% % % % % % % % % % % % % % 
%
%	U.S. Chartbook
%	Brian W. Dew (brianwdew@gmail.com)
%	Updated: September 23, 2019
%	GitHub repo contains to do list (issues)
%   https://github.com/bdecon/US-chartbook
%
% % % % % % % % % % % % % %
\PassOptionsToPackage{table}{xcolor}
\documentclass{report}

%
% % % % % % Packages % % % % % % % % % 
%
	
	\usepackage[letterpaper, margin=1.18in]{geometry}
	\usepackage{microtype}
	\usepackage[default]{lato}
	\usepackage{pgfplots, pgfplotstable}
	\usepackage{xcolor}
	\usepackage{array}
	\usepackage{fontawesome}
	\usepackage{hyperref}
	\hypersetup{colorlinks=true, linkcolor=blue}
	\usetikzlibrary{pgfplots.dateplot}

%
% % % % % Document Settings % % % % % % % 
%

	% Paragraph spacing
		\setlength{\parskip}{8pt}
		\setlength{\parindent}{0pt}
		
%
% % % % % Graph Settings % % % % % % % 
%
	
	% Color square
	\newcommand{\cbox}[1]{
		\begin{tikzpicture} \draw [#1, line width=6](0,0) -- (.2,0);  
		\end{tikzpicture}}
	\newcommand{\colorline}[2]{
		\begin{tikzpicture} \draw [#1, line width=1.8](0,0.2) -- +(0.6,0) node[right, black!80] {#2}; 
		\end{tikzpicture}}
	
	% Last two digits of year
	\makeatletter
	\newcommand*\short[1]{\expandafter\@gobbletwo\number\numexpr#1\relax}
	\makeatother	
	
	% Column width and alignment
	\newcolumntype{R}[1]{>{\raggedleft\let\newline\\\arraybackslash\hspace{0pt}}m{#1}}	
	
	% Style for date plots
	\pgfplotsset{compat=newest, 
		scaled y ticks=false,
		axis line style={black!20}, 
		xtick style={black!20}, ytick style={draw=none},
		every tick label/.style={black!50, font=\scriptsize,
			/pgf/number format/assume math mode=true},
		width=13.0cm, height=4.8cm, 
		xticklabel style={align=left}, 
		yticklabel style={text width=0.85em, align=right},       
		axis x line*=bottom, x axis line style={black!50},
	    axis y line=left, y axis line style={opacity=0},
	    ymajorgrids, grid style={very thin, black!10},	        
	    every node near coord/.style={/pgf/number format/fixed,
	    	font=\scriptsize, style={black!70}},
	    legend style={legend columns=-1, draw=none, fill=none,
	    	/tikz/every even column/.append style={column sep=0.3cm}}}
	
	% stacked diverging bar
	\newcommand{\sbar}[4]{
		\addplot[ybar stacked, bar width=2.7pt, draw opacity=0, fill=#1] 
			table [x=#2, y=#3, col sep=comma]{#4};}
					
	% text node
	\newcommand{\stdnode}[3]{\node[below, align=left, shift=({#1,#2})]{#3};}	        
		        
	% Date (X) Axis Tick Marks, one tick per year, every even year labeled
	\newcommand{\dateaxisticks}{
		date coordinates in=x, axis line style={draw=none},
		xmax={2019-10-01},
		max space between ticks=40,	    
		xtick={{1990-01-01}, {1992-01-01}, {1994-01-01}, 
			{1996-01-01}, {1998-01-01}, {2000-01-01}, 
			{2002-01-01}, {2004-01-01}, {2006-01-01},
			{2008-01-01}, {2010-01-01}, {2012-01-01}, {2014-01-01},
		    {2016-01-01}, {2018-01-01}},
		minor xtick={{1989-01-01}, {1991-01-01}, {1993-01-01},
			{1995-01-01}, {1997-01-01}, {1999-01-01}, 
			{2001-01-01}, {2003-01-01}, {2005-01-01}, {2007-01-01},
		    {2009-01-01}, {2011-01-01}, {2013-01-01}, {2015-01-01},
		    {2017-01-01}, {2019-01-01}},
		enlarge y limits={0.06}, enlarge x limits={0.01},
		}
		
	% Settings for y label text in horizontal bar charts
	\newcommand{\barylab}[2]{yticklabel style={text width=#1, align=right, 
		style={black!70}, text height=#2},}
	
	% Solid bars at significant  x or y values
	\newcommand{\bbar}[2]{extra #1 ticks = {{#2}}, extra #1 tick labels = ,
		extra #1 tick style = {grid=major, grid style={thick, black!25}},}
		
	% Standard line
	\newcommand{\stdline}[4]{\addplot[very thick, no markers, color=#1] 
		table [x=#2, y=#3, col sep=comma] {#4};	}
		
	% Thicker line
	\newcommand{\thickline}[4]{\addplot[ultra thick, no markers, color=#1] 
		table [x=#2, y=#3, col sep=comma] {#4};	}
		
	% Style for bar plots legend symbol		
		\pgfplotsset{/pgfplots/area legend/.style={/pgfplots/legend image code/.code={
            \fill[##1] (0cm,-0.1cm) rectangle (0.6cm,0.1cm);}},}		
		
	% Additional bar plot settings
	\newcommand{\barplotnogrid}{xbar=0pt,
	    y axis line style={opacity=0},   
	    x axis line style={opacity=0}, 
	    yticklabel style={align=left, anchor=east},
      		xmajorticks=false, ymajorgrids=false,   
	    ytick=data, tickwidth=0pt, area legend, reverse legend,
	    nodes near coords, nodes near coords align={horizontal},}  
		
	% Recession bars		
	\newcommand{\rbars}{
		\fill[color=black!10] (axis cs:{1990-07-01},\pgfkeysvalueof{/pgfplots/ymin}) rectangle 
			(axis cs:{1991-03-01}, \pgfkeysvalueof{/pgfplots/ymax});
		\fill[color=black!10] (axis cs:{2007-12-01},\pgfkeysvalueof{/pgfplots/ymin}) rectangle 
			(axis cs:{2009-07-01}, \pgfkeysvalueof{/pgfplots/ymax});
		\fill[color=black!10] (axis cs:{2001-03-01},\pgfkeysvalueof{/pgfplots/ymin}) rectangle 
			(axis cs:{2001-11-01}, \pgfkeysvalueof{/pgfplots/ymax});}
			
	\newcommand{\rebars}{
		\fill[color=black!10] (axis cs:{2007-12-01},\pgfkeysvalueof{/pgfplots/ymin}) rectangle 
			(axis cs:{2009-07-01}, \pgfkeysvalueof{/pgfplots/ymax});
		\fill[color=black!10] (axis cs:{2001-03-01},\pgfkeysvalueof{/pgfplots/ymin}) rectangle 
			(axis cs:{2001-11-01}, \pgfkeysvalueof{/pgfplots/ymax});}
	
	\newfontfamily\seriffont{SourceSerifPro}	
	
	\pgfplotstableread[header=true, col sep=comma]{data/cpi_comp.csv}\cpi
	\pgfplotstableread[header=true, col sep=semicolon]{data/ip_comp.csv}\ip
			    		    
% % % % % % % %
%
%  Begin Document
%
% % % % % % % %		
\begin{document}

\chapter*{
		\textcolor{blue!70}{\rule[-1pt]{6pt}{20pt}}
		\textcolor{green!70!blue}{\rule[-1pt]{6pt}{32pt}} \ \color{darkgray} \seriffont US Chartbook}
\vspace*{-16mm}
\footnotesize \hspace{11mm} v0.0, \today \normalsize \\

\vspace{12mm}

\begin{minipage}{0.76\textwidth}
\subsection*{\color{black!70} {\seriffont Notes}}

{\color{red} \textbf{Very early stage draft}} -- Contents not considered reliable.\\


\subsection*{\color{black!70} {\seriffont Contact}}

\textbf{Brian Wilson Dew} \ 

{\color{gray} \faEnvelope} \ brian.w.dew@gmail.com \ 

{\color{gray} \faTwitter} \ @bd\_econ \

{\color{gray} \faGithub} \ \ bdecon\\

\end{minipage}
\thispagestyle{empty}

\newpage

\subsection*{\color{black!70} {\seriffont Contents}}

\begin{description}

\item {\seriffont Overall Economic Activity}

\item {\seriffont Overall Financial Activity}

\item {\seriffont Households}

\item {\seriffont Businesses}

\item {\seriffont Government}

\item {\seriffont External Sector}

\item {\seriffont Labor Markets}

\item {\seriffont Capital Markets}

\item {\seriffont Prices}

\item {\seriffont International Comparisons}

\item {\seriffont References}

\end{description}

\vspace{2mm}

\newpage

\subsection*{\color{black!70} {\seriffont Ideas/Suggestions/To Do}}

\begin{minipage}{0.76\textwidth}

\small Continue to fill out the content of the document. Additionally, refactor some of the older code and clean up some of the issues with older charts and text. \\

It will be nice to have a section showing the top five indicators: GDP growth, wages, epop, cpi inflation, 10-year treasury yields. \\

It will also be nice to have a section in that puts some context on numbers generally. The key example that I've tried to do before is to put a threshold on GDP growth that marks how much is needed for population growth and depreciation and then calculate how much one extra pp of growth (beyond the previous amount) is worth, per person. For example, if population growth is 0.6pp and depreciation is 0.8pp, then it would take 1.4pp to keep the same level of real per capita production. Beyond that, an extra percentage point of GDP might mean something like \$900 per person in additional goods and services. \\

Section listing recent updates and upcoming releases would be nice. This would require some thinking in terms of implementation. \\

Get the table of contents up and running soon. Also look into options for links to footnotes at the end of the document. Add in some table and release numbers/data where available.\\

Beyond content, I still need to do/add: links to subsection, links to sources, links to data, links to code, date of last update, list of charts and numbering system, links between charts and references, marks for recent updates, explicitly note seasonal adjustment, adjusting to make text associated with values of less than one singular instead of plural (e.g. "0.1 percentage point"), and much much more. \\

Major LT developments: rise of imports, computers in the 1990s, welfare reform in 1996, rise in education level, aging of the population. Major MT developments: increase in health care costs, housing bubble, government austerity from 2010 to 2014. Major ST developments: low business investment, higher wages, increased employment, low interest rates on LT debt, low productivity growth. 

\end{minipage}


\newpage

\section*{\color{darkgray} \LARGE  \seriffont Overall Economic Activity}
\input{text/gdp.txt}

\begin{minipage}{0.76\textwidth}
\subsection*{\color{black!70}\seriffont Economic Growth}

\small GDP (see\cbox{red!95!black}) \input{text/gdp_gr.txt}
\vspace{5mm}

\noindent \normalsize \textbf{Real Gross Domestic Product Growth}\\
\footnotesize{\textit{quarterly growth at seasonally adjusted annual rate, percent}}\\
\noindent \hspace*{-2mm} \begin{tikzpicture}
	\begin{axis}[\bbar{y}{0}, \dateaxisticks ytick={-5, 0, 5}, 
		xticklabel={`\short{\year}}, clip=false, height=4.0cm]
	\rbars
	\sbar{red!95!black}{date}{A191RL}{data/gdp.csv}
	\node[above, align=left] at (axis cs:2017-01-01,-8.5) {\scriptsize \input{data/gdp.txt}};
	\end{axis}
\end{tikzpicture}\\
\footnotesize{Source: Bureau of Economic Analysis} 

\subsection*{\color{black!70} \seriffont Components of Growth}

\small The \textbf{expenditure approach} compiles GDP from the sum of spending on domestic goods and services. Major spending categories are consumer spending (see\cbox{yellow!80!orange}), private investment (gross spending on capital goods) and changes in private inventories (see\cbox{blue!70!black}), government spending and investment (see\cbox{cyan!50!white}), and net exports (see\cbox{green!60!black}) which is measured as foreign spending on US goods and services less US spending on goods and services produced by the rest of the world. 
\vspace{5mm}

\noindent \normalsize \textbf{Real GDP Growth by Expenditure Type}\\
\footnotesize{\textit{percentage point contribution to GDP growth}}\\
\noindent \hspace*{-2mm} \begin{tikzpicture}
	\begin{axis}[\bbar{y}{0}, \dateaxisticks ytick={-5, 0, 5},
		xticklabel={`\short{\year}}, clip=false, legend style={at={(0.95, 1.13)}}]
	\rbars
	\sbar{yellow!80!orange}{date}{DPCERY}{data/comp.csv}
	\sbar{blue!70!black}{date}{A006RY}{data/comp.csv}
	\sbar{cyan!50!white}{date}{A822RY}{data/comp.csv}
	\sbar{green!60!black}{date}{A019RY}{data/comp.csv}
	\stdnode{2.2cm}{0.4cm}{\footnotesize $^*$ Includes change in private inventories}
	\legend{Consumer Spending, Investment$^*$, Government, Net Exports};
	\end{axis}
\end{tikzpicture}\\
\footnotesize{Source: Bureau of Economic Analysis}
\end{minipage}

\newpage
\begin{minipage}{0.76\textwidth}
\small The \textbf{production approach} calculates GDP as the sum of gross value added--output minus inputs--in each sector. This identifies contributions from: goods-producing sectors combined with trade, transportation, and utilities (see\cbox{purple!70!blue}), finance, insurance, and real estate (see\cbox{red!90!white}), other service-providing sectors (see\cbox{blue!90!white}), and government (see\cbox{orange!80!white}).
\vspace{5mm}

\noindent \normalsize \textbf{Real GDP Growth by Industry Group}\\
\footnotesize{\textit{percentage point contribution to GDP growth}}\\
\noindent \hspace*{-2mm} \begin{tikzpicture}
	\begin{axis}[\bbar{y}{0}, \dateaxisticks ytick={-5, 0, 5},
		xticklabel={`\short{\year}}, clip=false, legend style={at={(0.95, 1.13)}}]
	\rbars
	\draw [dashed] (axis cs:{2004-10-01},\pgfkeysvalueof{/pgfplots/ymin}) -- (axis cs:{2004-10-01},
		\pgfkeysvalueof{/pgfplots/ymax});
	\draw [dashed] (axis cs:{1998-01-01},\pgfkeysvalueof{/pgfplots/ymin}) -- (axis cs:{1998-01-01},
		\pgfkeysvalueof{/pgfplots/ymax});
	\sbar{orange!80!white}{date}{Government}{data/gdpva.csv}
	\sbar{blue!90!white}{date}{Oth_Serv}{data/gdpva.csv}
	\sbar{red!90!white}{date}{FIRE}{data/gdpva.csv}
	\sbar{purple!70!blue}{date}{GoodsTTU}{data/gdpva.csv}
	\stdnode{0.65cm}{0.4cm}{\scriptsize historical data}
	\stdnode{4.1cm}{0.4cm}{\scriptsize annual data}
	\legend{Government, Other Services, FIRE, Goods and TTU};
	\end{axis}
\end{tikzpicture}\\
\footnotesize{Source: Bureau of Economic Analysis}
\vspace{6mm}

\small The \textbf{income approach} calculates GDP as the sum of market income to persons (in exchange for labor (see\cbox{magenta!90!blue}) or from returns on capital (see\cbox{yellow!60!orange})), indirect taxes such as sales taxes or tariffs (see\cbox{violet}), and depreciation (see\cbox{teal!60!white}). 
\vspace{5mm}

\noindent \normalsize \textbf{Real Gross Domestic Income Growth}\\
\footnotesize{\textit{percentage point contribution to GDI growth}}\\
\noindent \hspace*{-2mm} \begin{tikzpicture}
	\begin{axis}[\bbar{y}{0}, \dateaxisticks ytick={-5, 0, 5},
		xticklabel={`\short{\year}}, clip=false, 
		legend style={at={(0.95, 1.13)}}]
	\rbars
	\sbar{magenta!90!blue}{date}{A4002C}{data/gdi.csv}
	\sbar{yellow!60!orange}{date}{W271RC}{data/gdi.csv}
	\sbar{teal!60!white}{date}{A262RC}{data/gdi.csv}
	\sbar{violet}{date}{indirect}{data/gdi.csv}	
	\legend{Labor, Profit, Depreciation, Indirect Taxes};
	\end{axis}
\end{tikzpicture}\\
\footnotesize{Source: Bureau of Economic Analysis}
\vspace{6mm}

\small Changes to GDP can be assigned to changes in \textbf{household inputs}: population (see\cbox{lime}), employment rates (see\cbox{green!30!teal!90!black}), average hours worked (see\cbox{blue}), and total economy productivity (see\cbox{cyan!50!white}). 
\vspace{5mm}

\noindent \normalsize \textbf{Real GDP Growth by Inputs}\\
\footnotesize{\textit{percentage point contribution to GDP growth}}\\
\noindent \hspace*{-2mm} \begin{tikzpicture}
	\begin{axis}[\bbar{y}{0}, \dateaxisticks ytick={-10, -5, 0, 5, 10},
		xticklabel={`\short{\year}}, clip=false, 
		legend style={at={(0.95, 1.13)}}]
	\rbars
	\sbar{lime}{date}{pop_contr}{data/gdpjobs.csv}
	\sbar{green!30!teal!90!black}{date}{epop_contr}{data/gdpjobs.csv}
	\sbar{cyan!50!white}{date}{prod}{data/gdpjobs.csv}
	\sbar{blue}{date}{hours_contr}{data/gdpjobs.csv}	
	\legend{Population, Employment Rate, Productivity, Average Hours};
	\end{axis}
\end{tikzpicture}\\
\footnotesize{Source: Author's Calculations}
\end{minipage}

\newpage

\noindent \normalsize \textbf{Components of Economic Growth}\\
\footnotesize{\textit{percentage point contribution to real GDP/GDI growth \hspace{30mm} moving averages}\\ \vspace{4mm}
\hspace*{-2mm} \noindent \rowcolors{1}{}{black!5} \setlength{\tabcolsep}{3.5pt} \color{black!90}
		{\renewcommand{\arraystretch}{1.55}
		 \begin{tabular}{p{2mm} p{35.2mm} R{6.8mm} R{6.8mm} R{6.8mm} R{6.8mm} R{6.8mm} p{0mm} R{6.8mm} R{6.8mm} R{6.8mm} }
& & 2020 Q3 & '20 Q2 & '20 Q1 & '19 Q4 & '19 Q3 & & 3-year & 10-year & 30-year \\
\cbox{red!95!black} & \textbf{Gross Domestic Product} & 33.1 & -31.4 & -5.0 & 2.4 & 2.6 & & 1.8 &  2.0 & 2.4 \\
\cbox{yellow!80!orange} & \hspace{2mm} Consumer Spending & 25.22 & -24.01 & -4.75 & 1.07 & 1.83 & & 1.06 &  1.45 & 1.74 \\
& \hspace{4mm} Durable Goods & 5.20 & 0.00 & -0.93 & 0.22 & 0.44 & & 0.66 &  0.51 & 0.46 \\
& \hspace{4mm} Non-durable Goods  & 4.29 & -2.05 & 0.97 & -0.10 & 0.43 & & 0.53 &  0.37 & 0.35 \\
& \hspace{4mm} Services  & 15.73 & -21.95 & -4.78 & 0.96 & 0.96 & & -0.13 &  0.57 & 0.93 \\
\cbox{blue!70!black} & \hspace{2mm} Gross Investment & 11.78 & -8.77 & -1.56 & -0.64 & 0.34 & & 0.55 &  0.75 & 0.61 \\
& \hspace{4mm} Non-residential  & 3.06 & -3.67 & -0.91 & -0.04 & 0.25 & & 0.32 &  0.54 & 0.52 \\
& \hspace{4mm} Residential  & 2.17 & -1.60 & 0.68 & 0.22 & 0.17 & & 0.09 &  0.16 & 0.05 \\
& \hspace{4mm} Change in inventories  & 6.55 & -3.50 & -1.34 & -0.82 & -0.09 & & 0.14 &  0.05 & 0.04 \\
\cbox{cyan!50!white} & \hspace{2mm} Government  & -0.76 & 0.77 & 0.22 & 0.42 & 0.37 & & 0.30 &  -0.01 & 0.22 \\
& \hspace{4mm} Federal  & -0.38 & 1.17 & 0.10 & 0.26 & 0.31 & & 0.24 &  -0.02 & 0.07 \\
& \hspace{4mm} State and Local  & -0.38 & -0.40 & 0.12 & 0.16 & 0.06 & & 0.05 &  0.01 & 0.15 \\
\cbox{green!60!black} & \hspace{2mm} Net Exports  & -3.18 & 0.62 & 1.13 & 1.52 & 0.04 & & -0.16 &  -0.16 & -0.16 \\
& \hspace{4mm} Exports  & 4.95 & -9.51 & -1.12 & 0.39 & 0.10 & & -0.27 &  0.21 & 0.43 \\
& \hspace{4mm} Imports  & -8.12 & 10.13 & 2.25 & 1.13 & -0.06 & & 0.11 &  -0.37 & -0.59 \\
& & & & & & & & & & \\
\cbox{purple!70!blue} & \hspace{2mm} Goods and TTU  & -- & -12.45 & -1.09 & 0.49 & 1.08 & & -0.37 &  0.40 & 0.77 \\
& \hspace{4mm} Manufacturing  & -- & -4.10 & -0.70 & 0.00 & 0.53 & & -0.14 &  0.04 & 0.29 \\
& \hspace{4mm} Construction  & -- & -1.12 & 0.02 & 0.00 & 0.00 & & -0.06 &  0.04 & -0.02 \\
& \hspace{4mm} Retail Trade  & -- & -1.75 & -0.39 & 0.14 & 0.21 & & -0.04 &  0.08 & 0.17 \\
\cbox{red!90!white} & \hspace{2mm} FIRE+  & -- & -0.53 & -1.27 & 1.12 & 0.28 & & 0.44 &  0.59 & 0.71 \\
\cbox{blue!90!white} & \hspace{2mm} Other Services  & -- & -16.47 & -2.36 & 0.47 & 1.14 & & -0.69 &  0.28 & 0.47 \\
& \hspace{4mm} Education \& Healthcare  & -- & -4.54 & -0.59 & 0.19 & 0.17 & & -0.23 &  0.06 & 0.15 \\
& \hspace{4mm} Professional \& Business & -- & -3.84 & -0.24 & 0.39 & 0.63 & & 0.18 &  0.34 & 0.30 \\
& \hspace{4mm} Information  & -- & -0.29 & -0.15 & 0.38 & 0.37 & & 0.28 &  0.27 & 0.24 \\
\cbox{orange!80!white} & \hspace{2mm} Government  & -- & -1.93 & -0.30 & 0.34 & 0.14 & & -0.04 &  -0.02 & 0.09 \\
& & & & & & & & & & \\
\cbox{lime!90!green} & \hspace{2mm} Population  & 0.76 & 0.31 & 0.40 & 0.57 & 0.57 & & 0.52 &  0.65 & 0.94 \\
\cbox{green!30!teal!90!black} & \hspace{2mm} Employment Rate  & 27.37 & -49.19 & -2.48 & 1.43 & 1.83 & & -1.23 &  0.07 & -0.13 \\
\cbox{blue} & \hspace{2mm} Average Hours & 0.47 & -4.68 & -3.19 & 0.38 & 0.71 & & -0.19 &  0.05 & -0.04 \\
\cbox{cyan!60!white} & \hspace{2mm} Productivity  & 4.47 & 22.17 & 0.32 & -0.01 & -0.54 & & 2.65 &  1.26 & 1.63 \\
& & & & & & & & & & \\& \textbf{Gross Domestic Income}  & 25.5 & -32.6 & -2.5 & 3.3 & 0.8 & & 0.9 &  1.9 & 2.4 \\
\cbox{magenta!90!blue} & \hspace{2mm} Labor  & 10.75 & -10.73 & 1.11 & 1.59 & 0.13 & & 1.11 &  1.16 & 1.25 \\
\cbox{yellow!60!orange} & \hspace{2mm} Profit  & 16.02 & -4.23 & -4.38 & 1.30 & 0.28 & & 0.80 &  0.77 & 0.75 \\
\cbox{teal!60!white} & \hspace{2mm} Depreciation  & 0.12 & 0.71 & 0.37 & 0.34 & 0.51 & & 0.41 &  0.36 & 0.41 \\
\cbox{violet} & \hspace{2mm} Indirect Taxes  & -1.40 & -18.33 & 0.37 & 0.07 & -0.09 & & -1.40 &  -0.35 & 0.00 \\

		\end{tabular}}	\\
\footnotesize{Source: Bureau of Economic Analysis and Author's Calculations}
		
\newpage

\noindent \normalsize \textbf{Real GDP Growth by State}\\
\footnotesize{\textit{percentage point change in real GDP}}\\
\vspace{-2mm}
\hspace{-8mm} %% Creator: Matplotlib, PGF backend
%%
%% To include the figure in your LaTeX document, write
%%   \input{<filename>.pgf}
%%
%% Make sure the required packages are loaded in your preamble
%%   \usepackage{pgf}
%%
%% Also ensure that all the required font packages are loaded; for instance,
%% the lmodern package is sometimes necessary when using math font.
%%   \usepackage{lmodern}
%%
%% Figures using additional raster images can only be included by \input if
%% they are in the same directory as the main LaTeX file. For loading figures
%% from other directories you can use the `import` package
%%   \usepackage{import}
%%
%% and then include the figures with
%%   \import{<path to file>}{<filename>.pgf}
%%
%% Matplotlib used the following preamble
%%   
%%   \usepackage{fontspec}
%%   \setmainfont{DejaVuSerif.ttf}[Path=\detokenize{/home/brian/miniconda3/lib/python3.8/site-packages/matplotlib/mpl-data/fonts/ttf/}]
%%   \setsansfont{DejaVuSans.ttf}[Path=\detokenize{/home/brian/miniconda3/lib/python3.8/site-packages/matplotlib/mpl-data/fonts/ttf/}]
%%   \setmonofont{DejaVuSansMono.ttf}[Path=\detokenize{/home/brian/miniconda3/lib/python3.8/site-packages/matplotlib/mpl-data/fonts/ttf/}]
%%   \makeatletter\@ifpackageloaded{underscore}{}{\usepackage[strings]{underscore}}\makeatother
%%
\begingroup%
\makeatletter%
\begin{pgfpicture}%
\pgfpathrectangle{\pgfpointorigin}{\pgfqpoint{6.714028in}{2.186394in}}%
\pgfusepath{use as bounding box, clip}%
\begin{pgfscope}%
\pgfsetbuttcap%
\pgfsetmiterjoin%
\definecolor{currentfill}{rgb}{1.000000,1.000000,1.000000}%
\pgfsetfillcolor{currentfill}%
\pgfsetlinewidth{0.000000pt}%
\definecolor{currentstroke}{rgb}{1.000000,1.000000,1.000000}%
\pgfsetstrokecolor{currentstroke}%
\pgfsetdash{}{0pt}%
\pgfpathmoveto{\pgfqpoint{0.000000in}{0.000000in}}%
\pgfpathlineto{\pgfqpoint{6.714028in}{0.000000in}}%
\pgfpathlineto{\pgfqpoint{6.714028in}{2.186394in}}%
\pgfpathlineto{\pgfqpoint{0.000000in}{2.186394in}}%
\pgfpathlineto{\pgfqpoint{0.000000in}{0.000000in}}%
\pgfpathclose%
\pgfusepath{fill}%
\end{pgfscope}%
\begin{pgfscope}%
\pgfpathrectangle{\pgfqpoint{0.100000in}{0.100000in}}{\pgfqpoint{2.989028in}{1.913466in}}%
\pgfusepath{clip}%
\pgfsetbuttcap%
\pgfsetmiterjoin%
\definecolor{currentfill}{rgb}{0.959631,0.983852,0.686044}%
\pgfsetfillcolor{currentfill}%
\pgfsetlinewidth{0.000000pt}%
\definecolor{currentstroke}{rgb}{0.000000,0.000000,0.000000}%
\pgfsetstrokecolor{currentstroke}%
\pgfsetstrokeopacity{0.000000}%
\pgfsetdash{}{0pt}%
\pgfpathmoveto{\pgfqpoint{1.097679in}{0.548675in}}%
\pgfpathlineto{\pgfqpoint{1.089089in}{0.548815in}}%
\pgfpathlineto{\pgfqpoint{1.083242in}{0.555287in}}%
\pgfpathlineto{\pgfqpoint{1.079266in}{0.572473in}}%
\pgfpathlineto{\pgfqpoint{1.088506in}{0.574627in}}%
\pgfpathlineto{\pgfqpoint{1.097893in}{0.572311in}}%
\pgfpathlineto{\pgfqpoint{1.107446in}{0.565540in}}%
\pgfpathlineto{\pgfqpoint{1.107441in}{0.556177in}}%
\pgfpathlineto{\pgfqpoint{1.097679in}{0.548675in}}%
\pgfpathclose%
\pgfusepath{fill}%
\end{pgfscope}%
\begin{pgfscope}%
\pgfpathrectangle{\pgfqpoint{0.100000in}{0.100000in}}{\pgfqpoint{2.989028in}{1.913466in}}%
\pgfusepath{clip}%
\pgfsetbuttcap%
\pgfsetmiterjoin%
\definecolor{currentfill}{rgb}{0.959631,0.983852,0.686044}%
\pgfsetfillcolor{currentfill}%
\pgfsetlinewidth{0.000000pt}%
\definecolor{currentstroke}{rgb}{0.000000,0.000000,0.000000}%
\pgfsetstrokecolor{currentstroke}%
\pgfsetstrokeopacity{0.000000}%
\pgfsetdash{}{0pt}%
\pgfpathmoveto{\pgfqpoint{1.149970in}{0.446640in}}%
\pgfpathlineto{\pgfqpoint{1.140359in}{0.450300in}}%
\pgfpathlineto{\pgfqpoint{1.140315in}{0.457479in}}%
\pgfpathlineto{\pgfqpoint{1.128705in}{0.463549in}}%
\pgfpathlineto{\pgfqpoint{1.130119in}{0.478835in}}%
\pgfpathlineto{\pgfqpoint{1.133481in}{0.486048in}}%
\pgfpathlineto{\pgfqpoint{1.140581in}{0.479409in}}%
\pgfpathlineto{\pgfqpoint{1.152127in}{0.480518in}}%
\pgfpathlineto{\pgfqpoint{1.149211in}{0.459391in}}%
\pgfpathlineto{\pgfqpoint{1.149970in}{0.446640in}}%
\pgfpathclose%
\pgfusepath{fill}%
\end{pgfscope}%
\begin{pgfscope}%
\pgfpathrectangle{\pgfqpoint{0.100000in}{0.100000in}}{\pgfqpoint{2.989028in}{1.913466in}}%
\pgfusepath{clip}%
\pgfsetbuttcap%
\pgfsetmiterjoin%
\definecolor{currentfill}{rgb}{0.959631,0.983852,0.686044}%
\pgfsetfillcolor{currentfill}%
\pgfsetlinewidth{0.000000pt}%
\definecolor{currentstroke}{rgb}{0.000000,0.000000,0.000000}%
\pgfsetstrokecolor{currentstroke}%
\pgfsetstrokeopacity{0.000000}%
\pgfsetdash{}{0pt}%
\pgfpathmoveto{\pgfqpoint{1.190310in}{0.399517in}}%
\pgfpathlineto{\pgfqpoint{1.178114in}{0.401594in}}%
\pgfpathlineto{\pgfqpoint{1.171369in}{0.412035in}}%
\pgfpathlineto{\pgfqpoint{1.170417in}{0.420620in}}%
\pgfpathlineto{\pgfqpoint{1.190310in}{0.399517in}}%
\pgfpathclose%
\pgfusepath{fill}%
\end{pgfscope}%
\begin{pgfscope}%
\pgfpathrectangle{\pgfqpoint{0.100000in}{0.100000in}}{\pgfqpoint{2.989028in}{1.913466in}}%
\pgfusepath{clip}%
\pgfsetbuttcap%
\pgfsetmiterjoin%
\definecolor{currentfill}{rgb}{0.959631,0.983852,0.686044}%
\pgfsetfillcolor{currentfill}%
\pgfsetlinewidth{0.000000pt}%
\definecolor{currentstroke}{rgb}{0.000000,0.000000,0.000000}%
\pgfsetstrokecolor{currentstroke}%
\pgfsetstrokeopacity{0.000000}%
\pgfsetdash{}{0pt}%
\pgfpathmoveto{\pgfqpoint{1.165317in}{0.401338in}}%
\pgfpathlineto{\pgfqpoint{1.171711in}{0.396147in}}%
\pgfpathlineto{\pgfqpoint{1.173099in}{0.387222in}}%
\pgfpathlineto{\pgfqpoint{1.160227in}{0.387715in}}%
\pgfpathlineto{\pgfqpoint{1.165317in}{0.401338in}}%
\pgfpathclose%
\pgfusepath{fill}%
\end{pgfscope}%
\begin{pgfscope}%
\pgfpathrectangle{\pgfqpoint{0.100000in}{0.100000in}}{\pgfqpoint{2.989028in}{1.913466in}}%
\pgfusepath{clip}%
\pgfsetbuttcap%
\pgfsetmiterjoin%
\definecolor{currentfill}{rgb}{0.959631,0.983852,0.686044}%
\pgfsetfillcolor{currentfill}%
\pgfsetlinewidth{0.000000pt}%
\definecolor{currentstroke}{rgb}{0.000000,0.000000,0.000000}%
\pgfsetstrokecolor{currentstroke}%
\pgfsetstrokeopacity{0.000000}%
\pgfsetdash{}{0pt}%
\pgfpathmoveto{\pgfqpoint{1.194683in}{0.351485in}}%
\pgfpathlineto{\pgfqpoint{1.181837in}{0.356103in}}%
\pgfpathlineto{\pgfqpoint{1.185518in}{0.367298in}}%
\pgfpathlineto{\pgfqpoint{1.180134in}{0.378705in}}%
\pgfpathlineto{\pgfqpoint{1.181825in}{0.386546in}}%
\pgfpathlineto{\pgfqpoint{1.192162in}{0.389535in}}%
\pgfpathlineto{\pgfqpoint{1.191963in}{0.376972in}}%
\pgfpathlineto{\pgfqpoint{1.204190in}{0.370003in}}%
\pgfpathlineto{\pgfqpoint{1.211183in}{0.353155in}}%
\pgfpathlineto{\pgfqpoint{1.203389in}{0.346667in}}%
\pgfpathlineto{\pgfqpoint{1.194683in}{0.351485in}}%
\pgfpathclose%
\pgfusepath{fill}%
\end{pgfscope}%
\begin{pgfscope}%
\pgfpathrectangle{\pgfqpoint{0.100000in}{0.100000in}}{\pgfqpoint{2.989028in}{1.913466in}}%
\pgfusepath{clip}%
\pgfsetbuttcap%
\pgfsetmiterjoin%
\definecolor{currentfill}{rgb}{0.959631,0.983852,0.686044}%
\pgfsetfillcolor{currentfill}%
\pgfsetlinewidth{0.000000pt}%
\definecolor{currentstroke}{rgb}{0.000000,0.000000,0.000000}%
\pgfsetstrokecolor{currentstroke}%
\pgfsetstrokeopacity{0.000000}%
\pgfsetdash{}{0pt}%
\pgfpathmoveto{\pgfqpoint{1.148289in}{0.237569in}}%
\pgfpathlineto{\pgfqpoint{1.142169in}{0.250211in}}%
\pgfpathlineto{\pgfqpoint{1.144879in}{0.258715in}}%
\pgfpathlineto{\pgfqpoint{1.156883in}{0.270485in}}%
\pgfpathlineto{\pgfqpoint{1.158019in}{0.279508in}}%
\pgfpathlineto{\pgfqpoint{1.165875in}{0.299484in}}%
\pgfpathlineto{\pgfqpoint{1.186536in}{0.302897in}}%
\pgfpathlineto{\pgfqpoint{1.190491in}{0.314590in}}%
\pgfpathlineto{\pgfqpoint{1.199600in}{0.310205in}}%
\pgfpathlineto{\pgfqpoint{1.218110in}{0.277388in}}%
\pgfpathlineto{\pgfqpoint{1.218344in}{0.266674in}}%
\pgfpathlineto{\pgfqpoint{1.213112in}{0.257400in}}%
\pgfpathlineto{\pgfqpoint{1.219750in}{0.236108in}}%
\pgfpathlineto{\pgfqpoint{1.215111in}{0.232676in}}%
\pgfpathlineto{\pgfqpoint{1.199311in}{0.233674in}}%
\pgfpathlineto{\pgfqpoint{1.180311in}{0.240024in}}%
\pgfpathlineto{\pgfqpoint{1.164342in}{0.242646in}}%
\pgfpathlineto{\pgfqpoint{1.160076in}{0.238921in}}%
\pgfpathlineto{\pgfqpoint{1.148289in}{0.237569in}}%
\pgfpathclose%
\pgfusepath{fill}%
\end{pgfscope}%
\begin{pgfscope}%
\pgfpathrectangle{\pgfqpoint{0.100000in}{0.100000in}}{\pgfqpoint{2.989028in}{1.913466in}}%
\pgfusepath{clip}%
\pgfsetbuttcap%
\pgfsetmiterjoin%
\definecolor{currentfill}{rgb}{0.711419,0.883276,0.634833}%
\pgfsetfillcolor{currentfill}%
\pgfsetlinewidth{0.000000pt}%
\definecolor{currentstroke}{rgb}{0.000000,0.000000,0.000000}%
\pgfsetstrokecolor{currentstroke}%
\pgfsetstrokeopacity{0.000000}%
\pgfsetdash{}{0pt}%
\pgfpathmoveto{\pgfqpoint{0.810222in}{1.900931in}}%
\pgfpathlineto{\pgfqpoint{0.794722in}{1.839512in}}%
\pgfpathlineto{\pgfqpoint{0.783700in}{1.795504in}}%
\pgfpathlineto{\pgfqpoint{0.770595in}{1.742564in}}%
\pgfpathlineto{\pgfqpoint{0.768022in}{1.729516in}}%
\pgfpathlineto{\pgfqpoint{0.769703in}{1.717547in}}%
\pgfpathlineto{\pgfqpoint{0.767475in}{1.706523in}}%
\pgfpathlineto{\pgfqpoint{0.675998in}{1.730394in}}%
\pgfpathlineto{\pgfqpoint{0.667734in}{1.727622in}}%
\pgfpathlineto{\pgfqpoint{0.660663in}{1.730028in}}%
\pgfpathlineto{\pgfqpoint{0.627921in}{1.730814in}}%
\pgfpathlineto{\pgfqpoint{0.616848in}{1.727692in}}%
\pgfpathlineto{\pgfqpoint{0.605851in}{1.728747in}}%
\pgfpathlineto{\pgfqpoint{0.601190in}{1.733634in}}%
\pgfpathlineto{\pgfqpoint{0.571877in}{1.732568in}}%
\pgfpathlineto{\pgfqpoint{0.567248in}{1.740554in}}%
\pgfpathlineto{\pgfqpoint{0.558460in}{1.744762in}}%
\pgfpathlineto{\pgfqpoint{0.545673in}{1.747325in}}%
\pgfpathlineto{\pgfqpoint{0.523603in}{1.743571in}}%
\pgfpathlineto{\pgfqpoint{0.515456in}{1.747249in}}%
\pgfpathlineto{\pgfqpoint{0.502895in}{1.757031in}}%
\pgfpathlineto{\pgfqpoint{0.506670in}{1.776202in}}%
\pgfpathlineto{\pgfqpoint{0.505347in}{1.785992in}}%
\pgfpathlineto{\pgfqpoint{0.495391in}{1.796386in}}%
\pgfpathlineto{\pgfqpoint{0.489015in}{1.795741in}}%
\pgfpathlineto{\pgfqpoint{0.484440in}{1.806255in}}%
\pgfpathlineto{\pgfqpoint{0.473604in}{1.810504in}}%
\pgfpathlineto{\pgfqpoint{0.465727in}{1.809963in}}%
\pgfpathlineto{\pgfqpoint{0.463179in}{1.820760in}}%
\pgfpathlineto{\pgfqpoint{0.471026in}{1.819611in}}%
\pgfpathlineto{\pgfqpoint{0.470402in}{1.834619in}}%
\pgfpathlineto{\pgfqpoint{0.466982in}{1.843530in}}%
\pgfpathlineto{\pgfqpoint{0.468490in}{1.854935in}}%
\pgfpathlineto{\pgfqpoint{0.472595in}{1.863514in}}%
\pgfpathlineto{\pgfqpoint{0.472364in}{1.879792in}}%
\pgfpathlineto{\pgfqpoint{0.470172in}{1.885699in}}%
\pgfpathlineto{\pgfqpoint{0.473957in}{1.904648in}}%
\pgfpathlineto{\pgfqpoint{0.472865in}{1.916860in}}%
\pgfpathlineto{\pgfqpoint{0.469081in}{1.922716in}}%
\pgfpathlineto{\pgfqpoint{0.469658in}{1.941861in}}%
\pgfpathlineto{\pgfqpoint{0.475162in}{1.955941in}}%
\pgfpathlineto{\pgfqpoint{0.481149in}{1.952495in}}%
\pgfpathlineto{\pgfqpoint{0.500972in}{1.932018in}}%
\pgfpathlineto{\pgfqpoint{0.524993in}{1.920812in}}%
\pgfpathlineto{\pgfqpoint{0.537292in}{1.919486in}}%
\pgfpathlineto{\pgfqpoint{0.548880in}{1.914715in}}%
\pgfpathlineto{\pgfqpoint{0.552168in}{1.898661in}}%
\pgfpathlineto{\pgfqpoint{0.542014in}{1.895620in}}%
\pgfpathlineto{\pgfqpoint{0.535285in}{1.882967in}}%
\pgfpathlineto{\pgfqpoint{0.543185in}{1.883670in}}%
\pgfpathlineto{\pgfqpoint{0.546372in}{1.889349in}}%
\pgfpathlineto{\pgfqpoint{0.557514in}{1.896451in}}%
\pgfpathlineto{\pgfqpoint{0.556939in}{1.885905in}}%
\pgfpathlineto{\pgfqpoint{0.549499in}{1.884204in}}%
\pgfpathlineto{\pgfqpoint{0.550734in}{1.870648in}}%
\pgfpathlineto{\pgfqpoint{0.541674in}{1.857069in}}%
\pgfpathlineto{\pgfqpoint{0.535876in}{1.863526in}}%
\pgfpathlineto{\pgfqpoint{0.518645in}{1.860015in}}%
\pgfpathlineto{\pgfqpoint{0.517738in}{1.852116in}}%
\pgfpathlineto{\pgfqpoint{0.523678in}{1.847303in}}%
\pgfpathlineto{\pgfqpoint{0.532741in}{1.846830in}}%
\pgfpathlineto{\pgfqpoint{0.545271in}{1.857114in}}%
\pgfpathlineto{\pgfqpoint{0.555189in}{1.857968in}}%
\pgfpathlineto{\pgfqpoint{0.555569in}{1.869352in}}%
\pgfpathlineto{\pgfqpoint{0.560680in}{1.886072in}}%
\pgfpathlineto{\pgfqpoint{0.572793in}{1.898506in}}%
\pgfpathlineto{\pgfqpoint{0.568754in}{1.908138in}}%
\pgfpathlineto{\pgfqpoint{0.571513in}{1.918507in}}%
\pgfpathlineto{\pgfqpoint{0.566139in}{1.928842in}}%
\pgfpathlineto{\pgfqpoint{0.573798in}{1.942057in}}%
\pgfpathlineto{\pgfqpoint{0.574945in}{1.950008in}}%
\pgfpathlineto{\pgfqpoint{0.568296in}{1.955234in}}%
\pgfpathlineto{\pgfqpoint{0.569403in}{1.968600in}}%
\pgfpathlineto{\pgfqpoint{0.649072in}{1.944395in}}%
\pgfpathlineto{\pgfqpoint{0.733673in}{1.920671in}}%
\pgfpathlineto{\pgfqpoint{0.810222in}{1.900931in}}%
\pgfpathclose%
\pgfusepath{fill}%
\end{pgfscope}%
\begin{pgfscope}%
\pgfpathrectangle{\pgfqpoint{0.100000in}{0.100000in}}{\pgfqpoint{2.989028in}{1.913466in}}%
\pgfusepath{clip}%
\pgfsetbuttcap%
\pgfsetmiterjoin%
\definecolor{currentfill}{rgb}{0.711419,0.883276,0.634833}%
\pgfsetfillcolor{currentfill}%
\pgfsetlinewidth{0.000000pt}%
\definecolor{currentstroke}{rgb}{0.000000,0.000000,0.000000}%
\pgfsetstrokecolor{currentstroke}%
\pgfsetstrokeopacity{0.000000}%
\pgfsetdash{}{0pt}%
\pgfpathmoveto{\pgfqpoint{0.556297in}{1.922359in}}%
\pgfpathlineto{\pgfqpoint{0.560220in}{1.907235in}}%
\pgfpathlineto{\pgfqpoint{0.566372in}{1.902211in}}%
\pgfpathlineto{\pgfqpoint{0.561463in}{1.895816in}}%
\pgfpathlineto{\pgfqpoint{0.556679in}{1.905186in}}%
\pgfpathlineto{\pgfqpoint{0.556297in}{1.922359in}}%
\pgfpathclose%
\pgfusepath{fill}%
\end{pgfscope}%
\begin{pgfscope}%
\pgfpathrectangle{\pgfqpoint{0.100000in}{0.100000in}}{\pgfqpoint{2.989028in}{1.913466in}}%
\pgfusepath{clip}%
\pgfsetbuttcap%
\pgfsetmiterjoin%
\definecolor{currentfill}{rgb}{0.793080,0.916494,0.618224}%
\pgfsetfillcolor{currentfill}%
\pgfsetlinewidth{0.000000pt}%
\definecolor{currentstroke}{rgb}{0.000000,0.000000,0.000000}%
\pgfsetstrokecolor{currentstroke}%
\pgfsetstrokeopacity{0.000000}%
\pgfsetdash{}{0pt}%
\pgfpathmoveto{\pgfqpoint{0.851305in}{1.891004in}}%
\pgfpathlineto{\pgfqpoint{0.936467in}{1.871893in}}%
\pgfpathlineto{\pgfqpoint{1.016670in}{1.855681in}}%
\pgfpathlineto{\pgfqpoint{1.078378in}{1.844375in}}%
\pgfpathlineto{\pgfqpoint{1.132175in}{1.835337in}}%
\pgfpathlineto{\pgfqpoint{1.186092in}{1.827040in}}%
\pgfpathlineto{\pgfqpoint{1.232007in}{1.820572in}}%
\pgfpathlineto{\pgfqpoint{1.277995in}{1.814641in}}%
\pgfpathlineto{\pgfqpoint{1.324048in}{1.809248in}}%
\pgfpathlineto{\pgfqpoint{1.367448in}{1.804664in}}%
\pgfpathlineto{\pgfqpoint{1.361479in}{1.738522in}}%
\pgfpathlineto{\pgfqpoint{1.352595in}{1.649140in}}%
\pgfpathlineto{\pgfqpoint{1.347936in}{1.603169in}}%
\pgfpathlineto{\pgfqpoint{1.341234in}{1.541215in}}%
\pgfpathlineto{\pgfqpoint{1.293794in}{1.546395in}}%
\pgfpathlineto{\pgfqpoint{1.239479in}{1.552561in}}%
\pgfpathlineto{\pgfqpoint{1.164064in}{1.562882in}}%
\pgfpathlineto{\pgfqpoint{1.130381in}{1.567687in}}%
\pgfpathlineto{\pgfqpoint{1.047396in}{1.580713in}}%
\pgfpathlineto{\pgfqpoint{1.018829in}{1.585939in}}%
\pgfpathlineto{\pgfqpoint{1.012875in}{1.552087in}}%
\pgfpathlineto{\pgfqpoint{1.009620in}{1.554502in}}%
\pgfpathlineto{\pgfqpoint{1.003502in}{1.570788in}}%
\pgfpathlineto{\pgfqpoint{0.995639in}{1.569785in}}%
\pgfpathlineto{\pgfqpoint{0.990011in}{1.561552in}}%
\pgfpathlineto{\pgfqpoint{0.978331in}{1.560451in}}%
\pgfpathlineto{\pgfqpoint{0.975984in}{1.564390in}}%
\pgfpathlineto{\pgfqpoint{0.965043in}{1.563913in}}%
\pgfpathlineto{\pgfqpoint{0.959543in}{1.567727in}}%
\pgfpathlineto{\pgfqpoint{0.951852in}{1.561826in}}%
\pgfpathlineto{\pgfqpoint{0.933172in}{1.567136in}}%
\pgfpathlineto{\pgfqpoint{0.925020in}{1.564285in}}%
\pgfpathlineto{\pgfqpoint{0.920634in}{1.576938in}}%
\pgfpathlineto{\pgfqpoint{0.920429in}{1.588978in}}%
\pgfpathlineto{\pgfqpoint{0.907552in}{1.597637in}}%
\pgfpathlineto{\pgfqpoint{0.909652in}{1.610116in}}%
\pgfpathlineto{\pgfqpoint{0.900483in}{1.630729in}}%
\pgfpathlineto{\pgfqpoint{0.901101in}{1.648251in}}%
\pgfpathlineto{\pgfqpoint{0.893220in}{1.656827in}}%
\pgfpathlineto{\pgfqpoint{0.885827in}{1.649240in}}%
\pgfpathlineto{\pgfqpoint{0.875763in}{1.645136in}}%
\pgfpathlineto{\pgfqpoint{0.868102in}{1.653594in}}%
\pgfpathlineto{\pgfqpoint{0.869348in}{1.667101in}}%
\pgfpathlineto{\pgfqpoint{0.878535in}{1.672422in}}%
\pgfpathlineto{\pgfqpoint{0.876093in}{1.680980in}}%
\pgfpathlineto{\pgfqpoint{0.892063in}{1.722399in}}%
\pgfpathlineto{\pgfqpoint{0.879470in}{1.723437in}}%
\pgfpathlineto{\pgfqpoint{0.878176in}{1.730870in}}%
\pgfpathlineto{\pgfqpoint{0.869003in}{1.737279in}}%
\pgfpathlineto{\pgfqpoint{0.869589in}{1.744445in}}%
\pgfpathlineto{\pgfqpoint{0.864750in}{1.750011in}}%
\pgfpathlineto{\pgfqpoint{0.856873in}{1.770222in}}%
\pgfpathlineto{\pgfqpoint{0.849675in}{1.774343in}}%
\pgfpathlineto{\pgfqpoint{0.844521in}{1.791786in}}%
\pgfpathlineto{\pgfqpoint{0.845737in}{1.803380in}}%
\pgfpathlineto{\pgfqpoint{0.836121in}{1.824733in}}%
\pgfpathlineto{\pgfqpoint{0.851305in}{1.891004in}}%
\pgfpathclose%
\pgfusepath{fill}%
\end{pgfscope}%
\begin{pgfscope}%
\pgfpathrectangle{\pgfqpoint{0.100000in}{0.100000in}}{\pgfqpoint{2.989028in}{1.913466in}}%
\pgfusepath{clip}%
\pgfsetbuttcap%
\pgfsetmiterjoin%
\definecolor{currentfill}{rgb}{0.963476,0.985390,0.692042}%
\pgfsetfillcolor{currentfill}%
\pgfsetlinewidth{0.000000pt}%
\definecolor{currentstroke}{rgb}{0.000000,0.000000,0.000000}%
\pgfsetstrokecolor{currentstroke}%
\pgfsetstrokeopacity{0.000000}%
\pgfsetdash{}{0pt}%
\pgfpathmoveto{\pgfqpoint{2.907175in}{1.550163in}}%
\pgfpathlineto{\pgfqpoint{2.905471in}{1.557411in}}%
\pgfpathlineto{\pgfqpoint{2.895998in}{1.563760in}}%
\pgfpathlineto{\pgfqpoint{2.870964in}{1.645645in}}%
\pgfpathlineto{\pgfqpoint{2.857471in}{1.685161in}}%
\pgfpathlineto{\pgfqpoint{2.868319in}{1.696546in}}%
\pgfpathlineto{\pgfqpoint{2.878864in}{1.724407in}}%
\pgfpathlineto{\pgfqpoint{2.884123in}{1.733311in}}%
\pgfpathlineto{\pgfqpoint{2.880397in}{1.737050in}}%
\pgfpathlineto{\pgfqpoint{2.879149in}{1.761782in}}%
\pgfpathlineto{\pgfqpoint{2.883854in}{1.769364in}}%
\pgfpathlineto{\pgfqpoint{2.881792in}{1.786986in}}%
\pgfpathlineto{\pgfqpoint{2.900570in}{1.844804in}}%
\pgfpathlineto{\pgfqpoint{2.909031in}{1.845120in}}%
\pgfpathlineto{\pgfqpoint{2.912543in}{1.834792in}}%
\pgfpathlineto{\pgfqpoint{2.920062in}{1.831826in}}%
\pgfpathlineto{\pgfqpoint{2.934266in}{1.843974in}}%
\pgfpathlineto{\pgfqpoint{2.945403in}{1.851149in}}%
\pgfpathlineto{\pgfqpoint{2.969943in}{1.838514in}}%
\pgfpathlineto{\pgfqpoint{2.992113in}{1.768360in}}%
\pgfpathlineto{\pgfqpoint{2.996328in}{1.751101in}}%
\pgfpathlineto{\pgfqpoint{3.006039in}{1.749076in}}%
\pgfpathlineto{\pgfqpoint{3.019209in}{1.737354in}}%
\pgfpathlineto{\pgfqpoint{3.018463in}{1.730524in}}%
\pgfpathlineto{\pgfqpoint{3.027435in}{1.722428in}}%
\pgfpathlineto{\pgfqpoint{3.034723in}{1.727073in}}%
\pgfpathlineto{\pgfqpoint{3.049974in}{1.709287in}}%
\pgfpathlineto{\pgfqpoint{3.043158in}{1.695053in}}%
\pgfpathlineto{\pgfqpoint{3.034018in}{1.694771in}}%
\pgfpathlineto{\pgfqpoint{3.026685in}{1.682065in}}%
\pgfpathlineto{\pgfqpoint{3.017802in}{1.680263in}}%
\pgfpathlineto{\pgfqpoint{3.011280in}{1.673497in}}%
\pgfpathlineto{\pgfqpoint{2.991803in}{1.666241in}}%
\pgfpathlineto{\pgfqpoint{2.980014in}{1.654548in}}%
\pgfpathlineto{\pgfqpoint{2.974025in}{1.662941in}}%
\pgfpathlineto{\pgfqpoint{2.968584in}{1.656926in}}%
\pgfpathlineto{\pgfqpoint{2.970074in}{1.632607in}}%
\pgfpathlineto{\pgfqpoint{2.965765in}{1.622974in}}%
\pgfpathlineto{\pgfqpoint{2.956367in}{1.625605in}}%
\pgfpathlineto{\pgfqpoint{2.954850in}{1.615730in}}%
\pgfpathlineto{\pgfqpoint{2.950789in}{1.611669in}}%
\pgfpathlineto{\pgfqpoint{2.942662in}{1.612659in}}%
\pgfpathlineto{\pgfqpoint{2.943174in}{1.603425in}}%
\pgfpathlineto{\pgfqpoint{2.930855in}{1.606048in}}%
\pgfpathlineto{\pgfqpoint{2.924128in}{1.593255in}}%
\pgfpathlineto{\pgfqpoint{2.926668in}{1.586561in}}%
\pgfpathlineto{\pgfqpoint{2.922666in}{1.575429in}}%
\pgfpathlineto{\pgfqpoint{2.916367in}{1.567240in}}%
\pgfpathlineto{\pgfqpoint{2.914781in}{1.550146in}}%
\pgfpathlineto{\pgfqpoint{2.907175in}{1.550163in}}%
\pgfpathclose%
\pgfusepath{fill}%
\end{pgfscope}%
\begin{pgfscope}%
\pgfpathrectangle{\pgfqpoint{0.100000in}{0.100000in}}{\pgfqpoint{2.989028in}{1.913466in}}%
\pgfusepath{clip}%
\pgfsetbuttcap%
\pgfsetmiterjoin%
\definecolor{currentfill}{rgb}{0.963476,0.985390,0.692042}%
\pgfsetfillcolor{currentfill}%
\pgfsetlinewidth{0.000000pt}%
\definecolor{currentstroke}{rgb}{0.000000,0.000000,0.000000}%
\pgfsetstrokecolor{currentstroke}%
\pgfsetstrokeopacity{0.000000}%
\pgfsetdash{}{0pt}%
\pgfpathmoveto{\pgfqpoint{2.995301in}{1.661247in}}%
\pgfpathlineto{\pgfqpoint{3.000863in}{1.667073in}}%
\pgfpathlineto{\pgfqpoint{3.006158in}{1.661597in}}%
\pgfpathlineto{\pgfqpoint{2.996647in}{1.654342in}}%
\pgfpathlineto{\pgfqpoint{2.995301in}{1.661247in}}%
\pgfpathclose%
\pgfusepath{fill}%
\end{pgfscope}%
\begin{pgfscope}%
\pgfpathrectangle{\pgfqpoint{0.100000in}{0.100000in}}{\pgfqpoint{2.989028in}{1.913466in}}%
\pgfusepath{clip}%
\pgfsetbuttcap%
\pgfsetmiterjoin%
\definecolor{currentfill}{rgb}{0.591003,0.835525,0.644291}%
\pgfsetfillcolor{currentfill}%
\pgfsetlinewidth{0.000000pt}%
\definecolor{currentstroke}{rgb}{0.000000,0.000000,0.000000}%
\pgfsetstrokecolor{currentstroke}%
\pgfsetstrokeopacity{0.000000}%
\pgfsetdash{}{0pt}%
\pgfpathmoveto{\pgfqpoint{1.347936in}{1.603169in}}%
\pgfpathlineto{\pgfqpoint{1.352595in}{1.649140in}}%
\pgfpathlineto{\pgfqpoint{1.361479in}{1.738522in}}%
\pgfpathlineto{\pgfqpoint{1.367448in}{1.804664in}}%
\pgfpathlineto{\pgfqpoint{1.416329in}{1.800078in}}%
\pgfpathlineto{\pgfqpoint{1.478862in}{1.795100in}}%
\pgfpathlineto{\pgfqpoint{1.536021in}{1.791418in}}%
\pgfpathlineto{\pgfqpoint{1.587777in}{1.788799in}}%
\pgfpathlineto{\pgfqpoint{1.665014in}{1.786150in}}%
\pgfpathlineto{\pgfqpoint{1.670285in}{1.764381in}}%
\pgfpathlineto{\pgfqpoint{1.668398in}{1.753978in}}%
\pgfpathlineto{\pgfqpoint{1.667777in}{1.732596in}}%
\pgfpathlineto{\pgfqpoint{1.671363in}{1.716582in}}%
\pgfpathlineto{\pgfqpoint{1.679579in}{1.692956in}}%
\pgfpathlineto{\pgfqpoint{1.679560in}{1.663212in}}%
\pgfpathlineto{\pgfqpoint{1.681080in}{1.628658in}}%
\pgfpathlineto{\pgfqpoint{1.683167in}{1.619346in}}%
\pgfpathlineto{\pgfqpoint{1.689274in}{1.609122in}}%
\pgfpathlineto{\pgfqpoint{1.691288in}{1.593192in}}%
\pgfpathlineto{\pgfqpoint{1.690419in}{1.582557in}}%
\pgfpathlineto{\pgfqpoint{1.625668in}{1.583959in}}%
\pgfpathlineto{\pgfqpoint{1.578564in}{1.586347in}}%
\pgfpathlineto{\pgfqpoint{1.509511in}{1.590023in}}%
\pgfpathlineto{\pgfqpoint{1.441415in}{1.594860in}}%
\pgfpathlineto{\pgfqpoint{1.396059in}{1.598519in}}%
\pgfpathlineto{\pgfqpoint{1.347936in}{1.603169in}}%
\pgfpathclose%
\pgfusepath{fill}%
\end{pgfscope}%
\begin{pgfscope}%
\pgfpathrectangle{\pgfqpoint{0.100000in}{0.100000in}}{\pgfqpoint{2.989028in}{1.913466in}}%
\pgfusepath{clip}%
\pgfsetbuttcap%
\pgfsetmiterjoin%
\definecolor{currentfill}{rgb}{0.940408,0.976163,0.656055}%
\pgfsetfillcolor{currentfill}%
\pgfsetlinewidth{0.000000pt}%
\definecolor{currentstroke}{rgb}{0.000000,0.000000,0.000000}%
\pgfsetstrokecolor{currentstroke}%
\pgfsetstrokeopacity{0.000000}%
\pgfsetdash{}{0pt}%
\pgfpathmoveto{\pgfqpoint{1.328363in}{1.410780in}}%
\pgfpathlineto{\pgfqpoint{1.335839in}{1.487840in}}%
\pgfpathlineto{\pgfqpoint{1.341234in}{1.541215in}}%
\pgfpathlineto{\pgfqpoint{1.347936in}{1.603169in}}%
\pgfpathlineto{\pgfqpoint{1.396059in}{1.598519in}}%
\pgfpathlineto{\pgfqpoint{1.441415in}{1.594860in}}%
\pgfpathlineto{\pgfqpoint{1.509511in}{1.590023in}}%
\pgfpathlineto{\pgfqpoint{1.578564in}{1.586347in}}%
\pgfpathlineto{\pgfqpoint{1.625668in}{1.583959in}}%
\pgfpathlineto{\pgfqpoint{1.690419in}{1.582557in}}%
\pgfpathlineto{\pgfqpoint{1.686034in}{1.569764in}}%
\pgfpathlineto{\pgfqpoint{1.677284in}{1.559721in}}%
\pgfpathlineto{\pgfqpoint{1.683985in}{1.548158in}}%
\pgfpathlineto{\pgfqpoint{1.691374in}{1.545688in}}%
\pgfpathlineto{\pgfqpoint{1.694880in}{1.539046in}}%
\pgfpathlineto{\pgfqpoint{1.694075in}{1.490552in}}%
\pgfpathlineto{\pgfqpoint{1.692728in}{1.422304in}}%
\pgfpathlineto{\pgfqpoint{1.687744in}{1.404379in}}%
\pgfpathlineto{\pgfqpoint{1.692201in}{1.394464in}}%
\pgfpathlineto{\pgfqpoint{1.683252in}{1.373157in}}%
\pgfpathlineto{\pgfqpoint{1.692680in}{1.355981in}}%
\pgfpathlineto{\pgfqpoint{1.684670in}{1.357295in}}%
\pgfpathlineto{\pgfqpoint{1.679205in}{1.367982in}}%
\pgfpathlineto{\pgfqpoint{1.659700in}{1.375307in}}%
\pgfpathlineto{\pgfqpoint{1.647400in}{1.381750in}}%
\pgfpathlineto{\pgfqpoint{1.626759in}{1.382286in}}%
\pgfpathlineto{\pgfqpoint{1.619594in}{1.376416in}}%
\pgfpathlineto{\pgfqpoint{1.596231in}{1.387975in}}%
\pgfpathlineto{\pgfqpoint{1.594439in}{1.391630in}}%
\pgfpathlineto{\pgfqpoint{1.512902in}{1.395489in}}%
\pgfpathlineto{\pgfqpoint{1.463365in}{1.398329in}}%
\pgfpathlineto{\pgfqpoint{1.422447in}{1.401533in}}%
\pgfpathlineto{\pgfqpoint{1.354836in}{1.407921in}}%
\pgfpathlineto{\pgfqpoint{1.328363in}{1.410780in}}%
\pgfpathclose%
\pgfusepath{fill}%
\end{pgfscope}%
\begin{pgfscope}%
\pgfpathrectangle{\pgfqpoint{0.100000in}{0.100000in}}{\pgfqpoint{2.989028in}{1.913466in}}%
\pgfusepath{clip}%
\pgfsetbuttcap%
\pgfsetmiterjoin%
\definecolor{currentfill}{rgb}{0.368627,0.309804,0.635294}%
\pgfsetfillcolor{currentfill}%
\pgfsetlinewidth{0.000000pt}%
\definecolor{currentstroke}{rgb}{0.000000,0.000000,0.000000}%
\pgfsetstrokecolor{currentstroke}%
\pgfsetstrokeopacity{0.000000}%
\pgfsetdash{}{0pt}%
\pgfpathmoveto{\pgfqpoint{1.315540in}{1.280295in}}%
\pgfpathlineto{\pgfqpoint{1.272072in}{1.284265in}}%
\pgfpathlineto{\pgfqpoint{1.177321in}{1.295959in}}%
\pgfpathlineto{\pgfqpoint{1.125778in}{1.303357in}}%
\pgfpathlineto{\pgfqpoint{1.070495in}{1.311293in}}%
\pgfpathlineto{\pgfqpoint{1.023928in}{1.318744in}}%
\pgfpathlineto{\pgfqpoint{0.972815in}{1.327496in}}%
\pgfpathlineto{\pgfqpoint{0.984436in}{1.391963in}}%
\pgfpathlineto{\pgfqpoint{0.996208in}{1.458068in}}%
\pgfpathlineto{\pgfqpoint{1.012875in}{1.552087in}}%
\pgfpathlineto{\pgfqpoint{1.018829in}{1.585939in}}%
\pgfpathlineto{\pgfqpoint{1.047396in}{1.580713in}}%
\pgfpathlineto{\pgfqpoint{1.130381in}{1.567687in}}%
\pgfpathlineto{\pgfqpoint{1.164064in}{1.562882in}}%
\pgfpathlineto{\pgfqpoint{1.239479in}{1.552561in}}%
\pgfpathlineto{\pgfqpoint{1.293794in}{1.546395in}}%
\pgfpathlineto{\pgfqpoint{1.341234in}{1.541215in}}%
\pgfpathlineto{\pgfqpoint{1.335839in}{1.487840in}}%
\pgfpathlineto{\pgfqpoint{1.328363in}{1.410780in}}%
\pgfpathlineto{\pgfqpoint{1.321947in}{1.345289in}}%
\pgfpathlineto{\pgfqpoint{1.315540in}{1.280295in}}%
\pgfpathclose%
\pgfusepath{fill}%
\end{pgfscope}%
\begin{pgfscope}%
\pgfpathrectangle{\pgfqpoint{0.100000in}{0.100000in}}{\pgfqpoint{2.989028in}{1.913466in}}%
\pgfusepath{clip}%
\pgfsetbuttcap%
\pgfsetmiterjoin%
\definecolor{currentfill}{rgb}{0.999616,0.988082,0.729027}%
\pgfsetfillcolor{currentfill}%
\pgfsetlinewidth{0.000000pt}%
\definecolor{currentstroke}{rgb}{0.000000,0.000000,0.000000}%
\pgfsetstrokecolor{currentstroke}%
\pgfsetstrokeopacity{0.000000}%
\pgfsetdash{}{0pt}%
\pgfpathmoveto{\pgfqpoint{2.110291in}{1.369221in}}%
\pgfpathlineto{\pgfqpoint{2.055206in}{1.365308in}}%
\pgfpathlineto{\pgfqpoint{1.973100in}{1.361809in}}%
\pgfpathlineto{\pgfqpoint{1.969980in}{1.370103in}}%
\pgfpathlineto{\pgfqpoint{1.951772in}{1.376303in}}%
\pgfpathlineto{\pgfqpoint{1.947756in}{1.388016in}}%
\pgfpathlineto{\pgfqpoint{1.946075in}{1.402507in}}%
\pgfpathlineto{\pgfqpoint{1.950178in}{1.409931in}}%
\pgfpathlineto{\pgfqpoint{1.943697in}{1.417056in}}%
\pgfpathlineto{\pgfqpoint{1.942134in}{1.425554in}}%
\pgfpathlineto{\pgfqpoint{1.940045in}{1.444352in}}%
\pgfpathlineto{\pgfqpoint{1.933835in}{1.454564in}}%
\pgfpathlineto{\pgfqpoint{1.922805in}{1.460295in}}%
\pgfpathlineto{\pgfqpoint{1.910803in}{1.469752in}}%
\pgfpathlineto{\pgfqpoint{1.904599in}{1.480942in}}%
\pgfpathlineto{\pgfqpoint{1.893466in}{1.485448in}}%
\pgfpathlineto{\pgfqpoint{1.886922in}{1.492782in}}%
\pgfpathlineto{\pgfqpoint{1.878994in}{1.494027in}}%
\pgfpathlineto{\pgfqpoint{1.864840in}{1.504916in}}%
\pgfpathlineto{\pgfqpoint{1.867139in}{1.517447in}}%
\pgfpathlineto{\pgfqpoint{1.866696in}{1.541277in}}%
\pgfpathlineto{\pgfqpoint{1.871061in}{1.547836in}}%
\pgfpathlineto{\pgfqpoint{1.867139in}{1.557741in}}%
\pgfpathlineto{\pgfqpoint{1.860230in}{1.559657in}}%
\pgfpathlineto{\pgfqpoint{1.860801in}{1.568355in}}%
\pgfpathlineto{\pgfqpoint{1.869373in}{1.582098in}}%
\pgfpathlineto{\pgfqpoint{1.886348in}{1.592965in}}%
\pgfpathlineto{\pgfqpoint{1.885290in}{1.631611in}}%
\pgfpathlineto{\pgfqpoint{1.893784in}{1.637416in}}%
\pgfpathlineto{\pgfqpoint{1.901822in}{1.633549in}}%
\pgfpathlineto{\pgfqpoint{1.918197in}{1.639211in}}%
\pgfpathlineto{\pgfqpoint{1.949004in}{1.653439in}}%
\pgfpathlineto{\pgfqpoint{1.953017in}{1.649033in}}%
\pgfpathlineto{\pgfqpoint{1.947182in}{1.629079in}}%
\pgfpathlineto{\pgfqpoint{1.955863in}{1.633458in}}%
\pgfpathlineto{\pgfqpoint{1.970731in}{1.629085in}}%
\pgfpathlineto{\pgfqpoint{1.979867in}{1.625436in}}%
\pgfpathlineto{\pgfqpoint{1.984989in}{1.614740in}}%
\pgfpathlineto{\pgfqpoint{2.031860in}{1.604646in}}%
\pgfpathlineto{\pgfqpoint{2.045871in}{1.597722in}}%
\pgfpathlineto{\pgfqpoint{2.060116in}{1.597803in}}%
\pgfpathlineto{\pgfqpoint{2.074752in}{1.594947in}}%
\pgfpathlineto{\pgfqpoint{2.084251in}{1.585189in}}%
\pgfpathlineto{\pgfqpoint{2.093328in}{1.580368in}}%
\pgfpathlineto{\pgfqpoint{2.094996in}{1.566462in}}%
\pgfpathlineto{\pgfqpoint{2.092314in}{1.557730in}}%
\pgfpathlineto{\pgfqpoint{2.102430in}{1.557084in}}%
\pgfpathlineto{\pgfqpoint{2.099022in}{1.546957in}}%
\pgfpathlineto{\pgfqpoint{2.102264in}{1.543360in}}%
\pgfpathlineto{\pgfqpoint{2.105488in}{1.533800in}}%
\pgfpathlineto{\pgfqpoint{2.095629in}{1.528739in}}%
\pgfpathlineto{\pgfqpoint{2.089905in}{1.514637in}}%
\pgfpathlineto{\pgfqpoint{2.088109in}{1.504665in}}%
\pgfpathlineto{\pgfqpoint{2.093600in}{1.502958in}}%
\pgfpathlineto{\pgfqpoint{2.100632in}{1.510436in}}%
\pgfpathlineto{\pgfqpoint{2.106610in}{1.523394in}}%
\pgfpathlineto{\pgfqpoint{2.114698in}{1.527895in}}%
\pgfpathlineto{\pgfqpoint{2.120776in}{1.522041in}}%
\pgfpathlineto{\pgfqpoint{2.114793in}{1.504322in}}%
\pgfpathlineto{\pgfqpoint{2.112916in}{1.490567in}}%
\pgfpathlineto{\pgfqpoint{2.114684in}{1.480679in}}%
\pgfpathlineto{\pgfqpoint{2.109128in}{1.475085in}}%
\pgfpathlineto{\pgfqpoint{2.106353in}{1.461416in}}%
\pgfpathlineto{\pgfqpoint{2.108620in}{1.447050in}}%
\pgfpathlineto{\pgfqpoint{2.102064in}{1.425804in}}%
\pgfpathlineto{\pgfqpoint{2.102202in}{1.415188in}}%
\pgfpathlineto{\pgfqpoint{2.107383in}{1.392182in}}%
\pgfpathlineto{\pgfqpoint{2.110741in}{1.388236in}}%
\pgfpathlineto{\pgfqpoint{2.110291in}{1.369221in}}%
\pgfpathclose%
\pgfusepath{fill}%
\end{pgfscope}%
\begin{pgfscope}%
\pgfpathrectangle{\pgfqpoint{0.100000in}{0.100000in}}{\pgfqpoint{2.989028in}{1.913466in}}%
\pgfusepath{clip}%
\pgfsetbuttcap%
\pgfsetmiterjoin%
\definecolor{currentfill}{rgb}{0.999616,0.988082,0.729027}%
\pgfsetfillcolor{currentfill}%
\pgfsetlinewidth{0.000000pt}%
\definecolor{currentstroke}{rgb}{0.000000,0.000000,0.000000}%
\pgfsetstrokecolor{currentstroke}%
\pgfsetstrokeopacity{0.000000}%
\pgfsetdash{}{0pt}%
\pgfpathmoveto{\pgfqpoint{2.130944in}{1.555675in}}%
\pgfpathlineto{\pgfqpoint{2.132173in}{1.546498in}}%
\pgfpathlineto{\pgfqpoint{2.120909in}{1.522320in}}%
\pgfpathlineto{\pgfqpoint{2.115904in}{1.529323in}}%
\pgfpathlineto{\pgfqpoint{2.130944in}{1.555675in}}%
\pgfpathclose%
\pgfusepath{fill}%
\end{pgfscope}%
\begin{pgfscope}%
\pgfpathrectangle{\pgfqpoint{0.100000in}{0.100000in}}{\pgfqpoint{2.989028in}{1.913466in}}%
\pgfusepath{clip}%
\pgfsetbuttcap%
\pgfsetmiterjoin%
\definecolor{currentfill}{rgb}{0.944252,0.977701,0.662053}%
\pgfsetfillcolor{currentfill}%
\pgfsetlinewidth{0.000000pt}%
\definecolor{currentstroke}{rgb}{0.000000,0.000000,0.000000}%
\pgfsetstrokecolor{currentstroke}%
\pgfsetstrokeopacity{0.000000}%
\pgfsetdash{}{0pt}%
\pgfpathmoveto{\pgfqpoint{0.767475in}{1.706523in}}%
\pgfpathlineto{\pgfqpoint{0.769703in}{1.717547in}}%
\pgfpathlineto{\pgfqpoint{0.768022in}{1.729516in}}%
\pgfpathlineto{\pgfqpoint{0.770595in}{1.742564in}}%
\pgfpathlineto{\pgfqpoint{0.783700in}{1.795504in}}%
\pgfpathlineto{\pgfqpoint{0.794722in}{1.839512in}}%
\pgfpathlineto{\pgfqpoint{0.810222in}{1.900931in}}%
\pgfpathlineto{\pgfqpoint{0.851305in}{1.891004in}}%
\pgfpathlineto{\pgfqpoint{0.836121in}{1.824733in}}%
\pgfpathlineto{\pgfqpoint{0.845737in}{1.803380in}}%
\pgfpathlineto{\pgfqpoint{0.844521in}{1.791786in}}%
\pgfpathlineto{\pgfqpoint{0.849675in}{1.774343in}}%
\pgfpathlineto{\pgfqpoint{0.856873in}{1.770222in}}%
\pgfpathlineto{\pgfqpoint{0.864750in}{1.750011in}}%
\pgfpathlineto{\pgfqpoint{0.869589in}{1.744445in}}%
\pgfpathlineto{\pgfqpoint{0.869003in}{1.737279in}}%
\pgfpathlineto{\pgfqpoint{0.878176in}{1.730870in}}%
\pgfpathlineto{\pgfqpoint{0.879470in}{1.723437in}}%
\pgfpathlineto{\pgfqpoint{0.892063in}{1.722399in}}%
\pgfpathlineto{\pgfqpoint{0.876093in}{1.680980in}}%
\pgfpathlineto{\pgfqpoint{0.878535in}{1.672422in}}%
\pgfpathlineto{\pgfqpoint{0.869348in}{1.667101in}}%
\pgfpathlineto{\pgfqpoint{0.868102in}{1.653594in}}%
\pgfpathlineto{\pgfqpoint{0.875763in}{1.645136in}}%
\pgfpathlineto{\pgfqpoint{0.885827in}{1.649240in}}%
\pgfpathlineto{\pgfqpoint{0.893220in}{1.656827in}}%
\pgfpathlineto{\pgfqpoint{0.901101in}{1.648251in}}%
\pgfpathlineto{\pgfqpoint{0.900483in}{1.630729in}}%
\pgfpathlineto{\pgfqpoint{0.909652in}{1.610116in}}%
\pgfpathlineto{\pgfqpoint{0.907552in}{1.597637in}}%
\pgfpathlineto{\pgfqpoint{0.920429in}{1.588978in}}%
\pgfpathlineto{\pgfqpoint{0.920634in}{1.576938in}}%
\pgfpathlineto{\pgfqpoint{0.925020in}{1.564285in}}%
\pgfpathlineto{\pgfqpoint{0.933172in}{1.567136in}}%
\pgfpathlineto{\pgfqpoint{0.951852in}{1.561826in}}%
\pgfpathlineto{\pgfqpoint{0.959543in}{1.567727in}}%
\pgfpathlineto{\pgfqpoint{0.965043in}{1.563913in}}%
\pgfpathlineto{\pgfqpoint{0.975984in}{1.564390in}}%
\pgfpathlineto{\pgfqpoint{0.978331in}{1.560451in}}%
\pgfpathlineto{\pgfqpoint{0.990011in}{1.561552in}}%
\pgfpathlineto{\pgfqpoint{0.995639in}{1.569785in}}%
\pgfpathlineto{\pgfqpoint{1.003502in}{1.570788in}}%
\pgfpathlineto{\pgfqpoint{1.009620in}{1.554502in}}%
\pgfpathlineto{\pgfqpoint{1.012875in}{1.552087in}}%
\pgfpathlineto{\pgfqpoint{0.996208in}{1.458068in}}%
\pgfpathlineto{\pgfqpoint{0.984436in}{1.391963in}}%
\pgfpathlineto{\pgfqpoint{0.891592in}{1.409886in}}%
\pgfpathlineto{\pgfqpoint{0.841442in}{1.419862in}}%
\pgfpathlineto{\pgfqpoint{0.794534in}{1.430177in}}%
\pgfpathlineto{\pgfqpoint{0.750689in}{1.440103in}}%
\pgfpathlineto{\pgfqpoint{0.699946in}{1.452341in}}%
\pgfpathlineto{\pgfqpoint{0.726119in}{1.559712in}}%
\pgfpathlineto{\pgfqpoint{0.727531in}{1.567559in}}%
\pgfpathlineto{\pgfqpoint{0.739170in}{1.588120in}}%
\pgfpathlineto{\pgfqpoint{0.727118in}{1.600478in}}%
\pgfpathlineto{\pgfqpoint{0.729620in}{1.612613in}}%
\pgfpathlineto{\pgfqpoint{0.734208in}{1.615650in}}%
\pgfpathlineto{\pgfqpoint{0.742402in}{1.628155in}}%
\pgfpathlineto{\pgfqpoint{0.754345in}{1.636806in}}%
\pgfpathlineto{\pgfqpoint{0.754997in}{1.643207in}}%
\pgfpathlineto{\pgfqpoint{0.762206in}{1.649603in}}%
\pgfpathlineto{\pgfqpoint{0.768191in}{1.661575in}}%
\pgfpathlineto{\pgfqpoint{0.780469in}{1.674240in}}%
\pgfpathlineto{\pgfqpoint{0.779597in}{1.686749in}}%
\pgfpathlineto{\pgfqpoint{0.771203in}{1.693698in}}%
\pgfpathlineto{\pgfqpoint{0.767475in}{1.706523in}}%
\pgfpathclose%
\pgfusepath{fill}%
\end{pgfscope}%
\begin{pgfscope}%
\pgfpathrectangle{\pgfqpoint{0.100000in}{0.100000in}}{\pgfqpoint{2.989028in}{1.913466in}}%
\pgfusepath{clip}%
\pgfsetbuttcap%
\pgfsetmiterjoin%
\definecolor{currentfill}{rgb}{0.995463,0.847674,0.519262}%
\pgfsetfillcolor{currentfill}%
\pgfsetlinewidth{0.000000pt}%
\definecolor{currentstroke}{rgb}{0.000000,0.000000,0.000000}%
\pgfsetstrokecolor{currentstroke}%
\pgfsetstrokeopacity{0.000000}%
\pgfsetdash{}{0pt}%
\pgfpathmoveto{\pgfqpoint{2.799865in}{1.496989in}}%
\pgfpathlineto{\pgfqpoint{2.796007in}{1.509167in}}%
\pgfpathlineto{\pgfqpoint{2.788848in}{1.546108in}}%
\pgfpathlineto{\pgfqpoint{2.779593in}{1.557473in}}%
\pgfpathlineto{\pgfqpoint{2.771474in}{1.577911in}}%
\pgfpathlineto{\pgfqpoint{2.774140in}{1.593064in}}%
\pgfpathlineto{\pgfqpoint{2.772002in}{1.604233in}}%
\pgfpathlineto{\pgfqpoint{2.765065in}{1.615202in}}%
\pgfpathlineto{\pgfqpoint{2.764780in}{1.627285in}}%
\pgfpathlineto{\pgfqpoint{2.760752in}{1.640318in}}%
\pgfpathlineto{\pgfqpoint{2.796797in}{1.649189in}}%
\pgfpathlineto{\pgfqpoint{2.843616in}{1.661826in}}%
\pgfpathlineto{\pgfqpoint{2.845485in}{1.654580in}}%
\pgfpathlineto{\pgfqpoint{2.842504in}{1.643043in}}%
\pgfpathlineto{\pgfqpoint{2.849502in}{1.633812in}}%
\pgfpathlineto{\pgfqpoint{2.845785in}{1.622115in}}%
\pgfpathlineto{\pgfqpoint{2.831015in}{1.607411in}}%
\pgfpathlineto{\pgfqpoint{2.835075in}{1.596366in}}%
\pgfpathlineto{\pgfqpoint{2.832066in}{1.573021in}}%
\pgfpathlineto{\pgfqpoint{2.827854in}{1.559858in}}%
\pgfpathlineto{\pgfqpoint{2.833228in}{1.522426in}}%
\pgfpathlineto{\pgfqpoint{2.832504in}{1.509256in}}%
\pgfpathlineto{\pgfqpoint{2.837703in}{1.505030in}}%
\pgfpathlineto{\pgfqpoint{2.799865in}{1.496989in}}%
\pgfpathclose%
\pgfusepath{fill}%
\end{pgfscope}%
\begin{pgfscope}%
\pgfpathrectangle{\pgfqpoint{0.100000in}{0.100000in}}{\pgfqpoint{2.989028in}{1.913466in}}%
\pgfusepath{clip}%
\pgfsetbuttcap%
\pgfsetmiterjoin%
\definecolor{currentfill}{rgb}{0.959631,0.983852,0.686044}%
\pgfsetfillcolor{currentfill}%
\pgfsetlinewidth{0.000000pt}%
\definecolor{currentstroke}{rgb}{0.000000,0.000000,0.000000}%
\pgfsetstrokecolor{currentstroke}%
\pgfsetstrokeopacity{0.000000}%
\pgfsetdash{}{0pt}%
\pgfpathmoveto{\pgfqpoint{1.692728in}{1.422304in}}%
\pgfpathlineto{\pgfqpoint{1.694075in}{1.490552in}}%
\pgfpathlineto{\pgfqpoint{1.694880in}{1.539046in}}%
\pgfpathlineto{\pgfqpoint{1.691374in}{1.545688in}}%
\pgfpathlineto{\pgfqpoint{1.683985in}{1.548158in}}%
\pgfpathlineto{\pgfqpoint{1.677284in}{1.559721in}}%
\pgfpathlineto{\pgfqpoint{1.686034in}{1.569764in}}%
\pgfpathlineto{\pgfqpoint{1.690419in}{1.582557in}}%
\pgfpathlineto{\pgfqpoint{1.691288in}{1.593192in}}%
\pgfpathlineto{\pgfqpoint{1.689274in}{1.609122in}}%
\pgfpathlineto{\pgfqpoint{1.683167in}{1.619346in}}%
\pgfpathlineto{\pgfqpoint{1.681080in}{1.628658in}}%
\pgfpathlineto{\pgfqpoint{1.679560in}{1.663212in}}%
\pgfpathlineto{\pgfqpoint{1.679579in}{1.692956in}}%
\pgfpathlineto{\pgfqpoint{1.671363in}{1.716582in}}%
\pgfpathlineto{\pgfqpoint{1.667777in}{1.732596in}}%
\pgfpathlineto{\pgfqpoint{1.668398in}{1.753978in}}%
\pgfpathlineto{\pgfqpoint{1.670285in}{1.764381in}}%
\pgfpathlineto{\pgfqpoint{1.665014in}{1.786150in}}%
\pgfpathlineto{\pgfqpoint{1.700900in}{1.785431in}}%
\pgfpathlineto{\pgfqpoint{1.755411in}{1.784963in}}%
\pgfpathlineto{\pgfqpoint{1.755709in}{1.809705in}}%
\pgfpathlineto{\pgfqpoint{1.769584in}{1.806981in}}%
\pgfpathlineto{\pgfqpoint{1.776233in}{1.776812in}}%
\pgfpathlineto{\pgfqpoint{1.781136in}{1.765965in}}%
\pgfpathlineto{\pgfqpoint{1.793328in}{1.765645in}}%
\pgfpathlineto{\pgfqpoint{1.796056in}{1.761964in}}%
\pgfpathlineto{\pgfqpoint{1.813062in}{1.760334in}}%
\pgfpathlineto{\pgfqpoint{1.815920in}{1.752855in}}%
\pgfpathlineto{\pgfqpoint{1.827640in}{1.754536in}}%
\pgfpathlineto{\pgfqpoint{1.836742in}{1.761531in}}%
\pgfpathlineto{\pgfqpoint{1.852440in}{1.761270in}}%
\pgfpathlineto{\pgfqpoint{1.862153in}{1.755644in}}%
\pgfpathlineto{\pgfqpoint{1.863270in}{1.750368in}}%
\pgfpathlineto{\pgfqpoint{1.872512in}{1.749264in}}%
\pgfpathlineto{\pgfqpoint{1.878543in}{1.734873in}}%
\pgfpathlineto{\pgfqpoint{1.882434in}{1.743721in}}%
\pgfpathlineto{\pgfqpoint{1.893051in}{1.744265in}}%
\pgfpathlineto{\pgfqpoint{1.895736in}{1.737374in}}%
\pgfpathlineto{\pgfqpoint{1.907682in}{1.734228in}}%
\pgfpathlineto{\pgfqpoint{1.914270in}{1.724302in}}%
\pgfpathlineto{\pgfqpoint{1.928880in}{1.727384in}}%
\pgfpathlineto{\pgfqpoint{1.944957in}{1.739565in}}%
\pgfpathlineto{\pgfqpoint{1.950818in}{1.728827in}}%
\pgfpathlineto{\pgfqpoint{1.977198in}{1.731766in}}%
\pgfpathlineto{\pgfqpoint{1.988474in}{1.724397in}}%
\pgfpathlineto{\pgfqpoint{1.995037in}{1.727028in}}%
\pgfpathlineto{\pgfqpoint{2.000300in}{1.722879in}}%
\pgfpathlineto{\pgfqpoint{1.984687in}{1.713035in}}%
\pgfpathlineto{\pgfqpoint{1.962401in}{1.704265in}}%
\pgfpathlineto{\pgfqpoint{1.940329in}{1.686734in}}%
\pgfpathlineto{\pgfqpoint{1.921193in}{1.663680in}}%
\pgfpathlineto{\pgfqpoint{1.906707in}{1.650050in}}%
\pgfpathlineto{\pgfqpoint{1.894018in}{1.640682in}}%
\pgfpathlineto{\pgfqpoint{1.885290in}{1.631611in}}%
\pgfpathlineto{\pgfqpoint{1.886348in}{1.592965in}}%
\pgfpathlineto{\pgfqpoint{1.869373in}{1.582098in}}%
\pgfpathlineto{\pgfqpoint{1.860801in}{1.568355in}}%
\pgfpathlineto{\pgfqpoint{1.860230in}{1.559657in}}%
\pgfpathlineto{\pgfqpoint{1.867139in}{1.557741in}}%
\pgfpathlineto{\pgfqpoint{1.871061in}{1.547836in}}%
\pgfpathlineto{\pgfqpoint{1.866696in}{1.541277in}}%
\pgfpathlineto{\pgfqpoint{1.867139in}{1.517447in}}%
\pgfpathlineto{\pgfqpoint{1.864840in}{1.504916in}}%
\pgfpathlineto{\pgfqpoint{1.878994in}{1.494027in}}%
\pgfpathlineto{\pgfqpoint{1.886922in}{1.492782in}}%
\pgfpathlineto{\pgfqpoint{1.893466in}{1.485448in}}%
\pgfpathlineto{\pgfqpoint{1.904599in}{1.480942in}}%
\pgfpathlineto{\pgfqpoint{1.910803in}{1.469752in}}%
\pgfpathlineto{\pgfqpoint{1.922805in}{1.460295in}}%
\pgfpathlineto{\pgfqpoint{1.933835in}{1.454564in}}%
\pgfpathlineto{\pgfqpoint{1.940045in}{1.444352in}}%
\pgfpathlineto{\pgfqpoint{1.942134in}{1.425554in}}%
\pgfpathlineto{\pgfqpoint{1.883600in}{1.423428in}}%
\pgfpathlineto{\pgfqpoint{1.833705in}{1.422385in}}%
\pgfpathlineto{\pgfqpoint{1.788246in}{1.421717in}}%
\pgfpathlineto{\pgfqpoint{1.740155in}{1.421790in}}%
\pgfpathlineto{\pgfqpoint{1.692728in}{1.422304in}}%
\pgfpathclose%
\pgfusepath{fill}%
\end{pgfscope}%
\begin{pgfscope}%
\pgfpathrectangle{\pgfqpoint{0.100000in}{0.100000in}}{\pgfqpoint{2.989028in}{1.913466in}}%
\pgfusepath{clip}%
\pgfsetbuttcap%
\pgfsetmiterjoin%
\definecolor{currentfill}{rgb}{0.883814,0.953403,0.599769}%
\pgfsetfillcolor{currentfill}%
\pgfsetlinewidth{0.000000pt}%
\definecolor{currentstroke}{rgb}{0.000000,0.000000,0.000000}%
\pgfsetstrokecolor{currentstroke}%
\pgfsetstrokeopacity{0.000000}%
\pgfsetdash{}{0pt}%
\pgfpathmoveto{\pgfqpoint{0.363713in}{1.549891in}}%
\pgfpathlineto{\pgfqpoint{0.359077in}{1.558421in}}%
\pgfpathlineto{\pgfqpoint{0.362037in}{1.580316in}}%
\pgfpathlineto{\pgfqpoint{0.367750in}{1.591758in}}%
\pgfpathlineto{\pgfqpoint{0.364894in}{1.607237in}}%
\pgfpathlineto{\pgfqpoint{0.370854in}{1.613747in}}%
\pgfpathlineto{\pgfqpoint{0.381727in}{1.631292in}}%
\pgfpathlineto{\pgfqpoint{0.390946in}{1.641883in}}%
\pgfpathlineto{\pgfqpoint{0.404422in}{1.664951in}}%
\pgfpathlineto{\pgfqpoint{0.414758in}{1.690079in}}%
\pgfpathlineto{\pgfqpoint{0.425730in}{1.713646in}}%
\pgfpathlineto{\pgfqpoint{0.427948in}{1.723489in}}%
\pgfpathlineto{\pgfqpoint{0.443085in}{1.751647in}}%
\pgfpathlineto{\pgfqpoint{0.445999in}{1.764014in}}%
\pgfpathlineto{\pgfqpoint{0.452444in}{1.777001in}}%
\pgfpathlineto{\pgfqpoint{0.454747in}{1.792843in}}%
\pgfpathlineto{\pgfqpoint{0.467051in}{1.800568in}}%
\pgfpathlineto{\pgfqpoint{0.481639in}{1.804461in}}%
\pgfpathlineto{\pgfqpoint{0.489015in}{1.795741in}}%
\pgfpathlineto{\pgfqpoint{0.495391in}{1.796386in}}%
\pgfpathlineto{\pgfqpoint{0.505347in}{1.785992in}}%
\pgfpathlineto{\pgfqpoint{0.506670in}{1.776202in}}%
\pgfpathlineto{\pgfqpoint{0.502895in}{1.757031in}}%
\pgfpathlineto{\pgfqpoint{0.515456in}{1.747249in}}%
\pgfpathlineto{\pgfqpoint{0.523603in}{1.743571in}}%
\pgfpathlineto{\pgfqpoint{0.545673in}{1.747325in}}%
\pgfpathlineto{\pgfqpoint{0.558460in}{1.744762in}}%
\pgfpathlineto{\pgfqpoint{0.567248in}{1.740554in}}%
\pgfpathlineto{\pgfqpoint{0.571877in}{1.732568in}}%
\pgfpathlineto{\pgfqpoint{0.601190in}{1.733634in}}%
\pgfpathlineto{\pgfqpoint{0.605851in}{1.728747in}}%
\pgfpathlineto{\pgfqpoint{0.616848in}{1.727692in}}%
\pgfpathlineto{\pgfqpoint{0.627921in}{1.730814in}}%
\pgfpathlineto{\pgfqpoint{0.660663in}{1.730028in}}%
\pgfpathlineto{\pgfqpoint{0.667734in}{1.727622in}}%
\pgfpathlineto{\pgfqpoint{0.675998in}{1.730394in}}%
\pgfpathlineto{\pgfqpoint{0.767475in}{1.706523in}}%
\pgfpathlineto{\pgfqpoint{0.771203in}{1.693698in}}%
\pgfpathlineto{\pgfqpoint{0.779597in}{1.686749in}}%
\pgfpathlineto{\pgfqpoint{0.780469in}{1.674240in}}%
\pgfpathlineto{\pgfqpoint{0.768191in}{1.661575in}}%
\pgfpathlineto{\pgfqpoint{0.762206in}{1.649603in}}%
\pgfpathlineto{\pgfqpoint{0.754997in}{1.643207in}}%
\pgfpathlineto{\pgfqpoint{0.754345in}{1.636806in}}%
\pgfpathlineto{\pgfqpoint{0.742402in}{1.628155in}}%
\pgfpathlineto{\pgfqpoint{0.734208in}{1.615650in}}%
\pgfpathlineto{\pgfqpoint{0.729620in}{1.612613in}}%
\pgfpathlineto{\pgfqpoint{0.727118in}{1.600478in}}%
\pgfpathlineto{\pgfqpoint{0.739170in}{1.588120in}}%
\pgfpathlineto{\pgfqpoint{0.727531in}{1.567559in}}%
\pgfpathlineto{\pgfqpoint{0.726119in}{1.559712in}}%
\pgfpathlineto{\pgfqpoint{0.699946in}{1.452341in}}%
\pgfpathlineto{\pgfqpoint{0.644893in}{1.466447in}}%
\pgfpathlineto{\pgfqpoint{0.591784in}{1.480141in}}%
\pgfpathlineto{\pgfqpoint{0.559744in}{1.489057in}}%
\pgfpathlineto{\pgfqpoint{0.518578in}{1.500784in}}%
\pgfpathlineto{\pgfqpoint{0.452913in}{1.521452in}}%
\pgfpathlineto{\pgfqpoint{0.381532in}{1.543676in}}%
\pgfpathlineto{\pgfqpoint{0.363713in}{1.549891in}}%
\pgfpathclose%
\pgfusepath{fill}%
\end{pgfscope}%
\begin{pgfscope}%
\pgfpathrectangle{\pgfqpoint{0.100000in}{0.100000in}}{\pgfqpoint{2.989028in}{1.913466in}}%
\pgfusepath{clip}%
\pgfsetbuttcap%
\pgfsetmiterjoin%
\definecolor{currentfill}{rgb}{0.955786,0.982314,0.680046}%
\pgfsetfillcolor{currentfill}%
\pgfsetlinewidth{0.000000pt}%
\definecolor{currentstroke}{rgb}{0.000000,0.000000,0.000000}%
\pgfsetstrokecolor{currentstroke}%
\pgfsetstrokeopacity{0.000000}%
\pgfsetdash{}{0pt}%
\pgfpathmoveto{\pgfqpoint{2.837703in}{1.505030in}}%
\pgfpathlineto{\pgfqpoint{2.832504in}{1.509256in}}%
\pgfpathlineto{\pgfqpoint{2.833228in}{1.522426in}}%
\pgfpathlineto{\pgfqpoint{2.827854in}{1.559858in}}%
\pgfpathlineto{\pgfqpoint{2.832066in}{1.573021in}}%
\pgfpathlineto{\pgfqpoint{2.835075in}{1.596366in}}%
\pgfpathlineto{\pgfqpoint{2.831015in}{1.607411in}}%
\pgfpathlineto{\pgfqpoint{2.845785in}{1.622115in}}%
\pgfpathlineto{\pgfqpoint{2.849502in}{1.633812in}}%
\pgfpathlineto{\pgfqpoint{2.842504in}{1.643043in}}%
\pgfpathlineto{\pgfqpoint{2.845485in}{1.654580in}}%
\pgfpathlineto{\pgfqpoint{2.843616in}{1.661826in}}%
\pgfpathlineto{\pgfqpoint{2.845221in}{1.677346in}}%
\pgfpathlineto{\pgfqpoint{2.848221in}{1.682137in}}%
\pgfpathlineto{\pgfqpoint{2.857471in}{1.685161in}}%
\pgfpathlineto{\pgfqpoint{2.870964in}{1.645645in}}%
\pgfpathlineto{\pgfqpoint{2.895998in}{1.563760in}}%
\pgfpathlineto{\pgfqpoint{2.905471in}{1.557411in}}%
\pgfpathlineto{\pgfqpoint{2.907175in}{1.550163in}}%
\pgfpathlineto{\pgfqpoint{2.912175in}{1.547235in}}%
\pgfpathlineto{\pgfqpoint{2.911795in}{1.534090in}}%
\pgfpathlineto{\pgfqpoint{2.906497in}{1.533879in}}%
\pgfpathlineto{\pgfqpoint{2.895764in}{1.525694in}}%
\pgfpathlineto{\pgfqpoint{2.892669in}{1.517482in}}%
\pgfpathlineto{\pgfqpoint{2.863942in}{1.510374in}}%
\pgfpathlineto{\pgfqpoint{2.837703in}{1.505030in}}%
\pgfpathclose%
\pgfusepath{fill}%
\end{pgfscope}%
\begin{pgfscope}%
\pgfpathrectangle{\pgfqpoint{0.100000in}{0.100000in}}{\pgfqpoint{2.989028in}{1.913466in}}%
\pgfusepath{clip}%
\pgfsetbuttcap%
\pgfsetmiterjoin%
\definecolor{currentfill}{rgb}{0.936563,0.974625,0.650058}%
\pgfsetfillcolor{currentfill}%
\pgfsetlinewidth{0.000000pt}%
\definecolor{currentstroke}{rgb}{0.000000,0.000000,0.000000}%
\pgfsetstrokecolor{currentstroke}%
\pgfsetstrokeopacity{0.000000}%
\pgfsetdash{}{0pt}%
\pgfpathmoveto{\pgfqpoint{1.939431in}{1.220143in}}%
\pgfpathlineto{\pgfqpoint{1.924266in}{1.235161in}}%
\pgfpathlineto{\pgfqpoint{1.875789in}{1.232365in}}%
\pgfpathlineto{\pgfqpoint{1.800184in}{1.229874in}}%
\pgfpathlineto{\pgfqpoint{1.724124in}{1.231060in}}%
\pgfpathlineto{\pgfqpoint{1.718785in}{1.240370in}}%
\pgfpathlineto{\pgfqpoint{1.720965in}{1.249511in}}%
\pgfpathlineto{\pgfqpoint{1.719930in}{1.265170in}}%
\pgfpathlineto{\pgfqpoint{1.715910in}{1.280338in}}%
\pgfpathlineto{\pgfqpoint{1.716378in}{1.288348in}}%
\pgfpathlineto{\pgfqpoint{1.707833in}{1.297398in}}%
\pgfpathlineto{\pgfqpoint{1.709692in}{1.309990in}}%
\pgfpathlineto{\pgfqpoint{1.696585in}{1.334864in}}%
\pgfpathlineto{\pgfqpoint{1.692680in}{1.355981in}}%
\pgfpathlineto{\pgfqpoint{1.683252in}{1.373157in}}%
\pgfpathlineto{\pgfqpoint{1.692201in}{1.394464in}}%
\pgfpathlineto{\pgfqpoint{1.687744in}{1.404379in}}%
\pgfpathlineto{\pgfqpoint{1.692728in}{1.422304in}}%
\pgfpathlineto{\pgfqpoint{1.740155in}{1.421790in}}%
\pgfpathlineto{\pgfqpoint{1.788246in}{1.421717in}}%
\pgfpathlineto{\pgfqpoint{1.833705in}{1.422385in}}%
\pgfpathlineto{\pgfqpoint{1.883600in}{1.423428in}}%
\pgfpathlineto{\pgfqpoint{1.942134in}{1.425554in}}%
\pgfpathlineto{\pgfqpoint{1.943697in}{1.417056in}}%
\pgfpathlineto{\pgfqpoint{1.950178in}{1.409931in}}%
\pgfpathlineto{\pgfqpoint{1.946075in}{1.402507in}}%
\pgfpathlineto{\pgfqpoint{1.947756in}{1.388016in}}%
\pgfpathlineto{\pgfqpoint{1.951772in}{1.376303in}}%
\pgfpathlineto{\pgfqpoint{1.969980in}{1.370103in}}%
\pgfpathlineto{\pgfqpoint{1.973100in}{1.361809in}}%
\pgfpathlineto{\pgfqpoint{1.983091in}{1.352497in}}%
\pgfpathlineto{\pgfqpoint{1.987167in}{1.342864in}}%
\pgfpathlineto{\pgfqpoint{1.997291in}{1.336402in}}%
\pgfpathlineto{\pgfqpoint{1.998875in}{1.328623in}}%
\pgfpathlineto{\pgfqpoint{1.996904in}{1.316845in}}%
\pgfpathlineto{\pgfqpoint{1.991751in}{1.313315in}}%
\pgfpathlineto{\pgfqpoint{1.990197in}{1.302104in}}%
\pgfpathlineto{\pgfqpoint{1.975389in}{1.293200in}}%
\pgfpathlineto{\pgfqpoint{1.956087in}{1.288322in}}%
\pgfpathlineto{\pgfqpoint{1.954318in}{1.277106in}}%
\pgfpathlineto{\pgfqpoint{1.961764in}{1.269093in}}%
\pgfpathlineto{\pgfqpoint{1.962068in}{1.259018in}}%
\pgfpathlineto{\pgfqpoint{1.956060in}{1.251098in}}%
\pgfpathlineto{\pgfqpoint{1.952907in}{1.239332in}}%
\pgfpathlineto{\pgfqpoint{1.942465in}{1.235438in}}%
\pgfpathlineto{\pgfqpoint{1.943130in}{1.222326in}}%
\pgfpathlineto{\pgfqpoint{1.939431in}{1.220143in}}%
\pgfpathclose%
\pgfusepath{fill}%
\end{pgfscope}%
\begin{pgfscope}%
\pgfpathrectangle{\pgfqpoint{0.100000in}{0.100000in}}{\pgfqpoint{2.989028in}{1.913466in}}%
\pgfusepath{clip}%
\pgfsetbuttcap%
\pgfsetmiterjoin%
\definecolor{currentfill}{rgb}{0.711419,0.883276,0.634833}%
\pgfsetfillcolor{currentfill}%
\pgfsetlinewidth{0.000000pt}%
\definecolor{currentstroke}{rgb}{0.000000,0.000000,0.000000}%
\pgfsetstrokecolor{currentstroke}%
\pgfsetstrokeopacity{0.000000}%
\pgfsetdash{}{0pt}%
\pgfpathmoveto{\pgfqpoint{2.880364in}{1.467139in}}%
\pgfpathlineto{\pgfqpoint{2.866000in}{1.464872in}}%
\pgfpathlineto{\pgfqpoint{2.800022in}{1.449880in}}%
\pgfpathlineto{\pgfqpoint{2.798869in}{1.451627in}}%
\pgfpathlineto{\pgfqpoint{2.799865in}{1.496989in}}%
\pgfpathlineto{\pgfqpoint{2.837703in}{1.505030in}}%
\pgfpathlineto{\pgfqpoint{2.863942in}{1.510374in}}%
\pgfpathlineto{\pgfqpoint{2.892669in}{1.517482in}}%
\pgfpathlineto{\pgfqpoint{2.895764in}{1.525694in}}%
\pgfpathlineto{\pgfqpoint{2.906497in}{1.533879in}}%
\pgfpathlineto{\pgfqpoint{2.911795in}{1.534090in}}%
\pgfpathlineto{\pgfqpoint{2.918759in}{1.522149in}}%
\pgfpathlineto{\pgfqpoint{2.912440in}{1.504717in}}%
\pgfpathlineto{\pgfqpoint{2.911512in}{1.494495in}}%
\pgfpathlineto{\pgfqpoint{2.924304in}{1.495455in}}%
\pgfpathlineto{\pgfqpoint{2.930105in}{1.490552in}}%
\pgfpathlineto{\pgfqpoint{2.943114in}{1.470509in}}%
\pgfpathlineto{\pgfqpoint{2.949579in}{1.468072in}}%
\pgfpathlineto{\pgfqpoint{2.960410in}{1.468976in}}%
\pgfpathlineto{\pgfqpoint{2.967946in}{1.475787in}}%
\pgfpathlineto{\pgfqpoint{2.972956in}{1.469708in}}%
\pgfpathlineto{\pgfqpoint{2.941451in}{1.453083in}}%
\pgfpathlineto{\pgfqpoint{2.940467in}{1.465013in}}%
\pgfpathlineto{\pgfqpoint{2.926155in}{1.446471in}}%
\pgfpathlineto{\pgfqpoint{2.921123in}{1.443277in}}%
\pgfpathlineto{\pgfqpoint{2.914105in}{1.453976in}}%
\pgfpathlineto{\pgfqpoint{2.912183in}{1.455442in}}%
\pgfpathlineto{\pgfqpoint{2.905650in}{1.458903in}}%
\pgfpathlineto{\pgfqpoint{2.899952in}{1.472940in}}%
\pgfpathlineto{\pgfqpoint{2.880364in}{1.467139in}}%
\pgfpathclose%
\pgfusepath{fill}%
\end{pgfscope}%
\begin{pgfscope}%
\pgfpathrectangle{\pgfqpoint{0.100000in}{0.100000in}}{\pgfqpoint{2.989028in}{1.913466in}}%
\pgfusepath{clip}%
\pgfsetbuttcap%
\pgfsetmiterjoin%
\definecolor{currentfill}{rgb}{0.256055,0.600231,0.713495}%
\pgfsetfillcolor{currentfill}%
\pgfsetlinewidth{0.000000pt}%
\definecolor{currentstroke}{rgb}{0.000000,0.000000,0.000000}%
\pgfsetstrokecolor{currentstroke}%
\pgfsetstrokeopacity{0.000000}%
\pgfsetdash{}{0pt}%
\pgfpathmoveto{\pgfqpoint{1.408935in}{1.205970in}}%
\pgfpathlineto{\pgfqpoint{1.414195in}{1.271196in}}%
\pgfpathlineto{\pgfqpoint{1.384404in}{1.273614in}}%
\pgfpathlineto{\pgfqpoint{1.315540in}{1.280295in}}%
\pgfpathlineto{\pgfqpoint{1.321947in}{1.345289in}}%
\pgfpathlineto{\pgfqpoint{1.328363in}{1.410780in}}%
\pgfpathlineto{\pgfqpoint{1.354836in}{1.407921in}}%
\pgfpathlineto{\pgfqpoint{1.422447in}{1.401533in}}%
\pgfpathlineto{\pgfqpoint{1.463365in}{1.398329in}}%
\pgfpathlineto{\pgfqpoint{1.512902in}{1.395489in}}%
\pgfpathlineto{\pgfqpoint{1.594439in}{1.391630in}}%
\pgfpathlineto{\pgfqpoint{1.596231in}{1.387975in}}%
\pgfpathlineto{\pgfqpoint{1.619594in}{1.376416in}}%
\pgfpathlineto{\pgfqpoint{1.626759in}{1.382286in}}%
\pgfpathlineto{\pgfqpoint{1.647400in}{1.381750in}}%
\pgfpathlineto{\pgfqpoint{1.659700in}{1.375307in}}%
\pgfpathlineto{\pgfqpoint{1.679205in}{1.367982in}}%
\pgfpathlineto{\pgfqpoint{1.684670in}{1.357295in}}%
\pgfpathlineto{\pgfqpoint{1.692680in}{1.355981in}}%
\pgfpathlineto{\pgfqpoint{1.696585in}{1.334864in}}%
\pgfpathlineto{\pgfqpoint{1.709692in}{1.309990in}}%
\pgfpathlineto{\pgfqpoint{1.707833in}{1.297398in}}%
\pgfpathlineto{\pgfqpoint{1.716378in}{1.288348in}}%
\pgfpathlineto{\pgfqpoint{1.715910in}{1.280338in}}%
\pgfpathlineto{\pgfqpoint{1.719930in}{1.265170in}}%
\pgfpathlineto{\pgfqpoint{1.720965in}{1.249511in}}%
\pgfpathlineto{\pgfqpoint{1.718785in}{1.240370in}}%
\pgfpathlineto{\pgfqpoint{1.724124in}{1.231060in}}%
\pgfpathlineto{\pgfqpoint{1.731447in}{1.214132in}}%
\pgfpathlineto{\pgfqpoint{1.738455in}{1.207242in}}%
\pgfpathlineto{\pgfqpoint{1.746810in}{1.192312in}}%
\pgfpathlineto{\pgfqpoint{1.723131in}{1.192066in}}%
\pgfpathlineto{\pgfqpoint{1.671933in}{1.192859in}}%
\pgfpathlineto{\pgfqpoint{1.615367in}{1.194592in}}%
\pgfpathlineto{\pgfqpoint{1.558468in}{1.196775in}}%
\pgfpathlineto{\pgfqpoint{1.474809in}{1.201345in}}%
\pgfpathlineto{\pgfqpoint{1.408935in}{1.205970in}}%
\pgfpathclose%
\pgfusepath{fill}%
\end{pgfscope}%
\begin{pgfscope}%
\pgfpathrectangle{\pgfqpoint{0.100000in}{0.100000in}}{\pgfqpoint{2.989028in}{1.913466in}}%
\pgfusepath{clip}%
\pgfsetbuttcap%
\pgfsetmiterjoin%
\definecolor{currentfill}{rgb}{0.975010,0.990004,0.710035}%
\pgfsetfillcolor{currentfill}%
\pgfsetlinewidth{0.000000pt}%
\definecolor{currentstroke}{rgb}{0.000000,0.000000,0.000000}%
\pgfsetstrokecolor{currentstroke}%
\pgfsetstrokeopacity{0.000000}%
\pgfsetdash{}{0pt}%
\pgfpathmoveto{\pgfqpoint{2.498213in}{1.402714in}}%
\pgfpathlineto{\pgfqpoint{2.524409in}{1.427692in}}%
\pgfpathlineto{\pgfqpoint{2.527665in}{1.436570in}}%
\pgfpathlineto{\pgfqpoint{2.535341in}{1.444171in}}%
\pgfpathlineto{\pgfqpoint{2.529581in}{1.455238in}}%
\pgfpathlineto{\pgfqpoint{2.522360in}{1.461711in}}%
\pgfpathlineto{\pgfqpoint{2.520273in}{1.473182in}}%
\pgfpathlineto{\pgfqpoint{2.547151in}{1.484959in}}%
\pgfpathlineto{\pgfqpoint{2.569405in}{1.488685in}}%
\pgfpathlineto{\pgfqpoint{2.581363in}{1.488934in}}%
\pgfpathlineto{\pgfqpoint{2.590473in}{1.484433in}}%
\pgfpathlineto{\pgfqpoint{2.599352in}{1.488453in}}%
\pgfpathlineto{\pgfqpoint{2.621010in}{1.492956in}}%
\pgfpathlineto{\pgfqpoint{2.628493in}{1.498769in}}%
\pgfpathlineto{\pgfqpoint{2.639589in}{1.511638in}}%
\pgfpathlineto{\pgfqpoint{2.649683in}{1.517325in}}%
\pgfpathlineto{\pgfqpoint{2.650395in}{1.522779in}}%
\pgfpathlineto{\pgfqpoint{2.645097in}{1.535222in}}%
\pgfpathlineto{\pgfqpoint{2.648927in}{1.542544in}}%
\pgfpathlineto{\pgfqpoint{2.643770in}{1.550418in}}%
\pgfpathlineto{\pgfqpoint{2.635873in}{1.550968in}}%
\pgfpathlineto{\pgfqpoint{2.655638in}{1.574751in}}%
\pgfpathlineto{\pgfqpoint{2.657985in}{1.583815in}}%
\pgfpathlineto{\pgfqpoint{2.673541in}{1.606931in}}%
\pgfpathlineto{\pgfqpoint{2.687981in}{1.619440in}}%
\pgfpathlineto{\pgfqpoint{2.697897in}{1.624664in}}%
\pgfpathlineto{\pgfqpoint{2.730342in}{1.632016in}}%
\pgfpathlineto{\pgfqpoint{2.760752in}{1.640318in}}%
\pgfpathlineto{\pgfqpoint{2.764780in}{1.627285in}}%
\pgfpathlineto{\pgfqpoint{2.765065in}{1.615202in}}%
\pgfpathlineto{\pgfqpoint{2.772002in}{1.604233in}}%
\pgfpathlineto{\pgfqpoint{2.774140in}{1.593064in}}%
\pgfpathlineto{\pgfqpoint{2.771474in}{1.577911in}}%
\pgfpathlineto{\pgfqpoint{2.779593in}{1.557473in}}%
\pgfpathlineto{\pgfqpoint{2.788848in}{1.546108in}}%
\pgfpathlineto{\pgfqpoint{2.796007in}{1.509167in}}%
\pgfpathlineto{\pgfqpoint{2.799865in}{1.496989in}}%
\pgfpathlineto{\pgfqpoint{2.798869in}{1.451627in}}%
\pgfpathlineto{\pgfqpoint{2.800022in}{1.449880in}}%
\pgfpathlineto{\pgfqpoint{2.808450in}{1.401092in}}%
\pgfpathlineto{\pgfqpoint{2.813176in}{1.396647in}}%
\pgfpathlineto{\pgfqpoint{2.803010in}{1.386769in}}%
\pgfpathlineto{\pgfqpoint{2.808026in}{1.381100in}}%
\pgfpathlineto{\pgfqpoint{2.803617in}{1.372526in}}%
\pgfpathlineto{\pgfqpoint{2.803654in}{1.368875in}}%
\pgfpathlineto{\pgfqpoint{2.798142in}{1.365581in}}%
\pgfpathlineto{\pgfqpoint{2.795458in}{1.358300in}}%
\pgfpathlineto{\pgfqpoint{2.796310in}{1.378322in}}%
\pgfpathlineto{\pgfqpoint{2.779225in}{1.382730in}}%
\pgfpathlineto{\pgfqpoint{2.752507in}{1.391844in}}%
\pgfpathlineto{\pgfqpoint{2.748728in}{1.396333in}}%
\pgfpathlineto{\pgfqpoint{2.737564in}{1.397408in}}%
\pgfpathlineto{\pgfqpoint{2.731142in}{1.404090in}}%
\pgfpathlineto{\pgfqpoint{2.727676in}{1.417397in}}%
\pgfpathlineto{\pgfqpoint{2.718572in}{1.419052in}}%
\pgfpathlineto{\pgfqpoint{2.712484in}{1.426033in}}%
\pgfpathlineto{\pgfqpoint{2.634972in}{1.410406in}}%
\pgfpathlineto{\pgfqpoint{2.597928in}{1.402745in}}%
\pgfpathlineto{\pgfqpoint{2.541683in}{1.392479in}}%
\pgfpathlineto{\pgfqpoint{2.501180in}{1.385646in}}%
\pgfpathlineto{\pgfqpoint{2.498213in}{1.402714in}}%
\pgfpathclose%
\pgfusepath{fill}%
\end{pgfscope}%
\begin{pgfscope}%
\pgfpathrectangle{\pgfqpoint{0.100000in}{0.100000in}}{\pgfqpoint{2.989028in}{1.913466in}}%
\pgfusepath{clip}%
\pgfsetbuttcap%
\pgfsetmiterjoin%
\definecolor{currentfill}{rgb}{0.975010,0.990004,0.710035}%
\pgfsetfillcolor{currentfill}%
\pgfsetlinewidth{0.000000pt}%
\definecolor{currentstroke}{rgb}{0.000000,0.000000,0.000000}%
\pgfsetstrokecolor{currentstroke}%
\pgfsetstrokeopacity{0.000000}%
\pgfsetdash{}{0pt}%
\pgfpathmoveto{\pgfqpoint{2.809390in}{1.354233in}}%
\pgfpathlineto{\pgfqpoint{2.797403in}{1.350498in}}%
\pgfpathlineto{\pgfqpoint{2.795386in}{1.353932in}}%
\pgfpathlineto{\pgfqpoint{2.799234in}{1.365456in}}%
\pgfpathlineto{\pgfqpoint{2.806360in}{1.366591in}}%
\pgfpathlineto{\pgfqpoint{2.805722in}{1.370219in}}%
\pgfpathlineto{\pgfqpoint{2.812120in}{1.375665in}}%
\pgfpathlineto{\pgfqpoint{2.830618in}{1.380004in}}%
\pgfpathlineto{\pgfqpoint{2.838853in}{1.386571in}}%
\pgfpathlineto{\pgfqpoint{2.857353in}{1.392045in}}%
\pgfpathlineto{\pgfqpoint{2.865790in}{1.390042in}}%
\pgfpathlineto{\pgfqpoint{2.872888in}{1.398866in}}%
\pgfpathlineto{\pgfqpoint{2.883615in}{1.400032in}}%
\pgfpathlineto{\pgfqpoint{2.865314in}{1.382820in}}%
\pgfpathlineto{\pgfqpoint{2.809390in}{1.354233in}}%
\pgfpathclose%
\pgfusepath{fill}%
\end{pgfscope}%
\begin{pgfscope}%
\pgfpathrectangle{\pgfqpoint{0.100000in}{0.100000in}}{\pgfqpoint{2.989028in}{1.913466in}}%
\pgfusepath{clip}%
\pgfsetbuttcap%
\pgfsetmiterjoin%
\definecolor{currentfill}{rgb}{0.793080,0.916494,0.618224}%
\pgfsetfillcolor{currentfill}%
\pgfsetlinewidth{0.000000pt}%
\definecolor{currentstroke}{rgb}{0.000000,0.000000,0.000000}%
\pgfsetstrokecolor{currentstroke}%
\pgfsetstrokeopacity{0.000000}%
\pgfsetdash{}{0pt}%
\pgfpathmoveto{\pgfqpoint{2.540072in}{1.240581in}}%
\pgfpathlineto{\pgfqpoint{2.488234in}{1.232010in}}%
\pgfpathlineto{\pgfqpoint{2.478831in}{1.291252in}}%
\pgfpathlineto{\pgfqpoint{2.464869in}{1.378577in}}%
\pgfpathlineto{\pgfqpoint{2.498213in}{1.402714in}}%
\pgfpathlineto{\pgfqpoint{2.501180in}{1.385646in}}%
\pgfpathlineto{\pgfqpoint{2.541683in}{1.392479in}}%
\pgfpathlineto{\pgfqpoint{2.597928in}{1.402745in}}%
\pgfpathlineto{\pgfqpoint{2.634972in}{1.410406in}}%
\pgfpathlineto{\pgfqpoint{2.712484in}{1.426033in}}%
\pgfpathlineto{\pgfqpoint{2.718572in}{1.419052in}}%
\pgfpathlineto{\pgfqpoint{2.727676in}{1.417397in}}%
\pgfpathlineto{\pgfqpoint{2.731142in}{1.404090in}}%
\pgfpathlineto{\pgfqpoint{2.737564in}{1.397408in}}%
\pgfpathlineto{\pgfqpoint{2.748728in}{1.396333in}}%
\pgfpathlineto{\pgfqpoint{2.752507in}{1.391844in}}%
\pgfpathlineto{\pgfqpoint{2.748671in}{1.388380in}}%
\pgfpathlineto{\pgfqpoint{2.745215in}{1.376126in}}%
\pgfpathlineto{\pgfqpoint{2.737116in}{1.362357in}}%
\pgfpathlineto{\pgfqpoint{2.742552in}{1.356369in}}%
\pgfpathlineto{\pgfqpoint{2.738104in}{1.349953in}}%
\pgfpathlineto{\pgfqpoint{2.740613in}{1.335884in}}%
\pgfpathlineto{\pgfqpoint{2.748625in}{1.328499in}}%
\pgfpathlineto{\pgfqpoint{2.767634in}{1.316500in}}%
\pgfpathlineto{\pgfqpoint{2.752326in}{1.299570in}}%
\pgfpathlineto{\pgfqpoint{2.752111in}{1.293145in}}%
\pgfpathlineto{\pgfqpoint{2.739648in}{1.284858in}}%
\pgfpathlineto{\pgfqpoint{2.725867in}{1.283302in}}%
\pgfpathlineto{\pgfqpoint{2.722458in}{1.276100in}}%
\pgfpathlineto{\pgfqpoint{2.684127in}{1.267806in}}%
\pgfpathlineto{\pgfqpoint{2.639399in}{1.258797in}}%
\pgfpathlineto{\pgfqpoint{2.608655in}{1.253292in}}%
\pgfpathlineto{\pgfqpoint{2.540072in}{1.240581in}}%
\pgfpathclose%
\pgfusepath{fill}%
\end{pgfscope}%
\begin{pgfscope}%
\pgfpathrectangle{\pgfqpoint{0.100000in}{0.100000in}}{\pgfqpoint{2.989028in}{1.913466in}}%
\pgfusepath{clip}%
\pgfsetbuttcap%
\pgfsetmiterjoin%
\definecolor{currentfill}{rgb}{0.951942,0.980777,0.674048}%
\pgfsetfillcolor{currentfill}%
\pgfsetlinewidth{0.000000pt}%
\definecolor{currentstroke}{rgb}{0.000000,0.000000,0.000000}%
\pgfsetstrokecolor{currentstroke}%
\pgfsetstrokeopacity{0.000000}%
\pgfsetdash{}{0pt}%
\pgfpathmoveto{\pgfqpoint{2.800022in}{1.449880in}}%
\pgfpathlineto{\pgfqpoint{2.866000in}{1.464872in}}%
\pgfpathlineto{\pgfqpoint{2.880364in}{1.467139in}}%
\pgfpathlineto{\pgfqpoint{2.889871in}{1.429760in}}%
\pgfpathlineto{\pgfqpoint{2.888366in}{1.423067in}}%
\pgfpathlineto{\pgfqpoint{2.857827in}{1.411244in}}%
\pgfpathlineto{\pgfqpoint{2.839596in}{1.407111in}}%
\pgfpathlineto{\pgfqpoint{2.831849in}{1.397842in}}%
\pgfpathlineto{\pgfqpoint{2.808026in}{1.381100in}}%
\pgfpathlineto{\pgfqpoint{2.803010in}{1.386769in}}%
\pgfpathlineto{\pgfqpoint{2.813176in}{1.396647in}}%
\pgfpathlineto{\pgfqpoint{2.808450in}{1.401092in}}%
\pgfpathlineto{\pgfqpoint{2.800022in}{1.449880in}}%
\pgfpathclose%
\pgfusepath{fill}%
\end{pgfscope}%
\begin{pgfscope}%
\pgfpathrectangle{\pgfqpoint{0.100000in}{0.100000in}}{\pgfqpoint{2.989028in}{1.913466in}}%
\pgfusepath{clip}%
\pgfsetbuttcap%
\pgfsetmiterjoin%
\definecolor{currentfill}{rgb}{0.959631,0.983852,0.686044}%
\pgfsetfillcolor{currentfill}%
\pgfsetlinewidth{0.000000pt}%
\definecolor{currentstroke}{rgb}{0.000000,0.000000,0.000000}%
\pgfsetstrokecolor{currentstroke}%
\pgfsetstrokeopacity{0.000000}%
\pgfsetdash{}{0pt}%
\pgfpathmoveto{\pgfqpoint{2.880364in}{1.467139in}}%
\pgfpathlineto{\pgfqpoint{2.899952in}{1.472940in}}%
\pgfpathlineto{\pgfqpoint{2.905650in}{1.458903in}}%
\pgfpathlineto{\pgfqpoint{2.912183in}{1.455442in}}%
\pgfpathlineto{\pgfqpoint{2.904121in}{1.449529in}}%
\pgfpathlineto{\pgfqpoint{2.905138in}{1.432164in}}%
\pgfpathlineto{\pgfqpoint{2.888366in}{1.423067in}}%
\pgfpathlineto{\pgfqpoint{2.889871in}{1.429760in}}%
\pgfpathlineto{\pgfqpoint{2.880364in}{1.467139in}}%
\pgfpathclose%
\pgfusepath{fill}%
\end{pgfscope}%
\begin{pgfscope}%
\pgfpathrectangle{\pgfqpoint{0.100000in}{0.100000in}}{\pgfqpoint{2.989028in}{1.913466in}}%
\pgfusepath{clip}%
\pgfsetbuttcap%
\pgfsetmiterjoin%
\definecolor{currentfill}{rgb}{0.793080,0.916494,0.618224}%
\pgfsetfillcolor{currentfill}%
\pgfsetlinewidth{0.000000pt}%
\definecolor{currentstroke}{rgb}{0.000000,0.000000,0.000000}%
\pgfsetstrokecolor{currentstroke}%
\pgfsetstrokeopacity{0.000000}%
\pgfsetdash{}{0pt}%
\pgfpathmoveto{\pgfqpoint{2.737440in}{1.278737in}}%
\pgfpathlineto{\pgfqpoint{2.739648in}{1.284858in}}%
\pgfpathlineto{\pgfqpoint{2.752111in}{1.293145in}}%
\pgfpathlineto{\pgfqpoint{2.752326in}{1.299570in}}%
\pgfpathlineto{\pgfqpoint{2.767634in}{1.316500in}}%
\pgfpathlineto{\pgfqpoint{2.748625in}{1.328499in}}%
\pgfpathlineto{\pgfqpoint{2.740613in}{1.335884in}}%
\pgfpathlineto{\pgfqpoint{2.738104in}{1.349953in}}%
\pgfpathlineto{\pgfqpoint{2.742552in}{1.356369in}}%
\pgfpathlineto{\pgfqpoint{2.737116in}{1.362357in}}%
\pgfpathlineto{\pgfqpoint{2.745215in}{1.376126in}}%
\pgfpathlineto{\pgfqpoint{2.748671in}{1.388380in}}%
\pgfpathlineto{\pgfqpoint{2.752507in}{1.391844in}}%
\pgfpathlineto{\pgfqpoint{2.779225in}{1.382730in}}%
\pgfpathlineto{\pgfqpoint{2.796310in}{1.378322in}}%
\pgfpathlineto{\pgfqpoint{2.795458in}{1.358300in}}%
\pgfpathlineto{\pgfqpoint{2.785080in}{1.343116in}}%
\pgfpathlineto{\pgfqpoint{2.793632in}{1.340882in}}%
\pgfpathlineto{\pgfqpoint{2.802511in}{1.334366in}}%
\pgfpathlineto{\pgfqpoint{2.803034in}{1.316545in}}%
\pgfpathlineto{\pgfqpoint{2.800347in}{1.303928in}}%
\pgfpathlineto{\pgfqpoint{2.802135in}{1.293566in}}%
\pgfpathlineto{\pgfqpoint{2.786701in}{1.258597in}}%
\pgfpathlineto{\pgfqpoint{2.781154in}{1.242268in}}%
\pgfpathlineto{\pgfqpoint{2.773409in}{1.250205in}}%
\pgfpathlineto{\pgfqpoint{2.763139in}{1.248853in}}%
\pgfpathlineto{\pgfqpoint{2.737435in}{1.263703in}}%
\pgfpathlineto{\pgfqpoint{2.734805in}{1.271656in}}%
\pgfpathlineto{\pgfqpoint{2.737440in}{1.278737in}}%
\pgfpathclose%
\pgfusepath{fill}%
\end{pgfscope}%
\begin{pgfscope}%
\pgfpathrectangle{\pgfqpoint{0.100000in}{0.100000in}}{\pgfqpoint{2.989028in}{1.913466in}}%
\pgfusepath{clip}%
\pgfsetbuttcap%
\pgfsetmiterjoin%
\definecolor{currentfill}{rgb}{0.913495,0.965398,0.614072}%
\pgfsetfillcolor{currentfill}%
\pgfsetlinewidth{0.000000pt}%
\definecolor{currentstroke}{rgb}{0.000000,0.000000,0.000000}%
\pgfsetstrokecolor{currentstroke}%
\pgfsetstrokeopacity{0.000000}%
\pgfsetdash{}{0pt}%
\pgfpathmoveto{\pgfqpoint{2.122290in}{1.062454in}}%
\pgfpathlineto{\pgfqpoint{2.121088in}{1.078676in}}%
\pgfpathlineto{\pgfqpoint{2.125759in}{1.086724in}}%
\pgfpathlineto{\pgfqpoint{2.122681in}{1.090668in}}%
\pgfpathlineto{\pgfqpoint{2.133907in}{1.103580in}}%
\pgfpathlineto{\pgfqpoint{2.133263in}{1.106692in}}%
\pgfpathlineto{\pgfqpoint{2.144170in}{1.129219in}}%
\pgfpathlineto{\pgfqpoint{2.141966in}{1.140083in}}%
\pgfpathlineto{\pgfqpoint{2.134057in}{1.151457in}}%
\pgfpathlineto{\pgfqpoint{2.139547in}{1.165295in}}%
\pgfpathlineto{\pgfqpoint{2.132431in}{1.256358in}}%
\pgfpathlineto{\pgfqpoint{2.127306in}{1.320243in}}%
\pgfpathlineto{\pgfqpoint{2.134385in}{1.314949in}}%
\pgfpathlineto{\pgfqpoint{2.142283in}{1.315093in}}%
\pgfpathlineto{\pgfqpoint{2.160936in}{1.325898in}}%
\pgfpathlineto{\pgfqpoint{2.218150in}{1.331236in}}%
\pgfpathlineto{\pgfqpoint{2.260498in}{1.335699in}}%
\pgfpathlineto{\pgfqpoint{2.260872in}{1.331556in}}%
\pgfpathlineto{\pgfqpoint{2.270539in}{1.244011in}}%
\pgfpathlineto{\pgfqpoint{2.278860in}{1.162554in}}%
\pgfpathlineto{\pgfqpoint{2.275273in}{1.158730in}}%
\pgfpathlineto{\pgfqpoint{2.280783in}{1.149238in}}%
\pgfpathlineto{\pgfqpoint{2.280755in}{1.142397in}}%
\pgfpathlineto{\pgfqpoint{2.272893in}{1.140679in}}%
\pgfpathlineto{\pgfqpoint{2.264096in}{1.134084in}}%
\pgfpathlineto{\pgfqpoint{2.258139in}{1.136680in}}%
\pgfpathlineto{\pgfqpoint{2.249225in}{1.132459in}}%
\pgfpathlineto{\pgfqpoint{2.251983in}{1.123981in}}%
\pgfpathlineto{\pgfqpoint{2.242820in}{1.115463in}}%
\pgfpathlineto{\pgfqpoint{2.240290in}{1.105610in}}%
\pgfpathlineto{\pgfqpoint{2.233989in}{1.103993in}}%
\pgfpathlineto{\pgfqpoint{2.229287in}{1.096531in}}%
\pgfpathlineto{\pgfqpoint{2.229903in}{1.089016in}}%
\pgfpathlineto{\pgfqpoint{2.224371in}{1.083732in}}%
\pgfpathlineto{\pgfqpoint{2.216048in}{1.084548in}}%
\pgfpathlineto{\pgfqpoint{2.209721in}{1.092656in}}%
\pgfpathlineto{\pgfqpoint{2.199000in}{1.084848in}}%
\pgfpathlineto{\pgfqpoint{2.199753in}{1.078025in}}%
\pgfpathlineto{\pgfqpoint{2.187829in}{1.074039in}}%
\pgfpathlineto{\pgfqpoint{2.184834in}{1.079056in}}%
\pgfpathlineto{\pgfqpoint{2.173226in}{1.073367in}}%
\pgfpathlineto{\pgfqpoint{2.168985in}{1.065222in}}%
\pgfpathlineto{\pgfqpoint{2.155006in}{1.073574in}}%
\pgfpathlineto{\pgfqpoint{2.132850in}{1.068041in}}%
\pgfpathlineto{\pgfqpoint{2.122290in}{1.062454in}}%
\pgfpathclose%
\pgfusepath{fill}%
\end{pgfscope}%
\begin{pgfscope}%
\pgfpathrectangle{\pgfqpoint{0.100000in}{0.100000in}}{\pgfqpoint{2.989028in}{1.913466in}}%
\pgfusepath{clip}%
\pgfsetbuttcap%
\pgfsetmiterjoin%
\definecolor{currentfill}{rgb}{0.865667,0.946021,0.603460}%
\pgfsetfillcolor{currentfill}%
\pgfsetlinewidth{0.000000pt}%
\definecolor{currentstroke}{rgb}{0.000000,0.000000,0.000000}%
\pgfsetstrokecolor{currentstroke}%
\pgfsetstrokeopacity{0.000000}%
\pgfsetdash{}{0pt}%
\pgfpathmoveto{\pgfqpoint{0.559744in}{1.489057in}}%
\pgfpathlineto{\pgfqpoint{0.591784in}{1.480141in}}%
\pgfpathlineto{\pgfqpoint{0.644893in}{1.466447in}}%
\pgfpathlineto{\pgfqpoint{0.699946in}{1.452341in}}%
\pgfpathlineto{\pgfqpoint{0.750689in}{1.440103in}}%
\pgfpathlineto{\pgfqpoint{0.794534in}{1.430177in}}%
\pgfpathlineto{\pgfqpoint{0.841442in}{1.419862in}}%
\pgfpathlineto{\pgfqpoint{0.827889in}{1.355901in}}%
\pgfpathlineto{\pgfqpoint{0.815815in}{1.299101in}}%
\pgfpathlineto{\pgfqpoint{0.796008in}{1.207454in}}%
\pgfpathlineto{\pgfqpoint{0.781137in}{1.138280in}}%
\pgfpathlineto{\pgfqpoint{0.773103in}{1.099678in}}%
\pgfpathlineto{\pgfqpoint{0.762801in}{1.049576in}}%
\pgfpathlineto{\pgfqpoint{0.755674in}{1.039412in}}%
\pgfpathlineto{\pgfqpoint{0.749938in}{1.039072in}}%
\pgfpathlineto{\pgfqpoint{0.745846in}{1.047945in}}%
\pgfpathlineto{\pgfqpoint{0.736451in}{1.051167in}}%
\pgfpathlineto{\pgfqpoint{0.726328in}{1.050037in}}%
\pgfpathlineto{\pgfqpoint{0.723454in}{1.042780in}}%
\pgfpathlineto{\pgfqpoint{0.723155in}{1.017578in}}%
\pgfpathlineto{\pgfqpoint{0.720123in}{1.011825in}}%
\pgfpathlineto{\pgfqpoint{0.721856in}{0.991592in}}%
\pgfpathlineto{\pgfqpoint{0.715525in}{0.978113in}}%
\pgfpathlineto{\pgfqpoint{0.664188in}{1.057164in}}%
\pgfpathlineto{\pgfqpoint{0.613630in}{1.133980in}}%
\pgfpathlineto{\pgfqpoint{0.587312in}{1.174260in}}%
\pgfpathlineto{\pgfqpoint{0.565358in}{1.209028in}}%
\pgfpathlineto{\pgfqpoint{0.537695in}{1.252036in}}%
\pgfpathlineto{\pgfqpoint{0.506381in}{1.300279in}}%
\pgfpathlineto{\pgfqpoint{0.519255in}{1.346069in}}%
\pgfpathlineto{\pgfqpoint{0.545163in}{1.437904in}}%
\pgfpathlineto{\pgfqpoint{0.559744in}{1.489057in}}%
\pgfpathclose%
\pgfusepath{fill}%
\end{pgfscope}%
\begin{pgfscope}%
\pgfpathrectangle{\pgfqpoint{0.100000in}{0.100000in}}{\pgfqpoint{2.989028in}{1.913466in}}%
\pgfusepath{clip}%
\pgfsetbuttcap%
\pgfsetmiterjoin%
\definecolor{currentfill}{rgb}{0.774933,0.909112,0.621915}%
\pgfsetfillcolor{currentfill}%
\pgfsetlinewidth{0.000000pt}%
\definecolor{currentstroke}{rgb}{0.000000,0.000000,0.000000}%
\pgfsetstrokecolor{currentstroke}%
\pgfsetstrokeopacity{0.000000}%
\pgfsetdash{}{0pt}%
\pgfpathmoveto{\pgfqpoint{0.841442in}{1.419862in}}%
\pgfpathlineto{\pgfqpoint{0.891592in}{1.409886in}}%
\pgfpathlineto{\pgfqpoint{0.984436in}{1.391963in}}%
\pgfpathlineto{\pgfqpoint{0.972815in}{1.327496in}}%
\pgfpathlineto{\pgfqpoint{1.023928in}{1.318744in}}%
\pgfpathlineto{\pgfqpoint{1.070495in}{1.311293in}}%
\pgfpathlineto{\pgfqpoint{1.062404in}{1.260326in}}%
\pgfpathlineto{\pgfqpoint{1.053829in}{1.205365in}}%
\pgfpathlineto{\pgfqpoint{1.042350in}{1.133200in}}%
\pgfpathlineto{\pgfqpoint{1.042053in}{1.127151in}}%
\pgfpathlineto{\pgfqpoint{1.030137in}{1.052362in}}%
\pgfpathlineto{\pgfqpoint{0.981086in}{1.059964in}}%
\pgfpathlineto{\pgfqpoint{0.956091in}{1.064984in}}%
\pgfpathlineto{\pgfqpoint{0.865746in}{1.080865in}}%
\pgfpathlineto{\pgfqpoint{0.831724in}{1.087575in}}%
\pgfpathlineto{\pgfqpoint{0.773103in}{1.099678in}}%
\pgfpathlineto{\pgfqpoint{0.781137in}{1.138280in}}%
\pgfpathlineto{\pgfqpoint{0.796008in}{1.207454in}}%
\pgfpathlineto{\pgfqpoint{0.815815in}{1.299101in}}%
\pgfpathlineto{\pgfqpoint{0.827889in}{1.355901in}}%
\pgfpathlineto{\pgfqpoint{0.841442in}{1.419862in}}%
\pgfpathclose%
\pgfusepath{fill}%
\end{pgfscope}%
\begin{pgfscope}%
\pgfpathrectangle{\pgfqpoint{0.100000in}{0.100000in}}{\pgfqpoint{2.989028in}{1.913466in}}%
\pgfusepath{clip}%
\pgfsetbuttcap%
\pgfsetmiterjoin%
\definecolor{currentfill}{rgb}{0.729566,0.890657,0.631142}%
\pgfsetfillcolor{currentfill}%
\pgfsetlinewidth{0.000000pt}%
\definecolor{currentstroke}{rgb}{0.000000,0.000000,0.000000}%
\pgfsetstrokecolor{currentstroke}%
\pgfsetstrokeopacity{0.000000}%
\pgfsetdash{}{0pt}%
\pgfpathmoveto{\pgfqpoint{0.363713in}{1.549891in}}%
\pgfpathlineto{\pgfqpoint{0.381532in}{1.543676in}}%
\pgfpathlineto{\pgfqpoint{0.452913in}{1.521452in}}%
\pgfpathlineto{\pgfqpoint{0.518578in}{1.500784in}}%
\pgfpathlineto{\pgfqpoint{0.559744in}{1.489057in}}%
\pgfpathlineto{\pgfqpoint{0.545163in}{1.437904in}}%
\pgfpathlineto{\pgfqpoint{0.519255in}{1.346069in}}%
\pgfpathlineto{\pgfqpoint{0.506381in}{1.300279in}}%
\pgfpathlineto{\pgfqpoint{0.537695in}{1.252036in}}%
\pgfpathlineto{\pgfqpoint{0.565358in}{1.209028in}}%
\pgfpathlineto{\pgfqpoint{0.587312in}{1.174260in}}%
\pgfpathlineto{\pgfqpoint{0.613630in}{1.133980in}}%
\pgfpathlineto{\pgfqpoint{0.664188in}{1.057164in}}%
\pgfpathlineto{\pgfqpoint{0.715525in}{0.978113in}}%
\pgfpathlineto{\pgfqpoint{0.713464in}{0.970274in}}%
\pgfpathlineto{\pgfqpoint{0.719656in}{0.957789in}}%
\pgfpathlineto{\pgfqpoint{0.720881in}{0.940711in}}%
\pgfpathlineto{\pgfqpoint{0.729874in}{0.932426in}}%
\pgfpathlineto{\pgfqpoint{0.730230in}{0.925750in}}%
\pgfpathlineto{\pgfqpoint{0.714127in}{0.918174in}}%
\pgfpathlineto{\pgfqpoint{0.706455in}{0.910589in}}%
\pgfpathlineto{\pgfqpoint{0.704063in}{0.893839in}}%
\pgfpathlineto{\pgfqpoint{0.700180in}{0.884704in}}%
\pgfpathlineto{\pgfqpoint{0.691998in}{0.877004in}}%
\pgfpathlineto{\pgfqpoint{0.683861in}{0.856982in}}%
\pgfpathlineto{\pgfqpoint{0.695303in}{0.846552in}}%
\pgfpathlineto{\pgfqpoint{0.693826in}{0.837954in}}%
\pgfpathlineto{\pgfqpoint{0.684444in}{0.831972in}}%
\pgfpathlineto{\pgfqpoint{0.677975in}{0.833027in}}%
\pgfpathlineto{\pgfqpoint{0.601622in}{0.843623in}}%
\pgfpathlineto{\pgfqpoint{0.545229in}{0.851623in}}%
\pgfpathlineto{\pgfqpoint{0.547695in}{0.860728in}}%
\pgfpathlineto{\pgfqpoint{0.544024in}{0.875844in}}%
\pgfpathlineto{\pgfqpoint{0.543653in}{0.891092in}}%
\pgfpathlineto{\pgfqpoint{0.541254in}{0.900014in}}%
\pgfpathlineto{\pgfqpoint{0.533864in}{0.912772in}}%
\pgfpathlineto{\pgfqpoint{0.512605in}{0.942205in}}%
\pgfpathlineto{\pgfqpoint{0.505643in}{0.945811in}}%
\pgfpathlineto{\pgfqpoint{0.496672in}{0.945755in}}%
\pgfpathlineto{\pgfqpoint{0.498721in}{0.955040in}}%
\pgfpathlineto{\pgfqpoint{0.494473in}{0.966651in}}%
\pgfpathlineto{\pgfqpoint{0.473608in}{0.972422in}}%
\pgfpathlineto{\pgfqpoint{0.460889in}{0.983108in}}%
\pgfpathlineto{\pgfqpoint{0.459834in}{0.989647in}}%
\pgfpathlineto{\pgfqpoint{0.445189in}{1.005835in}}%
\pgfpathlineto{\pgfqpoint{0.431249in}{1.008932in}}%
\pgfpathlineto{\pgfqpoint{0.418332in}{1.017159in}}%
\pgfpathlineto{\pgfqpoint{0.401344in}{1.020005in}}%
\pgfpathlineto{\pgfqpoint{0.394086in}{1.030977in}}%
\pgfpathlineto{\pgfqpoint{0.400979in}{1.048345in}}%
\pgfpathlineto{\pgfqpoint{0.398871in}{1.052250in}}%
\pgfpathlineto{\pgfqpoint{0.404611in}{1.066728in}}%
\pgfpathlineto{\pgfqpoint{0.394400in}{1.074432in}}%
\pgfpathlineto{\pgfqpoint{0.397713in}{1.088401in}}%
\pgfpathlineto{\pgfqpoint{0.392259in}{1.091971in}}%
\pgfpathlineto{\pgfqpoint{0.387550in}{1.105174in}}%
\pgfpathlineto{\pgfqpoint{0.381867in}{1.109200in}}%
\pgfpathlineto{\pgfqpoint{0.381440in}{1.118736in}}%
\pgfpathlineto{\pgfqpoint{0.376988in}{1.125461in}}%
\pgfpathlineto{\pgfqpoint{0.370279in}{1.148109in}}%
\pgfpathlineto{\pgfqpoint{0.362938in}{1.158956in}}%
\pgfpathlineto{\pgfqpoint{0.364533in}{1.177370in}}%
\pgfpathlineto{\pgfqpoint{0.373166in}{1.179223in}}%
\pgfpathlineto{\pgfqpoint{0.378787in}{1.189211in}}%
\pgfpathlineto{\pgfqpoint{0.375390in}{1.200043in}}%
\pgfpathlineto{\pgfqpoint{0.366209in}{1.201851in}}%
\pgfpathlineto{\pgfqpoint{0.361629in}{1.206918in}}%
\pgfpathlineto{\pgfqpoint{0.354211in}{1.225555in}}%
\pgfpathlineto{\pgfqpoint{0.357678in}{1.232248in}}%
\pgfpathlineto{\pgfqpoint{0.355219in}{1.244711in}}%
\pgfpathlineto{\pgfqpoint{0.360664in}{1.260852in}}%
\pgfpathlineto{\pgfqpoint{0.367043in}{1.254907in}}%
\pgfpathlineto{\pgfqpoint{0.364148in}{1.247890in}}%
\pgfpathlineto{\pgfqpoint{0.375202in}{1.236766in}}%
\pgfpathlineto{\pgfqpoint{0.374497in}{1.253285in}}%
\pgfpathlineto{\pgfqpoint{0.369760in}{1.257715in}}%
\pgfpathlineto{\pgfqpoint{0.375174in}{1.272240in}}%
\pgfpathlineto{\pgfqpoint{0.395652in}{1.269927in}}%
\pgfpathlineto{\pgfqpoint{0.392896in}{1.275328in}}%
\pgfpathlineto{\pgfqpoint{0.379381in}{1.274797in}}%
\pgfpathlineto{\pgfqpoint{0.372951in}{1.282977in}}%
\pgfpathlineto{\pgfqpoint{0.364858in}{1.275713in}}%
\pgfpathlineto{\pgfqpoint{0.360564in}{1.263570in}}%
\pgfpathlineto{\pgfqpoint{0.349102in}{1.279909in}}%
\pgfpathlineto{\pgfqpoint{0.344687in}{1.282884in}}%
\pgfpathlineto{\pgfqpoint{0.346369in}{1.300667in}}%
\pgfpathlineto{\pgfqpoint{0.342875in}{1.311153in}}%
\pgfpathlineto{\pgfqpoint{0.336542in}{1.321006in}}%
\pgfpathlineto{\pgfqpoint{0.323573in}{1.351193in}}%
\pgfpathlineto{\pgfqpoint{0.327813in}{1.357869in}}%
\pgfpathlineto{\pgfqpoint{0.327764in}{1.378965in}}%
\pgfpathlineto{\pgfqpoint{0.334763in}{1.390732in}}%
\pgfpathlineto{\pgfqpoint{0.336368in}{1.409121in}}%
\pgfpathlineto{\pgfqpoint{0.329768in}{1.430191in}}%
\pgfpathlineto{\pgfqpoint{0.321012in}{1.443602in}}%
\pgfpathlineto{\pgfqpoint{0.322574in}{1.455704in}}%
\pgfpathlineto{\pgfqpoint{0.347163in}{1.484994in}}%
\pgfpathlineto{\pgfqpoint{0.348384in}{1.494985in}}%
\pgfpathlineto{\pgfqpoint{0.359464in}{1.514044in}}%
\pgfpathlineto{\pgfqpoint{0.361001in}{1.532109in}}%
\pgfpathlineto{\pgfqpoint{0.357439in}{1.536719in}}%
\pgfpathlineto{\pgfqpoint{0.363713in}{1.549891in}}%
\pgfpathclose%
\pgfusepath{fill}%
\end{pgfscope}%
\begin{pgfscope}%
\pgfpathrectangle{\pgfqpoint{0.100000in}{0.100000in}}{\pgfqpoint{2.989028in}{1.913466in}}%
\pgfusepath{clip}%
\pgfsetbuttcap%
\pgfsetmiterjoin%
\definecolor{currentfill}{rgb}{0.865667,0.946021,0.603460}%
\pgfsetfillcolor{currentfill}%
\pgfsetlinewidth{0.000000pt}%
\definecolor{currentstroke}{rgb}{0.000000,0.000000,0.000000}%
\pgfsetstrokecolor{currentstroke}%
\pgfsetstrokeopacity{0.000000}%
\pgfsetdash{}{0pt}%
\pgfpathmoveto{\pgfqpoint{2.278860in}{1.162554in}}%
\pgfpathlineto{\pgfqpoint{2.270539in}{1.244011in}}%
\pgfpathlineto{\pgfqpoint{2.260872in}{1.331556in}}%
\pgfpathlineto{\pgfqpoint{2.324180in}{1.340984in}}%
\pgfpathlineto{\pgfqpoint{2.340982in}{1.336611in}}%
\pgfpathlineto{\pgfqpoint{2.349041in}{1.331864in}}%
\pgfpathlineto{\pgfqpoint{2.359139in}{1.333180in}}%
\pgfpathlineto{\pgfqpoint{2.372453in}{1.325321in}}%
\pgfpathlineto{\pgfqpoint{2.397252in}{1.337018in}}%
\pgfpathlineto{\pgfqpoint{2.410938in}{1.337409in}}%
\pgfpathlineto{\pgfqpoint{2.426920in}{1.355239in}}%
\pgfpathlineto{\pgfqpoint{2.443180in}{1.366084in}}%
\pgfpathlineto{\pgfqpoint{2.464869in}{1.378577in}}%
\pgfpathlineto{\pgfqpoint{2.478831in}{1.291252in}}%
\pgfpathlineto{\pgfqpoint{2.472344in}{1.285643in}}%
\pgfpathlineto{\pgfqpoint{2.476532in}{1.280492in}}%
\pgfpathlineto{\pgfqpoint{2.478201in}{1.269199in}}%
\pgfpathlineto{\pgfqpoint{2.474951in}{1.258593in}}%
\pgfpathlineto{\pgfqpoint{2.474521in}{1.243089in}}%
\pgfpathlineto{\pgfqpoint{2.471469in}{1.222911in}}%
\pgfpathlineto{\pgfqpoint{2.456028in}{1.204915in}}%
\pgfpathlineto{\pgfqpoint{2.449528in}{1.201094in}}%
\pgfpathlineto{\pgfqpoint{2.444473in}{1.204383in}}%
\pgfpathlineto{\pgfqpoint{2.431963in}{1.187230in}}%
\pgfpathlineto{\pgfqpoint{2.433135in}{1.170714in}}%
\pgfpathlineto{\pgfqpoint{2.419203in}{1.174686in}}%
\pgfpathlineto{\pgfqpoint{2.412606in}{1.158191in}}%
\pgfpathlineto{\pgfqpoint{2.415941in}{1.146502in}}%
\pgfpathlineto{\pgfqpoint{2.410720in}{1.144776in}}%
\pgfpathlineto{\pgfqpoint{2.409990in}{1.135540in}}%
\pgfpathlineto{\pgfqpoint{2.397176in}{1.131815in}}%
\pgfpathlineto{\pgfqpoint{2.390518in}{1.139272in}}%
\pgfpathlineto{\pgfqpoint{2.381950in}{1.142163in}}%
\pgfpathlineto{\pgfqpoint{2.379937in}{1.149724in}}%
\pgfpathlineto{\pgfqpoint{2.372191in}{1.148394in}}%
\pgfpathlineto{\pgfqpoint{2.367116in}{1.141440in}}%
\pgfpathlineto{\pgfqpoint{2.359855in}{1.138994in}}%
\pgfpathlineto{\pgfqpoint{2.347031in}{1.143916in}}%
\pgfpathlineto{\pgfqpoint{2.339938in}{1.138060in}}%
\pgfpathlineto{\pgfqpoint{2.329857in}{1.144988in}}%
\pgfpathlineto{\pgfqpoint{2.310517in}{1.147114in}}%
\pgfpathlineto{\pgfqpoint{2.304684in}{1.159700in}}%
\pgfpathlineto{\pgfqpoint{2.297293in}{1.165276in}}%
\pgfpathlineto{\pgfqpoint{2.290128in}{1.161707in}}%
\pgfpathlineto{\pgfqpoint{2.278860in}{1.162554in}}%
\pgfpathclose%
\pgfusepath{fill}%
\end{pgfscope}%
\begin{pgfscope}%
\pgfpathrectangle{\pgfqpoint{0.100000in}{0.100000in}}{\pgfqpoint{2.989028in}{1.913466in}}%
\pgfusepath{clip}%
\pgfsetbuttcap%
\pgfsetmiterjoin%
\definecolor{currentfill}{rgb}{0.928874,0.971549,0.638062}%
\pgfsetfillcolor{currentfill}%
\pgfsetlinewidth{0.000000pt}%
\definecolor{currentstroke}{rgb}{0.000000,0.000000,0.000000}%
\pgfsetstrokecolor{currentstroke}%
\pgfsetstrokeopacity{0.000000}%
\pgfsetdash{}{0pt}%
\pgfpathmoveto{\pgfqpoint{2.069266in}{1.005020in}}%
\pgfpathlineto{\pgfqpoint{2.061849in}{1.011045in}}%
\pgfpathlineto{\pgfqpoint{2.055801in}{1.008178in}}%
\pgfpathlineto{\pgfqpoint{2.048069in}{1.022616in}}%
\pgfpathlineto{\pgfqpoint{2.052019in}{1.031690in}}%
\pgfpathlineto{\pgfqpoint{2.046326in}{1.041904in}}%
\pgfpathlineto{\pgfqpoint{2.046020in}{1.049946in}}%
\pgfpathlineto{\pgfqpoint{2.038328in}{1.052811in}}%
\pgfpathlineto{\pgfqpoint{2.034760in}{1.058870in}}%
\pgfpathlineto{\pgfqpoint{2.019719in}{1.066437in}}%
\pgfpathlineto{\pgfqpoint{2.006657in}{1.075760in}}%
\pgfpathlineto{\pgfqpoint{2.000587in}{1.082799in}}%
\pgfpathlineto{\pgfqpoint{2.000462in}{1.091383in}}%
\pgfpathlineto{\pgfqpoint{2.008584in}{1.107865in}}%
\pgfpathlineto{\pgfqpoint{2.011082in}{1.120472in}}%
\pgfpathlineto{\pgfqpoint{2.004465in}{1.127595in}}%
\pgfpathlineto{\pgfqpoint{1.995702in}{1.130279in}}%
\pgfpathlineto{\pgfqpoint{1.985070in}{1.124410in}}%
\pgfpathlineto{\pgfqpoint{1.980570in}{1.134486in}}%
\pgfpathlineto{\pgfqpoint{1.978299in}{1.148144in}}%
\pgfpathlineto{\pgfqpoint{1.962630in}{1.160319in}}%
\pgfpathlineto{\pgfqpoint{1.959499in}{1.165722in}}%
\pgfpathlineto{\pgfqpoint{1.945188in}{1.177954in}}%
\pgfpathlineto{\pgfqpoint{1.940703in}{1.186846in}}%
\pgfpathlineto{\pgfqpoint{1.936659in}{1.204480in}}%
\pgfpathlineto{\pgfqpoint{1.939431in}{1.220143in}}%
\pgfpathlineto{\pgfqpoint{1.943130in}{1.222326in}}%
\pgfpathlineto{\pgfqpoint{1.942465in}{1.235438in}}%
\pgfpathlineto{\pgfqpoint{1.952907in}{1.239332in}}%
\pgfpathlineto{\pgfqpoint{1.956060in}{1.251098in}}%
\pgfpathlineto{\pgfqpoint{1.962068in}{1.259018in}}%
\pgfpathlineto{\pgfqpoint{1.961764in}{1.269093in}}%
\pgfpathlineto{\pgfqpoint{1.954318in}{1.277106in}}%
\pgfpathlineto{\pgfqpoint{1.956087in}{1.288322in}}%
\pgfpathlineto{\pgfqpoint{1.975389in}{1.293200in}}%
\pgfpathlineto{\pgfqpoint{1.990197in}{1.302104in}}%
\pgfpathlineto{\pgfqpoint{1.991751in}{1.313315in}}%
\pgfpathlineto{\pgfqpoint{1.996904in}{1.316845in}}%
\pgfpathlineto{\pgfqpoint{1.998875in}{1.328623in}}%
\pgfpathlineto{\pgfqpoint{1.997291in}{1.336402in}}%
\pgfpathlineto{\pgfqpoint{1.987167in}{1.342864in}}%
\pgfpathlineto{\pgfqpoint{1.983091in}{1.352497in}}%
\pgfpathlineto{\pgfqpoint{1.973100in}{1.361809in}}%
\pgfpathlineto{\pgfqpoint{2.055206in}{1.365308in}}%
\pgfpathlineto{\pgfqpoint{2.110291in}{1.369221in}}%
\pgfpathlineto{\pgfqpoint{2.109283in}{1.357634in}}%
\pgfpathlineto{\pgfqpoint{2.118671in}{1.341651in}}%
\pgfpathlineto{\pgfqpoint{2.122612in}{1.327996in}}%
\pgfpathlineto{\pgfqpoint{2.127306in}{1.320243in}}%
\pgfpathlineto{\pgfqpoint{2.132431in}{1.256358in}}%
\pgfpathlineto{\pgfqpoint{2.139547in}{1.165295in}}%
\pgfpathlineto{\pgfqpoint{2.134057in}{1.151457in}}%
\pgfpathlineto{\pgfqpoint{2.141966in}{1.140083in}}%
\pgfpathlineto{\pgfqpoint{2.144170in}{1.129219in}}%
\pgfpathlineto{\pgfqpoint{2.133263in}{1.106692in}}%
\pgfpathlineto{\pgfqpoint{2.133907in}{1.103580in}}%
\pgfpathlineto{\pgfqpoint{2.122681in}{1.090668in}}%
\pgfpathlineto{\pgfqpoint{2.125759in}{1.086724in}}%
\pgfpathlineto{\pgfqpoint{2.121088in}{1.078676in}}%
\pgfpathlineto{\pgfqpoint{2.122290in}{1.062454in}}%
\pgfpathlineto{\pgfqpoint{2.116617in}{1.052499in}}%
\pgfpathlineto{\pgfqpoint{2.121231in}{1.040736in}}%
\pgfpathlineto{\pgfqpoint{2.101900in}{1.034344in}}%
\pgfpathlineto{\pgfqpoint{2.100119in}{1.027393in}}%
\pgfpathlineto{\pgfqpoint{2.105397in}{1.018582in}}%
\pgfpathlineto{\pgfqpoint{2.102969in}{1.012840in}}%
\pgfpathlineto{\pgfqpoint{2.082240in}{1.019933in}}%
\pgfpathlineto{\pgfqpoint{2.075401in}{1.020647in}}%
\pgfpathlineto{\pgfqpoint{2.066874in}{1.009863in}}%
\pgfpathlineto{\pgfqpoint{2.069266in}{1.005020in}}%
\pgfpathclose%
\pgfusepath{fill}%
\end{pgfscope}%
\begin{pgfscope}%
\pgfpathrectangle{\pgfqpoint{0.100000in}{0.100000in}}{\pgfqpoint{2.989028in}{1.913466in}}%
\pgfusepath{clip}%
\pgfsetbuttcap%
\pgfsetmiterjoin%
\definecolor{currentfill}{rgb}{0.905805,0.962322,0.602076}%
\pgfsetfillcolor{currentfill}%
\pgfsetlinewidth{0.000000pt}%
\definecolor{currentstroke}{rgb}{0.000000,0.000000,0.000000}%
\pgfsetstrokecolor{currentstroke}%
\pgfsetstrokeopacity{0.000000}%
\pgfsetdash{}{0pt}%
\pgfpathmoveto{\pgfqpoint{2.672645in}{1.203210in}}%
\pgfpathlineto{\pgfqpoint{2.666941in}{1.211681in}}%
\pgfpathlineto{\pgfqpoint{2.670160in}{1.216421in}}%
\pgfpathlineto{\pgfqpoint{2.678043in}{1.211098in}}%
\pgfpathlineto{\pgfqpoint{2.672645in}{1.203210in}}%
\pgfpathclose%
\pgfusepath{fill}%
\end{pgfscope}%
\begin{pgfscope}%
\pgfpathrectangle{\pgfqpoint{0.100000in}{0.100000in}}{\pgfqpoint{2.989028in}{1.913466in}}%
\pgfusepath{clip}%
\pgfsetbuttcap%
\pgfsetmiterjoin%
\definecolor{currentfill}{rgb}{0.998231,0.945175,0.657055}%
\pgfsetfillcolor{currentfill}%
\pgfsetlinewidth{0.000000pt}%
\definecolor{currentstroke}{rgb}{0.000000,0.000000,0.000000}%
\pgfsetstrokecolor{currentstroke}%
\pgfsetstrokeopacity{0.000000}%
\pgfsetdash{}{0pt}%
\pgfpathmoveto{\pgfqpoint{2.722458in}{1.276100in}}%
\pgfpathlineto{\pgfqpoint{2.725867in}{1.283302in}}%
\pgfpathlineto{\pgfqpoint{2.739648in}{1.284858in}}%
\pgfpathlineto{\pgfqpoint{2.737440in}{1.278737in}}%
\pgfpathlineto{\pgfqpoint{2.732893in}{1.270918in}}%
\pgfpathlineto{\pgfqpoint{2.735974in}{1.261602in}}%
\pgfpathlineto{\pgfqpoint{2.748133in}{1.250439in}}%
\pgfpathlineto{\pgfqpoint{2.750955in}{1.238685in}}%
\pgfpathlineto{\pgfqpoint{2.764963in}{1.224047in}}%
\pgfpathlineto{\pgfqpoint{2.770458in}{1.224674in}}%
\pgfpathlineto{\pgfqpoint{2.777304in}{1.202700in}}%
\pgfpathlineto{\pgfqpoint{2.776181in}{1.202481in}}%
\pgfpathlineto{\pgfqpoint{2.774933in}{1.202236in}}%
\pgfpathlineto{\pgfqpoint{2.744401in}{1.196391in}}%
\pgfpathlineto{\pgfqpoint{2.728066in}{1.254485in}}%
\pgfpathlineto{\pgfqpoint{2.722458in}{1.276100in}}%
\pgfpathclose%
\pgfusepath{fill}%
\end{pgfscope}%
\begin{pgfscope}%
\pgfpathrectangle{\pgfqpoint{0.100000in}{0.100000in}}{\pgfqpoint{2.989028in}{1.913466in}}%
\pgfusepath{clip}%
\pgfsetbuttcap%
\pgfsetmiterjoin%
\definecolor{currentfill}{rgb}{0.601615,0.839677,0.644137}%
\pgfsetfillcolor{currentfill}%
\pgfsetlinewidth{0.000000pt}%
\definecolor{currentstroke}{rgb}{0.000000,0.000000,0.000000}%
\pgfsetstrokecolor{currentstroke}%
\pgfsetstrokeopacity{0.000000}%
\pgfsetdash{}{0pt}%
\pgfpathmoveto{\pgfqpoint{2.437284in}{1.079184in}}%
\pgfpathlineto{\pgfqpoint{2.427636in}{1.079539in}}%
\pgfpathlineto{\pgfqpoint{2.418746in}{1.085662in}}%
\pgfpathlineto{\pgfqpoint{2.406736in}{1.104274in}}%
\pgfpathlineto{\pgfqpoint{2.396513in}{1.114131in}}%
\pgfpathlineto{\pgfqpoint{2.399185in}{1.121744in}}%
\pgfpathlineto{\pgfqpoint{2.397176in}{1.131815in}}%
\pgfpathlineto{\pgfqpoint{2.409990in}{1.135540in}}%
\pgfpathlineto{\pgfqpoint{2.410720in}{1.144776in}}%
\pgfpathlineto{\pgfqpoint{2.415941in}{1.146502in}}%
\pgfpathlineto{\pgfqpoint{2.412606in}{1.158191in}}%
\pgfpathlineto{\pgfqpoint{2.419203in}{1.174686in}}%
\pgfpathlineto{\pgfqpoint{2.433135in}{1.170714in}}%
\pgfpathlineto{\pgfqpoint{2.431963in}{1.187230in}}%
\pgfpathlineto{\pgfqpoint{2.444473in}{1.204383in}}%
\pgfpathlineto{\pgfqpoint{2.449528in}{1.201094in}}%
\pgfpathlineto{\pgfqpoint{2.456028in}{1.204915in}}%
\pgfpathlineto{\pgfqpoint{2.471469in}{1.222911in}}%
\pgfpathlineto{\pgfqpoint{2.474521in}{1.243089in}}%
\pgfpathlineto{\pgfqpoint{2.474951in}{1.258593in}}%
\pgfpathlineto{\pgfqpoint{2.478201in}{1.269199in}}%
\pgfpathlineto{\pgfqpoint{2.476532in}{1.280492in}}%
\pgfpathlineto{\pgfqpoint{2.472344in}{1.285643in}}%
\pgfpathlineto{\pgfqpoint{2.478831in}{1.291252in}}%
\pgfpathlineto{\pgfqpoint{2.488234in}{1.232010in}}%
\pgfpathlineto{\pgfqpoint{2.540072in}{1.240581in}}%
\pgfpathlineto{\pgfqpoint{2.545445in}{1.206768in}}%
\pgfpathlineto{\pgfqpoint{2.564213in}{1.229086in}}%
\pgfpathlineto{\pgfqpoint{2.568635in}{1.226861in}}%
\pgfpathlineto{\pgfqpoint{2.575140in}{1.239793in}}%
\pgfpathlineto{\pgfqpoint{2.584152in}{1.235738in}}%
\pgfpathlineto{\pgfqpoint{2.592010in}{1.236509in}}%
\pgfpathlineto{\pgfqpoint{2.597194in}{1.245499in}}%
\pgfpathlineto{\pgfqpoint{2.604719in}{1.250500in}}%
\pgfpathlineto{\pgfqpoint{2.615180in}{1.246096in}}%
\pgfpathlineto{\pgfqpoint{2.622064in}{1.247618in}}%
\pgfpathlineto{\pgfqpoint{2.631926in}{1.230550in}}%
\pgfpathlineto{\pgfqpoint{2.629074in}{1.217629in}}%
\pgfpathlineto{\pgfqpoint{2.603317in}{1.232177in}}%
\pgfpathlineto{\pgfqpoint{2.598492in}{1.220234in}}%
\pgfpathlineto{\pgfqpoint{2.600084in}{1.214739in}}%
\pgfpathlineto{\pgfqpoint{2.589121in}{1.196164in}}%
\pgfpathlineto{\pgfqpoint{2.583947in}{1.192470in}}%
\pgfpathlineto{\pgfqpoint{2.581611in}{1.184279in}}%
\pgfpathlineto{\pgfqpoint{2.572791in}{1.185136in}}%
\pgfpathlineto{\pgfqpoint{2.566456in}{1.162779in}}%
\pgfpathlineto{\pgfqpoint{2.562911in}{1.157648in}}%
\pgfpathlineto{\pgfqpoint{2.553799in}{1.159359in}}%
\pgfpathlineto{\pgfqpoint{2.544480in}{1.166411in}}%
\pgfpathlineto{\pgfqpoint{2.544157in}{1.155588in}}%
\pgfpathlineto{\pgfqpoint{2.535135in}{1.137399in}}%
\pgfpathlineto{\pgfqpoint{2.532891in}{1.124451in}}%
\pgfpathlineto{\pgfqpoint{2.525969in}{1.115827in}}%
\pgfpathlineto{\pgfqpoint{2.520721in}{1.102055in}}%
\pgfpathlineto{\pgfqpoint{2.520463in}{1.089306in}}%
\pgfpathlineto{\pgfqpoint{2.512358in}{1.086925in}}%
\pgfpathlineto{\pgfqpoint{2.503164in}{1.079708in}}%
\pgfpathlineto{\pgfqpoint{2.494311in}{1.078334in}}%
\pgfpathlineto{\pgfqpoint{2.492271in}{1.072225in}}%
\pgfpathlineto{\pgfqpoint{2.478007in}{1.065970in}}%
\pgfpathlineto{\pgfqpoint{2.470035in}{1.071300in}}%
\pgfpathlineto{\pgfqpoint{2.461126in}{1.061179in}}%
\pgfpathlineto{\pgfqpoint{2.445810in}{1.064162in}}%
\pgfpathlineto{\pgfqpoint{2.436416in}{1.074785in}}%
\pgfpathlineto{\pgfqpoint{2.437284in}{1.079184in}}%
\pgfpathclose%
\pgfusepath{fill}%
\end{pgfscope}%
\begin{pgfscope}%
\pgfpathrectangle{\pgfqpoint{0.100000in}{0.100000in}}{\pgfqpoint{2.989028in}{1.913466in}}%
\pgfusepath{clip}%
\pgfsetbuttcap%
\pgfsetmiterjoin%
\definecolor{currentfill}{rgb}{0.892887,0.957093,0.597924}%
\pgfsetfillcolor{currentfill}%
\pgfsetlinewidth{0.000000pt}%
\definecolor{currentstroke}{rgb}{0.000000,0.000000,0.000000}%
\pgfsetstrokecolor{currentstroke}%
\pgfsetstrokeopacity{0.000000}%
\pgfsetdash{}{0pt}%
\pgfpathmoveto{\pgfqpoint{2.774933in}{1.202236in}}%
\pgfpathlineto{\pgfqpoint{2.774542in}{1.190298in}}%
\pgfpathlineto{\pgfqpoint{2.769947in}{1.184429in}}%
\pgfpathlineto{\pgfqpoint{2.767015in}{1.171401in}}%
\pgfpathlineto{\pgfqpoint{2.753775in}{1.165395in}}%
\pgfpathlineto{\pgfqpoint{2.742667in}{1.163645in}}%
\pgfpathlineto{\pgfqpoint{2.745904in}{1.172214in}}%
\pgfpathlineto{\pgfqpoint{2.728800in}{1.179393in}}%
\pgfpathlineto{\pgfqpoint{2.714904in}{1.188383in}}%
\pgfpathlineto{\pgfqpoint{2.723507in}{1.201821in}}%
\pgfpathlineto{\pgfqpoint{2.716613in}{1.212266in}}%
\pgfpathlineto{\pgfqpoint{2.717424in}{1.221276in}}%
\pgfpathlineto{\pgfqpoint{2.712439in}{1.223749in}}%
\pgfpathlineto{\pgfqpoint{2.708394in}{1.238386in}}%
\pgfpathlineto{\pgfqpoint{2.717960in}{1.258229in}}%
\pgfpathlineto{\pgfqpoint{2.710773in}{1.261469in}}%
\pgfpathlineto{\pgfqpoint{2.708910in}{1.251680in}}%
\pgfpathlineto{\pgfqpoint{2.698651in}{1.248954in}}%
\pgfpathlineto{\pgfqpoint{2.699730in}{1.216746in}}%
\pgfpathlineto{\pgfqpoint{2.697854in}{1.206381in}}%
\pgfpathlineto{\pgfqpoint{2.703000in}{1.191607in}}%
\pgfpathlineto{\pgfqpoint{2.710916in}{1.184499in}}%
\pgfpathlineto{\pgfqpoint{2.707326in}{1.180036in}}%
\pgfpathlineto{\pgfqpoint{2.715407in}{1.173514in}}%
\pgfpathlineto{\pgfqpoint{2.718328in}{1.162929in}}%
\pgfpathlineto{\pgfqpoint{2.703517in}{1.171694in}}%
\pgfpathlineto{\pgfqpoint{2.694143in}{1.170556in}}%
\pgfpathlineto{\pgfqpoint{2.686865in}{1.179573in}}%
\pgfpathlineto{\pgfqpoint{2.679454in}{1.180451in}}%
\pgfpathlineto{\pgfqpoint{2.668921in}{1.175935in}}%
\pgfpathlineto{\pgfqpoint{2.664837in}{1.181563in}}%
\pgfpathlineto{\pgfqpoint{2.670205in}{1.193376in}}%
\pgfpathlineto{\pgfqpoint{2.672645in}{1.203210in}}%
\pgfpathlineto{\pgfqpoint{2.678043in}{1.211098in}}%
\pgfpathlineto{\pgfqpoint{2.670160in}{1.216421in}}%
\pgfpathlineto{\pgfqpoint{2.666941in}{1.211681in}}%
\pgfpathlineto{\pgfqpoint{2.659066in}{1.216492in}}%
\pgfpathlineto{\pgfqpoint{2.645120in}{1.219698in}}%
\pgfpathlineto{\pgfqpoint{2.646385in}{1.226747in}}%
\pgfpathlineto{\pgfqpoint{2.640064in}{1.230836in}}%
\pgfpathlineto{\pgfqpoint{2.631926in}{1.230550in}}%
\pgfpathlineto{\pgfqpoint{2.622064in}{1.247618in}}%
\pgfpathlineto{\pgfqpoint{2.615180in}{1.246096in}}%
\pgfpathlineto{\pgfqpoint{2.604719in}{1.250500in}}%
\pgfpathlineto{\pgfqpoint{2.597194in}{1.245499in}}%
\pgfpathlineto{\pgfqpoint{2.592010in}{1.236509in}}%
\pgfpathlineto{\pgfqpoint{2.584152in}{1.235738in}}%
\pgfpathlineto{\pgfqpoint{2.575140in}{1.239793in}}%
\pgfpathlineto{\pgfqpoint{2.568635in}{1.226861in}}%
\pgfpathlineto{\pgfqpoint{2.564213in}{1.229086in}}%
\pgfpathlineto{\pgfqpoint{2.545445in}{1.206768in}}%
\pgfpathlineto{\pgfqpoint{2.540072in}{1.240581in}}%
\pgfpathlineto{\pgfqpoint{2.608655in}{1.253292in}}%
\pgfpathlineto{\pgfqpoint{2.639399in}{1.258797in}}%
\pgfpathlineto{\pgfqpoint{2.684127in}{1.267806in}}%
\pgfpathlineto{\pgfqpoint{2.722458in}{1.276100in}}%
\pgfpathlineto{\pgfqpoint{2.728066in}{1.254485in}}%
\pgfpathlineto{\pgfqpoint{2.744401in}{1.196391in}}%
\pgfpathlineto{\pgfqpoint{2.774933in}{1.202236in}}%
\pgfpathclose%
\pgfusepath{fill}%
\end{pgfscope}%
\begin{pgfscope}%
\pgfpathrectangle{\pgfqpoint{0.100000in}{0.100000in}}{\pgfqpoint{2.989028in}{1.913466in}}%
\pgfusepath{clip}%
\pgfsetbuttcap%
\pgfsetmiterjoin%
\definecolor{currentfill}{rgb}{0.756786,0.901730,0.625606}%
\pgfsetfillcolor{currentfill}%
\pgfsetlinewidth{0.000000pt}%
\definecolor{currentstroke}{rgb}{0.000000,0.000000,0.000000}%
\pgfsetstrokecolor{currentstroke}%
\pgfsetstrokeopacity{0.000000}%
\pgfsetdash{}{0pt}%
\pgfpathmoveto{\pgfqpoint{1.394404in}{1.009408in}}%
\pgfpathlineto{\pgfqpoint{1.344368in}{1.014183in}}%
\pgfpathlineto{\pgfqpoint{1.292460in}{1.018783in}}%
\pgfpathlineto{\pgfqpoint{1.232477in}{1.025040in}}%
\pgfpathlineto{\pgfqpoint{1.143359in}{1.035639in}}%
\pgfpathlineto{\pgfqpoint{1.111744in}{1.040510in}}%
\pgfpathlineto{\pgfqpoint{1.030137in}{1.052362in}}%
\pgfpathlineto{\pgfqpoint{1.042053in}{1.127151in}}%
\pgfpathlineto{\pgfqpoint{1.042350in}{1.133200in}}%
\pgfpathlineto{\pgfqpoint{1.053829in}{1.205365in}}%
\pgfpathlineto{\pgfqpoint{1.062404in}{1.260326in}}%
\pgfpathlineto{\pgfqpoint{1.070495in}{1.311293in}}%
\pgfpathlineto{\pgfqpoint{1.125778in}{1.303357in}}%
\pgfpathlineto{\pgfqpoint{1.177321in}{1.295959in}}%
\pgfpathlineto{\pgfqpoint{1.272072in}{1.284265in}}%
\pgfpathlineto{\pgfqpoint{1.315540in}{1.280295in}}%
\pgfpathlineto{\pgfqpoint{1.384404in}{1.273614in}}%
\pgfpathlineto{\pgfqpoint{1.414195in}{1.271196in}}%
\pgfpathlineto{\pgfqpoint{1.408935in}{1.205970in}}%
\pgfpathlineto{\pgfqpoint{1.404183in}{1.143169in}}%
\pgfpathlineto{\pgfqpoint{1.400359in}{1.092033in}}%
\pgfpathlineto{\pgfqpoint{1.394404in}{1.009408in}}%
\pgfpathclose%
\pgfusepath{fill}%
\end{pgfscope}%
\begin{pgfscope}%
\pgfpathrectangle{\pgfqpoint{0.100000in}{0.100000in}}{\pgfqpoint{2.989028in}{1.913466in}}%
\pgfusepath{clip}%
\pgfsetbuttcap%
\pgfsetmiterjoin%
\definecolor{currentfill}{rgb}{0.747712,0.898039,0.627451}%
\pgfsetfillcolor{currentfill}%
\pgfsetlinewidth{0.000000pt}%
\definecolor{currentstroke}{rgb}{0.000000,0.000000,0.000000}%
\pgfsetstrokecolor{currentstroke}%
\pgfsetstrokeopacity{0.000000}%
\pgfsetdash{}{0pt}%
\pgfpathmoveto{\pgfqpoint{2.056343in}{0.972329in}}%
\pgfpathlineto{\pgfqpoint{2.058030in}{0.979890in}}%
\pgfpathlineto{\pgfqpoint{2.066773in}{0.978188in}}%
\pgfpathlineto{\pgfqpoint{2.069266in}{1.005020in}}%
\pgfpathlineto{\pgfqpoint{2.066874in}{1.009863in}}%
\pgfpathlineto{\pgfqpoint{2.075401in}{1.020647in}}%
\pgfpathlineto{\pgfqpoint{2.082240in}{1.019933in}}%
\pgfpathlineto{\pgfqpoint{2.102969in}{1.012840in}}%
\pgfpathlineto{\pgfqpoint{2.105397in}{1.018582in}}%
\pgfpathlineto{\pgfqpoint{2.100119in}{1.027393in}}%
\pgfpathlineto{\pgfqpoint{2.101900in}{1.034344in}}%
\pgfpathlineto{\pgfqpoint{2.121231in}{1.040736in}}%
\pgfpathlineto{\pgfqpoint{2.116617in}{1.052499in}}%
\pgfpathlineto{\pgfqpoint{2.122290in}{1.062454in}}%
\pgfpathlineto{\pgfqpoint{2.132850in}{1.068041in}}%
\pgfpathlineto{\pgfqpoint{2.155006in}{1.073574in}}%
\pgfpathlineto{\pgfqpoint{2.168985in}{1.065222in}}%
\pgfpathlineto{\pgfqpoint{2.173226in}{1.073367in}}%
\pgfpathlineto{\pgfqpoint{2.184834in}{1.079056in}}%
\pgfpathlineto{\pgfqpoint{2.187829in}{1.074039in}}%
\pgfpathlineto{\pgfqpoint{2.199753in}{1.078025in}}%
\pgfpathlineto{\pgfqpoint{2.199000in}{1.084848in}}%
\pgfpathlineto{\pgfqpoint{2.209721in}{1.092656in}}%
\pgfpathlineto{\pgfqpoint{2.216048in}{1.084548in}}%
\pgfpathlineto{\pgfqpoint{2.224371in}{1.083732in}}%
\pgfpathlineto{\pgfqpoint{2.229903in}{1.089016in}}%
\pgfpathlineto{\pgfqpoint{2.229287in}{1.096531in}}%
\pgfpathlineto{\pgfqpoint{2.233989in}{1.103993in}}%
\pgfpathlineto{\pgfqpoint{2.240290in}{1.105610in}}%
\pgfpathlineto{\pgfqpoint{2.242820in}{1.115463in}}%
\pgfpathlineto{\pgfqpoint{2.251983in}{1.123981in}}%
\pgfpathlineto{\pgfqpoint{2.249225in}{1.132459in}}%
\pgfpathlineto{\pgfqpoint{2.258139in}{1.136680in}}%
\pgfpathlineto{\pgfqpoint{2.264096in}{1.134084in}}%
\pgfpathlineto{\pgfqpoint{2.272893in}{1.140679in}}%
\pgfpathlineto{\pgfqpoint{2.280755in}{1.142397in}}%
\pgfpathlineto{\pgfqpoint{2.280783in}{1.149238in}}%
\pgfpathlineto{\pgfqpoint{2.275273in}{1.158730in}}%
\pgfpathlineto{\pgfqpoint{2.278860in}{1.162554in}}%
\pgfpathlineto{\pgfqpoint{2.290128in}{1.161707in}}%
\pgfpathlineto{\pgfqpoint{2.297293in}{1.165276in}}%
\pgfpathlineto{\pgfqpoint{2.304684in}{1.159700in}}%
\pgfpathlineto{\pgfqpoint{2.310517in}{1.147114in}}%
\pgfpathlineto{\pgfqpoint{2.329857in}{1.144988in}}%
\pgfpathlineto{\pgfqpoint{2.339938in}{1.138060in}}%
\pgfpathlineto{\pgfqpoint{2.347031in}{1.143916in}}%
\pgfpathlineto{\pgfqpoint{2.359855in}{1.138994in}}%
\pgfpathlineto{\pgfqpoint{2.367116in}{1.141440in}}%
\pgfpathlineto{\pgfqpoint{2.372191in}{1.148394in}}%
\pgfpathlineto{\pgfqpoint{2.379937in}{1.149724in}}%
\pgfpathlineto{\pgfqpoint{2.381950in}{1.142163in}}%
\pgfpathlineto{\pgfqpoint{2.390518in}{1.139272in}}%
\pgfpathlineto{\pgfqpoint{2.397176in}{1.131815in}}%
\pgfpathlineto{\pgfqpoint{2.399185in}{1.121744in}}%
\pgfpathlineto{\pgfqpoint{2.396513in}{1.114131in}}%
\pgfpathlineto{\pgfqpoint{2.406736in}{1.104274in}}%
\pgfpathlineto{\pgfqpoint{2.418746in}{1.085662in}}%
\pgfpathlineto{\pgfqpoint{2.427636in}{1.079539in}}%
\pgfpathlineto{\pgfqpoint{2.437284in}{1.079184in}}%
\pgfpathlineto{\pgfqpoint{2.419469in}{1.058743in}}%
\pgfpathlineto{\pgfqpoint{2.401934in}{1.046371in}}%
\pgfpathlineto{\pgfqpoint{2.395481in}{1.036545in}}%
\pgfpathlineto{\pgfqpoint{2.395590in}{1.031213in}}%
\pgfpathlineto{\pgfqpoint{2.386092in}{1.027126in}}%
\pgfpathlineto{\pgfqpoint{2.383370in}{1.019436in}}%
\pgfpathlineto{\pgfqpoint{2.363584in}{1.011690in}}%
\pgfpathlineto{\pgfqpoint{2.356571in}{1.006660in}}%
\pgfpathlineto{\pgfqpoint{2.355622in}{1.005586in}}%
\pgfpathlineto{\pgfqpoint{2.298708in}{1.000190in}}%
\pgfpathlineto{\pgfqpoint{2.264337in}{0.997299in}}%
\pgfpathlineto{\pgfqpoint{2.207938in}{0.994107in}}%
\pgfpathlineto{\pgfqpoint{2.137674in}{0.987156in}}%
\pgfpathlineto{\pgfqpoint{2.126065in}{0.988764in}}%
\pgfpathlineto{\pgfqpoint{2.128480in}{0.976914in}}%
\pgfpathlineto{\pgfqpoint{2.056343in}{0.972329in}}%
\pgfpathclose%
\pgfusepath{fill}%
\end{pgfscope}%
\begin{pgfscope}%
\pgfpathrectangle{\pgfqpoint{0.100000in}{0.100000in}}{\pgfqpoint{2.989028in}{1.913466in}}%
\pgfusepath{clip}%
\pgfsetbuttcap%
\pgfsetmiterjoin%
\definecolor{currentfill}{rgb}{0.307728,0.388697,0.672664}%
\pgfsetfillcolor{currentfill}%
\pgfsetlinewidth{0.000000pt}%
\definecolor{currentstroke}{rgb}{0.000000,0.000000,0.000000}%
\pgfsetstrokecolor{currentstroke}%
\pgfsetstrokeopacity{0.000000}%
\pgfsetdash{}{0pt}%
\pgfpathmoveto{\pgfqpoint{1.394404in}{1.009408in}}%
\pgfpathlineto{\pgfqpoint{1.400359in}{1.092033in}}%
\pgfpathlineto{\pgfqpoint{1.404183in}{1.143169in}}%
\pgfpathlineto{\pgfqpoint{1.408935in}{1.205970in}}%
\pgfpathlineto{\pgfqpoint{1.474809in}{1.201345in}}%
\pgfpathlineto{\pgfqpoint{1.558468in}{1.196775in}}%
\pgfpathlineto{\pgfqpoint{1.615367in}{1.194592in}}%
\pgfpathlineto{\pgfqpoint{1.671933in}{1.192859in}}%
\pgfpathlineto{\pgfqpoint{1.723131in}{1.192066in}}%
\pgfpathlineto{\pgfqpoint{1.746810in}{1.192312in}}%
\pgfpathlineto{\pgfqpoint{1.757233in}{1.183806in}}%
\pgfpathlineto{\pgfqpoint{1.765397in}{1.185521in}}%
\pgfpathlineto{\pgfqpoint{1.765653in}{1.178102in}}%
\pgfpathlineto{\pgfqpoint{1.759590in}{1.165269in}}%
\pgfpathlineto{\pgfqpoint{1.760247in}{1.157161in}}%
\pgfpathlineto{\pgfqpoint{1.766052in}{1.151814in}}%
\pgfpathlineto{\pgfqpoint{1.771384in}{1.140674in}}%
\pgfpathlineto{\pgfqpoint{1.782201in}{1.134279in}}%
\pgfpathlineto{\pgfqpoint{1.781840in}{1.092287in}}%
\pgfpathlineto{\pgfqpoint{1.782158in}{0.995725in}}%
\pgfpathlineto{\pgfqpoint{1.709656in}{0.996060in}}%
\pgfpathlineto{\pgfqpoint{1.633309in}{0.997396in}}%
\pgfpathlineto{\pgfqpoint{1.577105in}{0.999379in}}%
\pgfpathlineto{\pgfqpoint{1.530191in}{1.001189in}}%
\pgfpathlineto{\pgfqpoint{1.451155in}{1.005849in}}%
\pgfpathlineto{\pgfqpoint{1.394404in}{1.009408in}}%
\pgfpathclose%
\pgfusepath{fill}%
\end{pgfscope}%
\begin{pgfscope}%
\pgfpathrectangle{\pgfqpoint{0.100000in}{0.100000in}}{\pgfqpoint{2.989028in}{1.913466in}}%
\pgfusepath{clip}%
\pgfsetbuttcap%
\pgfsetmiterjoin%
\definecolor{currentfill}{rgb}{0.917339,0.966936,0.620069}%
\pgfsetfillcolor{currentfill}%
\pgfsetlinewidth{0.000000pt}%
\definecolor{currentstroke}{rgb}{0.000000,0.000000,0.000000}%
\pgfsetstrokecolor{currentstroke}%
\pgfsetstrokeopacity{0.000000}%
\pgfsetdash{}{0pt}%
\pgfpathmoveto{\pgfqpoint{2.461189in}{1.020348in}}%
\pgfpathlineto{\pgfqpoint{2.380888in}{1.009068in}}%
\pgfpathlineto{\pgfqpoint{2.356571in}{1.006660in}}%
\pgfpathlineto{\pgfqpoint{2.363584in}{1.011690in}}%
\pgfpathlineto{\pgfqpoint{2.383370in}{1.019436in}}%
\pgfpathlineto{\pgfqpoint{2.386092in}{1.027126in}}%
\pgfpathlineto{\pgfqpoint{2.395590in}{1.031213in}}%
\pgfpathlineto{\pgfqpoint{2.395481in}{1.036545in}}%
\pgfpathlineto{\pgfqpoint{2.401934in}{1.046371in}}%
\pgfpathlineto{\pgfqpoint{2.419469in}{1.058743in}}%
\pgfpathlineto{\pgfqpoint{2.437284in}{1.079184in}}%
\pgfpathlineto{\pgfqpoint{2.436416in}{1.074785in}}%
\pgfpathlineto{\pgfqpoint{2.445810in}{1.064162in}}%
\pgfpathlineto{\pgfqpoint{2.461126in}{1.061179in}}%
\pgfpathlineto{\pgfqpoint{2.470035in}{1.071300in}}%
\pgfpathlineto{\pgfqpoint{2.478007in}{1.065970in}}%
\pgfpathlineto{\pgfqpoint{2.492271in}{1.072225in}}%
\pgfpathlineto{\pgfqpoint{2.494311in}{1.078334in}}%
\pgfpathlineto{\pgfqpoint{2.503164in}{1.079708in}}%
\pgfpathlineto{\pgfqpoint{2.512358in}{1.086925in}}%
\pgfpathlineto{\pgfqpoint{2.520463in}{1.089306in}}%
\pgfpathlineto{\pgfqpoint{2.520721in}{1.102055in}}%
\pgfpathlineto{\pgfqpoint{2.525969in}{1.115827in}}%
\pgfpathlineto{\pgfqpoint{2.532891in}{1.124451in}}%
\pgfpathlineto{\pgfqpoint{2.535135in}{1.137399in}}%
\pgfpathlineto{\pgfqpoint{2.544157in}{1.155588in}}%
\pgfpathlineto{\pgfqpoint{2.544480in}{1.166411in}}%
\pgfpathlineto{\pgfqpoint{2.553799in}{1.159359in}}%
\pgfpathlineto{\pgfqpoint{2.562911in}{1.157648in}}%
\pgfpathlineto{\pgfqpoint{2.566456in}{1.162779in}}%
\pgfpathlineto{\pgfqpoint{2.572791in}{1.185136in}}%
\pgfpathlineto{\pgfqpoint{2.581611in}{1.184279in}}%
\pgfpathlineto{\pgfqpoint{2.583947in}{1.192470in}}%
\pgfpathlineto{\pgfqpoint{2.589121in}{1.196164in}}%
\pgfpathlineto{\pgfqpoint{2.600084in}{1.214739in}}%
\pgfpathlineto{\pgfqpoint{2.598492in}{1.220234in}}%
\pgfpathlineto{\pgfqpoint{2.603317in}{1.232177in}}%
\pgfpathlineto{\pgfqpoint{2.629074in}{1.217629in}}%
\pgfpathlineto{\pgfqpoint{2.631926in}{1.230550in}}%
\pgfpathlineto{\pgfqpoint{2.640064in}{1.230836in}}%
\pgfpathlineto{\pgfqpoint{2.646385in}{1.226747in}}%
\pgfpathlineto{\pgfqpoint{2.645120in}{1.219698in}}%
\pgfpathlineto{\pgfqpoint{2.659066in}{1.216492in}}%
\pgfpathlineto{\pgfqpoint{2.666941in}{1.211681in}}%
\pgfpathlineto{\pgfqpoint{2.672645in}{1.203210in}}%
\pgfpathlineto{\pgfqpoint{2.670205in}{1.193376in}}%
\pgfpathlineto{\pgfqpoint{2.665277in}{1.192570in}}%
\pgfpathlineto{\pgfqpoint{2.662419in}{1.177731in}}%
\pgfpathlineto{\pgfqpoint{2.668680in}{1.171929in}}%
\pgfpathlineto{\pgfqpoint{2.677493in}{1.176619in}}%
\pgfpathlineto{\pgfqpoint{2.685671in}{1.166716in}}%
\pgfpathlineto{\pgfqpoint{2.703942in}{1.164936in}}%
\pgfpathlineto{\pgfqpoint{2.707089in}{1.159239in}}%
\pgfpathlineto{\pgfqpoint{2.723998in}{1.153699in}}%
\pgfpathlineto{\pgfqpoint{2.721912in}{1.147161in}}%
\pgfpathlineto{\pgfqpoint{2.724111in}{1.128683in}}%
\pgfpathlineto{\pgfqpoint{2.730936in}{1.121706in}}%
\pgfpathlineto{\pgfqpoint{2.720866in}{1.118761in}}%
\pgfpathlineto{\pgfqpoint{2.722201in}{1.110873in}}%
\pgfpathlineto{\pgfqpoint{2.732944in}{1.104185in}}%
\pgfpathlineto{\pgfqpoint{2.733912in}{1.097588in}}%
\pgfpathlineto{\pgfqpoint{2.727847in}{1.092630in}}%
\pgfpathlineto{\pgfqpoint{2.715609in}{1.104375in}}%
\pgfpathlineto{\pgfqpoint{2.714399in}{1.095806in}}%
\pgfpathlineto{\pgfqpoint{2.724644in}{1.091731in}}%
\pgfpathlineto{\pgfqpoint{2.725669in}{1.087517in}}%
\pgfpathlineto{\pgfqpoint{2.739721in}{1.093086in}}%
\pgfpathlineto{\pgfqpoint{2.750489in}{1.094559in}}%
\pgfpathlineto{\pgfqpoint{2.761520in}{1.072304in}}%
\pgfpathlineto{\pgfqpoint{2.760289in}{1.072063in}}%
\pgfpathlineto{\pgfqpoint{2.755305in}{1.071033in}}%
\pgfpathlineto{\pgfqpoint{2.753836in}{1.070726in}}%
\pgfpathlineto{\pgfqpoint{2.752865in}{1.070536in}}%
\pgfpathlineto{\pgfqpoint{2.687189in}{1.056913in}}%
\pgfpathlineto{\pgfqpoint{2.628447in}{1.044890in}}%
\pgfpathlineto{\pgfqpoint{2.547229in}{1.030898in}}%
\pgfpathlineto{\pgfqpoint{2.478251in}{1.021788in}}%
\pgfpathlineto{\pgfqpoint{2.461189in}{1.020348in}}%
\pgfpathclose%
\pgfusepath{fill}%
\end{pgfscope}%
\begin{pgfscope}%
\pgfpathrectangle{\pgfqpoint{0.100000in}{0.100000in}}{\pgfqpoint{2.989028in}{1.913466in}}%
\pgfusepath{clip}%
\pgfsetbuttcap%
\pgfsetmiterjoin%
\definecolor{currentfill}{rgb}{0.917339,0.966936,0.620069}%
\pgfsetfillcolor{currentfill}%
\pgfsetlinewidth{0.000000pt}%
\definecolor{currentstroke}{rgb}{0.000000,0.000000,0.000000}%
\pgfsetstrokecolor{currentstroke}%
\pgfsetstrokeopacity{0.000000}%
\pgfsetdash{}{0pt}%
\pgfpathmoveto{\pgfqpoint{2.753775in}{1.165395in}}%
\pgfpathlineto{\pgfqpoint{2.767015in}{1.171401in}}%
\pgfpathlineto{\pgfqpoint{2.756442in}{1.140471in}}%
\pgfpathlineto{\pgfqpoint{2.749450in}{1.124092in}}%
\pgfpathlineto{\pgfqpoint{2.744413in}{1.133361in}}%
\pgfpathlineto{\pgfqpoint{2.753370in}{1.155552in}}%
\pgfpathlineto{\pgfqpoint{2.753775in}{1.165395in}}%
\pgfpathclose%
\pgfusepath{fill}%
\end{pgfscope}%
\begin{pgfscope}%
\pgfpathrectangle{\pgfqpoint{0.100000in}{0.100000in}}{\pgfqpoint{2.989028in}{1.913466in}}%
\pgfusepath{clip}%
\pgfsetbuttcap%
\pgfsetmiterjoin%
\definecolor{currentfill}{rgb}{0.999769,0.992849,0.737024}%
\pgfsetfillcolor{currentfill}%
\pgfsetlinewidth{0.000000pt}%
\definecolor{currentstroke}{rgb}{0.000000,0.000000,0.000000}%
\pgfsetstrokecolor{currentstroke}%
\pgfsetstrokeopacity{0.000000}%
\pgfsetdash{}{0pt}%
\pgfpathmoveto{\pgfqpoint{2.056343in}{0.972329in}}%
\pgfpathlineto{\pgfqpoint{2.053146in}{0.971866in}}%
\pgfpathlineto{\pgfqpoint{2.050131in}{0.971655in}}%
\pgfpathlineto{\pgfqpoint{2.051421in}{0.962397in}}%
\pgfpathlineto{\pgfqpoint{2.043642in}{0.953066in}}%
\pgfpathlineto{\pgfqpoint{2.042109in}{0.938464in}}%
\pgfpathlineto{\pgfqpoint{2.007334in}{0.935892in}}%
\pgfpathlineto{\pgfqpoint{2.010365in}{0.942748in}}%
\pgfpathlineto{\pgfqpoint{2.022904in}{0.955269in}}%
\pgfpathlineto{\pgfqpoint{2.023251in}{0.962527in}}%
\pgfpathlineto{\pgfqpoint{2.017700in}{0.969399in}}%
\pgfpathlineto{\pgfqpoint{1.965941in}{0.966662in}}%
\pgfpathlineto{\pgfqpoint{1.875448in}{0.963789in}}%
\pgfpathlineto{\pgfqpoint{1.822491in}{0.962824in}}%
\pgfpathlineto{\pgfqpoint{1.782462in}{0.962463in}}%
\pgfpathlineto{\pgfqpoint{1.782158in}{0.995725in}}%
\pgfpathlineto{\pgfqpoint{1.781840in}{1.092287in}}%
\pgfpathlineto{\pgfqpoint{1.782201in}{1.134279in}}%
\pgfpathlineto{\pgfqpoint{1.771384in}{1.140674in}}%
\pgfpathlineto{\pgfqpoint{1.766052in}{1.151814in}}%
\pgfpathlineto{\pgfqpoint{1.760247in}{1.157161in}}%
\pgfpathlineto{\pgfqpoint{1.759590in}{1.165269in}}%
\pgfpathlineto{\pgfqpoint{1.765653in}{1.178102in}}%
\pgfpathlineto{\pgfqpoint{1.765397in}{1.185521in}}%
\pgfpathlineto{\pgfqpoint{1.757233in}{1.183806in}}%
\pgfpathlineto{\pgfqpoint{1.746810in}{1.192312in}}%
\pgfpathlineto{\pgfqpoint{1.738455in}{1.207242in}}%
\pgfpathlineto{\pgfqpoint{1.731447in}{1.214132in}}%
\pgfpathlineto{\pgfqpoint{1.724124in}{1.231060in}}%
\pgfpathlineto{\pgfqpoint{1.800184in}{1.229874in}}%
\pgfpathlineto{\pgfqpoint{1.875789in}{1.232365in}}%
\pgfpathlineto{\pgfqpoint{1.924266in}{1.235161in}}%
\pgfpathlineto{\pgfqpoint{1.939431in}{1.220143in}}%
\pgfpathlineto{\pgfqpoint{1.936659in}{1.204480in}}%
\pgfpathlineto{\pgfqpoint{1.940703in}{1.186846in}}%
\pgfpathlineto{\pgfqpoint{1.945188in}{1.177954in}}%
\pgfpathlineto{\pgfqpoint{1.959499in}{1.165722in}}%
\pgfpathlineto{\pgfqpoint{1.962630in}{1.160319in}}%
\pgfpathlineto{\pgfqpoint{1.978299in}{1.148144in}}%
\pgfpathlineto{\pgfqpoint{1.980570in}{1.134486in}}%
\pgfpathlineto{\pgfqpoint{1.985070in}{1.124410in}}%
\pgfpathlineto{\pgfqpoint{1.995702in}{1.130279in}}%
\pgfpathlineto{\pgfqpoint{2.004465in}{1.127595in}}%
\pgfpathlineto{\pgfqpoint{2.011082in}{1.120472in}}%
\pgfpathlineto{\pgfqpoint{2.008584in}{1.107865in}}%
\pgfpathlineto{\pgfqpoint{2.000462in}{1.091383in}}%
\pgfpathlineto{\pgfqpoint{2.000587in}{1.082799in}}%
\pgfpathlineto{\pgfqpoint{2.006657in}{1.075760in}}%
\pgfpathlineto{\pgfqpoint{2.019719in}{1.066437in}}%
\pgfpathlineto{\pgfqpoint{2.034760in}{1.058870in}}%
\pgfpathlineto{\pgfqpoint{2.038328in}{1.052811in}}%
\pgfpathlineto{\pgfqpoint{2.046020in}{1.049946in}}%
\pgfpathlineto{\pgfqpoint{2.046326in}{1.041904in}}%
\pgfpathlineto{\pgfqpoint{2.052019in}{1.031690in}}%
\pgfpathlineto{\pgfqpoint{2.048069in}{1.022616in}}%
\pgfpathlineto{\pgfqpoint{2.055801in}{1.008178in}}%
\pgfpathlineto{\pgfqpoint{2.061849in}{1.011045in}}%
\pgfpathlineto{\pgfqpoint{2.069266in}{1.005020in}}%
\pgfpathlineto{\pgfqpoint{2.066773in}{0.978188in}}%
\pgfpathlineto{\pgfqpoint{2.058030in}{0.979890in}}%
\pgfpathlineto{\pgfqpoint{2.056343in}{0.972329in}}%
\pgfpathclose%
\pgfusepath{fill}%
\end{pgfscope}%
\begin{pgfscope}%
\pgfpathrectangle{\pgfqpoint{0.100000in}{0.100000in}}{\pgfqpoint{2.989028in}{1.913466in}}%
\pgfusepath{clip}%
\pgfsetbuttcap%
\pgfsetmiterjoin%
\definecolor{currentfill}{rgb}{0.909650,0.963860,0.608074}%
\pgfsetfillcolor{currentfill}%
\pgfsetlinewidth{0.000000pt}%
\definecolor{currentstroke}{rgb}{0.000000,0.000000,0.000000}%
\pgfsetstrokecolor{currentstroke}%
\pgfsetstrokeopacity{0.000000}%
\pgfsetdash{}{0pt}%
\pgfpathmoveto{\pgfqpoint{0.715525in}{0.978113in}}%
\pgfpathlineto{\pgfqpoint{0.721856in}{0.991592in}}%
\pgfpathlineto{\pgfqpoint{0.720123in}{1.011825in}}%
\pgfpathlineto{\pgfqpoint{0.723155in}{1.017578in}}%
\pgfpathlineto{\pgfqpoint{0.723454in}{1.042780in}}%
\pgfpathlineto{\pgfqpoint{0.726328in}{1.050037in}}%
\pgfpathlineto{\pgfqpoint{0.736451in}{1.051167in}}%
\pgfpathlineto{\pgfqpoint{0.745846in}{1.047945in}}%
\pgfpathlineto{\pgfqpoint{0.749938in}{1.039072in}}%
\pgfpathlineto{\pgfqpoint{0.755674in}{1.039412in}}%
\pgfpathlineto{\pgfqpoint{0.762801in}{1.049576in}}%
\pgfpathlineto{\pgfqpoint{0.773103in}{1.099678in}}%
\pgfpathlineto{\pgfqpoint{0.831724in}{1.087575in}}%
\pgfpathlineto{\pgfqpoint{0.865746in}{1.080865in}}%
\pgfpathlineto{\pgfqpoint{0.956091in}{1.064984in}}%
\pgfpathlineto{\pgfqpoint{0.981086in}{1.059964in}}%
\pgfpathlineto{\pgfqpoint{1.030137in}{1.052362in}}%
\pgfpathlineto{\pgfqpoint{1.020079in}{0.987601in}}%
\pgfpathlineto{\pgfqpoint{1.009619in}{0.920049in}}%
\pgfpathlineto{\pgfqpoint{0.997573in}{0.844052in}}%
\pgfpathlineto{\pgfqpoint{0.984025in}{0.756815in}}%
\pgfpathlineto{\pgfqpoint{0.973080in}{0.685166in}}%
\pgfpathlineto{\pgfqpoint{0.894687in}{0.697614in}}%
\pgfpathlineto{\pgfqpoint{0.860244in}{0.703513in}}%
\pgfpathlineto{\pgfqpoint{0.844857in}{0.712725in}}%
\pgfpathlineto{\pgfqpoint{0.744535in}{0.773254in}}%
\pgfpathlineto{\pgfqpoint{0.669248in}{0.819189in}}%
\pgfpathlineto{\pgfqpoint{0.671756in}{0.827331in}}%
\pgfpathlineto{\pgfqpoint{0.677975in}{0.833027in}}%
\pgfpathlineto{\pgfqpoint{0.684444in}{0.831972in}}%
\pgfpathlineto{\pgfqpoint{0.693826in}{0.837954in}}%
\pgfpathlineto{\pgfqpoint{0.695303in}{0.846552in}}%
\pgfpathlineto{\pgfqpoint{0.683861in}{0.856982in}}%
\pgfpathlineto{\pgfqpoint{0.691998in}{0.877004in}}%
\pgfpathlineto{\pgfqpoint{0.700180in}{0.884704in}}%
\pgfpathlineto{\pgfqpoint{0.704063in}{0.893839in}}%
\pgfpathlineto{\pgfqpoint{0.706455in}{0.910589in}}%
\pgfpathlineto{\pgfqpoint{0.714127in}{0.918174in}}%
\pgfpathlineto{\pgfqpoint{0.730230in}{0.925750in}}%
\pgfpathlineto{\pgfqpoint{0.729874in}{0.932426in}}%
\pgfpathlineto{\pgfqpoint{0.720881in}{0.940711in}}%
\pgfpathlineto{\pgfqpoint{0.719656in}{0.957789in}}%
\pgfpathlineto{\pgfqpoint{0.713464in}{0.970274in}}%
\pgfpathlineto{\pgfqpoint{0.715525in}{0.978113in}}%
\pgfpathclose%
\pgfusepath{fill}%
\end{pgfscope}%
\begin{pgfscope}%
\pgfpathrectangle{\pgfqpoint{0.100000in}{0.100000in}}{\pgfqpoint{2.989028in}{1.913466in}}%
\pgfusepath{clip}%
\pgfsetbuttcap%
\pgfsetmiterjoin%
\definecolor{currentfill}{rgb}{0.442445,0.777393,0.646444}%
\pgfsetfillcolor{currentfill}%
\pgfsetlinewidth{0.000000pt}%
\definecolor{currentstroke}{rgb}{0.000000,0.000000,0.000000}%
\pgfsetstrokecolor{currentstroke}%
\pgfsetstrokeopacity{0.000000}%
\pgfsetdash{}{0pt}%
\pgfpathmoveto{\pgfqpoint{1.790948in}{0.774957in}}%
\pgfpathlineto{\pgfqpoint{1.776416in}{0.779451in}}%
\pgfpathlineto{\pgfqpoint{1.757795in}{0.793711in}}%
\pgfpathlineto{\pgfqpoint{1.749524in}{0.796776in}}%
\pgfpathlineto{\pgfqpoint{1.744268in}{0.790617in}}%
\pgfpathlineto{\pgfqpoint{1.737631in}{0.790306in}}%
\pgfpathlineto{\pgfqpoint{1.729234in}{0.795535in}}%
\pgfpathlineto{\pgfqpoint{1.716057in}{0.788835in}}%
\pgfpathlineto{\pgfqpoint{1.711367in}{0.792135in}}%
\pgfpathlineto{\pgfqpoint{1.698401in}{0.784318in}}%
\pgfpathlineto{\pgfqpoint{1.684722in}{0.785761in}}%
\pgfpathlineto{\pgfqpoint{1.674172in}{0.790840in}}%
\pgfpathlineto{\pgfqpoint{1.666783in}{0.788928in}}%
\pgfpathlineto{\pgfqpoint{1.654956in}{0.796112in}}%
\pgfpathlineto{\pgfqpoint{1.648377in}{0.783442in}}%
\pgfpathlineto{\pgfqpoint{1.641649in}{0.794338in}}%
\pgfpathlineto{\pgfqpoint{1.628360in}{0.790111in}}%
\pgfpathlineto{\pgfqpoint{1.616736in}{0.800460in}}%
\pgfpathlineto{\pgfqpoint{1.606579in}{0.792118in}}%
\pgfpathlineto{\pgfqpoint{1.601493in}{0.799784in}}%
\pgfpathlineto{\pgfqpoint{1.594161in}{0.802273in}}%
\pgfpathlineto{\pgfqpoint{1.589696in}{0.809664in}}%
\pgfpathlineto{\pgfqpoint{1.580106in}{0.811767in}}%
\pgfpathlineto{\pgfqpoint{1.574573in}{0.806204in}}%
\pgfpathlineto{\pgfqpoint{1.565171in}{0.813414in}}%
\pgfpathlineto{\pgfqpoint{1.560789in}{0.811767in}}%
\pgfpathlineto{\pgfqpoint{1.545207in}{0.817609in}}%
\pgfpathlineto{\pgfqpoint{1.535448in}{0.818261in}}%
\pgfpathlineto{\pgfqpoint{1.531080in}{0.830667in}}%
\pgfpathlineto{\pgfqpoint{1.523256in}{0.829122in}}%
\pgfpathlineto{\pgfqpoint{1.514296in}{0.832183in}}%
\pgfpathlineto{\pgfqpoint{1.508395in}{0.830414in}}%
\pgfpathlineto{\pgfqpoint{1.495741in}{0.844347in}}%
\pgfpathlineto{\pgfqpoint{1.492215in}{0.843436in}}%
\pgfpathlineto{\pgfqpoint{1.495417in}{0.899940in}}%
\pgfpathlineto{\pgfqpoint{1.498906in}{0.969876in}}%
\pgfpathlineto{\pgfqpoint{1.441639in}{0.973080in}}%
\pgfpathlineto{\pgfqpoint{1.385128in}{0.977352in}}%
\pgfpathlineto{\pgfqpoint{1.341469in}{0.981143in}}%
\pgfpathlineto{\pgfqpoint{1.344368in}{1.014183in}}%
\pgfpathlineto{\pgfqpoint{1.394404in}{1.009408in}}%
\pgfpathlineto{\pgfqpoint{1.451155in}{1.005849in}}%
\pgfpathlineto{\pgfqpoint{1.530191in}{1.001189in}}%
\pgfpathlineto{\pgfqpoint{1.577105in}{0.999379in}}%
\pgfpathlineto{\pgfqpoint{1.633309in}{0.997396in}}%
\pgfpathlineto{\pgfqpoint{1.709656in}{0.996060in}}%
\pgfpathlineto{\pgfqpoint{1.782158in}{0.995725in}}%
\pgfpathlineto{\pgfqpoint{1.782462in}{0.962463in}}%
\pgfpathlineto{\pgfqpoint{1.786531in}{0.937404in}}%
\pgfpathlineto{\pgfqpoint{1.792853in}{0.891120in}}%
\pgfpathlineto{\pgfqpoint{1.791549in}{0.812079in}}%
\pgfpathlineto{\pgfqpoint{1.790948in}{0.774957in}}%
\pgfpathclose%
\pgfusepath{fill}%
\end{pgfscope}%
\begin{pgfscope}%
\pgfpathrectangle{\pgfqpoint{0.100000in}{0.100000in}}{\pgfqpoint{2.989028in}{1.913466in}}%
\pgfusepath{clip}%
\pgfsetbuttcap%
\pgfsetmiterjoin%
\definecolor{currentfill}{rgb}{0.932718,0.973087,0.644060}%
\pgfsetfillcolor{currentfill}%
\pgfsetlinewidth{0.000000pt}%
\definecolor{currentstroke}{rgb}{0.000000,0.000000,0.000000}%
\pgfsetstrokecolor{currentstroke}%
\pgfsetstrokeopacity{0.000000}%
\pgfsetdash{}{0pt}%
\pgfpathmoveto{\pgfqpoint{2.335069in}{0.897708in}}%
\pgfpathlineto{\pgfqpoint{2.335112in}{0.912353in}}%
\pgfpathlineto{\pgfqpoint{2.347839in}{0.917977in}}%
\pgfpathlineto{\pgfqpoint{2.348372in}{0.926967in}}%
\pgfpathlineto{\pgfqpoint{2.359767in}{0.937949in}}%
\pgfpathlineto{\pgfqpoint{2.374011in}{0.940198in}}%
\pgfpathlineto{\pgfqpoint{2.386533in}{0.952301in}}%
\pgfpathlineto{\pgfqpoint{2.399610in}{0.957711in}}%
\pgfpathlineto{\pgfqpoint{2.409425in}{0.973951in}}%
\pgfpathlineto{\pgfqpoint{2.418364in}{0.972772in}}%
\pgfpathlineto{\pgfqpoint{2.437256in}{0.987570in}}%
\pgfpathlineto{\pgfqpoint{2.447232in}{0.987849in}}%
\pgfpathlineto{\pgfqpoint{2.451445in}{0.999123in}}%
\pgfpathlineto{\pgfqpoint{2.459375in}{1.006973in}}%
\pgfpathlineto{\pgfqpoint{2.461189in}{1.020348in}}%
\pgfpathlineto{\pgfqpoint{2.478251in}{1.021788in}}%
\pgfpathlineto{\pgfqpoint{2.547229in}{1.030898in}}%
\pgfpathlineto{\pgfqpoint{2.628447in}{1.044890in}}%
\pgfpathlineto{\pgfqpoint{2.687189in}{1.056913in}}%
\pgfpathlineto{\pgfqpoint{2.752865in}{1.070536in}}%
\pgfpathlineto{\pgfqpoint{2.771631in}{1.044770in}}%
\pgfpathlineto{\pgfqpoint{2.761466in}{1.046415in}}%
\pgfpathlineto{\pgfqpoint{2.748417in}{1.043236in}}%
\pgfpathlineto{\pgfqpoint{2.735559in}{1.030105in}}%
\pgfpathlineto{\pgfqpoint{2.726318in}{1.031050in}}%
\pgfpathlineto{\pgfqpoint{2.725149in}{1.023251in}}%
\pgfpathlineto{\pgfqpoint{2.741852in}{1.029418in}}%
\pgfpathlineto{\pgfqpoint{2.758660in}{1.031961in}}%
\pgfpathlineto{\pgfqpoint{2.764901in}{1.028577in}}%
\pgfpathlineto{\pgfqpoint{2.773300in}{1.032435in}}%
\pgfpathlineto{\pgfqpoint{2.777634in}{1.029732in}}%
\pgfpathlineto{\pgfqpoint{2.781470in}{1.016862in}}%
\pgfpathlineto{\pgfqpoint{2.773487in}{1.012876in}}%
\pgfpathlineto{\pgfqpoint{2.768008in}{0.997182in}}%
\pgfpathlineto{\pgfqpoint{2.762263in}{0.991072in}}%
\pgfpathlineto{\pgfqpoint{2.744653in}{0.992409in}}%
\pgfpathlineto{\pgfqpoint{2.745600in}{1.001624in}}%
\pgfpathlineto{\pgfqpoint{2.735982in}{0.997599in}}%
\pgfpathlineto{\pgfqpoint{2.733890in}{0.989913in}}%
\pgfpathlineto{\pgfqpoint{2.740483in}{0.981952in}}%
\pgfpathlineto{\pgfqpoint{2.742722in}{0.968440in}}%
\pgfpathlineto{\pgfqpoint{2.731935in}{0.960741in}}%
\pgfpathlineto{\pgfqpoint{2.743601in}{0.958032in}}%
\pgfpathlineto{\pgfqpoint{2.748881in}{0.963694in}}%
\pgfpathlineto{\pgfqpoint{2.760553in}{0.964383in}}%
\pgfpathlineto{\pgfqpoint{2.754555in}{0.952158in}}%
\pgfpathlineto{\pgfqpoint{2.747208in}{0.946268in}}%
\pgfpathlineto{\pgfqpoint{2.724984in}{0.940370in}}%
\pgfpathlineto{\pgfqpoint{2.702166in}{0.919675in}}%
\pgfpathlineto{\pgfqpoint{2.688096in}{0.899284in}}%
\pgfpathlineto{\pgfqpoint{2.688013in}{0.891003in}}%
\pgfpathlineto{\pgfqpoint{2.682393in}{0.879575in}}%
\pgfpathlineto{\pgfqpoint{2.653548in}{0.872057in}}%
\pgfpathlineto{\pgfqpoint{2.584952in}{0.921230in}}%
\pgfpathlineto{\pgfqpoint{2.524543in}{0.912354in}}%
\pgfpathlineto{\pgfqpoint{2.524037in}{0.920550in}}%
\pgfpathlineto{\pgfqpoint{2.514855in}{0.929779in}}%
\pgfpathlineto{\pgfqpoint{2.507904in}{0.932038in}}%
\pgfpathlineto{\pgfqpoint{2.442269in}{0.925218in}}%
\pgfpathlineto{\pgfqpoint{2.427183in}{0.920102in}}%
\pgfpathlineto{\pgfqpoint{2.399943in}{0.906551in}}%
\pgfpathlineto{\pgfqpoint{2.376402in}{0.902801in}}%
\pgfpathlineto{\pgfqpoint{2.335069in}{0.897708in}}%
\pgfpathclose%
\pgfusepath{fill}%
\end{pgfscope}%
\begin{pgfscope}%
\pgfpathrectangle{\pgfqpoint{0.100000in}{0.100000in}}{\pgfqpoint{2.989028in}{1.913466in}}%
\pgfusepath{clip}%
\pgfsetbuttcap%
\pgfsetmiterjoin%
\definecolor{currentfill}{rgb}{0.955786,0.982314,0.680046}%
\pgfsetfillcolor{currentfill}%
\pgfsetlinewidth{0.000000pt}%
\definecolor{currentstroke}{rgb}{0.000000,0.000000,0.000000}%
\pgfsetstrokecolor{currentstroke}%
\pgfsetstrokeopacity{0.000000}%
\pgfsetdash{}{0pt}%
\pgfpathmoveto{\pgfqpoint{2.335069in}{0.897708in}}%
\pgfpathlineto{\pgfqpoint{2.266375in}{0.890163in}}%
\pgfpathlineto{\pgfqpoint{2.203533in}{0.884512in}}%
\pgfpathlineto{\pgfqpoint{2.138945in}{0.880335in}}%
\pgfpathlineto{\pgfqpoint{2.127858in}{0.878728in}}%
\pgfpathlineto{\pgfqpoint{2.084319in}{0.875379in}}%
\pgfpathlineto{\pgfqpoint{2.014584in}{0.871310in}}%
\pgfpathlineto{\pgfqpoint{2.026990in}{0.881608in}}%
\pgfpathlineto{\pgfqpoint{2.024159in}{0.892449in}}%
\pgfpathlineto{\pgfqpoint{2.026902in}{0.905590in}}%
\pgfpathlineto{\pgfqpoint{2.031092in}{0.911777in}}%
\pgfpathlineto{\pgfqpoint{2.030934in}{0.920384in}}%
\pgfpathlineto{\pgfqpoint{2.042090in}{0.925797in}}%
\pgfpathlineto{\pgfqpoint{2.042109in}{0.938464in}}%
\pgfpathlineto{\pgfqpoint{2.043642in}{0.953066in}}%
\pgfpathlineto{\pgfqpoint{2.051421in}{0.962397in}}%
\pgfpathlineto{\pgfqpoint{2.050131in}{0.971655in}}%
\pgfpathlineto{\pgfqpoint{2.053146in}{0.971866in}}%
\pgfpathlineto{\pgfqpoint{2.056343in}{0.972329in}}%
\pgfpathlineto{\pgfqpoint{2.128480in}{0.976914in}}%
\pgfpathlineto{\pgfqpoint{2.126065in}{0.988764in}}%
\pgfpathlineto{\pgfqpoint{2.137674in}{0.987156in}}%
\pgfpathlineto{\pgfqpoint{2.207938in}{0.994107in}}%
\pgfpathlineto{\pgfqpoint{2.264337in}{0.997299in}}%
\pgfpathlineto{\pgfqpoint{2.298708in}{1.000190in}}%
\pgfpathlineto{\pgfqpoint{2.355622in}{1.005586in}}%
\pgfpathlineto{\pgfqpoint{2.356571in}{1.006660in}}%
\pgfpathlineto{\pgfqpoint{2.380888in}{1.009068in}}%
\pgfpathlineto{\pgfqpoint{2.461189in}{1.020348in}}%
\pgfpathlineto{\pgfqpoint{2.459375in}{1.006973in}}%
\pgfpathlineto{\pgfqpoint{2.451445in}{0.999123in}}%
\pgfpathlineto{\pgfqpoint{2.447232in}{0.987849in}}%
\pgfpathlineto{\pgfqpoint{2.437256in}{0.987570in}}%
\pgfpathlineto{\pgfqpoint{2.418364in}{0.972772in}}%
\pgfpathlineto{\pgfqpoint{2.409425in}{0.973951in}}%
\pgfpathlineto{\pgfqpoint{2.399610in}{0.957711in}}%
\pgfpathlineto{\pgfqpoint{2.386533in}{0.952301in}}%
\pgfpathlineto{\pgfqpoint{2.374011in}{0.940198in}}%
\pgfpathlineto{\pgfqpoint{2.359767in}{0.937949in}}%
\pgfpathlineto{\pgfqpoint{2.348372in}{0.926967in}}%
\pgfpathlineto{\pgfqpoint{2.347839in}{0.917977in}}%
\pgfpathlineto{\pgfqpoint{2.335112in}{0.912353in}}%
\pgfpathlineto{\pgfqpoint{2.335069in}{0.897708in}}%
\pgfpathclose%
\pgfusepath{fill}%
\end{pgfscope}%
\begin{pgfscope}%
\pgfpathrectangle{\pgfqpoint{0.100000in}{0.100000in}}{\pgfqpoint{2.989028in}{1.913466in}}%
\pgfusepath{clip}%
\pgfsetbuttcap%
\pgfsetmiterjoin%
\definecolor{currentfill}{rgb}{0.384006,0.742945,0.654441}%
\pgfsetfillcolor{currentfill}%
\pgfsetlinewidth{0.000000pt}%
\definecolor{currentstroke}{rgb}{0.000000,0.000000,0.000000}%
\pgfsetstrokecolor{currentstroke}%
\pgfsetstrokeopacity{0.000000}%
\pgfsetdash{}{0pt}%
\pgfpathmoveto{\pgfqpoint{1.116758in}{0.694333in}}%
\pgfpathlineto{\pgfqpoint{1.111859in}{0.702205in}}%
\pgfpathlineto{\pgfqpoint{1.113886in}{0.708987in}}%
\pgfpathlineto{\pgfqpoint{1.148324in}{0.704715in}}%
\pgfpathlineto{\pgfqpoint{1.212359in}{0.697442in}}%
\pgfpathlineto{\pgfqpoint{1.258660in}{0.692892in}}%
\pgfpathlineto{\pgfqpoint{1.312135in}{0.687496in}}%
\pgfpathlineto{\pgfqpoint{1.315044in}{0.721225in}}%
\pgfpathlineto{\pgfqpoint{1.324140in}{0.807015in}}%
\pgfpathlineto{\pgfqpoint{1.330041in}{0.867115in}}%
\pgfpathlineto{\pgfqpoint{1.335142in}{0.924525in}}%
\pgfpathlineto{\pgfqpoint{1.339896in}{0.981231in}}%
\pgfpathlineto{\pgfqpoint{1.341469in}{0.981143in}}%
\pgfpathlineto{\pgfqpoint{1.385128in}{0.977352in}}%
\pgfpathlineto{\pgfqpoint{1.441639in}{0.973080in}}%
\pgfpathlineto{\pgfqpoint{1.498906in}{0.969876in}}%
\pgfpathlineto{\pgfqpoint{1.495417in}{0.899940in}}%
\pgfpathlineto{\pgfqpoint{1.492215in}{0.843436in}}%
\pgfpathlineto{\pgfqpoint{1.495741in}{0.844347in}}%
\pgfpathlineto{\pgfqpoint{1.508395in}{0.830414in}}%
\pgfpathlineto{\pgfqpoint{1.514296in}{0.832183in}}%
\pgfpathlineto{\pgfqpoint{1.523256in}{0.829122in}}%
\pgfpathlineto{\pgfqpoint{1.531080in}{0.830667in}}%
\pgfpathlineto{\pgfqpoint{1.535448in}{0.818261in}}%
\pgfpathlineto{\pgfqpoint{1.545207in}{0.817609in}}%
\pgfpathlineto{\pgfqpoint{1.560789in}{0.811767in}}%
\pgfpathlineto{\pgfqpoint{1.565171in}{0.813414in}}%
\pgfpathlineto{\pgfqpoint{1.574573in}{0.806204in}}%
\pgfpathlineto{\pgfqpoint{1.580106in}{0.811767in}}%
\pgfpathlineto{\pgfqpoint{1.589696in}{0.809664in}}%
\pgfpathlineto{\pgfqpoint{1.594161in}{0.802273in}}%
\pgfpathlineto{\pgfqpoint{1.601493in}{0.799784in}}%
\pgfpathlineto{\pgfqpoint{1.606579in}{0.792118in}}%
\pgfpathlineto{\pgfqpoint{1.616736in}{0.800460in}}%
\pgfpathlineto{\pgfqpoint{1.628360in}{0.790111in}}%
\pgfpathlineto{\pgfqpoint{1.641649in}{0.794338in}}%
\pgfpathlineto{\pgfqpoint{1.648377in}{0.783442in}}%
\pgfpathlineto{\pgfqpoint{1.654956in}{0.796112in}}%
\pgfpathlineto{\pgfqpoint{1.666783in}{0.788928in}}%
\pgfpathlineto{\pgfqpoint{1.674172in}{0.790840in}}%
\pgfpathlineto{\pgfqpoint{1.684722in}{0.785761in}}%
\pgfpathlineto{\pgfqpoint{1.698401in}{0.784318in}}%
\pgfpathlineto{\pgfqpoint{1.711367in}{0.792135in}}%
\pgfpathlineto{\pgfqpoint{1.716057in}{0.788835in}}%
\pgfpathlineto{\pgfqpoint{1.729234in}{0.795535in}}%
\pgfpathlineto{\pgfqpoint{1.737631in}{0.790306in}}%
\pgfpathlineto{\pgfqpoint{1.744268in}{0.790617in}}%
\pgfpathlineto{\pgfqpoint{1.749524in}{0.796776in}}%
\pgfpathlineto{\pgfqpoint{1.757795in}{0.793711in}}%
\pgfpathlineto{\pgfqpoint{1.776416in}{0.779451in}}%
\pgfpathlineto{\pgfqpoint{1.790948in}{0.774957in}}%
\pgfpathlineto{\pgfqpoint{1.796775in}{0.769451in}}%
\pgfpathlineto{\pgfqpoint{1.804070in}{0.772449in}}%
\pgfpathlineto{\pgfqpoint{1.815123in}{0.770153in}}%
\pgfpathlineto{\pgfqpoint{1.815340in}{0.735108in}}%
\pgfpathlineto{\pgfqpoint{1.816262in}{0.667326in}}%
\pgfpathlineto{\pgfqpoint{1.823939in}{0.660813in}}%
\pgfpathlineto{\pgfqpoint{1.830092in}{0.648811in}}%
\pgfpathlineto{\pgfqpoint{1.827929in}{0.640781in}}%
\pgfpathlineto{\pgfqpoint{1.832761in}{0.637381in}}%
\pgfpathlineto{\pgfqpoint{1.836688in}{0.622574in}}%
\pgfpathlineto{\pgfqpoint{1.844209in}{0.614644in}}%
\pgfpathlineto{\pgfqpoint{1.845894in}{0.597808in}}%
\pgfpathlineto{\pgfqpoint{1.844586in}{0.590680in}}%
\pgfpathlineto{\pgfqpoint{1.834363in}{0.571815in}}%
\pgfpathlineto{\pgfqpoint{1.833187in}{0.558251in}}%
\pgfpathlineto{\pgfqpoint{1.836655in}{0.555420in}}%
\pgfpathlineto{\pgfqpoint{1.836956in}{0.543537in}}%
\pgfpathlineto{\pgfqpoint{1.833432in}{0.536068in}}%
\pgfpathlineto{\pgfqpoint{1.827950in}{0.533255in}}%
\pgfpathlineto{\pgfqpoint{1.822582in}{0.523496in}}%
\pgfpathlineto{\pgfqpoint{1.829424in}{0.514043in}}%
\pgfpathlineto{\pgfqpoint{1.816144in}{0.513857in}}%
\pgfpathlineto{\pgfqpoint{1.780636in}{0.497611in}}%
\pgfpathlineto{\pgfqpoint{1.787415in}{0.507333in}}%
\pgfpathlineto{\pgfqpoint{1.779206in}{0.512575in}}%
\pgfpathlineto{\pgfqpoint{1.777490in}{0.521488in}}%
\pgfpathlineto{\pgfqpoint{1.761468in}{0.505964in}}%
\pgfpathlineto{\pgfqpoint{1.768579in}{0.495354in}}%
\pgfpathlineto{\pgfqpoint{1.758435in}{0.481844in}}%
\pgfpathlineto{\pgfqpoint{1.752976in}{0.482126in}}%
\pgfpathlineto{\pgfqpoint{1.747841in}{0.467413in}}%
\pgfpathlineto{\pgfqpoint{1.731600in}{0.455840in}}%
\pgfpathlineto{\pgfqpoint{1.716427in}{0.451669in}}%
\pgfpathlineto{\pgfqpoint{1.689964in}{0.440819in}}%
\pgfpathlineto{\pgfqpoint{1.687222in}{0.446871in}}%
\pgfpathlineto{\pgfqpoint{1.670737in}{0.441300in}}%
\pgfpathlineto{\pgfqpoint{1.680876in}{0.431814in}}%
\pgfpathlineto{\pgfqpoint{1.664883in}{0.423573in}}%
\pgfpathlineto{\pgfqpoint{1.647633in}{0.411131in}}%
\pgfpathlineto{\pgfqpoint{1.643490in}{0.416915in}}%
\pgfpathlineto{\pgfqpoint{1.629373in}{0.408246in}}%
\pgfpathlineto{\pgfqpoint{1.643210in}{0.404954in}}%
\pgfpathlineto{\pgfqpoint{1.632777in}{0.391324in}}%
\pgfpathlineto{\pgfqpoint{1.627685in}{0.395380in}}%
\pgfpathlineto{\pgfqpoint{1.620845in}{0.388865in}}%
\pgfpathlineto{\pgfqpoint{1.629369in}{0.383174in}}%
\pgfpathlineto{\pgfqpoint{1.624329in}{0.374856in}}%
\pgfpathlineto{\pgfqpoint{1.618135in}{0.355164in}}%
\pgfpathlineto{\pgfqpoint{1.609200in}{0.336199in}}%
\pgfpathlineto{\pgfqpoint{1.613211in}{0.323767in}}%
\pgfpathlineto{\pgfqpoint{1.620029in}{0.294458in}}%
\pgfpathlineto{\pgfqpoint{1.620643in}{0.282596in}}%
\pgfpathlineto{\pgfqpoint{1.631178in}{0.267038in}}%
\pgfpathlineto{\pgfqpoint{1.623054in}{0.267947in}}%
\pgfpathlineto{\pgfqpoint{1.615164in}{0.260098in}}%
\pgfpathlineto{\pgfqpoint{1.605923in}{0.266267in}}%
\pgfpathlineto{\pgfqpoint{1.602605in}{0.272421in}}%
\pgfpathlineto{\pgfqpoint{1.589453in}{0.275284in}}%
\pgfpathlineto{\pgfqpoint{1.569363in}{0.275635in}}%
\pgfpathlineto{\pgfqpoint{1.554561in}{0.287289in}}%
\pgfpathlineto{\pgfqpoint{1.541120in}{0.289236in}}%
\pgfpathlineto{\pgfqpoint{1.532966in}{0.298497in}}%
\pgfpathlineto{\pgfqpoint{1.515889in}{0.302225in}}%
\pgfpathlineto{\pgfqpoint{1.506552in}{0.332004in}}%
\pgfpathlineto{\pgfqpoint{1.497004in}{0.343934in}}%
\pgfpathlineto{\pgfqpoint{1.498626in}{0.355270in}}%
\pgfpathlineto{\pgfqpoint{1.492706in}{0.363554in}}%
\pgfpathlineto{\pgfqpoint{1.496421in}{0.374877in}}%
\pgfpathlineto{\pgfqpoint{1.493356in}{0.383175in}}%
\pgfpathlineto{\pgfqpoint{1.483762in}{0.386934in}}%
\pgfpathlineto{\pgfqpoint{1.474791in}{0.396502in}}%
\pgfpathlineto{\pgfqpoint{1.471522in}{0.409312in}}%
\pgfpathlineto{\pgfqpoint{1.463033in}{0.420954in}}%
\pgfpathlineto{\pgfqpoint{1.451735in}{0.430012in}}%
\pgfpathlineto{\pgfqpoint{1.441558in}{0.455996in}}%
\pgfpathlineto{\pgfqpoint{1.433340in}{0.484413in}}%
\pgfpathlineto{\pgfqpoint{1.426597in}{0.495650in}}%
\pgfpathlineto{\pgfqpoint{1.414908in}{0.505109in}}%
\pgfpathlineto{\pgfqpoint{1.411987in}{0.511975in}}%
\pgfpathlineto{\pgfqpoint{1.401042in}{0.516238in}}%
\pgfpathlineto{\pgfqpoint{1.390239in}{0.534431in}}%
\pgfpathlineto{\pgfqpoint{1.380374in}{0.533035in}}%
\pgfpathlineto{\pgfqpoint{1.370356in}{0.537579in}}%
\pgfpathlineto{\pgfqpoint{1.356156in}{0.536700in}}%
\pgfpathlineto{\pgfqpoint{1.341731in}{0.544199in}}%
\pgfpathlineto{\pgfqpoint{1.337663in}{0.537067in}}%
\pgfpathlineto{\pgfqpoint{1.320804in}{0.536889in}}%
\pgfpathlineto{\pgfqpoint{1.312222in}{0.523370in}}%
\pgfpathlineto{\pgfqpoint{1.304757in}{0.506631in}}%
\pgfpathlineto{\pgfqpoint{1.288883in}{0.488663in}}%
\pgfpathlineto{\pgfqpoint{1.270899in}{0.496548in}}%
\pgfpathlineto{\pgfqpoint{1.268347in}{0.501760in}}%
\pgfpathlineto{\pgfqpoint{1.257418in}{0.505734in}}%
\pgfpathlineto{\pgfqpoint{1.254061in}{0.511171in}}%
\pgfpathlineto{\pgfqpoint{1.239552in}{0.516660in}}%
\pgfpathlineto{\pgfqpoint{1.231436in}{0.527879in}}%
\pgfpathlineto{\pgfqpoint{1.221957in}{0.533292in}}%
\pgfpathlineto{\pgfqpoint{1.213802in}{0.542738in}}%
\pgfpathlineto{\pgfqpoint{1.207448in}{0.558735in}}%
\pgfpathlineto{\pgfqpoint{1.208159in}{0.580595in}}%
\pgfpathlineto{\pgfqpoint{1.200706in}{0.591641in}}%
\pgfpathlineto{\pgfqpoint{1.199848in}{0.603605in}}%
\pgfpathlineto{\pgfqpoint{1.183312in}{0.621518in}}%
\pgfpathlineto{\pgfqpoint{1.173690in}{0.625342in}}%
\pgfpathlineto{\pgfqpoint{1.163479in}{0.642057in}}%
\pgfpathlineto{\pgfqpoint{1.154778in}{0.648698in}}%
\pgfpathlineto{\pgfqpoint{1.143723in}{0.664869in}}%
\pgfpathlineto{\pgfqpoint{1.132411in}{0.671867in}}%
\pgfpathlineto{\pgfqpoint{1.125003in}{0.689800in}}%
\pgfpathlineto{\pgfqpoint{1.116758in}{0.694333in}}%
\pgfpathclose%
\pgfusepath{fill}%
\end{pgfscope}%
\begin{pgfscope}%
\pgfpathrectangle{\pgfqpoint{0.100000in}{0.100000in}}{\pgfqpoint{2.989028in}{1.913466in}}%
\pgfusepath{clip}%
\pgfsetbuttcap%
\pgfsetmiterjoin%
\definecolor{currentfill}{rgb}{0.580392,0.831373,0.644444}%
\pgfsetfillcolor{currentfill}%
\pgfsetlinewidth{0.000000pt}%
\definecolor{currentstroke}{rgb}{0.000000,0.000000,0.000000}%
\pgfsetstrokecolor{currentstroke}%
\pgfsetstrokeopacity{0.000000}%
\pgfsetdash{}{0pt}%
\pgfpathmoveto{\pgfqpoint{1.030137in}{1.052362in}}%
\pgfpathlineto{\pgfqpoint{1.111744in}{1.040510in}}%
\pgfpathlineto{\pgfqpoint{1.143359in}{1.035639in}}%
\pgfpathlineto{\pgfqpoint{1.232477in}{1.025040in}}%
\pgfpathlineto{\pgfqpoint{1.292460in}{1.018783in}}%
\pgfpathlineto{\pgfqpoint{1.344368in}{1.014183in}}%
\pgfpathlineto{\pgfqpoint{1.341469in}{0.981143in}}%
\pgfpathlineto{\pgfqpoint{1.339896in}{0.981231in}}%
\pgfpathlineto{\pgfqpoint{1.335142in}{0.924525in}}%
\pgfpathlineto{\pgfqpoint{1.330041in}{0.867115in}}%
\pgfpathlineto{\pgfqpoint{1.324140in}{0.807015in}}%
\pgfpathlineto{\pgfqpoint{1.315044in}{0.721225in}}%
\pgfpathlineto{\pgfqpoint{1.312135in}{0.687496in}}%
\pgfpathlineto{\pgfqpoint{1.258660in}{0.692892in}}%
\pgfpathlineto{\pgfqpoint{1.212359in}{0.697442in}}%
\pgfpathlineto{\pgfqpoint{1.148324in}{0.704715in}}%
\pgfpathlineto{\pgfqpoint{1.113886in}{0.708987in}}%
\pgfpathlineto{\pgfqpoint{1.111859in}{0.702205in}}%
\pgfpathlineto{\pgfqpoint{1.116758in}{0.694333in}}%
\pgfpathlineto{\pgfqpoint{1.075370in}{0.699708in}}%
\pgfpathlineto{\pgfqpoint{1.024306in}{0.707037in}}%
\pgfpathlineto{\pgfqpoint{1.019659in}{0.678155in}}%
\pgfpathlineto{\pgfqpoint{0.973080in}{0.685166in}}%
\pgfpathlineto{\pgfqpoint{0.984025in}{0.756815in}}%
\pgfpathlineto{\pgfqpoint{0.997573in}{0.844052in}}%
\pgfpathlineto{\pgfqpoint{1.009619in}{0.920049in}}%
\pgfpathlineto{\pgfqpoint{1.020079in}{0.987601in}}%
\pgfpathlineto{\pgfqpoint{1.030137in}{1.052362in}}%
\pgfpathclose%
\pgfusepath{fill}%
\end{pgfscope}%
\begin{pgfscope}%
\pgfpathrectangle{\pgfqpoint{0.100000in}{0.100000in}}{\pgfqpoint{2.989028in}{1.913466in}}%
\pgfusepath{clip}%
\pgfsetbuttcap%
\pgfsetmiterjoin%
\definecolor{currentfill}{rgb}{0.982699,0.993080,0.722030}%
\pgfsetfillcolor{currentfill}%
\pgfsetlinewidth{0.000000pt}%
\definecolor{currentstroke}{rgb}{0.000000,0.000000,0.000000}%
\pgfsetstrokecolor{currentstroke}%
\pgfsetstrokeopacity{0.000000}%
\pgfsetdash{}{0pt}%
\pgfpathmoveto{\pgfqpoint{2.327687in}{0.632577in}}%
\pgfpathlineto{\pgfqpoint{2.249834in}{0.623930in}}%
\pgfpathlineto{\pgfqpoint{2.181207in}{0.618602in}}%
\pgfpathlineto{\pgfqpoint{2.180349in}{0.610192in}}%
\pgfpathlineto{\pgfqpoint{2.186652in}{0.602188in}}%
\pgfpathlineto{\pgfqpoint{2.194346in}{0.597485in}}%
\pgfpathlineto{\pgfqpoint{2.194230in}{0.585103in}}%
\pgfpathlineto{\pgfqpoint{2.192191in}{0.576831in}}%
\pgfpathlineto{\pgfqpoint{2.185400in}{0.570869in}}%
\pgfpathlineto{\pgfqpoint{2.175936in}{0.571500in}}%
\pgfpathlineto{\pgfqpoint{2.166987in}{0.578887in}}%
\pgfpathlineto{\pgfqpoint{2.165392in}{0.592032in}}%
\pgfpathlineto{\pgfqpoint{2.158727in}{0.599677in}}%
\pgfpathlineto{\pgfqpoint{2.154190in}{0.572315in}}%
\pgfpathlineto{\pgfqpoint{2.138775in}{0.574915in}}%
\pgfpathlineto{\pgfqpoint{2.133415in}{0.622761in}}%
\pgfpathlineto{\pgfqpoint{2.127610in}{0.673185in}}%
\pgfpathlineto{\pgfqpoint{2.129766in}{0.745935in}}%
\pgfpathlineto{\pgfqpoint{2.131104in}{0.795857in}}%
\pgfpathlineto{\pgfqpoint{2.133950in}{0.872025in}}%
\pgfpathlineto{\pgfqpoint{2.127858in}{0.878728in}}%
\pgfpathlineto{\pgfqpoint{2.138945in}{0.880335in}}%
\pgfpathlineto{\pgfqpoint{2.203533in}{0.884512in}}%
\pgfpathlineto{\pgfqpoint{2.266375in}{0.890163in}}%
\pgfpathlineto{\pgfqpoint{2.282908in}{0.832251in}}%
\pgfpathlineto{\pgfqpoint{2.294184in}{0.789743in}}%
\pgfpathlineto{\pgfqpoint{2.304291in}{0.754156in}}%
\pgfpathlineto{\pgfqpoint{2.310136in}{0.739826in}}%
\pgfpathlineto{\pgfqpoint{2.319340in}{0.726491in}}%
\pgfpathlineto{\pgfqpoint{2.317853in}{0.719705in}}%
\pgfpathlineto{\pgfqpoint{2.324428in}{0.716398in}}%
\pgfpathlineto{\pgfqpoint{2.316659in}{0.706111in}}%
\pgfpathlineto{\pgfqpoint{2.314034in}{0.687792in}}%
\pgfpathlineto{\pgfqpoint{2.321628in}{0.666365in}}%
\pgfpathlineto{\pgfqpoint{2.320573in}{0.644805in}}%
\pgfpathlineto{\pgfqpoint{2.327687in}{0.632577in}}%
\pgfpathclose%
\pgfusepath{fill}%
\end{pgfscope}%
\begin{pgfscope}%
\pgfpathrectangle{\pgfqpoint{0.100000in}{0.100000in}}{\pgfqpoint{2.989028in}{1.913466in}}%
\pgfusepath{clip}%
\pgfsetbuttcap%
\pgfsetmiterjoin%
\definecolor{currentfill}{rgb}{0.995617,0.855363,0.525721}%
\pgfsetfillcolor{currentfill}%
\pgfsetlinewidth{0.000000pt}%
\definecolor{currentstroke}{rgb}{0.000000,0.000000,0.000000}%
\pgfsetstrokecolor{currentstroke}%
\pgfsetstrokeopacity{0.000000}%
\pgfsetdash{}{0pt}%
\pgfpathmoveto{\pgfqpoint{2.138775in}{0.574915in}}%
\pgfpathlineto{\pgfqpoint{2.122957in}{0.570389in}}%
\pgfpathlineto{\pgfqpoint{2.111166in}{0.573295in}}%
\pgfpathlineto{\pgfqpoint{2.089268in}{0.566330in}}%
\pgfpathlineto{\pgfqpoint{2.072753in}{0.557360in}}%
\pgfpathlineto{\pgfqpoint{2.065977in}{0.573569in}}%
\pgfpathlineto{\pgfqpoint{2.058992in}{0.580363in}}%
\pgfpathlineto{\pgfqpoint{2.055622in}{0.588493in}}%
\pgfpathlineto{\pgfqpoint{2.060562in}{0.610498in}}%
\pgfpathlineto{\pgfqpoint{2.013761in}{0.607617in}}%
\pgfpathlineto{\pgfqpoint{1.953073in}{0.604986in}}%
\pgfpathlineto{\pgfqpoint{1.956684in}{0.610463in}}%
\pgfpathlineto{\pgfqpoint{1.952221in}{0.620808in}}%
\pgfpathlineto{\pgfqpoint{1.956668in}{0.622916in}}%
\pgfpathlineto{\pgfqpoint{1.955681in}{0.632856in}}%
\pgfpathlineto{\pgfqpoint{1.963432in}{0.642495in}}%
\pgfpathlineto{\pgfqpoint{1.967673in}{0.661207in}}%
\pgfpathlineto{\pgfqpoint{1.981859in}{0.673535in}}%
\pgfpathlineto{\pgfqpoint{1.976601in}{0.685508in}}%
\pgfpathlineto{\pgfqpoint{1.986660in}{0.692400in}}%
\pgfpathlineto{\pgfqpoint{1.977983in}{0.704877in}}%
\pgfpathlineto{\pgfqpoint{1.971703in}{0.733439in}}%
\pgfpathlineto{\pgfqpoint{1.974085in}{0.738625in}}%
\pgfpathlineto{\pgfqpoint{1.977845in}{0.748573in}}%
\pgfpathlineto{\pgfqpoint{1.974345in}{0.759012in}}%
\pgfpathlineto{\pgfqpoint{1.976057in}{0.763762in}}%
\pgfpathlineto{\pgfqpoint{1.969899in}{0.781706in}}%
\pgfpathlineto{\pgfqpoint{1.988077in}{0.813926in}}%
\pgfpathlineto{\pgfqpoint{1.988922in}{0.822501in}}%
\pgfpathlineto{\pgfqpoint{1.997678in}{0.828737in}}%
\pgfpathlineto{\pgfqpoint{2.007021in}{0.849323in}}%
\pgfpathlineto{\pgfqpoint{2.006577in}{0.857690in}}%
\pgfpathlineto{\pgfqpoint{2.018214in}{0.866243in}}%
\pgfpathlineto{\pgfqpoint{2.014584in}{0.871310in}}%
\pgfpathlineto{\pgfqpoint{2.084319in}{0.875379in}}%
\pgfpathlineto{\pgfqpoint{2.127858in}{0.878728in}}%
\pgfpathlineto{\pgfqpoint{2.133950in}{0.872025in}}%
\pgfpathlineto{\pgfqpoint{2.131104in}{0.795857in}}%
\pgfpathlineto{\pgfqpoint{2.129766in}{0.745935in}}%
\pgfpathlineto{\pgfqpoint{2.127610in}{0.673185in}}%
\pgfpathlineto{\pgfqpoint{2.133415in}{0.622761in}}%
\pgfpathlineto{\pgfqpoint{2.138775in}{0.574915in}}%
\pgfpathclose%
\pgfusepath{fill}%
\end{pgfscope}%
\begin{pgfscope}%
\pgfpathrectangle{\pgfqpoint{0.100000in}{0.100000in}}{\pgfqpoint{2.989028in}{1.913466in}}%
\pgfusepath{clip}%
\pgfsetbuttcap%
\pgfsetmiterjoin%
\definecolor{currentfill}{rgb}{0.982699,0.993080,0.722030}%
\pgfsetfillcolor{currentfill}%
\pgfsetlinewidth{0.000000pt}%
\definecolor{currentstroke}{rgb}{0.000000,0.000000,0.000000}%
\pgfsetstrokecolor{currentstroke}%
\pgfsetstrokeopacity{0.000000}%
\pgfsetdash{}{0pt}%
\pgfpathmoveto{\pgfqpoint{2.266375in}{0.890163in}}%
\pgfpathlineto{\pgfqpoint{2.335069in}{0.897708in}}%
\pgfpathlineto{\pgfqpoint{2.376402in}{0.902801in}}%
\pgfpathlineto{\pgfqpoint{2.399943in}{0.906551in}}%
\pgfpathlineto{\pgfqpoint{2.390192in}{0.891171in}}%
\pgfpathlineto{\pgfqpoint{2.390198in}{0.883919in}}%
\pgfpathlineto{\pgfqpoint{2.400235in}{0.880009in}}%
\pgfpathlineto{\pgfqpoint{2.407061in}{0.873695in}}%
\pgfpathlineto{\pgfqpoint{2.417377in}{0.872915in}}%
\pgfpathlineto{\pgfqpoint{2.427019in}{0.855143in}}%
\pgfpathlineto{\pgfqpoint{2.437478in}{0.842609in}}%
\pgfpathlineto{\pgfqpoint{2.456053in}{0.832075in}}%
\pgfpathlineto{\pgfqpoint{2.461332in}{0.823674in}}%
\pgfpathlineto{\pgfqpoint{2.477896in}{0.814587in}}%
\pgfpathlineto{\pgfqpoint{2.479142in}{0.807998in}}%
\pgfpathlineto{\pgfqpoint{2.492579in}{0.794896in}}%
\pgfpathlineto{\pgfqpoint{2.502872in}{0.791165in}}%
\pgfpathlineto{\pgfqpoint{2.510161in}{0.778793in}}%
\pgfpathlineto{\pgfqpoint{2.513387in}{0.764886in}}%
\pgfpathlineto{\pgfqpoint{2.524078in}{0.759511in}}%
\pgfpathlineto{\pgfqpoint{2.532685in}{0.744563in}}%
\pgfpathlineto{\pgfqpoint{2.535460in}{0.733570in}}%
\pgfpathlineto{\pgfqpoint{2.548073in}{0.728890in}}%
\pgfpathlineto{\pgfqpoint{2.537490in}{0.708626in}}%
\pgfpathlineto{\pgfqpoint{2.533507in}{0.696651in}}%
\pgfpathlineto{\pgfqpoint{2.536105in}{0.691037in}}%
\pgfpathlineto{\pgfqpoint{2.531568in}{0.681708in}}%
\pgfpathlineto{\pgfqpoint{2.529639in}{0.668810in}}%
\pgfpathlineto{\pgfqpoint{2.521570in}{0.666408in}}%
\pgfpathlineto{\pgfqpoint{2.525831in}{0.654599in}}%
\pgfpathlineto{\pgfqpoint{2.525559in}{0.639617in}}%
\pgfpathlineto{\pgfqpoint{2.511217in}{0.640178in}}%
\pgfpathlineto{\pgfqpoint{2.501225in}{0.642912in}}%
\pgfpathlineto{\pgfqpoint{2.496902in}{0.637830in}}%
\pgfpathlineto{\pgfqpoint{2.500128in}{0.625826in}}%
\pgfpathlineto{\pgfqpoint{2.499427in}{0.611870in}}%
\pgfpathlineto{\pgfqpoint{2.493123in}{0.610802in}}%
\pgfpathlineto{\pgfqpoint{2.488024in}{0.623836in}}%
\pgfpathlineto{\pgfqpoint{2.436103in}{0.620376in}}%
\pgfpathlineto{\pgfqpoint{2.370635in}{0.616804in}}%
\pgfpathlineto{\pgfqpoint{2.337617in}{0.614480in}}%
\pgfpathlineto{\pgfqpoint{2.327687in}{0.632577in}}%
\pgfpathlineto{\pgfqpoint{2.320573in}{0.644805in}}%
\pgfpathlineto{\pgfqpoint{2.321628in}{0.666365in}}%
\pgfpathlineto{\pgfqpoint{2.314034in}{0.687792in}}%
\pgfpathlineto{\pgfqpoint{2.316659in}{0.706111in}}%
\pgfpathlineto{\pgfqpoint{2.324428in}{0.716398in}}%
\pgfpathlineto{\pgfqpoint{2.317853in}{0.719705in}}%
\pgfpathlineto{\pgfqpoint{2.319340in}{0.726491in}}%
\pgfpathlineto{\pgfqpoint{2.310136in}{0.739826in}}%
\pgfpathlineto{\pgfqpoint{2.304291in}{0.754156in}}%
\pgfpathlineto{\pgfqpoint{2.294184in}{0.789743in}}%
\pgfpathlineto{\pgfqpoint{2.282908in}{0.832251in}}%
\pgfpathlineto{\pgfqpoint{2.266375in}{0.890163in}}%
\pgfpathclose%
\pgfusepath{fill}%
\end{pgfscope}%
\begin{pgfscope}%
\pgfpathrectangle{\pgfqpoint{0.100000in}{0.100000in}}{\pgfqpoint{2.989028in}{1.913466in}}%
\pgfusepath{clip}%
\pgfsetbuttcap%
\pgfsetmiterjoin%
\definecolor{currentfill}{rgb}{0.865667,0.946021,0.603460}%
\pgfsetfillcolor{currentfill}%
\pgfsetlinewidth{0.000000pt}%
\definecolor{currentstroke}{rgb}{0.000000,0.000000,0.000000}%
\pgfsetstrokecolor{currentstroke}%
\pgfsetstrokeopacity{0.000000}%
\pgfsetdash{}{0pt}%
\pgfpathmoveto{\pgfqpoint{2.399943in}{0.906551in}}%
\pgfpathlineto{\pgfqpoint{2.427183in}{0.920102in}}%
\pgfpathlineto{\pgfqpoint{2.442269in}{0.925218in}}%
\pgfpathlineto{\pgfqpoint{2.507904in}{0.932038in}}%
\pgfpathlineto{\pgfqpoint{2.514855in}{0.929779in}}%
\pgfpathlineto{\pgfqpoint{2.524037in}{0.920550in}}%
\pgfpathlineto{\pgfqpoint{2.524543in}{0.912354in}}%
\pgfpathlineto{\pgfqpoint{2.584952in}{0.921230in}}%
\pgfpathlineto{\pgfqpoint{2.653548in}{0.872057in}}%
\pgfpathlineto{\pgfqpoint{2.640684in}{0.858649in}}%
\pgfpathlineto{\pgfqpoint{2.623033in}{0.827511in}}%
\pgfpathlineto{\pgfqpoint{2.628057in}{0.820819in}}%
\pgfpathlineto{\pgfqpoint{2.618641in}{0.807830in}}%
\pgfpathlineto{\pgfqpoint{2.609275in}{0.806359in}}%
\pgfpathlineto{\pgfqpoint{2.609240in}{0.798554in}}%
\pgfpathlineto{\pgfqpoint{2.602464in}{0.790386in}}%
\pgfpathlineto{\pgfqpoint{2.594032in}{0.788730in}}%
\pgfpathlineto{\pgfqpoint{2.595872in}{0.781473in}}%
\pgfpathlineto{\pgfqpoint{2.591167in}{0.775902in}}%
\pgfpathlineto{\pgfqpoint{2.579911in}{0.771089in}}%
\pgfpathlineto{\pgfqpoint{2.566192in}{0.760141in}}%
\pgfpathlineto{\pgfqpoint{2.568782in}{0.753058in}}%
\pgfpathlineto{\pgfqpoint{2.560148in}{0.748611in}}%
\pgfpathlineto{\pgfqpoint{2.552866in}{0.753287in}}%
\pgfpathlineto{\pgfqpoint{2.547540in}{0.732958in}}%
\pgfpathlineto{\pgfqpoint{2.535460in}{0.733570in}}%
\pgfpathlineto{\pgfqpoint{2.532685in}{0.744563in}}%
\pgfpathlineto{\pgfqpoint{2.524078in}{0.759511in}}%
\pgfpathlineto{\pgfqpoint{2.513387in}{0.764886in}}%
\pgfpathlineto{\pgfqpoint{2.510161in}{0.778793in}}%
\pgfpathlineto{\pgfqpoint{2.502872in}{0.791165in}}%
\pgfpathlineto{\pgfqpoint{2.492579in}{0.794896in}}%
\pgfpathlineto{\pgfqpoint{2.479142in}{0.807998in}}%
\pgfpathlineto{\pgfqpoint{2.477896in}{0.814587in}}%
\pgfpathlineto{\pgfqpoint{2.461332in}{0.823674in}}%
\pgfpathlineto{\pgfqpoint{2.456053in}{0.832075in}}%
\pgfpathlineto{\pgfqpoint{2.437478in}{0.842609in}}%
\pgfpathlineto{\pgfqpoint{2.427019in}{0.855143in}}%
\pgfpathlineto{\pgfqpoint{2.417377in}{0.872915in}}%
\pgfpathlineto{\pgfqpoint{2.407061in}{0.873695in}}%
\pgfpathlineto{\pgfqpoint{2.400235in}{0.880009in}}%
\pgfpathlineto{\pgfqpoint{2.390198in}{0.883919in}}%
\pgfpathlineto{\pgfqpoint{2.390192in}{0.891171in}}%
\pgfpathlineto{\pgfqpoint{2.399943in}{0.906551in}}%
\pgfpathclose%
\pgfusepath{fill}%
\end{pgfscope}%
\begin{pgfscope}%
\pgfpathrectangle{\pgfqpoint{0.100000in}{0.100000in}}{\pgfqpoint{2.989028in}{1.913466in}}%
\pgfusepath{clip}%
\pgfsetbuttcap%
\pgfsetmiterjoin%
\definecolor{currentfill}{rgb}{0.999000,0.969012,0.697040}%
\pgfsetfillcolor{currentfill}%
\pgfsetlinewidth{0.000000pt}%
\definecolor{currentstroke}{rgb}{0.000000,0.000000,0.000000}%
\pgfsetstrokecolor{currentstroke}%
\pgfsetstrokeopacity{0.000000}%
\pgfsetdash{}{0pt}%
\pgfpathmoveto{\pgfqpoint{1.782462in}{0.962463in}}%
\pgfpathlineto{\pgfqpoint{1.822491in}{0.962824in}}%
\pgfpathlineto{\pgfqpoint{1.875448in}{0.963789in}}%
\pgfpathlineto{\pgfqpoint{1.965941in}{0.966662in}}%
\pgfpathlineto{\pgfqpoint{2.017700in}{0.969399in}}%
\pgfpathlineto{\pgfqpoint{2.023251in}{0.962527in}}%
\pgfpathlineto{\pgfqpoint{2.022904in}{0.955269in}}%
\pgfpathlineto{\pgfqpoint{2.010365in}{0.942748in}}%
\pgfpathlineto{\pgfqpoint{2.007334in}{0.935892in}}%
\pgfpathlineto{\pgfqpoint{2.042109in}{0.938464in}}%
\pgfpathlineto{\pgfqpoint{2.042090in}{0.925797in}}%
\pgfpathlineto{\pgfqpoint{2.030934in}{0.920384in}}%
\pgfpathlineto{\pgfqpoint{2.031092in}{0.911777in}}%
\pgfpathlineto{\pgfqpoint{2.026902in}{0.905590in}}%
\pgfpathlineto{\pgfqpoint{2.024159in}{0.892449in}}%
\pgfpathlineto{\pgfqpoint{2.026990in}{0.881608in}}%
\pgfpathlineto{\pgfqpoint{2.014584in}{0.871310in}}%
\pgfpathlineto{\pgfqpoint{2.018214in}{0.866243in}}%
\pgfpathlineto{\pgfqpoint{2.006577in}{0.857690in}}%
\pgfpathlineto{\pgfqpoint{2.007021in}{0.849323in}}%
\pgfpathlineto{\pgfqpoint{1.997678in}{0.828737in}}%
\pgfpathlineto{\pgfqpoint{1.988922in}{0.822501in}}%
\pgfpathlineto{\pgfqpoint{1.988077in}{0.813926in}}%
\pgfpathlineto{\pgfqpoint{1.969899in}{0.781706in}}%
\pgfpathlineto{\pgfqpoint{1.976057in}{0.763762in}}%
\pgfpathlineto{\pgfqpoint{1.974345in}{0.759012in}}%
\pgfpathlineto{\pgfqpoint{1.977845in}{0.748573in}}%
\pgfpathlineto{\pgfqpoint{1.974085in}{0.738625in}}%
\pgfpathlineto{\pgfqpoint{1.924378in}{0.736573in}}%
\pgfpathlineto{\pgfqpoint{1.859848in}{0.735510in}}%
\pgfpathlineto{\pgfqpoint{1.815340in}{0.735108in}}%
\pgfpathlineto{\pgfqpoint{1.815123in}{0.770153in}}%
\pgfpathlineto{\pgfqpoint{1.804070in}{0.772449in}}%
\pgfpathlineto{\pgfqpoint{1.796775in}{0.769451in}}%
\pgfpathlineto{\pgfqpoint{1.790948in}{0.774957in}}%
\pgfpathlineto{\pgfqpoint{1.791549in}{0.812079in}}%
\pgfpathlineto{\pgfqpoint{1.792853in}{0.891120in}}%
\pgfpathlineto{\pgfqpoint{1.786531in}{0.937404in}}%
\pgfpathlineto{\pgfqpoint{1.782462in}{0.962463in}}%
\pgfpathclose%
\pgfusepath{fill}%
\end{pgfscope}%
\begin{pgfscope}%
\pgfpathrectangle{\pgfqpoint{0.100000in}{0.100000in}}{\pgfqpoint{2.989028in}{1.913466in}}%
\pgfusepath{clip}%
\pgfsetbuttcap%
\pgfsetmiterjoin%
\definecolor{currentfill}{rgb}{0.665283,0.864591,0.643214}%
\pgfsetfillcolor{currentfill}%
\pgfsetlinewidth{0.000000pt}%
\definecolor{currentstroke}{rgb}{0.000000,0.000000,0.000000}%
\pgfsetstrokecolor{currentstroke}%
\pgfsetstrokeopacity{0.000000}%
\pgfsetdash{}{0pt}%
\pgfpathmoveto{\pgfqpoint{1.815340in}{0.735108in}}%
\pgfpathlineto{\pgfqpoint{1.859848in}{0.735510in}}%
\pgfpathlineto{\pgfqpoint{1.924378in}{0.736573in}}%
\pgfpathlineto{\pgfqpoint{1.974085in}{0.738625in}}%
\pgfpathlineto{\pgfqpoint{1.971703in}{0.733439in}}%
\pgfpathlineto{\pgfqpoint{1.977983in}{0.704877in}}%
\pgfpathlineto{\pgfqpoint{1.986660in}{0.692400in}}%
\pgfpathlineto{\pgfqpoint{1.976601in}{0.685508in}}%
\pgfpathlineto{\pgfqpoint{1.981859in}{0.673535in}}%
\pgfpathlineto{\pgfqpoint{1.967673in}{0.661207in}}%
\pgfpathlineto{\pgfqpoint{1.963432in}{0.642495in}}%
\pgfpathlineto{\pgfqpoint{1.955681in}{0.632856in}}%
\pgfpathlineto{\pgfqpoint{1.956668in}{0.622916in}}%
\pgfpathlineto{\pgfqpoint{1.952221in}{0.620808in}}%
\pgfpathlineto{\pgfqpoint{1.956684in}{0.610463in}}%
\pgfpathlineto{\pgfqpoint{1.953073in}{0.604986in}}%
\pgfpathlineto{\pgfqpoint{2.013761in}{0.607617in}}%
\pgfpathlineto{\pgfqpoint{2.060562in}{0.610498in}}%
\pgfpathlineto{\pgfqpoint{2.055622in}{0.588493in}}%
\pgfpathlineto{\pgfqpoint{2.058992in}{0.580363in}}%
\pgfpathlineto{\pgfqpoint{2.065977in}{0.573569in}}%
\pgfpathlineto{\pgfqpoint{2.072753in}{0.557360in}}%
\pgfpathlineto{\pgfqpoint{2.051350in}{0.561094in}}%
\pgfpathlineto{\pgfqpoint{2.043455in}{0.567234in}}%
\pgfpathlineto{\pgfqpoint{2.034048in}{0.567526in}}%
\pgfpathlineto{\pgfqpoint{2.024158in}{0.554082in}}%
\pgfpathlineto{\pgfqpoint{2.026129in}{0.547960in}}%
\pgfpathlineto{\pgfqpoint{2.057724in}{0.544257in}}%
\pgfpathlineto{\pgfqpoint{2.065995in}{0.537220in}}%
\pgfpathlineto{\pgfqpoint{2.079454in}{0.533600in}}%
\pgfpathlineto{\pgfqpoint{2.073576in}{0.525268in}}%
\pgfpathlineto{\pgfqpoint{2.063672in}{0.518003in}}%
\pgfpathlineto{\pgfqpoint{2.089813in}{0.501664in}}%
\pgfpathlineto{\pgfqpoint{2.101990in}{0.499111in}}%
\pgfpathlineto{\pgfqpoint{2.107054in}{0.485755in}}%
\pgfpathlineto{\pgfqpoint{2.095589in}{0.483298in}}%
\pgfpathlineto{\pgfqpoint{2.074041in}{0.500096in}}%
\pgfpathlineto{\pgfqpoint{2.065616in}{0.502430in}}%
\pgfpathlineto{\pgfqpoint{2.061539in}{0.509066in}}%
\pgfpathlineto{\pgfqpoint{2.049043in}{0.506343in}}%
\pgfpathlineto{\pgfqpoint{2.045130in}{0.497792in}}%
\pgfpathlineto{\pgfqpoint{2.047608in}{0.488264in}}%
\pgfpathlineto{\pgfqpoint{2.039195in}{0.482642in}}%
\pgfpathlineto{\pgfqpoint{2.031517in}{0.496493in}}%
\pgfpathlineto{\pgfqpoint{2.018069in}{0.492363in}}%
\pgfpathlineto{\pgfqpoint{2.013030in}{0.484098in}}%
\pgfpathlineto{\pgfqpoint{2.003479in}{0.486449in}}%
\pgfpathlineto{\pgfqpoint{1.997373in}{0.496886in}}%
\pgfpathlineto{\pgfqpoint{1.981173in}{0.500333in}}%
\pgfpathlineto{\pgfqpoint{1.978115in}{0.505764in}}%
\pgfpathlineto{\pgfqpoint{1.961236in}{0.515225in}}%
\pgfpathlineto{\pgfqpoint{1.957047in}{0.523496in}}%
\pgfpathlineto{\pgfqpoint{1.942910in}{0.520116in}}%
\pgfpathlineto{\pgfqpoint{1.944705in}{0.527689in}}%
\pgfpathlineto{\pgfqpoint{1.927134in}{0.519920in}}%
\pgfpathlineto{\pgfqpoint{1.931854in}{0.511860in}}%
\pgfpathlineto{\pgfqpoint{1.918284in}{0.507095in}}%
\pgfpathlineto{\pgfqpoint{1.900311in}{0.509714in}}%
\pgfpathlineto{\pgfqpoint{1.863934in}{0.522174in}}%
\pgfpathlineto{\pgfqpoint{1.835864in}{0.519701in}}%
\pgfpathlineto{\pgfqpoint{1.833432in}{0.536068in}}%
\pgfpathlineto{\pgfqpoint{1.836956in}{0.543537in}}%
\pgfpathlineto{\pgfqpoint{1.836655in}{0.555420in}}%
\pgfpathlineto{\pgfqpoint{1.833187in}{0.558251in}}%
\pgfpathlineto{\pgfqpoint{1.834363in}{0.571815in}}%
\pgfpathlineto{\pgfqpoint{1.844586in}{0.590680in}}%
\pgfpathlineto{\pgfqpoint{1.845894in}{0.597808in}}%
\pgfpathlineto{\pgfqpoint{1.844209in}{0.614644in}}%
\pgfpathlineto{\pgfqpoint{1.836688in}{0.622574in}}%
\pgfpathlineto{\pgfqpoint{1.832761in}{0.637381in}}%
\pgfpathlineto{\pgfqpoint{1.827929in}{0.640781in}}%
\pgfpathlineto{\pgfqpoint{1.830092in}{0.648811in}}%
\pgfpathlineto{\pgfqpoint{1.823939in}{0.660813in}}%
\pgfpathlineto{\pgfqpoint{1.816262in}{0.667326in}}%
\pgfpathlineto{\pgfqpoint{1.815340in}{0.735108in}}%
\pgfpathclose%
\pgfusepath{fill}%
\end{pgfscope}%
\begin{pgfscope}%
\pgfpathrectangle{\pgfqpoint{0.100000in}{0.100000in}}{\pgfqpoint{2.989028in}{1.913466in}}%
\pgfusepath{clip}%
\pgfsetbuttcap%
\pgfsetmiterjoin%
\definecolor{currentfill}{rgb}{0.665283,0.864591,0.643214}%
\pgfsetfillcolor{currentfill}%
\pgfsetlinewidth{0.000000pt}%
\definecolor{currentstroke}{rgb}{0.000000,0.000000,0.000000}%
\pgfsetstrokecolor{currentstroke}%
\pgfsetstrokeopacity{0.000000}%
\pgfsetdash{}{0pt}%
\pgfpathmoveto{\pgfqpoint{1.934378in}{0.511243in}}%
\pgfpathlineto{\pgfqpoint{1.940817in}{0.515073in}}%
\pgfpathlineto{\pgfqpoint{1.948633in}{0.510545in}}%
\pgfpathlineto{\pgfqpoint{1.944274in}{0.504314in}}%
\pgfpathlineto{\pgfqpoint{1.934378in}{0.511243in}}%
\pgfpathclose%
\pgfusepath{fill}%
\end{pgfscope}%
\begin{pgfscope}%
\pgfpathrectangle{\pgfqpoint{0.100000in}{0.100000in}}{\pgfqpoint{2.989028in}{1.913466in}}%
\pgfusepath{clip}%
\pgfsetbuttcap%
\pgfsetmiterjoin%
\definecolor{currentfill}{rgb}{0.802153,0.920185,0.616378}%
\pgfsetfillcolor{currentfill}%
\pgfsetlinewidth{0.000000pt}%
\definecolor{currentstroke}{rgb}{0.000000,0.000000,0.000000}%
\pgfsetstrokecolor{currentstroke}%
\pgfsetstrokeopacity{0.000000}%
\pgfsetdash{}{0pt}%
\pgfpathmoveto{\pgfqpoint{2.194230in}{0.585103in}}%
\pgfpathlineto{\pgfqpoint{2.194346in}{0.597485in}}%
\pgfpathlineto{\pgfqpoint{2.186652in}{0.602188in}}%
\pgfpathlineto{\pgfqpoint{2.180349in}{0.610192in}}%
\pgfpathlineto{\pgfqpoint{2.181207in}{0.618602in}}%
\pgfpathlineto{\pgfqpoint{2.249834in}{0.623930in}}%
\pgfpathlineto{\pgfqpoint{2.327687in}{0.632577in}}%
\pgfpathlineto{\pgfqpoint{2.337617in}{0.614480in}}%
\pgfpathlineto{\pgfqpoint{2.370635in}{0.616804in}}%
\pgfpathlineto{\pgfqpoint{2.436103in}{0.620376in}}%
\pgfpathlineto{\pgfqpoint{2.488024in}{0.623836in}}%
\pgfpathlineto{\pgfqpoint{2.493123in}{0.610802in}}%
\pgfpathlineto{\pgfqpoint{2.499427in}{0.611870in}}%
\pgfpathlineto{\pgfqpoint{2.500128in}{0.625826in}}%
\pgfpathlineto{\pgfqpoint{2.496902in}{0.637830in}}%
\pgfpathlineto{\pgfqpoint{2.501225in}{0.642912in}}%
\pgfpathlineto{\pgfqpoint{2.511217in}{0.640178in}}%
\pgfpathlineto{\pgfqpoint{2.525559in}{0.639617in}}%
\pgfpathlineto{\pgfqpoint{2.527764in}{0.628892in}}%
\pgfpathlineto{\pgfqpoint{2.532176in}{0.622764in}}%
\pgfpathlineto{\pgfqpoint{2.535621in}{0.609356in}}%
\pgfpathlineto{\pgfqpoint{2.546296in}{0.588607in}}%
\pgfpathlineto{\pgfqpoint{2.546349in}{0.582995in}}%
\pgfpathlineto{\pgfqpoint{2.562126in}{0.558631in}}%
\pgfpathlineto{\pgfqpoint{2.563652in}{0.553598in}}%
\pgfpathlineto{\pgfqpoint{2.587320in}{0.519188in}}%
\pgfpathlineto{\pgfqpoint{2.583570in}{0.518632in}}%
\pgfpathlineto{\pgfqpoint{2.593548in}{0.494158in}}%
\pgfpathlineto{\pgfqpoint{2.620959in}{0.451609in}}%
\pgfpathlineto{\pgfqpoint{2.636873in}{0.420325in}}%
\pgfpathlineto{\pgfqpoint{2.642192in}{0.414848in}}%
\pgfpathlineto{\pgfqpoint{2.651306in}{0.395211in}}%
\pgfpathlineto{\pgfqpoint{2.654468in}{0.363769in}}%
\pgfpathlineto{\pgfqpoint{2.655727in}{0.340258in}}%
\pgfpathlineto{\pgfqpoint{2.654207in}{0.325205in}}%
\pgfpathlineto{\pgfqpoint{2.649320in}{0.314444in}}%
\pgfpathlineto{\pgfqpoint{2.651590in}{0.300332in}}%
\pgfpathlineto{\pgfqpoint{2.646329in}{0.289163in}}%
\pgfpathlineto{\pgfqpoint{2.630707in}{0.280003in}}%
\pgfpathlineto{\pgfqpoint{2.620547in}{0.280672in}}%
\pgfpathlineto{\pgfqpoint{2.613949in}{0.275884in}}%
\pgfpathlineto{\pgfqpoint{2.612059in}{0.288668in}}%
\pgfpathlineto{\pgfqpoint{2.601137in}{0.292042in}}%
\pgfpathlineto{\pgfqpoint{2.591346in}{0.309880in}}%
\pgfpathlineto{\pgfqpoint{2.590241in}{0.318003in}}%
\pgfpathlineto{\pgfqpoint{2.572719in}{0.322936in}}%
\pgfpathlineto{\pgfqpoint{2.561428in}{0.321868in}}%
\pgfpathlineto{\pgfqpoint{2.555040in}{0.333654in}}%
\pgfpathlineto{\pgfqpoint{2.547725in}{0.354839in}}%
\pgfpathlineto{\pgfqpoint{2.537575in}{0.359133in}}%
\pgfpathlineto{\pgfqpoint{2.532072in}{0.371273in}}%
\pgfpathlineto{\pgfqpoint{2.518614in}{0.378428in}}%
\pgfpathlineto{\pgfqpoint{2.510835in}{0.387377in}}%
\pgfpathlineto{\pgfqpoint{2.497416in}{0.411708in}}%
\pgfpathlineto{\pgfqpoint{2.495838in}{0.428833in}}%
\pgfpathlineto{\pgfqpoint{2.503150in}{0.439789in}}%
\pgfpathlineto{\pgfqpoint{2.502419in}{0.447403in}}%
\pgfpathlineto{\pgfqpoint{2.493969in}{0.448217in}}%
\pgfpathlineto{\pgfqpoint{2.486909in}{0.453513in}}%
\pgfpathlineto{\pgfqpoint{2.483042in}{0.447046in}}%
\pgfpathlineto{\pgfqpoint{2.489818in}{0.441777in}}%
\pgfpathlineto{\pgfqpoint{2.484887in}{0.432302in}}%
\pgfpathlineto{\pgfqpoint{2.476908in}{0.440166in}}%
\pgfpathlineto{\pgfqpoint{2.477834in}{0.461974in}}%
\pgfpathlineto{\pgfqpoint{2.481682in}{0.479667in}}%
\pgfpathlineto{\pgfqpoint{2.479704in}{0.510026in}}%
\pgfpathlineto{\pgfqpoint{2.471751in}{0.517257in}}%
\pgfpathlineto{\pgfqpoint{2.467752in}{0.526536in}}%
\pgfpathlineto{\pgfqpoint{2.454052in}{0.526329in}}%
\pgfpathlineto{\pgfqpoint{2.440510in}{0.541597in}}%
\pgfpathlineto{\pgfqpoint{2.431430in}{0.546173in}}%
\pgfpathlineto{\pgfqpoint{2.428737in}{0.555842in}}%
\pgfpathlineto{\pgfqpoint{2.419856in}{0.559238in}}%
\pgfpathlineto{\pgfqpoint{2.412496in}{0.569913in}}%
\pgfpathlineto{\pgfqpoint{2.393052in}{0.578628in}}%
\pgfpathlineto{\pgfqpoint{2.377937in}{0.578850in}}%
\pgfpathlineto{\pgfqpoint{2.371356in}{0.575503in}}%
\pgfpathlineto{\pgfqpoint{2.372991in}{0.565056in}}%
\pgfpathlineto{\pgfqpoint{2.366130in}{0.565551in}}%
\pgfpathlineto{\pgfqpoint{2.345022in}{0.550943in}}%
\pgfpathlineto{\pgfqpoint{2.319668in}{0.545153in}}%
\pgfpathlineto{\pgfqpoint{2.319252in}{0.552317in}}%
\pgfpathlineto{\pgfqpoint{2.313633in}{0.559318in}}%
\pgfpathlineto{\pgfqpoint{2.298558in}{0.568984in}}%
\pgfpathlineto{\pgfqpoint{2.276918in}{0.578894in}}%
\pgfpathlineto{\pgfqpoint{2.253414in}{0.584141in}}%
\pgfpathlineto{\pgfqpoint{2.249002in}{0.591275in}}%
\pgfpathlineto{\pgfqpoint{2.240528in}{0.585325in}}%
\pgfpathlineto{\pgfqpoint{2.230323in}{0.584012in}}%
\pgfpathlineto{\pgfqpoint{2.194801in}{0.574653in}}%
\pgfpathlineto{\pgfqpoint{2.194230in}{0.585103in}}%
\pgfpathclose%
\pgfusepath{fill}%
\end{pgfscope}%
\begin{pgfscope}%
\pgfpathrectangle{\pgfqpoint{0.100000in}{0.100000in}}{\pgfqpoint{2.989028in}{1.913466in}}%
\pgfusepath{clip}%
\pgfsetbuttcap%
\pgfsetmiterjoin%
\definecolor{currentfill}{rgb}{0.802153,0.920185,0.616378}%
\pgfsetfillcolor{currentfill}%
\pgfsetlinewidth{0.000000pt}%
\definecolor{currentstroke}{rgb}{0.000000,0.000000,0.000000}%
\pgfsetstrokecolor{currentstroke}%
\pgfsetstrokeopacity{0.000000}%
\pgfsetdash{}{0pt}%
\pgfpathmoveto{\pgfqpoint{2.590769in}{0.519678in}}%
\pgfpathlineto{\pgfqpoint{2.599959in}{0.508987in}}%
\pgfpathlineto{\pgfqpoint{2.589344in}{0.508314in}}%
\pgfpathlineto{\pgfqpoint{2.590769in}{0.519678in}}%
\pgfpathclose%
\pgfusepath{fill}%
\end{pgfscope}%
\begin{pgfscope}%
\pgfpathrectangle{\pgfqpoint{0.100000in}{0.100000in}}{\pgfqpoint{2.989028in}{1.913466in}}%
\pgfusepath{clip}%
\pgfsetbuttcap%
\pgfsetmiterjoin%
\definecolor{currentfill}{rgb}{0.820300,0.927566,0.612687}%
\pgfsetfillcolor{currentfill}%
\pgfsetlinewidth{0.000000pt}%
\definecolor{currentstroke}{rgb}{0.000000,0.000000,0.000000}%
\pgfsetstrokecolor{currentstroke}%
\pgfsetstrokeopacity{0.000000}%
\pgfsetdash{}{0pt}%
\pgfpathmoveto{\pgfqpoint{2.049740in}{1.740361in}}%
\pgfpathlineto{\pgfqpoint{2.044753in}{1.730656in}}%
\pgfpathlineto{\pgfqpoint{2.032857in}{1.724992in}}%
\pgfpathlineto{\pgfqpoint{2.027705in}{1.717368in}}%
\pgfpathlineto{\pgfqpoint{2.021669in}{1.722859in}}%
\pgfpathlineto{\pgfqpoint{2.049740in}{1.740361in}}%
\pgfpathclose%
\pgfusepath{fill}%
\end{pgfscope}%
\begin{pgfscope}%
\pgfpathrectangle{\pgfqpoint{0.100000in}{0.100000in}}{\pgfqpoint{2.989028in}{1.913466in}}%
\pgfusepath{clip}%
\pgfsetbuttcap%
\pgfsetmiterjoin%
\definecolor{currentfill}{rgb}{0.820300,0.927566,0.612687}%
\pgfsetfillcolor{currentfill}%
\pgfsetlinewidth{0.000000pt}%
\definecolor{currentstroke}{rgb}{0.000000,0.000000,0.000000}%
\pgfsetstrokecolor{currentstroke}%
\pgfsetstrokeopacity{0.000000}%
\pgfsetdash{}{0pt}%
\pgfpathmoveto{\pgfqpoint{2.053784in}{1.681953in}}%
\pgfpathlineto{\pgfqpoint{2.065956in}{1.693308in}}%
\pgfpathlineto{\pgfqpoint{2.084730in}{1.696281in}}%
\pgfpathlineto{\pgfqpoint{2.079559in}{1.688384in}}%
\pgfpathlineto{\pgfqpoint{2.066688in}{1.676964in}}%
\pgfpathlineto{\pgfqpoint{2.059164in}{1.662306in}}%
\pgfpathlineto{\pgfqpoint{2.055977in}{1.670256in}}%
\pgfpathlineto{\pgfqpoint{2.050318in}{1.671398in}}%
\pgfpathlineto{\pgfqpoint{2.049750in}{1.678576in}}%
\pgfpathlineto{\pgfqpoint{2.053784in}{1.681953in}}%
\pgfpathclose%
\pgfusepath{fill}%
\end{pgfscope}%
\begin{pgfscope}%
\pgfpathrectangle{\pgfqpoint{0.100000in}{0.100000in}}{\pgfqpoint{2.989028in}{1.913466in}}%
\pgfusepath{clip}%
\pgfsetbuttcap%
\pgfsetmiterjoin%
\definecolor{currentfill}{rgb}{0.820300,0.927566,0.612687}%
\pgfsetfillcolor{currentfill}%
\pgfsetlinewidth{0.000000pt}%
\definecolor{currentstroke}{rgb}{0.000000,0.000000,0.000000}%
\pgfsetstrokecolor{currentstroke}%
\pgfsetstrokeopacity{0.000000}%
\pgfsetdash{}{0pt}%
\pgfpathmoveto{\pgfqpoint{2.102264in}{1.543360in}}%
\pgfpathlineto{\pgfqpoint{2.099022in}{1.546957in}}%
\pgfpathlineto{\pgfqpoint{2.102430in}{1.557084in}}%
\pgfpathlineto{\pgfqpoint{2.092314in}{1.557730in}}%
\pgfpathlineto{\pgfqpoint{2.094996in}{1.566462in}}%
\pgfpathlineto{\pgfqpoint{2.093328in}{1.580368in}}%
\pgfpathlineto{\pgfqpoint{2.084251in}{1.585189in}}%
\pgfpathlineto{\pgfqpoint{2.074752in}{1.594947in}}%
\pgfpathlineto{\pgfqpoint{2.060116in}{1.597803in}}%
\pgfpathlineto{\pgfqpoint{2.045871in}{1.597722in}}%
\pgfpathlineto{\pgfqpoint{2.031860in}{1.604646in}}%
\pgfpathlineto{\pgfqpoint{1.984989in}{1.614740in}}%
\pgfpathlineto{\pgfqpoint{1.979867in}{1.625436in}}%
\pgfpathlineto{\pgfqpoint{1.970731in}{1.629085in}}%
\pgfpathlineto{\pgfqpoint{1.987989in}{1.637274in}}%
\pgfpathlineto{\pgfqpoint{1.997738in}{1.647490in}}%
\pgfpathlineto{\pgfqpoint{2.015897in}{1.650260in}}%
\pgfpathlineto{\pgfqpoint{2.027071in}{1.660666in}}%
\pgfpathlineto{\pgfqpoint{2.032934in}{1.661083in}}%
\pgfpathlineto{\pgfqpoint{2.037408in}{1.668504in}}%
\pgfpathlineto{\pgfqpoint{2.048315in}{1.677295in}}%
\pgfpathlineto{\pgfqpoint{2.049265in}{1.671081in}}%
\pgfpathlineto{\pgfqpoint{2.054179in}{1.669798in}}%
\pgfpathlineto{\pgfqpoint{2.057877in}{1.662388in}}%
\pgfpathlineto{\pgfqpoint{2.058543in}{1.649753in}}%
\pgfpathlineto{\pgfqpoint{2.069656in}{1.657304in}}%
\pgfpathlineto{\pgfqpoint{2.082606in}{1.658880in}}%
\pgfpathlineto{\pgfqpoint{2.093660in}{1.654925in}}%
\pgfpathlineto{\pgfqpoint{2.108651in}{1.634360in}}%
\pgfpathlineto{\pgfqpoint{2.125023in}{1.637647in}}%
\pgfpathlineto{\pgfqpoint{2.131676in}{1.632143in}}%
\pgfpathlineto{\pgfqpoint{2.142376in}{1.631651in}}%
\pgfpathlineto{\pgfqpoint{2.149485in}{1.641522in}}%
\pgfpathlineto{\pgfqpoint{2.162974in}{1.650248in}}%
\pgfpathlineto{\pgfqpoint{2.175939in}{1.652980in}}%
\pgfpathlineto{\pgfqpoint{2.192026in}{1.653265in}}%
\pgfpathlineto{\pgfqpoint{2.203782in}{1.660005in}}%
\pgfpathlineto{\pgfqpoint{2.213376in}{1.656901in}}%
\pgfpathlineto{\pgfqpoint{2.215419in}{1.642645in}}%
\pgfpathlineto{\pgfqpoint{2.236002in}{1.640413in}}%
\pgfpathlineto{\pgfqpoint{2.247182in}{1.647086in}}%
\pgfpathlineto{\pgfqpoint{2.254932in}{1.632053in}}%
\pgfpathlineto{\pgfqpoint{2.269719in}{1.614670in}}%
\pgfpathlineto{\pgfqpoint{2.249022in}{1.614771in}}%
\pgfpathlineto{\pgfqpoint{2.242482in}{1.612622in}}%
\pgfpathlineto{\pgfqpoint{2.233519in}{1.615414in}}%
\pgfpathlineto{\pgfqpoint{2.232916in}{1.603380in}}%
\pgfpathlineto{\pgfqpoint{2.216640in}{1.612785in}}%
\pgfpathlineto{\pgfqpoint{2.195708in}{1.615659in}}%
\pgfpathlineto{\pgfqpoint{2.189946in}{1.606500in}}%
\pgfpathlineto{\pgfqpoint{2.162559in}{1.602046in}}%
\pgfpathlineto{\pgfqpoint{2.159409in}{1.594288in}}%
\pgfpathlineto{\pgfqpoint{2.140351in}{1.591960in}}%
\pgfpathlineto{\pgfqpoint{2.134590in}{1.584073in}}%
\pgfpathlineto{\pgfqpoint{2.124509in}{1.581958in}}%
\pgfpathlineto{\pgfqpoint{2.116458in}{1.563256in}}%
\pgfpathlineto{\pgfqpoint{2.106264in}{1.545142in}}%
\pgfpathlineto{\pgfqpoint{2.102264in}{1.543360in}}%
\pgfpathclose%
\pgfusepath{fill}%
\end{pgfscope}%
\begin{pgfscope}%
\pgfpathrectangle{\pgfqpoint{0.100000in}{0.100000in}}{\pgfqpoint{2.989028in}{1.913466in}}%
\pgfusepath{clip}%
\pgfsetbuttcap%
\pgfsetmiterjoin%
\definecolor{currentfill}{rgb}{0.820300,0.927566,0.612687}%
\pgfsetfillcolor{currentfill}%
\pgfsetlinewidth{0.000000pt}%
\definecolor{currentstroke}{rgb}{0.000000,0.000000,0.000000}%
\pgfsetstrokecolor{currentstroke}%
\pgfsetstrokeopacity{0.000000}%
\pgfsetdash{}{0pt}%
\pgfpathmoveto{\pgfqpoint{2.160936in}{1.325898in}}%
\pgfpathlineto{\pgfqpoint{2.170655in}{1.336128in}}%
\pgfpathlineto{\pgfqpoint{2.175083in}{1.350960in}}%
\pgfpathlineto{\pgfqpoint{2.180349in}{1.359557in}}%
\pgfpathlineto{\pgfqpoint{2.183578in}{1.371256in}}%
\pgfpathlineto{\pgfqpoint{2.184581in}{1.394584in}}%
\pgfpathlineto{\pgfqpoint{2.179720in}{1.416940in}}%
\pgfpathlineto{\pgfqpoint{2.163654in}{1.451174in}}%
\pgfpathlineto{\pgfqpoint{2.168005in}{1.461938in}}%
\pgfpathlineto{\pgfqpoint{2.162337in}{1.476816in}}%
\pgfpathlineto{\pgfqpoint{2.172059in}{1.497392in}}%
\pgfpathlineto{\pgfqpoint{2.170475in}{1.520335in}}%
\pgfpathlineto{\pgfqpoint{2.177247in}{1.523223in}}%
\pgfpathlineto{\pgfqpoint{2.178098in}{1.534139in}}%
\pgfpathlineto{\pgfqpoint{2.190119in}{1.541146in}}%
\pgfpathlineto{\pgfqpoint{2.196918in}{1.540015in}}%
\pgfpathlineto{\pgfqpoint{2.198829in}{1.528307in}}%
\pgfpathlineto{\pgfqpoint{2.204135in}{1.527848in}}%
\pgfpathlineto{\pgfqpoint{2.208977in}{1.544735in}}%
\pgfpathlineto{\pgfqpoint{2.207331in}{1.557869in}}%
\pgfpathlineto{\pgfqpoint{2.210479in}{1.565411in}}%
\pgfpathlineto{\pgfqpoint{2.227498in}{1.573202in}}%
\pgfpathlineto{\pgfqpoint{2.219743in}{1.575975in}}%
\pgfpathlineto{\pgfqpoint{2.217216in}{1.582668in}}%
\pgfpathlineto{\pgfqpoint{2.222803in}{1.594421in}}%
\pgfpathlineto{\pgfqpoint{2.233821in}{1.598481in}}%
\pgfpathlineto{\pgfqpoint{2.246615in}{1.591546in}}%
\pgfpathlineto{\pgfqpoint{2.258677in}{1.591460in}}%
\pgfpathlineto{\pgfqpoint{2.264279in}{1.583359in}}%
\pgfpathlineto{\pgfqpoint{2.272727in}{1.583922in}}%
\pgfpathlineto{\pgfqpoint{2.298804in}{1.572661in}}%
\pgfpathlineto{\pgfqpoint{2.304014in}{1.561760in}}%
\pgfpathlineto{\pgfqpoint{2.298324in}{1.557985in}}%
\pgfpathlineto{\pgfqpoint{2.300077in}{1.549815in}}%
\pgfpathlineto{\pgfqpoint{2.305699in}{1.546182in}}%
\pgfpathlineto{\pgfqpoint{2.308831in}{1.536139in}}%
\pgfpathlineto{\pgfqpoint{2.308393in}{1.511672in}}%
\pgfpathlineto{\pgfqpoint{2.300971in}{1.505827in}}%
\pgfpathlineto{\pgfqpoint{2.299305in}{1.492975in}}%
\pgfpathlineto{\pgfqpoint{2.285538in}{1.481098in}}%
\pgfpathlineto{\pgfqpoint{2.286355in}{1.466722in}}%
\pgfpathlineto{\pgfqpoint{2.298421in}{1.461685in}}%
\pgfpathlineto{\pgfqpoint{2.312034in}{1.479623in}}%
\pgfpathlineto{\pgfqpoint{2.313159in}{1.486129in}}%
\pgfpathlineto{\pgfqpoint{2.330136in}{1.496942in}}%
\pgfpathlineto{\pgfqpoint{2.340932in}{1.491933in}}%
\pgfpathlineto{\pgfqpoint{2.347707in}{1.480613in}}%
\pgfpathlineto{\pgfqpoint{2.358636in}{1.441310in}}%
\pgfpathlineto{\pgfqpoint{2.364428in}{1.428876in}}%
\pgfpathlineto{\pgfqpoint{2.362615in}{1.423735in}}%
\pgfpathlineto{\pgfqpoint{2.362791in}{1.406232in}}%
\pgfpathlineto{\pgfqpoint{2.352255in}{1.407912in}}%
\pgfpathlineto{\pgfqpoint{2.346305in}{1.394832in}}%
\pgfpathlineto{\pgfqpoint{2.345480in}{1.385940in}}%
\pgfpathlineto{\pgfqpoint{2.337522in}{1.380232in}}%
\pgfpathlineto{\pgfqpoint{2.335753in}{1.362891in}}%
\pgfpathlineto{\pgfqpoint{2.324180in}{1.340984in}}%
\pgfpathlineto{\pgfqpoint{2.260872in}{1.331556in}}%
\pgfpathlineto{\pgfqpoint{2.260498in}{1.335699in}}%
\pgfpathlineto{\pgfqpoint{2.218150in}{1.331236in}}%
\pgfpathlineto{\pgfqpoint{2.160936in}{1.325898in}}%
\pgfpathclose%
\pgfusepath{fill}%
\end{pgfscope}%
\begin{pgfscope}%
\pgfpathrectangle{\pgfqpoint{0.100000in}{0.100000in}}{\pgfqpoint{2.989028in}{1.913466in}}%
\pgfusepath{clip}%
\pgfsetbuttcap%
\pgfsetmiterjoin%
\definecolor{currentfill}{rgb}{0.400000,0.760784,0.647059}%
\pgfsetfillcolor{currentfill}%
\pgfsetlinewidth{0.000000pt}%
\definecolor{currentstroke}{rgb}{0.000000,0.000000,0.000000}%
\pgfsetstrokecolor{currentstroke}%
\pgfsetstrokeopacity{0.000000}%
\pgfsetdash{}{0pt}%
\pgfpathmoveto{\pgfqpoint{0.562031in}{0.446426in}}%
\pgfpathlineto{\pgfqpoint{0.560767in}{0.445314in}}%
\pgfpathlineto{\pgfqpoint{0.554967in}{0.446797in}}%
\pgfpathlineto{\pgfqpoint{0.555400in}{0.449716in}}%
\pgfpathlineto{\pgfqpoint{0.558683in}{0.451053in}}%
\pgfpathlineto{\pgfqpoint{0.563364in}{0.457782in}}%
\pgfpathlineto{\pgfqpoint{0.564142in}{0.462330in}}%
\pgfpathlineto{\pgfqpoint{0.566773in}{0.463522in}}%
\pgfpathlineto{\pgfqpoint{0.571054in}{0.463526in}}%
\pgfpathlineto{\pgfqpoint{0.575530in}{0.478285in}}%
\pgfpathlineto{\pgfqpoint{0.577283in}{0.479499in}}%
\pgfpathlineto{\pgfqpoint{0.575711in}{0.482384in}}%
\pgfpathlineto{\pgfqpoint{0.572030in}{0.479466in}}%
\pgfpathlineto{\pgfqpoint{0.561215in}{0.483400in}}%
\pgfpathlineto{\pgfqpoint{0.554760in}{0.490029in}}%
\pgfpathlineto{\pgfqpoint{0.557575in}{0.493053in}}%
\pgfpathlineto{\pgfqpoint{0.556295in}{0.497319in}}%
\pgfpathlineto{\pgfqpoint{0.557083in}{0.502359in}}%
\pgfpathlineto{\pgfqpoint{0.555695in}{0.508250in}}%
\pgfpathlineto{\pgfqpoint{0.555473in}{0.514148in}}%
\pgfpathlineto{\pgfqpoint{0.561686in}{0.513161in}}%
\pgfpathlineto{\pgfqpoint{0.562916in}{0.514899in}}%
\pgfpathlineto{\pgfqpoint{0.566006in}{0.514731in}}%
\pgfpathlineto{\pgfqpoint{0.569565in}{0.508441in}}%
\pgfpathlineto{\pgfqpoint{0.572996in}{0.503835in}}%
\pgfpathlineto{\pgfqpoint{0.570248in}{0.500837in}}%
\pgfpathlineto{\pgfqpoint{0.575000in}{0.499687in}}%
\pgfpathlineto{\pgfqpoint{0.576083in}{0.501158in}}%
\pgfpathlineto{\pgfqpoint{0.573542in}{0.507123in}}%
\pgfpathlineto{\pgfqpoint{0.568190in}{0.512068in}}%
\pgfpathlineto{\pgfqpoint{0.568845in}{0.514140in}}%
\pgfpathlineto{\pgfqpoint{0.564574in}{0.519376in}}%
\pgfpathlineto{\pgfqpoint{0.568935in}{0.520439in}}%
\pgfpathlineto{\pgfqpoint{0.565157in}{0.531650in}}%
\pgfpathlineto{\pgfqpoint{0.568229in}{0.533204in}}%
\pgfpathlineto{\pgfqpoint{0.568978in}{0.535892in}}%
\pgfpathlineto{\pgfqpoint{0.566980in}{0.539085in}}%
\pgfpathlineto{\pgfqpoint{0.571104in}{0.539884in}}%
\pgfpathlineto{\pgfqpoint{0.571845in}{0.542299in}}%
\pgfpathlineto{\pgfqpoint{0.576765in}{0.538674in}}%
\pgfpathlineto{\pgfqpoint{0.577857in}{0.542605in}}%
\pgfpathlineto{\pgfqpoint{0.580443in}{0.544334in}}%
\pgfpathlineto{\pgfqpoint{0.593023in}{0.546739in}}%
\pgfpathlineto{\pgfqpoint{0.596885in}{0.545889in}}%
\pgfpathlineto{\pgfqpoint{0.600120in}{0.550663in}}%
\pgfpathlineto{\pgfqpoint{0.607797in}{0.553794in}}%
\pgfpathlineto{\pgfqpoint{0.613710in}{0.552606in}}%
\pgfpathlineto{\pgfqpoint{0.616407in}{0.548449in}}%
\pgfpathlineto{\pgfqpoint{0.616266in}{0.541844in}}%
\pgfpathlineto{\pgfqpoint{0.619292in}{0.540347in}}%
\pgfpathlineto{\pgfqpoint{0.625464in}{0.540556in}}%
\pgfpathlineto{\pgfqpoint{0.633370in}{0.542274in}}%
\pgfpathlineto{\pgfqpoint{0.633272in}{0.540162in}}%
\pgfpathlineto{\pgfqpoint{0.643266in}{0.534139in}}%
\pgfpathlineto{\pgfqpoint{0.650255in}{0.535754in}}%
\pgfpathlineto{\pgfqpoint{0.652313in}{0.537844in}}%
\pgfpathlineto{\pgfqpoint{0.654516in}{0.543197in}}%
\pgfpathlineto{\pgfqpoint{0.657248in}{0.546625in}}%
\pgfpathlineto{\pgfqpoint{0.657492in}{0.551762in}}%
\pgfpathlineto{\pgfqpoint{0.654990in}{0.553782in}}%
\pgfpathlineto{\pgfqpoint{0.658695in}{0.555608in}}%
\pgfpathlineto{\pgfqpoint{0.664920in}{0.552865in}}%
\pgfpathlineto{\pgfqpoint{0.666948in}{0.554559in}}%
\pgfpathlineto{\pgfqpoint{0.667305in}{0.561826in}}%
\pgfpathlineto{\pgfqpoint{0.663281in}{0.560307in}}%
\pgfpathlineto{\pgfqpoint{0.660490in}{0.562825in}}%
\pgfpathlineto{\pgfqpoint{0.655960in}{0.564395in}}%
\pgfpathlineto{\pgfqpoint{0.648770in}{0.564133in}}%
\pgfpathlineto{\pgfqpoint{0.641880in}{0.566521in}}%
\pgfpathlineto{\pgfqpoint{0.641535in}{0.572469in}}%
\pgfpathlineto{\pgfqpoint{0.635373in}{0.578787in}}%
\pgfpathlineto{\pgfqpoint{0.627559in}{0.581439in}}%
\pgfpathlineto{\pgfqpoint{0.620739in}{0.593671in}}%
\pgfpathlineto{\pgfqpoint{0.621298in}{0.598197in}}%
\pgfpathlineto{\pgfqpoint{0.624971in}{0.600261in}}%
\pgfpathlineto{\pgfqpoint{0.624430in}{0.604545in}}%
\pgfpathlineto{\pgfqpoint{0.625282in}{0.608808in}}%
\pgfpathlineto{\pgfqpoint{0.628188in}{0.605890in}}%
\pgfpathlineto{\pgfqpoint{0.632025in}{0.607107in}}%
\pgfpathlineto{\pgfqpoint{0.631833in}{0.610810in}}%
\pgfpathlineto{\pgfqpoint{0.626514in}{0.618243in}}%
\pgfpathlineto{\pgfqpoint{0.624956in}{0.625911in}}%
\pgfpathlineto{\pgfqpoint{0.629409in}{0.626496in}}%
\pgfpathlineto{\pgfqpoint{0.636074in}{0.626025in}}%
\pgfpathlineto{\pgfqpoint{0.638172in}{0.623460in}}%
\pgfpathlineto{\pgfqpoint{0.644147in}{0.624378in}}%
\pgfpathlineto{\pgfqpoint{0.649700in}{0.623173in}}%
\pgfpathlineto{\pgfqpoint{0.653454in}{0.621055in}}%
\pgfpathlineto{\pgfqpoint{0.656052in}{0.622023in}}%
\pgfpathlineto{\pgfqpoint{0.662951in}{0.619906in}}%
\pgfpathlineto{\pgfqpoint{0.663014in}{0.618032in}}%
\pgfpathlineto{\pgfqpoint{0.668268in}{0.618665in}}%
\pgfpathlineto{\pgfqpoint{0.675313in}{0.614092in}}%
\pgfpathlineto{\pgfqpoint{0.675039in}{0.611057in}}%
\pgfpathlineto{\pgfqpoint{0.671687in}{0.609656in}}%
\pgfpathlineto{\pgfqpoint{0.668124in}{0.606323in}}%
\pgfpathlineto{\pgfqpoint{0.667694in}{0.601746in}}%
\pgfpathlineto{\pgfqpoint{0.675369in}{0.595909in}}%
\pgfpathlineto{\pgfqpoint{0.674141in}{0.593112in}}%
\pgfpathlineto{\pgfqpoint{0.679602in}{0.590964in}}%
\pgfpathlineto{\pgfqpoint{0.680943in}{0.587455in}}%
\pgfpathlineto{\pgfqpoint{0.687873in}{0.590243in}}%
\pgfpathlineto{\pgfqpoint{0.690925in}{0.586589in}}%
\pgfpathlineto{\pgfqpoint{0.692334in}{0.588801in}}%
\pgfpathlineto{\pgfqpoint{0.688186in}{0.595999in}}%
\pgfpathlineto{\pgfqpoint{0.689660in}{0.598155in}}%
\pgfpathlineto{\pgfqpoint{0.690313in}{0.603841in}}%
\pgfpathlineto{\pgfqpoint{0.688661in}{0.606342in}}%
\pgfpathlineto{\pgfqpoint{0.689901in}{0.609704in}}%
\pgfpathlineto{\pgfqpoint{0.693961in}{0.609388in}}%
\pgfpathlineto{\pgfqpoint{0.693121in}{0.603668in}}%
\pgfpathlineto{\pgfqpoint{0.690529in}{0.601760in}}%
\pgfpathlineto{\pgfqpoint{0.690712in}{0.594065in}}%
\pgfpathlineto{\pgfqpoint{0.695364in}{0.593146in}}%
\pgfpathlineto{\pgfqpoint{0.695421in}{0.587424in}}%
\pgfpathlineto{\pgfqpoint{0.700501in}{0.583713in}}%
\pgfpathlineto{\pgfqpoint{0.702836in}{0.586053in}}%
\pgfpathlineto{\pgfqpoint{0.700284in}{0.593176in}}%
\pgfpathlineto{\pgfqpoint{0.697644in}{0.594979in}}%
\pgfpathlineto{\pgfqpoint{0.695047in}{0.593793in}}%
\pgfpathlineto{\pgfqpoint{0.692763in}{0.595306in}}%
\pgfpathlineto{\pgfqpoint{0.693161in}{0.601889in}}%
\pgfpathlineto{\pgfqpoint{0.696974in}{0.604581in}}%
\pgfpathlineto{\pgfqpoint{0.700726in}{0.604333in}}%
\pgfpathlineto{\pgfqpoint{0.699172in}{0.608058in}}%
\pgfpathlineto{\pgfqpoint{0.693547in}{0.610911in}}%
\pgfpathlineto{\pgfqpoint{0.686098in}{0.622277in}}%
\pgfpathlineto{\pgfqpoint{0.689816in}{0.627728in}}%
\pgfpathlineto{\pgfqpoint{0.691885in}{0.632863in}}%
\pgfpathlineto{\pgfqpoint{0.692048in}{0.643174in}}%
\pgfpathlineto{\pgfqpoint{0.691251in}{0.651872in}}%
\pgfpathlineto{\pgfqpoint{0.688978in}{0.657584in}}%
\pgfpathlineto{\pgfqpoint{0.689431in}{0.663294in}}%
\pgfpathlineto{\pgfqpoint{0.695244in}{0.668247in}}%
\pgfpathlineto{\pgfqpoint{0.701026in}{0.674133in}}%
\pgfpathlineto{\pgfqpoint{0.715519in}{0.662108in}}%
\pgfpathlineto{\pgfqpoint{0.724995in}{0.661221in}}%
\pgfpathlineto{\pgfqpoint{0.732817in}{0.664216in}}%
\pgfpathlineto{\pgfqpoint{0.739683in}{0.668159in}}%
\pgfpathlineto{\pgfqpoint{0.741141in}{0.670038in}}%
\pgfpathlineto{\pgfqpoint{0.758090in}{0.674322in}}%
\pgfpathlineto{\pgfqpoint{0.759562in}{0.671184in}}%
\pgfpathlineto{\pgfqpoint{0.765321in}{0.668615in}}%
\pgfpathlineto{\pgfqpoint{0.776313in}{0.669403in}}%
\pgfpathlineto{\pgfqpoint{0.778320in}{0.668702in}}%
\pgfpathlineto{\pgfqpoint{0.783260in}{0.670576in}}%
\pgfpathlineto{\pgfqpoint{0.785246in}{0.667335in}}%
\pgfpathlineto{\pgfqpoint{0.789990in}{0.664374in}}%
\pgfpathlineto{\pgfqpoint{0.792363in}{0.664709in}}%
\pgfpathlineto{\pgfqpoint{0.796956in}{0.661151in}}%
\pgfpathlineto{\pgfqpoint{0.804717in}{0.661805in}}%
\pgfpathlineto{\pgfqpoint{0.812469in}{0.664685in}}%
\pgfpathlineto{\pgfqpoint{0.818800in}{0.655475in}}%
\pgfpathlineto{\pgfqpoint{0.818309in}{0.653408in}}%
\pgfpathlineto{\pgfqpoint{0.811323in}{0.651032in}}%
\pgfpathlineto{\pgfqpoint{0.813190in}{0.647440in}}%
\pgfpathlineto{\pgfqpoint{0.819621in}{0.651221in}}%
\pgfpathlineto{\pgfqpoint{0.822246in}{0.650305in}}%
\pgfpathlineto{\pgfqpoint{0.823363in}{0.645783in}}%
\pgfpathlineto{\pgfqpoint{0.819524in}{0.643870in}}%
\pgfpathlineto{\pgfqpoint{0.822266in}{0.638154in}}%
\pgfpathlineto{\pgfqpoint{0.826248in}{0.639120in}}%
\pgfpathlineto{\pgfqpoint{0.832026in}{0.636170in}}%
\pgfpathlineto{\pgfqpoint{0.837551in}{0.627982in}}%
\pgfpathlineto{\pgfqpoint{0.832689in}{0.625708in}}%
\pgfpathlineto{\pgfqpoint{0.836985in}{0.619675in}}%
\pgfpathlineto{\pgfqpoint{0.833366in}{0.618264in}}%
\pgfpathlineto{\pgfqpoint{0.837868in}{0.612534in}}%
\pgfpathlineto{\pgfqpoint{0.843358in}{0.613145in}}%
\pgfpathlineto{\pgfqpoint{0.846249in}{0.610567in}}%
\pgfpathlineto{\pgfqpoint{0.853562in}{0.606765in}}%
\pgfpathlineto{\pgfqpoint{0.862995in}{0.593271in}}%
\pgfpathlineto{\pgfqpoint{0.862782in}{0.590043in}}%
\pgfpathlineto{\pgfqpoint{0.876928in}{0.579706in}}%
\pgfpathlineto{\pgfqpoint{0.877233in}{0.576049in}}%
\pgfpathlineto{\pgfqpoint{0.881003in}{0.570578in}}%
\pgfpathlineto{\pgfqpoint{0.892585in}{0.567843in}}%
\pgfpathlineto{\pgfqpoint{0.896881in}{0.565899in}}%
\pgfpathlineto{\pgfqpoint{0.900922in}{0.559868in}}%
\pgfpathlineto{\pgfqpoint{0.901442in}{0.554585in}}%
\pgfpathlineto{\pgfqpoint{0.904915in}{0.550279in}}%
\pgfpathlineto{\pgfqpoint{0.905958in}{0.545423in}}%
\pgfpathlineto{\pgfqpoint{0.909213in}{0.543649in}}%
\pgfpathlineto{\pgfqpoint{0.893089in}{0.514548in}}%
\pgfpathlineto{\pgfqpoint{0.859622in}{0.454133in}}%
\pgfpathlineto{\pgfqpoint{0.810518in}{0.365475in}}%
\pgfpathlineto{\pgfqpoint{0.791294in}{0.330756in}}%
\pgfpathlineto{\pgfqpoint{0.795388in}{0.326093in}}%
\pgfpathlineto{\pgfqpoint{0.797192in}{0.327581in}}%
\pgfpathlineto{\pgfqpoint{0.800850in}{0.322152in}}%
\pgfpathlineto{\pgfqpoint{0.805927in}{0.323727in}}%
\pgfpathlineto{\pgfqpoint{0.812667in}{0.320475in}}%
\pgfpathlineto{\pgfqpoint{0.808301in}{0.315393in}}%
\pgfpathlineto{\pgfqpoint{0.811672in}{0.308535in}}%
\pgfpathlineto{\pgfqpoint{0.810976in}{0.305196in}}%
\pgfpathlineto{\pgfqpoint{0.816353in}{0.287587in}}%
\pgfpathlineto{\pgfqpoint{0.814077in}{0.279580in}}%
\pgfpathlineto{\pgfqpoint{0.824350in}{0.281562in}}%
\pgfpathlineto{\pgfqpoint{0.826872in}{0.280263in}}%
\pgfpathlineto{\pgfqpoint{0.829610in}{0.282376in}}%
\pgfpathlineto{\pgfqpoint{0.831558in}{0.286368in}}%
\pgfpathlineto{\pgfqpoint{0.834347in}{0.288657in}}%
\pgfpathlineto{\pgfqpoint{0.846252in}{0.288367in}}%
\pgfpathlineto{\pgfqpoint{0.848943in}{0.280683in}}%
\pgfpathlineto{\pgfqpoint{0.846634in}{0.273938in}}%
\pgfpathlineto{\pgfqpoint{0.849257in}{0.271712in}}%
\pgfpathlineto{\pgfqpoint{0.850636in}{0.267905in}}%
\pgfpathlineto{\pgfqpoint{0.850336in}{0.260746in}}%
\pgfpathlineto{\pgfqpoint{0.853836in}{0.255613in}}%
\pgfpathlineto{\pgfqpoint{0.855755in}{0.247568in}}%
\pgfpathlineto{\pgfqpoint{0.854770in}{0.243223in}}%
\pgfpathlineto{\pgfqpoint{0.855958in}{0.238819in}}%
\pgfpathlineto{\pgfqpoint{0.857671in}{0.213318in}}%
\pgfpathlineto{\pgfqpoint{0.855216in}{0.211309in}}%
\pgfpathlineto{\pgfqpoint{0.858455in}{0.208533in}}%
\pgfpathlineto{\pgfqpoint{0.855910in}{0.205157in}}%
\pgfpathlineto{\pgfqpoint{0.858206in}{0.202405in}}%
\pgfpathlineto{\pgfqpoint{0.856727in}{0.197707in}}%
\pgfpathlineto{\pgfqpoint{0.860034in}{0.196433in}}%
\pgfpathlineto{\pgfqpoint{0.864287in}{0.189287in}}%
\pgfpathlineto{\pgfqpoint{0.867447in}{0.187134in}}%
\pgfpathlineto{\pgfqpoint{0.868274in}{0.183987in}}%
\pgfpathlineto{\pgfqpoint{0.870105in}{0.182623in}}%
\pgfpathlineto{\pgfqpoint{0.869723in}{0.180010in}}%
\pgfpathlineto{\pgfqpoint{0.873670in}{0.178116in}}%
\pgfpathlineto{\pgfqpoint{0.872696in}{0.173088in}}%
\pgfpathlineto{\pgfqpoint{0.869298in}{0.170454in}}%
\pgfpathlineto{\pgfqpoint{0.867755in}{0.165560in}}%
\pgfpathlineto{\pgfqpoint{0.867345in}{0.159006in}}%
\pgfpathlineto{\pgfqpoint{0.865458in}{0.158379in}}%
\pgfpathlineto{\pgfqpoint{0.859289in}{0.152634in}}%
\pgfpathlineto{\pgfqpoint{0.853806in}{0.151118in}}%
\pgfpathlineto{\pgfqpoint{0.851190in}{0.153435in}}%
\pgfpathlineto{\pgfqpoint{0.853718in}{0.159583in}}%
\pgfpathlineto{\pgfqpoint{0.852946in}{0.162601in}}%
\pgfpathlineto{\pgfqpoint{0.856568in}{0.164131in}}%
\pgfpathlineto{\pgfqpoint{0.859956in}{0.173027in}}%
\pgfpathlineto{\pgfqpoint{0.858525in}{0.180625in}}%
\pgfpathlineto{\pgfqpoint{0.856235in}{0.182419in}}%
\pgfpathlineto{\pgfqpoint{0.848578in}{0.181838in}}%
\pgfpathlineto{\pgfqpoint{0.850121in}{0.179853in}}%
\pgfpathlineto{\pgfqpoint{0.844579in}{0.174155in}}%
\pgfpathlineto{\pgfqpoint{0.842793in}{0.177323in}}%
\pgfpathlineto{\pgfqpoint{0.843258in}{0.181462in}}%
\pgfpathlineto{\pgfqpoint{0.846439in}{0.181386in}}%
\pgfpathlineto{\pgfqpoint{0.849272in}{0.184426in}}%
\pgfpathlineto{\pgfqpoint{0.850948in}{0.188866in}}%
\pgfpathlineto{\pgfqpoint{0.857264in}{0.187613in}}%
\pgfpathlineto{\pgfqpoint{0.852208in}{0.189984in}}%
\pgfpathlineto{\pgfqpoint{0.852653in}{0.193161in}}%
\pgfpathlineto{\pgfqpoint{0.850549in}{0.194596in}}%
\pgfpathlineto{\pgfqpoint{0.849687in}{0.198923in}}%
\pgfpathlineto{\pgfqpoint{0.851181in}{0.201177in}}%
\pgfpathlineto{\pgfqpoint{0.847619in}{0.208179in}}%
\pgfpathlineto{\pgfqpoint{0.847919in}{0.212633in}}%
\pgfpathlineto{\pgfqpoint{0.843746in}{0.216599in}}%
\pgfpathlineto{\pgfqpoint{0.841488in}{0.219808in}}%
\pgfpathlineto{\pgfqpoint{0.844606in}{0.222866in}}%
\pgfpathlineto{\pgfqpoint{0.845761in}{0.227437in}}%
\pgfpathlineto{\pgfqpoint{0.844730in}{0.230438in}}%
\pgfpathlineto{\pgfqpoint{0.845936in}{0.234017in}}%
\pgfpathlineto{\pgfqpoint{0.844399in}{0.245686in}}%
\pgfpathlineto{\pgfqpoint{0.842100in}{0.250980in}}%
\pgfpathlineto{\pgfqpoint{0.839471in}{0.252970in}}%
\pgfpathlineto{\pgfqpoint{0.840590in}{0.259857in}}%
\pgfpathlineto{\pgfqpoint{0.839801in}{0.265235in}}%
\pgfpathlineto{\pgfqpoint{0.841230in}{0.268685in}}%
\pgfpathlineto{\pgfqpoint{0.838272in}{0.269169in}}%
\pgfpathlineto{\pgfqpoint{0.837479in}{0.260789in}}%
\pgfpathlineto{\pgfqpoint{0.834154in}{0.251441in}}%
\pgfpathlineto{\pgfqpoint{0.831484in}{0.253570in}}%
\pgfpathlineto{\pgfqpoint{0.831127in}{0.256642in}}%
\pgfpathlineto{\pgfqpoint{0.827861in}{0.260839in}}%
\pgfpathlineto{\pgfqpoint{0.829521in}{0.263587in}}%
\pgfpathlineto{\pgfqpoint{0.829005in}{0.269772in}}%
\pgfpathlineto{\pgfqpoint{0.824778in}{0.267597in}}%
\pgfpathlineto{\pgfqpoint{0.825607in}{0.264242in}}%
\pgfpathlineto{\pgfqpoint{0.824811in}{0.259983in}}%
\pgfpathlineto{\pgfqpoint{0.820066in}{0.259990in}}%
\pgfpathlineto{\pgfqpoint{0.817939in}{0.262630in}}%
\pgfpathlineto{\pgfqpoint{0.814729in}{0.263134in}}%
\pgfpathlineto{\pgfqpoint{0.812569in}{0.265977in}}%
\pgfpathlineto{\pgfqpoint{0.808852in}{0.273782in}}%
\pgfpathlineto{\pgfqpoint{0.807648in}{0.279842in}}%
\pgfpathlineto{\pgfqpoint{0.808418in}{0.282418in}}%
\pgfpathlineto{\pgfqpoint{0.806620in}{0.288091in}}%
\pgfpathlineto{\pgfqpoint{0.803639in}{0.291375in}}%
\pgfpathlineto{\pgfqpoint{0.798483in}{0.299768in}}%
\pgfpathlineto{\pgfqpoint{0.794744in}{0.307057in}}%
\pgfpathlineto{\pgfqpoint{0.798559in}{0.307167in}}%
\pgfpathlineto{\pgfqpoint{0.800804in}{0.308698in}}%
\pgfpathlineto{\pgfqpoint{0.801246in}{0.313435in}}%
\pgfpathlineto{\pgfqpoint{0.790450in}{0.314007in}}%
\pgfpathlineto{\pgfqpoint{0.781267in}{0.324264in}}%
\pgfpathlineto{\pgfqpoint{0.783682in}{0.326962in}}%
\pgfpathlineto{\pgfqpoint{0.779241in}{0.327423in}}%
\pgfpathlineto{\pgfqpoint{0.770423in}{0.336959in}}%
\pgfpathlineto{\pgfqpoint{0.756451in}{0.341741in}}%
\pgfpathlineto{\pgfqpoint{0.754570in}{0.347345in}}%
\pgfpathlineto{\pgfqpoint{0.751671in}{0.350264in}}%
\pgfpathlineto{\pgfqpoint{0.751539in}{0.353005in}}%
\pgfpathlineto{\pgfqpoint{0.749358in}{0.355041in}}%
\pgfpathlineto{\pgfqpoint{0.752220in}{0.358442in}}%
\pgfpathlineto{\pgfqpoint{0.745693in}{0.358770in}}%
\pgfpathlineto{\pgfqpoint{0.744034in}{0.363173in}}%
\pgfpathlineto{\pgfqpoint{0.741063in}{0.365047in}}%
\pgfpathlineto{\pgfqpoint{0.747012in}{0.367309in}}%
\pgfpathlineto{\pgfqpoint{0.743918in}{0.371345in}}%
\pgfpathlineto{\pgfqpoint{0.738224in}{0.373038in}}%
\pgfpathlineto{\pgfqpoint{0.739110in}{0.380491in}}%
\pgfpathlineto{\pgfqpoint{0.737478in}{0.383130in}}%
\pgfpathlineto{\pgfqpoint{0.734538in}{0.384988in}}%
\pgfpathlineto{\pgfqpoint{0.732024in}{0.383124in}}%
\pgfpathlineto{\pgfqpoint{0.730538in}{0.384877in}}%
\pgfpathlineto{\pgfqpoint{0.726206in}{0.385379in}}%
\pgfpathlineto{\pgfqpoint{0.726555in}{0.392143in}}%
\pgfpathlineto{\pgfqpoint{0.720275in}{0.387574in}}%
\pgfpathlineto{\pgfqpoint{0.720407in}{0.378160in}}%
\pgfpathlineto{\pgfqpoint{0.713666in}{0.376355in}}%
\pgfpathlineto{\pgfqpoint{0.709263in}{0.372596in}}%
\pgfpathlineto{\pgfqpoint{0.705725in}{0.371804in}}%
\pgfpathlineto{\pgfqpoint{0.701895in}{0.375593in}}%
\pgfpathlineto{\pgfqpoint{0.698485in}{0.375090in}}%
\pgfpathlineto{\pgfqpoint{0.699118in}{0.378551in}}%
\pgfpathlineto{\pgfqpoint{0.694270in}{0.376688in}}%
\pgfpathlineto{\pgfqpoint{0.689779in}{0.376478in}}%
\pgfpathlineto{\pgfqpoint{0.684523in}{0.374589in}}%
\pgfpathlineto{\pgfqpoint{0.673855in}{0.375652in}}%
\pgfpathlineto{\pgfqpoint{0.670430in}{0.374953in}}%
\pgfpathlineto{\pgfqpoint{0.667959in}{0.378065in}}%
\pgfpathlineto{\pgfqpoint{0.663397in}{0.378376in}}%
\pgfpathlineto{\pgfqpoint{0.662449in}{0.382409in}}%
\pgfpathlineto{\pgfqpoint{0.665167in}{0.384606in}}%
\pgfpathlineto{\pgfqpoint{0.668357in}{0.384399in}}%
\pgfpathlineto{\pgfqpoint{0.671019in}{0.381403in}}%
\pgfpathlineto{\pgfqpoint{0.672772in}{0.386005in}}%
\pgfpathlineto{\pgfqpoint{0.670326in}{0.391102in}}%
\pgfpathlineto{\pgfqpoint{0.675863in}{0.395589in}}%
\pgfpathlineto{\pgfqpoint{0.681411in}{0.397243in}}%
\pgfpathlineto{\pgfqpoint{0.685273in}{0.399940in}}%
\pgfpathlineto{\pgfqpoint{0.687775in}{0.402846in}}%
\pgfpathlineto{\pgfqpoint{0.689172in}{0.407609in}}%
\pgfpathlineto{\pgfqpoint{0.693583in}{0.406312in}}%
\pgfpathlineto{\pgfqpoint{0.703148in}{0.406979in}}%
\pgfpathlineto{\pgfqpoint{0.705201in}{0.401756in}}%
\pgfpathlineto{\pgfqpoint{0.708556in}{0.401529in}}%
\pgfpathlineto{\pgfqpoint{0.714662in}{0.393534in}}%
\pgfpathlineto{\pgfqpoint{0.714673in}{0.396448in}}%
\pgfpathlineto{\pgfqpoint{0.712540in}{0.397396in}}%
\pgfpathlineto{\pgfqpoint{0.708636in}{0.406940in}}%
\pgfpathlineto{\pgfqpoint{0.710536in}{0.408236in}}%
\pgfpathlineto{\pgfqpoint{0.705322in}{0.413215in}}%
\pgfpathlineto{\pgfqpoint{0.700663in}{0.414184in}}%
\pgfpathlineto{\pgfqpoint{0.696223in}{0.412463in}}%
\pgfpathlineto{\pgfqpoint{0.688619in}{0.413804in}}%
\pgfpathlineto{\pgfqpoint{0.685013in}{0.411392in}}%
\pgfpathlineto{\pgfqpoint{0.676801in}{0.409529in}}%
\pgfpathlineto{\pgfqpoint{0.676050in}{0.406847in}}%
\pgfpathlineto{\pgfqpoint{0.669741in}{0.406115in}}%
\pgfpathlineto{\pgfqpoint{0.668122in}{0.403049in}}%
\pgfpathlineto{\pgfqpoint{0.664513in}{0.400803in}}%
\pgfpathlineto{\pgfqpoint{0.660154in}{0.401189in}}%
\pgfpathlineto{\pgfqpoint{0.657977in}{0.399018in}}%
\pgfpathlineto{\pgfqpoint{0.655231in}{0.399119in}}%
\pgfpathlineto{\pgfqpoint{0.652607in}{0.401074in}}%
\pgfpathlineto{\pgfqpoint{0.648019in}{0.400725in}}%
\pgfpathlineto{\pgfqpoint{0.646938in}{0.398924in}}%
\pgfpathlineto{\pgfqpoint{0.642119in}{0.400773in}}%
\pgfpathlineto{\pgfqpoint{0.638027in}{0.396000in}}%
\pgfpathlineto{\pgfqpoint{0.641950in}{0.391382in}}%
\pgfpathlineto{\pgfqpoint{0.643460in}{0.387584in}}%
\pgfpathlineto{\pgfqpoint{0.642068in}{0.384367in}}%
\pgfpathlineto{\pgfqpoint{0.636493in}{0.382148in}}%
\pgfpathlineto{\pgfqpoint{0.633263in}{0.383954in}}%
\pgfpathlineto{\pgfqpoint{0.629198in}{0.382976in}}%
\pgfpathlineto{\pgfqpoint{0.628708in}{0.380157in}}%
\pgfpathlineto{\pgfqpoint{0.623759in}{0.381154in}}%
\pgfpathlineto{\pgfqpoint{0.624855in}{0.378337in}}%
\pgfpathlineto{\pgfqpoint{0.617555in}{0.377743in}}%
\pgfpathlineto{\pgfqpoint{0.612830in}{0.381319in}}%
\pgfpathlineto{\pgfqpoint{0.610306in}{0.379045in}}%
\pgfpathlineto{\pgfqpoint{0.603590in}{0.381466in}}%
\pgfpathlineto{\pgfqpoint{0.601918in}{0.379014in}}%
\pgfpathlineto{\pgfqpoint{0.598588in}{0.377923in}}%
\pgfpathlineto{\pgfqpoint{0.595917in}{0.380797in}}%
\pgfpathlineto{\pgfqpoint{0.586784in}{0.380083in}}%
\pgfpathlineto{\pgfqpoint{0.586794in}{0.375843in}}%
\pgfpathlineto{\pgfqpoint{0.583125in}{0.374721in}}%
\pgfpathlineto{\pgfqpoint{0.580270in}{0.376900in}}%
\pgfpathlineto{\pgfqpoint{0.576497in}{0.375689in}}%
\pgfpathlineto{\pgfqpoint{0.571614in}{0.375401in}}%
\pgfpathlineto{\pgfqpoint{0.564340in}{0.378466in}}%
\pgfpathlineto{\pgfqpoint{0.560174in}{0.375305in}}%
\pgfpathlineto{\pgfqpoint{0.554216in}{0.379530in}}%
\pgfpathlineto{\pgfqpoint{0.552268in}{0.377929in}}%
\pgfpathlineto{\pgfqpoint{0.550900in}{0.373791in}}%
\pgfpathlineto{\pgfqpoint{0.547071in}{0.371365in}}%
\pgfpathlineto{\pgfqpoint{0.540825in}{0.374142in}}%
\pgfpathlineto{\pgfqpoint{0.530479in}{0.381003in}}%
\pgfpathlineto{\pgfqpoint{0.527463in}{0.381613in}}%
\pgfpathlineto{\pgfqpoint{0.525762in}{0.380183in}}%
\pgfpathlineto{\pgfqpoint{0.521211in}{0.381155in}}%
\pgfpathlineto{\pgfqpoint{0.518991in}{0.383290in}}%
\pgfpathlineto{\pgfqpoint{0.514445in}{0.384441in}}%
\pgfpathlineto{\pgfqpoint{0.508238in}{0.384507in}}%
\pgfpathlineto{\pgfqpoint{0.505773in}{0.387054in}}%
\pgfpathlineto{\pgfqpoint{0.510497in}{0.389690in}}%
\pgfpathlineto{\pgfqpoint{0.509395in}{0.392291in}}%
\pgfpathlineto{\pgfqpoint{0.506149in}{0.391707in}}%
\pgfpathlineto{\pgfqpoint{0.504528in}{0.389662in}}%
\pgfpathlineto{\pgfqpoint{0.496930in}{0.386730in}}%
\pgfpathlineto{\pgfqpoint{0.493088in}{0.387727in}}%
\pgfpathlineto{\pgfqpoint{0.491471in}{0.393681in}}%
\pgfpathlineto{\pgfqpoint{0.486679in}{0.391121in}}%
\pgfpathlineto{\pgfqpoint{0.485286in}{0.398667in}}%
\pgfpathlineto{\pgfqpoint{0.483015in}{0.396722in}}%
\pgfpathlineto{\pgfqpoint{0.483395in}{0.393819in}}%
\pgfpathlineto{\pgfqpoint{0.478173in}{0.394586in}}%
\pgfpathlineto{\pgfqpoint{0.483605in}{0.399646in}}%
\pgfpathlineto{\pgfqpoint{0.489110in}{0.396639in}}%
\pgfpathlineto{\pgfqpoint{0.490291in}{0.398148in}}%
\pgfpathlineto{\pgfqpoint{0.496082in}{0.396339in}}%
\pgfpathlineto{\pgfqpoint{0.495720in}{0.398399in}}%
\pgfpathlineto{\pgfqpoint{0.511521in}{0.399081in}}%
\pgfpathlineto{\pgfqpoint{0.519230in}{0.395868in}}%
\pgfpathlineto{\pgfqpoint{0.522212in}{0.392698in}}%
\pgfpathlineto{\pgfqpoint{0.521327in}{0.389931in}}%
\pgfpathlineto{\pgfqpoint{0.524323in}{0.388883in}}%
\pgfpathlineto{\pgfqpoint{0.531765in}{0.393251in}}%
\pgfpathlineto{\pgfqpoint{0.541392in}{0.393531in}}%
\pgfpathlineto{\pgfqpoint{0.550136in}{0.390898in}}%
\pgfpathlineto{\pgfqpoint{0.555089in}{0.391340in}}%
\pgfpathlineto{\pgfqpoint{0.556933in}{0.387255in}}%
\pgfpathlineto{\pgfqpoint{0.559703in}{0.391758in}}%
\pgfpathlineto{\pgfqpoint{0.561407in}{0.392733in}}%
\pgfpathlineto{\pgfqpoint{0.569288in}{0.394691in}}%
\pgfpathlineto{\pgfqpoint{0.572217in}{0.393494in}}%
\pgfpathlineto{\pgfqpoint{0.575385in}{0.394915in}}%
\pgfpathlineto{\pgfqpoint{0.579163in}{0.394172in}}%
\pgfpathlineto{\pgfqpoint{0.580055in}{0.395827in}}%
\pgfpathlineto{\pgfqpoint{0.587939in}{0.403420in}}%
\pgfpathlineto{\pgfqpoint{0.591278in}{0.404545in}}%
\pgfpathlineto{\pgfqpoint{0.593251in}{0.408644in}}%
\pgfpathlineto{\pgfqpoint{0.603722in}{0.411817in}}%
\pgfpathlineto{\pgfqpoint{0.605033in}{0.414238in}}%
\pgfpathlineto{\pgfqpoint{0.590192in}{0.417974in}}%
\pgfpathlineto{\pgfqpoint{0.590776in}{0.421993in}}%
\pgfpathlineto{\pgfqpoint{0.589250in}{0.427301in}}%
\pgfpathlineto{\pgfqpoint{0.585794in}{0.426860in}}%
\pgfpathlineto{\pgfqpoint{0.583443in}{0.419972in}}%
\pgfpathlineto{\pgfqpoint{0.580714in}{0.419404in}}%
\pgfpathlineto{\pgfqpoint{0.578938in}{0.421634in}}%
\pgfpathlineto{\pgfqpoint{0.581010in}{0.431388in}}%
\pgfpathlineto{\pgfqpoint{0.578947in}{0.435475in}}%
\pgfpathlineto{\pgfqpoint{0.574950in}{0.440293in}}%
\pgfpathlineto{\pgfqpoint{0.576871in}{0.443969in}}%
\pgfpathlineto{\pgfqpoint{0.568436in}{0.444102in}}%
\pgfpathlineto{\pgfqpoint{0.568628in}{0.445528in}}%
\pgfpathlineto{\pgfqpoint{0.563595in}{0.447526in}}%
\pgfpathlineto{\pgfqpoint{0.562031in}{0.446426in}}%
\pgfpathclose%
\pgfusepath{fill}%
\end{pgfscope}%
\begin{pgfscope}%
\pgfpathrectangle{\pgfqpoint{0.100000in}{0.100000in}}{\pgfqpoint{2.989028in}{1.913466in}}%
\pgfusepath{clip}%
\pgfsetbuttcap%
\pgfsetmiterjoin%
\definecolor{currentfill}{rgb}{0.400000,0.760784,0.647059}%
\pgfsetfillcolor{currentfill}%
\pgfsetlinewidth{0.000000pt}%
\definecolor{currentstroke}{rgb}{0.000000,0.000000,0.000000}%
\pgfsetstrokecolor{currentstroke}%
\pgfsetstrokeopacity{0.000000}%
\pgfsetdash{}{0pt}%
\pgfpathmoveto{\pgfqpoint{0.546979in}{0.517466in}}%
\pgfpathlineto{\pgfqpoint{0.545936in}{0.515633in}}%
\pgfpathlineto{\pgfqpoint{0.548696in}{0.511730in}}%
\pgfpathlineto{\pgfqpoint{0.544334in}{0.508079in}}%
\pgfpathlineto{\pgfqpoint{0.544455in}{0.504939in}}%
\pgfpathlineto{\pgfqpoint{0.542483in}{0.504198in}}%
\pgfpathlineto{\pgfqpoint{0.535704in}{0.508881in}}%
\pgfpathlineto{\pgfqpoint{0.530872in}{0.519132in}}%
\pgfpathlineto{\pgfqpoint{0.530549in}{0.522094in}}%
\pgfpathlineto{\pgfqpoint{0.531791in}{0.525640in}}%
\pgfpathlineto{\pgfqpoint{0.537027in}{0.520216in}}%
\pgfpathlineto{\pgfqpoint{0.538628in}{0.521262in}}%
\pgfpathlineto{\pgfqpoint{0.543070in}{0.520267in}}%
\pgfpathlineto{\pgfqpoint{0.546979in}{0.517466in}}%
\pgfpathclose%
\pgfusepath{fill}%
\end{pgfscope}%
\begin{pgfscope}%
\pgfpathrectangle{\pgfqpoint{0.100000in}{0.100000in}}{\pgfqpoint{2.989028in}{1.913466in}}%
\pgfusepath{clip}%
\pgfsetbuttcap%
\pgfsetmiterjoin%
\definecolor{currentfill}{rgb}{0.400000,0.760784,0.647059}%
\pgfsetfillcolor{currentfill}%
\pgfsetlinewidth{0.000000pt}%
\definecolor{currentstroke}{rgb}{0.000000,0.000000,0.000000}%
\pgfsetstrokecolor{currentstroke}%
\pgfsetstrokeopacity{0.000000}%
\pgfsetdash{}{0pt}%
\pgfpathmoveto{\pgfqpoint{0.466584in}{0.398705in}}%
\pgfpathlineto{\pgfqpoint{0.461820in}{0.397932in}}%
\pgfpathlineto{\pgfqpoint{0.458372in}{0.399525in}}%
\pgfpathlineto{\pgfqpoint{0.456865in}{0.402120in}}%
\pgfpathlineto{\pgfqpoint{0.458573in}{0.405842in}}%
\pgfpathlineto{\pgfqpoint{0.462377in}{0.404858in}}%
\pgfpathlineto{\pgfqpoint{0.468630in}{0.407066in}}%
\pgfpathlineto{\pgfqpoint{0.470820in}{0.404016in}}%
\pgfpathlineto{\pgfqpoint{0.472944in}{0.404579in}}%
\pgfpathlineto{\pgfqpoint{0.478084in}{0.402681in}}%
\pgfpathlineto{\pgfqpoint{0.480305in}{0.400269in}}%
\pgfpathlineto{\pgfqpoint{0.477595in}{0.393963in}}%
\pgfpathlineto{\pgfqpoint{0.474899in}{0.392626in}}%
\pgfpathlineto{\pgfqpoint{0.472476in}{0.393346in}}%
\pgfpathlineto{\pgfqpoint{0.466584in}{0.398705in}}%
\pgfpathclose%
\pgfusepath{fill}%
\end{pgfscope}%
\begin{pgfscope}%
\pgfpathrectangle{\pgfqpoint{0.100000in}{0.100000in}}{\pgfqpoint{2.989028in}{1.913466in}}%
\pgfusepath{clip}%
\pgfsetbuttcap%
\pgfsetmiterjoin%
\definecolor{currentfill}{rgb}{0.400000,0.760784,0.647059}%
\pgfsetfillcolor{currentfill}%
\pgfsetlinewidth{0.000000pt}%
\definecolor{currentstroke}{rgb}{0.000000,0.000000,0.000000}%
\pgfsetstrokecolor{currentstroke}%
\pgfsetstrokeopacity{0.000000}%
\pgfsetdash{}{0pt}%
\pgfpathmoveto{\pgfqpoint{0.423985in}{0.404436in}}%
\pgfpathlineto{\pgfqpoint{0.417234in}{0.407656in}}%
\pgfpathlineto{\pgfqpoint{0.410987in}{0.408614in}}%
\pgfpathlineto{\pgfqpoint{0.409018in}{0.409985in}}%
\pgfpathlineto{\pgfqpoint{0.411260in}{0.412685in}}%
\pgfpathlineto{\pgfqpoint{0.415239in}{0.409312in}}%
\pgfpathlineto{\pgfqpoint{0.418576in}{0.409777in}}%
\pgfpathlineto{\pgfqpoint{0.424004in}{0.412113in}}%
\pgfpathlineto{\pgfqpoint{0.424442in}{0.415120in}}%
\pgfpathlineto{\pgfqpoint{0.431109in}{0.413632in}}%
\pgfpathlineto{\pgfqpoint{0.429548in}{0.409103in}}%
\pgfpathlineto{\pgfqpoint{0.433794in}{0.408772in}}%
\pgfpathlineto{\pgfqpoint{0.430231in}{0.403385in}}%
\pgfpathlineto{\pgfqpoint{0.426781in}{0.405131in}}%
\pgfpathlineto{\pgfqpoint{0.423985in}{0.404436in}}%
\pgfpathclose%
\pgfusepath{fill}%
\end{pgfscope}%
\begin{pgfscope}%
\pgfpathrectangle{\pgfqpoint{0.100000in}{0.100000in}}{\pgfqpoint{2.989028in}{1.913466in}}%
\pgfusepath{clip}%
\pgfsetbuttcap%
\pgfsetmiterjoin%
\definecolor{currentfill}{rgb}{0.400000,0.760784,0.647059}%
\pgfsetfillcolor{currentfill}%
\pgfsetlinewidth{0.000000pt}%
\definecolor{currentstroke}{rgb}{0.000000,0.000000,0.000000}%
\pgfsetstrokecolor{currentstroke}%
\pgfsetstrokeopacity{0.000000}%
\pgfsetdash{}{0pt}%
\pgfpathmoveto{\pgfqpoint{0.412124in}{0.415568in}}%
\pgfpathlineto{\pgfqpoint{0.409643in}{0.414219in}}%
\pgfpathlineto{\pgfqpoint{0.402882in}{0.416293in}}%
\pgfpathlineto{\pgfqpoint{0.397830in}{0.415092in}}%
\pgfpathlineto{\pgfqpoint{0.395264in}{0.418504in}}%
\pgfpathlineto{\pgfqpoint{0.396383in}{0.420066in}}%
\pgfpathlineto{\pgfqpoint{0.400212in}{0.420327in}}%
\pgfpathlineto{\pgfqpoint{0.403140in}{0.419346in}}%
\pgfpathlineto{\pgfqpoint{0.406331in}{0.420898in}}%
\pgfpathlineto{\pgfqpoint{0.411223in}{0.418823in}}%
\pgfpathlineto{\pgfqpoint{0.412124in}{0.415568in}}%
\pgfpathclose%
\pgfusepath{fill}%
\end{pgfscope}%
\begin{pgfscope}%
\pgfpathrectangle{\pgfqpoint{0.100000in}{0.100000in}}{\pgfqpoint{2.989028in}{1.913466in}}%
\pgfusepath{clip}%
\pgfsetbuttcap%
\pgfsetmiterjoin%
\definecolor{currentfill}{rgb}{0.400000,0.760784,0.647059}%
\pgfsetfillcolor{currentfill}%
\pgfsetlinewidth{0.000000pt}%
\definecolor{currentstroke}{rgb}{0.000000,0.000000,0.000000}%
\pgfsetstrokecolor{currentstroke}%
\pgfsetstrokeopacity{0.000000}%
\pgfsetdash{}{0pt}%
\pgfpathmoveto{\pgfqpoint{0.332429in}{0.465782in}}%
\pgfpathlineto{\pgfqpoint{0.332250in}{0.462436in}}%
\pgfpathlineto{\pgfqpoint{0.327314in}{0.460334in}}%
\pgfpathlineto{\pgfqpoint{0.322807in}{0.465597in}}%
\pgfpathlineto{\pgfqpoint{0.327862in}{0.467350in}}%
\pgfpathlineto{\pgfqpoint{0.332429in}{0.465782in}}%
\pgfpathclose%
\pgfusepath{fill}%
\end{pgfscope}%
\begin{pgfscope}%
\pgfpathrectangle{\pgfqpoint{0.100000in}{0.100000in}}{\pgfqpoint{2.989028in}{1.913466in}}%
\pgfusepath{clip}%
\pgfsetbuttcap%
\pgfsetmiterjoin%
\definecolor{currentfill}{rgb}{0.400000,0.760784,0.647059}%
\pgfsetfillcolor{currentfill}%
\pgfsetlinewidth{0.000000pt}%
\definecolor{currentstroke}{rgb}{0.000000,0.000000,0.000000}%
\pgfsetstrokecolor{currentstroke}%
\pgfsetstrokeopacity{0.000000}%
\pgfsetdash{}{0pt}%
\pgfpathmoveto{\pgfqpoint{0.291722in}{0.484589in}}%
\pgfpathlineto{\pgfqpoint{0.292771in}{0.486043in}}%
\pgfpathlineto{\pgfqpoint{0.298825in}{0.485792in}}%
\pgfpathlineto{\pgfqpoint{0.299887in}{0.487956in}}%
\pgfpathlineto{\pgfqpoint{0.302543in}{0.486068in}}%
\pgfpathlineto{\pgfqpoint{0.301009in}{0.482435in}}%
\pgfpathlineto{\pgfqpoint{0.298753in}{0.482127in}}%
\pgfpathlineto{\pgfqpoint{0.293361in}{0.483267in}}%
\pgfpathlineto{\pgfqpoint{0.291722in}{0.484589in}}%
\pgfpathclose%
\pgfusepath{fill}%
\end{pgfscope}%
\begin{pgfscope}%
\pgfpathrectangle{\pgfqpoint{0.100000in}{0.100000in}}{\pgfqpoint{2.989028in}{1.913466in}}%
\pgfusepath{clip}%
\pgfsetbuttcap%
\pgfsetmiterjoin%
\definecolor{currentfill}{rgb}{0.400000,0.760784,0.647059}%
\pgfsetfillcolor{currentfill}%
\pgfsetlinewidth{0.000000pt}%
\definecolor{currentstroke}{rgb}{0.000000,0.000000,0.000000}%
\pgfsetstrokecolor{currentstroke}%
\pgfsetstrokeopacity{0.000000}%
\pgfsetdash{}{0pt}%
\pgfpathmoveto{\pgfqpoint{0.283137in}{0.495848in}}%
\pgfpathlineto{\pgfqpoint{0.282770in}{0.499125in}}%
\pgfpathlineto{\pgfqpoint{0.285929in}{0.498882in}}%
\pgfpathlineto{\pgfqpoint{0.285709in}{0.503484in}}%
\pgfpathlineto{\pgfqpoint{0.288886in}{0.501000in}}%
\pgfpathlineto{\pgfqpoint{0.287787in}{0.497411in}}%
\pgfpathlineto{\pgfqpoint{0.283137in}{0.495848in}}%
\pgfpathclose%
\pgfusepath{fill}%
\end{pgfscope}%
\begin{pgfscope}%
\pgfpathrectangle{\pgfqpoint{0.100000in}{0.100000in}}{\pgfqpoint{2.989028in}{1.913466in}}%
\pgfusepath{clip}%
\pgfsetbuttcap%
\pgfsetmiterjoin%
\definecolor{currentfill}{rgb}{0.400000,0.760784,0.647059}%
\pgfsetfillcolor{currentfill}%
\pgfsetlinewidth{0.000000pt}%
\definecolor{currentstroke}{rgb}{0.000000,0.000000,0.000000}%
\pgfsetstrokecolor{currentstroke}%
\pgfsetstrokeopacity{0.000000}%
\pgfsetdash{}{0pt}%
\pgfpathmoveto{\pgfqpoint{0.561438in}{0.612633in}}%
\pgfpathlineto{\pgfqpoint{0.556400in}{0.613721in}}%
\pgfpathlineto{\pgfqpoint{0.555318in}{0.617088in}}%
\pgfpathlineto{\pgfqpoint{0.557049in}{0.620453in}}%
\pgfpathlineto{\pgfqpoint{0.560927in}{0.622211in}}%
\pgfpathlineto{\pgfqpoint{0.565367in}{0.613572in}}%
\pgfpathlineto{\pgfqpoint{0.569395in}{0.613189in}}%
\pgfpathlineto{\pgfqpoint{0.572399in}{0.610523in}}%
\pgfpathlineto{\pgfqpoint{0.572340in}{0.606855in}}%
\pgfpathlineto{\pgfqpoint{0.570143in}{0.605097in}}%
\pgfpathlineto{\pgfqpoint{0.573716in}{0.594249in}}%
\pgfpathlineto{\pgfqpoint{0.577625in}{0.590269in}}%
\pgfpathlineto{\pgfqpoint{0.573628in}{0.589006in}}%
\pgfpathlineto{\pgfqpoint{0.570393in}{0.593472in}}%
\pgfpathlineto{\pgfqpoint{0.563363in}{0.592089in}}%
\pgfpathlineto{\pgfqpoint{0.565549in}{0.596509in}}%
\pgfpathlineto{\pgfqpoint{0.564531in}{0.598269in}}%
\pgfpathlineto{\pgfqpoint{0.565128in}{0.602876in}}%
\pgfpathlineto{\pgfqpoint{0.563735in}{0.611204in}}%
\pgfpathlineto{\pgfqpoint{0.561438in}{0.612633in}}%
\pgfpathclose%
\pgfusepath{fill}%
\end{pgfscope}%
\begin{pgfscope}%
\pgfpathrectangle{\pgfqpoint{0.100000in}{0.100000in}}{\pgfqpoint{2.989028in}{1.913466in}}%
\pgfusepath{clip}%
\pgfsetbuttcap%
\pgfsetmiterjoin%
\definecolor{currentfill}{rgb}{0.400000,0.760784,0.647059}%
\pgfsetfillcolor{currentfill}%
\pgfsetlinewidth{0.000000pt}%
\definecolor{currentstroke}{rgb}{0.000000,0.000000,0.000000}%
\pgfsetstrokecolor{currentstroke}%
\pgfsetstrokeopacity{0.000000}%
\pgfsetdash{}{0pt}%
\pgfpathmoveto{\pgfqpoint{0.738441in}{0.364487in}}%
\pgfpathlineto{\pgfqpoint{0.731138in}{0.364569in}}%
\pgfpathlineto{\pgfqpoint{0.731700in}{0.368235in}}%
\pgfpathlineto{\pgfqpoint{0.734623in}{0.369546in}}%
\pgfpathlineto{\pgfqpoint{0.738441in}{0.364487in}}%
\pgfpathclose%
\pgfusepath{fill}%
\end{pgfscope}%
\begin{pgfscope}%
\pgfpathrectangle{\pgfqpoint{0.100000in}{0.100000in}}{\pgfqpoint{2.989028in}{1.913466in}}%
\pgfusepath{clip}%
\pgfsetbuttcap%
\pgfsetmiterjoin%
\definecolor{currentfill}{rgb}{0.400000,0.760784,0.647059}%
\pgfsetfillcolor{currentfill}%
\pgfsetlinewidth{0.000000pt}%
\definecolor{currentstroke}{rgb}{0.000000,0.000000,0.000000}%
\pgfsetstrokecolor{currentstroke}%
\pgfsetstrokeopacity{0.000000}%
\pgfsetdash{}{0pt}%
\pgfpathmoveto{\pgfqpoint{0.728306in}{0.367807in}}%
\pgfpathlineto{\pgfqpoint{0.723074in}{0.366541in}}%
\pgfpathlineto{\pgfqpoint{0.718102in}{0.364373in}}%
\pgfpathlineto{\pgfqpoint{0.716567in}{0.366927in}}%
\pgfpathlineto{\pgfqpoint{0.725117in}{0.368657in}}%
\pgfpathlineto{\pgfqpoint{0.728306in}{0.367807in}}%
\pgfpathclose%
\pgfusepath{fill}%
\end{pgfscope}%
\begin{pgfscope}%
\pgfpathrectangle{\pgfqpoint{0.100000in}{0.100000in}}{\pgfqpoint{2.989028in}{1.913466in}}%
\pgfusepath{clip}%
\pgfsetbuttcap%
\pgfsetmiterjoin%
\definecolor{currentfill}{rgb}{0.400000,0.760784,0.647059}%
\pgfsetfillcolor{currentfill}%
\pgfsetlinewidth{0.000000pt}%
\definecolor{currentstroke}{rgb}{0.000000,0.000000,0.000000}%
\pgfsetstrokecolor{currentstroke}%
\pgfsetstrokeopacity{0.000000}%
\pgfsetdash{}{0pt}%
\pgfpathmoveto{\pgfqpoint{0.648103in}{0.365212in}}%
\pgfpathlineto{\pgfqpoint{0.648303in}{0.362676in}}%
\pgfpathlineto{\pgfqpoint{0.645288in}{0.361100in}}%
\pgfpathlineto{\pgfqpoint{0.636875in}{0.364714in}}%
\pgfpathlineto{\pgfqpoint{0.635570in}{0.363261in}}%
\pgfpathlineto{\pgfqpoint{0.633535in}{0.370138in}}%
\pgfpathlineto{\pgfqpoint{0.637750in}{0.371095in}}%
\pgfpathlineto{\pgfqpoint{0.639698in}{0.368269in}}%
\pgfpathlineto{\pgfqpoint{0.643929in}{0.371807in}}%
\pgfpathlineto{\pgfqpoint{0.648103in}{0.365212in}}%
\pgfpathclose%
\pgfusepath{fill}%
\end{pgfscope}%
\begin{pgfscope}%
\pgfpathrectangle{\pgfqpoint{0.100000in}{0.100000in}}{\pgfqpoint{2.989028in}{1.913466in}}%
\pgfusepath{clip}%
\pgfsetbuttcap%
\pgfsetmiterjoin%
\definecolor{currentfill}{rgb}{0.400000,0.760784,0.647059}%
\pgfsetfillcolor{currentfill}%
\pgfsetlinewidth{0.000000pt}%
\definecolor{currentstroke}{rgb}{0.000000,0.000000,0.000000}%
\pgfsetstrokecolor{currentstroke}%
\pgfsetstrokeopacity{0.000000}%
\pgfsetdash{}{0pt}%
\pgfpathmoveto{\pgfqpoint{0.837890in}{0.224669in}}%
\pgfpathlineto{\pgfqpoint{0.830835in}{0.222042in}}%
\pgfpathlineto{\pgfqpoint{0.827098in}{0.221885in}}%
\pgfpathlineto{\pgfqpoint{0.830139in}{0.227457in}}%
\pgfpathlineto{\pgfqpoint{0.832738in}{0.229535in}}%
\pgfpathlineto{\pgfqpoint{0.832654in}{0.235306in}}%
\pgfpathlineto{\pgfqpoint{0.835479in}{0.245723in}}%
\pgfpathlineto{\pgfqpoint{0.838081in}{0.247702in}}%
\pgfpathlineto{\pgfqpoint{0.844087in}{0.244837in}}%
\pgfpathlineto{\pgfqpoint{0.843210in}{0.243196in}}%
\pgfpathlineto{\pgfqpoint{0.843665in}{0.235310in}}%
\pgfpathlineto{\pgfqpoint{0.841718in}{0.233182in}}%
\pgfpathlineto{\pgfqpoint{0.840881in}{0.227581in}}%
\pgfpathlineto{\pgfqpoint{0.837890in}{0.224669in}}%
\pgfpathclose%
\pgfusepath{fill}%
\end{pgfscope}%
\begin{pgfscope}%
\pgfpathrectangle{\pgfqpoint{0.100000in}{0.100000in}}{\pgfqpoint{2.989028in}{1.913466in}}%
\pgfusepath{clip}%
\pgfsetbuttcap%
\pgfsetmiterjoin%
\definecolor{currentfill}{rgb}{0.400000,0.760784,0.647059}%
\pgfsetfillcolor{currentfill}%
\pgfsetlinewidth{0.000000pt}%
\definecolor{currentstroke}{rgb}{0.000000,0.000000,0.000000}%
\pgfsetstrokecolor{currentstroke}%
\pgfsetstrokeopacity{0.000000}%
\pgfsetdash{}{0pt}%
\pgfpathmoveto{\pgfqpoint{0.822266in}{0.250492in}}%
\pgfpathlineto{\pgfqpoint{0.826399in}{0.243355in}}%
\pgfpathlineto{\pgfqpoint{0.825633in}{0.241653in}}%
\pgfpathlineto{\pgfqpoint{0.831161in}{0.239922in}}%
\pgfpathlineto{\pgfqpoint{0.829574in}{0.233362in}}%
\pgfpathlineto{\pgfqpoint{0.827167in}{0.233665in}}%
\pgfpathlineto{\pgfqpoint{0.820842in}{0.245101in}}%
\pgfpathlineto{\pgfqpoint{0.821742in}{0.239954in}}%
\pgfpathlineto{\pgfqpoint{0.820664in}{0.236990in}}%
\pgfpathlineto{\pgfqpoint{0.817986in}{0.235671in}}%
\pgfpathlineto{\pgfqpoint{0.816463in}{0.237092in}}%
\pgfpathlineto{\pgfqpoint{0.815613in}{0.243682in}}%
\pgfpathlineto{\pgfqpoint{0.816115in}{0.247956in}}%
\pgfpathlineto{\pgfqpoint{0.814331in}{0.249822in}}%
\pgfpathlineto{\pgfqpoint{0.819043in}{0.255220in}}%
\pgfpathlineto{\pgfqpoint{0.822135in}{0.256822in}}%
\pgfpathlineto{\pgfqpoint{0.823165in}{0.254717in}}%
\pgfpathlineto{\pgfqpoint{0.826391in}{0.256346in}}%
\pgfpathlineto{\pgfqpoint{0.828728in}{0.251922in}}%
\pgfpathlineto{\pgfqpoint{0.833783in}{0.246295in}}%
\pgfpathlineto{\pgfqpoint{0.833093in}{0.243746in}}%
\pgfpathlineto{\pgfqpoint{0.830286in}{0.240969in}}%
\pgfpathlineto{\pgfqpoint{0.827691in}{0.242277in}}%
\pgfpathlineto{\pgfqpoint{0.822266in}{0.250492in}}%
\pgfpathclose%
\pgfusepath{fill}%
\end{pgfscope}%
\begin{pgfscope}%
\pgfpathrectangle{\pgfqpoint{0.100000in}{0.100000in}}{\pgfqpoint{2.989028in}{1.913466in}}%
\pgfusepath{clip}%
\pgfsetbuttcap%
\pgfsetmiterjoin%
\definecolor{currentfill}{rgb}{0.400000,0.760784,0.647059}%
\pgfsetfillcolor{currentfill}%
\pgfsetlinewidth{0.000000pt}%
\definecolor{currentstroke}{rgb}{0.000000,0.000000,0.000000}%
\pgfsetstrokecolor{currentstroke}%
\pgfsetstrokeopacity{0.000000}%
\pgfsetdash{}{0pt}%
\pgfpathmoveto{\pgfqpoint{0.617161in}{0.351684in}}%
\pgfpathlineto{\pgfqpoint{0.612041in}{0.351018in}}%
\pgfpathlineto{\pgfqpoint{0.604830in}{0.348246in}}%
\pgfpathlineto{\pgfqpoint{0.603148in}{0.349752in}}%
\pgfpathlineto{\pgfqpoint{0.610680in}{0.351454in}}%
\pgfpathlineto{\pgfqpoint{0.608661in}{0.352939in}}%
\pgfpathlineto{\pgfqpoint{0.603991in}{0.354027in}}%
\pgfpathlineto{\pgfqpoint{0.602744in}{0.357507in}}%
\pgfpathlineto{\pgfqpoint{0.605094in}{0.360506in}}%
\pgfpathlineto{\pgfqpoint{0.607073in}{0.367387in}}%
\pgfpathlineto{\pgfqpoint{0.614301in}{0.368669in}}%
\pgfpathlineto{\pgfqpoint{0.617801in}{0.365696in}}%
\pgfpathlineto{\pgfqpoint{0.621088in}{0.368573in}}%
\pgfpathlineto{\pgfqpoint{0.625211in}{0.367867in}}%
\pgfpathlineto{\pgfqpoint{0.628174in}{0.362553in}}%
\pgfpathlineto{\pgfqpoint{0.630683in}{0.367277in}}%
\pgfpathlineto{\pgfqpoint{0.637804in}{0.356228in}}%
\pgfpathlineto{\pgfqpoint{0.635232in}{0.355345in}}%
\pgfpathlineto{\pgfqpoint{0.633653in}{0.352950in}}%
\pgfpathlineto{\pgfqpoint{0.636636in}{0.350764in}}%
\pgfpathlineto{\pgfqpoint{0.632011in}{0.348715in}}%
\pgfpathlineto{\pgfqpoint{0.628656in}{0.349843in}}%
\pgfpathlineto{\pgfqpoint{0.622321in}{0.353583in}}%
\pgfpathlineto{\pgfqpoint{0.617161in}{0.351684in}}%
\pgfpathclose%
\pgfusepath{fill}%
\end{pgfscope}%
\begin{pgfscope}%
\pgfpathrectangle{\pgfqpoint{0.100000in}{0.100000in}}{\pgfqpoint{2.989028in}{1.913466in}}%
\pgfusepath{clip}%
\pgfsetbuttcap%
\pgfsetmiterjoin%
\definecolor{currentfill}{rgb}{0.400000,0.760784,0.647059}%
\pgfsetfillcolor{currentfill}%
\pgfsetlinewidth{0.000000pt}%
\definecolor{currentstroke}{rgb}{0.000000,0.000000,0.000000}%
\pgfsetstrokecolor{currentstroke}%
\pgfsetstrokeopacity{0.000000}%
\pgfsetdash{}{0pt}%
\pgfpathmoveto{\pgfqpoint{0.817967in}{0.203589in}}%
\pgfpathlineto{\pgfqpoint{0.816952in}{0.213793in}}%
\pgfpathlineto{\pgfqpoint{0.819688in}{0.217539in}}%
\pgfpathlineto{\pgfqpoint{0.816762in}{0.217800in}}%
\pgfpathlineto{\pgfqpoint{0.817832in}{0.225086in}}%
\pgfpathlineto{\pgfqpoint{0.819093in}{0.229412in}}%
\pgfpathlineto{\pgfqpoint{0.817997in}{0.235224in}}%
\pgfpathlineto{\pgfqpoint{0.820677in}{0.236349in}}%
\pgfpathlineto{\pgfqpoint{0.821551in}{0.237852in}}%
\pgfpathlineto{\pgfqpoint{0.824191in}{0.237282in}}%
\pgfpathlineto{\pgfqpoint{0.826476in}{0.232626in}}%
\pgfpathlineto{\pgfqpoint{0.826901in}{0.228360in}}%
\pgfpathlineto{\pgfqpoint{0.824105in}{0.215583in}}%
\pgfpathlineto{\pgfqpoint{0.817967in}{0.203589in}}%
\pgfpathclose%
\pgfusepath{fill}%
\end{pgfscope}%
\begin{pgfscope}%
\pgfpathrectangle{\pgfqpoint{0.100000in}{0.100000in}}{\pgfqpoint{2.989028in}{1.913466in}}%
\pgfusepath{clip}%
\pgfsetbuttcap%
\pgfsetmiterjoin%
\definecolor{currentfill}{rgb}{0.400000,0.760784,0.647059}%
\pgfsetfillcolor{currentfill}%
\pgfsetlinewidth{0.000000pt}%
\definecolor{currentstroke}{rgb}{0.000000,0.000000,0.000000}%
\pgfsetstrokecolor{currentstroke}%
\pgfsetstrokeopacity{0.000000}%
\pgfsetdash{}{0pt}%
\pgfpathmoveto{\pgfqpoint{0.817360in}{0.234737in}}%
\pgfpathlineto{\pgfqpoint{0.817219in}{0.229762in}}%
\pgfpathlineto{\pgfqpoint{0.814925in}{0.227503in}}%
\pgfpathlineto{\pgfqpoint{0.812273in}{0.228338in}}%
\pgfpathlineto{\pgfqpoint{0.815625in}{0.235565in}}%
\pgfpathlineto{\pgfqpoint{0.817360in}{0.234737in}}%
\pgfpathclose%
\pgfusepath{fill}%
\end{pgfscope}%
\begin{pgfscope}%
\pgfpathrectangle{\pgfqpoint{0.100000in}{0.100000in}}{\pgfqpoint{2.989028in}{1.913466in}}%
\pgfusepath{clip}%
\pgfsetbuttcap%
\pgfsetmiterjoin%
\definecolor{currentfill}{rgb}{0.400000,0.760784,0.647059}%
\pgfsetfillcolor{currentfill}%
\pgfsetlinewidth{0.000000pt}%
\definecolor{currentstroke}{rgb}{0.000000,0.000000,0.000000}%
\pgfsetstrokecolor{currentstroke}%
\pgfsetstrokeopacity{0.000000}%
\pgfsetdash{}{0pt}%
\pgfpathmoveto{\pgfqpoint{0.845513in}{0.212648in}}%
\pgfpathlineto{\pgfqpoint{0.843557in}{0.204322in}}%
\pgfpathlineto{\pgfqpoint{0.838998in}{0.201690in}}%
\pgfpathlineto{\pgfqpoint{0.833017in}{0.204115in}}%
\pgfpathlineto{\pgfqpoint{0.834665in}{0.207350in}}%
\pgfpathlineto{\pgfqpoint{0.834632in}{0.209591in}}%
\pgfpathlineto{\pgfqpoint{0.836855in}{0.213398in}}%
\pgfpathlineto{\pgfqpoint{0.835292in}{0.212632in}}%
\pgfpathlineto{\pgfqpoint{0.836432in}{0.214491in}}%
\pgfpathlineto{\pgfqpoint{0.834267in}{0.218715in}}%
\pgfpathlineto{\pgfqpoint{0.836723in}{0.219493in}}%
\pgfpathlineto{\pgfqpoint{0.842410in}{0.214551in}}%
\pgfpathlineto{\pgfqpoint{0.845513in}{0.212648in}}%
\pgfpathclose%
\pgfusepath{fill}%
\end{pgfscope}%
\begin{pgfscope}%
\pgfpathrectangle{\pgfqpoint{0.100000in}{0.100000in}}{\pgfqpoint{2.989028in}{1.913466in}}%
\pgfusepath{clip}%
\pgfsetbuttcap%
\pgfsetmiterjoin%
\definecolor{currentfill}{rgb}{0.400000,0.760784,0.647059}%
\pgfsetfillcolor{currentfill}%
\pgfsetlinewidth{0.000000pt}%
\definecolor{currentstroke}{rgb}{0.000000,0.000000,0.000000}%
\pgfsetstrokecolor{currentstroke}%
\pgfsetstrokeopacity{0.000000}%
\pgfsetdash{}{0pt}%
\pgfpathmoveto{\pgfqpoint{0.825557in}{0.197926in}}%
\pgfpathlineto{\pgfqpoint{0.822979in}{0.196923in}}%
\pgfpathlineto{\pgfqpoint{0.824451in}{0.205709in}}%
\pgfpathlineto{\pgfqpoint{0.827546in}{0.207760in}}%
\pgfpathlineto{\pgfqpoint{0.826791in}{0.214343in}}%
\pgfpathlineto{\pgfqpoint{0.828049in}{0.217262in}}%
\pgfpathlineto{\pgfqpoint{0.830536in}{0.218356in}}%
\pgfpathlineto{\pgfqpoint{0.833148in}{0.215493in}}%
\pgfpathlineto{\pgfqpoint{0.835389in}{0.211607in}}%
\pgfpathlineto{\pgfqpoint{0.828184in}{0.203995in}}%
\pgfpathlineto{\pgfqpoint{0.825557in}{0.197926in}}%
\pgfpathclose%
\pgfusepath{fill}%
\end{pgfscope}%
\begin{pgfscope}%
\pgfpathrectangle{\pgfqpoint{0.100000in}{0.100000in}}{\pgfqpoint{2.989028in}{1.913466in}}%
\pgfusepath{clip}%
\pgfsetbuttcap%
\pgfsetmiterjoin%
\definecolor{currentfill}{rgb}{0.400000,0.760784,0.647059}%
\pgfsetfillcolor{currentfill}%
\pgfsetlinewidth{0.000000pt}%
\definecolor{currentstroke}{rgb}{0.000000,0.000000,0.000000}%
\pgfsetstrokecolor{currentstroke}%
\pgfsetstrokeopacity{0.000000}%
\pgfsetdash{}{0pt}%
\pgfpathmoveto{\pgfqpoint{0.846657in}{0.207083in}}%
\pgfpathlineto{\pgfqpoint{0.847921in}{0.201020in}}%
\pgfpathlineto{\pgfqpoint{0.842275in}{0.201714in}}%
\pgfpathlineto{\pgfqpoint{0.846657in}{0.207083in}}%
\pgfpathclose%
\pgfusepath{fill}%
\end{pgfscope}%
\begin{pgfscope}%
\pgfpathrectangle{\pgfqpoint{0.100000in}{0.100000in}}{\pgfqpoint{2.989028in}{1.913466in}}%
\pgfusepath{clip}%
\pgfsetbuttcap%
\pgfsetmiterjoin%
\definecolor{currentfill}{rgb}{0.400000,0.760784,0.647059}%
\pgfsetfillcolor{currentfill}%
\pgfsetlinewidth{0.000000pt}%
\definecolor{currentstroke}{rgb}{0.000000,0.000000,0.000000}%
\pgfsetstrokecolor{currentstroke}%
\pgfsetstrokeopacity{0.000000}%
\pgfsetdash{}{0pt}%
\pgfpathmoveto{\pgfqpoint{0.849351in}{0.197819in}}%
\pgfpathlineto{\pgfqpoint{0.850276in}{0.194079in}}%
\pgfpathlineto{\pgfqpoint{0.852047in}{0.192882in}}%
\pgfpathlineto{\pgfqpoint{0.851725in}{0.189362in}}%
\pgfpathlineto{\pgfqpoint{0.849251in}{0.188177in}}%
\pgfpathlineto{\pgfqpoint{0.847242in}{0.193433in}}%
\pgfpathlineto{\pgfqpoint{0.849351in}{0.197819in}}%
\pgfpathclose%
\pgfusepath{fill}%
\end{pgfscope}%
\begin{pgfscope}%
\pgfpathrectangle{\pgfqpoint{0.100000in}{0.100000in}}{\pgfqpoint{2.989028in}{1.913466in}}%
\pgfusepath{clip}%
\pgfsetbuttcap%
\pgfsetmiterjoin%
\definecolor{currentfill}{rgb}{0.400000,0.760784,0.647059}%
\pgfsetfillcolor{currentfill}%
\pgfsetlinewidth{0.000000pt}%
\definecolor{currentstroke}{rgb}{0.000000,0.000000,0.000000}%
\pgfsetstrokecolor{currentstroke}%
\pgfsetstrokeopacity{0.000000}%
\pgfsetdash{}{0pt}%
\pgfpathmoveto{\pgfqpoint{0.844640in}{0.199254in}}%
\pgfpathlineto{\pgfqpoint{0.843104in}{0.194909in}}%
\pgfpathlineto{\pgfqpoint{0.840719in}{0.195069in}}%
\pgfpathlineto{\pgfqpoint{0.839215in}{0.198652in}}%
\pgfpathlineto{\pgfqpoint{0.841665in}{0.200349in}}%
\pgfpathlineto{\pgfqpoint{0.844640in}{0.199254in}}%
\pgfpathclose%
\pgfusepath{fill}%
\end{pgfscope}%
\begin{pgfscope}%
\pgfpathrectangle{\pgfqpoint{0.100000in}{0.100000in}}{\pgfqpoint{2.989028in}{1.913466in}}%
\pgfusepath{clip}%
\pgfsetbuttcap%
\pgfsetmiterjoin%
\definecolor{currentfill}{rgb}{0.400000,0.760784,0.647059}%
\pgfsetfillcolor{currentfill}%
\pgfsetlinewidth{0.000000pt}%
\definecolor{currentstroke}{rgb}{0.000000,0.000000,0.000000}%
\pgfsetstrokecolor{currentstroke}%
\pgfsetstrokeopacity{0.000000}%
\pgfsetdash{}{0pt}%
\pgfpathmoveto{\pgfqpoint{0.838205in}{0.176854in}}%
\pgfpathlineto{\pgfqpoint{0.840333in}{0.174527in}}%
\pgfpathlineto{\pgfqpoint{0.840802in}{0.169622in}}%
\pgfpathlineto{\pgfqpoint{0.841922in}{0.168197in}}%
\pgfpathlineto{\pgfqpoint{0.840092in}{0.163758in}}%
\pgfpathlineto{\pgfqpoint{0.837686in}{0.161260in}}%
\pgfpathlineto{\pgfqpoint{0.836405in}{0.156623in}}%
\pgfpathlineto{\pgfqpoint{0.833774in}{0.163848in}}%
\pgfpathlineto{\pgfqpoint{0.833338in}{0.170408in}}%
\pgfpathlineto{\pgfqpoint{0.831865in}{0.173673in}}%
\pgfpathlineto{\pgfqpoint{0.826795in}{0.176321in}}%
\pgfpathlineto{\pgfqpoint{0.831551in}{0.181841in}}%
\pgfpathlineto{\pgfqpoint{0.828380in}{0.184201in}}%
\pgfpathlineto{\pgfqpoint{0.831174in}{0.186552in}}%
\pgfpathlineto{\pgfqpoint{0.834882in}{0.195579in}}%
\pgfpathlineto{\pgfqpoint{0.831038in}{0.198762in}}%
\pgfpathlineto{\pgfqpoint{0.832519in}{0.201818in}}%
\pgfpathlineto{\pgfqpoint{0.837487in}{0.199107in}}%
\pgfpathlineto{\pgfqpoint{0.837990in}{0.194829in}}%
\pgfpathlineto{\pgfqpoint{0.836111in}{0.192394in}}%
\pgfpathlineto{\pgfqpoint{0.839836in}{0.189098in}}%
\pgfpathlineto{\pgfqpoint{0.841009in}{0.183965in}}%
\pgfpathlineto{\pgfqpoint{0.840850in}{0.176765in}}%
\pgfpathlineto{\pgfqpoint{0.838205in}{0.176854in}}%
\pgfpathclose%
\pgfusepath{fill}%
\end{pgfscope}%
\begin{pgfscope}%
\pgfpathrectangle{\pgfqpoint{0.100000in}{0.100000in}}{\pgfqpoint{2.989028in}{1.913466in}}%
\pgfusepath{clip}%
\pgfsetbuttcap%
\pgfsetmiterjoin%
\definecolor{currentfill}{rgb}{0.400000,0.760784,0.647059}%
\pgfsetfillcolor{currentfill}%
\pgfsetlinewidth{0.000000pt}%
\definecolor{currentstroke}{rgb}{0.000000,0.000000,0.000000}%
\pgfsetstrokecolor{currentstroke}%
\pgfsetstrokeopacity{0.000000}%
\pgfsetdash{}{0pt}%
\pgfpathmoveto{\pgfqpoint{0.847302in}{0.194926in}}%
\pgfpathlineto{\pgfqpoint{0.846143in}{0.192429in}}%
\pgfpathlineto{\pgfqpoint{0.847231in}{0.188374in}}%
\pgfpathlineto{\pgfqpoint{0.844445in}{0.188033in}}%
\pgfpathlineto{\pgfqpoint{0.841524in}{0.189911in}}%
\pgfpathlineto{\pgfqpoint{0.842472in}{0.194046in}}%
\pgfpathlineto{\pgfqpoint{0.847302in}{0.194926in}}%
\pgfpathclose%
\pgfusepath{fill}%
\end{pgfscope}%
\begin{pgfscope}%
\pgfpathrectangle{\pgfqpoint{0.100000in}{0.100000in}}{\pgfqpoint{2.989028in}{1.913466in}}%
\pgfusepath{clip}%
\pgfsetbuttcap%
\pgfsetmiterjoin%
\definecolor{currentfill}{rgb}{0.400000,0.760784,0.647059}%
\pgfsetfillcolor{currentfill}%
\pgfsetlinewidth{0.000000pt}%
\definecolor{currentstroke}{rgb}{0.000000,0.000000,0.000000}%
\pgfsetstrokecolor{currentstroke}%
\pgfsetstrokeopacity{0.000000}%
\pgfsetdash{}{0pt}%
\pgfpathmoveto{\pgfqpoint{0.834316in}{0.195209in}}%
\pgfpathlineto{\pgfqpoint{0.828167in}{0.192451in}}%
\pgfpathlineto{\pgfqpoint{0.826076in}{0.193444in}}%
\pgfpathlineto{\pgfqpoint{0.830270in}{0.197082in}}%
\pgfpathlineto{\pgfqpoint{0.834316in}{0.195209in}}%
\pgfpathclose%
\pgfusepath{fill}%
\end{pgfscope}%
\begin{pgfscope}%
\pgfpathrectangle{\pgfqpoint{0.100000in}{0.100000in}}{\pgfqpoint{2.989028in}{1.913466in}}%
\pgfusepath{clip}%
\pgfsetbuttcap%
\pgfsetmiterjoin%
\definecolor{currentfill}{rgb}{0.400000,0.760784,0.647059}%
\pgfsetfillcolor{currentfill}%
\pgfsetlinewidth{0.000000pt}%
\definecolor{currentstroke}{rgb}{0.000000,0.000000,0.000000}%
\pgfsetstrokecolor{currentstroke}%
\pgfsetstrokeopacity{0.000000}%
\pgfsetdash{}{0pt}%
\pgfpathmoveto{\pgfqpoint{0.847369in}{0.168302in}}%
\pgfpathlineto{\pgfqpoint{0.845903in}{0.172455in}}%
\pgfpathlineto{\pgfqpoint{0.849004in}{0.173477in}}%
\pgfpathlineto{\pgfqpoint{0.849943in}{0.177955in}}%
\pgfpathlineto{\pgfqpoint{0.853303in}{0.177960in}}%
\pgfpathlineto{\pgfqpoint{0.853289in}{0.181079in}}%
\pgfpathlineto{\pgfqpoint{0.857959in}{0.180341in}}%
\pgfpathlineto{\pgfqpoint{0.858755in}{0.171887in}}%
\pgfpathlineto{\pgfqpoint{0.856046in}{0.166498in}}%
\pgfpathlineto{\pgfqpoint{0.852027in}{0.163131in}}%
\pgfpathlineto{\pgfqpoint{0.847369in}{0.168302in}}%
\pgfpathclose%
\pgfusepath{fill}%
\end{pgfscope}%
\begin{pgfscope}%
\pgfpathrectangle{\pgfqpoint{0.100000in}{0.100000in}}{\pgfqpoint{2.989028in}{1.913466in}}%
\pgfusepath{clip}%
\pgfsetbuttcap%
\pgfsetmiterjoin%
\definecolor{currentfill}{rgb}{0.400000,0.760784,0.647059}%
\pgfsetfillcolor{currentfill}%
\pgfsetlinewidth{0.000000pt}%
\definecolor{currentstroke}{rgb}{0.000000,0.000000,0.000000}%
\pgfsetstrokecolor{currentstroke}%
\pgfsetstrokeopacity{0.000000}%
\pgfsetdash{}{0pt}%
\pgfpathmoveto{\pgfqpoint{0.845553in}{0.171496in}}%
\pgfpathlineto{\pgfqpoint{0.846878in}{0.167564in}}%
\pgfpathlineto{\pgfqpoint{0.843612in}{0.167086in}}%
\pgfpathlineto{\pgfqpoint{0.845553in}{0.171496in}}%
\pgfpathclose%
\pgfusepath{fill}%
\end{pgfscope}%
\begin{pgfscope}%
\pgfpathrectangle{\pgfqpoint{0.100000in}{0.100000in}}{\pgfqpoint{2.989028in}{1.913466in}}%
\pgfusepath{clip}%
\pgfsetbuttcap%
\pgfsetmiterjoin%
\definecolor{currentfill}{rgb}{0.400000,0.760784,0.647059}%
\pgfsetfillcolor{currentfill}%
\pgfsetlinewidth{0.000000pt}%
\definecolor{currentstroke}{rgb}{0.000000,0.000000,0.000000}%
\pgfsetstrokecolor{currentstroke}%
\pgfsetstrokeopacity{0.000000}%
\pgfsetdash{}{0pt}%
\pgfpathmoveto{\pgfqpoint{0.848306in}{0.165713in}}%
\pgfpathlineto{\pgfqpoint{0.847851in}{0.160700in}}%
\pgfpathlineto{\pgfqpoint{0.844406in}{0.161116in}}%
\pgfpathlineto{\pgfqpoint{0.845520in}{0.163537in}}%
\pgfpathlineto{\pgfqpoint{0.848306in}{0.165713in}}%
\pgfpathclose%
\pgfusepath{fill}%
\end{pgfscope}%
\begin{pgfscope}%
\pgfsetbuttcap%
\pgfsetmiterjoin%
\definecolor{currentfill}{rgb}{1.000000,1.000000,1.000000}%
\pgfsetfillcolor{currentfill}%
\pgfsetlinewidth{1.003750pt}%
\definecolor{currentstroke}{rgb}{0.827451,0.827451,0.827451}%
\pgfsetstrokecolor{currentstroke}%
\pgfsetdash{}{0pt}%
\pgfpathmoveto{\pgfqpoint{1.015239in}{1.931644in}}%
\pgfpathlineto{\pgfqpoint{2.626989in}{1.931644in}}%
\pgfpathlineto{\pgfqpoint{2.626989in}{2.127644in}}%
\pgfpathlineto{\pgfqpoint{1.015239in}{2.127644in}}%
\pgfpathlineto{\pgfqpoint{1.015239in}{1.931644in}}%
\pgfpathclose%
\pgfusepath{stroke,fill}%
\end{pgfscope}%
\begin{pgfscope}%
\definecolor{textcolor}{rgb}{0.000000,0.000000,0.000000}%
\pgfsetstrokecolor{textcolor}%
\pgfsetfillcolor{textcolor}%
\pgftext[x=1.056489in,y=1.994331in,left,base]{\color{textcolor}\setmainfont{Lato}\rmfamily\fontsize{9.000000}{10.800000}\selectfont Quarterly Growth, 2023 Q2}%
\end{pgfscope}%
\begin{pgfscope}%
\pgfpathrectangle{\pgfqpoint{3.625000in}{0.100000in}}{\pgfqpoint{2.989028in}{1.913466in}}%
\pgfusepath{clip}%
\pgfsetbuttcap%
\pgfsetmiterjoin%
\definecolor{currentfill}{rgb}{0.793080,0.916494,0.618224}%
\pgfsetfillcolor{currentfill}%
\pgfsetlinewidth{0.000000pt}%
\definecolor{currentstroke}{rgb}{0.000000,0.000000,0.000000}%
\pgfsetstrokecolor{currentstroke}%
\pgfsetstrokeopacity{0.000000}%
\pgfsetdash{}{0pt}%
\pgfpathmoveto{\pgfqpoint{4.622679in}{0.548675in}}%
\pgfpathlineto{\pgfqpoint{4.614089in}{0.548815in}}%
\pgfpathlineto{\pgfqpoint{4.608242in}{0.555287in}}%
\pgfpathlineto{\pgfqpoint{4.604266in}{0.572473in}}%
\pgfpathlineto{\pgfqpoint{4.613506in}{0.574627in}}%
\pgfpathlineto{\pgfqpoint{4.622893in}{0.572311in}}%
\pgfpathlineto{\pgfqpoint{4.632446in}{0.565540in}}%
\pgfpathlineto{\pgfqpoint{4.632441in}{0.556177in}}%
\pgfpathlineto{\pgfqpoint{4.622679in}{0.548675in}}%
\pgfpathclose%
\pgfusepath{fill}%
\end{pgfscope}%
\begin{pgfscope}%
\pgfpathrectangle{\pgfqpoint{3.625000in}{0.100000in}}{\pgfqpoint{2.989028in}{1.913466in}}%
\pgfusepath{clip}%
\pgfsetbuttcap%
\pgfsetmiterjoin%
\definecolor{currentfill}{rgb}{0.793080,0.916494,0.618224}%
\pgfsetfillcolor{currentfill}%
\pgfsetlinewidth{0.000000pt}%
\definecolor{currentstroke}{rgb}{0.000000,0.000000,0.000000}%
\pgfsetstrokecolor{currentstroke}%
\pgfsetstrokeopacity{0.000000}%
\pgfsetdash{}{0pt}%
\pgfpathmoveto{\pgfqpoint{4.674970in}{0.446640in}}%
\pgfpathlineto{\pgfqpoint{4.665359in}{0.450300in}}%
\pgfpathlineto{\pgfqpoint{4.665315in}{0.457479in}}%
\pgfpathlineto{\pgfqpoint{4.653705in}{0.463549in}}%
\pgfpathlineto{\pgfqpoint{4.655119in}{0.478835in}}%
\pgfpathlineto{\pgfqpoint{4.658481in}{0.486048in}}%
\pgfpathlineto{\pgfqpoint{4.665581in}{0.479409in}}%
\pgfpathlineto{\pgfqpoint{4.677127in}{0.480518in}}%
\pgfpathlineto{\pgfqpoint{4.674211in}{0.459391in}}%
\pgfpathlineto{\pgfqpoint{4.674970in}{0.446640in}}%
\pgfpathclose%
\pgfusepath{fill}%
\end{pgfscope}%
\begin{pgfscope}%
\pgfpathrectangle{\pgfqpoint{3.625000in}{0.100000in}}{\pgfqpoint{2.989028in}{1.913466in}}%
\pgfusepath{clip}%
\pgfsetbuttcap%
\pgfsetmiterjoin%
\definecolor{currentfill}{rgb}{0.793080,0.916494,0.618224}%
\pgfsetfillcolor{currentfill}%
\pgfsetlinewidth{0.000000pt}%
\definecolor{currentstroke}{rgb}{0.000000,0.000000,0.000000}%
\pgfsetstrokecolor{currentstroke}%
\pgfsetstrokeopacity{0.000000}%
\pgfsetdash{}{0pt}%
\pgfpathmoveto{\pgfqpoint{4.715310in}{0.399517in}}%
\pgfpathlineto{\pgfqpoint{4.703114in}{0.401594in}}%
\pgfpathlineto{\pgfqpoint{4.696369in}{0.412035in}}%
\pgfpathlineto{\pgfqpoint{4.695417in}{0.420620in}}%
\pgfpathlineto{\pgfqpoint{4.715310in}{0.399517in}}%
\pgfpathclose%
\pgfusepath{fill}%
\end{pgfscope}%
\begin{pgfscope}%
\pgfpathrectangle{\pgfqpoint{3.625000in}{0.100000in}}{\pgfqpoint{2.989028in}{1.913466in}}%
\pgfusepath{clip}%
\pgfsetbuttcap%
\pgfsetmiterjoin%
\definecolor{currentfill}{rgb}{0.793080,0.916494,0.618224}%
\pgfsetfillcolor{currentfill}%
\pgfsetlinewidth{0.000000pt}%
\definecolor{currentstroke}{rgb}{0.000000,0.000000,0.000000}%
\pgfsetstrokecolor{currentstroke}%
\pgfsetstrokeopacity{0.000000}%
\pgfsetdash{}{0pt}%
\pgfpathmoveto{\pgfqpoint{4.690317in}{0.401338in}}%
\pgfpathlineto{\pgfqpoint{4.696711in}{0.396147in}}%
\pgfpathlineto{\pgfqpoint{4.698099in}{0.387222in}}%
\pgfpathlineto{\pgfqpoint{4.685227in}{0.387715in}}%
\pgfpathlineto{\pgfqpoint{4.690317in}{0.401338in}}%
\pgfpathclose%
\pgfusepath{fill}%
\end{pgfscope}%
\begin{pgfscope}%
\pgfpathrectangle{\pgfqpoint{3.625000in}{0.100000in}}{\pgfqpoint{2.989028in}{1.913466in}}%
\pgfusepath{clip}%
\pgfsetbuttcap%
\pgfsetmiterjoin%
\definecolor{currentfill}{rgb}{0.793080,0.916494,0.618224}%
\pgfsetfillcolor{currentfill}%
\pgfsetlinewidth{0.000000pt}%
\definecolor{currentstroke}{rgb}{0.000000,0.000000,0.000000}%
\pgfsetstrokecolor{currentstroke}%
\pgfsetstrokeopacity{0.000000}%
\pgfsetdash{}{0pt}%
\pgfpathmoveto{\pgfqpoint{4.719683in}{0.351485in}}%
\pgfpathlineto{\pgfqpoint{4.706837in}{0.356103in}}%
\pgfpathlineto{\pgfqpoint{4.710518in}{0.367298in}}%
\pgfpathlineto{\pgfqpoint{4.705134in}{0.378705in}}%
\pgfpathlineto{\pgfqpoint{4.706825in}{0.386546in}}%
\pgfpathlineto{\pgfqpoint{4.717162in}{0.389535in}}%
\pgfpathlineto{\pgfqpoint{4.716963in}{0.376972in}}%
\pgfpathlineto{\pgfqpoint{4.729190in}{0.370003in}}%
\pgfpathlineto{\pgfqpoint{4.736183in}{0.353155in}}%
\pgfpathlineto{\pgfqpoint{4.728389in}{0.346667in}}%
\pgfpathlineto{\pgfqpoint{4.719683in}{0.351485in}}%
\pgfpathclose%
\pgfusepath{fill}%
\end{pgfscope}%
\begin{pgfscope}%
\pgfpathrectangle{\pgfqpoint{3.625000in}{0.100000in}}{\pgfqpoint{2.989028in}{1.913466in}}%
\pgfusepath{clip}%
\pgfsetbuttcap%
\pgfsetmiterjoin%
\definecolor{currentfill}{rgb}{0.793080,0.916494,0.618224}%
\pgfsetfillcolor{currentfill}%
\pgfsetlinewidth{0.000000pt}%
\definecolor{currentstroke}{rgb}{0.000000,0.000000,0.000000}%
\pgfsetstrokecolor{currentstroke}%
\pgfsetstrokeopacity{0.000000}%
\pgfsetdash{}{0pt}%
\pgfpathmoveto{\pgfqpoint{4.673289in}{0.237569in}}%
\pgfpathlineto{\pgfqpoint{4.667169in}{0.250211in}}%
\pgfpathlineto{\pgfqpoint{4.669879in}{0.258715in}}%
\pgfpathlineto{\pgfqpoint{4.681883in}{0.270485in}}%
\pgfpathlineto{\pgfqpoint{4.683019in}{0.279508in}}%
\pgfpathlineto{\pgfqpoint{4.690875in}{0.299484in}}%
\pgfpathlineto{\pgfqpoint{4.711536in}{0.302897in}}%
\pgfpathlineto{\pgfqpoint{4.715491in}{0.314590in}}%
\pgfpathlineto{\pgfqpoint{4.724600in}{0.310205in}}%
\pgfpathlineto{\pgfqpoint{4.743110in}{0.277388in}}%
\pgfpathlineto{\pgfqpoint{4.743344in}{0.266674in}}%
\pgfpathlineto{\pgfqpoint{4.738112in}{0.257400in}}%
\pgfpathlineto{\pgfqpoint{4.744750in}{0.236108in}}%
\pgfpathlineto{\pgfqpoint{4.740111in}{0.232676in}}%
\pgfpathlineto{\pgfqpoint{4.724311in}{0.233674in}}%
\pgfpathlineto{\pgfqpoint{4.705311in}{0.240024in}}%
\pgfpathlineto{\pgfqpoint{4.689342in}{0.242646in}}%
\pgfpathlineto{\pgfqpoint{4.685076in}{0.238921in}}%
\pgfpathlineto{\pgfqpoint{4.673289in}{0.237569in}}%
\pgfpathclose%
\pgfusepath{fill}%
\end{pgfscope}%
\begin{pgfscope}%
\pgfpathrectangle{\pgfqpoint{3.625000in}{0.100000in}}{\pgfqpoint{2.989028in}{1.913466in}}%
\pgfusepath{clip}%
\pgfsetbuttcap%
\pgfsetmiterjoin%
\definecolor{currentfill}{rgb}{0.344022,0.698347,0.672895}%
\pgfsetfillcolor{currentfill}%
\pgfsetlinewidth{0.000000pt}%
\definecolor{currentstroke}{rgb}{0.000000,0.000000,0.000000}%
\pgfsetstrokecolor{currentstroke}%
\pgfsetstrokeopacity{0.000000}%
\pgfsetdash{}{0pt}%
\pgfpathmoveto{\pgfqpoint{4.335222in}{1.900931in}}%
\pgfpathlineto{\pgfqpoint{4.319722in}{1.839512in}}%
\pgfpathlineto{\pgfqpoint{4.308700in}{1.795504in}}%
\pgfpathlineto{\pgfqpoint{4.295595in}{1.742564in}}%
\pgfpathlineto{\pgfqpoint{4.293022in}{1.729516in}}%
\pgfpathlineto{\pgfqpoint{4.294703in}{1.717547in}}%
\pgfpathlineto{\pgfqpoint{4.292475in}{1.706523in}}%
\pgfpathlineto{\pgfqpoint{4.200998in}{1.730394in}}%
\pgfpathlineto{\pgfqpoint{4.192734in}{1.727622in}}%
\pgfpathlineto{\pgfqpoint{4.185663in}{1.730028in}}%
\pgfpathlineto{\pgfqpoint{4.152921in}{1.730814in}}%
\pgfpathlineto{\pgfqpoint{4.141848in}{1.727692in}}%
\pgfpathlineto{\pgfqpoint{4.130851in}{1.728747in}}%
\pgfpathlineto{\pgfqpoint{4.126190in}{1.733634in}}%
\pgfpathlineto{\pgfqpoint{4.096877in}{1.732568in}}%
\pgfpathlineto{\pgfqpoint{4.092248in}{1.740554in}}%
\pgfpathlineto{\pgfqpoint{4.083460in}{1.744762in}}%
\pgfpathlineto{\pgfqpoint{4.070673in}{1.747325in}}%
\pgfpathlineto{\pgfqpoint{4.048603in}{1.743571in}}%
\pgfpathlineto{\pgfqpoint{4.040456in}{1.747249in}}%
\pgfpathlineto{\pgfqpoint{4.027895in}{1.757031in}}%
\pgfpathlineto{\pgfqpoint{4.031670in}{1.776202in}}%
\pgfpathlineto{\pgfqpoint{4.030347in}{1.785992in}}%
\pgfpathlineto{\pgfqpoint{4.020391in}{1.796386in}}%
\pgfpathlineto{\pgfqpoint{4.014015in}{1.795741in}}%
\pgfpathlineto{\pgfqpoint{4.009440in}{1.806255in}}%
\pgfpathlineto{\pgfqpoint{3.998604in}{1.810504in}}%
\pgfpathlineto{\pgfqpoint{3.990727in}{1.809963in}}%
\pgfpathlineto{\pgfqpoint{3.988179in}{1.820760in}}%
\pgfpathlineto{\pgfqpoint{3.996026in}{1.819611in}}%
\pgfpathlineto{\pgfqpoint{3.995402in}{1.834619in}}%
\pgfpathlineto{\pgfqpoint{3.991982in}{1.843530in}}%
\pgfpathlineto{\pgfqpoint{3.993490in}{1.854935in}}%
\pgfpathlineto{\pgfqpoint{3.997595in}{1.863514in}}%
\pgfpathlineto{\pgfqpoint{3.997364in}{1.879792in}}%
\pgfpathlineto{\pgfqpoint{3.995172in}{1.885699in}}%
\pgfpathlineto{\pgfqpoint{3.998957in}{1.904648in}}%
\pgfpathlineto{\pgfqpoint{3.997865in}{1.916860in}}%
\pgfpathlineto{\pgfqpoint{3.994081in}{1.922716in}}%
\pgfpathlineto{\pgfqpoint{3.994658in}{1.941861in}}%
\pgfpathlineto{\pgfqpoint{4.000162in}{1.955941in}}%
\pgfpathlineto{\pgfqpoint{4.006149in}{1.952495in}}%
\pgfpathlineto{\pgfqpoint{4.025972in}{1.932018in}}%
\pgfpathlineto{\pgfqpoint{4.049993in}{1.920812in}}%
\pgfpathlineto{\pgfqpoint{4.062292in}{1.919486in}}%
\pgfpathlineto{\pgfqpoint{4.073880in}{1.914715in}}%
\pgfpathlineto{\pgfqpoint{4.077168in}{1.898661in}}%
\pgfpathlineto{\pgfqpoint{4.067014in}{1.895620in}}%
\pgfpathlineto{\pgfqpoint{4.060285in}{1.882967in}}%
\pgfpathlineto{\pgfqpoint{4.068185in}{1.883670in}}%
\pgfpathlineto{\pgfqpoint{4.071372in}{1.889349in}}%
\pgfpathlineto{\pgfqpoint{4.082514in}{1.896451in}}%
\pgfpathlineto{\pgfqpoint{4.081939in}{1.885905in}}%
\pgfpathlineto{\pgfqpoint{4.074499in}{1.884204in}}%
\pgfpathlineto{\pgfqpoint{4.075734in}{1.870648in}}%
\pgfpathlineto{\pgfqpoint{4.066674in}{1.857069in}}%
\pgfpathlineto{\pgfqpoint{4.060876in}{1.863526in}}%
\pgfpathlineto{\pgfqpoint{4.043645in}{1.860015in}}%
\pgfpathlineto{\pgfqpoint{4.042738in}{1.852116in}}%
\pgfpathlineto{\pgfqpoint{4.048678in}{1.847303in}}%
\pgfpathlineto{\pgfqpoint{4.057741in}{1.846830in}}%
\pgfpathlineto{\pgfqpoint{4.070271in}{1.857114in}}%
\pgfpathlineto{\pgfqpoint{4.080189in}{1.857968in}}%
\pgfpathlineto{\pgfqpoint{4.080569in}{1.869352in}}%
\pgfpathlineto{\pgfqpoint{4.085680in}{1.886072in}}%
\pgfpathlineto{\pgfqpoint{4.097793in}{1.898506in}}%
\pgfpathlineto{\pgfqpoint{4.093754in}{1.908138in}}%
\pgfpathlineto{\pgfqpoint{4.096513in}{1.918507in}}%
\pgfpathlineto{\pgfqpoint{4.091139in}{1.928842in}}%
\pgfpathlineto{\pgfqpoint{4.098798in}{1.942057in}}%
\pgfpathlineto{\pgfqpoint{4.099945in}{1.950008in}}%
\pgfpathlineto{\pgfqpoint{4.093296in}{1.955234in}}%
\pgfpathlineto{\pgfqpoint{4.094403in}{1.968600in}}%
\pgfpathlineto{\pgfqpoint{4.174072in}{1.944395in}}%
\pgfpathlineto{\pgfqpoint{4.258673in}{1.920671in}}%
\pgfpathlineto{\pgfqpoint{4.335222in}{1.900931in}}%
\pgfpathclose%
\pgfusepath{fill}%
\end{pgfscope}%
\begin{pgfscope}%
\pgfpathrectangle{\pgfqpoint{3.625000in}{0.100000in}}{\pgfqpoint{2.989028in}{1.913466in}}%
\pgfusepath{clip}%
\pgfsetbuttcap%
\pgfsetmiterjoin%
\definecolor{currentfill}{rgb}{0.344022,0.698347,0.672895}%
\pgfsetfillcolor{currentfill}%
\pgfsetlinewidth{0.000000pt}%
\definecolor{currentstroke}{rgb}{0.000000,0.000000,0.000000}%
\pgfsetstrokecolor{currentstroke}%
\pgfsetstrokeopacity{0.000000}%
\pgfsetdash{}{0pt}%
\pgfpathmoveto{\pgfqpoint{4.081297in}{1.922359in}}%
\pgfpathlineto{\pgfqpoint{4.085220in}{1.907235in}}%
\pgfpathlineto{\pgfqpoint{4.091372in}{1.902211in}}%
\pgfpathlineto{\pgfqpoint{4.086463in}{1.895816in}}%
\pgfpathlineto{\pgfqpoint{4.081679in}{1.905186in}}%
\pgfpathlineto{\pgfqpoint{4.081297in}{1.922359in}}%
\pgfpathclose%
\pgfusepath{fill}%
\end{pgfscope}%
\begin{pgfscope}%
\pgfpathrectangle{\pgfqpoint{3.625000in}{0.100000in}}{\pgfqpoint{2.989028in}{1.913466in}}%
\pgfusepath{clip}%
\pgfsetbuttcap%
\pgfsetmiterjoin%
\definecolor{currentfill}{rgb}{0.847520,0.938639,0.607151}%
\pgfsetfillcolor{currentfill}%
\pgfsetlinewidth{0.000000pt}%
\definecolor{currentstroke}{rgb}{0.000000,0.000000,0.000000}%
\pgfsetstrokecolor{currentstroke}%
\pgfsetstrokeopacity{0.000000}%
\pgfsetdash{}{0pt}%
\pgfpathmoveto{\pgfqpoint{4.376305in}{1.891004in}}%
\pgfpathlineto{\pgfqpoint{4.461467in}{1.871893in}}%
\pgfpathlineto{\pgfqpoint{4.541670in}{1.855681in}}%
\pgfpathlineto{\pgfqpoint{4.603378in}{1.844375in}}%
\pgfpathlineto{\pgfqpoint{4.657175in}{1.835337in}}%
\pgfpathlineto{\pgfqpoint{4.711092in}{1.827040in}}%
\pgfpathlineto{\pgfqpoint{4.757007in}{1.820572in}}%
\pgfpathlineto{\pgfqpoint{4.802995in}{1.814641in}}%
\pgfpathlineto{\pgfqpoint{4.849048in}{1.809248in}}%
\pgfpathlineto{\pgfqpoint{4.892448in}{1.804664in}}%
\pgfpathlineto{\pgfqpoint{4.886479in}{1.738522in}}%
\pgfpathlineto{\pgfqpoint{4.877595in}{1.649140in}}%
\pgfpathlineto{\pgfqpoint{4.872936in}{1.603169in}}%
\pgfpathlineto{\pgfqpoint{4.866234in}{1.541215in}}%
\pgfpathlineto{\pgfqpoint{4.818794in}{1.546395in}}%
\pgfpathlineto{\pgfqpoint{4.764479in}{1.552561in}}%
\pgfpathlineto{\pgfqpoint{4.689064in}{1.562882in}}%
\pgfpathlineto{\pgfqpoint{4.655381in}{1.567687in}}%
\pgfpathlineto{\pgfqpoint{4.572396in}{1.580713in}}%
\pgfpathlineto{\pgfqpoint{4.543829in}{1.585939in}}%
\pgfpathlineto{\pgfqpoint{4.537875in}{1.552087in}}%
\pgfpathlineto{\pgfqpoint{4.534620in}{1.554502in}}%
\pgfpathlineto{\pgfqpoint{4.528502in}{1.570788in}}%
\pgfpathlineto{\pgfqpoint{4.520639in}{1.569785in}}%
\pgfpathlineto{\pgfqpoint{4.515011in}{1.561552in}}%
\pgfpathlineto{\pgfqpoint{4.503331in}{1.560451in}}%
\pgfpathlineto{\pgfqpoint{4.500984in}{1.564390in}}%
\pgfpathlineto{\pgfqpoint{4.490043in}{1.563913in}}%
\pgfpathlineto{\pgfqpoint{4.484543in}{1.567727in}}%
\pgfpathlineto{\pgfqpoint{4.476852in}{1.561826in}}%
\pgfpathlineto{\pgfqpoint{4.458172in}{1.567136in}}%
\pgfpathlineto{\pgfqpoint{4.450020in}{1.564285in}}%
\pgfpathlineto{\pgfqpoint{4.445634in}{1.576938in}}%
\pgfpathlineto{\pgfqpoint{4.445429in}{1.588978in}}%
\pgfpathlineto{\pgfqpoint{4.432552in}{1.597637in}}%
\pgfpathlineto{\pgfqpoint{4.434652in}{1.610116in}}%
\pgfpathlineto{\pgfqpoint{4.425483in}{1.630729in}}%
\pgfpathlineto{\pgfqpoint{4.426101in}{1.648251in}}%
\pgfpathlineto{\pgfqpoint{4.418220in}{1.656827in}}%
\pgfpathlineto{\pgfqpoint{4.410827in}{1.649240in}}%
\pgfpathlineto{\pgfqpoint{4.400763in}{1.645136in}}%
\pgfpathlineto{\pgfqpoint{4.393102in}{1.653594in}}%
\pgfpathlineto{\pgfqpoint{4.394348in}{1.667101in}}%
\pgfpathlineto{\pgfqpoint{4.403535in}{1.672422in}}%
\pgfpathlineto{\pgfqpoint{4.401093in}{1.680980in}}%
\pgfpathlineto{\pgfqpoint{4.417063in}{1.722399in}}%
\pgfpathlineto{\pgfqpoint{4.404470in}{1.723437in}}%
\pgfpathlineto{\pgfqpoint{4.403176in}{1.730870in}}%
\pgfpathlineto{\pgfqpoint{4.394003in}{1.737279in}}%
\pgfpathlineto{\pgfqpoint{4.394589in}{1.744445in}}%
\pgfpathlineto{\pgfqpoint{4.389750in}{1.750011in}}%
\pgfpathlineto{\pgfqpoint{4.381873in}{1.770222in}}%
\pgfpathlineto{\pgfqpoint{4.374675in}{1.774343in}}%
\pgfpathlineto{\pgfqpoint{4.369521in}{1.791786in}}%
\pgfpathlineto{\pgfqpoint{4.370737in}{1.803380in}}%
\pgfpathlineto{\pgfqpoint{4.361121in}{1.824733in}}%
\pgfpathlineto{\pgfqpoint{4.376305in}{1.891004in}}%
\pgfpathclose%
\pgfusepath{fill}%
\end{pgfscope}%
\begin{pgfscope}%
\pgfpathrectangle{\pgfqpoint{3.625000in}{0.100000in}}{\pgfqpoint{2.989028in}{1.913466in}}%
\pgfusepath{clip}%
\pgfsetbuttcap%
\pgfsetmiterjoin%
\definecolor{currentfill}{rgb}{0.932718,0.973087,0.644060}%
\pgfsetfillcolor{currentfill}%
\pgfsetlinewidth{0.000000pt}%
\definecolor{currentstroke}{rgb}{0.000000,0.000000,0.000000}%
\pgfsetstrokecolor{currentstroke}%
\pgfsetstrokeopacity{0.000000}%
\pgfsetdash{}{0pt}%
\pgfpathmoveto{\pgfqpoint{6.432175in}{1.550163in}}%
\pgfpathlineto{\pgfqpoint{6.430471in}{1.557411in}}%
\pgfpathlineto{\pgfqpoint{6.420998in}{1.563760in}}%
\pgfpathlineto{\pgfqpoint{6.395964in}{1.645645in}}%
\pgfpathlineto{\pgfqpoint{6.382471in}{1.685161in}}%
\pgfpathlineto{\pgfqpoint{6.393319in}{1.696546in}}%
\pgfpathlineto{\pgfqpoint{6.403864in}{1.724407in}}%
\pgfpathlineto{\pgfqpoint{6.409123in}{1.733311in}}%
\pgfpathlineto{\pgfqpoint{6.405397in}{1.737050in}}%
\pgfpathlineto{\pgfqpoint{6.404149in}{1.761782in}}%
\pgfpathlineto{\pgfqpoint{6.408854in}{1.769364in}}%
\pgfpathlineto{\pgfqpoint{6.406792in}{1.786986in}}%
\pgfpathlineto{\pgfqpoint{6.425570in}{1.844804in}}%
\pgfpathlineto{\pgfqpoint{6.434031in}{1.845120in}}%
\pgfpathlineto{\pgfqpoint{6.437543in}{1.834792in}}%
\pgfpathlineto{\pgfqpoint{6.445062in}{1.831826in}}%
\pgfpathlineto{\pgfqpoint{6.459266in}{1.843974in}}%
\pgfpathlineto{\pgfqpoint{6.470403in}{1.851149in}}%
\pgfpathlineto{\pgfqpoint{6.494943in}{1.838514in}}%
\pgfpathlineto{\pgfqpoint{6.517113in}{1.768360in}}%
\pgfpathlineto{\pgfqpoint{6.521328in}{1.751101in}}%
\pgfpathlineto{\pgfqpoint{6.531039in}{1.749076in}}%
\pgfpathlineto{\pgfqpoint{6.544209in}{1.737354in}}%
\pgfpathlineto{\pgfqpoint{6.543463in}{1.730524in}}%
\pgfpathlineto{\pgfqpoint{6.552435in}{1.722428in}}%
\pgfpathlineto{\pgfqpoint{6.559723in}{1.727073in}}%
\pgfpathlineto{\pgfqpoint{6.574974in}{1.709287in}}%
\pgfpathlineto{\pgfqpoint{6.568158in}{1.695053in}}%
\pgfpathlineto{\pgfqpoint{6.559018in}{1.694771in}}%
\pgfpathlineto{\pgfqpoint{6.551685in}{1.682065in}}%
\pgfpathlineto{\pgfqpoint{6.542802in}{1.680263in}}%
\pgfpathlineto{\pgfqpoint{6.536280in}{1.673497in}}%
\pgfpathlineto{\pgfqpoint{6.516803in}{1.666241in}}%
\pgfpathlineto{\pgfqpoint{6.505014in}{1.654548in}}%
\pgfpathlineto{\pgfqpoint{6.499025in}{1.662941in}}%
\pgfpathlineto{\pgfqpoint{6.493584in}{1.656926in}}%
\pgfpathlineto{\pgfqpoint{6.495074in}{1.632607in}}%
\pgfpathlineto{\pgfqpoint{6.490765in}{1.622974in}}%
\pgfpathlineto{\pgfqpoint{6.481367in}{1.625605in}}%
\pgfpathlineto{\pgfqpoint{6.479850in}{1.615730in}}%
\pgfpathlineto{\pgfqpoint{6.475789in}{1.611669in}}%
\pgfpathlineto{\pgfqpoint{6.467662in}{1.612659in}}%
\pgfpathlineto{\pgfqpoint{6.468174in}{1.603425in}}%
\pgfpathlineto{\pgfqpoint{6.455855in}{1.606048in}}%
\pgfpathlineto{\pgfqpoint{6.449128in}{1.593255in}}%
\pgfpathlineto{\pgfqpoint{6.451668in}{1.586561in}}%
\pgfpathlineto{\pgfqpoint{6.447666in}{1.575429in}}%
\pgfpathlineto{\pgfqpoint{6.441367in}{1.567240in}}%
\pgfpathlineto{\pgfqpoint{6.439781in}{1.550146in}}%
\pgfpathlineto{\pgfqpoint{6.432175in}{1.550163in}}%
\pgfpathclose%
\pgfusepath{fill}%
\end{pgfscope}%
\begin{pgfscope}%
\pgfpathrectangle{\pgfqpoint{3.625000in}{0.100000in}}{\pgfqpoint{2.989028in}{1.913466in}}%
\pgfusepath{clip}%
\pgfsetbuttcap%
\pgfsetmiterjoin%
\definecolor{currentfill}{rgb}{0.932718,0.973087,0.644060}%
\pgfsetfillcolor{currentfill}%
\pgfsetlinewidth{0.000000pt}%
\definecolor{currentstroke}{rgb}{0.000000,0.000000,0.000000}%
\pgfsetstrokecolor{currentstroke}%
\pgfsetstrokeopacity{0.000000}%
\pgfsetdash{}{0pt}%
\pgfpathmoveto{\pgfqpoint{6.520301in}{1.661247in}}%
\pgfpathlineto{\pgfqpoint{6.525863in}{1.667073in}}%
\pgfpathlineto{\pgfqpoint{6.531158in}{1.661597in}}%
\pgfpathlineto{\pgfqpoint{6.521647in}{1.654342in}}%
\pgfpathlineto{\pgfqpoint{6.520301in}{1.661247in}}%
\pgfpathclose%
\pgfusepath{fill}%
\end{pgfscope}%
\begin{pgfscope}%
\pgfpathrectangle{\pgfqpoint{3.625000in}{0.100000in}}{\pgfqpoint{2.989028in}{1.913466in}}%
\pgfusepath{clip}%
\pgfsetbuttcap%
\pgfsetmiterjoin%
\definecolor{currentfill}{rgb}{0.246828,0.467589,0.710035}%
\pgfsetfillcolor{currentfill}%
\pgfsetlinewidth{0.000000pt}%
\definecolor{currentstroke}{rgb}{0.000000,0.000000,0.000000}%
\pgfsetstrokecolor{currentstroke}%
\pgfsetstrokeopacity{0.000000}%
\pgfsetdash{}{0pt}%
\pgfpathmoveto{\pgfqpoint{4.872936in}{1.603169in}}%
\pgfpathlineto{\pgfqpoint{4.877595in}{1.649140in}}%
\pgfpathlineto{\pgfqpoint{4.886479in}{1.738522in}}%
\pgfpathlineto{\pgfqpoint{4.892448in}{1.804664in}}%
\pgfpathlineto{\pgfqpoint{4.941329in}{1.800078in}}%
\pgfpathlineto{\pgfqpoint{5.003862in}{1.795100in}}%
\pgfpathlineto{\pgfqpoint{5.061021in}{1.791418in}}%
\pgfpathlineto{\pgfqpoint{5.112777in}{1.788799in}}%
\pgfpathlineto{\pgfqpoint{5.190014in}{1.786150in}}%
\pgfpathlineto{\pgfqpoint{5.195285in}{1.764381in}}%
\pgfpathlineto{\pgfqpoint{5.193398in}{1.753978in}}%
\pgfpathlineto{\pgfqpoint{5.192777in}{1.732596in}}%
\pgfpathlineto{\pgfqpoint{5.196363in}{1.716582in}}%
\pgfpathlineto{\pgfqpoint{5.204579in}{1.692956in}}%
\pgfpathlineto{\pgfqpoint{5.204560in}{1.663212in}}%
\pgfpathlineto{\pgfqpoint{5.206080in}{1.628658in}}%
\pgfpathlineto{\pgfqpoint{5.208167in}{1.619346in}}%
\pgfpathlineto{\pgfqpoint{5.214274in}{1.609122in}}%
\pgfpathlineto{\pgfqpoint{5.216288in}{1.593192in}}%
\pgfpathlineto{\pgfqpoint{5.215419in}{1.582557in}}%
\pgfpathlineto{\pgfqpoint{5.150668in}{1.583959in}}%
\pgfpathlineto{\pgfqpoint{5.103564in}{1.586347in}}%
\pgfpathlineto{\pgfqpoint{5.034511in}{1.590023in}}%
\pgfpathlineto{\pgfqpoint{4.966415in}{1.594860in}}%
\pgfpathlineto{\pgfqpoint{4.921059in}{1.598519in}}%
\pgfpathlineto{\pgfqpoint{4.872936in}{1.603169in}}%
\pgfpathclose%
\pgfusepath{fill}%
\end{pgfscope}%
\begin{pgfscope}%
\pgfpathrectangle{\pgfqpoint{3.625000in}{0.100000in}}{\pgfqpoint{2.989028in}{1.913466in}}%
\pgfusepath{clip}%
\pgfsetbuttcap%
\pgfsetmiterjoin%
\definecolor{currentfill}{rgb}{0.905805,0.962322,0.602076}%
\pgfsetfillcolor{currentfill}%
\pgfsetlinewidth{0.000000pt}%
\definecolor{currentstroke}{rgb}{0.000000,0.000000,0.000000}%
\pgfsetstrokecolor{currentstroke}%
\pgfsetstrokeopacity{0.000000}%
\pgfsetdash{}{0pt}%
\pgfpathmoveto{\pgfqpoint{4.853363in}{1.410780in}}%
\pgfpathlineto{\pgfqpoint{4.860839in}{1.487840in}}%
\pgfpathlineto{\pgfqpoint{4.866234in}{1.541215in}}%
\pgfpathlineto{\pgfqpoint{4.872936in}{1.603169in}}%
\pgfpathlineto{\pgfqpoint{4.921059in}{1.598519in}}%
\pgfpathlineto{\pgfqpoint{4.966415in}{1.594860in}}%
\pgfpathlineto{\pgfqpoint{5.034511in}{1.590023in}}%
\pgfpathlineto{\pgfqpoint{5.103564in}{1.586347in}}%
\pgfpathlineto{\pgfqpoint{5.150668in}{1.583959in}}%
\pgfpathlineto{\pgfqpoint{5.215419in}{1.582557in}}%
\pgfpathlineto{\pgfqpoint{5.211034in}{1.569764in}}%
\pgfpathlineto{\pgfqpoint{5.202284in}{1.559721in}}%
\pgfpathlineto{\pgfqpoint{5.208985in}{1.548158in}}%
\pgfpathlineto{\pgfqpoint{5.216374in}{1.545688in}}%
\pgfpathlineto{\pgfqpoint{5.219880in}{1.539046in}}%
\pgfpathlineto{\pgfqpoint{5.219075in}{1.490552in}}%
\pgfpathlineto{\pgfqpoint{5.217728in}{1.422304in}}%
\pgfpathlineto{\pgfqpoint{5.212744in}{1.404379in}}%
\pgfpathlineto{\pgfqpoint{5.217201in}{1.394464in}}%
\pgfpathlineto{\pgfqpoint{5.208252in}{1.373157in}}%
\pgfpathlineto{\pgfqpoint{5.217680in}{1.355981in}}%
\pgfpathlineto{\pgfqpoint{5.209670in}{1.357295in}}%
\pgfpathlineto{\pgfqpoint{5.204205in}{1.367982in}}%
\pgfpathlineto{\pgfqpoint{5.184700in}{1.375307in}}%
\pgfpathlineto{\pgfqpoint{5.172400in}{1.381750in}}%
\pgfpathlineto{\pgfqpoint{5.151759in}{1.382286in}}%
\pgfpathlineto{\pgfqpoint{5.144594in}{1.376416in}}%
\pgfpathlineto{\pgfqpoint{5.121231in}{1.387975in}}%
\pgfpathlineto{\pgfqpoint{5.119439in}{1.391630in}}%
\pgfpathlineto{\pgfqpoint{5.037902in}{1.395489in}}%
\pgfpathlineto{\pgfqpoint{4.988365in}{1.398329in}}%
\pgfpathlineto{\pgfqpoint{4.947447in}{1.401533in}}%
\pgfpathlineto{\pgfqpoint{4.879836in}{1.407921in}}%
\pgfpathlineto{\pgfqpoint{4.853363in}{1.410780in}}%
\pgfpathclose%
\pgfusepath{fill}%
\end{pgfscope}%
\begin{pgfscope}%
\pgfpathrectangle{\pgfqpoint{3.625000in}{0.100000in}}{\pgfqpoint{2.989028in}{1.913466in}}%
\pgfusepath{clip}%
\pgfsetbuttcap%
\pgfsetmiterjoin%
\definecolor{currentfill}{rgb}{0.280661,0.423760,0.689273}%
\pgfsetfillcolor{currentfill}%
\pgfsetlinewidth{0.000000pt}%
\definecolor{currentstroke}{rgb}{0.000000,0.000000,0.000000}%
\pgfsetstrokecolor{currentstroke}%
\pgfsetstrokeopacity{0.000000}%
\pgfsetdash{}{0pt}%
\pgfpathmoveto{\pgfqpoint{4.840540in}{1.280295in}}%
\pgfpathlineto{\pgfqpoint{4.797072in}{1.284265in}}%
\pgfpathlineto{\pgfqpoint{4.702321in}{1.295959in}}%
\pgfpathlineto{\pgfqpoint{4.650778in}{1.303357in}}%
\pgfpathlineto{\pgfqpoint{4.595495in}{1.311293in}}%
\pgfpathlineto{\pgfqpoint{4.548928in}{1.318744in}}%
\pgfpathlineto{\pgfqpoint{4.497815in}{1.327496in}}%
\pgfpathlineto{\pgfqpoint{4.509436in}{1.391963in}}%
\pgfpathlineto{\pgfqpoint{4.521208in}{1.458068in}}%
\pgfpathlineto{\pgfqpoint{4.537875in}{1.552087in}}%
\pgfpathlineto{\pgfqpoint{4.543829in}{1.585939in}}%
\pgfpathlineto{\pgfqpoint{4.572396in}{1.580713in}}%
\pgfpathlineto{\pgfqpoint{4.655381in}{1.567687in}}%
\pgfpathlineto{\pgfqpoint{4.689064in}{1.562882in}}%
\pgfpathlineto{\pgfqpoint{4.764479in}{1.552561in}}%
\pgfpathlineto{\pgfqpoint{4.818794in}{1.546395in}}%
\pgfpathlineto{\pgfqpoint{4.866234in}{1.541215in}}%
\pgfpathlineto{\pgfqpoint{4.860839in}{1.487840in}}%
\pgfpathlineto{\pgfqpoint{4.853363in}{1.410780in}}%
\pgfpathlineto{\pgfqpoint{4.846947in}{1.345289in}}%
\pgfpathlineto{\pgfqpoint{4.840540in}{1.280295in}}%
\pgfpathclose%
\pgfusepath{fill}%
\end{pgfscope}%
\begin{pgfscope}%
\pgfpathrectangle{\pgfqpoint{3.625000in}{0.100000in}}{\pgfqpoint{2.989028in}{1.913466in}}%
\pgfusepath{clip}%
\pgfsetbuttcap%
\pgfsetmiterjoin%
\definecolor{currentfill}{rgb}{0.999616,0.988082,0.729027}%
\pgfsetfillcolor{currentfill}%
\pgfsetlinewidth{0.000000pt}%
\definecolor{currentstroke}{rgb}{0.000000,0.000000,0.000000}%
\pgfsetstrokecolor{currentstroke}%
\pgfsetstrokeopacity{0.000000}%
\pgfsetdash{}{0pt}%
\pgfpathmoveto{\pgfqpoint{5.635291in}{1.369221in}}%
\pgfpathlineto{\pgfqpoint{5.580206in}{1.365308in}}%
\pgfpathlineto{\pgfqpoint{5.498100in}{1.361809in}}%
\pgfpathlineto{\pgfqpoint{5.494980in}{1.370103in}}%
\pgfpathlineto{\pgfqpoint{5.476772in}{1.376303in}}%
\pgfpathlineto{\pgfqpoint{5.472756in}{1.388016in}}%
\pgfpathlineto{\pgfqpoint{5.471075in}{1.402507in}}%
\pgfpathlineto{\pgfqpoint{5.475178in}{1.409931in}}%
\pgfpathlineto{\pgfqpoint{5.468697in}{1.417056in}}%
\pgfpathlineto{\pgfqpoint{5.467134in}{1.425554in}}%
\pgfpathlineto{\pgfqpoint{5.465045in}{1.444352in}}%
\pgfpathlineto{\pgfqpoint{5.458835in}{1.454564in}}%
\pgfpathlineto{\pgfqpoint{5.447805in}{1.460295in}}%
\pgfpathlineto{\pgfqpoint{5.435803in}{1.469752in}}%
\pgfpathlineto{\pgfqpoint{5.429599in}{1.480942in}}%
\pgfpathlineto{\pgfqpoint{5.418466in}{1.485448in}}%
\pgfpathlineto{\pgfqpoint{5.411922in}{1.492782in}}%
\pgfpathlineto{\pgfqpoint{5.403994in}{1.494027in}}%
\pgfpathlineto{\pgfqpoint{5.389840in}{1.504916in}}%
\pgfpathlineto{\pgfqpoint{5.392139in}{1.517447in}}%
\pgfpathlineto{\pgfqpoint{5.391696in}{1.541277in}}%
\pgfpathlineto{\pgfqpoint{5.396061in}{1.547836in}}%
\pgfpathlineto{\pgfqpoint{5.392139in}{1.557741in}}%
\pgfpathlineto{\pgfqpoint{5.385230in}{1.559657in}}%
\pgfpathlineto{\pgfqpoint{5.385801in}{1.568355in}}%
\pgfpathlineto{\pgfqpoint{5.394373in}{1.582098in}}%
\pgfpathlineto{\pgfqpoint{5.411348in}{1.592965in}}%
\pgfpathlineto{\pgfqpoint{5.410290in}{1.631611in}}%
\pgfpathlineto{\pgfqpoint{5.418784in}{1.637416in}}%
\pgfpathlineto{\pgfqpoint{5.426822in}{1.633549in}}%
\pgfpathlineto{\pgfqpoint{5.443197in}{1.639211in}}%
\pgfpathlineto{\pgfqpoint{5.474004in}{1.653439in}}%
\pgfpathlineto{\pgfqpoint{5.478017in}{1.649033in}}%
\pgfpathlineto{\pgfqpoint{5.472182in}{1.629079in}}%
\pgfpathlineto{\pgfqpoint{5.480863in}{1.633458in}}%
\pgfpathlineto{\pgfqpoint{5.495731in}{1.629085in}}%
\pgfpathlineto{\pgfqpoint{5.504867in}{1.625436in}}%
\pgfpathlineto{\pgfqpoint{5.509989in}{1.614740in}}%
\pgfpathlineto{\pgfqpoint{5.556860in}{1.604646in}}%
\pgfpathlineto{\pgfqpoint{5.570871in}{1.597722in}}%
\pgfpathlineto{\pgfqpoint{5.585116in}{1.597803in}}%
\pgfpathlineto{\pgfqpoint{5.599752in}{1.594947in}}%
\pgfpathlineto{\pgfqpoint{5.609251in}{1.585189in}}%
\pgfpathlineto{\pgfqpoint{5.618328in}{1.580368in}}%
\pgfpathlineto{\pgfqpoint{5.619996in}{1.566462in}}%
\pgfpathlineto{\pgfqpoint{5.617314in}{1.557730in}}%
\pgfpathlineto{\pgfqpoint{5.627430in}{1.557084in}}%
\pgfpathlineto{\pgfqpoint{5.624022in}{1.546957in}}%
\pgfpathlineto{\pgfqpoint{5.627264in}{1.543360in}}%
\pgfpathlineto{\pgfqpoint{5.630488in}{1.533800in}}%
\pgfpathlineto{\pgfqpoint{5.620629in}{1.528739in}}%
\pgfpathlineto{\pgfqpoint{5.614905in}{1.514637in}}%
\pgfpathlineto{\pgfqpoint{5.613109in}{1.504665in}}%
\pgfpathlineto{\pgfqpoint{5.618600in}{1.502958in}}%
\pgfpathlineto{\pgfqpoint{5.625632in}{1.510436in}}%
\pgfpathlineto{\pgfqpoint{5.631610in}{1.523394in}}%
\pgfpathlineto{\pgfqpoint{5.639698in}{1.527895in}}%
\pgfpathlineto{\pgfqpoint{5.645776in}{1.522041in}}%
\pgfpathlineto{\pgfqpoint{5.639793in}{1.504322in}}%
\pgfpathlineto{\pgfqpoint{5.637916in}{1.490567in}}%
\pgfpathlineto{\pgfqpoint{5.639684in}{1.480679in}}%
\pgfpathlineto{\pgfqpoint{5.634128in}{1.475085in}}%
\pgfpathlineto{\pgfqpoint{5.631353in}{1.461416in}}%
\pgfpathlineto{\pgfqpoint{5.633620in}{1.447050in}}%
\pgfpathlineto{\pgfqpoint{5.627064in}{1.425804in}}%
\pgfpathlineto{\pgfqpoint{5.627202in}{1.415188in}}%
\pgfpathlineto{\pgfqpoint{5.632383in}{1.392182in}}%
\pgfpathlineto{\pgfqpoint{5.635741in}{1.388236in}}%
\pgfpathlineto{\pgfqpoint{5.635291in}{1.369221in}}%
\pgfpathclose%
\pgfusepath{fill}%
\end{pgfscope}%
\begin{pgfscope}%
\pgfpathrectangle{\pgfqpoint{3.625000in}{0.100000in}}{\pgfqpoint{2.989028in}{1.913466in}}%
\pgfusepath{clip}%
\pgfsetbuttcap%
\pgfsetmiterjoin%
\definecolor{currentfill}{rgb}{0.999616,0.988082,0.729027}%
\pgfsetfillcolor{currentfill}%
\pgfsetlinewidth{0.000000pt}%
\definecolor{currentstroke}{rgb}{0.000000,0.000000,0.000000}%
\pgfsetstrokecolor{currentstroke}%
\pgfsetstrokeopacity{0.000000}%
\pgfsetdash{}{0pt}%
\pgfpathmoveto{\pgfqpoint{5.655944in}{1.555675in}}%
\pgfpathlineto{\pgfqpoint{5.657173in}{1.546498in}}%
\pgfpathlineto{\pgfqpoint{5.645909in}{1.522320in}}%
\pgfpathlineto{\pgfqpoint{5.640904in}{1.529323in}}%
\pgfpathlineto{\pgfqpoint{5.655944in}{1.555675in}}%
\pgfpathclose%
\pgfusepath{fill}%
\end{pgfscope}%
\begin{pgfscope}%
\pgfpathrectangle{\pgfqpoint{3.625000in}{0.100000in}}{\pgfqpoint{2.989028in}{1.913466in}}%
\pgfusepath{clip}%
\pgfsetbuttcap%
\pgfsetmiterjoin%
\definecolor{currentfill}{rgb}{0.702345,0.879585,0.636678}%
\pgfsetfillcolor{currentfill}%
\pgfsetlinewidth{0.000000pt}%
\definecolor{currentstroke}{rgb}{0.000000,0.000000,0.000000}%
\pgfsetstrokecolor{currentstroke}%
\pgfsetstrokeopacity{0.000000}%
\pgfsetdash{}{0pt}%
\pgfpathmoveto{\pgfqpoint{4.292475in}{1.706523in}}%
\pgfpathlineto{\pgfqpoint{4.294703in}{1.717547in}}%
\pgfpathlineto{\pgfqpoint{4.293022in}{1.729516in}}%
\pgfpathlineto{\pgfqpoint{4.295595in}{1.742564in}}%
\pgfpathlineto{\pgfqpoint{4.308700in}{1.795504in}}%
\pgfpathlineto{\pgfqpoint{4.319722in}{1.839512in}}%
\pgfpathlineto{\pgfqpoint{4.335222in}{1.900931in}}%
\pgfpathlineto{\pgfqpoint{4.376305in}{1.891004in}}%
\pgfpathlineto{\pgfqpoint{4.361121in}{1.824733in}}%
\pgfpathlineto{\pgfqpoint{4.370737in}{1.803380in}}%
\pgfpathlineto{\pgfqpoint{4.369521in}{1.791786in}}%
\pgfpathlineto{\pgfqpoint{4.374675in}{1.774343in}}%
\pgfpathlineto{\pgfqpoint{4.381873in}{1.770222in}}%
\pgfpathlineto{\pgfqpoint{4.389750in}{1.750011in}}%
\pgfpathlineto{\pgfqpoint{4.394589in}{1.744445in}}%
\pgfpathlineto{\pgfqpoint{4.394003in}{1.737279in}}%
\pgfpathlineto{\pgfqpoint{4.403176in}{1.730870in}}%
\pgfpathlineto{\pgfqpoint{4.404470in}{1.723437in}}%
\pgfpathlineto{\pgfqpoint{4.417063in}{1.722399in}}%
\pgfpathlineto{\pgfqpoint{4.401093in}{1.680980in}}%
\pgfpathlineto{\pgfqpoint{4.403535in}{1.672422in}}%
\pgfpathlineto{\pgfqpoint{4.394348in}{1.667101in}}%
\pgfpathlineto{\pgfqpoint{4.393102in}{1.653594in}}%
\pgfpathlineto{\pgfqpoint{4.400763in}{1.645136in}}%
\pgfpathlineto{\pgfqpoint{4.410827in}{1.649240in}}%
\pgfpathlineto{\pgfqpoint{4.418220in}{1.656827in}}%
\pgfpathlineto{\pgfqpoint{4.426101in}{1.648251in}}%
\pgfpathlineto{\pgfqpoint{4.425483in}{1.630729in}}%
\pgfpathlineto{\pgfqpoint{4.434652in}{1.610116in}}%
\pgfpathlineto{\pgfqpoint{4.432552in}{1.597637in}}%
\pgfpathlineto{\pgfqpoint{4.445429in}{1.588978in}}%
\pgfpathlineto{\pgfqpoint{4.445634in}{1.576938in}}%
\pgfpathlineto{\pgfqpoint{4.450020in}{1.564285in}}%
\pgfpathlineto{\pgfqpoint{4.458172in}{1.567136in}}%
\pgfpathlineto{\pgfqpoint{4.476852in}{1.561826in}}%
\pgfpathlineto{\pgfqpoint{4.484543in}{1.567727in}}%
\pgfpathlineto{\pgfqpoint{4.490043in}{1.563913in}}%
\pgfpathlineto{\pgfqpoint{4.500984in}{1.564390in}}%
\pgfpathlineto{\pgfqpoint{4.503331in}{1.560451in}}%
\pgfpathlineto{\pgfqpoint{4.515011in}{1.561552in}}%
\pgfpathlineto{\pgfqpoint{4.520639in}{1.569785in}}%
\pgfpathlineto{\pgfqpoint{4.528502in}{1.570788in}}%
\pgfpathlineto{\pgfqpoint{4.534620in}{1.554502in}}%
\pgfpathlineto{\pgfqpoint{4.537875in}{1.552087in}}%
\pgfpathlineto{\pgfqpoint{4.521208in}{1.458068in}}%
\pgfpathlineto{\pgfqpoint{4.509436in}{1.391963in}}%
\pgfpathlineto{\pgfqpoint{4.416592in}{1.409886in}}%
\pgfpathlineto{\pgfqpoint{4.366442in}{1.419862in}}%
\pgfpathlineto{\pgfqpoint{4.319534in}{1.430177in}}%
\pgfpathlineto{\pgfqpoint{4.275689in}{1.440103in}}%
\pgfpathlineto{\pgfqpoint{4.224946in}{1.452341in}}%
\pgfpathlineto{\pgfqpoint{4.251119in}{1.559712in}}%
\pgfpathlineto{\pgfqpoint{4.252531in}{1.567559in}}%
\pgfpathlineto{\pgfqpoint{4.264170in}{1.588120in}}%
\pgfpathlineto{\pgfqpoint{4.252118in}{1.600478in}}%
\pgfpathlineto{\pgfqpoint{4.254620in}{1.612613in}}%
\pgfpathlineto{\pgfqpoint{4.259208in}{1.615650in}}%
\pgfpathlineto{\pgfqpoint{4.267402in}{1.628155in}}%
\pgfpathlineto{\pgfqpoint{4.279345in}{1.636806in}}%
\pgfpathlineto{\pgfqpoint{4.279997in}{1.643207in}}%
\pgfpathlineto{\pgfqpoint{4.287206in}{1.649603in}}%
\pgfpathlineto{\pgfqpoint{4.293191in}{1.661575in}}%
\pgfpathlineto{\pgfqpoint{4.305469in}{1.674240in}}%
\pgfpathlineto{\pgfqpoint{4.304597in}{1.686749in}}%
\pgfpathlineto{\pgfqpoint{4.296203in}{1.693698in}}%
\pgfpathlineto{\pgfqpoint{4.292475in}{1.706523in}}%
\pgfpathclose%
\pgfusepath{fill}%
\end{pgfscope}%
\begin{pgfscope}%
\pgfpathrectangle{\pgfqpoint{3.625000in}{0.100000in}}{\pgfqpoint{2.989028in}{1.913466in}}%
\pgfusepath{clip}%
\pgfsetbuttcap%
\pgfsetmiterjoin%
\definecolor{currentfill}{rgb}{0.975010,0.990004,0.710035}%
\pgfsetfillcolor{currentfill}%
\pgfsetlinewidth{0.000000pt}%
\definecolor{currentstroke}{rgb}{0.000000,0.000000,0.000000}%
\pgfsetstrokecolor{currentstroke}%
\pgfsetstrokeopacity{0.000000}%
\pgfsetdash{}{0pt}%
\pgfpathmoveto{\pgfqpoint{6.324865in}{1.496989in}}%
\pgfpathlineto{\pgfqpoint{6.321007in}{1.509167in}}%
\pgfpathlineto{\pgfqpoint{6.313848in}{1.546108in}}%
\pgfpathlineto{\pgfqpoint{6.304593in}{1.557473in}}%
\pgfpathlineto{\pgfqpoint{6.296474in}{1.577911in}}%
\pgfpathlineto{\pgfqpoint{6.299140in}{1.593064in}}%
\pgfpathlineto{\pgfqpoint{6.297002in}{1.604233in}}%
\pgfpathlineto{\pgfqpoint{6.290065in}{1.615202in}}%
\pgfpathlineto{\pgfqpoint{6.289780in}{1.627285in}}%
\pgfpathlineto{\pgfqpoint{6.285752in}{1.640318in}}%
\pgfpathlineto{\pgfqpoint{6.321797in}{1.649189in}}%
\pgfpathlineto{\pgfqpoint{6.368616in}{1.661826in}}%
\pgfpathlineto{\pgfqpoint{6.370485in}{1.654580in}}%
\pgfpathlineto{\pgfqpoint{6.367504in}{1.643043in}}%
\pgfpathlineto{\pgfqpoint{6.374502in}{1.633812in}}%
\pgfpathlineto{\pgfqpoint{6.370785in}{1.622115in}}%
\pgfpathlineto{\pgfqpoint{6.356015in}{1.607411in}}%
\pgfpathlineto{\pgfqpoint{6.360075in}{1.596366in}}%
\pgfpathlineto{\pgfqpoint{6.357066in}{1.573021in}}%
\pgfpathlineto{\pgfqpoint{6.352854in}{1.559858in}}%
\pgfpathlineto{\pgfqpoint{6.358228in}{1.522426in}}%
\pgfpathlineto{\pgfqpoint{6.357504in}{1.509256in}}%
\pgfpathlineto{\pgfqpoint{6.362703in}{1.505030in}}%
\pgfpathlineto{\pgfqpoint{6.324865in}{1.496989in}}%
\pgfpathclose%
\pgfusepath{fill}%
\end{pgfscope}%
\begin{pgfscope}%
\pgfpathrectangle{\pgfqpoint{3.625000in}{0.100000in}}{\pgfqpoint{2.989028in}{1.913466in}}%
\pgfusepath{clip}%
\pgfsetbuttcap%
\pgfsetmiterjoin%
\definecolor{currentfill}{rgb}{0.948097,0.979239,0.668051}%
\pgfsetfillcolor{currentfill}%
\pgfsetlinewidth{0.000000pt}%
\definecolor{currentstroke}{rgb}{0.000000,0.000000,0.000000}%
\pgfsetstrokecolor{currentstroke}%
\pgfsetstrokeopacity{0.000000}%
\pgfsetdash{}{0pt}%
\pgfpathmoveto{\pgfqpoint{5.217728in}{1.422304in}}%
\pgfpathlineto{\pgfqpoint{5.219075in}{1.490552in}}%
\pgfpathlineto{\pgfqpoint{5.219880in}{1.539046in}}%
\pgfpathlineto{\pgfqpoint{5.216374in}{1.545688in}}%
\pgfpathlineto{\pgfqpoint{5.208985in}{1.548158in}}%
\pgfpathlineto{\pgfqpoint{5.202284in}{1.559721in}}%
\pgfpathlineto{\pgfqpoint{5.211034in}{1.569764in}}%
\pgfpathlineto{\pgfqpoint{5.215419in}{1.582557in}}%
\pgfpathlineto{\pgfqpoint{5.216288in}{1.593192in}}%
\pgfpathlineto{\pgfqpoint{5.214274in}{1.609122in}}%
\pgfpathlineto{\pgfqpoint{5.208167in}{1.619346in}}%
\pgfpathlineto{\pgfqpoint{5.206080in}{1.628658in}}%
\pgfpathlineto{\pgfqpoint{5.204560in}{1.663212in}}%
\pgfpathlineto{\pgfqpoint{5.204579in}{1.692956in}}%
\pgfpathlineto{\pgfqpoint{5.196363in}{1.716582in}}%
\pgfpathlineto{\pgfqpoint{5.192777in}{1.732596in}}%
\pgfpathlineto{\pgfqpoint{5.193398in}{1.753978in}}%
\pgfpathlineto{\pgfqpoint{5.195285in}{1.764381in}}%
\pgfpathlineto{\pgfqpoint{5.190014in}{1.786150in}}%
\pgfpathlineto{\pgfqpoint{5.225900in}{1.785431in}}%
\pgfpathlineto{\pgfqpoint{5.280411in}{1.784963in}}%
\pgfpathlineto{\pgfqpoint{5.280709in}{1.809705in}}%
\pgfpathlineto{\pgfqpoint{5.294584in}{1.806981in}}%
\pgfpathlineto{\pgfqpoint{5.301233in}{1.776812in}}%
\pgfpathlineto{\pgfqpoint{5.306136in}{1.765965in}}%
\pgfpathlineto{\pgfqpoint{5.318328in}{1.765645in}}%
\pgfpathlineto{\pgfqpoint{5.321056in}{1.761964in}}%
\pgfpathlineto{\pgfqpoint{5.338062in}{1.760334in}}%
\pgfpathlineto{\pgfqpoint{5.340920in}{1.752855in}}%
\pgfpathlineto{\pgfqpoint{5.352640in}{1.754536in}}%
\pgfpathlineto{\pgfqpoint{5.361742in}{1.761531in}}%
\pgfpathlineto{\pgfqpoint{5.377440in}{1.761270in}}%
\pgfpathlineto{\pgfqpoint{5.387153in}{1.755644in}}%
\pgfpathlineto{\pgfqpoint{5.388270in}{1.750368in}}%
\pgfpathlineto{\pgfqpoint{5.397512in}{1.749264in}}%
\pgfpathlineto{\pgfqpoint{5.403543in}{1.734873in}}%
\pgfpathlineto{\pgfqpoint{5.407434in}{1.743721in}}%
\pgfpathlineto{\pgfqpoint{5.418051in}{1.744265in}}%
\pgfpathlineto{\pgfqpoint{5.420736in}{1.737374in}}%
\pgfpathlineto{\pgfqpoint{5.432682in}{1.734228in}}%
\pgfpathlineto{\pgfqpoint{5.439270in}{1.724302in}}%
\pgfpathlineto{\pgfqpoint{5.453880in}{1.727384in}}%
\pgfpathlineto{\pgfqpoint{5.469957in}{1.739565in}}%
\pgfpathlineto{\pgfqpoint{5.475818in}{1.728827in}}%
\pgfpathlineto{\pgfqpoint{5.502198in}{1.731766in}}%
\pgfpathlineto{\pgfqpoint{5.513474in}{1.724397in}}%
\pgfpathlineto{\pgfqpoint{5.520037in}{1.727028in}}%
\pgfpathlineto{\pgfqpoint{5.525300in}{1.722879in}}%
\pgfpathlineto{\pgfqpoint{5.509687in}{1.713035in}}%
\pgfpathlineto{\pgfqpoint{5.487401in}{1.704265in}}%
\pgfpathlineto{\pgfqpoint{5.465329in}{1.686734in}}%
\pgfpathlineto{\pgfqpoint{5.446193in}{1.663680in}}%
\pgfpathlineto{\pgfqpoint{5.431707in}{1.650050in}}%
\pgfpathlineto{\pgfqpoint{5.419018in}{1.640682in}}%
\pgfpathlineto{\pgfqpoint{5.410290in}{1.631611in}}%
\pgfpathlineto{\pgfqpoint{5.411348in}{1.592965in}}%
\pgfpathlineto{\pgfqpoint{5.394373in}{1.582098in}}%
\pgfpathlineto{\pgfqpoint{5.385801in}{1.568355in}}%
\pgfpathlineto{\pgfqpoint{5.385230in}{1.559657in}}%
\pgfpathlineto{\pgfqpoint{5.392139in}{1.557741in}}%
\pgfpathlineto{\pgfqpoint{5.396061in}{1.547836in}}%
\pgfpathlineto{\pgfqpoint{5.391696in}{1.541277in}}%
\pgfpathlineto{\pgfqpoint{5.392139in}{1.517447in}}%
\pgfpathlineto{\pgfqpoint{5.389840in}{1.504916in}}%
\pgfpathlineto{\pgfqpoint{5.403994in}{1.494027in}}%
\pgfpathlineto{\pgfqpoint{5.411922in}{1.492782in}}%
\pgfpathlineto{\pgfqpoint{5.418466in}{1.485448in}}%
\pgfpathlineto{\pgfqpoint{5.429599in}{1.480942in}}%
\pgfpathlineto{\pgfqpoint{5.435803in}{1.469752in}}%
\pgfpathlineto{\pgfqpoint{5.447805in}{1.460295in}}%
\pgfpathlineto{\pgfqpoint{5.458835in}{1.454564in}}%
\pgfpathlineto{\pgfqpoint{5.465045in}{1.444352in}}%
\pgfpathlineto{\pgfqpoint{5.467134in}{1.425554in}}%
\pgfpathlineto{\pgfqpoint{5.408600in}{1.423428in}}%
\pgfpathlineto{\pgfqpoint{5.358705in}{1.422385in}}%
\pgfpathlineto{\pgfqpoint{5.313246in}{1.421717in}}%
\pgfpathlineto{\pgfqpoint{5.265155in}{1.421790in}}%
\pgfpathlineto{\pgfqpoint{5.217728in}{1.422304in}}%
\pgfpathclose%
\pgfusepath{fill}%
\end{pgfscope}%
\begin{pgfscope}%
\pgfpathrectangle{\pgfqpoint{3.625000in}{0.100000in}}{\pgfqpoint{2.989028in}{1.913466in}}%
\pgfusepath{clip}%
\pgfsetbuttcap%
\pgfsetmiterjoin%
\definecolor{currentfill}{rgb}{0.838447,0.934948,0.608997}%
\pgfsetfillcolor{currentfill}%
\pgfsetlinewidth{0.000000pt}%
\definecolor{currentstroke}{rgb}{0.000000,0.000000,0.000000}%
\pgfsetstrokecolor{currentstroke}%
\pgfsetstrokeopacity{0.000000}%
\pgfsetdash{}{0pt}%
\pgfpathmoveto{\pgfqpoint{3.888713in}{1.549891in}}%
\pgfpathlineto{\pgfqpoint{3.884077in}{1.558421in}}%
\pgfpathlineto{\pgfqpoint{3.887037in}{1.580316in}}%
\pgfpathlineto{\pgfqpoint{3.892750in}{1.591758in}}%
\pgfpathlineto{\pgfqpoint{3.889894in}{1.607237in}}%
\pgfpathlineto{\pgfqpoint{3.895854in}{1.613747in}}%
\pgfpathlineto{\pgfqpoint{3.906727in}{1.631292in}}%
\pgfpathlineto{\pgfqpoint{3.915946in}{1.641883in}}%
\pgfpathlineto{\pgfqpoint{3.929422in}{1.664951in}}%
\pgfpathlineto{\pgfqpoint{3.939758in}{1.690079in}}%
\pgfpathlineto{\pgfqpoint{3.950730in}{1.713646in}}%
\pgfpathlineto{\pgfqpoint{3.952948in}{1.723489in}}%
\pgfpathlineto{\pgfqpoint{3.968085in}{1.751647in}}%
\pgfpathlineto{\pgfqpoint{3.970999in}{1.764014in}}%
\pgfpathlineto{\pgfqpoint{3.977444in}{1.777001in}}%
\pgfpathlineto{\pgfqpoint{3.979747in}{1.792843in}}%
\pgfpathlineto{\pgfqpoint{3.992051in}{1.800568in}}%
\pgfpathlineto{\pgfqpoint{4.006639in}{1.804461in}}%
\pgfpathlineto{\pgfqpoint{4.014015in}{1.795741in}}%
\pgfpathlineto{\pgfqpoint{4.020391in}{1.796386in}}%
\pgfpathlineto{\pgfqpoint{4.030347in}{1.785992in}}%
\pgfpathlineto{\pgfqpoint{4.031670in}{1.776202in}}%
\pgfpathlineto{\pgfqpoint{4.027895in}{1.757031in}}%
\pgfpathlineto{\pgfqpoint{4.040456in}{1.747249in}}%
\pgfpathlineto{\pgfqpoint{4.048603in}{1.743571in}}%
\pgfpathlineto{\pgfqpoint{4.070673in}{1.747325in}}%
\pgfpathlineto{\pgfqpoint{4.083460in}{1.744762in}}%
\pgfpathlineto{\pgfqpoint{4.092248in}{1.740554in}}%
\pgfpathlineto{\pgfqpoint{4.096877in}{1.732568in}}%
\pgfpathlineto{\pgfqpoint{4.126190in}{1.733634in}}%
\pgfpathlineto{\pgfqpoint{4.130851in}{1.728747in}}%
\pgfpathlineto{\pgfqpoint{4.141848in}{1.727692in}}%
\pgfpathlineto{\pgfqpoint{4.152921in}{1.730814in}}%
\pgfpathlineto{\pgfqpoint{4.185663in}{1.730028in}}%
\pgfpathlineto{\pgfqpoint{4.192734in}{1.727622in}}%
\pgfpathlineto{\pgfqpoint{4.200998in}{1.730394in}}%
\pgfpathlineto{\pgfqpoint{4.292475in}{1.706523in}}%
\pgfpathlineto{\pgfqpoint{4.296203in}{1.693698in}}%
\pgfpathlineto{\pgfqpoint{4.304597in}{1.686749in}}%
\pgfpathlineto{\pgfqpoint{4.305469in}{1.674240in}}%
\pgfpathlineto{\pgfqpoint{4.293191in}{1.661575in}}%
\pgfpathlineto{\pgfqpoint{4.287206in}{1.649603in}}%
\pgfpathlineto{\pgfqpoint{4.279997in}{1.643207in}}%
\pgfpathlineto{\pgfqpoint{4.279345in}{1.636806in}}%
\pgfpathlineto{\pgfqpoint{4.267402in}{1.628155in}}%
\pgfpathlineto{\pgfqpoint{4.259208in}{1.615650in}}%
\pgfpathlineto{\pgfqpoint{4.254620in}{1.612613in}}%
\pgfpathlineto{\pgfqpoint{4.252118in}{1.600478in}}%
\pgfpathlineto{\pgfqpoint{4.264170in}{1.588120in}}%
\pgfpathlineto{\pgfqpoint{4.252531in}{1.567559in}}%
\pgfpathlineto{\pgfqpoint{4.251119in}{1.559712in}}%
\pgfpathlineto{\pgfqpoint{4.224946in}{1.452341in}}%
\pgfpathlineto{\pgfqpoint{4.169893in}{1.466447in}}%
\pgfpathlineto{\pgfqpoint{4.116784in}{1.480141in}}%
\pgfpathlineto{\pgfqpoint{4.084744in}{1.489057in}}%
\pgfpathlineto{\pgfqpoint{4.043578in}{1.500784in}}%
\pgfpathlineto{\pgfqpoint{3.977913in}{1.521452in}}%
\pgfpathlineto{\pgfqpoint{3.906532in}{1.543676in}}%
\pgfpathlineto{\pgfqpoint{3.888713in}{1.549891in}}%
\pgfpathclose%
\pgfusepath{fill}%
\end{pgfscope}%
\begin{pgfscope}%
\pgfpathrectangle{\pgfqpoint{3.625000in}{0.100000in}}{\pgfqpoint{2.989028in}{1.913466in}}%
\pgfusepath{clip}%
\pgfsetbuttcap%
\pgfsetmiterjoin%
\definecolor{currentfill}{rgb}{0.982699,0.993080,0.722030}%
\pgfsetfillcolor{currentfill}%
\pgfsetlinewidth{0.000000pt}%
\definecolor{currentstroke}{rgb}{0.000000,0.000000,0.000000}%
\pgfsetstrokecolor{currentstroke}%
\pgfsetstrokeopacity{0.000000}%
\pgfsetdash{}{0pt}%
\pgfpathmoveto{\pgfqpoint{6.362703in}{1.505030in}}%
\pgfpathlineto{\pgfqpoint{6.357504in}{1.509256in}}%
\pgfpathlineto{\pgfqpoint{6.358228in}{1.522426in}}%
\pgfpathlineto{\pgfqpoint{6.352854in}{1.559858in}}%
\pgfpathlineto{\pgfqpoint{6.357066in}{1.573021in}}%
\pgfpathlineto{\pgfqpoint{6.360075in}{1.596366in}}%
\pgfpathlineto{\pgfqpoint{6.356015in}{1.607411in}}%
\pgfpathlineto{\pgfqpoint{6.370785in}{1.622115in}}%
\pgfpathlineto{\pgfqpoint{6.374502in}{1.633812in}}%
\pgfpathlineto{\pgfqpoint{6.367504in}{1.643043in}}%
\pgfpathlineto{\pgfqpoint{6.370485in}{1.654580in}}%
\pgfpathlineto{\pgfqpoint{6.368616in}{1.661826in}}%
\pgfpathlineto{\pgfqpoint{6.370221in}{1.677346in}}%
\pgfpathlineto{\pgfqpoint{6.373221in}{1.682137in}}%
\pgfpathlineto{\pgfqpoint{6.382471in}{1.685161in}}%
\pgfpathlineto{\pgfqpoint{6.395964in}{1.645645in}}%
\pgfpathlineto{\pgfqpoint{6.420998in}{1.563760in}}%
\pgfpathlineto{\pgfqpoint{6.430471in}{1.557411in}}%
\pgfpathlineto{\pgfqpoint{6.432175in}{1.550163in}}%
\pgfpathlineto{\pgfqpoint{6.437175in}{1.547235in}}%
\pgfpathlineto{\pgfqpoint{6.436795in}{1.534090in}}%
\pgfpathlineto{\pgfqpoint{6.431497in}{1.533879in}}%
\pgfpathlineto{\pgfqpoint{6.420764in}{1.525694in}}%
\pgfpathlineto{\pgfqpoint{6.417669in}{1.517482in}}%
\pgfpathlineto{\pgfqpoint{6.388942in}{1.510374in}}%
\pgfpathlineto{\pgfqpoint{6.362703in}{1.505030in}}%
\pgfpathclose%
\pgfusepath{fill}%
\end{pgfscope}%
\begin{pgfscope}%
\pgfpathrectangle{\pgfqpoint{3.625000in}{0.100000in}}{\pgfqpoint{2.989028in}{1.913466in}}%
\pgfusepath{clip}%
\pgfsetbuttcap%
\pgfsetmiterjoin%
\definecolor{currentfill}{rgb}{0.932718,0.973087,0.644060}%
\pgfsetfillcolor{currentfill}%
\pgfsetlinewidth{0.000000pt}%
\definecolor{currentstroke}{rgb}{0.000000,0.000000,0.000000}%
\pgfsetstrokecolor{currentstroke}%
\pgfsetstrokeopacity{0.000000}%
\pgfsetdash{}{0pt}%
\pgfpathmoveto{\pgfqpoint{5.464431in}{1.220143in}}%
\pgfpathlineto{\pgfqpoint{5.449266in}{1.235161in}}%
\pgfpathlineto{\pgfqpoint{5.400789in}{1.232365in}}%
\pgfpathlineto{\pgfqpoint{5.325184in}{1.229874in}}%
\pgfpathlineto{\pgfqpoint{5.249124in}{1.231060in}}%
\pgfpathlineto{\pgfqpoint{5.243785in}{1.240370in}}%
\pgfpathlineto{\pgfqpoint{5.245965in}{1.249511in}}%
\pgfpathlineto{\pgfqpoint{5.244930in}{1.265170in}}%
\pgfpathlineto{\pgfqpoint{5.240910in}{1.280338in}}%
\pgfpathlineto{\pgfqpoint{5.241378in}{1.288348in}}%
\pgfpathlineto{\pgfqpoint{5.232833in}{1.297398in}}%
\pgfpathlineto{\pgfqpoint{5.234692in}{1.309990in}}%
\pgfpathlineto{\pgfqpoint{5.221585in}{1.334864in}}%
\pgfpathlineto{\pgfqpoint{5.217680in}{1.355981in}}%
\pgfpathlineto{\pgfqpoint{5.208252in}{1.373157in}}%
\pgfpathlineto{\pgfqpoint{5.217201in}{1.394464in}}%
\pgfpathlineto{\pgfqpoint{5.212744in}{1.404379in}}%
\pgfpathlineto{\pgfqpoint{5.217728in}{1.422304in}}%
\pgfpathlineto{\pgfqpoint{5.265155in}{1.421790in}}%
\pgfpathlineto{\pgfqpoint{5.313246in}{1.421717in}}%
\pgfpathlineto{\pgfqpoint{5.358705in}{1.422385in}}%
\pgfpathlineto{\pgfqpoint{5.408600in}{1.423428in}}%
\pgfpathlineto{\pgfqpoint{5.467134in}{1.425554in}}%
\pgfpathlineto{\pgfqpoint{5.468697in}{1.417056in}}%
\pgfpathlineto{\pgfqpoint{5.475178in}{1.409931in}}%
\pgfpathlineto{\pgfqpoint{5.471075in}{1.402507in}}%
\pgfpathlineto{\pgfqpoint{5.472756in}{1.388016in}}%
\pgfpathlineto{\pgfqpoint{5.476772in}{1.376303in}}%
\pgfpathlineto{\pgfqpoint{5.494980in}{1.370103in}}%
\pgfpathlineto{\pgfqpoint{5.498100in}{1.361809in}}%
\pgfpathlineto{\pgfqpoint{5.508091in}{1.352497in}}%
\pgfpathlineto{\pgfqpoint{5.512167in}{1.342864in}}%
\pgfpathlineto{\pgfqpoint{5.522291in}{1.336402in}}%
\pgfpathlineto{\pgfqpoint{5.523875in}{1.328623in}}%
\pgfpathlineto{\pgfqpoint{5.521904in}{1.316845in}}%
\pgfpathlineto{\pgfqpoint{5.516751in}{1.313315in}}%
\pgfpathlineto{\pgfqpoint{5.515197in}{1.302104in}}%
\pgfpathlineto{\pgfqpoint{5.500389in}{1.293200in}}%
\pgfpathlineto{\pgfqpoint{5.481087in}{1.288322in}}%
\pgfpathlineto{\pgfqpoint{5.479318in}{1.277106in}}%
\pgfpathlineto{\pgfqpoint{5.486764in}{1.269093in}}%
\pgfpathlineto{\pgfqpoint{5.487068in}{1.259018in}}%
\pgfpathlineto{\pgfqpoint{5.481060in}{1.251098in}}%
\pgfpathlineto{\pgfqpoint{5.477907in}{1.239332in}}%
\pgfpathlineto{\pgfqpoint{5.467465in}{1.235438in}}%
\pgfpathlineto{\pgfqpoint{5.468130in}{1.222326in}}%
\pgfpathlineto{\pgfqpoint{5.464431in}{1.220143in}}%
\pgfpathclose%
\pgfusepath{fill}%
\end{pgfscope}%
\begin{pgfscope}%
\pgfpathrectangle{\pgfqpoint{3.625000in}{0.100000in}}{\pgfqpoint{2.989028in}{1.913466in}}%
\pgfusepath{clip}%
\pgfsetbuttcap%
\pgfsetmiterjoin%
\definecolor{currentfill}{rgb}{0.921184,0.968474,0.626067}%
\pgfsetfillcolor{currentfill}%
\pgfsetlinewidth{0.000000pt}%
\definecolor{currentstroke}{rgb}{0.000000,0.000000,0.000000}%
\pgfsetstrokecolor{currentstroke}%
\pgfsetstrokeopacity{0.000000}%
\pgfsetdash{}{0pt}%
\pgfpathmoveto{\pgfqpoint{6.405364in}{1.467139in}}%
\pgfpathlineto{\pgfqpoint{6.391000in}{1.464872in}}%
\pgfpathlineto{\pgfqpoint{6.325022in}{1.449880in}}%
\pgfpathlineto{\pgfqpoint{6.323869in}{1.451627in}}%
\pgfpathlineto{\pgfqpoint{6.324865in}{1.496989in}}%
\pgfpathlineto{\pgfqpoint{6.362703in}{1.505030in}}%
\pgfpathlineto{\pgfqpoint{6.388942in}{1.510374in}}%
\pgfpathlineto{\pgfqpoint{6.417669in}{1.517482in}}%
\pgfpathlineto{\pgfqpoint{6.420764in}{1.525694in}}%
\pgfpathlineto{\pgfqpoint{6.431497in}{1.533879in}}%
\pgfpathlineto{\pgfqpoint{6.436795in}{1.534090in}}%
\pgfpathlineto{\pgfqpoint{6.443759in}{1.522149in}}%
\pgfpathlineto{\pgfqpoint{6.437440in}{1.504717in}}%
\pgfpathlineto{\pgfqpoint{6.436512in}{1.494495in}}%
\pgfpathlineto{\pgfqpoint{6.449304in}{1.495455in}}%
\pgfpathlineto{\pgfqpoint{6.455105in}{1.490552in}}%
\pgfpathlineto{\pgfqpoint{6.468114in}{1.470509in}}%
\pgfpathlineto{\pgfqpoint{6.474579in}{1.468072in}}%
\pgfpathlineto{\pgfqpoint{6.485410in}{1.468976in}}%
\pgfpathlineto{\pgfqpoint{6.492946in}{1.475787in}}%
\pgfpathlineto{\pgfqpoint{6.497956in}{1.469708in}}%
\pgfpathlineto{\pgfqpoint{6.466451in}{1.453083in}}%
\pgfpathlineto{\pgfqpoint{6.465467in}{1.465013in}}%
\pgfpathlineto{\pgfqpoint{6.451155in}{1.446471in}}%
\pgfpathlineto{\pgfqpoint{6.446123in}{1.443277in}}%
\pgfpathlineto{\pgfqpoint{6.439105in}{1.453976in}}%
\pgfpathlineto{\pgfqpoint{6.437183in}{1.455442in}}%
\pgfpathlineto{\pgfqpoint{6.430650in}{1.458903in}}%
\pgfpathlineto{\pgfqpoint{6.424952in}{1.472940in}}%
\pgfpathlineto{\pgfqpoint{6.405364in}{1.467139in}}%
\pgfpathclose%
\pgfusepath{fill}%
\end{pgfscope}%
\begin{pgfscope}%
\pgfpathrectangle{\pgfqpoint{3.625000in}{0.100000in}}{\pgfqpoint{2.989028in}{1.913466in}}%
\pgfusepath{clip}%
\pgfsetbuttcap%
\pgfsetmiterjoin%
\definecolor{currentfill}{rgb}{0.328028,0.680507,0.680277}%
\pgfsetfillcolor{currentfill}%
\pgfsetlinewidth{0.000000pt}%
\definecolor{currentstroke}{rgb}{0.000000,0.000000,0.000000}%
\pgfsetstrokecolor{currentstroke}%
\pgfsetstrokeopacity{0.000000}%
\pgfsetdash{}{0pt}%
\pgfpathmoveto{\pgfqpoint{4.933935in}{1.205970in}}%
\pgfpathlineto{\pgfqpoint{4.939195in}{1.271196in}}%
\pgfpathlineto{\pgfqpoint{4.909404in}{1.273614in}}%
\pgfpathlineto{\pgfqpoint{4.840540in}{1.280295in}}%
\pgfpathlineto{\pgfqpoint{4.846947in}{1.345289in}}%
\pgfpathlineto{\pgfqpoint{4.853363in}{1.410780in}}%
\pgfpathlineto{\pgfqpoint{4.879836in}{1.407921in}}%
\pgfpathlineto{\pgfqpoint{4.947447in}{1.401533in}}%
\pgfpathlineto{\pgfqpoint{4.988365in}{1.398329in}}%
\pgfpathlineto{\pgfqpoint{5.037902in}{1.395489in}}%
\pgfpathlineto{\pgfqpoint{5.119439in}{1.391630in}}%
\pgfpathlineto{\pgfqpoint{5.121231in}{1.387975in}}%
\pgfpathlineto{\pgfqpoint{5.144594in}{1.376416in}}%
\pgfpathlineto{\pgfqpoint{5.151759in}{1.382286in}}%
\pgfpathlineto{\pgfqpoint{5.172400in}{1.381750in}}%
\pgfpathlineto{\pgfqpoint{5.184700in}{1.375307in}}%
\pgfpathlineto{\pgfqpoint{5.204205in}{1.367982in}}%
\pgfpathlineto{\pgfqpoint{5.209670in}{1.357295in}}%
\pgfpathlineto{\pgfqpoint{5.217680in}{1.355981in}}%
\pgfpathlineto{\pgfqpoint{5.221585in}{1.334864in}}%
\pgfpathlineto{\pgfqpoint{5.234692in}{1.309990in}}%
\pgfpathlineto{\pgfqpoint{5.232833in}{1.297398in}}%
\pgfpathlineto{\pgfqpoint{5.241378in}{1.288348in}}%
\pgfpathlineto{\pgfqpoint{5.240910in}{1.280338in}}%
\pgfpathlineto{\pgfqpoint{5.244930in}{1.265170in}}%
\pgfpathlineto{\pgfqpoint{5.245965in}{1.249511in}}%
\pgfpathlineto{\pgfqpoint{5.243785in}{1.240370in}}%
\pgfpathlineto{\pgfqpoint{5.249124in}{1.231060in}}%
\pgfpathlineto{\pgfqpoint{5.256447in}{1.214132in}}%
\pgfpathlineto{\pgfqpoint{5.263455in}{1.207242in}}%
\pgfpathlineto{\pgfqpoint{5.271810in}{1.192312in}}%
\pgfpathlineto{\pgfqpoint{5.248131in}{1.192066in}}%
\pgfpathlineto{\pgfqpoint{5.196933in}{1.192859in}}%
\pgfpathlineto{\pgfqpoint{5.140367in}{1.194592in}}%
\pgfpathlineto{\pgfqpoint{5.083468in}{1.196775in}}%
\pgfpathlineto{\pgfqpoint{4.999809in}{1.201345in}}%
\pgfpathlineto{\pgfqpoint{4.933935in}{1.205970in}}%
\pgfpathclose%
\pgfusepath{fill}%
\end{pgfscope}%
\begin{pgfscope}%
\pgfpathrectangle{\pgfqpoint{3.625000in}{0.100000in}}{\pgfqpoint{2.989028in}{1.913466in}}%
\pgfusepath{clip}%
\pgfsetbuttcap%
\pgfsetmiterjoin%
\definecolor{currentfill}{rgb}{0.971165,0.988466,0.704037}%
\pgfsetfillcolor{currentfill}%
\pgfsetlinewidth{0.000000pt}%
\definecolor{currentstroke}{rgb}{0.000000,0.000000,0.000000}%
\pgfsetstrokecolor{currentstroke}%
\pgfsetstrokeopacity{0.000000}%
\pgfsetdash{}{0pt}%
\pgfpathmoveto{\pgfqpoint{6.023213in}{1.402714in}}%
\pgfpathlineto{\pgfqpoint{6.049409in}{1.427692in}}%
\pgfpathlineto{\pgfqpoint{6.052665in}{1.436570in}}%
\pgfpathlineto{\pgfqpoint{6.060341in}{1.444171in}}%
\pgfpathlineto{\pgfqpoint{6.054581in}{1.455238in}}%
\pgfpathlineto{\pgfqpoint{6.047360in}{1.461711in}}%
\pgfpathlineto{\pgfqpoint{6.045273in}{1.473182in}}%
\pgfpathlineto{\pgfqpoint{6.072151in}{1.484959in}}%
\pgfpathlineto{\pgfqpoint{6.094405in}{1.488685in}}%
\pgfpathlineto{\pgfqpoint{6.106363in}{1.488934in}}%
\pgfpathlineto{\pgfqpoint{6.115473in}{1.484433in}}%
\pgfpathlineto{\pgfqpoint{6.124352in}{1.488453in}}%
\pgfpathlineto{\pgfqpoint{6.146010in}{1.492956in}}%
\pgfpathlineto{\pgfqpoint{6.153493in}{1.498769in}}%
\pgfpathlineto{\pgfqpoint{6.164589in}{1.511638in}}%
\pgfpathlineto{\pgfqpoint{6.174683in}{1.517325in}}%
\pgfpathlineto{\pgfqpoint{6.175395in}{1.522779in}}%
\pgfpathlineto{\pgfqpoint{6.170097in}{1.535222in}}%
\pgfpathlineto{\pgfqpoint{6.173927in}{1.542544in}}%
\pgfpathlineto{\pgfqpoint{6.168770in}{1.550418in}}%
\pgfpathlineto{\pgfqpoint{6.160873in}{1.550968in}}%
\pgfpathlineto{\pgfqpoint{6.180638in}{1.574751in}}%
\pgfpathlineto{\pgfqpoint{6.182985in}{1.583815in}}%
\pgfpathlineto{\pgfqpoint{6.198541in}{1.606931in}}%
\pgfpathlineto{\pgfqpoint{6.212981in}{1.619440in}}%
\pgfpathlineto{\pgfqpoint{6.222897in}{1.624664in}}%
\pgfpathlineto{\pgfqpoint{6.255342in}{1.632016in}}%
\pgfpathlineto{\pgfqpoint{6.285752in}{1.640318in}}%
\pgfpathlineto{\pgfqpoint{6.289780in}{1.627285in}}%
\pgfpathlineto{\pgfqpoint{6.290065in}{1.615202in}}%
\pgfpathlineto{\pgfqpoint{6.297002in}{1.604233in}}%
\pgfpathlineto{\pgfqpoint{6.299140in}{1.593064in}}%
\pgfpathlineto{\pgfqpoint{6.296474in}{1.577911in}}%
\pgfpathlineto{\pgfqpoint{6.304593in}{1.557473in}}%
\pgfpathlineto{\pgfqpoint{6.313848in}{1.546108in}}%
\pgfpathlineto{\pgfqpoint{6.321007in}{1.509167in}}%
\pgfpathlineto{\pgfqpoint{6.324865in}{1.496989in}}%
\pgfpathlineto{\pgfqpoint{6.323869in}{1.451627in}}%
\pgfpathlineto{\pgfqpoint{6.325022in}{1.449880in}}%
\pgfpathlineto{\pgfqpoint{6.333450in}{1.401092in}}%
\pgfpathlineto{\pgfqpoint{6.338176in}{1.396647in}}%
\pgfpathlineto{\pgfqpoint{6.328010in}{1.386769in}}%
\pgfpathlineto{\pgfqpoint{6.333026in}{1.381100in}}%
\pgfpathlineto{\pgfqpoint{6.328617in}{1.372526in}}%
\pgfpathlineto{\pgfqpoint{6.328654in}{1.368875in}}%
\pgfpathlineto{\pgfqpoint{6.323142in}{1.365581in}}%
\pgfpathlineto{\pgfqpoint{6.320458in}{1.358300in}}%
\pgfpathlineto{\pgfqpoint{6.321310in}{1.378322in}}%
\pgfpathlineto{\pgfqpoint{6.304225in}{1.382730in}}%
\pgfpathlineto{\pgfqpoint{6.277507in}{1.391844in}}%
\pgfpathlineto{\pgfqpoint{6.273728in}{1.396333in}}%
\pgfpathlineto{\pgfqpoint{6.262564in}{1.397408in}}%
\pgfpathlineto{\pgfqpoint{6.256142in}{1.404090in}}%
\pgfpathlineto{\pgfqpoint{6.252676in}{1.417397in}}%
\pgfpathlineto{\pgfqpoint{6.243572in}{1.419052in}}%
\pgfpathlineto{\pgfqpoint{6.237484in}{1.426033in}}%
\pgfpathlineto{\pgfqpoint{6.159972in}{1.410406in}}%
\pgfpathlineto{\pgfqpoint{6.122928in}{1.402745in}}%
\pgfpathlineto{\pgfqpoint{6.066683in}{1.392479in}}%
\pgfpathlineto{\pgfqpoint{6.026180in}{1.385646in}}%
\pgfpathlineto{\pgfqpoint{6.023213in}{1.402714in}}%
\pgfpathclose%
\pgfusepath{fill}%
\end{pgfscope}%
\begin{pgfscope}%
\pgfpathrectangle{\pgfqpoint{3.625000in}{0.100000in}}{\pgfqpoint{2.989028in}{1.913466in}}%
\pgfusepath{clip}%
\pgfsetbuttcap%
\pgfsetmiterjoin%
\definecolor{currentfill}{rgb}{0.971165,0.988466,0.704037}%
\pgfsetfillcolor{currentfill}%
\pgfsetlinewidth{0.000000pt}%
\definecolor{currentstroke}{rgb}{0.000000,0.000000,0.000000}%
\pgfsetstrokecolor{currentstroke}%
\pgfsetstrokeopacity{0.000000}%
\pgfsetdash{}{0pt}%
\pgfpathmoveto{\pgfqpoint{6.334390in}{1.354233in}}%
\pgfpathlineto{\pgfqpoint{6.322403in}{1.350498in}}%
\pgfpathlineto{\pgfqpoint{6.320386in}{1.353932in}}%
\pgfpathlineto{\pgfqpoint{6.324234in}{1.365456in}}%
\pgfpathlineto{\pgfqpoint{6.331360in}{1.366591in}}%
\pgfpathlineto{\pgfqpoint{6.330722in}{1.370219in}}%
\pgfpathlineto{\pgfqpoint{6.337120in}{1.375665in}}%
\pgfpathlineto{\pgfqpoint{6.355618in}{1.380004in}}%
\pgfpathlineto{\pgfqpoint{6.363853in}{1.386571in}}%
\pgfpathlineto{\pgfqpoint{6.382353in}{1.392045in}}%
\pgfpathlineto{\pgfqpoint{6.390790in}{1.390042in}}%
\pgfpathlineto{\pgfqpoint{6.397888in}{1.398866in}}%
\pgfpathlineto{\pgfqpoint{6.408615in}{1.400032in}}%
\pgfpathlineto{\pgfqpoint{6.390314in}{1.382820in}}%
\pgfpathlineto{\pgfqpoint{6.334390in}{1.354233in}}%
\pgfpathclose%
\pgfusepath{fill}%
\end{pgfscope}%
\begin{pgfscope}%
\pgfpathrectangle{\pgfqpoint{3.625000in}{0.100000in}}{\pgfqpoint{2.989028in}{1.913466in}}%
\pgfusepath{clip}%
\pgfsetbuttcap%
\pgfsetmiterjoin%
\definecolor{currentfill}{rgb}{0.874740,0.949712,0.601615}%
\pgfsetfillcolor{currentfill}%
\pgfsetlinewidth{0.000000pt}%
\definecolor{currentstroke}{rgb}{0.000000,0.000000,0.000000}%
\pgfsetstrokecolor{currentstroke}%
\pgfsetstrokeopacity{0.000000}%
\pgfsetdash{}{0pt}%
\pgfpathmoveto{\pgfqpoint{6.065072in}{1.240581in}}%
\pgfpathlineto{\pgfqpoint{6.013234in}{1.232010in}}%
\pgfpathlineto{\pgfqpoint{6.003831in}{1.291252in}}%
\pgfpathlineto{\pgfqpoint{5.989869in}{1.378577in}}%
\pgfpathlineto{\pgfqpoint{6.023213in}{1.402714in}}%
\pgfpathlineto{\pgfqpoint{6.026180in}{1.385646in}}%
\pgfpathlineto{\pgfqpoint{6.066683in}{1.392479in}}%
\pgfpathlineto{\pgfqpoint{6.122928in}{1.402745in}}%
\pgfpathlineto{\pgfqpoint{6.159972in}{1.410406in}}%
\pgfpathlineto{\pgfqpoint{6.237484in}{1.426033in}}%
\pgfpathlineto{\pgfqpoint{6.243572in}{1.419052in}}%
\pgfpathlineto{\pgfqpoint{6.252676in}{1.417397in}}%
\pgfpathlineto{\pgfqpoint{6.256142in}{1.404090in}}%
\pgfpathlineto{\pgfqpoint{6.262564in}{1.397408in}}%
\pgfpathlineto{\pgfqpoint{6.273728in}{1.396333in}}%
\pgfpathlineto{\pgfqpoint{6.277507in}{1.391844in}}%
\pgfpathlineto{\pgfqpoint{6.273671in}{1.388380in}}%
\pgfpathlineto{\pgfqpoint{6.270215in}{1.376126in}}%
\pgfpathlineto{\pgfqpoint{6.262116in}{1.362357in}}%
\pgfpathlineto{\pgfqpoint{6.267552in}{1.356369in}}%
\pgfpathlineto{\pgfqpoint{6.263104in}{1.349953in}}%
\pgfpathlineto{\pgfqpoint{6.265613in}{1.335884in}}%
\pgfpathlineto{\pgfqpoint{6.273625in}{1.328499in}}%
\pgfpathlineto{\pgfqpoint{6.292634in}{1.316500in}}%
\pgfpathlineto{\pgfqpoint{6.277326in}{1.299570in}}%
\pgfpathlineto{\pgfqpoint{6.277111in}{1.293145in}}%
\pgfpathlineto{\pgfqpoint{6.264648in}{1.284858in}}%
\pgfpathlineto{\pgfqpoint{6.250867in}{1.283302in}}%
\pgfpathlineto{\pgfqpoint{6.247458in}{1.276100in}}%
\pgfpathlineto{\pgfqpoint{6.209127in}{1.267806in}}%
\pgfpathlineto{\pgfqpoint{6.164399in}{1.258797in}}%
\pgfpathlineto{\pgfqpoint{6.133655in}{1.253292in}}%
\pgfpathlineto{\pgfqpoint{6.065072in}{1.240581in}}%
\pgfpathclose%
\pgfusepath{fill}%
\end{pgfscope}%
\begin{pgfscope}%
\pgfpathrectangle{\pgfqpoint{3.625000in}{0.100000in}}{\pgfqpoint{2.989028in}{1.913466in}}%
\pgfusepath{clip}%
\pgfsetbuttcap%
\pgfsetmiterjoin%
\definecolor{currentfill}{rgb}{0.820300,0.927566,0.612687}%
\pgfsetfillcolor{currentfill}%
\pgfsetlinewidth{0.000000pt}%
\definecolor{currentstroke}{rgb}{0.000000,0.000000,0.000000}%
\pgfsetstrokecolor{currentstroke}%
\pgfsetstrokeopacity{0.000000}%
\pgfsetdash{}{0pt}%
\pgfpathmoveto{\pgfqpoint{6.325022in}{1.449880in}}%
\pgfpathlineto{\pgfqpoint{6.391000in}{1.464872in}}%
\pgfpathlineto{\pgfqpoint{6.405364in}{1.467139in}}%
\pgfpathlineto{\pgfqpoint{6.414871in}{1.429760in}}%
\pgfpathlineto{\pgfqpoint{6.413366in}{1.423067in}}%
\pgfpathlineto{\pgfqpoint{6.382827in}{1.411244in}}%
\pgfpathlineto{\pgfqpoint{6.364596in}{1.407111in}}%
\pgfpathlineto{\pgfqpoint{6.356849in}{1.397842in}}%
\pgfpathlineto{\pgfqpoint{6.333026in}{1.381100in}}%
\pgfpathlineto{\pgfqpoint{6.328010in}{1.386769in}}%
\pgfpathlineto{\pgfqpoint{6.338176in}{1.396647in}}%
\pgfpathlineto{\pgfqpoint{6.333450in}{1.401092in}}%
\pgfpathlineto{\pgfqpoint{6.325022in}{1.449880in}}%
\pgfpathclose%
\pgfusepath{fill}%
\end{pgfscope}%
\begin{pgfscope}%
\pgfpathrectangle{\pgfqpoint{3.625000in}{0.100000in}}{\pgfqpoint{2.989028in}{1.913466in}}%
\pgfusepath{clip}%
\pgfsetbuttcap%
\pgfsetmiterjoin%
\definecolor{currentfill}{rgb}{0.982699,0.993080,0.722030}%
\pgfsetfillcolor{currentfill}%
\pgfsetlinewidth{0.000000pt}%
\definecolor{currentstroke}{rgb}{0.000000,0.000000,0.000000}%
\pgfsetstrokecolor{currentstroke}%
\pgfsetstrokeopacity{0.000000}%
\pgfsetdash{}{0pt}%
\pgfpathmoveto{\pgfqpoint{6.405364in}{1.467139in}}%
\pgfpathlineto{\pgfqpoint{6.424952in}{1.472940in}}%
\pgfpathlineto{\pgfqpoint{6.430650in}{1.458903in}}%
\pgfpathlineto{\pgfqpoint{6.437183in}{1.455442in}}%
\pgfpathlineto{\pgfqpoint{6.429121in}{1.449529in}}%
\pgfpathlineto{\pgfqpoint{6.430138in}{1.432164in}}%
\pgfpathlineto{\pgfqpoint{6.413366in}{1.423067in}}%
\pgfpathlineto{\pgfqpoint{6.414871in}{1.429760in}}%
\pgfpathlineto{\pgfqpoint{6.405364in}{1.467139in}}%
\pgfpathclose%
\pgfusepath{fill}%
\end{pgfscope}%
\begin{pgfscope}%
\pgfpathrectangle{\pgfqpoint{3.625000in}{0.100000in}}{\pgfqpoint{2.989028in}{1.913466in}}%
\pgfusepath{clip}%
\pgfsetbuttcap%
\pgfsetmiterjoin%
\definecolor{currentfill}{rgb}{0.940408,0.976163,0.656055}%
\pgfsetfillcolor{currentfill}%
\pgfsetlinewidth{0.000000pt}%
\definecolor{currentstroke}{rgb}{0.000000,0.000000,0.000000}%
\pgfsetstrokecolor{currentstroke}%
\pgfsetstrokeopacity{0.000000}%
\pgfsetdash{}{0pt}%
\pgfpathmoveto{\pgfqpoint{6.262440in}{1.278737in}}%
\pgfpathlineto{\pgfqpoint{6.264648in}{1.284858in}}%
\pgfpathlineto{\pgfqpoint{6.277111in}{1.293145in}}%
\pgfpathlineto{\pgfqpoint{6.277326in}{1.299570in}}%
\pgfpathlineto{\pgfqpoint{6.292634in}{1.316500in}}%
\pgfpathlineto{\pgfqpoint{6.273625in}{1.328499in}}%
\pgfpathlineto{\pgfqpoint{6.265613in}{1.335884in}}%
\pgfpathlineto{\pgfqpoint{6.263104in}{1.349953in}}%
\pgfpathlineto{\pgfqpoint{6.267552in}{1.356369in}}%
\pgfpathlineto{\pgfqpoint{6.262116in}{1.362357in}}%
\pgfpathlineto{\pgfqpoint{6.270215in}{1.376126in}}%
\pgfpathlineto{\pgfqpoint{6.273671in}{1.388380in}}%
\pgfpathlineto{\pgfqpoint{6.277507in}{1.391844in}}%
\pgfpathlineto{\pgfqpoint{6.304225in}{1.382730in}}%
\pgfpathlineto{\pgfqpoint{6.321310in}{1.378322in}}%
\pgfpathlineto{\pgfqpoint{6.320458in}{1.358300in}}%
\pgfpathlineto{\pgfqpoint{6.310080in}{1.343116in}}%
\pgfpathlineto{\pgfqpoint{6.318632in}{1.340882in}}%
\pgfpathlineto{\pgfqpoint{6.327511in}{1.334366in}}%
\pgfpathlineto{\pgfqpoint{6.328034in}{1.316545in}}%
\pgfpathlineto{\pgfqpoint{6.325347in}{1.303928in}}%
\pgfpathlineto{\pgfqpoint{6.327135in}{1.293566in}}%
\pgfpathlineto{\pgfqpoint{6.311701in}{1.258597in}}%
\pgfpathlineto{\pgfqpoint{6.306154in}{1.242268in}}%
\pgfpathlineto{\pgfqpoint{6.298409in}{1.250205in}}%
\pgfpathlineto{\pgfqpoint{6.288139in}{1.248853in}}%
\pgfpathlineto{\pgfqpoint{6.262435in}{1.263703in}}%
\pgfpathlineto{\pgfqpoint{6.259805in}{1.271656in}}%
\pgfpathlineto{\pgfqpoint{6.262440in}{1.278737in}}%
\pgfpathclose%
\pgfusepath{fill}%
\end{pgfscope}%
\begin{pgfscope}%
\pgfpathrectangle{\pgfqpoint{3.625000in}{0.100000in}}{\pgfqpoint{2.989028in}{1.913466in}}%
\pgfusepath{clip}%
\pgfsetbuttcap%
\pgfsetmiterjoin%
\definecolor{currentfill}{rgb}{0.951942,0.980777,0.674048}%
\pgfsetfillcolor{currentfill}%
\pgfsetlinewidth{0.000000pt}%
\definecolor{currentstroke}{rgb}{0.000000,0.000000,0.000000}%
\pgfsetstrokecolor{currentstroke}%
\pgfsetstrokeopacity{0.000000}%
\pgfsetdash{}{0pt}%
\pgfpathmoveto{\pgfqpoint{5.647290in}{1.062454in}}%
\pgfpathlineto{\pgfqpoint{5.646088in}{1.078676in}}%
\pgfpathlineto{\pgfqpoint{5.650759in}{1.086724in}}%
\pgfpathlineto{\pgfqpoint{5.647681in}{1.090668in}}%
\pgfpathlineto{\pgfqpoint{5.658907in}{1.103580in}}%
\pgfpathlineto{\pgfqpoint{5.658263in}{1.106692in}}%
\pgfpathlineto{\pgfqpoint{5.669170in}{1.129219in}}%
\pgfpathlineto{\pgfqpoint{5.666966in}{1.140083in}}%
\pgfpathlineto{\pgfqpoint{5.659057in}{1.151457in}}%
\pgfpathlineto{\pgfqpoint{5.664547in}{1.165295in}}%
\pgfpathlineto{\pgfqpoint{5.657431in}{1.256358in}}%
\pgfpathlineto{\pgfqpoint{5.652306in}{1.320243in}}%
\pgfpathlineto{\pgfqpoint{5.659385in}{1.314949in}}%
\pgfpathlineto{\pgfqpoint{5.667283in}{1.315093in}}%
\pgfpathlineto{\pgfqpoint{5.685936in}{1.325898in}}%
\pgfpathlineto{\pgfqpoint{5.743150in}{1.331236in}}%
\pgfpathlineto{\pgfqpoint{5.785498in}{1.335699in}}%
\pgfpathlineto{\pgfqpoint{5.785872in}{1.331556in}}%
\pgfpathlineto{\pgfqpoint{5.795539in}{1.244011in}}%
\pgfpathlineto{\pgfqpoint{5.803860in}{1.162554in}}%
\pgfpathlineto{\pgfqpoint{5.800273in}{1.158730in}}%
\pgfpathlineto{\pgfqpoint{5.805783in}{1.149238in}}%
\pgfpathlineto{\pgfqpoint{5.805755in}{1.142397in}}%
\pgfpathlineto{\pgfqpoint{5.797893in}{1.140679in}}%
\pgfpathlineto{\pgfqpoint{5.789096in}{1.134084in}}%
\pgfpathlineto{\pgfqpoint{5.783139in}{1.136680in}}%
\pgfpathlineto{\pgfqpoint{5.774225in}{1.132459in}}%
\pgfpathlineto{\pgfqpoint{5.776983in}{1.123981in}}%
\pgfpathlineto{\pgfqpoint{5.767820in}{1.115463in}}%
\pgfpathlineto{\pgfqpoint{5.765290in}{1.105610in}}%
\pgfpathlineto{\pgfqpoint{5.758989in}{1.103993in}}%
\pgfpathlineto{\pgfqpoint{5.754287in}{1.096531in}}%
\pgfpathlineto{\pgfqpoint{5.754903in}{1.089016in}}%
\pgfpathlineto{\pgfqpoint{5.749371in}{1.083732in}}%
\pgfpathlineto{\pgfqpoint{5.741048in}{1.084548in}}%
\pgfpathlineto{\pgfqpoint{5.734721in}{1.092656in}}%
\pgfpathlineto{\pgfqpoint{5.724000in}{1.084848in}}%
\pgfpathlineto{\pgfqpoint{5.724753in}{1.078025in}}%
\pgfpathlineto{\pgfqpoint{5.712829in}{1.074039in}}%
\pgfpathlineto{\pgfqpoint{5.709834in}{1.079056in}}%
\pgfpathlineto{\pgfqpoint{5.698226in}{1.073367in}}%
\pgfpathlineto{\pgfqpoint{5.693985in}{1.065222in}}%
\pgfpathlineto{\pgfqpoint{5.680006in}{1.073574in}}%
\pgfpathlineto{\pgfqpoint{5.657850in}{1.068041in}}%
\pgfpathlineto{\pgfqpoint{5.647290in}{1.062454in}}%
\pgfpathclose%
\pgfusepath{fill}%
\end{pgfscope}%
\begin{pgfscope}%
\pgfpathrectangle{\pgfqpoint{3.625000in}{0.100000in}}{\pgfqpoint{2.989028in}{1.913466in}}%
\pgfusepath{clip}%
\pgfsetbuttcap%
\pgfsetmiterjoin%
\definecolor{currentfill}{rgb}{0.905805,0.962322,0.602076}%
\pgfsetfillcolor{currentfill}%
\pgfsetlinewidth{0.000000pt}%
\definecolor{currentstroke}{rgb}{0.000000,0.000000,0.000000}%
\pgfsetstrokecolor{currentstroke}%
\pgfsetstrokeopacity{0.000000}%
\pgfsetdash{}{0pt}%
\pgfpathmoveto{\pgfqpoint{4.084744in}{1.489057in}}%
\pgfpathlineto{\pgfqpoint{4.116784in}{1.480141in}}%
\pgfpathlineto{\pgfqpoint{4.169893in}{1.466447in}}%
\pgfpathlineto{\pgfqpoint{4.224946in}{1.452341in}}%
\pgfpathlineto{\pgfqpoint{4.275689in}{1.440103in}}%
\pgfpathlineto{\pgfqpoint{4.319534in}{1.430177in}}%
\pgfpathlineto{\pgfqpoint{4.366442in}{1.419862in}}%
\pgfpathlineto{\pgfqpoint{4.352889in}{1.355901in}}%
\pgfpathlineto{\pgfqpoint{4.340815in}{1.299101in}}%
\pgfpathlineto{\pgfqpoint{4.321008in}{1.207454in}}%
\pgfpathlineto{\pgfqpoint{4.306137in}{1.138280in}}%
\pgfpathlineto{\pgfqpoint{4.298103in}{1.099678in}}%
\pgfpathlineto{\pgfqpoint{4.287801in}{1.049576in}}%
\pgfpathlineto{\pgfqpoint{4.280674in}{1.039412in}}%
\pgfpathlineto{\pgfqpoint{4.274938in}{1.039072in}}%
\pgfpathlineto{\pgfqpoint{4.270846in}{1.047945in}}%
\pgfpathlineto{\pgfqpoint{4.261451in}{1.051167in}}%
\pgfpathlineto{\pgfqpoint{4.251328in}{1.050037in}}%
\pgfpathlineto{\pgfqpoint{4.248454in}{1.042780in}}%
\pgfpathlineto{\pgfqpoint{4.248155in}{1.017578in}}%
\pgfpathlineto{\pgfqpoint{4.245123in}{1.011825in}}%
\pgfpathlineto{\pgfqpoint{4.246856in}{0.991592in}}%
\pgfpathlineto{\pgfqpoint{4.240525in}{0.978113in}}%
\pgfpathlineto{\pgfqpoint{4.189188in}{1.057164in}}%
\pgfpathlineto{\pgfqpoint{4.138630in}{1.133980in}}%
\pgfpathlineto{\pgfqpoint{4.112312in}{1.174260in}}%
\pgfpathlineto{\pgfqpoint{4.090358in}{1.209028in}}%
\pgfpathlineto{\pgfqpoint{4.062695in}{1.252036in}}%
\pgfpathlineto{\pgfqpoint{4.031381in}{1.300279in}}%
\pgfpathlineto{\pgfqpoint{4.044255in}{1.346069in}}%
\pgfpathlineto{\pgfqpoint{4.070163in}{1.437904in}}%
\pgfpathlineto{\pgfqpoint{4.084744in}{1.489057in}}%
\pgfpathclose%
\pgfusepath{fill}%
\end{pgfscope}%
\begin{pgfscope}%
\pgfpathrectangle{\pgfqpoint{3.625000in}{0.100000in}}{\pgfqpoint{2.989028in}{1.913466in}}%
\pgfusepath{clip}%
\pgfsetbuttcap%
\pgfsetmiterjoin%
\definecolor{currentfill}{rgb}{0.847520,0.938639,0.607151}%
\pgfsetfillcolor{currentfill}%
\pgfsetlinewidth{0.000000pt}%
\definecolor{currentstroke}{rgb}{0.000000,0.000000,0.000000}%
\pgfsetstrokecolor{currentstroke}%
\pgfsetstrokeopacity{0.000000}%
\pgfsetdash{}{0pt}%
\pgfpathmoveto{\pgfqpoint{4.366442in}{1.419862in}}%
\pgfpathlineto{\pgfqpoint{4.416592in}{1.409886in}}%
\pgfpathlineto{\pgfqpoint{4.509436in}{1.391963in}}%
\pgfpathlineto{\pgfqpoint{4.497815in}{1.327496in}}%
\pgfpathlineto{\pgfqpoint{4.548928in}{1.318744in}}%
\pgfpathlineto{\pgfqpoint{4.595495in}{1.311293in}}%
\pgfpathlineto{\pgfqpoint{4.587404in}{1.260326in}}%
\pgfpathlineto{\pgfqpoint{4.578829in}{1.205365in}}%
\pgfpathlineto{\pgfqpoint{4.567350in}{1.133200in}}%
\pgfpathlineto{\pgfqpoint{4.567053in}{1.127151in}}%
\pgfpathlineto{\pgfqpoint{4.555137in}{1.052362in}}%
\pgfpathlineto{\pgfqpoint{4.506086in}{1.059964in}}%
\pgfpathlineto{\pgfqpoint{4.481091in}{1.064984in}}%
\pgfpathlineto{\pgfqpoint{4.390746in}{1.080865in}}%
\pgfpathlineto{\pgfqpoint{4.356724in}{1.087575in}}%
\pgfpathlineto{\pgfqpoint{4.298103in}{1.099678in}}%
\pgfpathlineto{\pgfqpoint{4.306137in}{1.138280in}}%
\pgfpathlineto{\pgfqpoint{4.321008in}{1.207454in}}%
\pgfpathlineto{\pgfqpoint{4.340815in}{1.299101in}}%
\pgfpathlineto{\pgfqpoint{4.352889in}{1.355901in}}%
\pgfpathlineto{\pgfqpoint{4.366442in}{1.419862in}}%
\pgfpathclose%
\pgfusepath{fill}%
\end{pgfscope}%
\begin{pgfscope}%
\pgfpathrectangle{\pgfqpoint{3.625000in}{0.100000in}}{\pgfqpoint{2.989028in}{1.913466in}}%
\pgfusepath{clip}%
\pgfsetbuttcap%
\pgfsetmiterjoin%
\definecolor{currentfill}{rgb}{0.892887,0.957093,0.597924}%
\pgfsetfillcolor{currentfill}%
\pgfsetlinewidth{0.000000pt}%
\definecolor{currentstroke}{rgb}{0.000000,0.000000,0.000000}%
\pgfsetstrokecolor{currentstroke}%
\pgfsetstrokeopacity{0.000000}%
\pgfsetdash{}{0pt}%
\pgfpathmoveto{\pgfqpoint{3.888713in}{1.549891in}}%
\pgfpathlineto{\pgfqpoint{3.906532in}{1.543676in}}%
\pgfpathlineto{\pgfqpoint{3.977913in}{1.521452in}}%
\pgfpathlineto{\pgfqpoint{4.043578in}{1.500784in}}%
\pgfpathlineto{\pgfqpoint{4.084744in}{1.489057in}}%
\pgfpathlineto{\pgfqpoint{4.070163in}{1.437904in}}%
\pgfpathlineto{\pgfqpoint{4.044255in}{1.346069in}}%
\pgfpathlineto{\pgfqpoint{4.031381in}{1.300279in}}%
\pgfpathlineto{\pgfqpoint{4.062695in}{1.252036in}}%
\pgfpathlineto{\pgfqpoint{4.090358in}{1.209028in}}%
\pgfpathlineto{\pgfqpoint{4.112312in}{1.174260in}}%
\pgfpathlineto{\pgfqpoint{4.138630in}{1.133980in}}%
\pgfpathlineto{\pgfqpoint{4.189188in}{1.057164in}}%
\pgfpathlineto{\pgfqpoint{4.240525in}{0.978113in}}%
\pgfpathlineto{\pgfqpoint{4.238464in}{0.970274in}}%
\pgfpathlineto{\pgfqpoint{4.244656in}{0.957789in}}%
\pgfpathlineto{\pgfqpoint{4.245881in}{0.940711in}}%
\pgfpathlineto{\pgfqpoint{4.254874in}{0.932426in}}%
\pgfpathlineto{\pgfqpoint{4.255230in}{0.925750in}}%
\pgfpathlineto{\pgfqpoint{4.239127in}{0.918174in}}%
\pgfpathlineto{\pgfqpoint{4.231455in}{0.910589in}}%
\pgfpathlineto{\pgfqpoint{4.229063in}{0.893839in}}%
\pgfpathlineto{\pgfqpoint{4.225180in}{0.884704in}}%
\pgfpathlineto{\pgfqpoint{4.216998in}{0.877004in}}%
\pgfpathlineto{\pgfqpoint{4.208861in}{0.856982in}}%
\pgfpathlineto{\pgfqpoint{4.220303in}{0.846552in}}%
\pgfpathlineto{\pgfqpoint{4.218826in}{0.837954in}}%
\pgfpathlineto{\pgfqpoint{4.209444in}{0.831972in}}%
\pgfpathlineto{\pgfqpoint{4.202975in}{0.833027in}}%
\pgfpathlineto{\pgfqpoint{4.126622in}{0.843623in}}%
\pgfpathlineto{\pgfqpoint{4.070229in}{0.851623in}}%
\pgfpathlineto{\pgfqpoint{4.072695in}{0.860728in}}%
\pgfpathlineto{\pgfqpoint{4.069024in}{0.875844in}}%
\pgfpathlineto{\pgfqpoint{4.068653in}{0.891092in}}%
\pgfpathlineto{\pgfqpoint{4.066254in}{0.900014in}}%
\pgfpathlineto{\pgfqpoint{4.058864in}{0.912772in}}%
\pgfpathlineto{\pgfqpoint{4.037605in}{0.942205in}}%
\pgfpathlineto{\pgfqpoint{4.030643in}{0.945811in}}%
\pgfpathlineto{\pgfqpoint{4.021672in}{0.945755in}}%
\pgfpathlineto{\pgfqpoint{4.023721in}{0.955040in}}%
\pgfpathlineto{\pgfqpoint{4.019473in}{0.966651in}}%
\pgfpathlineto{\pgfqpoint{3.998608in}{0.972422in}}%
\pgfpathlineto{\pgfqpoint{3.985889in}{0.983108in}}%
\pgfpathlineto{\pgfqpoint{3.984834in}{0.989647in}}%
\pgfpathlineto{\pgfqpoint{3.970189in}{1.005835in}}%
\pgfpathlineto{\pgfqpoint{3.956249in}{1.008932in}}%
\pgfpathlineto{\pgfqpoint{3.943332in}{1.017159in}}%
\pgfpathlineto{\pgfqpoint{3.926344in}{1.020005in}}%
\pgfpathlineto{\pgfqpoint{3.919086in}{1.030977in}}%
\pgfpathlineto{\pgfqpoint{3.925979in}{1.048345in}}%
\pgfpathlineto{\pgfqpoint{3.923871in}{1.052250in}}%
\pgfpathlineto{\pgfqpoint{3.929611in}{1.066728in}}%
\pgfpathlineto{\pgfqpoint{3.919400in}{1.074432in}}%
\pgfpathlineto{\pgfqpoint{3.922713in}{1.088401in}}%
\pgfpathlineto{\pgfqpoint{3.917259in}{1.091971in}}%
\pgfpathlineto{\pgfqpoint{3.912550in}{1.105174in}}%
\pgfpathlineto{\pgfqpoint{3.906867in}{1.109200in}}%
\pgfpathlineto{\pgfqpoint{3.906440in}{1.118736in}}%
\pgfpathlineto{\pgfqpoint{3.901988in}{1.125461in}}%
\pgfpathlineto{\pgfqpoint{3.895279in}{1.148109in}}%
\pgfpathlineto{\pgfqpoint{3.887938in}{1.158956in}}%
\pgfpathlineto{\pgfqpoint{3.889533in}{1.177370in}}%
\pgfpathlineto{\pgfqpoint{3.898166in}{1.179223in}}%
\pgfpathlineto{\pgfqpoint{3.903787in}{1.189211in}}%
\pgfpathlineto{\pgfqpoint{3.900390in}{1.200043in}}%
\pgfpathlineto{\pgfqpoint{3.891209in}{1.201851in}}%
\pgfpathlineto{\pgfqpoint{3.886629in}{1.206918in}}%
\pgfpathlineto{\pgfqpoint{3.879211in}{1.225555in}}%
\pgfpathlineto{\pgfqpoint{3.882678in}{1.232248in}}%
\pgfpathlineto{\pgfqpoint{3.880219in}{1.244711in}}%
\pgfpathlineto{\pgfqpoint{3.885664in}{1.260852in}}%
\pgfpathlineto{\pgfqpoint{3.892043in}{1.254907in}}%
\pgfpathlineto{\pgfqpoint{3.889148in}{1.247890in}}%
\pgfpathlineto{\pgfqpoint{3.900202in}{1.236766in}}%
\pgfpathlineto{\pgfqpoint{3.899497in}{1.253285in}}%
\pgfpathlineto{\pgfqpoint{3.894760in}{1.257715in}}%
\pgfpathlineto{\pgfqpoint{3.900174in}{1.272240in}}%
\pgfpathlineto{\pgfqpoint{3.920652in}{1.269927in}}%
\pgfpathlineto{\pgfqpoint{3.917896in}{1.275328in}}%
\pgfpathlineto{\pgfqpoint{3.904381in}{1.274797in}}%
\pgfpathlineto{\pgfqpoint{3.897951in}{1.282977in}}%
\pgfpathlineto{\pgfqpoint{3.889858in}{1.275713in}}%
\pgfpathlineto{\pgfqpoint{3.885564in}{1.263570in}}%
\pgfpathlineto{\pgfqpoint{3.874102in}{1.279909in}}%
\pgfpathlineto{\pgfqpoint{3.869687in}{1.282884in}}%
\pgfpathlineto{\pgfqpoint{3.871369in}{1.300667in}}%
\pgfpathlineto{\pgfqpoint{3.867875in}{1.311153in}}%
\pgfpathlineto{\pgfqpoint{3.861542in}{1.321006in}}%
\pgfpathlineto{\pgfqpoint{3.848573in}{1.351193in}}%
\pgfpathlineto{\pgfqpoint{3.852813in}{1.357869in}}%
\pgfpathlineto{\pgfqpoint{3.852764in}{1.378965in}}%
\pgfpathlineto{\pgfqpoint{3.859763in}{1.390732in}}%
\pgfpathlineto{\pgfqpoint{3.861368in}{1.409121in}}%
\pgfpathlineto{\pgfqpoint{3.854768in}{1.430191in}}%
\pgfpathlineto{\pgfqpoint{3.846012in}{1.443602in}}%
\pgfpathlineto{\pgfqpoint{3.847574in}{1.455704in}}%
\pgfpathlineto{\pgfqpoint{3.872163in}{1.484994in}}%
\pgfpathlineto{\pgfqpoint{3.873384in}{1.494985in}}%
\pgfpathlineto{\pgfqpoint{3.884464in}{1.514044in}}%
\pgfpathlineto{\pgfqpoint{3.886001in}{1.532109in}}%
\pgfpathlineto{\pgfqpoint{3.882439in}{1.536719in}}%
\pgfpathlineto{\pgfqpoint{3.888713in}{1.549891in}}%
\pgfpathclose%
\pgfusepath{fill}%
\end{pgfscope}%
\begin{pgfscope}%
\pgfpathrectangle{\pgfqpoint{3.625000in}{0.100000in}}{\pgfqpoint{2.989028in}{1.913466in}}%
\pgfusepath{clip}%
\pgfsetbuttcap%
\pgfsetmiterjoin%
\definecolor{currentfill}{rgb}{0.955786,0.982314,0.680046}%
\pgfsetfillcolor{currentfill}%
\pgfsetlinewidth{0.000000pt}%
\definecolor{currentstroke}{rgb}{0.000000,0.000000,0.000000}%
\pgfsetstrokecolor{currentstroke}%
\pgfsetstrokeopacity{0.000000}%
\pgfsetdash{}{0pt}%
\pgfpathmoveto{\pgfqpoint{5.803860in}{1.162554in}}%
\pgfpathlineto{\pgfqpoint{5.795539in}{1.244011in}}%
\pgfpathlineto{\pgfqpoint{5.785872in}{1.331556in}}%
\pgfpathlineto{\pgfqpoint{5.849180in}{1.340984in}}%
\pgfpathlineto{\pgfqpoint{5.865982in}{1.336611in}}%
\pgfpathlineto{\pgfqpoint{5.874041in}{1.331864in}}%
\pgfpathlineto{\pgfqpoint{5.884139in}{1.333180in}}%
\pgfpathlineto{\pgfqpoint{5.897453in}{1.325321in}}%
\pgfpathlineto{\pgfqpoint{5.922252in}{1.337018in}}%
\pgfpathlineto{\pgfqpoint{5.935938in}{1.337409in}}%
\pgfpathlineto{\pgfqpoint{5.951920in}{1.355239in}}%
\pgfpathlineto{\pgfqpoint{5.968180in}{1.366084in}}%
\pgfpathlineto{\pgfqpoint{5.989869in}{1.378577in}}%
\pgfpathlineto{\pgfqpoint{6.003831in}{1.291252in}}%
\pgfpathlineto{\pgfqpoint{5.997344in}{1.285643in}}%
\pgfpathlineto{\pgfqpoint{6.001532in}{1.280492in}}%
\pgfpathlineto{\pgfqpoint{6.003201in}{1.269199in}}%
\pgfpathlineto{\pgfqpoint{5.999951in}{1.258593in}}%
\pgfpathlineto{\pgfqpoint{5.999521in}{1.243089in}}%
\pgfpathlineto{\pgfqpoint{5.996469in}{1.222911in}}%
\pgfpathlineto{\pgfqpoint{5.981028in}{1.204915in}}%
\pgfpathlineto{\pgfqpoint{5.974528in}{1.201094in}}%
\pgfpathlineto{\pgfqpoint{5.969473in}{1.204383in}}%
\pgfpathlineto{\pgfqpoint{5.956963in}{1.187230in}}%
\pgfpathlineto{\pgfqpoint{5.958135in}{1.170714in}}%
\pgfpathlineto{\pgfqpoint{5.944203in}{1.174686in}}%
\pgfpathlineto{\pgfqpoint{5.937606in}{1.158191in}}%
\pgfpathlineto{\pgfqpoint{5.940941in}{1.146502in}}%
\pgfpathlineto{\pgfqpoint{5.935720in}{1.144776in}}%
\pgfpathlineto{\pgfqpoint{5.934990in}{1.135540in}}%
\pgfpathlineto{\pgfqpoint{5.922176in}{1.131815in}}%
\pgfpathlineto{\pgfqpoint{5.915518in}{1.139272in}}%
\pgfpathlineto{\pgfqpoint{5.906950in}{1.142163in}}%
\pgfpathlineto{\pgfqpoint{5.904937in}{1.149724in}}%
\pgfpathlineto{\pgfqpoint{5.897191in}{1.148394in}}%
\pgfpathlineto{\pgfqpoint{5.892116in}{1.141440in}}%
\pgfpathlineto{\pgfqpoint{5.884855in}{1.138994in}}%
\pgfpathlineto{\pgfqpoint{5.872031in}{1.143916in}}%
\pgfpathlineto{\pgfqpoint{5.864938in}{1.138060in}}%
\pgfpathlineto{\pgfqpoint{5.854857in}{1.144988in}}%
\pgfpathlineto{\pgfqpoint{5.835517in}{1.147114in}}%
\pgfpathlineto{\pgfqpoint{5.829684in}{1.159700in}}%
\pgfpathlineto{\pgfqpoint{5.822293in}{1.165276in}}%
\pgfpathlineto{\pgfqpoint{5.815128in}{1.161707in}}%
\pgfpathlineto{\pgfqpoint{5.803860in}{1.162554in}}%
\pgfpathclose%
\pgfusepath{fill}%
\end{pgfscope}%
\begin{pgfscope}%
\pgfpathrectangle{\pgfqpoint{3.625000in}{0.100000in}}{\pgfqpoint{2.989028in}{1.913466in}}%
\pgfusepath{clip}%
\pgfsetbuttcap%
\pgfsetmiterjoin%
\definecolor{currentfill}{rgb}{0.944252,0.977701,0.662053}%
\pgfsetfillcolor{currentfill}%
\pgfsetlinewidth{0.000000pt}%
\definecolor{currentstroke}{rgb}{0.000000,0.000000,0.000000}%
\pgfsetstrokecolor{currentstroke}%
\pgfsetstrokeopacity{0.000000}%
\pgfsetdash{}{0pt}%
\pgfpathmoveto{\pgfqpoint{5.594266in}{1.005020in}}%
\pgfpathlineto{\pgfqpoint{5.586849in}{1.011045in}}%
\pgfpathlineto{\pgfqpoint{5.580801in}{1.008178in}}%
\pgfpathlineto{\pgfqpoint{5.573069in}{1.022616in}}%
\pgfpathlineto{\pgfqpoint{5.577019in}{1.031690in}}%
\pgfpathlineto{\pgfqpoint{5.571326in}{1.041904in}}%
\pgfpathlineto{\pgfqpoint{5.571020in}{1.049946in}}%
\pgfpathlineto{\pgfqpoint{5.563328in}{1.052811in}}%
\pgfpathlineto{\pgfqpoint{5.559760in}{1.058870in}}%
\pgfpathlineto{\pgfqpoint{5.544719in}{1.066437in}}%
\pgfpathlineto{\pgfqpoint{5.531657in}{1.075760in}}%
\pgfpathlineto{\pgfqpoint{5.525587in}{1.082799in}}%
\pgfpathlineto{\pgfqpoint{5.525462in}{1.091383in}}%
\pgfpathlineto{\pgfqpoint{5.533584in}{1.107865in}}%
\pgfpathlineto{\pgfqpoint{5.536082in}{1.120472in}}%
\pgfpathlineto{\pgfqpoint{5.529465in}{1.127595in}}%
\pgfpathlineto{\pgfqpoint{5.520702in}{1.130279in}}%
\pgfpathlineto{\pgfqpoint{5.510070in}{1.124410in}}%
\pgfpathlineto{\pgfqpoint{5.505570in}{1.134486in}}%
\pgfpathlineto{\pgfqpoint{5.503299in}{1.148144in}}%
\pgfpathlineto{\pgfqpoint{5.487630in}{1.160319in}}%
\pgfpathlineto{\pgfqpoint{5.484499in}{1.165722in}}%
\pgfpathlineto{\pgfqpoint{5.470188in}{1.177954in}}%
\pgfpathlineto{\pgfqpoint{5.465703in}{1.186846in}}%
\pgfpathlineto{\pgfqpoint{5.461659in}{1.204480in}}%
\pgfpathlineto{\pgfqpoint{5.464431in}{1.220143in}}%
\pgfpathlineto{\pgfqpoint{5.468130in}{1.222326in}}%
\pgfpathlineto{\pgfqpoint{5.467465in}{1.235438in}}%
\pgfpathlineto{\pgfqpoint{5.477907in}{1.239332in}}%
\pgfpathlineto{\pgfqpoint{5.481060in}{1.251098in}}%
\pgfpathlineto{\pgfqpoint{5.487068in}{1.259018in}}%
\pgfpathlineto{\pgfqpoint{5.486764in}{1.269093in}}%
\pgfpathlineto{\pgfqpoint{5.479318in}{1.277106in}}%
\pgfpathlineto{\pgfqpoint{5.481087in}{1.288322in}}%
\pgfpathlineto{\pgfqpoint{5.500389in}{1.293200in}}%
\pgfpathlineto{\pgfqpoint{5.515197in}{1.302104in}}%
\pgfpathlineto{\pgfqpoint{5.516751in}{1.313315in}}%
\pgfpathlineto{\pgfqpoint{5.521904in}{1.316845in}}%
\pgfpathlineto{\pgfqpoint{5.523875in}{1.328623in}}%
\pgfpathlineto{\pgfqpoint{5.522291in}{1.336402in}}%
\pgfpathlineto{\pgfqpoint{5.512167in}{1.342864in}}%
\pgfpathlineto{\pgfqpoint{5.508091in}{1.352497in}}%
\pgfpathlineto{\pgfqpoint{5.498100in}{1.361809in}}%
\pgfpathlineto{\pgfqpoint{5.580206in}{1.365308in}}%
\pgfpathlineto{\pgfqpoint{5.635291in}{1.369221in}}%
\pgfpathlineto{\pgfqpoint{5.634283in}{1.357634in}}%
\pgfpathlineto{\pgfqpoint{5.643671in}{1.341651in}}%
\pgfpathlineto{\pgfqpoint{5.647612in}{1.327996in}}%
\pgfpathlineto{\pgfqpoint{5.652306in}{1.320243in}}%
\pgfpathlineto{\pgfqpoint{5.657431in}{1.256358in}}%
\pgfpathlineto{\pgfqpoint{5.664547in}{1.165295in}}%
\pgfpathlineto{\pgfqpoint{5.659057in}{1.151457in}}%
\pgfpathlineto{\pgfqpoint{5.666966in}{1.140083in}}%
\pgfpathlineto{\pgfqpoint{5.669170in}{1.129219in}}%
\pgfpathlineto{\pgfqpoint{5.658263in}{1.106692in}}%
\pgfpathlineto{\pgfqpoint{5.658907in}{1.103580in}}%
\pgfpathlineto{\pgfqpoint{5.647681in}{1.090668in}}%
\pgfpathlineto{\pgfqpoint{5.650759in}{1.086724in}}%
\pgfpathlineto{\pgfqpoint{5.646088in}{1.078676in}}%
\pgfpathlineto{\pgfqpoint{5.647290in}{1.062454in}}%
\pgfpathlineto{\pgfqpoint{5.641617in}{1.052499in}}%
\pgfpathlineto{\pgfqpoint{5.646231in}{1.040736in}}%
\pgfpathlineto{\pgfqpoint{5.626900in}{1.034344in}}%
\pgfpathlineto{\pgfqpoint{5.625119in}{1.027393in}}%
\pgfpathlineto{\pgfqpoint{5.630397in}{1.018582in}}%
\pgfpathlineto{\pgfqpoint{5.627969in}{1.012840in}}%
\pgfpathlineto{\pgfqpoint{5.607240in}{1.019933in}}%
\pgfpathlineto{\pgfqpoint{5.600401in}{1.020647in}}%
\pgfpathlineto{\pgfqpoint{5.591874in}{1.009863in}}%
\pgfpathlineto{\pgfqpoint{5.594266in}{1.005020in}}%
\pgfpathclose%
\pgfusepath{fill}%
\end{pgfscope}%
\begin{pgfscope}%
\pgfpathrectangle{\pgfqpoint{3.625000in}{0.100000in}}{\pgfqpoint{2.989028in}{1.913466in}}%
\pgfusepath{clip}%
\pgfsetbuttcap%
\pgfsetmiterjoin%
\definecolor{currentfill}{rgb}{0.951942,0.980777,0.674048}%
\pgfsetfillcolor{currentfill}%
\pgfsetlinewidth{0.000000pt}%
\definecolor{currentstroke}{rgb}{0.000000,0.000000,0.000000}%
\pgfsetstrokecolor{currentstroke}%
\pgfsetstrokeopacity{0.000000}%
\pgfsetdash{}{0pt}%
\pgfpathmoveto{\pgfqpoint{6.197645in}{1.203210in}}%
\pgfpathlineto{\pgfqpoint{6.191941in}{1.211681in}}%
\pgfpathlineto{\pgfqpoint{6.195160in}{1.216421in}}%
\pgfpathlineto{\pgfqpoint{6.203043in}{1.211098in}}%
\pgfpathlineto{\pgfqpoint{6.197645in}{1.203210in}}%
\pgfpathclose%
\pgfusepath{fill}%
\end{pgfscope}%
\begin{pgfscope}%
\pgfpathrectangle{\pgfqpoint{3.625000in}{0.100000in}}{\pgfqpoint{2.989028in}{1.913466in}}%
\pgfusepath{clip}%
\pgfsetbuttcap%
\pgfsetmiterjoin%
\definecolor{currentfill}{rgb}{0.995617,0.855363,0.525721}%
\pgfsetfillcolor{currentfill}%
\pgfsetlinewidth{0.000000pt}%
\definecolor{currentstroke}{rgb}{0.000000,0.000000,0.000000}%
\pgfsetstrokecolor{currentstroke}%
\pgfsetstrokeopacity{0.000000}%
\pgfsetdash{}{0pt}%
\pgfpathmoveto{\pgfqpoint{6.247458in}{1.276100in}}%
\pgfpathlineto{\pgfqpoint{6.250867in}{1.283302in}}%
\pgfpathlineto{\pgfqpoint{6.264648in}{1.284858in}}%
\pgfpathlineto{\pgfqpoint{6.262440in}{1.278737in}}%
\pgfpathlineto{\pgfqpoint{6.257893in}{1.270918in}}%
\pgfpathlineto{\pgfqpoint{6.260974in}{1.261602in}}%
\pgfpathlineto{\pgfqpoint{6.273133in}{1.250439in}}%
\pgfpathlineto{\pgfqpoint{6.275955in}{1.238685in}}%
\pgfpathlineto{\pgfqpoint{6.289963in}{1.224047in}}%
\pgfpathlineto{\pgfqpoint{6.295458in}{1.224674in}}%
\pgfpathlineto{\pgfqpoint{6.302304in}{1.202700in}}%
\pgfpathlineto{\pgfqpoint{6.301181in}{1.202481in}}%
\pgfpathlineto{\pgfqpoint{6.299933in}{1.202236in}}%
\pgfpathlineto{\pgfqpoint{6.269401in}{1.196391in}}%
\pgfpathlineto{\pgfqpoint{6.253066in}{1.254485in}}%
\pgfpathlineto{\pgfqpoint{6.247458in}{1.276100in}}%
\pgfpathclose%
\pgfusepath{fill}%
\end{pgfscope}%
\begin{pgfscope}%
\pgfpathrectangle{\pgfqpoint{3.625000in}{0.100000in}}{\pgfqpoint{2.989028in}{1.913466in}}%
\pgfusepath{clip}%
\pgfsetbuttcap%
\pgfsetmiterjoin%
\definecolor{currentfill}{rgb}{0.280046,0.626990,0.702422}%
\pgfsetfillcolor{currentfill}%
\pgfsetlinewidth{0.000000pt}%
\definecolor{currentstroke}{rgb}{0.000000,0.000000,0.000000}%
\pgfsetstrokecolor{currentstroke}%
\pgfsetstrokeopacity{0.000000}%
\pgfsetdash{}{0pt}%
\pgfpathmoveto{\pgfqpoint{5.962284in}{1.079184in}}%
\pgfpathlineto{\pgfqpoint{5.952636in}{1.079539in}}%
\pgfpathlineto{\pgfqpoint{5.943746in}{1.085662in}}%
\pgfpathlineto{\pgfqpoint{5.931736in}{1.104274in}}%
\pgfpathlineto{\pgfqpoint{5.921513in}{1.114131in}}%
\pgfpathlineto{\pgfqpoint{5.924185in}{1.121744in}}%
\pgfpathlineto{\pgfqpoint{5.922176in}{1.131815in}}%
\pgfpathlineto{\pgfqpoint{5.934990in}{1.135540in}}%
\pgfpathlineto{\pgfqpoint{5.935720in}{1.144776in}}%
\pgfpathlineto{\pgfqpoint{5.940941in}{1.146502in}}%
\pgfpathlineto{\pgfqpoint{5.937606in}{1.158191in}}%
\pgfpathlineto{\pgfqpoint{5.944203in}{1.174686in}}%
\pgfpathlineto{\pgfqpoint{5.958135in}{1.170714in}}%
\pgfpathlineto{\pgfqpoint{5.956963in}{1.187230in}}%
\pgfpathlineto{\pgfqpoint{5.969473in}{1.204383in}}%
\pgfpathlineto{\pgfqpoint{5.974528in}{1.201094in}}%
\pgfpathlineto{\pgfqpoint{5.981028in}{1.204915in}}%
\pgfpathlineto{\pgfqpoint{5.996469in}{1.222911in}}%
\pgfpathlineto{\pgfqpoint{5.999521in}{1.243089in}}%
\pgfpathlineto{\pgfqpoint{5.999951in}{1.258593in}}%
\pgfpathlineto{\pgfqpoint{6.003201in}{1.269199in}}%
\pgfpathlineto{\pgfqpoint{6.001532in}{1.280492in}}%
\pgfpathlineto{\pgfqpoint{5.997344in}{1.285643in}}%
\pgfpathlineto{\pgfqpoint{6.003831in}{1.291252in}}%
\pgfpathlineto{\pgfqpoint{6.013234in}{1.232010in}}%
\pgfpathlineto{\pgfqpoint{6.065072in}{1.240581in}}%
\pgfpathlineto{\pgfqpoint{6.070445in}{1.206768in}}%
\pgfpathlineto{\pgfqpoint{6.089213in}{1.229086in}}%
\pgfpathlineto{\pgfqpoint{6.093635in}{1.226861in}}%
\pgfpathlineto{\pgfqpoint{6.100140in}{1.239793in}}%
\pgfpathlineto{\pgfqpoint{6.109152in}{1.235738in}}%
\pgfpathlineto{\pgfqpoint{6.117010in}{1.236509in}}%
\pgfpathlineto{\pgfqpoint{6.122194in}{1.245499in}}%
\pgfpathlineto{\pgfqpoint{6.129719in}{1.250500in}}%
\pgfpathlineto{\pgfqpoint{6.140180in}{1.246096in}}%
\pgfpathlineto{\pgfqpoint{6.147064in}{1.247618in}}%
\pgfpathlineto{\pgfqpoint{6.156926in}{1.230550in}}%
\pgfpathlineto{\pgfqpoint{6.154074in}{1.217629in}}%
\pgfpathlineto{\pgfqpoint{6.128317in}{1.232177in}}%
\pgfpathlineto{\pgfqpoint{6.123492in}{1.220234in}}%
\pgfpathlineto{\pgfqpoint{6.125084in}{1.214739in}}%
\pgfpathlineto{\pgfqpoint{6.114121in}{1.196164in}}%
\pgfpathlineto{\pgfqpoint{6.108947in}{1.192470in}}%
\pgfpathlineto{\pgfqpoint{6.106611in}{1.184279in}}%
\pgfpathlineto{\pgfqpoint{6.097791in}{1.185136in}}%
\pgfpathlineto{\pgfqpoint{6.091456in}{1.162779in}}%
\pgfpathlineto{\pgfqpoint{6.087911in}{1.157648in}}%
\pgfpathlineto{\pgfqpoint{6.078799in}{1.159359in}}%
\pgfpathlineto{\pgfqpoint{6.069480in}{1.166411in}}%
\pgfpathlineto{\pgfqpoint{6.069157in}{1.155588in}}%
\pgfpathlineto{\pgfqpoint{6.060135in}{1.137399in}}%
\pgfpathlineto{\pgfqpoint{6.057891in}{1.124451in}}%
\pgfpathlineto{\pgfqpoint{6.050969in}{1.115827in}}%
\pgfpathlineto{\pgfqpoint{6.045721in}{1.102055in}}%
\pgfpathlineto{\pgfqpoint{6.045463in}{1.089306in}}%
\pgfpathlineto{\pgfqpoint{6.037358in}{1.086925in}}%
\pgfpathlineto{\pgfqpoint{6.028164in}{1.079708in}}%
\pgfpathlineto{\pgfqpoint{6.019311in}{1.078334in}}%
\pgfpathlineto{\pgfqpoint{6.017271in}{1.072225in}}%
\pgfpathlineto{\pgfqpoint{6.003007in}{1.065970in}}%
\pgfpathlineto{\pgfqpoint{5.995035in}{1.071300in}}%
\pgfpathlineto{\pgfqpoint{5.986126in}{1.061179in}}%
\pgfpathlineto{\pgfqpoint{5.970810in}{1.064162in}}%
\pgfpathlineto{\pgfqpoint{5.961416in}{1.074785in}}%
\pgfpathlineto{\pgfqpoint{5.962284in}{1.079184in}}%
\pgfpathclose%
\pgfusepath{fill}%
\end{pgfscope}%
\begin{pgfscope}%
\pgfpathrectangle{\pgfqpoint{3.625000in}{0.100000in}}{\pgfqpoint{2.989028in}{1.913466in}}%
\pgfusepath{clip}%
\pgfsetbuttcap%
\pgfsetmiterjoin%
\definecolor{currentfill}{rgb}{0.838447,0.934948,0.608997}%
\pgfsetfillcolor{currentfill}%
\pgfsetlinewidth{0.000000pt}%
\definecolor{currentstroke}{rgb}{0.000000,0.000000,0.000000}%
\pgfsetstrokecolor{currentstroke}%
\pgfsetstrokeopacity{0.000000}%
\pgfsetdash{}{0pt}%
\pgfpathmoveto{\pgfqpoint{6.299933in}{1.202236in}}%
\pgfpathlineto{\pgfqpoint{6.299542in}{1.190298in}}%
\pgfpathlineto{\pgfqpoint{6.294947in}{1.184429in}}%
\pgfpathlineto{\pgfqpoint{6.292015in}{1.171401in}}%
\pgfpathlineto{\pgfqpoint{6.278775in}{1.165395in}}%
\pgfpathlineto{\pgfqpoint{6.267667in}{1.163645in}}%
\pgfpathlineto{\pgfqpoint{6.270904in}{1.172214in}}%
\pgfpathlineto{\pgfqpoint{6.253800in}{1.179393in}}%
\pgfpathlineto{\pgfqpoint{6.239904in}{1.188383in}}%
\pgfpathlineto{\pgfqpoint{6.248507in}{1.201821in}}%
\pgfpathlineto{\pgfqpoint{6.241613in}{1.212266in}}%
\pgfpathlineto{\pgfqpoint{6.242424in}{1.221276in}}%
\pgfpathlineto{\pgfqpoint{6.237439in}{1.223749in}}%
\pgfpathlineto{\pgfqpoint{6.233394in}{1.238386in}}%
\pgfpathlineto{\pgfqpoint{6.242960in}{1.258229in}}%
\pgfpathlineto{\pgfqpoint{6.235773in}{1.261469in}}%
\pgfpathlineto{\pgfqpoint{6.233910in}{1.251680in}}%
\pgfpathlineto{\pgfqpoint{6.223651in}{1.248954in}}%
\pgfpathlineto{\pgfqpoint{6.224730in}{1.216746in}}%
\pgfpathlineto{\pgfqpoint{6.222854in}{1.206381in}}%
\pgfpathlineto{\pgfqpoint{6.228000in}{1.191607in}}%
\pgfpathlineto{\pgfqpoint{6.235916in}{1.184499in}}%
\pgfpathlineto{\pgfqpoint{6.232326in}{1.180036in}}%
\pgfpathlineto{\pgfqpoint{6.240407in}{1.173514in}}%
\pgfpathlineto{\pgfqpoint{6.243328in}{1.162929in}}%
\pgfpathlineto{\pgfqpoint{6.228517in}{1.171694in}}%
\pgfpathlineto{\pgfqpoint{6.219143in}{1.170556in}}%
\pgfpathlineto{\pgfqpoint{6.211865in}{1.179573in}}%
\pgfpathlineto{\pgfqpoint{6.204454in}{1.180451in}}%
\pgfpathlineto{\pgfqpoint{6.193921in}{1.175935in}}%
\pgfpathlineto{\pgfqpoint{6.189837in}{1.181563in}}%
\pgfpathlineto{\pgfqpoint{6.195205in}{1.193376in}}%
\pgfpathlineto{\pgfqpoint{6.197645in}{1.203210in}}%
\pgfpathlineto{\pgfqpoint{6.203043in}{1.211098in}}%
\pgfpathlineto{\pgfqpoint{6.195160in}{1.216421in}}%
\pgfpathlineto{\pgfqpoint{6.191941in}{1.211681in}}%
\pgfpathlineto{\pgfqpoint{6.184066in}{1.216492in}}%
\pgfpathlineto{\pgfqpoint{6.170120in}{1.219698in}}%
\pgfpathlineto{\pgfqpoint{6.171385in}{1.226747in}}%
\pgfpathlineto{\pgfqpoint{6.165064in}{1.230836in}}%
\pgfpathlineto{\pgfqpoint{6.156926in}{1.230550in}}%
\pgfpathlineto{\pgfqpoint{6.147064in}{1.247618in}}%
\pgfpathlineto{\pgfqpoint{6.140180in}{1.246096in}}%
\pgfpathlineto{\pgfqpoint{6.129719in}{1.250500in}}%
\pgfpathlineto{\pgfqpoint{6.122194in}{1.245499in}}%
\pgfpathlineto{\pgfqpoint{6.117010in}{1.236509in}}%
\pgfpathlineto{\pgfqpoint{6.109152in}{1.235738in}}%
\pgfpathlineto{\pgfqpoint{6.100140in}{1.239793in}}%
\pgfpathlineto{\pgfqpoint{6.093635in}{1.226861in}}%
\pgfpathlineto{\pgfqpoint{6.089213in}{1.229086in}}%
\pgfpathlineto{\pgfqpoint{6.070445in}{1.206768in}}%
\pgfpathlineto{\pgfqpoint{6.065072in}{1.240581in}}%
\pgfpathlineto{\pgfqpoint{6.133655in}{1.253292in}}%
\pgfpathlineto{\pgfqpoint{6.164399in}{1.258797in}}%
\pgfpathlineto{\pgfqpoint{6.209127in}{1.267806in}}%
\pgfpathlineto{\pgfqpoint{6.247458in}{1.276100in}}%
\pgfpathlineto{\pgfqpoint{6.253066in}{1.254485in}}%
\pgfpathlineto{\pgfqpoint{6.269401in}{1.196391in}}%
\pgfpathlineto{\pgfqpoint{6.299933in}{1.202236in}}%
\pgfpathclose%
\pgfusepath{fill}%
\end{pgfscope}%
\begin{pgfscope}%
\pgfpathrectangle{\pgfqpoint{3.625000in}{0.100000in}}{\pgfqpoint{2.989028in}{1.913466in}}%
\pgfusepath{clip}%
\pgfsetbuttcap%
\pgfsetmiterjoin%
\definecolor{currentfill}{rgb}{0.684198,0.872203,0.640369}%
\pgfsetfillcolor{currentfill}%
\pgfsetlinewidth{0.000000pt}%
\definecolor{currentstroke}{rgb}{0.000000,0.000000,0.000000}%
\pgfsetstrokecolor{currentstroke}%
\pgfsetstrokeopacity{0.000000}%
\pgfsetdash{}{0pt}%
\pgfpathmoveto{\pgfqpoint{4.919404in}{1.009408in}}%
\pgfpathlineto{\pgfqpoint{4.869368in}{1.014183in}}%
\pgfpathlineto{\pgfqpoint{4.817460in}{1.018783in}}%
\pgfpathlineto{\pgfqpoint{4.757477in}{1.025040in}}%
\pgfpathlineto{\pgfqpoint{4.668359in}{1.035639in}}%
\pgfpathlineto{\pgfqpoint{4.636744in}{1.040510in}}%
\pgfpathlineto{\pgfqpoint{4.555137in}{1.052362in}}%
\pgfpathlineto{\pgfqpoint{4.567053in}{1.127151in}}%
\pgfpathlineto{\pgfqpoint{4.567350in}{1.133200in}}%
\pgfpathlineto{\pgfqpoint{4.578829in}{1.205365in}}%
\pgfpathlineto{\pgfqpoint{4.587404in}{1.260326in}}%
\pgfpathlineto{\pgfqpoint{4.595495in}{1.311293in}}%
\pgfpathlineto{\pgfqpoint{4.650778in}{1.303357in}}%
\pgfpathlineto{\pgfqpoint{4.702321in}{1.295959in}}%
\pgfpathlineto{\pgfqpoint{4.797072in}{1.284265in}}%
\pgfpathlineto{\pgfqpoint{4.840540in}{1.280295in}}%
\pgfpathlineto{\pgfqpoint{4.909404in}{1.273614in}}%
\pgfpathlineto{\pgfqpoint{4.939195in}{1.271196in}}%
\pgfpathlineto{\pgfqpoint{4.933935in}{1.205970in}}%
\pgfpathlineto{\pgfqpoint{4.929183in}{1.143169in}}%
\pgfpathlineto{\pgfqpoint{4.925359in}{1.092033in}}%
\pgfpathlineto{\pgfqpoint{4.919404in}{1.009408in}}%
\pgfpathclose%
\pgfusepath{fill}%
\end{pgfscope}%
\begin{pgfscope}%
\pgfpathrectangle{\pgfqpoint{3.625000in}{0.100000in}}{\pgfqpoint{2.989028in}{1.913466in}}%
\pgfusepath{clip}%
\pgfsetbuttcap%
\pgfsetmiterjoin%
\definecolor{currentfill}{rgb}{0.665283,0.864591,0.643214}%
\pgfsetfillcolor{currentfill}%
\pgfsetlinewidth{0.000000pt}%
\definecolor{currentstroke}{rgb}{0.000000,0.000000,0.000000}%
\pgfsetstrokecolor{currentstroke}%
\pgfsetstrokeopacity{0.000000}%
\pgfsetdash{}{0pt}%
\pgfpathmoveto{\pgfqpoint{5.581343in}{0.972329in}}%
\pgfpathlineto{\pgfqpoint{5.583030in}{0.979890in}}%
\pgfpathlineto{\pgfqpoint{5.591773in}{0.978188in}}%
\pgfpathlineto{\pgfqpoint{5.594266in}{1.005020in}}%
\pgfpathlineto{\pgfqpoint{5.591874in}{1.009863in}}%
\pgfpathlineto{\pgfqpoint{5.600401in}{1.020647in}}%
\pgfpathlineto{\pgfqpoint{5.607240in}{1.019933in}}%
\pgfpathlineto{\pgfqpoint{5.627969in}{1.012840in}}%
\pgfpathlineto{\pgfqpoint{5.630397in}{1.018582in}}%
\pgfpathlineto{\pgfqpoint{5.625119in}{1.027393in}}%
\pgfpathlineto{\pgfqpoint{5.626900in}{1.034344in}}%
\pgfpathlineto{\pgfqpoint{5.646231in}{1.040736in}}%
\pgfpathlineto{\pgfqpoint{5.641617in}{1.052499in}}%
\pgfpathlineto{\pgfqpoint{5.647290in}{1.062454in}}%
\pgfpathlineto{\pgfqpoint{5.657850in}{1.068041in}}%
\pgfpathlineto{\pgfqpoint{5.680006in}{1.073574in}}%
\pgfpathlineto{\pgfqpoint{5.693985in}{1.065222in}}%
\pgfpathlineto{\pgfqpoint{5.698226in}{1.073367in}}%
\pgfpathlineto{\pgfqpoint{5.709834in}{1.079056in}}%
\pgfpathlineto{\pgfqpoint{5.712829in}{1.074039in}}%
\pgfpathlineto{\pgfqpoint{5.724753in}{1.078025in}}%
\pgfpathlineto{\pgfqpoint{5.724000in}{1.084848in}}%
\pgfpathlineto{\pgfqpoint{5.734721in}{1.092656in}}%
\pgfpathlineto{\pgfqpoint{5.741048in}{1.084548in}}%
\pgfpathlineto{\pgfqpoint{5.749371in}{1.083732in}}%
\pgfpathlineto{\pgfqpoint{5.754903in}{1.089016in}}%
\pgfpathlineto{\pgfqpoint{5.754287in}{1.096531in}}%
\pgfpathlineto{\pgfqpoint{5.758989in}{1.103993in}}%
\pgfpathlineto{\pgfqpoint{5.765290in}{1.105610in}}%
\pgfpathlineto{\pgfqpoint{5.767820in}{1.115463in}}%
\pgfpathlineto{\pgfqpoint{5.776983in}{1.123981in}}%
\pgfpathlineto{\pgfqpoint{5.774225in}{1.132459in}}%
\pgfpathlineto{\pgfqpoint{5.783139in}{1.136680in}}%
\pgfpathlineto{\pgfqpoint{5.789096in}{1.134084in}}%
\pgfpathlineto{\pgfqpoint{5.797893in}{1.140679in}}%
\pgfpathlineto{\pgfqpoint{5.805755in}{1.142397in}}%
\pgfpathlineto{\pgfqpoint{5.805783in}{1.149238in}}%
\pgfpathlineto{\pgfqpoint{5.800273in}{1.158730in}}%
\pgfpathlineto{\pgfqpoint{5.803860in}{1.162554in}}%
\pgfpathlineto{\pgfqpoint{5.815128in}{1.161707in}}%
\pgfpathlineto{\pgfqpoint{5.822293in}{1.165276in}}%
\pgfpathlineto{\pgfqpoint{5.829684in}{1.159700in}}%
\pgfpathlineto{\pgfqpoint{5.835517in}{1.147114in}}%
\pgfpathlineto{\pgfqpoint{5.854857in}{1.144988in}}%
\pgfpathlineto{\pgfqpoint{5.864938in}{1.138060in}}%
\pgfpathlineto{\pgfqpoint{5.872031in}{1.143916in}}%
\pgfpathlineto{\pgfqpoint{5.884855in}{1.138994in}}%
\pgfpathlineto{\pgfqpoint{5.892116in}{1.141440in}}%
\pgfpathlineto{\pgfqpoint{5.897191in}{1.148394in}}%
\pgfpathlineto{\pgfqpoint{5.904937in}{1.149724in}}%
\pgfpathlineto{\pgfqpoint{5.906950in}{1.142163in}}%
\pgfpathlineto{\pgfqpoint{5.915518in}{1.139272in}}%
\pgfpathlineto{\pgfqpoint{5.922176in}{1.131815in}}%
\pgfpathlineto{\pgfqpoint{5.924185in}{1.121744in}}%
\pgfpathlineto{\pgfqpoint{5.921513in}{1.114131in}}%
\pgfpathlineto{\pgfqpoint{5.931736in}{1.104274in}}%
\pgfpathlineto{\pgfqpoint{5.943746in}{1.085662in}}%
\pgfpathlineto{\pgfqpoint{5.952636in}{1.079539in}}%
\pgfpathlineto{\pgfqpoint{5.962284in}{1.079184in}}%
\pgfpathlineto{\pgfqpoint{5.944469in}{1.058743in}}%
\pgfpathlineto{\pgfqpoint{5.926934in}{1.046371in}}%
\pgfpathlineto{\pgfqpoint{5.920481in}{1.036545in}}%
\pgfpathlineto{\pgfqpoint{5.920590in}{1.031213in}}%
\pgfpathlineto{\pgfqpoint{5.911092in}{1.027126in}}%
\pgfpathlineto{\pgfqpoint{5.908370in}{1.019436in}}%
\pgfpathlineto{\pgfqpoint{5.888584in}{1.011690in}}%
\pgfpathlineto{\pgfqpoint{5.881571in}{1.006660in}}%
\pgfpathlineto{\pgfqpoint{5.880622in}{1.005586in}}%
\pgfpathlineto{\pgfqpoint{5.823708in}{1.000190in}}%
\pgfpathlineto{\pgfqpoint{5.789337in}{0.997299in}}%
\pgfpathlineto{\pgfqpoint{5.732938in}{0.994107in}}%
\pgfpathlineto{\pgfqpoint{5.662674in}{0.987156in}}%
\pgfpathlineto{\pgfqpoint{5.651065in}{0.988764in}}%
\pgfpathlineto{\pgfqpoint{5.653480in}{0.976914in}}%
\pgfpathlineto{\pgfqpoint{5.581343in}{0.972329in}}%
\pgfpathclose%
\pgfusepath{fill}%
\end{pgfscope}%
\begin{pgfscope}%
\pgfpathrectangle{\pgfqpoint{3.625000in}{0.100000in}}{\pgfqpoint{2.989028in}{1.913466in}}%
\pgfusepath{clip}%
\pgfsetbuttcap%
\pgfsetmiterjoin%
\definecolor{currentfill}{rgb}{0.384006,0.742945,0.654441}%
\pgfsetfillcolor{currentfill}%
\pgfsetlinewidth{0.000000pt}%
\definecolor{currentstroke}{rgb}{0.000000,0.000000,0.000000}%
\pgfsetstrokecolor{currentstroke}%
\pgfsetstrokeopacity{0.000000}%
\pgfsetdash{}{0pt}%
\pgfpathmoveto{\pgfqpoint{4.919404in}{1.009408in}}%
\pgfpathlineto{\pgfqpoint{4.925359in}{1.092033in}}%
\pgfpathlineto{\pgfqpoint{4.929183in}{1.143169in}}%
\pgfpathlineto{\pgfqpoint{4.933935in}{1.205970in}}%
\pgfpathlineto{\pgfqpoint{4.999809in}{1.201345in}}%
\pgfpathlineto{\pgfqpoint{5.083468in}{1.196775in}}%
\pgfpathlineto{\pgfqpoint{5.140367in}{1.194592in}}%
\pgfpathlineto{\pgfqpoint{5.196933in}{1.192859in}}%
\pgfpathlineto{\pgfqpoint{5.248131in}{1.192066in}}%
\pgfpathlineto{\pgfqpoint{5.271810in}{1.192312in}}%
\pgfpathlineto{\pgfqpoint{5.282233in}{1.183806in}}%
\pgfpathlineto{\pgfqpoint{5.290397in}{1.185521in}}%
\pgfpathlineto{\pgfqpoint{5.290653in}{1.178102in}}%
\pgfpathlineto{\pgfqpoint{5.284590in}{1.165269in}}%
\pgfpathlineto{\pgfqpoint{5.285247in}{1.157161in}}%
\pgfpathlineto{\pgfqpoint{5.291052in}{1.151814in}}%
\pgfpathlineto{\pgfqpoint{5.296384in}{1.140674in}}%
\pgfpathlineto{\pgfqpoint{5.307201in}{1.134279in}}%
\pgfpathlineto{\pgfqpoint{5.306840in}{1.092287in}}%
\pgfpathlineto{\pgfqpoint{5.307158in}{0.995725in}}%
\pgfpathlineto{\pgfqpoint{5.234656in}{0.996060in}}%
\pgfpathlineto{\pgfqpoint{5.158309in}{0.997396in}}%
\pgfpathlineto{\pgfqpoint{5.102105in}{0.999379in}}%
\pgfpathlineto{\pgfqpoint{5.055191in}{1.001189in}}%
\pgfpathlineto{\pgfqpoint{4.976155in}{1.005849in}}%
\pgfpathlineto{\pgfqpoint{4.919404in}{1.009408in}}%
\pgfpathclose%
\pgfusepath{fill}%
\end{pgfscope}%
\begin{pgfscope}%
\pgfpathrectangle{\pgfqpoint{3.625000in}{0.100000in}}{\pgfqpoint{2.989028in}{1.913466in}}%
\pgfusepath{clip}%
\pgfsetbuttcap%
\pgfsetmiterjoin%
\definecolor{currentfill}{rgb}{0.774933,0.909112,0.621915}%
\pgfsetfillcolor{currentfill}%
\pgfsetlinewidth{0.000000pt}%
\definecolor{currentstroke}{rgb}{0.000000,0.000000,0.000000}%
\pgfsetstrokecolor{currentstroke}%
\pgfsetstrokeopacity{0.000000}%
\pgfsetdash{}{0pt}%
\pgfpathmoveto{\pgfqpoint{5.986189in}{1.020348in}}%
\pgfpathlineto{\pgfqpoint{5.905888in}{1.009068in}}%
\pgfpathlineto{\pgfqpoint{5.881571in}{1.006660in}}%
\pgfpathlineto{\pgfqpoint{5.888584in}{1.011690in}}%
\pgfpathlineto{\pgfqpoint{5.908370in}{1.019436in}}%
\pgfpathlineto{\pgfqpoint{5.911092in}{1.027126in}}%
\pgfpathlineto{\pgfqpoint{5.920590in}{1.031213in}}%
\pgfpathlineto{\pgfqpoint{5.920481in}{1.036545in}}%
\pgfpathlineto{\pgfqpoint{5.926934in}{1.046371in}}%
\pgfpathlineto{\pgfqpoint{5.944469in}{1.058743in}}%
\pgfpathlineto{\pgfqpoint{5.962284in}{1.079184in}}%
\pgfpathlineto{\pgfqpoint{5.961416in}{1.074785in}}%
\pgfpathlineto{\pgfqpoint{5.970810in}{1.064162in}}%
\pgfpathlineto{\pgfqpoint{5.986126in}{1.061179in}}%
\pgfpathlineto{\pgfqpoint{5.995035in}{1.071300in}}%
\pgfpathlineto{\pgfqpoint{6.003007in}{1.065970in}}%
\pgfpathlineto{\pgfqpoint{6.017271in}{1.072225in}}%
\pgfpathlineto{\pgfqpoint{6.019311in}{1.078334in}}%
\pgfpathlineto{\pgfqpoint{6.028164in}{1.079708in}}%
\pgfpathlineto{\pgfqpoint{6.037358in}{1.086925in}}%
\pgfpathlineto{\pgfqpoint{6.045463in}{1.089306in}}%
\pgfpathlineto{\pgfqpoint{6.045721in}{1.102055in}}%
\pgfpathlineto{\pgfqpoint{6.050969in}{1.115827in}}%
\pgfpathlineto{\pgfqpoint{6.057891in}{1.124451in}}%
\pgfpathlineto{\pgfqpoint{6.060135in}{1.137399in}}%
\pgfpathlineto{\pgfqpoint{6.069157in}{1.155588in}}%
\pgfpathlineto{\pgfqpoint{6.069480in}{1.166411in}}%
\pgfpathlineto{\pgfqpoint{6.078799in}{1.159359in}}%
\pgfpathlineto{\pgfqpoint{6.087911in}{1.157648in}}%
\pgfpathlineto{\pgfqpoint{6.091456in}{1.162779in}}%
\pgfpathlineto{\pgfqpoint{6.097791in}{1.185136in}}%
\pgfpathlineto{\pgfqpoint{6.106611in}{1.184279in}}%
\pgfpathlineto{\pgfqpoint{6.108947in}{1.192470in}}%
\pgfpathlineto{\pgfqpoint{6.114121in}{1.196164in}}%
\pgfpathlineto{\pgfqpoint{6.125084in}{1.214739in}}%
\pgfpathlineto{\pgfqpoint{6.123492in}{1.220234in}}%
\pgfpathlineto{\pgfqpoint{6.128317in}{1.232177in}}%
\pgfpathlineto{\pgfqpoint{6.154074in}{1.217629in}}%
\pgfpathlineto{\pgfqpoint{6.156926in}{1.230550in}}%
\pgfpathlineto{\pgfqpoint{6.165064in}{1.230836in}}%
\pgfpathlineto{\pgfqpoint{6.171385in}{1.226747in}}%
\pgfpathlineto{\pgfqpoint{6.170120in}{1.219698in}}%
\pgfpathlineto{\pgfqpoint{6.184066in}{1.216492in}}%
\pgfpathlineto{\pgfqpoint{6.191941in}{1.211681in}}%
\pgfpathlineto{\pgfqpoint{6.197645in}{1.203210in}}%
\pgfpathlineto{\pgfqpoint{6.195205in}{1.193376in}}%
\pgfpathlineto{\pgfqpoint{6.190277in}{1.192570in}}%
\pgfpathlineto{\pgfqpoint{6.187419in}{1.177731in}}%
\pgfpathlineto{\pgfqpoint{6.193680in}{1.171929in}}%
\pgfpathlineto{\pgfqpoint{6.202493in}{1.176619in}}%
\pgfpathlineto{\pgfqpoint{6.210671in}{1.166716in}}%
\pgfpathlineto{\pgfqpoint{6.228942in}{1.164936in}}%
\pgfpathlineto{\pgfqpoint{6.232089in}{1.159239in}}%
\pgfpathlineto{\pgfqpoint{6.248998in}{1.153699in}}%
\pgfpathlineto{\pgfqpoint{6.246912in}{1.147161in}}%
\pgfpathlineto{\pgfqpoint{6.249111in}{1.128683in}}%
\pgfpathlineto{\pgfqpoint{6.255936in}{1.121706in}}%
\pgfpathlineto{\pgfqpoint{6.245866in}{1.118761in}}%
\pgfpathlineto{\pgfqpoint{6.247201in}{1.110873in}}%
\pgfpathlineto{\pgfqpoint{6.257944in}{1.104185in}}%
\pgfpathlineto{\pgfqpoint{6.258912in}{1.097588in}}%
\pgfpathlineto{\pgfqpoint{6.252847in}{1.092630in}}%
\pgfpathlineto{\pgfqpoint{6.240609in}{1.104375in}}%
\pgfpathlineto{\pgfqpoint{6.239399in}{1.095806in}}%
\pgfpathlineto{\pgfqpoint{6.249644in}{1.091731in}}%
\pgfpathlineto{\pgfqpoint{6.250669in}{1.087517in}}%
\pgfpathlineto{\pgfqpoint{6.264721in}{1.093086in}}%
\pgfpathlineto{\pgfqpoint{6.275489in}{1.094559in}}%
\pgfpathlineto{\pgfqpoint{6.286520in}{1.072304in}}%
\pgfpathlineto{\pgfqpoint{6.285289in}{1.072063in}}%
\pgfpathlineto{\pgfqpoint{6.280305in}{1.071033in}}%
\pgfpathlineto{\pgfqpoint{6.278836in}{1.070726in}}%
\pgfpathlineto{\pgfqpoint{6.277865in}{1.070536in}}%
\pgfpathlineto{\pgfqpoint{6.212189in}{1.056913in}}%
\pgfpathlineto{\pgfqpoint{6.153447in}{1.044890in}}%
\pgfpathlineto{\pgfqpoint{6.072229in}{1.030898in}}%
\pgfpathlineto{\pgfqpoint{6.003251in}{1.021788in}}%
\pgfpathlineto{\pgfqpoint{5.986189in}{1.020348in}}%
\pgfpathclose%
\pgfusepath{fill}%
\end{pgfscope}%
\begin{pgfscope}%
\pgfpathrectangle{\pgfqpoint{3.625000in}{0.100000in}}{\pgfqpoint{2.989028in}{1.913466in}}%
\pgfusepath{clip}%
\pgfsetbuttcap%
\pgfsetmiterjoin%
\definecolor{currentfill}{rgb}{0.774933,0.909112,0.621915}%
\pgfsetfillcolor{currentfill}%
\pgfsetlinewidth{0.000000pt}%
\definecolor{currentstroke}{rgb}{0.000000,0.000000,0.000000}%
\pgfsetstrokecolor{currentstroke}%
\pgfsetstrokeopacity{0.000000}%
\pgfsetdash{}{0pt}%
\pgfpathmoveto{\pgfqpoint{6.278775in}{1.165395in}}%
\pgfpathlineto{\pgfqpoint{6.292015in}{1.171401in}}%
\pgfpathlineto{\pgfqpoint{6.281442in}{1.140471in}}%
\pgfpathlineto{\pgfqpoint{6.274450in}{1.124092in}}%
\pgfpathlineto{\pgfqpoint{6.269413in}{1.133361in}}%
\pgfpathlineto{\pgfqpoint{6.278370in}{1.155552in}}%
\pgfpathlineto{\pgfqpoint{6.278775in}{1.165395in}}%
\pgfpathclose%
\pgfusepath{fill}%
\end{pgfscope}%
\begin{pgfscope}%
\pgfpathrectangle{\pgfqpoint{3.625000in}{0.100000in}}{\pgfqpoint{2.989028in}{1.913466in}}%
\pgfusepath{clip}%
\pgfsetbuttcap%
\pgfsetmiterjoin%
\definecolor{currentfill}{rgb}{0.874740,0.949712,0.601615}%
\pgfsetfillcolor{currentfill}%
\pgfsetlinewidth{0.000000pt}%
\definecolor{currentstroke}{rgb}{0.000000,0.000000,0.000000}%
\pgfsetstrokecolor{currentstroke}%
\pgfsetstrokeopacity{0.000000}%
\pgfsetdash{}{0pt}%
\pgfpathmoveto{\pgfqpoint{5.581343in}{0.972329in}}%
\pgfpathlineto{\pgfqpoint{5.578146in}{0.971866in}}%
\pgfpathlineto{\pgfqpoint{5.575131in}{0.971655in}}%
\pgfpathlineto{\pgfqpoint{5.576421in}{0.962397in}}%
\pgfpathlineto{\pgfqpoint{5.568642in}{0.953066in}}%
\pgfpathlineto{\pgfqpoint{5.567109in}{0.938464in}}%
\pgfpathlineto{\pgfqpoint{5.532334in}{0.935892in}}%
\pgfpathlineto{\pgfqpoint{5.535365in}{0.942748in}}%
\pgfpathlineto{\pgfqpoint{5.547904in}{0.955269in}}%
\pgfpathlineto{\pgfqpoint{5.548251in}{0.962527in}}%
\pgfpathlineto{\pgfqpoint{5.542700in}{0.969399in}}%
\pgfpathlineto{\pgfqpoint{5.490941in}{0.966662in}}%
\pgfpathlineto{\pgfqpoint{5.400448in}{0.963789in}}%
\pgfpathlineto{\pgfqpoint{5.347491in}{0.962824in}}%
\pgfpathlineto{\pgfqpoint{5.307462in}{0.962463in}}%
\pgfpathlineto{\pgfqpoint{5.307158in}{0.995725in}}%
\pgfpathlineto{\pgfqpoint{5.306840in}{1.092287in}}%
\pgfpathlineto{\pgfqpoint{5.307201in}{1.134279in}}%
\pgfpathlineto{\pgfqpoint{5.296384in}{1.140674in}}%
\pgfpathlineto{\pgfqpoint{5.291052in}{1.151814in}}%
\pgfpathlineto{\pgfqpoint{5.285247in}{1.157161in}}%
\pgfpathlineto{\pgfqpoint{5.284590in}{1.165269in}}%
\pgfpathlineto{\pgfqpoint{5.290653in}{1.178102in}}%
\pgfpathlineto{\pgfqpoint{5.290397in}{1.185521in}}%
\pgfpathlineto{\pgfqpoint{5.282233in}{1.183806in}}%
\pgfpathlineto{\pgfqpoint{5.271810in}{1.192312in}}%
\pgfpathlineto{\pgfqpoint{5.263455in}{1.207242in}}%
\pgfpathlineto{\pgfqpoint{5.256447in}{1.214132in}}%
\pgfpathlineto{\pgfqpoint{5.249124in}{1.231060in}}%
\pgfpathlineto{\pgfqpoint{5.325184in}{1.229874in}}%
\pgfpathlineto{\pgfqpoint{5.400789in}{1.232365in}}%
\pgfpathlineto{\pgfqpoint{5.449266in}{1.235161in}}%
\pgfpathlineto{\pgfqpoint{5.464431in}{1.220143in}}%
\pgfpathlineto{\pgfqpoint{5.461659in}{1.204480in}}%
\pgfpathlineto{\pgfqpoint{5.465703in}{1.186846in}}%
\pgfpathlineto{\pgfqpoint{5.470188in}{1.177954in}}%
\pgfpathlineto{\pgfqpoint{5.484499in}{1.165722in}}%
\pgfpathlineto{\pgfqpoint{5.487630in}{1.160319in}}%
\pgfpathlineto{\pgfqpoint{5.503299in}{1.148144in}}%
\pgfpathlineto{\pgfqpoint{5.505570in}{1.134486in}}%
\pgfpathlineto{\pgfqpoint{5.510070in}{1.124410in}}%
\pgfpathlineto{\pgfqpoint{5.520702in}{1.130279in}}%
\pgfpathlineto{\pgfqpoint{5.529465in}{1.127595in}}%
\pgfpathlineto{\pgfqpoint{5.536082in}{1.120472in}}%
\pgfpathlineto{\pgfqpoint{5.533584in}{1.107865in}}%
\pgfpathlineto{\pgfqpoint{5.525462in}{1.091383in}}%
\pgfpathlineto{\pgfqpoint{5.525587in}{1.082799in}}%
\pgfpathlineto{\pgfqpoint{5.531657in}{1.075760in}}%
\pgfpathlineto{\pgfqpoint{5.544719in}{1.066437in}}%
\pgfpathlineto{\pgfqpoint{5.559760in}{1.058870in}}%
\pgfpathlineto{\pgfqpoint{5.563328in}{1.052811in}}%
\pgfpathlineto{\pgfqpoint{5.571020in}{1.049946in}}%
\pgfpathlineto{\pgfqpoint{5.571326in}{1.041904in}}%
\pgfpathlineto{\pgfqpoint{5.577019in}{1.031690in}}%
\pgfpathlineto{\pgfqpoint{5.573069in}{1.022616in}}%
\pgfpathlineto{\pgfqpoint{5.580801in}{1.008178in}}%
\pgfpathlineto{\pgfqpoint{5.586849in}{1.011045in}}%
\pgfpathlineto{\pgfqpoint{5.594266in}{1.005020in}}%
\pgfpathlineto{\pgfqpoint{5.591773in}{0.978188in}}%
\pgfpathlineto{\pgfqpoint{5.583030in}{0.979890in}}%
\pgfpathlineto{\pgfqpoint{5.581343in}{0.972329in}}%
\pgfpathclose%
\pgfusepath{fill}%
\end{pgfscope}%
\begin{pgfscope}%
\pgfpathrectangle{\pgfqpoint{3.625000in}{0.100000in}}{\pgfqpoint{2.989028in}{1.913466in}}%
\pgfusepath{clip}%
\pgfsetbuttcap%
\pgfsetmiterjoin%
\definecolor{currentfill}{rgb}{0.784006,0.912803,0.620069}%
\pgfsetfillcolor{currentfill}%
\pgfsetlinewidth{0.000000pt}%
\definecolor{currentstroke}{rgb}{0.000000,0.000000,0.000000}%
\pgfsetstrokecolor{currentstroke}%
\pgfsetstrokeopacity{0.000000}%
\pgfsetdash{}{0pt}%
\pgfpathmoveto{\pgfqpoint{4.240525in}{0.978113in}}%
\pgfpathlineto{\pgfqpoint{4.246856in}{0.991592in}}%
\pgfpathlineto{\pgfqpoint{4.245123in}{1.011825in}}%
\pgfpathlineto{\pgfqpoint{4.248155in}{1.017578in}}%
\pgfpathlineto{\pgfqpoint{4.248454in}{1.042780in}}%
\pgfpathlineto{\pgfqpoint{4.251328in}{1.050037in}}%
\pgfpathlineto{\pgfqpoint{4.261451in}{1.051167in}}%
\pgfpathlineto{\pgfqpoint{4.270846in}{1.047945in}}%
\pgfpathlineto{\pgfqpoint{4.274938in}{1.039072in}}%
\pgfpathlineto{\pgfqpoint{4.280674in}{1.039412in}}%
\pgfpathlineto{\pgfqpoint{4.287801in}{1.049576in}}%
\pgfpathlineto{\pgfqpoint{4.298103in}{1.099678in}}%
\pgfpathlineto{\pgfqpoint{4.356724in}{1.087575in}}%
\pgfpathlineto{\pgfqpoint{4.390746in}{1.080865in}}%
\pgfpathlineto{\pgfqpoint{4.481091in}{1.064984in}}%
\pgfpathlineto{\pgfqpoint{4.506086in}{1.059964in}}%
\pgfpathlineto{\pgfqpoint{4.555137in}{1.052362in}}%
\pgfpathlineto{\pgfqpoint{4.545079in}{0.987601in}}%
\pgfpathlineto{\pgfqpoint{4.534619in}{0.920049in}}%
\pgfpathlineto{\pgfqpoint{4.522573in}{0.844052in}}%
\pgfpathlineto{\pgfqpoint{4.509025in}{0.756815in}}%
\pgfpathlineto{\pgfqpoint{4.498080in}{0.685166in}}%
\pgfpathlineto{\pgfqpoint{4.419687in}{0.697614in}}%
\pgfpathlineto{\pgfqpoint{4.385244in}{0.703513in}}%
\pgfpathlineto{\pgfqpoint{4.369857in}{0.712725in}}%
\pgfpathlineto{\pgfqpoint{4.269535in}{0.773254in}}%
\pgfpathlineto{\pgfqpoint{4.194248in}{0.819189in}}%
\pgfpathlineto{\pgfqpoint{4.196756in}{0.827331in}}%
\pgfpathlineto{\pgfqpoint{4.202975in}{0.833027in}}%
\pgfpathlineto{\pgfqpoint{4.209444in}{0.831972in}}%
\pgfpathlineto{\pgfqpoint{4.218826in}{0.837954in}}%
\pgfpathlineto{\pgfqpoint{4.220303in}{0.846552in}}%
\pgfpathlineto{\pgfqpoint{4.208861in}{0.856982in}}%
\pgfpathlineto{\pgfqpoint{4.216998in}{0.877004in}}%
\pgfpathlineto{\pgfqpoint{4.225180in}{0.884704in}}%
\pgfpathlineto{\pgfqpoint{4.229063in}{0.893839in}}%
\pgfpathlineto{\pgfqpoint{4.231455in}{0.910589in}}%
\pgfpathlineto{\pgfqpoint{4.239127in}{0.918174in}}%
\pgfpathlineto{\pgfqpoint{4.255230in}{0.925750in}}%
\pgfpathlineto{\pgfqpoint{4.254874in}{0.932426in}}%
\pgfpathlineto{\pgfqpoint{4.245881in}{0.940711in}}%
\pgfpathlineto{\pgfqpoint{4.244656in}{0.957789in}}%
\pgfpathlineto{\pgfqpoint{4.238464in}{0.970274in}}%
\pgfpathlineto{\pgfqpoint{4.240525in}{0.978113in}}%
\pgfpathclose%
\pgfusepath{fill}%
\end{pgfscope}%
\begin{pgfscope}%
\pgfpathrectangle{\pgfqpoint{3.625000in}{0.100000in}}{\pgfqpoint{2.989028in}{1.913466in}}%
\pgfusepath{clip}%
\pgfsetbuttcap%
\pgfsetmiterjoin%
\definecolor{currentfill}{rgb}{0.256055,0.600231,0.713495}%
\pgfsetfillcolor{currentfill}%
\pgfsetlinewidth{0.000000pt}%
\definecolor{currentstroke}{rgb}{0.000000,0.000000,0.000000}%
\pgfsetstrokecolor{currentstroke}%
\pgfsetstrokeopacity{0.000000}%
\pgfsetdash{}{0pt}%
\pgfpathmoveto{\pgfqpoint{5.315948in}{0.774957in}}%
\pgfpathlineto{\pgfqpoint{5.301416in}{0.779451in}}%
\pgfpathlineto{\pgfqpoint{5.282795in}{0.793711in}}%
\pgfpathlineto{\pgfqpoint{5.274524in}{0.796776in}}%
\pgfpathlineto{\pgfqpoint{5.269268in}{0.790617in}}%
\pgfpathlineto{\pgfqpoint{5.262631in}{0.790306in}}%
\pgfpathlineto{\pgfqpoint{5.254234in}{0.795535in}}%
\pgfpathlineto{\pgfqpoint{5.241057in}{0.788835in}}%
\pgfpathlineto{\pgfqpoint{5.236367in}{0.792135in}}%
\pgfpathlineto{\pgfqpoint{5.223401in}{0.784318in}}%
\pgfpathlineto{\pgfqpoint{5.209722in}{0.785761in}}%
\pgfpathlineto{\pgfqpoint{5.199172in}{0.790840in}}%
\pgfpathlineto{\pgfqpoint{5.191783in}{0.788928in}}%
\pgfpathlineto{\pgfqpoint{5.179956in}{0.796112in}}%
\pgfpathlineto{\pgfqpoint{5.173377in}{0.783442in}}%
\pgfpathlineto{\pgfqpoint{5.166649in}{0.794338in}}%
\pgfpathlineto{\pgfqpoint{5.153360in}{0.790111in}}%
\pgfpathlineto{\pgfqpoint{5.141736in}{0.800460in}}%
\pgfpathlineto{\pgfqpoint{5.131579in}{0.792118in}}%
\pgfpathlineto{\pgfqpoint{5.126493in}{0.799784in}}%
\pgfpathlineto{\pgfqpoint{5.119161in}{0.802273in}}%
\pgfpathlineto{\pgfqpoint{5.114696in}{0.809664in}}%
\pgfpathlineto{\pgfqpoint{5.105106in}{0.811767in}}%
\pgfpathlineto{\pgfqpoint{5.099573in}{0.806204in}}%
\pgfpathlineto{\pgfqpoint{5.090171in}{0.813414in}}%
\pgfpathlineto{\pgfqpoint{5.085789in}{0.811767in}}%
\pgfpathlineto{\pgfqpoint{5.070207in}{0.817609in}}%
\pgfpathlineto{\pgfqpoint{5.060448in}{0.818261in}}%
\pgfpathlineto{\pgfqpoint{5.056080in}{0.830667in}}%
\pgfpathlineto{\pgfqpoint{5.048256in}{0.829122in}}%
\pgfpathlineto{\pgfqpoint{5.039296in}{0.832183in}}%
\pgfpathlineto{\pgfqpoint{5.033395in}{0.830414in}}%
\pgfpathlineto{\pgfqpoint{5.020741in}{0.844347in}}%
\pgfpathlineto{\pgfqpoint{5.017215in}{0.843436in}}%
\pgfpathlineto{\pgfqpoint{5.020417in}{0.899940in}}%
\pgfpathlineto{\pgfqpoint{5.023906in}{0.969876in}}%
\pgfpathlineto{\pgfqpoint{4.966639in}{0.973080in}}%
\pgfpathlineto{\pgfqpoint{4.910128in}{0.977352in}}%
\pgfpathlineto{\pgfqpoint{4.866469in}{0.981143in}}%
\pgfpathlineto{\pgfqpoint{4.869368in}{1.014183in}}%
\pgfpathlineto{\pgfqpoint{4.919404in}{1.009408in}}%
\pgfpathlineto{\pgfqpoint{4.976155in}{1.005849in}}%
\pgfpathlineto{\pgfqpoint{5.055191in}{1.001189in}}%
\pgfpathlineto{\pgfqpoint{5.102105in}{0.999379in}}%
\pgfpathlineto{\pgfqpoint{5.158309in}{0.997396in}}%
\pgfpathlineto{\pgfqpoint{5.234656in}{0.996060in}}%
\pgfpathlineto{\pgfqpoint{5.307158in}{0.995725in}}%
\pgfpathlineto{\pgfqpoint{5.307462in}{0.962463in}}%
\pgfpathlineto{\pgfqpoint{5.311531in}{0.937404in}}%
\pgfpathlineto{\pgfqpoint{5.317853in}{0.891120in}}%
\pgfpathlineto{\pgfqpoint{5.316549in}{0.812079in}}%
\pgfpathlineto{\pgfqpoint{5.315948in}{0.774957in}}%
\pgfpathclose%
\pgfusepath{fill}%
\end{pgfscope}%
\begin{pgfscope}%
\pgfpathrectangle{\pgfqpoint{3.625000in}{0.100000in}}{\pgfqpoint{2.989028in}{1.913466in}}%
\pgfusepath{clip}%
\pgfsetbuttcap%
\pgfsetmiterjoin%
\definecolor{currentfill}{rgb}{0.711419,0.883276,0.634833}%
\pgfsetfillcolor{currentfill}%
\pgfsetlinewidth{0.000000pt}%
\definecolor{currentstroke}{rgb}{0.000000,0.000000,0.000000}%
\pgfsetstrokecolor{currentstroke}%
\pgfsetstrokeopacity{0.000000}%
\pgfsetdash{}{0pt}%
\pgfpathmoveto{\pgfqpoint{5.860069in}{0.897708in}}%
\pgfpathlineto{\pgfqpoint{5.860112in}{0.912353in}}%
\pgfpathlineto{\pgfqpoint{5.872839in}{0.917977in}}%
\pgfpathlineto{\pgfqpoint{5.873372in}{0.926967in}}%
\pgfpathlineto{\pgfqpoint{5.884767in}{0.937949in}}%
\pgfpathlineto{\pgfqpoint{5.899011in}{0.940198in}}%
\pgfpathlineto{\pgfqpoint{5.911533in}{0.952301in}}%
\pgfpathlineto{\pgfqpoint{5.924610in}{0.957711in}}%
\pgfpathlineto{\pgfqpoint{5.934425in}{0.973951in}}%
\pgfpathlineto{\pgfqpoint{5.943364in}{0.972772in}}%
\pgfpathlineto{\pgfqpoint{5.962256in}{0.987570in}}%
\pgfpathlineto{\pgfqpoint{5.972232in}{0.987849in}}%
\pgfpathlineto{\pgfqpoint{5.976445in}{0.999123in}}%
\pgfpathlineto{\pgfqpoint{5.984375in}{1.006973in}}%
\pgfpathlineto{\pgfqpoint{5.986189in}{1.020348in}}%
\pgfpathlineto{\pgfqpoint{6.003251in}{1.021788in}}%
\pgfpathlineto{\pgfqpoint{6.072229in}{1.030898in}}%
\pgfpathlineto{\pgfqpoint{6.153447in}{1.044890in}}%
\pgfpathlineto{\pgfqpoint{6.212189in}{1.056913in}}%
\pgfpathlineto{\pgfqpoint{6.277865in}{1.070536in}}%
\pgfpathlineto{\pgfqpoint{6.296631in}{1.044770in}}%
\pgfpathlineto{\pgfqpoint{6.286466in}{1.046415in}}%
\pgfpathlineto{\pgfqpoint{6.273417in}{1.043236in}}%
\pgfpathlineto{\pgfqpoint{6.260559in}{1.030105in}}%
\pgfpathlineto{\pgfqpoint{6.251318in}{1.031050in}}%
\pgfpathlineto{\pgfqpoint{6.250149in}{1.023251in}}%
\pgfpathlineto{\pgfqpoint{6.266852in}{1.029418in}}%
\pgfpathlineto{\pgfqpoint{6.283660in}{1.031961in}}%
\pgfpathlineto{\pgfqpoint{6.289901in}{1.028577in}}%
\pgfpathlineto{\pgfqpoint{6.298300in}{1.032435in}}%
\pgfpathlineto{\pgfqpoint{6.302634in}{1.029732in}}%
\pgfpathlineto{\pgfqpoint{6.306470in}{1.016862in}}%
\pgfpathlineto{\pgfqpoint{6.298487in}{1.012876in}}%
\pgfpathlineto{\pgfqpoint{6.293008in}{0.997182in}}%
\pgfpathlineto{\pgfqpoint{6.287263in}{0.991072in}}%
\pgfpathlineto{\pgfqpoint{6.269653in}{0.992409in}}%
\pgfpathlineto{\pgfqpoint{6.270600in}{1.001624in}}%
\pgfpathlineto{\pgfqpoint{6.260982in}{0.997599in}}%
\pgfpathlineto{\pgfqpoint{6.258890in}{0.989913in}}%
\pgfpathlineto{\pgfqpoint{6.265483in}{0.981952in}}%
\pgfpathlineto{\pgfqpoint{6.267722in}{0.968440in}}%
\pgfpathlineto{\pgfqpoint{6.256935in}{0.960741in}}%
\pgfpathlineto{\pgfqpoint{6.268601in}{0.958032in}}%
\pgfpathlineto{\pgfqpoint{6.273881in}{0.963694in}}%
\pgfpathlineto{\pgfqpoint{6.285553in}{0.964383in}}%
\pgfpathlineto{\pgfqpoint{6.279555in}{0.952158in}}%
\pgfpathlineto{\pgfqpoint{6.272208in}{0.946268in}}%
\pgfpathlineto{\pgfqpoint{6.249984in}{0.940370in}}%
\pgfpathlineto{\pgfqpoint{6.227166in}{0.919675in}}%
\pgfpathlineto{\pgfqpoint{6.213096in}{0.899284in}}%
\pgfpathlineto{\pgfqpoint{6.213013in}{0.891003in}}%
\pgfpathlineto{\pgfqpoint{6.207393in}{0.879575in}}%
\pgfpathlineto{\pgfqpoint{6.178548in}{0.872057in}}%
\pgfpathlineto{\pgfqpoint{6.109952in}{0.921230in}}%
\pgfpathlineto{\pgfqpoint{6.049543in}{0.912354in}}%
\pgfpathlineto{\pgfqpoint{6.049037in}{0.920550in}}%
\pgfpathlineto{\pgfqpoint{6.039855in}{0.929779in}}%
\pgfpathlineto{\pgfqpoint{6.032904in}{0.932038in}}%
\pgfpathlineto{\pgfqpoint{5.967269in}{0.925218in}}%
\pgfpathlineto{\pgfqpoint{5.952183in}{0.920102in}}%
\pgfpathlineto{\pgfqpoint{5.924943in}{0.906551in}}%
\pgfpathlineto{\pgfqpoint{5.901402in}{0.902801in}}%
\pgfpathlineto{\pgfqpoint{5.860069in}{0.897708in}}%
\pgfpathclose%
\pgfusepath{fill}%
\end{pgfscope}%
\begin{pgfscope}%
\pgfpathrectangle{\pgfqpoint{3.625000in}{0.100000in}}{\pgfqpoint{2.989028in}{1.913466in}}%
\pgfusepath{clip}%
\pgfsetbuttcap%
\pgfsetmiterjoin%
\definecolor{currentfill}{rgb}{0.644060,0.856286,0.643522}%
\pgfsetfillcolor{currentfill}%
\pgfsetlinewidth{0.000000pt}%
\definecolor{currentstroke}{rgb}{0.000000,0.000000,0.000000}%
\pgfsetstrokecolor{currentstroke}%
\pgfsetstrokeopacity{0.000000}%
\pgfsetdash{}{0pt}%
\pgfpathmoveto{\pgfqpoint{5.860069in}{0.897708in}}%
\pgfpathlineto{\pgfqpoint{5.791375in}{0.890163in}}%
\pgfpathlineto{\pgfqpoint{5.728533in}{0.884512in}}%
\pgfpathlineto{\pgfqpoint{5.663945in}{0.880335in}}%
\pgfpathlineto{\pgfqpoint{5.652858in}{0.878728in}}%
\pgfpathlineto{\pgfqpoint{5.609319in}{0.875379in}}%
\pgfpathlineto{\pgfqpoint{5.539584in}{0.871310in}}%
\pgfpathlineto{\pgfqpoint{5.551990in}{0.881608in}}%
\pgfpathlineto{\pgfqpoint{5.549159in}{0.892449in}}%
\pgfpathlineto{\pgfqpoint{5.551902in}{0.905590in}}%
\pgfpathlineto{\pgfqpoint{5.556092in}{0.911777in}}%
\pgfpathlineto{\pgfqpoint{5.555934in}{0.920384in}}%
\pgfpathlineto{\pgfqpoint{5.567090in}{0.925797in}}%
\pgfpathlineto{\pgfqpoint{5.567109in}{0.938464in}}%
\pgfpathlineto{\pgfqpoint{5.568642in}{0.953066in}}%
\pgfpathlineto{\pgfqpoint{5.576421in}{0.962397in}}%
\pgfpathlineto{\pgfqpoint{5.575131in}{0.971655in}}%
\pgfpathlineto{\pgfqpoint{5.578146in}{0.971866in}}%
\pgfpathlineto{\pgfqpoint{5.581343in}{0.972329in}}%
\pgfpathlineto{\pgfqpoint{5.653480in}{0.976914in}}%
\pgfpathlineto{\pgfqpoint{5.651065in}{0.988764in}}%
\pgfpathlineto{\pgfqpoint{5.662674in}{0.987156in}}%
\pgfpathlineto{\pgfqpoint{5.732938in}{0.994107in}}%
\pgfpathlineto{\pgfqpoint{5.789337in}{0.997299in}}%
\pgfpathlineto{\pgfqpoint{5.823708in}{1.000190in}}%
\pgfpathlineto{\pgfqpoint{5.880622in}{1.005586in}}%
\pgfpathlineto{\pgfqpoint{5.881571in}{1.006660in}}%
\pgfpathlineto{\pgfqpoint{5.905888in}{1.009068in}}%
\pgfpathlineto{\pgfqpoint{5.986189in}{1.020348in}}%
\pgfpathlineto{\pgfqpoint{5.984375in}{1.006973in}}%
\pgfpathlineto{\pgfqpoint{5.976445in}{0.999123in}}%
\pgfpathlineto{\pgfqpoint{5.972232in}{0.987849in}}%
\pgfpathlineto{\pgfqpoint{5.962256in}{0.987570in}}%
\pgfpathlineto{\pgfqpoint{5.943364in}{0.972772in}}%
\pgfpathlineto{\pgfqpoint{5.934425in}{0.973951in}}%
\pgfpathlineto{\pgfqpoint{5.924610in}{0.957711in}}%
\pgfpathlineto{\pgfqpoint{5.911533in}{0.952301in}}%
\pgfpathlineto{\pgfqpoint{5.899011in}{0.940198in}}%
\pgfpathlineto{\pgfqpoint{5.884767in}{0.937949in}}%
\pgfpathlineto{\pgfqpoint{5.873372in}{0.926967in}}%
\pgfpathlineto{\pgfqpoint{5.872839in}{0.917977in}}%
\pgfpathlineto{\pgfqpoint{5.860112in}{0.912353in}}%
\pgfpathlineto{\pgfqpoint{5.860069in}{0.897708in}}%
\pgfpathclose%
\pgfusepath{fill}%
\end{pgfscope}%
\begin{pgfscope}%
\pgfpathrectangle{\pgfqpoint{3.625000in}{0.100000in}}{\pgfqpoint{2.989028in}{1.913466in}}%
\pgfusepath{clip}%
\pgfsetbuttcap%
\pgfsetmiterjoin%
\definecolor{currentfill}{rgb}{0.224068,0.564552,0.728258}%
\pgfsetfillcolor{currentfill}%
\pgfsetlinewidth{0.000000pt}%
\definecolor{currentstroke}{rgb}{0.000000,0.000000,0.000000}%
\pgfsetstrokecolor{currentstroke}%
\pgfsetstrokeopacity{0.000000}%
\pgfsetdash{}{0pt}%
\pgfpathmoveto{\pgfqpoint{4.641758in}{0.694333in}}%
\pgfpathlineto{\pgfqpoint{4.636859in}{0.702205in}}%
\pgfpathlineto{\pgfqpoint{4.638886in}{0.708987in}}%
\pgfpathlineto{\pgfqpoint{4.673324in}{0.704715in}}%
\pgfpathlineto{\pgfqpoint{4.737359in}{0.697442in}}%
\pgfpathlineto{\pgfqpoint{4.783660in}{0.692892in}}%
\pgfpathlineto{\pgfqpoint{4.837135in}{0.687496in}}%
\pgfpathlineto{\pgfqpoint{4.840044in}{0.721225in}}%
\pgfpathlineto{\pgfqpoint{4.849140in}{0.807015in}}%
\pgfpathlineto{\pgfqpoint{4.855041in}{0.867115in}}%
\pgfpathlineto{\pgfqpoint{4.860142in}{0.924525in}}%
\pgfpathlineto{\pgfqpoint{4.864896in}{0.981231in}}%
\pgfpathlineto{\pgfqpoint{4.866469in}{0.981143in}}%
\pgfpathlineto{\pgfqpoint{4.910128in}{0.977352in}}%
\pgfpathlineto{\pgfqpoint{4.966639in}{0.973080in}}%
\pgfpathlineto{\pgfqpoint{5.023906in}{0.969876in}}%
\pgfpathlineto{\pgfqpoint{5.020417in}{0.899940in}}%
\pgfpathlineto{\pgfqpoint{5.017215in}{0.843436in}}%
\pgfpathlineto{\pgfqpoint{5.020741in}{0.844347in}}%
\pgfpathlineto{\pgfqpoint{5.033395in}{0.830414in}}%
\pgfpathlineto{\pgfqpoint{5.039296in}{0.832183in}}%
\pgfpathlineto{\pgfqpoint{5.048256in}{0.829122in}}%
\pgfpathlineto{\pgfqpoint{5.056080in}{0.830667in}}%
\pgfpathlineto{\pgfqpoint{5.060448in}{0.818261in}}%
\pgfpathlineto{\pgfqpoint{5.070207in}{0.817609in}}%
\pgfpathlineto{\pgfqpoint{5.085789in}{0.811767in}}%
\pgfpathlineto{\pgfqpoint{5.090171in}{0.813414in}}%
\pgfpathlineto{\pgfqpoint{5.099573in}{0.806204in}}%
\pgfpathlineto{\pgfqpoint{5.105106in}{0.811767in}}%
\pgfpathlineto{\pgfqpoint{5.114696in}{0.809664in}}%
\pgfpathlineto{\pgfqpoint{5.119161in}{0.802273in}}%
\pgfpathlineto{\pgfqpoint{5.126493in}{0.799784in}}%
\pgfpathlineto{\pgfqpoint{5.131579in}{0.792118in}}%
\pgfpathlineto{\pgfqpoint{5.141736in}{0.800460in}}%
\pgfpathlineto{\pgfqpoint{5.153360in}{0.790111in}}%
\pgfpathlineto{\pgfqpoint{5.166649in}{0.794338in}}%
\pgfpathlineto{\pgfqpoint{5.173377in}{0.783442in}}%
\pgfpathlineto{\pgfqpoint{5.179956in}{0.796112in}}%
\pgfpathlineto{\pgfqpoint{5.191783in}{0.788928in}}%
\pgfpathlineto{\pgfqpoint{5.199172in}{0.790840in}}%
\pgfpathlineto{\pgfqpoint{5.209722in}{0.785761in}}%
\pgfpathlineto{\pgfqpoint{5.223401in}{0.784318in}}%
\pgfpathlineto{\pgfqpoint{5.236367in}{0.792135in}}%
\pgfpathlineto{\pgfqpoint{5.241057in}{0.788835in}}%
\pgfpathlineto{\pgfqpoint{5.254234in}{0.795535in}}%
\pgfpathlineto{\pgfqpoint{5.262631in}{0.790306in}}%
\pgfpathlineto{\pgfqpoint{5.269268in}{0.790617in}}%
\pgfpathlineto{\pgfqpoint{5.274524in}{0.796776in}}%
\pgfpathlineto{\pgfqpoint{5.282795in}{0.793711in}}%
\pgfpathlineto{\pgfqpoint{5.301416in}{0.779451in}}%
\pgfpathlineto{\pgfqpoint{5.315948in}{0.774957in}}%
\pgfpathlineto{\pgfqpoint{5.321775in}{0.769451in}}%
\pgfpathlineto{\pgfqpoint{5.329070in}{0.772449in}}%
\pgfpathlineto{\pgfqpoint{5.340123in}{0.770153in}}%
\pgfpathlineto{\pgfqpoint{5.340340in}{0.735108in}}%
\pgfpathlineto{\pgfqpoint{5.341262in}{0.667326in}}%
\pgfpathlineto{\pgfqpoint{5.348939in}{0.660813in}}%
\pgfpathlineto{\pgfqpoint{5.355092in}{0.648811in}}%
\pgfpathlineto{\pgfqpoint{5.352929in}{0.640781in}}%
\pgfpathlineto{\pgfqpoint{5.357761in}{0.637381in}}%
\pgfpathlineto{\pgfqpoint{5.361688in}{0.622574in}}%
\pgfpathlineto{\pgfqpoint{5.369209in}{0.614644in}}%
\pgfpathlineto{\pgfqpoint{5.370894in}{0.597808in}}%
\pgfpathlineto{\pgfqpoint{5.369586in}{0.590680in}}%
\pgfpathlineto{\pgfqpoint{5.359363in}{0.571815in}}%
\pgfpathlineto{\pgfqpoint{5.358187in}{0.558251in}}%
\pgfpathlineto{\pgfqpoint{5.361655in}{0.555420in}}%
\pgfpathlineto{\pgfqpoint{5.361956in}{0.543537in}}%
\pgfpathlineto{\pgfqpoint{5.358432in}{0.536068in}}%
\pgfpathlineto{\pgfqpoint{5.352950in}{0.533255in}}%
\pgfpathlineto{\pgfqpoint{5.347582in}{0.523496in}}%
\pgfpathlineto{\pgfqpoint{5.354424in}{0.514043in}}%
\pgfpathlineto{\pgfqpoint{5.341144in}{0.513857in}}%
\pgfpathlineto{\pgfqpoint{5.305636in}{0.497611in}}%
\pgfpathlineto{\pgfqpoint{5.312415in}{0.507333in}}%
\pgfpathlineto{\pgfqpoint{5.304206in}{0.512575in}}%
\pgfpathlineto{\pgfqpoint{5.302490in}{0.521488in}}%
\pgfpathlineto{\pgfqpoint{5.286468in}{0.505964in}}%
\pgfpathlineto{\pgfqpoint{5.293579in}{0.495354in}}%
\pgfpathlineto{\pgfqpoint{5.283435in}{0.481844in}}%
\pgfpathlineto{\pgfqpoint{5.277976in}{0.482126in}}%
\pgfpathlineto{\pgfqpoint{5.272841in}{0.467413in}}%
\pgfpathlineto{\pgfqpoint{5.256600in}{0.455840in}}%
\pgfpathlineto{\pgfqpoint{5.241427in}{0.451669in}}%
\pgfpathlineto{\pgfqpoint{5.214964in}{0.440819in}}%
\pgfpathlineto{\pgfqpoint{5.212222in}{0.446871in}}%
\pgfpathlineto{\pgfqpoint{5.195737in}{0.441300in}}%
\pgfpathlineto{\pgfqpoint{5.205876in}{0.431814in}}%
\pgfpathlineto{\pgfqpoint{5.189883in}{0.423573in}}%
\pgfpathlineto{\pgfqpoint{5.172633in}{0.411131in}}%
\pgfpathlineto{\pgfqpoint{5.168490in}{0.416915in}}%
\pgfpathlineto{\pgfqpoint{5.154373in}{0.408246in}}%
\pgfpathlineto{\pgfqpoint{5.168210in}{0.404954in}}%
\pgfpathlineto{\pgfqpoint{5.157777in}{0.391324in}}%
\pgfpathlineto{\pgfqpoint{5.152685in}{0.395380in}}%
\pgfpathlineto{\pgfqpoint{5.145845in}{0.388865in}}%
\pgfpathlineto{\pgfqpoint{5.154369in}{0.383174in}}%
\pgfpathlineto{\pgfqpoint{5.149329in}{0.374856in}}%
\pgfpathlineto{\pgfqpoint{5.143135in}{0.355164in}}%
\pgfpathlineto{\pgfqpoint{5.134200in}{0.336199in}}%
\pgfpathlineto{\pgfqpoint{5.138211in}{0.323767in}}%
\pgfpathlineto{\pgfqpoint{5.145029in}{0.294458in}}%
\pgfpathlineto{\pgfqpoint{5.145643in}{0.282596in}}%
\pgfpathlineto{\pgfqpoint{5.156178in}{0.267038in}}%
\pgfpathlineto{\pgfqpoint{5.148054in}{0.267947in}}%
\pgfpathlineto{\pgfqpoint{5.140164in}{0.260098in}}%
\pgfpathlineto{\pgfqpoint{5.130923in}{0.266267in}}%
\pgfpathlineto{\pgfqpoint{5.127605in}{0.272421in}}%
\pgfpathlineto{\pgfqpoint{5.114453in}{0.275284in}}%
\pgfpathlineto{\pgfqpoint{5.094363in}{0.275635in}}%
\pgfpathlineto{\pgfqpoint{5.079561in}{0.287289in}}%
\pgfpathlineto{\pgfqpoint{5.066120in}{0.289236in}}%
\pgfpathlineto{\pgfqpoint{5.057966in}{0.298497in}}%
\pgfpathlineto{\pgfqpoint{5.040889in}{0.302225in}}%
\pgfpathlineto{\pgfqpoint{5.031552in}{0.332004in}}%
\pgfpathlineto{\pgfqpoint{5.022004in}{0.343934in}}%
\pgfpathlineto{\pgfqpoint{5.023626in}{0.355270in}}%
\pgfpathlineto{\pgfqpoint{5.017706in}{0.363554in}}%
\pgfpathlineto{\pgfqpoint{5.021421in}{0.374877in}}%
\pgfpathlineto{\pgfqpoint{5.018356in}{0.383175in}}%
\pgfpathlineto{\pgfqpoint{5.008762in}{0.386934in}}%
\pgfpathlineto{\pgfqpoint{4.999791in}{0.396502in}}%
\pgfpathlineto{\pgfqpoint{4.996522in}{0.409312in}}%
\pgfpathlineto{\pgfqpoint{4.988033in}{0.420954in}}%
\pgfpathlineto{\pgfqpoint{4.976735in}{0.430012in}}%
\pgfpathlineto{\pgfqpoint{4.966558in}{0.455996in}}%
\pgfpathlineto{\pgfqpoint{4.958340in}{0.484413in}}%
\pgfpathlineto{\pgfqpoint{4.951597in}{0.495650in}}%
\pgfpathlineto{\pgfqpoint{4.939908in}{0.505109in}}%
\pgfpathlineto{\pgfqpoint{4.936987in}{0.511975in}}%
\pgfpathlineto{\pgfqpoint{4.926042in}{0.516238in}}%
\pgfpathlineto{\pgfqpoint{4.915239in}{0.534431in}}%
\pgfpathlineto{\pgfqpoint{4.905374in}{0.533035in}}%
\pgfpathlineto{\pgfqpoint{4.895356in}{0.537579in}}%
\pgfpathlineto{\pgfqpoint{4.881156in}{0.536700in}}%
\pgfpathlineto{\pgfqpoint{4.866731in}{0.544199in}}%
\pgfpathlineto{\pgfqpoint{4.862663in}{0.537067in}}%
\pgfpathlineto{\pgfqpoint{4.845804in}{0.536889in}}%
\pgfpathlineto{\pgfqpoint{4.837222in}{0.523370in}}%
\pgfpathlineto{\pgfqpoint{4.829757in}{0.506631in}}%
\pgfpathlineto{\pgfqpoint{4.813883in}{0.488663in}}%
\pgfpathlineto{\pgfqpoint{4.795899in}{0.496548in}}%
\pgfpathlineto{\pgfqpoint{4.793347in}{0.501760in}}%
\pgfpathlineto{\pgfqpoint{4.782418in}{0.505734in}}%
\pgfpathlineto{\pgfqpoint{4.779061in}{0.511171in}}%
\pgfpathlineto{\pgfqpoint{4.764552in}{0.516660in}}%
\pgfpathlineto{\pgfqpoint{4.756436in}{0.527879in}}%
\pgfpathlineto{\pgfqpoint{4.746957in}{0.533292in}}%
\pgfpathlineto{\pgfqpoint{4.738802in}{0.542738in}}%
\pgfpathlineto{\pgfqpoint{4.732448in}{0.558735in}}%
\pgfpathlineto{\pgfqpoint{4.733159in}{0.580595in}}%
\pgfpathlineto{\pgfqpoint{4.725706in}{0.591641in}}%
\pgfpathlineto{\pgfqpoint{4.724848in}{0.603605in}}%
\pgfpathlineto{\pgfqpoint{4.708312in}{0.621518in}}%
\pgfpathlineto{\pgfqpoint{4.698690in}{0.625342in}}%
\pgfpathlineto{\pgfqpoint{4.688479in}{0.642057in}}%
\pgfpathlineto{\pgfqpoint{4.679778in}{0.648698in}}%
\pgfpathlineto{\pgfqpoint{4.668723in}{0.664869in}}%
\pgfpathlineto{\pgfqpoint{4.657411in}{0.671867in}}%
\pgfpathlineto{\pgfqpoint{4.650003in}{0.689800in}}%
\pgfpathlineto{\pgfqpoint{4.641758in}{0.694333in}}%
\pgfpathclose%
\pgfusepath{fill}%
\end{pgfscope}%
\begin{pgfscope}%
\pgfpathrectangle{\pgfqpoint{3.625000in}{0.100000in}}{\pgfqpoint{2.989028in}{1.913466in}}%
\pgfusepath{clip}%
\pgfsetbuttcap%
\pgfsetmiterjoin%
\definecolor{currentfill}{rgb}{0.400000,0.760784,0.647059}%
\pgfsetfillcolor{currentfill}%
\pgfsetlinewidth{0.000000pt}%
\definecolor{currentstroke}{rgb}{0.000000,0.000000,0.000000}%
\pgfsetstrokecolor{currentstroke}%
\pgfsetstrokeopacity{0.000000}%
\pgfsetdash{}{0pt}%
\pgfpathmoveto{\pgfqpoint{4.555137in}{1.052362in}}%
\pgfpathlineto{\pgfqpoint{4.636744in}{1.040510in}}%
\pgfpathlineto{\pgfqpoint{4.668359in}{1.035639in}}%
\pgfpathlineto{\pgfqpoint{4.757477in}{1.025040in}}%
\pgfpathlineto{\pgfqpoint{4.817460in}{1.018783in}}%
\pgfpathlineto{\pgfqpoint{4.869368in}{1.014183in}}%
\pgfpathlineto{\pgfqpoint{4.866469in}{0.981143in}}%
\pgfpathlineto{\pgfqpoint{4.864896in}{0.981231in}}%
\pgfpathlineto{\pgfqpoint{4.860142in}{0.924525in}}%
\pgfpathlineto{\pgfqpoint{4.855041in}{0.867115in}}%
\pgfpathlineto{\pgfqpoint{4.849140in}{0.807015in}}%
\pgfpathlineto{\pgfqpoint{4.840044in}{0.721225in}}%
\pgfpathlineto{\pgfqpoint{4.837135in}{0.687496in}}%
\pgfpathlineto{\pgfqpoint{4.783660in}{0.692892in}}%
\pgfpathlineto{\pgfqpoint{4.737359in}{0.697442in}}%
\pgfpathlineto{\pgfqpoint{4.673324in}{0.704715in}}%
\pgfpathlineto{\pgfqpoint{4.638886in}{0.708987in}}%
\pgfpathlineto{\pgfqpoint{4.636859in}{0.702205in}}%
\pgfpathlineto{\pgfqpoint{4.641758in}{0.694333in}}%
\pgfpathlineto{\pgfqpoint{4.600370in}{0.699708in}}%
\pgfpathlineto{\pgfqpoint{4.549306in}{0.707037in}}%
\pgfpathlineto{\pgfqpoint{4.544659in}{0.678155in}}%
\pgfpathlineto{\pgfqpoint{4.498080in}{0.685166in}}%
\pgfpathlineto{\pgfqpoint{4.509025in}{0.756815in}}%
\pgfpathlineto{\pgfqpoint{4.522573in}{0.844052in}}%
\pgfpathlineto{\pgfqpoint{4.534619in}{0.920049in}}%
\pgfpathlineto{\pgfqpoint{4.545079in}{0.987601in}}%
\pgfpathlineto{\pgfqpoint{4.555137in}{1.052362in}}%
\pgfpathclose%
\pgfusepath{fill}%
\end{pgfscope}%
\begin{pgfscope}%
\pgfpathrectangle{\pgfqpoint{3.625000in}{0.100000in}}{\pgfqpoint{2.989028in}{1.913466in}}%
\pgfusepath{clip}%
\pgfsetbuttcap%
\pgfsetmiterjoin%
\definecolor{currentfill}{rgb}{0.756786,0.901730,0.625606}%
\pgfsetfillcolor{currentfill}%
\pgfsetlinewidth{0.000000pt}%
\definecolor{currentstroke}{rgb}{0.000000,0.000000,0.000000}%
\pgfsetstrokecolor{currentstroke}%
\pgfsetstrokeopacity{0.000000}%
\pgfsetdash{}{0pt}%
\pgfpathmoveto{\pgfqpoint{5.852687in}{0.632577in}}%
\pgfpathlineto{\pgfqpoint{5.774834in}{0.623930in}}%
\pgfpathlineto{\pgfqpoint{5.706207in}{0.618602in}}%
\pgfpathlineto{\pgfqpoint{5.705349in}{0.610192in}}%
\pgfpathlineto{\pgfqpoint{5.711652in}{0.602188in}}%
\pgfpathlineto{\pgfqpoint{5.719346in}{0.597485in}}%
\pgfpathlineto{\pgfqpoint{5.719230in}{0.585103in}}%
\pgfpathlineto{\pgfqpoint{5.717191in}{0.576831in}}%
\pgfpathlineto{\pgfqpoint{5.710400in}{0.570869in}}%
\pgfpathlineto{\pgfqpoint{5.700936in}{0.571500in}}%
\pgfpathlineto{\pgfqpoint{5.691987in}{0.578887in}}%
\pgfpathlineto{\pgfqpoint{5.690392in}{0.592032in}}%
\pgfpathlineto{\pgfqpoint{5.683727in}{0.599677in}}%
\pgfpathlineto{\pgfqpoint{5.679190in}{0.572315in}}%
\pgfpathlineto{\pgfqpoint{5.663775in}{0.574915in}}%
\pgfpathlineto{\pgfqpoint{5.658415in}{0.622761in}}%
\pgfpathlineto{\pgfqpoint{5.652610in}{0.673185in}}%
\pgfpathlineto{\pgfqpoint{5.654766in}{0.745935in}}%
\pgfpathlineto{\pgfqpoint{5.656104in}{0.795857in}}%
\pgfpathlineto{\pgfqpoint{5.658950in}{0.872025in}}%
\pgfpathlineto{\pgfqpoint{5.652858in}{0.878728in}}%
\pgfpathlineto{\pgfqpoint{5.663945in}{0.880335in}}%
\pgfpathlineto{\pgfqpoint{5.728533in}{0.884512in}}%
\pgfpathlineto{\pgfqpoint{5.791375in}{0.890163in}}%
\pgfpathlineto{\pgfqpoint{5.807908in}{0.832251in}}%
\pgfpathlineto{\pgfqpoint{5.819184in}{0.789743in}}%
\pgfpathlineto{\pgfqpoint{5.829291in}{0.754156in}}%
\pgfpathlineto{\pgfqpoint{5.835136in}{0.739826in}}%
\pgfpathlineto{\pgfqpoint{5.844340in}{0.726491in}}%
\pgfpathlineto{\pgfqpoint{5.842853in}{0.719705in}}%
\pgfpathlineto{\pgfqpoint{5.849428in}{0.716398in}}%
\pgfpathlineto{\pgfqpoint{5.841659in}{0.706111in}}%
\pgfpathlineto{\pgfqpoint{5.839034in}{0.687792in}}%
\pgfpathlineto{\pgfqpoint{5.846628in}{0.666365in}}%
\pgfpathlineto{\pgfqpoint{5.845573in}{0.644805in}}%
\pgfpathlineto{\pgfqpoint{5.852687in}{0.632577in}}%
\pgfpathclose%
\pgfusepath{fill}%
\end{pgfscope}%
\begin{pgfscope}%
\pgfpathrectangle{\pgfqpoint{3.625000in}{0.100000in}}{\pgfqpoint{2.989028in}{1.913466in}}%
\pgfusepath{clip}%
\pgfsetbuttcap%
\pgfsetmiterjoin%
\definecolor{currentfill}{rgb}{0.944252,0.977701,0.662053}%
\pgfsetfillcolor{currentfill}%
\pgfsetlinewidth{0.000000pt}%
\definecolor{currentstroke}{rgb}{0.000000,0.000000,0.000000}%
\pgfsetstrokecolor{currentstroke}%
\pgfsetstrokeopacity{0.000000}%
\pgfsetdash{}{0pt}%
\pgfpathmoveto{\pgfqpoint{5.663775in}{0.574915in}}%
\pgfpathlineto{\pgfqpoint{5.647957in}{0.570389in}}%
\pgfpathlineto{\pgfqpoint{5.636166in}{0.573295in}}%
\pgfpathlineto{\pgfqpoint{5.614268in}{0.566330in}}%
\pgfpathlineto{\pgfqpoint{5.597753in}{0.557360in}}%
\pgfpathlineto{\pgfqpoint{5.590977in}{0.573569in}}%
\pgfpathlineto{\pgfqpoint{5.583992in}{0.580363in}}%
\pgfpathlineto{\pgfqpoint{5.580622in}{0.588493in}}%
\pgfpathlineto{\pgfqpoint{5.585562in}{0.610498in}}%
\pgfpathlineto{\pgfqpoint{5.538761in}{0.607617in}}%
\pgfpathlineto{\pgfqpoint{5.478073in}{0.604986in}}%
\pgfpathlineto{\pgfqpoint{5.481684in}{0.610463in}}%
\pgfpathlineto{\pgfqpoint{5.477221in}{0.620808in}}%
\pgfpathlineto{\pgfqpoint{5.481668in}{0.622916in}}%
\pgfpathlineto{\pgfqpoint{5.480681in}{0.632856in}}%
\pgfpathlineto{\pgfqpoint{5.488432in}{0.642495in}}%
\pgfpathlineto{\pgfqpoint{5.492673in}{0.661207in}}%
\pgfpathlineto{\pgfqpoint{5.506859in}{0.673535in}}%
\pgfpathlineto{\pgfqpoint{5.501601in}{0.685508in}}%
\pgfpathlineto{\pgfqpoint{5.511660in}{0.692400in}}%
\pgfpathlineto{\pgfqpoint{5.502983in}{0.704877in}}%
\pgfpathlineto{\pgfqpoint{5.496703in}{0.733439in}}%
\pgfpathlineto{\pgfqpoint{5.499085in}{0.738625in}}%
\pgfpathlineto{\pgfqpoint{5.502845in}{0.748573in}}%
\pgfpathlineto{\pgfqpoint{5.499345in}{0.759012in}}%
\pgfpathlineto{\pgfqpoint{5.501057in}{0.763762in}}%
\pgfpathlineto{\pgfqpoint{5.494899in}{0.781706in}}%
\pgfpathlineto{\pgfqpoint{5.513077in}{0.813926in}}%
\pgfpathlineto{\pgfqpoint{5.513922in}{0.822501in}}%
\pgfpathlineto{\pgfqpoint{5.522678in}{0.828737in}}%
\pgfpathlineto{\pgfqpoint{5.532021in}{0.849323in}}%
\pgfpathlineto{\pgfqpoint{5.531577in}{0.857690in}}%
\pgfpathlineto{\pgfqpoint{5.543214in}{0.866243in}}%
\pgfpathlineto{\pgfqpoint{5.539584in}{0.871310in}}%
\pgfpathlineto{\pgfqpoint{5.609319in}{0.875379in}}%
\pgfpathlineto{\pgfqpoint{5.652858in}{0.878728in}}%
\pgfpathlineto{\pgfqpoint{5.658950in}{0.872025in}}%
\pgfpathlineto{\pgfqpoint{5.656104in}{0.795857in}}%
\pgfpathlineto{\pgfqpoint{5.654766in}{0.745935in}}%
\pgfpathlineto{\pgfqpoint{5.652610in}{0.673185in}}%
\pgfpathlineto{\pgfqpoint{5.658415in}{0.622761in}}%
\pgfpathlineto{\pgfqpoint{5.663775in}{0.574915in}}%
\pgfpathclose%
\pgfusepath{fill}%
\end{pgfscope}%
\begin{pgfscope}%
\pgfpathrectangle{\pgfqpoint{3.625000in}{0.100000in}}{\pgfqpoint{2.989028in}{1.913466in}}%
\pgfusepath{clip}%
\pgfsetbuttcap%
\pgfsetmiterjoin%
\definecolor{currentfill}{rgb}{0.990388,0.996155,0.734025}%
\pgfsetfillcolor{currentfill}%
\pgfsetlinewidth{0.000000pt}%
\definecolor{currentstroke}{rgb}{0.000000,0.000000,0.000000}%
\pgfsetstrokecolor{currentstroke}%
\pgfsetstrokeopacity{0.000000}%
\pgfsetdash{}{0pt}%
\pgfpathmoveto{\pgfqpoint{5.791375in}{0.890163in}}%
\pgfpathlineto{\pgfqpoint{5.860069in}{0.897708in}}%
\pgfpathlineto{\pgfqpoint{5.901402in}{0.902801in}}%
\pgfpathlineto{\pgfqpoint{5.924943in}{0.906551in}}%
\pgfpathlineto{\pgfqpoint{5.915192in}{0.891171in}}%
\pgfpathlineto{\pgfqpoint{5.915198in}{0.883919in}}%
\pgfpathlineto{\pgfqpoint{5.925235in}{0.880009in}}%
\pgfpathlineto{\pgfqpoint{5.932061in}{0.873695in}}%
\pgfpathlineto{\pgfqpoint{5.942377in}{0.872915in}}%
\pgfpathlineto{\pgfqpoint{5.952019in}{0.855143in}}%
\pgfpathlineto{\pgfqpoint{5.962478in}{0.842609in}}%
\pgfpathlineto{\pgfqpoint{5.981053in}{0.832075in}}%
\pgfpathlineto{\pgfqpoint{5.986332in}{0.823674in}}%
\pgfpathlineto{\pgfqpoint{6.002896in}{0.814587in}}%
\pgfpathlineto{\pgfqpoint{6.004142in}{0.807998in}}%
\pgfpathlineto{\pgfqpoint{6.017579in}{0.794896in}}%
\pgfpathlineto{\pgfqpoint{6.027872in}{0.791165in}}%
\pgfpathlineto{\pgfqpoint{6.035161in}{0.778793in}}%
\pgfpathlineto{\pgfqpoint{6.038387in}{0.764886in}}%
\pgfpathlineto{\pgfqpoint{6.049078in}{0.759511in}}%
\pgfpathlineto{\pgfqpoint{6.057685in}{0.744563in}}%
\pgfpathlineto{\pgfqpoint{6.060460in}{0.733570in}}%
\pgfpathlineto{\pgfqpoint{6.073073in}{0.728890in}}%
\pgfpathlineto{\pgfqpoint{6.062490in}{0.708626in}}%
\pgfpathlineto{\pgfqpoint{6.058507in}{0.696651in}}%
\pgfpathlineto{\pgfqpoint{6.061105in}{0.691037in}}%
\pgfpathlineto{\pgfqpoint{6.056568in}{0.681708in}}%
\pgfpathlineto{\pgfqpoint{6.054639in}{0.668810in}}%
\pgfpathlineto{\pgfqpoint{6.046570in}{0.666408in}}%
\pgfpathlineto{\pgfqpoint{6.050831in}{0.654599in}}%
\pgfpathlineto{\pgfqpoint{6.050559in}{0.639617in}}%
\pgfpathlineto{\pgfqpoint{6.036217in}{0.640178in}}%
\pgfpathlineto{\pgfqpoint{6.026225in}{0.642912in}}%
\pgfpathlineto{\pgfqpoint{6.021902in}{0.637830in}}%
\pgfpathlineto{\pgfqpoint{6.025128in}{0.625826in}}%
\pgfpathlineto{\pgfqpoint{6.024427in}{0.611870in}}%
\pgfpathlineto{\pgfqpoint{6.018123in}{0.610802in}}%
\pgfpathlineto{\pgfqpoint{6.013024in}{0.623836in}}%
\pgfpathlineto{\pgfqpoint{5.961103in}{0.620376in}}%
\pgfpathlineto{\pgfqpoint{5.895635in}{0.616804in}}%
\pgfpathlineto{\pgfqpoint{5.862617in}{0.614480in}}%
\pgfpathlineto{\pgfqpoint{5.852687in}{0.632577in}}%
\pgfpathlineto{\pgfqpoint{5.845573in}{0.644805in}}%
\pgfpathlineto{\pgfqpoint{5.846628in}{0.666365in}}%
\pgfpathlineto{\pgfqpoint{5.839034in}{0.687792in}}%
\pgfpathlineto{\pgfqpoint{5.841659in}{0.706111in}}%
\pgfpathlineto{\pgfqpoint{5.849428in}{0.716398in}}%
\pgfpathlineto{\pgfqpoint{5.842853in}{0.719705in}}%
\pgfpathlineto{\pgfqpoint{5.844340in}{0.726491in}}%
\pgfpathlineto{\pgfqpoint{5.835136in}{0.739826in}}%
\pgfpathlineto{\pgfqpoint{5.829291in}{0.754156in}}%
\pgfpathlineto{\pgfqpoint{5.819184in}{0.789743in}}%
\pgfpathlineto{\pgfqpoint{5.807908in}{0.832251in}}%
\pgfpathlineto{\pgfqpoint{5.791375in}{0.890163in}}%
\pgfpathclose%
\pgfusepath{fill}%
\end{pgfscope}%
\begin{pgfscope}%
\pgfpathrectangle{\pgfqpoint{3.625000in}{0.100000in}}{\pgfqpoint{2.989028in}{1.913466in}}%
\pgfusepath{clip}%
\pgfsetbuttcap%
\pgfsetmiterjoin%
\definecolor{currentfill}{rgb}{0.622837,0.847982,0.643829}%
\pgfsetfillcolor{currentfill}%
\pgfsetlinewidth{0.000000pt}%
\definecolor{currentstroke}{rgb}{0.000000,0.000000,0.000000}%
\pgfsetstrokecolor{currentstroke}%
\pgfsetstrokeopacity{0.000000}%
\pgfsetdash{}{0pt}%
\pgfpathmoveto{\pgfqpoint{5.924943in}{0.906551in}}%
\pgfpathlineto{\pgfqpoint{5.952183in}{0.920102in}}%
\pgfpathlineto{\pgfqpoint{5.967269in}{0.925218in}}%
\pgfpathlineto{\pgfqpoint{6.032904in}{0.932038in}}%
\pgfpathlineto{\pgfqpoint{6.039855in}{0.929779in}}%
\pgfpathlineto{\pgfqpoint{6.049037in}{0.920550in}}%
\pgfpathlineto{\pgfqpoint{6.049543in}{0.912354in}}%
\pgfpathlineto{\pgfqpoint{6.109952in}{0.921230in}}%
\pgfpathlineto{\pgfqpoint{6.178548in}{0.872057in}}%
\pgfpathlineto{\pgfqpoint{6.165684in}{0.858649in}}%
\pgfpathlineto{\pgfqpoint{6.148033in}{0.827511in}}%
\pgfpathlineto{\pgfqpoint{6.153057in}{0.820819in}}%
\pgfpathlineto{\pgfqpoint{6.143641in}{0.807830in}}%
\pgfpathlineto{\pgfqpoint{6.134275in}{0.806359in}}%
\pgfpathlineto{\pgfqpoint{6.134240in}{0.798554in}}%
\pgfpathlineto{\pgfqpoint{6.127464in}{0.790386in}}%
\pgfpathlineto{\pgfqpoint{6.119032in}{0.788730in}}%
\pgfpathlineto{\pgfqpoint{6.120872in}{0.781473in}}%
\pgfpathlineto{\pgfqpoint{6.116167in}{0.775902in}}%
\pgfpathlineto{\pgfqpoint{6.104911in}{0.771089in}}%
\pgfpathlineto{\pgfqpoint{6.091192in}{0.760141in}}%
\pgfpathlineto{\pgfqpoint{6.093782in}{0.753058in}}%
\pgfpathlineto{\pgfqpoint{6.085148in}{0.748611in}}%
\pgfpathlineto{\pgfqpoint{6.077866in}{0.753287in}}%
\pgfpathlineto{\pgfqpoint{6.072540in}{0.732958in}}%
\pgfpathlineto{\pgfqpoint{6.060460in}{0.733570in}}%
\pgfpathlineto{\pgfqpoint{6.057685in}{0.744563in}}%
\pgfpathlineto{\pgfqpoint{6.049078in}{0.759511in}}%
\pgfpathlineto{\pgfqpoint{6.038387in}{0.764886in}}%
\pgfpathlineto{\pgfqpoint{6.035161in}{0.778793in}}%
\pgfpathlineto{\pgfqpoint{6.027872in}{0.791165in}}%
\pgfpathlineto{\pgfqpoint{6.017579in}{0.794896in}}%
\pgfpathlineto{\pgfqpoint{6.004142in}{0.807998in}}%
\pgfpathlineto{\pgfqpoint{6.002896in}{0.814587in}}%
\pgfpathlineto{\pgfqpoint{5.986332in}{0.823674in}}%
\pgfpathlineto{\pgfqpoint{5.981053in}{0.832075in}}%
\pgfpathlineto{\pgfqpoint{5.962478in}{0.842609in}}%
\pgfpathlineto{\pgfqpoint{5.952019in}{0.855143in}}%
\pgfpathlineto{\pgfqpoint{5.942377in}{0.872915in}}%
\pgfpathlineto{\pgfqpoint{5.932061in}{0.873695in}}%
\pgfpathlineto{\pgfqpoint{5.925235in}{0.880009in}}%
\pgfpathlineto{\pgfqpoint{5.915198in}{0.883919in}}%
\pgfpathlineto{\pgfqpoint{5.915192in}{0.891171in}}%
\pgfpathlineto{\pgfqpoint{5.924943in}{0.906551in}}%
\pgfpathclose%
\pgfusepath{fill}%
\end{pgfscope}%
\begin{pgfscope}%
\pgfpathrectangle{\pgfqpoint{3.625000in}{0.100000in}}{\pgfqpoint{2.989028in}{1.913466in}}%
\pgfusepath{clip}%
\pgfsetbuttcap%
\pgfsetmiterjoin%
\definecolor{currentfill}{rgb}{0.537947,0.814764,0.645060}%
\pgfsetfillcolor{currentfill}%
\pgfsetlinewidth{0.000000pt}%
\definecolor{currentstroke}{rgb}{0.000000,0.000000,0.000000}%
\pgfsetstrokecolor{currentstroke}%
\pgfsetstrokeopacity{0.000000}%
\pgfsetdash{}{0pt}%
\pgfpathmoveto{\pgfqpoint{5.307462in}{0.962463in}}%
\pgfpathlineto{\pgfqpoint{5.347491in}{0.962824in}}%
\pgfpathlineto{\pgfqpoint{5.400448in}{0.963789in}}%
\pgfpathlineto{\pgfqpoint{5.490941in}{0.966662in}}%
\pgfpathlineto{\pgfqpoint{5.542700in}{0.969399in}}%
\pgfpathlineto{\pgfqpoint{5.548251in}{0.962527in}}%
\pgfpathlineto{\pgfqpoint{5.547904in}{0.955269in}}%
\pgfpathlineto{\pgfqpoint{5.535365in}{0.942748in}}%
\pgfpathlineto{\pgfqpoint{5.532334in}{0.935892in}}%
\pgfpathlineto{\pgfqpoint{5.567109in}{0.938464in}}%
\pgfpathlineto{\pgfqpoint{5.567090in}{0.925797in}}%
\pgfpathlineto{\pgfqpoint{5.555934in}{0.920384in}}%
\pgfpathlineto{\pgfqpoint{5.556092in}{0.911777in}}%
\pgfpathlineto{\pgfqpoint{5.551902in}{0.905590in}}%
\pgfpathlineto{\pgfqpoint{5.549159in}{0.892449in}}%
\pgfpathlineto{\pgfqpoint{5.551990in}{0.881608in}}%
\pgfpathlineto{\pgfqpoint{5.539584in}{0.871310in}}%
\pgfpathlineto{\pgfqpoint{5.543214in}{0.866243in}}%
\pgfpathlineto{\pgfqpoint{5.531577in}{0.857690in}}%
\pgfpathlineto{\pgfqpoint{5.532021in}{0.849323in}}%
\pgfpathlineto{\pgfqpoint{5.522678in}{0.828737in}}%
\pgfpathlineto{\pgfqpoint{5.513922in}{0.822501in}}%
\pgfpathlineto{\pgfqpoint{5.513077in}{0.813926in}}%
\pgfpathlineto{\pgfqpoint{5.494899in}{0.781706in}}%
\pgfpathlineto{\pgfqpoint{5.501057in}{0.763762in}}%
\pgfpathlineto{\pgfqpoint{5.499345in}{0.759012in}}%
\pgfpathlineto{\pgfqpoint{5.502845in}{0.748573in}}%
\pgfpathlineto{\pgfqpoint{5.499085in}{0.738625in}}%
\pgfpathlineto{\pgfqpoint{5.449378in}{0.736573in}}%
\pgfpathlineto{\pgfqpoint{5.384848in}{0.735510in}}%
\pgfpathlineto{\pgfqpoint{5.340340in}{0.735108in}}%
\pgfpathlineto{\pgfqpoint{5.340123in}{0.770153in}}%
\pgfpathlineto{\pgfqpoint{5.329070in}{0.772449in}}%
\pgfpathlineto{\pgfqpoint{5.321775in}{0.769451in}}%
\pgfpathlineto{\pgfqpoint{5.315948in}{0.774957in}}%
\pgfpathlineto{\pgfqpoint{5.316549in}{0.812079in}}%
\pgfpathlineto{\pgfqpoint{5.317853in}{0.891120in}}%
\pgfpathlineto{\pgfqpoint{5.311531in}{0.937404in}}%
\pgfpathlineto{\pgfqpoint{5.307462in}{0.962463in}}%
\pgfpathclose%
\pgfusepath{fill}%
\end{pgfscope}%
\begin{pgfscope}%
\pgfpathrectangle{\pgfqpoint{3.625000in}{0.100000in}}{\pgfqpoint{2.989028in}{1.913466in}}%
\pgfusepath{clip}%
\pgfsetbuttcap%
\pgfsetmiterjoin%
\definecolor{currentfill}{rgb}{0.738639,0.894348,0.629296}%
\pgfsetfillcolor{currentfill}%
\pgfsetlinewidth{0.000000pt}%
\definecolor{currentstroke}{rgb}{0.000000,0.000000,0.000000}%
\pgfsetstrokecolor{currentstroke}%
\pgfsetstrokeopacity{0.000000}%
\pgfsetdash{}{0pt}%
\pgfpathmoveto{\pgfqpoint{5.340340in}{0.735108in}}%
\pgfpathlineto{\pgfqpoint{5.384848in}{0.735510in}}%
\pgfpathlineto{\pgfqpoint{5.449378in}{0.736573in}}%
\pgfpathlineto{\pgfqpoint{5.499085in}{0.738625in}}%
\pgfpathlineto{\pgfqpoint{5.496703in}{0.733439in}}%
\pgfpathlineto{\pgfqpoint{5.502983in}{0.704877in}}%
\pgfpathlineto{\pgfqpoint{5.511660in}{0.692400in}}%
\pgfpathlineto{\pgfqpoint{5.501601in}{0.685508in}}%
\pgfpathlineto{\pgfqpoint{5.506859in}{0.673535in}}%
\pgfpathlineto{\pgfqpoint{5.492673in}{0.661207in}}%
\pgfpathlineto{\pgfqpoint{5.488432in}{0.642495in}}%
\pgfpathlineto{\pgfqpoint{5.480681in}{0.632856in}}%
\pgfpathlineto{\pgfqpoint{5.481668in}{0.622916in}}%
\pgfpathlineto{\pgfqpoint{5.477221in}{0.620808in}}%
\pgfpathlineto{\pgfqpoint{5.481684in}{0.610463in}}%
\pgfpathlineto{\pgfqpoint{5.478073in}{0.604986in}}%
\pgfpathlineto{\pgfqpoint{5.538761in}{0.607617in}}%
\pgfpathlineto{\pgfqpoint{5.585562in}{0.610498in}}%
\pgfpathlineto{\pgfqpoint{5.580622in}{0.588493in}}%
\pgfpathlineto{\pgfqpoint{5.583992in}{0.580363in}}%
\pgfpathlineto{\pgfqpoint{5.590977in}{0.573569in}}%
\pgfpathlineto{\pgfqpoint{5.597753in}{0.557360in}}%
\pgfpathlineto{\pgfqpoint{5.576350in}{0.561094in}}%
\pgfpathlineto{\pgfqpoint{5.568455in}{0.567234in}}%
\pgfpathlineto{\pgfqpoint{5.559048in}{0.567526in}}%
\pgfpathlineto{\pgfqpoint{5.549158in}{0.554082in}}%
\pgfpathlineto{\pgfqpoint{5.551129in}{0.547960in}}%
\pgfpathlineto{\pgfqpoint{5.582724in}{0.544257in}}%
\pgfpathlineto{\pgfqpoint{5.590995in}{0.537220in}}%
\pgfpathlineto{\pgfqpoint{5.604454in}{0.533600in}}%
\pgfpathlineto{\pgfqpoint{5.598576in}{0.525268in}}%
\pgfpathlineto{\pgfqpoint{5.588672in}{0.518003in}}%
\pgfpathlineto{\pgfqpoint{5.614813in}{0.501664in}}%
\pgfpathlineto{\pgfqpoint{5.626990in}{0.499111in}}%
\pgfpathlineto{\pgfqpoint{5.632054in}{0.485755in}}%
\pgfpathlineto{\pgfqpoint{5.620589in}{0.483298in}}%
\pgfpathlineto{\pgfqpoint{5.599041in}{0.500096in}}%
\pgfpathlineto{\pgfqpoint{5.590616in}{0.502430in}}%
\pgfpathlineto{\pgfqpoint{5.586539in}{0.509066in}}%
\pgfpathlineto{\pgfqpoint{5.574043in}{0.506343in}}%
\pgfpathlineto{\pgfqpoint{5.570130in}{0.497792in}}%
\pgfpathlineto{\pgfqpoint{5.572608in}{0.488264in}}%
\pgfpathlineto{\pgfqpoint{5.564195in}{0.482642in}}%
\pgfpathlineto{\pgfqpoint{5.556517in}{0.496493in}}%
\pgfpathlineto{\pgfqpoint{5.543069in}{0.492363in}}%
\pgfpathlineto{\pgfqpoint{5.538030in}{0.484098in}}%
\pgfpathlineto{\pgfqpoint{5.528479in}{0.486449in}}%
\pgfpathlineto{\pgfqpoint{5.522373in}{0.496886in}}%
\pgfpathlineto{\pgfqpoint{5.506173in}{0.500333in}}%
\pgfpathlineto{\pgfqpoint{5.503115in}{0.505764in}}%
\pgfpathlineto{\pgfqpoint{5.486236in}{0.515225in}}%
\pgfpathlineto{\pgfqpoint{5.482047in}{0.523496in}}%
\pgfpathlineto{\pgfqpoint{5.467910in}{0.520116in}}%
\pgfpathlineto{\pgfqpoint{5.469705in}{0.527689in}}%
\pgfpathlineto{\pgfqpoint{5.452134in}{0.519920in}}%
\pgfpathlineto{\pgfqpoint{5.456854in}{0.511860in}}%
\pgfpathlineto{\pgfqpoint{5.443284in}{0.507095in}}%
\pgfpathlineto{\pgfqpoint{5.425311in}{0.509714in}}%
\pgfpathlineto{\pgfqpoint{5.388934in}{0.522174in}}%
\pgfpathlineto{\pgfqpoint{5.360864in}{0.519701in}}%
\pgfpathlineto{\pgfqpoint{5.358432in}{0.536068in}}%
\pgfpathlineto{\pgfqpoint{5.361956in}{0.543537in}}%
\pgfpathlineto{\pgfqpoint{5.361655in}{0.555420in}}%
\pgfpathlineto{\pgfqpoint{5.358187in}{0.558251in}}%
\pgfpathlineto{\pgfqpoint{5.359363in}{0.571815in}}%
\pgfpathlineto{\pgfqpoint{5.369586in}{0.590680in}}%
\pgfpathlineto{\pgfqpoint{5.370894in}{0.597808in}}%
\pgfpathlineto{\pgfqpoint{5.369209in}{0.614644in}}%
\pgfpathlineto{\pgfqpoint{5.361688in}{0.622574in}}%
\pgfpathlineto{\pgfqpoint{5.357761in}{0.637381in}}%
\pgfpathlineto{\pgfqpoint{5.352929in}{0.640781in}}%
\pgfpathlineto{\pgfqpoint{5.355092in}{0.648811in}}%
\pgfpathlineto{\pgfqpoint{5.348939in}{0.660813in}}%
\pgfpathlineto{\pgfqpoint{5.341262in}{0.667326in}}%
\pgfpathlineto{\pgfqpoint{5.340340in}{0.735108in}}%
\pgfpathclose%
\pgfusepath{fill}%
\end{pgfscope}%
\begin{pgfscope}%
\pgfpathrectangle{\pgfqpoint{3.625000in}{0.100000in}}{\pgfqpoint{2.989028in}{1.913466in}}%
\pgfusepath{clip}%
\pgfsetbuttcap%
\pgfsetmiterjoin%
\definecolor{currentfill}{rgb}{0.738639,0.894348,0.629296}%
\pgfsetfillcolor{currentfill}%
\pgfsetlinewidth{0.000000pt}%
\definecolor{currentstroke}{rgb}{0.000000,0.000000,0.000000}%
\pgfsetstrokecolor{currentstroke}%
\pgfsetstrokeopacity{0.000000}%
\pgfsetdash{}{0pt}%
\pgfpathmoveto{\pgfqpoint{5.459378in}{0.511243in}}%
\pgfpathlineto{\pgfqpoint{5.465817in}{0.515073in}}%
\pgfpathlineto{\pgfqpoint{5.473633in}{0.510545in}}%
\pgfpathlineto{\pgfqpoint{5.469274in}{0.504314in}}%
\pgfpathlineto{\pgfqpoint{5.459378in}{0.511243in}}%
\pgfpathclose%
\pgfusepath{fill}%
\end{pgfscope}%
\begin{pgfscope}%
\pgfpathrectangle{\pgfqpoint{3.625000in}{0.100000in}}{\pgfqpoint{2.989028in}{1.913466in}}%
\pgfusepath{clip}%
\pgfsetbuttcap%
\pgfsetmiterjoin%
\definecolor{currentfill}{rgb}{0.368012,0.725106,0.661822}%
\pgfsetfillcolor{currentfill}%
\pgfsetlinewidth{0.000000pt}%
\definecolor{currentstroke}{rgb}{0.000000,0.000000,0.000000}%
\pgfsetstrokecolor{currentstroke}%
\pgfsetstrokeopacity{0.000000}%
\pgfsetdash{}{0pt}%
\pgfpathmoveto{\pgfqpoint{5.719230in}{0.585103in}}%
\pgfpathlineto{\pgfqpoint{5.719346in}{0.597485in}}%
\pgfpathlineto{\pgfqpoint{5.711652in}{0.602188in}}%
\pgfpathlineto{\pgfqpoint{5.705349in}{0.610192in}}%
\pgfpathlineto{\pgfqpoint{5.706207in}{0.618602in}}%
\pgfpathlineto{\pgfqpoint{5.774834in}{0.623930in}}%
\pgfpathlineto{\pgfqpoint{5.852687in}{0.632577in}}%
\pgfpathlineto{\pgfqpoint{5.862617in}{0.614480in}}%
\pgfpathlineto{\pgfqpoint{5.895635in}{0.616804in}}%
\pgfpathlineto{\pgfqpoint{5.961103in}{0.620376in}}%
\pgfpathlineto{\pgfqpoint{6.013024in}{0.623836in}}%
\pgfpathlineto{\pgfqpoint{6.018123in}{0.610802in}}%
\pgfpathlineto{\pgfqpoint{6.024427in}{0.611870in}}%
\pgfpathlineto{\pgfqpoint{6.025128in}{0.625826in}}%
\pgfpathlineto{\pgfqpoint{6.021902in}{0.637830in}}%
\pgfpathlineto{\pgfqpoint{6.026225in}{0.642912in}}%
\pgfpathlineto{\pgfqpoint{6.036217in}{0.640178in}}%
\pgfpathlineto{\pgfqpoint{6.050559in}{0.639617in}}%
\pgfpathlineto{\pgfqpoint{6.052764in}{0.628892in}}%
\pgfpathlineto{\pgfqpoint{6.057176in}{0.622764in}}%
\pgfpathlineto{\pgfqpoint{6.060621in}{0.609356in}}%
\pgfpathlineto{\pgfqpoint{6.071296in}{0.588607in}}%
\pgfpathlineto{\pgfqpoint{6.071349in}{0.582995in}}%
\pgfpathlineto{\pgfqpoint{6.087126in}{0.558631in}}%
\pgfpathlineto{\pgfqpoint{6.088652in}{0.553598in}}%
\pgfpathlineto{\pgfqpoint{6.112320in}{0.519188in}}%
\pgfpathlineto{\pgfqpoint{6.108570in}{0.518632in}}%
\pgfpathlineto{\pgfqpoint{6.118548in}{0.494158in}}%
\pgfpathlineto{\pgfqpoint{6.145959in}{0.451609in}}%
\pgfpathlineto{\pgfqpoint{6.161873in}{0.420325in}}%
\pgfpathlineto{\pgfqpoint{6.167192in}{0.414848in}}%
\pgfpathlineto{\pgfqpoint{6.176306in}{0.395211in}}%
\pgfpathlineto{\pgfqpoint{6.179468in}{0.363769in}}%
\pgfpathlineto{\pgfqpoint{6.180727in}{0.340258in}}%
\pgfpathlineto{\pgfqpoint{6.179207in}{0.325205in}}%
\pgfpathlineto{\pgfqpoint{6.174320in}{0.314444in}}%
\pgfpathlineto{\pgfqpoint{6.176590in}{0.300332in}}%
\pgfpathlineto{\pgfqpoint{6.171329in}{0.289163in}}%
\pgfpathlineto{\pgfqpoint{6.155707in}{0.280003in}}%
\pgfpathlineto{\pgfqpoint{6.145547in}{0.280672in}}%
\pgfpathlineto{\pgfqpoint{6.138949in}{0.275884in}}%
\pgfpathlineto{\pgfqpoint{6.137059in}{0.288668in}}%
\pgfpathlineto{\pgfqpoint{6.126137in}{0.292042in}}%
\pgfpathlineto{\pgfqpoint{6.116346in}{0.309880in}}%
\pgfpathlineto{\pgfqpoint{6.115241in}{0.318003in}}%
\pgfpathlineto{\pgfqpoint{6.097719in}{0.322936in}}%
\pgfpathlineto{\pgfqpoint{6.086428in}{0.321868in}}%
\pgfpathlineto{\pgfqpoint{6.080040in}{0.333654in}}%
\pgfpathlineto{\pgfqpoint{6.072725in}{0.354839in}}%
\pgfpathlineto{\pgfqpoint{6.062575in}{0.359133in}}%
\pgfpathlineto{\pgfqpoint{6.057072in}{0.371273in}}%
\pgfpathlineto{\pgfqpoint{6.043614in}{0.378428in}}%
\pgfpathlineto{\pgfqpoint{6.035835in}{0.387377in}}%
\pgfpathlineto{\pgfqpoint{6.022416in}{0.411708in}}%
\pgfpathlineto{\pgfqpoint{6.020838in}{0.428833in}}%
\pgfpathlineto{\pgfqpoint{6.028150in}{0.439789in}}%
\pgfpathlineto{\pgfqpoint{6.027419in}{0.447403in}}%
\pgfpathlineto{\pgfqpoint{6.018969in}{0.448217in}}%
\pgfpathlineto{\pgfqpoint{6.011909in}{0.453513in}}%
\pgfpathlineto{\pgfqpoint{6.008042in}{0.447046in}}%
\pgfpathlineto{\pgfqpoint{6.014818in}{0.441777in}}%
\pgfpathlineto{\pgfqpoint{6.009887in}{0.432302in}}%
\pgfpathlineto{\pgfqpoint{6.001908in}{0.440166in}}%
\pgfpathlineto{\pgfqpoint{6.002834in}{0.461974in}}%
\pgfpathlineto{\pgfqpoint{6.006682in}{0.479667in}}%
\pgfpathlineto{\pgfqpoint{6.004704in}{0.510026in}}%
\pgfpathlineto{\pgfqpoint{5.996751in}{0.517257in}}%
\pgfpathlineto{\pgfqpoint{5.992752in}{0.526536in}}%
\pgfpathlineto{\pgfqpoint{5.979052in}{0.526329in}}%
\pgfpathlineto{\pgfqpoint{5.965510in}{0.541597in}}%
\pgfpathlineto{\pgfqpoint{5.956430in}{0.546173in}}%
\pgfpathlineto{\pgfqpoint{5.953737in}{0.555842in}}%
\pgfpathlineto{\pgfqpoint{5.944856in}{0.559238in}}%
\pgfpathlineto{\pgfqpoint{5.937496in}{0.569913in}}%
\pgfpathlineto{\pgfqpoint{5.918052in}{0.578628in}}%
\pgfpathlineto{\pgfqpoint{5.902937in}{0.578850in}}%
\pgfpathlineto{\pgfqpoint{5.896356in}{0.575503in}}%
\pgfpathlineto{\pgfqpoint{5.897991in}{0.565056in}}%
\pgfpathlineto{\pgfqpoint{5.891130in}{0.565551in}}%
\pgfpathlineto{\pgfqpoint{5.870022in}{0.550943in}}%
\pgfpathlineto{\pgfqpoint{5.844668in}{0.545153in}}%
\pgfpathlineto{\pgfqpoint{5.844252in}{0.552317in}}%
\pgfpathlineto{\pgfqpoint{5.838633in}{0.559318in}}%
\pgfpathlineto{\pgfqpoint{5.823558in}{0.568984in}}%
\pgfpathlineto{\pgfqpoint{5.801918in}{0.578894in}}%
\pgfpathlineto{\pgfqpoint{5.778414in}{0.584141in}}%
\pgfpathlineto{\pgfqpoint{5.774002in}{0.591275in}}%
\pgfpathlineto{\pgfqpoint{5.765528in}{0.585325in}}%
\pgfpathlineto{\pgfqpoint{5.755323in}{0.584012in}}%
\pgfpathlineto{\pgfqpoint{5.719801in}{0.574653in}}%
\pgfpathlineto{\pgfqpoint{5.719230in}{0.585103in}}%
\pgfpathclose%
\pgfusepath{fill}%
\end{pgfscope}%
\begin{pgfscope}%
\pgfpathrectangle{\pgfqpoint{3.625000in}{0.100000in}}{\pgfqpoint{2.989028in}{1.913466in}}%
\pgfusepath{clip}%
\pgfsetbuttcap%
\pgfsetmiterjoin%
\definecolor{currentfill}{rgb}{0.368012,0.725106,0.661822}%
\pgfsetfillcolor{currentfill}%
\pgfsetlinewidth{0.000000pt}%
\definecolor{currentstroke}{rgb}{0.000000,0.000000,0.000000}%
\pgfsetstrokecolor{currentstroke}%
\pgfsetstrokeopacity{0.000000}%
\pgfsetdash{}{0pt}%
\pgfpathmoveto{\pgfqpoint{6.115769in}{0.519678in}}%
\pgfpathlineto{\pgfqpoint{6.124959in}{0.508987in}}%
\pgfpathlineto{\pgfqpoint{6.114344in}{0.508314in}}%
\pgfpathlineto{\pgfqpoint{6.115769in}{0.519678in}}%
\pgfpathclose%
\pgfusepath{fill}%
\end{pgfscope}%
\begin{pgfscope}%
\pgfpathrectangle{\pgfqpoint{3.625000in}{0.100000in}}{\pgfqpoint{2.989028in}{1.913466in}}%
\pgfusepath{clip}%
\pgfsetbuttcap%
\pgfsetmiterjoin%
\definecolor{currentfill}{rgb}{0.932718,0.973087,0.644060}%
\pgfsetfillcolor{currentfill}%
\pgfsetlinewidth{0.000000pt}%
\definecolor{currentstroke}{rgb}{0.000000,0.000000,0.000000}%
\pgfsetstrokecolor{currentstroke}%
\pgfsetstrokeopacity{0.000000}%
\pgfsetdash{}{0pt}%
\pgfpathmoveto{\pgfqpoint{5.574740in}{1.740361in}}%
\pgfpathlineto{\pgfqpoint{5.569753in}{1.730656in}}%
\pgfpathlineto{\pgfqpoint{5.557857in}{1.724992in}}%
\pgfpathlineto{\pgfqpoint{5.552705in}{1.717368in}}%
\pgfpathlineto{\pgfqpoint{5.546669in}{1.722859in}}%
\pgfpathlineto{\pgfqpoint{5.574740in}{1.740361in}}%
\pgfpathclose%
\pgfusepath{fill}%
\end{pgfscope}%
\begin{pgfscope}%
\pgfpathrectangle{\pgfqpoint{3.625000in}{0.100000in}}{\pgfqpoint{2.989028in}{1.913466in}}%
\pgfusepath{clip}%
\pgfsetbuttcap%
\pgfsetmiterjoin%
\definecolor{currentfill}{rgb}{0.932718,0.973087,0.644060}%
\pgfsetfillcolor{currentfill}%
\pgfsetlinewidth{0.000000pt}%
\definecolor{currentstroke}{rgb}{0.000000,0.000000,0.000000}%
\pgfsetstrokecolor{currentstroke}%
\pgfsetstrokeopacity{0.000000}%
\pgfsetdash{}{0pt}%
\pgfpathmoveto{\pgfqpoint{5.578784in}{1.681953in}}%
\pgfpathlineto{\pgfqpoint{5.590956in}{1.693308in}}%
\pgfpathlineto{\pgfqpoint{5.609730in}{1.696281in}}%
\pgfpathlineto{\pgfqpoint{5.604559in}{1.688384in}}%
\pgfpathlineto{\pgfqpoint{5.591688in}{1.676964in}}%
\pgfpathlineto{\pgfqpoint{5.584164in}{1.662306in}}%
\pgfpathlineto{\pgfqpoint{5.580977in}{1.670256in}}%
\pgfpathlineto{\pgfqpoint{5.575318in}{1.671398in}}%
\pgfpathlineto{\pgfqpoint{5.574750in}{1.678576in}}%
\pgfpathlineto{\pgfqpoint{5.578784in}{1.681953in}}%
\pgfpathclose%
\pgfusepath{fill}%
\end{pgfscope}%
\begin{pgfscope}%
\pgfpathrectangle{\pgfqpoint{3.625000in}{0.100000in}}{\pgfqpoint{2.989028in}{1.913466in}}%
\pgfusepath{clip}%
\pgfsetbuttcap%
\pgfsetmiterjoin%
\definecolor{currentfill}{rgb}{0.932718,0.973087,0.644060}%
\pgfsetfillcolor{currentfill}%
\pgfsetlinewidth{0.000000pt}%
\definecolor{currentstroke}{rgb}{0.000000,0.000000,0.000000}%
\pgfsetstrokecolor{currentstroke}%
\pgfsetstrokeopacity{0.000000}%
\pgfsetdash{}{0pt}%
\pgfpathmoveto{\pgfqpoint{5.627264in}{1.543360in}}%
\pgfpathlineto{\pgfqpoint{5.624022in}{1.546957in}}%
\pgfpathlineto{\pgfqpoint{5.627430in}{1.557084in}}%
\pgfpathlineto{\pgfqpoint{5.617314in}{1.557730in}}%
\pgfpathlineto{\pgfqpoint{5.619996in}{1.566462in}}%
\pgfpathlineto{\pgfqpoint{5.618328in}{1.580368in}}%
\pgfpathlineto{\pgfqpoint{5.609251in}{1.585189in}}%
\pgfpathlineto{\pgfqpoint{5.599752in}{1.594947in}}%
\pgfpathlineto{\pgfqpoint{5.585116in}{1.597803in}}%
\pgfpathlineto{\pgfqpoint{5.570871in}{1.597722in}}%
\pgfpathlineto{\pgfqpoint{5.556860in}{1.604646in}}%
\pgfpathlineto{\pgfqpoint{5.509989in}{1.614740in}}%
\pgfpathlineto{\pgfqpoint{5.504867in}{1.625436in}}%
\pgfpathlineto{\pgfqpoint{5.495731in}{1.629085in}}%
\pgfpathlineto{\pgfqpoint{5.512989in}{1.637274in}}%
\pgfpathlineto{\pgfqpoint{5.522738in}{1.647490in}}%
\pgfpathlineto{\pgfqpoint{5.540897in}{1.650260in}}%
\pgfpathlineto{\pgfqpoint{5.552071in}{1.660666in}}%
\pgfpathlineto{\pgfqpoint{5.557934in}{1.661083in}}%
\pgfpathlineto{\pgfqpoint{5.562408in}{1.668504in}}%
\pgfpathlineto{\pgfqpoint{5.573315in}{1.677295in}}%
\pgfpathlineto{\pgfqpoint{5.574265in}{1.671081in}}%
\pgfpathlineto{\pgfqpoint{5.579179in}{1.669798in}}%
\pgfpathlineto{\pgfqpoint{5.582877in}{1.662388in}}%
\pgfpathlineto{\pgfqpoint{5.583543in}{1.649753in}}%
\pgfpathlineto{\pgfqpoint{5.594656in}{1.657304in}}%
\pgfpathlineto{\pgfqpoint{5.607606in}{1.658880in}}%
\pgfpathlineto{\pgfqpoint{5.618660in}{1.654925in}}%
\pgfpathlineto{\pgfqpoint{5.633651in}{1.634360in}}%
\pgfpathlineto{\pgfqpoint{5.650023in}{1.637647in}}%
\pgfpathlineto{\pgfqpoint{5.656676in}{1.632143in}}%
\pgfpathlineto{\pgfqpoint{5.667376in}{1.631651in}}%
\pgfpathlineto{\pgfqpoint{5.674485in}{1.641522in}}%
\pgfpathlineto{\pgfqpoint{5.687974in}{1.650248in}}%
\pgfpathlineto{\pgfqpoint{5.700939in}{1.652980in}}%
\pgfpathlineto{\pgfqpoint{5.717026in}{1.653265in}}%
\pgfpathlineto{\pgfqpoint{5.728782in}{1.660005in}}%
\pgfpathlineto{\pgfqpoint{5.738376in}{1.656901in}}%
\pgfpathlineto{\pgfqpoint{5.740419in}{1.642645in}}%
\pgfpathlineto{\pgfqpoint{5.761002in}{1.640413in}}%
\pgfpathlineto{\pgfqpoint{5.772182in}{1.647086in}}%
\pgfpathlineto{\pgfqpoint{5.779932in}{1.632053in}}%
\pgfpathlineto{\pgfqpoint{5.794719in}{1.614670in}}%
\pgfpathlineto{\pgfqpoint{5.774022in}{1.614771in}}%
\pgfpathlineto{\pgfqpoint{5.767482in}{1.612622in}}%
\pgfpathlineto{\pgfqpoint{5.758519in}{1.615414in}}%
\pgfpathlineto{\pgfqpoint{5.757916in}{1.603380in}}%
\pgfpathlineto{\pgfqpoint{5.741640in}{1.612785in}}%
\pgfpathlineto{\pgfqpoint{5.720708in}{1.615659in}}%
\pgfpathlineto{\pgfqpoint{5.714946in}{1.606500in}}%
\pgfpathlineto{\pgfqpoint{5.687559in}{1.602046in}}%
\pgfpathlineto{\pgfqpoint{5.684409in}{1.594288in}}%
\pgfpathlineto{\pgfqpoint{5.665351in}{1.591960in}}%
\pgfpathlineto{\pgfqpoint{5.659590in}{1.584073in}}%
\pgfpathlineto{\pgfqpoint{5.649509in}{1.581958in}}%
\pgfpathlineto{\pgfqpoint{5.641458in}{1.563256in}}%
\pgfpathlineto{\pgfqpoint{5.631264in}{1.545142in}}%
\pgfpathlineto{\pgfqpoint{5.627264in}{1.543360in}}%
\pgfpathclose%
\pgfusepath{fill}%
\end{pgfscope}%
\begin{pgfscope}%
\pgfpathrectangle{\pgfqpoint{3.625000in}{0.100000in}}{\pgfqpoint{2.989028in}{1.913466in}}%
\pgfusepath{clip}%
\pgfsetbuttcap%
\pgfsetmiterjoin%
\definecolor{currentfill}{rgb}{0.932718,0.973087,0.644060}%
\pgfsetfillcolor{currentfill}%
\pgfsetlinewidth{0.000000pt}%
\definecolor{currentstroke}{rgb}{0.000000,0.000000,0.000000}%
\pgfsetstrokecolor{currentstroke}%
\pgfsetstrokeopacity{0.000000}%
\pgfsetdash{}{0pt}%
\pgfpathmoveto{\pgfqpoint{5.685936in}{1.325898in}}%
\pgfpathlineto{\pgfqpoint{5.695655in}{1.336128in}}%
\pgfpathlineto{\pgfqpoint{5.700083in}{1.350960in}}%
\pgfpathlineto{\pgfqpoint{5.705349in}{1.359557in}}%
\pgfpathlineto{\pgfqpoint{5.708578in}{1.371256in}}%
\pgfpathlineto{\pgfqpoint{5.709581in}{1.394584in}}%
\pgfpathlineto{\pgfqpoint{5.704720in}{1.416940in}}%
\pgfpathlineto{\pgfqpoint{5.688654in}{1.451174in}}%
\pgfpathlineto{\pgfqpoint{5.693005in}{1.461938in}}%
\pgfpathlineto{\pgfqpoint{5.687337in}{1.476816in}}%
\pgfpathlineto{\pgfqpoint{5.697059in}{1.497392in}}%
\pgfpathlineto{\pgfqpoint{5.695475in}{1.520335in}}%
\pgfpathlineto{\pgfqpoint{5.702247in}{1.523223in}}%
\pgfpathlineto{\pgfqpoint{5.703098in}{1.534139in}}%
\pgfpathlineto{\pgfqpoint{5.715119in}{1.541146in}}%
\pgfpathlineto{\pgfqpoint{5.721918in}{1.540015in}}%
\pgfpathlineto{\pgfqpoint{5.723829in}{1.528307in}}%
\pgfpathlineto{\pgfqpoint{5.729135in}{1.527848in}}%
\pgfpathlineto{\pgfqpoint{5.733977in}{1.544735in}}%
\pgfpathlineto{\pgfqpoint{5.732331in}{1.557869in}}%
\pgfpathlineto{\pgfqpoint{5.735479in}{1.565411in}}%
\pgfpathlineto{\pgfqpoint{5.752498in}{1.573202in}}%
\pgfpathlineto{\pgfqpoint{5.744743in}{1.575975in}}%
\pgfpathlineto{\pgfqpoint{5.742216in}{1.582668in}}%
\pgfpathlineto{\pgfqpoint{5.747803in}{1.594421in}}%
\pgfpathlineto{\pgfqpoint{5.758821in}{1.598481in}}%
\pgfpathlineto{\pgfqpoint{5.771615in}{1.591546in}}%
\pgfpathlineto{\pgfqpoint{5.783677in}{1.591460in}}%
\pgfpathlineto{\pgfqpoint{5.789279in}{1.583359in}}%
\pgfpathlineto{\pgfqpoint{5.797727in}{1.583922in}}%
\pgfpathlineto{\pgfqpoint{5.823804in}{1.572661in}}%
\pgfpathlineto{\pgfqpoint{5.829014in}{1.561760in}}%
\pgfpathlineto{\pgfqpoint{5.823324in}{1.557985in}}%
\pgfpathlineto{\pgfqpoint{5.825077in}{1.549815in}}%
\pgfpathlineto{\pgfqpoint{5.830699in}{1.546182in}}%
\pgfpathlineto{\pgfqpoint{5.833831in}{1.536139in}}%
\pgfpathlineto{\pgfqpoint{5.833393in}{1.511672in}}%
\pgfpathlineto{\pgfqpoint{5.825971in}{1.505827in}}%
\pgfpathlineto{\pgfqpoint{5.824305in}{1.492975in}}%
\pgfpathlineto{\pgfqpoint{5.810538in}{1.481098in}}%
\pgfpathlineto{\pgfqpoint{5.811355in}{1.466722in}}%
\pgfpathlineto{\pgfqpoint{5.823421in}{1.461685in}}%
\pgfpathlineto{\pgfqpoint{5.837034in}{1.479623in}}%
\pgfpathlineto{\pgfqpoint{5.838159in}{1.486129in}}%
\pgfpathlineto{\pgfqpoint{5.855136in}{1.496942in}}%
\pgfpathlineto{\pgfqpoint{5.865932in}{1.491933in}}%
\pgfpathlineto{\pgfqpoint{5.872707in}{1.480613in}}%
\pgfpathlineto{\pgfqpoint{5.883636in}{1.441310in}}%
\pgfpathlineto{\pgfqpoint{5.889428in}{1.428876in}}%
\pgfpathlineto{\pgfqpoint{5.887615in}{1.423735in}}%
\pgfpathlineto{\pgfqpoint{5.887791in}{1.406232in}}%
\pgfpathlineto{\pgfqpoint{5.877255in}{1.407912in}}%
\pgfpathlineto{\pgfqpoint{5.871305in}{1.394832in}}%
\pgfpathlineto{\pgfqpoint{5.870480in}{1.385940in}}%
\pgfpathlineto{\pgfqpoint{5.862522in}{1.380232in}}%
\pgfpathlineto{\pgfqpoint{5.860753in}{1.362891in}}%
\pgfpathlineto{\pgfqpoint{5.849180in}{1.340984in}}%
\pgfpathlineto{\pgfqpoint{5.785872in}{1.331556in}}%
\pgfpathlineto{\pgfqpoint{5.785498in}{1.335699in}}%
\pgfpathlineto{\pgfqpoint{5.743150in}{1.331236in}}%
\pgfpathlineto{\pgfqpoint{5.685936in}{1.325898in}}%
\pgfpathclose%
\pgfusepath{fill}%
\end{pgfscope}%
\begin{pgfscope}%
\pgfpathrectangle{\pgfqpoint{3.625000in}{0.100000in}}{\pgfqpoint{2.989028in}{1.913466in}}%
\pgfusepath{clip}%
\pgfsetbuttcap%
\pgfsetmiterjoin%
\definecolor{currentfill}{rgb}{0.312034,0.662668,0.687659}%
\pgfsetfillcolor{currentfill}%
\pgfsetlinewidth{0.000000pt}%
\definecolor{currentstroke}{rgb}{0.000000,0.000000,0.000000}%
\pgfsetstrokecolor{currentstroke}%
\pgfsetstrokeopacity{0.000000}%
\pgfsetdash{}{0pt}%
\pgfpathmoveto{\pgfqpoint{4.087031in}{0.446426in}}%
\pgfpathlineto{\pgfqpoint{4.085767in}{0.445314in}}%
\pgfpathlineto{\pgfqpoint{4.079967in}{0.446797in}}%
\pgfpathlineto{\pgfqpoint{4.080400in}{0.449716in}}%
\pgfpathlineto{\pgfqpoint{4.083683in}{0.451053in}}%
\pgfpathlineto{\pgfqpoint{4.088364in}{0.457782in}}%
\pgfpathlineto{\pgfqpoint{4.089142in}{0.462330in}}%
\pgfpathlineto{\pgfqpoint{4.091773in}{0.463522in}}%
\pgfpathlineto{\pgfqpoint{4.096054in}{0.463526in}}%
\pgfpathlineto{\pgfqpoint{4.100530in}{0.478285in}}%
\pgfpathlineto{\pgfqpoint{4.102283in}{0.479499in}}%
\pgfpathlineto{\pgfqpoint{4.100711in}{0.482384in}}%
\pgfpathlineto{\pgfqpoint{4.097030in}{0.479466in}}%
\pgfpathlineto{\pgfqpoint{4.086215in}{0.483400in}}%
\pgfpathlineto{\pgfqpoint{4.079760in}{0.490029in}}%
\pgfpathlineto{\pgfqpoint{4.082575in}{0.493053in}}%
\pgfpathlineto{\pgfqpoint{4.081295in}{0.497319in}}%
\pgfpathlineto{\pgfqpoint{4.082083in}{0.502359in}}%
\pgfpathlineto{\pgfqpoint{4.080695in}{0.508250in}}%
\pgfpathlineto{\pgfqpoint{4.080473in}{0.514148in}}%
\pgfpathlineto{\pgfqpoint{4.086686in}{0.513161in}}%
\pgfpathlineto{\pgfqpoint{4.087916in}{0.514899in}}%
\pgfpathlineto{\pgfqpoint{4.091006in}{0.514731in}}%
\pgfpathlineto{\pgfqpoint{4.094565in}{0.508441in}}%
\pgfpathlineto{\pgfqpoint{4.097996in}{0.503835in}}%
\pgfpathlineto{\pgfqpoint{4.095248in}{0.500837in}}%
\pgfpathlineto{\pgfqpoint{4.100000in}{0.499687in}}%
\pgfpathlineto{\pgfqpoint{4.101083in}{0.501158in}}%
\pgfpathlineto{\pgfqpoint{4.098542in}{0.507123in}}%
\pgfpathlineto{\pgfqpoint{4.093190in}{0.512068in}}%
\pgfpathlineto{\pgfqpoint{4.093845in}{0.514140in}}%
\pgfpathlineto{\pgfqpoint{4.089574in}{0.519376in}}%
\pgfpathlineto{\pgfqpoint{4.093935in}{0.520439in}}%
\pgfpathlineto{\pgfqpoint{4.090157in}{0.531650in}}%
\pgfpathlineto{\pgfqpoint{4.093229in}{0.533204in}}%
\pgfpathlineto{\pgfqpoint{4.093978in}{0.535892in}}%
\pgfpathlineto{\pgfqpoint{4.091980in}{0.539085in}}%
\pgfpathlineto{\pgfqpoint{4.096104in}{0.539884in}}%
\pgfpathlineto{\pgfqpoint{4.096845in}{0.542299in}}%
\pgfpathlineto{\pgfqpoint{4.101765in}{0.538674in}}%
\pgfpathlineto{\pgfqpoint{4.102857in}{0.542605in}}%
\pgfpathlineto{\pgfqpoint{4.105443in}{0.544334in}}%
\pgfpathlineto{\pgfqpoint{4.118023in}{0.546739in}}%
\pgfpathlineto{\pgfqpoint{4.121885in}{0.545889in}}%
\pgfpathlineto{\pgfqpoint{4.125120in}{0.550663in}}%
\pgfpathlineto{\pgfqpoint{4.132797in}{0.553794in}}%
\pgfpathlineto{\pgfqpoint{4.138710in}{0.552606in}}%
\pgfpathlineto{\pgfqpoint{4.141407in}{0.548449in}}%
\pgfpathlineto{\pgfqpoint{4.141266in}{0.541844in}}%
\pgfpathlineto{\pgfqpoint{4.144292in}{0.540347in}}%
\pgfpathlineto{\pgfqpoint{4.150464in}{0.540556in}}%
\pgfpathlineto{\pgfqpoint{4.158370in}{0.542274in}}%
\pgfpathlineto{\pgfqpoint{4.158272in}{0.540162in}}%
\pgfpathlineto{\pgfqpoint{4.168266in}{0.534139in}}%
\pgfpathlineto{\pgfqpoint{4.175255in}{0.535754in}}%
\pgfpathlineto{\pgfqpoint{4.177313in}{0.537844in}}%
\pgfpathlineto{\pgfqpoint{4.179516in}{0.543197in}}%
\pgfpathlineto{\pgfqpoint{4.182248in}{0.546625in}}%
\pgfpathlineto{\pgfqpoint{4.182492in}{0.551762in}}%
\pgfpathlineto{\pgfqpoint{4.179990in}{0.553782in}}%
\pgfpathlineto{\pgfqpoint{4.183695in}{0.555608in}}%
\pgfpathlineto{\pgfqpoint{4.189920in}{0.552865in}}%
\pgfpathlineto{\pgfqpoint{4.191948in}{0.554559in}}%
\pgfpathlineto{\pgfqpoint{4.192305in}{0.561826in}}%
\pgfpathlineto{\pgfqpoint{4.188281in}{0.560307in}}%
\pgfpathlineto{\pgfqpoint{4.185490in}{0.562825in}}%
\pgfpathlineto{\pgfqpoint{4.180960in}{0.564395in}}%
\pgfpathlineto{\pgfqpoint{4.173770in}{0.564133in}}%
\pgfpathlineto{\pgfqpoint{4.166880in}{0.566521in}}%
\pgfpathlineto{\pgfqpoint{4.166535in}{0.572469in}}%
\pgfpathlineto{\pgfqpoint{4.160373in}{0.578787in}}%
\pgfpathlineto{\pgfqpoint{4.152559in}{0.581439in}}%
\pgfpathlineto{\pgfqpoint{4.145739in}{0.593671in}}%
\pgfpathlineto{\pgfqpoint{4.146298in}{0.598197in}}%
\pgfpathlineto{\pgfqpoint{4.149971in}{0.600261in}}%
\pgfpathlineto{\pgfqpoint{4.149430in}{0.604545in}}%
\pgfpathlineto{\pgfqpoint{4.150282in}{0.608808in}}%
\pgfpathlineto{\pgfqpoint{4.153188in}{0.605890in}}%
\pgfpathlineto{\pgfqpoint{4.157025in}{0.607107in}}%
\pgfpathlineto{\pgfqpoint{4.156833in}{0.610810in}}%
\pgfpathlineto{\pgfqpoint{4.151514in}{0.618243in}}%
\pgfpathlineto{\pgfqpoint{4.149956in}{0.625911in}}%
\pgfpathlineto{\pgfqpoint{4.154409in}{0.626496in}}%
\pgfpathlineto{\pgfqpoint{4.161074in}{0.626025in}}%
\pgfpathlineto{\pgfqpoint{4.163172in}{0.623460in}}%
\pgfpathlineto{\pgfqpoint{4.169147in}{0.624378in}}%
\pgfpathlineto{\pgfqpoint{4.174700in}{0.623173in}}%
\pgfpathlineto{\pgfqpoint{4.178454in}{0.621055in}}%
\pgfpathlineto{\pgfqpoint{4.181052in}{0.622023in}}%
\pgfpathlineto{\pgfqpoint{4.187951in}{0.619906in}}%
\pgfpathlineto{\pgfqpoint{4.188014in}{0.618032in}}%
\pgfpathlineto{\pgfqpoint{4.193268in}{0.618665in}}%
\pgfpathlineto{\pgfqpoint{4.200313in}{0.614092in}}%
\pgfpathlineto{\pgfqpoint{4.200039in}{0.611057in}}%
\pgfpathlineto{\pgfqpoint{4.196687in}{0.609656in}}%
\pgfpathlineto{\pgfqpoint{4.193124in}{0.606323in}}%
\pgfpathlineto{\pgfqpoint{4.192694in}{0.601746in}}%
\pgfpathlineto{\pgfqpoint{4.200369in}{0.595909in}}%
\pgfpathlineto{\pgfqpoint{4.199141in}{0.593112in}}%
\pgfpathlineto{\pgfqpoint{4.204602in}{0.590964in}}%
\pgfpathlineto{\pgfqpoint{4.205943in}{0.587455in}}%
\pgfpathlineto{\pgfqpoint{4.212873in}{0.590243in}}%
\pgfpathlineto{\pgfqpoint{4.215925in}{0.586589in}}%
\pgfpathlineto{\pgfqpoint{4.217334in}{0.588801in}}%
\pgfpathlineto{\pgfqpoint{4.213186in}{0.595999in}}%
\pgfpathlineto{\pgfqpoint{4.214660in}{0.598155in}}%
\pgfpathlineto{\pgfqpoint{4.215313in}{0.603841in}}%
\pgfpathlineto{\pgfqpoint{4.213661in}{0.606342in}}%
\pgfpathlineto{\pgfqpoint{4.214901in}{0.609704in}}%
\pgfpathlineto{\pgfqpoint{4.218961in}{0.609388in}}%
\pgfpathlineto{\pgfqpoint{4.218121in}{0.603668in}}%
\pgfpathlineto{\pgfqpoint{4.215529in}{0.601760in}}%
\pgfpathlineto{\pgfqpoint{4.215712in}{0.594065in}}%
\pgfpathlineto{\pgfqpoint{4.220364in}{0.593146in}}%
\pgfpathlineto{\pgfqpoint{4.220421in}{0.587424in}}%
\pgfpathlineto{\pgfqpoint{4.225501in}{0.583713in}}%
\pgfpathlineto{\pgfqpoint{4.227836in}{0.586053in}}%
\pgfpathlineto{\pgfqpoint{4.225284in}{0.593176in}}%
\pgfpathlineto{\pgfqpoint{4.222644in}{0.594979in}}%
\pgfpathlineto{\pgfqpoint{4.220047in}{0.593793in}}%
\pgfpathlineto{\pgfqpoint{4.217763in}{0.595306in}}%
\pgfpathlineto{\pgfqpoint{4.218161in}{0.601889in}}%
\pgfpathlineto{\pgfqpoint{4.221974in}{0.604581in}}%
\pgfpathlineto{\pgfqpoint{4.225726in}{0.604333in}}%
\pgfpathlineto{\pgfqpoint{4.224172in}{0.608058in}}%
\pgfpathlineto{\pgfqpoint{4.218547in}{0.610911in}}%
\pgfpathlineto{\pgfqpoint{4.211098in}{0.622277in}}%
\pgfpathlineto{\pgfqpoint{4.214816in}{0.627728in}}%
\pgfpathlineto{\pgfqpoint{4.216885in}{0.632863in}}%
\pgfpathlineto{\pgfqpoint{4.217048in}{0.643174in}}%
\pgfpathlineto{\pgfqpoint{4.216251in}{0.651872in}}%
\pgfpathlineto{\pgfqpoint{4.213978in}{0.657584in}}%
\pgfpathlineto{\pgfqpoint{4.214431in}{0.663294in}}%
\pgfpathlineto{\pgfqpoint{4.220244in}{0.668247in}}%
\pgfpathlineto{\pgfqpoint{4.226026in}{0.674133in}}%
\pgfpathlineto{\pgfqpoint{4.240519in}{0.662108in}}%
\pgfpathlineto{\pgfqpoint{4.249995in}{0.661221in}}%
\pgfpathlineto{\pgfqpoint{4.257817in}{0.664216in}}%
\pgfpathlineto{\pgfqpoint{4.264683in}{0.668159in}}%
\pgfpathlineto{\pgfqpoint{4.266141in}{0.670038in}}%
\pgfpathlineto{\pgfqpoint{4.283090in}{0.674322in}}%
\pgfpathlineto{\pgfqpoint{4.284562in}{0.671184in}}%
\pgfpathlineto{\pgfqpoint{4.290321in}{0.668615in}}%
\pgfpathlineto{\pgfqpoint{4.301313in}{0.669403in}}%
\pgfpathlineto{\pgfqpoint{4.303320in}{0.668702in}}%
\pgfpathlineto{\pgfqpoint{4.308260in}{0.670576in}}%
\pgfpathlineto{\pgfqpoint{4.310246in}{0.667335in}}%
\pgfpathlineto{\pgfqpoint{4.314990in}{0.664374in}}%
\pgfpathlineto{\pgfqpoint{4.317363in}{0.664709in}}%
\pgfpathlineto{\pgfqpoint{4.321956in}{0.661151in}}%
\pgfpathlineto{\pgfqpoint{4.329717in}{0.661805in}}%
\pgfpathlineto{\pgfqpoint{4.337469in}{0.664685in}}%
\pgfpathlineto{\pgfqpoint{4.343800in}{0.655475in}}%
\pgfpathlineto{\pgfqpoint{4.343309in}{0.653408in}}%
\pgfpathlineto{\pgfqpoint{4.336323in}{0.651032in}}%
\pgfpathlineto{\pgfqpoint{4.338190in}{0.647440in}}%
\pgfpathlineto{\pgfqpoint{4.344621in}{0.651221in}}%
\pgfpathlineto{\pgfqpoint{4.347246in}{0.650305in}}%
\pgfpathlineto{\pgfqpoint{4.348363in}{0.645783in}}%
\pgfpathlineto{\pgfqpoint{4.344524in}{0.643870in}}%
\pgfpathlineto{\pgfqpoint{4.347266in}{0.638154in}}%
\pgfpathlineto{\pgfqpoint{4.351248in}{0.639120in}}%
\pgfpathlineto{\pgfqpoint{4.357026in}{0.636170in}}%
\pgfpathlineto{\pgfqpoint{4.362551in}{0.627982in}}%
\pgfpathlineto{\pgfqpoint{4.357689in}{0.625708in}}%
\pgfpathlineto{\pgfqpoint{4.361985in}{0.619675in}}%
\pgfpathlineto{\pgfqpoint{4.358366in}{0.618264in}}%
\pgfpathlineto{\pgfqpoint{4.362868in}{0.612534in}}%
\pgfpathlineto{\pgfqpoint{4.368358in}{0.613145in}}%
\pgfpathlineto{\pgfqpoint{4.371249in}{0.610567in}}%
\pgfpathlineto{\pgfqpoint{4.378562in}{0.606765in}}%
\pgfpathlineto{\pgfqpoint{4.387995in}{0.593271in}}%
\pgfpathlineto{\pgfqpoint{4.387782in}{0.590043in}}%
\pgfpathlineto{\pgfqpoint{4.401928in}{0.579706in}}%
\pgfpathlineto{\pgfqpoint{4.402233in}{0.576049in}}%
\pgfpathlineto{\pgfqpoint{4.406003in}{0.570578in}}%
\pgfpathlineto{\pgfqpoint{4.417585in}{0.567843in}}%
\pgfpathlineto{\pgfqpoint{4.421881in}{0.565899in}}%
\pgfpathlineto{\pgfqpoint{4.425922in}{0.559868in}}%
\pgfpathlineto{\pgfqpoint{4.426442in}{0.554585in}}%
\pgfpathlineto{\pgfqpoint{4.429915in}{0.550279in}}%
\pgfpathlineto{\pgfqpoint{4.430958in}{0.545423in}}%
\pgfpathlineto{\pgfqpoint{4.434213in}{0.543649in}}%
\pgfpathlineto{\pgfqpoint{4.418089in}{0.514548in}}%
\pgfpathlineto{\pgfqpoint{4.384622in}{0.454133in}}%
\pgfpathlineto{\pgfqpoint{4.335518in}{0.365475in}}%
\pgfpathlineto{\pgfqpoint{4.316294in}{0.330756in}}%
\pgfpathlineto{\pgfqpoint{4.320388in}{0.326093in}}%
\pgfpathlineto{\pgfqpoint{4.322192in}{0.327581in}}%
\pgfpathlineto{\pgfqpoint{4.325850in}{0.322152in}}%
\pgfpathlineto{\pgfqpoint{4.330927in}{0.323727in}}%
\pgfpathlineto{\pgfqpoint{4.337667in}{0.320475in}}%
\pgfpathlineto{\pgfqpoint{4.333301in}{0.315393in}}%
\pgfpathlineto{\pgfqpoint{4.336672in}{0.308535in}}%
\pgfpathlineto{\pgfqpoint{4.335976in}{0.305196in}}%
\pgfpathlineto{\pgfqpoint{4.341353in}{0.287587in}}%
\pgfpathlineto{\pgfqpoint{4.339077in}{0.279580in}}%
\pgfpathlineto{\pgfqpoint{4.349350in}{0.281562in}}%
\pgfpathlineto{\pgfqpoint{4.351872in}{0.280263in}}%
\pgfpathlineto{\pgfqpoint{4.354610in}{0.282376in}}%
\pgfpathlineto{\pgfqpoint{4.356558in}{0.286368in}}%
\pgfpathlineto{\pgfqpoint{4.359347in}{0.288657in}}%
\pgfpathlineto{\pgfqpoint{4.371252in}{0.288367in}}%
\pgfpathlineto{\pgfqpoint{4.373943in}{0.280683in}}%
\pgfpathlineto{\pgfqpoint{4.371634in}{0.273938in}}%
\pgfpathlineto{\pgfqpoint{4.374257in}{0.271712in}}%
\pgfpathlineto{\pgfqpoint{4.375636in}{0.267905in}}%
\pgfpathlineto{\pgfqpoint{4.375336in}{0.260746in}}%
\pgfpathlineto{\pgfqpoint{4.378836in}{0.255613in}}%
\pgfpathlineto{\pgfqpoint{4.380755in}{0.247568in}}%
\pgfpathlineto{\pgfqpoint{4.379770in}{0.243223in}}%
\pgfpathlineto{\pgfqpoint{4.380958in}{0.238819in}}%
\pgfpathlineto{\pgfqpoint{4.382671in}{0.213318in}}%
\pgfpathlineto{\pgfqpoint{4.380216in}{0.211309in}}%
\pgfpathlineto{\pgfqpoint{4.383455in}{0.208533in}}%
\pgfpathlineto{\pgfqpoint{4.380910in}{0.205157in}}%
\pgfpathlineto{\pgfqpoint{4.383206in}{0.202405in}}%
\pgfpathlineto{\pgfqpoint{4.381727in}{0.197707in}}%
\pgfpathlineto{\pgfqpoint{4.385034in}{0.196433in}}%
\pgfpathlineto{\pgfqpoint{4.389287in}{0.189287in}}%
\pgfpathlineto{\pgfqpoint{4.392447in}{0.187134in}}%
\pgfpathlineto{\pgfqpoint{4.393274in}{0.183987in}}%
\pgfpathlineto{\pgfqpoint{4.395105in}{0.182623in}}%
\pgfpathlineto{\pgfqpoint{4.394723in}{0.180010in}}%
\pgfpathlineto{\pgfqpoint{4.398670in}{0.178116in}}%
\pgfpathlineto{\pgfqpoint{4.397696in}{0.173088in}}%
\pgfpathlineto{\pgfqpoint{4.394298in}{0.170454in}}%
\pgfpathlineto{\pgfqpoint{4.392755in}{0.165560in}}%
\pgfpathlineto{\pgfqpoint{4.392345in}{0.159006in}}%
\pgfpathlineto{\pgfqpoint{4.390458in}{0.158379in}}%
\pgfpathlineto{\pgfqpoint{4.384289in}{0.152634in}}%
\pgfpathlineto{\pgfqpoint{4.378806in}{0.151118in}}%
\pgfpathlineto{\pgfqpoint{4.376190in}{0.153435in}}%
\pgfpathlineto{\pgfqpoint{4.378718in}{0.159583in}}%
\pgfpathlineto{\pgfqpoint{4.377946in}{0.162601in}}%
\pgfpathlineto{\pgfqpoint{4.381568in}{0.164131in}}%
\pgfpathlineto{\pgfqpoint{4.384956in}{0.173027in}}%
\pgfpathlineto{\pgfqpoint{4.383525in}{0.180625in}}%
\pgfpathlineto{\pgfqpoint{4.381235in}{0.182419in}}%
\pgfpathlineto{\pgfqpoint{4.373578in}{0.181838in}}%
\pgfpathlineto{\pgfqpoint{4.375121in}{0.179853in}}%
\pgfpathlineto{\pgfqpoint{4.369579in}{0.174155in}}%
\pgfpathlineto{\pgfqpoint{4.367793in}{0.177323in}}%
\pgfpathlineto{\pgfqpoint{4.368258in}{0.181462in}}%
\pgfpathlineto{\pgfqpoint{4.371439in}{0.181386in}}%
\pgfpathlineto{\pgfqpoint{4.374272in}{0.184426in}}%
\pgfpathlineto{\pgfqpoint{4.375948in}{0.188866in}}%
\pgfpathlineto{\pgfqpoint{4.382264in}{0.187613in}}%
\pgfpathlineto{\pgfqpoint{4.377208in}{0.189984in}}%
\pgfpathlineto{\pgfqpoint{4.377653in}{0.193161in}}%
\pgfpathlineto{\pgfqpoint{4.375549in}{0.194596in}}%
\pgfpathlineto{\pgfqpoint{4.374687in}{0.198923in}}%
\pgfpathlineto{\pgfqpoint{4.376181in}{0.201177in}}%
\pgfpathlineto{\pgfqpoint{4.372619in}{0.208179in}}%
\pgfpathlineto{\pgfqpoint{4.372919in}{0.212633in}}%
\pgfpathlineto{\pgfqpoint{4.368746in}{0.216599in}}%
\pgfpathlineto{\pgfqpoint{4.366488in}{0.219808in}}%
\pgfpathlineto{\pgfqpoint{4.369606in}{0.222866in}}%
\pgfpathlineto{\pgfqpoint{4.370761in}{0.227437in}}%
\pgfpathlineto{\pgfqpoint{4.369730in}{0.230438in}}%
\pgfpathlineto{\pgfqpoint{4.370936in}{0.234017in}}%
\pgfpathlineto{\pgfqpoint{4.369399in}{0.245686in}}%
\pgfpathlineto{\pgfqpoint{4.367100in}{0.250980in}}%
\pgfpathlineto{\pgfqpoint{4.364471in}{0.252970in}}%
\pgfpathlineto{\pgfqpoint{4.365590in}{0.259857in}}%
\pgfpathlineto{\pgfqpoint{4.364801in}{0.265235in}}%
\pgfpathlineto{\pgfqpoint{4.366230in}{0.268685in}}%
\pgfpathlineto{\pgfqpoint{4.363272in}{0.269169in}}%
\pgfpathlineto{\pgfqpoint{4.362479in}{0.260789in}}%
\pgfpathlineto{\pgfqpoint{4.359154in}{0.251441in}}%
\pgfpathlineto{\pgfqpoint{4.356484in}{0.253570in}}%
\pgfpathlineto{\pgfqpoint{4.356127in}{0.256642in}}%
\pgfpathlineto{\pgfqpoint{4.352861in}{0.260839in}}%
\pgfpathlineto{\pgfqpoint{4.354521in}{0.263587in}}%
\pgfpathlineto{\pgfqpoint{4.354005in}{0.269772in}}%
\pgfpathlineto{\pgfqpoint{4.349778in}{0.267597in}}%
\pgfpathlineto{\pgfqpoint{4.350607in}{0.264242in}}%
\pgfpathlineto{\pgfqpoint{4.349811in}{0.259983in}}%
\pgfpathlineto{\pgfqpoint{4.345066in}{0.259990in}}%
\pgfpathlineto{\pgfqpoint{4.342939in}{0.262630in}}%
\pgfpathlineto{\pgfqpoint{4.339729in}{0.263134in}}%
\pgfpathlineto{\pgfqpoint{4.337569in}{0.265977in}}%
\pgfpathlineto{\pgfqpoint{4.333852in}{0.273782in}}%
\pgfpathlineto{\pgfqpoint{4.332648in}{0.279842in}}%
\pgfpathlineto{\pgfqpoint{4.333418in}{0.282418in}}%
\pgfpathlineto{\pgfqpoint{4.331620in}{0.288091in}}%
\pgfpathlineto{\pgfqpoint{4.328639in}{0.291375in}}%
\pgfpathlineto{\pgfqpoint{4.323483in}{0.299768in}}%
\pgfpathlineto{\pgfqpoint{4.319744in}{0.307057in}}%
\pgfpathlineto{\pgfqpoint{4.323559in}{0.307167in}}%
\pgfpathlineto{\pgfqpoint{4.325804in}{0.308698in}}%
\pgfpathlineto{\pgfqpoint{4.326246in}{0.313435in}}%
\pgfpathlineto{\pgfqpoint{4.315450in}{0.314007in}}%
\pgfpathlineto{\pgfqpoint{4.306267in}{0.324264in}}%
\pgfpathlineto{\pgfqpoint{4.308682in}{0.326962in}}%
\pgfpathlineto{\pgfqpoint{4.304241in}{0.327423in}}%
\pgfpathlineto{\pgfqpoint{4.295423in}{0.336959in}}%
\pgfpathlineto{\pgfqpoint{4.281451in}{0.341741in}}%
\pgfpathlineto{\pgfqpoint{4.279570in}{0.347345in}}%
\pgfpathlineto{\pgfqpoint{4.276671in}{0.350264in}}%
\pgfpathlineto{\pgfqpoint{4.276539in}{0.353005in}}%
\pgfpathlineto{\pgfqpoint{4.274358in}{0.355041in}}%
\pgfpathlineto{\pgfqpoint{4.277220in}{0.358442in}}%
\pgfpathlineto{\pgfqpoint{4.270693in}{0.358770in}}%
\pgfpathlineto{\pgfqpoint{4.269034in}{0.363173in}}%
\pgfpathlineto{\pgfqpoint{4.266063in}{0.365047in}}%
\pgfpathlineto{\pgfqpoint{4.272012in}{0.367309in}}%
\pgfpathlineto{\pgfqpoint{4.268918in}{0.371345in}}%
\pgfpathlineto{\pgfqpoint{4.263224in}{0.373038in}}%
\pgfpathlineto{\pgfqpoint{4.264110in}{0.380491in}}%
\pgfpathlineto{\pgfqpoint{4.262478in}{0.383130in}}%
\pgfpathlineto{\pgfqpoint{4.259538in}{0.384988in}}%
\pgfpathlineto{\pgfqpoint{4.257024in}{0.383124in}}%
\pgfpathlineto{\pgfqpoint{4.255538in}{0.384877in}}%
\pgfpathlineto{\pgfqpoint{4.251206in}{0.385379in}}%
\pgfpathlineto{\pgfqpoint{4.251555in}{0.392143in}}%
\pgfpathlineto{\pgfqpoint{4.245275in}{0.387574in}}%
\pgfpathlineto{\pgfqpoint{4.245407in}{0.378160in}}%
\pgfpathlineto{\pgfqpoint{4.238666in}{0.376355in}}%
\pgfpathlineto{\pgfqpoint{4.234263in}{0.372596in}}%
\pgfpathlineto{\pgfqpoint{4.230725in}{0.371804in}}%
\pgfpathlineto{\pgfqpoint{4.226895in}{0.375593in}}%
\pgfpathlineto{\pgfqpoint{4.223485in}{0.375090in}}%
\pgfpathlineto{\pgfqpoint{4.224118in}{0.378551in}}%
\pgfpathlineto{\pgfqpoint{4.219270in}{0.376688in}}%
\pgfpathlineto{\pgfqpoint{4.214779in}{0.376478in}}%
\pgfpathlineto{\pgfqpoint{4.209523in}{0.374589in}}%
\pgfpathlineto{\pgfqpoint{4.198855in}{0.375652in}}%
\pgfpathlineto{\pgfqpoint{4.195430in}{0.374953in}}%
\pgfpathlineto{\pgfqpoint{4.192959in}{0.378065in}}%
\pgfpathlineto{\pgfqpoint{4.188397in}{0.378376in}}%
\pgfpathlineto{\pgfqpoint{4.187449in}{0.382409in}}%
\pgfpathlineto{\pgfqpoint{4.190167in}{0.384606in}}%
\pgfpathlineto{\pgfqpoint{4.193357in}{0.384399in}}%
\pgfpathlineto{\pgfqpoint{4.196019in}{0.381403in}}%
\pgfpathlineto{\pgfqpoint{4.197772in}{0.386005in}}%
\pgfpathlineto{\pgfqpoint{4.195326in}{0.391102in}}%
\pgfpathlineto{\pgfqpoint{4.200863in}{0.395589in}}%
\pgfpathlineto{\pgfqpoint{4.206411in}{0.397243in}}%
\pgfpathlineto{\pgfqpoint{4.210273in}{0.399940in}}%
\pgfpathlineto{\pgfqpoint{4.212775in}{0.402846in}}%
\pgfpathlineto{\pgfqpoint{4.214172in}{0.407609in}}%
\pgfpathlineto{\pgfqpoint{4.218583in}{0.406312in}}%
\pgfpathlineto{\pgfqpoint{4.228148in}{0.406979in}}%
\pgfpathlineto{\pgfqpoint{4.230201in}{0.401756in}}%
\pgfpathlineto{\pgfqpoint{4.233556in}{0.401529in}}%
\pgfpathlineto{\pgfqpoint{4.239662in}{0.393534in}}%
\pgfpathlineto{\pgfqpoint{4.239673in}{0.396448in}}%
\pgfpathlineto{\pgfqpoint{4.237540in}{0.397396in}}%
\pgfpathlineto{\pgfqpoint{4.233636in}{0.406940in}}%
\pgfpathlineto{\pgfqpoint{4.235536in}{0.408236in}}%
\pgfpathlineto{\pgfqpoint{4.230322in}{0.413215in}}%
\pgfpathlineto{\pgfqpoint{4.225663in}{0.414184in}}%
\pgfpathlineto{\pgfqpoint{4.221223in}{0.412463in}}%
\pgfpathlineto{\pgfqpoint{4.213619in}{0.413804in}}%
\pgfpathlineto{\pgfqpoint{4.210013in}{0.411392in}}%
\pgfpathlineto{\pgfqpoint{4.201801in}{0.409529in}}%
\pgfpathlineto{\pgfqpoint{4.201050in}{0.406847in}}%
\pgfpathlineto{\pgfqpoint{4.194741in}{0.406115in}}%
\pgfpathlineto{\pgfqpoint{4.193122in}{0.403049in}}%
\pgfpathlineto{\pgfqpoint{4.189513in}{0.400803in}}%
\pgfpathlineto{\pgfqpoint{4.185154in}{0.401189in}}%
\pgfpathlineto{\pgfqpoint{4.182977in}{0.399018in}}%
\pgfpathlineto{\pgfqpoint{4.180231in}{0.399119in}}%
\pgfpathlineto{\pgfqpoint{4.177607in}{0.401074in}}%
\pgfpathlineto{\pgfqpoint{4.173019in}{0.400725in}}%
\pgfpathlineto{\pgfqpoint{4.171938in}{0.398924in}}%
\pgfpathlineto{\pgfqpoint{4.167119in}{0.400773in}}%
\pgfpathlineto{\pgfqpoint{4.163027in}{0.396000in}}%
\pgfpathlineto{\pgfqpoint{4.166950in}{0.391382in}}%
\pgfpathlineto{\pgfqpoint{4.168460in}{0.387584in}}%
\pgfpathlineto{\pgfqpoint{4.167068in}{0.384367in}}%
\pgfpathlineto{\pgfqpoint{4.161493in}{0.382148in}}%
\pgfpathlineto{\pgfqpoint{4.158263in}{0.383954in}}%
\pgfpathlineto{\pgfqpoint{4.154198in}{0.382976in}}%
\pgfpathlineto{\pgfqpoint{4.153708in}{0.380157in}}%
\pgfpathlineto{\pgfqpoint{4.148759in}{0.381154in}}%
\pgfpathlineto{\pgfqpoint{4.149855in}{0.378337in}}%
\pgfpathlineto{\pgfqpoint{4.142555in}{0.377743in}}%
\pgfpathlineto{\pgfqpoint{4.137830in}{0.381319in}}%
\pgfpathlineto{\pgfqpoint{4.135306in}{0.379045in}}%
\pgfpathlineto{\pgfqpoint{4.128590in}{0.381466in}}%
\pgfpathlineto{\pgfqpoint{4.126918in}{0.379014in}}%
\pgfpathlineto{\pgfqpoint{4.123588in}{0.377923in}}%
\pgfpathlineto{\pgfqpoint{4.120917in}{0.380797in}}%
\pgfpathlineto{\pgfqpoint{4.111784in}{0.380083in}}%
\pgfpathlineto{\pgfqpoint{4.111794in}{0.375843in}}%
\pgfpathlineto{\pgfqpoint{4.108125in}{0.374721in}}%
\pgfpathlineto{\pgfqpoint{4.105270in}{0.376900in}}%
\pgfpathlineto{\pgfqpoint{4.101497in}{0.375689in}}%
\pgfpathlineto{\pgfqpoint{4.096614in}{0.375401in}}%
\pgfpathlineto{\pgfqpoint{4.089340in}{0.378466in}}%
\pgfpathlineto{\pgfqpoint{4.085174in}{0.375305in}}%
\pgfpathlineto{\pgfqpoint{4.079216in}{0.379530in}}%
\pgfpathlineto{\pgfqpoint{4.077268in}{0.377929in}}%
\pgfpathlineto{\pgfqpoint{4.075900in}{0.373791in}}%
\pgfpathlineto{\pgfqpoint{4.072071in}{0.371365in}}%
\pgfpathlineto{\pgfqpoint{4.065825in}{0.374142in}}%
\pgfpathlineto{\pgfqpoint{4.055479in}{0.381003in}}%
\pgfpathlineto{\pgfqpoint{4.052463in}{0.381613in}}%
\pgfpathlineto{\pgfqpoint{4.050762in}{0.380183in}}%
\pgfpathlineto{\pgfqpoint{4.046211in}{0.381155in}}%
\pgfpathlineto{\pgfqpoint{4.043991in}{0.383290in}}%
\pgfpathlineto{\pgfqpoint{4.039445in}{0.384441in}}%
\pgfpathlineto{\pgfqpoint{4.033238in}{0.384507in}}%
\pgfpathlineto{\pgfqpoint{4.030773in}{0.387054in}}%
\pgfpathlineto{\pgfqpoint{4.035497in}{0.389690in}}%
\pgfpathlineto{\pgfqpoint{4.034395in}{0.392291in}}%
\pgfpathlineto{\pgfqpoint{4.031149in}{0.391707in}}%
\pgfpathlineto{\pgfqpoint{4.029528in}{0.389662in}}%
\pgfpathlineto{\pgfqpoint{4.021930in}{0.386730in}}%
\pgfpathlineto{\pgfqpoint{4.018088in}{0.387727in}}%
\pgfpathlineto{\pgfqpoint{4.016471in}{0.393681in}}%
\pgfpathlineto{\pgfqpoint{4.011679in}{0.391121in}}%
\pgfpathlineto{\pgfqpoint{4.010286in}{0.398667in}}%
\pgfpathlineto{\pgfqpoint{4.008015in}{0.396722in}}%
\pgfpathlineto{\pgfqpoint{4.008395in}{0.393819in}}%
\pgfpathlineto{\pgfqpoint{4.003173in}{0.394586in}}%
\pgfpathlineto{\pgfqpoint{4.008605in}{0.399646in}}%
\pgfpathlineto{\pgfqpoint{4.014110in}{0.396639in}}%
\pgfpathlineto{\pgfqpoint{4.015291in}{0.398148in}}%
\pgfpathlineto{\pgfqpoint{4.021082in}{0.396339in}}%
\pgfpathlineto{\pgfqpoint{4.020720in}{0.398399in}}%
\pgfpathlineto{\pgfqpoint{4.036521in}{0.399081in}}%
\pgfpathlineto{\pgfqpoint{4.044230in}{0.395868in}}%
\pgfpathlineto{\pgfqpoint{4.047212in}{0.392698in}}%
\pgfpathlineto{\pgfqpoint{4.046327in}{0.389931in}}%
\pgfpathlineto{\pgfqpoint{4.049323in}{0.388883in}}%
\pgfpathlineto{\pgfqpoint{4.056765in}{0.393251in}}%
\pgfpathlineto{\pgfqpoint{4.066392in}{0.393531in}}%
\pgfpathlineto{\pgfqpoint{4.075136in}{0.390898in}}%
\pgfpathlineto{\pgfqpoint{4.080089in}{0.391340in}}%
\pgfpathlineto{\pgfqpoint{4.081933in}{0.387255in}}%
\pgfpathlineto{\pgfqpoint{4.084703in}{0.391758in}}%
\pgfpathlineto{\pgfqpoint{4.086407in}{0.392733in}}%
\pgfpathlineto{\pgfqpoint{4.094288in}{0.394691in}}%
\pgfpathlineto{\pgfqpoint{4.097217in}{0.393494in}}%
\pgfpathlineto{\pgfqpoint{4.100385in}{0.394915in}}%
\pgfpathlineto{\pgfqpoint{4.104163in}{0.394172in}}%
\pgfpathlineto{\pgfqpoint{4.105055in}{0.395827in}}%
\pgfpathlineto{\pgfqpoint{4.112939in}{0.403420in}}%
\pgfpathlineto{\pgfqpoint{4.116278in}{0.404545in}}%
\pgfpathlineto{\pgfqpoint{4.118251in}{0.408644in}}%
\pgfpathlineto{\pgfqpoint{4.128722in}{0.411817in}}%
\pgfpathlineto{\pgfqpoint{4.130033in}{0.414238in}}%
\pgfpathlineto{\pgfqpoint{4.115192in}{0.417974in}}%
\pgfpathlineto{\pgfqpoint{4.115776in}{0.421993in}}%
\pgfpathlineto{\pgfqpoint{4.114250in}{0.427301in}}%
\pgfpathlineto{\pgfqpoint{4.110794in}{0.426860in}}%
\pgfpathlineto{\pgfqpoint{4.108443in}{0.419972in}}%
\pgfpathlineto{\pgfqpoint{4.105714in}{0.419404in}}%
\pgfpathlineto{\pgfqpoint{4.103938in}{0.421634in}}%
\pgfpathlineto{\pgfqpoint{4.106010in}{0.431388in}}%
\pgfpathlineto{\pgfqpoint{4.103947in}{0.435475in}}%
\pgfpathlineto{\pgfqpoint{4.099950in}{0.440293in}}%
\pgfpathlineto{\pgfqpoint{4.101871in}{0.443969in}}%
\pgfpathlineto{\pgfqpoint{4.093436in}{0.444102in}}%
\pgfpathlineto{\pgfqpoint{4.093628in}{0.445528in}}%
\pgfpathlineto{\pgfqpoint{4.088595in}{0.447526in}}%
\pgfpathlineto{\pgfqpoint{4.087031in}{0.446426in}}%
\pgfpathclose%
\pgfusepath{fill}%
\end{pgfscope}%
\begin{pgfscope}%
\pgfpathrectangle{\pgfqpoint{3.625000in}{0.100000in}}{\pgfqpoint{2.989028in}{1.913466in}}%
\pgfusepath{clip}%
\pgfsetbuttcap%
\pgfsetmiterjoin%
\definecolor{currentfill}{rgb}{0.312034,0.662668,0.687659}%
\pgfsetfillcolor{currentfill}%
\pgfsetlinewidth{0.000000pt}%
\definecolor{currentstroke}{rgb}{0.000000,0.000000,0.000000}%
\pgfsetstrokecolor{currentstroke}%
\pgfsetstrokeopacity{0.000000}%
\pgfsetdash{}{0pt}%
\pgfpathmoveto{\pgfqpoint{4.071979in}{0.517466in}}%
\pgfpathlineto{\pgfqpoint{4.070936in}{0.515633in}}%
\pgfpathlineto{\pgfqpoint{4.073696in}{0.511730in}}%
\pgfpathlineto{\pgfqpoint{4.069334in}{0.508079in}}%
\pgfpathlineto{\pgfqpoint{4.069455in}{0.504939in}}%
\pgfpathlineto{\pgfqpoint{4.067483in}{0.504198in}}%
\pgfpathlineto{\pgfqpoint{4.060704in}{0.508881in}}%
\pgfpathlineto{\pgfqpoint{4.055872in}{0.519132in}}%
\pgfpathlineto{\pgfqpoint{4.055549in}{0.522094in}}%
\pgfpathlineto{\pgfqpoint{4.056791in}{0.525640in}}%
\pgfpathlineto{\pgfqpoint{4.062027in}{0.520216in}}%
\pgfpathlineto{\pgfqpoint{4.063628in}{0.521262in}}%
\pgfpathlineto{\pgfqpoint{4.068070in}{0.520267in}}%
\pgfpathlineto{\pgfqpoint{4.071979in}{0.517466in}}%
\pgfpathclose%
\pgfusepath{fill}%
\end{pgfscope}%
\begin{pgfscope}%
\pgfpathrectangle{\pgfqpoint{3.625000in}{0.100000in}}{\pgfqpoint{2.989028in}{1.913466in}}%
\pgfusepath{clip}%
\pgfsetbuttcap%
\pgfsetmiterjoin%
\definecolor{currentfill}{rgb}{0.312034,0.662668,0.687659}%
\pgfsetfillcolor{currentfill}%
\pgfsetlinewidth{0.000000pt}%
\definecolor{currentstroke}{rgb}{0.000000,0.000000,0.000000}%
\pgfsetstrokecolor{currentstroke}%
\pgfsetstrokeopacity{0.000000}%
\pgfsetdash{}{0pt}%
\pgfpathmoveto{\pgfqpoint{3.991584in}{0.398705in}}%
\pgfpathlineto{\pgfqpoint{3.986820in}{0.397932in}}%
\pgfpathlineto{\pgfqpoint{3.983372in}{0.399525in}}%
\pgfpathlineto{\pgfqpoint{3.981865in}{0.402120in}}%
\pgfpathlineto{\pgfqpoint{3.983573in}{0.405842in}}%
\pgfpathlineto{\pgfqpoint{3.987377in}{0.404858in}}%
\pgfpathlineto{\pgfqpoint{3.993630in}{0.407066in}}%
\pgfpathlineto{\pgfqpoint{3.995820in}{0.404016in}}%
\pgfpathlineto{\pgfqpoint{3.997944in}{0.404579in}}%
\pgfpathlineto{\pgfqpoint{4.003084in}{0.402681in}}%
\pgfpathlineto{\pgfqpoint{4.005305in}{0.400269in}}%
\pgfpathlineto{\pgfqpoint{4.002595in}{0.393963in}}%
\pgfpathlineto{\pgfqpoint{3.999899in}{0.392626in}}%
\pgfpathlineto{\pgfqpoint{3.997476in}{0.393346in}}%
\pgfpathlineto{\pgfqpoint{3.991584in}{0.398705in}}%
\pgfpathclose%
\pgfusepath{fill}%
\end{pgfscope}%
\begin{pgfscope}%
\pgfpathrectangle{\pgfqpoint{3.625000in}{0.100000in}}{\pgfqpoint{2.989028in}{1.913466in}}%
\pgfusepath{clip}%
\pgfsetbuttcap%
\pgfsetmiterjoin%
\definecolor{currentfill}{rgb}{0.312034,0.662668,0.687659}%
\pgfsetfillcolor{currentfill}%
\pgfsetlinewidth{0.000000pt}%
\definecolor{currentstroke}{rgb}{0.000000,0.000000,0.000000}%
\pgfsetstrokecolor{currentstroke}%
\pgfsetstrokeopacity{0.000000}%
\pgfsetdash{}{0pt}%
\pgfpathmoveto{\pgfqpoint{3.948985in}{0.404436in}}%
\pgfpathlineto{\pgfqpoint{3.942234in}{0.407656in}}%
\pgfpathlineto{\pgfqpoint{3.935987in}{0.408614in}}%
\pgfpathlineto{\pgfqpoint{3.934018in}{0.409985in}}%
\pgfpathlineto{\pgfqpoint{3.936260in}{0.412685in}}%
\pgfpathlineto{\pgfqpoint{3.940239in}{0.409312in}}%
\pgfpathlineto{\pgfqpoint{3.943576in}{0.409777in}}%
\pgfpathlineto{\pgfqpoint{3.949004in}{0.412113in}}%
\pgfpathlineto{\pgfqpoint{3.949442in}{0.415120in}}%
\pgfpathlineto{\pgfqpoint{3.956109in}{0.413632in}}%
\pgfpathlineto{\pgfqpoint{3.954548in}{0.409103in}}%
\pgfpathlineto{\pgfqpoint{3.958794in}{0.408772in}}%
\pgfpathlineto{\pgfqpoint{3.955231in}{0.403385in}}%
\pgfpathlineto{\pgfqpoint{3.951781in}{0.405131in}}%
\pgfpathlineto{\pgfqpoint{3.948985in}{0.404436in}}%
\pgfpathclose%
\pgfusepath{fill}%
\end{pgfscope}%
\begin{pgfscope}%
\pgfpathrectangle{\pgfqpoint{3.625000in}{0.100000in}}{\pgfqpoint{2.989028in}{1.913466in}}%
\pgfusepath{clip}%
\pgfsetbuttcap%
\pgfsetmiterjoin%
\definecolor{currentfill}{rgb}{0.312034,0.662668,0.687659}%
\pgfsetfillcolor{currentfill}%
\pgfsetlinewidth{0.000000pt}%
\definecolor{currentstroke}{rgb}{0.000000,0.000000,0.000000}%
\pgfsetstrokecolor{currentstroke}%
\pgfsetstrokeopacity{0.000000}%
\pgfsetdash{}{0pt}%
\pgfpathmoveto{\pgfqpoint{3.937124in}{0.415568in}}%
\pgfpathlineto{\pgfqpoint{3.934643in}{0.414219in}}%
\pgfpathlineto{\pgfqpoint{3.927882in}{0.416293in}}%
\pgfpathlineto{\pgfqpoint{3.922830in}{0.415092in}}%
\pgfpathlineto{\pgfqpoint{3.920264in}{0.418504in}}%
\pgfpathlineto{\pgfqpoint{3.921383in}{0.420066in}}%
\pgfpathlineto{\pgfqpoint{3.925212in}{0.420327in}}%
\pgfpathlineto{\pgfqpoint{3.928140in}{0.419346in}}%
\pgfpathlineto{\pgfqpoint{3.931331in}{0.420898in}}%
\pgfpathlineto{\pgfqpoint{3.936223in}{0.418823in}}%
\pgfpathlineto{\pgfqpoint{3.937124in}{0.415568in}}%
\pgfpathclose%
\pgfusepath{fill}%
\end{pgfscope}%
\begin{pgfscope}%
\pgfpathrectangle{\pgfqpoint{3.625000in}{0.100000in}}{\pgfqpoint{2.989028in}{1.913466in}}%
\pgfusepath{clip}%
\pgfsetbuttcap%
\pgfsetmiterjoin%
\definecolor{currentfill}{rgb}{0.312034,0.662668,0.687659}%
\pgfsetfillcolor{currentfill}%
\pgfsetlinewidth{0.000000pt}%
\definecolor{currentstroke}{rgb}{0.000000,0.000000,0.000000}%
\pgfsetstrokecolor{currentstroke}%
\pgfsetstrokeopacity{0.000000}%
\pgfsetdash{}{0pt}%
\pgfpathmoveto{\pgfqpoint{3.857429in}{0.465782in}}%
\pgfpathlineto{\pgfqpoint{3.857250in}{0.462436in}}%
\pgfpathlineto{\pgfqpoint{3.852314in}{0.460334in}}%
\pgfpathlineto{\pgfqpoint{3.847807in}{0.465597in}}%
\pgfpathlineto{\pgfqpoint{3.852862in}{0.467350in}}%
\pgfpathlineto{\pgfqpoint{3.857429in}{0.465782in}}%
\pgfpathclose%
\pgfusepath{fill}%
\end{pgfscope}%
\begin{pgfscope}%
\pgfpathrectangle{\pgfqpoint{3.625000in}{0.100000in}}{\pgfqpoint{2.989028in}{1.913466in}}%
\pgfusepath{clip}%
\pgfsetbuttcap%
\pgfsetmiterjoin%
\definecolor{currentfill}{rgb}{0.312034,0.662668,0.687659}%
\pgfsetfillcolor{currentfill}%
\pgfsetlinewidth{0.000000pt}%
\definecolor{currentstroke}{rgb}{0.000000,0.000000,0.000000}%
\pgfsetstrokecolor{currentstroke}%
\pgfsetstrokeopacity{0.000000}%
\pgfsetdash{}{0pt}%
\pgfpathmoveto{\pgfqpoint{3.816722in}{0.484589in}}%
\pgfpathlineto{\pgfqpoint{3.817771in}{0.486043in}}%
\pgfpathlineto{\pgfqpoint{3.823825in}{0.485792in}}%
\pgfpathlineto{\pgfqpoint{3.824887in}{0.487956in}}%
\pgfpathlineto{\pgfqpoint{3.827543in}{0.486068in}}%
\pgfpathlineto{\pgfqpoint{3.826009in}{0.482435in}}%
\pgfpathlineto{\pgfqpoint{3.823753in}{0.482127in}}%
\pgfpathlineto{\pgfqpoint{3.818361in}{0.483267in}}%
\pgfpathlineto{\pgfqpoint{3.816722in}{0.484589in}}%
\pgfpathclose%
\pgfusepath{fill}%
\end{pgfscope}%
\begin{pgfscope}%
\pgfpathrectangle{\pgfqpoint{3.625000in}{0.100000in}}{\pgfqpoint{2.989028in}{1.913466in}}%
\pgfusepath{clip}%
\pgfsetbuttcap%
\pgfsetmiterjoin%
\definecolor{currentfill}{rgb}{0.312034,0.662668,0.687659}%
\pgfsetfillcolor{currentfill}%
\pgfsetlinewidth{0.000000pt}%
\definecolor{currentstroke}{rgb}{0.000000,0.000000,0.000000}%
\pgfsetstrokecolor{currentstroke}%
\pgfsetstrokeopacity{0.000000}%
\pgfsetdash{}{0pt}%
\pgfpathmoveto{\pgfqpoint{3.808137in}{0.495848in}}%
\pgfpathlineto{\pgfqpoint{3.807770in}{0.499125in}}%
\pgfpathlineto{\pgfqpoint{3.810929in}{0.498882in}}%
\pgfpathlineto{\pgfqpoint{3.810709in}{0.503484in}}%
\pgfpathlineto{\pgfqpoint{3.813886in}{0.501000in}}%
\pgfpathlineto{\pgfqpoint{3.812787in}{0.497411in}}%
\pgfpathlineto{\pgfqpoint{3.808137in}{0.495848in}}%
\pgfpathclose%
\pgfusepath{fill}%
\end{pgfscope}%
\begin{pgfscope}%
\pgfpathrectangle{\pgfqpoint{3.625000in}{0.100000in}}{\pgfqpoint{2.989028in}{1.913466in}}%
\pgfusepath{clip}%
\pgfsetbuttcap%
\pgfsetmiterjoin%
\definecolor{currentfill}{rgb}{0.312034,0.662668,0.687659}%
\pgfsetfillcolor{currentfill}%
\pgfsetlinewidth{0.000000pt}%
\definecolor{currentstroke}{rgb}{0.000000,0.000000,0.000000}%
\pgfsetstrokecolor{currentstroke}%
\pgfsetstrokeopacity{0.000000}%
\pgfsetdash{}{0pt}%
\pgfpathmoveto{\pgfqpoint{4.086438in}{0.612633in}}%
\pgfpathlineto{\pgfqpoint{4.081400in}{0.613721in}}%
\pgfpathlineto{\pgfqpoint{4.080318in}{0.617088in}}%
\pgfpathlineto{\pgfqpoint{4.082049in}{0.620453in}}%
\pgfpathlineto{\pgfqpoint{4.085927in}{0.622211in}}%
\pgfpathlineto{\pgfqpoint{4.090367in}{0.613572in}}%
\pgfpathlineto{\pgfqpoint{4.094395in}{0.613189in}}%
\pgfpathlineto{\pgfqpoint{4.097399in}{0.610523in}}%
\pgfpathlineto{\pgfqpoint{4.097340in}{0.606855in}}%
\pgfpathlineto{\pgfqpoint{4.095143in}{0.605097in}}%
\pgfpathlineto{\pgfqpoint{4.098716in}{0.594249in}}%
\pgfpathlineto{\pgfqpoint{4.102625in}{0.590269in}}%
\pgfpathlineto{\pgfqpoint{4.098628in}{0.589006in}}%
\pgfpathlineto{\pgfqpoint{4.095393in}{0.593472in}}%
\pgfpathlineto{\pgfqpoint{4.088363in}{0.592089in}}%
\pgfpathlineto{\pgfqpoint{4.090549in}{0.596509in}}%
\pgfpathlineto{\pgfqpoint{4.089531in}{0.598269in}}%
\pgfpathlineto{\pgfqpoint{4.090128in}{0.602876in}}%
\pgfpathlineto{\pgfqpoint{4.088735in}{0.611204in}}%
\pgfpathlineto{\pgfqpoint{4.086438in}{0.612633in}}%
\pgfpathclose%
\pgfusepath{fill}%
\end{pgfscope}%
\begin{pgfscope}%
\pgfpathrectangle{\pgfqpoint{3.625000in}{0.100000in}}{\pgfqpoint{2.989028in}{1.913466in}}%
\pgfusepath{clip}%
\pgfsetbuttcap%
\pgfsetmiterjoin%
\definecolor{currentfill}{rgb}{0.312034,0.662668,0.687659}%
\pgfsetfillcolor{currentfill}%
\pgfsetlinewidth{0.000000pt}%
\definecolor{currentstroke}{rgb}{0.000000,0.000000,0.000000}%
\pgfsetstrokecolor{currentstroke}%
\pgfsetstrokeopacity{0.000000}%
\pgfsetdash{}{0pt}%
\pgfpathmoveto{\pgfqpoint{4.263441in}{0.364487in}}%
\pgfpathlineto{\pgfqpoint{4.256138in}{0.364569in}}%
\pgfpathlineto{\pgfqpoint{4.256700in}{0.368235in}}%
\pgfpathlineto{\pgfqpoint{4.259623in}{0.369546in}}%
\pgfpathlineto{\pgfqpoint{4.263441in}{0.364487in}}%
\pgfpathclose%
\pgfusepath{fill}%
\end{pgfscope}%
\begin{pgfscope}%
\pgfpathrectangle{\pgfqpoint{3.625000in}{0.100000in}}{\pgfqpoint{2.989028in}{1.913466in}}%
\pgfusepath{clip}%
\pgfsetbuttcap%
\pgfsetmiterjoin%
\definecolor{currentfill}{rgb}{0.312034,0.662668,0.687659}%
\pgfsetfillcolor{currentfill}%
\pgfsetlinewidth{0.000000pt}%
\definecolor{currentstroke}{rgb}{0.000000,0.000000,0.000000}%
\pgfsetstrokecolor{currentstroke}%
\pgfsetstrokeopacity{0.000000}%
\pgfsetdash{}{0pt}%
\pgfpathmoveto{\pgfqpoint{4.253306in}{0.367807in}}%
\pgfpathlineto{\pgfqpoint{4.248074in}{0.366541in}}%
\pgfpathlineto{\pgfqpoint{4.243102in}{0.364373in}}%
\pgfpathlineto{\pgfqpoint{4.241567in}{0.366927in}}%
\pgfpathlineto{\pgfqpoint{4.250117in}{0.368657in}}%
\pgfpathlineto{\pgfqpoint{4.253306in}{0.367807in}}%
\pgfpathclose%
\pgfusepath{fill}%
\end{pgfscope}%
\begin{pgfscope}%
\pgfpathrectangle{\pgfqpoint{3.625000in}{0.100000in}}{\pgfqpoint{2.989028in}{1.913466in}}%
\pgfusepath{clip}%
\pgfsetbuttcap%
\pgfsetmiterjoin%
\definecolor{currentfill}{rgb}{0.312034,0.662668,0.687659}%
\pgfsetfillcolor{currentfill}%
\pgfsetlinewidth{0.000000pt}%
\definecolor{currentstroke}{rgb}{0.000000,0.000000,0.000000}%
\pgfsetstrokecolor{currentstroke}%
\pgfsetstrokeopacity{0.000000}%
\pgfsetdash{}{0pt}%
\pgfpathmoveto{\pgfqpoint{4.173103in}{0.365212in}}%
\pgfpathlineto{\pgfqpoint{4.173303in}{0.362676in}}%
\pgfpathlineto{\pgfqpoint{4.170288in}{0.361100in}}%
\pgfpathlineto{\pgfqpoint{4.161875in}{0.364714in}}%
\pgfpathlineto{\pgfqpoint{4.160570in}{0.363261in}}%
\pgfpathlineto{\pgfqpoint{4.158535in}{0.370138in}}%
\pgfpathlineto{\pgfqpoint{4.162750in}{0.371095in}}%
\pgfpathlineto{\pgfqpoint{4.164698in}{0.368269in}}%
\pgfpathlineto{\pgfqpoint{4.168929in}{0.371807in}}%
\pgfpathlineto{\pgfqpoint{4.173103in}{0.365212in}}%
\pgfpathclose%
\pgfusepath{fill}%
\end{pgfscope}%
\begin{pgfscope}%
\pgfpathrectangle{\pgfqpoint{3.625000in}{0.100000in}}{\pgfqpoint{2.989028in}{1.913466in}}%
\pgfusepath{clip}%
\pgfsetbuttcap%
\pgfsetmiterjoin%
\definecolor{currentfill}{rgb}{0.312034,0.662668,0.687659}%
\pgfsetfillcolor{currentfill}%
\pgfsetlinewidth{0.000000pt}%
\definecolor{currentstroke}{rgb}{0.000000,0.000000,0.000000}%
\pgfsetstrokecolor{currentstroke}%
\pgfsetstrokeopacity{0.000000}%
\pgfsetdash{}{0pt}%
\pgfpathmoveto{\pgfqpoint{4.362890in}{0.224669in}}%
\pgfpathlineto{\pgfqpoint{4.355835in}{0.222042in}}%
\pgfpathlineto{\pgfqpoint{4.352098in}{0.221885in}}%
\pgfpathlineto{\pgfqpoint{4.355139in}{0.227457in}}%
\pgfpathlineto{\pgfqpoint{4.357738in}{0.229535in}}%
\pgfpathlineto{\pgfqpoint{4.357654in}{0.235306in}}%
\pgfpathlineto{\pgfqpoint{4.360479in}{0.245723in}}%
\pgfpathlineto{\pgfqpoint{4.363081in}{0.247702in}}%
\pgfpathlineto{\pgfqpoint{4.369087in}{0.244837in}}%
\pgfpathlineto{\pgfqpoint{4.368210in}{0.243196in}}%
\pgfpathlineto{\pgfqpoint{4.368665in}{0.235310in}}%
\pgfpathlineto{\pgfqpoint{4.366718in}{0.233182in}}%
\pgfpathlineto{\pgfqpoint{4.365881in}{0.227581in}}%
\pgfpathlineto{\pgfqpoint{4.362890in}{0.224669in}}%
\pgfpathclose%
\pgfusepath{fill}%
\end{pgfscope}%
\begin{pgfscope}%
\pgfpathrectangle{\pgfqpoint{3.625000in}{0.100000in}}{\pgfqpoint{2.989028in}{1.913466in}}%
\pgfusepath{clip}%
\pgfsetbuttcap%
\pgfsetmiterjoin%
\definecolor{currentfill}{rgb}{0.312034,0.662668,0.687659}%
\pgfsetfillcolor{currentfill}%
\pgfsetlinewidth{0.000000pt}%
\definecolor{currentstroke}{rgb}{0.000000,0.000000,0.000000}%
\pgfsetstrokecolor{currentstroke}%
\pgfsetstrokeopacity{0.000000}%
\pgfsetdash{}{0pt}%
\pgfpathmoveto{\pgfqpoint{4.347266in}{0.250492in}}%
\pgfpathlineto{\pgfqpoint{4.351399in}{0.243355in}}%
\pgfpathlineto{\pgfqpoint{4.350633in}{0.241653in}}%
\pgfpathlineto{\pgfqpoint{4.356161in}{0.239922in}}%
\pgfpathlineto{\pgfqpoint{4.354574in}{0.233362in}}%
\pgfpathlineto{\pgfqpoint{4.352167in}{0.233665in}}%
\pgfpathlineto{\pgfqpoint{4.345842in}{0.245101in}}%
\pgfpathlineto{\pgfqpoint{4.346742in}{0.239954in}}%
\pgfpathlineto{\pgfqpoint{4.345664in}{0.236990in}}%
\pgfpathlineto{\pgfqpoint{4.342986in}{0.235671in}}%
\pgfpathlineto{\pgfqpoint{4.341463in}{0.237092in}}%
\pgfpathlineto{\pgfqpoint{4.340613in}{0.243682in}}%
\pgfpathlineto{\pgfqpoint{4.341115in}{0.247956in}}%
\pgfpathlineto{\pgfqpoint{4.339331in}{0.249822in}}%
\pgfpathlineto{\pgfqpoint{4.344043in}{0.255220in}}%
\pgfpathlineto{\pgfqpoint{4.347135in}{0.256822in}}%
\pgfpathlineto{\pgfqpoint{4.348165in}{0.254717in}}%
\pgfpathlineto{\pgfqpoint{4.351391in}{0.256346in}}%
\pgfpathlineto{\pgfqpoint{4.353728in}{0.251922in}}%
\pgfpathlineto{\pgfqpoint{4.358783in}{0.246295in}}%
\pgfpathlineto{\pgfqpoint{4.358093in}{0.243746in}}%
\pgfpathlineto{\pgfqpoint{4.355286in}{0.240969in}}%
\pgfpathlineto{\pgfqpoint{4.352691in}{0.242277in}}%
\pgfpathlineto{\pgfqpoint{4.347266in}{0.250492in}}%
\pgfpathclose%
\pgfusepath{fill}%
\end{pgfscope}%
\begin{pgfscope}%
\pgfpathrectangle{\pgfqpoint{3.625000in}{0.100000in}}{\pgfqpoint{2.989028in}{1.913466in}}%
\pgfusepath{clip}%
\pgfsetbuttcap%
\pgfsetmiterjoin%
\definecolor{currentfill}{rgb}{0.312034,0.662668,0.687659}%
\pgfsetfillcolor{currentfill}%
\pgfsetlinewidth{0.000000pt}%
\definecolor{currentstroke}{rgb}{0.000000,0.000000,0.000000}%
\pgfsetstrokecolor{currentstroke}%
\pgfsetstrokeopacity{0.000000}%
\pgfsetdash{}{0pt}%
\pgfpathmoveto{\pgfqpoint{4.142161in}{0.351684in}}%
\pgfpathlineto{\pgfqpoint{4.137041in}{0.351018in}}%
\pgfpathlineto{\pgfqpoint{4.129830in}{0.348246in}}%
\pgfpathlineto{\pgfqpoint{4.128148in}{0.349752in}}%
\pgfpathlineto{\pgfqpoint{4.135680in}{0.351454in}}%
\pgfpathlineto{\pgfqpoint{4.133661in}{0.352939in}}%
\pgfpathlineto{\pgfqpoint{4.128991in}{0.354027in}}%
\pgfpathlineto{\pgfqpoint{4.127744in}{0.357507in}}%
\pgfpathlineto{\pgfqpoint{4.130094in}{0.360506in}}%
\pgfpathlineto{\pgfqpoint{4.132073in}{0.367387in}}%
\pgfpathlineto{\pgfqpoint{4.139301in}{0.368669in}}%
\pgfpathlineto{\pgfqpoint{4.142801in}{0.365696in}}%
\pgfpathlineto{\pgfqpoint{4.146088in}{0.368573in}}%
\pgfpathlineto{\pgfqpoint{4.150211in}{0.367867in}}%
\pgfpathlineto{\pgfqpoint{4.153174in}{0.362553in}}%
\pgfpathlineto{\pgfqpoint{4.155683in}{0.367277in}}%
\pgfpathlineto{\pgfqpoint{4.162804in}{0.356228in}}%
\pgfpathlineto{\pgfqpoint{4.160232in}{0.355345in}}%
\pgfpathlineto{\pgfqpoint{4.158653in}{0.352950in}}%
\pgfpathlineto{\pgfqpoint{4.161636in}{0.350764in}}%
\pgfpathlineto{\pgfqpoint{4.157011in}{0.348715in}}%
\pgfpathlineto{\pgfqpoint{4.153656in}{0.349843in}}%
\pgfpathlineto{\pgfqpoint{4.147321in}{0.353583in}}%
\pgfpathlineto{\pgfqpoint{4.142161in}{0.351684in}}%
\pgfpathclose%
\pgfusepath{fill}%
\end{pgfscope}%
\begin{pgfscope}%
\pgfpathrectangle{\pgfqpoint{3.625000in}{0.100000in}}{\pgfqpoint{2.989028in}{1.913466in}}%
\pgfusepath{clip}%
\pgfsetbuttcap%
\pgfsetmiterjoin%
\definecolor{currentfill}{rgb}{0.312034,0.662668,0.687659}%
\pgfsetfillcolor{currentfill}%
\pgfsetlinewidth{0.000000pt}%
\definecolor{currentstroke}{rgb}{0.000000,0.000000,0.000000}%
\pgfsetstrokecolor{currentstroke}%
\pgfsetstrokeopacity{0.000000}%
\pgfsetdash{}{0pt}%
\pgfpathmoveto{\pgfqpoint{4.342967in}{0.203589in}}%
\pgfpathlineto{\pgfqpoint{4.341952in}{0.213793in}}%
\pgfpathlineto{\pgfqpoint{4.344688in}{0.217539in}}%
\pgfpathlineto{\pgfqpoint{4.341762in}{0.217800in}}%
\pgfpathlineto{\pgfqpoint{4.342832in}{0.225086in}}%
\pgfpathlineto{\pgfqpoint{4.344093in}{0.229412in}}%
\pgfpathlineto{\pgfqpoint{4.342997in}{0.235224in}}%
\pgfpathlineto{\pgfqpoint{4.345677in}{0.236349in}}%
\pgfpathlineto{\pgfqpoint{4.346551in}{0.237852in}}%
\pgfpathlineto{\pgfqpoint{4.349191in}{0.237282in}}%
\pgfpathlineto{\pgfqpoint{4.351476in}{0.232626in}}%
\pgfpathlineto{\pgfqpoint{4.351901in}{0.228360in}}%
\pgfpathlineto{\pgfqpoint{4.349105in}{0.215583in}}%
\pgfpathlineto{\pgfqpoint{4.342967in}{0.203589in}}%
\pgfpathclose%
\pgfusepath{fill}%
\end{pgfscope}%
\begin{pgfscope}%
\pgfpathrectangle{\pgfqpoint{3.625000in}{0.100000in}}{\pgfqpoint{2.989028in}{1.913466in}}%
\pgfusepath{clip}%
\pgfsetbuttcap%
\pgfsetmiterjoin%
\definecolor{currentfill}{rgb}{0.312034,0.662668,0.687659}%
\pgfsetfillcolor{currentfill}%
\pgfsetlinewidth{0.000000pt}%
\definecolor{currentstroke}{rgb}{0.000000,0.000000,0.000000}%
\pgfsetstrokecolor{currentstroke}%
\pgfsetstrokeopacity{0.000000}%
\pgfsetdash{}{0pt}%
\pgfpathmoveto{\pgfqpoint{4.342360in}{0.234737in}}%
\pgfpathlineto{\pgfqpoint{4.342219in}{0.229762in}}%
\pgfpathlineto{\pgfqpoint{4.339925in}{0.227503in}}%
\pgfpathlineto{\pgfqpoint{4.337273in}{0.228338in}}%
\pgfpathlineto{\pgfqpoint{4.340625in}{0.235565in}}%
\pgfpathlineto{\pgfqpoint{4.342360in}{0.234737in}}%
\pgfpathclose%
\pgfusepath{fill}%
\end{pgfscope}%
\begin{pgfscope}%
\pgfpathrectangle{\pgfqpoint{3.625000in}{0.100000in}}{\pgfqpoint{2.989028in}{1.913466in}}%
\pgfusepath{clip}%
\pgfsetbuttcap%
\pgfsetmiterjoin%
\definecolor{currentfill}{rgb}{0.312034,0.662668,0.687659}%
\pgfsetfillcolor{currentfill}%
\pgfsetlinewidth{0.000000pt}%
\definecolor{currentstroke}{rgb}{0.000000,0.000000,0.000000}%
\pgfsetstrokecolor{currentstroke}%
\pgfsetstrokeopacity{0.000000}%
\pgfsetdash{}{0pt}%
\pgfpathmoveto{\pgfqpoint{4.370513in}{0.212648in}}%
\pgfpathlineto{\pgfqpoint{4.368557in}{0.204322in}}%
\pgfpathlineto{\pgfqpoint{4.363998in}{0.201690in}}%
\pgfpathlineto{\pgfqpoint{4.358017in}{0.204115in}}%
\pgfpathlineto{\pgfqpoint{4.359665in}{0.207350in}}%
\pgfpathlineto{\pgfqpoint{4.359632in}{0.209591in}}%
\pgfpathlineto{\pgfqpoint{4.361855in}{0.213398in}}%
\pgfpathlineto{\pgfqpoint{4.360292in}{0.212632in}}%
\pgfpathlineto{\pgfqpoint{4.361432in}{0.214491in}}%
\pgfpathlineto{\pgfqpoint{4.359267in}{0.218715in}}%
\pgfpathlineto{\pgfqpoint{4.361723in}{0.219493in}}%
\pgfpathlineto{\pgfqpoint{4.367410in}{0.214551in}}%
\pgfpathlineto{\pgfqpoint{4.370513in}{0.212648in}}%
\pgfpathclose%
\pgfusepath{fill}%
\end{pgfscope}%
\begin{pgfscope}%
\pgfpathrectangle{\pgfqpoint{3.625000in}{0.100000in}}{\pgfqpoint{2.989028in}{1.913466in}}%
\pgfusepath{clip}%
\pgfsetbuttcap%
\pgfsetmiterjoin%
\definecolor{currentfill}{rgb}{0.312034,0.662668,0.687659}%
\pgfsetfillcolor{currentfill}%
\pgfsetlinewidth{0.000000pt}%
\definecolor{currentstroke}{rgb}{0.000000,0.000000,0.000000}%
\pgfsetstrokecolor{currentstroke}%
\pgfsetstrokeopacity{0.000000}%
\pgfsetdash{}{0pt}%
\pgfpathmoveto{\pgfqpoint{4.350557in}{0.197926in}}%
\pgfpathlineto{\pgfqpoint{4.347979in}{0.196923in}}%
\pgfpathlineto{\pgfqpoint{4.349451in}{0.205709in}}%
\pgfpathlineto{\pgfqpoint{4.352546in}{0.207760in}}%
\pgfpathlineto{\pgfqpoint{4.351791in}{0.214343in}}%
\pgfpathlineto{\pgfqpoint{4.353049in}{0.217262in}}%
\pgfpathlineto{\pgfqpoint{4.355536in}{0.218356in}}%
\pgfpathlineto{\pgfqpoint{4.358148in}{0.215493in}}%
\pgfpathlineto{\pgfqpoint{4.360389in}{0.211607in}}%
\pgfpathlineto{\pgfqpoint{4.353184in}{0.203995in}}%
\pgfpathlineto{\pgfqpoint{4.350557in}{0.197926in}}%
\pgfpathclose%
\pgfusepath{fill}%
\end{pgfscope}%
\begin{pgfscope}%
\pgfpathrectangle{\pgfqpoint{3.625000in}{0.100000in}}{\pgfqpoint{2.989028in}{1.913466in}}%
\pgfusepath{clip}%
\pgfsetbuttcap%
\pgfsetmiterjoin%
\definecolor{currentfill}{rgb}{0.312034,0.662668,0.687659}%
\pgfsetfillcolor{currentfill}%
\pgfsetlinewidth{0.000000pt}%
\definecolor{currentstroke}{rgb}{0.000000,0.000000,0.000000}%
\pgfsetstrokecolor{currentstroke}%
\pgfsetstrokeopacity{0.000000}%
\pgfsetdash{}{0pt}%
\pgfpathmoveto{\pgfqpoint{4.371657in}{0.207083in}}%
\pgfpathlineto{\pgfqpoint{4.372921in}{0.201020in}}%
\pgfpathlineto{\pgfqpoint{4.367275in}{0.201714in}}%
\pgfpathlineto{\pgfqpoint{4.371657in}{0.207083in}}%
\pgfpathclose%
\pgfusepath{fill}%
\end{pgfscope}%
\begin{pgfscope}%
\pgfpathrectangle{\pgfqpoint{3.625000in}{0.100000in}}{\pgfqpoint{2.989028in}{1.913466in}}%
\pgfusepath{clip}%
\pgfsetbuttcap%
\pgfsetmiterjoin%
\definecolor{currentfill}{rgb}{0.312034,0.662668,0.687659}%
\pgfsetfillcolor{currentfill}%
\pgfsetlinewidth{0.000000pt}%
\definecolor{currentstroke}{rgb}{0.000000,0.000000,0.000000}%
\pgfsetstrokecolor{currentstroke}%
\pgfsetstrokeopacity{0.000000}%
\pgfsetdash{}{0pt}%
\pgfpathmoveto{\pgfqpoint{4.374351in}{0.197819in}}%
\pgfpathlineto{\pgfqpoint{4.375276in}{0.194079in}}%
\pgfpathlineto{\pgfqpoint{4.377047in}{0.192882in}}%
\pgfpathlineto{\pgfqpoint{4.376725in}{0.189362in}}%
\pgfpathlineto{\pgfqpoint{4.374251in}{0.188177in}}%
\pgfpathlineto{\pgfqpoint{4.372242in}{0.193433in}}%
\pgfpathlineto{\pgfqpoint{4.374351in}{0.197819in}}%
\pgfpathclose%
\pgfusepath{fill}%
\end{pgfscope}%
\begin{pgfscope}%
\pgfpathrectangle{\pgfqpoint{3.625000in}{0.100000in}}{\pgfqpoint{2.989028in}{1.913466in}}%
\pgfusepath{clip}%
\pgfsetbuttcap%
\pgfsetmiterjoin%
\definecolor{currentfill}{rgb}{0.312034,0.662668,0.687659}%
\pgfsetfillcolor{currentfill}%
\pgfsetlinewidth{0.000000pt}%
\definecolor{currentstroke}{rgb}{0.000000,0.000000,0.000000}%
\pgfsetstrokecolor{currentstroke}%
\pgfsetstrokeopacity{0.000000}%
\pgfsetdash{}{0pt}%
\pgfpathmoveto{\pgfqpoint{4.369640in}{0.199254in}}%
\pgfpathlineto{\pgfqpoint{4.368104in}{0.194909in}}%
\pgfpathlineto{\pgfqpoint{4.365719in}{0.195069in}}%
\pgfpathlineto{\pgfqpoint{4.364215in}{0.198652in}}%
\pgfpathlineto{\pgfqpoint{4.366665in}{0.200349in}}%
\pgfpathlineto{\pgfqpoint{4.369640in}{0.199254in}}%
\pgfpathclose%
\pgfusepath{fill}%
\end{pgfscope}%
\begin{pgfscope}%
\pgfpathrectangle{\pgfqpoint{3.625000in}{0.100000in}}{\pgfqpoint{2.989028in}{1.913466in}}%
\pgfusepath{clip}%
\pgfsetbuttcap%
\pgfsetmiterjoin%
\definecolor{currentfill}{rgb}{0.312034,0.662668,0.687659}%
\pgfsetfillcolor{currentfill}%
\pgfsetlinewidth{0.000000pt}%
\definecolor{currentstroke}{rgb}{0.000000,0.000000,0.000000}%
\pgfsetstrokecolor{currentstroke}%
\pgfsetstrokeopacity{0.000000}%
\pgfsetdash{}{0pt}%
\pgfpathmoveto{\pgfqpoint{4.363205in}{0.176854in}}%
\pgfpathlineto{\pgfqpoint{4.365333in}{0.174527in}}%
\pgfpathlineto{\pgfqpoint{4.365802in}{0.169622in}}%
\pgfpathlineto{\pgfqpoint{4.366922in}{0.168197in}}%
\pgfpathlineto{\pgfqpoint{4.365092in}{0.163758in}}%
\pgfpathlineto{\pgfqpoint{4.362686in}{0.161260in}}%
\pgfpathlineto{\pgfqpoint{4.361405in}{0.156623in}}%
\pgfpathlineto{\pgfqpoint{4.358774in}{0.163848in}}%
\pgfpathlineto{\pgfqpoint{4.358338in}{0.170408in}}%
\pgfpathlineto{\pgfqpoint{4.356865in}{0.173673in}}%
\pgfpathlineto{\pgfqpoint{4.351795in}{0.176321in}}%
\pgfpathlineto{\pgfqpoint{4.356551in}{0.181841in}}%
\pgfpathlineto{\pgfqpoint{4.353380in}{0.184201in}}%
\pgfpathlineto{\pgfqpoint{4.356174in}{0.186552in}}%
\pgfpathlineto{\pgfqpoint{4.359882in}{0.195579in}}%
\pgfpathlineto{\pgfqpoint{4.356038in}{0.198762in}}%
\pgfpathlineto{\pgfqpoint{4.357519in}{0.201818in}}%
\pgfpathlineto{\pgfqpoint{4.362487in}{0.199107in}}%
\pgfpathlineto{\pgfqpoint{4.362990in}{0.194829in}}%
\pgfpathlineto{\pgfqpoint{4.361111in}{0.192394in}}%
\pgfpathlineto{\pgfqpoint{4.364836in}{0.189098in}}%
\pgfpathlineto{\pgfqpoint{4.366009in}{0.183965in}}%
\pgfpathlineto{\pgfqpoint{4.365850in}{0.176765in}}%
\pgfpathlineto{\pgfqpoint{4.363205in}{0.176854in}}%
\pgfpathclose%
\pgfusepath{fill}%
\end{pgfscope}%
\begin{pgfscope}%
\pgfpathrectangle{\pgfqpoint{3.625000in}{0.100000in}}{\pgfqpoint{2.989028in}{1.913466in}}%
\pgfusepath{clip}%
\pgfsetbuttcap%
\pgfsetmiterjoin%
\definecolor{currentfill}{rgb}{0.312034,0.662668,0.687659}%
\pgfsetfillcolor{currentfill}%
\pgfsetlinewidth{0.000000pt}%
\definecolor{currentstroke}{rgb}{0.000000,0.000000,0.000000}%
\pgfsetstrokecolor{currentstroke}%
\pgfsetstrokeopacity{0.000000}%
\pgfsetdash{}{0pt}%
\pgfpathmoveto{\pgfqpoint{4.372302in}{0.194926in}}%
\pgfpathlineto{\pgfqpoint{4.371143in}{0.192429in}}%
\pgfpathlineto{\pgfqpoint{4.372231in}{0.188374in}}%
\pgfpathlineto{\pgfqpoint{4.369445in}{0.188033in}}%
\pgfpathlineto{\pgfqpoint{4.366524in}{0.189911in}}%
\pgfpathlineto{\pgfqpoint{4.367472in}{0.194046in}}%
\pgfpathlineto{\pgfqpoint{4.372302in}{0.194926in}}%
\pgfpathclose%
\pgfusepath{fill}%
\end{pgfscope}%
\begin{pgfscope}%
\pgfpathrectangle{\pgfqpoint{3.625000in}{0.100000in}}{\pgfqpoint{2.989028in}{1.913466in}}%
\pgfusepath{clip}%
\pgfsetbuttcap%
\pgfsetmiterjoin%
\definecolor{currentfill}{rgb}{0.312034,0.662668,0.687659}%
\pgfsetfillcolor{currentfill}%
\pgfsetlinewidth{0.000000pt}%
\definecolor{currentstroke}{rgb}{0.000000,0.000000,0.000000}%
\pgfsetstrokecolor{currentstroke}%
\pgfsetstrokeopacity{0.000000}%
\pgfsetdash{}{0pt}%
\pgfpathmoveto{\pgfqpoint{4.359316in}{0.195209in}}%
\pgfpathlineto{\pgfqpoint{4.353167in}{0.192451in}}%
\pgfpathlineto{\pgfqpoint{4.351076in}{0.193444in}}%
\pgfpathlineto{\pgfqpoint{4.355270in}{0.197082in}}%
\pgfpathlineto{\pgfqpoint{4.359316in}{0.195209in}}%
\pgfpathclose%
\pgfusepath{fill}%
\end{pgfscope}%
\begin{pgfscope}%
\pgfpathrectangle{\pgfqpoint{3.625000in}{0.100000in}}{\pgfqpoint{2.989028in}{1.913466in}}%
\pgfusepath{clip}%
\pgfsetbuttcap%
\pgfsetmiterjoin%
\definecolor{currentfill}{rgb}{0.312034,0.662668,0.687659}%
\pgfsetfillcolor{currentfill}%
\pgfsetlinewidth{0.000000pt}%
\definecolor{currentstroke}{rgb}{0.000000,0.000000,0.000000}%
\pgfsetstrokecolor{currentstroke}%
\pgfsetstrokeopacity{0.000000}%
\pgfsetdash{}{0pt}%
\pgfpathmoveto{\pgfqpoint{4.372369in}{0.168302in}}%
\pgfpathlineto{\pgfqpoint{4.370903in}{0.172455in}}%
\pgfpathlineto{\pgfqpoint{4.374004in}{0.173477in}}%
\pgfpathlineto{\pgfqpoint{4.374943in}{0.177955in}}%
\pgfpathlineto{\pgfqpoint{4.378303in}{0.177960in}}%
\pgfpathlineto{\pgfqpoint{4.378289in}{0.181079in}}%
\pgfpathlineto{\pgfqpoint{4.382959in}{0.180341in}}%
\pgfpathlineto{\pgfqpoint{4.383755in}{0.171887in}}%
\pgfpathlineto{\pgfqpoint{4.381046in}{0.166498in}}%
\pgfpathlineto{\pgfqpoint{4.377027in}{0.163131in}}%
\pgfpathlineto{\pgfqpoint{4.372369in}{0.168302in}}%
\pgfpathclose%
\pgfusepath{fill}%
\end{pgfscope}%
\begin{pgfscope}%
\pgfpathrectangle{\pgfqpoint{3.625000in}{0.100000in}}{\pgfqpoint{2.989028in}{1.913466in}}%
\pgfusepath{clip}%
\pgfsetbuttcap%
\pgfsetmiterjoin%
\definecolor{currentfill}{rgb}{0.312034,0.662668,0.687659}%
\pgfsetfillcolor{currentfill}%
\pgfsetlinewidth{0.000000pt}%
\definecolor{currentstroke}{rgb}{0.000000,0.000000,0.000000}%
\pgfsetstrokecolor{currentstroke}%
\pgfsetstrokeopacity{0.000000}%
\pgfsetdash{}{0pt}%
\pgfpathmoveto{\pgfqpoint{4.370553in}{0.171496in}}%
\pgfpathlineto{\pgfqpoint{4.371878in}{0.167564in}}%
\pgfpathlineto{\pgfqpoint{4.368612in}{0.167086in}}%
\pgfpathlineto{\pgfqpoint{4.370553in}{0.171496in}}%
\pgfpathclose%
\pgfusepath{fill}%
\end{pgfscope}%
\begin{pgfscope}%
\pgfpathrectangle{\pgfqpoint{3.625000in}{0.100000in}}{\pgfqpoint{2.989028in}{1.913466in}}%
\pgfusepath{clip}%
\pgfsetbuttcap%
\pgfsetmiterjoin%
\definecolor{currentfill}{rgb}{0.312034,0.662668,0.687659}%
\pgfsetfillcolor{currentfill}%
\pgfsetlinewidth{0.000000pt}%
\definecolor{currentstroke}{rgb}{0.000000,0.000000,0.000000}%
\pgfsetstrokecolor{currentstroke}%
\pgfsetstrokeopacity{0.000000}%
\pgfsetdash{}{0pt}%
\pgfpathmoveto{\pgfqpoint{4.373306in}{0.165713in}}%
\pgfpathlineto{\pgfqpoint{4.372851in}{0.160700in}}%
\pgfpathlineto{\pgfqpoint{4.369406in}{0.161116in}}%
\pgfpathlineto{\pgfqpoint{4.370520in}{0.163537in}}%
\pgfpathlineto{\pgfqpoint{4.373306in}{0.165713in}}%
\pgfpathclose%
\pgfusepath{fill}%
\end{pgfscope}%
\begin{pgfscope}%
\pgfsetbuttcap%
\pgfsetmiterjoin%
\definecolor{currentfill}{rgb}{1.000000,1.000000,1.000000}%
\pgfsetfillcolor{currentfill}%
\pgfsetlinewidth{1.003750pt}%
\definecolor{currentstroke}{rgb}{0.827451,0.827451,0.827451}%
\pgfsetstrokecolor{currentstroke}%
\pgfsetdash{}{0pt}%
\pgfpathmoveto{\pgfqpoint{4.540239in}{1.931644in}}%
\pgfpathlineto{\pgfqpoint{6.189614in}{1.931644in}}%
\pgfpathlineto{\pgfqpoint{6.189614in}{2.127644in}}%
\pgfpathlineto{\pgfqpoint{4.540239in}{2.127644in}}%
\pgfpathlineto{\pgfqpoint{4.540239in}{1.931644in}}%
\pgfpathclose%
\pgfusepath{stroke,fill}%
\end{pgfscope}%
\begin{pgfscope}%
\definecolor{textcolor}{rgb}{0.000000,0.000000,0.000000}%
\pgfsetstrokecolor{textcolor}%
\pgfsetfillcolor{textcolor}%
\pgftext[x=4.581489in,y=1.994331in,left,base]{\color{textcolor}\setmainfont{Lato}\rmfamily\fontsize{9.000000}{10.800000}\selectfont One-Year Growth, 2023 Q2*}%
\end{pgfscope}%
\begin{pgfscope}%
\pgfpathrectangle{\pgfqpoint{3.019583in}{0.169444in}}{\pgfqpoint{0.896708in}{1.339426in}}%
\pgfusepath{clip}%
\pgfsetbuttcap%
\pgfsetmiterjoin%
\definecolor{currentfill}{rgb}{0.619608,0.003922,0.258824}%
\pgfsetfillcolor{currentfill}%
\pgfsetlinewidth{0.000000pt}%
\definecolor{currentstroke}{rgb}{0.000000,0.000000,0.000000}%
\pgfsetstrokecolor{currentstroke}%
\pgfsetstrokeopacity{0.000000}%
\pgfsetdash{}{0pt}%
\pgfpathmoveto{\pgfqpoint{3.279629in}{0.169444in}}%
\pgfpathlineto{\pgfqpoint{3.243760in}{0.169444in}}%
\pgfpathlineto{\pgfqpoint{3.243760in}{0.190373in}}%
\pgfpathlineto{\pgfqpoint{3.279629in}{0.190373in}}%
\pgfpathlineto{\pgfqpoint{3.279629in}{0.169444in}}%
\pgfpathclose%
\pgfusepath{fill}%
\end{pgfscope}%
\begin{pgfscope}%
\pgfpathrectangle{\pgfqpoint{3.019583in}{0.169444in}}{\pgfqpoint{0.896708in}{1.339426in}}%
\pgfusepath{clip}%
\pgfsetbuttcap%
\pgfsetmiterjoin%
\definecolor{currentfill}{rgb}{0.653441,0.041446,0.266820}%
\pgfsetfillcolor{currentfill}%
\pgfsetlinewidth{0.000000pt}%
\definecolor{currentstroke}{rgb}{0.000000,0.000000,0.000000}%
\pgfsetstrokecolor{currentstroke}%
\pgfsetstrokeopacity{0.000000}%
\pgfsetdash{}{0pt}%
\pgfpathmoveto{\pgfqpoint{3.279629in}{0.190373in}}%
\pgfpathlineto{\pgfqpoint{3.243760in}{0.190373in}}%
\pgfpathlineto{\pgfqpoint{3.243760in}{0.211302in}}%
\pgfpathlineto{\pgfqpoint{3.279629in}{0.211302in}}%
\pgfpathlineto{\pgfqpoint{3.279629in}{0.190373in}}%
\pgfpathclose%
\pgfusepath{fill}%
\end{pgfscope}%
\begin{pgfscope}%
\pgfpathrectangle{\pgfqpoint{3.019583in}{0.169444in}}{\pgfqpoint{0.896708in}{1.339426in}}%
\pgfusepath{clip}%
\pgfsetbuttcap%
\pgfsetmiterjoin%
\definecolor{currentfill}{rgb}{0.687274,0.078970,0.274817}%
\pgfsetfillcolor{currentfill}%
\pgfsetlinewidth{0.000000pt}%
\definecolor{currentstroke}{rgb}{0.000000,0.000000,0.000000}%
\pgfsetstrokecolor{currentstroke}%
\pgfsetstrokeopacity{0.000000}%
\pgfsetdash{}{0pt}%
\pgfpathmoveto{\pgfqpoint{3.279629in}{0.211302in}}%
\pgfpathlineto{\pgfqpoint{3.243760in}{0.211302in}}%
\pgfpathlineto{\pgfqpoint{3.243760in}{0.232230in}}%
\pgfpathlineto{\pgfqpoint{3.279629in}{0.232230in}}%
\pgfpathlineto{\pgfqpoint{3.279629in}{0.211302in}}%
\pgfpathclose%
\pgfusepath{fill}%
\end{pgfscope}%
\begin{pgfscope}%
\pgfpathrectangle{\pgfqpoint{3.019583in}{0.169444in}}{\pgfqpoint{0.896708in}{1.339426in}}%
\pgfusepath{clip}%
\pgfsetbuttcap%
\pgfsetmiterjoin%
\definecolor{currentfill}{rgb}{0.721107,0.116494,0.282814}%
\pgfsetfillcolor{currentfill}%
\pgfsetlinewidth{0.000000pt}%
\definecolor{currentstroke}{rgb}{0.000000,0.000000,0.000000}%
\pgfsetstrokecolor{currentstroke}%
\pgfsetstrokeopacity{0.000000}%
\pgfsetdash{}{0pt}%
\pgfpathmoveto{\pgfqpoint{3.279629in}{0.232230in}}%
\pgfpathlineto{\pgfqpoint{3.243760in}{0.232230in}}%
\pgfpathlineto{\pgfqpoint{3.243760in}{0.253159in}}%
\pgfpathlineto{\pgfqpoint{3.279629in}{0.253159in}}%
\pgfpathlineto{\pgfqpoint{3.279629in}{0.232230in}}%
\pgfpathclose%
\pgfusepath{fill}%
\end{pgfscope}%
\begin{pgfscope}%
\pgfpathrectangle{\pgfqpoint{3.019583in}{0.169444in}}{\pgfqpoint{0.896708in}{1.339426in}}%
\pgfusepath{clip}%
\pgfsetbuttcap%
\pgfsetmiterjoin%
\definecolor{currentfill}{rgb}{0.754940,0.154018,0.290811}%
\pgfsetfillcolor{currentfill}%
\pgfsetlinewidth{0.000000pt}%
\definecolor{currentstroke}{rgb}{0.000000,0.000000,0.000000}%
\pgfsetstrokecolor{currentstroke}%
\pgfsetstrokeopacity{0.000000}%
\pgfsetdash{}{0pt}%
\pgfpathmoveto{\pgfqpoint{3.279629in}{0.253159in}}%
\pgfpathlineto{\pgfqpoint{3.243760in}{0.253159in}}%
\pgfpathlineto{\pgfqpoint{3.243760in}{0.274087in}}%
\pgfpathlineto{\pgfqpoint{3.279629in}{0.274087in}}%
\pgfpathlineto{\pgfqpoint{3.279629in}{0.253159in}}%
\pgfpathclose%
\pgfusepath{fill}%
\end{pgfscope}%
\begin{pgfscope}%
\pgfpathrectangle{\pgfqpoint{3.019583in}{0.169444in}}{\pgfqpoint{0.896708in}{1.339426in}}%
\pgfusepath{clip}%
\pgfsetbuttcap%
\pgfsetmiterjoin%
\definecolor{currentfill}{rgb}{0.788774,0.191542,0.298808}%
\pgfsetfillcolor{currentfill}%
\pgfsetlinewidth{0.000000pt}%
\definecolor{currentstroke}{rgb}{0.000000,0.000000,0.000000}%
\pgfsetstrokecolor{currentstroke}%
\pgfsetstrokeopacity{0.000000}%
\pgfsetdash{}{0pt}%
\pgfpathmoveto{\pgfqpoint{3.279629in}{0.274087in}}%
\pgfpathlineto{\pgfqpoint{3.243760in}{0.274087in}}%
\pgfpathlineto{\pgfqpoint{3.243760in}{0.295016in}}%
\pgfpathlineto{\pgfqpoint{3.279629in}{0.295016in}}%
\pgfpathlineto{\pgfqpoint{3.279629in}{0.274087in}}%
\pgfpathclose%
\pgfusepath{fill}%
\end{pgfscope}%
\begin{pgfscope}%
\pgfpathrectangle{\pgfqpoint{3.019583in}{0.169444in}}{\pgfqpoint{0.896708in}{1.339426in}}%
\pgfusepath{clip}%
\pgfsetbuttcap%
\pgfsetmiterjoin%
\definecolor{currentfill}{rgb}{0.822607,0.229066,0.306805}%
\pgfsetfillcolor{currentfill}%
\pgfsetlinewidth{0.000000pt}%
\definecolor{currentstroke}{rgb}{0.000000,0.000000,0.000000}%
\pgfsetstrokecolor{currentstroke}%
\pgfsetstrokeopacity{0.000000}%
\pgfsetdash{}{0pt}%
\pgfpathmoveto{\pgfqpoint{3.279629in}{0.295016in}}%
\pgfpathlineto{\pgfqpoint{3.243760in}{0.295016in}}%
\pgfpathlineto{\pgfqpoint{3.243760in}{0.315944in}}%
\pgfpathlineto{\pgfqpoint{3.279629in}{0.315944in}}%
\pgfpathlineto{\pgfqpoint{3.279629in}{0.295016in}}%
\pgfpathclose%
\pgfusepath{fill}%
\end{pgfscope}%
\begin{pgfscope}%
\pgfpathrectangle{\pgfqpoint{3.019583in}{0.169444in}}{\pgfqpoint{0.896708in}{1.339426in}}%
\pgfusepath{clip}%
\pgfsetbuttcap%
\pgfsetmiterjoin%
\definecolor{currentfill}{rgb}{0.847213,0.261207,0.305190}%
\pgfsetfillcolor{currentfill}%
\pgfsetlinewidth{0.000000pt}%
\definecolor{currentstroke}{rgb}{0.000000,0.000000,0.000000}%
\pgfsetstrokecolor{currentstroke}%
\pgfsetstrokeopacity{0.000000}%
\pgfsetdash{}{0pt}%
\pgfpathmoveto{\pgfqpoint{3.279629in}{0.315944in}}%
\pgfpathlineto{\pgfqpoint{3.243760in}{0.315944in}}%
\pgfpathlineto{\pgfqpoint{3.243760in}{0.336873in}}%
\pgfpathlineto{\pgfqpoint{3.279629in}{0.336873in}}%
\pgfpathlineto{\pgfqpoint{3.279629in}{0.315944in}}%
\pgfpathclose%
\pgfusepath{fill}%
\end{pgfscope}%
\begin{pgfscope}%
\pgfpathrectangle{\pgfqpoint{3.019583in}{0.169444in}}{\pgfqpoint{0.896708in}{1.339426in}}%
\pgfusepath{clip}%
\pgfsetbuttcap%
\pgfsetmiterjoin%
\definecolor{currentfill}{rgb}{0.866282,0.290119,0.297809}%
\pgfsetfillcolor{currentfill}%
\pgfsetlinewidth{0.000000pt}%
\definecolor{currentstroke}{rgb}{0.000000,0.000000,0.000000}%
\pgfsetstrokecolor{currentstroke}%
\pgfsetstrokeopacity{0.000000}%
\pgfsetdash{}{0pt}%
\pgfpathmoveto{\pgfqpoint{3.279629in}{0.336873in}}%
\pgfpathlineto{\pgfqpoint{3.243760in}{0.336873in}}%
\pgfpathlineto{\pgfqpoint{3.243760in}{0.357801in}}%
\pgfpathlineto{\pgfqpoint{3.279629in}{0.357801in}}%
\pgfpathlineto{\pgfqpoint{3.279629in}{0.336873in}}%
\pgfpathclose%
\pgfusepath{fill}%
\end{pgfscope}%
\begin{pgfscope}%
\pgfpathrectangle{\pgfqpoint{3.019583in}{0.169444in}}{\pgfqpoint{0.896708in}{1.339426in}}%
\pgfusepath{clip}%
\pgfsetbuttcap%
\pgfsetmiterjoin%
\definecolor{currentfill}{rgb}{0.885352,0.319031,0.290427}%
\pgfsetfillcolor{currentfill}%
\pgfsetlinewidth{0.000000pt}%
\definecolor{currentstroke}{rgb}{0.000000,0.000000,0.000000}%
\pgfsetstrokecolor{currentstroke}%
\pgfsetstrokeopacity{0.000000}%
\pgfsetdash{}{0pt}%
\pgfpathmoveto{\pgfqpoint{3.279629in}{0.357801in}}%
\pgfpathlineto{\pgfqpoint{3.243760in}{0.357801in}}%
\pgfpathlineto{\pgfqpoint{3.243760in}{0.378730in}}%
\pgfpathlineto{\pgfqpoint{3.279629in}{0.378730in}}%
\pgfpathlineto{\pgfqpoint{3.279629in}{0.357801in}}%
\pgfpathclose%
\pgfusepath{fill}%
\end{pgfscope}%
\begin{pgfscope}%
\pgfpathrectangle{\pgfqpoint{3.019583in}{0.169444in}}{\pgfqpoint{0.896708in}{1.339426in}}%
\pgfusepath{clip}%
\pgfsetbuttcap%
\pgfsetmiterjoin%
\definecolor{currentfill}{rgb}{0.904421,0.347943,0.283045}%
\pgfsetfillcolor{currentfill}%
\pgfsetlinewidth{0.000000pt}%
\definecolor{currentstroke}{rgb}{0.000000,0.000000,0.000000}%
\pgfsetstrokecolor{currentstroke}%
\pgfsetstrokeopacity{0.000000}%
\pgfsetdash{}{0pt}%
\pgfpathmoveto{\pgfqpoint{3.279629in}{0.378730in}}%
\pgfpathlineto{\pgfqpoint{3.243760in}{0.378730in}}%
\pgfpathlineto{\pgfqpoint{3.243760in}{0.399658in}}%
\pgfpathlineto{\pgfqpoint{3.279629in}{0.399658in}}%
\pgfpathlineto{\pgfqpoint{3.279629in}{0.378730in}}%
\pgfpathclose%
\pgfusepath{fill}%
\end{pgfscope}%
\begin{pgfscope}%
\pgfpathrectangle{\pgfqpoint{3.019583in}{0.169444in}}{\pgfqpoint{0.896708in}{1.339426in}}%
\pgfusepath{clip}%
\pgfsetbuttcap%
\pgfsetmiterjoin%
\definecolor{currentfill}{rgb}{0.923491,0.376855,0.275663}%
\pgfsetfillcolor{currentfill}%
\pgfsetlinewidth{0.000000pt}%
\definecolor{currentstroke}{rgb}{0.000000,0.000000,0.000000}%
\pgfsetstrokecolor{currentstroke}%
\pgfsetstrokeopacity{0.000000}%
\pgfsetdash{}{0pt}%
\pgfpathmoveto{\pgfqpoint{3.279629in}{0.399658in}}%
\pgfpathlineto{\pgfqpoint{3.243760in}{0.399658in}}%
\pgfpathlineto{\pgfqpoint{3.243760in}{0.420587in}}%
\pgfpathlineto{\pgfqpoint{3.279629in}{0.420587in}}%
\pgfpathlineto{\pgfqpoint{3.279629in}{0.399658in}}%
\pgfpathclose%
\pgfusepath{fill}%
\end{pgfscope}%
\begin{pgfscope}%
\pgfpathrectangle{\pgfqpoint{3.019583in}{0.169444in}}{\pgfqpoint{0.896708in}{1.339426in}}%
\pgfusepath{clip}%
\pgfsetbuttcap%
\pgfsetmiterjoin%
\definecolor{currentfill}{rgb}{0.942561,0.405767,0.268281}%
\pgfsetfillcolor{currentfill}%
\pgfsetlinewidth{0.000000pt}%
\definecolor{currentstroke}{rgb}{0.000000,0.000000,0.000000}%
\pgfsetstrokecolor{currentstroke}%
\pgfsetstrokeopacity{0.000000}%
\pgfsetdash{}{0pt}%
\pgfpathmoveto{\pgfqpoint{3.279629in}{0.420587in}}%
\pgfpathlineto{\pgfqpoint{3.243760in}{0.420587in}}%
\pgfpathlineto{\pgfqpoint{3.243760in}{0.441515in}}%
\pgfpathlineto{\pgfqpoint{3.279629in}{0.441515in}}%
\pgfpathlineto{\pgfqpoint{3.279629in}{0.420587in}}%
\pgfpathclose%
\pgfusepath{fill}%
\end{pgfscope}%
\begin{pgfscope}%
\pgfpathrectangle{\pgfqpoint{3.019583in}{0.169444in}}{\pgfqpoint{0.896708in}{1.339426in}}%
\pgfusepath{clip}%
\pgfsetbuttcap%
\pgfsetmiterjoin%
\definecolor{currentfill}{rgb}{0.958247,0.437447,0.267359}%
\pgfsetfillcolor{currentfill}%
\pgfsetlinewidth{0.000000pt}%
\definecolor{currentstroke}{rgb}{0.000000,0.000000,0.000000}%
\pgfsetstrokecolor{currentstroke}%
\pgfsetstrokeopacity{0.000000}%
\pgfsetdash{}{0pt}%
\pgfpathmoveto{\pgfqpoint{3.279629in}{0.441515in}}%
\pgfpathlineto{\pgfqpoint{3.243760in}{0.441515in}}%
\pgfpathlineto{\pgfqpoint{3.243760in}{0.462444in}}%
\pgfpathlineto{\pgfqpoint{3.279629in}{0.462444in}}%
\pgfpathlineto{\pgfqpoint{3.279629in}{0.441515in}}%
\pgfpathclose%
\pgfusepath{fill}%
\end{pgfscope}%
\begin{pgfscope}%
\pgfpathrectangle{\pgfqpoint{3.019583in}{0.169444in}}{\pgfqpoint{0.896708in}{1.339426in}}%
\pgfusepath{clip}%
\pgfsetbuttcap%
\pgfsetmiterjoin%
\definecolor{currentfill}{rgb}{0.963783,0.477432,0.285813}%
\pgfsetfillcolor{currentfill}%
\pgfsetlinewidth{0.000000pt}%
\definecolor{currentstroke}{rgb}{0.000000,0.000000,0.000000}%
\pgfsetstrokecolor{currentstroke}%
\pgfsetstrokeopacity{0.000000}%
\pgfsetdash{}{0pt}%
\pgfpathmoveto{\pgfqpoint{3.279629in}{0.462444in}}%
\pgfpathlineto{\pgfqpoint{3.243760in}{0.462444in}}%
\pgfpathlineto{\pgfqpoint{3.243760in}{0.483372in}}%
\pgfpathlineto{\pgfqpoint{3.279629in}{0.483372in}}%
\pgfpathlineto{\pgfqpoint{3.279629in}{0.462444in}}%
\pgfpathclose%
\pgfusepath{fill}%
\end{pgfscope}%
\begin{pgfscope}%
\pgfpathrectangle{\pgfqpoint{3.019583in}{0.169444in}}{\pgfqpoint{0.896708in}{1.339426in}}%
\pgfusepath{clip}%
\pgfsetbuttcap%
\pgfsetmiterjoin%
\definecolor{currentfill}{rgb}{0.969319,0.517416,0.304268}%
\pgfsetfillcolor{currentfill}%
\pgfsetlinewidth{0.000000pt}%
\definecolor{currentstroke}{rgb}{0.000000,0.000000,0.000000}%
\pgfsetstrokecolor{currentstroke}%
\pgfsetstrokeopacity{0.000000}%
\pgfsetdash{}{0pt}%
\pgfpathmoveto{\pgfqpoint{3.279629in}{0.483372in}}%
\pgfpathlineto{\pgfqpoint{3.243760in}{0.483372in}}%
\pgfpathlineto{\pgfqpoint{3.243760in}{0.504301in}}%
\pgfpathlineto{\pgfqpoint{3.279629in}{0.504301in}}%
\pgfpathlineto{\pgfqpoint{3.279629in}{0.483372in}}%
\pgfpathclose%
\pgfusepath{fill}%
\end{pgfscope}%
\begin{pgfscope}%
\pgfpathrectangle{\pgfqpoint{3.019583in}{0.169444in}}{\pgfqpoint{0.896708in}{1.339426in}}%
\pgfusepath{clip}%
\pgfsetbuttcap%
\pgfsetmiterjoin%
\definecolor{currentfill}{rgb}{0.974856,0.557401,0.322722}%
\pgfsetfillcolor{currentfill}%
\pgfsetlinewidth{0.000000pt}%
\definecolor{currentstroke}{rgb}{0.000000,0.000000,0.000000}%
\pgfsetstrokecolor{currentstroke}%
\pgfsetstrokeopacity{0.000000}%
\pgfsetdash{}{0pt}%
\pgfpathmoveto{\pgfqpoint{3.279629in}{0.504301in}}%
\pgfpathlineto{\pgfqpoint{3.243760in}{0.504301in}}%
\pgfpathlineto{\pgfqpoint{3.243760in}{0.525230in}}%
\pgfpathlineto{\pgfqpoint{3.279629in}{0.525230in}}%
\pgfpathlineto{\pgfqpoint{3.279629in}{0.504301in}}%
\pgfpathclose%
\pgfusepath{fill}%
\end{pgfscope}%
\begin{pgfscope}%
\pgfpathrectangle{\pgfqpoint{3.019583in}{0.169444in}}{\pgfqpoint{0.896708in}{1.339426in}}%
\pgfusepath{clip}%
\pgfsetbuttcap%
\pgfsetmiterjoin%
\definecolor{currentfill}{rgb}{0.980392,0.597386,0.341176}%
\pgfsetfillcolor{currentfill}%
\pgfsetlinewidth{0.000000pt}%
\definecolor{currentstroke}{rgb}{0.000000,0.000000,0.000000}%
\pgfsetstrokecolor{currentstroke}%
\pgfsetstrokeopacity{0.000000}%
\pgfsetdash{}{0pt}%
\pgfpathmoveto{\pgfqpoint{3.279629in}{0.525230in}}%
\pgfpathlineto{\pgfqpoint{3.243760in}{0.525230in}}%
\pgfpathlineto{\pgfqpoint{3.243760in}{0.546158in}}%
\pgfpathlineto{\pgfqpoint{3.279629in}{0.546158in}}%
\pgfpathlineto{\pgfqpoint{3.279629in}{0.525230in}}%
\pgfpathclose%
\pgfusepath{fill}%
\end{pgfscope}%
\begin{pgfscope}%
\pgfpathrectangle{\pgfqpoint{3.019583in}{0.169444in}}{\pgfqpoint{0.896708in}{1.339426in}}%
\pgfusepath{clip}%
\pgfsetbuttcap%
\pgfsetmiterjoin%
\definecolor{currentfill}{rgb}{0.985928,0.637370,0.359631}%
\pgfsetfillcolor{currentfill}%
\pgfsetlinewidth{0.000000pt}%
\definecolor{currentstroke}{rgb}{0.000000,0.000000,0.000000}%
\pgfsetstrokecolor{currentstroke}%
\pgfsetstrokeopacity{0.000000}%
\pgfsetdash{}{0pt}%
\pgfpathmoveto{\pgfqpoint{3.279629in}{0.546158in}}%
\pgfpathlineto{\pgfqpoint{3.243760in}{0.546158in}}%
\pgfpathlineto{\pgfqpoint{3.243760in}{0.567087in}}%
\pgfpathlineto{\pgfqpoint{3.279629in}{0.567087in}}%
\pgfpathlineto{\pgfqpoint{3.279629in}{0.546158in}}%
\pgfpathclose%
\pgfusepath{fill}%
\end{pgfscope}%
\begin{pgfscope}%
\pgfpathrectangle{\pgfqpoint{3.019583in}{0.169444in}}{\pgfqpoint{0.896708in}{1.339426in}}%
\pgfusepath{clip}%
\pgfsetbuttcap%
\pgfsetmiterjoin%
\definecolor{currentfill}{rgb}{0.991465,0.677355,0.378085}%
\pgfsetfillcolor{currentfill}%
\pgfsetlinewidth{0.000000pt}%
\definecolor{currentstroke}{rgb}{0.000000,0.000000,0.000000}%
\pgfsetstrokecolor{currentstroke}%
\pgfsetstrokeopacity{0.000000}%
\pgfsetdash{}{0pt}%
\pgfpathmoveto{\pgfqpoint{3.279629in}{0.567087in}}%
\pgfpathlineto{\pgfqpoint{3.243760in}{0.567087in}}%
\pgfpathlineto{\pgfqpoint{3.243760in}{0.588015in}}%
\pgfpathlineto{\pgfqpoint{3.279629in}{0.588015in}}%
\pgfpathlineto{\pgfqpoint{3.279629in}{0.567087in}}%
\pgfpathclose%
\pgfusepath{fill}%
\end{pgfscope}%
\begin{pgfscope}%
\pgfpathrectangle{\pgfqpoint{3.019583in}{0.169444in}}{\pgfqpoint{0.896708in}{1.339426in}}%
\pgfusepath{clip}%
\pgfsetbuttcap%
\pgfsetmiterjoin%
\definecolor{currentfill}{rgb}{0.992695,0.709266,0.402999}%
\pgfsetfillcolor{currentfill}%
\pgfsetlinewidth{0.000000pt}%
\definecolor{currentstroke}{rgb}{0.000000,0.000000,0.000000}%
\pgfsetstrokecolor{currentstroke}%
\pgfsetstrokeopacity{0.000000}%
\pgfsetdash{}{0pt}%
\pgfpathmoveto{\pgfqpoint{3.279629in}{0.588015in}}%
\pgfpathlineto{\pgfqpoint{3.243760in}{0.588015in}}%
\pgfpathlineto{\pgfqpoint{3.243760in}{0.608944in}}%
\pgfpathlineto{\pgfqpoint{3.279629in}{0.608944in}}%
\pgfpathlineto{\pgfqpoint{3.279629in}{0.588015in}}%
\pgfpathclose%
\pgfusepath{fill}%
\end{pgfscope}%
\begin{pgfscope}%
\pgfpathrectangle{\pgfqpoint{3.019583in}{0.169444in}}{\pgfqpoint{0.896708in}{1.339426in}}%
\pgfusepath{clip}%
\pgfsetbuttcap%
\pgfsetmiterjoin%
\definecolor{currentfill}{rgb}{0.993310,0.740023,0.428835}%
\pgfsetfillcolor{currentfill}%
\pgfsetlinewidth{0.000000pt}%
\definecolor{currentstroke}{rgb}{0.000000,0.000000,0.000000}%
\pgfsetstrokecolor{currentstroke}%
\pgfsetstrokeopacity{0.000000}%
\pgfsetdash{}{0pt}%
\pgfpathmoveto{\pgfqpoint{3.279629in}{0.608944in}}%
\pgfpathlineto{\pgfqpoint{3.243760in}{0.608944in}}%
\pgfpathlineto{\pgfqpoint{3.243760in}{0.629872in}}%
\pgfpathlineto{\pgfqpoint{3.279629in}{0.629872in}}%
\pgfpathlineto{\pgfqpoint{3.279629in}{0.608944in}}%
\pgfpathclose%
\pgfusepath{fill}%
\end{pgfscope}%
\begin{pgfscope}%
\pgfpathrectangle{\pgfqpoint{3.019583in}{0.169444in}}{\pgfqpoint{0.896708in}{1.339426in}}%
\pgfusepath{clip}%
\pgfsetbuttcap%
\pgfsetmiterjoin%
\definecolor{currentfill}{rgb}{0.993925,0.770780,0.454671}%
\pgfsetfillcolor{currentfill}%
\pgfsetlinewidth{0.000000pt}%
\definecolor{currentstroke}{rgb}{0.000000,0.000000,0.000000}%
\pgfsetstrokecolor{currentstroke}%
\pgfsetstrokeopacity{0.000000}%
\pgfsetdash{}{0pt}%
\pgfpathmoveto{\pgfqpoint{3.279629in}{0.629872in}}%
\pgfpathlineto{\pgfqpoint{3.243760in}{0.629872in}}%
\pgfpathlineto{\pgfqpoint{3.243760in}{0.650801in}}%
\pgfpathlineto{\pgfqpoint{3.279629in}{0.650801in}}%
\pgfpathlineto{\pgfqpoint{3.279629in}{0.629872in}}%
\pgfpathclose%
\pgfusepath{fill}%
\end{pgfscope}%
\begin{pgfscope}%
\pgfpathrectangle{\pgfqpoint{3.019583in}{0.169444in}}{\pgfqpoint{0.896708in}{1.339426in}}%
\pgfusepath{clip}%
\pgfsetbuttcap%
\pgfsetmiterjoin%
\definecolor{currentfill}{rgb}{0.994541,0.801538,0.480507}%
\pgfsetfillcolor{currentfill}%
\pgfsetlinewidth{0.000000pt}%
\definecolor{currentstroke}{rgb}{0.000000,0.000000,0.000000}%
\pgfsetstrokecolor{currentstroke}%
\pgfsetstrokeopacity{0.000000}%
\pgfsetdash{}{0pt}%
\pgfpathmoveto{\pgfqpoint{3.279629in}{0.650801in}}%
\pgfpathlineto{\pgfqpoint{3.243760in}{0.650801in}}%
\pgfpathlineto{\pgfqpoint{3.243760in}{0.671729in}}%
\pgfpathlineto{\pgfqpoint{3.279629in}{0.671729in}}%
\pgfpathlineto{\pgfqpoint{3.279629in}{0.650801in}}%
\pgfpathclose%
\pgfusepath{fill}%
\end{pgfscope}%
\begin{pgfscope}%
\pgfpathrectangle{\pgfqpoint{3.019583in}{0.169444in}}{\pgfqpoint{0.896708in}{1.339426in}}%
\pgfusepath{clip}%
\pgfsetbuttcap%
\pgfsetmiterjoin%
\definecolor{currentfill}{rgb}{0.995156,0.832295,0.506344}%
\pgfsetfillcolor{currentfill}%
\pgfsetlinewidth{0.000000pt}%
\definecolor{currentstroke}{rgb}{0.000000,0.000000,0.000000}%
\pgfsetstrokecolor{currentstroke}%
\pgfsetstrokeopacity{0.000000}%
\pgfsetdash{}{0pt}%
\pgfpathmoveto{\pgfqpoint{3.279629in}{0.671729in}}%
\pgfpathlineto{\pgfqpoint{3.243760in}{0.671729in}}%
\pgfpathlineto{\pgfqpoint{3.243760in}{0.692658in}}%
\pgfpathlineto{\pgfqpoint{3.279629in}{0.692658in}}%
\pgfpathlineto{\pgfqpoint{3.279629in}{0.671729in}}%
\pgfpathclose%
\pgfusepath{fill}%
\end{pgfscope}%
\begin{pgfscope}%
\pgfpathrectangle{\pgfqpoint{3.019583in}{0.169444in}}{\pgfqpoint{0.896708in}{1.339426in}}%
\pgfusepath{clip}%
\pgfsetbuttcap%
\pgfsetmiterjoin%
\definecolor{currentfill}{rgb}{0.995771,0.863053,0.532180}%
\pgfsetfillcolor{currentfill}%
\pgfsetlinewidth{0.000000pt}%
\definecolor{currentstroke}{rgb}{0.000000,0.000000,0.000000}%
\pgfsetstrokecolor{currentstroke}%
\pgfsetstrokeopacity{0.000000}%
\pgfsetdash{}{0pt}%
\pgfpathmoveto{\pgfqpoint{3.279629in}{0.692658in}}%
\pgfpathlineto{\pgfqpoint{3.243760in}{0.692658in}}%
\pgfpathlineto{\pgfqpoint{3.243760in}{0.713586in}}%
\pgfpathlineto{\pgfqpoint{3.279629in}{0.713586in}}%
\pgfpathlineto{\pgfqpoint{3.279629in}{0.692658in}}%
\pgfpathclose%
\pgfusepath{fill}%
\end{pgfscope}%
\begin{pgfscope}%
\pgfpathrectangle{\pgfqpoint{3.019583in}{0.169444in}}{\pgfqpoint{0.896708in}{1.339426in}}%
\pgfusepath{clip}%
\pgfsetbuttcap%
\pgfsetmiterjoin%
\definecolor{currentfill}{rgb}{0.996386,0.887966,0.561092}%
\pgfsetfillcolor{currentfill}%
\pgfsetlinewidth{0.000000pt}%
\definecolor{currentstroke}{rgb}{0.000000,0.000000,0.000000}%
\pgfsetstrokecolor{currentstroke}%
\pgfsetstrokeopacity{0.000000}%
\pgfsetdash{}{0pt}%
\pgfpathmoveto{\pgfqpoint{3.279629in}{0.713586in}}%
\pgfpathlineto{\pgfqpoint{3.243760in}{0.713586in}}%
\pgfpathlineto{\pgfqpoint{3.243760in}{0.734515in}}%
\pgfpathlineto{\pgfqpoint{3.279629in}{0.734515in}}%
\pgfpathlineto{\pgfqpoint{3.279629in}{0.713586in}}%
\pgfpathclose%
\pgfusepath{fill}%
\end{pgfscope}%
\begin{pgfscope}%
\pgfpathrectangle{\pgfqpoint{3.019583in}{0.169444in}}{\pgfqpoint{0.896708in}{1.339426in}}%
\pgfusepath{clip}%
\pgfsetbuttcap%
\pgfsetmiterjoin%
\definecolor{currentfill}{rgb}{0.997001,0.907036,0.593080}%
\pgfsetfillcolor{currentfill}%
\pgfsetlinewidth{0.000000pt}%
\definecolor{currentstroke}{rgb}{0.000000,0.000000,0.000000}%
\pgfsetstrokecolor{currentstroke}%
\pgfsetstrokeopacity{0.000000}%
\pgfsetdash{}{0pt}%
\pgfpathmoveto{\pgfqpoint{3.279629in}{0.734515in}}%
\pgfpathlineto{\pgfqpoint{3.243760in}{0.734515in}}%
\pgfpathlineto{\pgfqpoint{3.243760in}{0.755443in}}%
\pgfpathlineto{\pgfqpoint{3.279629in}{0.755443in}}%
\pgfpathlineto{\pgfqpoint{3.279629in}{0.734515in}}%
\pgfpathclose%
\pgfusepath{fill}%
\end{pgfscope}%
\begin{pgfscope}%
\pgfpathrectangle{\pgfqpoint{3.019583in}{0.169444in}}{\pgfqpoint{0.896708in}{1.339426in}}%
\pgfusepath{clip}%
\pgfsetbuttcap%
\pgfsetmiterjoin%
\definecolor{currentfill}{rgb}{0.997616,0.926105,0.625067}%
\pgfsetfillcolor{currentfill}%
\pgfsetlinewidth{0.000000pt}%
\definecolor{currentstroke}{rgb}{0.000000,0.000000,0.000000}%
\pgfsetstrokecolor{currentstroke}%
\pgfsetstrokeopacity{0.000000}%
\pgfsetdash{}{0pt}%
\pgfpathmoveto{\pgfqpoint{3.279629in}{0.755443in}}%
\pgfpathlineto{\pgfqpoint{3.243760in}{0.755443in}}%
\pgfpathlineto{\pgfqpoint{3.243760in}{0.776372in}}%
\pgfpathlineto{\pgfqpoint{3.279629in}{0.776372in}}%
\pgfpathlineto{\pgfqpoint{3.279629in}{0.755443in}}%
\pgfpathclose%
\pgfusepath{fill}%
\end{pgfscope}%
\begin{pgfscope}%
\pgfpathrectangle{\pgfqpoint{3.019583in}{0.169444in}}{\pgfqpoint{0.896708in}{1.339426in}}%
\pgfusepath{clip}%
\pgfsetbuttcap%
\pgfsetmiterjoin%
\definecolor{currentfill}{rgb}{0.998231,0.945175,0.657055}%
\pgfsetfillcolor{currentfill}%
\pgfsetlinewidth{0.000000pt}%
\definecolor{currentstroke}{rgb}{0.000000,0.000000,0.000000}%
\pgfsetstrokecolor{currentstroke}%
\pgfsetstrokeopacity{0.000000}%
\pgfsetdash{}{0pt}%
\pgfpathmoveto{\pgfqpoint{3.279629in}{0.776372in}}%
\pgfpathlineto{\pgfqpoint{3.243760in}{0.776372in}}%
\pgfpathlineto{\pgfqpoint{3.243760in}{0.797300in}}%
\pgfpathlineto{\pgfqpoint{3.279629in}{0.797300in}}%
\pgfpathlineto{\pgfqpoint{3.279629in}{0.776372in}}%
\pgfpathclose%
\pgfusepath{fill}%
\end{pgfscope}%
\begin{pgfscope}%
\pgfpathrectangle{\pgfqpoint{3.019583in}{0.169444in}}{\pgfqpoint{0.896708in}{1.339426in}}%
\pgfusepath{clip}%
\pgfsetbuttcap%
\pgfsetmiterjoin%
\definecolor{currentfill}{rgb}{0.998847,0.964245,0.689043}%
\pgfsetfillcolor{currentfill}%
\pgfsetlinewidth{0.000000pt}%
\definecolor{currentstroke}{rgb}{0.000000,0.000000,0.000000}%
\pgfsetstrokecolor{currentstroke}%
\pgfsetstrokeopacity{0.000000}%
\pgfsetdash{}{0pt}%
\pgfpathmoveto{\pgfqpoint{3.279629in}{0.797300in}}%
\pgfpathlineto{\pgfqpoint{3.243760in}{0.797300in}}%
\pgfpathlineto{\pgfqpoint{3.243760in}{0.818229in}}%
\pgfpathlineto{\pgfqpoint{3.279629in}{0.818229in}}%
\pgfpathlineto{\pgfqpoint{3.279629in}{0.797300in}}%
\pgfpathclose%
\pgfusepath{fill}%
\end{pgfscope}%
\begin{pgfscope}%
\pgfpathrectangle{\pgfqpoint{3.019583in}{0.169444in}}{\pgfqpoint{0.896708in}{1.339426in}}%
\pgfusepath{clip}%
\pgfsetbuttcap%
\pgfsetmiterjoin%
\definecolor{currentfill}{rgb}{0.999462,0.983314,0.721030}%
\pgfsetfillcolor{currentfill}%
\pgfsetlinewidth{0.000000pt}%
\definecolor{currentstroke}{rgb}{0.000000,0.000000,0.000000}%
\pgfsetstrokecolor{currentstroke}%
\pgfsetstrokeopacity{0.000000}%
\pgfsetdash{}{0pt}%
\pgfpathmoveto{\pgfqpoint{3.279629in}{0.818229in}}%
\pgfpathlineto{\pgfqpoint{3.243760in}{0.818229in}}%
\pgfpathlineto{\pgfqpoint{3.243760in}{0.839157in}}%
\pgfpathlineto{\pgfqpoint{3.279629in}{0.839157in}}%
\pgfpathlineto{\pgfqpoint{3.279629in}{0.818229in}}%
\pgfpathclose%
\pgfusepath{fill}%
\end{pgfscope}%
\begin{pgfscope}%
\pgfpathrectangle{\pgfqpoint{3.019583in}{0.169444in}}{\pgfqpoint{0.896708in}{1.339426in}}%
\pgfusepath{clip}%
\pgfsetbuttcap%
\pgfsetmiterjoin%
\definecolor{currentfill}{rgb}{0.998078,0.999231,0.746021}%
\pgfsetfillcolor{currentfill}%
\pgfsetlinewidth{0.000000pt}%
\definecolor{currentstroke}{rgb}{0.000000,0.000000,0.000000}%
\pgfsetstrokecolor{currentstroke}%
\pgfsetstrokeopacity{0.000000}%
\pgfsetdash{}{0pt}%
\pgfpathmoveto{\pgfqpoint{3.279629in}{0.839157in}}%
\pgfpathlineto{\pgfqpoint{3.243760in}{0.839157in}}%
\pgfpathlineto{\pgfqpoint{3.243760in}{0.860086in}}%
\pgfpathlineto{\pgfqpoint{3.279629in}{0.860086in}}%
\pgfpathlineto{\pgfqpoint{3.279629in}{0.839157in}}%
\pgfpathclose%
\pgfusepath{fill}%
\end{pgfscope}%
\begin{pgfscope}%
\pgfpathrectangle{\pgfqpoint{3.019583in}{0.169444in}}{\pgfqpoint{0.896708in}{1.339426in}}%
\pgfusepath{clip}%
\pgfsetbuttcap%
\pgfsetmiterjoin%
\definecolor{currentfill}{rgb}{0.982699,0.993080,0.722030}%
\pgfsetfillcolor{currentfill}%
\pgfsetlinewidth{0.000000pt}%
\definecolor{currentstroke}{rgb}{0.000000,0.000000,0.000000}%
\pgfsetstrokecolor{currentstroke}%
\pgfsetstrokeopacity{0.000000}%
\pgfsetdash{}{0pt}%
\pgfpathmoveto{\pgfqpoint{3.279629in}{0.860086in}}%
\pgfpathlineto{\pgfqpoint{3.243760in}{0.860086in}}%
\pgfpathlineto{\pgfqpoint{3.243760in}{0.881015in}}%
\pgfpathlineto{\pgfqpoint{3.279629in}{0.881015in}}%
\pgfpathlineto{\pgfqpoint{3.279629in}{0.860086in}}%
\pgfpathclose%
\pgfusepath{fill}%
\end{pgfscope}%
\begin{pgfscope}%
\pgfpathrectangle{\pgfqpoint{3.019583in}{0.169444in}}{\pgfqpoint{0.896708in}{1.339426in}}%
\pgfusepath{clip}%
\pgfsetbuttcap%
\pgfsetmiterjoin%
\definecolor{currentfill}{rgb}{0.967320,0.986928,0.698039}%
\pgfsetfillcolor{currentfill}%
\pgfsetlinewidth{0.000000pt}%
\definecolor{currentstroke}{rgb}{0.000000,0.000000,0.000000}%
\pgfsetstrokecolor{currentstroke}%
\pgfsetstrokeopacity{0.000000}%
\pgfsetdash{}{0pt}%
\pgfpathmoveto{\pgfqpoint{3.279629in}{0.881015in}}%
\pgfpathlineto{\pgfqpoint{3.243760in}{0.881015in}}%
\pgfpathlineto{\pgfqpoint{3.243760in}{0.901943in}}%
\pgfpathlineto{\pgfqpoint{3.279629in}{0.901943in}}%
\pgfpathlineto{\pgfqpoint{3.279629in}{0.881015in}}%
\pgfpathclose%
\pgfusepath{fill}%
\end{pgfscope}%
\begin{pgfscope}%
\pgfpathrectangle{\pgfqpoint{3.019583in}{0.169444in}}{\pgfqpoint{0.896708in}{1.339426in}}%
\pgfusepath{clip}%
\pgfsetbuttcap%
\pgfsetmiterjoin%
\definecolor{currentfill}{rgb}{0.951942,0.980777,0.674048}%
\pgfsetfillcolor{currentfill}%
\pgfsetlinewidth{0.000000pt}%
\definecolor{currentstroke}{rgb}{0.000000,0.000000,0.000000}%
\pgfsetstrokecolor{currentstroke}%
\pgfsetstrokeopacity{0.000000}%
\pgfsetdash{}{0pt}%
\pgfpathmoveto{\pgfqpoint{3.279629in}{0.901943in}}%
\pgfpathlineto{\pgfqpoint{3.243760in}{0.901943in}}%
\pgfpathlineto{\pgfqpoint{3.243760in}{0.922872in}}%
\pgfpathlineto{\pgfqpoint{3.279629in}{0.922872in}}%
\pgfpathlineto{\pgfqpoint{3.279629in}{0.901943in}}%
\pgfpathclose%
\pgfusepath{fill}%
\end{pgfscope}%
\begin{pgfscope}%
\pgfpathrectangle{\pgfqpoint{3.019583in}{0.169444in}}{\pgfqpoint{0.896708in}{1.339426in}}%
\pgfusepath{clip}%
\pgfsetbuttcap%
\pgfsetmiterjoin%
\definecolor{currentfill}{rgb}{0.936563,0.974625,0.650058}%
\pgfsetfillcolor{currentfill}%
\pgfsetlinewidth{0.000000pt}%
\definecolor{currentstroke}{rgb}{0.000000,0.000000,0.000000}%
\pgfsetstrokecolor{currentstroke}%
\pgfsetstrokeopacity{0.000000}%
\pgfsetdash{}{0pt}%
\pgfpathmoveto{\pgfqpoint{3.279629in}{0.922872in}}%
\pgfpathlineto{\pgfqpoint{3.243760in}{0.922872in}}%
\pgfpathlineto{\pgfqpoint{3.243760in}{0.943800in}}%
\pgfpathlineto{\pgfqpoint{3.279629in}{0.943800in}}%
\pgfpathlineto{\pgfqpoint{3.279629in}{0.922872in}}%
\pgfpathclose%
\pgfusepath{fill}%
\end{pgfscope}%
\begin{pgfscope}%
\pgfpathrectangle{\pgfqpoint{3.019583in}{0.169444in}}{\pgfqpoint{0.896708in}{1.339426in}}%
\pgfusepath{clip}%
\pgfsetbuttcap%
\pgfsetmiterjoin%
\definecolor{currentfill}{rgb}{0.921184,0.968474,0.626067}%
\pgfsetfillcolor{currentfill}%
\pgfsetlinewidth{0.000000pt}%
\definecolor{currentstroke}{rgb}{0.000000,0.000000,0.000000}%
\pgfsetstrokecolor{currentstroke}%
\pgfsetstrokeopacity{0.000000}%
\pgfsetdash{}{0pt}%
\pgfpathmoveto{\pgfqpoint{3.279629in}{0.943800in}}%
\pgfpathlineto{\pgfqpoint{3.243760in}{0.943800in}}%
\pgfpathlineto{\pgfqpoint{3.243760in}{0.964729in}}%
\pgfpathlineto{\pgfqpoint{3.279629in}{0.964729in}}%
\pgfpathlineto{\pgfqpoint{3.279629in}{0.943800in}}%
\pgfpathclose%
\pgfusepath{fill}%
\end{pgfscope}%
\begin{pgfscope}%
\pgfpathrectangle{\pgfqpoint{3.019583in}{0.169444in}}{\pgfqpoint{0.896708in}{1.339426in}}%
\pgfusepath{clip}%
\pgfsetbuttcap%
\pgfsetmiterjoin%
\definecolor{currentfill}{rgb}{0.905805,0.962322,0.602076}%
\pgfsetfillcolor{currentfill}%
\pgfsetlinewidth{0.000000pt}%
\definecolor{currentstroke}{rgb}{0.000000,0.000000,0.000000}%
\pgfsetstrokecolor{currentstroke}%
\pgfsetstrokeopacity{0.000000}%
\pgfsetdash{}{0pt}%
\pgfpathmoveto{\pgfqpoint{3.279629in}{0.964729in}}%
\pgfpathlineto{\pgfqpoint{3.243760in}{0.964729in}}%
\pgfpathlineto{\pgfqpoint{3.243760in}{0.985657in}}%
\pgfpathlineto{\pgfqpoint{3.279629in}{0.985657in}}%
\pgfpathlineto{\pgfqpoint{3.279629in}{0.964729in}}%
\pgfpathclose%
\pgfusepath{fill}%
\end{pgfscope}%
\begin{pgfscope}%
\pgfpathrectangle{\pgfqpoint{3.019583in}{0.169444in}}{\pgfqpoint{0.896708in}{1.339426in}}%
\pgfusepath{clip}%
\pgfsetbuttcap%
\pgfsetmiterjoin%
\definecolor{currentfill}{rgb}{0.874740,0.949712,0.601615}%
\pgfsetfillcolor{currentfill}%
\pgfsetlinewidth{0.000000pt}%
\definecolor{currentstroke}{rgb}{0.000000,0.000000,0.000000}%
\pgfsetstrokecolor{currentstroke}%
\pgfsetstrokeopacity{0.000000}%
\pgfsetdash{}{0pt}%
\pgfpathmoveto{\pgfqpoint{3.279629in}{0.985657in}}%
\pgfpathlineto{\pgfqpoint{3.243760in}{0.985657in}}%
\pgfpathlineto{\pgfqpoint{3.243760in}{1.006586in}}%
\pgfpathlineto{\pgfqpoint{3.279629in}{1.006586in}}%
\pgfpathlineto{\pgfqpoint{3.279629in}{0.985657in}}%
\pgfpathclose%
\pgfusepath{fill}%
\end{pgfscope}%
\begin{pgfscope}%
\pgfpathrectangle{\pgfqpoint{3.019583in}{0.169444in}}{\pgfqpoint{0.896708in}{1.339426in}}%
\pgfusepath{clip}%
\pgfsetbuttcap%
\pgfsetmiterjoin%
\definecolor{currentfill}{rgb}{0.838447,0.934948,0.608997}%
\pgfsetfillcolor{currentfill}%
\pgfsetlinewidth{0.000000pt}%
\definecolor{currentstroke}{rgb}{0.000000,0.000000,0.000000}%
\pgfsetstrokecolor{currentstroke}%
\pgfsetstrokeopacity{0.000000}%
\pgfsetdash{}{0pt}%
\pgfpathmoveto{\pgfqpoint{3.279629in}{1.006586in}}%
\pgfpathlineto{\pgfqpoint{3.243760in}{1.006586in}}%
\pgfpathlineto{\pgfqpoint{3.243760in}{1.027514in}}%
\pgfpathlineto{\pgfqpoint{3.279629in}{1.027514in}}%
\pgfpathlineto{\pgfqpoint{3.279629in}{1.006586in}}%
\pgfpathclose%
\pgfusepath{fill}%
\end{pgfscope}%
\begin{pgfscope}%
\pgfpathrectangle{\pgfqpoint{3.019583in}{0.169444in}}{\pgfqpoint{0.896708in}{1.339426in}}%
\pgfusepath{clip}%
\pgfsetbuttcap%
\pgfsetmiterjoin%
\definecolor{currentfill}{rgb}{0.802153,0.920185,0.616378}%
\pgfsetfillcolor{currentfill}%
\pgfsetlinewidth{0.000000pt}%
\definecolor{currentstroke}{rgb}{0.000000,0.000000,0.000000}%
\pgfsetstrokecolor{currentstroke}%
\pgfsetstrokeopacity{0.000000}%
\pgfsetdash{}{0pt}%
\pgfpathmoveto{\pgfqpoint{3.279629in}{1.027514in}}%
\pgfpathlineto{\pgfqpoint{3.243760in}{1.027514in}}%
\pgfpathlineto{\pgfqpoint{3.243760in}{1.048443in}}%
\pgfpathlineto{\pgfqpoint{3.279629in}{1.048443in}}%
\pgfpathlineto{\pgfqpoint{3.279629in}{1.027514in}}%
\pgfpathclose%
\pgfusepath{fill}%
\end{pgfscope}%
\begin{pgfscope}%
\pgfpathrectangle{\pgfqpoint{3.019583in}{0.169444in}}{\pgfqpoint{0.896708in}{1.339426in}}%
\pgfusepath{clip}%
\pgfsetbuttcap%
\pgfsetmiterjoin%
\definecolor{currentfill}{rgb}{0.765859,0.905421,0.623760}%
\pgfsetfillcolor{currentfill}%
\pgfsetlinewidth{0.000000pt}%
\definecolor{currentstroke}{rgb}{0.000000,0.000000,0.000000}%
\pgfsetstrokecolor{currentstroke}%
\pgfsetstrokeopacity{0.000000}%
\pgfsetdash{}{0pt}%
\pgfpathmoveto{\pgfqpoint{3.279629in}{1.048443in}}%
\pgfpathlineto{\pgfqpoint{3.243760in}{1.048443in}}%
\pgfpathlineto{\pgfqpoint{3.243760in}{1.069371in}}%
\pgfpathlineto{\pgfqpoint{3.279629in}{1.069371in}}%
\pgfpathlineto{\pgfqpoint{3.279629in}{1.048443in}}%
\pgfpathclose%
\pgfusepath{fill}%
\end{pgfscope}%
\begin{pgfscope}%
\pgfpathrectangle{\pgfqpoint{3.019583in}{0.169444in}}{\pgfqpoint{0.896708in}{1.339426in}}%
\pgfusepath{clip}%
\pgfsetbuttcap%
\pgfsetmiterjoin%
\definecolor{currentfill}{rgb}{0.729566,0.890657,0.631142}%
\pgfsetfillcolor{currentfill}%
\pgfsetlinewidth{0.000000pt}%
\definecolor{currentstroke}{rgb}{0.000000,0.000000,0.000000}%
\pgfsetstrokecolor{currentstroke}%
\pgfsetstrokeopacity{0.000000}%
\pgfsetdash{}{0pt}%
\pgfpathmoveto{\pgfqpoint{3.279629in}{1.069371in}}%
\pgfpathlineto{\pgfqpoint{3.243760in}{1.069371in}}%
\pgfpathlineto{\pgfqpoint{3.243760in}{1.090300in}}%
\pgfpathlineto{\pgfqpoint{3.279629in}{1.090300in}}%
\pgfpathlineto{\pgfqpoint{3.279629in}{1.069371in}}%
\pgfpathclose%
\pgfusepath{fill}%
\end{pgfscope}%
\begin{pgfscope}%
\pgfpathrectangle{\pgfqpoint{3.019583in}{0.169444in}}{\pgfqpoint{0.896708in}{1.339426in}}%
\pgfusepath{clip}%
\pgfsetbuttcap%
\pgfsetmiterjoin%
\definecolor{currentfill}{rgb}{0.693272,0.875894,0.638524}%
\pgfsetfillcolor{currentfill}%
\pgfsetlinewidth{0.000000pt}%
\definecolor{currentstroke}{rgb}{0.000000,0.000000,0.000000}%
\pgfsetstrokecolor{currentstroke}%
\pgfsetstrokeopacity{0.000000}%
\pgfsetdash{}{0pt}%
\pgfpathmoveto{\pgfqpoint{3.279629in}{1.090300in}}%
\pgfpathlineto{\pgfqpoint{3.243760in}{1.090300in}}%
\pgfpathlineto{\pgfqpoint{3.243760in}{1.111228in}}%
\pgfpathlineto{\pgfqpoint{3.279629in}{1.111228in}}%
\pgfpathlineto{\pgfqpoint{3.279629in}{1.090300in}}%
\pgfpathclose%
\pgfusepath{fill}%
\end{pgfscope}%
\begin{pgfscope}%
\pgfpathrectangle{\pgfqpoint{3.019583in}{0.169444in}}{\pgfqpoint{0.896708in}{1.339426in}}%
\pgfusepath{clip}%
\pgfsetbuttcap%
\pgfsetmiterjoin%
\definecolor{currentfill}{rgb}{0.654671,0.860438,0.643368}%
\pgfsetfillcolor{currentfill}%
\pgfsetlinewidth{0.000000pt}%
\definecolor{currentstroke}{rgb}{0.000000,0.000000,0.000000}%
\pgfsetstrokecolor{currentstroke}%
\pgfsetstrokeopacity{0.000000}%
\pgfsetdash{}{0pt}%
\pgfpathmoveto{\pgfqpoint{3.279629in}{1.111228in}}%
\pgfpathlineto{\pgfqpoint{3.243760in}{1.111228in}}%
\pgfpathlineto{\pgfqpoint{3.243760in}{1.132157in}}%
\pgfpathlineto{\pgfqpoint{3.279629in}{1.132157in}}%
\pgfpathlineto{\pgfqpoint{3.279629in}{1.111228in}}%
\pgfpathclose%
\pgfusepath{fill}%
\end{pgfscope}%
\begin{pgfscope}%
\pgfpathrectangle{\pgfqpoint{3.019583in}{0.169444in}}{\pgfqpoint{0.896708in}{1.339426in}}%
\pgfusepath{clip}%
\pgfsetbuttcap%
\pgfsetmiterjoin%
\definecolor{currentfill}{rgb}{0.612226,0.843829,0.643983}%
\pgfsetfillcolor{currentfill}%
\pgfsetlinewidth{0.000000pt}%
\definecolor{currentstroke}{rgb}{0.000000,0.000000,0.000000}%
\pgfsetstrokecolor{currentstroke}%
\pgfsetstrokeopacity{0.000000}%
\pgfsetdash{}{0pt}%
\pgfpathmoveto{\pgfqpoint{3.279629in}{1.132157in}}%
\pgfpathlineto{\pgfqpoint{3.243760in}{1.132157in}}%
\pgfpathlineto{\pgfqpoint{3.243760in}{1.153085in}}%
\pgfpathlineto{\pgfqpoint{3.279629in}{1.153085in}}%
\pgfpathlineto{\pgfqpoint{3.279629in}{1.132157in}}%
\pgfpathclose%
\pgfusepath{fill}%
\end{pgfscope}%
\begin{pgfscope}%
\pgfpathrectangle{\pgfqpoint{3.019583in}{0.169444in}}{\pgfqpoint{0.896708in}{1.339426in}}%
\pgfusepath{clip}%
\pgfsetbuttcap%
\pgfsetmiterjoin%
\definecolor{currentfill}{rgb}{0.569781,0.827220,0.644598}%
\pgfsetfillcolor{currentfill}%
\pgfsetlinewidth{0.000000pt}%
\definecolor{currentstroke}{rgb}{0.000000,0.000000,0.000000}%
\pgfsetstrokecolor{currentstroke}%
\pgfsetstrokeopacity{0.000000}%
\pgfsetdash{}{0pt}%
\pgfpathmoveto{\pgfqpoint{3.279629in}{1.153085in}}%
\pgfpathlineto{\pgfqpoint{3.243760in}{1.153085in}}%
\pgfpathlineto{\pgfqpoint{3.243760in}{1.174014in}}%
\pgfpathlineto{\pgfqpoint{3.279629in}{1.174014in}}%
\pgfpathlineto{\pgfqpoint{3.279629in}{1.153085in}}%
\pgfpathclose%
\pgfusepath{fill}%
\end{pgfscope}%
\begin{pgfscope}%
\pgfpathrectangle{\pgfqpoint{3.019583in}{0.169444in}}{\pgfqpoint{0.896708in}{1.339426in}}%
\pgfusepath{clip}%
\pgfsetbuttcap%
\pgfsetmiterjoin%
\definecolor{currentfill}{rgb}{0.527336,0.810611,0.645213}%
\pgfsetfillcolor{currentfill}%
\pgfsetlinewidth{0.000000pt}%
\definecolor{currentstroke}{rgb}{0.000000,0.000000,0.000000}%
\pgfsetstrokecolor{currentstroke}%
\pgfsetstrokeopacity{0.000000}%
\pgfsetdash{}{0pt}%
\pgfpathmoveto{\pgfqpoint{3.279629in}{1.174014in}}%
\pgfpathlineto{\pgfqpoint{3.243760in}{1.174014in}}%
\pgfpathlineto{\pgfqpoint{3.243760in}{1.194943in}}%
\pgfpathlineto{\pgfqpoint{3.279629in}{1.194943in}}%
\pgfpathlineto{\pgfqpoint{3.279629in}{1.174014in}}%
\pgfpathclose%
\pgfusepath{fill}%
\end{pgfscope}%
\begin{pgfscope}%
\pgfpathrectangle{\pgfqpoint{3.019583in}{0.169444in}}{\pgfqpoint{0.896708in}{1.339426in}}%
\pgfusepath{clip}%
\pgfsetbuttcap%
\pgfsetmiterjoin%
\definecolor{currentfill}{rgb}{0.484890,0.794002,0.645829}%
\pgfsetfillcolor{currentfill}%
\pgfsetlinewidth{0.000000pt}%
\definecolor{currentstroke}{rgb}{0.000000,0.000000,0.000000}%
\pgfsetstrokecolor{currentstroke}%
\pgfsetstrokeopacity{0.000000}%
\pgfsetdash{}{0pt}%
\pgfpathmoveto{\pgfqpoint{3.279629in}{1.194943in}}%
\pgfpathlineto{\pgfqpoint{3.243760in}{1.194943in}}%
\pgfpathlineto{\pgfqpoint{3.243760in}{1.215871in}}%
\pgfpathlineto{\pgfqpoint{3.279629in}{1.215871in}}%
\pgfpathlineto{\pgfqpoint{3.279629in}{1.194943in}}%
\pgfpathclose%
\pgfusepath{fill}%
\end{pgfscope}%
\begin{pgfscope}%
\pgfpathrectangle{\pgfqpoint{3.019583in}{0.169444in}}{\pgfqpoint{0.896708in}{1.339426in}}%
\pgfusepath{clip}%
\pgfsetbuttcap%
\pgfsetmiterjoin%
\definecolor{currentfill}{rgb}{0.442445,0.777393,0.646444}%
\pgfsetfillcolor{currentfill}%
\pgfsetlinewidth{0.000000pt}%
\definecolor{currentstroke}{rgb}{0.000000,0.000000,0.000000}%
\pgfsetstrokecolor{currentstroke}%
\pgfsetstrokeopacity{0.000000}%
\pgfsetdash{}{0pt}%
\pgfpathmoveto{\pgfqpoint{3.279629in}{1.215871in}}%
\pgfpathlineto{\pgfqpoint{3.243760in}{1.215871in}}%
\pgfpathlineto{\pgfqpoint{3.243760in}{1.236800in}}%
\pgfpathlineto{\pgfqpoint{3.279629in}{1.236800in}}%
\pgfpathlineto{\pgfqpoint{3.279629in}{1.215871in}}%
\pgfpathclose%
\pgfusepath{fill}%
\end{pgfscope}%
\begin{pgfscope}%
\pgfpathrectangle{\pgfqpoint{3.019583in}{0.169444in}}{\pgfqpoint{0.896708in}{1.339426in}}%
\pgfusepath{clip}%
\pgfsetbuttcap%
\pgfsetmiterjoin%
\definecolor{currentfill}{rgb}{0.400000,0.760784,0.647059}%
\pgfsetfillcolor{currentfill}%
\pgfsetlinewidth{0.000000pt}%
\definecolor{currentstroke}{rgb}{0.000000,0.000000,0.000000}%
\pgfsetstrokecolor{currentstroke}%
\pgfsetstrokeopacity{0.000000}%
\pgfsetdash{}{0pt}%
\pgfpathmoveto{\pgfqpoint{3.279629in}{1.236800in}}%
\pgfpathlineto{\pgfqpoint{3.243760in}{1.236800in}}%
\pgfpathlineto{\pgfqpoint{3.243760in}{1.257728in}}%
\pgfpathlineto{\pgfqpoint{3.279629in}{1.257728in}}%
\pgfpathlineto{\pgfqpoint{3.279629in}{1.236800in}}%
\pgfpathclose%
\pgfusepath{fill}%
\end{pgfscope}%
\begin{pgfscope}%
\pgfpathrectangle{\pgfqpoint{3.019583in}{0.169444in}}{\pgfqpoint{0.896708in}{1.339426in}}%
\pgfusepath{clip}%
\pgfsetbuttcap%
\pgfsetmiterjoin%
\definecolor{currentfill}{rgb}{0.368012,0.725106,0.661822}%
\pgfsetfillcolor{currentfill}%
\pgfsetlinewidth{0.000000pt}%
\definecolor{currentstroke}{rgb}{0.000000,0.000000,0.000000}%
\pgfsetstrokecolor{currentstroke}%
\pgfsetstrokeopacity{0.000000}%
\pgfsetdash{}{0pt}%
\pgfpathmoveto{\pgfqpoint{3.279629in}{1.257728in}}%
\pgfpathlineto{\pgfqpoint{3.243760in}{1.257728in}}%
\pgfpathlineto{\pgfqpoint{3.243760in}{1.278657in}}%
\pgfpathlineto{\pgfqpoint{3.279629in}{1.278657in}}%
\pgfpathlineto{\pgfqpoint{3.279629in}{1.257728in}}%
\pgfpathclose%
\pgfusepath{fill}%
\end{pgfscope}%
\begin{pgfscope}%
\pgfpathrectangle{\pgfqpoint{3.019583in}{0.169444in}}{\pgfqpoint{0.896708in}{1.339426in}}%
\pgfusepath{clip}%
\pgfsetbuttcap%
\pgfsetmiterjoin%
\definecolor{currentfill}{rgb}{0.336025,0.689427,0.676586}%
\pgfsetfillcolor{currentfill}%
\pgfsetlinewidth{0.000000pt}%
\definecolor{currentstroke}{rgb}{0.000000,0.000000,0.000000}%
\pgfsetstrokecolor{currentstroke}%
\pgfsetstrokeopacity{0.000000}%
\pgfsetdash{}{0pt}%
\pgfpathmoveto{\pgfqpoint{3.279629in}{1.278657in}}%
\pgfpathlineto{\pgfqpoint{3.243760in}{1.278657in}}%
\pgfpathlineto{\pgfqpoint{3.243760in}{1.299585in}}%
\pgfpathlineto{\pgfqpoint{3.279629in}{1.299585in}}%
\pgfpathlineto{\pgfqpoint{3.279629in}{1.278657in}}%
\pgfpathclose%
\pgfusepath{fill}%
\end{pgfscope}%
\begin{pgfscope}%
\pgfpathrectangle{\pgfqpoint{3.019583in}{0.169444in}}{\pgfqpoint{0.896708in}{1.339426in}}%
\pgfusepath{clip}%
\pgfsetbuttcap%
\pgfsetmiterjoin%
\definecolor{currentfill}{rgb}{0.304037,0.653749,0.691349}%
\pgfsetfillcolor{currentfill}%
\pgfsetlinewidth{0.000000pt}%
\definecolor{currentstroke}{rgb}{0.000000,0.000000,0.000000}%
\pgfsetstrokecolor{currentstroke}%
\pgfsetstrokeopacity{0.000000}%
\pgfsetdash{}{0pt}%
\pgfpathmoveto{\pgfqpoint{3.279629in}{1.299585in}}%
\pgfpathlineto{\pgfqpoint{3.243760in}{1.299585in}}%
\pgfpathlineto{\pgfqpoint{3.243760in}{1.320514in}}%
\pgfpathlineto{\pgfqpoint{3.279629in}{1.320514in}}%
\pgfpathlineto{\pgfqpoint{3.279629in}{1.299585in}}%
\pgfpathclose%
\pgfusepath{fill}%
\end{pgfscope}%
\begin{pgfscope}%
\pgfpathrectangle{\pgfqpoint{3.019583in}{0.169444in}}{\pgfqpoint{0.896708in}{1.339426in}}%
\pgfusepath{clip}%
\pgfsetbuttcap%
\pgfsetmiterjoin%
\definecolor{currentfill}{rgb}{0.272049,0.618070,0.706113}%
\pgfsetfillcolor{currentfill}%
\pgfsetlinewidth{0.000000pt}%
\definecolor{currentstroke}{rgb}{0.000000,0.000000,0.000000}%
\pgfsetstrokecolor{currentstroke}%
\pgfsetstrokeopacity{0.000000}%
\pgfsetdash{}{0pt}%
\pgfpathmoveto{\pgfqpoint{3.279629in}{1.320514in}}%
\pgfpathlineto{\pgfqpoint{3.243760in}{1.320514in}}%
\pgfpathlineto{\pgfqpoint{3.243760in}{1.341442in}}%
\pgfpathlineto{\pgfqpoint{3.279629in}{1.341442in}}%
\pgfpathlineto{\pgfqpoint{3.279629in}{1.320514in}}%
\pgfpathclose%
\pgfusepath{fill}%
\end{pgfscope}%
\begin{pgfscope}%
\pgfpathrectangle{\pgfqpoint{3.019583in}{0.169444in}}{\pgfqpoint{0.896708in}{1.339426in}}%
\pgfusepath{clip}%
\pgfsetbuttcap%
\pgfsetmiterjoin%
\definecolor{currentfill}{rgb}{0.240062,0.582391,0.720877}%
\pgfsetfillcolor{currentfill}%
\pgfsetlinewidth{0.000000pt}%
\definecolor{currentstroke}{rgb}{0.000000,0.000000,0.000000}%
\pgfsetstrokecolor{currentstroke}%
\pgfsetstrokeopacity{0.000000}%
\pgfsetdash{}{0pt}%
\pgfpathmoveto{\pgfqpoint{3.279629in}{1.341442in}}%
\pgfpathlineto{\pgfqpoint{3.243760in}{1.341442in}}%
\pgfpathlineto{\pgfqpoint{3.243760in}{1.362371in}}%
\pgfpathlineto{\pgfqpoint{3.279629in}{1.362371in}}%
\pgfpathlineto{\pgfqpoint{3.279629in}{1.341442in}}%
\pgfpathclose%
\pgfusepath{fill}%
\end{pgfscope}%
\begin{pgfscope}%
\pgfpathrectangle{\pgfqpoint{3.019583in}{0.169444in}}{\pgfqpoint{0.896708in}{1.339426in}}%
\pgfusepath{clip}%
\pgfsetbuttcap%
\pgfsetmiterjoin%
\definecolor{currentfill}{rgb}{0.208074,0.546713,0.735640}%
\pgfsetfillcolor{currentfill}%
\pgfsetlinewidth{0.000000pt}%
\definecolor{currentstroke}{rgb}{0.000000,0.000000,0.000000}%
\pgfsetstrokecolor{currentstroke}%
\pgfsetstrokeopacity{0.000000}%
\pgfsetdash{}{0pt}%
\pgfpathmoveto{\pgfqpoint{3.279629in}{1.362371in}}%
\pgfpathlineto{\pgfqpoint{3.243760in}{1.362371in}}%
\pgfpathlineto{\pgfqpoint{3.243760in}{1.383299in}}%
\pgfpathlineto{\pgfqpoint{3.279629in}{1.383299in}}%
\pgfpathlineto{\pgfqpoint{3.279629in}{1.362371in}}%
\pgfpathclose%
\pgfusepath{fill}%
\end{pgfscope}%
\begin{pgfscope}%
\pgfpathrectangle{\pgfqpoint{3.019583in}{0.169444in}}{\pgfqpoint{0.896708in}{1.339426in}}%
\pgfusepath{clip}%
\pgfsetbuttcap%
\pgfsetmiterjoin%
\definecolor{currentfill}{rgb}{0.212995,0.511419,0.730796}%
\pgfsetfillcolor{currentfill}%
\pgfsetlinewidth{0.000000pt}%
\definecolor{currentstroke}{rgb}{0.000000,0.000000,0.000000}%
\pgfsetstrokecolor{currentstroke}%
\pgfsetstrokeopacity{0.000000}%
\pgfsetdash{}{0pt}%
\pgfpathmoveto{\pgfqpoint{3.279629in}{1.383299in}}%
\pgfpathlineto{\pgfqpoint{3.243760in}{1.383299in}}%
\pgfpathlineto{\pgfqpoint{3.243760in}{1.404228in}}%
\pgfpathlineto{\pgfqpoint{3.279629in}{1.404228in}}%
\pgfpathlineto{\pgfqpoint{3.279629in}{1.383299in}}%
\pgfpathclose%
\pgfusepath{fill}%
\end{pgfscope}%
\begin{pgfscope}%
\pgfpathrectangle{\pgfqpoint{3.019583in}{0.169444in}}{\pgfqpoint{0.896708in}{1.339426in}}%
\pgfusepath{clip}%
\pgfsetbuttcap%
\pgfsetmiterjoin%
\definecolor{currentfill}{rgb}{0.240062,0.476355,0.714187}%
\pgfsetfillcolor{currentfill}%
\pgfsetlinewidth{0.000000pt}%
\definecolor{currentstroke}{rgb}{0.000000,0.000000,0.000000}%
\pgfsetstrokecolor{currentstroke}%
\pgfsetstrokeopacity{0.000000}%
\pgfsetdash{}{0pt}%
\pgfpathmoveto{\pgfqpoint{3.279629in}{1.404228in}}%
\pgfpathlineto{\pgfqpoint{3.243760in}{1.404228in}}%
\pgfpathlineto{\pgfqpoint{3.243760in}{1.425156in}}%
\pgfpathlineto{\pgfqpoint{3.279629in}{1.425156in}}%
\pgfpathlineto{\pgfqpoint{3.279629in}{1.404228in}}%
\pgfpathclose%
\pgfusepath{fill}%
\end{pgfscope}%
\begin{pgfscope}%
\pgfpathrectangle{\pgfqpoint{3.019583in}{0.169444in}}{\pgfqpoint{0.896708in}{1.339426in}}%
\pgfusepath{clip}%
\pgfsetbuttcap%
\pgfsetmiterjoin%
\definecolor{currentfill}{rgb}{0.267128,0.441292,0.697578}%
\pgfsetfillcolor{currentfill}%
\pgfsetlinewidth{0.000000pt}%
\definecolor{currentstroke}{rgb}{0.000000,0.000000,0.000000}%
\pgfsetstrokecolor{currentstroke}%
\pgfsetstrokeopacity{0.000000}%
\pgfsetdash{}{0pt}%
\pgfpathmoveto{\pgfqpoint{3.279629in}{1.425156in}}%
\pgfpathlineto{\pgfqpoint{3.243760in}{1.425156in}}%
\pgfpathlineto{\pgfqpoint{3.243760in}{1.446085in}}%
\pgfpathlineto{\pgfqpoint{3.279629in}{1.446085in}}%
\pgfpathlineto{\pgfqpoint{3.279629in}{1.425156in}}%
\pgfpathclose%
\pgfusepath{fill}%
\end{pgfscope}%
\begin{pgfscope}%
\pgfpathrectangle{\pgfqpoint{3.019583in}{0.169444in}}{\pgfqpoint{0.896708in}{1.339426in}}%
\pgfusepath{clip}%
\pgfsetbuttcap%
\pgfsetmiterjoin%
\definecolor{currentfill}{rgb}{0.294195,0.406228,0.680969}%
\pgfsetfillcolor{currentfill}%
\pgfsetlinewidth{0.000000pt}%
\definecolor{currentstroke}{rgb}{0.000000,0.000000,0.000000}%
\pgfsetstrokecolor{currentstroke}%
\pgfsetstrokeopacity{0.000000}%
\pgfsetdash{}{0pt}%
\pgfpathmoveto{\pgfqpoint{3.279629in}{1.446085in}}%
\pgfpathlineto{\pgfqpoint{3.243760in}{1.446085in}}%
\pgfpathlineto{\pgfqpoint{3.243760in}{1.467013in}}%
\pgfpathlineto{\pgfqpoint{3.279629in}{1.467013in}}%
\pgfpathlineto{\pgfqpoint{3.279629in}{1.446085in}}%
\pgfpathclose%
\pgfusepath{fill}%
\end{pgfscope}%
\begin{pgfscope}%
\pgfpathrectangle{\pgfqpoint{3.019583in}{0.169444in}}{\pgfqpoint{0.896708in}{1.339426in}}%
\pgfusepath{clip}%
\pgfsetbuttcap%
\pgfsetmiterjoin%
\definecolor{currentfill}{rgb}{0.321261,0.371165,0.664360}%
\pgfsetfillcolor{currentfill}%
\pgfsetlinewidth{0.000000pt}%
\definecolor{currentstroke}{rgb}{0.000000,0.000000,0.000000}%
\pgfsetstrokecolor{currentstroke}%
\pgfsetstrokeopacity{0.000000}%
\pgfsetdash{}{0pt}%
\pgfpathmoveto{\pgfqpoint{3.279629in}{1.467013in}}%
\pgfpathlineto{\pgfqpoint{3.243760in}{1.467013in}}%
\pgfpathlineto{\pgfqpoint{3.243760in}{1.487942in}}%
\pgfpathlineto{\pgfqpoint{3.279629in}{1.487942in}}%
\pgfpathlineto{\pgfqpoint{3.279629in}{1.467013in}}%
\pgfpathclose%
\pgfusepath{fill}%
\end{pgfscope}%
\begin{pgfscope}%
\pgfpathrectangle{\pgfqpoint{3.019583in}{0.169444in}}{\pgfqpoint{0.896708in}{1.339426in}}%
\pgfusepath{clip}%
\pgfsetbuttcap%
\pgfsetmiterjoin%
\definecolor{currentfill}{rgb}{0.348328,0.336101,0.647751}%
\pgfsetfillcolor{currentfill}%
\pgfsetlinewidth{0.000000pt}%
\definecolor{currentstroke}{rgb}{0.000000,0.000000,0.000000}%
\pgfsetstrokecolor{currentstroke}%
\pgfsetstrokeopacity{0.000000}%
\pgfsetdash{}{0pt}%
\pgfpathmoveto{\pgfqpoint{3.279629in}{1.487942in}}%
\pgfpathlineto{\pgfqpoint{3.243760in}{1.487942in}}%
\pgfpathlineto{\pgfqpoint{3.243760in}{1.508871in}}%
\pgfpathlineto{\pgfqpoint{3.279629in}{1.508871in}}%
\pgfpathlineto{\pgfqpoint{3.279629in}{1.487942in}}%
\pgfpathclose%
\pgfusepath{fill}%
\end{pgfscope}%
\begin{pgfscope}%
\pgfpathrectangle{\pgfqpoint{3.019583in}{0.169444in}}{\pgfqpoint{0.896708in}{1.339426in}}%
\pgfusepath{clip}%
\pgfsetbuttcap%
\pgfsetmiterjoin%
\definecolor{currentfill}{rgb}{0.368627,0.309804,0.635294}%
\pgfsetfillcolor{currentfill}%
\pgfsetlinewidth{0.000000pt}%
\definecolor{currentstroke}{rgb}{0.000000,0.000000,0.000000}%
\pgfsetstrokecolor{currentstroke}%
\pgfsetstrokeopacity{0.000000}%
\pgfsetdash{}{0pt}%
\pgfpathmoveto{\pgfqpoint{3.279629in}{1.508871in}}%
\pgfpathlineto{\pgfqpoint{3.243760in}{1.508871in}}%
\pgfpathlineto{\pgfqpoint{3.243760in}{1.529799in}}%
\pgfpathlineto{\pgfqpoint{3.279629in}{1.529799in}}%
\pgfpathlineto{\pgfqpoint{3.279629in}{1.508871in}}%
\pgfpathclose%
\pgfusepath{fill}%
\end{pgfscope}%
\begin{pgfscope}%
\pgfpathrectangle{\pgfqpoint{3.019583in}{0.169444in}}{\pgfqpoint{0.896708in}{1.339426in}}%
\pgfusepath{clip}%
\pgfsetbuttcap%
\pgfsetmiterjoin%
\definecolor{currentfill}{rgb}{0.995156,0.832295,0.506344}%
\pgfsetfillcolor{currentfill}%
\pgfsetlinewidth{0.000000pt}%
\definecolor{currentstroke}{rgb}{0.000000,0.000000,0.000000}%
\pgfsetstrokecolor{currentstroke}%
\pgfsetstrokeopacity{0.000000}%
\pgfsetdash{}{0pt}%
\pgfpathmoveto{\pgfqpoint{3.163057in}{0.671729in}}%
\pgfpathlineto{\pgfqpoint{3.198925in}{0.671729in}}%
\pgfpathlineto{\pgfqpoint{3.198925in}{0.713586in}}%
\pgfpathlineto{\pgfqpoint{3.163057in}{0.713586in}}%
\pgfpathlineto{\pgfqpoint{3.163057in}{0.671729in}}%
\pgfpathclose%
\pgfusepath{fill}%
\end{pgfscope}%
\begin{pgfscope}%
\pgfpathrectangle{\pgfqpoint{3.019583in}{0.169444in}}{\pgfqpoint{0.896708in}{1.339426in}}%
\pgfusepath{clip}%
\pgfsetbuttcap%
\pgfsetmiterjoin%
\definecolor{currentfill}{rgb}{0.997616,0.926105,0.625067}%
\pgfsetfillcolor{currentfill}%
\pgfsetlinewidth{0.000000pt}%
\definecolor{currentstroke}{rgb}{0.000000,0.000000,0.000000}%
\pgfsetstrokecolor{currentstroke}%
\pgfsetstrokeopacity{0.000000}%
\pgfsetdash{}{0pt}%
\pgfpathmoveto{\pgfqpoint{3.180991in}{0.755443in}}%
\pgfpathlineto{\pgfqpoint{3.198925in}{0.755443in}}%
\pgfpathlineto{\pgfqpoint{3.198925in}{0.797300in}}%
\pgfpathlineto{\pgfqpoint{3.180991in}{0.797300in}}%
\pgfpathlineto{\pgfqpoint{3.180991in}{0.755443in}}%
\pgfpathclose%
\pgfusepath{fill}%
\end{pgfscope}%
\begin{pgfscope}%
\pgfpathrectangle{\pgfqpoint{3.019583in}{0.169444in}}{\pgfqpoint{0.896708in}{1.339426in}}%
\pgfusepath{clip}%
\pgfsetbuttcap%
\pgfsetmiterjoin%
\definecolor{currentfill}{rgb}{0.998847,0.964245,0.689043}%
\pgfsetfillcolor{currentfill}%
\pgfsetlinewidth{0.000000pt}%
\definecolor{currentstroke}{rgb}{0.000000,0.000000,0.000000}%
\pgfsetstrokecolor{currentstroke}%
\pgfsetstrokeopacity{0.000000}%
\pgfsetdash{}{0pt}%
\pgfpathmoveto{\pgfqpoint{3.145122in}{0.797300in}}%
\pgfpathlineto{\pgfqpoint{3.198925in}{0.797300in}}%
\pgfpathlineto{\pgfqpoint{3.198925in}{0.839157in}}%
\pgfpathlineto{\pgfqpoint{3.145122in}{0.839157in}}%
\pgfpathlineto{\pgfqpoint{3.145122in}{0.797300in}}%
\pgfpathclose%
\pgfusepath{fill}%
\end{pgfscope}%
\begin{pgfscope}%
\pgfpathrectangle{\pgfqpoint{3.019583in}{0.169444in}}{\pgfqpoint{0.896708in}{1.339426in}}%
\pgfusepath{clip}%
\pgfsetbuttcap%
\pgfsetmiterjoin%
\definecolor{currentfill}{rgb}{0.998078,0.999231,0.746021}%
\pgfsetfillcolor{currentfill}%
\pgfsetlinewidth{0.000000pt}%
\definecolor{currentstroke}{rgb}{0.000000,0.000000,0.000000}%
\pgfsetstrokecolor{currentstroke}%
\pgfsetstrokeopacity{0.000000}%
\pgfsetdash{}{0pt}%
\pgfpathmoveto{\pgfqpoint{3.145122in}{0.839157in}}%
\pgfpathlineto{\pgfqpoint{3.198925in}{0.839157in}}%
\pgfpathlineto{\pgfqpoint{3.198925in}{0.881015in}}%
\pgfpathlineto{\pgfqpoint{3.145122in}{0.881015in}}%
\pgfpathlineto{\pgfqpoint{3.145122in}{0.839157in}}%
\pgfpathclose%
\pgfusepath{fill}%
\end{pgfscope}%
\begin{pgfscope}%
\pgfpathrectangle{\pgfqpoint{3.019583in}{0.169444in}}{\pgfqpoint{0.896708in}{1.339426in}}%
\pgfusepath{clip}%
\pgfsetbuttcap%
\pgfsetmiterjoin%
\definecolor{currentfill}{rgb}{0.967320,0.986928,0.698039}%
\pgfsetfillcolor{currentfill}%
\pgfsetlinewidth{0.000000pt}%
\definecolor{currentstroke}{rgb}{0.000000,0.000000,0.000000}%
\pgfsetstrokecolor{currentstroke}%
\pgfsetstrokeopacity{0.000000}%
\pgfsetdash{}{0pt}%
\pgfpathmoveto{\pgfqpoint{3.037517in}{0.881015in}}%
\pgfpathlineto{\pgfqpoint{3.198925in}{0.881015in}}%
\pgfpathlineto{\pgfqpoint{3.198925in}{0.922872in}}%
\pgfpathlineto{\pgfqpoint{3.037517in}{0.922872in}}%
\pgfpathlineto{\pgfqpoint{3.037517in}{0.881015in}}%
\pgfpathclose%
\pgfusepath{fill}%
\end{pgfscope}%
\begin{pgfscope}%
\pgfpathrectangle{\pgfqpoint{3.019583in}{0.169444in}}{\pgfqpoint{0.896708in}{1.339426in}}%
\pgfusepath{clip}%
\pgfsetbuttcap%
\pgfsetmiterjoin%
\definecolor{currentfill}{rgb}{0.936563,0.974625,0.650058}%
\pgfsetfillcolor{currentfill}%
\pgfsetlinewidth{0.000000pt}%
\definecolor{currentstroke}{rgb}{0.000000,0.000000,0.000000}%
\pgfsetstrokecolor{currentstroke}%
\pgfsetstrokeopacity{0.000000}%
\pgfsetdash{}{0pt}%
\pgfpathmoveto{\pgfqpoint{3.091320in}{0.922872in}}%
\pgfpathlineto{\pgfqpoint{3.198925in}{0.922872in}}%
\pgfpathlineto{\pgfqpoint{3.198925in}{0.964729in}}%
\pgfpathlineto{\pgfqpoint{3.091320in}{0.964729in}}%
\pgfpathlineto{\pgfqpoint{3.091320in}{0.922872in}}%
\pgfpathclose%
\pgfusepath{fill}%
\end{pgfscope}%
\begin{pgfscope}%
\pgfpathrectangle{\pgfqpoint{3.019583in}{0.169444in}}{\pgfqpoint{0.896708in}{1.339426in}}%
\pgfusepath{clip}%
\pgfsetbuttcap%
\pgfsetmiterjoin%
\definecolor{currentfill}{rgb}{0.905805,0.962322,0.602076}%
\pgfsetfillcolor{currentfill}%
\pgfsetlinewidth{0.000000pt}%
\definecolor{currentstroke}{rgb}{0.000000,0.000000,0.000000}%
\pgfsetstrokecolor{currentstroke}%
\pgfsetstrokeopacity{0.000000}%
\pgfsetdash{}{0pt}%
\pgfpathmoveto{\pgfqpoint{3.091320in}{0.964729in}}%
\pgfpathlineto{\pgfqpoint{3.198925in}{0.964729in}}%
\pgfpathlineto{\pgfqpoint{3.198925in}{1.006586in}}%
\pgfpathlineto{\pgfqpoint{3.091320in}{1.006586in}}%
\pgfpathlineto{\pgfqpoint{3.091320in}{0.964729in}}%
\pgfpathclose%
\pgfusepath{fill}%
\end{pgfscope}%
\begin{pgfscope}%
\pgfpathrectangle{\pgfqpoint{3.019583in}{0.169444in}}{\pgfqpoint{0.896708in}{1.339426in}}%
\pgfusepath{clip}%
\pgfsetbuttcap%
\pgfsetmiterjoin%
\definecolor{currentfill}{rgb}{0.838447,0.934948,0.608997}%
\pgfsetfillcolor{currentfill}%
\pgfsetlinewidth{0.000000pt}%
\definecolor{currentstroke}{rgb}{0.000000,0.000000,0.000000}%
\pgfsetstrokecolor{currentstroke}%
\pgfsetstrokeopacity{0.000000}%
\pgfsetdash{}{0pt}%
\pgfpathmoveto{\pgfqpoint{3.091320in}{1.006586in}}%
\pgfpathlineto{\pgfqpoint{3.198925in}{1.006586in}}%
\pgfpathlineto{\pgfqpoint{3.198925in}{1.048443in}}%
\pgfpathlineto{\pgfqpoint{3.091320in}{1.048443in}}%
\pgfpathlineto{\pgfqpoint{3.091320in}{1.006586in}}%
\pgfpathclose%
\pgfusepath{fill}%
\end{pgfscope}%
\begin{pgfscope}%
\pgfpathrectangle{\pgfqpoint{3.019583in}{0.169444in}}{\pgfqpoint{0.896708in}{1.339426in}}%
\pgfusepath{clip}%
\pgfsetbuttcap%
\pgfsetmiterjoin%
\definecolor{currentfill}{rgb}{0.765859,0.905421,0.623760}%
\pgfsetfillcolor{currentfill}%
\pgfsetlinewidth{0.000000pt}%
\definecolor{currentstroke}{rgb}{0.000000,0.000000,0.000000}%
\pgfsetstrokecolor{currentstroke}%
\pgfsetstrokeopacity{0.000000}%
\pgfsetdash{}{0pt}%
\pgfpathmoveto{\pgfqpoint{3.109254in}{1.048443in}}%
\pgfpathlineto{\pgfqpoint{3.198925in}{1.048443in}}%
\pgfpathlineto{\pgfqpoint{3.198925in}{1.090300in}}%
\pgfpathlineto{\pgfqpoint{3.109254in}{1.090300in}}%
\pgfpathlineto{\pgfqpoint{3.109254in}{1.048443in}}%
\pgfpathclose%
\pgfusepath{fill}%
\end{pgfscope}%
\begin{pgfscope}%
\pgfpathrectangle{\pgfqpoint{3.019583in}{0.169444in}}{\pgfqpoint{0.896708in}{1.339426in}}%
\pgfusepath{clip}%
\pgfsetbuttcap%
\pgfsetmiterjoin%
\definecolor{currentfill}{rgb}{0.693272,0.875894,0.638524}%
\pgfsetfillcolor{currentfill}%
\pgfsetlinewidth{0.000000pt}%
\definecolor{currentstroke}{rgb}{0.000000,0.000000,0.000000}%
\pgfsetstrokecolor{currentstroke}%
\pgfsetstrokeopacity{0.000000}%
\pgfsetdash{}{0pt}%
\pgfpathmoveto{\pgfqpoint{3.180991in}{1.090300in}}%
\pgfpathlineto{\pgfqpoint{3.198925in}{1.090300in}}%
\pgfpathlineto{\pgfqpoint{3.198925in}{1.132157in}}%
\pgfpathlineto{\pgfqpoint{3.180991in}{1.132157in}}%
\pgfpathlineto{\pgfqpoint{3.180991in}{1.090300in}}%
\pgfpathclose%
\pgfusepath{fill}%
\end{pgfscope}%
\begin{pgfscope}%
\pgfpathrectangle{\pgfqpoint{3.019583in}{0.169444in}}{\pgfqpoint{0.896708in}{1.339426in}}%
\pgfusepath{clip}%
\pgfsetbuttcap%
\pgfsetmiterjoin%
\definecolor{currentfill}{rgb}{0.612226,0.843829,0.643983}%
\pgfsetfillcolor{currentfill}%
\pgfsetlinewidth{0.000000pt}%
\definecolor{currentstroke}{rgb}{0.000000,0.000000,0.000000}%
\pgfsetstrokecolor{currentstroke}%
\pgfsetstrokeopacity{0.000000}%
\pgfsetdash{}{0pt}%
\pgfpathmoveto{\pgfqpoint{3.145122in}{1.132157in}}%
\pgfpathlineto{\pgfqpoint{3.198925in}{1.132157in}}%
\pgfpathlineto{\pgfqpoint{3.198925in}{1.174014in}}%
\pgfpathlineto{\pgfqpoint{3.145122in}{1.174014in}}%
\pgfpathlineto{\pgfqpoint{3.145122in}{1.132157in}}%
\pgfpathclose%
\pgfusepath{fill}%
\end{pgfscope}%
\begin{pgfscope}%
\pgfpathrectangle{\pgfqpoint{3.019583in}{0.169444in}}{\pgfqpoint{0.896708in}{1.339426in}}%
\pgfusepath{clip}%
\pgfsetbuttcap%
\pgfsetmiterjoin%
\definecolor{currentfill}{rgb}{0.442445,0.777393,0.646444}%
\pgfsetfillcolor{currentfill}%
\pgfsetlinewidth{0.000000pt}%
\definecolor{currentstroke}{rgb}{0.000000,0.000000,0.000000}%
\pgfsetstrokecolor{currentstroke}%
\pgfsetstrokeopacity{0.000000}%
\pgfsetdash{}{0pt}%
\pgfpathmoveto{\pgfqpoint{3.145122in}{1.215871in}}%
\pgfpathlineto{\pgfqpoint{3.198925in}{1.215871in}}%
\pgfpathlineto{\pgfqpoint{3.198925in}{1.257728in}}%
\pgfpathlineto{\pgfqpoint{3.145122in}{1.257728in}}%
\pgfpathlineto{\pgfqpoint{3.145122in}{1.215871in}}%
\pgfpathclose%
\pgfusepath{fill}%
\end{pgfscope}%
\begin{pgfscope}%
\pgfpathrectangle{\pgfqpoint{3.019583in}{0.169444in}}{\pgfqpoint{0.896708in}{1.339426in}}%
\pgfusepath{clip}%
\pgfsetbuttcap%
\pgfsetmiterjoin%
\definecolor{currentfill}{rgb}{0.304037,0.653749,0.691349}%
\pgfsetfillcolor{currentfill}%
\pgfsetlinewidth{0.000000pt}%
\definecolor{currentstroke}{rgb}{0.000000,0.000000,0.000000}%
\pgfsetstrokecolor{currentstroke}%
\pgfsetstrokeopacity{0.000000}%
\pgfsetdash{}{0pt}%
\pgfpathmoveto{\pgfqpoint{3.180991in}{1.299585in}}%
\pgfpathlineto{\pgfqpoint{3.198925in}{1.299585in}}%
\pgfpathlineto{\pgfqpoint{3.198925in}{1.341442in}}%
\pgfpathlineto{\pgfqpoint{3.180991in}{1.341442in}}%
\pgfpathlineto{\pgfqpoint{3.180991in}{1.299585in}}%
\pgfpathclose%
\pgfusepath{fill}%
\end{pgfscope}%
\begin{pgfscope}%
\pgfpathrectangle{\pgfqpoint{3.019583in}{0.169444in}}{\pgfqpoint{0.896708in}{1.339426in}}%
\pgfusepath{clip}%
\pgfsetbuttcap%
\pgfsetmiterjoin%
\definecolor{currentfill}{rgb}{0.267128,0.441292,0.697578}%
\pgfsetfillcolor{currentfill}%
\pgfsetlinewidth{0.000000pt}%
\definecolor{currentstroke}{rgb}{0.000000,0.000000,0.000000}%
\pgfsetstrokecolor{currentstroke}%
\pgfsetstrokeopacity{0.000000}%
\pgfsetdash{}{0pt}%
\pgfpathmoveto{\pgfqpoint{3.180991in}{1.425156in}}%
\pgfpathlineto{\pgfqpoint{3.198925in}{1.425156in}}%
\pgfpathlineto{\pgfqpoint{3.198925in}{1.467013in}}%
\pgfpathlineto{\pgfqpoint{3.180991in}{1.467013in}}%
\pgfpathlineto{\pgfqpoint{3.180991in}{1.425156in}}%
\pgfpathclose%
\pgfusepath{fill}%
\end{pgfscope}%
\begin{pgfscope}%
\pgfpathrectangle{\pgfqpoint{3.019583in}{0.169444in}}{\pgfqpoint{0.896708in}{1.339426in}}%
\pgfusepath{clip}%
\pgfsetbuttcap%
\pgfsetmiterjoin%
\definecolor{currentfill}{rgb}{0.321261,0.371165,0.664360}%
\pgfsetfillcolor{currentfill}%
\pgfsetlinewidth{0.000000pt}%
\definecolor{currentstroke}{rgb}{0.000000,0.000000,0.000000}%
\pgfsetstrokecolor{currentstroke}%
\pgfsetstrokeopacity{0.000000}%
\pgfsetdash{}{0pt}%
\pgfpathmoveto{\pgfqpoint{3.180991in}{1.467013in}}%
\pgfpathlineto{\pgfqpoint{3.198925in}{1.467013in}}%
\pgfpathlineto{\pgfqpoint{3.198925in}{1.508871in}}%
\pgfpathlineto{\pgfqpoint{3.180991in}{1.508871in}}%
\pgfpathlineto{\pgfqpoint{3.180991in}{1.467013in}}%
\pgfpathclose%
\pgfusepath{fill}%
\end{pgfscope}%
\begin{pgfscope}%
\pgfpathrectangle{\pgfqpoint{3.019583in}{0.169444in}}{\pgfqpoint{0.896708in}{1.339426in}}%
\pgfusepath{clip}%
\pgfsetbuttcap%
\pgfsetmiterjoin%
\definecolor{currentfill}{rgb}{0.619608,0.003922,0.258824}%
\pgfsetfillcolor{currentfill}%
\pgfsetlinewidth{0.000000pt}%
\definecolor{currentstroke}{rgb}{0.000000,0.000000,0.000000}%
\pgfsetstrokecolor{currentstroke}%
\pgfsetstrokeopacity{0.000000}%
\pgfsetdash{}{0pt}%
\pgfpathmoveto{\pgfqpoint{3.521740in}{0.169444in}}%
\pgfpathlineto{\pgfqpoint{3.485872in}{0.169444in}}%
\pgfpathlineto{\pgfqpoint{3.485872in}{0.190373in}}%
\pgfpathlineto{\pgfqpoint{3.521740in}{0.190373in}}%
\pgfpathlineto{\pgfqpoint{3.521740in}{0.169444in}}%
\pgfpathclose%
\pgfusepath{fill}%
\end{pgfscope}%
\begin{pgfscope}%
\pgfpathrectangle{\pgfqpoint{3.019583in}{0.169444in}}{\pgfqpoint{0.896708in}{1.339426in}}%
\pgfusepath{clip}%
\pgfsetbuttcap%
\pgfsetmiterjoin%
\definecolor{currentfill}{rgb}{0.653441,0.041446,0.266820}%
\pgfsetfillcolor{currentfill}%
\pgfsetlinewidth{0.000000pt}%
\definecolor{currentstroke}{rgb}{0.000000,0.000000,0.000000}%
\pgfsetstrokecolor{currentstroke}%
\pgfsetstrokeopacity{0.000000}%
\pgfsetdash{}{0pt}%
\pgfpathmoveto{\pgfqpoint{3.521740in}{0.190373in}}%
\pgfpathlineto{\pgfqpoint{3.485872in}{0.190373in}}%
\pgfpathlineto{\pgfqpoint{3.485872in}{0.211302in}}%
\pgfpathlineto{\pgfqpoint{3.521740in}{0.211302in}}%
\pgfpathlineto{\pgfqpoint{3.521740in}{0.190373in}}%
\pgfpathclose%
\pgfusepath{fill}%
\end{pgfscope}%
\begin{pgfscope}%
\pgfpathrectangle{\pgfqpoint{3.019583in}{0.169444in}}{\pgfqpoint{0.896708in}{1.339426in}}%
\pgfusepath{clip}%
\pgfsetbuttcap%
\pgfsetmiterjoin%
\definecolor{currentfill}{rgb}{0.687274,0.078970,0.274817}%
\pgfsetfillcolor{currentfill}%
\pgfsetlinewidth{0.000000pt}%
\definecolor{currentstroke}{rgb}{0.000000,0.000000,0.000000}%
\pgfsetstrokecolor{currentstroke}%
\pgfsetstrokeopacity{0.000000}%
\pgfsetdash{}{0pt}%
\pgfpathmoveto{\pgfqpoint{3.521740in}{0.211302in}}%
\pgfpathlineto{\pgfqpoint{3.485872in}{0.211302in}}%
\pgfpathlineto{\pgfqpoint{3.485872in}{0.232230in}}%
\pgfpathlineto{\pgfqpoint{3.521740in}{0.232230in}}%
\pgfpathlineto{\pgfqpoint{3.521740in}{0.211302in}}%
\pgfpathclose%
\pgfusepath{fill}%
\end{pgfscope}%
\begin{pgfscope}%
\pgfpathrectangle{\pgfqpoint{3.019583in}{0.169444in}}{\pgfqpoint{0.896708in}{1.339426in}}%
\pgfusepath{clip}%
\pgfsetbuttcap%
\pgfsetmiterjoin%
\definecolor{currentfill}{rgb}{0.721107,0.116494,0.282814}%
\pgfsetfillcolor{currentfill}%
\pgfsetlinewidth{0.000000pt}%
\definecolor{currentstroke}{rgb}{0.000000,0.000000,0.000000}%
\pgfsetstrokecolor{currentstroke}%
\pgfsetstrokeopacity{0.000000}%
\pgfsetdash{}{0pt}%
\pgfpathmoveto{\pgfqpoint{3.521740in}{0.232230in}}%
\pgfpathlineto{\pgfqpoint{3.485872in}{0.232230in}}%
\pgfpathlineto{\pgfqpoint{3.485872in}{0.253159in}}%
\pgfpathlineto{\pgfqpoint{3.521740in}{0.253159in}}%
\pgfpathlineto{\pgfqpoint{3.521740in}{0.232230in}}%
\pgfpathclose%
\pgfusepath{fill}%
\end{pgfscope}%
\begin{pgfscope}%
\pgfpathrectangle{\pgfqpoint{3.019583in}{0.169444in}}{\pgfqpoint{0.896708in}{1.339426in}}%
\pgfusepath{clip}%
\pgfsetbuttcap%
\pgfsetmiterjoin%
\definecolor{currentfill}{rgb}{0.754940,0.154018,0.290811}%
\pgfsetfillcolor{currentfill}%
\pgfsetlinewidth{0.000000pt}%
\definecolor{currentstroke}{rgb}{0.000000,0.000000,0.000000}%
\pgfsetstrokecolor{currentstroke}%
\pgfsetstrokeopacity{0.000000}%
\pgfsetdash{}{0pt}%
\pgfpathmoveto{\pgfqpoint{3.521740in}{0.253159in}}%
\pgfpathlineto{\pgfqpoint{3.485872in}{0.253159in}}%
\pgfpathlineto{\pgfqpoint{3.485872in}{0.274087in}}%
\pgfpathlineto{\pgfqpoint{3.521740in}{0.274087in}}%
\pgfpathlineto{\pgfqpoint{3.521740in}{0.253159in}}%
\pgfpathclose%
\pgfusepath{fill}%
\end{pgfscope}%
\begin{pgfscope}%
\pgfpathrectangle{\pgfqpoint{3.019583in}{0.169444in}}{\pgfqpoint{0.896708in}{1.339426in}}%
\pgfusepath{clip}%
\pgfsetbuttcap%
\pgfsetmiterjoin%
\definecolor{currentfill}{rgb}{0.788774,0.191542,0.298808}%
\pgfsetfillcolor{currentfill}%
\pgfsetlinewidth{0.000000pt}%
\definecolor{currentstroke}{rgb}{0.000000,0.000000,0.000000}%
\pgfsetstrokecolor{currentstroke}%
\pgfsetstrokeopacity{0.000000}%
\pgfsetdash{}{0pt}%
\pgfpathmoveto{\pgfqpoint{3.521740in}{0.274087in}}%
\pgfpathlineto{\pgfqpoint{3.485872in}{0.274087in}}%
\pgfpathlineto{\pgfqpoint{3.485872in}{0.295016in}}%
\pgfpathlineto{\pgfqpoint{3.521740in}{0.295016in}}%
\pgfpathlineto{\pgfqpoint{3.521740in}{0.274087in}}%
\pgfpathclose%
\pgfusepath{fill}%
\end{pgfscope}%
\begin{pgfscope}%
\pgfpathrectangle{\pgfqpoint{3.019583in}{0.169444in}}{\pgfqpoint{0.896708in}{1.339426in}}%
\pgfusepath{clip}%
\pgfsetbuttcap%
\pgfsetmiterjoin%
\definecolor{currentfill}{rgb}{0.822607,0.229066,0.306805}%
\pgfsetfillcolor{currentfill}%
\pgfsetlinewidth{0.000000pt}%
\definecolor{currentstroke}{rgb}{0.000000,0.000000,0.000000}%
\pgfsetstrokecolor{currentstroke}%
\pgfsetstrokeopacity{0.000000}%
\pgfsetdash{}{0pt}%
\pgfpathmoveto{\pgfqpoint{3.521740in}{0.295016in}}%
\pgfpathlineto{\pgfqpoint{3.485872in}{0.295016in}}%
\pgfpathlineto{\pgfqpoint{3.485872in}{0.315944in}}%
\pgfpathlineto{\pgfqpoint{3.521740in}{0.315944in}}%
\pgfpathlineto{\pgfqpoint{3.521740in}{0.295016in}}%
\pgfpathclose%
\pgfusepath{fill}%
\end{pgfscope}%
\begin{pgfscope}%
\pgfpathrectangle{\pgfqpoint{3.019583in}{0.169444in}}{\pgfqpoint{0.896708in}{1.339426in}}%
\pgfusepath{clip}%
\pgfsetbuttcap%
\pgfsetmiterjoin%
\definecolor{currentfill}{rgb}{0.847213,0.261207,0.305190}%
\pgfsetfillcolor{currentfill}%
\pgfsetlinewidth{0.000000pt}%
\definecolor{currentstroke}{rgb}{0.000000,0.000000,0.000000}%
\pgfsetstrokecolor{currentstroke}%
\pgfsetstrokeopacity{0.000000}%
\pgfsetdash{}{0pt}%
\pgfpathmoveto{\pgfqpoint{3.521740in}{0.315944in}}%
\pgfpathlineto{\pgfqpoint{3.485872in}{0.315944in}}%
\pgfpathlineto{\pgfqpoint{3.485872in}{0.336873in}}%
\pgfpathlineto{\pgfqpoint{3.521740in}{0.336873in}}%
\pgfpathlineto{\pgfqpoint{3.521740in}{0.315944in}}%
\pgfpathclose%
\pgfusepath{fill}%
\end{pgfscope}%
\begin{pgfscope}%
\pgfpathrectangle{\pgfqpoint{3.019583in}{0.169444in}}{\pgfqpoint{0.896708in}{1.339426in}}%
\pgfusepath{clip}%
\pgfsetbuttcap%
\pgfsetmiterjoin%
\definecolor{currentfill}{rgb}{0.866282,0.290119,0.297809}%
\pgfsetfillcolor{currentfill}%
\pgfsetlinewidth{0.000000pt}%
\definecolor{currentstroke}{rgb}{0.000000,0.000000,0.000000}%
\pgfsetstrokecolor{currentstroke}%
\pgfsetstrokeopacity{0.000000}%
\pgfsetdash{}{0pt}%
\pgfpathmoveto{\pgfqpoint{3.521740in}{0.336873in}}%
\pgfpathlineto{\pgfqpoint{3.485872in}{0.336873in}}%
\pgfpathlineto{\pgfqpoint{3.485872in}{0.357801in}}%
\pgfpathlineto{\pgfqpoint{3.521740in}{0.357801in}}%
\pgfpathlineto{\pgfqpoint{3.521740in}{0.336873in}}%
\pgfpathclose%
\pgfusepath{fill}%
\end{pgfscope}%
\begin{pgfscope}%
\pgfpathrectangle{\pgfqpoint{3.019583in}{0.169444in}}{\pgfqpoint{0.896708in}{1.339426in}}%
\pgfusepath{clip}%
\pgfsetbuttcap%
\pgfsetmiterjoin%
\definecolor{currentfill}{rgb}{0.885352,0.319031,0.290427}%
\pgfsetfillcolor{currentfill}%
\pgfsetlinewidth{0.000000pt}%
\definecolor{currentstroke}{rgb}{0.000000,0.000000,0.000000}%
\pgfsetstrokecolor{currentstroke}%
\pgfsetstrokeopacity{0.000000}%
\pgfsetdash{}{0pt}%
\pgfpathmoveto{\pgfqpoint{3.521740in}{0.357801in}}%
\pgfpathlineto{\pgfqpoint{3.485872in}{0.357801in}}%
\pgfpathlineto{\pgfqpoint{3.485872in}{0.378730in}}%
\pgfpathlineto{\pgfqpoint{3.521740in}{0.378730in}}%
\pgfpathlineto{\pgfqpoint{3.521740in}{0.357801in}}%
\pgfpathclose%
\pgfusepath{fill}%
\end{pgfscope}%
\begin{pgfscope}%
\pgfpathrectangle{\pgfqpoint{3.019583in}{0.169444in}}{\pgfqpoint{0.896708in}{1.339426in}}%
\pgfusepath{clip}%
\pgfsetbuttcap%
\pgfsetmiterjoin%
\definecolor{currentfill}{rgb}{0.904421,0.347943,0.283045}%
\pgfsetfillcolor{currentfill}%
\pgfsetlinewidth{0.000000pt}%
\definecolor{currentstroke}{rgb}{0.000000,0.000000,0.000000}%
\pgfsetstrokecolor{currentstroke}%
\pgfsetstrokeopacity{0.000000}%
\pgfsetdash{}{0pt}%
\pgfpathmoveto{\pgfqpoint{3.521740in}{0.378730in}}%
\pgfpathlineto{\pgfqpoint{3.485872in}{0.378730in}}%
\pgfpathlineto{\pgfqpoint{3.485872in}{0.399658in}}%
\pgfpathlineto{\pgfqpoint{3.521740in}{0.399658in}}%
\pgfpathlineto{\pgfqpoint{3.521740in}{0.378730in}}%
\pgfpathclose%
\pgfusepath{fill}%
\end{pgfscope}%
\begin{pgfscope}%
\pgfpathrectangle{\pgfqpoint{3.019583in}{0.169444in}}{\pgfqpoint{0.896708in}{1.339426in}}%
\pgfusepath{clip}%
\pgfsetbuttcap%
\pgfsetmiterjoin%
\definecolor{currentfill}{rgb}{0.923491,0.376855,0.275663}%
\pgfsetfillcolor{currentfill}%
\pgfsetlinewidth{0.000000pt}%
\definecolor{currentstroke}{rgb}{0.000000,0.000000,0.000000}%
\pgfsetstrokecolor{currentstroke}%
\pgfsetstrokeopacity{0.000000}%
\pgfsetdash{}{0pt}%
\pgfpathmoveto{\pgfqpoint{3.521740in}{0.399658in}}%
\pgfpathlineto{\pgfqpoint{3.485872in}{0.399658in}}%
\pgfpathlineto{\pgfqpoint{3.485872in}{0.420587in}}%
\pgfpathlineto{\pgfqpoint{3.521740in}{0.420587in}}%
\pgfpathlineto{\pgfqpoint{3.521740in}{0.399658in}}%
\pgfpathclose%
\pgfusepath{fill}%
\end{pgfscope}%
\begin{pgfscope}%
\pgfpathrectangle{\pgfqpoint{3.019583in}{0.169444in}}{\pgfqpoint{0.896708in}{1.339426in}}%
\pgfusepath{clip}%
\pgfsetbuttcap%
\pgfsetmiterjoin%
\definecolor{currentfill}{rgb}{0.942561,0.405767,0.268281}%
\pgfsetfillcolor{currentfill}%
\pgfsetlinewidth{0.000000pt}%
\definecolor{currentstroke}{rgb}{0.000000,0.000000,0.000000}%
\pgfsetstrokecolor{currentstroke}%
\pgfsetstrokeopacity{0.000000}%
\pgfsetdash{}{0pt}%
\pgfpathmoveto{\pgfqpoint{3.521740in}{0.420587in}}%
\pgfpathlineto{\pgfqpoint{3.485872in}{0.420587in}}%
\pgfpathlineto{\pgfqpoint{3.485872in}{0.441515in}}%
\pgfpathlineto{\pgfqpoint{3.521740in}{0.441515in}}%
\pgfpathlineto{\pgfqpoint{3.521740in}{0.420587in}}%
\pgfpathclose%
\pgfusepath{fill}%
\end{pgfscope}%
\begin{pgfscope}%
\pgfpathrectangle{\pgfqpoint{3.019583in}{0.169444in}}{\pgfqpoint{0.896708in}{1.339426in}}%
\pgfusepath{clip}%
\pgfsetbuttcap%
\pgfsetmiterjoin%
\definecolor{currentfill}{rgb}{0.958247,0.437447,0.267359}%
\pgfsetfillcolor{currentfill}%
\pgfsetlinewidth{0.000000pt}%
\definecolor{currentstroke}{rgb}{0.000000,0.000000,0.000000}%
\pgfsetstrokecolor{currentstroke}%
\pgfsetstrokeopacity{0.000000}%
\pgfsetdash{}{0pt}%
\pgfpathmoveto{\pgfqpoint{3.521740in}{0.441515in}}%
\pgfpathlineto{\pgfqpoint{3.485872in}{0.441515in}}%
\pgfpathlineto{\pgfqpoint{3.485872in}{0.462444in}}%
\pgfpathlineto{\pgfqpoint{3.521740in}{0.462444in}}%
\pgfpathlineto{\pgfqpoint{3.521740in}{0.441515in}}%
\pgfpathclose%
\pgfusepath{fill}%
\end{pgfscope}%
\begin{pgfscope}%
\pgfpathrectangle{\pgfqpoint{3.019583in}{0.169444in}}{\pgfqpoint{0.896708in}{1.339426in}}%
\pgfusepath{clip}%
\pgfsetbuttcap%
\pgfsetmiterjoin%
\definecolor{currentfill}{rgb}{0.963783,0.477432,0.285813}%
\pgfsetfillcolor{currentfill}%
\pgfsetlinewidth{0.000000pt}%
\definecolor{currentstroke}{rgb}{0.000000,0.000000,0.000000}%
\pgfsetstrokecolor{currentstroke}%
\pgfsetstrokeopacity{0.000000}%
\pgfsetdash{}{0pt}%
\pgfpathmoveto{\pgfqpoint{3.521740in}{0.462444in}}%
\pgfpathlineto{\pgfqpoint{3.485872in}{0.462444in}}%
\pgfpathlineto{\pgfqpoint{3.485872in}{0.483372in}}%
\pgfpathlineto{\pgfqpoint{3.521740in}{0.483372in}}%
\pgfpathlineto{\pgfqpoint{3.521740in}{0.462444in}}%
\pgfpathclose%
\pgfusepath{fill}%
\end{pgfscope}%
\begin{pgfscope}%
\pgfpathrectangle{\pgfqpoint{3.019583in}{0.169444in}}{\pgfqpoint{0.896708in}{1.339426in}}%
\pgfusepath{clip}%
\pgfsetbuttcap%
\pgfsetmiterjoin%
\definecolor{currentfill}{rgb}{0.969319,0.517416,0.304268}%
\pgfsetfillcolor{currentfill}%
\pgfsetlinewidth{0.000000pt}%
\definecolor{currentstroke}{rgb}{0.000000,0.000000,0.000000}%
\pgfsetstrokecolor{currentstroke}%
\pgfsetstrokeopacity{0.000000}%
\pgfsetdash{}{0pt}%
\pgfpathmoveto{\pgfqpoint{3.521740in}{0.483372in}}%
\pgfpathlineto{\pgfqpoint{3.485872in}{0.483372in}}%
\pgfpathlineto{\pgfqpoint{3.485872in}{0.504301in}}%
\pgfpathlineto{\pgfqpoint{3.521740in}{0.504301in}}%
\pgfpathlineto{\pgfqpoint{3.521740in}{0.483372in}}%
\pgfpathclose%
\pgfusepath{fill}%
\end{pgfscope}%
\begin{pgfscope}%
\pgfpathrectangle{\pgfqpoint{3.019583in}{0.169444in}}{\pgfqpoint{0.896708in}{1.339426in}}%
\pgfusepath{clip}%
\pgfsetbuttcap%
\pgfsetmiterjoin%
\definecolor{currentfill}{rgb}{0.974856,0.557401,0.322722}%
\pgfsetfillcolor{currentfill}%
\pgfsetlinewidth{0.000000pt}%
\definecolor{currentstroke}{rgb}{0.000000,0.000000,0.000000}%
\pgfsetstrokecolor{currentstroke}%
\pgfsetstrokeopacity{0.000000}%
\pgfsetdash{}{0pt}%
\pgfpathmoveto{\pgfqpoint{3.521740in}{0.504301in}}%
\pgfpathlineto{\pgfqpoint{3.485872in}{0.504301in}}%
\pgfpathlineto{\pgfqpoint{3.485872in}{0.525230in}}%
\pgfpathlineto{\pgfqpoint{3.521740in}{0.525230in}}%
\pgfpathlineto{\pgfqpoint{3.521740in}{0.504301in}}%
\pgfpathclose%
\pgfusepath{fill}%
\end{pgfscope}%
\begin{pgfscope}%
\pgfpathrectangle{\pgfqpoint{3.019583in}{0.169444in}}{\pgfqpoint{0.896708in}{1.339426in}}%
\pgfusepath{clip}%
\pgfsetbuttcap%
\pgfsetmiterjoin%
\definecolor{currentfill}{rgb}{0.980392,0.597386,0.341176}%
\pgfsetfillcolor{currentfill}%
\pgfsetlinewidth{0.000000pt}%
\definecolor{currentstroke}{rgb}{0.000000,0.000000,0.000000}%
\pgfsetstrokecolor{currentstroke}%
\pgfsetstrokeopacity{0.000000}%
\pgfsetdash{}{0pt}%
\pgfpathmoveto{\pgfqpoint{3.521740in}{0.525230in}}%
\pgfpathlineto{\pgfqpoint{3.485872in}{0.525230in}}%
\pgfpathlineto{\pgfqpoint{3.485872in}{0.546158in}}%
\pgfpathlineto{\pgfqpoint{3.521740in}{0.546158in}}%
\pgfpathlineto{\pgfqpoint{3.521740in}{0.525230in}}%
\pgfpathclose%
\pgfusepath{fill}%
\end{pgfscope}%
\begin{pgfscope}%
\pgfpathrectangle{\pgfqpoint{3.019583in}{0.169444in}}{\pgfqpoint{0.896708in}{1.339426in}}%
\pgfusepath{clip}%
\pgfsetbuttcap%
\pgfsetmiterjoin%
\definecolor{currentfill}{rgb}{0.985928,0.637370,0.359631}%
\pgfsetfillcolor{currentfill}%
\pgfsetlinewidth{0.000000pt}%
\definecolor{currentstroke}{rgb}{0.000000,0.000000,0.000000}%
\pgfsetstrokecolor{currentstroke}%
\pgfsetstrokeopacity{0.000000}%
\pgfsetdash{}{0pt}%
\pgfpathmoveto{\pgfqpoint{3.521740in}{0.546158in}}%
\pgfpathlineto{\pgfqpoint{3.485872in}{0.546158in}}%
\pgfpathlineto{\pgfqpoint{3.485872in}{0.567087in}}%
\pgfpathlineto{\pgfqpoint{3.521740in}{0.567087in}}%
\pgfpathlineto{\pgfqpoint{3.521740in}{0.546158in}}%
\pgfpathclose%
\pgfusepath{fill}%
\end{pgfscope}%
\begin{pgfscope}%
\pgfpathrectangle{\pgfqpoint{3.019583in}{0.169444in}}{\pgfqpoint{0.896708in}{1.339426in}}%
\pgfusepath{clip}%
\pgfsetbuttcap%
\pgfsetmiterjoin%
\definecolor{currentfill}{rgb}{0.991465,0.677355,0.378085}%
\pgfsetfillcolor{currentfill}%
\pgfsetlinewidth{0.000000pt}%
\definecolor{currentstroke}{rgb}{0.000000,0.000000,0.000000}%
\pgfsetstrokecolor{currentstroke}%
\pgfsetstrokeopacity{0.000000}%
\pgfsetdash{}{0pt}%
\pgfpathmoveto{\pgfqpoint{3.521740in}{0.567087in}}%
\pgfpathlineto{\pgfqpoint{3.485872in}{0.567087in}}%
\pgfpathlineto{\pgfqpoint{3.485872in}{0.588015in}}%
\pgfpathlineto{\pgfqpoint{3.521740in}{0.588015in}}%
\pgfpathlineto{\pgfqpoint{3.521740in}{0.567087in}}%
\pgfpathclose%
\pgfusepath{fill}%
\end{pgfscope}%
\begin{pgfscope}%
\pgfpathrectangle{\pgfqpoint{3.019583in}{0.169444in}}{\pgfqpoint{0.896708in}{1.339426in}}%
\pgfusepath{clip}%
\pgfsetbuttcap%
\pgfsetmiterjoin%
\definecolor{currentfill}{rgb}{0.992695,0.709266,0.402999}%
\pgfsetfillcolor{currentfill}%
\pgfsetlinewidth{0.000000pt}%
\definecolor{currentstroke}{rgb}{0.000000,0.000000,0.000000}%
\pgfsetstrokecolor{currentstroke}%
\pgfsetstrokeopacity{0.000000}%
\pgfsetdash{}{0pt}%
\pgfpathmoveto{\pgfqpoint{3.521740in}{0.588015in}}%
\pgfpathlineto{\pgfqpoint{3.485872in}{0.588015in}}%
\pgfpathlineto{\pgfqpoint{3.485872in}{0.608944in}}%
\pgfpathlineto{\pgfqpoint{3.521740in}{0.608944in}}%
\pgfpathlineto{\pgfqpoint{3.521740in}{0.588015in}}%
\pgfpathclose%
\pgfusepath{fill}%
\end{pgfscope}%
\begin{pgfscope}%
\pgfpathrectangle{\pgfqpoint{3.019583in}{0.169444in}}{\pgfqpoint{0.896708in}{1.339426in}}%
\pgfusepath{clip}%
\pgfsetbuttcap%
\pgfsetmiterjoin%
\definecolor{currentfill}{rgb}{0.993310,0.740023,0.428835}%
\pgfsetfillcolor{currentfill}%
\pgfsetlinewidth{0.000000pt}%
\definecolor{currentstroke}{rgb}{0.000000,0.000000,0.000000}%
\pgfsetstrokecolor{currentstroke}%
\pgfsetstrokeopacity{0.000000}%
\pgfsetdash{}{0pt}%
\pgfpathmoveto{\pgfqpoint{3.521740in}{0.608944in}}%
\pgfpathlineto{\pgfqpoint{3.485872in}{0.608944in}}%
\pgfpathlineto{\pgfqpoint{3.485872in}{0.629872in}}%
\pgfpathlineto{\pgfqpoint{3.521740in}{0.629872in}}%
\pgfpathlineto{\pgfqpoint{3.521740in}{0.608944in}}%
\pgfpathclose%
\pgfusepath{fill}%
\end{pgfscope}%
\begin{pgfscope}%
\pgfpathrectangle{\pgfqpoint{3.019583in}{0.169444in}}{\pgfqpoint{0.896708in}{1.339426in}}%
\pgfusepath{clip}%
\pgfsetbuttcap%
\pgfsetmiterjoin%
\definecolor{currentfill}{rgb}{0.993925,0.770780,0.454671}%
\pgfsetfillcolor{currentfill}%
\pgfsetlinewidth{0.000000pt}%
\definecolor{currentstroke}{rgb}{0.000000,0.000000,0.000000}%
\pgfsetstrokecolor{currentstroke}%
\pgfsetstrokeopacity{0.000000}%
\pgfsetdash{}{0pt}%
\pgfpathmoveto{\pgfqpoint{3.521740in}{0.629872in}}%
\pgfpathlineto{\pgfqpoint{3.485872in}{0.629872in}}%
\pgfpathlineto{\pgfqpoint{3.485872in}{0.650801in}}%
\pgfpathlineto{\pgfqpoint{3.521740in}{0.650801in}}%
\pgfpathlineto{\pgfqpoint{3.521740in}{0.629872in}}%
\pgfpathclose%
\pgfusepath{fill}%
\end{pgfscope}%
\begin{pgfscope}%
\pgfpathrectangle{\pgfqpoint{3.019583in}{0.169444in}}{\pgfqpoint{0.896708in}{1.339426in}}%
\pgfusepath{clip}%
\pgfsetbuttcap%
\pgfsetmiterjoin%
\definecolor{currentfill}{rgb}{0.994541,0.801538,0.480507}%
\pgfsetfillcolor{currentfill}%
\pgfsetlinewidth{0.000000pt}%
\definecolor{currentstroke}{rgb}{0.000000,0.000000,0.000000}%
\pgfsetstrokecolor{currentstroke}%
\pgfsetstrokeopacity{0.000000}%
\pgfsetdash{}{0pt}%
\pgfpathmoveto{\pgfqpoint{3.521740in}{0.650801in}}%
\pgfpathlineto{\pgfqpoint{3.485872in}{0.650801in}}%
\pgfpathlineto{\pgfqpoint{3.485872in}{0.671729in}}%
\pgfpathlineto{\pgfqpoint{3.521740in}{0.671729in}}%
\pgfpathlineto{\pgfqpoint{3.521740in}{0.650801in}}%
\pgfpathclose%
\pgfusepath{fill}%
\end{pgfscope}%
\begin{pgfscope}%
\pgfpathrectangle{\pgfqpoint{3.019583in}{0.169444in}}{\pgfqpoint{0.896708in}{1.339426in}}%
\pgfusepath{clip}%
\pgfsetbuttcap%
\pgfsetmiterjoin%
\definecolor{currentfill}{rgb}{0.995156,0.832295,0.506344}%
\pgfsetfillcolor{currentfill}%
\pgfsetlinewidth{0.000000pt}%
\definecolor{currentstroke}{rgb}{0.000000,0.000000,0.000000}%
\pgfsetstrokecolor{currentstroke}%
\pgfsetstrokeopacity{0.000000}%
\pgfsetdash{}{0pt}%
\pgfpathmoveto{\pgfqpoint{3.521740in}{0.671729in}}%
\pgfpathlineto{\pgfqpoint{3.485872in}{0.671729in}}%
\pgfpathlineto{\pgfqpoint{3.485872in}{0.692658in}}%
\pgfpathlineto{\pgfqpoint{3.521740in}{0.692658in}}%
\pgfpathlineto{\pgfqpoint{3.521740in}{0.671729in}}%
\pgfpathclose%
\pgfusepath{fill}%
\end{pgfscope}%
\begin{pgfscope}%
\pgfpathrectangle{\pgfqpoint{3.019583in}{0.169444in}}{\pgfqpoint{0.896708in}{1.339426in}}%
\pgfusepath{clip}%
\pgfsetbuttcap%
\pgfsetmiterjoin%
\definecolor{currentfill}{rgb}{0.995771,0.863053,0.532180}%
\pgfsetfillcolor{currentfill}%
\pgfsetlinewidth{0.000000pt}%
\definecolor{currentstroke}{rgb}{0.000000,0.000000,0.000000}%
\pgfsetstrokecolor{currentstroke}%
\pgfsetstrokeopacity{0.000000}%
\pgfsetdash{}{0pt}%
\pgfpathmoveto{\pgfqpoint{3.521740in}{0.692658in}}%
\pgfpathlineto{\pgfqpoint{3.485872in}{0.692658in}}%
\pgfpathlineto{\pgfqpoint{3.485872in}{0.713586in}}%
\pgfpathlineto{\pgfqpoint{3.521740in}{0.713586in}}%
\pgfpathlineto{\pgfqpoint{3.521740in}{0.692658in}}%
\pgfpathclose%
\pgfusepath{fill}%
\end{pgfscope}%
\begin{pgfscope}%
\pgfpathrectangle{\pgfqpoint{3.019583in}{0.169444in}}{\pgfqpoint{0.896708in}{1.339426in}}%
\pgfusepath{clip}%
\pgfsetbuttcap%
\pgfsetmiterjoin%
\definecolor{currentfill}{rgb}{0.996386,0.887966,0.561092}%
\pgfsetfillcolor{currentfill}%
\pgfsetlinewidth{0.000000pt}%
\definecolor{currentstroke}{rgb}{0.000000,0.000000,0.000000}%
\pgfsetstrokecolor{currentstroke}%
\pgfsetstrokeopacity{0.000000}%
\pgfsetdash{}{0pt}%
\pgfpathmoveto{\pgfqpoint{3.521740in}{0.713586in}}%
\pgfpathlineto{\pgfqpoint{3.485872in}{0.713586in}}%
\pgfpathlineto{\pgfqpoint{3.485872in}{0.734515in}}%
\pgfpathlineto{\pgfqpoint{3.521740in}{0.734515in}}%
\pgfpathlineto{\pgfqpoint{3.521740in}{0.713586in}}%
\pgfpathclose%
\pgfusepath{fill}%
\end{pgfscope}%
\begin{pgfscope}%
\pgfpathrectangle{\pgfqpoint{3.019583in}{0.169444in}}{\pgfqpoint{0.896708in}{1.339426in}}%
\pgfusepath{clip}%
\pgfsetbuttcap%
\pgfsetmiterjoin%
\definecolor{currentfill}{rgb}{0.997001,0.907036,0.593080}%
\pgfsetfillcolor{currentfill}%
\pgfsetlinewidth{0.000000pt}%
\definecolor{currentstroke}{rgb}{0.000000,0.000000,0.000000}%
\pgfsetstrokecolor{currentstroke}%
\pgfsetstrokeopacity{0.000000}%
\pgfsetdash{}{0pt}%
\pgfpathmoveto{\pgfqpoint{3.521740in}{0.734515in}}%
\pgfpathlineto{\pgfqpoint{3.485872in}{0.734515in}}%
\pgfpathlineto{\pgfqpoint{3.485872in}{0.755443in}}%
\pgfpathlineto{\pgfqpoint{3.521740in}{0.755443in}}%
\pgfpathlineto{\pgfqpoint{3.521740in}{0.734515in}}%
\pgfpathclose%
\pgfusepath{fill}%
\end{pgfscope}%
\begin{pgfscope}%
\pgfpathrectangle{\pgfqpoint{3.019583in}{0.169444in}}{\pgfqpoint{0.896708in}{1.339426in}}%
\pgfusepath{clip}%
\pgfsetbuttcap%
\pgfsetmiterjoin%
\definecolor{currentfill}{rgb}{0.997616,0.926105,0.625067}%
\pgfsetfillcolor{currentfill}%
\pgfsetlinewidth{0.000000pt}%
\definecolor{currentstroke}{rgb}{0.000000,0.000000,0.000000}%
\pgfsetstrokecolor{currentstroke}%
\pgfsetstrokeopacity{0.000000}%
\pgfsetdash{}{0pt}%
\pgfpathmoveto{\pgfqpoint{3.521740in}{0.755443in}}%
\pgfpathlineto{\pgfqpoint{3.485872in}{0.755443in}}%
\pgfpathlineto{\pgfqpoint{3.485872in}{0.776372in}}%
\pgfpathlineto{\pgfqpoint{3.521740in}{0.776372in}}%
\pgfpathlineto{\pgfqpoint{3.521740in}{0.755443in}}%
\pgfpathclose%
\pgfusepath{fill}%
\end{pgfscope}%
\begin{pgfscope}%
\pgfpathrectangle{\pgfqpoint{3.019583in}{0.169444in}}{\pgfqpoint{0.896708in}{1.339426in}}%
\pgfusepath{clip}%
\pgfsetbuttcap%
\pgfsetmiterjoin%
\definecolor{currentfill}{rgb}{0.998231,0.945175,0.657055}%
\pgfsetfillcolor{currentfill}%
\pgfsetlinewidth{0.000000pt}%
\definecolor{currentstroke}{rgb}{0.000000,0.000000,0.000000}%
\pgfsetstrokecolor{currentstroke}%
\pgfsetstrokeopacity{0.000000}%
\pgfsetdash{}{0pt}%
\pgfpathmoveto{\pgfqpoint{3.521740in}{0.776372in}}%
\pgfpathlineto{\pgfqpoint{3.485872in}{0.776372in}}%
\pgfpathlineto{\pgfqpoint{3.485872in}{0.797300in}}%
\pgfpathlineto{\pgfqpoint{3.521740in}{0.797300in}}%
\pgfpathlineto{\pgfqpoint{3.521740in}{0.776372in}}%
\pgfpathclose%
\pgfusepath{fill}%
\end{pgfscope}%
\begin{pgfscope}%
\pgfpathrectangle{\pgfqpoint{3.019583in}{0.169444in}}{\pgfqpoint{0.896708in}{1.339426in}}%
\pgfusepath{clip}%
\pgfsetbuttcap%
\pgfsetmiterjoin%
\definecolor{currentfill}{rgb}{0.998847,0.964245,0.689043}%
\pgfsetfillcolor{currentfill}%
\pgfsetlinewidth{0.000000pt}%
\definecolor{currentstroke}{rgb}{0.000000,0.000000,0.000000}%
\pgfsetstrokecolor{currentstroke}%
\pgfsetstrokeopacity{0.000000}%
\pgfsetdash{}{0pt}%
\pgfpathmoveto{\pgfqpoint{3.521740in}{0.797300in}}%
\pgfpathlineto{\pgfqpoint{3.485872in}{0.797300in}}%
\pgfpathlineto{\pgfqpoint{3.485872in}{0.818229in}}%
\pgfpathlineto{\pgfqpoint{3.521740in}{0.818229in}}%
\pgfpathlineto{\pgfqpoint{3.521740in}{0.797300in}}%
\pgfpathclose%
\pgfusepath{fill}%
\end{pgfscope}%
\begin{pgfscope}%
\pgfpathrectangle{\pgfqpoint{3.019583in}{0.169444in}}{\pgfqpoint{0.896708in}{1.339426in}}%
\pgfusepath{clip}%
\pgfsetbuttcap%
\pgfsetmiterjoin%
\definecolor{currentfill}{rgb}{0.999462,0.983314,0.721030}%
\pgfsetfillcolor{currentfill}%
\pgfsetlinewidth{0.000000pt}%
\definecolor{currentstroke}{rgb}{0.000000,0.000000,0.000000}%
\pgfsetstrokecolor{currentstroke}%
\pgfsetstrokeopacity{0.000000}%
\pgfsetdash{}{0pt}%
\pgfpathmoveto{\pgfqpoint{3.521740in}{0.818229in}}%
\pgfpathlineto{\pgfqpoint{3.485872in}{0.818229in}}%
\pgfpathlineto{\pgfqpoint{3.485872in}{0.839157in}}%
\pgfpathlineto{\pgfqpoint{3.521740in}{0.839157in}}%
\pgfpathlineto{\pgfqpoint{3.521740in}{0.818229in}}%
\pgfpathclose%
\pgfusepath{fill}%
\end{pgfscope}%
\begin{pgfscope}%
\pgfpathrectangle{\pgfqpoint{3.019583in}{0.169444in}}{\pgfqpoint{0.896708in}{1.339426in}}%
\pgfusepath{clip}%
\pgfsetbuttcap%
\pgfsetmiterjoin%
\definecolor{currentfill}{rgb}{0.998078,0.999231,0.746021}%
\pgfsetfillcolor{currentfill}%
\pgfsetlinewidth{0.000000pt}%
\definecolor{currentstroke}{rgb}{0.000000,0.000000,0.000000}%
\pgfsetstrokecolor{currentstroke}%
\pgfsetstrokeopacity{0.000000}%
\pgfsetdash{}{0pt}%
\pgfpathmoveto{\pgfqpoint{3.521740in}{0.839157in}}%
\pgfpathlineto{\pgfqpoint{3.485872in}{0.839157in}}%
\pgfpathlineto{\pgfqpoint{3.485872in}{0.860086in}}%
\pgfpathlineto{\pgfqpoint{3.521740in}{0.860086in}}%
\pgfpathlineto{\pgfqpoint{3.521740in}{0.839157in}}%
\pgfpathclose%
\pgfusepath{fill}%
\end{pgfscope}%
\begin{pgfscope}%
\pgfpathrectangle{\pgfqpoint{3.019583in}{0.169444in}}{\pgfqpoint{0.896708in}{1.339426in}}%
\pgfusepath{clip}%
\pgfsetbuttcap%
\pgfsetmiterjoin%
\definecolor{currentfill}{rgb}{0.982699,0.993080,0.722030}%
\pgfsetfillcolor{currentfill}%
\pgfsetlinewidth{0.000000pt}%
\definecolor{currentstroke}{rgb}{0.000000,0.000000,0.000000}%
\pgfsetstrokecolor{currentstroke}%
\pgfsetstrokeopacity{0.000000}%
\pgfsetdash{}{0pt}%
\pgfpathmoveto{\pgfqpoint{3.521740in}{0.860086in}}%
\pgfpathlineto{\pgfqpoint{3.485872in}{0.860086in}}%
\pgfpathlineto{\pgfqpoint{3.485872in}{0.881015in}}%
\pgfpathlineto{\pgfqpoint{3.521740in}{0.881015in}}%
\pgfpathlineto{\pgfqpoint{3.521740in}{0.860086in}}%
\pgfpathclose%
\pgfusepath{fill}%
\end{pgfscope}%
\begin{pgfscope}%
\pgfpathrectangle{\pgfqpoint{3.019583in}{0.169444in}}{\pgfqpoint{0.896708in}{1.339426in}}%
\pgfusepath{clip}%
\pgfsetbuttcap%
\pgfsetmiterjoin%
\definecolor{currentfill}{rgb}{0.967320,0.986928,0.698039}%
\pgfsetfillcolor{currentfill}%
\pgfsetlinewidth{0.000000pt}%
\definecolor{currentstroke}{rgb}{0.000000,0.000000,0.000000}%
\pgfsetstrokecolor{currentstroke}%
\pgfsetstrokeopacity{0.000000}%
\pgfsetdash{}{0pt}%
\pgfpathmoveto{\pgfqpoint{3.521740in}{0.881015in}}%
\pgfpathlineto{\pgfqpoint{3.485872in}{0.881015in}}%
\pgfpathlineto{\pgfqpoint{3.485872in}{0.901943in}}%
\pgfpathlineto{\pgfqpoint{3.521740in}{0.901943in}}%
\pgfpathlineto{\pgfqpoint{3.521740in}{0.881015in}}%
\pgfpathclose%
\pgfusepath{fill}%
\end{pgfscope}%
\begin{pgfscope}%
\pgfpathrectangle{\pgfqpoint{3.019583in}{0.169444in}}{\pgfqpoint{0.896708in}{1.339426in}}%
\pgfusepath{clip}%
\pgfsetbuttcap%
\pgfsetmiterjoin%
\definecolor{currentfill}{rgb}{0.951942,0.980777,0.674048}%
\pgfsetfillcolor{currentfill}%
\pgfsetlinewidth{0.000000pt}%
\definecolor{currentstroke}{rgb}{0.000000,0.000000,0.000000}%
\pgfsetstrokecolor{currentstroke}%
\pgfsetstrokeopacity{0.000000}%
\pgfsetdash{}{0pt}%
\pgfpathmoveto{\pgfqpoint{3.521740in}{0.901943in}}%
\pgfpathlineto{\pgfqpoint{3.485872in}{0.901943in}}%
\pgfpathlineto{\pgfqpoint{3.485872in}{0.922872in}}%
\pgfpathlineto{\pgfqpoint{3.521740in}{0.922872in}}%
\pgfpathlineto{\pgfqpoint{3.521740in}{0.901943in}}%
\pgfpathclose%
\pgfusepath{fill}%
\end{pgfscope}%
\begin{pgfscope}%
\pgfpathrectangle{\pgfqpoint{3.019583in}{0.169444in}}{\pgfqpoint{0.896708in}{1.339426in}}%
\pgfusepath{clip}%
\pgfsetbuttcap%
\pgfsetmiterjoin%
\definecolor{currentfill}{rgb}{0.936563,0.974625,0.650058}%
\pgfsetfillcolor{currentfill}%
\pgfsetlinewidth{0.000000pt}%
\definecolor{currentstroke}{rgb}{0.000000,0.000000,0.000000}%
\pgfsetstrokecolor{currentstroke}%
\pgfsetstrokeopacity{0.000000}%
\pgfsetdash{}{0pt}%
\pgfpathmoveto{\pgfqpoint{3.521740in}{0.922872in}}%
\pgfpathlineto{\pgfqpoint{3.485872in}{0.922872in}}%
\pgfpathlineto{\pgfqpoint{3.485872in}{0.943800in}}%
\pgfpathlineto{\pgfqpoint{3.521740in}{0.943800in}}%
\pgfpathlineto{\pgfqpoint{3.521740in}{0.922872in}}%
\pgfpathclose%
\pgfusepath{fill}%
\end{pgfscope}%
\begin{pgfscope}%
\pgfpathrectangle{\pgfqpoint{3.019583in}{0.169444in}}{\pgfqpoint{0.896708in}{1.339426in}}%
\pgfusepath{clip}%
\pgfsetbuttcap%
\pgfsetmiterjoin%
\definecolor{currentfill}{rgb}{0.921184,0.968474,0.626067}%
\pgfsetfillcolor{currentfill}%
\pgfsetlinewidth{0.000000pt}%
\definecolor{currentstroke}{rgb}{0.000000,0.000000,0.000000}%
\pgfsetstrokecolor{currentstroke}%
\pgfsetstrokeopacity{0.000000}%
\pgfsetdash{}{0pt}%
\pgfpathmoveto{\pgfqpoint{3.521740in}{0.943800in}}%
\pgfpathlineto{\pgfqpoint{3.485872in}{0.943800in}}%
\pgfpathlineto{\pgfqpoint{3.485872in}{0.964729in}}%
\pgfpathlineto{\pgfqpoint{3.521740in}{0.964729in}}%
\pgfpathlineto{\pgfqpoint{3.521740in}{0.943800in}}%
\pgfpathclose%
\pgfusepath{fill}%
\end{pgfscope}%
\begin{pgfscope}%
\pgfpathrectangle{\pgfqpoint{3.019583in}{0.169444in}}{\pgfqpoint{0.896708in}{1.339426in}}%
\pgfusepath{clip}%
\pgfsetbuttcap%
\pgfsetmiterjoin%
\definecolor{currentfill}{rgb}{0.905805,0.962322,0.602076}%
\pgfsetfillcolor{currentfill}%
\pgfsetlinewidth{0.000000pt}%
\definecolor{currentstroke}{rgb}{0.000000,0.000000,0.000000}%
\pgfsetstrokecolor{currentstroke}%
\pgfsetstrokeopacity{0.000000}%
\pgfsetdash{}{0pt}%
\pgfpathmoveto{\pgfqpoint{3.521740in}{0.964729in}}%
\pgfpathlineto{\pgfqpoint{3.485872in}{0.964729in}}%
\pgfpathlineto{\pgfqpoint{3.485872in}{0.985657in}}%
\pgfpathlineto{\pgfqpoint{3.521740in}{0.985657in}}%
\pgfpathlineto{\pgfqpoint{3.521740in}{0.964729in}}%
\pgfpathclose%
\pgfusepath{fill}%
\end{pgfscope}%
\begin{pgfscope}%
\pgfpathrectangle{\pgfqpoint{3.019583in}{0.169444in}}{\pgfqpoint{0.896708in}{1.339426in}}%
\pgfusepath{clip}%
\pgfsetbuttcap%
\pgfsetmiterjoin%
\definecolor{currentfill}{rgb}{0.874740,0.949712,0.601615}%
\pgfsetfillcolor{currentfill}%
\pgfsetlinewidth{0.000000pt}%
\definecolor{currentstroke}{rgb}{0.000000,0.000000,0.000000}%
\pgfsetstrokecolor{currentstroke}%
\pgfsetstrokeopacity{0.000000}%
\pgfsetdash{}{0pt}%
\pgfpathmoveto{\pgfqpoint{3.521740in}{0.985657in}}%
\pgfpathlineto{\pgfqpoint{3.485872in}{0.985657in}}%
\pgfpathlineto{\pgfqpoint{3.485872in}{1.006586in}}%
\pgfpathlineto{\pgfqpoint{3.521740in}{1.006586in}}%
\pgfpathlineto{\pgfqpoint{3.521740in}{0.985657in}}%
\pgfpathclose%
\pgfusepath{fill}%
\end{pgfscope}%
\begin{pgfscope}%
\pgfpathrectangle{\pgfqpoint{3.019583in}{0.169444in}}{\pgfqpoint{0.896708in}{1.339426in}}%
\pgfusepath{clip}%
\pgfsetbuttcap%
\pgfsetmiterjoin%
\definecolor{currentfill}{rgb}{0.838447,0.934948,0.608997}%
\pgfsetfillcolor{currentfill}%
\pgfsetlinewidth{0.000000pt}%
\definecolor{currentstroke}{rgb}{0.000000,0.000000,0.000000}%
\pgfsetstrokecolor{currentstroke}%
\pgfsetstrokeopacity{0.000000}%
\pgfsetdash{}{0pt}%
\pgfpathmoveto{\pgfqpoint{3.521740in}{1.006586in}}%
\pgfpathlineto{\pgfqpoint{3.485872in}{1.006586in}}%
\pgfpathlineto{\pgfqpoint{3.485872in}{1.027514in}}%
\pgfpathlineto{\pgfqpoint{3.521740in}{1.027514in}}%
\pgfpathlineto{\pgfqpoint{3.521740in}{1.006586in}}%
\pgfpathclose%
\pgfusepath{fill}%
\end{pgfscope}%
\begin{pgfscope}%
\pgfpathrectangle{\pgfqpoint{3.019583in}{0.169444in}}{\pgfqpoint{0.896708in}{1.339426in}}%
\pgfusepath{clip}%
\pgfsetbuttcap%
\pgfsetmiterjoin%
\definecolor{currentfill}{rgb}{0.802153,0.920185,0.616378}%
\pgfsetfillcolor{currentfill}%
\pgfsetlinewidth{0.000000pt}%
\definecolor{currentstroke}{rgb}{0.000000,0.000000,0.000000}%
\pgfsetstrokecolor{currentstroke}%
\pgfsetstrokeopacity{0.000000}%
\pgfsetdash{}{0pt}%
\pgfpathmoveto{\pgfqpoint{3.521740in}{1.027514in}}%
\pgfpathlineto{\pgfqpoint{3.485872in}{1.027514in}}%
\pgfpathlineto{\pgfqpoint{3.485872in}{1.048443in}}%
\pgfpathlineto{\pgfqpoint{3.521740in}{1.048443in}}%
\pgfpathlineto{\pgfqpoint{3.521740in}{1.027514in}}%
\pgfpathclose%
\pgfusepath{fill}%
\end{pgfscope}%
\begin{pgfscope}%
\pgfpathrectangle{\pgfqpoint{3.019583in}{0.169444in}}{\pgfqpoint{0.896708in}{1.339426in}}%
\pgfusepath{clip}%
\pgfsetbuttcap%
\pgfsetmiterjoin%
\definecolor{currentfill}{rgb}{0.765859,0.905421,0.623760}%
\pgfsetfillcolor{currentfill}%
\pgfsetlinewidth{0.000000pt}%
\definecolor{currentstroke}{rgb}{0.000000,0.000000,0.000000}%
\pgfsetstrokecolor{currentstroke}%
\pgfsetstrokeopacity{0.000000}%
\pgfsetdash{}{0pt}%
\pgfpathmoveto{\pgfqpoint{3.521740in}{1.048443in}}%
\pgfpathlineto{\pgfqpoint{3.485872in}{1.048443in}}%
\pgfpathlineto{\pgfqpoint{3.485872in}{1.069371in}}%
\pgfpathlineto{\pgfqpoint{3.521740in}{1.069371in}}%
\pgfpathlineto{\pgfqpoint{3.521740in}{1.048443in}}%
\pgfpathclose%
\pgfusepath{fill}%
\end{pgfscope}%
\begin{pgfscope}%
\pgfpathrectangle{\pgfqpoint{3.019583in}{0.169444in}}{\pgfqpoint{0.896708in}{1.339426in}}%
\pgfusepath{clip}%
\pgfsetbuttcap%
\pgfsetmiterjoin%
\definecolor{currentfill}{rgb}{0.729566,0.890657,0.631142}%
\pgfsetfillcolor{currentfill}%
\pgfsetlinewidth{0.000000pt}%
\definecolor{currentstroke}{rgb}{0.000000,0.000000,0.000000}%
\pgfsetstrokecolor{currentstroke}%
\pgfsetstrokeopacity{0.000000}%
\pgfsetdash{}{0pt}%
\pgfpathmoveto{\pgfqpoint{3.521740in}{1.069371in}}%
\pgfpathlineto{\pgfqpoint{3.485872in}{1.069371in}}%
\pgfpathlineto{\pgfqpoint{3.485872in}{1.090300in}}%
\pgfpathlineto{\pgfqpoint{3.521740in}{1.090300in}}%
\pgfpathlineto{\pgfqpoint{3.521740in}{1.069371in}}%
\pgfpathclose%
\pgfusepath{fill}%
\end{pgfscope}%
\begin{pgfscope}%
\pgfpathrectangle{\pgfqpoint{3.019583in}{0.169444in}}{\pgfqpoint{0.896708in}{1.339426in}}%
\pgfusepath{clip}%
\pgfsetbuttcap%
\pgfsetmiterjoin%
\definecolor{currentfill}{rgb}{0.693272,0.875894,0.638524}%
\pgfsetfillcolor{currentfill}%
\pgfsetlinewidth{0.000000pt}%
\definecolor{currentstroke}{rgb}{0.000000,0.000000,0.000000}%
\pgfsetstrokecolor{currentstroke}%
\pgfsetstrokeopacity{0.000000}%
\pgfsetdash{}{0pt}%
\pgfpathmoveto{\pgfqpoint{3.521740in}{1.090300in}}%
\pgfpathlineto{\pgfqpoint{3.485872in}{1.090300in}}%
\pgfpathlineto{\pgfqpoint{3.485872in}{1.111228in}}%
\pgfpathlineto{\pgfqpoint{3.521740in}{1.111228in}}%
\pgfpathlineto{\pgfqpoint{3.521740in}{1.090300in}}%
\pgfpathclose%
\pgfusepath{fill}%
\end{pgfscope}%
\begin{pgfscope}%
\pgfpathrectangle{\pgfqpoint{3.019583in}{0.169444in}}{\pgfqpoint{0.896708in}{1.339426in}}%
\pgfusepath{clip}%
\pgfsetbuttcap%
\pgfsetmiterjoin%
\definecolor{currentfill}{rgb}{0.654671,0.860438,0.643368}%
\pgfsetfillcolor{currentfill}%
\pgfsetlinewidth{0.000000pt}%
\definecolor{currentstroke}{rgb}{0.000000,0.000000,0.000000}%
\pgfsetstrokecolor{currentstroke}%
\pgfsetstrokeopacity{0.000000}%
\pgfsetdash{}{0pt}%
\pgfpathmoveto{\pgfqpoint{3.521740in}{1.111228in}}%
\pgfpathlineto{\pgfqpoint{3.485872in}{1.111228in}}%
\pgfpathlineto{\pgfqpoint{3.485872in}{1.132157in}}%
\pgfpathlineto{\pgfqpoint{3.521740in}{1.132157in}}%
\pgfpathlineto{\pgfqpoint{3.521740in}{1.111228in}}%
\pgfpathclose%
\pgfusepath{fill}%
\end{pgfscope}%
\begin{pgfscope}%
\pgfpathrectangle{\pgfqpoint{3.019583in}{0.169444in}}{\pgfqpoint{0.896708in}{1.339426in}}%
\pgfusepath{clip}%
\pgfsetbuttcap%
\pgfsetmiterjoin%
\definecolor{currentfill}{rgb}{0.612226,0.843829,0.643983}%
\pgfsetfillcolor{currentfill}%
\pgfsetlinewidth{0.000000pt}%
\definecolor{currentstroke}{rgb}{0.000000,0.000000,0.000000}%
\pgfsetstrokecolor{currentstroke}%
\pgfsetstrokeopacity{0.000000}%
\pgfsetdash{}{0pt}%
\pgfpathmoveto{\pgfqpoint{3.521740in}{1.132157in}}%
\pgfpathlineto{\pgfqpoint{3.485872in}{1.132157in}}%
\pgfpathlineto{\pgfqpoint{3.485872in}{1.153085in}}%
\pgfpathlineto{\pgfqpoint{3.521740in}{1.153085in}}%
\pgfpathlineto{\pgfqpoint{3.521740in}{1.132157in}}%
\pgfpathclose%
\pgfusepath{fill}%
\end{pgfscope}%
\begin{pgfscope}%
\pgfpathrectangle{\pgfqpoint{3.019583in}{0.169444in}}{\pgfqpoint{0.896708in}{1.339426in}}%
\pgfusepath{clip}%
\pgfsetbuttcap%
\pgfsetmiterjoin%
\definecolor{currentfill}{rgb}{0.569781,0.827220,0.644598}%
\pgfsetfillcolor{currentfill}%
\pgfsetlinewidth{0.000000pt}%
\definecolor{currentstroke}{rgb}{0.000000,0.000000,0.000000}%
\pgfsetstrokecolor{currentstroke}%
\pgfsetstrokeopacity{0.000000}%
\pgfsetdash{}{0pt}%
\pgfpathmoveto{\pgfqpoint{3.521740in}{1.153085in}}%
\pgfpathlineto{\pgfqpoint{3.485872in}{1.153085in}}%
\pgfpathlineto{\pgfqpoint{3.485872in}{1.174014in}}%
\pgfpathlineto{\pgfqpoint{3.521740in}{1.174014in}}%
\pgfpathlineto{\pgfqpoint{3.521740in}{1.153085in}}%
\pgfpathclose%
\pgfusepath{fill}%
\end{pgfscope}%
\begin{pgfscope}%
\pgfpathrectangle{\pgfqpoint{3.019583in}{0.169444in}}{\pgfqpoint{0.896708in}{1.339426in}}%
\pgfusepath{clip}%
\pgfsetbuttcap%
\pgfsetmiterjoin%
\definecolor{currentfill}{rgb}{0.527336,0.810611,0.645213}%
\pgfsetfillcolor{currentfill}%
\pgfsetlinewidth{0.000000pt}%
\definecolor{currentstroke}{rgb}{0.000000,0.000000,0.000000}%
\pgfsetstrokecolor{currentstroke}%
\pgfsetstrokeopacity{0.000000}%
\pgfsetdash{}{0pt}%
\pgfpathmoveto{\pgfqpoint{3.521740in}{1.174014in}}%
\pgfpathlineto{\pgfqpoint{3.485872in}{1.174014in}}%
\pgfpathlineto{\pgfqpoint{3.485872in}{1.194943in}}%
\pgfpathlineto{\pgfqpoint{3.521740in}{1.194943in}}%
\pgfpathlineto{\pgfqpoint{3.521740in}{1.174014in}}%
\pgfpathclose%
\pgfusepath{fill}%
\end{pgfscope}%
\begin{pgfscope}%
\pgfpathrectangle{\pgfqpoint{3.019583in}{0.169444in}}{\pgfqpoint{0.896708in}{1.339426in}}%
\pgfusepath{clip}%
\pgfsetbuttcap%
\pgfsetmiterjoin%
\definecolor{currentfill}{rgb}{0.484890,0.794002,0.645829}%
\pgfsetfillcolor{currentfill}%
\pgfsetlinewidth{0.000000pt}%
\definecolor{currentstroke}{rgb}{0.000000,0.000000,0.000000}%
\pgfsetstrokecolor{currentstroke}%
\pgfsetstrokeopacity{0.000000}%
\pgfsetdash{}{0pt}%
\pgfpathmoveto{\pgfqpoint{3.521740in}{1.194943in}}%
\pgfpathlineto{\pgfqpoint{3.485872in}{1.194943in}}%
\pgfpathlineto{\pgfqpoint{3.485872in}{1.215871in}}%
\pgfpathlineto{\pgfqpoint{3.521740in}{1.215871in}}%
\pgfpathlineto{\pgfqpoint{3.521740in}{1.194943in}}%
\pgfpathclose%
\pgfusepath{fill}%
\end{pgfscope}%
\begin{pgfscope}%
\pgfpathrectangle{\pgfqpoint{3.019583in}{0.169444in}}{\pgfqpoint{0.896708in}{1.339426in}}%
\pgfusepath{clip}%
\pgfsetbuttcap%
\pgfsetmiterjoin%
\definecolor{currentfill}{rgb}{0.442445,0.777393,0.646444}%
\pgfsetfillcolor{currentfill}%
\pgfsetlinewidth{0.000000pt}%
\definecolor{currentstroke}{rgb}{0.000000,0.000000,0.000000}%
\pgfsetstrokecolor{currentstroke}%
\pgfsetstrokeopacity{0.000000}%
\pgfsetdash{}{0pt}%
\pgfpathmoveto{\pgfqpoint{3.521740in}{1.215871in}}%
\pgfpathlineto{\pgfqpoint{3.485872in}{1.215871in}}%
\pgfpathlineto{\pgfqpoint{3.485872in}{1.236800in}}%
\pgfpathlineto{\pgfqpoint{3.521740in}{1.236800in}}%
\pgfpathlineto{\pgfqpoint{3.521740in}{1.215871in}}%
\pgfpathclose%
\pgfusepath{fill}%
\end{pgfscope}%
\begin{pgfscope}%
\pgfpathrectangle{\pgfqpoint{3.019583in}{0.169444in}}{\pgfqpoint{0.896708in}{1.339426in}}%
\pgfusepath{clip}%
\pgfsetbuttcap%
\pgfsetmiterjoin%
\definecolor{currentfill}{rgb}{0.400000,0.760784,0.647059}%
\pgfsetfillcolor{currentfill}%
\pgfsetlinewidth{0.000000pt}%
\definecolor{currentstroke}{rgb}{0.000000,0.000000,0.000000}%
\pgfsetstrokecolor{currentstroke}%
\pgfsetstrokeopacity{0.000000}%
\pgfsetdash{}{0pt}%
\pgfpathmoveto{\pgfqpoint{3.521740in}{1.236800in}}%
\pgfpathlineto{\pgfqpoint{3.485872in}{1.236800in}}%
\pgfpathlineto{\pgfqpoint{3.485872in}{1.257728in}}%
\pgfpathlineto{\pgfqpoint{3.521740in}{1.257728in}}%
\pgfpathlineto{\pgfqpoint{3.521740in}{1.236800in}}%
\pgfpathclose%
\pgfusepath{fill}%
\end{pgfscope}%
\begin{pgfscope}%
\pgfpathrectangle{\pgfqpoint{3.019583in}{0.169444in}}{\pgfqpoint{0.896708in}{1.339426in}}%
\pgfusepath{clip}%
\pgfsetbuttcap%
\pgfsetmiterjoin%
\definecolor{currentfill}{rgb}{0.368012,0.725106,0.661822}%
\pgfsetfillcolor{currentfill}%
\pgfsetlinewidth{0.000000pt}%
\definecolor{currentstroke}{rgb}{0.000000,0.000000,0.000000}%
\pgfsetstrokecolor{currentstroke}%
\pgfsetstrokeopacity{0.000000}%
\pgfsetdash{}{0pt}%
\pgfpathmoveto{\pgfqpoint{3.521740in}{1.257728in}}%
\pgfpathlineto{\pgfqpoint{3.485872in}{1.257728in}}%
\pgfpathlineto{\pgfqpoint{3.485872in}{1.278657in}}%
\pgfpathlineto{\pgfqpoint{3.521740in}{1.278657in}}%
\pgfpathlineto{\pgfqpoint{3.521740in}{1.257728in}}%
\pgfpathclose%
\pgfusepath{fill}%
\end{pgfscope}%
\begin{pgfscope}%
\pgfpathrectangle{\pgfqpoint{3.019583in}{0.169444in}}{\pgfqpoint{0.896708in}{1.339426in}}%
\pgfusepath{clip}%
\pgfsetbuttcap%
\pgfsetmiterjoin%
\definecolor{currentfill}{rgb}{0.336025,0.689427,0.676586}%
\pgfsetfillcolor{currentfill}%
\pgfsetlinewidth{0.000000pt}%
\definecolor{currentstroke}{rgb}{0.000000,0.000000,0.000000}%
\pgfsetstrokecolor{currentstroke}%
\pgfsetstrokeopacity{0.000000}%
\pgfsetdash{}{0pt}%
\pgfpathmoveto{\pgfqpoint{3.521740in}{1.278657in}}%
\pgfpathlineto{\pgfqpoint{3.485872in}{1.278657in}}%
\pgfpathlineto{\pgfqpoint{3.485872in}{1.299585in}}%
\pgfpathlineto{\pgfqpoint{3.521740in}{1.299585in}}%
\pgfpathlineto{\pgfqpoint{3.521740in}{1.278657in}}%
\pgfpathclose%
\pgfusepath{fill}%
\end{pgfscope}%
\begin{pgfscope}%
\pgfpathrectangle{\pgfqpoint{3.019583in}{0.169444in}}{\pgfqpoint{0.896708in}{1.339426in}}%
\pgfusepath{clip}%
\pgfsetbuttcap%
\pgfsetmiterjoin%
\definecolor{currentfill}{rgb}{0.304037,0.653749,0.691349}%
\pgfsetfillcolor{currentfill}%
\pgfsetlinewidth{0.000000pt}%
\definecolor{currentstroke}{rgb}{0.000000,0.000000,0.000000}%
\pgfsetstrokecolor{currentstroke}%
\pgfsetstrokeopacity{0.000000}%
\pgfsetdash{}{0pt}%
\pgfpathmoveto{\pgfqpoint{3.521740in}{1.299585in}}%
\pgfpathlineto{\pgfqpoint{3.485872in}{1.299585in}}%
\pgfpathlineto{\pgfqpoint{3.485872in}{1.320514in}}%
\pgfpathlineto{\pgfqpoint{3.521740in}{1.320514in}}%
\pgfpathlineto{\pgfqpoint{3.521740in}{1.299585in}}%
\pgfpathclose%
\pgfusepath{fill}%
\end{pgfscope}%
\begin{pgfscope}%
\pgfpathrectangle{\pgfqpoint{3.019583in}{0.169444in}}{\pgfqpoint{0.896708in}{1.339426in}}%
\pgfusepath{clip}%
\pgfsetbuttcap%
\pgfsetmiterjoin%
\definecolor{currentfill}{rgb}{0.272049,0.618070,0.706113}%
\pgfsetfillcolor{currentfill}%
\pgfsetlinewidth{0.000000pt}%
\definecolor{currentstroke}{rgb}{0.000000,0.000000,0.000000}%
\pgfsetstrokecolor{currentstroke}%
\pgfsetstrokeopacity{0.000000}%
\pgfsetdash{}{0pt}%
\pgfpathmoveto{\pgfqpoint{3.521740in}{1.320514in}}%
\pgfpathlineto{\pgfqpoint{3.485872in}{1.320514in}}%
\pgfpathlineto{\pgfqpoint{3.485872in}{1.341442in}}%
\pgfpathlineto{\pgfqpoint{3.521740in}{1.341442in}}%
\pgfpathlineto{\pgfqpoint{3.521740in}{1.320514in}}%
\pgfpathclose%
\pgfusepath{fill}%
\end{pgfscope}%
\begin{pgfscope}%
\pgfpathrectangle{\pgfqpoint{3.019583in}{0.169444in}}{\pgfqpoint{0.896708in}{1.339426in}}%
\pgfusepath{clip}%
\pgfsetbuttcap%
\pgfsetmiterjoin%
\definecolor{currentfill}{rgb}{0.240062,0.582391,0.720877}%
\pgfsetfillcolor{currentfill}%
\pgfsetlinewidth{0.000000pt}%
\definecolor{currentstroke}{rgb}{0.000000,0.000000,0.000000}%
\pgfsetstrokecolor{currentstroke}%
\pgfsetstrokeopacity{0.000000}%
\pgfsetdash{}{0pt}%
\pgfpathmoveto{\pgfqpoint{3.521740in}{1.341442in}}%
\pgfpathlineto{\pgfqpoint{3.485872in}{1.341442in}}%
\pgfpathlineto{\pgfqpoint{3.485872in}{1.362371in}}%
\pgfpathlineto{\pgfqpoint{3.521740in}{1.362371in}}%
\pgfpathlineto{\pgfqpoint{3.521740in}{1.341442in}}%
\pgfpathclose%
\pgfusepath{fill}%
\end{pgfscope}%
\begin{pgfscope}%
\pgfpathrectangle{\pgfqpoint{3.019583in}{0.169444in}}{\pgfqpoint{0.896708in}{1.339426in}}%
\pgfusepath{clip}%
\pgfsetbuttcap%
\pgfsetmiterjoin%
\definecolor{currentfill}{rgb}{0.208074,0.546713,0.735640}%
\pgfsetfillcolor{currentfill}%
\pgfsetlinewidth{0.000000pt}%
\definecolor{currentstroke}{rgb}{0.000000,0.000000,0.000000}%
\pgfsetstrokecolor{currentstroke}%
\pgfsetstrokeopacity{0.000000}%
\pgfsetdash{}{0pt}%
\pgfpathmoveto{\pgfqpoint{3.521740in}{1.362371in}}%
\pgfpathlineto{\pgfqpoint{3.485872in}{1.362371in}}%
\pgfpathlineto{\pgfqpoint{3.485872in}{1.383299in}}%
\pgfpathlineto{\pgfqpoint{3.521740in}{1.383299in}}%
\pgfpathlineto{\pgfqpoint{3.521740in}{1.362371in}}%
\pgfpathclose%
\pgfusepath{fill}%
\end{pgfscope}%
\begin{pgfscope}%
\pgfpathrectangle{\pgfqpoint{3.019583in}{0.169444in}}{\pgfqpoint{0.896708in}{1.339426in}}%
\pgfusepath{clip}%
\pgfsetbuttcap%
\pgfsetmiterjoin%
\definecolor{currentfill}{rgb}{0.212995,0.511419,0.730796}%
\pgfsetfillcolor{currentfill}%
\pgfsetlinewidth{0.000000pt}%
\definecolor{currentstroke}{rgb}{0.000000,0.000000,0.000000}%
\pgfsetstrokecolor{currentstroke}%
\pgfsetstrokeopacity{0.000000}%
\pgfsetdash{}{0pt}%
\pgfpathmoveto{\pgfqpoint{3.521740in}{1.383299in}}%
\pgfpathlineto{\pgfqpoint{3.485872in}{1.383299in}}%
\pgfpathlineto{\pgfqpoint{3.485872in}{1.404228in}}%
\pgfpathlineto{\pgfqpoint{3.521740in}{1.404228in}}%
\pgfpathlineto{\pgfqpoint{3.521740in}{1.383299in}}%
\pgfpathclose%
\pgfusepath{fill}%
\end{pgfscope}%
\begin{pgfscope}%
\pgfpathrectangle{\pgfqpoint{3.019583in}{0.169444in}}{\pgfqpoint{0.896708in}{1.339426in}}%
\pgfusepath{clip}%
\pgfsetbuttcap%
\pgfsetmiterjoin%
\definecolor{currentfill}{rgb}{0.240062,0.476355,0.714187}%
\pgfsetfillcolor{currentfill}%
\pgfsetlinewidth{0.000000pt}%
\definecolor{currentstroke}{rgb}{0.000000,0.000000,0.000000}%
\pgfsetstrokecolor{currentstroke}%
\pgfsetstrokeopacity{0.000000}%
\pgfsetdash{}{0pt}%
\pgfpathmoveto{\pgfqpoint{3.521740in}{1.404228in}}%
\pgfpathlineto{\pgfqpoint{3.485872in}{1.404228in}}%
\pgfpathlineto{\pgfqpoint{3.485872in}{1.425156in}}%
\pgfpathlineto{\pgfqpoint{3.521740in}{1.425156in}}%
\pgfpathlineto{\pgfqpoint{3.521740in}{1.404228in}}%
\pgfpathclose%
\pgfusepath{fill}%
\end{pgfscope}%
\begin{pgfscope}%
\pgfpathrectangle{\pgfqpoint{3.019583in}{0.169444in}}{\pgfqpoint{0.896708in}{1.339426in}}%
\pgfusepath{clip}%
\pgfsetbuttcap%
\pgfsetmiterjoin%
\definecolor{currentfill}{rgb}{0.267128,0.441292,0.697578}%
\pgfsetfillcolor{currentfill}%
\pgfsetlinewidth{0.000000pt}%
\definecolor{currentstroke}{rgb}{0.000000,0.000000,0.000000}%
\pgfsetstrokecolor{currentstroke}%
\pgfsetstrokeopacity{0.000000}%
\pgfsetdash{}{0pt}%
\pgfpathmoveto{\pgfqpoint{3.521740in}{1.425156in}}%
\pgfpathlineto{\pgfqpoint{3.485872in}{1.425156in}}%
\pgfpathlineto{\pgfqpoint{3.485872in}{1.446085in}}%
\pgfpathlineto{\pgfqpoint{3.521740in}{1.446085in}}%
\pgfpathlineto{\pgfqpoint{3.521740in}{1.425156in}}%
\pgfpathclose%
\pgfusepath{fill}%
\end{pgfscope}%
\begin{pgfscope}%
\pgfpathrectangle{\pgfqpoint{3.019583in}{0.169444in}}{\pgfqpoint{0.896708in}{1.339426in}}%
\pgfusepath{clip}%
\pgfsetbuttcap%
\pgfsetmiterjoin%
\definecolor{currentfill}{rgb}{0.294195,0.406228,0.680969}%
\pgfsetfillcolor{currentfill}%
\pgfsetlinewidth{0.000000pt}%
\definecolor{currentstroke}{rgb}{0.000000,0.000000,0.000000}%
\pgfsetstrokecolor{currentstroke}%
\pgfsetstrokeopacity{0.000000}%
\pgfsetdash{}{0pt}%
\pgfpathmoveto{\pgfqpoint{3.521740in}{1.446085in}}%
\pgfpathlineto{\pgfqpoint{3.485872in}{1.446085in}}%
\pgfpathlineto{\pgfqpoint{3.485872in}{1.467013in}}%
\pgfpathlineto{\pgfqpoint{3.521740in}{1.467013in}}%
\pgfpathlineto{\pgfqpoint{3.521740in}{1.446085in}}%
\pgfpathclose%
\pgfusepath{fill}%
\end{pgfscope}%
\begin{pgfscope}%
\pgfpathrectangle{\pgfqpoint{3.019583in}{0.169444in}}{\pgfqpoint{0.896708in}{1.339426in}}%
\pgfusepath{clip}%
\pgfsetbuttcap%
\pgfsetmiterjoin%
\definecolor{currentfill}{rgb}{0.321261,0.371165,0.664360}%
\pgfsetfillcolor{currentfill}%
\pgfsetlinewidth{0.000000pt}%
\definecolor{currentstroke}{rgb}{0.000000,0.000000,0.000000}%
\pgfsetstrokecolor{currentstroke}%
\pgfsetstrokeopacity{0.000000}%
\pgfsetdash{}{0pt}%
\pgfpathmoveto{\pgfqpoint{3.521740in}{1.467013in}}%
\pgfpathlineto{\pgfqpoint{3.485872in}{1.467013in}}%
\pgfpathlineto{\pgfqpoint{3.485872in}{1.487942in}}%
\pgfpathlineto{\pgfqpoint{3.521740in}{1.487942in}}%
\pgfpathlineto{\pgfqpoint{3.521740in}{1.467013in}}%
\pgfpathclose%
\pgfusepath{fill}%
\end{pgfscope}%
\begin{pgfscope}%
\pgfpathrectangle{\pgfqpoint{3.019583in}{0.169444in}}{\pgfqpoint{0.896708in}{1.339426in}}%
\pgfusepath{clip}%
\pgfsetbuttcap%
\pgfsetmiterjoin%
\definecolor{currentfill}{rgb}{0.348328,0.336101,0.647751}%
\pgfsetfillcolor{currentfill}%
\pgfsetlinewidth{0.000000pt}%
\definecolor{currentstroke}{rgb}{0.000000,0.000000,0.000000}%
\pgfsetstrokecolor{currentstroke}%
\pgfsetstrokeopacity{0.000000}%
\pgfsetdash{}{0pt}%
\pgfpathmoveto{\pgfqpoint{3.521740in}{1.487942in}}%
\pgfpathlineto{\pgfqpoint{3.485872in}{1.487942in}}%
\pgfpathlineto{\pgfqpoint{3.485872in}{1.508871in}}%
\pgfpathlineto{\pgfqpoint{3.521740in}{1.508871in}}%
\pgfpathlineto{\pgfqpoint{3.521740in}{1.487942in}}%
\pgfpathclose%
\pgfusepath{fill}%
\end{pgfscope}%
\begin{pgfscope}%
\pgfpathrectangle{\pgfqpoint{3.019583in}{0.169444in}}{\pgfqpoint{0.896708in}{1.339426in}}%
\pgfusepath{clip}%
\pgfsetbuttcap%
\pgfsetmiterjoin%
\definecolor{currentfill}{rgb}{0.368627,0.309804,0.635294}%
\pgfsetfillcolor{currentfill}%
\pgfsetlinewidth{0.000000pt}%
\definecolor{currentstroke}{rgb}{0.000000,0.000000,0.000000}%
\pgfsetstrokecolor{currentstroke}%
\pgfsetstrokeopacity{0.000000}%
\pgfsetdash{}{0pt}%
\pgfpathmoveto{\pgfqpoint{3.521740in}{1.508871in}}%
\pgfpathlineto{\pgfqpoint{3.485872in}{1.508871in}}%
\pgfpathlineto{\pgfqpoint{3.485872in}{1.529799in}}%
\pgfpathlineto{\pgfqpoint{3.521740in}{1.529799in}}%
\pgfpathlineto{\pgfqpoint{3.521740in}{1.508871in}}%
\pgfpathclose%
\pgfusepath{fill}%
\end{pgfscope}%
\begin{pgfscope}%
\pgfpathrectangle{\pgfqpoint{3.019583in}{0.169444in}}{\pgfqpoint{0.896708in}{1.339426in}}%
\pgfusepath{clip}%
\pgfsetbuttcap%
\pgfsetmiterjoin%
\definecolor{currentfill}{rgb}{0.995156,0.832295,0.506344}%
\pgfsetfillcolor{currentfill}%
\pgfsetlinewidth{0.000000pt}%
\definecolor{currentstroke}{rgb}{0.000000,0.000000,0.000000}%
\pgfsetstrokecolor{currentstroke}%
\pgfsetstrokeopacity{0.000000}%
\pgfsetdash{}{0pt}%
\pgfpathmoveto{\pgfqpoint{3.566575in}{0.671729in}}%
\pgfpathlineto{\pgfqpoint{3.584510in}{0.671729in}}%
\pgfpathlineto{\pgfqpoint{3.584510in}{0.713586in}}%
\pgfpathlineto{\pgfqpoint{3.566575in}{0.713586in}}%
\pgfpathlineto{\pgfqpoint{3.566575in}{0.671729in}}%
\pgfpathclose%
\pgfusepath{fill}%
\end{pgfscope}%
\begin{pgfscope}%
\pgfpathrectangle{\pgfqpoint{3.019583in}{0.169444in}}{\pgfqpoint{0.896708in}{1.339426in}}%
\pgfusepath{clip}%
\pgfsetbuttcap%
\pgfsetmiterjoin%
\definecolor{currentfill}{rgb}{0.998847,0.964245,0.689043}%
\pgfsetfillcolor{currentfill}%
\pgfsetlinewidth{0.000000pt}%
\definecolor{currentstroke}{rgb}{0.000000,0.000000,0.000000}%
\pgfsetstrokecolor{currentstroke}%
\pgfsetstrokeopacity{0.000000}%
\pgfsetdash{}{0pt}%
\pgfpathmoveto{\pgfqpoint{3.566575in}{0.797300in}}%
\pgfpathlineto{\pgfqpoint{3.584510in}{0.797300in}}%
\pgfpathlineto{\pgfqpoint{3.584510in}{0.839157in}}%
\pgfpathlineto{\pgfqpoint{3.566575in}{0.839157in}}%
\pgfpathlineto{\pgfqpoint{3.566575in}{0.797300in}}%
\pgfpathclose%
\pgfusepath{fill}%
\end{pgfscope}%
\begin{pgfscope}%
\pgfpathrectangle{\pgfqpoint{3.019583in}{0.169444in}}{\pgfqpoint{0.896708in}{1.339426in}}%
\pgfusepath{clip}%
\pgfsetbuttcap%
\pgfsetmiterjoin%
\definecolor{currentfill}{rgb}{0.998078,0.999231,0.746021}%
\pgfsetfillcolor{currentfill}%
\pgfsetlinewidth{0.000000pt}%
\definecolor{currentstroke}{rgb}{0.000000,0.000000,0.000000}%
\pgfsetstrokecolor{currentstroke}%
\pgfsetstrokeopacity{0.000000}%
\pgfsetdash{}{0pt}%
\pgfpathmoveto{\pgfqpoint{3.566575in}{0.839157in}}%
\pgfpathlineto{\pgfqpoint{3.656246in}{0.839157in}}%
\pgfpathlineto{\pgfqpoint{3.656246in}{0.881015in}}%
\pgfpathlineto{\pgfqpoint{3.566575in}{0.881015in}}%
\pgfpathlineto{\pgfqpoint{3.566575in}{0.839157in}}%
\pgfpathclose%
\pgfusepath{fill}%
\end{pgfscope}%
\begin{pgfscope}%
\pgfpathrectangle{\pgfqpoint{3.019583in}{0.169444in}}{\pgfqpoint{0.896708in}{1.339426in}}%
\pgfusepath{clip}%
\pgfsetbuttcap%
\pgfsetmiterjoin%
\definecolor{currentfill}{rgb}{0.967320,0.986928,0.698039}%
\pgfsetfillcolor{currentfill}%
\pgfsetlinewidth{0.000000pt}%
\definecolor{currentstroke}{rgb}{0.000000,0.000000,0.000000}%
\pgfsetstrokecolor{currentstroke}%
\pgfsetstrokeopacity{0.000000}%
\pgfsetdash{}{0pt}%
\pgfpathmoveto{\pgfqpoint{3.566575in}{0.881015in}}%
\pgfpathlineto{\pgfqpoint{3.692115in}{0.881015in}}%
\pgfpathlineto{\pgfqpoint{3.692115in}{0.922872in}}%
\pgfpathlineto{\pgfqpoint{3.566575in}{0.922872in}}%
\pgfpathlineto{\pgfqpoint{3.566575in}{0.881015in}}%
\pgfpathclose%
\pgfusepath{fill}%
\end{pgfscope}%
\begin{pgfscope}%
\pgfpathrectangle{\pgfqpoint{3.019583in}{0.169444in}}{\pgfqpoint{0.896708in}{1.339426in}}%
\pgfusepath{clip}%
\pgfsetbuttcap%
\pgfsetmiterjoin%
\definecolor{currentfill}{rgb}{0.936563,0.974625,0.650058}%
\pgfsetfillcolor{currentfill}%
\pgfsetlinewidth{0.000000pt}%
\definecolor{currentstroke}{rgb}{0.000000,0.000000,0.000000}%
\pgfsetstrokecolor{currentstroke}%
\pgfsetstrokeopacity{0.000000}%
\pgfsetdash{}{0pt}%
\pgfpathmoveto{\pgfqpoint{3.566575in}{0.922872in}}%
\pgfpathlineto{\pgfqpoint{3.638312in}{0.922872in}}%
\pgfpathlineto{\pgfqpoint{3.638312in}{0.964729in}}%
\pgfpathlineto{\pgfqpoint{3.566575in}{0.964729in}}%
\pgfpathlineto{\pgfqpoint{3.566575in}{0.922872in}}%
\pgfpathclose%
\pgfusepath{fill}%
\end{pgfscope}%
\begin{pgfscope}%
\pgfpathrectangle{\pgfqpoint{3.019583in}{0.169444in}}{\pgfqpoint{0.896708in}{1.339426in}}%
\pgfusepath{clip}%
\pgfsetbuttcap%
\pgfsetmiterjoin%
\definecolor{currentfill}{rgb}{0.905805,0.962322,0.602076}%
\pgfsetfillcolor{currentfill}%
\pgfsetlinewidth{0.000000pt}%
\definecolor{currentstroke}{rgb}{0.000000,0.000000,0.000000}%
\pgfsetstrokecolor{currentstroke}%
\pgfsetstrokeopacity{0.000000}%
\pgfsetdash{}{0pt}%
\pgfpathmoveto{\pgfqpoint{3.566575in}{0.964729in}}%
\pgfpathlineto{\pgfqpoint{3.692115in}{0.964729in}}%
\pgfpathlineto{\pgfqpoint{3.692115in}{1.006586in}}%
\pgfpathlineto{\pgfqpoint{3.566575in}{1.006586in}}%
\pgfpathlineto{\pgfqpoint{3.566575in}{0.964729in}}%
\pgfpathclose%
\pgfusepath{fill}%
\end{pgfscope}%
\begin{pgfscope}%
\pgfpathrectangle{\pgfqpoint{3.019583in}{0.169444in}}{\pgfqpoint{0.896708in}{1.339426in}}%
\pgfusepath{clip}%
\pgfsetbuttcap%
\pgfsetmiterjoin%
\definecolor{currentfill}{rgb}{0.838447,0.934948,0.608997}%
\pgfsetfillcolor{currentfill}%
\pgfsetlinewidth{0.000000pt}%
\definecolor{currentstroke}{rgb}{0.000000,0.000000,0.000000}%
\pgfsetstrokecolor{currentstroke}%
\pgfsetstrokeopacity{0.000000}%
\pgfsetdash{}{0pt}%
\pgfpathmoveto{\pgfqpoint{3.566575in}{1.006586in}}%
\pgfpathlineto{\pgfqpoint{3.674180in}{1.006586in}}%
\pgfpathlineto{\pgfqpoint{3.674180in}{1.048443in}}%
\pgfpathlineto{\pgfqpoint{3.566575in}{1.048443in}}%
\pgfpathlineto{\pgfqpoint{3.566575in}{1.006586in}}%
\pgfpathclose%
\pgfusepath{fill}%
\end{pgfscope}%
\begin{pgfscope}%
\pgfpathrectangle{\pgfqpoint{3.019583in}{0.169444in}}{\pgfqpoint{0.896708in}{1.339426in}}%
\pgfusepath{clip}%
\pgfsetbuttcap%
\pgfsetmiterjoin%
\definecolor{currentfill}{rgb}{0.765859,0.905421,0.623760}%
\pgfsetfillcolor{currentfill}%
\pgfsetlinewidth{0.000000pt}%
\definecolor{currentstroke}{rgb}{0.000000,0.000000,0.000000}%
\pgfsetstrokecolor{currentstroke}%
\pgfsetstrokeopacity{0.000000}%
\pgfsetdash{}{0pt}%
\pgfpathmoveto{\pgfqpoint{3.566575in}{1.048443in}}%
\pgfpathlineto{\pgfqpoint{3.638312in}{1.048443in}}%
\pgfpathlineto{\pgfqpoint{3.638312in}{1.090300in}}%
\pgfpathlineto{\pgfqpoint{3.566575in}{1.090300in}}%
\pgfpathlineto{\pgfqpoint{3.566575in}{1.048443in}}%
\pgfpathclose%
\pgfusepath{fill}%
\end{pgfscope}%
\begin{pgfscope}%
\pgfpathrectangle{\pgfqpoint{3.019583in}{0.169444in}}{\pgfqpoint{0.896708in}{1.339426in}}%
\pgfusepath{clip}%
\pgfsetbuttcap%
\pgfsetmiterjoin%
\definecolor{currentfill}{rgb}{0.693272,0.875894,0.638524}%
\pgfsetfillcolor{currentfill}%
\pgfsetlinewidth{0.000000pt}%
\definecolor{currentstroke}{rgb}{0.000000,0.000000,0.000000}%
\pgfsetstrokecolor{currentstroke}%
\pgfsetstrokeopacity{0.000000}%
\pgfsetdash{}{0pt}%
\pgfpathmoveto{\pgfqpoint{3.566575in}{1.090300in}}%
\pgfpathlineto{\pgfqpoint{3.638312in}{1.090300in}}%
\pgfpathlineto{\pgfqpoint{3.638312in}{1.132157in}}%
\pgfpathlineto{\pgfqpoint{3.566575in}{1.132157in}}%
\pgfpathlineto{\pgfqpoint{3.566575in}{1.090300in}}%
\pgfpathclose%
\pgfusepath{fill}%
\end{pgfscope}%
\begin{pgfscope}%
\pgfpathrectangle{\pgfqpoint{3.019583in}{0.169444in}}{\pgfqpoint{0.896708in}{1.339426in}}%
\pgfusepath{clip}%
\pgfsetbuttcap%
\pgfsetmiterjoin%
\definecolor{currentfill}{rgb}{0.612226,0.843829,0.643983}%
\pgfsetfillcolor{currentfill}%
\pgfsetlinewidth{0.000000pt}%
\definecolor{currentstroke}{rgb}{0.000000,0.000000,0.000000}%
\pgfsetstrokecolor{currentstroke}%
\pgfsetstrokeopacity{0.000000}%
\pgfsetdash{}{0pt}%
\pgfpathmoveto{\pgfqpoint{3.566575in}{1.132157in}}%
\pgfpathlineto{\pgfqpoint{3.584510in}{1.132157in}}%
\pgfpathlineto{\pgfqpoint{3.584510in}{1.174014in}}%
\pgfpathlineto{\pgfqpoint{3.566575in}{1.174014in}}%
\pgfpathlineto{\pgfqpoint{3.566575in}{1.132157in}}%
\pgfpathclose%
\pgfusepath{fill}%
\end{pgfscope}%
\begin{pgfscope}%
\pgfpathrectangle{\pgfqpoint{3.019583in}{0.169444in}}{\pgfqpoint{0.896708in}{1.339426in}}%
\pgfusepath{clip}%
\pgfsetbuttcap%
\pgfsetmiterjoin%
\definecolor{currentfill}{rgb}{0.442445,0.777393,0.646444}%
\pgfsetfillcolor{currentfill}%
\pgfsetlinewidth{0.000000pt}%
\definecolor{currentstroke}{rgb}{0.000000,0.000000,0.000000}%
\pgfsetstrokecolor{currentstroke}%
\pgfsetstrokeopacity{0.000000}%
\pgfsetdash{}{0pt}%
\pgfpathmoveto{\pgfqpoint{3.566575in}{1.215871in}}%
\pgfpathlineto{\pgfqpoint{3.602444in}{1.215871in}}%
\pgfpathlineto{\pgfqpoint{3.602444in}{1.257728in}}%
\pgfpathlineto{\pgfqpoint{3.566575in}{1.257728in}}%
\pgfpathlineto{\pgfqpoint{3.566575in}{1.215871in}}%
\pgfpathclose%
\pgfusepath{fill}%
\end{pgfscope}%
\begin{pgfscope}%
\pgfpathrectangle{\pgfqpoint{3.019583in}{0.169444in}}{\pgfqpoint{0.896708in}{1.339426in}}%
\pgfusepath{clip}%
\pgfsetbuttcap%
\pgfsetmiterjoin%
\definecolor{currentfill}{rgb}{0.368012,0.725106,0.661822}%
\pgfsetfillcolor{currentfill}%
\pgfsetlinewidth{0.000000pt}%
\definecolor{currentstroke}{rgb}{0.000000,0.000000,0.000000}%
\pgfsetstrokecolor{currentstroke}%
\pgfsetstrokeopacity{0.000000}%
\pgfsetdash{}{0pt}%
\pgfpathmoveto{\pgfqpoint{3.566575in}{1.257728in}}%
\pgfpathlineto{\pgfqpoint{3.638312in}{1.257728in}}%
\pgfpathlineto{\pgfqpoint{3.638312in}{1.299585in}}%
\pgfpathlineto{\pgfqpoint{3.566575in}{1.299585in}}%
\pgfpathlineto{\pgfqpoint{3.566575in}{1.257728in}}%
\pgfpathclose%
\pgfusepath{fill}%
\end{pgfscope}%
\begin{pgfscope}%
\pgfpathrectangle{\pgfqpoint{3.019583in}{0.169444in}}{\pgfqpoint{0.896708in}{1.339426in}}%
\pgfusepath{clip}%
\pgfsetbuttcap%
\pgfsetmiterjoin%
\definecolor{currentfill}{rgb}{0.304037,0.653749,0.691349}%
\pgfsetfillcolor{currentfill}%
\pgfsetlinewidth{0.000000pt}%
\definecolor{currentstroke}{rgb}{0.000000,0.000000,0.000000}%
\pgfsetstrokecolor{currentstroke}%
\pgfsetstrokeopacity{0.000000}%
\pgfsetdash{}{0pt}%
\pgfpathmoveto{\pgfqpoint{3.566575in}{1.299585in}}%
\pgfpathlineto{\pgfqpoint{3.602444in}{1.299585in}}%
\pgfpathlineto{\pgfqpoint{3.602444in}{1.341442in}}%
\pgfpathlineto{\pgfqpoint{3.566575in}{1.341442in}}%
\pgfpathlineto{\pgfqpoint{3.566575in}{1.299585in}}%
\pgfpathclose%
\pgfusepath{fill}%
\end{pgfscope}%
\begin{pgfscope}%
\pgfpathrectangle{\pgfqpoint{3.019583in}{0.169444in}}{\pgfqpoint{0.896708in}{1.339426in}}%
\pgfusepath{clip}%
\pgfsetbuttcap%
\pgfsetmiterjoin%
\definecolor{currentfill}{rgb}{0.240062,0.582391,0.720877}%
\pgfsetfillcolor{currentfill}%
\pgfsetlinewidth{0.000000pt}%
\definecolor{currentstroke}{rgb}{0.000000,0.000000,0.000000}%
\pgfsetstrokecolor{currentstroke}%
\pgfsetstrokeopacity{0.000000}%
\pgfsetdash{}{0pt}%
\pgfpathmoveto{\pgfqpoint{3.566575in}{1.341442in}}%
\pgfpathlineto{\pgfqpoint{3.584510in}{1.341442in}}%
\pgfpathlineto{\pgfqpoint{3.584510in}{1.383299in}}%
\pgfpathlineto{\pgfqpoint{3.566575in}{1.383299in}}%
\pgfpathlineto{\pgfqpoint{3.566575in}{1.341442in}}%
\pgfpathclose%
\pgfusepath{fill}%
\end{pgfscope}%
\begin{pgfscope}%
\pgfpathrectangle{\pgfqpoint{3.019583in}{0.169444in}}{\pgfqpoint{0.896708in}{1.339426in}}%
\pgfusepath{clip}%
\pgfsetbuttcap%
\pgfsetmiterjoin%
\definecolor{currentfill}{rgb}{0.212995,0.511419,0.730796}%
\pgfsetfillcolor{currentfill}%
\pgfsetlinewidth{0.000000pt}%
\definecolor{currentstroke}{rgb}{0.000000,0.000000,0.000000}%
\pgfsetstrokecolor{currentstroke}%
\pgfsetstrokeopacity{0.000000}%
\pgfsetdash{}{0pt}%
\pgfpathmoveto{\pgfqpoint{3.566575in}{1.383299in}}%
\pgfpathlineto{\pgfqpoint{3.584510in}{1.383299in}}%
\pgfpathlineto{\pgfqpoint{3.584510in}{1.425156in}}%
\pgfpathlineto{\pgfqpoint{3.566575in}{1.425156in}}%
\pgfpathlineto{\pgfqpoint{3.566575in}{1.383299in}}%
\pgfpathclose%
\pgfusepath{fill}%
\end{pgfscope}%
\begin{pgfscope}%
\pgfpathrectangle{\pgfqpoint{3.019583in}{0.169444in}}{\pgfqpoint{0.896708in}{1.339426in}}%
\pgfusepath{clip}%
\pgfsetbuttcap%
\pgfsetmiterjoin%
\definecolor{currentfill}{rgb}{0.267128,0.441292,0.697578}%
\pgfsetfillcolor{currentfill}%
\pgfsetlinewidth{0.000000pt}%
\definecolor{currentstroke}{rgb}{0.000000,0.000000,0.000000}%
\pgfsetstrokecolor{currentstroke}%
\pgfsetstrokeopacity{0.000000}%
\pgfsetdash{}{0pt}%
\pgfpathmoveto{\pgfqpoint{3.566575in}{1.425156in}}%
\pgfpathlineto{\pgfqpoint{3.584510in}{1.425156in}}%
\pgfpathlineto{\pgfqpoint{3.584510in}{1.467013in}}%
\pgfpathlineto{\pgfqpoint{3.566575in}{1.467013in}}%
\pgfpathlineto{\pgfqpoint{3.566575in}{1.425156in}}%
\pgfpathclose%
\pgfusepath{fill}%
\end{pgfscope}%
\begin{pgfscope}%
\definecolor{textcolor}{rgb}{0.000000,0.000000,0.000000}%
\pgfsetstrokecolor{textcolor}%
\pgfsetfillcolor{textcolor}%
\pgftext[x=3.378267in,y=0.169444in,,]{\color{textcolor}\setmainfont{Lato}\rmfamily\fontsize{8.000000}{9.600000}\selectfont -8}%
\end{pgfscope}%
\begin{pgfscope}%
\definecolor{textcolor}{rgb}{0.000000,0.000000,0.000000}%
\pgfsetstrokecolor{textcolor}%
\pgfsetfillcolor{textcolor}%
\pgftext[x=3.378267in,y=0.504301in,,]{\color{textcolor}\setmainfont{Lato}\rmfamily\fontsize{8.000000}{9.600000}\selectfont -4}%
\end{pgfscope}%
\begin{pgfscope}%
\definecolor{textcolor}{rgb}{0.000000,0.000000,0.000000}%
\pgfsetstrokecolor{textcolor}%
\pgfsetfillcolor{textcolor}%
\pgftext[x=3.378267in,y=0.839157in,,]{\color{textcolor}\setmainfont{Lato}\rmfamily\fontsize{8.000000}{9.600000}\selectfont 0}%
\end{pgfscope}%
\begin{pgfscope}%
\definecolor{textcolor}{rgb}{0.000000,0.000000,0.000000}%
\pgfsetstrokecolor{textcolor}%
\pgfsetfillcolor{textcolor}%
\pgftext[x=3.378267in,y=1.174014in,,]{\color{textcolor}\setmainfont{Lato}\rmfamily\fontsize{8.000000}{9.600000}\selectfont 4}%
\end{pgfscope}%
\begin{pgfscope}%
\definecolor{textcolor}{rgb}{0.000000,0.000000,0.000000}%
\pgfsetstrokecolor{textcolor}%
\pgfsetfillcolor{textcolor}%
\pgftext[x=3.378267in,y=1.508871in,,]{\color{textcolor}\setmainfont{Lato}\rmfamily\fontsize{8.000000}{9.600000}\selectfont 8+}%
\end{pgfscope}%
\end{pgfpicture}%
\makeatother%
\endgroup%
 \\
\footnotesize{Source: Bureau of Economic Analysis}\\

\vspace{2mm}

\begin{minipage}{0.76\textwidth}

\small \input{text/gdp_state.txt} \\

\end{minipage}

\vspace{2mm}

\noindent \normalsize \textbf{Real GDP Growth by State}\\
\footnotesize{\textit{quarterly growth at seasonally adjusted annualized rate \hspace{20mm} total growth, \input{text/gdp_state_date.txt}}\\ 

\vspace{-4.5mm}
\hspace{-2mm} \noindent \rowcolors{1}{}{black!5} \setlength{\tabcolsep}{3.7pt} \color{black!90}
		{\renewcommand{\arraystretch}{1.44}
		 \begin{tabular}{p{30mm} R{7mm} R{7mm} R{7mm} R{7mm} R{7mm} p{0mm} R{9mm} R{9mm} R{10mm} }
 & 2021 Q3 & '21 Q2 & '21 Q1 & '20 Q4 & '20 Q3 & & 1-year* & 3-year & 10-year \\
\textbf{United States}  & 2.3 & 6.7 & 6.3 & 4.5 & 33.8 &  & 4.9 & 4.8 & 22.7 \\
\hspace{1mm} \textbf{Pacific}  & 2.9 & 7.9 & 10.7 & 1.7 & 33.2 &  & 5.8 & 7.8 & 37.2 \\
\hspace{3mm}  Washington  & 2.7 & 8.2 & 8.9 & 3.5 & 30.8 &  & 5.8 & 10.4 & 47.7 \\
\hspace{3mm}  California  & 2.9 & 8.1 & 11.7 & 1.2 & 33.6 &  & 5.9 & 8.2 & 38.1 \\
\hspace{3mm}  Oregon  & 3.5 & 6.0 & 7.4 & 2.5 & 35.9 &  & 4.8 & 5.0 & 29.9 \\
\hspace{3mm}  Hawaii  & 6.0 & 8.9 & 5.2 & 4.4 & 34.5 &  & 6.1 & -5.8 & 7.0 \\
\hspace{3mm}  Alaska  & -0.6 & 1.8 & -4.5 & 4.0 & 27.6 &  & 0.2 & -6.0 & -8.2 \\
\hspace{1mm} \textbf{West South Central}  & 2.6 & 5.8 & 3.4 & 5.3 & 36.0 &  & 4.2 & 4.5 & 28.4 \\
\hspace{3mm}  Texas  & 3.5 & 6.4 & 3.9 & 5.6 & 35.6 &  & 4.8 & 6.2 & 36.2 \\
\hspace{3mm}  Oklahoma  & 1.0 & 3.5 & -0.4 & 5.0 & 38.8 &  & 2.2 & -0.1 & 19.8 \\
\hspace{3mm}  Arkansas  & 0.7 & 4.2 & 7.2 & 4.8 & 38.1 &  & 4.2 & 4.5 & 12.5 \\
\hspace{3mm}  Louisiana  & -2.7 & 4.0 & 0.7 & 3.1 & 35.7 &  & 1.2 & -4.1 & -3.1 \\
\hspace{1mm} \textbf{Mountain}  & 2.0 & 6.4 & 4.5 & 3.6 & 32.8 &  & 4.1 & 6.2 & 26.3 \\
\hspace{3mm}  Utah  & 2.7 & 6.3 & 4.0 & 2.4 & 30.8 &  & 3.8 & 10.0 & 40.0 \\
\hspace{3mm}  Colorado  & 2.3 & 7.2 & 7.3 & 5.3 & 29.0 &  & 5.5 & 7.2 & 35.3 \\
\hspace{3mm}  Idaho  & -1.0 & 4.0 & 7.4 & 1.1 & 38.6 &  & 2.8 & 7.9 & 31.8 \\
\hspace{3mm}  Arizona  & 3.2 & 5.6 & 0.8 & 3.5 & 32.5 &  & 3.2 & 6.5 & 26.2 \\
\hspace{3mm}  Nevada  & 2.6 & 9.7 & 6.2 & 4.0 & 45.3 &  & 5.6 & 2.2 & 16.8 \\
\hspace{3mm}  Montana  & -0.6 & 5.5 & 11.2 & 2.0 & 30.8 &  & 4.4 & 3.7 & 16.5 \\
\hspace{3mm}  New Mexico  & -0.5 & 5.3 & 0.8 & 3.3 & 30.8 &  & 2.2 & 4.9 & 9.3 \\
\multicolumn{3}{l}{continued on next page . . .} & &  & & & & & \\
\end{tabular} } \\ \newpage

\hspace{-2mm} \noindent \rowcolors{1}{}{black!5} 
            \setlength{\tabcolsep}{3.8pt} \color{black!90}
            {\renewcommand{\arraystretch}{1.44}
             \begin{tabular}{p{30mm} R{7mm} R{7mm} R{7mm} R{7mm} 
             R{7mm} p{0mm} R{9mm} R{9mm} R{10mm} }
 & 2021 Q3 & '21 Q2 & '21 Q1 & '20 Q4 & '20 Q3 & & 1-year* & 3-year & 10-year \\
\multicolumn{3}{l}{continued from previous page . . .}  & &  & & & & & \\
\hspace{3mm}  Wyoming  & -1.5 & 2.3 & 3.7 & 1.2 & 28.9 &  & 1.4 & -3.2 & -7.0 \\
\hspace{1mm} \textbf{South Atlantic}  & 2.9 & 6.2 & 6.1 & 4.4 & 31.3 &  & 4.9 & 4.8 & 20.9 \\
\hspace{3mm}  Florida  & 3.7 & 6.7 & 7.5 & 3.4 & 32.6 &  & 5.3 & 6.5 & 29.4 \\
\hspace{3mm}  Georgia  & 3.3 & 6.0 & 6.9 & 4.1 & 30.4 &  & 5.1 & 5.6 & 29.3 \\
\hspace{3mm}  South Carolina  & 1.6 & 6.1 & 5.1 & 4.4 & 39.1 &  & 4.3 & 5.9 & 24.6 \\
\hspace{3mm}  North Carolina  & 2.4 & 6.2 & 9.5 & 5.5 & 34.8 &  & 5.9 & 5.7 & 19.4 \\
\hspace{3mm}  District of Columbia  & 3.9 & 7.2 & -1.6 & 5.5 & 18.6 &  & 3.7 & 2.5 & 13.7 \\
\hspace{3mm}  Maryland  & 1.8 & 6.1 & 8.9 & 4.2 & 27.8 &  & 5.2 & 1.5 & 12.2 \\
\hspace{3mm}  Virginia  & 2.8 & 5.8 & 1.4 & 5.2 & 29.2 &  & 3.8 & 3.6 & 11.6 \\
\hspace{3mm}  West Virginia  & -0.6 & 7.3 & 1.1 & 4.1 & 33.3 &  & 3.0 & -1.0 & 2.9 \\
\hspace{3mm}  Delaware  & 4.7 & 2.8 & -3.6 & 8.1 & 27.1 &  & 2.9 & 4.7 & 2.5 \\
\hspace{1mm} \textbf{West North Central}  & 0.8 & 6.2 & 5.8 & 6.7 & 35.4 &  & 4.8 & 4.3 & 17.5 \\
\hspace{3mm}  North Dakota  & -3.3 & 5.5 & 10.1 & 4.6 & 26.9 &  & 4.1 & 2.2 & 39.9 \\
\hspace{3mm}  Nebraska  & 0.8 & 2.9 & 6.4 & 11.8 & 37.1 &  & 5.4 & 7.5 & 22.4 \\
\hspace{3mm}  Iowa  & 0.5 & 7.7 & 8.6 & 9.3 & 38.1 &  & 6.5 & 5.9 & 21.1 \\
\hspace{3mm}  Kansas  & -0.3 & 6.1 & 3.5 & 6.2 & 39.2 &  & 3.8 & 4.0 & 17.4 \\
\hspace{3mm}  Minnesota  & 1.8 & 6.5 & 5.3 & 5.6 & 33.9 &  & 4.8 & 2.3 & 16.9 \\
\hspace{3mm}  South Dakota  & -0.8 & 4.2 & 6.3 & 5.9 & 33.1 &  & 3.9 & 4.9 & 14.4 \\
\hspace{3mm}  Missouri  & 1.6 & 6.7 & 4.7 & 5.3 & 34.7 &  & 4.5 & 4.7 & 11.3 \\
\hspace{1mm} \textbf{Middle Atlantic}  & 2.5 & 7.2 & 6.2 & 6.1 & 31.3 &  & 5.5 & 3.7 & 16.2 \\
\hspace{3mm}  New York  & 2.2 & 8.1 & 7.1 & 6.8 & 28.9 &  & 6.0 & 4.2 & 19.2 \\
\hspace{3mm}  Pennsylvania  & 2.2 & 5.9 & 3.0 & 5.2 & 34.8 &  & 4.1 & 2.2 & 13.0 \\
\hspace{3mm}  New Jersey  & 3.7 & 6.4 & 7.9 & 5.1 & 33.3 &  & 5.8 & 4.2 & 12.6 \\
\hspace{1mm} \textbf{East South Central}  & 1.8 & 5.3 & 7.3 & 5.6 & 43.8 &  & 5.0 & 4.3 & 15.4 \\
\hspace{3mm}  Tennessee  & 3.1 & 5.6 & 13.4 & 7.6 & 51.2 &  & 7.4 & 6.2 & 24.5 \\
\hspace{3mm}  Kentucky  & 1.1 & 6.5 & 5.5 & 4.7 & 40.8 &  & 4.4 & 4.3 & 12.4 \\
\hspace{3mm}  Alabama  & 1.3 & 4.1 & 2.4 & 3.5 & 36.0 &  & 2.8 & 1.8 & 9.6 \\
\hspace{3mm}  Mississippi  & 0.2 & 4.5 & 2.0 & 5.0 & 43.1 &  & 2.9 & 2.6 & 6.0 \\
\hspace{1mm} \textbf{New England}  & 2.6 & 7.1 & 3.8 & 3.5 & 35.2 &  & 4.2 & 3.2 & 13.5 \\
\hspace{3mm}  Massachusetts  & 3.7 & 8.0 & 6.1 & 1.4 & 35.2 &  & 4.8 & 5.1 & 20.5 \\
\hspace{3mm}  New Hampshire  & -3.3 & 5.6 & 2.5 & 5.3 & 40.0 &  & 2.5 & 3.3 & 15.6 \\
\hspace{3mm}  Maine  & 1.7 & 5.5 & 1.3 & 3.8 & 37.2 &  & 3.1 & 5.6 & 14.4 \\
\hspace{3mm}  Rhode Island  & 2.2 & 7.5 & -2.9 & 6.0 & 31.6 &  & 3.1 & 1.7 & 3.3 \\
\hspace{3mm}  Vermont  & 0.4 & 6.5 & 0.9 & 3.6 & 43.0 &  & 2.8 & 0.7 & 3.3 \\
\hspace{3mm}  Connecticut  & 2.7 & 5.9 & 1.8 & 6.8 & 33.0 &  & 4.3 & -0.8 & 3.3 \\
\hspace{1mm} \textbf{East North Central}  & 0.8 & 6.4 & 5.2 & 5.1 & 36.9 &  & 4.4 & 2.1 & 13.3 \\
\hspace{3mm}  Indiana  & 0.2 & 6.1 & 9.4 & 5.0 & 43.1 &  & 5.2 & 4.5 & 17.9 \\
\hspace{3mm}  Ohio  & 0.9 & 5.2 & 3.5 & 5.5 & 35.7 &  & 3.8 & 3.4 & 14.9 \\
\hspace{3mm}  Michigan  & -0.3 & 8.3 & 2.8 & 5.0 & 39.8 &  & 3.9 & 0.2 & 14.2 \\
\hspace{3mm}  Wisconsin  & -0.2 & 5.7 & 1.2 & 3.6 & 38.8 &  & 2.6 & 1.4 & 11.5 \\
\hspace{3mm}  Illinois  & 1.9 & 6.8 & 7.7 & 5.6 & 32.9 &  & 5.5 & 1.5 & 10.3 \\
 \hline
		\end{tabular}}	\\

\vspace{-3mm}	
\footnotesize{Source: Bureau of Economic Analysis}

\newpage

\begin{minipage}{0.76\textwidth}

\section*{\color{darkgray}\LARGE \seriffont Financial Accounts}

\small A high-level overview of US financial activities can be provided by dividing the world economy into three sectors: the US private sector (see\cbox{green!70!black}), the US government (see\cbox{yellow!70!orange}), and the rest of the world (see\cbox{blue!90!black}), then examining the net lending and borrowing between the groups, which must sum to zero at an aggregate level. That is, if one sector is running a deficit, another sector must be running a surplus.\\

\vspace{0mm}

\noindent \normalsize \textbf{Sectoral Financial Balance}\\
\footnotesize{\textit{net lending (+) or borrowing (-), NIPA basis, by sector, as share of GDP}}\\
\noindent \hspace*{-3mm} \begin{tikzpicture}
	\begin{axis}[\bbar{y}{0}, \dateaxisticks ytick={-10, 0, 10},
		xticklabel={`\short{\year}}, yticklabel style={text width=1.5em}, clip=false, 
		legend style={at={(0.95, 1.13)}}]
	\rbars
	\sbar{green!70!black}{date}{PRIV}{data/sectbal.csv}
	\sbar{yellow!70!orange}{date}{GOV}{data/sectbal.csv}
	\sbar{blue!90!black}{date}{ROW}{data/sectbal.csv}
    \node[align=left] (source) at (axis cs:2017-06-15,10.5) {%
      	\tiny TCJA repatriation};
    \node (destination) at (axis cs:2017-11-01,3.1) {};
    \draw[-] (source)--(destination);
	\legend{Private, Government, Rest of World};
	\end{axis}
\end{tikzpicture}\\
\footnotesize{Source: Bureau of Economic Analysis} \\

\vspace{2mm}

\small \input{text/sectbal.txt} \\

\vspace{2mm}

\noindent \normalsize \textbf{Domestic Private Sector Financial Balance}\\
\footnotesize{\textit{net lending (+) or borrowing (-), NIPA basis, by sector, as share of GDP}}\\
\noindent \hspace*{-3mm} \begin{tikzpicture}
	\begin{axis}[\bbar{y}{0}, \dateaxisticks ytick={-5, 0, 5, 10},
		xticklabel={`\short{\year}}, yticklabel style={text width=1.5em}, clip=false, 
		legend style={at={(0.95, 1.13)}}]
	\rbars
	\sbar{purple!50!red}{date}{W995RC}{data/sectbal2.csv}
	\sbar{orange!90!yellow}{date}{W996RC}{data/sectbal2.csv}
	\legend{Private Businesses, Households};
	\end{axis}
\end{tikzpicture}\\
\footnotesize{Source: Bureau of Economic Analysis}

\end{minipage}

\newpage

\begin{minipage}{0.76\textwidth}

\subsection*{\color{black!70} \seriffont Liabilities}

\small The contribution of different sectors to the total change in borrowing can identify potential risks in the domestic economy. For example, the tech bubble of the late 1990s and early 2000s shows up as a large increase in corporate borrowing. The housing bubble from the 1990s to 2007 shows up as an increase in household borrowing. Government borrowing increased following the collapse of the housing bubble, in an effort to compensate for the massive fall in wage income. Keep in mind, however, that the vast majority of liabilities in the domestic economy are to other domestic parties.\\

&   2021  Q4 & `21  Q3 & `21  Q2 & `21  Q1 & `20  Q4 &3-year&10-year&30-year\\ Total&-0.73&-0.09&0.68&6.09&8.31&4.58&3.35&3.87\\  \hspace{-2mm}\cbox{lime!70}Corporate  Business &-0.11&0.16&0.06&0.82&1.37&1.61&1.25&1.07\\  \hspace{4mm}  Debt  Securities &-0.24&-0.12&-0.14&0.47&0.68&0.26&0.36&0.34\\  \hspace{4mm}  Loans &0.21&-0.02&-0.29&-0.14&0.34&0.32&0.20&0.10\\  \hspace{-2mm}\cbox{green!72!black}Non-corporate  Business &-0.14&-0.12&0.00&0.43&0.62&0.36&0.39&0.45\\  \hspace{4mm}  Commercial  Mortgages &-0.04&-0.02&-0.03&0.03&0.07&0.04&0.06&0.07\\  \hspace{-2mm}\cbox{orange!70}Household  \&  Nonprofit &0.32&0.46&0.51&0.55&0.53&0.39&0.12&0.89\\  \hspace{4mm}  Home  Mortgages &0.25&0.29&0.27&0.38&0.40&0.24&-0.08&0.58\\  \hspace{4mm}  Consumer  Credit &0.02&0.02&0.00&-0.07&-0.07&0.07&0.16&0.21\\  \hspace{-2mm}\cbox{cyan!52}State  \&  Local  Government &-0.84&-1.01&-1.14&-1.04&-0.18&-0.22&0.04&0.36\\  \hspace{-2mm}\cbox{blue!70}Federal  Government &0.04&0.43&1.24&5.33&5.97&2.45&1.55&1.10\\  \\

\vspace{2mm}

\noindent \normalsize \textbf{Real Debt Growth}\\
\footnotesize{\textit{contribution to one-year percent change in liabilities, adjusted by PCE price deflator}}\\
\noindent \hspace*{-3mm} \begin{tikzpicture}
	\begin{axis}[\bbar{y}{0}, \dateaxisticks ytick={-2, 0, 2, 4, 6, 8}, 
		xmin = 1989-01-01, height=5.0cm, 
		xticklabel={`\short{\year}}, yticklabel style={text width=1.5em}, clip=false, 
		legend style={at={(0.95, 1.13)}}]
	\rbars
	\sbar{blue!70}{date}{Federal Government}{data/liabgr.csv}
	\sbar{cyan!70}{date}{State and Local Government}{data/liabgr.csv}
	\sbar{orange!70}{date}{Household and Nonprofit}{data/liabgr.csv}
	\sbar{lime!70}{date}{Corporate Business}{data/liabgr.csv}
	\sbar{green!80!black}{date}{Non-corporate Business}{data/liabgr.csv}
	\legend{Fed. Gov.,State/Loc. Gov., Households, Corp bus., Non-corp. bus.};
	\end{axis}
\end{tikzpicture}\\
\footnotesize{Source: Federal Reserve, Bureau of Economic Analysis}\\

\vspace{4mm}

\normalsize  [TABLE HERE]
		
			
\end{minipage}

\newpage

\begin{minipage}{0.76\textwidth}

\subsection*{\color{black!70} \seriffont Wealth}

\small \textbf{Total US wealth} is the tangible assets of all non-corporate sectors of the US, plus the market value of domestic corporate equities, less US financial obligations to the rest of the world. \input{text/wealthgdp.txt}\\

\vspace{1mm}

\noindent \normalsize \textbf{Total US Wealth to GDP Ratio}\\
\footnotesize{\textit{total US wealth divided by GDP}}\\
\noindent \hspace*{-3mm} \begin{tikzpicture}
	\begin{axis}[\bbar{y}{0}, \dateaxisticks ytick={0, 1, 2, 3, 4, 5}, ymin=-0.2,
		xticklabel={`\short{\year}}, clip=false, 
		legend style={at={(0.95, 1.13)}}]
	\rbars
	\sbar{cyan!35!white}{date}{Other}{data/wealthgdp.csv}
	\sbar{green!80!blue}{date}{Real Estate}{data/wealthgdp.csv}
	\sbar{magenta!50!violet}{date}{Corporate Equities}{data/wealthgdp.csv}
	\legend{Other, Residential Real Estate, Corporate Equities};
	\end{axis}
\end{tikzpicture}\\
\footnotesize{Source: Federal Reserve}
\end{minipage}

\newpage
\subsection*{\color{black!70} \seriffont Investment}

\begin{minipage}{0.76\textwidth}

\small Private fixed investment, as measured in the national accounts, includes construction and improvement of houses, apartment buildings, and other residential property (see\cbox{blue!90!black}), but not automobiles, appliances, or furniture. Non-residential private fixed investment includes the construction and improvement of offices, warehouses, factories, and other commercial and industrial property (see\cbox{yellow!50!orange}), as well as purchases of equipment and intellectual property products. The change in private inventories (see\cbox{red}) at the end of the accounting period is also, at times, grouped with investment.\\

\vspace{4mm} 

\noindent \normalsize \textbf{Private Fixed Investment}\\
\footnotesize{\textit{percentage point contribution to GDP growth}}\\
\noindent \hspace*{-3mm} \begin{tikzpicture}
	\begin{axis}[\bbar{y}{0}, \dateaxisticks ytick={-5, -2, 0, 2, 5},
		xticklabel={`\short{\year}}, clip=false, 
		legend style={at={(0.95, 1.13)}}]
	\rbars
	\sbar{yellow!50!orange}{date}{A008RY}{data/inv.csv}
	\sbar{blue!90!black}{date}{A011RY}{data/inv.csv}
	\sbar{red}{date}{A014RY}{data/inv.csv}
	\legend{Non-Residential, Residential, Change in Private Inventories};
	\end{axis}
\end{tikzpicture}\\
\footnotesize{Source: Bureau of Economic Analysis}

\end{minipage}

\newpage
\section*{\color{darkgray}\LARGE \seriffont Households}

\begin{minipage}{0.76\textwidth}
\small This section covers the household sector of the economy loosely defined, and touches on demographics, personal income and outlays, residential fixed investment, household balance sheets, home ownership, housing prices, and housing construction and permitting.

\vspace{2mm}

[Table or chart on population] \\

\end{minipage}

\vspace{8mm}

\noindent \normalsize \textbf{Age Group Share of Commuter Zone Population, 2018}

\hspace{2mm} Age 0 to 17 \hspace{62mm} Age 65+

\vspace{-3mm}
\hspace{-9mm} %% Creator: Matplotlib, PGF backend
%%
%% To include the figure in your LaTeX document, write
%%   \input{<filename>.pgf}
%%
%% Make sure the required packages are loaded in your preamble
%%   \usepackage{pgf}
%%
%% Also ensure that all the required font packages are loaded; for instance,
%% the lmodern package is sometimes necessary when using math font.
%%   \usepackage{lmodern}
%%
%% Figures using additional raster images can only be included by \input if
%% they are in the same directory as the main LaTeX file. For loading figures
%% from other directories you can use the `import` package
%%   \usepackage{import}
%%
%% and then include the figures with
%%   \import{<path to file>}{<filename>.pgf}
%%
%% Matplotlib used the following preamble
%%   
%%   \usepackage{fontspec}
%%   \setmainfont{DejaVuSerif.ttf}[Path=\detokenize{/home/brian/miniconda3/lib/python3.8/site-packages/matplotlib/mpl-data/fonts/ttf/}]
%%   \setsansfont{DejaVuSans.ttf}[Path=\detokenize{/home/brian/miniconda3/lib/python3.8/site-packages/matplotlib/mpl-data/fonts/ttf/}]
%%   \setmonofont{DejaVuSansMono.ttf}[Path=\detokenize{/home/brian/miniconda3/lib/python3.8/site-packages/matplotlib/mpl-data/fonts/ttf/}]
%%   \makeatletter\@ifpackageloaded{underscore}{}{\usepackage[strings]{underscore}}\makeatother
%%
\begingroup%
\makeatletter%
\begin{pgfpicture}%
\pgfpathrectangle{\pgfpointorigin}{\pgfqpoint{3.347362in}{2.125000in}}%
\pgfusepath{use as bounding box, clip}%
\begin{pgfscope}%
\pgfsetbuttcap%
\pgfsetmiterjoin%
\pgfsetlinewidth{0.000000pt}%
\definecolor{currentstroke}{rgb}{0.000000,0.000000,0.000000}%
\pgfsetstrokecolor{currentstroke}%
\pgfsetstrokeopacity{0.000000}%
\pgfsetdash{}{0pt}%
\pgfpathmoveto{\pgfqpoint{0.000000in}{0.000000in}}%
\pgfpathlineto{\pgfqpoint{3.347362in}{0.000000in}}%
\pgfpathlineto{\pgfqpoint{3.347362in}{2.125000in}}%
\pgfpathlineto{\pgfqpoint{0.000000in}{2.125000in}}%
\pgfpathlineto{\pgfqpoint{0.000000in}{0.000000in}}%
\pgfpathclose%
\pgfusepath{}%
\end{pgfscope}%
\begin{pgfscope}%
\pgfpathrectangle{\pgfqpoint{0.100000in}{0.100000in}}{\pgfqpoint{3.007045in}{1.925000in}}%
\pgfusepath{clip}%
\pgfsetbuttcap%
\pgfsetmiterjoin%
\pgfsetlinewidth{0.000000pt}%
\definecolor{currentstroke}{rgb}{0.000000,0.000000,0.000000}%
\pgfsetstrokecolor{currentstroke}%
\pgfsetstrokeopacity{0.000000}%
\pgfsetdash{}{0pt}%
\pgfpathmoveto{\pgfqpoint{0.100000in}{0.100000in}}%
\pgfpathlineto{\pgfqpoint{3.107045in}{0.100000in}}%
\pgfpathlineto{\pgfqpoint{3.107045in}{2.025000in}}%
\pgfpathlineto{\pgfqpoint{0.100000in}{2.025000in}}%
\pgfpathlineto{\pgfqpoint{0.100000in}{0.100000in}}%
\pgfpathclose%
\pgfusepath{}%
\end{pgfscope}%
\begin{pgfscope}%
\pgfpathrectangle{\pgfqpoint{0.100000in}{0.100000in}}{\pgfqpoint{3.007045in}{1.925000in}}%
\pgfusepath{clip}%
\pgfsetbuttcap%
\pgfsetmiterjoin%
\definecolor{currentfill}{rgb}{0.366644,0.646182,0.818547}%
\pgfsetfillcolor{currentfill}%
\pgfsetlinewidth{0.000000pt}%
\definecolor{currentstroke}{rgb}{0.000000,0.000000,0.000000}%
\pgfsetstrokecolor{currentstroke}%
\pgfsetstrokeopacity{0.000000}%
\pgfsetdash{}{0pt}%
\pgfpathmoveto{\pgfqpoint{1.787185in}{1.275241in}}%
\pgfpathlineto{\pgfqpoint{1.764595in}{1.275306in}}%
\pgfpathlineto{\pgfqpoint{1.764683in}{1.298101in}}%
\pgfpathlineto{\pgfqpoint{1.747728in}{1.298216in}}%
\pgfpathlineto{\pgfqpoint{1.744815in}{1.304579in}}%
\pgfpathlineto{\pgfqpoint{1.744960in}{1.321802in}}%
\pgfpathlineto{\pgfqpoint{1.784860in}{1.321668in}}%
\pgfpathlineto{\pgfqpoint{1.784942in}{1.304507in}}%
\pgfpathlineto{\pgfqpoint{1.787084in}{1.304515in}}%
\pgfpathlineto{\pgfqpoint{1.787185in}{1.275241in}}%
\pgfpathclose%
\pgfusepath{fill}%
\end{pgfscope}%
\begin{pgfscope}%
\pgfpathrectangle{\pgfqpoint{0.100000in}{0.100000in}}{\pgfqpoint{3.007045in}{1.925000in}}%
\pgfusepath{clip}%
\pgfsetbuttcap%
\pgfsetmiterjoin%
\definecolor{currentfill}{rgb}{0.401953,0.670296,0.832326}%
\pgfsetfillcolor{currentfill}%
\pgfsetlinewidth{0.000000pt}%
\definecolor{currentstroke}{rgb}{0.000000,0.000000,0.000000}%
\pgfsetstrokecolor{currentstroke}%
\pgfsetstrokeopacity{0.000000}%
\pgfsetdash{}{0pt}%
\pgfpathmoveto{\pgfqpoint{2.287383in}{1.019436in}}%
\pgfpathlineto{\pgfqpoint{2.292665in}{1.018214in}}%
\pgfpathlineto{\pgfqpoint{2.306270in}{1.030998in}}%
\pgfpathlineto{\pgfqpoint{2.302206in}{1.033847in}}%
\pgfpathlineto{\pgfqpoint{2.301947in}{1.038398in}}%
\pgfpathlineto{\pgfqpoint{2.310463in}{1.047437in}}%
\pgfpathlineto{\pgfqpoint{2.318137in}{1.056117in}}%
\pgfpathlineto{\pgfqpoint{2.321129in}{1.054650in}}%
\pgfpathlineto{\pgfqpoint{2.333634in}{1.044388in}}%
\pgfpathlineto{\pgfqpoint{2.332415in}{1.038665in}}%
\pgfpathlineto{\pgfqpoint{2.334697in}{1.028538in}}%
\pgfpathlineto{\pgfqpoint{2.332746in}{1.021710in}}%
\pgfpathlineto{\pgfqpoint{2.336554in}{1.012826in}}%
\pgfpathlineto{\pgfqpoint{2.341210in}{1.008147in}}%
\pgfpathlineto{\pgfqpoint{2.340163in}{1.003243in}}%
\pgfpathlineto{\pgfqpoint{2.336026in}{1.001374in}}%
\pgfpathlineto{\pgfqpoint{2.338194in}{0.987373in}}%
\pgfpathlineto{\pgfqpoint{2.334580in}{0.979199in}}%
\pgfpathlineto{\pgfqpoint{2.325950in}{0.982785in}}%
\pgfpathlineto{\pgfqpoint{2.317969in}{0.990753in}}%
\pgfpathlineto{\pgfqpoint{2.319984in}{0.992600in}}%
\pgfpathlineto{\pgfqpoint{2.315188in}{1.000641in}}%
\pgfpathlineto{\pgfqpoint{2.311789in}{1.005478in}}%
\pgfpathlineto{\pgfqpoint{2.285799in}{1.004151in}}%
\pgfpathlineto{\pgfqpoint{2.287383in}{1.019436in}}%
\pgfpathclose%
\pgfusepath{fill}%
\end{pgfscope}%
\begin{pgfscope}%
\pgfpathrectangle{\pgfqpoint{0.100000in}{0.100000in}}{\pgfqpoint{3.007045in}{1.925000in}}%
\pgfusepath{clip}%
\pgfsetbuttcap%
\pgfsetmiterjoin%
\definecolor{currentfill}{rgb}{0.166967,0.480692,0.729150}%
\pgfsetfillcolor{currentfill}%
\pgfsetlinewidth{0.000000pt}%
\definecolor{currentstroke}{rgb}{0.000000,0.000000,0.000000}%
\pgfsetstrokecolor{currentstroke}%
\pgfsetstrokeopacity{0.000000}%
\pgfsetdash{}{0pt}%
\pgfpathmoveto{\pgfqpoint{1.471341in}{0.833174in}}%
\pgfpathlineto{\pgfqpoint{1.469347in}{0.801604in}}%
\pgfpathlineto{\pgfqpoint{1.440696in}{0.803294in}}%
\pgfpathlineto{\pgfqpoint{1.442781in}{0.834883in}}%
\pgfpathlineto{\pgfqpoint{1.471341in}{0.833174in}}%
\pgfpathclose%
\pgfusepath{fill}%
\end{pgfscope}%
\begin{pgfscope}%
\pgfpathrectangle{\pgfqpoint{0.100000in}{0.100000in}}{\pgfqpoint{3.007045in}{1.925000in}}%
\pgfusepath{clip}%
\pgfsetbuttcap%
\pgfsetmiterjoin%
\definecolor{currentfill}{rgb}{0.417086,0.680631,0.838231}%
\pgfsetfillcolor{currentfill}%
\pgfsetlinewidth{0.000000pt}%
\definecolor{currentstroke}{rgb}{0.000000,0.000000,0.000000}%
\pgfsetstrokecolor{currentstroke}%
\pgfsetstrokeopacity{0.000000}%
\pgfsetdash{}{0pt}%
\pgfpathmoveto{\pgfqpoint{1.408374in}{1.098989in}}%
\pgfpathlineto{\pgfqpoint{1.408351in}{1.098572in}}%
\pgfpathlineto{\pgfqpoint{1.405795in}{1.064151in}}%
\pgfpathlineto{\pgfqpoint{1.405343in}{1.057957in}}%
\pgfpathlineto{\pgfqpoint{1.351564in}{1.062383in}}%
\pgfpathlineto{\pgfqpoint{1.300552in}{1.067248in}}%
\pgfpathlineto{\pgfqpoint{1.301086in}{1.073162in}}%
\pgfpathlineto{\pgfqpoint{1.271819in}{1.087203in}}%
\pgfpathlineto{\pgfqpoint{1.252696in}{1.088003in}}%
\pgfpathlineto{\pgfqpoint{1.251074in}{1.090450in}}%
\pgfpathlineto{\pgfqpoint{1.253600in}{1.112871in}}%
\pgfpathlineto{\pgfqpoint{1.259199in}{1.112273in}}%
\pgfpathlineto{\pgfqpoint{1.261017in}{1.129389in}}%
\pgfpathlineto{\pgfqpoint{1.306529in}{1.124802in}}%
\pgfpathlineto{\pgfqpoint{1.334664in}{1.121680in}}%
\pgfpathlineto{\pgfqpoint{1.333267in}{1.105178in}}%
\pgfpathlineto{\pgfqpoint{1.372398in}{1.101802in}}%
\pgfpathlineto{\pgfqpoint{1.408374in}{1.098989in}}%
\pgfpathclose%
\pgfusepath{fill}%
\end{pgfscope}%
\begin{pgfscope}%
\pgfpathrectangle{\pgfqpoint{0.100000in}{0.100000in}}{\pgfqpoint{3.007045in}{1.925000in}}%
\pgfusepath{clip}%
\pgfsetbuttcap%
\pgfsetmiterjoin%
\definecolor{currentfill}{rgb}{0.381776,0.656517,0.824452}%
\pgfsetfillcolor{currentfill}%
\pgfsetlinewidth{0.000000pt}%
\definecolor{currentstroke}{rgb}{0.000000,0.000000,0.000000}%
\pgfsetstrokecolor{currentstroke}%
\pgfsetstrokeopacity{0.000000}%
\pgfsetdash{}{0pt}%
\pgfpathmoveto{\pgfqpoint{1.769446in}{1.131864in}}%
\pgfpathlineto{\pgfqpoint{1.764032in}{1.131806in}}%
\pgfpathlineto{\pgfqpoint{1.763130in}{1.141541in}}%
\pgfpathlineto{\pgfqpoint{1.763181in}{1.160646in}}%
\pgfpathlineto{\pgfqpoint{1.743298in}{1.160738in}}%
\pgfpathlineto{\pgfqpoint{1.743716in}{1.176139in}}%
\pgfpathlineto{\pgfqpoint{1.755095in}{1.176082in}}%
\pgfpathlineto{\pgfqpoint{1.766486in}{1.176026in}}%
\pgfpathlineto{\pgfqpoint{1.769538in}{1.174111in}}%
\pgfpathlineto{\pgfqpoint{1.767101in}{1.168173in}}%
\pgfpathlineto{\pgfqpoint{1.772963in}{1.162326in}}%
\pgfpathlineto{\pgfqpoint{1.780164in}{1.150731in}}%
\pgfpathlineto{\pgfqpoint{1.777411in}{1.146354in}}%
\pgfpathlineto{\pgfqpoint{1.776978in}{1.132450in}}%
\pgfpathlineto{\pgfqpoint{1.769446in}{1.131864in}}%
\pgfpathclose%
\pgfusepath{fill}%
\end{pgfscope}%
\begin{pgfscope}%
\pgfpathrectangle{\pgfqpoint{0.100000in}{0.100000in}}{\pgfqpoint{3.007045in}{1.925000in}}%
\pgfusepath{clip}%
\pgfsetbuttcap%
\pgfsetmiterjoin%
\definecolor{currentfill}{rgb}{0.346467,0.632403,0.810673}%
\pgfsetfillcolor{currentfill}%
\pgfsetlinewidth{0.000000pt}%
\definecolor{currentstroke}{rgb}{0.000000,0.000000,0.000000}%
\pgfsetstrokecolor{currentstroke}%
\pgfsetstrokeopacity{0.000000}%
\pgfsetdash{}{0pt}%
\pgfpathmoveto{\pgfqpoint{2.317320in}{0.826818in}}%
\pgfpathlineto{\pgfqpoint{2.298990in}{0.824653in}}%
\pgfpathlineto{\pgfqpoint{2.297886in}{0.828687in}}%
\pgfpathlineto{\pgfqpoint{2.290820in}{0.827401in}}%
\pgfpathlineto{\pgfqpoint{2.279304in}{0.827039in}}%
\pgfpathlineto{\pgfqpoint{2.271751in}{0.844592in}}%
\pgfpathlineto{\pgfqpoint{2.275437in}{0.845959in}}%
\pgfpathlineto{\pgfqpoint{2.282179in}{0.853795in}}%
\pgfpathlineto{\pgfqpoint{2.284621in}{0.861875in}}%
\pgfpathlineto{\pgfqpoint{2.287764in}{0.864843in}}%
\pgfpathlineto{\pgfqpoint{2.286587in}{0.869005in}}%
\pgfpathlineto{\pgfqpoint{2.311838in}{0.871613in}}%
\pgfpathlineto{\pgfqpoint{2.311402in}{0.880390in}}%
\pgfpathlineto{\pgfqpoint{2.305031in}{0.879714in}}%
\pgfpathlineto{\pgfqpoint{2.306322in}{0.886459in}}%
\pgfpathlineto{\pgfqpoint{2.310101in}{0.886859in}}%
\pgfpathlineto{\pgfqpoint{2.313442in}{0.898402in}}%
\pgfpathlineto{\pgfqpoint{2.332547in}{0.900599in}}%
\pgfpathlineto{\pgfqpoint{2.333466in}{0.891984in}}%
\pgfpathlineto{\pgfqpoint{2.335741in}{0.890096in}}%
\pgfpathlineto{\pgfqpoint{2.332579in}{0.883358in}}%
\pgfpathlineto{\pgfqpoint{2.334077in}{0.871563in}}%
\pgfpathlineto{\pgfqpoint{2.337348in}{0.840608in}}%
\pgfpathlineto{\pgfqpoint{2.322977in}{0.839287in}}%
\pgfpathlineto{\pgfqpoint{2.320886in}{0.830332in}}%
\pgfpathlineto{\pgfqpoint{2.317320in}{0.826818in}}%
\pgfpathclose%
\pgfusepath{fill}%
\end{pgfscope}%
\begin{pgfscope}%
\pgfpathrectangle{\pgfqpoint{0.100000in}{0.100000in}}{\pgfqpoint{3.007045in}{1.925000in}}%
\pgfusepath{clip}%
\pgfsetbuttcap%
\pgfsetmiterjoin%
\definecolor{currentfill}{rgb}{0.244106,0.557832,0.768889}%
\pgfsetfillcolor{currentfill}%
\pgfsetlinewidth{0.000000pt}%
\definecolor{currentstroke}{rgb}{0.000000,0.000000,0.000000}%
\pgfsetstrokecolor{currentstroke}%
\pgfsetstrokeopacity{0.000000}%
\pgfsetdash{}{0pt}%
\pgfpathmoveto{\pgfqpoint{1.447246in}{1.572779in}}%
\pgfpathlineto{\pgfqpoint{1.445034in}{1.544102in}}%
\pgfpathlineto{\pgfqpoint{1.400379in}{1.547840in}}%
\pgfpathlineto{\pgfqpoint{1.401374in}{1.559284in}}%
\pgfpathlineto{\pgfqpoint{1.406341in}{1.607594in}}%
\pgfpathlineto{\pgfqpoint{1.403892in}{1.607820in}}%
\pgfpathlineto{\pgfqpoint{1.405844in}{1.630008in}}%
\pgfpathlineto{\pgfqpoint{1.409091in}{1.629777in}}%
\pgfpathlineto{\pgfqpoint{1.410950in}{1.652859in}}%
\pgfpathlineto{\pgfqpoint{1.448770in}{1.649792in}}%
\pgfpathlineto{\pgfqpoint{1.450846in}{1.649617in}}%
\pgfpathlineto{\pgfqpoint{1.449255in}{1.626605in}}%
\pgfpathlineto{\pgfqpoint{1.451489in}{1.626431in}}%
\pgfpathlineto{\pgfqpoint{1.450280in}{1.611196in}}%
\pgfpathlineto{\pgfqpoint{1.447246in}{1.572779in}}%
\pgfpathclose%
\pgfusepath{fill}%
\end{pgfscope}%
\begin{pgfscope}%
\pgfpathrectangle{\pgfqpoint{0.100000in}{0.100000in}}{\pgfqpoint{3.007045in}{1.925000in}}%
\pgfusepath{clip}%
\pgfsetbuttcap%
\pgfsetmiterjoin%
\definecolor{currentfill}{rgb}{0.730950,0.839477,0.921323}%
\pgfsetfillcolor{currentfill}%
\pgfsetlinewidth{0.000000pt}%
\definecolor{currentstroke}{rgb}{0.000000,0.000000,0.000000}%
\pgfsetstrokecolor{currentstroke}%
\pgfsetstrokeopacity{0.000000}%
\pgfsetdash{}{0pt}%
\pgfpathmoveto{\pgfqpoint{2.765291in}{1.169198in}}%
\pgfpathlineto{\pgfqpoint{2.770616in}{1.173606in}}%
\pgfpathlineto{\pgfqpoint{2.789514in}{1.179938in}}%
\pgfpathlineto{\pgfqpoint{2.785420in}{1.167710in}}%
\pgfpathlineto{\pgfqpoint{2.782226in}{1.168112in}}%
\pgfpathlineto{\pgfqpoint{2.778730in}{1.161873in}}%
\pgfpathlineto{\pgfqpoint{2.776737in}{1.148749in}}%
\pgfpathlineto{\pgfqpoint{2.778073in}{1.146484in}}%
\pgfpathlineto{\pgfqpoint{2.774225in}{1.130181in}}%
\pgfpathlineto{\pgfqpoint{2.772699in}{1.120440in}}%
\pgfpathlineto{\pgfqpoint{2.768773in}{1.114106in}}%
\pgfpathlineto{\pgfqpoint{2.765038in}{1.113269in}}%
\pgfpathlineto{\pgfqpoint{2.760275in}{1.121718in}}%
\pgfpathlineto{\pgfqpoint{2.760413in}{1.142327in}}%
\pgfpathlineto{\pgfqpoint{2.763321in}{1.150894in}}%
\pgfpathlineto{\pgfqpoint{2.763523in}{1.158416in}}%
\pgfpathlineto{\pgfqpoint{2.768180in}{1.159798in}}%
\pgfpathlineto{\pgfqpoint{2.769212in}{1.165623in}}%
\pgfpathlineto{\pgfqpoint{2.765291in}{1.169198in}}%
\pgfpathclose%
\pgfusepath{fill}%
\end{pgfscope}%
\begin{pgfscope}%
\pgfpathrectangle{\pgfqpoint{0.100000in}{0.100000in}}{\pgfqpoint{3.007045in}{1.925000in}}%
\pgfusepath{clip}%
\pgfsetbuttcap%
\pgfsetmiterjoin%
\definecolor{currentfill}{rgb}{0.627605,0.795556,0.885152}%
\pgfsetfillcolor{currentfill}%
\pgfsetlinewidth{0.000000pt}%
\definecolor{currentstroke}{rgb}{0.000000,0.000000,0.000000}%
\pgfsetstrokecolor{currentstroke}%
\pgfsetstrokeopacity{0.000000}%
\pgfsetdash{}{0pt}%
\pgfpathmoveto{\pgfqpoint{2.090405in}{1.513286in}}%
\pgfpathlineto{\pgfqpoint{2.073413in}{1.512435in}}%
\pgfpathlineto{\pgfqpoint{2.073048in}{1.518236in}}%
\pgfpathlineto{\pgfqpoint{2.043986in}{1.516491in}}%
\pgfpathlineto{\pgfqpoint{2.042126in}{1.545379in}}%
\pgfpathlineto{\pgfqpoint{2.032705in}{1.544832in}}%
\pgfpathlineto{\pgfqpoint{2.031351in}{1.567858in}}%
\pgfpathlineto{\pgfqpoint{2.054639in}{1.569212in}}%
\pgfpathlineto{\pgfqpoint{2.054884in}{1.563482in}}%
\pgfpathlineto{\pgfqpoint{2.066412in}{1.564272in}}%
\pgfpathlineto{\pgfqpoint{2.067108in}{1.552783in}}%
\pgfpathlineto{\pgfqpoint{2.069326in}{1.547117in}}%
\pgfpathlineto{\pgfqpoint{2.076668in}{1.547607in}}%
\pgfpathlineto{\pgfqpoint{2.077656in}{1.530311in}}%
\pgfpathlineto{\pgfqpoint{2.088935in}{1.530951in}}%
\pgfpathlineto{\pgfqpoint{2.090405in}{1.513286in}}%
\pgfpathclose%
\pgfusepath{fill}%
\end{pgfscope}%
\begin{pgfscope}%
\pgfpathrectangle{\pgfqpoint{0.100000in}{0.100000in}}{\pgfqpoint{3.007045in}{1.925000in}}%
\pgfusepath{clip}%
\pgfsetbuttcap%
\pgfsetmiterjoin%
\definecolor{currentfill}{rgb}{0.825928,0.891795,0.953741}%
\pgfsetfillcolor{currentfill}%
\pgfsetlinewidth{0.000000pt}%
\definecolor{currentstroke}{rgb}{0.000000,0.000000,0.000000}%
\pgfsetstrokecolor{currentstroke}%
\pgfsetstrokeopacity{0.000000}%
\pgfsetdash{}{0pt}%
\pgfpathmoveto{\pgfqpoint{2.836904in}{1.462520in}}%
\pgfpathlineto{\pgfqpoint{2.816097in}{1.458112in}}%
\pgfpathlineto{\pgfqpoint{2.815641in}{1.458005in}}%
\pgfpathlineto{\pgfqpoint{2.814539in}{1.460221in}}%
\pgfpathlineto{\pgfqpoint{2.815796in}{1.505313in}}%
\pgfpathlineto{\pgfqpoint{2.813736in}{1.511035in}}%
\pgfpathlineto{\pgfqpoint{2.807383in}{1.541913in}}%
\pgfpathlineto{\pgfqpoint{2.825651in}{1.545463in}}%
\pgfpathlineto{\pgfqpoint{2.828637in}{1.543127in}}%
\pgfpathlineto{\pgfqpoint{2.828599in}{1.533431in}}%
\pgfpathlineto{\pgfqpoint{2.822507in}{1.532326in}}%
\pgfpathlineto{\pgfqpoint{2.824976in}{1.520081in}}%
\pgfpathlineto{\pgfqpoint{2.829097in}{1.520876in}}%
\pgfpathlineto{\pgfqpoint{2.831656in}{1.508759in}}%
\pgfpathlineto{\pgfqpoint{2.827848in}{1.505323in}}%
\pgfpathlineto{\pgfqpoint{2.832224in}{1.502258in}}%
\pgfpathlineto{\pgfqpoint{2.833508in}{1.484777in}}%
\pgfpathlineto{\pgfqpoint{2.836904in}{1.462520in}}%
\pgfpathclose%
\pgfusepath{fill}%
\end{pgfscope}%
\begin{pgfscope}%
\pgfpathrectangle{\pgfqpoint{0.100000in}{0.100000in}}{\pgfqpoint{3.007045in}{1.925000in}}%
\pgfusepath{clip}%
\pgfsetbuttcap%
\pgfsetmiterjoin%
\definecolor{currentfill}{rgb}{0.516863,0.735748,0.860192}%
\pgfsetfillcolor{currentfill}%
\pgfsetlinewidth{0.000000pt}%
\definecolor{currentstroke}{rgb}{0.000000,0.000000,0.000000}%
\pgfsetstrokecolor{currentstroke}%
\pgfsetstrokeopacity{0.000000}%
\pgfsetdash{}{0pt}%
\pgfpathmoveto{\pgfqpoint{2.429643in}{1.109432in}}%
\pgfpathlineto{\pgfqpoint{2.425369in}{1.103463in}}%
\pgfpathlineto{\pgfqpoint{2.420050in}{1.106990in}}%
\pgfpathlineto{\pgfqpoint{2.421148in}{1.110912in}}%
\pgfpathlineto{\pgfqpoint{2.410869in}{1.119825in}}%
\pgfpathlineto{\pgfqpoint{2.411679in}{1.127319in}}%
\pgfpathlineto{\pgfqpoint{2.401483in}{1.125255in}}%
\pgfpathlineto{\pgfqpoint{2.396061in}{1.120463in}}%
\pgfpathlineto{\pgfqpoint{2.398193in}{1.116290in}}%
\pgfpathlineto{\pgfqpoint{2.392249in}{1.108544in}}%
\pgfpathlineto{\pgfqpoint{2.383396in}{1.107611in}}%
\pgfpathlineto{\pgfqpoint{2.379152in}{1.110286in}}%
\pgfpathlineto{\pgfqpoint{2.370833in}{1.107548in}}%
\pgfpathlineto{\pgfqpoint{2.359204in}{1.116262in}}%
\pgfpathlineto{\pgfqpoint{2.365037in}{1.122682in}}%
\pgfpathlineto{\pgfqpoint{2.364836in}{1.127691in}}%
\pgfpathlineto{\pgfqpoint{2.368680in}{1.131320in}}%
\pgfpathlineto{\pgfqpoint{2.372393in}{1.126861in}}%
\pgfpathlineto{\pgfqpoint{2.378692in}{1.128851in}}%
\pgfpathlineto{\pgfqpoint{2.377693in}{1.133822in}}%
\pgfpathlineto{\pgfqpoint{2.379735in}{1.139390in}}%
\pgfpathlineto{\pgfqpoint{2.384672in}{1.146612in}}%
\pgfpathlineto{\pgfqpoint{2.385587in}{1.155405in}}%
\pgfpathlineto{\pgfqpoint{2.392277in}{1.158355in}}%
\pgfpathlineto{\pgfqpoint{2.395723in}{1.148881in}}%
\pgfpathlineto{\pgfqpoint{2.401519in}{1.149007in}}%
\pgfpathlineto{\pgfqpoint{2.404127in}{1.159387in}}%
\pgfpathlineto{\pgfqpoint{2.403876in}{1.166074in}}%
\pgfpathlineto{\pgfqpoint{2.407704in}{1.166293in}}%
\pgfpathlineto{\pgfqpoint{2.407930in}{1.161976in}}%
\pgfpathlineto{\pgfqpoint{2.413032in}{1.162228in}}%
\pgfpathlineto{\pgfqpoint{2.414411in}{1.156435in}}%
\pgfpathlineto{\pgfqpoint{2.420478in}{1.156867in}}%
\pgfpathlineto{\pgfqpoint{2.420941in}{1.150887in}}%
\pgfpathlineto{\pgfqpoint{2.428157in}{1.152335in}}%
\pgfpathlineto{\pgfqpoint{2.437602in}{1.145887in}}%
\pgfpathlineto{\pgfqpoint{2.438948in}{1.139413in}}%
\pgfpathlineto{\pgfqpoint{2.431774in}{1.134957in}}%
\pgfpathlineto{\pgfqpoint{2.428704in}{1.126732in}}%
\pgfpathlineto{\pgfqpoint{2.428858in}{1.122661in}}%
\pgfpathlineto{\pgfqpoint{2.433994in}{1.115719in}}%
\pgfpathlineto{\pgfqpoint{2.429643in}{1.109432in}}%
\pgfpathclose%
\pgfusepath{fill}%
\end{pgfscope}%
\begin{pgfscope}%
\pgfpathrectangle{\pgfqpoint{0.100000in}{0.100000in}}{\pgfqpoint{3.007045in}{1.925000in}}%
\pgfusepath{clip}%
\pgfsetbuttcap%
\pgfsetmiterjoin%
\definecolor{currentfill}{rgb}{0.765398,0.854118,0.933379}%
\pgfsetfillcolor{currentfill}%
\pgfsetlinewidth{0.000000pt}%
\definecolor{currentstroke}{rgb}{0.000000,0.000000,0.000000}%
\pgfsetstrokecolor{currentstroke}%
\pgfsetstrokeopacity{0.000000}%
\pgfsetdash{}{0pt}%
\pgfpathmoveto{\pgfqpoint{1.194470in}{0.906193in}}%
\pgfpathlineto{\pgfqpoint{1.166043in}{0.909544in}}%
\pgfpathlineto{\pgfqpoint{1.171643in}{0.956432in}}%
\pgfpathlineto{\pgfqpoint{1.164202in}{0.962160in}}%
\pgfpathlineto{\pgfqpoint{1.165289in}{0.970829in}}%
\pgfpathlineto{\pgfqpoint{1.173700in}{0.972532in}}%
\pgfpathlineto{\pgfqpoint{1.139908in}{0.976686in}}%
\pgfpathlineto{\pgfqpoint{1.141795in}{0.990914in}}%
\pgfpathlineto{\pgfqpoint{1.102849in}{0.996413in}}%
\pgfpathlineto{\pgfqpoint{1.108498in}{1.034565in}}%
\pgfpathlineto{\pgfqpoint{1.118519in}{1.040536in}}%
\pgfpathlineto{\pgfqpoint{1.120573in}{1.045885in}}%
\pgfpathlineto{\pgfqpoint{1.169817in}{1.038957in}}%
\pgfpathlineto{\pgfqpoint{1.235453in}{1.031177in}}%
\pgfpathlineto{\pgfqpoint{1.235393in}{1.026018in}}%
\pgfpathlineto{\pgfqpoint{1.231839in}{1.013157in}}%
\pgfpathlineto{\pgfqpoint{1.225227in}{1.010762in}}%
\pgfpathlineto{\pgfqpoint{1.225457in}{0.993376in}}%
\pgfpathlineto{\pgfqpoint{1.223675in}{0.983885in}}%
\pgfpathlineto{\pgfqpoint{1.221151in}{0.983550in}}%
\pgfpathlineto{\pgfqpoint{1.218512in}{0.975812in}}%
\pgfpathlineto{\pgfqpoint{1.209230in}{0.966315in}}%
\pgfpathlineto{\pgfqpoint{1.201592in}{0.967633in}}%
\pgfpathlineto{\pgfqpoint{1.194470in}{0.906193in}}%
\pgfpathclose%
\pgfusepath{fill}%
\end{pgfscope}%
\begin{pgfscope}%
\pgfpathrectangle{\pgfqpoint{0.100000in}{0.100000in}}{\pgfqpoint{3.007045in}{1.925000in}}%
\pgfusepath{clip}%
\pgfsetbuttcap%
\pgfsetmiterjoin%
\definecolor{currentfill}{rgb}{0.381776,0.656517,0.824452}%
\pgfsetfillcolor{currentfill}%
\pgfsetlinewidth{0.000000pt}%
\definecolor{currentstroke}{rgb}{0.000000,0.000000,0.000000}%
\pgfsetstrokecolor{currentstroke}%
\pgfsetstrokeopacity{0.000000}%
\pgfsetdash{}{0pt}%
\pgfpathmoveto{\pgfqpoint{2.022844in}{1.028625in}}%
\pgfpathlineto{\pgfqpoint{2.005455in}{1.027915in}}%
\pgfpathlineto{\pgfqpoint{2.005554in}{1.024952in}}%
\pgfpathlineto{\pgfqpoint{1.995876in}{1.024384in}}%
\pgfpathlineto{\pgfqpoint{1.993598in}{1.031047in}}%
\pgfpathlineto{\pgfqpoint{1.993117in}{1.046153in}}%
\pgfpathlineto{\pgfqpoint{1.981728in}{1.045692in}}%
\pgfpathlineto{\pgfqpoint{1.973010in}{1.044275in}}%
\pgfpathlineto{\pgfqpoint{1.972796in}{1.051471in}}%
\pgfpathlineto{\pgfqpoint{1.975453in}{1.054398in}}%
\pgfpathlineto{\pgfqpoint{1.974361in}{1.084952in}}%
\pgfpathlineto{\pgfqpoint{1.990680in}{1.085683in}}%
\pgfpathlineto{\pgfqpoint{2.001501in}{1.072967in}}%
\pgfpathlineto{\pgfqpoint{2.010046in}{1.075951in}}%
\pgfpathlineto{\pgfqpoint{2.014367in}{1.079611in}}%
\pgfpathlineto{\pgfqpoint{2.020734in}{1.079515in}}%
\pgfpathlineto{\pgfqpoint{2.024754in}{1.078038in}}%
\pgfpathlineto{\pgfqpoint{2.031875in}{1.071974in}}%
\pgfpathlineto{\pgfqpoint{2.034604in}{1.066814in}}%
\pgfpathlineto{\pgfqpoint{2.040262in}{1.068526in}}%
\pgfpathlineto{\pgfqpoint{2.045818in}{1.065000in}}%
\pgfpathlineto{\pgfqpoint{2.054473in}{1.056687in}}%
\pgfpathlineto{\pgfqpoint{2.058159in}{1.055368in}}%
\pgfpathlineto{\pgfqpoint{2.060389in}{1.049493in}}%
\pgfpathlineto{\pgfqpoint{2.058254in}{1.047208in}}%
\pgfpathlineto{\pgfqpoint{2.053627in}{1.049022in}}%
\pgfpathlineto{\pgfqpoint{2.025561in}{1.047408in}}%
\pgfpathlineto{\pgfqpoint{2.026620in}{1.028636in}}%
\pgfpathlineto{\pgfqpoint{2.022844in}{1.028625in}}%
\pgfpathclose%
\pgfusepath{fill}%
\end{pgfscope}%
\begin{pgfscope}%
\pgfpathrectangle{\pgfqpoint{0.100000in}{0.100000in}}{\pgfqpoint{3.007045in}{1.925000in}}%
\pgfusepath{clip}%
\pgfsetbuttcap%
\pgfsetmiterjoin%
\definecolor{currentfill}{rgb}{0.541961,0.749527,0.865606}%
\pgfsetfillcolor{currentfill}%
\pgfsetlinewidth{0.000000pt}%
\definecolor{currentstroke}{rgb}{0.000000,0.000000,0.000000}%
\pgfsetstrokecolor{currentstroke}%
\pgfsetstrokeopacity{0.000000}%
\pgfsetdash{}{0pt}%
\pgfpathmoveto{\pgfqpoint{2.309942in}{0.708924in}}%
\pgfpathlineto{\pgfqpoint{2.309841in}{0.700126in}}%
\pgfpathlineto{\pgfqpoint{2.301073in}{0.698033in}}%
\pgfpathlineto{\pgfqpoint{2.297933in}{0.689896in}}%
\pgfpathlineto{\pgfqpoint{2.290454in}{0.689136in}}%
\pgfpathlineto{\pgfqpoint{2.289927in}{0.694879in}}%
\pgfpathlineto{\pgfqpoint{2.284645in}{0.694345in}}%
\pgfpathlineto{\pgfqpoint{2.283592in}{0.699581in}}%
\pgfpathlineto{\pgfqpoint{2.277794in}{0.699248in}}%
\pgfpathlineto{\pgfqpoint{2.276404in}{0.718014in}}%
\pgfpathlineto{\pgfqpoint{2.274136in}{0.722813in}}%
\pgfpathlineto{\pgfqpoint{2.281053in}{0.729035in}}%
\pgfpathlineto{\pgfqpoint{2.285996in}{0.729613in}}%
\pgfpathlineto{\pgfqpoint{2.285327in}{0.735328in}}%
\pgfpathlineto{\pgfqpoint{2.290941in}{0.736895in}}%
\pgfpathlineto{\pgfqpoint{2.290204in}{0.743581in}}%
\pgfpathlineto{\pgfqpoint{2.295738in}{0.746225in}}%
\pgfpathlineto{\pgfqpoint{2.312802in}{0.748173in}}%
\pgfpathlineto{\pgfqpoint{2.321001in}{0.750113in}}%
\pgfpathlineto{\pgfqpoint{2.323388in}{0.743508in}}%
\pgfpathlineto{\pgfqpoint{2.330141in}{0.735377in}}%
\pgfpathlineto{\pgfqpoint{2.327177in}{0.732545in}}%
\pgfpathlineto{\pgfqpoint{2.311941in}{0.730610in}}%
\pgfpathlineto{\pgfqpoint{2.312554in}{0.726822in}}%
\pgfpathlineto{\pgfqpoint{2.306778in}{0.726142in}}%
\pgfpathlineto{\pgfqpoint{2.308027in}{0.714591in}}%
\pgfpathlineto{\pgfqpoint{2.309942in}{0.708924in}}%
\pgfpathclose%
\pgfusepath{fill}%
\end{pgfscope}%
\begin{pgfscope}%
\pgfpathrectangle{\pgfqpoint{0.100000in}{0.100000in}}{\pgfqpoint{3.007045in}{1.925000in}}%
\pgfusepath{clip}%
\pgfsetbuttcap%
\pgfsetmiterjoin%
\definecolor{currentfill}{rgb}{0.296025,0.597955,0.790988}%
\pgfsetfillcolor{currentfill}%
\pgfsetlinewidth{0.000000pt}%
\definecolor{currentstroke}{rgb}{0.000000,0.000000,0.000000}%
\pgfsetstrokecolor{currentstroke}%
\pgfsetstrokeopacity{0.000000}%
\pgfsetdash{}{0pt}%
\pgfpathmoveto{\pgfqpoint{0.829274in}{1.562729in}}%
\pgfpathlineto{\pgfqpoint{0.828248in}{1.558524in}}%
\pgfpathlineto{\pgfqpoint{0.823175in}{1.550976in}}%
\pgfpathlineto{\pgfqpoint{0.824049in}{1.547593in}}%
\pgfpathlineto{\pgfqpoint{0.819243in}{1.542361in}}%
\pgfpathlineto{\pgfqpoint{0.809080in}{1.497950in}}%
\pgfpathlineto{\pgfqpoint{0.811385in}{1.497272in}}%
\pgfpathlineto{\pgfqpoint{0.798116in}{1.438303in}}%
\pgfpathlineto{\pgfqpoint{0.734607in}{1.453046in}}%
\pgfpathlineto{\pgfqpoint{0.703710in}{1.460774in}}%
\pgfpathlineto{\pgfqpoint{0.703346in}{1.460924in}}%
\pgfpathlineto{\pgfqpoint{0.729959in}{1.568628in}}%
\pgfpathlineto{\pgfqpoint{0.733534in}{1.580107in}}%
\pgfpathlineto{\pgfqpoint{0.736446in}{1.580574in}}%
\pgfpathlineto{\pgfqpoint{0.739760in}{1.573820in}}%
\pgfpathlineto{\pgfqpoint{0.746496in}{1.573138in}}%
\pgfpathlineto{\pgfqpoint{0.749229in}{1.584343in}}%
\pgfpathlineto{\pgfqpoint{0.755702in}{1.582709in}}%
\pgfpathlineto{\pgfqpoint{0.758886in}{1.587771in}}%
\pgfpathlineto{\pgfqpoint{0.762552in}{1.586835in}}%
\pgfpathlineto{\pgfqpoint{0.763908in}{1.592366in}}%
\pgfpathlineto{\pgfqpoint{0.768216in}{1.591309in}}%
\pgfpathlineto{\pgfqpoint{0.771340in}{1.602215in}}%
\pgfpathlineto{\pgfqpoint{0.775430in}{1.609514in}}%
\pgfpathlineto{\pgfqpoint{0.783037in}{1.611433in}}%
\pgfpathlineto{\pgfqpoint{0.780379in}{1.600195in}}%
\pgfpathlineto{\pgfqpoint{0.777625in}{1.600841in}}%
\pgfpathlineto{\pgfqpoint{0.774977in}{1.589715in}}%
\pgfpathlineto{\pgfqpoint{0.779546in}{1.588394in}}%
\pgfpathlineto{\pgfqpoint{0.781320in}{1.594007in}}%
\pgfpathlineto{\pgfqpoint{0.808189in}{1.587646in}}%
\pgfpathlineto{\pgfqpoint{0.811192in}{1.587476in}}%
\pgfpathlineto{\pgfqpoint{0.820142in}{1.591987in}}%
\pgfpathlineto{\pgfqpoint{0.824845in}{1.588003in}}%
\pgfpathlineto{\pgfqpoint{0.824028in}{1.581837in}}%
\pgfpathlineto{\pgfqpoint{0.829905in}{1.577513in}}%
\pgfpathlineto{\pgfqpoint{0.827488in}{1.571131in}}%
\pgfpathlineto{\pgfqpoint{0.829274in}{1.562729in}}%
\pgfpathclose%
\pgfusepath{fill}%
\end{pgfscope}%
\begin{pgfscope}%
\pgfpathrectangle{\pgfqpoint{0.100000in}{0.100000in}}{\pgfqpoint{3.007045in}{1.925000in}}%
\pgfusepath{clip}%
\pgfsetbuttcap%
\pgfsetmiterjoin%
\definecolor{currentfill}{rgb}{0.231926,0.545652,0.762614}%
\pgfsetfillcolor{currentfill}%
\pgfsetlinewidth{0.000000pt}%
\definecolor{currentstroke}{rgb}{0.000000,0.000000,0.000000}%
\pgfsetstrokecolor{currentstroke}%
\pgfsetstrokeopacity{0.000000}%
\pgfsetdash{}{0pt}%
\pgfpathmoveto{\pgfqpoint{1.910648in}{0.511564in}}%
\pgfpathlineto{\pgfqpoint{1.903450in}{0.513741in}}%
\pgfpathlineto{\pgfqpoint{1.888369in}{0.520432in}}%
\pgfpathlineto{\pgfqpoint{1.876777in}{0.523740in}}%
\pgfpathlineto{\pgfqpoint{1.868009in}{0.522526in}}%
\pgfpathlineto{\pgfqpoint{1.859158in}{0.523013in}}%
\pgfpathlineto{\pgfqpoint{1.843789in}{0.520220in}}%
\pgfpathlineto{\pgfqpoint{1.839662in}{0.517608in}}%
\pgfpathlineto{\pgfqpoint{1.834496in}{0.525796in}}%
\pgfpathlineto{\pgfqpoint{1.840585in}{0.532547in}}%
\pgfpathlineto{\pgfqpoint{1.842229in}{0.537436in}}%
\pgfpathlineto{\pgfqpoint{1.846781in}{0.541896in}}%
\pgfpathlineto{\pgfqpoint{1.846835in}{0.557616in}}%
\pgfpathlineto{\pgfqpoint{1.843280in}{0.560440in}}%
\pgfpathlineto{\pgfqpoint{1.847088in}{0.567676in}}%
\pgfpathlineto{\pgfqpoint{1.844939in}{0.573686in}}%
\pgfpathlineto{\pgfqpoint{1.847738in}{0.581014in}}%
\pgfpathlineto{\pgfqpoint{1.850658in}{0.583534in}}%
\pgfpathlineto{\pgfqpoint{1.853934in}{0.596710in}}%
\pgfpathlineto{\pgfqpoint{1.856414in}{0.600398in}}%
\pgfpathlineto{\pgfqpoint{1.854154in}{0.604333in}}%
\pgfpathlineto{\pgfqpoint{1.854631in}{0.617364in}}%
\pgfpathlineto{\pgfqpoint{1.855514in}{0.623416in}}%
\pgfpathlineto{\pgfqpoint{1.860787in}{0.623525in}}%
\pgfpathlineto{\pgfqpoint{1.863641in}{0.629386in}}%
\pgfpathlineto{\pgfqpoint{1.872399in}{0.629573in}}%
\pgfpathlineto{\pgfqpoint{1.883960in}{0.629796in}}%
\pgfpathlineto{\pgfqpoint{1.886870in}{0.628687in}}%
\pgfpathlineto{\pgfqpoint{1.892685in}{0.626874in}}%
\pgfpathlineto{\pgfqpoint{1.895632in}{0.622616in}}%
\pgfpathlineto{\pgfqpoint{1.896550in}{0.598540in}}%
\pgfpathlineto{\pgfqpoint{1.909431in}{0.599271in}}%
\pgfpathlineto{\pgfqpoint{1.910006in}{0.578352in}}%
\pgfpathlineto{\pgfqpoint{1.908086in}{0.571754in}}%
\pgfpathlineto{\pgfqpoint{1.909910in}{0.542133in}}%
\pgfpathlineto{\pgfqpoint{1.910648in}{0.511564in}}%
\pgfpathclose%
\pgfusepath{fill}%
\end{pgfscope}%
\begin{pgfscope}%
\pgfpathrectangle{\pgfqpoint{0.100000in}{0.100000in}}{\pgfqpoint{3.007045in}{1.925000in}}%
\pgfusepath{clip}%
\pgfsetbuttcap%
\pgfsetmiterjoin%
\definecolor{currentfill}{rgb}{0.784591,0.864237,0.939962}%
\pgfsetfillcolor{currentfill}%
\pgfsetlinewidth{0.000000pt}%
\definecolor{currentstroke}{rgb}{0.000000,0.000000,0.000000}%
\pgfsetstrokecolor{currentstroke}%
\pgfsetstrokeopacity{0.000000}%
\pgfsetdash{}{0pt}%
\pgfpathmoveto{\pgfqpoint{0.677176in}{1.047687in}}%
\pgfpathlineto{\pgfqpoint{0.672324in}{1.047770in}}%
\pgfpathlineto{\pgfqpoint{0.654898in}{1.051840in}}%
\pgfpathlineto{\pgfqpoint{0.593532in}{1.066855in}}%
\pgfpathlineto{\pgfqpoint{0.573744in}{1.071996in}}%
\pgfpathlineto{\pgfqpoint{0.554172in}{1.076466in}}%
\pgfpathlineto{\pgfqpoint{0.557260in}{1.082859in}}%
\pgfpathlineto{\pgfqpoint{0.556142in}{1.096690in}}%
\pgfpathlineto{\pgfqpoint{0.557824in}{1.101480in}}%
\pgfpathlineto{\pgfqpoint{0.556496in}{1.110688in}}%
\pgfpathlineto{\pgfqpoint{0.558630in}{1.113262in}}%
\pgfpathlineto{\pgfqpoint{0.553369in}{1.124234in}}%
\pgfpathlineto{\pgfqpoint{0.554104in}{1.127797in}}%
\pgfpathlineto{\pgfqpoint{0.551392in}{1.137382in}}%
\pgfpathlineto{\pgfqpoint{0.551954in}{1.148285in}}%
\pgfpathlineto{\pgfqpoint{0.554321in}{1.151293in}}%
\pgfpathlineto{\pgfqpoint{0.552868in}{1.161985in}}%
\pgfpathlineto{\pgfqpoint{0.546742in}{1.170366in}}%
\pgfpathlineto{\pgfqpoint{0.543660in}{1.171437in}}%
\pgfpathlineto{\pgfqpoint{0.543712in}{1.184135in}}%
\pgfpathlineto{\pgfqpoint{0.540581in}{1.185427in}}%
\pgfpathlineto{\pgfqpoint{0.543469in}{1.191912in}}%
\pgfpathlineto{\pgfqpoint{0.539193in}{1.195591in}}%
\pgfpathlineto{\pgfqpoint{0.537345in}{1.200997in}}%
\pgfpathlineto{\pgfqpoint{0.532644in}{1.204682in}}%
\pgfpathlineto{\pgfqpoint{0.532191in}{1.211551in}}%
\pgfpathlineto{\pgfqpoint{0.530069in}{1.215499in}}%
\pgfpathlineto{\pgfqpoint{0.522889in}{1.217818in}}%
\pgfpathlineto{\pgfqpoint{0.527157in}{1.220456in}}%
\pgfpathlineto{\pgfqpoint{0.527593in}{1.228218in}}%
\pgfpathlineto{\pgfqpoint{0.524524in}{1.231551in}}%
\pgfpathlineto{\pgfqpoint{0.525021in}{1.240725in}}%
\pgfpathlineto{\pgfqpoint{0.514807in}{1.248637in}}%
\pgfpathlineto{\pgfqpoint{0.513336in}{1.257851in}}%
\pgfpathlineto{\pgfqpoint{0.514663in}{1.260246in}}%
\pgfpathlineto{\pgfqpoint{0.519485in}{1.264180in}}%
\pgfpathlineto{\pgfqpoint{0.522497in}{1.269886in}}%
\pgfpathlineto{\pgfqpoint{0.520542in}{1.277526in}}%
\pgfpathlineto{\pgfqpoint{0.524302in}{1.283464in}}%
\pgfpathlineto{\pgfqpoint{0.533766in}{1.269179in}}%
\pgfpathlineto{\pgfqpoint{0.540209in}{1.259221in}}%
\pgfpathlineto{\pgfqpoint{0.567833in}{1.216455in}}%
\pgfpathlineto{\pgfqpoint{0.590725in}{1.181138in}}%
\pgfpathlineto{\pgfqpoint{0.616671in}{1.141013in}}%
\pgfpathlineto{\pgfqpoint{0.667024in}{1.063268in}}%
\pgfpathlineto{\pgfqpoint{0.677176in}{1.047687in}}%
\pgfpathclose%
\pgfusepath{fill}%
\end{pgfscope}%
\begin{pgfscope}%
\pgfpathrectangle{\pgfqpoint{0.100000in}{0.100000in}}{\pgfqpoint{3.007045in}{1.925000in}}%
\pgfusepath{clip}%
\pgfsetbuttcap%
\pgfsetmiterjoin%
\definecolor{currentfill}{rgb}{0.260715,0.573841,0.777209}%
\pgfsetfillcolor{currentfill}%
\pgfsetlinewidth{0.000000pt}%
\definecolor{currentstroke}{rgb}{0.000000,0.000000,0.000000}%
\pgfsetstrokecolor{currentstroke}%
\pgfsetstrokeopacity{0.000000}%
\pgfsetdash{}{0pt}%
\pgfpathmoveto{\pgfqpoint{2.442052in}{0.679496in}}%
\pgfpathlineto{\pgfqpoint{2.428996in}{0.678445in}}%
\pgfpathlineto{\pgfqpoint{2.425943in}{0.682862in}}%
\pgfpathlineto{\pgfqpoint{2.421279in}{0.683375in}}%
\pgfpathlineto{\pgfqpoint{2.420802in}{0.691745in}}%
\pgfpathlineto{\pgfqpoint{2.422816in}{0.692785in}}%
\pgfpathlineto{\pgfqpoint{2.420123in}{0.701779in}}%
\pgfpathlineto{\pgfqpoint{2.437033in}{0.704008in}}%
\pgfpathlineto{\pgfqpoint{2.447789in}{0.700910in}}%
\pgfpathlineto{\pgfqpoint{2.448668in}{0.692985in}}%
\pgfpathlineto{\pgfqpoint{2.445730in}{0.689304in}}%
\pgfpathlineto{\pgfqpoint{2.443946in}{0.680145in}}%
\pgfpathlineto{\pgfqpoint{2.442052in}{0.679496in}}%
\pgfpathclose%
\pgfusepath{fill}%
\end{pgfscope}%
\begin{pgfscope}%
\pgfpathrectangle{\pgfqpoint{0.100000in}{0.100000in}}{\pgfqpoint{3.007045in}{1.925000in}}%
\pgfusepath{clip}%
\pgfsetbuttcap%
\pgfsetmiterjoin%
\definecolor{currentfill}{rgb}{0.622684,0.793464,0.883429}%
\pgfsetfillcolor{currentfill}%
\pgfsetlinewidth{0.000000pt}%
\definecolor{currentstroke}{rgb}{0.000000,0.000000,0.000000}%
\pgfsetstrokecolor{currentstroke}%
\pgfsetstrokeopacity{0.000000}%
\pgfsetdash{}{0pt}%
\pgfpathmoveto{\pgfqpoint{0.404299in}{1.600395in}}%
\pgfpathlineto{\pgfqpoint{0.415190in}{1.591253in}}%
\pgfpathlineto{\pgfqpoint{0.419385in}{1.591902in}}%
\pgfpathlineto{\pgfqpoint{0.426249in}{1.588654in}}%
\pgfpathlineto{\pgfqpoint{0.429770in}{1.590236in}}%
\pgfpathlineto{\pgfqpoint{0.436255in}{1.587889in}}%
\pgfpathlineto{\pgfqpoint{0.446705in}{1.586464in}}%
\pgfpathlineto{\pgfqpoint{0.460693in}{1.593122in}}%
\pgfpathlineto{\pgfqpoint{0.465330in}{1.591720in}}%
\pgfpathlineto{\pgfqpoint{0.469189in}{1.594432in}}%
\pgfpathlineto{\pgfqpoint{0.474662in}{1.592743in}}%
\pgfpathlineto{\pgfqpoint{0.455007in}{1.530523in}}%
\pgfpathlineto{\pgfqpoint{0.397486in}{1.548285in}}%
\pgfpathlineto{\pgfqpoint{0.383232in}{1.552564in}}%
\pgfpathlineto{\pgfqpoint{0.384587in}{1.562231in}}%
\pgfpathlineto{\pgfqpoint{0.389591in}{1.567203in}}%
\pgfpathlineto{\pgfqpoint{0.387396in}{1.575981in}}%
\pgfpathlineto{\pgfqpoint{0.381248in}{1.578474in}}%
\pgfpathlineto{\pgfqpoint{0.384929in}{1.586076in}}%
\pgfpathlineto{\pgfqpoint{0.390435in}{1.585257in}}%
\pgfpathlineto{\pgfqpoint{0.395865in}{1.592654in}}%
\pgfpathlineto{\pgfqpoint{0.404299in}{1.600395in}}%
\pgfpathclose%
\pgfusepath{fill}%
\end{pgfscope}%
\begin{pgfscope}%
\pgfpathrectangle{\pgfqpoint{0.100000in}{0.100000in}}{\pgfqpoint{3.007045in}{1.925000in}}%
\pgfusepath{clip}%
\pgfsetbuttcap%
\pgfsetmiterjoin%
\definecolor{currentfill}{rgb}{0.560784,0.759862,0.869666}%
\pgfsetfillcolor{currentfill}%
\pgfsetlinewidth{0.000000pt}%
\definecolor{currentstroke}{rgb}{0.000000,0.000000,0.000000}%
\pgfsetstrokecolor{currentstroke}%
\pgfsetstrokeopacity{0.000000}%
\pgfsetdash{}{0pt}%
\pgfpathmoveto{\pgfqpoint{2.295738in}{0.746225in}}%
\pgfpathlineto{\pgfqpoint{2.290204in}{0.743581in}}%
\pgfpathlineto{\pgfqpoint{2.290941in}{0.736895in}}%
\pgfpathlineto{\pgfqpoint{2.285327in}{0.735328in}}%
\pgfpathlineto{\pgfqpoint{2.285996in}{0.729613in}}%
\pgfpathlineto{\pgfqpoint{2.281053in}{0.729035in}}%
\pgfpathlineto{\pgfqpoint{2.279667in}{0.746313in}}%
\pgfpathlineto{\pgfqpoint{2.252087in}{0.743598in}}%
\pgfpathlineto{\pgfqpoint{2.247289in}{0.747109in}}%
\pgfpathlineto{\pgfqpoint{2.243973in}{0.753410in}}%
\pgfpathlineto{\pgfqpoint{2.242492in}{0.770671in}}%
\pgfpathlineto{\pgfqpoint{2.244576in}{0.775293in}}%
\pgfpathlineto{\pgfqpoint{2.249772in}{0.779385in}}%
\pgfpathlineto{\pgfqpoint{2.248013in}{0.785562in}}%
\pgfpathlineto{\pgfqpoint{2.257385in}{0.796619in}}%
\pgfpathlineto{\pgfqpoint{2.255427in}{0.799367in}}%
\pgfpathlineto{\pgfqpoint{2.258979in}{0.805773in}}%
\pgfpathlineto{\pgfqpoint{2.264190in}{0.805986in}}%
\pgfpathlineto{\pgfqpoint{2.267838in}{0.800489in}}%
\pgfpathlineto{\pgfqpoint{2.278832in}{0.801530in}}%
\pgfpathlineto{\pgfqpoint{2.281013in}{0.795971in}}%
\pgfpathlineto{\pgfqpoint{2.287725in}{0.796480in}}%
\pgfpathlineto{\pgfqpoint{2.289813in}{0.770791in}}%
\pgfpathlineto{\pgfqpoint{2.293144in}{0.771183in}}%
\pgfpathlineto{\pgfqpoint{2.295738in}{0.746225in}}%
\pgfpathclose%
\pgfusepath{fill}%
\end{pgfscope}%
\begin{pgfscope}%
\pgfpathrectangle{\pgfqpoint{0.100000in}{0.100000in}}{\pgfqpoint{3.007045in}{1.925000in}}%
\pgfusepath{clip}%
\pgfsetbuttcap%
\pgfsetmiterjoin%
\definecolor{currentfill}{rgb}{0.498039,0.725413,0.856132}%
\pgfsetfillcolor{currentfill}%
\pgfsetlinewidth{0.000000pt}%
\definecolor{currentstroke}{rgb}{0.000000,0.000000,0.000000}%
\pgfsetstrokecolor{currentstroke}%
\pgfsetstrokeopacity{0.000000}%
\pgfsetdash{}{0pt}%
\pgfpathmoveto{\pgfqpoint{2.067386in}{1.211625in}}%
\pgfpathlineto{\pgfqpoint{2.066190in}{1.226950in}}%
\pgfpathlineto{\pgfqpoint{2.060239in}{1.229418in}}%
\pgfpathlineto{\pgfqpoint{2.058847in}{1.247095in}}%
\pgfpathlineto{\pgfqpoint{2.065605in}{1.247690in}}%
\pgfpathlineto{\pgfqpoint{2.072749in}{1.252662in}}%
\pgfpathlineto{\pgfqpoint{2.075117in}{1.258653in}}%
\pgfpathlineto{\pgfqpoint{2.073487in}{1.281826in}}%
\pgfpathlineto{\pgfqpoint{2.090562in}{1.283138in}}%
\pgfpathlineto{\pgfqpoint{2.107101in}{1.284740in}}%
\pgfpathlineto{\pgfqpoint{2.107905in}{1.276923in}}%
\pgfpathlineto{\pgfqpoint{2.110362in}{1.252216in}}%
\pgfpathlineto{\pgfqpoint{2.099165in}{1.251343in}}%
\pgfpathlineto{\pgfqpoint{2.100744in}{1.229280in}}%
\pgfpathlineto{\pgfqpoint{2.094960in}{1.228863in}}%
\pgfpathlineto{\pgfqpoint{2.087435in}{1.213369in}}%
\pgfpathlineto{\pgfqpoint{2.067386in}{1.211625in}}%
\pgfpathclose%
\pgfusepath{fill}%
\end{pgfscope}%
\begin{pgfscope}%
\pgfpathrectangle{\pgfqpoint{0.100000in}{0.100000in}}{\pgfqpoint{3.007045in}{1.925000in}}%
\pgfusepath{clip}%
\pgfsetbuttcap%
\pgfsetmiterjoin%
\definecolor{currentfill}{rgb}{0.256286,0.570012,0.775163}%
\pgfsetfillcolor{currentfill}%
\pgfsetlinewidth{0.000000pt}%
\definecolor{currentstroke}{rgb}{0.000000,0.000000,0.000000}%
\pgfsetstrokecolor{currentstroke}%
\pgfsetstrokeopacity{0.000000}%
\pgfsetdash{}{0pt}%
\pgfpathmoveto{\pgfqpoint{1.952131in}{0.625210in}}%
\pgfpathlineto{\pgfqpoint{1.961233in}{0.620328in}}%
\pgfpathlineto{\pgfqpoint{1.958991in}{0.616692in}}%
\pgfpathlineto{\pgfqpoint{1.958935in}{0.610678in}}%
\pgfpathlineto{\pgfqpoint{1.954440in}{0.605869in}}%
\pgfpathlineto{\pgfqpoint{1.954185in}{0.597454in}}%
\pgfpathlineto{\pgfqpoint{1.930070in}{0.596671in}}%
\pgfpathlineto{\pgfqpoint{1.927440in}{0.602431in}}%
\pgfpathlineto{\pgfqpoint{1.921122in}{0.606604in}}%
\pgfpathlineto{\pgfqpoint{1.915657in}{0.600227in}}%
\pgfpathlineto{\pgfqpoint{1.909431in}{0.599271in}}%
\pgfpathlineto{\pgfqpoint{1.896550in}{0.598540in}}%
\pgfpathlineto{\pgfqpoint{1.895632in}{0.622616in}}%
\pgfpathlineto{\pgfqpoint{1.892685in}{0.626874in}}%
\pgfpathlineto{\pgfqpoint{1.886870in}{0.628687in}}%
\pgfpathlineto{\pgfqpoint{1.890720in}{0.630209in}}%
\pgfpathlineto{\pgfqpoint{1.901476in}{0.640393in}}%
\pgfpathlineto{\pgfqpoint{1.896682in}{0.645552in}}%
\pgfpathlineto{\pgfqpoint{1.890709in}{0.647179in}}%
\pgfpathlineto{\pgfqpoint{1.886783in}{0.652756in}}%
\pgfpathlineto{\pgfqpoint{1.906746in}{0.653283in}}%
\pgfpathlineto{\pgfqpoint{1.906637in}{0.659096in}}%
\pgfpathlineto{\pgfqpoint{1.921110in}{0.659462in}}%
\pgfpathlineto{\pgfqpoint{1.922010in}{0.647995in}}%
\pgfpathlineto{\pgfqpoint{1.926639in}{0.641691in}}%
\pgfpathlineto{\pgfqpoint{1.933198in}{0.637846in}}%
\pgfpathlineto{\pgfqpoint{1.933963in}{0.633867in}}%
\pgfpathlineto{\pgfqpoint{1.942513in}{0.628295in}}%
\pgfpathlineto{\pgfqpoint{1.952131in}{0.625210in}}%
\pgfpathclose%
\pgfusepath{fill}%
\end{pgfscope}%
\begin{pgfscope}%
\pgfpathrectangle{\pgfqpoint{0.100000in}{0.100000in}}{\pgfqpoint{3.007045in}{1.925000in}}%
\pgfusepath{clip}%
\pgfsetbuttcap%
\pgfsetmiterjoin%
\definecolor{currentfill}{rgb}{0.429020,0.687520,0.841246}%
\pgfsetfillcolor{currentfill}%
\pgfsetlinewidth{0.000000pt}%
\definecolor{currentstroke}{rgb}{0.000000,0.000000,0.000000}%
\pgfsetstrokecolor{currentstroke}%
\pgfsetstrokeopacity{0.000000}%
\pgfsetdash{}{0pt}%
\pgfpathmoveto{\pgfqpoint{2.131121in}{1.073626in}}%
\pgfpathlineto{\pgfqpoint{2.128732in}{1.074544in}}%
\pgfpathlineto{\pgfqpoint{2.116398in}{1.073683in}}%
\pgfpathlineto{\pgfqpoint{2.114940in}{1.096476in}}%
\pgfpathlineto{\pgfqpoint{2.097811in}{1.095356in}}%
\pgfpathlineto{\pgfqpoint{2.096418in}{1.118342in}}%
\pgfpathlineto{\pgfqpoint{2.119277in}{1.119542in}}%
\pgfpathlineto{\pgfqpoint{2.118010in}{1.123238in}}%
\pgfpathlineto{\pgfqpoint{2.117823in}{1.135815in}}%
\pgfpathlineto{\pgfqpoint{2.135746in}{1.137366in}}%
\pgfpathlineto{\pgfqpoint{2.136969in}{1.118978in}}%
\pgfpathlineto{\pgfqpoint{2.135045in}{1.113976in}}%
\pgfpathlineto{\pgfqpoint{2.134603in}{1.106062in}}%
\pgfpathlineto{\pgfqpoint{2.136503in}{1.101602in}}%
\pgfpathlineto{\pgfqpoint{2.134514in}{1.098251in}}%
\pgfpathlineto{\pgfqpoint{2.137521in}{1.089690in}}%
\pgfpathlineto{\pgfqpoint{2.131121in}{1.073626in}}%
\pgfpathclose%
\pgfusepath{fill}%
\end{pgfscope}%
\begin{pgfscope}%
\pgfpathrectangle{\pgfqpoint{0.100000in}{0.100000in}}{\pgfqpoint{3.007045in}{1.925000in}}%
\pgfusepath{clip}%
\pgfsetbuttcap%
\pgfsetmiterjoin%
\definecolor{currentfill}{rgb}{0.642368,0.801830,0.890319}%
\pgfsetfillcolor{currentfill}%
\pgfsetlinewidth{0.000000pt}%
\definecolor{currentstroke}{rgb}{0.000000,0.000000,0.000000}%
\pgfsetstrokecolor{currentstroke}%
\pgfsetstrokeopacity{0.000000}%
\pgfsetdash{}{0pt}%
\pgfpathmoveto{\pgfqpoint{2.450037in}{1.161421in}}%
\pgfpathlineto{\pgfqpoint{2.440950in}{1.158535in}}%
\pgfpathlineto{\pgfqpoint{2.435873in}{1.154519in}}%
\pgfpathlineto{\pgfqpoint{2.437602in}{1.145887in}}%
\pgfpathlineto{\pgfqpoint{2.428157in}{1.152335in}}%
\pgfpathlineto{\pgfqpoint{2.420941in}{1.150887in}}%
\pgfpathlineto{\pgfqpoint{2.420478in}{1.156867in}}%
\pgfpathlineto{\pgfqpoint{2.414411in}{1.156435in}}%
\pgfpathlineto{\pgfqpoint{2.413032in}{1.162228in}}%
\pgfpathlineto{\pgfqpoint{2.407930in}{1.161976in}}%
\pgfpathlineto{\pgfqpoint{2.407704in}{1.166293in}}%
\pgfpathlineto{\pgfqpoint{2.413821in}{1.167145in}}%
\pgfpathlineto{\pgfqpoint{2.413102in}{1.179720in}}%
\pgfpathlineto{\pgfqpoint{2.418866in}{1.180027in}}%
\pgfpathlineto{\pgfqpoint{2.418057in}{1.191742in}}%
\pgfpathlineto{\pgfqpoint{2.419602in}{1.197771in}}%
\pgfpathlineto{\pgfqpoint{2.417290in}{1.203518in}}%
\pgfpathlineto{\pgfqpoint{2.416942in}{1.209349in}}%
\pgfpathlineto{\pgfqpoint{2.422680in}{1.209735in}}%
\pgfpathlineto{\pgfqpoint{2.422257in}{1.215672in}}%
\pgfpathlineto{\pgfqpoint{2.427935in}{1.216104in}}%
\pgfpathlineto{\pgfqpoint{2.428341in}{1.210131in}}%
\pgfpathlineto{\pgfqpoint{2.439065in}{1.211016in}}%
\pgfpathlineto{\pgfqpoint{2.439887in}{1.202342in}}%
\pgfpathlineto{\pgfqpoint{2.446995in}{1.200159in}}%
\pgfpathlineto{\pgfqpoint{2.446120in}{1.194088in}}%
\pgfpathlineto{\pgfqpoint{2.447277in}{1.183280in}}%
\pgfpathlineto{\pgfqpoint{2.441696in}{1.172351in}}%
\pgfpathlineto{\pgfqpoint{2.450037in}{1.161421in}}%
\pgfpathclose%
\pgfusepath{fill}%
\end{pgfscope}%
\begin{pgfscope}%
\pgfpathrectangle{\pgfqpoint{0.100000in}{0.100000in}}{\pgfqpoint{3.007045in}{1.925000in}}%
\pgfusepath{clip}%
\pgfsetbuttcap%
\pgfsetmiterjoin%
\definecolor{currentfill}{rgb}{0.361599,0.642737,0.816578}%
\pgfsetfillcolor{currentfill}%
\pgfsetlinewidth{0.000000pt}%
\definecolor{currentstroke}{rgb}{0.000000,0.000000,0.000000}%
\pgfsetstrokecolor{currentstroke}%
\pgfsetstrokeopacity{0.000000}%
\pgfsetdash{}{0pt}%
\pgfpathmoveto{\pgfqpoint{1.594392in}{1.173857in}}%
\pgfpathlineto{\pgfqpoint{1.623649in}{1.172778in}}%
\pgfpathlineto{\pgfqpoint{1.623055in}{1.155630in}}%
\pgfpathlineto{\pgfqpoint{1.622833in}{1.149886in}}%
\pgfpathlineto{\pgfqpoint{1.594289in}{1.150925in}}%
\pgfpathlineto{\pgfqpoint{1.594074in}{1.145204in}}%
\pgfpathlineto{\pgfqpoint{1.565600in}{1.146405in}}%
\pgfpathlineto{\pgfqpoint{1.567058in}{1.175010in}}%
\pgfpathlineto{\pgfqpoint{1.594392in}{1.173857in}}%
\pgfpathclose%
\pgfusepath{fill}%
\end{pgfscope}%
\begin{pgfscope}%
\pgfpathrectangle{\pgfqpoint{0.100000in}{0.100000in}}{\pgfqpoint{3.007045in}{1.925000in}}%
\pgfusepath{clip}%
\pgfsetbuttcap%
\pgfsetmiterjoin%
\definecolor{currentfill}{rgb}{0.491765,0.721968,0.854779}%
\pgfsetfillcolor{currentfill}%
\pgfsetlinewidth{0.000000pt}%
\definecolor{currentstroke}{rgb}{0.000000,0.000000,0.000000}%
\pgfsetstrokecolor{currentstroke}%
\pgfsetstrokeopacity{0.000000}%
\pgfsetdash{}{0pt}%
\pgfpathmoveto{\pgfqpoint{2.695089in}{1.071478in}}%
\pgfpathlineto{\pgfqpoint{2.691815in}{1.071736in}}%
\pgfpathlineto{\pgfqpoint{2.691645in}{1.081236in}}%
\pgfpathlineto{\pgfqpoint{2.685692in}{1.079312in}}%
\pgfpathlineto{\pgfqpoint{2.680930in}{1.081273in}}%
\pgfpathlineto{\pgfqpoint{2.674370in}{1.085389in}}%
\pgfpathlineto{\pgfqpoint{2.667768in}{1.085077in}}%
\pgfpathlineto{\pgfqpoint{2.665311in}{1.095015in}}%
\pgfpathlineto{\pgfqpoint{2.670092in}{1.099194in}}%
\pgfpathlineto{\pgfqpoint{2.646388in}{1.101662in}}%
\pgfpathlineto{\pgfqpoint{2.645322in}{1.106286in}}%
\pgfpathlineto{\pgfqpoint{2.649572in}{1.112889in}}%
\pgfpathlineto{\pgfqpoint{2.649818in}{1.126222in}}%
\pgfpathlineto{\pgfqpoint{2.644603in}{1.131634in}}%
\pgfpathlineto{\pgfqpoint{2.647609in}{1.142718in}}%
\pgfpathlineto{\pgfqpoint{2.654207in}{1.139998in}}%
\pgfpathlineto{\pgfqpoint{2.658319in}{1.135059in}}%
\pgfpathlineto{\pgfqpoint{2.663369in}{1.133914in}}%
\pgfpathlineto{\pgfqpoint{2.665522in}{1.153025in}}%
\pgfpathlineto{\pgfqpoint{2.679314in}{1.148103in}}%
\pgfpathlineto{\pgfqpoint{2.681074in}{1.143862in}}%
\pgfpathlineto{\pgfqpoint{2.685472in}{1.142255in}}%
\pgfpathlineto{\pgfqpoint{2.692040in}{1.154397in}}%
\pgfpathlineto{\pgfqpoint{2.697515in}{1.156406in}}%
\pgfpathlineto{\pgfqpoint{2.701679in}{1.149182in}}%
\pgfpathlineto{\pgfqpoint{2.706818in}{1.146620in}}%
\pgfpathlineto{\pgfqpoint{2.713398in}{1.148167in}}%
\pgfpathlineto{\pgfqpoint{2.716860in}{1.144435in}}%
\pgfpathlineto{\pgfqpoint{2.723470in}{1.137350in}}%
\pgfpathlineto{\pgfqpoint{2.724698in}{1.127666in}}%
\pgfpathlineto{\pgfqpoint{2.713382in}{1.123521in}}%
\pgfpathlineto{\pgfqpoint{2.716713in}{1.110416in}}%
\pgfpathlineto{\pgfqpoint{2.711789in}{1.111134in}}%
\pgfpathlineto{\pgfqpoint{2.704080in}{1.100551in}}%
\pgfpathlineto{\pgfqpoint{2.713742in}{1.098160in}}%
\pgfpathlineto{\pgfqpoint{2.716714in}{1.091684in}}%
\pgfpathlineto{\pgfqpoint{2.695089in}{1.071478in}}%
\pgfpathclose%
\pgfusepath{fill}%
\end{pgfscope}%
\begin{pgfscope}%
\pgfpathrectangle{\pgfqpoint{0.100000in}{0.100000in}}{\pgfqpoint{3.007045in}{1.925000in}}%
\pgfusepath{clip}%
\pgfsetbuttcap%
\pgfsetmiterjoin%
\definecolor{currentfill}{rgb}{0.290980,0.594510,0.789020}%
\pgfsetfillcolor{currentfill}%
\pgfsetlinewidth{0.000000pt}%
\definecolor{currentstroke}{rgb}{0.000000,0.000000,0.000000}%
\pgfsetstrokecolor{currentstroke}%
\pgfsetstrokeopacity{0.000000}%
\pgfsetdash{}{0pt}%
\pgfpathmoveto{\pgfqpoint{1.681563in}{0.619854in}}%
\pgfpathlineto{\pgfqpoint{1.668209in}{0.612458in}}%
\pgfpathlineto{\pgfqpoint{1.654308in}{0.604797in}}%
\pgfpathlineto{\pgfqpoint{1.642936in}{0.624588in}}%
\pgfpathlineto{\pgfqpoint{1.639177in}{0.622324in}}%
\pgfpathlineto{\pgfqpoint{1.635041in}{0.627484in}}%
\pgfpathlineto{\pgfqpoint{1.620418in}{0.653867in}}%
\pgfpathlineto{\pgfqpoint{1.616016in}{0.651438in}}%
\pgfpathlineto{\pgfqpoint{1.603281in}{0.674775in}}%
\pgfpathlineto{\pgfqpoint{1.611326in}{0.679134in}}%
\pgfpathlineto{\pgfqpoint{1.624705in}{0.686669in}}%
\pgfpathlineto{\pgfqpoint{1.628904in}{0.682713in}}%
\pgfpathlineto{\pgfqpoint{1.633343in}{0.684748in}}%
\pgfpathlineto{\pgfqpoint{1.655346in}{0.689754in}}%
\pgfpathlineto{\pgfqpoint{1.663204in}{0.675422in}}%
\pgfpathlineto{\pgfqpoint{1.665733in}{0.676821in}}%
\pgfpathlineto{\pgfqpoint{1.675361in}{0.659431in}}%
\pgfpathlineto{\pgfqpoint{1.663193in}{0.652637in}}%
\pgfpathlineto{\pgfqpoint{1.681563in}{0.619854in}}%
\pgfpathclose%
\pgfusepath{fill}%
\end{pgfscope}%
\begin{pgfscope}%
\pgfpathrectangle{\pgfqpoint{0.100000in}{0.100000in}}{\pgfqpoint{3.007045in}{1.925000in}}%
\pgfusepath{clip}%
\pgfsetbuttcap%
\pgfsetmiterjoin%
\definecolor{currentfill}{rgb}{0.676817,0.816471,0.902376}%
\pgfsetfillcolor{currentfill}%
\pgfsetlinewidth{0.000000pt}%
\definecolor{currentstroke}{rgb}{0.000000,0.000000,0.000000}%
\pgfsetstrokecolor{currentstroke}%
\pgfsetstrokeopacity{0.000000}%
\pgfsetdash{}{0pt}%
\pgfpathmoveto{\pgfqpoint{2.498857in}{1.196503in}}%
\pgfpathlineto{\pgfqpoint{2.494442in}{1.196721in}}%
\pgfpathlineto{\pgfqpoint{2.487686in}{1.206566in}}%
\pgfpathlineto{\pgfqpoint{2.488733in}{1.208153in}}%
\pgfpathlineto{\pgfqpoint{2.500713in}{1.220454in}}%
\pgfpathlineto{\pgfqpoint{2.505896in}{1.220258in}}%
\pgfpathlineto{\pgfqpoint{2.507751in}{1.225110in}}%
\pgfpathlineto{\pgfqpoint{2.505584in}{1.228741in}}%
\pgfpathlineto{\pgfqpoint{2.510163in}{1.234931in}}%
\pgfpathlineto{\pgfqpoint{2.507767in}{1.239655in}}%
\pgfpathlineto{\pgfqpoint{2.554976in}{1.247539in}}%
\pgfpathlineto{\pgfqpoint{2.560339in}{1.213292in}}%
\pgfpathlineto{\pgfqpoint{2.549426in}{1.216536in}}%
\pgfpathlineto{\pgfqpoint{2.543719in}{1.212838in}}%
\pgfpathlineto{\pgfqpoint{2.538153in}{1.216728in}}%
\pgfpathlineto{\pgfqpoint{2.532780in}{1.212166in}}%
\pgfpathlineto{\pgfqpoint{2.525644in}{1.210611in}}%
\pgfpathlineto{\pgfqpoint{2.523965in}{1.201691in}}%
\pgfpathlineto{\pgfqpoint{2.520381in}{1.200511in}}%
\pgfpathlineto{\pgfqpoint{2.505727in}{1.202402in}}%
\pgfpathlineto{\pgfqpoint{2.498857in}{1.196503in}}%
\pgfpathclose%
\pgfusepath{fill}%
\end{pgfscope}%
\begin{pgfscope}%
\pgfpathrectangle{\pgfqpoint{0.100000in}{0.100000in}}{\pgfqpoint{3.007045in}{1.925000in}}%
\pgfusepath{clip}%
\pgfsetbuttcap%
\pgfsetmiterjoin%
\definecolor{currentfill}{rgb}{0.598431,0.780531,0.877785}%
\pgfsetfillcolor{currentfill}%
\pgfsetlinewidth{0.000000pt}%
\definecolor{currentstroke}{rgb}{0.000000,0.000000,0.000000}%
\pgfsetstrokecolor{currentstroke}%
\pgfsetstrokeopacity{0.000000}%
\pgfsetdash{}{0pt}%
\pgfpathmoveto{\pgfqpoint{2.677192in}{1.366767in}}%
\pgfpathlineto{\pgfqpoint{2.679466in}{1.361047in}}%
\pgfpathlineto{\pgfqpoint{2.686221in}{1.354519in}}%
\pgfpathlineto{\pgfqpoint{2.688600in}{1.346784in}}%
\pgfpathlineto{\pgfqpoint{2.692918in}{1.348382in}}%
\pgfpathlineto{\pgfqpoint{2.694956in}{1.344848in}}%
\pgfpathlineto{\pgfqpoint{2.680799in}{1.334029in}}%
\pgfpathlineto{\pgfqpoint{2.670437in}{1.328641in}}%
\pgfpathlineto{\pgfqpoint{2.664036in}{1.332084in}}%
\pgfpathlineto{\pgfqpoint{2.648248in}{1.330436in}}%
\pgfpathlineto{\pgfqpoint{2.645950in}{1.339764in}}%
\pgfpathlineto{\pgfqpoint{2.654171in}{1.354581in}}%
\pgfpathlineto{\pgfqpoint{2.661650in}{1.359044in}}%
\pgfpathlineto{\pgfqpoint{2.661755in}{1.363168in}}%
\pgfpathlineto{\pgfqpoint{2.669738in}{1.366595in}}%
\pgfpathlineto{\pgfqpoint{2.677192in}{1.366767in}}%
\pgfpathclose%
\pgfusepath{fill}%
\end{pgfscope}%
\begin{pgfscope}%
\pgfpathrectangle{\pgfqpoint{0.100000in}{0.100000in}}{\pgfqpoint{3.007045in}{1.925000in}}%
\pgfusepath{clip}%
\pgfsetbuttcap%
\pgfsetmiterjoin%
\definecolor{currentfill}{rgb}{0.567059,0.763306,0.871019}%
\pgfsetfillcolor{currentfill}%
\pgfsetlinewidth{0.000000pt}%
\definecolor{currentstroke}{rgb}{0.000000,0.000000,0.000000}%
\pgfsetstrokecolor{currentstroke}%
\pgfsetstrokeopacity{0.000000}%
\pgfsetdash{}{0pt}%
\pgfpathmoveto{\pgfqpoint{2.141224in}{0.664018in}}%
\pgfpathlineto{\pgfqpoint{2.139795in}{0.676959in}}%
\pgfpathlineto{\pgfqpoint{2.141241in}{0.722568in}}%
\pgfpathlineto{\pgfqpoint{2.141820in}{0.745927in}}%
\pgfpathlineto{\pgfqpoint{2.141932in}{0.750061in}}%
\pgfpathlineto{\pgfqpoint{2.151248in}{0.751063in}}%
\pgfpathlineto{\pgfqpoint{2.154101in}{0.754354in}}%
\pgfpathlineto{\pgfqpoint{2.158791in}{0.754738in}}%
\pgfpathlineto{\pgfqpoint{2.163129in}{0.759429in}}%
\pgfpathlineto{\pgfqpoint{2.168941in}{0.762907in}}%
\pgfpathlineto{\pgfqpoint{2.169944in}{0.753927in}}%
\pgfpathlineto{\pgfqpoint{2.176652in}{0.754493in}}%
\pgfpathlineto{\pgfqpoint{2.175967in}{0.749888in}}%
\pgfpathlineto{\pgfqpoint{2.172078in}{0.746133in}}%
\pgfpathlineto{\pgfqpoint{2.171759in}{0.737367in}}%
\pgfpathlineto{\pgfqpoint{2.174689in}{0.726922in}}%
\pgfpathlineto{\pgfqpoint{2.173613in}{0.721470in}}%
\pgfpathlineto{\pgfqpoint{2.178513in}{0.718958in}}%
\pgfpathlineto{\pgfqpoint{2.189892in}{0.719957in}}%
\pgfpathlineto{\pgfqpoint{2.190855in}{0.708411in}}%
\pgfpathlineto{\pgfqpoint{2.193997in}{0.705851in}}%
\pgfpathlineto{\pgfqpoint{2.191190in}{0.704948in}}%
\pgfpathlineto{\pgfqpoint{2.191937in}{0.696888in}}%
\pgfpathlineto{\pgfqpoint{2.186226in}{0.696389in}}%
\pgfpathlineto{\pgfqpoint{2.186978in}{0.688005in}}%
\pgfpathlineto{\pgfqpoint{2.184529in}{0.686866in}}%
\pgfpathlineto{\pgfqpoint{2.161771in}{0.684999in}}%
\pgfpathlineto{\pgfqpoint{2.156603in}{0.673169in}}%
\pgfpathlineto{\pgfqpoint{2.161161in}{0.669683in}}%
\pgfpathlineto{\pgfqpoint{2.162408in}{0.665683in}}%
\pgfpathlineto{\pgfqpoint{2.141224in}{0.664018in}}%
\pgfpathclose%
\pgfusepath{fill}%
\end{pgfscope}%
\begin{pgfscope}%
\pgfpathrectangle{\pgfqpoint{0.100000in}{0.100000in}}{\pgfqpoint{3.007045in}{1.925000in}}%
\pgfusepath{clip}%
\pgfsetbuttcap%
\pgfsetmiterjoin%
\definecolor{currentfill}{rgb}{0.412042,0.677186,0.836263}%
\pgfsetfillcolor{currentfill}%
\pgfsetlinewidth{0.000000pt}%
\definecolor{currentstroke}{rgb}{0.000000,0.000000,0.000000}%
\pgfsetstrokecolor{currentstroke}%
\pgfsetstrokeopacity{0.000000}%
\pgfsetdash{}{0pt}%
\pgfpathmoveto{\pgfqpoint{1.984723in}{0.815682in}}%
\pgfpathlineto{\pgfqpoint{1.980611in}{0.811051in}}%
\pgfpathlineto{\pgfqpoint{1.984377in}{0.804898in}}%
\pgfpathlineto{\pgfqpoint{1.974688in}{0.804739in}}%
\pgfpathlineto{\pgfqpoint{1.967563in}{0.808897in}}%
\pgfpathlineto{\pgfqpoint{1.966388in}{0.815925in}}%
\pgfpathlineto{\pgfqpoint{1.962924in}{0.816378in}}%
\pgfpathlineto{\pgfqpoint{1.961337in}{0.821725in}}%
\pgfpathlineto{\pgfqpoint{1.951971in}{0.821985in}}%
\pgfpathlineto{\pgfqpoint{1.951736in}{0.838426in}}%
\pgfpathlineto{\pgfqpoint{1.955232in}{0.842505in}}%
\pgfpathlineto{\pgfqpoint{1.955013in}{0.847952in}}%
\pgfpathlineto{\pgfqpoint{1.951519in}{0.850943in}}%
\pgfpathlineto{\pgfqpoint{1.951312in}{0.867300in}}%
\pgfpathlineto{\pgfqpoint{1.945345in}{0.867212in}}%
\pgfpathlineto{\pgfqpoint{1.945167in}{0.874395in}}%
\pgfpathlineto{\pgfqpoint{1.950043in}{0.876695in}}%
\pgfpathlineto{\pgfqpoint{1.956926in}{0.874466in}}%
\pgfpathlineto{\pgfqpoint{1.956817in}{0.878848in}}%
\pgfpathlineto{\pgfqpoint{1.963244in}{0.878965in}}%
\pgfpathlineto{\pgfqpoint{1.968811in}{0.876703in}}%
\pgfpathlineto{\pgfqpoint{1.972810in}{0.872450in}}%
\pgfpathlineto{\pgfqpoint{1.980505in}{0.873918in}}%
\pgfpathlineto{\pgfqpoint{1.980621in}{0.867935in}}%
\pgfpathlineto{\pgfqpoint{1.983593in}{0.865113in}}%
\pgfpathlineto{\pgfqpoint{1.983824in}{0.856375in}}%
\pgfpathlineto{\pgfqpoint{1.986708in}{0.856447in}}%
\pgfpathlineto{\pgfqpoint{1.987166in}{0.839058in}}%
\pgfpathlineto{\pgfqpoint{1.990126in}{0.839097in}}%
\pgfpathlineto{\pgfqpoint{1.990401in}{0.830295in}}%
\pgfpathlineto{\pgfqpoint{1.987469in}{0.830190in}}%
\pgfpathlineto{\pgfqpoint{1.986497in}{0.818543in}}%
\pgfpathlineto{\pgfqpoint{1.984723in}{0.815682in}}%
\pgfpathclose%
\pgfusepath{fill}%
\end{pgfscope}%
\begin{pgfscope}%
\pgfpathrectangle{\pgfqpoint{0.100000in}{0.100000in}}{\pgfqpoint{3.007045in}{1.925000in}}%
\pgfusepath{clip}%
\pgfsetbuttcap%
\pgfsetmiterjoin%
\definecolor{currentfill}{rgb}{0.401953,0.670296,0.832326}%
\pgfsetfillcolor{currentfill}%
\pgfsetlinewidth{0.000000pt}%
\definecolor{currentstroke}{rgb}{0.000000,0.000000,0.000000}%
\pgfsetstrokecolor{currentstroke}%
\pgfsetstrokeopacity{0.000000}%
\pgfsetdash{}{0pt}%
\pgfpathmoveto{\pgfqpoint{1.408374in}{1.098989in}}%
\pgfpathlineto{\pgfqpoint{1.372398in}{1.101802in}}%
\pgfpathlineto{\pgfqpoint{1.333267in}{1.105178in}}%
\pgfpathlineto{\pgfqpoint{1.334664in}{1.121680in}}%
\pgfpathlineto{\pgfqpoint{1.351745in}{1.120688in}}%
\pgfpathlineto{\pgfqpoint{1.353166in}{1.131470in}}%
\pgfpathlineto{\pgfqpoint{1.355807in}{1.160158in}}%
\pgfpathlineto{\pgfqpoint{1.358836in}{1.188865in}}%
\pgfpathlineto{\pgfqpoint{1.414767in}{1.184726in}}%
\pgfpathlineto{\pgfqpoint{1.414739in}{1.184340in}}%
\pgfpathlineto{\pgfqpoint{1.410528in}{1.127149in}}%
\pgfpathlineto{\pgfqpoint{1.408374in}{1.098989in}}%
\pgfpathclose%
\pgfusepath{fill}%
\end{pgfscope}%
\begin{pgfscope}%
\pgfpathrectangle{\pgfqpoint{0.100000in}{0.100000in}}{\pgfqpoint{3.007045in}{1.925000in}}%
\pgfusepath{clip}%
\pgfsetbuttcap%
\pgfsetmiterjoin%
\definecolor{currentfill}{rgb}{0.248166,0.561892,0.770980}%
\pgfsetfillcolor{currentfill}%
\pgfsetlinewidth{0.000000pt}%
\definecolor{currentstroke}{rgb}{0.000000,0.000000,0.000000}%
\pgfsetstrokecolor{currentstroke}%
\pgfsetstrokeopacity{0.000000}%
\pgfsetdash{}{0pt}%
\pgfpathmoveto{\pgfqpoint{2.013407in}{0.843839in}}%
\pgfpathlineto{\pgfqpoint{2.013158in}{0.836265in}}%
\pgfpathlineto{\pgfqpoint{2.003733in}{0.832951in}}%
\pgfpathlineto{\pgfqpoint{2.003512in}{0.827202in}}%
\pgfpathlineto{\pgfqpoint{1.993574in}{0.816135in}}%
\pgfpathlineto{\pgfqpoint{1.993621in}{0.816052in}}%
\pgfpathlineto{\pgfqpoint{1.984723in}{0.815682in}}%
\pgfpathlineto{\pgfqpoint{1.986497in}{0.818543in}}%
\pgfpathlineto{\pgfqpoint{1.987469in}{0.830190in}}%
\pgfpathlineto{\pgfqpoint{1.990401in}{0.830295in}}%
\pgfpathlineto{\pgfqpoint{1.990126in}{0.839097in}}%
\pgfpathlineto{\pgfqpoint{1.987166in}{0.839058in}}%
\pgfpathlineto{\pgfqpoint{1.986708in}{0.856447in}}%
\pgfpathlineto{\pgfqpoint{1.983824in}{0.856375in}}%
\pgfpathlineto{\pgfqpoint{1.983593in}{0.865113in}}%
\pgfpathlineto{\pgfqpoint{1.980621in}{0.867935in}}%
\pgfpathlineto{\pgfqpoint{1.980505in}{0.873918in}}%
\pgfpathlineto{\pgfqpoint{1.986161in}{0.874058in}}%
\pgfpathlineto{\pgfqpoint{1.985207in}{0.903228in}}%
\pgfpathlineto{\pgfqpoint{2.008210in}{0.904668in}}%
\pgfpathlineto{\pgfqpoint{2.025623in}{0.904867in}}%
\pgfpathlineto{\pgfqpoint{2.033514in}{0.905154in}}%
\pgfpathlineto{\pgfqpoint{2.037206in}{0.901885in}}%
\pgfpathlineto{\pgfqpoint{2.034157in}{0.893004in}}%
\pgfpathlineto{\pgfqpoint{2.037839in}{0.890956in}}%
\pgfpathlineto{\pgfqpoint{2.036782in}{0.884180in}}%
\pgfpathlineto{\pgfqpoint{2.032840in}{0.884060in}}%
\pgfpathlineto{\pgfqpoint{2.030962in}{0.877859in}}%
\pgfpathlineto{\pgfqpoint{2.026701in}{0.874367in}}%
\pgfpathlineto{\pgfqpoint{2.029442in}{0.869945in}}%
\pgfpathlineto{\pgfqpoint{2.020371in}{0.864609in}}%
\pgfpathlineto{\pgfqpoint{2.018227in}{0.853487in}}%
\pgfpathlineto{\pgfqpoint{2.012143in}{0.849788in}}%
\pgfpathlineto{\pgfqpoint{2.013407in}{0.843839in}}%
\pgfpathclose%
\pgfusepath{fill}%
\end{pgfscope}%
\begin{pgfscope}%
\pgfpathrectangle{\pgfqpoint{0.100000in}{0.100000in}}{\pgfqpoint{3.007045in}{1.925000in}}%
\pgfusepath{clip}%
\pgfsetbuttcap%
\pgfsetmiterjoin%
\definecolor{currentfill}{rgb}{0.376732,0.653072,0.822484}%
\pgfsetfillcolor{currentfill}%
\pgfsetlinewidth{0.000000pt}%
\definecolor{currentstroke}{rgb}{0.000000,0.000000,0.000000}%
\pgfsetstrokecolor{currentstroke}%
\pgfsetstrokeopacity{0.000000}%
\pgfsetdash{}{0pt}%
\pgfpathmoveto{\pgfqpoint{2.283647in}{1.145438in}}%
\pgfpathlineto{\pgfqpoint{2.275093in}{1.139834in}}%
\pgfpathlineto{\pgfqpoint{2.273431in}{1.154396in}}%
\pgfpathlineto{\pgfqpoint{2.261118in}{1.153025in}}%
\pgfpathlineto{\pgfqpoint{2.259410in}{1.171566in}}%
\pgfpathlineto{\pgfqpoint{2.253429in}{1.166775in}}%
\pgfpathlineto{\pgfqpoint{2.247318in}{1.166002in}}%
\pgfpathlineto{\pgfqpoint{2.245949in}{1.180376in}}%
\pgfpathlineto{\pgfqpoint{2.248039in}{1.187423in}}%
\pgfpathlineto{\pgfqpoint{2.264834in}{1.189201in}}%
\pgfpathlineto{\pgfqpoint{2.264298in}{1.193927in}}%
\pgfpathlineto{\pgfqpoint{2.288706in}{1.196382in}}%
\pgfpathlineto{\pgfqpoint{2.291530in}{1.169095in}}%
\pgfpathlineto{\pgfqpoint{2.288006in}{1.165183in}}%
\pgfpathlineto{\pgfqpoint{2.291496in}{1.161362in}}%
\pgfpathlineto{\pgfqpoint{2.289964in}{1.156159in}}%
\pgfpathlineto{\pgfqpoint{2.294975in}{1.154537in}}%
\pgfpathlineto{\pgfqpoint{2.290373in}{1.148367in}}%
\pgfpathlineto{\pgfqpoint{2.283647in}{1.145438in}}%
\pgfpathclose%
\pgfusepath{fill}%
\end{pgfscope}%
\begin{pgfscope}%
\pgfpathrectangle{\pgfqpoint{0.100000in}{0.100000in}}{\pgfqpoint{3.007045in}{1.925000in}}%
\pgfusepath{clip}%
\pgfsetbuttcap%
\pgfsetmiterjoin%
\definecolor{currentfill}{rgb}{0.435294,0.690965,0.842599}%
\pgfsetfillcolor{currentfill}%
\pgfsetlinewidth{0.000000pt}%
\definecolor{currentstroke}{rgb}{0.000000,0.000000,0.000000}%
\pgfsetstrokecolor{currentstroke}%
\pgfsetstrokeopacity{0.000000}%
\pgfsetdash{}{0pt}%
\pgfpathmoveto{\pgfqpoint{1.440696in}{0.803294in}}%
\pgfpathlineto{\pgfqpoint{1.469347in}{0.801604in}}%
\pgfpathlineto{\pgfqpoint{1.495136in}{0.800124in}}%
\pgfpathlineto{\pgfqpoint{1.514372in}{0.798412in}}%
\pgfpathlineto{\pgfqpoint{1.525327in}{0.794045in}}%
\pgfpathlineto{\pgfqpoint{1.526290in}{0.791703in}}%
\pgfpathlineto{\pgfqpoint{1.525314in}{0.769593in}}%
\pgfpathlineto{\pgfqpoint{1.523941in}{0.740360in}}%
\pgfpathlineto{\pgfqpoint{1.486605in}{0.742533in}}%
\pgfpathlineto{\pgfqpoint{1.465814in}{0.743950in}}%
\pgfpathlineto{\pgfqpoint{1.467658in}{0.772689in}}%
\pgfpathlineto{\pgfqpoint{1.438908in}{0.774476in}}%
\pgfpathlineto{\pgfqpoint{1.440696in}{0.803294in}}%
\pgfpathclose%
\pgfusepath{fill}%
\end{pgfscope}%
\begin{pgfscope}%
\pgfpathrectangle{\pgfqpoint{0.100000in}{0.100000in}}{\pgfqpoint{3.007045in}{1.925000in}}%
\pgfusepath{clip}%
\pgfsetbuttcap%
\pgfsetmiterjoin%
\definecolor{currentfill}{rgb}{0.341423,0.628958,0.808704}%
\pgfsetfillcolor{currentfill}%
\pgfsetlinewidth{0.000000pt}%
\definecolor{currentstroke}{rgb}{0.000000,0.000000,0.000000}%
\pgfsetstrokecolor{currentstroke}%
\pgfsetstrokeopacity{0.000000}%
\pgfsetdash{}{0pt}%
\pgfpathmoveto{\pgfqpoint{1.963991in}{0.607879in}}%
\pgfpathlineto{\pgfqpoint{1.958935in}{0.610678in}}%
\pgfpathlineto{\pgfqpoint{1.958991in}{0.616692in}}%
\pgfpathlineto{\pgfqpoint{1.961233in}{0.620328in}}%
\pgfpathlineto{\pgfqpoint{1.952131in}{0.625210in}}%
\pgfpathlineto{\pgfqpoint{1.949979in}{0.638422in}}%
\pgfpathlineto{\pgfqpoint{1.954469in}{0.643065in}}%
\pgfpathlineto{\pgfqpoint{1.959044in}{0.656761in}}%
\pgfpathlineto{\pgfqpoint{1.962071in}{0.656269in}}%
\pgfpathlineto{\pgfqpoint{1.965330in}{0.661044in}}%
\pgfpathlineto{\pgfqpoint{1.965268in}{0.666642in}}%
\pgfpathlineto{\pgfqpoint{1.969490in}{0.679240in}}%
\pgfpathlineto{\pgfqpoint{1.969121in}{0.688091in}}%
\pgfpathlineto{\pgfqpoint{1.978768in}{0.688413in}}%
\pgfpathlineto{\pgfqpoint{1.989120in}{0.691955in}}%
\pgfpathlineto{\pgfqpoint{1.987667in}{0.684255in}}%
\pgfpathlineto{\pgfqpoint{1.995075in}{0.682326in}}%
\pgfpathlineto{\pgfqpoint{1.993029in}{0.676683in}}%
\pgfpathlineto{\pgfqpoint{1.986876in}{0.672737in}}%
\pgfpathlineto{\pgfqpoint{1.986197in}{0.669162in}}%
\pgfpathlineto{\pgfqpoint{1.978327in}{0.664514in}}%
\pgfpathlineto{\pgfqpoint{1.981608in}{0.653144in}}%
\pgfpathlineto{\pgfqpoint{1.989789in}{0.649517in}}%
\pgfpathlineto{\pgfqpoint{2.013321in}{0.650638in}}%
\pgfpathlineto{\pgfqpoint{2.019201in}{0.650937in}}%
\pgfpathlineto{\pgfqpoint{2.020031in}{0.633536in}}%
\pgfpathlineto{\pgfqpoint{1.998230in}{0.632422in}}%
\pgfpathlineto{\pgfqpoint{1.993008in}{0.630363in}}%
\pgfpathlineto{\pgfqpoint{1.985698in}{0.633433in}}%
\pgfpathlineto{\pgfqpoint{1.972763in}{0.629511in}}%
\pgfpathlineto{\pgfqpoint{1.971855in}{0.623291in}}%
\pgfpathlineto{\pgfqpoint{1.967749in}{0.623734in}}%
\pgfpathlineto{\pgfqpoint{1.964384in}{0.616831in}}%
\pgfpathlineto{\pgfqpoint{1.967968in}{0.612514in}}%
\pgfpathlineto{\pgfqpoint{1.963991in}{0.607879in}}%
\pgfpathclose%
\pgfusepath{fill}%
\end{pgfscope}%
\begin{pgfscope}%
\pgfpathrectangle{\pgfqpoint{0.100000in}{0.100000in}}{\pgfqpoint{3.007045in}{1.925000in}}%
\pgfusepath{clip}%
\pgfsetbuttcap%
\pgfsetmiterjoin%
\definecolor{currentfill}{rgb}{0.401953,0.670296,0.832326}%
\pgfsetfillcolor{currentfill}%
\pgfsetlinewidth{0.000000pt}%
\definecolor{currentstroke}{rgb}{0.000000,0.000000,0.000000}%
\pgfsetstrokecolor{currentstroke}%
\pgfsetstrokeopacity{0.000000}%
\pgfsetdash{}{0pt}%
\pgfpathmoveto{\pgfqpoint{1.990535in}{1.152936in}}%
\pgfpathlineto{\pgfqpoint{1.967023in}{1.152094in}}%
\pgfpathlineto{\pgfqpoint{1.967135in}{1.146275in}}%
\pgfpathlineto{\pgfqpoint{1.963320in}{1.146179in}}%
\pgfpathlineto{\pgfqpoint{1.955717in}{1.146050in}}%
\pgfpathlineto{\pgfqpoint{1.953772in}{1.157587in}}%
\pgfpathlineto{\pgfqpoint{1.939480in}{1.158394in}}%
\pgfpathlineto{\pgfqpoint{1.938875in}{1.179366in}}%
\pgfpathlineto{\pgfqpoint{1.932193in}{1.179188in}}%
\pgfpathlineto{\pgfqpoint{1.931854in}{1.198270in}}%
\pgfpathlineto{\pgfqpoint{1.926181in}{1.198155in}}%
\pgfpathlineto{\pgfqpoint{1.925598in}{1.218425in}}%
\pgfpathlineto{\pgfqpoint{1.948406in}{1.218688in}}%
\pgfpathlineto{\pgfqpoint{1.948122in}{1.215494in}}%
\pgfpathlineto{\pgfqpoint{1.977971in}{1.216245in}}%
\pgfpathlineto{\pgfqpoint{1.978770in}{1.193340in}}%
\pgfpathlineto{\pgfqpoint{1.996210in}{1.193778in}}%
\pgfpathlineto{\pgfqpoint{1.992899in}{1.184317in}}%
\pgfpathlineto{\pgfqpoint{1.997146in}{1.174029in}}%
\pgfpathlineto{\pgfqpoint{1.995501in}{1.164493in}}%
\pgfpathlineto{\pgfqpoint{1.979066in}{1.163974in}}%
\pgfpathlineto{\pgfqpoint{1.986624in}{1.158384in}}%
\pgfpathlineto{\pgfqpoint{1.990535in}{1.152936in}}%
\pgfpathclose%
\pgfusepath{fill}%
\end{pgfscope}%
\begin{pgfscope}%
\pgfpathrectangle{\pgfqpoint{0.100000in}{0.100000in}}{\pgfqpoint{3.007045in}{1.925000in}}%
\pgfusepath{clip}%
\pgfsetbuttcap%
\pgfsetmiterjoin%
\definecolor{currentfill}{rgb}{0.622684,0.793464,0.883429}%
\pgfsetfillcolor{currentfill}%
\pgfsetlinewidth{0.000000pt}%
\definecolor{currentstroke}{rgb}{0.000000,0.000000,0.000000}%
\pgfsetstrokecolor{currentstroke}%
\pgfsetstrokeopacity{0.000000}%
\pgfsetdash{}{0pt}%
\pgfpathmoveto{\pgfqpoint{0.856506in}{1.901711in}}%
\pgfpathlineto{\pgfqpoint{0.848850in}{1.869247in}}%
\pgfpathlineto{\pgfqpoint{0.817161in}{1.876758in}}%
\pgfpathlineto{\pgfqpoint{0.822653in}{1.899281in}}%
\pgfpathlineto{\pgfqpoint{0.812070in}{1.901824in}}%
\pgfpathlineto{\pgfqpoint{0.814525in}{1.911735in}}%
\pgfpathlineto{\pgfqpoint{0.856506in}{1.901711in}}%
\pgfpathclose%
\pgfusepath{fill}%
\end{pgfscope}%
\begin{pgfscope}%
\pgfpathrectangle{\pgfqpoint{0.100000in}{0.100000in}}{\pgfqpoint{3.007045in}{1.925000in}}%
\pgfusepath{clip}%
\pgfsetbuttcap%
\pgfsetmiterjoin%
\definecolor{currentfill}{rgb}{0.460392,0.704744,0.848012}%
\pgfsetfillcolor{currentfill}%
\pgfsetlinewidth{0.000000pt}%
\definecolor{currentstroke}{rgb}{0.000000,0.000000,0.000000}%
\pgfsetstrokecolor{currentstroke}%
\pgfsetstrokeopacity{0.000000}%
\pgfsetdash{}{0pt}%
\pgfpathmoveto{\pgfqpoint{2.349537in}{1.370183in}}%
\pgfpathlineto{\pgfqpoint{2.343612in}{1.373779in}}%
\pgfpathlineto{\pgfqpoint{2.320460in}{1.370056in}}%
\pgfpathlineto{\pgfqpoint{2.303185in}{1.367262in}}%
\pgfpathlineto{\pgfqpoint{2.299946in}{1.390294in}}%
\pgfpathlineto{\pgfqpoint{2.296340in}{1.413257in}}%
\pgfpathlineto{\pgfqpoint{2.286455in}{1.412044in}}%
\pgfpathlineto{\pgfqpoint{2.283625in}{1.434436in}}%
\pgfpathlineto{\pgfqpoint{2.283546in}{1.435118in}}%
\pgfpathlineto{\pgfqpoint{2.304504in}{1.437926in}}%
\pgfpathlineto{\pgfqpoint{2.303651in}{1.443639in}}%
\pgfpathlineto{\pgfqpoint{2.326140in}{1.446645in}}%
\pgfpathlineto{\pgfqpoint{2.331232in}{1.448101in}}%
\pgfpathlineto{\pgfqpoint{2.330370in}{1.453796in}}%
\pgfpathlineto{\pgfqpoint{2.342275in}{1.455683in}}%
\pgfpathlineto{\pgfqpoint{2.338611in}{1.478397in}}%
\pgfpathlineto{\pgfqpoint{2.362681in}{1.482619in}}%
\pgfpathlineto{\pgfqpoint{2.364694in}{1.474272in}}%
\pgfpathlineto{\pgfqpoint{2.368291in}{1.465466in}}%
\pgfpathlineto{\pgfqpoint{2.371281in}{1.451948in}}%
\pgfpathlineto{\pgfqpoint{2.373711in}{1.444991in}}%
\pgfpathlineto{\pgfqpoint{2.377887in}{1.438966in}}%
\pgfpathlineto{\pgfqpoint{2.376582in}{1.425757in}}%
\pgfpathlineto{\pgfqpoint{2.377704in}{1.422799in}}%
\pgfpathlineto{\pgfqpoint{2.376418in}{1.412332in}}%
\pgfpathlineto{\pgfqpoint{2.367395in}{1.416634in}}%
\pgfpathlineto{\pgfqpoint{2.363048in}{1.413679in}}%
\pgfpathlineto{\pgfqpoint{2.360076in}{1.403017in}}%
\pgfpathlineto{\pgfqpoint{2.361096in}{1.399839in}}%
\pgfpathlineto{\pgfqpoint{2.359012in}{1.392852in}}%
\pgfpathlineto{\pgfqpoint{2.352893in}{1.389860in}}%
\pgfpathlineto{\pgfqpoint{2.350480in}{1.384293in}}%
\pgfpathlineto{\pgfqpoint{2.351493in}{1.374397in}}%
\pgfpathlineto{\pgfqpoint{2.349537in}{1.370183in}}%
\pgfpathclose%
\pgfusepath{fill}%
\end{pgfscope}%
\begin{pgfscope}%
\pgfpathrectangle{\pgfqpoint{0.100000in}{0.100000in}}{\pgfqpoint{3.007045in}{1.925000in}}%
\pgfusepath{clip}%
\pgfsetbuttcap%
\pgfsetmiterjoin%
\definecolor{currentfill}{rgb}{0.681738,0.818562,0.904098}%
\pgfsetfillcolor{currentfill}%
\pgfsetlinewidth{0.000000pt}%
\definecolor{currentstroke}{rgb}{0.000000,0.000000,0.000000}%
\pgfsetstrokecolor{currentstroke}%
\pgfsetstrokeopacity{0.000000}%
\pgfsetdash{}{0pt}%
\pgfpathmoveto{\pgfqpoint{0.652693in}{1.955469in}}%
\pgfpathlineto{\pgfqpoint{0.654305in}{1.933747in}}%
\pgfpathlineto{\pgfqpoint{0.650462in}{1.932235in}}%
\pgfpathlineto{\pgfqpoint{0.643432in}{1.933133in}}%
\pgfpathlineto{\pgfqpoint{0.623985in}{1.938935in}}%
\pgfpathlineto{\pgfqpoint{0.576681in}{1.953612in}}%
\pgfpathlineto{\pgfqpoint{0.577191in}{1.962609in}}%
\pgfpathlineto{\pgfqpoint{0.572821in}{1.961633in}}%
\pgfpathlineto{\pgfqpoint{0.569082in}{1.972907in}}%
\pgfpathlineto{\pgfqpoint{0.572639in}{1.979931in}}%
\pgfpathlineto{\pgfqpoint{0.614766in}{1.966556in}}%
\pgfpathlineto{\pgfqpoint{0.652693in}{1.955469in}}%
\pgfpathclose%
\pgfusepath{fill}%
\end{pgfscope}%
\begin{pgfscope}%
\pgfpathrectangle{\pgfqpoint{0.100000in}{0.100000in}}{\pgfqpoint{3.007045in}{1.925000in}}%
\pgfusepath{clip}%
\pgfsetbuttcap%
\pgfsetmiterjoin%
\definecolor{currentfill}{rgb}{0.199446,0.513172,0.745882}%
\pgfsetfillcolor{currentfill}%
\pgfsetlinewidth{0.000000pt}%
\definecolor{currentstroke}{rgb}{0.000000,0.000000,0.000000}%
\pgfsetstrokecolor{currentstroke}%
\pgfsetstrokeopacity{0.000000}%
\pgfsetdash{}{0pt}%
\pgfpathmoveto{\pgfqpoint{1.035972in}{1.058230in}}%
\pgfpathlineto{\pgfqpoint{1.025816in}{0.993300in}}%
\pgfpathlineto{\pgfqpoint{1.100725in}{0.982093in}}%
\pgfpathlineto{\pgfqpoint{1.117536in}{0.979664in}}%
\pgfpathlineto{\pgfqpoint{1.111260in}{0.934336in}}%
\pgfpathlineto{\pgfqpoint{1.049496in}{0.943259in}}%
\pgfpathlineto{\pgfqpoint{1.046069in}{0.920555in}}%
\pgfpathlineto{\pgfqpoint{1.015230in}{0.925309in}}%
\pgfpathlineto{\pgfqpoint{1.011345in}{0.900484in}}%
\pgfpathlineto{\pgfqpoint{1.003150in}{0.848168in}}%
\pgfpathlineto{\pgfqpoint{0.986809in}{0.850694in}}%
\pgfpathlineto{\pgfqpoint{0.978685in}{0.845988in}}%
\pgfpathlineto{\pgfqpoint{0.975090in}{0.841991in}}%
\pgfpathlineto{\pgfqpoint{0.971108in}{0.842140in}}%
\pgfpathlineto{\pgfqpoint{0.960834in}{0.834656in}}%
\pgfpathlineto{\pgfqpoint{0.954625in}{0.842378in}}%
\pgfpathlineto{\pgfqpoint{0.949199in}{0.843379in}}%
\pgfpathlineto{\pgfqpoint{0.953797in}{0.870982in}}%
\pgfpathlineto{\pgfqpoint{0.913272in}{0.877904in}}%
\pgfpathlineto{\pgfqpoint{0.916273in}{0.895048in}}%
\pgfpathlineto{\pgfqpoint{0.947453in}{1.073147in}}%
\pgfpathlineto{\pgfqpoint{0.986309in}{1.066160in}}%
\pgfpathlineto{\pgfqpoint{1.035972in}{1.058230in}}%
\pgfpathclose%
\pgfusepath{fill}%
\end{pgfscope}%
\begin{pgfscope}%
\pgfpathrectangle{\pgfqpoint{0.100000in}{0.100000in}}{\pgfqpoint{3.007045in}{1.925000in}}%
\pgfusepath{clip}%
\pgfsetbuttcap%
\pgfsetmiterjoin%
\definecolor{currentfill}{rgb}{0.548235,0.752972,0.866959}%
\pgfsetfillcolor{currentfill}%
\pgfsetlinewidth{0.000000pt}%
\definecolor{currentstroke}{rgb}{0.000000,0.000000,0.000000}%
\pgfsetstrokecolor{currentstroke}%
\pgfsetstrokeopacity{0.000000}%
\pgfsetdash{}{0pt}%
\pgfpathmoveto{\pgfqpoint{1.907975in}{1.607918in}}%
\pgfpathlineto{\pgfqpoint{1.930881in}{1.608680in}}%
\pgfpathlineto{\pgfqpoint{1.959683in}{1.609718in}}%
\pgfpathlineto{\pgfqpoint{1.960172in}{1.598201in}}%
\pgfpathlineto{\pgfqpoint{1.971604in}{1.598763in}}%
\pgfpathlineto{\pgfqpoint{1.973425in}{1.558760in}}%
\pgfpathlineto{\pgfqpoint{1.961960in}{1.558341in}}%
\pgfpathlineto{\pgfqpoint{1.962214in}{1.552559in}}%
\pgfpathlineto{\pgfqpoint{1.933521in}{1.551363in}}%
\pgfpathlineto{\pgfqpoint{1.933694in}{1.545701in}}%
\pgfpathlineto{\pgfqpoint{1.905032in}{1.544903in}}%
\pgfpathlineto{\pgfqpoint{1.904215in}{1.573411in}}%
\pgfpathlineto{\pgfqpoint{1.909946in}{1.573607in}}%
\pgfpathlineto{\pgfqpoint{1.909099in}{1.596409in}}%
\pgfpathlineto{\pgfqpoint{1.907975in}{1.607918in}}%
\pgfpathclose%
\pgfusepath{fill}%
\end{pgfscope}%
\begin{pgfscope}%
\pgfpathrectangle{\pgfqpoint{0.100000in}{0.100000in}}{\pgfqpoint{3.007045in}{1.925000in}}%
\pgfusepath{clip}%
\pgfsetbuttcap%
\pgfsetmiterjoin%
\definecolor{currentfill}{rgb}{0.504314,0.728858,0.857486}%
\pgfsetfillcolor{currentfill}%
\pgfsetlinewidth{0.000000pt}%
\definecolor{currentstroke}{rgb}{0.000000,0.000000,0.000000}%
\pgfsetstrokecolor{currentstroke}%
\pgfsetstrokeopacity{0.000000}%
\pgfsetdash{}{0pt}%
\pgfpathmoveto{\pgfqpoint{1.454083in}{0.686391in}}%
\pgfpathlineto{\pgfqpoint{1.425000in}{0.688410in}}%
\pgfpathlineto{\pgfqpoint{1.427496in}{0.717282in}}%
\pgfpathlineto{\pgfqpoint{1.429539in}{0.746331in}}%
\pgfpathlineto{\pgfqpoint{1.437038in}{0.746267in}}%
\pgfpathlineto{\pgfqpoint{1.438908in}{0.774476in}}%
\pgfpathlineto{\pgfqpoint{1.467658in}{0.772689in}}%
\pgfpathlineto{\pgfqpoint{1.465814in}{0.743950in}}%
\pgfpathlineto{\pgfqpoint{1.458235in}{0.744453in}}%
\pgfpathlineto{\pgfqpoint{1.454083in}{0.686391in}}%
\pgfpathclose%
\pgfusepath{fill}%
\end{pgfscope}%
\begin{pgfscope}%
\pgfpathrectangle{\pgfqpoint{0.100000in}{0.100000in}}{\pgfqpoint{3.007045in}{1.925000in}}%
\pgfusepath{clip}%
\pgfsetbuttcap%
\pgfsetmiterjoin%
\definecolor{currentfill}{rgb}{0.356555,0.639293,0.814610}%
\pgfsetfillcolor{currentfill}%
\pgfsetlinewidth{0.000000pt}%
\definecolor{currentstroke}{rgb}{0.000000,0.000000,0.000000}%
\pgfsetstrokecolor{currentstroke}%
\pgfsetstrokeopacity{0.000000}%
\pgfsetdash{}{0pt}%
\pgfpathmoveto{\pgfqpoint{2.079034in}{0.765390in}}%
\pgfpathlineto{\pgfqpoint{2.086624in}{0.765838in}}%
\pgfpathlineto{\pgfqpoint{2.087936in}{0.742436in}}%
\pgfpathlineto{\pgfqpoint{2.065136in}{0.740946in}}%
\pgfpathlineto{\pgfqpoint{2.065317in}{0.737960in}}%
\pgfpathlineto{\pgfqpoint{2.052237in}{0.736873in}}%
\pgfpathlineto{\pgfqpoint{2.031102in}{0.744404in}}%
\pgfpathlineto{\pgfqpoint{2.025625in}{0.749814in}}%
\pgfpathlineto{\pgfqpoint{2.031123in}{0.752696in}}%
\pgfpathlineto{\pgfqpoint{2.030481in}{0.763778in}}%
\pgfpathlineto{\pgfqpoint{2.033226in}{0.766351in}}%
\pgfpathlineto{\pgfqpoint{2.039047in}{0.766251in}}%
\pgfpathlineto{\pgfqpoint{2.062975in}{0.759697in}}%
\pgfpathlineto{\pgfqpoint{2.065349in}{0.764544in}}%
\pgfpathlineto{\pgfqpoint{2.079034in}{0.765390in}}%
\pgfpathclose%
\pgfusepath{fill}%
\end{pgfscope}%
\begin{pgfscope}%
\pgfpathrectangle{\pgfqpoint{0.100000in}{0.100000in}}{\pgfqpoint{3.007045in}{1.925000in}}%
\pgfusepath{clip}%
\pgfsetbuttcap%
\pgfsetmiterjoin%
\definecolor{currentfill}{rgb}{0.554510,0.756417,0.868312}%
\pgfsetfillcolor{currentfill}%
\pgfsetlinewidth{0.000000pt}%
\definecolor{currentstroke}{rgb}{0.000000,0.000000,0.000000}%
\pgfsetstrokecolor{currentstroke}%
\pgfsetstrokeopacity{0.000000}%
\pgfsetdash{}{0pt}%
\pgfpathmoveto{\pgfqpoint{1.973511in}{1.274071in}}%
\pgfpathlineto{\pgfqpoint{1.952591in}{1.273282in}}%
\pgfpathlineto{\pgfqpoint{1.952380in}{1.279020in}}%
\pgfpathlineto{\pgfqpoint{1.935244in}{1.278471in}}%
\pgfpathlineto{\pgfqpoint{1.923843in}{1.278117in}}%
\pgfpathlineto{\pgfqpoint{1.923061in}{1.300995in}}%
\pgfpathlineto{\pgfqpoint{1.928815in}{1.301237in}}%
\pgfpathlineto{\pgfqpoint{1.927902in}{1.324319in}}%
\pgfpathlineto{\pgfqpoint{1.950752in}{1.325057in}}%
\pgfpathlineto{\pgfqpoint{1.950534in}{1.330806in}}%
\pgfpathlineto{\pgfqpoint{1.973436in}{1.331705in}}%
\pgfpathlineto{\pgfqpoint{1.973965in}{1.320197in}}%
\pgfpathlineto{\pgfqpoint{1.974408in}{1.308715in}}%
\pgfpathlineto{\pgfqpoint{1.980128in}{1.308936in}}%
\pgfpathlineto{\pgfqpoint{1.980445in}{1.299472in}}%
\pgfpathlineto{\pgfqpoint{1.967742in}{1.296340in}}%
\pgfpathlineto{\pgfqpoint{1.964916in}{1.284730in}}%
\pgfpathlineto{\pgfqpoint{1.971270in}{1.279448in}}%
\pgfpathlineto{\pgfqpoint{1.973511in}{1.274071in}}%
\pgfpathclose%
\pgfusepath{fill}%
\end{pgfscope}%
\begin{pgfscope}%
\pgfpathrectangle{\pgfqpoint{0.100000in}{0.100000in}}{\pgfqpoint{3.007045in}{1.925000in}}%
\pgfusepath{clip}%
\pgfsetbuttcap%
\pgfsetmiterjoin%
\definecolor{currentfill}{rgb}{0.396909,0.666851,0.830358}%
\pgfsetfillcolor{currentfill}%
\pgfsetlinewidth{0.000000pt}%
\definecolor{currentstroke}{rgb}{0.000000,0.000000,0.000000}%
\pgfsetstrokecolor{currentstroke}%
\pgfsetstrokeopacity{0.000000}%
\pgfsetdash{}{0pt}%
\pgfpathmoveto{\pgfqpoint{1.765864in}{1.580217in}}%
\pgfpathlineto{\pgfqpoint{1.766199in}{1.556278in}}%
\pgfpathlineto{\pgfqpoint{1.720360in}{1.556564in}}%
\pgfpathlineto{\pgfqpoint{1.714573in}{1.556685in}}%
\pgfpathlineto{\pgfqpoint{1.714182in}{1.568146in}}%
\pgfpathlineto{\pgfqpoint{1.714476in}{1.591318in}}%
\pgfpathlineto{\pgfqpoint{1.714002in}{1.597084in}}%
\pgfpathlineto{\pgfqpoint{1.713584in}{1.614400in}}%
\pgfpathlineto{\pgfqpoint{1.713868in}{1.637553in}}%
\pgfpathlineto{\pgfqpoint{1.718828in}{1.637478in}}%
\pgfpathlineto{\pgfqpoint{1.718799in}{1.643280in}}%
\pgfpathlineto{\pgfqpoint{1.764876in}{1.642920in}}%
\pgfpathlineto{\pgfqpoint{1.765646in}{1.613993in}}%
\pgfpathlineto{\pgfqpoint{1.765864in}{1.580217in}}%
\pgfpathclose%
\pgfusepath{fill}%
\end{pgfscope}%
\begin{pgfscope}%
\pgfpathrectangle{\pgfqpoint{0.100000in}{0.100000in}}{\pgfqpoint{3.007045in}{1.925000in}}%
\pgfusepath{clip}%
\pgfsetbuttcap%
\pgfsetmiterjoin%
\definecolor{currentfill}{rgb}{0.346467,0.632403,0.810673}%
\pgfsetfillcolor{currentfill}%
\pgfsetlinewidth{0.000000pt}%
\definecolor{currentstroke}{rgb}{0.000000,0.000000,0.000000}%
\pgfsetstrokecolor{currentstroke}%
\pgfsetstrokeopacity{0.000000}%
\pgfsetdash{}{0pt}%
\pgfpathmoveto{\pgfqpoint{2.364743in}{1.174580in}}%
\pgfpathlineto{\pgfqpoint{2.353578in}{1.170749in}}%
\pgfpathlineto{\pgfqpoint{2.340324in}{1.169258in}}%
\pgfpathlineto{\pgfqpoint{2.338836in}{1.184076in}}%
\pgfpathlineto{\pgfqpoint{2.331627in}{1.183748in}}%
\pgfpathlineto{\pgfqpoint{2.330635in}{1.204428in}}%
\pgfpathlineto{\pgfqpoint{2.346250in}{1.205103in}}%
\pgfpathlineto{\pgfqpoint{2.345734in}{1.216099in}}%
\pgfpathlineto{\pgfqpoint{2.366064in}{1.217273in}}%
\pgfpathlineto{\pgfqpoint{2.366865in}{1.205475in}}%
\pgfpathlineto{\pgfqpoint{2.362689in}{1.195710in}}%
\pgfpathlineto{\pgfqpoint{2.363339in}{1.187311in}}%
\pgfpathlineto{\pgfqpoint{2.365382in}{1.186478in}}%
\pgfpathlineto{\pgfqpoint{2.364743in}{1.174580in}}%
\pgfpathclose%
\pgfusepath{fill}%
\end{pgfscope}%
\begin{pgfscope}%
\pgfpathrectangle{\pgfqpoint{0.100000in}{0.100000in}}{\pgfqpoint{3.007045in}{1.925000in}}%
\pgfusepath{clip}%
\pgfsetbuttcap%
\pgfsetmiterjoin%
\definecolor{currentfill}{rgb}{0.805260,0.878016,0.946851}%
\pgfsetfillcolor{currentfill}%
\pgfsetlinewidth{0.000000pt}%
\definecolor{currentstroke}{rgb}{0.000000,0.000000,0.000000}%
\pgfsetstrokecolor{currentstroke}%
\pgfsetstrokeopacity{0.000000}%
\pgfsetdash{}{0pt}%
\pgfpathmoveto{\pgfqpoint{3.013596in}{1.761178in}}%
\pgfpathlineto{\pgfqpoint{3.033217in}{1.758207in}}%
\pgfpathlineto{\pgfqpoint{3.030957in}{1.753391in}}%
\pgfpathlineto{\pgfqpoint{3.036928in}{1.745059in}}%
\pgfpathlineto{\pgfqpoint{3.036155in}{1.740067in}}%
\pgfpathlineto{\pgfqpoint{3.045228in}{1.732473in}}%
\pgfpathlineto{\pgfqpoint{3.046987in}{1.737585in}}%
\pgfpathlineto{\pgfqpoint{3.052581in}{1.737319in}}%
\pgfpathlineto{\pgfqpoint{3.064704in}{1.724060in}}%
\pgfpathlineto{\pgfqpoint{3.067422in}{1.717798in}}%
\pgfpathlineto{\pgfqpoint{3.063897in}{1.711796in}}%
\pgfpathlineto{\pgfqpoint{3.061530in}{1.704211in}}%
\pgfpathlineto{\pgfqpoint{3.052328in}{1.703378in}}%
\pgfpathlineto{\pgfqpoint{3.052800in}{1.697836in}}%
\pgfpathlineto{\pgfqpoint{3.045823in}{1.697647in}}%
\pgfpathlineto{\pgfqpoint{3.044375in}{1.690587in}}%
\pgfpathlineto{\pgfqpoint{3.036932in}{1.689276in}}%
\pgfpathlineto{\pgfqpoint{3.034646in}{1.679524in}}%
\pgfpathlineto{\pgfqpoint{3.031696in}{1.678779in}}%
\pgfpathlineto{\pgfqpoint{3.025401in}{1.689136in}}%
\pgfpathlineto{\pgfqpoint{3.014639in}{1.710091in}}%
\pgfpathlineto{\pgfqpoint{3.019840in}{1.712826in}}%
\pgfpathlineto{\pgfqpoint{3.010773in}{1.730803in}}%
\pgfpathlineto{\pgfqpoint{3.015744in}{1.733166in}}%
\pgfpathlineto{\pgfqpoint{3.003662in}{1.755298in}}%
\pgfpathlineto{\pgfqpoint{3.013596in}{1.761178in}}%
\pgfpathclose%
\pgfusepath{fill}%
\end{pgfscope}%
\begin{pgfscope}%
\pgfpathrectangle{\pgfqpoint{0.100000in}{0.100000in}}{\pgfqpoint{3.007045in}{1.925000in}}%
\pgfusepath{clip}%
\pgfsetbuttcap%
\pgfsetmiterjoin%
\definecolor{currentfill}{rgb}{0.417086,0.680631,0.838231}%
\pgfsetfillcolor{currentfill}%
\pgfsetlinewidth{0.000000pt}%
\definecolor{currentstroke}{rgb}{0.000000,0.000000,0.000000}%
\pgfsetstrokecolor{currentstroke}%
\pgfsetstrokeopacity{0.000000}%
\pgfsetdash{}{0pt}%
\pgfpathmoveto{\pgfqpoint{2.378309in}{1.338718in}}%
\pgfpathlineto{\pgfqpoint{2.365754in}{1.334376in}}%
\pgfpathlineto{\pgfqpoint{2.346907in}{1.331869in}}%
\pgfpathlineto{\pgfqpoint{2.342797in}{1.334180in}}%
\pgfpathlineto{\pgfqpoint{2.345273in}{1.312326in}}%
\pgfpathlineto{\pgfqpoint{2.322615in}{1.309553in}}%
\pgfpathlineto{\pgfqpoint{2.305531in}{1.307364in}}%
\pgfpathlineto{\pgfqpoint{2.303481in}{1.324508in}}%
\pgfpathlineto{\pgfqpoint{2.297909in}{1.323832in}}%
\pgfpathlineto{\pgfqpoint{2.295342in}{1.329269in}}%
\pgfpathlineto{\pgfqpoint{2.292953in}{1.341716in}}%
\pgfpathlineto{\pgfqpoint{2.323909in}{1.346635in}}%
\pgfpathlineto{\pgfqpoint{2.320460in}{1.370056in}}%
\pgfpathlineto{\pgfqpoint{2.343612in}{1.373779in}}%
\pgfpathlineto{\pgfqpoint{2.349537in}{1.370183in}}%
\pgfpathlineto{\pgfqpoint{2.343956in}{1.363335in}}%
\pgfpathlineto{\pgfqpoint{2.338987in}{1.354529in}}%
\pgfpathlineto{\pgfqpoint{2.338422in}{1.346684in}}%
\pgfpathlineto{\pgfqpoint{2.344381in}{1.346977in}}%
\pgfpathlineto{\pgfqpoint{2.358766in}{1.342811in}}%
\pgfpathlineto{\pgfqpoint{2.364318in}{1.338530in}}%
\pgfpathlineto{\pgfqpoint{2.375956in}{1.341494in}}%
\pgfpathlineto{\pgfqpoint{2.378309in}{1.338718in}}%
\pgfpathclose%
\pgfusepath{fill}%
\end{pgfscope}%
\begin{pgfscope}%
\pgfpathrectangle{\pgfqpoint{0.100000in}{0.100000in}}{\pgfqpoint{3.007045in}{1.925000in}}%
\pgfusepath{clip}%
\pgfsetbuttcap%
\pgfsetmiterjoin%
\definecolor{currentfill}{rgb}{0.696501,0.824837,0.909266}%
\pgfsetfillcolor{currentfill}%
\pgfsetlinewidth{0.000000pt}%
\definecolor{currentstroke}{rgb}{0.000000,0.000000,0.000000}%
\pgfsetstrokecolor{currentstroke}%
\pgfsetstrokeopacity{0.000000}%
\pgfsetdash{}{0pt}%
\pgfpathmoveto{\pgfqpoint{2.000756in}{1.600279in}}%
\pgfpathlineto{\pgfqpoint{2.002328in}{1.571903in}}%
\pgfpathlineto{\pgfqpoint{2.003003in}{1.560543in}}%
\pgfpathlineto{\pgfqpoint{1.973425in}{1.558760in}}%
\pgfpathlineto{\pgfqpoint{1.971604in}{1.598763in}}%
\pgfpathlineto{\pgfqpoint{1.960172in}{1.598201in}}%
\pgfpathlineto{\pgfqpoint{1.959683in}{1.609718in}}%
\pgfpathlineto{\pgfqpoint{1.930881in}{1.608680in}}%
\pgfpathlineto{\pgfqpoint{1.929386in}{1.648471in}}%
\pgfpathlineto{\pgfqpoint{1.937520in}{1.651178in}}%
\pgfpathlineto{\pgfqpoint{1.944689in}{1.656422in}}%
\pgfpathlineto{\pgfqpoint{1.949266in}{1.655981in}}%
\pgfpathlineto{\pgfqpoint{1.955129in}{1.661825in}}%
\pgfpathlineto{\pgfqpoint{1.960258in}{1.663453in}}%
\pgfpathlineto{\pgfqpoint{1.965406in}{1.659054in}}%
\pgfpathlineto{\pgfqpoint{1.959719in}{1.649522in}}%
\pgfpathlineto{\pgfqpoint{1.959203in}{1.638150in}}%
\pgfpathlineto{\pgfqpoint{1.968638in}{1.644005in}}%
\pgfpathlineto{\pgfqpoint{1.975507in}{1.638992in}}%
\pgfpathlineto{\pgfqpoint{1.976511in}{1.616232in}}%
\pgfpathlineto{\pgfqpoint{1.982301in}{1.616494in}}%
\pgfpathlineto{\pgfqpoint{1.982576in}{1.610763in}}%
\pgfpathlineto{\pgfqpoint{1.988231in}{1.611105in}}%
\pgfpathlineto{\pgfqpoint{1.988873in}{1.599607in}}%
\pgfpathlineto{\pgfqpoint{2.000756in}{1.600279in}}%
\pgfpathclose%
\pgfusepath{fill}%
\end{pgfscope}%
\begin{pgfscope}%
\pgfpathrectangle{\pgfqpoint{0.100000in}{0.100000in}}{\pgfqpoint{3.007045in}{1.925000in}}%
\pgfusepath{clip}%
\pgfsetbuttcap%
\pgfsetmiterjoin%
\definecolor{currentfill}{rgb}{0.447843,0.697855,0.845306}%
\pgfsetfillcolor{currentfill}%
\pgfsetlinewidth{0.000000pt}%
\definecolor{currentstroke}{rgb}{0.000000,0.000000,0.000000}%
\pgfsetstrokecolor{currentstroke}%
\pgfsetstrokeopacity{0.000000}%
\pgfsetdash{}{0pt}%
\pgfpathmoveto{\pgfqpoint{2.212181in}{1.234336in}}%
\pgfpathlineto{\pgfqpoint{2.212424in}{1.232038in}}%
\pgfpathlineto{\pgfqpoint{2.189712in}{1.229760in}}%
\pgfpathlineto{\pgfqpoint{2.189499in}{1.232103in}}%
\pgfpathlineto{\pgfqpoint{2.169632in}{1.230355in}}%
\pgfpathlineto{\pgfqpoint{2.171191in}{1.213179in}}%
\pgfpathlineto{\pgfqpoint{2.154631in}{1.211728in}}%
\pgfpathlineto{\pgfqpoint{2.154351in}{1.223276in}}%
\pgfpathlineto{\pgfqpoint{2.147976in}{1.224103in}}%
\pgfpathlineto{\pgfqpoint{2.149319in}{1.206705in}}%
\pgfpathlineto{\pgfqpoint{2.128961in}{1.204856in}}%
\pgfpathlineto{\pgfqpoint{2.126956in}{1.227545in}}%
\pgfpathlineto{\pgfqpoint{2.125964in}{1.244692in}}%
\pgfpathlineto{\pgfqpoint{2.146371in}{1.246681in}}%
\pgfpathlineto{\pgfqpoint{2.145056in}{1.262796in}}%
\pgfpathlineto{\pgfqpoint{2.166328in}{1.264572in}}%
\pgfpathlineto{\pgfqpoint{2.165690in}{1.271203in}}%
\pgfpathlineto{\pgfqpoint{2.171318in}{1.271716in}}%
\pgfpathlineto{\pgfqpoint{2.173650in}{1.276857in}}%
\pgfpathlineto{\pgfqpoint{2.190977in}{1.278330in}}%
\pgfpathlineto{\pgfqpoint{2.192034in}{1.266835in}}%
\pgfpathlineto{\pgfqpoint{2.195116in}{1.264251in}}%
\pgfpathlineto{\pgfqpoint{2.202654in}{1.264977in}}%
\pgfpathlineto{\pgfqpoint{2.204170in}{1.247864in}}%
\pgfpathlineto{\pgfqpoint{2.211166in}{1.244697in}}%
\pgfpathlineto{\pgfqpoint{2.212181in}{1.234336in}}%
\pgfpathclose%
\pgfusepath{fill}%
\end{pgfscope}%
\begin{pgfscope}%
\pgfpathrectangle{\pgfqpoint{0.100000in}{0.100000in}}{\pgfqpoint{3.007045in}{1.925000in}}%
\pgfusepath{clip}%
\pgfsetbuttcap%
\pgfsetmiterjoin%
\definecolor{currentfill}{rgb}{0.296025,0.597955,0.790988}%
\pgfsetfillcolor{currentfill}%
\pgfsetlinewidth{0.000000pt}%
\definecolor{currentstroke}{rgb}{0.000000,0.000000,0.000000}%
\pgfsetstrokecolor{currentstroke}%
\pgfsetstrokeopacity{0.000000}%
\pgfsetdash{}{0pt}%
\pgfpathmoveto{\pgfqpoint{1.851931in}{0.616696in}}%
\pgfpathlineto{\pgfqpoint{1.834269in}{0.615251in}}%
\pgfpathlineto{\pgfqpoint{1.821877in}{0.611209in}}%
\pgfpathlineto{\pgfqpoint{1.810103in}{0.619771in}}%
\pgfpathlineto{\pgfqpoint{1.804231in}{0.625251in}}%
\pgfpathlineto{\pgfqpoint{1.800480in}{0.631756in}}%
\pgfpathlineto{\pgfqpoint{1.787706in}{0.634802in}}%
\pgfpathlineto{\pgfqpoint{1.779925in}{0.639399in}}%
\pgfpathlineto{\pgfqpoint{1.776581in}{0.643123in}}%
\pgfpathlineto{\pgfqpoint{1.773764in}{0.656330in}}%
\pgfpathlineto{\pgfqpoint{1.775875in}{0.660550in}}%
\pgfpathlineto{\pgfqpoint{1.803211in}{0.660558in}}%
\pgfpathlineto{\pgfqpoint{1.799830in}{0.669148in}}%
\pgfpathlineto{\pgfqpoint{1.827733in}{0.669764in}}%
\pgfpathlineto{\pgfqpoint{1.835531in}{0.660856in}}%
\pgfpathlineto{\pgfqpoint{1.838775in}{0.656201in}}%
\pgfpathlineto{\pgfqpoint{1.839877in}{0.650274in}}%
\pgfpathlineto{\pgfqpoint{1.837976in}{0.644603in}}%
\pgfpathlineto{\pgfqpoint{1.843180in}{0.636011in}}%
\pgfpathlineto{\pgfqpoint{1.847519in}{0.631354in}}%
\pgfpathlineto{\pgfqpoint{1.846842in}{0.625571in}}%
\pgfpathlineto{\pgfqpoint{1.851931in}{0.616696in}}%
\pgfpathclose%
\pgfusepath{fill}%
\end{pgfscope}%
\begin{pgfscope}%
\pgfpathrectangle{\pgfqpoint{0.100000in}{0.100000in}}{\pgfqpoint{3.007045in}{1.925000in}}%
\pgfusepath{clip}%
\pgfsetbuttcap%
\pgfsetmiterjoin%
\definecolor{currentfill}{rgb}{0.637447,0.799739,0.888597}%
\pgfsetfillcolor{currentfill}%
\pgfsetlinewidth{0.000000pt}%
\definecolor{currentstroke}{rgb}{0.000000,0.000000,0.000000}%
\pgfsetstrokecolor{currentstroke}%
\pgfsetstrokeopacity{0.000000}%
\pgfsetdash{}{0pt}%
\pgfpathmoveto{\pgfqpoint{2.639205in}{0.819156in}}%
\pgfpathlineto{\pgfqpoint{2.628686in}{0.826071in}}%
\pgfpathlineto{\pgfqpoint{2.612678in}{0.829037in}}%
\pgfpathlineto{\pgfqpoint{2.597720in}{0.838651in}}%
\pgfpathlineto{\pgfqpoint{2.590755in}{0.837224in}}%
\pgfpathlineto{\pgfqpoint{2.595239in}{0.853039in}}%
\pgfpathlineto{\pgfqpoint{2.593582in}{0.855868in}}%
\pgfpathlineto{\pgfqpoint{2.597384in}{0.859907in}}%
\pgfpathlineto{\pgfqpoint{2.592778in}{0.867711in}}%
\pgfpathlineto{\pgfqpoint{2.596705in}{0.871224in}}%
\pgfpathlineto{\pgfqpoint{2.580510in}{0.880092in}}%
\pgfpathlineto{\pgfqpoint{2.582316in}{0.884878in}}%
\pgfpathlineto{\pgfqpoint{2.577854in}{0.890686in}}%
\pgfpathlineto{\pgfqpoint{2.571314in}{0.892287in}}%
\pgfpathlineto{\pgfqpoint{2.582994in}{0.902658in}}%
\pgfpathlineto{\pgfqpoint{2.595973in}{0.904310in}}%
\pgfpathlineto{\pgfqpoint{2.607323in}{0.893730in}}%
\pgfpathlineto{\pgfqpoint{2.613555in}{0.916458in}}%
\pgfpathlineto{\pgfqpoint{2.637551in}{0.899069in}}%
\pgfpathlineto{\pgfqpoint{2.671279in}{0.875019in}}%
\pgfpathlineto{\pgfqpoint{2.663098in}{0.870424in}}%
\pgfpathlineto{\pgfqpoint{2.657985in}{0.865431in}}%
\pgfpathlineto{\pgfqpoint{2.651920in}{0.856646in}}%
\pgfpathlineto{\pgfqpoint{2.646223in}{0.845821in}}%
\pgfpathlineto{\pgfqpoint{2.644013in}{0.838187in}}%
\pgfpathlineto{\pgfqpoint{2.644164in}{0.824590in}}%
\pgfpathlineto{\pgfqpoint{2.639205in}{0.819156in}}%
\pgfpathclose%
\pgfusepath{fill}%
\end{pgfscope}%
\begin{pgfscope}%
\pgfpathrectangle{\pgfqpoint{0.100000in}{0.100000in}}{\pgfqpoint{3.007045in}{1.925000in}}%
\pgfusepath{clip}%
\pgfsetbuttcap%
\pgfsetmiterjoin%
\definecolor{currentfill}{rgb}{0.770319,0.856209,0.935102}%
\pgfsetfillcolor{currentfill}%
\pgfsetlinewidth{0.000000pt}%
\definecolor{currentstroke}{rgb}{0.000000,0.000000,0.000000}%
\pgfsetstrokecolor{currentstroke}%
\pgfsetstrokeopacity{0.000000}%
\pgfsetdash{}{0pt}%
\pgfpathmoveto{\pgfqpoint{2.192935in}{1.614287in}}%
\pgfpathlineto{\pgfqpoint{2.188229in}{1.662007in}}%
\pgfpathlineto{\pgfqpoint{2.204739in}{1.662788in}}%
\pgfpathlineto{\pgfqpoint{2.213687in}{1.668241in}}%
\pgfpathlineto{\pgfqpoint{2.228170in}{1.671708in}}%
\pgfpathlineto{\pgfqpoint{2.226016in}{1.665581in}}%
\pgfpathlineto{\pgfqpoint{2.227291in}{1.652489in}}%
\pgfpathlineto{\pgfqpoint{2.236977in}{1.650715in}}%
\pgfpathlineto{\pgfqpoint{2.243546in}{1.654372in}}%
\pgfpathlineto{\pgfqpoint{2.247528in}{1.650275in}}%
\pgfpathlineto{\pgfqpoint{2.255094in}{1.656649in}}%
\pgfpathlineto{\pgfqpoint{2.269105in}{1.658515in}}%
\pgfpathlineto{\pgfqpoint{2.268196in}{1.652678in}}%
\pgfpathlineto{\pgfqpoint{2.270778in}{1.646529in}}%
\pgfpathlineto{\pgfqpoint{2.269357in}{1.637766in}}%
\pgfpathlineto{\pgfqpoint{2.275952in}{1.634432in}}%
\pgfpathlineto{\pgfqpoint{2.278409in}{1.628086in}}%
\pgfpathlineto{\pgfqpoint{2.287170in}{1.626728in}}%
\pgfpathlineto{\pgfqpoint{2.296641in}{1.634077in}}%
\pgfpathlineto{\pgfqpoint{2.302457in}{1.627925in}}%
\pgfpathlineto{\pgfqpoint{2.301033in}{1.623548in}}%
\pgfpathlineto{\pgfqpoint{2.294668in}{1.624039in}}%
\pgfpathlineto{\pgfqpoint{2.288149in}{1.622604in}}%
\pgfpathlineto{\pgfqpoint{2.283389in}{1.624709in}}%
\pgfpathlineto{\pgfqpoint{2.272380in}{1.623457in}}%
\pgfpathlineto{\pgfqpoint{2.261434in}{1.619276in}}%
\pgfpathlineto{\pgfqpoint{2.254060in}{1.620885in}}%
\pgfpathlineto{\pgfqpoint{2.251911in}{1.624956in}}%
\pgfpathlineto{\pgfqpoint{2.246315in}{1.624264in}}%
\pgfpathlineto{\pgfqpoint{2.244624in}{1.611374in}}%
\pgfpathlineto{\pgfqpoint{2.236991in}{1.615863in}}%
\pgfpathlineto{\pgfqpoint{2.232484in}{1.620710in}}%
\pgfpathlineto{\pgfqpoint{2.225249in}{1.622997in}}%
\pgfpathlineto{\pgfqpoint{2.213158in}{1.624960in}}%
\pgfpathlineto{\pgfqpoint{2.207852in}{1.624039in}}%
\pgfpathlineto{\pgfqpoint{2.201875in}{1.614996in}}%
\pgfpathlineto{\pgfqpoint{2.192935in}{1.614287in}}%
\pgfpathclose%
\pgfusepath{fill}%
\end{pgfscope}%
\begin{pgfscope}%
\pgfpathrectangle{\pgfqpoint{0.100000in}{0.100000in}}{\pgfqpoint{3.007045in}{1.925000in}}%
\pgfusepath{clip}%
\pgfsetbuttcap%
\pgfsetmiterjoin%
\definecolor{currentfill}{rgb}{0.391865,0.663406,0.828389}%
\pgfsetfillcolor{currentfill}%
\pgfsetlinewidth{0.000000pt}%
\definecolor{currentstroke}{rgb}{0.000000,0.000000,0.000000}%
\pgfsetstrokecolor{currentstroke}%
\pgfsetstrokeopacity{0.000000}%
\pgfsetdash{}{0pt}%
\pgfpathmoveto{\pgfqpoint{2.085014in}{0.905571in}}%
\pgfpathlineto{\pgfqpoint{2.069407in}{0.904987in}}%
\pgfpathlineto{\pgfqpoint{2.068214in}{0.916226in}}%
\pgfpathlineto{\pgfqpoint{2.061506in}{0.919581in}}%
\pgfpathlineto{\pgfqpoint{2.051337in}{0.919577in}}%
\pgfpathlineto{\pgfqpoint{2.045514in}{0.912506in}}%
\pgfpathlineto{\pgfqpoint{2.043371in}{0.916596in}}%
\pgfpathlineto{\pgfqpoint{2.042608in}{0.925600in}}%
\pgfpathlineto{\pgfqpoint{2.049440in}{0.927189in}}%
\pgfpathlineto{\pgfqpoint{2.052220in}{0.930837in}}%
\pgfpathlineto{\pgfqpoint{2.051897in}{0.936223in}}%
\pgfpathlineto{\pgfqpoint{2.054350in}{0.941105in}}%
\pgfpathlineto{\pgfqpoint{2.055729in}{0.949079in}}%
\pgfpathlineto{\pgfqpoint{2.060132in}{0.952292in}}%
\pgfpathlineto{\pgfqpoint{2.054510in}{0.958420in}}%
\pgfpathlineto{\pgfqpoint{2.062788in}{0.960749in}}%
\pgfpathlineto{\pgfqpoint{2.061897in}{0.966247in}}%
\pgfpathlineto{\pgfqpoint{2.061582in}{0.976889in}}%
\pgfpathlineto{\pgfqpoint{2.064440in}{0.977028in}}%
\pgfpathlineto{\pgfqpoint{2.068022in}{0.977336in}}%
\pgfpathlineto{\pgfqpoint{2.071414in}{0.971616in}}%
\pgfpathlineto{\pgfqpoint{2.068381in}{0.961543in}}%
\pgfpathlineto{\pgfqpoint{2.065703in}{0.958280in}}%
\pgfpathlineto{\pgfqpoint{2.083172in}{0.958841in}}%
\pgfpathlineto{\pgfqpoint{2.083454in}{0.947053in}}%
\pgfpathlineto{\pgfqpoint{2.074636in}{0.931335in}}%
\pgfpathlineto{\pgfqpoint{2.089884in}{0.925436in}}%
\pgfpathlineto{\pgfqpoint{2.090507in}{0.908195in}}%
\pgfpathlineto{\pgfqpoint{2.085014in}{0.905571in}}%
\pgfpathclose%
\pgfusepath{fill}%
\end{pgfscope}%
\begin{pgfscope}%
\pgfpathrectangle{\pgfqpoint{0.100000in}{0.100000in}}{\pgfqpoint{3.007045in}{1.925000in}}%
\pgfusepath{clip}%
\pgfsetbuttcap%
\pgfsetmiterjoin%
\definecolor{currentfill}{rgb}{0.454118,0.701300,0.846659}%
\pgfsetfillcolor{currentfill}%
\pgfsetlinewidth{0.000000pt}%
\definecolor{currentstroke}{rgb}{0.000000,0.000000,0.000000}%
\pgfsetstrokecolor{currentstroke}%
\pgfsetstrokeopacity{0.000000}%
\pgfsetdash{}{0pt}%
\pgfpathmoveto{\pgfqpoint{2.492448in}{0.960338in}}%
\pgfpathlineto{\pgfqpoint{2.498902in}{0.973883in}}%
\pgfpathlineto{\pgfqpoint{2.494999in}{0.975158in}}%
\pgfpathlineto{\pgfqpoint{2.489274in}{0.973776in}}%
\pgfpathlineto{\pgfqpoint{2.480170in}{0.978949in}}%
\pgfpathlineto{\pgfqpoint{2.474213in}{0.983871in}}%
\pgfpathlineto{\pgfqpoint{2.479657in}{0.995365in}}%
\pgfpathlineto{\pgfqpoint{2.486521in}{0.996045in}}%
\pgfpathlineto{\pgfqpoint{2.494094in}{0.994158in}}%
\pgfpathlineto{\pgfqpoint{2.499152in}{0.989884in}}%
\pgfpathlineto{\pgfqpoint{2.506001in}{0.993623in}}%
\pgfpathlineto{\pgfqpoint{2.510978in}{0.993811in}}%
\pgfpathlineto{\pgfqpoint{2.522081in}{0.997367in}}%
\pgfpathlineto{\pgfqpoint{2.532109in}{0.998638in}}%
\pgfpathlineto{\pgfqpoint{2.533363in}{0.985590in}}%
\pgfpathlineto{\pgfqpoint{2.531826in}{0.973469in}}%
\pgfpathlineto{\pgfqpoint{2.535380in}{0.962820in}}%
\pgfpathlineto{\pgfqpoint{2.523665in}{0.961730in}}%
\pgfpathlineto{\pgfqpoint{2.523136in}{0.963616in}}%
\pgfpathlineto{\pgfqpoint{2.492448in}{0.960338in}}%
\pgfpathclose%
\pgfusepath{fill}%
\end{pgfscope}%
\begin{pgfscope}%
\pgfpathrectangle{\pgfqpoint{0.100000in}{0.100000in}}{\pgfqpoint{3.007045in}{1.925000in}}%
\pgfusepath{clip}%
\pgfsetbuttcap%
\pgfsetmiterjoin%
\definecolor{currentfill}{rgb}{0.460392,0.704744,0.848012}%
\pgfsetfillcolor{currentfill}%
\pgfsetlinewidth{0.000000pt}%
\definecolor{currentstroke}{rgb}{0.000000,0.000000,0.000000}%
\pgfsetstrokecolor{currentstroke}%
\pgfsetstrokeopacity{0.000000}%
\pgfsetdash{}{0pt}%
\pgfpathmoveto{\pgfqpoint{2.350943in}{1.269110in}}%
\pgfpathlineto{\pgfqpoint{2.345898in}{1.268387in}}%
\pgfpathlineto{\pgfqpoint{2.327732in}{1.268420in}}%
\pgfpathlineto{\pgfqpoint{2.322615in}{1.309553in}}%
\pgfpathlineto{\pgfqpoint{2.345273in}{1.312326in}}%
\pgfpathlineto{\pgfqpoint{2.342797in}{1.334180in}}%
\pgfpathlineto{\pgfqpoint{2.346907in}{1.331869in}}%
\pgfpathlineto{\pgfqpoint{2.365754in}{1.334376in}}%
\pgfpathlineto{\pgfqpoint{2.371168in}{1.333253in}}%
\pgfpathlineto{\pgfqpoint{2.375866in}{1.305055in}}%
\pgfpathlineto{\pgfqpoint{2.361935in}{1.302986in}}%
\pgfpathlineto{\pgfqpoint{2.364496in}{1.284008in}}%
\pgfpathlineto{\pgfqpoint{2.349350in}{1.280970in}}%
\pgfpathlineto{\pgfqpoint{2.350943in}{1.269110in}}%
\pgfpathclose%
\pgfusepath{fill}%
\end{pgfscope}%
\begin{pgfscope}%
\pgfpathrectangle{\pgfqpoint{0.100000in}{0.100000in}}{\pgfqpoint{3.007045in}{1.925000in}}%
\pgfusepath{clip}%
\pgfsetbuttcap%
\pgfsetmiterjoin%
\definecolor{currentfill}{rgb}{0.666974,0.812288,0.898931}%
\pgfsetfillcolor{currentfill}%
\pgfsetlinewidth{0.000000pt}%
\definecolor{currentstroke}{rgb}{0.000000,0.000000,0.000000}%
\pgfsetstrokecolor{currentstroke}%
\pgfsetstrokeopacity{0.000000}%
\pgfsetdash{}{0pt}%
\pgfpathmoveto{\pgfqpoint{2.880647in}{1.472875in}}%
\pgfpathlineto{\pgfqpoint{2.836904in}{1.462520in}}%
\pgfpathlineto{\pgfqpoint{2.833508in}{1.484777in}}%
\pgfpathlineto{\pgfqpoint{2.832224in}{1.502258in}}%
\pgfpathlineto{\pgfqpoint{2.827848in}{1.505323in}}%
\pgfpathlineto{\pgfqpoint{2.831656in}{1.508759in}}%
\pgfpathlineto{\pgfqpoint{2.862374in}{1.515277in}}%
\pgfpathlineto{\pgfqpoint{2.865846in}{1.512168in}}%
\pgfpathlineto{\pgfqpoint{2.865008in}{1.506125in}}%
\pgfpathlineto{\pgfqpoint{2.867051in}{1.490726in}}%
\pgfpathlineto{\pgfqpoint{2.873175in}{1.485821in}}%
\pgfpathlineto{\pgfqpoint{2.872055in}{1.481134in}}%
\pgfpathlineto{\pgfqpoint{2.878530in}{1.481285in}}%
\pgfpathlineto{\pgfqpoint{2.880647in}{1.472875in}}%
\pgfpathclose%
\pgfusepath{fill}%
\end{pgfscope}%
\begin{pgfscope}%
\pgfpathrectangle{\pgfqpoint{0.100000in}{0.100000in}}{\pgfqpoint{3.007045in}{1.925000in}}%
\pgfusepath{clip}%
\pgfsetbuttcap%
\pgfsetmiterjoin%
\definecolor{currentfill}{rgb}{0.316201,0.611734,0.798862}%
\pgfsetfillcolor{currentfill}%
\pgfsetlinewidth{0.000000pt}%
\definecolor{currentstroke}{rgb}{0.000000,0.000000,0.000000}%
\pgfsetstrokecolor{currentstroke}%
\pgfsetstrokeopacity{0.000000}%
\pgfsetdash{}{0pt}%
\pgfpathmoveto{\pgfqpoint{1.631910in}{1.063521in}}%
\pgfpathlineto{\pgfqpoint{1.632419in}{1.080689in}}%
\pgfpathlineto{\pgfqpoint{1.621015in}{1.081045in}}%
\pgfpathlineto{\pgfqpoint{1.621825in}{1.109765in}}%
\pgfpathlineto{\pgfqpoint{1.650255in}{1.108902in}}%
\pgfpathlineto{\pgfqpoint{1.672984in}{1.108364in}}%
\pgfpathlineto{\pgfqpoint{1.672876in}{1.102642in}}%
\pgfpathlineto{\pgfqpoint{1.678559in}{1.102524in}}%
\pgfpathlineto{\pgfqpoint{1.677388in}{1.091095in}}%
\pgfpathlineto{\pgfqpoint{1.676880in}{1.073769in}}%
\pgfpathlineto{\pgfqpoint{1.660705in}{1.074262in}}%
\pgfpathlineto{\pgfqpoint{1.660467in}{1.062737in}}%
\pgfpathlineto{\pgfqpoint{1.631910in}{1.063521in}}%
\pgfpathclose%
\pgfusepath{fill}%
\end{pgfscope}%
\begin{pgfscope}%
\pgfpathrectangle{\pgfqpoint{0.100000in}{0.100000in}}{\pgfqpoint{3.007045in}{1.925000in}}%
\pgfusepath{clip}%
\pgfsetbuttcap%
\pgfsetmiterjoin%
\definecolor{currentfill}{rgb}{0.114802,0.424437,0.695194}%
\pgfsetfillcolor{currentfill}%
\pgfsetlinewidth{0.000000pt}%
\definecolor{currentstroke}{rgb}{0.000000,0.000000,0.000000}%
\pgfsetstrokecolor{currentstroke}%
\pgfsetstrokeopacity{0.000000}%
\pgfsetdash{}{0pt}%
\pgfpathmoveto{\pgfqpoint{1.449237in}{1.655533in}}%
\pgfpathlineto{\pgfqpoint{1.448770in}{1.649792in}}%
\pgfpathlineto{\pgfqpoint{1.410950in}{1.652859in}}%
\pgfpathlineto{\pgfqpoint{1.379987in}{1.655666in}}%
\pgfpathlineto{\pgfqpoint{1.382174in}{1.678842in}}%
\pgfpathlineto{\pgfqpoint{1.379578in}{1.679082in}}%
\pgfpathlineto{\pgfqpoint{1.381806in}{1.702240in}}%
\pgfpathlineto{\pgfqpoint{1.407305in}{1.699840in}}%
\pgfpathlineto{\pgfqpoint{1.409324in}{1.722778in}}%
\pgfpathlineto{\pgfqpoint{1.429757in}{1.720971in}}%
\pgfpathlineto{\pgfqpoint{1.433959in}{1.727722in}}%
\pgfpathlineto{\pgfqpoint{1.441706in}{1.725605in}}%
\pgfpathlineto{\pgfqpoint{1.447405in}{1.726710in}}%
\pgfpathlineto{\pgfqpoint{1.446373in}{1.716181in}}%
\pgfpathlineto{\pgfqpoint{1.448796in}{1.712766in}}%
\pgfpathlineto{\pgfqpoint{1.447485in}{1.696400in}}%
\pgfpathlineto{\pgfqpoint{1.450238in}{1.696191in}}%
\pgfpathlineto{\pgfqpoint{1.448415in}{1.673164in}}%
\pgfpathlineto{\pgfqpoint{1.450582in}{1.673001in}}%
\pgfpathlineto{\pgfqpoint{1.449237in}{1.655533in}}%
\pgfpathclose%
\pgfusepath{fill}%
\end{pgfscope}%
\begin{pgfscope}%
\pgfpathrectangle{\pgfqpoint{0.100000in}{0.100000in}}{\pgfqpoint{3.007045in}{1.925000in}}%
\pgfusepath{clip}%
\pgfsetbuttcap%
\pgfsetmiterjoin%
\definecolor{currentfill}{rgb}{0.412042,0.677186,0.836263}%
\pgfsetfillcolor{currentfill}%
\pgfsetlinewidth{0.000000pt}%
\definecolor{currentstroke}{rgb}{0.000000,0.000000,0.000000}%
\pgfsetstrokecolor{currentstroke}%
\pgfsetstrokeopacity{0.000000}%
\pgfsetdash{}{0pt}%
\pgfpathmoveto{\pgfqpoint{0.842950in}{0.338119in}}%
\pgfpathlineto{\pgfqpoint{0.844097in}{0.340724in}}%
\pgfpathlineto{\pgfqpoint{0.845519in}{0.341453in}}%
\pgfpathlineto{\pgfqpoint{0.848054in}{0.340022in}}%
\pgfpathlineto{\pgfqpoint{0.847662in}{0.338845in}}%
\pgfpathlineto{\pgfqpoint{0.850122in}{0.336629in}}%
\pgfpathlineto{\pgfqpoint{0.844580in}{0.336908in}}%
\pgfpathlineto{\pgfqpoint{0.842950in}{0.338119in}}%
\pgfpathclose%
\pgfusepath{fill}%
\end{pgfscope}%
\begin{pgfscope}%
\pgfpathrectangle{\pgfqpoint{0.100000in}{0.100000in}}{\pgfqpoint{3.007045in}{1.925000in}}%
\pgfusepath{clip}%
\pgfsetbuttcap%
\pgfsetmiterjoin%
\definecolor{currentfill}{rgb}{0.412042,0.677186,0.836263}%
\pgfsetfillcolor{currentfill}%
\pgfsetlinewidth{0.000000pt}%
\definecolor{currentstroke}{rgb}{0.000000,0.000000,0.000000}%
\pgfsetstrokecolor{currentstroke}%
\pgfsetstrokeopacity{0.000000}%
\pgfsetdash{}{0pt}%
\pgfpathmoveto{\pgfqpoint{0.824088in}{0.336340in}}%
\pgfpathlineto{\pgfqpoint{0.826449in}{0.337990in}}%
\pgfpathlineto{\pgfqpoint{0.828677in}{0.338112in}}%
\pgfpathlineto{\pgfqpoint{0.837136in}{0.340612in}}%
\pgfpathlineto{\pgfqpoint{0.837987in}{0.342496in}}%
\pgfpathlineto{\pgfqpoint{0.839837in}{0.341361in}}%
\pgfpathlineto{\pgfqpoint{0.839821in}{0.339455in}}%
\pgfpathlineto{\pgfqpoint{0.834944in}{0.338315in}}%
\pgfpathlineto{\pgfqpoint{0.831383in}{0.335875in}}%
\pgfpathlineto{\pgfqpoint{0.829597in}{0.336005in}}%
\pgfpathlineto{\pgfqpoint{0.828843in}{0.334684in}}%
\pgfpathlineto{\pgfqpoint{0.825115in}{0.335470in}}%
\pgfpathlineto{\pgfqpoint{0.824088in}{0.336340in}}%
\pgfpathclose%
\pgfusepath{fill}%
\end{pgfscope}%
\begin{pgfscope}%
\pgfpathrectangle{\pgfqpoint{0.100000in}{0.100000in}}{\pgfqpoint{3.007045in}{1.925000in}}%
\pgfusepath{clip}%
\pgfsetbuttcap%
\pgfsetmiterjoin%
\definecolor{currentfill}{rgb}{0.412042,0.677186,0.836263}%
\pgfsetfillcolor{currentfill}%
\pgfsetlinewidth{0.000000pt}%
\definecolor{currentstroke}{rgb}{0.000000,0.000000,0.000000}%
\pgfsetstrokecolor{currentstroke}%
\pgfsetstrokeopacity{0.000000}%
\pgfsetdash{}{0pt}%
\pgfpathmoveto{\pgfqpoint{0.829895in}{0.342488in}}%
\pgfpathlineto{\pgfqpoint{0.832685in}{0.348119in}}%
\pgfpathlineto{\pgfqpoint{0.833319in}{0.346252in}}%
\pgfpathlineto{\pgfqpoint{0.829895in}{0.342488in}}%
\pgfpathclose%
\pgfusepath{fill}%
\end{pgfscope}%
\begin{pgfscope}%
\pgfpathrectangle{\pgfqpoint{0.100000in}{0.100000in}}{\pgfqpoint{3.007045in}{1.925000in}}%
\pgfusepath{clip}%
\pgfsetbuttcap%
\pgfsetmiterjoin%
\definecolor{currentfill}{rgb}{0.412042,0.677186,0.836263}%
\pgfsetfillcolor{currentfill}%
\pgfsetlinewidth{0.000000pt}%
\definecolor{currentstroke}{rgb}{0.000000,0.000000,0.000000}%
\pgfsetstrokecolor{currentstroke}%
\pgfsetstrokeopacity{0.000000}%
\pgfsetdash{}{0pt}%
\pgfpathmoveto{\pgfqpoint{0.880633in}{0.393554in}}%
\pgfpathlineto{\pgfqpoint{0.884245in}{0.391268in}}%
\pgfpathlineto{\pgfqpoint{0.886512in}{0.394965in}}%
\pgfpathlineto{\pgfqpoint{0.895631in}{0.389258in}}%
\pgfpathlineto{\pgfqpoint{0.894498in}{0.387400in}}%
\pgfpathlineto{\pgfqpoint{0.899657in}{0.384205in}}%
\pgfpathlineto{\pgfqpoint{0.914264in}{0.375762in}}%
\pgfpathlineto{\pgfqpoint{0.912562in}{0.370845in}}%
\pgfpathlineto{\pgfqpoint{0.909515in}{0.369615in}}%
\pgfpathlineto{\pgfqpoint{0.910203in}{0.367541in}}%
\pgfpathlineto{\pgfqpoint{0.907829in}{0.364423in}}%
\pgfpathlineto{\pgfqpoint{0.910185in}{0.362649in}}%
\pgfpathlineto{\pgfqpoint{0.912886in}{0.364295in}}%
\pgfpathlineto{\pgfqpoint{0.913444in}{0.363209in}}%
\pgfpathlineto{\pgfqpoint{0.916564in}{0.361368in}}%
\pgfpathlineto{\pgfqpoint{0.915470in}{0.359574in}}%
\pgfpathlineto{\pgfqpoint{0.917368in}{0.358439in}}%
\pgfpathlineto{\pgfqpoint{0.916310in}{0.356673in}}%
\pgfpathlineto{\pgfqpoint{0.917983in}{0.355693in}}%
\pgfpathlineto{\pgfqpoint{0.913691in}{0.348399in}}%
\pgfpathlineto{\pgfqpoint{0.914753in}{0.346694in}}%
\pgfpathlineto{\pgfqpoint{0.912069in}{0.341923in}}%
\pgfpathlineto{\pgfqpoint{0.920301in}{0.337334in}}%
\pgfpathlineto{\pgfqpoint{0.911029in}{0.320607in}}%
\pgfpathlineto{\pgfqpoint{0.901489in}{0.303399in}}%
\pgfpathlineto{\pgfqpoint{0.895195in}{0.292046in}}%
\pgfpathlineto{\pgfqpoint{0.891908in}{0.297356in}}%
\pgfpathlineto{\pgfqpoint{0.895042in}{0.297561in}}%
\pgfpathlineto{\pgfqpoint{0.895865in}{0.299933in}}%
\pgfpathlineto{\pgfqpoint{0.894005in}{0.301147in}}%
\pgfpathlineto{\pgfqpoint{0.893616in}{0.299471in}}%
\pgfpathlineto{\pgfqpoint{0.889618in}{0.300158in}}%
\pgfpathlineto{\pgfqpoint{0.888432in}{0.302470in}}%
\pgfpathlineto{\pgfqpoint{0.880446in}{0.309836in}}%
\pgfpathlineto{\pgfqpoint{0.876880in}{0.310671in}}%
\pgfpathlineto{\pgfqpoint{0.872039in}{0.312853in}}%
\pgfpathlineto{\pgfqpoint{0.867319in}{0.314295in}}%
\pgfpathlineto{\pgfqpoint{0.866304in}{0.315660in}}%
\pgfpathlineto{\pgfqpoint{0.866224in}{0.318437in}}%
\pgfpathlineto{\pgfqpoint{0.865182in}{0.320626in}}%
\pgfpathlineto{\pgfqpoint{0.862537in}{0.322620in}}%
\pgfpathlineto{\pgfqpoint{0.862785in}{0.325009in}}%
\pgfpathlineto{\pgfqpoint{0.861746in}{0.326681in}}%
\pgfpathlineto{\pgfqpoint{0.864218in}{0.329401in}}%
\pgfpathlineto{\pgfqpoint{0.864631in}{0.330890in}}%
\pgfpathlineto{\pgfqpoint{0.863184in}{0.332010in}}%
\pgfpathlineto{\pgfqpoint{0.861073in}{0.330560in}}%
\pgfpathlineto{\pgfqpoint{0.858822in}{0.331645in}}%
\pgfpathlineto{\pgfqpoint{0.857771in}{0.330884in}}%
\pgfpathlineto{\pgfqpoint{0.855435in}{0.335097in}}%
\pgfpathlineto{\pgfqpoint{0.853673in}{0.335957in}}%
\pgfpathlineto{\pgfqpoint{0.855410in}{0.337624in}}%
\pgfpathlineto{\pgfqpoint{0.855863in}{0.341580in}}%
\pgfpathlineto{\pgfqpoint{0.855648in}{0.344279in}}%
\pgfpathlineto{\pgfqpoint{0.852144in}{0.345090in}}%
\pgfpathlineto{\pgfqpoint{0.853807in}{0.346777in}}%
\pgfpathlineto{\pgfqpoint{0.850106in}{0.348448in}}%
\pgfpathlineto{\pgfqpoint{0.850605in}{0.352065in}}%
\pgfpathlineto{\pgfqpoint{0.848839in}{0.353027in}}%
\pgfpathlineto{\pgfqpoint{0.848400in}{0.354633in}}%
\pgfpathlineto{\pgfqpoint{0.846238in}{0.354335in}}%
\pgfpathlineto{\pgfqpoint{0.845449in}{0.352516in}}%
\pgfpathlineto{\pgfqpoint{0.842638in}{0.355946in}}%
\pgfpathlineto{\pgfqpoint{0.843920in}{0.358581in}}%
\pgfpathlineto{\pgfqpoint{0.839850in}{0.357610in}}%
\pgfpathlineto{\pgfqpoint{0.835041in}{0.357359in}}%
\pgfpathlineto{\pgfqpoint{0.836568in}{0.359811in}}%
\pgfpathlineto{\pgfqpoint{0.838051in}{0.358886in}}%
\pgfpathlineto{\pgfqpoint{0.838232in}{0.362073in}}%
\pgfpathlineto{\pgfqpoint{0.836781in}{0.362125in}}%
\pgfpathlineto{\pgfqpoint{0.832732in}{0.359491in}}%
\pgfpathlineto{\pgfqpoint{0.829761in}{0.356685in}}%
\pgfpathlineto{\pgfqpoint{0.830722in}{0.356227in}}%
\pgfpathlineto{\pgfqpoint{0.832645in}{0.357725in}}%
\pgfpathlineto{\pgfqpoint{0.834243in}{0.356277in}}%
\pgfpathlineto{\pgfqpoint{0.832157in}{0.354595in}}%
\pgfpathlineto{\pgfqpoint{0.830411in}{0.354246in}}%
\pgfpathlineto{\pgfqpoint{0.827722in}{0.355289in}}%
\pgfpathlineto{\pgfqpoint{0.828994in}{0.351862in}}%
\pgfpathlineto{\pgfqpoint{0.830872in}{0.350681in}}%
\pgfpathlineto{\pgfqpoint{0.825792in}{0.348726in}}%
\pgfpathlineto{\pgfqpoint{0.825045in}{0.347737in}}%
\pgfpathlineto{\pgfqpoint{0.828029in}{0.344167in}}%
\pgfpathlineto{\pgfqpoint{0.827087in}{0.340105in}}%
\pgfpathlineto{\pgfqpoint{0.825414in}{0.340100in}}%
\pgfpathlineto{\pgfqpoint{0.823966in}{0.342216in}}%
\pgfpathlineto{\pgfqpoint{0.824700in}{0.344679in}}%
\pgfpathlineto{\pgfqpoint{0.823443in}{0.345830in}}%
\pgfpathlineto{\pgfqpoint{0.821225in}{0.343325in}}%
\pgfpathlineto{\pgfqpoint{0.819433in}{0.344425in}}%
\pgfpathlineto{\pgfqpoint{0.821535in}{0.347132in}}%
\pgfpathlineto{\pgfqpoint{0.825526in}{0.353057in}}%
\pgfpathlineto{\pgfqpoint{0.824494in}{0.353761in}}%
\pgfpathlineto{\pgfqpoint{0.828821in}{0.359977in}}%
\pgfpathlineto{\pgfqpoint{0.827922in}{0.360561in}}%
\pgfpathlineto{\pgfqpoint{0.829442in}{0.362834in}}%
\pgfpathlineto{\pgfqpoint{0.832191in}{0.361126in}}%
\pgfpathlineto{\pgfqpoint{0.836391in}{0.367548in}}%
\pgfpathlineto{\pgfqpoint{0.840140in}{0.372751in}}%
\pgfpathlineto{\pgfqpoint{0.852394in}{0.364648in}}%
\pgfpathlineto{\pgfqpoint{0.853061in}{0.365671in}}%
\pgfpathlineto{\pgfqpoint{0.855608in}{0.363995in}}%
\pgfpathlineto{\pgfqpoint{0.864498in}{0.378218in}}%
\pgfpathlineto{\pgfqpoint{0.865448in}{0.380135in}}%
\pgfpathlineto{\pgfqpoint{0.870796in}{0.376637in}}%
\pgfpathlineto{\pgfqpoint{0.878779in}{0.389457in}}%
\pgfpathlineto{\pgfqpoint{0.880633in}{0.393554in}}%
\pgfpathclose%
\pgfusepath{fill}%
\end{pgfscope}%
\begin{pgfscope}%
\pgfpathrectangle{\pgfqpoint{0.100000in}{0.100000in}}{\pgfqpoint{3.007045in}{1.925000in}}%
\pgfusepath{clip}%
\pgfsetbuttcap%
\pgfsetmiterjoin%
\definecolor{currentfill}{rgb}{0.523137,0.739193,0.861546}%
\pgfsetfillcolor{currentfill}%
\pgfsetlinewidth{0.000000pt}%
\definecolor{currentstroke}{rgb}{0.000000,0.000000,0.000000}%
\pgfsetstrokecolor{currentstroke}%
\pgfsetstrokeopacity{0.000000}%
\pgfsetdash{}{0pt}%
\pgfpathmoveto{\pgfqpoint{2.453185in}{1.084073in}}%
\pgfpathlineto{\pgfqpoint{2.456556in}{1.087012in}}%
\pgfpathlineto{\pgfqpoint{2.458232in}{1.094688in}}%
\pgfpathlineto{\pgfqpoint{2.463061in}{1.098120in}}%
\pgfpathlineto{\pgfqpoint{2.459342in}{1.100923in}}%
\pgfpathlineto{\pgfqpoint{2.471855in}{1.105407in}}%
\pgfpathlineto{\pgfqpoint{2.476705in}{1.103396in}}%
\pgfpathlineto{\pgfqpoint{2.484474in}{1.093009in}}%
\pgfpathlineto{\pgfqpoint{2.489686in}{1.089390in}}%
\pgfpathlineto{\pgfqpoint{2.487921in}{1.088350in}}%
\pgfpathlineto{\pgfqpoint{2.483702in}{1.075698in}}%
\pgfpathlineto{\pgfqpoint{2.475445in}{1.067081in}}%
\pgfpathlineto{\pgfqpoint{2.469086in}{1.065710in}}%
\pgfpathlineto{\pgfqpoint{2.462901in}{1.070185in}}%
\pgfpathlineto{\pgfqpoint{2.455254in}{1.073522in}}%
\pgfpathlineto{\pgfqpoint{2.452938in}{1.079250in}}%
\pgfpathlineto{\pgfqpoint{2.453185in}{1.084073in}}%
\pgfpathclose%
\pgfusepath{fill}%
\end{pgfscope}%
\begin{pgfscope}%
\pgfpathrectangle{\pgfqpoint{0.100000in}{0.100000in}}{\pgfqpoint{3.007045in}{1.925000in}}%
\pgfusepath{clip}%
\pgfsetbuttcap%
\pgfsetmiterjoin%
\definecolor{currentfill}{rgb}{0.535686,0.746082,0.864252}%
\pgfsetfillcolor{currentfill}%
\pgfsetlinewidth{0.000000pt}%
\definecolor{currentstroke}{rgb}{0.000000,0.000000,0.000000}%
\pgfsetstrokecolor{currentstroke}%
\pgfsetstrokeopacity{0.000000}%
\pgfsetdash{}{0pt}%
\pgfpathmoveto{\pgfqpoint{2.533154in}{0.479245in}}%
\pgfpathlineto{\pgfqpoint{2.531241in}{0.493155in}}%
\pgfpathlineto{\pgfqpoint{2.521853in}{0.495228in}}%
\pgfpathlineto{\pgfqpoint{2.517736in}{0.501110in}}%
\pgfpathlineto{\pgfqpoint{2.522009in}{0.510090in}}%
\pgfpathlineto{\pgfqpoint{2.512188in}{0.520102in}}%
\pgfpathlineto{\pgfqpoint{2.549990in}{0.525560in}}%
\pgfpathlineto{\pgfqpoint{2.550293in}{0.531506in}}%
\pgfpathlineto{\pgfqpoint{2.547864in}{0.546632in}}%
\pgfpathlineto{\pgfqpoint{2.550313in}{0.542005in}}%
\pgfpathlineto{\pgfqpoint{2.555178in}{0.540975in}}%
\pgfpathlineto{\pgfqpoint{2.557842in}{0.535907in}}%
\pgfpathlineto{\pgfqpoint{2.567244in}{0.529939in}}%
\pgfpathlineto{\pgfqpoint{2.570842in}{0.520187in}}%
\pgfpathlineto{\pgfqpoint{2.576314in}{0.520862in}}%
\pgfpathlineto{\pgfqpoint{2.581927in}{0.518998in}}%
\pgfpathlineto{\pgfqpoint{2.584158in}{0.521537in}}%
\pgfpathlineto{\pgfqpoint{2.592396in}{0.508481in}}%
\pgfpathlineto{\pgfqpoint{2.599396in}{0.502567in}}%
\pgfpathlineto{\pgfqpoint{2.599254in}{0.497364in}}%
\pgfpathlineto{\pgfqpoint{2.602369in}{0.492044in}}%
\pgfpathlineto{\pgfqpoint{2.609094in}{0.445229in}}%
\pgfpathlineto{\pgfqpoint{2.593275in}{0.442838in}}%
\pgfpathlineto{\pgfqpoint{2.592017in}{0.447697in}}%
\pgfpathlineto{\pgfqpoint{2.587084in}{0.454555in}}%
\pgfpathlineto{\pgfqpoint{2.581515in}{0.455951in}}%
\pgfpathlineto{\pgfqpoint{2.580449in}{0.459760in}}%
\pgfpathlineto{\pgfqpoint{2.575198in}{0.465448in}}%
\pgfpathlineto{\pgfqpoint{2.565859in}{0.472579in}}%
\pgfpathlineto{\pgfqpoint{2.562765in}{0.480015in}}%
\pgfpathlineto{\pgfqpoint{2.556944in}{0.479141in}}%
\pgfpathlineto{\pgfqpoint{2.556078in}{0.484967in}}%
\pgfpathlineto{\pgfqpoint{2.538576in}{0.482276in}}%
\pgfpathlineto{\pgfqpoint{2.533154in}{0.479245in}}%
\pgfpathclose%
\pgfusepath{fill}%
\end{pgfscope}%
\begin{pgfscope}%
\pgfpathrectangle{\pgfqpoint{0.100000in}{0.100000in}}{\pgfqpoint{3.007045in}{1.925000in}}%
\pgfusepath{clip}%
\pgfsetbuttcap%
\pgfsetmiterjoin%
\definecolor{currentfill}{rgb}{0.154787,0.468512,0.722876}%
\pgfsetfillcolor{currentfill}%
\pgfsetlinewidth{0.000000pt}%
\definecolor{currentstroke}{rgb}{0.000000,0.000000,0.000000}%
\pgfsetstrokecolor{currentstroke}%
\pgfsetstrokeopacity{0.000000}%
\pgfsetdash{}{0pt}%
\pgfpathmoveto{\pgfqpoint{1.453227in}{1.011793in}}%
\pgfpathlineto{\pgfqpoint{1.459593in}{1.011387in}}%
\pgfpathlineto{\pgfqpoint{1.456990in}{0.978623in}}%
\pgfpathlineto{\pgfqpoint{1.421577in}{0.981072in}}%
\pgfpathlineto{\pgfqpoint{1.419469in}{0.951860in}}%
\pgfpathlineto{\pgfqpoint{1.390781in}{0.954030in}}%
\pgfpathlineto{\pgfqpoint{1.393120in}{0.983287in}}%
\pgfpathlineto{\pgfqpoint{1.348759in}{0.987002in}}%
\pgfpathlineto{\pgfqpoint{1.351668in}{1.019814in}}%
\pgfpathlineto{\pgfqpoint{1.367585in}{1.018108in}}%
\pgfpathlineto{\pgfqpoint{1.401987in}{1.015175in}}%
\pgfpathlineto{\pgfqpoint{1.404039in}{1.041199in}}%
\pgfpathlineto{\pgfqpoint{1.429358in}{1.039255in}}%
\pgfpathlineto{\pgfqpoint{1.427563in}{1.013412in}}%
\pgfpathlineto{\pgfqpoint{1.453227in}{1.011793in}}%
\pgfpathclose%
\pgfusepath{fill}%
\end{pgfscope}%
\begin{pgfscope}%
\pgfpathrectangle{\pgfqpoint{0.100000in}{0.100000in}}{\pgfqpoint{3.007045in}{1.925000in}}%
\pgfusepath{clip}%
\pgfsetbuttcap%
\pgfsetmiterjoin%
\definecolor{currentfill}{rgb}{0.290980,0.594510,0.789020}%
\pgfsetfillcolor{currentfill}%
\pgfsetlinewidth{0.000000pt}%
\definecolor{currentstroke}{rgb}{0.000000,0.000000,0.000000}%
\pgfsetstrokecolor{currentstroke}%
\pgfsetstrokeopacity{0.000000}%
\pgfsetdash{}{0pt}%
\pgfpathmoveto{\pgfqpoint{2.218727in}{1.043400in}}%
\pgfpathlineto{\pgfqpoint{2.223300in}{1.033829in}}%
\pgfpathlineto{\pgfqpoint{2.216520in}{1.033633in}}%
\pgfpathlineto{\pgfqpoint{2.212201in}{1.031659in}}%
\pgfpathlineto{\pgfqpoint{2.209941in}{1.021313in}}%
\pgfpathlineto{\pgfqpoint{2.200660in}{1.024227in}}%
\pgfpathlineto{\pgfqpoint{2.189732in}{1.023554in}}%
\pgfpathlineto{\pgfqpoint{2.179536in}{1.021312in}}%
\pgfpathlineto{\pgfqpoint{2.174509in}{1.028592in}}%
\pgfpathlineto{\pgfqpoint{2.171148in}{1.034806in}}%
\pgfpathlineto{\pgfqpoint{2.174985in}{1.043028in}}%
\pgfpathlineto{\pgfqpoint{2.172018in}{1.046217in}}%
\pgfpathlineto{\pgfqpoint{2.170940in}{1.054187in}}%
\pgfpathlineto{\pgfqpoint{2.164342in}{1.057050in}}%
\pgfpathlineto{\pgfqpoint{2.171424in}{1.072069in}}%
\pgfpathlineto{\pgfqpoint{2.172010in}{1.077414in}}%
\pgfpathlineto{\pgfqpoint{2.184734in}{1.072536in}}%
\pgfpathlineto{\pgfqpoint{2.185322in}{1.077996in}}%
\pgfpathlineto{\pgfqpoint{2.197741in}{1.085673in}}%
\pgfpathlineto{\pgfqpoint{2.200736in}{1.080556in}}%
\pgfpathlineto{\pgfqpoint{2.205878in}{1.081086in}}%
\pgfpathlineto{\pgfqpoint{2.205852in}{1.076709in}}%
\pgfpathlineto{\pgfqpoint{2.204976in}{1.073343in}}%
\pgfpathlineto{\pgfqpoint{2.208703in}{1.064571in}}%
\pgfpathlineto{\pgfqpoint{2.207346in}{1.056620in}}%
\pgfpathlineto{\pgfqpoint{2.210809in}{1.047577in}}%
\pgfpathlineto{\pgfqpoint{2.217524in}{1.046026in}}%
\pgfpathlineto{\pgfqpoint{2.218727in}{1.043400in}}%
\pgfpathclose%
\pgfusepath{fill}%
\end{pgfscope}%
\begin{pgfscope}%
\pgfpathrectangle{\pgfqpoint{0.100000in}{0.100000in}}{\pgfqpoint{3.007045in}{1.925000in}}%
\pgfusepath{clip}%
\pgfsetbuttcap%
\pgfsetmiterjoin%
\definecolor{currentfill}{rgb}{0.381776,0.656517,0.824452}%
\pgfsetfillcolor{currentfill}%
\pgfsetlinewidth{0.000000pt}%
\definecolor{currentstroke}{rgb}{0.000000,0.000000,0.000000}%
\pgfsetstrokecolor{currentstroke}%
\pgfsetstrokeopacity{0.000000}%
\pgfsetdash{}{0pt}%
\pgfpathmoveto{\pgfqpoint{1.865473in}{1.453759in}}%
\pgfpathlineto{\pgfqpoint{1.882427in}{1.454160in}}%
\pgfpathlineto{\pgfqpoint{1.893937in}{1.453516in}}%
\pgfpathlineto{\pgfqpoint{1.894560in}{1.431480in}}%
\pgfpathlineto{\pgfqpoint{1.866993in}{1.430735in}}%
\pgfpathlineto{\pgfqpoint{1.837085in}{1.430186in}}%
\pgfpathlineto{\pgfqpoint{1.836747in}{1.453219in}}%
\pgfpathlineto{\pgfqpoint{1.865473in}{1.453759in}}%
\pgfpathclose%
\pgfusepath{fill}%
\end{pgfscope}%
\begin{pgfscope}%
\pgfpathrectangle{\pgfqpoint{0.100000in}{0.100000in}}{\pgfqpoint{3.007045in}{1.925000in}}%
\pgfusepath{clip}%
\pgfsetbuttcap%
\pgfsetmiterjoin%
\definecolor{currentfill}{rgb}{0.223806,0.537532,0.758431}%
\pgfsetfillcolor{currentfill}%
\pgfsetlinewidth{0.000000pt}%
\definecolor{currentstroke}{rgb}{0.000000,0.000000,0.000000}%
\pgfsetstrokecolor{currentstroke}%
\pgfsetstrokeopacity{0.000000}%
\pgfsetdash{}{0pt}%
\pgfpathmoveto{\pgfqpoint{1.674656in}{1.382036in}}%
\pgfpathlineto{\pgfqpoint{1.667711in}{1.384730in}}%
\pgfpathlineto{\pgfqpoint{1.668294in}{1.409172in}}%
\pgfpathlineto{\pgfqpoint{1.656789in}{1.409421in}}%
\pgfpathlineto{\pgfqpoint{1.657257in}{1.431226in}}%
\pgfpathlineto{\pgfqpoint{1.647409in}{1.431503in}}%
\pgfpathlineto{\pgfqpoint{1.648008in}{1.454576in}}%
\pgfpathlineto{\pgfqpoint{1.682306in}{1.453680in}}%
\pgfpathlineto{\pgfqpoint{1.703058in}{1.453389in}}%
\pgfpathlineto{\pgfqpoint{1.702699in}{1.430317in}}%
\pgfpathlineto{\pgfqpoint{1.695709in}{1.430437in}}%
\pgfpathlineto{\pgfqpoint{1.695898in}{1.425999in}}%
\pgfpathlineto{\pgfqpoint{1.699167in}{1.423436in}}%
\pgfpathlineto{\pgfqpoint{1.698770in}{1.417142in}}%
\pgfpathlineto{\pgfqpoint{1.701304in}{1.411915in}}%
\pgfpathlineto{\pgfqpoint{1.702210in}{1.402774in}}%
\pgfpathlineto{\pgfqpoint{1.685257in}{1.403080in}}%
\pgfpathlineto{\pgfqpoint{1.684732in}{1.378050in}}%
\pgfpathlineto{\pgfqpoint{1.674656in}{1.382036in}}%
\pgfpathclose%
\pgfusepath{fill}%
\end{pgfscope}%
\begin{pgfscope}%
\pgfpathrectangle{\pgfqpoint{0.100000in}{0.100000in}}{\pgfqpoint{3.007045in}{1.925000in}}%
\pgfusepath{clip}%
\pgfsetbuttcap%
\pgfsetmiterjoin%
\definecolor{currentfill}{rgb}{0.371688,0.649627,0.820515}%
\pgfsetfillcolor{currentfill}%
\pgfsetlinewidth{0.000000pt}%
\definecolor{currentstroke}{rgb}{0.000000,0.000000,0.000000}%
\pgfsetstrokecolor{currentstroke}%
\pgfsetstrokeopacity{0.000000}%
\pgfsetdash{}{0pt}%
\pgfpathmoveto{\pgfqpoint{1.698475in}{1.199567in}}%
\pgfpathlineto{\pgfqpoint{1.709799in}{1.199381in}}%
\pgfpathlineto{\pgfqpoint{1.709420in}{1.170767in}}%
\pgfpathlineto{\pgfqpoint{1.652076in}{1.171950in}}%
\pgfpathlineto{\pgfqpoint{1.652796in}{1.200602in}}%
\pgfpathlineto{\pgfqpoint{1.698475in}{1.199567in}}%
\pgfpathclose%
\pgfusepath{fill}%
\end{pgfscope}%
\begin{pgfscope}%
\pgfpathrectangle{\pgfqpoint{0.100000in}{0.100000in}}{\pgfqpoint{3.007045in}{1.925000in}}%
\pgfusepath{clip}%
\pgfsetbuttcap%
\pgfsetmiterjoin%
\definecolor{currentfill}{rgb}{0.523137,0.739193,0.861546}%
\pgfsetfillcolor{currentfill}%
\pgfsetlinewidth{0.000000pt}%
\definecolor{currentstroke}{rgb}{0.000000,0.000000,0.000000}%
\pgfsetstrokecolor{currentstroke}%
\pgfsetstrokeopacity{0.000000}%
\pgfsetdash{}{0pt}%
\pgfpathmoveto{\pgfqpoint{2.013813in}{1.217404in}}%
\pgfpathlineto{\pgfqpoint{2.001133in}{1.217077in}}%
\pgfpathlineto{\pgfqpoint{2.000904in}{1.222841in}}%
\pgfpathlineto{\pgfqpoint{1.977807in}{1.222266in}}%
\pgfpathlineto{\pgfqpoint{1.977034in}{1.245635in}}%
\pgfpathlineto{\pgfqpoint{1.982776in}{1.245666in}}%
\pgfpathlineto{\pgfqpoint{1.981668in}{1.274180in}}%
\pgfpathlineto{\pgfqpoint{1.998849in}{1.274698in}}%
\pgfpathlineto{\pgfqpoint{1.998647in}{1.280445in}}%
\pgfpathlineto{\pgfqpoint{2.021108in}{1.281507in}}%
\pgfpathlineto{\pgfqpoint{2.021691in}{1.270010in}}%
\pgfpathlineto{\pgfqpoint{2.022785in}{1.247040in}}%
\pgfpathlineto{\pgfqpoint{2.028561in}{1.247285in}}%
\pgfpathlineto{\pgfqpoint{2.029027in}{1.239956in}}%
\pgfpathlineto{\pgfqpoint{2.027473in}{1.235391in}}%
\pgfpathlineto{\pgfqpoint{2.021248in}{1.230411in}}%
\pgfpathlineto{\pgfqpoint{2.017601in}{1.220842in}}%
\pgfpathlineto{\pgfqpoint{2.013813in}{1.217404in}}%
\pgfpathclose%
\pgfusepath{fill}%
\end{pgfscope}%
\begin{pgfscope}%
\pgfpathrectangle{\pgfqpoint{0.100000in}{0.100000in}}{\pgfqpoint{3.007045in}{1.925000in}}%
\pgfusepath{clip}%
\pgfsetbuttcap%
\pgfsetmiterjoin%
\definecolor{currentfill}{rgb}{0.326290,0.618624,0.802799}%
\pgfsetfillcolor{currentfill}%
\pgfsetlinewidth{0.000000pt}%
\definecolor{currentstroke}{rgb}{0.000000,0.000000,0.000000}%
\pgfsetstrokecolor{currentstroke}%
\pgfsetstrokeopacity{0.000000}%
\pgfsetdash{}{0pt}%
\pgfpathmoveto{\pgfqpoint{2.291623in}{1.065687in}}%
\pgfpathlineto{\pgfqpoint{2.292317in}{1.056941in}}%
\pgfpathlineto{\pgfqpoint{2.285344in}{1.059690in}}%
\pgfpathlineto{\pgfqpoint{2.281792in}{1.056352in}}%
\pgfpathlineto{\pgfqpoint{2.276076in}{1.059145in}}%
\pgfpathlineto{\pgfqpoint{2.270021in}{1.058081in}}%
\pgfpathlineto{\pgfqpoint{2.267818in}{1.063028in}}%
\pgfpathlineto{\pgfqpoint{2.262113in}{1.062729in}}%
\pgfpathlineto{\pgfqpoint{2.261909in}{1.072347in}}%
\pgfpathlineto{\pgfqpoint{2.259515in}{1.072291in}}%
\pgfpathlineto{\pgfqpoint{2.253561in}{1.079316in}}%
\pgfpathlineto{\pgfqpoint{2.258362in}{1.081298in}}%
\pgfpathlineto{\pgfqpoint{2.265212in}{1.092380in}}%
\pgfpathlineto{\pgfqpoint{2.270164in}{1.089523in}}%
\pgfpathlineto{\pgfqpoint{2.281936in}{1.092901in}}%
\pgfpathlineto{\pgfqpoint{2.283241in}{1.088142in}}%
\pgfpathlineto{\pgfqpoint{2.289598in}{1.088435in}}%
\pgfpathlineto{\pgfqpoint{2.291416in}{1.086166in}}%
\pgfpathlineto{\pgfqpoint{2.291623in}{1.065687in}}%
\pgfpathclose%
\pgfusepath{fill}%
\end{pgfscope}%
\begin{pgfscope}%
\pgfpathrectangle{\pgfqpoint{0.100000in}{0.100000in}}{\pgfqpoint{3.007045in}{1.925000in}}%
\pgfusepath{clip}%
\pgfsetbuttcap%
\pgfsetmiterjoin%
\definecolor{currentfill}{rgb}{0.535686,0.746082,0.864252}%
\pgfsetfillcolor{currentfill}%
\pgfsetlinewidth{0.000000pt}%
\definecolor{currentstroke}{rgb}{0.000000,0.000000,0.000000}%
\pgfsetstrokecolor{currentstroke}%
\pgfsetstrokeopacity{0.000000}%
\pgfsetdash{}{0pt}%
\pgfpathmoveto{\pgfqpoint{2.406992in}{0.858796in}}%
\pgfpathlineto{\pgfqpoint{2.409663in}{0.861811in}}%
\pgfpathlineto{\pgfqpoint{2.422010in}{0.860947in}}%
\pgfpathlineto{\pgfqpoint{2.421120in}{0.857223in}}%
\pgfpathlineto{\pgfqpoint{2.431219in}{0.848011in}}%
\pgfpathlineto{\pgfqpoint{2.440442in}{0.846333in}}%
\pgfpathlineto{\pgfqpoint{2.435742in}{0.842463in}}%
\pgfpathlineto{\pgfqpoint{2.430794in}{0.832351in}}%
\pgfpathlineto{\pgfqpoint{2.430059in}{0.826283in}}%
\pgfpathlineto{\pgfqpoint{2.424340in}{0.826019in}}%
\pgfpathlineto{\pgfqpoint{2.414961in}{0.829005in}}%
\pgfpathlineto{\pgfqpoint{2.408663in}{0.823968in}}%
\pgfpathlineto{\pgfqpoint{2.403550in}{0.830476in}}%
\pgfpathlineto{\pgfqpoint{2.392091in}{0.836488in}}%
\pgfpathlineto{\pgfqpoint{2.387448in}{0.834640in}}%
\pgfpathlineto{\pgfqpoint{2.380905in}{0.840878in}}%
\pgfpathlineto{\pgfqpoint{2.382703in}{0.849302in}}%
\pgfpathlineto{\pgfqpoint{2.386057in}{0.845572in}}%
\pgfpathlineto{\pgfqpoint{2.397316in}{0.844754in}}%
\pgfpathlineto{\pgfqpoint{2.399267in}{0.840607in}}%
\pgfpathlineto{\pgfqpoint{2.406734in}{0.847180in}}%
\pgfpathlineto{\pgfqpoint{2.407803in}{0.851715in}}%
\pgfpathlineto{\pgfqpoint{2.404649in}{0.856728in}}%
\pgfpathlineto{\pgfqpoint{2.406992in}{0.858796in}}%
\pgfpathclose%
\pgfusepath{fill}%
\end{pgfscope}%
\begin{pgfscope}%
\pgfpathrectangle{\pgfqpoint{0.100000in}{0.100000in}}{\pgfqpoint{3.007045in}{1.925000in}}%
\pgfusepath{clip}%
\pgfsetbuttcap%
\pgfsetmiterjoin%
\definecolor{currentfill}{rgb}{0.376732,0.653072,0.822484}%
\pgfsetfillcolor{currentfill}%
\pgfsetlinewidth{0.000000pt}%
\definecolor{currentstroke}{rgb}{0.000000,0.000000,0.000000}%
\pgfsetstrokecolor{currentstroke}%
\pgfsetstrokeopacity{0.000000}%
\pgfsetdash{}{0pt}%
\pgfpathmoveto{\pgfqpoint{1.565129in}{1.341345in}}%
\pgfpathlineto{\pgfqpoint{1.543010in}{1.342313in}}%
\pgfpathlineto{\pgfqpoint{1.519516in}{1.343574in}}%
\pgfpathlineto{\pgfqpoint{1.520062in}{1.366510in}}%
\pgfpathlineto{\pgfqpoint{1.521365in}{1.389410in}}%
\pgfpathlineto{\pgfqpoint{1.521452in}{1.403789in}}%
\pgfpathlineto{\pgfqpoint{1.566978in}{1.401393in}}%
\pgfpathlineto{\pgfqpoint{1.565129in}{1.341345in}}%
\pgfpathclose%
\pgfusepath{fill}%
\end{pgfscope}%
\begin{pgfscope}%
\pgfpathrectangle{\pgfqpoint{0.100000in}{0.100000in}}{\pgfqpoint{3.007045in}{1.925000in}}%
\pgfusepath{clip}%
\pgfsetbuttcap%
\pgfsetmiterjoin%
\definecolor{currentfill}{rgb}{0.162907,0.476632,0.727059}%
\pgfsetfillcolor{currentfill}%
\pgfsetlinewidth{0.000000pt}%
\definecolor{currentstroke}{rgb}{0.000000,0.000000,0.000000}%
\pgfsetstrokecolor{currentstroke}%
\pgfsetstrokeopacity{0.000000}%
\pgfsetdash{}{0pt}%
\pgfpathmoveto{\pgfqpoint{1.716390in}{1.322247in}}%
\pgfpathlineto{\pgfqpoint{1.719990in}{1.317507in}}%
\pgfpathlineto{\pgfqpoint{1.717094in}{1.310143in}}%
\pgfpathlineto{\pgfqpoint{1.701191in}{1.310424in}}%
\pgfpathlineto{\pgfqpoint{1.701276in}{1.314251in}}%
\pgfpathlineto{\pgfqpoint{1.678558in}{1.314735in}}%
\pgfpathlineto{\pgfqpoint{1.672961in}{1.314877in}}%
\pgfpathlineto{\pgfqpoint{1.673497in}{1.337786in}}%
\pgfpathlineto{\pgfqpoint{1.683057in}{1.337584in}}%
\pgfpathlineto{\pgfqpoint{1.683358in}{1.349049in}}%
\pgfpathlineto{\pgfqpoint{1.688055in}{1.349861in}}%
\pgfpathlineto{\pgfqpoint{1.706125in}{1.349454in}}%
\pgfpathlineto{\pgfqpoint{1.706612in}{1.341900in}}%
\pgfpathlineto{\pgfqpoint{1.713888in}{1.330082in}}%
\pgfpathlineto{\pgfqpoint{1.716907in}{1.329278in}}%
\pgfpathlineto{\pgfqpoint{1.716390in}{1.322247in}}%
\pgfpathclose%
\pgfusepath{fill}%
\end{pgfscope}%
\begin{pgfscope}%
\pgfpathrectangle{\pgfqpoint{0.100000in}{0.100000in}}{\pgfqpoint{3.007045in}{1.925000in}}%
\pgfusepath{clip}%
\pgfsetbuttcap%
\pgfsetmiterjoin%
\definecolor{currentfill}{rgb}{0.351511,0.635848,0.812641}%
\pgfsetfillcolor{currentfill}%
\pgfsetlinewidth{0.000000pt}%
\definecolor{currentstroke}{rgb}{0.000000,0.000000,0.000000}%
\pgfsetstrokecolor{currentstroke}%
\pgfsetstrokeopacity{0.000000}%
\pgfsetdash{}{0pt}%
\pgfpathmoveto{\pgfqpoint{1.898889in}{1.500432in}}%
\pgfpathlineto{\pgfqpoint{1.889296in}{1.502422in}}%
\pgfpathlineto{\pgfqpoint{1.884271in}{1.505208in}}%
\pgfpathlineto{\pgfqpoint{1.876275in}{1.505700in}}%
\pgfpathlineto{\pgfqpoint{1.876437in}{1.499997in}}%
\pgfpathlineto{\pgfqpoint{1.864831in}{1.497891in}}%
\pgfpathlineto{\pgfqpoint{1.864898in}{1.494975in}}%
\pgfpathlineto{\pgfqpoint{1.853444in}{1.494748in}}%
\pgfpathlineto{\pgfqpoint{1.853375in}{1.499519in}}%
\pgfpathlineto{\pgfqpoint{1.841921in}{1.499291in}}%
\pgfpathlineto{\pgfqpoint{1.823694in}{1.499014in}}%
\pgfpathlineto{\pgfqpoint{1.824600in}{1.510542in}}%
\pgfpathlineto{\pgfqpoint{1.818879in}{1.510489in}}%
\pgfpathlineto{\pgfqpoint{1.818599in}{1.527791in}}%
\pgfpathlineto{\pgfqpoint{1.807185in}{1.527732in}}%
\pgfpathlineto{\pgfqpoint{1.806790in}{1.547902in}}%
\pgfpathlineto{\pgfqpoint{1.813546in}{1.550942in}}%
\pgfpathlineto{\pgfqpoint{1.816671in}{1.557157in}}%
\pgfpathlineto{\pgfqpoint{1.830410in}{1.548866in}}%
\pgfpathlineto{\pgfqpoint{1.838457in}{1.548803in}}%
\pgfpathlineto{\pgfqpoint{1.841735in}{1.545791in}}%
\pgfpathlineto{\pgfqpoint{1.841127in}{1.578186in}}%
\pgfpathlineto{\pgfqpoint{1.872272in}{1.578521in}}%
\pgfpathlineto{\pgfqpoint{1.870534in}{1.568228in}}%
\pgfpathlineto{\pgfqpoint{1.875816in}{1.567788in}}%
\pgfpathlineto{\pgfqpoint{1.881731in}{1.559757in}}%
\pgfpathlineto{\pgfqpoint{1.881663in}{1.556766in}}%
\pgfpathlineto{\pgfqpoint{1.876801in}{1.549513in}}%
\pgfpathlineto{\pgfqpoint{1.876967in}{1.544122in}}%
\pgfpathlineto{\pgfqpoint{1.905032in}{1.544903in}}%
\pgfpathlineto{\pgfqpoint{1.906167in}{1.539095in}}%
\pgfpathlineto{\pgfqpoint{1.907419in}{1.500598in}}%
\pgfpathlineto{\pgfqpoint{1.898889in}{1.500432in}}%
\pgfpathclose%
\pgfusepath{fill}%
\end{pgfscope}%
\begin{pgfscope}%
\pgfpathrectangle{\pgfqpoint{0.100000in}{0.100000in}}{\pgfqpoint{3.007045in}{1.925000in}}%
\pgfusepath{clip}%
\pgfsetbuttcap%
\pgfsetmiterjoin%
\definecolor{currentfill}{rgb}{0.326290,0.618624,0.802799}%
\pgfsetfillcolor{currentfill}%
\pgfsetlinewidth{0.000000pt}%
\definecolor{currentstroke}{rgb}{0.000000,0.000000,0.000000}%
\pgfsetstrokecolor{currentstroke}%
\pgfsetstrokeopacity{0.000000}%
\pgfsetdash{}{0pt}%
\pgfpathmoveto{\pgfqpoint{2.037435in}{0.611183in}}%
\pgfpathlineto{\pgfqpoint{2.043709in}{0.592924in}}%
\pgfpathlineto{\pgfqpoint{2.046065in}{0.559905in}}%
\pgfpathlineto{\pgfqpoint{2.042685in}{0.564352in}}%
\pgfpathlineto{\pgfqpoint{2.032903in}{0.562233in}}%
\pgfpathlineto{\pgfqpoint{2.029617in}{0.558203in}}%
\pgfpathlineto{\pgfqpoint{2.023323in}{0.554899in}}%
\pgfpathlineto{\pgfqpoint{2.007733in}{0.550626in}}%
\pgfpathlineto{\pgfqpoint{2.005096in}{0.547382in}}%
\pgfpathlineto{\pgfqpoint{2.002634in}{0.548120in}}%
\pgfpathlineto{\pgfqpoint{1.990185in}{0.543999in}}%
\pgfpathlineto{\pgfqpoint{1.976175in}{0.548583in}}%
\pgfpathlineto{\pgfqpoint{1.975186in}{0.557014in}}%
\pgfpathlineto{\pgfqpoint{1.970123in}{0.557651in}}%
\pgfpathlineto{\pgfqpoint{1.966602in}{0.562051in}}%
\pgfpathlineto{\pgfqpoint{1.965180in}{0.570849in}}%
\pgfpathlineto{\pgfqpoint{1.959526in}{0.575009in}}%
\pgfpathlineto{\pgfqpoint{1.958167in}{0.578691in}}%
\pgfpathlineto{\pgfqpoint{1.959192in}{0.587885in}}%
\pgfpathlineto{\pgfqpoint{1.954185in}{0.597454in}}%
\pgfpathlineto{\pgfqpoint{1.954440in}{0.605869in}}%
\pgfpathlineto{\pgfqpoint{1.958935in}{0.610678in}}%
\pgfpathlineto{\pgfqpoint{1.963991in}{0.607879in}}%
\pgfpathlineto{\pgfqpoint{2.010177in}{0.609791in}}%
\pgfpathlineto{\pgfqpoint{2.037435in}{0.611183in}}%
\pgfpathclose%
\pgfusepath{fill}%
\end{pgfscope}%
\begin{pgfscope}%
\pgfpathrectangle{\pgfqpoint{0.100000in}{0.100000in}}{\pgfqpoint{3.007045in}{1.925000in}}%
\pgfusepath{clip}%
\pgfsetbuttcap%
\pgfsetmiterjoin%
\definecolor{currentfill}{rgb}{0.301069,0.601399,0.792957}%
\pgfsetfillcolor{currentfill}%
\pgfsetlinewidth{0.000000pt}%
\definecolor{currentstroke}{rgb}{0.000000,0.000000,0.000000}%
\pgfsetstrokecolor{currentstroke}%
\pgfsetstrokeopacity{0.000000}%
\pgfsetdash{}{0pt}%
\pgfpathmoveto{\pgfqpoint{1.596180in}{1.004385in}}%
\pgfpathlineto{\pgfqpoint{1.561807in}{1.005883in}}%
\pgfpathlineto{\pgfqpoint{1.562899in}{1.031213in}}%
\pgfpathlineto{\pgfqpoint{1.534548in}{1.032337in}}%
\pgfpathlineto{\pgfqpoint{1.534859in}{1.055660in}}%
\pgfpathlineto{\pgfqpoint{1.563272in}{1.054193in}}%
\pgfpathlineto{\pgfqpoint{1.563552in}{1.060256in}}%
\pgfpathlineto{\pgfqpoint{1.591676in}{1.059068in}}%
\pgfpathlineto{\pgfqpoint{1.591838in}{1.053004in}}%
\pgfpathlineto{\pgfqpoint{1.590957in}{1.030045in}}%
\pgfpathlineto{\pgfqpoint{1.596980in}{1.029821in}}%
\pgfpathlineto{\pgfqpoint{1.596180in}{1.004385in}}%
\pgfpathclose%
\pgfusepath{fill}%
\end{pgfscope}%
\begin{pgfscope}%
\pgfpathrectangle{\pgfqpoint{0.100000in}{0.100000in}}{\pgfqpoint{3.007045in}{1.925000in}}%
\pgfusepath{clip}%
\pgfsetbuttcap%
\pgfsetmiterjoin%
\definecolor{currentfill}{rgb}{0.485490,0.718524,0.853426}%
\pgfsetfillcolor{currentfill}%
\pgfsetlinewidth{0.000000pt}%
\definecolor{currentstroke}{rgb}{0.000000,0.000000,0.000000}%
\pgfsetstrokecolor{currentstroke}%
\pgfsetstrokeopacity{0.000000}%
\pgfsetdash{}{0pt}%
\pgfpathmoveto{\pgfqpoint{2.099536in}{1.444895in}}%
\pgfpathlineto{\pgfqpoint{2.088107in}{1.444073in}}%
\pgfpathlineto{\pgfqpoint{2.087655in}{1.449833in}}%
\pgfpathlineto{\pgfqpoint{2.058743in}{1.447959in}}%
\pgfpathlineto{\pgfqpoint{2.047314in}{1.447880in}}%
\pgfpathlineto{\pgfqpoint{2.046797in}{1.456021in}}%
\pgfpathlineto{\pgfqpoint{2.050014in}{1.463525in}}%
\pgfpathlineto{\pgfqpoint{2.049555in}{1.470541in}}%
\pgfpathlineto{\pgfqpoint{2.029155in}{1.469240in}}%
\pgfpathlineto{\pgfqpoint{2.028120in}{1.486633in}}%
\pgfpathlineto{\pgfqpoint{2.045773in}{1.487575in}}%
\pgfpathlineto{\pgfqpoint{2.043986in}{1.516491in}}%
\pgfpathlineto{\pgfqpoint{2.073048in}{1.518236in}}%
\pgfpathlineto{\pgfqpoint{2.073413in}{1.512435in}}%
\pgfpathlineto{\pgfqpoint{2.090405in}{1.513286in}}%
\pgfpathlineto{\pgfqpoint{2.092961in}{1.513610in}}%
\pgfpathlineto{\pgfqpoint{2.094574in}{1.490855in}}%
\pgfpathlineto{\pgfqpoint{2.101634in}{1.491321in}}%
\pgfpathlineto{\pgfqpoint{2.103472in}{1.468301in}}%
\pgfpathlineto{\pgfqpoint{2.097758in}{1.467853in}}%
\pgfpathlineto{\pgfqpoint{2.099536in}{1.444895in}}%
\pgfpathclose%
\pgfusepath{fill}%
\end{pgfscope}%
\begin{pgfscope}%
\pgfpathrectangle{\pgfqpoint{0.100000in}{0.100000in}}{\pgfqpoint{3.007045in}{1.925000in}}%
\pgfusepath{clip}%
\pgfsetbuttcap%
\pgfsetmiterjoin%
\definecolor{currentfill}{rgb}{0.466667,0.708189,0.849366}%
\pgfsetfillcolor{currentfill}%
\pgfsetlinewidth{0.000000pt}%
\definecolor{currentstroke}{rgb}{0.000000,0.000000,0.000000}%
\pgfsetstrokecolor{currentstroke}%
\pgfsetstrokeopacity{0.000000}%
\pgfsetdash{}{0pt}%
\pgfpathmoveto{\pgfqpoint{2.639205in}{0.819156in}}%
\pgfpathlineto{\pgfqpoint{2.636694in}{0.812102in}}%
\pgfpathlineto{\pgfqpoint{2.628381in}{0.812335in}}%
\pgfpathlineto{\pgfqpoint{2.623221in}{0.806040in}}%
\pgfpathlineto{\pgfqpoint{2.626021in}{0.803536in}}%
\pgfpathlineto{\pgfqpoint{2.619999in}{0.798689in}}%
\pgfpathlineto{\pgfqpoint{2.618917in}{0.795608in}}%
\pgfpathlineto{\pgfqpoint{2.611802in}{0.789793in}}%
\pgfpathlineto{\pgfqpoint{2.611478in}{0.786111in}}%
\pgfpathlineto{\pgfqpoint{2.605521in}{0.780009in}}%
\pgfpathlineto{\pgfqpoint{2.599364in}{0.777779in}}%
\pgfpathlineto{\pgfqpoint{2.588994in}{0.768511in}}%
\pgfpathlineto{\pgfqpoint{2.581007in}{0.767715in}}%
\pgfpathlineto{\pgfqpoint{2.576021in}{0.772345in}}%
\pgfpathlineto{\pgfqpoint{2.572033in}{0.772743in}}%
\pgfpathlineto{\pgfqpoint{2.567570in}{0.778188in}}%
\pgfpathlineto{\pgfqpoint{2.559434in}{0.779009in}}%
\pgfpathlineto{\pgfqpoint{2.554010in}{0.786819in}}%
\pgfpathlineto{\pgfqpoint{2.548566in}{0.790531in}}%
\pgfpathlineto{\pgfqpoint{2.542055in}{0.797900in}}%
\pgfpathlineto{\pgfqpoint{2.556125in}{0.810180in}}%
\pgfpathlineto{\pgfqpoint{2.570619in}{0.823049in}}%
\pgfpathlineto{\pgfqpoint{2.575563in}{0.818701in}}%
\pgfpathlineto{\pgfqpoint{2.584582in}{0.822960in}}%
\pgfpathlineto{\pgfqpoint{2.584761in}{0.832670in}}%
\pgfpathlineto{\pgfqpoint{2.590755in}{0.837224in}}%
\pgfpathlineto{\pgfqpoint{2.597720in}{0.838651in}}%
\pgfpathlineto{\pgfqpoint{2.612678in}{0.829037in}}%
\pgfpathlineto{\pgfqpoint{2.628686in}{0.826071in}}%
\pgfpathlineto{\pgfqpoint{2.639205in}{0.819156in}}%
\pgfpathclose%
\pgfusepath{fill}%
\end{pgfscope}%
\begin{pgfscope}%
\pgfpathrectangle{\pgfqpoint{0.100000in}{0.100000in}}{\pgfqpoint{3.007045in}{1.925000in}}%
\pgfusepath{clip}%
\pgfsetbuttcap%
\pgfsetmiterjoin%
\definecolor{currentfill}{rgb}{0.396909,0.666851,0.830358}%
\pgfsetfillcolor{currentfill}%
\pgfsetlinewidth{0.000000pt}%
\definecolor{currentstroke}{rgb}{0.000000,0.000000,0.000000}%
\pgfsetstrokecolor{currentstroke}%
\pgfsetstrokeopacity{0.000000}%
\pgfsetdash{}{0pt}%
\pgfpathmoveto{\pgfqpoint{2.675874in}{0.992417in}}%
\pgfpathlineto{\pgfqpoint{2.673153in}{0.980638in}}%
\pgfpathlineto{\pgfqpoint{2.674050in}{0.976293in}}%
\pgfpathlineto{\pgfqpoint{2.667375in}{0.972671in}}%
\pgfpathlineto{\pgfqpoint{2.661577in}{0.967658in}}%
\pgfpathlineto{\pgfqpoint{2.657195in}{0.967322in}}%
\pgfpathlineto{\pgfqpoint{2.653930in}{0.970205in}}%
\pgfpathlineto{\pgfqpoint{2.650737in}{0.964955in}}%
\pgfpathlineto{\pgfqpoint{2.641093in}{0.964571in}}%
\pgfpathlineto{\pgfqpoint{2.625940in}{0.956912in}}%
\pgfpathlineto{\pgfqpoint{2.618329in}{0.960663in}}%
\pgfpathlineto{\pgfqpoint{2.608908in}{0.972810in}}%
\pgfpathlineto{\pgfqpoint{2.608759in}{0.975733in}}%
\pgfpathlineto{\pgfqpoint{2.597898in}{0.973667in}}%
\pgfpathlineto{\pgfqpoint{2.594940in}{0.995133in}}%
\pgfpathlineto{\pgfqpoint{2.611008in}{0.997970in}}%
\pgfpathlineto{\pgfqpoint{2.608855in}{1.001837in}}%
\pgfpathlineto{\pgfqpoint{2.605433in}{1.023762in}}%
\pgfpathlineto{\pgfqpoint{2.610860in}{1.024597in}}%
\pgfpathlineto{\pgfqpoint{2.608257in}{1.044199in}}%
\pgfpathlineto{\pgfqpoint{2.643747in}{1.050644in}}%
\pgfpathlineto{\pgfqpoint{2.641967in}{1.044405in}}%
\pgfpathlineto{\pgfqpoint{2.645335in}{1.026823in}}%
\pgfpathlineto{\pgfqpoint{2.650549in}{1.026900in}}%
\pgfpathlineto{\pgfqpoint{2.654897in}{1.034173in}}%
\pgfpathlineto{\pgfqpoint{2.664233in}{1.034590in}}%
\pgfpathlineto{\pgfqpoint{2.671365in}{1.033005in}}%
\pgfpathlineto{\pgfqpoint{2.663029in}{1.005596in}}%
\pgfpathlineto{\pgfqpoint{2.668059in}{1.000440in}}%
\pgfpathlineto{\pgfqpoint{2.672444in}{0.992955in}}%
\pgfpathlineto{\pgfqpoint{2.675874in}{0.992417in}}%
\pgfpathclose%
\pgfusepath{fill}%
\end{pgfscope}%
\begin{pgfscope}%
\pgfpathrectangle{\pgfqpoint{0.100000in}{0.100000in}}{\pgfqpoint{3.007045in}{1.925000in}}%
\pgfusepath{clip}%
\pgfsetbuttcap%
\pgfsetmiterjoin%
\definecolor{currentfill}{rgb}{0.401953,0.670296,0.832326}%
\pgfsetfillcolor{currentfill}%
\pgfsetlinewidth{0.000000pt}%
\definecolor{currentstroke}{rgb}{0.000000,0.000000,0.000000}%
\pgfsetstrokecolor{currentstroke}%
\pgfsetstrokeopacity{0.000000}%
\pgfsetdash{}{0pt}%
\pgfpathmoveto{\pgfqpoint{1.584352in}{1.202908in}}%
\pgfpathlineto{\pgfqpoint{1.561531in}{1.203886in}}%
\pgfpathlineto{\pgfqpoint{1.563506in}{1.247230in}}%
\pgfpathlineto{\pgfqpoint{1.551658in}{1.248467in}}%
\pgfpathlineto{\pgfqpoint{1.552431in}{1.273213in}}%
\pgfpathlineto{\pgfqpoint{1.563271in}{1.272731in}}%
\pgfpathlineto{\pgfqpoint{1.564548in}{1.295579in}}%
\pgfpathlineto{\pgfqpoint{1.587220in}{1.294590in}}%
\pgfpathlineto{\pgfqpoint{1.586068in}{1.271711in}}%
\pgfpathlineto{\pgfqpoint{1.587386in}{1.271651in}}%
\pgfpathlineto{\pgfqpoint{1.586476in}{1.248770in}}%
\pgfpathlineto{\pgfqpoint{1.584352in}{1.202908in}}%
\pgfpathclose%
\pgfusepath{fill}%
\end{pgfscope}%
\begin{pgfscope}%
\pgfpathrectangle{\pgfqpoint{0.100000in}{0.100000in}}{\pgfqpoint{3.007045in}{1.925000in}}%
\pgfusepath{clip}%
\pgfsetbuttcap%
\pgfsetmiterjoin%
\definecolor{currentfill}{rgb}{0.361599,0.642737,0.816578}%
\pgfsetfillcolor{currentfill}%
\pgfsetlinewidth{0.000000pt}%
\definecolor{currentstroke}{rgb}{0.000000,0.000000,0.000000}%
\pgfsetstrokecolor{currentstroke}%
\pgfsetstrokeopacity{0.000000}%
\pgfsetdash{}{0pt}%
\pgfpathmoveto{\pgfqpoint{0.713573in}{1.866288in}}%
\pgfpathlineto{\pgfqpoint{0.708324in}{1.867692in}}%
\pgfpathlineto{\pgfqpoint{0.702364in}{1.864190in}}%
\pgfpathlineto{\pgfqpoint{0.700822in}{1.858600in}}%
\pgfpathlineto{\pgfqpoint{0.695822in}{1.854965in}}%
\pgfpathlineto{\pgfqpoint{0.691919in}{1.848024in}}%
\pgfpathlineto{\pgfqpoint{0.689171in}{1.848788in}}%
\pgfpathlineto{\pgfqpoint{0.682133in}{1.844866in}}%
\pgfpathlineto{\pgfqpoint{0.680569in}{1.839333in}}%
\pgfpathlineto{\pgfqpoint{0.665649in}{1.843218in}}%
\pgfpathlineto{\pgfqpoint{0.662469in}{1.839071in}}%
\pgfpathlineto{\pgfqpoint{0.657936in}{1.837275in}}%
\pgfpathlineto{\pgfqpoint{0.656028in}{1.830896in}}%
\pgfpathlineto{\pgfqpoint{0.652989in}{1.834668in}}%
\pgfpathlineto{\pgfqpoint{0.640168in}{1.838234in}}%
\pgfpathlineto{\pgfqpoint{0.635771in}{1.843623in}}%
\pgfpathlineto{\pgfqpoint{0.629717in}{1.846964in}}%
\pgfpathlineto{\pgfqpoint{0.627702in}{1.851025in}}%
\pgfpathlineto{\pgfqpoint{0.620178in}{1.855997in}}%
\pgfpathlineto{\pgfqpoint{0.618659in}{1.862354in}}%
\pgfpathlineto{\pgfqpoint{0.615087in}{1.868865in}}%
\pgfpathlineto{\pgfqpoint{0.619147in}{1.875248in}}%
\pgfpathlineto{\pgfqpoint{0.617908in}{1.888997in}}%
\pgfpathlineto{\pgfqpoint{0.622765in}{1.897664in}}%
\pgfpathlineto{\pgfqpoint{0.631834in}{1.900038in}}%
\pgfpathlineto{\pgfqpoint{0.634619in}{1.904375in}}%
\pgfpathlineto{\pgfqpoint{0.630805in}{1.915337in}}%
\pgfpathlineto{\pgfqpoint{0.633546in}{1.923434in}}%
\pgfpathlineto{\pgfqpoint{0.643966in}{1.926557in}}%
\pgfpathlineto{\pgfqpoint{0.650253in}{1.923532in}}%
\pgfpathlineto{\pgfqpoint{0.650154in}{1.917880in}}%
\pgfpathlineto{\pgfqpoint{0.654234in}{1.907098in}}%
\pgfpathlineto{\pgfqpoint{0.659239in}{1.897832in}}%
\pgfpathlineto{\pgfqpoint{0.657890in}{1.895448in}}%
\pgfpathlineto{\pgfqpoint{0.666800in}{1.883212in}}%
\pgfpathlineto{\pgfqpoint{0.667442in}{1.879016in}}%
\pgfpathlineto{\pgfqpoint{0.675310in}{1.876711in}}%
\pgfpathlineto{\pgfqpoint{0.676269in}{1.883139in}}%
\pgfpathlineto{\pgfqpoint{0.683348in}{1.884220in}}%
\pgfpathlineto{\pgfqpoint{0.685273in}{1.876771in}}%
\pgfpathlineto{\pgfqpoint{0.695862in}{1.879153in}}%
\pgfpathlineto{\pgfqpoint{0.699664in}{1.876125in}}%
\pgfpathlineto{\pgfqpoint{0.710511in}{1.880181in}}%
\pgfpathlineto{\pgfqpoint{0.713541in}{1.878149in}}%
\pgfpathlineto{\pgfqpoint{0.713573in}{1.866288in}}%
\pgfpathclose%
\pgfusepath{fill}%
\end{pgfscope}%
\begin{pgfscope}%
\pgfpathrectangle{\pgfqpoint{0.100000in}{0.100000in}}{\pgfqpoint{3.007045in}{1.925000in}}%
\pgfusepath{clip}%
\pgfsetbuttcap%
\pgfsetmiterjoin%
\definecolor{currentfill}{rgb}{0.371688,0.649627,0.820515}%
\pgfsetfillcolor{currentfill}%
\pgfsetlinewidth{0.000000pt}%
\definecolor{currentstroke}{rgb}{0.000000,0.000000,0.000000}%
\pgfsetstrokecolor{currentstroke}%
\pgfsetstrokeopacity{0.000000}%
\pgfsetdash{}{0pt}%
\pgfpathmoveto{\pgfqpoint{2.087534in}{0.559447in}}%
\pgfpathlineto{\pgfqpoint{2.082124in}{0.561840in}}%
\pgfpathlineto{\pgfqpoint{2.077352in}{0.576813in}}%
\pgfpathlineto{\pgfqpoint{2.072286in}{0.580627in}}%
\pgfpathlineto{\pgfqpoint{2.067276in}{0.590315in}}%
\pgfpathlineto{\pgfqpoint{2.067754in}{0.598346in}}%
\pgfpathlineto{\pgfqpoint{2.070931in}{0.604028in}}%
\pgfpathlineto{\pgfqpoint{2.072666in}{0.613237in}}%
\pgfpathlineto{\pgfqpoint{2.076881in}{0.613497in}}%
\pgfpathlineto{\pgfqpoint{2.085520in}{0.614773in}}%
\pgfpathlineto{\pgfqpoint{2.094715in}{0.614481in}}%
\pgfpathlineto{\pgfqpoint{2.095084in}{0.608430in}}%
\pgfpathlineto{\pgfqpoint{2.123918in}{0.610337in}}%
\pgfpathlineto{\pgfqpoint{2.121151in}{0.645024in}}%
\pgfpathlineto{\pgfqpoint{2.143322in}{0.646719in}}%
\pgfpathlineto{\pgfqpoint{2.146781in}{0.617810in}}%
\pgfpathlineto{\pgfqpoint{2.151467in}{0.577078in}}%
\pgfpathlineto{\pgfqpoint{2.147381in}{0.572588in}}%
\pgfpathlineto{\pgfqpoint{2.139247in}{0.575561in}}%
\pgfpathlineto{\pgfqpoint{2.134440in}{0.573344in}}%
\pgfpathlineto{\pgfqpoint{2.121286in}{0.575705in}}%
\pgfpathlineto{\pgfqpoint{2.113119in}{0.573784in}}%
\pgfpathlineto{\pgfqpoint{2.098215in}{0.568019in}}%
\pgfpathlineto{\pgfqpoint{2.087534in}{0.559447in}}%
\pgfpathclose%
\pgfusepath{fill}%
\end{pgfscope}%
\begin{pgfscope}%
\pgfpathrectangle{\pgfqpoint{0.100000in}{0.100000in}}{\pgfqpoint{3.007045in}{1.925000in}}%
\pgfusepath{clip}%
\pgfsetbuttcap%
\pgfsetmiterjoin%
\definecolor{currentfill}{rgb}{0.968627,0.984314,1.000000}%
\pgfsetfillcolor{currentfill}%
\pgfsetlinewidth{0.000000pt}%
\definecolor{currentstroke}{rgb}{0.000000,0.000000,0.000000}%
\pgfsetstrokecolor{currentstroke}%
\pgfsetstrokeopacity{0.000000}%
\pgfsetdash{}{0pt}%
\pgfpathmoveto{\pgfqpoint{2.990796in}{1.458726in}}%
\pgfpathlineto{\pgfqpoint{2.997029in}{1.452912in}}%
\pgfpathlineto{\pgfqpoint{2.995205in}{1.449674in}}%
\pgfpathlineto{\pgfqpoint{2.988017in}{1.448533in}}%
\pgfpathlineto{\pgfqpoint{2.992375in}{1.455283in}}%
\pgfpathlineto{\pgfqpoint{2.990796in}{1.458726in}}%
\pgfpathclose%
\pgfusepath{fill}%
\end{pgfscope}%
\begin{pgfscope}%
\pgfpathrectangle{\pgfqpoint{0.100000in}{0.100000in}}{\pgfqpoint{3.007045in}{1.925000in}}%
\pgfusepath{clip}%
\pgfsetbuttcap%
\pgfsetmiterjoin%
\definecolor{currentfill}{rgb}{0.691580,0.822745,0.907543}%
\pgfsetfillcolor{currentfill}%
\pgfsetlinewidth{0.000000pt}%
\definecolor{currentstroke}{rgb}{0.000000,0.000000,0.000000}%
\pgfsetstrokecolor{currentstroke}%
\pgfsetstrokeopacity{0.000000}%
\pgfsetdash{}{0pt}%
\pgfpathmoveto{\pgfqpoint{2.537236in}{1.043373in}}%
\pgfpathlineto{\pgfqpoint{2.526060in}{1.056473in}}%
\pgfpathlineto{\pgfqpoint{2.521063in}{1.052157in}}%
\pgfpathlineto{\pgfqpoint{2.511594in}{1.060569in}}%
\pgfpathlineto{\pgfqpoint{2.508760in}{1.065976in}}%
\pgfpathlineto{\pgfqpoint{2.512374in}{1.068702in}}%
\pgfpathlineto{\pgfqpoint{2.502786in}{1.075737in}}%
\pgfpathlineto{\pgfqpoint{2.510605in}{1.081728in}}%
\pgfpathlineto{\pgfqpoint{2.509239in}{1.087027in}}%
\pgfpathlineto{\pgfqpoint{2.516736in}{1.108073in}}%
\pgfpathlineto{\pgfqpoint{2.527642in}{1.106923in}}%
\pgfpathlineto{\pgfqpoint{2.535565in}{1.108737in}}%
\pgfpathlineto{\pgfqpoint{2.544389in}{1.103816in}}%
\pgfpathlineto{\pgfqpoint{2.554928in}{1.112611in}}%
\pgfpathlineto{\pgfqpoint{2.559027in}{1.119923in}}%
\pgfpathlineto{\pgfqpoint{2.566770in}{1.118700in}}%
\pgfpathlineto{\pgfqpoint{2.567398in}{1.112031in}}%
\pgfpathlineto{\pgfqpoint{2.578351in}{1.105297in}}%
\pgfpathlineto{\pgfqpoint{2.573073in}{1.099388in}}%
\pgfpathlineto{\pgfqpoint{2.570652in}{1.100774in}}%
\pgfpathlineto{\pgfqpoint{2.564677in}{1.094671in}}%
\pgfpathlineto{\pgfqpoint{2.562917in}{1.088318in}}%
\pgfpathlineto{\pgfqpoint{2.564592in}{1.081776in}}%
\pgfpathlineto{\pgfqpoint{2.574988in}{1.077569in}}%
\pgfpathlineto{\pgfqpoint{2.578887in}{1.072582in}}%
\pgfpathlineto{\pgfqpoint{2.578534in}{1.060078in}}%
\pgfpathlineto{\pgfqpoint{2.572984in}{1.054602in}}%
\pgfpathlineto{\pgfqpoint{2.547424in}{1.055944in}}%
\pgfpathlineto{\pgfqpoint{2.543457in}{1.053394in}}%
\pgfpathlineto{\pgfqpoint{2.537236in}{1.043373in}}%
\pgfpathclose%
\pgfusepath{fill}%
\end{pgfscope}%
\begin{pgfscope}%
\pgfpathrectangle{\pgfqpoint{0.100000in}{0.100000in}}{\pgfqpoint{3.007045in}{1.925000in}}%
\pgfusepath{clip}%
\pgfsetbuttcap%
\pgfsetmiterjoin%
\definecolor{currentfill}{rgb}{0.260715,0.573841,0.777209}%
\pgfsetfillcolor{currentfill}%
\pgfsetlinewidth{0.000000pt}%
\definecolor{currentstroke}{rgb}{0.000000,0.000000,0.000000}%
\pgfsetstrokecolor{currentstroke}%
\pgfsetstrokeopacity{0.000000}%
\pgfsetdash{}{0pt}%
\pgfpathmoveto{\pgfqpoint{1.072883in}{1.296139in}}%
\pgfpathlineto{\pgfqpoint{1.067620in}{1.298977in}}%
\pgfpathlineto{\pgfqpoint{1.068540in}{1.305641in}}%
\pgfpathlineto{\pgfqpoint{1.067332in}{1.310685in}}%
\pgfpathlineto{\pgfqpoint{1.057913in}{1.312141in}}%
\pgfpathlineto{\pgfqpoint{1.057220in}{1.307768in}}%
\pgfpathlineto{\pgfqpoint{1.052269in}{1.305131in}}%
\pgfpathlineto{\pgfqpoint{1.050961in}{1.311223in}}%
\pgfpathlineto{\pgfqpoint{1.033610in}{1.309798in}}%
\pgfpathlineto{\pgfqpoint{1.027613in}{1.314125in}}%
\pgfpathlineto{\pgfqpoint{1.029601in}{1.326060in}}%
\pgfpathlineto{\pgfqpoint{1.005052in}{1.330045in}}%
\pgfpathlineto{\pgfqpoint{0.978339in}{1.334944in}}%
\pgfpathlineto{\pgfqpoint{0.981285in}{1.351407in}}%
\pgfpathlineto{\pgfqpoint{0.972493in}{1.347031in}}%
\pgfpathlineto{\pgfqpoint{0.969963in}{1.352958in}}%
\pgfpathlineto{\pgfqpoint{0.970085in}{1.358692in}}%
\pgfpathlineto{\pgfqpoint{0.963794in}{1.361956in}}%
\pgfpathlineto{\pgfqpoint{0.960688in}{1.366618in}}%
\pgfpathlineto{\pgfqpoint{0.965270in}{1.373029in}}%
\pgfpathlineto{\pgfqpoint{0.968734in}{1.380626in}}%
\pgfpathlineto{\pgfqpoint{0.967550in}{1.392277in}}%
\pgfpathlineto{\pgfqpoint{0.965948in}{1.394443in}}%
\pgfpathlineto{\pgfqpoint{0.968744in}{1.400234in}}%
\pgfpathlineto{\pgfqpoint{0.967728in}{1.403968in}}%
\pgfpathlineto{\pgfqpoint{0.989963in}{1.400079in}}%
\pgfpathlineto{\pgfqpoint{1.001890in}{1.466220in}}%
\pgfpathlineto{\pgfqpoint{1.005297in}{1.485446in}}%
\pgfpathlineto{\pgfqpoint{1.016022in}{1.482462in}}%
\pgfpathlineto{\pgfqpoint{1.015288in}{1.478274in}}%
\pgfpathlineto{\pgfqpoint{1.026443in}{1.476331in}}%
\pgfpathlineto{\pgfqpoint{1.027092in}{1.480000in}}%
\pgfpathlineto{\pgfqpoint{1.038193in}{1.478061in}}%
\pgfpathlineto{\pgfqpoint{1.039144in}{1.483711in}}%
\pgfpathlineto{\pgfqpoint{1.052724in}{1.481408in}}%
\pgfpathlineto{\pgfqpoint{1.053944in}{1.487032in}}%
\pgfpathlineto{\pgfqpoint{1.067969in}{1.484605in}}%
\pgfpathlineto{\pgfqpoint{1.069796in}{1.477757in}}%
\pgfpathlineto{\pgfqpoint{1.069165in}{1.468405in}}%
\pgfpathlineto{\pgfqpoint{1.070708in}{1.461628in}}%
\pgfpathlineto{\pgfqpoint{1.075159in}{1.453148in}}%
\pgfpathlineto{\pgfqpoint{1.079954in}{1.447589in}}%
\pgfpathlineto{\pgfqpoint{1.085388in}{1.435891in}}%
\pgfpathlineto{\pgfqpoint{1.092414in}{1.429227in}}%
\pgfpathlineto{\pgfqpoint{1.089929in}{1.412247in}}%
\pgfpathlineto{\pgfqpoint{1.089437in}{1.400986in}}%
\pgfpathlineto{\pgfqpoint{1.139789in}{1.393330in}}%
\pgfpathlineto{\pgfqpoint{1.162889in}{1.390045in}}%
\pgfpathlineto{\pgfqpoint{1.162230in}{1.378562in}}%
\pgfpathlineto{\pgfqpoint{1.158048in}{1.350480in}}%
\pgfpathlineto{\pgfqpoint{1.137581in}{1.353507in}}%
\pgfpathlineto{\pgfqpoint{1.135759in}{1.336378in}}%
\pgfpathlineto{\pgfqpoint{1.131992in}{1.310519in}}%
\pgfpathlineto{\pgfqpoint{1.102069in}{1.314741in}}%
\pgfpathlineto{\pgfqpoint{1.076315in}{1.318750in}}%
\pgfpathlineto{\pgfqpoint{1.072883in}{1.296139in}}%
\pgfpathclose%
\pgfusepath{fill}%
\end{pgfscope}%
\begin{pgfscope}%
\pgfpathrectangle{\pgfqpoint{0.100000in}{0.100000in}}{\pgfqpoint{3.007045in}{1.925000in}}%
\pgfusepath{clip}%
\pgfsetbuttcap%
\pgfsetmiterjoin%
\definecolor{currentfill}{rgb}{0.560784,0.759862,0.869666}%
\pgfsetfillcolor{currentfill}%
\pgfsetlinewidth{0.000000pt}%
\definecolor{currentstroke}{rgb}{0.000000,0.000000,0.000000}%
\pgfsetstrokecolor{currentstroke}%
\pgfsetstrokeopacity{0.000000}%
\pgfsetdash{}{0pt}%
\pgfpathmoveto{\pgfqpoint{2.117083in}{1.446231in}}%
\pgfpathlineto{\pgfqpoint{2.099536in}{1.444895in}}%
\pgfpathlineto{\pgfqpoint{2.097758in}{1.467853in}}%
\pgfpathlineto{\pgfqpoint{2.103472in}{1.468301in}}%
\pgfpathlineto{\pgfqpoint{2.101634in}{1.491321in}}%
\pgfpathlineto{\pgfqpoint{2.108968in}{1.491853in}}%
\pgfpathlineto{\pgfqpoint{2.108536in}{1.497608in}}%
\pgfpathlineto{\pgfqpoint{2.124853in}{1.498938in}}%
\pgfpathlineto{\pgfqpoint{2.126881in}{1.489860in}}%
\pgfpathlineto{\pgfqpoint{2.121101in}{1.483818in}}%
\pgfpathlineto{\pgfqpoint{2.118059in}{1.468764in}}%
\pgfpathlineto{\pgfqpoint{2.120658in}{1.456120in}}%
\pgfpathlineto{\pgfqpoint{2.117083in}{1.446231in}}%
\pgfpathclose%
\pgfusepath{fill}%
\end{pgfscope}%
\begin{pgfscope}%
\pgfpathrectangle{\pgfqpoint{0.100000in}{0.100000in}}{\pgfqpoint{3.007045in}{1.925000in}}%
\pgfusepath{clip}%
\pgfsetbuttcap%
\pgfsetmiterjoin%
\definecolor{currentfill}{rgb}{0.150727,0.464452,0.720784}%
\pgfsetfillcolor{currentfill}%
\pgfsetlinewidth{0.000000pt}%
\definecolor{currentstroke}{rgb}{0.000000,0.000000,0.000000}%
\pgfsetstrokecolor{currentstroke}%
\pgfsetstrokeopacity{0.000000}%
\pgfsetdash{}{0pt}%
\pgfpathmoveto{\pgfqpoint{1.330767in}{0.811238in}}%
\pgfpathlineto{\pgfqpoint{1.333803in}{0.843365in}}%
\pgfpathlineto{\pgfqpoint{1.361996in}{0.840962in}}%
\pgfpathlineto{\pgfqpoint{1.364391in}{0.869531in}}%
\pgfpathlineto{\pgfqpoint{1.392891in}{0.867332in}}%
\pgfpathlineto{\pgfqpoint{1.390704in}{0.838664in}}%
\pgfpathlineto{\pgfqpoint{1.385676in}{0.839061in}}%
\pgfpathlineto{\pgfqpoint{1.383417in}{0.806844in}}%
\pgfpathlineto{\pgfqpoint{1.330767in}{0.811238in}}%
\pgfpathclose%
\pgfusepath{fill}%
\end{pgfscope}%
\begin{pgfscope}%
\pgfpathrectangle{\pgfqpoint{0.100000in}{0.100000in}}{\pgfqpoint{3.007045in}{1.925000in}}%
\pgfusepath{clip}%
\pgfsetbuttcap%
\pgfsetmiterjoin%
\definecolor{currentfill}{rgb}{0.541961,0.749527,0.865606}%
\pgfsetfillcolor{currentfill}%
\pgfsetlinewidth{0.000000pt}%
\definecolor{currentstroke}{rgb}{0.000000,0.000000,0.000000}%
\pgfsetstrokecolor{currentstroke}%
\pgfsetstrokeopacity{0.000000}%
\pgfsetdash{}{0pt}%
\pgfpathmoveto{\pgfqpoint{0.481516in}{1.783693in}}%
\pgfpathlineto{\pgfqpoint{0.465616in}{1.788403in}}%
\pgfpathlineto{\pgfqpoint{0.454831in}{1.792649in}}%
\pgfpathlineto{\pgfqpoint{0.462406in}{1.810296in}}%
\pgfpathlineto{\pgfqpoint{0.462308in}{1.820138in}}%
\pgfpathlineto{\pgfqpoint{0.467265in}{1.814731in}}%
\pgfpathlineto{\pgfqpoint{0.473024in}{1.816689in}}%
\pgfpathlineto{\pgfqpoint{0.479649in}{1.816372in}}%
\pgfpathlineto{\pgfqpoint{0.476951in}{1.821800in}}%
\pgfpathlineto{\pgfqpoint{0.468427in}{1.820044in}}%
\pgfpathlineto{\pgfqpoint{0.462073in}{1.828534in}}%
\pgfpathlineto{\pgfqpoint{0.465398in}{1.837277in}}%
\pgfpathlineto{\pgfqpoint{0.473406in}{1.837959in}}%
\pgfpathlineto{\pgfqpoint{0.472606in}{1.850993in}}%
\pgfpathlineto{\pgfqpoint{0.469231in}{1.854603in}}%
\pgfpathlineto{\pgfqpoint{0.471018in}{1.865773in}}%
\pgfpathlineto{\pgfqpoint{0.474450in}{1.863703in}}%
\pgfpathlineto{\pgfqpoint{0.482840in}{1.866326in}}%
\pgfpathlineto{\pgfqpoint{0.472793in}{1.873445in}}%
\pgfpathlineto{\pgfqpoint{0.474678in}{1.891400in}}%
\pgfpathlineto{\pgfqpoint{0.472885in}{1.900270in}}%
\pgfpathlineto{\pgfqpoint{0.474818in}{1.908630in}}%
\pgfpathlineto{\pgfqpoint{0.487655in}{1.903257in}}%
\pgfpathlineto{\pgfqpoint{0.510848in}{1.895622in}}%
\pgfpathlineto{\pgfqpoint{0.505442in}{1.879195in}}%
\pgfpathlineto{\pgfqpoint{0.502572in}{1.867867in}}%
\pgfpathlineto{\pgfqpoint{0.514994in}{1.864013in}}%
\pgfpathlineto{\pgfqpoint{0.515078in}{1.857792in}}%
\pgfpathlineto{\pgfqpoint{0.510872in}{1.844976in}}%
\pgfpathlineto{\pgfqpoint{0.501743in}{1.847839in}}%
\pgfpathlineto{\pgfqpoint{0.493956in}{1.821847in}}%
\pgfpathlineto{\pgfqpoint{0.542761in}{1.806508in}}%
\pgfpathlineto{\pgfqpoint{0.536092in}{1.785540in}}%
\pgfpathlineto{\pgfqpoint{0.529311in}{1.780896in}}%
\pgfpathlineto{\pgfqpoint{0.524572in}{1.784544in}}%
\pgfpathlineto{\pgfqpoint{0.510888in}{1.782623in}}%
\pgfpathlineto{\pgfqpoint{0.507744in}{1.796488in}}%
\pgfpathlineto{\pgfqpoint{0.498370in}{1.806808in}}%
\pgfpathlineto{\pgfqpoint{0.488870in}{1.806888in}}%
\pgfpathlineto{\pgfqpoint{0.481516in}{1.783693in}}%
\pgfpathclose%
\pgfusepath{fill}%
\end{pgfscope}%
\begin{pgfscope}%
\pgfpathrectangle{\pgfqpoint{0.100000in}{0.100000in}}{\pgfqpoint{3.007045in}{1.925000in}}%
\pgfusepath{clip}%
\pgfsetbuttcap%
\pgfsetmiterjoin%
\definecolor{currentfill}{rgb}{0.560784,0.759862,0.869666}%
\pgfsetfillcolor{currentfill}%
\pgfsetlinewidth{0.000000pt}%
\definecolor{currentstroke}{rgb}{0.000000,0.000000,0.000000}%
\pgfsetstrokecolor{currentstroke}%
\pgfsetstrokeopacity{0.000000}%
\pgfsetdash{}{0pt}%
\pgfpathmoveto{\pgfqpoint{0.812473in}{1.770492in}}%
\pgfpathlineto{\pgfqpoint{0.807745in}{1.750662in}}%
\pgfpathlineto{\pgfqpoint{0.794807in}{1.753888in}}%
\pgfpathlineto{\pgfqpoint{0.789165in}{1.749329in}}%
\pgfpathlineto{\pgfqpoint{0.774865in}{1.752771in}}%
\pgfpathlineto{\pgfqpoint{0.773003in}{1.745268in}}%
\pgfpathlineto{\pgfqpoint{0.765843in}{1.746677in}}%
\pgfpathlineto{\pgfqpoint{0.765426in}{1.755324in}}%
\pgfpathlineto{\pgfqpoint{0.760765in}{1.761473in}}%
\pgfpathlineto{\pgfqpoint{0.758923in}{1.767006in}}%
\pgfpathlineto{\pgfqpoint{0.752113in}{1.767443in}}%
\pgfpathlineto{\pgfqpoint{0.748808in}{1.770376in}}%
\pgfpathlineto{\pgfqpoint{0.740246in}{1.767188in}}%
\pgfpathlineto{\pgfqpoint{0.724939in}{1.766494in}}%
\pgfpathlineto{\pgfqpoint{0.724517in}{1.778735in}}%
\pgfpathlineto{\pgfqpoint{0.726344in}{1.778628in}}%
\pgfpathlineto{\pgfqpoint{0.735917in}{1.779747in}}%
\pgfpathlineto{\pgfqpoint{0.740338in}{1.787082in}}%
\pgfpathlineto{\pgfqpoint{0.746029in}{1.809361in}}%
\pgfpathlineto{\pgfqpoint{0.786311in}{1.799097in}}%
\pgfpathlineto{\pgfqpoint{0.784213in}{1.790565in}}%
\pgfpathlineto{\pgfqpoint{0.794601in}{1.781859in}}%
\pgfpathlineto{\pgfqpoint{0.808364in}{1.778465in}}%
\pgfpathlineto{\pgfqpoint{0.813833in}{1.776182in}}%
\pgfpathlineto{\pgfqpoint{0.812473in}{1.770492in}}%
\pgfpathclose%
\pgfusepath{fill}%
\end{pgfscope}%
\begin{pgfscope}%
\pgfpathrectangle{\pgfqpoint{0.100000in}{0.100000in}}{\pgfqpoint{3.007045in}{1.925000in}}%
\pgfusepath{clip}%
\pgfsetbuttcap%
\pgfsetmiterjoin%
\definecolor{currentfill}{rgb}{0.579608,0.770196,0.873725}%
\pgfsetfillcolor{currentfill}%
\pgfsetlinewidth{0.000000pt}%
\definecolor{currentstroke}{rgb}{0.000000,0.000000,0.000000}%
\pgfsetstrokecolor{currentstroke}%
\pgfsetstrokeopacity{0.000000}%
\pgfsetdash{}{0pt}%
\pgfpathmoveto{\pgfqpoint{0.748143in}{0.777058in}}%
\pgfpathlineto{\pgfqpoint{0.754327in}{0.807329in}}%
\pgfpathlineto{\pgfqpoint{0.795773in}{0.799049in}}%
\pgfpathlineto{\pgfqpoint{0.851184in}{0.788621in}}%
\pgfpathlineto{\pgfqpoint{0.912796in}{0.778126in}}%
\pgfpathlineto{\pgfqpoint{0.911810in}{0.772480in}}%
\pgfpathlineto{\pgfqpoint{0.927312in}{0.769778in}}%
\pgfpathlineto{\pgfqpoint{0.989331in}{0.759721in}}%
\pgfpathlineto{\pgfqpoint{0.978016in}{0.688001in}}%
\pgfpathlineto{\pgfqpoint{0.924756in}{0.696630in}}%
\pgfpathlineto{\pgfqpoint{0.864661in}{0.706945in}}%
\pgfpathlineto{\pgfqpoint{0.849482in}{0.716019in}}%
\pgfpathlineto{\pgfqpoint{0.766721in}{0.765622in}}%
\pgfpathlineto{\pgfqpoint{0.748143in}{0.777058in}}%
\pgfpathclose%
\pgfusepath{fill}%
\end{pgfscope}%
\begin{pgfscope}%
\pgfpathrectangle{\pgfqpoint{0.100000in}{0.100000in}}{\pgfqpoint{3.007045in}{1.925000in}}%
\pgfusepath{clip}%
\pgfsetbuttcap%
\pgfsetmiterjoin%
\definecolor{currentfill}{rgb}{0.341423,0.628958,0.808704}%
\pgfsetfillcolor{currentfill}%
\pgfsetlinewidth{0.000000pt}%
\definecolor{currentstroke}{rgb}{0.000000,0.000000,0.000000}%
\pgfsetstrokecolor{currentstroke}%
\pgfsetstrokeopacity{0.000000}%
\pgfsetdash{}{0pt}%
\pgfpathmoveto{\pgfqpoint{2.644431in}{1.224390in}}%
\pgfpathlineto{\pgfqpoint{2.643347in}{1.220746in}}%
\pgfpathlineto{\pgfqpoint{2.637469in}{1.212817in}}%
\pgfpathlineto{\pgfqpoint{2.634232in}{1.214856in}}%
\pgfpathlineto{\pgfqpoint{2.621536in}{1.211734in}}%
\pgfpathlineto{\pgfqpoint{2.616014in}{1.216485in}}%
\pgfpathlineto{\pgfqpoint{2.614276in}{1.211702in}}%
\pgfpathlineto{\pgfqpoint{2.609375in}{1.212759in}}%
\pgfpathlineto{\pgfqpoint{2.604647in}{1.216037in}}%
\pgfpathlineto{\pgfqpoint{2.589864in}{1.218745in}}%
\pgfpathlineto{\pgfqpoint{2.585351in}{1.220532in}}%
\pgfpathlineto{\pgfqpoint{2.590743in}{1.227736in}}%
\pgfpathlineto{\pgfqpoint{2.592721in}{1.234363in}}%
\pgfpathlineto{\pgfqpoint{2.598024in}{1.242727in}}%
\pgfpathlineto{\pgfqpoint{2.607688in}{1.243264in}}%
\pgfpathlineto{\pgfqpoint{2.614371in}{1.241096in}}%
\pgfpathlineto{\pgfqpoint{2.644431in}{1.224390in}}%
\pgfpathclose%
\pgfusepath{fill}%
\end{pgfscope}%
\begin{pgfscope}%
\pgfpathrectangle{\pgfqpoint{0.100000in}{0.100000in}}{\pgfqpoint{3.007045in}{1.925000in}}%
\pgfusepath{clip}%
\pgfsetbuttcap%
\pgfsetmiterjoin%
\definecolor{currentfill}{rgb}{0.158847,0.472572,0.724967}%
\pgfsetfillcolor{currentfill}%
\pgfsetlinewidth{0.000000pt}%
\definecolor{currentstroke}{rgb}{0.000000,0.000000,0.000000}%
\pgfsetstrokecolor{currentstroke}%
\pgfsetstrokeopacity{0.000000}%
\pgfsetdash{}{0pt}%
\pgfpathmoveto{\pgfqpoint{1.654308in}{0.604797in}}%
\pgfpathlineto{\pgfqpoint{1.643382in}{0.598629in}}%
\pgfpathlineto{\pgfqpoint{1.639929in}{0.589583in}}%
\pgfpathlineto{\pgfqpoint{1.622469in}{0.597909in}}%
\pgfpathlineto{\pgfqpoint{1.610932in}{0.600948in}}%
\pgfpathlineto{\pgfqpoint{1.606454in}{0.609350in}}%
\pgfpathlineto{\pgfqpoint{1.576374in}{0.610043in}}%
\pgfpathlineto{\pgfqpoint{1.571686in}{0.614167in}}%
\pgfpathlineto{\pgfqpoint{1.569891in}{0.623670in}}%
\pgfpathlineto{\pgfqpoint{1.586901in}{0.635387in}}%
\pgfpathlineto{\pgfqpoint{1.616016in}{0.651438in}}%
\pgfpathlineto{\pgfqpoint{1.620418in}{0.653867in}}%
\pgfpathlineto{\pgfqpoint{1.635041in}{0.627484in}}%
\pgfpathlineto{\pgfqpoint{1.639177in}{0.622324in}}%
\pgfpathlineto{\pgfqpoint{1.642936in}{0.624588in}}%
\pgfpathlineto{\pgfqpoint{1.654308in}{0.604797in}}%
\pgfpathclose%
\pgfusepath{fill}%
\end{pgfscope}%
\begin{pgfscope}%
\pgfpathrectangle{\pgfqpoint{0.100000in}{0.100000in}}{\pgfqpoint{3.007045in}{1.925000in}}%
\pgfusepath{clip}%
\pgfsetbuttcap%
\pgfsetmiterjoin%
\definecolor{currentfill}{rgb}{0.031373,0.241015,0.497978}%
\pgfsetfillcolor{currentfill}%
\pgfsetlinewidth{0.000000pt}%
\definecolor{currentstroke}{rgb}{0.000000,0.000000,0.000000}%
\pgfsetstrokecolor{currentstroke}%
\pgfsetstrokeopacity{0.000000}%
\pgfsetdash{}{0pt}%
\pgfpathmoveto{\pgfqpoint{1.521148in}{0.313814in}}%
\pgfpathlineto{\pgfqpoint{1.530987in}{0.320341in}}%
\pgfpathlineto{\pgfqpoint{1.534731in}{0.327636in}}%
\pgfpathlineto{\pgfqpoint{1.566652in}{0.326194in}}%
\pgfpathlineto{\pgfqpoint{1.592904in}{0.325092in}}%
\pgfpathlineto{\pgfqpoint{1.592550in}{0.313976in}}%
\pgfpathlineto{\pgfqpoint{1.599953in}{0.312537in}}%
\pgfpathlineto{\pgfqpoint{1.625397in}{0.311924in}}%
\pgfpathlineto{\pgfqpoint{1.627117in}{0.305150in}}%
\pgfpathlineto{\pgfqpoint{1.625131in}{0.302682in}}%
\pgfpathlineto{\pgfqpoint{1.630801in}{0.295513in}}%
\pgfpathlineto{\pgfqpoint{1.630560in}{0.289662in}}%
\pgfpathlineto{\pgfqpoint{1.633409in}{0.284357in}}%
\pgfpathlineto{\pgfqpoint{1.633735in}{0.277987in}}%
\pgfpathlineto{\pgfqpoint{1.641756in}{0.274374in}}%
\pgfpathlineto{\pgfqpoint{1.642081in}{0.269076in}}%
\pgfpathlineto{\pgfqpoint{1.628506in}{0.265666in}}%
\pgfpathlineto{\pgfqpoint{1.624844in}{0.260718in}}%
\pgfpathlineto{\pgfqpoint{1.619737in}{0.263995in}}%
\pgfpathlineto{\pgfqpoint{1.609852in}{0.273649in}}%
\pgfpathlineto{\pgfqpoint{1.597319in}{0.276746in}}%
\pgfpathlineto{\pgfqpoint{1.586571in}{0.275707in}}%
\pgfpathlineto{\pgfqpoint{1.575367in}{0.277979in}}%
\pgfpathlineto{\pgfqpoint{1.562903in}{0.288739in}}%
\pgfpathlineto{\pgfqpoint{1.550740in}{0.289846in}}%
\pgfpathlineto{\pgfqpoint{1.542376in}{0.299233in}}%
\pgfpathlineto{\pgfqpoint{1.540217in}{0.298637in}}%
\pgfpathlineto{\pgfqpoint{1.524231in}{0.303862in}}%
\pgfpathlineto{\pgfqpoint{1.525543in}{0.307224in}}%
\pgfpathlineto{\pgfqpoint{1.521148in}{0.313814in}}%
\pgfpathclose%
\pgfusepath{fill}%
\end{pgfscope}%
\begin{pgfscope}%
\pgfpathrectangle{\pgfqpoint{0.100000in}{0.100000in}}{\pgfqpoint{3.007045in}{1.925000in}}%
\pgfusepath{clip}%
\pgfsetbuttcap%
\pgfsetmiterjoin%
\definecolor{currentfill}{rgb}{0.406997,0.673741,0.834295}%
\pgfsetfillcolor{currentfill}%
\pgfsetlinewidth{0.000000pt}%
\definecolor{currentstroke}{rgb}{0.000000,0.000000,0.000000}%
\pgfsetstrokecolor{currentstroke}%
\pgfsetstrokeopacity{0.000000}%
\pgfsetdash{}{0pt}%
\pgfpathmoveto{\pgfqpoint{2.096418in}{1.118342in}}%
\pgfpathlineto{\pgfqpoint{2.095572in}{1.132813in}}%
\pgfpathlineto{\pgfqpoint{2.072910in}{1.131159in}}%
\pgfpathlineto{\pgfqpoint{2.073272in}{1.125381in}}%
\pgfpathlineto{\pgfqpoint{2.067309in}{1.125374in}}%
\pgfpathlineto{\pgfqpoint{2.065520in}{1.156638in}}%
\pgfpathlineto{\pgfqpoint{2.071143in}{1.157025in}}%
\pgfpathlineto{\pgfqpoint{2.105221in}{1.159153in}}%
\pgfpathlineto{\pgfqpoint{2.105428in}{1.156288in}}%
\pgfpathlineto{\pgfqpoint{2.142303in}{1.159619in}}%
\pgfpathlineto{\pgfqpoint{2.148381in}{1.158667in}}%
\pgfpathlineto{\pgfqpoint{2.151777in}{1.152326in}}%
\pgfpathlineto{\pgfqpoint{2.152006in}{1.147645in}}%
\pgfpathlineto{\pgfqpoint{2.155147in}{1.142366in}}%
\pgfpathlineto{\pgfqpoint{2.154997in}{1.139057in}}%
\pgfpathlineto{\pgfqpoint{2.135746in}{1.137366in}}%
\pgfpathlineto{\pgfqpoint{2.117823in}{1.135815in}}%
\pgfpathlineto{\pgfqpoint{2.118010in}{1.123238in}}%
\pgfpathlineto{\pgfqpoint{2.119277in}{1.119542in}}%
\pgfpathlineto{\pgfqpoint{2.096418in}{1.118342in}}%
\pgfpathclose%
\pgfusepath{fill}%
\end{pgfscope}%
\begin{pgfscope}%
\pgfpathrectangle{\pgfqpoint{0.100000in}{0.100000in}}{\pgfqpoint{3.007045in}{1.925000in}}%
\pgfusepath{clip}%
\pgfsetbuttcap%
\pgfsetmiterjoin%
\definecolor{currentfill}{rgb}{0.441569,0.694410,0.843952}%
\pgfsetfillcolor{currentfill}%
\pgfsetlinewidth{0.000000pt}%
\definecolor{currentstroke}{rgb}{0.000000,0.000000,0.000000}%
\pgfsetstrokecolor{currentstroke}%
\pgfsetstrokeopacity{0.000000}%
\pgfsetdash{}{0pt}%
\pgfpathmoveto{\pgfqpoint{2.547659in}{0.980532in}}%
\pgfpathlineto{\pgfqpoint{2.537459in}{0.985766in}}%
\pgfpathlineto{\pgfqpoint{2.533363in}{0.985590in}}%
\pgfpathlineto{\pgfqpoint{2.532109in}{0.998638in}}%
\pgfpathlineto{\pgfqpoint{2.522081in}{0.997367in}}%
\pgfpathlineto{\pgfqpoint{2.520017in}{1.015115in}}%
\pgfpathlineto{\pgfqpoint{2.514084in}{1.019211in}}%
\pgfpathlineto{\pgfqpoint{2.513186in}{1.021977in}}%
\pgfpathlineto{\pgfqpoint{2.515896in}{1.030173in}}%
\pgfpathlineto{\pgfqpoint{2.531083in}{1.032312in}}%
\pgfpathlineto{\pgfqpoint{2.547798in}{1.034084in}}%
\pgfpathlineto{\pgfqpoint{2.561819in}{1.036295in}}%
\pgfpathlineto{\pgfqpoint{2.566736in}{1.001545in}}%
\pgfpathlineto{\pgfqpoint{2.557574in}{1.001121in}}%
\pgfpathlineto{\pgfqpoint{2.549576in}{0.995557in}}%
\pgfpathlineto{\pgfqpoint{2.549802in}{0.987470in}}%
\pgfpathlineto{\pgfqpoint{2.545915in}{0.985956in}}%
\pgfpathlineto{\pgfqpoint{2.547659in}{0.980532in}}%
\pgfpathclose%
\pgfusepath{fill}%
\end{pgfscope}%
\begin{pgfscope}%
\pgfpathrectangle{\pgfqpoint{0.100000in}{0.100000in}}{\pgfqpoint{3.007045in}{1.925000in}}%
\pgfusepath{clip}%
\pgfsetbuttcap%
\pgfsetmiterjoin%
\definecolor{currentfill}{rgb}{0.366644,0.646182,0.818547}%
\pgfsetfillcolor{currentfill}%
\pgfsetlinewidth{0.000000pt}%
\definecolor{currentstroke}{rgb}{0.000000,0.000000,0.000000}%
\pgfsetstrokecolor{currentstroke}%
\pgfsetstrokeopacity{0.000000}%
\pgfsetdash{}{0pt}%
\pgfpathmoveto{\pgfqpoint{2.815570in}{1.283815in}}%
\pgfpathlineto{\pgfqpoint{2.811323in}{1.284246in}}%
\pgfpathlineto{\pgfqpoint{2.806809in}{1.301783in}}%
\pgfpathlineto{\pgfqpoint{2.791685in}{1.322963in}}%
\pgfpathlineto{\pgfqpoint{2.789776in}{1.325534in}}%
\pgfpathlineto{\pgfqpoint{2.799417in}{1.334517in}}%
\pgfpathlineto{\pgfqpoint{2.804907in}{1.347247in}}%
\pgfpathlineto{\pgfqpoint{2.809088in}{1.348480in}}%
\pgfpathlineto{\pgfqpoint{2.816490in}{1.347154in}}%
\pgfpathlineto{\pgfqpoint{2.819196in}{1.340247in}}%
\pgfpathlineto{\pgfqpoint{2.819265in}{1.328442in}}%
\pgfpathlineto{\pgfqpoint{2.821303in}{1.305330in}}%
\pgfpathlineto{\pgfqpoint{2.820085in}{1.298464in}}%
\pgfpathlineto{\pgfqpoint{2.815570in}{1.283815in}}%
\pgfpathclose%
\pgfusepath{fill}%
\end{pgfscope}%
\begin{pgfscope}%
\pgfpathrectangle{\pgfqpoint{0.100000in}{0.100000in}}{\pgfqpoint{3.007045in}{1.925000in}}%
\pgfusepath{clip}%
\pgfsetbuttcap%
\pgfsetmiterjoin%
\definecolor{currentfill}{rgb}{0.270804,0.580730,0.781146}%
\pgfsetfillcolor{currentfill}%
\pgfsetlinewidth{0.000000pt}%
\definecolor{currentstroke}{rgb}{0.000000,0.000000,0.000000}%
\pgfsetstrokecolor{currentstroke}%
\pgfsetstrokeopacity{0.000000}%
\pgfsetdash{}{0pt}%
\pgfpathmoveto{\pgfqpoint{1.633052in}{1.541386in}}%
\pgfpathlineto{\pgfqpoint{1.632109in}{1.506861in}}%
\pgfpathlineto{\pgfqpoint{1.597846in}{1.508282in}}%
\pgfpathlineto{\pgfqpoint{1.598571in}{1.525674in}}%
\pgfpathlineto{\pgfqpoint{1.557925in}{1.527548in}}%
\pgfpathlineto{\pgfqpoint{1.559045in}{1.550570in}}%
\pgfpathlineto{\pgfqpoint{1.552729in}{1.550914in}}%
\pgfpathlineto{\pgfqpoint{1.554716in}{1.596972in}}%
\pgfpathlineto{\pgfqpoint{1.587550in}{1.595331in}}%
\pgfpathlineto{\pgfqpoint{1.634891in}{1.593241in}}%
\pgfpathlineto{\pgfqpoint{1.633052in}{1.541386in}}%
\pgfpathclose%
\pgfusepath{fill}%
\end{pgfscope}%
\begin{pgfscope}%
\pgfpathrectangle{\pgfqpoint{0.100000in}{0.100000in}}{\pgfqpoint{3.007045in}{1.925000in}}%
\pgfusepath{clip}%
\pgfsetbuttcap%
\pgfsetmiterjoin%
\definecolor{currentfill}{rgb}{0.270804,0.580730,0.781146}%
\pgfsetfillcolor{currentfill}%
\pgfsetlinewidth{0.000000pt}%
\definecolor{currentstroke}{rgb}{0.000000,0.000000,0.000000}%
\pgfsetstrokecolor{currentstroke}%
\pgfsetstrokeopacity{0.000000}%
\pgfsetdash{}{0pt}%
\pgfpathmoveto{\pgfqpoint{1.827733in}{0.669764in}}%
\pgfpathlineto{\pgfqpoint{1.799830in}{0.669148in}}%
\pgfpathlineto{\pgfqpoint{1.803211in}{0.660558in}}%
\pgfpathlineto{\pgfqpoint{1.775875in}{0.660550in}}%
\pgfpathlineto{\pgfqpoint{1.773171in}{0.660592in}}%
\pgfpathlineto{\pgfqpoint{1.773020in}{0.706433in}}%
\pgfpathlineto{\pgfqpoint{1.763758in}{0.708620in}}%
\pgfpathlineto{\pgfqpoint{1.763868in}{0.730641in}}%
\pgfpathlineto{\pgfqpoint{1.787896in}{0.730793in}}%
\pgfpathlineto{\pgfqpoint{1.788689in}{0.729118in}}%
\pgfpathlineto{\pgfqpoint{1.825570in}{0.729490in}}%
\pgfpathlineto{\pgfqpoint{1.826203in}{0.670688in}}%
\pgfpathlineto{\pgfqpoint{1.827733in}{0.669764in}}%
\pgfpathclose%
\pgfusepath{fill}%
\end{pgfscope}%
\begin{pgfscope}%
\pgfpathrectangle{\pgfqpoint{0.100000in}{0.100000in}}{\pgfqpoint{3.007045in}{1.925000in}}%
\pgfusepath{clip}%
\pgfsetbuttcap%
\pgfsetmiterjoin%
\definecolor{currentfill}{rgb}{0.093272,0.396878,0.673664}%
\pgfsetfillcolor{currentfill}%
\pgfsetlinewidth{0.000000pt}%
\definecolor{currentstroke}{rgb}{0.000000,0.000000,0.000000}%
\pgfsetstrokecolor{currentstroke}%
\pgfsetstrokeopacity{0.000000}%
\pgfsetdash{}{0pt}%
\pgfpathmoveto{\pgfqpoint{0.877896in}{0.551221in}}%
\pgfpathlineto{\pgfqpoint{0.873853in}{0.546928in}}%
\pgfpathlineto{\pgfqpoint{0.875594in}{0.545533in}}%
\pgfpathlineto{\pgfqpoint{0.874170in}{0.543737in}}%
\pgfpathlineto{\pgfqpoint{0.876010in}{0.542324in}}%
\pgfpathlineto{\pgfqpoint{0.874603in}{0.540599in}}%
\pgfpathlineto{\pgfqpoint{0.876407in}{0.539190in}}%
\pgfpathlineto{\pgfqpoint{0.874516in}{0.537747in}}%
\pgfpathlineto{\pgfqpoint{0.870380in}{0.532422in}}%
\pgfpathlineto{\pgfqpoint{0.873947in}{0.529693in}}%
\pgfpathlineto{\pgfqpoint{0.872574in}{0.527836in}}%
\pgfpathlineto{\pgfqpoint{0.873845in}{0.526915in}}%
\pgfpathlineto{\pgfqpoint{0.867902in}{0.519376in}}%
\pgfpathlineto{\pgfqpoint{0.865978in}{0.517880in}}%
\pgfpathlineto{\pgfqpoint{0.860718in}{0.522144in}}%
\pgfpathlineto{\pgfqpoint{0.857919in}{0.518605in}}%
\pgfpathlineto{\pgfqpoint{0.852921in}{0.522581in}}%
\pgfpathlineto{\pgfqpoint{0.848387in}{0.517598in}}%
\pgfpathlineto{\pgfqpoint{0.845093in}{0.520174in}}%
\pgfpathlineto{\pgfqpoint{0.846505in}{0.521977in}}%
\pgfpathlineto{\pgfqpoint{0.844770in}{0.523398in}}%
\pgfpathlineto{\pgfqpoint{0.846127in}{0.525116in}}%
\pgfpathlineto{\pgfqpoint{0.842673in}{0.527910in}}%
\pgfpathlineto{\pgfqpoint{0.839868in}{0.524417in}}%
\pgfpathlineto{\pgfqpoint{0.836381in}{0.527215in}}%
\pgfpathlineto{\pgfqpoint{0.839219in}{0.530767in}}%
\pgfpathlineto{\pgfqpoint{0.832267in}{0.536448in}}%
\pgfpathlineto{\pgfqpoint{0.826641in}{0.529420in}}%
\pgfpathlineto{\pgfqpoint{0.818104in}{0.536633in}}%
\pgfpathlineto{\pgfqpoint{0.815410in}{0.533363in}}%
\pgfpathlineto{\pgfqpoint{0.810396in}{0.537663in}}%
\pgfpathlineto{\pgfqpoint{0.803129in}{0.529197in}}%
\pgfpathlineto{\pgfqpoint{0.801466in}{0.530632in}}%
\pgfpathlineto{\pgfqpoint{0.799988in}{0.528978in}}%
\pgfpathlineto{\pgfqpoint{0.790016in}{0.537796in}}%
\pgfpathlineto{\pgfqpoint{0.775661in}{0.550783in}}%
\pgfpathlineto{\pgfqpoint{0.768863in}{0.557118in}}%
\pgfpathlineto{\pgfqpoint{0.771871in}{0.560318in}}%
\pgfpathlineto{\pgfqpoint{0.770248in}{0.561865in}}%
\pgfpathlineto{\pgfqpoint{0.773326in}{0.565108in}}%
\pgfpathlineto{\pgfqpoint{0.771216in}{0.567098in}}%
\pgfpathlineto{\pgfqpoint{0.777383in}{0.573526in}}%
\pgfpathlineto{\pgfqpoint{0.776856in}{0.574080in}}%
\pgfpathlineto{\pgfqpoint{0.782998in}{0.580535in}}%
\pgfpathlineto{\pgfqpoint{0.784268in}{0.583082in}}%
\pgfpathlineto{\pgfqpoint{0.786955in}{0.580900in}}%
\pgfpathlineto{\pgfqpoint{0.788384in}{0.577765in}}%
\pgfpathlineto{\pgfqpoint{0.786641in}{0.576148in}}%
\pgfpathlineto{\pgfqpoint{0.784316in}{0.575077in}}%
\pgfpathlineto{\pgfqpoint{0.782913in}{0.573171in}}%
\pgfpathlineto{\pgfqpoint{0.780593in}{0.573219in}}%
\pgfpathlineto{\pgfqpoint{0.779806in}{0.568879in}}%
\pgfpathlineto{\pgfqpoint{0.781934in}{0.567331in}}%
\pgfpathlineto{\pgfqpoint{0.785364in}{0.563844in}}%
\pgfpathlineto{\pgfqpoint{0.788254in}{0.562327in}}%
\pgfpathlineto{\pgfqpoint{0.787949in}{0.560680in}}%
\pgfpathlineto{\pgfqpoint{0.789850in}{0.558307in}}%
\pgfpathlineto{\pgfqpoint{0.792509in}{0.557623in}}%
\pgfpathlineto{\pgfqpoint{0.793630in}{0.555844in}}%
\pgfpathlineto{\pgfqpoint{0.792534in}{0.553561in}}%
\pgfpathlineto{\pgfqpoint{0.797356in}{0.555466in}}%
\pgfpathlineto{\pgfqpoint{0.800166in}{0.557163in}}%
\pgfpathlineto{\pgfqpoint{0.801439in}{0.554540in}}%
\pgfpathlineto{\pgfqpoint{0.804446in}{0.553565in}}%
\pgfpathlineto{\pgfqpoint{0.804535in}{0.556258in}}%
\pgfpathlineto{\pgfqpoint{0.802504in}{0.559954in}}%
\pgfpathlineto{\pgfqpoint{0.799473in}{0.561628in}}%
\pgfpathlineto{\pgfqpoint{0.801889in}{0.563583in}}%
\pgfpathlineto{\pgfqpoint{0.801986in}{0.567136in}}%
\pgfpathlineto{\pgfqpoint{0.802817in}{0.569907in}}%
\pgfpathlineto{\pgfqpoint{0.801188in}{0.572131in}}%
\pgfpathlineto{\pgfqpoint{0.802671in}{0.576011in}}%
\pgfpathlineto{\pgfqpoint{0.804683in}{0.576334in}}%
\pgfpathlineto{\pgfqpoint{0.806090in}{0.575510in}}%
\pgfpathlineto{\pgfqpoint{0.805398in}{0.570181in}}%
\pgfpathlineto{\pgfqpoint{0.803112in}{0.567234in}}%
\pgfpathlineto{\pgfqpoint{0.803364in}{0.561395in}}%
\pgfpathlineto{\pgfqpoint{0.807010in}{0.559720in}}%
\pgfpathlineto{\pgfqpoint{0.807723in}{0.558079in}}%
\pgfpathlineto{\pgfqpoint{0.807544in}{0.554762in}}%
\pgfpathlineto{\pgfqpoint{0.812317in}{0.550291in}}%
\pgfpathlineto{\pgfqpoint{0.814525in}{0.552070in}}%
\pgfpathlineto{\pgfqpoint{0.815449in}{0.555639in}}%
\pgfpathlineto{\pgfqpoint{0.812956in}{0.557139in}}%
\pgfpathlineto{\pgfqpoint{0.812165in}{0.560243in}}%
\pgfpathlineto{\pgfqpoint{0.809060in}{0.561120in}}%
\pgfpathlineto{\pgfqpoint{0.807259in}{0.560638in}}%
\pgfpathlineto{\pgfqpoint{0.805657in}{0.562677in}}%
\pgfpathlineto{\pgfqpoint{0.805494in}{0.568104in}}%
\pgfpathlineto{\pgfqpoint{0.807400in}{0.569549in}}%
\pgfpathlineto{\pgfqpoint{0.811973in}{0.571987in}}%
\pgfpathlineto{\pgfqpoint{0.811391in}{0.574218in}}%
\pgfpathlineto{\pgfqpoint{0.806259in}{0.576839in}}%
\pgfpathlineto{\pgfqpoint{0.807373in}{0.578266in}}%
\pgfpathlineto{\pgfqpoint{0.803872in}{0.581047in}}%
\pgfpathlineto{\pgfqpoint{0.801672in}{0.583533in}}%
\pgfpathlineto{\pgfqpoint{0.799005in}{0.588521in}}%
\pgfpathlineto{\pgfqpoint{0.802634in}{0.593263in}}%
\pgfpathlineto{\pgfqpoint{0.804432in}{0.598840in}}%
\pgfpathlineto{\pgfqpoint{0.804689in}{0.601267in}}%
\pgfpathlineto{\pgfqpoint{0.803744in}{0.605764in}}%
\pgfpathlineto{\pgfqpoint{0.803822in}{0.608129in}}%
\pgfpathlineto{\pgfqpoint{0.803267in}{0.616806in}}%
\pgfpathlineto{\pgfqpoint{0.809438in}{0.610810in}}%
\pgfpathlineto{\pgfqpoint{0.813031in}{0.614554in}}%
\pgfpathlineto{\pgfqpoint{0.825701in}{0.602628in}}%
\pgfpathlineto{\pgfqpoint{0.826925in}{0.603954in}}%
\pgfpathlineto{\pgfqpoint{0.832109in}{0.599208in}}%
\pgfpathlineto{\pgfqpoint{0.830895in}{0.597873in}}%
\pgfpathlineto{\pgfqpoint{0.834401in}{0.594707in}}%
\pgfpathlineto{\pgfqpoint{0.835607in}{0.596048in}}%
\pgfpathlineto{\pgfqpoint{0.838673in}{0.593306in}}%
\pgfpathlineto{\pgfqpoint{0.837473in}{0.591959in}}%
\pgfpathlineto{\pgfqpoint{0.844619in}{0.585660in}}%
\pgfpathlineto{\pgfqpoint{0.845795in}{0.587009in}}%
\pgfpathlineto{\pgfqpoint{0.848948in}{0.584275in}}%
\pgfpathlineto{\pgfqpoint{0.847780in}{0.582922in}}%
\pgfpathlineto{\pgfqpoint{0.859713in}{0.572782in}}%
\pgfpathlineto{\pgfqpoint{0.867050in}{0.566744in}}%
\pgfpathlineto{\pgfqpoint{0.865552in}{0.564907in}}%
\pgfpathlineto{\pgfqpoint{0.872509in}{0.559210in}}%
\pgfpathlineto{\pgfqpoint{0.871135in}{0.557489in}}%
\pgfpathlineto{\pgfqpoint{0.878340in}{0.551804in}}%
\pgfpathlineto{\pgfqpoint{0.877896in}{0.551221in}}%
\pgfpathclose%
\pgfusepath{fill}%
\end{pgfscope}%
\begin{pgfscope}%
\pgfpathrectangle{\pgfqpoint{0.100000in}{0.100000in}}{\pgfqpoint{3.007045in}{1.925000in}}%
\pgfusepath{clip}%
\pgfsetbuttcap%
\pgfsetmiterjoin%
\definecolor{currentfill}{rgb}{0.031373,0.285675,0.564291}%
\pgfsetfillcolor{currentfill}%
\pgfsetlinewidth{0.000000pt}%
\definecolor{currentstroke}{rgb}{0.000000,0.000000,0.000000}%
\pgfsetstrokecolor{currentstroke}%
\pgfsetstrokeopacity{0.000000}%
\pgfsetdash{}{0pt}%
\pgfpathmoveto{\pgfqpoint{1.517880in}{1.481436in}}%
\pgfpathlineto{\pgfqpoint{1.516674in}{1.460211in}}%
\pgfpathlineto{\pgfqpoint{1.517464in}{1.451554in}}%
\pgfpathlineto{\pgfqpoint{1.511045in}{1.452183in}}%
\pgfpathlineto{\pgfqpoint{1.506479in}{1.454584in}}%
\pgfpathlineto{\pgfqpoint{1.496174in}{1.453745in}}%
\pgfpathlineto{\pgfqpoint{1.490267in}{1.460196in}}%
\pgfpathlineto{\pgfqpoint{1.483456in}{1.461953in}}%
\pgfpathlineto{\pgfqpoint{1.485701in}{1.483457in}}%
\pgfpathlineto{\pgfqpoint{1.517880in}{1.481436in}}%
\pgfpathclose%
\pgfusepath{fill}%
\end{pgfscope}%
\begin{pgfscope}%
\pgfpathrectangle{\pgfqpoint{0.100000in}{0.100000in}}{\pgfqpoint{3.007045in}{1.925000in}}%
\pgfusepath{clip}%
\pgfsetbuttcap%
\pgfsetmiterjoin%
\definecolor{currentfill}{rgb}{0.412042,0.677186,0.836263}%
\pgfsetfillcolor{currentfill}%
\pgfsetlinewidth{0.000000pt}%
\definecolor{currentstroke}{rgb}{0.000000,0.000000,0.000000}%
\pgfsetstrokecolor{currentstroke}%
\pgfsetstrokeopacity{0.000000}%
\pgfsetdash{}{0pt}%
\pgfpathmoveto{\pgfqpoint{2.006146in}{1.322530in}}%
\pgfpathlineto{\pgfqpoint{2.000310in}{1.318595in}}%
\pgfpathlineto{\pgfqpoint{1.996691in}{1.320570in}}%
\pgfpathlineto{\pgfqpoint{1.973965in}{1.320197in}}%
\pgfpathlineto{\pgfqpoint{1.973436in}{1.331705in}}%
\pgfpathlineto{\pgfqpoint{1.972422in}{1.354713in}}%
\pgfpathlineto{\pgfqpoint{1.983756in}{1.355205in}}%
\pgfpathlineto{\pgfqpoint{1.983506in}{1.360972in}}%
\pgfpathlineto{\pgfqpoint{1.992817in}{1.361377in}}%
\pgfpathlineto{\pgfqpoint{1.995792in}{1.357804in}}%
\pgfpathlineto{\pgfqpoint{1.997258in}{1.351410in}}%
\pgfpathlineto{\pgfqpoint{2.001078in}{1.349385in}}%
\pgfpathlineto{\pgfqpoint{2.008917in}{1.344598in}}%
\pgfpathlineto{\pgfqpoint{2.010393in}{1.337303in}}%
\pgfpathlineto{\pgfqpoint{2.009028in}{1.324274in}}%
\pgfpathlineto{\pgfqpoint{2.006146in}{1.322530in}}%
\pgfpathclose%
\pgfusepath{fill}%
\end{pgfscope}%
\begin{pgfscope}%
\pgfpathrectangle{\pgfqpoint{0.100000in}{0.100000in}}{\pgfqpoint{3.007045in}{1.925000in}}%
\pgfusepath{clip}%
\pgfsetbuttcap%
\pgfsetmiterjoin%
\definecolor{currentfill}{rgb}{0.351511,0.635848,0.812641}%
\pgfsetfillcolor{currentfill}%
\pgfsetlinewidth{0.000000pt}%
\definecolor{currentstroke}{rgb}{0.000000,0.000000,0.000000}%
\pgfsetstrokecolor{currentstroke}%
\pgfsetstrokeopacity{0.000000}%
\pgfsetdash{}{0pt}%
\pgfpathmoveto{\pgfqpoint{1.561531in}{1.203886in}}%
\pgfpathlineto{\pgfqpoint{1.584352in}{1.202908in}}%
\pgfpathlineto{\pgfqpoint{1.595536in}{1.202468in}}%
\pgfpathlineto{\pgfqpoint{1.594392in}{1.173857in}}%
\pgfpathlineto{\pgfqpoint{1.567058in}{1.175010in}}%
\pgfpathlineto{\pgfqpoint{1.537445in}{1.176396in}}%
\pgfpathlineto{\pgfqpoint{1.539001in}{1.204970in}}%
\pgfpathlineto{\pgfqpoint{1.561531in}{1.203886in}}%
\pgfpathclose%
\pgfusepath{fill}%
\end{pgfscope}%
\begin{pgfscope}%
\pgfpathrectangle{\pgfqpoint{0.100000in}{0.100000in}}{\pgfqpoint{3.007045in}{1.925000in}}%
\pgfusepath{clip}%
\pgfsetbuttcap%
\pgfsetmiterjoin%
\definecolor{currentfill}{rgb}{0.516863,0.735748,0.860192}%
\pgfsetfillcolor{currentfill}%
\pgfsetlinewidth{0.000000pt}%
\definecolor{currentstroke}{rgb}{0.000000,0.000000,0.000000}%
\pgfsetstrokecolor{currentstroke}%
\pgfsetstrokeopacity{0.000000}%
\pgfsetdash{}{0pt}%
\pgfpathmoveto{\pgfqpoint{1.995501in}{1.164493in}}%
\pgfpathlineto{\pgfqpoint{1.997146in}{1.174029in}}%
\pgfpathlineto{\pgfqpoint{1.992899in}{1.184317in}}%
\pgfpathlineto{\pgfqpoint{1.996210in}{1.193778in}}%
\pgfpathlineto{\pgfqpoint{1.978770in}{1.193340in}}%
\pgfpathlineto{\pgfqpoint{1.977971in}{1.216245in}}%
\pgfpathlineto{\pgfqpoint{1.977807in}{1.222266in}}%
\pgfpathlineto{\pgfqpoint{2.000904in}{1.222841in}}%
\pgfpathlineto{\pgfqpoint{2.001133in}{1.217077in}}%
\pgfpathlineto{\pgfqpoint{2.013813in}{1.217404in}}%
\pgfpathlineto{\pgfqpoint{2.008276in}{1.212623in}}%
\pgfpathlineto{\pgfqpoint{2.009474in}{1.209217in}}%
\pgfpathlineto{\pgfqpoint{2.014487in}{1.208775in}}%
\pgfpathlineto{\pgfqpoint{2.024381in}{1.212989in}}%
\pgfpathlineto{\pgfqpoint{2.025601in}{1.191743in}}%
\pgfpathlineto{\pgfqpoint{2.029984in}{1.174625in}}%
\pgfpathlineto{\pgfqpoint{2.018445in}{1.173868in}}%
\pgfpathlineto{\pgfqpoint{2.019642in}{1.156899in}}%
\pgfpathlineto{\pgfqpoint{2.011513in}{1.150677in}}%
\pgfpathlineto{\pgfqpoint{2.002639in}{1.150292in}}%
\pgfpathlineto{\pgfqpoint{1.997266in}{1.151743in}}%
\pgfpathlineto{\pgfqpoint{1.995501in}{1.164493in}}%
\pgfpathclose%
\pgfusepath{fill}%
\end{pgfscope}%
\begin{pgfscope}%
\pgfpathrectangle{\pgfqpoint{0.100000in}{0.100000in}}{\pgfqpoint{3.007045in}{1.925000in}}%
\pgfusepath{clip}%
\pgfsetbuttcap%
\pgfsetmiterjoin%
\definecolor{currentfill}{rgb}{0.429020,0.687520,0.841246}%
\pgfsetfillcolor{currentfill}%
\pgfsetlinewidth{0.000000pt}%
\definecolor{currentstroke}{rgb}{0.000000,0.000000,0.000000}%
\pgfsetstrokecolor{currentstroke}%
\pgfsetstrokeopacity{0.000000}%
\pgfsetdash{}{0pt}%
\pgfpathmoveto{\pgfqpoint{2.664388in}{1.534234in}}%
\pgfpathlineto{\pgfqpoint{2.661128in}{1.541996in}}%
\pgfpathlineto{\pgfqpoint{2.664794in}{1.548806in}}%
\pgfpathlineto{\pgfqpoint{2.662584in}{1.552711in}}%
\pgfpathlineto{\pgfqpoint{2.655181in}{1.554501in}}%
\pgfpathlineto{\pgfqpoint{2.651982in}{1.562017in}}%
\pgfpathlineto{\pgfqpoint{2.659262in}{1.569668in}}%
\pgfpathlineto{\pgfqpoint{2.658006in}{1.573468in}}%
\pgfpathlineto{\pgfqpoint{2.668332in}{1.581558in}}%
\pgfpathlineto{\pgfqpoint{2.671705in}{1.586619in}}%
\pgfpathlineto{\pgfqpoint{2.673065in}{1.592568in}}%
\pgfpathlineto{\pgfqpoint{2.682395in}{1.607388in}}%
\pgfpathlineto{\pgfqpoint{2.696457in}{1.624732in}}%
\pgfpathlineto{\pgfqpoint{2.702880in}{1.630799in}}%
\pgfpathlineto{\pgfqpoint{2.709073in}{1.634394in}}%
\pgfpathlineto{\pgfqpoint{2.713963in}{1.634075in}}%
\pgfpathlineto{\pgfqpoint{2.718437in}{1.632224in}}%
\pgfpathlineto{\pgfqpoint{2.735305in}{1.582994in}}%
\pgfpathlineto{\pgfqpoint{2.735921in}{1.578293in}}%
\pgfpathlineto{\pgfqpoint{2.721573in}{1.573055in}}%
\pgfpathlineto{\pgfqpoint{2.712187in}{1.569616in}}%
\pgfpathlineto{\pgfqpoint{2.706539in}{1.571514in}}%
\pgfpathlineto{\pgfqpoint{2.716233in}{1.541048in}}%
\pgfpathlineto{\pgfqpoint{2.699199in}{1.524078in}}%
\pgfpathlineto{\pgfqpoint{2.688054in}{1.525089in}}%
\pgfpathlineto{\pgfqpoint{2.684161in}{1.539019in}}%
\pgfpathlineto{\pgfqpoint{2.672204in}{1.537694in}}%
\pgfpathlineto{\pgfqpoint{2.664388in}{1.534234in}}%
\pgfpathclose%
\pgfusepath{fill}%
\end{pgfscope}%
\begin{pgfscope}%
\pgfpathrectangle{\pgfqpoint{0.100000in}{0.100000in}}{\pgfqpoint{3.007045in}{1.925000in}}%
\pgfusepath{clip}%
\pgfsetbuttcap%
\pgfsetmiterjoin%
\definecolor{currentfill}{rgb}{0.361599,0.642737,0.816578}%
\pgfsetfillcolor{currentfill}%
\pgfsetlinewidth{0.000000pt}%
\definecolor{currentstroke}{rgb}{0.000000,0.000000,0.000000}%
\pgfsetstrokecolor{currentstroke}%
\pgfsetstrokeopacity{0.000000}%
\pgfsetdash{}{0pt}%
\pgfpathmoveto{\pgfqpoint{1.907064in}{1.059811in}}%
\pgfpathlineto{\pgfqpoint{1.900128in}{1.055674in}}%
\pgfpathlineto{\pgfqpoint{1.896036in}{1.061694in}}%
\pgfpathlineto{\pgfqpoint{1.872518in}{1.061815in}}%
\pgfpathlineto{\pgfqpoint{1.872670in}{1.072352in}}%
\pgfpathlineto{\pgfqpoint{1.871618in}{1.080502in}}%
\pgfpathlineto{\pgfqpoint{1.872048in}{1.113887in}}%
\pgfpathlineto{\pgfqpoint{1.883385in}{1.113543in}}%
\pgfpathlineto{\pgfqpoint{1.883415in}{1.107800in}}%
\pgfpathlineto{\pgfqpoint{1.894852in}{1.097146in}}%
\pgfpathlineto{\pgfqpoint{1.906297in}{1.097100in}}%
\pgfpathlineto{\pgfqpoint{1.906201in}{1.091563in}}%
\pgfpathlineto{\pgfqpoint{1.911673in}{1.091410in}}%
\pgfpathlineto{\pgfqpoint{1.917287in}{1.088655in}}%
\pgfpathlineto{\pgfqpoint{1.918382in}{1.070506in}}%
\pgfpathlineto{\pgfqpoint{1.906913in}{1.070484in}}%
\pgfpathlineto{\pgfqpoint{1.907064in}{1.059811in}}%
\pgfpathclose%
\pgfusepath{fill}%
\end{pgfscope}%
\begin{pgfscope}%
\pgfpathrectangle{\pgfqpoint{0.100000in}{0.100000in}}{\pgfqpoint{3.007045in}{1.925000in}}%
\pgfusepath{clip}%
\pgfsetbuttcap%
\pgfsetmiterjoin%
\definecolor{currentfill}{rgb}{0.296025,0.597955,0.790988}%
\pgfsetfillcolor{currentfill}%
\pgfsetlinewidth{0.000000pt}%
\definecolor{currentstroke}{rgb}{0.000000,0.000000,0.000000}%
\pgfsetstrokecolor{currentstroke}%
\pgfsetstrokeopacity{0.000000}%
\pgfsetdash{}{0pt}%
\pgfpathmoveto{\pgfqpoint{2.045514in}{0.912506in}}%
\pgfpathlineto{\pgfqpoint{2.038540in}{0.912838in}}%
\pgfpathlineto{\pgfqpoint{2.039932in}{0.907365in}}%
\pgfpathlineto{\pgfqpoint{2.033514in}{0.905154in}}%
\pgfpathlineto{\pgfqpoint{2.025623in}{0.904867in}}%
\pgfpathlineto{\pgfqpoint{2.023624in}{0.941619in}}%
\pgfpathlineto{\pgfqpoint{2.021688in}{0.947698in}}%
\pgfpathlineto{\pgfqpoint{2.026615in}{0.954215in}}%
\pgfpathlineto{\pgfqpoint{2.033532in}{0.960342in}}%
\pgfpathlineto{\pgfqpoint{2.034251in}{0.967797in}}%
\pgfpathlineto{\pgfqpoint{2.025516in}{0.974832in}}%
\pgfpathlineto{\pgfqpoint{2.028875in}{0.983778in}}%
\pgfpathlineto{\pgfqpoint{2.038888in}{0.984228in}}%
\pgfpathlineto{\pgfqpoint{2.039044in}{0.968405in}}%
\pgfpathlineto{\pgfqpoint{2.050739in}{0.968865in}}%
\pgfpathlineto{\pgfqpoint{2.059156in}{0.970494in}}%
\pgfpathlineto{\pgfqpoint{2.061897in}{0.966247in}}%
\pgfpathlineto{\pgfqpoint{2.062788in}{0.960749in}}%
\pgfpathlineto{\pgfqpoint{2.054510in}{0.958420in}}%
\pgfpathlineto{\pgfqpoint{2.060132in}{0.952292in}}%
\pgfpathlineto{\pgfqpoint{2.055729in}{0.949079in}}%
\pgfpathlineto{\pgfqpoint{2.054350in}{0.941105in}}%
\pgfpathlineto{\pgfqpoint{2.051897in}{0.936223in}}%
\pgfpathlineto{\pgfqpoint{2.052220in}{0.930837in}}%
\pgfpathlineto{\pgfqpoint{2.049440in}{0.927189in}}%
\pgfpathlineto{\pgfqpoint{2.042608in}{0.925600in}}%
\pgfpathlineto{\pgfqpoint{2.043371in}{0.916596in}}%
\pgfpathlineto{\pgfqpoint{2.045514in}{0.912506in}}%
\pgfpathclose%
\pgfusepath{fill}%
\end{pgfscope}%
\begin{pgfscope}%
\pgfpathrectangle{\pgfqpoint{0.100000in}{0.100000in}}{\pgfqpoint{3.007045in}{1.925000in}}%
\pgfusepath{clip}%
\pgfsetbuttcap%
\pgfsetmiterjoin%
\definecolor{currentfill}{rgb}{0.376732,0.653072,0.822484}%
\pgfsetfillcolor{currentfill}%
\pgfsetlinewidth{0.000000pt}%
\definecolor{currentstroke}{rgb}{0.000000,0.000000,0.000000}%
\pgfsetstrokecolor{currentstroke}%
\pgfsetstrokeopacity{0.000000}%
\pgfsetdash{}{0pt}%
\pgfpathmoveto{\pgfqpoint{2.512057in}{0.734800in}}%
\pgfpathlineto{\pgfqpoint{2.509344in}{0.740112in}}%
\pgfpathlineto{\pgfqpoint{2.485433in}{0.742115in}}%
\pgfpathlineto{\pgfqpoint{2.478955in}{0.739426in}}%
\pgfpathlineto{\pgfqpoint{2.475118in}{0.743801in}}%
\pgfpathlineto{\pgfqpoint{2.476049in}{0.748944in}}%
\pgfpathlineto{\pgfqpoint{2.469526in}{0.751792in}}%
\pgfpathlineto{\pgfqpoint{2.460691in}{0.751658in}}%
\pgfpathlineto{\pgfqpoint{2.463033in}{0.756433in}}%
\pgfpathlineto{\pgfqpoint{2.469397in}{0.761860in}}%
\pgfpathlineto{\pgfqpoint{2.467299in}{0.764175in}}%
\pgfpathlineto{\pgfqpoint{2.470222in}{0.769671in}}%
\pgfpathlineto{\pgfqpoint{2.475814in}{0.775674in}}%
\pgfpathlineto{\pgfqpoint{2.482589in}{0.774200in}}%
\pgfpathlineto{\pgfqpoint{2.485475in}{0.775665in}}%
\pgfpathlineto{\pgfqpoint{2.488225in}{0.782556in}}%
\pgfpathlineto{\pgfqpoint{2.500186in}{0.786765in}}%
\pgfpathlineto{\pgfqpoint{2.505569in}{0.784566in}}%
\pgfpathlineto{\pgfqpoint{2.516660in}{0.795332in}}%
\pgfpathlineto{\pgfqpoint{2.519777in}{0.793317in}}%
\pgfpathlineto{\pgfqpoint{2.519904in}{0.788430in}}%
\pgfpathlineto{\pgfqpoint{2.525012in}{0.782861in}}%
\pgfpathlineto{\pgfqpoint{2.526324in}{0.773753in}}%
\pgfpathlineto{\pgfqpoint{2.529482in}{0.767093in}}%
\pgfpathlineto{\pgfqpoint{2.522083in}{0.760192in}}%
\pgfpathlineto{\pgfqpoint{2.523508in}{0.752566in}}%
\pgfpathlineto{\pgfqpoint{2.528796in}{0.747623in}}%
\pgfpathlineto{\pgfqpoint{2.530406in}{0.743490in}}%
\pgfpathlineto{\pgfqpoint{2.512057in}{0.734800in}}%
\pgfpathclose%
\pgfusepath{fill}%
\end{pgfscope}%
\begin{pgfscope}%
\pgfpathrectangle{\pgfqpoint{0.100000in}{0.100000in}}{\pgfqpoint{3.007045in}{1.925000in}}%
\pgfusepath{clip}%
\pgfsetbuttcap%
\pgfsetmiterjoin%
\definecolor{currentfill}{rgb}{0.187266,0.500992,0.739608}%
\pgfsetfillcolor{currentfill}%
\pgfsetlinewidth{0.000000pt}%
\definecolor{currentstroke}{rgb}{0.000000,0.000000,0.000000}%
\pgfsetstrokecolor{currentstroke}%
\pgfsetstrokeopacity{0.000000}%
\pgfsetdash{}{0pt}%
\pgfpathmoveto{\pgfqpoint{1.215312in}{0.701261in}}%
\pgfpathlineto{\pgfqpoint{1.212900in}{0.676457in}}%
\pgfpathlineto{\pgfqpoint{1.207983in}{0.628535in}}%
\pgfpathlineto{\pgfqpoint{1.205698in}{0.612940in}}%
\pgfpathlineto{\pgfqpoint{1.201869in}{0.611048in}}%
\pgfpathlineto{\pgfqpoint{1.194762in}{0.619908in}}%
\pgfpathlineto{\pgfqpoint{1.189798in}{0.624237in}}%
\pgfpathlineto{\pgfqpoint{1.180001in}{0.628883in}}%
\pgfpathlineto{\pgfqpoint{1.179940in}{0.630968in}}%
\pgfpathlineto{\pgfqpoint{1.171824in}{0.638589in}}%
\pgfpathlineto{\pgfqpoint{1.169910in}{0.645050in}}%
\pgfpathlineto{\pgfqpoint{1.160977in}{0.651771in}}%
\pgfpathlineto{\pgfqpoint{1.152306in}{0.666060in}}%
\pgfpathlineto{\pgfqpoint{1.145301in}{0.669338in}}%
\pgfpathlineto{\pgfqpoint{1.138853in}{0.674660in}}%
\pgfpathlineto{\pgfqpoint{1.131339in}{0.693210in}}%
\pgfpathlineto{\pgfqpoint{1.123511in}{0.697668in}}%
\pgfpathlineto{\pgfqpoint{1.080529in}{0.703340in}}%
\pgfpathlineto{\pgfqpoint{1.087735in}{0.757358in}}%
\pgfpathlineto{\pgfqpoint{1.089302in}{0.768831in}}%
\pgfpathlineto{\pgfqpoint{1.112021in}{0.765725in}}%
\pgfpathlineto{\pgfqpoint{1.112390in}{0.768586in}}%
\pgfpathlineto{\pgfqpoint{1.144512in}{0.779815in}}%
\pgfpathlineto{\pgfqpoint{1.141737in}{0.773832in}}%
\pgfpathlineto{\pgfqpoint{1.133803in}{0.710900in}}%
\pgfpathlineto{\pgfqpoint{1.186737in}{0.704479in}}%
\pgfpathlineto{\pgfqpoint{1.215312in}{0.701261in}}%
\pgfpathclose%
\pgfusepath{fill}%
\end{pgfscope}%
\begin{pgfscope}%
\pgfpathrectangle{\pgfqpoint{0.100000in}{0.100000in}}{\pgfqpoint{3.007045in}{1.925000in}}%
\pgfusepath{clip}%
\pgfsetbuttcap%
\pgfsetmiterjoin%
\definecolor{currentfill}{rgb}{0.256286,0.570012,0.775163}%
\pgfsetfillcolor{currentfill}%
\pgfsetlinewidth{0.000000pt}%
\definecolor{currentstroke}{rgb}{0.000000,0.000000,0.000000}%
\pgfsetstrokecolor{currentstroke}%
\pgfsetstrokeopacity{0.000000}%
\pgfsetdash{}{0pt}%
\pgfpathmoveto{\pgfqpoint{2.321001in}{0.750113in}}%
\pgfpathlineto{\pgfqpoint{2.317396in}{0.757985in}}%
\pgfpathlineto{\pgfqpoint{2.335233in}{0.759815in}}%
\pgfpathlineto{\pgfqpoint{2.333138in}{0.783244in}}%
\pgfpathlineto{\pgfqpoint{2.352426in}{0.785269in}}%
\pgfpathlineto{\pgfqpoint{2.352032in}{0.771233in}}%
\pgfpathlineto{\pgfqpoint{2.354579in}{0.763090in}}%
\pgfpathlineto{\pgfqpoint{2.361889in}{0.757434in}}%
\pgfpathlineto{\pgfqpoint{2.367977in}{0.755519in}}%
\pgfpathlineto{\pgfqpoint{2.360665in}{0.742287in}}%
\pgfpathlineto{\pgfqpoint{2.350162in}{0.739256in}}%
\pgfpathlineto{\pgfqpoint{2.349352in}{0.736500in}}%
\pgfpathlineto{\pgfqpoint{2.351237in}{0.719223in}}%
\pgfpathlineto{\pgfqpoint{2.336528in}{0.717430in}}%
\pgfpathlineto{\pgfqpoint{2.329526in}{0.710196in}}%
\pgfpathlineto{\pgfqpoint{2.330327in}{0.705545in}}%
\pgfpathlineto{\pgfqpoint{2.323124in}{0.704669in}}%
\pgfpathlineto{\pgfqpoint{2.318447in}{0.709939in}}%
\pgfpathlineto{\pgfqpoint{2.309942in}{0.708924in}}%
\pgfpathlineto{\pgfqpoint{2.308027in}{0.714591in}}%
\pgfpathlineto{\pgfqpoint{2.306778in}{0.726142in}}%
\pgfpathlineto{\pgfqpoint{2.312554in}{0.726822in}}%
\pgfpathlineto{\pgfqpoint{2.311941in}{0.730610in}}%
\pgfpathlineto{\pgfqpoint{2.327177in}{0.732545in}}%
\pgfpathlineto{\pgfqpoint{2.330141in}{0.735377in}}%
\pgfpathlineto{\pgfqpoint{2.323388in}{0.743508in}}%
\pgfpathlineto{\pgfqpoint{2.321001in}{0.750113in}}%
\pgfpathclose%
\pgfusepath{fill}%
\end{pgfscope}%
\begin{pgfscope}%
\pgfpathrectangle{\pgfqpoint{0.100000in}{0.100000in}}{\pgfqpoint{3.007045in}{1.925000in}}%
\pgfusepath{clip}%
\pgfsetbuttcap%
\pgfsetmiterjoin%
\definecolor{currentfill}{rgb}{0.361599,0.642737,0.816578}%
\pgfsetfillcolor{currentfill}%
\pgfsetlinewidth{0.000000pt}%
\definecolor{currentstroke}{rgb}{0.000000,0.000000,0.000000}%
\pgfsetstrokecolor{currentstroke}%
\pgfsetstrokeopacity{0.000000}%
\pgfsetdash{}{0pt}%
\pgfpathmoveto{\pgfqpoint{2.252087in}{0.743598in}}%
\pgfpathlineto{\pgfqpoint{2.250388in}{0.740313in}}%
\pgfpathlineto{\pgfqpoint{2.233682in}{0.738645in}}%
\pgfpathlineto{\pgfqpoint{2.233994in}{0.735748in}}%
\pgfpathlineto{\pgfqpoint{2.222669in}{0.734886in}}%
\pgfpathlineto{\pgfqpoint{2.217078in}{0.734332in}}%
\pgfpathlineto{\pgfqpoint{2.215985in}{0.745794in}}%
\pgfpathlineto{\pgfqpoint{2.223920in}{0.746448in}}%
\pgfpathlineto{\pgfqpoint{2.222615in}{0.760540in}}%
\pgfpathlineto{\pgfqpoint{2.217222in}{0.763282in}}%
\pgfpathlineto{\pgfqpoint{2.213720in}{0.767453in}}%
\pgfpathlineto{\pgfqpoint{2.213183in}{0.772767in}}%
\pgfpathlineto{\pgfqpoint{2.204331in}{0.776956in}}%
\pgfpathlineto{\pgfqpoint{2.197889in}{0.784340in}}%
\pgfpathlineto{\pgfqpoint{2.198442in}{0.789193in}}%
\pgfpathlineto{\pgfqpoint{2.202677in}{0.796254in}}%
\pgfpathlineto{\pgfqpoint{2.207356in}{0.797594in}}%
\pgfpathlineto{\pgfqpoint{2.209929in}{0.801633in}}%
\pgfpathlineto{\pgfqpoint{2.210327in}{0.807475in}}%
\pgfpathlineto{\pgfqpoint{2.213109in}{0.813406in}}%
\pgfpathlineto{\pgfqpoint{2.206348in}{0.816022in}}%
\pgfpathlineto{\pgfqpoint{2.204539in}{0.819210in}}%
\pgfpathlineto{\pgfqpoint{2.202564in}{0.841744in}}%
\pgfpathlineto{\pgfqpoint{2.202469in}{0.842698in}}%
\pgfpathlineto{\pgfqpoint{2.231240in}{0.844683in}}%
\pgfpathlineto{\pgfqpoint{2.236928in}{0.845086in}}%
\pgfpathlineto{\pgfqpoint{2.247672in}{0.832584in}}%
\pgfpathlineto{\pgfqpoint{2.246429in}{0.828970in}}%
\pgfpathlineto{\pgfqpoint{2.249130in}{0.825060in}}%
\pgfpathlineto{\pgfqpoint{2.254073in}{0.825883in}}%
\pgfpathlineto{\pgfqpoint{2.257720in}{0.820289in}}%
\pgfpathlineto{\pgfqpoint{2.262319in}{0.816941in}}%
\pgfpathlineto{\pgfqpoint{2.263640in}{0.812852in}}%
\pgfpathlineto{\pgfqpoint{2.258979in}{0.805773in}}%
\pgfpathlineto{\pgfqpoint{2.255427in}{0.799367in}}%
\pgfpathlineto{\pgfqpoint{2.257385in}{0.796619in}}%
\pgfpathlineto{\pgfqpoint{2.248013in}{0.785562in}}%
\pgfpathlineto{\pgfqpoint{2.249772in}{0.779385in}}%
\pgfpathlineto{\pgfqpoint{2.244576in}{0.775293in}}%
\pgfpathlineto{\pgfqpoint{2.242492in}{0.770671in}}%
\pgfpathlineto{\pgfqpoint{2.243973in}{0.753410in}}%
\pgfpathlineto{\pgfqpoint{2.247289in}{0.747109in}}%
\pgfpathlineto{\pgfqpoint{2.252087in}{0.743598in}}%
\pgfpathclose%
\pgfusepath{fill}%
\end{pgfscope}%
\begin{pgfscope}%
\pgfpathrectangle{\pgfqpoint{0.100000in}{0.100000in}}{\pgfqpoint{3.007045in}{1.925000in}}%
\pgfusepath{clip}%
\pgfsetbuttcap%
\pgfsetmiterjoin%
\definecolor{currentfill}{rgb}{0.472941,0.711634,0.850719}%
\pgfsetfillcolor{currentfill}%
\pgfsetlinewidth{0.000000pt}%
\definecolor{currentstroke}{rgb}{0.000000,0.000000,0.000000}%
\pgfsetstrokecolor{currentstroke}%
\pgfsetstrokeopacity{0.000000}%
\pgfsetdash{}{0pt}%
\pgfpathmoveto{\pgfqpoint{1.710630in}{1.795591in}}%
\pgfpathlineto{\pgfqpoint{1.710357in}{1.776679in}}%
\pgfpathlineto{\pgfqpoint{1.710937in}{1.765107in}}%
\pgfpathlineto{\pgfqpoint{1.676648in}{1.765752in}}%
\pgfpathlineto{\pgfqpoint{1.677958in}{1.760345in}}%
\pgfpathlineto{\pgfqpoint{1.676638in}{1.755236in}}%
\pgfpathlineto{\pgfqpoint{1.678294in}{1.748130in}}%
\pgfpathlineto{\pgfqpoint{1.677065in}{1.742344in}}%
\pgfpathlineto{\pgfqpoint{1.643166in}{1.743370in}}%
\pgfpathlineto{\pgfqpoint{1.625869in}{1.743967in}}%
\pgfpathlineto{\pgfqpoint{1.626307in}{1.755616in}}%
\pgfpathlineto{\pgfqpoint{1.625579in}{1.767286in}}%
\pgfpathlineto{\pgfqpoint{1.596746in}{1.768470in}}%
\pgfpathlineto{\pgfqpoint{1.595955in}{1.780137in}}%
\pgfpathlineto{\pgfqpoint{1.596789in}{1.798977in}}%
\pgfpathlineto{\pgfqpoint{1.638516in}{1.797389in}}%
\pgfpathlineto{\pgfqpoint{1.674481in}{1.796361in}}%
\pgfpathlineto{\pgfqpoint{1.710630in}{1.795591in}}%
\pgfpathclose%
\pgfusepath{fill}%
\end{pgfscope}%
\begin{pgfscope}%
\pgfpathrectangle{\pgfqpoint{0.100000in}{0.100000in}}{\pgfqpoint{3.007045in}{1.925000in}}%
\pgfusepath{clip}%
\pgfsetbuttcap%
\pgfsetmiterjoin%
\definecolor{currentfill}{rgb}{0.031373,0.196355,0.431665}%
\pgfsetfillcolor{currentfill}%
\pgfsetlinewidth{0.000000pt}%
\definecolor{currentstroke}{rgb}{0.000000,0.000000,0.000000}%
\pgfsetstrokecolor{currentstroke}%
\pgfsetstrokeopacity{0.000000}%
\pgfsetdash{}{0pt}%
\pgfpathmoveto{\pgfqpoint{1.521148in}{0.313814in}}%
\pgfpathlineto{\pgfqpoint{1.519248in}{0.324225in}}%
\pgfpathlineto{\pgfqpoint{1.516036in}{0.332359in}}%
\pgfpathlineto{\pgfqpoint{1.509870in}{0.338738in}}%
\pgfpathlineto{\pgfqpoint{1.505834in}{0.347871in}}%
\pgfpathlineto{\pgfqpoint{1.507664in}{0.355223in}}%
\pgfpathlineto{\pgfqpoint{1.503531in}{0.366333in}}%
\pgfpathlineto{\pgfqpoint{1.505470in}{0.376130in}}%
\pgfpathlineto{\pgfqpoint{1.502792in}{0.376996in}}%
\pgfpathlineto{\pgfqpoint{1.502216in}{0.384359in}}%
\pgfpathlineto{\pgfqpoint{1.491739in}{0.388953in}}%
\pgfpathlineto{\pgfqpoint{1.486685in}{0.396463in}}%
\pgfpathlineto{\pgfqpoint{1.482967in}{0.398437in}}%
\pgfpathlineto{\pgfqpoint{1.482372in}{0.405110in}}%
\pgfpathlineto{\pgfqpoint{1.476970in}{0.411788in}}%
\pgfpathlineto{\pgfqpoint{1.472047in}{0.422119in}}%
\pgfpathlineto{\pgfqpoint{1.464650in}{0.426083in}}%
\pgfpathlineto{\pgfqpoint{1.470461in}{0.425819in}}%
\pgfpathlineto{\pgfqpoint{1.512921in}{0.424052in}}%
\pgfpathlineto{\pgfqpoint{1.512666in}{0.412373in}}%
\pgfpathlineto{\pgfqpoint{1.547367in}{0.412624in}}%
\pgfpathlineto{\pgfqpoint{1.545690in}{0.365433in}}%
\pgfpathlineto{\pgfqpoint{1.560356in}{0.365181in}}%
\pgfpathlineto{\pgfqpoint{1.563594in}{0.356706in}}%
\pgfpathlineto{\pgfqpoint{1.564709in}{0.344532in}}%
\pgfpathlineto{\pgfqpoint{1.567623in}{0.344460in}}%
\pgfpathlineto{\pgfqpoint{1.566652in}{0.326194in}}%
\pgfpathlineto{\pgfqpoint{1.534731in}{0.327636in}}%
\pgfpathlineto{\pgfqpoint{1.530987in}{0.320341in}}%
\pgfpathlineto{\pgfqpoint{1.521148in}{0.313814in}}%
\pgfpathclose%
\pgfusepath{fill}%
\end{pgfscope}%
\begin{pgfscope}%
\pgfpathrectangle{\pgfqpoint{0.100000in}{0.100000in}}{\pgfqpoint{3.007045in}{1.925000in}}%
\pgfusepath{clip}%
\pgfsetbuttcap%
\pgfsetmiterjoin%
\definecolor{currentfill}{rgb}{0.429020,0.687520,0.841246}%
\pgfsetfillcolor{currentfill}%
\pgfsetlinewidth{0.000000pt}%
\definecolor{currentstroke}{rgb}{0.000000,0.000000,0.000000}%
\pgfsetstrokecolor{currentstroke}%
\pgfsetstrokeopacity{0.000000}%
\pgfsetdash{}{0pt}%
\pgfpathmoveto{\pgfqpoint{2.503300in}{0.656930in}}%
\pgfpathlineto{\pgfqpoint{2.498304in}{0.661191in}}%
\pgfpathlineto{\pgfqpoint{2.497961in}{0.669319in}}%
\pgfpathlineto{\pgfqpoint{2.503124in}{0.674785in}}%
\pgfpathlineto{\pgfqpoint{2.508359in}{0.677599in}}%
\pgfpathlineto{\pgfqpoint{2.505019in}{0.681519in}}%
\pgfpathlineto{\pgfqpoint{2.511887in}{0.680666in}}%
\pgfpathlineto{\pgfqpoint{2.518569in}{0.682840in}}%
\pgfpathlineto{\pgfqpoint{2.523188in}{0.681053in}}%
\pgfpathlineto{\pgfqpoint{2.528105in}{0.691406in}}%
\pgfpathlineto{\pgfqpoint{2.524353in}{0.695341in}}%
\pgfpathlineto{\pgfqpoint{2.529648in}{0.698799in}}%
\pgfpathlineto{\pgfqpoint{2.532623in}{0.707455in}}%
\pgfpathlineto{\pgfqpoint{2.536542in}{0.704088in}}%
\pgfpathlineto{\pgfqpoint{2.545964in}{0.706161in}}%
\pgfpathlineto{\pgfqpoint{2.550608in}{0.701257in}}%
\pgfpathlineto{\pgfqpoint{2.551853in}{0.698652in}}%
\pgfpathlineto{\pgfqpoint{2.548410in}{0.688938in}}%
\pgfpathlineto{\pgfqpoint{2.550045in}{0.680517in}}%
\pgfpathlineto{\pgfqpoint{2.543234in}{0.669808in}}%
\pgfpathlineto{\pgfqpoint{2.542300in}{0.662555in}}%
\pgfpathlineto{\pgfqpoint{2.533990in}{0.667284in}}%
\pgfpathlineto{\pgfqpoint{2.521337in}{0.669031in}}%
\pgfpathlineto{\pgfqpoint{2.514003in}{0.660961in}}%
\pgfpathlineto{\pgfqpoint{2.506695in}{0.661778in}}%
\pgfpathlineto{\pgfqpoint{2.503300in}{0.656930in}}%
\pgfpathclose%
\pgfusepath{fill}%
\end{pgfscope}%
\begin{pgfscope}%
\pgfpathrectangle{\pgfqpoint{0.100000in}{0.100000in}}{\pgfqpoint{3.007045in}{1.925000in}}%
\pgfusepath{clip}%
\pgfsetbuttcap%
\pgfsetmiterjoin%
\definecolor{currentfill}{rgb}{0.376732,0.653072,0.822484}%
\pgfsetfillcolor{currentfill}%
\pgfsetlinewidth{0.000000pt}%
\definecolor{currentstroke}{rgb}{0.000000,0.000000,0.000000}%
\pgfsetstrokecolor{currentstroke}%
\pgfsetstrokeopacity{0.000000}%
\pgfsetdash{}{0pt}%
\pgfpathmoveto{\pgfqpoint{1.326529in}{1.324213in}}%
\pgfpathlineto{\pgfqpoint{1.333387in}{1.392952in}}%
\pgfpathlineto{\pgfqpoint{1.335923in}{1.418503in}}%
\pgfpathlineto{\pgfqpoint{1.386507in}{1.413753in}}%
\pgfpathlineto{\pgfqpoint{1.396545in}{1.412828in}}%
\pgfpathlineto{\pgfqpoint{1.396236in}{1.398628in}}%
\pgfpathlineto{\pgfqpoint{1.394290in}{1.375922in}}%
\pgfpathlineto{\pgfqpoint{1.395452in}{1.375827in}}%
\pgfpathlineto{\pgfqpoint{1.393848in}{1.352944in}}%
\pgfpathlineto{\pgfqpoint{1.396518in}{1.346955in}}%
\pgfpathlineto{\pgfqpoint{1.395435in}{1.334971in}}%
\pgfpathlineto{\pgfqpoint{1.396817in}{1.327688in}}%
\pgfpathlineto{\pgfqpoint{1.395381in}{1.309582in}}%
\pgfpathlineto{\pgfqpoint{1.359322in}{1.312674in}}%
\pgfpathlineto{\pgfqpoint{1.358996in}{1.309848in}}%
\pgfpathlineto{\pgfqpoint{1.325420in}{1.312996in}}%
\pgfpathlineto{\pgfqpoint{1.326529in}{1.324213in}}%
\pgfpathclose%
\pgfusepath{fill}%
\end{pgfscope}%
\begin{pgfscope}%
\pgfpathrectangle{\pgfqpoint{0.100000in}{0.100000in}}{\pgfqpoint{3.007045in}{1.925000in}}%
\pgfusepath{clip}%
\pgfsetbuttcap%
\pgfsetmiterjoin%
\definecolor{currentfill}{rgb}{0.270804,0.580730,0.781146}%
\pgfsetfillcolor{currentfill}%
\pgfsetlinewidth{0.000000pt}%
\definecolor{currentstroke}{rgb}{0.000000,0.000000,0.000000}%
\pgfsetstrokecolor{currentstroke}%
\pgfsetstrokeopacity{0.000000}%
\pgfsetdash{}{0pt}%
\pgfpathmoveto{\pgfqpoint{2.011147in}{0.717467in}}%
\pgfpathlineto{\pgfqpoint{2.010976in}{0.714602in}}%
\pgfpathlineto{\pgfqpoint{2.003446in}{0.714212in}}%
\pgfpathlineto{\pgfqpoint{2.003803in}{0.705398in}}%
\pgfpathlineto{\pgfqpoint{1.998667in}{0.705092in}}%
\pgfpathlineto{\pgfqpoint{1.997671in}{0.713980in}}%
\pgfpathlineto{\pgfqpoint{1.993233in}{0.713829in}}%
\pgfpathlineto{\pgfqpoint{1.988496in}{0.714303in}}%
\pgfpathlineto{\pgfqpoint{1.986961in}{0.717838in}}%
\pgfpathlineto{\pgfqpoint{1.991838in}{0.723800in}}%
\pgfpathlineto{\pgfqpoint{1.985938in}{0.725835in}}%
\pgfpathlineto{\pgfqpoint{1.986138in}{0.729844in}}%
\pgfpathlineto{\pgfqpoint{1.990829in}{0.734444in}}%
\pgfpathlineto{\pgfqpoint{1.989679in}{0.740203in}}%
\pgfpathlineto{\pgfqpoint{1.985200in}{0.741858in}}%
\pgfpathlineto{\pgfqpoint{1.988976in}{0.750911in}}%
\pgfpathlineto{\pgfqpoint{1.988710in}{0.756370in}}%
\pgfpathlineto{\pgfqpoint{1.985739in}{0.761415in}}%
\pgfpathlineto{\pgfqpoint{1.985442in}{0.766767in}}%
\pgfpathlineto{\pgfqpoint{1.980431in}{0.778978in}}%
\pgfpathlineto{\pgfqpoint{1.985316in}{0.780979in}}%
\pgfpathlineto{\pgfqpoint{1.980623in}{0.785127in}}%
\pgfpathlineto{\pgfqpoint{1.984613in}{0.793228in}}%
\pgfpathlineto{\pgfqpoint{1.992182in}{0.792843in}}%
\pgfpathlineto{\pgfqpoint{1.988015in}{0.799119in}}%
\pgfpathlineto{\pgfqpoint{1.991139in}{0.803150in}}%
\pgfpathlineto{\pgfqpoint{1.986772in}{0.806255in}}%
\pgfpathlineto{\pgfqpoint{1.996950in}{0.809911in}}%
\pgfpathlineto{\pgfqpoint{1.998299in}{0.813452in}}%
\pgfpathlineto{\pgfqpoint{1.993621in}{0.816052in}}%
\pgfpathlineto{\pgfqpoint{1.993574in}{0.816135in}}%
\pgfpathlineto{\pgfqpoint{2.009788in}{0.816800in}}%
\pgfpathlineto{\pgfqpoint{2.010153in}{0.808128in}}%
\pgfpathlineto{\pgfqpoint{2.021640in}{0.808564in}}%
\pgfpathlineto{\pgfqpoint{2.022165in}{0.797050in}}%
\pgfpathlineto{\pgfqpoint{2.023674in}{0.765200in}}%
\pgfpathlineto{\pgfqpoint{2.018021in}{0.764899in}}%
\pgfpathlineto{\pgfqpoint{2.018223in}{0.761023in}}%
\pgfpathlineto{\pgfqpoint{2.009327in}{0.760562in}}%
\pgfpathlineto{\pgfqpoint{2.013096in}{0.749239in}}%
\pgfpathlineto{\pgfqpoint{2.013725in}{0.737732in}}%
\pgfpathlineto{\pgfqpoint{2.008261in}{0.731716in}}%
\pgfpathlineto{\pgfqpoint{2.011096in}{0.729886in}}%
\pgfpathlineto{\pgfqpoint{2.011147in}{0.717467in}}%
\pgfpathclose%
\pgfusepath{fill}%
\end{pgfscope}%
\begin{pgfscope}%
\pgfpathrectangle{\pgfqpoint{0.100000in}{0.100000in}}{\pgfqpoint{3.007045in}{1.925000in}}%
\pgfusepath{clip}%
\pgfsetbuttcap%
\pgfsetmiterjoin%
\definecolor{currentfill}{rgb}{0.662053,0.810196,0.897209}%
\pgfsetfillcolor{currentfill}%
\pgfsetlinewidth{0.000000pt}%
\definecolor{currentstroke}{rgb}{0.000000,0.000000,0.000000}%
\pgfsetstrokecolor{currentstroke}%
\pgfsetstrokeopacity{0.000000}%
\pgfsetdash{}{0pt}%
\pgfpathmoveto{\pgfqpoint{0.416964in}{1.700001in}}%
\pgfpathlineto{\pgfqpoint{0.427465in}{1.723088in}}%
\pgfpathlineto{\pgfqpoint{0.429108in}{1.730577in}}%
\pgfpathlineto{\pgfqpoint{0.441718in}{1.753288in}}%
\pgfpathlineto{\pgfqpoint{0.445758in}{1.765737in}}%
\pgfpathlineto{\pgfqpoint{0.453664in}{1.784658in}}%
\pgfpathlineto{\pgfqpoint{0.454831in}{1.792649in}}%
\pgfpathlineto{\pgfqpoint{0.465616in}{1.788403in}}%
\pgfpathlineto{\pgfqpoint{0.481516in}{1.783693in}}%
\pgfpathlineto{\pgfqpoint{0.480074in}{1.779212in}}%
\pgfpathlineto{\pgfqpoint{0.475846in}{1.775523in}}%
\pgfpathlineto{\pgfqpoint{0.475014in}{1.769793in}}%
\pgfpathlineto{\pgfqpoint{0.469874in}{1.763357in}}%
\pgfpathlineto{\pgfqpoint{0.465504in}{1.749616in}}%
\pgfpathlineto{\pgfqpoint{0.451249in}{1.754244in}}%
\pgfpathlineto{\pgfqpoint{0.448349in}{1.745454in}}%
\pgfpathlineto{\pgfqpoint{0.451037in}{1.744559in}}%
\pgfpathlineto{\pgfqpoint{0.444068in}{1.723357in}}%
\pgfpathlineto{\pgfqpoint{0.449171in}{1.720447in}}%
\pgfpathlineto{\pgfqpoint{0.443505in}{1.702226in}}%
\pgfpathlineto{\pgfqpoint{0.437975in}{1.704042in}}%
\pgfpathlineto{\pgfqpoint{0.436719in}{1.699407in}}%
\pgfpathlineto{\pgfqpoint{0.432427in}{1.695452in}}%
\pgfpathlineto{\pgfqpoint{0.416964in}{1.700001in}}%
\pgfpathclose%
\pgfusepath{fill}%
\end{pgfscope}%
\begin{pgfscope}%
\pgfpathrectangle{\pgfqpoint{0.100000in}{0.100000in}}{\pgfqpoint{3.007045in}{1.925000in}}%
\pgfusepath{clip}%
\pgfsetbuttcap%
\pgfsetmiterjoin%
\definecolor{currentfill}{rgb}{0.093272,0.396878,0.673664}%
\pgfsetfillcolor{currentfill}%
\pgfsetlinewidth{0.000000pt}%
\definecolor{currentstroke}{rgb}{0.000000,0.000000,0.000000}%
\pgfsetstrokecolor{currentstroke}%
\pgfsetstrokeopacity{0.000000}%
\pgfsetdash{}{0pt}%
\pgfpathmoveto{\pgfqpoint{0.377147in}{0.493245in}}%
\pgfpathlineto{\pgfqpoint{0.376520in}{0.497403in}}%
\pgfpathlineto{\pgfqpoint{0.377477in}{0.497571in}}%
\pgfpathlineto{\pgfqpoint{0.377147in}{0.493245in}}%
\pgfpathclose%
\pgfusepath{fill}%
\end{pgfscope}%
\begin{pgfscope}%
\pgfpathrectangle{\pgfqpoint{0.100000in}{0.100000in}}{\pgfqpoint{3.007045in}{1.925000in}}%
\pgfusepath{clip}%
\pgfsetbuttcap%
\pgfsetmiterjoin%
\definecolor{currentfill}{rgb}{0.093272,0.396878,0.673664}%
\pgfsetfillcolor{currentfill}%
\pgfsetlinewidth{0.000000pt}%
\definecolor{currentstroke}{rgb}{0.000000,0.000000,0.000000}%
\pgfsetstrokecolor{currentstroke}%
\pgfsetstrokeopacity{0.000000}%
\pgfsetdash{}{0pt}%
\pgfpathmoveto{\pgfqpoint{0.405385in}{0.465085in}}%
\pgfpathlineto{\pgfqpoint{0.405017in}{0.468510in}}%
\pgfpathlineto{\pgfqpoint{0.406749in}{0.467215in}}%
\pgfpathlineto{\pgfqpoint{0.407796in}{0.467652in}}%
\pgfpathlineto{\pgfqpoint{0.407185in}{0.472036in}}%
\pgfpathlineto{\pgfqpoint{0.408305in}{0.472399in}}%
\pgfpathlineto{\pgfqpoint{0.410270in}{0.469591in}}%
\pgfpathlineto{\pgfqpoint{0.409867in}{0.468623in}}%
\pgfpathlineto{\pgfqpoint{0.410756in}{0.466052in}}%
\pgfpathlineto{\pgfqpoint{0.406364in}{0.465975in}}%
\pgfpathlineto{\pgfqpoint{0.405385in}{0.465085in}}%
\pgfpathclose%
\pgfusepath{fill}%
\end{pgfscope}%
\begin{pgfscope}%
\pgfpathrectangle{\pgfqpoint{0.100000in}{0.100000in}}{\pgfqpoint{3.007045in}{1.925000in}}%
\pgfusepath{clip}%
\pgfsetbuttcap%
\pgfsetmiterjoin%
\definecolor{currentfill}{rgb}{0.093272,0.396878,0.673664}%
\pgfsetfillcolor{currentfill}%
\pgfsetlinewidth{0.000000pt}%
\definecolor{currentstroke}{rgb}{0.000000,0.000000,0.000000}%
\pgfsetstrokecolor{currentstroke}%
\pgfsetstrokeopacity{0.000000}%
\pgfsetdash{}{0pt}%
\pgfpathmoveto{\pgfqpoint{0.415072in}{0.452560in}}%
\pgfpathlineto{\pgfqpoint{0.414645in}{0.455401in}}%
\pgfpathlineto{\pgfqpoint{0.419140in}{0.455415in}}%
\pgfpathlineto{\pgfqpoint{0.420979in}{0.457297in}}%
\pgfpathlineto{\pgfqpoint{0.422131in}{0.456956in}}%
\pgfpathlineto{\pgfqpoint{0.423763in}{0.455028in}}%
\pgfpathlineto{\pgfqpoint{0.422007in}{0.454672in}}%
\pgfpathlineto{\pgfqpoint{0.421045in}{0.453407in}}%
\pgfpathlineto{\pgfqpoint{0.422643in}{0.451021in}}%
\pgfpathlineto{\pgfqpoint{0.423045in}{0.449284in}}%
\pgfpathlineto{\pgfqpoint{0.421244in}{0.449179in}}%
\pgfpathlineto{\pgfqpoint{0.420180in}{0.450490in}}%
\pgfpathlineto{\pgfqpoint{0.417429in}{0.452432in}}%
\pgfpathlineto{\pgfqpoint{0.415072in}{0.452560in}}%
\pgfpathclose%
\pgfusepath{fill}%
\end{pgfscope}%
\begin{pgfscope}%
\pgfpathrectangle{\pgfqpoint{0.100000in}{0.100000in}}{\pgfqpoint{3.007045in}{1.925000in}}%
\pgfusepath{clip}%
\pgfsetbuttcap%
\pgfsetmiterjoin%
\definecolor{currentfill}{rgb}{0.093272,0.396878,0.673664}%
\pgfsetfillcolor{currentfill}%
\pgfsetlinewidth{0.000000pt}%
\definecolor{currentstroke}{rgb}{0.000000,0.000000,0.000000}%
\pgfsetstrokecolor{currentstroke}%
\pgfsetstrokeopacity{0.000000}%
\pgfsetdash{}{0pt}%
\pgfpathmoveto{\pgfqpoint{0.412280in}{0.458704in}}%
\pgfpathlineto{\pgfqpoint{0.412274in}{0.460723in}}%
\pgfpathlineto{\pgfqpoint{0.408425in}{0.462631in}}%
\pgfpathlineto{\pgfqpoint{0.410054in}{0.463833in}}%
\pgfpathlineto{\pgfqpoint{0.412335in}{0.461427in}}%
\pgfpathlineto{\pgfqpoint{0.415525in}{0.460193in}}%
\pgfpathlineto{\pgfqpoint{0.417642in}{0.461549in}}%
\pgfpathlineto{\pgfqpoint{0.418112in}{0.459372in}}%
\pgfpathlineto{\pgfqpoint{0.415856in}{0.459267in}}%
\pgfpathlineto{\pgfqpoint{0.414135in}{0.457815in}}%
\pgfpathlineto{\pgfqpoint{0.412280in}{0.458704in}}%
\pgfpathclose%
\pgfusepath{fill}%
\end{pgfscope}%
\begin{pgfscope}%
\pgfpathrectangle{\pgfqpoint{0.100000in}{0.100000in}}{\pgfqpoint{3.007045in}{1.925000in}}%
\pgfusepath{clip}%
\pgfsetbuttcap%
\pgfsetmiterjoin%
\definecolor{currentfill}{rgb}{0.093272,0.396878,0.673664}%
\pgfsetfillcolor{currentfill}%
\pgfsetlinewidth{0.000000pt}%
\definecolor{currentstroke}{rgb}{0.000000,0.000000,0.000000}%
\pgfsetstrokecolor{currentstroke}%
\pgfsetstrokeopacity{0.000000}%
\pgfsetdash{}{0pt}%
\pgfpathmoveto{\pgfqpoint{0.389271in}{0.496402in}}%
\pgfpathlineto{\pgfqpoint{0.388843in}{0.498609in}}%
\pgfpathlineto{\pgfqpoint{0.391517in}{0.499040in}}%
\pgfpathlineto{\pgfqpoint{0.392195in}{0.498040in}}%
\pgfpathlineto{\pgfqpoint{0.391561in}{0.496110in}}%
\pgfpathlineto{\pgfqpoint{0.389271in}{0.496402in}}%
\pgfpathclose%
\pgfusepath{fill}%
\end{pgfscope}%
\begin{pgfscope}%
\pgfpathrectangle{\pgfqpoint{0.100000in}{0.100000in}}{\pgfqpoint{3.007045in}{1.925000in}}%
\pgfusepath{clip}%
\pgfsetbuttcap%
\pgfsetmiterjoin%
\definecolor{currentfill}{rgb}{0.093272,0.396878,0.673664}%
\pgfsetfillcolor{currentfill}%
\pgfsetlinewidth{0.000000pt}%
\definecolor{currentstroke}{rgb}{0.000000,0.000000,0.000000}%
\pgfsetstrokecolor{currentstroke}%
\pgfsetstrokeopacity{0.000000}%
\pgfsetdash{}{0pt}%
\pgfpathmoveto{\pgfqpoint{0.424629in}{0.447465in}}%
\pgfpathlineto{\pgfqpoint{0.424048in}{0.449531in}}%
\pgfpathlineto{\pgfqpoint{0.427830in}{0.447738in}}%
\pgfpathlineto{\pgfqpoint{0.426724in}{0.446647in}}%
\pgfpathlineto{\pgfqpoint{0.424629in}{0.447465in}}%
\pgfpathclose%
\pgfusepath{fill}%
\end{pgfscope}%
\begin{pgfscope}%
\pgfpathrectangle{\pgfqpoint{0.100000in}{0.100000in}}{\pgfqpoint{3.007045in}{1.925000in}}%
\pgfusepath{clip}%
\pgfsetbuttcap%
\pgfsetmiterjoin%
\definecolor{currentfill}{rgb}{0.093272,0.396878,0.673664}%
\pgfsetfillcolor{currentfill}%
\pgfsetlinewidth{0.000000pt}%
\definecolor{currentstroke}{rgb}{0.000000,0.000000,0.000000}%
\pgfsetstrokecolor{currentstroke}%
\pgfsetstrokeopacity{0.000000}%
\pgfsetdash{}{0pt}%
\pgfpathmoveto{\pgfqpoint{0.380099in}{0.509690in}}%
\pgfpathlineto{\pgfqpoint{0.380270in}{0.511273in}}%
\pgfpathlineto{\pgfqpoint{0.382192in}{0.510622in}}%
\pgfpathlineto{\pgfqpoint{0.381722in}{0.509141in}}%
\pgfpathlineto{\pgfqpoint{0.380099in}{0.509690in}}%
\pgfpathclose%
\pgfusepath{fill}%
\end{pgfscope}%
\begin{pgfscope}%
\pgfpathrectangle{\pgfqpoint{0.100000in}{0.100000in}}{\pgfqpoint{3.007045in}{1.925000in}}%
\pgfusepath{clip}%
\pgfsetbuttcap%
\pgfsetmiterjoin%
\definecolor{currentfill}{rgb}{0.093272,0.396878,0.673664}%
\pgfsetfillcolor{currentfill}%
\pgfsetlinewidth{0.000000pt}%
\definecolor{currentstroke}{rgb}{0.000000,0.000000,0.000000}%
\pgfsetstrokecolor{currentstroke}%
\pgfsetstrokeopacity{0.000000}%
\pgfsetdash{}{0pt}%
\pgfpathmoveto{\pgfqpoint{0.428107in}{0.449417in}}%
\pgfpathlineto{\pgfqpoint{0.427960in}{0.451570in}}%
\pgfpathlineto{\pgfqpoint{0.429777in}{0.452594in}}%
\pgfpathlineto{\pgfqpoint{0.430464in}{0.450596in}}%
\pgfpathlineto{\pgfqpoint{0.428107in}{0.449417in}}%
\pgfpathclose%
\pgfusepath{fill}%
\end{pgfscope}%
\begin{pgfscope}%
\pgfpathrectangle{\pgfqpoint{0.100000in}{0.100000in}}{\pgfqpoint{3.007045in}{1.925000in}}%
\pgfusepath{clip}%
\pgfsetbuttcap%
\pgfsetmiterjoin%
\definecolor{currentfill}{rgb}{0.093272,0.396878,0.673664}%
\pgfsetfillcolor{currentfill}%
\pgfsetlinewidth{0.000000pt}%
\definecolor{currentstroke}{rgb}{0.000000,0.000000,0.000000}%
\pgfsetstrokecolor{currentstroke}%
\pgfsetstrokeopacity{0.000000}%
\pgfsetdash{}{0pt}%
\pgfpathmoveto{\pgfqpoint{0.369259in}{0.522705in}}%
\pgfpathlineto{\pgfqpoint{0.369319in}{0.524685in}}%
\pgfpathlineto{\pgfqpoint{0.372098in}{0.524246in}}%
\pgfpathlineto{\pgfqpoint{0.373941in}{0.523135in}}%
\pgfpathlineto{\pgfqpoint{0.376137in}{0.523717in}}%
\pgfpathlineto{\pgfqpoint{0.377311in}{0.522297in}}%
\pgfpathlineto{\pgfqpoint{0.374074in}{0.522045in}}%
\pgfpathlineto{\pgfqpoint{0.373276in}{0.520714in}}%
\pgfpathlineto{\pgfqpoint{0.371566in}{0.523261in}}%
\pgfpathlineto{\pgfqpoint{0.369259in}{0.522705in}}%
\pgfpathclose%
\pgfusepath{fill}%
\end{pgfscope}%
\begin{pgfscope}%
\pgfpathrectangle{\pgfqpoint{0.100000in}{0.100000in}}{\pgfqpoint{3.007045in}{1.925000in}}%
\pgfusepath{clip}%
\pgfsetbuttcap%
\pgfsetmiterjoin%
\definecolor{currentfill}{rgb}{0.093272,0.396878,0.673664}%
\pgfsetfillcolor{currentfill}%
\pgfsetlinewidth{0.000000pt}%
\definecolor{currentstroke}{rgb}{0.000000,0.000000,0.000000}%
\pgfsetstrokecolor{currentstroke}%
\pgfsetstrokeopacity{0.000000}%
\pgfsetdash{}{0pt}%
\pgfpathmoveto{\pgfqpoint{0.465052in}{0.415402in}}%
\pgfpathlineto{\pgfqpoint{0.464568in}{0.416834in}}%
\pgfpathlineto{\pgfqpoint{0.468037in}{0.416734in}}%
\pgfpathlineto{\pgfqpoint{0.469080in}{0.414904in}}%
\pgfpathlineto{\pgfqpoint{0.468333in}{0.414126in}}%
\pgfpathlineto{\pgfqpoint{0.465052in}{0.415402in}}%
\pgfpathclose%
\pgfusepath{fill}%
\end{pgfscope}%
\begin{pgfscope}%
\pgfpathrectangle{\pgfqpoint{0.100000in}{0.100000in}}{\pgfqpoint{3.007045in}{1.925000in}}%
\pgfusepath{clip}%
\pgfsetbuttcap%
\pgfsetmiterjoin%
\definecolor{currentfill}{rgb}{0.093272,0.396878,0.673664}%
\pgfsetfillcolor{currentfill}%
\pgfsetlinewidth{0.000000pt}%
\definecolor{currentstroke}{rgb}{0.000000,0.000000,0.000000}%
\pgfsetstrokecolor{currentstroke}%
\pgfsetstrokeopacity{0.000000}%
\pgfsetdash{}{0pt}%
\pgfpathmoveto{\pgfqpoint{0.351227in}{0.571842in}}%
\pgfpathlineto{\pgfqpoint{0.350829in}{0.574370in}}%
\pgfpathlineto{\pgfqpoint{0.351503in}{0.575341in}}%
\pgfpathlineto{\pgfqpoint{0.353352in}{0.574845in}}%
\pgfpathlineto{\pgfqpoint{0.353069in}{0.573567in}}%
\pgfpathlineto{\pgfqpoint{0.351227in}{0.571842in}}%
\pgfpathclose%
\pgfusepath{fill}%
\end{pgfscope}%
\begin{pgfscope}%
\pgfpathrectangle{\pgfqpoint{0.100000in}{0.100000in}}{\pgfqpoint{3.007045in}{1.925000in}}%
\pgfusepath{clip}%
\pgfsetbuttcap%
\pgfsetmiterjoin%
\definecolor{currentfill}{rgb}{0.093272,0.396878,0.673664}%
\pgfsetfillcolor{currentfill}%
\pgfsetlinewidth{0.000000pt}%
\definecolor{currentstroke}{rgb}{0.000000,0.000000,0.000000}%
\pgfsetstrokecolor{currentstroke}%
\pgfsetstrokeopacity{0.000000}%
\pgfsetdash{}{0pt}%
\pgfpathmoveto{\pgfqpoint{0.486654in}{0.401495in}}%
\pgfpathlineto{\pgfqpoint{0.487380in}{0.403172in}}%
\pgfpathlineto{\pgfqpoint{0.490544in}{0.402448in}}%
\pgfpathlineto{\pgfqpoint{0.489751in}{0.400770in}}%
\pgfpathlineto{\pgfqpoint{0.486654in}{0.401495in}}%
\pgfpathclose%
\pgfusepath{fill}%
\end{pgfscope}%
\begin{pgfscope}%
\pgfpathrectangle{\pgfqpoint{0.100000in}{0.100000in}}{\pgfqpoint{3.007045in}{1.925000in}}%
\pgfusepath{clip}%
\pgfsetbuttcap%
\pgfsetmiterjoin%
\definecolor{currentfill}{rgb}{0.093272,0.396878,0.673664}%
\pgfsetfillcolor{currentfill}%
\pgfsetlinewidth{0.000000pt}%
\definecolor{currentstroke}{rgb}{0.000000,0.000000,0.000000}%
\pgfsetstrokecolor{currentstroke}%
\pgfsetstrokeopacity{0.000000}%
\pgfsetdash{}{0pt}%
\pgfpathmoveto{\pgfqpoint{0.355532in}{0.583863in}}%
\pgfpathlineto{\pgfqpoint{0.355346in}{0.585775in}}%
\pgfpathlineto{\pgfqpoint{0.352345in}{0.588784in}}%
\pgfpathlineto{\pgfqpoint{0.354611in}{0.589330in}}%
\pgfpathlineto{\pgfqpoint{0.353122in}{0.591113in}}%
\pgfpathlineto{\pgfqpoint{0.354013in}{0.592136in}}%
\pgfpathlineto{\pgfqpoint{0.356001in}{0.592468in}}%
\pgfpathlineto{\pgfqpoint{0.357361in}{0.589332in}}%
\pgfpathlineto{\pgfqpoint{0.359106in}{0.586517in}}%
\pgfpathlineto{\pgfqpoint{0.359311in}{0.584053in}}%
\pgfpathlineto{\pgfqpoint{0.357819in}{0.583219in}}%
\pgfpathlineto{\pgfqpoint{0.355532in}{0.583863in}}%
\pgfpathclose%
\pgfusepath{fill}%
\end{pgfscope}%
\begin{pgfscope}%
\pgfpathrectangle{\pgfqpoint{0.100000in}{0.100000in}}{\pgfqpoint{3.007045in}{1.925000in}}%
\pgfusepath{clip}%
\pgfsetbuttcap%
\pgfsetmiterjoin%
\definecolor{currentfill}{rgb}{0.093272,0.396878,0.673664}%
\pgfsetfillcolor{currentfill}%
\pgfsetlinewidth{0.000000pt}%
\definecolor{currentstroke}{rgb}{0.000000,0.000000,0.000000}%
\pgfsetstrokecolor{currentstroke}%
\pgfsetstrokeopacity{0.000000}%
\pgfsetdash{}{0pt}%
\pgfpathmoveto{\pgfqpoint{0.528105in}{0.380782in}}%
\pgfpathlineto{\pgfqpoint{0.527637in}{0.383157in}}%
\pgfpathlineto{\pgfqpoint{0.528964in}{0.383518in}}%
\pgfpathlineto{\pgfqpoint{0.532084in}{0.382526in}}%
\pgfpathlineto{\pgfqpoint{0.533585in}{0.381267in}}%
\pgfpathlineto{\pgfqpoint{0.535312in}{0.380961in}}%
\pgfpathlineto{\pgfqpoint{0.536723in}{0.379546in}}%
\pgfpathlineto{\pgfqpoint{0.538639in}{0.381047in}}%
\pgfpathlineto{\pgfqpoint{0.541713in}{0.382292in}}%
\pgfpathlineto{\pgfqpoint{0.544259in}{0.379787in}}%
\pgfpathlineto{\pgfqpoint{0.542790in}{0.385194in}}%
\pgfpathlineto{\pgfqpoint{0.544767in}{0.386107in}}%
\pgfpathlineto{\pgfqpoint{0.546961in}{0.385637in}}%
\pgfpathlineto{\pgfqpoint{0.550504in}{0.383713in}}%
\pgfpathlineto{\pgfqpoint{0.549689in}{0.382189in}}%
\pgfpathlineto{\pgfqpoint{0.551450in}{0.380178in}}%
\pgfpathlineto{\pgfqpoint{0.553278in}{0.380675in}}%
\pgfpathlineto{\pgfqpoint{0.552656in}{0.378168in}}%
\pgfpathlineto{\pgfqpoint{0.549658in}{0.377895in}}%
\pgfpathlineto{\pgfqpoint{0.548439in}{0.374718in}}%
\pgfpathlineto{\pgfqpoint{0.543340in}{0.376546in}}%
\pgfpathlineto{\pgfqpoint{0.541469in}{0.376728in}}%
\pgfpathlineto{\pgfqpoint{0.540243in}{0.375113in}}%
\pgfpathlineto{\pgfqpoint{0.538796in}{0.377500in}}%
\pgfpathlineto{\pgfqpoint{0.537382in}{0.378201in}}%
\pgfpathlineto{\pgfqpoint{0.533254in}{0.378923in}}%
\pgfpathlineto{\pgfqpoint{0.531395in}{0.379938in}}%
\pgfpathlineto{\pgfqpoint{0.528105in}{0.380782in}}%
\pgfpathclose%
\pgfusepath{fill}%
\end{pgfscope}%
\begin{pgfscope}%
\pgfpathrectangle{\pgfqpoint{0.100000in}{0.100000in}}{\pgfqpoint{3.007045in}{1.925000in}}%
\pgfusepath{clip}%
\pgfsetbuttcap%
\pgfsetmiterjoin%
\definecolor{currentfill}{rgb}{0.093272,0.396878,0.673664}%
\pgfsetfillcolor{currentfill}%
\pgfsetlinewidth{0.000000pt}%
\definecolor{currentstroke}{rgb}{0.000000,0.000000,0.000000}%
\pgfsetstrokecolor{currentstroke}%
\pgfsetstrokeopacity{0.000000}%
\pgfsetdash{}{0pt}%
\pgfpathmoveto{\pgfqpoint{0.500539in}{0.394151in}}%
\pgfpathlineto{\pgfqpoint{0.500162in}{0.396419in}}%
\pgfpathlineto{\pgfqpoint{0.502003in}{0.396647in}}%
\pgfpathlineto{\pgfqpoint{0.502049in}{0.394805in}}%
\pgfpathlineto{\pgfqpoint{0.500539in}{0.394151in}}%
\pgfpathclose%
\pgfusepath{fill}%
\end{pgfscope}%
\begin{pgfscope}%
\pgfpathrectangle{\pgfqpoint{0.100000in}{0.100000in}}{\pgfqpoint{3.007045in}{1.925000in}}%
\pgfusepath{clip}%
\pgfsetbuttcap%
\pgfsetmiterjoin%
\definecolor{currentfill}{rgb}{0.093272,0.396878,0.673664}%
\pgfsetfillcolor{currentfill}%
\pgfsetlinewidth{0.000000pt}%
\definecolor{currentstroke}{rgb}{0.000000,0.000000,0.000000}%
\pgfsetstrokecolor{currentstroke}%
\pgfsetstrokeopacity{0.000000}%
\pgfsetdash{}{0pt}%
\pgfpathmoveto{\pgfqpoint{0.507877in}{0.388268in}}%
\pgfpathlineto{\pgfqpoint{0.508331in}{0.389076in}}%
\pgfpathlineto{\pgfqpoint{0.513005in}{0.389208in}}%
\pgfpathlineto{\pgfqpoint{0.515777in}{0.391018in}}%
\pgfpathlineto{\pgfqpoint{0.519331in}{0.390800in}}%
\pgfpathlineto{\pgfqpoint{0.521540in}{0.387996in}}%
\pgfpathlineto{\pgfqpoint{0.523561in}{0.390645in}}%
\pgfpathlineto{\pgfqpoint{0.525758in}{0.391376in}}%
\pgfpathlineto{\pgfqpoint{0.528197in}{0.390858in}}%
\pgfpathlineto{\pgfqpoint{0.530050in}{0.389563in}}%
\pgfpathlineto{\pgfqpoint{0.530693in}{0.387435in}}%
\pgfpathlineto{\pgfqpoint{0.530317in}{0.386066in}}%
\pgfpathlineto{\pgfqpoint{0.528322in}{0.384961in}}%
\pgfpathlineto{\pgfqpoint{0.521632in}{0.386913in}}%
\pgfpathlineto{\pgfqpoint{0.517279in}{0.385660in}}%
\pgfpathlineto{\pgfqpoint{0.515618in}{0.386416in}}%
\pgfpathlineto{\pgfqpoint{0.507877in}{0.388268in}}%
\pgfpathclose%
\pgfusepath{fill}%
\end{pgfscope}%
\begin{pgfscope}%
\pgfpathrectangle{\pgfqpoint{0.100000in}{0.100000in}}{\pgfqpoint{3.007045in}{1.925000in}}%
\pgfusepath{clip}%
\pgfsetbuttcap%
\pgfsetmiterjoin%
\definecolor{currentfill}{rgb}{0.093272,0.396878,0.673664}%
\pgfsetfillcolor{currentfill}%
\pgfsetlinewidth{0.000000pt}%
\definecolor{currentstroke}{rgb}{0.000000,0.000000,0.000000}%
\pgfsetstrokecolor{currentstroke}%
\pgfsetstrokeopacity{0.000000}%
\pgfsetdash{}{0pt}%
\pgfpathmoveto{\pgfqpoint{0.453614in}{0.421300in}}%
\pgfpathlineto{\pgfqpoint{0.451899in}{0.423133in}}%
\pgfpathlineto{\pgfqpoint{0.450880in}{0.425283in}}%
\pgfpathlineto{\pgfqpoint{0.449241in}{0.426901in}}%
\pgfpathlineto{\pgfqpoint{0.449633in}{0.429133in}}%
\pgfpathlineto{\pgfqpoint{0.453091in}{0.425783in}}%
\pgfpathlineto{\pgfqpoint{0.455129in}{0.421602in}}%
\pgfpathlineto{\pgfqpoint{0.453614in}{0.421300in}}%
\pgfpathclose%
\pgfusepath{fill}%
\end{pgfscope}%
\begin{pgfscope}%
\pgfpathrectangle{\pgfqpoint{0.100000in}{0.100000in}}{\pgfqpoint{3.007045in}{1.925000in}}%
\pgfusepath{clip}%
\pgfsetbuttcap%
\pgfsetmiterjoin%
\definecolor{currentfill}{rgb}{0.093272,0.396878,0.673664}%
\pgfsetfillcolor{currentfill}%
\pgfsetlinewidth{0.000000pt}%
\definecolor{currentstroke}{rgb}{0.000000,0.000000,0.000000}%
\pgfsetstrokecolor{currentstroke}%
\pgfsetstrokeopacity{0.000000}%
\pgfsetdash{}{0pt}%
\pgfpathmoveto{\pgfqpoint{0.444349in}{0.432536in}}%
\pgfpathlineto{\pgfqpoint{0.443210in}{0.434576in}}%
\pgfpathlineto{\pgfqpoint{0.440911in}{0.435309in}}%
\pgfpathlineto{\pgfqpoint{0.440187in}{0.438580in}}%
\pgfpathlineto{\pgfqpoint{0.442735in}{0.437137in}}%
\pgfpathlineto{\pgfqpoint{0.443790in}{0.435233in}}%
\pgfpathlineto{\pgfqpoint{0.445554in}{0.435638in}}%
\pgfpathlineto{\pgfqpoint{0.448969in}{0.435304in}}%
\pgfpathlineto{\pgfqpoint{0.450814in}{0.436419in}}%
\pgfpathlineto{\pgfqpoint{0.452553in}{0.435695in}}%
\pgfpathlineto{\pgfqpoint{0.453462in}{0.433672in}}%
\pgfpathlineto{\pgfqpoint{0.453143in}{0.432430in}}%
\pgfpathlineto{\pgfqpoint{0.451239in}{0.431442in}}%
\pgfpathlineto{\pgfqpoint{0.449605in}{0.432844in}}%
\pgfpathlineto{\pgfqpoint{0.448738in}{0.430447in}}%
\pgfpathlineto{\pgfqpoint{0.447526in}{0.430880in}}%
\pgfpathlineto{\pgfqpoint{0.445671in}{0.433246in}}%
\pgfpathlineto{\pgfqpoint{0.444349in}{0.432536in}}%
\pgfpathclose%
\pgfusepath{fill}%
\end{pgfscope}%
\begin{pgfscope}%
\pgfpathrectangle{\pgfqpoint{0.100000in}{0.100000in}}{\pgfqpoint{3.007045in}{1.925000in}}%
\pgfusepath{clip}%
\pgfsetbuttcap%
\pgfsetmiterjoin%
\definecolor{currentfill}{rgb}{0.579608,0.770196,0.873725}%
\pgfsetfillcolor{currentfill}%
\pgfsetlinewidth{0.000000pt}%
\definecolor{currentstroke}{rgb}{0.000000,0.000000,0.000000}%
\pgfsetstrokecolor{currentstroke}%
\pgfsetstrokeopacity{0.000000}%
\pgfsetdash{}{0pt}%
\pgfpathmoveto{\pgfqpoint{1.876321in}{0.768157in}}%
\pgfpathlineto{\pgfqpoint{1.870545in}{0.767215in}}%
\pgfpathlineto{\pgfqpoint{1.856034in}{0.767401in}}%
\pgfpathlineto{\pgfqpoint{1.856051in}{0.769311in}}%
\pgfpathlineto{\pgfqpoint{1.842812in}{0.769497in}}%
\pgfpathlineto{\pgfqpoint{1.836570in}{0.777705in}}%
\pgfpathlineto{\pgfqpoint{1.829258in}{0.781767in}}%
\pgfpathlineto{\pgfqpoint{1.829586in}{0.787129in}}%
\pgfpathlineto{\pgfqpoint{1.828440in}{0.791464in}}%
\pgfpathlineto{\pgfqpoint{1.824314in}{0.796016in}}%
\pgfpathlineto{\pgfqpoint{1.822337in}{0.805756in}}%
\pgfpathlineto{\pgfqpoint{1.821492in}{0.816167in}}%
\pgfpathlineto{\pgfqpoint{1.813675in}{0.816167in}}%
\pgfpathlineto{\pgfqpoint{1.812989in}{0.826934in}}%
\pgfpathlineto{\pgfqpoint{1.830412in}{0.826752in}}%
\pgfpathlineto{\pgfqpoint{1.830433in}{0.847601in}}%
\pgfpathlineto{\pgfqpoint{1.833670in}{0.850751in}}%
\pgfpathlineto{\pgfqpoint{1.842321in}{0.853020in}}%
\pgfpathlineto{\pgfqpoint{1.859472in}{0.853168in}}%
\pgfpathlineto{\pgfqpoint{1.865214in}{0.855241in}}%
\pgfpathlineto{\pgfqpoint{1.882494in}{0.855536in}}%
\pgfpathlineto{\pgfqpoint{1.882114in}{0.846605in}}%
\pgfpathlineto{\pgfqpoint{1.887792in}{0.846596in}}%
\pgfpathlineto{\pgfqpoint{1.891528in}{0.843729in}}%
\pgfpathlineto{\pgfqpoint{1.892466in}{0.838016in}}%
\pgfpathlineto{\pgfqpoint{1.898412in}{0.836142in}}%
\pgfpathlineto{\pgfqpoint{1.899400in}{0.832281in}}%
\pgfpathlineto{\pgfqpoint{1.899320in}{0.814784in}}%
\pgfpathlineto{\pgfqpoint{1.887940in}{0.814920in}}%
\pgfpathlineto{\pgfqpoint{1.887657in}{0.795131in}}%
\pgfpathlineto{\pgfqpoint{1.884971in}{0.791103in}}%
\pgfpathlineto{\pgfqpoint{1.876541in}{0.789626in}}%
\pgfpathlineto{\pgfqpoint{1.876321in}{0.768157in}}%
\pgfpathclose%
\pgfusepath{fill}%
\end{pgfscope}%
\begin{pgfscope}%
\pgfpathrectangle{\pgfqpoint{0.100000in}{0.100000in}}{\pgfqpoint{3.007045in}{1.925000in}}%
\pgfusepath{clip}%
\pgfsetbuttcap%
\pgfsetmiterjoin%
\definecolor{currentfill}{rgb}{0.346467,0.632403,0.810673}%
\pgfsetfillcolor{currentfill}%
\pgfsetlinewidth{0.000000pt}%
\definecolor{currentstroke}{rgb}{0.000000,0.000000,0.000000}%
\pgfsetstrokecolor{currentstroke}%
\pgfsetstrokeopacity{0.000000}%
\pgfsetdash{}{0pt}%
\pgfpathmoveto{\pgfqpoint{0.596303in}{1.682378in}}%
\pgfpathlineto{\pgfqpoint{0.563962in}{1.691905in}}%
\pgfpathlineto{\pgfqpoint{0.533873in}{1.701067in}}%
\pgfpathlineto{\pgfqpoint{0.536189in}{1.704696in}}%
\pgfpathlineto{\pgfqpoint{0.535329in}{1.713270in}}%
\pgfpathlineto{\pgfqpoint{0.541696in}{1.715411in}}%
\pgfpathlineto{\pgfqpoint{0.541409in}{1.723741in}}%
\pgfpathlineto{\pgfqpoint{0.545552in}{1.726930in}}%
\pgfpathlineto{\pgfqpoint{0.547023in}{1.735645in}}%
\pgfpathlineto{\pgfqpoint{0.540688in}{1.747112in}}%
\pgfpathlineto{\pgfqpoint{0.542490in}{1.755480in}}%
\pgfpathlineto{\pgfqpoint{0.545786in}{1.757536in}}%
\pgfpathlineto{\pgfqpoint{0.551653in}{1.755806in}}%
\pgfpathlineto{\pgfqpoint{0.561232in}{1.755167in}}%
\pgfpathlineto{\pgfqpoint{0.558944in}{1.759738in}}%
\pgfpathlineto{\pgfqpoint{0.563789in}{1.776383in}}%
\pgfpathlineto{\pgfqpoint{0.567701in}{1.775221in}}%
\pgfpathlineto{\pgfqpoint{0.616044in}{1.760740in}}%
\pgfpathlineto{\pgfqpoint{0.640972in}{1.753699in}}%
\pgfpathlineto{\pgfqpoint{0.637219in}{1.740644in}}%
\pgfpathlineto{\pgfqpoint{0.630951in}{1.740790in}}%
\pgfpathlineto{\pgfqpoint{0.619982in}{1.737934in}}%
\pgfpathlineto{\pgfqpoint{0.610892in}{1.738691in}}%
\pgfpathlineto{\pgfqpoint{0.606293in}{1.736988in}}%
\pgfpathlineto{\pgfqpoint{0.608690in}{1.732151in}}%
\pgfpathlineto{\pgfqpoint{0.608895in}{1.725281in}}%
\pgfpathlineto{\pgfqpoint{0.602494in}{1.724955in}}%
\pgfpathlineto{\pgfqpoint{0.595993in}{1.712179in}}%
\pgfpathlineto{\pgfqpoint{0.598195in}{1.705394in}}%
\pgfpathlineto{\pgfqpoint{0.593379in}{1.692150in}}%
\pgfpathlineto{\pgfqpoint{0.593360in}{1.686222in}}%
\pgfpathlineto{\pgfqpoint{0.596303in}{1.682378in}}%
\pgfpathclose%
\pgfusepath{fill}%
\end{pgfscope}%
\begin{pgfscope}%
\pgfpathrectangle{\pgfqpoint{0.100000in}{0.100000in}}{\pgfqpoint{3.007045in}{1.925000in}}%
\pgfusepath{clip}%
\pgfsetbuttcap%
\pgfsetmiterjoin%
\definecolor{currentfill}{rgb}{0.108651,0.416563,0.689043}%
\pgfsetfillcolor{currentfill}%
\pgfsetlinewidth{0.000000pt}%
\definecolor{currentstroke}{rgb}{0.000000,0.000000,0.000000}%
\pgfsetstrokecolor{currentstroke}%
\pgfsetstrokeopacity{0.000000}%
\pgfsetdash{}{0pt}%
\pgfpathmoveto{\pgfqpoint{1.473659in}{0.890543in}}%
\pgfpathlineto{\pgfqpoint{1.444186in}{0.892364in}}%
\pgfpathlineto{\pgfqpoint{1.415306in}{0.894451in}}%
\pgfpathlineto{\pgfqpoint{1.417413in}{0.923222in}}%
\pgfpathlineto{\pgfqpoint{1.419469in}{0.951860in}}%
\pgfpathlineto{\pgfqpoint{1.477103in}{0.948069in}}%
\pgfpathlineto{\pgfqpoint{1.505836in}{0.946347in}}%
\pgfpathlineto{\pgfqpoint{1.505202in}{0.934838in}}%
\pgfpathlineto{\pgfqpoint{1.504252in}{0.917589in}}%
\pgfpathlineto{\pgfqpoint{1.475346in}{0.919280in}}%
\pgfpathlineto{\pgfqpoint{1.473659in}{0.890543in}}%
\pgfpathclose%
\pgfusepath{fill}%
\end{pgfscope}%
\begin{pgfscope}%
\pgfpathrectangle{\pgfqpoint{0.100000in}{0.100000in}}{\pgfqpoint{3.007045in}{1.925000in}}%
\pgfusepath{clip}%
\pgfsetbuttcap%
\pgfsetmiterjoin%
\definecolor{currentfill}{rgb}{0.504314,0.728858,0.857486}%
\pgfsetfillcolor{currentfill}%
\pgfsetlinewidth{0.000000pt}%
\definecolor{currentstroke}{rgb}{0.000000,0.000000,0.000000}%
\pgfsetstrokecolor{currentstroke}%
\pgfsetstrokeopacity{0.000000}%
\pgfsetdash{}{0pt}%
\pgfpathmoveto{\pgfqpoint{1.586901in}{0.635387in}}%
\pgfpathlineto{\pgfqpoint{1.569891in}{0.623670in}}%
\pgfpathlineto{\pgfqpoint{1.571686in}{0.614167in}}%
\pgfpathlineto{\pgfqpoint{1.576374in}{0.610043in}}%
\pgfpathlineto{\pgfqpoint{1.575754in}{0.602872in}}%
\pgfpathlineto{\pgfqpoint{1.546174in}{0.604074in}}%
\pgfpathlineto{\pgfqpoint{1.538952in}{0.605702in}}%
\pgfpathlineto{\pgfqpoint{1.540604in}{0.640218in}}%
\pgfpathlineto{\pgfqpoint{1.532433in}{0.642045in}}%
\pgfpathlineto{\pgfqpoint{1.527773in}{0.637409in}}%
\pgfpathlineto{\pgfqpoint{1.520806in}{0.642095in}}%
\pgfpathlineto{\pgfqpoint{1.510361in}{0.642115in}}%
\pgfpathlineto{\pgfqpoint{1.505217in}{0.648714in}}%
\pgfpathlineto{\pgfqpoint{1.507461in}{0.683119in}}%
\pgfpathlineto{\pgfqpoint{1.551784in}{0.680740in}}%
\pgfpathlineto{\pgfqpoint{1.565148in}{0.655103in}}%
\pgfpathlineto{\pgfqpoint{1.575122in}{0.656245in}}%
\pgfpathlineto{\pgfqpoint{1.586901in}{0.635387in}}%
\pgfpathclose%
\pgfusepath{fill}%
\end{pgfscope}%
\begin{pgfscope}%
\pgfpathrectangle{\pgfqpoint{0.100000in}{0.100000in}}{\pgfqpoint{3.007045in}{1.925000in}}%
\pgfusepath{clip}%
\pgfsetbuttcap%
\pgfsetmiterjoin%
\definecolor{currentfill}{rgb}{0.548235,0.752972,0.866959}%
\pgfsetfillcolor{currentfill}%
\pgfsetlinewidth{0.000000pt}%
\definecolor{currentstroke}{rgb}{0.000000,0.000000,0.000000}%
\pgfsetstrokecolor{currentstroke}%
\pgfsetstrokeopacity{0.000000}%
\pgfsetdash{}{0pt}%
\pgfpathmoveto{\pgfqpoint{2.442205in}{0.558231in}}%
\pgfpathlineto{\pgfqpoint{2.433773in}{0.561843in}}%
\pgfpathlineto{\pgfqpoint{2.427536in}{0.571164in}}%
\pgfpathlineto{\pgfqpoint{2.419054in}{0.576060in}}%
\pgfpathlineto{\pgfqpoint{2.409749in}{0.579100in}}%
\pgfpathlineto{\pgfqpoint{2.405641in}{0.582074in}}%
\pgfpathlineto{\pgfqpoint{2.408711in}{0.592974in}}%
\pgfpathlineto{\pgfqpoint{2.415159in}{0.598741in}}%
\pgfpathlineto{\pgfqpoint{2.413904in}{0.601638in}}%
\pgfpathlineto{\pgfqpoint{2.418230in}{0.606839in}}%
\pgfpathlineto{\pgfqpoint{2.416778in}{0.611640in}}%
\pgfpathlineto{\pgfqpoint{2.423706in}{0.617445in}}%
\pgfpathlineto{\pgfqpoint{2.422704in}{0.621971in}}%
\pgfpathlineto{\pgfqpoint{2.481583in}{0.625736in}}%
\pgfpathlineto{\pgfqpoint{2.488716in}{0.626214in}}%
\pgfpathlineto{\pgfqpoint{2.492884in}{0.596677in}}%
\pgfpathlineto{\pgfqpoint{2.488025in}{0.585155in}}%
\pgfpathlineto{\pgfqpoint{2.487697in}{0.577489in}}%
\pgfpathlineto{\pgfqpoint{2.485231in}{0.574838in}}%
\pgfpathlineto{\pgfqpoint{2.479736in}{0.576043in}}%
\pgfpathlineto{\pgfqpoint{2.476107in}{0.580682in}}%
\pgfpathlineto{\pgfqpoint{2.471783in}{0.578648in}}%
\pgfpathlineto{\pgfqpoint{2.470809in}{0.572873in}}%
\pgfpathlineto{\pgfqpoint{2.446412in}{0.569336in}}%
\pgfpathlineto{\pgfqpoint{2.444618in}{0.558566in}}%
\pgfpathlineto{\pgfqpoint{2.442205in}{0.558231in}}%
\pgfpathclose%
\pgfusepath{fill}%
\end{pgfscope}%
\begin{pgfscope}%
\pgfpathrectangle{\pgfqpoint{0.100000in}{0.100000in}}{\pgfqpoint{3.007045in}{1.925000in}}%
\pgfusepath{clip}%
\pgfsetbuttcap%
\pgfsetmiterjoin%
\definecolor{currentfill}{rgb}{0.454118,0.701300,0.846659}%
\pgfsetfillcolor{currentfill}%
\pgfsetlinewidth{0.000000pt}%
\definecolor{currentstroke}{rgb}{0.000000,0.000000,0.000000}%
\pgfsetstrokecolor{currentstroke}%
\pgfsetstrokeopacity{0.000000}%
\pgfsetdash{}{0pt}%
\pgfpathmoveto{\pgfqpoint{2.135867in}{0.998378in}}%
\pgfpathlineto{\pgfqpoint{2.141030in}{0.985137in}}%
\pgfpathlineto{\pgfqpoint{2.140155in}{0.982191in}}%
\pgfpathlineto{\pgfqpoint{2.117091in}{0.980740in}}%
\pgfpathlineto{\pgfqpoint{2.116101in}{0.997143in}}%
\pgfpathlineto{\pgfqpoint{2.135867in}{0.998378in}}%
\pgfpathclose%
\pgfusepath{fill}%
\end{pgfscope}%
\begin{pgfscope}%
\pgfpathrectangle{\pgfqpoint{0.100000in}{0.100000in}}{\pgfqpoint{3.007045in}{1.925000in}}%
\pgfusepath{clip}%
\pgfsetbuttcap%
\pgfsetmiterjoin%
\definecolor{currentfill}{rgb}{0.472941,0.711634,0.850719}%
\pgfsetfillcolor{currentfill}%
\pgfsetlinewidth{0.000000pt}%
\definecolor{currentstroke}{rgb}{0.000000,0.000000,0.000000}%
\pgfsetstrokecolor{currentstroke}%
\pgfsetstrokeopacity{0.000000}%
\pgfsetdash{}{0pt}%
\pgfpathmoveto{\pgfqpoint{1.861397in}{1.624038in}}%
\pgfpathlineto{\pgfqpoint{1.861813in}{1.606700in}}%
\pgfpathlineto{\pgfqpoint{1.844441in}{1.606089in}}%
\pgfpathlineto{\pgfqpoint{1.844395in}{1.612207in}}%
\pgfpathlineto{\pgfqpoint{1.826933in}{1.611742in}}%
\pgfpathlineto{\pgfqpoint{1.827033in}{1.605997in}}%
\pgfpathlineto{\pgfqpoint{1.801088in}{1.605771in}}%
\pgfpathlineto{\pgfqpoint{1.802129in}{1.614963in}}%
\pgfpathlineto{\pgfqpoint{1.798627in}{1.617392in}}%
\pgfpathlineto{\pgfqpoint{1.793675in}{1.615402in}}%
\pgfpathlineto{\pgfqpoint{1.782448in}{1.621431in}}%
\pgfpathlineto{\pgfqpoint{1.782023in}{1.648669in}}%
\pgfpathlineto{\pgfqpoint{1.787806in}{1.648648in}}%
\pgfpathlineto{\pgfqpoint{1.787190in}{1.689017in}}%
\pgfpathlineto{\pgfqpoint{1.798628in}{1.689186in}}%
\pgfpathlineto{\pgfqpoint{1.798346in}{1.718105in}}%
\pgfpathlineto{\pgfqpoint{1.827086in}{1.718431in}}%
\pgfpathlineto{\pgfqpoint{1.827059in}{1.721935in}}%
\pgfpathlineto{\pgfqpoint{1.858121in}{1.722001in}}%
\pgfpathlineto{\pgfqpoint{1.859255in}{1.710576in}}%
\pgfpathlineto{\pgfqpoint{1.859346in}{1.688805in}}%
\pgfpathlineto{\pgfqpoint{1.860483in}{1.664391in}}%
\pgfpathlineto{\pgfqpoint{1.860671in}{1.638052in}}%
\pgfpathlineto{\pgfqpoint{1.861397in}{1.624038in}}%
\pgfpathclose%
\pgfusepath{fill}%
\end{pgfscope}%
\begin{pgfscope}%
\pgfpathrectangle{\pgfqpoint{0.100000in}{0.100000in}}{\pgfqpoint{3.007045in}{1.925000in}}%
\pgfusepath{clip}%
\pgfsetbuttcap%
\pgfsetmiterjoin%
\definecolor{currentfill}{rgb}{0.321246,0.615179,0.800830}%
\pgfsetfillcolor{currentfill}%
\pgfsetlinewidth{0.000000pt}%
\definecolor{currentstroke}{rgb}{0.000000,0.000000,0.000000}%
\pgfsetstrokecolor{currentstroke}%
\pgfsetstrokeopacity{0.000000}%
\pgfsetdash{}{0pt}%
\pgfpathmoveto{\pgfqpoint{1.348840in}{1.549898in}}%
\pgfpathlineto{\pgfqpoint{1.303187in}{1.554906in}}%
\pgfpathlineto{\pgfqpoint{1.305764in}{1.578064in}}%
\pgfpathlineto{\pgfqpoint{1.308128in}{1.577801in}}%
\pgfpathlineto{\pgfqpoint{1.311526in}{1.606441in}}%
\pgfpathlineto{\pgfqpoint{1.313493in}{1.606222in}}%
\pgfpathlineto{\pgfqpoint{1.316018in}{1.629285in}}%
\pgfpathlineto{\pgfqpoint{1.318468in}{1.629024in}}%
\pgfpathlineto{\pgfqpoint{1.335389in}{1.627188in}}%
\pgfpathlineto{\pgfqpoint{1.334802in}{1.621794in}}%
\pgfpathlineto{\pgfqpoint{1.340576in}{1.621215in}}%
\pgfpathlineto{\pgfqpoint{1.339970in}{1.615629in}}%
\pgfpathlineto{\pgfqpoint{1.351714in}{1.614337in}}%
\pgfpathlineto{\pgfqpoint{1.355655in}{1.612365in}}%
\pgfpathlineto{\pgfqpoint{1.403892in}{1.607820in}}%
\pgfpathlineto{\pgfqpoint{1.406341in}{1.607594in}}%
\pgfpathlineto{\pgfqpoint{1.401374in}{1.559284in}}%
\pgfpathlineto{\pgfqpoint{1.351084in}{1.564019in}}%
\pgfpathlineto{\pgfqpoint{1.348840in}{1.549898in}}%
\pgfpathclose%
\pgfusepath{fill}%
\end{pgfscope}%
\begin{pgfscope}%
\pgfpathrectangle{\pgfqpoint{0.100000in}{0.100000in}}{\pgfqpoint{3.007045in}{1.925000in}}%
\pgfusepath{clip}%
\pgfsetbuttcap%
\pgfsetmiterjoin%
\definecolor{currentfill}{rgb}{0.642368,0.801830,0.890319}%
\pgfsetfillcolor{currentfill}%
\pgfsetlinewidth{0.000000pt}%
\definecolor{currentstroke}{rgb}{0.000000,0.000000,0.000000}%
\pgfsetstrokecolor{currentstroke}%
\pgfsetstrokeopacity{0.000000}%
\pgfsetdash{}{0pt}%
\pgfpathmoveto{\pgfqpoint{2.199018in}{1.471645in}}%
\pgfpathlineto{\pgfqpoint{2.180375in}{1.470113in}}%
\pgfpathlineto{\pgfqpoint{2.177932in}{1.479844in}}%
\pgfpathlineto{\pgfqpoint{2.175073in}{1.484492in}}%
\pgfpathlineto{\pgfqpoint{2.178913in}{1.490543in}}%
\pgfpathlineto{\pgfqpoint{2.184775in}{1.505418in}}%
\pgfpathlineto{\pgfqpoint{2.185259in}{1.517003in}}%
\pgfpathlineto{\pgfqpoint{2.204865in}{1.518579in}}%
\pgfpathlineto{\pgfqpoint{2.207015in}{1.495613in}}%
\pgfpathlineto{\pgfqpoint{2.196560in}{1.494814in}}%
\pgfpathlineto{\pgfqpoint{2.199018in}{1.471645in}}%
\pgfpathclose%
\pgfusepath{fill}%
\end{pgfscope}%
\begin{pgfscope}%
\pgfpathrectangle{\pgfqpoint{0.100000in}{0.100000in}}{\pgfqpoint{3.007045in}{1.925000in}}%
\pgfusepath{clip}%
\pgfsetbuttcap%
\pgfsetmiterjoin%
\definecolor{currentfill}{rgb}{0.084045,0.385067,0.664437}%
\pgfsetfillcolor{currentfill}%
\pgfsetlinewidth{0.000000pt}%
\definecolor{currentstroke}{rgb}{0.000000,0.000000,0.000000}%
\pgfsetstrokecolor{currentstroke}%
\pgfsetstrokeopacity{0.000000}%
\pgfsetdash{}{0pt}%
\pgfpathmoveto{\pgfqpoint{1.011802in}{1.226459in}}%
\pgfpathlineto{\pgfqpoint{1.010713in}{1.219317in}}%
\pgfpathlineto{\pgfqpoint{1.007174in}{1.215093in}}%
\pgfpathlineto{\pgfqpoint{1.007785in}{1.211607in}}%
\pgfpathlineto{\pgfqpoint{1.001745in}{1.203071in}}%
\pgfpathlineto{\pgfqpoint{0.998322in}{1.188795in}}%
\pgfpathlineto{\pgfqpoint{1.000947in}{1.182123in}}%
\pgfpathlineto{\pgfqpoint{0.999346in}{1.179674in}}%
\pgfpathlineto{\pgfqpoint{1.002802in}{1.166666in}}%
\pgfpathlineto{\pgfqpoint{1.001312in}{1.163987in}}%
\pgfpathlineto{\pgfqpoint{0.968213in}{1.169632in}}%
\pgfpathlineto{\pgfqpoint{0.936286in}{1.175309in}}%
\pgfpathlineto{\pgfqpoint{0.936407in}{1.175969in}}%
\pgfpathlineto{\pgfqpoint{0.942870in}{1.209747in}}%
\pgfpathlineto{\pgfqpoint{0.914996in}{1.214937in}}%
\pgfpathlineto{\pgfqpoint{0.907070in}{1.217261in}}%
\pgfpathlineto{\pgfqpoint{0.910572in}{1.236135in}}%
\pgfpathlineto{\pgfqpoint{0.915780in}{1.238145in}}%
\pgfpathlineto{\pgfqpoint{0.924573in}{1.236589in}}%
\pgfpathlineto{\pgfqpoint{0.927198in}{1.241278in}}%
\pgfpathlineto{\pgfqpoint{0.929754in}{1.256515in}}%
\pgfpathlineto{\pgfqpoint{0.937487in}{1.261054in}}%
\pgfpathlineto{\pgfqpoint{0.934985in}{1.263476in}}%
\pgfpathlineto{\pgfqpoint{0.974041in}{1.256437in}}%
\pgfpathlineto{\pgfqpoint{0.990893in}{1.253051in}}%
\pgfpathlineto{\pgfqpoint{1.022492in}{1.247657in}}%
\pgfpathlineto{\pgfqpoint{1.016300in}{1.244056in}}%
\pgfpathlineto{\pgfqpoint{1.015989in}{1.239279in}}%
\pgfpathlineto{\pgfqpoint{1.012248in}{1.233395in}}%
\pgfpathlineto{\pgfqpoint{1.011802in}{1.226459in}}%
\pgfpathclose%
\pgfusepath{fill}%
\end{pgfscope}%
\begin{pgfscope}%
\pgfpathrectangle{\pgfqpoint{0.100000in}{0.100000in}}{\pgfqpoint{3.007045in}{1.925000in}}%
\pgfusepath{clip}%
\pgfsetbuttcap%
\pgfsetmiterjoin%
\definecolor{currentfill}{rgb}{0.706344,0.829020,0.912710}%
\pgfsetfillcolor{currentfill}%
\pgfsetlinewidth{0.000000pt}%
\definecolor{currentstroke}{rgb}{0.000000,0.000000,0.000000}%
\pgfsetstrokecolor{currentstroke}%
\pgfsetstrokeopacity{0.000000}%
\pgfsetdash{}{0pt}%
\pgfpathmoveto{\pgfqpoint{0.397486in}{1.548285in}}%
\pgfpathlineto{\pgfqpoint{0.393613in}{1.542963in}}%
\pgfpathlineto{\pgfqpoint{0.385318in}{1.540362in}}%
\pgfpathlineto{\pgfqpoint{0.384988in}{1.532462in}}%
\pgfpathlineto{\pgfqpoint{0.380582in}{1.527032in}}%
\pgfpathlineto{\pgfqpoint{0.381897in}{1.521054in}}%
\pgfpathlineto{\pgfqpoint{0.381028in}{1.514238in}}%
\pgfpathlineto{\pgfqpoint{0.378145in}{1.511673in}}%
\pgfpathlineto{\pgfqpoint{0.372999in}{1.513265in}}%
\pgfpathlineto{\pgfqpoint{0.374697in}{1.518495in}}%
\pgfpathlineto{\pgfqpoint{0.360884in}{1.523094in}}%
\pgfpathlineto{\pgfqpoint{0.362297in}{1.539826in}}%
\pgfpathlineto{\pgfqpoint{0.358579in}{1.545963in}}%
\pgfpathlineto{\pgfqpoint{0.364222in}{1.554694in}}%
\pgfpathlineto{\pgfqpoint{0.365111in}{1.559248in}}%
\pgfpathlineto{\pgfqpoint{0.360771in}{1.567349in}}%
\pgfpathlineto{\pgfqpoint{0.362017in}{1.572568in}}%
\pgfpathlineto{\pgfqpoint{0.361686in}{1.582534in}}%
\pgfpathlineto{\pgfqpoint{0.365712in}{1.592867in}}%
\pgfpathlineto{\pgfqpoint{0.368947in}{1.597569in}}%
\pgfpathlineto{\pgfqpoint{0.369577in}{1.603351in}}%
\pgfpathlineto{\pgfqpoint{0.366697in}{1.609297in}}%
\pgfpathlineto{\pgfqpoint{0.367089in}{1.616757in}}%
\pgfpathlineto{\pgfqpoint{0.372778in}{1.622728in}}%
\pgfpathlineto{\pgfqpoint{0.382988in}{1.619317in}}%
\pgfpathlineto{\pgfqpoint{0.386071in}{1.611691in}}%
\pgfpathlineto{\pgfqpoint{0.382499in}{1.599727in}}%
\pgfpathlineto{\pgfqpoint{0.389344in}{1.598489in}}%
\pgfpathlineto{\pgfqpoint{0.397064in}{1.602137in}}%
\pgfpathlineto{\pgfqpoint{0.404299in}{1.600395in}}%
\pgfpathlineto{\pgfqpoint{0.395865in}{1.592654in}}%
\pgfpathlineto{\pgfqpoint{0.390435in}{1.585257in}}%
\pgfpathlineto{\pgfqpoint{0.384929in}{1.586076in}}%
\pgfpathlineto{\pgfqpoint{0.381248in}{1.578474in}}%
\pgfpathlineto{\pgfqpoint{0.387396in}{1.575981in}}%
\pgfpathlineto{\pgfqpoint{0.389591in}{1.567203in}}%
\pgfpathlineto{\pgfqpoint{0.384587in}{1.562231in}}%
\pgfpathlineto{\pgfqpoint{0.383232in}{1.552564in}}%
\pgfpathlineto{\pgfqpoint{0.397486in}{1.548285in}}%
\pgfpathclose%
\pgfusepath{fill}%
\end{pgfscope}%
\begin{pgfscope}%
\pgfpathrectangle{\pgfqpoint{0.100000in}{0.100000in}}{\pgfqpoint{3.007045in}{1.925000in}}%
\pgfusepath{clip}%
\pgfsetbuttcap%
\pgfsetmiterjoin%
\definecolor{currentfill}{rgb}{0.541961,0.749527,0.865606}%
\pgfsetfillcolor{currentfill}%
\pgfsetlinewidth{0.000000pt}%
\definecolor{currentstroke}{rgb}{0.000000,0.000000,0.000000}%
\pgfsetstrokecolor{currentstroke}%
\pgfsetstrokeopacity{0.000000}%
\pgfsetdash{}{0pt}%
\pgfpathmoveto{\pgfqpoint{2.283471in}{1.248041in}}%
\pgfpathlineto{\pgfqpoint{2.285375in}{1.228022in}}%
\pgfpathlineto{\pgfqpoint{2.265786in}{1.225833in}}%
\pgfpathlineto{\pgfqpoint{2.266331in}{1.211577in}}%
\pgfpathlineto{\pgfqpoint{2.262289in}{1.211080in}}%
\pgfpathlineto{\pgfqpoint{2.247429in}{1.209448in}}%
\pgfpathlineto{\pgfqpoint{2.247375in}{1.219993in}}%
\pgfpathlineto{\pgfqpoint{2.233016in}{1.218375in}}%
\pgfpathlineto{\pgfqpoint{2.231222in}{1.236425in}}%
\pgfpathlineto{\pgfqpoint{2.228217in}{1.264785in}}%
\pgfpathlineto{\pgfqpoint{2.239459in}{1.266008in}}%
\pgfpathlineto{\pgfqpoint{2.248890in}{1.267034in}}%
\pgfpathlineto{\pgfqpoint{2.249581in}{1.261349in}}%
\pgfpathlineto{\pgfqpoint{2.261811in}{1.262682in}}%
\pgfpathlineto{\pgfqpoint{2.262780in}{1.245532in}}%
\pgfpathlineto{\pgfqpoint{2.283471in}{1.248041in}}%
\pgfpathclose%
\pgfusepath{fill}%
\end{pgfscope}%
\begin{pgfscope}%
\pgfpathrectangle{\pgfqpoint{0.100000in}{0.100000in}}{\pgfqpoint{3.007045in}{1.925000in}}%
\pgfusepath{clip}%
\pgfsetbuttcap%
\pgfsetmiterjoin%
\definecolor{currentfill}{rgb}{0.554510,0.756417,0.868312}%
\pgfsetfillcolor{currentfill}%
\pgfsetlinewidth{0.000000pt}%
\definecolor{currentstroke}{rgb}{0.000000,0.000000,0.000000}%
\pgfsetstrokecolor{currentstroke}%
\pgfsetstrokeopacity{0.000000}%
\pgfsetdash{}{0pt}%
\pgfpathmoveto{\pgfqpoint{2.358324in}{1.060460in}}%
\pgfpathlineto{\pgfqpoint{2.350874in}{1.053026in}}%
\pgfpathlineto{\pgfqpoint{2.347601in}{1.052815in}}%
\pgfpathlineto{\pgfqpoint{2.340592in}{1.057330in}}%
\pgfpathlineto{\pgfqpoint{2.334876in}{1.050418in}}%
\pgfpathlineto{\pgfqpoint{2.333634in}{1.044388in}}%
\pgfpathlineto{\pgfqpoint{2.321129in}{1.054650in}}%
\pgfpathlineto{\pgfqpoint{2.322848in}{1.065287in}}%
\pgfpathlineto{\pgfqpoint{2.327579in}{1.069371in}}%
\pgfpathlineto{\pgfqpoint{2.316608in}{1.083386in}}%
\pgfpathlineto{\pgfqpoint{2.328598in}{1.094602in}}%
\pgfpathlineto{\pgfqpoint{2.342975in}{1.090758in}}%
\pgfpathlineto{\pgfqpoint{2.350883in}{1.086320in}}%
\pgfpathlineto{\pgfqpoint{2.358589in}{1.084899in}}%
\pgfpathlineto{\pgfqpoint{2.353799in}{1.080602in}}%
\pgfpathlineto{\pgfqpoint{2.350488in}{1.072384in}}%
\pgfpathlineto{\pgfqpoint{2.352145in}{1.068229in}}%
\pgfpathlineto{\pgfqpoint{2.358324in}{1.060460in}}%
\pgfpathclose%
\pgfusepath{fill}%
\end{pgfscope}%
\begin{pgfscope}%
\pgfpathrectangle{\pgfqpoint{0.100000in}{0.100000in}}{\pgfqpoint{3.007045in}{1.925000in}}%
\pgfusepath{clip}%
\pgfsetbuttcap%
\pgfsetmiterjoin%
\definecolor{currentfill}{rgb}{0.401953,0.670296,0.832326}%
\pgfsetfillcolor{currentfill}%
\pgfsetlinewidth{0.000000pt}%
\definecolor{currentstroke}{rgb}{0.000000,0.000000,0.000000}%
\pgfsetstrokecolor{currentstroke}%
\pgfsetstrokeopacity{0.000000}%
\pgfsetdash{}{0pt}%
\pgfpathmoveto{\pgfqpoint{1.835531in}{0.660856in}}%
\pgfpathlineto{\pgfqpoint{1.860218in}{0.661283in}}%
\pgfpathlineto{\pgfqpoint{1.860335in}{0.652602in}}%
\pgfpathlineto{\pgfqpoint{1.866139in}{0.652691in}}%
\pgfpathlineto{\pgfqpoint{1.866438in}{0.640996in}}%
\pgfpathlineto{\pgfqpoint{1.872143in}{0.641119in}}%
\pgfpathlineto{\pgfqpoint{1.872399in}{0.629573in}}%
\pgfpathlineto{\pgfqpoint{1.863641in}{0.629386in}}%
\pgfpathlineto{\pgfqpoint{1.860787in}{0.623525in}}%
\pgfpathlineto{\pgfqpoint{1.855514in}{0.623416in}}%
\pgfpathlineto{\pgfqpoint{1.854631in}{0.617364in}}%
\pgfpathlineto{\pgfqpoint{1.851931in}{0.616696in}}%
\pgfpathlineto{\pgfqpoint{1.846842in}{0.625571in}}%
\pgfpathlineto{\pgfqpoint{1.847519in}{0.631354in}}%
\pgfpathlineto{\pgfqpoint{1.843180in}{0.636011in}}%
\pgfpathlineto{\pgfqpoint{1.837976in}{0.644603in}}%
\pgfpathlineto{\pgfqpoint{1.839877in}{0.650274in}}%
\pgfpathlineto{\pgfqpoint{1.838775in}{0.656201in}}%
\pgfpathlineto{\pgfqpoint{1.835531in}{0.660856in}}%
\pgfpathclose%
\pgfusepath{fill}%
\end{pgfscope}%
\begin{pgfscope}%
\pgfpathrectangle{\pgfqpoint{0.100000in}{0.100000in}}{\pgfqpoint{3.007045in}{1.925000in}}%
\pgfusepath{clip}%
\pgfsetbuttcap%
\pgfsetmiterjoin%
\definecolor{currentfill}{rgb}{0.321246,0.615179,0.800830}%
\pgfsetfillcolor{currentfill}%
\pgfsetlinewidth{0.000000pt}%
\definecolor{currentstroke}{rgb}{0.000000,0.000000,0.000000}%
\pgfsetstrokecolor{currentstroke}%
\pgfsetstrokeopacity{0.000000}%
\pgfsetdash{}{0pt}%
\pgfpathmoveto{\pgfqpoint{1.460453in}{1.255605in}}%
\pgfpathlineto{\pgfqpoint{1.420426in}{1.258524in}}%
\pgfpathlineto{\pgfqpoint{1.420694in}{1.261942in}}%
\pgfpathlineto{\pgfqpoint{1.390844in}{1.264339in}}%
\pgfpathlineto{\pgfqpoint{1.392134in}{1.280968in}}%
\pgfpathlineto{\pgfqpoint{1.394532in}{1.283801in}}%
\pgfpathlineto{\pgfqpoint{1.395479in}{1.295197in}}%
\pgfpathlineto{\pgfqpoint{1.422895in}{1.292969in}}%
\pgfpathlineto{\pgfqpoint{1.423796in}{1.304378in}}%
\pgfpathlineto{\pgfqpoint{1.427276in}{1.304079in}}%
\pgfpathlineto{\pgfqpoint{1.455754in}{1.302023in}}%
\pgfpathlineto{\pgfqpoint{1.462537in}{1.301517in}}%
\pgfpathlineto{\pgfqpoint{1.460924in}{1.278668in}}%
\pgfpathlineto{\pgfqpoint{1.461934in}{1.278597in}}%
\pgfpathlineto{\pgfqpoint{1.460453in}{1.255605in}}%
\pgfpathclose%
\pgfusepath{fill}%
\end{pgfscope}%
\begin{pgfscope}%
\pgfpathrectangle{\pgfqpoint{0.100000in}{0.100000in}}{\pgfqpoint{3.007045in}{1.925000in}}%
\pgfusepath{clip}%
\pgfsetbuttcap%
\pgfsetmiterjoin%
\definecolor{currentfill}{rgb}{0.306113,0.604844,0.794925}%
\pgfsetfillcolor{currentfill}%
\pgfsetlinewidth{0.000000pt}%
\definecolor{currentstroke}{rgb}{0.000000,0.000000,0.000000}%
\pgfsetstrokecolor{currentstroke}%
\pgfsetstrokeopacity{0.000000}%
\pgfsetdash{}{0pt}%
\pgfpathmoveto{\pgfqpoint{2.191739in}{0.996087in}}%
\pgfpathlineto{\pgfqpoint{2.189732in}{1.023554in}}%
\pgfpathlineto{\pgfqpoint{2.200660in}{1.024227in}}%
\pgfpathlineto{\pgfqpoint{2.209941in}{1.021313in}}%
\pgfpathlineto{\pgfqpoint{2.212201in}{1.031659in}}%
\pgfpathlineto{\pgfqpoint{2.216520in}{1.033633in}}%
\pgfpathlineto{\pgfqpoint{2.223300in}{1.033829in}}%
\pgfpathlineto{\pgfqpoint{2.218727in}{1.043400in}}%
\pgfpathlineto{\pgfqpoint{2.234875in}{1.045920in}}%
\pgfpathlineto{\pgfqpoint{2.241309in}{1.038598in}}%
\pgfpathlineto{\pgfqpoint{2.241044in}{1.027825in}}%
\pgfpathlineto{\pgfqpoint{2.236965in}{1.019520in}}%
\pgfpathlineto{\pgfqpoint{2.247870in}{1.007837in}}%
\pgfpathlineto{\pgfqpoint{2.248936in}{1.000483in}}%
\pgfpathlineto{\pgfqpoint{2.230102in}{0.999971in}}%
\pgfpathlineto{\pgfqpoint{2.216487in}{0.998935in}}%
\pgfpathlineto{\pgfqpoint{2.191739in}{0.996087in}}%
\pgfpathclose%
\pgfusepath{fill}%
\end{pgfscope}%
\begin{pgfscope}%
\pgfpathrectangle{\pgfqpoint{0.100000in}{0.100000in}}{\pgfqpoint{3.007045in}{1.925000in}}%
\pgfusepath{clip}%
\pgfsetbuttcap%
\pgfsetmiterjoin%
\definecolor{currentfill}{rgb}{0.142607,0.456332,0.716601}%
\pgfsetfillcolor{currentfill}%
\pgfsetlinewidth{0.000000pt}%
\definecolor{currentstroke}{rgb}{0.000000,0.000000,0.000000}%
\pgfsetstrokecolor{currentstroke}%
\pgfsetstrokeopacity{0.000000}%
\pgfsetdash{}{0pt}%
\pgfpathmoveto{\pgfqpoint{0.829274in}{1.562729in}}%
\pgfpathlineto{\pgfqpoint{0.834605in}{1.565249in}}%
\pgfpathlineto{\pgfqpoint{0.836443in}{1.560436in}}%
\pgfpathlineto{\pgfqpoint{0.842609in}{1.559070in}}%
\pgfpathlineto{\pgfqpoint{0.849873in}{1.555251in}}%
\pgfpathlineto{\pgfqpoint{0.860657in}{1.551817in}}%
\pgfpathlineto{\pgfqpoint{0.861425in}{1.547079in}}%
\pgfpathlineto{\pgfqpoint{0.872520in}{1.537698in}}%
\pgfpathlineto{\pgfqpoint{0.878844in}{1.527019in}}%
\pgfpathlineto{\pgfqpoint{0.882414in}{1.527693in}}%
\pgfpathlineto{\pgfqpoint{0.884911in}{1.517908in}}%
\pgfpathlineto{\pgfqpoint{0.883755in}{1.512410in}}%
\pgfpathlineto{\pgfqpoint{0.896614in}{1.509734in}}%
\pgfpathlineto{\pgfqpoint{0.895520in}{1.504390in}}%
\pgfpathlineto{\pgfqpoint{0.912230in}{1.501011in}}%
\pgfpathlineto{\pgfqpoint{0.909954in}{1.489763in}}%
\pgfpathlineto{\pgfqpoint{0.898811in}{1.492021in}}%
\pgfpathlineto{\pgfqpoint{0.894533in}{1.469396in}}%
\pgfpathlineto{\pgfqpoint{0.896143in}{1.463222in}}%
\pgfpathlineto{\pgfqpoint{0.892728in}{1.460530in}}%
\pgfpathlineto{\pgfqpoint{0.887062in}{1.465893in}}%
\pgfpathlineto{\pgfqpoint{0.882125in}{1.465664in}}%
\pgfpathlineto{\pgfqpoint{0.884523in}{1.477298in}}%
\pgfpathlineto{\pgfqpoint{0.887326in}{1.476723in}}%
\pgfpathlineto{\pgfqpoint{0.891952in}{1.499400in}}%
\pgfpathlineto{\pgfqpoint{0.877767in}{1.502355in}}%
\pgfpathlineto{\pgfqpoint{0.873074in}{1.479709in}}%
\pgfpathlineto{\pgfqpoint{0.869518in}{1.474687in}}%
\pgfpathlineto{\pgfqpoint{0.861547in}{1.476417in}}%
\pgfpathlineto{\pgfqpoint{0.858463in}{1.461570in}}%
\pgfpathlineto{\pgfqpoint{0.851744in}{1.461596in}}%
\pgfpathlineto{\pgfqpoint{0.850305in}{1.455334in}}%
\pgfpathlineto{\pgfqpoint{0.840009in}{1.457550in}}%
\pgfpathlineto{\pgfqpoint{0.834301in}{1.430229in}}%
\pgfpathlineto{\pgfqpoint{0.808116in}{1.436493in}}%
\pgfpathlineto{\pgfqpoint{0.798116in}{1.438303in}}%
\pgfpathlineto{\pgfqpoint{0.811385in}{1.497272in}}%
\pgfpathlineto{\pgfqpoint{0.809080in}{1.497950in}}%
\pgfpathlineto{\pgfqpoint{0.819243in}{1.542361in}}%
\pgfpathlineto{\pgfqpoint{0.824049in}{1.547593in}}%
\pgfpathlineto{\pgfqpoint{0.823175in}{1.550976in}}%
\pgfpathlineto{\pgfqpoint{0.828248in}{1.558524in}}%
\pgfpathlineto{\pgfqpoint{0.829274in}{1.562729in}}%
\pgfpathclose%
\pgfusepath{fill}%
\end{pgfscope}%
\begin{pgfscope}%
\pgfpathrectangle{\pgfqpoint{0.100000in}{0.100000in}}{\pgfqpoint{3.007045in}{1.925000in}}%
\pgfusepath{clip}%
\pgfsetbuttcap%
\pgfsetmiterjoin%
\definecolor{currentfill}{rgb}{0.696501,0.824837,0.909266}%
\pgfsetfillcolor{currentfill}%
\pgfsetlinewidth{0.000000pt}%
\definecolor{currentstroke}{rgb}{0.000000,0.000000,0.000000}%
\pgfsetstrokecolor{currentstroke}%
\pgfsetstrokeopacity{0.000000}%
\pgfsetdash{}{0pt}%
\pgfpathmoveto{\pgfqpoint{0.429023in}{1.481634in}}%
\pgfpathlineto{\pgfqpoint{0.424324in}{1.471482in}}%
\pgfpathlineto{\pgfqpoint{0.420836in}{1.470570in}}%
\pgfpathlineto{\pgfqpoint{0.412778in}{1.458416in}}%
\pgfpathlineto{\pgfqpoint{0.407285in}{1.452726in}}%
\pgfpathlineto{\pgfqpoint{0.407911in}{1.446662in}}%
\pgfpathlineto{\pgfqpoint{0.399108in}{1.444237in}}%
\pgfpathlineto{\pgfqpoint{0.395246in}{1.441320in}}%
\pgfpathlineto{\pgfqpoint{0.390077in}{1.441086in}}%
\pgfpathlineto{\pgfqpoint{0.385422in}{1.437519in}}%
\pgfpathlineto{\pgfqpoint{0.384276in}{1.433228in}}%
\pgfpathlineto{\pgfqpoint{0.387316in}{1.430255in}}%
\pgfpathlineto{\pgfqpoint{0.385058in}{1.423808in}}%
\pgfpathlineto{\pgfqpoint{0.384397in}{1.412475in}}%
\pgfpathlineto{\pgfqpoint{0.355032in}{1.421828in}}%
\pgfpathlineto{\pgfqpoint{0.355543in}{1.423387in}}%
\pgfpathlineto{\pgfqpoint{0.332528in}{1.430977in}}%
\pgfpathlineto{\pgfqpoint{0.330534in}{1.438279in}}%
\pgfpathlineto{\pgfqpoint{0.327431in}{1.441471in}}%
\pgfpathlineto{\pgfqpoint{0.321640in}{1.452224in}}%
\pgfpathlineto{\pgfqpoint{0.323491in}{1.455583in}}%
\pgfpathlineto{\pgfqpoint{0.323252in}{1.464314in}}%
\pgfpathlineto{\pgfqpoint{0.332049in}{1.475470in}}%
\pgfpathlineto{\pgfqpoint{0.345804in}{1.489825in}}%
\pgfpathlineto{\pgfqpoint{0.349534in}{1.496541in}}%
\pgfpathlineto{\pgfqpoint{0.348214in}{1.499832in}}%
\pgfpathlineto{\pgfqpoint{0.355538in}{1.511740in}}%
\pgfpathlineto{\pgfqpoint{0.360884in}{1.523094in}}%
\pgfpathlineto{\pgfqpoint{0.374697in}{1.518495in}}%
\pgfpathlineto{\pgfqpoint{0.372999in}{1.513265in}}%
\pgfpathlineto{\pgfqpoint{0.378145in}{1.511673in}}%
\pgfpathlineto{\pgfqpoint{0.385689in}{1.509253in}}%
\pgfpathlineto{\pgfqpoint{0.386307in}{1.493716in}}%
\pgfpathlineto{\pgfqpoint{0.391169in}{1.489496in}}%
\pgfpathlineto{\pgfqpoint{0.391596in}{1.486250in}}%
\pgfpathlineto{\pgfqpoint{0.397986in}{1.484025in}}%
\pgfpathlineto{\pgfqpoint{0.399966in}{1.478464in}}%
\pgfpathlineto{\pgfqpoint{0.407088in}{1.477930in}}%
\pgfpathlineto{\pgfqpoint{0.405051in}{1.484139in}}%
\pgfpathlineto{\pgfqpoint{0.408114in}{1.487902in}}%
\pgfpathlineto{\pgfqpoint{0.414605in}{1.487528in}}%
\pgfpathlineto{\pgfqpoint{0.428385in}{1.494011in}}%
\pgfpathlineto{\pgfqpoint{0.431912in}{1.491802in}}%
\pgfpathlineto{\pgfqpoint{0.429023in}{1.481634in}}%
\pgfpathclose%
\pgfusepath{fill}%
\end{pgfscope}%
\begin{pgfscope}%
\pgfpathrectangle{\pgfqpoint{0.100000in}{0.100000in}}{\pgfqpoint{3.007045in}{1.925000in}}%
\pgfusepath{clip}%
\pgfsetbuttcap%
\pgfsetmiterjoin%
\definecolor{currentfill}{rgb}{0.401953,0.670296,0.832326}%
\pgfsetfillcolor{currentfill}%
\pgfsetlinewidth{0.000000pt}%
\definecolor{currentstroke}{rgb}{0.000000,0.000000,0.000000}%
\pgfsetstrokecolor{currentstroke}%
\pgfsetstrokeopacity{0.000000}%
\pgfsetdash{}{0pt}%
\pgfpathmoveto{\pgfqpoint{1.889132in}{1.155695in}}%
\pgfpathlineto{\pgfqpoint{1.884869in}{1.157037in}}%
\pgfpathlineto{\pgfqpoint{1.882047in}{1.149212in}}%
\pgfpathlineto{\pgfqpoint{1.877817in}{1.149309in}}%
\pgfpathlineto{\pgfqpoint{1.875772in}{1.154964in}}%
\pgfpathlineto{\pgfqpoint{1.870359in}{1.156300in}}%
\pgfpathlineto{\pgfqpoint{1.867949in}{1.161555in}}%
\pgfpathlineto{\pgfqpoint{1.862920in}{1.162275in}}%
\pgfpathlineto{\pgfqpoint{1.859930in}{1.165889in}}%
\pgfpathlineto{\pgfqpoint{1.860786in}{1.170457in}}%
\pgfpathlineto{\pgfqpoint{1.859241in}{1.178993in}}%
\pgfpathlineto{\pgfqpoint{1.855189in}{1.180142in}}%
\pgfpathlineto{\pgfqpoint{1.855015in}{1.197513in}}%
\pgfpathlineto{\pgfqpoint{1.854945in}{1.201835in}}%
\pgfpathlineto{\pgfqpoint{1.880439in}{1.202637in}}%
\pgfpathlineto{\pgfqpoint{1.880879in}{1.202651in}}%
\pgfpathlineto{\pgfqpoint{1.880876in}{1.180425in}}%
\pgfpathlineto{\pgfqpoint{1.889421in}{1.180509in}}%
\pgfpathlineto{\pgfqpoint{1.889132in}{1.155695in}}%
\pgfpathclose%
\pgfusepath{fill}%
\end{pgfscope}%
\begin{pgfscope}%
\pgfpathrectangle{\pgfqpoint{0.100000in}{0.100000in}}{\pgfqpoint{3.007045in}{1.925000in}}%
\pgfusepath{clip}%
\pgfsetbuttcap%
\pgfsetmiterjoin%
\definecolor{currentfill}{rgb}{0.248166,0.561892,0.770980}%
\pgfsetfillcolor{currentfill}%
\pgfsetlinewidth{0.000000pt}%
\definecolor{currentstroke}{rgb}{0.000000,0.000000,0.000000}%
\pgfsetstrokecolor{currentstroke}%
\pgfsetstrokeopacity{0.000000}%
\pgfsetdash{}{0pt}%
\pgfpathmoveto{\pgfqpoint{1.564310in}{0.846924in}}%
\pgfpathlineto{\pgfqpoint{1.564075in}{0.841182in}}%
\pgfpathlineto{\pgfqpoint{1.573014in}{0.840820in}}%
\pgfpathlineto{\pgfqpoint{1.572553in}{0.829339in}}%
\pgfpathlineto{\pgfqpoint{1.575401in}{0.829239in}}%
\pgfpathlineto{\pgfqpoint{1.574901in}{0.817578in}}%
\pgfpathlineto{\pgfqpoint{1.570469in}{0.816172in}}%
\pgfpathlineto{\pgfqpoint{1.560430in}{0.818079in}}%
\pgfpathlineto{\pgfqpoint{1.556351in}{0.822025in}}%
\pgfpathlineto{\pgfqpoint{1.543414in}{0.822716in}}%
\pgfpathlineto{\pgfqpoint{1.542619in}{0.830859in}}%
\pgfpathlineto{\pgfqpoint{1.542518in}{0.841248in}}%
\pgfpathlineto{\pgfqpoint{1.545976in}{0.847321in}}%
\pgfpathlineto{\pgfqpoint{1.564310in}{0.846924in}}%
\pgfpathclose%
\pgfusepath{fill}%
\end{pgfscope}%
\begin{pgfscope}%
\pgfpathrectangle{\pgfqpoint{0.100000in}{0.100000in}}{\pgfqpoint{3.007045in}{1.925000in}}%
\pgfusepath{clip}%
\pgfsetbuttcap%
\pgfsetmiterjoin%
\definecolor{currentfill}{rgb}{0.260715,0.573841,0.777209}%
\pgfsetfillcolor{currentfill}%
\pgfsetlinewidth{0.000000pt}%
\definecolor{currentstroke}{rgb}{0.000000,0.000000,0.000000}%
\pgfsetstrokecolor{currentstroke}%
\pgfsetstrokeopacity{0.000000}%
\pgfsetdash{}{0pt}%
\pgfpathmoveto{\pgfqpoint{1.721870in}{1.430045in}}%
\pgfpathlineto{\pgfqpoint{1.702699in}{1.430317in}}%
\pgfpathlineto{\pgfqpoint{1.703058in}{1.453389in}}%
\pgfpathlineto{\pgfqpoint{1.703851in}{1.499277in}}%
\pgfpathlineto{\pgfqpoint{1.703956in}{1.505078in}}%
\pgfpathlineto{\pgfqpoint{1.743883in}{1.504541in}}%
\pgfpathlineto{\pgfqpoint{1.744270in}{1.498734in}}%
\pgfpathlineto{\pgfqpoint{1.744188in}{1.475811in}}%
\pgfpathlineto{\pgfqpoint{1.721890in}{1.476069in}}%
\pgfpathlineto{\pgfqpoint{1.722176in}{1.453096in}}%
\pgfpathlineto{\pgfqpoint{1.721870in}{1.430045in}}%
\pgfpathclose%
\pgfusepath{fill}%
\end{pgfscope}%
\begin{pgfscope}%
\pgfpathrectangle{\pgfqpoint{0.100000in}{0.100000in}}{\pgfqpoint{3.007045in}{1.925000in}}%
\pgfusepath{clip}%
\pgfsetbuttcap%
\pgfsetmiterjoin%
\definecolor{currentfill}{rgb}{0.460392,0.704744,0.848012}%
\pgfsetfillcolor{currentfill}%
\pgfsetlinewidth{0.000000pt}%
\definecolor{currentstroke}{rgb}{0.000000,0.000000,0.000000}%
\pgfsetstrokecolor{currentstroke}%
\pgfsetstrokeopacity{0.000000}%
\pgfsetdash{}{0pt}%
\pgfpathmoveto{\pgfqpoint{1.234424in}{1.392114in}}%
\pgfpathlineto{\pgfqpoint{1.231571in}{1.369429in}}%
\pgfpathlineto{\pgfqpoint{1.226347in}{1.324123in}}%
\pgfpathlineto{\pgfqpoint{1.213860in}{1.325677in}}%
\pgfpathlineto{\pgfqpoint{1.210701in}{1.299766in}}%
\pgfpathlineto{\pgfqpoint{1.184259in}{1.303392in}}%
\pgfpathlineto{\pgfqpoint{1.131992in}{1.310519in}}%
\pgfpathlineto{\pgfqpoint{1.135759in}{1.336378in}}%
\pgfpathlineto{\pgfqpoint{1.137581in}{1.353507in}}%
\pgfpathlineto{\pgfqpoint{1.158048in}{1.350480in}}%
\pgfpathlineto{\pgfqpoint{1.162230in}{1.378562in}}%
\pgfpathlineto{\pgfqpoint{1.162889in}{1.390045in}}%
\pgfpathlineto{\pgfqpoint{1.164443in}{1.401325in}}%
\pgfpathlineto{\pgfqpoint{1.207752in}{1.395280in}}%
\pgfpathlineto{\pgfqpoint{1.234424in}{1.392114in}}%
\pgfpathclose%
\pgfusepath{fill}%
\end{pgfscope}%
\begin{pgfscope}%
\pgfpathrectangle{\pgfqpoint{0.100000in}{0.100000in}}{\pgfqpoint{3.007045in}{1.925000in}}%
\pgfusepath{clip}%
\pgfsetbuttcap%
\pgfsetmiterjoin%
\definecolor{currentfill}{rgb}{0.396909,0.666851,0.830358}%
\pgfsetfillcolor{currentfill}%
\pgfsetlinewidth{0.000000pt}%
\definecolor{currentstroke}{rgb}{0.000000,0.000000,0.000000}%
\pgfsetstrokecolor{currentstroke}%
\pgfsetstrokeopacity{0.000000}%
\pgfsetdash{}{0pt}%
\pgfpathmoveto{\pgfqpoint{1.652796in}{1.200602in}}%
\pgfpathlineto{\pgfqpoint{1.653431in}{1.223538in}}%
\pgfpathlineto{\pgfqpoint{1.654636in}{1.269418in}}%
\pgfpathlineto{\pgfqpoint{1.677475in}{1.268823in}}%
\pgfpathlineto{\pgfqpoint{1.699583in}{1.268350in}}%
\pgfpathlineto{\pgfqpoint{1.699296in}{1.251149in}}%
\pgfpathlineto{\pgfqpoint{1.698475in}{1.199567in}}%
\pgfpathlineto{\pgfqpoint{1.652796in}{1.200602in}}%
\pgfpathclose%
\pgfusepath{fill}%
\end{pgfscope}%
\begin{pgfscope}%
\pgfpathrectangle{\pgfqpoint{0.100000in}{0.100000in}}{\pgfqpoint{3.007045in}{1.925000in}}%
\pgfusepath{clip}%
\pgfsetbuttcap%
\pgfsetmiterjoin%
\definecolor{currentfill}{rgb}{0.441569,0.694410,0.843952}%
\pgfsetfillcolor{currentfill}%
\pgfsetlinewidth{0.000000pt}%
\definecolor{currentstroke}{rgb}{0.000000,0.000000,0.000000}%
\pgfsetstrokecolor{currentstroke}%
\pgfsetstrokeopacity{0.000000}%
\pgfsetdash{}{0pt}%
\pgfpathmoveto{\pgfqpoint{1.650048in}{1.679481in}}%
\pgfpathlineto{\pgfqpoint{1.638529in}{1.679874in}}%
\pgfpathlineto{\pgfqpoint{1.637567in}{1.685608in}}%
\pgfpathlineto{\pgfqpoint{1.638404in}{1.708645in}}%
\pgfpathlineto{\pgfqpoint{1.642987in}{1.708503in}}%
\pgfpathlineto{\pgfqpoint{1.643745in}{1.731622in}}%
\pgfpathlineto{\pgfqpoint{1.643166in}{1.743370in}}%
\pgfpathlineto{\pgfqpoint{1.677065in}{1.742344in}}%
\pgfpathlineto{\pgfqpoint{1.676789in}{1.740990in}}%
\pgfpathlineto{\pgfqpoint{1.705532in}{1.740473in}}%
\pgfpathlineto{\pgfqpoint{1.706015in}{1.718669in}}%
\pgfpathlineto{\pgfqpoint{1.711793in}{1.718564in}}%
\pgfpathlineto{\pgfqpoint{1.711752in}{1.712779in}}%
\pgfpathlineto{\pgfqpoint{1.734818in}{1.712519in}}%
\pgfpathlineto{\pgfqpoint{1.734882in}{1.718354in}}%
\pgfpathlineto{\pgfqpoint{1.740597in}{1.718328in}}%
\pgfpathlineto{\pgfqpoint{1.740624in}{1.724147in}}%
\pgfpathlineto{\pgfqpoint{1.746314in}{1.723984in}}%
\pgfpathlineto{\pgfqpoint{1.746352in}{1.706621in}}%
\pgfpathlineto{\pgfqpoint{1.747387in}{1.695072in}}%
\pgfpathlineto{\pgfqpoint{1.724288in}{1.695202in}}%
\pgfpathlineto{\pgfqpoint{1.689129in}{1.695752in}}%
\pgfpathlineto{\pgfqpoint{1.689494in}{1.678334in}}%
\pgfpathlineto{\pgfqpoint{1.650048in}{1.679481in}}%
\pgfpathclose%
\pgfusepath{fill}%
\end{pgfscope}%
\begin{pgfscope}%
\pgfpathrectangle{\pgfqpoint{0.100000in}{0.100000in}}{\pgfqpoint{3.007045in}{1.925000in}}%
\pgfusepath{clip}%
\pgfsetbuttcap%
\pgfsetmiterjoin%
\definecolor{currentfill}{rgb}{0.346467,0.632403,0.810673}%
\pgfsetfillcolor{currentfill}%
\pgfsetlinewidth{0.000000pt}%
\definecolor{currentstroke}{rgb}{0.000000,0.000000,0.000000}%
\pgfsetstrokecolor{currentstroke}%
\pgfsetstrokeopacity{0.000000}%
\pgfsetdash{}{0pt}%
\pgfpathmoveto{\pgfqpoint{1.767169in}{1.475679in}}%
\pgfpathlineto{\pgfqpoint{1.744188in}{1.475811in}}%
\pgfpathlineto{\pgfqpoint{1.744270in}{1.498734in}}%
\pgfpathlineto{\pgfqpoint{1.755330in}{1.498635in}}%
\pgfpathlineto{\pgfqpoint{1.755381in}{1.509019in}}%
\pgfpathlineto{\pgfqpoint{1.749628in}{1.512662in}}%
\pgfpathlineto{\pgfqpoint{1.749650in}{1.521778in}}%
\pgfpathlineto{\pgfqpoint{1.783670in}{1.521817in}}%
\pgfpathlineto{\pgfqpoint{1.795889in}{1.521880in}}%
\pgfpathlineto{\pgfqpoint{1.795936in}{1.510293in}}%
\pgfpathlineto{\pgfqpoint{1.789742in}{1.510236in}}%
\pgfpathlineto{\pgfqpoint{1.790073in}{1.492987in}}%
\pgfpathlineto{\pgfqpoint{1.782199in}{1.492987in}}%
\pgfpathlineto{\pgfqpoint{1.778660in}{1.495730in}}%
\pgfpathlineto{\pgfqpoint{1.778661in}{1.481448in}}%
\pgfpathlineto{\pgfqpoint{1.767240in}{1.481415in}}%
\pgfpathlineto{\pgfqpoint{1.767169in}{1.475679in}}%
\pgfpathclose%
\pgfusepath{fill}%
\end{pgfscope}%
\begin{pgfscope}%
\pgfpathrectangle{\pgfqpoint{0.100000in}{0.100000in}}{\pgfqpoint{3.007045in}{1.925000in}}%
\pgfusepath{clip}%
\pgfsetbuttcap%
\pgfsetmiterjoin%
\definecolor{currentfill}{rgb}{0.721107,0.835294,0.917878}%
\pgfsetfillcolor{currentfill}%
\pgfsetlinewidth{0.000000pt}%
\definecolor{currentstroke}{rgb}{0.000000,0.000000,0.000000}%
\pgfsetstrokecolor{currentstroke}%
\pgfsetstrokeopacity{0.000000}%
\pgfsetdash{}{0pt}%
\pgfpathmoveto{\pgfqpoint{1.948847in}{1.744739in}}%
\pgfpathlineto{\pgfqpoint{1.956874in}{1.748840in}}%
\pgfpathlineto{\pgfqpoint{1.961605in}{1.738804in}}%
\pgfpathlineto{\pgfqpoint{1.978390in}{1.739623in}}%
\pgfpathlineto{\pgfqpoint{1.988931in}{1.741585in}}%
\pgfpathlineto{\pgfqpoint{1.994032in}{1.740034in}}%
\pgfpathlineto{\pgfqpoint{1.995567in}{1.736342in}}%
\pgfpathlineto{\pgfqpoint{2.001096in}{1.733792in}}%
\pgfpathlineto{\pgfqpoint{2.005613in}{1.736504in}}%
\pgfpathlineto{\pgfqpoint{2.012380in}{1.736151in}}%
\pgfpathlineto{\pgfqpoint{2.011685in}{1.732318in}}%
\pgfpathlineto{\pgfqpoint{1.997324in}{1.723369in}}%
\pgfpathlineto{\pgfqpoint{1.987813in}{1.719063in}}%
\pgfpathlineto{\pgfqpoint{1.972465in}{1.713418in}}%
\pgfpathlineto{\pgfqpoint{1.963898in}{1.707804in}}%
\pgfpathlineto{\pgfqpoint{1.951342in}{1.696561in}}%
\pgfpathlineto{\pgfqpoint{1.950215in}{1.724734in}}%
\pgfpathlineto{\pgfqpoint{1.948847in}{1.744739in}}%
\pgfpathclose%
\pgfusepath{fill}%
\end{pgfscope}%
\begin{pgfscope}%
\pgfpathrectangle{\pgfqpoint{0.100000in}{0.100000in}}{\pgfqpoint{3.007045in}{1.925000in}}%
\pgfusepath{clip}%
\pgfsetbuttcap%
\pgfsetmiterjoin%
\definecolor{currentfill}{rgb}{0.406997,0.673741,0.834295}%
\pgfsetfillcolor{currentfill}%
\pgfsetlinewidth{0.000000pt}%
\definecolor{currentstroke}{rgb}{0.000000,0.000000,0.000000}%
\pgfsetstrokecolor{currentstroke}%
\pgfsetstrokeopacity{0.000000}%
\pgfsetdash{}{0pt}%
\pgfpathmoveto{\pgfqpoint{2.385587in}{1.155405in}}%
\pgfpathlineto{\pgfqpoint{2.380064in}{1.147699in}}%
\pgfpathlineto{\pgfqpoint{2.374431in}{1.146809in}}%
\pgfpathlineto{\pgfqpoint{2.370900in}{1.172765in}}%
\pgfpathlineto{\pgfqpoint{2.364743in}{1.174580in}}%
\pgfpathlineto{\pgfqpoint{2.365382in}{1.186478in}}%
\pgfpathlineto{\pgfqpoint{2.363339in}{1.187311in}}%
\pgfpathlineto{\pgfqpoint{2.362689in}{1.195710in}}%
\pgfpathlineto{\pgfqpoint{2.366865in}{1.205475in}}%
\pgfpathlineto{\pgfqpoint{2.381162in}{1.206723in}}%
\pgfpathlineto{\pgfqpoint{2.393794in}{1.205858in}}%
\pgfpathlineto{\pgfqpoint{2.394275in}{1.199273in}}%
\pgfpathlineto{\pgfqpoint{2.405917in}{1.200954in}}%
\pgfpathlineto{\pgfqpoint{2.405836in}{1.202691in}}%
\pgfpathlineto{\pgfqpoint{2.417290in}{1.203518in}}%
\pgfpathlineto{\pgfqpoint{2.419602in}{1.197771in}}%
\pgfpathlineto{\pgfqpoint{2.418057in}{1.191742in}}%
\pgfpathlineto{\pgfqpoint{2.418866in}{1.180027in}}%
\pgfpathlineto{\pgfqpoint{2.413102in}{1.179720in}}%
\pgfpathlineto{\pgfqpoint{2.413821in}{1.167145in}}%
\pgfpathlineto{\pgfqpoint{2.407704in}{1.166293in}}%
\pgfpathlineto{\pgfqpoint{2.403876in}{1.166074in}}%
\pgfpathlineto{\pgfqpoint{2.404127in}{1.159387in}}%
\pgfpathlineto{\pgfqpoint{2.401519in}{1.149007in}}%
\pgfpathlineto{\pgfqpoint{2.395723in}{1.148881in}}%
\pgfpathlineto{\pgfqpoint{2.392277in}{1.158355in}}%
\pgfpathlineto{\pgfqpoint{2.385587in}{1.155405in}}%
\pgfpathclose%
\pgfusepath{fill}%
\end{pgfscope}%
\begin{pgfscope}%
\pgfpathrectangle{\pgfqpoint{0.100000in}{0.100000in}}{\pgfqpoint{3.007045in}{1.925000in}}%
\pgfusepath{clip}%
\pgfsetbuttcap%
\pgfsetmiterjoin%
\definecolor{currentfill}{rgb}{0.406997,0.673741,0.834295}%
\pgfsetfillcolor{currentfill}%
\pgfsetlinewidth{0.000000pt}%
\definecolor{currentstroke}{rgb}{0.000000,0.000000,0.000000}%
\pgfsetstrokecolor{currentstroke}%
\pgfsetstrokeopacity{0.000000}%
\pgfsetdash{}{0pt}%
\pgfpathmoveto{\pgfqpoint{2.018995in}{1.370976in}}%
\pgfpathlineto{\pgfqpoint{2.020468in}{1.350608in}}%
\pgfpathlineto{\pgfqpoint{2.001078in}{1.349385in}}%
\pgfpathlineto{\pgfqpoint{1.997258in}{1.351410in}}%
\pgfpathlineto{\pgfqpoint{1.995792in}{1.357804in}}%
\pgfpathlineto{\pgfqpoint{1.992817in}{1.361377in}}%
\pgfpathlineto{\pgfqpoint{1.983506in}{1.360972in}}%
\pgfpathlineto{\pgfqpoint{1.983756in}{1.355205in}}%
\pgfpathlineto{\pgfqpoint{1.972422in}{1.354713in}}%
\pgfpathlineto{\pgfqpoint{1.938345in}{1.353373in}}%
\pgfpathlineto{\pgfqpoint{1.937355in}{1.364751in}}%
\pgfpathlineto{\pgfqpoint{1.936000in}{1.405159in}}%
\pgfpathlineto{\pgfqpoint{1.934699in}{1.432796in}}%
\pgfpathlineto{\pgfqpoint{1.953526in}{1.433541in}}%
\pgfpathlineto{\pgfqpoint{1.954322in}{1.428443in}}%
\pgfpathlineto{\pgfqpoint{1.954703in}{1.423577in}}%
\pgfpathlineto{\pgfqpoint{1.961879in}{1.417685in}}%
\pgfpathlineto{\pgfqpoint{1.956575in}{1.409480in}}%
\pgfpathlineto{\pgfqpoint{1.958754in}{1.394373in}}%
\pgfpathlineto{\pgfqpoint{1.960884in}{1.393037in}}%
\pgfpathlineto{\pgfqpoint{1.963562in}{1.383583in}}%
\pgfpathlineto{\pgfqpoint{1.969155in}{1.380236in}}%
\pgfpathlineto{\pgfqpoint{1.980017in}{1.377888in}}%
\pgfpathlineto{\pgfqpoint{1.984280in}{1.369339in}}%
\pgfpathlineto{\pgfqpoint{2.018995in}{1.370976in}}%
\pgfpathclose%
\pgfusepath{fill}%
\end{pgfscope}%
\begin{pgfscope}%
\pgfpathrectangle{\pgfqpoint{0.100000in}{0.100000in}}{\pgfqpoint{3.007045in}{1.925000in}}%
\pgfusepath{clip}%
\pgfsetbuttcap%
\pgfsetmiterjoin%
\definecolor{currentfill}{rgb}{0.632526,0.797647,0.886874}%
\pgfsetfillcolor{currentfill}%
\pgfsetlinewidth{0.000000pt}%
\definecolor{currentstroke}{rgb}{0.000000,0.000000,0.000000}%
\pgfsetstrokecolor{currentstroke}%
\pgfsetstrokeopacity{0.000000}%
\pgfsetdash{}{0pt}%
\pgfpathmoveto{\pgfqpoint{2.604673in}{1.078358in}}%
\pgfpathlineto{\pgfqpoint{2.599528in}{1.076992in}}%
\pgfpathlineto{\pgfqpoint{2.596091in}{1.081050in}}%
\pgfpathlineto{\pgfqpoint{2.591032in}{1.080451in}}%
\pgfpathlineto{\pgfqpoint{2.584592in}{1.075050in}}%
\pgfpathlineto{\pgfqpoint{2.578887in}{1.072582in}}%
\pgfpathlineto{\pgfqpoint{2.574988in}{1.077569in}}%
\pgfpathlineto{\pgfqpoint{2.564592in}{1.081776in}}%
\pgfpathlineto{\pgfqpoint{2.562917in}{1.088318in}}%
\pgfpathlineto{\pgfqpoint{2.564677in}{1.094671in}}%
\pgfpathlineto{\pgfqpoint{2.570652in}{1.100774in}}%
\pgfpathlineto{\pgfqpoint{2.573073in}{1.099388in}}%
\pgfpathlineto{\pgfqpoint{2.578351in}{1.105297in}}%
\pgfpathlineto{\pgfqpoint{2.579736in}{1.109975in}}%
\pgfpathlineto{\pgfqpoint{2.584797in}{1.114957in}}%
\pgfpathlineto{\pgfqpoint{2.587434in}{1.124379in}}%
\pgfpathlineto{\pgfqpoint{2.593692in}{1.126779in}}%
\pgfpathlineto{\pgfqpoint{2.598994in}{1.120903in}}%
\pgfpathlineto{\pgfqpoint{2.602817in}{1.120583in}}%
\pgfpathlineto{\pgfqpoint{2.609095in}{1.111445in}}%
\pgfpathlineto{\pgfqpoint{2.612770in}{1.112672in}}%
\pgfpathlineto{\pgfqpoint{2.621502in}{1.105990in}}%
\pgfpathlineto{\pgfqpoint{2.626460in}{1.104802in}}%
\pgfpathlineto{\pgfqpoint{2.623729in}{1.094309in}}%
\pgfpathlineto{\pgfqpoint{2.616873in}{1.090165in}}%
\pgfpathlineto{\pgfqpoint{2.614846in}{1.077543in}}%
\pgfpathlineto{\pgfqpoint{2.604673in}{1.078358in}}%
\pgfpathclose%
\pgfusepath{fill}%
\end{pgfscope}%
\begin{pgfscope}%
\pgfpathrectangle{\pgfqpoint{0.100000in}{0.100000in}}{\pgfqpoint{3.007045in}{1.925000in}}%
\pgfusepath{clip}%
\pgfsetbuttcap%
\pgfsetmiterjoin%
\definecolor{currentfill}{rgb}{0.396909,0.666851,0.830358}%
\pgfsetfillcolor{currentfill}%
\pgfsetlinewidth{0.000000pt}%
\definecolor{currentstroke}{rgb}{0.000000,0.000000,0.000000}%
\pgfsetstrokecolor{currentstroke}%
\pgfsetstrokeopacity{0.000000}%
\pgfsetdash{}{0pt}%
\pgfpathmoveto{\pgfqpoint{0.530203in}{1.618670in}}%
\pgfpathlineto{\pgfqpoint{0.573938in}{1.605496in}}%
\pgfpathlineto{\pgfqpoint{0.596099in}{1.599150in}}%
\pgfpathlineto{\pgfqpoint{0.588372in}{1.571730in}}%
\pgfpathlineto{\pgfqpoint{0.586719in}{1.572193in}}%
\pgfpathlineto{\pgfqpoint{0.578445in}{1.544847in}}%
\pgfpathlineto{\pgfqpoint{0.605512in}{1.537505in}}%
\pgfpathlineto{\pgfqpoint{0.592563in}{1.489427in}}%
\pgfpathlineto{\pgfqpoint{0.562393in}{1.497838in}}%
\pgfpathlineto{\pgfqpoint{0.520911in}{1.509711in}}%
\pgfpathlineto{\pgfqpoint{0.534635in}{1.557261in}}%
\pgfpathlineto{\pgfqpoint{0.513001in}{1.563845in}}%
\pgfpathlineto{\pgfqpoint{0.524578in}{1.602426in}}%
\pgfpathlineto{\pgfqpoint{0.530203in}{1.618670in}}%
\pgfpathclose%
\pgfusepath{fill}%
\end{pgfscope}%
\begin{pgfscope}%
\pgfpathrectangle{\pgfqpoint{0.100000in}{0.100000in}}{\pgfqpoint{3.007045in}{1.925000in}}%
\pgfusepath{clip}%
\pgfsetbuttcap%
\pgfsetmiterjoin%
\definecolor{currentfill}{rgb}{0.270804,0.580730,0.781146}%
\pgfsetfillcolor{currentfill}%
\pgfsetlinewidth{0.000000pt}%
\definecolor{currentstroke}{rgb}{0.000000,0.000000,0.000000}%
\pgfsetstrokecolor{currentstroke}%
\pgfsetstrokeopacity{0.000000}%
\pgfsetdash{}{0pt}%
\pgfpathmoveto{\pgfqpoint{1.799463in}{0.919527in}}%
\pgfpathlineto{\pgfqpoint{1.795605in}{0.946063in}}%
\pgfpathlineto{\pgfqpoint{1.771725in}{0.946010in}}%
\pgfpathlineto{\pgfqpoint{1.772350in}{0.979565in}}%
\pgfpathlineto{\pgfqpoint{1.792522in}{0.979422in}}%
\pgfpathlineto{\pgfqpoint{1.792483in}{0.985928in}}%
\pgfpathlineto{\pgfqpoint{1.821509in}{0.984906in}}%
\pgfpathlineto{\pgfqpoint{1.821240in}{0.968478in}}%
\pgfpathlineto{\pgfqpoint{1.832387in}{0.968604in}}%
\pgfpathlineto{\pgfqpoint{1.832421in}{0.955990in}}%
\pgfpathlineto{\pgfqpoint{1.843659in}{0.955855in}}%
\pgfpathlineto{\pgfqpoint{1.845759in}{0.949170in}}%
\pgfpathlineto{\pgfqpoint{1.849514in}{0.944275in}}%
\pgfpathlineto{\pgfqpoint{1.855176in}{0.941374in}}%
\pgfpathlineto{\pgfqpoint{1.855032in}{0.933900in}}%
\pgfpathlineto{\pgfqpoint{1.851232in}{0.928170in}}%
\pgfpathlineto{\pgfqpoint{1.851596in}{0.920385in}}%
\pgfpathlineto{\pgfqpoint{1.830635in}{0.919840in}}%
\pgfpathlineto{\pgfqpoint{1.819183in}{0.919101in}}%
\pgfpathlineto{\pgfqpoint{1.799463in}{0.919527in}}%
\pgfpathclose%
\pgfusepath{fill}%
\end{pgfscope}%
\begin{pgfscope}%
\pgfpathrectangle{\pgfqpoint{0.100000in}{0.100000in}}{\pgfqpoint{3.007045in}{1.925000in}}%
\pgfusepath{clip}%
\pgfsetbuttcap%
\pgfsetmiterjoin%
\definecolor{currentfill}{rgb}{0.435294,0.690965,0.842599}%
\pgfsetfillcolor{currentfill}%
\pgfsetlinewidth{0.000000pt}%
\definecolor{currentstroke}{rgb}{0.000000,0.000000,0.000000}%
\pgfsetstrokecolor{currentstroke}%
\pgfsetstrokeopacity{0.000000}%
\pgfsetdash{}{0pt}%
\pgfpathmoveto{\pgfqpoint{2.584761in}{0.832670in}}%
\pgfpathlineto{\pgfqpoint{2.584582in}{0.822960in}}%
\pgfpathlineto{\pgfqpoint{2.575563in}{0.818701in}}%
\pgfpathlineto{\pgfqpoint{2.570619in}{0.823049in}}%
\pgfpathlineto{\pgfqpoint{2.556125in}{0.810180in}}%
\pgfpathlineto{\pgfqpoint{2.545370in}{0.817990in}}%
\pgfpathlineto{\pgfqpoint{2.532363in}{0.822168in}}%
\pgfpathlineto{\pgfqpoint{2.525782in}{0.825781in}}%
\pgfpathlineto{\pgfqpoint{2.521533in}{0.825837in}}%
\pgfpathlineto{\pgfqpoint{2.530038in}{0.837997in}}%
\pgfpathlineto{\pgfqpoint{2.516115in}{0.841918in}}%
\pgfpathlineto{\pgfqpoint{2.506929in}{0.849528in}}%
\pgfpathlineto{\pgfqpoint{2.503173in}{0.844767in}}%
\pgfpathlineto{\pgfqpoint{2.492621in}{0.846691in}}%
\pgfpathlineto{\pgfqpoint{2.488445in}{0.853368in}}%
\pgfpathlineto{\pgfqpoint{2.482424in}{0.851583in}}%
\pgfpathlineto{\pgfqpoint{2.486659in}{0.864555in}}%
\pgfpathlineto{\pgfqpoint{2.484386in}{0.869099in}}%
\pgfpathlineto{\pgfqpoint{2.485317in}{0.874629in}}%
\pgfpathlineto{\pgfqpoint{2.490797in}{0.879764in}}%
\pgfpathlineto{\pgfqpoint{2.496666in}{0.891862in}}%
\pgfpathlineto{\pgfqpoint{2.502098in}{0.888878in}}%
\pgfpathlineto{\pgfqpoint{2.508933in}{0.891663in}}%
\pgfpathlineto{\pgfqpoint{2.508172in}{0.896146in}}%
\pgfpathlineto{\pgfqpoint{2.537783in}{0.898762in}}%
\pgfpathlineto{\pgfqpoint{2.538562in}{0.893175in}}%
\pgfpathlineto{\pgfqpoint{2.545276in}{0.888835in}}%
\pgfpathlineto{\pgfqpoint{2.541892in}{0.875274in}}%
\pgfpathlineto{\pgfqpoint{2.551190in}{0.869149in}}%
\pgfpathlineto{\pgfqpoint{2.555420in}{0.871742in}}%
\pgfpathlineto{\pgfqpoint{2.557919in}{0.868208in}}%
\pgfpathlineto{\pgfqpoint{2.558099in}{0.855702in}}%
\pgfpathlineto{\pgfqpoint{2.560497in}{0.852920in}}%
\pgfpathlineto{\pgfqpoint{2.559661in}{0.847879in}}%
\pgfpathlineto{\pgfqpoint{2.566054in}{0.842571in}}%
\pgfpathlineto{\pgfqpoint{2.571176in}{0.834427in}}%
\pgfpathlineto{\pgfqpoint{2.577826in}{0.830633in}}%
\pgfpathlineto{\pgfqpoint{2.584761in}{0.832670in}}%
\pgfpathclose%
\pgfusepath{fill}%
\end{pgfscope}%
\begin{pgfscope}%
\pgfpathrectangle{\pgfqpoint{0.100000in}{0.100000in}}{\pgfqpoint{3.007045in}{1.925000in}}%
\pgfusepath{clip}%
\pgfsetbuttcap%
\pgfsetmiterjoin%
\definecolor{currentfill}{rgb}{0.793449,0.870142,0.942914}%
\pgfsetfillcolor{currentfill}%
\pgfsetlinewidth{0.000000pt}%
\definecolor{currentstroke}{rgb}{0.000000,0.000000,0.000000}%
\pgfsetstrokecolor{currentstroke}%
\pgfsetstrokeopacity{0.000000}%
\pgfsetdash{}{0pt}%
\pgfpathmoveto{\pgfqpoint{2.413670in}{0.911416in}}%
\pgfpathlineto{\pgfqpoint{2.413500in}{0.907511in}}%
\pgfpathlineto{\pgfqpoint{2.408265in}{0.902647in}}%
\pgfpathlineto{\pgfqpoint{2.403903in}{0.896059in}}%
\pgfpathlineto{\pgfqpoint{2.402990in}{0.891059in}}%
\pgfpathlineto{\pgfqpoint{2.391995in}{0.891918in}}%
\pgfpathlineto{\pgfqpoint{2.389462in}{0.896298in}}%
\pgfpathlineto{\pgfqpoint{2.385967in}{0.896018in}}%
\pgfpathlineto{\pgfqpoint{2.385392in}{0.899513in}}%
\pgfpathlineto{\pgfqpoint{2.391763in}{0.908652in}}%
\pgfpathlineto{\pgfqpoint{2.383505in}{0.917672in}}%
\pgfpathlineto{\pgfqpoint{2.378628in}{0.917255in}}%
\pgfpathlineto{\pgfqpoint{2.379773in}{0.923563in}}%
\pgfpathlineto{\pgfqpoint{2.381105in}{0.925986in}}%
\pgfpathlineto{\pgfqpoint{2.392521in}{0.928792in}}%
\pgfpathlineto{\pgfqpoint{2.398481in}{0.931422in}}%
\pgfpathlineto{\pgfqpoint{2.406157in}{0.920864in}}%
\pgfpathlineto{\pgfqpoint{2.410454in}{0.917486in}}%
\pgfpathlineto{\pgfqpoint{2.413670in}{0.911416in}}%
\pgfpathclose%
\pgfusepath{fill}%
\end{pgfscope}%
\begin{pgfscope}%
\pgfpathrectangle{\pgfqpoint{0.100000in}{0.100000in}}{\pgfqpoint{3.007045in}{1.925000in}}%
\pgfusepath{clip}%
\pgfsetbuttcap%
\pgfsetmiterjoin%
\definecolor{currentfill}{rgb}{0.504314,0.728858,0.857486}%
\pgfsetfillcolor{currentfill}%
\pgfsetlinewidth{0.000000pt}%
\definecolor{currentstroke}{rgb}{0.000000,0.000000,0.000000}%
\pgfsetstrokecolor{currentstroke}%
\pgfsetstrokeopacity{0.000000}%
\pgfsetdash{}{0pt}%
\pgfpathmoveto{\pgfqpoint{1.923472in}{0.954935in}}%
\pgfpathlineto{\pgfqpoint{1.921677in}{0.946416in}}%
\pgfpathlineto{\pgfqpoint{1.921721in}{0.941554in}}%
\pgfpathlineto{\pgfqpoint{1.915999in}{0.941499in}}%
\pgfpathlineto{\pgfqpoint{1.916049in}{0.935756in}}%
\pgfpathlineto{\pgfqpoint{1.910330in}{0.935711in}}%
\pgfpathlineto{\pgfqpoint{1.910317in}{0.941444in}}%
\pgfpathlineto{\pgfqpoint{1.887586in}{0.941336in}}%
\pgfpathlineto{\pgfqpoint{1.884698in}{0.944172in}}%
\pgfpathlineto{\pgfqpoint{1.884426in}{0.969017in}}%
\pgfpathlineto{\pgfqpoint{1.886046in}{0.969526in}}%
\pgfpathlineto{\pgfqpoint{1.890382in}{0.969613in}}%
\pgfpathlineto{\pgfqpoint{1.890280in}{0.989939in}}%
\pgfpathlineto{\pgfqpoint{1.924728in}{0.990132in}}%
\pgfpathlineto{\pgfqpoint{1.924925in}{0.970656in}}%
\pgfpathlineto{\pgfqpoint{1.923336in}{0.970590in}}%
\pgfpathlineto{\pgfqpoint{1.923472in}{0.954935in}}%
\pgfpathclose%
\pgfusepath{fill}%
\end{pgfscope}%
\begin{pgfscope}%
\pgfpathrectangle{\pgfqpoint{0.100000in}{0.100000in}}{\pgfqpoint{3.007045in}{1.925000in}}%
\pgfusepath{clip}%
\pgfsetbuttcap%
\pgfsetmiterjoin%
\definecolor{currentfill}{rgb}{0.652211,0.806013,0.893764}%
\pgfsetfillcolor{currentfill}%
\pgfsetlinewidth{0.000000pt}%
\definecolor{currentstroke}{rgb}{0.000000,0.000000,0.000000}%
\pgfsetstrokecolor{currentstroke}%
\pgfsetstrokeopacity{0.000000}%
\pgfsetdash{}{0pt}%
\pgfpathmoveto{\pgfqpoint{2.680930in}{1.081273in}}%
\pgfpathlineto{\pgfqpoint{2.683306in}{1.069582in}}%
\pgfpathlineto{\pgfqpoint{2.679621in}{1.057585in}}%
\pgfpathlineto{\pgfqpoint{2.672729in}{1.056217in}}%
\pgfpathlineto{\pgfqpoint{2.643747in}{1.050644in}}%
\pgfpathlineto{\pgfqpoint{2.608257in}{1.044199in}}%
\pgfpathlineto{\pgfqpoint{2.604048in}{1.043452in}}%
\pgfpathlineto{\pgfqpoint{2.604673in}{1.078358in}}%
\pgfpathlineto{\pgfqpoint{2.614846in}{1.077543in}}%
\pgfpathlineto{\pgfqpoint{2.616873in}{1.090165in}}%
\pgfpathlineto{\pgfqpoint{2.623729in}{1.094309in}}%
\pgfpathlineto{\pgfqpoint{2.626460in}{1.104802in}}%
\pgfpathlineto{\pgfqpoint{2.633816in}{1.102244in}}%
\pgfpathlineto{\pgfqpoint{2.641373in}{1.123470in}}%
\pgfpathlineto{\pgfqpoint{2.641185in}{1.127114in}}%
\pgfpathlineto{\pgfqpoint{2.644603in}{1.131634in}}%
\pgfpathlineto{\pgfqpoint{2.649818in}{1.126222in}}%
\pgfpathlineto{\pgfqpoint{2.649572in}{1.112889in}}%
\pgfpathlineto{\pgfqpoint{2.645322in}{1.106286in}}%
\pgfpathlineto{\pgfqpoint{2.646388in}{1.101662in}}%
\pgfpathlineto{\pgfqpoint{2.670092in}{1.099194in}}%
\pgfpathlineto{\pgfqpoint{2.665311in}{1.095015in}}%
\pgfpathlineto{\pgfqpoint{2.667768in}{1.085077in}}%
\pgfpathlineto{\pgfqpoint{2.674370in}{1.085389in}}%
\pgfpathlineto{\pgfqpoint{2.680930in}{1.081273in}}%
\pgfpathclose%
\pgfusepath{fill}%
\end{pgfscope}%
\begin{pgfscope}%
\pgfpathrectangle{\pgfqpoint{0.100000in}{0.100000in}}{\pgfqpoint{3.007045in}{1.925000in}}%
\pgfusepath{clip}%
\pgfsetbuttcap%
\pgfsetmiterjoin%
\definecolor{currentfill}{rgb}{0.491765,0.721968,0.854779}%
\pgfsetfillcolor{currentfill}%
\pgfsetlinewidth{0.000000pt}%
\definecolor{currentstroke}{rgb}{0.000000,0.000000,0.000000}%
\pgfsetstrokecolor{currentstroke}%
\pgfsetstrokeopacity{0.000000}%
\pgfsetdash{}{0pt}%
\pgfpathmoveto{\pgfqpoint{0.878844in}{1.527019in}}%
\pgfpathlineto{\pgfqpoint{0.872520in}{1.537698in}}%
\pgfpathlineto{\pgfqpoint{0.861425in}{1.547079in}}%
\pgfpathlineto{\pgfqpoint{0.860657in}{1.551817in}}%
\pgfpathlineto{\pgfqpoint{0.849873in}{1.555251in}}%
\pgfpathlineto{\pgfqpoint{0.842609in}{1.559070in}}%
\pgfpathlineto{\pgfqpoint{0.836443in}{1.560436in}}%
\pgfpathlineto{\pgfqpoint{0.834605in}{1.565249in}}%
\pgfpathlineto{\pgfqpoint{0.829274in}{1.562729in}}%
\pgfpathlineto{\pgfqpoint{0.827488in}{1.571131in}}%
\pgfpathlineto{\pgfqpoint{0.829905in}{1.577513in}}%
\pgfpathlineto{\pgfqpoint{0.824028in}{1.581837in}}%
\pgfpathlineto{\pgfqpoint{0.824845in}{1.588003in}}%
\pgfpathlineto{\pgfqpoint{0.820142in}{1.591987in}}%
\pgfpathlineto{\pgfqpoint{0.824171in}{1.596209in}}%
\pgfpathlineto{\pgfqpoint{0.825412in}{1.600994in}}%
\pgfpathlineto{\pgfqpoint{0.822960in}{1.604421in}}%
\pgfpathlineto{\pgfqpoint{0.824810in}{1.609423in}}%
\pgfpathlineto{\pgfqpoint{0.830622in}{1.610723in}}%
\pgfpathlineto{\pgfqpoint{0.834363in}{1.617027in}}%
\pgfpathlineto{\pgfqpoint{0.841346in}{1.613085in}}%
\pgfpathlineto{\pgfqpoint{0.853704in}{1.621066in}}%
\pgfpathlineto{\pgfqpoint{0.857696in}{1.638685in}}%
\pgfpathlineto{\pgfqpoint{0.860069in}{1.641241in}}%
\pgfpathlineto{\pgfqpoint{0.865574in}{1.646714in}}%
\pgfpathlineto{\pgfqpoint{0.859318in}{1.660383in}}%
\pgfpathlineto{\pgfqpoint{0.865364in}{1.658676in}}%
\pgfpathlineto{\pgfqpoint{0.871070in}{1.663297in}}%
\pgfpathlineto{\pgfqpoint{0.873352in}{1.663504in}}%
\pgfpathlineto{\pgfqpoint{0.879064in}{1.657147in}}%
\pgfpathlineto{\pgfqpoint{0.890559in}{1.659065in}}%
\pgfpathlineto{\pgfqpoint{0.894600in}{1.662965in}}%
\pgfpathlineto{\pgfqpoint{0.901414in}{1.666132in}}%
\pgfpathlineto{\pgfqpoint{0.906088in}{1.658804in}}%
\pgfpathlineto{\pgfqpoint{0.903714in}{1.653787in}}%
\pgfpathlineto{\pgfqpoint{0.906555in}{1.644348in}}%
\pgfpathlineto{\pgfqpoint{0.905525in}{1.640271in}}%
\pgfpathlineto{\pgfqpoint{0.910956in}{1.624666in}}%
\pgfpathlineto{\pgfqpoint{0.914945in}{1.620213in}}%
\pgfpathlineto{\pgfqpoint{0.912228in}{1.607641in}}%
\pgfpathlineto{\pgfqpoint{0.916159in}{1.603777in}}%
\pgfpathlineto{\pgfqpoint{0.923418in}{1.601071in}}%
\pgfpathlineto{\pgfqpoint{0.926655in}{1.595511in}}%
\pgfpathlineto{\pgfqpoint{0.928393in}{1.581446in}}%
\pgfpathlineto{\pgfqpoint{0.927577in}{1.576545in}}%
\pgfpathlineto{\pgfqpoint{0.933731in}{1.569068in}}%
\pgfpathlineto{\pgfqpoint{0.935411in}{1.569991in}}%
\pgfpathlineto{\pgfqpoint{0.933521in}{1.560492in}}%
\pgfpathlineto{\pgfqpoint{0.905554in}{1.566329in}}%
\pgfpathlineto{\pgfqpoint{0.903257in}{1.555054in}}%
\pgfpathlineto{\pgfqpoint{0.900875in}{1.550400in}}%
\pgfpathlineto{\pgfqpoint{0.897650in}{1.536577in}}%
\pgfpathlineto{\pgfqpoint{0.891068in}{1.533017in}}%
\pgfpathlineto{\pgfqpoint{0.881395in}{1.530161in}}%
\pgfpathlineto{\pgfqpoint{0.878844in}{1.527019in}}%
\pgfpathclose%
\pgfusepath{fill}%
\end{pgfscope}%
\begin{pgfscope}%
\pgfpathrectangle{\pgfqpoint{0.100000in}{0.100000in}}{\pgfqpoint{3.007045in}{1.925000in}}%
\pgfusepath{clip}%
\pgfsetbuttcap%
\pgfsetmiterjoin%
\definecolor{currentfill}{rgb}{0.142607,0.456332,0.716601}%
\pgfsetfillcolor{currentfill}%
\pgfsetlinewidth{0.000000pt}%
\definecolor{currentstroke}{rgb}{0.000000,0.000000,0.000000}%
\pgfsetstrokecolor{currentstroke}%
\pgfsetstrokeopacity{0.000000}%
\pgfsetdash{}{0pt}%
\pgfpathmoveto{\pgfqpoint{1.002522in}{1.140541in}}%
\pgfpathlineto{\pgfqpoint{0.997732in}{1.138462in}}%
\pgfpathlineto{\pgfqpoint{0.992394in}{1.131729in}}%
\pgfpathlineto{\pgfqpoint{0.990253in}{1.132805in}}%
\pgfpathlineto{\pgfqpoint{0.982614in}{1.127004in}}%
\pgfpathlineto{\pgfqpoint{0.975147in}{1.126984in}}%
\pgfpathlineto{\pgfqpoint{0.967830in}{1.113017in}}%
\pgfpathlineto{\pgfqpoint{0.961951in}{1.110649in}}%
\pgfpathlineto{\pgfqpoint{0.958933in}{1.107125in}}%
\pgfpathlineto{\pgfqpoint{0.937370in}{1.110856in}}%
\pgfpathlineto{\pgfqpoint{0.870763in}{1.123009in}}%
\pgfpathlineto{\pgfqpoint{0.854337in}{1.126666in}}%
\pgfpathlineto{\pgfqpoint{0.857375in}{1.143704in}}%
\pgfpathlineto{\pgfqpoint{0.863012in}{1.142510in}}%
\pgfpathlineto{\pgfqpoint{0.864002in}{1.147526in}}%
\pgfpathlineto{\pgfqpoint{0.869732in}{1.146966in}}%
\pgfpathlineto{\pgfqpoint{0.872431in}{1.163681in}}%
\pgfpathlineto{\pgfqpoint{0.877415in}{1.166653in}}%
\pgfpathlineto{\pgfqpoint{0.882037in}{1.174273in}}%
\pgfpathlineto{\pgfqpoint{0.878976in}{1.180671in}}%
\pgfpathlineto{\pgfqpoint{0.875518in}{1.183498in}}%
\pgfpathlineto{\pgfqpoint{0.875936in}{1.191519in}}%
\pgfpathlineto{\pgfqpoint{0.880713in}{1.197892in}}%
\pgfpathlineto{\pgfqpoint{0.887613in}{1.196226in}}%
\pgfpathlineto{\pgfqpoint{0.892855in}{1.198668in}}%
\pgfpathlineto{\pgfqpoint{0.894249in}{1.207621in}}%
\pgfpathlineto{\pgfqpoint{0.900045in}{1.212653in}}%
\pgfpathlineto{\pgfqpoint{0.903422in}{1.212075in}}%
\pgfpathlineto{\pgfqpoint{0.907070in}{1.217261in}}%
\pgfpathlineto{\pgfqpoint{0.914996in}{1.214937in}}%
\pgfpathlineto{\pgfqpoint{0.942870in}{1.209747in}}%
\pgfpathlineto{\pgfqpoint{0.936407in}{1.175969in}}%
\pgfpathlineto{\pgfqpoint{0.913573in}{1.180156in}}%
\pgfpathlineto{\pgfqpoint{0.907847in}{1.175524in}}%
\pgfpathlineto{\pgfqpoint{0.904834in}{1.157752in}}%
\pgfpathlineto{\pgfqpoint{0.963729in}{1.147104in}}%
\pgfpathlineto{\pgfqpoint{1.002522in}{1.140541in}}%
\pgfpathclose%
\pgfusepath{fill}%
\end{pgfscope}%
\begin{pgfscope}%
\pgfpathrectangle{\pgfqpoint{0.100000in}{0.100000in}}{\pgfqpoint{3.007045in}{1.925000in}}%
\pgfusepath{clip}%
\pgfsetbuttcap%
\pgfsetmiterjoin%
\definecolor{currentfill}{rgb}{0.361599,0.642737,0.816578}%
\pgfsetfillcolor{currentfill}%
\pgfsetlinewidth{0.000000pt}%
\definecolor{currentstroke}{rgb}{0.000000,0.000000,0.000000}%
\pgfsetstrokecolor{currentstroke}%
\pgfsetstrokeopacity{0.000000}%
\pgfsetdash{}{0pt}%
\pgfpathmoveto{\pgfqpoint{1.768458in}{1.001242in}}%
\pgfpathlineto{\pgfqpoint{1.750868in}{1.001271in}}%
\pgfpathlineto{\pgfqpoint{1.749585in}{0.997555in}}%
\pgfpathlineto{\pgfqpoint{1.749489in}{0.974797in}}%
\pgfpathlineto{\pgfqpoint{1.729327in}{0.974933in}}%
\pgfpathlineto{\pgfqpoint{1.729236in}{0.963455in}}%
\pgfpathlineto{\pgfqpoint{1.719212in}{0.963573in}}%
\pgfpathlineto{\pgfqpoint{1.719649in}{1.001488in}}%
\pgfpathlineto{\pgfqpoint{1.721571in}{1.001483in}}%
\pgfpathlineto{\pgfqpoint{1.722263in}{1.049928in}}%
\pgfpathlineto{\pgfqpoint{1.767740in}{1.049552in}}%
\pgfpathlineto{\pgfqpoint{1.768402in}{1.026569in}}%
\pgfpathlineto{\pgfqpoint{1.768458in}{1.001242in}}%
\pgfpathclose%
\pgfusepath{fill}%
\end{pgfscope}%
\begin{pgfscope}%
\pgfpathrectangle{\pgfqpoint{0.100000in}{0.100000in}}{\pgfqpoint{3.007045in}{1.925000in}}%
\pgfusepath{clip}%
\pgfsetbuttcap%
\pgfsetmiterjoin%
\definecolor{currentfill}{rgb}{0.460392,0.704744,0.848012}%
\pgfsetfillcolor{currentfill}%
\pgfsetlinewidth{0.000000pt}%
\definecolor{currentstroke}{rgb}{0.000000,0.000000,0.000000}%
\pgfsetstrokecolor{currentstroke}%
\pgfsetstrokeopacity{0.000000}%
\pgfsetdash{}{0pt}%
\pgfpathmoveto{\pgfqpoint{1.192875in}{1.732905in}}%
\pgfpathlineto{\pgfqpoint{1.188345in}{1.720102in}}%
\pgfpathlineto{\pgfqpoint{1.188167in}{1.711464in}}%
\pgfpathlineto{\pgfqpoint{1.189559in}{1.703789in}}%
\pgfpathlineto{\pgfqpoint{1.186086in}{1.695738in}}%
\pgfpathlineto{\pgfqpoint{1.187986in}{1.693328in}}%
\pgfpathlineto{\pgfqpoint{1.189998in}{1.686616in}}%
\pgfpathlineto{\pgfqpoint{1.153729in}{1.691500in}}%
\pgfpathlineto{\pgfqpoint{1.119627in}{1.696907in}}%
\pgfpathlineto{\pgfqpoint{1.119016in}{1.693144in}}%
\pgfpathlineto{\pgfqpoint{1.102971in}{1.695747in}}%
\pgfpathlineto{\pgfqpoint{1.102514in}{1.707496in}}%
\pgfpathlineto{\pgfqpoint{1.104710in}{1.709101in}}%
\pgfpathlineto{\pgfqpoint{1.107755in}{1.728110in}}%
\pgfpathlineto{\pgfqpoint{1.103070in}{1.731808in}}%
\pgfpathlineto{\pgfqpoint{1.097369in}{1.732746in}}%
\pgfpathlineto{\pgfqpoint{1.092154in}{1.736531in}}%
\pgfpathlineto{\pgfqpoint{1.094250in}{1.746968in}}%
\pgfpathlineto{\pgfqpoint{1.107591in}{1.752201in}}%
\pgfpathlineto{\pgfqpoint{1.110393in}{1.763495in}}%
\pgfpathlineto{\pgfqpoint{1.123379in}{1.762908in}}%
\pgfpathlineto{\pgfqpoint{1.126744in}{1.761331in}}%
\pgfpathlineto{\pgfqpoint{1.135562in}{1.764609in}}%
\pgfpathlineto{\pgfqpoint{1.141833in}{1.762846in}}%
\pgfpathlineto{\pgfqpoint{1.148828in}{1.763186in}}%
\pgfpathlineto{\pgfqpoint{1.152002in}{1.758091in}}%
\pgfpathlineto{\pgfqpoint{1.156612in}{1.751283in}}%
\pgfpathlineto{\pgfqpoint{1.164883in}{1.747708in}}%
\pgfpathlineto{\pgfqpoint{1.170837in}{1.747018in}}%
\pgfpathlineto{\pgfqpoint{1.174368in}{1.744062in}}%
\pgfpathlineto{\pgfqpoint{1.187367in}{1.743646in}}%
\pgfpathlineto{\pgfqpoint{1.193102in}{1.740744in}}%
\pgfpathlineto{\pgfqpoint{1.192875in}{1.732905in}}%
\pgfpathclose%
\pgfusepath{fill}%
\end{pgfscope}%
\begin{pgfscope}%
\pgfpathrectangle{\pgfqpoint{0.100000in}{0.100000in}}{\pgfqpoint{3.007045in}{1.925000in}}%
\pgfusepath{clip}%
\pgfsetbuttcap%
\pgfsetmiterjoin%
\definecolor{currentfill}{rgb}{0.256286,0.570012,0.775163}%
\pgfsetfillcolor{currentfill}%
\pgfsetlinewidth{0.000000pt}%
\definecolor{currentstroke}{rgb}{0.000000,0.000000,0.000000}%
\pgfsetstrokecolor{currentstroke}%
\pgfsetstrokeopacity{0.000000}%
\pgfsetdash{}{0pt}%
\pgfpathmoveto{\pgfqpoint{1.685427in}{0.912299in}}%
\pgfpathlineto{\pgfqpoint{1.685760in}{0.932254in}}%
\pgfpathlineto{\pgfqpoint{1.685998in}{0.946658in}}%
\pgfpathlineto{\pgfqpoint{1.675476in}{0.946791in}}%
\pgfpathlineto{\pgfqpoint{1.675597in}{0.952567in}}%
\pgfpathlineto{\pgfqpoint{1.669953in}{0.952684in}}%
\pgfpathlineto{\pgfqpoint{1.670084in}{0.958402in}}%
\pgfpathlineto{\pgfqpoint{1.664393in}{0.958534in}}%
\pgfpathlineto{\pgfqpoint{1.664644in}{0.969981in}}%
\pgfpathlineto{\pgfqpoint{1.671642in}{0.972779in}}%
\pgfpathlineto{\pgfqpoint{1.670693in}{0.976025in}}%
\pgfpathlineto{\pgfqpoint{1.642047in}{0.976241in}}%
\pgfpathlineto{\pgfqpoint{1.642737in}{1.002945in}}%
\pgfpathlineto{\pgfqpoint{1.659290in}{1.002553in}}%
\pgfpathlineto{\pgfqpoint{1.719649in}{1.001488in}}%
\pgfpathlineto{\pgfqpoint{1.719212in}{0.963573in}}%
\pgfpathlineto{\pgfqpoint{1.729236in}{0.963455in}}%
\pgfpathlineto{\pgfqpoint{1.729327in}{0.974933in}}%
\pgfpathlineto{\pgfqpoint{1.749489in}{0.974797in}}%
\pgfpathlineto{\pgfqpoint{1.749585in}{0.997555in}}%
\pgfpathlineto{\pgfqpoint{1.750868in}{1.001271in}}%
\pgfpathlineto{\pgfqpoint{1.768458in}{1.001242in}}%
\pgfpathlineto{\pgfqpoint{1.772364in}{0.997534in}}%
\pgfpathlineto{\pgfqpoint{1.772350in}{0.979565in}}%
\pgfpathlineto{\pgfqpoint{1.771725in}{0.946010in}}%
\pgfpathlineto{\pgfqpoint{1.766020in}{0.946003in}}%
\pgfpathlineto{\pgfqpoint{1.766027in}{0.940252in}}%
\pgfpathlineto{\pgfqpoint{1.761242in}{0.940278in}}%
\pgfpathlineto{\pgfqpoint{1.757448in}{0.932993in}}%
\pgfpathlineto{\pgfqpoint{1.758114in}{0.923040in}}%
\pgfpathlineto{\pgfqpoint{1.746777in}{0.922489in}}%
\pgfpathlineto{\pgfqpoint{1.740818in}{0.919900in}}%
\pgfpathlineto{\pgfqpoint{1.737553in}{0.926013in}}%
\pgfpathlineto{\pgfqpoint{1.731364in}{0.926038in}}%
\pgfpathlineto{\pgfqpoint{1.708573in}{0.926325in}}%
\pgfpathlineto{\pgfqpoint{1.708380in}{0.911948in}}%
\pgfpathlineto{\pgfqpoint{1.685427in}{0.912299in}}%
\pgfpathclose%
\pgfusepath{fill}%
\end{pgfscope}%
\begin{pgfscope}%
\pgfpathrectangle{\pgfqpoint{0.100000in}{0.100000in}}{\pgfqpoint{3.007045in}{1.925000in}}%
\pgfusepath{clip}%
\pgfsetbuttcap%
\pgfsetmiterjoin%
\definecolor{currentfill}{rgb}{0.031373,0.285675,0.564291}%
\pgfsetfillcolor{currentfill}%
\pgfsetlinewidth{0.000000pt}%
\definecolor{currentstroke}{rgb}{0.000000,0.000000,0.000000}%
\pgfsetstrokecolor{currentstroke}%
\pgfsetstrokeopacity{0.000000}%
\pgfsetdash{}{0pt}%
\pgfpathmoveto{\pgfqpoint{1.551371in}{1.482830in}}%
\pgfpathlineto{\pgfqpoint{1.552579in}{1.474853in}}%
\pgfpathlineto{\pgfqpoint{1.546378in}{1.477641in}}%
\pgfpathlineto{\pgfqpoint{1.542308in}{1.477127in}}%
\pgfpathlineto{\pgfqpoint{1.538292in}{1.482046in}}%
\pgfpathlineto{\pgfqpoint{1.517980in}{1.483159in}}%
\pgfpathlineto{\pgfqpoint{1.517880in}{1.481436in}}%
\pgfpathlineto{\pgfqpoint{1.485701in}{1.483457in}}%
\pgfpathlineto{\pgfqpoint{1.480026in}{1.483812in}}%
\pgfpathlineto{\pgfqpoint{1.481358in}{1.495327in}}%
\pgfpathlineto{\pgfqpoint{1.482943in}{1.518221in}}%
\pgfpathlineto{\pgfqpoint{1.484012in}{1.521937in}}%
\pgfpathlineto{\pgfqpoint{1.489275in}{1.520056in}}%
\pgfpathlineto{\pgfqpoint{1.498577in}{1.523374in}}%
\pgfpathlineto{\pgfqpoint{1.501256in}{1.521545in}}%
\pgfpathlineto{\pgfqpoint{1.506367in}{1.525118in}}%
\pgfpathlineto{\pgfqpoint{1.512562in}{1.521360in}}%
\pgfpathlineto{\pgfqpoint{1.519350in}{1.527670in}}%
\pgfpathlineto{\pgfqpoint{1.518970in}{1.529708in}}%
\pgfpathlineto{\pgfqpoint{1.553171in}{1.527805in}}%
\pgfpathlineto{\pgfqpoint{1.551887in}{1.504699in}}%
\pgfpathlineto{\pgfqpoint{1.551371in}{1.482830in}}%
\pgfpathclose%
\pgfusepath{fill}%
\end{pgfscope}%
\begin{pgfscope}%
\pgfpathrectangle{\pgfqpoint{0.100000in}{0.100000in}}{\pgfqpoint{3.007045in}{1.925000in}}%
\pgfusepath{clip}%
\pgfsetbuttcap%
\pgfsetmiterjoin%
\definecolor{currentfill}{rgb}{0.326290,0.618624,0.802799}%
\pgfsetfillcolor{currentfill}%
\pgfsetlinewidth{0.000000pt}%
\definecolor{currentstroke}{rgb}{0.000000,0.000000,0.000000}%
\pgfsetstrokecolor{currentstroke}%
\pgfsetstrokeopacity{0.000000}%
\pgfsetdash{}{0pt}%
\pgfpathmoveto{\pgfqpoint{2.020031in}{0.633536in}}%
\pgfpathlineto{\pgfqpoint{2.019201in}{0.650937in}}%
\pgfpathlineto{\pgfqpoint{2.013321in}{0.650638in}}%
\pgfpathlineto{\pgfqpoint{2.013078in}{0.656435in}}%
\pgfpathlineto{\pgfqpoint{2.038852in}{0.657801in}}%
\pgfpathlineto{\pgfqpoint{2.047327in}{0.661747in}}%
\pgfpathlineto{\pgfqpoint{2.055851in}{0.662822in}}%
\pgfpathlineto{\pgfqpoint{2.056646in}{0.646958in}}%
\pgfpathlineto{\pgfqpoint{2.058100in}{0.638257in}}%
\pgfpathlineto{\pgfqpoint{2.053466in}{0.638280in}}%
\pgfpathlineto{\pgfqpoint{2.049764in}{0.635164in}}%
\pgfpathlineto{\pgfqpoint{2.020031in}{0.633536in}}%
\pgfpathclose%
\pgfusepath{fill}%
\end{pgfscope}%
\begin{pgfscope}%
\pgfpathrectangle{\pgfqpoint{0.100000in}{0.100000in}}{\pgfqpoint{3.007045in}{1.925000in}}%
\pgfusepath{clip}%
\pgfsetbuttcap%
\pgfsetmiterjoin%
\definecolor{currentfill}{rgb}{0.031373,0.297855,0.582376}%
\pgfsetfillcolor{currentfill}%
\pgfsetlinewidth{0.000000pt}%
\definecolor{currentstroke}{rgb}{0.000000,0.000000,0.000000}%
\pgfsetstrokecolor{currentstroke}%
\pgfsetstrokeopacity{0.000000}%
\pgfsetdash{}{0pt}%
\pgfpathmoveto{\pgfqpoint{1.386813in}{0.624056in}}%
\pgfpathlineto{\pgfqpoint{1.351088in}{0.627381in}}%
\pgfpathlineto{\pgfqpoint{1.349725in}{0.635825in}}%
\pgfpathlineto{\pgfqpoint{1.337100in}{0.644492in}}%
\pgfpathlineto{\pgfqpoint{1.331766in}{0.642740in}}%
\pgfpathlineto{\pgfqpoint{1.333802in}{0.666433in}}%
\pgfpathlineto{\pgfqpoint{1.332017in}{0.666591in}}%
\pgfpathlineto{\pgfqpoint{1.334479in}{0.695294in}}%
\pgfpathlineto{\pgfqpoint{1.319633in}{0.696676in}}%
\pgfpathlineto{\pgfqpoint{1.322173in}{0.725422in}}%
\pgfpathlineto{\pgfqpoint{1.370186in}{0.721452in}}%
\pgfpathlineto{\pgfqpoint{1.398836in}{0.719387in}}%
\pgfpathlineto{\pgfqpoint{1.396350in}{0.690465in}}%
\pgfpathlineto{\pgfqpoint{1.391806in}{0.690752in}}%
\pgfpathlineto{\pgfqpoint{1.386813in}{0.624056in}}%
\pgfpathclose%
\pgfusepath{fill}%
\end{pgfscope}%
\begin{pgfscope}%
\pgfpathrectangle{\pgfqpoint{0.100000in}{0.100000in}}{\pgfqpoint{3.007045in}{1.925000in}}%
\pgfusepath{clip}%
\pgfsetbuttcap%
\pgfsetmiterjoin%
\definecolor{currentfill}{rgb}{0.491765,0.721968,0.854779}%
\pgfsetfillcolor{currentfill}%
\pgfsetlinewidth{0.000000pt}%
\definecolor{currentstroke}{rgb}{0.000000,0.000000,0.000000}%
\pgfsetstrokecolor{currentstroke}%
\pgfsetstrokeopacity{0.000000}%
\pgfsetdash{}{0pt}%
\pgfpathmoveto{\pgfqpoint{2.044424in}{1.118250in}}%
\pgfpathlineto{\pgfqpoint{2.020049in}{1.117235in}}%
\pgfpathlineto{\pgfqpoint{2.018572in}{1.121893in}}%
\pgfpathlineto{\pgfqpoint{2.022871in}{1.129850in}}%
\pgfpathlineto{\pgfqpoint{2.015419in}{1.134149in}}%
\pgfpathlineto{\pgfqpoint{2.005193in}{1.136744in}}%
\pgfpathlineto{\pgfqpoint{2.000724in}{1.130408in}}%
\pgfpathlineto{\pgfqpoint{1.994783in}{1.133102in}}%
\pgfpathlineto{\pgfqpoint{1.991482in}{1.141762in}}%
\pgfpathlineto{\pgfqpoint{1.992952in}{1.144098in}}%
\pgfpathlineto{\pgfqpoint{1.990535in}{1.152936in}}%
\pgfpathlineto{\pgfqpoint{1.986624in}{1.158384in}}%
\pgfpathlineto{\pgfqpoint{1.979066in}{1.163974in}}%
\pgfpathlineto{\pgfqpoint{1.995501in}{1.164493in}}%
\pgfpathlineto{\pgfqpoint{1.997266in}{1.151743in}}%
\pgfpathlineto{\pgfqpoint{2.002639in}{1.150292in}}%
\pgfpathlineto{\pgfqpoint{2.011513in}{1.150677in}}%
\pgfpathlineto{\pgfqpoint{2.019642in}{1.156899in}}%
\pgfpathlineto{\pgfqpoint{2.018445in}{1.173868in}}%
\pgfpathlineto{\pgfqpoint{2.029984in}{1.174625in}}%
\pgfpathlineto{\pgfqpoint{2.049876in}{1.175940in}}%
\pgfpathlineto{\pgfqpoint{2.050719in}{1.164389in}}%
\pgfpathlineto{\pgfqpoint{2.070604in}{1.165647in}}%
\pgfpathlineto{\pgfqpoint{2.071143in}{1.157025in}}%
\pgfpathlineto{\pgfqpoint{2.065520in}{1.156638in}}%
\pgfpathlineto{\pgfqpoint{2.067309in}{1.125374in}}%
\pgfpathlineto{\pgfqpoint{2.049708in}{1.124375in}}%
\pgfpathlineto{\pgfqpoint{2.050157in}{1.118638in}}%
\pgfpathlineto{\pgfqpoint{2.044424in}{1.118250in}}%
\pgfpathclose%
\pgfusepath{fill}%
\end{pgfscope}%
\begin{pgfscope}%
\pgfpathrectangle{\pgfqpoint{0.100000in}{0.100000in}}{\pgfqpoint{3.007045in}{1.925000in}}%
\pgfusepath{clip}%
\pgfsetbuttcap%
\pgfsetmiterjoin%
\definecolor{currentfill}{rgb}{0.396909,0.666851,0.830358}%
\pgfsetfillcolor{currentfill}%
\pgfsetlinewidth{0.000000pt}%
\definecolor{currentstroke}{rgb}{0.000000,0.000000,0.000000}%
\pgfsetstrokecolor{currentstroke}%
\pgfsetstrokeopacity{0.000000}%
\pgfsetdash{}{0pt}%
\pgfpathmoveto{\pgfqpoint{1.968862in}{0.741315in}}%
\pgfpathlineto{\pgfqpoint{1.935072in}{0.740260in}}%
\pgfpathlineto{\pgfqpoint{1.932837in}{0.741766in}}%
\pgfpathlineto{\pgfqpoint{1.930951in}{0.750384in}}%
\pgfpathlineto{\pgfqpoint{1.931371in}{0.753155in}}%
\pgfpathlineto{\pgfqpoint{1.939283in}{0.759647in}}%
\pgfpathlineto{\pgfqpoint{1.938599in}{0.766214in}}%
\pgfpathlineto{\pgfqpoint{1.968042in}{0.766663in}}%
\pgfpathlineto{\pgfqpoint{1.968862in}{0.741315in}}%
\pgfpathclose%
\pgfusepath{fill}%
\end{pgfscope}%
\begin{pgfscope}%
\pgfpathrectangle{\pgfqpoint{0.100000in}{0.100000in}}{\pgfqpoint{3.007045in}{1.925000in}}%
\pgfusepath{clip}%
\pgfsetbuttcap%
\pgfsetmiterjoin%
\definecolor{currentfill}{rgb}{0.356555,0.639293,0.814610}%
\pgfsetfillcolor{currentfill}%
\pgfsetlinewidth{0.000000pt}%
\definecolor{currentstroke}{rgb}{0.000000,0.000000,0.000000}%
\pgfsetstrokecolor{currentstroke}%
\pgfsetstrokeopacity{0.000000}%
\pgfsetdash{}{0pt}%
\pgfpathmoveto{\pgfqpoint{1.455754in}{1.302023in}}%
\pgfpathlineto{\pgfqpoint{1.427276in}{1.304079in}}%
\pgfpathlineto{\pgfqpoint{1.428992in}{1.326930in}}%
\pgfpathlineto{\pgfqpoint{1.424949in}{1.327230in}}%
\pgfpathlineto{\pgfqpoint{1.426380in}{1.344786in}}%
\pgfpathlineto{\pgfqpoint{1.428115in}{1.350443in}}%
\pgfpathlineto{\pgfqpoint{1.429987in}{1.373229in}}%
\pgfpathlineto{\pgfqpoint{1.428602in}{1.373346in}}%
\pgfpathlineto{\pgfqpoint{1.430345in}{1.395850in}}%
\pgfpathlineto{\pgfqpoint{1.430688in}{1.409985in}}%
\pgfpathlineto{\pgfqpoint{1.471839in}{1.406868in}}%
\pgfpathlineto{\pgfqpoint{1.521452in}{1.403789in}}%
\pgfpathlineto{\pgfqpoint{1.521365in}{1.389410in}}%
\pgfpathlineto{\pgfqpoint{1.520062in}{1.366510in}}%
\pgfpathlineto{\pgfqpoint{1.519516in}{1.343574in}}%
\pgfpathlineto{\pgfqpoint{1.486374in}{1.345738in}}%
\pgfpathlineto{\pgfqpoint{1.485080in}{1.322839in}}%
\pgfpathlineto{\pgfqpoint{1.457408in}{1.324892in}}%
\pgfpathlineto{\pgfqpoint{1.455754in}{1.302023in}}%
\pgfpathclose%
\pgfusepath{fill}%
\end{pgfscope}%
\begin{pgfscope}%
\pgfpathrectangle{\pgfqpoint{0.100000in}{0.100000in}}{\pgfqpoint{3.007045in}{1.925000in}}%
\pgfusepath{clip}%
\pgfsetbuttcap%
\pgfsetmiterjoin%
\definecolor{currentfill}{rgb}{0.510588,0.732303,0.858839}%
\pgfsetfillcolor{currentfill}%
\pgfsetlinewidth{0.000000pt}%
\definecolor{currentstroke}{rgb}{0.000000,0.000000,0.000000}%
\pgfsetstrokecolor{currentstroke}%
\pgfsetstrokeopacity{0.000000}%
\pgfsetdash{}{0pt}%
\pgfpathmoveto{\pgfqpoint{2.292953in}{1.341716in}}%
\pgfpathlineto{\pgfqpoint{2.273138in}{1.338797in}}%
\pgfpathlineto{\pgfqpoint{2.271732in}{1.342879in}}%
\pgfpathlineto{\pgfqpoint{2.269373in}{1.363358in}}%
\pgfpathlineto{\pgfqpoint{2.275068in}{1.363856in}}%
\pgfpathlineto{\pgfqpoint{2.272022in}{1.386856in}}%
\pgfpathlineto{\pgfqpoint{2.299946in}{1.390294in}}%
\pgfpathlineto{\pgfqpoint{2.303185in}{1.367262in}}%
\pgfpathlineto{\pgfqpoint{2.320460in}{1.370056in}}%
\pgfpathlineto{\pgfqpoint{2.323909in}{1.346635in}}%
\pgfpathlineto{\pgfqpoint{2.292953in}{1.341716in}}%
\pgfpathclose%
\pgfusepath{fill}%
\end{pgfscope}%
\begin{pgfscope}%
\pgfpathrectangle{\pgfqpoint{0.100000in}{0.100000in}}{\pgfqpoint{3.007045in}{1.925000in}}%
\pgfusepath{clip}%
\pgfsetbuttcap%
\pgfsetmiterjoin%
\definecolor{currentfill}{rgb}{0.516863,0.735748,0.860192}%
\pgfsetfillcolor{currentfill}%
\pgfsetlinewidth{0.000000pt}%
\definecolor{currentstroke}{rgb}{0.000000,0.000000,0.000000}%
\pgfsetstrokecolor{currentstroke}%
\pgfsetstrokeopacity{0.000000}%
\pgfsetdash{}{0pt}%
\pgfpathmoveto{\pgfqpoint{2.741068in}{1.332531in}}%
\pgfpathlineto{\pgfqpoint{2.720167in}{1.343542in}}%
\pgfpathlineto{\pgfqpoint{2.725895in}{1.348689in}}%
\pgfpathlineto{\pgfqpoint{2.711768in}{1.357635in}}%
\pgfpathlineto{\pgfqpoint{2.723365in}{1.366527in}}%
\pgfpathlineto{\pgfqpoint{2.720526in}{1.370514in}}%
\pgfpathlineto{\pgfqpoint{2.723613in}{1.374911in}}%
\pgfpathlineto{\pgfqpoint{2.727132in}{1.374300in}}%
\pgfpathlineto{\pgfqpoint{2.735363in}{1.367506in}}%
\pgfpathlineto{\pgfqpoint{2.732190in}{1.364248in}}%
\pgfpathlineto{\pgfqpoint{2.738618in}{1.356791in}}%
\pgfpathlineto{\pgfqpoint{2.746998in}{1.361954in}}%
\pgfpathlineto{\pgfqpoint{2.753651in}{1.370464in}}%
\pgfpathlineto{\pgfqpoint{2.758253in}{1.365099in}}%
\pgfpathlineto{\pgfqpoint{2.756767in}{1.359194in}}%
\pgfpathlineto{\pgfqpoint{2.753885in}{1.357653in}}%
\pgfpathlineto{\pgfqpoint{2.755066in}{1.346497in}}%
\pgfpathlineto{\pgfqpoint{2.746311in}{1.336376in}}%
\pgfpathlineto{\pgfqpoint{2.741068in}{1.332531in}}%
\pgfpathclose%
\pgfusepath{fill}%
\end{pgfscope}%
\begin{pgfscope}%
\pgfpathrectangle{\pgfqpoint{0.100000in}{0.100000in}}{\pgfqpoint{3.007045in}{1.925000in}}%
\pgfusepath{clip}%
\pgfsetbuttcap%
\pgfsetmiterjoin%
\definecolor{currentfill}{rgb}{0.447843,0.697855,0.845306}%
\pgfsetfillcolor{currentfill}%
\pgfsetlinewidth{0.000000pt}%
\definecolor{currentstroke}{rgb}{0.000000,0.000000,0.000000}%
\pgfsetstrokecolor{currentstroke}%
\pgfsetstrokeopacity{0.000000}%
\pgfsetdash{}{0pt}%
\pgfpathmoveto{\pgfqpoint{1.538952in}{0.605702in}}%
\pgfpathlineto{\pgfqpoint{1.546174in}{0.604074in}}%
\pgfpathlineto{\pgfqpoint{1.544968in}{0.575956in}}%
\pgfpathlineto{\pgfqpoint{1.525500in}{0.576932in}}%
\pgfpathlineto{\pgfqpoint{1.515156in}{0.577420in}}%
\pgfpathlineto{\pgfqpoint{1.516589in}{0.606731in}}%
\pgfpathlineto{\pgfqpoint{1.538952in}{0.605702in}}%
\pgfpathclose%
\pgfusepath{fill}%
\end{pgfscope}%
\begin{pgfscope}%
\pgfpathrectangle{\pgfqpoint{0.100000in}{0.100000in}}{\pgfqpoint{3.007045in}{1.925000in}}%
\pgfusepath{clip}%
\pgfsetbuttcap%
\pgfsetmiterjoin%
\definecolor{currentfill}{rgb}{0.447843,0.697855,0.845306}%
\pgfsetfillcolor{currentfill}%
\pgfsetlinewidth{0.000000pt}%
\definecolor{currentstroke}{rgb}{0.000000,0.000000,0.000000}%
\pgfsetstrokecolor{currentstroke}%
\pgfsetstrokeopacity{0.000000}%
\pgfsetdash{}{0pt}%
\pgfpathmoveto{\pgfqpoint{1.880439in}{1.202637in}}%
\pgfpathlineto{\pgfqpoint{1.854945in}{1.201835in}}%
\pgfpathlineto{\pgfqpoint{1.853670in}{1.237862in}}%
\pgfpathlineto{\pgfqpoint{1.886676in}{1.239179in}}%
\pgfpathlineto{\pgfqpoint{1.889157in}{1.227613in}}%
\pgfpathlineto{\pgfqpoint{1.888599in}{1.223016in}}%
\pgfpathlineto{\pgfqpoint{1.879993in}{1.222750in}}%
\pgfpathlineto{\pgfqpoint{1.880439in}{1.202637in}}%
\pgfpathclose%
\pgfusepath{fill}%
\end{pgfscope}%
\begin{pgfscope}%
\pgfpathrectangle{\pgfqpoint{0.100000in}{0.100000in}}{\pgfqpoint{3.007045in}{1.925000in}}%
\pgfusepath{clip}%
\pgfsetbuttcap%
\pgfsetmiterjoin%
\definecolor{currentfill}{rgb}{0.657132,0.808105,0.895486}%
\pgfsetfillcolor{currentfill}%
\pgfsetlinewidth{0.000000pt}%
\definecolor{currentstroke}{rgb}{0.000000,0.000000,0.000000}%
\pgfsetstrokecolor{currentstroke}%
\pgfsetstrokeopacity{0.000000}%
\pgfsetdash{}{0pt}%
\pgfpathmoveto{\pgfqpoint{2.612770in}{1.112672in}}%
\pgfpathlineto{\pgfqpoint{2.609095in}{1.111445in}}%
\pgfpathlineto{\pgfqpoint{2.602817in}{1.120583in}}%
\pgfpathlineto{\pgfqpoint{2.598994in}{1.120903in}}%
\pgfpathlineto{\pgfqpoint{2.593692in}{1.126779in}}%
\pgfpathlineto{\pgfqpoint{2.587434in}{1.124379in}}%
\pgfpathlineto{\pgfqpoint{2.584797in}{1.114957in}}%
\pgfpathlineto{\pgfqpoint{2.579736in}{1.109975in}}%
\pgfpathlineto{\pgfqpoint{2.578351in}{1.105297in}}%
\pgfpathlineto{\pgfqpoint{2.567398in}{1.112031in}}%
\pgfpathlineto{\pgfqpoint{2.566770in}{1.118700in}}%
\pgfpathlineto{\pgfqpoint{2.559027in}{1.119923in}}%
\pgfpathlineto{\pgfqpoint{2.554928in}{1.112611in}}%
\pgfpathlineto{\pgfqpoint{2.544389in}{1.103816in}}%
\pgfpathlineto{\pgfqpoint{2.535565in}{1.108737in}}%
\pgfpathlineto{\pgfqpoint{2.538295in}{1.117062in}}%
\pgfpathlineto{\pgfqpoint{2.547549in}{1.130953in}}%
\pgfpathlineto{\pgfqpoint{2.548827in}{1.135761in}}%
\pgfpathlineto{\pgfqpoint{2.549531in}{1.143698in}}%
\pgfpathlineto{\pgfqpoint{2.555307in}{1.150523in}}%
\pgfpathlineto{\pgfqpoint{2.553841in}{1.152723in}}%
\pgfpathlineto{\pgfqpoint{2.558583in}{1.161971in}}%
\pgfpathlineto{\pgfqpoint{2.558875in}{1.172717in}}%
\pgfpathlineto{\pgfqpoint{2.565022in}{1.171056in}}%
\pgfpathlineto{\pgfqpoint{2.569128in}{1.165501in}}%
\pgfpathlineto{\pgfqpoint{2.577995in}{1.164006in}}%
\pgfpathlineto{\pgfqpoint{2.581715in}{1.169068in}}%
\pgfpathlineto{\pgfqpoint{2.609006in}{1.155780in}}%
\pgfpathlineto{\pgfqpoint{2.608923in}{1.148127in}}%
\pgfpathlineto{\pgfqpoint{2.606282in}{1.144642in}}%
\pgfpathlineto{\pgfqpoint{2.616773in}{1.129743in}}%
\pgfpathlineto{\pgfqpoint{2.617784in}{1.122931in}}%
\pgfpathlineto{\pgfqpoint{2.613630in}{1.118818in}}%
\pgfpathlineto{\pgfqpoint{2.612770in}{1.112672in}}%
\pgfpathclose%
\pgfusepath{fill}%
\end{pgfscope}%
\begin{pgfscope}%
\pgfpathrectangle{\pgfqpoint{0.100000in}{0.100000in}}{\pgfqpoint{3.007045in}{1.925000in}}%
\pgfusepath{clip}%
\pgfsetbuttcap%
\pgfsetmiterjoin%
\definecolor{currentfill}{rgb}{0.447843,0.697855,0.845306}%
\pgfsetfillcolor{currentfill}%
\pgfsetlinewidth{0.000000pt}%
\definecolor{currentstroke}{rgb}{0.000000,0.000000,0.000000}%
\pgfsetstrokecolor{currentstroke}%
\pgfsetstrokeopacity{0.000000}%
\pgfsetdash{}{0pt}%
\pgfpathmoveto{\pgfqpoint{1.525500in}{0.576932in}}%
\pgfpathlineto{\pgfqpoint{1.524943in}{0.562745in}}%
\pgfpathlineto{\pgfqpoint{1.498977in}{0.564304in}}%
\pgfpathlineto{\pgfqpoint{1.478167in}{0.565413in}}%
\pgfpathlineto{\pgfqpoint{1.479753in}{0.593339in}}%
\pgfpathlineto{\pgfqpoint{1.481225in}{0.618416in}}%
\pgfpathlineto{\pgfqpoint{1.510333in}{0.616812in}}%
\pgfpathlineto{\pgfqpoint{1.509816in}{0.607078in}}%
\pgfpathlineto{\pgfqpoint{1.516589in}{0.606731in}}%
\pgfpathlineto{\pgfqpoint{1.515156in}{0.577420in}}%
\pgfpathlineto{\pgfqpoint{1.525500in}{0.576932in}}%
\pgfpathclose%
\pgfusepath{fill}%
\end{pgfscope}%
\begin{pgfscope}%
\pgfpathrectangle{\pgfqpoint{0.100000in}{0.100000in}}{\pgfqpoint{3.007045in}{1.925000in}}%
\pgfusepath{clip}%
\pgfsetbuttcap%
\pgfsetmiterjoin%
\definecolor{currentfill}{rgb}{0.721107,0.835294,0.917878}%
\pgfsetfillcolor{currentfill}%
\pgfsetlinewidth{0.000000pt}%
\definecolor{currentstroke}{rgb}{0.000000,0.000000,0.000000}%
\pgfsetstrokecolor{currentstroke}%
\pgfsetstrokeopacity{0.000000}%
\pgfsetdash{}{0pt}%
\pgfpathmoveto{\pgfqpoint{2.671279in}{0.875019in}}%
\pgfpathlineto{\pgfqpoint{2.637551in}{0.899069in}}%
\pgfpathlineto{\pgfqpoint{2.641884in}{0.909457in}}%
\pgfpathlineto{\pgfqpoint{2.647369in}{0.914791in}}%
\pgfpathlineto{\pgfqpoint{2.647207in}{0.927059in}}%
\pgfpathlineto{\pgfqpoint{2.640458in}{0.935526in}}%
\pgfpathlineto{\pgfqpoint{2.645656in}{0.937482in}}%
\pgfpathlineto{\pgfqpoint{2.661874in}{0.940850in}}%
\pgfpathlineto{\pgfqpoint{2.668672in}{0.934954in}}%
\pgfpathlineto{\pgfqpoint{2.678361in}{0.924174in}}%
\pgfpathlineto{\pgfqpoint{2.683843in}{0.935913in}}%
\pgfpathlineto{\pgfqpoint{2.706932in}{0.940219in}}%
\pgfpathlineto{\pgfqpoint{2.714846in}{0.926915in}}%
\pgfpathlineto{\pgfqpoint{2.719169in}{0.923665in}}%
\pgfpathlineto{\pgfqpoint{2.709222in}{0.910276in}}%
\pgfpathlineto{\pgfqpoint{2.705650in}{0.901507in}}%
\pgfpathlineto{\pgfqpoint{2.702584in}{0.882702in}}%
\pgfpathlineto{\pgfqpoint{2.691067in}{0.882886in}}%
\pgfpathlineto{\pgfqpoint{2.680626in}{0.880256in}}%
\pgfpathlineto{\pgfqpoint{2.671279in}{0.875019in}}%
\pgfpathclose%
\pgfusepath{fill}%
\end{pgfscope}%
\begin{pgfscope}%
\pgfpathrectangle{\pgfqpoint{0.100000in}{0.100000in}}{\pgfqpoint{3.007045in}{1.925000in}}%
\pgfusepath{clip}%
\pgfsetbuttcap%
\pgfsetmiterjoin%
\definecolor{currentfill}{rgb}{0.541961,0.749527,0.865606}%
\pgfsetfillcolor{currentfill}%
\pgfsetlinewidth{0.000000pt}%
\definecolor{currentstroke}{rgb}{0.000000,0.000000,0.000000}%
\pgfsetstrokecolor{currentstroke}%
\pgfsetstrokeopacity{0.000000}%
\pgfsetdash{}{0pt}%
\pgfpathmoveto{\pgfqpoint{0.540209in}{1.259221in}}%
\pgfpathlineto{\pgfqpoint{0.533766in}{1.269179in}}%
\pgfpathlineto{\pgfqpoint{0.536090in}{1.281820in}}%
\pgfpathlineto{\pgfqpoint{0.533368in}{1.282853in}}%
\pgfpathlineto{\pgfqpoint{0.536518in}{1.297107in}}%
\pgfpathlineto{\pgfqpoint{0.542583in}{1.297900in}}%
\pgfpathlineto{\pgfqpoint{0.544151in}{1.303815in}}%
\pgfpathlineto{\pgfqpoint{0.522474in}{1.309901in}}%
\pgfpathlineto{\pgfqpoint{0.522542in}{1.311873in}}%
\pgfpathlineto{\pgfqpoint{0.510601in}{1.315136in}}%
\pgfpathlineto{\pgfqpoint{0.516565in}{1.336204in}}%
\pgfpathlineto{\pgfqpoint{0.521615in}{1.353758in}}%
\pgfpathlineto{\pgfqpoint{0.535789in}{1.402977in}}%
\pgfpathlineto{\pgfqpoint{0.547844in}{1.446415in}}%
\pgfpathlineto{\pgfqpoint{0.562393in}{1.497838in}}%
\pgfpathlineto{\pgfqpoint{0.592563in}{1.489427in}}%
\pgfpathlineto{\pgfqpoint{0.594261in}{1.488948in}}%
\pgfpathlineto{\pgfqpoint{0.584193in}{1.452083in}}%
\pgfpathlineto{\pgfqpoint{0.575970in}{1.423296in}}%
\pgfpathlineto{\pgfqpoint{0.576936in}{1.423028in}}%
\pgfpathlineto{\pgfqpoint{0.569070in}{1.394980in}}%
\pgfpathlineto{\pgfqpoint{0.567831in}{1.395284in}}%
\pgfpathlineto{\pgfqpoint{0.558741in}{1.362309in}}%
\pgfpathlineto{\pgfqpoint{0.594471in}{1.352638in}}%
\pgfpathlineto{\pgfqpoint{0.646486in}{1.339327in}}%
\pgfpathlineto{\pgfqpoint{0.648824in}{1.333373in}}%
\pgfpathlineto{\pgfqpoint{0.644679in}{1.322619in}}%
\pgfpathlineto{\pgfqpoint{0.644313in}{1.314171in}}%
\pgfpathlineto{\pgfqpoint{0.641368in}{1.309018in}}%
\pgfpathlineto{\pgfqpoint{0.632081in}{1.306373in}}%
\pgfpathlineto{\pgfqpoint{0.625081in}{1.300716in}}%
\pgfpathlineto{\pgfqpoint{0.625302in}{1.292738in}}%
\pgfpathlineto{\pgfqpoint{0.621234in}{1.289668in}}%
\pgfpathlineto{\pgfqpoint{0.620230in}{1.284320in}}%
\pgfpathlineto{\pgfqpoint{0.615468in}{1.284198in}}%
\pgfpathlineto{\pgfqpoint{0.562826in}{1.298061in}}%
\pgfpathlineto{\pgfqpoint{0.556383in}{1.290907in}}%
\pgfpathlineto{\pgfqpoint{0.554814in}{1.285052in}}%
\pgfpathlineto{\pgfqpoint{0.560395in}{1.283604in}}%
\pgfpathlineto{\pgfqpoint{0.552667in}{1.255842in}}%
\pgfpathlineto{\pgfqpoint{0.540209in}{1.259221in}}%
\pgfpathclose%
\pgfusepath{fill}%
\end{pgfscope}%
\begin{pgfscope}%
\pgfpathrectangle{\pgfqpoint{0.100000in}{0.100000in}}{\pgfqpoint{3.007045in}{1.925000in}}%
\pgfusepath{clip}%
\pgfsetbuttcap%
\pgfsetmiterjoin%
\definecolor{currentfill}{rgb}{0.290980,0.594510,0.789020}%
\pgfsetfillcolor{currentfill}%
\pgfsetlinewidth{0.000000pt}%
\definecolor{currentstroke}{rgb}{0.000000,0.000000,0.000000}%
\pgfsetstrokecolor{currentstroke}%
\pgfsetstrokeopacity{0.000000}%
\pgfsetdash{}{0pt}%
\pgfpathmoveto{\pgfqpoint{1.591111in}{1.462290in}}%
\pgfpathlineto{\pgfqpoint{1.585428in}{1.462527in}}%
\pgfpathlineto{\pgfqpoint{1.586215in}{1.479799in}}%
\pgfpathlineto{\pgfqpoint{1.596903in}{1.479359in}}%
\pgfpathlineto{\pgfqpoint{1.597846in}{1.508282in}}%
\pgfpathlineto{\pgfqpoint{1.632109in}{1.506861in}}%
\pgfpathlineto{\pgfqpoint{1.637825in}{1.506641in}}%
\pgfpathlineto{\pgfqpoint{1.637786in}{1.500890in}}%
\pgfpathlineto{\pgfqpoint{1.637041in}{1.477807in}}%
\pgfpathlineto{\pgfqpoint{1.614383in}{1.478679in}}%
\pgfpathlineto{\pgfqpoint{1.613769in}{1.461576in}}%
\pgfpathlineto{\pgfqpoint{1.591111in}{1.462290in}}%
\pgfpathclose%
\pgfusepath{fill}%
\end{pgfscope}%
\begin{pgfscope}%
\pgfpathrectangle{\pgfqpoint{0.100000in}{0.100000in}}{\pgfqpoint{3.007045in}{1.925000in}}%
\pgfusepath{clip}%
\pgfsetbuttcap%
\pgfsetmiterjoin%
\definecolor{currentfill}{rgb}{0.730950,0.839477,0.921323}%
\pgfsetfillcolor{currentfill}%
\pgfsetlinewidth{0.000000pt}%
\definecolor{currentstroke}{rgb}{0.000000,0.000000,0.000000}%
\pgfsetstrokecolor{currentstroke}%
\pgfsetstrokeopacity{0.000000}%
\pgfsetdash{}{0pt}%
\pgfpathmoveto{\pgfqpoint{2.735476in}{1.140922in}}%
\pgfpathlineto{\pgfqpoint{2.729649in}{1.140417in}}%
\pgfpathlineto{\pgfqpoint{2.725113in}{1.146693in}}%
\pgfpathlineto{\pgfqpoint{2.720436in}{1.148513in}}%
\pgfpathlineto{\pgfqpoint{2.716860in}{1.144435in}}%
\pgfpathlineto{\pgfqpoint{2.713398in}{1.148167in}}%
\pgfpathlineto{\pgfqpoint{2.706818in}{1.146620in}}%
\pgfpathlineto{\pgfqpoint{2.701679in}{1.149182in}}%
\pgfpathlineto{\pgfqpoint{2.697515in}{1.156406in}}%
\pgfpathlineto{\pgfqpoint{2.690990in}{1.162746in}}%
\pgfpathlineto{\pgfqpoint{2.695805in}{1.170685in}}%
\pgfpathlineto{\pgfqpoint{2.697143in}{1.177380in}}%
\pgfpathlineto{\pgfqpoint{2.704729in}{1.171795in}}%
\pgfpathlineto{\pgfqpoint{2.718443in}{1.173001in}}%
\pgfpathlineto{\pgfqpoint{2.728305in}{1.164174in}}%
\pgfpathlineto{\pgfqpoint{2.740946in}{1.160176in}}%
\pgfpathlineto{\pgfqpoint{2.739016in}{1.152520in}}%
\pgfpathlineto{\pgfqpoint{2.740280in}{1.146999in}}%
\pgfpathlineto{\pgfqpoint{2.738357in}{1.140814in}}%
\pgfpathlineto{\pgfqpoint{2.735476in}{1.140922in}}%
\pgfpathclose%
\pgfusepath{fill}%
\end{pgfscope}%
\begin{pgfscope}%
\pgfpathrectangle{\pgfqpoint{0.100000in}{0.100000in}}{\pgfqpoint{3.007045in}{1.925000in}}%
\pgfusepath{clip}%
\pgfsetbuttcap%
\pgfsetmiterjoin%
\definecolor{currentfill}{rgb}{0.435294,0.690965,0.842599}%
\pgfsetfillcolor{currentfill}%
\pgfsetlinewidth{0.000000pt}%
\definecolor{currentstroke}{rgb}{0.000000,0.000000,0.000000}%
\pgfsetstrokecolor{currentstroke}%
\pgfsetstrokeopacity{0.000000}%
\pgfsetdash{}{0pt}%
\pgfpathmoveto{\pgfqpoint{1.631910in}{1.063521in}}%
\pgfpathlineto{\pgfqpoint{1.631737in}{1.051813in}}%
\pgfpathlineto{\pgfqpoint{1.591838in}{1.053004in}}%
\pgfpathlineto{\pgfqpoint{1.591676in}{1.059068in}}%
\pgfpathlineto{\pgfqpoint{1.593165in}{1.116517in}}%
\pgfpathlineto{\pgfqpoint{1.593668in}{1.127954in}}%
\pgfpathlineto{\pgfqpoint{1.622153in}{1.126974in}}%
\pgfpathlineto{\pgfqpoint{1.621825in}{1.109765in}}%
\pgfpathlineto{\pgfqpoint{1.621015in}{1.081045in}}%
\pgfpathlineto{\pgfqpoint{1.632419in}{1.080689in}}%
\pgfpathlineto{\pgfqpoint{1.631910in}{1.063521in}}%
\pgfpathclose%
\pgfusepath{fill}%
\end{pgfscope}%
\begin{pgfscope}%
\pgfpathrectangle{\pgfqpoint{0.100000in}{0.100000in}}{\pgfqpoint{3.007045in}{1.925000in}}%
\pgfusepath{clip}%
\pgfsetbuttcap%
\pgfsetmiterjoin%
\definecolor{currentfill}{rgb}{0.280892,0.587620,0.785083}%
\pgfsetfillcolor{currentfill}%
\pgfsetlinewidth{0.000000pt}%
\definecolor{currentstroke}{rgb}{0.000000,0.000000,0.000000}%
\pgfsetstrokecolor{currentstroke}%
\pgfsetstrokeopacity{0.000000}%
\pgfsetdash{}{0pt}%
\pgfpathmoveto{\pgfqpoint{2.397002in}{0.647345in}}%
\pgfpathlineto{\pgfqpoint{2.375927in}{0.644931in}}%
\pgfpathlineto{\pgfqpoint{2.366514in}{0.643850in}}%
\pgfpathlineto{\pgfqpoint{2.365478in}{0.655570in}}%
\pgfpathlineto{\pgfqpoint{2.359552in}{0.655091in}}%
\pgfpathlineto{\pgfqpoint{2.358460in}{0.666668in}}%
\pgfpathlineto{\pgfqpoint{2.366771in}{0.667222in}}%
\pgfpathlineto{\pgfqpoint{2.370955in}{0.669671in}}%
\pgfpathlineto{\pgfqpoint{2.367869in}{0.675418in}}%
\pgfpathlineto{\pgfqpoint{2.367559in}{0.680287in}}%
\pgfpathlineto{\pgfqpoint{2.362322in}{0.683671in}}%
\pgfpathlineto{\pgfqpoint{2.357840in}{0.689197in}}%
\pgfpathlineto{\pgfqpoint{2.356903in}{0.698990in}}%
\pgfpathlineto{\pgfqpoint{2.394856in}{0.702917in}}%
\pgfpathlineto{\pgfqpoint{2.395249in}{0.698742in}}%
\pgfpathlineto{\pgfqpoint{2.402438in}{0.696626in}}%
\pgfpathlineto{\pgfqpoint{2.404240in}{0.684768in}}%
\pgfpathlineto{\pgfqpoint{2.408654in}{0.685257in}}%
\pgfpathlineto{\pgfqpoint{2.412962in}{0.682194in}}%
\pgfpathlineto{\pgfqpoint{2.414666in}{0.666517in}}%
\pgfpathlineto{\pgfqpoint{2.395188in}{0.664380in}}%
\pgfpathlineto{\pgfqpoint{2.397002in}{0.647345in}}%
\pgfpathclose%
\pgfusepath{fill}%
\end{pgfscope}%
\begin{pgfscope}%
\pgfpathrectangle{\pgfqpoint{0.100000in}{0.100000in}}{\pgfqpoint{3.007045in}{1.925000in}}%
\pgfusepath{clip}%
\pgfsetbuttcap%
\pgfsetmiterjoin%
\definecolor{currentfill}{rgb}{0.686659,0.820654,0.905821}%
\pgfsetfillcolor{currentfill}%
\pgfsetlinewidth{0.000000pt}%
\definecolor{currentstroke}{rgb}{0.000000,0.000000,0.000000}%
\pgfsetstrokecolor{currentstroke}%
\pgfsetstrokeopacity{0.000000}%
\pgfsetdash{}{0pt}%
\pgfpathmoveto{\pgfqpoint{2.209634in}{1.589574in}}%
\pgfpathlineto{\pgfqpoint{2.206269in}{1.590978in}}%
\pgfpathlineto{\pgfqpoint{2.208079in}{1.601957in}}%
\pgfpathlineto{\pgfqpoint{2.211439in}{1.600059in}}%
\pgfpathlineto{\pgfqpoint{2.212615in}{1.592362in}}%
\pgfpathlineto{\pgfqpoint{2.209634in}{1.589574in}}%
\pgfpathclose%
\pgfusepath{fill}%
\end{pgfscope}%
\begin{pgfscope}%
\pgfpathrectangle{\pgfqpoint{0.100000in}{0.100000in}}{\pgfqpoint{3.007045in}{1.925000in}}%
\pgfusepath{clip}%
\pgfsetbuttcap%
\pgfsetmiterjoin%
\definecolor{currentfill}{rgb}{0.686659,0.820654,0.905821}%
\pgfsetfillcolor{currentfill}%
\pgfsetlinewidth{0.000000pt}%
\definecolor{currentstroke}{rgb}{0.000000,0.000000,0.000000}%
\pgfsetstrokecolor{currentstroke}%
\pgfsetstrokeopacity{0.000000}%
\pgfsetdash{}{0pt}%
\pgfpathmoveto{\pgfqpoint{2.271744in}{1.600141in}}%
\pgfpathlineto{\pgfqpoint{2.269589in}{1.599880in}}%
\pgfpathlineto{\pgfqpoint{2.273175in}{1.571785in}}%
\pgfpathlineto{\pgfqpoint{2.250552in}{1.569368in}}%
\pgfpathlineto{\pgfqpoint{2.251176in}{1.563619in}}%
\pgfpathlineto{\pgfqpoint{2.245496in}{1.563087in}}%
\pgfpathlineto{\pgfqpoint{2.228428in}{1.561303in}}%
\pgfpathlineto{\pgfqpoint{2.227849in}{1.567100in}}%
\pgfpathlineto{\pgfqpoint{2.220169in}{1.566322in}}%
\pgfpathlineto{\pgfqpoint{2.223475in}{1.574310in}}%
\pgfpathlineto{\pgfqpoint{2.230581in}{1.578556in}}%
\pgfpathlineto{\pgfqpoint{2.234669in}{1.583276in}}%
\pgfpathlineto{\pgfqpoint{2.231732in}{1.586161in}}%
\pgfpathlineto{\pgfqpoint{2.230119in}{1.592502in}}%
\pgfpathlineto{\pgfqpoint{2.237120in}{1.602372in}}%
\pgfpathlineto{\pgfqpoint{2.243327in}{1.604920in}}%
\pgfpathlineto{\pgfqpoint{2.244332in}{1.607881in}}%
\pgfpathlineto{\pgfqpoint{2.259869in}{1.600439in}}%
\pgfpathlineto{\pgfqpoint{2.265685in}{1.601992in}}%
\pgfpathlineto{\pgfqpoint{2.271744in}{1.600141in}}%
\pgfpathclose%
\pgfusepath{fill}%
\end{pgfscope}%
\begin{pgfscope}%
\pgfpathrectangle{\pgfqpoint{0.100000in}{0.100000in}}{\pgfqpoint{3.007045in}{1.925000in}}%
\pgfusepath{clip}%
\pgfsetbuttcap%
\pgfsetmiterjoin%
\definecolor{currentfill}{rgb}{0.652211,0.806013,0.893764}%
\pgfsetfillcolor{currentfill}%
\pgfsetlinewidth{0.000000pt}%
\definecolor{currentstroke}{rgb}{0.000000,0.000000,0.000000}%
\pgfsetstrokecolor{currentstroke}%
\pgfsetstrokeopacity{0.000000}%
\pgfsetdash{}{0pt}%
\pgfpathmoveto{\pgfqpoint{2.415134in}{0.621502in}}%
\pgfpathlineto{\pgfqpoint{2.422704in}{0.621971in}}%
\pgfpathlineto{\pgfqpoint{2.423706in}{0.617445in}}%
\pgfpathlineto{\pgfqpoint{2.416778in}{0.611640in}}%
\pgfpathlineto{\pgfqpoint{2.418230in}{0.606839in}}%
\pgfpathlineto{\pgfqpoint{2.413904in}{0.601638in}}%
\pgfpathlineto{\pgfqpoint{2.415159in}{0.598741in}}%
\pgfpathlineto{\pgfqpoint{2.408711in}{0.592974in}}%
\pgfpathlineto{\pgfqpoint{2.405641in}{0.582074in}}%
\pgfpathlineto{\pgfqpoint{2.395790in}{0.579657in}}%
\pgfpathlineto{\pgfqpoint{2.389487in}{0.580324in}}%
\pgfpathlineto{\pgfqpoint{2.384731in}{0.575233in}}%
\pgfpathlineto{\pgfqpoint{2.386694in}{0.571228in}}%
\pgfpathlineto{\pgfqpoint{2.374564in}{0.573201in}}%
\pgfpathlineto{\pgfqpoint{2.347979in}{0.570392in}}%
\pgfpathlineto{\pgfqpoint{2.340428in}{0.571556in}}%
\pgfpathlineto{\pgfqpoint{2.339480in}{0.580648in}}%
\pgfpathlineto{\pgfqpoint{2.324742in}{0.580445in}}%
\pgfpathlineto{\pgfqpoint{2.322486in}{0.604682in}}%
\pgfpathlineto{\pgfqpoint{2.316671in}{0.604111in}}%
\pgfpathlineto{\pgfqpoint{2.317482in}{0.595459in}}%
\pgfpathlineto{\pgfqpoint{2.296561in}{0.593554in}}%
\pgfpathlineto{\pgfqpoint{2.288959in}{0.589362in}}%
\pgfpathlineto{\pgfqpoint{2.295880in}{0.596792in}}%
\pgfpathlineto{\pgfqpoint{2.291737in}{0.606164in}}%
\pgfpathlineto{\pgfqpoint{2.294588in}{0.610608in}}%
\pgfpathlineto{\pgfqpoint{2.284322in}{0.609983in}}%
\pgfpathlineto{\pgfqpoint{2.282576in}{0.629115in}}%
\pgfpathlineto{\pgfqpoint{2.341073in}{0.635732in}}%
\pgfpathlineto{\pgfqpoint{2.345288in}{0.628691in}}%
\pgfpathlineto{\pgfqpoint{2.346074in}{0.624253in}}%
\pgfpathlineto{\pgfqpoint{2.351014in}{0.617483in}}%
\pgfpathlineto{\pgfqpoint{2.378692in}{0.619214in}}%
\pgfpathlineto{\pgfqpoint{2.415134in}{0.621502in}}%
\pgfpathclose%
\pgfusepath{fill}%
\end{pgfscope}%
\begin{pgfscope}%
\pgfpathrectangle{\pgfqpoint{0.100000in}{0.100000in}}{\pgfqpoint{3.007045in}{1.925000in}}%
\pgfusepath{clip}%
\pgfsetbuttcap%
\pgfsetmiterjoin%
\definecolor{currentfill}{rgb}{0.114802,0.424437,0.695194}%
\pgfsetfillcolor{currentfill}%
\pgfsetlinewidth{0.000000pt}%
\definecolor{currentstroke}{rgb}{0.000000,0.000000,0.000000}%
\pgfsetstrokecolor{currentstroke}%
\pgfsetstrokeopacity{0.000000}%
\pgfsetdash{}{0pt}%
\pgfpathmoveto{\pgfqpoint{1.448770in}{1.649792in}}%
\pgfpathlineto{\pgfqpoint{1.449237in}{1.655533in}}%
\pgfpathlineto{\pgfqpoint{1.466420in}{1.654245in}}%
\pgfpathlineto{\pgfqpoint{1.465987in}{1.648477in}}%
\pgfpathlineto{\pgfqpoint{1.485073in}{1.647065in}}%
\pgfpathlineto{\pgfqpoint{1.483880in}{1.629904in}}%
\pgfpathlineto{\pgfqpoint{1.495332in}{1.629121in}}%
\pgfpathlineto{\pgfqpoint{1.495994in}{1.623275in}}%
\pgfpathlineto{\pgfqpoint{1.482806in}{1.614036in}}%
\pgfpathlineto{\pgfqpoint{1.480364in}{1.610710in}}%
\pgfpathlineto{\pgfqpoint{1.470594in}{1.606845in}}%
\pgfpathlineto{\pgfqpoint{1.459052in}{1.608999in}}%
\pgfpathlineto{\pgfqpoint{1.455103in}{1.611872in}}%
\pgfpathlineto{\pgfqpoint{1.450280in}{1.611196in}}%
\pgfpathlineto{\pgfqpoint{1.451489in}{1.626431in}}%
\pgfpathlineto{\pgfqpoint{1.449255in}{1.626605in}}%
\pgfpathlineto{\pgfqpoint{1.450846in}{1.649617in}}%
\pgfpathlineto{\pgfqpoint{1.448770in}{1.649792in}}%
\pgfpathclose%
\pgfusepath{fill}%
\end{pgfscope}%
\begin{pgfscope}%
\pgfpathrectangle{\pgfqpoint{0.100000in}{0.100000in}}{\pgfqpoint{3.007045in}{1.925000in}}%
\pgfusepath{clip}%
\pgfsetbuttcap%
\pgfsetmiterjoin%
\definecolor{currentfill}{rgb}{0.351511,0.635848,0.812641}%
\pgfsetfillcolor{currentfill}%
\pgfsetlinewidth{0.000000pt}%
\definecolor{currentstroke}{rgb}{0.000000,0.000000,0.000000}%
\pgfsetstrokecolor{currentstroke}%
\pgfsetstrokeopacity{0.000000}%
\pgfsetdash{}{0pt}%
\pgfpathmoveto{\pgfqpoint{1.408351in}{1.098572in}}%
\pgfpathlineto{\pgfqpoint{1.408374in}{1.098989in}}%
\pgfpathlineto{\pgfqpoint{1.410528in}{1.127149in}}%
\pgfpathlineto{\pgfqpoint{1.439278in}{1.125169in}}%
\pgfpathlineto{\pgfqpoint{1.480135in}{1.122427in}}%
\pgfpathlineto{\pgfqpoint{1.478511in}{1.093767in}}%
\pgfpathlineto{\pgfqpoint{1.432949in}{1.096770in}}%
\pgfpathlineto{\pgfqpoint{1.408351in}{1.098572in}}%
\pgfpathclose%
\pgfusepath{fill}%
\end{pgfscope}%
\begin{pgfscope}%
\pgfpathrectangle{\pgfqpoint{0.100000in}{0.100000in}}{\pgfqpoint{3.007045in}{1.925000in}}%
\pgfusepath{clip}%
\pgfsetbuttcap%
\pgfsetmiterjoin%
\definecolor{currentfill}{rgb}{0.296025,0.597955,0.790988}%
\pgfsetfillcolor{currentfill}%
\pgfsetlinewidth{0.000000pt}%
\definecolor{currentstroke}{rgb}{0.000000,0.000000,0.000000}%
\pgfsetstrokecolor{currentstroke}%
\pgfsetstrokeopacity{0.000000}%
\pgfsetdash{}{0pt}%
\pgfpathmoveto{\pgfqpoint{1.753660in}{1.413597in}}%
\pgfpathlineto{\pgfqpoint{1.753560in}{1.390798in}}%
\pgfpathlineto{\pgfqpoint{1.730762in}{1.390911in}}%
\pgfpathlineto{\pgfqpoint{1.697951in}{1.391327in}}%
\pgfpathlineto{\pgfqpoint{1.699908in}{1.394517in}}%
\pgfpathlineto{\pgfqpoint{1.699433in}{1.400596in}}%
\pgfpathlineto{\pgfqpoint{1.702210in}{1.402774in}}%
\pgfpathlineto{\pgfqpoint{1.701304in}{1.411915in}}%
\pgfpathlineto{\pgfqpoint{1.698770in}{1.417142in}}%
\pgfpathlineto{\pgfqpoint{1.699167in}{1.423436in}}%
\pgfpathlineto{\pgfqpoint{1.695898in}{1.425999in}}%
\pgfpathlineto{\pgfqpoint{1.695709in}{1.430437in}}%
\pgfpathlineto{\pgfqpoint{1.702699in}{1.430317in}}%
\pgfpathlineto{\pgfqpoint{1.721870in}{1.430045in}}%
\pgfpathlineto{\pgfqpoint{1.731061in}{1.429926in}}%
\pgfpathlineto{\pgfqpoint{1.730861in}{1.413920in}}%
\pgfpathlineto{\pgfqpoint{1.753660in}{1.413597in}}%
\pgfpathclose%
\pgfusepath{fill}%
\end{pgfscope}%
\begin{pgfscope}%
\pgfpathrectangle{\pgfqpoint{0.100000in}{0.100000in}}{\pgfqpoint{3.007045in}{1.925000in}}%
\pgfusepath{clip}%
\pgfsetbuttcap%
\pgfsetmiterjoin%
\definecolor{currentfill}{rgb}{0.701423,0.826928,0.910988}%
\pgfsetfillcolor{currentfill}%
\pgfsetlinewidth{0.000000pt}%
\definecolor{currentstroke}{rgb}{0.000000,0.000000,0.000000}%
\pgfsetstrokecolor{currentstroke}%
\pgfsetstrokeopacity{0.000000}%
\pgfsetdash{}{0pt}%
\pgfpathmoveto{\pgfqpoint{2.867151in}{1.544262in}}%
\pgfpathlineto{\pgfqpoint{2.869055in}{1.532209in}}%
\pgfpathlineto{\pgfqpoint{2.871705in}{1.532829in}}%
\pgfpathlineto{\pgfqpoint{2.875745in}{1.523338in}}%
\pgfpathlineto{\pgfqpoint{2.879160in}{1.518963in}}%
\pgfpathlineto{\pgfqpoint{2.862374in}{1.515277in}}%
\pgfpathlineto{\pgfqpoint{2.831656in}{1.508759in}}%
\pgfpathlineto{\pgfqpoint{2.829097in}{1.520876in}}%
\pgfpathlineto{\pgfqpoint{2.824976in}{1.520081in}}%
\pgfpathlineto{\pgfqpoint{2.822507in}{1.532326in}}%
\pgfpathlineto{\pgfqpoint{2.828599in}{1.533431in}}%
\pgfpathlineto{\pgfqpoint{2.828637in}{1.543127in}}%
\pgfpathlineto{\pgfqpoint{2.835404in}{1.542568in}}%
\pgfpathlineto{\pgfqpoint{2.846989in}{1.546225in}}%
\pgfpathlineto{\pgfqpoint{2.850891in}{1.542069in}}%
\pgfpathlineto{\pgfqpoint{2.860753in}{1.546240in}}%
\pgfpathlineto{\pgfqpoint{2.867151in}{1.544262in}}%
\pgfpathclose%
\pgfusepath{fill}%
\end{pgfscope}%
\begin{pgfscope}%
\pgfpathrectangle{\pgfqpoint{0.100000in}{0.100000in}}{\pgfqpoint{3.007045in}{1.925000in}}%
\pgfusepath{clip}%
\pgfsetbuttcap%
\pgfsetmiterjoin%
\definecolor{currentfill}{rgb}{0.529412,0.742637,0.862899}%
\pgfsetfillcolor{currentfill}%
\pgfsetlinewidth{0.000000pt}%
\definecolor{currentstroke}{rgb}{0.000000,0.000000,0.000000}%
\pgfsetstrokecolor{currentstroke}%
\pgfsetstrokeopacity{0.000000}%
\pgfsetdash{}{0pt}%
\pgfpathmoveto{\pgfqpoint{1.147775in}{0.452101in}}%
\pgfpathlineto{\pgfqpoint{1.141912in}{0.462482in}}%
\pgfpathlineto{\pgfqpoint{1.134776in}{0.466628in}}%
\pgfpathlineto{\pgfqpoint{1.136483in}{0.471729in}}%
\pgfpathlineto{\pgfqpoint{1.136838in}{0.481546in}}%
\pgfpathlineto{\pgfqpoint{1.140040in}{0.487692in}}%
\pgfpathlineto{\pgfqpoint{1.147213in}{0.482417in}}%
\pgfpathlineto{\pgfqpoint{1.157196in}{0.484131in}}%
\pgfpathlineto{\pgfqpoint{1.159892in}{0.482747in}}%
\pgfpathlineto{\pgfqpoint{1.158835in}{0.470012in}}%
\pgfpathlineto{\pgfqpoint{1.155158in}{0.464696in}}%
\pgfpathlineto{\pgfqpoint{1.158256in}{0.454687in}}%
\pgfpathlineto{\pgfqpoint{1.157522in}{0.448171in}}%
\pgfpathlineto{\pgfqpoint{1.147775in}{0.452101in}}%
\pgfpathclose%
\pgfusepath{fill}%
\end{pgfscope}%
\begin{pgfscope}%
\pgfpathrectangle{\pgfqpoint{0.100000in}{0.100000in}}{\pgfqpoint{3.007045in}{1.925000in}}%
\pgfusepath{clip}%
\pgfsetbuttcap%
\pgfsetmiterjoin%
\definecolor{currentfill}{rgb}{0.429020,0.687520,0.841246}%
\pgfsetfillcolor{currentfill}%
\pgfsetlinewidth{0.000000pt}%
\definecolor{currentstroke}{rgb}{0.000000,0.000000,0.000000}%
\pgfsetstrokecolor{currentstroke}%
\pgfsetstrokeopacity{0.000000}%
\pgfsetdash{}{0pt}%
\pgfpathmoveto{\pgfqpoint{0.495869in}{0.921526in}}%
\pgfpathlineto{\pgfqpoint{0.488883in}{0.925335in}}%
\pgfpathlineto{\pgfqpoint{0.489033in}{0.933786in}}%
\pgfpathlineto{\pgfqpoint{0.495364in}{0.929362in}}%
\pgfpathlineto{\pgfqpoint{0.495869in}{0.921526in}}%
\pgfpathclose%
\pgfusepath{fill}%
\end{pgfscope}%
\begin{pgfscope}%
\pgfpathrectangle{\pgfqpoint{0.100000in}{0.100000in}}{\pgfqpoint{3.007045in}{1.925000in}}%
\pgfusepath{clip}%
\pgfsetbuttcap%
\pgfsetmiterjoin%
\definecolor{currentfill}{rgb}{0.429020,0.687520,0.841246}%
\pgfsetfillcolor{currentfill}%
\pgfsetlinewidth{0.000000pt}%
\definecolor{currentstroke}{rgb}{0.000000,0.000000,0.000000}%
\pgfsetstrokecolor{currentstroke}%
\pgfsetstrokeopacity{0.000000}%
\pgfsetdash{}{0pt}%
\pgfpathmoveto{\pgfqpoint{0.536458in}{0.917204in}}%
\pgfpathlineto{\pgfqpoint{0.528897in}{0.929641in}}%
\pgfpathlineto{\pgfqpoint{0.522315in}{0.935716in}}%
\pgfpathlineto{\pgfqpoint{0.515735in}{0.945831in}}%
\pgfpathlineto{\pgfqpoint{0.511433in}{0.949293in}}%
\pgfpathlineto{\pgfqpoint{0.504379in}{0.947346in}}%
\pgfpathlineto{\pgfqpoint{0.498832in}{0.951061in}}%
\pgfpathlineto{\pgfqpoint{0.501532in}{0.957039in}}%
\pgfpathlineto{\pgfqpoint{0.499515in}{0.967768in}}%
\pgfpathlineto{\pgfqpoint{0.496988in}{0.971830in}}%
\pgfpathlineto{\pgfqpoint{0.486075in}{0.974392in}}%
\pgfpathlineto{\pgfqpoint{0.481940in}{0.973863in}}%
\pgfpathlineto{\pgfqpoint{0.463162in}{0.988331in}}%
\pgfpathlineto{\pgfqpoint{0.461865in}{0.997033in}}%
\pgfpathlineto{\pgfqpoint{0.453370in}{1.006642in}}%
\pgfpathlineto{\pgfqpoint{0.456601in}{1.011141in}}%
\pgfpathlineto{\pgfqpoint{0.464444in}{1.039777in}}%
\pgfpathlineto{\pgfqpoint{0.472706in}{1.036029in}}%
\pgfpathlineto{\pgfqpoint{0.473306in}{1.031380in}}%
\pgfpathlineto{\pgfqpoint{0.515049in}{1.020412in}}%
\pgfpathlineto{\pgfqpoint{0.557933in}{1.009610in}}%
\pgfpathlineto{\pgfqpoint{0.573744in}{1.071996in}}%
\pgfpathlineto{\pgfqpoint{0.593532in}{1.066855in}}%
\pgfpathlineto{\pgfqpoint{0.654898in}{1.051840in}}%
\pgfpathlineto{\pgfqpoint{0.672324in}{1.047770in}}%
\pgfpathlineto{\pgfqpoint{0.677176in}{1.047687in}}%
\pgfpathlineto{\pgfqpoint{0.718691in}{0.983740in}}%
\pgfpathlineto{\pgfqpoint{0.716895in}{0.976496in}}%
\pgfpathlineto{\pgfqpoint{0.719619in}{0.967635in}}%
\pgfpathlineto{\pgfqpoint{0.723182in}{0.963163in}}%
\pgfpathlineto{\pgfqpoint{0.723368in}{0.955271in}}%
\pgfpathlineto{\pgfqpoint{0.726621in}{0.944753in}}%
\pgfpathlineto{\pgfqpoint{0.735192in}{0.932928in}}%
\pgfpathlineto{\pgfqpoint{0.740331in}{0.928918in}}%
\pgfpathlineto{\pgfqpoint{0.762651in}{0.922152in}}%
\pgfpathlineto{\pgfqpoint{0.767196in}{0.926377in}}%
\pgfpathlineto{\pgfqpoint{0.778392in}{0.924882in}}%
\pgfpathlineto{\pgfqpoint{0.774157in}{0.904247in}}%
\pgfpathlineto{\pgfqpoint{0.765905in}{0.863949in}}%
\pgfpathlineto{\pgfqpoint{0.732137in}{0.870987in}}%
\pgfpathlineto{\pgfqpoint{0.733331in}{0.876625in}}%
\pgfpathlineto{\pgfqpoint{0.716592in}{0.880125in}}%
\pgfpathlineto{\pgfqpoint{0.710499in}{0.852063in}}%
\pgfpathlineto{\pgfqpoint{0.689447in}{0.856914in}}%
\pgfpathlineto{\pgfqpoint{0.687845in}{0.862115in}}%
\pgfpathlineto{\pgfqpoint{0.691629in}{0.872688in}}%
\pgfpathlineto{\pgfqpoint{0.691055in}{0.881658in}}%
\pgfpathlineto{\pgfqpoint{0.673193in}{0.887262in}}%
\pgfpathlineto{\pgfqpoint{0.618157in}{0.899792in}}%
\pgfpathlineto{\pgfqpoint{0.567402in}{0.912130in}}%
\pgfpathlineto{\pgfqpoint{0.556193in}{0.915257in}}%
\pgfpathlineto{\pgfqpoint{0.550214in}{0.920766in}}%
\pgfpathlineto{\pgfqpoint{0.543010in}{0.923557in}}%
\pgfpathlineto{\pgfqpoint{0.538502in}{0.921211in}}%
\pgfpathlineto{\pgfqpoint{0.536458in}{0.917204in}}%
\pgfpathclose%
\pgfusepath{fill}%
\end{pgfscope}%
\begin{pgfscope}%
\pgfpathrectangle{\pgfqpoint{0.100000in}{0.100000in}}{\pgfqpoint{3.007045in}{1.925000in}}%
\pgfusepath{clip}%
\pgfsetbuttcap%
\pgfsetmiterjoin%
\definecolor{currentfill}{rgb}{0.510588,0.732303,0.858839}%
\pgfsetfillcolor{currentfill}%
\pgfsetlinewidth{0.000000pt}%
\definecolor{currentstroke}{rgb}{0.000000,0.000000,0.000000}%
\pgfsetstrokecolor{currentstroke}%
\pgfsetstrokeopacity{0.000000}%
\pgfsetdash{}{0pt}%
\pgfpathmoveto{\pgfqpoint{2.241536in}{0.660624in}}%
\pgfpathlineto{\pgfqpoint{2.252934in}{0.661782in}}%
\pgfpathlineto{\pgfqpoint{2.252110in}{0.670372in}}%
\pgfpathlineto{\pgfqpoint{2.254972in}{0.670665in}}%
\pgfpathlineto{\pgfqpoint{2.253069in}{0.691091in}}%
\pgfpathlineto{\pgfqpoint{2.254880in}{0.697001in}}%
\pgfpathlineto{\pgfqpoint{2.260621in}{0.697611in}}%
\pgfpathlineto{\pgfqpoint{2.261248in}{0.691899in}}%
\pgfpathlineto{\pgfqpoint{2.278425in}{0.693743in}}%
\pgfpathlineto{\pgfqpoint{2.277794in}{0.699248in}}%
\pgfpathlineto{\pgfqpoint{2.283592in}{0.699581in}}%
\pgfpathlineto{\pgfqpoint{2.284645in}{0.694345in}}%
\pgfpathlineto{\pgfqpoint{2.289927in}{0.694879in}}%
\pgfpathlineto{\pgfqpoint{2.290454in}{0.689136in}}%
\pgfpathlineto{\pgfqpoint{2.297933in}{0.689896in}}%
\pgfpathlineto{\pgfqpoint{2.298312in}{0.683587in}}%
\pgfpathlineto{\pgfqpoint{2.294624in}{0.672055in}}%
\pgfpathlineto{\pgfqpoint{2.272250in}{0.669798in}}%
\pgfpathlineto{\pgfqpoint{2.270062in}{0.663726in}}%
\pgfpathlineto{\pgfqpoint{2.273936in}{0.628314in}}%
\pgfpathlineto{\pgfqpoint{2.245494in}{0.625658in}}%
\pgfpathlineto{\pgfqpoint{2.244783in}{0.625988in}}%
\pgfpathlineto{\pgfqpoint{2.241536in}{0.660624in}}%
\pgfpathclose%
\pgfusepath{fill}%
\end{pgfscope}%
\begin{pgfscope}%
\pgfpathrectangle{\pgfqpoint{0.100000in}{0.100000in}}{\pgfqpoint{3.007045in}{1.925000in}}%
\pgfusepath{clip}%
\pgfsetbuttcap%
\pgfsetmiterjoin%
\definecolor{currentfill}{rgb}{0.235986,0.549712,0.764706}%
\pgfsetfillcolor{currentfill}%
\pgfsetlinewidth{0.000000pt}%
\definecolor{currentstroke}{rgb}{0.000000,0.000000,0.000000}%
\pgfsetstrokecolor{currentstroke}%
\pgfsetstrokeopacity{0.000000}%
\pgfsetdash{}{0pt}%
\pgfpathmoveto{\pgfqpoint{2.033514in}{0.905154in}}%
\pgfpathlineto{\pgfqpoint{2.039932in}{0.907365in}}%
\pgfpathlineto{\pgfqpoint{2.038540in}{0.912838in}}%
\pgfpathlineto{\pgfqpoint{2.045514in}{0.912506in}}%
\pgfpathlineto{\pgfqpoint{2.051337in}{0.919577in}}%
\pgfpathlineto{\pgfqpoint{2.061506in}{0.919581in}}%
\pgfpathlineto{\pgfqpoint{2.068214in}{0.916226in}}%
\pgfpathlineto{\pgfqpoint{2.069407in}{0.904987in}}%
\pgfpathlineto{\pgfqpoint{2.085014in}{0.905571in}}%
\pgfpathlineto{\pgfqpoint{2.085939in}{0.878995in}}%
\pgfpathlineto{\pgfqpoint{2.077608in}{0.878428in}}%
\pgfpathlineto{\pgfqpoint{2.078219in}{0.868579in}}%
\pgfpathlineto{\pgfqpoint{2.081112in}{0.868758in}}%
\pgfpathlineto{\pgfqpoint{2.082243in}{0.851523in}}%
\pgfpathlineto{\pgfqpoint{2.085442in}{0.845945in}}%
\pgfpathlineto{\pgfqpoint{2.072679in}{0.849028in}}%
\pgfpathlineto{\pgfqpoint{2.033456in}{0.846814in}}%
\pgfpathlineto{\pgfqpoint{2.027867in}{0.843557in}}%
\pgfpathlineto{\pgfqpoint{2.028291in}{0.837950in}}%
\pgfpathlineto{\pgfqpoint{2.022978in}{0.837629in}}%
\pgfpathlineto{\pgfqpoint{2.013407in}{0.843839in}}%
\pgfpathlineto{\pgfqpoint{2.012143in}{0.849788in}}%
\pgfpathlineto{\pgfqpoint{2.018227in}{0.853487in}}%
\pgfpathlineto{\pgfqpoint{2.020371in}{0.864609in}}%
\pgfpathlineto{\pgfqpoint{2.029442in}{0.869945in}}%
\pgfpathlineto{\pgfqpoint{2.026701in}{0.874367in}}%
\pgfpathlineto{\pgfqpoint{2.030962in}{0.877859in}}%
\pgfpathlineto{\pgfqpoint{2.032840in}{0.884060in}}%
\pgfpathlineto{\pgfqpoint{2.036782in}{0.884180in}}%
\pgfpathlineto{\pgfqpoint{2.037839in}{0.890956in}}%
\pgfpathlineto{\pgfqpoint{2.034157in}{0.893004in}}%
\pgfpathlineto{\pgfqpoint{2.037206in}{0.901885in}}%
\pgfpathlineto{\pgfqpoint{2.033514in}{0.905154in}}%
\pgfpathclose%
\pgfusepath{fill}%
\end{pgfscope}%
\begin{pgfscope}%
\pgfpathrectangle{\pgfqpoint{0.100000in}{0.100000in}}{\pgfqpoint{3.007045in}{1.925000in}}%
\pgfusepath{clip}%
\pgfsetbuttcap%
\pgfsetmiterjoin%
\definecolor{currentfill}{rgb}{0.391865,0.663406,0.828389}%
\pgfsetfillcolor{currentfill}%
\pgfsetlinewidth{0.000000pt}%
\definecolor{currentstroke}{rgb}{0.000000,0.000000,0.000000}%
\pgfsetstrokecolor{currentstroke}%
\pgfsetstrokeopacity{0.000000}%
\pgfsetdash{}{0pt}%
\pgfpathmoveto{\pgfqpoint{2.372484in}{1.056245in}}%
\pgfpathlineto{\pgfqpoint{2.363870in}{1.060375in}}%
\pgfpathlineto{\pgfqpoint{2.358324in}{1.060460in}}%
\pgfpathlineto{\pgfqpoint{2.352145in}{1.068229in}}%
\pgfpathlineto{\pgfqpoint{2.350488in}{1.072384in}}%
\pgfpathlineto{\pgfqpoint{2.353799in}{1.080602in}}%
\pgfpathlineto{\pgfqpoint{2.358589in}{1.084899in}}%
\pgfpathlineto{\pgfqpoint{2.362396in}{1.092621in}}%
\pgfpathlineto{\pgfqpoint{2.367300in}{1.093259in}}%
\pgfpathlineto{\pgfqpoint{2.371064in}{1.101491in}}%
\pgfpathlineto{\pgfqpoint{2.370833in}{1.107548in}}%
\pgfpathlineto{\pgfqpoint{2.379152in}{1.110286in}}%
\pgfpathlineto{\pgfqpoint{2.383396in}{1.107611in}}%
\pgfpathlineto{\pgfqpoint{2.392249in}{1.108544in}}%
\pgfpathlineto{\pgfqpoint{2.396141in}{1.102157in}}%
\pgfpathlineto{\pgfqpoint{2.389841in}{1.100458in}}%
\pgfpathlineto{\pgfqpoint{2.381551in}{1.091127in}}%
\pgfpathlineto{\pgfqpoint{2.382872in}{1.085587in}}%
\pgfpathlineto{\pgfqpoint{2.391489in}{1.082969in}}%
\pgfpathlineto{\pgfqpoint{2.400437in}{1.076137in}}%
\pgfpathlineto{\pgfqpoint{2.392967in}{1.076431in}}%
\pgfpathlineto{\pgfqpoint{2.392156in}{1.068534in}}%
\pgfpathlineto{\pgfqpoint{2.384647in}{1.067115in}}%
\pgfpathlineto{\pgfqpoint{2.378157in}{1.062296in}}%
\pgfpathlineto{\pgfqpoint{2.375741in}{1.064118in}}%
\pgfpathlineto{\pgfqpoint{2.370868in}{1.060399in}}%
\pgfpathlineto{\pgfqpoint{2.372484in}{1.056245in}}%
\pgfpathclose%
\pgfusepath{fill}%
\end{pgfscope}%
\begin{pgfscope}%
\pgfpathrectangle{\pgfqpoint{0.100000in}{0.100000in}}{\pgfqpoint{3.007045in}{1.925000in}}%
\pgfusepath{clip}%
\pgfsetbuttcap%
\pgfsetmiterjoin%
\definecolor{currentfill}{rgb}{0.793449,0.870142,0.942914}%
\pgfsetfillcolor{currentfill}%
\pgfsetlinewidth{0.000000pt}%
\definecolor{currentstroke}{rgb}{0.000000,0.000000,0.000000}%
\pgfsetstrokecolor{currentstroke}%
\pgfsetstrokeopacity{0.000000}%
\pgfsetdash{}{0pt}%
\pgfpathmoveto{\pgfqpoint{2.623105in}{0.353228in}}%
\pgfpathlineto{\pgfqpoint{2.628250in}{0.323017in}}%
\pgfpathlineto{\pgfqpoint{2.599950in}{0.318474in}}%
\pgfpathlineto{\pgfqpoint{2.594969in}{0.321852in}}%
\pgfpathlineto{\pgfqpoint{2.580427in}{0.321862in}}%
\pgfpathlineto{\pgfqpoint{2.573972in}{0.324804in}}%
\pgfpathlineto{\pgfqpoint{2.569745in}{0.333328in}}%
\pgfpathlineto{\pgfqpoint{2.565657in}{0.347157in}}%
\pgfpathlineto{\pgfqpoint{2.562738in}{0.352174in}}%
\pgfpathlineto{\pgfqpoint{2.555247in}{0.358065in}}%
\pgfpathlineto{\pgfqpoint{2.547708in}{0.358122in}}%
\pgfpathlineto{\pgfqpoint{2.543685in}{0.367836in}}%
\pgfpathlineto{\pgfqpoint{2.546700in}{0.369561in}}%
\pgfpathlineto{\pgfqpoint{2.547508in}{0.376728in}}%
\pgfpathlineto{\pgfqpoint{2.577000in}{0.380965in}}%
\pgfpathlineto{\pgfqpoint{2.579681in}{0.363910in}}%
\pgfpathlineto{\pgfqpoint{2.597057in}{0.366782in}}%
\pgfpathlineto{\pgfqpoint{2.599929in}{0.349210in}}%
\pgfpathlineto{\pgfqpoint{2.623105in}{0.353228in}}%
\pgfpathclose%
\pgfusepath{fill}%
\end{pgfscope}%
\begin{pgfscope}%
\pgfpathrectangle{\pgfqpoint{0.100000in}{0.100000in}}{\pgfqpoint{3.007045in}{1.925000in}}%
\pgfusepath{clip}%
\pgfsetbuttcap%
\pgfsetmiterjoin%
\definecolor{currentfill}{rgb}{0.485490,0.718524,0.853426}%
\pgfsetfillcolor{currentfill}%
\pgfsetlinewidth{0.000000pt}%
\definecolor{currentstroke}{rgb}{0.000000,0.000000,0.000000}%
\pgfsetstrokecolor{currentstroke}%
\pgfsetstrokeopacity{0.000000}%
\pgfsetdash{}{0pt}%
\pgfpathmoveto{\pgfqpoint{0.440258in}{1.107424in}}%
\pgfpathlineto{\pgfqpoint{0.406868in}{1.117231in}}%
\pgfpathlineto{\pgfqpoint{0.382061in}{1.124859in}}%
\pgfpathlineto{\pgfqpoint{0.377822in}{1.132382in}}%
\pgfpathlineto{\pgfqpoint{0.377753in}{1.140606in}}%
\pgfpathlineto{\pgfqpoint{0.375122in}{1.142276in}}%
\pgfpathlineto{\pgfqpoint{0.370888in}{1.155493in}}%
\pgfpathlineto{\pgfqpoint{0.366684in}{1.159674in}}%
\pgfpathlineto{\pgfqpoint{0.363995in}{1.165891in}}%
\pgfpathlineto{\pgfqpoint{0.365227in}{1.184145in}}%
\pgfpathlineto{\pgfqpoint{0.371166in}{1.184004in}}%
\pgfpathlineto{\pgfqpoint{0.374775in}{1.187299in}}%
\pgfpathlineto{\pgfqpoint{0.378870in}{1.195364in}}%
\pgfpathlineto{\pgfqpoint{0.377503in}{1.204628in}}%
\pgfpathlineto{\pgfqpoint{0.365785in}{1.209702in}}%
\pgfpathlineto{\pgfqpoint{0.361242in}{1.215875in}}%
\pgfpathlineto{\pgfqpoint{0.359301in}{1.222111in}}%
\pgfpathlineto{\pgfqpoint{0.359639in}{1.227533in}}%
\pgfpathlineto{\pgfqpoint{0.368463in}{1.226772in}}%
\pgfpathlineto{\pgfqpoint{0.368846in}{1.236860in}}%
\pgfpathlineto{\pgfqpoint{0.370754in}{1.240933in}}%
\pgfpathlineto{\pgfqpoint{0.377116in}{1.242201in}}%
\pgfpathlineto{\pgfqpoint{0.384479in}{1.238341in}}%
\pgfpathlineto{\pgfqpoint{0.388058in}{1.239323in}}%
\pgfpathlineto{\pgfqpoint{0.407661in}{1.233230in}}%
\pgfpathlineto{\pgfqpoint{0.408920in}{1.226474in}}%
\pgfpathlineto{\pgfqpoint{0.404848in}{1.220444in}}%
\pgfpathlineto{\pgfqpoint{0.404869in}{1.211825in}}%
\pgfpathlineto{\pgfqpoint{0.413366in}{1.209180in}}%
\pgfpathlineto{\pgfqpoint{0.413081in}{1.203149in}}%
\pgfpathlineto{\pgfqpoint{0.409170in}{1.194540in}}%
\pgfpathlineto{\pgfqpoint{0.412185in}{1.187494in}}%
\pgfpathlineto{\pgfqpoint{0.421673in}{1.178083in}}%
\pgfpathlineto{\pgfqpoint{0.433028in}{1.157462in}}%
\pgfpathlineto{\pgfqpoint{0.430464in}{1.147237in}}%
\pgfpathlineto{\pgfqpoint{0.426809in}{1.147328in}}%
\pgfpathlineto{\pgfqpoint{0.423734in}{1.137658in}}%
\pgfpathlineto{\pgfqpoint{0.428309in}{1.126977in}}%
\pgfpathlineto{\pgfqpoint{0.432260in}{1.122040in}}%
\pgfpathlineto{\pgfqpoint{0.435906in}{1.120684in}}%
\pgfpathlineto{\pgfqpoint{0.440258in}{1.107424in}}%
\pgfpathclose%
\pgfusepath{fill}%
\end{pgfscope}%
\begin{pgfscope}%
\pgfpathrectangle{\pgfqpoint{0.100000in}{0.100000in}}{\pgfqpoint{3.007045in}{1.925000in}}%
\pgfusepath{clip}%
\pgfsetbuttcap%
\pgfsetmiterjoin%
\definecolor{currentfill}{rgb}{0.412042,0.677186,0.836263}%
\pgfsetfillcolor{currentfill}%
\pgfsetlinewidth{0.000000pt}%
\definecolor{currentstroke}{rgb}{0.000000,0.000000,0.000000}%
\pgfsetstrokecolor{currentstroke}%
\pgfsetstrokeopacity{0.000000}%
\pgfsetdash{}{0pt}%
\pgfpathmoveto{\pgfqpoint{0.966052in}{0.192399in}}%
\pgfpathlineto{\pgfqpoint{0.964721in}{0.193135in}}%
\pgfpathlineto{\pgfqpoint{0.963493in}{0.195565in}}%
\pgfpathlineto{\pgfqpoint{0.958655in}{0.195219in}}%
\pgfpathlineto{\pgfqpoint{0.957247in}{0.193413in}}%
\pgfpathlineto{\pgfqpoint{0.955692in}{0.192812in}}%
\pgfpathlineto{\pgfqpoint{0.954245in}{0.194809in}}%
\pgfpathlineto{\pgfqpoint{0.951030in}{0.194631in}}%
\pgfpathlineto{\pgfqpoint{0.952673in}{0.196998in}}%
\pgfpathlineto{\pgfqpoint{0.954685in}{0.197089in}}%
\pgfpathlineto{\pgfqpoint{0.954041in}{0.199019in}}%
\pgfpathlineto{\pgfqpoint{0.954053in}{0.201532in}}%
\pgfpathlineto{\pgfqpoint{0.954782in}{0.203274in}}%
\pgfpathlineto{\pgfqpoint{0.953594in}{0.205042in}}%
\pgfpathlineto{\pgfqpoint{0.954325in}{0.206175in}}%
\pgfpathlineto{\pgfqpoint{0.956732in}{0.206941in}}%
\pgfpathlineto{\pgfqpoint{0.957233in}{0.209806in}}%
\pgfpathlineto{\pgfqpoint{0.954751in}{0.209244in}}%
\pgfpathlineto{\pgfqpoint{0.954872in}{0.213244in}}%
\pgfpathlineto{\pgfqpoint{0.953569in}{0.215767in}}%
\pgfpathlineto{\pgfqpoint{0.954140in}{0.218651in}}%
\pgfpathlineto{\pgfqpoint{0.953788in}{0.220128in}}%
\pgfpathlineto{\pgfqpoint{0.952256in}{0.220622in}}%
\pgfpathlineto{\pgfqpoint{0.949173in}{0.223004in}}%
\pgfpathlineto{\pgfqpoint{0.948744in}{0.225034in}}%
\pgfpathlineto{\pgfqpoint{0.949609in}{0.226010in}}%
\pgfpathlineto{\pgfqpoint{0.948660in}{0.227534in}}%
\pgfpathlineto{\pgfqpoint{0.949321in}{0.229307in}}%
\pgfpathlineto{\pgfqpoint{0.949075in}{0.234015in}}%
\pgfpathlineto{\pgfqpoint{0.949908in}{0.234459in}}%
\pgfpathlineto{\pgfqpoint{0.950927in}{0.237714in}}%
\pgfpathlineto{\pgfqpoint{0.949190in}{0.237528in}}%
\pgfpathlineto{\pgfqpoint{0.949382in}{0.239911in}}%
\pgfpathlineto{\pgfqpoint{0.951032in}{0.247085in}}%
\pgfpathlineto{\pgfqpoint{0.951672in}{0.251760in}}%
\pgfpathlineto{\pgfqpoint{0.949767in}{0.251172in}}%
\pgfpathlineto{\pgfqpoint{0.950024in}{0.249097in}}%
\pgfpathlineto{\pgfqpoint{0.948795in}{0.248793in}}%
\pgfpathlineto{\pgfqpoint{0.948847in}{0.246567in}}%
\pgfpathlineto{\pgfqpoint{0.947286in}{0.241526in}}%
\pgfpathlineto{\pgfqpoint{0.946946in}{0.236340in}}%
\pgfpathlineto{\pgfqpoint{0.945783in}{0.232618in}}%
\pgfpathlineto{\pgfqpoint{0.944897in}{0.227958in}}%
\pgfpathlineto{\pgfqpoint{0.942843in}{0.225828in}}%
\pgfpathlineto{\pgfqpoint{0.941004in}{0.228956in}}%
\pgfpathlineto{\pgfqpoint{0.941264in}{0.231409in}}%
\pgfpathlineto{\pgfqpoint{0.939615in}{0.232446in}}%
\pgfpathlineto{\pgfqpoint{0.935598in}{0.233694in}}%
\pgfpathlineto{\pgfqpoint{0.937893in}{0.236489in}}%
\pgfpathlineto{\pgfqpoint{0.938389in}{0.240874in}}%
\pgfpathlineto{\pgfqpoint{0.938282in}{0.243236in}}%
\pgfpathlineto{\pgfqpoint{0.935953in}{0.243285in}}%
\pgfpathlineto{\pgfqpoint{0.934974in}{0.245815in}}%
\pgfpathlineto{\pgfqpoint{0.933942in}{0.246488in}}%
\pgfpathlineto{\pgfqpoint{0.934379in}{0.249543in}}%
\pgfpathlineto{\pgfqpoint{0.931764in}{0.250440in}}%
\pgfpathlineto{\pgfqpoint{0.930132in}{0.252504in}}%
\pgfpathlineto{\pgfqpoint{0.929218in}{0.251104in}}%
\pgfpathlineto{\pgfqpoint{0.931463in}{0.248759in}}%
\pgfpathlineto{\pgfqpoint{0.931593in}{0.245629in}}%
\pgfpathlineto{\pgfqpoint{0.933934in}{0.242455in}}%
\pgfpathlineto{\pgfqpoint{0.931815in}{0.241739in}}%
\pgfpathlineto{\pgfqpoint{0.932820in}{0.240335in}}%
\pgfpathlineto{\pgfqpoint{0.934512in}{0.240731in}}%
\pgfpathlineto{\pgfqpoint{0.934225in}{0.234374in}}%
\pgfpathlineto{\pgfqpoint{0.930700in}{0.234187in}}%
\pgfpathlineto{\pgfqpoint{0.931316in}{0.236252in}}%
\pgfpathlineto{\pgfqpoint{0.928812in}{0.235626in}}%
\pgfpathlineto{\pgfqpoint{0.925421in}{0.233892in}}%
\pgfpathlineto{\pgfqpoint{0.925428in}{0.237038in}}%
\pgfpathlineto{\pgfqpoint{0.924460in}{0.239424in}}%
\pgfpathlineto{\pgfqpoint{0.922320in}{0.240051in}}%
\pgfpathlineto{\pgfqpoint{0.919294in}{0.247052in}}%
\pgfpathlineto{\pgfqpoint{0.919630in}{0.248516in}}%
\pgfpathlineto{\pgfqpoint{0.917767in}{0.252248in}}%
\pgfpathlineto{\pgfqpoint{0.917999in}{0.256789in}}%
\pgfpathlineto{\pgfqpoint{0.916349in}{0.261630in}}%
\pgfpathlineto{\pgfqpoint{0.913473in}{0.265334in}}%
\pgfpathlineto{\pgfqpoint{0.912193in}{0.268097in}}%
\pgfpathlineto{\pgfqpoint{0.908339in}{0.273326in}}%
\pgfpathlineto{\pgfqpoint{0.904581in}{0.280446in}}%
\pgfpathlineto{\pgfqpoint{0.907013in}{0.280057in}}%
\pgfpathlineto{\pgfqpoint{0.910459in}{0.281718in}}%
\pgfpathlineto{\pgfqpoint{0.910823in}{0.286283in}}%
\pgfpathlineto{\pgfqpoint{0.911674in}{0.287962in}}%
\pgfpathlineto{\pgfqpoint{0.907295in}{0.286400in}}%
\pgfpathlineto{\pgfqpoint{0.905838in}{0.288835in}}%
\pgfpathlineto{\pgfqpoint{0.902783in}{0.288080in}}%
\pgfpathlineto{\pgfqpoint{0.901946in}{0.286374in}}%
\pgfpathlineto{\pgfqpoint{0.897692in}{0.289155in}}%
\pgfpathlineto{\pgfqpoint{0.895195in}{0.292046in}}%
\pgfpathlineto{\pgfqpoint{0.901489in}{0.303399in}}%
\pgfpathlineto{\pgfqpoint{0.905283in}{0.298997in}}%
\pgfpathlineto{\pgfqpoint{0.907695in}{0.299493in}}%
\pgfpathlineto{\pgfqpoint{0.910640in}{0.295087in}}%
\pgfpathlineto{\pgfqpoint{0.915744in}{0.296619in}}%
\pgfpathlineto{\pgfqpoint{0.920918in}{0.294241in}}%
\pgfpathlineto{\pgfqpoint{0.921920in}{0.292840in}}%
\pgfpathlineto{\pgfqpoint{0.917937in}{0.288586in}}%
\pgfpathlineto{\pgfqpoint{0.918519in}{0.285751in}}%
\pgfpathlineto{\pgfqpoint{0.921178in}{0.282019in}}%
\pgfpathlineto{\pgfqpoint{0.920601in}{0.278454in}}%
\pgfpathlineto{\pgfqpoint{0.925898in}{0.261651in}}%
\pgfpathlineto{\pgfqpoint{0.924680in}{0.256502in}}%
\pgfpathlineto{\pgfqpoint{0.923189in}{0.253738in}}%
\pgfpathlineto{\pgfqpoint{0.924096in}{0.253355in}}%
\pgfpathlineto{\pgfqpoint{0.927167in}{0.254345in}}%
\pgfpathlineto{\pgfqpoint{0.932410in}{0.254987in}}%
\pgfpathlineto{\pgfqpoint{0.936447in}{0.254101in}}%
\pgfpathlineto{\pgfqpoint{0.938989in}{0.256067in}}%
\pgfpathlineto{\pgfqpoint{0.940309in}{0.258917in}}%
\pgfpathlineto{\pgfqpoint{0.942720in}{0.259083in}}%
\pgfpathlineto{\pgfqpoint{0.946116in}{0.262219in}}%
\pgfpathlineto{\pgfqpoint{0.948733in}{0.261499in}}%
\pgfpathlineto{\pgfqpoint{0.954529in}{0.261913in}}%
\pgfpathlineto{\pgfqpoint{0.955655in}{0.261214in}}%
\pgfpathlineto{\pgfqpoint{0.957563in}{0.255964in}}%
\pgfpathlineto{\pgfqpoint{0.957437in}{0.253580in}}%
\pgfpathlineto{\pgfqpoint{0.955678in}{0.251817in}}%
\pgfpathlineto{\pgfqpoint{0.955644in}{0.247945in}}%
\pgfpathlineto{\pgfqpoint{0.958149in}{0.245787in}}%
\pgfpathlineto{\pgfqpoint{0.958180in}{0.243117in}}%
\pgfpathlineto{\pgfqpoint{0.959385in}{0.241184in}}%
\pgfpathlineto{\pgfqpoint{0.958932in}{0.238807in}}%
\pgfpathlineto{\pgfqpoint{0.959232in}{0.235056in}}%
\pgfpathlineto{\pgfqpoint{0.962530in}{0.230044in}}%
\pgfpathlineto{\pgfqpoint{0.962897in}{0.225216in}}%
\pgfpathlineto{\pgfqpoint{0.964318in}{0.222393in}}%
\pgfpathlineto{\pgfqpoint{0.963412in}{0.220631in}}%
\pgfpathlineto{\pgfqpoint{0.964300in}{0.213106in}}%
\pgfpathlineto{\pgfqpoint{0.964057in}{0.210474in}}%
\pgfpathlineto{\pgfqpoint{0.964942in}{0.205541in}}%
\pgfpathlineto{\pgfqpoint{0.965301in}{0.196768in}}%
\pgfpathlineto{\pgfqpoint{0.966052in}{0.192399in}}%
\pgfpathclose%
\pgfusepath{fill}%
\end{pgfscope}%
\begin{pgfscope}%
\pgfpathrectangle{\pgfqpoint{0.100000in}{0.100000in}}{\pgfqpoint{3.007045in}{1.925000in}}%
\pgfusepath{clip}%
\pgfsetbuttcap%
\pgfsetmiterjoin%
\definecolor{currentfill}{rgb}{0.412042,0.677186,0.836263}%
\pgfsetfillcolor{currentfill}%
\pgfsetlinewidth{0.000000pt}%
\definecolor{currentstroke}{rgb}{0.000000,0.000000,0.000000}%
\pgfsetstrokecolor{currentstroke}%
\pgfsetstrokeopacity{0.000000}%
\pgfsetdash{}{0pt}%
\pgfpathmoveto{\pgfqpoint{0.936626in}{0.197068in}}%
\pgfpathlineto{\pgfqpoint{0.938877in}{0.201656in}}%
\pgfpathlineto{\pgfqpoint{0.940558in}{0.202970in}}%
\pgfpathlineto{\pgfqpoint{0.942011in}{0.205039in}}%
\pgfpathlineto{\pgfqpoint{0.942374in}{0.211894in}}%
\pgfpathlineto{\pgfqpoint{0.944008in}{0.216194in}}%
\pgfpathlineto{\pgfqpoint{0.944696in}{0.219988in}}%
\pgfpathlineto{\pgfqpoint{0.945984in}{0.221447in}}%
\pgfpathlineto{\pgfqpoint{0.945425in}{0.223366in}}%
\pgfpathlineto{\pgfqpoint{0.945878in}{0.227745in}}%
\pgfpathlineto{\pgfqpoint{0.947540in}{0.225877in}}%
\pgfpathlineto{\pgfqpoint{0.947168in}{0.222342in}}%
\pgfpathlineto{\pgfqpoint{0.948688in}{0.221616in}}%
\pgfpathlineto{\pgfqpoint{0.952311in}{0.218681in}}%
\pgfpathlineto{\pgfqpoint{0.952653in}{0.214058in}}%
\pgfpathlineto{\pgfqpoint{0.952346in}{0.208455in}}%
\pgfpathlineto{\pgfqpoint{0.950987in}{0.209391in}}%
\pgfpathlineto{\pgfqpoint{0.951266in}{0.216399in}}%
\pgfpathlineto{\pgfqpoint{0.950054in}{0.216109in}}%
\pgfpathlineto{\pgfqpoint{0.948419in}{0.213095in}}%
\pgfpathlineto{\pgfqpoint{0.949973in}{0.212546in}}%
\pgfpathlineto{\pgfqpoint{0.949906in}{0.210661in}}%
\pgfpathlineto{\pgfqpoint{0.948880in}{0.207820in}}%
\pgfpathlineto{\pgfqpoint{0.950125in}{0.207070in}}%
\pgfpathlineto{\pgfqpoint{0.949683in}{0.203304in}}%
\pgfpathlineto{\pgfqpoint{0.947074in}{0.204157in}}%
\pgfpathlineto{\pgfqpoint{0.948879in}{0.200841in}}%
\pgfpathlineto{\pgfqpoint{0.947064in}{0.199809in}}%
\pgfpathlineto{\pgfqpoint{0.946264in}{0.200919in}}%
\pgfpathlineto{\pgfqpoint{0.944697in}{0.199414in}}%
\pgfpathlineto{\pgfqpoint{0.942282in}{0.199077in}}%
\pgfpathlineto{\pgfqpoint{0.939613in}{0.197250in}}%
\pgfpathlineto{\pgfqpoint{0.936626in}{0.197068in}}%
\pgfpathclose%
\pgfusepath{fill}%
\end{pgfscope}%
\begin{pgfscope}%
\pgfpathrectangle{\pgfqpoint{0.100000in}{0.100000in}}{\pgfqpoint{3.007045in}{1.925000in}}%
\pgfusepath{clip}%
\pgfsetbuttcap%
\pgfsetmiterjoin%
\definecolor{currentfill}{rgb}{0.412042,0.677186,0.836263}%
\pgfsetfillcolor{currentfill}%
\pgfsetlinewidth{0.000000pt}%
\definecolor{currentstroke}{rgb}{0.000000,0.000000,0.000000}%
\pgfsetstrokecolor{currentstroke}%
\pgfsetstrokeopacity{0.000000}%
\pgfsetdash{}{0pt}%
\pgfpathmoveto{\pgfqpoint{0.924230in}{0.224566in}}%
\pgfpathlineto{\pgfqpoint{0.924247in}{0.226972in}}%
\pgfpathlineto{\pgfqpoint{0.925304in}{0.230453in}}%
\pgfpathlineto{\pgfqpoint{0.928832in}{0.230375in}}%
\pgfpathlineto{\pgfqpoint{0.930194in}{0.229696in}}%
\pgfpathlineto{\pgfqpoint{0.931247in}{0.231217in}}%
\pgfpathlineto{\pgfqpoint{0.932446in}{0.229785in}}%
\pgfpathlineto{\pgfqpoint{0.935561in}{0.230884in}}%
\pgfpathlineto{\pgfqpoint{0.937650in}{0.227795in}}%
\pgfpathlineto{\pgfqpoint{0.938131in}{0.224744in}}%
\pgfpathlineto{\pgfqpoint{0.941809in}{0.222536in}}%
\pgfpathlineto{\pgfqpoint{0.943001in}{0.220916in}}%
\pgfpathlineto{\pgfqpoint{0.942398in}{0.218706in}}%
\pgfpathlineto{\pgfqpoint{0.941010in}{0.217751in}}%
\pgfpathlineto{\pgfqpoint{0.939636in}{0.215477in}}%
\pgfpathlineto{\pgfqpoint{0.936886in}{0.216547in}}%
\pgfpathlineto{\pgfqpoint{0.934778in}{0.220682in}}%
\pgfpathlineto{\pgfqpoint{0.933152in}{0.222753in}}%
\pgfpathlineto{\pgfqpoint{0.931898in}{0.225478in}}%
\pgfpathlineto{\pgfqpoint{0.928775in}{0.222468in}}%
\pgfpathlineto{\pgfqpoint{0.924230in}{0.224566in}}%
\pgfpathclose%
\pgfusepath{fill}%
\end{pgfscope}%
\begin{pgfscope}%
\pgfpathrectangle{\pgfqpoint{0.100000in}{0.100000in}}{\pgfqpoint{3.007045in}{1.925000in}}%
\pgfusepath{clip}%
\pgfsetbuttcap%
\pgfsetmiterjoin%
\definecolor{currentfill}{rgb}{0.535686,0.746082,0.864252}%
\pgfsetfillcolor{currentfill}%
\pgfsetlinewidth{0.000000pt}%
\definecolor{currentstroke}{rgb}{0.000000,0.000000,0.000000}%
\pgfsetstrokecolor{currentstroke}%
\pgfsetstrokeopacity{0.000000}%
\pgfsetdash{}{0pt}%
\pgfpathmoveto{\pgfqpoint{2.803517in}{1.405975in}}%
\pgfpathlineto{\pgfqpoint{2.803333in}{1.417011in}}%
\pgfpathlineto{\pgfqpoint{2.822125in}{1.424128in}}%
\pgfpathlineto{\pgfqpoint{2.824695in}{1.409031in}}%
\pgfpathlineto{\pgfqpoint{2.829259in}{1.404526in}}%
\pgfpathlineto{\pgfqpoint{2.819156in}{1.394508in}}%
\pgfpathlineto{\pgfqpoint{2.824324in}{1.387968in}}%
\pgfpathlineto{\pgfqpoint{2.826877in}{1.382803in}}%
\pgfpathlineto{\pgfqpoint{2.833070in}{1.385813in}}%
\pgfpathlineto{\pgfqpoint{2.846367in}{1.387920in}}%
\pgfpathlineto{\pgfqpoint{2.853875in}{1.393696in}}%
\pgfpathlineto{\pgfqpoint{2.867459in}{1.397196in}}%
\pgfpathlineto{\pgfqpoint{2.873958in}{1.399810in}}%
\pgfpathlineto{\pgfqpoint{2.883054in}{1.409613in}}%
\pgfpathlineto{\pgfqpoint{2.884754in}{1.413273in}}%
\pgfpathlineto{\pgfqpoint{2.890760in}{1.408191in}}%
\pgfpathlineto{\pgfqpoint{2.895504in}{1.410186in}}%
\pgfpathlineto{\pgfqpoint{2.899044in}{1.407001in}}%
\pgfpathlineto{\pgfqpoint{2.904883in}{1.413872in}}%
\pgfpathlineto{\pgfqpoint{2.909280in}{1.413649in}}%
\pgfpathlineto{\pgfqpoint{2.882440in}{1.391754in}}%
\pgfpathlineto{\pgfqpoint{2.865813in}{1.380465in}}%
\pgfpathlineto{\pgfqpoint{2.859459in}{1.375218in}}%
\pgfpathlineto{\pgfqpoint{2.849223in}{1.369251in}}%
\pgfpathlineto{\pgfqpoint{2.845081in}{1.369000in}}%
\pgfpathlineto{\pgfqpoint{2.835181in}{1.363297in}}%
\pgfpathlineto{\pgfqpoint{2.824782in}{1.361319in}}%
\pgfpathlineto{\pgfqpoint{2.818078in}{1.357032in}}%
\pgfpathlineto{\pgfqpoint{2.810932in}{1.358841in}}%
\pgfpathlineto{\pgfqpoint{2.808914in}{1.354582in}}%
\pgfpathlineto{\pgfqpoint{2.802639in}{1.349500in}}%
\pgfpathlineto{\pgfqpoint{2.803807in}{1.360108in}}%
\pgfpathlineto{\pgfqpoint{2.810062in}{1.361931in}}%
\pgfpathlineto{\pgfqpoint{2.812809in}{1.384940in}}%
\pgfpathlineto{\pgfqpoint{2.810869in}{1.395086in}}%
\pgfpathlineto{\pgfqpoint{2.805188in}{1.401167in}}%
\pgfpathlineto{\pgfqpoint{2.803517in}{1.405975in}}%
\pgfpathclose%
\pgfusepath{fill}%
\end{pgfscope}%
\begin{pgfscope}%
\pgfpathrectangle{\pgfqpoint{0.100000in}{0.100000in}}{\pgfqpoint{3.007045in}{1.925000in}}%
\pgfusepath{clip}%
\pgfsetbuttcap%
\pgfsetmiterjoin%
\definecolor{currentfill}{rgb}{0.429020,0.687520,0.841246}%
\pgfsetfillcolor{currentfill}%
\pgfsetlinewidth{0.000000pt}%
\definecolor{currentstroke}{rgb}{0.000000,0.000000,0.000000}%
\pgfsetstrokecolor{currentstroke}%
\pgfsetstrokeopacity{0.000000}%
\pgfsetdash{}{0pt}%
\pgfpathmoveto{\pgfqpoint{0.981446in}{1.653571in}}%
\pgfpathlineto{\pgfqpoint{0.972309in}{1.661350in}}%
\pgfpathlineto{\pgfqpoint{0.973053in}{1.678906in}}%
\pgfpathlineto{\pgfqpoint{0.969357in}{1.685011in}}%
\pgfpathlineto{\pgfqpoint{0.971079in}{1.692732in}}%
\pgfpathlineto{\pgfqpoint{0.973346in}{1.694790in}}%
\pgfpathlineto{\pgfqpoint{0.982127in}{1.696489in}}%
\pgfpathlineto{\pgfqpoint{0.985045in}{1.700348in}}%
\pgfpathlineto{\pgfqpoint{0.987484in}{1.710382in}}%
\pgfpathlineto{\pgfqpoint{0.983438in}{1.717518in}}%
\pgfpathlineto{\pgfqpoint{0.977851in}{1.718526in}}%
\pgfpathlineto{\pgfqpoint{0.979570in}{1.727393in}}%
\pgfpathlineto{\pgfqpoint{0.967944in}{1.729505in}}%
\pgfpathlineto{\pgfqpoint{0.972359in}{1.752059in}}%
\pgfpathlineto{\pgfqpoint{0.960752in}{1.754525in}}%
\pgfpathlineto{\pgfqpoint{0.964785in}{1.774681in}}%
\pgfpathlineto{\pgfqpoint{0.962444in}{1.782513in}}%
\pgfpathlineto{\pgfqpoint{0.963166in}{1.790823in}}%
\pgfpathlineto{\pgfqpoint{0.967284in}{1.791973in}}%
\pgfpathlineto{\pgfqpoint{0.969863in}{1.800673in}}%
\pgfpathlineto{\pgfqpoint{0.974188in}{1.804466in}}%
\pgfpathlineto{\pgfqpoint{0.975711in}{1.795488in}}%
\pgfpathlineto{\pgfqpoint{0.973893in}{1.786912in}}%
\pgfpathlineto{\pgfqpoint{0.976971in}{1.780376in}}%
\pgfpathlineto{\pgfqpoint{0.987897in}{1.779953in}}%
\pgfpathlineto{\pgfqpoint{0.994400in}{1.776622in}}%
\pgfpathlineto{\pgfqpoint{0.994442in}{1.773321in}}%
\pgfpathlineto{\pgfqpoint{0.999503in}{1.769057in}}%
\pgfpathlineto{\pgfqpoint{1.009841in}{1.767764in}}%
\pgfpathlineto{\pgfqpoint{1.005933in}{1.746597in}}%
\pgfpathlineto{\pgfqpoint{1.010013in}{1.742368in}}%
\pgfpathlineto{\pgfqpoint{1.016513in}{1.740285in}}%
\pgfpathlineto{\pgfqpoint{1.013851in}{1.726227in}}%
\pgfpathlineto{\pgfqpoint{1.019738in}{1.725093in}}%
\pgfpathlineto{\pgfqpoint{1.019449in}{1.720433in}}%
\pgfpathlineto{\pgfqpoint{1.025028in}{1.717016in}}%
\pgfpathlineto{\pgfqpoint{1.027516in}{1.705647in}}%
\pgfpathlineto{\pgfqpoint{1.030764in}{1.699834in}}%
\pgfpathlineto{\pgfqpoint{1.033247in}{1.689867in}}%
\pgfpathlineto{\pgfqpoint{1.037077in}{1.690684in}}%
\pgfpathlineto{\pgfqpoint{1.040593in}{1.686492in}}%
\pgfpathlineto{\pgfqpoint{1.036887in}{1.681674in}}%
\pgfpathlineto{\pgfqpoint{1.037941in}{1.673119in}}%
\pgfpathlineto{\pgfqpoint{1.025044in}{1.675119in}}%
\pgfpathlineto{\pgfqpoint{1.024131in}{1.670887in}}%
\pgfpathlineto{\pgfqpoint{1.019996in}{1.667445in}}%
\pgfpathlineto{\pgfqpoint{1.016570in}{1.660701in}}%
\pgfpathlineto{\pgfqpoint{1.010507in}{1.658210in}}%
\pgfpathlineto{\pgfqpoint{1.004464in}{1.652753in}}%
\pgfpathlineto{\pgfqpoint{0.997011in}{1.655494in}}%
\pgfpathlineto{\pgfqpoint{0.994682in}{1.658385in}}%
\pgfpathlineto{\pgfqpoint{0.988214in}{1.659319in}}%
\pgfpathlineto{\pgfqpoint{0.981446in}{1.653571in}}%
\pgfpathclose%
\pgfusepath{fill}%
\end{pgfscope}%
\begin{pgfscope}%
\pgfpathrectangle{\pgfqpoint{0.100000in}{0.100000in}}{\pgfqpoint{3.007045in}{1.925000in}}%
\pgfusepath{clip}%
\pgfsetbuttcap%
\pgfsetmiterjoin%
\definecolor{currentfill}{rgb}{0.371688,0.649627,0.820515}%
\pgfsetfillcolor{currentfill}%
\pgfsetlinewidth{0.000000pt}%
\definecolor{currentstroke}{rgb}{0.000000,0.000000,0.000000}%
\pgfsetstrokecolor{currentstroke}%
\pgfsetstrokeopacity{0.000000}%
\pgfsetdash{}{0pt}%
\pgfpathmoveto{\pgfqpoint{1.653431in}{1.223538in}}%
\pgfpathlineto{\pgfqpoint{1.630558in}{1.224197in}}%
\pgfpathlineto{\pgfqpoint{1.631190in}{1.247136in}}%
\pgfpathlineto{\pgfqpoint{1.632031in}{1.278421in}}%
\pgfpathlineto{\pgfqpoint{1.635249in}{1.282362in}}%
\pgfpathlineto{\pgfqpoint{1.643733in}{1.288581in}}%
\pgfpathlineto{\pgfqpoint{1.655235in}{1.292354in}}%
\pgfpathlineto{\pgfqpoint{1.654636in}{1.269418in}}%
\pgfpathlineto{\pgfqpoint{1.653431in}{1.223538in}}%
\pgfpathclose%
\pgfusepath{fill}%
\end{pgfscope}%
\begin{pgfscope}%
\pgfpathrectangle{\pgfqpoint{0.100000in}{0.100000in}}{\pgfqpoint{3.007045in}{1.925000in}}%
\pgfusepath{clip}%
\pgfsetbuttcap%
\pgfsetmiterjoin%
\definecolor{currentfill}{rgb}{0.260715,0.573841,0.777209}%
\pgfsetfillcolor{currentfill}%
\pgfsetlinewidth{0.000000pt}%
\definecolor{currentstroke}{rgb}{0.000000,0.000000,0.000000}%
\pgfsetstrokecolor{currentstroke}%
\pgfsetstrokeopacity{0.000000}%
\pgfsetdash{}{0pt}%
\pgfpathmoveto{\pgfqpoint{0.630991in}{1.684466in}}%
\pgfpathlineto{\pgfqpoint{0.628849in}{1.673291in}}%
\pgfpathlineto{\pgfqpoint{0.619536in}{1.640423in}}%
\pgfpathlineto{\pgfqpoint{0.608444in}{1.643504in}}%
\pgfpathlineto{\pgfqpoint{0.609918in}{1.648785in}}%
\pgfpathlineto{\pgfqpoint{0.605211in}{1.653481in}}%
\pgfpathlineto{\pgfqpoint{0.588656in}{1.658125in}}%
\pgfpathlineto{\pgfqpoint{0.594595in}{1.678719in}}%
\pgfpathlineto{\pgfqpoint{0.596303in}{1.682378in}}%
\pgfpathlineto{\pgfqpoint{0.593360in}{1.686222in}}%
\pgfpathlineto{\pgfqpoint{0.593379in}{1.692150in}}%
\pgfpathlineto{\pgfqpoint{0.598195in}{1.705394in}}%
\pgfpathlineto{\pgfqpoint{0.595993in}{1.712179in}}%
\pgfpathlineto{\pgfqpoint{0.602494in}{1.724955in}}%
\pgfpathlineto{\pgfqpoint{0.608895in}{1.725281in}}%
\pgfpathlineto{\pgfqpoint{0.608690in}{1.732151in}}%
\pgfpathlineto{\pgfqpoint{0.606293in}{1.736988in}}%
\pgfpathlineto{\pgfqpoint{0.610892in}{1.738691in}}%
\pgfpathlineto{\pgfqpoint{0.619982in}{1.737934in}}%
\pgfpathlineto{\pgfqpoint{0.630951in}{1.740790in}}%
\pgfpathlineto{\pgfqpoint{0.620630in}{1.705467in}}%
\pgfpathlineto{\pgfqpoint{0.626202in}{1.703795in}}%
\pgfpathlineto{\pgfqpoint{0.624611in}{1.698303in}}%
\pgfpathlineto{\pgfqpoint{0.630102in}{1.696622in}}%
\pgfpathlineto{\pgfqpoint{0.625624in}{1.685979in}}%
\pgfpathlineto{\pgfqpoint{0.630991in}{1.684466in}}%
\pgfpathclose%
\pgfusepath{fill}%
\end{pgfscope}%
\begin{pgfscope}%
\pgfpathrectangle{\pgfqpoint{0.100000in}{0.100000in}}{\pgfqpoint{3.007045in}{1.925000in}}%
\pgfusepath{clip}%
\pgfsetbuttcap%
\pgfsetmiterjoin%
\definecolor{currentfill}{rgb}{0.290980,0.594510,0.789020}%
\pgfsetfillcolor{currentfill}%
\pgfsetlinewidth{0.000000pt}%
\definecolor{currentstroke}{rgb}{0.000000,0.000000,0.000000}%
\pgfsetstrokecolor{currentstroke}%
\pgfsetstrokeopacity{0.000000}%
\pgfsetdash{}{0pt}%
\pgfpathmoveto{\pgfqpoint{1.510401in}{1.206498in}}%
\pgfpathlineto{\pgfqpoint{1.453477in}{1.210184in}}%
\pgfpathlineto{\pgfqpoint{1.455104in}{1.233038in}}%
\pgfpathlineto{\pgfqpoint{1.418635in}{1.235643in}}%
\pgfpathlineto{\pgfqpoint{1.420426in}{1.258524in}}%
\pgfpathlineto{\pgfqpoint{1.460453in}{1.255605in}}%
\pgfpathlineto{\pgfqpoint{1.483897in}{1.254208in}}%
\pgfpathlineto{\pgfqpoint{1.482455in}{1.231145in}}%
\pgfpathlineto{\pgfqpoint{1.511500in}{1.229428in}}%
\pgfpathlineto{\pgfqpoint{1.510401in}{1.206498in}}%
\pgfpathclose%
\pgfusepath{fill}%
\end{pgfscope}%
\begin{pgfscope}%
\pgfpathrectangle{\pgfqpoint{0.100000in}{0.100000in}}{\pgfqpoint{3.007045in}{1.925000in}}%
\pgfusepath{clip}%
\pgfsetbuttcap%
\pgfsetmiterjoin%
\definecolor{currentfill}{rgb}{0.479216,0.715079,0.852072}%
\pgfsetfillcolor{currentfill}%
\pgfsetlinewidth{0.000000pt}%
\definecolor{currentstroke}{rgb}{0.000000,0.000000,0.000000}%
\pgfsetstrokecolor{currentstroke}%
\pgfsetstrokeopacity{0.000000}%
\pgfsetdash{}{0pt}%
\pgfpathmoveto{\pgfqpoint{1.035972in}{1.058230in}}%
\pgfpathlineto{\pgfqpoint{0.986309in}{1.066160in}}%
\pgfpathlineto{\pgfqpoint{0.947453in}{1.073147in}}%
\pgfpathlineto{\pgfqpoint{0.915481in}{1.078785in}}%
\pgfpathlineto{\pgfqpoint{0.923032in}{1.079857in}}%
\pgfpathlineto{\pgfqpoint{0.926450in}{1.083530in}}%
\pgfpathlineto{\pgfqpoint{0.932657in}{1.082202in}}%
\pgfpathlineto{\pgfqpoint{0.941624in}{1.085677in}}%
\pgfpathlineto{\pgfqpoint{0.944689in}{1.092843in}}%
\pgfpathlineto{\pgfqpoint{0.951769in}{1.095219in}}%
\pgfpathlineto{\pgfqpoint{0.958933in}{1.107125in}}%
\pgfpathlineto{\pgfqpoint{0.961951in}{1.110649in}}%
\pgfpathlineto{\pgfqpoint{0.967830in}{1.113017in}}%
\pgfpathlineto{\pgfqpoint{0.975147in}{1.126984in}}%
\pgfpathlineto{\pgfqpoint{0.982614in}{1.127004in}}%
\pgfpathlineto{\pgfqpoint{0.990253in}{1.132805in}}%
\pgfpathlineto{\pgfqpoint{0.992394in}{1.131729in}}%
\pgfpathlineto{\pgfqpoint{0.997732in}{1.138462in}}%
\pgfpathlineto{\pgfqpoint{1.002522in}{1.140541in}}%
\pgfpathlineto{\pgfqpoint{1.005017in}{1.143547in}}%
\pgfpathlineto{\pgfqpoint{1.002930in}{1.149720in}}%
\pgfpathlineto{\pgfqpoint{1.001312in}{1.163987in}}%
\pgfpathlineto{\pgfqpoint{1.050444in}{1.156057in}}%
\pgfpathlineto{\pgfqpoint{1.048160in}{1.141447in}}%
\pgfpathlineto{\pgfqpoint{1.047972in}{1.134089in}}%
\pgfpathlineto{\pgfqpoint{1.045110in}{1.115632in}}%
\pgfpathlineto{\pgfqpoint{1.056798in}{1.114784in}}%
\pgfpathlineto{\pgfqpoint{1.085598in}{1.110401in}}%
\pgfpathlineto{\pgfqpoint{1.087280in}{1.105261in}}%
\pgfpathlineto{\pgfqpoint{1.098980in}{1.105934in}}%
\pgfpathlineto{\pgfqpoint{1.104848in}{1.099688in}}%
\pgfpathlineto{\pgfqpoint{1.103247in}{1.096115in}}%
\pgfpathlineto{\pgfqpoint{1.094867in}{1.088434in}}%
\pgfpathlineto{\pgfqpoint{1.093042in}{1.080708in}}%
\pgfpathlineto{\pgfqpoint{1.088472in}{1.075836in}}%
\pgfpathlineto{\pgfqpoint{1.083485in}{1.074787in}}%
\pgfpathlineto{\pgfqpoint{1.077407in}{1.066910in}}%
\pgfpathlineto{\pgfqpoint{1.073504in}{1.056252in}}%
\pgfpathlineto{\pgfqpoint{1.070635in}{1.052989in}}%
\pgfpathlineto{\pgfqpoint{1.035972in}{1.058230in}}%
\pgfpathclose%
\pgfusepath{fill}%
\end{pgfscope}%
\begin{pgfscope}%
\pgfpathrectangle{\pgfqpoint{0.100000in}{0.100000in}}{\pgfqpoint{3.007045in}{1.925000in}}%
\pgfusepath{clip}%
\pgfsetbuttcap%
\pgfsetmiterjoin%
\definecolor{currentfill}{rgb}{0.417086,0.680631,0.838231}%
\pgfsetfillcolor{currentfill}%
\pgfsetlinewidth{0.000000pt}%
\definecolor{currentstroke}{rgb}{0.000000,0.000000,0.000000}%
\pgfsetstrokecolor{currentstroke}%
\pgfsetstrokeopacity{0.000000}%
\pgfsetdash{}{0pt}%
\pgfpathmoveto{\pgfqpoint{2.241623in}{0.891209in}}%
\pgfpathlineto{\pgfqpoint{2.240204in}{0.883283in}}%
\pgfpathlineto{\pgfqpoint{2.241813in}{0.880528in}}%
\pgfpathlineto{\pgfqpoint{2.241299in}{0.867393in}}%
\pgfpathlineto{\pgfqpoint{2.244466in}{0.861498in}}%
\pgfpathlineto{\pgfqpoint{2.238722in}{0.856984in}}%
\pgfpathlineto{\pgfqpoint{2.232658in}{0.857963in}}%
\pgfpathlineto{\pgfqpoint{2.231240in}{0.844683in}}%
\pgfpathlineto{\pgfqpoint{2.202469in}{0.842698in}}%
\pgfpathlineto{\pgfqpoint{2.202564in}{0.841744in}}%
\pgfpathlineto{\pgfqpoint{2.179696in}{0.840159in}}%
\pgfpathlineto{\pgfqpoint{2.178278in}{0.857446in}}%
\pgfpathlineto{\pgfqpoint{2.183086in}{0.867245in}}%
\pgfpathlineto{\pgfqpoint{2.182587in}{0.873244in}}%
\pgfpathlineto{\pgfqpoint{2.189352in}{0.871261in}}%
\pgfpathlineto{\pgfqpoint{2.193931in}{0.875265in}}%
\pgfpathlineto{\pgfqpoint{2.193121in}{0.887325in}}%
\pgfpathlineto{\pgfqpoint{2.213336in}{0.888623in}}%
\pgfpathlineto{\pgfqpoint{2.212159in}{0.906542in}}%
\pgfpathlineto{\pgfqpoint{2.216592in}{0.906228in}}%
\pgfpathlineto{\pgfqpoint{2.223886in}{0.911812in}}%
\pgfpathlineto{\pgfqpoint{2.223823in}{0.914321in}}%
\pgfpathlineto{\pgfqpoint{2.235740in}{0.906849in}}%
\pgfpathlineto{\pgfqpoint{2.237551in}{0.900340in}}%
\pgfpathlineto{\pgfqpoint{2.240394in}{0.900117in}}%
\pgfpathlineto{\pgfqpoint{2.241623in}{0.891209in}}%
\pgfpathclose%
\pgfusepath{fill}%
\end{pgfscope}%
\begin{pgfscope}%
\pgfpathrectangle{\pgfqpoint{0.100000in}{0.100000in}}{\pgfqpoint{3.007045in}{1.925000in}}%
\pgfusepath{clip}%
\pgfsetbuttcap%
\pgfsetmiterjoin%
\definecolor{currentfill}{rgb}{0.361599,0.642737,0.816578}%
\pgfsetfillcolor{currentfill}%
\pgfsetlinewidth{0.000000pt}%
\definecolor{currentstroke}{rgb}{0.000000,0.000000,0.000000}%
\pgfsetstrokecolor{currentstroke}%
\pgfsetstrokeopacity{0.000000}%
\pgfsetdash{}{0pt}%
\pgfpathmoveto{\pgfqpoint{1.062663in}{1.231515in}}%
\pgfpathlineto{\pgfqpoint{1.072883in}{1.296139in}}%
\pgfpathlineto{\pgfqpoint{1.076315in}{1.318750in}}%
\pgfpathlineto{\pgfqpoint{1.102069in}{1.314741in}}%
\pgfpathlineto{\pgfqpoint{1.131992in}{1.310519in}}%
\pgfpathlineto{\pgfqpoint{1.184259in}{1.303392in}}%
\pgfpathlineto{\pgfqpoint{1.183883in}{1.298271in}}%
\pgfpathlineto{\pgfqpoint{1.192988in}{1.292052in}}%
\pgfpathlineto{\pgfqpoint{1.193426in}{1.288196in}}%
\pgfpathlineto{\pgfqpoint{1.188332in}{1.274642in}}%
\pgfpathlineto{\pgfqpoint{1.189706in}{1.265642in}}%
\pgfpathlineto{\pgfqpoint{1.189821in}{1.258740in}}%
\pgfpathlineto{\pgfqpoint{1.186728in}{1.236591in}}%
\pgfpathlineto{\pgfqpoint{1.186601in}{1.231472in}}%
\pgfpathlineto{\pgfqpoint{1.166255in}{1.233746in}}%
\pgfpathlineto{\pgfqpoint{1.167476in}{1.245055in}}%
\pgfpathlineto{\pgfqpoint{1.153503in}{1.246887in}}%
\pgfpathlineto{\pgfqpoint{1.151850in}{1.235425in}}%
\pgfpathlineto{\pgfqpoint{1.146223in}{1.236379in}}%
\pgfpathlineto{\pgfqpoint{1.145415in}{1.230549in}}%
\pgfpathlineto{\pgfqpoint{1.120000in}{1.233895in}}%
\pgfpathlineto{\pgfqpoint{1.118778in}{1.225384in}}%
\pgfpathlineto{\pgfqpoint{1.087872in}{1.229867in}}%
\pgfpathlineto{\pgfqpoint{1.087451in}{1.227034in}}%
\pgfpathlineto{\pgfqpoint{1.062663in}{1.231515in}}%
\pgfpathclose%
\pgfusepath{fill}%
\end{pgfscope}%
\begin{pgfscope}%
\pgfpathrectangle{\pgfqpoint{0.100000in}{0.100000in}}{\pgfqpoint{3.007045in}{1.925000in}}%
\pgfusepath{clip}%
\pgfsetbuttcap%
\pgfsetmiterjoin%
\definecolor{currentfill}{rgb}{0.050211,0.341761,0.630604}%
\pgfsetfillcolor{currentfill}%
\pgfsetlinewidth{0.000000pt}%
\definecolor{currentstroke}{rgb}{0.000000,0.000000,0.000000}%
\pgfsetstrokecolor{currentstroke}%
\pgfsetstrokeopacity{0.000000}%
\pgfsetdash{}{0pt}%
\pgfpathmoveto{\pgfqpoint{1.441506in}{0.486973in}}%
\pgfpathlineto{\pgfqpoint{1.473940in}{0.485183in}}%
\pgfpathlineto{\pgfqpoint{1.470461in}{0.425819in}}%
\pgfpathlineto{\pgfqpoint{1.464650in}{0.426083in}}%
\pgfpathlineto{\pgfqpoint{1.460228in}{0.431872in}}%
\pgfpathlineto{\pgfqpoint{1.458257in}{0.442114in}}%
\pgfpathlineto{\pgfqpoint{1.454336in}{0.450453in}}%
\pgfpathlineto{\pgfqpoint{1.455112in}{0.452787in}}%
\pgfpathlineto{\pgfqpoint{1.449609in}{0.458112in}}%
\pgfpathlineto{\pgfqpoint{1.447489in}{0.469258in}}%
\pgfpathlineto{\pgfqpoint{1.441856in}{0.477516in}}%
\pgfpathlineto{\pgfqpoint{1.441506in}{0.486973in}}%
\pgfpathclose%
\pgfusepath{fill}%
\end{pgfscope}%
\begin{pgfscope}%
\pgfpathrectangle{\pgfqpoint{0.100000in}{0.100000in}}{\pgfqpoint{3.007045in}{1.925000in}}%
\pgfusepath{clip}%
\pgfsetbuttcap%
\pgfsetmiterjoin%
\definecolor{currentfill}{rgb}{0.371688,0.649627,0.820515}%
\pgfsetfillcolor{currentfill}%
\pgfsetlinewidth{0.000000pt}%
\definecolor{currentstroke}{rgb}{0.000000,0.000000,0.000000}%
\pgfsetstrokecolor{currentstroke}%
\pgfsetstrokeopacity{0.000000}%
\pgfsetdash{}{0pt}%
\pgfpathmoveto{\pgfqpoint{1.775875in}{0.660550in}}%
\pgfpathlineto{\pgfqpoint{1.773764in}{0.656330in}}%
\pgfpathlineto{\pgfqpoint{1.776581in}{0.643123in}}%
\pgfpathlineto{\pgfqpoint{1.779925in}{0.639399in}}%
\pgfpathlineto{\pgfqpoint{1.772144in}{0.632693in}}%
\pgfpathlineto{\pgfqpoint{1.766321in}{0.636474in}}%
\pgfpathlineto{\pgfqpoint{1.763851in}{0.642226in}}%
\pgfpathlineto{\pgfqpoint{1.756860in}{0.643794in}}%
\pgfpathlineto{\pgfqpoint{1.735433in}{0.640512in}}%
\pgfpathlineto{\pgfqpoint{1.730423in}{0.638033in}}%
\pgfpathlineto{\pgfqpoint{1.732103in}{0.644712in}}%
\pgfpathlineto{\pgfqpoint{1.723459in}{0.650585in}}%
\pgfpathlineto{\pgfqpoint{1.722872in}{0.654828in}}%
\pgfpathlineto{\pgfqpoint{1.718592in}{0.656205in}}%
\pgfpathlineto{\pgfqpoint{1.712375in}{0.673683in}}%
\pgfpathlineto{\pgfqpoint{1.703043in}{0.687922in}}%
\pgfpathlineto{\pgfqpoint{1.694814in}{0.693189in}}%
\pgfpathlineto{\pgfqpoint{1.712066in}{0.694873in}}%
\pgfpathlineto{\pgfqpoint{1.712393in}{0.726751in}}%
\pgfpathlineto{\pgfqpoint{1.720316in}{0.726579in}}%
\pgfpathlineto{\pgfqpoint{1.725321in}{0.722021in}}%
\pgfpathlineto{\pgfqpoint{1.728058in}{0.722452in}}%
\pgfpathlineto{\pgfqpoint{1.736915in}{0.718690in}}%
\pgfpathlineto{\pgfqpoint{1.735350in}{0.734610in}}%
\pgfpathlineto{\pgfqpoint{1.755158in}{0.734645in}}%
\pgfpathlineto{\pgfqpoint{1.762705in}{0.734578in}}%
\pgfpathlineto{\pgfqpoint{1.763868in}{0.730641in}}%
\pgfpathlineto{\pgfqpoint{1.763758in}{0.708620in}}%
\pgfpathlineto{\pgfqpoint{1.773020in}{0.706433in}}%
\pgfpathlineto{\pgfqpoint{1.773171in}{0.660592in}}%
\pgfpathlineto{\pgfqpoint{1.775875in}{0.660550in}}%
\pgfpathclose%
\pgfusepath{fill}%
\end{pgfscope}%
\begin{pgfscope}%
\pgfpathrectangle{\pgfqpoint{0.100000in}{0.100000in}}{\pgfqpoint{3.007045in}{1.925000in}}%
\pgfusepath{clip}%
\pgfsetbuttcap%
\pgfsetmiterjoin%
\definecolor{currentfill}{rgb}{0.361599,0.642737,0.816578}%
\pgfsetfillcolor{currentfill}%
\pgfsetlinewidth{0.000000pt}%
\definecolor{currentstroke}{rgb}{0.000000,0.000000,0.000000}%
\pgfsetstrokecolor{currentstroke}%
\pgfsetstrokeopacity{0.000000}%
\pgfsetdash{}{0pt}%
\pgfpathmoveto{\pgfqpoint{0.407690in}{1.233195in}}%
\pgfpathlineto{\pgfqpoint{0.404604in}{1.238315in}}%
\pgfpathlineto{\pgfqpoint{0.409870in}{1.255826in}}%
\pgfpathlineto{\pgfqpoint{0.409743in}{1.259767in}}%
\pgfpathlineto{\pgfqpoint{0.413844in}{1.273461in}}%
\pgfpathlineto{\pgfqpoint{0.409876in}{1.274771in}}%
\pgfpathlineto{\pgfqpoint{0.409917in}{1.279165in}}%
\pgfpathlineto{\pgfqpoint{0.414282in}{1.280552in}}%
\pgfpathlineto{\pgfqpoint{0.417431in}{1.287454in}}%
\pgfpathlineto{\pgfqpoint{0.412425in}{1.289011in}}%
\pgfpathlineto{\pgfqpoint{0.416471in}{1.302466in}}%
\pgfpathlineto{\pgfqpoint{0.404388in}{1.306594in}}%
\pgfpathlineto{\pgfqpoint{0.400092in}{1.304827in}}%
\pgfpathlineto{\pgfqpoint{0.396070in}{1.307741in}}%
\pgfpathlineto{\pgfqpoint{0.395623in}{1.317666in}}%
\pgfpathlineto{\pgfqpoint{0.393792in}{1.321038in}}%
\pgfpathlineto{\pgfqpoint{0.393337in}{1.331101in}}%
\pgfpathlineto{\pgfqpoint{0.388567in}{1.334258in}}%
\pgfpathlineto{\pgfqpoint{0.392426in}{1.337198in}}%
\pgfpathlineto{\pgfqpoint{0.417135in}{1.329622in}}%
\pgfpathlineto{\pgfqpoint{0.421047in}{1.323928in}}%
\pgfpathlineto{\pgfqpoint{0.420345in}{1.320339in}}%
\pgfpathlineto{\pgfqpoint{0.424973in}{1.314267in}}%
\pgfpathlineto{\pgfqpoint{0.431828in}{1.313269in}}%
\pgfpathlineto{\pgfqpoint{0.435140in}{1.324290in}}%
\pgfpathlineto{\pgfqpoint{0.439260in}{1.328328in}}%
\pgfpathlineto{\pgfqpoint{0.445384in}{1.329930in}}%
\pgfpathlineto{\pgfqpoint{0.450179in}{1.340731in}}%
\pgfpathlineto{\pgfqpoint{0.460387in}{1.347970in}}%
\pgfpathlineto{\pgfqpoint{0.465761in}{1.347164in}}%
\pgfpathlineto{\pgfqpoint{0.480726in}{1.347606in}}%
\pgfpathlineto{\pgfqpoint{0.486181in}{1.350484in}}%
\pgfpathlineto{\pgfqpoint{0.489895in}{1.349175in}}%
\pgfpathlineto{\pgfqpoint{0.492015in}{1.343273in}}%
\pgfpathlineto{\pgfqpoint{0.516565in}{1.336204in}}%
\pgfpathlineto{\pgfqpoint{0.510601in}{1.315136in}}%
\pgfpathlineto{\pgfqpoint{0.522542in}{1.311873in}}%
\pgfpathlineto{\pgfqpoint{0.522474in}{1.309901in}}%
\pgfpathlineto{\pgfqpoint{0.544151in}{1.303815in}}%
\pgfpathlineto{\pgfqpoint{0.542583in}{1.297900in}}%
\pgfpathlineto{\pgfqpoint{0.536518in}{1.297107in}}%
\pgfpathlineto{\pgfqpoint{0.533368in}{1.282853in}}%
\pgfpathlineto{\pgfqpoint{0.536090in}{1.281820in}}%
\pgfpathlineto{\pgfqpoint{0.533766in}{1.269179in}}%
\pgfpathlineto{\pgfqpoint{0.524302in}{1.283464in}}%
\pgfpathlineto{\pgfqpoint{0.520542in}{1.277526in}}%
\pgfpathlineto{\pgfqpoint{0.522497in}{1.269886in}}%
\pgfpathlineto{\pgfqpoint{0.519485in}{1.264180in}}%
\pgfpathlineto{\pgfqpoint{0.514663in}{1.260246in}}%
\pgfpathlineto{\pgfqpoint{0.510697in}{1.267522in}}%
\pgfpathlineto{\pgfqpoint{0.502964in}{1.265519in}}%
\pgfpathlineto{\pgfqpoint{0.497623in}{1.272306in}}%
\pgfpathlineto{\pgfqpoint{0.488781in}{1.270732in}}%
\pgfpathlineto{\pgfqpoint{0.475080in}{1.259356in}}%
\pgfpathlineto{\pgfqpoint{0.464303in}{1.248933in}}%
\pgfpathlineto{\pgfqpoint{0.455151in}{1.243275in}}%
\pgfpathlineto{\pgfqpoint{0.446079in}{1.262743in}}%
\pgfpathlineto{\pgfqpoint{0.440085in}{1.241248in}}%
\pgfpathlineto{\pgfqpoint{0.436885in}{1.243705in}}%
\pgfpathlineto{\pgfqpoint{0.424991in}{1.242670in}}%
\pgfpathlineto{\pgfqpoint{0.407690in}{1.233195in}}%
\pgfpathclose%
\pgfusepath{fill}%
\end{pgfscope}%
\begin{pgfscope}%
\pgfpathrectangle{\pgfqpoint{0.100000in}{0.100000in}}{\pgfqpoint{3.007045in}{1.925000in}}%
\pgfusepath{clip}%
\pgfsetbuttcap%
\pgfsetmiterjoin%
\definecolor{currentfill}{rgb}{0.296025,0.597955,0.790988}%
\pgfsetfillcolor{currentfill}%
\pgfsetlinewidth{0.000000pt}%
\definecolor{currentstroke}{rgb}{0.000000,0.000000,0.000000}%
\pgfsetstrokecolor{currentstroke}%
\pgfsetstrokeopacity{0.000000}%
\pgfsetdash{}{0pt}%
\pgfpathmoveto{\pgfqpoint{1.588665in}{1.618076in}}%
\pgfpathlineto{\pgfqpoint{1.587099in}{1.618157in}}%
\pgfpathlineto{\pgfqpoint{1.588163in}{1.641202in}}%
\pgfpathlineto{\pgfqpoint{1.569331in}{1.642166in}}%
\pgfpathlineto{\pgfqpoint{1.570614in}{1.665351in}}%
\pgfpathlineto{\pgfqpoint{1.569042in}{1.665415in}}%
\pgfpathlineto{\pgfqpoint{1.570193in}{1.688446in}}%
\pgfpathlineto{\pgfqpoint{1.579862in}{1.687980in}}%
\pgfpathlineto{\pgfqpoint{1.614445in}{1.686458in}}%
\pgfpathlineto{\pgfqpoint{1.615660in}{1.680679in}}%
\pgfpathlineto{\pgfqpoint{1.638529in}{1.679874in}}%
\pgfpathlineto{\pgfqpoint{1.650048in}{1.679481in}}%
\pgfpathlineto{\pgfqpoint{1.650635in}{1.662065in}}%
\pgfpathlineto{\pgfqpoint{1.649938in}{1.638890in}}%
\pgfpathlineto{\pgfqpoint{1.633892in}{1.639443in}}%
\pgfpathlineto{\pgfqpoint{1.633089in}{1.616328in}}%
\pgfpathlineto{\pgfqpoint{1.588665in}{1.618076in}}%
\pgfpathclose%
\pgfusepath{fill}%
\end{pgfscope}%
\begin{pgfscope}%
\pgfpathrectangle{\pgfqpoint{0.100000in}{0.100000in}}{\pgfqpoint{3.007045in}{1.925000in}}%
\pgfusepath{clip}%
\pgfsetbuttcap%
\pgfsetmiterjoin%
\definecolor{currentfill}{rgb}{0.790496,0.868174,0.941930}%
\pgfsetfillcolor{currentfill}%
\pgfsetlinewidth{0.000000pt}%
\definecolor{currentstroke}{rgb}{0.000000,0.000000,0.000000}%
\pgfsetstrokecolor{currentstroke}%
\pgfsetstrokeopacity{0.000000}%
\pgfsetdash{}{0pt}%
\pgfpathmoveto{\pgfqpoint{2.867151in}{1.544262in}}%
\pgfpathlineto{\pgfqpoint{2.860753in}{1.546240in}}%
\pgfpathlineto{\pgfqpoint{2.850891in}{1.542069in}}%
\pgfpathlineto{\pgfqpoint{2.846989in}{1.546225in}}%
\pgfpathlineto{\pgfqpoint{2.835404in}{1.542568in}}%
\pgfpathlineto{\pgfqpoint{2.828637in}{1.543127in}}%
\pgfpathlineto{\pgfqpoint{2.825651in}{1.545463in}}%
\pgfpathlineto{\pgfqpoint{2.829267in}{1.550294in}}%
\pgfpathlineto{\pgfqpoint{2.829522in}{1.557864in}}%
\pgfpathlineto{\pgfqpoint{2.826130in}{1.558806in}}%
\pgfpathlineto{\pgfqpoint{2.827305in}{1.571893in}}%
\pgfpathlineto{\pgfqpoint{2.821240in}{1.572694in}}%
\pgfpathlineto{\pgfqpoint{2.821564in}{1.578723in}}%
\pgfpathlineto{\pgfqpoint{2.811802in}{1.580671in}}%
\pgfpathlineto{\pgfqpoint{2.815683in}{1.583923in}}%
\pgfpathlineto{\pgfqpoint{2.819866in}{1.593654in}}%
\pgfpathlineto{\pgfqpoint{2.811115in}{1.594372in}}%
\pgfpathlineto{\pgfqpoint{2.808838in}{1.600051in}}%
\pgfpathlineto{\pgfqpoint{2.808017in}{1.607896in}}%
\pgfpathlineto{\pgfqpoint{2.810189in}{1.620019in}}%
\pgfpathlineto{\pgfqpoint{2.817950in}{1.618314in}}%
\pgfpathlineto{\pgfqpoint{2.819281in}{1.624179in}}%
\pgfpathlineto{\pgfqpoint{2.825014in}{1.621968in}}%
\pgfpathlineto{\pgfqpoint{2.826283in}{1.627777in}}%
\pgfpathlineto{\pgfqpoint{2.837240in}{1.625008in}}%
\pgfpathlineto{\pgfqpoint{2.834224in}{1.609178in}}%
\pgfpathlineto{\pgfqpoint{2.837477in}{1.608579in}}%
\pgfpathlineto{\pgfqpoint{2.848338in}{1.611817in}}%
\pgfpathlineto{\pgfqpoint{2.846880in}{1.617175in}}%
\pgfpathlineto{\pgfqpoint{2.849656in}{1.622164in}}%
\pgfpathlineto{\pgfqpoint{2.854814in}{1.623529in}}%
\pgfpathlineto{\pgfqpoint{2.858546in}{1.629287in}}%
\pgfpathlineto{\pgfqpoint{2.862905in}{1.625347in}}%
\pgfpathlineto{\pgfqpoint{2.868383in}{1.624903in}}%
\pgfpathlineto{\pgfqpoint{2.869747in}{1.621836in}}%
\pgfpathlineto{\pgfqpoint{2.875904in}{1.623095in}}%
\pgfpathlineto{\pgfqpoint{2.877796in}{1.621050in}}%
\pgfpathlineto{\pgfqpoint{2.882477in}{1.612503in}}%
\pgfpathlineto{\pgfqpoint{2.885308in}{1.602759in}}%
\pgfpathlineto{\pgfqpoint{2.876355in}{1.598852in}}%
\pgfpathlineto{\pgfqpoint{2.881617in}{1.586847in}}%
\pgfpathlineto{\pgfqpoint{2.875977in}{1.584608in}}%
\pgfpathlineto{\pgfqpoint{2.876946in}{1.580426in}}%
\pgfpathlineto{\pgfqpoint{2.870513in}{1.573950in}}%
\pgfpathlineto{\pgfqpoint{2.867917in}{1.577405in}}%
\pgfpathlineto{\pgfqpoint{2.863308in}{1.564293in}}%
\pgfpathlineto{\pgfqpoint{2.861539in}{1.556030in}}%
\pgfpathlineto{\pgfqpoint{2.867535in}{1.548390in}}%
\pgfpathlineto{\pgfqpoint{2.867151in}{1.544262in}}%
\pgfpathclose%
\pgfusepath{fill}%
\end{pgfscope}%
\begin{pgfscope}%
\pgfpathrectangle{\pgfqpoint{0.100000in}{0.100000in}}{\pgfqpoint{3.007045in}{1.925000in}}%
\pgfusepath{clip}%
\pgfsetbuttcap%
\pgfsetmiterjoin%
\definecolor{currentfill}{rgb}{0.554510,0.756417,0.868312}%
\pgfsetfillcolor{currentfill}%
\pgfsetlinewidth{0.000000pt}%
\definecolor{currentstroke}{rgb}{0.000000,0.000000,0.000000}%
\pgfsetstrokecolor{currentstroke}%
\pgfsetstrokeopacity{0.000000}%
\pgfsetdash{}{0pt}%
\pgfpathmoveto{\pgfqpoint{2.552895in}{1.336088in}}%
\pgfpathlineto{\pgfqpoint{2.550029in}{1.332847in}}%
\pgfpathlineto{\pgfqpoint{2.540085in}{1.329055in}}%
\pgfpathlineto{\pgfqpoint{2.533624in}{1.328350in}}%
\pgfpathlineto{\pgfqpoint{2.528039in}{1.339978in}}%
\pgfpathlineto{\pgfqpoint{2.513140in}{1.337398in}}%
\pgfpathlineto{\pgfqpoint{2.509694in}{1.358087in}}%
\pgfpathlineto{\pgfqpoint{2.503675in}{1.356445in}}%
\pgfpathlineto{\pgfqpoint{2.484276in}{1.354588in}}%
\pgfpathlineto{\pgfqpoint{2.479266in}{1.385678in}}%
\pgfpathlineto{\pgfqpoint{2.494271in}{1.396072in}}%
\pgfpathlineto{\pgfqpoint{2.507781in}{1.407380in}}%
\pgfpathlineto{\pgfqpoint{2.516225in}{1.413233in}}%
\pgfpathlineto{\pgfqpoint{2.526196in}{1.422773in}}%
\pgfpathlineto{\pgfqpoint{2.534317in}{1.431974in}}%
\pgfpathlineto{\pgfqpoint{2.539326in}{1.435418in}}%
\pgfpathlineto{\pgfqpoint{2.543334in}{1.433990in}}%
\pgfpathlineto{\pgfqpoint{2.549510in}{1.399000in}}%
\pgfpathlineto{\pgfqpoint{2.556395in}{1.400149in}}%
\pgfpathlineto{\pgfqpoint{2.558369in}{1.389969in}}%
\pgfpathlineto{\pgfqpoint{2.556898in}{1.389007in}}%
\pgfpathlineto{\pgfqpoint{2.558934in}{1.375533in}}%
\pgfpathlineto{\pgfqpoint{2.559925in}{1.364184in}}%
\pgfpathlineto{\pgfqpoint{2.554563in}{1.361326in}}%
\pgfpathlineto{\pgfqpoint{2.555437in}{1.355961in}}%
\pgfpathlineto{\pgfqpoint{2.549986in}{1.354398in}}%
\pgfpathlineto{\pgfqpoint{2.552895in}{1.336088in}}%
\pgfpathclose%
\pgfusepath{fill}%
\end{pgfscope}%
\begin{pgfscope}%
\pgfpathrectangle{\pgfqpoint{0.100000in}{0.100000in}}{\pgfqpoint{3.007045in}{1.925000in}}%
\pgfusepath{clip}%
\pgfsetbuttcap%
\pgfsetmiterjoin%
\definecolor{currentfill}{rgb}{0.187266,0.500992,0.739608}%
\pgfsetfillcolor{currentfill}%
\pgfsetlinewidth{0.000000pt}%
\definecolor{currentstroke}{rgb}{0.000000,0.000000,0.000000}%
\pgfsetstrokecolor{currentstroke}%
\pgfsetstrokeopacity{0.000000}%
\pgfsetdash{}{0pt}%
\pgfpathmoveto{\pgfqpoint{1.677475in}{1.268823in}}%
\pgfpathlineto{\pgfqpoint{1.654636in}{1.269418in}}%
\pgfpathlineto{\pgfqpoint{1.655235in}{1.292354in}}%
\pgfpathlineto{\pgfqpoint{1.643733in}{1.288581in}}%
\pgfpathlineto{\pgfqpoint{1.643851in}{1.292672in}}%
\pgfpathlineto{\pgfqpoint{1.638646in}{1.292816in}}%
\pgfpathlineto{\pgfqpoint{1.638392in}{1.287167in}}%
\pgfpathlineto{\pgfqpoint{1.621155in}{1.286648in}}%
\pgfpathlineto{\pgfqpoint{1.618684in}{1.284817in}}%
\pgfpathlineto{\pgfqpoint{1.610083in}{1.285141in}}%
\pgfpathlineto{\pgfqpoint{1.609594in}{1.293697in}}%
\pgfpathlineto{\pgfqpoint{1.610215in}{1.316599in}}%
\pgfpathlineto{\pgfqpoint{1.610638in}{1.328045in}}%
\pgfpathlineto{\pgfqpoint{1.633318in}{1.327373in}}%
\pgfpathlineto{\pgfqpoint{1.633082in}{1.315887in}}%
\pgfpathlineto{\pgfqpoint{1.672961in}{1.314877in}}%
\pgfpathlineto{\pgfqpoint{1.678558in}{1.314735in}}%
\pgfpathlineto{\pgfqpoint{1.677475in}{1.268823in}}%
\pgfpathclose%
\pgfusepath{fill}%
\end{pgfscope}%
\begin{pgfscope}%
\pgfpathrectangle{\pgfqpoint{0.100000in}{0.100000in}}{\pgfqpoint{3.007045in}{1.925000in}}%
\pgfusepath{clip}%
\pgfsetbuttcap%
\pgfsetmiterjoin%
\definecolor{currentfill}{rgb}{0.491765,0.721968,0.854779}%
\pgfsetfillcolor{currentfill}%
\pgfsetlinewidth{0.000000pt}%
\definecolor{currentstroke}{rgb}{0.000000,0.000000,0.000000}%
\pgfsetstrokecolor{currentstroke}%
\pgfsetstrokeopacity{0.000000}%
\pgfsetdash{}{0pt}%
\pgfpathmoveto{\pgfqpoint{1.779266in}{1.452736in}}%
\pgfpathlineto{\pgfqpoint{1.778980in}{1.469921in}}%
\pgfpathlineto{\pgfqpoint{1.767569in}{1.469929in}}%
\pgfpathlineto{\pgfqpoint{1.767169in}{1.475679in}}%
\pgfpathlineto{\pgfqpoint{1.767240in}{1.481415in}}%
\pgfpathlineto{\pgfqpoint{1.778661in}{1.481448in}}%
\pgfpathlineto{\pgfqpoint{1.778660in}{1.495730in}}%
\pgfpathlineto{\pgfqpoint{1.782199in}{1.492987in}}%
\pgfpathlineto{\pgfqpoint{1.790073in}{1.492987in}}%
\pgfpathlineto{\pgfqpoint{1.822887in}{1.493279in}}%
\pgfpathlineto{\pgfqpoint{1.823694in}{1.499014in}}%
\pgfpathlineto{\pgfqpoint{1.841921in}{1.499291in}}%
\pgfpathlineto{\pgfqpoint{1.842251in}{1.476305in}}%
\pgfpathlineto{\pgfqpoint{1.830743in}{1.476124in}}%
\pgfpathlineto{\pgfqpoint{1.831065in}{1.453126in}}%
\pgfpathlineto{\pgfqpoint{1.779266in}{1.452736in}}%
\pgfpathclose%
\pgfusepath{fill}%
\end{pgfscope}%
\begin{pgfscope}%
\pgfpathrectangle{\pgfqpoint{0.100000in}{0.100000in}}{\pgfqpoint{3.007045in}{1.925000in}}%
\pgfusepath{clip}%
\pgfsetbuttcap%
\pgfsetmiterjoin%
\definecolor{currentfill}{rgb}{0.356555,0.639293,0.814610}%
\pgfsetfillcolor{currentfill}%
\pgfsetlinewidth{0.000000pt}%
\definecolor{currentstroke}{rgb}{0.000000,0.000000,0.000000}%
\pgfsetstrokecolor{currentstroke}%
\pgfsetstrokeopacity{0.000000}%
\pgfsetdash{}{0pt}%
\pgfpathmoveto{\pgfqpoint{1.890498in}{1.239370in}}%
\pgfpathlineto{\pgfqpoint{1.889907in}{1.259660in}}%
\pgfpathlineto{\pgfqpoint{1.867080in}{1.259096in}}%
\pgfpathlineto{\pgfqpoint{1.866638in}{1.276378in}}%
\pgfpathlineto{\pgfqpoint{1.855298in}{1.276147in}}%
\pgfpathlineto{\pgfqpoint{1.854836in}{1.299029in}}%
\pgfpathlineto{\pgfqpoint{1.905724in}{1.300373in}}%
\pgfpathlineto{\pgfqpoint{1.923061in}{1.300995in}}%
\pgfpathlineto{\pgfqpoint{1.923843in}{1.278117in}}%
\pgfpathlineto{\pgfqpoint{1.935244in}{1.278471in}}%
\pgfpathlineto{\pgfqpoint{1.935891in}{1.242279in}}%
\pgfpathlineto{\pgfqpoint{1.925211in}{1.241392in}}%
\pgfpathlineto{\pgfqpoint{1.890498in}{1.239370in}}%
\pgfpathclose%
\pgfusepath{fill}%
\end{pgfscope}%
\begin{pgfscope}%
\pgfpathrectangle{\pgfqpoint{0.100000in}{0.100000in}}{\pgfqpoint{3.007045in}{1.925000in}}%
\pgfusepath{clip}%
\pgfsetbuttcap%
\pgfsetmiterjoin%
\definecolor{currentfill}{rgb}{0.037908,0.326013,0.618301}%
\pgfsetfillcolor{currentfill}%
\pgfsetlinewidth{0.000000pt}%
\definecolor{currentstroke}{rgb}{0.000000,0.000000,0.000000}%
\pgfsetstrokecolor{currentstroke}%
\pgfsetstrokeopacity{0.000000}%
\pgfsetdash{}{0pt}%
\pgfpathmoveto{\pgfqpoint{0.967728in}{1.403968in}}%
\pgfpathlineto{\pgfqpoint{0.968744in}{1.400234in}}%
\pgfpathlineto{\pgfqpoint{0.965948in}{1.394443in}}%
\pgfpathlineto{\pgfqpoint{0.967550in}{1.392277in}}%
\pgfpathlineto{\pgfqpoint{0.968734in}{1.380626in}}%
\pgfpathlineto{\pgfqpoint{0.965270in}{1.373029in}}%
\pgfpathlineto{\pgfqpoint{0.960688in}{1.366618in}}%
\pgfpathlineto{\pgfqpoint{0.951603in}{1.367889in}}%
\pgfpathlineto{\pgfqpoint{0.947822in}{1.365378in}}%
\pgfpathlineto{\pgfqpoint{0.942058in}{1.369750in}}%
\pgfpathlineto{\pgfqpoint{0.934476in}{1.365698in}}%
\pgfpathlineto{\pgfqpoint{0.924294in}{1.367661in}}%
\pgfpathlineto{\pgfqpoint{0.908601in}{1.353242in}}%
\pgfpathlineto{\pgfqpoint{0.892707in}{1.351230in}}%
\pgfpathlineto{\pgfqpoint{0.832155in}{1.363654in}}%
\pgfpathlineto{\pgfqpoint{0.845804in}{1.427734in}}%
\pgfpathlineto{\pgfqpoint{0.853680in}{1.425681in}}%
\pgfpathlineto{\pgfqpoint{0.895826in}{1.417699in}}%
\pgfpathlineto{\pgfqpoint{0.900100in}{1.438993in}}%
\pgfpathlineto{\pgfqpoint{0.916709in}{1.435786in}}%
\pgfpathlineto{\pgfqpoint{0.917819in}{1.441425in}}%
\pgfpathlineto{\pgfqpoint{0.925285in}{1.439963in}}%
\pgfpathlineto{\pgfqpoint{0.926367in}{1.445605in}}%
\pgfpathlineto{\pgfqpoint{0.931789in}{1.444541in}}%
\pgfpathlineto{\pgfqpoint{0.934955in}{1.436848in}}%
\pgfpathlineto{\pgfqpoint{0.941439in}{1.428069in}}%
\pgfpathlineto{\pgfqpoint{0.948249in}{1.426793in}}%
\pgfpathlineto{\pgfqpoint{0.952072in}{1.424129in}}%
\pgfpathlineto{\pgfqpoint{0.955057in}{1.434288in}}%
\pgfpathlineto{\pgfqpoint{0.968314in}{1.431784in}}%
\pgfpathlineto{\pgfqpoint{0.969777in}{1.427234in}}%
\pgfpathlineto{\pgfqpoint{0.968279in}{1.422237in}}%
\pgfpathlineto{\pgfqpoint{0.964506in}{1.418586in}}%
\pgfpathlineto{\pgfqpoint{0.964652in}{1.410841in}}%
\pgfpathlineto{\pgfqpoint{0.967940in}{1.408953in}}%
\pgfpathlineto{\pgfqpoint{0.967728in}{1.403968in}}%
\pgfpathclose%
\pgfusepath{fill}%
\end{pgfscope}%
\begin{pgfscope}%
\pgfpathrectangle{\pgfqpoint{0.100000in}{0.100000in}}{\pgfqpoint{3.007045in}{1.925000in}}%
\pgfusepath{clip}%
\pgfsetbuttcap%
\pgfsetmiterjoin%
\definecolor{currentfill}{rgb}{0.381776,0.656517,0.824452}%
\pgfsetfillcolor{currentfill}%
\pgfsetlinewidth{0.000000pt}%
\definecolor{currentstroke}{rgb}{0.000000,0.000000,0.000000}%
\pgfsetstrokecolor{currentstroke}%
\pgfsetstrokeopacity{0.000000}%
\pgfsetdash{}{0pt}%
\pgfpathmoveto{\pgfqpoint{2.359204in}{1.116262in}}%
\pgfpathlineto{\pgfqpoint{2.353384in}{1.121802in}}%
\pgfpathlineto{\pgfqpoint{2.347373in}{1.122044in}}%
\pgfpathlineto{\pgfqpoint{2.339247in}{1.131363in}}%
\pgfpathlineto{\pgfqpoint{2.341607in}{1.134315in}}%
\pgfpathlineto{\pgfqpoint{2.338933in}{1.139975in}}%
\pgfpathlineto{\pgfqpoint{2.332965in}{1.140303in}}%
\pgfpathlineto{\pgfqpoint{2.326824in}{1.138484in}}%
\pgfpathlineto{\pgfqpoint{2.323384in}{1.154002in}}%
\pgfpathlineto{\pgfqpoint{2.331133in}{1.151744in}}%
\pgfpathlineto{\pgfqpoint{2.336776in}{1.153283in}}%
\pgfpathlineto{\pgfqpoint{2.342358in}{1.152328in}}%
\pgfpathlineto{\pgfqpoint{2.348520in}{1.145975in}}%
\pgfpathlineto{\pgfqpoint{2.354011in}{1.144847in}}%
\pgfpathlineto{\pgfqpoint{2.356171in}{1.148869in}}%
\pgfpathlineto{\pgfqpoint{2.360800in}{1.150692in}}%
\pgfpathlineto{\pgfqpoint{2.374431in}{1.146809in}}%
\pgfpathlineto{\pgfqpoint{2.380064in}{1.147699in}}%
\pgfpathlineto{\pgfqpoint{2.385587in}{1.155405in}}%
\pgfpathlineto{\pgfqpoint{2.384672in}{1.146612in}}%
\pgfpathlineto{\pgfqpoint{2.379735in}{1.139390in}}%
\pgfpathlineto{\pgfqpoint{2.377693in}{1.133822in}}%
\pgfpathlineto{\pgfqpoint{2.378692in}{1.128851in}}%
\pgfpathlineto{\pgfqpoint{2.372393in}{1.126861in}}%
\pgfpathlineto{\pgfqpoint{2.368680in}{1.131320in}}%
\pgfpathlineto{\pgfqpoint{2.364836in}{1.127691in}}%
\pgfpathlineto{\pgfqpoint{2.365037in}{1.122682in}}%
\pgfpathlineto{\pgfqpoint{2.359204in}{1.116262in}}%
\pgfpathclose%
\pgfusepath{fill}%
\end{pgfscope}%
\begin{pgfscope}%
\pgfpathrectangle{\pgfqpoint{0.100000in}{0.100000in}}{\pgfqpoint{3.007045in}{1.925000in}}%
\pgfusepath{clip}%
\pgfsetbuttcap%
\pgfsetmiterjoin%
\definecolor{currentfill}{rgb}{0.331334,0.622068,0.804767}%
\pgfsetfillcolor{currentfill}%
\pgfsetlinewidth{0.000000pt}%
\definecolor{currentstroke}{rgb}{0.000000,0.000000,0.000000}%
\pgfsetstrokecolor{currentstroke}%
\pgfsetstrokeopacity{0.000000}%
\pgfsetdash{}{0pt}%
\pgfpathmoveto{\pgfqpoint{2.382677in}{0.742228in}}%
\pgfpathlineto{\pgfqpoint{2.378977in}{0.740714in}}%
\pgfpathlineto{\pgfqpoint{2.375605in}{0.733287in}}%
\pgfpathlineto{\pgfqpoint{2.368826in}{0.730586in}}%
\pgfpathlineto{\pgfqpoint{2.363559in}{0.734763in}}%
\pgfpathlineto{\pgfqpoint{2.360665in}{0.742287in}}%
\pgfpathlineto{\pgfqpoint{2.367977in}{0.755519in}}%
\pgfpathlineto{\pgfqpoint{2.373071in}{0.752288in}}%
\pgfpathlineto{\pgfqpoint{2.375145in}{0.759957in}}%
\pgfpathlineto{\pgfqpoint{2.375546in}{0.768744in}}%
\pgfpathlineto{\pgfqpoint{2.379793in}{0.770306in}}%
\pgfpathlineto{\pgfqpoint{2.377909in}{0.787069in}}%
\pgfpathlineto{\pgfqpoint{2.387448in}{0.788188in}}%
\pgfpathlineto{\pgfqpoint{2.390153in}{0.787078in}}%
\pgfpathlineto{\pgfqpoint{2.390888in}{0.783936in}}%
\pgfpathlineto{\pgfqpoint{2.405437in}{0.788426in}}%
\pgfpathlineto{\pgfqpoint{2.411719in}{0.790134in}}%
\pgfpathlineto{\pgfqpoint{2.417835in}{0.773607in}}%
\pgfpathlineto{\pgfqpoint{2.415340in}{0.771423in}}%
\pgfpathlineto{\pgfqpoint{2.427990in}{0.752079in}}%
\pgfpathlineto{\pgfqpoint{2.434230in}{0.742118in}}%
\pgfpathlineto{\pgfqpoint{2.432076in}{0.743806in}}%
\pgfpathlineto{\pgfqpoint{2.425722in}{0.735586in}}%
\pgfpathlineto{\pgfqpoint{2.422656in}{0.729020in}}%
\pgfpathlineto{\pgfqpoint{2.427863in}{0.724959in}}%
\pgfpathlineto{\pgfqpoint{2.425917in}{0.719822in}}%
\pgfpathlineto{\pgfqpoint{2.410702in}{0.718714in}}%
\pgfpathlineto{\pgfqpoint{2.409404in}{0.729924in}}%
\pgfpathlineto{\pgfqpoint{2.395922in}{0.728401in}}%
\pgfpathlineto{\pgfqpoint{2.395370in}{0.738234in}}%
\pgfpathlineto{\pgfqpoint{2.392692in}{0.741268in}}%
\pgfpathlineto{\pgfqpoint{2.382677in}{0.742228in}}%
\pgfpathclose%
\pgfusepath{fill}%
\end{pgfscope}%
\begin{pgfscope}%
\pgfpathrectangle{\pgfqpoint{0.100000in}{0.100000in}}{\pgfqpoint{3.007045in}{1.925000in}}%
\pgfusepath{clip}%
\pgfsetbuttcap%
\pgfsetmiterjoin%
\definecolor{currentfill}{rgb}{0.270804,0.580730,0.781146}%
\pgfsetfillcolor{currentfill}%
\pgfsetlinewidth{0.000000pt}%
\definecolor{currentstroke}{rgb}{0.000000,0.000000,0.000000}%
\pgfsetstrokecolor{currentstroke}%
\pgfsetstrokeopacity{0.000000}%
\pgfsetdash{}{0pt}%
\pgfpathmoveto{\pgfqpoint{1.668319in}{1.540277in}}%
\pgfpathlineto{\pgfqpoint{1.633052in}{1.541386in}}%
\pgfpathlineto{\pgfqpoint{1.634891in}{1.593241in}}%
\pgfpathlineto{\pgfqpoint{1.669507in}{1.592201in}}%
\pgfpathlineto{\pgfqpoint{1.700191in}{1.591568in}}%
\pgfpathlineto{\pgfqpoint{1.699144in}{1.583940in}}%
\pgfpathlineto{\pgfqpoint{1.695859in}{1.579141in}}%
\pgfpathlineto{\pgfqpoint{1.687099in}{1.572723in}}%
\pgfpathlineto{\pgfqpoint{1.686157in}{1.569986in}}%
\pgfpathlineto{\pgfqpoint{1.694140in}{1.556870in}}%
\pgfpathlineto{\pgfqpoint{1.701484in}{1.554430in}}%
\pgfpathlineto{\pgfqpoint{1.703819in}{1.551137in}}%
\pgfpathlineto{\pgfqpoint{1.679482in}{1.551623in}}%
\pgfpathlineto{\pgfqpoint{1.678758in}{1.549649in}}%
\pgfpathlineto{\pgfqpoint{1.668554in}{1.549944in}}%
\pgfpathlineto{\pgfqpoint{1.668319in}{1.540277in}}%
\pgfpathclose%
\pgfusepath{fill}%
\end{pgfscope}%
\begin{pgfscope}%
\pgfpathrectangle{\pgfqpoint{0.100000in}{0.100000in}}{\pgfqpoint{3.007045in}{1.925000in}}%
\pgfusepath{clip}%
\pgfsetbuttcap%
\pgfsetmiterjoin%
\definecolor{currentfill}{rgb}{0.585882,0.773641,0.875079}%
\pgfsetfillcolor{currentfill}%
\pgfsetlinewidth{0.000000pt}%
\definecolor{currentstroke}{rgb}{0.000000,0.000000,0.000000}%
\pgfsetstrokecolor{currentstroke}%
\pgfsetstrokeopacity{0.000000}%
\pgfsetdash{}{0pt}%
\pgfpathmoveto{\pgfqpoint{2.493292in}{1.298554in}}%
\pgfpathlineto{\pgfqpoint{2.486521in}{1.293715in}}%
\pgfpathlineto{\pgfqpoint{2.476757in}{1.293301in}}%
\pgfpathlineto{\pgfqpoint{2.473535in}{1.295735in}}%
\pgfpathlineto{\pgfqpoint{2.472733in}{1.301176in}}%
\pgfpathlineto{\pgfqpoint{2.464330in}{1.299934in}}%
\pgfpathlineto{\pgfqpoint{2.461738in}{1.316890in}}%
\pgfpathlineto{\pgfqpoint{2.466018in}{1.317526in}}%
\pgfpathlineto{\pgfqpoint{2.460651in}{1.350974in}}%
\pgfpathlineto{\pgfqpoint{2.484276in}{1.354588in}}%
\pgfpathlineto{\pgfqpoint{2.503675in}{1.356445in}}%
\pgfpathlineto{\pgfqpoint{2.509694in}{1.358087in}}%
\pgfpathlineto{\pgfqpoint{2.513140in}{1.337398in}}%
\pgfpathlineto{\pgfqpoint{2.506874in}{1.324911in}}%
\pgfpathlineto{\pgfqpoint{2.508763in}{1.315516in}}%
\pgfpathlineto{\pgfqpoint{2.491069in}{1.312382in}}%
\pgfpathlineto{\pgfqpoint{2.493292in}{1.298554in}}%
\pgfpathclose%
\pgfusepath{fill}%
\end{pgfscope}%
\begin{pgfscope}%
\pgfpathrectangle{\pgfqpoint{0.100000in}{0.100000in}}{\pgfqpoint{3.007045in}{1.925000in}}%
\pgfusepath{clip}%
\pgfsetbuttcap%
\pgfsetmiterjoin%
\definecolor{currentfill}{rgb}{0.479216,0.715079,0.852072}%
\pgfsetfillcolor{currentfill}%
\pgfsetlinewidth{0.000000pt}%
\definecolor{currentstroke}{rgb}{0.000000,0.000000,0.000000}%
\pgfsetstrokecolor{currentstroke}%
\pgfsetstrokeopacity{0.000000}%
\pgfsetdash{}{0pt}%
\pgfpathmoveto{\pgfqpoint{1.386507in}{1.413753in}}%
\pgfpathlineto{\pgfqpoint{1.335923in}{1.418503in}}%
\pgfpathlineto{\pgfqpoint{1.339144in}{1.451533in}}%
\pgfpathlineto{\pgfqpoint{1.343611in}{1.496079in}}%
\pgfpathlineto{\pgfqpoint{1.348840in}{1.549898in}}%
\pgfpathlineto{\pgfqpoint{1.351084in}{1.564019in}}%
\pgfpathlineto{\pgfqpoint{1.401374in}{1.559284in}}%
\pgfpathlineto{\pgfqpoint{1.400379in}{1.547840in}}%
\pgfpathlineto{\pgfqpoint{1.445034in}{1.544102in}}%
\pgfpathlineto{\pgfqpoint{1.442332in}{1.509325in}}%
\pgfpathlineto{\pgfqpoint{1.439374in}{1.475257in}}%
\pgfpathlineto{\pgfqpoint{1.437357in}{1.456409in}}%
\pgfpathlineto{\pgfqpoint{1.431599in}{1.456389in}}%
\pgfpathlineto{\pgfqpoint{1.429680in}{1.455658in}}%
\pgfpathlineto{\pgfqpoint{1.399572in}{1.458189in}}%
\pgfpathlineto{\pgfqpoint{1.395146in}{1.457695in}}%
\pgfpathlineto{\pgfqpoint{1.390013in}{1.453572in}}%
\pgfpathlineto{\pgfqpoint{1.386507in}{1.413753in}}%
\pgfpathclose%
\pgfusepath{fill}%
\end{pgfscope}%
\begin{pgfscope}%
\pgfpathrectangle{\pgfqpoint{0.100000in}{0.100000in}}{\pgfqpoint{3.007045in}{1.925000in}}%
\pgfusepath{clip}%
\pgfsetbuttcap%
\pgfsetmiterjoin%
\definecolor{currentfill}{rgb}{0.435294,0.690965,0.842599}%
\pgfsetfillcolor{currentfill}%
\pgfsetlinewidth{0.000000pt}%
\definecolor{currentstroke}{rgb}{0.000000,0.000000,0.000000}%
\pgfsetstrokecolor{currentstroke}%
\pgfsetstrokeopacity{0.000000}%
\pgfsetdash{}{0pt}%
\pgfpathmoveto{\pgfqpoint{2.406434in}{1.016470in}}%
\pgfpathlineto{\pgfqpoint{2.399781in}{1.010130in}}%
\pgfpathlineto{\pgfqpoint{2.396925in}{1.003949in}}%
\pgfpathlineto{\pgfqpoint{2.392615in}{1.001420in}}%
\pgfpathlineto{\pgfqpoint{2.388640in}{1.002575in}}%
\pgfpathlineto{\pgfqpoint{2.380497in}{0.997923in}}%
\pgfpathlineto{\pgfqpoint{2.372662in}{0.995532in}}%
\pgfpathlineto{\pgfqpoint{2.361719in}{1.000172in}}%
\pgfpathlineto{\pgfqpoint{2.358542in}{0.999323in}}%
\pgfpathlineto{\pgfqpoint{2.353818in}{1.009495in}}%
\pgfpathlineto{\pgfqpoint{2.355024in}{1.013333in}}%
\pgfpathlineto{\pgfqpoint{2.360603in}{1.018528in}}%
\pgfpathlineto{\pgfqpoint{2.362675in}{1.024583in}}%
\pgfpathlineto{\pgfqpoint{2.376071in}{1.035433in}}%
\pgfpathlineto{\pgfqpoint{2.378235in}{1.029769in}}%
\pgfpathlineto{\pgfqpoint{2.378204in}{1.022084in}}%
\pgfpathlineto{\pgfqpoint{2.380873in}{1.017870in}}%
\pgfpathlineto{\pgfqpoint{2.373385in}{1.015073in}}%
\pgfpathlineto{\pgfqpoint{2.372965in}{1.012422in}}%
\pgfpathlineto{\pgfqpoint{2.406434in}{1.016470in}}%
\pgfpathclose%
\pgfusepath{fill}%
\end{pgfscope}%
\begin{pgfscope}%
\pgfpathrectangle{\pgfqpoint{0.100000in}{0.100000in}}{\pgfqpoint{3.007045in}{1.925000in}}%
\pgfusepath{clip}%
\pgfsetbuttcap%
\pgfsetmiterjoin%
\definecolor{currentfill}{rgb}{0.784591,0.864237,0.939962}%
\pgfsetfillcolor{currentfill}%
\pgfsetlinewidth{0.000000pt}%
\definecolor{currentstroke}{rgb}{0.000000,0.000000,0.000000}%
\pgfsetstrokecolor{currentstroke}%
\pgfsetstrokeopacity{0.000000}%
\pgfsetdash{}{0pt}%
\pgfpathmoveto{\pgfqpoint{2.385967in}{0.896018in}}%
\pgfpathlineto{\pgfqpoint{2.377707in}{0.892532in}}%
\pgfpathlineto{\pgfqpoint{2.375790in}{0.888043in}}%
\pgfpathlineto{\pgfqpoint{2.371189in}{0.888722in}}%
\pgfpathlineto{\pgfqpoint{2.366727in}{0.881594in}}%
\pgfpathlineto{\pgfqpoint{2.360142in}{0.881259in}}%
\pgfpathlineto{\pgfqpoint{2.362404in}{0.891696in}}%
\pgfpathlineto{\pgfqpoint{2.356919in}{0.901553in}}%
\pgfpathlineto{\pgfqpoint{2.348646in}{0.902478in}}%
\pgfpathlineto{\pgfqpoint{2.348504in}{0.918229in}}%
\pgfpathlineto{\pgfqpoint{2.351726in}{0.921503in}}%
\pgfpathlineto{\pgfqpoint{2.356471in}{0.920502in}}%
\pgfpathlineto{\pgfqpoint{2.361956in}{0.924269in}}%
\pgfpathlineto{\pgfqpoint{2.366318in}{0.919795in}}%
\pgfpathlineto{\pgfqpoint{2.372864in}{0.923192in}}%
\pgfpathlineto{\pgfqpoint{2.379773in}{0.923563in}}%
\pgfpathlineto{\pgfqpoint{2.378628in}{0.917255in}}%
\pgfpathlineto{\pgfqpoint{2.383505in}{0.917672in}}%
\pgfpathlineto{\pgfqpoint{2.391763in}{0.908652in}}%
\pgfpathlineto{\pgfqpoint{2.385392in}{0.899513in}}%
\pgfpathlineto{\pgfqpoint{2.385967in}{0.896018in}}%
\pgfpathclose%
\pgfusepath{fill}%
\end{pgfscope}%
\begin{pgfscope}%
\pgfpathrectangle{\pgfqpoint{0.100000in}{0.100000in}}{\pgfqpoint{3.007045in}{1.925000in}}%
\pgfusepath{clip}%
\pgfsetbuttcap%
\pgfsetmiterjoin%
\definecolor{currentfill}{rgb}{0.548235,0.752972,0.866959}%
\pgfsetfillcolor{currentfill}%
\pgfsetlinewidth{0.000000pt}%
\definecolor{currentstroke}{rgb}{0.000000,0.000000,0.000000}%
\pgfsetstrokecolor{currentstroke}%
\pgfsetstrokeopacity{0.000000}%
\pgfsetdash{}{0pt}%
\pgfpathmoveto{\pgfqpoint{2.410399in}{1.309530in}}%
\pgfpathlineto{\pgfqpoint{2.407666in}{1.313899in}}%
\pgfpathlineto{\pgfqpoint{2.407064in}{1.318707in}}%
\pgfpathlineto{\pgfqpoint{2.411263in}{1.323487in}}%
\pgfpathlineto{\pgfqpoint{2.416075in}{1.324178in}}%
\pgfpathlineto{\pgfqpoint{2.415460in}{1.329120in}}%
\pgfpathlineto{\pgfqpoint{2.420098in}{1.329773in}}%
\pgfpathlineto{\pgfqpoint{2.419430in}{1.334702in}}%
\pgfpathlineto{\pgfqpoint{2.414804in}{1.334085in}}%
\pgfpathlineto{\pgfqpoint{2.413516in}{1.344212in}}%
\pgfpathlineto{\pgfqpoint{2.424873in}{1.344692in}}%
\pgfpathlineto{\pgfqpoint{2.430966in}{1.350397in}}%
\pgfpathlineto{\pgfqpoint{2.437543in}{1.358914in}}%
\pgfpathlineto{\pgfqpoint{2.456731in}{1.373827in}}%
\pgfpathlineto{\pgfqpoint{2.461753in}{1.375576in}}%
\pgfpathlineto{\pgfqpoint{2.479266in}{1.385678in}}%
\pgfpathlineto{\pgfqpoint{2.484276in}{1.354588in}}%
\pgfpathlineto{\pgfqpoint{2.460651in}{1.350974in}}%
\pgfpathlineto{\pgfqpoint{2.466018in}{1.317526in}}%
\pgfpathlineto{\pgfqpoint{2.461738in}{1.316890in}}%
\pgfpathlineto{\pgfqpoint{2.445530in}{1.314439in}}%
\pgfpathlineto{\pgfqpoint{2.446146in}{1.309080in}}%
\pgfpathlineto{\pgfqpoint{2.434855in}{1.307908in}}%
\pgfpathlineto{\pgfqpoint{2.434169in}{1.312710in}}%
\pgfpathlineto{\pgfqpoint{2.410399in}{1.309530in}}%
\pgfpathclose%
\pgfusepath{fill}%
\end{pgfscope}%
\begin{pgfscope}%
\pgfpathrectangle{\pgfqpoint{0.100000in}{0.100000in}}{\pgfqpoint{3.007045in}{1.925000in}}%
\pgfusepath{clip}%
\pgfsetbuttcap%
\pgfsetmiterjoin%
\definecolor{currentfill}{rgb}{0.195386,0.509112,0.743791}%
\pgfsetfillcolor{currentfill}%
\pgfsetlinewidth{0.000000pt}%
\definecolor{currentstroke}{rgb}{0.000000,0.000000,0.000000}%
\pgfsetstrokecolor{currentstroke}%
\pgfsetstrokeopacity{0.000000}%
\pgfsetdash{}{0pt}%
\pgfpathmoveto{\pgfqpoint{0.630991in}{1.684466in}}%
\pgfpathlineto{\pgfqpoint{0.625624in}{1.685979in}}%
\pgfpathlineto{\pgfqpoint{0.630102in}{1.696622in}}%
\pgfpathlineto{\pgfqpoint{0.624611in}{1.698303in}}%
\pgfpathlineto{\pgfqpoint{0.626202in}{1.703795in}}%
\pgfpathlineto{\pgfqpoint{0.620630in}{1.705467in}}%
\pgfpathlineto{\pgfqpoint{0.630951in}{1.740790in}}%
\pgfpathlineto{\pgfqpoint{0.637219in}{1.740644in}}%
\pgfpathlineto{\pgfqpoint{0.640972in}{1.753699in}}%
\pgfpathlineto{\pgfqpoint{0.651182in}{1.791391in}}%
\pgfpathlineto{\pgfqpoint{0.662371in}{1.789397in}}%
\pgfpathlineto{\pgfqpoint{0.668689in}{1.793331in}}%
\pgfpathlineto{\pgfqpoint{0.670581in}{1.789487in}}%
\pgfpathlineto{\pgfqpoint{0.675279in}{1.792196in}}%
\pgfpathlineto{\pgfqpoint{0.702590in}{1.784664in}}%
\pgfpathlineto{\pgfqpoint{0.726344in}{1.778628in}}%
\pgfpathlineto{\pgfqpoint{0.724517in}{1.778735in}}%
\pgfpathlineto{\pgfqpoint{0.724939in}{1.766494in}}%
\pgfpathlineto{\pgfqpoint{0.740246in}{1.767188in}}%
\pgfpathlineto{\pgfqpoint{0.737197in}{1.757418in}}%
\pgfpathlineto{\pgfqpoint{0.742787in}{1.756014in}}%
\pgfpathlineto{\pgfqpoint{0.743350in}{1.746702in}}%
\pgfpathlineto{\pgfqpoint{0.746177in}{1.746042in}}%
\pgfpathlineto{\pgfqpoint{0.741011in}{1.724091in}}%
\pgfpathlineto{\pgfqpoint{0.724313in}{1.728496in}}%
\pgfpathlineto{\pgfqpoint{0.722204in}{1.719436in}}%
\pgfpathlineto{\pgfqpoint{0.718256in}{1.717477in}}%
\pgfpathlineto{\pgfqpoint{0.715135in}{1.709465in}}%
\pgfpathlineto{\pgfqpoint{0.712881in}{1.710060in}}%
\pgfpathlineto{\pgfqpoint{0.709253in}{1.696086in}}%
\pgfpathlineto{\pgfqpoint{0.704883in}{1.694237in}}%
\pgfpathlineto{\pgfqpoint{0.694542in}{1.697073in}}%
\pgfpathlineto{\pgfqpoint{0.693304in}{1.692341in}}%
\pgfpathlineto{\pgfqpoint{0.681006in}{1.694917in}}%
\pgfpathlineto{\pgfqpoint{0.680362in}{1.684919in}}%
\pgfpathlineto{\pgfqpoint{0.685286in}{1.683617in}}%
\pgfpathlineto{\pgfqpoint{0.683125in}{1.670416in}}%
\pgfpathlineto{\pgfqpoint{0.630991in}{1.684466in}}%
\pgfpathclose%
\pgfusepath{fill}%
\end{pgfscope}%
\begin{pgfscope}%
\pgfpathrectangle{\pgfqpoint{0.100000in}{0.100000in}}{\pgfqpoint{3.007045in}{1.925000in}}%
\pgfusepath{clip}%
\pgfsetbuttcap%
\pgfsetmiterjoin%
\definecolor{currentfill}{rgb}{0.627605,0.795556,0.885152}%
\pgfsetfillcolor{currentfill}%
\pgfsetlinewidth{0.000000pt}%
\definecolor{currentstroke}{rgb}{0.000000,0.000000,0.000000}%
\pgfsetstrokecolor{currentstroke}%
\pgfsetstrokeopacity{0.000000}%
\pgfsetdash{}{0pt}%
\pgfpathmoveto{\pgfqpoint{1.351564in}{1.062383in}}%
\pgfpathlineto{\pgfqpoint{1.347270in}{1.020215in}}%
\pgfpathlineto{\pgfqpoint{1.298942in}{1.024507in}}%
\pgfpathlineto{\pgfqpoint{1.277959in}{1.026457in}}%
\pgfpathlineto{\pgfqpoint{1.238883in}{1.030797in}}%
\pgfpathlineto{\pgfqpoint{1.241118in}{1.050301in}}%
\pgfpathlineto{\pgfqpoint{1.242720in}{1.057254in}}%
\pgfpathlineto{\pgfqpoint{1.241493in}{1.067690in}}%
\pgfpathlineto{\pgfqpoint{1.236457in}{1.074727in}}%
\pgfpathlineto{\pgfqpoint{1.225618in}{1.073383in}}%
\pgfpathlineto{\pgfqpoint{1.231022in}{1.088493in}}%
\pgfpathlineto{\pgfqpoint{1.229068in}{1.091653in}}%
\pgfpathlineto{\pgfqpoint{1.232103in}{1.090942in}}%
\pgfpathlineto{\pgfqpoint{1.241682in}{1.094643in}}%
\pgfpathlineto{\pgfqpoint{1.245702in}{1.097910in}}%
\pgfpathlineto{\pgfqpoint{1.251074in}{1.090450in}}%
\pgfpathlineto{\pgfqpoint{1.252696in}{1.088003in}}%
\pgfpathlineto{\pgfqpoint{1.271819in}{1.087203in}}%
\pgfpathlineto{\pgfqpoint{1.301086in}{1.073162in}}%
\pgfpathlineto{\pgfqpoint{1.300552in}{1.067248in}}%
\pgfpathlineto{\pgfqpoint{1.351564in}{1.062383in}}%
\pgfpathclose%
\pgfusepath{fill}%
\end{pgfscope}%
\begin{pgfscope}%
\pgfpathrectangle{\pgfqpoint{0.100000in}{0.100000in}}{\pgfqpoint{3.007045in}{1.925000in}}%
\pgfusepath{clip}%
\pgfsetbuttcap%
\pgfsetmiterjoin%
\definecolor{currentfill}{rgb}{0.429020,0.687520,0.841246}%
\pgfsetfillcolor{currentfill}%
\pgfsetlinewidth{0.000000pt}%
\definecolor{currentstroke}{rgb}{0.000000,0.000000,0.000000}%
\pgfsetstrokecolor{currentstroke}%
\pgfsetstrokeopacity{0.000000}%
\pgfsetdash{}{0pt}%
\pgfpathmoveto{\pgfqpoint{1.972796in}{1.051471in}}%
\pgfpathlineto{\pgfqpoint{1.964595in}{1.051309in}}%
\pgfpathlineto{\pgfqpoint{1.964484in}{1.057050in}}%
\pgfpathlineto{\pgfqpoint{1.953009in}{1.056717in}}%
\pgfpathlineto{\pgfqpoint{1.952290in}{1.080650in}}%
\pgfpathlineto{\pgfqpoint{1.946082in}{1.080499in}}%
\pgfpathlineto{\pgfqpoint{1.945763in}{1.089375in}}%
\pgfpathlineto{\pgfqpoint{1.944978in}{1.116719in}}%
\pgfpathlineto{\pgfqpoint{1.949655in}{1.115080in}}%
\pgfpathlineto{\pgfqpoint{1.956351in}{1.117548in}}%
\pgfpathlineto{\pgfqpoint{1.956143in}{1.126474in}}%
\pgfpathlineto{\pgfqpoint{1.963877in}{1.126688in}}%
\pgfpathlineto{\pgfqpoint{1.963320in}{1.146179in}}%
\pgfpathlineto{\pgfqpoint{1.967135in}{1.146275in}}%
\pgfpathlineto{\pgfqpoint{1.967023in}{1.152094in}}%
\pgfpathlineto{\pgfqpoint{1.990535in}{1.152936in}}%
\pgfpathlineto{\pgfqpoint{1.992952in}{1.144098in}}%
\pgfpathlineto{\pgfqpoint{1.991482in}{1.141762in}}%
\pgfpathlineto{\pgfqpoint{1.994783in}{1.133102in}}%
\pgfpathlineto{\pgfqpoint{2.000724in}{1.130408in}}%
\pgfpathlineto{\pgfqpoint{2.005193in}{1.136744in}}%
\pgfpathlineto{\pgfqpoint{2.015419in}{1.134149in}}%
\pgfpathlineto{\pgfqpoint{2.022871in}{1.129850in}}%
\pgfpathlineto{\pgfqpoint{2.018572in}{1.121893in}}%
\pgfpathlineto{\pgfqpoint{2.020049in}{1.117235in}}%
\pgfpathlineto{\pgfqpoint{2.044424in}{1.118250in}}%
\pgfpathlineto{\pgfqpoint{2.046290in}{1.089601in}}%
\pgfpathlineto{\pgfqpoint{2.029064in}{1.088879in}}%
\pgfpathlineto{\pgfqpoint{2.029399in}{1.083124in}}%
\pgfpathlineto{\pgfqpoint{2.020734in}{1.079515in}}%
\pgfpathlineto{\pgfqpoint{2.014367in}{1.079611in}}%
\pgfpathlineto{\pgfqpoint{2.010046in}{1.075951in}}%
\pgfpathlineto{\pgfqpoint{2.001501in}{1.072967in}}%
\pgfpathlineto{\pgfqpoint{1.990680in}{1.085683in}}%
\pgfpathlineto{\pgfqpoint{1.974361in}{1.084952in}}%
\pgfpathlineto{\pgfqpoint{1.975453in}{1.054398in}}%
\pgfpathlineto{\pgfqpoint{1.972796in}{1.051471in}}%
\pgfpathclose%
\pgfusepath{fill}%
\end{pgfscope}%
\begin{pgfscope}%
\pgfpathrectangle{\pgfqpoint{0.100000in}{0.100000in}}{\pgfqpoint{3.007045in}{1.925000in}}%
\pgfusepath{clip}%
\pgfsetbuttcap%
\pgfsetmiterjoin%
\definecolor{currentfill}{rgb}{0.311157,0.608289,0.796894}%
\pgfsetfillcolor{currentfill}%
\pgfsetlinewidth{0.000000pt}%
\definecolor{currentstroke}{rgb}{0.000000,0.000000,0.000000}%
\pgfsetstrokecolor{currentstroke}%
\pgfsetstrokeopacity{0.000000}%
\pgfsetdash{}{0pt}%
\pgfpathmoveto{\pgfqpoint{1.792356in}{1.069644in}}%
\pgfpathlineto{\pgfqpoint{1.768296in}{1.069654in}}%
\pgfpathlineto{\pgfqpoint{1.768960in}{1.084214in}}%
\pgfpathlineto{\pgfqpoint{1.768947in}{1.112944in}}%
\pgfpathlineto{\pgfqpoint{1.769426in}{1.115817in}}%
\pgfpathlineto{\pgfqpoint{1.769446in}{1.131864in}}%
\pgfpathlineto{\pgfqpoint{1.776978in}{1.132450in}}%
\pgfpathlineto{\pgfqpoint{1.777411in}{1.146354in}}%
\pgfpathlineto{\pgfqpoint{1.780164in}{1.150731in}}%
\pgfpathlineto{\pgfqpoint{1.772963in}{1.162326in}}%
\pgfpathlineto{\pgfqpoint{1.767101in}{1.168173in}}%
\pgfpathlineto{\pgfqpoint{1.792583in}{1.168136in}}%
\pgfpathlineto{\pgfqpoint{1.792627in}{1.163111in}}%
\pgfpathlineto{\pgfqpoint{1.812471in}{1.163163in}}%
\pgfpathlineto{\pgfqpoint{1.812537in}{1.167939in}}%
\pgfpathlineto{\pgfqpoint{1.835366in}{1.168017in}}%
\pgfpathlineto{\pgfqpoint{1.835689in}{1.147108in}}%
\pgfpathlineto{\pgfqpoint{1.840824in}{1.149749in}}%
\pgfpathlineto{\pgfqpoint{1.848856in}{1.148030in}}%
\pgfpathlineto{\pgfqpoint{1.849349in}{1.129915in}}%
\pgfpathlineto{\pgfqpoint{1.849007in}{1.104433in}}%
\pgfpathlineto{\pgfqpoint{1.820514in}{1.104792in}}%
\pgfpathlineto{\pgfqpoint{1.820190in}{1.081544in}}%
\pgfpathlineto{\pgfqpoint{1.821187in}{1.069825in}}%
\pgfpathlineto{\pgfqpoint{1.811752in}{1.070677in}}%
\pgfpathlineto{\pgfqpoint{1.792345in}{1.071169in}}%
\pgfpathlineto{\pgfqpoint{1.792356in}{1.069644in}}%
\pgfpathclose%
\pgfusepath{fill}%
\end{pgfscope}%
\begin{pgfscope}%
\pgfpathrectangle{\pgfqpoint{0.100000in}{0.100000in}}{\pgfqpoint{3.007045in}{1.925000in}}%
\pgfusepath{clip}%
\pgfsetbuttcap%
\pgfsetmiterjoin%
\definecolor{currentfill}{rgb}{0.183206,0.496932,0.737516}%
\pgfsetfillcolor{currentfill}%
\pgfsetlinewidth{0.000000pt}%
\definecolor{currentstroke}{rgb}{0.000000,0.000000,0.000000}%
\pgfsetstrokecolor{currentstroke}%
\pgfsetstrokeopacity{0.000000}%
\pgfsetdash{}{0pt}%
\pgfpathmoveto{\pgfqpoint{1.285700in}{1.824835in}}%
\pgfpathlineto{\pgfqpoint{1.331685in}{1.819464in}}%
\pgfpathlineto{\pgfqpoint{1.331869in}{1.813477in}}%
\pgfpathlineto{\pgfqpoint{1.329938in}{1.796177in}}%
\pgfpathlineto{\pgfqpoint{1.332164in}{1.790080in}}%
\pgfpathlineto{\pgfqpoint{1.295653in}{1.794279in}}%
\pgfpathlineto{\pgfqpoint{1.286131in}{1.795421in}}%
\pgfpathlineto{\pgfqpoint{1.288237in}{1.812721in}}%
\pgfpathlineto{\pgfqpoint{1.282530in}{1.813419in}}%
\pgfpathlineto{\pgfqpoint{1.285700in}{1.824835in}}%
\pgfpathclose%
\pgfusepath{fill}%
\end{pgfscope}%
\begin{pgfscope}%
\pgfpathrectangle{\pgfqpoint{0.100000in}{0.100000in}}{\pgfqpoint{3.007045in}{1.925000in}}%
\pgfusepath{clip}%
\pgfsetbuttcap%
\pgfsetmiterjoin%
\definecolor{currentfill}{rgb}{0.401953,0.670296,0.832326}%
\pgfsetfillcolor{currentfill}%
\pgfsetlinewidth{0.000000pt}%
\definecolor{currentstroke}{rgb}{0.000000,0.000000,0.000000}%
\pgfsetstrokecolor{currentstroke}%
\pgfsetstrokeopacity{0.000000}%
\pgfsetdash{}{0pt}%
\pgfpathmoveto{\pgfqpoint{2.568608in}{0.576327in}}%
\pgfpathlineto{\pgfqpoint{2.562689in}{0.572385in}}%
\pgfpathlineto{\pgfqpoint{2.551269in}{0.570455in}}%
\pgfpathlineto{\pgfqpoint{2.549830in}{0.579481in}}%
\pgfpathlineto{\pgfqpoint{2.545862in}{0.584302in}}%
\pgfpathlineto{\pgfqpoint{2.532632in}{0.582139in}}%
\pgfpathlineto{\pgfqpoint{2.526180in}{0.575201in}}%
\pgfpathlineto{\pgfqpoint{2.520191in}{0.572396in}}%
\pgfpathlineto{\pgfqpoint{2.516161in}{0.600413in}}%
\pgfpathlineto{\pgfqpoint{2.492884in}{0.596677in}}%
\pgfpathlineto{\pgfqpoint{2.488716in}{0.626214in}}%
\pgfpathlineto{\pgfqpoint{2.502750in}{0.627127in}}%
\pgfpathlineto{\pgfqpoint{2.504409in}{0.641856in}}%
\pgfpathlineto{\pgfqpoint{2.488999in}{0.640420in}}%
\pgfpathlineto{\pgfqpoint{2.487249in}{0.654855in}}%
\pgfpathlineto{\pgfqpoint{2.503300in}{0.656930in}}%
\pgfpathlineto{\pgfqpoint{2.506695in}{0.661778in}}%
\pgfpathlineto{\pgfqpoint{2.514003in}{0.660961in}}%
\pgfpathlineto{\pgfqpoint{2.521337in}{0.669031in}}%
\pgfpathlineto{\pgfqpoint{2.533990in}{0.667284in}}%
\pgfpathlineto{\pgfqpoint{2.542300in}{0.662555in}}%
\pgfpathlineto{\pgfqpoint{2.545263in}{0.656150in}}%
\pgfpathlineto{\pgfqpoint{2.543774in}{0.645180in}}%
\pgfpathlineto{\pgfqpoint{2.546129in}{0.642161in}}%
\pgfpathlineto{\pgfqpoint{2.546297in}{0.634841in}}%
\pgfpathlineto{\pgfqpoint{2.551061in}{0.622627in}}%
\pgfpathlineto{\pgfqpoint{2.552074in}{0.616494in}}%
\pgfpathlineto{\pgfqpoint{2.560429in}{0.594452in}}%
\pgfpathlineto{\pgfqpoint{2.568608in}{0.576327in}}%
\pgfpathclose%
\pgfusepath{fill}%
\end{pgfscope}%
\begin{pgfscope}%
\pgfpathrectangle{\pgfqpoint{0.100000in}{0.100000in}}{\pgfqpoint{3.007045in}{1.925000in}}%
\pgfusepath{clip}%
\pgfsetbuttcap%
\pgfsetmiterjoin%
\definecolor{currentfill}{rgb}{0.311157,0.608289,0.796894}%
\pgfsetfillcolor{currentfill}%
\pgfsetlinewidth{0.000000pt}%
\definecolor{currentstroke}{rgb}{0.000000,0.000000,0.000000}%
\pgfsetstrokecolor{currentstroke}%
\pgfsetstrokeopacity{0.000000}%
\pgfsetdash{}{0pt}%
\pgfpathmoveto{\pgfqpoint{1.342266in}{0.937180in}}%
\pgfpathlineto{\pgfqpoint{1.341578in}{0.929505in}}%
\pgfpathlineto{\pgfqpoint{1.337618in}{0.885562in}}%
\pgfpathlineto{\pgfqpoint{1.324670in}{0.886744in}}%
\pgfpathlineto{\pgfqpoint{1.324074in}{0.880866in}}%
\pgfpathlineto{\pgfqpoint{1.318390in}{0.881414in}}%
\pgfpathlineto{\pgfqpoint{1.317814in}{0.875699in}}%
\pgfpathlineto{\pgfqpoint{1.312137in}{0.876255in}}%
\pgfpathlineto{\pgfqpoint{1.311575in}{0.870525in}}%
\pgfpathlineto{\pgfqpoint{1.300241in}{0.871662in}}%
\pgfpathlineto{\pgfqpoint{1.299641in}{0.865888in}}%
\pgfpathlineto{\pgfqpoint{1.286524in}{0.867195in}}%
\pgfpathlineto{\pgfqpoint{1.276783in}{0.868180in}}%
\pgfpathlineto{\pgfqpoint{1.280556in}{0.903442in}}%
\pgfpathlineto{\pgfqpoint{1.295559in}{0.908588in}}%
\pgfpathlineto{\pgfqpoint{1.307382in}{0.907359in}}%
\pgfpathlineto{\pgfqpoint{1.308509in}{0.919262in}}%
\pgfpathlineto{\pgfqpoint{1.303832in}{0.919724in}}%
\pgfpathlineto{\pgfqpoint{1.292797in}{0.945962in}}%
\pgfpathlineto{\pgfqpoint{1.290299in}{0.944870in}}%
\pgfpathlineto{\pgfqpoint{1.271875in}{0.946736in}}%
\pgfpathlineto{\pgfqpoint{1.273425in}{0.954257in}}%
\pgfpathlineto{\pgfqpoint{1.273752in}{0.963960in}}%
\pgfpathlineto{\pgfqpoint{1.271190in}{0.975810in}}%
\pgfpathlineto{\pgfqpoint{1.293808in}{0.973457in}}%
\pgfpathlineto{\pgfqpoint{1.293521in}{0.970602in}}%
\pgfpathlineto{\pgfqpoint{1.304909in}{0.969450in}}%
\pgfpathlineto{\pgfqpoint{1.304327in}{0.963661in}}%
\pgfpathlineto{\pgfqpoint{1.327114in}{0.961542in}}%
\pgfpathlineto{\pgfqpoint{1.324480in}{0.938840in}}%
\pgfpathlineto{\pgfqpoint{1.342266in}{0.937180in}}%
\pgfpathclose%
\pgfusepath{fill}%
\end{pgfscope}%
\begin{pgfscope}%
\pgfpathrectangle{\pgfqpoint{0.100000in}{0.100000in}}{\pgfqpoint{3.007045in}{1.925000in}}%
\pgfusepath{clip}%
\pgfsetbuttcap%
\pgfsetmiterjoin%
\definecolor{currentfill}{rgb}{0.711265,0.831111,0.914433}%
\pgfsetfillcolor{currentfill}%
\pgfsetlinewidth{0.000000pt}%
\definecolor{currentstroke}{rgb}{0.000000,0.000000,0.000000}%
\pgfsetstrokecolor{currentstroke}%
\pgfsetstrokeopacity{0.000000}%
\pgfsetdash{}{0pt}%
\pgfpathmoveto{\pgfqpoint{2.625680in}{1.382482in}}%
\pgfpathlineto{\pgfqpoint{2.619453in}{1.411977in}}%
\pgfpathlineto{\pgfqpoint{2.612717in}{1.410639in}}%
\pgfpathlineto{\pgfqpoint{2.608126in}{1.441492in}}%
\pgfpathlineto{\pgfqpoint{2.609773in}{1.449142in}}%
\pgfpathlineto{\pgfqpoint{2.634427in}{1.453744in}}%
\pgfpathlineto{\pgfqpoint{2.635144in}{1.448720in}}%
\pgfpathlineto{\pgfqpoint{2.647379in}{1.448629in}}%
\pgfpathlineto{\pgfqpoint{2.646021in}{1.453686in}}%
\pgfpathlineto{\pgfqpoint{2.661352in}{1.457374in}}%
\pgfpathlineto{\pgfqpoint{2.656543in}{1.461343in}}%
\pgfpathlineto{\pgfqpoint{2.675608in}{1.465295in}}%
\pgfpathlineto{\pgfqpoint{2.679109in}{1.451464in}}%
\pgfpathlineto{\pgfqpoint{2.680763in}{1.444336in}}%
\pgfpathlineto{\pgfqpoint{2.672580in}{1.444093in}}%
\pgfpathlineto{\pgfqpoint{2.667190in}{1.440512in}}%
\pgfpathlineto{\pgfqpoint{2.669988in}{1.422124in}}%
\pgfpathlineto{\pgfqpoint{2.652249in}{1.418613in}}%
\pgfpathlineto{\pgfqpoint{2.660061in}{1.392938in}}%
\pgfpathlineto{\pgfqpoint{2.656505in}{1.389182in}}%
\pgfpathlineto{\pgfqpoint{2.625680in}{1.382482in}}%
\pgfpathclose%
\pgfusepath{fill}%
\end{pgfscope}%
\begin{pgfscope}%
\pgfpathrectangle{\pgfqpoint{0.100000in}{0.100000in}}{\pgfqpoint{3.007045in}{1.925000in}}%
\pgfusepath{clip}%
\pgfsetbuttcap%
\pgfsetmiterjoin%
\definecolor{currentfill}{rgb}{0.346467,0.632403,0.810673}%
\pgfsetfillcolor{currentfill}%
\pgfsetlinewidth{0.000000pt}%
\definecolor{currentstroke}{rgb}{0.000000,0.000000,0.000000}%
\pgfsetstrokecolor{currentstroke}%
\pgfsetstrokeopacity{0.000000}%
\pgfsetdash{}{0pt}%
\pgfpathmoveto{\pgfqpoint{0.683738in}{1.619619in}}%
\pgfpathlineto{\pgfqpoint{0.687625in}{1.619603in}}%
\pgfpathlineto{\pgfqpoint{0.695577in}{1.627414in}}%
\pgfpathlineto{\pgfqpoint{0.698781in}{1.628508in}}%
\pgfpathlineto{\pgfqpoint{0.716216in}{1.624121in}}%
\pgfpathlineto{\pgfqpoint{0.720108in}{1.619262in}}%
\pgfpathlineto{\pgfqpoint{0.718698in}{1.613732in}}%
\pgfpathlineto{\pgfqpoint{0.730914in}{1.610662in}}%
\pgfpathlineto{\pgfqpoint{0.733470in}{1.622014in}}%
\pgfpathlineto{\pgfqpoint{0.738581in}{1.626973in}}%
\pgfpathlineto{\pgfqpoint{0.746366in}{1.637288in}}%
\pgfpathlineto{\pgfqpoint{0.754151in}{1.641671in}}%
\pgfpathlineto{\pgfqpoint{0.766659in}{1.638492in}}%
\pgfpathlineto{\pgfqpoint{0.763942in}{1.627183in}}%
\pgfpathlineto{\pgfqpoint{0.767582in}{1.623021in}}%
\pgfpathlineto{\pgfqpoint{0.765709in}{1.615018in}}%
\pgfpathlineto{\pgfqpoint{0.770478in}{1.610681in}}%
\pgfpathlineto{\pgfqpoint{0.775430in}{1.609514in}}%
\pgfpathlineto{\pgfqpoint{0.771340in}{1.602215in}}%
\pgfpathlineto{\pgfqpoint{0.768216in}{1.591309in}}%
\pgfpathlineto{\pgfqpoint{0.763908in}{1.592366in}}%
\pgfpathlineto{\pgfqpoint{0.762552in}{1.586835in}}%
\pgfpathlineto{\pgfqpoint{0.758886in}{1.587771in}}%
\pgfpathlineto{\pgfqpoint{0.755702in}{1.582709in}}%
\pgfpathlineto{\pgfqpoint{0.749229in}{1.584343in}}%
\pgfpathlineto{\pgfqpoint{0.746496in}{1.573138in}}%
\pgfpathlineto{\pgfqpoint{0.739760in}{1.573820in}}%
\pgfpathlineto{\pgfqpoint{0.736446in}{1.580574in}}%
\pgfpathlineto{\pgfqpoint{0.733534in}{1.580107in}}%
\pgfpathlineto{\pgfqpoint{0.729959in}{1.568628in}}%
\pgfpathlineto{\pgfqpoint{0.703346in}{1.460924in}}%
\pgfpathlineto{\pgfqpoint{0.647669in}{1.474848in}}%
\pgfpathlineto{\pgfqpoint{0.651488in}{1.492874in}}%
\pgfpathlineto{\pgfqpoint{0.662044in}{1.533668in}}%
\pgfpathlineto{\pgfqpoint{0.661467in}{1.533821in}}%
\pgfpathlineto{\pgfqpoint{0.675770in}{1.589097in}}%
\pgfpathlineto{\pgfqpoint{0.680340in}{1.605736in}}%
\pgfpathlineto{\pgfqpoint{0.683738in}{1.619619in}}%
\pgfpathclose%
\pgfusepath{fill}%
\end{pgfscope}%
\begin{pgfscope}%
\pgfpathrectangle{\pgfqpoint{0.100000in}{0.100000in}}{\pgfqpoint{3.007045in}{1.925000in}}%
\pgfusepath{clip}%
\pgfsetbuttcap%
\pgfsetmiterjoin%
\definecolor{currentfill}{rgb}{0.417086,0.680631,0.838231}%
\pgfsetfillcolor{currentfill}%
\pgfsetlinewidth{0.000000pt}%
\definecolor{currentstroke}{rgb}{0.000000,0.000000,0.000000}%
\pgfsetstrokecolor{currentstroke}%
\pgfsetstrokeopacity{0.000000}%
\pgfsetdash{}{0pt}%
\pgfpathmoveto{\pgfqpoint{2.402438in}{0.696626in}}%
\pgfpathlineto{\pgfqpoint{2.395249in}{0.698742in}}%
\pgfpathlineto{\pgfqpoint{2.394856in}{0.702917in}}%
\pgfpathlineto{\pgfqpoint{2.389600in}{0.714570in}}%
\pgfpathlineto{\pgfqpoint{2.386745in}{0.718055in}}%
\pgfpathlineto{\pgfqpoint{2.389930in}{0.724977in}}%
\pgfpathlineto{\pgfqpoint{2.395922in}{0.728401in}}%
\pgfpathlineto{\pgfqpoint{2.409404in}{0.729924in}}%
\pgfpathlineto{\pgfqpoint{2.410702in}{0.718714in}}%
\pgfpathlineto{\pgfqpoint{2.425917in}{0.719822in}}%
\pgfpathlineto{\pgfqpoint{2.428514in}{0.718344in}}%
\pgfpathlineto{\pgfqpoint{2.429949in}{0.711648in}}%
\pgfpathlineto{\pgfqpoint{2.433519in}{0.709598in}}%
\pgfpathlineto{\pgfqpoint{2.437033in}{0.704008in}}%
\pgfpathlineto{\pgfqpoint{2.420123in}{0.701779in}}%
\pgfpathlineto{\pgfqpoint{2.412679in}{0.701290in}}%
\pgfpathlineto{\pgfqpoint{2.413072in}{0.697996in}}%
\pgfpathlineto{\pgfqpoint{2.402438in}{0.696626in}}%
\pgfpathclose%
\pgfusepath{fill}%
\end{pgfscope}%
\begin{pgfscope}%
\pgfpathrectangle{\pgfqpoint{0.100000in}{0.100000in}}{\pgfqpoint{3.007045in}{1.925000in}}%
\pgfusepath{clip}%
\pgfsetbuttcap%
\pgfsetmiterjoin%
\definecolor{currentfill}{rgb}{0.326290,0.618624,0.802799}%
\pgfsetfillcolor{currentfill}%
\pgfsetlinewidth{0.000000pt}%
\definecolor{currentstroke}{rgb}{0.000000,0.000000,0.000000}%
\pgfsetstrokecolor{currentstroke}%
\pgfsetstrokeopacity{0.000000}%
\pgfsetdash{}{0pt}%
\pgfpathmoveto{\pgfqpoint{1.511195in}{1.206454in}}%
\pgfpathlineto{\pgfqpoint{1.510401in}{1.206498in}}%
\pgfpathlineto{\pgfqpoint{1.511500in}{1.229428in}}%
\pgfpathlineto{\pgfqpoint{1.482455in}{1.231145in}}%
\pgfpathlineto{\pgfqpoint{1.483897in}{1.254208in}}%
\pgfpathlineto{\pgfqpoint{1.511547in}{1.252543in}}%
\pgfpathlineto{\pgfqpoint{1.512812in}{1.275327in}}%
\pgfpathlineto{\pgfqpoint{1.552431in}{1.273213in}}%
\pgfpathlineto{\pgfqpoint{1.551658in}{1.248467in}}%
\pgfpathlineto{\pgfqpoint{1.563506in}{1.247230in}}%
\pgfpathlineto{\pgfqpoint{1.561531in}{1.203886in}}%
\pgfpathlineto{\pgfqpoint{1.539001in}{1.204970in}}%
\pgfpathlineto{\pgfqpoint{1.511195in}{1.206454in}}%
\pgfpathclose%
\pgfusepath{fill}%
\end{pgfscope}%
\begin{pgfscope}%
\pgfpathrectangle{\pgfqpoint{0.100000in}{0.100000in}}{\pgfqpoint{3.007045in}{1.925000in}}%
\pgfusepath{clip}%
\pgfsetbuttcap%
\pgfsetmiterjoin%
\definecolor{currentfill}{rgb}{0.341423,0.628958,0.808704}%
\pgfsetfillcolor{currentfill}%
\pgfsetlinewidth{0.000000pt}%
\definecolor{currentstroke}{rgb}{0.000000,0.000000,0.000000}%
\pgfsetstrokecolor{currentstroke}%
\pgfsetstrokeopacity{0.000000}%
\pgfsetdash{}{0pt}%
\pgfpathmoveto{\pgfqpoint{1.298841in}{1.455523in}}%
\pgfpathlineto{\pgfqpoint{1.339144in}{1.451533in}}%
\pgfpathlineto{\pgfqpoint{1.335923in}{1.418503in}}%
\pgfpathlineto{\pgfqpoint{1.333387in}{1.392952in}}%
\pgfpathlineto{\pgfqpoint{1.292810in}{1.397021in}}%
\pgfpathlineto{\pgfqpoint{1.293878in}{1.409890in}}%
\pgfpathlineto{\pgfqpoint{1.298841in}{1.455523in}}%
\pgfpathclose%
\pgfusepath{fill}%
\end{pgfscope}%
\begin{pgfscope}%
\pgfpathrectangle{\pgfqpoint{0.100000in}{0.100000in}}{\pgfqpoint{3.007045in}{1.925000in}}%
\pgfusepath{clip}%
\pgfsetbuttcap%
\pgfsetmiterjoin%
\definecolor{currentfill}{rgb}{0.632526,0.797647,0.886874}%
\pgfsetfillcolor{currentfill}%
\pgfsetlinewidth{0.000000pt}%
\definecolor{currentstroke}{rgb}{0.000000,0.000000,0.000000}%
\pgfsetstrokecolor{currentstroke}%
\pgfsetstrokeopacity{0.000000}%
\pgfsetdash{}{0pt}%
\pgfpathmoveto{\pgfqpoint{0.374195in}{1.289386in}}%
\pgfpathlineto{\pgfqpoint{0.377156in}{1.290848in}}%
\pgfpathlineto{\pgfqpoint{0.377641in}{1.299051in}}%
\pgfpathlineto{\pgfqpoint{0.376145in}{1.309067in}}%
\pgfpathlineto{\pgfqpoint{0.371022in}{1.321407in}}%
\pgfpathlineto{\pgfqpoint{0.373315in}{1.325445in}}%
\pgfpathlineto{\pgfqpoint{0.382079in}{1.325306in}}%
\pgfpathlineto{\pgfqpoint{0.388700in}{1.330958in}}%
\pgfpathlineto{\pgfqpoint{0.388567in}{1.334258in}}%
\pgfpathlineto{\pgfqpoint{0.393337in}{1.331101in}}%
\pgfpathlineto{\pgfqpoint{0.393792in}{1.321038in}}%
\pgfpathlineto{\pgfqpoint{0.395623in}{1.317666in}}%
\pgfpathlineto{\pgfqpoint{0.396070in}{1.307741in}}%
\pgfpathlineto{\pgfqpoint{0.400092in}{1.304827in}}%
\pgfpathlineto{\pgfqpoint{0.404388in}{1.306594in}}%
\pgfpathlineto{\pgfqpoint{0.416471in}{1.302466in}}%
\pgfpathlineto{\pgfqpoint{0.412425in}{1.289011in}}%
\pgfpathlineto{\pgfqpoint{0.417431in}{1.287454in}}%
\pgfpathlineto{\pgfqpoint{0.414282in}{1.280552in}}%
\pgfpathlineto{\pgfqpoint{0.409917in}{1.279165in}}%
\pgfpathlineto{\pgfqpoint{0.409876in}{1.274771in}}%
\pgfpathlineto{\pgfqpoint{0.413844in}{1.273461in}}%
\pgfpathlineto{\pgfqpoint{0.409743in}{1.259767in}}%
\pgfpathlineto{\pgfqpoint{0.409870in}{1.255826in}}%
\pgfpathlineto{\pgfqpoint{0.404604in}{1.238315in}}%
\pgfpathlineto{\pgfqpoint{0.407690in}{1.233195in}}%
\pgfpathlineto{\pgfqpoint{0.407661in}{1.233230in}}%
\pgfpathlineto{\pgfqpoint{0.388058in}{1.239323in}}%
\pgfpathlineto{\pgfqpoint{0.384479in}{1.238341in}}%
\pgfpathlineto{\pgfqpoint{0.377116in}{1.242201in}}%
\pgfpathlineto{\pgfqpoint{0.370754in}{1.240933in}}%
\pgfpathlineto{\pgfqpoint{0.368846in}{1.236860in}}%
\pgfpathlineto{\pgfqpoint{0.368463in}{1.226772in}}%
\pgfpathlineto{\pgfqpoint{0.359639in}{1.227533in}}%
\pgfpathlineto{\pgfqpoint{0.359301in}{1.222111in}}%
\pgfpathlineto{\pgfqpoint{0.355662in}{1.228408in}}%
\pgfpathlineto{\pgfqpoint{0.355631in}{1.232502in}}%
\pgfpathlineto{\pgfqpoint{0.358662in}{1.239132in}}%
\pgfpathlineto{\pgfqpoint{0.358656in}{1.248068in}}%
\pgfpathlineto{\pgfqpoint{0.356187in}{1.251923in}}%
\pgfpathlineto{\pgfqpoint{0.359786in}{1.258991in}}%
\pgfpathlineto{\pgfqpoint{0.361404in}{1.267963in}}%
\pgfpathlineto{\pgfqpoint{0.367391in}{1.268171in}}%
\pgfpathlineto{\pgfqpoint{0.365394in}{1.253654in}}%
\pgfpathlineto{\pgfqpoint{0.369838in}{1.250965in}}%
\pgfpathlineto{\pgfqpoint{0.373837in}{1.244770in}}%
\pgfpathlineto{\pgfqpoint{0.376610in}{1.245915in}}%
\pgfpathlineto{\pgfqpoint{0.376740in}{1.254309in}}%
\pgfpathlineto{\pgfqpoint{0.370828in}{1.266174in}}%
\pgfpathlineto{\pgfqpoint{0.373731in}{1.272170in}}%
\pgfpathlineto{\pgfqpoint{0.370644in}{1.277375in}}%
\pgfpathlineto{\pgfqpoint{0.379239in}{1.281168in}}%
\pgfpathlineto{\pgfqpoint{0.378012in}{1.285740in}}%
\pgfpathlineto{\pgfqpoint{0.374195in}{1.289386in}}%
\pgfpathclose%
\pgfusepath{fill}%
\end{pgfscope}%
\begin{pgfscope}%
\pgfpathrectangle{\pgfqpoint{0.100000in}{0.100000in}}{\pgfqpoint{3.007045in}{1.925000in}}%
\pgfusepath{clip}%
\pgfsetbuttcap%
\pgfsetmiterjoin%
\definecolor{currentfill}{rgb}{0.632526,0.797647,0.886874}%
\pgfsetfillcolor{currentfill}%
\pgfsetlinewidth{0.000000pt}%
\definecolor{currentstroke}{rgb}{0.000000,0.000000,0.000000}%
\pgfsetstrokecolor{currentstroke}%
\pgfsetstrokeopacity{0.000000}%
\pgfsetdash{}{0pt}%
\pgfpathmoveto{\pgfqpoint{0.369150in}{1.288189in}}%
\pgfpathlineto{\pgfqpoint{0.366091in}{1.274818in}}%
\pgfpathlineto{\pgfqpoint{0.361852in}{1.270865in}}%
\pgfpathlineto{\pgfqpoint{0.357053in}{1.277694in}}%
\pgfpathlineto{\pgfqpoint{0.354298in}{1.277874in}}%
\pgfpathlineto{\pgfqpoint{0.349286in}{1.287909in}}%
\pgfpathlineto{\pgfqpoint{0.344216in}{1.290431in}}%
\pgfpathlineto{\pgfqpoint{0.347199in}{1.297896in}}%
\pgfpathlineto{\pgfqpoint{0.347482in}{1.307924in}}%
\pgfpathlineto{\pgfqpoint{0.352980in}{1.307630in}}%
\pgfpathlineto{\pgfqpoint{0.358679in}{1.298249in}}%
\pgfpathlineto{\pgfqpoint{0.366648in}{1.294372in}}%
\pgfpathlineto{\pgfqpoint{0.369150in}{1.288189in}}%
\pgfpathclose%
\pgfusepath{fill}%
\end{pgfscope}%
\begin{pgfscope}%
\pgfpathrectangle{\pgfqpoint{0.100000in}{0.100000in}}{\pgfqpoint{3.007045in}{1.925000in}}%
\pgfusepath{clip}%
\pgfsetbuttcap%
\pgfsetmiterjoin%
\definecolor{currentfill}{rgb}{0.396909,0.666851,0.830358}%
\pgfsetfillcolor{currentfill}%
\pgfsetlinewidth{0.000000pt}%
\definecolor{currentstroke}{rgb}{0.000000,0.000000,0.000000}%
\pgfsetstrokecolor{currentstroke}%
\pgfsetstrokeopacity{0.000000}%
\pgfsetdash{}{0pt}%
\pgfpathmoveto{\pgfqpoint{2.025542in}{1.310434in}}%
\pgfpathlineto{\pgfqpoint{2.009626in}{1.309620in}}%
\pgfpathlineto{\pgfqpoint{2.006146in}{1.322530in}}%
\pgfpathlineto{\pgfqpoint{2.009028in}{1.324274in}}%
\pgfpathlineto{\pgfqpoint{2.010393in}{1.337303in}}%
\pgfpathlineto{\pgfqpoint{2.008917in}{1.344598in}}%
\pgfpathlineto{\pgfqpoint{2.001078in}{1.349385in}}%
\pgfpathlineto{\pgfqpoint{2.020468in}{1.350608in}}%
\pgfpathlineto{\pgfqpoint{2.018995in}{1.370976in}}%
\pgfpathlineto{\pgfqpoint{2.046233in}{1.372218in}}%
\pgfpathlineto{\pgfqpoint{2.078237in}{1.373879in}}%
\pgfpathlineto{\pgfqpoint{2.079874in}{1.351505in}}%
\pgfpathlineto{\pgfqpoint{2.085588in}{1.351908in}}%
\pgfpathlineto{\pgfqpoint{2.085344in}{1.346129in}}%
\pgfpathlineto{\pgfqpoint{2.087331in}{1.317506in}}%
\pgfpathlineto{\pgfqpoint{2.059573in}{1.315459in}}%
\pgfpathlineto{\pgfqpoint{2.059792in}{1.312587in}}%
\pgfpathlineto{\pgfqpoint{2.025542in}{1.310434in}}%
\pgfpathclose%
\pgfusepath{fill}%
\end{pgfscope}%
\begin{pgfscope}%
\pgfpathrectangle{\pgfqpoint{0.100000in}{0.100000in}}{\pgfqpoint{3.007045in}{1.925000in}}%
\pgfusepath{clip}%
\pgfsetbuttcap%
\pgfsetmiterjoin%
\definecolor{currentfill}{rgb}{0.447843,0.697855,0.845306}%
\pgfsetfillcolor{currentfill}%
\pgfsetlinewidth{0.000000pt}%
\definecolor{currentstroke}{rgb}{0.000000,0.000000,0.000000}%
\pgfsetstrokecolor{currentstroke}%
\pgfsetstrokeopacity{0.000000}%
\pgfsetdash{}{0pt}%
\pgfpathmoveto{\pgfqpoint{2.169528in}{1.143936in}}%
\pgfpathlineto{\pgfqpoint{2.175194in}{1.144388in}}%
\pgfpathlineto{\pgfqpoint{2.175011in}{1.140154in}}%
\pgfpathlineto{\pgfqpoint{2.170118in}{1.138155in}}%
\pgfpathlineto{\pgfqpoint{2.168758in}{1.129637in}}%
\pgfpathlineto{\pgfqpoint{2.171500in}{1.120112in}}%
\pgfpathlineto{\pgfqpoint{2.160227in}{1.118407in}}%
\pgfpathlineto{\pgfqpoint{2.154476in}{1.114634in}}%
\pgfpathlineto{\pgfqpoint{2.150356in}{1.119886in}}%
\pgfpathlineto{\pgfqpoint{2.136969in}{1.118978in}}%
\pgfpathlineto{\pgfqpoint{2.135746in}{1.137366in}}%
\pgfpathlineto{\pgfqpoint{2.154997in}{1.139057in}}%
\pgfpathlineto{\pgfqpoint{2.155147in}{1.142366in}}%
\pgfpathlineto{\pgfqpoint{2.169528in}{1.143936in}}%
\pgfpathclose%
\pgfusepath{fill}%
\end{pgfscope}%
\begin{pgfscope}%
\pgfpathrectangle{\pgfqpoint{0.100000in}{0.100000in}}{\pgfqpoint{3.007045in}{1.925000in}}%
\pgfusepath{clip}%
\pgfsetbuttcap%
\pgfsetmiterjoin%
\definecolor{currentfill}{rgb}{0.296025,0.597955,0.790988}%
\pgfsetfillcolor{currentfill}%
\pgfsetlinewidth{0.000000pt}%
\definecolor{currentstroke}{rgb}{0.000000,0.000000,0.000000}%
\pgfsetstrokecolor{currentstroke}%
\pgfsetstrokeopacity{0.000000}%
\pgfsetdash{}{0pt}%
\pgfpathmoveto{\pgfqpoint{1.735784in}{0.842654in}}%
\pgfpathlineto{\pgfqpoint{1.735756in}{0.836902in}}%
\pgfpathlineto{\pgfqpoint{1.729891in}{0.836933in}}%
\pgfpathlineto{\pgfqpoint{1.729695in}{0.813812in}}%
\pgfpathlineto{\pgfqpoint{1.718089in}{0.813933in}}%
\pgfpathlineto{\pgfqpoint{1.695332in}{0.814267in}}%
\pgfpathlineto{\pgfqpoint{1.695616in}{0.831530in}}%
\pgfpathlineto{\pgfqpoint{1.689815in}{0.831695in}}%
\pgfpathlineto{\pgfqpoint{1.689906in}{0.837361in}}%
\pgfpathlineto{\pgfqpoint{1.672844in}{0.837714in}}%
\pgfpathlineto{\pgfqpoint{1.672960in}{0.843473in}}%
\pgfpathlineto{\pgfqpoint{1.667244in}{0.843587in}}%
\pgfpathlineto{\pgfqpoint{1.667600in}{0.860848in}}%
\pgfpathlineto{\pgfqpoint{1.670026in}{0.865689in}}%
\pgfpathlineto{\pgfqpoint{1.674047in}{0.866744in}}%
\pgfpathlineto{\pgfqpoint{1.678967in}{0.860772in}}%
\pgfpathlineto{\pgfqpoint{1.683133in}{0.863693in}}%
\pgfpathlineto{\pgfqpoint{1.691602in}{0.864102in}}%
\pgfpathlineto{\pgfqpoint{1.691843in}{0.877659in}}%
\pgfpathlineto{\pgfqpoint{1.694493in}{0.877607in}}%
\pgfpathlineto{\pgfqpoint{1.694674in}{0.889113in}}%
\pgfpathlineto{\pgfqpoint{1.719456in}{0.888768in}}%
\pgfpathlineto{\pgfqpoint{1.719231in}{0.879638in}}%
\pgfpathlineto{\pgfqpoint{1.713476in}{0.872989in}}%
\pgfpathlineto{\pgfqpoint{1.713003in}{0.848605in}}%
\pgfpathlineto{\pgfqpoint{1.724413in}{0.848474in}}%
\pgfpathlineto{\pgfqpoint{1.724353in}{0.842730in}}%
\pgfpathlineto{\pgfqpoint{1.735784in}{0.842654in}}%
\pgfpathclose%
\pgfusepath{fill}%
\end{pgfscope}%
\begin{pgfscope}%
\pgfpathrectangle{\pgfqpoint{0.100000in}{0.100000in}}{\pgfqpoint{3.007045in}{1.925000in}}%
\pgfusepath{clip}%
\pgfsetbuttcap%
\pgfsetmiterjoin%
\definecolor{currentfill}{rgb}{0.466667,0.708189,0.849366}%
\pgfsetfillcolor{currentfill}%
\pgfsetlinewidth{0.000000pt}%
\definecolor{currentstroke}{rgb}{0.000000,0.000000,0.000000}%
\pgfsetstrokecolor{currentstroke}%
\pgfsetstrokeopacity{0.000000}%
\pgfsetdash{}{0pt}%
\pgfpathmoveto{\pgfqpoint{2.643747in}{1.050644in}}%
\pgfpathlineto{\pgfqpoint{2.672729in}{1.056217in}}%
\pgfpathlineto{\pgfqpoint{2.674093in}{1.045577in}}%
\pgfpathlineto{\pgfqpoint{2.671365in}{1.033005in}}%
\pgfpathlineto{\pgfqpoint{2.664233in}{1.034590in}}%
\pgfpathlineto{\pgfqpoint{2.654897in}{1.034173in}}%
\pgfpathlineto{\pgfqpoint{2.650549in}{1.026900in}}%
\pgfpathlineto{\pgfqpoint{2.645335in}{1.026823in}}%
\pgfpathlineto{\pgfqpoint{2.641967in}{1.044405in}}%
\pgfpathlineto{\pgfqpoint{2.643747in}{1.050644in}}%
\pgfpathclose%
\pgfusepath{fill}%
\end{pgfscope}%
\begin{pgfscope}%
\pgfpathrectangle{\pgfqpoint{0.100000in}{0.100000in}}{\pgfqpoint{3.007045in}{1.925000in}}%
\pgfusepath{clip}%
\pgfsetbuttcap%
\pgfsetmiterjoin%
\definecolor{currentfill}{rgb}{0.311157,0.608289,0.796894}%
\pgfsetfillcolor{currentfill}%
\pgfsetlinewidth{0.000000pt}%
\definecolor{currentstroke}{rgb}{0.000000,0.000000,0.000000}%
\pgfsetstrokecolor{currentstroke}%
\pgfsetstrokeopacity{0.000000}%
\pgfsetdash{}{0pt}%
\pgfpathmoveto{\pgfqpoint{1.003150in}{0.848168in}}%
\pgfpathlineto{\pgfqpoint{0.997334in}{0.810949in}}%
\pgfpathlineto{\pgfqpoint{0.989331in}{0.759721in}}%
\pgfpathlineto{\pgfqpoint{0.927312in}{0.769778in}}%
\pgfpathlineto{\pgfqpoint{0.911810in}{0.772480in}}%
\pgfpathlineto{\pgfqpoint{0.912796in}{0.778126in}}%
\pgfpathlineto{\pgfqpoint{0.920548in}{0.822552in}}%
\pgfpathlineto{\pgfqpoint{0.923055in}{0.834676in}}%
\pgfpathlineto{\pgfqpoint{0.930194in}{0.836921in}}%
\pgfpathlineto{\pgfqpoint{0.938796in}{0.837657in}}%
\pgfpathlineto{\pgfqpoint{0.947992in}{0.836130in}}%
\pgfpathlineto{\pgfqpoint{0.949199in}{0.843379in}}%
\pgfpathlineto{\pgfqpoint{0.954625in}{0.842378in}}%
\pgfpathlineto{\pgfqpoint{0.960834in}{0.834656in}}%
\pgfpathlineto{\pgfqpoint{0.971108in}{0.842140in}}%
\pgfpathlineto{\pgfqpoint{0.975090in}{0.841991in}}%
\pgfpathlineto{\pgfqpoint{0.978685in}{0.845988in}}%
\pgfpathlineto{\pgfqpoint{0.986809in}{0.850694in}}%
\pgfpathlineto{\pgfqpoint{1.003150in}{0.848168in}}%
\pgfpathclose%
\pgfusepath{fill}%
\end{pgfscope}%
\begin{pgfscope}%
\pgfpathrectangle{\pgfqpoint{0.100000in}{0.100000in}}{\pgfqpoint{3.007045in}{1.925000in}}%
\pgfusepath{clip}%
\pgfsetbuttcap%
\pgfsetmiterjoin%
\definecolor{currentfill}{rgb}{0.331334,0.622068,0.804767}%
\pgfsetfillcolor{currentfill}%
\pgfsetlinewidth{0.000000pt}%
\definecolor{currentstroke}{rgb}{0.000000,0.000000,0.000000}%
\pgfsetstrokecolor{currentstroke}%
\pgfsetstrokeopacity{0.000000}%
\pgfsetdash{}{0pt}%
\pgfpathmoveto{\pgfqpoint{1.730423in}{0.638033in}}%
\pgfpathlineto{\pgfqpoint{1.734214in}{0.624182in}}%
\pgfpathlineto{\pgfqpoint{1.731113in}{0.622526in}}%
\pgfpathlineto{\pgfqpoint{1.728210in}{0.613654in}}%
\pgfpathlineto{\pgfqpoint{1.728806in}{0.610818in}}%
\pgfpathlineto{\pgfqpoint{1.716755in}{0.610772in}}%
\pgfpathlineto{\pgfqpoint{1.701562in}{0.603092in}}%
\pgfpathlineto{\pgfqpoint{1.696915in}{0.623980in}}%
\pgfpathlineto{\pgfqpoint{1.697477in}{0.628654in}}%
\pgfpathlineto{\pgfqpoint{1.681563in}{0.619854in}}%
\pgfpathlineto{\pgfqpoint{1.663193in}{0.652637in}}%
\pgfpathlineto{\pgfqpoint{1.675361in}{0.659431in}}%
\pgfpathlineto{\pgfqpoint{1.665733in}{0.676821in}}%
\pgfpathlineto{\pgfqpoint{1.694814in}{0.693189in}}%
\pgfpathlineto{\pgfqpoint{1.703043in}{0.687922in}}%
\pgfpathlineto{\pgfqpoint{1.712375in}{0.673683in}}%
\pgfpathlineto{\pgfqpoint{1.718592in}{0.656205in}}%
\pgfpathlineto{\pgfqpoint{1.722872in}{0.654828in}}%
\pgfpathlineto{\pgfqpoint{1.723459in}{0.650585in}}%
\pgfpathlineto{\pgfqpoint{1.732103in}{0.644712in}}%
\pgfpathlineto{\pgfqpoint{1.730423in}{0.638033in}}%
\pgfpathclose%
\pgfusepath{fill}%
\end{pgfscope}%
\begin{pgfscope}%
\pgfpathrectangle{\pgfqpoint{0.100000in}{0.100000in}}{\pgfqpoint{3.007045in}{1.925000in}}%
\pgfusepath{clip}%
\pgfsetbuttcap%
\pgfsetmiterjoin%
\definecolor{currentfill}{rgb}{0.441569,0.694410,0.843952}%
\pgfsetfillcolor{currentfill}%
\pgfsetlinewidth{0.000000pt}%
\definecolor{currentstroke}{rgb}{0.000000,0.000000,0.000000}%
\pgfsetstrokecolor{currentstroke}%
\pgfsetstrokeopacity{0.000000}%
\pgfsetdash{}{0pt}%
\pgfpathmoveto{\pgfqpoint{1.643166in}{1.743370in}}%
\pgfpathlineto{\pgfqpoint{1.643745in}{1.731622in}}%
\pgfpathlineto{\pgfqpoint{1.642987in}{1.708503in}}%
\pgfpathlineto{\pgfqpoint{1.638404in}{1.708645in}}%
\pgfpathlineto{\pgfqpoint{1.615296in}{1.709477in}}%
\pgfpathlineto{\pgfqpoint{1.614607in}{1.721113in}}%
\pgfpathlineto{\pgfqpoint{1.580095in}{1.722634in}}%
\pgfpathlineto{\pgfqpoint{1.557103in}{1.723835in}}%
\pgfpathlineto{\pgfqpoint{1.556105in}{1.735501in}}%
\pgfpathlineto{\pgfqpoint{1.557416in}{1.758816in}}%
\pgfpathlineto{\pgfqpoint{1.573086in}{1.757987in}}%
\pgfpathlineto{\pgfqpoint{1.596215in}{1.756877in}}%
\pgfpathlineto{\pgfqpoint{1.596746in}{1.768470in}}%
\pgfpathlineto{\pgfqpoint{1.625579in}{1.767286in}}%
\pgfpathlineto{\pgfqpoint{1.626307in}{1.755616in}}%
\pgfpathlineto{\pgfqpoint{1.625869in}{1.743967in}}%
\pgfpathlineto{\pgfqpoint{1.643166in}{1.743370in}}%
\pgfpathclose%
\pgfusepath{fill}%
\end{pgfscope}%
\begin{pgfscope}%
\pgfpathrectangle{\pgfqpoint{0.100000in}{0.100000in}}{\pgfqpoint{3.007045in}{1.925000in}}%
\pgfusepath{clip}%
\pgfsetbuttcap%
\pgfsetmiterjoin%
\definecolor{currentfill}{rgb}{0.376732,0.653072,0.822484}%
\pgfsetfillcolor{currentfill}%
\pgfsetlinewidth{0.000000pt}%
\definecolor{currentstroke}{rgb}{0.000000,0.000000,0.000000}%
\pgfsetstrokecolor{currentstroke}%
\pgfsetstrokeopacity{0.000000}%
\pgfsetdash{}{0pt}%
\pgfpathmoveto{\pgfqpoint{2.060711in}{0.508116in}}%
\pgfpathlineto{\pgfqpoint{2.060425in}{0.500127in}}%
\pgfpathlineto{\pgfqpoint{2.056411in}{0.498585in}}%
\pgfpathlineto{\pgfqpoint{2.060074in}{0.492771in}}%
\pgfpathlineto{\pgfqpoint{2.058740in}{0.489589in}}%
\pgfpathlineto{\pgfqpoint{2.051172in}{0.484014in}}%
\pgfpathlineto{\pgfqpoint{2.048181in}{0.488038in}}%
\pgfpathlineto{\pgfqpoint{2.046171in}{0.495142in}}%
\pgfpathlineto{\pgfqpoint{2.042744in}{0.497419in}}%
\pgfpathlineto{\pgfqpoint{2.040294in}{0.492004in}}%
\pgfpathlineto{\pgfqpoint{2.034473in}{0.496696in}}%
\pgfpathlineto{\pgfqpoint{2.026621in}{0.490352in}}%
\pgfpathlineto{\pgfqpoint{2.026344in}{0.485680in}}%
\pgfpathlineto{\pgfqpoint{2.021444in}{0.485001in}}%
\pgfpathlineto{\pgfqpoint{2.014806in}{0.481223in}}%
\pgfpathlineto{\pgfqpoint{2.007886in}{0.488255in}}%
\pgfpathlineto{\pgfqpoint{2.000041in}{0.488340in}}%
\pgfpathlineto{\pgfqpoint{1.989313in}{0.491975in}}%
\pgfpathlineto{\pgfqpoint{1.985959in}{0.494974in}}%
\pgfpathlineto{\pgfqpoint{1.991281in}{0.500512in}}%
\pgfpathlineto{\pgfqpoint{1.989147in}{0.507119in}}%
\pgfpathlineto{\pgfqpoint{1.973270in}{0.512709in}}%
\pgfpathlineto{\pgfqpoint{1.968012in}{0.516450in}}%
\pgfpathlineto{\pgfqpoint{1.968031in}{0.523710in}}%
\pgfpathlineto{\pgfqpoint{1.958964in}{0.523855in}}%
\pgfpathlineto{\pgfqpoint{1.953040in}{0.522060in}}%
\pgfpathlineto{\pgfqpoint{1.962543in}{0.530961in}}%
\pgfpathlineto{\pgfqpoint{1.967294in}{0.537756in}}%
\pgfpathlineto{\pgfqpoint{1.976086in}{0.538805in}}%
\pgfpathlineto{\pgfqpoint{1.982084in}{0.530926in}}%
\pgfpathlineto{\pgfqpoint{1.981476in}{0.527516in}}%
\pgfpathlineto{\pgfqpoint{1.993498in}{0.524333in}}%
\pgfpathlineto{\pgfqpoint{1.998738in}{0.526353in}}%
\pgfpathlineto{\pgfqpoint{1.997286in}{0.531031in}}%
\pgfpathlineto{\pgfqpoint{1.992985in}{0.532196in}}%
\pgfpathlineto{\pgfqpoint{1.988235in}{0.542241in}}%
\pgfpathlineto{\pgfqpoint{1.990185in}{0.543999in}}%
\pgfpathlineto{\pgfqpoint{2.002634in}{0.548120in}}%
\pgfpathlineto{\pgfqpoint{2.005096in}{0.547382in}}%
\pgfpathlineto{\pgfqpoint{2.005579in}{0.541500in}}%
\pgfpathlineto{\pgfqpoint{2.010082in}{0.536853in}}%
\pgfpathlineto{\pgfqpoint{2.030059in}{0.536911in}}%
\pgfpathlineto{\pgfqpoint{2.034075in}{0.531818in}}%
\pgfpathlineto{\pgfqpoint{2.040150in}{0.528704in}}%
\pgfpathlineto{\pgfqpoint{2.042354in}{0.524234in}}%
\pgfpathlineto{\pgfqpoint{2.050269in}{0.524553in}}%
\pgfpathlineto{\pgfqpoint{2.053131in}{0.522603in}}%
\pgfpathlineto{\pgfqpoint{2.051671in}{0.516053in}}%
\pgfpathlineto{\pgfqpoint{2.054795in}{0.510464in}}%
\pgfpathlineto{\pgfqpoint{2.060711in}{0.508116in}}%
\pgfpathclose%
\pgfusepath{fill}%
\end{pgfscope}%
\begin{pgfscope}%
\pgfpathrectangle{\pgfqpoint{0.100000in}{0.100000in}}{\pgfqpoint{3.007045in}{1.925000in}}%
\pgfusepath{clip}%
\pgfsetbuttcap%
\pgfsetmiterjoin%
\definecolor{currentfill}{rgb}{0.244106,0.557832,0.768889}%
\pgfsetfillcolor{currentfill}%
\pgfsetlinewidth{0.000000pt}%
\definecolor{currentstroke}{rgb}{0.000000,0.000000,0.000000}%
\pgfsetstrokecolor{currentstroke}%
\pgfsetstrokeopacity{0.000000}%
\pgfsetdash{}{0pt}%
\pgfpathmoveto{\pgfqpoint{1.502115in}{0.878788in}}%
\pgfpathlineto{\pgfqpoint{1.504252in}{0.917589in}}%
\pgfpathlineto{\pgfqpoint{1.505202in}{0.934838in}}%
\pgfpathlineto{\pgfqpoint{1.510237in}{0.939913in}}%
\pgfpathlineto{\pgfqpoint{1.515519in}{0.937448in}}%
\pgfpathlineto{\pgfqpoint{1.520249in}{0.932634in}}%
\pgfpathlineto{\pgfqpoint{1.528739in}{0.932611in}}%
\pgfpathlineto{\pgfqpoint{1.532268in}{0.939536in}}%
\pgfpathlineto{\pgfqpoint{1.538682in}{0.941826in}}%
\pgfpathlineto{\pgfqpoint{1.538310in}{0.928586in}}%
\pgfpathlineto{\pgfqpoint{1.578070in}{0.926846in}}%
\pgfpathlineto{\pgfqpoint{1.577863in}{0.909634in}}%
\pgfpathlineto{\pgfqpoint{1.576781in}{0.879670in}}%
\pgfpathlineto{\pgfqpoint{1.569929in}{0.881210in}}%
\pgfpathlineto{\pgfqpoint{1.514160in}{0.883851in}}%
\pgfpathlineto{\pgfqpoint{1.513852in}{0.878138in}}%
\pgfpathlineto{\pgfqpoint{1.502115in}{0.878788in}}%
\pgfpathclose%
\pgfusepath{fill}%
\end{pgfscope}%
\begin{pgfscope}%
\pgfpathrectangle{\pgfqpoint{0.100000in}{0.100000in}}{\pgfqpoint{3.007045in}{1.925000in}}%
\pgfusepath{clip}%
\pgfsetbuttcap%
\pgfsetmiterjoin%
\definecolor{currentfill}{rgb}{0.491765,0.721968,0.854779}%
\pgfsetfillcolor{currentfill}%
\pgfsetlinewidth{0.000000pt}%
\definecolor{currentstroke}{rgb}{0.000000,0.000000,0.000000}%
\pgfsetstrokecolor{currentstroke}%
\pgfsetstrokeopacity{0.000000}%
\pgfsetdash{}{0pt}%
\pgfpathmoveto{\pgfqpoint{2.523965in}{1.201691in}}%
\pgfpathlineto{\pgfqpoint{2.534247in}{1.194214in}}%
\pgfpathlineto{\pgfqpoint{2.532223in}{1.188646in}}%
\pgfpathlineto{\pgfqpoint{2.532999in}{1.178683in}}%
\pgfpathlineto{\pgfqpoint{2.527409in}{1.177624in}}%
\pgfpathlineto{\pgfqpoint{2.525737in}{1.174087in}}%
\pgfpathlineto{\pgfqpoint{2.517598in}{1.175060in}}%
\pgfpathlineto{\pgfqpoint{2.514455in}{1.181489in}}%
\pgfpathlineto{\pgfqpoint{2.507565in}{1.184564in}}%
\pgfpathlineto{\pgfqpoint{2.500337in}{1.190919in}}%
\pgfpathlineto{\pgfqpoint{2.498857in}{1.196503in}}%
\pgfpathlineto{\pgfqpoint{2.505727in}{1.202402in}}%
\pgfpathlineto{\pgfqpoint{2.520381in}{1.200511in}}%
\pgfpathlineto{\pgfqpoint{2.523965in}{1.201691in}}%
\pgfpathclose%
\pgfusepath{fill}%
\end{pgfscope}%
\begin{pgfscope}%
\pgfpathrectangle{\pgfqpoint{0.100000in}{0.100000in}}{\pgfqpoint{3.007045in}{1.925000in}}%
\pgfusepath{clip}%
\pgfsetbuttcap%
\pgfsetmiterjoin%
\definecolor{currentfill}{rgb}{0.441569,0.694410,0.843952}%
\pgfsetfillcolor{currentfill}%
\pgfsetlinewidth{0.000000pt}%
\definecolor{currentstroke}{rgb}{0.000000,0.000000,0.000000}%
\pgfsetstrokecolor{currentstroke}%
\pgfsetstrokeopacity{0.000000}%
\pgfsetdash{}{0pt}%
\pgfpathmoveto{\pgfqpoint{2.362396in}{1.092621in}}%
\pgfpathlineto{\pgfqpoint{2.358589in}{1.084899in}}%
\pgfpathlineto{\pgfqpoint{2.350883in}{1.086320in}}%
\pgfpathlineto{\pgfqpoint{2.342975in}{1.090758in}}%
\pgfpathlineto{\pgfqpoint{2.328598in}{1.094602in}}%
\pgfpathlineto{\pgfqpoint{2.316608in}{1.083386in}}%
\pgfpathlineto{\pgfqpoint{2.314503in}{1.080433in}}%
\pgfpathlineto{\pgfqpoint{2.300122in}{1.093676in}}%
\pgfpathlineto{\pgfqpoint{2.294182in}{1.093073in}}%
\pgfpathlineto{\pgfqpoint{2.289598in}{1.088435in}}%
\pgfpathlineto{\pgfqpoint{2.283241in}{1.088142in}}%
\pgfpathlineto{\pgfqpoint{2.281936in}{1.092901in}}%
\pgfpathlineto{\pgfqpoint{2.287143in}{1.100325in}}%
\pgfpathlineto{\pgfqpoint{2.288667in}{1.106429in}}%
\pgfpathlineto{\pgfqpoint{2.288467in}{1.113713in}}%
\pgfpathlineto{\pgfqpoint{2.292494in}{1.120777in}}%
\pgfpathlineto{\pgfqpoint{2.287382in}{1.127701in}}%
\pgfpathlineto{\pgfqpoint{2.285748in}{1.132441in}}%
\pgfpathlineto{\pgfqpoint{2.282277in}{1.134362in}}%
\pgfpathlineto{\pgfqpoint{2.288955in}{1.139422in}}%
\pgfpathlineto{\pgfqpoint{2.296120in}{1.144067in}}%
\pgfpathlineto{\pgfqpoint{2.297453in}{1.137486in}}%
\pgfpathlineto{\pgfqpoint{2.308461in}{1.129091in}}%
\pgfpathlineto{\pgfqpoint{2.317016in}{1.135906in}}%
\pgfpathlineto{\pgfqpoint{2.326824in}{1.138484in}}%
\pgfpathlineto{\pgfqpoint{2.332965in}{1.140303in}}%
\pgfpathlineto{\pgfqpoint{2.338933in}{1.139975in}}%
\pgfpathlineto{\pgfqpoint{2.341607in}{1.134315in}}%
\pgfpathlineto{\pgfqpoint{2.339247in}{1.131363in}}%
\pgfpathlineto{\pgfqpoint{2.347373in}{1.122044in}}%
\pgfpathlineto{\pgfqpoint{2.336897in}{1.108708in}}%
\pgfpathlineto{\pgfqpoint{2.344556in}{1.097617in}}%
\pgfpathlineto{\pgfqpoint{2.355335in}{1.096669in}}%
\pgfpathlineto{\pgfqpoint{2.362396in}{1.092621in}}%
\pgfpathclose%
\pgfusepath{fill}%
\end{pgfscope}%
\begin{pgfscope}%
\pgfpathrectangle{\pgfqpoint{0.100000in}{0.100000in}}{\pgfqpoint{3.007045in}{1.925000in}}%
\pgfusepath{clip}%
\pgfsetbuttcap%
\pgfsetmiterjoin%
\definecolor{currentfill}{rgb}{0.376732,0.653072,0.822484}%
\pgfsetfillcolor{currentfill}%
\pgfsetlinewidth{0.000000pt}%
\definecolor{currentstroke}{rgb}{0.000000,0.000000,0.000000}%
\pgfsetstrokecolor{currentstroke}%
\pgfsetstrokeopacity{0.000000}%
\pgfsetdash{}{0pt}%
\pgfpathmoveto{\pgfqpoint{2.231222in}{1.236425in}}%
\pgfpathlineto{\pgfqpoint{2.212181in}{1.234336in}}%
\pgfpathlineto{\pgfqpoint{2.211166in}{1.244697in}}%
\pgfpathlineto{\pgfqpoint{2.204170in}{1.247864in}}%
\pgfpathlineto{\pgfqpoint{2.202654in}{1.264977in}}%
\pgfpathlineto{\pgfqpoint{2.195116in}{1.264251in}}%
\pgfpathlineto{\pgfqpoint{2.192034in}{1.266835in}}%
\pgfpathlineto{\pgfqpoint{2.190977in}{1.278330in}}%
\pgfpathlineto{\pgfqpoint{2.173650in}{1.276857in}}%
\pgfpathlineto{\pgfqpoint{2.172150in}{1.293918in}}%
\pgfpathlineto{\pgfqpoint{2.195043in}{1.295965in}}%
\pgfpathlineto{\pgfqpoint{2.196588in}{1.278782in}}%
\pgfpathlineto{\pgfqpoint{2.211433in}{1.280191in}}%
\pgfpathlineto{\pgfqpoint{2.210917in}{1.285878in}}%
\pgfpathlineto{\pgfqpoint{2.221882in}{1.287153in}}%
\pgfpathlineto{\pgfqpoint{2.224496in}{1.264365in}}%
\pgfpathlineto{\pgfqpoint{2.228217in}{1.264785in}}%
\pgfpathlineto{\pgfqpoint{2.231222in}{1.236425in}}%
\pgfpathclose%
\pgfusepath{fill}%
\end{pgfscope}%
\begin{pgfscope}%
\pgfpathrectangle{\pgfqpoint{0.100000in}{0.100000in}}{\pgfqpoint{3.007045in}{1.925000in}}%
\pgfusepath{clip}%
\pgfsetbuttcap%
\pgfsetmiterjoin%
\definecolor{currentfill}{rgb}{0.422745,0.684075,0.839892}%
\pgfsetfillcolor{currentfill}%
\pgfsetlinewidth{0.000000pt}%
\definecolor{currentstroke}{rgb}{0.000000,0.000000,0.000000}%
\pgfsetstrokecolor{currentstroke}%
\pgfsetstrokeopacity{0.000000}%
\pgfsetdash{}{0pt}%
\pgfpathmoveto{\pgfqpoint{1.980505in}{0.873918in}}%
\pgfpathlineto{\pgfqpoint{1.972810in}{0.872450in}}%
\pgfpathlineto{\pgfqpoint{1.968811in}{0.876703in}}%
\pgfpathlineto{\pgfqpoint{1.963244in}{0.878965in}}%
\pgfpathlineto{\pgfqpoint{1.956817in}{0.878848in}}%
\pgfpathlineto{\pgfqpoint{1.956926in}{0.874466in}}%
\pgfpathlineto{\pgfqpoint{1.950043in}{0.876695in}}%
\pgfpathlineto{\pgfqpoint{1.945167in}{0.874395in}}%
\pgfpathlineto{\pgfqpoint{1.941240in}{0.873795in}}%
\pgfpathlineto{\pgfqpoint{1.933967in}{0.876483in}}%
\pgfpathlineto{\pgfqpoint{1.927950in}{0.876198in}}%
\pgfpathlineto{\pgfqpoint{1.927775in}{0.895762in}}%
\pgfpathlineto{\pgfqpoint{1.941745in}{0.895933in}}%
\pgfpathlineto{\pgfqpoint{1.944537in}{0.902457in}}%
\pgfpathlineto{\pgfqpoint{1.944456in}{0.907639in}}%
\pgfpathlineto{\pgfqpoint{1.967250in}{0.907985in}}%
\pgfpathlineto{\pgfqpoint{1.966993in}{0.915480in}}%
\pgfpathlineto{\pgfqpoint{1.969720in}{0.923885in}}%
\pgfpathlineto{\pgfqpoint{1.975423in}{0.932334in}}%
\pgfpathlineto{\pgfqpoint{1.984303in}{0.932387in}}%
\pgfpathlineto{\pgfqpoint{1.985207in}{0.903228in}}%
\pgfpathlineto{\pgfqpoint{1.986161in}{0.874058in}}%
\pgfpathlineto{\pgfqpoint{1.980505in}{0.873918in}}%
\pgfpathclose%
\pgfusepath{fill}%
\end{pgfscope}%
\begin{pgfscope}%
\pgfpathrectangle{\pgfqpoint{0.100000in}{0.100000in}}{\pgfqpoint{3.007045in}{1.925000in}}%
\pgfusepath{clip}%
\pgfsetbuttcap%
\pgfsetmiterjoin%
\definecolor{currentfill}{rgb}{0.346467,0.632403,0.810673}%
\pgfsetfillcolor{currentfill}%
\pgfsetlinewidth{0.000000pt}%
\definecolor{currentstroke}{rgb}{0.000000,0.000000,0.000000}%
\pgfsetstrokecolor{currentstroke}%
\pgfsetstrokeopacity{0.000000}%
\pgfsetdash{}{0pt}%
\pgfpathmoveto{\pgfqpoint{1.553928in}{0.768166in}}%
\pgfpathlineto{\pgfqpoint{1.555230in}{0.797022in}}%
\pgfpathlineto{\pgfqpoint{1.556351in}{0.822025in}}%
\pgfpathlineto{\pgfqpoint{1.560430in}{0.818079in}}%
\pgfpathlineto{\pgfqpoint{1.570469in}{0.816172in}}%
\pgfpathlineto{\pgfqpoint{1.574901in}{0.817578in}}%
\pgfpathlineto{\pgfqpoint{1.581430in}{0.811070in}}%
\pgfpathlineto{\pgfqpoint{1.586260in}{0.813360in}}%
\pgfpathlineto{\pgfqpoint{1.588268in}{0.817057in}}%
\pgfpathlineto{\pgfqpoint{1.598818in}{0.813845in}}%
\pgfpathlineto{\pgfqpoint{1.605264in}{0.805706in}}%
\pgfpathlineto{\pgfqpoint{1.610850in}{0.805245in}}%
\pgfpathlineto{\pgfqpoint{1.608887in}{0.798669in}}%
\pgfpathlineto{\pgfqpoint{1.607913in}{0.770715in}}%
\pgfpathlineto{\pgfqpoint{1.583500in}{0.771553in}}%
\pgfpathlineto{\pgfqpoint{1.583365in}{0.766882in}}%
\pgfpathlineto{\pgfqpoint{1.553928in}{0.768166in}}%
\pgfpathclose%
\pgfusepath{fill}%
\end{pgfscope}%
\begin{pgfscope}%
\pgfpathrectangle{\pgfqpoint{0.100000in}{0.100000in}}{\pgfqpoint{3.007045in}{1.925000in}}%
\pgfusepath{clip}%
\pgfsetbuttcap%
\pgfsetmiterjoin%
\definecolor{currentfill}{rgb}{0.326290,0.618624,0.802799}%
\pgfsetfillcolor{currentfill}%
\pgfsetlinewidth{0.000000pt}%
\definecolor{currentstroke}{rgb}{0.000000,0.000000,0.000000}%
\pgfsetstrokecolor{currentstroke}%
\pgfsetstrokeopacity{0.000000}%
\pgfsetdash{}{0pt}%
\pgfpathmoveto{\pgfqpoint{1.872670in}{1.072352in}}%
\pgfpathlineto{\pgfqpoint{1.849904in}{1.072618in}}%
\pgfpathlineto{\pgfqpoint{1.849656in}{1.062106in}}%
\pgfpathlineto{\pgfqpoint{1.846715in}{1.062147in}}%
\pgfpathlineto{\pgfqpoint{1.843696in}{1.056469in}}%
\pgfpathlineto{\pgfqpoint{1.834159in}{1.056680in}}%
\pgfpathlineto{\pgfqpoint{1.834274in}{1.060508in}}%
\pgfpathlineto{\pgfqpoint{1.820896in}{1.060894in}}%
\pgfpathlineto{\pgfqpoint{1.821187in}{1.069825in}}%
\pgfpathlineto{\pgfqpoint{1.820190in}{1.081544in}}%
\pgfpathlineto{\pgfqpoint{1.820514in}{1.104792in}}%
\pgfpathlineto{\pgfqpoint{1.849007in}{1.104433in}}%
\pgfpathlineto{\pgfqpoint{1.849349in}{1.129915in}}%
\pgfpathlineto{\pgfqpoint{1.872212in}{1.129384in}}%
\pgfpathlineto{\pgfqpoint{1.872048in}{1.113887in}}%
\pgfpathlineto{\pgfqpoint{1.871618in}{1.080502in}}%
\pgfpathlineto{\pgfqpoint{1.872670in}{1.072352in}}%
\pgfpathclose%
\pgfusepath{fill}%
\end{pgfscope}%
\begin{pgfscope}%
\pgfpathrectangle{\pgfqpoint{0.100000in}{0.100000in}}{\pgfqpoint{3.007045in}{1.925000in}}%
\pgfusepath{clip}%
\pgfsetbuttcap%
\pgfsetmiterjoin%
\definecolor{currentfill}{rgb}{0.567059,0.763306,0.871019}%
\pgfsetfillcolor{currentfill}%
\pgfsetlinewidth{0.000000pt}%
\definecolor{currentstroke}{rgb}{0.000000,0.000000,0.000000}%
\pgfsetstrokecolor{currentstroke}%
\pgfsetstrokeopacity{0.000000}%
\pgfsetdash{}{0pt}%
\pgfpathmoveto{\pgfqpoint{0.433096in}{0.983854in}}%
\pgfpathlineto{\pgfqpoint{0.425437in}{0.987474in}}%
\pgfpathlineto{\pgfqpoint{0.427947in}{0.992997in}}%
\pgfpathlineto{\pgfqpoint{0.436545in}{0.987083in}}%
\pgfpathlineto{\pgfqpoint{0.433096in}{0.983854in}}%
\pgfpathclose%
\pgfusepath{fill}%
\end{pgfscope}%
\begin{pgfscope}%
\pgfpathrectangle{\pgfqpoint{0.100000in}{0.100000in}}{\pgfqpoint{3.007045in}{1.925000in}}%
\pgfusepath{clip}%
\pgfsetbuttcap%
\pgfsetmiterjoin%
\definecolor{currentfill}{rgb}{0.567059,0.763306,0.871019}%
\pgfsetfillcolor{currentfill}%
\pgfsetlinewidth{0.000000pt}%
\definecolor{currentstroke}{rgb}{0.000000,0.000000,0.000000}%
\pgfsetstrokecolor{currentstroke}%
\pgfsetstrokeopacity{0.000000}%
\pgfsetdash{}{0pt}%
\pgfpathmoveto{\pgfqpoint{0.411502in}{0.985534in}}%
\pgfpathlineto{\pgfqpoint{0.407141in}{0.993168in}}%
\pgfpathlineto{\pgfqpoint{0.412033in}{0.994414in}}%
\pgfpathlineto{\pgfqpoint{0.419874in}{0.989359in}}%
\pgfpathlineto{\pgfqpoint{0.418366in}{0.986829in}}%
\pgfpathlineto{\pgfqpoint{0.411502in}{0.985534in}}%
\pgfpathclose%
\pgfusepath{fill}%
\end{pgfscope}%
\begin{pgfscope}%
\pgfpathrectangle{\pgfqpoint{0.100000in}{0.100000in}}{\pgfqpoint{3.007045in}{1.925000in}}%
\pgfusepath{clip}%
\pgfsetbuttcap%
\pgfsetmiterjoin%
\definecolor{currentfill}{rgb}{0.567059,0.763306,0.871019}%
\pgfsetfillcolor{currentfill}%
\pgfsetlinewidth{0.000000pt}%
\definecolor{currentstroke}{rgb}{0.000000,0.000000,0.000000}%
\pgfsetstrokecolor{currentstroke}%
\pgfsetstrokeopacity{0.000000}%
\pgfsetdash{}{0pt}%
\pgfpathmoveto{\pgfqpoint{0.464444in}{1.039777in}}%
\pgfpathlineto{\pgfqpoint{0.456601in}{1.011141in}}%
\pgfpathlineto{\pgfqpoint{0.453370in}{1.006642in}}%
\pgfpathlineto{\pgfqpoint{0.447005in}{1.011573in}}%
\pgfpathlineto{\pgfqpoint{0.440321in}{1.011780in}}%
\pgfpathlineto{\pgfqpoint{0.433003in}{1.014607in}}%
\pgfpathlineto{\pgfqpoint{0.427568in}{1.019544in}}%
\pgfpathlineto{\pgfqpoint{0.416032in}{1.023886in}}%
\pgfpathlineto{\pgfqpoint{0.403356in}{1.025371in}}%
\pgfpathlineto{\pgfqpoint{0.401817in}{1.031469in}}%
\pgfpathlineto{\pgfqpoint{0.395689in}{1.036947in}}%
\pgfpathlineto{\pgfqpoint{0.400361in}{1.044762in}}%
\pgfpathlineto{\pgfqpoint{0.399420in}{1.048155in}}%
\pgfpathlineto{\pgfqpoint{0.402613in}{1.053373in}}%
\pgfpathlineto{\pgfqpoint{0.400409in}{1.057912in}}%
\pgfpathlineto{\pgfqpoint{0.403882in}{1.063816in}}%
\pgfpathlineto{\pgfqpoint{0.405983in}{1.073145in}}%
\pgfpathlineto{\pgfqpoint{0.400206in}{1.075730in}}%
\pgfpathlineto{\pgfqpoint{0.395182in}{1.083867in}}%
\pgfpathlineto{\pgfqpoint{0.399136in}{1.090944in}}%
\pgfpathlineto{\pgfqpoint{0.398305in}{1.096127in}}%
\pgfpathlineto{\pgfqpoint{0.393442in}{1.098296in}}%
\pgfpathlineto{\pgfqpoint{0.388518in}{1.111365in}}%
\pgfpathlineto{\pgfqpoint{0.382876in}{1.115594in}}%
\pgfpathlineto{\pgfqpoint{0.382061in}{1.124859in}}%
\pgfpathlineto{\pgfqpoint{0.406868in}{1.117231in}}%
\pgfpathlineto{\pgfqpoint{0.440258in}{1.107424in}}%
\pgfpathlineto{\pgfqpoint{0.441277in}{1.107129in}}%
\pgfpathlineto{\pgfqpoint{0.438131in}{1.096030in}}%
\pgfpathlineto{\pgfqpoint{0.443707in}{1.094459in}}%
\pgfpathlineto{\pgfqpoint{0.442087in}{1.088884in}}%
\pgfpathlineto{\pgfqpoint{0.445664in}{1.083905in}}%
\pgfpathlineto{\pgfqpoint{0.451207in}{1.080317in}}%
\pgfpathlineto{\pgfqpoint{0.449631in}{1.074733in}}%
\pgfpathlineto{\pgfqpoint{0.453296in}{1.073681in}}%
\pgfpathlineto{\pgfqpoint{0.451737in}{1.068135in}}%
\pgfpathlineto{\pgfqpoint{0.459131in}{1.065999in}}%
\pgfpathlineto{\pgfqpoint{0.457558in}{1.060425in}}%
\pgfpathlineto{\pgfqpoint{0.463546in}{1.059117in}}%
\pgfpathlineto{\pgfqpoint{0.461538in}{1.053350in}}%
\pgfpathlineto{\pgfqpoint{0.465944in}{1.051393in}}%
\pgfpathlineto{\pgfqpoint{0.464444in}{1.039777in}}%
\pgfpathclose%
\pgfusepath{fill}%
\end{pgfscope}%
\begin{pgfscope}%
\pgfpathrectangle{\pgfqpoint{0.100000in}{0.100000in}}{\pgfqpoint{3.007045in}{1.925000in}}%
\pgfusepath{clip}%
\pgfsetbuttcap%
\pgfsetmiterjoin%
\definecolor{currentfill}{rgb}{0.735871,0.841569,0.923045}%
\pgfsetfillcolor{currentfill}%
\pgfsetlinewidth{0.000000pt}%
\definecolor{currentstroke}{rgb}{0.000000,0.000000,0.000000}%
\pgfsetstrokecolor{currentstroke}%
\pgfsetstrokeopacity{0.000000}%
\pgfsetdash{}{0pt}%
\pgfpathmoveto{\pgfqpoint{2.825651in}{1.545463in}}%
\pgfpathlineto{\pgfqpoint{2.807383in}{1.541913in}}%
\pgfpathlineto{\pgfqpoint{2.803917in}{1.557332in}}%
\pgfpathlineto{\pgfqpoint{2.794553in}{1.564410in}}%
\pgfpathlineto{\pgfqpoint{2.793875in}{1.572201in}}%
\pgfpathlineto{\pgfqpoint{2.787488in}{1.586749in}}%
\pgfpathlineto{\pgfqpoint{2.787394in}{1.596652in}}%
\pgfpathlineto{\pgfqpoint{2.789757in}{1.602899in}}%
\pgfpathlineto{\pgfqpoint{2.787188in}{1.608949in}}%
\pgfpathlineto{\pgfqpoint{2.786419in}{1.617481in}}%
\pgfpathlineto{\pgfqpoint{2.780749in}{1.624141in}}%
\pgfpathlineto{\pgfqpoint{2.780372in}{1.636703in}}%
\pgfpathlineto{\pgfqpoint{2.777816in}{1.638392in}}%
\pgfpathlineto{\pgfqpoint{2.778407in}{1.644079in}}%
\pgfpathlineto{\pgfqpoint{2.776771in}{1.649915in}}%
\pgfpathlineto{\pgfqpoint{2.809280in}{1.658194in}}%
\pgfpathlineto{\pgfqpoint{2.812663in}{1.658624in}}%
\pgfpathlineto{\pgfqpoint{2.816481in}{1.647652in}}%
\pgfpathlineto{\pgfqpoint{2.815249in}{1.643795in}}%
\pgfpathlineto{\pgfqpoint{2.823531in}{1.641882in}}%
\pgfpathlineto{\pgfqpoint{2.822013in}{1.635422in}}%
\pgfpathlineto{\pgfqpoint{2.827912in}{1.634106in}}%
\pgfpathlineto{\pgfqpoint{2.826283in}{1.627777in}}%
\pgfpathlineto{\pgfqpoint{2.825014in}{1.621968in}}%
\pgfpathlineto{\pgfqpoint{2.819281in}{1.624179in}}%
\pgfpathlineto{\pgfqpoint{2.817950in}{1.618314in}}%
\pgfpathlineto{\pgfqpoint{2.810189in}{1.620019in}}%
\pgfpathlineto{\pgfqpoint{2.808017in}{1.607896in}}%
\pgfpathlineto{\pgfqpoint{2.808838in}{1.600051in}}%
\pgfpathlineto{\pgfqpoint{2.811115in}{1.594372in}}%
\pgfpathlineto{\pgfqpoint{2.819866in}{1.593654in}}%
\pgfpathlineto{\pgfqpoint{2.815683in}{1.583923in}}%
\pgfpathlineto{\pgfqpoint{2.811802in}{1.580671in}}%
\pgfpathlineto{\pgfqpoint{2.821564in}{1.578723in}}%
\pgfpathlineto{\pgfqpoint{2.821240in}{1.572694in}}%
\pgfpathlineto{\pgfqpoint{2.827305in}{1.571893in}}%
\pgfpathlineto{\pgfqpoint{2.826130in}{1.558806in}}%
\pgfpathlineto{\pgfqpoint{2.829522in}{1.557864in}}%
\pgfpathlineto{\pgfqpoint{2.829267in}{1.550294in}}%
\pgfpathlineto{\pgfqpoint{2.825651in}{1.545463in}}%
\pgfpathclose%
\pgfusepath{fill}%
\end{pgfscope}%
\begin{pgfscope}%
\pgfpathrectangle{\pgfqpoint{0.100000in}{0.100000in}}{\pgfqpoint{3.007045in}{1.925000in}}%
\pgfusepath{clip}%
\pgfsetbuttcap%
\pgfsetmiterjoin%
\definecolor{currentfill}{rgb}{0.401953,0.670296,0.832326}%
\pgfsetfillcolor{currentfill}%
\pgfsetlinewidth{0.000000pt}%
\definecolor{currentstroke}{rgb}{0.000000,0.000000,0.000000}%
\pgfsetstrokecolor{currentstroke}%
\pgfsetstrokeopacity{0.000000}%
\pgfsetdash{}{0pt}%
\pgfpathmoveto{\pgfqpoint{1.969121in}{0.688091in}}%
\pgfpathlineto{\pgfqpoint{1.966832in}{0.693993in}}%
\pgfpathlineto{\pgfqpoint{1.970011in}{0.695703in}}%
\pgfpathlineto{\pgfqpoint{1.968808in}{0.707431in}}%
\pgfpathlineto{\pgfqpoint{1.970224in}{0.710328in}}%
\pgfpathlineto{\pgfqpoint{1.988311in}{0.711075in}}%
\pgfpathlineto{\pgfqpoint{1.993233in}{0.713829in}}%
\pgfpathlineto{\pgfqpoint{1.997671in}{0.713980in}}%
\pgfpathlineto{\pgfqpoint{1.998667in}{0.705092in}}%
\pgfpathlineto{\pgfqpoint{2.003803in}{0.705398in}}%
\pgfpathlineto{\pgfqpoint{2.003446in}{0.714212in}}%
\pgfpathlineto{\pgfqpoint{2.010976in}{0.714602in}}%
\pgfpathlineto{\pgfqpoint{2.011147in}{0.717467in}}%
\pgfpathlineto{\pgfqpoint{2.020779in}{0.710545in}}%
\pgfpathlineto{\pgfqpoint{2.018058in}{0.708669in}}%
\pgfpathlineto{\pgfqpoint{2.016613in}{0.701553in}}%
\pgfpathlineto{\pgfqpoint{2.012854in}{0.699525in}}%
\pgfpathlineto{\pgfqpoint{2.011898in}{0.691368in}}%
\pgfpathlineto{\pgfqpoint{2.013470in}{0.675004in}}%
\pgfpathlineto{\pgfqpoint{2.013078in}{0.656435in}}%
\pgfpathlineto{\pgfqpoint{2.013321in}{0.650638in}}%
\pgfpathlineto{\pgfqpoint{1.989789in}{0.649517in}}%
\pgfpathlineto{\pgfqpoint{1.981608in}{0.653144in}}%
\pgfpathlineto{\pgfqpoint{1.978327in}{0.664514in}}%
\pgfpathlineto{\pgfqpoint{1.986197in}{0.669162in}}%
\pgfpathlineto{\pgfqpoint{1.986876in}{0.672737in}}%
\pgfpathlineto{\pgfqpoint{1.993029in}{0.676683in}}%
\pgfpathlineto{\pgfqpoint{1.995075in}{0.682326in}}%
\pgfpathlineto{\pgfqpoint{1.987667in}{0.684255in}}%
\pgfpathlineto{\pgfqpoint{1.989120in}{0.691955in}}%
\pgfpathlineto{\pgfqpoint{1.978768in}{0.688413in}}%
\pgfpathlineto{\pgfqpoint{1.969121in}{0.688091in}}%
\pgfpathclose%
\pgfusepath{fill}%
\end{pgfscope}%
\begin{pgfscope}%
\pgfpathrectangle{\pgfqpoint{0.100000in}{0.100000in}}{\pgfqpoint{3.007045in}{1.925000in}}%
\pgfusepath{clip}%
\pgfsetbuttcap%
\pgfsetmiterjoin%
\definecolor{currentfill}{rgb}{0.059439,0.353572,0.639831}%
\pgfsetfillcolor{currentfill}%
\pgfsetlinewidth{0.000000pt}%
\definecolor{currentstroke}{rgb}{0.000000,0.000000,0.000000}%
\pgfsetstrokecolor{currentstroke}%
\pgfsetstrokeopacity{0.000000}%
\pgfsetdash{}{0pt}%
\pgfpathmoveto{\pgfqpoint{0.441277in}{1.107129in}}%
\pgfpathlineto{\pgfqpoint{0.440258in}{1.107424in}}%
\pgfpathlineto{\pgfqpoint{0.435906in}{1.120684in}}%
\pgfpathlineto{\pgfqpoint{0.432260in}{1.122040in}}%
\pgfpathlineto{\pgfqpoint{0.428309in}{1.126977in}}%
\pgfpathlineto{\pgfqpoint{0.423734in}{1.137658in}}%
\pgfpathlineto{\pgfqpoint{0.426809in}{1.147328in}}%
\pgfpathlineto{\pgfqpoint{0.430464in}{1.147237in}}%
\pgfpathlineto{\pgfqpoint{0.433028in}{1.157462in}}%
\pgfpathlineto{\pgfqpoint{0.421673in}{1.178083in}}%
\pgfpathlineto{\pgfqpoint{0.450589in}{1.194118in}}%
\pgfpathlineto{\pgfqpoint{0.461047in}{1.194974in}}%
\pgfpathlineto{\pgfqpoint{0.473609in}{1.193492in}}%
\pgfpathlineto{\pgfqpoint{0.492430in}{1.204219in}}%
\pgfpathlineto{\pgfqpoint{0.498018in}{1.202710in}}%
\pgfpathlineto{\pgfqpoint{0.502749in}{1.206653in}}%
\pgfpathlineto{\pgfqpoint{0.503936in}{1.210812in}}%
\pgfpathlineto{\pgfqpoint{0.521553in}{1.220832in}}%
\pgfpathlineto{\pgfqpoint{0.522889in}{1.217818in}}%
\pgfpathlineto{\pgfqpoint{0.530069in}{1.215499in}}%
\pgfpathlineto{\pgfqpoint{0.532191in}{1.211551in}}%
\pgfpathlineto{\pgfqpoint{0.532644in}{1.204682in}}%
\pgfpathlineto{\pgfqpoint{0.537345in}{1.200997in}}%
\pgfpathlineto{\pgfqpoint{0.539193in}{1.195591in}}%
\pgfpathlineto{\pgfqpoint{0.543469in}{1.191912in}}%
\pgfpathlineto{\pgfqpoint{0.540581in}{1.185427in}}%
\pgfpathlineto{\pgfqpoint{0.543712in}{1.184135in}}%
\pgfpathlineto{\pgfqpoint{0.543660in}{1.171437in}}%
\pgfpathlineto{\pgfqpoint{0.546742in}{1.170366in}}%
\pgfpathlineto{\pgfqpoint{0.552868in}{1.161985in}}%
\pgfpathlineto{\pgfqpoint{0.554321in}{1.151293in}}%
\pgfpathlineto{\pgfqpoint{0.551954in}{1.148285in}}%
\pgfpathlineto{\pgfqpoint{0.551392in}{1.137382in}}%
\pgfpathlineto{\pgfqpoint{0.554104in}{1.127797in}}%
\pgfpathlineto{\pgfqpoint{0.553369in}{1.124234in}}%
\pgfpathlineto{\pgfqpoint{0.558630in}{1.113262in}}%
\pgfpathlineto{\pgfqpoint{0.556496in}{1.110688in}}%
\pgfpathlineto{\pgfqpoint{0.557824in}{1.101480in}}%
\pgfpathlineto{\pgfqpoint{0.556142in}{1.096690in}}%
\pgfpathlineto{\pgfqpoint{0.557260in}{1.082859in}}%
\pgfpathlineto{\pgfqpoint{0.554172in}{1.076466in}}%
\pgfpathlineto{\pgfqpoint{0.522151in}{1.084892in}}%
\pgfpathlineto{\pgfqpoint{0.475084in}{1.097667in}}%
\pgfpathlineto{\pgfqpoint{0.441277in}{1.107129in}}%
\pgfpathclose%
\pgfusepath{fill}%
\end{pgfscope}%
\begin{pgfscope}%
\pgfpathrectangle{\pgfqpoint{0.100000in}{0.100000in}}{\pgfqpoint{3.007045in}{1.925000in}}%
\pgfusepath{clip}%
\pgfsetbuttcap%
\pgfsetmiterjoin%
\definecolor{currentfill}{rgb}{0.417086,0.680631,0.838231}%
\pgfsetfillcolor{currentfill}%
\pgfsetlinewidth{0.000000pt}%
\definecolor{currentstroke}{rgb}{0.000000,0.000000,0.000000}%
\pgfsetstrokecolor{currentstroke}%
\pgfsetstrokeopacity{0.000000}%
\pgfsetdash{}{0pt}%
\pgfpathmoveto{\pgfqpoint{0.856506in}{1.901711in}}%
\pgfpathlineto{\pgfqpoint{0.892466in}{1.893289in}}%
\pgfpathlineto{\pgfqpoint{0.913025in}{1.888772in}}%
\pgfpathlineto{\pgfqpoint{0.912328in}{1.878097in}}%
\pgfpathlineto{\pgfqpoint{0.913309in}{1.873552in}}%
\pgfpathlineto{\pgfqpoint{0.910046in}{1.867809in}}%
\pgfpathlineto{\pgfqpoint{0.901194in}{1.867977in}}%
\pgfpathlineto{\pgfqpoint{0.901644in}{1.862062in}}%
\pgfpathlineto{\pgfqpoint{0.896692in}{1.839394in}}%
\pgfpathlineto{\pgfqpoint{0.889217in}{1.841046in}}%
\pgfpathlineto{\pgfqpoint{0.886583in}{1.827387in}}%
\pgfpathlineto{\pgfqpoint{0.880330in}{1.828801in}}%
\pgfpathlineto{\pgfqpoint{0.877958in}{1.823506in}}%
\pgfpathlineto{\pgfqpoint{0.873246in}{1.821840in}}%
\pgfpathlineto{\pgfqpoint{0.862566in}{1.825406in}}%
\pgfpathlineto{\pgfqpoint{0.863113in}{1.830232in}}%
\pgfpathlineto{\pgfqpoint{0.860078in}{1.837613in}}%
\pgfpathlineto{\pgfqpoint{0.859696in}{1.849285in}}%
\pgfpathlineto{\pgfqpoint{0.845385in}{1.846522in}}%
\pgfpathlineto{\pgfqpoint{0.844425in}{1.850601in}}%
\pgfpathlineto{\pgfqpoint{0.848850in}{1.869247in}}%
\pgfpathlineto{\pgfqpoint{0.856506in}{1.901711in}}%
\pgfpathclose%
\pgfusepath{fill}%
\end{pgfscope}%
\begin{pgfscope}%
\pgfpathrectangle{\pgfqpoint{0.100000in}{0.100000in}}{\pgfqpoint{3.007045in}{1.925000in}}%
\pgfusepath{clip}%
\pgfsetbuttcap%
\pgfsetmiterjoin%
\definecolor{currentfill}{rgb}{0.386820,0.659962,0.826421}%
\pgfsetfillcolor{currentfill}%
\pgfsetlinewidth{0.000000pt}%
\definecolor{currentstroke}{rgb}{0.000000,0.000000,0.000000}%
\pgfsetstrokecolor{currentstroke}%
\pgfsetstrokeopacity{0.000000}%
\pgfsetdash{}{0pt}%
\pgfpathmoveto{\pgfqpoint{2.113075in}{0.848655in}}%
\pgfpathlineto{\pgfqpoint{2.112919in}{0.865058in}}%
\pgfpathlineto{\pgfqpoint{2.110591in}{0.871650in}}%
\pgfpathlineto{\pgfqpoint{2.106779in}{0.871379in}}%
\pgfpathlineto{\pgfqpoint{2.106178in}{0.880380in}}%
\pgfpathlineto{\pgfqpoint{2.108156in}{0.880516in}}%
\pgfpathlineto{\pgfqpoint{2.107278in}{0.897155in}}%
\pgfpathlineto{\pgfqpoint{2.110613in}{0.897382in}}%
\pgfpathlineto{\pgfqpoint{2.114334in}{0.905115in}}%
\pgfpathlineto{\pgfqpoint{2.117945in}{0.906798in}}%
\pgfpathlineto{\pgfqpoint{2.129176in}{0.907553in}}%
\pgfpathlineto{\pgfqpoint{2.129095in}{0.910042in}}%
\pgfpathlineto{\pgfqpoint{2.142348in}{0.908782in}}%
\pgfpathlineto{\pgfqpoint{2.147248in}{0.909640in}}%
\pgfpathlineto{\pgfqpoint{2.150055in}{0.903459in}}%
\pgfpathlineto{\pgfqpoint{2.151351in}{0.884373in}}%
\pgfpathlineto{\pgfqpoint{2.139583in}{0.883616in}}%
\pgfpathlineto{\pgfqpoint{2.145830in}{0.876417in}}%
\pgfpathlineto{\pgfqpoint{2.144807in}{0.847920in}}%
\pgfpathlineto{\pgfqpoint{2.123839in}{0.846519in}}%
\pgfpathlineto{\pgfqpoint{2.123649in}{0.849414in}}%
\pgfpathlineto{\pgfqpoint{2.113075in}{0.848655in}}%
\pgfpathclose%
\pgfusepath{fill}%
\end{pgfscope}%
\begin{pgfscope}%
\pgfpathrectangle{\pgfqpoint{0.100000in}{0.100000in}}{\pgfqpoint{3.007045in}{1.925000in}}%
\pgfusepath{clip}%
\pgfsetbuttcap%
\pgfsetmiterjoin%
\definecolor{currentfill}{rgb}{0.888658,0.933133,0.974410}%
\pgfsetfillcolor{currentfill}%
\pgfsetlinewidth{0.000000pt}%
\definecolor{currentstroke}{rgb}{0.000000,0.000000,0.000000}%
\pgfsetstrokecolor{currentstroke}%
\pgfsetstrokeopacity{0.000000}%
\pgfsetdash{}{0pt}%
\pgfpathmoveto{\pgfqpoint{0.556363in}{1.910254in}}%
\pgfpathlineto{\pgfqpoint{0.556850in}{1.906039in}}%
\pgfpathlineto{\pgfqpoint{0.549783in}{1.902072in}}%
\pgfpathlineto{\pgfqpoint{0.546025in}{1.896043in}}%
\pgfpathlineto{\pgfqpoint{0.536363in}{1.893635in}}%
\pgfpathlineto{\pgfqpoint{0.512685in}{1.901288in}}%
\pgfpathlineto{\pgfqpoint{0.510848in}{1.895622in}}%
\pgfpathlineto{\pgfqpoint{0.487655in}{1.903257in}}%
\pgfpathlineto{\pgfqpoint{0.474818in}{1.908630in}}%
\pgfpathlineto{\pgfqpoint{0.476195in}{1.922659in}}%
\pgfpathlineto{\pgfqpoint{0.474995in}{1.928796in}}%
\pgfpathlineto{\pgfqpoint{0.470655in}{1.935298in}}%
\pgfpathlineto{\pgfqpoint{0.470537in}{1.940585in}}%
\pgfpathlineto{\pgfqpoint{0.473848in}{1.954693in}}%
\pgfpathlineto{\pgfqpoint{0.481028in}{1.966614in}}%
\pgfpathlineto{\pgfqpoint{0.489160in}{1.956885in}}%
\pgfpathlineto{\pgfqpoint{0.494976in}{1.953083in}}%
\pgfpathlineto{\pgfqpoint{0.502419in}{1.944351in}}%
\pgfpathlineto{\pgfqpoint{0.509754in}{1.940761in}}%
\pgfpathlineto{\pgfqpoint{0.515790in}{1.939825in}}%
\pgfpathlineto{\pgfqpoint{0.520125in}{1.935808in}}%
\pgfpathlineto{\pgfqpoint{0.533931in}{1.930174in}}%
\pgfpathlineto{\pgfqpoint{0.539348in}{1.932466in}}%
\pgfpathlineto{\pgfqpoint{0.542272in}{1.925230in}}%
\pgfpathlineto{\pgfqpoint{0.555512in}{1.925149in}}%
\pgfpathlineto{\pgfqpoint{0.557364in}{1.921639in}}%
\pgfpathlineto{\pgfqpoint{0.555607in}{1.915548in}}%
\pgfpathlineto{\pgfqpoint{0.556363in}{1.910254in}}%
\pgfpathclose%
\pgfusepath{fill}%
\end{pgfscope}%
\begin{pgfscope}%
\pgfpathrectangle{\pgfqpoint{0.100000in}{0.100000in}}{\pgfqpoint{3.007045in}{1.925000in}}%
\pgfusepath{clip}%
\pgfsetbuttcap%
\pgfsetmiterjoin%
\definecolor{currentfill}{rgb}{0.396909,0.666851,0.830358}%
\pgfsetfillcolor{currentfill}%
\pgfsetlinewidth{0.000000pt}%
\definecolor{currentstroke}{rgb}{0.000000,0.000000,0.000000}%
\pgfsetstrokecolor{currentstroke}%
\pgfsetstrokeopacity{0.000000}%
\pgfsetdash{}{0pt}%
\pgfpathmoveto{\pgfqpoint{2.288393in}{1.199350in}}%
\pgfpathlineto{\pgfqpoint{2.288706in}{1.196382in}}%
\pgfpathlineto{\pgfqpoint{2.264298in}{1.193927in}}%
\pgfpathlineto{\pgfqpoint{2.262289in}{1.211080in}}%
\pgfpathlineto{\pgfqpoint{2.266331in}{1.211577in}}%
\pgfpathlineto{\pgfqpoint{2.265786in}{1.225833in}}%
\pgfpathlineto{\pgfqpoint{2.285375in}{1.228022in}}%
\pgfpathlineto{\pgfqpoint{2.288393in}{1.199350in}}%
\pgfpathclose%
\pgfusepath{fill}%
\end{pgfscope}%
\begin{pgfscope}%
\pgfpathrectangle{\pgfqpoint{0.100000in}{0.100000in}}{\pgfqpoint{3.007045in}{1.925000in}}%
\pgfusepath{clip}%
\pgfsetbuttcap%
\pgfsetmiterjoin%
\definecolor{currentfill}{rgb}{0.466667,0.708189,0.849366}%
\pgfsetfillcolor{currentfill}%
\pgfsetlinewidth{0.000000pt}%
\definecolor{currentstroke}{rgb}{0.000000,0.000000,0.000000}%
\pgfsetstrokecolor{currentstroke}%
\pgfsetstrokeopacity{0.000000}%
\pgfsetdash{}{0pt}%
\pgfpathmoveto{\pgfqpoint{1.905032in}{1.544903in}}%
\pgfpathlineto{\pgfqpoint{1.876967in}{1.544122in}}%
\pgfpathlineto{\pgfqpoint{1.876801in}{1.549513in}}%
\pgfpathlineto{\pgfqpoint{1.881663in}{1.556766in}}%
\pgfpathlineto{\pgfqpoint{1.881731in}{1.559757in}}%
\pgfpathlineto{\pgfqpoint{1.875816in}{1.567788in}}%
\pgfpathlineto{\pgfqpoint{1.870534in}{1.568228in}}%
\pgfpathlineto{\pgfqpoint{1.872272in}{1.578521in}}%
\pgfpathlineto{\pgfqpoint{1.874776in}{1.580802in}}%
\pgfpathlineto{\pgfqpoint{1.877917in}{1.589345in}}%
\pgfpathlineto{\pgfqpoint{1.881222in}{1.592179in}}%
\pgfpathlineto{\pgfqpoint{1.894425in}{1.598113in}}%
\pgfpathlineto{\pgfqpoint{1.896908in}{1.602038in}}%
\pgfpathlineto{\pgfqpoint{1.896759in}{1.607543in}}%
\pgfpathlineto{\pgfqpoint{1.907975in}{1.607918in}}%
\pgfpathlineto{\pgfqpoint{1.909099in}{1.596409in}}%
\pgfpathlineto{\pgfqpoint{1.909946in}{1.573607in}}%
\pgfpathlineto{\pgfqpoint{1.904215in}{1.573411in}}%
\pgfpathlineto{\pgfqpoint{1.905032in}{1.544903in}}%
\pgfpathclose%
\pgfusepath{fill}%
\end{pgfscope}%
\begin{pgfscope}%
\pgfpathrectangle{\pgfqpoint{0.100000in}{0.100000in}}{\pgfqpoint{3.007045in}{1.925000in}}%
\pgfusepath{clip}%
\pgfsetbuttcap%
\pgfsetmiterjoin%
\definecolor{currentfill}{rgb}{0.270804,0.580730,0.781146}%
\pgfsetfillcolor{currentfill}%
\pgfsetlinewidth{0.000000pt}%
\definecolor{currentstroke}{rgb}{0.000000,0.000000,0.000000}%
\pgfsetstrokecolor{currentstroke}%
\pgfsetstrokeopacity{0.000000}%
\pgfsetdash{}{0pt}%
\pgfpathmoveto{\pgfqpoint{2.052237in}{0.736873in}}%
\pgfpathlineto{\pgfqpoint{2.048959in}{0.728755in}}%
\pgfpathlineto{\pgfqpoint{2.040123in}{0.721921in}}%
\pgfpathlineto{\pgfqpoint{2.027190in}{0.716066in}}%
\pgfpathlineto{\pgfqpoint{2.020779in}{0.710545in}}%
\pgfpathlineto{\pgfqpoint{2.011147in}{0.717467in}}%
\pgfpathlineto{\pgfqpoint{2.011096in}{0.729886in}}%
\pgfpathlineto{\pgfqpoint{2.008261in}{0.731716in}}%
\pgfpathlineto{\pgfqpoint{2.013725in}{0.737732in}}%
\pgfpathlineto{\pgfqpoint{2.013096in}{0.749239in}}%
\pgfpathlineto{\pgfqpoint{2.009327in}{0.760562in}}%
\pgfpathlineto{\pgfqpoint{2.018223in}{0.761023in}}%
\pgfpathlineto{\pgfqpoint{2.018021in}{0.764899in}}%
\pgfpathlineto{\pgfqpoint{2.023674in}{0.765200in}}%
\pgfpathlineto{\pgfqpoint{2.023847in}{0.761087in}}%
\pgfpathlineto{\pgfqpoint{2.030481in}{0.763778in}}%
\pgfpathlineto{\pgfqpoint{2.031123in}{0.752696in}}%
\pgfpathlineto{\pgfqpoint{2.025625in}{0.749814in}}%
\pgfpathlineto{\pgfqpoint{2.031102in}{0.744404in}}%
\pgfpathlineto{\pgfqpoint{2.052237in}{0.736873in}}%
\pgfpathclose%
\pgfusepath{fill}%
\end{pgfscope}%
\begin{pgfscope}%
\pgfpathrectangle{\pgfqpoint{0.100000in}{0.100000in}}{\pgfqpoint{3.007045in}{1.925000in}}%
\pgfusepath{clip}%
\pgfsetbuttcap%
\pgfsetmiterjoin%
\definecolor{currentfill}{rgb}{0.381776,0.656517,0.824452}%
\pgfsetfillcolor{currentfill}%
\pgfsetlinewidth{0.000000pt}%
\definecolor{currentstroke}{rgb}{0.000000,0.000000,0.000000}%
\pgfsetstrokecolor{currentstroke}%
\pgfsetstrokeopacity{0.000000}%
\pgfsetdash{}{0pt}%
\pgfpathmoveto{\pgfqpoint{1.827033in}{1.605997in}}%
\pgfpathlineto{\pgfqpoint{1.827186in}{1.594478in}}%
\pgfpathlineto{\pgfqpoint{1.829097in}{1.594472in}}%
\pgfpathlineto{\pgfqpoint{1.829469in}{1.583832in}}%
\pgfpathlineto{\pgfqpoint{1.802241in}{1.583541in}}%
\pgfpathlineto{\pgfqpoint{1.806019in}{1.580465in}}%
\pgfpathlineto{\pgfqpoint{1.765864in}{1.580217in}}%
\pgfpathlineto{\pgfqpoint{1.765646in}{1.613993in}}%
\pgfpathlineto{\pgfqpoint{1.764876in}{1.642920in}}%
\pgfpathlineto{\pgfqpoint{1.764918in}{1.648716in}}%
\pgfpathlineto{\pgfqpoint{1.782023in}{1.648669in}}%
\pgfpathlineto{\pgfqpoint{1.782448in}{1.621431in}}%
\pgfpathlineto{\pgfqpoint{1.793675in}{1.615402in}}%
\pgfpathlineto{\pgfqpoint{1.798627in}{1.617392in}}%
\pgfpathlineto{\pgfqpoint{1.802129in}{1.614963in}}%
\pgfpathlineto{\pgfqpoint{1.801088in}{1.605771in}}%
\pgfpathlineto{\pgfqpoint{1.827033in}{1.605997in}}%
\pgfpathclose%
\pgfusepath{fill}%
\end{pgfscope}%
\begin{pgfscope}%
\pgfpathrectangle{\pgfqpoint{0.100000in}{0.100000in}}{\pgfqpoint{3.007045in}{1.925000in}}%
\pgfusepath{clip}%
\pgfsetbuttcap%
\pgfsetmiterjoin%
\definecolor{currentfill}{rgb}{0.321246,0.615179,0.800830}%
\pgfsetfillcolor{currentfill}%
\pgfsetlinewidth{0.000000pt}%
\definecolor{currentstroke}{rgb}{0.000000,0.000000,0.000000}%
\pgfsetstrokecolor{currentstroke}%
\pgfsetstrokeopacity{0.000000}%
\pgfsetdash{}{0pt}%
\pgfpathmoveto{\pgfqpoint{1.876321in}{0.768157in}}%
\pgfpathlineto{\pgfqpoint{1.876274in}{0.763349in}}%
\pgfpathlineto{\pgfqpoint{1.884045in}{0.763316in}}%
\pgfpathlineto{\pgfqpoint{1.884006in}{0.739484in}}%
\pgfpathlineto{\pgfqpoint{1.838686in}{0.738812in}}%
\pgfpathlineto{\pgfqpoint{1.835563in}{0.743537in}}%
\pgfpathlineto{\pgfqpoint{1.836073in}{0.747122in}}%
\pgfpathlineto{\pgfqpoint{1.841164in}{0.748497in}}%
\pgfpathlineto{\pgfqpoint{1.846121in}{0.759629in}}%
\pgfpathlineto{\pgfqpoint{1.842406in}{0.765728in}}%
\pgfpathlineto{\pgfqpoint{1.842812in}{0.769497in}}%
\pgfpathlineto{\pgfqpoint{1.856051in}{0.769311in}}%
\pgfpathlineto{\pgfqpoint{1.856034in}{0.767401in}}%
\pgfpathlineto{\pgfqpoint{1.870545in}{0.767215in}}%
\pgfpathlineto{\pgfqpoint{1.876321in}{0.768157in}}%
\pgfpathclose%
\pgfusepath{fill}%
\end{pgfscope}%
\begin{pgfscope}%
\pgfpathrectangle{\pgfqpoint{0.100000in}{0.100000in}}{\pgfqpoint{3.007045in}{1.925000in}}%
\pgfusepath{clip}%
\pgfsetbuttcap%
\pgfsetmiterjoin%
\definecolor{currentfill}{rgb}{0.585882,0.773641,0.875079}%
\pgfsetfillcolor{currentfill}%
\pgfsetlinewidth{0.000000pt}%
\definecolor{currentstroke}{rgb}{0.000000,0.000000,0.000000}%
\pgfsetstrokecolor{currentstroke}%
\pgfsetstrokeopacity{0.000000}%
\pgfsetdash{}{0pt}%
\pgfpathmoveto{\pgfqpoint{2.061841in}{1.396171in}}%
\pgfpathlineto{\pgfqpoint{2.044679in}{1.394919in}}%
\pgfpathlineto{\pgfqpoint{2.021967in}{1.394387in}}%
\pgfpathlineto{\pgfqpoint{2.022119in}{1.391513in}}%
\pgfpathlineto{\pgfqpoint{1.993663in}{1.389926in}}%
\pgfpathlineto{\pgfqpoint{1.992238in}{1.415509in}}%
\pgfpathlineto{\pgfqpoint{1.986647in}{1.415927in}}%
\pgfpathlineto{\pgfqpoint{1.980936in}{1.413027in}}%
\pgfpathlineto{\pgfqpoint{1.980060in}{1.429597in}}%
\pgfpathlineto{\pgfqpoint{1.979503in}{1.438168in}}%
\pgfpathlineto{\pgfqpoint{1.996750in}{1.439088in}}%
\pgfpathlineto{\pgfqpoint{1.996384in}{1.444829in}}%
\pgfpathlineto{\pgfqpoint{2.021528in}{1.446204in}}%
\pgfpathlineto{\pgfqpoint{2.030394in}{1.446824in}}%
\pgfpathlineto{\pgfqpoint{2.029155in}{1.469240in}}%
\pgfpathlineto{\pgfqpoint{2.049555in}{1.470541in}}%
\pgfpathlineto{\pgfqpoint{2.050014in}{1.463525in}}%
\pgfpathlineto{\pgfqpoint{2.046797in}{1.456021in}}%
\pgfpathlineto{\pgfqpoint{2.047314in}{1.447880in}}%
\pgfpathlineto{\pgfqpoint{2.058743in}{1.447959in}}%
\pgfpathlineto{\pgfqpoint{2.061841in}{1.396171in}}%
\pgfpathclose%
\pgfusepath{fill}%
\end{pgfscope}%
\begin{pgfscope}%
\pgfpathrectangle{\pgfqpoint{0.100000in}{0.100000in}}{\pgfqpoint{3.007045in}{1.925000in}}%
\pgfusepath{clip}%
\pgfsetbuttcap%
\pgfsetmiterjoin%
\definecolor{currentfill}{rgb}{0.485490,0.718524,0.853426}%
\pgfsetfillcolor{currentfill}%
\pgfsetlinewidth{0.000000pt}%
\definecolor{currentstroke}{rgb}{0.000000,0.000000,0.000000}%
\pgfsetstrokecolor{currentstroke}%
\pgfsetstrokeopacity{0.000000}%
\pgfsetdash{}{0pt}%
\pgfpathmoveto{\pgfqpoint{1.082392in}{1.585392in}}%
\pgfpathlineto{\pgfqpoint{1.114575in}{1.580480in}}%
\pgfpathlineto{\pgfqpoint{1.115456in}{1.579920in}}%
\pgfpathlineto{\pgfqpoint{1.154054in}{1.574061in}}%
\pgfpathlineto{\pgfqpoint{1.169718in}{1.571914in}}%
\pgfpathlineto{\pgfqpoint{1.170691in}{1.563235in}}%
\pgfpathlineto{\pgfqpoint{1.174953in}{1.552958in}}%
\pgfpathlineto{\pgfqpoint{1.180687in}{1.550302in}}%
\pgfpathlineto{\pgfqpoint{1.188802in}{1.543168in}}%
\pgfpathlineto{\pgfqpoint{1.193646in}{1.534813in}}%
\pgfpathlineto{\pgfqpoint{1.197692in}{1.530943in}}%
\pgfpathlineto{\pgfqpoint{1.199309in}{1.522550in}}%
\pgfpathlineto{\pgfqpoint{1.197829in}{1.512182in}}%
\pgfpathlineto{\pgfqpoint{1.131931in}{1.521748in}}%
\pgfpathlineto{\pgfqpoint{1.131047in}{1.515926in}}%
\pgfpathlineto{\pgfqpoint{1.119707in}{1.517659in}}%
\pgfpathlineto{\pgfqpoint{1.118845in}{1.511867in}}%
\pgfpathlineto{\pgfqpoint{1.113003in}{1.512751in}}%
\pgfpathlineto{\pgfqpoint{1.112286in}{1.507103in}}%
\pgfpathlineto{\pgfqpoint{1.103807in}{1.508400in}}%
\pgfpathlineto{\pgfqpoint{1.102940in}{1.502630in}}%
\pgfpathlineto{\pgfqpoint{1.092509in}{1.504079in}}%
\pgfpathlineto{\pgfqpoint{1.087640in}{1.512697in}}%
\pgfpathlineto{\pgfqpoint{1.083266in}{1.515471in}}%
\pgfpathlineto{\pgfqpoint{1.078184in}{1.513711in}}%
\pgfpathlineto{\pgfqpoint{1.076162in}{1.510378in}}%
\pgfpathlineto{\pgfqpoint{1.070042in}{1.506895in}}%
\pgfpathlineto{\pgfqpoint{1.067064in}{1.517262in}}%
\pgfpathlineto{\pgfqpoint{1.062108in}{1.518116in}}%
\pgfpathlineto{\pgfqpoint{1.059788in}{1.522455in}}%
\pgfpathlineto{\pgfqpoint{1.061138in}{1.530583in}}%
\pgfpathlineto{\pgfqpoint{1.058870in}{1.534775in}}%
\pgfpathlineto{\pgfqpoint{1.057765in}{1.542345in}}%
\pgfpathlineto{\pgfqpoint{1.053058in}{1.551890in}}%
\pgfpathlineto{\pgfqpoint{1.054900in}{1.557314in}}%
\pgfpathlineto{\pgfqpoint{1.051087in}{1.562499in}}%
\pgfpathlineto{\pgfqpoint{1.037496in}{1.564829in}}%
\pgfpathlineto{\pgfqpoint{1.038445in}{1.570273in}}%
\pgfpathlineto{\pgfqpoint{1.020483in}{1.573439in}}%
\pgfpathlineto{\pgfqpoint{1.024388in}{1.595276in}}%
\pgfpathlineto{\pgfqpoint{1.040422in}{1.591840in}}%
\pgfpathlineto{\pgfqpoint{1.082392in}{1.585392in}}%
\pgfpathclose%
\pgfusepath{fill}%
\end{pgfscope}%
\begin{pgfscope}%
\pgfpathrectangle{\pgfqpoint{0.100000in}{0.100000in}}{\pgfqpoint{3.007045in}{1.925000in}}%
\pgfusepath{clip}%
\pgfsetbuttcap%
\pgfsetmiterjoin%
\definecolor{currentfill}{rgb}{0.381776,0.656517,0.824452}%
\pgfsetfillcolor{currentfill}%
\pgfsetlinewidth{0.000000pt}%
\definecolor{currentstroke}{rgb}{0.000000,0.000000,0.000000}%
\pgfsetstrokecolor{currentstroke}%
\pgfsetstrokeopacity{0.000000}%
\pgfsetdash{}{0pt}%
\pgfpathmoveto{\pgfqpoint{2.143299in}{0.800190in}}%
\pgfpathlineto{\pgfqpoint{2.139121in}{0.799917in}}%
\pgfpathlineto{\pgfqpoint{2.136914in}{0.795797in}}%
\pgfpathlineto{\pgfqpoint{2.129211in}{0.792870in}}%
\pgfpathlineto{\pgfqpoint{2.117772in}{0.794063in}}%
\pgfpathlineto{\pgfqpoint{2.117152in}{0.802691in}}%
\pgfpathlineto{\pgfqpoint{2.105570in}{0.801993in}}%
\pgfpathlineto{\pgfqpoint{2.105804in}{0.798159in}}%
\pgfpathlineto{\pgfqpoint{2.091654in}{0.795102in}}%
\pgfpathlineto{\pgfqpoint{2.074344in}{0.794001in}}%
\pgfpathlineto{\pgfqpoint{2.073753in}{0.803617in}}%
\pgfpathlineto{\pgfqpoint{2.072446in}{0.823053in}}%
\pgfpathlineto{\pgfqpoint{2.086876in}{0.823888in}}%
\pgfpathlineto{\pgfqpoint{2.085917in}{0.838289in}}%
\pgfpathlineto{\pgfqpoint{2.103193in}{0.839402in}}%
\pgfpathlineto{\pgfqpoint{2.110699in}{0.840860in}}%
\pgfpathlineto{\pgfqpoint{2.110261in}{0.847530in}}%
\pgfpathlineto{\pgfqpoint{2.113075in}{0.848655in}}%
\pgfpathlineto{\pgfqpoint{2.123649in}{0.849414in}}%
\pgfpathlineto{\pgfqpoint{2.123839in}{0.846519in}}%
\pgfpathlineto{\pgfqpoint{2.144807in}{0.847920in}}%
\pgfpathlineto{\pgfqpoint{2.144570in}{0.838481in}}%
\pgfpathlineto{\pgfqpoint{2.143299in}{0.800190in}}%
\pgfpathclose%
\pgfusepath{fill}%
\end{pgfscope}%
\begin{pgfscope}%
\pgfpathrectangle{\pgfqpoint{0.100000in}{0.100000in}}{\pgfqpoint{3.007045in}{1.925000in}}%
\pgfusepath{clip}%
\pgfsetbuttcap%
\pgfsetmiterjoin%
\definecolor{currentfill}{rgb}{0.805260,0.878016,0.946851}%
\pgfsetfillcolor{currentfill}%
\pgfsetlinewidth{0.000000pt}%
\definecolor{currentstroke}{rgb}{0.000000,0.000000,0.000000}%
\pgfsetstrokecolor{currentstroke}%
\pgfsetstrokeopacity{0.000000}%
\pgfsetdash{}{0pt}%
\pgfpathmoveto{\pgfqpoint{2.898990in}{1.789167in}}%
\pgfpathlineto{\pgfqpoint{2.897996in}{1.797410in}}%
\pgfpathlineto{\pgfqpoint{2.917566in}{1.855793in}}%
\pgfpathlineto{\pgfqpoint{2.926613in}{1.855322in}}%
\pgfpathlineto{\pgfqpoint{2.928868in}{1.845060in}}%
\pgfpathlineto{\pgfqpoint{2.936844in}{1.841940in}}%
\pgfpathlineto{\pgfqpoint{2.948719in}{1.853007in}}%
\pgfpathlineto{\pgfqpoint{2.957139in}{1.855503in}}%
\pgfpathlineto{\pgfqpoint{2.956602in}{1.859933in}}%
\pgfpathlineto{\pgfqpoint{2.962545in}{1.861779in}}%
\pgfpathlineto{\pgfqpoint{2.977427in}{1.855560in}}%
\pgfpathlineto{\pgfqpoint{2.982323in}{1.850596in}}%
\pgfpathlineto{\pgfqpoint{2.987134in}{1.849352in}}%
\pgfpathlineto{\pgfqpoint{2.996362in}{1.819764in}}%
\pgfpathlineto{\pgfqpoint{3.009567in}{1.778140in}}%
\pgfpathlineto{\pgfqpoint{3.009737in}{1.774049in}}%
\pgfpathlineto{\pgfqpoint{3.013596in}{1.761178in}}%
\pgfpathlineto{\pgfqpoint{3.003662in}{1.755298in}}%
\pgfpathlineto{\pgfqpoint{2.988875in}{1.746178in}}%
\pgfpathlineto{\pgfqpoint{2.987728in}{1.746399in}}%
\pgfpathlineto{\pgfqpoint{2.981003in}{1.769545in}}%
\pgfpathlineto{\pgfqpoint{2.972249in}{1.797222in}}%
\pgfpathlineto{\pgfqpoint{2.955175in}{1.793081in}}%
\pgfpathlineto{\pgfqpoint{2.951724in}{1.804308in}}%
\pgfpathlineto{\pgfqpoint{2.912260in}{1.792925in}}%
\pgfpathlineto{\pgfqpoint{2.898990in}{1.789167in}}%
\pgfpathclose%
\pgfusepath{fill}%
\end{pgfscope}%
\begin{pgfscope}%
\pgfpathrectangle{\pgfqpoint{0.100000in}{0.100000in}}{\pgfqpoint{3.007045in}{1.925000in}}%
\pgfusepath{clip}%
\pgfsetbuttcap%
\pgfsetmiterjoin%
\definecolor{currentfill}{rgb}{0.351511,0.635848,0.812641}%
\pgfsetfillcolor{currentfill}%
\pgfsetlinewidth{0.000000pt}%
\definecolor{currentstroke}{rgb}{0.000000,0.000000,0.000000}%
\pgfsetstrokecolor{currentstroke}%
\pgfsetstrokeopacity{0.000000}%
\pgfsetdash{}{0pt}%
\pgfpathmoveto{\pgfqpoint{2.278491in}{1.292509in}}%
\pgfpathlineto{\pgfqpoint{2.273138in}{1.338797in}}%
\pgfpathlineto{\pgfqpoint{2.292953in}{1.341716in}}%
\pgfpathlineto{\pgfqpoint{2.295342in}{1.329269in}}%
\pgfpathlineto{\pgfqpoint{2.297909in}{1.323832in}}%
\pgfpathlineto{\pgfqpoint{2.303481in}{1.324508in}}%
\pgfpathlineto{\pgfqpoint{2.305531in}{1.307364in}}%
\pgfpathlineto{\pgfqpoint{2.299961in}{1.306677in}}%
\pgfpathlineto{\pgfqpoint{2.301283in}{1.295237in}}%
\pgfpathlineto{\pgfqpoint{2.278491in}{1.292509in}}%
\pgfpathclose%
\pgfusepath{fill}%
\end{pgfscope}%
\begin{pgfscope}%
\pgfpathrectangle{\pgfqpoint{0.100000in}{0.100000in}}{\pgfqpoint{3.007045in}{1.925000in}}%
\pgfusepath{clip}%
\pgfsetbuttcap%
\pgfsetmiterjoin%
\definecolor{currentfill}{rgb}{0.361599,0.642737,0.816578}%
\pgfsetfillcolor{currentfill}%
\pgfsetlinewidth{0.000000pt}%
\definecolor{currentstroke}{rgb}{0.000000,0.000000,0.000000}%
\pgfsetstrokecolor{currentstroke}%
\pgfsetstrokeopacity{0.000000}%
\pgfsetdash{}{0pt}%
\pgfpathmoveto{\pgfqpoint{1.102849in}{0.996413in}}%
\pgfpathlineto{\pgfqpoint{1.100725in}{0.982093in}}%
\pgfpathlineto{\pgfqpoint{1.025816in}{0.993300in}}%
\pgfpathlineto{\pgfqpoint{1.035972in}{1.058230in}}%
\pgfpathlineto{\pgfqpoint{1.070635in}{1.052989in}}%
\pgfpathlineto{\pgfqpoint{1.073504in}{1.056252in}}%
\pgfpathlineto{\pgfqpoint{1.077407in}{1.066910in}}%
\pgfpathlineto{\pgfqpoint{1.083485in}{1.074787in}}%
\pgfpathlineto{\pgfqpoint{1.088472in}{1.075836in}}%
\pgfpathlineto{\pgfqpoint{1.093042in}{1.080708in}}%
\pgfpathlineto{\pgfqpoint{1.094867in}{1.088434in}}%
\pgfpathlineto{\pgfqpoint{1.103247in}{1.096115in}}%
\pgfpathlineto{\pgfqpoint{1.104848in}{1.099688in}}%
\pgfpathlineto{\pgfqpoint{1.112363in}{1.107267in}}%
\pgfpathlineto{\pgfqpoint{1.121625in}{1.109932in}}%
\pgfpathlineto{\pgfqpoint{1.123300in}{1.108295in}}%
\pgfpathlineto{\pgfqpoint{1.124305in}{1.096464in}}%
\pgfpathlineto{\pgfqpoint{1.121195in}{1.073917in}}%
\pgfpathlineto{\pgfqpoint{1.139523in}{1.071438in}}%
\pgfpathlineto{\pgfqpoint{1.139253in}{1.069444in}}%
\pgfpathlineto{\pgfqpoint{1.162663in}{1.066636in}}%
\pgfpathlineto{\pgfqpoint{1.161246in}{1.055657in}}%
\pgfpathlineto{\pgfqpoint{1.169817in}{1.038957in}}%
\pgfpathlineto{\pgfqpoint{1.120573in}{1.045885in}}%
\pgfpathlineto{\pgfqpoint{1.118519in}{1.040536in}}%
\pgfpathlineto{\pgfqpoint{1.108498in}{1.034565in}}%
\pgfpathlineto{\pgfqpoint{1.102849in}{0.996413in}}%
\pgfpathclose%
\pgfusepath{fill}%
\end{pgfscope}%
\begin{pgfscope}%
\pgfpathrectangle{\pgfqpoint{0.100000in}{0.100000in}}{\pgfqpoint{3.007045in}{1.925000in}}%
\pgfusepath{clip}%
\pgfsetbuttcap%
\pgfsetmiterjoin%
\definecolor{currentfill}{rgb}{0.760477,0.852026,0.931657}%
\pgfsetfillcolor{currentfill}%
\pgfsetlinewidth{0.000000pt}%
\definecolor{currentstroke}{rgb}{0.000000,0.000000,0.000000}%
\pgfsetstrokecolor{currentstroke}%
\pgfsetstrokeopacity{0.000000}%
\pgfsetdash{}{0pt}%
\pgfpathmoveto{\pgfqpoint{2.033429in}{1.724806in}}%
\pgfpathlineto{\pgfqpoint{2.028764in}{1.727403in}}%
\pgfpathlineto{\pgfqpoint{2.031261in}{1.731523in}}%
\pgfpathlineto{\pgfqpoint{2.049089in}{1.741926in}}%
\pgfpathlineto{\pgfqpoint{2.055441in}{1.748010in}}%
\pgfpathlineto{\pgfqpoint{2.063105in}{1.751149in}}%
\pgfpathlineto{\pgfqpoint{2.058606in}{1.742131in}}%
\pgfpathlineto{\pgfqpoint{2.053691in}{1.738050in}}%
\pgfpathlineto{\pgfqpoint{2.042387in}{1.732402in}}%
\pgfpathlineto{\pgfqpoint{2.044192in}{1.729855in}}%
\pgfpathlineto{\pgfqpoint{2.033429in}{1.724806in}}%
\pgfpathclose%
\pgfusepath{fill}%
\end{pgfscope}%
\begin{pgfscope}%
\pgfpathrectangle{\pgfqpoint{0.100000in}{0.100000in}}{\pgfqpoint{3.007045in}{1.925000in}}%
\pgfusepath{clip}%
\pgfsetbuttcap%
\pgfsetmiterjoin%
\definecolor{currentfill}{rgb}{0.760477,0.852026,0.931657}%
\pgfsetfillcolor{currentfill}%
\pgfsetlinewidth{0.000000pt}%
\definecolor{currentstroke}{rgb}{0.000000,0.000000,0.000000}%
\pgfsetstrokecolor{currentstroke}%
\pgfsetstrokeopacity{0.000000}%
\pgfsetdash{}{0pt}%
\pgfpathmoveto{\pgfqpoint{2.088047in}{1.667928in}}%
\pgfpathlineto{\pgfqpoint{2.088768in}{1.658202in}}%
\pgfpathlineto{\pgfqpoint{2.091627in}{1.652681in}}%
\pgfpathlineto{\pgfqpoint{2.085981in}{1.652184in}}%
\pgfpathlineto{\pgfqpoint{2.087227in}{1.635040in}}%
\pgfpathlineto{\pgfqpoint{2.047301in}{1.632207in}}%
\pgfpathlineto{\pgfqpoint{2.046234in}{1.649481in}}%
\pgfpathlineto{\pgfqpoint{2.051916in}{1.649810in}}%
\pgfpathlineto{\pgfqpoint{2.048453in}{1.655274in}}%
\pgfpathlineto{\pgfqpoint{2.047218in}{1.673072in}}%
\pgfpathlineto{\pgfqpoint{2.048913in}{1.677724in}}%
\pgfpathlineto{\pgfqpoint{2.058859in}{1.686684in}}%
\pgfpathlineto{\pgfqpoint{2.062270in}{1.688002in}}%
\pgfpathlineto{\pgfqpoint{2.068764in}{1.697183in}}%
\pgfpathlineto{\pgfqpoint{2.077462in}{1.703060in}}%
\pgfpathlineto{\pgfqpoint{2.089417in}{1.705852in}}%
\pgfpathlineto{\pgfqpoint{2.096001in}{1.705831in}}%
\pgfpathlineto{\pgfqpoint{2.100493in}{1.701354in}}%
\pgfpathlineto{\pgfqpoint{2.091665in}{1.700318in}}%
\pgfpathlineto{\pgfqpoint{2.090480in}{1.696264in}}%
\pgfpathlineto{\pgfqpoint{2.085473in}{1.693327in}}%
\pgfpathlineto{\pgfqpoint{2.078290in}{1.686386in}}%
\pgfpathlineto{\pgfqpoint{2.078268in}{1.682642in}}%
\pgfpathlineto{\pgfqpoint{2.069709in}{1.670334in}}%
\pgfpathlineto{\pgfqpoint{2.068597in}{1.661571in}}%
\pgfpathlineto{\pgfqpoint{2.071052in}{1.658913in}}%
\pgfpathlineto{\pgfqpoint{2.078864in}{1.668386in}}%
\pgfpathlineto{\pgfqpoint{2.082700in}{1.671038in}}%
\pgfpathlineto{\pgfqpoint{2.088047in}{1.667928in}}%
\pgfpathclose%
\pgfusepath{fill}%
\end{pgfscope}%
\begin{pgfscope}%
\pgfpathrectangle{\pgfqpoint{0.100000in}{0.100000in}}{\pgfqpoint{3.007045in}{1.925000in}}%
\pgfusepath{clip}%
\pgfsetbuttcap%
\pgfsetmiterjoin%
\definecolor{currentfill}{rgb}{0.391865,0.663406,0.828389}%
\pgfsetfillcolor{currentfill}%
\pgfsetlinewidth{0.000000pt}%
\definecolor{currentstroke}{rgb}{0.000000,0.000000,0.000000}%
\pgfsetstrokecolor{currentstroke}%
\pgfsetstrokeopacity{0.000000}%
\pgfsetdash{}{0pt}%
\pgfpathmoveto{\pgfqpoint{1.251074in}{1.090450in}}%
\pgfpathlineto{\pgfqpoint{1.245702in}{1.097910in}}%
\pgfpathlineto{\pgfqpoint{1.241682in}{1.094643in}}%
\pgfpathlineto{\pgfqpoint{1.232103in}{1.090942in}}%
\pgfpathlineto{\pgfqpoint{1.229068in}{1.091653in}}%
\pgfpathlineto{\pgfqpoint{1.223896in}{1.096979in}}%
\pgfpathlineto{\pgfqpoint{1.220511in}{1.109227in}}%
\pgfpathlineto{\pgfqpoint{1.205700in}{1.130953in}}%
\pgfpathlineto{\pgfqpoint{1.213054in}{1.137702in}}%
\pgfpathlineto{\pgfqpoint{1.213461in}{1.141563in}}%
\pgfpathlineto{\pgfqpoint{1.209760in}{1.146848in}}%
\pgfpathlineto{\pgfqpoint{1.242494in}{1.143237in}}%
\pgfpathlineto{\pgfqpoint{1.245658in}{1.171547in}}%
\pgfpathlineto{\pgfqpoint{1.310620in}{1.164502in}}%
\pgfpathlineto{\pgfqpoint{1.308723in}{1.147516in}}%
\pgfpathlineto{\pgfqpoint{1.306529in}{1.124802in}}%
\pgfpathlineto{\pgfqpoint{1.261017in}{1.129389in}}%
\pgfpathlineto{\pgfqpoint{1.259199in}{1.112273in}}%
\pgfpathlineto{\pgfqpoint{1.253600in}{1.112871in}}%
\pgfpathlineto{\pgfqpoint{1.251074in}{1.090450in}}%
\pgfpathclose%
\pgfusepath{fill}%
\end{pgfscope}%
\begin{pgfscope}%
\pgfpathrectangle{\pgfqpoint{0.100000in}{0.100000in}}{\pgfqpoint{3.007045in}{1.925000in}}%
\pgfusepath{clip}%
\pgfsetbuttcap%
\pgfsetmiterjoin%
\definecolor{currentfill}{rgb}{0.647290,0.803922,0.892042}%
\pgfsetfillcolor{currentfill}%
\pgfsetlinewidth{0.000000pt}%
\definecolor{currentstroke}{rgb}{0.000000,0.000000,0.000000}%
\pgfsetstrokecolor{currentstroke}%
\pgfsetstrokeopacity{0.000000}%
\pgfsetdash{}{0pt}%
\pgfpathmoveto{\pgfqpoint{2.498296in}{1.267386in}}%
\pgfpathlineto{\pgfqpoint{2.480209in}{1.264505in}}%
\pgfpathlineto{\pgfqpoint{2.478385in}{1.281786in}}%
\pgfpathlineto{\pgfqpoint{2.474602in}{1.281312in}}%
\pgfpathlineto{\pgfqpoint{2.474283in}{1.290040in}}%
\pgfpathlineto{\pgfqpoint{2.476757in}{1.293301in}}%
\pgfpathlineto{\pgfqpoint{2.486521in}{1.293715in}}%
\pgfpathlineto{\pgfqpoint{2.493292in}{1.298554in}}%
\pgfpathlineto{\pgfqpoint{2.498296in}{1.267386in}}%
\pgfpathclose%
\pgfusepath{fill}%
\end{pgfscope}%
\begin{pgfscope}%
\pgfpathrectangle{\pgfqpoint{0.100000in}{0.100000in}}{\pgfqpoint{3.007045in}{1.925000in}}%
\pgfusepath{clip}%
\pgfsetbuttcap%
\pgfsetmiterjoin%
\definecolor{currentfill}{rgb}{0.454118,0.701300,0.846659}%
\pgfsetfillcolor{currentfill}%
\pgfsetlinewidth{0.000000pt}%
\definecolor{currentstroke}{rgb}{0.000000,0.000000,0.000000}%
\pgfsetstrokecolor{currentstroke}%
\pgfsetstrokeopacity{0.000000}%
\pgfsetdash{}{0pt}%
\pgfpathmoveto{\pgfqpoint{2.161409in}{0.940303in}}%
\pgfpathlineto{\pgfqpoint{2.160890in}{0.928692in}}%
\pgfpathlineto{\pgfqpoint{2.166375in}{0.924298in}}%
\pgfpathlineto{\pgfqpoint{2.163243in}{0.916924in}}%
\pgfpathlineto{\pgfqpoint{2.149635in}{0.914318in}}%
\pgfpathlineto{\pgfqpoint{2.147248in}{0.909640in}}%
\pgfpathlineto{\pgfqpoint{2.142348in}{0.908782in}}%
\pgfpathlineto{\pgfqpoint{2.129095in}{0.910042in}}%
\pgfpathlineto{\pgfqpoint{2.125902in}{0.914003in}}%
\pgfpathlineto{\pgfqpoint{2.120897in}{0.915606in}}%
\pgfpathlineto{\pgfqpoint{2.114744in}{0.920210in}}%
\pgfpathlineto{\pgfqpoint{2.114171in}{0.933474in}}%
\pgfpathlineto{\pgfqpoint{2.116000in}{0.935950in}}%
\pgfpathlineto{\pgfqpoint{2.136863in}{0.936941in}}%
\pgfpathlineto{\pgfqpoint{2.143491in}{0.939640in}}%
\pgfpathlineto{\pgfqpoint{2.144945in}{0.937174in}}%
\pgfpathlineto{\pgfqpoint{2.151093in}{0.939458in}}%
\pgfpathlineto{\pgfqpoint{2.161409in}{0.940303in}}%
\pgfpathclose%
\pgfusepath{fill}%
\end{pgfscope}%
\begin{pgfscope}%
\pgfpathrectangle{\pgfqpoint{0.100000in}{0.100000in}}{\pgfqpoint{3.007045in}{1.925000in}}%
\pgfusepath{clip}%
\pgfsetbuttcap%
\pgfsetmiterjoin%
\definecolor{currentfill}{rgb}{0.191326,0.505052,0.741699}%
\pgfsetfillcolor{currentfill}%
\pgfsetlinewidth{0.000000pt}%
\definecolor{currentstroke}{rgb}{0.000000,0.000000,0.000000}%
\pgfsetstrokecolor{currentstroke}%
\pgfsetstrokeopacity{0.000000}%
\pgfsetdash{}{0pt}%
\pgfpathmoveto{\pgfqpoint{1.282225in}{0.694354in}}%
\pgfpathlineto{\pgfqpoint{1.265329in}{0.695996in}}%
\pgfpathlineto{\pgfqpoint{1.219256in}{0.700832in}}%
\pgfpathlineto{\pgfqpoint{1.222778in}{0.735139in}}%
\pgfpathlineto{\pgfqpoint{1.194858in}{0.738140in}}%
\pgfpathlineto{\pgfqpoint{1.198272in}{0.767309in}}%
\pgfpathlineto{\pgfqpoint{1.200282in}{0.767082in}}%
\pgfpathlineto{\pgfqpoint{1.201518in}{0.778286in}}%
\pgfpathlineto{\pgfqpoint{1.224219in}{0.776176in}}%
\pgfpathlineto{\pgfqpoint{1.226820in}{0.793144in}}%
\pgfpathlineto{\pgfqpoint{1.231742in}{0.838561in}}%
\pgfpathlineto{\pgfqpoint{1.237556in}{0.837894in}}%
\pgfpathlineto{\pgfqpoint{1.236948in}{0.832126in}}%
\pgfpathlineto{\pgfqpoint{1.271248in}{0.828230in}}%
\pgfpathlineto{\pgfqpoint{1.271855in}{0.834010in}}%
\pgfpathlineto{\pgfqpoint{1.283287in}{0.832832in}}%
\pgfpathlineto{\pgfqpoint{1.288966in}{0.832202in}}%
\pgfpathlineto{\pgfqpoint{1.287264in}{0.814931in}}%
\pgfpathlineto{\pgfqpoint{1.294107in}{0.814272in}}%
\pgfpathlineto{\pgfqpoint{1.292993in}{0.803535in}}%
\pgfpathlineto{\pgfqpoint{1.304431in}{0.802517in}}%
\pgfpathlineto{\pgfqpoint{1.303863in}{0.796794in}}%
\pgfpathlineto{\pgfqpoint{1.292309in}{0.797855in}}%
\pgfpathlineto{\pgfqpoint{1.291172in}{0.786411in}}%
\pgfpathlineto{\pgfqpoint{1.288691in}{0.786623in}}%
\pgfpathlineto{\pgfqpoint{1.285953in}{0.758267in}}%
\pgfpathlineto{\pgfqpoint{1.283283in}{0.758525in}}%
\pgfpathlineto{\pgfqpoint{1.280477in}{0.729323in}}%
\pgfpathlineto{\pgfqpoint{1.285541in}{0.728841in}}%
\pgfpathlineto{\pgfqpoint{1.282225in}{0.694354in}}%
\pgfpathclose%
\pgfusepath{fill}%
\end{pgfscope}%
\begin{pgfscope}%
\pgfpathrectangle{\pgfqpoint{0.100000in}{0.100000in}}{\pgfqpoint{3.007045in}{1.925000in}}%
\pgfusepath{clip}%
\pgfsetbuttcap%
\pgfsetmiterjoin%
\definecolor{currentfill}{rgb}{0.460392,0.704744,0.848012}%
\pgfsetfillcolor{currentfill}%
\pgfsetlinewidth{0.000000pt}%
\definecolor{currentstroke}{rgb}{0.000000,0.000000,0.000000}%
\pgfsetstrokecolor{currentstroke}%
\pgfsetstrokeopacity{0.000000}%
\pgfsetdash{}{0pt}%
\pgfpathmoveto{\pgfqpoint{1.193280in}{1.837329in}}%
\pgfpathlineto{\pgfqpoint{1.189784in}{1.820439in}}%
\pgfpathlineto{\pgfqpoint{1.187177in}{1.820830in}}%
\pgfpathlineto{\pgfqpoint{1.185494in}{1.809331in}}%
\pgfpathlineto{\pgfqpoint{1.182880in}{1.799848in}}%
\pgfpathlineto{\pgfqpoint{1.180237in}{1.801764in}}%
\pgfpathlineto{\pgfqpoint{1.177030in}{1.787023in}}%
\pgfpathlineto{\pgfqpoint{1.174638in}{1.770993in}}%
\pgfpathlineto{\pgfqpoint{1.167096in}{1.773103in}}%
\pgfpathlineto{\pgfqpoint{1.165616in}{1.768587in}}%
\pgfpathlineto{\pgfqpoint{1.154039in}{1.770483in}}%
\pgfpathlineto{\pgfqpoint{1.152002in}{1.758091in}}%
\pgfpathlineto{\pgfqpoint{1.148828in}{1.763186in}}%
\pgfpathlineto{\pgfqpoint{1.141833in}{1.762846in}}%
\pgfpathlineto{\pgfqpoint{1.135562in}{1.764609in}}%
\pgfpathlineto{\pgfqpoint{1.126744in}{1.761331in}}%
\pgfpathlineto{\pgfqpoint{1.123379in}{1.762908in}}%
\pgfpathlineto{\pgfqpoint{1.127771in}{1.788966in}}%
\pgfpathlineto{\pgfqpoint{1.119270in}{1.790180in}}%
\pgfpathlineto{\pgfqpoint{1.120232in}{1.796028in}}%
\pgfpathlineto{\pgfqpoint{1.115258in}{1.796858in}}%
\pgfpathlineto{\pgfqpoint{1.115936in}{1.802454in}}%
\pgfpathlineto{\pgfqpoint{1.081806in}{1.808264in}}%
\pgfpathlineto{\pgfqpoint{1.080804in}{1.802553in}}%
\pgfpathlineto{\pgfqpoint{1.071976in}{1.804099in}}%
\pgfpathlineto{\pgfqpoint{1.070974in}{1.798414in}}%
\pgfpathlineto{\pgfqpoint{1.045450in}{1.802919in}}%
\pgfpathlineto{\pgfqpoint{1.046502in}{1.808652in}}%
\pgfpathlineto{\pgfqpoint{1.052373in}{1.807583in}}%
\pgfpathlineto{\pgfqpoint{1.061807in}{1.858712in}}%
\pgfpathlineto{\pgfqpoint{1.107359in}{1.850859in}}%
\pgfpathlineto{\pgfqpoint{1.160277in}{1.842278in}}%
\pgfpathlineto{\pgfqpoint{1.193280in}{1.837329in}}%
\pgfpathclose%
\pgfusepath{fill}%
\end{pgfscope}%
\begin{pgfscope}%
\pgfpathrectangle{\pgfqpoint{0.100000in}{0.100000in}}{\pgfqpoint{3.007045in}{1.925000in}}%
\pgfusepath{clip}%
\pgfsetbuttcap%
\pgfsetmiterjoin%
\definecolor{currentfill}{rgb}{0.726028,0.837386,0.919600}%
\pgfsetfillcolor{currentfill}%
\pgfsetlinewidth{0.000000pt}%
\definecolor{currentstroke}{rgb}{0.000000,0.000000,0.000000}%
\pgfsetstrokecolor{currentstroke}%
\pgfsetstrokeopacity{0.000000}%
\pgfsetdash{}{0pt}%
\pgfpathmoveto{\pgfqpoint{2.755737in}{0.991248in}}%
\pgfpathlineto{\pgfqpoint{2.755915in}{0.984607in}}%
\pgfpathlineto{\pgfqpoint{2.744917in}{0.981075in}}%
\pgfpathlineto{\pgfqpoint{2.737785in}{0.985361in}}%
\pgfpathlineto{\pgfqpoint{2.724069in}{0.990555in}}%
\pgfpathlineto{\pgfqpoint{2.723536in}{0.997192in}}%
\pgfpathlineto{\pgfqpoint{2.725670in}{0.999941in}}%
\pgfpathlineto{\pgfqpoint{2.722156in}{1.004752in}}%
\pgfpathlineto{\pgfqpoint{2.720462in}{1.011491in}}%
\pgfpathlineto{\pgfqpoint{2.710520in}{1.014800in}}%
\pgfpathlineto{\pgfqpoint{2.709850in}{1.020460in}}%
\pgfpathlineto{\pgfqpoint{2.705428in}{1.026285in}}%
\pgfpathlineto{\pgfqpoint{2.709028in}{1.030469in}}%
\pgfpathlineto{\pgfqpoint{2.715659in}{1.028449in}}%
\pgfpathlineto{\pgfqpoint{2.716657in}{1.024638in}}%
\pgfpathlineto{\pgfqpoint{2.722671in}{1.024298in}}%
\pgfpathlineto{\pgfqpoint{2.728149in}{1.020939in}}%
\pgfpathlineto{\pgfqpoint{2.731963in}{1.024371in}}%
\pgfpathlineto{\pgfqpoint{2.735129in}{1.019199in}}%
\pgfpathlineto{\pgfqpoint{2.738228in}{1.024682in}}%
\pgfpathlineto{\pgfqpoint{2.741385in}{1.024328in}}%
\pgfpathlineto{\pgfqpoint{2.743345in}{1.029859in}}%
\pgfpathlineto{\pgfqpoint{2.751811in}{1.031450in}}%
\pgfpathlineto{\pgfqpoint{2.758023in}{1.035375in}}%
\pgfpathlineto{\pgfqpoint{2.761386in}{1.033527in}}%
\pgfpathlineto{\pgfqpoint{2.767606in}{1.037752in}}%
\pgfpathlineto{\pgfqpoint{2.775648in}{1.039823in}}%
\pgfpathlineto{\pgfqpoint{2.782087in}{1.034841in}}%
\pgfpathlineto{\pgfqpoint{2.784638in}{1.040494in}}%
\pgfpathlineto{\pgfqpoint{2.789579in}{1.040321in}}%
\pgfpathlineto{\pgfqpoint{2.793501in}{1.036335in}}%
\pgfpathlineto{\pgfqpoint{2.797982in}{1.024799in}}%
\pgfpathlineto{\pgfqpoint{2.796768in}{1.016487in}}%
\pgfpathlineto{\pgfqpoint{2.790807in}{1.014831in}}%
\pgfpathlineto{\pgfqpoint{2.786449in}{1.006708in}}%
\pgfpathlineto{\pgfqpoint{2.785854in}{1.002606in}}%
\pgfpathlineto{\pgfqpoint{2.779981in}{0.995686in}}%
\pgfpathlineto{\pgfqpoint{2.772764in}{0.995498in}}%
\pgfpathlineto{\pgfqpoint{2.764411in}{0.997827in}}%
\pgfpathlineto{\pgfqpoint{2.762533in}{0.994980in}}%
\pgfpathlineto{\pgfqpoint{2.754523in}{0.995316in}}%
\pgfpathlineto{\pgfqpoint{2.755737in}{0.991248in}}%
\pgfpathclose%
\pgfusepath{fill}%
\end{pgfscope}%
\begin{pgfscope}%
\pgfpathrectangle{\pgfqpoint{0.100000in}{0.100000in}}{\pgfqpoint{3.007045in}{1.925000in}}%
\pgfusepath{clip}%
\pgfsetbuttcap%
\pgfsetmiterjoin%
\definecolor{currentfill}{rgb}{0.726028,0.837386,0.919600}%
\pgfsetfillcolor{currentfill}%
\pgfsetlinewidth{0.000000pt}%
\definecolor{currentstroke}{rgb}{0.000000,0.000000,0.000000}%
\pgfsetstrokecolor{currentstroke}%
\pgfsetstrokeopacity{0.000000}%
\pgfsetdash{}{0pt}%
\pgfpathmoveto{\pgfqpoint{2.787444in}{1.058403in}}%
\pgfpathlineto{\pgfqpoint{2.798434in}{1.043326in}}%
\pgfpathlineto{\pgfqpoint{2.793014in}{1.044103in}}%
\pgfpathlineto{\pgfqpoint{2.786756in}{1.058061in}}%
\pgfpathlineto{\pgfqpoint{2.787444in}{1.058403in}}%
\pgfpathclose%
\pgfusepath{fill}%
\end{pgfscope}%
\begin{pgfscope}%
\pgfpathrectangle{\pgfqpoint{0.100000in}{0.100000in}}{\pgfqpoint{3.007045in}{1.925000in}}%
\pgfusepath{clip}%
\pgfsetbuttcap%
\pgfsetmiterjoin%
\definecolor{currentfill}{rgb}{0.579608,0.770196,0.873725}%
\pgfsetfillcolor{currentfill}%
\pgfsetlinewidth{0.000000pt}%
\definecolor{currentstroke}{rgb}{0.000000,0.000000,0.000000}%
\pgfsetstrokecolor{currentstroke}%
\pgfsetstrokeopacity{0.000000}%
\pgfsetdash{}{0pt}%
\pgfpathmoveto{\pgfqpoint{1.619645in}{0.516644in}}%
\pgfpathlineto{\pgfqpoint{1.621659in}{0.515303in}}%
\pgfpathlineto{\pgfqpoint{1.638148in}{0.525132in}}%
\pgfpathlineto{\pgfqpoint{1.648009in}{0.514439in}}%
\pgfpathlineto{\pgfqpoint{1.644405in}{0.511356in}}%
\pgfpathlineto{\pgfqpoint{1.641992in}{0.498317in}}%
\pgfpathlineto{\pgfqpoint{1.619767in}{0.480550in}}%
\pgfpathlineto{\pgfqpoint{1.613261in}{0.488228in}}%
\pgfpathlineto{\pgfqpoint{1.605940in}{0.497218in}}%
\pgfpathlineto{\pgfqpoint{1.619645in}{0.516644in}}%
\pgfpathclose%
\pgfusepath{fill}%
\end{pgfscope}%
\begin{pgfscope}%
\pgfpathrectangle{\pgfqpoint{0.100000in}{0.100000in}}{\pgfqpoint{3.007045in}{1.925000in}}%
\pgfusepath{clip}%
\pgfsetbuttcap%
\pgfsetmiterjoin%
\definecolor{currentfill}{rgb}{0.134487,0.448212,0.712418}%
\pgfsetfillcolor{currentfill}%
\pgfsetlinewidth{0.000000pt}%
\definecolor{currentstroke}{rgb}{0.000000,0.000000,0.000000}%
\pgfsetstrokecolor{currentstroke}%
\pgfsetstrokeopacity{0.000000}%
\pgfsetdash{}{0pt}%
\pgfpathmoveto{\pgfqpoint{1.590530in}{1.433489in}}%
\pgfpathlineto{\pgfqpoint{1.647409in}{1.431503in}}%
\pgfpathlineto{\pgfqpoint{1.657257in}{1.431226in}}%
\pgfpathlineto{\pgfqpoint{1.656789in}{1.409421in}}%
\pgfpathlineto{\pgfqpoint{1.624011in}{1.410336in}}%
\pgfpathlineto{\pgfqpoint{1.622340in}{1.404124in}}%
\pgfpathlineto{\pgfqpoint{1.622963in}{1.397529in}}%
\pgfpathlineto{\pgfqpoint{1.621472in}{1.391045in}}%
\pgfpathlineto{\pgfqpoint{1.618866in}{1.388633in}}%
\pgfpathlineto{\pgfqpoint{1.612271in}{1.391873in}}%
\pgfpathlineto{\pgfqpoint{1.604815in}{1.396442in}}%
\pgfpathlineto{\pgfqpoint{1.603438in}{1.399865in}}%
\pgfpathlineto{\pgfqpoint{1.602167in}{1.402735in}}%
\pgfpathlineto{\pgfqpoint{1.594388in}{1.405377in}}%
\pgfpathlineto{\pgfqpoint{1.590580in}{1.410117in}}%
\pgfpathlineto{\pgfqpoint{1.586747in}{1.410853in}}%
\pgfpathlineto{\pgfqpoint{1.584652in}{1.417090in}}%
\pgfpathlineto{\pgfqpoint{1.576855in}{1.422263in}}%
\pgfpathlineto{\pgfqpoint{1.572614in}{1.429955in}}%
\pgfpathlineto{\pgfqpoint{1.568472in}{1.430306in}}%
\pgfpathlineto{\pgfqpoint{1.566436in}{1.434578in}}%
\pgfpathlineto{\pgfqpoint{1.590530in}{1.433489in}}%
\pgfpathclose%
\pgfusepath{fill}%
\end{pgfscope}%
\begin{pgfscope}%
\pgfpathrectangle{\pgfqpoint{0.100000in}{0.100000in}}{\pgfqpoint{3.007045in}{1.925000in}}%
\pgfusepath{clip}%
\pgfsetbuttcap%
\pgfsetmiterjoin%
\definecolor{currentfill}{rgb}{0.301069,0.601399,0.792957}%
\pgfsetfillcolor{currentfill}%
\pgfsetlinewidth{0.000000pt}%
\definecolor{currentstroke}{rgb}{0.000000,0.000000,0.000000}%
\pgfsetstrokecolor{currentstroke}%
\pgfsetstrokeopacity{0.000000}%
\pgfsetdash{}{0pt}%
\pgfpathmoveto{\pgfqpoint{2.256834in}{1.132262in}}%
\pgfpathlineto{\pgfqpoint{2.245396in}{1.130991in}}%
\pgfpathlineto{\pgfqpoint{2.240875in}{1.128586in}}%
\pgfpathlineto{\pgfqpoint{2.239682in}{1.138991in}}%
\pgfpathlineto{\pgfqpoint{2.235786in}{1.140573in}}%
\pgfpathlineto{\pgfqpoint{2.219718in}{1.138972in}}%
\pgfpathlineto{\pgfqpoint{2.218022in}{1.153845in}}%
\pgfpathlineto{\pgfqpoint{2.215737in}{1.157449in}}%
\pgfpathlineto{\pgfqpoint{2.227838in}{1.158721in}}%
\pgfpathlineto{\pgfqpoint{2.225671in}{1.177961in}}%
\pgfpathlineto{\pgfqpoint{2.245949in}{1.180376in}}%
\pgfpathlineto{\pgfqpoint{2.247318in}{1.166002in}}%
\pgfpathlineto{\pgfqpoint{2.253429in}{1.166775in}}%
\pgfpathlineto{\pgfqpoint{2.259410in}{1.171566in}}%
\pgfpathlineto{\pgfqpoint{2.261118in}{1.153025in}}%
\pgfpathlineto{\pgfqpoint{2.253275in}{1.146329in}}%
\pgfpathlineto{\pgfqpoint{2.249568in}{1.145924in}}%
\pgfpathlineto{\pgfqpoint{2.250151in}{1.140216in}}%
\pgfpathlineto{\pgfqpoint{2.256292in}{1.137004in}}%
\pgfpathlineto{\pgfqpoint{2.256834in}{1.132262in}}%
\pgfpathclose%
\pgfusepath{fill}%
\end{pgfscope}%
\begin{pgfscope}%
\pgfpathrectangle{\pgfqpoint{0.100000in}{0.100000in}}{\pgfqpoint{3.007045in}{1.925000in}}%
\pgfusepath{clip}%
\pgfsetbuttcap%
\pgfsetmiterjoin%
\definecolor{currentfill}{rgb}{0.326290,0.618624,0.802799}%
\pgfsetfillcolor{currentfill}%
\pgfsetlinewidth{0.000000pt}%
\definecolor{currentstroke}{rgb}{0.000000,0.000000,0.000000}%
\pgfsetstrokecolor{currentstroke}%
\pgfsetstrokeopacity{0.000000}%
\pgfsetdash{}{0pt}%
\pgfpathmoveto{\pgfqpoint{1.719649in}{1.001488in}}%
\pgfpathlineto{\pgfqpoint{1.659290in}{1.002553in}}%
\pgfpathlineto{\pgfqpoint{1.659741in}{1.033948in}}%
\pgfpathlineto{\pgfqpoint{1.692552in}{1.033341in}}%
\pgfpathlineto{\pgfqpoint{1.693377in}{1.073520in}}%
\pgfpathlineto{\pgfqpoint{1.701914in}{1.073348in}}%
\pgfpathlineto{\pgfqpoint{1.702042in}{1.079070in}}%
\pgfpathlineto{\pgfqpoint{1.722658in}{1.078703in}}%
\pgfpathlineto{\pgfqpoint{1.722556in}{1.070090in}}%
\pgfpathlineto{\pgfqpoint{1.722263in}{1.049928in}}%
\pgfpathlineto{\pgfqpoint{1.721571in}{1.001483in}}%
\pgfpathlineto{\pgfqpoint{1.719649in}{1.001488in}}%
\pgfpathclose%
\pgfusepath{fill}%
\end{pgfscope}%
\begin{pgfscope}%
\pgfpathrectangle{\pgfqpoint{0.100000in}{0.100000in}}{\pgfqpoint{3.007045in}{1.925000in}}%
\pgfusepath{clip}%
\pgfsetbuttcap%
\pgfsetmiterjoin%
\definecolor{currentfill}{rgb}{0.671895,0.814379,0.900654}%
\pgfsetfillcolor{currentfill}%
\pgfsetlinewidth{0.000000pt}%
\definecolor{currentstroke}{rgb}{0.000000,0.000000,0.000000}%
\pgfsetstrokecolor{currentstroke}%
\pgfsetstrokeopacity{0.000000}%
\pgfsetdash{}{0pt}%
\pgfpathmoveto{\pgfqpoint{2.406434in}{1.016470in}}%
\pgfpathlineto{\pgfqpoint{2.372965in}{1.012422in}}%
\pgfpathlineto{\pgfqpoint{2.373385in}{1.015073in}}%
\pgfpathlineto{\pgfqpoint{2.380873in}{1.017870in}}%
\pgfpathlineto{\pgfqpoint{2.397238in}{1.025171in}}%
\pgfpathlineto{\pgfqpoint{2.399471in}{1.032629in}}%
\pgfpathlineto{\pgfqpoint{2.409387in}{1.036508in}}%
\pgfpathlineto{\pgfqpoint{2.409125in}{1.041720in}}%
\pgfpathlineto{\pgfqpoint{2.415182in}{1.047525in}}%
\pgfpathlineto{\pgfqpoint{2.415558in}{1.052709in}}%
\pgfpathlineto{\pgfqpoint{2.423037in}{1.058784in}}%
\pgfpathlineto{\pgfqpoint{2.435154in}{1.067105in}}%
\pgfpathlineto{\pgfqpoint{2.440019in}{1.062167in}}%
\pgfpathlineto{\pgfqpoint{2.446669in}{1.052394in}}%
\pgfpathlineto{\pgfqpoint{2.437440in}{1.045801in}}%
\pgfpathlineto{\pgfqpoint{2.439575in}{1.041742in}}%
\pgfpathlineto{\pgfqpoint{2.434200in}{1.039055in}}%
\pgfpathlineto{\pgfqpoint{2.422280in}{1.037551in}}%
\pgfpathlineto{\pgfqpoint{2.416735in}{1.033753in}}%
\pgfpathlineto{\pgfqpoint{2.413702in}{1.026726in}}%
\pgfpathlineto{\pgfqpoint{2.407690in}{1.022099in}}%
\pgfpathlineto{\pgfqpoint{2.406434in}{1.016470in}}%
\pgfpathclose%
\pgfusepath{fill}%
\end{pgfscope}%
\begin{pgfscope}%
\pgfpathrectangle{\pgfqpoint{0.100000in}{0.100000in}}{\pgfqpoint{3.007045in}{1.925000in}}%
\pgfusepath{clip}%
\pgfsetbuttcap%
\pgfsetmiterjoin%
\definecolor{currentfill}{rgb}{0.108651,0.416563,0.689043}%
\pgfsetfillcolor{currentfill}%
\pgfsetlinewidth{0.000000pt}%
\definecolor{currentstroke}{rgb}{0.000000,0.000000,0.000000}%
\pgfsetstrokecolor{currentstroke}%
\pgfsetstrokeopacity{0.000000}%
\pgfsetdash{}{0pt}%
\pgfpathmoveto{\pgfqpoint{1.419469in}{0.951860in}}%
\pgfpathlineto{\pgfqpoint{1.417413in}{0.923222in}}%
\pgfpathlineto{\pgfqpoint{1.388570in}{0.925898in}}%
\pgfpathlineto{\pgfqpoint{1.341578in}{0.929505in}}%
\pgfpathlineto{\pgfqpoint{1.342266in}{0.937180in}}%
\pgfpathlineto{\pgfqpoint{1.346674in}{0.987189in}}%
\pgfpathlineto{\pgfqpoint{1.348759in}{0.987002in}}%
\pgfpathlineto{\pgfqpoint{1.393120in}{0.983287in}}%
\pgfpathlineto{\pgfqpoint{1.390781in}{0.954030in}}%
\pgfpathlineto{\pgfqpoint{1.419469in}{0.951860in}}%
\pgfpathclose%
\pgfusepath{fill}%
\end{pgfscope}%
\begin{pgfscope}%
\pgfpathrectangle{\pgfqpoint{0.100000in}{0.100000in}}{\pgfqpoint{3.007045in}{1.925000in}}%
\pgfusepath{clip}%
\pgfsetbuttcap%
\pgfsetmiterjoin%
\definecolor{currentfill}{rgb}{0.256286,0.570012,0.775163}%
\pgfsetfillcolor{currentfill}%
\pgfsetlinewidth{0.000000pt}%
\definecolor{currentstroke}{rgb}{0.000000,0.000000,0.000000}%
\pgfsetstrokecolor{currentstroke}%
\pgfsetstrokeopacity{0.000000}%
\pgfsetdash{}{0pt}%
\pgfpathmoveto{\pgfqpoint{1.605884in}{0.418493in}}%
\pgfpathlineto{\pgfqpoint{1.600656in}{0.414463in}}%
\pgfpathlineto{\pgfqpoint{1.601772in}{0.410582in}}%
\pgfpathlineto{\pgfqpoint{1.547367in}{0.412624in}}%
\pgfpathlineto{\pgfqpoint{1.549167in}{0.452106in}}%
\pgfpathlineto{\pgfqpoint{1.550168in}{0.481716in}}%
\pgfpathlineto{\pgfqpoint{1.514641in}{0.483384in}}%
\pgfpathlineto{\pgfqpoint{1.516499in}{0.519146in}}%
\pgfpathlineto{\pgfqpoint{1.505414in}{0.519674in}}%
\pgfpathlineto{\pgfqpoint{1.506378in}{0.538351in}}%
\pgfpathlineto{\pgfqpoint{1.530916in}{0.536323in}}%
\pgfpathlineto{\pgfqpoint{1.545579in}{0.528102in}}%
\pgfpathlineto{\pgfqpoint{1.546459in}{0.551872in}}%
\pgfpathlineto{\pgfqpoint{1.565594in}{0.551126in}}%
\pgfpathlineto{\pgfqpoint{1.575104in}{0.537324in}}%
\pgfpathlineto{\pgfqpoint{1.582069in}{0.543767in}}%
\pgfpathlineto{\pgfqpoint{1.597051in}{0.530605in}}%
\pgfpathlineto{\pgfqpoint{1.598646in}{0.524138in}}%
\pgfpathlineto{\pgfqpoint{1.606065in}{0.530962in}}%
\pgfpathlineto{\pgfqpoint{1.615672in}{0.518019in}}%
\pgfpathlineto{\pgfqpoint{1.619645in}{0.516644in}}%
\pgfpathlineto{\pgfqpoint{1.605940in}{0.497218in}}%
\pgfpathlineto{\pgfqpoint{1.613261in}{0.488228in}}%
\pgfpathlineto{\pgfqpoint{1.585508in}{0.466431in}}%
\pgfpathlineto{\pgfqpoint{1.595954in}{0.453216in}}%
\pgfpathlineto{\pgfqpoint{1.590934in}{0.451560in}}%
\pgfpathlineto{\pgfqpoint{1.606432in}{0.418922in}}%
\pgfpathlineto{\pgfqpoint{1.605884in}{0.418493in}}%
\pgfpathclose%
\pgfusepath{fill}%
\end{pgfscope}%
\begin{pgfscope}%
\pgfpathrectangle{\pgfqpoint{0.100000in}{0.100000in}}{\pgfqpoint{3.007045in}{1.925000in}}%
\pgfusepath{clip}%
\pgfsetbuttcap%
\pgfsetmiterjoin%
\definecolor{currentfill}{rgb}{0.248166,0.561892,0.770980}%
\pgfsetfillcolor{currentfill}%
\pgfsetlinewidth{0.000000pt}%
\definecolor{currentstroke}{rgb}{0.000000,0.000000,0.000000}%
\pgfsetstrokecolor{currentstroke}%
\pgfsetstrokeopacity{0.000000}%
\pgfsetdash{}{0pt}%
\pgfpathmoveto{\pgfqpoint{1.542619in}{0.830859in}}%
\pgfpathlineto{\pgfqpoint{1.533732in}{0.839225in}}%
\pgfpathlineto{\pgfqpoint{1.531947in}{0.833744in}}%
\pgfpathlineto{\pgfqpoint{1.526280in}{0.836772in}}%
\pgfpathlineto{\pgfqpoint{1.521539in}{0.834387in}}%
\pgfpathlineto{\pgfqpoint{1.516379in}{0.834892in}}%
\pgfpathlineto{\pgfqpoint{1.504435in}{0.848612in}}%
\pgfpathlineto{\pgfqpoint{1.500565in}{0.847796in}}%
\pgfpathlineto{\pgfqpoint{1.501084in}{0.860066in}}%
\pgfpathlineto{\pgfqpoint{1.502115in}{0.878788in}}%
\pgfpathlineto{\pgfqpoint{1.513852in}{0.878138in}}%
\pgfpathlineto{\pgfqpoint{1.514160in}{0.883851in}}%
\pgfpathlineto{\pgfqpoint{1.569929in}{0.881210in}}%
\pgfpathlineto{\pgfqpoint{1.576781in}{0.879670in}}%
\pgfpathlineto{\pgfqpoint{1.576210in}{0.863673in}}%
\pgfpathlineto{\pgfqpoint{1.565031in}{0.864162in}}%
\pgfpathlineto{\pgfqpoint{1.564310in}{0.846924in}}%
\pgfpathlineto{\pgfqpoint{1.545976in}{0.847321in}}%
\pgfpathlineto{\pgfqpoint{1.542518in}{0.841248in}}%
\pgfpathlineto{\pgfqpoint{1.542619in}{0.830859in}}%
\pgfpathclose%
\pgfusepath{fill}%
\end{pgfscope}%
\begin{pgfscope}%
\pgfpathrectangle{\pgfqpoint{0.100000in}{0.100000in}}{\pgfqpoint{3.007045in}{1.925000in}}%
\pgfusepath{clip}%
\pgfsetbuttcap%
\pgfsetmiterjoin%
\definecolor{currentfill}{rgb}{0.376732,0.653072,0.822484}%
\pgfsetfillcolor{currentfill}%
\pgfsetlinewidth{0.000000pt}%
\definecolor{currentstroke}{rgb}{0.000000,0.000000,0.000000}%
\pgfsetstrokecolor{currentstroke}%
\pgfsetstrokeopacity{0.000000}%
\pgfsetdash{}{0pt}%
\pgfpathmoveto{\pgfqpoint{1.702839in}{1.116383in}}%
\pgfpathlineto{\pgfqpoint{1.702564in}{1.102052in}}%
\pgfpathlineto{\pgfqpoint{1.678559in}{1.102524in}}%
\pgfpathlineto{\pgfqpoint{1.672876in}{1.102642in}}%
\pgfpathlineto{\pgfqpoint{1.672984in}{1.108364in}}%
\pgfpathlineto{\pgfqpoint{1.650255in}{1.108902in}}%
\pgfpathlineto{\pgfqpoint{1.651571in}{1.154782in}}%
\pgfpathlineto{\pgfqpoint{1.652076in}{1.171950in}}%
\pgfpathlineto{\pgfqpoint{1.709420in}{1.170767in}}%
\pgfpathlineto{\pgfqpoint{1.719727in}{1.170628in}}%
\pgfpathlineto{\pgfqpoint{1.719484in}{1.147603in}}%
\pgfpathlineto{\pgfqpoint{1.712256in}{1.145853in}}%
\pgfpathlineto{\pgfqpoint{1.708830in}{1.147175in}}%
\pgfpathlineto{\pgfqpoint{1.701412in}{1.144959in}}%
\pgfpathlineto{\pgfqpoint{1.701186in}{1.136436in}}%
\pgfpathlineto{\pgfqpoint{1.695539in}{1.136550in}}%
\pgfpathlineto{\pgfqpoint{1.695279in}{1.122232in}}%
\pgfpathlineto{\pgfqpoint{1.700977in}{1.122113in}}%
\pgfpathlineto{\pgfqpoint{1.702839in}{1.116383in}}%
\pgfpathclose%
\pgfusepath{fill}%
\end{pgfscope}%
\begin{pgfscope}%
\pgfpathrectangle{\pgfqpoint{0.100000in}{0.100000in}}{\pgfqpoint{3.007045in}{1.925000in}}%
\pgfusepath{clip}%
\pgfsetbuttcap%
\pgfsetmiterjoin%
\definecolor{currentfill}{rgb}{0.460392,0.704744,0.848012}%
\pgfsetfillcolor{currentfill}%
\pgfsetlinewidth{0.000000pt}%
\definecolor{currentstroke}{rgb}{0.000000,0.000000,0.000000}%
\pgfsetstrokecolor{currentstroke}%
\pgfsetstrokeopacity{0.000000}%
\pgfsetdash{}{0pt}%
\pgfpathmoveto{\pgfqpoint{2.505332in}{0.935315in}}%
\pgfpathlineto{\pgfqpoint{2.472948in}{0.932025in}}%
\pgfpathlineto{\pgfqpoint{2.472311in}{0.936614in}}%
\pgfpathlineto{\pgfqpoint{2.460723in}{0.945065in}}%
\pgfpathlineto{\pgfqpoint{2.455447in}{0.943242in}}%
\pgfpathlineto{\pgfqpoint{2.453980in}{0.946316in}}%
\pgfpathlineto{\pgfqpoint{2.455786in}{0.951722in}}%
\pgfpathlineto{\pgfqpoint{2.459120in}{0.952751in}}%
\pgfpathlineto{\pgfqpoint{2.469542in}{0.954025in}}%
\pgfpathlineto{\pgfqpoint{2.476332in}{0.956183in}}%
\pgfpathlineto{\pgfqpoint{2.480332in}{0.961447in}}%
\pgfpathlineto{\pgfqpoint{2.486422in}{0.958962in}}%
\pgfpathlineto{\pgfqpoint{2.492448in}{0.960338in}}%
\pgfpathlineto{\pgfqpoint{2.523136in}{0.963616in}}%
\pgfpathlineto{\pgfqpoint{2.523665in}{0.961730in}}%
\pgfpathlineto{\pgfqpoint{2.524892in}{0.952370in}}%
\pgfpathlineto{\pgfqpoint{2.523035in}{0.944991in}}%
\pgfpathlineto{\pgfqpoint{2.523780in}{0.937038in}}%
\pgfpathlineto{\pgfqpoint{2.505332in}{0.935315in}}%
\pgfpathclose%
\pgfusepath{fill}%
\end{pgfscope}%
\begin{pgfscope}%
\pgfpathrectangle{\pgfqpoint{0.100000in}{0.100000in}}{\pgfqpoint{3.007045in}{1.925000in}}%
\pgfusepath{clip}%
\pgfsetbuttcap%
\pgfsetmiterjoin%
\definecolor{currentfill}{rgb}{0.479216,0.715079,0.852072}%
\pgfsetfillcolor{currentfill}%
\pgfsetlinewidth{0.000000pt}%
\definecolor{currentstroke}{rgb}{0.000000,0.000000,0.000000}%
\pgfsetstrokecolor{currentstroke}%
\pgfsetstrokeopacity{0.000000}%
\pgfsetdash{}{0pt}%
\pgfpathmoveto{\pgfqpoint{2.452246in}{0.846939in}}%
\pgfpathlineto{\pgfqpoint{2.447000in}{0.848563in}}%
\pgfpathlineto{\pgfqpoint{2.440442in}{0.846333in}}%
\pgfpathlineto{\pgfqpoint{2.431219in}{0.848011in}}%
\pgfpathlineto{\pgfqpoint{2.421120in}{0.857223in}}%
\pgfpathlineto{\pgfqpoint{2.422010in}{0.860947in}}%
\pgfpathlineto{\pgfqpoint{2.409663in}{0.861811in}}%
\pgfpathlineto{\pgfqpoint{2.406992in}{0.858796in}}%
\pgfpathlineto{\pgfqpoint{2.404051in}{0.865135in}}%
\pgfpathlineto{\pgfqpoint{2.403763in}{0.872065in}}%
\pgfpathlineto{\pgfqpoint{2.399169in}{0.874931in}}%
\pgfpathlineto{\pgfqpoint{2.402644in}{0.888304in}}%
\pgfpathlineto{\pgfqpoint{2.403995in}{0.889257in}}%
\pgfpathlineto{\pgfqpoint{2.410336in}{0.885056in}}%
\pgfpathlineto{\pgfqpoint{2.413604in}{0.885230in}}%
\pgfpathlineto{\pgfqpoint{2.420077in}{0.879039in}}%
\pgfpathlineto{\pgfqpoint{2.423854in}{0.877561in}}%
\pgfpathlineto{\pgfqpoint{2.429221in}{0.879173in}}%
\pgfpathlineto{\pgfqpoint{2.439934in}{0.864954in}}%
\pgfpathlineto{\pgfqpoint{2.442202in}{0.858557in}}%
\pgfpathlineto{\pgfqpoint{2.449961in}{0.851491in}}%
\pgfpathlineto{\pgfqpoint{2.452246in}{0.846939in}}%
\pgfpathclose%
\pgfusepath{fill}%
\end{pgfscope}%
\begin{pgfscope}%
\pgfpathrectangle{\pgfqpoint{0.100000in}{0.100000in}}{\pgfqpoint{3.007045in}{1.925000in}}%
\pgfusepath{clip}%
\pgfsetbuttcap%
\pgfsetmiterjoin%
\definecolor{currentfill}{rgb}{0.429020,0.687520,0.841246}%
\pgfsetfillcolor{currentfill}%
\pgfsetlinewidth{0.000000pt}%
\definecolor{currentstroke}{rgb}{0.000000,0.000000,0.000000}%
\pgfsetstrokecolor{currentstroke}%
\pgfsetstrokeopacity{0.000000}%
\pgfsetdash{}{0pt}%
\pgfpathmoveto{\pgfqpoint{0.816917in}{1.293210in}}%
\pgfpathlineto{\pgfqpoint{0.811909in}{1.269809in}}%
\pgfpathlineto{\pgfqpoint{0.799942in}{1.214078in}}%
\pgfpathlineto{\pgfqpoint{0.752166in}{1.224502in}}%
\pgfpathlineto{\pgfqpoint{0.714021in}{1.265975in}}%
\pgfpathlineto{\pgfqpoint{0.679545in}{1.274162in}}%
\pgfpathlineto{\pgfqpoint{0.643335in}{1.283223in}}%
\pgfpathlineto{\pgfqpoint{0.620230in}{1.284320in}}%
\pgfpathlineto{\pgfqpoint{0.621234in}{1.289668in}}%
\pgfpathlineto{\pgfqpoint{0.625302in}{1.292738in}}%
\pgfpathlineto{\pgfqpoint{0.625081in}{1.300716in}}%
\pgfpathlineto{\pgfqpoint{0.632081in}{1.306373in}}%
\pgfpathlineto{\pgfqpoint{0.641368in}{1.309018in}}%
\pgfpathlineto{\pgfqpoint{0.644313in}{1.314171in}}%
\pgfpathlineto{\pgfqpoint{0.644679in}{1.322619in}}%
\pgfpathlineto{\pgfqpoint{0.648824in}{1.333373in}}%
\pgfpathlineto{\pgfqpoint{0.646486in}{1.339327in}}%
\pgfpathlineto{\pgfqpoint{0.671162in}{1.376694in}}%
\pgfpathlineto{\pgfqpoint{0.682240in}{1.373981in}}%
\pgfpathlineto{\pgfqpoint{0.687900in}{1.396831in}}%
\pgfpathlineto{\pgfqpoint{0.703710in}{1.460774in}}%
\pgfpathlineto{\pgfqpoint{0.734607in}{1.453046in}}%
\pgfpathlineto{\pgfqpoint{0.798116in}{1.438303in}}%
\pgfpathlineto{\pgfqpoint{0.808116in}{1.436493in}}%
\pgfpathlineto{\pgfqpoint{0.834301in}{1.430229in}}%
\pgfpathlineto{\pgfqpoint{0.845804in}{1.427734in}}%
\pgfpathlineto{\pgfqpoint{0.832155in}{1.363654in}}%
\pgfpathlineto{\pgfqpoint{0.816917in}{1.293210in}}%
\pgfpathclose%
\pgfusepath{fill}%
\end{pgfscope}%
\begin{pgfscope}%
\pgfpathrectangle{\pgfqpoint{0.100000in}{0.100000in}}{\pgfqpoint{3.007045in}{1.925000in}}%
\pgfusepath{clip}%
\pgfsetbuttcap%
\pgfsetmiterjoin%
\definecolor{currentfill}{rgb}{0.031373,0.285675,0.564291}%
\pgfsetfillcolor{currentfill}%
\pgfsetlinewidth{0.000000pt}%
\definecolor{currentstroke}{rgb}{0.000000,0.000000,0.000000}%
\pgfsetstrokecolor{currentstroke}%
\pgfsetstrokeopacity{0.000000}%
\pgfsetdash{}{0pt}%
\pgfpathmoveto{\pgfqpoint{1.566436in}{1.434578in}}%
\pgfpathlineto{\pgfqpoint{1.568472in}{1.430306in}}%
\pgfpathlineto{\pgfqpoint{1.572614in}{1.429955in}}%
\pgfpathlineto{\pgfqpoint{1.576855in}{1.422263in}}%
\pgfpathlineto{\pgfqpoint{1.584652in}{1.417090in}}%
\pgfpathlineto{\pgfqpoint{1.586747in}{1.410853in}}%
\pgfpathlineto{\pgfqpoint{1.590580in}{1.410117in}}%
\pgfpathlineto{\pgfqpoint{1.594388in}{1.405377in}}%
\pgfpathlineto{\pgfqpoint{1.602167in}{1.402735in}}%
\pgfpathlineto{\pgfqpoint{1.603438in}{1.399865in}}%
\pgfpathlineto{\pgfqpoint{1.566978in}{1.401393in}}%
\pgfpathlineto{\pgfqpoint{1.521452in}{1.403789in}}%
\pgfpathlineto{\pgfqpoint{1.521204in}{1.412401in}}%
\pgfpathlineto{\pgfqpoint{1.522496in}{1.435436in}}%
\pgfpathlineto{\pgfqpoint{1.522608in}{1.451050in}}%
\pgfpathlineto{\pgfqpoint{1.531144in}{1.449134in}}%
\pgfpathlineto{\pgfqpoint{1.541165in}{1.448356in}}%
\pgfpathlineto{\pgfqpoint{1.551671in}{1.451429in}}%
\pgfpathlineto{\pgfqpoint{1.550851in}{1.435352in}}%
\pgfpathlineto{\pgfqpoint{1.566436in}{1.434578in}}%
\pgfpathclose%
\pgfusepath{fill}%
\end{pgfscope}%
\begin{pgfscope}%
\pgfpathrectangle{\pgfqpoint{0.100000in}{0.100000in}}{\pgfqpoint{3.007045in}{1.925000in}}%
\pgfusepath{clip}%
\pgfsetbuttcap%
\pgfsetmiterjoin%
\definecolor{currentfill}{rgb}{0.447843,0.697855,0.845306}%
\pgfsetfillcolor{currentfill}%
\pgfsetlinewidth{0.000000pt}%
\definecolor{currentstroke}{rgb}{0.000000,0.000000,0.000000}%
\pgfsetstrokecolor{currentstroke}%
\pgfsetstrokeopacity{0.000000}%
\pgfsetdash{}{0pt}%
\pgfpathmoveto{\pgfqpoint{2.359552in}{0.655091in}}%
\pgfpathlineto{\pgfqpoint{2.343997in}{0.653135in}}%
\pgfpathlineto{\pgfqpoint{2.345074in}{0.640987in}}%
\pgfpathlineto{\pgfqpoint{2.339044in}{0.640515in}}%
\pgfpathlineto{\pgfqpoint{2.334338in}{0.646016in}}%
\pgfpathlineto{\pgfqpoint{2.332928in}{0.653635in}}%
\pgfpathlineto{\pgfqpoint{2.334926in}{0.671284in}}%
\pgfpathlineto{\pgfqpoint{2.333467in}{0.676210in}}%
\pgfpathlineto{\pgfqpoint{2.329091in}{0.680831in}}%
\pgfpathlineto{\pgfqpoint{2.312691in}{0.679850in}}%
\pgfpathlineto{\pgfqpoint{2.313278in}{0.674089in}}%
\pgfpathlineto{\pgfqpoint{2.294624in}{0.672055in}}%
\pgfpathlineto{\pgfqpoint{2.298312in}{0.683587in}}%
\pgfpathlineto{\pgfqpoint{2.297933in}{0.689896in}}%
\pgfpathlineto{\pgfqpoint{2.301073in}{0.698033in}}%
\pgfpathlineto{\pgfqpoint{2.309841in}{0.700126in}}%
\pgfpathlineto{\pgfqpoint{2.309942in}{0.708924in}}%
\pgfpathlineto{\pgfqpoint{2.318447in}{0.709939in}}%
\pgfpathlineto{\pgfqpoint{2.323124in}{0.704669in}}%
\pgfpathlineto{\pgfqpoint{2.330327in}{0.705545in}}%
\pgfpathlineto{\pgfqpoint{2.331743in}{0.699866in}}%
\pgfpathlineto{\pgfqpoint{2.339099in}{0.697275in}}%
\pgfpathlineto{\pgfqpoint{2.356903in}{0.698990in}}%
\pgfpathlineto{\pgfqpoint{2.357840in}{0.689197in}}%
\pgfpathlineto{\pgfqpoint{2.362322in}{0.683671in}}%
\pgfpathlineto{\pgfqpoint{2.367559in}{0.680287in}}%
\pgfpathlineto{\pgfqpoint{2.367869in}{0.675418in}}%
\pgfpathlineto{\pgfqpoint{2.370955in}{0.669671in}}%
\pgfpathlineto{\pgfqpoint{2.366771in}{0.667222in}}%
\pgfpathlineto{\pgfqpoint{2.358460in}{0.666668in}}%
\pgfpathlineto{\pgfqpoint{2.359552in}{0.655091in}}%
\pgfpathclose%
\pgfusepath{fill}%
\end{pgfscope}%
\begin{pgfscope}%
\pgfpathrectangle{\pgfqpoint{0.100000in}{0.100000in}}{\pgfqpoint{3.007045in}{1.925000in}}%
\pgfusepath{clip}%
\pgfsetbuttcap%
\pgfsetmiterjoin%
\definecolor{currentfill}{rgb}{0.681738,0.818562,0.904098}%
\pgfsetfillcolor{currentfill}%
\pgfsetlinewidth{0.000000pt}%
\definecolor{currentstroke}{rgb}{0.000000,0.000000,0.000000}%
\pgfsetstrokecolor{currentstroke}%
\pgfsetstrokeopacity{0.000000}%
\pgfsetdash{}{0pt}%
\pgfpathmoveto{\pgfqpoint{0.923792in}{1.677916in}}%
\pgfpathlineto{\pgfqpoint{0.921890in}{1.674121in}}%
\pgfpathlineto{\pgfqpoint{0.914135in}{1.673457in}}%
\pgfpathlineto{\pgfqpoint{0.908010in}{1.668930in}}%
\pgfpathlineto{\pgfqpoint{0.901414in}{1.666132in}}%
\pgfpathlineto{\pgfqpoint{0.894600in}{1.662965in}}%
\pgfpathlineto{\pgfqpoint{0.890559in}{1.659065in}}%
\pgfpathlineto{\pgfqpoint{0.879064in}{1.657147in}}%
\pgfpathlineto{\pgfqpoint{0.873352in}{1.663504in}}%
\pgfpathlineto{\pgfqpoint{0.871070in}{1.663297in}}%
\pgfpathlineto{\pgfqpoint{0.875599in}{1.669863in}}%
\pgfpathlineto{\pgfqpoint{0.874072in}{1.677313in}}%
\pgfpathlineto{\pgfqpoint{0.877655in}{1.681396in}}%
\pgfpathlineto{\pgfqpoint{0.883784in}{1.683111in}}%
\pgfpathlineto{\pgfqpoint{0.881292in}{1.694653in}}%
\pgfpathlineto{\pgfqpoint{0.885040in}{1.701682in}}%
\pgfpathlineto{\pgfqpoint{0.885687in}{1.708694in}}%
\pgfpathlineto{\pgfqpoint{0.891175in}{1.716853in}}%
\pgfpathlineto{\pgfqpoint{0.897388in}{1.731726in}}%
\pgfpathlineto{\pgfqpoint{0.895843in}{1.733369in}}%
\pgfpathlineto{\pgfqpoint{0.884196in}{1.733912in}}%
\pgfpathlineto{\pgfqpoint{0.884881in}{1.738405in}}%
\pgfpathlineto{\pgfqpoint{0.873285in}{1.749159in}}%
\pgfpathlineto{\pgfqpoint{0.873084in}{1.756371in}}%
\pgfpathlineto{\pgfqpoint{0.869191in}{1.761494in}}%
\pgfpathlineto{\pgfqpoint{0.861882in}{1.780983in}}%
\pgfpathlineto{\pgfqpoint{0.853297in}{1.785851in}}%
\pgfpathlineto{\pgfqpoint{0.853094in}{1.789354in}}%
\pgfpathlineto{\pgfqpoint{0.845810in}{1.797316in}}%
\pgfpathlineto{\pgfqpoint{0.851627in}{1.798562in}}%
\pgfpathlineto{\pgfqpoint{0.859543in}{1.795823in}}%
\pgfpathlineto{\pgfqpoint{0.866931in}{1.795554in}}%
\pgfpathlineto{\pgfqpoint{0.878913in}{1.786617in}}%
\pgfpathlineto{\pgfqpoint{0.878165in}{1.781128in}}%
\pgfpathlineto{\pgfqpoint{0.883118in}{1.777032in}}%
\pgfpathlineto{\pgfqpoint{0.890065in}{1.775890in}}%
\pgfpathlineto{\pgfqpoint{0.900754in}{1.769170in}}%
\pgfpathlineto{\pgfqpoint{0.906982in}{1.761729in}}%
\pgfpathlineto{\pgfqpoint{0.910196in}{1.762234in}}%
\pgfpathlineto{\pgfqpoint{0.921428in}{1.759848in}}%
\pgfpathlineto{\pgfqpoint{0.926970in}{1.761628in}}%
\pgfpathlineto{\pgfqpoint{0.926220in}{1.769071in}}%
\pgfpathlineto{\pgfqpoint{0.927695in}{1.775797in}}%
\pgfpathlineto{\pgfqpoint{0.925310in}{1.782011in}}%
\pgfpathlineto{\pgfqpoint{0.927953in}{1.789677in}}%
\pgfpathlineto{\pgfqpoint{0.948336in}{1.785640in}}%
\pgfpathlineto{\pgfqpoint{0.942694in}{1.758174in}}%
\pgfpathlineto{\pgfqpoint{0.949935in}{1.756722in}}%
\pgfpathlineto{\pgfqpoint{0.945308in}{1.734086in}}%
\pgfpathlineto{\pgfqpoint{0.940260in}{1.735106in}}%
\pgfpathlineto{\pgfqpoint{0.936258in}{1.730024in}}%
\pgfpathlineto{\pgfqpoint{0.927484in}{1.729880in}}%
\pgfpathlineto{\pgfqpoint{0.926725in}{1.726127in}}%
\pgfpathlineto{\pgfqpoint{0.921093in}{1.727298in}}%
\pgfpathlineto{\pgfqpoint{0.915399in}{1.718366in}}%
\pgfpathlineto{\pgfqpoint{0.914970in}{1.712349in}}%
\pgfpathlineto{\pgfqpoint{0.917545in}{1.708081in}}%
\pgfpathlineto{\pgfqpoint{0.917747in}{1.701623in}}%
\pgfpathlineto{\pgfqpoint{0.914111in}{1.694407in}}%
\pgfpathlineto{\pgfqpoint{0.913359in}{1.689243in}}%
\pgfpathlineto{\pgfqpoint{0.917372in}{1.685810in}}%
\pgfpathlineto{\pgfqpoint{0.918191in}{1.680868in}}%
\pgfpathlineto{\pgfqpoint{0.923792in}{1.677916in}}%
\pgfpathclose%
\pgfusepath{fill}%
\end{pgfscope}%
\begin{pgfscope}%
\pgfpathrectangle{\pgfqpoint{0.100000in}{0.100000in}}{\pgfqpoint{3.007045in}{1.925000in}}%
\pgfusepath{clip}%
\pgfsetbuttcap%
\pgfsetmiterjoin%
\definecolor{currentfill}{rgb}{0.541961,0.749527,0.865606}%
\pgfsetfillcolor{currentfill}%
\pgfsetlinewidth{0.000000pt}%
\definecolor{currentstroke}{rgb}{0.000000,0.000000,0.000000}%
\pgfsetstrokecolor{currentstroke}%
\pgfsetstrokeopacity{0.000000}%
\pgfsetdash{}{0pt}%
\pgfpathmoveto{\pgfqpoint{1.844441in}{1.606089in}}%
\pgfpathlineto{\pgfqpoint{1.861813in}{1.606700in}}%
\pgfpathlineto{\pgfqpoint{1.861397in}{1.624038in}}%
\pgfpathlineto{\pgfqpoint{1.896284in}{1.624795in}}%
\pgfpathlineto{\pgfqpoint{1.896759in}{1.607543in}}%
\pgfpathlineto{\pgfqpoint{1.896908in}{1.602038in}}%
\pgfpathlineto{\pgfqpoint{1.894425in}{1.598113in}}%
\pgfpathlineto{\pgfqpoint{1.881222in}{1.592179in}}%
\pgfpathlineto{\pgfqpoint{1.877917in}{1.589345in}}%
\pgfpathlineto{\pgfqpoint{1.874776in}{1.580802in}}%
\pgfpathlineto{\pgfqpoint{1.872272in}{1.578521in}}%
\pgfpathlineto{\pgfqpoint{1.841127in}{1.578186in}}%
\pgfpathlineto{\pgfqpoint{1.840662in}{1.594594in}}%
\pgfpathlineto{\pgfqpoint{1.844608in}{1.594634in}}%
\pgfpathlineto{\pgfqpoint{1.844441in}{1.606089in}}%
\pgfpathclose%
\pgfusepath{fill}%
\end{pgfscope}%
\begin{pgfscope}%
\pgfpathrectangle{\pgfqpoint{0.100000in}{0.100000in}}{\pgfqpoint{3.007045in}{1.925000in}}%
\pgfusepath{clip}%
\pgfsetbuttcap%
\pgfsetmiterjoin%
\definecolor{currentfill}{rgb}{0.301069,0.601399,0.792957}%
\pgfsetfillcolor{currentfill}%
\pgfsetlinewidth{0.000000pt}%
\definecolor{currentstroke}{rgb}{0.000000,0.000000,0.000000}%
\pgfsetstrokecolor{currentstroke}%
\pgfsetstrokeopacity{0.000000}%
\pgfsetdash{}{0pt}%
\pgfpathmoveto{\pgfqpoint{1.561807in}{1.005883in}}%
\pgfpathlineto{\pgfqpoint{1.537864in}{1.006990in}}%
\pgfpathlineto{\pgfqpoint{1.533400in}{1.007220in}}%
\pgfpathlineto{\pgfqpoint{1.534548in}{1.032337in}}%
\pgfpathlineto{\pgfqpoint{1.562899in}{1.031213in}}%
\pgfpathlineto{\pgfqpoint{1.561807in}{1.005883in}}%
\pgfpathclose%
\pgfusepath{fill}%
\end{pgfscope}%
\begin{pgfscope}%
\pgfpathrectangle{\pgfqpoint{0.100000in}{0.100000in}}{\pgfqpoint{3.007045in}{1.925000in}}%
\pgfusepath{clip}%
\pgfsetbuttcap%
\pgfsetmiterjoin%
\definecolor{currentfill}{rgb}{0.371688,0.649627,0.820515}%
\pgfsetfillcolor{currentfill}%
\pgfsetlinewidth{0.000000pt}%
\definecolor{currentstroke}{rgb}{0.000000,0.000000,0.000000}%
\pgfsetstrokecolor{currentstroke}%
\pgfsetstrokeopacity{0.000000}%
\pgfsetdash{}{0pt}%
\pgfpathmoveto{\pgfqpoint{2.174509in}{1.028592in}}%
\pgfpathlineto{\pgfqpoint{2.164956in}{1.024377in}}%
\pgfpathlineto{\pgfqpoint{2.156379in}{1.026602in}}%
\pgfpathlineto{\pgfqpoint{2.155476in}{1.018069in}}%
\pgfpathlineto{\pgfqpoint{2.152882in}{1.014762in}}%
\pgfpathlineto{\pgfqpoint{2.146362in}{1.013286in}}%
\pgfpathlineto{\pgfqpoint{2.132817in}{1.006098in}}%
\pgfpathlineto{\pgfqpoint{2.127814in}{1.020206in}}%
\pgfpathlineto{\pgfqpoint{2.129826in}{1.024933in}}%
\pgfpathlineto{\pgfqpoint{2.127608in}{1.032537in}}%
\pgfpathlineto{\pgfqpoint{2.119654in}{1.040653in}}%
\pgfpathlineto{\pgfqpoint{2.123429in}{1.044085in}}%
\pgfpathlineto{\pgfqpoint{2.133456in}{1.046059in}}%
\pgfpathlineto{\pgfqpoint{2.128798in}{1.058421in}}%
\pgfpathlineto{\pgfqpoint{2.134908in}{1.067855in}}%
\pgfpathlineto{\pgfqpoint{2.141157in}{1.068895in}}%
\pgfpathlineto{\pgfqpoint{2.139094in}{1.074231in}}%
\pgfpathlineto{\pgfqpoint{2.144755in}{1.073731in}}%
\pgfpathlineto{\pgfqpoint{2.152373in}{1.076112in}}%
\pgfpathlineto{\pgfqpoint{2.156206in}{1.071747in}}%
\pgfpathlineto{\pgfqpoint{2.158994in}{1.078092in}}%
\pgfpathlineto{\pgfqpoint{2.165193in}{1.079905in}}%
\pgfpathlineto{\pgfqpoint{2.172010in}{1.077414in}}%
\pgfpathlineto{\pgfqpoint{2.171424in}{1.072069in}}%
\pgfpathlineto{\pgfqpoint{2.164342in}{1.057050in}}%
\pgfpathlineto{\pgfqpoint{2.170940in}{1.054187in}}%
\pgfpathlineto{\pgfqpoint{2.172018in}{1.046217in}}%
\pgfpathlineto{\pgfqpoint{2.174985in}{1.043028in}}%
\pgfpathlineto{\pgfqpoint{2.171148in}{1.034806in}}%
\pgfpathlineto{\pgfqpoint{2.174509in}{1.028592in}}%
\pgfpathclose%
\pgfusepath{fill}%
\end{pgfscope}%
\begin{pgfscope}%
\pgfpathrectangle{\pgfqpoint{0.100000in}{0.100000in}}{\pgfqpoint{3.007045in}{1.925000in}}%
\pgfusepath{clip}%
\pgfsetbuttcap%
\pgfsetmiterjoin%
\definecolor{currentfill}{rgb}{0.171027,0.484752,0.731242}%
\pgfsetfillcolor{currentfill}%
\pgfsetlinewidth{0.000000pt}%
\definecolor{currentstroke}{rgb}{0.000000,0.000000,0.000000}%
\pgfsetstrokecolor{currentstroke}%
\pgfsetstrokeopacity{0.000000}%
\pgfsetdash{}{0pt}%
\pgfpathmoveto{\pgfqpoint{1.381806in}{1.702240in}}%
\pgfpathlineto{\pgfqpoint{1.379578in}{1.679082in}}%
\pgfpathlineto{\pgfqpoint{1.382174in}{1.678842in}}%
\pgfpathlineto{\pgfqpoint{1.379987in}{1.655666in}}%
\pgfpathlineto{\pgfqpoint{1.371277in}{1.656487in}}%
\pgfpathlineto{\pgfqpoint{1.370707in}{1.650612in}}%
\pgfpathlineto{\pgfqpoint{1.359600in}{1.651719in}}%
\pgfpathlineto{\pgfqpoint{1.360257in}{1.658366in}}%
\pgfpathlineto{\pgfqpoint{1.346246in}{1.659789in}}%
\pgfpathlineto{\pgfqpoint{1.343644in}{1.662962in}}%
\pgfpathlineto{\pgfqpoint{1.335063in}{1.663866in}}%
\pgfpathlineto{\pgfqpoint{1.336444in}{1.675484in}}%
\pgfpathlineto{\pgfqpoint{1.317293in}{1.677601in}}%
\pgfpathlineto{\pgfqpoint{1.315821in}{1.681622in}}%
\pgfpathlineto{\pgfqpoint{1.310066in}{1.682269in}}%
\pgfpathlineto{\pgfqpoint{1.310508in}{1.686101in}}%
\pgfpathlineto{\pgfqpoint{1.304728in}{1.686756in}}%
\pgfpathlineto{\pgfqpoint{1.306273in}{1.700261in}}%
\pgfpathlineto{\pgfqpoint{1.302614in}{1.700688in}}%
\pgfpathlineto{\pgfqpoint{1.305247in}{1.723687in}}%
\pgfpathlineto{\pgfqpoint{1.307173in}{1.723478in}}%
\pgfpathlineto{\pgfqpoint{1.309172in}{1.740759in}}%
\pgfpathlineto{\pgfqpoint{1.314888in}{1.740131in}}%
\pgfpathlineto{\pgfqpoint{1.314224in}{1.734361in}}%
\pgfpathlineto{\pgfqpoint{1.325673in}{1.733068in}}%
\pgfpathlineto{\pgfqpoint{1.325013in}{1.727299in}}%
\pgfpathlineto{\pgfqpoint{1.342199in}{1.725419in}}%
\pgfpathlineto{\pgfqpoint{1.341580in}{1.719656in}}%
\pgfpathlineto{\pgfqpoint{1.345202in}{1.719293in}}%
\pgfpathlineto{\pgfqpoint{1.343969in}{1.707734in}}%
\pgfpathlineto{\pgfqpoint{1.361096in}{1.705933in}}%
\pgfpathlineto{\pgfqpoint{1.364804in}{1.703953in}}%
\pgfpathlineto{\pgfqpoint{1.381806in}{1.702240in}}%
\pgfpathclose%
\pgfusepath{fill}%
\end{pgfscope}%
\begin{pgfscope}%
\pgfpathrectangle{\pgfqpoint{0.100000in}{0.100000in}}{\pgfqpoint{3.007045in}{1.925000in}}%
\pgfusepath{clip}%
\pgfsetbuttcap%
\pgfsetmiterjoin%
\definecolor{currentfill}{rgb}{0.598431,0.780531,0.877785}%
\pgfsetfillcolor{currentfill}%
\pgfsetlinewidth{0.000000pt}%
\definecolor{currentstroke}{rgb}{0.000000,0.000000,0.000000}%
\pgfsetstrokecolor{currentstroke}%
\pgfsetstrokeopacity{0.000000}%
\pgfsetdash{}{0pt}%
\pgfpathmoveto{\pgfqpoint{2.128961in}{1.204856in}}%
\pgfpathlineto{\pgfqpoint{2.128488in}{1.192008in}}%
\pgfpathlineto{\pgfqpoint{2.123761in}{1.189439in}}%
\pgfpathlineto{\pgfqpoint{2.103104in}{1.187835in}}%
\pgfpathlineto{\pgfqpoint{2.105221in}{1.159153in}}%
\pgfpathlineto{\pgfqpoint{2.071143in}{1.157025in}}%
\pgfpathlineto{\pgfqpoint{2.070604in}{1.165647in}}%
\pgfpathlineto{\pgfqpoint{2.076427in}{1.165836in}}%
\pgfpathlineto{\pgfqpoint{2.075079in}{1.186079in}}%
\pgfpathlineto{\pgfqpoint{2.069337in}{1.185778in}}%
\pgfpathlineto{\pgfqpoint{2.068581in}{1.195334in}}%
\pgfpathlineto{\pgfqpoint{2.064705in}{1.195908in}}%
\pgfpathlineto{\pgfqpoint{2.064257in}{1.202733in}}%
\pgfpathlineto{\pgfqpoint{2.068009in}{1.203031in}}%
\pgfpathlineto{\pgfqpoint{2.067386in}{1.211625in}}%
\pgfpathlineto{\pgfqpoint{2.087435in}{1.213369in}}%
\pgfpathlineto{\pgfqpoint{2.094960in}{1.228863in}}%
\pgfpathlineto{\pgfqpoint{2.100744in}{1.229280in}}%
\pgfpathlineto{\pgfqpoint{2.099165in}{1.251343in}}%
\pgfpathlineto{\pgfqpoint{2.110362in}{1.252216in}}%
\pgfpathlineto{\pgfqpoint{2.107905in}{1.276923in}}%
\pgfpathlineto{\pgfqpoint{2.113613in}{1.277542in}}%
\pgfpathlineto{\pgfqpoint{2.116838in}{1.244113in}}%
\pgfpathlineto{\pgfqpoint{2.125964in}{1.244692in}}%
\pgfpathlineto{\pgfqpoint{2.126956in}{1.227545in}}%
\pgfpathlineto{\pgfqpoint{2.128961in}{1.204856in}}%
\pgfpathclose%
\pgfusepath{fill}%
\end{pgfscope}%
\begin{pgfscope}%
\pgfpathrectangle{\pgfqpoint{0.100000in}{0.100000in}}{\pgfqpoint{3.007045in}{1.925000in}}%
\pgfusepath{clip}%
\pgfsetbuttcap%
\pgfsetmiterjoin%
\definecolor{currentfill}{rgb}{0.171027,0.484752,0.731242}%
\pgfsetfillcolor{currentfill}%
\pgfsetlinewidth{0.000000pt}%
\definecolor{currentstroke}{rgb}{0.000000,0.000000,0.000000}%
\pgfsetstrokecolor{currentstroke}%
\pgfsetstrokeopacity{0.000000}%
\pgfsetdash{}{0pt}%
\pgfpathmoveto{\pgfqpoint{1.381806in}{1.702240in}}%
\pgfpathlineto{\pgfqpoint{1.364804in}{1.703953in}}%
\pgfpathlineto{\pgfqpoint{1.361096in}{1.705933in}}%
\pgfpathlineto{\pgfqpoint{1.343969in}{1.707734in}}%
\pgfpathlineto{\pgfqpoint{1.345202in}{1.719293in}}%
\pgfpathlineto{\pgfqpoint{1.341580in}{1.719656in}}%
\pgfpathlineto{\pgfqpoint{1.342199in}{1.725419in}}%
\pgfpathlineto{\pgfqpoint{1.325013in}{1.727299in}}%
\pgfpathlineto{\pgfqpoint{1.325673in}{1.733068in}}%
\pgfpathlineto{\pgfqpoint{1.314224in}{1.734361in}}%
\pgfpathlineto{\pgfqpoint{1.314888in}{1.740131in}}%
\pgfpathlineto{\pgfqpoint{1.317232in}{1.745061in}}%
\pgfpathlineto{\pgfqpoint{1.318715in}{1.758150in}}%
\pgfpathlineto{\pgfqpoint{1.324957in}{1.756945in}}%
\pgfpathlineto{\pgfqpoint{1.335725in}{1.759869in}}%
\pgfpathlineto{\pgfqpoint{1.341818in}{1.758120in}}%
\pgfpathlineto{\pgfqpoint{1.351556in}{1.758299in}}%
\pgfpathlineto{\pgfqpoint{1.355902in}{1.752996in}}%
\pgfpathlineto{\pgfqpoint{1.366019in}{1.751221in}}%
\pgfpathlineto{\pgfqpoint{1.369270in}{1.748112in}}%
\pgfpathlineto{\pgfqpoint{1.376795in}{1.745675in}}%
\pgfpathlineto{\pgfqpoint{1.386114in}{1.754885in}}%
\pgfpathlineto{\pgfqpoint{1.389403in}{1.754089in}}%
\pgfpathlineto{\pgfqpoint{1.390611in}{1.748693in}}%
\pgfpathlineto{\pgfqpoint{1.394175in}{1.746414in}}%
\pgfpathlineto{\pgfqpoint{1.403722in}{1.748968in}}%
\pgfpathlineto{\pgfqpoint{1.407533in}{1.753794in}}%
\pgfpathlineto{\pgfqpoint{1.422411in}{1.752597in}}%
\pgfpathlineto{\pgfqpoint{1.425483in}{1.768138in}}%
\pgfpathlineto{\pgfqpoint{1.422948in}{1.768339in}}%
\pgfpathlineto{\pgfqpoint{1.423931in}{1.779899in}}%
\pgfpathlineto{\pgfqpoint{1.452658in}{1.777477in}}%
\pgfpathlineto{\pgfqpoint{1.464168in}{1.776657in}}%
\pgfpathlineto{\pgfqpoint{1.463268in}{1.765000in}}%
\pgfpathlineto{\pgfqpoint{1.465533in}{1.764842in}}%
\pgfpathlineto{\pgfqpoint{1.463751in}{1.741609in}}%
\pgfpathlineto{\pgfqpoint{1.466009in}{1.741430in}}%
\pgfpathlineto{\pgfqpoint{1.465127in}{1.729765in}}%
\pgfpathlineto{\pgfqpoint{1.442195in}{1.731563in}}%
\pgfpathlineto{\pgfqpoint{1.441706in}{1.725605in}}%
\pgfpathlineto{\pgfqpoint{1.433959in}{1.727722in}}%
\pgfpathlineto{\pgfqpoint{1.429757in}{1.720971in}}%
\pgfpathlineto{\pgfqpoint{1.409324in}{1.722778in}}%
\pgfpathlineto{\pgfqpoint{1.407305in}{1.699840in}}%
\pgfpathlineto{\pgfqpoint{1.381806in}{1.702240in}}%
\pgfpathclose%
\pgfusepath{fill}%
\end{pgfscope}%
\begin{pgfscope}%
\pgfpathrectangle{\pgfqpoint{0.100000in}{0.100000in}}{\pgfqpoint{3.007045in}{1.925000in}}%
\pgfusepath{clip}%
\pgfsetbuttcap%
\pgfsetmiterjoin%
\definecolor{currentfill}{rgb}{0.968627,0.984314,1.000000}%
\pgfsetfillcolor{currentfill}%
\pgfsetlinewidth{0.000000pt}%
\definecolor{currentstroke}{rgb}{0.000000,0.000000,0.000000}%
\pgfsetstrokecolor{currentstroke}%
\pgfsetstrokeopacity{0.000000}%
\pgfsetdash{}{0pt}%
\pgfpathmoveto{\pgfqpoint{2.962613in}{1.457141in}}%
\pgfpathlineto{\pgfqpoint{2.972406in}{1.450863in}}%
\pgfpathlineto{\pgfqpoint{2.960151in}{1.447007in}}%
\pgfpathlineto{\pgfqpoint{2.958968in}{1.452964in}}%
\pgfpathlineto{\pgfqpoint{2.962613in}{1.457141in}}%
\pgfpathclose%
\pgfusepath{fill}%
\end{pgfscope}%
\begin{pgfscope}%
\pgfpathrectangle{\pgfqpoint{0.100000in}{0.100000in}}{\pgfqpoint{3.007045in}{1.925000in}}%
\pgfusepath{clip}%
\pgfsetbuttcap%
\pgfsetmiterjoin%
\definecolor{currentfill}{rgb}{0.617255,0.790865,0.881845}%
\pgfsetfillcolor{currentfill}%
\pgfsetlinewidth{0.000000pt}%
\definecolor{currentstroke}{rgb}{0.000000,0.000000,0.000000}%
\pgfsetstrokecolor{currentstroke}%
\pgfsetstrokeopacity{0.000000}%
\pgfsetdash{}{0pt}%
\pgfpathmoveto{\pgfqpoint{2.644603in}{1.131634in}}%
\pgfpathlineto{\pgfqpoint{2.641185in}{1.127114in}}%
\pgfpathlineto{\pgfqpoint{2.641373in}{1.123470in}}%
\pgfpathlineto{\pgfqpoint{2.633816in}{1.102244in}}%
\pgfpathlineto{\pgfqpoint{2.626460in}{1.104802in}}%
\pgfpathlineto{\pgfqpoint{2.621502in}{1.105990in}}%
\pgfpathlineto{\pgfqpoint{2.612770in}{1.112672in}}%
\pgfpathlineto{\pgfqpoint{2.613630in}{1.118818in}}%
\pgfpathlineto{\pgfqpoint{2.617784in}{1.122931in}}%
\pgfpathlineto{\pgfqpoint{2.616773in}{1.129743in}}%
\pgfpathlineto{\pgfqpoint{2.606282in}{1.144642in}}%
\pgfpathlineto{\pgfqpoint{2.608923in}{1.148127in}}%
\pgfpathlineto{\pgfqpoint{2.609006in}{1.155780in}}%
\pgfpathlineto{\pgfqpoint{2.613166in}{1.160149in}}%
\pgfpathlineto{\pgfqpoint{2.619686in}{1.171111in}}%
\pgfpathlineto{\pgfqpoint{2.621002in}{1.178691in}}%
\pgfpathlineto{\pgfqpoint{2.624882in}{1.186857in}}%
\pgfpathlineto{\pgfqpoint{2.631429in}{1.181706in}}%
\pgfpathlineto{\pgfqpoint{2.634662in}{1.181683in}}%
\pgfpathlineto{\pgfqpoint{2.644402in}{1.195106in}}%
\pgfpathlineto{\pgfqpoint{2.648785in}{1.188481in}}%
\pgfpathlineto{\pgfqpoint{2.658001in}{1.178784in}}%
\pgfpathlineto{\pgfqpoint{2.663143in}{1.179525in}}%
\pgfpathlineto{\pgfqpoint{2.650904in}{1.157743in}}%
\pgfpathlineto{\pgfqpoint{2.656779in}{1.157595in}}%
\pgfpathlineto{\pgfqpoint{2.665522in}{1.153025in}}%
\pgfpathlineto{\pgfqpoint{2.663369in}{1.133914in}}%
\pgfpathlineto{\pgfqpoint{2.658319in}{1.135059in}}%
\pgfpathlineto{\pgfqpoint{2.654207in}{1.139998in}}%
\pgfpathlineto{\pgfqpoint{2.647609in}{1.142718in}}%
\pgfpathlineto{\pgfqpoint{2.644603in}{1.131634in}}%
\pgfpathclose%
\pgfusepath{fill}%
\end{pgfscope}%
\begin{pgfscope}%
\pgfpathrectangle{\pgfqpoint{0.100000in}{0.100000in}}{\pgfqpoint{3.007045in}{1.925000in}}%
\pgfusepath{clip}%
\pgfsetbuttcap%
\pgfsetmiterjoin%
\definecolor{currentfill}{rgb}{0.199446,0.513172,0.745882}%
\pgfsetfillcolor{currentfill}%
\pgfsetlinewidth{0.000000pt}%
\definecolor{currentstroke}{rgb}{0.000000,0.000000,0.000000}%
\pgfsetstrokecolor{currentstroke}%
\pgfsetstrokeopacity{0.000000}%
\pgfsetdash{}{0pt}%
\pgfpathmoveto{\pgfqpoint{1.444186in}{0.892364in}}%
\pgfpathlineto{\pgfqpoint{1.442027in}{0.863792in}}%
\pgfpathlineto{\pgfqpoint{1.421405in}{0.865168in}}%
\pgfpathlineto{\pgfqpoint{1.392891in}{0.867332in}}%
\pgfpathlineto{\pgfqpoint{1.364391in}{0.869531in}}%
\pgfpathlineto{\pgfqpoint{1.336407in}{0.871962in}}%
\pgfpathlineto{\pgfqpoint{1.337618in}{0.885562in}}%
\pgfpathlineto{\pgfqpoint{1.341578in}{0.929505in}}%
\pgfpathlineto{\pgfqpoint{1.388570in}{0.925898in}}%
\pgfpathlineto{\pgfqpoint{1.417413in}{0.923222in}}%
\pgfpathlineto{\pgfqpoint{1.415306in}{0.894451in}}%
\pgfpathlineto{\pgfqpoint{1.444186in}{0.892364in}}%
\pgfpathclose%
\pgfusepath{fill}%
\end{pgfscope}%
\begin{pgfscope}%
\pgfpathrectangle{\pgfqpoint{0.100000in}{0.100000in}}{\pgfqpoint{3.007045in}{1.925000in}}%
\pgfusepath{clip}%
\pgfsetbuttcap%
\pgfsetmiterjoin%
\definecolor{currentfill}{rgb}{0.326290,0.618624,0.802799}%
\pgfsetfillcolor{currentfill}%
\pgfsetlinewidth{0.000000pt}%
\definecolor{currentstroke}{rgb}{0.000000,0.000000,0.000000}%
\pgfsetstrokecolor{currentstroke}%
\pgfsetstrokeopacity{0.000000}%
\pgfsetdash{}{0pt}%
\pgfpathmoveto{\pgfqpoint{1.918185in}{0.809351in}}%
\pgfpathlineto{\pgfqpoint{1.909437in}{0.814697in}}%
\pgfpathlineto{\pgfqpoint{1.899320in}{0.814784in}}%
\pgfpathlineto{\pgfqpoint{1.899400in}{0.832281in}}%
\pgfpathlineto{\pgfqpoint{1.898412in}{0.836142in}}%
\pgfpathlineto{\pgfqpoint{1.892466in}{0.838016in}}%
\pgfpathlineto{\pgfqpoint{1.891528in}{0.843729in}}%
\pgfpathlineto{\pgfqpoint{1.887792in}{0.846596in}}%
\pgfpathlineto{\pgfqpoint{1.882114in}{0.846605in}}%
\pgfpathlineto{\pgfqpoint{1.882494in}{0.855536in}}%
\pgfpathlineto{\pgfqpoint{1.865214in}{0.855241in}}%
\pgfpathlineto{\pgfqpoint{1.864117in}{0.861985in}}%
\pgfpathlineto{\pgfqpoint{1.871864in}{0.868767in}}%
\pgfpathlineto{\pgfqpoint{1.884196in}{0.881637in}}%
\pgfpathlineto{\pgfqpoint{1.888066in}{0.882118in}}%
\pgfpathlineto{\pgfqpoint{1.887849in}{0.901272in}}%
\pgfpathlineto{\pgfqpoint{1.907800in}{0.901293in}}%
\pgfpathlineto{\pgfqpoint{1.907859in}{0.895543in}}%
\pgfpathlineto{\pgfqpoint{1.927775in}{0.895762in}}%
\pgfpathlineto{\pgfqpoint{1.927950in}{0.876198in}}%
\pgfpathlineto{\pgfqpoint{1.933967in}{0.876483in}}%
\pgfpathlineto{\pgfqpoint{1.941240in}{0.873795in}}%
\pgfpathlineto{\pgfqpoint{1.945167in}{0.874395in}}%
\pgfpathlineto{\pgfqpoint{1.945345in}{0.867212in}}%
\pgfpathlineto{\pgfqpoint{1.951312in}{0.867300in}}%
\pgfpathlineto{\pgfqpoint{1.951519in}{0.850943in}}%
\pgfpathlineto{\pgfqpoint{1.955013in}{0.847952in}}%
\pgfpathlineto{\pgfqpoint{1.955232in}{0.842505in}}%
\pgfpathlineto{\pgfqpoint{1.951736in}{0.838426in}}%
\pgfpathlineto{\pgfqpoint{1.924362in}{0.838176in}}%
\pgfpathlineto{\pgfqpoint{1.924611in}{0.820832in}}%
\pgfpathlineto{\pgfqpoint{1.923831in}{0.809696in}}%
\pgfpathlineto{\pgfqpoint{1.918185in}{0.809351in}}%
\pgfpathclose%
\pgfusepath{fill}%
\end{pgfscope}%
\begin{pgfscope}%
\pgfpathrectangle{\pgfqpoint{0.100000in}{0.100000in}}{\pgfqpoint{3.007045in}{1.925000in}}%
\pgfusepath{clip}%
\pgfsetbuttcap%
\pgfsetmiterjoin%
\definecolor{currentfill}{rgb}{0.183206,0.496932,0.737516}%
\pgfsetfillcolor{currentfill}%
\pgfsetlinewidth{0.000000pt}%
\definecolor{currentstroke}{rgb}{0.000000,0.000000,0.000000}%
\pgfsetstrokecolor{currentstroke}%
\pgfsetstrokeopacity{0.000000}%
\pgfsetdash{}{0pt}%
\pgfpathmoveto{\pgfqpoint{1.311526in}{1.606441in}}%
\pgfpathlineto{\pgfqpoint{1.308128in}{1.577801in}}%
\pgfpathlineto{\pgfqpoint{1.305764in}{1.578064in}}%
\pgfpathlineto{\pgfqpoint{1.303187in}{1.554906in}}%
\pgfpathlineto{\pgfqpoint{1.262489in}{1.559658in}}%
\pgfpathlineto{\pgfqpoint{1.257251in}{1.559802in}}%
\pgfpathlineto{\pgfqpoint{1.246154in}{1.561183in}}%
\pgfpathlineto{\pgfqpoint{1.247672in}{1.573410in}}%
\pgfpathlineto{\pgfqpoint{1.224309in}{1.576380in}}%
\pgfpathlineto{\pgfqpoint{1.227555in}{1.587451in}}%
\pgfpathlineto{\pgfqpoint{1.230291in}{1.608576in}}%
\pgfpathlineto{\pgfqpoint{1.221965in}{1.610407in}}%
\pgfpathlineto{\pgfqpoint{1.222383in}{1.622797in}}%
\pgfpathlineto{\pgfqpoint{1.212290in}{1.629948in}}%
\pgfpathlineto{\pgfqpoint{1.200926in}{1.631495in}}%
\pgfpathlineto{\pgfqpoint{1.201711in}{1.637190in}}%
\pgfpathlineto{\pgfqpoint{1.197757in}{1.637726in}}%
\pgfpathlineto{\pgfqpoint{1.201035in}{1.646271in}}%
\pgfpathlineto{\pgfqpoint{1.195568in}{1.655776in}}%
\pgfpathlineto{\pgfqpoint{1.192292in}{1.666923in}}%
\pgfpathlineto{\pgfqpoint{1.188964in}{1.669306in}}%
\pgfpathlineto{\pgfqpoint{1.190856in}{1.681412in}}%
\pgfpathlineto{\pgfqpoint{1.189998in}{1.686616in}}%
\pgfpathlineto{\pgfqpoint{1.187986in}{1.693328in}}%
\pgfpathlineto{\pgfqpoint{1.240870in}{1.686663in}}%
\pgfpathlineto{\pgfqpoint{1.240737in}{1.685718in}}%
\pgfpathlineto{\pgfqpoint{1.269340in}{1.682158in}}%
\pgfpathlineto{\pgfqpoint{1.267668in}{1.681381in}}%
\pgfpathlineto{\pgfqpoint{1.264869in}{1.658439in}}%
\pgfpathlineto{\pgfqpoint{1.263286in}{1.658626in}}%
\pgfpathlineto{\pgfqpoint{1.260561in}{1.635790in}}%
\pgfpathlineto{\pgfqpoint{1.258817in}{1.635994in}}%
\pgfpathlineto{\pgfqpoint{1.255957in}{1.613050in}}%
\pgfpathlineto{\pgfqpoint{1.311526in}{1.606441in}}%
\pgfpathclose%
\pgfusepath{fill}%
\end{pgfscope}%
\begin{pgfscope}%
\pgfpathrectangle{\pgfqpoint{0.100000in}{0.100000in}}{\pgfqpoint{3.007045in}{1.925000in}}%
\pgfusepath{clip}%
\pgfsetbuttcap%
\pgfsetmiterjoin%
\definecolor{currentfill}{rgb}{0.290980,0.594510,0.789020}%
\pgfsetfillcolor{currentfill}%
\pgfsetlinewidth{0.000000pt}%
\definecolor{currentstroke}{rgb}{0.000000,0.000000,0.000000}%
\pgfsetstrokecolor{currentstroke}%
\pgfsetstrokeopacity{0.000000}%
\pgfsetdash{}{0pt}%
\pgfpathmoveto{\pgfqpoint{2.498935in}{0.687218in}}%
\pgfpathlineto{\pgfqpoint{2.492063in}{0.691931in}}%
\pgfpathlineto{\pgfqpoint{2.477917in}{0.698299in}}%
\pgfpathlineto{\pgfqpoint{2.457652in}{0.694920in}}%
\pgfpathlineto{\pgfqpoint{2.455203in}{0.702082in}}%
\pgfpathlineto{\pgfqpoint{2.447789in}{0.700910in}}%
\pgfpathlineto{\pgfqpoint{2.437033in}{0.704008in}}%
\pgfpathlineto{\pgfqpoint{2.433519in}{0.709598in}}%
\pgfpathlineto{\pgfqpoint{2.429949in}{0.711648in}}%
\pgfpathlineto{\pgfqpoint{2.428514in}{0.718344in}}%
\pgfpathlineto{\pgfqpoint{2.425917in}{0.719822in}}%
\pgfpathlineto{\pgfqpoint{2.427863in}{0.724959in}}%
\pgfpathlineto{\pgfqpoint{2.422656in}{0.729020in}}%
\pgfpathlineto{\pgfqpoint{2.425722in}{0.735586in}}%
\pgfpathlineto{\pgfqpoint{2.432076in}{0.743806in}}%
\pgfpathlineto{\pgfqpoint{2.434230in}{0.742118in}}%
\pgfpathlineto{\pgfqpoint{2.444853in}{0.725072in}}%
\pgfpathlineto{\pgfqpoint{2.451822in}{0.729156in}}%
\pgfpathlineto{\pgfqpoint{2.458380in}{0.737732in}}%
\pgfpathlineto{\pgfqpoint{2.457386in}{0.739684in}}%
\pgfpathlineto{\pgfqpoint{2.460691in}{0.751658in}}%
\pgfpathlineto{\pgfqpoint{2.469526in}{0.751792in}}%
\pgfpathlineto{\pgfqpoint{2.476049in}{0.748944in}}%
\pgfpathlineto{\pgfqpoint{2.475118in}{0.743801in}}%
\pgfpathlineto{\pgfqpoint{2.478955in}{0.739426in}}%
\pgfpathlineto{\pgfqpoint{2.485433in}{0.742115in}}%
\pgfpathlineto{\pgfqpoint{2.489704in}{0.732476in}}%
\pgfpathlineto{\pgfqpoint{2.489650in}{0.715531in}}%
\pgfpathlineto{\pgfqpoint{2.496462in}{0.714530in}}%
\pgfpathlineto{\pgfqpoint{2.499221in}{0.711875in}}%
\pgfpathlineto{\pgfqpoint{2.496078in}{0.707079in}}%
\pgfpathlineto{\pgfqpoint{2.498935in}{0.687218in}}%
\pgfpathclose%
\pgfusepath{fill}%
\end{pgfscope}%
\begin{pgfscope}%
\pgfpathrectangle{\pgfqpoint{0.100000in}{0.100000in}}{\pgfqpoint{3.007045in}{1.925000in}}%
\pgfusepath{clip}%
\pgfsetbuttcap%
\pgfsetmiterjoin%
\definecolor{currentfill}{rgb}{0.429020,0.687520,0.841246}%
\pgfsetfillcolor{currentfill}%
\pgfsetlinewidth{0.000000pt}%
\definecolor{currentstroke}{rgb}{0.000000,0.000000,0.000000}%
\pgfsetstrokecolor{currentstroke}%
\pgfsetstrokeopacity{0.000000}%
\pgfsetdash{}{0pt}%
\pgfpathmoveto{\pgfqpoint{2.741244in}{1.095623in}}%
\pgfpathlineto{\pgfqpoint{2.738779in}{1.094235in}}%
\pgfpathlineto{\pgfqpoint{2.723519in}{1.072852in}}%
\pgfpathlineto{\pgfqpoint{2.723871in}{1.066185in}}%
\pgfpathlineto{\pgfqpoint{2.701345in}{1.062144in}}%
\pgfpathlineto{\pgfqpoint{2.695089in}{1.071478in}}%
\pgfpathlineto{\pgfqpoint{2.716714in}{1.091684in}}%
\pgfpathlineto{\pgfqpoint{2.713742in}{1.098160in}}%
\pgfpathlineto{\pgfqpoint{2.704080in}{1.100551in}}%
\pgfpathlineto{\pgfqpoint{2.711789in}{1.111134in}}%
\pgfpathlineto{\pgfqpoint{2.716713in}{1.110416in}}%
\pgfpathlineto{\pgfqpoint{2.713382in}{1.123521in}}%
\pgfpathlineto{\pgfqpoint{2.724698in}{1.127666in}}%
\pgfpathlineto{\pgfqpoint{2.723470in}{1.137350in}}%
\pgfpathlineto{\pgfqpoint{2.716860in}{1.144435in}}%
\pgfpathlineto{\pgfqpoint{2.720436in}{1.148513in}}%
\pgfpathlineto{\pgfqpoint{2.725113in}{1.146693in}}%
\pgfpathlineto{\pgfqpoint{2.729649in}{1.140417in}}%
\pgfpathlineto{\pgfqpoint{2.735476in}{1.140922in}}%
\pgfpathlineto{\pgfqpoint{2.739950in}{1.134391in}}%
\pgfpathlineto{\pgfqpoint{2.744726in}{1.133904in}}%
\pgfpathlineto{\pgfqpoint{2.747294in}{1.127106in}}%
\pgfpathlineto{\pgfqpoint{2.744962in}{1.123172in}}%
\pgfpathlineto{\pgfqpoint{2.740631in}{1.125917in}}%
\pgfpathlineto{\pgfqpoint{2.742495in}{1.113720in}}%
\pgfpathlineto{\pgfqpoint{2.750051in}{1.107970in}}%
\pgfpathlineto{\pgfqpoint{2.749079in}{1.102041in}}%
\pgfpathlineto{\pgfqpoint{2.745560in}{1.100962in}}%
\pgfpathlineto{\pgfqpoint{2.741244in}{1.095623in}}%
\pgfpathclose%
\pgfusepath{fill}%
\end{pgfscope}%
\begin{pgfscope}%
\pgfpathrectangle{\pgfqpoint{0.100000in}{0.100000in}}{\pgfqpoint{3.007045in}{1.925000in}}%
\pgfusepath{clip}%
\pgfsetbuttcap%
\pgfsetmiterjoin%
\definecolor{currentfill}{rgb}{0.516863,0.735748,0.860192}%
\pgfsetfillcolor{currentfill}%
\pgfsetlinewidth{0.000000pt}%
\definecolor{currentstroke}{rgb}{0.000000,0.000000,0.000000}%
\pgfsetstrokecolor{currentstroke}%
\pgfsetstrokeopacity{0.000000}%
\pgfsetdash{}{0pt}%
\pgfpathmoveto{\pgfqpoint{2.386694in}{0.571228in}}%
\pgfpathlineto{\pgfqpoint{2.386777in}{0.566973in}}%
\pgfpathlineto{\pgfqpoint{2.379860in}{0.568127in}}%
\pgfpathlineto{\pgfqpoint{2.372613in}{0.564198in}}%
\pgfpathlineto{\pgfqpoint{2.358211in}{0.553384in}}%
\pgfpathlineto{\pgfqpoint{2.349321in}{0.550333in}}%
\pgfpathlineto{\pgfqpoint{2.344406in}{0.550144in}}%
\pgfpathlineto{\pgfqpoint{2.336004in}{0.546671in}}%
\pgfpathlineto{\pgfqpoint{2.332928in}{0.547891in}}%
\pgfpathlineto{\pgfqpoint{2.332519in}{0.554903in}}%
\pgfpathlineto{\pgfqpoint{2.324536in}{0.563497in}}%
\pgfpathlineto{\pgfqpoint{2.320470in}{0.564006in}}%
\pgfpathlineto{\pgfqpoint{2.314549in}{0.569015in}}%
\pgfpathlineto{\pgfqpoint{2.302272in}{0.575567in}}%
\pgfpathlineto{\pgfqpoint{2.289539in}{0.581381in}}%
\pgfpathlineto{\pgfqpoint{2.288959in}{0.589362in}}%
\pgfpathlineto{\pgfqpoint{2.296561in}{0.593554in}}%
\pgfpathlineto{\pgfqpoint{2.317482in}{0.595459in}}%
\pgfpathlineto{\pgfqpoint{2.316671in}{0.604111in}}%
\pgfpathlineto{\pgfqpoint{2.322486in}{0.604682in}}%
\pgfpathlineto{\pgfqpoint{2.324742in}{0.580445in}}%
\pgfpathlineto{\pgfqpoint{2.339480in}{0.580648in}}%
\pgfpathlineto{\pgfqpoint{2.340428in}{0.571556in}}%
\pgfpathlineto{\pgfqpoint{2.347979in}{0.570392in}}%
\pgfpathlineto{\pgfqpoint{2.374564in}{0.573201in}}%
\pgfpathlineto{\pgfqpoint{2.386694in}{0.571228in}}%
\pgfpathclose%
\pgfusepath{fill}%
\end{pgfscope}%
\begin{pgfscope}%
\pgfpathrectangle{\pgfqpoint{0.100000in}{0.100000in}}{\pgfqpoint{3.007045in}{1.925000in}}%
\pgfusepath{clip}%
\pgfsetbuttcap%
\pgfsetmiterjoin%
\definecolor{currentfill}{rgb}{0.652211,0.806013,0.893764}%
\pgfsetfillcolor{currentfill}%
\pgfsetlinewidth{0.000000pt}%
\definecolor{currentstroke}{rgb}{0.000000,0.000000,0.000000}%
\pgfsetstrokecolor{currentstroke}%
\pgfsetstrokeopacity{0.000000}%
\pgfsetdash{}{0pt}%
\pgfpathmoveto{\pgfqpoint{2.578534in}{1.060078in}}%
\pgfpathlineto{\pgfqpoint{2.578144in}{1.039001in}}%
\pgfpathlineto{\pgfqpoint{2.561819in}{1.036295in}}%
\pgfpathlineto{\pgfqpoint{2.547798in}{1.034084in}}%
\pgfpathlineto{\pgfqpoint{2.531083in}{1.032312in}}%
\pgfpathlineto{\pgfqpoint{2.529927in}{1.036274in}}%
\pgfpathlineto{\pgfqpoint{2.537514in}{1.041038in}}%
\pgfpathlineto{\pgfqpoint{2.537236in}{1.043373in}}%
\pgfpathlineto{\pgfqpoint{2.543457in}{1.053394in}}%
\pgfpathlineto{\pgfqpoint{2.547424in}{1.055944in}}%
\pgfpathlineto{\pgfqpoint{2.572984in}{1.054602in}}%
\pgfpathlineto{\pgfqpoint{2.578534in}{1.060078in}}%
\pgfpathclose%
\pgfusepath{fill}%
\end{pgfscope}%
\begin{pgfscope}%
\pgfpathrectangle{\pgfqpoint{0.100000in}{0.100000in}}{\pgfqpoint{3.007045in}{1.925000in}}%
\pgfusepath{clip}%
\pgfsetbuttcap%
\pgfsetmiterjoin%
\definecolor{currentfill}{rgb}{0.311157,0.608289,0.796894}%
\pgfsetfillcolor{currentfill}%
\pgfsetlinewidth{0.000000pt}%
\definecolor{currentstroke}{rgb}{0.000000,0.000000,0.000000}%
\pgfsetstrokecolor{currentstroke}%
\pgfsetstrokeopacity{0.000000}%
\pgfsetdash{}{0pt}%
\pgfpathmoveto{\pgfqpoint{1.286524in}{0.867195in}}%
\pgfpathlineto{\pgfqpoint{1.283287in}{0.832832in}}%
\pgfpathlineto{\pgfqpoint{1.271855in}{0.834010in}}%
\pgfpathlineto{\pgfqpoint{1.271248in}{0.828230in}}%
\pgfpathlineto{\pgfqpoint{1.236948in}{0.832126in}}%
\pgfpathlineto{\pgfqpoint{1.237556in}{0.837894in}}%
\pgfpathlineto{\pgfqpoint{1.231742in}{0.838561in}}%
\pgfpathlineto{\pgfqpoint{1.233670in}{0.855551in}}%
\pgfpathlineto{\pgfqpoint{1.210832in}{0.858083in}}%
\pgfpathlineto{\pgfqpoint{1.212799in}{0.875172in}}%
\pgfpathlineto{\pgfqpoint{1.214018in}{0.875058in}}%
\pgfpathlineto{\pgfqpoint{1.217220in}{0.903588in}}%
\pgfpathlineto{\pgfqpoint{1.218495in}{0.915021in}}%
\pgfpathlineto{\pgfqpoint{1.281006in}{0.908273in}}%
\pgfpathlineto{\pgfqpoint{1.280556in}{0.903442in}}%
\pgfpathlineto{\pgfqpoint{1.276783in}{0.868180in}}%
\pgfpathlineto{\pgfqpoint{1.286524in}{0.867195in}}%
\pgfpathclose%
\pgfusepath{fill}%
\end{pgfscope}%
\begin{pgfscope}%
\pgfpathrectangle{\pgfqpoint{0.100000in}{0.100000in}}{\pgfqpoint{3.007045in}{1.925000in}}%
\pgfusepath{clip}%
\pgfsetbuttcap%
\pgfsetmiterjoin%
\definecolor{currentfill}{rgb}{0.632526,0.797647,0.886874}%
\pgfsetfillcolor{currentfill}%
\pgfsetlinewidth{0.000000pt}%
\definecolor{currentstroke}{rgb}{0.000000,0.000000,0.000000}%
\pgfsetstrokecolor{currentstroke}%
\pgfsetstrokeopacity{0.000000}%
\pgfsetdash{}{0pt}%
\pgfpathmoveto{\pgfqpoint{2.959007in}{1.458513in}}%
\pgfpathlineto{\pgfqpoint{2.957672in}{1.458775in}}%
\pgfpathlineto{\pgfqpoint{2.959346in}{1.462818in}}%
\pgfpathlineto{\pgfqpoint{2.955719in}{1.471405in}}%
\pgfpathlineto{\pgfqpoint{2.948844in}{1.463490in}}%
\pgfpathlineto{\pgfqpoint{2.944479in}{1.471321in}}%
\pgfpathlineto{\pgfqpoint{2.933228in}{1.482168in}}%
\pgfpathlineto{\pgfqpoint{2.929528in}{1.490206in}}%
\pgfpathlineto{\pgfqpoint{2.917977in}{1.479575in}}%
\pgfpathlineto{\pgfqpoint{2.916591in}{1.481492in}}%
\pgfpathlineto{\pgfqpoint{2.896973in}{1.475592in}}%
\pgfpathlineto{\pgfqpoint{2.880647in}{1.472875in}}%
\pgfpathlineto{\pgfqpoint{2.878530in}{1.481285in}}%
\pgfpathlineto{\pgfqpoint{2.872055in}{1.481134in}}%
\pgfpathlineto{\pgfqpoint{2.873175in}{1.485821in}}%
\pgfpathlineto{\pgfqpoint{2.867051in}{1.490726in}}%
\pgfpathlineto{\pgfqpoint{2.865008in}{1.506125in}}%
\pgfpathlineto{\pgfqpoint{2.865846in}{1.512168in}}%
\pgfpathlineto{\pgfqpoint{2.862374in}{1.515277in}}%
\pgfpathlineto{\pgfqpoint{2.879160in}{1.518963in}}%
\pgfpathlineto{\pgfqpoint{2.909180in}{1.525824in}}%
\pgfpathlineto{\pgfqpoint{2.914625in}{1.535783in}}%
\pgfpathlineto{\pgfqpoint{2.922960in}{1.542355in}}%
\pgfpathlineto{\pgfqpoint{2.928476in}{1.542999in}}%
\pgfpathlineto{\pgfqpoint{2.934550in}{1.531865in}}%
\pgfpathlineto{\pgfqpoint{2.942818in}{1.530677in}}%
\pgfpathlineto{\pgfqpoint{2.939653in}{1.526607in}}%
\pgfpathlineto{\pgfqpoint{2.932124in}{1.521534in}}%
\pgfpathlineto{\pgfqpoint{2.933679in}{1.519559in}}%
\pgfpathlineto{\pgfqpoint{2.928376in}{1.511191in}}%
\pgfpathlineto{\pgfqpoint{2.930098in}{1.505248in}}%
\pgfpathlineto{\pgfqpoint{2.935189in}{1.506115in}}%
\pgfpathlineto{\pgfqpoint{2.944880in}{1.501863in}}%
\pgfpathlineto{\pgfqpoint{2.950388in}{1.495521in}}%
\pgfpathlineto{\pgfqpoint{2.949426in}{1.488196in}}%
\pgfpathlineto{\pgfqpoint{2.958190in}{1.486046in}}%
\pgfpathlineto{\pgfqpoint{2.959875in}{1.479333in}}%
\pgfpathlineto{\pgfqpoint{2.966716in}{1.476431in}}%
\pgfpathlineto{\pgfqpoint{2.973137in}{1.477451in}}%
\pgfpathlineto{\pgfqpoint{2.988983in}{1.486082in}}%
\pgfpathlineto{\pgfqpoint{2.991233in}{1.478471in}}%
\pgfpathlineto{\pgfqpoint{2.968516in}{1.467337in}}%
\pgfpathlineto{\pgfqpoint{2.967427in}{1.463763in}}%
\pgfpathlineto{\pgfqpoint{2.959007in}{1.458513in}}%
\pgfpathclose%
\pgfusepath{fill}%
\end{pgfscope}%
\begin{pgfscope}%
\pgfpathrectangle{\pgfqpoint{0.100000in}{0.100000in}}{\pgfqpoint{3.007045in}{1.925000in}}%
\pgfusepath{clip}%
\pgfsetbuttcap%
\pgfsetmiterjoin%
\definecolor{currentfill}{rgb}{0.093272,0.396878,0.673664}%
\pgfsetfillcolor{currentfill}%
\pgfsetlinewidth{0.000000pt}%
\definecolor{currentstroke}{rgb}{0.000000,0.000000,0.000000}%
\pgfsetstrokecolor{currentstroke}%
\pgfsetstrokeopacity{0.000000}%
\pgfsetdash{}{0pt}%
\pgfpathmoveto{\pgfqpoint{0.678803in}{0.558557in}}%
\pgfpathlineto{\pgfqpoint{0.679997in}{0.561480in}}%
\pgfpathlineto{\pgfqpoint{0.680052in}{0.564426in}}%
\pgfpathlineto{\pgfqpoint{0.679031in}{0.568728in}}%
\pgfpathlineto{\pgfqpoint{0.679664in}{0.571354in}}%
\pgfpathlineto{\pgfqpoint{0.678267in}{0.576636in}}%
\pgfpathlineto{\pgfqpoint{0.676653in}{0.578838in}}%
\pgfpathlineto{\pgfqpoint{0.671826in}{0.579804in}}%
\pgfpathlineto{\pgfqpoint{0.671241in}{0.581751in}}%
\pgfpathlineto{\pgfqpoint{0.671704in}{0.585285in}}%
\pgfpathlineto{\pgfqpoint{0.673540in}{0.587189in}}%
\pgfpathlineto{\pgfqpoint{0.675922in}{0.587938in}}%
\pgfpathlineto{\pgfqpoint{0.678005in}{0.589551in}}%
\pgfpathlineto{\pgfqpoint{0.678705in}{0.588737in}}%
\pgfpathlineto{\pgfqpoint{0.677833in}{0.586473in}}%
\pgfpathlineto{\pgfqpoint{0.680118in}{0.580300in}}%
\pgfpathlineto{\pgfqpoint{0.681622in}{0.579560in}}%
\pgfpathlineto{\pgfqpoint{0.686508in}{0.578300in}}%
\pgfpathlineto{\pgfqpoint{0.687618in}{0.575930in}}%
\pgfpathlineto{\pgfqpoint{0.687530in}{0.573360in}}%
\pgfpathlineto{\pgfqpoint{0.685839in}{0.571449in}}%
\pgfpathlineto{\pgfqpoint{0.687044in}{0.567758in}}%
\pgfpathlineto{\pgfqpoint{0.687213in}{0.565076in}}%
\pgfpathlineto{\pgfqpoint{0.688961in}{0.561585in}}%
\pgfpathlineto{\pgfqpoint{0.691199in}{0.560100in}}%
\pgfpathlineto{\pgfqpoint{0.692475in}{0.557535in}}%
\pgfpathlineto{\pgfqpoint{0.691676in}{0.556369in}}%
\pgfpathlineto{\pgfqpoint{0.688983in}{0.555610in}}%
\pgfpathlineto{\pgfqpoint{0.687659in}{0.557811in}}%
\pgfpathlineto{\pgfqpoint{0.685611in}{0.559618in}}%
\pgfpathlineto{\pgfqpoint{0.682975in}{0.560083in}}%
\pgfpathlineto{\pgfqpoint{0.678803in}{0.558557in}}%
\pgfpathclose%
\pgfusepath{fill}%
\end{pgfscope}%
\begin{pgfscope}%
\pgfpathrectangle{\pgfqpoint{0.100000in}{0.100000in}}{\pgfqpoint{3.007045in}{1.925000in}}%
\pgfusepath{clip}%
\pgfsetbuttcap%
\pgfsetmiterjoin%
\definecolor{currentfill}{rgb}{0.093272,0.396878,0.673664}%
\pgfsetfillcolor{currentfill}%
\pgfsetlinewidth{0.000000pt}%
\definecolor{currentstroke}{rgb}{0.000000,0.000000,0.000000}%
\pgfsetstrokecolor{currentstroke}%
\pgfsetstrokeopacity{0.000000}%
\pgfsetdash{}{0pt}%
\pgfpathmoveto{\pgfqpoint{0.784268in}{0.583082in}}%
\pgfpathlineto{\pgfqpoint{0.782998in}{0.580535in}}%
\pgfpathlineto{\pgfqpoint{0.776856in}{0.574080in}}%
\pgfpathlineto{\pgfqpoint{0.777383in}{0.573526in}}%
\pgfpathlineto{\pgfqpoint{0.771216in}{0.567098in}}%
\pgfpathlineto{\pgfqpoint{0.773326in}{0.565108in}}%
\pgfpathlineto{\pgfqpoint{0.770248in}{0.561865in}}%
\pgfpathlineto{\pgfqpoint{0.771871in}{0.560318in}}%
\pgfpathlineto{\pgfqpoint{0.768863in}{0.557118in}}%
\pgfpathlineto{\pgfqpoint{0.775661in}{0.550783in}}%
\pgfpathlineto{\pgfqpoint{0.790016in}{0.537796in}}%
\pgfpathlineto{\pgfqpoint{0.799988in}{0.528978in}}%
\pgfpathlineto{\pgfqpoint{0.801466in}{0.530632in}}%
\pgfpathlineto{\pgfqpoint{0.803129in}{0.529197in}}%
\pgfpathlineto{\pgfqpoint{0.804798in}{0.527778in}}%
\pgfpathlineto{\pgfqpoint{0.800473in}{0.522699in}}%
\pgfpathlineto{\pgfqpoint{0.798922in}{0.524065in}}%
\pgfpathlineto{\pgfqpoint{0.793312in}{0.517519in}}%
\pgfpathlineto{\pgfqpoint{0.791670in}{0.518914in}}%
\pgfpathlineto{\pgfqpoint{0.788832in}{0.515575in}}%
\pgfpathlineto{\pgfqpoint{0.787134in}{0.516986in}}%
\pgfpathlineto{\pgfqpoint{0.775774in}{0.503541in}}%
\pgfpathlineto{\pgfqpoint{0.777669in}{0.501945in}}%
\pgfpathlineto{\pgfqpoint{0.773225in}{0.497185in}}%
\pgfpathlineto{\pgfqpoint{0.771756in}{0.498478in}}%
\pgfpathlineto{\pgfqpoint{0.770286in}{0.496757in}}%
\pgfpathlineto{\pgfqpoint{0.768805in}{0.498004in}}%
\pgfpathlineto{\pgfqpoint{0.767409in}{0.496302in}}%
\pgfpathlineto{\pgfqpoint{0.764086in}{0.499185in}}%
\pgfpathlineto{\pgfqpoint{0.762810in}{0.497757in}}%
\pgfpathlineto{\pgfqpoint{0.761153in}{0.499198in}}%
\pgfpathlineto{\pgfqpoint{0.756937in}{0.494067in}}%
\pgfpathlineto{\pgfqpoint{0.755431in}{0.495345in}}%
\pgfpathlineto{\pgfqpoint{0.751165in}{0.490590in}}%
\pgfpathlineto{\pgfqpoint{0.741587in}{0.499176in}}%
\pgfpathlineto{\pgfqpoint{0.742883in}{0.500799in}}%
\pgfpathlineto{\pgfqpoint{0.741434in}{0.502134in}}%
\pgfpathlineto{\pgfqpoint{0.742884in}{0.503844in}}%
\pgfpathlineto{\pgfqpoint{0.741312in}{0.505279in}}%
\pgfpathlineto{\pgfqpoint{0.742738in}{0.506896in}}%
\pgfpathlineto{\pgfqpoint{0.741050in}{0.508543in}}%
\pgfpathlineto{\pgfqpoint{0.746586in}{0.510216in}}%
\pgfpathlineto{\pgfqpoint{0.746712in}{0.512326in}}%
\pgfpathlineto{\pgfqpoint{0.745027in}{0.513729in}}%
\pgfpathlineto{\pgfqpoint{0.746433in}{0.515178in}}%
\pgfpathlineto{\pgfqpoint{0.748958in}{0.513680in}}%
\pgfpathlineto{\pgfqpoint{0.748481in}{0.510045in}}%
\pgfpathlineto{\pgfqpoint{0.750210in}{0.505718in}}%
\pgfpathlineto{\pgfqpoint{0.755351in}{0.502374in}}%
\pgfpathlineto{\pgfqpoint{0.757588in}{0.502208in}}%
\pgfpathlineto{\pgfqpoint{0.761221in}{0.502993in}}%
\pgfpathlineto{\pgfqpoint{0.764189in}{0.504114in}}%
\pgfpathlineto{\pgfqpoint{0.765417in}{0.505253in}}%
\pgfpathlineto{\pgfqpoint{0.767984in}{0.511139in}}%
\pgfpathlineto{\pgfqpoint{0.770424in}{0.514447in}}%
\pgfpathlineto{\pgfqpoint{0.771090in}{0.519688in}}%
\pgfpathlineto{\pgfqpoint{0.769265in}{0.521681in}}%
\pgfpathlineto{\pgfqpoint{0.771550in}{0.523004in}}%
\pgfpathlineto{\pgfqpoint{0.773952in}{0.519957in}}%
\pgfpathlineto{\pgfqpoint{0.778165in}{0.520246in}}%
\pgfpathlineto{\pgfqpoint{0.780075in}{0.522157in}}%
\pgfpathlineto{\pgfqpoint{0.781283in}{0.524482in}}%
\pgfpathlineto{\pgfqpoint{0.780153in}{0.528966in}}%
\pgfpathlineto{\pgfqpoint{0.776623in}{0.527750in}}%
\pgfpathlineto{\pgfqpoint{0.774378in}{0.527915in}}%
\pgfpathlineto{\pgfqpoint{0.773486in}{0.530158in}}%
\pgfpathlineto{\pgfqpoint{0.770430in}{0.530368in}}%
\pgfpathlineto{\pgfqpoint{0.767348in}{0.531211in}}%
\pgfpathlineto{\pgfqpoint{0.764853in}{0.531128in}}%
\pgfpathlineto{\pgfqpoint{0.757318in}{0.529341in}}%
\pgfpathlineto{\pgfqpoint{0.759189in}{0.532678in}}%
\pgfpathlineto{\pgfqpoint{0.759643in}{0.537554in}}%
\pgfpathlineto{\pgfqpoint{0.757195in}{0.537867in}}%
\pgfpathlineto{\pgfqpoint{0.757549in}{0.535521in}}%
\pgfpathlineto{\pgfqpoint{0.755714in}{0.534840in}}%
\pgfpathlineto{\pgfqpoint{0.755754in}{0.537318in}}%
\pgfpathlineto{\pgfqpoint{0.754836in}{0.539913in}}%
\pgfpathlineto{\pgfqpoint{0.749644in}{0.544946in}}%
\pgfpathlineto{\pgfqpoint{0.743760in}{0.546795in}}%
\pgfpathlineto{\pgfqpoint{0.742220in}{0.547611in}}%
\pgfpathlineto{\pgfqpoint{0.735804in}{0.558236in}}%
\pgfpathlineto{\pgfqpoint{0.734630in}{0.560668in}}%
\pgfpathlineto{\pgfqpoint{0.734596in}{0.562774in}}%
\pgfpathlineto{\pgfqpoint{0.736774in}{0.565731in}}%
\pgfpathlineto{\pgfqpoint{0.738729in}{0.567328in}}%
\pgfpathlineto{\pgfqpoint{0.739066in}{0.574040in}}%
\pgfpathlineto{\pgfqpoint{0.742356in}{0.572549in}}%
\pgfpathlineto{\pgfqpoint{0.746458in}{0.574359in}}%
\pgfpathlineto{\pgfqpoint{0.742466in}{0.581233in}}%
\pgfpathlineto{\pgfqpoint{0.740777in}{0.583420in}}%
\pgfpathlineto{\pgfqpoint{0.739815in}{0.588701in}}%
\pgfpathlineto{\pgfqpoint{0.738785in}{0.592111in}}%
\pgfpathlineto{\pgfqpoint{0.740864in}{0.594187in}}%
\pgfpathlineto{\pgfqpoint{0.750011in}{0.591581in}}%
\pgfpathlineto{\pgfqpoint{0.750589in}{0.589917in}}%
\pgfpathlineto{\pgfqpoint{0.753942in}{0.589343in}}%
\pgfpathlineto{\pgfqpoint{0.753820in}{0.591119in}}%
\pgfpathlineto{\pgfqpoint{0.762748in}{0.589868in}}%
\pgfpathlineto{\pgfqpoint{0.764937in}{0.585166in}}%
\pgfpathlineto{\pgfqpoint{0.766490in}{0.583615in}}%
\pgfpathlineto{\pgfqpoint{0.768297in}{0.583840in}}%
\pgfpathlineto{\pgfqpoint{0.767184in}{0.586211in}}%
\pgfpathlineto{\pgfqpoint{0.770001in}{0.588417in}}%
\pgfpathlineto{\pgfqpoint{0.777196in}{0.586185in}}%
\pgfpathlineto{\pgfqpoint{0.781761in}{0.584481in}}%
\pgfpathlineto{\pgfqpoint{0.784268in}{0.583082in}}%
\pgfpathclose%
\pgfusepath{fill}%
\end{pgfscope}%
\begin{pgfscope}%
\pgfpathrectangle{\pgfqpoint{0.100000in}{0.100000in}}{\pgfqpoint{3.007045in}{1.925000in}}%
\pgfusepath{clip}%
\pgfsetbuttcap%
\pgfsetmiterjoin%
\definecolor{currentfill}{rgb}{0.031373,0.285675,0.564291}%
\pgfsetfillcolor{currentfill}%
\pgfsetlinewidth{0.000000pt}%
\definecolor{currentstroke}{rgb}{0.000000,0.000000,0.000000}%
\pgfsetstrokecolor{currentstroke}%
\pgfsetstrokeopacity{0.000000}%
\pgfsetdash{}{0pt}%
\pgfpathmoveto{\pgfqpoint{1.521452in}{1.403789in}}%
\pgfpathlineto{\pgfqpoint{1.471839in}{1.406868in}}%
\pgfpathlineto{\pgfqpoint{1.473591in}{1.432656in}}%
\pgfpathlineto{\pgfqpoint{1.475452in}{1.459167in}}%
\pgfpathlineto{\pgfqpoint{1.483456in}{1.461953in}}%
\pgfpathlineto{\pgfqpoint{1.490267in}{1.460196in}}%
\pgfpathlineto{\pgfqpoint{1.496174in}{1.453745in}}%
\pgfpathlineto{\pgfqpoint{1.506479in}{1.454584in}}%
\pgfpathlineto{\pgfqpoint{1.511045in}{1.452183in}}%
\pgfpathlineto{\pgfqpoint{1.517464in}{1.451554in}}%
\pgfpathlineto{\pgfqpoint{1.522608in}{1.451050in}}%
\pgfpathlineto{\pgfqpoint{1.522496in}{1.435436in}}%
\pgfpathlineto{\pgfqpoint{1.521204in}{1.412401in}}%
\pgfpathlineto{\pgfqpoint{1.521452in}{1.403789in}}%
\pgfpathclose%
\pgfusepath{fill}%
\end{pgfscope}%
\begin{pgfscope}%
\pgfpathrectangle{\pgfqpoint{0.100000in}{0.100000in}}{\pgfqpoint{3.007045in}{1.925000in}}%
\pgfusepath{clip}%
\pgfsetbuttcap%
\pgfsetmiterjoin%
\definecolor{currentfill}{rgb}{0.485490,0.718524,0.853426}%
\pgfsetfillcolor{currentfill}%
\pgfsetlinewidth{0.000000pt}%
\definecolor{currentstroke}{rgb}{0.000000,0.000000,0.000000}%
\pgfsetstrokecolor{currentstroke}%
\pgfsetstrokeopacity{0.000000}%
\pgfsetdash{}{0pt}%
\pgfpathmoveto{\pgfqpoint{0.763180in}{1.655715in}}%
\pgfpathlineto{\pgfqpoint{0.759381in}{1.652797in}}%
\pgfpathlineto{\pgfqpoint{0.758646in}{1.647388in}}%
\pgfpathlineto{\pgfqpoint{0.754151in}{1.641671in}}%
\pgfpathlineto{\pgfqpoint{0.746366in}{1.637288in}}%
\pgfpathlineto{\pgfqpoint{0.738581in}{1.626973in}}%
\pgfpathlineto{\pgfqpoint{0.733470in}{1.622014in}}%
\pgfpathlineto{\pgfqpoint{0.730914in}{1.610662in}}%
\pgfpathlineto{\pgfqpoint{0.718698in}{1.613732in}}%
\pgfpathlineto{\pgfqpoint{0.720108in}{1.619262in}}%
\pgfpathlineto{\pgfqpoint{0.716216in}{1.624121in}}%
\pgfpathlineto{\pgfqpoint{0.698781in}{1.628508in}}%
\pgfpathlineto{\pgfqpoint{0.695577in}{1.627414in}}%
\pgfpathlineto{\pgfqpoint{0.687625in}{1.619603in}}%
\pgfpathlineto{\pgfqpoint{0.683738in}{1.619619in}}%
\pgfpathlineto{\pgfqpoint{0.671575in}{1.622748in}}%
\pgfpathlineto{\pgfqpoint{0.677402in}{1.629304in}}%
\pgfpathlineto{\pgfqpoint{0.678278in}{1.634204in}}%
\pgfpathlineto{\pgfqpoint{0.685898in}{1.641698in}}%
\pgfpathlineto{\pgfqpoint{0.684947in}{1.644999in}}%
\pgfpathlineto{\pgfqpoint{0.678193in}{1.649100in}}%
\pgfpathlineto{\pgfqpoint{0.678258in}{1.651609in}}%
\pgfpathlineto{\pgfqpoint{0.689599in}{1.651922in}}%
\pgfpathlineto{\pgfqpoint{0.689503in}{1.657799in}}%
\pgfpathlineto{\pgfqpoint{0.694447in}{1.659187in}}%
\pgfpathlineto{\pgfqpoint{0.695518in}{1.663103in}}%
\pgfpathlineto{\pgfqpoint{0.690132in}{1.668259in}}%
\pgfpathlineto{\pgfqpoint{0.683125in}{1.670416in}}%
\pgfpathlineto{\pgfqpoint{0.685286in}{1.683617in}}%
\pgfpathlineto{\pgfqpoint{0.680362in}{1.684919in}}%
\pgfpathlineto{\pgfqpoint{0.681006in}{1.694917in}}%
\pgfpathlineto{\pgfqpoint{0.693304in}{1.692341in}}%
\pgfpathlineto{\pgfqpoint{0.694542in}{1.697073in}}%
\pgfpathlineto{\pgfqpoint{0.704883in}{1.694237in}}%
\pgfpathlineto{\pgfqpoint{0.709253in}{1.696086in}}%
\pgfpathlineto{\pgfqpoint{0.712881in}{1.710060in}}%
\pgfpathlineto{\pgfqpoint{0.715135in}{1.709465in}}%
\pgfpathlineto{\pgfqpoint{0.718256in}{1.717477in}}%
\pgfpathlineto{\pgfqpoint{0.722204in}{1.719436in}}%
\pgfpathlineto{\pgfqpoint{0.732306in}{1.716868in}}%
\pgfpathlineto{\pgfqpoint{0.727679in}{1.706314in}}%
\pgfpathlineto{\pgfqpoint{0.729208in}{1.700974in}}%
\pgfpathlineto{\pgfqpoint{0.727519in}{1.694386in}}%
\pgfpathlineto{\pgfqpoint{0.728377in}{1.682025in}}%
\pgfpathlineto{\pgfqpoint{0.732751in}{1.676168in}}%
\pgfpathlineto{\pgfqpoint{0.733120in}{1.669073in}}%
\pgfpathlineto{\pgfqpoint{0.742633in}{1.666720in}}%
\pgfpathlineto{\pgfqpoint{0.741293in}{1.661247in}}%
\pgfpathlineto{\pgfqpoint{0.763180in}{1.655715in}}%
\pgfpathclose%
\pgfusepath{fill}%
\end{pgfscope}%
\begin{pgfscope}%
\pgfpathrectangle{\pgfqpoint{0.100000in}{0.100000in}}{\pgfqpoint{3.007045in}{1.925000in}}%
\pgfusepath{clip}%
\pgfsetbuttcap%
\pgfsetmiterjoin%
\definecolor{currentfill}{rgb}{0.351511,0.635848,0.812641}%
\pgfsetfillcolor{currentfill}%
\pgfsetlinewidth{0.000000pt}%
\definecolor{currentstroke}{rgb}{0.000000,0.000000,0.000000}%
\pgfsetstrokecolor{currentstroke}%
\pgfsetstrokeopacity{0.000000}%
\pgfsetdash{}{0pt}%
\pgfpathmoveto{\pgfqpoint{1.509458in}{1.149192in}}%
\pgfpathlineto{\pgfqpoint{1.507527in}{1.120611in}}%
\pgfpathlineto{\pgfqpoint{1.480135in}{1.122427in}}%
\pgfpathlineto{\pgfqpoint{1.439278in}{1.125169in}}%
\pgfpathlineto{\pgfqpoint{1.441646in}{1.153706in}}%
\pgfpathlineto{\pgfqpoint{1.446054in}{1.153418in}}%
\pgfpathlineto{\pgfqpoint{1.448195in}{1.181868in}}%
\pgfpathlineto{\pgfqpoint{1.509436in}{1.177873in}}%
\pgfpathlineto{\pgfqpoint{1.510385in}{1.177845in}}%
\pgfpathlineto{\pgfqpoint{1.508618in}{1.149222in}}%
\pgfpathlineto{\pgfqpoint{1.509458in}{1.149192in}}%
\pgfpathclose%
\pgfusepath{fill}%
\end{pgfscope}%
\begin{pgfscope}%
\pgfpathrectangle{\pgfqpoint{0.100000in}{0.100000in}}{\pgfqpoint{3.007045in}{1.925000in}}%
\pgfusepath{clip}%
\pgfsetbuttcap%
\pgfsetmiterjoin%
\definecolor{currentfill}{rgb}{0.435294,0.690965,0.842599}%
\pgfsetfillcolor{currentfill}%
\pgfsetlinewidth{0.000000pt}%
\definecolor{currentstroke}{rgb}{0.000000,0.000000,0.000000}%
\pgfsetstrokecolor{currentstroke}%
\pgfsetstrokeopacity{0.000000}%
\pgfsetdash{}{0pt}%
\pgfpathmoveto{\pgfqpoint{2.070604in}{1.165647in}}%
\pgfpathlineto{\pgfqpoint{2.050719in}{1.164389in}}%
\pgfpathlineto{\pgfqpoint{2.049876in}{1.175940in}}%
\pgfpathlineto{\pgfqpoint{2.029984in}{1.174625in}}%
\pgfpathlineto{\pgfqpoint{2.025601in}{1.191743in}}%
\pgfpathlineto{\pgfqpoint{2.024381in}{1.212989in}}%
\pgfpathlineto{\pgfqpoint{2.025473in}{1.215274in}}%
\pgfpathlineto{\pgfqpoint{2.032407in}{1.214497in}}%
\pgfpathlineto{\pgfqpoint{2.039739in}{1.216849in}}%
\pgfpathlineto{\pgfqpoint{2.044100in}{1.215046in}}%
\pgfpathlineto{\pgfqpoint{2.043262in}{1.228036in}}%
\pgfpathlineto{\pgfqpoint{2.060239in}{1.229418in}}%
\pgfpathlineto{\pgfqpoint{2.066190in}{1.226950in}}%
\pgfpathlineto{\pgfqpoint{2.067386in}{1.211625in}}%
\pgfpathlineto{\pgfqpoint{2.068009in}{1.203031in}}%
\pgfpathlineto{\pgfqpoint{2.064257in}{1.202733in}}%
\pgfpathlineto{\pgfqpoint{2.064705in}{1.195908in}}%
\pgfpathlineto{\pgfqpoint{2.068581in}{1.195334in}}%
\pgfpathlineto{\pgfqpoint{2.069337in}{1.185778in}}%
\pgfpathlineto{\pgfqpoint{2.075079in}{1.186079in}}%
\pgfpathlineto{\pgfqpoint{2.076427in}{1.165836in}}%
\pgfpathlineto{\pgfqpoint{2.070604in}{1.165647in}}%
\pgfpathclose%
\pgfusepath{fill}%
\end{pgfscope}%
\begin{pgfscope}%
\pgfpathrectangle{\pgfqpoint{0.100000in}{0.100000in}}{\pgfqpoint{3.007045in}{1.925000in}}%
\pgfusepath{clip}%
\pgfsetbuttcap%
\pgfsetmiterjoin%
\definecolor{currentfill}{rgb}{0.491765,0.721968,0.854779}%
\pgfsetfillcolor{currentfill}%
\pgfsetlinewidth{0.000000pt}%
\definecolor{currentstroke}{rgb}{0.000000,0.000000,0.000000}%
\pgfsetstrokecolor{currentstroke}%
\pgfsetstrokeopacity{0.000000}%
\pgfsetdash{}{0pt}%
\pgfpathmoveto{\pgfqpoint{2.803517in}{1.405975in}}%
\pgfpathlineto{\pgfqpoint{2.794033in}{1.391447in}}%
\pgfpathlineto{\pgfqpoint{2.768627in}{1.400117in}}%
\pgfpathlineto{\pgfqpoint{2.765754in}{1.403877in}}%
\pgfpathlineto{\pgfqpoint{2.760698in}{1.403338in}}%
\pgfpathlineto{\pgfqpoint{2.752320in}{1.406665in}}%
\pgfpathlineto{\pgfqpoint{2.747986in}{1.413389in}}%
\pgfpathlineto{\pgfqpoint{2.745672in}{1.421591in}}%
\pgfpathlineto{\pgfqpoint{2.739778in}{1.426983in}}%
\pgfpathlineto{\pgfqpoint{2.768356in}{1.454967in}}%
\pgfpathlineto{\pgfqpoint{2.776040in}{1.453025in}}%
\pgfpathlineto{\pgfqpoint{2.787399in}{1.454470in}}%
\pgfpathlineto{\pgfqpoint{2.787830in}{1.459571in}}%
\pgfpathlineto{\pgfqpoint{2.794361in}{1.458299in}}%
\pgfpathlineto{\pgfqpoint{2.795846in}{1.455210in}}%
\pgfpathlineto{\pgfqpoint{2.806124in}{1.452725in}}%
\pgfpathlineto{\pgfqpoint{2.815310in}{1.453061in}}%
\pgfpathlineto{\pgfqpoint{2.815641in}{1.458005in}}%
\pgfpathlineto{\pgfqpoint{2.816097in}{1.458112in}}%
\pgfpathlineto{\pgfqpoint{2.822125in}{1.424128in}}%
\pgfpathlineto{\pgfqpoint{2.803333in}{1.417011in}}%
\pgfpathlineto{\pgfqpoint{2.803517in}{1.405975in}}%
\pgfpathclose%
\pgfusepath{fill}%
\end{pgfscope}%
\begin{pgfscope}%
\pgfpathrectangle{\pgfqpoint{0.100000in}{0.100000in}}{\pgfqpoint{3.007045in}{1.925000in}}%
\pgfusepath{clip}%
\pgfsetbuttcap%
\pgfsetmiterjoin%
\definecolor{currentfill}{rgb}{0.485490,0.718524,0.853426}%
\pgfsetfillcolor{currentfill}%
\pgfsetlinewidth{0.000000pt}%
\definecolor{currentstroke}{rgb}{0.000000,0.000000,0.000000}%
\pgfsetstrokecolor{currentstroke}%
\pgfsetstrokeopacity{0.000000}%
\pgfsetdash{}{0pt}%
\pgfpathmoveto{\pgfqpoint{1.219256in}{0.700832in}}%
\pgfpathlineto{\pgfqpoint{1.215312in}{0.701261in}}%
\pgfpathlineto{\pgfqpoint{1.186737in}{0.704479in}}%
\pgfpathlineto{\pgfqpoint{1.133803in}{0.710900in}}%
\pgfpathlineto{\pgfqpoint{1.141737in}{0.773832in}}%
\pgfpathlineto{\pgfqpoint{1.144512in}{0.779815in}}%
\pgfpathlineto{\pgfqpoint{1.147002in}{0.802056in}}%
\pgfpathlineto{\pgfqpoint{1.146247in}{0.808107in}}%
\pgfpathlineto{\pgfqpoint{1.147690in}{0.819949in}}%
\pgfpathlineto{\pgfqpoint{1.165157in}{0.817109in}}%
\pgfpathlineto{\pgfqpoint{1.166565in}{0.828636in}}%
\pgfpathlineto{\pgfqpoint{1.173550in}{0.827766in}}%
\pgfpathlineto{\pgfqpoint{1.177035in}{0.856282in}}%
\pgfpathlineto{\pgfqpoint{1.210171in}{0.852394in}}%
\pgfpathlineto{\pgfqpoint{1.210832in}{0.858083in}}%
\pgfpathlineto{\pgfqpoint{1.233670in}{0.855551in}}%
\pgfpathlineto{\pgfqpoint{1.231742in}{0.838561in}}%
\pgfpathlineto{\pgfqpoint{1.226820in}{0.793144in}}%
\pgfpathlineto{\pgfqpoint{1.224219in}{0.776176in}}%
\pgfpathlineto{\pgfqpoint{1.201518in}{0.778286in}}%
\pgfpathlineto{\pgfqpoint{1.200282in}{0.767082in}}%
\pgfpathlineto{\pgfqpoint{1.198272in}{0.767309in}}%
\pgfpathlineto{\pgfqpoint{1.194858in}{0.738140in}}%
\pgfpathlineto{\pgfqpoint{1.222778in}{0.735139in}}%
\pgfpathlineto{\pgfqpoint{1.219256in}{0.700832in}}%
\pgfpathclose%
\pgfusepath{fill}%
\end{pgfscope}%
\begin{pgfscope}%
\pgfpathrectangle{\pgfqpoint{0.100000in}{0.100000in}}{\pgfqpoint{3.007045in}{1.925000in}}%
\pgfusepath{clip}%
\pgfsetbuttcap%
\pgfsetmiterjoin%
\definecolor{currentfill}{rgb}{0.422745,0.684075,0.839892}%
\pgfsetfillcolor{currentfill}%
\pgfsetlinewidth{0.000000pt}%
\definecolor{currentstroke}{rgb}{0.000000,0.000000,0.000000}%
\pgfsetstrokecolor{currentstroke}%
\pgfsetstrokeopacity{0.000000}%
\pgfsetdash{}{0pt}%
\pgfpathmoveto{\pgfqpoint{2.394856in}{0.702917in}}%
\pgfpathlineto{\pgfqpoint{2.356903in}{0.698990in}}%
\pgfpathlineto{\pgfqpoint{2.339099in}{0.697275in}}%
\pgfpathlineto{\pgfqpoint{2.331743in}{0.699866in}}%
\pgfpathlineto{\pgfqpoint{2.330327in}{0.705545in}}%
\pgfpathlineto{\pgfqpoint{2.329526in}{0.710196in}}%
\pgfpathlineto{\pgfqpoint{2.336528in}{0.717430in}}%
\pgfpathlineto{\pgfqpoint{2.351237in}{0.719223in}}%
\pgfpathlineto{\pgfqpoint{2.349352in}{0.736500in}}%
\pgfpathlineto{\pgfqpoint{2.350162in}{0.739256in}}%
\pgfpathlineto{\pgfqpoint{2.360665in}{0.742287in}}%
\pgfpathlineto{\pgfqpoint{2.363559in}{0.734763in}}%
\pgfpathlineto{\pgfqpoint{2.368826in}{0.730586in}}%
\pgfpathlineto{\pgfqpoint{2.375605in}{0.733287in}}%
\pgfpathlineto{\pgfqpoint{2.378977in}{0.740714in}}%
\pgfpathlineto{\pgfqpoint{2.382677in}{0.742228in}}%
\pgfpathlineto{\pgfqpoint{2.392692in}{0.741268in}}%
\pgfpathlineto{\pgfqpoint{2.395370in}{0.738234in}}%
\pgfpathlineto{\pgfqpoint{2.395922in}{0.728401in}}%
\pgfpathlineto{\pgfqpoint{2.389930in}{0.724977in}}%
\pgfpathlineto{\pgfqpoint{2.386745in}{0.718055in}}%
\pgfpathlineto{\pgfqpoint{2.389600in}{0.714570in}}%
\pgfpathlineto{\pgfqpoint{2.394856in}{0.702917in}}%
\pgfpathclose%
\pgfusepath{fill}%
\end{pgfscope}%
\begin{pgfscope}%
\pgfpathrectangle{\pgfqpoint{0.100000in}{0.100000in}}{\pgfqpoint{3.007045in}{1.925000in}}%
\pgfusepath{clip}%
\pgfsetbuttcap%
\pgfsetmiterjoin%
\definecolor{currentfill}{rgb}{0.784591,0.864237,0.939962}%
\pgfsetfillcolor{currentfill}%
\pgfsetlinewidth{0.000000pt}%
\definecolor{currentstroke}{rgb}{0.000000,0.000000,0.000000}%
\pgfsetstrokecolor{currentstroke}%
\pgfsetstrokeopacity{0.000000}%
\pgfsetdash{}{0pt}%
\pgfpathmoveto{\pgfqpoint{2.313368in}{1.507664in}}%
\pgfpathlineto{\pgfqpoint{2.298344in}{1.505689in}}%
\pgfpathlineto{\pgfqpoint{2.291082in}{1.562710in}}%
\pgfpathlineto{\pgfqpoint{2.290081in}{1.574045in}}%
\pgfpathlineto{\pgfqpoint{2.273175in}{1.571785in}}%
\pgfpathlineto{\pgfqpoint{2.269589in}{1.599880in}}%
\pgfpathlineto{\pgfqpoint{2.271744in}{1.600141in}}%
\pgfpathlineto{\pgfqpoint{2.277048in}{1.592666in}}%
\pgfpathlineto{\pgfqpoint{2.285970in}{1.592606in}}%
\pgfpathlineto{\pgfqpoint{2.292882in}{1.588599in}}%
\pgfpathlineto{\pgfqpoint{2.308293in}{1.585034in}}%
\pgfpathlineto{\pgfqpoint{2.311509in}{1.579156in}}%
\pgfpathlineto{\pgfqpoint{2.316771in}{1.573334in}}%
\pgfpathlineto{\pgfqpoint{2.320179in}{1.565731in}}%
\pgfpathlineto{\pgfqpoint{2.311941in}{1.567027in}}%
\pgfpathlineto{\pgfqpoint{2.313141in}{1.558957in}}%
\pgfpathlineto{\pgfqpoint{2.319127in}{1.556401in}}%
\pgfpathlineto{\pgfqpoint{2.322517in}{1.545363in}}%
\pgfpathlineto{\pgfqpoint{2.321374in}{1.538390in}}%
\pgfpathlineto{\pgfqpoint{2.322792in}{1.520480in}}%
\pgfpathlineto{\pgfqpoint{2.318985in}{1.515613in}}%
\pgfpathlineto{\pgfqpoint{2.314903in}{1.514937in}}%
\pgfpathlineto{\pgfqpoint{2.313368in}{1.507664in}}%
\pgfpathclose%
\pgfusepath{fill}%
\end{pgfscope}%
\begin{pgfscope}%
\pgfpathrectangle{\pgfqpoint{0.100000in}{0.100000in}}{\pgfqpoint{3.007045in}{1.925000in}}%
\pgfusepath{clip}%
\pgfsetbuttcap%
\pgfsetmiterjoin%
\definecolor{currentfill}{rgb}{0.326290,0.618624,0.802799}%
\pgfsetfillcolor{currentfill}%
\pgfsetlinewidth{0.000000pt}%
\definecolor{currentstroke}{rgb}{0.000000,0.000000,0.000000}%
\pgfsetstrokecolor{currentstroke}%
\pgfsetstrokeopacity{0.000000}%
\pgfsetdash{}{0pt}%
\pgfpathmoveto{\pgfqpoint{2.327732in}{1.268420in}}%
\pgfpathlineto{\pgfqpoint{2.322114in}{1.267499in}}%
\pgfpathlineto{\pgfqpoint{2.322524in}{1.264076in}}%
\pgfpathlineto{\pgfqpoint{2.305349in}{1.261913in}}%
\pgfpathlineto{\pgfqpoint{2.306160in}{1.255224in}}%
\pgfpathlineto{\pgfqpoint{2.301562in}{1.253044in}}%
\pgfpathlineto{\pgfqpoint{2.283146in}{1.250837in}}%
\pgfpathlineto{\pgfqpoint{2.278491in}{1.292509in}}%
\pgfpathlineto{\pgfqpoint{2.301283in}{1.295237in}}%
\pgfpathlineto{\pgfqpoint{2.299961in}{1.306677in}}%
\pgfpathlineto{\pgfqpoint{2.305531in}{1.307364in}}%
\pgfpathlineto{\pgfqpoint{2.322615in}{1.309553in}}%
\pgfpathlineto{\pgfqpoint{2.327732in}{1.268420in}}%
\pgfpathclose%
\pgfusepath{fill}%
\end{pgfscope}%
\begin{pgfscope}%
\pgfpathrectangle{\pgfqpoint{0.100000in}{0.100000in}}{\pgfqpoint{3.007045in}{1.925000in}}%
\pgfusepath{clip}%
\pgfsetbuttcap%
\pgfsetmiterjoin%
\definecolor{currentfill}{rgb}{0.585882,0.773641,0.875079}%
\pgfsetfillcolor{currentfill}%
\pgfsetlinewidth{0.000000pt}%
\definecolor{currentstroke}{rgb}{0.000000,0.000000,0.000000}%
\pgfsetstrokecolor{currentstroke}%
\pgfsetstrokeopacity{0.000000}%
\pgfsetdash{}{0pt}%
\pgfpathmoveto{\pgfqpoint{2.625680in}{1.382482in}}%
\pgfpathlineto{\pgfqpoint{2.626550in}{1.378374in}}%
\pgfpathlineto{\pgfqpoint{2.607627in}{1.374496in}}%
\pgfpathlineto{\pgfqpoint{2.604647in}{1.373975in}}%
\pgfpathlineto{\pgfqpoint{2.595492in}{1.381828in}}%
\pgfpathlineto{\pgfqpoint{2.581784in}{1.380075in}}%
\pgfpathlineto{\pgfqpoint{2.558934in}{1.375533in}}%
\pgfpathlineto{\pgfqpoint{2.556898in}{1.389007in}}%
\pgfpathlineto{\pgfqpoint{2.558369in}{1.389969in}}%
\pgfpathlineto{\pgfqpoint{2.556395in}{1.400149in}}%
\pgfpathlineto{\pgfqpoint{2.549510in}{1.399000in}}%
\pgfpathlineto{\pgfqpoint{2.543334in}{1.433990in}}%
\pgfpathlineto{\pgfqpoint{2.546681in}{1.434064in}}%
\pgfpathlineto{\pgfqpoint{2.552432in}{1.429022in}}%
\pgfpathlineto{\pgfqpoint{2.557784in}{1.431044in}}%
\pgfpathlineto{\pgfqpoint{2.570755in}{1.438725in}}%
\pgfpathlineto{\pgfqpoint{2.572079in}{1.437927in}}%
\pgfpathlineto{\pgfqpoint{2.601922in}{1.443430in}}%
\pgfpathlineto{\pgfqpoint{2.608126in}{1.441492in}}%
\pgfpathlineto{\pgfqpoint{2.612717in}{1.410639in}}%
\pgfpathlineto{\pgfqpoint{2.619453in}{1.411977in}}%
\pgfpathlineto{\pgfqpoint{2.625680in}{1.382482in}}%
\pgfpathclose%
\pgfusepath{fill}%
\end{pgfscope}%
\begin{pgfscope}%
\pgfpathrectangle{\pgfqpoint{0.100000in}{0.100000in}}{\pgfqpoint{3.007045in}{1.925000in}}%
\pgfusepath{clip}%
\pgfsetbuttcap%
\pgfsetmiterjoin%
\definecolor{currentfill}{rgb}{0.592157,0.777086,0.876432}%
\pgfsetfillcolor{currentfill}%
\pgfsetlinewidth{0.000000pt}%
\definecolor{currentstroke}{rgb}{0.000000,0.000000,0.000000}%
\pgfsetstrokecolor{currentstroke}%
\pgfsetstrokeopacity{0.000000}%
\pgfsetdash{}{0pt}%
\pgfpathmoveto{\pgfqpoint{2.252299in}{1.407809in}}%
\pgfpathlineto{\pgfqpoint{2.263712in}{1.409069in}}%
\pgfpathlineto{\pgfqpoint{2.261163in}{1.431916in}}%
\pgfpathlineto{\pgfqpoint{2.283625in}{1.434436in}}%
\pgfpathlineto{\pgfqpoint{2.286455in}{1.412044in}}%
\pgfpathlineto{\pgfqpoint{2.296340in}{1.413257in}}%
\pgfpathlineto{\pgfqpoint{2.299946in}{1.390294in}}%
\pgfpathlineto{\pgfqpoint{2.272022in}{1.386856in}}%
\pgfpathlineto{\pgfqpoint{2.254932in}{1.384910in}}%
\pgfpathlineto{\pgfqpoint{2.252299in}{1.407809in}}%
\pgfpathclose%
\pgfusepath{fill}%
\end{pgfscope}%
\begin{pgfscope}%
\pgfpathrectangle{\pgfqpoint{0.100000in}{0.100000in}}{\pgfqpoint{3.007045in}{1.925000in}}%
\pgfusepath{clip}%
\pgfsetbuttcap%
\pgfsetmiterjoin%
\definecolor{currentfill}{rgb}{0.290980,0.594510,0.789020}%
\pgfsetfillcolor{currentfill}%
\pgfsetlinewidth{0.000000pt}%
\definecolor{currentstroke}{rgb}{0.000000,0.000000,0.000000}%
\pgfsetstrokecolor{currentstroke}%
\pgfsetstrokeopacity{0.000000}%
\pgfsetdash{}{0pt}%
\pgfpathmoveto{\pgfqpoint{1.677065in}{1.742344in}}%
\pgfpathlineto{\pgfqpoint{1.678294in}{1.748130in}}%
\pgfpathlineto{\pgfqpoint{1.676638in}{1.755236in}}%
\pgfpathlineto{\pgfqpoint{1.677958in}{1.760345in}}%
\pgfpathlineto{\pgfqpoint{1.676648in}{1.765752in}}%
\pgfpathlineto{\pgfqpoint{1.710937in}{1.765107in}}%
\pgfpathlineto{\pgfqpoint{1.745680in}{1.764366in}}%
\pgfpathlineto{\pgfqpoint{1.745890in}{1.729762in}}%
\pgfpathlineto{\pgfqpoint{1.746314in}{1.723984in}}%
\pgfpathlineto{\pgfqpoint{1.740624in}{1.724147in}}%
\pgfpathlineto{\pgfqpoint{1.740597in}{1.718328in}}%
\pgfpathlineto{\pgfqpoint{1.734882in}{1.718354in}}%
\pgfpathlineto{\pgfqpoint{1.734818in}{1.712519in}}%
\pgfpathlineto{\pgfqpoint{1.711752in}{1.712779in}}%
\pgfpathlineto{\pgfqpoint{1.711793in}{1.718564in}}%
\pgfpathlineto{\pgfqpoint{1.706015in}{1.718669in}}%
\pgfpathlineto{\pgfqpoint{1.705532in}{1.740473in}}%
\pgfpathlineto{\pgfqpoint{1.676789in}{1.740990in}}%
\pgfpathlineto{\pgfqpoint{1.677065in}{1.742344in}}%
\pgfpathclose%
\pgfusepath{fill}%
\end{pgfscope}%
\begin{pgfscope}%
\pgfpathrectangle{\pgfqpoint{0.100000in}{0.100000in}}{\pgfqpoint{3.007045in}{1.925000in}}%
\pgfusepath{clip}%
\pgfsetbuttcap%
\pgfsetmiterjoin%
\definecolor{currentfill}{rgb}{0.460392,0.704744,0.848012}%
\pgfsetfillcolor{currentfill}%
\pgfsetlinewidth{0.000000pt}%
\definecolor{currentstroke}{rgb}{0.000000,0.000000,0.000000}%
\pgfsetstrokecolor{currentstroke}%
\pgfsetstrokeopacity{0.000000}%
\pgfsetdash{}{0pt}%
\pgfpathmoveto{\pgfqpoint{2.667563in}{1.268330in}}%
\pgfpathlineto{\pgfqpoint{2.655502in}{1.265959in}}%
\pgfpathlineto{\pgfqpoint{2.652060in}{1.280358in}}%
\pgfpathlineto{\pgfqpoint{2.638827in}{1.299653in}}%
\pgfpathlineto{\pgfqpoint{2.636603in}{1.298784in}}%
\pgfpathlineto{\pgfqpoint{2.632407in}{1.305014in}}%
\pgfpathlineto{\pgfqpoint{2.627758in}{1.303889in}}%
\pgfpathlineto{\pgfqpoint{2.625195in}{1.307053in}}%
\pgfpathlineto{\pgfqpoint{2.628110in}{1.312979in}}%
\pgfpathlineto{\pgfqpoint{2.626236in}{1.316064in}}%
\pgfpathlineto{\pgfqpoint{2.633083in}{1.330389in}}%
\pgfpathlineto{\pgfqpoint{2.645950in}{1.339764in}}%
\pgfpathlineto{\pgfqpoint{2.648248in}{1.330436in}}%
\pgfpathlineto{\pgfqpoint{2.664036in}{1.332084in}}%
\pgfpathlineto{\pgfqpoint{2.670437in}{1.328641in}}%
\pgfpathlineto{\pgfqpoint{2.680799in}{1.334029in}}%
\pgfpathlineto{\pgfqpoint{2.690299in}{1.329057in}}%
\pgfpathlineto{\pgfqpoint{2.684366in}{1.322430in}}%
\pgfpathlineto{\pgfqpoint{2.693385in}{1.305552in}}%
\pgfpathlineto{\pgfqpoint{2.687398in}{1.300784in}}%
\pgfpathlineto{\pgfqpoint{2.690460in}{1.296019in}}%
\pgfpathlineto{\pgfqpoint{2.697015in}{1.296543in}}%
\pgfpathlineto{\pgfqpoint{2.701481in}{1.290448in}}%
\pgfpathlineto{\pgfqpoint{2.706007in}{1.289043in}}%
\pgfpathlineto{\pgfqpoint{2.716037in}{1.278304in}}%
\pgfpathlineto{\pgfqpoint{2.667563in}{1.268330in}}%
\pgfpathclose%
\pgfusepath{fill}%
\end{pgfscope}%
\begin{pgfscope}%
\pgfpathrectangle{\pgfqpoint{0.100000in}{0.100000in}}{\pgfqpoint{3.007045in}{1.925000in}}%
\pgfusepath{clip}%
\pgfsetbuttcap%
\pgfsetmiterjoin%
\definecolor{currentfill}{rgb}{0.381776,0.656517,0.824452}%
\pgfsetfillcolor{currentfill}%
\pgfsetlinewidth{0.000000pt}%
\definecolor{currentstroke}{rgb}{0.000000,0.000000,0.000000}%
\pgfsetstrokecolor{currentstroke}%
\pgfsetstrokeopacity{0.000000}%
\pgfsetdash{}{0pt}%
\pgfpathmoveto{\pgfqpoint{1.395381in}{1.309582in}}%
\pgfpathlineto{\pgfqpoint{1.396484in}{1.306583in}}%
\pgfpathlineto{\pgfqpoint{1.395479in}{1.295197in}}%
\pgfpathlineto{\pgfqpoint{1.394532in}{1.283801in}}%
\pgfpathlineto{\pgfqpoint{1.392134in}{1.280968in}}%
\pgfpathlineto{\pgfqpoint{1.346503in}{1.285013in}}%
\pgfpathlineto{\pgfqpoint{1.294492in}{1.290219in}}%
\pgfpathlineto{\pgfqpoint{1.262254in}{1.293547in}}%
\pgfpathlineto{\pgfqpoint{1.267098in}{1.336682in}}%
\pgfpathlineto{\pgfqpoint{1.297671in}{1.333074in}}%
\pgfpathlineto{\pgfqpoint{1.297072in}{1.327304in}}%
\pgfpathlineto{\pgfqpoint{1.326529in}{1.324213in}}%
\pgfpathlineto{\pgfqpoint{1.325420in}{1.312996in}}%
\pgfpathlineto{\pgfqpoint{1.358996in}{1.309848in}}%
\pgfpathlineto{\pgfqpoint{1.359322in}{1.312674in}}%
\pgfpathlineto{\pgfqpoint{1.395381in}{1.309582in}}%
\pgfpathclose%
\pgfusepath{fill}%
\end{pgfscope}%
\begin{pgfscope}%
\pgfpathrectangle{\pgfqpoint{0.100000in}{0.100000in}}{\pgfqpoint{3.007045in}{1.925000in}}%
\pgfusepath{clip}%
\pgfsetbuttcap%
\pgfsetmiterjoin%
\definecolor{currentfill}{rgb}{0.849550,0.907543,0.961615}%
\pgfsetfillcolor{currentfill}%
\pgfsetlinewidth{0.000000pt}%
\definecolor{currentstroke}{rgb}{0.000000,0.000000,0.000000}%
\pgfsetstrokecolor{currentstroke}%
\pgfsetstrokeopacity{0.000000}%
\pgfsetdash{}{0pt}%
\pgfpathmoveto{\pgfqpoint{1.201403in}{1.171929in}}%
\pgfpathlineto{\pgfqpoint{1.200990in}{1.169461in}}%
\pgfpathlineto{\pgfqpoint{1.208767in}{1.163511in}}%
\pgfpathlineto{\pgfqpoint{1.213448in}{1.158253in}}%
\pgfpathlineto{\pgfqpoint{1.213877in}{1.153875in}}%
\pgfpathlineto{\pgfqpoint{1.209760in}{1.146848in}}%
\pgfpathlineto{\pgfqpoint{1.213461in}{1.141563in}}%
\pgfpathlineto{\pgfqpoint{1.213054in}{1.137702in}}%
\pgfpathlineto{\pgfqpoint{1.205700in}{1.130953in}}%
\pgfpathlineto{\pgfqpoint{1.202180in}{1.129872in}}%
\pgfpathlineto{\pgfqpoint{1.193414in}{1.130886in}}%
\pgfpathlineto{\pgfqpoint{1.189161in}{1.136735in}}%
\pgfpathlineto{\pgfqpoint{1.187185in}{1.151775in}}%
\pgfpathlineto{\pgfqpoint{1.188060in}{1.158199in}}%
\pgfpathlineto{\pgfqpoint{1.193507in}{1.162201in}}%
\pgfpathlineto{\pgfqpoint{1.186288in}{1.164150in}}%
\pgfpathlineto{\pgfqpoint{1.180201in}{1.170746in}}%
\pgfpathlineto{\pgfqpoint{1.181782in}{1.174522in}}%
\pgfpathlineto{\pgfqpoint{1.201403in}{1.171929in}}%
\pgfpathclose%
\pgfusepath{fill}%
\end{pgfscope}%
\begin{pgfscope}%
\pgfpathrectangle{\pgfqpoint{0.100000in}{0.100000in}}{\pgfqpoint{3.007045in}{1.925000in}}%
\pgfusepath{clip}%
\pgfsetbuttcap%
\pgfsetmiterjoin%
\definecolor{currentfill}{rgb}{0.516863,0.735748,0.860192}%
\pgfsetfillcolor{currentfill}%
\pgfsetlinewidth{0.000000pt}%
\definecolor{currentstroke}{rgb}{0.000000,0.000000,0.000000}%
\pgfsetstrokecolor{currentstroke}%
\pgfsetstrokeopacity{0.000000}%
\pgfsetdash{}{0pt}%
\pgfpathmoveto{\pgfqpoint{2.114171in}{0.933474in}}%
\pgfpathlineto{\pgfqpoint{2.108828in}{0.933214in}}%
\pgfpathlineto{\pgfqpoint{2.108338in}{0.951144in}}%
\pgfpathlineto{\pgfqpoint{2.110066in}{0.954874in}}%
\pgfpathlineto{\pgfqpoint{2.116626in}{0.955211in}}%
\pgfpathlineto{\pgfqpoint{2.115651in}{0.980661in}}%
\pgfpathlineto{\pgfqpoint{2.117091in}{0.980740in}}%
\pgfpathlineto{\pgfqpoint{2.140155in}{0.982191in}}%
\pgfpathlineto{\pgfqpoint{2.140247in}{0.978276in}}%
\pgfpathlineto{\pgfqpoint{2.146159in}{0.970611in}}%
\pgfpathlineto{\pgfqpoint{2.146928in}{0.965917in}}%
\pgfpathlineto{\pgfqpoint{2.158552in}{0.966327in}}%
\pgfpathlineto{\pgfqpoint{2.166450in}{0.964466in}}%
\pgfpathlineto{\pgfqpoint{2.169480in}{0.968102in}}%
\pgfpathlineto{\pgfqpoint{2.169549in}{0.973787in}}%
\pgfpathlineto{\pgfqpoint{2.181638in}{0.973954in}}%
\pgfpathlineto{\pgfqpoint{2.188596in}{0.973610in}}%
\pgfpathlineto{\pgfqpoint{2.187491in}{0.967104in}}%
\pgfpathlineto{\pgfqpoint{2.188978in}{0.963509in}}%
\pgfpathlineto{\pgfqpoint{2.188043in}{0.950490in}}%
\pgfpathlineto{\pgfqpoint{2.170276in}{0.951262in}}%
\pgfpathlineto{\pgfqpoint{2.161575in}{0.944740in}}%
\pgfpathlineto{\pgfqpoint{2.161409in}{0.940303in}}%
\pgfpathlineto{\pgfqpoint{2.151093in}{0.939458in}}%
\pgfpathlineto{\pgfqpoint{2.144945in}{0.937174in}}%
\pgfpathlineto{\pgfqpoint{2.143491in}{0.939640in}}%
\pgfpathlineto{\pgfqpoint{2.136863in}{0.936941in}}%
\pgfpathlineto{\pgfqpoint{2.116000in}{0.935950in}}%
\pgfpathlineto{\pgfqpoint{2.114171in}{0.933474in}}%
\pgfpathclose%
\pgfusepath{fill}%
\end{pgfscope}%
\begin{pgfscope}%
\pgfpathrectangle{\pgfqpoint{0.100000in}{0.100000in}}{\pgfqpoint{3.007045in}{1.925000in}}%
\pgfusepath{clip}%
\pgfsetbuttcap%
\pgfsetmiterjoin%
\definecolor{currentfill}{rgb}{0.417086,0.680631,0.838231}%
\pgfsetfillcolor{currentfill}%
\pgfsetlinewidth{0.000000pt}%
\definecolor{currentstroke}{rgb}{0.000000,0.000000,0.000000}%
\pgfsetstrokecolor{currentstroke}%
\pgfsetstrokeopacity{0.000000}%
\pgfsetdash{}{0pt}%
\pgfpathmoveto{\pgfqpoint{0.920301in}{0.337334in}}%
\pgfpathlineto{\pgfqpoint{0.912069in}{0.341923in}}%
\pgfpathlineto{\pgfqpoint{0.914753in}{0.346694in}}%
\pgfpathlineto{\pgfqpoint{0.913691in}{0.348399in}}%
\pgfpathlineto{\pgfqpoint{0.917983in}{0.355693in}}%
\pgfpathlineto{\pgfqpoint{0.916310in}{0.356673in}}%
\pgfpathlineto{\pgfqpoint{0.917368in}{0.358439in}}%
\pgfpathlineto{\pgfqpoint{0.915470in}{0.359574in}}%
\pgfpathlineto{\pgfqpoint{0.916564in}{0.361368in}}%
\pgfpathlineto{\pgfqpoint{0.913444in}{0.363209in}}%
\pgfpathlineto{\pgfqpoint{0.912886in}{0.364295in}}%
\pgfpathlineto{\pgfqpoint{0.910185in}{0.362649in}}%
\pgfpathlineto{\pgfqpoint{0.907829in}{0.364423in}}%
\pgfpathlineto{\pgfqpoint{0.910203in}{0.367541in}}%
\pgfpathlineto{\pgfqpoint{0.909515in}{0.369615in}}%
\pgfpathlineto{\pgfqpoint{0.912562in}{0.370845in}}%
\pgfpathlineto{\pgfqpoint{0.914264in}{0.375762in}}%
\pgfpathlineto{\pgfqpoint{0.899657in}{0.384205in}}%
\pgfpathlineto{\pgfqpoint{0.894498in}{0.387400in}}%
\pgfpathlineto{\pgfqpoint{0.895631in}{0.389258in}}%
\pgfpathlineto{\pgfqpoint{0.886512in}{0.394965in}}%
\pgfpathlineto{\pgfqpoint{0.884245in}{0.391268in}}%
\pgfpathlineto{\pgfqpoint{0.880633in}{0.393554in}}%
\pgfpathlineto{\pgfqpoint{0.886359in}{0.402751in}}%
\pgfpathlineto{\pgfqpoint{0.875033in}{0.409830in}}%
\pgfpathlineto{\pgfqpoint{0.871795in}{0.412013in}}%
\pgfpathlineto{\pgfqpoint{0.869886in}{0.409139in}}%
\pgfpathlineto{\pgfqpoint{0.856121in}{0.418510in}}%
\pgfpathlineto{\pgfqpoint{0.847814in}{0.419834in}}%
\pgfpathlineto{\pgfqpoint{0.827109in}{0.423257in}}%
\pgfpathlineto{\pgfqpoint{0.826027in}{0.421875in}}%
\pgfpathlineto{\pgfqpoint{0.816153in}{0.429462in}}%
\pgfpathlineto{\pgfqpoint{0.809857in}{0.421160in}}%
\pgfpathlineto{\pgfqpoint{0.807974in}{0.419972in}}%
\pgfpathlineto{\pgfqpoint{0.805001in}{0.422296in}}%
\pgfpathlineto{\pgfqpoint{0.803696in}{0.420629in}}%
\pgfpathlineto{\pgfqpoint{0.800378in}{0.423260in}}%
\pgfpathlineto{\pgfqpoint{0.798438in}{0.420763in}}%
\pgfpathlineto{\pgfqpoint{0.796550in}{0.422002in}}%
\pgfpathlineto{\pgfqpoint{0.787335in}{0.429069in}}%
\pgfpathlineto{\pgfqpoint{0.776021in}{0.437944in}}%
\pgfpathlineto{\pgfqpoint{0.768555in}{0.444177in}}%
\pgfpathlineto{\pgfqpoint{0.770089in}{0.446056in}}%
\pgfpathlineto{\pgfqpoint{0.757361in}{0.456738in}}%
\pgfpathlineto{\pgfqpoint{0.755991in}{0.455118in}}%
\pgfpathlineto{\pgfqpoint{0.749271in}{0.460813in}}%
\pgfpathlineto{\pgfqpoint{0.747517in}{0.458771in}}%
\pgfpathlineto{\pgfqpoint{0.741635in}{0.463832in}}%
\pgfpathlineto{\pgfqpoint{0.733232in}{0.471329in}}%
\pgfpathlineto{\pgfqpoint{0.737384in}{0.476047in}}%
\pgfpathlineto{\pgfqpoint{0.736206in}{0.477104in}}%
\pgfpathlineto{\pgfqpoint{0.741905in}{0.483472in}}%
\pgfpathlineto{\pgfqpoint{0.742309in}{0.483156in}}%
\pgfpathlineto{\pgfqpoint{0.748071in}{0.489521in}}%
\pgfpathlineto{\pgfqpoint{0.750170in}{0.491012in}}%
\pgfpathlineto{\pgfqpoint{0.751165in}{0.490590in}}%
\pgfpathlineto{\pgfqpoint{0.755431in}{0.495345in}}%
\pgfpathlineto{\pgfqpoint{0.756937in}{0.494067in}}%
\pgfpathlineto{\pgfqpoint{0.761153in}{0.499198in}}%
\pgfpathlineto{\pgfqpoint{0.762810in}{0.497757in}}%
\pgfpathlineto{\pgfqpoint{0.764086in}{0.499185in}}%
\pgfpathlineto{\pgfqpoint{0.767409in}{0.496302in}}%
\pgfpathlineto{\pgfqpoint{0.768805in}{0.498004in}}%
\pgfpathlineto{\pgfqpoint{0.770286in}{0.496757in}}%
\pgfpathlineto{\pgfqpoint{0.771756in}{0.498478in}}%
\pgfpathlineto{\pgfqpoint{0.773225in}{0.497185in}}%
\pgfpathlineto{\pgfqpoint{0.777669in}{0.501945in}}%
\pgfpathlineto{\pgfqpoint{0.775774in}{0.503541in}}%
\pgfpathlineto{\pgfqpoint{0.787134in}{0.516986in}}%
\pgfpathlineto{\pgfqpoint{0.788832in}{0.515575in}}%
\pgfpathlineto{\pgfqpoint{0.791670in}{0.518914in}}%
\pgfpathlineto{\pgfqpoint{0.793312in}{0.517519in}}%
\pgfpathlineto{\pgfqpoint{0.798922in}{0.524065in}}%
\pgfpathlineto{\pgfqpoint{0.800473in}{0.522699in}}%
\pgfpathlineto{\pgfqpoint{0.804798in}{0.527778in}}%
\pgfpathlineto{\pgfqpoint{0.803129in}{0.529197in}}%
\pgfpathlineto{\pgfqpoint{0.810396in}{0.537663in}}%
\pgfpathlineto{\pgfqpoint{0.815410in}{0.533363in}}%
\pgfpathlineto{\pgfqpoint{0.818104in}{0.536633in}}%
\pgfpathlineto{\pgfqpoint{0.826641in}{0.529420in}}%
\pgfpathlineto{\pgfqpoint{0.832267in}{0.536448in}}%
\pgfpathlineto{\pgfqpoint{0.839219in}{0.530767in}}%
\pgfpathlineto{\pgfqpoint{0.836381in}{0.527215in}}%
\pgfpathlineto{\pgfqpoint{0.839868in}{0.524417in}}%
\pgfpathlineto{\pgfqpoint{0.842673in}{0.527910in}}%
\pgfpathlineto{\pgfqpoint{0.846127in}{0.525116in}}%
\pgfpathlineto{\pgfqpoint{0.844770in}{0.523398in}}%
\pgfpathlineto{\pgfqpoint{0.846505in}{0.521977in}}%
\pgfpathlineto{\pgfqpoint{0.845093in}{0.520174in}}%
\pgfpathlineto{\pgfqpoint{0.848387in}{0.517598in}}%
\pgfpathlineto{\pgfqpoint{0.852921in}{0.522581in}}%
\pgfpathlineto{\pgfqpoint{0.857919in}{0.518605in}}%
\pgfpathlineto{\pgfqpoint{0.860718in}{0.522144in}}%
\pgfpathlineto{\pgfqpoint{0.865978in}{0.517880in}}%
\pgfpathlineto{\pgfqpoint{0.867902in}{0.519376in}}%
\pgfpathlineto{\pgfqpoint{0.873845in}{0.526915in}}%
\pgfpathlineto{\pgfqpoint{0.872574in}{0.527836in}}%
\pgfpathlineto{\pgfqpoint{0.873947in}{0.529693in}}%
\pgfpathlineto{\pgfqpoint{0.870380in}{0.532422in}}%
\pgfpathlineto{\pgfqpoint{0.874516in}{0.537747in}}%
\pgfpathlineto{\pgfqpoint{0.876407in}{0.539190in}}%
\pgfpathlineto{\pgfqpoint{0.874603in}{0.540599in}}%
\pgfpathlineto{\pgfqpoint{0.876010in}{0.542324in}}%
\pgfpathlineto{\pgfqpoint{0.874170in}{0.543737in}}%
\pgfpathlineto{\pgfqpoint{0.875594in}{0.545533in}}%
\pgfpathlineto{\pgfqpoint{0.873853in}{0.546928in}}%
\pgfpathlineto{\pgfqpoint{0.877896in}{0.551221in}}%
\pgfpathlineto{\pgfqpoint{0.895892in}{0.537283in}}%
\pgfpathlineto{\pgfqpoint{0.919823in}{0.519872in}}%
\pgfpathlineto{\pgfqpoint{0.941118in}{0.505304in}}%
\pgfpathlineto{\pgfqpoint{0.944682in}{0.503522in}}%
\pgfpathlineto{\pgfqpoint{0.952820in}{0.498342in}}%
\pgfpathlineto{\pgfqpoint{0.959476in}{0.508253in}}%
\pgfpathlineto{\pgfqpoint{0.966890in}{0.503916in}}%
\pgfpathlineto{\pgfqpoint{0.989639in}{0.489920in}}%
\pgfpathlineto{\pgfqpoint{1.001047in}{0.482991in}}%
\pgfpathlineto{\pgfqpoint{0.992914in}{0.468320in}}%
\pgfpathlineto{\pgfqpoint{0.981844in}{0.448351in}}%
\pgfpathlineto{\pgfqpoint{0.970915in}{0.428636in}}%
\pgfpathlineto{\pgfqpoint{0.960932in}{0.410628in}}%
\pgfpathlineto{\pgfqpoint{0.950240in}{0.391342in}}%
\pgfpathlineto{\pgfqpoint{0.939656in}{0.372249in}}%
\pgfpathlineto{\pgfqpoint{0.929170in}{0.353333in}}%
\pgfpathlineto{\pgfqpoint{0.920301in}{0.337334in}}%
\pgfpathclose%
\pgfusepath{fill}%
\end{pgfscope}%
\begin{pgfscope}%
\pgfpathrectangle{\pgfqpoint{0.100000in}{0.100000in}}{\pgfqpoint{3.007045in}{1.925000in}}%
\pgfusepath{clip}%
\pgfsetbuttcap%
\pgfsetmiterjoin%
\definecolor{currentfill}{rgb}{0.396909,0.666851,0.830358}%
\pgfsetfillcolor{currentfill}%
\pgfsetlinewidth{0.000000pt}%
\definecolor{currentstroke}{rgb}{0.000000,0.000000,0.000000}%
\pgfsetstrokecolor{currentstroke}%
\pgfsetstrokeopacity{0.000000}%
\pgfsetdash{}{0pt}%
\pgfpathmoveto{\pgfqpoint{1.924728in}{0.990132in}}%
\pgfpathlineto{\pgfqpoint{1.890280in}{0.989939in}}%
\pgfpathlineto{\pgfqpoint{1.859810in}{0.990030in}}%
\pgfpathlineto{\pgfqpoint{1.859842in}{1.001580in}}%
\pgfpathlineto{\pgfqpoint{1.845487in}{1.001557in}}%
\pgfpathlineto{\pgfqpoint{1.845462in}{1.020387in}}%
\pgfpathlineto{\pgfqpoint{1.844374in}{1.020395in}}%
\pgfpathlineto{\pgfqpoint{1.844766in}{1.050707in}}%
\pgfpathlineto{\pgfqpoint{1.843696in}{1.056469in}}%
\pgfpathlineto{\pgfqpoint{1.846715in}{1.062147in}}%
\pgfpathlineto{\pgfqpoint{1.849656in}{1.062106in}}%
\pgfpathlineto{\pgfqpoint{1.849904in}{1.072618in}}%
\pgfpathlineto{\pgfqpoint{1.872670in}{1.072352in}}%
\pgfpathlineto{\pgfqpoint{1.872518in}{1.061815in}}%
\pgfpathlineto{\pgfqpoint{1.896036in}{1.061694in}}%
\pgfpathlineto{\pgfqpoint{1.900128in}{1.055674in}}%
\pgfpathlineto{\pgfqpoint{1.907064in}{1.059811in}}%
\pgfpathlineto{\pgfqpoint{1.907105in}{1.050239in}}%
\pgfpathlineto{\pgfqpoint{1.915821in}{1.046178in}}%
\pgfpathlineto{\pgfqpoint{1.915878in}{1.043286in}}%
\pgfpathlineto{\pgfqpoint{1.916864in}{1.007392in}}%
\pgfpathlineto{\pgfqpoint{1.925420in}{1.007434in}}%
\pgfpathlineto{\pgfqpoint{1.924728in}{0.990132in}}%
\pgfpathclose%
\pgfusepath{fill}%
\end{pgfscope}%
\begin{pgfscope}%
\pgfpathrectangle{\pgfqpoint{0.100000in}{0.100000in}}{\pgfqpoint{3.007045in}{1.925000in}}%
\pgfusepath{clip}%
\pgfsetbuttcap%
\pgfsetmiterjoin%
\definecolor{currentfill}{rgb}{0.275848,0.584175,0.783114}%
\pgfsetfillcolor{currentfill}%
\pgfsetlinewidth{0.000000pt}%
\definecolor{currentstroke}{rgb}{0.000000,0.000000,0.000000}%
\pgfsetstrokecolor{currentstroke}%
\pgfsetstrokeopacity{0.000000}%
\pgfsetdash{}{0pt}%
\pgfpathmoveto{\pgfqpoint{1.829586in}{0.787129in}}%
\pgfpathlineto{\pgfqpoint{1.825054in}{0.789133in}}%
\pgfpathlineto{\pgfqpoint{1.814189in}{0.790353in}}%
\pgfpathlineto{\pgfqpoint{1.806094in}{0.799451in}}%
\pgfpathlineto{\pgfqpoint{1.801001in}{0.799445in}}%
\pgfpathlineto{\pgfqpoint{1.800644in}{0.779398in}}%
\pgfpathlineto{\pgfqpoint{1.786299in}{0.783647in}}%
\pgfpathlineto{\pgfqpoint{1.779488in}{0.786480in}}%
\pgfpathlineto{\pgfqpoint{1.774881in}{0.791133in}}%
\pgfpathlineto{\pgfqpoint{1.760297in}{0.800742in}}%
\pgfpathlineto{\pgfqpoint{1.756132in}{0.794865in}}%
\pgfpathlineto{\pgfqpoint{1.749934in}{0.794479in}}%
\pgfpathlineto{\pgfqpoint{1.742020in}{0.795983in}}%
\pgfpathlineto{\pgfqpoint{1.740181in}{0.799040in}}%
\pgfpathlineto{\pgfqpoint{1.730486in}{0.795042in}}%
\pgfpathlineto{\pgfqpoint{1.726173in}{0.795162in}}%
\pgfpathlineto{\pgfqpoint{1.720879in}{0.800114in}}%
\pgfpathlineto{\pgfqpoint{1.720987in}{0.811838in}}%
\pgfpathlineto{\pgfqpoint{1.718089in}{0.813933in}}%
\pgfpathlineto{\pgfqpoint{1.729695in}{0.813812in}}%
\pgfpathlineto{\pgfqpoint{1.729891in}{0.836933in}}%
\pgfpathlineto{\pgfqpoint{1.735756in}{0.836902in}}%
\pgfpathlineto{\pgfqpoint{1.735784in}{0.842654in}}%
\pgfpathlineto{\pgfqpoint{1.744358in}{0.842618in}}%
\pgfpathlineto{\pgfqpoint{1.744379in}{0.848364in}}%
\pgfpathlineto{\pgfqpoint{1.775709in}{0.848268in}}%
\pgfpathlineto{\pgfqpoint{1.775725in}{0.836774in}}%
\pgfpathlineto{\pgfqpoint{1.801665in}{0.836897in}}%
\pgfpathlineto{\pgfqpoint{1.801936in}{0.851532in}}%
\pgfpathlineto{\pgfqpoint{1.809383in}{0.849499in}}%
\pgfpathlineto{\pgfqpoint{1.825651in}{0.849322in}}%
\pgfpathlineto{\pgfqpoint{1.830433in}{0.847601in}}%
\pgfpathlineto{\pgfqpoint{1.830412in}{0.826752in}}%
\pgfpathlineto{\pgfqpoint{1.812989in}{0.826934in}}%
\pgfpathlineto{\pgfqpoint{1.813675in}{0.816167in}}%
\pgfpathlineto{\pgfqpoint{1.821492in}{0.816167in}}%
\pgfpathlineto{\pgfqpoint{1.822337in}{0.805756in}}%
\pgfpathlineto{\pgfqpoint{1.824314in}{0.796016in}}%
\pgfpathlineto{\pgfqpoint{1.828440in}{0.791464in}}%
\pgfpathlineto{\pgfqpoint{1.829586in}{0.787129in}}%
\pgfpathclose%
\pgfusepath{fill}%
\end{pgfscope}%
\begin{pgfscope}%
\pgfpathrectangle{\pgfqpoint{0.100000in}{0.100000in}}{\pgfqpoint{3.007045in}{1.925000in}}%
\pgfusepath{clip}%
\pgfsetbuttcap%
\pgfsetmiterjoin%
\definecolor{currentfill}{rgb}{0.336378,0.625513,0.806736}%
\pgfsetfillcolor{currentfill}%
\pgfsetlinewidth{0.000000pt}%
\definecolor{currentstroke}{rgb}{0.000000,0.000000,0.000000}%
\pgfsetstrokecolor{currentstroke}%
\pgfsetstrokeopacity{0.000000}%
\pgfsetdash{}{0pt}%
\pgfpathmoveto{\pgfqpoint{1.798597in}{1.275256in}}%
\pgfpathlineto{\pgfqpoint{1.787185in}{1.275241in}}%
\pgfpathlineto{\pgfqpoint{1.787084in}{1.304515in}}%
\pgfpathlineto{\pgfqpoint{1.784942in}{1.304507in}}%
\pgfpathlineto{\pgfqpoint{1.784860in}{1.321668in}}%
\pgfpathlineto{\pgfqpoint{1.813376in}{1.321894in}}%
\pgfpathlineto{\pgfqpoint{1.813145in}{1.344757in}}%
\pgfpathlineto{\pgfqpoint{1.847419in}{1.345161in}}%
\pgfpathlineto{\pgfqpoint{1.858723in}{1.345342in}}%
\pgfpathlineto{\pgfqpoint{1.859156in}{1.322514in}}%
\pgfpathlineto{\pgfqpoint{1.853457in}{1.322433in}}%
\pgfpathlineto{\pgfqpoint{1.853741in}{1.305138in}}%
\pgfpathlineto{\pgfqpoint{1.854836in}{1.299029in}}%
\pgfpathlineto{\pgfqpoint{1.855298in}{1.276147in}}%
\pgfpathlineto{\pgfqpoint{1.843908in}{1.275998in}}%
\pgfpathlineto{\pgfqpoint{1.826979in}{1.275432in}}%
\pgfpathlineto{\pgfqpoint{1.798597in}{1.275256in}}%
\pgfpathclose%
\pgfusepath{fill}%
\end{pgfscope}%
\begin{pgfscope}%
\pgfpathrectangle{\pgfqpoint{0.100000in}{0.100000in}}{\pgfqpoint{3.007045in}{1.925000in}}%
\pgfusepath{clip}%
\pgfsetbuttcap%
\pgfsetmiterjoin%
\definecolor{currentfill}{rgb}{0.108651,0.416563,0.689043}%
\pgfsetfillcolor{currentfill}%
\pgfsetlinewidth{0.000000pt}%
\definecolor{currentstroke}{rgb}{0.000000,0.000000,0.000000}%
\pgfsetstrokecolor{currentstroke}%
\pgfsetstrokeopacity{0.000000}%
\pgfsetdash{}{0pt}%
\pgfpathmoveto{\pgfqpoint{1.502115in}{0.878788in}}%
\pgfpathlineto{\pgfqpoint{1.501084in}{0.860066in}}%
\pgfpathlineto{\pgfqpoint{1.478574in}{0.861428in}}%
\pgfpathlineto{\pgfqpoint{1.471813in}{0.861849in}}%
\pgfpathlineto{\pgfqpoint{1.473659in}{0.890543in}}%
\pgfpathlineto{\pgfqpoint{1.475346in}{0.919280in}}%
\pgfpathlineto{\pgfqpoint{1.504252in}{0.917589in}}%
\pgfpathlineto{\pgfqpoint{1.502115in}{0.878788in}}%
\pgfpathclose%
\pgfusepath{fill}%
\end{pgfscope}%
\begin{pgfscope}%
\pgfpathrectangle{\pgfqpoint{0.100000in}{0.100000in}}{\pgfqpoint{3.007045in}{1.925000in}}%
\pgfusepath{clip}%
\pgfsetbuttcap%
\pgfsetmiterjoin%
\definecolor{currentfill}{rgb}{0.191326,0.505052,0.741699}%
\pgfsetfillcolor{currentfill}%
\pgfsetlinewidth{0.000000pt}%
\definecolor{currentstroke}{rgb}{0.000000,0.000000,0.000000}%
\pgfsetstrokecolor{currentstroke}%
\pgfsetstrokeopacity{0.000000}%
\pgfsetdash{}{0pt}%
\pgfpathmoveto{\pgfqpoint{1.594488in}{0.908994in}}%
\pgfpathlineto{\pgfqpoint{1.577863in}{0.909634in}}%
\pgfpathlineto{\pgfqpoint{1.578070in}{0.926846in}}%
\pgfpathlineto{\pgfqpoint{1.538310in}{0.928586in}}%
\pgfpathlineto{\pgfqpoint{1.538682in}{0.941826in}}%
\pgfpathlineto{\pgfqpoint{1.539102in}{0.951802in}}%
\pgfpathlineto{\pgfqpoint{1.561768in}{0.950581in}}%
\pgfpathlineto{\pgfqpoint{1.562805in}{0.979407in}}%
\pgfpathlineto{\pgfqpoint{1.553847in}{0.991951in}}%
\pgfpathlineto{\pgfqpoint{1.547875in}{0.992381in}}%
\pgfpathlineto{\pgfqpoint{1.541765in}{0.998932in}}%
\pgfpathlineto{\pgfqpoint{1.537864in}{1.006990in}}%
\pgfpathlineto{\pgfqpoint{1.561807in}{1.005883in}}%
\pgfpathlineto{\pgfqpoint{1.596180in}{1.004385in}}%
\pgfpathlineto{\pgfqpoint{1.642737in}{1.002945in}}%
\pgfpathlineto{\pgfqpoint{1.642047in}{0.976241in}}%
\pgfpathlineto{\pgfqpoint{1.641329in}{0.947962in}}%
\pgfpathlineto{\pgfqpoint{1.601428in}{0.949236in}}%
\pgfpathlineto{\pgfqpoint{1.600584in}{0.920240in}}%
\pgfpathlineto{\pgfqpoint{1.594900in}{0.920448in}}%
\pgfpathlineto{\pgfqpoint{1.594488in}{0.908994in}}%
\pgfpathclose%
\pgfusepath{fill}%
\end{pgfscope}%
\begin{pgfscope}%
\pgfpathrectangle{\pgfqpoint{0.100000in}{0.100000in}}{\pgfqpoint{3.007045in}{1.925000in}}%
\pgfusepath{clip}%
\pgfsetbuttcap%
\pgfsetmiterjoin%
\definecolor{currentfill}{rgb}{0.031373,0.188235,0.419608}%
\pgfsetfillcolor{currentfill}%
\pgfsetlinewidth{0.000000pt}%
\definecolor{currentstroke}{rgb}{0.000000,0.000000,0.000000}%
\pgfsetstrokecolor{currentstroke}%
\pgfsetstrokeopacity{0.000000}%
\pgfsetdash{}{0pt}%
\pgfpathmoveto{\pgfqpoint{0.907070in}{1.217261in}}%
\pgfpathlineto{\pgfqpoint{0.903422in}{1.212075in}}%
\pgfpathlineto{\pgfqpoint{0.900045in}{1.212653in}}%
\pgfpathlineto{\pgfqpoint{0.894249in}{1.207621in}}%
\pgfpathlineto{\pgfqpoint{0.892855in}{1.198668in}}%
\pgfpathlineto{\pgfqpoint{0.887613in}{1.196226in}}%
\pgfpathlineto{\pgfqpoint{0.880713in}{1.197892in}}%
\pgfpathlineto{\pgfqpoint{0.875936in}{1.191519in}}%
\pgfpathlineto{\pgfqpoint{0.850288in}{1.196720in}}%
\pgfpathlineto{\pgfqpoint{0.798495in}{1.207353in}}%
\pgfpathlineto{\pgfqpoint{0.799942in}{1.214078in}}%
\pgfpathlineto{\pgfqpoint{0.811909in}{1.269809in}}%
\pgfpathlineto{\pgfqpoint{0.816917in}{1.293210in}}%
\pgfpathlineto{\pgfqpoint{0.878712in}{1.280320in}}%
\pgfpathlineto{\pgfqpoint{0.901338in}{1.275958in}}%
\pgfpathlineto{\pgfqpoint{0.910706in}{1.281361in}}%
\pgfpathlineto{\pgfqpoint{0.914434in}{1.291420in}}%
\pgfpathlineto{\pgfqpoint{0.913219in}{1.297816in}}%
\pgfpathlineto{\pgfqpoint{0.914647in}{1.310630in}}%
\pgfpathlineto{\pgfqpoint{0.923286in}{1.309346in}}%
\pgfpathlineto{\pgfqpoint{0.925436in}{1.305547in}}%
\pgfpathlineto{\pgfqpoint{0.937610in}{1.310762in}}%
\pgfpathlineto{\pgfqpoint{0.947451in}{1.310370in}}%
\pgfpathlineto{\pgfqpoint{0.943668in}{1.304520in}}%
\pgfpathlineto{\pgfqpoint{0.950156in}{1.292847in}}%
\pgfpathlineto{\pgfqpoint{0.957665in}{1.289984in}}%
\pgfpathlineto{\pgfqpoint{0.959849in}{1.282582in}}%
\pgfpathlineto{\pgfqpoint{0.957377in}{1.275606in}}%
\pgfpathlineto{\pgfqpoint{0.964301in}{1.266613in}}%
\pgfpathlineto{\pgfqpoint{0.963825in}{1.264056in}}%
\pgfpathlineto{\pgfqpoint{0.973337in}{1.262333in}}%
\pgfpathlineto{\pgfqpoint{0.974041in}{1.256437in}}%
\pgfpathlineto{\pgfqpoint{0.934985in}{1.263476in}}%
\pgfpathlineto{\pgfqpoint{0.937487in}{1.261054in}}%
\pgfpathlineto{\pgfqpoint{0.929754in}{1.256515in}}%
\pgfpathlineto{\pgfqpoint{0.927198in}{1.241278in}}%
\pgfpathlineto{\pgfqpoint{0.924573in}{1.236589in}}%
\pgfpathlineto{\pgfqpoint{0.915780in}{1.238145in}}%
\pgfpathlineto{\pgfqpoint{0.910572in}{1.236135in}}%
\pgfpathlineto{\pgfqpoint{0.907070in}{1.217261in}}%
\pgfpathclose%
\pgfusepath{fill}%
\end{pgfscope}%
\begin{pgfscope}%
\pgfpathrectangle{\pgfqpoint{0.100000in}{0.100000in}}{\pgfqpoint{3.007045in}{1.925000in}}%
\pgfusepath{clip}%
\pgfsetbuttcap%
\pgfsetmiterjoin%
\definecolor{currentfill}{rgb}{0.460392,0.704744,0.848012}%
\pgfsetfillcolor{currentfill}%
\pgfsetlinewidth{0.000000pt}%
\definecolor{currentstroke}{rgb}{0.000000,0.000000,0.000000}%
\pgfsetstrokecolor{currentstroke}%
\pgfsetstrokeopacity{0.000000}%
\pgfsetdash{}{0pt}%
\pgfpathmoveto{\pgfqpoint{2.375866in}{1.305055in}}%
\pgfpathlineto{\pgfqpoint{2.371168in}{1.333253in}}%
\pgfpathlineto{\pgfqpoint{2.365754in}{1.334376in}}%
\pgfpathlineto{\pgfqpoint{2.378309in}{1.338718in}}%
\pgfpathlineto{\pgfqpoint{2.383725in}{1.333912in}}%
\pgfpathlineto{\pgfqpoint{2.389914in}{1.332610in}}%
\pgfpathlineto{\pgfqpoint{2.400156in}{1.337527in}}%
\pgfpathlineto{\pgfqpoint{2.411268in}{1.344518in}}%
\pgfpathlineto{\pgfqpoint{2.413516in}{1.344212in}}%
\pgfpathlineto{\pgfqpoint{2.414804in}{1.334085in}}%
\pgfpathlineto{\pgfqpoint{2.419430in}{1.334702in}}%
\pgfpathlineto{\pgfqpoint{2.420098in}{1.329773in}}%
\pgfpathlineto{\pgfqpoint{2.415460in}{1.329120in}}%
\pgfpathlineto{\pgfqpoint{2.416075in}{1.324178in}}%
\pgfpathlineto{\pgfqpoint{2.411263in}{1.323487in}}%
\pgfpathlineto{\pgfqpoint{2.407064in}{1.318707in}}%
\pgfpathlineto{\pgfqpoint{2.407666in}{1.313899in}}%
\pgfpathlineto{\pgfqpoint{2.394589in}{1.312199in}}%
\pgfpathlineto{\pgfqpoint{2.395446in}{1.307499in}}%
\pgfpathlineto{\pgfqpoint{2.375866in}{1.305055in}}%
\pgfpathclose%
\pgfusepath{fill}%
\end{pgfscope}%
\begin{pgfscope}%
\pgfpathrectangle{\pgfqpoint{0.100000in}{0.100000in}}{\pgfqpoint{3.007045in}{1.925000in}}%
\pgfusepath{clip}%
\pgfsetbuttcap%
\pgfsetmiterjoin%
\definecolor{currentfill}{rgb}{0.183206,0.496932,0.737516}%
\pgfsetfillcolor{currentfill}%
\pgfsetlinewidth{0.000000pt}%
\definecolor{currentstroke}{rgb}{0.000000,0.000000,0.000000}%
\pgfsetstrokecolor{currentstroke}%
\pgfsetstrokeopacity{0.000000}%
\pgfsetdash{}{0pt}%
\pgfpathmoveto{\pgfqpoint{1.359600in}{1.651719in}}%
\pgfpathlineto{\pgfqpoint{1.355655in}{1.612365in}}%
\pgfpathlineto{\pgfqpoint{1.351714in}{1.614337in}}%
\pgfpathlineto{\pgfqpoint{1.339970in}{1.615629in}}%
\pgfpathlineto{\pgfqpoint{1.340576in}{1.621215in}}%
\pgfpathlineto{\pgfqpoint{1.334802in}{1.621794in}}%
\pgfpathlineto{\pgfqpoint{1.335389in}{1.627188in}}%
\pgfpathlineto{\pgfqpoint{1.318468in}{1.629024in}}%
\pgfpathlineto{\pgfqpoint{1.320048in}{1.651933in}}%
\pgfpathlineto{\pgfqpoint{1.327908in}{1.651084in}}%
\pgfpathlineto{\pgfqpoint{1.329154in}{1.662581in}}%
\pgfpathlineto{\pgfqpoint{1.335063in}{1.663866in}}%
\pgfpathlineto{\pgfqpoint{1.343644in}{1.662962in}}%
\pgfpathlineto{\pgfqpoint{1.346246in}{1.659789in}}%
\pgfpathlineto{\pgfqpoint{1.360257in}{1.658366in}}%
\pgfpathlineto{\pgfqpoint{1.359600in}{1.651719in}}%
\pgfpathclose%
\pgfusepath{fill}%
\end{pgfscope}%
\begin{pgfscope}%
\pgfpathrectangle{\pgfqpoint{0.100000in}{0.100000in}}{\pgfqpoint{3.007045in}{1.925000in}}%
\pgfusepath{clip}%
\pgfsetbuttcap%
\pgfsetmiterjoin%
\definecolor{currentfill}{rgb}{0.592157,0.777086,0.876432}%
\pgfsetfillcolor{currentfill}%
\pgfsetlinewidth{0.000000pt}%
\definecolor{currentstroke}{rgb}{0.000000,0.000000,0.000000}%
\pgfsetstrokecolor{currentstroke}%
\pgfsetstrokeopacity{0.000000}%
\pgfsetdash{}{0pt}%
\pgfpathmoveto{\pgfqpoint{2.765291in}{1.169198in}}%
\pgfpathlineto{\pgfqpoint{2.759352in}{1.165916in}}%
\pgfpathlineto{\pgfqpoint{2.756954in}{1.169348in}}%
\pgfpathlineto{\pgfqpoint{2.759204in}{1.174214in}}%
\pgfpathlineto{\pgfqpoint{2.757805in}{1.180412in}}%
\pgfpathlineto{\pgfqpoint{2.751639in}{1.179119in}}%
\pgfpathlineto{\pgfqpoint{2.751955in}{1.186579in}}%
\pgfpathlineto{\pgfqpoint{2.751060in}{1.190437in}}%
\pgfpathlineto{\pgfqpoint{2.753781in}{1.202785in}}%
\pgfpathlineto{\pgfqpoint{2.758832in}{1.209240in}}%
\pgfpathlineto{\pgfqpoint{2.745705in}{1.255995in}}%
\pgfpathlineto{\pgfqpoint{2.751195in}{1.256851in}}%
\pgfpathlineto{\pgfqpoint{2.757050in}{1.263166in}}%
\pgfpathlineto{\pgfqpoint{2.761478in}{1.260531in}}%
\pgfpathlineto{\pgfqpoint{2.765354in}{1.253983in}}%
\pgfpathlineto{\pgfqpoint{2.766799in}{1.245106in}}%
\pgfpathlineto{\pgfqpoint{2.770462in}{1.242822in}}%
\pgfpathlineto{\pgfqpoint{2.773747in}{1.236358in}}%
\pgfpathlineto{\pgfqpoint{2.781281in}{1.230541in}}%
\pgfpathlineto{\pgfqpoint{2.786967in}{1.229663in}}%
\pgfpathlineto{\pgfqpoint{2.793235in}{1.209405in}}%
\pgfpathlineto{\pgfqpoint{2.791000in}{1.184936in}}%
\pgfpathlineto{\pgfqpoint{2.789514in}{1.179938in}}%
\pgfpathlineto{\pgfqpoint{2.770616in}{1.173606in}}%
\pgfpathlineto{\pgfqpoint{2.765291in}{1.169198in}}%
\pgfpathclose%
\pgfusepath{fill}%
\end{pgfscope}%
\begin{pgfscope}%
\pgfpathrectangle{\pgfqpoint{0.100000in}{0.100000in}}{\pgfqpoint{3.007045in}{1.925000in}}%
\pgfusepath{clip}%
\pgfsetbuttcap%
\pgfsetmiterjoin%
\definecolor{currentfill}{rgb}{0.321246,0.615179,0.800830}%
\pgfsetfillcolor{currentfill}%
\pgfsetlinewidth{0.000000pt}%
\definecolor{currentstroke}{rgb}{0.000000,0.000000,0.000000}%
\pgfsetstrokecolor{currentstroke}%
\pgfsetstrokeopacity{0.000000}%
\pgfsetdash{}{0pt}%
\pgfpathmoveto{\pgfqpoint{1.779241in}{1.344556in}}%
\pgfpathlineto{\pgfqpoint{1.739541in}{1.344779in}}%
\pgfpathlineto{\pgfqpoint{1.739732in}{1.362155in}}%
\pgfpathlineto{\pgfqpoint{1.736283in}{1.367899in}}%
\pgfpathlineto{\pgfqpoint{1.730520in}{1.367928in}}%
\pgfpathlineto{\pgfqpoint{1.730762in}{1.390911in}}%
\pgfpathlineto{\pgfqpoint{1.753560in}{1.390798in}}%
\pgfpathlineto{\pgfqpoint{1.776480in}{1.390760in}}%
\pgfpathlineto{\pgfqpoint{1.776477in}{1.362015in}}%
\pgfpathlineto{\pgfqpoint{1.779293in}{1.361999in}}%
\pgfpathlineto{\pgfqpoint{1.779241in}{1.344556in}}%
\pgfpathclose%
\pgfusepath{fill}%
\end{pgfscope}%
\begin{pgfscope}%
\pgfpathrectangle{\pgfqpoint{0.100000in}{0.100000in}}{\pgfqpoint{3.007045in}{1.925000in}}%
\pgfusepath{clip}%
\pgfsetbuttcap%
\pgfsetmiterjoin%
\definecolor{currentfill}{rgb}{0.031757,0.318139,0.612149}%
\pgfsetfillcolor{currentfill}%
\pgfsetlinewidth{0.000000pt}%
\definecolor{currentstroke}{rgb}{0.000000,0.000000,0.000000}%
\pgfsetstrokecolor{currentstroke}%
\pgfsetstrokeopacity{0.000000}%
\pgfsetdash{}{0pt}%
\pgfpathmoveto{\pgfqpoint{1.011802in}{1.226459in}}%
\pgfpathlineto{\pgfqpoint{1.057846in}{1.219034in}}%
\pgfpathlineto{\pgfqpoint{1.061002in}{1.220921in}}%
\pgfpathlineto{\pgfqpoint{1.059704in}{1.212390in}}%
\pgfpathlineto{\pgfqpoint{1.050444in}{1.156057in}}%
\pgfpathlineto{\pgfqpoint{1.001312in}{1.163987in}}%
\pgfpathlineto{\pgfqpoint{1.002802in}{1.166666in}}%
\pgfpathlineto{\pgfqpoint{0.999346in}{1.179674in}}%
\pgfpathlineto{\pgfqpoint{1.000947in}{1.182123in}}%
\pgfpathlineto{\pgfqpoint{0.998322in}{1.188795in}}%
\pgfpathlineto{\pgfqpoint{1.001745in}{1.203071in}}%
\pgfpathlineto{\pgfqpoint{1.007785in}{1.211607in}}%
\pgfpathlineto{\pgfqpoint{1.007174in}{1.215093in}}%
\pgfpathlineto{\pgfqpoint{1.010713in}{1.219317in}}%
\pgfpathlineto{\pgfqpoint{1.011802in}{1.226459in}}%
\pgfpathclose%
\pgfusepath{fill}%
\end{pgfscope}%
\begin{pgfscope}%
\pgfpathrectangle{\pgfqpoint{0.100000in}{0.100000in}}{\pgfqpoint{3.007045in}{1.925000in}}%
\pgfusepath{clip}%
\pgfsetbuttcap%
\pgfsetmiterjoin%
\definecolor{currentfill}{rgb}{0.585882,0.773641,0.875079}%
\pgfsetfillcolor{currentfill}%
\pgfsetlinewidth{0.000000pt}%
\definecolor{currentstroke}{rgb}{0.000000,0.000000,0.000000}%
\pgfsetstrokecolor{currentstroke}%
\pgfsetstrokeopacity{0.000000}%
\pgfsetdash{}{0pt}%
\pgfpathmoveto{\pgfqpoint{2.334077in}{0.871563in}}%
\pgfpathlineto{\pgfqpoint{2.332579in}{0.883358in}}%
\pgfpathlineto{\pgfqpoint{2.335741in}{0.890096in}}%
\pgfpathlineto{\pgfqpoint{2.333466in}{0.891984in}}%
\pgfpathlineto{\pgfqpoint{2.332547in}{0.900599in}}%
\pgfpathlineto{\pgfqpoint{2.348646in}{0.902478in}}%
\pgfpathlineto{\pgfqpoint{2.356919in}{0.901553in}}%
\pgfpathlineto{\pgfqpoint{2.362404in}{0.891696in}}%
\pgfpathlineto{\pgfqpoint{2.360142in}{0.881259in}}%
\pgfpathlineto{\pgfqpoint{2.358828in}{0.878027in}}%
\pgfpathlineto{\pgfqpoint{2.351405in}{0.875886in}}%
\pgfpathlineto{\pgfqpoint{2.349378in}{0.873307in}}%
\pgfpathlineto{\pgfqpoint{2.334077in}{0.871563in}}%
\pgfpathclose%
\pgfusepath{fill}%
\end{pgfscope}%
\begin{pgfscope}%
\pgfpathrectangle{\pgfqpoint{0.100000in}{0.100000in}}{\pgfqpoint{3.007045in}{1.925000in}}%
\pgfusepath{clip}%
\pgfsetbuttcap%
\pgfsetmiterjoin%
\definecolor{currentfill}{rgb}{0.691580,0.822745,0.907543}%
\pgfsetfillcolor{currentfill}%
\pgfsetlinewidth{0.000000pt}%
\definecolor{currentstroke}{rgb}{0.000000,0.000000,0.000000}%
\pgfsetstrokecolor{currentstroke}%
\pgfsetstrokeopacity{0.000000}%
\pgfsetdash{}{0pt}%
\pgfpathmoveto{\pgfqpoint{0.372778in}{1.622728in}}%
\pgfpathlineto{\pgfqpoint{0.383904in}{1.641105in}}%
\pgfpathlineto{\pgfqpoint{0.396803in}{1.656959in}}%
\pgfpathlineto{\pgfqpoint{0.406441in}{1.674762in}}%
\pgfpathlineto{\pgfqpoint{0.416059in}{1.671542in}}%
\pgfpathlineto{\pgfqpoint{0.423095in}{1.674955in}}%
\pgfpathlineto{\pgfqpoint{0.432010in}{1.670306in}}%
\pgfpathlineto{\pgfqpoint{0.434242in}{1.663279in}}%
\pgfpathlineto{\pgfqpoint{0.441475in}{1.657513in}}%
\pgfpathlineto{\pgfqpoint{0.451005in}{1.654418in}}%
\pgfpathlineto{\pgfqpoint{0.447535in}{1.643476in}}%
\pgfpathlineto{\pgfqpoint{0.447557in}{1.638893in}}%
\pgfpathlineto{\pgfqpoint{0.464341in}{1.633883in}}%
\pgfpathlineto{\pgfqpoint{0.462254in}{1.627100in}}%
\pgfpathlineto{\pgfqpoint{0.490196in}{1.618627in}}%
\pgfpathlineto{\pgfqpoint{0.493741in}{1.612121in}}%
\pgfpathlineto{\pgfqpoint{0.493766in}{1.605432in}}%
\pgfpathlineto{\pgfqpoint{0.487657in}{1.599535in}}%
\pgfpathlineto{\pgfqpoint{0.485002in}{1.594976in}}%
\pgfpathlineto{\pgfqpoint{0.476029in}{1.597225in}}%
\pgfpathlineto{\pgfqpoint{0.474662in}{1.592743in}}%
\pgfpathlineto{\pgfqpoint{0.469189in}{1.594432in}}%
\pgfpathlineto{\pgfqpoint{0.465330in}{1.591720in}}%
\pgfpathlineto{\pgfqpoint{0.460693in}{1.593122in}}%
\pgfpathlineto{\pgfqpoint{0.446705in}{1.586464in}}%
\pgfpathlineto{\pgfqpoint{0.436255in}{1.587889in}}%
\pgfpathlineto{\pgfqpoint{0.429770in}{1.590236in}}%
\pgfpathlineto{\pgfqpoint{0.426249in}{1.588654in}}%
\pgfpathlineto{\pgfqpoint{0.419385in}{1.591902in}}%
\pgfpathlineto{\pgfqpoint{0.415190in}{1.591253in}}%
\pgfpathlineto{\pgfqpoint{0.404299in}{1.600395in}}%
\pgfpathlineto{\pgfqpoint{0.397064in}{1.602137in}}%
\pgfpathlineto{\pgfqpoint{0.389344in}{1.598489in}}%
\pgfpathlineto{\pgfqpoint{0.382499in}{1.599727in}}%
\pgfpathlineto{\pgfqpoint{0.386071in}{1.611691in}}%
\pgfpathlineto{\pgfqpoint{0.382988in}{1.619317in}}%
\pgfpathlineto{\pgfqpoint{0.372778in}{1.622728in}}%
\pgfpathclose%
\pgfusepath{fill}%
\end{pgfscope}%
\begin{pgfscope}%
\pgfpathrectangle{\pgfqpoint{0.100000in}{0.100000in}}{\pgfqpoint{3.007045in}{1.925000in}}%
\pgfusepath{clip}%
\pgfsetbuttcap%
\pgfsetmiterjoin%
\definecolor{currentfill}{rgb}{0.760477,0.852026,0.931657}%
\pgfsetfillcolor{currentfill}%
\pgfsetlinewidth{0.000000pt}%
\definecolor{currentstroke}{rgb}{0.000000,0.000000,0.000000}%
\pgfsetstrokecolor{currentstroke}%
\pgfsetstrokeopacity{0.000000}%
\pgfsetdash{}{0pt}%
\pgfpathmoveto{\pgfqpoint{2.605371in}{0.522544in}}%
\pgfpathlineto{\pgfqpoint{2.591696in}{0.520372in}}%
\pgfpathlineto{\pgfqpoint{2.592396in}{0.508481in}}%
\pgfpathlineto{\pgfqpoint{2.584158in}{0.521537in}}%
\pgfpathlineto{\pgfqpoint{2.581927in}{0.518998in}}%
\pgfpathlineto{\pgfqpoint{2.576314in}{0.520862in}}%
\pgfpathlineto{\pgfqpoint{2.570842in}{0.520187in}}%
\pgfpathlineto{\pgfqpoint{2.567244in}{0.529939in}}%
\pgfpathlineto{\pgfqpoint{2.557842in}{0.535907in}}%
\pgfpathlineto{\pgfqpoint{2.555178in}{0.540975in}}%
\pgfpathlineto{\pgfqpoint{2.550313in}{0.542005in}}%
\pgfpathlineto{\pgfqpoint{2.547864in}{0.546632in}}%
\pgfpathlineto{\pgfqpoint{2.541214in}{0.552000in}}%
\pgfpathlineto{\pgfqpoint{2.538085in}{0.559398in}}%
\pgfpathlineto{\pgfqpoint{2.533920in}{0.561053in}}%
\pgfpathlineto{\pgfqpoint{2.522149in}{0.556021in}}%
\pgfpathlineto{\pgfqpoint{2.520191in}{0.572396in}}%
\pgfpathlineto{\pgfqpoint{2.526180in}{0.575201in}}%
\pgfpathlineto{\pgfqpoint{2.532632in}{0.582139in}}%
\pgfpathlineto{\pgfqpoint{2.545862in}{0.584302in}}%
\pgfpathlineto{\pgfqpoint{2.549830in}{0.579481in}}%
\pgfpathlineto{\pgfqpoint{2.551269in}{0.570455in}}%
\pgfpathlineto{\pgfqpoint{2.562689in}{0.572385in}}%
\pgfpathlineto{\pgfqpoint{2.568608in}{0.576327in}}%
\pgfpathlineto{\pgfqpoint{2.581840in}{0.553861in}}%
\pgfpathlineto{\pgfqpoint{2.588561in}{0.543384in}}%
\pgfpathlineto{\pgfqpoint{2.605371in}{0.522544in}}%
\pgfpathclose%
\pgfusepath{fill}%
\end{pgfscope}%
\begin{pgfscope}%
\pgfpathrectangle{\pgfqpoint{0.100000in}{0.100000in}}{\pgfqpoint{3.007045in}{1.925000in}}%
\pgfusepath{clip}%
\pgfsetbuttcap%
\pgfsetmiterjoin%
\definecolor{currentfill}{rgb}{0.422745,0.684075,0.839892}%
\pgfsetfillcolor{currentfill}%
\pgfsetlinewidth{0.000000pt}%
\definecolor{currentstroke}{rgb}{0.000000,0.000000,0.000000}%
\pgfsetstrokecolor{currentstroke}%
\pgfsetstrokeopacity{0.000000}%
\pgfsetdash{}{0pt}%
\pgfpathmoveto{\pgfqpoint{0.778392in}{0.924882in}}%
\pgfpathlineto{\pgfqpoint{0.767196in}{0.926377in}}%
\pgfpathlineto{\pgfqpoint{0.762651in}{0.922152in}}%
\pgfpathlineto{\pgfqpoint{0.740331in}{0.928918in}}%
\pgfpathlineto{\pgfqpoint{0.735192in}{0.932928in}}%
\pgfpathlineto{\pgfqpoint{0.726621in}{0.944753in}}%
\pgfpathlineto{\pgfqpoint{0.723368in}{0.955271in}}%
\pgfpathlineto{\pgfqpoint{0.723182in}{0.963163in}}%
\pgfpathlineto{\pgfqpoint{0.719619in}{0.967635in}}%
\pgfpathlineto{\pgfqpoint{0.716895in}{0.976496in}}%
\pgfpathlineto{\pgfqpoint{0.718691in}{0.983740in}}%
\pgfpathlineto{\pgfqpoint{0.677176in}{1.047687in}}%
\pgfpathlineto{\pgfqpoint{0.667024in}{1.063268in}}%
\pgfpathlineto{\pgfqpoint{0.616671in}{1.141013in}}%
\pgfpathlineto{\pgfqpoint{0.590725in}{1.181138in}}%
\pgfpathlineto{\pgfqpoint{0.567833in}{1.216455in}}%
\pgfpathlineto{\pgfqpoint{0.571645in}{1.215280in}}%
\pgfpathlineto{\pgfqpoint{0.614328in}{1.243704in}}%
\pgfpathlineto{\pgfqpoint{0.596506in}{1.278590in}}%
\pgfpathlineto{\pgfqpoint{0.597849in}{1.283744in}}%
\pgfpathlineto{\pgfqpoint{0.615468in}{1.284198in}}%
\pgfpathlineto{\pgfqpoint{0.620230in}{1.284320in}}%
\pgfpathlineto{\pgfqpoint{0.643335in}{1.283223in}}%
\pgfpathlineto{\pgfqpoint{0.679545in}{1.274162in}}%
\pgfpathlineto{\pgfqpoint{0.714021in}{1.265975in}}%
\pgfpathlineto{\pgfqpoint{0.752166in}{1.224502in}}%
\pgfpathlineto{\pgfqpoint{0.799942in}{1.214078in}}%
\pgfpathlineto{\pgfqpoint{0.798495in}{1.207353in}}%
\pgfpathlineto{\pgfqpoint{0.788118in}{1.158139in}}%
\pgfpathlineto{\pgfqpoint{0.776912in}{1.106019in}}%
\pgfpathlineto{\pgfqpoint{0.836260in}{1.093783in}}%
\pgfpathlineto{\pgfqpoint{0.854802in}{1.090140in}}%
\pgfpathlineto{\pgfqpoint{0.845589in}{1.070038in}}%
\pgfpathlineto{\pgfqpoint{0.842643in}{1.059879in}}%
\pgfpathlineto{\pgfqpoint{0.842909in}{1.052579in}}%
\pgfpathlineto{\pgfqpoint{0.839109in}{1.048947in}}%
\pgfpathlineto{\pgfqpoint{0.832280in}{1.046410in}}%
\pgfpathlineto{\pgfqpoint{0.818051in}{1.044246in}}%
\pgfpathlineto{\pgfqpoint{0.809522in}{1.041947in}}%
\pgfpathlineto{\pgfqpoint{0.807519in}{1.038421in}}%
\pgfpathlineto{\pgfqpoint{0.803014in}{1.039936in}}%
\pgfpathlineto{\pgfqpoint{0.800098in}{1.036517in}}%
\pgfpathlineto{\pgfqpoint{0.800968in}{1.030983in}}%
\pgfpathlineto{\pgfqpoint{0.798014in}{1.020963in}}%
\pgfpathlineto{\pgfqpoint{0.794371in}{1.003188in}}%
\pgfpathlineto{\pgfqpoint{0.778392in}{0.924882in}}%
\pgfpathclose%
\pgfusepath{fill}%
\end{pgfscope}%
\begin{pgfscope}%
\pgfpathrectangle{\pgfqpoint{0.100000in}{0.100000in}}{\pgfqpoint{3.007045in}{1.925000in}}%
\pgfusepath{clip}%
\pgfsetbuttcap%
\pgfsetmiterjoin%
\definecolor{currentfill}{rgb}{0.381776,0.656517,0.824452}%
\pgfsetfillcolor{currentfill}%
\pgfsetlinewidth{0.000000pt}%
\definecolor{currentstroke}{rgb}{0.000000,0.000000,0.000000}%
\pgfsetstrokecolor{currentstroke}%
\pgfsetstrokeopacity{0.000000}%
\pgfsetdash{}{0pt}%
\pgfpathmoveto{\pgfqpoint{2.480209in}{1.264505in}}%
\pgfpathlineto{\pgfqpoint{2.463037in}{1.262554in}}%
\pgfpathlineto{\pgfqpoint{2.457021in}{1.264583in}}%
\pgfpathlineto{\pgfqpoint{2.440547in}{1.262665in}}%
\pgfpathlineto{\pgfqpoint{2.441070in}{1.257982in}}%
\pgfpathlineto{\pgfqpoint{2.415184in}{1.255293in}}%
\pgfpathlineto{\pgfqpoint{2.414093in}{1.259940in}}%
\pgfpathlineto{\pgfqpoint{2.411854in}{1.281854in}}%
\pgfpathlineto{\pgfqpoint{2.409849in}{1.281241in}}%
\pgfpathlineto{\pgfqpoint{2.408904in}{1.287720in}}%
\pgfpathlineto{\pgfqpoint{2.413572in}{1.288426in}}%
\pgfpathlineto{\pgfqpoint{2.410399in}{1.309530in}}%
\pgfpathlineto{\pgfqpoint{2.434169in}{1.312710in}}%
\pgfpathlineto{\pgfqpoint{2.434855in}{1.307908in}}%
\pgfpathlineto{\pgfqpoint{2.446146in}{1.309080in}}%
\pgfpathlineto{\pgfqpoint{2.445530in}{1.314439in}}%
\pgfpathlineto{\pgfqpoint{2.461738in}{1.316890in}}%
\pgfpathlineto{\pgfqpoint{2.464330in}{1.299934in}}%
\pgfpathlineto{\pgfqpoint{2.472733in}{1.301176in}}%
\pgfpathlineto{\pgfqpoint{2.473535in}{1.295735in}}%
\pgfpathlineto{\pgfqpoint{2.476757in}{1.293301in}}%
\pgfpathlineto{\pgfqpoint{2.474283in}{1.290040in}}%
\pgfpathlineto{\pgfqpoint{2.474602in}{1.281312in}}%
\pgfpathlineto{\pgfqpoint{2.478385in}{1.281786in}}%
\pgfpathlineto{\pgfqpoint{2.480209in}{1.264505in}}%
\pgfpathclose%
\pgfusepath{fill}%
\end{pgfscope}%
\begin{pgfscope}%
\pgfpathrectangle{\pgfqpoint{0.100000in}{0.100000in}}{\pgfqpoint{3.007045in}{1.925000in}}%
\pgfusepath{clip}%
\pgfsetbuttcap%
\pgfsetmiterjoin%
\definecolor{currentfill}{rgb}{0.676817,0.816471,0.902376}%
\pgfsetfillcolor{currentfill}%
\pgfsetlinewidth{0.000000pt}%
\definecolor{currentstroke}{rgb}{0.000000,0.000000,0.000000}%
\pgfsetstrokecolor{currentstroke}%
\pgfsetstrokeopacity{0.000000}%
\pgfsetdash{}{0pt}%
\pgfpathmoveto{\pgfqpoint{2.514084in}{1.019211in}}%
\pgfpathlineto{\pgfqpoint{2.520017in}{1.015115in}}%
\pgfpathlineto{\pgfqpoint{2.522081in}{0.997367in}}%
\pgfpathlineto{\pgfqpoint{2.510978in}{0.993811in}}%
\pgfpathlineto{\pgfqpoint{2.506001in}{0.993623in}}%
\pgfpathlineto{\pgfqpoint{2.499152in}{0.989884in}}%
\pgfpathlineto{\pgfqpoint{2.494094in}{0.994158in}}%
\pgfpathlineto{\pgfqpoint{2.486521in}{0.996045in}}%
\pgfpathlineto{\pgfqpoint{2.490105in}{1.002416in}}%
\pgfpathlineto{\pgfqpoint{2.479391in}{1.010461in}}%
\pgfpathlineto{\pgfqpoint{2.473634in}{1.012683in}}%
\pgfpathlineto{\pgfqpoint{2.475252in}{1.017742in}}%
\pgfpathlineto{\pgfqpoint{2.475027in}{1.025720in}}%
\pgfpathlineto{\pgfqpoint{2.492133in}{1.027383in}}%
\pgfpathlineto{\pgfqpoint{2.492311in}{1.025615in}}%
\pgfpathlineto{\pgfqpoint{2.500787in}{1.015480in}}%
\pgfpathlineto{\pgfqpoint{2.506602in}{1.020237in}}%
\pgfpathlineto{\pgfqpoint{2.514084in}{1.019211in}}%
\pgfpathclose%
\pgfusepath{fill}%
\end{pgfscope}%
\begin{pgfscope}%
\pgfpathrectangle{\pgfqpoint{0.100000in}{0.100000in}}{\pgfqpoint{3.007045in}{1.925000in}}%
\pgfusepath{clip}%
\pgfsetbuttcap%
\pgfsetmiterjoin%
\definecolor{currentfill}{rgb}{0.376732,0.653072,0.822484}%
\pgfsetfillcolor{currentfill}%
\pgfsetlinewidth{0.000000pt}%
\definecolor{currentstroke}{rgb}{0.000000,0.000000,0.000000}%
\pgfsetstrokecolor{currentstroke}%
\pgfsetstrokeopacity{0.000000}%
\pgfsetdash{}{0pt}%
\pgfpathmoveto{\pgfqpoint{1.396518in}{1.346955in}}%
\pgfpathlineto{\pgfqpoint{1.426380in}{1.344786in}}%
\pgfpathlineto{\pgfqpoint{1.424949in}{1.327230in}}%
\pgfpathlineto{\pgfqpoint{1.428992in}{1.326930in}}%
\pgfpathlineto{\pgfqpoint{1.427276in}{1.304079in}}%
\pgfpathlineto{\pgfqpoint{1.423796in}{1.304378in}}%
\pgfpathlineto{\pgfqpoint{1.422895in}{1.292969in}}%
\pgfpathlineto{\pgfqpoint{1.395479in}{1.295197in}}%
\pgfpathlineto{\pgfqpoint{1.396484in}{1.306583in}}%
\pgfpathlineto{\pgfqpoint{1.395381in}{1.309582in}}%
\pgfpathlineto{\pgfqpoint{1.396817in}{1.327688in}}%
\pgfpathlineto{\pgfqpoint{1.395435in}{1.334971in}}%
\pgfpathlineto{\pgfqpoint{1.396518in}{1.346955in}}%
\pgfpathclose%
\pgfusepath{fill}%
\end{pgfscope}%
\begin{pgfscope}%
\pgfpathrectangle{\pgfqpoint{0.100000in}{0.100000in}}{\pgfqpoint{3.007045in}{1.925000in}}%
\pgfusepath{clip}%
\pgfsetbuttcap%
\pgfsetmiterjoin%
\definecolor{currentfill}{rgb}{0.296025,0.597955,0.790988}%
\pgfsetfillcolor{currentfill}%
\pgfsetlinewidth{0.000000pt}%
\definecolor{currentstroke}{rgb}{0.000000,0.000000,0.000000}%
\pgfsetstrokecolor{currentstroke}%
\pgfsetstrokeopacity{0.000000}%
\pgfsetdash{}{0pt}%
\pgfpathmoveto{\pgfqpoint{2.285799in}{1.004151in}}%
\pgfpathlineto{\pgfqpoint{2.285270in}{0.997505in}}%
\pgfpathlineto{\pgfqpoint{2.278118in}{0.995018in}}%
\pgfpathlineto{\pgfqpoint{2.275809in}{0.988285in}}%
\pgfpathlineto{\pgfqpoint{2.271083in}{0.992318in}}%
\pgfpathlineto{\pgfqpoint{2.263992in}{0.994959in}}%
\pgfpathlineto{\pgfqpoint{2.258356in}{0.992865in}}%
\pgfpathlineto{\pgfqpoint{2.258514in}{1.000440in}}%
\pgfpathlineto{\pgfqpoint{2.248936in}{1.000483in}}%
\pgfpathlineto{\pgfqpoint{2.247870in}{1.007837in}}%
\pgfpathlineto{\pgfqpoint{2.236965in}{1.019520in}}%
\pgfpathlineto{\pgfqpoint{2.241044in}{1.027825in}}%
\pgfpathlineto{\pgfqpoint{2.241309in}{1.038598in}}%
\pgfpathlineto{\pgfqpoint{2.234875in}{1.045920in}}%
\pgfpathlineto{\pgfqpoint{2.238225in}{1.047271in}}%
\pgfpathlineto{\pgfqpoint{2.239874in}{1.053973in}}%
\pgfpathlineto{\pgfqpoint{2.260372in}{1.054200in}}%
\pgfpathlineto{\pgfqpoint{2.259090in}{1.046090in}}%
\pgfpathlineto{\pgfqpoint{2.260460in}{1.038312in}}%
\pgfpathlineto{\pgfqpoint{2.263654in}{1.034861in}}%
\pgfpathlineto{\pgfqpoint{2.269307in}{1.034406in}}%
\pgfpathlineto{\pgfqpoint{2.271241in}{1.027836in}}%
\pgfpathlineto{\pgfqpoint{2.274407in}{1.023601in}}%
\pgfpathlineto{\pgfqpoint{2.286020in}{1.023971in}}%
\pgfpathlineto{\pgfqpoint{2.287383in}{1.019436in}}%
\pgfpathlineto{\pgfqpoint{2.285799in}{1.004151in}}%
\pgfpathclose%
\pgfusepath{fill}%
\end{pgfscope}%
\begin{pgfscope}%
\pgfpathrectangle{\pgfqpoint{0.100000in}{0.100000in}}{\pgfqpoint{3.007045in}{1.925000in}}%
\pgfusepath{clip}%
\pgfsetbuttcap%
\pgfsetmiterjoin%
\definecolor{currentfill}{rgb}{0.376732,0.653072,0.822484}%
\pgfsetfillcolor{currentfill}%
\pgfsetlinewidth{0.000000pt}%
\definecolor{currentstroke}{rgb}{0.000000,0.000000,0.000000}%
\pgfsetstrokecolor{currentstroke}%
\pgfsetstrokeopacity{0.000000}%
\pgfsetdash{}{0pt}%
\pgfpathmoveto{\pgfqpoint{2.688736in}{0.994006in}}%
\pgfpathlineto{\pgfqpoint{2.675874in}{0.992417in}}%
\pgfpathlineto{\pgfqpoint{2.672444in}{0.992955in}}%
\pgfpathlineto{\pgfqpoint{2.668059in}{1.000440in}}%
\pgfpathlineto{\pgfqpoint{2.663029in}{1.005596in}}%
\pgfpathlineto{\pgfqpoint{2.671365in}{1.033005in}}%
\pgfpathlineto{\pgfqpoint{2.679615in}{1.030971in}}%
\pgfpathlineto{\pgfqpoint{2.689119in}{1.033149in}}%
\pgfpathlineto{\pgfqpoint{2.697499in}{1.030502in}}%
\pgfpathlineto{\pgfqpoint{2.698517in}{1.027123in}}%
\pgfpathlineto{\pgfqpoint{2.705428in}{1.026285in}}%
\pgfpathlineto{\pgfqpoint{2.709850in}{1.020460in}}%
\pgfpathlineto{\pgfqpoint{2.710520in}{1.014800in}}%
\pgfpathlineto{\pgfqpoint{2.708269in}{1.015278in}}%
\pgfpathlineto{\pgfqpoint{2.688736in}{0.994006in}}%
\pgfpathclose%
\pgfusepath{fill}%
\end{pgfscope}%
\begin{pgfscope}%
\pgfpathrectangle{\pgfqpoint{0.100000in}{0.100000in}}{\pgfqpoint{3.007045in}{1.925000in}}%
\pgfusepath{clip}%
\pgfsetbuttcap%
\pgfsetmiterjoin%
\definecolor{currentfill}{rgb}{0.142607,0.456332,0.716601}%
\pgfsetfillcolor{currentfill}%
\pgfsetlinewidth{0.000000pt}%
\definecolor{currentstroke}{rgb}{0.000000,0.000000,0.000000}%
\pgfsetstrokecolor{currentstroke}%
\pgfsetstrokeopacity{0.000000}%
\pgfsetdash{}{0pt}%
\pgfpathmoveto{\pgfqpoint{0.892728in}{1.460530in}}%
\pgfpathlineto{\pgfqpoint{0.895172in}{1.457614in}}%
\pgfpathlineto{\pgfqpoint{0.903388in}{1.455868in}}%
\pgfpathlineto{\pgfqpoint{0.900100in}{1.438993in}}%
\pgfpathlineto{\pgfqpoint{0.895826in}{1.417699in}}%
\pgfpathlineto{\pgfqpoint{0.853680in}{1.425681in}}%
\pgfpathlineto{\pgfqpoint{0.845804in}{1.427734in}}%
\pgfpathlineto{\pgfqpoint{0.834301in}{1.430229in}}%
\pgfpathlineto{\pgfqpoint{0.840009in}{1.457550in}}%
\pgfpathlineto{\pgfqpoint{0.850305in}{1.455334in}}%
\pgfpathlineto{\pgfqpoint{0.851744in}{1.461596in}}%
\pgfpathlineto{\pgfqpoint{0.858463in}{1.461570in}}%
\pgfpathlineto{\pgfqpoint{0.861547in}{1.476417in}}%
\pgfpathlineto{\pgfqpoint{0.869518in}{1.474687in}}%
\pgfpathlineto{\pgfqpoint{0.873074in}{1.479709in}}%
\pgfpathlineto{\pgfqpoint{0.877767in}{1.502355in}}%
\pgfpathlineto{\pgfqpoint{0.891952in}{1.499400in}}%
\pgfpathlineto{\pgfqpoint{0.887326in}{1.476723in}}%
\pgfpathlineto{\pgfqpoint{0.884523in}{1.477298in}}%
\pgfpathlineto{\pgfqpoint{0.882125in}{1.465664in}}%
\pgfpathlineto{\pgfqpoint{0.887062in}{1.465893in}}%
\pgfpathlineto{\pgfqpoint{0.892728in}{1.460530in}}%
\pgfpathclose%
\pgfusepath{fill}%
\end{pgfscope}%
\begin{pgfscope}%
\pgfpathrectangle{\pgfqpoint{0.100000in}{0.100000in}}{\pgfqpoint{3.007045in}{1.925000in}}%
\pgfusepath{clip}%
\pgfsetbuttcap%
\pgfsetmiterjoin%
\definecolor{currentfill}{rgb}{0.498039,0.725413,0.856132}%
\pgfsetfillcolor{currentfill}%
\pgfsetlinewidth{0.000000pt}%
\definecolor{currentstroke}{rgb}{0.000000,0.000000,0.000000}%
\pgfsetstrokecolor{currentstroke}%
\pgfsetstrokeopacity{0.000000}%
\pgfsetdash{}{0pt}%
\pgfpathmoveto{\pgfqpoint{2.286587in}{0.869005in}}%
\pgfpathlineto{\pgfqpoint{2.279674in}{0.894582in}}%
\pgfpathlineto{\pgfqpoint{2.265651in}{0.893385in}}%
\pgfpathlineto{\pgfqpoint{2.264140in}{0.908633in}}%
\pgfpathlineto{\pgfqpoint{2.261069in}{0.912869in}}%
\pgfpathlineto{\pgfqpoint{2.262417in}{0.916982in}}%
\pgfpathlineto{\pgfqpoint{2.261477in}{0.928516in}}%
\pgfpathlineto{\pgfqpoint{2.264733in}{0.928140in}}%
\pgfpathlineto{\pgfqpoint{2.278483in}{0.930848in}}%
\pgfpathlineto{\pgfqpoint{2.285385in}{0.933858in}}%
\pgfpathlineto{\pgfqpoint{2.287005in}{0.929042in}}%
\pgfpathlineto{\pgfqpoint{2.297460in}{0.921002in}}%
\pgfpathlineto{\pgfqpoint{2.299822in}{0.927995in}}%
\pgfpathlineto{\pgfqpoint{2.308274in}{0.925835in}}%
\pgfpathlineto{\pgfqpoint{2.313145in}{0.917931in}}%
\pgfpathlineto{\pgfqpoint{2.309785in}{0.907283in}}%
\pgfpathlineto{\pgfqpoint{2.313442in}{0.898402in}}%
\pgfpathlineto{\pgfqpoint{2.310101in}{0.886859in}}%
\pgfpathlineto{\pgfqpoint{2.306322in}{0.886459in}}%
\pgfpathlineto{\pgfqpoint{2.305031in}{0.879714in}}%
\pgfpathlineto{\pgfqpoint{2.311402in}{0.880390in}}%
\pgfpathlineto{\pgfqpoint{2.311838in}{0.871613in}}%
\pgfpathlineto{\pgfqpoint{2.286587in}{0.869005in}}%
\pgfpathclose%
\pgfusepath{fill}%
\end{pgfscope}%
\begin{pgfscope}%
\pgfpathrectangle{\pgfqpoint{0.100000in}{0.100000in}}{\pgfqpoint{3.007045in}{1.925000in}}%
\pgfusepath{clip}%
\pgfsetbuttcap%
\pgfsetmiterjoin%
\definecolor{currentfill}{rgb}{0.361599,0.642737,0.816578}%
\pgfsetfillcolor{currentfill}%
\pgfsetlinewidth{0.000000pt}%
\definecolor{currentstroke}{rgb}{0.000000,0.000000,0.000000}%
\pgfsetstrokecolor{currentstroke}%
\pgfsetstrokeopacity{0.000000}%
\pgfsetdash{}{0pt}%
\pgfpathmoveto{\pgfqpoint{2.239459in}{1.266008in}}%
\pgfpathlineto{\pgfqpoint{2.228217in}{1.264785in}}%
\pgfpathlineto{\pgfqpoint{2.224496in}{1.264365in}}%
\pgfpathlineto{\pgfqpoint{2.221882in}{1.287153in}}%
\pgfpathlineto{\pgfqpoint{2.210917in}{1.285878in}}%
\pgfpathlineto{\pgfqpoint{2.211433in}{1.280191in}}%
\pgfpathlineto{\pgfqpoint{2.196588in}{1.278782in}}%
\pgfpathlineto{\pgfqpoint{2.195043in}{1.295965in}}%
\pgfpathlineto{\pgfqpoint{2.215428in}{1.298031in}}%
\pgfpathlineto{\pgfqpoint{2.213486in}{1.315239in}}%
\pgfpathlineto{\pgfqpoint{2.233410in}{1.317320in}}%
\pgfpathlineto{\pgfqpoint{2.234453in}{1.308012in}}%
\pgfpathlineto{\pgfqpoint{2.233530in}{1.300204in}}%
\pgfpathlineto{\pgfqpoint{2.234886in}{1.288646in}}%
\pgfpathlineto{\pgfqpoint{2.236831in}{1.288875in}}%
\pgfpathlineto{\pgfqpoint{2.239459in}{1.266008in}}%
\pgfpathclose%
\pgfusepath{fill}%
\end{pgfscope}%
\begin{pgfscope}%
\pgfpathrectangle{\pgfqpoint{0.100000in}{0.100000in}}{\pgfqpoint{3.007045in}{1.925000in}}%
\pgfusepath{clip}%
\pgfsetbuttcap%
\pgfsetmiterjoin%
\definecolor{currentfill}{rgb}{0.637447,0.799739,0.888597}%
\pgfsetfillcolor{currentfill}%
\pgfsetlinewidth{0.000000pt}%
\definecolor{currentstroke}{rgb}{0.000000,0.000000,0.000000}%
\pgfsetstrokecolor{currentstroke}%
\pgfsetstrokeopacity{0.000000}%
\pgfsetdash{}{0pt}%
\pgfpathmoveto{\pgfqpoint{0.374195in}{1.289386in}}%
\pgfpathlineto{\pgfqpoint{0.369150in}{1.288189in}}%
\pgfpathlineto{\pgfqpoint{0.366648in}{1.294372in}}%
\pgfpathlineto{\pgfqpoint{0.358679in}{1.298249in}}%
\pgfpathlineto{\pgfqpoint{0.352980in}{1.307630in}}%
\pgfpathlineto{\pgfqpoint{0.347482in}{1.307924in}}%
\pgfpathlineto{\pgfqpoint{0.344373in}{1.319665in}}%
\pgfpathlineto{\pgfqpoint{0.339643in}{1.325177in}}%
\pgfpathlineto{\pgfqpoint{0.335632in}{1.333256in}}%
\pgfpathlineto{\pgfqpoint{0.333488in}{1.341560in}}%
\pgfpathlineto{\pgfqpoint{0.327012in}{1.352333in}}%
\pgfpathlineto{\pgfqpoint{0.324629in}{1.358271in}}%
\pgfpathlineto{\pgfqpoint{0.328917in}{1.365788in}}%
\pgfpathlineto{\pgfqpoint{0.328154in}{1.376358in}}%
\pgfpathlineto{\pgfqpoint{0.328676in}{1.386959in}}%
\pgfpathlineto{\pgfqpoint{0.331174in}{1.392985in}}%
\pgfpathlineto{\pgfqpoint{0.335747in}{1.398988in}}%
\pgfpathlineto{\pgfqpoint{0.337136in}{1.407814in}}%
\pgfpathlineto{\pgfqpoint{0.337272in}{1.417649in}}%
\pgfpathlineto{\pgfqpoint{0.332528in}{1.430977in}}%
\pgfpathlineto{\pgfqpoint{0.355543in}{1.423387in}}%
\pgfpathlineto{\pgfqpoint{0.355032in}{1.421828in}}%
\pgfpathlineto{\pgfqpoint{0.384397in}{1.412475in}}%
\pgfpathlineto{\pgfqpoint{0.380635in}{1.401231in}}%
\pgfpathlineto{\pgfqpoint{0.381027in}{1.394938in}}%
\pgfpathlineto{\pgfqpoint{0.378819in}{1.386738in}}%
\pgfpathlineto{\pgfqpoint{0.386077in}{1.384484in}}%
\pgfpathlineto{\pgfqpoint{0.383653in}{1.376276in}}%
\pgfpathlineto{\pgfqpoint{0.379169in}{1.366194in}}%
\pgfpathlineto{\pgfqpoint{0.383879in}{1.360469in}}%
\pgfpathlineto{\pgfqpoint{0.389453in}{1.357782in}}%
\pgfpathlineto{\pgfqpoint{0.387533in}{1.347636in}}%
\pgfpathlineto{\pgfqpoint{0.390745in}{1.343728in}}%
\pgfpathlineto{\pgfqpoint{0.392426in}{1.337198in}}%
\pgfpathlineto{\pgfqpoint{0.388567in}{1.334258in}}%
\pgfpathlineto{\pgfqpoint{0.388700in}{1.330958in}}%
\pgfpathlineto{\pgfqpoint{0.382079in}{1.325306in}}%
\pgfpathlineto{\pgfqpoint{0.373315in}{1.325445in}}%
\pgfpathlineto{\pgfqpoint{0.371022in}{1.321407in}}%
\pgfpathlineto{\pgfqpoint{0.376145in}{1.309067in}}%
\pgfpathlineto{\pgfqpoint{0.377641in}{1.299051in}}%
\pgfpathlineto{\pgfqpoint{0.377156in}{1.290848in}}%
\pgfpathlineto{\pgfqpoint{0.374195in}{1.289386in}}%
\pgfpathclose%
\pgfusepath{fill}%
\end{pgfscope}%
\begin{pgfscope}%
\pgfpathrectangle{\pgfqpoint{0.100000in}{0.100000in}}{\pgfqpoint{3.007045in}{1.925000in}}%
\pgfusepath{clip}%
\pgfsetbuttcap%
\pgfsetmiterjoin%
\definecolor{currentfill}{rgb}{0.579608,0.770196,0.873725}%
\pgfsetfillcolor{currentfill}%
\pgfsetlinewidth{0.000000pt}%
\definecolor{currentstroke}{rgb}{0.000000,0.000000,0.000000}%
\pgfsetstrokecolor{currentstroke}%
\pgfsetstrokeopacity{0.000000}%
\pgfsetdash{}{0pt}%
\pgfpathmoveto{\pgfqpoint{2.458232in}{1.094688in}}%
\pgfpathlineto{\pgfqpoint{2.446820in}{1.093542in}}%
\pgfpathlineto{\pgfqpoint{2.441207in}{1.098026in}}%
\pgfpathlineto{\pgfqpoint{2.440031in}{1.105220in}}%
\pgfpathlineto{\pgfqpoint{2.435308in}{1.112304in}}%
\pgfpathlineto{\pgfqpoint{2.429643in}{1.109432in}}%
\pgfpathlineto{\pgfqpoint{2.433994in}{1.115719in}}%
\pgfpathlineto{\pgfqpoint{2.428858in}{1.122661in}}%
\pgfpathlineto{\pgfqpoint{2.428704in}{1.126732in}}%
\pgfpathlineto{\pgfqpoint{2.431774in}{1.134957in}}%
\pgfpathlineto{\pgfqpoint{2.438948in}{1.139413in}}%
\pgfpathlineto{\pgfqpoint{2.437602in}{1.145887in}}%
\pgfpathlineto{\pgfqpoint{2.435873in}{1.154519in}}%
\pgfpathlineto{\pgfqpoint{2.440950in}{1.158535in}}%
\pgfpathlineto{\pgfqpoint{2.450037in}{1.161421in}}%
\pgfpathlineto{\pgfqpoint{2.458391in}{1.154238in}}%
\pgfpathlineto{\pgfqpoint{2.463450in}{1.158871in}}%
\pgfpathlineto{\pgfqpoint{2.466955in}{1.155210in}}%
\pgfpathlineto{\pgfqpoint{2.480993in}{1.155854in}}%
\pgfpathlineto{\pgfqpoint{2.487851in}{1.166255in}}%
\pgfpathlineto{\pgfqpoint{2.494103in}{1.161327in}}%
\pgfpathlineto{\pgfqpoint{2.497293in}{1.156805in}}%
\pgfpathlineto{\pgfqpoint{2.496539in}{1.150737in}}%
\pgfpathlineto{\pgfqpoint{2.481701in}{1.138352in}}%
\pgfpathlineto{\pgfqpoint{2.477515in}{1.130920in}}%
\pgfpathlineto{\pgfqpoint{2.476974in}{1.118028in}}%
\pgfpathlineto{\pgfqpoint{2.470294in}{1.118309in}}%
\pgfpathlineto{\pgfqpoint{2.467580in}{1.113861in}}%
\pgfpathlineto{\pgfqpoint{2.471855in}{1.105407in}}%
\pgfpathlineto{\pgfqpoint{2.459342in}{1.100923in}}%
\pgfpathlineto{\pgfqpoint{2.463061in}{1.098120in}}%
\pgfpathlineto{\pgfqpoint{2.458232in}{1.094688in}}%
\pgfpathclose%
\pgfusepath{fill}%
\end{pgfscope}%
\begin{pgfscope}%
\pgfpathrectangle{\pgfqpoint{0.100000in}{0.100000in}}{\pgfqpoint{3.007045in}{1.925000in}}%
\pgfusepath{clip}%
\pgfsetbuttcap%
\pgfsetmiterjoin%
\definecolor{currentfill}{rgb}{0.460392,0.704744,0.848012}%
\pgfsetfillcolor{currentfill}%
\pgfsetlinewidth{0.000000pt}%
\definecolor{currentstroke}{rgb}{0.000000,0.000000,0.000000}%
\pgfsetstrokecolor{currentstroke}%
\pgfsetstrokeopacity{0.000000}%
\pgfsetdash{}{0pt}%
\pgfpathmoveto{\pgfqpoint{1.298841in}{1.455523in}}%
\pgfpathlineto{\pgfqpoint{1.293878in}{1.409890in}}%
\pgfpathlineto{\pgfqpoint{1.292810in}{1.397021in}}%
\pgfpathlineto{\pgfqpoint{1.273764in}{1.398937in}}%
\pgfpathlineto{\pgfqpoint{1.272590in}{1.387487in}}%
\pgfpathlineto{\pgfqpoint{1.270231in}{1.387736in}}%
\pgfpathlineto{\pgfqpoint{1.267095in}{1.378734in}}%
\pgfpathlineto{\pgfqpoint{1.255653in}{1.381012in}}%
\pgfpathlineto{\pgfqpoint{1.256175in}{1.385798in}}%
\pgfpathlineto{\pgfqpoint{1.259419in}{1.388949in}}%
\pgfpathlineto{\pgfqpoint{1.234424in}{1.392114in}}%
\pgfpathlineto{\pgfqpoint{1.207752in}{1.395280in}}%
\pgfpathlineto{\pgfqpoint{1.164443in}{1.401325in}}%
\pgfpathlineto{\pgfqpoint{1.163522in}{1.401453in}}%
\pgfpathlineto{\pgfqpoint{1.166601in}{1.424158in}}%
\pgfpathlineto{\pgfqpoint{1.168621in}{1.423872in}}%
\pgfpathlineto{\pgfqpoint{1.171716in}{1.446544in}}%
\pgfpathlineto{\pgfqpoint{1.174139in}{1.469186in}}%
\pgfpathlineto{\pgfqpoint{1.173551in}{1.471182in}}%
\pgfpathlineto{\pgfqpoint{1.193681in}{1.468364in}}%
\pgfpathlineto{\pgfqpoint{1.218116in}{1.464571in}}%
\pgfpathlineto{\pgfqpoint{1.245609in}{1.461365in}}%
\pgfpathlineto{\pgfqpoint{1.298841in}{1.455523in}}%
\pgfpathclose%
\pgfusepath{fill}%
\end{pgfscope}%
\begin{pgfscope}%
\pgfpathrectangle{\pgfqpoint{0.100000in}{0.100000in}}{\pgfqpoint{3.007045in}{1.925000in}}%
\pgfusepath{clip}%
\pgfsetbuttcap%
\pgfsetmiterjoin%
\definecolor{currentfill}{rgb}{0.371688,0.649627,0.820515}%
\pgfsetfillcolor{currentfill}%
\pgfsetlinewidth{0.000000pt}%
\definecolor{currentstroke}{rgb}{0.000000,0.000000,0.000000}%
\pgfsetstrokecolor{currentstroke}%
\pgfsetstrokeopacity{0.000000}%
\pgfsetdash{}{0pt}%
\pgfpathmoveto{\pgfqpoint{1.756771in}{1.198935in}}%
\pgfpathlineto{\pgfqpoint{1.755176in}{1.198941in}}%
\pgfpathlineto{\pgfqpoint{1.709799in}{1.199381in}}%
\pgfpathlineto{\pgfqpoint{1.698475in}{1.199567in}}%
\pgfpathlineto{\pgfqpoint{1.699296in}{1.251149in}}%
\pgfpathlineto{\pgfqpoint{1.730711in}{1.250739in}}%
\pgfpathlineto{\pgfqpoint{1.727958in}{1.247635in}}%
\pgfpathlineto{\pgfqpoint{1.732568in}{1.242722in}}%
\pgfpathlineto{\pgfqpoint{1.734017in}{1.237625in}}%
\pgfpathlineto{\pgfqpoint{1.741287in}{1.219682in}}%
\pgfpathlineto{\pgfqpoint{1.747673in}{1.215377in}}%
\pgfpathlineto{\pgfqpoint{1.748211in}{1.211203in}}%
\pgfpathlineto{\pgfqpoint{1.751959in}{1.207454in}}%
\pgfpathlineto{\pgfqpoint{1.751069in}{1.202842in}}%
\pgfpathlineto{\pgfqpoint{1.756771in}{1.198935in}}%
\pgfpathclose%
\pgfusepath{fill}%
\end{pgfscope}%
\begin{pgfscope}%
\pgfpathrectangle{\pgfqpoint{0.100000in}{0.100000in}}{\pgfqpoint{3.007045in}{1.925000in}}%
\pgfusepath{clip}%
\pgfsetbuttcap%
\pgfsetmiterjoin%
\definecolor{currentfill}{rgb}{0.087120,0.389004,0.667512}%
\pgfsetfillcolor{currentfill}%
\pgfsetlinewidth{0.000000pt}%
\definecolor{currentstroke}{rgb}{0.000000,0.000000,0.000000}%
\pgfsetstrokecolor{currentstroke}%
\pgfsetstrokeopacity{0.000000}%
\pgfsetdash{}{0pt}%
\pgfpathmoveto{\pgfqpoint{1.370186in}{0.721452in}}%
\pgfpathlineto{\pgfqpoint{1.322173in}{0.725422in}}%
\pgfpathlineto{\pgfqpoint{1.319633in}{0.696676in}}%
\pgfpathlineto{\pgfqpoint{1.319120in}{0.690958in}}%
\pgfpathlineto{\pgfqpoint{1.282225in}{0.694354in}}%
\pgfpathlineto{\pgfqpoint{1.285541in}{0.728841in}}%
\pgfpathlineto{\pgfqpoint{1.280477in}{0.729323in}}%
\pgfpathlineto{\pgfqpoint{1.283283in}{0.758525in}}%
\pgfpathlineto{\pgfqpoint{1.285953in}{0.758267in}}%
\pgfpathlineto{\pgfqpoint{1.288691in}{0.786623in}}%
\pgfpathlineto{\pgfqpoint{1.291172in}{0.786411in}}%
\pgfpathlineto{\pgfqpoint{1.292309in}{0.797855in}}%
\pgfpathlineto{\pgfqpoint{1.303863in}{0.796794in}}%
\pgfpathlineto{\pgfqpoint{1.328991in}{0.794521in}}%
\pgfpathlineto{\pgfqpoint{1.327701in}{0.782533in}}%
\pgfpathlineto{\pgfqpoint{1.381713in}{0.778104in}}%
\pgfpathlineto{\pgfqpoint{1.379483in}{0.749728in}}%
\pgfpathlineto{\pgfqpoint{1.372150in}{0.750256in}}%
\pgfpathlineto{\pgfqpoint{1.370186in}{0.721452in}}%
\pgfpathclose%
\pgfusepath{fill}%
\end{pgfscope}%
\begin{pgfscope}%
\pgfpathrectangle{\pgfqpoint{0.100000in}{0.100000in}}{\pgfqpoint{3.007045in}{1.925000in}}%
\pgfusepath{clip}%
\pgfsetbuttcap%
\pgfsetmiterjoin%
\definecolor{currentfill}{rgb}{0.422745,0.684075,0.839892}%
\pgfsetfillcolor{currentfill}%
\pgfsetlinewidth{0.000000pt}%
\definecolor{currentstroke}{rgb}{0.000000,0.000000,0.000000}%
\pgfsetstrokecolor{currentstroke}%
\pgfsetstrokeopacity{0.000000}%
\pgfsetdash{}{0pt}%
\pgfpathmoveto{\pgfqpoint{2.139901in}{1.326614in}}%
\pgfpathlineto{\pgfqpoint{2.141986in}{1.299655in}}%
\pgfpathlineto{\pgfqpoint{2.117970in}{1.297350in}}%
\pgfpathlineto{\pgfqpoint{2.118514in}{1.291662in}}%
\pgfpathlineto{\pgfqpoint{2.107035in}{1.290511in}}%
\pgfpathlineto{\pgfqpoint{2.107101in}{1.284740in}}%
\pgfpathlineto{\pgfqpoint{2.090562in}{1.283138in}}%
\pgfpathlineto{\pgfqpoint{2.087331in}{1.317506in}}%
\pgfpathlineto{\pgfqpoint{2.085344in}{1.346129in}}%
\pgfpathlineto{\pgfqpoint{2.085588in}{1.351908in}}%
\pgfpathlineto{\pgfqpoint{2.079874in}{1.351505in}}%
\pgfpathlineto{\pgfqpoint{2.078237in}{1.373879in}}%
\pgfpathlineto{\pgfqpoint{2.122132in}{1.377026in}}%
\pgfpathlineto{\pgfqpoint{2.121650in}{1.363916in}}%
\pgfpathlineto{\pgfqpoint{2.123749in}{1.358371in}}%
\pgfpathlineto{\pgfqpoint{2.130184in}{1.350131in}}%
\pgfpathlineto{\pgfqpoint{2.134976in}{1.335269in}}%
\pgfpathlineto{\pgfqpoint{2.139901in}{1.326614in}}%
\pgfpathclose%
\pgfusepath{fill}%
\end{pgfscope}%
\begin{pgfscope}%
\pgfpathrectangle{\pgfqpoint{0.100000in}{0.100000in}}{\pgfqpoint{3.007045in}{1.925000in}}%
\pgfusepath{clip}%
\pgfsetbuttcap%
\pgfsetmiterjoin%
\definecolor{currentfill}{rgb}{0.371688,0.649627,0.820515}%
\pgfsetfillcolor{currentfill}%
\pgfsetlinewidth{0.000000pt}%
\definecolor{currentstroke}{rgb}{0.000000,0.000000,0.000000}%
\pgfsetstrokecolor{currentstroke}%
\pgfsetstrokeopacity{0.000000}%
\pgfsetdash{}{0pt}%
\pgfpathmoveto{\pgfqpoint{1.652076in}{1.171950in}}%
\pgfpathlineto{\pgfqpoint{1.651571in}{1.154782in}}%
\pgfpathlineto{\pgfqpoint{1.623055in}{1.155630in}}%
\pgfpathlineto{\pgfqpoint{1.623649in}{1.172778in}}%
\pgfpathlineto{\pgfqpoint{1.594392in}{1.173857in}}%
\pgfpathlineto{\pgfqpoint{1.595536in}{1.202468in}}%
\pgfpathlineto{\pgfqpoint{1.607157in}{1.202049in}}%
\pgfpathlineto{\pgfqpoint{1.652796in}{1.200602in}}%
\pgfpathlineto{\pgfqpoint{1.652076in}{1.171950in}}%
\pgfpathclose%
\pgfusepath{fill}%
\end{pgfscope}%
\begin{pgfscope}%
\pgfpathrectangle{\pgfqpoint{0.100000in}{0.100000in}}{\pgfqpoint{3.007045in}{1.925000in}}%
\pgfusepath{clip}%
\pgfsetbuttcap%
\pgfsetmiterjoin%
\definecolor{currentfill}{rgb}{0.074817,0.373256,0.655210}%
\pgfsetfillcolor{currentfill}%
\pgfsetlinewidth{0.000000pt}%
\definecolor{currentstroke}{rgb}{0.000000,0.000000,0.000000}%
\pgfsetstrokecolor{currentstroke}%
\pgfsetstrokeopacity{0.000000}%
\pgfsetdash{}{0pt}%
\pgfpathmoveto{\pgfqpoint{1.426380in}{1.344786in}}%
\pgfpathlineto{\pgfqpoint{1.396518in}{1.346955in}}%
\pgfpathlineto{\pgfqpoint{1.393848in}{1.352944in}}%
\pgfpathlineto{\pgfqpoint{1.395452in}{1.375827in}}%
\pgfpathlineto{\pgfqpoint{1.394290in}{1.375922in}}%
\pgfpathlineto{\pgfqpoint{1.396236in}{1.398628in}}%
\pgfpathlineto{\pgfqpoint{1.396545in}{1.412828in}}%
\pgfpathlineto{\pgfqpoint{1.386507in}{1.413753in}}%
\pgfpathlineto{\pgfqpoint{1.390013in}{1.453572in}}%
\pgfpathlineto{\pgfqpoint{1.395146in}{1.457695in}}%
\pgfpathlineto{\pgfqpoint{1.399572in}{1.458189in}}%
\pgfpathlineto{\pgfqpoint{1.429680in}{1.455658in}}%
\pgfpathlineto{\pgfqpoint{1.431599in}{1.456389in}}%
\pgfpathlineto{\pgfqpoint{1.430442in}{1.441625in}}%
\pgfpathlineto{\pgfqpoint{1.431343in}{1.435804in}}%
\pgfpathlineto{\pgfqpoint{1.473591in}{1.432656in}}%
\pgfpathlineto{\pgfqpoint{1.471839in}{1.406868in}}%
\pgfpathlineto{\pgfqpoint{1.430688in}{1.409985in}}%
\pgfpathlineto{\pgfqpoint{1.430345in}{1.395850in}}%
\pgfpathlineto{\pgfqpoint{1.428602in}{1.373346in}}%
\pgfpathlineto{\pgfqpoint{1.429987in}{1.373229in}}%
\pgfpathlineto{\pgfqpoint{1.428115in}{1.350443in}}%
\pgfpathlineto{\pgfqpoint{1.426380in}{1.344786in}}%
\pgfpathclose%
\pgfusepath{fill}%
\end{pgfscope}%
\begin{pgfscope}%
\pgfpathrectangle{\pgfqpoint{0.100000in}{0.100000in}}{\pgfqpoint{3.007045in}{1.925000in}}%
\pgfusepath{clip}%
\pgfsetbuttcap%
\pgfsetmiterjoin%
\definecolor{currentfill}{rgb}{0.541961,0.749527,0.865606}%
\pgfsetfillcolor{currentfill}%
\pgfsetlinewidth{0.000000pt}%
\definecolor{currentstroke}{rgb}{0.000000,0.000000,0.000000}%
\pgfsetstrokecolor{currentstroke}%
\pgfsetstrokeopacity{0.000000}%
\pgfsetdash{}{0pt}%
\pgfpathmoveto{\pgfqpoint{1.797461in}{1.775190in}}%
\pgfpathlineto{\pgfqpoint{1.804817in}{1.775072in}}%
\pgfpathlineto{\pgfqpoint{1.805689in}{1.772077in}}%
\pgfpathlineto{\pgfqpoint{1.823819in}{1.770365in}}%
\pgfpathlineto{\pgfqpoint{1.825676in}{1.763110in}}%
\pgfpathlineto{\pgfqpoint{1.840111in}{1.765309in}}%
\pgfpathlineto{\pgfqpoint{1.840141in}{1.768138in}}%
\pgfpathlineto{\pgfqpoint{1.851487in}{1.771989in}}%
\pgfpathlineto{\pgfqpoint{1.856761in}{1.771102in}}%
\pgfpathlineto{\pgfqpoint{1.857189in}{1.733722in}}%
\pgfpathlineto{\pgfqpoint{1.858121in}{1.722001in}}%
\pgfpathlineto{\pgfqpoint{1.827059in}{1.721935in}}%
\pgfpathlineto{\pgfqpoint{1.827086in}{1.718431in}}%
\pgfpathlineto{\pgfqpoint{1.798346in}{1.718105in}}%
\pgfpathlineto{\pgfqpoint{1.797671in}{1.752902in}}%
\pgfpathlineto{\pgfqpoint{1.797461in}{1.775190in}}%
\pgfpathclose%
\pgfusepath{fill}%
\end{pgfscope}%
\begin{pgfscope}%
\pgfpathrectangle{\pgfqpoint{0.100000in}{0.100000in}}{\pgfqpoint{3.007045in}{1.925000in}}%
\pgfusepath{clip}%
\pgfsetbuttcap%
\pgfsetmiterjoin%
\definecolor{currentfill}{rgb}{0.326290,0.618624,0.802799}%
\pgfsetfillcolor{currentfill}%
\pgfsetlinewidth{0.000000pt}%
\definecolor{currentstroke}{rgb}{0.000000,0.000000,0.000000}%
\pgfsetstrokecolor{currentstroke}%
\pgfsetstrokeopacity{0.000000}%
\pgfsetdash{}{0pt}%
\pgfpathmoveto{\pgfqpoint{2.145056in}{1.262796in}}%
\pgfpathlineto{\pgfqpoint{2.146371in}{1.246681in}}%
\pgfpathlineto{\pgfqpoint{2.125964in}{1.244692in}}%
\pgfpathlineto{\pgfqpoint{2.116838in}{1.244113in}}%
\pgfpathlineto{\pgfqpoint{2.113613in}{1.277542in}}%
\pgfpathlineto{\pgfqpoint{2.107905in}{1.276923in}}%
\pgfpathlineto{\pgfqpoint{2.107101in}{1.284740in}}%
\pgfpathlineto{\pgfqpoint{2.107035in}{1.290511in}}%
\pgfpathlineto{\pgfqpoint{2.118514in}{1.291662in}}%
\pgfpathlineto{\pgfqpoint{2.117970in}{1.297350in}}%
\pgfpathlineto{\pgfqpoint{2.141986in}{1.299655in}}%
\pgfpathlineto{\pgfqpoint{2.145056in}{1.262796in}}%
\pgfpathclose%
\pgfusepath{fill}%
\end{pgfscope}%
\begin{pgfscope}%
\pgfpathrectangle{\pgfqpoint{0.100000in}{0.100000in}}{\pgfqpoint{3.007045in}{1.925000in}}%
\pgfusepath{clip}%
\pgfsetbuttcap%
\pgfsetmiterjoin%
\definecolor{currentfill}{rgb}{0.406997,0.673741,0.834295}%
\pgfsetfillcolor{currentfill}%
\pgfsetlinewidth{0.000000pt}%
\definecolor{currentstroke}{rgb}{0.000000,0.000000,0.000000}%
\pgfsetstrokecolor{currentstroke}%
\pgfsetstrokeopacity{0.000000}%
\pgfsetdash{}{0pt}%
\pgfpathmoveto{\pgfqpoint{2.060711in}{0.508116in}}%
\pgfpathlineto{\pgfqpoint{2.054795in}{0.510464in}}%
\pgfpathlineto{\pgfqpoint{2.051671in}{0.516053in}}%
\pgfpathlineto{\pgfqpoint{2.053131in}{0.522603in}}%
\pgfpathlineto{\pgfqpoint{2.050269in}{0.524553in}}%
\pgfpathlineto{\pgfqpoint{2.042354in}{0.524234in}}%
\pgfpathlineto{\pgfqpoint{2.040150in}{0.528704in}}%
\pgfpathlineto{\pgfqpoint{2.034075in}{0.531818in}}%
\pgfpathlineto{\pgfqpoint{2.030059in}{0.536911in}}%
\pgfpathlineto{\pgfqpoint{2.010082in}{0.536853in}}%
\pgfpathlineto{\pgfqpoint{2.005579in}{0.541500in}}%
\pgfpathlineto{\pgfqpoint{2.005096in}{0.547382in}}%
\pgfpathlineto{\pgfqpoint{2.007733in}{0.550626in}}%
\pgfpathlineto{\pgfqpoint{2.023323in}{0.554899in}}%
\pgfpathlineto{\pgfqpoint{2.029617in}{0.558203in}}%
\pgfpathlineto{\pgfqpoint{2.032903in}{0.562233in}}%
\pgfpathlineto{\pgfqpoint{2.042685in}{0.564352in}}%
\pgfpathlineto{\pgfqpoint{2.046065in}{0.559905in}}%
\pgfpathlineto{\pgfqpoint{2.043709in}{0.592924in}}%
\pgfpathlineto{\pgfqpoint{2.037435in}{0.611183in}}%
\pgfpathlineto{\pgfqpoint{2.072666in}{0.613237in}}%
\pgfpathlineto{\pgfqpoint{2.070931in}{0.604028in}}%
\pgfpathlineto{\pgfqpoint{2.067754in}{0.598346in}}%
\pgfpathlineto{\pgfqpoint{2.067276in}{0.590315in}}%
\pgfpathlineto{\pgfqpoint{2.072286in}{0.580627in}}%
\pgfpathlineto{\pgfqpoint{2.077352in}{0.576813in}}%
\pgfpathlineto{\pgfqpoint{2.082124in}{0.561840in}}%
\pgfpathlineto{\pgfqpoint{2.087534in}{0.559447in}}%
\pgfpathlineto{\pgfqpoint{2.082198in}{0.557399in}}%
\pgfpathlineto{\pgfqpoint{2.076836in}{0.548759in}}%
\pgfpathlineto{\pgfqpoint{2.071543in}{0.549265in}}%
\pgfpathlineto{\pgfqpoint{2.070214in}{0.542954in}}%
\pgfpathlineto{\pgfqpoint{2.075634in}{0.542956in}}%
\pgfpathlineto{\pgfqpoint{2.077930in}{0.538591in}}%
\pgfpathlineto{\pgfqpoint{2.084493in}{0.539170in}}%
\pgfpathlineto{\pgfqpoint{2.085010in}{0.546770in}}%
\pgfpathlineto{\pgfqpoint{2.089855in}{0.550151in}}%
\pgfpathlineto{\pgfqpoint{2.100634in}{0.542949in}}%
\pgfpathlineto{\pgfqpoint{2.094872in}{0.530452in}}%
\pgfpathlineto{\pgfqpoint{2.081945in}{0.530204in}}%
\pgfpathlineto{\pgfqpoint{2.084051in}{0.524154in}}%
\pgfpathlineto{\pgfqpoint{2.081293in}{0.516251in}}%
\pgfpathlineto{\pgfqpoint{2.087768in}{0.513497in}}%
\pgfpathlineto{\pgfqpoint{2.093764in}{0.507672in}}%
\pgfpathlineto{\pgfqpoint{2.109818in}{0.505073in}}%
\pgfpathlineto{\pgfqpoint{2.117129in}{0.495373in}}%
\pgfpathlineto{\pgfqpoint{2.122084in}{0.494513in}}%
\pgfpathlineto{\pgfqpoint{2.119160in}{0.486899in}}%
\pgfpathlineto{\pgfqpoint{2.113765in}{0.483500in}}%
\pgfpathlineto{\pgfqpoint{2.109844in}{0.485263in}}%
\pgfpathlineto{\pgfqpoint{2.102674in}{0.484267in}}%
\pgfpathlineto{\pgfqpoint{2.093049in}{0.495483in}}%
\pgfpathlineto{\pgfqpoint{2.079626in}{0.500114in}}%
\pgfpathlineto{\pgfqpoint{2.072262in}{0.500507in}}%
\pgfpathlineto{\pgfqpoint{2.073553in}{0.504658in}}%
\pgfpathlineto{\pgfqpoint{2.072286in}{0.510828in}}%
\pgfpathlineto{\pgfqpoint{2.060711in}{0.508116in}}%
\pgfpathclose%
\pgfusepath{fill}%
\end{pgfscope}%
\begin{pgfscope}%
\pgfpathrectangle{\pgfqpoint{0.100000in}{0.100000in}}{\pgfqpoint{3.007045in}{1.925000in}}%
\pgfusepath{clip}%
\pgfsetbuttcap%
\pgfsetmiterjoin%
\definecolor{currentfill}{rgb}{0.406997,0.673741,0.834295}%
\pgfsetfillcolor{currentfill}%
\pgfsetlinewidth{0.000000pt}%
\definecolor{currentstroke}{rgb}{0.000000,0.000000,0.000000}%
\pgfsetstrokecolor{currentstroke}%
\pgfsetstrokeopacity{0.000000}%
\pgfsetdash{}{0pt}%
\pgfpathmoveto{\pgfqpoint{1.938345in}{1.353373in}}%
\pgfpathlineto{\pgfqpoint{1.904114in}{1.352301in}}%
\pgfpathlineto{\pgfqpoint{1.881224in}{1.351626in}}%
\pgfpathlineto{\pgfqpoint{1.881395in}{1.345922in}}%
\pgfpathlineto{\pgfqpoint{1.869963in}{1.345597in}}%
\pgfpathlineto{\pgfqpoint{1.869522in}{1.362845in}}%
\pgfpathlineto{\pgfqpoint{1.868339in}{1.362827in}}%
\pgfpathlineto{\pgfqpoint{1.867818in}{1.391658in}}%
\pgfpathlineto{\pgfqpoint{1.913408in}{1.392837in}}%
\pgfpathlineto{\pgfqpoint{1.913064in}{1.404438in}}%
\pgfpathlineto{\pgfqpoint{1.936000in}{1.405159in}}%
\pgfpathlineto{\pgfqpoint{1.937355in}{1.364751in}}%
\pgfpathlineto{\pgfqpoint{1.938345in}{1.353373in}}%
\pgfpathclose%
\pgfusepath{fill}%
\end{pgfscope}%
\begin{pgfscope}%
\pgfpathrectangle{\pgfqpoint{0.100000in}{0.100000in}}{\pgfqpoint{3.007045in}{1.925000in}}%
\pgfusepath{clip}%
\pgfsetbuttcap%
\pgfsetmiterjoin%
\definecolor{currentfill}{rgb}{0.311157,0.608289,0.796894}%
\pgfsetfillcolor{currentfill}%
\pgfsetlinewidth{0.000000pt}%
\definecolor{currentstroke}{rgb}{0.000000,0.000000,0.000000}%
\pgfsetstrokecolor{currentstroke}%
\pgfsetstrokeopacity{0.000000}%
\pgfsetdash{}{0pt}%
\pgfpathmoveto{\pgfqpoint{2.405437in}{0.788426in}}%
\pgfpathlineto{\pgfqpoint{2.390888in}{0.783936in}}%
\pgfpathlineto{\pgfqpoint{2.390153in}{0.787078in}}%
\pgfpathlineto{\pgfqpoint{2.390080in}{0.791606in}}%
\pgfpathlineto{\pgfqpoint{2.386212in}{0.799739in}}%
\pgfpathlineto{\pgfqpoint{2.382395in}{0.804196in}}%
\pgfpathlineto{\pgfqpoint{2.375369in}{0.793380in}}%
\pgfpathlineto{\pgfqpoint{2.365423in}{0.795571in}}%
\pgfpathlineto{\pgfqpoint{2.357695in}{0.794657in}}%
\pgfpathlineto{\pgfqpoint{2.355976in}{0.788053in}}%
\pgfpathlineto{\pgfqpoint{2.352410in}{0.787703in}}%
\pgfpathlineto{\pgfqpoint{2.346769in}{0.793773in}}%
\pgfpathlineto{\pgfqpoint{2.343332in}{0.803548in}}%
\pgfpathlineto{\pgfqpoint{2.332358in}{0.802455in}}%
\pgfpathlineto{\pgfqpoint{2.327623in}{0.805924in}}%
\pgfpathlineto{\pgfqpoint{2.326336in}{0.819580in}}%
\pgfpathlineto{\pgfqpoint{2.318665in}{0.820809in}}%
\pgfpathlineto{\pgfqpoint{2.317320in}{0.826818in}}%
\pgfpathlineto{\pgfqpoint{2.320886in}{0.830332in}}%
\pgfpathlineto{\pgfqpoint{2.322977in}{0.839287in}}%
\pgfpathlineto{\pgfqpoint{2.337348in}{0.840608in}}%
\pgfpathlineto{\pgfqpoint{2.334077in}{0.871563in}}%
\pgfpathlineto{\pgfqpoint{2.349378in}{0.873307in}}%
\pgfpathlineto{\pgfqpoint{2.351405in}{0.875886in}}%
\pgfpathlineto{\pgfqpoint{2.358828in}{0.878027in}}%
\pgfpathlineto{\pgfqpoint{2.359050in}{0.873951in}}%
\pgfpathlineto{\pgfqpoint{2.365118in}{0.869291in}}%
\pgfpathlineto{\pgfqpoint{2.371582in}{0.867326in}}%
\pgfpathlineto{\pgfqpoint{2.375538in}{0.858536in}}%
\pgfpathlineto{\pgfqpoint{2.372820in}{0.852600in}}%
\pgfpathlineto{\pgfqpoint{2.380206in}{0.847045in}}%
\pgfpathlineto{\pgfqpoint{2.382703in}{0.849302in}}%
\pgfpathlineto{\pgfqpoint{2.380905in}{0.840878in}}%
\pgfpathlineto{\pgfqpoint{2.387448in}{0.834640in}}%
\pgfpathlineto{\pgfqpoint{2.392091in}{0.836488in}}%
\pgfpathlineto{\pgfqpoint{2.403550in}{0.830476in}}%
\pgfpathlineto{\pgfqpoint{2.408663in}{0.823968in}}%
\pgfpathlineto{\pgfqpoint{2.417379in}{0.810753in}}%
\pgfpathlineto{\pgfqpoint{2.403890in}{0.805755in}}%
\pgfpathlineto{\pgfqpoint{2.405437in}{0.788426in}}%
\pgfpathclose%
\pgfusepath{fill}%
\end{pgfscope}%
\begin{pgfscope}%
\pgfpathrectangle{\pgfqpoint{0.100000in}{0.100000in}}{\pgfqpoint{3.007045in}{1.925000in}}%
\pgfusepath{clip}%
\pgfsetbuttcap%
\pgfsetmiterjoin%
\definecolor{currentfill}{rgb}{0.787543,0.866205,0.940946}%
\pgfsetfillcolor{currentfill}%
\pgfsetlinewidth{0.000000pt}%
\definecolor{currentstroke}{rgb}{0.000000,0.000000,0.000000}%
\pgfsetstrokecolor{currentstroke}%
\pgfsetstrokeopacity{0.000000}%
\pgfsetdash{}{0pt}%
\pgfpathmoveto{\pgfqpoint{2.418790in}{0.913875in}}%
\pgfpathlineto{\pgfqpoint{2.413670in}{0.911416in}}%
\pgfpathlineto{\pgfqpoint{2.410454in}{0.917486in}}%
\pgfpathlineto{\pgfqpoint{2.406157in}{0.920864in}}%
\pgfpathlineto{\pgfqpoint{2.398481in}{0.931422in}}%
\pgfpathlineto{\pgfqpoint{2.392521in}{0.928792in}}%
\pgfpathlineto{\pgfqpoint{2.381105in}{0.925986in}}%
\pgfpathlineto{\pgfqpoint{2.379773in}{0.923563in}}%
\pgfpathlineto{\pgfqpoint{2.372864in}{0.923192in}}%
\pgfpathlineto{\pgfqpoint{2.366318in}{0.919795in}}%
\pgfpathlineto{\pgfqpoint{2.361956in}{0.924269in}}%
\pgfpathlineto{\pgfqpoint{2.361650in}{0.931386in}}%
\pgfpathlineto{\pgfqpoint{2.364223in}{0.935952in}}%
\pgfpathlineto{\pgfqpoint{2.373522in}{0.943638in}}%
\pgfpathlineto{\pgfqpoint{2.379259in}{0.944820in}}%
\pgfpathlineto{\pgfqpoint{2.388077in}{0.945524in}}%
\pgfpathlineto{\pgfqpoint{2.399943in}{0.955887in}}%
\pgfpathlineto{\pgfqpoint{2.403998in}{0.954773in}}%
\pgfpathlineto{\pgfqpoint{2.405701in}{0.945045in}}%
\pgfpathlineto{\pgfqpoint{2.420735in}{0.932481in}}%
\pgfpathlineto{\pgfqpoint{2.421150in}{0.927599in}}%
\pgfpathlineto{\pgfqpoint{2.416370in}{0.915795in}}%
\pgfpathlineto{\pgfqpoint{2.418790in}{0.913875in}}%
\pgfpathclose%
\pgfusepath{fill}%
\end{pgfscope}%
\begin{pgfscope}%
\pgfpathrectangle{\pgfqpoint{0.100000in}{0.100000in}}{\pgfqpoint{3.007045in}{1.925000in}}%
\pgfusepath{clip}%
\pgfsetbuttcap%
\pgfsetmiterjoin%
\definecolor{currentfill}{rgb}{0.454118,0.701300,0.846659}%
\pgfsetfillcolor{currentfill}%
\pgfsetlinewidth{0.000000pt}%
\definecolor{currentstroke}{rgb}{0.000000,0.000000,0.000000}%
\pgfsetstrokecolor{currentstroke}%
\pgfsetstrokeopacity{0.000000}%
\pgfsetdash{}{0pt}%
\pgfpathmoveto{\pgfqpoint{2.044679in}{1.394919in}}%
\pgfpathlineto{\pgfqpoint{2.046233in}{1.372218in}}%
\pgfpathlineto{\pgfqpoint{2.018995in}{1.370976in}}%
\pgfpathlineto{\pgfqpoint{1.984280in}{1.369339in}}%
\pgfpathlineto{\pgfqpoint{1.980017in}{1.377888in}}%
\pgfpathlineto{\pgfqpoint{1.969155in}{1.380236in}}%
\pgfpathlineto{\pgfqpoint{1.963562in}{1.383583in}}%
\pgfpathlineto{\pgfqpoint{1.960884in}{1.393037in}}%
\pgfpathlineto{\pgfqpoint{1.958754in}{1.394373in}}%
\pgfpathlineto{\pgfqpoint{1.956575in}{1.409480in}}%
\pgfpathlineto{\pgfqpoint{1.961879in}{1.417685in}}%
\pgfpathlineto{\pgfqpoint{1.954703in}{1.423577in}}%
\pgfpathlineto{\pgfqpoint{1.954322in}{1.428443in}}%
\pgfpathlineto{\pgfqpoint{1.980060in}{1.429597in}}%
\pgfpathlineto{\pgfqpoint{1.980936in}{1.413027in}}%
\pgfpathlineto{\pgfqpoint{1.986647in}{1.415927in}}%
\pgfpathlineto{\pgfqpoint{1.992238in}{1.415509in}}%
\pgfpathlineto{\pgfqpoint{1.993663in}{1.389926in}}%
\pgfpathlineto{\pgfqpoint{2.022119in}{1.391513in}}%
\pgfpathlineto{\pgfqpoint{2.021967in}{1.394387in}}%
\pgfpathlineto{\pgfqpoint{2.044679in}{1.394919in}}%
\pgfpathclose%
\pgfusepath{fill}%
\end{pgfscope}%
\begin{pgfscope}%
\pgfpathrectangle{\pgfqpoint{0.100000in}{0.100000in}}{\pgfqpoint{3.007045in}{1.925000in}}%
\pgfusepath{clip}%
\pgfsetbuttcap%
\pgfsetmiterjoin%
\definecolor{currentfill}{rgb}{0.381776,0.656517,0.824452}%
\pgfsetfillcolor{currentfill}%
\pgfsetlinewidth{0.000000pt}%
\definecolor{currentstroke}{rgb}{0.000000,0.000000,0.000000}%
\pgfsetstrokecolor{currentstroke}%
\pgfsetstrokeopacity{0.000000}%
\pgfsetdash{}{0pt}%
\pgfpathmoveto{\pgfqpoint{1.923472in}{0.954935in}}%
\pgfpathlineto{\pgfqpoint{1.923336in}{0.970590in}}%
\pgfpathlineto{\pgfqpoint{1.924925in}{0.970656in}}%
\pgfpathlineto{\pgfqpoint{1.924728in}{0.990132in}}%
\pgfpathlineto{\pgfqpoint{1.925420in}{1.007434in}}%
\pgfpathlineto{\pgfqpoint{1.916864in}{1.007392in}}%
\pgfpathlineto{\pgfqpoint{1.915878in}{1.043286in}}%
\pgfpathlineto{\pgfqpoint{1.941697in}{1.043763in}}%
\pgfpathlineto{\pgfqpoint{1.942096in}{1.032256in}}%
\pgfpathlineto{\pgfqpoint{1.972430in}{1.032823in}}%
\pgfpathlineto{\pgfqpoint{1.972674in}{1.026092in}}%
\pgfpathlineto{\pgfqpoint{1.975616in}{1.018755in}}%
\pgfpathlineto{\pgfqpoint{1.981501in}{1.015319in}}%
\pgfpathlineto{\pgfqpoint{1.981594in}{1.012166in}}%
\pgfpathlineto{\pgfqpoint{1.971148in}{1.011094in}}%
\pgfpathlineto{\pgfqpoint{1.971458in}{0.993921in}}%
\pgfpathlineto{\pgfqpoint{1.977205in}{0.994031in}}%
\pgfpathlineto{\pgfqpoint{1.977561in}{0.972537in}}%
\pgfpathlineto{\pgfqpoint{1.962703in}{0.971887in}}%
\pgfpathlineto{\pgfqpoint{1.960431in}{0.971833in}}%
\pgfpathlineto{\pgfqpoint{1.960613in}{0.961138in}}%
\pgfpathlineto{\pgfqpoint{1.954694in}{0.961038in}}%
\pgfpathlineto{\pgfqpoint{1.954901in}{0.955275in}}%
\pgfpathlineto{\pgfqpoint{1.923472in}{0.954935in}}%
\pgfpathclose%
\pgfusepath{fill}%
\end{pgfscope}%
\begin{pgfscope}%
\pgfpathrectangle{\pgfqpoint{0.100000in}{0.100000in}}{\pgfqpoint{3.007045in}{1.925000in}}%
\pgfusepath{clip}%
\pgfsetbuttcap%
\pgfsetmiterjoin%
\definecolor{currentfill}{rgb}{0.366644,0.646182,0.818547}%
\pgfsetfillcolor{currentfill}%
\pgfsetlinewidth{0.000000pt}%
\definecolor{currentstroke}{rgb}{0.000000,0.000000,0.000000}%
\pgfsetstrokecolor{currentstroke}%
\pgfsetstrokeopacity{0.000000}%
\pgfsetdash{}{0pt}%
\pgfpathmoveto{\pgfqpoint{1.018543in}{1.560865in}}%
\pgfpathlineto{\pgfqpoint{1.020483in}{1.573439in}}%
\pgfpathlineto{\pgfqpoint{1.038445in}{1.570273in}}%
\pgfpathlineto{\pgfqpoint{1.037496in}{1.564829in}}%
\pgfpathlineto{\pgfqpoint{1.051087in}{1.562499in}}%
\pgfpathlineto{\pgfqpoint{1.054900in}{1.557314in}}%
\pgfpathlineto{\pgfqpoint{1.053058in}{1.551890in}}%
\pgfpathlineto{\pgfqpoint{1.057765in}{1.542345in}}%
\pgfpathlineto{\pgfqpoint{1.058870in}{1.534775in}}%
\pgfpathlineto{\pgfqpoint{1.061138in}{1.530583in}}%
\pgfpathlineto{\pgfqpoint{1.059788in}{1.522455in}}%
\pgfpathlineto{\pgfqpoint{1.053944in}{1.487032in}}%
\pgfpathlineto{\pgfqpoint{1.052724in}{1.481408in}}%
\pgfpathlineto{\pgfqpoint{1.039144in}{1.483711in}}%
\pgfpathlineto{\pgfqpoint{1.038193in}{1.478061in}}%
\pgfpathlineto{\pgfqpoint{1.027092in}{1.480000in}}%
\pgfpathlineto{\pgfqpoint{1.026443in}{1.476331in}}%
\pgfpathlineto{\pgfqpoint{1.015288in}{1.478274in}}%
\pgfpathlineto{\pgfqpoint{1.016022in}{1.482462in}}%
\pgfpathlineto{\pgfqpoint{1.005297in}{1.485446in}}%
\pgfpathlineto{\pgfqpoint{1.007407in}{1.497511in}}%
\pgfpathlineto{\pgfqpoint{1.000614in}{1.501315in}}%
\pgfpathlineto{\pgfqpoint{0.999466in}{1.507030in}}%
\pgfpathlineto{\pgfqpoint{0.992126in}{1.508384in}}%
\pgfpathlineto{\pgfqpoint{0.995747in}{1.527955in}}%
\pgfpathlineto{\pgfqpoint{1.005109in}{1.526977in}}%
\pgfpathlineto{\pgfqpoint{1.012932in}{1.528900in}}%
\pgfpathlineto{\pgfqpoint{1.018543in}{1.560865in}}%
\pgfpathclose%
\pgfusepath{fill}%
\end{pgfscope}%
\begin{pgfscope}%
\pgfpathrectangle{\pgfqpoint{0.100000in}{0.100000in}}{\pgfqpoint{3.007045in}{1.925000in}}%
\pgfusepath{clip}%
\pgfsetbuttcap%
\pgfsetmiterjoin%
\definecolor{currentfill}{rgb}{0.326290,0.618624,0.802799}%
\pgfsetfillcolor{currentfill}%
\pgfsetlinewidth{0.000000pt}%
\definecolor{currentstroke}{rgb}{0.000000,0.000000,0.000000}%
\pgfsetstrokecolor{currentstroke}%
\pgfsetstrokeopacity{0.000000}%
\pgfsetdash{}{0pt}%
\pgfpathmoveto{\pgfqpoint{1.578119in}{0.708542in}}%
\pgfpathlineto{\pgfqpoint{1.577602in}{0.694365in}}%
\pgfpathlineto{\pgfqpoint{1.551784in}{0.680740in}}%
\pgfpathlineto{\pgfqpoint{1.507461in}{0.683119in}}%
\pgfpathlineto{\pgfqpoint{1.482882in}{0.684491in}}%
\pgfpathlineto{\pgfqpoint{1.486605in}{0.742533in}}%
\pgfpathlineto{\pgfqpoint{1.523941in}{0.740360in}}%
\pgfpathlineto{\pgfqpoint{1.573659in}{0.737980in}}%
\pgfpathlineto{\pgfqpoint{1.572479in}{0.708902in}}%
\pgfpathlineto{\pgfqpoint{1.578119in}{0.708542in}}%
\pgfpathclose%
\pgfusepath{fill}%
\end{pgfscope}%
\begin{pgfscope}%
\pgfpathrectangle{\pgfqpoint{0.100000in}{0.100000in}}{\pgfqpoint{3.007045in}{1.925000in}}%
\pgfusepath{clip}%
\pgfsetbuttcap%
\pgfsetmiterjoin%
\definecolor{currentfill}{rgb}{0.142607,0.456332,0.716601}%
\pgfsetfillcolor{currentfill}%
\pgfsetlinewidth{0.000000pt}%
\definecolor{currentstroke}{rgb}{0.000000,0.000000,0.000000}%
\pgfsetstrokecolor{currentstroke}%
\pgfsetstrokeopacity{0.000000}%
\pgfsetdash{}{0pt}%
\pgfpathmoveto{\pgfqpoint{0.936407in}{1.175969in}}%
\pgfpathlineto{\pgfqpoint{0.936286in}{1.175309in}}%
\pgfpathlineto{\pgfqpoint{0.968213in}{1.169632in}}%
\pgfpathlineto{\pgfqpoint{1.001312in}{1.163987in}}%
\pgfpathlineto{\pgfqpoint{1.002930in}{1.149720in}}%
\pgfpathlineto{\pgfqpoint{1.005017in}{1.143547in}}%
\pgfpathlineto{\pgfqpoint{1.002522in}{1.140541in}}%
\pgfpathlineto{\pgfqpoint{0.963729in}{1.147104in}}%
\pgfpathlineto{\pgfqpoint{0.904834in}{1.157752in}}%
\pgfpathlineto{\pgfqpoint{0.907847in}{1.175524in}}%
\pgfpathlineto{\pgfqpoint{0.913573in}{1.180156in}}%
\pgfpathlineto{\pgfqpoint{0.936407in}{1.175969in}}%
\pgfpathclose%
\pgfusepath{fill}%
\end{pgfscope}%
\begin{pgfscope}%
\pgfpathrectangle{\pgfqpoint{0.100000in}{0.100000in}}{\pgfqpoint{3.007045in}{1.925000in}}%
\pgfusepath{clip}%
\pgfsetbuttcap%
\pgfsetmiterjoin%
\definecolor{currentfill}{rgb}{0.301069,0.601399,0.792957}%
\pgfsetfillcolor{currentfill}%
\pgfsetlinewidth{0.000000pt}%
\definecolor{currentstroke}{rgb}{0.000000,0.000000,0.000000}%
\pgfsetstrokecolor{currentstroke}%
\pgfsetstrokeopacity{0.000000}%
\pgfsetdash{}{0pt}%
\pgfpathmoveto{\pgfqpoint{1.786299in}{0.783647in}}%
\pgfpathlineto{\pgfqpoint{1.800644in}{0.779398in}}%
\pgfpathlineto{\pgfqpoint{1.801001in}{0.799445in}}%
\pgfpathlineto{\pgfqpoint{1.806094in}{0.799451in}}%
\pgfpathlineto{\pgfqpoint{1.814189in}{0.790353in}}%
\pgfpathlineto{\pgfqpoint{1.825054in}{0.789133in}}%
\pgfpathlineto{\pgfqpoint{1.829586in}{0.787129in}}%
\pgfpathlineto{\pgfqpoint{1.829258in}{0.781767in}}%
\pgfpathlineto{\pgfqpoint{1.836570in}{0.777705in}}%
\pgfpathlineto{\pgfqpoint{1.842812in}{0.769497in}}%
\pgfpathlineto{\pgfqpoint{1.842406in}{0.765728in}}%
\pgfpathlineto{\pgfqpoint{1.846121in}{0.759629in}}%
\pgfpathlineto{\pgfqpoint{1.841164in}{0.748497in}}%
\pgfpathlineto{\pgfqpoint{1.836073in}{0.747122in}}%
\pgfpathlineto{\pgfqpoint{1.835563in}{0.743537in}}%
\pgfpathlineto{\pgfqpoint{1.838686in}{0.738812in}}%
\pgfpathlineto{\pgfqpoint{1.825469in}{0.738647in}}%
\pgfpathlineto{\pgfqpoint{1.825570in}{0.729490in}}%
\pgfpathlineto{\pgfqpoint{1.788689in}{0.729118in}}%
\pgfpathlineto{\pgfqpoint{1.787896in}{0.730793in}}%
\pgfpathlineto{\pgfqpoint{1.763868in}{0.730641in}}%
\pgfpathlineto{\pgfqpoint{1.762705in}{0.734578in}}%
\pgfpathlineto{\pgfqpoint{1.755158in}{0.734645in}}%
\pgfpathlineto{\pgfqpoint{1.755274in}{0.762070in}}%
\pgfpathlineto{\pgfqpoint{1.767169in}{0.763420in}}%
\pgfpathlineto{\pgfqpoint{1.779169in}{0.763215in}}%
\pgfpathlineto{\pgfqpoint{1.786323in}{0.758894in}}%
\pgfpathlineto{\pgfqpoint{1.786299in}{0.783647in}}%
\pgfpathclose%
\pgfusepath{fill}%
\end{pgfscope}%
\begin{pgfscope}%
\pgfpathrectangle{\pgfqpoint{0.100000in}{0.100000in}}{\pgfqpoint{3.007045in}{1.925000in}}%
\pgfusepath{clip}%
\pgfsetbuttcap%
\pgfsetmiterjoin%
\definecolor{currentfill}{rgb}{0.750634,0.847843,0.928212}%
\pgfsetfillcolor{currentfill}%
\pgfsetlinewidth{0.000000pt}%
\definecolor{currentstroke}{rgb}{0.000000,0.000000,0.000000}%
\pgfsetstrokecolor{currentstroke}%
\pgfsetstrokeopacity{0.000000}%
\pgfsetdash{}{0pt}%
\pgfpathmoveto{\pgfqpoint{2.517736in}{0.501110in}}%
\pgfpathlineto{\pgfqpoint{2.508454in}{0.501648in}}%
\pgfpathlineto{\pgfqpoint{2.495655in}{0.499863in}}%
\pgfpathlineto{\pgfqpoint{2.494027in}{0.512423in}}%
\pgfpathlineto{\pgfqpoint{2.486048in}{0.518807in}}%
\pgfpathlineto{\pgfqpoint{2.488304in}{0.521528in}}%
\pgfpathlineto{\pgfqpoint{2.494299in}{0.520921in}}%
\pgfpathlineto{\pgfqpoint{2.498437in}{0.523892in}}%
\pgfpathlineto{\pgfqpoint{2.496864in}{0.535136in}}%
\pgfpathlineto{\pgfqpoint{2.504526in}{0.536259in}}%
\pgfpathlineto{\pgfqpoint{2.501841in}{0.554053in}}%
\pgfpathlineto{\pgfqpoint{2.513059in}{0.555566in}}%
\pgfpathlineto{\pgfqpoint{2.513609in}{0.551975in}}%
\pgfpathlineto{\pgfqpoint{2.522149in}{0.556021in}}%
\pgfpathlineto{\pgfqpoint{2.533920in}{0.561053in}}%
\pgfpathlineto{\pgfqpoint{2.538085in}{0.559398in}}%
\pgfpathlineto{\pgfqpoint{2.541214in}{0.552000in}}%
\pgfpathlineto{\pgfqpoint{2.547864in}{0.546632in}}%
\pgfpathlineto{\pgfqpoint{2.550293in}{0.531506in}}%
\pgfpathlineto{\pgfqpoint{2.549990in}{0.525560in}}%
\pgfpathlineto{\pgfqpoint{2.512188in}{0.520102in}}%
\pgfpathlineto{\pgfqpoint{2.522009in}{0.510090in}}%
\pgfpathlineto{\pgfqpoint{2.517736in}{0.501110in}}%
\pgfpathclose%
\pgfusepath{fill}%
\end{pgfscope}%
\begin{pgfscope}%
\pgfpathrectangle{\pgfqpoint{0.100000in}{0.100000in}}{\pgfqpoint{3.007045in}{1.925000in}}%
\pgfusepath{clip}%
\pgfsetbuttcap%
\pgfsetmiterjoin%
\definecolor{currentfill}{rgb}{0.341423,0.628958,0.808704}%
\pgfsetfillcolor{currentfill}%
\pgfsetlinewidth{0.000000pt}%
\definecolor{currentstroke}{rgb}{0.000000,0.000000,0.000000}%
\pgfsetstrokecolor{currentstroke}%
\pgfsetstrokeopacity{0.000000}%
\pgfsetdash{}{0pt}%
\pgfpathmoveto{\pgfqpoint{1.685427in}{0.912299in}}%
\pgfpathlineto{\pgfqpoint{1.708380in}{0.911948in}}%
\pgfpathlineto{\pgfqpoint{1.708573in}{0.926325in}}%
\pgfpathlineto{\pgfqpoint{1.731364in}{0.926038in}}%
\pgfpathlineto{\pgfqpoint{1.731290in}{0.917416in}}%
\pgfpathlineto{\pgfqpoint{1.734135in}{0.917404in}}%
\pgfpathlineto{\pgfqpoint{1.734052in}{0.905918in}}%
\pgfpathlineto{\pgfqpoint{1.728181in}{0.905962in}}%
\pgfpathlineto{\pgfqpoint{1.728070in}{0.900172in}}%
\pgfpathlineto{\pgfqpoint{1.725208in}{0.894438in}}%
\pgfpathlineto{\pgfqpoint{1.719521in}{0.894505in}}%
\pgfpathlineto{\pgfqpoint{1.719456in}{0.888768in}}%
\pgfpathlineto{\pgfqpoint{1.694674in}{0.889113in}}%
\pgfpathlineto{\pgfqpoint{1.691799in}{0.898304in}}%
\pgfpathlineto{\pgfqpoint{1.684983in}{0.896595in}}%
\pgfpathlineto{\pgfqpoint{1.685427in}{0.912299in}}%
\pgfpathclose%
\pgfusepath{fill}%
\end{pgfscope}%
\begin{pgfscope}%
\pgfpathrectangle{\pgfqpoint{0.100000in}{0.100000in}}{\pgfqpoint{3.007045in}{1.925000in}}%
\pgfusepath{clip}%
\pgfsetbuttcap%
\pgfsetmiterjoin%
\definecolor{currentfill}{rgb}{0.840692,0.901638,0.958662}%
\pgfsetfillcolor{currentfill}%
\pgfsetlinewidth{0.000000pt}%
\definecolor{currentstroke}{rgb}{0.000000,0.000000,0.000000}%
\pgfsetstrokecolor{currentstroke}%
\pgfsetstrokeopacity{0.000000}%
\pgfsetdash{}{0pt}%
\pgfpathmoveto{\pgfqpoint{2.474213in}{0.983871in}}%
\pgfpathlineto{\pgfqpoint{2.468967in}{0.985644in}}%
\pgfpathlineto{\pgfqpoint{2.465463in}{0.979232in}}%
\pgfpathlineto{\pgfqpoint{2.462402in}{0.980829in}}%
\pgfpathlineto{\pgfqpoint{2.458453in}{0.990986in}}%
\pgfpathlineto{\pgfqpoint{2.460956in}{0.992711in}}%
\pgfpathlineto{\pgfqpoint{2.463756in}{1.002325in}}%
\pgfpathlineto{\pgfqpoint{2.458161in}{1.009811in}}%
\pgfpathlineto{\pgfqpoint{2.461449in}{1.016280in}}%
\pgfpathlineto{\pgfqpoint{2.462097in}{1.022483in}}%
\pgfpathlineto{\pgfqpoint{2.466964in}{1.026275in}}%
\pgfpathlineto{\pgfqpoint{2.476402in}{1.027509in}}%
\pgfpathlineto{\pgfqpoint{2.475027in}{1.025720in}}%
\pgfpathlineto{\pgfqpoint{2.475252in}{1.017742in}}%
\pgfpathlineto{\pgfqpoint{2.473634in}{1.012683in}}%
\pgfpathlineto{\pgfqpoint{2.479391in}{1.010461in}}%
\pgfpathlineto{\pgfqpoint{2.490105in}{1.002416in}}%
\pgfpathlineto{\pgfqpoint{2.486521in}{0.996045in}}%
\pgfpathlineto{\pgfqpoint{2.479657in}{0.995365in}}%
\pgfpathlineto{\pgfqpoint{2.474213in}{0.983871in}}%
\pgfpathclose%
\pgfusepath{fill}%
\end{pgfscope}%
\begin{pgfscope}%
\pgfpathrectangle{\pgfqpoint{0.100000in}{0.100000in}}{\pgfqpoint{3.007045in}{1.925000in}}%
\pgfusepath{clip}%
\pgfsetbuttcap%
\pgfsetmiterjoin%
\definecolor{currentfill}{rgb}{0.270804,0.580730,0.781146}%
\pgfsetfillcolor{currentfill}%
\pgfsetlinewidth{0.000000pt}%
\definecolor{currentstroke}{rgb}{0.000000,0.000000,0.000000}%
\pgfsetstrokecolor{currentstroke}%
\pgfsetstrokeopacity{0.000000}%
\pgfsetdash{}{0pt}%
\pgfpathmoveto{\pgfqpoint{1.527075in}{1.552396in}}%
\pgfpathlineto{\pgfqpoint{1.552729in}{1.550914in}}%
\pgfpathlineto{\pgfqpoint{1.559045in}{1.550570in}}%
\pgfpathlineto{\pgfqpoint{1.557925in}{1.527548in}}%
\pgfpathlineto{\pgfqpoint{1.553171in}{1.527805in}}%
\pgfpathlineto{\pgfqpoint{1.518970in}{1.529708in}}%
\pgfpathlineto{\pgfqpoint{1.518468in}{1.535998in}}%
\pgfpathlineto{\pgfqpoint{1.524818in}{1.537110in}}%
\pgfpathlineto{\pgfqpoint{1.523533in}{1.542968in}}%
\pgfpathlineto{\pgfqpoint{1.527075in}{1.552396in}}%
\pgfpathclose%
\pgfusepath{fill}%
\end{pgfscope}%
\begin{pgfscope}%
\pgfpathrectangle{\pgfqpoint{0.100000in}{0.100000in}}{\pgfqpoint{3.007045in}{1.925000in}}%
\pgfusepath{clip}%
\pgfsetbuttcap%
\pgfsetmiterjoin%
\definecolor{currentfill}{rgb}{0.031373,0.285675,0.564291}%
\pgfsetfillcolor{currentfill}%
\pgfsetlinewidth{0.000000pt}%
\definecolor{currentstroke}{rgb}{0.000000,0.000000,0.000000}%
\pgfsetstrokecolor{currentstroke}%
\pgfsetstrokeopacity{0.000000}%
\pgfsetdash{}{0pt}%
\pgfpathmoveto{\pgfqpoint{1.265329in}{0.695996in}}%
\pgfpathlineto{\pgfqpoint{1.282225in}{0.694354in}}%
\pgfpathlineto{\pgfqpoint{1.319120in}{0.690958in}}%
\pgfpathlineto{\pgfqpoint{1.319633in}{0.696676in}}%
\pgfpathlineto{\pgfqpoint{1.334479in}{0.695294in}}%
\pgfpathlineto{\pgfqpoint{1.332017in}{0.666591in}}%
\pgfpathlineto{\pgfqpoint{1.333802in}{0.666433in}}%
\pgfpathlineto{\pgfqpoint{1.331766in}{0.642740in}}%
\pgfpathlineto{\pgfqpoint{1.328596in}{0.641304in}}%
\pgfpathlineto{\pgfqpoint{1.322677in}{0.647278in}}%
\pgfpathlineto{\pgfqpoint{1.318428in}{0.649073in}}%
\pgfpathlineto{\pgfqpoint{1.282209in}{0.612003in}}%
\pgfpathlineto{\pgfqpoint{1.255027in}{0.637292in}}%
\pgfpathlineto{\pgfqpoint{1.265329in}{0.695996in}}%
\pgfpathclose%
\pgfusepath{fill}%
\end{pgfscope}%
\begin{pgfscope}%
\pgfpathrectangle{\pgfqpoint{0.100000in}{0.100000in}}{\pgfqpoint{3.007045in}{1.925000in}}%
\pgfusepath{clip}%
\pgfsetbuttcap%
\pgfsetmiterjoin%
\definecolor{currentfill}{rgb}{0.406997,0.673741,0.834295}%
\pgfsetfillcolor{currentfill}%
\pgfsetlinewidth{0.000000pt}%
\definecolor{currentstroke}{rgb}{0.000000,0.000000,0.000000}%
\pgfsetstrokecolor{currentstroke}%
\pgfsetstrokeopacity{0.000000}%
\pgfsetdash{}{0pt}%
\pgfpathmoveto{\pgfqpoint{2.742724in}{1.035580in}}%
\pgfpathlineto{\pgfqpoint{2.738199in}{1.043269in}}%
\pgfpathlineto{\pgfqpoint{2.738529in}{1.052500in}}%
\pgfpathlineto{\pgfqpoint{2.733124in}{1.055886in}}%
\pgfpathlineto{\pgfqpoint{2.724911in}{1.056794in}}%
\pgfpathlineto{\pgfqpoint{2.723871in}{1.066185in}}%
\pgfpathlineto{\pgfqpoint{2.723519in}{1.072852in}}%
\pgfpathlineto{\pgfqpoint{2.738779in}{1.094235in}}%
\pgfpathlineto{\pgfqpoint{2.741244in}{1.095623in}}%
\pgfpathlineto{\pgfqpoint{2.746605in}{1.094479in}}%
\pgfpathlineto{\pgfqpoint{2.750900in}{1.100328in}}%
\pgfpathlineto{\pgfqpoint{2.761478in}{1.098464in}}%
\pgfpathlineto{\pgfqpoint{2.766403in}{1.100448in}}%
\pgfpathlineto{\pgfqpoint{2.771017in}{1.090586in}}%
\pgfpathlineto{\pgfqpoint{2.777353in}{1.079986in}}%
\pgfpathlineto{\pgfqpoint{2.787444in}{1.058403in}}%
\pgfpathlineto{\pgfqpoint{2.786756in}{1.058061in}}%
\pgfpathlineto{\pgfqpoint{2.780444in}{1.069950in}}%
\pgfpathlineto{\pgfqpoint{2.777839in}{1.065735in}}%
\pgfpathlineto{\pgfqpoint{2.788743in}{1.048436in}}%
\pgfpathlineto{\pgfqpoint{2.782567in}{1.052153in}}%
\pgfpathlineto{\pgfqpoint{2.776592in}{1.052297in}}%
\pgfpathlineto{\pgfqpoint{2.766420in}{1.044865in}}%
\pgfpathlineto{\pgfqpoint{2.757806in}{1.041934in}}%
\pgfpathlineto{\pgfqpoint{2.752708in}{1.036350in}}%
\pgfpathlineto{\pgfqpoint{2.742724in}{1.035580in}}%
\pgfpathclose%
\pgfusepath{fill}%
\end{pgfscope}%
\begin{pgfscope}%
\pgfpathrectangle{\pgfqpoint{0.100000in}{0.100000in}}{\pgfqpoint{3.007045in}{1.925000in}}%
\pgfusepath{clip}%
\pgfsetbuttcap%
\pgfsetmiterjoin%
\definecolor{currentfill}{rgb}{0.529412,0.742637,0.862899}%
\pgfsetfillcolor{currentfill}%
\pgfsetlinewidth{0.000000pt}%
\definecolor{currentstroke}{rgb}{0.000000,0.000000,0.000000}%
\pgfsetstrokecolor{currentstroke}%
\pgfsetstrokeopacity{0.000000}%
\pgfsetdash{}{0pt}%
\pgfpathmoveto{\pgfqpoint{1.685760in}{0.932254in}}%
\pgfpathlineto{\pgfqpoint{1.658027in}{0.932849in}}%
\pgfpathlineto{\pgfqpoint{1.646710in}{0.935974in}}%
\pgfpathlineto{\pgfqpoint{1.647016in}{0.947800in}}%
\pgfpathlineto{\pgfqpoint{1.641329in}{0.947962in}}%
\pgfpathlineto{\pgfqpoint{1.642047in}{0.976241in}}%
\pgfpathlineto{\pgfqpoint{1.670693in}{0.976025in}}%
\pgfpathlineto{\pgfqpoint{1.671642in}{0.972779in}}%
\pgfpathlineto{\pgfqpoint{1.664644in}{0.969981in}}%
\pgfpathlineto{\pgfqpoint{1.664393in}{0.958534in}}%
\pgfpathlineto{\pgfqpoint{1.670084in}{0.958402in}}%
\pgfpathlineto{\pgfqpoint{1.669953in}{0.952684in}}%
\pgfpathlineto{\pgfqpoint{1.675597in}{0.952567in}}%
\pgfpathlineto{\pgfqpoint{1.675476in}{0.946791in}}%
\pgfpathlineto{\pgfqpoint{1.685998in}{0.946658in}}%
\pgfpathlineto{\pgfqpoint{1.685760in}{0.932254in}}%
\pgfpathclose%
\pgfusepath{fill}%
\end{pgfscope}%
\begin{pgfscope}%
\pgfpathrectangle{\pgfqpoint{0.100000in}{0.100000in}}{\pgfqpoint{3.007045in}{1.925000in}}%
\pgfusepath{clip}%
\pgfsetbuttcap%
\pgfsetmiterjoin%
\definecolor{currentfill}{rgb}{0.187266,0.500992,0.739608}%
\pgfsetfillcolor{currentfill}%
\pgfsetlinewidth{0.000000pt}%
\definecolor{currentstroke}{rgb}{0.000000,0.000000,0.000000}%
\pgfsetstrokecolor{currentstroke}%
\pgfsetstrokeopacity{0.000000}%
\pgfsetdash{}{0pt}%
\pgfpathmoveto{\pgfqpoint{1.564548in}{1.295579in}}%
\pgfpathlineto{\pgfqpoint{1.563271in}{1.272731in}}%
\pgfpathlineto{\pgfqpoint{1.552431in}{1.273213in}}%
\pgfpathlineto{\pgfqpoint{1.512812in}{1.275327in}}%
\pgfpathlineto{\pgfqpoint{1.514112in}{1.321020in}}%
\pgfpathlineto{\pgfqpoint{1.485080in}{1.322839in}}%
\pgfpathlineto{\pgfqpoint{1.486374in}{1.345738in}}%
\pgfpathlineto{\pgfqpoint{1.519516in}{1.343574in}}%
\pgfpathlineto{\pgfqpoint{1.543010in}{1.342313in}}%
\pgfpathlineto{\pgfqpoint{1.541840in}{1.319542in}}%
\pgfpathlineto{\pgfqpoint{1.565174in}{1.318418in}}%
\pgfpathlineto{\pgfqpoint{1.564548in}{1.295579in}}%
\pgfpathclose%
\pgfusepath{fill}%
\end{pgfscope}%
\begin{pgfscope}%
\pgfpathrectangle{\pgfqpoint{0.100000in}{0.100000in}}{\pgfqpoint{3.007045in}{1.925000in}}%
\pgfusepath{clip}%
\pgfsetbuttcap%
\pgfsetmiterjoin%
\definecolor{currentfill}{rgb}{0.265759,0.577286,0.779177}%
\pgfsetfillcolor{currentfill}%
\pgfsetlinewidth{0.000000pt}%
\definecolor{currentstroke}{rgb}{0.000000,0.000000,0.000000}%
\pgfsetstrokecolor{currentstroke}%
\pgfsetstrokeopacity{0.000000}%
\pgfsetdash{}{0pt}%
\pgfpathmoveto{\pgfqpoint{2.025623in}{0.904867in}}%
\pgfpathlineto{\pgfqpoint{2.008210in}{0.904668in}}%
\pgfpathlineto{\pgfqpoint{1.985207in}{0.903228in}}%
\pgfpathlineto{\pgfqpoint{1.984303in}{0.932387in}}%
\pgfpathlineto{\pgfqpoint{1.975423in}{0.932334in}}%
\pgfpathlineto{\pgfqpoint{1.966949in}{0.932026in}}%
\pgfpathlineto{\pgfqpoint{1.966442in}{0.954472in}}%
\pgfpathlineto{\pgfqpoint{1.971212in}{0.955492in}}%
\pgfpathlineto{\pgfqpoint{1.969967in}{0.964093in}}%
\pgfpathlineto{\pgfqpoint{1.962703in}{0.971887in}}%
\pgfpathlineto{\pgfqpoint{1.977561in}{0.972537in}}%
\pgfpathlineto{\pgfqpoint{2.025516in}{0.974832in}}%
\pgfpathlineto{\pgfqpoint{2.034251in}{0.967797in}}%
\pgfpathlineto{\pgfqpoint{2.033532in}{0.960342in}}%
\pgfpathlineto{\pgfqpoint{2.026615in}{0.954215in}}%
\pgfpathlineto{\pgfqpoint{2.021688in}{0.947698in}}%
\pgfpathlineto{\pgfqpoint{2.023624in}{0.941619in}}%
\pgfpathlineto{\pgfqpoint{2.025623in}{0.904867in}}%
\pgfpathclose%
\pgfusepath{fill}%
\end{pgfscope}%
\begin{pgfscope}%
\pgfpathrectangle{\pgfqpoint{0.100000in}{0.100000in}}{\pgfqpoint{3.007045in}{1.925000in}}%
\pgfusepath{clip}%
\pgfsetbuttcap%
\pgfsetmiterjoin%
\definecolor{currentfill}{rgb}{0.479216,0.715079,0.852072}%
\pgfsetfillcolor{currentfill}%
\pgfsetlinewidth{0.000000pt}%
\definecolor{currentstroke}{rgb}{0.000000,0.000000,0.000000}%
\pgfsetstrokecolor{currentstroke}%
\pgfsetstrokeopacity{0.000000}%
\pgfsetdash{}{0pt}%
\pgfpathmoveto{\pgfqpoint{1.169817in}{1.038957in}}%
\pgfpathlineto{\pgfqpoint{1.161246in}{1.055657in}}%
\pgfpathlineto{\pgfqpoint{1.162663in}{1.066636in}}%
\pgfpathlineto{\pgfqpoint{1.139253in}{1.069444in}}%
\pgfpathlineto{\pgfqpoint{1.139523in}{1.071438in}}%
\pgfpathlineto{\pgfqpoint{1.140887in}{1.088045in}}%
\pgfpathlineto{\pgfqpoint{1.143557in}{1.105192in}}%
\pgfpathlineto{\pgfqpoint{1.150792in}{1.105369in}}%
\pgfpathlineto{\pgfqpoint{1.154865in}{1.136033in}}%
\pgfpathlineto{\pgfqpoint{1.193414in}{1.130886in}}%
\pgfpathlineto{\pgfqpoint{1.202180in}{1.129872in}}%
\pgfpathlineto{\pgfqpoint{1.205700in}{1.130953in}}%
\pgfpathlineto{\pgfqpoint{1.220511in}{1.109227in}}%
\pgfpathlineto{\pgfqpoint{1.223896in}{1.096979in}}%
\pgfpathlineto{\pgfqpoint{1.229068in}{1.091653in}}%
\pgfpathlineto{\pgfqpoint{1.231022in}{1.088493in}}%
\pgfpathlineto{\pgfqpoint{1.225618in}{1.073383in}}%
\pgfpathlineto{\pgfqpoint{1.236457in}{1.074727in}}%
\pgfpathlineto{\pgfqpoint{1.241493in}{1.067690in}}%
\pgfpathlineto{\pgfqpoint{1.242720in}{1.057254in}}%
\pgfpathlineto{\pgfqpoint{1.241118in}{1.050301in}}%
\pgfpathlineto{\pgfqpoint{1.238883in}{1.030797in}}%
\pgfpathlineto{\pgfqpoint{1.235453in}{1.031177in}}%
\pgfpathlineto{\pgfqpoint{1.169817in}{1.038957in}}%
\pgfpathclose%
\pgfusepath{fill}%
\end{pgfscope}%
\begin{pgfscope}%
\pgfpathrectangle{\pgfqpoint{0.100000in}{0.100000in}}{\pgfqpoint{3.007045in}{1.925000in}}%
\pgfusepath{clip}%
\pgfsetbuttcap%
\pgfsetmiterjoin%
\definecolor{currentfill}{rgb}{0.535686,0.746082,0.864252}%
\pgfsetfillcolor{currentfill}%
\pgfsetlinewidth{0.000000pt}%
\definecolor{currentstroke}{rgb}{0.000000,0.000000,0.000000}%
\pgfsetstrokecolor{currentstroke}%
\pgfsetstrokeopacity{0.000000}%
\pgfsetdash{}{0pt}%
\pgfpathmoveto{\pgfqpoint{1.853670in}{1.237862in}}%
\pgfpathlineto{\pgfqpoint{1.844539in}{1.237709in}}%
\pgfpathlineto{\pgfqpoint{1.843908in}{1.275998in}}%
\pgfpathlineto{\pgfqpoint{1.855298in}{1.276147in}}%
\pgfpathlineto{\pgfqpoint{1.866638in}{1.276378in}}%
\pgfpathlineto{\pgfqpoint{1.867080in}{1.259096in}}%
\pgfpathlineto{\pgfqpoint{1.889907in}{1.259660in}}%
\pgfpathlineto{\pgfqpoint{1.890498in}{1.239370in}}%
\pgfpathlineto{\pgfqpoint{1.886676in}{1.239179in}}%
\pgfpathlineto{\pgfqpoint{1.853670in}{1.237862in}}%
\pgfpathclose%
\pgfusepath{fill}%
\end{pgfscope}%
\begin{pgfscope}%
\pgfpathrectangle{\pgfqpoint{0.100000in}{0.100000in}}{\pgfqpoint{3.007045in}{1.925000in}}%
\pgfusepath{clip}%
\pgfsetbuttcap%
\pgfsetmiterjoin%
\definecolor{currentfill}{rgb}{0.435294,0.690965,0.842599}%
\pgfsetfillcolor{currentfill}%
\pgfsetlinewidth{0.000000pt}%
\definecolor{currentstroke}{rgb}{0.000000,0.000000,0.000000}%
\pgfsetstrokecolor{currentstroke}%
\pgfsetstrokeopacity{0.000000}%
\pgfsetdash{}{0pt}%
\pgfpathmoveto{\pgfqpoint{1.638148in}{0.525132in}}%
\pgfpathlineto{\pgfqpoint{1.621659in}{0.515303in}}%
\pgfpathlineto{\pgfqpoint{1.619645in}{0.516644in}}%
\pgfpathlineto{\pgfqpoint{1.615672in}{0.518019in}}%
\pgfpathlineto{\pgfqpoint{1.606065in}{0.530962in}}%
\pgfpathlineto{\pgfqpoint{1.598646in}{0.524138in}}%
\pgfpathlineto{\pgfqpoint{1.597051in}{0.530605in}}%
\pgfpathlineto{\pgfqpoint{1.582069in}{0.543767in}}%
\pgfpathlineto{\pgfqpoint{1.575104in}{0.537324in}}%
\pgfpathlineto{\pgfqpoint{1.565594in}{0.551126in}}%
\pgfpathlineto{\pgfqpoint{1.566327in}{0.575160in}}%
\pgfpathlineto{\pgfqpoint{1.576662in}{0.574910in}}%
\pgfpathlineto{\pgfqpoint{1.592881in}{0.569267in}}%
\pgfpathlineto{\pgfqpoint{1.593576in}{0.574175in}}%
\pgfpathlineto{\pgfqpoint{1.602998in}{0.592878in}}%
\pgfpathlineto{\pgfqpoint{1.610932in}{0.600948in}}%
\pgfpathlineto{\pgfqpoint{1.622469in}{0.597909in}}%
\pgfpathlineto{\pgfqpoint{1.639929in}{0.589583in}}%
\pgfpathlineto{\pgfqpoint{1.643382in}{0.598629in}}%
\pgfpathlineto{\pgfqpoint{1.654308in}{0.604797in}}%
\pgfpathlineto{\pgfqpoint{1.668209in}{0.612458in}}%
\pgfpathlineto{\pgfqpoint{1.670600in}{0.609460in}}%
\pgfpathlineto{\pgfqpoint{1.672091in}{0.600546in}}%
\pgfpathlineto{\pgfqpoint{1.678179in}{0.593593in}}%
\pgfpathlineto{\pgfqpoint{1.679583in}{0.587280in}}%
\pgfpathlineto{\pgfqpoint{1.659758in}{0.576115in}}%
\pgfpathlineto{\pgfqpoint{1.662849in}{0.573743in}}%
\pgfpathlineto{\pgfqpoint{1.665066in}{0.567789in}}%
\pgfpathlineto{\pgfqpoint{1.672046in}{0.560070in}}%
\pgfpathlineto{\pgfqpoint{1.677967in}{0.558409in}}%
\pgfpathlineto{\pgfqpoint{1.672597in}{0.554943in}}%
\pgfpathlineto{\pgfqpoint{1.668934in}{0.549504in}}%
\pgfpathlineto{\pgfqpoint{1.665918in}{0.546597in}}%
\pgfpathlineto{\pgfqpoint{1.655528in}{0.542512in}}%
\pgfpathlineto{\pgfqpoint{1.638148in}{0.525132in}}%
\pgfpathclose%
\pgfusepath{fill}%
\end{pgfscope}%
\begin{pgfscope}%
\pgfpathrectangle{\pgfqpoint{0.100000in}{0.100000in}}{\pgfqpoint{3.007045in}{1.925000in}}%
\pgfusepath{clip}%
\pgfsetbuttcap%
\pgfsetmiterjoin%
\definecolor{currentfill}{rgb}{0.311157,0.608289,0.796894}%
\pgfsetfillcolor{currentfill}%
\pgfsetlinewidth{0.000000pt}%
\definecolor{currentstroke}{rgb}{0.000000,0.000000,0.000000}%
\pgfsetstrokecolor{currentstroke}%
\pgfsetstrokeopacity{0.000000}%
\pgfsetdash{}{0pt}%
\pgfpathmoveto{\pgfqpoint{1.923673in}{0.668219in}}%
\pgfpathlineto{\pgfqpoint{1.922946in}{0.694307in}}%
\pgfpathlineto{\pgfqpoint{1.916979in}{0.699909in}}%
\pgfpathlineto{\pgfqpoint{1.916505in}{0.717286in}}%
\pgfpathlineto{\pgfqpoint{1.910744in}{0.717166in}}%
\pgfpathlineto{\pgfqpoint{1.906378in}{0.722687in}}%
\pgfpathlineto{\pgfqpoint{1.899049in}{0.722750in}}%
\pgfpathlineto{\pgfqpoint{1.898651in}{0.739626in}}%
\pgfpathlineto{\pgfqpoint{1.935072in}{0.740260in}}%
\pgfpathlineto{\pgfqpoint{1.968862in}{0.741315in}}%
\pgfpathlineto{\pgfqpoint{1.970228in}{0.741380in}}%
\pgfpathlineto{\pgfqpoint{1.967255in}{0.733621in}}%
\pgfpathlineto{\pgfqpoint{1.964704in}{0.732579in}}%
\pgfpathlineto{\pgfqpoint{1.959700in}{0.720849in}}%
\pgfpathlineto{\pgfqpoint{1.962387in}{0.712972in}}%
\pgfpathlineto{\pgfqpoint{1.970799in}{0.713279in}}%
\pgfpathlineto{\pgfqpoint{1.970224in}{0.710328in}}%
\pgfpathlineto{\pgfqpoint{1.968808in}{0.707431in}}%
\pgfpathlineto{\pgfqpoint{1.970011in}{0.695703in}}%
\pgfpathlineto{\pgfqpoint{1.966832in}{0.693993in}}%
\pgfpathlineto{\pgfqpoint{1.969121in}{0.688091in}}%
\pgfpathlineto{\pgfqpoint{1.969490in}{0.679240in}}%
\pgfpathlineto{\pgfqpoint{1.965268in}{0.666642in}}%
\pgfpathlineto{\pgfqpoint{1.963857in}{0.672381in}}%
\pgfpathlineto{\pgfqpoint{1.953769in}{0.665525in}}%
\pgfpathlineto{\pgfqpoint{1.949500in}{0.671101in}}%
\pgfpathlineto{\pgfqpoint{1.946656in}{0.668963in}}%
\pgfpathlineto{\pgfqpoint{1.923673in}{0.668219in}}%
\pgfpathclose%
\pgfusepath{fill}%
\end{pgfscope}%
\begin{pgfscope}%
\pgfpathrectangle{\pgfqpoint{0.100000in}{0.100000in}}{\pgfqpoint{3.007045in}{1.925000in}}%
\pgfusepath{clip}%
\pgfsetbuttcap%
\pgfsetmiterjoin%
\definecolor{currentfill}{rgb}{0.765398,0.854118,0.933379}%
\pgfsetfillcolor{currentfill}%
\pgfsetlinewidth{0.000000pt}%
\definecolor{currentstroke}{rgb}{0.000000,0.000000,0.000000}%
\pgfsetstrokecolor{currentstroke}%
\pgfsetstrokeopacity{0.000000}%
\pgfsetdash{}{0pt}%
\pgfpathmoveto{\pgfqpoint{2.192935in}{1.614287in}}%
\pgfpathlineto{\pgfqpoint{2.174903in}{1.611101in}}%
\pgfpathlineto{\pgfqpoint{2.172417in}{1.608214in}}%
\pgfpathlineto{\pgfqpoint{2.171847in}{1.600827in}}%
\pgfpathlineto{\pgfqpoint{2.163982in}{1.596790in}}%
\pgfpathlineto{\pgfqpoint{2.160048in}{1.590165in}}%
\pgfpathlineto{\pgfqpoint{2.155554in}{1.591476in}}%
\pgfpathlineto{\pgfqpoint{2.160765in}{1.598791in}}%
\pgfpathlineto{\pgfqpoint{2.163198in}{1.604703in}}%
\pgfpathlineto{\pgfqpoint{2.152294in}{1.599865in}}%
\pgfpathlineto{\pgfqpoint{2.149656in}{1.593559in}}%
\pgfpathlineto{\pgfqpoint{2.144154in}{1.589747in}}%
\pgfpathlineto{\pgfqpoint{2.139521in}{1.593745in}}%
\pgfpathlineto{\pgfqpoint{2.134646in}{1.588277in}}%
\pgfpathlineto{\pgfqpoint{2.131069in}{1.580709in}}%
\pgfpathlineto{\pgfqpoint{2.128131in}{1.580457in}}%
\pgfpathlineto{\pgfqpoint{2.126263in}{1.603472in}}%
\pgfpathlineto{\pgfqpoint{2.123850in}{1.609038in}}%
\pgfpathlineto{\pgfqpoint{2.112389in}{1.608131in}}%
\pgfpathlineto{\pgfqpoint{2.111036in}{1.625398in}}%
\pgfpathlineto{\pgfqpoint{2.088086in}{1.623582in}}%
\pgfpathlineto{\pgfqpoint{2.087227in}{1.635040in}}%
\pgfpathlineto{\pgfqpoint{2.085981in}{1.652184in}}%
\pgfpathlineto{\pgfqpoint{2.091627in}{1.652681in}}%
\pgfpathlineto{\pgfqpoint{2.088768in}{1.658202in}}%
\pgfpathlineto{\pgfqpoint{2.088047in}{1.667928in}}%
\pgfpathlineto{\pgfqpoint{2.094537in}{1.668479in}}%
\pgfpathlineto{\pgfqpoint{2.107303in}{1.662856in}}%
\pgfpathlineto{\pgfqpoint{2.109669in}{1.658235in}}%
\pgfpathlineto{\pgfqpoint{2.122391in}{1.643026in}}%
\pgfpathlineto{\pgfqpoint{2.132464in}{1.643757in}}%
\pgfpathlineto{\pgfqpoint{2.136667in}{1.646574in}}%
\pgfpathlineto{\pgfqpoint{2.143748in}{1.640811in}}%
\pgfpathlineto{\pgfqpoint{2.153377in}{1.649798in}}%
\pgfpathlineto{\pgfqpoint{2.156972in}{1.644308in}}%
\pgfpathlineto{\pgfqpoint{2.161790in}{1.650391in}}%
\pgfpathlineto{\pgfqpoint{2.176831in}{1.658426in}}%
\pgfpathlineto{\pgfqpoint{2.188229in}{1.662007in}}%
\pgfpathlineto{\pgfqpoint{2.192935in}{1.614287in}}%
\pgfpathclose%
\pgfusepath{fill}%
\end{pgfscope}%
\begin{pgfscope}%
\pgfpathrectangle{\pgfqpoint{0.100000in}{0.100000in}}{\pgfqpoint{3.007045in}{1.925000in}}%
\pgfusepath{clip}%
\pgfsetbuttcap%
\pgfsetmiterjoin%
\definecolor{currentfill}{rgb}{0.632526,0.797647,0.886874}%
\pgfsetfillcolor{currentfill}%
\pgfsetlinewidth{0.000000pt}%
\definecolor{currentstroke}{rgb}{0.000000,0.000000,0.000000}%
\pgfsetstrokecolor{currentstroke}%
\pgfsetstrokeopacity{0.000000}%
\pgfsetdash{}{0pt}%
\pgfpathmoveto{\pgfqpoint{2.753651in}{1.370464in}}%
\pgfpathlineto{\pgfqpoint{2.746998in}{1.361954in}}%
\pgfpathlineto{\pgfqpoint{2.738618in}{1.356791in}}%
\pgfpathlineto{\pgfqpoint{2.732190in}{1.364248in}}%
\pgfpathlineto{\pgfqpoint{2.735363in}{1.367506in}}%
\pgfpathlineto{\pgfqpoint{2.727132in}{1.374300in}}%
\pgfpathlineto{\pgfqpoint{2.723613in}{1.374911in}}%
\pgfpathlineto{\pgfqpoint{2.720526in}{1.370514in}}%
\pgfpathlineto{\pgfqpoint{2.723365in}{1.366527in}}%
\pgfpathlineto{\pgfqpoint{2.711768in}{1.357635in}}%
\pgfpathlineto{\pgfqpoint{2.701015in}{1.357842in}}%
\pgfpathlineto{\pgfqpoint{2.698167in}{1.352791in}}%
\pgfpathlineto{\pgfqpoint{2.698167in}{1.347287in}}%
\pgfpathlineto{\pgfqpoint{2.694956in}{1.344848in}}%
\pgfpathlineto{\pgfqpoint{2.692918in}{1.348382in}}%
\pgfpathlineto{\pgfqpoint{2.688600in}{1.346784in}}%
\pgfpathlineto{\pgfqpoint{2.686221in}{1.354519in}}%
\pgfpathlineto{\pgfqpoint{2.679466in}{1.361047in}}%
\pgfpathlineto{\pgfqpoint{2.677192in}{1.366767in}}%
\pgfpathlineto{\pgfqpoint{2.679506in}{1.367379in}}%
\pgfpathlineto{\pgfqpoint{2.686454in}{1.378941in}}%
\pgfpathlineto{\pgfqpoint{2.691171in}{1.379954in}}%
\pgfpathlineto{\pgfqpoint{2.692396in}{1.393752in}}%
\pgfpathlineto{\pgfqpoint{2.691983in}{1.402891in}}%
\pgfpathlineto{\pgfqpoint{2.695962in}{1.404095in}}%
\pgfpathlineto{\pgfqpoint{2.727566in}{1.410162in}}%
\pgfpathlineto{\pgfqpoint{2.721527in}{1.432985in}}%
\pgfpathlineto{\pgfqpoint{2.727454in}{1.434295in}}%
\pgfpathlineto{\pgfqpoint{2.732166in}{1.431244in}}%
\pgfpathlineto{\pgfqpoint{2.733867in}{1.427016in}}%
\pgfpathlineto{\pgfqpoint{2.739778in}{1.426983in}}%
\pgfpathlineto{\pgfqpoint{2.745672in}{1.421591in}}%
\pgfpathlineto{\pgfqpoint{2.747986in}{1.413389in}}%
\pgfpathlineto{\pgfqpoint{2.752320in}{1.406665in}}%
\pgfpathlineto{\pgfqpoint{2.760698in}{1.403338in}}%
\pgfpathlineto{\pgfqpoint{2.765754in}{1.403877in}}%
\pgfpathlineto{\pgfqpoint{2.768627in}{1.400117in}}%
\pgfpathlineto{\pgfqpoint{2.763125in}{1.394125in}}%
\pgfpathlineto{\pgfqpoint{2.762147in}{1.386718in}}%
\pgfpathlineto{\pgfqpoint{2.753651in}{1.370464in}}%
\pgfpathclose%
\pgfusepath{fill}%
\end{pgfscope}%
\begin{pgfscope}%
\pgfpathrectangle{\pgfqpoint{0.100000in}{0.100000in}}{\pgfqpoint{3.007045in}{1.925000in}}%
\pgfusepath{clip}%
\pgfsetbuttcap%
\pgfsetmiterjoin%
\definecolor{currentfill}{rgb}{0.351511,0.635848,0.812641}%
\pgfsetfillcolor{currentfill}%
\pgfsetlinewidth{0.000000pt}%
\definecolor{currentstroke}{rgb}{0.000000,0.000000,0.000000}%
\pgfsetstrokecolor{currentstroke}%
\pgfsetstrokeopacity{0.000000}%
\pgfsetdash{}{0pt}%
\pgfpathmoveto{\pgfqpoint{1.565600in}{1.146405in}}%
\pgfpathlineto{\pgfqpoint{1.594074in}{1.145204in}}%
\pgfpathlineto{\pgfqpoint{1.593668in}{1.127954in}}%
\pgfpathlineto{\pgfqpoint{1.593165in}{1.116517in}}%
\pgfpathlineto{\pgfqpoint{1.536726in}{1.118994in}}%
\pgfpathlineto{\pgfqpoint{1.507527in}{1.120611in}}%
\pgfpathlineto{\pgfqpoint{1.509458in}{1.149192in}}%
\pgfpathlineto{\pgfqpoint{1.565600in}{1.146405in}}%
\pgfpathclose%
\pgfusepath{fill}%
\end{pgfscope}%
\begin{pgfscope}%
\pgfpathrectangle{\pgfqpoint{0.100000in}{0.100000in}}{\pgfqpoint{3.007045in}{1.925000in}}%
\pgfusepath{clip}%
\pgfsetbuttcap%
\pgfsetmiterjoin%
\definecolor{currentfill}{rgb}{0.093272,0.396878,0.673664}%
\pgfsetfillcolor{currentfill}%
\pgfsetlinewidth{0.000000pt}%
\definecolor{currentstroke}{rgb}{0.000000,0.000000,0.000000}%
\pgfsetstrokecolor{currentstroke}%
\pgfsetstrokeopacity{0.000000}%
\pgfsetdash{}{0pt}%
\pgfpathmoveto{\pgfqpoint{0.558440in}{0.376370in}}%
\pgfpathlineto{\pgfqpoint{0.557442in}{0.379760in}}%
\pgfpathlineto{\pgfqpoint{0.559412in}{0.380419in}}%
\pgfpathlineto{\pgfqpoint{0.561194in}{0.379545in}}%
\pgfpathlineto{\pgfqpoint{0.561660in}{0.376234in}}%
\pgfpathlineto{\pgfqpoint{0.560353in}{0.375728in}}%
\pgfpathlineto{\pgfqpoint{0.558440in}{0.376370in}}%
\pgfpathclose%
\pgfusepath{fill}%
\end{pgfscope}%
\begin{pgfscope}%
\pgfpathrectangle{\pgfqpoint{0.100000in}{0.100000in}}{\pgfqpoint{3.007045in}{1.925000in}}%
\pgfusepath{clip}%
\pgfsetbuttcap%
\pgfsetmiterjoin%
\definecolor{currentfill}{rgb}{0.093272,0.396878,0.673664}%
\pgfsetfillcolor{currentfill}%
\pgfsetlinewidth{0.000000pt}%
\definecolor{currentstroke}{rgb}{0.000000,0.000000,0.000000}%
\pgfsetstrokecolor{currentstroke}%
\pgfsetstrokeopacity{0.000000}%
\pgfsetdash{}{0pt}%
\pgfpathmoveto{\pgfqpoint{0.562662in}{0.374546in}}%
\pgfpathlineto{\pgfqpoint{0.563801in}{0.376835in}}%
\pgfpathlineto{\pgfqpoint{0.565086in}{0.376399in}}%
\pgfpathlineto{\pgfqpoint{0.565958in}{0.374725in}}%
\pgfpathlineto{\pgfqpoint{0.562662in}{0.374546in}}%
\pgfpathclose%
\pgfusepath{fill}%
\end{pgfscope}%
\begin{pgfscope}%
\pgfpathrectangle{\pgfqpoint{0.100000in}{0.100000in}}{\pgfqpoint{3.007045in}{1.925000in}}%
\pgfusepath{clip}%
\pgfsetbuttcap%
\pgfsetmiterjoin%
\definecolor{currentfill}{rgb}{0.093272,0.396878,0.673664}%
\pgfsetfillcolor{currentfill}%
\pgfsetlinewidth{0.000000pt}%
\definecolor{currentstroke}{rgb}{0.000000,0.000000,0.000000}%
\pgfsetstrokecolor{currentstroke}%
\pgfsetstrokeopacity{0.000000}%
\pgfsetdash{}{0pt}%
\pgfpathmoveto{\pgfqpoint{0.630180in}{0.347609in}}%
\pgfpathlineto{\pgfqpoint{0.628737in}{0.345721in}}%
\pgfpathlineto{\pgfqpoint{0.627895in}{0.348703in}}%
\pgfpathlineto{\pgfqpoint{0.625581in}{0.348130in}}%
\pgfpathlineto{\pgfqpoint{0.628285in}{0.352021in}}%
\pgfpathlineto{\pgfqpoint{0.631284in}{0.352146in}}%
\pgfpathlineto{\pgfqpoint{0.630506in}{0.350620in}}%
\pgfpathlineto{\pgfqpoint{0.631605in}{0.349217in}}%
\pgfpathlineto{\pgfqpoint{0.634087in}{0.348313in}}%
\pgfpathlineto{\pgfqpoint{0.632370in}{0.345968in}}%
\pgfpathlineto{\pgfqpoint{0.631280in}{0.347967in}}%
\pgfpathlineto{\pgfqpoint{0.630180in}{0.347609in}}%
\pgfpathclose%
\pgfusepath{fill}%
\end{pgfscope}%
\begin{pgfscope}%
\pgfpathrectangle{\pgfqpoint{0.100000in}{0.100000in}}{\pgfqpoint{3.007045in}{1.925000in}}%
\pgfusepath{clip}%
\pgfsetbuttcap%
\pgfsetmiterjoin%
\definecolor{currentfill}{rgb}{0.093272,0.396878,0.673664}%
\pgfsetfillcolor{currentfill}%
\pgfsetlinewidth{0.000000pt}%
\definecolor{currentstroke}{rgb}{0.000000,0.000000,0.000000}%
\pgfsetstrokecolor{currentstroke}%
\pgfsetstrokeopacity{0.000000}%
\pgfsetdash{}{0pt}%
\pgfpathmoveto{\pgfqpoint{0.606094in}{0.356546in}}%
\pgfpathlineto{\pgfqpoint{0.606285in}{0.359081in}}%
\pgfpathlineto{\pgfqpoint{0.608596in}{0.359253in}}%
\pgfpathlineto{\pgfqpoint{0.608292in}{0.357179in}}%
\pgfpathlineto{\pgfqpoint{0.606094in}{0.356546in}}%
\pgfpathclose%
\pgfusepath{fill}%
\end{pgfscope}%
\begin{pgfscope}%
\pgfpathrectangle{\pgfqpoint{0.100000in}{0.100000in}}{\pgfqpoint{3.007045in}{1.925000in}}%
\pgfusepath{clip}%
\pgfsetbuttcap%
\pgfsetmiterjoin%
\definecolor{currentfill}{rgb}{0.093272,0.396878,0.673664}%
\pgfsetfillcolor{currentfill}%
\pgfsetlinewidth{0.000000pt}%
\definecolor{currentstroke}{rgb}{0.000000,0.000000,0.000000}%
\pgfsetstrokecolor{currentstroke}%
\pgfsetstrokeopacity{0.000000}%
\pgfsetdash{}{0pt}%
\pgfpathmoveto{\pgfqpoint{0.628222in}{0.338692in}}%
\pgfpathlineto{\pgfqpoint{0.634226in}{0.342115in}}%
\pgfpathlineto{\pgfqpoint{0.635742in}{0.340232in}}%
\pgfpathlineto{\pgfqpoint{0.633794in}{0.339431in}}%
\pgfpathlineto{\pgfqpoint{0.628222in}{0.338692in}}%
\pgfpathclose%
\pgfusepath{fill}%
\end{pgfscope}%
\begin{pgfscope}%
\pgfpathrectangle{\pgfqpoint{0.100000in}{0.100000in}}{\pgfqpoint{3.007045in}{1.925000in}}%
\pgfusepath{clip}%
\pgfsetbuttcap%
\pgfsetmiterjoin%
\definecolor{currentfill}{rgb}{0.093272,0.396878,0.673664}%
\pgfsetfillcolor{currentfill}%
\pgfsetlinewidth{0.000000pt}%
\definecolor{currentstroke}{rgb}{0.000000,0.000000,0.000000}%
\pgfsetstrokecolor{currentstroke}%
\pgfsetstrokeopacity{0.000000}%
\pgfsetdash{}{0pt}%
\pgfpathmoveto{\pgfqpoint{0.646196in}{0.346313in}}%
\pgfpathlineto{\pgfqpoint{0.645147in}{0.347754in}}%
\pgfpathlineto{\pgfqpoint{0.648533in}{0.349423in}}%
\pgfpathlineto{\pgfqpoint{0.647621in}{0.351703in}}%
\pgfpathlineto{\pgfqpoint{0.645804in}{0.351697in}}%
\pgfpathlineto{\pgfqpoint{0.644687in}{0.352622in}}%
\pgfpathlineto{\pgfqpoint{0.640184in}{0.353124in}}%
\pgfpathlineto{\pgfqpoint{0.636431in}{0.353197in}}%
\pgfpathlineto{\pgfqpoint{0.634892in}{0.354431in}}%
\pgfpathlineto{\pgfqpoint{0.633377in}{0.353497in}}%
\pgfpathlineto{\pgfqpoint{0.631296in}{0.353562in}}%
\pgfpathlineto{\pgfqpoint{0.631571in}{0.355560in}}%
\pgfpathlineto{\pgfqpoint{0.627061in}{0.355667in}}%
\pgfpathlineto{\pgfqpoint{0.624673in}{0.356334in}}%
\pgfpathlineto{\pgfqpoint{0.622641in}{0.358315in}}%
\pgfpathlineto{\pgfqpoint{0.624401in}{0.359859in}}%
\pgfpathlineto{\pgfqpoint{0.627993in}{0.361739in}}%
\pgfpathlineto{\pgfqpoint{0.626212in}{0.363517in}}%
\pgfpathlineto{\pgfqpoint{0.624339in}{0.363616in}}%
\pgfpathlineto{\pgfqpoint{0.620521in}{0.360581in}}%
\pgfpathlineto{\pgfqpoint{0.617999in}{0.360096in}}%
\pgfpathlineto{\pgfqpoint{0.615778in}{0.360562in}}%
\pgfpathlineto{\pgfqpoint{0.614484in}{0.358255in}}%
\pgfpathlineto{\pgfqpoint{0.612634in}{0.358313in}}%
\pgfpathlineto{\pgfqpoint{0.608448in}{0.361014in}}%
\pgfpathlineto{\pgfqpoint{0.608279in}{0.362605in}}%
\pgfpathlineto{\pgfqpoint{0.609968in}{0.362762in}}%
\pgfpathlineto{\pgfqpoint{0.610410in}{0.364673in}}%
\pgfpathlineto{\pgfqpoint{0.610099in}{0.367347in}}%
\pgfpathlineto{\pgfqpoint{0.606743in}{0.363415in}}%
\pgfpathlineto{\pgfqpoint{0.606556in}{0.361958in}}%
\pgfpathlineto{\pgfqpoint{0.603910in}{0.362284in}}%
\pgfpathlineto{\pgfqpoint{0.602151in}{0.363752in}}%
\pgfpathlineto{\pgfqpoint{0.603264in}{0.366602in}}%
\pgfpathlineto{\pgfqpoint{0.602235in}{0.369208in}}%
\pgfpathlineto{\pgfqpoint{0.601172in}{0.368904in}}%
\pgfpathlineto{\pgfqpoint{0.600989in}{0.365044in}}%
\pgfpathlineto{\pgfqpoint{0.595049in}{0.365340in}}%
\pgfpathlineto{\pgfqpoint{0.596365in}{0.363673in}}%
\pgfpathlineto{\pgfqpoint{0.594450in}{0.363174in}}%
\pgfpathlineto{\pgfqpoint{0.593766in}{0.364861in}}%
\pgfpathlineto{\pgfqpoint{0.592021in}{0.364091in}}%
\pgfpathlineto{\pgfqpoint{0.589288in}{0.365861in}}%
\pgfpathlineto{\pgfqpoint{0.585892in}{0.369053in}}%
\pgfpathlineto{\pgfqpoint{0.581946in}{0.370113in}}%
\pgfpathlineto{\pgfqpoint{0.578510in}{0.369321in}}%
\pgfpathlineto{\pgfqpoint{0.576281in}{0.370552in}}%
\pgfpathlineto{\pgfqpoint{0.575149in}{0.374376in}}%
\pgfpathlineto{\pgfqpoint{0.576135in}{0.376449in}}%
\pgfpathlineto{\pgfqpoint{0.580707in}{0.375972in}}%
\pgfpathlineto{\pgfqpoint{0.586276in}{0.378070in}}%
\pgfpathlineto{\pgfqpoint{0.587415in}{0.376136in}}%
\pgfpathlineto{\pgfqpoint{0.589357in}{0.375160in}}%
\pgfpathlineto{\pgfqpoint{0.590661in}{0.375506in}}%
\pgfpathlineto{\pgfqpoint{0.595051in}{0.373915in}}%
\pgfpathlineto{\pgfqpoint{0.597649in}{0.371586in}}%
\pgfpathlineto{\pgfqpoint{0.596738in}{0.368224in}}%
\pgfpathlineto{\pgfqpoint{0.598337in}{0.367723in}}%
\pgfpathlineto{\pgfqpoint{0.600669in}{0.370774in}}%
\pgfpathlineto{\pgfqpoint{0.604276in}{0.369985in}}%
\pgfpathlineto{\pgfqpoint{0.605343in}{0.367909in}}%
\pgfpathlineto{\pgfqpoint{0.607337in}{0.368055in}}%
\pgfpathlineto{\pgfqpoint{0.620695in}{0.370601in}}%
\pgfpathlineto{\pgfqpoint{0.625712in}{0.370166in}}%
\pgfpathlineto{\pgfqpoint{0.628012in}{0.370317in}}%
\pgfpathlineto{\pgfqpoint{0.633395in}{0.367875in}}%
\pgfpathlineto{\pgfqpoint{0.636602in}{0.364776in}}%
\pgfpathlineto{\pgfqpoint{0.635557in}{0.363488in}}%
\pgfpathlineto{\pgfqpoint{0.635523in}{0.359748in}}%
\pgfpathlineto{\pgfqpoint{0.637405in}{0.360249in}}%
\pgfpathlineto{\pgfqpoint{0.640275in}{0.358605in}}%
\pgfpathlineto{\pgfqpoint{0.639530in}{0.356749in}}%
\pgfpathlineto{\pgfqpoint{0.640742in}{0.355326in}}%
\pgfpathlineto{\pgfqpoint{0.642306in}{0.356377in}}%
\pgfpathlineto{\pgfqpoint{0.641680in}{0.358719in}}%
\pgfpathlineto{\pgfqpoint{0.640362in}{0.360039in}}%
\pgfpathlineto{\pgfqpoint{0.641503in}{0.361170in}}%
\pgfpathlineto{\pgfqpoint{0.644772in}{0.362563in}}%
\pgfpathlineto{\pgfqpoint{0.646911in}{0.363995in}}%
\pgfpathlineto{\pgfqpoint{0.651915in}{0.364863in}}%
\pgfpathlineto{\pgfqpoint{0.654420in}{0.364388in}}%
\pgfpathlineto{\pgfqpoint{0.656995in}{0.364705in}}%
\pgfpathlineto{\pgfqpoint{0.659517in}{0.364305in}}%
\pgfpathlineto{\pgfqpoint{0.665903in}{0.362144in}}%
\pgfpathlineto{\pgfqpoint{0.665627in}{0.363168in}}%
\pgfpathlineto{\pgfqpoint{0.669120in}{0.362795in}}%
\pgfpathlineto{\pgfqpoint{0.670704in}{0.361448in}}%
\pgfpathlineto{\pgfqpoint{0.668605in}{0.361721in}}%
\pgfpathlineto{\pgfqpoint{0.667310in}{0.360198in}}%
\pgfpathlineto{\pgfqpoint{0.664908in}{0.362318in}}%
\pgfpathlineto{\pgfqpoint{0.663515in}{0.360688in}}%
\pgfpathlineto{\pgfqpoint{0.660386in}{0.363372in}}%
\pgfpathlineto{\pgfqpoint{0.659049in}{0.361833in}}%
\pgfpathlineto{\pgfqpoint{0.655995in}{0.364392in}}%
\pgfpathlineto{\pgfqpoint{0.653338in}{0.361411in}}%
\pgfpathlineto{\pgfqpoint{0.654252in}{0.360591in}}%
\pgfpathlineto{\pgfqpoint{0.649075in}{0.354518in}}%
\pgfpathlineto{\pgfqpoint{0.647002in}{0.353387in}}%
\pgfpathlineto{\pgfqpoint{0.650028in}{0.350774in}}%
\pgfpathlineto{\pgfqpoint{0.646196in}{0.346313in}}%
\pgfpathclose%
\pgfusepath{fill}%
\end{pgfscope}%
\begin{pgfscope}%
\pgfpathrectangle{\pgfqpoint{0.100000in}{0.100000in}}{\pgfqpoint{3.007045in}{1.925000in}}%
\pgfusepath{clip}%
\pgfsetbuttcap%
\pgfsetmiterjoin%
\definecolor{currentfill}{rgb}{0.296025,0.597955,0.790988}%
\pgfsetfillcolor{currentfill}%
\pgfsetlinewidth{0.000000pt}%
\definecolor{currentstroke}{rgb}{0.000000,0.000000,0.000000}%
\pgfsetstrokecolor{currentstroke}%
\pgfsetstrokeopacity{0.000000}%
\pgfsetdash{}{0pt}%
\pgfpathmoveto{\pgfqpoint{1.588665in}{1.618076in}}%
\pgfpathlineto{\pgfqpoint{1.587550in}{1.595331in}}%
\pgfpathlineto{\pgfqpoint{1.554716in}{1.596972in}}%
\pgfpathlineto{\pgfqpoint{1.518744in}{1.599173in}}%
\pgfpathlineto{\pgfqpoint{1.514131in}{1.607335in}}%
\pgfpathlineto{\pgfqpoint{1.514093in}{1.614101in}}%
\pgfpathlineto{\pgfqpoint{1.517258in}{1.619082in}}%
\pgfpathlineto{\pgfqpoint{1.518046in}{1.627144in}}%
\pgfpathlineto{\pgfqpoint{1.516525in}{1.631523in}}%
\pgfpathlineto{\pgfqpoint{1.519039in}{1.638417in}}%
\pgfpathlineto{\pgfqpoint{1.518209in}{1.642479in}}%
\pgfpathlineto{\pgfqpoint{1.514103in}{1.645425in}}%
\pgfpathlineto{\pgfqpoint{1.540553in}{1.643789in}}%
\pgfpathlineto{\pgfqpoint{1.569331in}{1.642166in}}%
\pgfpathlineto{\pgfqpoint{1.588163in}{1.641202in}}%
\pgfpathlineto{\pgfqpoint{1.587099in}{1.618157in}}%
\pgfpathlineto{\pgfqpoint{1.588665in}{1.618076in}}%
\pgfpathclose%
\pgfusepath{fill}%
\end{pgfscope}%
\begin{pgfscope}%
\pgfpathrectangle{\pgfqpoint{0.100000in}{0.100000in}}{\pgfqpoint{3.007045in}{1.925000in}}%
\pgfusepath{clip}%
\pgfsetbuttcap%
\pgfsetmiterjoin%
\definecolor{currentfill}{rgb}{0.071742,0.369319,0.652134}%
\pgfsetfillcolor{currentfill}%
\pgfsetlinewidth{0.000000pt}%
\definecolor{currentstroke}{rgb}{0.000000,0.000000,0.000000}%
\pgfsetstrokecolor{currentstroke}%
\pgfsetstrokeopacity{0.000000}%
\pgfsetdash{}{0pt}%
\pgfpathmoveto{\pgfqpoint{1.509153in}{1.008633in}}%
\pgfpathlineto{\pgfqpoint{1.507272in}{0.975589in}}%
\pgfpathlineto{\pgfqpoint{1.456990in}{0.978623in}}%
\pgfpathlineto{\pgfqpoint{1.459593in}{1.011387in}}%
\pgfpathlineto{\pgfqpoint{1.453227in}{1.011793in}}%
\pgfpathlineto{\pgfqpoint{1.454880in}{1.037434in}}%
\pgfpathlineto{\pgfqpoint{1.476575in}{1.036016in}}%
\pgfpathlineto{\pgfqpoint{1.476919in}{1.041763in}}%
\pgfpathlineto{\pgfqpoint{1.505407in}{1.040070in}}%
\pgfpathlineto{\pgfqpoint{1.505902in}{1.034236in}}%
\pgfpathlineto{\pgfqpoint{1.504587in}{1.008926in}}%
\pgfpathlineto{\pgfqpoint{1.509153in}{1.008633in}}%
\pgfpathclose%
\pgfusepath{fill}%
\end{pgfscope}%
\begin{pgfscope}%
\pgfpathrectangle{\pgfqpoint{0.100000in}{0.100000in}}{\pgfqpoint{3.007045in}{1.925000in}}%
\pgfusepath{clip}%
\pgfsetbuttcap%
\pgfsetmiterjoin%
\definecolor{currentfill}{rgb}{0.265759,0.577286,0.779177}%
\pgfsetfillcolor{currentfill}%
\pgfsetlinewidth{0.000000pt}%
\definecolor{currentstroke}{rgb}{0.000000,0.000000,0.000000}%
\pgfsetstrokecolor{currentstroke}%
\pgfsetstrokeopacity{0.000000}%
\pgfsetdash{}{0pt}%
\pgfpathmoveto{\pgfqpoint{1.753660in}{1.413597in}}%
\pgfpathlineto{\pgfqpoint{1.730861in}{1.413920in}}%
\pgfpathlineto{\pgfqpoint{1.731061in}{1.429926in}}%
\pgfpathlineto{\pgfqpoint{1.721870in}{1.430045in}}%
\pgfpathlineto{\pgfqpoint{1.722176in}{1.453096in}}%
\pgfpathlineto{\pgfqpoint{1.721890in}{1.476069in}}%
\pgfpathlineto{\pgfqpoint{1.744188in}{1.475811in}}%
\pgfpathlineto{\pgfqpoint{1.767169in}{1.475679in}}%
\pgfpathlineto{\pgfqpoint{1.767569in}{1.469929in}}%
\pgfpathlineto{\pgfqpoint{1.778980in}{1.469921in}}%
\pgfpathlineto{\pgfqpoint{1.779266in}{1.452736in}}%
\pgfpathlineto{\pgfqpoint{1.779303in}{1.429779in}}%
\pgfpathlineto{\pgfqpoint{1.776405in}{1.429765in}}%
\pgfpathlineto{\pgfqpoint{1.753741in}{1.429789in}}%
\pgfpathlineto{\pgfqpoint{1.753660in}{1.413597in}}%
\pgfpathclose%
\pgfusepath{fill}%
\end{pgfscope}%
\begin{pgfscope}%
\pgfpathrectangle{\pgfqpoint{0.100000in}{0.100000in}}{\pgfqpoint{3.007045in}{1.925000in}}%
\pgfusepath{clip}%
\pgfsetbuttcap%
\pgfsetmiterjoin%
\definecolor{currentfill}{rgb}{0.535686,0.746082,0.864252}%
\pgfsetfillcolor{currentfill}%
\pgfsetlinewidth{0.000000pt}%
\definecolor{currentstroke}{rgb}{0.000000,0.000000,0.000000}%
\pgfsetstrokecolor{currentstroke}%
\pgfsetstrokeopacity{0.000000}%
\pgfsetdash{}{0pt}%
\pgfpathmoveto{\pgfqpoint{1.798792in}{1.236593in}}%
\pgfpathlineto{\pgfqpoint{1.790743in}{1.236578in}}%
\pgfpathlineto{\pgfqpoint{1.734017in}{1.237625in}}%
\pgfpathlineto{\pgfqpoint{1.732568in}{1.242722in}}%
\pgfpathlineto{\pgfqpoint{1.727958in}{1.247635in}}%
\pgfpathlineto{\pgfqpoint{1.730711in}{1.250739in}}%
\pgfpathlineto{\pgfqpoint{1.731811in}{1.258499in}}%
\pgfpathlineto{\pgfqpoint{1.753153in}{1.258349in}}%
\pgfpathlineto{\pgfqpoint{1.753309in}{1.275315in}}%
\pgfpathlineto{\pgfqpoint{1.764595in}{1.275306in}}%
\pgfpathlineto{\pgfqpoint{1.787185in}{1.275241in}}%
\pgfpathlineto{\pgfqpoint{1.798597in}{1.275256in}}%
\pgfpathlineto{\pgfqpoint{1.798792in}{1.236593in}}%
\pgfpathclose%
\pgfusepath{fill}%
\end{pgfscope}%
\begin{pgfscope}%
\pgfpathrectangle{\pgfqpoint{0.100000in}{0.100000in}}{\pgfqpoint{3.007045in}{1.925000in}}%
\pgfusepath{clip}%
\pgfsetbuttcap%
\pgfsetmiterjoin%
\definecolor{currentfill}{rgb}{0.260715,0.573841,0.777209}%
\pgfsetfillcolor{currentfill}%
\pgfsetlinewidth{0.000000pt}%
\definecolor{currentstroke}{rgb}{0.000000,0.000000,0.000000}%
\pgfsetstrokecolor{currentstroke}%
\pgfsetstrokeopacity{0.000000}%
\pgfsetdash{}{0pt}%
\pgfpathmoveto{\pgfqpoint{1.634574in}{0.311704in}}%
\pgfpathlineto{\pgfqpoint{1.632618in}{0.311751in}}%
\pgfpathlineto{\pgfqpoint{1.628886in}{0.326991in}}%
\pgfpathlineto{\pgfqpoint{1.629356in}{0.349314in}}%
\pgfpathlineto{\pgfqpoint{1.631677in}{0.363791in}}%
\pgfpathlineto{\pgfqpoint{1.647149in}{0.392795in}}%
\pgfpathlineto{\pgfqpoint{1.651127in}{0.395908in}}%
\pgfpathlineto{\pgfqpoint{1.655482in}{0.405989in}}%
\pgfpathlineto{\pgfqpoint{1.661560in}{0.414117in}}%
\pgfpathlineto{\pgfqpoint{1.664160in}{0.412781in}}%
\pgfpathlineto{\pgfqpoint{1.662731in}{0.409312in}}%
\pgfpathlineto{\pgfqpoint{1.652282in}{0.397408in}}%
\pgfpathlineto{\pgfqpoint{1.643553in}{0.383920in}}%
\pgfpathlineto{\pgfqpoint{1.637879in}{0.372867in}}%
\pgfpathlineto{\pgfqpoint{1.633473in}{0.362042in}}%
\pgfpathlineto{\pgfqpoint{1.630879in}{0.352084in}}%
\pgfpathlineto{\pgfqpoint{1.629805in}{0.336635in}}%
\pgfpathlineto{\pgfqpoint{1.631675in}{0.322573in}}%
\pgfpathlineto{\pgfqpoint{1.634574in}{0.311704in}}%
\pgfpathclose%
\pgfusepath{fill}%
\end{pgfscope}%
\begin{pgfscope}%
\pgfpathrectangle{\pgfqpoint{0.100000in}{0.100000in}}{\pgfqpoint{3.007045in}{1.925000in}}%
\pgfusepath{clip}%
\pgfsetbuttcap%
\pgfsetmiterjoin%
\definecolor{currentfill}{rgb}{0.260715,0.573841,0.777209}%
\pgfsetfillcolor{currentfill}%
\pgfsetlinewidth{0.000000pt}%
\definecolor{currentstroke}{rgb}{0.000000,0.000000,0.000000}%
\pgfsetstrokecolor{currentstroke}%
\pgfsetstrokeopacity{0.000000}%
\pgfsetdash{}{0pt}%
\pgfpathmoveto{\pgfqpoint{1.625397in}{0.311924in}}%
\pgfpathlineto{\pgfqpoint{1.599953in}{0.312537in}}%
\pgfpathlineto{\pgfqpoint{1.592550in}{0.313976in}}%
\pgfpathlineto{\pgfqpoint{1.592904in}{0.325092in}}%
\pgfpathlineto{\pgfqpoint{1.566652in}{0.326194in}}%
\pgfpathlineto{\pgfqpoint{1.567623in}{0.344460in}}%
\pgfpathlineto{\pgfqpoint{1.564709in}{0.344532in}}%
\pgfpathlineto{\pgfqpoint{1.563594in}{0.356706in}}%
\pgfpathlineto{\pgfqpoint{1.560356in}{0.365181in}}%
\pgfpathlineto{\pgfqpoint{1.545690in}{0.365433in}}%
\pgfpathlineto{\pgfqpoint{1.547367in}{0.412624in}}%
\pgfpathlineto{\pgfqpoint{1.601772in}{0.410582in}}%
\pgfpathlineto{\pgfqpoint{1.600656in}{0.414463in}}%
\pgfpathlineto{\pgfqpoint{1.605884in}{0.418493in}}%
\pgfpathlineto{\pgfqpoint{1.620569in}{0.414900in}}%
\pgfpathlineto{\pgfqpoint{1.632363in}{0.431928in}}%
\pgfpathlineto{\pgfqpoint{1.645296in}{0.442654in}}%
\pgfpathlineto{\pgfqpoint{1.651860in}{0.442260in}}%
\pgfpathlineto{\pgfqpoint{1.655495in}{0.438831in}}%
\pgfpathlineto{\pgfqpoint{1.661661in}{0.438978in}}%
\pgfpathlineto{\pgfqpoint{1.667536in}{0.431953in}}%
\pgfpathlineto{\pgfqpoint{1.666217in}{0.423659in}}%
\pgfpathlineto{\pgfqpoint{1.665504in}{0.419888in}}%
\pgfpathlineto{\pgfqpoint{1.658131in}{0.413570in}}%
\pgfpathlineto{\pgfqpoint{1.654267in}{0.413508in}}%
\pgfpathlineto{\pgfqpoint{1.651083in}{0.406992in}}%
\pgfpathlineto{\pgfqpoint{1.642502in}{0.393804in}}%
\pgfpathlineto{\pgfqpoint{1.637586in}{0.397797in}}%
\pgfpathlineto{\pgfqpoint{1.631800in}{0.395290in}}%
\pgfpathlineto{\pgfqpoint{1.632028in}{0.388063in}}%
\pgfpathlineto{\pgfqpoint{1.638379in}{0.385411in}}%
\pgfpathlineto{\pgfqpoint{1.634179in}{0.376996in}}%
\pgfpathlineto{\pgfqpoint{1.628882in}{0.361108in}}%
\pgfpathlineto{\pgfqpoint{1.626077in}{0.346812in}}%
\pgfpathlineto{\pgfqpoint{1.619237in}{0.338192in}}%
\pgfpathlineto{\pgfqpoint{1.618817in}{0.328087in}}%
\pgfpathlineto{\pgfqpoint{1.623729in}{0.322672in}}%
\pgfpathlineto{\pgfqpoint{1.625397in}{0.311924in}}%
\pgfpathclose%
\pgfusepath{fill}%
\end{pgfscope}%
\begin{pgfscope}%
\pgfpathrectangle{\pgfqpoint{0.100000in}{0.100000in}}{\pgfqpoint{3.007045in}{1.925000in}}%
\pgfusepath{clip}%
\pgfsetbuttcap%
\pgfsetmiterjoin%
\definecolor{currentfill}{rgb}{0.290980,0.594510,0.789020}%
\pgfsetfillcolor{currentfill}%
\pgfsetlinewidth{0.000000pt}%
\definecolor{currentstroke}{rgb}{0.000000,0.000000,0.000000}%
\pgfsetstrokecolor{currentstroke}%
\pgfsetstrokeopacity{0.000000}%
\pgfsetdash{}{0pt}%
\pgfpathmoveto{\pgfqpoint{1.703058in}{1.453389in}}%
\pgfpathlineto{\pgfqpoint{1.682306in}{1.453680in}}%
\pgfpathlineto{\pgfqpoint{1.682752in}{1.476661in}}%
\pgfpathlineto{\pgfqpoint{1.637239in}{1.477803in}}%
\pgfpathlineto{\pgfqpoint{1.637041in}{1.477807in}}%
\pgfpathlineto{\pgfqpoint{1.637786in}{1.500890in}}%
\pgfpathlineto{\pgfqpoint{1.671967in}{1.499872in}}%
\pgfpathlineto{\pgfqpoint{1.703851in}{1.499277in}}%
\pgfpathlineto{\pgfqpoint{1.703058in}{1.453389in}}%
\pgfpathclose%
\pgfusepath{fill}%
\end{pgfscope}%
\begin{pgfscope}%
\pgfpathrectangle{\pgfqpoint{0.100000in}{0.100000in}}{\pgfqpoint{3.007045in}{1.925000in}}%
\pgfusepath{clip}%
\pgfsetbuttcap%
\pgfsetmiterjoin%
\definecolor{currentfill}{rgb}{0.585882,0.773641,0.875079}%
\pgfsetfillcolor{currentfill}%
\pgfsetlinewidth{0.000000pt}%
\definecolor{currentstroke}{rgb}{0.000000,0.000000,0.000000}%
\pgfsetstrokecolor{currentstroke}%
\pgfsetstrokeopacity{0.000000}%
\pgfsetdash{}{0pt}%
\pgfpathmoveto{\pgfqpoint{1.944456in}{0.907639in}}%
\pgfpathlineto{\pgfqpoint{1.944537in}{0.902457in}}%
\pgfpathlineto{\pgfqpoint{1.941745in}{0.895933in}}%
\pgfpathlineto{\pgfqpoint{1.927775in}{0.895762in}}%
\pgfpathlineto{\pgfqpoint{1.907859in}{0.895543in}}%
\pgfpathlineto{\pgfqpoint{1.907800in}{0.901293in}}%
\pgfpathlineto{\pgfqpoint{1.887849in}{0.901272in}}%
\pgfpathlineto{\pgfqpoint{1.889712in}{0.906997in}}%
\pgfpathlineto{\pgfqpoint{1.889632in}{0.918440in}}%
\pgfpathlineto{\pgfqpoint{1.899200in}{0.918414in}}%
\pgfpathlineto{\pgfqpoint{1.899149in}{0.923195in}}%
\pgfpathlineto{\pgfqpoint{1.910553in}{0.923358in}}%
\pgfpathlineto{\pgfqpoint{1.910330in}{0.935711in}}%
\pgfpathlineto{\pgfqpoint{1.916049in}{0.935756in}}%
\pgfpathlineto{\pgfqpoint{1.915999in}{0.941499in}}%
\pgfpathlineto{\pgfqpoint{1.921721in}{0.941554in}}%
\pgfpathlineto{\pgfqpoint{1.921677in}{0.946416in}}%
\pgfpathlineto{\pgfqpoint{1.929258in}{0.940159in}}%
\pgfpathlineto{\pgfqpoint{1.927752in}{0.935378in}}%
\pgfpathlineto{\pgfqpoint{1.935018in}{0.933586in}}%
\pgfpathlineto{\pgfqpoint{1.940723in}{0.927923in}}%
\pgfpathlineto{\pgfqpoint{1.939696in}{0.925677in}}%
\pgfpathlineto{\pgfqpoint{1.944264in}{0.919360in}}%
\pgfpathlineto{\pgfqpoint{1.944456in}{0.907639in}}%
\pgfpathclose%
\pgfusepath{fill}%
\end{pgfscope}%
\begin{pgfscope}%
\pgfpathrectangle{\pgfqpoint{0.100000in}{0.100000in}}{\pgfqpoint{3.007045in}{1.925000in}}%
\pgfusepath{clip}%
\pgfsetbuttcap%
\pgfsetmiterjoin%
\definecolor{currentfill}{rgb}{0.441569,0.694410,0.843952}%
\pgfsetfillcolor{currentfill}%
\pgfsetlinewidth{0.000000pt}%
\definecolor{currentstroke}{rgb}{0.000000,0.000000,0.000000}%
\pgfsetstrokecolor{currentstroke}%
\pgfsetstrokeopacity{0.000000}%
\pgfsetdash{}{0pt}%
\pgfpathmoveto{\pgfqpoint{2.482424in}{0.851583in}}%
\pgfpathlineto{\pgfqpoint{2.474703in}{0.848311in}}%
\pgfpathlineto{\pgfqpoint{2.477899in}{0.842105in}}%
\pgfpathlineto{\pgfqpoint{2.475134in}{0.834677in}}%
\pgfpathlineto{\pgfqpoint{2.480732in}{0.829446in}}%
\pgfpathlineto{\pgfqpoint{2.479069in}{0.826299in}}%
\pgfpathlineto{\pgfqpoint{2.473536in}{0.831913in}}%
\pgfpathlineto{\pgfqpoint{2.468037in}{0.838774in}}%
\pgfpathlineto{\pgfqpoint{2.452246in}{0.846939in}}%
\pgfpathlineto{\pgfqpoint{2.449961in}{0.851491in}}%
\pgfpathlineto{\pgfqpoint{2.442202in}{0.858557in}}%
\pgfpathlineto{\pgfqpoint{2.439934in}{0.864954in}}%
\pgfpathlineto{\pgfqpoint{2.429221in}{0.879173in}}%
\pgfpathlineto{\pgfqpoint{2.423854in}{0.877561in}}%
\pgfpathlineto{\pgfqpoint{2.420077in}{0.879039in}}%
\pgfpathlineto{\pgfqpoint{2.413604in}{0.885230in}}%
\pgfpathlineto{\pgfqpoint{2.410336in}{0.885056in}}%
\pgfpathlineto{\pgfqpoint{2.403995in}{0.889257in}}%
\pgfpathlineto{\pgfqpoint{2.402990in}{0.891059in}}%
\pgfpathlineto{\pgfqpoint{2.403903in}{0.896059in}}%
\pgfpathlineto{\pgfqpoint{2.408265in}{0.902647in}}%
\pgfpathlineto{\pgfqpoint{2.413500in}{0.907511in}}%
\pgfpathlineto{\pgfqpoint{2.413670in}{0.911416in}}%
\pgfpathlineto{\pgfqpoint{2.418790in}{0.913875in}}%
\pgfpathlineto{\pgfqpoint{2.435961in}{0.922177in}}%
\pgfpathlineto{\pgfqpoint{2.452389in}{0.929337in}}%
\pgfpathlineto{\pgfqpoint{2.459680in}{0.930712in}}%
\pgfpathlineto{\pgfqpoint{2.462329in}{0.908061in}}%
\pgfpathlineto{\pgfqpoint{2.476040in}{0.897437in}}%
\pgfpathlineto{\pgfqpoint{2.481602in}{0.894314in}}%
\pgfpathlineto{\pgfqpoint{2.496666in}{0.891862in}}%
\pgfpathlineto{\pgfqpoint{2.490797in}{0.879764in}}%
\pgfpathlineto{\pgfqpoint{2.485317in}{0.874629in}}%
\pgfpathlineto{\pgfqpoint{2.484386in}{0.869099in}}%
\pgfpathlineto{\pgfqpoint{2.486659in}{0.864555in}}%
\pgfpathlineto{\pgfqpoint{2.482424in}{0.851583in}}%
\pgfpathclose%
\pgfusepath{fill}%
\end{pgfscope}%
\begin{pgfscope}%
\pgfpathrectangle{\pgfqpoint{0.100000in}{0.100000in}}{\pgfqpoint{3.007045in}{1.925000in}}%
\pgfusepath{clip}%
\pgfsetbuttcap%
\pgfsetmiterjoin%
\definecolor{currentfill}{rgb}{0.406997,0.673741,0.834295}%
\pgfsetfillcolor{currentfill}%
\pgfsetlinewidth{0.000000pt}%
\definecolor{currentstroke}{rgb}{0.000000,0.000000,0.000000}%
\pgfsetstrokecolor{currentstroke}%
\pgfsetstrokeopacity{0.000000}%
\pgfsetdash{}{0pt}%
\pgfpathmoveto{\pgfqpoint{2.253561in}{1.079316in}}%
\pgfpathlineto{\pgfqpoint{2.249127in}{1.081944in}}%
\pgfpathlineto{\pgfqpoint{2.246327in}{1.087565in}}%
\pgfpathlineto{\pgfqpoint{2.241377in}{1.090992in}}%
\pgfpathlineto{\pgfqpoint{2.229804in}{1.090057in}}%
\pgfpathlineto{\pgfqpoint{2.224931in}{1.092417in}}%
\pgfpathlineto{\pgfqpoint{2.223926in}{1.097599in}}%
\pgfpathlineto{\pgfqpoint{2.219675in}{1.101261in}}%
\pgfpathlineto{\pgfqpoint{2.217176in}{1.095188in}}%
\pgfpathlineto{\pgfqpoint{2.214133in}{1.095845in}}%
\pgfpathlineto{\pgfqpoint{2.213695in}{1.101544in}}%
\pgfpathlineto{\pgfqpoint{2.207958in}{1.101054in}}%
\pgfpathlineto{\pgfqpoint{2.207678in}{1.104917in}}%
\pgfpathlineto{\pgfqpoint{2.202065in}{1.104184in}}%
\pgfpathlineto{\pgfqpoint{2.201173in}{1.112813in}}%
\pgfpathlineto{\pgfqpoint{2.220327in}{1.114564in}}%
\pgfpathlineto{\pgfqpoint{2.218133in}{1.136791in}}%
\pgfpathlineto{\pgfqpoint{2.219718in}{1.138972in}}%
\pgfpathlineto{\pgfqpoint{2.235786in}{1.140573in}}%
\pgfpathlineto{\pgfqpoint{2.239682in}{1.138991in}}%
\pgfpathlineto{\pgfqpoint{2.240875in}{1.128586in}}%
\pgfpathlineto{\pgfqpoint{2.245396in}{1.130991in}}%
\pgfpathlineto{\pgfqpoint{2.256834in}{1.132262in}}%
\pgfpathlineto{\pgfqpoint{2.264234in}{1.131721in}}%
\pgfpathlineto{\pgfqpoint{2.264410in}{1.127604in}}%
\pgfpathlineto{\pgfqpoint{2.272501in}{1.126660in}}%
\pgfpathlineto{\pgfqpoint{2.277544in}{1.133106in}}%
\pgfpathlineto{\pgfqpoint{2.282277in}{1.134362in}}%
\pgfpathlineto{\pgfqpoint{2.285748in}{1.132441in}}%
\pgfpathlineto{\pgfqpoint{2.287382in}{1.127701in}}%
\pgfpathlineto{\pgfqpoint{2.292494in}{1.120777in}}%
\pgfpathlineto{\pgfqpoint{2.288467in}{1.113713in}}%
\pgfpathlineto{\pgfqpoint{2.288667in}{1.106429in}}%
\pgfpathlineto{\pgfqpoint{2.287143in}{1.100325in}}%
\pgfpathlineto{\pgfqpoint{2.281936in}{1.092901in}}%
\pgfpathlineto{\pgfqpoint{2.270164in}{1.089523in}}%
\pgfpathlineto{\pgfqpoint{2.265212in}{1.092380in}}%
\pgfpathlineto{\pgfqpoint{2.258362in}{1.081298in}}%
\pgfpathlineto{\pgfqpoint{2.253561in}{1.079316in}}%
\pgfpathclose%
\pgfusepath{fill}%
\end{pgfscope}%
\begin{pgfscope}%
\pgfpathrectangle{\pgfqpoint{0.100000in}{0.100000in}}{\pgfqpoint{3.007045in}{1.925000in}}%
\pgfusepath{clip}%
\pgfsetbuttcap%
\pgfsetmiterjoin%
\definecolor{currentfill}{rgb}{0.366644,0.646182,0.818547}%
\pgfsetfillcolor{currentfill}%
\pgfsetlinewidth{0.000000pt}%
\definecolor{currentstroke}{rgb}{0.000000,0.000000,0.000000}%
\pgfsetstrokecolor{currentstroke}%
\pgfsetstrokeopacity{0.000000}%
\pgfsetdash{}{0pt}%
\pgfpathmoveto{\pgfqpoint{1.610638in}{1.328045in}}%
\pgfpathlineto{\pgfqpoint{1.610215in}{1.316599in}}%
\pgfpathlineto{\pgfqpoint{1.587433in}{1.317426in}}%
\pgfpathlineto{\pgfqpoint{1.588274in}{1.340363in}}%
\pgfpathlineto{\pgfqpoint{1.565129in}{1.341345in}}%
\pgfpathlineto{\pgfqpoint{1.566978in}{1.401393in}}%
\pgfpathlineto{\pgfqpoint{1.603438in}{1.399865in}}%
\pgfpathlineto{\pgfqpoint{1.604815in}{1.396442in}}%
\pgfpathlineto{\pgfqpoint{1.612271in}{1.391873in}}%
\pgfpathlineto{\pgfqpoint{1.611668in}{1.362462in}}%
\pgfpathlineto{\pgfqpoint{1.610638in}{1.328045in}}%
\pgfpathclose%
\pgfusepath{fill}%
\end{pgfscope}%
\begin{pgfscope}%
\pgfpathrectangle{\pgfqpoint{0.100000in}{0.100000in}}{\pgfqpoint{3.007045in}{1.925000in}}%
\pgfusepath{clip}%
\pgfsetbuttcap%
\pgfsetmiterjoin%
\definecolor{currentfill}{rgb}{0.211626,0.525352,0.752157}%
\pgfsetfillcolor{currentfill}%
\pgfsetlinewidth{0.000000pt}%
\definecolor{currentstroke}{rgb}{0.000000,0.000000,0.000000}%
\pgfsetstrokecolor{currentstroke}%
\pgfsetstrokeopacity{0.000000}%
\pgfsetdash{}{0pt}%
\pgfpathmoveto{\pgfqpoint{2.283146in}{1.250837in}}%
\pgfpathlineto{\pgfqpoint{2.283471in}{1.248041in}}%
\pgfpathlineto{\pgfqpoint{2.262780in}{1.245532in}}%
\pgfpathlineto{\pgfqpoint{2.261811in}{1.262682in}}%
\pgfpathlineto{\pgfqpoint{2.249581in}{1.261349in}}%
\pgfpathlineto{\pgfqpoint{2.248890in}{1.267034in}}%
\pgfpathlineto{\pgfqpoint{2.239459in}{1.266008in}}%
\pgfpathlineto{\pgfqpoint{2.236831in}{1.288875in}}%
\pgfpathlineto{\pgfqpoint{2.234886in}{1.288646in}}%
\pgfpathlineto{\pgfqpoint{2.233530in}{1.300204in}}%
\pgfpathlineto{\pgfqpoint{2.234453in}{1.308012in}}%
\pgfpathlineto{\pgfqpoint{2.240301in}{1.306723in}}%
\pgfpathlineto{\pgfqpoint{2.257315in}{1.308409in}}%
\pgfpathlineto{\pgfqpoint{2.253575in}{1.340858in}}%
\pgfpathlineto{\pgfqpoint{2.271732in}{1.342879in}}%
\pgfpathlineto{\pgfqpoint{2.273138in}{1.338797in}}%
\pgfpathlineto{\pgfqpoint{2.278491in}{1.292509in}}%
\pgfpathlineto{\pgfqpoint{2.283146in}{1.250837in}}%
\pgfpathclose%
\pgfusepath{fill}%
\end{pgfscope}%
\begin{pgfscope}%
\pgfpathrectangle{\pgfqpoint{0.100000in}{0.100000in}}{\pgfqpoint{3.007045in}{1.925000in}}%
\pgfusepath{clip}%
\pgfsetbuttcap%
\pgfsetmiterjoin%
\definecolor{currentfill}{rgb}{0.371688,0.649627,0.820515}%
\pgfsetfillcolor{currentfill}%
\pgfsetlinewidth{0.000000pt}%
\definecolor{currentstroke}{rgb}{0.000000,0.000000,0.000000}%
\pgfsetstrokecolor{currentstroke}%
\pgfsetstrokeopacity{0.000000}%
\pgfsetdash{}{0pt}%
\pgfpathmoveto{\pgfqpoint{2.380873in}{1.017870in}}%
\pgfpathlineto{\pgfqpoint{2.378204in}{1.022084in}}%
\pgfpathlineto{\pgfqpoint{2.378235in}{1.029769in}}%
\pgfpathlineto{\pgfqpoint{2.376071in}{1.035433in}}%
\pgfpathlineto{\pgfqpoint{2.371954in}{1.050139in}}%
\pgfpathlineto{\pgfqpoint{2.372484in}{1.056245in}}%
\pgfpathlineto{\pgfqpoint{2.370868in}{1.060399in}}%
\pgfpathlineto{\pgfqpoint{2.375741in}{1.064118in}}%
\pgfpathlineto{\pgfqpoint{2.378157in}{1.062296in}}%
\pgfpathlineto{\pgfqpoint{2.384647in}{1.067115in}}%
\pgfpathlineto{\pgfqpoint{2.392156in}{1.068534in}}%
\pgfpathlineto{\pgfqpoint{2.392967in}{1.076431in}}%
\pgfpathlineto{\pgfqpoint{2.400437in}{1.076137in}}%
\pgfpathlineto{\pgfqpoint{2.410452in}{1.072460in}}%
\pgfpathlineto{\pgfqpoint{2.412270in}{1.064935in}}%
\pgfpathlineto{\pgfqpoint{2.417226in}{1.060057in}}%
\pgfpathlineto{\pgfqpoint{2.423037in}{1.058784in}}%
\pgfpathlineto{\pgfqpoint{2.415558in}{1.052709in}}%
\pgfpathlineto{\pgfqpoint{2.415182in}{1.047525in}}%
\pgfpathlineto{\pgfqpoint{2.409125in}{1.041720in}}%
\pgfpathlineto{\pgfqpoint{2.409387in}{1.036508in}}%
\pgfpathlineto{\pgfqpoint{2.399471in}{1.032629in}}%
\pgfpathlineto{\pgfqpoint{2.397238in}{1.025171in}}%
\pgfpathlineto{\pgfqpoint{2.380873in}{1.017870in}}%
\pgfpathclose%
\pgfusepath{fill}%
\end{pgfscope}%
\begin{pgfscope}%
\pgfpathrectangle{\pgfqpoint{0.100000in}{0.100000in}}{\pgfqpoint{3.007045in}{1.925000in}}%
\pgfusepath{clip}%
\pgfsetbuttcap%
\pgfsetmiterjoin%
\definecolor{currentfill}{rgb}{0.346467,0.632403,0.810673}%
\pgfsetfillcolor{currentfill}%
\pgfsetlinewidth{0.000000pt}%
\definecolor{currentstroke}{rgb}{0.000000,0.000000,0.000000}%
\pgfsetstrokecolor{currentstroke}%
\pgfsetstrokeopacity{0.000000}%
\pgfsetdash{}{0pt}%
\pgfpathmoveto{\pgfqpoint{2.508172in}{0.896146in}}%
\pgfpathlineto{\pgfqpoint{2.508933in}{0.891663in}}%
\pgfpathlineto{\pgfqpoint{2.502098in}{0.888878in}}%
\pgfpathlineto{\pgfqpoint{2.496666in}{0.891862in}}%
\pgfpathlineto{\pgfqpoint{2.481602in}{0.894314in}}%
\pgfpathlineto{\pgfqpoint{2.476040in}{0.897437in}}%
\pgfpathlineto{\pgfqpoint{2.462329in}{0.908061in}}%
\pgfpathlineto{\pgfqpoint{2.459680in}{0.930712in}}%
\pgfpathlineto{\pgfqpoint{2.452389in}{0.929337in}}%
\pgfpathlineto{\pgfqpoint{2.451924in}{0.935651in}}%
\pgfpathlineto{\pgfqpoint{2.455447in}{0.943242in}}%
\pgfpathlineto{\pgfqpoint{2.460723in}{0.945065in}}%
\pgfpathlineto{\pgfqpoint{2.472311in}{0.936614in}}%
\pgfpathlineto{\pgfqpoint{2.472948in}{0.932025in}}%
\pgfpathlineto{\pgfqpoint{2.505332in}{0.935315in}}%
\pgfpathlineto{\pgfqpoint{2.504209in}{0.927233in}}%
\pgfpathlineto{\pgfqpoint{2.500183in}{0.925858in}}%
\pgfpathlineto{\pgfqpoint{2.503825in}{0.913860in}}%
\pgfpathlineto{\pgfqpoint{2.505466in}{0.902631in}}%
\pgfpathlineto{\pgfqpoint{2.508172in}{0.896146in}}%
\pgfpathclose%
\pgfusepath{fill}%
\end{pgfscope}%
\begin{pgfscope}%
\pgfpathrectangle{\pgfqpoint{0.100000in}{0.100000in}}{\pgfqpoint{3.007045in}{1.925000in}}%
\pgfusepath{clip}%
\pgfsetbuttcap%
\pgfsetmiterjoin%
\definecolor{currentfill}{rgb}{0.093272,0.396878,0.673664}%
\pgfsetfillcolor{currentfill}%
\pgfsetlinewidth{0.000000pt}%
\definecolor{currentstroke}{rgb}{0.000000,0.000000,0.000000}%
\pgfsetstrokecolor{currentstroke}%
\pgfsetstrokeopacity{0.000000}%
\pgfsetdash{}{0pt}%
\pgfpathmoveto{\pgfqpoint{0.928770in}{0.169774in}}%
\pgfpathlineto{\pgfqpoint{0.929087in}{0.172237in}}%
\pgfpathlineto{\pgfqpoint{0.930780in}{0.170565in}}%
\pgfpathlineto{\pgfqpoint{0.928770in}{0.169774in}}%
\pgfpathclose%
\pgfusepath{fill}%
\end{pgfscope}%
\begin{pgfscope}%
\pgfpathrectangle{\pgfqpoint{0.100000in}{0.100000in}}{\pgfqpoint{3.007045in}{1.925000in}}%
\pgfusepath{clip}%
\pgfsetbuttcap%
\pgfsetmiterjoin%
\definecolor{currentfill}{rgb}{0.093272,0.396878,0.673664}%
\pgfsetfillcolor{currentfill}%
\pgfsetlinewidth{0.000000pt}%
\definecolor{currentstroke}{rgb}{0.000000,0.000000,0.000000}%
\pgfsetstrokecolor{currentstroke}%
\pgfsetstrokeopacity{0.000000}%
\pgfsetdash{}{0pt}%
\pgfpathmoveto{\pgfqpoint{0.927632in}{0.179344in}}%
\pgfpathlineto{\pgfqpoint{0.926956in}{0.181069in}}%
\pgfpathlineto{\pgfqpoint{0.926534in}{0.189086in}}%
\pgfpathlineto{\pgfqpoint{0.927077in}{0.189993in}}%
\pgfpathlineto{\pgfqpoint{0.926777in}{0.193425in}}%
\pgfpathlineto{\pgfqpoint{0.927516in}{0.194491in}}%
\pgfpathlineto{\pgfqpoint{0.926137in}{0.195868in}}%
\pgfpathlineto{\pgfqpoint{0.925991in}{0.197780in}}%
\pgfpathlineto{\pgfqpoint{0.928814in}{0.198830in}}%
\pgfpathlineto{\pgfqpoint{0.929984in}{0.200875in}}%
\pgfpathlineto{\pgfqpoint{0.928490in}{0.201725in}}%
\pgfpathlineto{\pgfqpoint{0.928312in}{0.203305in}}%
\pgfpathlineto{\pgfqpoint{0.929750in}{0.205372in}}%
\pgfpathlineto{\pgfqpoint{0.928020in}{0.209653in}}%
\pgfpathlineto{\pgfqpoint{0.932348in}{0.213098in}}%
\pgfpathlineto{\pgfqpoint{0.934167in}{0.209712in}}%
\pgfpathlineto{\pgfqpoint{0.937918in}{0.207074in}}%
\pgfpathlineto{\pgfqpoint{0.935283in}{0.206082in}}%
\pgfpathlineto{\pgfqpoint{0.936138in}{0.203890in}}%
\pgfpathlineto{\pgfqpoint{0.935915in}{0.201464in}}%
\pgfpathlineto{\pgfqpoint{0.934812in}{0.198205in}}%
\pgfpathlineto{\pgfqpoint{0.934411in}{0.195163in}}%
\pgfpathlineto{\pgfqpoint{0.933099in}{0.192388in}}%
\pgfpathlineto{\pgfqpoint{0.933499in}{0.191353in}}%
\pgfpathlineto{\pgfqpoint{0.932518in}{0.188498in}}%
\pgfpathlineto{\pgfqpoint{0.930488in}{0.187744in}}%
\pgfpathlineto{\pgfqpoint{0.931046in}{0.186350in}}%
\pgfpathlineto{\pgfqpoint{0.929012in}{0.184009in}}%
\pgfpathlineto{\pgfqpoint{0.927632in}{0.179344in}}%
\pgfpathclose%
\pgfusepath{fill}%
\end{pgfscope}%
\begin{pgfscope}%
\pgfpathrectangle{\pgfqpoint{0.100000in}{0.100000in}}{\pgfqpoint{3.007045in}{1.925000in}}%
\pgfusepath{clip}%
\pgfsetbuttcap%
\pgfsetmiterjoin%
\definecolor{currentfill}{rgb}{0.093272,0.396878,0.673664}%
\pgfsetfillcolor{currentfill}%
\pgfsetlinewidth{0.000000pt}%
\definecolor{currentstroke}{rgb}{0.000000,0.000000,0.000000}%
\pgfsetstrokecolor{currentstroke}%
\pgfsetstrokeopacity{0.000000}%
\pgfsetdash{}{0pt}%
\pgfpathmoveto{\pgfqpoint{0.921995in}{0.203350in}}%
\pgfpathlineto{\pgfqpoint{0.924523in}{0.205630in}}%
\pgfpathlineto{\pgfqpoint{0.924613in}{0.208652in}}%
\pgfpathlineto{\pgfqpoint{0.925424in}{0.210159in}}%
\pgfpathlineto{\pgfqpoint{0.927073in}{0.209363in}}%
\pgfpathlineto{\pgfqpoint{0.926551in}{0.205362in}}%
\pgfpathlineto{\pgfqpoint{0.925876in}{0.203745in}}%
\pgfpathlineto{\pgfqpoint{0.924259in}{0.202862in}}%
\pgfpathlineto{\pgfqpoint{0.921995in}{0.203350in}}%
\pgfpathclose%
\pgfusepath{fill}%
\end{pgfscope}%
\begin{pgfscope}%
\pgfpathrectangle{\pgfqpoint{0.100000in}{0.100000in}}{\pgfqpoint{3.007045in}{1.925000in}}%
\pgfusepath{clip}%
\pgfsetbuttcap%
\pgfsetmiterjoin%
\definecolor{currentfill}{rgb}{0.093272,0.396878,0.673664}%
\pgfsetfillcolor{currentfill}%
\pgfsetlinewidth{0.000000pt}%
\definecolor{currentstroke}{rgb}{0.000000,0.000000,0.000000}%
\pgfsetstrokecolor{currentstroke}%
\pgfsetstrokeopacity{0.000000}%
\pgfsetdash{}{0pt}%
\pgfpathmoveto{\pgfqpoint{0.924230in}{0.224566in}}%
\pgfpathlineto{\pgfqpoint{0.928775in}{0.222468in}}%
\pgfpathlineto{\pgfqpoint{0.931898in}{0.225478in}}%
\pgfpathlineto{\pgfqpoint{0.933152in}{0.222753in}}%
\pgfpathlineto{\pgfqpoint{0.934778in}{0.220682in}}%
\pgfpathlineto{\pgfqpoint{0.936886in}{0.216547in}}%
\pgfpathlineto{\pgfqpoint{0.939636in}{0.215477in}}%
\pgfpathlineto{\pgfqpoint{0.940303in}{0.214612in}}%
\pgfpathlineto{\pgfqpoint{0.938629in}{0.208282in}}%
\pgfpathlineto{\pgfqpoint{0.936276in}{0.208683in}}%
\pgfpathlineto{\pgfqpoint{0.931068in}{0.218130in}}%
\pgfpathlineto{\pgfqpoint{0.930637in}{0.216360in}}%
\pgfpathlineto{\pgfqpoint{0.931418in}{0.214692in}}%
\pgfpathlineto{\pgfqpoint{0.930183in}{0.212277in}}%
\pgfpathlineto{\pgfqpoint{0.927651in}{0.210591in}}%
\pgfpathlineto{\pgfqpoint{0.926290in}{0.211782in}}%
\pgfpathlineto{\pgfqpoint{0.925901in}{0.214793in}}%
\pgfpathlineto{\pgfqpoint{0.926598in}{0.218053in}}%
\pgfpathlineto{\pgfqpoint{0.924230in}{0.224566in}}%
\pgfpathclose%
\pgfusepath{fill}%
\end{pgfscope}%
\begin{pgfscope}%
\pgfpathrectangle{\pgfqpoint{0.100000in}{0.100000in}}{\pgfqpoint{3.007045in}{1.925000in}}%
\pgfusepath{clip}%
\pgfsetbuttcap%
\pgfsetmiterjoin%
\definecolor{currentfill}{rgb}{0.093272,0.396878,0.673664}%
\pgfsetfillcolor{currentfill}%
\pgfsetlinewidth{0.000000pt}%
\definecolor{currentstroke}{rgb}{0.000000,0.000000,0.000000}%
\pgfsetstrokecolor{currentstroke}%
\pgfsetstrokeopacity{0.000000}%
\pgfsetdash{}{0pt}%
\pgfpathmoveto{\pgfqpoint{0.949543in}{0.170923in}}%
\pgfpathlineto{\pgfqpoint{0.948341in}{0.174468in}}%
\pgfpathlineto{\pgfqpoint{0.951022in}{0.176053in}}%
\pgfpathlineto{\pgfqpoint{0.953034in}{0.175105in}}%
\pgfpathlineto{\pgfqpoint{0.953826in}{0.173319in}}%
\pgfpathlineto{\pgfqpoint{0.951936in}{0.170704in}}%
\pgfpathlineto{\pgfqpoint{0.949543in}{0.170923in}}%
\pgfpathclose%
\pgfusepath{fill}%
\end{pgfscope}%
\begin{pgfscope}%
\pgfpathrectangle{\pgfqpoint{0.100000in}{0.100000in}}{\pgfqpoint{3.007045in}{1.925000in}}%
\pgfusepath{clip}%
\pgfsetbuttcap%
\pgfsetmiterjoin%
\definecolor{currentfill}{rgb}{0.093272,0.396878,0.673664}%
\pgfsetfillcolor{currentfill}%
\pgfsetlinewidth{0.000000pt}%
\definecolor{currentstroke}{rgb}{0.000000,0.000000,0.000000}%
\pgfsetstrokecolor{currentstroke}%
\pgfsetstrokeopacity{0.000000}%
\pgfsetdash{}{0pt}%
\pgfpathmoveto{\pgfqpoint{0.941866in}{0.183097in}}%
\pgfpathlineto{\pgfqpoint{0.939091in}{0.182574in}}%
\pgfpathlineto{\pgfqpoint{0.938900in}{0.178993in}}%
\pgfpathlineto{\pgfqpoint{0.936985in}{0.178377in}}%
\pgfpathlineto{\pgfqpoint{0.937239in}{0.175986in}}%
\pgfpathlineto{\pgfqpoint{0.935359in}{0.173870in}}%
\pgfpathlineto{\pgfqpoint{0.934641in}{0.176663in}}%
\pgfpathlineto{\pgfqpoint{0.933054in}{0.175528in}}%
\pgfpathlineto{\pgfqpoint{0.932146in}{0.176239in}}%
\pgfpathlineto{\pgfqpoint{0.933479in}{0.179062in}}%
\pgfpathlineto{\pgfqpoint{0.935593in}{0.180734in}}%
\pgfpathlineto{\pgfqpoint{0.937053in}{0.180678in}}%
\pgfpathlineto{\pgfqpoint{0.938116in}{0.183219in}}%
\pgfpathlineto{\pgfqpoint{0.936870in}{0.183669in}}%
\pgfpathlineto{\pgfqpoint{0.935994in}{0.185420in}}%
\pgfpathlineto{\pgfqpoint{0.936562in}{0.187485in}}%
\pgfpathlineto{\pgfqpoint{0.936865in}{0.191439in}}%
\pgfpathlineto{\pgfqpoint{0.939368in}{0.192607in}}%
\pgfpathlineto{\pgfqpoint{0.942659in}{0.187128in}}%
\pgfpathlineto{\pgfqpoint{0.943420in}{0.184913in}}%
\pgfpathlineto{\pgfqpoint{0.941866in}{0.183097in}}%
\pgfpathclose%
\pgfusepath{fill}%
\end{pgfscope}%
\begin{pgfscope}%
\pgfpathrectangle{\pgfqpoint{0.100000in}{0.100000in}}{\pgfqpoint{3.007045in}{1.925000in}}%
\pgfusepath{clip}%
\pgfsetbuttcap%
\pgfsetmiterjoin%
\definecolor{currentfill}{rgb}{0.093272,0.396878,0.673664}%
\pgfsetfillcolor{currentfill}%
\pgfsetlinewidth{0.000000pt}%
\definecolor{currentstroke}{rgb}{0.000000,0.000000,0.000000}%
\pgfsetstrokecolor{currentstroke}%
\pgfsetstrokeopacity{0.000000}%
\pgfsetdash{}{0pt}%
\pgfpathmoveto{\pgfqpoint{0.951030in}{0.194631in}}%
\pgfpathlineto{\pgfqpoint{0.954245in}{0.194809in}}%
\pgfpathlineto{\pgfqpoint{0.955692in}{0.192812in}}%
\pgfpathlineto{\pgfqpoint{0.957247in}{0.193413in}}%
\pgfpathlineto{\pgfqpoint{0.958655in}{0.195219in}}%
\pgfpathlineto{\pgfqpoint{0.963493in}{0.195565in}}%
\pgfpathlineto{\pgfqpoint{0.964721in}{0.193135in}}%
\pgfpathlineto{\pgfqpoint{0.966052in}{0.192399in}}%
\pgfpathlineto{\pgfqpoint{0.966135in}{0.189796in}}%
\pgfpathlineto{\pgfqpoint{0.963633in}{0.186915in}}%
\pgfpathlineto{\pgfqpoint{0.966486in}{0.184667in}}%
\pgfpathlineto{\pgfqpoint{0.964468in}{0.180678in}}%
\pgfpathlineto{\pgfqpoint{0.966318in}{0.178588in}}%
\pgfpathlineto{\pgfqpoint{0.965355in}{0.173338in}}%
\pgfpathlineto{\pgfqpoint{0.967092in}{0.172775in}}%
\pgfpathlineto{\pgfqpoint{0.969300in}{0.170288in}}%
\pgfpathlineto{\pgfqpoint{0.972690in}{0.165137in}}%
\pgfpathlineto{\pgfqpoint{0.972584in}{0.162951in}}%
\pgfpathlineto{\pgfqpoint{0.970864in}{0.160909in}}%
\pgfpathlineto{\pgfqpoint{0.968598in}{0.161733in}}%
\pgfpathlineto{\pgfqpoint{0.965820in}{0.161252in}}%
\pgfpathlineto{\pgfqpoint{0.966014in}{0.159668in}}%
\pgfpathlineto{\pgfqpoint{0.964181in}{0.158832in}}%
\pgfpathlineto{\pgfqpoint{0.962912in}{0.161570in}}%
\pgfpathlineto{\pgfqpoint{0.960550in}{0.161270in}}%
\pgfpathlineto{\pgfqpoint{0.958681in}{0.160262in}}%
\pgfpathlineto{\pgfqpoint{0.958436in}{0.159148in}}%
\pgfpathlineto{\pgfqpoint{0.956199in}{0.158645in}}%
\pgfpathlineto{\pgfqpoint{0.956678in}{0.156700in}}%
\pgfpathlineto{\pgfqpoint{0.955416in}{0.154694in}}%
\pgfpathlineto{\pgfqpoint{0.953820in}{0.154455in}}%
\pgfpathlineto{\pgfqpoint{0.952217in}{0.157810in}}%
\pgfpathlineto{\pgfqpoint{0.955289in}{0.158119in}}%
\pgfpathlineto{\pgfqpoint{0.956244in}{0.159613in}}%
\pgfpathlineto{\pgfqpoint{0.958348in}{0.160762in}}%
\pgfpathlineto{\pgfqpoint{0.959317in}{0.163914in}}%
\pgfpathlineto{\pgfqpoint{0.960954in}{0.164908in}}%
\pgfpathlineto{\pgfqpoint{0.959218in}{0.166639in}}%
\pgfpathlineto{\pgfqpoint{0.958611in}{0.164391in}}%
\pgfpathlineto{\pgfqpoint{0.955800in}{0.164056in}}%
\pgfpathlineto{\pgfqpoint{0.954820in}{0.161400in}}%
\pgfpathlineto{\pgfqpoint{0.952564in}{0.162138in}}%
\pgfpathlineto{\pgfqpoint{0.952336in}{0.166328in}}%
\pgfpathlineto{\pgfqpoint{0.950396in}{0.166275in}}%
\pgfpathlineto{\pgfqpoint{0.951336in}{0.169331in}}%
\pgfpathlineto{\pgfqpoint{0.955934in}{0.171121in}}%
\pgfpathlineto{\pgfqpoint{0.955518in}{0.168806in}}%
\pgfpathlineto{\pgfqpoint{0.956516in}{0.167155in}}%
\pgfpathlineto{\pgfqpoint{0.957594in}{0.173312in}}%
\pgfpathlineto{\pgfqpoint{0.958602in}{0.175079in}}%
\pgfpathlineto{\pgfqpoint{0.961488in}{0.177541in}}%
\pgfpathlineto{\pgfqpoint{0.957917in}{0.179285in}}%
\pgfpathlineto{\pgfqpoint{0.958031in}{0.180854in}}%
\pgfpathlineto{\pgfqpoint{0.956711in}{0.183035in}}%
\pgfpathlineto{\pgfqpoint{0.956083in}{0.186705in}}%
\pgfpathlineto{\pgfqpoint{0.957478in}{0.186958in}}%
\pgfpathlineto{\pgfqpoint{0.957475in}{0.188952in}}%
\pgfpathlineto{\pgfqpoint{0.955586in}{0.189502in}}%
\pgfpathlineto{\pgfqpoint{0.954193in}{0.191220in}}%
\pgfpathlineto{\pgfqpoint{0.953928in}{0.193026in}}%
\pgfpathlineto{\pgfqpoint{0.952297in}{0.192656in}}%
\pgfpathlineto{\pgfqpoint{0.951030in}{0.194631in}}%
\pgfpathclose%
\pgfusepath{fill}%
\end{pgfscope}%
\begin{pgfscope}%
\pgfpathrectangle{\pgfqpoint{0.100000in}{0.100000in}}{\pgfqpoint{3.007045in}{1.925000in}}%
\pgfusepath{clip}%
\pgfsetbuttcap%
\pgfsetmiterjoin%
\definecolor{currentfill}{rgb}{0.093272,0.396878,0.673664}%
\pgfsetfillcolor{currentfill}%
\pgfsetlinewidth{0.000000pt}%
\definecolor{currentstroke}{rgb}{0.000000,0.000000,0.000000}%
\pgfsetstrokecolor{currentstroke}%
\pgfsetstrokeopacity{0.000000}%
\pgfsetdash{}{0pt}%
\pgfpathmoveto{\pgfqpoint{0.942241in}{0.179713in}}%
\pgfpathlineto{\pgfqpoint{0.944515in}{0.186094in}}%
\pgfpathlineto{\pgfqpoint{0.944156in}{0.190021in}}%
\pgfpathlineto{\pgfqpoint{0.944652in}{0.190685in}}%
\pgfpathlineto{\pgfqpoint{0.943856in}{0.194420in}}%
\pgfpathlineto{\pgfqpoint{0.945632in}{0.194990in}}%
\pgfpathlineto{\pgfqpoint{0.951121in}{0.190206in}}%
\pgfpathlineto{\pgfqpoint{0.954271in}{0.188027in}}%
\pgfpathlineto{\pgfqpoint{0.954701in}{0.185800in}}%
\pgfpathlineto{\pgfqpoint{0.953794in}{0.184014in}}%
\pgfpathlineto{\pgfqpoint{0.955507in}{0.182225in}}%
\pgfpathlineto{\pgfqpoint{0.956828in}{0.176810in}}%
\pgfpathlineto{\pgfqpoint{0.953089in}{0.176262in}}%
\pgfpathlineto{\pgfqpoint{0.951273in}{0.177255in}}%
\pgfpathlineto{\pgfqpoint{0.953068in}{0.180485in}}%
\pgfpathlineto{\pgfqpoint{0.951111in}{0.180223in}}%
\pgfpathlineto{\pgfqpoint{0.949807in}{0.178518in}}%
\pgfpathlineto{\pgfqpoint{0.947328in}{0.177601in}}%
\pgfpathlineto{\pgfqpoint{0.945308in}{0.179613in}}%
\pgfpathlineto{\pgfqpoint{0.944254in}{0.178896in}}%
\pgfpathlineto{\pgfqpoint{0.942241in}{0.179713in}}%
\pgfpathclose%
\pgfusepath{fill}%
\end{pgfscope}%
\begin{pgfscope}%
\pgfpathrectangle{\pgfqpoint{0.100000in}{0.100000in}}{\pgfqpoint{3.007045in}{1.925000in}}%
\pgfusepath{clip}%
\pgfsetbuttcap%
\pgfsetmiterjoin%
\definecolor{currentfill}{rgb}{0.735871,0.841569,0.923045}%
\pgfsetfillcolor{currentfill}%
\pgfsetlinewidth{0.000000pt}%
\definecolor{currentstroke}{rgb}{0.000000,0.000000,0.000000}%
\pgfsetstrokecolor{currentstroke}%
\pgfsetstrokeopacity{0.000000}%
\pgfsetdash{}{0pt}%
\pgfpathmoveto{\pgfqpoint{0.981446in}{1.653571in}}%
\pgfpathlineto{\pgfqpoint{0.966686in}{1.656395in}}%
\pgfpathlineto{\pgfqpoint{0.964271in}{1.652566in}}%
\pgfpathlineto{\pgfqpoint{0.957302in}{1.649942in}}%
\pgfpathlineto{\pgfqpoint{0.955593in}{1.658524in}}%
\pgfpathlineto{\pgfqpoint{0.944841in}{1.665499in}}%
\pgfpathlineto{\pgfqpoint{0.941856in}{1.670351in}}%
\pgfpathlineto{\pgfqpoint{0.936043in}{1.670714in}}%
\pgfpathlineto{\pgfqpoint{0.926926in}{1.665686in}}%
\pgfpathlineto{\pgfqpoint{0.923792in}{1.677916in}}%
\pgfpathlineto{\pgfqpoint{0.918191in}{1.680868in}}%
\pgfpathlineto{\pgfqpoint{0.917372in}{1.685810in}}%
\pgfpathlineto{\pgfqpoint{0.913359in}{1.689243in}}%
\pgfpathlineto{\pgfqpoint{0.914111in}{1.694407in}}%
\pgfpathlineto{\pgfqpoint{0.917747in}{1.701623in}}%
\pgfpathlineto{\pgfqpoint{0.917545in}{1.708081in}}%
\pgfpathlineto{\pgfqpoint{0.914970in}{1.712349in}}%
\pgfpathlineto{\pgfqpoint{0.915399in}{1.718366in}}%
\pgfpathlineto{\pgfqpoint{0.921093in}{1.727298in}}%
\pgfpathlineto{\pgfqpoint{0.926725in}{1.726127in}}%
\pgfpathlineto{\pgfqpoint{0.927484in}{1.729880in}}%
\pgfpathlineto{\pgfqpoint{0.936258in}{1.730024in}}%
\pgfpathlineto{\pgfqpoint{0.940260in}{1.735106in}}%
\pgfpathlineto{\pgfqpoint{0.945308in}{1.734086in}}%
\pgfpathlineto{\pgfqpoint{0.949935in}{1.756722in}}%
\pgfpathlineto{\pgfqpoint{0.942694in}{1.758174in}}%
\pgfpathlineto{\pgfqpoint{0.948336in}{1.785640in}}%
\pgfpathlineto{\pgfqpoint{0.962444in}{1.782513in}}%
\pgfpathlineto{\pgfqpoint{0.964785in}{1.774681in}}%
\pgfpathlineto{\pgfqpoint{0.960752in}{1.754525in}}%
\pgfpathlineto{\pgfqpoint{0.972359in}{1.752059in}}%
\pgfpathlineto{\pgfqpoint{0.967944in}{1.729505in}}%
\pgfpathlineto{\pgfqpoint{0.979570in}{1.727393in}}%
\pgfpathlineto{\pgfqpoint{0.977851in}{1.718526in}}%
\pgfpathlineto{\pgfqpoint{0.983438in}{1.717518in}}%
\pgfpathlineto{\pgfqpoint{0.987484in}{1.710382in}}%
\pgfpathlineto{\pgfqpoint{0.985045in}{1.700348in}}%
\pgfpathlineto{\pgfqpoint{0.982127in}{1.696489in}}%
\pgfpathlineto{\pgfqpoint{0.973346in}{1.694790in}}%
\pgfpathlineto{\pgfqpoint{0.971079in}{1.692732in}}%
\pgfpathlineto{\pgfqpoint{0.969357in}{1.685011in}}%
\pgfpathlineto{\pgfqpoint{0.973053in}{1.678906in}}%
\pgfpathlineto{\pgfqpoint{0.972309in}{1.661350in}}%
\pgfpathlineto{\pgfqpoint{0.981446in}{1.653571in}}%
\pgfpathclose%
\pgfusepath{fill}%
\end{pgfscope}%
\begin{pgfscope}%
\pgfpathrectangle{\pgfqpoint{0.100000in}{0.100000in}}{\pgfqpoint{3.007045in}{1.925000in}}%
\pgfusepath{clip}%
\pgfsetbuttcap%
\pgfsetmiterjoin%
\definecolor{currentfill}{rgb}{0.179146,0.492872,0.735425}%
\pgfsetfillcolor{currentfill}%
\pgfsetlinewidth{0.000000pt}%
\definecolor{currentstroke}{rgb}{0.000000,0.000000,0.000000}%
\pgfsetstrokecolor{currentstroke}%
\pgfsetstrokeopacity{0.000000}%
\pgfsetdash{}{0pt}%
\pgfpathmoveto{\pgfqpoint{1.631190in}{1.247136in}}%
\pgfpathlineto{\pgfqpoint{1.586476in}{1.248770in}}%
\pgfpathlineto{\pgfqpoint{1.587386in}{1.271651in}}%
\pgfpathlineto{\pgfqpoint{1.586068in}{1.271711in}}%
\pgfpathlineto{\pgfqpoint{1.587220in}{1.294590in}}%
\pgfpathlineto{\pgfqpoint{1.609594in}{1.293697in}}%
\pgfpathlineto{\pgfqpoint{1.610083in}{1.285141in}}%
\pgfpathlineto{\pgfqpoint{1.618684in}{1.284817in}}%
\pgfpathlineto{\pgfqpoint{1.621155in}{1.286648in}}%
\pgfpathlineto{\pgfqpoint{1.638392in}{1.287167in}}%
\pgfpathlineto{\pgfqpoint{1.638646in}{1.292816in}}%
\pgfpathlineto{\pgfqpoint{1.643851in}{1.292672in}}%
\pgfpathlineto{\pgfqpoint{1.643733in}{1.288581in}}%
\pgfpathlineto{\pgfqpoint{1.635249in}{1.282362in}}%
\pgfpathlineto{\pgfqpoint{1.632031in}{1.278421in}}%
\pgfpathlineto{\pgfqpoint{1.631190in}{1.247136in}}%
\pgfpathclose%
\pgfusepath{fill}%
\end{pgfscope}%
\begin{pgfscope}%
\pgfpathrectangle{\pgfqpoint{0.100000in}{0.100000in}}{\pgfqpoint{3.007045in}{1.925000in}}%
\pgfusepath{clip}%
\pgfsetbuttcap%
\pgfsetmiterjoin%
\definecolor{currentfill}{rgb}{0.260715,0.573841,0.777209}%
\pgfsetfillcolor{currentfill}%
\pgfsetlinewidth{0.000000pt}%
\definecolor{currentstroke}{rgb}{0.000000,0.000000,0.000000}%
\pgfsetstrokecolor{currentstroke}%
\pgfsetstrokeopacity{0.000000}%
\pgfsetdash{}{0pt}%
\pgfpathmoveto{\pgfqpoint{0.683125in}{1.670416in}}%
\pgfpathlineto{\pgfqpoint{0.690132in}{1.668259in}}%
\pgfpathlineto{\pgfqpoint{0.695518in}{1.663103in}}%
\pgfpathlineto{\pgfqpoint{0.694447in}{1.659187in}}%
\pgfpathlineto{\pgfqpoint{0.689503in}{1.657799in}}%
\pgfpathlineto{\pgfqpoint{0.689599in}{1.651922in}}%
\pgfpathlineto{\pgfqpoint{0.678258in}{1.651609in}}%
\pgfpathlineto{\pgfqpoint{0.678193in}{1.649100in}}%
\pgfpathlineto{\pgfqpoint{0.684947in}{1.644999in}}%
\pgfpathlineto{\pgfqpoint{0.685898in}{1.641698in}}%
\pgfpathlineto{\pgfqpoint{0.678278in}{1.634204in}}%
\pgfpathlineto{\pgfqpoint{0.677402in}{1.629304in}}%
\pgfpathlineto{\pgfqpoint{0.671575in}{1.622748in}}%
\pgfpathlineto{\pgfqpoint{0.683738in}{1.619619in}}%
\pgfpathlineto{\pgfqpoint{0.680340in}{1.605736in}}%
\pgfpathlineto{\pgfqpoint{0.653418in}{1.613391in}}%
\pgfpathlineto{\pgfqpoint{0.651898in}{1.607795in}}%
\pgfpathlineto{\pgfqpoint{0.613279in}{1.618283in}}%
\pgfpathlineto{\pgfqpoint{0.619536in}{1.640423in}}%
\pgfpathlineto{\pgfqpoint{0.628849in}{1.673291in}}%
\pgfpathlineto{\pgfqpoint{0.630991in}{1.684466in}}%
\pgfpathlineto{\pgfqpoint{0.683125in}{1.670416in}}%
\pgfpathclose%
\pgfusepath{fill}%
\end{pgfscope}%
\begin{pgfscope}%
\pgfpathrectangle{\pgfqpoint{0.100000in}{0.100000in}}{\pgfqpoint{3.007045in}{1.925000in}}%
\pgfusepath{clip}%
\pgfsetbuttcap%
\pgfsetmiterjoin%
\definecolor{currentfill}{rgb}{0.031757,0.318139,0.612149}%
\pgfsetfillcolor{currentfill}%
\pgfsetlinewidth{0.000000pt}%
\definecolor{currentstroke}{rgb}{0.000000,0.000000,0.000000}%
\pgfsetstrokecolor{currentstroke}%
\pgfsetstrokeopacity{0.000000}%
\pgfsetdash{}{0pt}%
\pgfpathmoveto{\pgfqpoint{1.072883in}{1.296139in}}%
\pgfpathlineto{\pgfqpoint{1.062663in}{1.231515in}}%
\pgfpathlineto{\pgfqpoint{1.061002in}{1.220921in}}%
\pgfpathlineto{\pgfqpoint{1.057846in}{1.219034in}}%
\pgfpathlineto{\pgfqpoint{1.011802in}{1.226459in}}%
\pgfpathlineto{\pgfqpoint{1.012248in}{1.233395in}}%
\pgfpathlineto{\pgfqpoint{1.015989in}{1.239279in}}%
\pgfpathlineto{\pgfqpoint{1.016300in}{1.244056in}}%
\pgfpathlineto{\pgfqpoint{1.022492in}{1.247657in}}%
\pgfpathlineto{\pgfqpoint{0.990893in}{1.253051in}}%
\pgfpathlineto{\pgfqpoint{0.974041in}{1.256437in}}%
\pgfpathlineto{\pgfqpoint{0.973337in}{1.262333in}}%
\pgfpathlineto{\pgfqpoint{0.981761in}{1.312933in}}%
\pgfpathlineto{\pgfqpoint{0.989953in}{1.316140in}}%
\pgfpathlineto{\pgfqpoint{0.994498in}{1.314874in}}%
\pgfpathlineto{\pgfqpoint{1.008923in}{1.315572in}}%
\pgfpathlineto{\pgfqpoint{1.015074in}{1.317685in}}%
\pgfpathlineto{\pgfqpoint{1.019386in}{1.315423in}}%
\pgfpathlineto{\pgfqpoint{1.027613in}{1.314125in}}%
\pgfpathlineto{\pgfqpoint{1.033610in}{1.309798in}}%
\pgfpathlineto{\pgfqpoint{1.050961in}{1.311223in}}%
\pgfpathlineto{\pgfqpoint{1.052269in}{1.305131in}}%
\pgfpathlineto{\pgfqpoint{1.057220in}{1.307768in}}%
\pgfpathlineto{\pgfqpoint{1.057913in}{1.312141in}}%
\pgfpathlineto{\pgfqpoint{1.067332in}{1.310685in}}%
\pgfpathlineto{\pgfqpoint{1.068540in}{1.305641in}}%
\pgfpathlineto{\pgfqpoint{1.067620in}{1.298977in}}%
\pgfpathlineto{\pgfqpoint{1.072883in}{1.296139in}}%
\pgfpathclose%
\pgfusepath{fill}%
\end{pgfscope}%
\begin{pgfscope}%
\pgfpathrectangle{\pgfqpoint{0.100000in}{0.100000in}}{\pgfqpoint{3.007045in}{1.925000in}}%
\pgfusepath{clip}%
\pgfsetbuttcap%
\pgfsetmiterjoin%
\definecolor{currentfill}{rgb}{0.280892,0.587620,0.785083}%
\pgfsetfillcolor{currentfill}%
\pgfsetlinewidth{0.000000pt}%
\definecolor{currentstroke}{rgb}{0.000000,0.000000,0.000000}%
\pgfsetstrokecolor{currentstroke}%
\pgfsetstrokeopacity{0.000000}%
\pgfsetdash{}{0pt}%
\pgfpathmoveto{\pgfqpoint{1.772350in}{0.979565in}}%
\pgfpathlineto{\pgfqpoint{1.772364in}{0.997534in}}%
\pgfpathlineto{\pgfqpoint{1.768458in}{1.001242in}}%
\pgfpathlineto{\pgfqpoint{1.768402in}{1.026569in}}%
\pgfpathlineto{\pgfqpoint{1.767740in}{1.049552in}}%
\pgfpathlineto{\pgfqpoint{1.768296in}{1.069654in}}%
\pgfpathlineto{\pgfqpoint{1.792356in}{1.069644in}}%
\pgfpathlineto{\pgfqpoint{1.792358in}{1.025328in}}%
\pgfpathlineto{\pgfqpoint{1.820409in}{1.024532in}}%
\pgfpathlineto{\pgfqpoint{1.822007in}{1.020648in}}%
\pgfpathlineto{\pgfqpoint{1.821509in}{0.984906in}}%
\pgfpathlineto{\pgfqpoint{1.792483in}{0.985928in}}%
\pgfpathlineto{\pgfqpoint{1.792522in}{0.979422in}}%
\pgfpathlineto{\pgfqpoint{1.772350in}{0.979565in}}%
\pgfpathclose%
\pgfusepath{fill}%
\end{pgfscope}%
\begin{pgfscope}%
\pgfpathrectangle{\pgfqpoint{0.100000in}{0.100000in}}{\pgfqpoint{3.007045in}{1.925000in}}%
\pgfusepath{clip}%
\pgfsetbuttcap%
\pgfsetmiterjoin%
\definecolor{currentfill}{rgb}{0.567059,0.763306,0.871019}%
\pgfsetfillcolor{currentfill}%
\pgfsetlinewidth{0.000000pt}%
\definecolor{currentstroke}{rgb}{0.000000,0.000000,0.000000}%
\pgfsetstrokecolor{currentstroke}%
\pgfsetstrokeopacity{0.000000}%
\pgfsetdash{}{0pt}%
\pgfpathmoveto{\pgfqpoint{2.584120in}{0.229934in}}%
\pgfpathlineto{\pgfqpoint{2.584920in}{0.232373in}}%
\pgfpathlineto{\pgfqpoint{2.594250in}{0.236772in}}%
\pgfpathlineto{\pgfqpoint{2.597851in}{0.242148in}}%
\pgfpathlineto{\pgfqpoint{2.603807in}{0.246137in}}%
\pgfpathlineto{\pgfqpoint{2.612413in}{0.245063in}}%
\pgfpathlineto{\pgfqpoint{2.616307in}{0.242603in}}%
\pgfpathlineto{\pgfqpoint{2.608919in}{0.238771in}}%
\pgfpathlineto{\pgfqpoint{2.602623in}{0.240537in}}%
\pgfpathlineto{\pgfqpoint{2.601276in}{0.237478in}}%
\pgfpathlineto{\pgfqpoint{2.591656in}{0.231982in}}%
\pgfpathlineto{\pgfqpoint{2.584120in}{0.229934in}}%
\pgfpathclose%
\pgfusepath{fill}%
\end{pgfscope}%
\begin{pgfscope}%
\pgfpathrectangle{\pgfqpoint{0.100000in}{0.100000in}}{\pgfqpoint{3.007045in}{1.925000in}}%
\pgfusepath{clip}%
\pgfsetbuttcap%
\pgfsetmiterjoin%
\definecolor{currentfill}{rgb}{0.567059,0.763306,0.871019}%
\pgfsetfillcolor{currentfill}%
\pgfsetlinewidth{0.000000pt}%
\definecolor{currentstroke}{rgb}{0.000000,0.000000,0.000000}%
\pgfsetstrokecolor{currentstroke}%
\pgfsetstrokeopacity{0.000000}%
\pgfsetdash{}{0pt}%
\pgfpathmoveto{\pgfqpoint{2.599950in}{0.318474in}}%
\pgfpathlineto{\pgfqpoint{2.628250in}{0.323017in}}%
\pgfpathlineto{\pgfqpoint{2.623105in}{0.353228in}}%
\pgfpathlineto{\pgfqpoint{2.622244in}{0.358169in}}%
\pgfpathlineto{\pgfqpoint{2.661292in}{0.364487in}}%
\pgfpathlineto{\pgfqpoint{2.670514in}{0.365060in}}%
\pgfpathlineto{\pgfqpoint{2.671006in}{0.349163in}}%
\pgfpathlineto{\pgfqpoint{2.673290in}{0.330949in}}%
\pgfpathlineto{\pgfqpoint{2.672635in}{0.325576in}}%
\pgfpathlineto{\pgfqpoint{2.666787in}{0.322924in}}%
\pgfpathlineto{\pgfqpoint{2.663898in}{0.305786in}}%
\pgfpathlineto{\pgfqpoint{2.665452in}{0.296790in}}%
\pgfpathlineto{\pgfqpoint{2.656966in}{0.286274in}}%
\pgfpathlineto{\pgfqpoint{2.652560in}{0.288018in}}%
\pgfpathlineto{\pgfqpoint{2.650535in}{0.283393in}}%
\pgfpathlineto{\pgfqpoint{2.645495in}{0.280171in}}%
\pgfpathlineto{\pgfqpoint{2.636303in}{0.281237in}}%
\pgfpathlineto{\pgfqpoint{2.633609in}{0.278212in}}%
\pgfpathlineto{\pgfqpoint{2.622499in}{0.274880in}}%
\pgfpathlineto{\pgfqpoint{2.616272in}{0.281159in}}%
\pgfpathlineto{\pgfqpoint{2.616427in}{0.288354in}}%
\pgfpathlineto{\pgfqpoint{2.614021in}{0.297409in}}%
\pgfpathlineto{\pgfqpoint{2.611003in}{0.299907in}}%
\pgfpathlineto{\pgfqpoint{2.607596in}{0.307893in}}%
\pgfpathlineto{\pgfqpoint{2.601344in}{0.312735in}}%
\pgfpathlineto{\pgfqpoint{2.599950in}{0.318474in}}%
\pgfpathclose%
\pgfusepath{fill}%
\end{pgfscope}%
\begin{pgfscope}%
\pgfpathrectangle{\pgfqpoint{0.100000in}{0.100000in}}{\pgfqpoint{3.007045in}{1.925000in}}%
\pgfusepath{clip}%
\pgfsetbuttcap%
\pgfsetmiterjoin%
\definecolor{currentfill}{rgb}{0.361599,0.642737,0.816578}%
\pgfsetfillcolor{currentfill}%
\pgfsetlinewidth{0.000000pt}%
\definecolor{currentstroke}{rgb}{0.000000,0.000000,0.000000}%
\pgfsetstrokecolor{currentstroke}%
\pgfsetstrokeopacity{0.000000}%
\pgfsetdash{}{0pt}%
\pgfpathmoveto{\pgfqpoint{2.139795in}{0.676959in}}%
\pgfpathlineto{\pgfqpoint{2.130566in}{0.673645in}}%
\pgfpathlineto{\pgfqpoint{2.115516in}{0.670799in}}%
\pgfpathlineto{\pgfqpoint{2.113560in}{0.697106in}}%
\pgfpathlineto{\pgfqpoint{2.090671in}{0.695575in}}%
\pgfpathlineto{\pgfqpoint{2.087936in}{0.742436in}}%
\pgfpathlineto{\pgfqpoint{2.116053in}{0.743899in}}%
\pgfpathlineto{\pgfqpoint{2.141820in}{0.745927in}}%
\pgfpathlineto{\pgfqpoint{2.141241in}{0.722568in}}%
\pgfpathlineto{\pgfqpoint{2.139795in}{0.676959in}}%
\pgfpathclose%
\pgfusepath{fill}%
\end{pgfscope}%
\begin{pgfscope}%
\pgfpathrectangle{\pgfqpoint{0.100000in}{0.100000in}}{\pgfqpoint{3.007045in}{1.925000in}}%
\pgfusepath{clip}%
\pgfsetbuttcap%
\pgfsetmiterjoin%
\definecolor{currentfill}{rgb}{0.485490,0.718524,0.853426}%
\pgfsetfillcolor{currentfill}%
\pgfsetlinewidth{0.000000pt}%
\definecolor{currentstroke}{rgb}{0.000000,0.000000,0.000000}%
\pgfsetstrokecolor{currentstroke}%
\pgfsetstrokeopacity{0.000000}%
\pgfsetdash{}{0pt}%
\pgfpathmoveto{\pgfqpoint{1.054085in}{0.578918in}}%
\pgfpathlineto{\pgfqpoint{1.057990in}{0.585618in}}%
\pgfpathlineto{\pgfqpoint{1.069716in}{0.583261in}}%
\pgfpathlineto{\pgfqpoint{1.065530in}{0.579554in}}%
\pgfpathlineto{\pgfqpoint{1.060064in}{0.581478in}}%
\pgfpathlineto{\pgfqpoint{1.054085in}{0.578918in}}%
\pgfpathclose%
\pgfusepath{fill}%
\end{pgfscope}%
\begin{pgfscope}%
\pgfpathrectangle{\pgfqpoint{0.100000in}{0.100000in}}{\pgfqpoint{3.007045in}{1.925000in}}%
\pgfusepath{clip}%
\pgfsetbuttcap%
\pgfsetmiterjoin%
\definecolor{currentfill}{rgb}{0.485490,0.718524,0.853426}%
\pgfsetfillcolor{currentfill}%
\pgfsetlinewidth{0.000000pt}%
\definecolor{currentstroke}{rgb}{0.000000,0.000000,0.000000}%
\pgfsetstrokecolor{currentstroke}%
\pgfsetstrokeopacity{0.000000}%
\pgfsetdash{}{0pt}%
\pgfpathmoveto{\pgfqpoint{1.094825in}{0.552023in}}%
\pgfpathlineto{\pgfqpoint{1.087738in}{0.562041in}}%
\pgfpathlineto{\pgfqpoint{1.084656in}{0.570735in}}%
\pgfpathlineto{\pgfqpoint{1.085398in}{0.574624in}}%
\pgfpathlineto{\pgfqpoint{1.094010in}{0.578238in}}%
\pgfpathlineto{\pgfqpoint{1.104098in}{0.576218in}}%
\pgfpathlineto{\pgfqpoint{1.107468in}{0.572007in}}%
\pgfpathlineto{\pgfqpoint{1.113289in}{0.568890in}}%
\pgfpathlineto{\pgfqpoint{1.115269in}{0.562992in}}%
\pgfpathlineto{\pgfqpoint{1.112904in}{0.558446in}}%
\pgfpathlineto{\pgfqpoint{1.108126in}{0.556840in}}%
\pgfpathlineto{\pgfqpoint{1.104455in}{0.552095in}}%
\pgfpathlineto{\pgfqpoint{1.094825in}{0.552023in}}%
\pgfpathclose%
\pgfusepath{fill}%
\end{pgfscope}%
\begin{pgfscope}%
\pgfpathrectangle{\pgfqpoint{0.100000in}{0.100000in}}{\pgfqpoint{3.007045in}{1.925000in}}%
\pgfusepath{clip}%
\pgfsetbuttcap%
\pgfsetmiterjoin%
\definecolor{currentfill}{rgb}{0.296025,0.597955,0.790988}%
\pgfsetfillcolor{currentfill}%
\pgfsetlinewidth{0.000000pt}%
\definecolor{currentstroke}{rgb}{0.000000,0.000000,0.000000}%
\pgfsetstrokecolor{currentstroke}%
\pgfsetstrokeopacity{0.000000}%
\pgfsetdash{}{0pt}%
\pgfpathmoveto{\pgfqpoint{1.638404in}{1.708645in}}%
\pgfpathlineto{\pgfqpoint{1.637567in}{1.685608in}}%
\pgfpathlineto{\pgfqpoint{1.638529in}{1.679874in}}%
\pgfpathlineto{\pgfqpoint{1.615660in}{1.680679in}}%
\pgfpathlineto{\pgfqpoint{1.614445in}{1.686458in}}%
\pgfpathlineto{\pgfqpoint{1.615296in}{1.709477in}}%
\pgfpathlineto{\pgfqpoint{1.638404in}{1.708645in}}%
\pgfpathclose%
\pgfusepath{fill}%
\end{pgfscope}%
\begin{pgfscope}%
\pgfpathrectangle{\pgfqpoint{0.100000in}{0.100000in}}{\pgfqpoint{3.007045in}{1.925000in}}%
\pgfusepath{clip}%
\pgfsetbuttcap%
\pgfsetmiterjoin%
\definecolor{currentfill}{rgb}{0.371688,0.649627,0.820515}%
\pgfsetfillcolor{currentfill}%
\pgfsetlinewidth{0.000000pt}%
\definecolor{currentstroke}{rgb}{0.000000,0.000000,0.000000}%
\pgfsetstrokecolor{currentstroke}%
\pgfsetstrokeopacity{0.000000}%
\pgfsetdash{}{0pt}%
\pgfpathmoveto{\pgfqpoint{1.245609in}{1.461365in}}%
\pgfpathlineto{\pgfqpoint{1.218116in}{1.464571in}}%
\pgfpathlineto{\pgfqpoint{1.193681in}{1.468364in}}%
\pgfpathlineto{\pgfqpoint{1.197829in}{1.512182in}}%
\pgfpathlineto{\pgfqpoint{1.199309in}{1.522550in}}%
\pgfpathlineto{\pgfqpoint{1.197692in}{1.530943in}}%
\pgfpathlineto{\pgfqpoint{1.193646in}{1.534813in}}%
\pgfpathlineto{\pgfqpoint{1.188802in}{1.543168in}}%
\pgfpathlineto{\pgfqpoint{1.180687in}{1.550302in}}%
\pgfpathlineto{\pgfqpoint{1.174953in}{1.552958in}}%
\pgfpathlineto{\pgfqpoint{1.170691in}{1.563235in}}%
\pgfpathlineto{\pgfqpoint{1.169718in}{1.571914in}}%
\pgfpathlineto{\pgfqpoint{1.154054in}{1.574061in}}%
\pgfpathlineto{\pgfqpoint{1.160708in}{1.586055in}}%
\pgfpathlineto{\pgfqpoint{1.163161in}{1.587525in}}%
\pgfpathlineto{\pgfqpoint{1.135376in}{1.591485in}}%
\pgfpathlineto{\pgfqpoint{1.138319in}{1.605460in}}%
\pgfpathlineto{\pgfqpoint{1.140335in}{1.607233in}}%
\pgfpathlineto{\pgfqpoint{1.155409in}{1.604740in}}%
\pgfpathlineto{\pgfqpoint{1.167306in}{1.606868in}}%
\pgfpathlineto{\pgfqpoint{1.169858in}{1.624279in}}%
\pgfpathlineto{\pgfqpoint{1.171947in}{1.631734in}}%
\pgfpathlineto{\pgfqpoint{1.179523in}{1.630622in}}%
\pgfpathlineto{\pgfqpoint{1.184148in}{1.635775in}}%
\pgfpathlineto{\pgfqpoint{1.189792in}{1.634979in}}%
\pgfpathlineto{\pgfqpoint{1.190334in}{1.638786in}}%
\pgfpathlineto{\pgfqpoint{1.197757in}{1.637726in}}%
\pgfpathlineto{\pgfqpoint{1.201711in}{1.637190in}}%
\pgfpathlineto{\pgfqpoint{1.200926in}{1.631495in}}%
\pgfpathlineto{\pgfqpoint{1.212290in}{1.629948in}}%
\pgfpathlineto{\pgfqpoint{1.222383in}{1.622797in}}%
\pgfpathlineto{\pgfqpoint{1.221965in}{1.610407in}}%
\pgfpathlineto{\pgfqpoint{1.230291in}{1.608576in}}%
\pgfpathlineto{\pgfqpoint{1.227555in}{1.587451in}}%
\pgfpathlineto{\pgfqpoint{1.224309in}{1.576380in}}%
\pgfpathlineto{\pgfqpoint{1.247672in}{1.573410in}}%
\pgfpathlineto{\pgfqpoint{1.246154in}{1.561183in}}%
\pgfpathlineto{\pgfqpoint{1.257251in}{1.559802in}}%
\pgfpathlineto{\pgfqpoint{1.255504in}{1.539978in}}%
\pgfpathlineto{\pgfqpoint{1.251044in}{1.505633in}}%
\pgfpathlineto{\pgfqpoint{1.245609in}{1.461365in}}%
\pgfpathclose%
\pgfusepath{fill}%
\end{pgfscope}%
\begin{pgfscope}%
\pgfpathrectangle{\pgfqpoint{0.100000in}{0.100000in}}{\pgfqpoint{3.007045in}{1.925000in}}%
\pgfusepath{clip}%
\pgfsetbuttcap%
\pgfsetmiterjoin%
\definecolor{currentfill}{rgb}{0.371688,0.649627,0.820515}%
\pgfsetfillcolor{currentfill}%
\pgfsetlinewidth{0.000000pt}%
\definecolor{currentstroke}{rgb}{0.000000,0.000000,0.000000}%
\pgfsetstrokecolor{currentstroke}%
\pgfsetstrokeopacity{0.000000}%
\pgfsetdash{}{0pt}%
\pgfpathmoveto{\pgfqpoint{1.836747in}{1.453219in}}%
\pgfpathlineto{\pgfqpoint{1.831065in}{1.453126in}}%
\pgfpathlineto{\pgfqpoint{1.830743in}{1.476124in}}%
\pgfpathlineto{\pgfqpoint{1.842251in}{1.476305in}}%
\pgfpathlineto{\pgfqpoint{1.841921in}{1.499291in}}%
\pgfpathlineto{\pgfqpoint{1.853375in}{1.499519in}}%
\pgfpathlineto{\pgfqpoint{1.853444in}{1.494748in}}%
\pgfpathlineto{\pgfqpoint{1.864898in}{1.494975in}}%
\pgfpathlineto{\pgfqpoint{1.865473in}{1.453759in}}%
\pgfpathlineto{\pgfqpoint{1.836747in}{1.453219in}}%
\pgfpathclose%
\pgfusepath{fill}%
\end{pgfscope}%
\begin{pgfscope}%
\pgfpathrectangle{\pgfqpoint{0.100000in}{0.100000in}}{\pgfqpoint{3.007045in}{1.925000in}}%
\pgfusepath{clip}%
\pgfsetbuttcap%
\pgfsetmiterjoin%
\definecolor{currentfill}{rgb}{0.351511,0.635848,0.812641}%
\pgfsetfillcolor{currentfill}%
\pgfsetlinewidth{0.000000pt}%
\definecolor{currentstroke}{rgb}{0.000000,0.000000,0.000000}%
\pgfsetstrokecolor{currentstroke}%
\pgfsetstrokeopacity{0.000000}%
\pgfsetdash{}{0pt}%
\pgfpathmoveto{\pgfqpoint{1.507527in}{1.120611in}}%
\pgfpathlineto{\pgfqpoint{1.536726in}{1.118994in}}%
\pgfpathlineto{\pgfqpoint{1.535329in}{1.090418in}}%
\pgfpathlineto{\pgfqpoint{1.502124in}{1.092214in}}%
\pgfpathlineto{\pgfqpoint{1.478511in}{1.093767in}}%
\pgfpathlineto{\pgfqpoint{1.480135in}{1.122427in}}%
\pgfpathlineto{\pgfqpoint{1.507527in}{1.120611in}}%
\pgfpathclose%
\pgfusepath{fill}%
\end{pgfscope}%
\begin{pgfscope}%
\pgfpathrectangle{\pgfqpoint{0.100000in}{0.100000in}}{\pgfqpoint{3.007045in}{1.925000in}}%
\pgfusepath{clip}%
\pgfsetbuttcap%
\pgfsetmiterjoin%
\definecolor{currentfill}{rgb}{0.059439,0.353572,0.639831}%
\pgfsetfillcolor{currentfill}%
\pgfsetlinewidth{0.000000pt}%
\definecolor{currentstroke}{rgb}{0.000000,0.000000,0.000000}%
\pgfsetstrokecolor{currentstroke}%
\pgfsetstrokeopacity{0.000000}%
\pgfsetdash{}{0pt}%
\pgfpathmoveto{\pgfqpoint{1.450280in}{1.611196in}}%
\pgfpathlineto{\pgfqpoint{1.455103in}{1.611872in}}%
\pgfpathlineto{\pgfqpoint{1.459052in}{1.608999in}}%
\pgfpathlineto{\pgfqpoint{1.470594in}{1.606845in}}%
\pgfpathlineto{\pgfqpoint{1.480364in}{1.610710in}}%
\pgfpathlineto{\pgfqpoint{1.482806in}{1.614036in}}%
\pgfpathlineto{\pgfqpoint{1.495994in}{1.623275in}}%
\pgfpathlineto{\pgfqpoint{1.503158in}{1.631150in}}%
\pgfpathlineto{\pgfqpoint{1.506325in}{1.629275in}}%
\pgfpathlineto{\pgfqpoint{1.516525in}{1.631523in}}%
\pgfpathlineto{\pgfqpoint{1.518046in}{1.627144in}}%
\pgfpathlineto{\pgfqpoint{1.517258in}{1.619082in}}%
\pgfpathlineto{\pgfqpoint{1.514093in}{1.614101in}}%
\pgfpathlineto{\pgfqpoint{1.514131in}{1.607335in}}%
\pgfpathlineto{\pgfqpoint{1.518744in}{1.599173in}}%
\pgfpathlineto{\pgfqpoint{1.523621in}{1.593929in}}%
\pgfpathlineto{\pgfqpoint{1.526579in}{1.583044in}}%
\pgfpathlineto{\pgfqpoint{1.522749in}{1.580062in}}%
\pgfpathlineto{\pgfqpoint{1.518319in}{1.571035in}}%
\pgfpathlineto{\pgfqpoint{1.524166in}{1.567570in}}%
\pgfpathlineto{\pgfqpoint{1.494665in}{1.569370in}}%
\pgfpathlineto{\pgfqpoint{1.447246in}{1.572779in}}%
\pgfpathlineto{\pgfqpoint{1.450280in}{1.611196in}}%
\pgfpathclose%
\pgfusepath{fill}%
\end{pgfscope}%
\begin{pgfscope}%
\pgfpathrectangle{\pgfqpoint{0.100000in}{0.100000in}}{\pgfqpoint{3.007045in}{1.925000in}}%
\pgfusepath{clip}%
\pgfsetbuttcap%
\pgfsetmiterjoin%
\definecolor{currentfill}{rgb}{0.778685,0.860300,0.937993}%
\pgfsetfillcolor{currentfill}%
\pgfsetlinewidth{0.000000pt}%
\definecolor{currentstroke}{rgb}{0.000000,0.000000,0.000000}%
\pgfsetstrokecolor{currentstroke}%
\pgfsetstrokeopacity{0.000000}%
\pgfsetdash{}{0pt}%
\pgfpathmoveto{\pgfqpoint{1.139523in}{1.071438in}}%
\pgfpathlineto{\pgfqpoint{1.121195in}{1.073917in}}%
\pgfpathlineto{\pgfqpoint{1.124305in}{1.096464in}}%
\pgfpathlineto{\pgfqpoint{1.123300in}{1.108295in}}%
\pgfpathlineto{\pgfqpoint{1.121625in}{1.109932in}}%
\pgfpathlineto{\pgfqpoint{1.123423in}{1.117299in}}%
\pgfpathlineto{\pgfqpoint{1.123455in}{1.127357in}}%
\pgfpathlineto{\pgfqpoint{1.121297in}{1.132386in}}%
\pgfpathlineto{\pgfqpoint{1.128210in}{1.131426in}}%
\pgfpathlineto{\pgfqpoint{1.136431in}{1.191196in}}%
\pgfpathlineto{\pgfqpoint{1.142160in}{1.192960in}}%
\pgfpathlineto{\pgfqpoint{1.143492in}{1.187243in}}%
\pgfpathlineto{\pgfqpoint{1.146599in}{1.183208in}}%
\pgfpathlineto{\pgfqpoint{1.157547in}{1.181738in}}%
\pgfpathlineto{\pgfqpoint{1.164471in}{1.172432in}}%
\pgfpathlineto{\pgfqpoint{1.169868in}{1.170865in}}%
\pgfpathlineto{\pgfqpoint{1.172957in}{1.174640in}}%
\pgfpathlineto{\pgfqpoint{1.177735in}{1.174146in}}%
\pgfpathlineto{\pgfqpoint{1.180201in}{1.170746in}}%
\pgfpathlineto{\pgfqpoint{1.186288in}{1.164150in}}%
\pgfpathlineto{\pgfqpoint{1.193507in}{1.162201in}}%
\pgfpathlineto{\pgfqpoint{1.188060in}{1.158199in}}%
\pgfpathlineto{\pgfqpoint{1.187185in}{1.151775in}}%
\pgfpathlineto{\pgfqpoint{1.189161in}{1.136735in}}%
\pgfpathlineto{\pgfqpoint{1.193414in}{1.130886in}}%
\pgfpathlineto{\pgfqpoint{1.154865in}{1.136033in}}%
\pgfpathlineto{\pgfqpoint{1.150792in}{1.105369in}}%
\pgfpathlineto{\pgfqpoint{1.143557in}{1.105192in}}%
\pgfpathlineto{\pgfqpoint{1.140887in}{1.088045in}}%
\pgfpathlineto{\pgfqpoint{1.139523in}{1.071438in}}%
\pgfpathclose%
\pgfusepath{fill}%
\end{pgfscope}%
\begin{pgfscope}%
\pgfpathrectangle{\pgfqpoint{0.100000in}{0.100000in}}{\pgfqpoint{3.007045in}{1.925000in}}%
\pgfusepath{clip}%
\pgfsetbuttcap%
\pgfsetmiterjoin%
\definecolor{currentfill}{rgb}{0.031373,0.285675,0.564291}%
\pgfsetfillcolor{currentfill}%
\pgfsetlinewidth{0.000000pt}%
\definecolor{currentstroke}{rgb}{0.000000,0.000000,0.000000}%
\pgfsetstrokecolor{currentstroke}%
\pgfsetstrokeopacity{0.000000}%
\pgfsetdash{}{0pt}%
\pgfpathmoveto{\pgfqpoint{1.447246in}{1.572779in}}%
\pgfpathlineto{\pgfqpoint{1.494665in}{1.569370in}}%
\pgfpathlineto{\pgfqpoint{1.524166in}{1.567570in}}%
\pgfpathlineto{\pgfqpoint{1.526953in}{1.561321in}}%
\pgfpathlineto{\pgfqpoint{1.524722in}{1.557253in}}%
\pgfpathlineto{\pgfqpoint{1.527075in}{1.552396in}}%
\pgfpathlineto{\pgfqpoint{1.523533in}{1.542968in}}%
\pgfpathlineto{\pgfqpoint{1.524818in}{1.537110in}}%
\pgfpathlineto{\pgfqpoint{1.518468in}{1.535998in}}%
\pgfpathlineto{\pgfqpoint{1.518970in}{1.529708in}}%
\pgfpathlineto{\pgfqpoint{1.519350in}{1.527670in}}%
\pgfpathlineto{\pgfqpoint{1.512562in}{1.521360in}}%
\pgfpathlineto{\pgfqpoint{1.506367in}{1.525118in}}%
\pgfpathlineto{\pgfqpoint{1.501256in}{1.521545in}}%
\pgfpathlineto{\pgfqpoint{1.498577in}{1.523374in}}%
\pgfpathlineto{\pgfqpoint{1.489275in}{1.520056in}}%
\pgfpathlineto{\pgfqpoint{1.484012in}{1.521937in}}%
\pgfpathlineto{\pgfqpoint{1.479161in}{1.518521in}}%
\pgfpathlineto{\pgfqpoint{1.473297in}{1.518824in}}%
\pgfpathlineto{\pgfqpoint{1.464688in}{1.511398in}}%
\pgfpathlineto{\pgfqpoint{1.458166in}{1.512912in}}%
\pgfpathlineto{\pgfqpoint{1.452730in}{1.510166in}}%
\pgfpathlineto{\pgfqpoint{1.442332in}{1.509325in}}%
\pgfpathlineto{\pgfqpoint{1.445034in}{1.544102in}}%
\pgfpathlineto{\pgfqpoint{1.447246in}{1.572779in}}%
\pgfpathclose%
\pgfusepath{fill}%
\end{pgfscope}%
\begin{pgfscope}%
\pgfpathrectangle{\pgfqpoint{0.100000in}{0.100000in}}{\pgfqpoint{3.007045in}{1.925000in}}%
\pgfusepath{clip}%
\pgfsetbuttcap%
\pgfsetmiterjoin%
\definecolor{currentfill}{rgb}{0.535686,0.746082,0.864252}%
\pgfsetfillcolor{currentfill}%
\pgfsetlinewidth{0.000000pt}%
\definecolor{currentstroke}{rgb}{0.000000,0.000000,0.000000}%
\pgfsetstrokecolor{currentstroke}%
\pgfsetstrokeopacity{0.000000}%
\pgfsetdash{}{0pt}%
\pgfpathmoveto{\pgfqpoint{2.379259in}{0.944820in}}%
\pgfpathlineto{\pgfqpoint{2.373522in}{0.943638in}}%
\pgfpathlineto{\pgfqpoint{2.364223in}{0.935952in}}%
\pgfpathlineto{\pgfqpoint{2.354830in}{0.942630in}}%
\pgfpathlineto{\pgfqpoint{2.341668in}{0.946890in}}%
\pgfpathlineto{\pgfqpoint{2.329679in}{0.943109in}}%
\pgfpathlineto{\pgfqpoint{2.327024in}{0.948030in}}%
\pgfpathlineto{\pgfqpoint{2.321351in}{0.950113in}}%
\pgfpathlineto{\pgfqpoint{2.317579in}{0.954327in}}%
\pgfpathlineto{\pgfqpoint{2.322488in}{0.960571in}}%
\pgfpathlineto{\pgfqpoint{2.319716in}{0.965996in}}%
\pgfpathlineto{\pgfqpoint{2.308631in}{0.975465in}}%
\pgfpathlineto{\pgfqpoint{2.309261in}{0.983763in}}%
\pgfpathlineto{\pgfqpoint{2.317969in}{0.990753in}}%
\pgfpathlineto{\pgfqpoint{2.325950in}{0.982785in}}%
\pgfpathlineto{\pgfqpoint{2.334580in}{0.979199in}}%
\pgfpathlineto{\pgfqpoint{2.338194in}{0.987373in}}%
\pgfpathlineto{\pgfqpoint{2.336026in}{1.001374in}}%
\pgfpathlineto{\pgfqpoint{2.340163in}{1.003243in}}%
\pgfpathlineto{\pgfqpoint{2.341210in}{1.008147in}}%
\pgfpathlineto{\pgfqpoint{2.353818in}{1.009495in}}%
\pgfpathlineto{\pgfqpoint{2.358542in}{0.999323in}}%
\pgfpathlineto{\pgfqpoint{2.361719in}{1.000172in}}%
\pgfpathlineto{\pgfqpoint{2.372662in}{0.995532in}}%
\pgfpathlineto{\pgfqpoint{2.370660in}{0.984801in}}%
\pgfpathlineto{\pgfqpoint{2.376395in}{0.971244in}}%
\pgfpathlineto{\pgfqpoint{2.369682in}{0.964768in}}%
\pgfpathlineto{\pgfqpoint{2.374820in}{0.956140in}}%
\pgfpathlineto{\pgfqpoint{2.378270in}{0.952775in}}%
\pgfpathlineto{\pgfqpoint{2.379259in}{0.944820in}}%
\pgfpathclose%
\pgfusepath{fill}%
\end{pgfscope}%
\begin{pgfscope}%
\pgfpathrectangle{\pgfqpoint{0.100000in}{0.100000in}}{\pgfqpoint{3.007045in}{1.925000in}}%
\pgfusepath{clip}%
\pgfsetbuttcap%
\pgfsetmiterjoin%
\definecolor{currentfill}{rgb}{0.351511,0.635848,0.812641}%
\pgfsetfillcolor{currentfill}%
\pgfsetlinewidth{0.000000pt}%
\definecolor{currentstroke}{rgb}{0.000000,0.000000,0.000000}%
\pgfsetstrokecolor{currentstroke}%
\pgfsetstrokeopacity{0.000000}%
\pgfsetdash{}{0pt}%
\pgfpathmoveto{\pgfqpoint{1.511195in}{1.206454in}}%
\pgfpathlineto{\pgfqpoint{1.509436in}{1.177873in}}%
\pgfpathlineto{\pgfqpoint{1.448195in}{1.181868in}}%
\pgfpathlineto{\pgfqpoint{1.446935in}{1.181960in}}%
\pgfpathlineto{\pgfqpoint{1.449096in}{1.210470in}}%
\pgfpathlineto{\pgfqpoint{1.453477in}{1.210184in}}%
\pgfpathlineto{\pgfqpoint{1.510401in}{1.206498in}}%
\pgfpathlineto{\pgfqpoint{1.511195in}{1.206454in}}%
\pgfpathclose%
\pgfusepath{fill}%
\end{pgfscope}%
\begin{pgfscope}%
\pgfpathrectangle{\pgfqpoint{0.100000in}{0.100000in}}{\pgfqpoint{3.007045in}{1.925000in}}%
\pgfusepath{clip}%
\pgfsetbuttcap%
\pgfsetmiterjoin%
\definecolor{currentfill}{rgb}{0.447843,0.697855,0.845306}%
\pgfsetfillcolor{currentfill}%
\pgfsetlinewidth{0.000000pt}%
\definecolor{currentstroke}{rgb}{0.000000,0.000000,0.000000}%
\pgfsetstrokecolor{currentstroke}%
\pgfsetstrokeopacity{0.000000}%
\pgfsetdash{}{0pt}%
\pgfpathmoveto{\pgfqpoint{2.585937in}{0.924191in}}%
\pgfpathlineto{\pgfqpoint{2.599449in}{0.926351in}}%
\pgfpathlineto{\pgfqpoint{2.613555in}{0.916458in}}%
\pgfpathlineto{\pgfqpoint{2.607323in}{0.893730in}}%
\pgfpathlineto{\pgfqpoint{2.595973in}{0.904310in}}%
\pgfpathlineto{\pgfqpoint{2.582994in}{0.902658in}}%
\pgfpathlineto{\pgfqpoint{2.571314in}{0.892287in}}%
\pgfpathlineto{\pgfqpoint{2.567423in}{0.900852in}}%
\pgfpathlineto{\pgfqpoint{2.562238in}{0.907450in}}%
\pgfpathlineto{\pgfqpoint{2.551891in}{0.919345in}}%
\pgfpathlineto{\pgfqpoint{2.585937in}{0.924191in}}%
\pgfpathclose%
\pgfusepath{fill}%
\end{pgfscope}%
\begin{pgfscope}%
\pgfpathrectangle{\pgfqpoint{0.100000in}{0.100000in}}{\pgfqpoint{3.007045in}{1.925000in}}%
\pgfusepath{clip}%
\pgfsetbuttcap%
\pgfsetmiterjoin%
\definecolor{currentfill}{rgb}{0.056363,0.349635,0.636755}%
\pgfsetfillcolor{currentfill}%
\pgfsetlinewidth{0.000000pt}%
\definecolor{currentstroke}{rgb}{0.000000,0.000000,0.000000}%
\pgfsetstrokecolor{currentstroke}%
\pgfsetstrokeopacity{0.000000}%
\pgfsetdash{}{0pt}%
\pgfpathmoveto{\pgfqpoint{1.473940in}{0.485183in}}%
\pgfpathlineto{\pgfqpoint{1.441506in}{0.486973in}}%
\pgfpathlineto{\pgfqpoint{1.435567in}{0.493311in}}%
\pgfpathlineto{\pgfqpoint{1.434480in}{0.498439in}}%
\pgfpathlineto{\pgfqpoint{1.422748in}{0.507253in}}%
\pgfpathlineto{\pgfqpoint{1.418771in}{0.514227in}}%
\pgfpathlineto{\pgfqpoint{1.415118in}{0.514942in}}%
\pgfpathlineto{\pgfqpoint{1.410285in}{0.521202in}}%
\pgfpathlineto{\pgfqpoint{1.406272in}{0.522948in}}%
\pgfpathlineto{\pgfqpoint{1.401030in}{0.533989in}}%
\pgfpathlineto{\pgfqpoint{1.394375in}{0.537885in}}%
\pgfpathlineto{\pgfqpoint{1.390298in}{0.536623in}}%
\pgfpathlineto{\pgfqpoint{1.369726in}{0.539824in}}%
\pgfpathlineto{\pgfqpoint{1.364524in}{0.539321in}}%
\pgfpathlineto{\pgfqpoint{1.349461in}{0.546748in}}%
\pgfpathlineto{\pgfqpoint{1.336243in}{0.559459in}}%
\pgfpathlineto{\pgfqpoint{1.337515in}{0.574775in}}%
\pgfpathlineto{\pgfqpoint{1.350354in}{0.573776in}}%
\pgfpathlineto{\pgfqpoint{1.352050in}{0.594662in}}%
\pgfpathlineto{\pgfqpoint{1.363729in}{0.593646in}}%
\pgfpathlineto{\pgfqpoint{1.364071in}{0.597523in}}%
\pgfpathlineto{\pgfqpoint{1.386971in}{0.595647in}}%
\pgfpathlineto{\pgfqpoint{1.390664in}{0.594316in}}%
\pgfpathlineto{\pgfqpoint{1.391278in}{0.588321in}}%
\pgfpathlineto{\pgfqpoint{1.385588in}{0.580234in}}%
\pgfpathlineto{\pgfqpoint{1.389923in}{0.575693in}}%
\pgfpathlineto{\pgfqpoint{1.383908in}{0.571430in}}%
\pgfpathlineto{\pgfqpoint{1.429685in}{0.568203in}}%
\pgfpathlineto{\pgfqpoint{1.444627in}{0.567276in}}%
\pgfpathlineto{\pgfqpoint{1.441878in}{0.523060in}}%
\pgfpathlineto{\pgfqpoint{1.475915in}{0.520991in}}%
\pgfpathlineto{\pgfqpoint{1.473940in}{0.485183in}}%
\pgfpathclose%
\pgfusepath{fill}%
\end{pgfscope}%
\begin{pgfscope}%
\pgfpathrectangle{\pgfqpoint{0.100000in}{0.100000in}}{\pgfqpoint{3.007045in}{1.925000in}}%
\pgfusepath{clip}%
\pgfsetbuttcap%
\pgfsetmiterjoin%
\definecolor{currentfill}{rgb}{0.647290,0.803922,0.892042}%
\pgfsetfillcolor{currentfill}%
\pgfsetlinewidth{0.000000pt}%
\definecolor{currentstroke}{rgb}{0.000000,0.000000,0.000000}%
\pgfsetstrokecolor{currentstroke}%
\pgfsetstrokeopacity{0.000000}%
\pgfsetdash{}{0pt}%
\pgfpathmoveto{\pgfqpoint{2.559174in}{1.248298in}}%
\pgfpathlineto{\pgfqpoint{2.559394in}{1.254953in}}%
\pgfpathlineto{\pgfqpoint{2.556847in}{1.259865in}}%
\pgfpathlineto{\pgfqpoint{2.561814in}{1.273283in}}%
\pgfpathlineto{\pgfqpoint{2.567920in}{1.283023in}}%
\pgfpathlineto{\pgfqpoint{2.574047in}{1.302973in}}%
\pgfpathlineto{\pgfqpoint{2.576932in}{1.318538in}}%
\pgfpathlineto{\pgfqpoint{2.599096in}{1.322608in}}%
\pgfpathlineto{\pgfqpoint{2.606011in}{1.320523in}}%
\pgfpathlineto{\pgfqpoint{2.609389in}{1.317404in}}%
\pgfpathlineto{\pgfqpoint{2.608640in}{1.313232in}}%
\pgfpathlineto{\pgfqpoint{2.612866in}{1.308899in}}%
\pgfpathlineto{\pgfqpoint{2.608820in}{1.295800in}}%
\pgfpathlineto{\pgfqpoint{2.610752in}{1.291249in}}%
\pgfpathlineto{\pgfqpoint{2.616493in}{1.288353in}}%
\pgfpathlineto{\pgfqpoint{2.613885in}{1.282571in}}%
\pgfpathlineto{\pgfqpoint{2.614370in}{1.276431in}}%
\pgfpathlineto{\pgfqpoint{2.612460in}{1.264832in}}%
\pgfpathlineto{\pgfqpoint{2.609634in}{1.257414in}}%
\pgfpathlineto{\pgfqpoint{2.559174in}{1.248298in}}%
\pgfpathclose%
\pgfusepath{fill}%
\end{pgfscope}%
\begin{pgfscope}%
\pgfpathrectangle{\pgfqpoint{0.100000in}{0.100000in}}{\pgfqpoint{3.007045in}{1.925000in}}%
\pgfusepath{clip}%
\pgfsetbuttcap%
\pgfsetmiterjoin%
\definecolor{currentfill}{rgb}{0.301069,0.601399,0.792957}%
\pgfsetfillcolor{currentfill}%
\pgfsetlinewidth{0.000000pt}%
\definecolor{currentstroke}{rgb}{0.000000,0.000000,0.000000}%
\pgfsetstrokecolor{currentstroke}%
\pgfsetstrokeopacity{0.000000}%
\pgfsetdash{}{0pt}%
\pgfpathmoveto{\pgfqpoint{1.963320in}{1.146179in}}%
\pgfpathlineto{\pgfqpoint{1.963877in}{1.126688in}}%
\pgfpathlineto{\pgfqpoint{1.956143in}{1.126474in}}%
\pgfpathlineto{\pgfqpoint{1.956351in}{1.117548in}}%
\pgfpathlineto{\pgfqpoint{1.949655in}{1.115080in}}%
\pgfpathlineto{\pgfqpoint{1.944978in}{1.116719in}}%
\pgfpathlineto{\pgfqpoint{1.944438in}{1.140114in}}%
\pgfpathlineto{\pgfqpoint{1.920082in}{1.139613in}}%
\pgfpathlineto{\pgfqpoint{1.920007in}{1.151193in}}%
\pgfpathlineto{\pgfqpoint{1.909292in}{1.151308in}}%
\pgfpathlineto{\pgfqpoint{1.909303in}{1.157973in}}%
\pgfpathlineto{\pgfqpoint{1.939480in}{1.158394in}}%
\pgfpathlineto{\pgfqpoint{1.953772in}{1.157587in}}%
\pgfpathlineto{\pgfqpoint{1.955717in}{1.146050in}}%
\pgfpathlineto{\pgfqpoint{1.963320in}{1.146179in}}%
\pgfpathclose%
\pgfusepath{fill}%
\end{pgfscope}%
\begin{pgfscope}%
\pgfpathrectangle{\pgfqpoint{0.100000in}{0.100000in}}{\pgfqpoint{3.007045in}{1.925000in}}%
\pgfusepath{clip}%
\pgfsetbuttcap%
\pgfsetmiterjoin%
\definecolor{currentfill}{rgb}{0.429020,0.687520,0.841246}%
\pgfsetfillcolor{currentfill}%
\pgfsetlinewidth{0.000000pt}%
\definecolor{currentstroke}{rgb}{0.000000,0.000000,0.000000}%
\pgfsetstrokecolor{currentstroke}%
\pgfsetstrokeopacity{0.000000}%
\pgfsetdash{}{0pt}%
\pgfpathmoveto{\pgfqpoint{1.835366in}{1.168017in}}%
\pgfpathlineto{\pgfqpoint{1.812537in}{1.167939in}}%
\pgfpathlineto{\pgfqpoint{1.812471in}{1.163163in}}%
\pgfpathlineto{\pgfqpoint{1.792627in}{1.163111in}}%
\pgfpathlineto{\pgfqpoint{1.792583in}{1.168136in}}%
\pgfpathlineto{\pgfqpoint{1.767101in}{1.168173in}}%
\pgfpathlineto{\pgfqpoint{1.769538in}{1.174111in}}%
\pgfpathlineto{\pgfqpoint{1.766486in}{1.176026in}}%
\pgfpathlineto{\pgfqpoint{1.755095in}{1.176082in}}%
\pgfpathlineto{\pgfqpoint{1.755176in}{1.198941in}}%
\pgfpathlineto{\pgfqpoint{1.756771in}{1.198935in}}%
\pgfpathlineto{\pgfqpoint{1.762123in}{1.192600in}}%
\pgfpathlineto{\pgfqpoint{1.768029in}{1.189813in}}%
\pgfpathlineto{\pgfqpoint{1.772661in}{1.193312in}}%
\pgfpathlineto{\pgfqpoint{1.770162in}{1.207494in}}%
\pgfpathlineto{\pgfqpoint{1.792177in}{1.207263in}}%
\pgfpathlineto{\pgfqpoint{1.792230in}{1.201536in}}%
\pgfpathlineto{\pgfqpoint{1.811728in}{1.201377in}}%
\pgfpathlineto{\pgfqpoint{1.811789in}{1.208062in}}%
\pgfpathlineto{\pgfqpoint{1.834576in}{1.208087in}}%
\pgfpathlineto{\pgfqpoint{1.834849in}{1.196667in}}%
\pgfpathlineto{\pgfqpoint{1.835366in}{1.168017in}}%
\pgfpathclose%
\pgfusepath{fill}%
\end{pgfscope}%
\begin{pgfscope}%
\pgfpathrectangle{\pgfqpoint{0.100000in}{0.100000in}}{\pgfqpoint{3.007045in}{1.925000in}}%
\pgfusepath{clip}%
\pgfsetbuttcap%
\pgfsetmiterjoin%
\definecolor{currentfill}{rgb}{0.406997,0.673741,0.834295}%
\pgfsetfillcolor{currentfill}%
\pgfsetlinewidth{0.000000pt}%
\definecolor{currentstroke}{rgb}{0.000000,0.000000,0.000000}%
\pgfsetstrokecolor{currentstroke}%
\pgfsetstrokeopacity{0.000000}%
\pgfsetdash{}{0pt}%
\pgfpathmoveto{\pgfqpoint{2.068022in}{0.977336in}}%
\pgfpathlineto{\pgfqpoint{2.064440in}{0.977028in}}%
\pgfpathlineto{\pgfqpoint{2.061582in}{0.976889in}}%
\pgfpathlineto{\pgfqpoint{2.061897in}{0.966247in}}%
\pgfpathlineto{\pgfqpoint{2.059156in}{0.970494in}}%
\pgfpathlineto{\pgfqpoint{2.050739in}{0.968865in}}%
\pgfpathlineto{\pgfqpoint{2.039044in}{0.968405in}}%
\pgfpathlineto{\pgfqpoint{2.038888in}{0.984228in}}%
\pgfpathlineto{\pgfqpoint{2.028875in}{0.983778in}}%
\pgfpathlineto{\pgfqpoint{2.028045in}{0.989826in}}%
\pgfpathlineto{\pgfqpoint{2.022047in}{1.002653in}}%
\pgfpathlineto{\pgfqpoint{2.029271in}{1.013848in}}%
\pgfpathlineto{\pgfqpoint{2.023343in}{1.013571in}}%
\pgfpathlineto{\pgfqpoint{2.022844in}{1.028625in}}%
\pgfpathlineto{\pgfqpoint{2.026620in}{1.028636in}}%
\pgfpathlineto{\pgfqpoint{2.025561in}{1.047408in}}%
\pgfpathlineto{\pgfqpoint{2.053627in}{1.049022in}}%
\pgfpathlineto{\pgfqpoint{2.058254in}{1.047208in}}%
\pgfpathlineto{\pgfqpoint{2.064198in}{1.036075in}}%
\pgfpathlineto{\pgfqpoint{2.059538in}{1.028875in}}%
\pgfpathlineto{\pgfqpoint{2.067443in}{1.014097in}}%
\pgfpathlineto{\pgfqpoint{2.072603in}{1.010193in}}%
\pgfpathlineto{\pgfqpoint{2.077300in}{1.012074in}}%
\pgfpathlineto{\pgfqpoint{2.081022in}{1.010070in}}%
\pgfpathlineto{\pgfqpoint{2.082880in}{1.008816in}}%
\pgfpathlineto{\pgfqpoint{2.081317in}{1.001182in}}%
\pgfpathlineto{\pgfqpoint{2.082341in}{0.997150in}}%
\pgfpathlineto{\pgfqpoint{2.078631in}{0.992336in}}%
\pgfpathlineto{\pgfqpoint{2.080971in}{0.989204in}}%
\pgfpathlineto{\pgfqpoint{2.077777in}{0.982593in}}%
\pgfpathlineto{\pgfqpoint{2.070320in}{0.985858in}}%
\pgfpathlineto{\pgfqpoint{2.068022in}{0.977336in}}%
\pgfpathclose%
\pgfusepath{fill}%
\end{pgfscope}%
\begin{pgfscope}%
\pgfpathrectangle{\pgfqpoint{0.100000in}{0.100000in}}{\pgfqpoint{3.007045in}{1.925000in}}%
\pgfusepath{clip}%
\pgfsetbuttcap%
\pgfsetmiterjoin%
\definecolor{currentfill}{rgb}{0.171027,0.484752,0.731242}%
\pgfsetfillcolor{currentfill}%
\pgfsetlinewidth{0.000000pt}%
\definecolor{currentstroke}{rgb}{0.000000,0.000000,0.000000}%
\pgfsetstrokecolor{currentstroke}%
\pgfsetstrokeopacity{0.000000}%
\pgfsetdash{}{0pt}%
\pgfpathmoveto{\pgfqpoint{2.139795in}{0.676959in}}%
\pgfpathlineto{\pgfqpoint{2.141224in}{0.664018in}}%
\pgfpathlineto{\pgfqpoint{2.143322in}{0.646719in}}%
\pgfpathlineto{\pgfqpoint{2.121151in}{0.645024in}}%
\pgfpathlineto{\pgfqpoint{2.089615in}{0.643002in}}%
\pgfpathlineto{\pgfqpoint{2.088016in}{0.667046in}}%
\pgfpathlineto{\pgfqpoint{2.073893in}{0.665116in}}%
\pgfpathlineto{\pgfqpoint{2.072910in}{0.680552in}}%
\pgfpathlineto{\pgfqpoint{2.068533in}{0.682516in}}%
\pgfpathlineto{\pgfqpoint{2.067842in}{0.694087in}}%
\pgfpathlineto{\pgfqpoint{2.090671in}{0.695575in}}%
\pgfpathlineto{\pgfqpoint{2.113560in}{0.697106in}}%
\pgfpathlineto{\pgfqpoint{2.115516in}{0.670799in}}%
\pgfpathlineto{\pgfqpoint{2.130566in}{0.673645in}}%
\pgfpathlineto{\pgfqpoint{2.139795in}{0.676959in}}%
\pgfpathclose%
\pgfusepath{fill}%
\end{pgfscope}%
\begin{pgfscope}%
\pgfpathrectangle{\pgfqpoint{0.100000in}{0.100000in}}{\pgfqpoint{3.007045in}{1.925000in}}%
\pgfusepath{clip}%
\pgfsetbuttcap%
\pgfsetmiterjoin%
\definecolor{currentfill}{rgb}{0.183206,0.496932,0.737516}%
\pgfsetfillcolor{currentfill}%
\pgfsetlinewidth{0.000000pt}%
\definecolor{currentstroke}{rgb}{0.000000,0.000000,0.000000}%
\pgfsetstrokecolor{currentstroke}%
\pgfsetstrokeopacity{0.000000}%
\pgfsetdash{}{0pt}%
\pgfpathmoveto{\pgfqpoint{1.314888in}{1.740131in}}%
\pgfpathlineto{\pgfqpoint{1.309172in}{1.740759in}}%
\pgfpathlineto{\pgfqpoint{1.307173in}{1.723478in}}%
\pgfpathlineto{\pgfqpoint{1.305247in}{1.723687in}}%
\pgfpathlineto{\pgfqpoint{1.302614in}{1.700688in}}%
\pgfpathlineto{\pgfqpoint{1.283442in}{1.702901in}}%
\pgfpathlineto{\pgfqpoint{1.282748in}{1.697095in}}%
\pgfpathlineto{\pgfqpoint{1.277023in}{1.697772in}}%
\pgfpathlineto{\pgfqpoint{1.277730in}{1.703596in}}%
\pgfpathlineto{\pgfqpoint{1.272031in}{1.704281in}}%
\pgfpathlineto{\pgfqpoint{1.268284in}{1.704741in}}%
\pgfpathlineto{\pgfqpoint{1.269690in}{1.716224in}}%
\pgfpathlineto{\pgfqpoint{1.263981in}{1.716912in}}%
\pgfpathlineto{\pgfqpoint{1.267101in}{1.728242in}}%
\pgfpathlineto{\pgfqpoint{1.269873in}{1.750687in}}%
\pgfpathlineto{\pgfqpoint{1.265560in}{1.751233in}}%
\pgfpathlineto{\pgfqpoint{1.263629in}{1.757459in}}%
\pgfpathlineto{\pgfqpoint{1.265605in}{1.764425in}}%
\pgfpathlineto{\pgfqpoint{1.272584in}{1.760976in}}%
\pgfpathlineto{\pgfqpoint{1.289529in}{1.757826in}}%
\pgfpathlineto{\pgfqpoint{1.291202in}{1.771653in}}%
\pgfpathlineto{\pgfqpoint{1.292930in}{1.771456in}}%
\pgfpathlineto{\pgfqpoint{1.295653in}{1.794279in}}%
\pgfpathlineto{\pgfqpoint{1.332164in}{1.790080in}}%
\pgfpathlineto{\pgfqpoint{1.341649in}{1.789040in}}%
\pgfpathlineto{\pgfqpoint{1.341017in}{1.783290in}}%
\pgfpathlineto{\pgfqpoint{1.346744in}{1.782663in}}%
\pgfpathlineto{\pgfqpoint{1.346116in}{1.776879in}}%
\pgfpathlineto{\pgfqpoint{1.371755in}{1.774248in}}%
\pgfpathlineto{\pgfqpoint{1.369270in}{1.748112in}}%
\pgfpathlineto{\pgfqpoint{1.366019in}{1.751221in}}%
\pgfpathlineto{\pgfqpoint{1.355902in}{1.752996in}}%
\pgfpathlineto{\pgfqpoint{1.351556in}{1.758299in}}%
\pgfpathlineto{\pgfqpoint{1.341818in}{1.758120in}}%
\pgfpathlineto{\pgfqpoint{1.335725in}{1.759869in}}%
\pgfpathlineto{\pgfqpoint{1.324957in}{1.756945in}}%
\pgfpathlineto{\pgfqpoint{1.318715in}{1.758150in}}%
\pgfpathlineto{\pgfqpoint{1.317232in}{1.745061in}}%
\pgfpathlineto{\pgfqpoint{1.314888in}{1.740131in}}%
\pgfpathclose%
\pgfusepath{fill}%
\end{pgfscope}%
\begin{pgfscope}%
\pgfpathrectangle{\pgfqpoint{0.100000in}{0.100000in}}{\pgfqpoint{3.007045in}{1.925000in}}%
\pgfusepath{clip}%
\pgfsetbuttcap%
\pgfsetmiterjoin%
\definecolor{currentfill}{rgb}{0.031373,0.261315,0.528120}%
\pgfsetfillcolor{currentfill}%
\pgfsetlinewidth{0.000000pt}%
\definecolor{currentstroke}{rgb}{0.000000,0.000000,0.000000}%
\pgfsetstrokecolor{currentstroke}%
\pgfsetstrokeopacity{0.000000}%
\pgfsetdash{}{0pt}%
\pgfpathmoveto{\pgfqpoint{1.512921in}{0.424052in}}%
\pgfpathlineto{\pgfqpoint{1.470461in}{0.425819in}}%
\pgfpathlineto{\pgfqpoint{1.473940in}{0.485183in}}%
\pgfpathlineto{\pgfqpoint{1.514641in}{0.483384in}}%
\pgfpathlineto{\pgfqpoint{1.513462in}{0.453257in}}%
\pgfpathlineto{\pgfqpoint{1.514242in}{0.453233in}}%
\pgfpathlineto{\pgfqpoint{1.512921in}{0.424052in}}%
\pgfpathclose%
\pgfusepath{fill}%
\end{pgfscope}%
\begin{pgfscope}%
\pgfpathrectangle{\pgfqpoint{0.100000in}{0.100000in}}{\pgfqpoint{3.007045in}{1.925000in}}%
\pgfusepath{clip}%
\pgfsetbuttcap%
\pgfsetmiterjoin%
\definecolor{currentfill}{rgb}{0.031373,0.285675,0.564291}%
\pgfsetfillcolor{currentfill}%
\pgfsetlinewidth{0.000000pt}%
\definecolor{currentstroke}{rgb}{0.000000,0.000000,0.000000}%
\pgfsetstrokecolor{currentstroke}%
\pgfsetstrokeopacity{0.000000}%
\pgfsetdash{}{0pt}%
\pgfpathmoveto{\pgfqpoint{1.219256in}{0.700832in}}%
\pgfpathlineto{\pgfqpoint{1.265329in}{0.695996in}}%
\pgfpathlineto{\pgfqpoint{1.255027in}{0.637292in}}%
\pgfpathlineto{\pgfqpoint{1.205698in}{0.612940in}}%
\pgfpathlineto{\pgfqpoint{1.207983in}{0.628535in}}%
\pgfpathlineto{\pgfqpoint{1.212900in}{0.676457in}}%
\pgfpathlineto{\pgfqpoint{1.215312in}{0.701261in}}%
\pgfpathlineto{\pgfqpoint{1.219256in}{0.700832in}}%
\pgfpathclose%
\pgfusepath{fill}%
\end{pgfscope}%
\begin{pgfscope}%
\pgfpathrectangle{\pgfqpoint{0.100000in}{0.100000in}}{\pgfqpoint{3.007045in}{1.925000in}}%
\pgfusepath{clip}%
\pgfsetbuttcap%
\pgfsetmiterjoin%
\definecolor{currentfill}{rgb}{0.311157,0.608289,0.796894}%
\pgfsetfillcolor{currentfill}%
\pgfsetlinewidth{0.000000pt}%
\definecolor{currentstroke}{rgb}{0.000000,0.000000,0.000000}%
\pgfsetstrokecolor{currentstroke}%
\pgfsetstrokeopacity{0.000000}%
\pgfsetdash{}{0pt}%
\pgfpathmoveto{\pgfqpoint{1.347270in}{1.020215in}}%
\pgfpathlineto{\pgfqpoint{1.351668in}{1.019814in}}%
\pgfpathlineto{\pgfqpoint{1.348759in}{0.987002in}}%
\pgfpathlineto{\pgfqpoint{1.346674in}{0.987189in}}%
\pgfpathlineto{\pgfqpoint{1.342266in}{0.937180in}}%
\pgfpathlineto{\pgfqpoint{1.324480in}{0.938840in}}%
\pgfpathlineto{\pgfqpoint{1.327114in}{0.961542in}}%
\pgfpathlineto{\pgfqpoint{1.304327in}{0.963661in}}%
\pgfpathlineto{\pgfqpoint{1.304909in}{0.969450in}}%
\pgfpathlineto{\pgfqpoint{1.293521in}{0.970602in}}%
\pgfpathlineto{\pgfqpoint{1.293808in}{0.973457in}}%
\pgfpathlineto{\pgfqpoint{1.298942in}{1.024507in}}%
\pgfpathlineto{\pgfqpoint{1.347270in}{1.020215in}}%
\pgfpathclose%
\pgfusepath{fill}%
\end{pgfscope}%
\begin{pgfscope}%
\pgfpathrectangle{\pgfqpoint{0.100000in}{0.100000in}}{\pgfqpoint{3.007045in}{1.925000in}}%
\pgfusepath{clip}%
\pgfsetbuttcap%
\pgfsetmiterjoin%
\definecolor{currentfill}{rgb}{0.093272,0.396878,0.673664}%
\pgfsetfillcolor{currentfill}%
\pgfsetlinewidth{0.000000pt}%
\definecolor{currentstroke}{rgb}{0.000000,0.000000,0.000000}%
\pgfsetstrokecolor{currentstroke}%
\pgfsetstrokeopacity{0.000000}%
\pgfsetdash{}{0pt}%
\pgfpathmoveto{\pgfqpoint{0.610456in}{0.538425in}}%
\pgfpathlineto{\pgfqpoint{0.607882in}{0.541289in}}%
\pgfpathlineto{\pgfqpoint{0.608049in}{0.546486in}}%
\pgfpathlineto{\pgfqpoint{0.607286in}{0.548897in}}%
\pgfpathlineto{\pgfqpoint{0.610143in}{0.549575in}}%
\pgfpathlineto{\pgfqpoint{0.608906in}{0.546215in}}%
\pgfpathlineto{\pgfqpoint{0.609173in}{0.543568in}}%
\pgfpathlineto{\pgfqpoint{0.611044in}{0.540017in}}%
\pgfpathlineto{\pgfqpoint{0.610456in}{0.538425in}}%
\pgfpathclose%
\pgfusepath{fill}%
\end{pgfscope}%
\begin{pgfscope}%
\pgfpathrectangle{\pgfqpoint{0.100000in}{0.100000in}}{\pgfqpoint{3.007045in}{1.925000in}}%
\pgfusepath{clip}%
\pgfsetbuttcap%
\pgfsetmiterjoin%
\definecolor{currentfill}{rgb}{0.093272,0.396878,0.673664}%
\pgfsetfillcolor{currentfill}%
\pgfsetlinewidth{0.000000pt}%
\definecolor{currentstroke}{rgb}{0.000000,0.000000,0.000000}%
\pgfsetstrokecolor{currentstroke}%
\pgfsetstrokeopacity{0.000000}%
\pgfsetdash{}{0pt}%
\pgfpathmoveto{\pgfqpoint{0.652105in}{0.476446in}}%
\pgfpathlineto{\pgfqpoint{0.651830in}{0.477944in}}%
\pgfpathlineto{\pgfqpoint{0.649670in}{0.479666in}}%
\pgfpathlineto{\pgfqpoint{0.648834in}{0.482194in}}%
\pgfpathlineto{\pgfqpoint{0.648697in}{0.484544in}}%
\pgfpathlineto{\pgfqpoint{0.647047in}{0.487579in}}%
\pgfpathlineto{\pgfqpoint{0.647342in}{0.491406in}}%
\pgfpathlineto{\pgfqpoint{0.648359in}{0.493435in}}%
\pgfpathlineto{\pgfqpoint{0.649744in}{0.493109in}}%
\pgfpathlineto{\pgfqpoint{0.652856in}{0.488673in}}%
\pgfpathlineto{\pgfqpoint{0.656744in}{0.489158in}}%
\pgfpathlineto{\pgfqpoint{0.659223in}{0.488519in}}%
\pgfpathlineto{\pgfqpoint{0.659628in}{0.487236in}}%
\pgfpathlineto{\pgfqpoint{0.661768in}{0.486250in}}%
\pgfpathlineto{\pgfqpoint{0.661612in}{0.484269in}}%
\pgfpathlineto{\pgfqpoint{0.663795in}{0.482269in}}%
\pgfpathlineto{\pgfqpoint{0.663710in}{0.479661in}}%
\pgfpathlineto{\pgfqpoint{0.661944in}{0.477630in}}%
\pgfpathlineto{\pgfqpoint{0.660195in}{0.476581in}}%
\pgfpathlineto{\pgfqpoint{0.659577in}{0.472416in}}%
\pgfpathlineto{\pgfqpoint{0.657595in}{0.473029in}}%
\pgfpathlineto{\pgfqpoint{0.655097in}{0.475183in}}%
\pgfpathlineto{\pgfqpoint{0.652105in}{0.476446in}}%
\pgfpathclose%
\pgfusepath{fill}%
\end{pgfscope}%
\begin{pgfscope}%
\pgfpathrectangle{\pgfqpoint{0.100000in}{0.100000in}}{\pgfqpoint{3.007045in}{1.925000in}}%
\pgfusepath{clip}%
\pgfsetbuttcap%
\pgfsetmiterjoin%
\definecolor{currentfill}{rgb}{0.093272,0.396878,0.673664}%
\pgfsetfillcolor{currentfill}%
\pgfsetlinewidth{0.000000pt}%
\definecolor{currentstroke}{rgb}{0.000000,0.000000,0.000000}%
\pgfsetstrokecolor{currentstroke}%
\pgfsetstrokeopacity{0.000000}%
\pgfsetdash{}{0pt}%
\pgfpathmoveto{\pgfqpoint{0.809857in}{0.421160in}}%
\pgfpathlineto{\pgfqpoint{0.796605in}{0.403352in}}%
\pgfpathlineto{\pgfqpoint{0.796295in}{0.405189in}}%
\pgfpathlineto{\pgfqpoint{0.792854in}{0.407923in}}%
\pgfpathlineto{\pgfqpoint{0.789615in}{0.403627in}}%
\pgfpathlineto{\pgfqpoint{0.788663in}{0.401187in}}%
\pgfpathlineto{\pgfqpoint{0.784950in}{0.396276in}}%
\pgfpathlineto{\pgfqpoint{0.775612in}{0.403305in}}%
\pgfpathlineto{\pgfqpoint{0.761232in}{0.414538in}}%
\pgfpathlineto{\pgfqpoint{0.750064in}{0.423440in}}%
\pgfpathlineto{\pgfqpoint{0.751507in}{0.425233in}}%
\pgfpathlineto{\pgfqpoint{0.748296in}{0.427861in}}%
\pgfpathlineto{\pgfqpoint{0.746839in}{0.426097in}}%
\pgfpathlineto{\pgfqpoint{0.745138in}{0.427348in}}%
\pgfpathlineto{\pgfqpoint{0.743807in}{0.425686in}}%
\pgfpathlineto{\pgfqpoint{0.742145in}{0.427074in}}%
\pgfpathlineto{\pgfqpoint{0.743447in}{0.428714in}}%
\pgfpathlineto{\pgfqpoint{0.734378in}{0.436156in}}%
\pgfpathlineto{\pgfqpoint{0.733009in}{0.434561in}}%
\pgfpathlineto{\pgfqpoint{0.731767in}{0.435552in}}%
\pgfpathlineto{\pgfqpoint{0.730325in}{0.433863in}}%
\pgfpathlineto{\pgfqpoint{0.728779in}{0.435116in}}%
\pgfpathlineto{\pgfqpoint{0.726974in}{0.433888in}}%
\pgfpathlineto{\pgfqpoint{0.724234in}{0.430706in}}%
\pgfpathlineto{\pgfqpoint{0.722451in}{0.432254in}}%
\pgfpathlineto{\pgfqpoint{0.721076in}{0.430661in}}%
\pgfpathlineto{\pgfqpoint{0.719488in}{0.432032in}}%
\pgfpathlineto{\pgfqpoint{0.716770in}{0.428913in}}%
\pgfpathlineto{\pgfqpoint{0.715184in}{0.430242in}}%
\pgfpathlineto{\pgfqpoint{0.713230in}{0.429162in}}%
\pgfpathlineto{\pgfqpoint{0.710582in}{0.426191in}}%
\pgfpathlineto{\pgfqpoint{0.709062in}{0.427491in}}%
\pgfpathlineto{\pgfqpoint{0.706317in}{0.424353in}}%
\pgfpathlineto{\pgfqpoint{0.704186in}{0.426137in}}%
\pgfpathlineto{\pgfqpoint{0.701434in}{0.423011in}}%
\pgfpathlineto{\pgfqpoint{0.699846in}{0.424401in}}%
\pgfpathlineto{\pgfqpoint{0.697072in}{0.421245in}}%
\pgfpathlineto{\pgfqpoint{0.695145in}{0.420096in}}%
\pgfpathlineto{\pgfqpoint{0.693623in}{0.421487in}}%
\pgfpathlineto{\pgfqpoint{0.691710in}{0.420379in}}%
\pgfpathlineto{\pgfqpoint{0.688938in}{0.417253in}}%
\pgfpathlineto{\pgfqpoint{0.687384in}{0.418587in}}%
\pgfpathlineto{\pgfqpoint{0.686127in}{0.417159in}}%
\pgfpathlineto{\pgfqpoint{0.684613in}{0.418474in}}%
\pgfpathlineto{\pgfqpoint{0.681584in}{0.415593in}}%
\pgfpathlineto{\pgfqpoint{0.681431in}{0.415610in}}%
\pgfpathlineto{\pgfqpoint{0.678710in}{0.418051in}}%
\pgfpathlineto{\pgfqpoint{0.677618in}{0.416243in}}%
\pgfpathlineto{\pgfqpoint{0.676726in}{0.415156in}}%
\pgfpathlineto{\pgfqpoint{0.670768in}{0.416525in}}%
\pgfpathlineto{\pgfqpoint{0.671401in}{0.419366in}}%
\pgfpathlineto{\pgfqpoint{0.674143in}{0.420400in}}%
\pgfpathlineto{\pgfqpoint{0.677137in}{0.423980in}}%
\pgfpathlineto{\pgfqpoint{0.678234in}{0.426008in}}%
\pgfpathlineto{\pgfqpoint{0.678253in}{0.428853in}}%
\pgfpathlineto{\pgfqpoint{0.679315in}{0.431241in}}%
\pgfpathlineto{\pgfqpoint{0.680432in}{0.430540in}}%
\pgfpathlineto{\pgfqpoint{0.681892in}{0.432397in}}%
\pgfpathlineto{\pgfqpoint{0.685663in}{0.432497in}}%
\pgfpathlineto{\pgfqpoint{0.686784in}{0.434521in}}%
\pgfpathlineto{\pgfqpoint{0.688419in}{0.439211in}}%
\pgfpathlineto{\pgfqpoint{0.688889in}{0.441645in}}%
\pgfpathlineto{\pgfqpoint{0.690056in}{0.443550in}}%
\pgfpathlineto{\pgfqpoint{0.690186in}{0.446179in}}%
\pgfpathlineto{\pgfqpoint{0.692680in}{0.447617in}}%
\pgfpathlineto{\pgfqpoint{0.692600in}{0.450868in}}%
\pgfpathlineto{\pgfqpoint{0.690991in}{0.451746in}}%
\pgfpathlineto{\pgfqpoint{0.688110in}{0.448749in}}%
\pgfpathlineto{\pgfqpoint{0.685848in}{0.448554in}}%
\pgfpathlineto{\pgfqpoint{0.684889in}{0.449901in}}%
\pgfpathlineto{\pgfqpoint{0.680145in}{0.450430in}}%
\pgfpathlineto{\pgfqpoint{0.675467in}{0.452772in}}%
\pgfpathlineto{\pgfqpoint{0.673367in}{0.454382in}}%
\pgfpathlineto{\pgfqpoint{0.670995in}{0.457386in}}%
\pgfpathlineto{\pgfqpoint{0.670409in}{0.459079in}}%
\pgfpathlineto{\pgfqpoint{0.671394in}{0.461167in}}%
\pgfpathlineto{\pgfqpoint{0.672846in}{0.461844in}}%
\pgfpathlineto{\pgfqpoint{0.672427in}{0.464293in}}%
\pgfpathlineto{\pgfqpoint{0.672655in}{0.467035in}}%
\pgfpathlineto{\pgfqpoint{0.673992in}{0.467451in}}%
\pgfpathlineto{\pgfqpoint{0.672827in}{0.471052in}}%
\pgfpathlineto{\pgfqpoint{0.673952in}{0.471769in}}%
\pgfpathlineto{\pgfqpoint{0.671748in}{0.473544in}}%
\pgfpathlineto{\pgfqpoint{0.671417in}{0.476039in}}%
\pgfpathlineto{\pgfqpoint{0.672034in}{0.478136in}}%
\pgfpathlineto{\pgfqpoint{0.673606in}{0.477783in}}%
\pgfpathlineto{\pgfqpoint{0.674154in}{0.479291in}}%
\pgfpathlineto{\pgfqpoint{0.671865in}{0.479944in}}%
\pgfpathlineto{\pgfqpoint{0.671055in}{0.482481in}}%
\pgfpathlineto{\pgfqpoint{0.674291in}{0.481708in}}%
\pgfpathlineto{\pgfqpoint{0.678236in}{0.483360in}}%
\pgfpathlineto{\pgfqpoint{0.680888in}{0.483413in}}%
\pgfpathlineto{\pgfqpoint{0.679590in}{0.486507in}}%
\pgfpathlineto{\pgfqpoint{0.680253in}{0.488047in}}%
\pgfpathlineto{\pgfqpoint{0.682631in}{0.488869in}}%
\pgfpathlineto{\pgfqpoint{0.682850in}{0.492468in}}%
\pgfpathlineto{\pgfqpoint{0.680782in}{0.491150in}}%
\pgfpathlineto{\pgfqpoint{0.679720in}{0.494316in}}%
\pgfpathlineto{\pgfqpoint{0.682387in}{0.496907in}}%
\pgfpathlineto{\pgfqpoint{0.680696in}{0.499993in}}%
\pgfpathlineto{\pgfqpoint{0.681929in}{0.501414in}}%
\pgfpathlineto{\pgfqpoint{0.683631in}{0.501161in}}%
\pgfpathlineto{\pgfqpoint{0.684549in}{0.502655in}}%
\pgfpathlineto{\pgfqpoint{0.684007in}{0.504296in}}%
\pgfpathlineto{\pgfqpoint{0.681235in}{0.504565in}}%
\pgfpathlineto{\pgfqpoint{0.682519in}{0.506916in}}%
\pgfpathlineto{\pgfqpoint{0.686294in}{0.507512in}}%
\pgfpathlineto{\pgfqpoint{0.687166in}{0.510160in}}%
\pgfpathlineto{\pgfqpoint{0.689933in}{0.507661in}}%
\pgfpathlineto{\pgfqpoint{0.692175in}{0.508261in}}%
\pgfpathlineto{\pgfqpoint{0.693238in}{0.510668in}}%
\pgfpathlineto{\pgfqpoint{0.695178in}{0.511813in}}%
\pgfpathlineto{\pgfqpoint{0.704163in}{0.514029in}}%
\pgfpathlineto{\pgfqpoint{0.708310in}{0.514388in}}%
\pgfpathlineto{\pgfqpoint{0.709700in}{0.512813in}}%
\pgfpathlineto{\pgfqpoint{0.712038in}{0.513756in}}%
\pgfpathlineto{\pgfqpoint{0.712689in}{0.516016in}}%
\pgfpathlineto{\pgfqpoint{0.717386in}{0.519817in}}%
\pgfpathlineto{\pgfqpoint{0.721882in}{0.521512in}}%
\pgfpathlineto{\pgfqpoint{0.726704in}{0.521249in}}%
\pgfpathlineto{\pgfqpoint{0.729612in}{0.518760in}}%
\pgfpathlineto{\pgfqpoint{0.730444in}{0.514976in}}%
\pgfpathlineto{\pgfqpoint{0.730775in}{0.511205in}}%
\pgfpathlineto{\pgfqpoint{0.731879in}{0.508992in}}%
\pgfpathlineto{\pgfqpoint{0.734192in}{0.507876in}}%
\pgfpathlineto{\pgfqpoint{0.738215in}{0.508943in}}%
\pgfpathlineto{\pgfqpoint{0.741050in}{0.508543in}}%
\pgfpathlineto{\pgfqpoint{0.742738in}{0.506896in}}%
\pgfpathlineto{\pgfqpoint{0.741312in}{0.505279in}}%
\pgfpathlineto{\pgfqpoint{0.742884in}{0.503844in}}%
\pgfpathlineto{\pgfqpoint{0.741434in}{0.502134in}}%
\pgfpathlineto{\pgfqpoint{0.742883in}{0.500799in}}%
\pgfpathlineto{\pgfqpoint{0.741587in}{0.499176in}}%
\pgfpathlineto{\pgfqpoint{0.751165in}{0.490590in}}%
\pgfpathlineto{\pgfqpoint{0.750170in}{0.491012in}}%
\pgfpathlineto{\pgfqpoint{0.748071in}{0.489521in}}%
\pgfpathlineto{\pgfqpoint{0.742309in}{0.483156in}}%
\pgfpathlineto{\pgfqpoint{0.741905in}{0.483472in}}%
\pgfpathlineto{\pgfqpoint{0.736206in}{0.477104in}}%
\pgfpathlineto{\pgfqpoint{0.737384in}{0.476047in}}%
\pgfpathlineto{\pgfqpoint{0.733232in}{0.471329in}}%
\pgfpathlineto{\pgfqpoint{0.741635in}{0.463832in}}%
\pgfpathlineto{\pgfqpoint{0.747517in}{0.458771in}}%
\pgfpathlineto{\pgfqpoint{0.749271in}{0.460813in}}%
\pgfpathlineto{\pgfqpoint{0.755991in}{0.455118in}}%
\pgfpathlineto{\pgfqpoint{0.757361in}{0.456738in}}%
\pgfpathlineto{\pgfqpoint{0.770089in}{0.446056in}}%
\pgfpathlineto{\pgfqpoint{0.768555in}{0.444177in}}%
\pgfpathlineto{\pgfqpoint{0.776021in}{0.437944in}}%
\pgfpathlineto{\pgfqpoint{0.787335in}{0.429069in}}%
\pgfpathlineto{\pgfqpoint{0.796550in}{0.422002in}}%
\pgfpathlineto{\pgfqpoint{0.798438in}{0.420763in}}%
\pgfpathlineto{\pgfqpoint{0.800378in}{0.423260in}}%
\pgfpathlineto{\pgfqpoint{0.803696in}{0.420629in}}%
\pgfpathlineto{\pgfqpoint{0.805001in}{0.422296in}}%
\pgfpathlineto{\pgfqpoint{0.807974in}{0.419972in}}%
\pgfpathlineto{\pgfqpoint{0.809857in}{0.421160in}}%
\pgfpathclose%
\pgfusepath{fill}%
\end{pgfscope}%
\begin{pgfscope}%
\pgfpathrectangle{\pgfqpoint{0.100000in}{0.100000in}}{\pgfqpoint{3.007045in}{1.925000in}}%
\pgfusepath{clip}%
\pgfsetbuttcap%
\pgfsetmiterjoin%
\definecolor{currentfill}{rgb}{0.491765,0.721968,0.854779}%
\pgfsetfillcolor{currentfill}%
\pgfsetlinewidth{0.000000pt}%
\definecolor{currentstroke}{rgb}{0.000000,0.000000,0.000000}%
\pgfsetstrokecolor{currentstroke}%
\pgfsetstrokeopacity{0.000000}%
\pgfsetdash{}{0pt}%
\pgfpathmoveto{\pgfqpoint{0.860069in}{1.641241in}}%
\pgfpathlineto{\pgfqpoint{0.857696in}{1.638685in}}%
\pgfpathlineto{\pgfqpoint{0.853704in}{1.621066in}}%
\pgfpathlineto{\pgfqpoint{0.841346in}{1.613085in}}%
\pgfpathlineto{\pgfqpoint{0.834363in}{1.617027in}}%
\pgfpathlineto{\pgfqpoint{0.830622in}{1.610723in}}%
\pgfpathlineto{\pgfqpoint{0.824810in}{1.609423in}}%
\pgfpathlineto{\pgfqpoint{0.822960in}{1.604421in}}%
\pgfpathlineto{\pgfqpoint{0.825412in}{1.600994in}}%
\pgfpathlineto{\pgfqpoint{0.824171in}{1.596209in}}%
\pgfpathlineto{\pgfqpoint{0.820142in}{1.591987in}}%
\pgfpathlineto{\pgfqpoint{0.811192in}{1.587476in}}%
\pgfpathlineto{\pgfqpoint{0.808189in}{1.587646in}}%
\pgfpathlineto{\pgfqpoint{0.781320in}{1.594007in}}%
\pgfpathlineto{\pgfqpoint{0.779546in}{1.588394in}}%
\pgfpathlineto{\pgfqpoint{0.774977in}{1.589715in}}%
\pgfpathlineto{\pgfqpoint{0.777625in}{1.600841in}}%
\pgfpathlineto{\pgfqpoint{0.780379in}{1.600195in}}%
\pgfpathlineto{\pgfqpoint{0.783037in}{1.611433in}}%
\pgfpathlineto{\pgfqpoint{0.775430in}{1.609514in}}%
\pgfpathlineto{\pgfqpoint{0.770478in}{1.610681in}}%
\pgfpathlineto{\pgfqpoint{0.765709in}{1.615018in}}%
\pgfpathlineto{\pgfqpoint{0.767582in}{1.623021in}}%
\pgfpathlineto{\pgfqpoint{0.763942in}{1.627183in}}%
\pgfpathlineto{\pgfqpoint{0.766659in}{1.638492in}}%
\pgfpathlineto{\pgfqpoint{0.754151in}{1.641671in}}%
\pgfpathlineto{\pgfqpoint{0.758646in}{1.647388in}}%
\pgfpathlineto{\pgfqpoint{0.759381in}{1.652797in}}%
\pgfpathlineto{\pgfqpoint{0.763180in}{1.655715in}}%
\pgfpathlineto{\pgfqpoint{0.766636in}{1.659171in}}%
\pgfpathlineto{\pgfqpoint{0.770539in}{1.666797in}}%
\pgfpathlineto{\pgfqpoint{0.786144in}{1.663045in}}%
\pgfpathlineto{\pgfqpoint{0.786641in}{1.653470in}}%
\pgfpathlineto{\pgfqpoint{0.792690in}{1.650559in}}%
\pgfpathlineto{\pgfqpoint{0.802276in}{1.654267in}}%
\pgfpathlineto{\pgfqpoint{0.860069in}{1.641241in}}%
\pgfpathclose%
\pgfusepath{fill}%
\end{pgfscope}%
\begin{pgfscope}%
\pgfpathrectangle{\pgfqpoint{0.100000in}{0.100000in}}{\pgfqpoint{3.007045in}{1.925000in}}%
\pgfusepath{clip}%
\pgfsetbuttcap%
\pgfsetmiterjoin%
\definecolor{currentfill}{rgb}{0.548235,0.752972,0.866959}%
\pgfsetfillcolor{currentfill}%
\pgfsetlinewidth{0.000000pt}%
\definecolor{currentstroke}{rgb}{0.000000,0.000000,0.000000}%
\pgfsetstrokecolor{currentstroke}%
\pgfsetstrokeopacity{0.000000}%
\pgfsetdash{}{0pt}%
\pgfpathmoveto{\pgfqpoint{2.081022in}{1.010070in}}%
\pgfpathlineto{\pgfqpoint{2.077300in}{1.012074in}}%
\pgfpathlineto{\pgfqpoint{2.072603in}{1.010193in}}%
\pgfpathlineto{\pgfqpoint{2.067443in}{1.014097in}}%
\pgfpathlineto{\pgfqpoint{2.059538in}{1.028875in}}%
\pgfpathlineto{\pgfqpoint{2.064198in}{1.036075in}}%
\pgfpathlineto{\pgfqpoint{2.058254in}{1.047208in}}%
\pgfpathlineto{\pgfqpoint{2.060389in}{1.049493in}}%
\pgfpathlineto{\pgfqpoint{2.058159in}{1.055368in}}%
\pgfpathlineto{\pgfqpoint{2.054473in}{1.056687in}}%
\pgfpathlineto{\pgfqpoint{2.045818in}{1.065000in}}%
\pgfpathlineto{\pgfqpoint{2.040262in}{1.068526in}}%
\pgfpathlineto{\pgfqpoint{2.034604in}{1.066814in}}%
\pgfpathlineto{\pgfqpoint{2.031875in}{1.071974in}}%
\pgfpathlineto{\pgfqpoint{2.024754in}{1.078038in}}%
\pgfpathlineto{\pgfqpoint{2.020734in}{1.079515in}}%
\pgfpathlineto{\pgfqpoint{2.029399in}{1.083124in}}%
\pgfpathlineto{\pgfqpoint{2.029064in}{1.088879in}}%
\pgfpathlineto{\pgfqpoint{2.046290in}{1.089601in}}%
\pgfpathlineto{\pgfqpoint{2.075029in}{1.090942in}}%
\pgfpathlineto{\pgfqpoint{2.075282in}{1.085153in}}%
\pgfpathlineto{\pgfqpoint{2.098280in}{1.086724in}}%
\pgfpathlineto{\pgfqpoint{2.099179in}{1.072352in}}%
\pgfpathlineto{\pgfqpoint{2.100489in}{1.052130in}}%
\pgfpathlineto{\pgfqpoint{2.101574in}{1.034888in}}%
\pgfpathlineto{\pgfqpoint{2.091233in}{1.034083in}}%
\pgfpathlineto{\pgfqpoint{2.090698in}{1.026843in}}%
\pgfpathlineto{\pgfqpoint{2.085427in}{1.025478in}}%
\pgfpathlineto{\pgfqpoint{2.078167in}{1.013748in}}%
\pgfpathlineto{\pgfqpoint{2.081022in}{1.010070in}}%
\pgfpathclose%
\pgfusepath{fill}%
\end{pgfscope}%
\begin{pgfscope}%
\pgfpathrectangle{\pgfqpoint{0.100000in}{0.100000in}}{\pgfqpoint{3.007045in}{1.925000in}}%
\pgfusepath{clip}%
\pgfsetbuttcap%
\pgfsetmiterjoin%
\definecolor{currentfill}{rgb}{0.401953,0.670296,0.832326}%
\pgfsetfillcolor{currentfill}%
\pgfsetlinewidth{0.000000pt}%
\definecolor{currentstroke}{rgb}{0.000000,0.000000,0.000000}%
\pgfsetstrokecolor{currentstroke}%
\pgfsetstrokeopacity{0.000000}%
\pgfsetdash{}{0pt}%
\pgfpathmoveto{\pgfqpoint{2.721216in}{1.279405in}}%
\pgfpathlineto{\pgfqpoint{2.716371in}{1.278367in}}%
\pgfpathlineto{\pgfqpoint{2.716037in}{1.278304in}}%
\pgfpathlineto{\pgfqpoint{2.706007in}{1.289043in}}%
\pgfpathlineto{\pgfqpoint{2.701481in}{1.290448in}}%
\pgfpathlineto{\pgfqpoint{2.697015in}{1.296543in}}%
\pgfpathlineto{\pgfqpoint{2.690460in}{1.296019in}}%
\pgfpathlineto{\pgfqpoint{2.687398in}{1.300784in}}%
\pgfpathlineto{\pgfqpoint{2.693385in}{1.305552in}}%
\pgfpathlineto{\pgfqpoint{2.684366in}{1.322430in}}%
\pgfpathlineto{\pgfqpoint{2.690299in}{1.329057in}}%
\pgfpathlineto{\pgfqpoint{2.680799in}{1.334029in}}%
\pgfpathlineto{\pgfqpoint{2.694956in}{1.344848in}}%
\pgfpathlineto{\pgfqpoint{2.698167in}{1.347287in}}%
\pgfpathlineto{\pgfqpoint{2.698167in}{1.352791in}}%
\pgfpathlineto{\pgfqpoint{2.701015in}{1.357842in}}%
\pgfpathlineto{\pgfqpoint{2.711768in}{1.357635in}}%
\pgfpathlineto{\pgfqpoint{2.725895in}{1.348689in}}%
\pgfpathlineto{\pgfqpoint{2.720167in}{1.343542in}}%
\pgfpathlineto{\pgfqpoint{2.741068in}{1.332531in}}%
\pgfpathlineto{\pgfqpoint{2.735728in}{1.317531in}}%
\pgfpathlineto{\pgfqpoint{2.725428in}{1.306338in}}%
\pgfpathlineto{\pgfqpoint{2.726798in}{1.302007in}}%
\pgfpathlineto{\pgfqpoint{2.725049in}{1.295828in}}%
\pgfpathlineto{\pgfqpoint{2.725778in}{1.289195in}}%
\pgfpathlineto{\pgfqpoint{2.724100in}{1.282597in}}%
\pgfpathlineto{\pgfqpoint{2.721216in}{1.279405in}}%
\pgfpathclose%
\pgfusepath{fill}%
\end{pgfscope}%
\begin{pgfscope}%
\pgfpathrectangle{\pgfqpoint{0.100000in}{0.100000in}}{\pgfqpoint{3.007045in}{1.925000in}}%
\pgfusepath{clip}%
\pgfsetbuttcap%
\pgfsetmiterjoin%
\definecolor{currentfill}{rgb}{0.248166,0.561892,0.770980}%
\pgfsetfillcolor{currentfill}%
\pgfsetlinewidth{0.000000pt}%
\definecolor{currentstroke}{rgb}{0.000000,0.000000,0.000000}%
\pgfsetstrokecolor{currentstroke}%
\pgfsetstrokeopacity{0.000000}%
\pgfsetdash{}{0pt}%
\pgfpathmoveto{\pgfqpoint{2.422704in}{0.621971in}}%
\pgfpathlineto{\pgfqpoint{2.415134in}{0.621502in}}%
\pgfpathlineto{\pgfqpoint{2.412441in}{0.646597in}}%
\pgfpathlineto{\pgfqpoint{2.397277in}{0.644995in}}%
\pgfpathlineto{\pgfqpoint{2.397002in}{0.647345in}}%
\pgfpathlineto{\pgfqpoint{2.395188in}{0.664380in}}%
\pgfpathlineto{\pgfqpoint{2.414666in}{0.666517in}}%
\pgfpathlineto{\pgfqpoint{2.412962in}{0.682194in}}%
\pgfpathlineto{\pgfqpoint{2.408654in}{0.685257in}}%
\pgfpathlineto{\pgfqpoint{2.404240in}{0.684768in}}%
\pgfpathlineto{\pgfqpoint{2.402438in}{0.696626in}}%
\pgfpathlineto{\pgfqpoint{2.413072in}{0.697996in}}%
\pgfpathlineto{\pgfqpoint{2.412679in}{0.701290in}}%
\pgfpathlineto{\pgfqpoint{2.420123in}{0.701779in}}%
\pgfpathlineto{\pgfqpoint{2.422816in}{0.692785in}}%
\pgfpathlineto{\pgfqpoint{2.420802in}{0.691745in}}%
\pgfpathlineto{\pgfqpoint{2.421279in}{0.683375in}}%
\pgfpathlineto{\pgfqpoint{2.425943in}{0.682862in}}%
\pgfpathlineto{\pgfqpoint{2.428996in}{0.678445in}}%
\pgfpathlineto{\pgfqpoint{2.442052in}{0.679496in}}%
\pgfpathlineto{\pgfqpoint{2.443165in}{0.674743in}}%
\pgfpathlineto{\pgfqpoint{2.450078in}{0.667343in}}%
\pgfpathlineto{\pgfqpoint{2.450133in}{0.661231in}}%
\pgfpathlineto{\pgfqpoint{2.454390in}{0.661837in}}%
\pgfpathlineto{\pgfqpoint{2.457148in}{0.641063in}}%
\pgfpathlineto{\pgfqpoint{2.464523in}{0.639748in}}%
\pgfpathlineto{\pgfqpoint{2.469936in}{0.633604in}}%
\pgfpathlineto{\pgfqpoint{2.479352in}{0.633603in}}%
\pgfpathlineto{\pgfqpoint{2.481583in}{0.625736in}}%
\pgfpathlineto{\pgfqpoint{2.422704in}{0.621971in}}%
\pgfpathclose%
\pgfusepath{fill}%
\end{pgfscope}%
\begin{pgfscope}%
\pgfpathrectangle{\pgfqpoint{0.100000in}{0.100000in}}{\pgfqpoint{3.007045in}{1.925000in}}%
\pgfusepath{clip}%
\pgfsetbuttcap%
\pgfsetmiterjoin%
\definecolor{currentfill}{rgb}{0.356555,0.639293,0.814610}%
\pgfsetfillcolor{currentfill}%
\pgfsetlinewidth{0.000000pt}%
\definecolor{currentstroke}{rgb}{0.000000,0.000000,0.000000}%
\pgfsetstrokecolor{currentstroke}%
\pgfsetstrokeopacity{0.000000}%
\pgfsetdash{}{0pt}%
\pgfpathmoveto{\pgfqpoint{2.030481in}{0.763778in}}%
\pgfpathlineto{\pgfqpoint{2.023847in}{0.761087in}}%
\pgfpathlineto{\pgfqpoint{2.023674in}{0.765200in}}%
\pgfpathlineto{\pgfqpoint{2.022165in}{0.797050in}}%
\pgfpathlineto{\pgfqpoint{2.036938in}{0.797675in}}%
\pgfpathlineto{\pgfqpoint{2.030424in}{0.792915in}}%
\pgfpathlineto{\pgfqpoint{2.039761in}{0.792018in}}%
\pgfpathlineto{\pgfqpoint{2.039467in}{0.797809in}}%
\pgfpathlineto{\pgfqpoint{2.050947in}{0.799400in}}%
\pgfpathlineto{\pgfqpoint{2.050685in}{0.804294in}}%
\pgfpathlineto{\pgfqpoint{2.059401in}{0.802761in}}%
\pgfpathlineto{\pgfqpoint{2.073753in}{0.803617in}}%
\pgfpathlineto{\pgfqpoint{2.074344in}{0.794001in}}%
\pgfpathlineto{\pgfqpoint{2.075429in}{0.776700in}}%
\pgfpathlineto{\pgfqpoint{2.078426in}{0.775122in}}%
\pgfpathlineto{\pgfqpoint{2.079034in}{0.765390in}}%
\pgfpathlineto{\pgfqpoint{2.065349in}{0.764544in}}%
\pgfpathlineto{\pgfqpoint{2.062975in}{0.759697in}}%
\pgfpathlineto{\pgfqpoint{2.039047in}{0.766251in}}%
\pgfpathlineto{\pgfqpoint{2.033226in}{0.766351in}}%
\pgfpathlineto{\pgfqpoint{2.030481in}{0.763778in}}%
\pgfpathclose%
\pgfusepath{fill}%
\end{pgfscope}%
\begin{pgfscope}%
\pgfpathrectangle{\pgfqpoint{0.100000in}{0.100000in}}{\pgfqpoint{3.007045in}{1.925000in}}%
\pgfusepath{clip}%
\pgfsetbuttcap%
\pgfsetmiterjoin%
\definecolor{currentfill}{rgb}{0.207566,0.521292,0.750065}%
\pgfsetfillcolor{currentfill}%
\pgfsetlinewidth{0.000000pt}%
\definecolor{currentstroke}{rgb}{0.000000,0.000000,0.000000}%
\pgfsetstrokecolor{currentstroke}%
\pgfsetstrokeopacity{0.000000}%
\pgfsetdash{}{0pt}%
\pgfpathmoveto{\pgfqpoint{1.611326in}{0.679134in}}%
\pgfpathlineto{\pgfqpoint{1.607101in}{0.688968in}}%
\pgfpathlineto{\pgfqpoint{1.600832in}{0.707630in}}%
\pgfpathlineto{\pgfqpoint{1.578119in}{0.708542in}}%
\pgfpathlineto{\pgfqpoint{1.572479in}{0.708902in}}%
\pgfpathlineto{\pgfqpoint{1.573659in}{0.737980in}}%
\pgfpathlineto{\pgfqpoint{1.581896in}{0.737526in}}%
\pgfpathlineto{\pgfqpoint{1.582071in}{0.741206in}}%
\pgfpathlineto{\pgfqpoint{1.610081in}{0.739788in}}%
\pgfpathlineto{\pgfqpoint{1.611193in}{0.768407in}}%
\pgfpathlineto{\pgfqpoint{1.607913in}{0.770715in}}%
\pgfpathlineto{\pgfqpoint{1.608887in}{0.798669in}}%
\pgfpathlineto{\pgfqpoint{1.614953in}{0.795791in}}%
\pgfpathlineto{\pgfqpoint{1.625906in}{0.804863in}}%
\pgfpathlineto{\pgfqpoint{1.631785in}{0.798613in}}%
\pgfpathlineto{\pgfqpoint{1.636022in}{0.799536in}}%
\pgfpathlineto{\pgfqpoint{1.634992in}{0.767696in}}%
\pgfpathlineto{\pgfqpoint{1.640734in}{0.767480in}}%
\pgfpathlineto{\pgfqpoint{1.639106in}{0.738203in}}%
\pgfpathlineto{\pgfqpoint{1.659429in}{0.737338in}}%
\pgfpathlineto{\pgfqpoint{1.658448in}{0.708465in}}%
\pgfpathlineto{\pgfqpoint{1.655743in}{0.708579in}}%
\pgfpathlineto{\pgfqpoint{1.655346in}{0.689754in}}%
\pgfpathlineto{\pgfqpoint{1.633343in}{0.684748in}}%
\pgfpathlineto{\pgfqpoint{1.628904in}{0.682713in}}%
\pgfpathlineto{\pgfqpoint{1.624705in}{0.686669in}}%
\pgfpathlineto{\pgfqpoint{1.611326in}{0.679134in}}%
\pgfpathclose%
\pgfusepath{fill}%
\end{pgfscope}%
\begin{pgfscope}%
\pgfpathrectangle{\pgfqpoint{0.100000in}{0.100000in}}{\pgfqpoint{3.007045in}{1.925000in}}%
\pgfusepath{clip}%
\pgfsetbuttcap%
\pgfsetmiterjoin%
\definecolor{currentfill}{rgb}{0.366644,0.646182,0.818547}%
\pgfsetfillcolor{currentfill}%
\pgfsetlinewidth{0.000000pt}%
\definecolor{currentstroke}{rgb}{0.000000,0.000000,0.000000}%
\pgfsetstrokecolor{currentstroke}%
\pgfsetstrokeopacity{0.000000}%
\pgfsetdash{}{0pt}%
\pgfpathmoveto{\pgfqpoint{2.265651in}{0.893385in}}%
\pgfpathlineto{\pgfqpoint{2.241623in}{0.891209in}}%
\pgfpathlineto{\pgfqpoint{2.240394in}{0.900117in}}%
\pgfpathlineto{\pgfqpoint{2.237551in}{0.900340in}}%
\pgfpathlineto{\pgfqpoint{2.235740in}{0.906849in}}%
\pgfpathlineto{\pgfqpoint{2.223823in}{0.914321in}}%
\pgfpathlineto{\pgfqpoint{2.220248in}{0.918653in}}%
\pgfpathlineto{\pgfqpoint{2.219736in}{0.935214in}}%
\pgfpathlineto{\pgfqpoint{2.226333in}{0.935993in}}%
\pgfpathlineto{\pgfqpoint{2.228851in}{0.933436in}}%
\pgfpathlineto{\pgfqpoint{2.240739in}{0.937396in}}%
\pgfpathlineto{\pgfqpoint{2.251280in}{0.935449in}}%
\pgfpathlineto{\pgfqpoint{2.254813in}{0.936868in}}%
\pgfpathlineto{\pgfqpoint{2.257324in}{0.931388in}}%
\pgfpathlineto{\pgfqpoint{2.261477in}{0.928516in}}%
\pgfpathlineto{\pgfqpoint{2.262417in}{0.916982in}}%
\pgfpathlineto{\pgfqpoint{2.261069in}{0.912869in}}%
\pgfpathlineto{\pgfqpoint{2.264140in}{0.908633in}}%
\pgfpathlineto{\pgfqpoint{2.265651in}{0.893385in}}%
\pgfpathclose%
\pgfusepath{fill}%
\end{pgfscope}%
\begin{pgfscope}%
\pgfpathrectangle{\pgfqpoint{0.100000in}{0.100000in}}{\pgfqpoint{3.007045in}{1.925000in}}%
\pgfusepath{clip}%
\pgfsetbuttcap%
\pgfsetmiterjoin%
\definecolor{currentfill}{rgb}{0.516863,0.735748,0.860192}%
\pgfsetfillcolor{currentfill}%
\pgfsetlinewidth{0.000000pt}%
\definecolor{currentstroke}{rgb}{0.000000,0.000000,0.000000}%
\pgfsetstrokecolor{currentstroke}%
\pgfsetstrokeopacity{0.000000}%
\pgfsetdash{}{0pt}%
\pgfpathmoveto{\pgfqpoint{2.141820in}{0.745927in}}%
\pgfpathlineto{\pgfqpoint{2.116053in}{0.743899in}}%
\pgfpathlineto{\pgfqpoint{2.114414in}{0.767709in}}%
\pgfpathlineto{\pgfqpoint{2.122360in}{0.768291in}}%
\pgfpathlineto{\pgfqpoint{2.121230in}{0.782758in}}%
\pgfpathlineto{\pgfqpoint{2.128772in}{0.785353in}}%
\pgfpathlineto{\pgfqpoint{2.131570in}{0.789934in}}%
\pgfpathlineto{\pgfqpoint{2.129211in}{0.792870in}}%
\pgfpathlineto{\pgfqpoint{2.136914in}{0.795797in}}%
\pgfpathlineto{\pgfqpoint{2.139121in}{0.799917in}}%
\pgfpathlineto{\pgfqpoint{2.143299in}{0.800190in}}%
\pgfpathlineto{\pgfqpoint{2.142921in}{0.786166in}}%
\pgfpathlineto{\pgfqpoint{2.166832in}{0.787387in}}%
\pgfpathlineto{\pgfqpoint{2.168941in}{0.762907in}}%
\pgfpathlineto{\pgfqpoint{2.163129in}{0.759429in}}%
\pgfpathlineto{\pgfqpoint{2.158791in}{0.754738in}}%
\pgfpathlineto{\pgfqpoint{2.154101in}{0.754354in}}%
\pgfpathlineto{\pgfqpoint{2.151248in}{0.751063in}}%
\pgfpathlineto{\pgfqpoint{2.141932in}{0.750061in}}%
\pgfpathlineto{\pgfqpoint{2.141820in}{0.745927in}}%
\pgfpathclose%
\pgfusepath{fill}%
\end{pgfscope}%
\begin{pgfscope}%
\pgfpathrectangle{\pgfqpoint{0.100000in}{0.100000in}}{\pgfqpoint{3.007045in}{1.925000in}}%
\pgfusepath{clip}%
\pgfsetbuttcap%
\pgfsetmiterjoin%
\definecolor{currentfill}{rgb}{0.657132,0.808105,0.895486}%
\pgfsetfillcolor{currentfill}%
\pgfsetlinewidth{0.000000pt}%
\definecolor{currentstroke}{rgb}{0.000000,0.000000,0.000000}%
\pgfsetstrokecolor{currentstroke}%
\pgfsetstrokeopacity{0.000000}%
\pgfsetdash{}{0pt}%
\pgfpathmoveto{\pgfqpoint{2.695962in}{1.404095in}}%
\pgfpathlineto{\pgfqpoint{2.689741in}{1.426141in}}%
\pgfpathlineto{\pgfqpoint{2.669988in}{1.422124in}}%
\pgfpathlineto{\pgfqpoint{2.667190in}{1.440512in}}%
\pgfpathlineto{\pgfqpoint{2.672580in}{1.444093in}}%
\pgfpathlineto{\pgfqpoint{2.680763in}{1.444336in}}%
\pgfpathlineto{\pgfqpoint{2.679109in}{1.451464in}}%
\pgfpathlineto{\pgfqpoint{2.697553in}{1.455914in}}%
\pgfpathlineto{\pgfqpoint{2.700669in}{1.446056in}}%
\pgfpathlineto{\pgfqpoint{2.710631in}{1.447455in}}%
\pgfpathlineto{\pgfqpoint{2.711428in}{1.444046in}}%
\pgfpathlineto{\pgfqpoint{2.721879in}{1.446280in}}%
\pgfpathlineto{\pgfqpoint{2.723883in}{1.436359in}}%
\pgfpathlineto{\pgfqpoint{2.727454in}{1.434295in}}%
\pgfpathlineto{\pgfqpoint{2.721527in}{1.432985in}}%
\pgfpathlineto{\pgfqpoint{2.727566in}{1.410162in}}%
\pgfpathlineto{\pgfqpoint{2.695962in}{1.404095in}}%
\pgfpathclose%
\pgfusepath{fill}%
\end{pgfscope}%
\begin{pgfscope}%
\pgfpathrectangle{\pgfqpoint{0.100000in}{0.100000in}}{\pgfqpoint{3.007045in}{1.925000in}}%
\pgfusepath{clip}%
\pgfsetbuttcap%
\pgfsetmiterjoin%
\definecolor{currentfill}{rgb}{0.316201,0.611734,0.798862}%
\pgfsetfillcolor{currentfill}%
\pgfsetlinewidth{0.000000pt}%
\definecolor{currentstroke}{rgb}{0.000000,0.000000,0.000000}%
\pgfsetstrokecolor{currentstroke}%
\pgfsetstrokeopacity{0.000000}%
\pgfsetdash{}{0pt}%
\pgfpathmoveto{\pgfqpoint{1.700709in}{0.760907in}}%
\pgfpathlineto{\pgfqpoint{1.695830in}{0.764016in}}%
\pgfpathlineto{\pgfqpoint{1.670614in}{0.764978in}}%
\pgfpathlineto{\pgfqpoint{1.664958in}{0.765812in}}%
\pgfpathlineto{\pgfqpoint{1.665670in}{0.801052in}}%
\pgfpathlineto{\pgfqpoint{1.668937in}{0.795613in}}%
\pgfpathlineto{\pgfqpoint{1.678257in}{0.793022in}}%
\pgfpathlineto{\pgfqpoint{1.683354in}{0.804224in}}%
\pgfpathlineto{\pgfqpoint{1.695332in}{0.814267in}}%
\pgfpathlineto{\pgfqpoint{1.718089in}{0.813933in}}%
\pgfpathlineto{\pgfqpoint{1.720987in}{0.811838in}}%
\pgfpathlineto{\pgfqpoint{1.720879in}{0.800114in}}%
\pgfpathlineto{\pgfqpoint{1.726173in}{0.795162in}}%
\pgfpathlineto{\pgfqpoint{1.730486in}{0.795042in}}%
\pgfpathlineto{\pgfqpoint{1.725337in}{0.792070in}}%
\pgfpathlineto{\pgfqpoint{1.724887in}{0.764449in}}%
\pgfpathlineto{\pgfqpoint{1.700709in}{0.760907in}}%
\pgfpathclose%
\pgfusepath{fill}%
\end{pgfscope}%
\begin{pgfscope}%
\pgfpathrectangle{\pgfqpoint{0.100000in}{0.100000in}}{\pgfqpoint{3.007045in}{1.925000in}}%
\pgfusepath{clip}%
\pgfsetbuttcap%
\pgfsetmiterjoin%
\definecolor{currentfill}{rgb}{0.396909,0.666851,0.830358}%
\pgfsetfillcolor{currentfill}%
\pgfsetlinewidth{0.000000pt}%
\definecolor{currentstroke}{rgb}{0.000000,0.000000,0.000000}%
\pgfsetstrokecolor{currentstroke}%
\pgfsetstrokeopacity{0.000000}%
\pgfsetdash{}{0pt}%
\pgfpathmoveto{\pgfqpoint{2.151467in}{0.577078in}}%
\pgfpathlineto{\pgfqpoint{2.146781in}{0.617810in}}%
\pgfpathlineto{\pgfqpoint{2.143322in}{0.646719in}}%
\pgfpathlineto{\pgfqpoint{2.141224in}{0.664018in}}%
\pgfpathlineto{\pgfqpoint{2.162408in}{0.665683in}}%
\pgfpathlineto{\pgfqpoint{2.161161in}{0.669683in}}%
\pgfpathlineto{\pgfqpoint{2.156603in}{0.673169in}}%
\pgfpathlineto{\pgfqpoint{2.161771in}{0.684999in}}%
\pgfpathlineto{\pgfqpoint{2.184529in}{0.686866in}}%
\pgfpathlineto{\pgfqpoint{2.186978in}{0.688005in}}%
\pgfpathlineto{\pgfqpoint{2.186226in}{0.696389in}}%
\pgfpathlineto{\pgfqpoint{2.191937in}{0.696888in}}%
\pgfpathlineto{\pgfqpoint{2.191190in}{0.704948in}}%
\pgfpathlineto{\pgfqpoint{2.193997in}{0.705851in}}%
\pgfpathlineto{\pgfqpoint{2.202255in}{0.702095in}}%
\pgfpathlineto{\pgfqpoint{2.211673in}{0.692874in}}%
\pgfpathlineto{\pgfqpoint{2.226858in}{0.694238in}}%
\pgfpathlineto{\pgfqpoint{2.228144in}{0.679871in}}%
\pgfpathlineto{\pgfqpoint{2.194774in}{0.676909in}}%
\pgfpathlineto{\pgfqpoint{2.194637in}{0.668142in}}%
\pgfpathlineto{\pgfqpoint{2.191786in}{0.667860in}}%
\pgfpathlineto{\pgfqpoint{2.192871in}{0.650628in}}%
\pgfpathlineto{\pgfqpoint{2.187693in}{0.649271in}}%
\pgfpathlineto{\pgfqpoint{2.184550in}{0.645723in}}%
\pgfpathlineto{\pgfqpoint{2.185584in}{0.641031in}}%
\pgfpathlineto{\pgfqpoint{2.191464in}{0.637662in}}%
\pgfpathlineto{\pgfqpoint{2.193761in}{0.621372in}}%
\pgfpathlineto{\pgfqpoint{2.192424in}{0.612482in}}%
\pgfpathlineto{\pgfqpoint{2.199079in}{0.604683in}}%
\pgfpathlineto{\pgfqpoint{2.206456in}{0.600906in}}%
\pgfpathlineto{\pgfqpoint{2.207470in}{0.596991in}}%
\pgfpathlineto{\pgfqpoint{2.205014in}{0.590892in}}%
\pgfpathlineto{\pgfqpoint{2.210055in}{0.585128in}}%
\pgfpathlineto{\pgfqpoint{2.206102in}{0.581924in}}%
\pgfpathlineto{\pgfqpoint{2.202262in}{0.574181in}}%
\pgfpathlineto{\pgfqpoint{2.194379in}{0.571471in}}%
\pgfpathlineto{\pgfqpoint{2.188678in}{0.573144in}}%
\pgfpathlineto{\pgfqpoint{2.179172in}{0.581268in}}%
\pgfpathlineto{\pgfqpoint{2.177184in}{0.585669in}}%
\pgfpathlineto{\pgfqpoint{2.178736in}{0.589471in}}%
\pgfpathlineto{\pgfqpoint{2.176721in}{0.596910in}}%
\pgfpathlineto{\pgfqpoint{2.173439in}{0.598944in}}%
\pgfpathlineto{\pgfqpoint{2.169431in}{0.595956in}}%
\pgfpathlineto{\pgfqpoint{2.167717in}{0.586842in}}%
\pgfpathlineto{\pgfqpoint{2.168005in}{0.577878in}}%
\pgfpathlineto{\pgfqpoint{2.166722in}{0.574427in}}%
\pgfpathlineto{\pgfqpoint{2.154103in}{0.578631in}}%
\pgfpathlineto{\pgfqpoint{2.151467in}{0.577078in}}%
\pgfpathclose%
\pgfusepath{fill}%
\end{pgfscope}%
\begin{pgfscope}%
\pgfpathrectangle{\pgfqpoint{0.100000in}{0.100000in}}{\pgfqpoint{3.007045in}{1.925000in}}%
\pgfusepath{clip}%
\pgfsetbuttcap%
\pgfsetmiterjoin%
\definecolor{currentfill}{rgb}{0.510588,0.732303,0.858839}%
\pgfsetfillcolor{currentfill}%
\pgfsetlinewidth{0.000000pt}%
\definecolor{currentstroke}{rgb}{0.000000,0.000000,0.000000}%
\pgfsetstrokecolor{currentstroke}%
\pgfsetstrokeopacity{0.000000}%
\pgfsetdash{}{0pt}%
\pgfpathmoveto{\pgfqpoint{1.217220in}{0.903588in}}%
\pgfpathlineto{\pgfqpoint{1.214018in}{0.875058in}}%
\pgfpathlineto{\pgfqpoint{1.212799in}{0.875172in}}%
\pgfpathlineto{\pgfqpoint{1.210832in}{0.858083in}}%
\pgfpathlineto{\pgfqpoint{1.210171in}{0.852394in}}%
\pgfpathlineto{\pgfqpoint{1.177035in}{0.856282in}}%
\pgfpathlineto{\pgfqpoint{1.150364in}{0.859555in}}%
\pgfpathlineto{\pgfqpoint{1.151795in}{0.871370in}}%
\pgfpathlineto{\pgfqpoint{1.133587in}{0.879798in}}%
\pgfpathlineto{\pgfqpoint{1.127835in}{0.883763in}}%
\pgfpathlineto{\pgfqpoint{1.082425in}{0.889855in}}%
\pgfpathlineto{\pgfqpoint{1.047397in}{0.894955in}}%
\pgfpathlineto{\pgfqpoint{1.011345in}{0.900484in}}%
\pgfpathlineto{\pgfqpoint{1.015230in}{0.925309in}}%
\pgfpathlineto{\pgfqpoint{1.046069in}{0.920555in}}%
\pgfpathlineto{\pgfqpoint{1.049496in}{0.943259in}}%
\pgfpathlineto{\pgfqpoint{1.111260in}{0.934336in}}%
\pgfpathlineto{\pgfqpoint{1.117536in}{0.979664in}}%
\pgfpathlineto{\pgfqpoint{1.100725in}{0.982093in}}%
\pgfpathlineto{\pgfqpoint{1.102849in}{0.996413in}}%
\pgfpathlineto{\pgfqpoint{1.141795in}{0.990914in}}%
\pgfpathlineto{\pgfqpoint{1.139908in}{0.976686in}}%
\pgfpathlineto{\pgfqpoint{1.173700in}{0.972532in}}%
\pgfpathlineto{\pgfqpoint{1.165289in}{0.970829in}}%
\pgfpathlineto{\pgfqpoint{1.164202in}{0.962160in}}%
\pgfpathlineto{\pgfqpoint{1.171643in}{0.956432in}}%
\pgfpathlineto{\pgfqpoint{1.166043in}{0.909544in}}%
\pgfpathlineto{\pgfqpoint{1.194470in}{0.906193in}}%
\pgfpathlineto{\pgfqpoint{1.217220in}{0.903588in}}%
\pgfpathclose%
\pgfusepath{fill}%
\end{pgfscope}%
\begin{pgfscope}%
\pgfpathrectangle{\pgfqpoint{0.100000in}{0.100000in}}{\pgfqpoint{3.007045in}{1.925000in}}%
\pgfusepath{clip}%
\pgfsetbuttcap%
\pgfsetmiterjoin%
\definecolor{currentfill}{rgb}{0.535686,0.746082,0.864252}%
\pgfsetfillcolor{currentfill}%
\pgfsetlinewidth{0.000000pt}%
\definecolor{currentstroke}{rgb}{0.000000,0.000000,0.000000}%
\pgfsetstrokecolor{currentstroke}%
\pgfsetstrokeopacity{0.000000}%
\pgfsetdash{}{0pt}%
\pgfpathmoveto{\pgfqpoint{1.677967in}{0.558409in}}%
\pgfpathlineto{\pgfqpoint{1.672046in}{0.560070in}}%
\pgfpathlineto{\pgfqpoint{1.665066in}{0.567789in}}%
\pgfpathlineto{\pgfqpoint{1.662849in}{0.573743in}}%
\pgfpathlineto{\pgfqpoint{1.659758in}{0.576115in}}%
\pgfpathlineto{\pgfqpoint{1.679583in}{0.587280in}}%
\pgfpathlineto{\pgfqpoint{1.678179in}{0.593593in}}%
\pgfpathlineto{\pgfqpoint{1.672091in}{0.600546in}}%
\pgfpathlineto{\pgfqpoint{1.670600in}{0.609460in}}%
\pgfpathlineto{\pgfqpoint{1.668209in}{0.612458in}}%
\pgfpathlineto{\pgfqpoint{1.681563in}{0.619854in}}%
\pgfpathlineto{\pgfqpoint{1.697477in}{0.628654in}}%
\pgfpathlineto{\pgfqpoint{1.696915in}{0.623980in}}%
\pgfpathlineto{\pgfqpoint{1.701562in}{0.603092in}}%
\pgfpathlineto{\pgfqpoint{1.705582in}{0.593025in}}%
\pgfpathlineto{\pgfqpoint{1.723044in}{0.595566in}}%
\pgfpathlineto{\pgfqpoint{1.724776in}{0.579988in}}%
\pgfpathlineto{\pgfqpoint{1.726035in}{0.554362in}}%
\pgfpathlineto{\pgfqpoint{1.709639in}{0.553402in}}%
\pgfpathlineto{\pgfqpoint{1.709428in}{0.556522in}}%
\pgfpathlineto{\pgfqpoint{1.704636in}{0.564588in}}%
\pgfpathlineto{\pgfqpoint{1.696897in}{0.563937in}}%
\pgfpathlineto{\pgfqpoint{1.690544in}{0.561282in}}%
\pgfpathlineto{\pgfqpoint{1.677967in}{0.558409in}}%
\pgfpathclose%
\pgfusepath{fill}%
\end{pgfscope}%
\begin{pgfscope}%
\pgfpathrectangle{\pgfqpoint{0.100000in}{0.100000in}}{\pgfqpoint{3.007045in}{1.925000in}}%
\pgfusepath{clip}%
\pgfsetbuttcap%
\pgfsetmiterjoin%
\definecolor{currentfill}{rgb}{0.516863,0.735748,0.860192}%
\pgfsetfillcolor{currentfill}%
\pgfsetlinewidth{0.000000pt}%
\definecolor{currentstroke}{rgb}{0.000000,0.000000,0.000000}%
\pgfsetstrokecolor{currentstroke}%
\pgfsetstrokeopacity{0.000000}%
\pgfsetdash{}{0pt}%
\pgfpathmoveto{\pgfqpoint{2.509239in}{1.087027in}}%
\pgfpathlineto{\pgfqpoint{2.495503in}{1.095488in}}%
\pgfpathlineto{\pgfqpoint{2.490450in}{1.093853in}}%
\pgfpathlineto{\pgfqpoint{2.489686in}{1.089390in}}%
\pgfpathlineto{\pgfqpoint{2.484474in}{1.093009in}}%
\pgfpathlineto{\pgfqpoint{2.476705in}{1.103396in}}%
\pgfpathlineto{\pgfqpoint{2.471855in}{1.105407in}}%
\pgfpathlineto{\pgfqpoint{2.467580in}{1.113861in}}%
\pgfpathlineto{\pgfqpoint{2.470294in}{1.118309in}}%
\pgfpathlineto{\pgfqpoint{2.476974in}{1.118028in}}%
\pgfpathlineto{\pgfqpoint{2.477515in}{1.130920in}}%
\pgfpathlineto{\pgfqpoint{2.481701in}{1.138352in}}%
\pgfpathlineto{\pgfqpoint{2.490413in}{1.135206in}}%
\pgfpathlineto{\pgfqpoint{2.494710in}{1.137649in}}%
\pgfpathlineto{\pgfqpoint{2.501225in}{1.130708in}}%
\pgfpathlineto{\pgfqpoint{2.510167in}{1.130498in}}%
\pgfpathlineto{\pgfqpoint{2.522292in}{1.144995in}}%
\pgfpathlineto{\pgfqpoint{2.526634in}{1.142930in}}%
\pgfpathlineto{\pgfqpoint{2.527646in}{1.135652in}}%
\pgfpathlineto{\pgfqpoint{2.533411in}{1.132073in}}%
\pgfpathlineto{\pgfqpoint{2.538421in}{1.132193in}}%
\pgfpathlineto{\pgfqpoint{2.548827in}{1.135761in}}%
\pgfpathlineto{\pgfqpoint{2.547549in}{1.130953in}}%
\pgfpathlineto{\pgfqpoint{2.538295in}{1.117062in}}%
\pgfpathlineto{\pgfqpoint{2.535565in}{1.108737in}}%
\pgfpathlineto{\pgfqpoint{2.527642in}{1.106923in}}%
\pgfpathlineto{\pgfqpoint{2.516736in}{1.108073in}}%
\pgfpathlineto{\pgfqpoint{2.509239in}{1.087027in}}%
\pgfpathclose%
\pgfusepath{fill}%
\end{pgfscope}%
\begin{pgfscope}%
\pgfpathrectangle{\pgfqpoint{0.100000in}{0.100000in}}{\pgfqpoint{3.007045in}{1.925000in}}%
\pgfusepath{clip}%
\pgfsetbuttcap%
\pgfsetmiterjoin%
\definecolor{currentfill}{rgb}{0.701423,0.826928,0.910988}%
\pgfsetfillcolor{currentfill}%
\pgfsetlinewidth{0.000000pt}%
\definecolor{currentstroke}{rgb}{0.000000,0.000000,0.000000}%
\pgfsetstrokecolor{currentstroke}%
\pgfsetstrokeopacity{0.000000}%
\pgfsetdash{}{0pt}%
\pgfpathmoveto{\pgfqpoint{2.768356in}{1.454967in}}%
\pgfpathlineto{\pgfqpoint{2.739778in}{1.426983in}}%
\pgfpathlineto{\pgfqpoint{2.733867in}{1.427016in}}%
\pgfpathlineto{\pgfqpoint{2.732166in}{1.431244in}}%
\pgfpathlineto{\pgfqpoint{2.727454in}{1.434295in}}%
\pgfpathlineto{\pgfqpoint{2.723883in}{1.436359in}}%
\pgfpathlineto{\pgfqpoint{2.721879in}{1.446280in}}%
\pgfpathlineto{\pgfqpoint{2.711428in}{1.444046in}}%
\pgfpathlineto{\pgfqpoint{2.710631in}{1.447455in}}%
\pgfpathlineto{\pgfqpoint{2.700669in}{1.446056in}}%
\pgfpathlineto{\pgfqpoint{2.697553in}{1.455914in}}%
\pgfpathlineto{\pgfqpoint{2.692078in}{1.475538in}}%
\pgfpathlineto{\pgfqpoint{2.719947in}{1.482951in}}%
\pgfpathlineto{\pgfqpoint{2.722365in}{1.491084in}}%
\pgfpathlineto{\pgfqpoint{2.726801in}{1.495572in}}%
\pgfpathlineto{\pgfqpoint{2.737175in}{1.492249in}}%
\pgfpathlineto{\pgfqpoint{2.737431in}{1.497269in}}%
\pgfpathlineto{\pgfqpoint{2.743391in}{1.496243in}}%
\pgfpathlineto{\pgfqpoint{2.749311in}{1.495307in}}%
\pgfpathlineto{\pgfqpoint{2.749558in}{1.490026in}}%
\pgfpathlineto{\pgfqpoint{2.753121in}{1.482442in}}%
\pgfpathlineto{\pgfqpoint{2.750879in}{1.474563in}}%
\pgfpathlineto{\pgfqpoint{2.756645in}{1.469554in}}%
\pgfpathlineto{\pgfqpoint{2.766813in}{1.466787in}}%
\pgfpathlineto{\pgfqpoint{2.763804in}{1.456101in}}%
\pgfpathlineto{\pgfqpoint{2.768356in}{1.454967in}}%
\pgfpathclose%
\pgfusepath{fill}%
\end{pgfscope}%
\begin{pgfscope}%
\pgfpathrectangle{\pgfqpoint{0.100000in}{0.100000in}}{\pgfqpoint{3.007045in}{1.925000in}}%
\pgfusepath{clip}%
\pgfsetbuttcap%
\pgfsetmiterjoin%
\definecolor{currentfill}{rgb}{0.296025,0.597955,0.790988}%
\pgfsetfillcolor{currentfill}%
\pgfsetlinewidth{0.000000pt}%
\definecolor{currentstroke}{rgb}{0.000000,0.000000,0.000000}%
\pgfsetstrokecolor{currentstroke}%
\pgfsetstrokeopacity{0.000000}%
\pgfsetdash{}{0pt}%
\pgfpathmoveto{\pgfqpoint{1.614445in}{1.686458in}}%
\pgfpathlineto{\pgfqpoint{1.579862in}{1.687980in}}%
\pgfpathlineto{\pgfqpoint{1.580956in}{1.710973in}}%
\pgfpathlineto{\pgfqpoint{1.580095in}{1.722634in}}%
\pgfpathlineto{\pgfqpoint{1.614607in}{1.721113in}}%
\pgfpathlineto{\pgfqpoint{1.615296in}{1.709477in}}%
\pgfpathlineto{\pgfqpoint{1.614445in}{1.686458in}}%
\pgfpathclose%
\pgfusepath{fill}%
\end{pgfscope}%
\begin{pgfscope}%
\pgfpathrectangle{\pgfqpoint{0.100000in}{0.100000in}}{\pgfqpoint{3.007045in}{1.925000in}}%
\pgfusepath{clip}%
\pgfsetbuttcap%
\pgfsetmiterjoin%
\definecolor{currentfill}{rgb}{0.256286,0.570012,0.775163}%
\pgfsetfillcolor{currentfill}%
\pgfsetlinewidth{0.000000pt}%
\definecolor{currentstroke}{rgb}{0.000000,0.000000,0.000000}%
\pgfsetstrokecolor{currentstroke}%
\pgfsetstrokeopacity{0.000000}%
\pgfsetdash{}{0pt}%
\pgfpathmoveto{\pgfqpoint{1.627761in}{0.861901in}}%
\pgfpathlineto{\pgfqpoint{1.627401in}{0.850427in}}%
\pgfpathlineto{\pgfqpoint{1.604562in}{0.851135in}}%
\pgfpathlineto{\pgfqpoint{1.604926in}{0.862611in}}%
\pgfpathlineto{\pgfqpoint{1.576210in}{0.863673in}}%
\pgfpathlineto{\pgfqpoint{1.576781in}{0.879670in}}%
\pgfpathlineto{\pgfqpoint{1.577863in}{0.909634in}}%
\pgfpathlineto{\pgfqpoint{1.594488in}{0.908994in}}%
\pgfpathlineto{\pgfqpoint{1.594438in}{0.897482in}}%
\pgfpathlineto{\pgfqpoint{1.609800in}{0.896970in}}%
\pgfpathlineto{\pgfqpoint{1.619336in}{0.893761in}}%
\pgfpathlineto{\pgfqpoint{1.628500in}{0.893638in}}%
\pgfpathlineto{\pgfqpoint{1.627761in}{0.861901in}}%
\pgfpathclose%
\pgfusepath{fill}%
\end{pgfscope}%
\begin{pgfscope}%
\pgfpathrectangle{\pgfqpoint{0.100000in}{0.100000in}}{\pgfqpoint{3.007045in}{1.925000in}}%
\pgfusepath{clip}%
\pgfsetbuttcap%
\pgfsetmiterjoin%
\definecolor{currentfill}{rgb}{0.341423,0.628958,0.808704}%
\pgfsetfillcolor{currentfill}%
\pgfsetlinewidth{0.000000pt}%
\definecolor{currentstroke}{rgb}{0.000000,0.000000,0.000000}%
\pgfsetstrokecolor{currentstroke}%
\pgfsetstrokeopacity{0.000000}%
\pgfsetdash{}{0pt}%
\pgfpathmoveto{\pgfqpoint{1.821649in}{1.429987in}}%
\pgfpathlineto{\pgfqpoint{1.822052in}{1.390899in}}%
\pgfpathlineto{\pgfqpoint{1.776480in}{1.390760in}}%
\pgfpathlineto{\pgfqpoint{1.753560in}{1.390798in}}%
\pgfpathlineto{\pgfqpoint{1.753660in}{1.413597in}}%
\pgfpathlineto{\pgfqpoint{1.753741in}{1.429789in}}%
\pgfpathlineto{\pgfqpoint{1.776405in}{1.429765in}}%
\pgfpathlineto{\pgfqpoint{1.776471in}{1.413556in}}%
\pgfpathlineto{\pgfqpoint{1.799116in}{1.413633in}}%
\pgfpathlineto{\pgfqpoint{1.799028in}{1.429855in}}%
\pgfpathlineto{\pgfqpoint{1.821649in}{1.429987in}}%
\pgfpathclose%
\pgfusepath{fill}%
\end{pgfscope}%
\begin{pgfscope}%
\pgfpathrectangle{\pgfqpoint{0.100000in}{0.100000in}}{\pgfqpoint{3.007045in}{1.925000in}}%
\pgfusepath{clip}%
\pgfsetbuttcap%
\pgfsetmiterjoin%
\definecolor{currentfill}{rgb}{0.381776,0.656517,0.824452}%
\pgfsetfillcolor{currentfill}%
\pgfsetlinewidth{0.000000pt}%
\definecolor{currentstroke}{rgb}{0.000000,0.000000,0.000000}%
\pgfsetstrokecolor{currentstroke}%
\pgfsetstrokeopacity{0.000000}%
\pgfsetdash{}{0pt}%
\pgfpathmoveto{\pgfqpoint{2.370833in}{1.107548in}}%
\pgfpathlineto{\pgfqpoint{2.371064in}{1.101491in}}%
\pgfpathlineto{\pgfqpoint{2.367300in}{1.093259in}}%
\pgfpathlineto{\pgfqpoint{2.362396in}{1.092621in}}%
\pgfpathlineto{\pgfqpoint{2.355335in}{1.096669in}}%
\pgfpathlineto{\pgfqpoint{2.344556in}{1.097617in}}%
\pgfpathlineto{\pgfqpoint{2.336897in}{1.108708in}}%
\pgfpathlineto{\pgfqpoint{2.347373in}{1.122044in}}%
\pgfpathlineto{\pgfqpoint{2.353384in}{1.121802in}}%
\pgfpathlineto{\pgfqpoint{2.359204in}{1.116262in}}%
\pgfpathlineto{\pgfqpoint{2.370833in}{1.107548in}}%
\pgfpathclose%
\pgfusepath{fill}%
\end{pgfscope}%
\begin{pgfscope}%
\pgfpathrectangle{\pgfqpoint{0.100000in}{0.100000in}}{\pgfqpoint{3.007045in}{1.925000in}}%
\pgfusepath{clip}%
\pgfsetbuttcap%
\pgfsetmiterjoin%
\definecolor{currentfill}{rgb}{0.356555,0.639293,0.814610}%
\pgfsetfillcolor{currentfill}%
\pgfsetlinewidth{0.000000pt}%
\definecolor{currentstroke}{rgb}{0.000000,0.000000,0.000000}%
\pgfsetstrokecolor{currentstroke}%
\pgfsetstrokeopacity{0.000000}%
\pgfsetdash{}{0pt}%
\pgfpathmoveto{\pgfqpoint{2.172150in}{1.293918in}}%
\pgfpathlineto{\pgfqpoint{2.173650in}{1.276857in}}%
\pgfpathlineto{\pgfqpoint{2.171318in}{1.271716in}}%
\pgfpathlineto{\pgfqpoint{2.165690in}{1.271203in}}%
\pgfpathlineto{\pgfqpoint{2.166328in}{1.264572in}}%
\pgfpathlineto{\pgfqpoint{2.145056in}{1.262796in}}%
\pgfpathlineto{\pgfqpoint{2.141986in}{1.299655in}}%
\pgfpathlineto{\pgfqpoint{2.139901in}{1.326614in}}%
\pgfpathlineto{\pgfqpoint{2.145664in}{1.322773in}}%
\pgfpathlineto{\pgfqpoint{2.155565in}{1.322406in}}%
\pgfpathlineto{\pgfqpoint{2.168897in}{1.329206in}}%
\pgfpathlineto{\pgfqpoint{2.172150in}{1.293918in}}%
\pgfpathclose%
\pgfusepath{fill}%
\end{pgfscope}%
\begin{pgfscope}%
\pgfpathrectangle{\pgfqpoint{0.100000in}{0.100000in}}{\pgfqpoint{3.007045in}{1.925000in}}%
\pgfusepath{clip}%
\pgfsetbuttcap%
\pgfsetmiterjoin%
\definecolor{currentfill}{rgb}{0.316201,0.611734,0.798862}%
\pgfsetfillcolor{currentfill}%
\pgfsetlinewidth{0.000000pt}%
\definecolor{currentstroke}{rgb}{0.000000,0.000000,0.000000}%
\pgfsetstrokecolor{currentstroke}%
\pgfsetstrokeopacity{0.000000}%
\pgfsetdash{}{0pt}%
\pgfpathmoveto{\pgfqpoint{1.553928in}{0.768166in}}%
\pgfpathlineto{\pgfqpoint{1.583365in}{0.766882in}}%
\pgfpathlineto{\pgfqpoint{1.583500in}{0.771553in}}%
\pgfpathlineto{\pgfqpoint{1.607913in}{0.770715in}}%
\pgfpathlineto{\pgfqpoint{1.611193in}{0.768407in}}%
\pgfpathlineto{\pgfqpoint{1.610081in}{0.739788in}}%
\pgfpathlineto{\pgfqpoint{1.582071in}{0.741206in}}%
\pgfpathlineto{\pgfqpoint{1.581896in}{0.737526in}}%
\pgfpathlineto{\pgfqpoint{1.573659in}{0.737980in}}%
\pgfpathlineto{\pgfqpoint{1.523941in}{0.740360in}}%
\pgfpathlineto{\pgfqpoint{1.525314in}{0.769593in}}%
\pgfpathlineto{\pgfqpoint{1.553928in}{0.768166in}}%
\pgfpathclose%
\pgfusepath{fill}%
\end{pgfscope}%
\begin{pgfscope}%
\pgfpathrectangle{\pgfqpoint{0.100000in}{0.100000in}}{\pgfqpoint{3.007045in}{1.925000in}}%
\pgfusepath{clip}%
\pgfsetbuttcap%
\pgfsetmiterjoin%
\definecolor{currentfill}{rgb}{0.361599,0.642737,0.816578}%
\pgfsetfillcolor{currentfill}%
\pgfsetlinewidth{0.000000pt}%
\definecolor{currentstroke}{rgb}{0.000000,0.000000,0.000000}%
\pgfsetstrokecolor{currentstroke}%
\pgfsetstrokeopacity{0.000000}%
\pgfsetdash{}{0pt}%
\pgfpathmoveto{\pgfqpoint{1.939480in}{1.158394in}}%
\pgfpathlineto{\pgfqpoint{1.909303in}{1.157973in}}%
\pgfpathlineto{\pgfqpoint{1.909292in}{1.151308in}}%
\pgfpathlineto{\pgfqpoint{1.903384in}{1.151278in}}%
\pgfpathlineto{\pgfqpoint{1.889132in}{1.155695in}}%
\pgfpathlineto{\pgfqpoint{1.889421in}{1.180509in}}%
\pgfpathlineto{\pgfqpoint{1.880876in}{1.180425in}}%
\pgfpathlineto{\pgfqpoint{1.880879in}{1.202651in}}%
\pgfpathlineto{\pgfqpoint{1.906198in}{1.203318in}}%
\pgfpathlineto{\pgfqpoint{1.906383in}{1.197788in}}%
\pgfpathlineto{\pgfqpoint{1.926181in}{1.198155in}}%
\pgfpathlineto{\pgfqpoint{1.931854in}{1.198270in}}%
\pgfpathlineto{\pgfqpoint{1.932193in}{1.179188in}}%
\pgfpathlineto{\pgfqpoint{1.938875in}{1.179366in}}%
\pgfpathlineto{\pgfqpoint{1.939480in}{1.158394in}}%
\pgfpathclose%
\pgfusepath{fill}%
\end{pgfscope}%
\begin{pgfscope}%
\pgfpathrectangle{\pgfqpoint{0.100000in}{0.100000in}}{\pgfqpoint{3.007045in}{1.925000in}}%
\pgfusepath{clip}%
\pgfsetbuttcap%
\pgfsetmiterjoin%
\definecolor{currentfill}{rgb}{0.306113,0.604844,0.794925}%
\pgfsetfillcolor{currentfill}%
\pgfsetlinewidth{0.000000pt}%
\definecolor{currentstroke}{rgb}{0.000000,0.000000,0.000000}%
\pgfsetstrokecolor{currentstroke}%
\pgfsetstrokeopacity{0.000000}%
\pgfsetdash{}{0pt}%
\pgfpathmoveto{\pgfqpoint{1.719231in}{0.879638in}}%
\pgfpathlineto{\pgfqpoint{1.719456in}{0.888768in}}%
\pgfpathlineto{\pgfqpoint{1.719521in}{0.894505in}}%
\pgfpathlineto{\pgfqpoint{1.725208in}{0.894438in}}%
\pgfpathlineto{\pgfqpoint{1.728070in}{0.900172in}}%
\pgfpathlineto{\pgfqpoint{1.728181in}{0.905962in}}%
\pgfpathlineto{\pgfqpoint{1.734052in}{0.905918in}}%
\pgfpathlineto{\pgfqpoint{1.734135in}{0.917404in}}%
\pgfpathlineto{\pgfqpoint{1.731290in}{0.917416in}}%
\pgfpathlineto{\pgfqpoint{1.731364in}{0.926038in}}%
\pgfpathlineto{\pgfqpoint{1.737553in}{0.926013in}}%
\pgfpathlineto{\pgfqpoint{1.740818in}{0.919900in}}%
\pgfpathlineto{\pgfqpoint{1.746777in}{0.922489in}}%
\pgfpathlineto{\pgfqpoint{1.758114in}{0.923040in}}%
\pgfpathlineto{\pgfqpoint{1.757448in}{0.932993in}}%
\pgfpathlineto{\pgfqpoint{1.761242in}{0.940278in}}%
\pgfpathlineto{\pgfqpoint{1.766027in}{0.940252in}}%
\pgfpathlineto{\pgfqpoint{1.766020in}{0.946003in}}%
\pgfpathlineto{\pgfqpoint{1.771725in}{0.946010in}}%
\pgfpathlineto{\pgfqpoint{1.795605in}{0.946063in}}%
\pgfpathlineto{\pgfqpoint{1.799463in}{0.919527in}}%
\pgfpathlineto{\pgfqpoint{1.800606in}{0.911576in}}%
\pgfpathlineto{\pgfqpoint{1.765492in}{0.911520in}}%
\pgfpathlineto{\pgfqpoint{1.765215in}{0.904084in}}%
\pgfpathlineto{\pgfqpoint{1.768002in}{0.897857in}}%
\pgfpathlineto{\pgfqpoint{1.763675in}{0.893735in}}%
\pgfpathlineto{\pgfqpoint{1.759980in}{0.886994in}}%
\pgfpathlineto{\pgfqpoint{1.751555in}{0.890916in}}%
\pgfpathlineto{\pgfqpoint{1.741580in}{0.885648in}}%
\pgfpathlineto{\pgfqpoint{1.732284in}{0.882217in}}%
\pgfpathlineto{\pgfqpoint{1.725397in}{0.881888in}}%
\pgfpathlineto{\pgfqpoint{1.719231in}{0.879638in}}%
\pgfpathclose%
\pgfusepath{fill}%
\end{pgfscope}%
\begin{pgfscope}%
\pgfpathrectangle{\pgfqpoint{0.100000in}{0.100000in}}{\pgfqpoint{3.007045in}{1.925000in}}%
\pgfusepath{clip}%
\pgfsetbuttcap%
\pgfsetmiterjoin%
\definecolor{currentfill}{rgb}{0.447843,0.697855,0.845306}%
\pgfsetfillcolor{currentfill}%
\pgfsetlinewidth{0.000000pt}%
\definecolor{currentstroke}{rgb}{0.000000,0.000000,0.000000}%
\pgfsetstrokecolor{currentstroke}%
\pgfsetstrokeopacity{0.000000}%
\pgfsetdash{}{0pt}%
\pgfpathmoveto{\pgfqpoint{1.756771in}{1.198935in}}%
\pgfpathlineto{\pgfqpoint{1.751069in}{1.202842in}}%
\pgfpathlineto{\pgfqpoint{1.751959in}{1.207454in}}%
\pgfpathlineto{\pgfqpoint{1.748211in}{1.211203in}}%
\pgfpathlineto{\pgfqpoint{1.747673in}{1.215377in}}%
\pgfpathlineto{\pgfqpoint{1.741287in}{1.219682in}}%
\pgfpathlineto{\pgfqpoint{1.734017in}{1.237625in}}%
\pgfpathlineto{\pgfqpoint{1.790743in}{1.236578in}}%
\pgfpathlineto{\pgfqpoint{1.792476in}{1.224430in}}%
\pgfpathlineto{\pgfqpoint{1.792177in}{1.207263in}}%
\pgfpathlineto{\pgfqpoint{1.770162in}{1.207494in}}%
\pgfpathlineto{\pgfqpoint{1.772661in}{1.193312in}}%
\pgfpathlineto{\pgfqpoint{1.768029in}{1.189813in}}%
\pgfpathlineto{\pgfqpoint{1.762123in}{1.192600in}}%
\pgfpathlineto{\pgfqpoint{1.756771in}{1.198935in}}%
\pgfpathclose%
\pgfusepath{fill}%
\end{pgfscope}%
\begin{pgfscope}%
\pgfpathrectangle{\pgfqpoint{0.100000in}{0.100000in}}{\pgfqpoint{3.007045in}{1.925000in}}%
\pgfusepath{clip}%
\pgfsetbuttcap%
\pgfsetmiterjoin%
\definecolor{currentfill}{rgb}{0.573333,0.766751,0.872372}%
\pgfsetfillcolor{currentfill}%
\pgfsetlinewidth{0.000000pt}%
\definecolor{currentstroke}{rgb}{0.000000,0.000000,0.000000}%
\pgfsetstrokecolor{currentstroke}%
\pgfsetstrokeopacity{0.000000}%
\pgfsetdash{}{0pt}%
\pgfpathmoveto{\pgfqpoint{2.285385in}{0.933858in}}%
\pgfpathlineto{\pgfqpoint{2.278483in}{0.930848in}}%
\pgfpathlineto{\pgfqpoint{2.264733in}{0.928140in}}%
\pgfpathlineto{\pgfqpoint{2.261477in}{0.928516in}}%
\pgfpathlineto{\pgfqpoint{2.257324in}{0.931388in}}%
\pgfpathlineto{\pgfqpoint{2.254813in}{0.936868in}}%
\pgfpathlineto{\pgfqpoint{2.254335in}{0.941757in}}%
\pgfpathlineto{\pgfqpoint{2.257521in}{0.951222in}}%
\pgfpathlineto{\pgfqpoint{2.251263in}{0.956468in}}%
\pgfpathlineto{\pgfqpoint{2.248417in}{0.962697in}}%
\pgfpathlineto{\pgfqpoint{2.253187in}{0.966596in}}%
\pgfpathlineto{\pgfqpoint{2.258371in}{0.965969in}}%
\pgfpathlineto{\pgfqpoint{2.261147in}{0.968764in}}%
\pgfpathlineto{\pgfqpoint{2.263169in}{0.965439in}}%
\pgfpathlineto{\pgfqpoint{2.267876in}{0.965337in}}%
\pgfpathlineto{\pgfqpoint{2.269975in}{0.962101in}}%
\pgfpathlineto{\pgfqpoint{2.277359in}{0.967152in}}%
\pgfpathlineto{\pgfqpoint{2.285629in}{0.965628in}}%
\pgfpathlineto{\pgfqpoint{2.291000in}{0.961775in}}%
\pgfpathlineto{\pgfqpoint{2.293939in}{0.955321in}}%
\pgfpathlineto{\pgfqpoint{2.292762in}{0.944860in}}%
\pgfpathlineto{\pgfqpoint{2.285385in}{0.933858in}}%
\pgfpathclose%
\pgfusepath{fill}%
\end{pgfscope}%
\begin{pgfscope}%
\pgfpathrectangle{\pgfqpoint{0.100000in}{0.100000in}}{\pgfqpoint{3.007045in}{1.925000in}}%
\pgfusepath{clip}%
\pgfsetbuttcap%
\pgfsetmiterjoin%
\definecolor{currentfill}{rgb}{0.265759,0.577286,0.779177}%
\pgfsetfillcolor{currentfill}%
\pgfsetlinewidth{0.000000pt}%
\definecolor{currentstroke}{rgb}{0.000000,0.000000,0.000000}%
\pgfsetstrokecolor{currentstroke}%
\pgfsetstrokeopacity{0.000000}%
\pgfsetdash{}{0pt}%
\pgfpathmoveto{\pgfqpoint{1.193280in}{1.837329in}}%
\pgfpathlineto{\pgfqpoint{1.244920in}{1.830121in}}%
\pgfpathlineto{\pgfqpoint{1.285700in}{1.824835in}}%
\pgfpathlineto{\pgfqpoint{1.282530in}{1.813419in}}%
\pgfpathlineto{\pgfqpoint{1.288237in}{1.812721in}}%
\pgfpathlineto{\pgfqpoint{1.286131in}{1.795421in}}%
\pgfpathlineto{\pgfqpoint{1.295653in}{1.794279in}}%
\pgfpathlineto{\pgfqpoint{1.292930in}{1.771456in}}%
\pgfpathlineto{\pgfqpoint{1.291202in}{1.771653in}}%
\pgfpathlineto{\pgfqpoint{1.289529in}{1.757826in}}%
\pgfpathlineto{\pgfqpoint{1.272584in}{1.760976in}}%
\pgfpathlineto{\pgfqpoint{1.265605in}{1.764425in}}%
\pgfpathlineto{\pgfqpoint{1.263629in}{1.757459in}}%
\pgfpathlineto{\pgfqpoint{1.260272in}{1.755961in}}%
\pgfpathlineto{\pgfqpoint{1.255859in}{1.748834in}}%
\pgfpathlineto{\pgfqpoint{1.246943in}{1.744290in}}%
\pgfpathlineto{\pgfqpoint{1.241819in}{1.745330in}}%
\pgfpathlineto{\pgfqpoint{1.240243in}{1.741597in}}%
\pgfpathlineto{\pgfqpoint{1.224257in}{1.742711in}}%
\pgfpathlineto{\pgfqpoint{1.217097in}{1.745725in}}%
\pgfpathlineto{\pgfqpoint{1.214896in}{1.741346in}}%
\pgfpathlineto{\pgfqpoint{1.206785in}{1.744306in}}%
\pgfpathlineto{\pgfqpoint{1.201819in}{1.739713in}}%
\pgfpathlineto{\pgfqpoint{1.195585in}{1.736909in}}%
\pgfpathlineto{\pgfqpoint{1.192875in}{1.732905in}}%
\pgfpathlineto{\pgfqpoint{1.193102in}{1.740744in}}%
\pgfpathlineto{\pgfqpoint{1.187367in}{1.743646in}}%
\pgfpathlineto{\pgfqpoint{1.174368in}{1.744062in}}%
\pgfpathlineto{\pgfqpoint{1.170837in}{1.747018in}}%
\pgfpathlineto{\pgfqpoint{1.164883in}{1.747708in}}%
\pgfpathlineto{\pgfqpoint{1.156612in}{1.751283in}}%
\pgfpathlineto{\pgfqpoint{1.152002in}{1.758091in}}%
\pgfpathlineto{\pgfqpoint{1.154039in}{1.770483in}}%
\pgfpathlineto{\pgfqpoint{1.165616in}{1.768587in}}%
\pgfpathlineto{\pgfqpoint{1.167096in}{1.773103in}}%
\pgfpathlineto{\pgfqpoint{1.174638in}{1.770993in}}%
\pgfpathlineto{\pgfqpoint{1.177030in}{1.787023in}}%
\pgfpathlineto{\pgfqpoint{1.180237in}{1.801764in}}%
\pgfpathlineto{\pgfqpoint{1.182880in}{1.799848in}}%
\pgfpathlineto{\pgfqpoint{1.185494in}{1.809331in}}%
\pgfpathlineto{\pgfqpoint{1.187177in}{1.820830in}}%
\pgfpathlineto{\pgfqpoint{1.189784in}{1.820439in}}%
\pgfpathlineto{\pgfqpoint{1.193280in}{1.837329in}}%
\pgfpathclose%
\pgfusepath{fill}%
\end{pgfscope}%
\begin{pgfscope}%
\pgfpathrectangle{\pgfqpoint{0.100000in}{0.100000in}}{\pgfqpoint{3.007045in}{1.925000in}}%
\pgfusepath{clip}%
\pgfsetbuttcap%
\pgfsetmiterjoin%
\definecolor{currentfill}{rgb}{0.760477,0.852026,0.931657}%
\pgfsetfillcolor{currentfill}%
\pgfsetlinewidth{0.000000pt}%
\definecolor{currentstroke}{rgb}{0.000000,0.000000,0.000000}%
\pgfsetstrokecolor{currentstroke}%
\pgfsetstrokeopacity{0.000000}%
\pgfsetdash{}{0pt}%
\pgfpathmoveto{\pgfqpoint{2.642481in}{0.444772in}}%
\pgfpathlineto{\pgfqpoint{2.615609in}{0.440573in}}%
\pgfpathlineto{\pgfqpoint{2.614673in}{0.446167in}}%
\pgfpathlineto{\pgfqpoint{2.609094in}{0.445229in}}%
\pgfpathlineto{\pgfqpoint{2.602369in}{0.492044in}}%
\pgfpathlineto{\pgfqpoint{2.599254in}{0.497364in}}%
\pgfpathlineto{\pgfqpoint{2.599396in}{0.502567in}}%
\pgfpathlineto{\pgfqpoint{2.592396in}{0.508481in}}%
\pgfpathlineto{\pgfqpoint{2.591696in}{0.520372in}}%
\pgfpathlineto{\pgfqpoint{2.605371in}{0.522544in}}%
\pgfpathlineto{\pgfqpoint{2.616569in}{0.510498in}}%
\pgfpathlineto{\pgfqpoint{2.620884in}{0.502421in}}%
\pgfpathlineto{\pgfqpoint{2.617307in}{0.495527in}}%
\pgfpathlineto{\pgfqpoint{2.618107in}{0.489509in}}%
\pgfpathlineto{\pgfqpoint{2.622871in}{0.477008in}}%
\pgfpathlineto{\pgfqpoint{2.636790in}{0.456290in}}%
\pgfpathlineto{\pgfqpoint{2.642481in}{0.444772in}}%
\pgfpathclose%
\pgfusepath{fill}%
\end{pgfscope}%
\begin{pgfscope}%
\pgfpathrectangle{\pgfqpoint{0.100000in}{0.100000in}}{\pgfqpoint{3.007045in}{1.925000in}}%
\pgfusepath{clip}%
\pgfsetbuttcap%
\pgfsetmiterjoin%
\definecolor{currentfill}{rgb}{0.321246,0.615179,0.800830}%
\pgfsetfillcolor{currentfill}%
\pgfsetlinewidth{0.000000pt}%
\definecolor{currentstroke}{rgb}{0.000000,0.000000,0.000000}%
\pgfsetstrokecolor{currentstroke}%
\pgfsetstrokeopacity{0.000000}%
\pgfsetdash{}{0pt}%
\pgfpathmoveto{\pgfqpoint{1.865214in}{0.855241in}}%
\pgfpathlineto{\pgfqpoint{1.859472in}{0.853168in}}%
\pgfpathlineto{\pgfqpoint{1.842321in}{0.853020in}}%
\pgfpathlineto{\pgfqpoint{1.842357in}{0.871145in}}%
\pgfpathlineto{\pgfqpoint{1.822375in}{0.871354in}}%
\pgfpathlineto{\pgfqpoint{1.818712in}{0.873240in}}%
\pgfpathlineto{\pgfqpoint{1.818790in}{0.879030in}}%
\pgfpathlineto{\pgfqpoint{1.821794in}{0.883783in}}%
\pgfpathlineto{\pgfqpoint{1.822064in}{0.899000in}}%
\pgfpathlineto{\pgfqpoint{1.824480in}{0.903095in}}%
\pgfpathlineto{\pgfqpoint{1.821906in}{0.907584in}}%
\pgfpathlineto{\pgfqpoint{1.821996in}{0.913307in}}%
\pgfpathlineto{\pgfqpoint{1.826781in}{0.913265in}}%
\pgfpathlineto{\pgfqpoint{1.830635in}{0.919840in}}%
\pgfpathlineto{\pgfqpoint{1.851596in}{0.920385in}}%
\pgfpathlineto{\pgfqpoint{1.851607in}{0.918465in}}%
\pgfpathlineto{\pgfqpoint{1.889632in}{0.918440in}}%
\pgfpathlineto{\pgfqpoint{1.889712in}{0.906997in}}%
\pgfpathlineto{\pgfqpoint{1.887849in}{0.901272in}}%
\pgfpathlineto{\pgfqpoint{1.888066in}{0.882118in}}%
\pgfpathlineto{\pgfqpoint{1.884196in}{0.881637in}}%
\pgfpathlineto{\pgfqpoint{1.871864in}{0.868767in}}%
\pgfpathlineto{\pgfqpoint{1.864117in}{0.861985in}}%
\pgfpathlineto{\pgfqpoint{1.865214in}{0.855241in}}%
\pgfpathclose%
\pgfusepath{fill}%
\end{pgfscope}%
\begin{pgfscope}%
\pgfpathrectangle{\pgfqpoint{0.100000in}{0.100000in}}{\pgfqpoint{3.007045in}{1.925000in}}%
\pgfusepath{clip}%
\pgfsetbuttcap%
\pgfsetmiterjoin%
\definecolor{currentfill}{rgb}{0.371688,0.649627,0.820515}%
\pgfsetfillcolor{currentfill}%
\pgfsetlinewidth{0.000000pt}%
\definecolor{currentstroke}{rgb}{0.000000,0.000000,0.000000}%
\pgfsetstrokecolor{currentstroke}%
\pgfsetstrokeopacity{0.000000}%
\pgfsetdash{}{0pt}%
\pgfpathmoveto{\pgfqpoint{0.912796in}{0.778126in}}%
\pgfpathlineto{\pgfqpoint{0.851184in}{0.788621in}}%
\pgfpathlineto{\pgfqpoint{0.795773in}{0.799049in}}%
\pgfpathlineto{\pgfqpoint{0.754327in}{0.807329in}}%
\pgfpathlineto{\pgfqpoint{0.765905in}{0.863949in}}%
\pgfpathlineto{\pgfqpoint{0.774157in}{0.904247in}}%
\pgfpathlineto{\pgfqpoint{0.805861in}{0.897916in}}%
\pgfpathlineto{\pgfqpoint{0.829566in}{0.885369in}}%
\pgfpathlineto{\pgfqpoint{0.837638in}{0.894986in}}%
\pgfpathlineto{\pgfqpoint{0.860696in}{0.887387in}}%
\pgfpathlineto{\pgfqpoint{0.873121in}{0.885088in}}%
\pgfpathlineto{\pgfqpoint{0.875838in}{0.894759in}}%
\pgfpathlineto{\pgfqpoint{0.863233in}{0.897098in}}%
\pgfpathlineto{\pgfqpoint{0.868902in}{0.911853in}}%
\pgfpathlineto{\pgfqpoint{0.876178in}{0.916207in}}%
\pgfpathlineto{\pgfqpoint{0.883290in}{0.911640in}}%
\pgfpathlineto{\pgfqpoint{0.890889in}{0.911616in}}%
\pgfpathlineto{\pgfqpoint{0.896277in}{0.906751in}}%
\pgfpathlineto{\pgfqpoint{0.903247in}{0.903940in}}%
\pgfpathlineto{\pgfqpoint{0.911088in}{0.896114in}}%
\pgfpathlineto{\pgfqpoint{0.916273in}{0.895048in}}%
\pgfpathlineto{\pgfqpoint{0.913272in}{0.877904in}}%
\pgfpathlineto{\pgfqpoint{0.953797in}{0.870982in}}%
\pgfpathlineto{\pgfqpoint{0.949199in}{0.843379in}}%
\pgfpathlineto{\pgfqpoint{0.947992in}{0.836130in}}%
\pgfpathlineto{\pgfqpoint{0.938796in}{0.837657in}}%
\pgfpathlineto{\pgfqpoint{0.930194in}{0.836921in}}%
\pgfpathlineto{\pgfqpoint{0.923055in}{0.834676in}}%
\pgfpathlineto{\pgfqpoint{0.920548in}{0.822552in}}%
\pgfpathlineto{\pgfqpoint{0.912796in}{0.778126in}}%
\pgfpathclose%
\pgfusepath{fill}%
\end{pgfscope}%
\begin{pgfscope}%
\pgfpathrectangle{\pgfqpoint{0.100000in}{0.100000in}}{\pgfqpoint{3.007045in}{1.925000in}}%
\pgfusepath{clip}%
\pgfsetbuttcap%
\pgfsetmiterjoin%
\definecolor{currentfill}{rgb}{0.290980,0.594510,0.789020}%
\pgfsetfillcolor{currentfill}%
\pgfsetlinewidth{0.000000pt}%
\definecolor{currentstroke}{rgb}{0.000000,0.000000,0.000000}%
\pgfsetstrokecolor{currentstroke}%
\pgfsetstrokeopacity{0.000000}%
\pgfsetdash{}{0pt}%
\pgfpathmoveto{\pgfqpoint{1.710630in}{1.795591in}}%
\pgfpathlineto{\pgfqpoint{1.765577in}{1.795047in}}%
\pgfpathlineto{\pgfqpoint{1.765638in}{1.820857in}}%
\pgfpathlineto{\pgfqpoint{1.769776in}{1.818763in}}%
\pgfpathlineto{\pgfqpoint{1.774181in}{1.819903in}}%
\pgfpathlineto{\pgfqpoint{1.779963in}{1.814824in}}%
\pgfpathlineto{\pgfqpoint{1.786263in}{1.787384in}}%
\pgfpathlineto{\pgfqpoint{1.785945in}{1.780293in}}%
\pgfpathlineto{\pgfqpoint{1.790535in}{1.776268in}}%
\pgfpathlineto{\pgfqpoint{1.797461in}{1.775190in}}%
\pgfpathlineto{\pgfqpoint{1.797671in}{1.752902in}}%
\pgfpathlineto{\pgfqpoint{1.762966in}{1.752742in}}%
\pgfpathlineto{\pgfqpoint{1.762948in}{1.764365in}}%
\pgfpathlineto{\pgfqpoint{1.745680in}{1.764366in}}%
\pgfpathlineto{\pgfqpoint{1.710937in}{1.765107in}}%
\pgfpathlineto{\pgfqpoint{1.710357in}{1.776679in}}%
\pgfpathlineto{\pgfqpoint{1.710630in}{1.795591in}}%
\pgfpathclose%
\pgfusepath{fill}%
\end{pgfscope}%
\begin{pgfscope}%
\pgfpathrectangle{\pgfqpoint{0.100000in}{0.100000in}}{\pgfqpoint{3.007045in}{1.925000in}}%
\pgfusepath{clip}%
\pgfsetbuttcap%
\pgfsetmiterjoin%
\definecolor{currentfill}{rgb}{0.491765,0.721968,0.854779}%
\pgfsetfillcolor{currentfill}%
\pgfsetlinewidth{0.000000pt}%
\definecolor{currentstroke}{rgb}{0.000000,0.000000,0.000000}%
\pgfsetstrokecolor{currentstroke}%
\pgfsetstrokeopacity{0.000000}%
\pgfsetdash{}{0pt}%
\pgfpathmoveto{\pgfqpoint{0.851627in}{1.798562in}}%
\pgfpathlineto{\pgfqpoint{0.847085in}{1.804318in}}%
\pgfpathlineto{\pgfqpoint{0.850775in}{1.806624in}}%
\pgfpathlineto{\pgfqpoint{0.850754in}{1.813504in}}%
\pgfpathlineto{\pgfqpoint{0.847161in}{1.818241in}}%
\pgfpathlineto{\pgfqpoint{0.840821in}{1.835141in}}%
\pgfpathlineto{\pgfqpoint{0.844425in}{1.850601in}}%
\pgfpathlineto{\pgfqpoint{0.845385in}{1.846522in}}%
\pgfpathlineto{\pgfqpoint{0.859696in}{1.849285in}}%
\pgfpathlineto{\pgfqpoint{0.860078in}{1.837613in}}%
\pgfpathlineto{\pgfqpoint{0.863113in}{1.830232in}}%
\pgfpathlineto{\pgfqpoint{0.862566in}{1.825406in}}%
\pgfpathlineto{\pgfqpoint{0.873246in}{1.821840in}}%
\pgfpathlineto{\pgfqpoint{0.877958in}{1.823506in}}%
\pgfpathlineto{\pgfqpoint{0.880330in}{1.828801in}}%
\pgfpathlineto{\pgfqpoint{0.886583in}{1.827387in}}%
\pgfpathlineto{\pgfqpoint{0.889217in}{1.841046in}}%
\pgfpathlineto{\pgfqpoint{0.896692in}{1.839394in}}%
\pgfpathlineto{\pgfqpoint{0.901644in}{1.862062in}}%
\pgfpathlineto{\pgfqpoint{0.901194in}{1.867977in}}%
\pgfpathlineto{\pgfqpoint{0.910046in}{1.867809in}}%
\pgfpathlineto{\pgfqpoint{0.913309in}{1.873552in}}%
\pgfpathlineto{\pgfqpoint{0.912328in}{1.878097in}}%
\pgfpathlineto{\pgfqpoint{0.913025in}{1.888772in}}%
\pgfpathlineto{\pgfqpoint{0.941334in}{1.882540in}}%
\pgfpathlineto{\pgfqpoint{0.941225in}{1.874500in}}%
\pgfpathlineto{\pgfqpoint{0.943633in}{1.869658in}}%
\pgfpathlineto{\pgfqpoint{0.949121in}{1.871112in}}%
\pgfpathlineto{\pgfqpoint{0.953218in}{1.860951in}}%
\pgfpathlineto{\pgfqpoint{0.949618in}{1.854808in}}%
\pgfpathlineto{\pgfqpoint{0.961059in}{1.847574in}}%
\pgfpathlineto{\pgfqpoint{0.959285in}{1.841945in}}%
\pgfpathlineto{\pgfqpoint{0.964152in}{1.834807in}}%
\pgfpathlineto{\pgfqpoint{0.961768in}{1.833418in}}%
\pgfpathlineto{\pgfqpoint{0.967054in}{1.825877in}}%
\pgfpathlineto{\pgfqpoint{0.966264in}{1.820814in}}%
\pgfpathlineto{\pgfqpoint{0.975124in}{1.816423in}}%
\pgfpathlineto{\pgfqpoint{0.978555in}{1.808720in}}%
\pgfpathlineto{\pgfqpoint{0.974188in}{1.804466in}}%
\pgfpathlineto{\pgfqpoint{0.969863in}{1.800673in}}%
\pgfpathlineto{\pgfqpoint{0.967284in}{1.791973in}}%
\pgfpathlineto{\pgfqpoint{0.963166in}{1.790823in}}%
\pgfpathlineto{\pgfqpoint{0.962444in}{1.782513in}}%
\pgfpathlineto{\pgfqpoint{0.948336in}{1.785640in}}%
\pgfpathlineto{\pgfqpoint{0.927953in}{1.789677in}}%
\pgfpathlineto{\pgfqpoint{0.925310in}{1.782011in}}%
\pgfpathlineto{\pgfqpoint{0.927695in}{1.775797in}}%
\pgfpathlineto{\pgfqpoint{0.926220in}{1.769071in}}%
\pgfpathlineto{\pgfqpoint{0.926970in}{1.761628in}}%
\pgfpathlineto{\pgfqpoint{0.921428in}{1.759848in}}%
\pgfpathlineto{\pgfqpoint{0.910196in}{1.762234in}}%
\pgfpathlineto{\pgfqpoint{0.906982in}{1.761729in}}%
\pgfpathlineto{\pgfqpoint{0.900754in}{1.769170in}}%
\pgfpathlineto{\pgfqpoint{0.890065in}{1.775890in}}%
\pgfpathlineto{\pgfqpoint{0.883118in}{1.777032in}}%
\pgfpathlineto{\pgfqpoint{0.878165in}{1.781128in}}%
\pgfpathlineto{\pgfqpoint{0.878913in}{1.786617in}}%
\pgfpathlineto{\pgfqpoint{0.866931in}{1.795554in}}%
\pgfpathlineto{\pgfqpoint{0.859543in}{1.795823in}}%
\pgfpathlineto{\pgfqpoint{0.851627in}{1.798562in}}%
\pgfpathclose%
\pgfusepath{fill}%
\end{pgfscope}%
\begin{pgfscope}%
\pgfpathrectangle{\pgfqpoint{0.100000in}{0.100000in}}{\pgfqpoint{3.007045in}{1.925000in}}%
\pgfusepath{clip}%
\pgfsetbuttcap%
\pgfsetmiterjoin%
\definecolor{currentfill}{rgb}{0.031373,0.285675,0.564291}%
\pgfsetfillcolor{currentfill}%
\pgfsetlinewidth{0.000000pt}%
\definecolor{currentstroke}{rgb}{0.000000,0.000000,0.000000}%
\pgfsetstrokecolor{currentstroke}%
\pgfsetstrokeopacity{0.000000}%
\pgfsetdash{}{0pt}%
\pgfpathmoveto{\pgfqpoint{1.336243in}{0.559459in}}%
\pgfpathlineto{\pgfqpoint{1.349461in}{0.546748in}}%
\pgfpathlineto{\pgfqpoint{1.344620in}{0.539783in}}%
\pgfpathlineto{\pgfqpoint{1.328189in}{0.539358in}}%
\pgfpathlineto{\pgfqpoint{1.323294in}{0.529791in}}%
\pgfpathlineto{\pgfqpoint{1.319302in}{0.525390in}}%
\pgfpathlineto{\pgfqpoint{1.316511in}{0.514705in}}%
\pgfpathlineto{\pgfqpoint{1.305003in}{0.500040in}}%
\pgfpathlineto{\pgfqpoint{1.299683in}{0.496151in}}%
\pgfpathlineto{\pgfqpoint{1.298360in}{0.491273in}}%
\pgfpathlineto{\pgfqpoint{1.291128in}{0.491622in}}%
\pgfpathlineto{\pgfqpoint{1.278806in}{0.498569in}}%
\pgfpathlineto{\pgfqpoint{1.273825in}{0.505053in}}%
\pgfpathlineto{\pgfqpoint{1.264636in}{0.507489in}}%
\pgfpathlineto{\pgfqpoint{1.261257in}{0.513470in}}%
\pgfpathlineto{\pgfqpoint{1.250316in}{0.516616in}}%
\pgfpathlineto{\pgfqpoint{1.241232in}{0.523169in}}%
\pgfpathlineto{\pgfqpoint{1.236587in}{0.531465in}}%
\pgfpathlineto{\pgfqpoint{1.231030in}{0.533550in}}%
\pgfpathlineto{\pgfqpoint{1.221419in}{0.542714in}}%
\pgfpathlineto{\pgfqpoint{1.219216in}{0.552154in}}%
\pgfpathlineto{\pgfqpoint{1.214219in}{0.561569in}}%
\pgfpathlineto{\pgfqpoint{1.213608in}{0.571913in}}%
\pgfpathlineto{\pgfqpoint{1.214655in}{0.583241in}}%
\pgfpathlineto{\pgfqpoint{1.207041in}{0.594453in}}%
\pgfpathlineto{\pgfqpoint{1.207348in}{0.601413in}}%
\pgfpathlineto{\pgfqpoint{1.204717in}{0.609116in}}%
\pgfpathlineto{\pgfqpoint{1.201869in}{0.611048in}}%
\pgfpathlineto{\pgfqpoint{1.205698in}{0.612940in}}%
\pgfpathlineto{\pgfqpoint{1.255027in}{0.637292in}}%
\pgfpathlineto{\pgfqpoint{1.282209in}{0.612003in}}%
\pgfpathlineto{\pgfqpoint{1.336243in}{0.559459in}}%
\pgfpathclose%
\pgfusepath{fill}%
\end{pgfscope}%
\begin{pgfscope}%
\pgfpathrectangle{\pgfqpoint{0.100000in}{0.100000in}}{\pgfqpoint{3.007045in}{1.925000in}}%
\pgfusepath{clip}%
\pgfsetbuttcap%
\pgfsetmiterjoin%
\definecolor{currentfill}{rgb}{0.336378,0.625513,0.806736}%
\pgfsetfillcolor{currentfill}%
\pgfsetlinewidth{0.000000pt}%
\definecolor{currentstroke}{rgb}{0.000000,0.000000,0.000000}%
\pgfsetstrokecolor{currentstroke}%
\pgfsetstrokeopacity{0.000000}%
\pgfsetdash{}{0pt}%
\pgfpathmoveto{\pgfqpoint{2.279304in}{0.827039in}}%
\pgfpathlineto{\pgfqpoint{2.271634in}{0.825054in}}%
\pgfpathlineto{\pgfqpoint{2.262319in}{0.816941in}}%
\pgfpathlineto{\pgfqpoint{2.257720in}{0.820289in}}%
\pgfpathlineto{\pgfqpoint{2.254073in}{0.825883in}}%
\pgfpathlineto{\pgfqpoint{2.249130in}{0.825060in}}%
\pgfpathlineto{\pgfqpoint{2.246429in}{0.828970in}}%
\pgfpathlineto{\pgfqpoint{2.247672in}{0.832584in}}%
\pgfpathlineto{\pgfqpoint{2.236928in}{0.845086in}}%
\pgfpathlineto{\pgfqpoint{2.231240in}{0.844683in}}%
\pgfpathlineto{\pgfqpoint{2.232658in}{0.857963in}}%
\pgfpathlineto{\pgfqpoint{2.238722in}{0.856984in}}%
\pgfpathlineto{\pgfqpoint{2.244466in}{0.861498in}}%
\pgfpathlineto{\pgfqpoint{2.241299in}{0.867393in}}%
\pgfpathlineto{\pgfqpoint{2.241813in}{0.880528in}}%
\pgfpathlineto{\pgfqpoint{2.240204in}{0.883283in}}%
\pgfpathlineto{\pgfqpoint{2.241623in}{0.891209in}}%
\pgfpathlineto{\pgfqpoint{2.265651in}{0.893385in}}%
\pgfpathlineto{\pgfqpoint{2.279674in}{0.894582in}}%
\pgfpathlineto{\pgfqpoint{2.286587in}{0.869005in}}%
\pgfpathlineto{\pgfqpoint{2.287764in}{0.864843in}}%
\pgfpathlineto{\pgfqpoint{2.284621in}{0.861875in}}%
\pgfpathlineto{\pgfqpoint{2.282179in}{0.853795in}}%
\pgfpathlineto{\pgfqpoint{2.275437in}{0.845959in}}%
\pgfpathlineto{\pgfqpoint{2.271751in}{0.844592in}}%
\pgfpathlineto{\pgfqpoint{2.279304in}{0.827039in}}%
\pgfpathclose%
\pgfusepath{fill}%
\end{pgfscope}%
\begin{pgfscope}%
\pgfpathrectangle{\pgfqpoint{0.100000in}{0.100000in}}{\pgfqpoint{3.007045in}{1.925000in}}%
\pgfusepath{clip}%
\pgfsetbuttcap%
\pgfsetmiterjoin%
\definecolor{currentfill}{rgb}{0.541961,0.749527,0.865606}%
\pgfsetfillcolor{currentfill}%
\pgfsetlinewidth{0.000000pt}%
\definecolor{currentstroke}{rgb}{0.000000,0.000000,0.000000}%
\pgfsetstrokecolor{currentstroke}%
\pgfsetstrokeopacity{0.000000}%
\pgfsetdash{}{0pt}%
\pgfpathmoveto{\pgfqpoint{0.521615in}{1.353758in}}%
\pgfpathlineto{\pgfqpoint{0.517066in}{1.358013in}}%
\pgfpathlineto{\pgfqpoint{0.520278in}{1.368998in}}%
\pgfpathlineto{\pgfqpoint{0.517118in}{1.374961in}}%
\pgfpathlineto{\pgfqpoint{0.518020in}{1.379672in}}%
\pgfpathlineto{\pgfqpoint{0.512166in}{1.383346in}}%
\pgfpathlineto{\pgfqpoint{0.503882in}{1.397397in}}%
\pgfpathlineto{\pgfqpoint{0.495392in}{1.401958in}}%
\pgfpathlineto{\pgfqpoint{0.488838in}{1.398768in}}%
\pgfpathlineto{\pgfqpoint{0.482087in}{1.399330in}}%
\pgfpathlineto{\pgfqpoint{0.479961in}{1.402391in}}%
\pgfpathlineto{\pgfqpoint{0.483507in}{1.414404in}}%
\pgfpathlineto{\pgfqpoint{0.470680in}{1.418127in}}%
\pgfpathlineto{\pgfqpoint{0.479723in}{1.447096in}}%
\pgfpathlineto{\pgfqpoint{0.484393in}{1.464821in}}%
\pgfpathlineto{\pgfqpoint{0.429023in}{1.481634in}}%
\pgfpathlineto{\pgfqpoint{0.431912in}{1.491802in}}%
\pgfpathlineto{\pgfqpoint{0.428385in}{1.494011in}}%
\pgfpathlineto{\pgfqpoint{0.414605in}{1.487528in}}%
\pgfpathlineto{\pgfqpoint{0.408114in}{1.487902in}}%
\pgfpathlineto{\pgfqpoint{0.405051in}{1.484139in}}%
\pgfpathlineto{\pgfqpoint{0.407088in}{1.477930in}}%
\pgfpathlineto{\pgfqpoint{0.399966in}{1.478464in}}%
\pgfpathlineto{\pgfqpoint{0.397986in}{1.484025in}}%
\pgfpathlineto{\pgfqpoint{0.391596in}{1.486250in}}%
\pgfpathlineto{\pgfqpoint{0.391169in}{1.489496in}}%
\pgfpathlineto{\pgfqpoint{0.386307in}{1.493716in}}%
\pgfpathlineto{\pgfqpoint{0.385689in}{1.509253in}}%
\pgfpathlineto{\pgfqpoint{0.378145in}{1.511673in}}%
\pgfpathlineto{\pgfqpoint{0.381028in}{1.514238in}}%
\pgfpathlineto{\pgfqpoint{0.381897in}{1.521054in}}%
\pgfpathlineto{\pgfqpoint{0.380582in}{1.527032in}}%
\pgfpathlineto{\pgfqpoint{0.384988in}{1.532462in}}%
\pgfpathlineto{\pgfqpoint{0.385318in}{1.540362in}}%
\pgfpathlineto{\pgfqpoint{0.393613in}{1.542963in}}%
\pgfpathlineto{\pgfqpoint{0.397486in}{1.548285in}}%
\pgfpathlineto{\pgfqpoint{0.455007in}{1.530523in}}%
\pgfpathlineto{\pgfqpoint{0.474662in}{1.592743in}}%
\pgfpathlineto{\pgfqpoint{0.476029in}{1.597225in}}%
\pgfpathlineto{\pgfqpoint{0.485002in}{1.594976in}}%
\pgfpathlineto{\pgfqpoint{0.487657in}{1.599535in}}%
\pgfpathlineto{\pgfqpoint{0.493766in}{1.605432in}}%
\pgfpathlineto{\pgfqpoint{0.493741in}{1.612121in}}%
\pgfpathlineto{\pgfqpoint{0.490196in}{1.618627in}}%
\pgfpathlineto{\pgfqpoint{0.492514in}{1.625981in}}%
\pgfpathlineto{\pgfqpoint{0.499509in}{1.627848in}}%
\pgfpathlineto{\pgfqpoint{0.530203in}{1.618670in}}%
\pgfpathlineto{\pgfqpoint{0.524578in}{1.602426in}}%
\pgfpathlineto{\pgfqpoint{0.513001in}{1.563845in}}%
\pgfpathlineto{\pgfqpoint{0.534635in}{1.557261in}}%
\pgfpathlineto{\pgfqpoint{0.520911in}{1.509711in}}%
\pgfpathlineto{\pgfqpoint{0.562393in}{1.497838in}}%
\pgfpathlineto{\pgfqpoint{0.547844in}{1.446415in}}%
\pgfpathlineto{\pgfqpoint{0.535789in}{1.402977in}}%
\pgfpathlineto{\pgfqpoint{0.521615in}{1.353758in}}%
\pgfpathclose%
\pgfusepath{fill}%
\end{pgfscope}%
\begin{pgfscope}%
\pgfpathrectangle{\pgfqpoint{0.100000in}{0.100000in}}{\pgfqpoint{3.007045in}{1.925000in}}%
\pgfusepath{clip}%
\pgfsetbuttcap%
\pgfsetmiterjoin%
\definecolor{currentfill}{rgb}{0.391865,0.663406,0.828389}%
\pgfsetfillcolor{currentfill}%
\pgfsetlinewidth{0.000000pt}%
\definecolor{currentstroke}{rgb}{0.000000,0.000000,0.000000}%
\pgfsetstrokecolor{currentstroke}%
\pgfsetstrokeopacity{0.000000}%
\pgfsetdash{}{0pt}%
\pgfpathmoveto{\pgfqpoint{2.213486in}{1.315239in}}%
\pgfpathlineto{\pgfqpoint{2.215428in}{1.298031in}}%
\pgfpathlineto{\pgfqpoint{2.195043in}{1.295965in}}%
\pgfpathlineto{\pgfqpoint{2.172150in}{1.293918in}}%
\pgfpathlineto{\pgfqpoint{2.168897in}{1.329206in}}%
\pgfpathlineto{\pgfqpoint{2.180297in}{1.338873in}}%
\pgfpathlineto{\pgfqpoint{2.184660in}{1.346048in}}%
\pgfpathlineto{\pgfqpoint{2.188143in}{1.357873in}}%
\pgfpathlineto{\pgfqpoint{2.193342in}{1.366778in}}%
\pgfpathlineto{\pgfqpoint{2.200252in}{1.367456in}}%
\pgfpathlineto{\pgfqpoint{2.201376in}{1.356174in}}%
\pgfpathlineto{\pgfqpoint{2.223806in}{1.358262in}}%
\pgfpathlineto{\pgfqpoint{2.225748in}{1.340539in}}%
\pgfpathlineto{\pgfqpoint{2.224488in}{1.337759in}}%
\pgfpathlineto{\pgfqpoint{2.211201in}{1.336530in}}%
\pgfpathlineto{\pgfqpoint{2.213486in}{1.315239in}}%
\pgfpathclose%
\pgfusepath{fill}%
\end{pgfscope}%
\begin{pgfscope}%
\pgfpathrectangle{\pgfqpoint{0.100000in}{0.100000in}}{\pgfqpoint{3.007045in}{1.925000in}}%
\pgfusepath{clip}%
\pgfsetbuttcap%
\pgfsetmiterjoin%
\definecolor{currentfill}{rgb}{0.331334,0.622068,0.804767}%
\pgfsetfillcolor{currentfill}%
\pgfsetlinewidth{0.000000pt}%
\definecolor{currentstroke}{rgb}{0.000000,0.000000,0.000000}%
\pgfsetstrokecolor{currentstroke}%
\pgfsetstrokeopacity{0.000000}%
\pgfsetdash{}{0pt}%
\pgfpathmoveto{\pgfqpoint{2.013078in}{0.656435in}}%
\pgfpathlineto{\pgfqpoint{2.013470in}{0.675004in}}%
\pgfpathlineto{\pgfqpoint{2.011898in}{0.691368in}}%
\pgfpathlineto{\pgfqpoint{2.012854in}{0.699525in}}%
\pgfpathlineto{\pgfqpoint{2.016613in}{0.701553in}}%
\pgfpathlineto{\pgfqpoint{2.018058in}{0.708669in}}%
\pgfpathlineto{\pgfqpoint{2.020779in}{0.710545in}}%
\pgfpathlineto{\pgfqpoint{2.027190in}{0.716066in}}%
\pgfpathlineto{\pgfqpoint{2.040123in}{0.721921in}}%
\pgfpathlineto{\pgfqpoint{2.048959in}{0.728755in}}%
\pgfpathlineto{\pgfqpoint{2.052237in}{0.736873in}}%
\pgfpathlineto{\pgfqpoint{2.065317in}{0.737960in}}%
\pgfpathlineto{\pgfqpoint{2.065136in}{0.740946in}}%
\pgfpathlineto{\pgfqpoint{2.087936in}{0.742436in}}%
\pgfpathlineto{\pgfqpoint{2.090671in}{0.695575in}}%
\pgfpathlineto{\pgfqpoint{2.067842in}{0.694087in}}%
\pgfpathlineto{\pgfqpoint{2.068533in}{0.682516in}}%
\pgfpathlineto{\pgfqpoint{2.072910in}{0.680552in}}%
\pgfpathlineto{\pgfqpoint{2.073893in}{0.665116in}}%
\pgfpathlineto{\pgfqpoint{2.055851in}{0.662822in}}%
\pgfpathlineto{\pgfqpoint{2.047327in}{0.661747in}}%
\pgfpathlineto{\pgfqpoint{2.038852in}{0.657801in}}%
\pgfpathlineto{\pgfqpoint{2.013078in}{0.656435in}}%
\pgfpathclose%
\pgfusepath{fill}%
\end{pgfscope}%
\begin{pgfscope}%
\pgfpathrectangle{\pgfqpoint{0.100000in}{0.100000in}}{\pgfqpoint{3.007045in}{1.925000in}}%
\pgfusepath{clip}%
\pgfsetbuttcap%
\pgfsetmiterjoin%
\definecolor{currentfill}{rgb}{0.316201,0.611734,0.798862}%
\pgfsetfillcolor{currentfill}%
\pgfsetlinewidth{0.000000pt}%
\definecolor{currentstroke}{rgb}{0.000000,0.000000,0.000000}%
\pgfsetstrokecolor{currentstroke}%
\pgfsetstrokeopacity{0.000000}%
\pgfsetdash{}{0pt}%
\pgfpathmoveto{\pgfqpoint{1.886783in}{0.652756in}}%
\pgfpathlineto{\pgfqpoint{1.890709in}{0.647179in}}%
\pgfpathlineto{\pgfqpoint{1.896682in}{0.645552in}}%
\pgfpathlineto{\pgfqpoint{1.901476in}{0.640393in}}%
\pgfpathlineto{\pgfqpoint{1.890720in}{0.630209in}}%
\pgfpathlineto{\pgfqpoint{1.886870in}{0.628687in}}%
\pgfpathlineto{\pgfqpoint{1.883960in}{0.629796in}}%
\pgfpathlineto{\pgfqpoint{1.872399in}{0.629573in}}%
\pgfpathlineto{\pgfqpoint{1.872143in}{0.641119in}}%
\pgfpathlineto{\pgfqpoint{1.866438in}{0.640996in}}%
\pgfpathlineto{\pgfqpoint{1.866139in}{0.652691in}}%
\pgfpathlineto{\pgfqpoint{1.860335in}{0.652602in}}%
\pgfpathlineto{\pgfqpoint{1.860218in}{0.661283in}}%
\pgfpathlineto{\pgfqpoint{1.835531in}{0.660856in}}%
\pgfpathlineto{\pgfqpoint{1.827733in}{0.669764in}}%
\pgfpathlineto{\pgfqpoint{1.826203in}{0.670688in}}%
\pgfpathlineto{\pgfqpoint{1.825570in}{0.729490in}}%
\pgfpathlineto{\pgfqpoint{1.825469in}{0.738647in}}%
\pgfpathlineto{\pgfqpoint{1.838686in}{0.738812in}}%
\pgfpathlineto{\pgfqpoint{1.884006in}{0.739484in}}%
\pgfpathlineto{\pgfqpoint{1.898651in}{0.739626in}}%
\pgfpathlineto{\pgfqpoint{1.899049in}{0.722750in}}%
\pgfpathlineto{\pgfqpoint{1.893306in}{0.722562in}}%
\pgfpathlineto{\pgfqpoint{1.890590in}{0.711973in}}%
\pgfpathlineto{\pgfqpoint{1.890952in}{0.702305in}}%
\pgfpathlineto{\pgfqpoint{1.896662in}{0.702398in}}%
\pgfpathlineto{\pgfqpoint{1.897220in}{0.687919in}}%
\pgfpathlineto{\pgfqpoint{1.895056in}{0.682024in}}%
\pgfpathlineto{\pgfqpoint{1.888095in}{0.681930in}}%
\pgfpathlineto{\pgfqpoint{1.891446in}{0.668175in}}%
\pgfpathlineto{\pgfqpoint{1.888359in}{0.662427in}}%
\pgfpathlineto{\pgfqpoint{1.886783in}{0.652756in}}%
\pgfpathclose%
\pgfusepath{fill}%
\end{pgfscope}%
\begin{pgfscope}%
\pgfpathrectangle{\pgfqpoint{0.100000in}{0.100000in}}{\pgfqpoint{3.007045in}{1.925000in}}%
\pgfusepath{clip}%
\pgfsetbuttcap%
\pgfsetmiterjoin%
\definecolor{currentfill}{rgb}{0.548235,0.752972,0.866959}%
\pgfsetfillcolor{currentfill}%
\pgfsetlinewidth{0.000000pt}%
\definecolor{currentstroke}{rgb}{0.000000,0.000000,0.000000}%
\pgfsetstrokecolor{currentstroke}%
\pgfsetstrokeopacity{0.000000}%
\pgfsetdash{}{0pt}%
\pgfpathmoveto{\pgfqpoint{2.399943in}{0.955887in}}%
\pgfpathlineto{\pgfqpoint{2.388077in}{0.945524in}}%
\pgfpathlineto{\pgfqpoint{2.379259in}{0.944820in}}%
\pgfpathlineto{\pgfqpoint{2.378270in}{0.952775in}}%
\pgfpathlineto{\pgfqpoint{2.374820in}{0.956140in}}%
\pgfpathlineto{\pgfqpoint{2.369682in}{0.964768in}}%
\pgfpathlineto{\pgfqpoint{2.376395in}{0.971244in}}%
\pgfpathlineto{\pgfqpoint{2.370660in}{0.984801in}}%
\pgfpathlineto{\pgfqpoint{2.372662in}{0.995532in}}%
\pgfpathlineto{\pgfqpoint{2.380497in}{0.997923in}}%
\pgfpathlineto{\pgfqpoint{2.388640in}{1.002575in}}%
\pgfpathlineto{\pgfqpoint{2.392615in}{1.001420in}}%
\pgfpathlineto{\pgfqpoint{2.394844in}{0.994687in}}%
\pgfpathlineto{\pgfqpoint{2.399497in}{0.998207in}}%
\pgfpathlineto{\pgfqpoint{2.404364in}{0.992821in}}%
\pgfpathlineto{\pgfqpoint{2.400322in}{0.986669in}}%
\pgfpathlineto{\pgfqpoint{2.406942in}{0.983024in}}%
\pgfpathlineto{\pgfqpoint{2.416656in}{0.974681in}}%
\pgfpathlineto{\pgfqpoint{2.414244in}{0.964014in}}%
\pgfpathlineto{\pgfqpoint{2.404400in}{0.961065in}}%
\pgfpathlineto{\pgfqpoint{2.399943in}{0.955887in}}%
\pgfpathclose%
\pgfusepath{fill}%
\end{pgfscope}%
\begin{pgfscope}%
\pgfpathrectangle{\pgfqpoint{0.100000in}{0.100000in}}{\pgfqpoint{3.007045in}{1.925000in}}%
\pgfusepath{clip}%
\pgfsetbuttcap%
\pgfsetmiterjoin%
\definecolor{currentfill}{rgb}{0.296025,0.597955,0.790988}%
\pgfsetfillcolor{currentfill}%
\pgfsetlinewidth{0.000000pt}%
\definecolor{currentstroke}{rgb}{0.000000,0.000000,0.000000}%
\pgfsetstrokecolor{currentstroke}%
\pgfsetstrokeopacity{0.000000}%
\pgfsetdash{}{0pt}%
\pgfpathmoveto{\pgfqpoint{1.092509in}{1.504079in}}%
\pgfpathlineto{\pgfqpoint{1.102940in}{1.502630in}}%
\pgfpathlineto{\pgfqpoint{1.103807in}{1.508400in}}%
\pgfpathlineto{\pgfqpoint{1.112286in}{1.507103in}}%
\pgfpathlineto{\pgfqpoint{1.113003in}{1.512751in}}%
\pgfpathlineto{\pgfqpoint{1.118845in}{1.511867in}}%
\pgfpathlineto{\pgfqpoint{1.119707in}{1.517659in}}%
\pgfpathlineto{\pgfqpoint{1.131047in}{1.515926in}}%
\pgfpathlineto{\pgfqpoint{1.131931in}{1.521748in}}%
\pgfpathlineto{\pgfqpoint{1.197829in}{1.512182in}}%
\pgfpathlineto{\pgfqpoint{1.193681in}{1.468364in}}%
\pgfpathlineto{\pgfqpoint{1.173551in}{1.471182in}}%
\pgfpathlineto{\pgfqpoint{1.170426in}{1.469676in}}%
\pgfpathlineto{\pgfqpoint{1.143778in}{1.473368in}}%
\pgfpathlineto{\pgfqpoint{1.135060in}{1.473922in}}%
\pgfpathlineto{\pgfqpoint{1.118840in}{1.481195in}}%
\pgfpathlineto{\pgfqpoint{1.119290in}{1.484034in}}%
\pgfpathlineto{\pgfqpoint{1.108398in}{1.488604in}}%
\pgfpathlineto{\pgfqpoint{1.097179in}{1.490393in}}%
\pgfpathlineto{\pgfqpoint{1.098004in}{1.495582in}}%
\pgfpathlineto{\pgfqpoint{1.092509in}{1.504079in}}%
\pgfpathclose%
\pgfusepath{fill}%
\end{pgfscope}%
\begin{pgfscope}%
\pgfpathrectangle{\pgfqpoint{0.100000in}{0.100000in}}{\pgfqpoint{3.007045in}{1.925000in}}%
\pgfusepath{clip}%
\pgfsetbuttcap%
\pgfsetmiterjoin%
\definecolor{currentfill}{rgb}{0.396909,0.666851,0.830358}%
\pgfsetfillcolor{currentfill}%
\pgfsetlinewidth{0.000000pt}%
\definecolor{currentstroke}{rgb}{0.000000,0.000000,0.000000}%
\pgfsetstrokecolor{currentstroke}%
\pgfsetstrokeopacity{0.000000}%
\pgfsetdash{}{0pt}%
\pgfpathmoveto{\pgfqpoint{2.119150in}{1.400006in}}%
\pgfpathlineto{\pgfqpoint{2.073275in}{1.396646in}}%
\pgfpathlineto{\pgfqpoint{2.061841in}{1.396171in}}%
\pgfpathlineto{\pgfqpoint{2.058743in}{1.447959in}}%
\pgfpathlineto{\pgfqpoint{2.087655in}{1.449833in}}%
\pgfpathlineto{\pgfqpoint{2.088107in}{1.444073in}}%
\pgfpathlineto{\pgfqpoint{2.099536in}{1.444895in}}%
\pgfpathlineto{\pgfqpoint{2.117083in}{1.446231in}}%
\pgfpathlineto{\pgfqpoint{2.116939in}{1.440991in}}%
\pgfpathlineto{\pgfqpoint{2.114427in}{1.436064in}}%
\pgfpathlineto{\pgfqpoint{2.113005in}{1.425983in}}%
\pgfpathlineto{\pgfqpoint{2.119150in}{1.400006in}}%
\pgfpathclose%
\pgfusepath{fill}%
\end{pgfscope}%
\begin{pgfscope}%
\pgfpathrectangle{\pgfqpoint{0.100000in}{0.100000in}}{\pgfqpoint{3.007045in}{1.925000in}}%
\pgfusepath{clip}%
\pgfsetbuttcap%
\pgfsetmiterjoin%
\definecolor{currentfill}{rgb}{0.504314,0.728858,0.857486}%
\pgfsetfillcolor{currentfill}%
\pgfsetlinewidth{0.000000pt}%
\definecolor{currentstroke}{rgb}{0.000000,0.000000,0.000000}%
\pgfsetstrokecolor{currentstroke}%
\pgfsetstrokeopacity{0.000000}%
\pgfsetdash{}{0pt}%
\pgfpathmoveto{\pgfqpoint{0.618157in}{0.899792in}}%
\pgfpathlineto{\pgfqpoint{0.613032in}{0.877119in}}%
\pgfpathlineto{\pgfqpoint{0.611833in}{0.877391in}}%
\pgfpathlineto{\pgfqpoint{0.604721in}{0.848024in}}%
\pgfpathlineto{\pgfqpoint{0.548111in}{0.856043in}}%
\pgfpathlineto{\pgfqpoint{0.548819in}{0.861290in}}%
\pgfpathlineto{\pgfqpoint{0.543625in}{0.866151in}}%
\pgfpathlineto{\pgfqpoint{0.546518in}{0.879093in}}%
\pgfpathlineto{\pgfqpoint{0.547290in}{0.888052in}}%
\pgfpathlineto{\pgfqpoint{0.545659in}{0.899091in}}%
\pgfpathlineto{\pgfqpoint{0.540654in}{0.912060in}}%
\pgfpathlineto{\pgfqpoint{0.536458in}{0.917204in}}%
\pgfpathlineto{\pgfqpoint{0.538502in}{0.921211in}}%
\pgfpathlineto{\pgfqpoint{0.543010in}{0.923557in}}%
\pgfpathlineto{\pgfqpoint{0.550214in}{0.920766in}}%
\pgfpathlineto{\pgfqpoint{0.556193in}{0.915257in}}%
\pgfpathlineto{\pgfqpoint{0.567402in}{0.912130in}}%
\pgfpathlineto{\pgfqpoint{0.618157in}{0.899792in}}%
\pgfpathclose%
\pgfusepath{fill}%
\end{pgfscope}%
\begin{pgfscope}%
\pgfpathrectangle{\pgfqpoint{0.100000in}{0.100000in}}{\pgfqpoint{3.007045in}{1.925000in}}%
\pgfusepath{clip}%
\pgfsetbuttcap%
\pgfsetmiterjoin%
\definecolor{currentfill}{rgb}{0.745713,0.845752,0.926490}%
\pgfsetfillcolor{currentfill}%
\pgfsetlinewidth{0.000000pt}%
\definecolor{currentstroke}{rgb}{0.000000,0.000000,0.000000}%
\pgfsetstrokecolor{currentstroke}%
\pgfsetstrokeopacity{0.000000}%
\pgfsetdash{}{0pt}%
\pgfpathmoveto{\pgfqpoint{2.000756in}{1.600279in}}%
\pgfpathlineto{\pgfqpoint{1.988873in}{1.599607in}}%
\pgfpathlineto{\pgfqpoint{1.988231in}{1.611105in}}%
\pgfpathlineto{\pgfqpoint{1.982576in}{1.610763in}}%
\pgfpathlineto{\pgfqpoint{1.982301in}{1.616494in}}%
\pgfpathlineto{\pgfqpoint{1.976511in}{1.616232in}}%
\pgfpathlineto{\pgfqpoint{1.975507in}{1.638992in}}%
\pgfpathlineto{\pgfqpoint{1.980609in}{1.637782in}}%
\pgfpathlineto{\pgfqpoint{1.984352in}{1.640321in}}%
\pgfpathlineto{\pgfqpoint{1.998858in}{1.646114in}}%
\pgfpathlineto{\pgfqpoint{2.009289in}{1.656368in}}%
\pgfpathlineto{\pgfqpoint{2.026300in}{1.659095in}}%
\pgfpathlineto{\pgfqpoint{2.033497in}{1.663365in}}%
\pgfpathlineto{\pgfqpoint{2.037806in}{1.669257in}}%
\pgfpathlineto{\pgfqpoint{2.044910in}{1.670501in}}%
\pgfpathlineto{\pgfqpoint{2.047218in}{1.673072in}}%
\pgfpathlineto{\pgfqpoint{2.048453in}{1.655274in}}%
\pgfpathlineto{\pgfqpoint{2.051916in}{1.649810in}}%
\pgfpathlineto{\pgfqpoint{2.046234in}{1.649481in}}%
\pgfpathlineto{\pgfqpoint{2.047301in}{1.632207in}}%
\pgfpathlineto{\pgfqpoint{2.048726in}{1.610846in}}%
\pgfpathlineto{\pgfqpoint{2.043917in}{1.613272in}}%
\pgfpathlineto{\pgfqpoint{2.004866in}{1.621636in}}%
\pgfpathlineto{\pgfqpoint{2.006057in}{1.600572in}}%
\pgfpathlineto{\pgfqpoint{2.000756in}{1.600279in}}%
\pgfpathclose%
\pgfusepath{fill}%
\end{pgfscope}%
\begin{pgfscope}%
\pgfpathrectangle{\pgfqpoint{0.100000in}{0.100000in}}{\pgfqpoint{3.007045in}{1.925000in}}%
\pgfusepath{clip}%
\pgfsetbuttcap%
\pgfsetmiterjoin%
\definecolor{currentfill}{rgb}{0.401953,0.670296,0.832326}%
\pgfsetfillcolor{currentfill}%
\pgfsetlinewidth{0.000000pt}%
\definecolor{currentstroke}{rgb}{0.000000,0.000000,0.000000}%
\pgfsetstrokecolor{currentstroke}%
\pgfsetstrokeopacity{0.000000}%
\pgfsetdash{}{0pt}%
\pgfpathmoveto{\pgfqpoint{1.330895in}{1.191503in}}%
\pgfpathlineto{\pgfqpoint{1.358836in}{1.188865in}}%
\pgfpathlineto{\pgfqpoint{1.355807in}{1.160158in}}%
\pgfpathlineto{\pgfqpoint{1.353166in}{1.131470in}}%
\pgfpathlineto{\pgfqpoint{1.351745in}{1.120688in}}%
\pgfpathlineto{\pgfqpoint{1.334664in}{1.121680in}}%
\pgfpathlineto{\pgfqpoint{1.306529in}{1.124802in}}%
\pgfpathlineto{\pgfqpoint{1.308723in}{1.147516in}}%
\pgfpathlineto{\pgfqpoint{1.325845in}{1.145704in}}%
\pgfpathlineto{\pgfqpoint{1.330895in}{1.191503in}}%
\pgfpathclose%
\pgfusepath{fill}%
\end{pgfscope}%
\begin{pgfscope}%
\pgfpathrectangle{\pgfqpoint{0.100000in}{0.100000in}}{\pgfqpoint{3.007045in}{1.925000in}}%
\pgfusepath{clip}%
\pgfsetbuttcap%
\pgfsetmiterjoin%
\definecolor{currentfill}{rgb}{0.316201,0.611734,0.798862}%
\pgfsetfillcolor{currentfill}%
\pgfsetlinewidth{0.000000pt}%
\definecolor{currentstroke}{rgb}{0.000000,0.000000,0.000000}%
\pgfsetstrokecolor{currentstroke}%
\pgfsetstrokeopacity{0.000000}%
\pgfsetdash{}{0pt}%
\pgfpathmoveto{\pgfqpoint{1.768296in}{1.069654in}}%
\pgfpathlineto{\pgfqpoint{1.767740in}{1.049552in}}%
\pgfpathlineto{\pgfqpoint{1.722263in}{1.049928in}}%
\pgfpathlineto{\pgfqpoint{1.722556in}{1.070090in}}%
\pgfpathlineto{\pgfqpoint{1.745387in}{1.069764in}}%
\pgfpathlineto{\pgfqpoint{1.746105in}{1.095854in}}%
\pgfpathlineto{\pgfqpoint{1.746586in}{1.115888in}}%
\pgfpathlineto{\pgfqpoint{1.769426in}{1.115817in}}%
\pgfpathlineto{\pgfqpoint{1.768947in}{1.112944in}}%
\pgfpathlineto{\pgfqpoint{1.768960in}{1.084214in}}%
\pgfpathlineto{\pgfqpoint{1.768296in}{1.069654in}}%
\pgfpathclose%
\pgfusepath{fill}%
\end{pgfscope}%
\begin{pgfscope}%
\pgfpathrectangle{\pgfqpoint{0.100000in}{0.100000in}}{\pgfqpoint{3.007045in}{1.925000in}}%
\pgfusepath{clip}%
\pgfsetbuttcap%
\pgfsetmiterjoin%
\definecolor{currentfill}{rgb}{0.516863,0.735748,0.860192}%
\pgfsetfillcolor{currentfill}%
\pgfsetlinewidth{0.000000pt}%
\definecolor{currentstroke}{rgb}{0.000000,0.000000,0.000000}%
\pgfsetstrokecolor{currentstroke}%
\pgfsetstrokeopacity{0.000000}%
\pgfsetdash{}{0pt}%
\pgfpathmoveto{\pgfqpoint{1.980060in}{1.429597in}}%
\pgfpathlineto{\pgfqpoint{1.954322in}{1.428443in}}%
\pgfpathlineto{\pgfqpoint{1.953526in}{1.433541in}}%
\pgfpathlineto{\pgfqpoint{1.934699in}{1.432796in}}%
\pgfpathlineto{\pgfqpoint{1.928977in}{1.432597in}}%
\pgfpathlineto{\pgfqpoint{1.928151in}{1.455464in}}%
\pgfpathlineto{\pgfqpoint{1.949409in}{1.456290in}}%
\pgfpathlineto{\pgfqpoint{1.942334in}{1.465087in}}%
\pgfpathlineto{\pgfqpoint{1.947232in}{1.465292in}}%
\pgfpathlineto{\pgfqpoint{1.947712in}{1.470634in}}%
\pgfpathlineto{\pgfqpoint{1.963568in}{1.471696in}}%
\pgfpathlineto{\pgfqpoint{1.966485in}{1.477622in}}%
\pgfpathlineto{\pgfqpoint{1.994624in}{1.478781in}}%
\pgfpathlineto{\pgfqpoint{1.994330in}{1.484956in}}%
\pgfpathlineto{\pgfqpoint{2.013699in}{1.486047in}}%
\pgfpathlineto{\pgfqpoint{2.008307in}{1.475250in}}%
\pgfpathlineto{\pgfqpoint{2.012462in}{1.466036in}}%
\pgfpathlineto{\pgfqpoint{2.012501in}{1.460181in}}%
\pgfpathlineto{\pgfqpoint{2.017195in}{1.455365in}}%
\pgfpathlineto{\pgfqpoint{2.021528in}{1.446204in}}%
\pgfpathlineto{\pgfqpoint{1.996384in}{1.444829in}}%
\pgfpathlineto{\pgfqpoint{1.996750in}{1.439088in}}%
\pgfpathlineto{\pgfqpoint{1.979503in}{1.438168in}}%
\pgfpathlineto{\pgfqpoint{1.980060in}{1.429597in}}%
\pgfpathclose%
\pgfusepath{fill}%
\end{pgfscope}%
\begin{pgfscope}%
\pgfpathrectangle{\pgfqpoint{0.100000in}{0.100000in}}{\pgfqpoint{3.007045in}{1.925000in}}%
\pgfusepath{clip}%
\pgfsetbuttcap%
\pgfsetmiterjoin%
\definecolor{currentfill}{rgb}{0.435294,0.690965,0.842599}%
\pgfsetfillcolor{currentfill}%
\pgfsetlinewidth{0.000000pt}%
\definecolor{currentstroke}{rgb}{0.000000,0.000000,0.000000}%
\pgfsetstrokecolor{currentstroke}%
\pgfsetstrokeopacity{0.000000}%
\pgfsetdash{}{0pt}%
\pgfpathmoveto{\pgfqpoint{1.045450in}{1.802919in}}%
\pgfpathlineto{\pgfqpoint{1.043735in}{1.793392in}}%
\pgfpathlineto{\pgfqpoint{1.018457in}{1.797935in}}%
\pgfpathlineto{\pgfqpoint{1.016942in}{1.805156in}}%
\pgfpathlineto{\pgfqpoint{1.011784in}{1.809131in}}%
\pgfpathlineto{\pgfqpoint{0.975124in}{1.816423in}}%
\pgfpathlineto{\pgfqpoint{0.966264in}{1.820814in}}%
\pgfpathlineto{\pgfqpoint{0.967054in}{1.825877in}}%
\pgfpathlineto{\pgfqpoint{0.961768in}{1.833418in}}%
\pgfpathlineto{\pgfqpoint{0.964152in}{1.834807in}}%
\pgfpathlineto{\pgfqpoint{0.959285in}{1.841945in}}%
\pgfpathlineto{\pgfqpoint{0.961059in}{1.847574in}}%
\pgfpathlineto{\pgfqpoint{0.949618in}{1.854808in}}%
\pgfpathlineto{\pgfqpoint{0.953218in}{1.860951in}}%
\pgfpathlineto{\pgfqpoint{0.949121in}{1.871112in}}%
\pgfpathlineto{\pgfqpoint{0.943633in}{1.869658in}}%
\pgfpathlineto{\pgfqpoint{0.941225in}{1.874500in}}%
\pgfpathlineto{\pgfqpoint{0.941334in}{1.882540in}}%
\pgfpathlineto{\pgfqpoint{0.981051in}{1.874255in}}%
\pgfpathlineto{\pgfqpoint{1.010982in}{1.868345in}}%
\pgfpathlineto{\pgfqpoint{1.061807in}{1.858712in}}%
\pgfpathlineto{\pgfqpoint{1.052373in}{1.807583in}}%
\pgfpathlineto{\pgfqpoint{1.046502in}{1.808652in}}%
\pgfpathlineto{\pgfqpoint{1.045450in}{1.802919in}}%
\pgfpathclose%
\pgfusepath{fill}%
\end{pgfscope}%
\begin{pgfscope}%
\pgfpathrectangle{\pgfqpoint{0.100000in}{0.100000in}}{\pgfqpoint{3.007045in}{1.925000in}}%
\pgfusepath{clip}%
\pgfsetbuttcap%
\pgfsetmiterjoin%
\definecolor{currentfill}{rgb}{0.240046,0.553772,0.766797}%
\pgfsetfillcolor{currentfill}%
\pgfsetlinewidth{0.000000pt}%
\definecolor{currentstroke}{rgb}{0.000000,0.000000,0.000000}%
\pgfsetstrokecolor{currentstroke}%
\pgfsetstrokeopacity{0.000000}%
\pgfsetdash{}{0pt}%
\pgfpathmoveto{\pgfqpoint{1.522557in}{1.725806in}}%
\pgfpathlineto{\pgfqpoint{1.465127in}{1.729765in}}%
\pgfpathlineto{\pgfqpoint{1.466009in}{1.741430in}}%
\pgfpathlineto{\pgfqpoint{1.463751in}{1.741609in}}%
\pgfpathlineto{\pgfqpoint{1.465533in}{1.764842in}}%
\pgfpathlineto{\pgfqpoint{1.463268in}{1.765000in}}%
\pgfpathlineto{\pgfqpoint{1.464168in}{1.776657in}}%
\pgfpathlineto{\pgfqpoint{1.452658in}{1.777477in}}%
\pgfpathlineto{\pgfqpoint{1.453609in}{1.789081in}}%
\pgfpathlineto{\pgfqpoint{1.457171in}{1.788804in}}%
\pgfpathlineto{\pgfqpoint{1.457650in}{1.794593in}}%
\pgfpathlineto{\pgfqpoint{1.463375in}{1.794138in}}%
\pgfpathlineto{\pgfqpoint{1.464391in}{1.806947in}}%
\pgfpathlineto{\pgfqpoint{1.508366in}{1.803801in}}%
\pgfpathlineto{\pgfqpoint{1.544920in}{1.801546in}}%
\pgfpathlineto{\pgfqpoint{1.543838in}{1.782832in}}%
\pgfpathlineto{\pgfqpoint{1.544810in}{1.771147in}}%
\pgfpathlineto{\pgfqpoint{1.539033in}{1.771449in}}%
\pgfpathlineto{\pgfqpoint{1.540108in}{1.759776in}}%
\pgfpathlineto{\pgfqpoint{1.538782in}{1.736495in}}%
\pgfpathlineto{\pgfqpoint{1.539856in}{1.724787in}}%
\pgfpathlineto{\pgfqpoint{1.522557in}{1.725806in}}%
\pgfpathclose%
\pgfusepath{fill}%
\end{pgfscope}%
\begin{pgfscope}%
\pgfpathrectangle{\pgfqpoint{0.100000in}{0.100000in}}{\pgfqpoint{3.007045in}{1.925000in}}%
\pgfusepath{clip}%
\pgfsetbuttcap%
\pgfsetmiterjoin%
\definecolor{currentfill}{rgb}{0.124029,0.436248,0.704421}%
\pgfsetfillcolor{currentfill}%
\pgfsetlinewidth{0.000000pt}%
\definecolor{currentstroke}{rgb}{0.000000,0.000000,0.000000}%
\pgfsetstrokecolor{currentstroke}%
\pgfsetstrokeopacity{0.000000}%
\pgfsetdash{}{0pt}%
\pgfpathmoveto{\pgfqpoint{1.591111in}{1.462290in}}%
\pgfpathlineto{\pgfqpoint{1.590530in}{1.433489in}}%
\pgfpathlineto{\pgfqpoint{1.566436in}{1.434578in}}%
\pgfpathlineto{\pgfqpoint{1.550851in}{1.435352in}}%
\pgfpathlineto{\pgfqpoint{1.551671in}{1.451429in}}%
\pgfpathlineto{\pgfqpoint{1.541165in}{1.448356in}}%
\pgfpathlineto{\pgfqpoint{1.531144in}{1.449134in}}%
\pgfpathlineto{\pgfqpoint{1.522608in}{1.451050in}}%
\pgfpathlineto{\pgfqpoint{1.517464in}{1.451554in}}%
\pgfpathlineto{\pgfqpoint{1.516674in}{1.460211in}}%
\pgfpathlineto{\pgfqpoint{1.517880in}{1.481436in}}%
\pgfpathlineto{\pgfqpoint{1.517980in}{1.483159in}}%
\pgfpathlineto{\pgfqpoint{1.538292in}{1.482046in}}%
\pgfpathlineto{\pgfqpoint{1.542308in}{1.477127in}}%
\pgfpathlineto{\pgfqpoint{1.546378in}{1.477641in}}%
\pgfpathlineto{\pgfqpoint{1.552579in}{1.474853in}}%
\pgfpathlineto{\pgfqpoint{1.551371in}{1.482830in}}%
\pgfpathlineto{\pgfqpoint{1.555401in}{1.480969in}}%
\pgfpathlineto{\pgfqpoint{1.586215in}{1.479799in}}%
\pgfpathlineto{\pgfqpoint{1.585428in}{1.462527in}}%
\pgfpathlineto{\pgfqpoint{1.591111in}{1.462290in}}%
\pgfpathclose%
\pgfusepath{fill}%
\end{pgfscope}%
\begin{pgfscope}%
\pgfpathrectangle{\pgfqpoint{0.100000in}{0.100000in}}{\pgfqpoint{3.007045in}{1.925000in}}%
\pgfusepath{clip}%
\pgfsetbuttcap%
\pgfsetmiterjoin%
\definecolor{currentfill}{rgb}{0.479216,0.715079,0.852072}%
\pgfsetfillcolor{currentfill}%
\pgfsetlinewidth{0.000000pt}%
\definecolor{currentstroke}{rgb}{0.000000,0.000000,0.000000}%
\pgfsetstrokecolor{currentstroke}%
\pgfsetstrokeopacity{0.000000}%
\pgfsetdash{}{0pt}%
\pgfpathmoveto{\pgfqpoint{2.429643in}{1.109432in}}%
\pgfpathlineto{\pgfqpoint{2.435308in}{1.112304in}}%
\pgfpathlineto{\pgfqpoint{2.440031in}{1.105220in}}%
\pgfpathlineto{\pgfqpoint{2.441207in}{1.098026in}}%
\pgfpathlineto{\pgfqpoint{2.446820in}{1.093542in}}%
\pgfpathlineto{\pgfqpoint{2.458232in}{1.094688in}}%
\pgfpathlineto{\pgfqpoint{2.456556in}{1.087012in}}%
\pgfpathlineto{\pgfqpoint{2.453185in}{1.084073in}}%
\pgfpathlineto{\pgfqpoint{2.450852in}{1.085445in}}%
\pgfpathlineto{\pgfqpoint{2.435154in}{1.067105in}}%
\pgfpathlineto{\pgfqpoint{2.423037in}{1.058784in}}%
\pgfpathlineto{\pgfqpoint{2.417226in}{1.060057in}}%
\pgfpathlineto{\pgfqpoint{2.412270in}{1.064935in}}%
\pgfpathlineto{\pgfqpoint{2.410452in}{1.072460in}}%
\pgfpathlineto{\pgfqpoint{2.400437in}{1.076137in}}%
\pgfpathlineto{\pgfqpoint{2.391489in}{1.082969in}}%
\pgfpathlineto{\pgfqpoint{2.382872in}{1.085587in}}%
\pgfpathlineto{\pgfqpoint{2.381551in}{1.091127in}}%
\pgfpathlineto{\pgfqpoint{2.389841in}{1.100458in}}%
\pgfpathlineto{\pgfqpoint{2.396141in}{1.102157in}}%
\pgfpathlineto{\pgfqpoint{2.392249in}{1.108544in}}%
\pgfpathlineto{\pgfqpoint{2.398193in}{1.116290in}}%
\pgfpathlineto{\pgfqpoint{2.396061in}{1.120463in}}%
\pgfpathlineto{\pgfqpoint{2.401483in}{1.125255in}}%
\pgfpathlineto{\pgfqpoint{2.411679in}{1.127319in}}%
\pgfpathlineto{\pgfqpoint{2.410869in}{1.119825in}}%
\pgfpathlineto{\pgfqpoint{2.421148in}{1.110912in}}%
\pgfpathlineto{\pgfqpoint{2.420050in}{1.106990in}}%
\pgfpathlineto{\pgfqpoint{2.425369in}{1.103463in}}%
\pgfpathlineto{\pgfqpoint{2.429643in}{1.109432in}}%
\pgfpathclose%
\pgfusepath{fill}%
\end{pgfscope}%
\begin{pgfscope}%
\pgfpathrectangle{\pgfqpoint{0.100000in}{0.100000in}}{\pgfqpoint{3.007045in}{1.925000in}}%
\pgfusepath{clip}%
\pgfsetbuttcap%
\pgfsetmiterjoin%
\definecolor{currentfill}{rgb}{0.498039,0.725413,0.856132}%
\pgfsetfillcolor{currentfill}%
\pgfsetlinewidth{0.000000pt}%
\definecolor{currentstroke}{rgb}{0.000000,0.000000,0.000000}%
\pgfsetstrokecolor{currentstroke}%
\pgfsetstrokeopacity{0.000000}%
\pgfsetdash{}{0pt}%
\pgfpathmoveto{\pgfqpoint{1.844917in}{1.391266in}}%
\pgfpathlineto{\pgfqpoint{1.822052in}{1.390899in}}%
\pgfpathlineto{\pgfqpoint{1.821649in}{1.429987in}}%
\pgfpathlineto{\pgfqpoint{1.837085in}{1.430186in}}%
\pgfpathlineto{\pgfqpoint{1.866993in}{1.430735in}}%
\pgfpathlineto{\pgfqpoint{1.867818in}{1.391658in}}%
\pgfpathlineto{\pgfqpoint{1.844917in}{1.391266in}}%
\pgfpathclose%
\pgfusepath{fill}%
\end{pgfscope}%
\begin{pgfscope}%
\pgfpathrectangle{\pgfqpoint{0.100000in}{0.100000in}}{\pgfqpoint{3.007045in}{1.925000in}}%
\pgfusepath{clip}%
\pgfsetbuttcap%
\pgfsetmiterjoin%
\definecolor{currentfill}{rgb}{0.627605,0.795556,0.885152}%
\pgfsetfillcolor{currentfill}%
\pgfsetlinewidth{0.000000pt}%
\definecolor{currentstroke}{rgb}{0.000000,0.000000,0.000000}%
\pgfsetstrokecolor{currentstroke}%
\pgfsetstrokeopacity{0.000000}%
\pgfsetdash{}{0pt}%
\pgfpathmoveto{\pgfqpoint{2.463450in}{1.158871in}}%
\pgfpathlineto{\pgfqpoint{2.458391in}{1.154238in}}%
\pgfpathlineto{\pgfqpoint{2.450037in}{1.161421in}}%
\pgfpathlineto{\pgfqpoint{2.441696in}{1.172351in}}%
\pgfpathlineto{\pgfqpoint{2.447277in}{1.183280in}}%
\pgfpathlineto{\pgfqpoint{2.446120in}{1.194088in}}%
\pgfpathlineto{\pgfqpoint{2.446995in}{1.200159in}}%
\pgfpathlineto{\pgfqpoint{2.439887in}{1.202342in}}%
\pgfpathlineto{\pgfqpoint{2.439065in}{1.211016in}}%
\pgfpathlineto{\pgfqpoint{2.439738in}{1.214050in}}%
\pgfpathlineto{\pgfqpoint{2.445630in}{1.214029in}}%
\pgfpathlineto{\pgfqpoint{2.445563in}{1.221325in}}%
\pgfpathlineto{\pgfqpoint{2.456449in}{1.222465in}}%
\pgfpathlineto{\pgfqpoint{2.465782in}{1.225372in}}%
\pgfpathlineto{\pgfqpoint{2.467867in}{1.223907in}}%
\pgfpathlineto{\pgfqpoint{2.478476in}{1.225076in}}%
\pgfpathlineto{\pgfqpoint{2.476467in}{1.219730in}}%
\pgfpathlineto{\pgfqpoint{2.485346in}{1.213775in}}%
\pgfpathlineto{\pgfqpoint{2.488733in}{1.208153in}}%
\pgfpathlineto{\pgfqpoint{2.487686in}{1.206566in}}%
\pgfpathlineto{\pgfqpoint{2.494442in}{1.196721in}}%
\pgfpathlineto{\pgfqpoint{2.489753in}{1.191584in}}%
\pgfpathlineto{\pgfqpoint{2.484306in}{1.188484in}}%
\pgfpathlineto{\pgfqpoint{2.478749in}{1.189995in}}%
\pgfpathlineto{\pgfqpoint{2.469733in}{1.178659in}}%
\pgfpathlineto{\pgfqpoint{2.463999in}{1.180859in}}%
\pgfpathlineto{\pgfqpoint{2.460301in}{1.177194in}}%
\pgfpathlineto{\pgfqpoint{2.461626in}{1.162571in}}%
\pgfpathlineto{\pgfqpoint{2.463450in}{1.158871in}}%
\pgfpathclose%
\pgfusepath{fill}%
\end{pgfscope}%
\begin{pgfscope}%
\pgfpathrectangle{\pgfqpoint{0.100000in}{0.100000in}}{\pgfqpoint{3.007045in}{1.925000in}}%
\pgfusepath{clip}%
\pgfsetbuttcap%
\pgfsetmiterjoin%
\definecolor{currentfill}{rgb}{0.510588,0.732303,0.858839}%
\pgfsetfillcolor{currentfill}%
\pgfsetlinewidth{0.000000pt}%
\definecolor{currentstroke}{rgb}{0.000000,0.000000,0.000000}%
\pgfsetstrokecolor{currentstroke}%
\pgfsetstrokeopacity{0.000000}%
\pgfsetdash{}{0pt}%
\pgfpathmoveto{\pgfqpoint{1.898889in}{1.500432in}}%
\pgfpathlineto{\pgfqpoint{1.907419in}{1.500598in}}%
\pgfpathlineto{\pgfqpoint{1.906167in}{1.539095in}}%
\pgfpathlineto{\pgfqpoint{1.905032in}{1.544903in}}%
\pgfpathlineto{\pgfqpoint{1.933694in}{1.545701in}}%
\pgfpathlineto{\pgfqpoint{1.933521in}{1.551363in}}%
\pgfpathlineto{\pgfqpoint{1.962214in}{1.552559in}}%
\pgfpathlineto{\pgfqpoint{1.964639in}{1.500815in}}%
\pgfpathlineto{\pgfqpoint{1.970355in}{1.501053in}}%
\pgfpathlineto{\pgfqpoint{1.970609in}{1.495307in}}%
\pgfpathlineto{\pgfqpoint{1.993541in}{1.496568in}}%
\pgfpathlineto{\pgfqpoint{1.994330in}{1.484956in}}%
\pgfpathlineto{\pgfqpoint{1.994624in}{1.478781in}}%
\pgfpathlineto{\pgfqpoint{1.966485in}{1.477622in}}%
\pgfpathlineto{\pgfqpoint{1.963568in}{1.471696in}}%
\pgfpathlineto{\pgfqpoint{1.947712in}{1.470634in}}%
\pgfpathlineto{\pgfqpoint{1.947232in}{1.465292in}}%
\pgfpathlineto{\pgfqpoint{1.942334in}{1.465087in}}%
\pgfpathlineto{\pgfqpoint{1.949409in}{1.456290in}}%
\pgfpathlineto{\pgfqpoint{1.928151in}{1.455464in}}%
\pgfpathlineto{\pgfqpoint{1.911530in}{1.454907in}}%
\pgfpathlineto{\pgfqpoint{1.910791in}{1.477769in}}%
\pgfpathlineto{\pgfqpoint{1.921271in}{1.478116in}}%
\pgfpathlineto{\pgfqpoint{1.917765in}{1.487298in}}%
\pgfpathlineto{\pgfqpoint{1.911427in}{1.491667in}}%
\pgfpathlineto{\pgfqpoint{1.903859in}{1.493796in}}%
\pgfpathlineto{\pgfqpoint{1.898889in}{1.500432in}}%
\pgfpathclose%
\pgfusepath{fill}%
\end{pgfscope}%
\begin{pgfscope}%
\pgfpathrectangle{\pgfqpoint{0.100000in}{0.100000in}}{\pgfqpoint{3.007045in}{1.925000in}}%
\pgfusepath{clip}%
\pgfsetbuttcap%
\pgfsetmiterjoin%
\definecolor{currentfill}{rgb}{0.341423,0.628958,0.808704}%
\pgfsetfillcolor{currentfill}%
\pgfsetlinewidth{0.000000pt}%
\definecolor{currentstroke}{rgb}{0.000000,0.000000,0.000000}%
\pgfsetstrokecolor{currentstroke}%
\pgfsetstrokeopacity{0.000000}%
\pgfsetdash{}{0pt}%
\pgfpathmoveto{\pgfqpoint{1.632363in}{0.431928in}}%
\pgfpathlineto{\pgfqpoint{1.620569in}{0.414900in}}%
\pgfpathlineto{\pgfqpoint{1.605884in}{0.418493in}}%
\pgfpathlineto{\pgfqpoint{1.606432in}{0.418922in}}%
\pgfpathlineto{\pgfqpoint{1.590934in}{0.451560in}}%
\pgfpathlineto{\pgfqpoint{1.595954in}{0.453216in}}%
\pgfpathlineto{\pgfqpoint{1.585508in}{0.466431in}}%
\pgfpathlineto{\pgfqpoint{1.613261in}{0.488228in}}%
\pgfpathlineto{\pgfqpoint{1.619767in}{0.480550in}}%
\pgfpathlineto{\pgfqpoint{1.611272in}{0.473936in}}%
\pgfpathlineto{\pgfqpoint{1.621451in}{0.460671in}}%
\pgfpathlineto{\pgfqpoint{1.609200in}{0.451305in}}%
\pgfpathlineto{\pgfqpoint{1.613769in}{0.442713in}}%
\pgfpathlineto{\pgfqpoint{1.621822in}{0.440642in}}%
\pgfpathlineto{\pgfqpoint{1.622101in}{0.436398in}}%
\pgfpathlineto{\pgfqpoint{1.632363in}{0.431928in}}%
\pgfpathclose%
\pgfusepath{fill}%
\end{pgfscope}%
\begin{pgfscope}%
\pgfpathrectangle{\pgfqpoint{0.100000in}{0.100000in}}{\pgfqpoint{3.007045in}{1.925000in}}%
\pgfusepath{clip}%
\pgfsetbuttcap%
\pgfsetmiterjoin%
\definecolor{currentfill}{rgb}{0.366644,0.646182,0.818547}%
\pgfsetfillcolor{currentfill}%
\pgfsetlinewidth{0.000000pt}%
\definecolor{currentstroke}{rgb}{0.000000,0.000000,0.000000}%
\pgfsetstrokecolor{currentstroke}%
\pgfsetstrokeopacity{0.000000}%
\pgfsetdash{}{0pt}%
\pgfpathmoveto{\pgfqpoint{2.516660in}{0.795332in}}%
\pgfpathlineto{\pgfqpoint{2.505569in}{0.784566in}}%
\pgfpathlineto{\pgfqpoint{2.500186in}{0.786765in}}%
\pgfpathlineto{\pgfqpoint{2.488225in}{0.782556in}}%
\pgfpathlineto{\pgfqpoint{2.485475in}{0.775665in}}%
\pgfpathlineto{\pgfqpoint{2.482589in}{0.774200in}}%
\pgfpathlineto{\pgfqpoint{2.475814in}{0.775674in}}%
\pgfpathlineto{\pgfqpoint{2.470222in}{0.769671in}}%
\pgfpathlineto{\pgfqpoint{2.464508in}{0.773648in}}%
\pgfpathlineto{\pgfqpoint{2.464577in}{0.779288in}}%
\pgfpathlineto{\pgfqpoint{2.461072in}{0.785247in}}%
\pgfpathlineto{\pgfqpoint{2.454423in}{0.792007in}}%
\pgfpathlineto{\pgfqpoint{2.449695in}{0.794403in}}%
\pgfpathlineto{\pgfqpoint{2.448640in}{0.798533in}}%
\pgfpathlineto{\pgfqpoint{2.441136in}{0.811210in}}%
\pgfpathlineto{\pgfqpoint{2.442122in}{0.815272in}}%
\pgfpathlineto{\pgfqpoint{2.446193in}{0.816402in}}%
\pgfpathlineto{\pgfqpoint{2.449159in}{0.822767in}}%
\pgfpathlineto{\pgfqpoint{2.452424in}{0.825463in}}%
\pgfpathlineto{\pgfqpoint{2.460350in}{0.826839in}}%
\pgfpathlineto{\pgfqpoint{2.468621in}{0.832178in}}%
\pgfpathlineto{\pgfqpoint{2.473536in}{0.831913in}}%
\pgfpathlineto{\pgfqpoint{2.479069in}{0.826299in}}%
\pgfpathlineto{\pgfqpoint{2.480732in}{0.829446in}}%
\pgfpathlineto{\pgfqpoint{2.475134in}{0.834677in}}%
\pgfpathlineto{\pgfqpoint{2.477899in}{0.842105in}}%
\pgfpathlineto{\pgfqpoint{2.474703in}{0.848311in}}%
\pgfpathlineto{\pgfqpoint{2.482424in}{0.851583in}}%
\pgfpathlineto{\pgfqpoint{2.488445in}{0.853368in}}%
\pgfpathlineto{\pgfqpoint{2.492621in}{0.846691in}}%
\pgfpathlineto{\pgfqpoint{2.503173in}{0.844767in}}%
\pgfpathlineto{\pgfqpoint{2.506929in}{0.849528in}}%
\pgfpathlineto{\pgfqpoint{2.516115in}{0.841918in}}%
\pgfpathlineto{\pgfqpoint{2.530038in}{0.837997in}}%
\pgfpathlineto{\pgfqpoint{2.521533in}{0.825837in}}%
\pgfpathlineto{\pgfqpoint{2.503376in}{0.803575in}}%
\pgfpathlineto{\pgfqpoint{2.507037in}{0.798645in}}%
\pgfpathlineto{\pgfqpoint{2.516660in}{0.795332in}}%
\pgfpathclose%
\pgfusepath{fill}%
\end{pgfscope}%
\begin{pgfscope}%
\pgfpathrectangle{\pgfqpoint{0.100000in}{0.100000in}}{\pgfqpoint{3.007045in}{1.925000in}}%
\pgfusepath{clip}%
\pgfsetbuttcap%
\pgfsetmiterjoin%
\definecolor{currentfill}{rgb}{0.417086,0.680631,0.838231}%
\pgfsetfillcolor{currentfill}%
\pgfsetlinewidth{0.000000pt}%
\definecolor{currentstroke}{rgb}{0.000000,0.000000,0.000000}%
\pgfsetstrokecolor{currentstroke}%
\pgfsetstrokeopacity{0.000000}%
\pgfsetdash{}{0pt}%
\pgfpathmoveto{\pgfqpoint{2.719169in}{0.923665in}}%
\pgfpathlineto{\pgfqpoint{2.714846in}{0.926915in}}%
\pgfpathlineto{\pgfqpoint{2.706932in}{0.940219in}}%
\pgfpathlineto{\pgfqpoint{2.705959in}{0.953701in}}%
\pgfpathlineto{\pgfqpoint{2.699683in}{0.959124in}}%
\pgfpathlineto{\pgfqpoint{2.696911in}{0.966874in}}%
\pgfpathlineto{\pgfqpoint{2.693069in}{0.968298in}}%
\pgfpathlineto{\pgfqpoint{2.688736in}{0.994006in}}%
\pgfpathlineto{\pgfqpoint{2.708269in}{1.015278in}}%
\pgfpathlineto{\pgfqpoint{2.710520in}{1.014800in}}%
\pgfpathlineto{\pgfqpoint{2.720462in}{1.011491in}}%
\pgfpathlineto{\pgfqpoint{2.722156in}{1.004752in}}%
\pgfpathlineto{\pgfqpoint{2.725670in}{0.999941in}}%
\pgfpathlineto{\pgfqpoint{2.723536in}{0.997192in}}%
\pgfpathlineto{\pgfqpoint{2.724069in}{0.990555in}}%
\pgfpathlineto{\pgfqpoint{2.737785in}{0.985361in}}%
\pgfpathlineto{\pgfqpoint{2.744917in}{0.981075in}}%
\pgfpathlineto{\pgfqpoint{2.755915in}{0.984607in}}%
\pgfpathlineto{\pgfqpoint{2.755737in}{0.991248in}}%
\pgfpathlineto{\pgfqpoint{2.762672in}{0.991320in}}%
\pgfpathlineto{\pgfqpoint{2.764291in}{0.988344in}}%
\pgfpathlineto{\pgfqpoint{2.761049in}{0.976325in}}%
\pgfpathlineto{\pgfqpoint{2.757169in}{0.971840in}}%
\pgfpathlineto{\pgfqpoint{2.760232in}{0.968674in}}%
\pgfpathlineto{\pgfqpoint{2.769233in}{0.972061in}}%
\pgfpathlineto{\pgfqpoint{2.773670in}{0.970838in}}%
\pgfpathlineto{\pgfqpoint{2.777250in}{0.973729in}}%
\pgfpathlineto{\pgfqpoint{2.783649in}{0.973537in}}%
\pgfpathlineto{\pgfqpoint{2.771520in}{0.950478in}}%
\pgfpathlineto{\pgfqpoint{2.767312in}{0.947670in}}%
\pgfpathlineto{\pgfqpoint{2.760927in}{0.948913in}}%
\pgfpathlineto{\pgfqpoint{2.754301in}{0.947904in}}%
\pgfpathlineto{\pgfqpoint{2.736174in}{0.939738in}}%
\pgfpathlineto{\pgfqpoint{2.719169in}{0.923665in}}%
\pgfpathclose%
\pgfusepath{fill}%
\end{pgfscope}%
\begin{pgfscope}%
\pgfpathrectangle{\pgfqpoint{0.100000in}{0.100000in}}{\pgfqpoint{3.007045in}{1.925000in}}%
\pgfusepath{clip}%
\pgfsetbuttcap%
\pgfsetmiterjoin%
\definecolor{currentfill}{rgb}{0.280892,0.587620,0.785083}%
\pgfsetfillcolor{currentfill}%
\pgfsetlinewidth{0.000000pt}%
\definecolor{currentstroke}{rgb}{0.000000,0.000000,0.000000}%
\pgfsetstrokecolor{currentstroke}%
\pgfsetstrokeopacity{0.000000}%
\pgfsetdash{}{0pt}%
\pgfpathmoveto{\pgfqpoint{1.481225in}{0.618416in}}%
\pgfpathlineto{\pgfqpoint{1.479753in}{0.593339in}}%
\pgfpathlineto{\pgfqpoint{1.431514in}{0.595996in}}%
\pgfpathlineto{\pgfqpoint{1.433065in}{0.620988in}}%
\pgfpathlineto{\pgfqpoint{1.415290in}{0.621980in}}%
\pgfpathlineto{\pgfqpoint{1.417736in}{0.651743in}}%
\pgfpathlineto{\pgfqpoint{1.420501in}{0.688714in}}%
\pgfpathlineto{\pgfqpoint{1.425000in}{0.688410in}}%
\pgfpathlineto{\pgfqpoint{1.454083in}{0.686391in}}%
\pgfpathlineto{\pgfqpoint{1.478210in}{0.684741in}}%
\pgfpathlineto{\pgfqpoint{1.476325in}{0.651539in}}%
\pgfpathlineto{\pgfqpoint{1.483290in}{0.651066in}}%
\pgfpathlineto{\pgfqpoint{1.481225in}{0.618416in}}%
\pgfpathclose%
\pgfusepath{fill}%
\end{pgfscope}%
\begin{pgfscope}%
\pgfpathrectangle{\pgfqpoint{0.100000in}{0.100000in}}{\pgfqpoint{3.007045in}{1.925000in}}%
\pgfusepath{clip}%
\pgfsetbuttcap%
\pgfsetmiterjoin%
\definecolor{currentfill}{rgb}{0.371688,0.649627,0.820515}%
\pgfsetfillcolor{currentfill}%
\pgfsetlinewidth{0.000000pt}%
\definecolor{currentstroke}{rgb}{0.000000,0.000000,0.000000}%
\pgfsetstrokecolor{currentstroke}%
\pgfsetstrokeopacity{0.000000}%
\pgfsetdash{}{0pt}%
\pgfpathmoveto{\pgfqpoint{2.382703in}{0.849302in}}%
\pgfpathlineto{\pgfqpoint{2.380206in}{0.847045in}}%
\pgfpathlineto{\pgfqpoint{2.372820in}{0.852600in}}%
\pgfpathlineto{\pgfqpoint{2.375538in}{0.858536in}}%
\pgfpathlineto{\pgfqpoint{2.371582in}{0.867326in}}%
\pgfpathlineto{\pgfqpoint{2.365118in}{0.869291in}}%
\pgfpathlineto{\pgfqpoint{2.359050in}{0.873951in}}%
\pgfpathlineto{\pgfqpoint{2.358828in}{0.878027in}}%
\pgfpathlineto{\pgfqpoint{2.360142in}{0.881259in}}%
\pgfpathlineto{\pgfqpoint{2.366727in}{0.881594in}}%
\pgfpathlineto{\pgfqpoint{2.371189in}{0.888722in}}%
\pgfpathlineto{\pgfqpoint{2.375790in}{0.888043in}}%
\pgfpathlineto{\pgfqpoint{2.377707in}{0.892532in}}%
\pgfpathlineto{\pgfqpoint{2.385967in}{0.896018in}}%
\pgfpathlineto{\pgfqpoint{2.389462in}{0.896298in}}%
\pgfpathlineto{\pgfqpoint{2.391995in}{0.891918in}}%
\pgfpathlineto{\pgfqpoint{2.402990in}{0.891059in}}%
\pgfpathlineto{\pgfqpoint{2.403995in}{0.889257in}}%
\pgfpathlineto{\pgfqpoint{2.402644in}{0.888304in}}%
\pgfpathlineto{\pgfqpoint{2.399169in}{0.874931in}}%
\pgfpathlineto{\pgfqpoint{2.403763in}{0.872065in}}%
\pgfpathlineto{\pgfqpoint{2.404051in}{0.865135in}}%
\pgfpathlineto{\pgfqpoint{2.406992in}{0.858796in}}%
\pgfpathlineto{\pgfqpoint{2.404649in}{0.856728in}}%
\pgfpathlineto{\pgfqpoint{2.407803in}{0.851715in}}%
\pgfpathlineto{\pgfqpoint{2.406734in}{0.847180in}}%
\pgfpathlineto{\pgfqpoint{2.399267in}{0.840607in}}%
\pgfpathlineto{\pgfqpoint{2.397316in}{0.844754in}}%
\pgfpathlineto{\pgfqpoint{2.386057in}{0.845572in}}%
\pgfpathlineto{\pgfqpoint{2.382703in}{0.849302in}}%
\pgfpathclose%
\pgfusepath{fill}%
\end{pgfscope}%
\begin{pgfscope}%
\pgfpathrectangle{\pgfqpoint{0.100000in}{0.100000in}}{\pgfqpoint{3.007045in}{1.925000in}}%
\pgfusepath{clip}%
\pgfsetbuttcap%
\pgfsetmiterjoin%
\definecolor{currentfill}{rgb}{0.510588,0.732303,0.858839}%
\pgfsetfillcolor{currentfill}%
\pgfsetlinewidth{0.000000pt}%
\definecolor{currentstroke}{rgb}{0.000000,0.000000,0.000000}%
\pgfsetstrokecolor{currentstroke}%
\pgfsetstrokeopacity{0.000000}%
\pgfsetdash{}{0pt}%
\pgfpathmoveto{\pgfqpoint{1.746105in}{1.095854in}}%
\pgfpathlineto{\pgfqpoint{1.723295in}{1.096039in}}%
\pgfpathlineto{\pgfqpoint{1.723731in}{1.116097in}}%
\pgfpathlineto{\pgfqpoint{1.702839in}{1.116383in}}%
\pgfpathlineto{\pgfqpoint{1.700977in}{1.122113in}}%
\pgfpathlineto{\pgfqpoint{1.695279in}{1.122232in}}%
\pgfpathlineto{\pgfqpoint{1.695539in}{1.136550in}}%
\pgfpathlineto{\pgfqpoint{1.701186in}{1.136436in}}%
\pgfpathlineto{\pgfqpoint{1.701412in}{1.144959in}}%
\pgfpathlineto{\pgfqpoint{1.708830in}{1.147175in}}%
\pgfpathlineto{\pgfqpoint{1.712256in}{1.145853in}}%
\pgfpathlineto{\pgfqpoint{1.719484in}{1.147603in}}%
\pgfpathlineto{\pgfqpoint{1.719727in}{1.170628in}}%
\pgfpathlineto{\pgfqpoint{1.732247in}{1.170499in}}%
\pgfpathlineto{\pgfqpoint{1.732315in}{1.176225in}}%
\pgfpathlineto{\pgfqpoint{1.743716in}{1.176139in}}%
\pgfpathlineto{\pgfqpoint{1.743298in}{1.160738in}}%
\pgfpathlineto{\pgfqpoint{1.763181in}{1.160646in}}%
\pgfpathlineto{\pgfqpoint{1.763130in}{1.141541in}}%
\pgfpathlineto{\pgfqpoint{1.764032in}{1.131806in}}%
\pgfpathlineto{\pgfqpoint{1.769446in}{1.131864in}}%
\pgfpathlineto{\pgfqpoint{1.769426in}{1.115817in}}%
\pgfpathlineto{\pgfqpoint{1.746586in}{1.115888in}}%
\pgfpathlineto{\pgfqpoint{1.746105in}{1.095854in}}%
\pgfpathclose%
\pgfusepath{fill}%
\end{pgfscope}%
\begin{pgfscope}%
\pgfpathrectangle{\pgfqpoint{0.100000in}{0.100000in}}{\pgfqpoint{3.007045in}{1.925000in}}%
\pgfusepath{clip}%
\pgfsetbuttcap%
\pgfsetmiterjoin%
\definecolor{currentfill}{rgb}{0.301069,0.601399,0.792957}%
\pgfsetfillcolor{currentfill}%
\pgfsetlinewidth{0.000000pt}%
\definecolor{currentstroke}{rgb}{0.000000,0.000000,0.000000}%
\pgfsetstrokecolor{currentstroke}%
\pgfsetstrokeopacity{0.000000}%
\pgfsetdash{}{0pt}%
\pgfpathmoveto{\pgfqpoint{1.704139in}{0.513794in}}%
\pgfpathlineto{\pgfqpoint{1.696363in}{0.505939in}}%
\pgfpathlineto{\pgfqpoint{1.693878in}{0.498881in}}%
\pgfpathlineto{\pgfqpoint{1.676720in}{0.488443in}}%
\pgfpathlineto{\pgfqpoint{1.681469in}{0.494214in}}%
\pgfpathlineto{\pgfqpoint{1.668257in}{0.510605in}}%
\pgfpathlineto{\pgfqpoint{1.663618in}{0.514403in}}%
\pgfpathlineto{\pgfqpoint{1.648009in}{0.514439in}}%
\pgfpathlineto{\pgfqpoint{1.638148in}{0.525132in}}%
\pgfpathlineto{\pgfqpoint{1.655528in}{0.542512in}}%
\pgfpathlineto{\pgfqpoint{1.665918in}{0.546597in}}%
\pgfpathlineto{\pgfqpoint{1.668934in}{0.549504in}}%
\pgfpathlineto{\pgfqpoint{1.675326in}{0.548748in}}%
\pgfpathlineto{\pgfqpoint{1.683561in}{0.532352in}}%
\pgfpathlineto{\pgfqpoint{1.690540in}{0.526743in}}%
\pgfpathlineto{\pgfqpoint{1.695319in}{0.526005in}}%
\pgfpathlineto{\pgfqpoint{1.699409in}{0.516089in}}%
\pgfpathlineto{\pgfqpoint{1.704139in}{0.513794in}}%
\pgfpathclose%
\pgfusepath{fill}%
\end{pgfscope}%
\begin{pgfscope}%
\pgfpathrectangle{\pgfqpoint{0.100000in}{0.100000in}}{\pgfqpoint{3.007045in}{1.925000in}}%
\pgfusepath{clip}%
\pgfsetbuttcap%
\pgfsetmiterjoin%
\definecolor{currentfill}{rgb}{0.429020,0.687520,0.841246}%
\pgfsetfillcolor{currentfill}%
\pgfsetlinewidth{0.000000pt}%
\definecolor{currentstroke}{rgb}{0.000000,0.000000,0.000000}%
\pgfsetstrokecolor{currentstroke}%
\pgfsetstrokeopacity{0.000000}%
\pgfsetdash{}{0pt}%
\pgfpathmoveto{\pgfqpoint{0.646486in}{1.339327in}}%
\pgfpathlineto{\pgfqpoint{0.594471in}{1.352638in}}%
\pgfpathlineto{\pgfqpoint{0.558741in}{1.362309in}}%
\pgfpathlineto{\pgfqpoint{0.567831in}{1.395284in}}%
\pgfpathlineto{\pgfqpoint{0.569070in}{1.394980in}}%
\pgfpathlineto{\pgfqpoint{0.576936in}{1.423028in}}%
\pgfpathlineto{\pgfqpoint{0.575970in}{1.423296in}}%
\pgfpathlineto{\pgfqpoint{0.584193in}{1.452083in}}%
\pgfpathlineto{\pgfqpoint{0.594261in}{1.488948in}}%
\pgfpathlineto{\pgfqpoint{0.623923in}{1.480760in}}%
\pgfpathlineto{\pgfqpoint{0.647669in}{1.474848in}}%
\pgfpathlineto{\pgfqpoint{0.703346in}{1.460924in}}%
\pgfpathlineto{\pgfqpoint{0.703710in}{1.460774in}}%
\pgfpathlineto{\pgfqpoint{0.687900in}{1.396831in}}%
\pgfpathlineto{\pgfqpoint{0.682240in}{1.373981in}}%
\pgfpathlineto{\pgfqpoint{0.671162in}{1.376694in}}%
\pgfpathlineto{\pgfqpoint{0.646486in}{1.339327in}}%
\pgfpathclose%
\pgfusepath{fill}%
\end{pgfscope}%
\begin{pgfscope}%
\pgfpathrectangle{\pgfqpoint{0.100000in}{0.100000in}}{\pgfqpoint{3.007045in}{1.925000in}}%
\pgfusepath{clip}%
\pgfsetbuttcap%
\pgfsetmiterjoin%
\definecolor{currentfill}{rgb}{0.187266,0.500992,0.739608}%
\pgfsetfillcolor{currentfill}%
\pgfsetlinewidth{0.000000pt}%
\definecolor{currentstroke}{rgb}{0.000000,0.000000,0.000000}%
\pgfsetstrokecolor{currentstroke}%
\pgfsetstrokeopacity{0.000000}%
\pgfsetdash{}{0pt}%
\pgfpathmoveto{\pgfqpoint{1.674656in}{1.382036in}}%
\pgfpathlineto{\pgfqpoint{1.673964in}{1.354981in}}%
\pgfpathlineto{\pgfqpoint{1.656900in}{1.355436in}}%
\pgfpathlineto{\pgfqpoint{1.657059in}{1.361181in}}%
\pgfpathlineto{\pgfqpoint{1.611668in}{1.362462in}}%
\pgfpathlineto{\pgfqpoint{1.612271in}{1.391873in}}%
\pgfpathlineto{\pgfqpoint{1.618866in}{1.388633in}}%
\pgfpathlineto{\pgfqpoint{1.621472in}{1.391045in}}%
\pgfpathlineto{\pgfqpoint{1.622963in}{1.397529in}}%
\pgfpathlineto{\pgfqpoint{1.622340in}{1.404124in}}%
\pgfpathlineto{\pgfqpoint{1.624011in}{1.410336in}}%
\pgfpathlineto{\pgfqpoint{1.656789in}{1.409421in}}%
\pgfpathlineto{\pgfqpoint{1.668294in}{1.409172in}}%
\pgfpathlineto{\pgfqpoint{1.667711in}{1.384730in}}%
\pgfpathlineto{\pgfqpoint{1.674656in}{1.382036in}}%
\pgfpathclose%
\pgfusepath{fill}%
\end{pgfscope}%
\begin{pgfscope}%
\pgfpathrectangle{\pgfqpoint{0.100000in}{0.100000in}}{\pgfqpoint{3.007045in}{1.925000in}}%
\pgfusepath{clip}%
\pgfsetbuttcap%
\pgfsetmiterjoin%
\definecolor{currentfill}{rgb}{0.265759,0.577286,0.779177}%
\pgfsetfillcolor{currentfill}%
\pgfsetlinewidth{0.000000pt}%
\definecolor{currentstroke}{rgb}{0.000000,0.000000,0.000000}%
\pgfsetstrokecolor{currentstroke}%
\pgfsetstrokeopacity{0.000000}%
\pgfsetdash{}{0pt}%
\pgfpathmoveto{\pgfqpoint{2.675874in}{0.992417in}}%
\pgfpathlineto{\pgfqpoint{2.688736in}{0.994006in}}%
\pgfpathlineto{\pgfqpoint{2.693069in}{0.968298in}}%
\pgfpathlineto{\pgfqpoint{2.696911in}{0.966874in}}%
\pgfpathlineto{\pgfqpoint{2.699683in}{0.959124in}}%
\pgfpathlineto{\pgfqpoint{2.705959in}{0.953701in}}%
\pgfpathlineto{\pgfqpoint{2.706932in}{0.940219in}}%
\pgfpathlineto{\pgfqpoint{2.683843in}{0.935913in}}%
\pgfpathlineto{\pgfqpoint{2.678361in}{0.924174in}}%
\pgfpathlineto{\pgfqpoint{2.668672in}{0.934954in}}%
\pgfpathlineto{\pgfqpoint{2.661874in}{0.940850in}}%
\pgfpathlineto{\pgfqpoint{2.652361in}{0.948046in}}%
\pgfpathlineto{\pgfqpoint{2.650566in}{0.955387in}}%
\pgfpathlineto{\pgfqpoint{2.650737in}{0.964955in}}%
\pgfpathlineto{\pgfqpoint{2.653930in}{0.970205in}}%
\pgfpathlineto{\pgfqpoint{2.657195in}{0.967322in}}%
\pgfpathlineto{\pgfqpoint{2.661577in}{0.967658in}}%
\pgfpathlineto{\pgfqpoint{2.667375in}{0.972671in}}%
\pgfpathlineto{\pgfqpoint{2.674050in}{0.976293in}}%
\pgfpathlineto{\pgfqpoint{2.673153in}{0.980638in}}%
\pgfpathlineto{\pgfqpoint{2.675874in}{0.992417in}}%
\pgfpathclose%
\pgfusepath{fill}%
\end{pgfscope}%
\begin{pgfscope}%
\pgfpathrectangle{\pgfqpoint{0.100000in}{0.100000in}}{\pgfqpoint{3.007045in}{1.925000in}}%
\pgfusepath{clip}%
\pgfsetbuttcap%
\pgfsetmiterjoin%
\definecolor{currentfill}{rgb}{0.351511,0.635848,0.812641}%
\pgfsetfillcolor{currentfill}%
\pgfsetlinewidth{0.000000pt}%
\definecolor{currentstroke}{rgb}{0.000000,0.000000,0.000000}%
\pgfsetstrokecolor{currentstroke}%
\pgfsetstrokeopacity{0.000000}%
\pgfsetdash{}{0pt}%
\pgfpathmoveto{\pgfqpoint{2.339044in}{0.640515in}}%
\pgfpathlineto{\pgfqpoint{2.341073in}{0.635732in}}%
\pgfpathlineto{\pgfqpoint{2.282576in}{0.629115in}}%
\pgfpathlineto{\pgfqpoint{2.273936in}{0.628314in}}%
\pgfpathlineto{\pgfqpoint{2.270062in}{0.663726in}}%
\pgfpathlineto{\pgfqpoint{2.272250in}{0.669798in}}%
\pgfpathlineto{\pgfqpoint{2.294624in}{0.672055in}}%
\pgfpathlineto{\pgfqpoint{2.313278in}{0.674089in}}%
\pgfpathlineto{\pgfqpoint{2.312691in}{0.679850in}}%
\pgfpathlineto{\pgfqpoint{2.329091in}{0.680831in}}%
\pgfpathlineto{\pgfqpoint{2.333467in}{0.676210in}}%
\pgfpathlineto{\pgfqpoint{2.334926in}{0.671284in}}%
\pgfpathlineto{\pgfqpoint{2.332928in}{0.653635in}}%
\pgfpathlineto{\pgfqpoint{2.334338in}{0.646016in}}%
\pgfpathlineto{\pgfqpoint{2.339044in}{0.640515in}}%
\pgfpathclose%
\pgfusepath{fill}%
\end{pgfscope}%
\begin{pgfscope}%
\pgfpathrectangle{\pgfqpoint{0.100000in}{0.100000in}}{\pgfqpoint{3.007045in}{1.925000in}}%
\pgfusepath{clip}%
\pgfsetbuttcap%
\pgfsetmiterjoin%
\definecolor{currentfill}{rgb}{0.454118,0.701300,0.846659}%
\pgfsetfillcolor{currentfill}%
\pgfsetlinewidth{0.000000pt}%
\definecolor{currentstroke}{rgb}{0.000000,0.000000,0.000000}%
\pgfsetstrokecolor{currentstroke}%
\pgfsetstrokeopacity{0.000000}%
\pgfsetdash{}{0pt}%
\pgfpathmoveto{\pgfqpoint{0.746029in}{1.809361in}}%
\pgfpathlineto{\pgfqpoint{0.701521in}{1.821189in}}%
\pgfpathlineto{\pgfqpoint{0.713636in}{1.864996in}}%
\pgfpathlineto{\pgfqpoint{0.719784in}{1.864006in}}%
\pgfpathlineto{\pgfqpoint{0.730840in}{1.859613in}}%
\pgfpathlineto{\pgfqpoint{0.735184in}{1.852084in}}%
\pgfpathlineto{\pgfqpoint{0.744909in}{1.856652in}}%
\pgfpathlineto{\pgfqpoint{0.748542in}{1.851631in}}%
\pgfpathlineto{\pgfqpoint{0.749934in}{1.845470in}}%
\pgfpathlineto{\pgfqpoint{0.759267in}{1.846349in}}%
\pgfpathlineto{\pgfqpoint{0.762060in}{1.843258in}}%
\pgfpathlineto{\pgfqpoint{0.768647in}{1.844176in}}%
\pgfpathlineto{\pgfqpoint{0.773351in}{1.839140in}}%
\pgfpathlineto{\pgfqpoint{0.777516in}{1.855475in}}%
\pgfpathlineto{\pgfqpoint{0.780353in}{1.860745in}}%
\pgfpathlineto{\pgfqpoint{0.789000in}{1.894475in}}%
\pgfpathlineto{\pgfqpoint{0.783449in}{1.895856in}}%
\pgfpathlineto{\pgfqpoint{0.787685in}{1.900755in}}%
\pgfpathlineto{\pgfqpoint{0.789229in}{1.906864in}}%
\pgfpathlineto{\pgfqpoint{0.797625in}{1.916024in}}%
\pgfpathlineto{\pgfqpoint{0.814525in}{1.911735in}}%
\pgfpathlineto{\pgfqpoint{0.812070in}{1.901824in}}%
\pgfpathlineto{\pgfqpoint{0.822653in}{1.899281in}}%
\pgfpathlineto{\pgfqpoint{0.817161in}{1.876758in}}%
\pgfpathlineto{\pgfqpoint{0.848850in}{1.869247in}}%
\pgfpathlineto{\pgfqpoint{0.844425in}{1.850601in}}%
\pgfpathlineto{\pgfqpoint{0.840821in}{1.835141in}}%
\pgfpathlineto{\pgfqpoint{0.847161in}{1.818241in}}%
\pgfpathlineto{\pgfqpoint{0.850754in}{1.813504in}}%
\pgfpathlineto{\pgfqpoint{0.850775in}{1.806624in}}%
\pgfpathlineto{\pgfqpoint{0.847085in}{1.804318in}}%
\pgfpathlineto{\pgfqpoint{0.851627in}{1.798562in}}%
\pgfpathlineto{\pgfqpoint{0.845810in}{1.797316in}}%
\pgfpathlineto{\pgfqpoint{0.853094in}{1.789354in}}%
\pgfpathlineto{\pgfqpoint{0.853297in}{1.785851in}}%
\pgfpathlineto{\pgfqpoint{0.861882in}{1.780983in}}%
\pgfpathlineto{\pgfqpoint{0.869191in}{1.761494in}}%
\pgfpathlineto{\pgfqpoint{0.873084in}{1.756371in}}%
\pgfpathlineto{\pgfqpoint{0.844253in}{1.762934in}}%
\pgfpathlineto{\pgfqpoint{0.812473in}{1.770492in}}%
\pgfpathlineto{\pgfqpoint{0.813833in}{1.776182in}}%
\pgfpathlineto{\pgfqpoint{0.808364in}{1.778465in}}%
\pgfpathlineto{\pgfqpoint{0.794601in}{1.781859in}}%
\pgfpathlineto{\pgfqpoint{0.784213in}{1.790565in}}%
\pgfpathlineto{\pgfqpoint{0.786311in}{1.799097in}}%
\pgfpathlineto{\pgfqpoint{0.746029in}{1.809361in}}%
\pgfpathclose%
\pgfusepath{fill}%
\end{pgfscope}%
\begin{pgfscope}%
\pgfpathrectangle{\pgfqpoint{0.100000in}{0.100000in}}{\pgfqpoint{3.007045in}{1.925000in}}%
\pgfusepath{clip}%
\pgfsetbuttcap%
\pgfsetmiterjoin%
\definecolor{currentfill}{rgb}{0.306113,0.604844,0.794925}%
\pgfsetfillcolor{currentfill}%
\pgfsetlinewidth{0.000000pt}%
\definecolor{currentstroke}{rgb}{0.000000,0.000000,0.000000}%
\pgfsetstrokecolor{currentstroke}%
\pgfsetstrokeopacity{0.000000}%
\pgfsetdash{}{0pt}%
\pgfpathmoveto{\pgfqpoint{1.839662in}{0.517608in}}%
\pgfpathlineto{\pgfqpoint{1.824849in}{0.515348in}}%
\pgfpathlineto{\pgfqpoint{1.809818in}{0.508665in}}%
\pgfpathlineto{\pgfqpoint{1.809447in}{0.530410in}}%
\pgfpathlineto{\pgfqpoint{1.804575in}{0.530516in}}%
\pgfpathlineto{\pgfqpoint{1.804317in}{0.545414in}}%
\pgfpathlineto{\pgfqpoint{1.795599in}{0.545307in}}%
\pgfpathlineto{\pgfqpoint{1.787662in}{0.570458in}}%
\pgfpathlineto{\pgfqpoint{1.774291in}{0.576196in}}%
\pgfpathlineto{\pgfqpoint{1.770975in}{0.581765in}}%
\pgfpathlineto{\pgfqpoint{1.765743in}{0.581820in}}%
\pgfpathlineto{\pgfqpoint{1.764790in}{0.589806in}}%
\pgfpathlineto{\pgfqpoint{1.760890in}{0.592713in}}%
\pgfpathlineto{\pgfqpoint{1.775795in}{0.607167in}}%
\pgfpathlineto{\pgfqpoint{1.781271in}{0.614134in}}%
\pgfpathlineto{\pgfqpoint{1.786479in}{0.611427in}}%
\pgfpathlineto{\pgfqpoint{1.793823in}{0.610228in}}%
\pgfpathlineto{\pgfqpoint{1.803223in}{0.606687in}}%
\pgfpathlineto{\pgfqpoint{1.821877in}{0.611209in}}%
\pgfpathlineto{\pgfqpoint{1.834269in}{0.615251in}}%
\pgfpathlineto{\pgfqpoint{1.851931in}{0.616696in}}%
\pgfpathlineto{\pgfqpoint{1.854631in}{0.617364in}}%
\pgfpathlineto{\pgfqpoint{1.854154in}{0.604333in}}%
\pgfpathlineto{\pgfqpoint{1.856414in}{0.600398in}}%
\pgfpathlineto{\pgfqpoint{1.853934in}{0.596710in}}%
\pgfpathlineto{\pgfqpoint{1.850658in}{0.583534in}}%
\pgfpathlineto{\pgfqpoint{1.847738in}{0.581014in}}%
\pgfpathlineto{\pgfqpoint{1.844939in}{0.573686in}}%
\pgfpathlineto{\pgfqpoint{1.847088in}{0.567676in}}%
\pgfpathlineto{\pgfqpoint{1.843280in}{0.560440in}}%
\pgfpathlineto{\pgfqpoint{1.846835in}{0.557616in}}%
\pgfpathlineto{\pgfqpoint{1.846781in}{0.541896in}}%
\pgfpathlineto{\pgfqpoint{1.842229in}{0.537436in}}%
\pgfpathlineto{\pgfqpoint{1.840585in}{0.532547in}}%
\pgfpathlineto{\pgfqpoint{1.834496in}{0.525796in}}%
\pgfpathlineto{\pgfqpoint{1.839662in}{0.517608in}}%
\pgfpathclose%
\pgfusepath{fill}%
\end{pgfscope}%
\begin{pgfscope}%
\pgfpathrectangle{\pgfqpoint{0.100000in}{0.100000in}}{\pgfqpoint{3.007045in}{1.925000in}}%
\pgfusepath{clip}%
\pgfsetbuttcap%
\pgfsetmiterjoin%
\definecolor{currentfill}{rgb}{0.260715,0.573841,0.777209}%
\pgfsetfillcolor{currentfill}%
\pgfsetlinewidth{0.000000pt}%
\definecolor{currentstroke}{rgb}{0.000000,0.000000,0.000000}%
\pgfsetstrokecolor{currentstroke}%
\pgfsetstrokeopacity{0.000000}%
\pgfsetdash{}{0pt}%
\pgfpathmoveto{\pgfqpoint{2.697515in}{1.156406in}}%
\pgfpathlineto{\pgfqpoint{2.692040in}{1.154397in}}%
\pgfpathlineto{\pgfqpoint{2.685472in}{1.142255in}}%
\pgfpathlineto{\pgfqpoint{2.681074in}{1.143862in}}%
\pgfpathlineto{\pgfqpoint{2.679314in}{1.148103in}}%
\pgfpathlineto{\pgfqpoint{2.665522in}{1.153025in}}%
\pgfpathlineto{\pgfqpoint{2.656779in}{1.157595in}}%
\pgfpathlineto{\pgfqpoint{2.650904in}{1.157743in}}%
\pgfpathlineto{\pgfqpoint{2.663143in}{1.179525in}}%
\pgfpathlineto{\pgfqpoint{2.662595in}{1.183012in}}%
\pgfpathlineto{\pgfqpoint{2.668861in}{1.192884in}}%
\pgfpathlineto{\pgfqpoint{2.678549in}{1.188294in}}%
\pgfpathlineto{\pgfqpoint{2.680316in}{1.180573in}}%
\pgfpathlineto{\pgfqpoint{2.682867in}{1.178238in}}%
\pgfpathlineto{\pgfqpoint{2.691552in}{1.182331in}}%
\pgfpathlineto{\pgfqpoint{2.695566in}{1.180728in}}%
\pgfpathlineto{\pgfqpoint{2.697143in}{1.177380in}}%
\pgfpathlineto{\pgfqpoint{2.695805in}{1.170685in}}%
\pgfpathlineto{\pgfqpoint{2.690990in}{1.162746in}}%
\pgfpathlineto{\pgfqpoint{2.697515in}{1.156406in}}%
\pgfpathclose%
\pgfusepath{fill}%
\end{pgfscope}%
\begin{pgfscope}%
\pgfpathrectangle{\pgfqpoint{0.100000in}{0.100000in}}{\pgfqpoint{3.007045in}{1.925000in}}%
\pgfusepath{clip}%
\pgfsetbuttcap%
\pgfsetmiterjoin%
\definecolor{currentfill}{rgb}{0.056363,0.349635,0.636755}%
\pgfsetfillcolor{currentfill}%
\pgfsetlinewidth{0.000000pt}%
\definecolor{currentstroke}{rgb}{0.000000,0.000000,0.000000}%
\pgfsetstrokecolor{currentstroke}%
\pgfsetstrokeopacity{0.000000}%
\pgfsetdash{}{0pt}%
\pgfpathmoveto{\pgfqpoint{1.534548in}{1.032337in}}%
\pgfpathlineto{\pgfqpoint{1.533400in}{1.007220in}}%
\pgfpathlineto{\pgfqpoint{1.509153in}{1.008633in}}%
\pgfpathlineto{\pgfqpoint{1.504587in}{1.008926in}}%
\pgfpathlineto{\pgfqpoint{1.505902in}{1.034236in}}%
\pgfpathlineto{\pgfqpoint{1.505407in}{1.040070in}}%
\pgfpathlineto{\pgfqpoint{1.476919in}{1.041763in}}%
\pgfpathlineto{\pgfqpoint{1.478439in}{1.065027in}}%
\pgfpathlineto{\pgfqpoint{1.478495in}{1.076500in}}%
\pgfpathlineto{\pgfqpoint{1.501163in}{1.075036in}}%
\pgfpathlineto{\pgfqpoint{1.502124in}{1.092214in}}%
\pgfpathlineto{\pgfqpoint{1.535329in}{1.090418in}}%
\pgfpathlineto{\pgfqpoint{1.536034in}{1.090382in}}%
\pgfpathlineto{\pgfqpoint{1.534600in}{1.061649in}}%
\pgfpathlineto{\pgfqpoint{1.534859in}{1.055660in}}%
\pgfpathlineto{\pgfqpoint{1.534548in}{1.032337in}}%
\pgfpathclose%
\pgfusepath{fill}%
\end{pgfscope}%
\begin{pgfscope}%
\pgfpathrectangle{\pgfqpoint{0.100000in}{0.100000in}}{\pgfqpoint{3.007045in}{1.925000in}}%
\pgfusepath{clip}%
\pgfsetbuttcap%
\pgfsetmiterjoin%
\definecolor{currentfill}{rgb}{0.504314,0.728858,0.857486}%
\pgfsetfillcolor{currentfill}%
\pgfsetlinewidth{0.000000pt}%
\definecolor{currentstroke}{rgb}{0.000000,0.000000,0.000000}%
\pgfsetstrokecolor{currentstroke}%
\pgfsetstrokeopacity{0.000000}%
\pgfsetdash{}{0pt}%
\pgfpathmoveto{\pgfqpoint{1.482882in}{0.684491in}}%
\pgfpathlineto{\pgfqpoint{1.478210in}{0.684741in}}%
\pgfpathlineto{\pgfqpoint{1.454083in}{0.686391in}}%
\pgfpathlineto{\pgfqpoint{1.458235in}{0.744453in}}%
\pgfpathlineto{\pgfqpoint{1.465814in}{0.743950in}}%
\pgfpathlineto{\pgfqpoint{1.486605in}{0.742533in}}%
\pgfpathlineto{\pgfqpoint{1.482882in}{0.684491in}}%
\pgfpathclose%
\pgfusepath{fill}%
\end{pgfscope}%
\begin{pgfscope}%
\pgfpathrectangle{\pgfqpoint{0.100000in}{0.100000in}}{\pgfqpoint{3.007045in}{1.925000in}}%
\pgfusepath{clip}%
\pgfsetbuttcap%
\pgfsetmiterjoin%
\definecolor{currentfill}{rgb}{0.056363,0.349635,0.636755}%
\pgfsetfillcolor{currentfill}%
\pgfsetlinewidth{0.000000pt}%
\definecolor{currentstroke}{rgb}{0.000000,0.000000,0.000000}%
\pgfsetstrokecolor{currentstroke}%
\pgfsetstrokeopacity{0.000000}%
\pgfsetdash{}{0pt}%
\pgfpathmoveto{\pgfqpoint{1.476919in}{1.041763in}}%
\pgfpathlineto{\pgfqpoint{1.476575in}{1.036016in}}%
\pgfpathlineto{\pgfqpoint{1.454880in}{1.037434in}}%
\pgfpathlineto{\pgfqpoint{1.453744in}{1.037504in}}%
\pgfpathlineto{\pgfqpoint{1.455281in}{1.060433in}}%
\pgfpathlineto{\pgfqpoint{1.405795in}{1.064151in}}%
\pgfpathlineto{\pgfqpoint{1.408351in}{1.098572in}}%
\pgfpathlineto{\pgfqpoint{1.432949in}{1.096770in}}%
\pgfpathlineto{\pgfqpoint{1.478511in}{1.093767in}}%
\pgfpathlineto{\pgfqpoint{1.502124in}{1.092214in}}%
\pgfpathlineto{\pgfqpoint{1.501163in}{1.075036in}}%
\pgfpathlineto{\pgfqpoint{1.478495in}{1.076500in}}%
\pgfpathlineto{\pgfqpoint{1.478439in}{1.065027in}}%
\pgfpathlineto{\pgfqpoint{1.476919in}{1.041763in}}%
\pgfpathclose%
\pgfusepath{fill}%
\end{pgfscope}%
\begin{pgfscope}%
\pgfpathrectangle{\pgfqpoint{0.100000in}{0.100000in}}{\pgfqpoint{3.007045in}{1.925000in}}%
\pgfusepath{clip}%
\pgfsetbuttcap%
\pgfsetmiterjoin%
\definecolor{currentfill}{rgb}{0.429020,0.687520,0.841246}%
\pgfsetfillcolor{currentfill}%
\pgfsetlinewidth{0.000000pt}%
\definecolor{currentstroke}{rgb}{0.000000,0.000000,0.000000}%
\pgfsetstrokecolor{currentstroke}%
\pgfsetstrokeopacity{0.000000}%
\pgfsetdash{}{0pt}%
\pgfpathmoveto{\pgfqpoint{1.917287in}{1.088655in}}%
\pgfpathlineto{\pgfqpoint{1.945763in}{1.089375in}}%
\pgfpathlineto{\pgfqpoint{1.946082in}{1.080499in}}%
\pgfpathlineto{\pgfqpoint{1.952290in}{1.080650in}}%
\pgfpathlineto{\pgfqpoint{1.953009in}{1.056717in}}%
\pgfpathlineto{\pgfqpoint{1.964484in}{1.057050in}}%
\pgfpathlineto{\pgfqpoint{1.964595in}{1.051309in}}%
\pgfpathlineto{\pgfqpoint{1.972796in}{1.051471in}}%
\pgfpathlineto{\pgfqpoint{1.973010in}{1.044275in}}%
\pgfpathlineto{\pgfqpoint{1.981728in}{1.045692in}}%
\pgfpathlineto{\pgfqpoint{1.993117in}{1.046153in}}%
\pgfpathlineto{\pgfqpoint{1.993598in}{1.031047in}}%
\pgfpathlineto{\pgfqpoint{1.995876in}{1.024384in}}%
\pgfpathlineto{\pgfqpoint{1.996111in}{1.017492in}}%
\pgfpathlineto{\pgfqpoint{1.994242in}{1.009746in}}%
\pgfpathlineto{\pgfqpoint{1.984516in}{1.009519in}}%
\pgfpathlineto{\pgfqpoint{1.981594in}{1.012166in}}%
\pgfpathlineto{\pgfqpoint{1.981501in}{1.015319in}}%
\pgfpathlineto{\pgfqpoint{1.975616in}{1.018755in}}%
\pgfpathlineto{\pgfqpoint{1.972674in}{1.026092in}}%
\pgfpathlineto{\pgfqpoint{1.972430in}{1.032823in}}%
\pgfpathlineto{\pgfqpoint{1.942096in}{1.032256in}}%
\pgfpathlineto{\pgfqpoint{1.941697in}{1.043763in}}%
\pgfpathlineto{\pgfqpoint{1.915878in}{1.043286in}}%
\pgfpathlineto{\pgfqpoint{1.915821in}{1.046178in}}%
\pgfpathlineto{\pgfqpoint{1.907105in}{1.050239in}}%
\pgfpathlineto{\pgfqpoint{1.907064in}{1.059811in}}%
\pgfpathlineto{\pgfqpoint{1.906913in}{1.070484in}}%
\pgfpathlineto{\pgfqpoint{1.918382in}{1.070506in}}%
\pgfpathlineto{\pgfqpoint{1.917287in}{1.088655in}}%
\pgfpathclose%
\pgfusepath{fill}%
\end{pgfscope}%
\begin{pgfscope}%
\pgfpathrectangle{\pgfqpoint{0.100000in}{0.100000in}}{\pgfqpoint{3.007045in}{1.925000in}}%
\pgfusepath{clip}%
\pgfsetbuttcap%
\pgfsetmiterjoin%
\definecolor{currentfill}{rgb}{0.376732,0.653072,0.822484}%
\pgfsetfillcolor{currentfill}%
\pgfsetlinewidth{0.000000pt}%
\definecolor{currentstroke}{rgb}{0.000000,0.000000,0.000000}%
\pgfsetstrokecolor{currentstroke}%
\pgfsetstrokeopacity{0.000000}%
\pgfsetdash{}{0pt}%
\pgfpathmoveto{\pgfqpoint{2.375927in}{0.644931in}}%
\pgfpathlineto{\pgfqpoint{2.378692in}{0.619214in}}%
\pgfpathlineto{\pgfqpoint{2.351014in}{0.617483in}}%
\pgfpathlineto{\pgfqpoint{2.346074in}{0.624253in}}%
\pgfpathlineto{\pgfqpoint{2.345288in}{0.628691in}}%
\pgfpathlineto{\pgfqpoint{2.341073in}{0.635732in}}%
\pgfpathlineto{\pgfqpoint{2.339044in}{0.640515in}}%
\pgfpathlineto{\pgfqpoint{2.345074in}{0.640987in}}%
\pgfpathlineto{\pgfqpoint{2.343997in}{0.653135in}}%
\pgfpathlineto{\pgfqpoint{2.359552in}{0.655091in}}%
\pgfpathlineto{\pgfqpoint{2.365478in}{0.655570in}}%
\pgfpathlineto{\pgfqpoint{2.366514in}{0.643850in}}%
\pgfpathlineto{\pgfqpoint{2.375927in}{0.644931in}}%
\pgfpathclose%
\pgfusepath{fill}%
\end{pgfscope}%
\begin{pgfscope}%
\pgfpathrectangle{\pgfqpoint{0.100000in}{0.100000in}}{\pgfqpoint{3.007045in}{1.925000in}}%
\pgfusepath{clip}%
\pgfsetbuttcap%
\pgfsetmiterjoin%
\definecolor{currentfill}{rgb}{0.166967,0.480692,0.729150}%
\pgfsetfillcolor{currentfill}%
\pgfsetlinewidth{0.000000pt}%
\definecolor{currentstroke}{rgb}{0.000000,0.000000,0.000000}%
\pgfsetstrokecolor{currentstroke}%
\pgfsetstrokeopacity{0.000000}%
\pgfsetdash{}{0pt}%
\pgfpathmoveto{\pgfqpoint{1.704139in}{0.513794in}}%
\pgfpathlineto{\pgfqpoint{1.699409in}{0.516089in}}%
\pgfpathlineto{\pgfqpoint{1.695319in}{0.526005in}}%
\pgfpathlineto{\pgfqpoint{1.690540in}{0.526743in}}%
\pgfpathlineto{\pgfqpoint{1.683561in}{0.532352in}}%
\pgfpathlineto{\pgfqpoint{1.675326in}{0.548748in}}%
\pgfpathlineto{\pgfqpoint{1.668934in}{0.549504in}}%
\pgfpathlineto{\pgfqpoint{1.672597in}{0.554943in}}%
\pgfpathlineto{\pgfqpoint{1.677967in}{0.558409in}}%
\pgfpathlineto{\pgfqpoint{1.690544in}{0.561282in}}%
\pgfpathlineto{\pgfqpoint{1.696897in}{0.563937in}}%
\pgfpathlineto{\pgfqpoint{1.704636in}{0.564588in}}%
\pgfpathlineto{\pgfqpoint{1.709428in}{0.556522in}}%
\pgfpathlineto{\pgfqpoint{1.709639in}{0.553402in}}%
\pgfpathlineto{\pgfqpoint{1.726035in}{0.554362in}}%
\pgfpathlineto{\pgfqpoint{1.724776in}{0.579988in}}%
\pgfpathlineto{\pgfqpoint{1.737969in}{0.571808in}}%
\pgfpathlineto{\pgfqpoint{1.751572in}{0.571734in}}%
\pgfpathlineto{\pgfqpoint{1.753626in}{0.595060in}}%
\pgfpathlineto{\pgfqpoint{1.757986in}{0.598171in}}%
\pgfpathlineto{\pgfqpoint{1.760890in}{0.592713in}}%
\pgfpathlineto{\pgfqpoint{1.764790in}{0.589806in}}%
\pgfpathlineto{\pgfqpoint{1.765743in}{0.581820in}}%
\pgfpathlineto{\pgfqpoint{1.770975in}{0.581765in}}%
\pgfpathlineto{\pgfqpoint{1.774291in}{0.576196in}}%
\pgfpathlineto{\pgfqpoint{1.787662in}{0.570458in}}%
\pgfpathlineto{\pgfqpoint{1.795599in}{0.545307in}}%
\pgfpathlineto{\pgfqpoint{1.804317in}{0.545414in}}%
\pgfpathlineto{\pgfqpoint{1.804575in}{0.530516in}}%
\pgfpathlineto{\pgfqpoint{1.809447in}{0.530410in}}%
\pgfpathlineto{\pgfqpoint{1.809818in}{0.508665in}}%
\pgfpathlineto{\pgfqpoint{1.794978in}{0.501878in}}%
\pgfpathlineto{\pgfqpoint{1.797904in}{0.509417in}}%
\pgfpathlineto{\pgfqpoint{1.787377in}{0.506124in}}%
\pgfpathlineto{\pgfqpoint{1.790301in}{0.517701in}}%
\pgfpathlineto{\pgfqpoint{1.787535in}{0.523539in}}%
\pgfpathlineto{\pgfqpoint{1.783015in}{0.521586in}}%
\pgfpathlineto{\pgfqpoint{1.779708in}{0.515842in}}%
\pgfpathlineto{\pgfqpoint{1.774796in}{0.517671in}}%
\pgfpathlineto{\pgfqpoint{1.771403in}{0.513633in}}%
\pgfpathlineto{\pgfqpoint{1.771383in}{0.507282in}}%
\pgfpathlineto{\pgfqpoint{1.777596in}{0.504206in}}%
\pgfpathlineto{\pgfqpoint{1.778463in}{0.493573in}}%
\pgfpathlineto{\pgfqpoint{1.788184in}{0.493368in}}%
\pgfpathlineto{\pgfqpoint{1.765360in}{0.476226in}}%
\pgfpathlineto{\pgfqpoint{1.769052in}{0.484340in}}%
\pgfpathlineto{\pgfqpoint{1.758777in}{0.502145in}}%
\pgfpathlineto{\pgfqpoint{1.759753in}{0.508170in}}%
\pgfpathlineto{\pgfqpoint{1.757144in}{0.510663in}}%
\pgfpathlineto{\pgfqpoint{1.747709in}{0.509837in}}%
\pgfpathlineto{\pgfqpoint{1.745433in}{0.500436in}}%
\pgfpathlineto{\pgfqpoint{1.740375in}{0.500409in}}%
\pgfpathlineto{\pgfqpoint{1.738650in}{0.493869in}}%
\pgfpathlineto{\pgfqpoint{1.734233in}{0.490553in}}%
\pgfpathlineto{\pgfqpoint{1.728488in}{0.492886in}}%
\pgfpathlineto{\pgfqpoint{1.722932in}{0.488802in}}%
\pgfpathlineto{\pgfqpoint{1.717041in}{0.493709in}}%
\pgfpathlineto{\pgfqpoint{1.714651in}{0.500322in}}%
\pgfpathlineto{\pgfqpoint{1.710733in}{0.503026in}}%
\pgfpathlineto{\pgfqpoint{1.710725in}{0.510845in}}%
\pgfpathlineto{\pgfqpoint{1.704139in}{0.513794in}}%
\pgfpathclose%
\pgfusepath{fill}%
\end{pgfscope}%
\begin{pgfscope}%
\pgfpathrectangle{\pgfqpoint{0.100000in}{0.100000in}}{\pgfqpoint{3.007045in}{1.925000in}}%
\pgfusepath{clip}%
\pgfsetbuttcap%
\pgfsetmiterjoin%
\definecolor{currentfill}{rgb}{0.275848,0.584175,0.783114}%
\pgfsetfillcolor{currentfill}%
\pgfsetlinewidth{0.000000pt}%
\definecolor{currentstroke}{rgb}{0.000000,0.000000,0.000000}%
\pgfsetstrokecolor{currentstroke}%
\pgfsetstrokeopacity{0.000000}%
\pgfsetdash{}{0pt}%
\pgfpathmoveto{\pgfqpoint{1.631785in}{0.798613in}}%
\pgfpathlineto{\pgfqpoint{1.625906in}{0.804863in}}%
\pgfpathlineto{\pgfqpoint{1.614953in}{0.795791in}}%
\pgfpathlineto{\pgfqpoint{1.608887in}{0.798669in}}%
\pgfpathlineto{\pgfqpoint{1.610850in}{0.805245in}}%
\pgfpathlineto{\pgfqpoint{1.605264in}{0.805706in}}%
\pgfpathlineto{\pgfqpoint{1.598818in}{0.813845in}}%
\pgfpathlineto{\pgfqpoint{1.588268in}{0.817057in}}%
\pgfpathlineto{\pgfqpoint{1.586260in}{0.813360in}}%
\pgfpathlineto{\pgfqpoint{1.581430in}{0.811070in}}%
\pgfpathlineto{\pgfqpoint{1.574901in}{0.817578in}}%
\pgfpathlineto{\pgfqpoint{1.575401in}{0.829239in}}%
\pgfpathlineto{\pgfqpoint{1.572553in}{0.829339in}}%
\pgfpathlineto{\pgfqpoint{1.573014in}{0.840820in}}%
\pgfpathlineto{\pgfqpoint{1.564075in}{0.841182in}}%
\pgfpathlineto{\pgfqpoint{1.564310in}{0.846924in}}%
\pgfpathlineto{\pgfqpoint{1.565031in}{0.864162in}}%
\pgfpathlineto{\pgfqpoint{1.576210in}{0.863673in}}%
\pgfpathlineto{\pgfqpoint{1.604926in}{0.862611in}}%
\pgfpathlineto{\pgfqpoint{1.604562in}{0.851135in}}%
\pgfpathlineto{\pgfqpoint{1.627401in}{0.850427in}}%
\pgfpathlineto{\pgfqpoint{1.633138in}{0.850249in}}%
\pgfpathlineto{\pgfqpoint{1.631785in}{0.798613in}}%
\pgfpathclose%
\pgfusepath{fill}%
\end{pgfscope}%
\begin{pgfscope}%
\pgfpathrectangle{\pgfqpoint{0.100000in}{0.100000in}}{\pgfqpoint{3.007045in}{1.925000in}}%
\pgfusepath{clip}%
\pgfsetbuttcap%
\pgfsetmiterjoin%
\definecolor{currentfill}{rgb}{0.691580,0.822745,0.907543}%
\pgfsetfillcolor{currentfill}%
\pgfsetlinewidth{0.000000pt}%
\definecolor{currentstroke}{rgb}{0.000000,0.000000,0.000000}%
\pgfsetstrokecolor{currentstroke}%
\pgfsetstrokeopacity{0.000000}%
\pgfsetdash{}{0pt}%
\pgfpathmoveto{\pgfqpoint{2.815641in}{1.458005in}}%
\pgfpathlineto{\pgfqpoint{2.815310in}{1.453061in}}%
\pgfpathlineto{\pgfqpoint{2.806124in}{1.452725in}}%
\pgfpathlineto{\pgfqpoint{2.795846in}{1.455210in}}%
\pgfpathlineto{\pgfqpoint{2.794361in}{1.458299in}}%
\pgfpathlineto{\pgfqpoint{2.787830in}{1.459571in}}%
\pgfpathlineto{\pgfqpoint{2.787399in}{1.454470in}}%
\pgfpathlineto{\pgfqpoint{2.776040in}{1.453025in}}%
\pgfpathlineto{\pgfqpoint{2.768356in}{1.454967in}}%
\pgfpathlineto{\pgfqpoint{2.763804in}{1.456101in}}%
\pgfpathlineto{\pgfqpoint{2.766813in}{1.466787in}}%
\pgfpathlineto{\pgfqpoint{2.756645in}{1.469554in}}%
\pgfpathlineto{\pgfqpoint{2.750879in}{1.474563in}}%
\pgfpathlineto{\pgfqpoint{2.753121in}{1.482442in}}%
\pgfpathlineto{\pgfqpoint{2.749558in}{1.490026in}}%
\pgfpathlineto{\pgfqpoint{2.749311in}{1.495307in}}%
\pgfpathlineto{\pgfqpoint{2.759273in}{1.493756in}}%
\pgfpathlineto{\pgfqpoint{2.766923in}{1.496257in}}%
\pgfpathlineto{\pgfqpoint{2.774908in}{1.505824in}}%
\pgfpathlineto{\pgfqpoint{2.756305in}{1.557859in}}%
\pgfpathlineto{\pgfqpoint{2.763396in}{1.560589in}}%
\pgfpathlineto{\pgfqpoint{2.793875in}{1.572201in}}%
\pgfpathlineto{\pgfqpoint{2.794553in}{1.564410in}}%
\pgfpathlineto{\pgfqpoint{2.803917in}{1.557332in}}%
\pgfpathlineto{\pgfqpoint{2.807383in}{1.541913in}}%
\pgfpathlineto{\pgfqpoint{2.813736in}{1.511035in}}%
\pgfpathlineto{\pgfqpoint{2.815796in}{1.505313in}}%
\pgfpathlineto{\pgfqpoint{2.814539in}{1.460221in}}%
\pgfpathlineto{\pgfqpoint{2.815641in}{1.458005in}}%
\pgfpathclose%
\pgfusepath{fill}%
\end{pgfscope}%
\begin{pgfscope}%
\pgfpathrectangle{\pgfqpoint{0.100000in}{0.100000in}}{\pgfqpoint{3.007045in}{1.925000in}}%
\pgfusepath{clip}%
\pgfsetbuttcap%
\pgfsetmiterjoin%
\definecolor{currentfill}{rgb}{0.361599,0.642737,0.816578}%
\pgfsetfillcolor{currentfill}%
\pgfsetlinewidth{0.000000pt}%
\definecolor{currentstroke}{rgb}{0.000000,0.000000,0.000000}%
\pgfsetstrokecolor{currentstroke}%
\pgfsetstrokeopacity{0.000000}%
\pgfsetdash{}{0pt}%
\pgfpathmoveto{\pgfqpoint{1.154054in}{1.574061in}}%
\pgfpathlineto{\pgfqpoint{1.115456in}{1.579920in}}%
\pgfpathlineto{\pgfqpoint{1.114575in}{1.580480in}}%
\pgfpathlineto{\pgfqpoint{1.082392in}{1.585392in}}%
\pgfpathlineto{\pgfqpoint{1.084238in}{1.596483in}}%
\pgfpathlineto{\pgfqpoint{1.072000in}{1.598524in}}%
\pgfpathlineto{\pgfqpoint{1.073953in}{1.610065in}}%
\pgfpathlineto{\pgfqpoint{1.079975in}{1.609049in}}%
\pgfpathlineto{\pgfqpoint{1.081871in}{1.620429in}}%
\pgfpathlineto{\pgfqpoint{1.094101in}{1.624217in}}%
\pgfpathlineto{\pgfqpoint{1.099750in}{1.623315in}}%
\pgfpathlineto{\pgfqpoint{1.101489in}{1.634717in}}%
\pgfpathlineto{\pgfqpoint{1.105971in}{1.645829in}}%
\pgfpathlineto{\pgfqpoint{1.109783in}{1.645289in}}%
\pgfpathlineto{\pgfqpoint{1.110938in}{1.650768in}}%
\pgfpathlineto{\pgfqpoint{1.102412in}{1.652202in}}%
\pgfpathlineto{\pgfqpoint{1.101102in}{1.658254in}}%
\pgfpathlineto{\pgfqpoint{1.102024in}{1.663968in}}%
\pgfpathlineto{\pgfqpoint{1.113428in}{1.662137in}}%
\pgfpathlineto{\pgfqpoint{1.119016in}{1.693144in}}%
\pgfpathlineto{\pgfqpoint{1.119627in}{1.696907in}}%
\pgfpathlineto{\pgfqpoint{1.153729in}{1.691500in}}%
\pgfpathlineto{\pgfqpoint{1.189998in}{1.686616in}}%
\pgfpathlineto{\pgfqpoint{1.190856in}{1.681412in}}%
\pgfpathlineto{\pgfqpoint{1.188964in}{1.669306in}}%
\pgfpathlineto{\pgfqpoint{1.192292in}{1.666923in}}%
\pgfpathlineto{\pgfqpoint{1.195568in}{1.655776in}}%
\pgfpathlineto{\pgfqpoint{1.201035in}{1.646271in}}%
\pgfpathlineto{\pgfqpoint{1.197757in}{1.637726in}}%
\pgfpathlineto{\pgfqpoint{1.190334in}{1.638786in}}%
\pgfpathlineto{\pgfqpoint{1.189792in}{1.634979in}}%
\pgfpathlineto{\pgfqpoint{1.184148in}{1.635775in}}%
\pgfpathlineto{\pgfqpoint{1.179523in}{1.630622in}}%
\pgfpathlineto{\pgfqpoint{1.171947in}{1.631734in}}%
\pgfpathlineto{\pgfqpoint{1.169858in}{1.624279in}}%
\pgfpathlineto{\pgfqpoint{1.167306in}{1.606868in}}%
\pgfpathlineto{\pgfqpoint{1.155409in}{1.604740in}}%
\pgfpathlineto{\pgfqpoint{1.140335in}{1.607233in}}%
\pgfpathlineto{\pgfqpoint{1.138319in}{1.605460in}}%
\pgfpathlineto{\pgfqpoint{1.135376in}{1.591485in}}%
\pgfpathlineto{\pgfqpoint{1.163161in}{1.587525in}}%
\pgfpathlineto{\pgfqpoint{1.160708in}{1.586055in}}%
\pgfpathlineto{\pgfqpoint{1.154054in}{1.574061in}}%
\pgfpathclose%
\pgfusepath{fill}%
\end{pgfscope}%
\begin{pgfscope}%
\pgfpathrectangle{\pgfqpoint{0.100000in}{0.100000in}}{\pgfqpoint{3.007045in}{1.925000in}}%
\pgfusepath{clip}%
\pgfsetbuttcap%
\pgfsetmiterjoin%
\definecolor{currentfill}{rgb}{0.579608,0.770196,0.873725}%
\pgfsetfillcolor{currentfill}%
\pgfsetlinewidth{0.000000pt}%
\definecolor{currentstroke}{rgb}{0.000000,0.000000,0.000000}%
\pgfsetstrokecolor{currentstroke}%
\pgfsetstrokeopacity{0.000000}%
\pgfsetdash{}{0pt}%
\pgfpathmoveto{\pgfqpoint{2.427990in}{0.752079in}}%
\pgfpathlineto{\pgfqpoint{2.415340in}{0.771423in}}%
\pgfpathlineto{\pgfqpoint{2.417835in}{0.773607in}}%
\pgfpathlineto{\pgfqpoint{2.411719in}{0.790134in}}%
\pgfpathlineto{\pgfqpoint{2.405437in}{0.788426in}}%
\pgfpathlineto{\pgfqpoint{2.403890in}{0.805755in}}%
\pgfpathlineto{\pgfqpoint{2.417379in}{0.810753in}}%
\pgfpathlineto{\pgfqpoint{2.424800in}{0.803148in}}%
\pgfpathlineto{\pgfqpoint{2.432103in}{0.811727in}}%
\pgfpathlineto{\pgfqpoint{2.441136in}{0.811210in}}%
\pgfpathlineto{\pgfqpoint{2.448640in}{0.798533in}}%
\pgfpathlineto{\pgfqpoint{2.443204in}{0.795028in}}%
\pgfpathlineto{\pgfqpoint{2.441806in}{0.790953in}}%
\pgfpathlineto{\pgfqpoint{2.437607in}{0.788428in}}%
\pgfpathlineto{\pgfqpoint{2.432968in}{0.782406in}}%
\pgfpathlineto{\pgfqpoint{2.433317in}{0.776989in}}%
\pgfpathlineto{\pgfqpoint{2.436633in}{0.771428in}}%
\pgfpathlineto{\pgfqpoint{2.441164in}{0.768340in}}%
\pgfpathlineto{\pgfqpoint{2.441826in}{0.762195in}}%
\pgfpathlineto{\pgfqpoint{2.427990in}{0.752079in}}%
\pgfpathclose%
\pgfusepath{fill}%
\end{pgfscope}%
\begin{pgfscope}%
\pgfpathrectangle{\pgfqpoint{0.100000in}{0.100000in}}{\pgfqpoint{3.007045in}{1.925000in}}%
\pgfusepath{clip}%
\pgfsetbuttcap%
\pgfsetmiterjoin%
\definecolor{currentfill}{rgb}{0.183206,0.496932,0.737516}%
\pgfsetfillcolor{currentfill}%
\pgfsetlinewidth{0.000000pt}%
\definecolor{currentstroke}{rgb}{0.000000,0.000000,0.000000}%
\pgfsetstrokecolor{currentstroke}%
\pgfsetstrokeopacity{0.000000}%
\pgfsetdash{}{0pt}%
\pgfpathmoveto{\pgfqpoint{1.263629in}{1.757459in}}%
\pgfpathlineto{\pgfqpoint{1.265560in}{1.751233in}}%
\pgfpathlineto{\pgfqpoint{1.269873in}{1.750687in}}%
\pgfpathlineto{\pgfqpoint{1.267101in}{1.728242in}}%
\pgfpathlineto{\pgfqpoint{1.263981in}{1.716912in}}%
\pgfpathlineto{\pgfqpoint{1.269690in}{1.716224in}}%
\pgfpathlineto{\pgfqpoint{1.268284in}{1.704741in}}%
\pgfpathlineto{\pgfqpoint{1.272031in}{1.704281in}}%
\pgfpathlineto{\pgfqpoint{1.269340in}{1.682158in}}%
\pgfpathlineto{\pgfqpoint{1.240737in}{1.685718in}}%
\pgfpathlineto{\pgfqpoint{1.240870in}{1.686663in}}%
\pgfpathlineto{\pgfqpoint{1.187986in}{1.693328in}}%
\pgfpathlineto{\pgfqpoint{1.186086in}{1.695738in}}%
\pgfpathlineto{\pgfqpoint{1.189559in}{1.703789in}}%
\pgfpathlineto{\pgfqpoint{1.188167in}{1.711464in}}%
\pgfpathlineto{\pgfqpoint{1.188345in}{1.720102in}}%
\pgfpathlineto{\pgfqpoint{1.192875in}{1.732905in}}%
\pgfpathlineto{\pgfqpoint{1.195585in}{1.736909in}}%
\pgfpathlineto{\pgfqpoint{1.201819in}{1.739713in}}%
\pgfpathlineto{\pgfqpoint{1.206785in}{1.744306in}}%
\pgfpathlineto{\pgfqpoint{1.214896in}{1.741346in}}%
\pgfpathlineto{\pgfqpoint{1.217097in}{1.745725in}}%
\pgfpathlineto{\pgfqpoint{1.224257in}{1.742711in}}%
\pgfpathlineto{\pgfqpoint{1.240243in}{1.741597in}}%
\pgfpathlineto{\pgfqpoint{1.241819in}{1.745330in}}%
\pgfpathlineto{\pgfqpoint{1.246943in}{1.744290in}}%
\pgfpathlineto{\pgfqpoint{1.255859in}{1.748834in}}%
\pgfpathlineto{\pgfqpoint{1.260272in}{1.755961in}}%
\pgfpathlineto{\pgfqpoint{1.263629in}{1.757459in}}%
\pgfpathclose%
\pgfusepath{fill}%
\end{pgfscope}%
\begin{pgfscope}%
\pgfpathrectangle{\pgfqpoint{0.100000in}{0.100000in}}{\pgfqpoint{3.007045in}{1.925000in}}%
\pgfusepath{clip}%
\pgfsetbuttcap%
\pgfsetmiterjoin%
\definecolor{currentfill}{rgb}{0.466667,0.708189,0.849366}%
\pgfsetfillcolor{currentfill}%
\pgfsetlinewidth{0.000000pt}%
\definecolor{currentstroke}{rgb}{0.000000,0.000000,0.000000}%
\pgfsetstrokecolor{currentstroke}%
\pgfsetstrokeopacity{0.000000}%
\pgfsetdash{}{0pt}%
\pgfpathmoveto{\pgfqpoint{2.644431in}{1.224390in}}%
\pgfpathlineto{\pgfqpoint{2.614371in}{1.241096in}}%
\pgfpathlineto{\pgfqpoint{2.607688in}{1.243264in}}%
\pgfpathlineto{\pgfqpoint{2.608755in}{1.249542in}}%
\pgfpathlineto{\pgfqpoint{2.613340in}{1.256035in}}%
\pgfpathlineto{\pgfqpoint{2.609634in}{1.257414in}}%
\pgfpathlineto{\pgfqpoint{2.612460in}{1.264832in}}%
\pgfpathlineto{\pgfqpoint{2.614370in}{1.276431in}}%
\pgfpathlineto{\pgfqpoint{2.613885in}{1.282571in}}%
\pgfpathlineto{\pgfqpoint{2.616493in}{1.288353in}}%
\pgfpathlineto{\pgfqpoint{2.627614in}{1.286206in}}%
\pgfpathlineto{\pgfqpoint{2.631148in}{1.284175in}}%
\pgfpathlineto{\pgfqpoint{2.636603in}{1.298784in}}%
\pgfpathlineto{\pgfqpoint{2.638827in}{1.299653in}}%
\pgfpathlineto{\pgfqpoint{2.652060in}{1.280358in}}%
\pgfpathlineto{\pgfqpoint{2.655502in}{1.265959in}}%
\pgfpathlineto{\pgfqpoint{2.651264in}{1.258406in}}%
\pgfpathlineto{\pgfqpoint{2.649600in}{1.238259in}}%
\pgfpathlineto{\pgfqpoint{2.646870in}{1.236936in}}%
\pgfpathlineto{\pgfqpoint{2.644431in}{1.224390in}}%
\pgfpathclose%
\pgfusepath{fill}%
\end{pgfscope}%
\begin{pgfscope}%
\pgfpathrectangle{\pgfqpoint{0.100000in}{0.100000in}}{\pgfqpoint{3.007045in}{1.925000in}}%
\pgfusepath{clip}%
\pgfsetbuttcap%
\pgfsetmiterjoin%
\definecolor{currentfill}{rgb}{0.093272,0.396878,0.673664}%
\pgfsetfillcolor{currentfill}%
\pgfsetlinewidth{0.000000pt}%
\definecolor{currentstroke}{rgb}{0.000000,0.000000,0.000000}%
\pgfsetstrokecolor{currentstroke}%
\pgfsetstrokeopacity{0.000000}%
\pgfsetdash{}{0pt}%
\pgfpathmoveto{\pgfqpoint{0.677092in}{0.410736in}}%
\pgfpathlineto{\pgfqpoint{0.679449in}{0.413309in}}%
\pgfpathlineto{\pgfqpoint{0.683920in}{0.412319in}}%
\pgfpathlineto{\pgfqpoint{0.681471in}{0.410739in}}%
\pgfpathlineto{\pgfqpoint{0.677092in}{0.410736in}}%
\pgfpathclose%
\pgfusepath{fill}%
\end{pgfscope}%
\begin{pgfscope}%
\pgfpathrectangle{\pgfqpoint{0.100000in}{0.100000in}}{\pgfqpoint{3.007045in}{1.925000in}}%
\pgfusepath{clip}%
\pgfsetbuttcap%
\pgfsetmiterjoin%
\definecolor{currentfill}{rgb}{0.093272,0.396878,0.673664}%
\pgfsetfillcolor{currentfill}%
\pgfsetlinewidth{0.000000pt}%
\definecolor{currentstroke}{rgb}{0.000000,0.000000,0.000000}%
\pgfsetstrokecolor{currentstroke}%
\pgfsetstrokeopacity{0.000000}%
\pgfsetdash{}{0pt}%
\pgfpathmoveto{\pgfqpoint{0.784950in}{0.396276in}}%
\pgfpathlineto{\pgfqpoint{0.784096in}{0.394320in}}%
\pgfpathlineto{\pgfqpoint{0.778986in}{0.387487in}}%
\pgfpathlineto{\pgfqpoint{0.777700in}{0.388433in}}%
\pgfpathlineto{\pgfqpoint{0.772579in}{0.381674in}}%
\pgfpathlineto{\pgfqpoint{0.771354in}{0.382565in}}%
\pgfpathlineto{\pgfqpoint{0.766473in}{0.376050in}}%
\pgfpathlineto{\pgfqpoint{0.765233in}{0.376990in}}%
\pgfpathlineto{\pgfqpoint{0.763938in}{0.375291in}}%
\pgfpathlineto{\pgfqpoint{0.762258in}{0.376547in}}%
\pgfpathlineto{\pgfqpoint{0.759707in}{0.373154in}}%
\pgfpathlineto{\pgfqpoint{0.758045in}{0.374400in}}%
\pgfpathlineto{\pgfqpoint{0.756787in}{0.372685in}}%
\pgfpathlineto{\pgfqpoint{0.753894in}{0.374816in}}%
\pgfpathlineto{\pgfqpoint{0.751397in}{0.371451in}}%
\pgfpathlineto{\pgfqpoint{0.748076in}{0.373959in}}%
\pgfpathlineto{\pgfqpoint{0.745512in}{0.370591in}}%
\pgfpathlineto{\pgfqpoint{0.746018in}{0.370169in}}%
\pgfpathlineto{\pgfqpoint{0.741011in}{0.363689in}}%
\pgfpathlineto{\pgfqpoint{0.740274in}{0.361603in}}%
\pgfpathlineto{\pgfqpoint{0.743549in}{0.359090in}}%
\pgfpathlineto{\pgfqpoint{0.740865in}{0.355528in}}%
\pgfpathlineto{\pgfqpoint{0.738983in}{0.355722in}}%
\pgfpathlineto{\pgfqpoint{0.737813in}{0.354884in}}%
\pgfpathlineto{\pgfqpoint{0.734357in}{0.355834in}}%
\pgfpathlineto{\pgfqpoint{0.731400in}{0.357708in}}%
\pgfpathlineto{\pgfqpoint{0.730107in}{0.357059in}}%
\pgfpathlineto{\pgfqpoint{0.728487in}{0.358327in}}%
\pgfpathlineto{\pgfqpoint{0.726759in}{0.358371in}}%
\pgfpathlineto{\pgfqpoint{0.722429in}{0.352828in}}%
\pgfpathlineto{\pgfqpoint{0.719341in}{0.353168in}}%
\pgfpathlineto{\pgfqpoint{0.717812in}{0.354377in}}%
\pgfpathlineto{\pgfqpoint{0.715201in}{0.352581in}}%
\pgfpathlineto{\pgfqpoint{0.712047in}{0.351741in}}%
\pgfpathlineto{\pgfqpoint{0.711910in}{0.351797in}}%
\pgfpathlineto{\pgfqpoint{0.712359in}{0.352736in}}%
\pgfpathlineto{\pgfqpoint{0.708693in}{0.353375in}}%
\pgfpathlineto{\pgfqpoint{0.707132in}{0.354637in}}%
\pgfpathlineto{\pgfqpoint{0.704495in}{0.354085in}}%
\pgfpathlineto{\pgfqpoint{0.703476in}{0.352730in}}%
\pgfpathlineto{\pgfqpoint{0.701818in}{0.353160in}}%
\pgfpathlineto{\pgfqpoint{0.698993in}{0.352386in}}%
\pgfpathlineto{\pgfqpoint{0.697922in}{0.351049in}}%
\pgfpathlineto{\pgfqpoint{0.701239in}{0.348319in}}%
\pgfpathlineto{\pgfqpoint{0.697491in}{0.346230in}}%
\pgfpathlineto{\pgfqpoint{0.696501in}{0.348097in}}%
\pgfpathlineto{\pgfqpoint{0.693475in}{0.346785in}}%
\pgfpathlineto{\pgfqpoint{0.692417in}{0.347973in}}%
\pgfpathlineto{\pgfqpoint{0.683563in}{0.349485in}}%
\pgfpathlineto{\pgfqpoint{0.682454in}{0.347098in}}%
\pgfpathlineto{\pgfqpoint{0.680426in}{0.348551in}}%
\pgfpathlineto{\pgfqpoint{0.680359in}{0.350534in}}%
\pgfpathlineto{\pgfqpoint{0.678074in}{0.351002in}}%
\pgfpathlineto{\pgfqpoint{0.677768in}{0.349141in}}%
\pgfpathlineto{\pgfqpoint{0.676340in}{0.347791in}}%
\pgfpathlineto{\pgfqpoint{0.672925in}{0.348867in}}%
\pgfpathlineto{\pgfqpoint{0.670574in}{0.351024in}}%
\pgfpathlineto{\pgfqpoint{0.667546in}{0.349762in}}%
\pgfpathlineto{\pgfqpoint{0.669799in}{0.346596in}}%
\pgfpathlineto{\pgfqpoint{0.664976in}{0.345492in}}%
\pgfpathlineto{\pgfqpoint{0.662946in}{0.343281in}}%
\pgfpathlineto{\pgfqpoint{0.662231in}{0.345505in}}%
\pgfpathlineto{\pgfqpoint{0.659923in}{0.345201in}}%
\pgfpathlineto{\pgfqpoint{0.658186in}{0.346446in}}%
\pgfpathlineto{\pgfqpoint{0.656689in}{0.346479in}}%
\pgfpathlineto{\pgfqpoint{0.652000in}{0.348708in}}%
\pgfpathlineto{\pgfqpoint{0.649972in}{0.347731in}}%
\pgfpathlineto{\pgfqpoint{0.648426in}{0.347772in}}%
\pgfpathlineto{\pgfqpoint{0.646196in}{0.346313in}}%
\pgfpathlineto{\pgfqpoint{0.650028in}{0.350774in}}%
\pgfpathlineto{\pgfqpoint{0.647002in}{0.353387in}}%
\pgfpathlineto{\pgfqpoint{0.649075in}{0.354518in}}%
\pgfpathlineto{\pgfqpoint{0.654252in}{0.360591in}}%
\pgfpathlineto{\pgfqpoint{0.653338in}{0.361411in}}%
\pgfpathlineto{\pgfqpoint{0.655995in}{0.364392in}}%
\pgfpathlineto{\pgfqpoint{0.659049in}{0.361833in}}%
\pgfpathlineto{\pgfqpoint{0.660386in}{0.363372in}}%
\pgfpathlineto{\pgfqpoint{0.663515in}{0.360688in}}%
\pgfpathlineto{\pgfqpoint{0.664908in}{0.362318in}}%
\pgfpathlineto{\pgfqpoint{0.667310in}{0.360198in}}%
\pgfpathlineto{\pgfqpoint{0.668605in}{0.361721in}}%
\pgfpathlineto{\pgfqpoint{0.670704in}{0.361448in}}%
\pgfpathlineto{\pgfqpoint{0.672992in}{0.359286in}}%
\pgfpathlineto{\pgfqpoint{0.675516in}{0.363060in}}%
\pgfpathlineto{\pgfqpoint{0.680318in}{0.364924in}}%
\pgfpathlineto{\pgfqpoint{0.685284in}{0.365748in}}%
\pgfpathlineto{\pgfqpoint{0.687937in}{0.364755in}}%
\pgfpathlineto{\pgfqpoint{0.690707in}{0.366122in}}%
\pgfpathlineto{\pgfqpoint{0.694569in}{0.365827in}}%
\pgfpathlineto{\pgfqpoint{0.695772in}{0.367709in}}%
\pgfpathlineto{\pgfqpoint{0.704688in}{0.375601in}}%
\pgfpathlineto{\pgfqpoint{0.706614in}{0.375694in}}%
\pgfpathlineto{\pgfqpoint{0.706494in}{0.377302in}}%
\pgfpathlineto{\pgfqpoint{0.709397in}{0.380342in}}%
\pgfpathlineto{\pgfqpoint{0.713956in}{0.381377in}}%
\pgfpathlineto{\pgfqpoint{0.723393in}{0.373987in}}%
\pgfpathlineto{\pgfqpoint{0.727658in}{0.379320in}}%
\pgfpathlineto{\pgfqpoint{0.720329in}{0.385044in}}%
\pgfpathlineto{\pgfqpoint{0.707689in}{0.387813in}}%
\pgfpathlineto{\pgfqpoint{0.705303in}{0.388793in}}%
\pgfpathlineto{\pgfqpoint{0.704791in}{0.390712in}}%
\pgfpathlineto{\pgfqpoint{0.705523in}{0.393336in}}%
\pgfpathlineto{\pgfqpoint{0.704478in}{0.394497in}}%
\pgfpathlineto{\pgfqpoint{0.707497in}{0.397391in}}%
\pgfpathlineto{\pgfqpoint{0.706468in}{0.397884in}}%
\pgfpathlineto{\pgfqpoint{0.700476in}{0.396199in}}%
\pgfpathlineto{\pgfqpoint{0.698526in}{0.391552in}}%
\pgfpathlineto{\pgfqpoint{0.697132in}{0.389984in}}%
\pgfpathlineto{\pgfqpoint{0.695255in}{0.390285in}}%
\pgfpathlineto{\pgfqpoint{0.694284in}{0.392577in}}%
\pgfpathlineto{\pgfqpoint{0.695108in}{0.393300in}}%
\pgfpathlineto{\pgfqpoint{0.695145in}{0.396106in}}%
\pgfpathlineto{\pgfqpoint{0.696329in}{0.402560in}}%
\pgfpathlineto{\pgfqpoint{0.695632in}{0.403858in}}%
\pgfpathlineto{\pgfqpoint{0.697277in}{0.406002in}}%
\pgfpathlineto{\pgfqpoint{0.695522in}{0.406874in}}%
\pgfpathlineto{\pgfqpoint{0.694538in}{0.405951in}}%
\pgfpathlineto{\pgfqpoint{0.691532in}{0.405166in}}%
\pgfpathlineto{\pgfqpoint{0.691789in}{0.407114in}}%
\pgfpathlineto{\pgfqpoint{0.690598in}{0.411229in}}%
\pgfpathlineto{\pgfqpoint{0.692288in}{0.413752in}}%
\pgfpathlineto{\pgfqpoint{0.690110in}{0.414243in}}%
\pgfpathlineto{\pgfqpoint{0.685710in}{0.414422in}}%
\pgfpathlineto{\pgfqpoint{0.681584in}{0.415593in}}%
\pgfpathlineto{\pgfqpoint{0.684613in}{0.418474in}}%
\pgfpathlineto{\pgfqpoint{0.686127in}{0.417159in}}%
\pgfpathlineto{\pgfqpoint{0.687384in}{0.418587in}}%
\pgfpathlineto{\pgfqpoint{0.688938in}{0.417253in}}%
\pgfpathlineto{\pgfqpoint{0.691710in}{0.420379in}}%
\pgfpathlineto{\pgfqpoint{0.693623in}{0.421487in}}%
\pgfpathlineto{\pgfqpoint{0.695145in}{0.420096in}}%
\pgfpathlineto{\pgfqpoint{0.697072in}{0.421245in}}%
\pgfpathlineto{\pgfqpoint{0.699846in}{0.424401in}}%
\pgfpathlineto{\pgfqpoint{0.701434in}{0.423011in}}%
\pgfpathlineto{\pgfqpoint{0.704186in}{0.426137in}}%
\pgfpathlineto{\pgfqpoint{0.706317in}{0.424353in}}%
\pgfpathlineto{\pgfqpoint{0.709062in}{0.427491in}}%
\pgfpathlineto{\pgfqpoint{0.710582in}{0.426191in}}%
\pgfpathlineto{\pgfqpoint{0.713230in}{0.429162in}}%
\pgfpathlineto{\pgfqpoint{0.715184in}{0.430242in}}%
\pgfpathlineto{\pgfqpoint{0.716770in}{0.428913in}}%
\pgfpathlineto{\pgfqpoint{0.719488in}{0.432032in}}%
\pgfpathlineto{\pgfqpoint{0.721076in}{0.430661in}}%
\pgfpathlineto{\pgfqpoint{0.722451in}{0.432254in}}%
\pgfpathlineto{\pgfqpoint{0.724234in}{0.430706in}}%
\pgfpathlineto{\pgfqpoint{0.726974in}{0.433888in}}%
\pgfpathlineto{\pgfqpoint{0.728779in}{0.435116in}}%
\pgfpathlineto{\pgfqpoint{0.730325in}{0.433863in}}%
\pgfpathlineto{\pgfqpoint{0.731767in}{0.435552in}}%
\pgfpathlineto{\pgfqpoint{0.733009in}{0.434561in}}%
\pgfpathlineto{\pgfqpoint{0.734378in}{0.436156in}}%
\pgfpathlineto{\pgfqpoint{0.743447in}{0.428714in}}%
\pgfpathlineto{\pgfqpoint{0.742145in}{0.427074in}}%
\pgfpathlineto{\pgfqpoint{0.743807in}{0.425686in}}%
\pgfpathlineto{\pgfqpoint{0.745138in}{0.427348in}}%
\pgfpathlineto{\pgfqpoint{0.746839in}{0.426097in}}%
\pgfpathlineto{\pgfqpoint{0.748296in}{0.427861in}}%
\pgfpathlineto{\pgfqpoint{0.751507in}{0.425233in}}%
\pgfpathlineto{\pgfqpoint{0.750064in}{0.423440in}}%
\pgfpathlineto{\pgfqpoint{0.761232in}{0.414538in}}%
\pgfpathlineto{\pgfqpoint{0.775612in}{0.403305in}}%
\pgfpathlineto{\pgfqpoint{0.784950in}{0.396276in}}%
\pgfpathclose%
\pgfusepath{fill}%
\end{pgfscope}%
\begin{pgfscope}%
\pgfpathrectangle{\pgfqpoint{0.100000in}{0.100000in}}{\pgfqpoint{3.007045in}{1.925000in}}%
\pgfusepath{clip}%
\pgfsetbuttcap%
\pgfsetmiterjoin%
\definecolor{currentfill}{rgb}{0.093272,0.396878,0.673664}%
\pgfsetfillcolor{currentfill}%
\pgfsetlinewidth{0.000000pt}%
\definecolor{currentstroke}{rgb}{0.000000,0.000000,0.000000}%
\pgfsetstrokecolor{currentstroke}%
\pgfsetstrokeopacity{0.000000}%
\pgfsetdash{}{0pt}%
\pgfpathmoveto{\pgfqpoint{0.677618in}{0.416243in}}%
\pgfpathlineto{\pgfqpoint{0.678710in}{0.418051in}}%
\pgfpathlineto{\pgfqpoint{0.681431in}{0.415610in}}%
\pgfpathlineto{\pgfqpoint{0.677618in}{0.416243in}}%
\pgfpathclose%
\pgfusepath{fill}%
\end{pgfscope}%
\begin{pgfscope}%
\pgfpathrectangle{\pgfqpoint{0.100000in}{0.100000in}}{\pgfqpoint{3.007045in}{1.925000in}}%
\pgfusepath{clip}%
\pgfsetbuttcap%
\pgfsetmiterjoin%
\definecolor{currentfill}{rgb}{0.735871,0.841569,0.923045}%
\pgfsetfillcolor{currentfill}%
\pgfsetlinewidth{0.000000pt}%
\definecolor{currentstroke}{rgb}{0.000000,0.000000,0.000000}%
\pgfsetstrokecolor{currentstroke}%
\pgfsetstrokeopacity{0.000000}%
\pgfsetdash{}{0pt}%
\pgfpathmoveto{\pgfqpoint{0.957302in}{1.649942in}}%
\pgfpathlineto{\pgfqpoint{0.955356in}{1.640352in}}%
\pgfpathlineto{\pgfqpoint{0.959812in}{1.636263in}}%
\pgfpathlineto{\pgfqpoint{0.963238in}{1.636470in}}%
\pgfpathlineto{\pgfqpoint{0.964235in}{1.629990in}}%
\pgfpathlineto{\pgfqpoint{0.960968in}{1.612983in}}%
\pgfpathlineto{\pgfqpoint{0.966618in}{1.611886in}}%
\pgfpathlineto{\pgfqpoint{0.965491in}{1.606172in}}%
\pgfpathlineto{\pgfqpoint{0.972074in}{1.604894in}}%
\pgfpathlineto{\pgfqpoint{0.971016in}{1.593385in}}%
\pgfpathlineto{\pgfqpoint{0.975154in}{1.589701in}}%
\pgfpathlineto{\pgfqpoint{0.988238in}{1.587058in}}%
\pgfpathlineto{\pgfqpoint{0.987365in}{1.582354in}}%
\pgfpathlineto{\pgfqpoint{1.001653in}{1.579735in}}%
\pgfpathlineto{\pgfqpoint{0.998004in}{1.571906in}}%
\pgfpathlineto{\pgfqpoint{0.989150in}{1.572014in}}%
\pgfpathlineto{\pgfqpoint{0.983095in}{1.569716in}}%
\pgfpathlineto{\pgfqpoint{0.981565in}{1.573774in}}%
\pgfpathlineto{\pgfqpoint{0.970525in}{1.573027in}}%
\pgfpathlineto{\pgfqpoint{0.962156in}{1.577626in}}%
\pgfpathlineto{\pgfqpoint{0.958428in}{1.575728in}}%
\pgfpathlineto{\pgfqpoint{0.956111in}{1.570703in}}%
\pgfpathlineto{\pgfqpoint{0.952519in}{1.573584in}}%
\pgfpathlineto{\pgfqpoint{0.938309in}{1.576703in}}%
\pgfpathlineto{\pgfqpoint{0.935411in}{1.569991in}}%
\pgfpathlineto{\pgfqpoint{0.933731in}{1.569068in}}%
\pgfpathlineto{\pgfqpoint{0.927577in}{1.576545in}}%
\pgfpathlineto{\pgfqpoint{0.928393in}{1.581446in}}%
\pgfpathlineto{\pgfqpoint{0.926655in}{1.595511in}}%
\pgfpathlineto{\pgfqpoint{0.923418in}{1.601071in}}%
\pgfpathlineto{\pgfqpoint{0.916159in}{1.603777in}}%
\pgfpathlineto{\pgfqpoint{0.912228in}{1.607641in}}%
\pgfpathlineto{\pgfqpoint{0.914945in}{1.620213in}}%
\pgfpathlineto{\pgfqpoint{0.910956in}{1.624666in}}%
\pgfpathlineto{\pgfqpoint{0.905525in}{1.640271in}}%
\pgfpathlineto{\pgfqpoint{0.906555in}{1.644348in}}%
\pgfpathlineto{\pgfqpoint{0.903714in}{1.653787in}}%
\pgfpathlineto{\pgfqpoint{0.906088in}{1.658804in}}%
\pgfpathlineto{\pgfqpoint{0.901414in}{1.666132in}}%
\pgfpathlineto{\pgfqpoint{0.908010in}{1.668930in}}%
\pgfpathlineto{\pgfqpoint{0.914135in}{1.673457in}}%
\pgfpathlineto{\pgfqpoint{0.921890in}{1.674121in}}%
\pgfpathlineto{\pgfqpoint{0.923792in}{1.677916in}}%
\pgfpathlineto{\pgfqpoint{0.926926in}{1.665686in}}%
\pgfpathlineto{\pgfqpoint{0.936043in}{1.670714in}}%
\pgfpathlineto{\pgfqpoint{0.941856in}{1.670351in}}%
\pgfpathlineto{\pgfqpoint{0.944841in}{1.665499in}}%
\pgfpathlineto{\pgfqpoint{0.955593in}{1.658524in}}%
\pgfpathlineto{\pgfqpoint{0.957302in}{1.649942in}}%
\pgfpathclose%
\pgfusepath{fill}%
\end{pgfscope}%
\begin{pgfscope}%
\pgfpathrectangle{\pgfqpoint{0.100000in}{0.100000in}}{\pgfqpoint{3.007045in}{1.925000in}}%
\pgfusepath{clip}%
\pgfsetbuttcap%
\pgfsetmiterjoin%
\definecolor{currentfill}{rgb}{0.485490,0.718524,0.853426}%
\pgfsetfillcolor{currentfill}%
\pgfsetlinewidth{0.000000pt}%
\definecolor{currentstroke}{rgb}{0.000000,0.000000,0.000000}%
\pgfsetstrokecolor{currentstroke}%
\pgfsetstrokeopacity{0.000000}%
\pgfsetdash{}{0pt}%
\pgfpathmoveto{\pgfqpoint{0.741011in}{1.724091in}}%
\pgfpathlineto{\pgfqpoint{0.771689in}{1.716185in}}%
\pgfpathlineto{\pgfqpoint{0.774980in}{1.703762in}}%
\pgfpathlineto{\pgfqpoint{0.778737in}{1.702487in}}%
\pgfpathlineto{\pgfqpoint{0.784656in}{1.695189in}}%
\pgfpathlineto{\pgfqpoint{0.786067in}{1.686768in}}%
\pgfpathlineto{\pgfqpoint{0.780091in}{1.680597in}}%
\pgfpathlineto{\pgfqpoint{0.770539in}{1.666797in}}%
\pgfpathlineto{\pgfqpoint{0.766636in}{1.659171in}}%
\pgfpathlineto{\pgfqpoint{0.763180in}{1.655715in}}%
\pgfpathlineto{\pgfqpoint{0.741293in}{1.661247in}}%
\pgfpathlineto{\pgfqpoint{0.742633in}{1.666720in}}%
\pgfpathlineto{\pgfqpoint{0.733120in}{1.669073in}}%
\pgfpathlineto{\pgfqpoint{0.732751in}{1.676168in}}%
\pgfpathlineto{\pgfqpoint{0.728377in}{1.682025in}}%
\pgfpathlineto{\pgfqpoint{0.727519in}{1.694386in}}%
\pgfpathlineto{\pgfqpoint{0.729208in}{1.700974in}}%
\pgfpathlineto{\pgfqpoint{0.727679in}{1.706314in}}%
\pgfpathlineto{\pgfqpoint{0.732306in}{1.716868in}}%
\pgfpathlineto{\pgfqpoint{0.722204in}{1.719436in}}%
\pgfpathlineto{\pgfqpoint{0.724313in}{1.728496in}}%
\pgfpathlineto{\pgfqpoint{0.741011in}{1.724091in}}%
\pgfpathclose%
\pgfusepath{fill}%
\end{pgfscope}%
\begin{pgfscope}%
\pgfpathrectangle{\pgfqpoint{0.100000in}{0.100000in}}{\pgfqpoint{3.007045in}{1.925000in}}%
\pgfusepath{clip}%
\pgfsetbuttcap%
\pgfsetmiterjoin%
\definecolor{currentfill}{rgb}{0.351511,0.635848,0.812641}%
\pgfsetfillcolor{currentfill}%
\pgfsetlinewidth{0.000000pt}%
\definecolor{currentstroke}{rgb}{0.000000,0.000000,0.000000}%
\pgfsetstrokecolor{currentstroke}%
\pgfsetstrokeopacity{0.000000}%
\pgfsetdash{}{0pt}%
\pgfpathmoveto{\pgfqpoint{1.439278in}{1.125169in}}%
\pgfpathlineto{\pgfqpoint{1.410528in}{1.127149in}}%
\pgfpathlineto{\pgfqpoint{1.414739in}{1.184340in}}%
\pgfpathlineto{\pgfqpoint{1.446935in}{1.181960in}}%
\pgfpathlineto{\pgfqpoint{1.448195in}{1.181868in}}%
\pgfpathlineto{\pgfqpoint{1.446054in}{1.153418in}}%
\pgfpathlineto{\pgfqpoint{1.441646in}{1.153706in}}%
\pgfpathlineto{\pgfqpoint{1.439278in}{1.125169in}}%
\pgfpathclose%
\pgfusepath{fill}%
\end{pgfscope}%
\begin{pgfscope}%
\pgfpathrectangle{\pgfqpoint{0.100000in}{0.100000in}}{\pgfqpoint{3.007045in}{1.925000in}}%
\pgfusepath{clip}%
\pgfsetbuttcap%
\pgfsetmiterjoin%
\definecolor{currentfill}{rgb}{0.223806,0.537532,0.758431}%
\pgfsetfillcolor{currentfill}%
\pgfsetlinewidth{0.000000pt}%
\definecolor{currentstroke}{rgb}{0.000000,0.000000,0.000000}%
\pgfsetstrokecolor{currentstroke}%
\pgfsetstrokeopacity{0.000000}%
\pgfsetdash{}{0pt}%
\pgfpathmoveto{\pgfqpoint{1.712393in}{0.726751in}}%
\pgfpathlineto{\pgfqpoint{1.712066in}{0.694873in}}%
\pgfpathlineto{\pgfqpoint{1.694814in}{0.693189in}}%
\pgfpathlineto{\pgfqpoint{1.665733in}{0.676821in}}%
\pgfpathlineto{\pgfqpoint{1.663204in}{0.675422in}}%
\pgfpathlineto{\pgfqpoint{1.655346in}{0.689754in}}%
\pgfpathlineto{\pgfqpoint{1.655743in}{0.708579in}}%
\pgfpathlineto{\pgfqpoint{1.658448in}{0.708465in}}%
\pgfpathlineto{\pgfqpoint{1.659429in}{0.737338in}}%
\pgfpathlineto{\pgfqpoint{1.639106in}{0.738203in}}%
\pgfpathlineto{\pgfqpoint{1.640734in}{0.767480in}}%
\pgfpathlineto{\pgfqpoint{1.634992in}{0.767696in}}%
\pgfpathlineto{\pgfqpoint{1.636022in}{0.799536in}}%
\pgfpathlineto{\pgfqpoint{1.638052in}{0.793421in}}%
\pgfpathlineto{\pgfqpoint{1.641958in}{0.793032in}}%
\pgfpathlineto{\pgfqpoint{1.649125in}{0.798383in}}%
\pgfpathlineto{\pgfqpoint{1.652538in}{0.797937in}}%
\pgfpathlineto{\pgfqpoint{1.651130in}{0.791612in}}%
\pgfpathlineto{\pgfqpoint{1.653822in}{0.786416in}}%
\pgfpathlineto{\pgfqpoint{1.656967in}{0.786599in}}%
\pgfpathlineto{\pgfqpoint{1.665670in}{0.801052in}}%
\pgfpathlineto{\pgfqpoint{1.664958in}{0.765812in}}%
\pgfpathlineto{\pgfqpoint{1.670614in}{0.764978in}}%
\pgfpathlineto{\pgfqpoint{1.695830in}{0.764016in}}%
\pgfpathlineto{\pgfqpoint{1.700709in}{0.760907in}}%
\pgfpathlineto{\pgfqpoint{1.700123in}{0.727126in}}%
\pgfpathlineto{\pgfqpoint{1.712393in}{0.726751in}}%
\pgfpathclose%
\pgfusepath{fill}%
\end{pgfscope}%
\begin{pgfscope}%
\pgfpathrectangle{\pgfqpoint{0.100000in}{0.100000in}}{\pgfqpoint{3.007045in}{1.925000in}}%
\pgfusepath{clip}%
\pgfsetbuttcap%
\pgfsetmiterjoin%
\definecolor{currentfill}{rgb}{0.280892,0.587620,0.785083}%
\pgfsetfillcolor{currentfill}%
\pgfsetlinewidth{0.000000pt}%
\definecolor{currentstroke}{rgb}{0.000000,0.000000,0.000000}%
\pgfsetstrokecolor{currentstroke}%
\pgfsetstrokeopacity{0.000000}%
\pgfsetdash{}{0pt}%
\pgfpathmoveto{\pgfqpoint{1.665670in}{0.801052in}}%
\pgfpathlineto{\pgfqpoint{1.656967in}{0.786599in}}%
\pgfpathlineto{\pgfqpoint{1.653822in}{0.786416in}}%
\pgfpathlineto{\pgfqpoint{1.651130in}{0.791612in}}%
\pgfpathlineto{\pgfqpoint{1.652538in}{0.797937in}}%
\pgfpathlineto{\pgfqpoint{1.649125in}{0.798383in}}%
\pgfpathlineto{\pgfqpoint{1.641958in}{0.793032in}}%
\pgfpathlineto{\pgfqpoint{1.638052in}{0.793421in}}%
\pgfpathlineto{\pgfqpoint{1.636022in}{0.799536in}}%
\pgfpathlineto{\pgfqpoint{1.631785in}{0.798613in}}%
\pgfpathlineto{\pgfqpoint{1.633138in}{0.850249in}}%
\pgfpathlineto{\pgfqpoint{1.627401in}{0.850427in}}%
\pgfpathlineto{\pgfqpoint{1.627761in}{0.861901in}}%
\pgfpathlineto{\pgfqpoint{1.667600in}{0.860848in}}%
\pgfpathlineto{\pgfqpoint{1.667244in}{0.843587in}}%
\pgfpathlineto{\pgfqpoint{1.672960in}{0.843473in}}%
\pgfpathlineto{\pgfqpoint{1.672844in}{0.837714in}}%
\pgfpathlineto{\pgfqpoint{1.689906in}{0.837361in}}%
\pgfpathlineto{\pgfqpoint{1.689815in}{0.831695in}}%
\pgfpathlineto{\pgfqpoint{1.695616in}{0.831530in}}%
\pgfpathlineto{\pgfqpoint{1.695332in}{0.814267in}}%
\pgfpathlineto{\pgfqpoint{1.683354in}{0.804224in}}%
\pgfpathlineto{\pgfqpoint{1.678257in}{0.793022in}}%
\pgfpathlineto{\pgfqpoint{1.668937in}{0.795613in}}%
\pgfpathlineto{\pgfqpoint{1.665670in}{0.801052in}}%
\pgfpathclose%
\pgfusepath{fill}%
\end{pgfscope}%
\begin{pgfscope}%
\pgfpathrectangle{\pgfqpoint{0.100000in}{0.100000in}}{\pgfqpoint{3.007045in}{1.925000in}}%
\pgfusepath{clip}%
\pgfsetbuttcap%
\pgfsetmiterjoin%
\definecolor{currentfill}{rgb}{0.485490,0.718524,0.853426}%
\pgfsetfillcolor{currentfill}%
\pgfsetlinewidth{0.000000pt}%
\definecolor{currentstroke}{rgb}{0.000000,0.000000,0.000000}%
\pgfsetstrokecolor{currentstroke}%
\pgfsetstrokeopacity{0.000000}%
\pgfsetdash{}{0pt}%
\pgfpathmoveto{\pgfqpoint{1.146247in}{0.808107in}}%
\pgfpathlineto{\pgfqpoint{1.147002in}{0.802056in}}%
\pgfpathlineto{\pgfqpoint{1.144512in}{0.779815in}}%
\pgfpathlineto{\pgfqpoint{1.112390in}{0.768586in}}%
\pgfpathlineto{\pgfqpoint{1.112021in}{0.765725in}}%
\pgfpathlineto{\pgfqpoint{1.089302in}{0.768831in}}%
\pgfpathlineto{\pgfqpoint{1.087735in}{0.757358in}}%
\pgfpathlineto{\pgfqpoint{1.070654in}{0.759703in}}%
\pgfpathlineto{\pgfqpoint{1.063758in}{0.762298in}}%
\pgfpathlineto{\pgfqpoint{1.065981in}{0.771962in}}%
\pgfpathlineto{\pgfqpoint{1.063832in}{0.777370in}}%
\pgfpathlineto{\pgfqpoint{1.064806in}{0.782977in}}%
\pgfpathlineto{\pgfqpoint{1.061362in}{0.785689in}}%
\pgfpathlineto{\pgfqpoint{1.062578in}{0.792196in}}%
\pgfpathlineto{\pgfqpoint{1.061462in}{0.800953in}}%
\pgfpathlineto{\pgfqpoint{1.054598in}{0.801868in}}%
\pgfpathlineto{\pgfqpoint{1.057206in}{0.819980in}}%
\pgfpathlineto{\pgfqpoint{1.060540in}{0.819258in}}%
\pgfpathlineto{\pgfqpoint{1.115027in}{0.811851in}}%
\pgfpathlineto{\pgfqpoint{1.146247in}{0.808107in}}%
\pgfpathclose%
\pgfusepath{fill}%
\end{pgfscope}%
\begin{pgfscope}%
\pgfpathrectangle{\pgfqpoint{0.100000in}{0.100000in}}{\pgfqpoint{3.007045in}{1.925000in}}%
\pgfusepath{clip}%
\pgfsetbuttcap%
\pgfsetmiterjoin%
\definecolor{currentfill}{rgb}{0.396909,0.666851,0.830358}%
\pgfsetfillcolor{currentfill}%
\pgfsetlinewidth{0.000000pt}%
\definecolor{currentstroke}{rgb}{0.000000,0.000000,0.000000}%
\pgfsetstrokecolor{currentstroke}%
\pgfsetstrokeopacity{0.000000}%
\pgfsetdash{}{0pt}%
\pgfpathmoveto{\pgfqpoint{1.718799in}{1.643280in}}%
\pgfpathlineto{\pgfqpoint{1.718828in}{1.637478in}}%
\pgfpathlineto{\pgfqpoint{1.713868in}{1.637553in}}%
\pgfpathlineto{\pgfqpoint{1.713584in}{1.614400in}}%
\pgfpathlineto{\pgfqpoint{1.714002in}{1.597084in}}%
\pgfpathlineto{\pgfqpoint{1.699675in}{1.597325in}}%
\pgfpathlineto{\pgfqpoint{1.700191in}{1.591568in}}%
\pgfpathlineto{\pgfqpoint{1.669507in}{1.592201in}}%
\pgfpathlineto{\pgfqpoint{1.668146in}{1.596346in}}%
\pgfpathlineto{\pgfqpoint{1.667675in}{1.615294in}}%
\pgfpathlineto{\pgfqpoint{1.668280in}{1.638340in}}%
\pgfpathlineto{\pgfqpoint{1.649938in}{1.638890in}}%
\pgfpathlineto{\pgfqpoint{1.650635in}{1.662065in}}%
\pgfpathlineto{\pgfqpoint{1.650048in}{1.679481in}}%
\pgfpathlineto{\pgfqpoint{1.689494in}{1.678334in}}%
\pgfpathlineto{\pgfqpoint{1.689129in}{1.695752in}}%
\pgfpathlineto{\pgfqpoint{1.724288in}{1.695202in}}%
\pgfpathlineto{\pgfqpoint{1.724037in}{1.672065in}}%
\pgfpathlineto{\pgfqpoint{1.718261in}{1.672097in}}%
\pgfpathlineto{\pgfqpoint{1.719048in}{1.660538in}}%
\pgfpathlineto{\pgfqpoint{1.718799in}{1.643280in}}%
\pgfpathclose%
\pgfusepath{fill}%
\end{pgfscope}%
\begin{pgfscope}%
\pgfpathrectangle{\pgfqpoint{0.100000in}{0.100000in}}{\pgfqpoint{3.007045in}{1.925000in}}%
\pgfusepath{clip}%
\pgfsetbuttcap%
\pgfsetmiterjoin%
\definecolor{currentfill}{rgb}{0.784591,0.864237,0.939962}%
\pgfsetfillcolor{currentfill}%
\pgfsetlinewidth{0.000000pt}%
\definecolor{currentstroke}{rgb}{0.000000,0.000000,0.000000}%
\pgfsetstrokecolor{currentstroke}%
\pgfsetstrokeopacity{0.000000}%
\pgfsetdash{}{0pt}%
\pgfpathmoveto{\pgfqpoint{2.298344in}{1.505689in}}%
\pgfpathlineto{\pgfqpoint{2.264263in}{1.501550in}}%
\pgfpathlineto{\pgfqpoint{2.252766in}{1.500298in}}%
\pgfpathlineto{\pgfqpoint{2.250233in}{1.523286in}}%
\pgfpathlineto{\pgfqpoint{2.247824in}{1.546163in}}%
\pgfpathlineto{\pgfqpoint{2.245496in}{1.563087in}}%
\pgfpathlineto{\pgfqpoint{2.251176in}{1.563619in}}%
\pgfpathlineto{\pgfqpoint{2.250552in}{1.569368in}}%
\pgfpathlineto{\pgfqpoint{2.273175in}{1.571785in}}%
\pgfpathlineto{\pgfqpoint{2.290081in}{1.574045in}}%
\pgfpathlineto{\pgfqpoint{2.291082in}{1.562710in}}%
\pgfpathlineto{\pgfqpoint{2.298344in}{1.505689in}}%
\pgfpathclose%
\pgfusepath{fill}%
\end{pgfscope}%
\begin{pgfscope}%
\pgfpathrectangle{\pgfqpoint{0.100000in}{0.100000in}}{\pgfqpoint{3.007045in}{1.925000in}}%
\pgfusepath{clip}%
\pgfsetbuttcap%
\pgfsetmiterjoin%
\definecolor{currentfill}{rgb}{0.867266,0.919354,0.967520}%
\pgfsetfillcolor{currentfill}%
\pgfsetlinewidth{0.000000pt}%
\definecolor{currentstroke}{rgb}{0.000000,0.000000,0.000000}%
\pgfsetstrokecolor{currentstroke}%
\pgfsetstrokeopacity{0.000000}%
\pgfsetdash{}{0pt}%
\pgfpathmoveto{\pgfqpoint{2.574578in}{0.398555in}}%
\pgfpathlineto{\pgfqpoint{2.577000in}{0.380965in}}%
\pgfpathlineto{\pgfqpoint{2.547508in}{0.376728in}}%
\pgfpathlineto{\pgfqpoint{2.547051in}{0.383388in}}%
\pgfpathlineto{\pgfqpoint{2.542240in}{0.387941in}}%
\pgfpathlineto{\pgfqpoint{2.539505in}{0.384621in}}%
\pgfpathlineto{\pgfqpoint{2.542207in}{0.377186in}}%
\pgfpathlineto{\pgfqpoint{2.536927in}{0.376168in}}%
\pgfpathlineto{\pgfqpoint{2.534833in}{0.376193in}}%
\pgfpathlineto{\pgfqpoint{2.521960in}{0.392832in}}%
\pgfpathlineto{\pgfqpoint{2.512567in}{0.407613in}}%
\pgfpathlineto{\pgfqpoint{2.503920in}{0.416388in}}%
\pgfpathlineto{\pgfqpoint{2.505714in}{0.422690in}}%
\pgfpathlineto{\pgfqpoint{2.510105in}{0.430828in}}%
\pgfpathlineto{\pgfqpoint{2.539586in}{0.435076in}}%
\pgfpathlineto{\pgfqpoint{2.542441in}{0.414576in}}%
\pgfpathlineto{\pgfqpoint{2.571480in}{0.418988in}}%
\pgfpathlineto{\pgfqpoint{2.574578in}{0.398555in}}%
\pgfpathclose%
\pgfusepath{fill}%
\end{pgfscope}%
\begin{pgfscope}%
\pgfpathrectangle{\pgfqpoint{0.100000in}{0.100000in}}{\pgfqpoint{3.007045in}{1.925000in}}%
\pgfusepath{clip}%
\pgfsetbuttcap%
\pgfsetmiterjoin%
\definecolor{currentfill}{rgb}{0.686659,0.820654,0.905821}%
\pgfsetfillcolor{currentfill}%
\pgfsetlinewidth{0.000000pt}%
\definecolor{currentstroke}{rgb}{0.000000,0.000000,0.000000}%
\pgfsetstrokecolor{currentstroke}%
\pgfsetstrokeopacity{0.000000}%
\pgfsetdash{}{0pt}%
\pgfpathmoveto{\pgfqpoint{2.555437in}{1.355961in}}%
\pgfpathlineto{\pgfqpoint{2.554563in}{1.361326in}}%
\pgfpathlineto{\pgfqpoint{2.559925in}{1.364184in}}%
\pgfpathlineto{\pgfqpoint{2.558934in}{1.375533in}}%
\pgfpathlineto{\pgfqpoint{2.581784in}{1.380075in}}%
\pgfpathlineto{\pgfqpoint{2.595492in}{1.381828in}}%
\pgfpathlineto{\pgfqpoint{2.604647in}{1.373975in}}%
\pgfpathlineto{\pgfqpoint{2.607627in}{1.374496in}}%
\pgfpathlineto{\pgfqpoint{2.608964in}{1.367546in}}%
\pgfpathlineto{\pgfqpoint{2.605715in}{1.356835in}}%
\pgfpathlineto{\pgfqpoint{2.573880in}{1.354042in}}%
\pgfpathlineto{\pgfqpoint{2.561808in}{1.358520in}}%
\pgfpathlineto{\pgfqpoint{2.555437in}{1.355961in}}%
\pgfpathclose%
\pgfusepath{fill}%
\end{pgfscope}%
\begin{pgfscope}%
\pgfpathrectangle{\pgfqpoint{0.100000in}{0.100000in}}{\pgfqpoint{3.007045in}{1.925000in}}%
\pgfusepath{clip}%
\pgfsetbuttcap%
\pgfsetmiterjoin%
\definecolor{currentfill}{rgb}{0.516863,0.735748,0.860192}%
\pgfsetfillcolor{currentfill}%
\pgfsetlinewidth{0.000000pt}%
\definecolor{currentstroke}{rgb}{0.000000,0.000000,0.000000}%
\pgfsetstrokecolor{currentstroke}%
\pgfsetstrokeopacity{0.000000}%
\pgfsetdash{}{0pt}%
\pgfpathmoveto{\pgfqpoint{2.169528in}{1.143936in}}%
\pgfpathlineto{\pgfqpoint{2.155147in}{1.142366in}}%
\pgfpathlineto{\pgfqpoint{2.152006in}{1.147645in}}%
\pgfpathlineto{\pgfqpoint{2.151777in}{1.152326in}}%
\pgfpathlineto{\pgfqpoint{2.148381in}{1.158667in}}%
\pgfpathlineto{\pgfqpoint{2.150340in}{1.161666in}}%
\pgfpathlineto{\pgfqpoint{2.148720in}{1.168849in}}%
\pgfpathlineto{\pgfqpoint{2.152293in}{1.171574in}}%
\pgfpathlineto{\pgfqpoint{2.149319in}{1.206705in}}%
\pgfpathlineto{\pgfqpoint{2.147976in}{1.224103in}}%
\pgfpathlineto{\pgfqpoint{2.154351in}{1.223276in}}%
\pgfpathlineto{\pgfqpoint{2.154631in}{1.211728in}}%
\pgfpathlineto{\pgfqpoint{2.171191in}{1.213179in}}%
\pgfpathlineto{\pgfqpoint{2.169632in}{1.230355in}}%
\pgfpathlineto{\pgfqpoint{2.189499in}{1.232103in}}%
\pgfpathlineto{\pgfqpoint{2.189712in}{1.229760in}}%
\pgfpathlineto{\pgfqpoint{2.193431in}{1.193828in}}%
\pgfpathlineto{\pgfqpoint{2.195596in}{1.188844in}}%
\pgfpathlineto{\pgfqpoint{2.194423in}{1.183326in}}%
\pgfpathlineto{\pgfqpoint{2.181588in}{1.182377in}}%
\pgfpathlineto{\pgfqpoint{2.182227in}{1.173751in}}%
\pgfpathlineto{\pgfqpoint{2.176581in}{1.173291in}}%
\pgfpathlineto{\pgfqpoint{2.177539in}{1.161835in}}%
\pgfpathlineto{\pgfqpoint{2.168048in}{1.161256in}}%
\pgfpathlineto{\pgfqpoint{2.169528in}{1.143936in}}%
\pgfpathclose%
\pgfusepath{fill}%
\end{pgfscope}%
\begin{pgfscope}%
\pgfpathrectangle{\pgfqpoint{0.100000in}{0.100000in}}{\pgfqpoint{3.007045in}{1.925000in}}%
\pgfusepath{clip}%
\pgfsetbuttcap%
\pgfsetmiterjoin%
\definecolor{currentfill}{rgb}{0.341423,0.628958,0.808704}%
\pgfsetfillcolor{currentfill}%
\pgfsetlinewidth{0.000000pt}%
\definecolor{currentstroke}{rgb}{0.000000,0.000000,0.000000}%
\pgfsetstrokecolor{currentstroke}%
\pgfsetstrokeopacity{0.000000}%
\pgfsetdash{}{0pt}%
\pgfpathmoveto{\pgfqpoint{1.333387in}{1.392952in}}%
\pgfpathlineto{\pgfqpoint{1.326529in}{1.324213in}}%
\pgfpathlineto{\pgfqpoint{1.297072in}{1.327304in}}%
\pgfpathlineto{\pgfqpoint{1.297671in}{1.333074in}}%
\pgfpathlineto{\pgfqpoint{1.267098in}{1.336682in}}%
\pgfpathlineto{\pgfqpoint{1.269866in}{1.358232in}}%
\pgfpathlineto{\pgfqpoint{1.272590in}{1.387487in}}%
\pgfpathlineto{\pgfqpoint{1.273764in}{1.398937in}}%
\pgfpathlineto{\pgfqpoint{1.292810in}{1.397021in}}%
\pgfpathlineto{\pgfqpoint{1.333387in}{1.392952in}}%
\pgfpathclose%
\pgfusepath{fill}%
\end{pgfscope}%
\begin{pgfscope}%
\pgfpathrectangle{\pgfqpoint{0.100000in}{0.100000in}}{\pgfqpoint{3.007045in}{1.925000in}}%
\pgfusepath{clip}%
\pgfsetbuttcap%
\pgfsetmiterjoin%
\definecolor{currentfill}{rgb}{0.154787,0.468512,0.722876}%
\pgfsetfillcolor{currentfill}%
\pgfsetlinewidth{0.000000pt}%
\definecolor{currentstroke}{rgb}{0.000000,0.000000,0.000000}%
\pgfsetstrokecolor{currentstroke}%
\pgfsetstrokeopacity{0.000000}%
\pgfsetdash{}{0pt}%
\pgfpathmoveto{\pgfqpoint{1.392891in}{0.867332in}}%
\pgfpathlineto{\pgfqpoint{1.421405in}{0.865168in}}%
\pgfpathlineto{\pgfqpoint{1.419358in}{0.836488in}}%
\pgfpathlineto{\pgfqpoint{1.442781in}{0.834883in}}%
\pgfpathlineto{\pgfqpoint{1.440696in}{0.803294in}}%
\pgfpathlineto{\pgfqpoint{1.383417in}{0.806844in}}%
\pgfpathlineto{\pgfqpoint{1.385676in}{0.839061in}}%
\pgfpathlineto{\pgfqpoint{1.390704in}{0.838664in}}%
\pgfpathlineto{\pgfqpoint{1.392891in}{0.867332in}}%
\pgfpathclose%
\pgfusepath{fill}%
\end{pgfscope}%
\begin{pgfscope}%
\pgfpathrectangle{\pgfqpoint{0.100000in}{0.100000in}}{\pgfqpoint{3.007045in}{1.925000in}}%
\pgfusepath{clip}%
\pgfsetbuttcap%
\pgfsetmiterjoin%
\definecolor{currentfill}{rgb}{0.637447,0.799739,0.888597}%
\pgfsetfillcolor{currentfill}%
\pgfsetlinewidth{0.000000pt}%
\definecolor{currentstroke}{rgb}{0.000000,0.000000,0.000000}%
\pgfsetstrokecolor{currentstroke}%
\pgfsetstrokeopacity{0.000000}%
\pgfsetdash{}{0pt}%
\pgfpathmoveto{\pgfqpoint{1.180201in}{1.170746in}}%
\pgfpathlineto{\pgfqpoint{1.177735in}{1.174146in}}%
\pgfpathlineto{\pgfqpoint{1.172957in}{1.174640in}}%
\pgfpathlineto{\pgfqpoint{1.169868in}{1.170865in}}%
\pgfpathlineto{\pgfqpoint{1.164471in}{1.172432in}}%
\pgfpathlineto{\pgfqpoint{1.157547in}{1.181738in}}%
\pgfpathlineto{\pgfqpoint{1.146599in}{1.183208in}}%
\pgfpathlineto{\pgfqpoint{1.143492in}{1.187243in}}%
\pgfpathlineto{\pgfqpoint{1.142160in}{1.192960in}}%
\pgfpathlineto{\pgfqpoint{1.139110in}{1.197239in}}%
\pgfpathlineto{\pgfqpoint{1.141305in}{1.200383in}}%
\pgfpathlineto{\pgfqpoint{1.091973in}{1.207431in}}%
\pgfpathlineto{\pgfqpoint{1.059704in}{1.212390in}}%
\pgfpathlineto{\pgfqpoint{1.061002in}{1.220921in}}%
\pgfpathlineto{\pgfqpoint{1.062663in}{1.231515in}}%
\pgfpathlineto{\pgfqpoint{1.087451in}{1.227034in}}%
\pgfpathlineto{\pgfqpoint{1.087872in}{1.229867in}}%
\pgfpathlineto{\pgfqpoint{1.118778in}{1.225384in}}%
\pgfpathlineto{\pgfqpoint{1.120000in}{1.233895in}}%
\pgfpathlineto{\pgfqpoint{1.145415in}{1.230549in}}%
\pgfpathlineto{\pgfqpoint{1.146223in}{1.236379in}}%
\pgfpathlineto{\pgfqpoint{1.151850in}{1.235425in}}%
\pgfpathlineto{\pgfqpoint{1.153503in}{1.246887in}}%
\pgfpathlineto{\pgfqpoint{1.167476in}{1.245055in}}%
\pgfpathlineto{\pgfqpoint{1.166255in}{1.233746in}}%
\pgfpathlineto{\pgfqpoint{1.186601in}{1.231472in}}%
\pgfpathlineto{\pgfqpoint{1.205530in}{1.228275in}}%
\pgfpathlineto{\pgfqpoint{1.212843in}{1.220333in}}%
\pgfpathlineto{\pgfqpoint{1.214794in}{1.212322in}}%
\pgfpathlineto{\pgfqpoint{1.220004in}{1.212317in}}%
\pgfpathlineto{\pgfqpoint{1.226581in}{1.206947in}}%
\pgfpathlineto{\pgfqpoint{1.216380in}{1.193870in}}%
\pgfpathlineto{\pgfqpoint{1.203807in}{1.188442in}}%
\pgfpathlineto{\pgfqpoint{1.202987in}{1.177221in}}%
\pgfpathlineto{\pgfqpoint{1.201403in}{1.171929in}}%
\pgfpathlineto{\pgfqpoint{1.181782in}{1.174522in}}%
\pgfpathlineto{\pgfqpoint{1.180201in}{1.170746in}}%
\pgfpathclose%
\pgfusepath{fill}%
\end{pgfscope}%
\begin{pgfscope}%
\pgfpathrectangle{\pgfqpoint{0.100000in}{0.100000in}}{\pgfqpoint{3.007045in}{1.925000in}}%
\pgfusepath{clip}%
\pgfsetbuttcap%
\pgfsetmiterjoin%
\definecolor{currentfill}{rgb}{0.435294,0.690965,0.842599}%
\pgfsetfillcolor{currentfill}%
\pgfsetlinewidth{0.000000pt}%
\definecolor{currentstroke}{rgb}{0.000000,0.000000,0.000000}%
\pgfsetstrokecolor{currentstroke}%
\pgfsetstrokeopacity{0.000000}%
\pgfsetdash{}{0pt}%
\pgfpathmoveto{\pgfqpoint{2.593275in}{0.442838in}}%
\pgfpathlineto{\pgfqpoint{2.590385in}{0.434304in}}%
\pgfpathlineto{\pgfqpoint{2.597645in}{0.425944in}}%
\pgfpathlineto{\pgfqpoint{2.600616in}{0.426095in}}%
\pgfpathlineto{\pgfqpoint{2.607766in}{0.416135in}}%
\pgfpathlineto{\pgfqpoint{2.596119in}{0.413805in}}%
\pgfpathlineto{\pgfqpoint{2.597054in}{0.407964in}}%
\pgfpathlineto{\pgfqpoint{2.591178in}{0.407023in}}%
\pgfpathlineto{\pgfqpoint{2.592108in}{0.401133in}}%
\pgfpathlineto{\pgfqpoint{2.574578in}{0.398555in}}%
\pgfpathlineto{\pgfqpoint{2.571480in}{0.418988in}}%
\pgfpathlineto{\pgfqpoint{2.542441in}{0.414576in}}%
\pgfpathlineto{\pgfqpoint{2.539586in}{0.435076in}}%
\pgfpathlineto{\pgfqpoint{2.534472in}{0.469934in}}%
\pgfpathlineto{\pgfqpoint{2.530722in}{0.475320in}}%
\pgfpathlineto{\pgfqpoint{2.533154in}{0.479245in}}%
\pgfpathlineto{\pgfqpoint{2.538576in}{0.482276in}}%
\pgfpathlineto{\pgfqpoint{2.556078in}{0.484967in}}%
\pgfpathlineto{\pgfqpoint{2.556944in}{0.479141in}}%
\pgfpathlineto{\pgfqpoint{2.562765in}{0.480015in}}%
\pgfpathlineto{\pgfqpoint{2.565859in}{0.472579in}}%
\pgfpathlineto{\pgfqpoint{2.575198in}{0.465448in}}%
\pgfpathlineto{\pgfqpoint{2.580449in}{0.459760in}}%
\pgfpathlineto{\pgfqpoint{2.581515in}{0.455951in}}%
\pgfpathlineto{\pgfqpoint{2.587084in}{0.454555in}}%
\pgfpathlineto{\pgfqpoint{2.592017in}{0.447697in}}%
\pgfpathlineto{\pgfqpoint{2.593275in}{0.442838in}}%
\pgfpathclose%
\pgfusepath{fill}%
\end{pgfscope}%
\begin{pgfscope}%
\pgfpathrectangle{\pgfqpoint{0.100000in}{0.100000in}}{\pgfqpoint{3.007045in}{1.925000in}}%
\pgfusepath{clip}%
\pgfsetbuttcap%
\pgfsetmiterjoin%
\definecolor{currentfill}{rgb}{0.040984,0.329950,0.621376}%
\pgfsetfillcolor{currentfill}%
\pgfsetlinewidth{0.000000pt}%
\definecolor{currentstroke}{rgb}{0.000000,0.000000,0.000000}%
\pgfsetstrokecolor{currentstroke}%
\pgfsetstrokeopacity{0.000000}%
\pgfsetdash{}{0pt}%
\pgfpathmoveto{\pgfqpoint{0.441277in}{1.107129in}}%
\pgfpathlineto{\pgfqpoint{0.475084in}{1.097667in}}%
\pgfpathlineto{\pgfqpoint{0.522151in}{1.084892in}}%
\pgfpathlineto{\pgfqpoint{0.554172in}{1.076466in}}%
\pgfpathlineto{\pgfqpoint{0.573744in}{1.071996in}}%
\pgfpathlineto{\pgfqpoint{0.557933in}{1.009610in}}%
\pgfpathlineto{\pgfqpoint{0.515049in}{1.020412in}}%
\pgfpathlineto{\pgfqpoint{0.473306in}{1.031380in}}%
\pgfpathlineto{\pgfqpoint{0.472706in}{1.036029in}}%
\pgfpathlineto{\pgfqpoint{0.464444in}{1.039777in}}%
\pgfpathlineto{\pgfqpoint{0.465944in}{1.051393in}}%
\pgfpathlineto{\pgfqpoint{0.461538in}{1.053350in}}%
\pgfpathlineto{\pgfqpoint{0.463546in}{1.059117in}}%
\pgfpathlineto{\pgfqpoint{0.457558in}{1.060425in}}%
\pgfpathlineto{\pgfqpoint{0.459131in}{1.065999in}}%
\pgfpathlineto{\pgfqpoint{0.451737in}{1.068135in}}%
\pgfpathlineto{\pgfqpoint{0.453296in}{1.073681in}}%
\pgfpathlineto{\pgfqpoint{0.449631in}{1.074733in}}%
\pgfpathlineto{\pgfqpoint{0.451207in}{1.080317in}}%
\pgfpathlineto{\pgfqpoint{0.445664in}{1.083905in}}%
\pgfpathlineto{\pgfqpoint{0.442087in}{1.088884in}}%
\pgfpathlineto{\pgfqpoint{0.443707in}{1.094459in}}%
\pgfpathlineto{\pgfqpoint{0.438131in}{1.096030in}}%
\pgfpathlineto{\pgfqpoint{0.441277in}{1.107129in}}%
\pgfpathclose%
\pgfusepath{fill}%
\end{pgfscope}%
\begin{pgfscope}%
\pgfpathrectangle{\pgfqpoint{0.100000in}{0.100000in}}{\pgfqpoint{3.007045in}{1.925000in}}%
\pgfusepath{clip}%
\pgfsetbuttcap%
\pgfsetmiterjoin%
\definecolor{currentfill}{rgb}{0.535686,0.746082,0.864252}%
\pgfsetfillcolor{currentfill}%
\pgfsetlinewidth{0.000000pt}%
\definecolor{currentstroke}{rgb}{0.000000,0.000000,0.000000}%
\pgfsetstrokecolor{currentstroke}%
\pgfsetstrokeopacity{0.000000}%
\pgfsetdash{}{0pt}%
\pgfpathmoveto{\pgfqpoint{1.844539in}{1.237709in}}%
\pgfpathlineto{\pgfqpoint{1.833661in}{1.237372in}}%
\pgfpathlineto{\pgfqpoint{1.798792in}{1.236593in}}%
\pgfpathlineto{\pgfqpoint{1.798597in}{1.275256in}}%
\pgfpathlineto{\pgfqpoint{1.826979in}{1.275432in}}%
\pgfpathlineto{\pgfqpoint{1.843908in}{1.275998in}}%
\pgfpathlineto{\pgfqpoint{1.844539in}{1.237709in}}%
\pgfpathclose%
\pgfusepath{fill}%
\end{pgfscope}%
\begin{pgfscope}%
\pgfpathrectangle{\pgfqpoint{0.100000in}{0.100000in}}{\pgfqpoint{3.007045in}{1.925000in}}%
\pgfusepath{clip}%
\pgfsetbuttcap%
\pgfsetmiterjoin%
\definecolor{currentfill}{rgb}{0.429020,0.687520,0.841246}%
\pgfsetfillcolor{currentfill}%
\pgfsetlinewidth{0.000000pt}%
\definecolor{currentstroke}{rgb}{0.000000,0.000000,0.000000}%
\pgfsetstrokecolor{currentstroke}%
\pgfsetstrokeopacity{0.000000}%
\pgfsetdash{}{0pt}%
\pgfpathmoveto{\pgfqpoint{2.287383in}{1.019436in}}%
\pgfpathlineto{\pgfqpoint{2.286020in}{1.023971in}}%
\pgfpathlineto{\pgfqpoint{2.274407in}{1.023601in}}%
\pgfpathlineto{\pgfqpoint{2.271241in}{1.027836in}}%
\pgfpathlineto{\pgfqpoint{2.269307in}{1.034406in}}%
\pgfpathlineto{\pgfqpoint{2.263654in}{1.034861in}}%
\pgfpathlineto{\pgfqpoint{2.260460in}{1.038312in}}%
\pgfpathlineto{\pgfqpoint{2.259090in}{1.046090in}}%
\pgfpathlineto{\pgfqpoint{2.260372in}{1.054200in}}%
\pgfpathlineto{\pgfqpoint{2.261228in}{1.057564in}}%
\pgfpathlineto{\pgfqpoint{2.270021in}{1.058081in}}%
\pgfpathlineto{\pgfqpoint{2.276076in}{1.059145in}}%
\pgfpathlineto{\pgfqpoint{2.281792in}{1.056352in}}%
\pgfpathlineto{\pgfqpoint{2.285344in}{1.059690in}}%
\pgfpathlineto{\pgfqpoint{2.292317in}{1.056941in}}%
\pgfpathlineto{\pgfqpoint{2.291623in}{1.065687in}}%
\pgfpathlineto{\pgfqpoint{2.301227in}{1.065647in}}%
\pgfpathlineto{\pgfqpoint{2.302296in}{1.059040in}}%
\pgfpathlineto{\pgfqpoint{2.309458in}{1.055417in}}%
\pgfpathlineto{\pgfqpoint{2.310463in}{1.047437in}}%
\pgfpathlineto{\pgfqpoint{2.301947in}{1.038398in}}%
\pgfpathlineto{\pgfqpoint{2.302206in}{1.033847in}}%
\pgfpathlineto{\pgfqpoint{2.306270in}{1.030998in}}%
\pgfpathlineto{\pgfqpoint{2.292665in}{1.018214in}}%
\pgfpathlineto{\pgfqpoint{2.287383in}{1.019436in}}%
\pgfpathclose%
\pgfusepath{fill}%
\end{pgfscope}%
\begin{pgfscope}%
\pgfpathrectangle{\pgfqpoint{0.100000in}{0.100000in}}{\pgfqpoint{3.007045in}{1.925000in}}%
\pgfusepath{clip}%
\pgfsetbuttcap%
\pgfsetmiterjoin%
\definecolor{currentfill}{rgb}{0.108651,0.416563,0.689043}%
\pgfsetfillcolor{currentfill}%
\pgfsetlinewidth{0.000000pt}%
\definecolor{currentstroke}{rgb}{0.000000,0.000000,0.000000}%
\pgfsetstrokecolor{currentstroke}%
\pgfsetstrokeopacity{0.000000}%
\pgfsetdash{}{0pt}%
\pgfpathmoveto{\pgfqpoint{1.442781in}{0.834883in}}%
\pgfpathlineto{\pgfqpoint{1.419358in}{0.836488in}}%
\pgfpathlineto{\pgfqpoint{1.421405in}{0.865168in}}%
\pgfpathlineto{\pgfqpoint{1.442027in}{0.863792in}}%
\pgfpathlineto{\pgfqpoint{1.444186in}{0.892364in}}%
\pgfpathlineto{\pgfqpoint{1.473659in}{0.890543in}}%
\pgfpathlineto{\pgfqpoint{1.471813in}{0.861849in}}%
\pgfpathlineto{\pgfqpoint{1.478574in}{0.861428in}}%
\pgfpathlineto{\pgfqpoint{1.476764in}{0.832809in}}%
\pgfpathlineto{\pgfqpoint{1.471341in}{0.833174in}}%
\pgfpathlineto{\pgfqpoint{1.442781in}{0.834883in}}%
\pgfpathclose%
\pgfusepath{fill}%
\end{pgfscope}%
\begin{pgfscope}%
\pgfpathrectangle{\pgfqpoint{0.100000in}{0.100000in}}{\pgfqpoint{3.007045in}{1.925000in}}%
\pgfusepath{clip}%
\pgfsetbuttcap%
\pgfsetmiterjoin%
\definecolor{currentfill}{rgb}{0.479216,0.715079,0.852072}%
\pgfsetfillcolor{currentfill}%
\pgfsetlinewidth{0.000000pt}%
\definecolor{currentstroke}{rgb}{0.000000,0.000000,0.000000}%
\pgfsetstrokecolor{currentstroke}%
\pgfsetstrokeopacity{0.000000}%
\pgfsetdash{}{0pt}%
\pgfpathmoveto{\pgfqpoint{1.405343in}{1.057957in}}%
\pgfpathlineto{\pgfqpoint{1.404039in}{1.041199in}}%
\pgfpathlineto{\pgfqpoint{1.401987in}{1.015175in}}%
\pgfpathlineto{\pgfqpoint{1.367585in}{1.018108in}}%
\pgfpathlineto{\pgfqpoint{1.351668in}{1.019814in}}%
\pgfpathlineto{\pgfqpoint{1.347270in}{1.020215in}}%
\pgfpathlineto{\pgfqpoint{1.351564in}{1.062383in}}%
\pgfpathlineto{\pgfqpoint{1.405343in}{1.057957in}}%
\pgfpathclose%
\pgfusepath{fill}%
\end{pgfscope}%
\begin{pgfscope}%
\pgfpathrectangle{\pgfqpoint{0.100000in}{0.100000in}}{\pgfqpoint{3.007045in}{1.925000in}}%
\pgfusepath{clip}%
\pgfsetbuttcap%
\pgfsetmiterjoin%
\definecolor{currentfill}{rgb}{0.381776,0.656517,0.824452}%
\pgfsetfillcolor{currentfill}%
\pgfsetlinewidth{0.000000pt}%
\definecolor{currentstroke}{rgb}{0.000000,0.000000,0.000000}%
\pgfsetstrokecolor{currentstroke}%
\pgfsetstrokeopacity{0.000000}%
\pgfsetdash{}{0pt}%
\pgfpathmoveto{\pgfqpoint{2.254813in}{0.936868in}}%
\pgfpathlineto{\pgfqpoint{2.251280in}{0.935449in}}%
\pgfpathlineto{\pgfqpoint{2.240739in}{0.937396in}}%
\pgfpathlineto{\pgfqpoint{2.228851in}{0.933436in}}%
\pgfpathlineto{\pgfqpoint{2.226333in}{0.935993in}}%
\pgfpathlineto{\pgfqpoint{2.219736in}{0.935214in}}%
\pgfpathlineto{\pgfqpoint{2.210327in}{0.936644in}}%
\pgfpathlineto{\pgfqpoint{2.188078in}{0.943307in}}%
\pgfpathlineto{\pgfqpoint{2.188043in}{0.950490in}}%
\pgfpathlineto{\pgfqpoint{2.188978in}{0.963509in}}%
\pgfpathlineto{\pgfqpoint{2.187491in}{0.967104in}}%
\pgfpathlineto{\pgfqpoint{2.188596in}{0.973610in}}%
\pgfpathlineto{\pgfqpoint{2.181638in}{0.973954in}}%
\pgfpathlineto{\pgfqpoint{2.183256in}{0.977798in}}%
\pgfpathlineto{\pgfqpoint{2.189194in}{0.983871in}}%
\pgfpathlineto{\pgfqpoint{2.188886in}{0.995787in}}%
\pgfpathlineto{\pgfqpoint{2.191739in}{0.996087in}}%
\pgfpathlineto{\pgfqpoint{2.216487in}{0.998935in}}%
\pgfpathlineto{\pgfqpoint{2.230102in}{0.999971in}}%
\pgfpathlineto{\pgfqpoint{2.248936in}{1.000483in}}%
\pgfpathlineto{\pgfqpoint{2.258514in}{1.000440in}}%
\pgfpathlineto{\pgfqpoint{2.258356in}{0.992865in}}%
\pgfpathlineto{\pgfqpoint{2.258038in}{0.978829in}}%
\pgfpathlineto{\pgfqpoint{2.261071in}{0.973710in}}%
\pgfpathlineto{\pgfqpoint{2.261147in}{0.968764in}}%
\pgfpathlineto{\pgfqpoint{2.258371in}{0.965969in}}%
\pgfpathlineto{\pgfqpoint{2.253187in}{0.966596in}}%
\pgfpathlineto{\pgfqpoint{2.248417in}{0.962697in}}%
\pgfpathlineto{\pgfqpoint{2.251263in}{0.956468in}}%
\pgfpathlineto{\pgfqpoint{2.257521in}{0.951222in}}%
\pgfpathlineto{\pgfqpoint{2.254335in}{0.941757in}}%
\pgfpathlineto{\pgfqpoint{2.254813in}{0.936868in}}%
\pgfpathclose%
\pgfusepath{fill}%
\end{pgfscope}%
\begin{pgfscope}%
\pgfpathrectangle{\pgfqpoint{0.100000in}{0.100000in}}{\pgfqpoint{3.007045in}{1.925000in}}%
\pgfusepath{clip}%
\pgfsetbuttcap%
\pgfsetmiterjoin%
\definecolor{currentfill}{rgb}{0.130427,0.444152,0.710327}%
\pgfsetfillcolor{currentfill}%
\pgfsetlinewidth{0.000000pt}%
\definecolor{currentstroke}{rgb}{0.000000,0.000000,0.000000}%
\pgfsetstrokecolor{currentstroke}%
\pgfsetstrokeopacity{0.000000}%
\pgfsetdash{}{0pt}%
\pgfpathmoveto{\pgfqpoint{0.421673in}{1.178083in}}%
\pgfpathlineto{\pgfqpoint{0.412185in}{1.187494in}}%
\pgfpathlineto{\pgfqpoint{0.409170in}{1.194540in}}%
\pgfpathlineto{\pgfqpoint{0.413081in}{1.203149in}}%
\pgfpathlineto{\pgfqpoint{0.413366in}{1.209180in}}%
\pgfpathlineto{\pgfqpoint{0.404869in}{1.211825in}}%
\pgfpathlineto{\pgfqpoint{0.404848in}{1.220444in}}%
\pgfpathlineto{\pgfqpoint{0.408920in}{1.226474in}}%
\pgfpathlineto{\pgfqpoint{0.407661in}{1.233230in}}%
\pgfpathlineto{\pgfqpoint{0.407690in}{1.233195in}}%
\pgfpathlineto{\pgfqpoint{0.424991in}{1.242670in}}%
\pgfpathlineto{\pgfqpoint{0.436885in}{1.243705in}}%
\pgfpathlineto{\pgfqpoint{0.440085in}{1.241248in}}%
\pgfpathlineto{\pgfqpoint{0.446079in}{1.262743in}}%
\pgfpathlineto{\pgfqpoint{0.455151in}{1.243275in}}%
\pgfpathlineto{\pgfqpoint{0.464303in}{1.248933in}}%
\pgfpathlineto{\pgfqpoint{0.475080in}{1.259356in}}%
\pgfpathlineto{\pgfqpoint{0.488781in}{1.270732in}}%
\pgfpathlineto{\pgfqpoint{0.497623in}{1.272306in}}%
\pgfpathlineto{\pgfqpoint{0.502964in}{1.265519in}}%
\pgfpathlineto{\pgfqpoint{0.510697in}{1.267522in}}%
\pgfpathlineto{\pgfqpoint{0.514663in}{1.260246in}}%
\pgfpathlineto{\pgfqpoint{0.513336in}{1.257851in}}%
\pgfpathlineto{\pgfqpoint{0.514807in}{1.248637in}}%
\pgfpathlineto{\pgfqpoint{0.525021in}{1.240725in}}%
\pgfpathlineto{\pgfqpoint{0.524524in}{1.231551in}}%
\pgfpathlineto{\pgfqpoint{0.527593in}{1.228218in}}%
\pgfpathlineto{\pgfqpoint{0.527157in}{1.220456in}}%
\pgfpathlineto{\pgfqpoint{0.522889in}{1.217818in}}%
\pgfpathlineto{\pgfqpoint{0.521553in}{1.220832in}}%
\pgfpathlineto{\pgfqpoint{0.503936in}{1.210812in}}%
\pgfpathlineto{\pgfqpoint{0.502749in}{1.206653in}}%
\pgfpathlineto{\pgfqpoint{0.498018in}{1.202710in}}%
\pgfpathlineto{\pgfqpoint{0.492430in}{1.204219in}}%
\pgfpathlineto{\pgfqpoint{0.473609in}{1.193492in}}%
\pgfpathlineto{\pgfqpoint{0.461047in}{1.194974in}}%
\pgfpathlineto{\pgfqpoint{0.450589in}{1.194118in}}%
\pgfpathlineto{\pgfqpoint{0.421673in}{1.178083in}}%
\pgfpathclose%
\pgfusepath{fill}%
\end{pgfscope}%
\begin{pgfscope}%
\pgfpathrectangle{\pgfqpoint{0.100000in}{0.100000in}}{\pgfqpoint{3.007045in}{1.925000in}}%
\pgfusepath{clip}%
\pgfsetbuttcap%
\pgfsetmiterjoin%
\definecolor{currentfill}{rgb}{0.114802,0.424437,0.695194}%
\pgfsetfillcolor{currentfill}%
\pgfsetlinewidth{0.000000pt}%
\definecolor{currentstroke}{rgb}{0.000000,0.000000,0.000000}%
\pgfsetstrokecolor{currentstroke}%
\pgfsetstrokeopacity{0.000000}%
\pgfsetdash{}{0pt}%
\pgfpathmoveto{\pgfqpoint{1.403892in}{1.607820in}}%
\pgfpathlineto{\pgfqpoint{1.355655in}{1.612365in}}%
\pgfpathlineto{\pgfqpoint{1.359600in}{1.651719in}}%
\pgfpathlineto{\pgfqpoint{1.370707in}{1.650612in}}%
\pgfpathlineto{\pgfqpoint{1.371277in}{1.656487in}}%
\pgfpathlineto{\pgfqpoint{1.379987in}{1.655666in}}%
\pgfpathlineto{\pgfqpoint{1.410950in}{1.652859in}}%
\pgfpathlineto{\pgfqpoint{1.409091in}{1.629777in}}%
\pgfpathlineto{\pgfqpoint{1.405844in}{1.630008in}}%
\pgfpathlineto{\pgfqpoint{1.403892in}{1.607820in}}%
\pgfpathclose%
\pgfusepath{fill}%
\end{pgfscope}%
\begin{pgfscope}%
\pgfpathrectangle{\pgfqpoint{0.100000in}{0.100000in}}{\pgfqpoint{3.007045in}{1.925000in}}%
\pgfusepath{clip}%
\pgfsetbuttcap%
\pgfsetmiterjoin%
\definecolor{currentfill}{rgb}{0.529412,0.742637,0.862899}%
\pgfsetfillcolor{currentfill}%
\pgfsetlinewidth{0.000000pt}%
\definecolor{currentstroke}{rgb}{0.000000,0.000000,0.000000}%
\pgfsetstrokecolor{currentstroke}%
\pgfsetstrokeopacity{0.000000}%
\pgfsetdash{}{0pt}%
\pgfpathmoveto{\pgfqpoint{2.751955in}{1.186579in}}%
\pgfpathlineto{\pgfqpoint{2.748588in}{1.187139in}}%
\pgfpathlineto{\pgfqpoint{2.739288in}{1.183920in}}%
\pgfpathlineto{\pgfqpoint{2.728190in}{1.196931in}}%
\pgfpathlineto{\pgfqpoint{2.729962in}{1.199364in}}%
\pgfpathlineto{\pgfqpoint{2.728781in}{1.207289in}}%
\pgfpathlineto{\pgfqpoint{2.734998in}{1.209843in}}%
\pgfpathlineto{\pgfqpoint{2.724066in}{1.216278in}}%
\pgfpathlineto{\pgfqpoint{2.728601in}{1.227398in}}%
\pgfpathlineto{\pgfqpoint{2.725297in}{1.231605in}}%
\pgfpathlineto{\pgfqpoint{2.725039in}{1.237269in}}%
\pgfpathlineto{\pgfqpoint{2.721783in}{1.241309in}}%
\pgfpathlineto{\pgfqpoint{2.724195in}{1.253074in}}%
\pgfpathlineto{\pgfqpoint{2.730272in}{1.258328in}}%
\pgfpathlineto{\pgfqpoint{2.738533in}{1.258792in}}%
\pgfpathlineto{\pgfqpoint{2.744257in}{1.261114in}}%
\pgfpathlineto{\pgfqpoint{2.745705in}{1.255995in}}%
\pgfpathlineto{\pgfqpoint{2.758832in}{1.209240in}}%
\pgfpathlineto{\pgfqpoint{2.753781in}{1.202785in}}%
\pgfpathlineto{\pgfqpoint{2.751060in}{1.190437in}}%
\pgfpathlineto{\pgfqpoint{2.751955in}{1.186579in}}%
\pgfpathclose%
\pgfusepath{fill}%
\end{pgfscope}%
\begin{pgfscope}%
\pgfpathrectangle{\pgfqpoint{0.100000in}{0.100000in}}{\pgfqpoint{3.007045in}{1.925000in}}%
\pgfusepath{clip}%
\pgfsetbuttcap%
\pgfsetmiterjoin%
\definecolor{currentfill}{rgb}{0.441569,0.694410,0.843952}%
\pgfsetfillcolor{currentfill}%
\pgfsetlinewidth{0.000000pt}%
\definecolor{currentstroke}{rgb}{0.000000,0.000000,0.000000}%
\pgfsetstrokecolor{currentstroke}%
\pgfsetstrokeopacity{0.000000}%
\pgfsetdash{}{0pt}%
\pgfpathmoveto{\pgfqpoint{2.124853in}{1.498938in}}%
\pgfpathlineto{\pgfqpoint{2.108536in}{1.497608in}}%
\pgfpathlineto{\pgfqpoint{2.108968in}{1.491853in}}%
\pgfpathlineto{\pgfqpoint{2.101634in}{1.491321in}}%
\pgfpathlineto{\pgfqpoint{2.094574in}{1.490855in}}%
\pgfpathlineto{\pgfqpoint{2.092961in}{1.513610in}}%
\pgfpathlineto{\pgfqpoint{2.090405in}{1.513286in}}%
\pgfpathlineto{\pgfqpoint{2.088935in}{1.530951in}}%
\pgfpathlineto{\pgfqpoint{2.077656in}{1.530311in}}%
\pgfpathlineto{\pgfqpoint{2.076668in}{1.547607in}}%
\pgfpathlineto{\pgfqpoint{2.069326in}{1.547117in}}%
\pgfpathlineto{\pgfqpoint{2.067108in}{1.552783in}}%
\pgfpathlineto{\pgfqpoint{2.066412in}{1.564272in}}%
\pgfpathlineto{\pgfqpoint{2.078017in}{1.564986in}}%
\pgfpathlineto{\pgfqpoint{2.083707in}{1.565240in}}%
\pgfpathlineto{\pgfqpoint{2.084400in}{1.553797in}}%
\pgfpathlineto{\pgfqpoint{2.090120in}{1.553996in}}%
\pgfpathlineto{\pgfqpoint{2.093706in}{1.548428in}}%
\pgfpathlineto{\pgfqpoint{2.094036in}{1.542696in}}%
\pgfpathlineto{\pgfqpoint{2.111077in}{1.540159in}}%
\pgfpathlineto{\pgfqpoint{2.102036in}{1.523219in}}%
\pgfpathlineto{\pgfqpoint{2.100053in}{1.512797in}}%
\pgfpathlineto{\pgfqpoint{2.104771in}{1.510691in}}%
\pgfpathlineto{\pgfqpoint{2.108185in}{1.516689in}}%
\pgfpathlineto{\pgfqpoint{2.112193in}{1.518509in}}%
\pgfpathlineto{\pgfqpoint{2.118815in}{1.532207in}}%
\pgfpathlineto{\pgfqpoint{2.124890in}{1.534323in}}%
\pgfpathlineto{\pgfqpoint{2.127996in}{1.537667in}}%
\pgfpathlineto{\pgfqpoint{2.133801in}{1.550138in}}%
\pgfpathlineto{\pgfqpoint{2.134382in}{1.555481in}}%
\pgfpathlineto{\pgfqpoint{2.143696in}{1.559230in}}%
\pgfpathlineto{\pgfqpoint{2.143649in}{1.552093in}}%
\pgfpathlineto{\pgfqpoint{2.137836in}{1.542472in}}%
\pgfpathlineto{\pgfqpoint{2.137692in}{1.536286in}}%
\pgfpathlineto{\pgfqpoint{2.134764in}{1.533978in}}%
\pgfpathlineto{\pgfqpoint{2.128245in}{1.517619in}}%
\pgfpathlineto{\pgfqpoint{2.124853in}{1.498938in}}%
\pgfpathclose%
\pgfusepath{fill}%
\end{pgfscope}%
\begin{pgfscope}%
\pgfpathrectangle{\pgfqpoint{0.100000in}{0.100000in}}{\pgfqpoint{3.007045in}{1.925000in}}%
\pgfusepath{clip}%
\pgfsetbuttcap%
\pgfsetmiterjoin%
\definecolor{currentfill}{rgb}{0.802307,0.876048,0.945867}%
\pgfsetfillcolor{currentfill}%
\pgfsetlinewidth{0.000000pt}%
\definecolor{currentstroke}{rgb}{0.000000,0.000000,0.000000}%
\pgfsetstrokecolor{currentstroke}%
\pgfsetstrokeopacity{0.000000}%
\pgfsetdash{}{0pt}%
\pgfpathmoveto{\pgfqpoint{2.636603in}{1.298784in}}%
\pgfpathlineto{\pgfqpoint{2.631148in}{1.284175in}}%
\pgfpathlineto{\pgfqpoint{2.627614in}{1.286206in}}%
\pgfpathlineto{\pgfqpoint{2.616493in}{1.288353in}}%
\pgfpathlineto{\pgfqpoint{2.610752in}{1.291249in}}%
\pgfpathlineto{\pgfqpoint{2.608820in}{1.295800in}}%
\pgfpathlineto{\pgfqpoint{2.612866in}{1.308899in}}%
\pgfpathlineto{\pgfqpoint{2.608640in}{1.313232in}}%
\pgfpathlineto{\pgfqpoint{2.609389in}{1.317404in}}%
\pgfpathlineto{\pgfqpoint{2.606011in}{1.320523in}}%
\pgfpathlineto{\pgfqpoint{2.599096in}{1.322608in}}%
\pgfpathlineto{\pgfqpoint{2.576932in}{1.318538in}}%
\pgfpathlineto{\pgfqpoint{2.574642in}{1.330275in}}%
\pgfpathlineto{\pgfqpoint{2.554469in}{1.327047in}}%
\pgfpathlineto{\pgfqpoint{2.552895in}{1.336088in}}%
\pgfpathlineto{\pgfqpoint{2.549986in}{1.354398in}}%
\pgfpathlineto{\pgfqpoint{2.555437in}{1.355961in}}%
\pgfpathlineto{\pgfqpoint{2.561808in}{1.358520in}}%
\pgfpathlineto{\pgfqpoint{2.573880in}{1.354042in}}%
\pgfpathlineto{\pgfqpoint{2.605715in}{1.356835in}}%
\pgfpathlineto{\pgfqpoint{2.609177in}{1.353242in}}%
\pgfpathlineto{\pgfqpoint{2.621003in}{1.357074in}}%
\pgfpathlineto{\pgfqpoint{2.623988in}{1.352894in}}%
\pgfpathlineto{\pgfqpoint{2.627962in}{1.353837in}}%
\pgfpathlineto{\pgfqpoint{2.637385in}{1.345841in}}%
\pgfpathlineto{\pgfqpoint{2.645166in}{1.349038in}}%
\pgfpathlineto{\pgfqpoint{2.654171in}{1.354581in}}%
\pgfpathlineto{\pgfqpoint{2.645950in}{1.339764in}}%
\pgfpathlineto{\pgfqpoint{2.633083in}{1.330389in}}%
\pgfpathlineto{\pgfqpoint{2.626236in}{1.316064in}}%
\pgfpathlineto{\pgfqpoint{2.628110in}{1.312979in}}%
\pgfpathlineto{\pgfqpoint{2.625195in}{1.307053in}}%
\pgfpathlineto{\pgfqpoint{2.627758in}{1.303889in}}%
\pgfpathlineto{\pgfqpoint{2.632407in}{1.305014in}}%
\pgfpathlineto{\pgfqpoint{2.636603in}{1.298784in}}%
\pgfpathclose%
\pgfusepath{fill}%
\end{pgfscope}%
\begin{pgfscope}%
\pgfpathrectangle{\pgfqpoint{0.100000in}{0.100000in}}{\pgfqpoint{3.007045in}{1.925000in}}%
\pgfusepath{clip}%
\pgfsetbuttcap%
\pgfsetmiterjoin%
\definecolor{currentfill}{rgb}{0.412042,0.677186,0.836263}%
\pgfsetfillcolor{currentfill}%
\pgfsetlinewidth{0.000000pt}%
\definecolor{currentstroke}{rgb}{0.000000,0.000000,0.000000}%
\pgfsetstrokecolor{currentstroke}%
\pgfsetstrokeopacity{0.000000}%
\pgfsetdash{}{0pt}%
\pgfpathmoveto{\pgfqpoint{1.847419in}{1.345161in}}%
\pgfpathlineto{\pgfqpoint{1.813145in}{1.344757in}}%
\pgfpathlineto{\pgfqpoint{1.779241in}{1.344556in}}%
\pgfpathlineto{\pgfqpoint{1.779293in}{1.361999in}}%
\pgfpathlineto{\pgfqpoint{1.776477in}{1.362015in}}%
\pgfpathlineto{\pgfqpoint{1.776480in}{1.390760in}}%
\pgfpathlineto{\pgfqpoint{1.822052in}{1.390899in}}%
\pgfpathlineto{\pgfqpoint{1.844917in}{1.391266in}}%
\pgfpathlineto{\pgfqpoint{1.845372in}{1.362394in}}%
\pgfpathlineto{\pgfqpoint{1.847146in}{1.362415in}}%
\pgfpathlineto{\pgfqpoint{1.847419in}{1.345161in}}%
\pgfpathclose%
\pgfusepath{fill}%
\end{pgfscope}%
\begin{pgfscope}%
\pgfpathrectangle{\pgfqpoint{0.100000in}{0.100000in}}{\pgfqpoint{3.007045in}{1.925000in}}%
\pgfusepath{clip}%
\pgfsetbuttcap%
\pgfsetmiterjoin%
\definecolor{currentfill}{rgb}{0.554510,0.756417,0.868312}%
\pgfsetfillcolor{currentfill}%
\pgfsetlinewidth{0.000000pt}%
\definecolor{currentstroke}{rgb}{0.000000,0.000000,0.000000}%
\pgfsetstrokecolor{currentstroke}%
\pgfsetstrokeopacity{0.000000}%
\pgfsetdash{}{0pt}%
\pgfpathmoveto{\pgfqpoint{2.905539in}{1.431014in}}%
\pgfpathlineto{\pgfqpoint{2.901522in}{1.431102in}}%
\pgfpathlineto{\pgfqpoint{2.893781in}{1.426472in}}%
\pgfpathlineto{\pgfqpoint{2.889192in}{1.426210in}}%
\pgfpathlineto{\pgfqpoint{2.881149in}{1.420623in}}%
\pgfpathlineto{\pgfqpoint{2.877803in}{1.421007in}}%
\pgfpathlineto{\pgfqpoint{2.855094in}{1.414428in}}%
\pgfpathlineto{\pgfqpoint{2.824324in}{1.387968in}}%
\pgfpathlineto{\pgfqpoint{2.819156in}{1.394508in}}%
\pgfpathlineto{\pgfqpoint{2.829259in}{1.404526in}}%
\pgfpathlineto{\pgfqpoint{2.824695in}{1.409031in}}%
\pgfpathlineto{\pgfqpoint{2.822125in}{1.424128in}}%
\pgfpathlineto{\pgfqpoint{2.816097in}{1.458112in}}%
\pgfpathlineto{\pgfqpoint{2.836904in}{1.462520in}}%
\pgfpathlineto{\pgfqpoint{2.880647in}{1.472875in}}%
\pgfpathlineto{\pgfqpoint{2.896973in}{1.475592in}}%
\pgfpathlineto{\pgfqpoint{2.903419in}{1.453280in}}%
\pgfpathlineto{\pgfqpoint{2.906901in}{1.437874in}}%
\pgfpathlineto{\pgfqpoint{2.905539in}{1.431014in}}%
\pgfpathclose%
\pgfusepath{fill}%
\end{pgfscope}%
\begin{pgfscope}%
\pgfpathrectangle{\pgfqpoint{0.100000in}{0.100000in}}{\pgfqpoint{3.007045in}{1.925000in}}%
\pgfusepath{clip}%
\pgfsetbuttcap%
\pgfsetmiterjoin%
\definecolor{currentfill}{rgb}{0.498039,0.725413,0.856132}%
\pgfsetfillcolor{currentfill}%
\pgfsetlinewidth{0.000000pt}%
\definecolor{currentstroke}{rgb}{0.000000,0.000000,0.000000}%
\pgfsetstrokecolor{currentstroke}%
\pgfsetstrokeopacity{0.000000}%
\pgfsetdash{}{0pt}%
\pgfpathmoveto{\pgfqpoint{1.977807in}{1.222266in}}%
\pgfpathlineto{\pgfqpoint{1.977971in}{1.216245in}}%
\pgfpathlineto{\pgfqpoint{1.948122in}{1.215494in}}%
\pgfpathlineto{\pgfqpoint{1.948406in}{1.218688in}}%
\pgfpathlineto{\pgfqpoint{1.925598in}{1.218425in}}%
\pgfpathlineto{\pgfqpoint{1.925211in}{1.241392in}}%
\pgfpathlineto{\pgfqpoint{1.935891in}{1.242279in}}%
\pgfpathlineto{\pgfqpoint{1.935244in}{1.278471in}}%
\pgfpathlineto{\pgfqpoint{1.952380in}{1.279020in}}%
\pgfpathlineto{\pgfqpoint{1.952591in}{1.273282in}}%
\pgfpathlineto{\pgfqpoint{1.973511in}{1.274071in}}%
\pgfpathlineto{\pgfqpoint{1.981668in}{1.274180in}}%
\pgfpathlineto{\pgfqpoint{1.982776in}{1.245666in}}%
\pgfpathlineto{\pgfqpoint{1.977034in}{1.245635in}}%
\pgfpathlineto{\pgfqpoint{1.977807in}{1.222266in}}%
\pgfpathclose%
\pgfusepath{fill}%
\end{pgfscope}%
\begin{pgfscope}%
\pgfpathrectangle{\pgfqpoint{0.100000in}{0.100000in}}{\pgfqpoint{3.007045in}{1.925000in}}%
\pgfusepath{clip}%
\pgfsetbuttcap%
\pgfsetmiterjoin%
\definecolor{currentfill}{rgb}{0.510588,0.732303,0.858839}%
\pgfsetfillcolor{currentfill}%
\pgfsetlinewidth{0.000000pt}%
\definecolor{currentstroke}{rgb}{0.000000,0.000000,0.000000}%
\pgfsetstrokecolor{currentstroke}%
\pgfsetstrokeopacity{0.000000}%
\pgfsetdash{}{0pt}%
\pgfpathmoveto{\pgfqpoint{2.348646in}{0.902478in}}%
\pgfpathlineto{\pgfqpoint{2.332547in}{0.900599in}}%
\pgfpathlineto{\pgfqpoint{2.313442in}{0.898402in}}%
\pgfpathlineto{\pgfqpoint{2.309785in}{0.907283in}}%
\pgfpathlineto{\pgfqpoint{2.313145in}{0.917931in}}%
\pgfpathlineto{\pgfqpoint{2.308274in}{0.925835in}}%
\pgfpathlineto{\pgfqpoint{2.313025in}{0.928226in}}%
\pgfpathlineto{\pgfqpoint{2.319657in}{0.939932in}}%
\pgfpathlineto{\pgfqpoint{2.318851in}{0.944884in}}%
\pgfpathlineto{\pgfqpoint{2.321351in}{0.950113in}}%
\pgfpathlineto{\pgfqpoint{2.327024in}{0.948030in}}%
\pgfpathlineto{\pgfqpoint{2.329679in}{0.943109in}}%
\pgfpathlineto{\pgfqpoint{2.341668in}{0.946890in}}%
\pgfpathlineto{\pgfqpoint{2.354830in}{0.942630in}}%
\pgfpathlineto{\pgfqpoint{2.364223in}{0.935952in}}%
\pgfpathlineto{\pgfqpoint{2.361650in}{0.931386in}}%
\pgfpathlineto{\pgfqpoint{2.361956in}{0.924269in}}%
\pgfpathlineto{\pgfqpoint{2.356471in}{0.920502in}}%
\pgfpathlineto{\pgfqpoint{2.351726in}{0.921503in}}%
\pgfpathlineto{\pgfqpoint{2.348504in}{0.918229in}}%
\pgfpathlineto{\pgfqpoint{2.348646in}{0.902478in}}%
\pgfpathclose%
\pgfusepath{fill}%
\end{pgfscope}%
\begin{pgfscope}%
\pgfpathrectangle{\pgfqpoint{0.100000in}{0.100000in}}{\pgfqpoint{3.007045in}{1.925000in}}%
\pgfusepath{clip}%
\pgfsetbuttcap%
\pgfsetmiterjoin%
\definecolor{currentfill}{rgb}{0.541961,0.749527,0.865606}%
\pgfsetfillcolor{currentfill}%
\pgfsetlinewidth{0.000000pt}%
\definecolor{currentstroke}{rgb}{0.000000,0.000000,0.000000}%
\pgfsetstrokecolor{currentstroke}%
\pgfsetstrokeopacity{0.000000}%
\pgfsetdash{}{0pt}%
\pgfpathmoveto{\pgfqpoint{1.280556in}{0.903442in}}%
\pgfpathlineto{\pgfqpoint{1.281006in}{0.908273in}}%
\pgfpathlineto{\pgfqpoint{1.218495in}{0.915021in}}%
\pgfpathlineto{\pgfqpoint{1.217220in}{0.903588in}}%
\pgfpathlineto{\pgfqpoint{1.194470in}{0.906193in}}%
\pgfpathlineto{\pgfqpoint{1.201592in}{0.967633in}}%
\pgfpathlineto{\pgfqpoint{1.209230in}{0.966315in}}%
\pgfpathlineto{\pgfqpoint{1.218512in}{0.975812in}}%
\pgfpathlineto{\pgfqpoint{1.221151in}{0.983550in}}%
\pgfpathlineto{\pgfqpoint{1.223675in}{0.983885in}}%
\pgfpathlineto{\pgfqpoint{1.225457in}{0.993376in}}%
\pgfpathlineto{\pgfqpoint{1.225227in}{1.010762in}}%
\pgfpathlineto{\pgfqpoint{1.231839in}{1.013157in}}%
\pgfpathlineto{\pgfqpoint{1.235393in}{1.026018in}}%
\pgfpathlineto{\pgfqpoint{1.235453in}{1.031177in}}%
\pgfpathlineto{\pgfqpoint{1.238883in}{1.030797in}}%
\pgfpathlineto{\pgfqpoint{1.277959in}{1.026457in}}%
\pgfpathlineto{\pgfqpoint{1.298942in}{1.024507in}}%
\pgfpathlineto{\pgfqpoint{1.293808in}{0.973457in}}%
\pgfpathlineto{\pgfqpoint{1.271190in}{0.975810in}}%
\pgfpathlineto{\pgfqpoint{1.273752in}{0.963960in}}%
\pgfpathlineto{\pgfqpoint{1.273425in}{0.954257in}}%
\pgfpathlineto{\pgfqpoint{1.271875in}{0.946736in}}%
\pgfpathlineto{\pgfqpoint{1.290299in}{0.944870in}}%
\pgfpathlineto{\pgfqpoint{1.292797in}{0.945962in}}%
\pgfpathlineto{\pgfqpoint{1.303832in}{0.919724in}}%
\pgfpathlineto{\pgfqpoint{1.308509in}{0.919262in}}%
\pgfpathlineto{\pgfqpoint{1.307382in}{0.907359in}}%
\pgfpathlineto{\pgfqpoint{1.295559in}{0.908588in}}%
\pgfpathlineto{\pgfqpoint{1.280556in}{0.903442in}}%
\pgfpathclose%
\pgfusepath{fill}%
\end{pgfscope}%
\begin{pgfscope}%
\pgfpathrectangle{\pgfqpoint{0.100000in}{0.100000in}}{\pgfqpoint{3.007045in}{1.925000in}}%
\pgfusepath{clip}%
\pgfsetbuttcap%
\pgfsetmiterjoin%
\definecolor{currentfill}{rgb}{0.326290,0.618624,0.802799}%
\pgfsetfillcolor{currentfill}%
\pgfsetlinewidth{0.000000pt}%
\definecolor{currentstroke}{rgb}{0.000000,0.000000,0.000000}%
\pgfsetstrokecolor{currentstroke}%
\pgfsetstrokeopacity{0.000000}%
\pgfsetdash{}{0pt}%
\pgfpathmoveto{\pgfqpoint{1.905724in}{1.300373in}}%
\pgfpathlineto{\pgfqpoint{1.854836in}{1.299029in}}%
\pgfpathlineto{\pgfqpoint{1.853741in}{1.305138in}}%
\pgfpathlineto{\pgfqpoint{1.853457in}{1.322433in}}%
\pgfpathlineto{\pgfqpoint{1.859156in}{1.322514in}}%
\pgfpathlineto{\pgfqpoint{1.858723in}{1.345342in}}%
\pgfpathlineto{\pgfqpoint{1.869963in}{1.345597in}}%
\pgfpathlineto{\pgfqpoint{1.881395in}{1.345922in}}%
\pgfpathlineto{\pgfqpoint{1.881224in}{1.351626in}}%
\pgfpathlineto{\pgfqpoint{1.904114in}{1.352301in}}%
\pgfpathlineto{\pgfqpoint{1.905724in}{1.300373in}}%
\pgfpathclose%
\pgfusepath{fill}%
\end{pgfscope}%
\begin{pgfscope}%
\pgfpathrectangle{\pgfqpoint{0.100000in}{0.100000in}}{\pgfqpoint{3.007045in}{1.925000in}}%
\pgfusepath{clip}%
\pgfsetbuttcap%
\pgfsetmiterjoin%
\definecolor{currentfill}{rgb}{0.429020,0.687520,0.841246}%
\pgfsetfillcolor{currentfill}%
\pgfsetlinewidth{0.000000pt}%
\definecolor{currentstroke}{rgb}{0.000000,0.000000,0.000000}%
\pgfsetstrokecolor{currentstroke}%
\pgfsetstrokeopacity{0.000000}%
\pgfsetdash{}{0pt}%
\pgfpathmoveto{\pgfqpoint{0.615468in}{1.284198in}}%
\pgfpathlineto{\pgfqpoint{0.597849in}{1.283744in}}%
\pgfpathlineto{\pgfqpoint{0.596506in}{1.278590in}}%
\pgfpathlineto{\pgfqpoint{0.614328in}{1.243704in}}%
\pgfpathlineto{\pgfqpoint{0.571645in}{1.215280in}}%
\pgfpathlineto{\pgfqpoint{0.567833in}{1.216455in}}%
\pgfpathlineto{\pgfqpoint{0.540209in}{1.259221in}}%
\pgfpathlineto{\pgfqpoint{0.552667in}{1.255842in}}%
\pgfpathlineto{\pgfqpoint{0.560395in}{1.283604in}}%
\pgfpathlineto{\pgfqpoint{0.554814in}{1.285052in}}%
\pgfpathlineto{\pgfqpoint{0.556383in}{1.290907in}}%
\pgfpathlineto{\pgfqpoint{0.562826in}{1.298061in}}%
\pgfpathlineto{\pgfqpoint{0.615468in}{1.284198in}}%
\pgfpathclose%
\pgfusepath{fill}%
\end{pgfscope}%
\begin{pgfscope}%
\pgfpathrectangle{\pgfqpoint{0.100000in}{0.100000in}}{\pgfqpoint{3.007045in}{1.925000in}}%
\pgfusepath{clip}%
\pgfsetbuttcap%
\pgfsetmiterjoin%
\definecolor{currentfill}{rgb}{0.676817,0.816471,0.902376}%
\pgfsetfillcolor{currentfill}%
\pgfsetlinewidth{0.000000pt}%
\definecolor{currentstroke}{rgb}{0.000000,0.000000,0.000000}%
\pgfsetstrokecolor{currentstroke}%
\pgfsetstrokeopacity{0.000000}%
\pgfsetdash{}{0pt}%
\pgfpathmoveto{\pgfqpoint{2.514084in}{1.019211in}}%
\pgfpathlineto{\pgfqpoint{2.506602in}{1.020237in}}%
\pgfpathlineto{\pgfqpoint{2.500787in}{1.015480in}}%
\pgfpathlineto{\pgfqpoint{2.492311in}{1.025615in}}%
\pgfpathlineto{\pgfqpoint{2.492133in}{1.027383in}}%
\pgfpathlineto{\pgfqpoint{2.475027in}{1.025720in}}%
\pgfpathlineto{\pgfqpoint{2.476402in}{1.027509in}}%
\pgfpathlineto{\pgfqpoint{2.481176in}{1.031217in}}%
\pgfpathlineto{\pgfqpoint{2.481841in}{1.034709in}}%
\pgfpathlineto{\pgfqpoint{2.495062in}{1.040390in}}%
\pgfpathlineto{\pgfqpoint{2.503517in}{1.042080in}}%
\pgfpathlineto{\pgfqpoint{2.505363in}{1.044619in}}%
\pgfpathlineto{\pgfqpoint{2.521063in}{1.052157in}}%
\pgfpathlineto{\pgfqpoint{2.526060in}{1.056473in}}%
\pgfpathlineto{\pgfqpoint{2.537236in}{1.043373in}}%
\pgfpathlineto{\pgfqpoint{2.537514in}{1.041038in}}%
\pgfpathlineto{\pgfqpoint{2.529927in}{1.036274in}}%
\pgfpathlineto{\pgfqpoint{2.531083in}{1.032312in}}%
\pgfpathlineto{\pgfqpoint{2.515896in}{1.030173in}}%
\pgfpathlineto{\pgfqpoint{2.513186in}{1.021977in}}%
\pgfpathlineto{\pgfqpoint{2.514084in}{1.019211in}}%
\pgfpathclose%
\pgfusepath{fill}%
\end{pgfscope}%
\begin{pgfscope}%
\pgfpathrectangle{\pgfqpoint{0.100000in}{0.100000in}}{\pgfqpoint{3.007045in}{1.925000in}}%
\pgfusepath{clip}%
\pgfsetbuttcap%
\pgfsetmiterjoin%
\definecolor{currentfill}{rgb}{0.376732,0.653072,0.822484}%
\pgfsetfillcolor{currentfill}%
\pgfsetlinewidth{0.000000pt}%
\definecolor{currentstroke}{rgb}{0.000000,0.000000,0.000000}%
\pgfsetstrokecolor{currentstroke}%
\pgfsetstrokeopacity{0.000000}%
\pgfsetdash{}{0pt}%
\pgfpathmoveto{\pgfqpoint{2.407666in}{1.313899in}}%
\pgfpathlineto{\pgfqpoint{2.410399in}{1.309530in}}%
\pgfpathlineto{\pgfqpoint{2.413572in}{1.288426in}}%
\pgfpathlineto{\pgfqpoint{2.408904in}{1.287720in}}%
\pgfpathlineto{\pgfqpoint{2.409849in}{1.281241in}}%
\pgfpathlineto{\pgfqpoint{2.402337in}{1.279030in}}%
\pgfpathlineto{\pgfqpoint{2.389048in}{1.277134in}}%
\pgfpathlineto{\pgfqpoint{2.390704in}{1.263733in}}%
\pgfpathlineto{\pgfqpoint{2.376485in}{1.262685in}}%
\pgfpathlineto{\pgfqpoint{2.376250in}{1.266421in}}%
\pgfpathlineto{\pgfqpoint{2.359900in}{1.266193in}}%
\pgfpathlineto{\pgfqpoint{2.359392in}{1.270321in}}%
\pgfpathlineto{\pgfqpoint{2.350943in}{1.269110in}}%
\pgfpathlineto{\pgfqpoint{2.349350in}{1.280970in}}%
\pgfpathlineto{\pgfqpoint{2.364496in}{1.284008in}}%
\pgfpathlineto{\pgfqpoint{2.361935in}{1.302986in}}%
\pgfpathlineto{\pgfqpoint{2.375866in}{1.305055in}}%
\pgfpathlineto{\pgfqpoint{2.395446in}{1.307499in}}%
\pgfpathlineto{\pgfqpoint{2.394589in}{1.312199in}}%
\pgfpathlineto{\pgfqpoint{2.407666in}{1.313899in}}%
\pgfpathclose%
\pgfusepath{fill}%
\end{pgfscope}%
\begin{pgfscope}%
\pgfpathrectangle{\pgfqpoint{0.100000in}{0.100000in}}{\pgfqpoint{3.007045in}{1.925000in}}%
\pgfusepath{clip}%
\pgfsetbuttcap%
\pgfsetmiterjoin%
\definecolor{currentfill}{rgb}{0.666974,0.812288,0.898931}%
\pgfsetfillcolor{currentfill}%
\pgfsetlinewidth{0.000000pt}%
\definecolor{currentstroke}{rgb}{0.000000,0.000000,0.000000}%
\pgfsetstrokecolor{currentstroke}%
\pgfsetstrokeopacity{0.000000}%
\pgfsetdash{}{0pt}%
\pgfpathmoveto{\pgfqpoint{2.285385in}{0.933858in}}%
\pgfpathlineto{\pgfqpoint{2.292762in}{0.944860in}}%
\pgfpathlineto{\pgfqpoint{2.293939in}{0.955321in}}%
\pgfpathlineto{\pgfqpoint{2.291000in}{0.961775in}}%
\pgfpathlineto{\pgfqpoint{2.291608in}{0.970240in}}%
\pgfpathlineto{\pgfqpoint{2.296056in}{0.970309in}}%
\pgfpathlineto{\pgfqpoint{2.299083in}{0.977051in}}%
\pgfpathlineto{\pgfqpoint{2.296204in}{0.983928in}}%
\pgfpathlineto{\pgfqpoint{2.295566in}{0.990662in}}%
\pgfpathlineto{\pgfqpoint{2.298509in}{1.001042in}}%
\pgfpathlineto{\pgfqpoint{2.303650in}{1.003101in}}%
\pgfpathlineto{\pgfqpoint{2.315188in}{1.000641in}}%
\pgfpathlineto{\pgfqpoint{2.319984in}{0.992600in}}%
\pgfpathlineto{\pgfqpoint{2.317969in}{0.990753in}}%
\pgfpathlineto{\pgfqpoint{2.309261in}{0.983763in}}%
\pgfpathlineto{\pgfqpoint{2.308631in}{0.975465in}}%
\pgfpathlineto{\pgfqpoint{2.319716in}{0.965996in}}%
\pgfpathlineto{\pgfqpoint{2.322488in}{0.960571in}}%
\pgfpathlineto{\pgfqpoint{2.317579in}{0.954327in}}%
\pgfpathlineto{\pgfqpoint{2.321351in}{0.950113in}}%
\pgfpathlineto{\pgfqpoint{2.318851in}{0.944884in}}%
\pgfpathlineto{\pgfqpoint{2.319657in}{0.939932in}}%
\pgfpathlineto{\pgfqpoint{2.313025in}{0.928226in}}%
\pgfpathlineto{\pgfqpoint{2.308274in}{0.925835in}}%
\pgfpathlineto{\pgfqpoint{2.299822in}{0.927995in}}%
\pgfpathlineto{\pgfqpoint{2.297460in}{0.921002in}}%
\pgfpathlineto{\pgfqpoint{2.287005in}{0.929042in}}%
\pgfpathlineto{\pgfqpoint{2.285385in}{0.933858in}}%
\pgfpathclose%
\pgfusepath{fill}%
\end{pgfscope}%
\begin{pgfscope}%
\pgfpathrectangle{\pgfqpoint{0.100000in}{0.100000in}}{\pgfqpoint{3.007045in}{1.925000in}}%
\pgfusepath{clip}%
\pgfsetbuttcap%
\pgfsetmiterjoin%
\definecolor{currentfill}{rgb}{0.657132,0.808105,0.895486}%
\pgfsetfillcolor{currentfill}%
\pgfsetlinewidth{0.000000pt}%
\definecolor{currentstroke}{rgb}{0.000000,0.000000,0.000000}%
\pgfsetstrokecolor{currentstroke}%
\pgfsetstrokeopacity{0.000000}%
\pgfsetdash{}{0pt}%
\pgfpathmoveto{\pgfqpoint{2.742724in}{1.035580in}}%
\pgfpathlineto{\pgfqpoint{2.743345in}{1.029859in}}%
\pgfpathlineto{\pgfqpoint{2.741385in}{1.024328in}}%
\pgfpathlineto{\pgfqpoint{2.738228in}{1.024682in}}%
\pgfpathlineto{\pgfqpoint{2.735129in}{1.019199in}}%
\pgfpathlineto{\pgfqpoint{2.731963in}{1.024371in}}%
\pgfpathlineto{\pgfqpoint{2.728149in}{1.020939in}}%
\pgfpathlineto{\pgfqpoint{2.722671in}{1.024298in}}%
\pgfpathlineto{\pgfqpoint{2.716657in}{1.024638in}}%
\pgfpathlineto{\pgfqpoint{2.715659in}{1.028449in}}%
\pgfpathlineto{\pgfqpoint{2.709028in}{1.030469in}}%
\pgfpathlineto{\pgfqpoint{2.705428in}{1.026285in}}%
\pgfpathlineto{\pgfqpoint{2.698517in}{1.027123in}}%
\pgfpathlineto{\pgfqpoint{2.697499in}{1.030502in}}%
\pgfpathlineto{\pgfqpoint{2.689119in}{1.033149in}}%
\pgfpathlineto{\pgfqpoint{2.679615in}{1.030971in}}%
\pgfpathlineto{\pgfqpoint{2.671365in}{1.033005in}}%
\pgfpathlineto{\pgfqpoint{2.674093in}{1.045577in}}%
\pgfpathlineto{\pgfqpoint{2.672729in}{1.056217in}}%
\pgfpathlineto{\pgfqpoint{2.679621in}{1.057585in}}%
\pgfpathlineto{\pgfqpoint{2.683306in}{1.069582in}}%
\pgfpathlineto{\pgfqpoint{2.680930in}{1.081273in}}%
\pgfpathlineto{\pgfqpoint{2.685692in}{1.079312in}}%
\pgfpathlineto{\pgfqpoint{2.691645in}{1.081236in}}%
\pgfpathlineto{\pgfqpoint{2.691815in}{1.071736in}}%
\pgfpathlineto{\pgfqpoint{2.695089in}{1.071478in}}%
\pgfpathlineto{\pgfqpoint{2.701345in}{1.062144in}}%
\pgfpathlineto{\pgfqpoint{2.723871in}{1.066185in}}%
\pgfpathlineto{\pgfqpoint{2.724911in}{1.056794in}}%
\pgfpathlineto{\pgfqpoint{2.733124in}{1.055886in}}%
\pgfpathlineto{\pgfqpoint{2.738529in}{1.052500in}}%
\pgfpathlineto{\pgfqpoint{2.738199in}{1.043269in}}%
\pgfpathlineto{\pgfqpoint{2.742724in}{1.035580in}}%
\pgfpathclose%
\pgfusepath{fill}%
\end{pgfscope}%
\begin{pgfscope}%
\pgfpathrectangle{\pgfqpoint{0.100000in}{0.100000in}}{\pgfqpoint{3.007045in}{1.925000in}}%
\pgfusepath{clip}%
\pgfsetbuttcap%
\pgfsetmiterjoin%
\definecolor{currentfill}{rgb}{0.175087,0.488812,0.733333}%
\pgfsetfillcolor{currentfill}%
\pgfsetlinewidth{0.000000pt}%
\definecolor{currentstroke}{rgb}{0.000000,0.000000,0.000000}%
\pgfsetstrokecolor{currentstroke}%
\pgfsetstrokeopacity{0.000000}%
\pgfsetdash{}{0pt}%
\pgfpathmoveto{\pgfqpoint{1.538682in}{0.941826in}}%
\pgfpathlineto{\pgfqpoint{1.532268in}{0.939536in}}%
\pgfpathlineto{\pgfqpoint{1.528739in}{0.932611in}}%
\pgfpathlineto{\pgfqpoint{1.520249in}{0.932634in}}%
\pgfpathlineto{\pgfqpoint{1.515519in}{0.937448in}}%
\pgfpathlineto{\pgfqpoint{1.510237in}{0.939913in}}%
\pgfpathlineto{\pgfqpoint{1.505202in}{0.934838in}}%
\pgfpathlineto{\pgfqpoint{1.505836in}{0.946347in}}%
\pgfpathlineto{\pgfqpoint{1.507272in}{0.975589in}}%
\pgfpathlineto{\pgfqpoint{1.509153in}{1.008633in}}%
\pgfpathlineto{\pgfqpoint{1.533400in}{1.007220in}}%
\pgfpathlineto{\pgfqpoint{1.537864in}{1.006990in}}%
\pgfpathlineto{\pgfqpoint{1.541765in}{0.998932in}}%
\pgfpathlineto{\pgfqpoint{1.547875in}{0.992381in}}%
\pgfpathlineto{\pgfqpoint{1.553847in}{0.991951in}}%
\pgfpathlineto{\pgfqpoint{1.562805in}{0.979407in}}%
\pgfpathlineto{\pgfqpoint{1.561768in}{0.950581in}}%
\pgfpathlineto{\pgfqpoint{1.539102in}{0.951802in}}%
\pgfpathlineto{\pgfqpoint{1.538682in}{0.941826in}}%
\pgfpathclose%
\pgfusepath{fill}%
\end{pgfscope}%
\begin{pgfscope}%
\pgfpathrectangle{\pgfqpoint{0.100000in}{0.100000in}}{\pgfqpoint{3.007045in}{1.925000in}}%
\pgfusepath{clip}%
\pgfsetbuttcap%
\pgfsetmiterjoin%
\definecolor{currentfill}{rgb}{0.336378,0.625513,0.806736}%
\pgfsetfillcolor{currentfill}%
\pgfsetlinewidth{0.000000pt}%
\definecolor{currentstroke}{rgb}{0.000000,0.000000,0.000000}%
\pgfsetstrokecolor{currentstroke}%
\pgfsetstrokeopacity{0.000000}%
\pgfsetdash{}{0pt}%
\pgfpathmoveto{\pgfqpoint{2.585937in}{0.924191in}}%
\pgfpathlineto{\pgfqpoint{2.551891in}{0.919345in}}%
\pgfpathlineto{\pgfqpoint{2.562238in}{0.907450in}}%
\pgfpathlineto{\pgfqpoint{2.555178in}{0.902655in}}%
\pgfpathlineto{\pgfqpoint{2.555983in}{0.898307in}}%
\pgfpathlineto{\pgfqpoint{2.550816in}{0.897393in}}%
\pgfpathlineto{\pgfqpoint{2.545027in}{0.899687in}}%
\pgfpathlineto{\pgfqpoint{2.538562in}{0.893175in}}%
\pgfpathlineto{\pgfqpoint{2.537783in}{0.898762in}}%
\pgfpathlineto{\pgfqpoint{2.508172in}{0.896146in}}%
\pgfpathlineto{\pgfqpoint{2.505466in}{0.902631in}}%
\pgfpathlineto{\pgfqpoint{2.503825in}{0.913860in}}%
\pgfpathlineto{\pgfqpoint{2.500183in}{0.925858in}}%
\pgfpathlineto{\pgfqpoint{2.504209in}{0.927233in}}%
\pgfpathlineto{\pgfqpoint{2.505332in}{0.935315in}}%
\pgfpathlineto{\pgfqpoint{2.523780in}{0.937038in}}%
\pgfpathlineto{\pgfqpoint{2.523035in}{0.944991in}}%
\pgfpathlineto{\pgfqpoint{2.524892in}{0.952370in}}%
\pgfpathlineto{\pgfqpoint{2.523665in}{0.961730in}}%
\pgfpathlineto{\pgfqpoint{2.535380in}{0.962820in}}%
\pgfpathlineto{\pgfqpoint{2.531826in}{0.973469in}}%
\pgfpathlineto{\pgfqpoint{2.533363in}{0.985590in}}%
\pgfpathlineto{\pgfqpoint{2.537459in}{0.985766in}}%
\pgfpathlineto{\pgfqpoint{2.547659in}{0.980532in}}%
\pgfpathlineto{\pgfqpoint{2.555149in}{0.977464in}}%
\pgfpathlineto{\pgfqpoint{2.562648in}{0.972115in}}%
\pgfpathlineto{\pgfqpoint{2.564813in}{0.967383in}}%
\pgfpathlineto{\pgfqpoint{2.569136in}{0.965197in}}%
\pgfpathlineto{\pgfqpoint{2.573216in}{0.959889in}}%
\pgfpathlineto{\pgfqpoint{2.571933in}{0.952085in}}%
\pgfpathlineto{\pgfqpoint{2.575191in}{0.947306in}}%
\pgfpathlineto{\pgfqpoint{2.575141in}{0.942834in}}%
\pgfpathlineto{\pgfqpoint{2.581758in}{0.944406in}}%
\pgfpathlineto{\pgfqpoint{2.586120in}{0.939582in}}%
\pgfpathlineto{\pgfqpoint{2.589091in}{0.931283in}}%
\pgfpathlineto{\pgfqpoint{2.585937in}{0.924191in}}%
\pgfpathclose%
\pgfusepath{fill}%
\end{pgfscope}%
\begin{pgfscope}%
\pgfpathrectangle{\pgfqpoint{0.100000in}{0.100000in}}{\pgfqpoint{3.007045in}{1.925000in}}%
\pgfusepath{clip}%
\pgfsetbuttcap%
\pgfsetmiterjoin%
\definecolor{currentfill}{rgb}{0.235986,0.549712,0.764706}%
\pgfsetfillcolor{currentfill}%
\pgfsetlinewidth{0.000000pt}%
\definecolor{currentstroke}{rgb}{0.000000,0.000000,0.000000}%
\pgfsetstrokecolor{currentstroke}%
\pgfsetstrokeopacity{0.000000}%
\pgfsetdash{}{0pt}%
\pgfpathmoveto{\pgfqpoint{1.712393in}{0.726751in}}%
\pgfpathlineto{\pgfqpoint{1.700123in}{0.727126in}}%
\pgfpathlineto{\pgfqpoint{1.700709in}{0.760907in}}%
\pgfpathlineto{\pgfqpoint{1.724887in}{0.764449in}}%
\pgfpathlineto{\pgfqpoint{1.725337in}{0.792070in}}%
\pgfpathlineto{\pgfqpoint{1.730486in}{0.795042in}}%
\pgfpathlineto{\pgfqpoint{1.740181in}{0.799040in}}%
\pgfpathlineto{\pgfqpoint{1.742020in}{0.795983in}}%
\pgfpathlineto{\pgfqpoint{1.749934in}{0.794479in}}%
\pgfpathlineto{\pgfqpoint{1.756132in}{0.794865in}}%
\pgfpathlineto{\pgfqpoint{1.760297in}{0.800742in}}%
\pgfpathlineto{\pgfqpoint{1.774881in}{0.791133in}}%
\pgfpathlineto{\pgfqpoint{1.779488in}{0.786480in}}%
\pgfpathlineto{\pgfqpoint{1.786299in}{0.783647in}}%
\pgfpathlineto{\pgfqpoint{1.786323in}{0.758894in}}%
\pgfpathlineto{\pgfqpoint{1.779169in}{0.763215in}}%
\pgfpathlineto{\pgfqpoint{1.767169in}{0.763420in}}%
\pgfpathlineto{\pgfqpoint{1.755274in}{0.762070in}}%
\pgfpathlineto{\pgfqpoint{1.755158in}{0.734645in}}%
\pgfpathlineto{\pgfqpoint{1.735350in}{0.734610in}}%
\pgfpathlineto{\pgfqpoint{1.736915in}{0.718690in}}%
\pgfpathlineto{\pgfqpoint{1.728058in}{0.722452in}}%
\pgfpathlineto{\pgfqpoint{1.725321in}{0.722021in}}%
\pgfpathlineto{\pgfqpoint{1.720316in}{0.726579in}}%
\pgfpathlineto{\pgfqpoint{1.712393in}{0.726751in}}%
\pgfpathclose%
\pgfusepath{fill}%
\end{pgfscope}%
\begin{pgfscope}%
\pgfpathrectangle{\pgfqpoint{0.100000in}{0.100000in}}{\pgfqpoint{3.007045in}{1.925000in}}%
\pgfusepath{clip}%
\pgfsetbuttcap%
\pgfsetmiterjoin%
\definecolor{currentfill}{rgb}{0.610980,0.787420,0.880492}%
\pgfsetfillcolor{currentfill}%
\pgfsetlinewidth{0.000000pt}%
\definecolor{currentstroke}{rgb}{0.000000,0.000000,0.000000}%
\pgfsetstrokecolor{currentstroke}%
\pgfsetstrokeopacity{0.000000}%
\pgfsetdash{}{0pt}%
\pgfpathmoveto{\pgfqpoint{2.484306in}{1.188484in}}%
\pgfpathlineto{\pgfqpoint{2.487519in}{1.178106in}}%
\pgfpathlineto{\pgfqpoint{2.485316in}{1.174226in}}%
\pgfpathlineto{\pgfqpoint{2.489719in}{1.169881in}}%
\pgfpathlineto{\pgfqpoint{2.487851in}{1.166255in}}%
\pgfpathlineto{\pgfqpoint{2.480993in}{1.155854in}}%
\pgfpathlineto{\pgfqpoint{2.466955in}{1.155210in}}%
\pgfpathlineto{\pgfqpoint{2.463450in}{1.158871in}}%
\pgfpathlineto{\pgfqpoint{2.461626in}{1.162571in}}%
\pgfpathlineto{\pgfqpoint{2.460301in}{1.177194in}}%
\pgfpathlineto{\pgfqpoint{2.463999in}{1.180859in}}%
\pgfpathlineto{\pgfqpoint{2.469733in}{1.178659in}}%
\pgfpathlineto{\pgfqpoint{2.478749in}{1.189995in}}%
\pgfpathlineto{\pgfqpoint{2.484306in}{1.188484in}}%
\pgfpathclose%
\pgfusepath{fill}%
\end{pgfscope}%
\begin{pgfscope}%
\pgfpathrectangle{\pgfqpoint{0.100000in}{0.100000in}}{\pgfqpoint{3.007045in}{1.925000in}}%
\pgfusepath{clip}%
\pgfsetbuttcap%
\pgfsetmiterjoin%
\definecolor{currentfill}{rgb}{0.117878,0.428374,0.698270}%
\pgfsetfillcolor{currentfill}%
\pgfsetlinewidth{0.000000pt}%
\definecolor{currentstroke}{rgb}{0.000000,0.000000,0.000000}%
\pgfsetstrokecolor{currentstroke}%
\pgfsetstrokeopacity{0.000000}%
\pgfsetdash{}{0pt}%
\pgfpathmoveto{\pgfqpoint{1.369270in}{1.748112in}}%
\pgfpathlineto{\pgfqpoint{1.371755in}{1.774248in}}%
\pgfpathlineto{\pgfqpoint{1.375731in}{1.814864in}}%
\pgfpathlineto{\pgfqpoint{1.424243in}{1.810279in}}%
\pgfpathlineto{\pgfqpoint{1.464391in}{1.806947in}}%
\pgfpathlineto{\pgfqpoint{1.463375in}{1.794138in}}%
\pgfpathlineto{\pgfqpoint{1.457650in}{1.794593in}}%
\pgfpathlineto{\pgfqpoint{1.457171in}{1.788804in}}%
\pgfpathlineto{\pgfqpoint{1.453609in}{1.789081in}}%
\pgfpathlineto{\pgfqpoint{1.452658in}{1.777477in}}%
\pgfpathlineto{\pgfqpoint{1.423931in}{1.779899in}}%
\pgfpathlineto{\pgfqpoint{1.422948in}{1.768339in}}%
\pgfpathlineto{\pgfqpoint{1.425483in}{1.768138in}}%
\pgfpathlineto{\pgfqpoint{1.422411in}{1.752597in}}%
\pgfpathlineto{\pgfqpoint{1.407533in}{1.753794in}}%
\pgfpathlineto{\pgfqpoint{1.403722in}{1.748968in}}%
\pgfpathlineto{\pgfqpoint{1.394175in}{1.746414in}}%
\pgfpathlineto{\pgfqpoint{1.390611in}{1.748693in}}%
\pgfpathlineto{\pgfqpoint{1.389403in}{1.754089in}}%
\pgfpathlineto{\pgfqpoint{1.386114in}{1.754885in}}%
\pgfpathlineto{\pgfqpoint{1.376795in}{1.745675in}}%
\pgfpathlineto{\pgfqpoint{1.369270in}{1.748112in}}%
\pgfpathclose%
\pgfusepath{fill}%
\end{pgfscope}%
\begin{pgfscope}%
\pgfpathrectangle{\pgfqpoint{0.100000in}{0.100000in}}{\pgfqpoint{3.007045in}{1.925000in}}%
\pgfusepath{clip}%
\pgfsetbuttcap%
\pgfsetmiterjoin%
\definecolor{currentfill}{rgb}{0.280892,0.587620,0.785083}%
\pgfsetfillcolor{currentfill}%
\pgfsetlinewidth{0.000000pt}%
\definecolor{currentstroke}{rgb}{0.000000,0.000000,0.000000}%
\pgfsetstrokecolor{currentstroke}%
\pgfsetstrokeopacity{0.000000}%
\pgfsetdash{}{0pt}%
\pgfpathmoveto{\pgfqpoint{1.330767in}{0.811238in}}%
\pgfpathlineto{\pgfqpoint{1.383417in}{0.806844in}}%
\pgfpathlineto{\pgfqpoint{1.440696in}{0.803294in}}%
\pgfpathlineto{\pgfqpoint{1.438908in}{0.774476in}}%
\pgfpathlineto{\pgfqpoint{1.437038in}{0.746267in}}%
\pgfpathlineto{\pgfqpoint{1.429539in}{0.746331in}}%
\pgfpathlineto{\pgfqpoint{1.379483in}{0.749728in}}%
\pgfpathlineto{\pgfqpoint{1.381713in}{0.778104in}}%
\pgfpathlineto{\pgfqpoint{1.327701in}{0.782533in}}%
\pgfpathlineto{\pgfqpoint{1.328991in}{0.794521in}}%
\pgfpathlineto{\pgfqpoint{1.330767in}{0.811238in}}%
\pgfpathclose%
\pgfusepath{fill}%
\end{pgfscope}%
\begin{pgfscope}%
\pgfpathrectangle{\pgfqpoint{0.100000in}{0.100000in}}{\pgfqpoint{3.007045in}{1.925000in}}%
\pgfusepath{clip}%
\pgfsetbuttcap%
\pgfsetmiterjoin%
\definecolor{currentfill}{rgb}{0.265759,0.577286,0.779177}%
\pgfsetfillcolor{currentfill}%
\pgfsetlinewidth{0.000000pt}%
\definecolor{currentstroke}{rgb}{0.000000,0.000000,0.000000}%
\pgfsetstrokecolor{currentstroke}%
\pgfsetstrokeopacity{0.000000}%
\pgfsetdash{}{0pt}%
\pgfpathmoveto{\pgfqpoint{2.637551in}{0.899069in}}%
\pgfpathlineto{\pgfqpoint{2.613555in}{0.916458in}}%
\pgfpathlineto{\pgfqpoint{2.599449in}{0.926351in}}%
\pgfpathlineto{\pgfqpoint{2.585937in}{0.924191in}}%
\pgfpathlineto{\pgfqpoint{2.589091in}{0.931283in}}%
\pgfpathlineto{\pgfqpoint{2.586120in}{0.939582in}}%
\pgfpathlineto{\pgfqpoint{2.581758in}{0.944406in}}%
\pgfpathlineto{\pgfqpoint{2.575141in}{0.942834in}}%
\pgfpathlineto{\pgfqpoint{2.575191in}{0.947306in}}%
\pgfpathlineto{\pgfqpoint{2.571933in}{0.952085in}}%
\pgfpathlineto{\pgfqpoint{2.573216in}{0.959889in}}%
\pgfpathlineto{\pgfqpoint{2.569136in}{0.965197in}}%
\pgfpathlineto{\pgfqpoint{2.564813in}{0.967383in}}%
\pgfpathlineto{\pgfqpoint{2.597898in}{0.973667in}}%
\pgfpathlineto{\pgfqpoint{2.608759in}{0.975733in}}%
\pgfpathlineto{\pgfqpoint{2.608908in}{0.972810in}}%
\pgfpathlineto{\pgfqpoint{2.618329in}{0.960663in}}%
\pgfpathlineto{\pgfqpoint{2.625940in}{0.956912in}}%
\pgfpathlineto{\pgfqpoint{2.641093in}{0.964571in}}%
\pgfpathlineto{\pgfqpoint{2.650737in}{0.964955in}}%
\pgfpathlineto{\pgfqpoint{2.650566in}{0.955387in}}%
\pgfpathlineto{\pgfqpoint{2.652361in}{0.948046in}}%
\pgfpathlineto{\pgfqpoint{2.661874in}{0.940850in}}%
\pgfpathlineto{\pgfqpoint{2.645656in}{0.937482in}}%
\pgfpathlineto{\pgfqpoint{2.640458in}{0.935526in}}%
\pgfpathlineto{\pgfqpoint{2.647207in}{0.927059in}}%
\pgfpathlineto{\pgfqpoint{2.647369in}{0.914791in}}%
\pgfpathlineto{\pgfqpoint{2.641884in}{0.909457in}}%
\pgfpathlineto{\pgfqpoint{2.637551in}{0.899069in}}%
\pgfpathclose%
\pgfusepath{fill}%
\end{pgfscope}%
\begin{pgfscope}%
\pgfpathrectangle{\pgfqpoint{0.100000in}{0.100000in}}{\pgfqpoint{3.007045in}{1.925000in}}%
\pgfusepath{clip}%
\pgfsetbuttcap%
\pgfsetmiterjoin%
\definecolor{currentfill}{rgb}{0.391865,0.663406,0.828389}%
\pgfsetfillcolor{currentfill}%
\pgfsetlinewidth{0.000000pt}%
\definecolor{currentstroke}{rgb}{0.000000,0.000000,0.000000}%
\pgfsetstrokecolor{currentstroke}%
\pgfsetstrokeopacity{0.000000}%
\pgfsetdash{}{0pt}%
\pgfpathmoveto{\pgfqpoint{1.653431in}{1.223538in}}%
\pgfpathlineto{\pgfqpoint{1.652796in}{1.200602in}}%
\pgfpathlineto{\pgfqpoint{1.607157in}{1.202049in}}%
\pgfpathlineto{\pgfqpoint{1.608008in}{1.224954in}}%
\pgfpathlineto{\pgfqpoint{1.630558in}{1.224197in}}%
\pgfpathlineto{\pgfqpoint{1.653431in}{1.223538in}}%
\pgfpathclose%
\pgfusepath{fill}%
\end{pgfscope}%
\begin{pgfscope}%
\pgfpathrectangle{\pgfqpoint{0.100000in}{0.100000in}}{\pgfqpoint{3.007045in}{1.925000in}}%
\pgfusepath{clip}%
\pgfsetbuttcap%
\pgfsetmiterjoin%
\definecolor{currentfill}{rgb}{0.454118,0.701300,0.846659}%
\pgfsetfillcolor{currentfill}%
\pgfsetlinewidth{0.000000pt}%
\definecolor{currentstroke}{rgb}{0.000000,0.000000,0.000000}%
\pgfsetstrokecolor{currentstroke}%
\pgfsetstrokeopacity{0.000000}%
\pgfsetdash{}{0pt}%
\pgfpathmoveto{\pgfqpoint{2.108338in}{0.951144in}}%
\pgfpathlineto{\pgfqpoint{2.098903in}{0.955455in}}%
\pgfpathlineto{\pgfqpoint{2.093449in}{0.959312in}}%
\pgfpathlineto{\pgfqpoint{2.083172in}{0.958841in}}%
\pgfpathlineto{\pgfqpoint{2.065703in}{0.958280in}}%
\pgfpathlineto{\pgfqpoint{2.068381in}{0.961543in}}%
\pgfpathlineto{\pgfqpoint{2.071414in}{0.971616in}}%
\pgfpathlineto{\pgfqpoint{2.068022in}{0.977336in}}%
\pgfpathlineto{\pgfqpoint{2.070320in}{0.985858in}}%
\pgfpathlineto{\pgfqpoint{2.077777in}{0.982593in}}%
\pgfpathlineto{\pgfqpoint{2.080971in}{0.989204in}}%
\pgfpathlineto{\pgfqpoint{2.078631in}{0.992336in}}%
\pgfpathlineto{\pgfqpoint{2.082341in}{0.997150in}}%
\pgfpathlineto{\pgfqpoint{2.098746in}{0.997437in}}%
\pgfpathlineto{\pgfqpoint{2.097982in}{1.008842in}}%
\pgfpathlineto{\pgfqpoint{2.115261in}{1.009765in}}%
\pgfpathlineto{\pgfqpoint{2.116101in}{0.997143in}}%
\pgfpathlineto{\pgfqpoint{2.117091in}{0.980740in}}%
\pgfpathlineto{\pgfqpoint{2.115651in}{0.980661in}}%
\pgfpathlineto{\pgfqpoint{2.116626in}{0.955211in}}%
\pgfpathlineto{\pgfqpoint{2.110066in}{0.954874in}}%
\pgfpathlineto{\pgfqpoint{2.108338in}{0.951144in}}%
\pgfpathclose%
\pgfusepath{fill}%
\end{pgfscope}%
\begin{pgfscope}%
\pgfpathrectangle{\pgfqpoint{0.100000in}{0.100000in}}{\pgfqpoint{3.007045in}{1.925000in}}%
\pgfusepath{clip}%
\pgfsetbuttcap%
\pgfsetmiterjoin%
\definecolor{currentfill}{rgb}{0.472941,0.711634,0.850719}%
\pgfsetfillcolor{currentfill}%
\pgfsetlinewidth{0.000000pt}%
\definecolor{currentstroke}{rgb}{0.000000,0.000000,0.000000}%
\pgfsetstrokecolor{currentstroke}%
\pgfsetstrokeopacity{0.000000}%
\pgfsetdash{}{0pt}%
\pgfpathmoveto{\pgfqpoint{1.184259in}{1.303392in}}%
\pgfpathlineto{\pgfqpoint{1.210701in}{1.299766in}}%
\pgfpathlineto{\pgfqpoint{1.213860in}{1.325677in}}%
\pgfpathlineto{\pgfqpoint{1.226347in}{1.324123in}}%
\pgfpathlineto{\pgfqpoint{1.231571in}{1.369429in}}%
\pgfpathlineto{\pgfqpoint{1.234424in}{1.392114in}}%
\pgfpathlineto{\pgfqpoint{1.259419in}{1.388949in}}%
\pgfpathlineto{\pgfqpoint{1.256175in}{1.385798in}}%
\pgfpathlineto{\pgfqpoint{1.255653in}{1.381012in}}%
\pgfpathlineto{\pgfqpoint{1.267095in}{1.378734in}}%
\pgfpathlineto{\pgfqpoint{1.270231in}{1.387736in}}%
\pgfpathlineto{\pgfqpoint{1.272590in}{1.387487in}}%
\pgfpathlineto{\pgfqpoint{1.269866in}{1.358232in}}%
\pgfpathlineto{\pgfqpoint{1.267098in}{1.336682in}}%
\pgfpathlineto{\pgfqpoint{1.262254in}{1.293547in}}%
\pgfpathlineto{\pgfqpoint{1.217135in}{1.298862in}}%
\pgfpathlineto{\pgfqpoint{1.216879in}{1.294670in}}%
\pgfpathlineto{\pgfqpoint{1.222250in}{1.286251in}}%
\pgfpathlineto{\pgfqpoint{1.229691in}{1.263388in}}%
\pgfpathlineto{\pgfqpoint{1.222855in}{1.255180in}}%
\pgfpathlineto{\pgfqpoint{1.220648in}{1.256986in}}%
\pgfpathlineto{\pgfqpoint{1.213800in}{1.255082in}}%
\pgfpathlineto{\pgfqpoint{1.205670in}{1.255974in}}%
\pgfpathlineto{\pgfqpoint{1.192494in}{1.261100in}}%
\pgfpathlineto{\pgfqpoint{1.189706in}{1.265642in}}%
\pgfpathlineto{\pgfqpoint{1.188332in}{1.274642in}}%
\pgfpathlineto{\pgfqpoint{1.193426in}{1.288196in}}%
\pgfpathlineto{\pgfqpoint{1.192988in}{1.292052in}}%
\pgfpathlineto{\pgfqpoint{1.183883in}{1.298271in}}%
\pgfpathlineto{\pgfqpoint{1.184259in}{1.303392in}}%
\pgfpathclose%
\pgfusepath{fill}%
\end{pgfscope}%
\begin{pgfscope}%
\pgfpathrectangle{\pgfqpoint{0.100000in}{0.100000in}}{\pgfqpoint{3.007045in}{1.925000in}}%
\pgfusepath{clip}%
\pgfsetbuttcap%
\pgfsetmiterjoin%
\definecolor{currentfill}{rgb}{0.548235,0.752972,0.866959}%
\pgfsetfillcolor{currentfill}%
\pgfsetlinewidth{0.000000pt}%
\definecolor{currentstroke}{rgb}{0.000000,0.000000,0.000000}%
\pgfsetstrokecolor{currentstroke}%
\pgfsetstrokeopacity{0.000000}%
\pgfsetdash{}{0pt}%
\pgfpathmoveto{\pgfqpoint{2.362681in}{1.482619in}}%
\pgfpathlineto{\pgfqpoint{2.338611in}{1.478397in}}%
\pgfpathlineto{\pgfqpoint{2.342275in}{1.455683in}}%
\pgfpathlineto{\pgfqpoint{2.330370in}{1.453796in}}%
\pgfpathlineto{\pgfqpoint{2.331232in}{1.448101in}}%
\pgfpathlineto{\pgfqpoint{2.326140in}{1.446645in}}%
\pgfpathlineto{\pgfqpoint{2.303651in}{1.443639in}}%
\pgfpathlineto{\pgfqpoint{2.304504in}{1.437926in}}%
\pgfpathlineto{\pgfqpoint{2.283546in}{1.435118in}}%
\pgfpathlineto{\pgfqpoint{2.280842in}{1.457266in}}%
\pgfpathlineto{\pgfqpoint{2.269523in}{1.455942in}}%
\pgfpathlineto{\pgfqpoint{2.264263in}{1.501550in}}%
\pgfpathlineto{\pgfqpoint{2.298344in}{1.505689in}}%
\pgfpathlineto{\pgfqpoint{2.313368in}{1.507664in}}%
\pgfpathlineto{\pgfqpoint{2.313689in}{1.503876in}}%
\pgfpathlineto{\pgfqpoint{2.305190in}{1.494752in}}%
\pgfpathlineto{\pgfqpoint{2.302323in}{1.494776in}}%
\pgfpathlineto{\pgfqpoint{2.298978in}{1.488134in}}%
\pgfpathlineto{\pgfqpoint{2.298537in}{1.477920in}}%
\pgfpathlineto{\pgfqpoint{2.300991in}{1.473532in}}%
\pgfpathlineto{\pgfqpoint{2.313383in}{1.469662in}}%
\pgfpathlineto{\pgfqpoint{2.329237in}{1.497116in}}%
\pgfpathlineto{\pgfqpoint{2.339139in}{1.501127in}}%
\pgfpathlineto{\pgfqpoint{2.344111in}{1.505414in}}%
\pgfpathlineto{\pgfqpoint{2.353364in}{1.501784in}}%
\pgfpathlineto{\pgfqpoint{2.360186in}{1.491680in}}%
\pgfpathlineto{\pgfqpoint{2.362681in}{1.482619in}}%
\pgfpathclose%
\pgfusepath{fill}%
\end{pgfscope}%
\begin{pgfscope}%
\pgfpathrectangle{\pgfqpoint{0.100000in}{0.100000in}}{\pgfqpoint{3.007045in}{1.925000in}}%
\pgfusepath{clip}%
\pgfsetbuttcap%
\pgfsetmiterjoin%
\definecolor{currentfill}{rgb}{0.523137,0.739193,0.861546}%
\pgfsetfillcolor{currentfill}%
\pgfsetlinewidth{0.000000pt}%
\definecolor{currentstroke}{rgb}{0.000000,0.000000,0.000000}%
\pgfsetstrokecolor{currentstroke}%
\pgfsetstrokeopacity{0.000000}%
\pgfsetdash{}{0pt}%
\pgfpathmoveto{\pgfqpoint{2.558875in}{1.172717in}}%
\pgfpathlineto{\pgfqpoint{2.558583in}{1.161971in}}%
\pgfpathlineto{\pgfqpoint{2.553841in}{1.152723in}}%
\pgfpathlineto{\pgfqpoint{2.555307in}{1.150523in}}%
\pgfpathlineto{\pgfqpoint{2.549531in}{1.143698in}}%
\pgfpathlineto{\pgfqpoint{2.548827in}{1.135761in}}%
\pgfpathlineto{\pgfqpoint{2.538421in}{1.132193in}}%
\pgfpathlineto{\pgfqpoint{2.533411in}{1.132073in}}%
\pgfpathlineto{\pgfqpoint{2.527646in}{1.135652in}}%
\pgfpathlineto{\pgfqpoint{2.526634in}{1.142930in}}%
\pgfpathlineto{\pgfqpoint{2.526995in}{1.150325in}}%
\pgfpathlineto{\pgfqpoint{2.530752in}{1.154455in}}%
\pgfpathlineto{\pgfqpoint{2.532425in}{1.163870in}}%
\pgfpathlineto{\pgfqpoint{2.525737in}{1.174087in}}%
\pgfpathlineto{\pgfqpoint{2.527409in}{1.177624in}}%
\pgfpathlineto{\pgfqpoint{2.532999in}{1.178683in}}%
\pgfpathlineto{\pgfqpoint{2.532223in}{1.188646in}}%
\pgfpathlineto{\pgfqpoint{2.534247in}{1.194214in}}%
\pgfpathlineto{\pgfqpoint{2.523965in}{1.201691in}}%
\pgfpathlineto{\pgfqpoint{2.525644in}{1.210611in}}%
\pgfpathlineto{\pgfqpoint{2.532780in}{1.212166in}}%
\pgfpathlineto{\pgfqpoint{2.538153in}{1.216728in}}%
\pgfpathlineto{\pgfqpoint{2.543719in}{1.212838in}}%
\pgfpathlineto{\pgfqpoint{2.549426in}{1.216536in}}%
\pgfpathlineto{\pgfqpoint{2.560339in}{1.213292in}}%
\pgfpathlineto{\pgfqpoint{2.567300in}{1.215781in}}%
\pgfpathlineto{\pgfqpoint{2.569988in}{1.214538in}}%
\pgfpathlineto{\pgfqpoint{2.568471in}{1.207938in}}%
\pgfpathlineto{\pgfqpoint{2.571223in}{1.207045in}}%
\pgfpathlineto{\pgfqpoint{2.569924in}{1.199083in}}%
\pgfpathlineto{\pgfqpoint{2.562770in}{1.192250in}}%
\pgfpathlineto{\pgfqpoint{2.563877in}{1.187087in}}%
\pgfpathlineto{\pgfqpoint{2.558875in}{1.172717in}}%
\pgfpathclose%
\pgfusepath{fill}%
\end{pgfscope}%
\begin{pgfscope}%
\pgfpathrectangle{\pgfqpoint{0.100000in}{0.100000in}}{\pgfqpoint{3.007045in}{1.925000in}}%
\pgfusepath{clip}%
\pgfsetbuttcap%
\pgfsetmiterjoin%
\definecolor{currentfill}{rgb}{0.541961,0.749527,0.865606}%
\pgfsetfillcolor{currentfill}%
\pgfsetlinewidth{0.000000pt}%
\definecolor{currentstroke}{rgb}{0.000000,0.000000,0.000000}%
\pgfsetstrokecolor{currentstroke}%
\pgfsetstrokeopacity{0.000000}%
\pgfsetdash{}{0pt}%
\pgfpathmoveto{\pgfqpoint{2.697553in}{1.455914in}}%
\pgfpathlineto{\pgfqpoint{2.679109in}{1.451464in}}%
\pgfpathlineto{\pgfqpoint{2.675608in}{1.465295in}}%
\pgfpathlineto{\pgfqpoint{2.656543in}{1.461343in}}%
\pgfpathlineto{\pgfqpoint{2.652395in}{1.466521in}}%
\pgfpathlineto{\pgfqpoint{2.648634in}{1.483094in}}%
\pgfpathlineto{\pgfqpoint{2.648812in}{1.490538in}}%
\pgfpathlineto{\pgfqpoint{2.644415in}{1.507385in}}%
\pgfpathlineto{\pgfqpoint{2.647780in}{1.512793in}}%
\pgfpathlineto{\pgfqpoint{2.656259in}{1.521763in}}%
\pgfpathlineto{\pgfqpoint{2.664991in}{1.524305in}}%
\pgfpathlineto{\pgfqpoint{2.664388in}{1.534234in}}%
\pgfpathlineto{\pgfqpoint{2.672204in}{1.537694in}}%
\pgfpathlineto{\pgfqpoint{2.684161in}{1.539019in}}%
\pgfpathlineto{\pgfqpoint{2.688054in}{1.525089in}}%
\pgfpathlineto{\pgfqpoint{2.699199in}{1.524078in}}%
\pgfpathlineto{\pgfqpoint{2.716233in}{1.541048in}}%
\pgfpathlineto{\pgfqpoint{2.706539in}{1.571514in}}%
\pgfpathlineto{\pgfqpoint{2.712187in}{1.569616in}}%
\pgfpathlineto{\pgfqpoint{2.721573in}{1.573055in}}%
\pgfpathlineto{\pgfqpoint{2.733723in}{1.536267in}}%
\pgfpathlineto{\pgfqpoint{2.731549in}{1.525832in}}%
\pgfpathlineto{\pgfqpoint{2.739601in}{1.524017in}}%
\pgfpathlineto{\pgfqpoint{2.742037in}{1.517084in}}%
\pgfpathlineto{\pgfqpoint{2.739690in}{1.510307in}}%
\pgfpathlineto{\pgfqpoint{2.742514in}{1.505188in}}%
\pgfpathlineto{\pgfqpoint{2.743391in}{1.496243in}}%
\pgfpathlineto{\pgfqpoint{2.737431in}{1.497269in}}%
\pgfpathlineto{\pgfqpoint{2.737175in}{1.492249in}}%
\pgfpathlineto{\pgfqpoint{2.726801in}{1.495572in}}%
\pgfpathlineto{\pgfqpoint{2.722365in}{1.491084in}}%
\pgfpathlineto{\pgfqpoint{2.719947in}{1.482951in}}%
\pgfpathlineto{\pgfqpoint{2.692078in}{1.475538in}}%
\pgfpathlineto{\pgfqpoint{2.697553in}{1.455914in}}%
\pgfpathclose%
\pgfusepath{fill}%
\end{pgfscope}%
\begin{pgfscope}%
\pgfpathrectangle{\pgfqpoint{0.100000in}{0.100000in}}{\pgfqpoint{3.007045in}{1.925000in}}%
\pgfusepath{clip}%
\pgfsetbuttcap%
\pgfsetmiterjoin%
\definecolor{currentfill}{rgb}{0.321246,0.615179,0.800830}%
\pgfsetfillcolor{currentfill}%
\pgfsetlinewidth{0.000000pt}%
\definecolor{currentstroke}{rgb}{0.000000,0.000000,0.000000}%
\pgfsetstrokecolor{currentstroke}%
\pgfsetstrokeopacity{0.000000}%
\pgfsetdash{}{0pt}%
\pgfpathmoveto{\pgfqpoint{1.779241in}{1.344556in}}%
\pgfpathlineto{\pgfqpoint{1.813145in}{1.344757in}}%
\pgfpathlineto{\pgfqpoint{1.813376in}{1.321894in}}%
\pgfpathlineto{\pgfqpoint{1.784860in}{1.321668in}}%
\pgfpathlineto{\pgfqpoint{1.744960in}{1.321802in}}%
\pgfpathlineto{\pgfqpoint{1.739290in}{1.321824in}}%
\pgfpathlineto{\pgfqpoint{1.739541in}{1.344779in}}%
\pgfpathlineto{\pgfqpoint{1.779241in}{1.344556in}}%
\pgfpathclose%
\pgfusepath{fill}%
\end{pgfscope}%
\begin{pgfscope}%
\pgfpathrectangle{\pgfqpoint{0.100000in}{0.100000in}}{\pgfqpoint{3.007045in}{1.925000in}}%
\pgfusepath{clip}%
\pgfsetbuttcap%
\pgfsetmiterjoin%
\definecolor{currentfill}{rgb}{0.516863,0.735748,0.860192}%
\pgfsetfillcolor{currentfill}%
\pgfsetlinewidth{0.000000pt}%
\definecolor{currentstroke}{rgb}{0.000000,0.000000,0.000000}%
\pgfsetstrokecolor{currentstroke}%
\pgfsetstrokeopacity{0.000000}%
\pgfsetdash{}{0pt}%
\pgfpathmoveto{\pgfqpoint{2.222669in}{0.734886in}}%
\pgfpathlineto{\pgfqpoint{2.223953in}{0.726994in}}%
\pgfpathlineto{\pgfqpoint{2.230267in}{0.714046in}}%
\pgfpathlineto{\pgfqpoint{2.225729in}{0.705930in}}%
\pgfpathlineto{\pgfqpoint{2.226858in}{0.694238in}}%
\pgfpathlineto{\pgfqpoint{2.211673in}{0.692874in}}%
\pgfpathlineto{\pgfqpoint{2.202255in}{0.702095in}}%
\pgfpathlineto{\pgfqpoint{2.193997in}{0.705851in}}%
\pgfpathlineto{\pgfqpoint{2.190855in}{0.708411in}}%
\pgfpathlineto{\pgfqpoint{2.189892in}{0.719957in}}%
\pgfpathlineto{\pgfqpoint{2.178513in}{0.718958in}}%
\pgfpathlineto{\pgfqpoint{2.173613in}{0.721470in}}%
\pgfpathlineto{\pgfqpoint{2.174689in}{0.726922in}}%
\pgfpathlineto{\pgfqpoint{2.171759in}{0.737367in}}%
\pgfpathlineto{\pgfqpoint{2.172078in}{0.746133in}}%
\pgfpathlineto{\pgfqpoint{2.175967in}{0.749888in}}%
\pgfpathlineto{\pgfqpoint{2.176652in}{0.754493in}}%
\pgfpathlineto{\pgfqpoint{2.169944in}{0.753927in}}%
\pgfpathlineto{\pgfqpoint{2.168941in}{0.762907in}}%
\pgfpathlineto{\pgfqpoint{2.166832in}{0.787387in}}%
\pgfpathlineto{\pgfqpoint{2.176418in}{0.787936in}}%
\pgfpathlineto{\pgfqpoint{2.177866in}{0.793913in}}%
\pgfpathlineto{\pgfqpoint{2.192221in}{0.794560in}}%
\pgfpathlineto{\pgfqpoint{2.198442in}{0.789193in}}%
\pgfpathlineto{\pgfqpoint{2.197889in}{0.784340in}}%
\pgfpathlineto{\pgfqpoint{2.204331in}{0.776956in}}%
\pgfpathlineto{\pgfqpoint{2.213183in}{0.772767in}}%
\pgfpathlineto{\pgfqpoint{2.213720in}{0.767453in}}%
\pgfpathlineto{\pgfqpoint{2.217222in}{0.763282in}}%
\pgfpathlineto{\pgfqpoint{2.222615in}{0.760540in}}%
\pgfpathlineto{\pgfqpoint{2.223920in}{0.746448in}}%
\pgfpathlineto{\pgfqpoint{2.215985in}{0.745794in}}%
\pgfpathlineto{\pgfqpoint{2.217078in}{0.734332in}}%
\pgfpathlineto{\pgfqpoint{2.222669in}{0.734886in}}%
\pgfpathclose%
\pgfusepath{fill}%
\end{pgfscope}%
\begin{pgfscope}%
\pgfpathrectangle{\pgfqpoint{0.100000in}{0.100000in}}{\pgfqpoint{3.007045in}{1.925000in}}%
\pgfusepath{clip}%
\pgfsetbuttcap%
\pgfsetmiterjoin%
\definecolor{currentfill}{rgb}{0.435294,0.690965,0.842599}%
\pgfsetfillcolor{currentfill}%
\pgfsetlinewidth{0.000000pt}%
\definecolor{currentstroke}{rgb}{0.000000,0.000000,0.000000}%
\pgfsetstrokecolor{currentstroke}%
\pgfsetstrokeopacity{0.000000}%
\pgfsetdash{}{0pt}%
\pgfpathmoveto{\pgfqpoint{1.869963in}{1.345597in}}%
\pgfpathlineto{\pgfqpoint{1.858723in}{1.345342in}}%
\pgfpathlineto{\pgfqpoint{1.847419in}{1.345161in}}%
\pgfpathlineto{\pgfqpoint{1.847146in}{1.362415in}}%
\pgfpathlineto{\pgfqpoint{1.845372in}{1.362394in}}%
\pgfpathlineto{\pgfqpoint{1.844917in}{1.391266in}}%
\pgfpathlineto{\pgfqpoint{1.867818in}{1.391658in}}%
\pgfpathlineto{\pgfqpoint{1.868339in}{1.362827in}}%
\pgfpathlineto{\pgfqpoint{1.869522in}{1.362845in}}%
\pgfpathlineto{\pgfqpoint{1.869963in}{1.345597in}}%
\pgfpathclose%
\pgfusepath{fill}%
\end{pgfscope}%
\begin{pgfscope}%
\pgfpathrectangle{\pgfqpoint{0.100000in}{0.100000in}}{\pgfqpoint{3.007045in}{1.925000in}}%
\pgfusepath{clip}%
\pgfsetbuttcap%
\pgfsetmiterjoin%
\definecolor{currentfill}{rgb}{0.676817,0.816471,0.902376}%
\pgfsetfillcolor{currentfill}%
\pgfsetlinewidth{0.000000pt}%
\definecolor{currentstroke}{rgb}{0.000000,0.000000,0.000000}%
\pgfsetstrokecolor{currentstroke}%
\pgfsetstrokeopacity{0.000000}%
\pgfsetdash{}{0pt}%
\pgfpathmoveto{\pgfqpoint{2.283546in}{1.435118in}}%
\pgfpathlineto{\pgfqpoint{2.283625in}{1.434436in}}%
\pgfpathlineto{\pgfqpoint{2.261163in}{1.431916in}}%
\pgfpathlineto{\pgfqpoint{2.258163in}{1.454659in}}%
\pgfpathlineto{\pgfqpoint{2.246778in}{1.453408in}}%
\pgfpathlineto{\pgfqpoint{2.241654in}{1.499264in}}%
\pgfpathlineto{\pgfqpoint{2.252766in}{1.500298in}}%
\pgfpathlineto{\pgfqpoint{2.264263in}{1.501550in}}%
\pgfpathlineto{\pgfqpoint{2.269523in}{1.455942in}}%
\pgfpathlineto{\pgfqpoint{2.280842in}{1.457266in}}%
\pgfpathlineto{\pgfqpoint{2.283546in}{1.435118in}}%
\pgfpathclose%
\pgfusepath{fill}%
\end{pgfscope}%
\begin{pgfscope}%
\pgfpathrectangle{\pgfqpoint{0.100000in}{0.100000in}}{\pgfqpoint{3.007045in}{1.925000in}}%
\pgfusepath{clip}%
\pgfsetbuttcap%
\pgfsetmiterjoin%
\definecolor{currentfill}{rgb}{0.730950,0.839477,0.921323}%
\pgfsetfillcolor{currentfill}%
\pgfsetlinewidth{0.000000pt}%
\definecolor{currentstroke}{rgb}{0.000000,0.000000,0.000000}%
\pgfsetstrokecolor{currentstroke}%
\pgfsetstrokeopacity{0.000000}%
\pgfsetdash{}{0pt}%
\pgfpathmoveto{\pgfqpoint{2.418790in}{0.913875in}}%
\pgfpathlineto{\pgfqpoint{2.416370in}{0.915795in}}%
\pgfpathlineto{\pgfqpoint{2.421150in}{0.927599in}}%
\pgfpathlineto{\pgfqpoint{2.420735in}{0.932481in}}%
\pgfpathlineto{\pgfqpoint{2.405701in}{0.945045in}}%
\pgfpathlineto{\pgfqpoint{2.403998in}{0.954773in}}%
\pgfpathlineto{\pgfqpoint{2.399943in}{0.955887in}}%
\pgfpathlineto{\pgfqpoint{2.404400in}{0.961065in}}%
\pgfpathlineto{\pgfqpoint{2.414244in}{0.964014in}}%
\pgfpathlineto{\pgfqpoint{2.416656in}{0.974681in}}%
\pgfpathlineto{\pgfqpoint{2.429667in}{0.984471in}}%
\pgfpathlineto{\pgfqpoint{2.431676in}{0.978202in}}%
\pgfpathlineto{\pgfqpoint{2.437163in}{0.979641in}}%
\pgfpathlineto{\pgfqpoint{2.437960in}{0.976572in}}%
\pgfpathlineto{\pgfqpoint{2.446969in}{0.969925in}}%
\pgfpathlineto{\pgfqpoint{2.454474in}{0.954214in}}%
\pgfpathlineto{\pgfqpoint{2.459120in}{0.952751in}}%
\pgfpathlineto{\pgfqpoint{2.455786in}{0.951722in}}%
\pgfpathlineto{\pgfqpoint{2.453980in}{0.946316in}}%
\pgfpathlineto{\pgfqpoint{2.455447in}{0.943242in}}%
\pgfpathlineto{\pgfqpoint{2.451924in}{0.935651in}}%
\pgfpathlineto{\pgfqpoint{2.452389in}{0.929337in}}%
\pgfpathlineto{\pgfqpoint{2.435961in}{0.922177in}}%
\pgfpathlineto{\pgfqpoint{2.418790in}{0.913875in}}%
\pgfpathclose%
\pgfusepath{fill}%
\end{pgfscope}%
\begin{pgfscope}%
\pgfpathrectangle{\pgfqpoint{0.100000in}{0.100000in}}{\pgfqpoint{3.007045in}{1.925000in}}%
\pgfusepath{clip}%
\pgfsetbuttcap%
\pgfsetmiterjoin%
\definecolor{currentfill}{rgb}{0.529412,0.742637,0.862899}%
\pgfsetfillcolor{currentfill}%
\pgfsetlinewidth{0.000000pt}%
\definecolor{currentstroke}{rgb}{0.000000,0.000000,0.000000}%
\pgfsetstrokecolor{currentstroke}%
\pgfsetstrokeopacity{0.000000}%
\pgfsetdash{}{0pt}%
\pgfpathmoveto{\pgfqpoint{1.910330in}{0.935711in}}%
\pgfpathlineto{\pgfqpoint{1.910553in}{0.923358in}}%
\pgfpathlineto{\pgfqpoint{1.899149in}{0.923195in}}%
\pgfpathlineto{\pgfqpoint{1.899200in}{0.918414in}}%
\pgfpathlineto{\pgfqpoint{1.889632in}{0.918440in}}%
\pgfpathlineto{\pgfqpoint{1.851607in}{0.918465in}}%
\pgfpathlineto{\pgfqpoint{1.851596in}{0.920385in}}%
\pgfpathlineto{\pgfqpoint{1.851232in}{0.928170in}}%
\pgfpathlineto{\pgfqpoint{1.855032in}{0.933900in}}%
\pgfpathlineto{\pgfqpoint{1.855176in}{0.941374in}}%
\pgfpathlineto{\pgfqpoint{1.849514in}{0.944275in}}%
\pgfpathlineto{\pgfqpoint{1.845759in}{0.949170in}}%
\pgfpathlineto{\pgfqpoint{1.843659in}{0.955855in}}%
\pgfpathlineto{\pgfqpoint{1.832421in}{0.955990in}}%
\pgfpathlineto{\pgfqpoint{1.832387in}{0.968604in}}%
\pgfpathlineto{\pgfqpoint{1.886046in}{0.969526in}}%
\pgfpathlineto{\pgfqpoint{1.884426in}{0.969017in}}%
\pgfpathlineto{\pgfqpoint{1.884698in}{0.944172in}}%
\pgfpathlineto{\pgfqpoint{1.887586in}{0.941336in}}%
\pgfpathlineto{\pgfqpoint{1.910317in}{0.941444in}}%
\pgfpathlineto{\pgfqpoint{1.910330in}{0.935711in}}%
\pgfpathclose%
\pgfusepath{fill}%
\end{pgfscope}%
\begin{pgfscope}%
\pgfpathrectangle{\pgfqpoint{0.100000in}{0.100000in}}{\pgfqpoint{3.007045in}{1.925000in}}%
\pgfusepath{clip}%
\pgfsetbuttcap%
\pgfsetmiterjoin%
\definecolor{currentfill}{rgb}{0.435294,0.690965,0.842599}%
\pgfsetfillcolor{currentfill}%
\pgfsetlinewidth{0.000000pt}%
\definecolor{currentstroke}{rgb}{0.000000,0.000000,0.000000}%
\pgfsetstrokecolor{currentstroke}%
\pgfsetstrokeopacity{0.000000}%
\pgfsetdash{}{0pt}%
\pgfpathmoveto{\pgfqpoint{2.122132in}{1.377026in}}%
\pgfpathlineto{\pgfqpoint{2.078237in}{1.373879in}}%
\pgfpathlineto{\pgfqpoint{2.046233in}{1.372218in}}%
\pgfpathlineto{\pgfqpoint{2.044679in}{1.394919in}}%
\pgfpathlineto{\pgfqpoint{2.061841in}{1.396171in}}%
\pgfpathlineto{\pgfqpoint{2.073275in}{1.396646in}}%
\pgfpathlineto{\pgfqpoint{2.119150in}{1.400006in}}%
\pgfpathlineto{\pgfqpoint{2.122786in}{1.396116in}}%
\pgfpathlineto{\pgfqpoint{2.120693in}{1.385146in}}%
\pgfpathlineto{\pgfqpoint{2.122132in}{1.377026in}}%
\pgfpathclose%
\pgfusepath{fill}%
\end{pgfscope}%
\begin{pgfscope}%
\pgfpathrectangle{\pgfqpoint{0.100000in}{0.100000in}}{\pgfqpoint{3.007045in}{1.925000in}}%
\pgfusepath{clip}%
\pgfsetbuttcap%
\pgfsetmiterjoin%
\definecolor{currentfill}{rgb}{0.138547,0.452272,0.714510}%
\pgfsetfillcolor{currentfill}%
\pgfsetlinewidth{0.000000pt}%
\definecolor{currentstroke}{rgb}{0.000000,0.000000,0.000000}%
\pgfsetstrokecolor{currentstroke}%
\pgfsetstrokeopacity{0.000000}%
\pgfsetdash{}{0pt}%
\pgfpathmoveto{\pgfqpoint{1.495136in}{0.800124in}}%
\pgfpathlineto{\pgfqpoint{1.469347in}{0.801604in}}%
\pgfpathlineto{\pgfqpoint{1.471341in}{0.833174in}}%
\pgfpathlineto{\pgfqpoint{1.476764in}{0.832809in}}%
\pgfpathlineto{\pgfqpoint{1.478574in}{0.861428in}}%
\pgfpathlineto{\pgfqpoint{1.501084in}{0.860066in}}%
\pgfpathlineto{\pgfqpoint{1.500565in}{0.847796in}}%
\pgfpathlineto{\pgfqpoint{1.499337in}{0.825611in}}%
\pgfpathlineto{\pgfqpoint{1.496660in}{0.826118in}}%
\pgfpathlineto{\pgfqpoint{1.495136in}{0.800124in}}%
\pgfpathclose%
\pgfusepath{fill}%
\end{pgfscope}%
\begin{pgfscope}%
\pgfpathrectangle{\pgfqpoint{0.100000in}{0.100000in}}{\pgfqpoint{3.007045in}{1.925000in}}%
\pgfusepath{clip}%
\pgfsetbuttcap%
\pgfsetmiterjoin%
\definecolor{currentfill}{rgb}{0.356555,0.639293,0.814610}%
\pgfsetfillcolor{currentfill}%
\pgfsetlinewidth{0.000000pt}%
\definecolor{currentstroke}{rgb}{0.000000,0.000000,0.000000}%
\pgfsetstrokecolor{currentstroke}%
\pgfsetstrokeopacity{0.000000}%
\pgfsetdash{}{0pt}%
\pgfpathmoveto{\pgfqpoint{1.965268in}{0.666642in}}%
\pgfpathlineto{\pgfqpoint{1.965330in}{0.661044in}}%
\pgfpathlineto{\pgfqpoint{1.962071in}{0.656269in}}%
\pgfpathlineto{\pgfqpoint{1.959044in}{0.656761in}}%
\pgfpathlineto{\pgfqpoint{1.954469in}{0.643065in}}%
\pgfpathlineto{\pgfqpoint{1.949979in}{0.638422in}}%
\pgfpathlineto{\pgfqpoint{1.952131in}{0.625210in}}%
\pgfpathlineto{\pgfqpoint{1.942513in}{0.628295in}}%
\pgfpathlineto{\pgfqpoint{1.933963in}{0.633867in}}%
\pgfpathlineto{\pgfqpoint{1.933198in}{0.637846in}}%
\pgfpathlineto{\pgfqpoint{1.926639in}{0.641691in}}%
\pgfpathlineto{\pgfqpoint{1.922010in}{0.647995in}}%
\pgfpathlineto{\pgfqpoint{1.921110in}{0.659462in}}%
\pgfpathlineto{\pgfqpoint{1.923673in}{0.668219in}}%
\pgfpathlineto{\pgfqpoint{1.946656in}{0.668963in}}%
\pgfpathlineto{\pgfqpoint{1.949500in}{0.671101in}}%
\pgfpathlineto{\pgfqpoint{1.953769in}{0.665525in}}%
\pgfpathlineto{\pgfqpoint{1.963857in}{0.672381in}}%
\pgfpathlineto{\pgfqpoint{1.965268in}{0.666642in}}%
\pgfpathclose%
\pgfusepath{fill}%
\end{pgfscope}%
\begin{pgfscope}%
\pgfpathrectangle{\pgfqpoint{0.100000in}{0.100000in}}{\pgfqpoint{3.007045in}{1.925000in}}%
\pgfusepath{clip}%
\pgfsetbuttcap%
\pgfsetmiterjoin%
\definecolor{currentfill}{rgb}{0.510588,0.732303,0.858839}%
\pgfsetfillcolor{currentfill}%
\pgfsetlinewidth{0.000000pt}%
\definecolor{currentstroke}{rgb}{0.000000,0.000000,0.000000}%
\pgfsetstrokecolor{currentstroke}%
\pgfsetstrokeopacity{0.000000}%
\pgfsetdash{}{0pt}%
\pgfpathmoveto{\pgfqpoint{2.198442in}{0.789193in}}%
\pgfpathlineto{\pgfqpoint{2.192221in}{0.794560in}}%
\pgfpathlineto{\pgfqpoint{2.177866in}{0.793913in}}%
\pgfpathlineto{\pgfqpoint{2.176418in}{0.787936in}}%
\pgfpathlineto{\pgfqpoint{2.166832in}{0.787387in}}%
\pgfpathlineto{\pgfqpoint{2.142921in}{0.786166in}}%
\pgfpathlineto{\pgfqpoint{2.143299in}{0.800190in}}%
\pgfpathlineto{\pgfqpoint{2.144570in}{0.838481in}}%
\pgfpathlineto{\pgfqpoint{2.179696in}{0.840159in}}%
\pgfpathlineto{\pgfqpoint{2.202564in}{0.841744in}}%
\pgfpathlineto{\pgfqpoint{2.204539in}{0.819210in}}%
\pgfpathlineto{\pgfqpoint{2.206348in}{0.816022in}}%
\pgfpathlineto{\pgfqpoint{2.213109in}{0.813406in}}%
\pgfpathlineto{\pgfqpoint{2.210327in}{0.807475in}}%
\pgfpathlineto{\pgfqpoint{2.209929in}{0.801633in}}%
\pgfpathlineto{\pgfqpoint{2.207356in}{0.797594in}}%
\pgfpathlineto{\pgfqpoint{2.202677in}{0.796254in}}%
\pgfpathlineto{\pgfqpoint{2.198442in}{0.789193in}}%
\pgfpathclose%
\pgfusepath{fill}%
\end{pgfscope}%
\begin{pgfscope}%
\pgfpathrectangle{\pgfqpoint{0.100000in}{0.100000in}}{\pgfqpoint{3.007045in}{1.925000in}}%
\pgfusepath{clip}%
\pgfsetbuttcap%
\pgfsetmiterjoin%
\definecolor{currentfill}{rgb}{0.735871,0.841569,0.923045}%
\pgfsetfillcolor{currentfill}%
\pgfsetlinewidth{0.000000pt}%
\definecolor{currentstroke}{rgb}{0.000000,0.000000,0.000000}%
\pgfsetstrokecolor{currentstroke}%
\pgfsetstrokeopacity{0.000000}%
\pgfsetdash{}{0pt}%
\pgfpathmoveto{\pgfqpoint{2.474213in}{0.983871in}}%
\pgfpathlineto{\pgfqpoint{2.480170in}{0.978949in}}%
\pgfpathlineto{\pgfqpoint{2.489274in}{0.973776in}}%
\pgfpathlineto{\pgfqpoint{2.494999in}{0.975158in}}%
\pgfpathlineto{\pgfqpoint{2.498902in}{0.973883in}}%
\pgfpathlineto{\pgfqpoint{2.492448in}{0.960338in}}%
\pgfpathlineto{\pgfqpoint{2.486422in}{0.958962in}}%
\pgfpathlineto{\pgfqpoint{2.480332in}{0.961447in}}%
\pgfpathlineto{\pgfqpoint{2.476332in}{0.956183in}}%
\pgfpathlineto{\pgfqpoint{2.469542in}{0.954025in}}%
\pgfpathlineto{\pgfqpoint{2.459120in}{0.952751in}}%
\pgfpathlineto{\pgfqpoint{2.454474in}{0.954214in}}%
\pgfpathlineto{\pgfqpoint{2.446969in}{0.969925in}}%
\pgfpathlineto{\pgfqpoint{2.437960in}{0.976572in}}%
\pgfpathlineto{\pgfqpoint{2.437163in}{0.979641in}}%
\pgfpathlineto{\pgfqpoint{2.441498in}{0.987564in}}%
\pgfpathlineto{\pgfqpoint{2.450485in}{0.993442in}}%
\pgfpathlineto{\pgfqpoint{2.458453in}{0.990986in}}%
\pgfpathlineto{\pgfqpoint{2.462402in}{0.980829in}}%
\pgfpathlineto{\pgfqpoint{2.465463in}{0.979232in}}%
\pgfpathlineto{\pgfqpoint{2.468967in}{0.985644in}}%
\pgfpathlineto{\pgfqpoint{2.474213in}{0.983871in}}%
\pgfpathclose%
\pgfusepath{fill}%
\end{pgfscope}%
\begin{pgfscope}%
\pgfpathrectangle{\pgfqpoint{0.100000in}{0.100000in}}{\pgfqpoint{3.007045in}{1.925000in}}%
\pgfusepath{clip}%
\pgfsetbuttcap%
\pgfsetmiterjoin%
\definecolor{currentfill}{rgb}{0.808212,0.879985,0.947835}%
\pgfsetfillcolor{currentfill}%
\pgfsetlinewidth{0.000000pt}%
\definecolor{currentstroke}{rgb}{0.000000,0.000000,0.000000}%
\pgfsetstrokecolor{currentstroke}%
\pgfsetstrokeopacity{0.000000}%
\pgfsetdash{}{0pt}%
\pgfpathmoveto{\pgfqpoint{0.558722in}{1.943515in}}%
\pgfpathlineto{\pgfqpoint{0.550057in}{1.949005in}}%
\pgfpathlineto{\pgfqpoint{0.546058in}{1.955336in}}%
\pgfpathlineto{\pgfqpoint{0.547224in}{1.961309in}}%
\pgfpathlineto{\pgfqpoint{0.552633in}{1.957537in}}%
\pgfpathlineto{\pgfqpoint{0.559107in}{1.963705in}}%
\pgfpathlineto{\pgfqpoint{0.566457in}{1.957886in}}%
\pgfpathlineto{\pgfqpoint{0.558722in}{1.943515in}}%
\pgfpathclose%
\pgfusepath{fill}%
\end{pgfscope}%
\begin{pgfscope}%
\pgfpathrectangle{\pgfqpoint{0.100000in}{0.100000in}}{\pgfqpoint{3.007045in}{1.925000in}}%
\pgfusepath{clip}%
\pgfsetbuttcap%
\pgfsetmiterjoin%
\definecolor{currentfill}{rgb}{0.447843,0.697855,0.845306}%
\pgfsetfillcolor{currentfill}%
\pgfsetlinewidth{0.000000pt}%
\definecolor{currentstroke}{rgb}{0.000000,0.000000,0.000000}%
\pgfsetstrokecolor{currentstroke}%
\pgfsetstrokeopacity{0.000000}%
\pgfsetdash{}{0pt}%
\pgfpathmoveto{\pgfqpoint{1.386813in}{0.624056in}}%
\pgfpathlineto{\pgfqpoint{1.415290in}{0.621980in}}%
\pgfpathlineto{\pgfqpoint{1.433065in}{0.620988in}}%
\pgfpathlineto{\pgfqpoint{1.431514in}{0.595996in}}%
\pgfpathlineto{\pgfqpoint{1.429685in}{0.568203in}}%
\pgfpathlineto{\pgfqpoint{1.383908in}{0.571430in}}%
\pgfpathlineto{\pgfqpoint{1.389923in}{0.575693in}}%
\pgfpathlineto{\pgfqpoint{1.385588in}{0.580234in}}%
\pgfpathlineto{\pgfqpoint{1.391278in}{0.588321in}}%
\pgfpathlineto{\pgfqpoint{1.390664in}{0.594316in}}%
\pgfpathlineto{\pgfqpoint{1.386971in}{0.595647in}}%
\pgfpathlineto{\pgfqpoint{1.364071in}{0.597523in}}%
\pgfpathlineto{\pgfqpoint{1.363729in}{0.593646in}}%
\pgfpathlineto{\pgfqpoint{1.352050in}{0.594662in}}%
\pgfpathlineto{\pgfqpoint{1.350354in}{0.573776in}}%
\pgfpathlineto{\pgfqpoint{1.337515in}{0.574775in}}%
\pgfpathlineto{\pgfqpoint{1.336243in}{0.559459in}}%
\pgfpathlineto{\pgfqpoint{1.282209in}{0.612003in}}%
\pgfpathlineto{\pgfqpoint{1.318428in}{0.649073in}}%
\pgfpathlineto{\pgfqpoint{1.322677in}{0.647278in}}%
\pgfpathlineto{\pgfqpoint{1.328596in}{0.641304in}}%
\pgfpathlineto{\pgfqpoint{1.331766in}{0.642740in}}%
\pgfpathlineto{\pgfqpoint{1.337100in}{0.644492in}}%
\pgfpathlineto{\pgfqpoint{1.349725in}{0.635825in}}%
\pgfpathlineto{\pgfqpoint{1.351088in}{0.627381in}}%
\pgfpathlineto{\pgfqpoint{1.386813in}{0.624056in}}%
\pgfpathclose%
\pgfusepath{fill}%
\end{pgfscope}%
\begin{pgfscope}%
\pgfpathrectangle{\pgfqpoint{0.100000in}{0.100000in}}{\pgfqpoint{3.007045in}{1.925000in}}%
\pgfusepath{clip}%
\pgfsetbuttcap%
\pgfsetmiterjoin%
\definecolor{currentfill}{rgb}{0.321246,0.615179,0.800830}%
\pgfsetfillcolor{currentfill}%
\pgfsetlinewidth{0.000000pt}%
\definecolor{currentstroke}{rgb}{0.000000,0.000000,0.000000}%
\pgfsetstrokecolor{currentstroke}%
\pgfsetstrokeopacity{0.000000}%
\pgfsetdash{}{0pt}%
\pgfpathmoveto{\pgfqpoint{1.930951in}{0.750384in}}%
\pgfpathlineto{\pgfqpoint{1.932837in}{0.741766in}}%
\pgfpathlineto{\pgfqpoint{1.935072in}{0.740260in}}%
\pgfpathlineto{\pgfqpoint{1.898651in}{0.739626in}}%
\pgfpathlineto{\pgfqpoint{1.884006in}{0.739484in}}%
\pgfpathlineto{\pgfqpoint{1.884045in}{0.763316in}}%
\pgfpathlineto{\pgfqpoint{1.876274in}{0.763349in}}%
\pgfpathlineto{\pgfqpoint{1.876321in}{0.768157in}}%
\pgfpathlineto{\pgfqpoint{1.876541in}{0.789626in}}%
\pgfpathlineto{\pgfqpoint{1.884971in}{0.791103in}}%
\pgfpathlineto{\pgfqpoint{1.887657in}{0.795131in}}%
\pgfpathlineto{\pgfqpoint{1.887940in}{0.814920in}}%
\pgfpathlineto{\pgfqpoint{1.899320in}{0.814784in}}%
\pgfpathlineto{\pgfqpoint{1.909437in}{0.814697in}}%
\pgfpathlineto{\pgfqpoint{1.918185in}{0.809351in}}%
\pgfpathlineto{\pgfqpoint{1.910795in}{0.808969in}}%
\pgfpathlineto{\pgfqpoint{1.911801in}{0.803402in}}%
\pgfpathlineto{\pgfqpoint{1.918485in}{0.794539in}}%
\pgfpathlineto{\pgfqpoint{1.920677in}{0.772374in}}%
\pgfpathlineto{\pgfqpoint{1.917305in}{0.764130in}}%
\pgfpathlineto{\pgfqpoint{1.918474in}{0.757862in}}%
\pgfpathlineto{\pgfqpoint{1.921911in}{0.757915in}}%
\pgfpathlineto{\pgfqpoint{1.930951in}{0.750384in}}%
\pgfpathclose%
\pgfusepath{fill}%
\end{pgfscope}%
\begin{pgfscope}%
\pgfpathrectangle{\pgfqpoint{0.100000in}{0.100000in}}{\pgfqpoint{3.007045in}{1.925000in}}%
\pgfusepath{clip}%
\pgfsetbuttcap%
\pgfsetmiterjoin%
\definecolor{currentfill}{rgb}{0.290980,0.594510,0.789020}%
\pgfsetfillcolor{currentfill}%
\pgfsetlinewidth{0.000000pt}%
\definecolor{currentstroke}{rgb}{0.000000,0.000000,0.000000}%
\pgfsetstrokecolor{currentstroke}%
\pgfsetstrokeopacity{0.000000}%
\pgfsetdash{}{0pt}%
\pgfpathmoveto{\pgfqpoint{1.512812in}{1.275327in}}%
\pgfpathlineto{\pgfqpoint{1.511547in}{1.252543in}}%
\pgfpathlineto{\pgfqpoint{1.483897in}{1.254208in}}%
\pgfpathlineto{\pgfqpoint{1.460453in}{1.255605in}}%
\pgfpathlineto{\pgfqpoint{1.461934in}{1.278597in}}%
\pgfpathlineto{\pgfqpoint{1.460924in}{1.278668in}}%
\pgfpathlineto{\pgfqpoint{1.462537in}{1.301517in}}%
\pgfpathlineto{\pgfqpoint{1.455754in}{1.302023in}}%
\pgfpathlineto{\pgfqpoint{1.457408in}{1.324892in}}%
\pgfpathlineto{\pgfqpoint{1.485080in}{1.322839in}}%
\pgfpathlineto{\pgfqpoint{1.514112in}{1.321020in}}%
\pgfpathlineto{\pgfqpoint{1.512812in}{1.275327in}}%
\pgfpathclose%
\pgfusepath{fill}%
\end{pgfscope}%
\begin{pgfscope}%
\pgfpathrectangle{\pgfqpoint{0.100000in}{0.100000in}}{\pgfqpoint{3.007045in}{1.925000in}}%
\pgfusepath{clip}%
\pgfsetbuttcap%
\pgfsetmiterjoin%
\definecolor{currentfill}{rgb}{0.412042,0.677186,0.836263}%
\pgfsetfillcolor{currentfill}%
\pgfsetlinewidth{0.000000pt}%
\definecolor{currentstroke}{rgb}{0.000000,0.000000,0.000000}%
\pgfsetstrokecolor{currentstroke}%
\pgfsetstrokeopacity{0.000000}%
\pgfsetdash{}{0pt}%
\pgfpathmoveto{\pgfqpoint{2.345734in}{1.216099in}}%
\pgfpathlineto{\pgfqpoint{2.346250in}{1.205103in}}%
\pgfpathlineto{\pgfqpoint{2.330635in}{1.204428in}}%
\pgfpathlineto{\pgfqpoint{2.314937in}{1.203868in}}%
\pgfpathlineto{\pgfqpoint{2.305334in}{1.201390in}}%
\pgfpathlineto{\pgfqpoint{2.288393in}{1.199350in}}%
\pgfpathlineto{\pgfqpoint{2.285375in}{1.228022in}}%
\pgfpathlineto{\pgfqpoint{2.283471in}{1.248041in}}%
\pgfpathlineto{\pgfqpoint{2.283146in}{1.250837in}}%
\pgfpathlineto{\pgfqpoint{2.301562in}{1.253044in}}%
\pgfpathlineto{\pgfqpoint{2.306160in}{1.255224in}}%
\pgfpathlineto{\pgfqpoint{2.305349in}{1.261913in}}%
\pgfpathlineto{\pgfqpoint{2.322524in}{1.264076in}}%
\pgfpathlineto{\pgfqpoint{2.322114in}{1.267499in}}%
\pgfpathlineto{\pgfqpoint{2.327732in}{1.268420in}}%
\pgfpathlineto{\pgfqpoint{2.345898in}{1.268387in}}%
\pgfpathlineto{\pgfqpoint{2.346617in}{1.250238in}}%
\pgfpathlineto{\pgfqpoint{2.349506in}{1.250356in}}%
\pgfpathlineto{\pgfqpoint{2.350594in}{1.236213in}}%
\pgfpathlineto{\pgfqpoint{2.348588in}{1.219904in}}%
\pgfpathlineto{\pgfqpoint{2.345734in}{1.216099in}}%
\pgfpathclose%
\pgfusepath{fill}%
\end{pgfscope}%
\begin{pgfscope}%
\pgfpathrectangle{\pgfqpoint{0.100000in}{0.100000in}}{\pgfqpoint{3.007045in}{1.925000in}}%
\pgfusepath{clip}%
\pgfsetbuttcap%
\pgfsetmiterjoin%
\definecolor{currentfill}{rgb}{0.391865,0.663406,0.828389}%
\pgfsetfillcolor{currentfill}%
\pgfsetlinewidth{0.000000pt}%
\definecolor{currentstroke}{rgb}{0.000000,0.000000,0.000000}%
\pgfsetstrokecolor{currentstroke}%
\pgfsetstrokeopacity{0.000000}%
\pgfsetdash{}{0pt}%
\pgfpathmoveto{\pgfqpoint{1.925598in}{1.218425in}}%
\pgfpathlineto{\pgfqpoint{1.926181in}{1.198155in}}%
\pgfpathlineto{\pgfqpoint{1.906383in}{1.197788in}}%
\pgfpathlineto{\pgfqpoint{1.906198in}{1.203318in}}%
\pgfpathlineto{\pgfqpoint{1.880879in}{1.202651in}}%
\pgfpathlineto{\pgfqpoint{1.880439in}{1.202637in}}%
\pgfpathlineto{\pgfqpoint{1.879993in}{1.222750in}}%
\pgfpathlineto{\pgfqpoint{1.888599in}{1.223016in}}%
\pgfpathlineto{\pgfqpoint{1.889157in}{1.227613in}}%
\pgfpathlineto{\pgfqpoint{1.886676in}{1.239179in}}%
\pgfpathlineto{\pgfqpoint{1.890498in}{1.239370in}}%
\pgfpathlineto{\pgfqpoint{1.925211in}{1.241392in}}%
\pgfpathlineto{\pgfqpoint{1.925598in}{1.218425in}}%
\pgfpathclose%
\pgfusepath{fill}%
\end{pgfscope}%
\begin{pgfscope}%
\pgfpathrectangle{\pgfqpoint{0.100000in}{0.100000in}}{\pgfqpoint{3.007045in}{1.925000in}}%
\pgfusepath{clip}%
\pgfsetbuttcap%
\pgfsetmiterjoin%
\definecolor{currentfill}{rgb}{0.316201,0.611734,0.798862}%
\pgfsetfillcolor{currentfill}%
\pgfsetlinewidth{0.000000pt}%
\definecolor{currentstroke}{rgb}{0.000000,0.000000,0.000000}%
\pgfsetstrokecolor{currentstroke}%
\pgfsetstrokeopacity{0.000000}%
\pgfsetdash{}{0pt}%
\pgfpathmoveto{\pgfqpoint{1.842321in}{0.853020in}}%
\pgfpathlineto{\pgfqpoint{1.833670in}{0.850751in}}%
\pgfpathlineto{\pgfqpoint{1.830433in}{0.847601in}}%
\pgfpathlineto{\pgfqpoint{1.825651in}{0.849322in}}%
\pgfpathlineto{\pgfqpoint{1.809383in}{0.849499in}}%
\pgfpathlineto{\pgfqpoint{1.801936in}{0.851532in}}%
\pgfpathlineto{\pgfqpoint{1.801665in}{0.836897in}}%
\pgfpathlineto{\pgfqpoint{1.775725in}{0.836774in}}%
\pgfpathlineto{\pgfqpoint{1.775709in}{0.848268in}}%
\pgfpathlineto{\pgfqpoint{1.744379in}{0.848364in}}%
\pgfpathlineto{\pgfqpoint{1.744358in}{0.842618in}}%
\pgfpathlineto{\pgfqpoint{1.735784in}{0.842654in}}%
\pgfpathlineto{\pgfqpoint{1.724353in}{0.842730in}}%
\pgfpathlineto{\pgfqpoint{1.724413in}{0.848474in}}%
\pgfpathlineto{\pgfqpoint{1.713003in}{0.848605in}}%
\pgfpathlineto{\pgfqpoint{1.713476in}{0.872989in}}%
\pgfpathlineto{\pgfqpoint{1.719231in}{0.879638in}}%
\pgfpathlineto{\pgfqpoint{1.725397in}{0.881888in}}%
\pgfpathlineto{\pgfqpoint{1.732284in}{0.882217in}}%
\pgfpathlineto{\pgfqpoint{1.741580in}{0.885648in}}%
\pgfpathlineto{\pgfqpoint{1.751555in}{0.890916in}}%
\pgfpathlineto{\pgfqpoint{1.759980in}{0.886994in}}%
\pgfpathlineto{\pgfqpoint{1.763675in}{0.893735in}}%
\pgfpathlineto{\pgfqpoint{1.768002in}{0.897857in}}%
\pgfpathlineto{\pgfqpoint{1.765215in}{0.904084in}}%
\pgfpathlineto{\pgfqpoint{1.765492in}{0.911520in}}%
\pgfpathlineto{\pgfqpoint{1.800606in}{0.911576in}}%
\pgfpathlineto{\pgfqpoint{1.799463in}{0.919527in}}%
\pgfpathlineto{\pgfqpoint{1.819183in}{0.919101in}}%
\pgfpathlineto{\pgfqpoint{1.830635in}{0.919840in}}%
\pgfpathlineto{\pgfqpoint{1.826781in}{0.913265in}}%
\pgfpathlineto{\pgfqpoint{1.821996in}{0.913307in}}%
\pgfpathlineto{\pgfqpoint{1.821906in}{0.907584in}}%
\pgfpathlineto{\pgfqpoint{1.824480in}{0.903095in}}%
\pgfpathlineto{\pgfqpoint{1.822064in}{0.899000in}}%
\pgfpathlineto{\pgfqpoint{1.821794in}{0.883783in}}%
\pgfpathlineto{\pgfqpoint{1.818790in}{0.879030in}}%
\pgfpathlineto{\pgfqpoint{1.818712in}{0.873240in}}%
\pgfpathlineto{\pgfqpoint{1.822375in}{0.871354in}}%
\pgfpathlineto{\pgfqpoint{1.842357in}{0.871145in}}%
\pgfpathlineto{\pgfqpoint{1.842321in}{0.853020in}}%
\pgfpathclose%
\pgfusepath{fill}%
\end{pgfscope}%
\begin{pgfscope}%
\pgfpathrectangle{\pgfqpoint{0.100000in}{0.100000in}}{\pgfqpoint{3.007045in}{1.925000in}}%
\pgfusepath{clip}%
\pgfsetbuttcap%
\pgfsetmiterjoin%
\definecolor{currentfill}{rgb}{0.781638,0.862268,0.938977}%
\pgfsetfillcolor{currentfill}%
\pgfsetlinewidth{0.000000pt}%
\definecolor{currentstroke}{rgb}{0.000000,0.000000,0.000000}%
\pgfsetstrokecolor{currentstroke}%
\pgfsetstrokeopacity{0.000000}%
\pgfsetdash{}{0pt}%
\pgfpathmoveto{\pgfqpoint{2.994174in}{1.647565in}}%
\pgfpathlineto{\pgfqpoint{3.000400in}{1.646818in}}%
\pgfpathlineto{\pgfqpoint{2.997849in}{1.641810in}}%
\pgfpathlineto{\pgfqpoint{2.994174in}{1.647565in}}%
\pgfpathclose%
\pgfusepath{fill}%
\end{pgfscope}%
\begin{pgfscope}%
\pgfpathrectangle{\pgfqpoint{0.100000in}{0.100000in}}{\pgfqpoint{3.007045in}{1.925000in}}%
\pgfusepath{clip}%
\pgfsetbuttcap%
\pgfsetmiterjoin%
\definecolor{currentfill}{rgb}{0.781638,0.862268,0.938977}%
\pgfsetfillcolor{currentfill}%
\pgfsetlinewidth{0.000000pt}%
\definecolor{currentstroke}{rgb}{0.000000,0.000000,0.000000}%
\pgfsetstrokecolor{currentstroke}%
\pgfsetstrokeopacity{0.000000}%
\pgfsetdash{}{0pt}%
\pgfpathmoveto{\pgfqpoint{2.975595in}{1.630839in}}%
\pgfpathlineto{\pgfqpoint{2.978270in}{1.638556in}}%
\pgfpathlineto{\pgfqpoint{2.972759in}{1.647244in}}%
\pgfpathlineto{\pgfqpoint{2.969629in}{1.646428in}}%
\pgfpathlineto{\pgfqpoint{2.965671in}{1.652396in}}%
\pgfpathlineto{\pgfqpoint{2.962617in}{1.653501in}}%
\pgfpathlineto{\pgfqpoint{2.963183in}{1.661995in}}%
\pgfpathlineto{\pgfqpoint{2.962312in}{1.673903in}}%
\pgfpathlineto{\pgfqpoint{2.960139in}{1.680504in}}%
\pgfpathlineto{\pgfqpoint{2.966515in}{1.680747in}}%
\pgfpathlineto{\pgfqpoint{2.956012in}{1.701938in}}%
\pgfpathlineto{\pgfqpoint{2.945232in}{1.694510in}}%
\pgfpathlineto{\pgfqpoint{2.928393in}{1.726358in}}%
\pgfpathlineto{\pgfqpoint{2.930414in}{1.732814in}}%
\pgfpathlineto{\pgfqpoint{2.925595in}{1.735765in}}%
\pgfpathlineto{\pgfqpoint{2.926062in}{1.747145in}}%
\pgfpathlineto{\pgfqpoint{2.922774in}{1.754622in}}%
\pgfpathlineto{\pgfqpoint{2.915056in}{1.781252in}}%
\pgfpathlineto{\pgfqpoint{2.912260in}{1.792925in}}%
\pgfpathlineto{\pgfqpoint{2.951724in}{1.804308in}}%
\pgfpathlineto{\pgfqpoint{2.955175in}{1.793081in}}%
\pgfpathlineto{\pgfqpoint{2.972249in}{1.797222in}}%
\pgfpathlineto{\pgfqpoint{2.981003in}{1.769545in}}%
\pgfpathlineto{\pgfqpoint{2.987728in}{1.746399in}}%
\pgfpathlineto{\pgfqpoint{2.988875in}{1.746178in}}%
\pgfpathlineto{\pgfqpoint{3.003662in}{1.755298in}}%
\pgfpathlineto{\pgfqpoint{3.015744in}{1.733166in}}%
\pgfpathlineto{\pgfqpoint{3.010773in}{1.730803in}}%
\pgfpathlineto{\pgfqpoint{3.019840in}{1.712826in}}%
\pgfpathlineto{\pgfqpoint{3.014639in}{1.710091in}}%
\pgfpathlineto{\pgfqpoint{3.025401in}{1.689136in}}%
\pgfpathlineto{\pgfqpoint{3.031696in}{1.678779in}}%
\pgfpathlineto{\pgfqpoint{3.027762in}{1.672640in}}%
\pgfpathlineto{\pgfqpoint{3.022621in}{1.680721in}}%
\pgfpathlineto{\pgfqpoint{3.016503in}{1.677395in}}%
\pgfpathlineto{\pgfqpoint{3.023371in}{1.670491in}}%
\pgfpathlineto{\pgfqpoint{3.017940in}{1.661449in}}%
\pgfpathlineto{\pgfqpoint{3.012026in}{1.665983in}}%
\pgfpathlineto{\pgfqpoint{3.012283in}{1.670432in}}%
\pgfpathlineto{\pgfqpoint{3.008113in}{1.674442in}}%
\pgfpathlineto{\pgfqpoint{3.005865in}{1.669939in}}%
\pgfpathlineto{\pgfqpoint{3.005023in}{1.661197in}}%
\pgfpathlineto{\pgfqpoint{2.996716in}{1.662802in}}%
\pgfpathlineto{\pgfqpoint{2.986928in}{1.668315in}}%
\pgfpathlineto{\pgfqpoint{2.984644in}{1.665649in}}%
\pgfpathlineto{\pgfqpoint{2.987952in}{1.658494in}}%
\pgfpathlineto{\pgfqpoint{2.985414in}{1.647330in}}%
\pgfpathlineto{\pgfqpoint{2.987840in}{1.640394in}}%
\pgfpathlineto{\pgfqpoint{2.984073in}{1.632614in}}%
\pgfpathlineto{\pgfqpoint{2.975595in}{1.630839in}}%
\pgfpathclose%
\pgfusepath{fill}%
\end{pgfscope}%
\begin{pgfscope}%
\pgfpathrectangle{\pgfqpoint{0.100000in}{0.100000in}}{\pgfqpoint{3.007045in}{1.925000in}}%
\pgfusepath{clip}%
\pgfsetbuttcap%
\pgfsetmiterjoin%
\definecolor{currentfill}{rgb}{0.516863,0.735748,0.860192}%
\pgfsetfillcolor{currentfill}%
\pgfsetlinewidth{0.000000pt}%
\definecolor{currentstroke}{rgb}{0.000000,0.000000,0.000000}%
\pgfsetstrokecolor{currentstroke}%
\pgfsetstrokeopacity{0.000000}%
\pgfsetdash{}{0pt}%
\pgfpathmoveto{\pgfqpoint{1.566327in}{0.575160in}}%
\pgfpathlineto{\pgfqpoint{1.544968in}{0.575956in}}%
\pgfpathlineto{\pgfqpoint{1.546174in}{0.604074in}}%
\pgfpathlineto{\pgfqpoint{1.575754in}{0.602872in}}%
\pgfpathlineto{\pgfqpoint{1.576374in}{0.610043in}}%
\pgfpathlineto{\pgfqpoint{1.606454in}{0.609350in}}%
\pgfpathlineto{\pgfqpoint{1.610932in}{0.600948in}}%
\pgfpathlineto{\pgfqpoint{1.602998in}{0.592878in}}%
\pgfpathlineto{\pgfqpoint{1.593576in}{0.574175in}}%
\pgfpathlineto{\pgfqpoint{1.592881in}{0.569267in}}%
\pgfpathlineto{\pgfqpoint{1.576662in}{0.574910in}}%
\pgfpathlineto{\pgfqpoint{1.566327in}{0.575160in}}%
\pgfpathclose%
\pgfusepath{fill}%
\end{pgfscope}%
\begin{pgfscope}%
\pgfpathrectangle{\pgfqpoint{0.100000in}{0.100000in}}{\pgfqpoint{3.007045in}{1.925000in}}%
\pgfusepath{clip}%
\pgfsetbuttcap%
\pgfsetmiterjoin%
\definecolor{currentfill}{rgb}{0.351511,0.635848,0.812641}%
\pgfsetfillcolor{currentfill}%
\pgfsetlinewidth{0.000000pt}%
\definecolor{currentstroke}{rgb}{0.000000,0.000000,0.000000}%
\pgfsetstrokecolor{currentstroke}%
\pgfsetstrokeopacity{0.000000}%
\pgfsetdash{}{0pt}%
\pgfpathmoveto{\pgfqpoint{1.985200in}{0.741858in}}%
\pgfpathlineto{\pgfqpoint{1.989679in}{0.740203in}}%
\pgfpathlineto{\pgfqpoint{1.990829in}{0.734444in}}%
\pgfpathlineto{\pgfqpoint{1.986138in}{0.729844in}}%
\pgfpathlineto{\pgfqpoint{1.985938in}{0.725835in}}%
\pgfpathlineto{\pgfqpoint{1.991838in}{0.723800in}}%
\pgfpathlineto{\pgfqpoint{1.986961in}{0.717838in}}%
\pgfpathlineto{\pgfqpoint{1.988496in}{0.714303in}}%
\pgfpathlineto{\pgfqpoint{1.993233in}{0.713829in}}%
\pgfpathlineto{\pgfqpoint{1.988311in}{0.711075in}}%
\pgfpathlineto{\pgfqpoint{1.970224in}{0.710328in}}%
\pgfpathlineto{\pgfqpoint{1.970799in}{0.713279in}}%
\pgfpathlineto{\pgfqpoint{1.962387in}{0.712972in}}%
\pgfpathlineto{\pgfqpoint{1.959700in}{0.720849in}}%
\pgfpathlineto{\pgfqpoint{1.964704in}{0.732579in}}%
\pgfpathlineto{\pgfqpoint{1.967255in}{0.733621in}}%
\pgfpathlineto{\pgfqpoint{1.970228in}{0.741380in}}%
\pgfpathlineto{\pgfqpoint{1.985200in}{0.741858in}}%
\pgfpathclose%
\pgfusepath{fill}%
\end{pgfscope}%
\begin{pgfscope}%
\pgfpathrectangle{\pgfqpoint{0.100000in}{0.100000in}}{\pgfqpoint{3.007045in}{1.925000in}}%
\pgfusepath{clip}%
\pgfsetbuttcap%
\pgfsetmiterjoin%
\definecolor{currentfill}{rgb}{0.523137,0.739193,0.861546}%
\pgfsetfillcolor{currentfill}%
\pgfsetlinewidth{0.000000pt}%
\definecolor{currentstroke}{rgb}{0.000000,0.000000,0.000000}%
\pgfsetstrokecolor{currentstroke}%
\pgfsetstrokeopacity{0.000000}%
\pgfsetdash{}{0pt}%
\pgfpathmoveto{\pgfqpoint{1.760890in}{0.592713in}}%
\pgfpathlineto{\pgfqpoint{1.757986in}{0.598171in}}%
\pgfpathlineto{\pgfqpoint{1.753626in}{0.595060in}}%
\pgfpathlineto{\pgfqpoint{1.751572in}{0.571734in}}%
\pgfpathlineto{\pgfqpoint{1.737969in}{0.571808in}}%
\pgfpathlineto{\pgfqpoint{1.724776in}{0.579988in}}%
\pgfpathlineto{\pgfqpoint{1.723044in}{0.595566in}}%
\pgfpathlineto{\pgfqpoint{1.705582in}{0.593025in}}%
\pgfpathlineto{\pgfqpoint{1.701562in}{0.603092in}}%
\pgfpathlineto{\pgfqpoint{1.716755in}{0.610772in}}%
\pgfpathlineto{\pgfqpoint{1.728806in}{0.610818in}}%
\pgfpathlineto{\pgfqpoint{1.728210in}{0.613654in}}%
\pgfpathlineto{\pgfqpoint{1.731113in}{0.622526in}}%
\pgfpathlineto{\pgfqpoint{1.734214in}{0.624182in}}%
\pgfpathlineto{\pgfqpoint{1.730423in}{0.638033in}}%
\pgfpathlineto{\pgfqpoint{1.735433in}{0.640512in}}%
\pgfpathlineto{\pgfqpoint{1.756860in}{0.643794in}}%
\pgfpathlineto{\pgfqpoint{1.763851in}{0.642226in}}%
\pgfpathlineto{\pgfqpoint{1.766321in}{0.636474in}}%
\pgfpathlineto{\pgfqpoint{1.772144in}{0.632693in}}%
\pgfpathlineto{\pgfqpoint{1.779925in}{0.639399in}}%
\pgfpathlineto{\pgfqpoint{1.787706in}{0.634802in}}%
\pgfpathlineto{\pgfqpoint{1.800480in}{0.631756in}}%
\pgfpathlineto{\pgfqpoint{1.804231in}{0.625251in}}%
\pgfpathlineto{\pgfqpoint{1.810103in}{0.619771in}}%
\pgfpathlineto{\pgfqpoint{1.821877in}{0.611209in}}%
\pgfpathlineto{\pgfqpoint{1.803223in}{0.606687in}}%
\pgfpathlineto{\pgfqpoint{1.793823in}{0.610228in}}%
\pgfpathlineto{\pgfqpoint{1.786479in}{0.611427in}}%
\pgfpathlineto{\pgfqpoint{1.781271in}{0.614134in}}%
\pgfpathlineto{\pgfqpoint{1.775795in}{0.607167in}}%
\pgfpathlineto{\pgfqpoint{1.760890in}{0.592713in}}%
\pgfpathclose%
\pgfusepath{fill}%
\end{pgfscope}%
\begin{pgfscope}%
\pgfpathrectangle{\pgfqpoint{0.100000in}{0.100000in}}{\pgfqpoint{3.007045in}{1.925000in}}%
\pgfusepath{clip}%
\pgfsetbuttcap%
\pgfsetmiterjoin%
\definecolor{currentfill}{rgb}{0.681738,0.818562,0.904098}%
\pgfsetfillcolor{currentfill}%
\pgfsetlinewidth{0.000000pt}%
\definecolor{currentstroke}{rgb}{0.000000,0.000000,0.000000}%
\pgfsetstrokecolor{currentstroke}%
\pgfsetstrokeopacity{0.000000}%
\pgfsetdash{}{0pt}%
\pgfpathmoveto{\pgfqpoint{2.437163in}{0.979641in}}%
\pgfpathlineto{\pgfqpoint{2.431676in}{0.978202in}}%
\pgfpathlineto{\pgfqpoint{2.429667in}{0.984471in}}%
\pgfpathlineto{\pgfqpoint{2.416656in}{0.974681in}}%
\pgfpathlineto{\pgfqpoint{2.406942in}{0.983024in}}%
\pgfpathlineto{\pgfqpoint{2.400322in}{0.986669in}}%
\pgfpathlineto{\pgfqpoint{2.404364in}{0.992821in}}%
\pgfpathlineto{\pgfqpoint{2.399497in}{0.998207in}}%
\pgfpathlineto{\pgfqpoint{2.394844in}{0.994687in}}%
\pgfpathlineto{\pgfqpoint{2.392615in}{1.001420in}}%
\pgfpathlineto{\pgfqpoint{2.396925in}{1.003949in}}%
\pgfpathlineto{\pgfqpoint{2.399781in}{1.010130in}}%
\pgfpathlineto{\pgfqpoint{2.406434in}{1.016470in}}%
\pgfpathlineto{\pgfqpoint{2.407690in}{1.022099in}}%
\pgfpathlineto{\pgfqpoint{2.413702in}{1.026726in}}%
\pgfpathlineto{\pgfqpoint{2.416735in}{1.033753in}}%
\pgfpathlineto{\pgfqpoint{2.422280in}{1.037551in}}%
\pgfpathlineto{\pgfqpoint{2.434200in}{1.039055in}}%
\pgfpathlineto{\pgfqpoint{2.439622in}{1.028698in}}%
\pgfpathlineto{\pgfqpoint{2.450725in}{1.036221in}}%
\pgfpathlineto{\pgfqpoint{2.451559in}{1.040086in}}%
\pgfpathlineto{\pgfqpoint{2.461325in}{1.043806in}}%
\pgfpathlineto{\pgfqpoint{2.464192in}{1.048074in}}%
\pgfpathlineto{\pgfqpoint{2.469632in}{1.050830in}}%
\pgfpathlineto{\pgfqpoint{2.483475in}{1.055101in}}%
\pgfpathlineto{\pgfqpoint{2.490316in}{1.048764in}}%
\pgfpathlineto{\pgfqpoint{2.495062in}{1.040390in}}%
\pgfpathlineto{\pgfqpoint{2.481841in}{1.034709in}}%
\pgfpathlineto{\pgfqpoint{2.481176in}{1.031217in}}%
\pgfpathlineto{\pgfqpoint{2.476402in}{1.027509in}}%
\pgfpathlineto{\pgfqpoint{2.466964in}{1.026275in}}%
\pgfpathlineto{\pgfqpoint{2.462097in}{1.022483in}}%
\pgfpathlineto{\pgfqpoint{2.461449in}{1.016280in}}%
\pgfpathlineto{\pgfqpoint{2.458161in}{1.009811in}}%
\pgfpathlineto{\pgfqpoint{2.463756in}{1.002325in}}%
\pgfpathlineto{\pgfqpoint{2.460956in}{0.992711in}}%
\pgfpathlineto{\pgfqpoint{2.458453in}{0.990986in}}%
\pgfpathlineto{\pgfqpoint{2.450485in}{0.993442in}}%
\pgfpathlineto{\pgfqpoint{2.441498in}{0.987564in}}%
\pgfpathlineto{\pgfqpoint{2.437163in}{0.979641in}}%
\pgfpathclose%
\pgfusepath{fill}%
\end{pgfscope}%
\begin{pgfscope}%
\pgfpathrectangle{\pgfqpoint{0.100000in}{0.100000in}}{\pgfqpoint{3.007045in}{1.925000in}}%
\pgfusepath{clip}%
\pgfsetbuttcap%
\pgfsetmiterjoin%
\definecolor{currentfill}{rgb}{0.676817,0.816471,0.902376}%
\pgfsetfillcolor{currentfill}%
\pgfsetlinewidth{0.000000pt}%
\definecolor{currentstroke}{rgb}{0.000000,0.000000,0.000000}%
\pgfsetstrokecolor{currentstroke}%
\pgfsetstrokeopacity{0.000000}%
\pgfsetdash{}{0pt}%
\pgfpathmoveto{\pgfqpoint{2.131069in}{1.580709in}}%
\pgfpathlineto{\pgfqpoint{2.124022in}{1.562475in}}%
\pgfpathlineto{\pgfqpoint{2.118308in}{1.554107in}}%
\pgfpathlineto{\pgfqpoint{2.116985in}{1.541346in}}%
\pgfpathlineto{\pgfqpoint{2.111077in}{1.540159in}}%
\pgfpathlineto{\pgfqpoint{2.094036in}{1.542696in}}%
\pgfpathlineto{\pgfqpoint{2.093706in}{1.548428in}}%
\pgfpathlineto{\pgfqpoint{2.090120in}{1.553996in}}%
\pgfpathlineto{\pgfqpoint{2.084400in}{1.553797in}}%
\pgfpathlineto{\pgfqpoint{2.083707in}{1.565240in}}%
\pgfpathlineto{\pgfqpoint{2.078017in}{1.564986in}}%
\pgfpathlineto{\pgfqpoint{2.076489in}{1.587837in}}%
\pgfpathlineto{\pgfqpoint{2.064911in}{1.587042in}}%
\pgfpathlineto{\pgfqpoint{2.063792in}{1.604125in}}%
\pgfpathlineto{\pgfqpoint{2.048726in}{1.610846in}}%
\pgfpathlineto{\pgfqpoint{2.047301in}{1.632207in}}%
\pgfpathlineto{\pgfqpoint{2.087227in}{1.635040in}}%
\pgfpathlineto{\pgfqpoint{2.088086in}{1.623582in}}%
\pgfpathlineto{\pgfqpoint{2.111036in}{1.625398in}}%
\pgfpathlineto{\pgfqpoint{2.112389in}{1.608131in}}%
\pgfpathlineto{\pgfqpoint{2.123850in}{1.609038in}}%
\pgfpathlineto{\pgfqpoint{2.126263in}{1.603472in}}%
\pgfpathlineto{\pgfqpoint{2.128131in}{1.580457in}}%
\pgfpathlineto{\pgfqpoint{2.131069in}{1.580709in}}%
\pgfpathclose%
\pgfusepath{fill}%
\end{pgfscope}%
\begin{pgfscope}%
\pgfpathrectangle{\pgfqpoint{0.100000in}{0.100000in}}{\pgfqpoint{3.007045in}{1.925000in}}%
\pgfusepath{clip}%
\pgfsetbuttcap%
\pgfsetmiterjoin%
\definecolor{currentfill}{rgb}{0.396909,0.666851,0.830358}%
\pgfsetfillcolor{currentfill}%
\pgfsetlinewidth{0.000000pt}%
\definecolor{currentstroke}{rgb}{0.000000,0.000000,0.000000}%
\pgfsetstrokecolor{currentstroke}%
\pgfsetstrokeopacity{0.000000}%
\pgfsetdash{}{0pt}%
\pgfpathmoveto{\pgfqpoint{2.106178in}{0.880380in}}%
\pgfpathlineto{\pgfqpoint{2.085939in}{0.878995in}}%
\pgfpathlineto{\pgfqpoint{2.085014in}{0.905571in}}%
\pgfpathlineto{\pgfqpoint{2.090507in}{0.908195in}}%
\pgfpathlineto{\pgfqpoint{2.089884in}{0.925436in}}%
\pgfpathlineto{\pgfqpoint{2.074636in}{0.931335in}}%
\pgfpathlineto{\pgfqpoint{2.083454in}{0.947053in}}%
\pgfpathlineto{\pgfqpoint{2.083172in}{0.958841in}}%
\pgfpathlineto{\pgfqpoint{2.093449in}{0.959312in}}%
\pgfpathlineto{\pgfqpoint{2.098903in}{0.955455in}}%
\pgfpathlineto{\pgfqpoint{2.108338in}{0.951144in}}%
\pgfpathlineto{\pgfqpoint{2.108828in}{0.933214in}}%
\pgfpathlineto{\pgfqpoint{2.114171in}{0.933474in}}%
\pgfpathlineto{\pgfqpoint{2.114744in}{0.920210in}}%
\pgfpathlineto{\pgfqpoint{2.120897in}{0.915606in}}%
\pgfpathlineto{\pgfqpoint{2.125902in}{0.914003in}}%
\pgfpathlineto{\pgfqpoint{2.129095in}{0.910042in}}%
\pgfpathlineto{\pgfqpoint{2.129176in}{0.907553in}}%
\pgfpathlineto{\pgfqpoint{2.117945in}{0.906798in}}%
\pgfpathlineto{\pgfqpoint{2.114334in}{0.905115in}}%
\pgfpathlineto{\pgfqpoint{2.110613in}{0.897382in}}%
\pgfpathlineto{\pgfqpoint{2.107278in}{0.897155in}}%
\pgfpathlineto{\pgfqpoint{2.108156in}{0.880516in}}%
\pgfpathlineto{\pgfqpoint{2.106178in}{0.880380in}}%
\pgfpathclose%
\pgfusepath{fill}%
\end{pgfscope}%
\begin{pgfscope}%
\pgfpathrectangle{\pgfqpoint{0.100000in}{0.100000in}}{\pgfqpoint{3.007045in}{1.925000in}}%
\pgfusepath{clip}%
\pgfsetbuttcap%
\pgfsetmiterjoin%
\definecolor{currentfill}{rgb}{0.371688,0.649627,0.820515}%
\pgfsetfillcolor{currentfill}%
\pgfsetlinewidth{0.000000pt}%
\definecolor{currentstroke}{rgb}{0.000000,0.000000,0.000000}%
\pgfsetstrokecolor{currentstroke}%
\pgfsetstrokeopacity{0.000000}%
\pgfsetdash{}{0pt}%
\pgfpathmoveto{\pgfqpoint{1.650255in}{1.108902in}}%
\pgfpathlineto{\pgfqpoint{1.621825in}{1.109765in}}%
\pgfpathlineto{\pgfqpoint{1.622153in}{1.126974in}}%
\pgfpathlineto{\pgfqpoint{1.593668in}{1.127954in}}%
\pgfpathlineto{\pgfqpoint{1.594074in}{1.145204in}}%
\pgfpathlineto{\pgfqpoint{1.594289in}{1.150925in}}%
\pgfpathlineto{\pgfqpoint{1.622833in}{1.149886in}}%
\pgfpathlineto{\pgfqpoint{1.623055in}{1.155630in}}%
\pgfpathlineto{\pgfqpoint{1.651571in}{1.154782in}}%
\pgfpathlineto{\pgfqpoint{1.650255in}{1.108902in}}%
\pgfpathclose%
\pgfusepath{fill}%
\end{pgfscope}%
\begin{pgfscope}%
\pgfpathrectangle{\pgfqpoint{0.100000in}{0.100000in}}{\pgfqpoint{3.007045in}{1.925000in}}%
\pgfusepath{clip}%
\pgfsetbuttcap%
\pgfsetmiterjoin%
\definecolor{currentfill}{rgb}{0.154787,0.468512,0.722876}%
\pgfsetfillcolor{currentfill}%
\pgfsetlinewidth{0.000000pt}%
\definecolor{currentstroke}{rgb}{0.000000,0.000000,0.000000}%
\pgfsetstrokecolor{currentstroke}%
\pgfsetstrokeopacity{0.000000}%
\pgfsetdash{}{0pt}%
\pgfpathmoveto{\pgfqpoint{2.140155in}{0.982191in}}%
\pgfpathlineto{\pgfqpoint{2.141030in}{0.985137in}}%
\pgfpathlineto{\pgfqpoint{2.135867in}{0.998378in}}%
\pgfpathlineto{\pgfqpoint{2.132817in}{1.006098in}}%
\pgfpathlineto{\pgfqpoint{2.146362in}{1.013286in}}%
\pgfpathlineto{\pgfqpoint{2.152882in}{1.014762in}}%
\pgfpathlineto{\pgfqpoint{2.155476in}{1.018069in}}%
\pgfpathlineto{\pgfqpoint{2.156379in}{1.026602in}}%
\pgfpathlineto{\pgfqpoint{2.164956in}{1.024377in}}%
\pgfpathlineto{\pgfqpoint{2.174509in}{1.028592in}}%
\pgfpathlineto{\pgfqpoint{2.179536in}{1.021312in}}%
\pgfpathlineto{\pgfqpoint{2.189732in}{1.023554in}}%
\pgfpathlineto{\pgfqpoint{2.191739in}{0.996087in}}%
\pgfpathlineto{\pgfqpoint{2.188886in}{0.995787in}}%
\pgfpathlineto{\pgfqpoint{2.189194in}{0.983871in}}%
\pgfpathlineto{\pgfqpoint{2.183256in}{0.977798in}}%
\pgfpathlineto{\pgfqpoint{2.181638in}{0.973954in}}%
\pgfpathlineto{\pgfqpoint{2.169549in}{0.973787in}}%
\pgfpathlineto{\pgfqpoint{2.169480in}{0.968102in}}%
\pgfpathlineto{\pgfqpoint{2.166450in}{0.964466in}}%
\pgfpathlineto{\pgfqpoint{2.158552in}{0.966327in}}%
\pgfpathlineto{\pgfqpoint{2.146928in}{0.965917in}}%
\pgfpathlineto{\pgfqpoint{2.146159in}{0.970611in}}%
\pgfpathlineto{\pgfqpoint{2.140247in}{0.978276in}}%
\pgfpathlineto{\pgfqpoint{2.140155in}{0.982191in}}%
\pgfpathclose%
\pgfusepath{fill}%
\end{pgfscope}%
\begin{pgfscope}%
\pgfpathrectangle{\pgfqpoint{0.100000in}{0.100000in}}{\pgfqpoint{3.007045in}{1.925000in}}%
\pgfusepath{clip}%
\pgfsetbuttcap%
\pgfsetmiterjoin%
\definecolor{currentfill}{rgb}{0.466667,0.708189,0.849366}%
\pgfsetfillcolor{currentfill}%
\pgfsetlinewidth{0.000000pt}%
\definecolor{currentstroke}{rgb}{0.000000,0.000000,0.000000}%
\pgfsetstrokecolor{currentstroke}%
\pgfsetstrokeopacity{0.000000}%
\pgfsetdash{}{0pt}%
\pgfpathmoveto{\pgfqpoint{2.721216in}{1.279405in}}%
\pgfpathlineto{\pgfqpoint{2.739138in}{1.283230in}}%
\pgfpathlineto{\pgfqpoint{2.742511in}{1.290165in}}%
\pgfpathlineto{\pgfqpoint{2.746343in}{1.292558in}}%
\pgfpathlineto{\pgfqpoint{2.753598in}{1.293002in}}%
\pgfpathlineto{\pgfqpoint{2.753954in}{1.289095in}}%
\pgfpathlineto{\pgfqpoint{2.751124in}{1.279964in}}%
\pgfpathlineto{\pgfqpoint{2.754264in}{1.276110in}}%
\pgfpathlineto{\pgfqpoint{2.754436in}{1.271463in}}%
\pgfpathlineto{\pgfqpoint{2.751790in}{1.268575in}}%
\pgfpathlineto{\pgfqpoint{2.757050in}{1.263166in}}%
\pgfpathlineto{\pgfqpoint{2.751195in}{1.256851in}}%
\pgfpathlineto{\pgfqpoint{2.745705in}{1.255995in}}%
\pgfpathlineto{\pgfqpoint{2.744257in}{1.261114in}}%
\pgfpathlineto{\pgfqpoint{2.738533in}{1.258792in}}%
\pgfpathlineto{\pgfqpoint{2.730272in}{1.258328in}}%
\pgfpathlineto{\pgfqpoint{2.732034in}{1.262134in}}%
\pgfpathlineto{\pgfqpoint{2.731538in}{1.268601in}}%
\pgfpathlineto{\pgfqpoint{2.726638in}{1.268556in}}%
\pgfpathlineto{\pgfqpoint{2.716371in}{1.278367in}}%
\pgfpathlineto{\pgfqpoint{2.721216in}{1.279405in}}%
\pgfpathclose%
\pgfusepath{fill}%
\end{pgfscope}%
\begin{pgfscope}%
\pgfpathrectangle{\pgfqpoint{0.100000in}{0.100000in}}{\pgfqpoint{3.007045in}{1.925000in}}%
\pgfusepath{clip}%
\pgfsetbuttcap%
\pgfsetmiterjoin%
\definecolor{currentfill}{rgb}{0.351511,0.635848,0.812641}%
\pgfsetfillcolor{currentfill}%
\pgfsetlinewidth{0.000000pt}%
\definecolor{currentstroke}{rgb}{0.000000,0.000000,0.000000}%
\pgfsetstrokecolor{currentstroke}%
\pgfsetstrokeopacity{0.000000}%
\pgfsetdash{}{0pt}%
\pgfpathmoveto{\pgfqpoint{1.509458in}{1.149192in}}%
\pgfpathlineto{\pgfqpoint{1.508618in}{1.149222in}}%
\pgfpathlineto{\pgfqpoint{1.510385in}{1.177845in}}%
\pgfpathlineto{\pgfqpoint{1.537445in}{1.176396in}}%
\pgfpathlineto{\pgfqpoint{1.567058in}{1.175010in}}%
\pgfpathlineto{\pgfqpoint{1.565600in}{1.146405in}}%
\pgfpathlineto{\pgfqpoint{1.509458in}{1.149192in}}%
\pgfpathclose%
\pgfusepath{fill}%
\end{pgfscope}%
\begin{pgfscope}%
\pgfpathrectangle{\pgfqpoint{0.100000in}{0.100000in}}{\pgfqpoint{3.007045in}{1.925000in}}%
\pgfusepath{clip}%
\pgfsetbuttcap%
\pgfsetmiterjoin%
\definecolor{currentfill}{rgb}{0.244106,0.557832,0.768889}%
\pgfsetfillcolor{currentfill}%
\pgfsetlinewidth{0.000000pt}%
\definecolor{currentstroke}{rgb}{0.000000,0.000000,0.000000}%
\pgfsetstrokecolor{currentstroke}%
\pgfsetstrokeopacity{0.000000}%
\pgfsetdash{}{0pt}%
\pgfpathmoveto{\pgfqpoint{1.596180in}{1.004385in}}%
\pgfpathlineto{\pgfqpoint{1.596980in}{1.029821in}}%
\pgfpathlineto{\pgfqpoint{1.590957in}{1.030045in}}%
\pgfpathlineto{\pgfqpoint{1.591838in}{1.053004in}}%
\pgfpathlineto{\pgfqpoint{1.631737in}{1.051813in}}%
\pgfpathlineto{\pgfqpoint{1.631910in}{1.063521in}}%
\pgfpathlineto{\pgfqpoint{1.660467in}{1.062737in}}%
\pgfpathlineto{\pgfqpoint{1.660705in}{1.074262in}}%
\pgfpathlineto{\pgfqpoint{1.676880in}{1.073769in}}%
\pgfpathlineto{\pgfqpoint{1.693377in}{1.073520in}}%
\pgfpathlineto{\pgfqpoint{1.692552in}{1.033341in}}%
\pgfpathlineto{\pgfqpoint{1.659741in}{1.033948in}}%
\pgfpathlineto{\pgfqpoint{1.659290in}{1.002553in}}%
\pgfpathlineto{\pgfqpoint{1.642737in}{1.002945in}}%
\pgfpathlineto{\pgfqpoint{1.596180in}{1.004385in}}%
\pgfpathclose%
\pgfusepath{fill}%
\end{pgfscope}%
\begin{pgfscope}%
\pgfpathrectangle{\pgfqpoint{0.100000in}{0.100000in}}{\pgfqpoint{3.007045in}{1.925000in}}%
\pgfusepath{clip}%
\pgfsetbuttcap%
\pgfsetmiterjoin%
\definecolor{currentfill}{rgb}{0.306113,0.604844,0.794925}%
\pgfsetfillcolor{currentfill}%
\pgfsetlinewidth{0.000000pt}%
\definecolor{currentstroke}{rgb}{0.000000,0.000000,0.000000}%
\pgfsetstrokecolor{currentstroke}%
\pgfsetstrokeopacity{0.000000}%
\pgfsetdash{}{0pt}%
\pgfpathmoveto{\pgfqpoint{1.526290in}{0.791703in}}%
\pgfpathlineto{\pgfqpoint{1.525327in}{0.794045in}}%
\pgfpathlineto{\pgfqpoint{1.514372in}{0.798412in}}%
\pgfpathlineto{\pgfqpoint{1.495136in}{0.800124in}}%
\pgfpathlineto{\pgfqpoint{1.496660in}{0.826118in}}%
\pgfpathlineto{\pgfqpoint{1.499337in}{0.825611in}}%
\pgfpathlineto{\pgfqpoint{1.500565in}{0.847796in}}%
\pgfpathlineto{\pgfqpoint{1.504435in}{0.848612in}}%
\pgfpathlineto{\pgfqpoint{1.516379in}{0.834892in}}%
\pgfpathlineto{\pgfqpoint{1.521539in}{0.834387in}}%
\pgfpathlineto{\pgfqpoint{1.526280in}{0.836772in}}%
\pgfpathlineto{\pgfqpoint{1.531947in}{0.833744in}}%
\pgfpathlineto{\pgfqpoint{1.533732in}{0.839225in}}%
\pgfpathlineto{\pgfqpoint{1.542619in}{0.830859in}}%
\pgfpathlineto{\pgfqpoint{1.543414in}{0.822716in}}%
\pgfpathlineto{\pgfqpoint{1.556351in}{0.822025in}}%
\pgfpathlineto{\pgfqpoint{1.555230in}{0.797022in}}%
\pgfpathlineto{\pgfqpoint{1.526556in}{0.798349in}}%
\pgfpathlineto{\pgfqpoint{1.526290in}{0.791703in}}%
\pgfpathclose%
\pgfusepath{fill}%
\end{pgfscope}%
\begin{pgfscope}%
\pgfpathrectangle{\pgfqpoint{0.100000in}{0.100000in}}{\pgfqpoint{3.007045in}{1.925000in}}%
\pgfusepath{clip}%
\pgfsetbuttcap%
\pgfsetmiterjoin%
\definecolor{currentfill}{rgb}{0.285936,0.591065,0.787051}%
\pgfsetfillcolor{currentfill}%
\pgfsetlinewidth{0.000000pt}%
\definecolor{currentstroke}{rgb}{0.000000,0.000000,0.000000}%
\pgfsetstrokecolor{currentstroke}%
\pgfsetstrokeopacity{0.000000}%
\pgfsetdash{}{0pt}%
\pgfpathmoveto{\pgfqpoint{0.819433in}{0.344425in}}%
\pgfpathlineto{\pgfqpoint{0.816099in}{0.346069in}}%
\pgfpathlineto{\pgfqpoint{0.814545in}{0.348123in}}%
\pgfpathlineto{\pgfqpoint{0.812387in}{0.348661in}}%
\pgfpathlineto{\pgfqpoint{0.812849in}{0.351051in}}%
\pgfpathlineto{\pgfqpoint{0.812080in}{0.351861in}}%
\pgfpathlineto{\pgfqpoint{0.810038in}{0.350072in}}%
\pgfpathlineto{\pgfqpoint{0.808207in}{0.349391in}}%
\pgfpathlineto{\pgfqpoint{0.806954in}{0.350824in}}%
\pgfpathlineto{\pgfqpoint{0.805473in}{0.350277in}}%
\pgfpathlineto{\pgfqpoint{0.804388in}{0.347206in}}%
\pgfpathlineto{\pgfqpoint{0.802543in}{0.351436in}}%
\pgfpathlineto{\pgfqpoint{0.802043in}{0.348092in}}%
\pgfpathlineto{\pgfqpoint{0.798587in}{0.347926in}}%
\pgfpathlineto{\pgfqpoint{0.798087in}{0.346708in}}%
\pgfpathlineto{\pgfqpoint{0.795113in}{0.346802in}}%
\pgfpathlineto{\pgfqpoint{0.791694in}{0.347913in}}%
\pgfpathlineto{\pgfqpoint{0.786584in}{0.347123in}}%
\pgfpathlineto{\pgfqpoint{0.784356in}{0.348068in}}%
\pgfpathlineto{\pgfqpoint{0.783495in}{0.346619in}}%
\pgfpathlineto{\pgfqpoint{0.781504in}{0.347865in}}%
\pgfpathlineto{\pgfqpoint{0.781399in}{0.349966in}}%
\pgfpathlineto{\pgfqpoint{0.779304in}{0.349025in}}%
\pgfpathlineto{\pgfqpoint{0.776366in}{0.349945in}}%
\pgfpathlineto{\pgfqpoint{0.776139in}{0.354814in}}%
\pgfpathlineto{\pgfqpoint{0.778619in}{0.356270in}}%
\pgfpathlineto{\pgfqpoint{0.781930in}{0.355808in}}%
\pgfpathlineto{\pgfqpoint{0.783469in}{0.354208in}}%
\pgfpathlineto{\pgfqpoint{0.785254in}{0.354993in}}%
\pgfpathlineto{\pgfqpoint{0.787547in}{0.354040in}}%
\pgfpathlineto{\pgfqpoint{0.788336in}{0.355429in}}%
\pgfpathlineto{\pgfqpoint{0.793181in}{0.356662in}}%
\pgfpathlineto{\pgfqpoint{0.790991in}{0.357629in}}%
\pgfpathlineto{\pgfqpoint{0.788388in}{0.357382in}}%
\pgfpathlineto{\pgfqpoint{0.785835in}{0.356265in}}%
\pgfpathlineto{\pgfqpoint{0.783898in}{0.359777in}}%
\pgfpathlineto{\pgfqpoint{0.783523in}{0.362343in}}%
\pgfpathlineto{\pgfqpoint{0.789148in}{0.366853in}}%
\pgfpathlineto{\pgfqpoint{0.793181in}{0.368023in}}%
\pgfpathlineto{\pgfqpoint{0.797910in}{0.370814in}}%
\pgfpathlineto{\pgfqpoint{0.800516in}{0.373276in}}%
\pgfpathlineto{\pgfqpoint{0.801383in}{0.377874in}}%
\pgfpathlineto{\pgfqpoint{0.803675in}{0.378105in}}%
\pgfpathlineto{\pgfqpoint{0.805964in}{0.377296in}}%
\pgfpathlineto{\pgfqpoint{0.811186in}{0.377817in}}%
\pgfpathlineto{\pgfqpoint{0.816257in}{0.377436in}}%
\pgfpathlineto{\pgfqpoint{0.816031in}{0.374807in}}%
\pgfpathlineto{\pgfqpoint{0.817543in}{0.372687in}}%
\pgfpathlineto{\pgfqpoint{0.822055in}{0.372655in}}%
\pgfpathlineto{\pgfqpoint{0.820708in}{0.377871in}}%
\pgfpathlineto{\pgfqpoint{0.822622in}{0.379076in}}%
\pgfpathlineto{\pgfqpoint{0.818515in}{0.382126in}}%
\pgfpathlineto{\pgfqpoint{0.817100in}{0.384096in}}%
\pgfpathlineto{\pgfqpoint{0.812559in}{0.384716in}}%
\pgfpathlineto{\pgfqpoint{0.809430in}{0.383487in}}%
\pgfpathlineto{\pgfqpoint{0.804683in}{0.384619in}}%
\pgfpathlineto{\pgfqpoint{0.800098in}{0.383859in}}%
\pgfpathlineto{\pgfqpoint{0.798506in}{0.380328in}}%
\pgfpathlineto{\pgfqpoint{0.797580in}{0.381676in}}%
\pgfpathlineto{\pgfqpoint{0.795111in}{0.381646in}}%
\pgfpathlineto{\pgfqpoint{0.790374in}{0.380576in}}%
\pgfpathlineto{\pgfqpoint{0.787763in}{0.377740in}}%
\pgfpathlineto{\pgfqpoint{0.785721in}{0.376868in}}%
\pgfpathlineto{\pgfqpoint{0.783285in}{0.376730in}}%
\pgfpathlineto{\pgfqpoint{0.781683in}{0.377953in}}%
\pgfpathlineto{\pgfqpoint{0.781141in}{0.373732in}}%
\pgfpathlineto{\pgfqpoint{0.777419in}{0.371708in}}%
\pgfpathlineto{\pgfqpoint{0.775658in}{0.371988in}}%
\pgfpathlineto{\pgfqpoint{0.774292in}{0.373277in}}%
\pgfpathlineto{\pgfqpoint{0.771887in}{0.370547in}}%
\pgfpathlineto{\pgfqpoint{0.769799in}{0.370122in}}%
\pgfpathlineto{\pgfqpoint{0.764543in}{0.372414in}}%
\pgfpathlineto{\pgfqpoint{0.763348in}{0.371469in}}%
\pgfpathlineto{\pgfqpoint{0.761565in}{0.371604in}}%
\pgfpathlineto{\pgfqpoint{0.760574in}{0.369853in}}%
\pgfpathlineto{\pgfqpoint{0.756927in}{0.371271in}}%
\pgfpathlineto{\pgfqpoint{0.753478in}{0.368527in}}%
\pgfpathlineto{\pgfqpoint{0.751025in}{0.367939in}}%
\pgfpathlineto{\pgfqpoint{0.752269in}{0.365262in}}%
\pgfpathlineto{\pgfqpoint{0.755024in}{0.362772in}}%
\pgfpathlineto{\pgfqpoint{0.757010in}{0.357656in}}%
\pgfpathlineto{\pgfqpoint{0.757017in}{0.355194in}}%
\pgfpathlineto{\pgfqpoint{0.755001in}{0.356698in}}%
\pgfpathlineto{\pgfqpoint{0.753292in}{0.354357in}}%
\pgfpathlineto{\pgfqpoint{0.749665in}{0.357072in}}%
\pgfpathlineto{\pgfqpoint{0.748423in}{0.355398in}}%
\pgfpathlineto{\pgfqpoint{0.743549in}{0.359090in}}%
\pgfpathlineto{\pgfqpoint{0.740274in}{0.361603in}}%
\pgfpathlineto{\pgfqpoint{0.741011in}{0.363689in}}%
\pgfpathlineto{\pgfqpoint{0.746018in}{0.370169in}}%
\pgfpathlineto{\pgfqpoint{0.745512in}{0.370591in}}%
\pgfpathlineto{\pgfqpoint{0.748076in}{0.373959in}}%
\pgfpathlineto{\pgfqpoint{0.751397in}{0.371451in}}%
\pgfpathlineto{\pgfqpoint{0.753894in}{0.374816in}}%
\pgfpathlineto{\pgfqpoint{0.756787in}{0.372685in}}%
\pgfpathlineto{\pgfqpoint{0.758045in}{0.374400in}}%
\pgfpathlineto{\pgfqpoint{0.759707in}{0.373154in}}%
\pgfpathlineto{\pgfqpoint{0.762258in}{0.376547in}}%
\pgfpathlineto{\pgfqpoint{0.763938in}{0.375291in}}%
\pgfpathlineto{\pgfqpoint{0.765233in}{0.376990in}}%
\pgfpathlineto{\pgfqpoint{0.766473in}{0.376050in}}%
\pgfpathlineto{\pgfqpoint{0.771354in}{0.382565in}}%
\pgfpathlineto{\pgfqpoint{0.772579in}{0.381674in}}%
\pgfpathlineto{\pgfqpoint{0.777700in}{0.388433in}}%
\pgfpathlineto{\pgfqpoint{0.778986in}{0.387487in}}%
\pgfpathlineto{\pgfqpoint{0.784096in}{0.394320in}}%
\pgfpathlineto{\pgfqpoint{0.784950in}{0.396276in}}%
\pgfpathlineto{\pgfqpoint{0.788663in}{0.401187in}}%
\pgfpathlineto{\pgfqpoint{0.789615in}{0.403627in}}%
\pgfpathlineto{\pgfqpoint{0.792854in}{0.407923in}}%
\pgfpathlineto{\pgfqpoint{0.796295in}{0.405189in}}%
\pgfpathlineto{\pgfqpoint{0.796605in}{0.403352in}}%
\pgfpathlineto{\pgfqpoint{0.809857in}{0.421160in}}%
\pgfpathlineto{\pgfqpoint{0.816153in}{0.429462in}}%
\pgfpathlineto{\pgfqpoint{0.826027in}{0.421875in}}%
\pgfpathlineto{\pgfqpoint{0.827109in}{0.423257in}}%
\pgfpathlineto{\pgfqpoint{0.847814in}{0.419834in}}%
\pgfpathlineto{\pgfqpoint{0.856121in}{0.418510in}}%
\pgfpathlineto{\pgfqpoint{0.869886in}{0.409139in}}%
\pgfpathlineto{\pgfqpoint{0.871795in}{0.412013in}}%
\pgfpathlineto{\pgfqpoint{0.875033in}{0.409830in}}%
\pgfpathlineto{\pgfqpoint{0.886359in}{0.402751in}}%
\pgfpathlineto{\pgfqpoint{0.880633in}{0.393554in}}%
\pgfpathlineto{\pgfqpoint{0.878779in}{0.389457in}}%
\pgfpathlineto{\pgfqpoint{0.870796in}{0.376637in}}%
\pgfpathlineto{\pgfqpoint{0.865448in}{0.380135in}}%
\pgfpathlineto{\pgfqpoint{0.864498in}{0.378218in}}%
\pgfpathlineto{\pgfqpoint{0.855608in}{0.363995in}}%
\pgfpathlineto{\pgfqpoint{0.853061in}{0.365671in}}%
\pgfpathlineto{\pgfqpoint{0.852394in}{0.364648in}}%
\pgfpathlineto{\pgfqpoint{0.840140in}{0.372751in}}%
\pgfpathlineto{\pgfqpoint{0.836391in}{0.367548in}}%
\pgfpathlineto{\pgfqpoint{0.832191in}{0.361126in}}%
\pgfpathlineto{\pgfqpoint{0.829442in}{0.362834in}}%
\pgfpathlineto{\pgfqpoint{0.827922in}{0.360561in}}%
\pgfpathlineto{\pgfqpoint{0.828821in}{0.359977in}}%
\pgfpathlineto{\pgfqpoint{0.824494in}{0.353761in}}%
\pgfpathlineto{\pgfqpoint{0.825526in}{0.353057in}}%
\pgfpathlineto{\pgfqpoint{0.821535in}{0.347132in}}%
\pgfpathlineto{\pgfqpoint{0.819433in}{0.344425in}}%
\pgfpathclose%
\pgfusepath{fill}%
\end{pgfscope}%
\begin{pgfscope}%
\pgfpathrectangle{\pgfqpoint{0.100000in}{0.100000in}}{\pgfqpoint{3.007045in}{1.925000in}}%
\pgfusepath{clip}%
\pgfsetbuttcap%
\pgfsetmiterjoin%
\definecolor{currentfill}{rgb}{0.642368,0.801830,0.890319}%
\pgfsetfillcolor{currentfill}%
\pgfsetlinewidth{0.000000pt}%
\definecolor{currentstroke}{rgb}{0.000000,0.000000,0.000000}%
\pgfsetstrokecolor{currentstroke}%
\pgfsetstrokeopacity{0.000000}%
\pgfsetdash{}{0pt}%
\pgfpathmoveto{\pgfqpoint{2.434200in}{1.039055in}}%
\pgfpathlineto{\pgfqpoint{2.439575in}{1.041742in}}%
\pgfpathlineto{\pgfqpoint{2.437440in}{1.045801in}}%
\pgfpathlineto{\pgfqpoint{2.446669in}{1.052394in}}%
\pgfpathlineto{\pgfqpoint{2.440019in}{1.062167in}}%
\pgfpathlineto{\pgfqpoint{2.435154in}{1.067105in}}%
\pgfpathlineto{\pgfqpoint{2.450852in}{1.085445in}}%
\pgfpathlineto{\pgfqpoint{2.453185in}{1.084073in}}%
\pgfpathlineto{\pgfqpoint{2.452938in}{1.079250in}}%
\pgfpathlineto{\pgfqpoint{2.455254in}{1.073522in}}%
\pgfpathlineto{\pgfqpoint{2.462901in}{1.070185in}}%
\pgfpathlineto{\pgfqpoint{2.469086in}{1.065710in}}%
\pgfpathlineto{\pgfqpoint{2.475445in}{1.067081in}}%
\pgfpathlineto{\pgfqpoint{2.483702in}{1.075698in}}%
\pgfpathlineto{\pgfqpoint{2.487921in}{1.088350in}}%
\pgfpathlineto{\pgfqpoint{2.489686in}{1.089390in}}%
\pgfpathlineto{\pgfqpoint{2.490450in}{1.093853in}}%
\pgfpathlineto{\pgfqpoint{2.495503in}{1.095488in}}%
\pgfpathlineto{\pgfqpoint{2.509239in}{1.087027in}}%
\pgfpathlineto{\pgfqpoint{2.510605in}{1.081728in}}%
\pgfpathlineto{\pgfqpoint{2.502786in}{1.075737in}}%
\pgfpathlineto{\pgfqpoint{2.512374in}{1.068702in}}%
\pgfpathlineto{\pgfqpoint{2.508760in}{1.065976in}}%
\pgfpathlineto{\pgfqpoint{2.511594in}{1.060569in}}%
\pgfpathlineto{\pgfqpoint{2.521063in}{1.052157in}}%
\pgfpathlineto{\pgfqpoint{2.505363in}{1.044619in}}%
\pgfpathlineto{\pgfqpoint{2.503517in}{1.042080in}}%
\pgfpathlineto{\pgfqpoint{2.495062in}{1.040390in}}%
\pgfpathlineto{\pgfqpoint{2.490316in}{1.048764in}}%
\pgfpathlineto{\pgfqpoint{2.483475in}{1.055101in}}%
\pgfpathlineto{\pgfqpoint{2.469632in}{1.050830in}}%
\pgfpathlineto{\pgfqpoint{2.464192in}{1.048074in}}%
\pgfpathlineto{\pgfqpoint{2.461325in}{1.043806in}}%
\pgfpathlineto{\pgfqpoint{2.451559in}{1.040086in}}%
\pgfpathlineto{\pgfqpoint{2.450725in}{1.036221in}}%
\pgfpathlineto{\pgfqpoint{2.439622in}{1.028698in}}%
\pgfpathlineto{\pgfqpoint{2.434200in}{1.039055in}}%
\pgfpathclose%
\pgfusepath{fill}%
\end{pgfscope}%
\begin{pgfscope}%
\pgfpathrectangle{\pgfqpoint{0.100000in}{0.100000in}}{\pgfqpoint{3.007045in}{1.925000in}}%
\pgfusepath{clip}%
\pgfsetbuttcap%
\pgfsetmiterjoin%
\definecolor{currentfill}{rgb}{0.296025,0.597955,0.790988}%
\pgfsetfillcolor{currentfill}%
\pgfsetlinewidth{0.000000pt}%
\definecolor{currentstroke}{rgb}{0.000000,0.000000,0.000000}%
\pgfsetstrokecolor{currentstroke}%
\pgfsetstrokeopacity{0.000000}%
\pgfsetdash{}{0pt}%
\pgfpathmoveto{\pgfqpoint{1.570193in}{1.688446in}}%
\pgfpathlineto{\pgfqpoint{1.569042in}{1.665415in}}%
\pgfpathlineto{\pgfqpoint{1.570614in}{1.665351in}}%
\pgfpathlineto{\pgfqpoint{1.569331in}{1.642166in}}%
\pgfpathlineto{\pgfqpoint{1.540553in}{1.643789in}}%
\pgfpathlineto{\pgfqpoint{1.542129in}{1.666889in}}%
\pgfpathlineto{\pgfqpoint{1.540344in}{1.666994in}}%
\pgfpathlineto{\pgfqpoint{1.541684in}{1.689945in}}%
\pgfpathlineto{\pgfqpoint{1.570193in}{1.688446in}}%
\pgfpathclose%
\pgfusepath{fill}%
\end{pgfscope}%
\begin{pgfscope}%
\pgfpathrectangle{\pgfqpoint{0.100000in}{0.100000in}}{\pgfqpoint{3.007045in}{1.925000in}}%
\pgfusepath{clip}%
\pgfsetbuttcap%
\pgfsetmiterjoin%
\definecolor{currentfill}{rgb}{0.108651,0.416563,0.689043}%
\pgfsetfillcolor{currentfill}%
\pgfsetlinewidth{0.000000pt}%
\definecolor{currentstroke}{rgb}{0.000000,0.000000,0.000000}%
\pgfsetstrokecolor{currentstroke}%
\pgfsetstrokeopacity{0.000000}%
\pgfsetdash{}{0pt}%
\pgfpathmoveto{\pgfqpoint{1.456990in}{0.978623in}}%
\pgfpathlineto{\pgfqpoint{1.507272in}{0.975589in}}%
\pgfpathlineto{\pgfqpoint{1.505836in}{0.946347in}}%
\pgfpathlineto{\pgfqpoint{1.477103in}{0.948069in}}%
\pgfpathlineto{\pgfqpoint{1.419469in}{0.951860in}}%
\pgfpathlineto{\pgfqpoint{1.421577in}{0.981072in}}%
\pgfpathlineto{\pgfqpoint{1.456990in}{0.978623in}}%
\pgfpathclose%
\pgfusepath{fill}%
\end{pgfscope}%
\begin{pgfscope}%
\pgfpathrectangle{\pgfqpoint{0.100000in}{0.100000in}}{\pgfqpoint{3.007045in}{1.925000in}}%
\pgfusepath{clip}%
\pgfsetbuttcap%
\pgfsetmiterjoin%
\definecolor{currentfill}{rgb}{0.361599,0.642737,0.816578}%
\pgfsetfillcolor{currentfill}%
\pgfsetlinewidth{0.000000pt}%
\definecolor{currentstroke}{rgb}{0.000000,0.000000,0.000000}%
\pgfsetstrokecolor{currentstroke}%
\pgfsetstrokeopacity{0.000000}%
\pgfsetdash{}{0pt}%
\pgfpathmoveto{\pgfqpoint{2.196663in}{1.378800in}}%
\pgfpathlineto{\pgfqpoint{2.198102in}{1.398160in}}%
\pgfpathlineto{\pgfqpoint{2.196389in}{1.411349in}}%
\pgfpathlineto{\pgfqpoint{2.193000in}{1.423406in}}%
\pgfpathlineto{\pgfqpoint{2.183445in}{1.440623in}}%
\pgfpathlineto{\pgfqpoint{2.177072in}{1.454810in}}%
\pgfpathlineto{\pgfqpoint{2.176584in}{1.460340in}}%
\pgfpathlineto{\pgfqpoint{2.180375in}{1.470113in}}%
\pgfpathlineto{\pgfqpoint{2.199018in}{1.471645in}}%
\pgfpathlineto{\pgfqpoint{2.221626in}{1.473931in}}%
\pgfpathlineto{\pgfqpoint{2.224008in}{1.451122in}}%
\pgfpathlineto{\pgfqpoint{2.246778in}{1.453408in}}%
\pgfpathlineto{\pgfqpoint{2.258163in}{1.454659in}}%
\pgfpathlineto{\pgfqpoint{2.261163in}{1.431916in}}%
\pgfpathlineto{\pgfqpoint{2.263712in}{1.409069in}}%
\pgfpathlineto{\pgfqpoint{2.252299in}{1.407809in}}%
\pgfpathlineto{\pgfqpoint{2.229609in}{1.405210in}}%
\pgfpathlineto{\pgfqpoint{2.232094in}{1.382445in}}%
\pgfpathlineto{\pgfqpoint{2.196663in}{1.378800in}}%
\pgfpathclose%
\pgfusepath{fill}%
\end{pgfscope}%
\begin{pgfscope}%
\pgfpathrectangle{\pgfqpoint{0.100000in}{0.100000in}}{\pgfqpoint{3.007045in}{1.925000in}}%
\pgfusepath{clip}%
\pgfsetbuttcap%
\pgfsetmiterjoin%
\definecolor{currentfill}{rgb}{0.130427,0.444152,0.710327}%
\pgfsetfillcolor{currentfill}%
\pgfsetlinewidth{0.000000pt}%
\definecolor{currentstroke}{rgb}{0.000000,0.000000,0.000000}%
\pgfsetstrokecolor{currentstroke}%
\pgfsetstrokeopacity{0.000000}%
\pgfsetdash{}{0pt}%
\pgfpathmoveto{\pgfqpoint{0.973337in}{1.262333in}}%
\pgfpathlineto{\pgfqpoint{0.963825in}{1.264056in}}%
\pgfpathlineto{\pgfqpoint{0.964301in}{1.266613in}}%
\pgfpathlineto{\pgfqpoint{0.957377in}{1.275606in}}%
\pgfpathlineto{\pgfqpoint{0.959849in}{1.282582in}}%
\pgfpathlineto{\pgfqpoint{0.957665in}{1.289984in}}%
\pgfpathlineto{\pgfqpoint{0.950156in}{1.292847in}}%
\pgfpathlineto{\pgfqpoint{0.943668in}{1.304520in}}%
\pgfpathlineto{\pgfqpoint{0.947451in}{1.310370in}}%
\pgfpathlineto{\pgfqpoint{0.937610in}{1.310762in}}%
\pgfpathlineto{\pgfqpoint{0.925436in}{1.305547in}}%
\pgfpathlineto{\pgfqpoint{0.923286in}{1.309346in}}%
\pgfpathlineto{\pgfqpoint{0.914647in}{1.310630in}}%
\pgfpathlineto{\pgfqpoint{0.913219in}{1.297816in}}%
\pgfpathlineto{\pgfqpoint{0.914434in}{1.291420in}}%
\pgfpathlineto{\pgfqpoint{0.910706in}{1.281361in}}%
\pgfpathlineto{\pgfqpoint{0.901338in}{1.275958in}}%
\pgfpathlineto{\pgfqpoint{0.878712in}{1.280320in}}%
\pgfpathlineto{\pgfqpoint{0.816917in}{1.293210in}}%
\pgfpathlineto{\pgfqpoint{0.832155in}{1.363654in}}%
\pgfpathlineto{\pgfqpoint{0.892707in}{1.351230in}}%
\pgfpathlineto{\pgfqpoint{0.908601in}{1.353242in}}%
\pgfpathlineto{\pgfqpoint{0.924294in}{1.367661in}}%
\pgfpathlineto{\pgfqpoint{0.934476in}{1.365698in}}%
\pgfpathlineto{\pgfqpoint{0.942058in}{1.369750in}}%
\pgfpathlineto{\pgfqpoint{0.947822in}{1.365378in}}%
\pgfpathlineto{\pgfqpoint{0.951603in}{1.367889in}}%
\pgfpathlineto{\pgfqpoint{0.960688in}{1.366618in}}%
\pgfpathlineto{\pgfqpoint{0.963794in}{1.361956in}}%
\pgfpathlineto{\pgfqpoint{0.970085in}{1.358692in}}%
\pgfpathlineto{\pgfqpoint{0.969963in}{1.352958in}}%
\pgfpathlineto{\pgfqpoint{0.972493in}{1.347031in}}%
\pgfpathlineto{\pgfqpoint{0.981285in}{1.351407in}}%
\pgfpathlineto{\pgfqpoint{0.978339in}{1.334944in}}%
\pgfpathlineto{\pgfqpoint{1.005052in}{1.330045in}}%
\pgfpathlineto{\pgfqpoint{1.029601in}{1.326060in}}%
\pgfpathlineto{\pgfqpoint{1.027613in}{1.314125in}}%
\pgfpathlineto{\pgfqpoint{1.019386in}{1.315423in}}%
\pgfpathlineto{\pgfqpoint{1.015074in}{1.317685in}}%
\pgfpathlineto{\pgfqpoint{1.008923in}{1.315572in}}%
\pgfpathlineto{\pgfqpoint{0.994498in}{1.314874in}}%
\pgfpathlineto{\pgfqpoint{0.989953in}{1.316140in}}%
\pgfpathlineto{\pgfqpoint{0.981761in}{1.312933in}}%
\pgfpathlineto{\pgfqpoint{0.973337in}{1.262333in}}%
\pgfpathclose%
\pgfusepath{fill}%
\end{pgfscope}%
\begin{pgfscope}%
\pgfpathrectangle{\pgfqpoint{0.100000in}{0.100000in}}{\pgfqpoint{3.007045in}{1.925000in}}%
\pgfusepath{clip}%
\pgfsetbuttcap%
\pgfsetmiterjoin%
\definecolor{currentfill}{rgb}{0.215686,0.529412,0.754248}%
\pgfsetfillcolor{currentfill}%
\pgfsetlinewidth{0.000000pt}%
\definecolor{currentstroke}{rgb}{0.000000,0.000000,0.000000}%
\pgfsetstrokecolor{currentstroke}%
\pgfsetstrokeopacity{0.000000}%
\pgfsetdash{}{0pt}%
\pgfpathmoveto{\pgfqpoint{1.730711in}{1.250739in}}%
\pgfpathlineto{\pgfqpoint{1.699296in}{1.251149in}}%
\pgfpathlineto{\pgfqpoint{1.699583in}{1.268350in}}%
\pgfpathlineto{\pgfqpoint{1.677475in}{1.268823in}}%
\pgfpathlineto{\pgfqpoint{1.678558in}{1.314735in}}%
\pgfpathlineto{\pgfqpoint{1.701276in}{1.314251in}}%
\pgfpathlineto{\pgfqpoint{1.701191in}{1.310424in}}%
\pgfpathlineto{\pgfqpoint{1.717094in}{1.310143in}}%
\pgfpathlineto{\pgfqpoint{1.719990in}{1.317507in}}%
\pgfpathlineto{\pgfqpoint{1.716390in}{1.322247in}}%
\pgfpathlineto{\pgfqpoint{1.739290in}{1.321824in}}%
\pgfpathlineto{\pgfqpoint{1.744960in}{1.321802in}}%
\pgfpathlineto{\pgfqpoint{1.744815in}{1.304579in}}%
\pgfpathlineto{\pgfqpoint{1.747728in}{1.298216in}}%
\pgfpathlineto{\pgfqpoint{1.764683in}{1.298101in}}%
\pgfpathlineto{\pgfqpoint{1.764595in}{1.275306in}}%
\pgfpathlineto{\pgfqpoint{1.753309in}{1.275315in}}%
\pgfpathlineto{\pgfqpoint{1.753153in}{1.258349in}}%
\pgfpathlineto{\pgfqpoint{1.731811in}{1.258499in}}%
\pgfpathlineto{\pgfqpoint{1.730711in}{1.250739in}}%
\pgfpathclose%
\pgfusepath{fill}%
\end{pgfscope}%
\begin{pgfscope}%
\pgfpathrectangle{\pgfqpoint{0.100000in}{0.100000in}}{\pgfqpoint{3.007045in}{1.925000in}}%
\pgfusepath{clip}%
\pgfsetbuttcap%
\pgfsetmiterjoin%
\definecolor{currentfill}{rgb}{0.371688,0.649627,0.820515}%
\pgfsetfillcolor{currentfill}%
\pgfsetlinewidth{0.000000pt}%
\definecolor{currentstroke}{rgb}{0.000000,0.000000,0.000000}%
\pgfsetstrokecolor{currentstroke}%
\pgfsetstrokeopacity{0.000000}%
\pgfsetdash{}{0pt}%
\pgfpathmoveto{\pgfqpoint{1.928151in}{1.455464in}}%
\pgfpathlineto{\pgfqpoint{1.928977in}{1.432597in}}%
\pgfpathlineto{\pgfqpoint{1.894560in}{1.431480in}}%
\pgfpathlineto{\pgfqpoint{1.893937in}{1.453516in}}%
\pgfpathlineto{\pgfqpoint{1.882427in}{1.454160in}}%
\pgfpathlineto{\pgfqpoint{1.865473in}{1.453759in}}%
\pgfpathlineto{\pgfqpoint{1.864898in}{1.494975in}}%
\pgfpathlineto{\pgfqpoint{1.864831in}{1.497891in}}%
\pgfpathlineto{\pgfqpoint{1.876437in}{1.499997in}}%
\pgfpathlineto{\pgfqpoint{1.876275in}{1.505700in}}%
\pgfpathlineto{\pgfqpoint{1.884271in}{1.505208in}}%
\pgfpathlineto{\pgfqpoint{1.889296in}{1.502422in}}%
\pgfpathlineto{\pgfqpoint{1.898889in}{1.500432in}}%
\pgfpathlineto{\pgfqpoint{1.903859in}{1.493796in}}%
\pgfpathlineto{\pgfqpoint{1.911427in}{1.491667in}}%
\pgfpathlineto{\pgfqpoint{1.917765in}{1.487298in}}%
\pgfpathlineto{\pgfqpoint{1.921271in}{1.478116in}}%
\pgfpathlineto{\pgfqpoint{1.910791in}{1.477769in}}%
\pgfpathlineto{\pgfqpoint{1.911530in}{1.454907in}}%
\pgfpathlineto{\pgfqpoint{1.928151in}{1.455464in}}%
\pgfpathclose%
\pgfusepath{fill}%
\end{pgfscope}%
\begin{pgfscope}%
\pgfpathrectangle{\pgfqpoint{0.100000in}{0.100000in}}{\pgfqpoint{3.007045in}{1.925000in}}%
\pgfusepath{clip}%
\pgfsetbuttcap%
\pgfsetmiterjoin%
\definecolor{currentfill}{rgb}{0.290980,0.594510,0.789020}%
\pgfsetfillcolor{currentfill}%
\pgfsetlinewidth{0.000000pt}%
\definecolor{currentstroke}{rgb}{0.000000,0.000000,0.000000}%
\pgfsetstrokecolor{currentstroke}%
\pgfsetstrokeopacity{0.000000}%
\pgfsetdash{}{0pt}%
\pgfpathmoveto{\pgfqpoint{1.637239in}{1.477803in}}%
\pgfpathlineto{\pgfqpoint{1.682752in}{1.476661in}}%
\pgfpathlineto{\pgfqpoint{1.682306in}{1.453680in}}%
\pgfpathlineto{\pgfqpoint{1.648008in}{1.454576in}}%
\pgfpathlineto{\pgfqpoint{1.636440in}{1.454988in}}%
\pgfpathlineto{\pgfqpoint{1.637239in}{1.477803in}}%
\pgfpathclose%
\pgfusepath{fill}%
\end{pgfscope}%
\begin{pgfscope}%
\pgfpathrectangle{\pgfqpoint{0.100000in}{0.100000in}}{\pgfqpoint{3.007045in}{1.925000in}}%
\pgfusepath{clip}%
\pgfsetbuttcap%
\pgfsetmiterjoin%
\definecolor{currentfill}{rgb}{0.316201,0.611734,0.798862}%
\pgfsetfillcolor{currentfill}%
\pgfsetlinewidth{0.000000pt}%
\definecolor{currentstroke}{rgb}{0.000000,0.000000,0.000000}%
\pgfsetstrokecolor{currentstroke}%
\pgfsetstrokeopacity{0.000000}%
\pgfsetdash{}{0pt}%
\pgfpathmoveto{\pgfqpoint{1.755176in}{1.198941in}}%
\pgfpathlineto{\pgfqpoint{1.755095in}{1.176082in}}%
\pgfpathlineto{\pgfqpoint{1.743716in}{1.176139in}}%
\pgfpathlineto{\pgfqpoint{1.732315in}{1.176225in}}%
\pgfpathlineto{\pgfqpoint{1.732247in}{1.170499in}}%
\pgfpathlineto{\pgfqpoint{1.719727in}{1.170628in}}%
\pgfpathlineto{\pgfqpoint{1.709420in}{1.170767in}}%
\pgfpathlineto{\pgfqpoint{1.709799in}{1.199381in}}%
\pgfpathlineto{\pgfqpoint{1.755176in}{1.198941in}}%
\pgfpathclose%
\pgfusepath{fill}%
\end{pgfscope}%
\begin{pgfscope}%
\pgfpathrectangle{\pgfqpoint{0.100000in}{0.100000in}}{\pgfqpoint{3.007045in}{1.925000in}}%
\pgfusepath{clip}%
\pgfsetbuttcap%
\pgfsetmiterjoin%
\definecolor{currentfill}{rgb}{0.523137,0.739193,0.861546}%
\pgfsetfillcolor{currentfill}%
\pgfsetlinewidth{0.000000pt}%
\definecolor{currentstroke}{rgb}{0.000000,0.000000,0.000000}%
\pgfsetstrokecolor{currentstroke}%
\pgfsetstrokeopacity{0.000000}%
\pgfsetdash{}{0pt}%
\pgfpathmoveto{\pgfqpoint{2.029155in}{1.469240in}}%
\pgfpathlineto{\pgfqpoint{2.030394in}{1.446824in}}%
\pgfpathlineto{\pgfqpoint{2.021528in}{1.446204in}}%
\pgfpathlineto{\pgfqpoint{2.017195in}{1.455365in}}%
\pgfpathlineto{\pgfqpoint{2.012501in}{1.460181in}}%
\pgfpathlineto{\pgfqpoint{2.012462in}{1.466036in}}%
\pgfpathlineto{\pgfqpoint{2.008307in}{1.475250in}}%
\pgfpathlineto{\pgfqpoint{2.013699in}{1.486047in}}%
\pgfpathlineto{\pgfqpoint{1.994330in}{1.484956in}}%
\pgfpathlineto{\pgfqpoint{1.993541in}{1.496568in}}%
\pgfpathlineto{\pgfqpoint{1.970609in}{1.495307in}}%
\pgfpathlineto{\pgfqpoint{1.970355in}{1.501053in}}%
\pgfpathlineto{\pgfqpoint{1.964639in}{1.500815in}}%
\pgfpathlineto{\pgfqpoint{1.962214in}{1.552559in}}%
\pgfpathlineto{\pgfqpoint{1.961960in}{1.558341in}}%
\pgfpathlineto{\pgfqpoint{1.973425in}{1.558760in}}%
\pgfpathlineto{\pgfqpoint{2.003003in}{1.560543in}}%
\pgfpathlineto{\pgfqpoint{2.002328in}{1.571903in}}%
\pgfpathlineto{\pgfqpoint{2.030834in}{1.573678in}}%
\pgfpathlineto{\pgfqpoint{2.031351in}{1.567858in}}%
\pgfpathlineto{\pgfqpoint{2.032705in}{1.544832in}}%
\pgfpathlineto{\pgfqpoint{2.042126in}{1.545379in}}%
\pgfpathlineto{\pgfqpoint{2.043986in}{1.516491in}}%
\pgfpathlineto{\pgfqpoint{2.045773in}{1.487575in}}%
\pgfpathlineto{\pgfqpoint{2.028120in}{1.486633in}}%
\pgfpathlineto{\pgfqpoint{2.029155in}{1.469240in}}%
\pgfpathclose%
\pgfusepath{fill}%
\end{pgfscope}%
\begin{pgfscope}%
\pgfpathrectangle{\pgfqpoint{0.100000in}{0.100000in}}{\pgfqpoint{3.007045in}{1.925000in}}%
\pgfusepath{clip}%
\pgfsetbuttcap%
\pgfsetmiterjoin%
\definecolor{currentfill}{rgb}{0.386820,0.659962,0.826421}%
\pgfsetfillcolor{currentfill}%
\pgfsetlinewidth{0.000000pt}%
\definecolor{currentstroke}{rgb}{0.000000,0.000000,0.000000}%
\pgfsetstrokecolor{currentstroke}%
\pgfsetstrokeopacity{0.000000}%
\pgfsetdash{}{0pt}%
\pgfpathmoveto{\pgfqpoint{0.459905in}{1.401168in}}%
\pgfpathlineto{\pgfqpoint{0.451708in}{1.400095in}}%
\pgfpathlineto{\pgfqpoint{0.446584in}{1.393519in}}%
\pgfpathlineto{\pgfqpoint{0.443854in}{1.394476in}}%
\pgfpathlineto{\pgfqpoint{0.437071in}{1.389645in}}%
\pgfpathlineto{\pgfqpoint{0.425415in}{1.393166in}}%
\pgfpathlineto{\pgfqpoint{0.423677in}{1.387762in}}%
\pgfpathlineto{\pgfqpoint{0.380635in}{1.401231in}}%
\pgfpathlineto{\pgfqpoint{0.384397in}{1.412475in}}%
\pgfpathlineto{\pgfqpoint{0.385058in}{1.423808in}}%
\pgfpathlineto{\pgfqpoint{0.387316in}{1.430255in}}%
\pgfpathlineto{\pgfqpoint{0.384276in}{1.433228in}}%
\pgfpathlineto{\pgfqpoint{0.385422in}{1.437519in}}%
\pgfpathlineto{\pgfqpoint{0.390077in}{1.441086in}}%
\pgfpathlineto{\pgfqpoint{0.395246in}{1.441320in}}%
\pgfpathlineto{\pgfqpoint{0.399108in}{1.444237in}}%
\pgfpathlineto{\pgfqpoint{0.407911in}{1.446662in}}%
\pgfpathlineto{\pgfqpoint{0.407285in}{1.452726in}}%
\pgfpathlineto{\pgfqpoint{0.412778in}{1.458416in}}%
\pgfpathlineto{\pgfqpoint{0.420836in}{1.470570in}}%
\pgfpathlineto{\pgfqpoint{0.424324in}{1.471482in}}%
\pgfpathlineto{\pgfqpoint{0.429023in}{1.481634in}}%
\pgfpathlineto{\pgfqpoint{0.484393in}{1.464821in}}%
\pgfpathlineto{\pgfqpoint{0.479723in}{1.447096in}}%
\pgfpathlineto{\pgfqpoint{0.470680in}{1.418127in}}%
\pgfpathlineto{\pgfqpoint{0.462527in}{1.420581in}}%
\pgfpathlineto{\pgfqpoint{0.462117in}{1.414113in}}%
\pgfpathlineto{\pgfqpoint{0.466986in}{1.410310in}}%
\pgfpathlineto{\pgfqpoint{0.464449in}{1.404081in}}%
\pgfpathlineto{\pgfqpoint{0.459905in}{1.401168in}}%
\pgfpathclose%
\pgfusepath{fill}%
\end{pgfscope}%
\begin{pgfscope}%
\pgfpathrectangle{\pgfqpoint{0.100000in}{0.100000in}}{\pgfqpoint{3.007045in}{1.925000in}}%
\pgfusepath{clip}%
\pgfsetbuttcap%
\pgfsetmiterjoin%
\definecolor{currentfill}{rgb}{0.391865,0.663406,0.828389}%
\pgfsetfillcolor{currentfill}%
\pgfsetlinewidth{0.000000pt}%
\definecolor{currentstroke}{rgb}{0.000000,0.000000,0.000000}%
\pgfsetstrokecolor{currentstroke}%
\pgfsetstrokeopacity{0.000000}%
\pgfsetdash{}{0pt}%
\pgfpathmoveto{\pgfqpoint{2.256834in}{1.132262in}}%
\pgfpathlineto{\pgfqpoint{2.256292in}{1.137004in}}%
\pgfpathlineto{\pgfqpoint{2.250151in}{1.140216in}}%
\pgfpathlineto{\pgfqpoint{2.249568in}{1.145924in}}%
\pgfpathlineto{\pgfqpoint{2.253275in}{1.146329in}}%
\pgfpathlineto{\pgfqpoint{2.261118in}{1.153025in}}%
\pgfpathlineto{\pgfqpoint{2.273431in}{1.154396in}}%
\pgfpathlineto{\pgfqpoint{2.275093in}{1.139834in}}%
\pgfpathlineto{\pgfqpoint{2.283647in}{1.145438in}}%
\pgfpathlineto{\pgfqpoint{2.288955in}{1.139422in}}%
\pgfpathlineto{\pgfqpoint{2.282277in}{1.134362in}}%
\pgfpathlineto{\pgfqpoint{2.277544in}{1.133106in}}%
\pgfpathlineto{\pgfqpoint{2.272501in}{1.126660in}}%
\pgfpathlineto{\pgfqpoint{2.264410in}{1.127604in}}%
\pgfpathlineto{\pgfqpoint{2.264234in}{1.131721in}}%
\pgfpathlineto{\pgfqpoint{2.256834in}{1.132262in}}%
\pgfpathclose%
\pgfusepath{fill}%
\end{pgfscope}%
\begin{pgfscope}%
\pgfpathrectangle{\pgfqpoint{0.100000in}{0.100000in}}{\pgfqpoint{3.007045in}{1.925000in}}%
\pgfusepath{clip}%
\pgfsetbuttcap%
\pgfsetmiterjoin%
\definecolor{currentfill}{rgb}{0.179146,0.492872,0.735425}%
\pgfsetfillcolor{currentfill}%
\pgfsetlinewidth{0.000000pt}%
\definecolor{currentstroke}{rgb}{0.000000,0.000000,0.000000}%
\pgfsetstrokecolor{currentstroke}%
\pgfsetstrokeopacity{0.000000}%
\pgfsetdash{}{0pt}%
\pgfpathmoveto{\pgfqpoint{1.610215in}{1.316599in}}%
\pgfpathlineto{\pgfqpoint{1.609594in}{1.293697in}}%
\pgfpathlineto{\pgfqpoint{1.587220in}{1.294590in}}%
\pgfpathlineto{\pgfqpoint{1.564548in}{1.295579in}}%
\pgfpathlineto{\pgfqpoint{1.565174in}{1.318418in}}%
\pgfpathlineto{\pgfqpoint{1.541840in}{1.319542in}}%
\pgfpathlineto{\pgfqpoint{1.543010in}{1.342313in}}%
\pgfpathlineto{\pgfqpoint{1.565129in}{1.341345in}}%
\pgfpathlineto{\pgfqpoint{1.588274in}{1.340363in}}%
\pgfpathlineto{\pgfqpoint{1.587433in}{1.317426in}}%
\pgfpathlineto{\pgfqpoint{1.610215in}{1.316599in}}%
\pgfpathclose%
\pgfusepath{fill}%
\end{pgfscope}%
\begin{pgfscope}%
\pgfpathrectangle{\pgfqpoint{0.100000in}{0.100000in}}{\pgfqpoint{3.007045in}{1.925000in}}%
\pgfusepath{clip}%
\pgfsetbuttcap%
\pgfsetmiterjoin%
\definecolor{currentfill}{rgb}{0.199446,0.513172,0.745882}%
\pgfsetfillcolor{currentfill}%
\pgfsetlinewidth{0.000000pt}%
\definecolor{currentstroke}{rgb}{0.000000,0.000000,0.000000}%
\pgfsetstrokecolor{currentstroke}%
\pgfsetstrokeopacity{0.000000}%
\pgfsetdash{}{0pt}%
\pgfpathmoveto{\pgfqpoint{1.656900in}{1.355436in}}%
\pgfpathlineto{\pgfqpoint{1.673964in}{1.354981in}}%
\pgfpathlineto{\pgfqpoint{1.674656in}{1.382036in}}%
\pgfpathlineto{\pgfqpoint{1.684732in}{1.378050in}}%
\pgfpathlineto{\pgfqpoint{1.685257in}{1.403080in}}%
\pgfpathlineto{\pgfqpoint{1.702210in}{1.402774in}}%
\pgfpathlineto{\pgfqpoint{1.699433in}{1.400596in}}%
\pgfpathlineto{\pgfqpoint{1.699908in}{1.394517in}}%
\pgfpathlineto{\pgfqpoint{1.697951in}{1.391327in}}%
\pgfpathlineto{\pgfqpoint{1.730762in}{1.390911in}}%
\pgfpathlineto{\pgfqpoint{1.730520in}{1.367928in}}%
\pgfpathlineto{\pgfqpoint{1.736283in}{1.367899in}}%
\pgfpathlineto{\pgfqpoint{1.739732in}{1.362155in}}%
\pgfpathlineto{\pgfqpoint{1.739541in}{1.344779in}}%
\pgfpathlineto{\pgfqpoint{1.739290in}{1.321824in}}%
\pgfpathlineto{\pgfqpoint{1.716390in}{1.322247in}}%
\pgfpathlineto{\pgfqpoint{1.716907in}{1.329278in}}%
\pgfpathlineto{\pgfqpoint{1.713888in}{1.330082in}}%
\pgfpathlineto{\pgfqpoint{1.706612in}{1.341900in}}%
\pgfpathlineto{\pgfqpoint{1.706125in}{1.349454in}}%
\pgfpathlineto{\pgfqpoint{1.688055in}{1.349861in}}%
\pgfpathlineto{\pgfqpoint{1.683358in}{1.349049in}}%
\pgfpathlineto{\pgfqpoint{1.683057in}{1.337584in}}%
\pgfpathlineto{\pgfqpoint{1.673497in}{1.337786in}}%
\pgfpathlineto{\pgfqpoint{1.656422in}{1.338221in}}%
\pgfpathlineto{\pgfqpoint{1.656900in}{1.355436in}}%
\pgfpathclose%
\pgfusepath{fill}%
\end{pgfscope}%
\begin{pgfscope}%
\pgfpathrectangle{\pgfqpoint{0.100000in}{0.100000in}}{\pgfqpoint{3.007045in}{1.925000in}}%
\pgfusepath{clip}%
\pgfsetbuttcap%
\pgfsetmiterjoin%
\definecolor{currentfill}{rgb}{0.491765,0.721968,0.854779}%
\pgfsetfillcolor{currentfill}%
\pgfsetlinewidth{0.000000pt}%
\definecolor{currentstroke}{rgb}{0.000000,0.000000,0.000000}%
\pgfsetstrokecolor{currentstroke}%
\pgfsetstrokeopacity{0.000000}%
\pgfsetdash{}{0pt}%
\pgfpathmoveto{\pgfqpoint{2.559434in}{0.779009in}}%
\pgfpathlineto{\pgfqpoint{2.556087in}{0.772933in}}%
\pgfpathlineto{\pgfqpoint{2.548564in}{0.780580in}}%
\pgfpathlineto{\pgfqpoint{2.535487in}{0.765030in}}%
\pgfpathlineto{\pgfqpoint{2.529482in}{0.767093in}}%
\pgfpathlineto{\pgfqpoint{2.526324in}{0.773753in}}%
\pgfpathlineto{\pgfqpoint{2.525012in}{0.782861in}}%
\pgfpathlineto{\pgfqpoint{2.519904in}{0.788430in}}%
\pgfpathlineto{\pgfqpoint{2.519777in}{0.793317in}}%
\pgfpathlineto{\pgfqpoint{2.516660in}{0.795332in}}%
\pgfpathlineto{\pgfqpoint{2.507037in}{0.798645in}}%
\pgfpathlineto{\pgfqpoint{2.503376in}{0.803575in}}%
\pgfpathlineto{\pgfqpoint{2.521533in}{0.825837in}}%
\pgfpathlineto{\pgfqpoint{2.525782in}{0.825781in}}%
\pgfpathlineto{\pgfqpoint{2.532363in}{0.822168in}}%
\pgfpathlineto{\pgfqpoint{2.545370in}{0.817990in}}%
\pgfpathlineto{\pgfqpoint{2.556125in}{0.810180in}}%
\pgfpathlineto{\pgfqpoint{2.542055in}{0.797900in}}%
\pgfpathlineto{\pgfqpoint{2.548566in}{0.790531in}}%
\pgfpathlineto{\pgfqpoint{2.554010in}{0.786819in}}%
\pgfpathlineto{\pgfqpoint{2.559434in}{0.779009in}}%
\pgfpathclose%
\pgfusepath{fill}%
\end{pgfscope}%
\begin{pgfscope}%
\pgfpathrectangle{\pgfqpoint{0.100000in}{0.100000in}}{\pgfqpoint{3.007045in}{1.925000in}}%
\pgfusepath{clip}%
\pgfsetbuttcap%
\pgfsetmiterjoin%
\definecolor{currentfill}{rgb}{0.321246,0.615179,0.800830}%
\pgfsetfillcolor{currentfill}%
\pgfsetlinewidth{0.000000pt}%
\definecolor{currentstroke}{rgb}{0.000000,0.000000,0.000000}%
\pgfsetstrokecolor{currentstroke}%
\pgfsetstrokeopacity{0.000000}%
\pgfsetdash{}{0pt}%
\pgfpathmoveto{\pgfqpoint{0.936059in}{0.163351in}}%
\pgfpathlineto{\pgfqpoint{0.936182in}{0.166254in}}%
\pgfpathlineto{\pgfqpoint{0.938191in}{0.164676in}}%
\pgfpathlineto{\pgfqpoint{0.936059in}{0.163351in}}%
\pgfpathclose%
\pgfusepath{fill}%
\end{pgfscope}%
\begin{pgfscope}%
\pgfpathrectangle{\pgfqpoint{0.100000in}{0.100000in}}{\pgfqpoint{3.007045in}{1.925000in}}%
\pgfusepath{clip}%
\pgfsetbuttcap%
\pgfsetmiterjoin%
\definecolor{currentfill}{rgb}{0.321246,0.615179,0.800830}%
\pgfsetfillcolor{currentfill}%
\pgfsetlinewidth{0.000000pt}%
\definecolor{currentstroke}{rgb}{0.000000,0.000000,0.000000}%
\pgfsetstrokecolor{currentstroke}%
\pgfsetstrokeopacity{0.000000}%
\pgfsetdash{}{0pt}%
\pgfpathmoveto{\pgfqpoint{0.933022in}{0.152128in}}%
\pgfpathlineto{\pgfqpoint{0.933743in}{0.154327in}}%
\pgfpathlineto{\pgfqpoint{0.935121in}{0.154547in}}%
\pgfpathlineto{\pgfqpoint{0.934597in}{0.151513in}}%
\pgfpathlineto{\pgfqpoint{0.933022in}{0.152128in}}%
\pgfpathclose%
\pgfusepath{fill}%
\end{pgfscope}%
\begin{pgfscope}%
\pgfpathrectangle{\pgfqpoint{0.100000in}{0.100000in}}{\pgfqpoint{3.007045in}{1.925000in}}%
\pgfusepath{clip}%
\pgfsetbuttcap%
\pgfsetmiterjoin%
\definecolor{currentfill}{rgb}{0.321246,0.615179,0.800830}%
\pgfsetfillcolor{currentfill}%
\pgfsetlinewidth{0.000000pt}%
\definecolor{currentstroke}{rgb}{0.000000,0.000000,0.000000}%
\pgfsetstrokecolor{currentstroke}%
\pgfsetstrokeopacity{0.000000}%
\pgfsetdash{}{0pt}%
\pgfpathmoveto{\pgfqpoint{0.935935in}{0.137518in}}%
\pgfpathlineto{\pgfqpoint{0.934705in}{0.142381in}}%
\pgfpathlineto{\pgfqpoint{0.935452in}{0.143773in}}%
\pgfpathlineto{\pgfqpoint{0.933837in}{0.145038in}}%
\pgfpathlineto{\pgfqpoint{0.935285in}{0.148282in}}%
\pgfpathlineto{\pgfqpoint{0.935191in}{0.150312in}}%
\pgfpathlineto{\pgfqpoint{0.936976in}{0.151065in}}%
\pgfpathlineto{\pgfqpoint{0.937372in}{0.148133in}}%
\pgfpathlineto{\pgfqpoint{0.935770in}{0.146761in}}%
\pgfpathlineto{\pgfqpoint{0.936802in}{0.145789in}}%
\pgfpathlineto{\pgfqpoint{0.936825in}{0.141016in}}%
\pgfpathlineto{\pgfqpoint{0.939296in}{0.140402in}}%
\pgfpathlineto{\pgfqpoint{0.935935in}{0.137518in}}%
\pgfpathclose%
\pgfusepath{fill}%
\end{pgfscope}%
\begin{pgfscope}%
\pgfpathrectangle{\pgfqpoint{0.100000in}{0.100000in}}{\pgfqpoint{3.007045in}{1.925000in}}%
\pgfusepath{clip}%
\pgfsetbuttcap%
\pgfsetmiterjoin%
\definecolor{currentfill}{rgb}{0.321246,0.615179,0.800830}%
\pgfsetfillcolor{currentfill}%
\pgfsetlinewidth{0.000000pt}%
\definecolor{currentstroke}{rgb}{0.000000,0.000000,0.000000}%
\pgfsetstrokecolor{currentstroke}%
\pgfsetstrokeopacity{0.000000}%
\pgfsetdash{}{0pt}%
\pgfpathmoveto{\pgfqpoint{0.957491in}{0.142127in}}%
\pgfpathlineto{\pgfqpoint{0.957670in}{0.139655in}}%
\pgfpathlineto{\pgfqpoint{0.956604in}{0.137353in}}%
\pgfpathlineto{\pgfqpoint{0.955422in}{0.137331in}}%
\pgfpathlineto{\pgfqpoint{0.954352in}{0.139936in}}%
\pgfpathlineto{\pgfqpoint{0.955529in}{0.140546in}}%
\pgfpathlineto{\pgfqpoint{0.955668in}{0.143108in}}%
\pgfpathlineto{\pgfqpoint{0.957491in}{0.142127in}}%
\pgfpathclose%
\pgfusepath{fill}%
\end{pgfscope}%
\begin{pgfscope}%
\pgfpathrectangle{\pgfqpoint{0.100000in}{0.100000in}}{\pgfqpoint{3.007045in}{1.925000in}}%
\pgfusepath{clip}%
\pgfsetbuttcap%
\pgfsetmiterjoin%
\definecolor{currentfill}{rgb}{0.321246,0.615179,0.800830}%
\pgfsetfillcolor{currentfill}%
\pgfsetlinewidth{0.000000pt}%
\definecolor{currentstroke}{rgb}{0.000000,0.000000,0.000000}%
\pgfsetstrokecolor{currentstroke}%
\pgfsetstrokeopacity{0.000000}%
\pgfsetdash{}{0pt}%
\pgfpathmoveto{\pgfqpoint{0.952217in}{0.157810in}}%
\pgfpathlineto{\pgfqpoint{0.953820in}{0.154455in}}%
\pgfpathlineto{\pgfqpoint{0.955416in}{0.154694in}}%
\pgfpathlineto{\pgfqpoint{0.956678in}{0.156700in}}%
\pgfpathlineto{\pgfqpoint{0.956199in}{0.158645in}}%
\pgfpathlineto{\pgfqpoint{0.958436in}{0.159148in}}%
\pgfpathlineto{\pgfqpoint{0.958681in}{0.160262in}}%
\pgfpathlineto{\pgfqpoint{0.960550in}{0.161270in}}%
\pgfpathlineto{\pgfqpoint{0.962912in}{0.161570in}}%
\pgfpathlineto{\pgfqpoint{0.964181in}{0.158832in}}%
\pgfpathlineto{\pgfqpoint{0.966014in}{0.159668in}}%
\pgfpathlineto{\pgfqpoint{0.965820in}{0.161252in}}%
\pgfpathlineto{\pgfqpoint{0.968598in}{0.161733in}}%
\pgfpathlineto{\pgfqpoint{0.970864in}{0.160909in}}%
\pgfpathlineto{\pgfqpoint{0.972584in}{0.162951in}}%
\pgfpathlineto{\pgfqpoint{0.972690in}{0.165137in}}%
\pgfpathlineto{\pgfqpoint{0.975165in}{0.163335in}}%
\pgfpathlineto{\pgfqpoint{0.977660in}{0.159042in}}%
\pgfpathlineto{\pgfqpoint{0.978047in}{0.156183in}}%
\pgfpathlineto{\pgfqpoint{0.979828in}{0.154295in}}%
\pgfpathlineto{\pgfqpoint{0.981654in}{0.154152in}}%
\pgfpathlineto{\pgfqpoint{0.981934in}{0.151716in}}%
\pgfpathlineto{\pgfqpoint{0.981007in}{0.149518in}}%
\pgfpathlineto{\pgfqpoint{0.977742in}{0.146985in}}%
\pgfpathlineto{\pgfqpoint{0.977518in}{0.144790in}}%
\pgfpathlineto{\pgfqpoint{0.976466in}{0.142679in}}%
\pgfpathlineto{\pgfqpoint{0.976184in}{0.138931in}}%
\pgfpathlineto{\pgfqpoint{0.975258in}{0.135551in}}%
\pgfpathlineto{\pgfqpoint{0.973176in}{0.134289in}}%
\pgfpathlineto{\pgfqpoint{0.967906in}{0.129628in}}%
\pgfpathlineto{\pgfqpoint{0.965959in}{0.130065in}}%
\pgfpathlineto{\pgfqpoint{0.963807in}{0.128960in}}%
\pgfpathlineto{\pgfqpoint{0.961890in}{0.130440in}}%
\pgfpathlineto{\pgfqpoint{0.959954in}{0.130233in}}%
\pgfpathlineto{\pgfqpoint{0.961401in}{0.136798in}}%
\pgfpathlineto{\pgfqpoint{0.961289in}{0.138714in}}%
\pgfpathlineto{\pgfqpoint{0.964237in}{0.140766in}}%
\pgfpathlineto{\pgfqpoint{0.965836in}{0.140351in}}%
\pgfpathlineto{\pgfqpoint{0.965862in}{0.142940in}}%
\pgfpathlineto{\pgfqpoint{0.967505in}{0.145942in}}%
\pgfpathlineto{\pgfqpoint{0.969065in}{0.151851in}}%
\pgfpathlineto{\pgfqpoint{0.966959in}{0.151516in}}%
\pgfpathlineto{\pgfqpoint{0.967397in}{0.149343in}}%
\pgfpathlineto{\pgfqpoint{0.966518in}{0.146372in}}%
\pgfpathlineto{\pgfqpoint{0.962740in}{0.141294in}}%
\pgfpathlineto{\pgfqpoint{0.959748in}{0.140140in}}%
\pgfpathlineto{\pgfqpoint{0.959257in}{0.141993in}}%
\pgfpathlineto{\pgfqpoint{0.956453in}{0.144467in}}%
\pgfpathlineto{\pgfqpoint{0.953985in}{0.143893in}}%
\pgfpathlineto{\pgfqpoint{0.952414in}{0.142851in}}%
\pgfpathlineto{\pgfqpoint{0.953506in}{0.147090in}}%
\pgfpathlineto{\pgfqpoint{0.955124in}{0.149168in}}%
\pgfpathlineto{\pgfqpoint{0.956748in}{0.149574in}}%
\pgfpathlineto{\pgfqpoint{0.957816in}{0.151764in}}%
\pgfpathlineto{\pgfqpoint{0.959671in}{0.153160in}}%
\pgfpathlineto{\pgfqpoint{0.959717in}{0.154964in}}%
\pgfpathlineto{\pgfqpoint{0.962294in}{0.155551in}}%
\pgfpathlineto{\pgfqpoint{0.962264in}{0.157330in}}%
\pgfpathlineto{\pgfqpoint{0.959964in}{0.158575in}}%
\pgfpathlineto{\pgfqpoint{0.957588in}{0.157646in}}%
\pgfpathlineto{\pgfqpoint{0.958906in}{0.156128in}}%
\pgfpathlineto{\pgfqpoint{0.956807in}{0.153413in}}%
\pgfpathlineto{\pgfqpoint{0.953556in}{0.150672in}}%
\pgfpathlineto{\pgfqpoint{0.951866in}{0.154211in}}%
\pgfpathlineto{\pgfqpoint{0.952217in}{0.157810in}}%
\pgfpathclose%
\pgfusepath{fill}%
\end{pgfscope}%
\begin{pgfscope}%
\pgfpathrectangle{\pgfqpoint{0.100000in}{0.100000in}}{\pgfqpoint{3.007045in}{1.925000in}}%
\pgfusepath{clip}%
\pgfsetbuttcap%
\pgfsetmiterjoin%
\definecolor{currentfill}{rgb}{0.321246,0.615179,0.800830}%
\pgfsetfillcolor{currentfill}%
\pgfsetlinewidth{0.000000pt}%
\definecolor{currentstroke}{rgb}{0.000000,0.000000,0.000000}%
\pgfsetstrokecolor{currentstroke}%
\pgfsetstrokeopacity{0.000000}%
\pgfsetdash{}{0pt}%
\pgfpathmoveto{\pgfqpoint{0.932602in}{0.158192in}}%
\pgfpathlineto{\pgfqpoint{0.930583in}{0.159046in}}%
\pgfpathlineto{\pgfqpoint{0.933442in}{0.160711in}}%
\pgfpathlineto{\pgfqpoint{0.934281in}{0.158960in}}%
\pgfpathlineto{\pgfqpoint{0.937131in}{0.159694in}}%
\pgfpathlineto{\pgfqpoint{0.937907in}{0.158017in}}%
\pgfpathlineto{\pgfqpoint{0.935667in}{0.157983in}}%
\pgfpathlineto{\pgfqpoint{0.934472in}{0.156058in}}%
\pgfpathlineto{\pgfqpoint{0.932593in}{0.155358in}}%
\pgfpathlineto{\pgfqpoint{0.931238in}{0.156086in}}%
\pgfpathlineto{\pgfqpoint{0.932602in}{0.158192in}}%
\pgfpathclose%
\pgfusepath{fill}%
\end{pgfscope}%
\begin{pgfscope}%
\pgfpathrectangle{\pgfqpoint{0.100000in}{0.100000in}}{\pgfqpoint{3.007045in}{1.925000in}}%
\pgfusepath{clip}%
\pgfsetbuttcap%
\pgfsetmiterjoin%
\definecolor{currentfill}{rgb}{0.321246,0.615179,0.800830}%
\pgfsetfillcolor{currentfill}%
\pgfsetlinewidth{0.000000pt}%
\definecolor{currentstroke}{rgb}{0.000000,0.000000,0.000000}%
\pgfsetstrokecolor{currentstroke}%
\pgfsetstrokeopacity{0.000000}%
\pgfsetdash{}{0pt}%
\pgfpathmoveto{\pgfqpoint{0.940503in}{0.148050in}}%
\pgfpathlineto{\pgfqpoint{0.940580in}{0.150009in}}%
\pgfpathlineto{\pgfqpoint{0.936401in}{0.153144in}}%
\pgfpathlineto{\pgfqpoint{0.939865in}{0.154873in}}%
\pgfpathlineto{\pgfqpoint{0.939188in}{0.156547in}}%
\pgfpathlineto{\pgfqpoint{0.940021in}{0.158363in}}%
\pgfpathlineto{\pgfqpoint{0.938477in}{0.159030in}}%
\pgfpathlineto{\pgfqpoint{0.937509in}{0.161120in}}%
\pgfpathlineto{\pgfqpoint{0.940507in}{0.163189in}}%
\pgfpathlineto{\pgfqpoint{0.940853in}{0.165668in}}%
\pgfpathlineto{\pgfqpoint{0.942540in}{0.166461in}}%
\pgfpathlineto{\pgfqpoint{0.942490in}{0.168936in}}%
\pgfpathlineto{\pgfqpoint{0.940371in}{0.169760in}}%
\pgfpathlineto{\pgfqpoint{0.935534in}{0.169087in}}%
\pgfpathlineto{\pgfqpoint{0.937048in}{0.170932in}}%
\pgfpathlineto{\pgfqpoint{0.938908in}{0.171837in}}%
\pgfpathlineto{\pgfqpoint{0.941019in}{0.171940in}}%
\pgfpathlineto{\pgfqpoint{0.940657in}{0.175170in}}%
\pgfpathlineto{\pgfqpoint{0.941701in}{0.177230in}}%
\pgfpathlineto{\pgfqpoint{0.943306in}{0.176542in}}%
\pgfpathlineto{\pgfqpoint{0.946910in}{0.174004in}}%
\pgfpathlineto{\pgfqpoint{0.946946in}{0.170521in}}%
\pgfpathlineto{\pgfqpoint{0.945241in}{0.169179in}}%
\pgfpathlineto{\pgfqpoint{0.945972in}{0.166130in}}%
\pgfpathlineto{\pgfqpoint{0.947228in}{0.166360in}}%
\pgfpathlineto{\pgfqpoint{0.948782in}{0.164350in}}%
\pgfpathlineto{\pgfqpoint{0.949856in}{0.160592in}}%
\pgfpathlineto{\pgfqpoint{0.949616in}{0.158564in}}%
\pgfpathlineto{\pgfqpoint{0.948423in}{0.157492in}}%
\pgfpathlineto{\pgfqpoint{0.950053in}{0.156174in}}%
\pgfpathlineto{\pgfqpoint{0.949902in}{0.154095in}}%
\pgfpathlineto{\pgfqpoint{0.948375in}{0.153768in}}%
\pgfpathlineto{\pgfqpoint{0.947443in}{0.155763in}}%
\pgfpathlineto{\pgfqpoint{0.946239in}{0.153864in}}%
\pgfpathlineto{\pgfqpoint{0.948621in}{0.152037in}}%
\pgfpathlineto{\pgfqpoint{0.948626in}{0.150372in}}%
\pgfpathlineto{\pgfqpoint{0.949742in}{0.147800in}}%
\pgfpathlineto{\pgfqpoint{0.948428in}{0.144747in}}%
\pgfpathlineto{\pgfqpoint{0.950997in}{0.145602in}}%
\pgfpathlineto{\pgfqpoint{0.950534in}{0.143772in}}%
\pgfpathlineto{\pgfqpoint{0.948944in}{0.142658in}}%
\pgfpathlineto{\pgfqpoint{0.947671in}{0.143289in}}%
\pgfpathlineto{\pgfqpoint{0.946770in}{0.140379in}}%
\pgfpathlineto{\pgfqpoint{0.949016in}{0.140644in}}%
\pgfpathlineto{\pgfqpoint{0.947609in}{0.137730in}}%
\pgfpathlineto{\pgfqpoint{0.947150in}{0.135497in}}%
\pgfpathlineto{\pgfqpoint{0.945969in}{0.134095in}}%
\pgfpathlineto{\pgfqpoint{0.943530in}{0.134941in}}%
\pgfpathlineto{\pgfqpoint{0.943106in}{0.139361in}}%
\pgfpathlineto{\pgfqpoint{0.941552in}{0.142463in}}%
\pgfpathlineto{\pgfqpoint{0.941399in}{0.144418in}}%
\pgfpathlineto{\pgfqpoint{0.942947in}{0.147821in}}%
\pgfpathlineto{\pgfqpoint{0.940503in}{0.148050in}}%
\pgfpathclose%
\pgfusepath{fill}%
\end{pgfscope}%
\begin{pgfscope}%
\pgfpathrectangle{\pgfqpoint{0.100000in}{0.100000in}}{\pgfqpoint{3.007045in}{1.925000in}}%
\pgfusepath{clip}%
\pgfsetbuttcap%
\pgfsetmiterjoin%
\definecolor{currentfill}{rgb}{0.248166,0.561892,0.770980}%
\pgfsetfillcolor{currentfill}%
\pgfsetlinewidth{0.000000pt}%
\definecolor{currentstroke}{rgb}{0.000000,0.000000,0.000000}%
\pgfsetstrokecolor{currentstroke}%
\pgfsetstrokeopacity{0.000000}%
\pgfsetdash{}{0pt}%
\pgfpathmoveto{\pgfqpoint{0.746029in}{1.809361in}}%
\pgfpathlineto{\pgfqpoint{0.740338in}{1.787082in}}%
\pgfpathlineto{\pgfqpoint{0.735917in}{1.779747in}}%
\pgfpathlineto{\pgfqpoint{0.726344in}{1.778628in}}%
\pgfpathlineto{\pgfqpoint{0.702590in}{1.784664in}}%
\pgfpathlineto{\pgfqpoint{0.675279in}{1.792196in}}%
\pgfpathlineto{\pgfqpoint{0.670581in}{1.789487in}}%
\pgfpathlineto{\pgfqpoint{0.668689in}{1.793331in}}%
\pgfpathlineto{\pgfqpoint{0.662371in}{1.789397in}}%
\pgfpathlineto{\pgfqpoint{0.651182in}{1.791391in}}%
\pgfpathlineto{\pgfqpoint{0.648904in}{1.795115in}}%
\pgfpathlineto{\pgfqpoint{0.651860in}{1.802864in}}%
\pgfpathlineto{\pgfqpoint{0.652827in}{1.812664in}}%
\pgfpathlineto{\pgfqpoint{0.652061in}{1.822418in}}%
\pgfpathlineto{\pgfqpoint{0.656028in}{1.830896in}}%
\pgfpathlineto{\pgfqpoint{0.657936in}{1.837275in}}%
\pgfpathlineto{\pgfqpoint{0.662469in}{1.839071in}}%
\pgfpathlineto{\pgfqpoint{0.665649in}{1.843218in}}%
\pgfpathlineto{\pgfqpoint{0.680569in}{1.839333in}}%
\pgfpathlineto{\pgfqpoint{0.682133in}{1.844866in}}%
\pgfpathlineto{\pgfqpoint{0.689171in}{1.848788in}}%
\pgfpathlineto{\pgfqpoint{0.691919in}{1.848024in}}%
\pgfpathlineto{\pgfqpoint{0.695822in}{1.854965in}}%
\pgfpathlineto{\pgfqpoint{0.700822in}{1.858600in}}%
\pgfpathlineto{\pgfqpoint{0.702364in}{1.864190in}}%
\pgfpathlineto{\pgfqpoint{0.708324in}{1.867692in}}%
\pgfpathlineto{\pgfqpoint{0.713573in}{1.866288in}}%
\pgfpathlineto{\pgfqpoint{0.713636in}{1.864996in}}%
\pgfpathlineto{\pgfqpoint{0.701521in}{1.821189in}}%
\pgfpathlineto{\pgfqpoint{0.746029in}{1.809361in}}%
\pgfpathclose%
\pgfusepath{fill}%
\end{pgfscope}%
\begin{pgfscope}%
\pgfpathrectangle{\pgfqpoint{0.100000in}{0.100000in}}{\pgfqpoint{3.007045in}{1.925000in}}%
\pgfusepath{clip}%
\pgfsetbuttcap%
\pgfsetmiterjoin%
\definecolor{currentfill}{rgb}{0.632526,0.797647,0.886874}%
\pgfsetfillcolor{currentfill}%
\pgfsetlinewidth{0.000000pt}%
\definecolor{currentstroke}{rgb}{0.000000,0.000000,0.000000}%
\pgfsetstrokecolor{currentstroke}%
\pgfsetstrokeopacity{0.000000}%
\pgfsetdash{}{0pt}%
\pgfpathmoveto{\pgfqpoint{2.488733in}{1.208153in}}%
\pgfpathlineto{\pgfqpoint{2.485346in}{1.213775in}}%
\pgfpathlineto{\pgfqpoint{2.476467in}{1.219730in}}%
\pgfpathlineto{\pgfqpoint{2.478476in}{1.225076in}}%
\pgfpathlineto{\pgfqpoint{2.467867in}{1.223907in}}%
\pgfpathlineto{\pgfqpoint{2.465782in}{1.225372in}}%
\pgfpathlineto{\pgfqpoint{2.462970in}{1.231712in}}%
\pgfpathlineto{\pgfqpoint{2.461646in}{1.242312in}}%
\pgfpathlineto{\pgfqpoint{2.465474in}{1.242785in}}%
\pgfpathlineto{\pgfqpoint{2.463037in}{1.262554in}}%
\pgfpathlineto{\pgfqpoint{2.480209in}{1.264505in}}%
\pgfpathlineto{\pgfqpoint{2.498296in}{1.267386in}}%
\pgfpathlineto{\pgfqpoint{2.502865in}{1.238879in}}%
\pgfpathlineto{\pgfqpoint{2.507767in}{1.239655in}}%
\pgfpathlineto{\pgfqpoint{2.510163in}{1.234931in}}%
\pgfpathlineto{\pgfqpoint{2.505584in}{1.228741in}}%
\pgfpathlineto{\pgfqpoint{2.507751in}{1.225110in}}%
\pgfpathlineto{\pgfqpoint{2.505896in}{1.220258in}}%
\pgfpathlineto{\pgfqpoint{2.500713in}{1.220454in}}%
\pgfpathlineto{\pgfqpoint{2.488733in}{1.208153in}}%
\pgfpathclose%
\pgfusepath{fill}%
\end{pgfscope}%
\begin{pgfscope}%
\pgfpathrectangle{\pgfqpoint{0.100000in}{0.100000in}}{\pgfqpoint{3.007045in}{1.925000in}}%
\pgfusepath{clip}%
\pgfsetbuttcap%
\pgfsetmiterjoin%
\definecolor{currentfill}{rgb}{0.637447,0.799739,0.888597}%
\pgfsetfillcolor{currentfill}%
\pgfsetlinewidth{0.000000pt}%
\definecolor{currentstroke}{rgb}{0.000000,0.000000,0.000000}%
\pgfsetstrokecolor{currentstroke}%
\pgfsetstrokeopacity{0.000000}%
\pgfsetdash{}{0pt}%
\pgfpathmoveto{\pgfqpoint{2.905539in}{1.431014in}}%
\pgfpathlineto{\pgfqpoint{2.906901in}{1.437874in}}%
\pgfpathlineto{\pgfqpoint{2.903419in}{1.453280in}}%
\pgfpathlineto{\pgfqpoint{2.896973in}{1.475592in}}%
\pgfpathlineto{\pgfqpoint{2.916591in}{1.481492in}}%
\pgfpathlineto{\pgfqpoint{2.917977in}{1.479575in}}%
\pgfpathlineto{\pgfqpoint{2.929528in}{1.490206in}}%
\pgfpathlineto{\pgfqpoint{2.933228in}{1.482168in}}%
\pgfpathlineto{\pgfqpoint{2.944479in}{1.471321in}}%
\pgfpathlineto{\pgfqpoint{2.948844in}{1.463490in}}%
\pgfpathlineto{\pgfqpoint{2.944989in}{1.461097in}}%
\pgfpathlineto{\pgfqpoint{2.945676in}{1.456613in}}%
\pgfpathlineto{\pgfqpoint{2.941792in}{1.452515in}}%
\pgfpathlineto{\pgfqpoint{2.928813in}{1.447884in}}%
\pgfpathlineto{\pgfqpoint{2.928414in}{1.457369in}}%
\pgfpathlineto{\pgfqpoint{2.925276in}{1.463337in}}%
\pgfpathlineto{\pgfqpoint{2.921321in}{1.458455in}}%
\pgfpathlineto{\pgfqpoint{2.921311in}{1.451942in}}%
\pgfpathlineto{\pgfqpoint{2.924101in}{1.446233in}}%
\pgfpathlineto{\pgfqpoint{2.922682in}{1.438922in}}%
\pgfpathlineto{\pgfqpoint{2.919165in}{1.438139in}}%
\pgfpathlineto{\pgfqpoint{2.905539in}{1.431014in}}%
\pgfpathclose%
\pgfusepath{fill}%
\end{pgfscope}%
\begin{pgfscope}%
\pgfpathrectangle{\pgfqpoint{0.100000in}{0.100000in}}{\pgfqpoint{3.007045in}{1.925000in}}%
\pgfusepath{clip}%
\pgfsetbuttcap%
\pgfsetmiterjoin%
\definecolor{currentfill}{rgb}{0.244106,0.557832,0.768889}%
\pgfsetfillcolor{currentfill}%
\pgfsetlinewidth{0.000000pt}%
\definecolor{currentstroke}{rgb}{0.000000,0.000000,0.000000}%
\pgfsetstrokecolor{currentstroke}%
\pgfsetstrokeopacity{0.000000}%
\pgfsetdash{}{0pt}%
\pgfpathmoveto{\pgfqpoint{1.806790in}{1.547902in}}%
\pgfpathlineto{\pgfqpoint{1.801048in}{1.550695in}}%
\pgfpathlineto{\pgfqpoint{1.783351in}{1.550568in}}%
\pgfpathlineto{\pgfqpoint{1.783347in}{1.556314in}}%
\pgfpathlineto{\pgfqpoint{1.766199in}{1.556278in}}%
\pgfpathlineto{\pgfqpoint{1.765864in}{1.580217in}}%
\pgfpathlineto{\pgfqpoint{1.806019in}{1.580465in}}%
\pgfpathlineto{\pgfqpoint{1.802241in}{1.583541in}}%
\pgfpathlineto{\pgfqpoint{1.829469in}{1.583832in}}%
\pgfpathlineto{\pgfqpoint{1.829097in}{1.594472in}}%
\pgfpathlineto{\pgfqpoint{1.827186in}{1.594478in}}%
\pgfpathlineto{\pgfqpoint{1.827033in}{1.605997in}}%
\pgfpathlineto{\pgfqpoint{1.826933in}{1.611742in}}%
\pgfpathlineto{\pgfqpoint{1.844395in}{1.612207in}}%
\pgfpathlineto{\pgfqpoint{1.844441in}{1.606089in}}%
\pgfpathlineto{\pgfqpoint{1.844608in}{1.594634in}}%
\pgfpathlineto{\pgfqpoint{1.840662in}{1.594594in}}%
\pgfpathlineto{\pgfqpoint{1.841127in}{1.578186in}}%
\pgfpathlineto{\pgfqpoint{1.841735in}{1.545791in}}%
\pgfpathlineto{\pgfqpoint{1.838457in}{1.548803in}}%
\pgfpathlineto{\pgfqpoint{1.830410in}{1.548866in}}%
\pgfpathlineto{\pgfqpoint{1.816671in}{1.557157in}}%
\pgfpathlineto{\pgfqpoint{1.813546in}{1.550942in}}%
\pgfpathlineto{\pgfqpoint{1.806790in}{1.547902in}}%
\pgfpathclose%
\pgfusepath{fill}%
\end{pgfscope}%
\begin{pgfscope}%
\pgfpathrectangle{\pgfqpoint{0.100000in}{0.100000in}}{\pgfqpoint{3.007045in}{1.925000in}}%
\pgfusepath{clip}%
\pgfsetbuttcap%
\pgfsetmiterjoin%
\definecolor{currentfill}{rgb}{0.114802,0.424437,0.695194}%
\pgfsetfillcolor{currentfill}%
\pgfsetlinewidth{0.000000pt}%
\definecolor{currentstroke}{rgb}{0.000000,0.000000,0.000000}%
\pgfsetstrokecolor{currentstroke}%
\pgfsetstrokeopacity{0.000000}%
\pgfsetdash{}{0pt}%
\pgfpathmoveto{\pgfqpoint{0.967728in}{1.403968in}}%
\pgfpathlineto{\pgfqpoint{0.967940in}{1.408953in}}%
\pgfpathlineto{\pgfqpoint{0.964652in}{1.410841in}}%
\pgfpathlineto{\pgfqpoint{0.964506in}{1.418586in}}%
\pgfpathlineto{\pgfqpoint{0.968279in}{1.422237in}}%
\pgfpathlineto{\pgfqpoint{0.969777in}{1.427234in}}%
\pgfpathlineto{\pgfqpoint{0.968314in}{1.431784in}}%
\pgfpathlineto{\pgfqpoint{0.955057in}{1.434288in}}%
\pgfpathlineto{\pgfqpoint{0.955530in}{1.447686in}}%
\pgfpathlineto{\pgfqpoint{0.954259in}{1.454699in}}%
\pgfpathlineto{\pgfqpoint{0.947546in}{1.454706in}}%
\pgfpathlineto{\pgfqpoint{0.946788in}{1.460996in}}%
\pgfpathlineto{\pgfqpoint{0.947898in}{1.467983in}}%
\pgfpathlineto{\pgfqpoint{0.953549in}{1.475280in}}%
\pgfpathlineto{\pgfqpoint{1.001890in}{1.466220in}}%
\pgfpathlineto{\pgfqpoint{0.989963in}{1.400079in}}%
\pgfpathlineto{\pgfqpoint{0.967728in}{1.403968in}}%
\pgfpathclose%
\pgfusepath{fill}%
\end{pgfscope}%
\begin{pgfscope}%
\pgfpathrectangle{\pgfqpoint{0.100000in}{0.100000in}}{\pgfqpoint{3.007045in}{1.925000in}}%
\pgfusepath{clip}%
\pgfsetbuttcap%
\pgfsetmiterjoin%
\definecolor{currentfill}{rgb}{0.248166,0.561892,0.770980}%
\pgfsetfillcolor{currentfill}%
\pgfsetlinewidth{0.000000pt}%
\definecolor{currentstroke}{rgb}{0.000000,0.000000,0.000000}%
\pgfsetstrokecolor{currentstroke}%
\pgfsetstrokeopacity{0.000000}%
\pgfsetdash{}{0pt}%
\pgfpathmoveto{\pgfqpoint{1.844374in}{1.020395in}}%
\pgfpathlineto{\pgfqpoint{1.822007in}{1.020648in}}%
\pgfpathlineto{\pgfqpoint{1.820409in}{1.024532in}}%
\pgfpathlineto{\pgfqpoint{1.792358in}{1.025328in}}%
\pgfpathlineto{\pgfqpoint{1.792356in}{1.069644in}}%
\pgfpathlineto{\pgfqpoint{1.792345in}{1.071169in}}%
\pgfpathlineto{\pgfqpoint{1.811752in}{1.070677in}}%
\pgfpathlineto{\pgfqpoint{1.821187in}{1.069825in}}%
\pgfpathlineto{\pgfqpoint{1.820896in}{1.060894in}}%
\pgfpathlineto{\pgfqpoint{1.834274in}{1.060508in}}%
\pgfpathlineto{\pgfqpoint{1.834159in}{1.056680in}}%
\pgfpathlineto{\pgfqpoint{1.843696in}{1.056469in}}%
\pgfpathlineto{\pgfqpoint{1.844766in}{1.050707in}}%
\pgfpathlineto{\pgfqpoint{1.844374in}{1.020395in}}%
\pgfpathclose%
\pgfusepath{fill}%
\end{pgfscope}%
\begin{pgfscope}%
\pgfpathrectangle{\pgfqpoint{0.100000in}{0.100000in}}{\pgfqpoint{3.007045in}{1.925000in}}%
\pgfusepath{clip}%
\pgfsetbuttcap%
\pgfsetmiterjoin%
\definecolor{currentfill}{rgb}{0.460392,0.704744,0.848012}%
\pgfsetfillcolor{currentfill}%
\pgfsetlinewidth{0.000000pt}%
\definecolor{currentstroke}{rgb}{0.000000,0.000000,0.000000}%
\pgfsetstrokecolor{currentstroke}%
\pgfsetstrokeopacity{0.000000}%
\pgfsetdash{}{0pt}%
\pgfpathmoveto{\pgfqpoint{2.597898in}{0.973667in}}%
\pgfpathlineto{\pgfqpoint{2.564813in}{0.967383in}}%
\pgfpathlineto{\pgfqpoint{2.562648in}{0.972115in}}%
\pgfpathlineto{\pgfqpoint{2.555149in}{0.977464in}}%
\pgfpathlineto{\pgfqpoint{2.547659in}{0.980532in}}%
\pgfpathlineto{\pgfqpoint{2.545915in}{0.985956in}}%
\pgfpathlineto{\pgfqpoint{2.549802in}{0.987470in}}%
\pgfpathlineto{\pgfqpoint{2.549576in}{0.995557in}}%
\pgfpathlineto{\pgfqpoint{2.557574in}{1.001121in}}%
\pgfpathlineto{\pgfqpoint{2.566736in}{1.001545in}}%
\pgfpathlineto{\pgfqpoint{2.561819in}{1.036295in}}%
\pgfpathlineto{\pgfqpoint{2.578144in}{1.039001in}}%
\pgfpathlineto{\pgfqpoint{2.578534in}{1.060078in}}%
\pgfpathlineto{\pgfqpoint{2.578887in}{1.072582in}}%
\pgfpathlineto{\pgfqpoint{2.584592in}{1.075050in}}%
\pgfpathlineto{\pgfqpoint{2.591032in}{1.080451in}}%
\pgfpathlineto{\pgfqpoint{2.596091in}{1.081050in}}%
\pgfpathlineto{\pgfqpoint{2.599528in}{1.076992in}}%
\pgfpathlineto{\pgfqpoint{2.604673in}{1.078358in}}%
\pgfpathlineto{\pgfqpoint{2.604048in}{1.043452in}}%
\pgfpathlineto{\pgfqpoint{2.608257in}{1.044199in}}%
\pgfpathlineto{\pgfqpoint{2.610860in}{1.024597in}}%
\pgfpathlineto{\pgfqpoint{2.605433in}{1.023762in}}%
\pgfpathlineto{\pgfqpoint{2.608855in}{1.001837in}}%
\pgfpathlineto{\pgfqpoint{2.611008in}{0.997970in}}%
\pgfpathlineto{\pgfqpoint{2.594940in}{0.995133in}}%
\pgfpathlineto{\pgfqpoint{2.597898in}{0.973667in}}%
\pgfpathclose%
\pgfusepath{fill}%
\end{pgfscope}%
\begin{pgfscope}%
\pgfpathrectangle{\pgfqpoint{0.100000in}{0.100000in}}{\pgfqpoint{3.007045in}{1.925000in}}%
\pgfusepath{clip}%
\pgfsetbuttcap%
\pgfsetmiterjoin%
\definecolor{currentfill}{rgb}{0.485490,0.718524,0.853426}%
\pgfsetfillcolor{currentfill}%
\pgfsetlinewidth{0.000000pt}%
\definecolor{currentstroke}{rgb}{0.000000,0.000000,0.000000}%
\pgfsetstrokecolor{currentstroke}%
\pgfsetstrokeopacity{0.000000}%
\pgfsetdash{}{0pt}%
\pgfpathmoveto{\pgfqpoint{1.166919in}{0.389823in}}%
\pgfpathlineto{\pgfqpoint{1.169914in}{0.398030in}}%
\pgfpathlineto{\pgfqpoint{1.170066in}{0.403152in}}%
\pgfpathlineto{\pgfqpoint{1.177565in}{0.397908in}}%
\pgfpathlineto{\pgfqpoint{1.179080in}{0.390976in}}%
\pgfpathlineto{\pgfqpoint{1.174659in}{0.386184in}}%
\pgfpathlineto{\pgfqpoint{1.166919in}{0.389823in}}%
\pgfpathclose%
\pgfusepath{fill}%
\end{pgfscope}%
\begin{pgfscope}%
\pgfpathrectangle{\pgfqpoint{0.100000in}{0.100000in}}{\pgfqpoint{3.007045in}{1.925000in}}%
\pgfusepath{clip}%
\pgfsetbuttcap%
\pgfsetmiterjoin%
\definecolor{currentfill}{rgb}{0.485490,0.718524,0.853426}%
\pgfsetfillcolor{currentfill}%
\pgfsetlinewidth{0.000000pt}%
\definecolor{currentstroke}{rgb}{0.000000,0.000000,0.000000}%
\pgfsetstrokecolor{currentstroke}%
\pgfsetstrokeopacity{0.000000}%
\pgfsetdash{}{0pt}%
\pgfpathmoveto{\pgfqpoint{1.171506in}{0.365544in}}%
\pgfpathlineto{\pgfqpoint{1.172099in}{0.368432in}}%
\pgfpathlineto{\pgfqpoint{1.181292in}{0.366754in}}%
\pgfpathlineto{\pgfqpoint{1.179030in}{0.362563in}}%
\pgfpathlineto{\pgfqpoint{1.171506in}{0.365544in}}%
\pgfpathclose%
\pgfusepath{fill}%
\end{pgfscope}%
\begin{pgfscope}%
\pgfpathrectangle{\pgfqpoint{0.100000in}{0.100000in}}{\pgfqpoint{3.007045in}{1.925000in}}%
\pgfusepath{clip}%
\pgfsetbuttcap%
\pgfsetmiterjoin%
\definecolor{currentfill}{rgb}{0.485490,0.718524,0.853426}%
\pgfsetfillcolor{currentfill}%
\pgfsetlinewidth{0.000000pt}%
\definecolor{currentstroke}{rgb}{0.000000,0.000000,0.000000}%
\pgfsetstrokecolor{currentstroke}%
\pgfsetstrokeopacity{0.000000}%
\pgfsetdash{}{0pt}%
\pgfpathmoveto{\pgfqpoint{1.189967in}{0.357288in}}%
\pgfpathlineto{\pgfqpoint{1.187711in}{0.361758in}}%
\pgfpathlineto{\pgfqpoint{1.194261in}{0.372053in}}%
\pgfpathlineto{\pgfqpoint{1.187347in}{0.380076in}}%
\pgfpathlineto{\pgfqpoint{1.187339in}{0.386651in}}%
\pgfpathlineto{\pgfqpoint{1.193857in}{0.392205in}}%
\pgfpathlineto{\pgfqpoint{1.198739in}{0.390671in}}%
\pgfpathlineto{\pgfqpoint{1.199911in}{0.385707in}}%
\pgfpathlineto{\pgfqpoint{1.198482in}{0.379133in}}%
\pgfpathlineto{\pgfqpoint{1.210453in}{0.373714in}}%
\pgfpathlineto{\pgfqpoint{1.213763in}{0.360359in}}%
\pgfpathlineto{\pgfqpoint{1.218074in}{0.354132in}}%
\pgfpathlineto{\pgfqpoint{1.215157in}{0.349080in}}%
\pgfpathlineto{\pgfqpoint{1.209803in}{0.348364in}}%
\pgfpathlineto{\pgfqpoint{1.200882in}{0.353223in}}%
\pgfpathlineto{\pgfqpoint{1.189967in}{0.357288in}}%
\pgfpathclose%
\pgfusepath{fill}%
\end{pgfscope}%
\begin{pgfscope}%
\pgfpathrectangle{\pgfqpoint{0.100000in}{0.100000in}}{\pgfqpoint{3.007045in}{1.925000in}}%
\pgfusepath{clip}%
\pgfsetbuttcap%
\pgfsetmiterjoin%
\definecolor{currentfill}{rgb}{0.485490,0.718524,0.853426}%
\pgfsetfillcolor{currentfill}%
\pgfsetlinewidth{0.000000pt}%
\definecolor{currentstroke}{rgb}{0.000000,0.000000,0.000000}%
\pgfsetstrokecolor{currentstroke}%
\pgfsetstrokeopacity{0.000000}%
\pgfsetdash{}{0pt}%
\pgfpathmoveto{\pgfqpoint{1.188373in}{0.411114in}}%
\pgfpathlineto{\pgfqpoint{1.197345in}{0.404388in}}%
\pgfpathlineto{\pgfqpoint{1.194069in}{0.401026in}}%
\pgfpathlineto{\pgfqpoint{1.187346in}{0.401919in}}%
\pgfpathlineto{\pgfqpoint{1.182053in}{0.407040in}}%
\pgfpathlineto{\pgfqpoint{1.178085in}{0.413644in}}%
\pgfpathlineto{\pgfqpoint{1.165729in}{0.423161in}}%
\pgfpathlineto{\pgfqpoint{1.174372in}{0.427023in}}%
\pgfpathlineto{\pgfqpoint{1.188373in}{0.411114in}}%
\pgfpathclose%
\pgfusepath{fill}%
\end{pgfscope}%
\begin{pgfscope}%
\pgfpathrectangle{\pgfqpoint{0.100000in}{0.100000in}}{\pgfqpoint{3.007045in}{1.925000in}}%
\pgfusepath{clip}%
\pgfsetbuttcap%
\pgfsetmiterjoin%
\definecolor{currentfill}{rgb}{0.280892,0.587620,0.785083}%
\pgfsetfillcolor{currentfill}%
\pgfsetlinewidth{0.000000pt}%
\definecolor{currentstroke}{rgb}{0.000000,0.000000,0.000000}%
\pgfsetstrokecolor{currentstroke}%
\pgfsetstrokeopacity{0.000000}%
\pgfsetdash{}{0pt}%
\pgfpathmoveto{\pgfqpoint{1.330767in}{0.811238in}}%
\pgfpathlineto{\pgfqpoint{1.328991in}{0.794521in}}%
\pgfpathlineto{\pgfqpoint{1.303863in}{0.796794in}}%
\pgfpathlineto{\pgfqpoint{1.304431in}{0.802517in}}%
\pgfpathlineto{\pgfqpoint{1.292993in}{0.803535in}}%
\pgfpathlineto{\pgfqpoint{1.294107in}{0.814272in}}%
\pgfpathlineto{\pgfqpoint{1.287264in}{0.814931in}}%
\pgfpathlineto{\pgfqpoint{1.288966in}{0.832202in}}%
\pgfpathlineto{\pgfqpoint{1.283287in}{0.832832in}}%
\pgfpathlineto{\pgfqpoint{1.286524in}{0.867195in}}%
\pgfpathlineto{\pgfqpoint{1.299641in}{0.865888in}}%
\pgfpathlineto{\pgfqpoint{1.300241in}{0.871662in}}%
\pgfpathlineto{\pgfqpoint{1.311575in}{0.870525in}}%
\pgfpathlineto{\pgfqpoint{1.312137in}{0.876255in}}%
\pgfpathlineto{\pgfqpoint{1.317814in}{0.875699in}}%
\pgfpathlineto{\pgfqpoint{1.318390in}{0.881414in}}%
\pgfpathlineto{\pgfqpoint{1.324074in}{0.880866in}}%
\pgfpathlineto{\pgfqpoint{1.324670in}{0.886744in}}%
\pgfpathlineto{\pgfqpoint{1.337618in}{0.885562in}}%
\pgfpathlineto{\pgfqpoint{1.336407in}{0.871962in}}%
\pgfpathlineto{\pgfqpoint{1.364391in}{0.869531in}}%
\pgfpathlineto{\pgfqpoint{1.361996in}{0.840962in}}%
\pgfpathlineto{\pgfqpoint{1.333803in}{0.843365in}}%
\pgfpathlineto{\pgfqpoint{1.330767in}{0.811238in}}%
\pgfpathclose%
\pgfusepath{fill}%
\end{pgfscope}%
\begin{pgfscope}%
\pgfpathrectangle{\pgfqpoint{0.100000in}{0.100000in}}{\pgfqpoint{3.007045in}{1.925000in}}%
\pgfusepath{clip}%
\pgfsetbuttcap%
\pgfsetmiterjoin%
\definecolor{currentfill}{rgb}{0.356555,0.639293,0.814610}%
\pgfsetfillcolor{currentfill}%
\pgfsetlinewidth{0.000000pt}%
\definecolor{currentstroke}{rgb}{0.000000,0.000000,0.000000}%
\pgfsetstrokecolor{currentstroke}%
\pgfsetstrokeopacity{0.000000}%
\pgfsetdash{}{0pt}%
\pgfpathmoveto{\pgfqpoint{2.590755in}{0.837224in}}%
\pgfpathlineto{\pgfqpoint{2.584761in}{0.832670in}}%
\pgfpathlineto{\pgfqpoint{2.577826in}{0.830633in}}%
\pgfpathlineto{\pgfqpoint{2.571176in}{0.834427in}}%
\pgfpathlineto{\pgfqpoint{2.566054in}{0.842571in}}%
\pgfpathlineto{\pgfqpoint{2.559661in}{0.847879in}}%
\pgfpathlineto{\pgfqpoint{2.560497in}{0.852920in}}%
\pgfpathlineto{\pgfqpoint{2.558099in}{0.855702in}}%
\pgfpathlineto{\pgfqpoint{2.557919in}{0.868208in}}%
\pgfpathlineto{\pgfqpoint{2.555420in}{0.871742in}}%
\pgfpathlineto{\pgfqpoint{2.551190in}{0.869149in}}%
\pgfpathlineto{\pgfqpoint{2.541892in}{0.875274in}}%
\pgfpathlineto{\pgfqpoint{2.545276in}{0.888835in}}%
\pgfpathlineto{\pgfqpoint{2.538562in}{0.893175in}}%
\pgfpathlineto{\pgfqpoint{2.545027in}{0.899687in}}%
\pgfpathlineto{\pgfqpoint{2.550816in}{0.897393in}}%
\pgfpathlineto{\pgfqpoint{2.555983in}{0.898307in}}%
\pgfpathlineto{\pgfqpoint{2.555178in}{0.902655in}}%
\pgfpathlineto{\pgfqpoint{2.562238in}{0.907450in}}%
\pgfpathlineto{\pgfqpoint{2.567423in}{0.900852in}}%
\pgfpathlineto{\pgfqpoint{2.571314in}{0.892287in}}%
\pgfpathlineto{\pgfqpoint{2.577854in}{0.890686in}}%
\pgfpathlineto{\pgfqpoint{2.582316in}{0.884878in}}%
\pgfpathlineto{\pgfqpoint{2.580510in}{0.880092in}}%
\pgfpathlineto{\pgfqpoint{2.596705in}{0.871224in}}%
\pgfpathlineto{\pgfqpoint{2.592778in}{0.867711in}}%
\pgfpathlineto{\pgfqpoint{2.597384in}{0.859907in}}%
\pgfpathlineto{\pgfqpoint{2.593582in}{0.855868in}}%
\pgfpathlineto{\pgfqpoint{2.595239in}{0.853039in}}%
\pgfpathlineto{\pgfqpoint{2.590755in}{0.837224in}}%
\pgfpathclose%
\pgfusepath{fill}%
\end{pgfscope}%
\begin{pgfscope}%
\pgfpathrectangle{\pgfqpoint{0.100000in}{0.100000in}}{\pgfqpoint{3.007045in}{1.925000in}}%
\pgfusepath{clip}%
\pgfsetbuttcap%
\pgfsetmiterjoin%
\definecolor{currentfill}{rgb}{0.183206,0.496932,0.737516}%
\pgfsetfillcolor{currentfill}%
\pgfsetlinewidth{0.000000pt}%
\definecolor{currentstroke}{rgb}{0.000000,0.000000,0.000000}%
\pgfsetstrokecolor{currentstroke}%
\pgfsetstrokeopacity{0.000000}%
\pgfsetdash{}{0pt}%
\pgfpathmoveto{\pgfqpoint{1.318468in}{1.629024in}}%
\pgfpathlineto{\pgfqpoint{1.316018in}{1.629285in}}%
\pgfpathlineto{\pgfqpoint{1.313493in}{1.606222in}}%
\pgfpathlineto{\pgfqpoint{1.311526in}{1.606441in}}%
\pgfpathlineto{\pgfqpoint{1.255957in}{1.613050in}}%
\pgfpathlineto{\pgfqpoint{1.258817in}{1.635994in}}%
\pgfpathlineto{\pgfqpoint{1.260561in}{1.635790in}}%
\pgfpathlineto{\pgfqpoint{1.263286in}{1.658626in}}%
\pgfpathlineto{\pgfqpoint{1.264869in}{1.658439in}}%
\pgfpathlineto{\pgfqpoint{1.267668in}{1.681381in}}%
\pgfpathlineto{\pgfqpoint{1.269340in}{1.682158in}}%
\pgfpathlineto{\pgfqpoint{1.272031in}{1.704281in}}%
\pgfpathlineto{\pgfqpoint{1.277730in}{1.703596in}}%
\pgfpathlineto{\pgfqpoint{1.277023in}{1.697772in}}%
\pgfpathlineto{\pgfqpoint{1.282748in}{1.697095in}}%
\pgfpathlineto{\pgfqpoint{1.283442in}{1.702901in}}%
\pgfpathlineto{\pgfqpoint{1.302614in}{1.700688in}}%
\pgfpathlineto{\pgfqpoint{1.306273in}{1.700261in}}%
\pgfpathlineto{\pgfqpoint{1.304728in}{1.686756in}}%
\pgfpathlineto{\pgfqpoint{1.310508in}{1.686101in}}%
\pgfpathlineto{\pgfqpoint{1.310066in}{1.682269in}}%
\pgfpathlineto{\pgfqpoint{1.315821in}{1.681622in}}%
\pgfpathlineto{\pgfqpoint{1.317293in}{1.677601in}}%
\pgfpathlineto{\pgfqpoint{1.336444in}{1.675484in}}%
\pgfpathlineto{\pgfqpoint{1.335063in}{1.663866in}}%
\pgfpathlineto{\pgfqpoint{1.329154in}{1.662581in}}%
\pgfpathlineto{\pgfqpoint{1.327908in}{1.651084in}}%
\pgfpathlineto{\pgfqpoint{1.320048in}{1.651933in}}%
\pgfpathlineto{\pgfqpoint{1.318468in}{1.629024in}}%
\pgfpathclose%
\pgfusepath{fill}%
\end{pgfscope}%
\begin{pgfscope}%
\pgfpathrectangle{\pgfqpoint{0.100000in}{0.100000in}}{\pgfqpoint{3.007045in}{1.925000in}}%
\pgfusepath{clip}%
\pgfsetbuttcap%
\pgfsetmiterjoin%
\definecolor{currentfill}{rgb}{0.203506,0.517232,0.747974}%
\pgfsetfillcolor{currentfill}%
\pgfsetlinewidth{0.000000pt}%
\definecolor{currentstroke}{rgb}{0.000000,0.000000,0.000000}%
\pgfsetstrokecolor{currentstroke}%
\pgfsetstrokeopacity{0.000000}%
\pgfsetdash{}{0pt}%
\pgfpathmoveto{\pgfqpoint{1.694633in}{0.447645in}}%
\pgfpathlineto{\pgfqpoint{1.695745in}{0.469116in}}%
\pgfpathlineto{\pgfqpoint{1.676720in}{0.488443in}}%
\pgfpathlineto{\pgfqpoint{1.693878in}{0.498881in}}%
\pgfpathlineto{\pgfqpoint{1.696363in}{0.505939in}}%
\pgfpathlineto{\pgfqpoint{1.704139in}{0.513794in}}%
\pgfpathlineto{\pgfqpoint{1.710725in}{0.510845in}}%
\pgfpathlineto{\pgfqpoint{1.710733in}{0.503026in}}%
\pgfpathlineto{\pgfqpoint{1.714651in}{0.500322in}}%
\pgfpathlineto{\pgfqpoint{1.717041in}{0.493709in}}%
\pgfpathlineto{\pgfqpoint{1.722932in}{0.488802in}}%
\pgfpathlineto{\pgfqpoint{1.728488in}{0.492886in}}%
\pgfpathlineto{\pgfqpoint{1.734233in}{0.490553in}}%
\pgfpathlineto{\pgfqpoint{1.738650in}{0.493869in}}%
\pgfpathlineto{\pgfqpoint{1.740375in}{0.500409in}}%
\pgfpathlineto{\pgfqpoint{1.745433in}{0.500436in}}%
\pgfpathlineto{\pgfqpoint{1.747709in}{0.509837in}}%
\pgfpathlineto{\pgfqpoint{1.757144in}{0.510663in}}%
\pgfpathlineto{\pgfqpoint{1.759753in}{0.508170in}}%
\pgfpathlineto{\pgfqpoint{1.758777in}{0.502145in}}%
\pgfpathlineto{\pgfqpoint{1.769052in}{0.484340in}}%
\pgfpathlineto{\pgfqpoint{1.765360in}{0.476226in}}%
\pgfpathlineto{\pgfqpoint{1.749852in}{0.462071in}}%
\pgfpathlineto{\pgfqpoint{1.745282in}{0.461023in}}%
\pgfpathlineto{\pgfqpoint{1.724620in}{0.448821in}}%
\pgfpathlineto{\pgfqpoint{1.714126in}{0.445000in}}%
\pgfpathlineto{\pgfqpoint{1.711446in}{0.448095in}}%
\pgfpathlineto{\pgfqpoint{1.700720in}{0.443275in}}%
\pgfpathlineto{\pgfqpoint{1.700851in}{0.451264in}}%
\pgfpathlineto{\pgfqpoint{1.694633in}{0.447645in}}%
\pgfpathclose%
\pgfusepath{fill}%
\end{pgfscope}%
\begin{pgfscope}%
\pgfpathrectangle{\pgfqpoint{0.100000in}{0.100000in}}{\pgfqpoint{3.007045in}{1.925000in}}%
\pgfusepath{clip}%
\pgfsetbuttcap%
\pgfsetmiterjoin%
\definecolor{currentfill}{rgb}{0.657132,0.808105,0.895486}%
\pgfsetfillcolor{currentfill}%
\pgfsetlinewidth{0.000000pt}%
\definecolor{currentstroke}{rgb}{0.000000,0.000000,0.000000}%
\pgfsetstrokecolor{currentstroke}%
\pgfsetstrokeopacity{0.000000}%
\pgfsetdash{}{0pt}%
\pgfpathmoveto{\pgfqpoint{2.609375in}{1.212759in}}%
\pgfpathlineto{\pgfqpoint{2.603332in}{1.202130in}}%
\pgfpathlineto{\pgfqpoint{2.599182in}{1.199561in}}%
\pgfpathlineto{\pgfqpoint{2.596426in}{1.190754in}}%
\pgfpathlineto{\pgfqpoint{2.589115in}{1.195276in}}%
\pgfpathlineto{\pgfqpoint{2.586940in}{1.188956in}}%
\pgfpathlineto{\pgfqpoint{2.569924in}{1.199083in}}%
\pgfpathlineto{\pgfqpoint{2.571223in}{1.207045in}}%
\pgfpathlineto{\pgfqpoint{2.568471in}{1.207938in}}%
\pgfpathlineto{\pgfqpoint{2.569988in}{1.214538in}}%
\pgfpathlineto{\pgfqpoint{2.567300in}{1.215781in}}%
\pgfpathlineto{\pgfqpoint{2.560339in}{1.213292in}}%
\pgfpathlineto{\pgfqpoint{2.554976in}{1.247539in}}%
\pgfpathlineto{\pgfqpoint{2.559174in}{1.248298in}}%
\pgfpathlineto{\pgfqpoint{2.609634in}{1.257414in}}%
\pgfpathlineto{\pgfqpoint{2.613340in}{1.256035in}}%
\pgfpathlineto{\pgfqpoint{2.608755in}{1.249542in}}%
\pgfpathlineto{\pgfqpoint{2.607688in}{1.243264in}}%
\pgfpathlineto{\pgfqpoint{2.598024in}{1.242727in}}%
\pgfpathlineto{\pgfqpoint{2.592721in}{1.234363in}}%
\pgfpathlineto{\pgfqpoint{2.590743in}{1.227736in}}%
\pgfpathlineto{\pgfqpoint{2.585351in}{1.220532in}}%
\pgfpathlineto{\pgfqpoint{2.589864in}{1.218745in}}%
\pgfpathlineto{\pgfqpoint{2.604647in}{1.216037in}}%
\pgfpathlineto{\pgfqpoint{2.609375in}{1.212759in}}%
\pgfpathclose%
\pgfusepath{fill}%
\end{pgfscope}%
\begin{pgfscope}%
\pgfpathrectangle{\pgfqpoint{0.100000in}{0.100000in}}{\pgfqpoint{3.007045in}{1.925000in}}%
\pgfusepath{clip}%
\pgfsetbuttcap%
\pgfsetmiterjoin%
\definecolor{currentfill}{rgb}{0.485490,0.718524,0.853426}%
\pgfsetfillcolor{currentfill}%
\pgfsetlinewidth{0.000000pt}%
\definecolor{currentstroke}{rgb}{0.000000,0.000000,0.000000}%
\pgfsetstrokecolor{currentstroke}%
\pgfsetstrokeopacity{0.000000}%
\pgfsetdash{}{0pt}%
\pgfpathmoveto{\pgfqpoint{2.749311in}{1.495307in}}%
\pgfpathlineto{\pgfqpoint{2.743391in}{1.496243in}}%
\pgfpathlineto{\pgfqpoint{2.742514in}{1.505188in}}%
\pgfpathlineto{\pgfqpoint{2.739690in}{1.510307in}}%
\pgfpathlineto{\pgfqpoint{2.742037in}{1.517084in}}%
\pgfpathlineto{\pgfqpoint{2.739601in}{1.524017in}}%
\pgfpathlineto{\pgfqpoint{2.731549in}{1.525832in}}%
\pgfpathlineto{\pgfqpoint{2.733723in}{1.536267in}}%
\pgfpathlineto{\pgfqpoint{2.721573in}{1.573055in}}%
\pgfpathlineto{\pgfqpoint{2.735921in}{1.578293in}}%
\pgfpathlineto{\pgfqpoint{2.747323in}{1.582366in}}%
\pgfpathlineto{\pgfqpoint{2.750775in}{1.572935in}}%
\pgfpathlineto{\pgfqpoint{2.747693in}{1.569195in}}%
\pgfpathlineto{\pgfqpoint{2.755118in}{1.563156in}}%
\pgfpathlineto{\pgfqpoint{2.763075in}{1.564033in}}%
\pgfpathlineto{\pgfqpoint{2.763396in}{1.560589in}}%
\pgfpathlineto{\pgfqpoint{2.756305in}{1.557859in}}%
\pgfpathlineto{\pgfqpoint{2.774908in}{1.505824in}}%
\pgfpathlineto{\pgfqpoint{2.766923in}{1.496257in}}%
\pgfpathlineto{\pgfqpoint{2.759273in}{1.493756in}}%
\pgfpathlineto{\pgfqpoint{2.749311in}{1.495307in}}%
\pgfpathclose%
\pgfusepath{fill}%
\end{pgfscope}%
\begin{pgfscope}%
\pgfpathrectangle{\pgfqpoint{0.100000in}{0.100000in}}{\pgfqpoint{3.007045in}{1.925000in}}%
\pgfusepath{clip}%
\pgfsetbuttcap%
\pgfsetmiterjoin%
\definecolor{currentfill}{rgb}{0.585882,0.773641,0.875079}%
\pgfsetfillcolor{currentfill}%
\pgfsetlinewidth{0.000000pt}%
\definecolor{currentstroke}{rgb}{0.000000,0.000000,0.000000}%
\pgfsetstrokecolor{currentstroke}%
\pgfsetstrokeopacity{0.000000}%
\pgfsetdash{}{0pt}%
\pgfpathmoveto{\pgfqpoint{2.148381in}{1.158667in}}%
\pgfpathlineto{\pgfqpoint{2.142303in}{1.159619in}}%
\pgfpathlineto{\pgfqpoint{2.105428in}{1.156288in}}%
\pgfpathlineto{\pgfqpoint{2.105221in}{1.159153in}}%
\pgfpathlineto{\pgfqpoint{2.103104in}{1.187835in}}%
\pgfpathlineto{\pgfqpoint{2.123761in}{1.189439in}}%
\pgfpathlineto{\pgfqpoint{2.128488in}{1.192008in}}%
\pgfpathlineto{\pgfqpoint{2.128961in}{1.204856in}}%
\pgfpathlineto{\pgfqpoint{2.149319in}{1.206705in}}%
\pgfpathlineto{\pgfqpoint{2.152293in}{1.171574in}}%
\pgfpathlineto{\pgfqpoint{2.148720in}{1.168849in}}%
\pgfpathlineto{\pgfqpoint{2.150340in}{1.161666in}}%
\pgfpathlineto{\pgfqpoint{2.148381in}{1.158667in}}%
\pgfpathclose%
\pgfusepath{fill}%
\end{pgfscope}%
\begin{pgfscope}%
\pgfpathrectangle{\pgfqpoint{0.100000in}{0.100000in}}{\pgfqpoint{3.007045in}{1.925000in}}%
\pgfusepath{clip}%
\pgfsetbuttcap%
\pgfsetmiterjoin%
\definecolor{currentfill}{rgb}{0.270804,0.580730,0.781146}%
\pgfsetfillcolor{currentfill}%
\pgfsetlinewidth{0.000000pt}%
\definecolor{currentstroke}{rgb}{0.000000,0.000000,0.000000}%
\pgfsetstrokecolor{currentstroke}%
\pgfsetstrokeopacity{0.000000}%
\pgfsetdash{}{0pt}%
\pgfpathmoveto{\pgfqpoint{1.552729in}{1.550914in}}%
\pgfpathlineto{\pgfqpoint{1.527075in}{1.552396in}}%
\pgfpathlineto{\pgfqpoint{1.524722in}{1.557253in}}%
\pgfpathlineto{\pgfqpoint{1.526953in}{1.561321in}}%
\pgfpathlineto{\pgfqpoint{1.524166in}{1.567570in}}%
\pgfpathlineto{\pgfqpoint{1.518319in}{1.571035in}}%
\pgfpathlineto{\pgfqpoint{1.522749in}{1.580062in}}%
\pgfpathlineto{\pgfqpoint{1.526579in}{1.583044in}}%
\pgfpathlineto{\pgfqpoint{1.523621in}{1.593929in}}%
\pgfpathlineto{\pgfqpoint{1.518744in}{1.599173in}}%
\pgfpathlineto{\pgfqpoint{1.554716in}{1.596972in}}%
\pgfpathlineto{\pgfqpoint{1.552729in}{1.550914in}}%
\pgfpathclose%
\pgfusepath{fill}%
\end{pgfscope}%
\begin{pgfscope}%
\pgfpathrectangle{\pgfqpoint{0.100000in}{0.100000in}}{\pgfqpoint{3.007045in}{1.925000in}}%
\pgfusepath{clip}%
\pgfsetbuttcap%
\pgfsetmiterjoin%
\definecolor{currentfill}{rgb}{0.093272,0.396878,0.673664}%
\pgfsetfillcolor{currentfill}%
\pgfsetlinewidth{0.000000pt}%
\definecolor{currentstroke}{rgb}{0.000000,0.000000,0.000000}%
\pgfsetstrokecolor{currentstroke}%
\pgfsetstrokeopacity{0.000000}%
\pgfsetdash{}{0pt}%
\pgfpathmoveto{\pgfqpoint{0.720329in}{0.385044in}}%
\pgfpathlineto{\pgfqpoint{0.727658in}{0.379320in}}%
\pgfpathlineto{\pgfqpoint{0.723393in}{0.373987in}}%
\pgfpathlineto{\pgfqpoint{0.713956in}{0.381377in}}%
\pgfpathlineto{\pgfqpoint{0.717385in}{0.381921in}}%
\pgfpathlineto{\pgfqpoint{0.720189in}{0.383616in}}%
\pgfpathlineto{\pgfqpoint{0.720329in}{0.385044in}}%
\pgfpathclose%
\pgfusepath{fill}%
\end{pgfscope}%
\begin{pgfscope}%
\pgfpathrectangle{\pgfqpoint{0.100000in}{0.100000in}}{\pgfqpoint{3.007045in}{1.925000in}}%
\pgfusepath{clip}%
\pgfsetbuttcap%
\pgfsetmiterjoin%
\definecolor{currentfill}{rgb}{0.031373,0.285675,0.564291}%
\pgfsetfillcolor{currentfill}%
\pgfsetlinewidth{0.000000pt}%
\definecolor{currentstroke}{rgb}{0.000000,0.000000,0.000000}%
\pgfsetstrokecolor{currentstroke}%
\pgfsetstrokeopacity{0.000000}%
\pgfsetdash{}{0pt}%
\pgfpathmoveto{\pgfqpoint{1.483456in}{1.461953in}}%
\pgfpathlineto{\pgfqpoint{1.475452in}{1.459167in}}%
\pgfpathlineto{\pgfqpoint{1.473591in}{1.432656in}}%
\pgfpathlineto{\pgfqpoint{1.431343in}{1.435804in}}%
\pgfpathlineto{\pgfqpoint{1.430442in}{1.441625in}}%
\pgfpathlineto{\pgfqpoint{1.431599in}{1.456389in}}%
\pgfpathlineto{\pgfqpoint{1.437357in}{1.456409in}}%
\pgfpathlineto{\pgfqpoint{1.439374in}{1.475257in}}%
\pgfpathlineto{\pgfqpoint{1.442332in}{1.509325in}}%
\pgfpathlineto{\pgfqpoint{1.452730in}{1.510166in}}%
\pgfpathlineto{\pgfqpoint{1.458166in}{1.512912in}}%
\pgfpathlineto{\pgfqpoint{1.464688in}{1.511398in}}%
\pgfpathlineto{\pgfqpoint{1.473297in}{1.518824in}}%
\pgfpathlineto{\pgfqpoint{1.479161in}{1.518521in}}%
\pgfpathlineto{\pgfqpoint{1.484012in}{1.521937in}}%
\pgfpathlineto{\pgfqpoint{1.482943in}{1.518221in}}%
\pgfpathlineto{\pgfqpoint{1.481358in}{1.495327in}}%
\pgfpathlineto{\pgfqpoint{1.480026in}{1.483812in}}%
\pgfpathlineto{\pgfqpoint{1.485701in}{1.483457in}}%
\pgfpathlineto{\pgfqpoint{1.483456in}{1.461953in}}%
\pgfpathclose%
\pgfusepath{fill}%
\end{pgfscope}%
\begin{pgfscope}%
\pgfpathrectangle{\pgfqpoint{0.100000in}{0.100000in}}{\pgfqpoint{3.007045in}{1.925000in}}%
\pgfusepath{clip}%
\pgfsetbuttcap%
\pgfsetmiterjoin%
\definecolor{currentfill}{rgb}{0.381776,0.656517,0.824452}%
\pgfsetfillcolor{currentfill}%
\pgfsetlinewidth{0.000000pt}%
\definecolor{currentstroke}{rgb}{0.000000,0.000000,0.000000}%
\pgfsetstrokecolor{currentstroke}%
\pgfsetstrokeopacity{0.000000}%
\pgfsetdash{}{0pt}%
\pgfpathmoveto{\pgfqpoint{2.021691in}{1.270010in}}%
\pgfpathlineto{\pgfqpoint{2.021108in}{1.281507in}}%
\pgfpathlineto{\pgfqpoint{1.998647in}{1.280445in}}%
\pgfpathlineto{\pgfqpoint{1.998849in}{1.274698in}}%
\pgfpathlineto{\pgfqpoint{1.981668in}{1.274180in}}%
\pgfpathlineto{\pgfqpoint{1.973511in}{1.274071in}}%
\pgfpathlineto{\pgfqpoint{1.971270in}{1.279448in}}%
\pgfpathlineto{\pgfqpoint{1.964916in}{1.284730in}}%
\pgfpathlineto{\pgfqpoint{1.967742in}{1.296340in}}%
\pgfpathlineto{\pgfqpoint{1.980445in}{1.299472in}}%
\pgfpathlineto{\pgfqpoint{1.980128in}{1.308936in}}%
\pgfpathlineto{\pgfqpoint{1.974408in}{1.308715in}}%
\pgfpathlineto{\pgfqpoint{1.973965in}{1.320197in}}%
\pgfpathlineto{\pgfqpoint{1.996691in}{1.320570in}}%
\pgfpathlineto{\pgfqpoint{2.000310in}{1.318595in}}%
\pgfpathlineto{\pgfqpoint{2.006146in}{1.322530in}}%
\pgfpathlineto{\pgfqpoint{2.009626in}{1.309620in}}%
\pgfpathlineto{\pgfqpoint{2.025542in}{1.310434in}}%
\pgfpathlineto{\pgfqpoint{2.027069in}{1.287453in}}%
\pgfpathlineto{\pgfqpoint{2.037908in}{1.288046in}}%
\pgfpathlineto{\pgfqpoint{2.038925in}{1.270934in}}%
\pgfpathlineto{\pgfqpoint{2.021691in}{1.270010in}}%
\pgfpathclose%
\pgfusepath{fill}%
\end{pgfscope}%
\begin{pgfscope}%
\pgfpathrectangle{\pgfqpoint{0.100000in}{0.100000in}}{\pgfqpoint{3.007045in}{1.925000in}}%
\pgfusepath{clip}%
\pgfsetbuttcap%
\pgfsetmiterjoin%
\definecolor{currentfill}{rgb}{0.460392,0.704744,0.848012}%
\pgfsetfillcolor{currentfill}%
\pgfsetlinewidth{0.000000pt}%
\definecolor{currentstroke}{rgb}{0.000000,0.000000,0.000000}%
\pgfsetstrokecolor{currentstroke}%
\pgfsetstrokeopacity{0.000000}%
\pgfsetdash{}{0pt}%
\pgfpathmoveto{\pgfqpoint{2.273936in}{0.628314in}}%
\pgfpathlineto{\pgfqpoint{2.282576in}{0.629115in}}%
\pgfpathlineto{\pgfqpoint{2.284322in}{0.609983in}}%
\pgfpathlineto{\pgfqpoint{2.294588in}{0.610608in}}%
\pgfpathlineto{\pgfqpoint{2.291737in}{0.606164in}}%
\pgfpathlineto{\pgfqpoint{2.295880in}{0.596792in}}%
\pgfpathlineto{\pgfqpoint{2.288959in}{0.589362in}}%
\pgfpathlineto{\pgfqpoint{2.289539in}{0.581381in}}%
\pgfpathlineto{\pgfqpoint{2.275662in}{0.585133in}}%
\pgfpathlineto{\pgfqpoint{2.265886in}{0.586394in}}%
\pgfpathlineto{\pgfqpoint{2.252297in}{0.586303in}}%
\pgfpathlineto{\pgfqpoint{2.238850in}{0.583884in}}%
\pgfpathlineto{\pgfqpoint{2.221344in}{0.578728in}}%
\pgfpathlineto{\pgfqpoint{2.212863in}{0.577456in}}%
\pgfpathlineto{\pgfqpoint{2.202262in}{0.574181in}}%
\pgfpathlineto{\pgfqpoint{2.206102in}{0.581924in}}%
\pgfpathlineto{\pgfqpoint{2.210055in}{0.585128in}}%
\pgfpathlineto{\pgfqpoint{2.205014in}{0.590892in}}%
\pgfpathlineto{\pgfqpoint{2.207470in}{0.596991in}}%
\pgfpathlineto{\pgfqpoint{2.206456in}{0.600906in}}%
\pgfpathlineto{\pgfqpoint{2.199079in}{0.604683in}}%
\pgfpathlineto{\pgfqpoint{2.192424in}{0.612482in}}%
\pgfpathlineto{\pgfqpoint{2.193761in}{0.621372in}}%
\pgfpathlineto{\pgfqpoint{2.245494in}{0.625658in}}%
\pgfpathlineto{\pgfqpoint{2.273936in}{0.628314in}}%
\pgfpathclose%
\pgfusepath{fill}%
\end{pgfscope}%
\begin{pgfscope}%
\pgfpathrectangle{\pgfqpoint{0.100000in}{0.100000in}}{\pgfqpoint{3.007045in}{1.925000in}}%
\pgfusepath{clip}%
\pgfsetbuttcap%
\pgfsetmiterjoin%
\definecolor{currentfill}{rgb}{0.351511,0.635848,0.812641}%
\pgfsetfillcolor{currentfill}%
\pgfsetlinewidth{0.000000pt}%
\definecolor{currentstroke}{rgb}{0.000000,0.000000,0.000000}%
\pgfsetstrokecolor{currentstroke}%
\pgfsetstrokeopacity{0.000000}%
\pgfsetdash{}{0pt}%
\pgfpathmoveto{\pgfqpoint{1.539001in}{1.204970in}}%
\pgfpathlineto{\pgfqpoint{1.537445in}{1.176396in}}%
\pgfpathlineto{\pgfqpoint{1.510385in}{1.177845in}}%
\pgfpathlineto{\pgfqpoint{1.509436in}{1.177873in}}%
\pgfpathlineto{\pgfqpoint{1.511195in}{1.206454in}}%
\pgfpathlineto{\pgfqpoint{1.539001in}{1.204970in}}%
\pgfpathclose%
\pgfusepath{fill}%
\end{pgfscope}%
\begin{pgfscope}%
\pgfpathrectangle{\pgfqpoint{0.100000in}{0.100000in}}{\pgfqpoint{3.007045in}{1.925000in}}%
\pgfusepath{clip}%
\pgfsetbuttcap%
\pgfsetmiterjoin%
\definecolor{currentfill}{rgb}{0.523137,0.739193,0.861546}%
\pgfsetfillcolor{currentfill}%
\pgfsetlinewidth{0.000000pt}%
\definecolor{currentstroke}{rgb}{0.000000,0.000000,0.000000}%
\pgfsetstrokecolor{currentstroke}%
\pgfsetstrokeopacity{0.000000}%
\pgfsetdash{}{0pt}%
\pgfpathmoveto{\pgfqpoint{1.975423in}{0.932334in}}%
\pgfpathlineto{\pgfqpoint{1.969720in}{0.923885in}}%
\pgfpathlineto{\pgfqpoint{1.966993in}{0.915480in}}%
\pgfpathlineto{\pgfqpoint{1.967250in}{0.907985in}}%
\pgfpathlineto{\pgfqpoint{1.944456in}{0.907639in}}%
\pgfpathlineto{\pgfqpoint{1.944264in}{0.919360in}}%
\pgfpathlineto{\pgfqpoint{1.939696in}{0.925677in}}%
\pgfpathlineto{\pgfqpoint{1.940723in}{0.927923in}}%
\pgfpathlineto{\pgfqpoint{1.935018in}{0.933586in}}%
\pgfpathlineto{\pgfqpoint{1.927752in}{0.935378in}}%
\pgfpathlineto{\pgfqpoint{1.929258in}{0.940159in}}%
\pgfpathlineto{\pgfqpoint{1.921677in}{0.946416in}}%
\pgfpathlineto{\pgfqpoint{1.923472in}{0.954935in}}%
\pgfpathlineto{\pgfqpoint{1.954901in}{0.955275in}}%
\pgfpathlineto{\pgfqpoint{1.954694in}{0.961038in}}%
\pgfpathlineto{\pgfqpoint{1.960613in}{0.961138in}}%
\pgfpathlineto{\pgfqpoint{1.960431in}{0.971833in}}%
\pgfpathlineto{\pgfqpoint{1.962703in}{0.971887in}}%
\pgfpathlineto{\pgfqpoint{1.969967in}{0.964093in}}%
\pgfpathlineto{\pgfqpoint{1.971212in}{0.955492in}}%
\pgfpathlineto{\pgfqpoint{1.966442in}{0.954472in}}%
\pgfpathlineto{\pgfqpoint{1.966949in}{0.932026in}}%
\pgfpathlineto{\pgfqpoint{1.975423in}{0.932334in}}%
\pgfpathclose%
\pgfusepath{fill}%
\end{pgfscope}%
\begin{pgfscope}%
\pgfpathrectangle{\pgfqpoint{0.100000in}{0.100000in}}{\pgfqpoint{3.007045in}{1.925000in}}%
\pgfusepath{clip}%
\pgfsetbuttcap%
\pgfsetmiterjoin%
\definecolor{currentfill}{rgb}{0.504314,0.728858,0.857486}%
\pgfsetfillcolor{currentfill}%
\pgfsetlinewidth{0.000000pt}%
\definecolor{currentstroke}{rgb}{0.000000,0.000000,0.000000}%
\pgfsetstrokecolor{currentstroke}%
\pgfsetstrokeopacity{0.000000}%
\pgfsetdash{}{0pt}%
\pgfpathmoveto{\pgfqpoint{1.944978in}{1.116719in}}%
\pgfpathlineto{\pgfqpoint{1.945763in}{1.089375in}}%
\pgfpathlineto{\pgfqpoint{1.917287in}{1.088655in}}%
\pgfpathlineto{\pgfqpoint{1.911673in}{1.091410in}}%
\pgfpathlineto{\pgfqpoint{1.906201in}{1.091563in}}%
\pgfpathlineto{\pgfqpoint{1.906297in}{1.097100in}}%
\pgfpathlineto{\pgfqpoint{1.894852in}{1.097146in}}%
\pgfpathlineto{\pgfqpoint{1.883415in}{1.107800in}}%
\pgfpathlineto{\pgfqpoint{1.883385in}{1.113543in}}%
\pgfpathlineto{\pgfqpoint{1.872048in}{1.113887in}}%
\pgfpathlineto{\pgfqpoint{1.872212in}{1.129384in}}%
\pgfpathlineto{\pgfqpoint{1.872251in}{1.132295in}}%
\pgfpathlineto{\pgfqpoint{1.877944in}{1.138491in}}%
\pgfpathlineto{\pgfqpoint{1.877465in}{1.141048in}}%
\pgfpathlineto{\pgfqpoint{1.882047in}{1.149212in}}%
\pgfpathlineto{\pgfqpoint{1.884869in}{1.157037in}}%
\pgfpathlineto{\pgfqpoint{1.889132in}{1.155695in}}%
\pgfpathlineto{\pgfqpoint{1.903384in}{1.151278in}}%
\pgfpathlineto{\pgfqpoint{1.909292in}{1.151308in}}%
\pgfpathlineto{\pgfqpoint{1.920007in}{1.151193in}}%
\pgfpathlineto{\pgfqpoint{1.920082in}{1.139613in}}%
\pgfpathlineto{\pgfqpoint{1.944438in}{1.140114in}}%
\pgfpathlineto{\pgfqpoint{1.944978in}{1.116719in}}%
\pgfpathclose%
\pgfusepath{fill}%
\end{pgfscope}%
\begin{pgfscope}%
\pgfpathrectangle{\pgfqpoint{0.100000in}{0.100000in}}{\pgfqpoint{3.007045in}{1.925000in}}%
\pgfusepath{clip}%
\pgfsetbuttcap%
\pgfsetmiterjoin%
\definecolor{currentfill}{rgb}{0.150727,0.464452,0.720784}%
\pgfsetfillcolor{currentfill}%
\pgfsetlinewidth{0.000000pt}%
\definecolor{currentstroke}{rgb}{0.000000,0.000000,0.000000}%
\pgfsetstrokecolor{currentstroke}%
\pgfsetstrokeopacity{0.000000}%
\pgfsetdash{}{0pt}%
\pgfpathmoveto{\pgfqpoint{2.505019in}{0.681519in}}%
\pgfpathlineto{\pgfqpoint{2.503207in}{0.687158in}}%
\pgfpathlineto{\pgfqpoint{2.498935in}{0.687218in}}%
\pgfpathlineto{\pgfqpoint{2.496078in}{0.707079in}}%
\pgfpathlineto{\pgfqpoint{2.499221in}{0.711875in}}%
\pgfpathlineto{\pgfqpoint{2.496462in}{0.714530in}}%
\pgfpathlineto{\pgfqpoint{2.489650in}{0.715531in}}%
\pgfpathlineto{\pgfqpoint{2.489704in}{0.732476in}}%
\pgfpathlineto{\pgfqpoint{2.485433in}{0.742115in}}%
\pgfpathlineto{\pgfqpoint{2.509344in}{0.740112in}}%
\pgfpathlineto{\pgfqpoint{2.512057in}{0.734800in}}%
\pgfpathlineto{\pgfqpoint{2.513924in}{0.731882in}}%
\pgfpathlineto{\pgfqpoint{2.520931in}{0.729238in}}%
\pgfpathlineto{\pgfqpoint{2.526011in}{0.723942in}}%
\pgfpathlineto{\pgfqpoint{2.534070in}{0.723697in}}%
\pgfpathlineto{\pgfqpoint{2.537534in}{0.718203in}}%
\pgfpathlineto{\pgfqpoint{2.544812in}{0.713097in}}%
\pgfpathlineto{\pgfqpoint{2.549716in}{0.711373in}}%
\pgfpathlineto{\pgfqpoint{2.553497in}{0.705848in}}%
\pgfpathlineto{\pgfqpoint{2.550608in}{0.701257in}}%
\pgfpathlineto{\pgfqpoint{2.545964in}{0.706161in}}%
\pgfpathlineto{\pgfqpoint{2.536542in}{0.704088in}}%
\pgfpathlineto{\pgfqpoint{2.532623in}{0.707455in}}%
\pgfpathlineto{\pgfqpoint{2.529648in}{0.698799in}}%
\pgfpathlineto{\pgfqpoint{2.524353in}{0.695341in}}%
\pgfpathlineto{\pgfqpoint{2.528105in}{0.691406in}}%
\pgfpathlineto{\pgfqpoint{2.523188in}{0.681053in}}%
\pgfpathlineto{\pgfqpoint{2.518569in}{0.682840in}}%
\pgfpathlineto{\pgfqpoint{2.511887in}{0.680666in}}%
\pgfpathlineto{\pgfqpoint{2.505019in}{0.681519in}}%
\pgfpathclose%
\pgfusepath{fill}%
\end{pgfscope}%
\begin{pgfscope}%
\pgfpathrectangle{\pgfqpoint{0.100000in}{0.100000in}}{\pgfqpoint{3.007045in}{1.925000in}}%
\pgfusepath{clip}%
\pgfsetbuttcap%
\pgfsetmiterjoin%
\definecolor{currentfill}{rgb}{0.270804,0.580730,0.781146}%
\pgfsetfillcolor{currentfill}%
\pgfsetlinewidth{0.000000pt}%
\definecolor{currentstroke}{rgb}{0.000000,0.000000,0.000000}%
\pgfsetstrokecolor{currentstroke}%
\pgfsetstrokeopacity{0.000000}%
\pgfsetdash{}{0pt}%
\pgfpathmoveto{\pgfqpoint{1.704306in}{1.528001in}}%
\pgfpathlineto{\pgfqpoint{1.703956in}{1.505078in}}%
\pgfpathlineto{\pgfqpoint{1.703851in}{1.499277in}}%
\pgfpathlineto{\pgfqpoint{1.671967in}{1.499872in}}%
\pgfpathlineto{\pgfqpoint{1.637786in}{1.500890in}}%
\pgfpathlineto{\pgfqpoint{1.637825in}{1.506641in}}%
\pgfpathlineto{\pgfqpoint{1.632109in}{1.506861in}}%
\pgfpathlineto{\pgfqpoint{1.633052in}{1.541386in}}%
\pgfpathlineto{\pgfqpoint{1.668319in}{1.540277in}}%
\pgfpathlineto{\pgfqpoint{1.684336in}{1.539811in}}%
\pgfpathlineto{\pgfqpoint{1.684152in}{1.528327in}}%
\pgfpathlineto{\pgfqpoint{1.704306in}{1.528001in}}%
\pgfpathclose%
\pgfusepath{fill}%
\end{pgfscope}%
\begin{pgfscope}%
\pgfpathrectangle{\pgfqpoint{0.100000in}{0.100000in}}{\pgfqpoint{3.007045in}{1.925000in}}%
\pgfusepath{clip}%
\pgfsetbuttcap%
\pgfsetmiterjoin%
\definecolor{currentfill}{rgb}{0.491765,0.721968,0.854779}%
\pgfsetfillcolor{currentfill}%
\pgfsetlinewidth{0.000000pt}%
\definecolor{currentstroke}{rgb}{0.000000,0.000000,0.000000}%
\pgfsetstrokecolor{currentstroke}%
\pgfsetstrokeopacity{0.000000}%
\pgfsetdash{}{0pt}%
\pgfpathmoveto{\pgfqpoint{1.616016in}{0.651438in}}%
\pgfpathlineto{\pgfqpoint{1.586901in}{0.635387in}}%
\pgfpathlineto{\pgfqpoint{1.575122in}{0.656245in}}%
\pgfpathlineto{\pgfqpoint{1.565148in}{0.655103in}}%
\pgfpathlineto{\pgfqpoint{1.551784in}{0.680740in}}%
\pgfpathlineto{\pgfqpoint{1.577602in}{0.694365in}}%
\pgfpathlineto{\pgfqpoint{1.578119in}{0.708542in}}%
\pgfpathlineto{\pgfqpoint{1.600832in}{0.707630in}}%
\pgfpathlineto{\pgfqpoint{1.607101in}{0.688968in}}%
\pgfpathlineto{\pgfqpoint{1.611326in}{0.679134in}}%
\pgfpathlineto{\pgfqpoint{1.603281in}{0.674775in}}%
\pgfpathlineto{\pgfqpoint{1.616016in}{0.651438in}}%
\pgfpathclose%
\pgfusepath{fill}%
\end{pgfscope}%
\begin{pgfscope}%
\pgfpathrectangle{\pgfqpoint{0.100000in}{0.100000in}}{\pgfqpoint{3.007045in}{1.925000in}}%
\pgfusepath{clip}%
\pgfsetbuttcap%
\pgfsetmiterjoin%
\definecolor{currentfill}{rgb}{0.567059,0.763306,0.871019}%
\pgfsetfillcolor{currentfill}%
\pgfsetlinewidth{0.000000pt}%
\definecolor{currentstroke}{rgb}{0.000000,0.000000,0.000000}%
\pgfsetstrokecolor{currentstroke}%
\pgfsetstrokeopacity{0.000000}%
\pgfsetdash{}{0pt}%
\pgfpathmoveto{\pgfqpoint{2.609375in}{1.212759in}}%
\pgfpathlineto{\pgfqpoint{2.614276in}{1.211702in}}%
\pgfpathlineto{\pgfqpoint{2.616014in}{1.216485in}}%
\pgfpathlineto{\pgfqpoint{2.621536in}{1.211734in}}%
\pgfpathlineto{\pgfqpoint{2.622986in}{1.207692in}}%
\pgfpathlineto{\pgfqpoint{2.620005in}{1.199545in}}%
\pgfpathlineto{\pgfqpoint{2.625996in}{1.195904in}}%
\pgfpathlineto{\pgfqpoint{2.624882in}{1.186857in}}%
\pgfpathlineto{\pgfqpoint{2.621002in}{1.178691in}}%
\pgfpathlineto{\pgfqpoint{2.619686in}{1.171111in}}%
\pgfpathlineto{\pgfqpoint{2.613166in}{1.160149in}}%
\pgfpathlineto{\pgfqpoint{2.609006in}{1.155780in}}%
\pgfpathlineto{\pgfqpoint{2.581715in}{1.169068in}}%
\pgfpathlineto{\pgfqpoint{2.577995in}{1.164006in}}%
\pgfpathlineto{\pgfqpoint{2.569128in}{1.165501in}}%
\pgfpathlineto{\pgfqpoint{2.565022in}{1.171056in}}%
\pgfpathlineto{\pgfqpoint{2.558875in}{1.172717in}}%
\pgfpathlineto{\pgfqpoint{2.563877in}{1.187087in}}%
\pgfpathlineto{\pgfqpoint{2.562770in}{1.192250in}}%
\pgfpathlineto{\pgfqpoint{2.569924in}{1.199083in}}%
\pgfpathlineto{\pgfqpoint{2.586940in}{1.188956in}}%
\pgfpathlineto{\pgfqpoint{2.589115in}{1.195276in}}%
\pgfpathlineto{\pgfqpoint{2.596426in}{1.190754in}}%
\pgfpathlineto{\pgfqpoint{2.599182in}{1.199561in}}%
\pgfpathlineto{\pgfqpoint{2.603332in}{1.202130in}}%
\pgfpathlineto{\pgfqpoint{2.609375in}{1.212759in}}%
\pgfpathclose%
\pgfusepath{fill}%
\end{pgfscope}%
\begin{pgfscope}%
\pgfpathrectangle{\pgfqpoint{0.100000in}{0.100000in}}{\pgfqpoint{3.007045in}{1.925000in}}%
\pgfusepath{clip}%
\pgfsetbuttcap%
\pgfsetmiterjoin%
\definecolor{currentfill}{rgb}{0.285936,0.591065,0.787051}%
\pgfsetfillcolor{currentfill}%
\pgfsetlinewidth{0.000000pt}%
\definecolor{currentstroke}{rgb}{0.000000,0.000000,0.000000}%
\pgfsetstrokecolor{currentstroke}%
\pgfsetstrokeopacity{0.000000}%
\pgfsetdash{}{0pt}%
\pgfpathmoveto{\pgfqpoint{2.028875in}{0.983778in}}%
\pgfpathlineto{\pgfqpoint{2.025516in}{0.974832in}}%
\pgfpathlineto{\pgfqpoint{1.977561in}{0.972537in}}%
\pgfpathlineto{\pgfqpoint{1.977205in}{0.994031in}}%
\pgfpathlineto{\pgfqpoint{1.971458in}{0.993921in}}%
\pgfpathlineto{\pgfqpoint{1.971148in}{1.011094in}}%
\pgfpathlineto{\pgfqpoint{1.981594in}{1.012166in}}%
\pgfpathlineto{\pgfqpoint{1.984516in}{1.009519in}}%
\pgfpathlineto{\pgfqpoint{1.994242in}{1.009746in}}%
\pgfpathlineto{\pgfqpoint{1.996111in}{1.017492in}}%
\pgfpathlineto{\pgfqpoint{1.995876in}{1.024384in}}%
\pgfpathlineto{\pgfqpoint{2.005554in}{1.024952in}}%
\pgfpathlineto{\pgfqpoint{2.005455in}{1.027915in}}%
\pgfpathlineto{\pgfqpoint{2.022844in}{1.028625in}}%
\pgfpathlineto{\pgfqpoint{2.023343in}{1.013571in}}%
\pgfpathlineto{\pgfqpoint{2.029271in}{1.013848in}}%
\pgfpathlineto{\pgfqpoint{2.022047in}{1.002653in}}%
\pgfpathlineto{\pgfqpoint{2.028045in}{0.989826in}}%
\pgfpathlineto{\pgfqpoint{2.028875in}{0.983778in}}%
\pgfpathclose%
\pgfusepath{fill}%
\end{pgfscope}%
\begin{pgfscope}%
\pgfpathrectangle{\pgfqpoint{0.100000in}{0.100000in}}{\pgfqpoint{3.007045in}{1.925000in}}%
\pgfusepath{clip}%
\pgfsetbuttcap%
\pgfsetmiterjoin%
\definecolor{currentfill}{rgb}{0.183206,0.496932,0.737516}%
\pgfsetfillcolor{currentfill}%
\pgfsetlinewidth{0.000000pt}%
\definecolor{currentstroke}{rgb}{0.000000,0.000000,0.000000}%
\pgfsetstrokecolor{currentstroke}%
\pgfsetstrokeopacity{0.000000}%
\pgfsetdash{}{0pt}%
\pgfpathmoveto{\pgfqpoint{1.331685in}{1.819464in}}%
\pgfpathlineto{\pgfqpoint{1.375731in}{1.814864in}}%
\pgfpathlineto{\pgfqpoint{1.371755in}{1.774248in}}%
\pgfpathlineto{\pgfqpoint{1.346116in}{1.776879in}}%
\pgfpathlineto{\pgfqpoint{1.346744in}{1.782663in}}%
\pgfpathlineto{\pgfqpoint{1.341017in}{1.783290in}}%
\pgfpathlineto{\pgfqpoint{1.341649in}{1.789040in}}%
\pgfpathlineto{\pgfqpoint{1.332164in}{1.790080in}}%
\pgfpathlineto{\pgfqpoint{1.329938in}{1.796177in}}%
\pgfpathlineto{\pgfqpoint{1.331869in}{1.813477in}}%
\pgfpathlineto{\pgfqpoint{1.331685in}{1.819464in}}%
\pgfpathclose%
\pgfusepath{fill}%
\end{pgfscope}%
\begin{pgfscope}%
\pgfpathrectangle{\pgfqpoint{0.100000in}{0.100000in}}{\pgfqpoint{3.007045in}{1.925000in}}%
\pgfusepath{clip}%
\pgfsetbuttcap%
\pgfsetmiterjoin%
\definecolor{currentfill}{rgb}{0.401953,0.670296,0.832326}%
\pgfsetfillcolor{currentfill}%
\pgfsetlinewidth{0.000000pt}%
\definecolor{currentstroke}{rgb}{0.000000,0.000000,0.000000}%
\pgfsetstrokecolor{currentstroke}%
\pgfsetstrokeopacity{0.000000}%
\pgfsetdash{}{0pt}%
\pgfpathmoveto{\pgfqpoint{1.782023in}{1.648669in}}%
\pgfpathlineto{\pgfqpoint{1.764918in}{1.648716in}}%
\pgfpathlineto{\pgfqpoint{1.764876in}{1.642920in}}%
\pgfpathlineto{\pgfqpoint{1.718799in}{1.643280in}}%
\pgfpathlineto{\pgfqpoint{1.719048in}{1.660538in}}%
\pgfpathlineto{\pgfqpoint{1.718261in}{1.672097in}}%
\pgfpathlineto{\pgfqpoint{1.724037in}{1.672065in}}%
\pgfpathlineto{\pgfqpoint{1.724288in}{1.695202in}}%
\pgfpathlineto{\pgfqpoint{1.747387in}{1.695072in}}%
\pgfpathlineto{\pgfqpoint{1.746352in}{1.706621in}}%
\pgfpathlineto{\pgfqpoint{1.746314in}{1.723984in}}%
\pgfpathlineto{\pgfqpoint{1.745890in}{1.729762in}}%
\pgfpathlineto{\pgfqpoint{1.745680in}{1.764366in}}%
\pgfpathlineto{\pgfqpoint{1.762948in}{1.764365in}}%
\pgfpathlineto{\pgfqpoint{1.762966in}{1.752742in}}%
\pgfpathlineto{\pgfqpoint{1.797671in}{1.752902in}}%
\pgfpathlineto{\pgfqpoint{1.798346in}{1.718105in}}%
\pgfpathlineto{\pgfqpoint{1.798628in}{1.689186in}}%
\pgfpathlineto{\pgfqpoint{1.787190in}{1.689017in}}%
\pgfpathlineto{\pgfqpoint{1.787806in}{1.648648in}}%
\pgfpathlineto{\pgfqpoint{1.782023in}{1.648669in}}%
\pgfpathclose%
\pgfusepath{fill}%
\end{pgfscope}%
\begin{pgfscope}%
\pgfpathrectangle{\pgfqpoint{0.100000in}{0.100000in}}{\pgfqpoint{3.007045in}{1.925000in}}%
\pgfusepath{clip}%
\pgfsetbuttcap%
\pgfsetmiterjoin%
\definecolor{currentfill}{rgb}{0.223806,0.537532,0.758431}%
\pgfsetfillcolor{currentfill}%
\pgfsetlinewidth{0.000000pt}%
\definecolor{currentstroke}{rgb}{0.000000,0.000000,0.000000}%
\pgfsetstrokecolor{currentstroke}%
\pgfsetstrokeopacity{0.000000}%
\pgfsetdash{}{0pt}%
\pgfpathmoveto{\pgfqpoint{1.910648in}{0.511564in}}%
\pgfpathlineto{\pgfqpoint{1.909910in}{0.542133in}}%
\pgfpathlineto{\pgfqpoint{1.908086in}{0.571754in}}%
\pgfpathlineto{\pgfqpoint{1.910006in}{0.578352in}}%
\pgfpathlineto{\pgfqpoint{1.909431in}{0.599271in}}%
\pgfpathlineto{\pgfqpoint{1.915657in}{0.600227in}}%
\pgfpathlineto{\pgfqpoint{1.921122in}{0.606604in}}%
\pgfpathlineto{\pgfqpoint{1.927440in}{0.602431in}}%
\pgfpathlineto{\pgfqpoint{1.930070in}{0.596671in}}%
\pgfpathlineto{\pgfqpoint{1.954185in}{0.597454in}}%
\pgfpathlineto{\pgfqpoint{1.959192in}{0.587885in}}%
\pgfpathlineto{\pgfqpoint{1.958167in}{0.578691in}}%
\pgfpathlineto{\pgfqpoint{1.959526in}{0.575009in}}%
\pgfpathlineto{\pgfqpoint{1.965180in}{0.570849in}}%
\pgfpathlineto{\pgfqpoint{1.966602in}{0.562051in}}%
\pgfpathlineto{\pgfqpoint{1.970123in}{0.557651in}}%
\pgfpathlineto{\pgfqpoint{1.975186in}{0.557014in}}%
\pgfpathlineto{\pgfqpoint{1.976175in}{0.548583in}}%
\pgfpathlineto{\pgfqpoint{1.990185in}{0.543999in}}%
\pgfpathlineto{\pgfqpoint{1.988235in}{0.542241in}}%
\pgfpathlineto{\pgfqpoint{1.992985in}{0.532196in}}%
\pgfpathlineto{\pgfqpoint{1.997286in}{0.531031in}}%
\pgfpathlineto{\pgfqpoint{1.998738in}{0.526353in}}%
\pgfpathlineto{\pgfqpoint{1.993498in}{0.524333in}}%
\pgfpathlineto{\pgfqpoint{1.981476in}{0.527516in}}%
\pgfpathlineto{\pgfqpoint{1.982084in}{0.530926in}}%
\pgfpathlineto{\pgfqpoint{1.976086in}{0.538805in}}%
\pgfpathlineto{\pgfqpoint{1.967294in}{0.537756in}}%
\pgfpathlineto{\pgfqpoint{1.962543in}{0.530961in}}%
\pgfpathlineto{\pgfqpoint{1.953040in}{0.522060in}}%
\pgfpathlineto{\pgfqpoint{1.952704in}{0.529849in}}%
\pgfpathlineto{\pgfqpoint{1.943821in}{0.525996in}}%
\pgfpathlineto{\pgfqpoint{1.937749in}{0.521438in}}%
\pgfpathlineto{\pgfqpoint{1.939627in}{0.515520in}}%
\pgfpathlineto{\pgfqpoint{1.946540in}{0.514748in}}%
\pgfpathlineto{\pgfqpoint{1.955604in}{0.516241in}}%
\pgfpathlineto{\pgfqpoint{1.963363in}{0.511582in}}%
\pgfpathlineto{\pgfqpoint{1.957159in}{0.505997in}}%
\pgfpathlineto{\pgfqpoint{1.943062in}{0.512593in}}%
\pgfpathlineto{\pgfqpoint{1.937453in}{0.512510in}}%
\pgfpathlineto{\pgfqpoint{1.928770in}{0.508924in}}%
\pgfpathlineto{\pgfqpoint{1.910648in}{0.511564in}}%
\pgfpathclose%
\pgfusepath{fill}%
\end{pgfscope}%
\begin{pgfscope}%
\pgfpathrectangle{\pgfqpoint{0.100000in}{0.100000in}}{\pgfqpoint{3.007045in}{1.925000in}}%
\pgfusepath{clip}%
\pgfsetbuttcap%
\pgfsetmiterjoin%
\definecolor{currentfill}{rgb}{0.215686,0.529412,0.754248}%
\pgfsetfillcolor{currentfill}%
\pgfsetlinewidth{0.000000pt}%
\definecolor{currentstroke}{rgb}{0.000000,0.000000,0.000000}%
\pgfsetstrokecolor{currentstroke}%
\pgfsetstrokeopacity{0.000000}%
\pgfsetdash{}{0pt}%
\pgfpathmoveto{\pgfqpoint{1.547367in}{0.412624in}}%
\pgfpathlineto{\pgfqpoint{1.512666in}{0.412373in}}%
\pgfpathlineto{\pgfqpoint{1.512921in}{0.424052in}}%
\pgfpathlineto{\pgfqpoint{1.514242in}{0.453233in}}%
\pgfpathlineto{\pgfqpoint{1.513462in}{0.453257in}}%
\pgfpathlineto{\pgfqpoint{1.514641in}{0.483384in}}%
\pgfpathlineto{\pgfqpoint{1.550168in}{0.481716in}}%
\pgfpathlineto{\pgfqpoint{1.549167in}{0.452106in}}%
\pgfpathlineto{\pgfqpoint{1.547367in}{0.412624in}}%
\pgfpathclose%
\pgfusepath{fill}%
\end{pgfscope}%
\begin{pgfscope}%
\pgfpathrectangle{\pgfqpoint{0.100000in}{0.100000in}}{\pgfqpoint{3.007045in}{1.925000in}}%
\pgfusepath{clip}%
\pgfsetbuttcap%
\pgfsetmiterjoin%
\definecolor{currentfill}{rgb}{0.285936,0.591065,0.787051}%
\pgfsetfillcolor{currentfill}%
\pgfsetlinewidth{0.000000pt}%
\definecolor{currentstroke}{rgb}{0.000000,0.000000,0.000000}%
\pgfsetstrokecolor{currentstroke}%
\pgfsetstrokeopacity{0.000000}%
\pgfsetdash{}{0pt}%
\pgfpathmoveto{\pgfqpoint{2.488716in}{0.626214in}}%
\pgfpathlineto{\pgfqpoint{2.481583in}{0.625736in}}%
\pgfpathlineto{\pgfqpoint{2.479352in}{0.633603in}}%
\pgfpathlineto{\pgfqpoint{2.469936in}{0.633604in}}%
\pgfpathlineto{\pgfqpoint{2.464523in}{0.639748in}}%
\pgfpathlineto{\pgfqpoint{2.457148in}{0.641063in}}%
\pgfpathlineto{\pgfqpoint{2.454390in}{0.661837in}}%
\pgfpathlineto{\pgfqpoint{2.450133in}{0.661231in}}%
\pgfpathlineto{\pgfqpoint{2.450078in}{0.667343in}}%
\pgfpathlineto{\pgfqpoint{2.443165in}{0.674743in}}%
\pgfpathlineto{\pgfqpoint{2.442052in}{0.679496in}}%
\pgfpathlineto{\pgfqpoint{2.443946in}{0.680145in}}%
\pgfpathlineto{\pgfqpoint{2.445730in}{0.689304in}}%
\pgfpathlineto{\pgfqpoint{2.448668in}{0.692985in}}%
\pgfpathlineto{\pgfqpoint{2.447789in}{0.700910in}}%
\pgfpathlineto{\pgfqpoint{2.455203in}{0.702082in}}%
\pgfpathlineto{\pgfqpoint{2.457652in}{0.694920in}}%
\pgfpathlineto{\pgfqpoint{2.477917in}{0.698299in}}%
\pgfpathlineto{\pgfqpoint{2.492063in}{0.691931in}}%
\pgfpathlineto{\pgfqpoint{2.498935in}{0.687218in}}%
\pgfpathlineto{\pgfqpoint{2.503207in}{0.687158in}}%
\pgfpathlineto{\pgfqpoint{2.505019in}{0.681519in}}%
\pgfpathlineto{\pgfqpoint{2.508359in}{0.677599in}}%
\pgfpathlineto{\pgfqpoint{2.503124in}{0.674785in}}%
\pgfpathlineto{\pgfqpoint{2.497961in}{0.669319in}}%
\pgfpathlineto{\pgfqpoint{2.498304in}{0.661191in}}%
\pgfpathlineto{\pgfqpoint{2.503300in}{0.656930in}}%
\pgfpathlineto{\pgfqpoint{2.487249in}{0.654855in}}%
\pgfpathlineto{\pgfqpoint{2.488999in}{0.640420in}}%
\pgfpathlineto{\pgfqpoint{2.504409in}{0.641856in}}%
\pgfpathlineto{\pgfqpoint{2.502750in}{0.627127in}}%
\pgfpathlineto{\pgfqpoint{2.488716in}{0.626214in}}%
\pgfpathclose%
\pgfusepath{fill}%
\end{pgfscope}%
\begin{pgfscope}%
\pgfpathrectangle{\pgfqpoint{0.100000in}{0.100000in}}{\pgfqpoint{3.007045in}{1.925000in}}%
\pgfusepath{clip}%
\pgfsetbuttcap%
\pgfsetmiterjoin%
\definecolor{currentfill}{rgb}{0.585882,0.773641,0.875079}%
\pgfsetfillcolor{currentfill}%
\pgfsetlinewidth{0.000000pt}%
\definecolor{currentstroke}{rgb}{0.000000,0.000000,0.000000}%
\pgfsetstrokecolor{currentstroke}%
\pgfsetstrokeopacity{0.000000}%
\pgfsetdash{}{0pt}%
\pgfpathmoveto{\pgfqpoint{0.499509in}{1.627848in}}%
\pgfpathlineto{\pgfqpoint{0.492514in}{1.625981in}}%
\pgfpathlineto{\pgfqpoint{0.490196in}{1.618627in}}%
\pgfpathlineto{\pgfqpoint{0.462254in}{1.627100in}}%
\pgfpathlineto{\pgfqpoint{0.464341in}{1.633883in}}%
\pgfpathlineto{\pgfqpoint{0.447557in}{1.638893in}}%
\pgfpathlineto{\pgfqpoint{0.447535in}{1.643476in}}%
\pgfpathlineto{\pgfqpoint{0.451005in}{1.654418in}}%
\pgfpathlineto{\pgfqpoint{0.441475in}{1.657513in}}%
\pgfpathlineto{\pgfqpoint{0.434242in}{1.663279in}}%
\pgfpathlineto{\pgfqpoint{0.432010in}{1.670306in}}%
\pgfpathlineto{\pgfqpoint{0.423095in}{1.674955in}}%
\pgfpathlineto{\pgfqpoint{0.416059in}{1.671542in}}%
\pgfpathlineto{\pgfqpoint{0.406441in}{1.674762in}}%
\pgfpathlineto{\pgfqpoint{0.413049in}{1.689210in}}%
\pgfpathlineto{\pgfqpoint{0.416964in}{1.700001in}}%
\pgfpathlineto{\pgfqpoint{0.432427in}{1.695452in}}%
\pgfpathlineto{\pgfqpoint{0.436719in}{1.699407in}}%
\pgfpathlineto{\pgfqpoint{0.437975in}{1.704042in}}%
\pgfpathlineto{\pgfqpoint{0.443505in}{1.702226in}}%
\pgfpathlineto{\pgfqpoint{0.449171in}{1.720447in}}%
\pgfpathlineto{\pgfqpoint{0.444068in}{1.723357in}}%
\pgfpathlineto{\pgfqpoint{0.451037in}{1.744559in}}%
\pgfpathlineto{\pgfqpoint{0.448349in}{1.745454in}}%
\pgfpathlineto{\pgfqpoint{0.451249in}{1.754244in}}%
\pgfpathlineto{\pgfqpoint{0.465504in}{1.749616in}}%
\pgfpathlineto{\pgfqpoint{0.469874in}{1.763357in}}%
\pgfpathlineto{\pgfqpoint{0.484457in}{1.758626in}}%
\pgfpathlineto{\pgfqpoint{0.488462in}{1.755366in}}%
\pgfpathlineto{\pgfqpoint{0.490020in}{1.750804in}}%
\pgfpathlineto{\pgfqpoint{0.496816in}{1.743625in}}%
\pgfpathlineto{\pgfqpoint{0.496421in}{1.737333in}}%
\pgfpathlineto{\pgfqpoint{0.493784in}{1.734570in}}%
\pgfpathlineto{\pgfqpoint{0.496057in}{1.728942in}}%
\pgfpathlineto{\pgfqpoint{0.500179in}{1.724897in}}%
\pgfpathlineto{\pgfqpoint{0.501234in}{1.718853in}}%
\pgfpathlineto{\pgfqpoint{0.506524in}{1.713649in}}%
\pgfpathlineto{\pgfqpoint{0.536189in}{1.704696in}}%
\pgfpathlineto{\pgfqpoint{0.533873in}{1.701067in}}%
\pgfpathlineto{\pgfqpoint{0.530035in}{1.695577in}}%
\pgfpathlineto{\pgfqpoint{0.527922in}{1.686852in}}%
\pgfpathlineto{\pgfqpoint{0.523556in}{1.681083in}}%
\pgfpathlineto{\pgfqpoint{0.521156in}{1.673183in}}%
\pgfpathlineto{\pgfqpoint{0.521441in}{1.664987in}}%
\pgfpathlineto{\pgfqpoint{0.519173in}{1.654672in}}%
\pgfpathlineto{\pgfqpoint{0.512713in}{1.647749in}}%
\pgfpathlineto{\pgfqpoint{0.505425in}{1.642739in}}%
\pgfpathlineto{\pgfqpoint{0.504292in}{1.636653in}}%
\pgfpathlineto{\pgfqpoint{0.499509in}{1.627848in}}%
\pgfpathclose%
\pgfusepath{fill}%
\end{pgfscope}%
\begin{pgfscope}%
\pgfpathrectangle{\pgfqpoint{0.100000in}{0.100000in}}{\pgfqpoint{3.007045in}{1.925000in}}%
\pgfusepath{clip}%
\pgfsetbuttcap%
\pgfsetmiterjoin%
\definecolor{currentfill}{rgb}{0.381776,0.656517,0.824452}%
\pgfsetfillcolor{currentfill}%
\pgfsetlinewidth{0.000000pt}%
\definecolor{currentstroke}{rgb}{0.000000,0.000000,0.000000}%
\pgfsetstrokecolor{currentstroke}%
\pgfsetstrokeopacity{0.000000}%
\pgfsetdash{}{0pt}%
\pgfpathmoveto{\pgfqpoint{2.321001in}{0.750113in}}%
\pgfpathlineto{\pgfqpoint{2.312802in}{0.748173in}}%
\pgfpathlineto{\pgfqpoint{2.295738in}{0.746225in}}%
\pgfpathlineto{\pgfqpoint{2.293144in}{0.771183in}}%
\pgfpathlineto{\pgfqpoint{2.289813in}{0.770791in}}%
\pgfpathlineto{\pgfqpoint{2.287725in}{0.796480in}}%
\pgfpathlineto{\pgfqpoint{2.281013in}{0.795971in}}%
\pgfpathlineto{\pgfqpoint{2.278832in}{0.801530in}}%
\pgfpathlineto{\pgfqpoint{2.267838in}{0.800489in}}%
\pgfpathlineto{\pgfqpoint{2.264190in}{0.805986in}}%
\pgfpathlineto{\pgfqpoint{2.258979in}{0.805773in}}%
\pgfpathlineto{\pgfqpoint{2.263640in}{0.812852in}}%
\pgfpathlineto{\pgfqpoint{2.262319in}{0.816941in}}%
\pgfpathlineto{\pgfqpoint{2.271634in}{0.825054in}}%
\pgfpathlineto{\pgfqpoint{2.279304in}{0.827039in}}%
\pgfpathlineto{\pgfqpoint{2.290820in}{0.827401in}}%
\pgfpathlineto{\pgfqpoint{2.297886in}{0.828687in}}%
\pgfpathlineto{\pgfqpoint{2.298990in}{0.824653in}}%
\pgfpathlineto{\pgfqpoint{2.317320in}{0.826818in}}%
\pgfpathlineto{\pgfqpoint{2.318665in}{0.820809in}}%
\pgfpathlineto{\pgfqpoint{2.326336in}{0.819580in}}%
\pgfpathlineto{\pgfqpoint{2.327623in}{0.805924in}}%
\pgfpathlineto{\pgfqpoint{2.332358in}{0.802455in}}%
\pgfpathlineto{\pgfqpoint{2.343332in}{0.803548in}}%
\pgfpathlineto{\pgfqpoint{2.346769in}{0.793773in}}%
\pgfpathlineto{\pgfqpoint{2.352410in}{0.787703in}}%
\pgfpathlineto{\pgfqpoint{2.352426in}{0.785269in}}%
\pgfpathlineto{\pgfqpoint{2.333138in}{0.783244in}}%
\pgfpathlineto{\pgfqpoint{2.335233in}{0.759815in}}%
\pgfpathlineto{\pgfqpoint{2.317396in}{0.757985in}}%
\pgfpathlineto{\pgfqpoint{2.321001in}{0.750113in}}%
\pgfpathclose%
\pgfusepath{fill}%
\end{pgfscope}%
\begin{pgfscope}%
\pgfpathrectangle{\pgfqpoint{0.100000in}{0.100000in}}{\pgfqpoint{3.007045in}{1.925000in}}%
\pgfusepath{clip}%
\pgfsetbuttcap%
\pgfsetmiterjoin%
\definecolor{currentfill}{rgb}{0.592157,0.777086,0.876432}%
\pgfsetfillcolor{currentfill}%
\pgfsetlinewidth{0.000000pt}%
\definecolor{currentstroke}{rgb}{0.000000,0.000000,0.000000}%
\pgfsetstrokecolor{currentstroke}%
\pgfsetstrokeopacity{0.000000}%
\pgfsetdash{}{0pt}%
\pgfpathmoveto{\pgfqpoint{1.104848in}{1.099688in}}%
\pgfpathlineto{\pgfqpoint{1.098980in}{1.105934in}}%
\pgfpathlineto{\pgfqpoint{1.087280in}{1.105261in}}%
\pgfpathlineto{\pgfqpoint{1.085598in}{1.110401in}}%
\pgfpathlineto{\pgfqpoint{1.056798in}{1.114784in}}%
\pgfpathlineto{\pgfqpoint{1.045110in}{1.115632in}}%
\pgfpathlineto{\pgfqpoint{1.047972in}{1.134089in}}%
\pgfpathlineto{\pgfqpoint{1.048160in}{1.141447in}}%
\pgfpathlineto{\pgfqpoint{1.050444in}{1.156057in}}%
\pgfpathlineto{\pgfqpoint{1.059704in}{1.212390in}}%
\pgfpathlineto{\pgfqpoint{1.091973in}{1.207431in}}%
\pgfpathlineto{\pgfqpoint{1.141305in}{1.200383in}}%
\pgfpathlineto{\pgfqpoint{1.139110in}{1.197239in}}%
\pgfpathlineto{\pgfqpoint{1.142160in}{1.192960in}}%
\pgfpathlineto{\pgfqpoint{1.136431in}{1.191196in}}%
\pgfpathlineto{\pgfqpoint{1.128210in}{1.131426in}}%
\pgfpathlineto{\pgfqpoint{1.121297in}{1.132386in}}%
\pgfpathlineto{\pgfqpoint{1.123455in}{1.127357in}}%
\pgfpathlineto{\pgfqpoint{1.123423in}{1.117299in}}%
\pgfpathlineto{\pgfqpoint{1.121625in}{1.109932in}}%
\pgfpathlineto{\pgfqpoint{1.112363in}{1.107267in}}%
\pgfpathlineto{\pgfqpoint{1.104848in}{1.099688in}}%
\pgfpathclose%
\pgfusepath{fill}%
\end{pgfscope}%
\begin{pgfscope}%
\pgfpathrectangle{\pgfqpoint{0.100000in}{0.100000in}}{\pgfqpoint{3.007045in}{1.925000in}}%
\pgfusepath{clip}%
\pgfsetbuttcap%
\pgfsetmiterjoin%
\definecolor{currentfill}{rgb}{0.093272,0.396878,0.673664}%
\pgfsetfillcolor{currentfill}%
\pgfsetlinewidth{0.000000pt}%
\definecolor{currentstroke}{rgb}{0.000000,0.000000,0.000000}%
\pgfsetstrokecolor{currentstroke}%
\pgfsetstrokeopacity{0.000000}%
\pgfsetdash{}{0pt}%
\pgfpathmoveto{\pgfqpoint{1.001047in}{0.482991in}}%
\pgfpathlineto{\pgfqpoint{0.989639in}{0.489920in}}%
\pgfpathlineto{\pgfqpoint{0.966890in}{0.503916in}}%
\pgfpathlineto{\pgfqpoint{0.959476in}{0.508253in}}%
\pgfpathlineto{\pgfqpoint{0.952820in}{0.498342in}}%
\pgfpathlineto{\pgfqpoint{0.944682in}{0.503522in}}%
\pgfpathlineto{\pgfqpoint{0.941118in}{0.505304in}}%
\pgfpathlineto{\pgfqpoint{0.919823in}{0.519872in}}%
\pgfpathlineto{\pgfqpoint{0.895892in}{0.537283in}}%
\pgfpathlineto{\pgfqpoint{0.877896in}{0.551221in}}%
\pgfpathlineto{\pgfqpoint{0.878340in}{0.551804in}}%
\pgfpathlineto{\pgfqpoint{0.871135in}{0.557489in}}%
\pgfpathlineto{\pgfqpoint{0.872509in}{0.559210in}}%
\pgfpathlineto{\pgfqpoint{0.865552in}{0.564907in}}%
\pgfpathlineto{\pgfqpoint{0.867050in}{0.566744in}}%
\pgfpathlineto{\pgfqpoint{0.859713in}{0.572782in}}%
\pgfpathlineto{\pgfqpoint{0.847780in}{0.582922in}}%
\pgfpathlineto{\pgfqpoint{0.848948in}{0.584275in}}%
\pgfpathlineto{\pgfqpoint{0.845795in}{0.587009in}}%
\pgfpathlineto{\pgfqpoint{0.844619in}{0.585660in}}%
\pgfpathlineto{\pgfqpoint{0.837473in}{0.591959in}}%
\pgfpathlineto{\pgfqpoint{0.838673in}{0.593306in}}%
\pgfpathlineto{\pgfqpoint{0.835607in}{0.596048in}}%
\pgfpathlineto{\pgfqpoint{0.834401in}{0.594707in}}%
\pgfpathlineto{\pgfqpoint{0.830895in}{0.597873in}}%
\pgfpathlineto{\pgfqpoint{0.832109in}{0.599208in}}%
\pgfpathlineto{\pgfqpoint{0.826925in}{0.603954in}}%
\pgfpathlineto{\pgfqpoint{0.825701in}{0.602628in}}%
\pgfpathlineto{\pgfqpoint{0.813031in}{0.614554in}}%
\pgfpathlineto{\pgfqpoint{0.809438in}{0.610810in}}%
\pgfpathlineto{\pgfqpoint{0.803267in}{0.616806in}}%
\pgfpathlineto{\pgfqpoint{0.800883in}{0.622716in}}%
\pgfpathlineto{\pgfqpoint{0.801608in}{0.625470in}}%
\pgfpathlineto{\pgfqpoint{0.801127in}{0.628309in}}%
\pgfpathlineto{\pgfqpoint{0.799100in}{0.632308in}}%
\pgfpathlineto{\pgfqpoint{0.801794in}{0.631253in}}%
\pgfpathlineto{\pgfqpoint{0.804644in}{0.631085in}}%
\pgfpathlineto{\pgfqpoint{0.806937in}{0.632414in}}%
\pgfpathlineto{\pgfqpoint{0.812022in}{0.637296in}}%
\pgfpathlineto{\pgfqpoint{0.813808in}{0.636711in}}%
\pgfpathlineto{\pgfqpoint{0.817714in}{0.632498in}}%
\pgfpathlineto{\pgfqpoint{0.826485in}{0.626672in}}%
\pgfpathlineto{\pgfqpoint{0.829143in}{0.625614in}}%
\pgfpathlineto{\pgfqpoint{0.834897in}{0.625269in}}%
\pgfpathlineto{\pgfqpoint{0.842826in}{0.626872in}}%
\pgfpathlineto{\pgfqpoint{0.846807in}{0.629874in}}%
\pgfpathlineto{\pgfqpoint{0.847784in}{0.631457in}}%
\pgfpathlineto{\pgfqpoint{0.850593in}{0.633815in}}%
\pgfpathlineto{\pgfqpoint{0.852477in}{0.634580in}}%
\pgfpathlineto{\pgfqpoint{0.858398in}{0.634527in}}%
\pgfpathlineto{\pgfqpoint{0.860204in}{0.636304in}}%
\pgfpathlineto{\pgfqpoint{0.864117in}{0.636616in}}%
\pgfpathlineto{\pgfqpoint{0.868789in}{0.637634in}}%
\pgfpathlineto{\pgfqpoint{0.871460in}{0.633851in}}%
\pgfpathlineto{\pgfqpoint{0.876444in}{0.632688in}}%
\pgfpathlineto{\pgfqpoint{0.885282in}{0.633242in}}%
\pgfpathlineto{\pgfqpoint{0.894306in}{0.634804in}}%
\pgfpathlineto{\pgfqpoint{0.897261in}{0.633456in}}%
\pgfpathlineto{\pgfqpoint{0.897619in}{0.631397in}}%
\pgfpathlineto{\pgfqpoint{0.901143in}{0.628563in}}%
\pgfpathlineto{\pgfqpoint{0.904509in}{0.626669in}}%
\pgfpathlineto{\pgfqpoint{0.907341in}{0.625778in}}%
\pgfpathlineto{\pgfqpoint{0.913294in}{0.626230in}}%
\pgfpathlineto{\pgfqpoint{0.921174in}{0.629150in}}%
\pgfpathlineto{\pgfqpoint{0.923879in}{0.629562in}}%
\pgfpathlineto{\pgfqpoint{0.925211in}{0.625453in}}%
\pgfpathlineto{\pgfqpoint{0.924492in}{0.622843in}}%
\pgfpathlineto{\pgfqpoint{0.926548in}{0.622888in}}%
\pgfpathlineto{\pgfqpoint{0.927653in}{0.620890in}}%
\pgfpathlineto{\pgfqpoint{0.927665in}{0.618526in}}%
\pgfpathlineto{\pgfqpoint{0.923257in}{0.616818in}}%
\pgfpathlineto{\pgfqpoint{0.919869in}{0.618044in}}%
\pgfpathlineto{\pgfqpoint{0.920400in}{0.614171in}}%
\pgfpathlineto{\pgfqpoint{0.923838in}{0.612830in}}%
\pgfpathlineto{\pgfqpoint{0.925840in}{0.614446in}}%
\pgfpathlineto{\pgfqpoint{0.930393in}{0.614357in}}%
\pgfpathlineto{\pgfqpoint{0.932785in}{0.611562in}}%
\pgfpathlineto{\pgfqpoint{0.930092in}{0.608495in}}%
\pgfpathlineto{\pgfqpoint{0.929589in}{0.606710in}}%
\pgfpathlineto{\pgfqpoint{0.931595in}{0.605231in}}%
\pgfpathlineto{\pgfqpoint{0.931935in}{0.603180in}}%
\pgfpathlineto{\pgfqpoint{0.934772in}{0.603947in}}%
\pgfpathlineto{\pgfqpoint{0.938741in}{0.601961in}}%
\pgfpathlineto{\pgfqpoint{0.941530in}{0.601138in}}%
\pgfpathlineto{\pgfqpoint{0.943304in}{0.596728in}}%
\pgfpathlineto{\pgfqpoint{0.945069in}{0.596940in}}%
\pgfpathlineto{\pgfqpoint{0.946594in}{0.594438in}}%
\pgfpathlineto{\pgfqpoint{0.941495in}{0.590893in}}%
\pgfpathlineto{\pgfqpoint{0.939975in}{0.589037in}}%
\pgfpathlineto{\pgfqpoint{0.945374in}{0.585331in}}%
\pgfpathlineto{\pgfqpoint{0.944737in}{0.583986in}}%
\pgfpathlineto{\pgfqpoint{0.942095in}{0.583582in}}%
\pgfpathlineto{\pgfqpoint{0.946113in}{0.580534in}}%
\pgfpathlineto{\pgfqpoint{0.946529in}{0.578018in}}%
\pgfpathlineto{\pgfqpoint{0.949099in}{0.579130in}}%
\pgfpathlineto{\pgfqpoint{0.953212in}{0.577928in}}%
\pgfpathlineto{\pgfqpoint{0.953739in}{0.576380in}}%
\pgfpathlineto{\pgfqpoint{0.951749in}{0.574813in}}%
\pgfpathlineto{\pgfqpoint{0.955961in}{0.573632in}}%
\pgfpathlineto{\pgfqpoint{0.958806in}{0.574118in}}%
\pgfpathlineto{\pgfqpoint{0.962206in}{0.570550in}}%
\pgfpathlineto{\pgfqpoint{0.963220in}{0.570510in}}%
\pgfpathlineto{\pgfqpoint{0.964026in}{0.566308in}}%
\pgfpathlineto{\pgfqpoint{0.966643in}{0.564927in}}%
\pgfpathlineto{\pgfqpoint{0.966745in}{0.562715in}}%
\pgfpathlineto{\pgfqpoint{0.968656in}{0.561617in}}%
\pgfpathlineto{\pgfqpoint{0.969071in}{0.557591in}}%
\pgfpathlineto{\pgfqpoint{0.974627in}{0.550898in}}%
\pgfpathlineto{\pgfqpoint{0.976869in}{0.550826in}}%
\pgfpathlineto{\pgfqpoint{0.980019in}{0.548774in}}%
\pgfpathlineto{\pgfqpoint{0.981988in}{0.546062in}}%
\pgfpathlineto{\pgfqpoint{0.984408in}{0.544629in}}%
\pgfpathlineto{\pgfqpoint{0.986456in}{0.538416in}}%
\pgfpathlineto{\pgfqpoint{0.988132in}{0.536134in}}%
\pgfpathlineto{\pgfqpoint{0.992596in}{0.535000in}}%
\pgfpathlineto{\pgfqpoint{0.998469in}{0.532993in}}%
\pgfpathlineto{\pgfqpoint{1.000842in}{0.532798in}}%
\pgfpathlineto{\pgfqpoint{1.003539in}{0.531122in}}%
\pgfpathlineto{\pgfqpoint{1.006445in}{0.526960in}}%
\pgfpathlineto{\pgfqpoint{1.008246in}{0.522245in}}%
\pgfpathlineto{\pgfqpoint{1.008217in}{0.520872in}}%
\pgfpathlineto{\pgfqpoint{1.010008in}{0.517358in}}%
\pgfpathlineto{\pgfqpoint{1.011929in}{0.514984in}}%
\pgfpathlineto{\pgfqpoint{1.012633in}{0.512658in}}%
\pgfpathlineto{\pgfqpoint{1.015157in}{0.509681in}}%
\pgfpathlineto{\pgfqpoint{1.014103in}{0.506544in}}%
\pgfpathlineto{\pgfqpoint{1.001047in}{0.482991in}}%
\pgfpathclose%
\pgfusepath{fill}%
\end{pgfscope}%
\begin{pgfscope}%
\pgfpathrectangle{\pgfqpoint{0.100000in}{0.100000in}}{\pgfqpoint{3.007045in}{1.925000in}}%
\pgfusepath{clip}%
\pgfsetbuttcap%
\pgfsetmiterjoin%
\definecolor{currentfill}{rgb}{0.541961,0.749527,0.865606}%
\pgfsetfillcolor{currentfill}%
\pgfsetlinewidth{0.000000pt}%
\definecolor{currentstroke}{rgb}{0.000000,0.000000,0.000000}%
\pgfsetstrokecolor{currentstroke}%
\pgfsetstrokeopacity{0.000000}%
\pgfsetdash{}{0pt}%
\pgfpathmoveto{\pgfqpoint{2.119654in}{1.040653in}}%
\pgfpathlineto{\pgfqpoint{2.116603in}{1.041678in}}%
\pgfpathlineto{\pgfqpoint{2.115921in}{1.053268in}}%
\pgfpathlineto{\pgfqpoint{2.100489in}{1.052130in}}%
\pgfpathlineto{\pgfqpoint{2.099179in}{1.072352in}}%
\pgfpathlineto{\pgfqpoint{2.116398in}{1.073683in}}%
\pgfpathlineto{\pgfqpoint{2.128732in}{1.074544in}}%
\pgfpathlineto{\pgfqpoint{2.131121in}{1.073626in}}%
\pgfpathlineto{\pgfqpoint{2.134908in}{1.067855in}}%
\pgfpathlineto{\pgfqpoint{2.128798in}{1.058421in}}%
\pgfpathlineto{\pgfqpoint{2.133456in}{1.046059in}}%
\pgfpathlineto{\pgfqpoint{2.123429in}{1.044085in}}%
\pgfpathlineto{\pgfqpoint{2.119654in}{1.040653in}}%
\pgfpathclose%
\pgfusepath{fill}%
\end{pgfscope}%
\begin{pgfscope}%
\pgfpathrectangle{\pgfqpoint{0.100000in}{0.100000in}}{\pgfqpoint{3.007045in}{1.925000in}}%
\pgfusepath{clip}%
\pgfsetbuttcap%
\pgfsetmiterjoin%
\definecolor{currentfill}{rgb}{0.376732,0.653072,0.822484}%
\pgfsetfillcolor{currentfill}%
\pgfsetlinewidth{0.000000pt}%
\definecolor{currentstroke}{rgb}{0.000000,0.000000,0.000000}%
\pgfsetstrokecolor{currentstroke}%
\pgfsetstrokeopacity{0.000000}%
\pgfsetdash{}{0pt}%
\pgfpathmoveto{\pgfqpoint{2.073753in}{0.803617in}}%
\pgfpathlineto{\pgfqpoint{2.059401in}{0.802761in}}%
\pgfpathlineto{\pgfqpoint{2.050685in}{0.804294in}}%
\pgfpathlineto{\pgfqpoint{2.050947in}{0.799400in}}%
\pgfpathlineto{\pgfqpoint{2.039467in}{0.797809in}}%
\pgfpathlineto{\pgfqpoint{2.039761in}{0.792018in}}%
\pgfpathlineto{\pgfqpoint{2.030424in}{0.792915in}}%
\pgfpathlineto{\pgfqpoint{2.036938in}{0.797675in}}%
\pgfpathlineto{\pgfqpoint{2.022165in}{0.797050in}}%
\pgfpathlineto{\pgfqpoint{2.021640in}{0.808564in}}%
\pgfpathlineto{\pgfqpoint{2.010153in}{0.808128in}}%
\pgfpathlineto{\pgfqpoint{2.009788in}{0.816800in}}%
\pgfpathlineto{\pgfqpoint{1.993574in}{0.816135in}}%
\pgfpathlineto{\pgfqpoint{2.003512in}{0.827202in}}%
\pgfpathlineto{\pgfqpoint{2.003733in}{0.832951in}}%
\pgfpathlineto{\pgfqpoint{2.013158in}{0.836265in}}%
\pgfpathlineto{\pgfqpoint{2.013407in}{0.843839in}}%
\pgfpathlineto{\pgfqpoint{2.022978in}{0.837629in}}%
\pgfpathlineto{\pgfqpoint{2.028291in}{0.837950in}}%
\pgfpathlineto{\pgfqpoint{2.027867in}{0.843557in}}%
\pgfpathlineto{\pgfqpoint{2.033456in}{0.846814in}}%
\pgfpathlineto{\pgfqpoint{2.072679in}{0.849028in}}%
\pgfpathlineto{\pgfqpoint{2.085442in}{0.845945in}}%
\pgfpathlineto{\pgfqpoint{2.085917in}{0.838289in}}%
\pgfpathlineto{\pgfqpoint{2.086876in}{0.823888in}}%
\pgfpathlineto{\pgfqpoint{2.072446in}{0.823053in}}%
\pgfpathlineto{\pgfqpoint{2.073753in}{0.803617in}}%
\pgfpathclose%
\pgfusepath{fill}%
\end{pgfscope}%
\begin{pgfscope}%
\pgfpathrectangle{\pgfqpoint{0.100000in}{0.100000in}}{\pgfqpoint{3.007045in}{1.925000in}}%
\pgfusepath{clip}%
\pgfsetbuttcap%
\pgfsetmiterjoin%
\definecolor{currentfill}{rgb}{0.726028,0.837386,0.919600}%
\pgfsetfillcolor{currentfill}%
\pgfsetlinewidth{0.000000pt}%
\definecolor{currentstroke}{rgb}{0.000000,0.000000,0.000000}%
\pgfsetstrokecolor{currentstroke}%
\pgfsetstrokeopacity{0.000000}%
\pgfsetdash{}{0pt}%
\pgfpathmoveto{\pgfqpoint{2.938382in}{1.591168in}}%
\pgfpathlineto{\pgfqpoint{2.930725in}{1.593936in}}%
\pgfpathlineto{\pgfqpoint{2.931682in}{1.596631in}}%
\pgfpathlineto{\pgfqpoint{2.926133in}{1.600139in}}%
\pgfpathlineto{\pgfqpoint{2.921331in}{1.597973in}}%
\pgfpathlineto{\pgfqpoint{2.920147in}{1.603400in}}%
\pgfpathlineto{\pgfqpoint{2.912256in}{1.602890in}}%
\pgfpathlineto{\pgfqpoint{2.904580in}{1.599504in}}%
\pgfpathlineto{\pgfqpoint{2.895122in}{1.630668in}}%
\pgfpathlineto{\pgfqpoint{2.885278in}{1.661827in}}%
\pgfpathlineto{\pgfqpoint{2.874149in}{1.694991in}}%
\pgfpathlineto{\pgfqpoint{2.879401in}{1.698668in}}%
\pgfpathlineto{\pgfqpoint{2.883573in}{1.693339in}}%
\pgfpathlineto{\pgfqpoint{2.886335in}{1.697801in}}%
\pgfpathlineto{\pgfqpoint{2.885576in}{1.706733in}}%
\pgfpathlineto{\pgfqpoint{2.889390in}{1.704990in}}%
\pgfpathlineto{\pgfqpoint{2.892641in}{1.707989in}}%
\pgfpathlineto{\pgfqpoint{2.886793in}{1.712832in}}%
\pgfpathlineto{\pgfqpoint{2.888624in}{1.719661in}}%
\pgfpathlineto{\pgfqpoint{2.897841in}{1.730562in}}%
\pgfpathlineto{\pgfqpoint{2.896524in}{1.736677in}}%
\pgfpathlineto{\pgfqpoint{2.900020in}{1.741049in}}%
\pgfpathlineto{\pgfqpoint{2.900699in}{1.746177in}}%
\pgfpathlineto{\pgfqpoint{2.895909in}{1.750061in}}%
\pgfpathlineto{\pgfqpoint{2.897245in}{1.759098in}}%
\pgfpathlineto{\pgfqpoint{2.893988in}{1.761345in}}%
\pgfpathlineto{\pgfqpoint{2.895208in}{1.771201in}}%
\pgfpathlineto{\pgfqpoint{2.900371in}{1.778621in}}%
\pgfpathlineto{\pgfqpoint{2.898990in}{1.789167in}}%
\pgfpathlineto{\pgfqpoint{2.912260in}{1.792925in}}%
\pgfpathlineto{\pgfqpoint{2.915056in}{1.781252in}}%
\pgfpathlineto{\pgfqpoint{2.922774in}{1.754622in}}%
\pgfpathlineto{\pgfqpoint{2.926062in}{1.747145in}}%
\pgfpathlineto{\pgfqpoint{2.925595in}{1.735765in}}%
\pgfpathlineto{\pgfqpoint{2.930414in}{1.732814in}}%
\pgfpathlineto{\pgfqpoint{2.928393in}{1.726358in}}%
\pgfpathlineto{\pgfqpoint{2.945232in}{1.694510in}}%
\pgfpathlineto{\pgfqpoint{2.956012in}{1.701938in}}%
\pgfpathlineto{\pgfqpoint{2.966515in}{1.680747in}}%
\pgfpathlineto{\pgfqpoint{2.960139in}{1.680504in}}%
\pgfpathlineto{\pgfqpoint{2.962312in}{1.673903in}}%
\pgfpathlineto{\pgfqpoint{2.963183in}{1.661995in}}%
\pgfpathlineto{\pgfqpoint{2.962617in}{1.653501in}}%
\pgfpathlineto{\pgfqpoint{2.965671in}{1.652396in}}%
\pgfpathlineto{\pgfqpoint{2.969629in}{1.646428in}}%
\pgfpathlineto{\pgfqpoint{2.972759in}{1.647244in}}%
\pgfpathlineto{\pgfqpoint{2.978270in}{1.638556in}}%
\pgfpathlineto{\pgfqpoint{2.975595in}{1.630839in}}%
\pgfpathlineto{\pgfqpoint{2.972528in}{1.630382in}}%
\pgfpathlineto{\pgfqpoint{2.972069in}{1.622078in}}%
\pgfpathlineto{\pgfqpoint{2.963086in}{1.616945in}}%
\pgfpathlineto{\pgfqpoint{2.962287in}{1.612214in}}%
\pgfpathlineto{\pgfqpoint{2.958299in}{1.609229in}}%
\pgfpathlineto{\pgfqpoint{2.947252in}{1.614168in}}%
\pgfpathlineto{\pgfqpoint{2.941831in}{1.608526in}}%
\pgfpathlineto{\pgfqpoint{2.940412in}{1.601930in}}%
\pgfpathlineto{\pgfqpoint{2.945057in}{1.595018in}}%
\pgfpathlineto{\pgfqpoint{2.938382in}{1.591168in}}%
\pgfpathclose%
\pgfusepath{fill}%
\end{pgfscope}%
\begin{pgfscope}%
\pgfpathrectangle{\pgfqpoint{0.100000in}{0.100000in}}{\pgfqpoint{3.007045in}{1.925000in}}%
\pgfusepath{clip}%
\pgfsetbuttcap%
\pgfsetmiterjoin%
\definecolor{currentfill}{rgb}{0.498039,0.725413,0.856132}%
\pgfsetfillcolor{currentfill}%
\pgfsetlinewidth{0.000000pt}%
\definecolor{currentstroke}{rgb}{0.000000,0.000000,0.000000}%
\pgfsetstrokecolor{currentstroke}%
\pgfsetstrokeopacity{0.000000}%
\pgfsetdash{}{0pt}%
\pgfpathmoveto{\pgfqpoint{2.116053in}{0.743899in}}%
\pgfpathlineto{\pgfqpoint{2.087936in}{0.742436in}}%
\pgfpathlineto{\pgfqpoint{2.086624in}{0.765838in}}%
\pgfpathlineto{\pgfqpoint{2.079034in}{0.765390in}}%
\pgfpathlineto{\pgfqpoint{2.078426in}{0.775122in}}%
\pgfpathlineto{\pgfqpoint{2.075429in}{0.776700in}}%
\pgfpathlineto{\pgfqpoint{2.074344in}{0.794001in}}%
\pgfpathlineto{\pgfqpoint{2.091654in}{0.795102in}}%
\pgfpathlineto{\pgfqpoint{2.105804in}{0.798159in}}%
\pgfpathlineto{\pgfqpoint{2.105570in}{0.801993in}}%
\pgfpathlineto{\pgfqpoint{2.117152in}{0.802691in}}%
\pgfpathlineto{\pgfqpoint{2.117772in}{0.794063in}}%
\pgfpathlineto{\pgfqpoint{2.129211in}{0.792870in}}%
\pgfpathlineto{\pgfqpoint{2.131570in}{0.789934in}}%
\pgfpathlineto{\pgfqpoint{2.128772in}{0.785353in}}%
\pgfpathlineto{\pgfqpoint{2.121230in}{0.782758in}}%
\pgfpathlineto{\pgfqpoint{2.122360in}{0.768291in}}%
\pgfpathlineto{\pgfqpoint{2.114414in}{0.767709in}}%
\pgfpathlineto{\pgfqpoint{2.116053in}{0.743899in}}%
\pgfpathclose%
\pgfusepath{fill}%
\end{pgfscope}%
\begin{pgfscope}%
\pgfpathrectangle{\pgfqpoint{0.100000in}{0.100000in}}{\pgfqpoint{3.007045in}{1.925000in}}%
\pgfusepath{clip}%
\pgfsetbuttcap%
\pgfsetmiterjoin%
\definecolor{currentfill}{rgb}{0.567059,0.763306,0.871019}%
\pgfsetfillcolor{currentfill}%
\pgfsetlinewidth{0.000000pt}%
\definecolor{currentstroke}{rgb}{0.000000,0.000000,0.000000}%
\pgfsetstrokecolor{currentstroke}%
\pgfsetstrokeopacity{0.000000}%
\pgfsetdash{}{0pt}%
\pgfpathmoveto{\pgfqpoint{0.536189in}{1.704696in}}%
\pgfpathlineto{\pgfqpoint{0.506524in}{1.713649in}}%
\pgfpathlineto{\pgfqpoint{0.501234in}{1.718853in}}%
\pgfpathlineto{\pgfqpoint{0.500179in}{1.724897in}}%
\pgfpathlineto{\pgfqpoint{0.496057in}{1.728942in}}%
\pgfpathlineto{\pgfqpoint{0.493784in}{1.734570in}}%
\pgfpathlineto{\pgfqpoint{0.496421in}{1.737333in}}%
\pgfpathlineto{\pgfqpoint{0.496816in}{1.743625in}}%
\pgfpathlineto{\pgfqpoint{0.490020in}{1.750804in}}%
\pgfpathlineto{\pgfqpoint{0.488462in}{1.755366in}}%
\pgfpathlineto{\pgfqpoint{0.484457in}{1.758626in}}%
\pgfpathlineto{\pgfqpoint{0.469874in}{1.763357in}}%
\pgfpathlineto{\pgfqpoint{0.475014in}{1.769793in}}%
\pgfpathlineto{\pgfqpoint{0.475846in}{1.775523in}}%
\pgfpathlineto{\pgfqpoint{0.480074in}{1.779212in}}%
\pgfpathlineto{\pgfqpoint{0.481516in}{1.783693in}}%
\pgfpathlineto{\pgfqpoint{0.488870in}{1.806888in}}%
\pgfpathlineto{\pgfqpoint{0.498370in}{1.806808in}}%
\pgfpathlineto{\pgfqpoint{0.507744in}{1.796488in}}%
\pgfpathlineto{\pgfqpoint{0.510888in}{1.782623in}}%
\pgfpathlineto{\pgfqpoint{0.524572in}{1.784544in}}%
\pgfpathlineto{\pgfqpoint{0.529311in}{1.780896in}}%
\pgfpathlineto{\pgfqpoint{0.536092in}{1.785540in}}%
\pgfpathlineto{\pgfqpoint{0.542761in}{1.806508in}}%
\pgfpathlineto{\pgfqpoint{0.574289in}{1.797104in}}%
\pgfpathlineto{\pgfqpoint{0.567701in}{1.775221in}}%
\pgfpathlineto{\pgfqpoint{0.563789in}{1.776383in}}%
\pgfpathlineto{\pgfqpoint{0.558944in}{1.759738in}}%
\pgfpathlineto{\pgfqpoint{0.561232in}{1.755167in}}%
\pgfpathlineto{\pgfqpoint{0.551653in}{1.755806in}}%
\pgfpathlineto{\pgfqpoint{0.545786in}{1.757536in}}%
\pgfpathlineto{\pgfqpoint{0.542490in}{1.755480in}}%
\pgfpathlineto{\pgfqpoint{0.540688in}{1.747112in}}%
\pgfpathlineto{\pgfqpoint{0.547023in}{1.735645in}}%
\pgfpathlineto{\pgfqpoint{0.545552in}{1.726930in}}%
\pgfpathlineto{\pgfqpoint{0.541409in}{1.723741in}}%
\pgfpathlineto{\pgfqpoint{0.541696in}{1.715411in}}%
\pgfpathlineto{\pgfqpoint{0.535329in}{1.713270in}}%
\pgfpathlineto{\pgfqpoint{0.536189in}{1.704696in}}%
\pgfpathclose%
\pgfusepath{fill}%
\end{pgfscope}%
\begin{pgfscope}%
\pgfpathrectangle{\pgfqpoint{0.100000in}{0.100000in}}{\pgfqpoint{3.007045in}{1.925000in}}%
\pgfusepath{clip}%
\pgfsetbuttcap%
\pgfsetmiterjoin%
\definecolor{currentfill}{rgb}{0.627605,0.795556,0.885152}%
\pgfsetfillcolor{currentfill}%
\pgfsetlinewidth{0.000000pt}%
\definecolor{currentstroke}{rgb}{0.000000,0.000000,0.000000}%
\pgfsetstrokecolor{currentstroke}%
\pgfsetstrokeopacity{0.000000}%
\pgfsetdash{}{0pt}%
\pgfpathmoveto{\pgfqpoint{2.539586in}{0.435076in}}%
\pgfpathlineto{\pgfqpoint{2.510105in}{0.430828in}}%
\pgfpathlineto{\pgfqpoint{2.515004in}{0.442453in}}%
\pgfpathlineto{\pgfqpoint{2.510239in}{0.443614in}}%
\pgfpathlineto{\pgfqpoint{2.504846in}{0.439692in}}%
\pgfpathlineto{\pgfqpoint{2.503926in}{0.433696in}}%
\pgfpathlineto{\pgfqpoint{2.500319in}{0.432366in}}%
\pgfpathlineto{\pgfqpoint{2.490587in}{0.443682in}}%
\pgfpathlineto{\pgfqpoint{2.492902in}{0.456197in}}%
\pgfpathlineto{\pgfqpoint{2.490893in}{0.464809in}}%
\pgfpathlineto{\pgfqpoint{2.493043in}{0.468035in}}%
\pgfpathlineto{\pgfqpoint{2.496096in}{0.492116in}}%
\pgfpathlineto{\pgfqpoint{2.495655in}{0.499863in}}%
\pgfpathlineto{\pgfqpoint{2.508454in}{0.501648in}}%
\pgfpathlineto{\pgfqpoint{2.517736in}{0.501110in}}%
\pgfpathlineto{\pgfqpoint{2.521853in}{0.495228in}}%
\pgfpathlineto{\pgfqpoint{2.531241in}{0.493155in}}%
\pgfpathlineto{\pgfqpoint{2.533154in}{0.479245in}}%
\pgfpathlineto{\pgfqpoint{2.530722in}{0.475320in}}%
\pgfpathlineto{\pgfqpoint{2.534472in}{0.469934in}}%
\pgfpathlineto{\pgfqpoint{2.539586in}{0.435076in}}%
\pgfpathclose%
\pgfusepath{fill}%
\end{pgfscope}%
\begin{pgfscope}%
\pgfpathrectangle{\pgfqpoint{0.100000in}{0.100000in}}{\pgfqpoint{3.007045in}{1.925000in}}%
\pgfusepath{clip}%
\pgfsetbuttcap%
\pgfsetmiterjoin%
\definecolor{currentfill}{rgb}{0.701423,0.826928,0.910988}%
\pgfsetfillcolor{currentfill}%
\pgfsetlinewidth{0.000000pt}%
\definecolor{currentstroke}{rgb}{0.000000,0.000000,0.000000}%
\pgfsetstrokecolor{currentstroke}%
\pgfsetstrokeopacity{0.000000}%
\pgfsetdash{}{0pt}%
\pgfpathmoveto{\pgfqpoint{2.867151in}{1.544262in}}%
\pgfpathlineto{\pgfqpoint{2.867535in}{1.548390in}}%
\pgfpathlineto{\pgfqpoint{2.861539in}{1.556030in}}%
\pgfpathlineto{\pgfqpoint{2.863308in}{1.564293in}}%
\pgfpathlineto{\pgfqpoint{2.867917in}{1.577405in}}%
\pgfpathlineto{\pgfqpoint{2.870513in}{1.573950in}}%
\pgfpathlineto{\pgfqpoint{2.876946in}{1.580426in}}%
\pgfpathlineto{\pgfqpoint{2.875977in}{1.584608in}}%
\pgfpathlineto{\pgfqpoint{2.881617in}{1.586847in}}%
\pgfpathlineto{\pgfqpoint{2.876355in}{1.598852in}}%
\pgfpathlineto{\pgfqpoint{2.885308in}{1.602759in}}%
\pgfpathlineto{\pgfqpoint{2.882477in}{1.612503in}}%
\pgfpathlineto{\pgfqpoint{2.877796in}{1.621050in}}%
\pgfpathlineto{\pgfqpoint{2.884101in}{1.617605in}}%
\pgfpathlineto{\pgfqpoint{2.884731in}{1.625099in}}%
\pgfpathlineto{\pgfqpoint{2.894447in}{1.627269in}}%
\pgfpathlineto{\pgfqpoint{2.895122in}{1.630668in}}%
\pgfpathlineto{\pgfqpoint{2.904580in}{1.599504in}}%
\pgfpathlineto{\pgfqpoint{2.912256in}{1.602890in}}%
\pgfpathlineto{\pgfqpoint{2.920147in}{1.603400in}}%
\pgfpathlineto{\pgfqpoint{2.921331in}{1.597973in}}%
\pgfpathlineto{\pgfqpoint{2.926133in}{1.600139in}}%
\pgfpathlineto{\pgfqpoint{2.931682in}{1.596631in}}%
\pgfpathlineto{\pgfqpoint{2.930725in}{1.593936in}}%
\pgfpathlineto{\pgfqpoint{2.938382in}{1.591168in}}%
\pgfpathlineto{\pgfqpoint{2.939431in}{1.584564in}}%
\pgfpathlineto{\pgfqpoint{2.937676in}{1.578867in}}%
\pgfpathlineto{\pgfqpoint{2.933280in}{1.576054in}}%
\pgfpathlineto{\pgfqpoint{2.932692in}{1.561193in}}%
\pgfpathlineto{\pgfqpoint{2.928157in}{1.545679in}}%
\pgfpathlineto{\pgfqpoint{2.928476in}{1.542999in}}%
\pgfpathlineto{\pgfqpoint{2.922960in}{1.542355in}}%
\pgfpathlineto{\pgfqpoint{2.914625in}{1.535783in}}%
\pgfpathlineto{\pgfqpoint{2.909180in}{1.525824in}}%
\pgfpathlineto{\pgfqpoint{2.879160in}{1.518963in}}%
\pgfpathlineto{\pgfqpoint{2.875745in}{1.523338in}}%
\pgfpathlineto{\pgfqpoint{2.871705in}{1.532829in}}%
\pgfpathlineto{\pgfqpoint{2.869055in}{1.532209in}}%
\pgfpathlineto{\pgfqpoint{2.867151in}{1.544262in}}%
\pgfpathclose%
\pgfusepath{fill}%
\end{pgfscope}%
\begin{pgfscope}%
\pgfpathrectangle{\pgfqpoint{0.100000in}{0.100000in}}{\pgfqpoint{3.007045in}{1.925000in}}%
\pgfusepath{clip}%
\pgfsetbuttcap%
\pgfsetmiterjoin%
\definecolor{currentfill}{rgb}{0.460392,0.704744,0.848012}%
\pgfsetfillcolor{currentfill}%
\pgfsetlinewidth{0.000000pt}%
\definecolor{currentstroke}{rgb}{0.000000,0.000000,0.000000}%
\pgfsetstrokecolor{currentstroke}%
\pgfsetstrokeopacity{0.000000}%
\pgfsetdash{}{0pt}%
\pgfpathmoveto{\pgfqpoint{2.082341in}{0.997150in}}%
\pgfpathlineto{\pgfqpoint{2.081317in}{1.001182in}}%
\pgfpathlineto{\pgfqpoint{2.082880in}{1.008816in}}%
\pgfpathlineto{\pgfqpoint{2.081022in}{1.010070in}}%
\pgfpathlineto{\pgfqpoint{2.078167in}{1.013748in}}%
\pgfpathlineto{\pgfqpoint{2.085427in}{1.025478in}}%
\pgfpathlineto{\pgfqpoint{2.090698in}{1.026843in}}%
\pgfpathlineto{\pgfqpoint{2.091233in}{1.034083in}}%
\pgfpathlineto{\pgfqpoint{2.101574in}{1.034888in}}%
\pgfpathlineto{\pgfqpoint{2.100489in}{1.052130in}}%
\pgfpathlineto{\pgfqpoint{2.115921in}{1.053268in}}%
\pgfpathlineto{\pgfqpoint{2.116603in}{1.041678in}}%
\pgfpathlineto{\pgfqpoint{2.119654in}{1.040653in}}%
\pgfpathlineto{\pgfqpoint{2.127608in}{1.032537in}}%
\pgfpathlineto{\pgfqpoint{2.129826in}{1.024933in}}%
\pgfpathlineto{\pgfqpoint{2.127814in}{1.020206in}}%
\pgfpathlineto{\pgfqpoint{2.132817in}{1.006098in}}%
\pgfpathlineto{\pgfqpoint{2.135867in}{0.998378in}}%
\pgfpathlineto{\pgfqpoint{2.116101in}{0.997143in}}%
\pgfpathlineto{\pgfqpoint{2.115261in}{1.009765in}}%
\pgfpathlineto{\pgfqpoint{2.097982in}{1.008842in}}%
\pgfpathlineto{\pgfqpoint{2.098746in}{0.997437in}}%
\pgfpathlineto{\pgfqpoint{2.082341in}{0.997150in}}%
\pgfpathclose%
\pgfusepath{fill}%
\end{pgfscope}%
\begin{pgfscope}%
\pgfpathrectangle{\pgfqpoint{0.100000in}{0.100000in}}{\pgfqpoint{3.007045in}{1.925000in}}%
\pgfusepath{clip}%
\pgfsetbuttcap%
\pgfsetmiterjoin%
\definecolor{currentfill}{rgb}{0.541961,0.749527,0.865606}%
\pgfsetfillcolor{currentfill}%
\pgfsetlinewidth{0.000000pt}%
\definecolor{currentstroke}{rgb}{0.000000,0.000000,0.000000}%
\pgfsetstrokecolor{currentstroke}%
\pgfsetstrokeopacity{0.000000}%
\pgfsetdash{}{0pt}%
\pgfpathmoveto{\pgfqpoint{2.654171in}{1.354581in}}%
\pgfpathlineto{\pgfqpoint{2.645166in}{1.349038in}}%
\pgfpathlineto{\pgfqpoint{2.637385in}{1.345841in}}%
\pgfpathlineto{\pgfqpoint{2.627962in}{1.353837in}}%
\pgfpathlineto{\pgfqpoint{2.623988in}{1.352894in}}%
\pgfpathlineto{\pgfqpoint{2.621003in}{1.357074in}}%
\pgfpathlineto{\pgfqpoint{2.609177in}{1.353242in}}%
\pgfpathlineto{\pgfqpoint{2.605715in}{1.356835in}}%
\pgfpathlineto{\pgfqpoint{2.608964in}{1.367546in}}%
\pgfpathlineto{\pgfqpoint{2.607627in}{1.374496in}}%
\pgfpathlineto{\pgfqpoint{2.626550in}{1.378374in}}%
\pgfpathlineto{\pgfqpoint{2.625680in}{1.382482in}}%
\pgfpathlineto{\pgfqpoint{2.656505in}{1.389182in}}%
\pgfpathlineto{\pgfqpoint{2.660061in}{1.392938in}}%
\pgfpathlineto{\pgfqpoint{2.652249in}{1.418613in}}%
\pgfpathlineto{\pgfqpoint{2.669988in}{1.422124in}}%
\pgfpathlineto{\pgfqpoint{2.689741in}{1.426141in}}%
\pgfpathlineto{\pgfqpoint{2.695962in}{1.404095in}}%
\pgfpathlineto{\pgfqpoint{2.691983in}{1.402891in}}%
\pgfpathlineto{\pgfqpoint{2.692396in}{1.393752in}}%
\pgfpathlineto{\pgfqpoint{2.691171in}{1.379954in}}%
\pgfpathlineto{\pgfqpoint{2.686454in}{1.378941in}}%
\pgfpathlineto{\pgfqpoint{2.679506in}{1.367379in}}%
\pgfpathlineto{\pgfqpoint{2.677192in}{1.366767in}}%
\pgfpathlineto{\pgfqpoint{2.669738in}{1.366595in}}%
\pgfpathlineto{\pgfqpoint{2.661755in}{1.363168in}}%
\pgfpathlineto{\pgfqpoint{2.661650in}{1.359044in}}%
\pgfpathlineto{\pgfqpoint{2.654171in}{1.354581in}}%
\pgfpathclose%
\pgfusepath{fill}%
\end{pgfscope}%
\begin{pgfscope}%
\pgfpathrectangle{\pgfqpoint{0.100000in}{0.100000in}}{\pgfqpoint{3.007045in}{1.925000in}}%
\pgfusepath{clip}%
\pgfsetbuttcap%
\pgfsetmiterjoin%
\definecolor{currentfill}{rgb}{0.523137,0.739193,0.861546}%
\pgfsetfillcolor{currentfill}%
\pgfsetlinewidth{0.000000pt}%
\definecolor{currentstroke}{rgb}{0.000000,0.000000,0.000000}%
\pgfsetstrokecolor{currentstroke}%
\pgfsetstrokeopacity{0.000000}%
\pgfsetdash{}{0pt}%
\pgfpathmoveto{\pgfqpoint{2.245494in}{0.625658in}}%
\pgfpathlineto{\pgfqpoint{2.193761in}{0.621372in}}%
\pgfpathlineto{\pgfqpoint{2.191464in}{0.637662in}}%
\pgfpathlineto{\pgfqpoint{2.185584in}{0.641031in}}%
\pgfpathlineto{\pgfqpoint{2.184550in}{0.645723in}}%
\pgfpathlineto{\pgfqpoint{2.187693in}{0.649271in}}%
\pgfpathlineto{\pgfqpoint{2.192871in}{0.650628in}}%
\pgfpathlineto{\pgfqpoint{2.191786in}{0.667860in}}%
\pgfpathlineto{\pgfqpoint{2.194637in}{0.668142in}}%
\pgfpathlineto{\pgfqpoint{2.194774in}{0.676909in}}%
\pgfpathlineto{\pgfqpoint{2.228144in}{0.679871in}}%
\pgfpathlineto{\pgfqpoint{2.229320in}{0.666780in}}%
\pgfpathlineto{\pgfqpoint{2.233755in}{0.660011in}}%
\pgfpathlineto{\pgfqpoint{2.241536in}{0.660624in}}%
\pgfpathlineto{\pgfqpoint{2.244783in}{0.625988in}}%
\pgfpathlineto{\pgfqpoint{2.245494in}{0.625658in}}%
\pgfpathclose%
\pgfusepath{fill}%
\end{pgfscope}%
\begin{pgfscope}%
\pgfpathrectangle{\pgfqpoint{0.100000in}{0.100000in}}{\pgfqpoint{3.007045in}{1.925000in}}%
\pgfusepath{clip}%
\pgfsetbuttcap%
\pgfsetmiterjoin%
\definecolor{currentfill}{rgb}{0.485490,0.718524,0.853426}%
\pgfsetfillcolor{currentfill}%
\pgfsetlinewidth{0.000000pt}%
\definecolor{currentstroke}{rgb}{0.000000,0.000000,0.000000}%
\pgfsetstrokecolor{currentstroke}%
\pgfsetstrokeopacity{0.000000}%
\pgfsetdash{}{0pt}%
\pgfpathmoveto{\pgfqpoint{1.054598in}{0.801868in}}%
\pgfpathlineto{\pgfqpoint{0.997334in}{0.810949in}}%
\pgfpathlineto{\pgfqpoint{1.003150in}{0.848168in}}%
\pgfpathlineto{\pgfqpoint{1.011345in}{0.900484in}}%
\pgfpathlineto{\pgfqpoint{1.047397in}{0.894955in}}%
\pgfpathlineto{\pgfqpoint{1.082425in}{0.889855in}}%
\pgfpathlineto{\pgfqpoint{1.127835in}{0.883763in}}%
\pgfpathlineto{\pgfqpoint{1.133587in}{0.879798in}}%
\pgfpathlineto{\pgfqpoint{1.151795in}{0.871370in}}%
\pgfpathlineto{\pgfqpoint{1.150364in}{0.859555in}}%
\pgfpathlineto{\pgfqpoint{1.177035in}{0.856282in}}%
\pgfpathlineto{\pgfqpoint{1.173550in}{0.827766in}}%
\pgfpathlineto{\pgfqpoint{1.166565in}{0.828636in}}%
\pgfpathlineto{\pgfqpoint{1.165157in}{0.817109in}}%
\pgfpathlineto{\pgfqpoint{1.147690in}{0.819949in}}%
\pgfpathlineto{\pgfqpoint{1.146247in}{0.808107in}}%
\pgfpathlineto{\pgfqpoint{1.115027in}{0.811851in}}%
\pgfpathlineto{\pgfqpoint{1.060540in}{0.819258in}}%
\pgfpathlineto{\pgfqpoint{1.057206in}{0.819980in}}%
\pgfpathlineto{\pgfqpoint{1.054598in}{0.801868in}}%
\pgfpathclose%
\pgfusepath{fill}%
\end{pgfscope}%
\begin{pgfscope}%
\pgfpathrectangle{\pgfqpoint{0.100000in}{0.100000in}}{\pgfqpoint{3.007045in}{1.925000in}}%
\pgfusepath{clip}%
\pgfsetbuttcap%
\pgfsetmiterjoin%
\definecolor{currentfill}{rgb}{0.579608,0.770196,0.873725}%
\pgfsetfillcolor{currentfill}%
\pgfsetlinewidth{0.000000pt}%
\definecolor{currentstroke}{rgb}{0.000000,0.000000,0.000000}%
\pgfsetstrokecolor{currentstroke}%
\pgfsetstrokeopacity{0.000000}%
\pgfsetdash{}{0pt}%
\pgfpathmoveto{\pgfqpoint{2.246778in}{1.453408in}}%
\pgfpathlineto{\pgfqpoint{2.224008in}{1.451122in}}%
\pgfpathlineto{\pgfqpoint{2.221626in}{1.473931in}}%
\pgfpathlineto{\pgfqpoint{2.199018in}{1.471645in}}%
\pgfpathlineto{\pgfqpoint{2.196560in}{1.494814in}}%
\pgfpathlineto{\pgfqpoint{2.207015in}{1.495613in}}%
\pgfpathlineto{\pgfqpoint{2.204865in}{1.518579in}}%
\pgfpathlineto{\pgfqpoint{2.250233in}{1.523286in}}%
\pgfpathlineto{\pgfqpoint{2.252766in}{1.500298in}}%
\pgfpathlineto{\pgfqpoint{2.241654in}{1.499264in}}%
\pgfpathlineto{\pgfqpoint{2.246778in}{1.453408in}}%
\pgfpathclose%
\pgfusepath{fill}%
\end{pgfscope}%
\begin{pgfscope}%
\pgfpathrectangle{\pgfqpoint{0.100000in}{0.100000in}}{\pgfqpoint{3.007045in}{1.925000in}}%
\pgfusepath{clip}%
\pgfsetbuttcap%
\pgfsetmiterjoin%
\definecolor{currentfill}{rgb}{0.381776,0.656517,0.824452}%
\pgfsetfillcolor{currentfill}%
\pgfsetlinewidth{0.000000pt}%
\definecolor{currentstroke}{rgb}{0.000000,0.000000,0.000000}%
\pgfsetstrokecolor{currentstroke}%
\pgfsetstrokeopacity{0.000000}%
\pgfsetdash{}{0pt}%
\pgfpathmoveto{\pgfqpoint{2.163243in}{0.916924in}}%
\pgfpathlineto{\pgfqpoint{2.166375in}{0.924298in}}%
\pgfpathlineto{\pgfqpoint{2.160890in}{0.928692in}}%
\pgfpathlineto{\pgfqpoint{2.161409in}{0.940303in}}%
\pgfpathlineto{\pgfqpoint{2.161575in}{0.944740in}}%
\pgfpathlineto{\pgfqpoint{2.170276in}{0.951262in}}%
\pgfpathlineto{\pgfqpoint{2.188043in}{0.950490in}}%
\pgfpathlineto{\pgfqpoint{2.188078in}{0.943307in}}%
\pgfpathlineto{\pgfqpoint{2.210327in}{0.936644in}}%
\pgfpathlineto{\pgfqpoint{2.219736in}{0.935214in}}%
\pgfpathlineto{\pgfqpoint{2.220248in}{0.918653in}}%
\pgfpathlineto{\pgfqpoint{2.223823in}{0.914321in}}%
\pgfpathlineto{\pgfqpoint{2.223886in}{0.911812in}}%
\pgfpathlineto{\pgfqpoint{2.216592in}{0.906228in}}%
\pgfpathlineto{\pgfqpoint{2.212159in}{0.906542in}}%
\pgfpathlineto{\pgfqpoint{2.213336in}{0.888623in}}%
\pgfpathlineto{\pgfqpoint{2.193121in}{0.887325in}}%
\pgfpathlineto{\pgfqpoint{2.192088in}{0.890541in}}%
\pgfpathlineto{\pgfqpoint{2.190733in}{0.912451in}}%
\pgfpathlineto{\pgfqpoint{2.189759in}{0.915550in}}%
\pgfpathlineto{\pgfqpoint{2.184378in}{0.916884in}}%
\pgfpathlineto{\pgfqpoint{2.171317in}{0.911958in}}%
\pgfpathlineto{\pgfqpoint{2.167138in}{0.916537in}}%
\pgfpathlineto{\pgfqpoint{2.163243in}{0.916924in}}%
\pgfpathclose%
\pgfusepath{fill}%
\end{pgfscope}%
\begin{pgfscope}%
\pgfpathrectangle{\pgfqpoint{0.100000in}{0.100000in}}{\pgfqpoint{3.007045in}{1.925000in}}%
\pgfusepath{clip}%
\pgfsetbuttcap%
\pgfsetmiterjoin%
\definecolor{currentfill}{rgb}{0.604706,0.783975,0.879139}%
\pgfsetfillcolor{currentfill}%
\pgfsetlinewidth{0.000000pt}%
\definecolor{currentstroke}{rgb}{0.000000,0.000000,0.000000}%
\pgfsetstrokecolor{currentstroke}%
\pgfsetstrokeopacity{0.000000}%
\pgfsetdash{}{0pt}%
\pgfpathmoveto{\pgfqpoint{2.525737in}{1.174087in}}%
\pgfpathlineto{\pgfqpoint{2.532425in}{1.163870in}}%
\pgfpathlineto{\pgfqpoint{2.530752in}{1.154455in}}%
\pgfpathlineto{\pgfqpoint{2.526995in}{1.150325in}}%
\pgfpathlineto{\pgfqpoint{2.526634in}{1.142930in}}%
\pgfpathlineto{\pgfqpoint{2.522292in}{1.144995in}}%
\pgfpathlineto{\pgfqpoint{2.510167in}{1.130498in}}%
\pgfpathlineto{\pgfqpoint{2.501225in}{1.130708in}}%
\pgfpathlineto{\pgfqpoint{2.494710in}{1.137649in}}%
\pgfpathlineto{\pgfqpoint{2.490413in}{1.135206in}}%
\pgfpathlineto{\pgfqpoint{2.481701in}{1.138352in}}%
\pgfpathlineto{\pgfqpoint{2.496539in}{1.150737in}}%
\pgfpathlineto{\pgfqpoint{2.497293in}{1.156805in}}%
\pgfpathlineto{\pgfqpoint{2.494103in}{1.161327in}}%
\pgfpathlineto{\pgfqpoint{2.487851in}{1.166255in}}%
\pgfpathlineto{\pgfqpoint{2.489719in}{1.169881in}}%
\pgfpathlineto{\pgfqpoint{2.485316in}{1.174226in}}%
\pgfpathlineto{\pgfqpoint{2.487519in}{1.178106in}}%
\pgfpathlineto{\pgfqpoint{2.484306in}{1.188484in}}%
\pgfpathlineto{\pgfqpoint{2.489753in}{1.191584in}}%
\pgfpathlineto{\pgfqpoint{2.494442in}{1.196721in}}%
\pgfpathlineto{\pgfqpoint{2.498857in}{1.196503in}}%
\pgfpathlineto{\pgfqpoint{2.500337in}{1.190919in}}%
\pgfpathlineto{\pgfqpoint{2.507565in}{1.184564in}}%
\pgfpathlineto{\pgfqpoint{2.514455in}{1.181489in}}%
\pgfpathlineto{\pgfqpoint{2.517598in}{1.175060in}}%
\pgfpathlineto{\pgfqpoint{2.525737in}{1.174087in}}%
\pgfpathclose%
\pgfusepath{fill}%
\end{pgfscope}%
\begin{pgfscope}%
\pgfpathrectangle{\pgfqpoint{0.100000in}{0.100000in}}{\pgfqpoint{3.007045in}{1.925000in}}%
\pgfusepath{clip}%
\pgfsetbuttcap%
\pgfsetmiterjoin%
\definecolor{currentfill}{rgb}{0.231926,0.545652,0.762614}%
\pgfsetfillcolor{currentfill}%
\pgfsetlinewidth{0.000000pt}%
\definecolor{currentstroke}{rgb}{0.000000,0.000000,0.000000}%
\pgfsetstrokecolor{currentstroke}%
\pgfsetstrokeopacity{0.000000}%
\pgfsetdash{}{0pt}%
\pgfpathmoveto{\pgfqpoint{0.836260in}{1.093783in}}%
\pgfpathlineto{\pgfqpoint{0.776912in}{1.106019in}}%
\pgfpathlineto{\pgfqpoint{0.788118in}{1.158139in}}%
\pgfpathlineto{\pgfqpoint{0.798495in}{1.207353in}}%
\pgfpathlineto{\pgfqpoint{0.850288in}{1.196720in}}%
\pgfpathlineto{\pgfqpoint{0.875936in}{1.191519in}}%
\pgfpathlineto{\pgfqpoint{0.875518in}{1.183498in}}%
\pgfpathlineto{\pgfqpoint{0.878976in}{1.180671in}}%
\pgfpathlineto{\pgfqpoint{0.882037in}{1.174273in}}%
\pgfpathlineto{\pgfqpoint{0.877415in}{1.166653in}}%
\pgfpathlineto{\pgfqpoint{0.872431in}{1.163681in}}%
\pgfpathlineto{\pgfqpoint{0.869732in}{1.146966in}}%
\pgfpathlineto{\pgfqpoint{0.864002in}{1.147526in}}%
\pgfpathlineto{\pgfqpoint{0.863012in}{1.142510in}}%
\pgfpathlineto{\pgfqpoint{0.857375in}{1.143704in}}%
\pgfpathlineto{\pgfqpoint{0.854337in}{1.126666in}}%
\pgfpathlineto{\pgfqpoint{0.843170in}{1.128894in}}%
\pgfpathlineto{\pgfqpoint{0.836260in}{1.093783in}}%
\pgfpathclose%
\pgfusepath{fill}%
\end{pgfscope}%
\begin{pgfscope}%
\pgfpathrectangle{\pgfqpoint{0.100000in}{0.100000in}}{\pgfqpoint{3.007045in}{1.925000in}}%
\pgfusepath{clip}%
\pgfsetbuttcap%
\pgfsetmiterjoin%
\definecolor{currentfill}{rgb}{0.260715,0.573841,0.777209}%
\pgfsetfillcolor{currentfill}%
\pgfsetlinewidth{0.000000pt}%
\definecolor{currentstroke}{rgb}{0.000000,0.000000,0.000000}%
\pgfsetstrokecolor{currentstroke}%
\pgfsetstrokeopacity{0.000000}%
\pgfsetdash{}{0pt}%
\pgfpathmoveto{\pgfqpoint{2.253561in}{1.079316in}}%
\pgfpathlineto{\pgfqpoint{2.259515in}{1.072291in}}%
\pgfpathlineto{\pgfqpoint{2.261909in}{1.072347in}}%
\pgfpathlineto{\pgfqpoint{2.262113in}{1.062729in}}%
\pgfpathlineto{\pgfqpoint{2.267818in}{1.063028in}}%
\pgfpathlineto{\pgfqpoint{2.270021in}{1.058081in}}%
\pgfpathlineto{\pgfqpoint{2.261228in}{1.057564in}}%
\pgfpathlineto{\pgfqpoint{2.260372in}{1.054200in}}%
\pgfpathlineto{\pgfqpoint{2.239874in}{1.053973in}}%
\pgfpathlineto{\pgfqpoint{2.238225in}{1.047271in}}%
\pgfpathlineto{\pgfqpoint{2.234875in}{1.045920in}}%
\pgfpathlineto{\pgfqpoint{2.218727in}{1.043400in}}%
\pgfpathlineto{\pgfqpoint{2.217524in}{1.046026in}}%
\pgfpathlineto{\pgfqpoint{2.210809in}{1.047577in}}%
\pgfpathlineto{\pgfqpoint{2.207346in}{1.056620in}}%
\pgfpathlineto{\pgfqpoint{2.208703in}{1.064571in}}%
\pgfpathlineto{\pgfqpoint{2.204976in}{1.073343in}}%
\pgfpathlineto{\pgfqpoint{2.205852in}{1.076709in}}%
\pgfpathlineto{\pgfqpoint{2.208829in}{1.082188in}}%
\pgfpathlineto{\pgfqpoint{2.212960in}{1.083080in}}%
\pgfpathlineto{\pgfqpoint{2.211772in}{1.090499in}}%
\pgfpathlineto{\pgfqpoint{2.214133in}{1.095845in}}%
\pgfpathlineto{\pgfqpoint{2.217176in}{1.095188in}}%
\pgfpathlineto{\pgfqpoint{2.219675in}{1.101261in}}%
\pgfpathlineto{\pgfqpoint{2.223926in}{1.097599in}}%
\pgfpathlineto{\pgfqpoint{2.224931in}{1.092417in}}%
\pgfpathlineto{\pgfqpoint{2.229804in}{1.090057in}}%
\pgfpathlineto{\pgfqpoint{2.241377in}{1.090992in}}%
\pgfpathlineto{\pgfqpoint{2.246327in}{1.087565in}}%
\pgfpathlineto{\pgfqpoint{2.249127in}{1.081944in}}%
\pgfpathlineto{\pgfqpoint{2.253561in}{1.079316in}}%
\pgfpathclose%
\pgfusepath{fill}%
\end{pgfscope}%
\begin{pgfscope}%
\pgfpathrectangle{\pgfqpoint{0.100000in}{0.100000in}}{\pgfqpoint{3.007045in}{1.925000in}}%
\pgfusepath{clip}%
\pgfsetbuttcap%
\pgfsetmiterjoin%
\definecolor{currentfill}{rgb}{0.296025,0.597955,0.790988}%
\pgfsetfillcolor{currentfill}%
\pgfsetlinewidth{0.000000pt}%
\definecolor{currentstroke}{rgb}{0.000000,0.000000,0.000000}%
\pgfsetstrokecolor{currentstroke}%
\pgfsetstrokeopacity{0.000000}%
\pgfsetdash{}{0pt}%
\pgfpathmoveto{\pgfqpoint{1.596789in}{1.798977in}}%
\pgfpathlineto{\pgfqpoint{1.595955in}{1.780137in}}%
\pgfpathlineto{\pgfqpoint{1.596746in}{1.768470in}}%
\pgfpathlineto{\pgfqpoint{1.596215in}{1.756877in}}%
\pgfpathlineto{\pgfqpoint{1.573086in}{1.757987in}}%
\pgfpathlineto{\pgfqpoint{1.573646in}{1.769590in}}%
\pgfpathlineto{\pgfqpoint{1.544810in}{1.771147in}}%
\pgfpathlineto{\pgfqpoint{1.543838in}{1.782832in}}%
\pgfpathlineto{\pgfqpoint{1.544920in}{1.801546in}}%
\pgfpathlineto{\pgfqpoint{1.596789in}{1.798977in}}%
\pgfpathclose%
\pgfusepath{fill}%
\end{pgfscope}%
\begin{pgfscope}%
\pgfpathrectangle{\pgfqpoint{0.100000in}{0.100000in}}{\pgfqpoint{3.007045in}{1.925000in}}%
\pgfusepath{clip}%
\pgfsetbuttcap%
\pgfsetmiterjoin%
\definecolor{currentfill}{rgb}{0.290980,0.594510,0.789020}%
\pgfsetfillcolor{currentfill}%
\pgfsetlinewidth{0.000000pt}%
\definecolor{currentstroke}{rgb}{0.000000,0.000000,0.000000}%
\pgfsetstrokecolor{currentstroke}%
\pgfsetstrokeopacity{0.000000}%
\pgfsetdash{}{0pt}%
\pgfpathmoveto{\pgfqpoint{1.694674in}{0.889113in}}%
\pgfpathlineto{\pgfqpoint{1.694493in}{0.877607in}}%
\pgfpathlineto{\pgfqpoint{1.691843in}{0.877659in}}%
\pgfpathlineto{\pgfqpoint{1.691602in}{0.864102in}}%
\pgfpathlineto{\pgfqpoint{1.683133in}{0.863693in}}%
\pgfpathlineto{\pgfqpoint{1.678967in}{0.860772in}}%
\pgfpathlineto{\pgfqpoint{1.674047in}{0.866744in}}%
\pgfpathlineto{\pgfqpoint{1.670026in}{0.865689in}}%
\pgfpathlineto{\pgfqpoint{1.667600in}{0.860848in}}%
\pgfpathlineto{\pgfqpoint{1.627761in}{0.861901in}}%
\pgfpathlineto{\pgfqpoint{1.628500in}{0.893638in}}%
\pgfpathlineto{\pgfqpoint{1.619336in}{0.893761in}}%
\pgfpathlineto{\pgfqpoint{1.609800in}{0.896970in}}%
\pgfpathlineto{\pgfqpoint{1.594438in}{0.897482in}}%
\pgfpathlineto{\pgfqpoint{1.594488in}{0.908994in}}%
\pgfpathlineto{\pgfqpoint{1.594900in}{0.920448in}}%
\pgfpathlineto{\pgfqpoint{1.600584in}{0.920240in}}%
\pgfpathlineto{\pgfqpoint{1.601428in}{0.949236in}}%
\pgfpathlineto{\pgfqpoint{1.641329in}{0.947962in}}%
\pgfpathlineto{\pgfqpoint{1.647016in}{0.947800in}}%
\pgfpathlineto{\pgfqpoint{1.646710in}{0.935974in}}%
\pgfpathlineto{\pgfqpoint{1.658027in}{0.932849in}}%
\pgfpathlineto{\pgfqpoint{1.685760in}{0.932254in}}%
\pgfpathlineto{\pgfqpoint{1.685427in}{0.912299in}}%
\pgfpathlineto{\pgfqpoint{1.684983in}{0.896595in}}%
\pgfpathlineto{\pgfqpoint{1.691799in}{0.898304in}}%
\pgfpathlineto{\pgfqpoint{1.694674in}{0.889113in}}%
\pgfpathclose%
\pgfusepath{fill}%
\end{pgfscope}%
\begin{pgfscope}%
\pgfpathrectangle{\pgfqpoint{0.100000in}{0.100000in}}{\pgfqpoint{3.007045in}{1.925000in}}%
\pgfusepath{clip}%
\pgfsetbuttcap%
\pgfsetmiterjoin%
\definecolor{currentfill}{rgb}{0.412042,0.677186,0.836263}%
\pgfsetfillcolor{currentfill}%
\pgfsetlinewidth{0.000000pt}%
\definecolor{currentstroke}{rgb}{0.000000,0.000000,0.000000}%
\pgfsetstrokecolor{currentstroke}%
\pgfsetstrokeopacity{0.000000}%
\pgfsetdash{}{0pt}%
\pgfpathmoveto{\pgfqpoint{1.776405in}{1.429765in}}%
\pgfpathlineto{\pgfqpoint{1.779303in}{1.429779in}}%
\pgfpathlineto{\pgfqpoint{1.779266in}{1.452736in}}%
\pgfpathlineto{\pgfqpoint{1.831065in}{1.453126in}}%
\pgfpathlineto{\pgfqpoint{1.836747in}{1.453219in}}%
\pgfpathlineto{\pgfqpoint{1.837085in}{1.430186in}}%
\pgfpathlineto{\pgfqpoint{1.821649in}{1.429987in}}%
\pgfpathlineto{\pgfqpoint{1.799028in}{1.429855in}}%
\pgfpathlineto{\pgfqpoint{1.799116in}{1.413633in}}%
\pgfpathlineto{\pgfqpoint{1.776471in}{1.413556in}}%
\pgfpathlineto{\pgfqpoint{1.776405in}{1.429765in}}%
\pgfpathclose%
\pgfusepath{fill}%
\end{pgfscope}%
\begin{pgfscope}%
\pgfpathrectangle{\pgfqpoint{0.100000in}{0.100000in}}{\pgfqpoint{3.007045in}{1.925000in}}%
\pgfusepath{clip}%
\pgfsetbuttcap%
\pgfsetmiterjoin%
\definecolor{currentfill}{rgb}{0.391865,0.663406,0.828389}%
\pgfsetfillcolor{currentfill}%
\pgfsetlinewidth{0.000000pt}%
\definecolor{currentstroke}{rgb}{0.000000,0.000000,0.000000}%
\pgfsetstrokecolor{currentstroke}%
\pgfsetstrokeopacity{0.000000}%
\pgfsetdash{}{0pt}%
\pgfpathmoveto{\pgfqpoint{2.353818in}{1.009495in}}%
\pgfpathlineto{\pgfqpoint{2.341210in}{1.008147in}}%
\pgfpathlineto{\pgfqpoint{2.336554in}{1.012826in}}%
\pgfpathlineto{\pgfqpoint{2.332746in}{1.021710in}}%
\pgfpathlineto{\pgfqpoint{2.334697in}{1.028538in}}%
\pgfpathlineto{\pgfqpoint{2.332415in}{1.038665in}}%
\pgfpathlineto{\pgfqpoint{2.333634in}{1.044388in}}%
\pgfpathlineto{\pgfqpoint{2.334876in}{1.050418in}}%
\pgfpathlineto{\pgfqpoint{2.340592in}{1.057330in}}%
\pgfpathlineto{\pgfqpoint{2.347601in}{1.052815in}}%
\pgfpathlineto{\pgfqpoint{2.350874in}{1.053026in}}%
\pgfpathlineto{\pgfqpoint{2.358324in}{1.060460in}}%
\pgfpathlineto{\pgfqpoint{2.363870in}{1.060375in}}%
\pgfpathlineto{\pgfqpoint{2.372484in}{1.056245in}}%
\pgfpathlineto{\pgfqpoint{2.371954in}{1.050139in}}%
\pgfpathlineto{\pgfqpoint{2.376071in}{1.035433in}}%
\pgfpathlineto{\pgfqpoint{2.362675in}{1.024583in}}%
\pgfpathlineto{\pgfqpoint{2.360603in}{1.018528in}}%
\pgfpathlineto{\pgfqpoint{2.355024in}{1.013333in}}%
\pgfpathlineto{\pgfqpoint{2.353818in}{1.009495in}}%
\pgfpathclose%
\pgfusepath{fill}%
\end{pgfscope}%
\begin{pgfscope}%
\pgfpathrectangle{\pgfqpoint{0.100000in}{0.100000in}}{\pgfqpoint{3.007045in}{1.925000in}}%
\pgfusepath{clip}%
\pgfsetbuttcap%
\pgfsetmiterjoin%
\definecolor{currentfill}{rgb}{0.351511,0.635848,0.812641}%
\pgfsetfillcolor{currentfill}%
\pgfsetlinewidth{0.000000pt}%
\definecolor{currentstroke}{rgb}{0.000000,0.000000,0.000000}%
\pgfsetstrokecolor{currentstroke}%
\pgfsetstrokeopacity{0.000000}%
\pgfsetdash{}{0pt}%
\pgfpathmoveto{\pgfqpoint{2.321129in}{1.054650in}}%
\pgfpathlineto{\pgfqpoint{2.318137in}{1.056117in}}%
\pgfpathlineto{\pgfqpoint{2.310463in}{1.047437in}}%
\pgfpathlineto{\pgfqpoint{2.309458in}{1.055417in}}%
\pgfpathlineto{\pgfqpoint{2.302296in}{1.059040in}}%
\pgfpathlineto{\pgfqpoint{2.301227in}{1.065647in}}%
\pgfpathlineto{\pgfqpoint{2.291623in}{1.065687in}}%
\pgfpathlineto{\pgfqpoint{2.291416in}{1.086166in}}%
\pgfpathlineto{\pgfqpoint{2.289598in}{1.088435in}}%
\pgfpathlineto{\pgfqpoint{2.294182in}{1.093073in}}%
\pgfpathlineto{\pgfqpoint{2.300122in}{1.093676in}}%
\pgfpathlineto{\pgfqpoint{2.314503in}{1.080433in}}%
\pgfpathlineto{\pgfqpoint{2.316608in}{1.083386in}}%
\pgfpathlineto{\pgfqpoint{2.327579in}{1.069371in}}%
\pgfpathlineto{\pgfqpoint{2.322848in}{1.065287in}}%
\pgfpathlineto{\pgfqpoint{2.321129in}{1.054650in}}%
\pgfpathclose%
\pgfusepath{fill}%
\end{pgfscope}%
\begin{pgfscope}%
\pgfpathrectangle{\pgfqpoint{0.100000in}{0.100000in}}{\pgfqpoint{3.007045in}{1.925000in}}%
\pgfusepath{clip}%
\pgfsetbuttcap%
\pgfsetmiterjoin%
\definecolor{currentfill}{rgb}{0.401953,0.670296,0.832326}%
\pgfsetfillcolor{currentfill}%
\pgfsetlinewidth{0.000000pt}%
\definecolor{currentstroke}{rgb}{0.000000,0.000000,0.000000}%
\pgfsetstrokecolor{currentstroke}%
\pgfsetstrokeopacity{0.000000}%
\pgfsetdash{}{0pt}%
\pgfpathmoveto{\pgfqpoint{1.418635in}{1.235643in}}%
\pgfpathlineto{\pgfqpoint{1.414767in}{1.184726in}}%
\pgfpathlineto{\pgfqpoint{1.358836in}{1.188865in}}%
\pgfpathlineto{\pgfqpoint{1.330895in}{1.191503in}}%
\pgfpathlineto{\pgfqpoint{1.333705in}{1.220024in}}%
\pgfpathlineto{\pgfqpoint{1.345447in}{1.218910in}}%
\pgfpathlineto{\pgfqpoint{1.348876in}{1.253120in}}%
\pgfpathlineto{\pgfqpoint{1.343169in}{1.253667in}}%
\pgfpathlineto{\pgfqpoint{1.346503in}{1.285013in}}%
\pgfpathlineto{\pgfqpoint{1.392134in}{1.280968in}}%
\pgfpathlineto{\pgfqpoint{1.390844in}{1.264339in}}%
\pgfpathlineto{\pgfqpoint{1.420694in}{1.261942in}}%
\pgfpathlineto{\pgfqpoint{1.420426in}{1.258524in}}%
\pgfpathlineto{\pgfqpoint{1.418635in}{1.235643in}}%
\pgfpathclose%
\pgfusepath{fill}%
\end{pgfscope}%
\begin{pgfscope}%
\pgfpathrectangle{\pgfqpoint{0.100000in}{0.100000in}}{\pgfqpoint{3.007045in}{1.925000in}}%
\pgfusepath{clip}%
\pgfsetbuttcap%
\pgfsetmiterjoin%
\definecolor{currentfill}{rgb}{0.306113,0.604844,0.794925}%
\pgfsetfillcolor{currentfill}%
\pgfsetlinewidth{0.000000pt}%
\definecolor{currentstroke}{rgb}{0.000000,0.000000,0.000000}%
\pgfsetstrokecolor{currentstroke}%
\pgfsetstrokeopacity{0.000000}%
\pgfsetdash{}{0pt}%
\pgfpathmoveto{\pgfqpoint{1.555230in}{0.797022in}}%
\pgfpathlineto{\pgfqpoint{1.553928in}{0.768166in}}%
\pgfpathlineto{\pgfqpoint{1.525314in}{0.769593in}}%
\pgfpathlineto{\pgfqpoint{1.526290in}{0.791703in}}%
\pgfpathlineto{\pgfqpoint{1.526556in}{0.798349in}}%
\pgfpathlineto{\pgfqpoint{1.555230in}{0.797022in}}%
\pgfpathclose%
\pgfusepath{fill}%
\end{pgfscope}%
\begin{pgfscope}%
\pgfpathrectangle{\pgfqpoint{0.100000in}{0.100000in}}{\pgfqpoint{3.007045in}{1.925000in}}%
\pgfusepath{clip}%
\pgfsetbuttcap%
\pgfsetmiterjoin%
\definecolor{currentfill}{rgb}{0.366644,0.646182,0.818547}%
\pgfsetfillcolor{currentfill}%
\pgfsetlinewidth{0.000000pt}%
\definecolor{currentstroke}{rgb}{0.000000,0.000000,0.000000}%
\pgfsetstrokecolor{currentstroke}%
\pgfsetstrokeopacity{0.000000}%
\pgfsetdash{}{0pt}%
\pgfpathmoveto{\pgfqpoint{1.882047in}{1.149212in}}%
\pgfpathlineto{\pgfqpoint{1.877465in}{1.141048in}}%
\pgfpathlineto{\pgfqpoint{1.877944in}{1.138491in}}%
\pgfpathlineto{\pgfqpoint{1.872251in}{1.132295in}}%
\pgfpathlineto{\pgfqpoint{1.872212in}{1.129384in}}%
\pgfpathlineto{\pgfqpoint{1.849349in}{1.129915in}}%
\pgfpathlineto{\pgfqpoint{1.848856in}{1.148030in}}%
\pgfpathlineto{\pgfqpoint{1.840824in}{1.149749in}}%
\pgfpathlineto{\pgfqpoint{1.835689in}{1.147108in}}%
\pgfpathlineto{\pgfqpoint{1.835366in}{1.168017in}}%
\pgfpathlineto{\pgfqpoint{1.834849in}{1.196667in}}%
\pgfpathlineto{\pgfqpoint{1.855015in}{1.197513in}}%
\pgfpathlineto{\pgfqpoint{1.855189in}{1.180142in}}%
\pgfpathlineto{\pgfqpoint{1.859241in}{1.178993in}}%
\pgfpathlineto{\pgfqpoint{1.860786in}{1.170457in}}%
\pgfpathlineto{\pgfqpoint{1.859930in}{1.165889in}}%
\pgfpathlineto{\pgfqpoint{1.862920in}{1.162275in}}%
\pgfpathlineto{\pgfqpoint{1.867949in}{1.161555in}}%
\pgfpathlineto{\pgfqpoint{1.870359in}{1.156300in}}%
\pgfpathlineto{\pgfqpoint{1.875772in}{1.154964in}}%
\pgfpathlineto{\pgfqpoint{1.877817in}{1.149309in}}%
\pgfpathlineto{\pgfqpoint{1.882047in}{1.149212in}}%
\pgfpathclose%
\pgfusepath{fill}%
\end{pgfscope}%
\begin{pgfscope}%
\pgfpathrectangle{\pgfqpoint{0.100000in}{0.100000in}}{\pgfqpoint{3.007045in}{1.925000in}}%
\pgfusepath{clip}%
\pgfsetbuttcap%
\pgfsetmiterjoin%
\definecolor{currentfill}{rgb}{0.336378,0.625513,0.806736}%
\pgfsetfillcolor{currentfill}%
\pgfsetlinewidth{0.000000pt}%
\definecolor{currentstroke}{rgb}{0.000000,0.000000,0.000000}%
\pgfsetstrokecolor{currentstroke}%
\pgfsetstrokeopacity{0.000000}%
\pgfsetdash{}{0pt}%
\pgfpathmoveto{\pgfqpoint{2.422257in}{1.215672in}}%
\pgfpathlineto{\pgfqpoint{2.422680in}{1.209735in}}%
\pgfpathlineto{\pgfqpoint{2.416942in}{1.209349in}}%
\pgfpathlineto{\pgfqpoint{2.417290in}{1.203518in}}%
\pgfpathlineto{\pgfqpoint{2.405836in}{1.202691in}}%
\pgfpathlineto{\pgfqpoint{2.405917in}{1.200954in}}%
\pgfpathlineto{\pgfqpoint{2.394275in}{1.199273in}}%
\pgfpathlineto{\pgfqpoint{2.393794in}{1.205858in}}%
\pgfpathlineto{\pgfqpoint{2.381162in}{1.206723in}}%
\pgfpathlineto{\pgfqpoint{2.366865in}{1.205475in}}%
\pgfpathlineto{\pgfqpoint{2.366064in}{1.217273in}}%
\pgfpathlineto{\pgfqpoint{2.345734in}{1.216099in}}%
\pgfpathlineto{\pgfqpoint{2.348588in}{1.219904in}}%
\pgfpathlineto{\pgfqpoint{2.350594in}{1.236213in}}%
\pgfpathlineto{\pgfqpoint{2.349506in}{1.250356in}}%
\pgfpathlineto{\pgfqpoint{2.346617in}{1.250238in}}%
\pgfpathlineto{\pgfqpoint{2.345898in}{1.268387in}}%
\pgfpathlineto{\pgfqpoint{2.350943in}{1.269110in}}%
\pgfpathlineto{\pgfqpoint{2.359392in}{1.270321in}}%
\pgfpathlineto{\pgfqpoint{2.359900in}{1.266193in}}%
\pgfpathlineto{\pgfqpoint{2.376250in}{1.266421in}}%
\pgfpathlineto{\pgfqpoint{2.376485in}{1.262685in}}%
\pgfpathlineto{\pgfqpoint{2.390704in}{1.263733in}}%
\pgfpathlineto{\pgfqpoint{2.389048in}{1.277134in}}%
\pgfpathlineto{\pgfqpoint{2.402337in}{1.279030in}}%
\pgfpathlineto{\pgfqpoint{2.409849in}{1.281241in}}%
\pgfpathlineto{\pgfqpoint{2.411854in}{1.281854in}}%
\pgfpathlineto{\pgfqpoint{2.414093in}{1.259940in}}%
\pgfpathlineto{\pgfqpoint{2.415184in}{1.255293in}}%
\pgfpathlineto{\pgfqpoint{2.416625in}{1.241076in}}%
\pgfpathlineto{\pgfqpoint{2.415203in}{1.238420in}}%
\pgfpathlineto{\pgfqpoint{2.403598in}{1.237933in}}%
\pgfpathlineto{\pgfqpoint{2.403956in}{1.231787in}}%
\pgfpathlineto{\pgfqpoint{2.407820in}{1.232038in}}%
\pgfpathlineto{\pgfqpoint{2.410154in}{1.226352in}}%
\pgfpathlineto{\pgfqpoint{2.410789in}{1.216745in}}%
\pgfpathlineto{\pgfqpoint{2.422257in}{1.215672in}}%
\pgfpathclose%
\pgfusepath{fill}%
\end{pgfscope}%
\begin{pgfscope}%
\pgfpathrectangle{\pgfqpoint{0.100000in}{0.100000in}}{\pgfqpoint{3.007045in}{1.925000in}}%
\pgfusepath{clip}%
\pgfsetbuttcap%
\pgfsetmiterjoin%
\definecolor{currentfill}{rgb}{0.447843,0.697855,0.845306}%
\pgfsetfillcolor{currentfill}%
\pgfsetlinewidth{0.000000pt}%
\definecolor{currentstroke}{rgb}{0.000000,0.000000,0.000000}%
\pgfsetstrokecolor{currentstroke}%
\pgfsetstrokeopacity{0.000000}%
\pgfsetdash{}{0pt}%
\pgfpathmoveto{\pgfqpoint{1.792177in}{1.207263in}}%
\pgfpathlineto{\pgfqpoint{1.792476in}{1.224430in}}%
\pgfpathlineto{\pgfqpoint{1.790743in}{1.236578in}}%
\pgfpathlineto{\pgfqpoint{1.798792in}{1.236593in}}%
\pgfpathlineto{\pgfqpoint{1.833661in}{1.237372in}}%
\pgfpathlineto{\pgfqpoint{1.834576in}{1.208087in}}%
\pgfpathlineto{\pgfqpoint{1.811789in}{1.208062in}}%
\pgfpathlineto{\pgfqpoint{1.811728in}{1.201377in}}%
\pgfpathlineto{\pgfqpoint{1.792230in}{1.201536in}}%
\pgfpathlineto{\pgfqpoint{1.792177in}{1.207263in}}%
\pgfpathclose%
\pgfusepath{fill}%
\end{pgfscope}%
\begin{pgfscope}%
\pgfpathrectangle{\pgfqpoint{0.100000in}{0.100000in}}{\pgfqpoint{3.007045in}{1.925000in}}%
\pgfusepath{clip}%
\pgfsetbuttcap%
\pgfsetmiterjoin%
\definecolor{currentfill}{rgb}{0.447843,0.697855,0.845306}%
\pgfsetfillcolor{currentfill}%
\pgfsetlinewidth{0.000000pt}%
\definecolor{currentstroke}{rgb}{0.000000,0.000000,0.000000}%
\pgfsetstrokecolor{currentstroke}%
\pgfsetstrokeopacity{0.000000}%
\pgfsetdash{}{0pt}%
\pgfpathmoveto{\pgfqpoint{1.744270in}{1.498734in}}%
\pgfpathlineto{\pgfqpoint{1.743883in}{1.504541in}}%
\pgfpathlineto{\pgfqpoint{1.703956in}{1.505078in}}%
\pgfpathlineto{\pgfqpoint{1.704306in}{1.528001in}}%
\pgfpathlineto{\pgfqpoint{1.704575in}{1.547315in}}%
\pgfpathlineto{\pgfqpoint{1.712446in}{1.545741in}}%
\pgfpathlineto{\pgfqpoint{1.717578in}{1.542068in}}%
\pgfpathlineto{\pgfqpoint{1.720167in}{1.545159in}}%
\pgfpathlineto{\pgfqpoint{1.720360in}{1.556564in}}%
\pgfpathlineto{\pgfqpoint{1.766199in}{1.556278in}}%
\pgfpathlineto{\pgfqpoint{1.783347in}{1.556314in}}%
\pgfpathlineto{\pgfqpoint{1.783351in}{1.550568in}}%
\pgfpathlineto{\pgfqpoint{1.783670in}{1.521817in}}%
\pgfpathlineto{\pgfqpoint{1.749650in}{1.521778in}}%
\pgfpathlineto{\pgfqpoint{1.749628in}{1.512662in}}%
\pgfpathlineto{\pgfqpoint{1.755381in}{1.509019in}}%
\pgfpathlineto{\pgfqpoint{1.755330in}{1.498635in}}%
\pgfpathlineto{\pgfqpoint{1.744270in}{1.498734in}}%
\pgfpathclose%
\pgfusepath{fill}%
\end{pgfscope}%
\begin{pgfscope}%
\pgfpathrectangle{\pgfqpoint{0.100000in}{0.100000in}}{\pgfqpoint{3.007045in}{1.925000in}}%
\pgfusepath{clip}%
\pgfsetbuttcap%
\pgfsetmiterjoin%
\definecolor{currentfill}{rgb}{0.460392,0.704744,0.848012}%
\pgfsetfillcolor{currentfill}%
\pgfsetlinewidth{0.000000pt}%
\definecolor{currentstroke}{rgb}{0.000000,0.000000,0.000000}%
\pgfsetstrokecolor{currentstroke}%
\pgfsetstrokeopacity{0.000000}%
\pgfsetdash{}{0pt}%
\pgfpathmoveto{\pgfqpoint{1.890280in}{0.989939in}}%
\pgfpathlineto{\pgfqpoint{1.890382in}{0.969613in}}%
\pgfpathlineto{\pgfqpoint{1.886046in}{0.969526in}}%
\pgfpathlineto{\pgfqpoint{1.832387in}{0.968604in}}%
\pgfpathlineto{\pgfqpoint{1.821240in}{0.968478in}}%
\pgfpathlineto{\pgfqpoint{1.821509in}{0.984906in}}%
\pgfpathlineto{\pgfqpoint{1.822007in}{1.020648in}}%
\pgfpathlineto{\pgfqpoint{1.844374in}{1.020395in}}%
\pgfpathlineto{\pgfqpoint{1.845462in}{1.020387in}}%
\pgfpathlineto{\pgfqpoint{1.845487in}{1.001557in}}%
\pgfpathlineto{\pgfqpoint{1.859842in}{1.001580in}}%
\pgfpathlineto{\pgfqpoint{1.859810in}{0.990030in}}%
\pgfpathlineto{\pgfqpoint{1.890280in}{0.989939in}}%
\pgfpathclose%
\pgfusepath{fill}%
\end{pgfscope}%
\begin{pgfscope}%
\pgfpathrectangle{\pgfqpoint{0.100000in}{0.100000in}}{\pgfqpoint{3.007045in}{1.925000in}}%
\pgfusepath{clip}%
\pgfsetbuttcap%
\pgfsetmiterjoin%
\definecolor{currentfill}{rgb}{0.491765,0.721968,0.854779}%
\pgfsetfillcolor{currentfill}%
\pgfsetlinewidth{0.000000pt}%
\definecolor{currentstroke}{rgb}{0.000000,0.000000,0.000000}%
\pgfsetstrokecolor{currentstroke}%
\pgfsetstrokeopacity{0.000000}%
\pgfsetdash{}{0pt}%
\pgfpathmoveto{\pgfqpoint{2.285799in}{1.004151in}}%
\pgfpathlineto{\pgfqpoint{2.311789in}{1.005478in}}%
\pgfpathlineto{\pgfqpoint{2.315188in}{1.000641in}}%
\pgfpathlineto{\pgfqpoint{2.303650in}{1.003101in}}%
\pgfpathlineto{\pgfqpoint{2.298509in}{1.001042in}}%
\pgfpathlineto{\pgfqpoint{2.295566in}{0.990662in}}%
\pgfpathlineto{\pgfqpoint{2.296204in}{0.983928in}}%
\pgfpathlineto{\pgfqpoint{2.299083in}{0.977051in}}%
\pgfpathlineto{\pgfqpoint{2.296056in}{0.970309in}}%
\pgfpathlineto{\pgfqpoint{2.291608in}{0.970240in}}%
\pgfpathlineto{\pgfqpoint{2.291000in}{0.961775in}}%
\pgfpathlineto{\pgfqpoint{2.285629in}{0.965628in}}%
\pgfpathlineto{\pgfqpoint{2.277359in}{0.967152in}}%
\pgfpathlineto{\pgfqpoint{2.269975in}{0.962101in}}%
\pgfpathlineto{\pgfqpoint{2.267876in}{0.965337in}}%
\pgfpathlineto{\pgfqpoint{2.263169in}{0.965439in}}%
\pgfpathlineto{\pgfqpoint{2.261147in}{0.968764in}}%
\pgfpathlineto{\pgfqpoint{2.261071in}{0.973710in}}%
\pgfpathlineto{\pgfqpoint{2.258038in}{0.978829in}}%
\pgfpathlineto{\pgfqpoint{2.258356in}{0.992865in}}%
\pgfpathlineto{\pgfqpoint{2.263992in}{0.994959in}}%
\pgfpathlineto{\pgfqpoint{2.271083in}{0.992318in}}%
\pgfpathlineto{\pgfqpoint{2.275809in}{0.988285in}}%
\pgfpathlineto{\pgfqpoint{2.278118in}{0.995018in}}%
\pgfpathlineto{\pgfqpoint{2.285270in}{0.997505in}}%
\pgfpathlineto{\pgfqpoint{2.285799in}{1.004151in}}%
\pgfpathclose%
\pgfusepath{fill}%
\end{pgfscope}%
\begin{pgfscope}%
\pgfpathrectangle{\pgfqpoint{0.100000in}{0.100000in}}{\pgfqpoint{3.007045in}{1.925000in}}%
\pgfusepath{clip}%
\pgfsetbuttcap%
\pgfsetmiterjoin%
\definecolor{currentfill}{rgb}{0.341423,0.628958,0.808704}%
\pgfsetfillcolor{currentfill}%
\pgfsetlinewidth{0.000000pt}%
\definecolor{currentstroke}{rgb}{0.000000,0.000000,0.000000}%
\pgfsetstrokecolor{currentstroke}%
\pgfsetstrokeopacity{0.000000}%
\pgfsetdash{}{0pt}%
\pgfpathmoveto{\pgfqpoint{1.339144in}{1.451533in}}%
\pgfpathlineto{\pgfqpoint{1.298841in}{1.455523in}}%
\pgfpathlineto{\pgfqpoint{1.245609in}{1.461365in}}%
\pgfpathlineto{\pgfqpoint{1.251044in}{1.505633in}}%
\pgfpathlineto{\pgfqpoint{1.255504in}{1.539978in}}%
\pgfpathlineto{\pgfqpoint{1.257251in}{1.559802in}}%
\pgfpathlineto{\pgfqpoint{1.262489in}{1.559658in}}%
\pgfpathlineto{\pgfqpoint{1.303187in}{1.554906in}}%
\pgfpathlineto{\pgfqpoint{1.348840in}{1.549898in}}%
\pgfpathlineto{\pgfqpoint{1.343611in}{1.496079in}}%
\pgfpathlineto{\pgfqpoint{1.339144in}{1.451533in}}%
\pgfpathclose%
\pgfusepath{fill}%
\end{pgfscope}%
\begin{pgfscope}%
\pgfpathrectangle{\pgfqpoint{0.100000in}{0.100000in}}{\pgfqpoint{3.007045in}{1.925000in}}%
\pgfusepath{clip}%
\pgfsetbuttcap%
\pgfsetmiterjoin%
\definecolor{currentfill}{rgb}{0.485490,0.718524,0.853426}%
\pgfsetfillcolor{currentfill}%
\pgfsetlinewidth{0.000000pt}%
\definecolor{currentstroke}{rgb}{0.000000,0.000000,0.000000}%
\pgfsetstrokecolor{currentstroke}%
\pgfsetstrokeopacity{0.000000}%
\pgfsetdash{}{0pt}%
\pgfpathmoveto{\pgfqpoint{1.154637in}{0.238482in}}%
\pgfpathlineto{\pgfqpoint{1.153566in}{0.243456in}}%
\pgfpathlineto{\pgfqpoint{1.148403in}{0.253718in}}%
\pgfpathlineto{\pgfqpoint{1.150064in}{0.259799in}}%
\pgfpathlineto{\pgfqpoint{1.161104in}{0.270073in}}%
\pgfpathlineto{\pgfqpoint{1.164300in}{0.280924in}}%
\pgfpathlineto{\pgfqpoint{1.168171in}{0.288539in}}%
\pgfpathlineto{\pgfqpoint{1.167443in}{0.293757in}}%
\pgfpathlineto{\pgfqpoint{1.171591in}{0.300793in}}%
\pgfpathlineto{\pgfqpoint{1.191981in}{0.303197in}}%
\pgfpathlineto{\pgfqpoint{1.194052in}{0.305195in}}%
\pgfpathlineto{\pgfqpoint{1.194899in}{0.313558in}}%
\pgfpathlineto{\pgfqpoint{1.199467in}{0.319810in}}%
\pgfpathlineto{\pgfqpoint{1.203895in}{0.319010in}}%
\pgfpathlineto{\pgfqpoint{1.206973in}{0.312220in}}%
\pgfpathlineto{\pgfqpoint{1.209897in}{0.301367in}}%
\pgfpathlineto{\pgfqpoint{1.215982in}{0.294627in}}%
\pgfpathlineto{\pgfqpoint{1.221823in}{0.282731in}}%
\pgfpathlineto{\pgfqpoint{1.224745in}{0.266660in}}%
\pgfpathlineto{\pgfqpoint{1.219165in}{0.259374in}}%
\pgfpathlineto{\pgfqpoint{1.223841in}{0.256519in}}%
\pgfpathlineto{\pgfqpoint{1.220981in}{0.250601in}}%
\pgfpathlineto{\pgfqpoint{1.221882in}{0.243306in}}%
\pgfpathlineto{\pgfqpoint{1.224870in}{0.237321in}}%
\pgfpathlineto{\pgfqpoint{1.222256in}{0.234938in}}%
\pgfpathlineto{\pgfqpoint{1.208527in}{0.234063in}}%
\pgfpathlineto{\pgfqpoint{1.197467in}{0.236205in}}%
\pgfpathlineto{\pgfqpoint{1.188592in}{0.242263in}}%
\pgfpathlineto{\pgfqpoint{1.183559in}{0.241689in}}%
\pgfpathlineto{\pgfqpoint{1.172818in}{0.243600in}}%
\pgfpathlineto{\pgfqpoint{1.154637in}{0.238482in}}%
\pgfpathclose%
\pgfusepath{fill}%
\end{pgfscope}%
\begin{pgfscope}%
\pgfpathrectangle{\pgfqpoint{0.100000in}{0.100000in}}{\pgfqpoint{3.007045in}{1.925000in}}%
\pgfusepath{clip}%
\pgfsetbuttcap%
\pgfsetmiterjoin%
\definecolor{currentfill}{rgb}{0.441569,0.694410,0.843952}%
\pgfsetfillcolor{currentfill}%
\pgfsetlinewidth{0.000000pt}%
\definecolor{currentstroke}{rgb}{0.000000,0.000000,0.000000}%
\pgfsetstrokecolor{currentstroke}%
\pgfsetstrokeopacity{0.000000}%
\pgfsetdash{}{0pt}%
\pgfpathmoveto{\pgfqpoint{1.823694in}{1.499014in}}%
\pgfpathlineto{\pgfqpoint{1.822887in}{1.493279in}}%
\pgfpathlineto{\pgfqpoint{1.790073in}{1.492987in}}%
\pgfpathlineto{\pgfqpoint{1.789742in}{1.510236in}}%
\pgfpathlineto{\pgfqpoint{1.795936in}{1.510293in}}%
\pgfpathlineto{\pgfqpoint{1.795889in}{1.521880in}}%
\pgfpathlineto{\pgfqpoint{1.783670in}{1.521817in}}%
\pgfpathlineto{\pgfqpoint{1.783351in}{1.550568in}}%
\pgfpathlineto{\pgfqpoint{1.801048in}{1.550695in}}%
\pgfpathlineto{\pgfqpoint{1.806790in}{1.547902in}}%
\pgfpathlineto{\pgfqpoint{1.807185in}{1.527732in}}%
\pgfpathlineto{\pgfqpoint{1.818599in}{1.527791in}}%
\pgfpathlineto{\pgfqpoint{1.818879in}{1.510489in}}%
\pgfpathlineto{\pgfqpoint{1.824600in}{1.510542in}}%
\pgfpathlineto{\pgfqpoint{1.823694in}{1.499014in}}%
\pgfpathclose%
\pgfusepath{fill}%
\end{pgfscope}%
\begin{pgfscope}%
\pgfpathrectangle{\pgfqpoint{0.100000in}{0.100000in}}{\pgfqpoint{3.007045in}{1.925000in}}%
\pgfusepath{clip}%
\pgfsetbuttcap%
\pgfsetmiterjoin%
\definecolor{currentfill}{rgb}{0.412042,0.677186,0.836263}%
\pgfsetfillcolor{currentfill}%
\pgfsetlinewidth{0.000000pt}%
\definecolor{currentstroke}{rgb}{0.000000,0.000000,0.000000}%
\pgfsetstrokecolor{currentstroke}%
\pgfsetstrokeopacity{0.000000}%
\pgfsetdash{}{0pt}%
\pgfpathmoveto{\pgfqpoint{2.214133in}{1.095845in}}%
\pgfpathlineto{\pgfqpoint{2.211772in}{1.090499in}}%
\pgfpathlineto{\pgfqpoint{2.212960in}{1.083080in}}%
\pgfpathlineto{\pgfqpoint{2.208829in}{1.082188in}}%
\pgfpathlineto{\pgfqpoint{2.205852in}{1.076709in}}%
\pgfpathlineto{\pgfqpoint{2.205878in}{1.081086in}}%
\pgfpathlineto{\pgfqpoint{2.200736in}{1.080556in}}%
\pgfpathlineto{\pgfqpoint{2.197741in}{1.085673in}}%
\pgfpathlineto{\pgfqpoint{2.185322in}{1.077996in}}%
\pgfpathlineto{\pgfqpoint{2.184734in}{1.072536in}}%
\pgfpathlineto{\pgfqpoint{2.172010in}{1.077414in}}%
\pgfpathlineto{\pgfqpoint{2.165193in}{1.079905in}}%
\pgfpathlineto{\pgfqpoint{2.158994in}{1.078092in}}%
\pgfpathlineto{\pgfqpoint{2.156206in}{1.071747in}}%
\pgfpathlineto{\pgfqpoint{2.152373in}{1.076112in}}%
\pgfpathlineto{\pgfqpoint{2.144755in}{1.073731in}}%
\pgfpathlineto{\pgfqpoint{2.139094in}{1.074231in}}%
\pgfpathlineto{\pgfqpoint{2.141157in}{1.068895in}}%
\pgfpathlineto{\pgfqpoint{2.134908in}{1.067855in}}%
\pgfpathlineto{\pgfqpoint{2.131121in}{1.073626in}}%
\pgfpathlineto{\pgfqpoint{2.137521in}{1.089690in}}%
\pgfpathlineto{\pgfqpoint{2.134514in}{1.098251in}}%
\pgfpathlineto{\pgfqpoint{2.136503in}{1.101602in}}%
\pgfpathlineto{\pgfqpoint{2.134603in}{1.106062in}}%
\pgfpathlineto{\pgfqpoint{2.135045in}{1.113976in}}%
\pgfpathlineto{\pgfqpoint{2.136969in}{1.118978in}}%
\pgfpathlineto{\pgfqpoint{2.150356in}{1.119886in}}%
\pgfpathlineto{\pgfqpoint{2.154476in}{1.114634in}}%
\pgfpathlineto{\pgfqpoint{2.160227in}{1.118407in}}%
\pgfpathlineto{\pgfqpoint{2.171500in}{1.120112in}}%
\pgfpathlineto{\pgfqpoint{2.178440in}{1.120233in}}%
\pgfpathlineto{\pgfqpoint{2.190430in}{1.118463in}}%
\pgfpathlineto{\pgfqpoint{2.192366in}{1.120724in}}%
\pgfpathlineto{\pgfqpoint{2.200334in}{1.121433in}}%
\pgfpathlineto{\pgfqpoint{2.201173in}{1.112813in}}%
\pgfpathlineto{\pgfqpoint{2.202065in}{1.104184in}}%
\pgfpathlineto{\pgfqpoint{2.207678in}{1.104917in}}%
\pgfpathlineto{\pgfqpoint{2.207958in}{1.101054in}}%
\pgfpathlineto{\pgfqpoint{2.213695in}{1.101544in}}%
\pgfpathlineto{\pgfqpoint{2.214133in}{1.095845in}}%
\pgfpathclose%
\pgfusepath{fill}%
\end{pgfscope}%
\begin{pgfscope}%
\pgfpathrectangle{\pgfqpoint{0.100000in}{0.100000in}}{\pgfqpoint{3.007045in}{1.925000in}}%
\pgfusepath{clip}%
\pgfsetbuttcap%
\pgfsetmiterjoin%
\definecolor{currentfill}{rgb}{0.711265,0.831111,0.914433}%
\pgfsetfillcolor{currentfill}%
\pgfsetlinewidth{0.000000pt}%
\definecolor{currentstroke}{rgb}{0.000000,0.000000,0.000000}%
\pgfsetstrokecolor{currentstroke}%
\pgfsetstrokeopacity{0.000000}%
\pgfsetdash{}{0pt}%
\pgfpathmoveto{\pgfqpoint{1.929386in}{1.648471in}}%
\pgfpathlineto{\pgfqpoint{1.930881in}{1.608680in}}%
\pgfpathlineto{\pgfqpoint{1.907975in}{1.607918in}}%
\pgfpathlineto{\pgfqpoint{1.896759in}{1.607543in}}%
\pgfpathlineto{\pgfqpoint{1.896284in}{1.624795in}}%
\pgfpathlineto{\pgfqpoint{1.861397in}{1.624038in}}%
\pgfpathlineto{\pgfqpoint{1.860671in}{1.638052in}}%
\pgfpathlineto{\pgfqpoint{1.860483in}{1.664391in}}%
\pgfpathlineto{\pgfqpoint{1.859346in}{1.688805in}}%
\pgfpathlineto{\pgfqpoint{1.859255in}{1.710576in}}%
\pgfpathlineto{\pgfqpoint{1.858121in}{1.722001in}}%
\pgfpathlineto{\pgfqpoint{1.857189in}{1.733722in}}%
\pgfpathlineto{\pgfqpoint{1.856761in}{1.771102in}}%
\pgfpathlineto{\pgfqpoint{1.861336in}{1.770943in}}%
\pgfpathlineto{\pgfqpoint{1.872825in}{1.765570in}}%
\pgfpathlineto{\pgfqpoint{1.876937in}{1.765915in}}%
\pgfpathlineto{\pgfqpoint{1.876187in}{1.758800in}}%
\pgfpathlineto{\pgfqpoint{1.882756in}{1.759721in}}%
\pgfpathlineto{\pgfqpoint{1.889291in}{1.744713in}}%
\pgfpathlineto{\pgfqpoint{1.893197in}{1.746419in}}%
\pgfpathlineto{\pgfqpoint{1.892633in}{1.752996in}}%
\pgfpathlineto{\pgfqpoint{1.902968in}{1.754402in}}%
\pgfpathlineto{\pgfqpoint{1.905332in}{1.748213in}}%
\pgfpathlineto{\pgfqpoint{1.911782in}{1.744531in}}%
\pgfpathlineto{\pgfqpoint{1.918443in}{1.744191in}}%
\pgfpathlineto{\pgfqpoint{1.918786in}{1.738606in}}%
\pgfpathlineto{\pgfqpoint{1.931137in}{1.734731in}}%
\pgfpathlineto{\pgfqpoint{1.939426in}{1.737368in}}%
\pgfpathlineto{\pgfqpoint{1.948847in}{1.744739in}}%
\pgfpathlineto{\pgfqpoint{1.950215in}{1.724734in}}%
\pgfpathlineto{\pgfqpoint{1.951342in}{1.696561in}}%
\pgfpathlineto{\pgfqpoint{1.941485in}{1.683918in}}%
\pgfpathlineto{\pgfqpoint{1.931784in}{1.673108in}}%
\pgfpathlineto{\pgfqpoint{1.927592in}{1.670463in}}%
\pgfpathlineto{\pgfqpoint{1.918507in}{1.660575in}}%
\pgfpathlineto{\pgfqpoint{1.904812in}{1.649918in}}%
\pgfpathlineto{\pgfqpoint{1.907843in}{1.644693in}}%
\pgfpathlineto{\pgfqpoint{1.911711in}{1.642752in}}%
\pgfpathlineto{\pgfqpoint{1.918643in}{1.643971in}}%
\pgfpathlineto{\pgfqpoint{1.929386in}{1.648471in}}%
\pgfpathclose%
\pgfusepath{fill}%
\end{pgfscope}%
\begin{pgfscope}%
\pgfpathrectangle{\pgfqpoint{0.100000in}{0.100000in}}{\pgfqpoint{3.007045in}{1.925000in}}%
\pgfusepath{clip}%
\pgfsetbuttcap%
\pgfsetmiterjoin%
\definecolor{currentfill}{rgb}{0.371688,0.649627,0.820515}%
\pgfsetfillcolor{currentfill}%
\pgfsetlinewidth{0.000000pt}%
\definecolor{currentstroke}{rgb}{0.000000,0.000000,0.000000}%
\pgfsetstrokecolor{currentstroke}%
\pgfsetstrokeopacity{0.000000}%
\pgfsetdash{}{0pt}%
\pgfpathmoveto{\pgfqpoint{1.923061in}{1.300995in}}%
\pgfpathlineto{\pgfqpoint{1.905724in}{1.300373in}}%
\pgfpathlineto{\pgfqpoint{1.904114in}{1.352301in}}%
\pgfpathlineto{\pgfqpoint{1.938345in}{1.353373in}}%
\pgfpathlineto{\pgfqpoint{1.972422in}{1.354713in}}%
\pgfpathlineto{\pgfqpoint{1.973436in}{1.331705in}}%
\pgfpathlineto{\pgfqpoint{1.950534in}{1.330806in}}%
\pgfpathlineto{\pgfqpoint{1.950752in}{1.325057in}}%
\pgfpathlineto{\pgfqpoint{1.927902in}{1.324319in}}%
\pgfpathlineto{\pgfqpoint{1.928815in}{1.301237in}}%
\pgfpathlineto{\pgfqpoint{1.923061in}{1.300995in}}%
\pgfpathclose%
\pgfusepath{fill}%
\end{pgfscope}%
\begin{pgfscope}%
\pgfpathrectangle{\pgfqpoint{0.100000in}{0.100000in}}{\pgfqpoint{3.007045in}{1.925000in}}%
\pgfusepath{clip}%
\pgfsetbuttcap%
\pgfsetmiterjoin%
\definecolor{currentfill}{rgb}{0.554510,0.756417,0.868312}%
\pgfsetfillcolor{currentfill}%
\pgfsetlinewidth{0.000000pt}%
\definecolor{currentstroke}{rgb}{0.000000,0.000000,0.000000}%
\pgfsetstrokecolor{currentstroke}%
\pgfsetstrokeopacity{0.000000}%
\pgfsetdash{}{0pt}%
\pgfpathmoveto{\pgfqpoint{2.549716in}{0.711373in}}%
\pgfpathlineto{\pgfqpoint{2.544812in}{0.713097in}}%
\pgfpathlineto{\pgfqpoint{2.537534in}{0.718203in}}%
\pgfpathlineto{\pgfqpoint{2.534070in}{0.723697in}}%
\pgfpathlineto{\pgfqpoint{2.526011in}{0.723942in}}%
\pgfpathlineto{\pgfqpoint{2.520931in}{0.729238in}}%
\pgfpathlineto{\pgfqpoint{2.513924in}{0.731882in}}%
\pgfpathlineto{\pgfqpoint{2.512057in}{0.734800in}}%
\pgfpathlineto{\pgfqpoint{2.530406in}{0.743490in}}%
\pgfpathlineto{\pgfqpoint{2.528796in}{0.747623in}}%
\pgfpathlineto{\pgfqpoint{2.523508in}{0.752566in}}%
\pgfpathlineto{\pgfqpoint{2.522083in}{0.760192in}}%
\pgfpathlineto{\pgfqpoint{2.529482in}{0.767093in}}%
\pgfpathlineto{\pgfqpoint{2.535487in}{0.765030in}}%
\pgfpathlineto{\pgfqpoint{2.548564in}{0.780580in}}%
\pgfpathlineto{\pgfqpoint{2.556087in}{0.772933in}}%
\pgfpathlineto{\pgfqpoint{2.559434in}{0.779009in}}%
\pgfpathlineto{\pgfqpoint{2.567570in}{0.778188in}}%
\pgfpathlineto{\pgfqpoint{2.572033in}{0.772743in}}%
\pgfpathlineto{\pgfqpoint{2.576021in}{0.772345in}}%
\pgfpathlineto{\pgfqpoint{2.581007in}{0.767715in}}%
\pgfpathlineto{\pgfqpoint{2.584609in}{0.761996in}}%
\pgfpathlineto{\pgfqpoint{2.584009in}{0.757368in}}%
\pgfpathlineto{\pgfqpoint{2.574779in}{0.751322in}}%
\pgfpathlineto{\pgfqpoint{2.570706in}{0.743972in}}%
\pgfpathlineto{\pgfqpoint{2.562709in}{0.737347in}}%
\pgfpathlineto{\pgfqpoint{2.564979in}{0.730555in}}%
\pgfpathlineto{\pgfqpoint{2.552611in}{0.712060in}}%
\pgfpathlineto{\pgfqpoint{2.549716in}{0.711373in}}%
\pgfpathclose%
\pgfusepath{fill}%
\end{pgfscope}%
\begin{pgfscope}%
\pgfpathrectangle{\pgfqpoint{0.100000in}{0.100000in}}{\pgfqpoint{3.007045in}{1.925000in}}%
\pgfusepath{clip}%
\pgfsetbuttcap%
\pgfsetmiterjoin%
\definecolor{currentfill}{rgb}{0.396909,0.666851,0.830358}%
\pgfsetfillcolor{currentfill}%
\pgfsetlinewidth{0.000000pt}%
\definecolor{currentstroke}{rgb}{0.000000,0.000000,0.000000}%
\pgfsetstrokecolor{currentstroke}%
\pgfsetstrokeopacity{0.000000}%
\pgfsetdash{}{0pt}%
\pgfpathmoveto{\pgfqpoint{0.613279in}{1.618283in}}%
\pgfpathlineto{\pgfqpoint{0.651898in}{1.607795in}}%
\pgfpathlineto{\pgfqpoint{0.653418in}{1.613391in}}%
\pgfpathlineto{\pgfqpoint{0.680340in}{1.605736in}}%
\pgfpathlineto{\pgfqpoint{0.675770in}{1.589097in}}%
\pgfpathlineto{\pgfqpoint{0.661467in}{1.533821in}}%
\pgfpathlineto{\pgfqpoint{0.662044in}{1.533668in}}%
\pgfpathlineto{\pgfqpoint{0.651488in}{1.492874in}}%
\pgfpathlineto{\pgfqpoint{0.647669in}{1.474848in}}%
\pgfpathlineto{\pgfqpoint{0.623923in}{1.480760in}}%
\pgfpathlineto{\pgfqpoint{0.594261in}{1.488948in}}%
\pgfpathlineto{\pgfqpoint{0.592563in}{1.489427in}}%
\pgfpathlineto{\pgfqpoint{0.605512in}{1.537505in}}%
\pgfpathlineto{\pgfqpoint{0.578445in}{1.544847in}}%
\pgfpathlineto{\pgfqpoint{0.586719in}{1.572193in}}%
\pgfpathlineto{\pgfqpoint{0.588372in}{1.571730in}}%
\pgfpathlineto{\pgfqpoint{0.596099in}{1.599150in}}%
\pgfpathlineto{\pgfqpoint{0.597590in}{1.604775in}}%
\pgfpathlineto{\pgfqpoint{0.603136in}{1.603204in}}%
\pgfpathlineto{\pgfqpoint{0.607867in}{1.619794in}}%
\pgfpathlineto{\pgfqpoint{0.613279in}{1.618283in}}%
\pgfpathclose%
\pgfusepath{fill}%
\end{pgfscope}%
\begin{pgfscope}%
\pgfpathrectangle{\pgfqpoint{0.100000in}{0.100000in}}{\pgfqpoint{3.007045in}{1.925000in}}%
\pgfusepath{clip}%
\pgfsetbuttcap%
\pgfsetmiterjoin%
\definecolor{currentfill}{rgb}{0.671895,0.814379,0.900654}%
\pgfsetfillcolor{currentfill}%
\pgfsetlinewidth{0.000000pt}%
\definecolor{currentstroke}{rgb}{0.000000,0.000000,0.000000}%
\pgfsetstrokecolor{currentstroke}%
\pgfsetstrokeopacity{0.000000}%
\pgfsetdash{}{0pt}%
\pgfpathmoveto{\pgfqpoint{2.623105in}{0.353228in}}%
\pgfpathlineto{\pgfqpoint{2.599929in}{0.349210in}}%
\pgfpathlineto{\pgfqpoint{2.597057in}{0.366782in}}%
\pgfpathlineto{\pgfqpoint{2.579681in}{0.363910in}}%
\pgfpathlineto{\pgfqpoint{2.577000in}{0.380965in}}%
\pgfpathlineto{\pgfqpoint{2.574578in}{0.398555in}}%
\pgfpathlineto{\pgfqpoint{2.592108in}{0.401133in}}%
\pgfpathlineto{\pgfqpoint{2.591178in}{0.407023in}}%
\pgfpathlineto{\pgfqpoint{2.597054in}{0.407964in}}%
\pgfpathlineto{\pgfqpoint{2.596119in}{0.413805in}}%
\pgfpathlineto{\pgfqpoint{2.607766in}{0.416135in}}%
\pgfpathlineto{\pgfqpoint{2.600616in}{0.426095in}}%
\pgfpathlineto{\pgfqpoint{2.597645in}{0.425944in}}%
\pgfpathlineto{\pgfqpoint{2.590385in}{0.434304in}}%
\pgfpathlineto{\pgfqpoint{2.593275in}{0.442838in}}%
\pgfpathlineto{\pgfqpoint{2.609094in}{0.445229in}}%
\pgfpathlineto{\pgfqpoint{2.614673in}{0.446167in}}%
\pgfpathlineto{\pgfqpoint{2.615609in}{0.440573in}}%
\pgfpathlineto{\pgfqpoint{2.642481in}{0.444772in}}%
\pgfpathlineto{\pgfqpoint{2.647094in}{0.435626in}}%
\pgfpathlineto{\pgfqpoint{2.655837in}{0.422084in}}%
\pgfpathlineto{\pgfqpoint{2.661202in}{0.412146in}}%
\pgfpathlineto{\pgfqpoint{2.667843in}{0.397101in}}%
\pgfpathlineto{\pgfqpoint{2.669777in}{0.383631in}}%
\pgfpathlineto{\pgfqpoint{2.670514in}{0.365060in}}%
\pgfpathlineto{\pgfqpoint{2.661292in}{0.364487in}}%
\pgfpathlineto{\pgfqpoint{2.622244in}{0.358169in}}%
\pgfpathlineto{\pgfqpoint{2.623105in}{0.353228in}}%
\pgfpathclose%
\pgfusepath{fill}%
\end{pgfscope}%
\begin{pgfscope}%
\pgfpathrectangle{\pgfqpoint{0.100000in}{0.100000in}}{\pgfqpoint{3.007045in}{1.925000in}}%
\pgfusepath{clip}%
\pgfsetbuttcap%
\pgfsetmiterjoin%
\definecolor{currentfill}{rgb}{0.585882,0.773641,0.875079}%
\pgfsetfillcolor{currentfill}%
\pgfsetlinewidth{0.000000pt}%
\definecolor{currentstroke}{rgb}{0.000000,0.000000,0.000000}%
\pgfsetstrokecolor{currentstroke}%
\pgfsetstrokeopacity{0.000000}%
\pgfsetdash{}{0pt}%
\pgfpathmoveto{\pgfqpoint{2.201173in}{1.112813in}}%
\pgfpathlineto{\pgfqpoint{2.200334in}{1.121433in}}%
\pgfpathlineto{\pgfqpoint{2.192366in}{1.120724in}}%
\pgfpathlineto{\pgfqpoint{2.190430in}{1.118463in}}%
\pgfpathlineto{\pgfqpoint{2.178440in}{1.120233in}}%
\pgfpathlineto{\pgfqpoint{2.171500in}{1.120112in}}%
\pgfpathlineto{\pgfqpoint{2.168758in}{1.129637in}}%
\pgfpathlineto{\pgfqpoint{2.170118in}{1.138155in}}%
\pgfpathlineto{\pgfqpoint{2.175011in}{1.140154in}}%
\pgfpathlineto{\pgfqpoint{2.175194in}{1.144388in}}%
\pgfpathlineto{\pgfqpoint{2.169528in}{1.143936in}}%
\pgfpathlineto{\pgfqpoint{2.168048in}{1.161256in}}%
\pgfpathlineto{\pgfqpoint{2.177539in}{1.161835in}}%
\pgfpathlineto{\pgfqpoint{2.176581in}{1.173291in}}%
\pgfpathlineto{\pgfqpoint{2.182227in}{1.173751in}}%
\pgfpathlineto{\pgfqpoint{2.181588in}{1.182377in}}%
\pgfpathlineto{\pgfqpoint{2.194423in}{1.183326in}}%
\pgfpathlineto{\pgfqpoint{2.197194in}{1.183592in}}%
\pgfpathlineto{\pgfqpoint{2.198043in}{1.175099in}}%
\pgfpathlineto{\pgfqpoint{2.210701in}{1.176219in}}%
\pgfpathlineto{\pgfqpoint{2.213096in}{1.157168in}}%
\pgfpathlineto{\pgfqpoint{2.215737in}{1.157449in}}%
\pgfpathlineto{\pgfqpoint{2.218022in}{1.153845in}}%
\pgfpathlineto{\pgfqpoint{2.219718in}{1.138972in}}%
\pgfpathlineto{\pgfqpoint{2.218133in}{1.136791in}}%
\pgfpathlineto{\pgfqpoint{2.220327in}{1.114564in}}%
\pgfpathlineto{\pgfqpoint{2.201173in}{1.112813in}}%
\pgfpathclose%
\pgfusepath{fill}%
\end{pgfscope}%
\begin{pgfscope}%
\pgfpathrectangle{\pgfqpoint{0.100000in}{0.100000in}}{\pgfqpoint{3.007045in}{1.925000in}}%
\pgfusepath{clip}%
\pgfsetbuttcap%
\pgfsetmiterjoin%
\definecolor{currentfill}{rgb}{0.381776,0.656517,0.824452}%
\pgfsetfillcolor{currentfill}%
\pgfsetlinewidth{0.000000pt}%
\definecolor{currentstroke}{rgb}{0.000000,0.000000,0.000000}%
\pgfsetstrokecolor{currentstroke}%
\pgfsetstrokeopacity{0.000000}%
\pgfsetdash{}{0pt}%
\pgfpathmoveto{\pgfqpoint{1.899049in}{0.722750in}}%
\pgfpathlineto{\pgfqpoint{1.906378in}{0.722687in}}%
\pgfpathlineto{\pgfqpoint{1.910744in}{0.717166in}}%
\pgfpathlineto{\pgfqpoint{1.916505in}{0.717286in}}%
\pgfpathlineto{\pgfqpoint{1.916979in}{0.699909in}}%
\pgfpathlineto{\pgfqpoint{1.922946in}{0.694307in}}%
\pgfpathlineto{\pgfqpoint{1.923673in}{0.668219in}}%
\pgfpathlineto{\pgfqpoint{1.921110in}{0.659462in}}%
\pgfpathlineto{\pgfqpoint{1.906637in}{0.659096in}}%
\pgfpathlineto{\pgfqpoint{1.906746in}{0.653283in}}%
\pgfpathlineto{\pgfqpoint{1.886783in}{0.652756in}}%
\pgfpathlineto{\pgfqpoint{1.888359in}{0.662427in}}%
\pgfpathlineto{\pgfqpoint{1.891446in}{0.668175in}}%
\pgfpathlineto{\pgfqpoint{1.888095in}{0.681930in}}%
\pgfpathlineto{\pgfqpoint{1.895056in}{0.682024in}}%
\pgfpathlineto{\pgfqpoint{1.897220in}{0.687919in}}%
\pgfpathlineto{\pgfqpoint{1.896662in}{0.702398in}}%
\pgfpathlineto{\pgfqpoint{1.890952in}{0.702305in}}%
\pgfpathlineto{\pgfqpoint{1.890590in}{0.711973in}}%
\pgfpathlineto{\pgfqpoint{1.893306in}{0.722562in}}%
\pgfpathlineto{\pgfqpoint{1.899049in}{0.722750in}}%
\pgfpathclose%
\pgfusepath{fill}%
\end{pgfscope}%
\begin{pgfscope}%
\pgfpathrectangle{\pgfqpoint{0.100000in}{0.100000in}}{\pgfqpoint{3.007045in}{1.925000in}}%
\pgfusepath{clip}%
\pgfsetbuttcap%
\pgfsetmiterjoin%
\definecolor{currentfill}{rgb}{0.223806,0.537532,0.758431}%
\pgfsetfillcolor{currentfill}%
\pgfsetlinewidth{0.000000pt}%
\definecolor{currentstroke}{rgb}{0.000000,0.000000,0.000000}%
\pgfsetstrokecolor{currentstroke}%
\pgfsetstrokeopacity{0.000000}%
\pgfsetdash{}{0pt}%
\pgfpathmoveto{\pgfqpoint{1.092509in}{1.504079in}}%
\pgfpathlineto{\pgfqpoint{1.098004in}{1.495582in}}%
\pgfpathlineto{\pgfqpoint{1.097179in}{1.490393in}}%
\pgfpathlineto{\pgfqpoint{1.108398in}{1.488604in}}%
\pgfpathlineto{\pgfqpoint{1.119290in}{1.484034in}}%
\pgfpathlineto{\pgfqpoint{1.118840in}{1.481195in}}%
\pgfpathlineto{\pgfqpoint{1.135060in}{1.473922in}}%
\pgfpathlineto{\pgfqpoint{1.143778in}{1.473368in}}%
\pgfpathlineto{\pgfqpoint{1.170426in}{1.469676in}}%
\pgfpathlineto{\pgfqpoint{1.173551in}{1.471182in}}%
\pgfpathlineto{\pgfqpoint{1.174139in}{1.469186in}}%
\pgfpathlineto{\pgfqpoint{1.171716in}{1.446544in}}%
\pgfpathlineto{\pgfqpoint{1.168621in}{1.423872in}}%
\pgfpathlineto{\pgfqpoint{1.166601in}{1.424158in}}%
\pgfpathlineto{\pgfqpoint{1.163522in}{1.401453in}}%
\pgfpathlineto{\pgfqpoint{1.164443in}{1.401325in}}%
\pgfpathlineto{\pgfqpoint{1.162889in}{1.390045in}}%
\pgfpathlineto{\pgfqpoint{1.139789in}{1.393330in}}%
\pgfpathlineto{\pgfqpoint{1.089437in}{1.400986in}}%
\pgfpathlineto{\pgfqpoint{1.089929in}{1.412247in}}%
\pgfpathlineto{\pgfqpoint{1.092414in}{1.429227in}}%
\pgfpathlineto{\pgfqpoint{1.085388in}{1.435891in}}%
\pgfpathlineto{\pgfqpoint{1.079954in}{1.447589in}}%
\pgfpathlineto{\pgfqpoint{1.075159in}{1.453148in}}%
\pgfpathlineto{\pgfqpoint{1.070708in}{1.461628in}}%
\pgfpathlineto{\pgfqpoint{1.069165in}{1.468405in}}%
\pgfpathlineto{\pgfqpoint{1.069796in}{1.477757in}}%
\pgfpathlineto{\pgfqpoint{1.067969in}{1.484605in}}%
\pgfpathlineto{\pgfqpoint{1.053944in}{1.487032in}}%
\pgfpathlineto{\pgfqpoint{1.059788in}{1.522455in}}%
\pgfpathlineto{\pgfqpoint{1.062108in}{1.518116in}}%
\pgfpathlineto{\pgfqpoint{1.067064in}{1.517262in}}%
\pgfpathlineto{\pgfqpoint{1.070042in}{1.506895in}}%
\pgfpathlineto{\pgfqpoint{1.076162in}{1.510378in}}%
\pgfpathlineto{\pgfqpoint{1.078184in}{1.513711in}}%
\pgfpathlineto{\pgfqpoint{1.083266in}{1.515471in}}%
\pgfpathlineto{\pgfqpoint{1.087640in}{1.512697in}}%
\pgfpathlineto{\pgfqpoint{1.092509in}{1.504079in}}%
\pgfpathclose%
\pgfusepath{fill}%
\end{pgfscope}%
\begin{pgfscope}%
\pgfpathrectangle{\pgfqpoint{0.100000in}{0.100000in}}{\pgfqpoint{3.007045in}{1.925000in}}%
\pgfusepath{clip}%
\pgfsetbuttcap%
\pgfsetmiterjoin%
\definecolor{currentfill}{rgb}{0.765398,0.854118,0.933379}%
\pgfsetfillcolor{currentfill}%
\pgfsetlinewidth{0.000000pt}%
\definecolor{currentstroke}{rgb}{0.000000,0.000000,0.000000}%
\pgfsetstrokecolor{currentstroke}%
\pgfsetstrokeopacity{0.000000}%
\pgfsetdash{}{0pt}%
\pgfpathmoveto{\pgfqpoint{2.031351in}{1.567858in}}%
\pgfpathlineto{\pgfqpoint{2.030834in}{1.573678in}}%
\pgfpathlineto{\pgfqpoint{2.002328in}{1.571903in}}%
\pgfpathlineto{\pgfqpoint{2.000756in}{1.600279in}}%
\pgfpathlineto{\pgfqpoint{2.006057in}{1.600572in}}%
\pgfpathlineto{\pgfqpoint{2.004866in}{1.621636in}}%
\pgfpathlineto{\pgfqpoint{2.043917in}{1.613272in}}%
\pgfpathlineto{\pgfqpoint{2.048726in}{1.610846in}}%
\pgfpathlineto{\pgfqpoint{2.063792in}{1.604125in}}%
\pgfpathlineto{\pgfqpoint{2.064911in}{1.587042in}}%
\pgfpathlineto{\pgfqpoint{2.076489in}{1.587837in}}%
\pgfpathlineto{\pgfqpoint{2.078017in}{1.564986in}}%
\pgfpathlineto{\pgfqpoint{2.066412in}{1.564272in}}%
\pgfpathlineto{\pgfqpoint{2.054884in}{1.563482in}}%
\pgfpathlineto{\pgfqpoint{2.054639in}{1.569212in}}%
\pgfpathlineto{\pgfqpoint{2.031351in}{1.567858in}}%
\pgfpathclose%
\pgfusepath{fill}%
\end{pgfscope}%
\begin{pgfscope}%
\pgfpathrectangle{\pgfqpoint{0.100000in}{0.100000in}}{\pgfqpoint{3.007045in}{1.925000in}}%
\pgfusepath{clip}%
\pgfsetbuttcap%
\pgfsetmiterjoin%
\definecolor{currentfill}{rgb}{0.102499,0.408689,0.682891}%
\pgfsetfillcolor{currentfill}%
\pgfsetlinewidth{0.000000pt}%
\definecolor{currentstroke}{rgb}{0.000000,0.000000,0.000000}%
\pgfsetstrokecolor{currentstroke}%
\pgfsetstrokeopacity{0.000000}%
\pgfsetdash{}{0pt}%
\pgfpathmoveto{\pgfqpoint{0.955057in}{1.434288in}}%
\pgfpathlineto{\pgfqpoint{0.952072in}{1.424129in}}%
\pgfpathlineto{\pgfqpoint{0.948249in}{1.426793in}}%
\pgfpathlineto{\pgfqpoint{0.941439in}{1.428069in}}%
\pgfpathlineto{\pgfqpoint{0.934955in}{1.436848in}}%
\pgfpathlineto{\pgfqpoint{0.931789in}{1.444541in}}%
\pgfpathlineto{\pgfqpoint{0.926367in}{1.445605in}}%
\pgfpathlineto{\pgfqpoint{0.925285in}{1.439963in}}%
\pgfpathlineto{\pgfqpoint{0.917819in}{1.441425in}}%
\pgfpathlineto{\pgfqpoint{0.916709in}{1.435786in}}%
\pgfpathlineto{\pgfqpoint{0.900100in}{1.438993in}}%
\pgfpathlineto{\pgfqpoint{0.903388in}{1.455868in}}%
\pgfpathlineto{\pgfqpoint{0.895172in}{1.457614in}}%
\pgfpathlineto{\pgfqpoint{0.892728in}{1.460530in}}%
\pgfpathlineto{\pgfqpoint{0.896143in}{1.463222in}}%
\pgfpathlineto{\pgfqpoint{0.894533in}{1.469396in}}%
\pgfpathlineto{\pgfqpoint{0.898811in}{1.492021in}}%
\pgfpathlineto{\pgfqpoint{0.909954in}{1.489763in}}%
\pgfpathlineto{\pgfqpoint{0.912230in}{1.501011in}}%
\pgfpathlineto{\pgfqpoint{0.895520in}{1.504390in}}%
\pgfpathlineto{\pgfqpoint{0.896614in}{1.509734in}}%
\pgfpathlineto{\pgfqpoint{0.883755in}{1.512410in}}%
\pgfpathlineto{\pgfqpoint{0.884911in}{1.517908in}}%
\pgfpathlineto{\pgfqpoint{0.882414in}{1.527693in}}%
\pgfpathlineto{\pgfqpoint{0.878844in}{1.527019in}}%
\pgfpathlineto{\pgfqpoint{0.881395in}{1.530161in}}%
\pgfpathlineto{\pgfqpoint{0.891068in}{1.533017in}}%
\pgfpathlineto{\pgfqpoint{0.897650in}{1.536577in}}%
\pgfpathlineto{\pgfqpoint{0.900875in}{1.550400in}}%
\pgfpathlineto{\pgfqpoint{0.903257in}{1.555054in}}%
\pgfpathlineto{\pgfqpoint{0.905554in}{1.566329in}}%
\pgfpathlineto{\pgfqpoint{0.933521in}{1.560492in}}%
\pgfpathlineto{\pgfqpoint{0.935411in}{1.569991in}}%
\pgfpathlineto{\pgfqpoint{0.938309in}{1.576703in}}%
\pgfpathlineto{\pgfqpoint{0.952519in}{1.573584in}}%
\pgfpathlineto{\pgfqpoint{0.956111in}{1.570703in}}%
\pgfpathlineto{\pgfqpoint{0.958428in}{1.575728in}}%
\pgfpathlineto{\pgfqpoint{0.962156in}{1.577626in}}%
\pgfpathlineto{\pgfqpoint{0.970525in}{1.573027in}}%
\pgfpathlineto{\pgfqpoint{0.981565in}{1.573774in}}%
\pgfpathlineto{\pgfqpoint{0.983095in}{1.569716in}}%
\pgfpathlineto{\pgfqpoint{0.989150in}{1.572014in}}%
\pgfpathlineto{\pgfqpoint{0.998004in}{1.571906in}}%
\pgfpathlineto{\pgfqpoint{1.001653in}{1.579735in}}%
\pgfpathlineto{\pgfqpoint{1.006564in}{1.581699in}}%
\pgfpathlineto{\pgfqpoint{1.009746in}{1.577792in}}%
\pgfpathlineto{\pgfqpoint{1.015015in}{1.563234in}}%
\pgfpathlineto{\pgfqpoint{1.018543in}{1.560865in}}%
\pgfpathlineto{\pgfqpoint{1.012932in}{1.528900in}}%
\pgfpathlineto{\pgfqpoint{1.005109in}{1.526977in}}%
\pgfpathlineto{\pgfqpoint{0.995747in}{1.527955in}}%
\pgfpathlineto{\pgfqpoint{0.992126in}{1.508384in}}%
\pgfpathlineto{\pgfqpoint{0.999466in}{1.507030in}}%
\pgfpathlineto{\pgfqpoint{1.000614in}{1.501315in}}%
\pgfpathlineto{\pgfqpoint{1.007407in}{1.497511in}}%
\pgfpathlineto{\pgfqpoint{1.005297in}{1.485446in}}%
\pgfpathlineto{\pgfqpoint{1.001890in}{1.466220in}}%
\pgfpathlineto{\pgfqpoint{0.953549in}{1.475280in}}%
\pgfpathlineto{\pgfqpoint{0.947898in}{1.467983in}}%
\pgfpathlineto{\pgfqpoint{0.946788in}{1.460996in}}%
\pgfpathlineto{\pgfqpoint{0.947546in}{1.454706in}}%
\pgfpathlineto{\pgfqpoint{0.954259in}{1.454699in}}%
\pgfpathlineto{\pgfqpoint{0.955530in}{1.447686in}}%
\pgfpathlineto{\pgfqpoint{0.955057in}{1.434288in}}%
\pgfpathclose%
\pgfusepath{fill}%
\end{pgfscope}%
\begin{pgfscope}%
\pgfpathrectangle{\pgfqpoint{0.100000in}{0.100000in}}{\pgfqpoint{3.007045in}{1.925000in}}%
\pgfusepath{clip}%
\pgfsetbuttcap%
\pgfsetmiterjoin%
\definecolor{currentfill}{rgb}{0.056363,0.349635,0.636755}%
\pgfsetfillcolor{currentfill}%
\pgfsetlinewidth{0.000000pt}%
\definecolor{currentstroke}{rgb}{0.000000,0.000000,0.000000}%
\pgfsetstrokecolor{currentstroke}%
\pgfsetstrokeopacity{0.000000}%
\pgfsetdash{}{0pt}%
\pgfpathmoveto{\pgfqpoint{0.615087in}{1.868865in}}%
\pgfpathlineto{\pgfqpoint{0.618659in}{1.862354in}}%
\pgfpathlineto{\pgfqpoint{0.620178in}{1.855997in}}%
\pgfpathlineto{\pgfqpoint{0.627702in}{1.851025in}}%
\pgfpathlineto{\pgfqpoint{0.629717in}{1.846964in}}%
\pgfpathlineto{\pgfqpoint{0.635771in}{1.843623in}}%
\pgfpathlineto{\pgfqpoint{0.640168in}{1.838234in}}%
\pgfpathlineto{\pgfqpoint{0.652989in}{1.834668in}}%
\pgfpathlineto{\pgfqpoint{0.656028in}{1.830896in}}%
\pgfpathlineto{\pgfqpoint{0.652061in}{1.822418in}}%
\pgfpathlineto{\pgfqpoint{0.652827in}{1.812664in}}%
\pgfpathlineto{\pgfqpoint{0.651860in}{1.802864in}}%
\pgfpathlineto{\pgfqpoint{0.648904in}{1.795115in}}%
\pgfpathlineto{\pgfqpoint{0.651182in}{1.791391in}}%
\pgfpathlineto{\pgfqpoint{0.640972in}{1.753699in}}%
\pgfpathlineto{\pgfqpoint{0.616044in}{1.760740in}}%
\pgfpathlineto{\pgfqpoint{0.567701in}{1.775221in}}%
\pgfpathlineto{\pgfqpoint{0.574289in}{1.797104in}}%
\pgfpathlineto{\pgfqpoint{0.579595in}{1.795500in}}%
\pgfpathlineto{\pgfqpoint{0.579759in}{1.803398in}}%
\pgfpathlineto{\pgfqpoint{0.585013in}{1.811079in}}%
\pgfpathlineto{\pgfqpoint{0.585552in}{1.821819in}}%
\pgfpathlineto{\pgfqpoint{0.583582in}{1.827974in}}%
\pgfpathlineto{\pgfqpoint{0.597616in}{1.841914in}}%
\pgfpathlineto{\pgfqpoint{0.599282in}{1.851636in}}%
\pgfpathlineto{\pgfqpoint{0.595548in}{1.853584in}}%
\pgfpathlineto{\pgfqpoint{0.595348in}{1.857817in}}%
\pgfpathlineto{\pgfqpoint{0.600903in}{1.863151in}}%
\pgfpathlineto{\pgfqpoint{0.610134in}{1.868414in}}%
\pgfpathlineto{\pgfqpoint{0.615087in}{1.868865in}}%
\pgfpathclose%
\pgfusepath{fill}%
\end{pgfscope}%
\begin{pgfscope}%
\pgfpathrectangle{\pgfqpoint{0.100000in}{0.100000in}}{\pgfqpoint{3.007045in}{1.925000in}}%
\pgfusepath{clip}%
\pgfsetbuttcap%
\pgfsetmiterjoin%
\definecolor{currentfill}{rgb}{0.351511,0.635848,0.812641}%
\pgfsetfillcolor{currentfill}%
\pgfsetlinewidth{0.000000pt}%
\definecolor{currentstroke}{rgb}{0.000000,0.000000,0.000000}%
\pgfsetstrokecolor{currentstroke}%
\pgfsetstrokeopacity{0.000000}%
\pgfsetdash{}{0pt}%
\pgfpathmoveto{\pgfqpoint{0.465761in}{1.347164in}}%
\pgfpathlineto{\pgfqpoint{0.460387in}{1.347970in}}%
\pgfpathlineto{\pgfqpoint{0.450179in}{1.340731in}}%
\pgfpathlineto{\pgfqpoint{0.445384in}{1.329930in}}%
\pgfpathlineto{\pgfqpoint{0.439260in}{1.328328in}}%
\pgfpathlineto{\pgfqpoint{0.435140in}{1.324290in}}%
\pgfpathlineto{\pgfqpoint{0.431828in}{1.313269in}}%
\pgfpathlineto{\pgfqpoint{0.424973in}{1.314267in}}%
\pgfpathlineto{\pgfqpoint{0.420345in}{1.320339in}}%
\pgfpathlineto{\pgfqpoint{0.421047in}{1.323928in}}%
\pgfpathlineto{\pgfqpoint{0.417135in}{1.329622in}}%
\pgfpathlineto{\pgfqpoint{0.392426in}{1.337198in}}%
\pgfpathlineto{\pgfqpoint{0.390745in}{1.343728in}}%
\pgfpathlineto{\pgfqpoint{0.387533in}{1.347636in}}%
\pgfpathlineto{\pgfqpoint{0.389453in}{1.357782in}}%
\pgfpathlineto{\pgfqpoint{0.383879in}{1.360469in}}%
\pgfpathlineto{\pgfqpoint{0.379169in}{1.366194in}}%
\pgfpathlineto{\pgfqpoint{0.383653in}{1.376276in}}%
\pgfpathlineto{\pgfqpoint{0.386077in}{1.384484in}}%
\pgfpathlineto{\pgfqpoint{0.378819in}{1.386738in}}%
\pgfpathlineto{\pgfqpoint{0.381027in}{1.394938in}}%
\pgfpathlineto{\pgfqpoint{0.380635in}{1.401231in}}%
\pgfpathlineto{\pgfqpoint{0.423677in}{1.387762in}}%
\pgfpathlineto{\pgfqpoint{0.425415in}{1.393166in}}%
\pgfpathlineto{\pgfqpoint{0.437071in}{1.389645in}}%
\pgfpathlineto{\pgfqpoint{0.443854in}{1.394476in}}%
\pgfpathlineto{\pgfqpoint{0.446584in}{1.393519in}}%
\pgfpathlineto{\pgfqpoint{0.451708in}{1.400095in}}%
\pgfpathlineto{\pgfqpoint{0.459905in}{1.401168in}}%
\pgfpathlineto{\pgfqpoint{0.461995in}{1.397568in}}%
\pgfpathlineto{\pgfqpoint{0.458164in}{1.392328in}}%
\pgfpathlineto{\pgfqpoint{0.455804in}{1.383109in}}%
\pgfpathlineto{\pgfqpoint{0.457960in}{1.381368in}}%
\pgfpathlineto{\pgfqpoint{0.467458in}{1.355133in}}%
\pgfpathlineto{\pgfqpoint{0.465761in}{1.347164in}}%
\pgfpathclose%
\pgfusepath{fill}%
\end{pgfscope}%
\begin{pgfscope}%
\pgfpathrectangle{\pgfqpoint{0.100000in}{0.100000in}}{\pgfqpoint{3.007045in}{1.925000in}}%
\pgfusepath{clip}%
\pgfsetbuttcap%
\pgfsetmiterjoin%
\definecolor{currentfill}{rgb}{0.341423,0.628958,0.808704}%
\pgfsetfillcolor{currentfill}%
\pgfsetlinewidth{0.000000pt}%
\definecolor{currentstroke}{rgb}{0.000000,0.000000,0.000000}%
\pgfsetstrokecolor{currentstroke}%
\pgfsetstrokeopacity{0.000000}%
\pgfsetdash{}{0pt}%
\pgfpathmoveto{\pgfqpoint{1.746105in}{1.095854in}}%
\pgfpathlineto{\pgfqpoint{1.745387in}{1.069764in}}%
\pgfpathlineto{\pgfqpoint{1.722556in}{1.070090in}}%
\pgfpathlineto{\pgfqpoint{1.722658in}{1.078703in}}%
\pgfpathlineto{\pgfqpoint{1.702042in}{1.079070in}}%
\pgfpathlineto{\pgfqpoint{1.701914in}{1.073348in}}%
\pgfpathlineto{\pgfqpoint{1.693377in}{1.073520in}}%
\pgfpathlineto{\pgfqpoint{1.676880in}{1.073769in}}%
\pgfpathlineto{\pgfqpoint{1.677388in}{1.091095in}}%
\pgfpathlineto{\pgfqpoint{1.678559in}{1.102524in}}%
\pgfpathlineto{\pgfqpoint{1.702564in}{1.102052in}}%
\pgfpathlineto{\pgfqpoint{1.702839in}{1.116383in}}%
\pgfpathlineto{\pgfqpoint{1.723731in}{1.116097in}}%
\pgfpathlineto{\pgfqpoint{1.723295in}{1.096039in}}%
\pgfpathlineto{\pgfqpoint{1.746105in}{1.095854in}}%
\pgfpathclose%
\pgfusepath{fill}%
\end{pgfscope}%
\begin{pgfscope}%
\pgfpathrectangle{\pgfqpoint{0.100000in}{0.100000in}}{\pgfqpoint{3.007045in}{1.925000in}}%
\pgfusepath{clip}%
\pgfsetbuttcap%
\pgfsetmiterjoin%
\definecolor{currentfill}{rgb}{0.429020,0.687520,0.841246}%
\pgfsetfillcolor{currentfill}%
\pgfsetlinewidth{0.000000pt}%
\definecolor{currentstroke}{rgb}{0.000000,0.000000,0.000000}%
\pgfsetstrokecolor{currentstroke}%
\pgfsetstrokeopacity{0.000000}%
\pgfsetdash{}{0pt}%
\pgfpathmoveto{\pgfqpoint{1.102971in}{1.695747in}}%
\pgfpathlineto{\pgfqpoint{1.099659in}{1.696257in}}%
\pgfpathlineto{\pgfqpoint{1.093536in}{1.701272in}}%
\pgfpathlineto{\pgfqpoint{1.070809in}{1.700125in}}%
\pgfpathlineto{\pgfqpoint{1.066660in}{1.704559in}}%
\pgfpathlineto{\pgfqpoint{1.059670in}{1.716732in}}%
\pgfpathlineto{\pgfqpoint{1.053818in}{1.723481in}}%
\pgfpathlineto{\pgfqpoint{1.043222in}{1.727442in}}%
\pgfpathlineto{\pgfqpoint{1.026102in}{1.729804in}}%
\pgfpathlineto{\pgfqpoint{1.025046in}{1.724117in}}%
\pgfpathlineto{\pgfqpoint{1.019738in}{1.725093in}}%
\pgfpathlineto{\pgfqpoint{1.013851in}{1.726227in}}%
\pgfpathlineto{\pgfqpoint{1.016513in}{1.740285in}}%
\pgfpathlineto{\pgfqpoint{1.010013in}{1.742368in}}%
\pgfpathlineto{\pgfqpoint{1.005933in}{1.746597in}}%
\pgfpathlineto{\pgfqpoint{1.009841in}{1.767764in}}%
\pgfpathlineto{\pgfqpoint{0.999503in}{1.769057in}}%
\pgfpathlineto{\pgfqpoint{0.994442in}{1.773321in}}%
\pgfpathlineto{\pgfqpoint{0.994400in}{1.776622in}}%
\pgfpathlineto{\pgfqpoint{0.987897in}{1.779953in}}%
\pgfpathlineto{\pgfqpoint{0.976971in}{1.780376in}}%
\pgfpathlineto{\pgfqpoint{0.973893in}{1.786912in}}%
\pgfpathlineto{\pgfqpoint{0.975711in}{1.795488in}}%
\pgfpathlineto{\pgfqpoint{0.974188in}{1.804466in}}%
\pgfpathlineto{\pgfqpoint{0.978555in}{1.808720in}}%
\pgfpathlineto{\pgfqpoint{0.975124in}{1.816423in}}%
\pgfpathlineto{\pgfqpoint{1.011784in}{1.809131in}}%
\pgfpathlineto{\pgfqpoint{1.016942in}{1.805156in}}%
\pgfpathlineto{\pgfqpoint{1.018457in}{1.797935in}}%
\pgfpathlineto{\pgfqpoint{1.043735in}{1.793392in}}%
\pgfpathlineto{\pgfqpoint{1.045450in}{1.802919in}}%
\pgfpathlineto{\pgfqpoint{1.070974in}{1.798414in}}%
\pgfpathlineto{\pgfqpoint{1.071976in}{1.804099in}}%
\pgfpathlineto{\pgfqpoint{1.080804in}{1.802553in}}%
\pgfpathlineto{\pgfqpoint{1.081806in}{1.808264in}}%
\pgfpathlineto{\pgfqpoint{1.115936in}{1.802454in}}%
\pgfpathlineto{\pgfqpoint{1.115258in}{1.796858in}}%
\pgfpathlineto{\pgfqpoint{1.120232in}{1.796028in}}%
\pgfpathlineto{\pgfqpoint{1.119270in}{1.790180in}}%
\pgfpathlineto{\pgfqpoint{1.127771in}{1.788966in}}%
\pgfpathlineto{\pgfqpoint{1.123379in}{1.762908in}}%
\pgfpathlineto{\pgfqpoint{1.110393in}{1.763495in}}%
\pgfpathlineto{\pgfqpoint{1.107591in}{1.752201in}}%
\pgfpathlineto{\pgfqpoint{1.094250in}{1.746968in}}%
\pgfpathlineto{\pgfqpoint{1.092154in}{1.736531in}}%
\pgfpathlineto{\pgfqpoint{1.097369in}{1.732746in}}%
\pgfpathlineto{\pgfqpoint{1.103070in}{1.731808in}}%
\pgfpathlineto{\pgfqpoint{1.107755in}{1.728110in}}%
\pgfpathlineto{\pgfqpoint{1.104710in}{1.709101in}}%
\pgfpathlineto{\pgfqpoint{1.102514in}{1.707496in}}%
\pgfpathlineto{\pgfqpoint{1.102971in}{1.695747in}}%
\pgfpathclose%
\pgfusepath{fill}%
\end{pgfscope}%
\begin{pgfscope}%
\pgfpathrectangle{\pgfqpoint{0.100000in}{0.100000in}}{\pgfqpoint{3.007045in}{1.925000in}}%
\pgfusepath{clip}%
\pgfsetbuttcap%
\pgfsetmiterjoin%
\definecolor{currentfill}{rgb}{0.285936,0.591065,0.787051}%
\pgfsetfillcolor{currentfill}%
\pgfsetlinewidth{0.000000pt}%
\definecolor{currentstroke}{rgb}{0.000000,0.000000,0.000000}%
\pgfsetstrokecolor{currentstroke}%
\pgfsetstrokeopacity{0.000000}%
\pgfsetdash{}{0pt}%
\pgfpathmoveto{\pgfqpoint{1.522557in}{1.725806in}}%
\pgfpathlineto{\pgfqpoint{1.539856in}{1.724787in}}%
\pgfpathlineto{\pgfqpoint{1.538782in}{1.736495in}}%
\pgfpathlineto{\pgfqpoint{1.540108in}{1.759776in}}%
\pgfpathlineto{\pgfqpoint{1.539033in}{1.771449in}}%
\pgfpathlineto{\pgfqpoint{1.544810in}{1.771147in}}%
\pgfpathlineto{\pgfqpoint{1.573646in}{1.769590in}}%
\pgfpathlineto{\pgfqpoint{1.573086in}{1.757987in}}%
\pgfpathlineto{\pgfqpoint{1.557416in}{1.758816in}}%
\pgfpathlineto{\pgfqpoint{1.556105in}{1.735501in}}%
\pgfpathlineto{\pgfqpoint{1.557103in}{1.723835in}}%
\pgfpathlineto{\pgfqpoint{1.580095in}{1.722634in}}%
\pgfpathlineto{\pgfqpoint{1.580956in}{1.710973in}}%
\pgfpathlineto{\pgfqpoint{1.579862in}{1.687980in}}%
\pgfpathlineto{\pgfqpoint{1.570193in}{1.688446in}}%
\pgfpathlineto{\pgfqpoint{1.541684in}{1.689945in}}%
\pgfpathlineto{\pgfqpoint{1.516455in}{1.691466in}}%
\pgfpathlineto{\pgfqpoint{1.517941in}{1.714507in}}%
\pgfpathlineto{\pgfqpoint{1.521849in}{1.714257in}}%
\pgfpathlineto{\pgfqpoint{1.522557in}{1.725806in}}%
\pgfpathclose%
\pgfusepath{fill}%
\end{pgfscope}%
\begin{pgfscope}%
\pgfpathrectangle{\pgfqpoint{0.100000in}{0.100000in}}{\pgfqpoint{3.007045in}{1.925000in}}%
\pgfusepath{clip}%
\pgfsetbuttcap%
\pgfsetmiterjoin%
\definecolor{currentfill}{rgb}{0.406997,0.673741,0.834295}%
\pgfsetfillcolor{currentfill}%
\pgfsetlinewidth{0.000000pt}%
\definecolor{currentstroke}{rgb}{0.000000,0.000000,0.000000}%
\pgfsetstrokecolor{currentstroke}%
\pgfsetstrokeopacity{0.000000}%
\pgfsetdash{}{0pt}%
\pgfpathmoveto{\pgfqpoint{2.024381in}{1.212989in}}%
\pgfpathlineto{\pgfqpoint{2.014487in}{1.208775in}}%
\pgfpathlineto{\pgfqpoint{2.009474in}{1.209217in}}%
\pgfpathlineto{\pgfqpoint{2.008276in}{1.212623in}}%
\pgfpathlineto{\pgfqpoint{2.013813in}{1.217404in}}%
\pgfpathlineto{\pgfqpoint{2.017601in}{1.220842in}}%
\pgfpathlineto{\pgfqpoint{2.021248in}{1.230411in}}%
\pgfpathlineto{\pgfqpoint{2.027473in}{1.235391in}}%
\pgfpathlineto{\pgfqpoint{2.029027in}{1.239956in}}%
\pgfpathlineto{\pgfqpoint{2.028561in}{1.247285in}}%
\pgfpathlineto{\pgfqpoint{2.022785in}{1.247040in}}%
\pgfpathlineto{\pgfqpoint{2.021691in}{1.270010in}}%
\pgfpathlineto{\pgfqpoint{2.038925in}{1.270934in}}%
\pgfpathlineto{\pgfqpoint{2.037908in}{1.288046in}}%
\pgfpathlineto{\pgfqpoint{2.027069in}{1.287453in}}%
\pgfpathlineto{\pgfqpoint{2.025542in}{1.310434in}}%
\pgfpathlineto{\pgfqpoint{2.059792in}{1.312587in}}%
\pgfpathlineto{\pgfqpoint{2.059573in}{1.315459in}}%
\pgfpathlineto{\pgfqpoint{2.087331in}{1.317506in}}%
\pgfpathlineto{\pgfqpoint{2.090562in}{1.283138in}}%
\pgfpathlineto{\pgfqpoint{2.073487in}{1.281826in}}%
\pgfpathlineto{\pgfqpoint{2.075117in}{1.258653in}}%
\pgfpathlineto{\pgfqpoint{2.072749in}{1.252662in}}%
\pgfpathlineto{\pgfqpoint{2.065605in}{1.247690in}}%
\pgfpathlineto{\pgfqpoint{2.058847in}{1.247095in}}%
\pgfpathlineto{\pgfqpoint{2.060239in}{1.229418in}}%
\pgfpathlineto{\pgfqpoint{2.043262in}{1.228036in}}%
\pgfpathlineto{\pgfqpoint{2.044100in}{1.215046in}}%
\pgfpathlineto{\pgfqpoint{2.039739in}{1.216849in}}%
\pgfpathlineto{\pgfqpoint{2.032407in}{1.214497in}}%
\pgfpathlineto{\pgfqpoint{2.025473in}{1.215274in}}%
\pgfpathlineto{\pgfqpoint{2.024381in}{1.212989in}}%
\pgfpathclose%
\pgfusepath{fill}%
\end{pgfscope}%
\begin{pgfscope}%
\pgfpathrectangle{\pgfqpoint{0.100000in}{0.100000in}}{\pgfqpoint{3.007045in}{1.925000in}}%
\pgfusepath{clip}%
\pgfsetbuttcap%
\pgfsetmiterjoin%
\definecolor{currentfill}{rgb}{0.195386,0.509112,0.743791}%
\pgfsetfillcolor{currentfill}%
\pgfsetlinewidth{0.000000pt}%
\definecolor{currentstroke}{rgb}{0.000000,0.000000,0.000000}%
\pgfsetstrokecolor{currentstroke}%
\pgfsetstrokeopacity{0.000000}%
\pgfsetdash{}{0pt}%
\pgfpathmoveto{\pgfqpoint{1.553171in}{1.527805in}}%
\pgfpathlineto{\pgfqpoint{1.557925in}{1.527548in}}%
\pgfpathlineto{\pgfqpoint{1.598571in}{1.525674in}}%
\pgfpathlineto{\pgfqpoint{1.597846in}{1.508282in}}%
\pgfpathlineto{\pgfqpoint{1.596903in}{1.479359in}}%
\pgfpathlineto{\pgfqpoint{1.586215in}{1.479799in}}%
\pgfpathlineto{\pgfqpoint{1.555401in}{1.480969in}}%
\pgfpathlineto{\pgfqpoint{1.551371in}{1.482830in}}%
\pgfpathlineto{\pgfqpoint{1.551887in}{1.504699in}}%
\pgfpathlineto{\pgfqpoint{1.553171in}{1.527805in}}%
\pgfpathclose%
\pgfusepath{fill}%
\end{pgfscope}%
\begin{pgfscope}%
\pgfpathrectangle{\pgfqpoint{0.100000in}{0.100000in}}{\pgfqpoint{3.007045in}{1.925000in}}%
\pgfusepath{clip}%
\pgfsetbuttcap%
\pgfsetmiterjoin%
\definecolor{currentfill}{rgb}{0.321246,0.615179,0.800830}%
\pgfsetfillcolor{currentfill}%
\pgfsetlinewidth{0.000000pt}%
\definecolor{currentstroke}{rgb}{0.000000,0.000000,0.000000}%
\pgfsetstrokecolor{currentstroke}%
\pgfsetstrokeopacity{0.000000}%
\pgfsetdash{}{0pt}%
\pgfpathmoveto{\pgfqpoint{1.963991in}{0.607879in}}%
\pgfpathlineto{\pgfqpoint{1.967968in}{0.612514in}}%
\pgfpathlineto{\pgfqpoint{1.964384in}{0.616831in}}%
\pgfpathlineto{\pgfqpoint{1.967749in}{0.623734in}}%
\pgfpathlineto{\pgfqpoint{1.971855in}{0.623291in}}%
\pgfpathlineto{\pgfqpoint{1.972763in}{0.629511in}}%
\pgfpathlineto{\pgfqpoint{1.985698in}{0.633433in}}%
\pgfpathlineto{\pgfqpoint{1.993008in}{0.630363in}}%
\pgfpathlineto{\pgfqpoint{1.998230in}{0.632422in}}%
\pgfpathlineto{\pgfqpoint{2.020031in}{0.633536in}}%
\pgfpathlineto{\pgfqpoint{2.049764in}{0.635164in}}%
\pgfpathlineto{\pgfqpoint{2.053466in}{0.638280in}}%
\pgfpathlineto{\pgfqpoint{2.058100in}{0.638257in}}%
\pgfpathlineto{\pgfqpoint{2.065780in}{0.638658in}}%
\pgfpathlineto{\pgfqpoint{2.065550in}{0.641578in}}%
\pgfpathlineto{\pgfqpoint{2.075185in}{0.642117in}}%
\pgfpathlineto{\pgfqpoint{2.076881in}{0.613497in}}%
\pgfpathlineto{\pgfqpoint{2.072666in}{0.613237in}}%
\pgfpathlineto{\pgfqpoint{2.037435in}{0.611183in}}%
\pgfpathlineto{\pgfqpoint{2.010177in}{0.609791in}}%
\pgfpathlineto{\pgfqpoint{1.963991in}{0.607879in}}%
\pgfpathclose%
\pgfusepath{fill}%
\end{pgfscope}%
\begin{pgfscope}%
\pgfpathrectangle{\pgfqpoint{0.100000in}{0.100000in}}{\pgfqpoint{3.007045in}{1.925000in}}%
\pgfusepath{clip}%
\pgfsetbuttcap%
\pgfsetmiterjoin%
\definecolor{currentfill}{rgb}{0.401953,0.670296,0.832326}%
\pgfsetfillcolor{currentfill}%
\pgfsetlinewidth{0.000000pt}%
\definecolor{currentstroke}{rgb}{0.000000,0.000000,0.000000}%
\pgfsetstrokecolor{currentstroke}%
\pgfsetstrokeopacity{0.000000}%
\pgfsetdash{}{0pt}%
\pgfpathmoveto{\pgfqpoint{1.867818in}{1.391658in}}%
\pgfpathlineto{\pgfqpoint{1.866993in}{1.430735in}}%
\pgfpathlineto{\pgfqpoint{1.894560in}{1.431480in}}%
\pgfpathlineto{\pgfqpoint{1.928977in}{1.432597in}}%
\pgfpathlineto{\pgfqpoint{1.934699in}{1.432796in}}%
\pgfpathlineto{\pgfqpoint{1.936000in}{1.405159in}}%
\pgfpathlineto{\pgfqpoint{1.913064in}{1.404438in}}%
\pgfpathlineto{\pgfqpoint{1.913408in}{1.392837in}}%
\pgfpathlineto{\pgfqpoint{1.867818in}{1.391658in}}%
\pgfpathclose%
\pgfusepath{fill}%
\end{pgfscope}%
\begin{pgfscope}%
\pgfpathrectangle{\pgfqpoint{0.100000in}{0.100000in}}{\pgfqpoint{3.007045in}{1.925000in}}%
\pgfusepath{clip}%
\pgfsetbuttcap%
\pgfsetmiterjoin%
\definecolor{currentfill}{rgb}{0.592157,0.777086,0.876432}%
\pgfsetfillcolor{currentfill}%
\pgfsetlinewidth{0.000000pt}%
\definecolor{currentstroke}{rgb}{0.000000,0.000000,0.000000}%
\pgfsetstrokecolor{currentstroke}%
\pgfsetstrokeopacity{0.000000}%
\pgfsetdash{}{0pt}%
\pgfpathmoveto{\pgfqpoint{2.608126in}{1.441492in}}%
\pgfpathlineto{\pgfqpoint{2.601922in}{1.443430in}}%
\pgfpathlineto{\pgfqpoint{2.572079in}{1.437927in}}%
\pgfpathlineto{\pgfqpoint{2.570755in}{1.438725in}}%
\pgfpathlineto{\pgfqpoint{2.557784in}{1.431044in}}%
\pgfpathlineto{\pgfqpoint{2.552432in}{1.429022in}}%
\pgfpathlineto{\pgfqpoint{2.546681in}{1.434064in}}%
\pgfpathlineto{\pgfqpoint{2.543334in}{1.433990in}}%
\pgfpathlineto{\pgfqpoint{2.539326in}{1.435418in}}%
\pgfpathlineto{\pgfqpoint{2.542194in}{1.444103in}}%
\pgfpathlineto{\pgfqpoint{2.547820in}{1.448105in}}%
\pgfpathlineto{\pgfqpoint{2.550364in}{1.452200in}}%
\pgfpathlineto{\pgfqpoint{2.544628in}{1.462243in}}%
\pgfpathlineto{\pgfqpoint{2.540521in}{1.463483in}}%
\pgfpathlineto{\pgfqpoint{2.539800in}{1.468753in}}%
\pgfpathlineto{\pgfqpoint{2.535305in}{1.480591in}}%
\pgfpathlineto{\pgfqpoint{2.544988in}{1.486580in}}%
\pgfpathlineto{\pgfqpoint{2.553954in}{1.490859in}}%
\pgfpathlineto{\pgfqpoint{2.565267in}{1.494184in}}%
\pgfpathlineto{\pgfqpoint{2.584950in}{1.497319in}}%
\pgfpathlineto{\pgfqpoint{2.595524in}{1.497454in}}%
\pgfpathlineto{\pgfqpoint{2.605383in}{1.492846in}}%
\pgfpathlineto{\pgfqpoint{2.614787in}{1.497022in}}%
\pgfpathlineto{\pgfqpoint{2.634861in}{1.500680in}}%
\pgfpathlineto{\pgfqpoint{2.644415in}{1.507385in}}%
\pgfpathlineto{\pgfqpoint{2.648812in}{1.490538in}}%
\pgfpathlineto{\pgfqpoint{2.648634in}{1.483094in}}%
\pgfpathlineto{\pgfqpoint{2.652395in}{1.466521in}}%
\pgfpathlineto{\pgfqpoint{2.656543in}{1.461343in}}%
\pgfpathlineto{\pgfqpoint{2.661352in}{1.457374in}}%
\pgfpathlineto{\pgfqpoint{2.646021in}{1.453686in}}%
\pgfpathlineto{\pgfqpoint{2.647379in}{1.448629in}}%
\pgfpathlineto{\pgfqpoint{2.635144in}{1.448720in}}%
\pgfpathlineto{\pgfqpoint{2.634427in}{1.453744in}}%
\pgfpathlineto{\pgfqpoint{2.609773in}{1.449142in}}%
\pgfpathlineto{\pgfqpoint{2.608126in}{1.441492in}}%
\pgfpathclose%
\pgfusepath{fill}%
\end{pgfscope}%
\begin{pgfscope}%
\pgfpathrectangle{\pgfqpoint{0.100000in}{0.100000in}}{\pgfqpoint{3.007045in}{1.925000in}}%
\pgfusepath{clip}%
\pgfsetbuttcap%
\pgfsetmiterjoin%
\definecolor{currentfill}{rgb}{0.676817,0.816471,0.902376}%
\pgfsetfillcolor{currentfill}%
\pgfsetlinewidth{0.000000pt}%
\definecolor{currentstroke}{rgb}{0.000000,0.000000,0.000000}%
\pgfsetstrokecolor{currentstroke}%
\pgfsetstrokeopacity{0.000000}%
\pgfsetdash{}{0pt}%
\pgfpathmoveto{\pgfqpoint{2.507767in}{1.239655in}}%
\pgfpathlineto{\pgfqpoint{2.502865in}{1.238879in}}%
\pgfpathlineto{\pgfqpoint{2.498296in}{1.267386in}}%
\pgfpathlineto{\pgfqpoint{2.493292in}{1.298554in}}%
\pgfpathlineto{\pgfqpoint{2.491069in}{1.312382in}}%
\pgfpathlineto{\pgfqpoint{2.508763in}{1.315516in}}%
\pgfpathlineto{\pgfqpoint{2.506874in}{1.324911in}}%
\pgfpathlineto{\pgfqpoint{2.513140in}{1.337398in}}%
\pgfpathlineto{\pgfqpoint{2.528039in}{1.339978in}}%
\pgfpathlineto{\pgfqpoint{2.533624in}{1.328350in}}%
\pgfpathlineto{\pgfqpoint{2.540085in}{1.329055in}}%
\pgfpathlineto{\pgfqpoint{2.550029in}{1.332847in}}%
\pgfpathlineto{\pgfqpoint{2.552895in}{1.336088in}}%
\pgfpathlineto{\pgfqpoint{2.554469in}{1.327047in}}%
\pgfpathlineto{\pgfqpoint{2.574642in}{1.330275in}}%
\pgfpathlineto{\pgfqpoint{2.576932in}{1.318538in}}%
\pgfpathlineto{\pgfqpoint{2.574047in}{1.302973in}}%
\pgfpathlineto{\pgfqpoint{2.567920in}{1.283023in}}%
\pgfpathlineto{\pgfqpoint{2.561814in}{1.273283in}}%
\pgfpathlineto{\pgfqpoint{2.556847in}{1.259865in}}%
\pgfpathlineto{\pgfqpoint{2.559394in}{1.254953in}}%
\pgfpathlineto{\pgfqpoint{2.559174in}{1.248298in}}%
\pgfpathlineto{\pgfqpoint{2.554976in}{1.247539in}}%
\pgfpathlineto{\pgfqpoint{2.507767in}{1.239655in}}%
\pgfpathclose%
\pgfusepath{fill}%
\end{pgfscope}%
\begin{pgfscope}%
\pgfpathrectangle{\pgfqpoint{0.100000in}{0.100000in}}{\pgfqpoint{3.007045in}{1.925000in}}%
\pgfusepath{clip}%
\pgfsetbuttcap%
\pgfsetmiterjoin%
\definecolor{currentfill}{rgb}{0.610980,0.787420,0.880492}%
\pgfsetfillcolor{currentfill}%
\pgfsetlinewidth{0.000000pt}%
\definecolor{currentstroke}{rgb}{0.000000,0.000000,0.000000}%
\pgfsetstrokecolor{currentstroke}%
\pgfsetstrokeopacity{0.000000}%
\pgfsetdash{}{0pt}%
\pgfpathmoveto{\pgfqpoint{1.506378in}{0.538351in}}%
\pgfpathlineto{\pgfqpoint{1.501272in}{0.538689in}}%
\pgfpathlineto{\pgfqpoint{1.501924in}{0.549696in}}%
\pgfpathlineto{\pgfqpoint{1.498020in}{0.549906in}}%
\pgfpathlineto{\pgfqpoint{1.498977in}{0.564304in}}%
\pgfpathlineto{\pgfqpoint{1.524943in}{0.562745in}}%
\pgfpathlineto{\pgfqpoint{1.525500in}{0.576932in}}%
\pgfpathlineto{\pgfqpoint{1.544968in}{0.575956in}}%
\pgfpathlineto{\pgfqpoint{1.566327in}{0.575160in}}%
\pgfpathlineto{\pgfqpoint{1.565594in}{0.551126in}}%
\pgfpathlineto{\pgfqpoint{1.546459in}{0.551872in}}%
\pgfpathlineto{\pgfqpoint{1.545579in}{0.528102in}}%
\pgfpathlineto{\pgfqpoint{1.530916in}{0.536323in}}%
\pgfpathlineto{\pgfqpoint{1.506378in}{0.538351in}}%
\pgfpathclose%
\pgfusepath{fill}%
\end{pgfscope}%
\begin{pgfscope}%
\pgfpathrectangle{\pgfqpoint{0.100000in}{0.100000in}}{\pgfqpoint{3.007045in}{1.925000in}}%
\pgfusepath{clip}%
\pgfsetbuttcap%
\pgfsetmiterjoin%
\definecolor{currentfill}{rgb}{0.386820,0.659962,0.826421}%
\pgfsetfillcolor{currentfill}%
\pgfsetlinewidth{0.000000pt}%
\definecolor{currentstroke}{rgb}{0.000000,0.000000,0.000000}%
\pgfsetstrokecolor{currentstroke}%
\pgfsetstrokeopacity{0.000000}%
\pgfsetdash{}{0pt}%
\pgfpathmoveto{\pgfqpoint{2.408663in}{0.823968in}}%
\pgfpathlineto{\pgfqpoint{2.414961in}{0.829005in}}%
\pgfpathlineto{\pgfqpoint{2.424340in}{0.826019in}}%
\pgfpathlineto{\pgfqpoint{2.430059in}{0.826283in}}%
\pgfpathlineto{\pgfqpoint{2.430794in}{0.832351in}}%
\pgfpathlineto{\pgfqpoint{2.435742in}{0.842463in}}%
\pgfpathlineto{\pgfqpoint{2.440442in}{0.846333in}}%
\pgfpathlineto{\pgfqpoint{2.447000in}{0.848563in}}%
\pgfpathlineto{\pgfqpoint{2.452246in}{0.846939in}}%
\pgfpathlineto{\pgfqpoint{2.468037in}{0.838774in}}%
\pgfpathlineto{\pgfqpoint{2.473536in}{0.831913in}}%
\pgfpathlineto{\pgfqpoint{2.468621in}{0.832178in}}%
\pgfpathlineto{\pgfqpoint{2.460350in}{0.826839in}}%
\pgfpathlineto{\pgfqpoint{2.452424in}{0.825463in}}%
\pgfpathlineto{\pgfqpoint{2.449159in}{0.822767in}}%
\pgfpathlineto{\pgfqpoint{2.446193in}{0.816402in}}%
\pgfpathlineto{\pgfqpoint{2.442122in}{0.815272in}}%
\pgfpathlineto{\pgfqpoint{2.441136in}{0.811210in}}%
\pgfpathlineto{\pgfqpoint{2.432103in}{0.811727in}}%
\pgfpathlineto{\pgfqpoint{2.424800in}{0.803148in}}%
\pgfpathlineto{\pgfqpoint{2.417379in}{0.810753in}}%
\pgfpathlineto{\pgfqpoint{2.408663in}{0.823968in}}%
\pgfpathclose%
\pgfusepath{fill}%
\end{pgfscope}%
\begin{pgfscope}%
\pgfpathrectangle{\pgfqpoint{0.100000in}{0.100000in}}{\pgfqpoint{3.007045in}{1.925000in}}%
\pgfusepath{clip}%
\pgfsetbuttcap%
\pgfsetmiterjoin%
\definecolor{currentfill}{rgb}{0.447843,0.697855,0.845306}%
\pgfsetfillcolor{currentfill}%
\pgfsetlinewidth{0.000000pt}%
\definecolor{currentstroke}{rgb}{0.000000,0.000000,0.000000}%
\pgfsetstrokecolor{currentstroke}%
\pgfsetstrokeopacity{0.000000}%
\pgfsetdash{}{0pt}%
\pgfpathmoveto{\pgfqpoint{1.854945in}{1.201835in}}%
\pgfpathlineto{\pgfqpoint{1.855015in}{1.197513in}}%
\pgfpathlineto{\pgfqpoint{1.834849in}{1.196667in}}%
\pgfpathlineto{\pgfqpoint{1.834576in}{1.208087in}}%
\pgfpathlineto{\pgfqpoint{1.833661in}{1.237372in}}%
\pgfpathlineto{\pgfqpoint{1.844539in}{1.237709in}}%
\pgfpathlineto{\pgfqpoint{1.853670in}{1.237862in}}%
\pgfpathlineto{\pgfqpoint{1.854945in}{1.201835in}}%
\pgfpathclose%
\pgfusepath{fill}%
\end{pgfscope}%
\begin{pgfscope}%
\pgfpathrectangle{\pgfqpoint{0.100000in}{0.100000in}}{\pgfqpoint{3.007045in}{1.925000in}}%
\pgfusepath{clip}%
\pgfsetbuttcap%
\pgfsetmiterjoin%
\definecolor{currentfill}{rgb}{0.472941,0.711634,0.850719}%
\pgfsetfillcolor{currentfill}%
\pgfsetlinewidth{0.000000pt}%
\definecolor{currentstroke}{rgb}{0.000000,0.000000,0.000000}%
\pgfsetstrokecolor{currentstroke}%
\pgfsetstrokeopacity{0.000000}%
\pgfsetdash{}{0pt}%
\pgfpathmoveto{\pgfqpoint{1.538952in}{0.605702in}}%
\pgfpathlineto{\pgfqpoint{1.516589in}{0.606731in}}%
\pgfpathlineto{\pgfqpoint{1.509816in}{0.607078in}}%
\pgfpathlineto{\pgfqpoint{1.510333in}{0.616812in}}%
\pgfpathlineto{\pgfqpoint{1.481225in}{0.618416in}}%
\pgfpathlineto{\pgfqpoint{1.483290in}{0.651066in}}%
\pgfpathlineto{\pgfqpoint{1.476325in}{0.651539in}}%
\pgfpathlineto{\pgfqpoint{1.478210in}{0.684741in}}%
\pgfpathlineto{\pgfqpoint{1.482882in}{0.684491in}}%
\pgfpathlineto{\pgfqpoint{1.507461in}{0.683119in}}%
\pgfpathlineto{\pgfqpoint{1.505217in}{0.648714in}}%
\pgfpathlineto{\pgfqpoint{1.510361in}{0.642115in}}%
\pgfpathlineto{\pgfqpoint{1.520806in}{0.642095in}}%
\pgfpathlineto{\pgfqpoint{1.527773in}{0.637409in}}%
\pgfpathlineto{\pgfqpoint{1.532433in}{0.642045in}}%
\pgfpathlineto{\pgfqpoint{1.540604in}{0.640218in}}%
\pgfpathlineto{\pgfqpoint{1.538952in}{0.605702in}}%
\pgfpathclose%
\pgfusepath{fill}%
\end{pgfscope}%
\begin{pgfscope}%
\pgfpathrectangle{\pgfqpoint{0.100000in}{0.100000in}}{\pgfqpoint{3.007045in}{1.925000in}}%
\pgfusepath{clip}%
\pgfsetbuttcap%
\pgfsetmiterjoin%
\definecolor{currentfill}{rgb}{0.265759,0.577286,0.779177}%
\pgfsetfillcolor{currentfill}%
\pgfsetlinewidth{0.000000pt}%
\definecolor{currentstroke}{rgb}{0.000000,0.000000,0.000000}%
\pgfsetstrokecolor{currentstroke}%
\pgfsetstrokeopacity{0.000000}%
\pgfsetdash{}{0pt}%
\pgfpathmoveto{\pgfqpoint{2.076881in}{0.613497in}}%
\pgfpathlineto{\pgfqpoint{2.075185in}{0.642117in}}%
\pgfpathlineto{\pgfqpoint{2.065550in}{0.641578in}}%
\pgfpathlineto{\pgfqpoint{2.065780in}{0.638658in}}%
\pgfpathlineto{\pgfqpoint{2.058100in}{0.638257in}}%
\pgfpathlineto{\pgfqpoint{2.056646in}{0.646958in}}%
\pgfpathlineto{\pgfqpoint{2.055851in}{0.662822in}}%
\pgfpathlineto{\pgfqpoint{2.073893in}{0.665116in}}%
\pgfpathlineto{\pgfqpoint{2.088016in}{0.667046in}}%
\pgfpathlineto{\pgfqpoint{2.089615in}{0.643002in}}%
\pgfpathlineto{\pgfqpoint{2.121151in}{0.645024in}}%
\pgfpathlineto{\pgfqpoint{2.123918in}{0.610337in}}%
\pgfpathlineto{\pgfqpoint{2.095084in}{0.608430in}}%
\pgfpathlineto{\pgfqpoint{2.094715in}{0.614481in}}%
\pgfpathlineto{\pgfqpoint{2.085520in}{0.614773in}}%
\pgfpathlineto{\pgfqpoint{2.076881in}{0.613497in}}%
\pgfpathclose%
\pgfusepath{fill}%
\end{pgfscope}%
\begin{pgfscope}%
\pgfpathrectangle{\pgfqpoint{0.100000in}{0.100000in}}{\pgfqpoint{3.007045in}{1.925000in}}%
\pgfusepath{clip}%
\pgfsetbuttcap%
\pgfsetmiterjoin%
\definecolor{currentfill}{rgb}{0.376732,0.653072,0.822484}%
\pgfsetfillcolor{currentfill}%
\pgfsetlinewidth{0.000000pt}%
\definecolor{currentstroke}{rgb}{0.000000,0.000000,0.000000}%
\pgfsetstrokecolor{currentstroke}%
\pgfsetstrokeopacity{0.000000}%
\pgfsetdash{}{0pt}%
\pgfpathmoveto{\pgfqpoint{2.624882in}{1.186857in}}%
\pgfpathlineto{\pgfqpoint{2.625996in}{1.195904in}}%
\pgfpathlineto{\pgfqpoint{2.620005in}{1.199545in}}%
\pgfpathlineto{\pgfqpoint{2.622986in}{1.207692in}}%
\pgfpathlineto{\pgfqpoint{2.621536in}{1.211734in}}%
\pgfpathlineto{\pgfqpoint{2.634232in}{1.214856in}}%
\pgfpathlineto{\pgfqpoint{2.637469in}{1.212817in}}%
\pgfpathlineto{\pgfqpoint{2.643347in}{1.220746in}}%
\pgfpathlineto{\pgfqpoint{2.644431in}{1.224390in}}%
\pgfpathlineto{\pgfqpoint{2.646870in}{1.236936in}}%
\pgfpathlineto{\pgfqpoint{2.649600in}{1.238259in}}%
\pgfpathlineto{\pgfqpoint{2.651264in}{1.258406in}}%
\pgfpathlineto{\pgfqpoint{2.655502in}{1.265959in}}%
\pgfpathlineto{\pgfqpoint{2.667563in}{1.268330in}}%
\pgfpathlineto{\pgfqpoint{2.664785in}{1.260556in}}%
\pgfpathlineto{\pgfqpoint{2.672090in}{1.258055in}}%
\pgfpathlineto{\pgfqpoint{2.675901in}{1.254838in}}%
\pgfpathlineto{\pgfqpoint{2.673952in}{1.243959in}}%
\pgfpathlineto{\pgfqpoint{2.677394in}{1.240185in}}%
\pgfpathlineto{\pgfqpoint{2.684505in}{1.237247in}}%
\pgfpathlineto{\pgfqpoint{2.686960in}{1.233890in}}%
\pgfpathlineto{\pgfqpoint{2.703023in}{1.225995in}}%
\pgfpathlineto{\pgfqpoint{2.705509in}{1.221124in}}%
\pgfpathlineto{\pgfqpoint{2.704557in}{1.214940in}}%
\pgfpathlineto{\pgfqpoint{2.706153in}{1.210930in}}%
\pgfpathlineto{\pgfqpoint{2.715119in}{1.210272in}}%
\pgfpathlineto{\pgfqpoint{2.717999in}{1.199175in}}%
\pgfpathlineto{\pgfqpoint{2.726182in}{1.190843in}}%
\pgfpathlineto{\pgfqpoint{2.729054in}{1.179712in}}%
\pgfpathlineto{\pgfqpoint{2.733376in}{1.175355in}}%
\pgfpathlineto{\pgfqpoint{2.732254in}{1.169909in}}%
\pgfpathlineto{\pgfqpoint{2.727131in}{1.173840in}}%
\pgfpathlineto{\pgfqpoint{2.714351in}{1.177897in}}%
\pgfpathlineto{\pgfqpoint{2.708831in}{1.176381in}}%
\pgfpathlineto{\pgfqpoint{2.695566in}{1.180728in}}%
\pgfpathlineto{\pgfqpoint{2.691552in}{1.182331in}}%
\pgfpathlineto{\pgfqpoint{2.682867in}{1.178238in}}%
\pgfpathlineto{\pgfqpoint{2.680316in}{1.180573in}}%
\pgfpathlineto{\pgfqpoint{2.678549in}{1.188294in}}%
\pgfpathlineto{\pgfqpoint{2.668861in}{1.192884in}}%
\pgfpathlineto{\pgfqpoint{2.662595in}{1.183012in}}%
\pgfpathlineto{\pgfqpoint{2.663143in}{1.179525in}}%
\pgfpathlineto{\pgfqpoint{2.658001in}{1.178784in}}%
\pgfpathlineto{\pgfqpoint{2.648785in}{1.188481in}}%
\pgfpathlineto{\pgfqpoint{2.644402in}{1.195106in}}%
\pgfpathlineto{\pgfqpoint{2.634662in}{1.181683in}}%
\pgfpathlineto{\pgfqpoint{2.631429in}{1.181706in}}%
\pgfpathlineto{\pgfqpoint{2.624882in}{1.186857in}}%
\pgfpathclose%
\pgfusepath{fill}%
\end{pgfscope}%
\begin{pgfscope}%
\pgfpathrectangle{\pgfqpoint{0.100000in}{0.100000in}}{\pgfqpoint{3.007045in}{1.925000in}}%
\pgfusepath{clip}%
\pgfsetbuttcap%
\pgfsetmiterjoin%
\definecolor{currentfill}{rgb}{0.681738,0.818562,0.904098}%
\pgfsetfillcolor{currentfill}%
\pgfsetlinewidth{0.000000pt}%
\definecolor{currentstroke}{rgb}{0.000000,0.000000,0.000000}%
\pgfsetstrokecolor{currentstroke}%
\pgfsetstrokeopacity{0.000000}%
\pgfsetdash{}{0pt}%
\pgfpathmoveto{\pgfqpoint{0.521615in}{1.353758in}}%
\pgfpathlineto{\pgfqpoint{0.516565in}{1.336204in}}%
\pgfpathlineto{\pgfqpoint{0.492015in}{1.343273in}}%
\pgfpathlineto{\pgfqpoint{0.489895in}{1.349175in}}%
\pgfpathlineto{\pgfqpoint{0.486181in}{1.350484in}}%
\pgfpathlineto{\pgfqpoint{0.480726in}{1.347606in}}%
\pgfpathlineto{\pgfqpoint{0.465761in}{1.347164in}}%
\pgfpathlineto{\pgfqpoint{0.467458in}{1.355133in}}%
\pgfpathlineto{\pgfqpoint{0.457960in}{1.381368in}}%
\pgfpathlineto{\pgfqpoint{0.455804in}{1.383109in}}%
\pgfpathlineto{\pgfqpoint{0.458164in}{1.392328in}}%
\pgfpathlineto{\pgfqpoint{0.461995in}{1.397568in}}%
\pgfpathlineto{\pgfqpoint{0.459905in}{1.401168in}}%
\pgfpathlineto{\pgfqpoint{0.464449in}{1.404081in}}%
\pgfpathlineto{\pgfqpoint{0.466986in}{1.410310in}}%
\pgfpathlineto{\pgfqpoint{0.462117in}{1.414113in}}%
\pgfpathlineto{\pgfqpoint{0.462527in}{1.420581in}}%
\pgfpathlineto{\pgfqpoint{0.470680in}{1.418127in}}%
\pgfpathlineto{\pgfqpoint{0.483507in}{1.414404in}}%
\pgfpathlineto{\pgfqpoint{0.479961in}{1.402391in}}%
\pgfpathlineto{\pgfqpoint{0.482087in}{1.399330in}}%
\pgfpathlineto{\pgfqpoint{0.488838in}{1.398768in}}%
\pgfpathlineto{\pgfqpoint{0.495392in}{1.401958in}}%
\pgfpathlineto{\pgfqpoint{0.503882in}{1.397397in}}%
\pgfpathlineto{\pgfqpoint{0.512166in}{1.383346in}}%
\pgfpathlineto{\pgfqpoint{0.518020in}{1.379672in}}%
\pgfpathlineto{\pgfqpoint{0.517118in}{1.374961in}}%
\pgfpathlineto{\pgfqpoint{0.520278in}{1.368998in}}%
\pgfpathlineto{\pgfqpoint{0.517066in}{1.358013in}}%
\pgfpathlineto{\pgfqpoint{0.521615in}{1.353758in}}%
\pgfpathclose%
\pgfusepath{fill}%
\end{pgfscope}%
\begin{pgfscope}%
\pgfpathrectangle{\pgfqpoint{0.100000in}{0.100000in}}{\pgfqpoint{3.007045in}{1.925000in}}%
\pgfusepath{clip}%
\pgfsetbuttcap%
\pgfsetmiterjoin%
\definecolor{currentfill}{rgb}{0.290980,0.594510,0.789020}%
\pgfsetfillcolor{currentfill}%
\pgfsetlinewidth{0.000000pt}%
\definecolor{currentstroke}{rgb}{0.000000,0.000000,0.000000}%
\pgfsetstrokecolor{currentstroke}%
\pgfsetstrokeopacity{0.000000}%
\pgfsetdash{}{0pt}%
\pgfpathmoveto{\pgfqpoint{1.590530in}{1.433489in}}%
\pgfpathlineto{\pgfqpoint{1.591111in}{1.462290in}}%
\pgfpathlineto{\pgfqpoint{1.613769in}{1.461576in}}%
\pgfpathlineto{\pgfqpoint{1.614383in}{1.478679in}}%
\pgfpathlineto{\pgfqpoint{1.637041in}{1.477807in}}%
\pgfpathlineto{\pgfqpoint{1.637239in}{1.477803in}}%
\pgfpathlineto{\pgfqpoint{1.636440in}{1.454988in}}%
\pgfpathlineto{\pgfqpoint{1.648008in}{1.454576in}}%
\pgfpathlineto{\pgfqpoint{1.647409in}{1.431503in}}%
\pgfpathlineto{\pgfqpoint{1.590530in}{1.433489in}}%
\pgfpathclose%
\pgfusepath{fill}%
\end{pgfscope}%
\begin{pgfscope}%
\pgfpathrectangle{\pgfqpoint{0.100000in}{0.100000in}}{\pgfqpoint{3.007045in}{1.925000in}}%
\pgfusepath{clip}%
\pgfsetbuttcap%
\pgfsetmiterjoin%
\definecolor{currentfill}{rgb}{0.472941,0.711634,0.850719}%
\pgfsetfillcolor{currentfill}%
\pgfsetlinewidth{0.000000pt}%
\definecolor{currentstroke}{rgb}{0.000000,0.000000,0.000000}%
\pgfsetstrokecolor{currentstroke}%
\pgfsetstrokeopacity{0.000000}%
\pgfsetdash{}{0pt}%
\pgfpathmoveto{\pgfqpoint{0.797625in}{1.916024in}}%
\pgfpathlineto{\pgfqpoint{0.789229in}{1.906864in}}%
\pgfpathlineto{\pgfqpoint{0.787685in}{1.900755in}}%
\pgfpathlineto{\pgfqpoint{0.783449in}{1.895856in}}%
\pgfpathlineto{\pgfqpoint{0.789000in}{1.894475in}}%
\pgfpathlineto{\pgfqpoint{0.780353in}{1.860745in}}%
\pgfpathlineto{\pgfqpoint{0.777516in}{1.855475in}}%
\pgfpathlineto{\pgfqpoint{0.773351in}{1.839140in}}%
\pgfpathlineto{\pgfqpoint{0.768647in}{1.844176in}}%
\pgfpathlineto{\pgfqpoint{0.762060in}{1.843258in}}%
\pgfpathlineto{\pgfqpoint{0.759267in}{1.846349in}}%
\pgfpathlineto{\pgfqpoint{0.749934in}{1.845470in}}%
\pgfpathlineto{\pgfqpoint{0.748542in}{1.851631in}}%
\pgfpathlineto{\pgfqpoint{0.744909in}{1.856652in}}%
\pgfpathlineto{\pgfqpoint{0.735184in}{1.852084in}}%
\pgfpathlineto{\pgfqpoint{0.730840in}{1.859613in}}%
\pgfpathlineto{\pgfqpoint{0.719784in}{1.864006in}}%
\pgfpathlineto{\pgfqpoint{0.713636in}{1.864996in}}%
\pgfpathlineto{\pgfqpoint{0.713573in}{1.866288in}}%
\pgfpathlineto{\pgfqpoint{0.713541in}{1.878149in}}%
\pgfpathlineto{\pgfqpoint{0.710511in}{1.880181in}}%
\pgfpathlineto{\pgfqpoint{0.699664in}{1.876125in}}%
\pgfpathlineto{\pgfqpoint{0.695862in}{1.879153in}}%
\pgfpathlineto{\pgfqpoint{0.685273in}{1.876771in}}%
\pgfpathlineto{\pgfqpoint{0.683348in}{1.884220in}}%
\pgfpathlineto{\pgfqpoint{0.676269in}{1.883139in}}%
\pgfpathlineto{\pgfqpoint{0.675310in}{1.876711in}}%
\pgfpathlineto{\pgfqpoint{0.667442in}{1.879016in}}%
\pgfpathlineto{\pgfqpoint{0.666800in}{1.883212in}}%
\pgfpathlineto{\pgfqpoint{0.657890in}{1.895448in}}%
\pgfpathlineto{\pgfqpoint{0.659239in}{1.897832in}}%
\pgfpathlineto{\pgfqpoint{0.654234in}{1.907098in}}%
\pgfpathlineto{\pgfqpoint{0.650154in}{1.917880in}}%
\pgfpathlineto{\pgfqpoint{0.650253in}{1.923532in}}%
\pgfpathlineto{\pgfqpoint{0.648297in}{1.930642in}}%
\pgfpathlineto{\pgfqpoint{0.650462in}{1.932235in}}%
\pgfpathlineto{\pgfqpoint{0.654305in}{1.933747in}}%
\pgfpathlineto{\pgfqpoint{0.652693in}{1.955469in}}%
\pgfpathlineto{\pgfqpoint{0.684552in}{1.946256in}}%
\pgfpathlineto{\pgfqpoint{0.723267in}{1.935529in}}%
\pgfpathlineto{\pgfqpoint{0.752077in}{1.927793in}}%
\pgfpathlineto{\pgfqpoint{0.797625in}{1.916024in}}%
\pgfpathclose%
\pgfusepath{fill}%
\end{pgfscope}%
\begin{pgfscope}%
\pgfpathrectangle{\pgfqpoint{0.100000in}{0.100000in}}{\pgfqpoint{3.007045in}{1.925000in}}%
\pgfusepath{clip}%
\pgfsetbuttcap%
\pgfsetmiterjoin%
\definecolor{currentfill}{rgb}{0.396909,0.666851,0.830358}%
\pgfsetfillcolor{currentfill}%
\pgfsetlinewidth{0.000000pt}%
\definecolor{currentstroke}{rgb}{0.000000,0.000000,0.000000}%
\pgfsetstrokecolor{currentstroke}%
\pgfsetstrokeopacity{0.000000}%
\pgfsetdash{}{0pt}%
\pgfpathmoveto{\pgfqpoint{2.367977in}{0.755519in}}%
\pgfpathlineto{\pgfqpoint{2.361889in}{0.757434in}}%
\pgfpathlineto{\pgfqpoint{2.354579in}{0.763090in}}%
\pgfpathlineto{\pgfqpoint{2.352032in}{0.771233in}}%
\pgfpathlineto{\pgfqpoint{2.352426in}{0.785269in}}%
\pgfpathlineto{\pgfqpoint{2.352410in}{0.787703in}}%
\pgfpathlineto{\pgfqpoint{2.355976in}{0.788053in}}%
\pgfpathlineto{\pgfqpoint{2.357695in}{0.794657in}}%
\pgfpathlineto{\pgfqpoint{2.365423in}{0.795571in}}%
\pgfpathlineto{\pgfqpoint{2.375369in}{0.793380in}}%
\pgfpathlineto{\pgfqpoint{2.382395in}{0.804196in}}%
\pgfpathlineto{\pgfqpoint{2.386212in}{0.799739in}}%
\pgfpathlineto{\pgfqpoint{2.390080in}{0.791606in}}%
\pgfpathlineto{\pgfqpoint{2.390153in}{0.787078in}}%
\pgfpathlineto{\pgfqpoint{2.387448in}{0.788188in}}%
\pgfpathlineto{\pgfqpoint{2.377909in}{0.787069in}}%
\pgfpathlineto{\pgfqpoint{2.379793in}{0.770306in}}%
\pgfpathlineto{\pgfqpoint{2.375546in}{0.768744in}}%
\pgfpathlineto{\pgfqpoint{2.375145in}{0.759957in}}%
\pgfpathlineto{\pgfqpoint{2.373071in}{0.752288in}}%
\pgfpathlineto{\pgfqpoint{2.367977in}{0.755519in}}%
\pgfpathclose%
\pgfusepath{fill}%
\end{pgfscope}%
\begin{pgfscope}%
\pgfpathrectangle{\pgfqpoint{0.100000in}{0.100000in}}{\pgfqpoint{3.007045in}{1.925000in}}%
\pgfusepath{clip}%
\pgfsetbuttcap%
\pgfsetmiterjoin%
\definecolor{currentfill}{rgb}{0.417086,0.680631,0.838231}%
\pgfsetfillcolor{currentfill}%
\pgfsetlinewidth{0.000000pt}%
\definecolor{currentstroke}{rgb}{0.000000,0.000000,0.000000}%
\pgfsetstrokecolor{currentstroke}%
\pgfsetstrokeopacity{0.000000}%
\pgfsetdash{}{0pt}%
\pgfpathmoveto{\pgfqpoint{2.715119in}{1.210272in}}%
\pgfpathlineto{\pgfqpoint{2.706153in}{1.210930in}}%
\pgfpathlineto{\pgfqpoint{2.704557in}{1.214940in}}%
\pgfpathlineto{\pgfqpoint{2.705509in}{1.221124in}}%
\pgfpathlineto{\pgfqpoint{2.703023in}{1.225995in}}%
\pgfpathlineto{\pgfqpoint{2.686960in}{1.233890in}}%
\pgfpathlineto{\pgfqpoint{2.684505in}{1.237247in}}%
\pgfpathlineto{\pgfqpoint{2.677394in}{1.240185in}}%
\pgfpathlineto{\pgfqpoint{2.673952in}{1.243959in}}%
\pgfpathlineto{\pgfqpoint{2.675901in}{1.254838in}}%
\pgfpathlineto{\pgfqpoint{2.672090in}{1.258055in}}%
\pgfpathlineto{\pgfqpoint{2.664785in}{1.260556in}}%
\pgfpathlineto{\pgfqpoint{2.667563in}{1.268330in}}%
\pgfpathlineto{\pgfqpoint{2.716037in}{1.278304in}}%
\pgfpathlineto{\pgfqpoint{2.716371in}{1.278367in}}%
\pgfpathlineto{\pgfqpoint{2.726638in}{1.268556in}}%
\pgfpathlineto{\pgfqpoint{2.726766in}{1.261212in}}%
\pgfpathlineto{\pgfqpoint{2.723696in}{1.256764in}}%
\pgfpathlineto{\pgfqpoint{2.715251in}{1.250311in}}%
\pgfpathlineto{\pgfqpoint{2.713010in}{1.242278in}}%
\pgfpathlineto{\pgfqpoint{2.715503in}{1.232898in}}%
\pgfpathlineto{\pgfqpoint{2.715994in}{1.223495in}}%
\pgfpathlineto{\pgfqpoint{2.713299in}{1.211895in}}%
\pgfpathlineto{\pgfqpoint{2.715119in}{1.210272in}}%
\pgfpathclose%
\pgfusepath{fill}%
\end{pgfscope}%
\begin{pgfscope}%
\pgfpathrectangle{\pgfqpoint{0.100000in}{0.100000in}}{\pgfqpoint{3.007045in}{1.925000in}}%
\pgfusepath{clip}%
\pgfsetbuttcap%
\pgfsetmiterjoin%
\definecolor{currentfill}{rgb}{0.301069,0.601399,0.792957}%
\pgfsetfillcolor{currentfill}%
\pgfsetlinewidth{0.000000pt}%
\definecolor{currentstroke}{rgb}{0.000000,0.000000,0.000000}%
\pgfsetstrokecolor{currentstroke}%
\pgfsetstrokeopacity{0.000000}%
\pgfsetdash{}{0pt}%
\pgfpathmoveto{\pgfqpoint{1.534859in}{1.055660in}}%
\pgfpathlineto{\pgfqpoint{1.534600in}{1.061649in}}%
\pgfpathlineto{\pgfqpoint{1.536034in}{1.090382in}}%
\pgfpathlineto{\pgfqpoint{1.535329in}{1.090418in}}%
\pgfpathlineto{\pgfqpoint{1.536726in}{1.118994in}}%
\pgfpathlineto{\pgfqpoint{1.593165in}{1.116517in}}%
\pgfpathlineto{\pgfqpoint{1.591676in}{1.059068in}}%
\pgfpathlineto{\pgfqpoint{1.563552in}{1.060256in}}%
\pgfpathlineto{\pgfqpoint{1.563272in}{1.054193in}}%
\pgfpathlineto{\pgfqpoint{1.534859in}{1.055660in}}%
\pgfpathclose%
\pgfusepath{fill}%
\end{pgfscope}%
\begin{pgfscope}%
\pgfpathrectangle{\pgfqpoint{0.100000in}{0.100000in}}{\pgfqpoint{3.007045in}{1.925000in}}%
\pgfusepath{clip}%
\pgfsetbuttcap%
\pgfsetmiterjoin%
\definecolor{currentfill}{rgb}{0.711265,0.831111,0.914433}%
\pgfsetfillcolor{currentfill}%
\pgfsetlinewidth{0.000000pt}%
\definecolor{currentstroke}{rgb}{0.000000,0.000000,0.000000}%
\pgfsetstrokecolor{currentstroke}%
\pgfsetstrokeopacity{0.000000}%
\pgfsetdash{}{0pt}%
\pgfpathmoveto{\pgfqpoint{2.204865in}{1.518579in}}%
\pgfpathlineto{\pgfqpoint{2.185259in}{1.517003in}}%
\pgfpathlineto{\pgfqpoint{2.183604in}{1.529014in}}%
\pgfpathlineto{\pgfqpoint{2.190595in}{1.532298in}}%
\pgfpathlineto{\pgfqpoint{2.190444in}{1.542079in}}%
\pgfpathlineto{\pgfqpoint{2.194250in}{1.543292in}}%
\pgfpathlineto{\pgfqpoint{2.196243in}{1.547670in}}%
\pgfpathlineto{\pgfqpoint{2.202517in}{1.547510in}}%
\pgfpathlineto{\pgfqpoint{2.204610in}{1.554457in}}%
\pgfpathlineto{\pgfqpoint{2.210201in}{1.559918in}}%
\pgfpathlineto{\pgfqpoint{2.211986in}{1.556493in}}%
\pgfpathlineto{\pgfqpoint{2.212190in}{1.546743in}}%
\pgfpathlineto{\pgfqpoint{2.210346in}{1.543780in}}%
\pgfpathlineto{\pgfqpoint{2.211388in}{1.536695in}}%
\pgfpathlineto{\pgfqpoint{2.216884in}{1.535425in}}%
\pgfpathlineto{\pgfqpoint{2.221340in}{1.546582in}}%
\pgfpathlineto{\pgfqpoint{2.222009in}{1.560126in}}%
\pgfpathlineto{\pgfqpoint{2.220169in}{1.566322in}}%
\pgfpathlineto{\pgfqpoint{2.227849in}{1.567100in}}%
\pgfpathlineto{\pgfqpoint{2.228428in}{1.561303in}}%
\pgfpathlineto{\pgfqpoint{2.245496in}{1.563087in}}%
\pgfpathlineto{\pgfqpoint{2.247824in}{1.546163in}}%
\pgfpathlineto{\pgfqpoint{2.250233in}{1.523286in}}%
\pgfpathlineto{\pgfqpoint{2.204865in}{1.518579in}}%
\pgfpathclose%
\pgfusepath{fill}%
\end{pgfscope}%
\begin{pgfscope}%
\pgfpathrectangle{\pgfqpoint{0.100000in}{0.100000in}}{\pgfqpoint{3.007045in}{1.925000in}}%
\pgfusepath{clip}%
\pgfsetbuttcap%
\pgfsetmiterjoin%
\definecolor{currentfill}{rgb}{0.316201,0.611734,0.798862}%
\pgfsetfillcolor{currentfill}%
\pgfsetlinewidth{0.000000pt}%
\definecolor{currentstroke}{rgb}{0.000000,0.000000,0.000000}%
\pgfsetstrokecolor{currentstroke}%
\pgfsetstrokeopacity{0.000000}%
\pgfsetdash{}{0pt}%
\pgfpathmoveto{\pgfqpoint{1.541684in}{1.689945in}}%
\pgfpathlineto{\pgfqpoint{1.540344in}{1.666994in}}%
\pgfpathlineto{\pgfqpoint{1.542129in}{1.666889in}}%
\pgfpathlineto{\pgfqpoint{1.540553in}{1.643789in}}%
\pgfpathlineto{\pgfqpoint{1.514103in}{1.645425in}}%
\pgfpathlineto{\pgfqpoint{1.518209in}{1.642479in}}%
\pgfpathlineto{\pgfqpoint{1.519039in}{1.638417in}}%
\pgfpathlineto{\pgfqpoint{1.516525in}{1.631523in}}%
\pgfpathlineto{\pgfqpoint{1.506325in}{1.629275in}}%
\pgfpathlineto{\pgfqpoint{1.503158in}{1.631150in}}%
\pgfpathlineto{\pgfqpoint{1.495994in}{1.623275in}}%
\pgfpathlineto{\pgfqpoint{1.495332in}{1.629121in}}%
\pgfpathlineto{\pgfqpoint{1.483880in}{1.629904in}}%
\pgfpathlineto{\pgfqpoint{1.485073in}{1.647065in}}%
\pgfpathlineto{\pgfqpoint{1.465987in}{1.648477in}}%
\pgfpathlineto{\pgfqpoint{1.466420in}{1.654245in}}%
\pgfpathlineto{\pgfqpoint{1.449237in}{1.655533in}}%
\pgfpathlineto{\pgfqpoint{1.450582in}{1.673001in}}%
\pgfpathlineto{\pgfqpoint{1.448415in}{1.673164in}}%
\pgfpathlineto{\pgfqpoint{1.450238in}{1.696191in}}%
\pgfpathlineto{\pgfqpoint{1.447485in}{1.696400in}}%
\pgfpathlineto{\pgfqpoint{1.448796in}{1.712766in}}%
\pgfpathlineto{\pgfqpoint{1.446373in}{1.716181in}}%
\pgfpathlineto{\pgfqpoint{1.447405in}{1.726710in}}%
\pgfpathlineto{\pgfqpoint{1.441706in}{1.725605in}}%
\pgfpathlineto{\pgfqpoint{1.442195in}{1.731563in}}%
\pgfpathlineto{\pgfqpoint{1.465127in}{1.729765in}}%
\pgfpathlineto{\pgfqpoint{1.522557in}{1.725806in}}%
\pgfpathlineto{\pgfqpoint{1.521849in}{1.714257in}}%
\pgfpathlineto{\pgfqpoint{1.517941in}{1.714507in}}%
\pgfpathlineto{\pgfqpoint{1.516455in}{1.691466in}}%
\pgfpathlineto{\pgfqpoint{1.541684in}{1.689945in}}%
\pgfpathclose%
\pgfusepath{fill}%
\end{pgfscope}%
\begin{pgfscope}%
\pgfpathrectangle{\pgfqpoint{0.100000in}{0.100000in}}{\pgfqpoint{3.007045in}{1.925000in}}%
\pgfusepath{clip}%
\pgfsetbuttcap%
\pgfsetmiterjoin%
\definecolor{currentfill}{rgb}{0.479216,0.715079,0.852072}%
\pgfsetfillcolor{currentfill}%
\pgfsetlinewidth{0.000000pt}%
\definecolor{currentstroke}{rgb}{0.000000,0.000000,0.000000}%
\pgfsetstrokecolor{currentstroke}%
\pgfsetstrokeopacity{0.000000}%
\pgfsetdash{}{0pt}%
\pgfpathmoveto{\pgfqpoint{2.754436in}{1.271463in}}%
\pgfpathlineto{\pgfqpoint{2.754264in}{1.276110in}}%
\pgfpathlineto{\pgfqpoint{2.751124in}{1.279964in}}%
\pgfpathlineto{\pgfqpoint{2.753954in}{1.289095in}}%
\pgfpathlineto{\pgfqpoint{2.753598in}{1.293002in}}%
\pgfpathlineto{\pgfqpoint{2.746343in}{1.292558in}}%
\pgfpathlineto{\pgfqpoint{2.742511in}{1.290165in}}%
\pgfpathlineto{\pgfqpoint{2.739138in}{1.283230in}}%
\pgfpathlineto{\pgfqpoint{2.721216in}{1.279405in}}%
\pgfpathlineto{\pgfqpoint{2.724100in}{1.282597in}}%
\pgfpathlineto{\pgfqpoint{2.725778in}{1.289195in}}%
\pgfpathlineto{\pgfqpoint{2.725049in}{1.295828in}}%
\pgfpathlineto{\pgfqpoint{2.726798in}{1.302007in}}%
\pgfpathlineto{\pgfqpoint{2.725428in}{1.306338in}}%
\pgfpathlineto{\pgfqpoint{2.735728in}{1.317531in}}%
\pgfpathlineto{\pgfqpoint{2.741068in}{1.332531in}}%
\pgfpathlineto{\pgfqpoint{2.746311in}{1.336376in}}%
\pgfpathlineto{\pgfqpoint{2.755066in}{1.346497in}}%
\pgfpathlineto{\pgfqpoint{2.762277in}{1.343620in}}%
\pgfpathlineto{\pgfqpoint{2.764535in}{1.335742in}}%
\pgfpathlineto{\pgfqpoint{2.769299in}{1.335490in}}%
\pgfpathlineto{\pgfqpoint{2.777561in}{1.327350in}}%
\pgfpathlineto{\pgfqpoint{2.791685in}{1.322963in}}%
\pgfpathlineto{\pgfqpoint{2.806809in}{1.301783in}}%
\pgfpathlineto{\pgfqpoint{2.811323in}{1.284246in}}%
\pgfpathlineto{\pgfqpoint{2.815570in}{1.283815in}}%
\pgfpathlineto{\pgfqpoint{2.811785in}{1.275012in}}%
\pgfpathlineto{\pgfqpoint{2.803631in}{1.265614in}}%
\pgfpathlineto{\pgfqpoint{2.798237in}{1.247006in}}%
\pgfpathlineto{\pgfqpoint{2.795440in}{1.242918in}}%
\pgfpathlineto{\pgfqpoint{2.790081in}{1.241725in}}%
\pgfpathlineto{\pgfqpoint{2.790953in}{1.257015in}}%
\pgfpathlineto{\pgfqpoint{2.775479in}{1.258863in}}%
\pgfpathlineto{\pgfqpoint{2.767772in}{1.261597in}}%
\pgfpathlineto{\pgfqpoint{2.757096in}{1.268170in}}%
\pgfpathlineto{\pgfqpoint{2.754436in}{1.271463in}}%
\pgfpathclose%
\pgfusepath{fill}%
\end{pgfscope}%
\begin{pgfscope}%
\pgfpathrectangle{\pgfqpoint{0.100000in}{0.100000in}}{\pgfqpoint{3.007045in}{1.925000in}}%
\pgfusepath{clip}%
\pgfsetbuttcap%
\pgfsetmiterjoin%
\definecolor{currentfill}{rgb}{0.056363,0.349635,0.636755}%
\pgfsetfillcolor{currentfill}%
\pgfsetlinewidth{0.000000pt}%
\definecolor{currentstroke}{rgb}{0.000000,0.000000,0.000000}%
\pgfsetstrokecolor{currentstroke}%
\pgfsetstrokeopacity{0.000000}%
\pgfsetdash{}{0pt}%
\pgfpathmoveto{\pgfqpoint{1.454880in}{1.037434in}}%
\pgfpathlineto{\pgfqpoint{1.453227in}{1.011793in}}%
\pgfpathlineto{\pgfqpoint{1.427563in}{1.013412in}}%
\pgfpathlineto{\pgfqpoint{1.429358in}{1.039255in}}%
\pgfpathlineto{\pgfqpoint{1.404039in}{1.041199in}}%
\pgfpathlineto{\pgfqpoint{1.405343in}{1.057957in}}%
\pgfpathlineto{\pgfqpoint{1.405795in}{1.064151in}}%
\pgfpathlineto{\pgfqpoint{1.455281in}{1.060433in}}%
\pgfpathlineto{\pgfqpoint{1.453744in}{1.037504in}}%
\pgfpathlineto{\pgfqpoint{1.454880in}{1.037434in}}%
\pgfpathclose%
\pgfusepath{fill}%
\end{pgfscope}%
\begin{pgfscope}%
\pgfpathrectangle{\pgfqpoint{0.100000in}{0.100000in}}{\pgfqpoint{3.007045in}{1.925000in}}%
\pgfusepath{clip}%
\pgfsetbuttcap%
\pgfsetmiterjoin%
\definecolor{currentfill}{rgb}{0.270804,0.580730,0.781146}%
\pgfsetfillcolor{currentfill}%
\pgfsetlinewidth{0.000000pt}%
\definecolor{currentstroke}{rgb}{0.000000,0.000000,0.000000}%
\pgfsetstrokecolor{currentstroke}%
\pgfsetstrokeopacity{0.000000}%
\pgfsetdash{}{0pt}%
\pgfpathmoveto{\pgfqpoint{2.397002in}{0.647345in}}%
\pgfpathlineto{\pgfqpoint{2.397277in}{0.644995in}}%
\pgfpathlineto{\pgfqpoint{2.412441in}{0.646597in}}%
\pgfpathlineto{\pgfqpoint{2.415134in}{0.621502in}}%
\pgfpathlineto{\pgfqpoint{2.378692in}{0.619214in}}%
\pgfpathlineto{\pgfqpoint{2.375927in}{0.644931in}}%
\pgfpathlineto{\pgfqpoint{2.397002in}{0.647345in}}%
\pgfpathclose%
\pgfusepath{fill}%
\end{pgfscope}%
\begin{pgfscope}%
\pgfpathrectangle{\pgfqpoint{0.100000in}{0.100000in}}{\pgfqpoint{3.007045in}{1.925000in}}%
\pgfusepath{clip}%
\pgfsetbuttcap%
\pgfsetmiterjoin%
\definecolor{currentfill}{rgb}{0.523137,0.739193,0.861546}%
\pgfsetfillcolor{currentfill}%
\pgfsetlinewidth{0.000000pt}%
\definecolor{currentstroke}{rgb}{0.000000,0.000000,0.000000}%
\pgfsetstrokecolor{currentstroke}%
\pgfsetstrokeopacity{0.000000}%
\pgfsetdash{}{0pt}%
\pgfpathmoveto{\pgfqpoint{2.116398in}{1.073683in}}%
\pgfpathlineto{\pgfqpoint{2.099179in}{1.072352in}}%
\pgfpathlineto{\pgfqpoint{2.098280in}{1.086724in}}%
\pgfpathlineto{\pgfqpoint{2.075282in}{1.085153in}}%
\pgfpathlineto{\pgfqpoint{2.075029in}{1.090942in}}%
\pgfpathlineto{\pgfqpoint{2.046290in}{1.089601in}}%
\pgfpathlineto{\pgfqpoint{2.044424in}{1.118250in}}%
\pgfpathlineto{\pgfqpoint{2.050157in}{1.118638in}}%
\pgfpathlineto{\pgfqpoint{2.049708in}{1.124375in}}%
\pgfpathlineto{\pgfqpoint{2.067309in}{1.125374in}}%
\pgfpathlineto{\pgfqpoint{2.073272in}{1.125381in}}%
\pgfpathlineto{\pgfqpoint{2.072910in}{1.131159in}}%
\pgfpathlineto{\pgfqpoint{2.095572in}{1.132813in}}%
\pgfpathlineto{\pgfqpoint{2.096418in}{1.118342in}}%
\pgfpathlineto{\pgfqpoint{2.097811in}{1.095356in}}%
\pgfpathlineto{\pgfqpoint{2.114940in}{1.096476in}}%
\pgfpathlineto{\pgfqpoint{2.116398in}{1.073683in}}%
\pgfpathclose%
\pgfusepath{fill}%
\end{pgfscope}%
\begin{pgfscope}%
\pgfpathrectangle{\pgfqpoint{0.100000in}{0.100000in}}{\pgfqpoint{3.007045in}{1.925000in}}%
\pgfusepath{clip}%
\pgfsetbuttcap%
\pgfsetmiterjoin%
\definecolor{currentfill}{rgb}{0.331334,0.622068,0.804767}%
\pgfsetfillcolor{currentfill}%
\pgfsetlinewidth{0.000000pt}%
\definecolor{currentstroke}{rgb}{0.000000,0.000000,0.000000}%
\pgfsetstrokecolor{currentstroke}%
\pgfsetstrokeopacity{0.000000}%
\pgfsetdash{}{0pt}%
\pgfpathmoveto{\pgfqpoint{2.241536in}{0.660624in}}%
\pgfpathlineto{\pgfqpoint{2.233755in}{0.660011in}}%
\pgfpathlineto{\pgfqpoint{2.229320in}{0.666780in}}%
\pgfpathlineto{\pgfqpoint{2.228144in}{0.679871in}}%
\pgfpathlineto{\pgfqpoint{2.226858in}{0.694238in}}%
\pgfpathlineto{\pgfqpoint{2.225729in}{0.705930in}}%
\pgfpathlineto{\pgfqpoint{2.230267in}{0.714046in}}%
\pgfpathlineto{\pgfqpoint{2.223953in}{0.726994in}}%
\pgfpathlineto{\pgfqpoint{2.222669in}{0.734886in}}%
\pgfpathlineto{\pgfqpoint{2.233994in}{0.735748in}}%
\pgfpathlineto{\pgfqpoint{2.233682in}{0.738645in}}%
\pgfpathlineto{\pgfqpoint{2.250388in}{0.740313in}}%
\pgfpathlineto{\pgfqpoint{2.252087in}{0.743598in}}%
\pgfpathlineto{\pgfqpoint{2.279667in}{0.746313in}}%
\pgfpathlineto{\pgfqpoint{2.281053in}{0.729035in}}%
\pgfpathlineto{\pgfqpoint{2.274136in}{0.722813in}}%
\pgfpathlineto{\pgfqpoint{2.276404in}{0.718014in}}%
\pgfpathlineto{\pgfqpoint{2.277794in}{0.699248in}}%
\pgfpathlineto{\pgfqpoint{2.278425in}{0.693743in}}%
\pgfpathlineto{\pgfqpoint{2.261248in}{0.691899in}}%
\pgfpathlineto{\pgfqpoint{2.260621in}{0.697611in}}%
\pgfpathlineto{\pgfqpoint{2.254880in}{0.697001in}}%
\pgfpathlineto{\pgfqpoint{2.253069in}{0.691091in}}%
\pgfpathlineto{\pgfqpoint{2.254972in}{0.670665in}}%
\pgfpathlineto{\pgfqpoint{2.252110in}{0.670372in}}%
\pgfpathlineto{\pgfqpoint{2.252934in}{0.661782in}}%
\pgfpathlineto{\pgfqpoint{2.241536in}{0.660624in}}%
\pgfpathclose%
\pgfusepath{fill}%
\end{pgfscope}%
\begin{pgfscope}%
\pgfpathrectangle{\pgfqpoint{0.100000in}{0.100000in}}{\pgfqpoint{3.007045in}{1.925000in}}%
\pgfusepath{clip}%
\pgfsetbuttcap%
\pgfsetmiterjoin%
\definecolor{currentfill}{rgb}{0.529412,0.742637,0.862899}%
\pgfsetfillcolor{currentfill}%
\pgfsetlinewidth{0.000000pt}%
\definecolor{currentstroke}{rgb}{0.000000,0.000000,0.000000}%
\pgfsetstrokecolor{currentstroke}%
\pgfsetstrokeopacity{0.000000}%
\pgfsetdash{}{0pt}%
\pgfpathmoveto{\pgfqpoint{0.565403in}{1.940383in}}%
\pgfpathlineto{\pgfqpoint{0.569196in}{1.933206in}}%
\pgfpathlineto{\pgfqpoint{0.564241in}{1.932920in}}%
\pgfpathlineto{\pgfqpoint{0.561680in}{1.929053in}}%
\pgfpathlineto{\pgfqpoint{0.565762in}{1.918633in}}%
\pgfpathlineto{\pgfqpoint{0.569487in}{1.913159in}}%
\pgfpathlineto{\pgfqpoint{0.569117in}{1.907951in}}%
\pgfpathlineto{\pgfqpoint{0.564737in}{1.906268in}}%
\pgfpathlineto{\pgfqpoint{0.563966in}{1.911840in}}%
\pgfpathlineto{\pgfqpoint{0.559562in}{1.916054in}}%
\pgfpathlineto{\pgfqpoint{0.561416in}{1.920783in}}%
\pgfpathlineto{\pgfqpoint{0.556562in}{1.930703in}}%
\pgfpathlineto{\pgfqpoint{0.565403in}{1.940383in}}%
\pgfpathclose%
\pgfusepath{fill}%
\end{pgfscope}%
\begin{pgfscope}%
\pgfpathrectangle{\pgfqpoint{0.100000in}{0.100000in}}{\pgfqpoint{3.007045in}{1.925000in}}%
\pgfusepath{clip}%
\pgfsetbuttcap%
\pgfsetmiterjoin%
\definecolor{currentfill}{rgb}{0.529412,0.742637,0.862899}%
\pgfsetfillcolor{currentfill}%
\pgfsetlinewidth{0.000000pt}%
\definecolor{currentstroke}{rgb}{0.000000,0.000000,0.000000}%
\pgfsetstrokecolor{currentstroke}%
\pgfsetstrokeopacity{0.000000}%
\pgfsetdash{}{0pt}%
\pgfpathmoveto{\pgfqpoint{0.574289in}{1.797104in}}%
\pgfpathlineto{\pgfqpoint{0.542761in}{1.806508in}}%
\pgfpathlineto{\pgfqpoint{0.493956in}{1.821847in}}%
\pgfpathlineto{\pgfqpoint{0.501743in}{1.847839in}}%
\pgfpathlineto{\pgfqpoint{0.510872in}{1.844976in}}%
\pgfpathlineto{\pgfqpoint{0.515078in}{1.857792in}}%
\pgfpathlineto{\pgfqpoint{0.514994in}{1.864013in}}%
\pgfpathlineto{\pgfqpoint{0.502572in}{1.867867in}}%
\pgfpathlineto{\pgfqpoint{0.505442in}{1.879195in}}%
\pgfpathlineto{\pgfqpoint{0.510848in}{1.895622in}}%
\pgfpathlineto{\pgfqpoint{0.512685in}{1.901288in}}%
\pgfpathlineto{\pgfqpoint{0.536363in}{1.893635in}}%
\pgfpathlineto{\pgfqpoint{0.546025in}{1.896043in}}%
\pgfpathlineto{\pgfqpoint{0.549783in}{1.902072in}}%
\pgfpathlineto{\pgfqpoint{0.556850in}{1.906039in}}%
\pgfpathlineto{\pgfqpoint{0.556363in}{1.910254in}}%
\pgfpathlineto{\pgfqpoint{0.560656in}{1.907168in}}%
\pgfpathlineto{\pgfqpoint{0.559298in}{1.890060in}}%
\pgfpathlineto{\pgfqpoint{0.563615in}{1.893690in}}%
\pgfpathlineto{\pgfqpoint{0.564306in}{1.899159in}}%
\pgfpathlineto{\pgfqpoint{0.570797in}{1.906935in}}%
\pgfpathlineto{\pgfqpoint{0.575526in}{1.910604in}}%
\pgfpathlineto{\pgfqpoint{0.572185in}{1.916693in}}%
\pgfpathlineto{\pgfqpoint{0.573093in}{1.922337in}}%
\pgfpathlineto{\pgfqpoint{0.567141in}{1.928935in}}%
\pgfpathlineto{\pgfqpoint{0.574554in}{1.930262in}}%
\pgfpathlineto{\pgfqpoint{0.568642in}{1.941046in}}%
\pgfpathlineto{\pgfqpoint{0.578403in}{1.950226in}}%
\pgfpathlineto{\pgfqpoint{0.576681in}{1.953612in}}%
\pgfpathlineto{\pgfqpoint{0.623985in}{1.938935in}}%
\pgfpathlineto{\pgfqpoint{0.643432in}{1.933133in}}%
\pgfpathlineto{\pgfqpoint{0.650462in}{1.932235in}}%
\pgfpathlineto{\pgfqpoint{0.648297in}{1.930642in}}%
\pgfpathlineto{\pgfqpoint{0.650253in}{1.923532in}}%
\pgfpathlineto{\pgfqpoint{0.643966in}{1.926557in}}%
\pgfpathlineto{\pgfqpoint{0.633546in}{1.923434in}}%
\pgfpathlineto{\pgfqpoint{0.630805in}{1.915337in}}%
\pgfpathlineto{\pgfqpoint{0.634619in}{1.904375in}}%
\pgfpathlineto{\pgfqpoint{0.631834in}{1.900038in}}%
\pgfpathlineto{\pgfqpoint{0.622765in}{1.897664in}}%
\pgfpathlineto{\pgfqpoint{0.617908in}{1.888997in}}%
\pgfpathlineto{\pgfqpoint{0.619147in}{1.875248in}}%
\pgfpathlineto{\pgfqpoint{0.615087in}{1.868865in}}%
\pgfpathlineto{\pgfqpoint{0.610134in}{1.868414in}}%
\pgfpathlineto{\pgfqpoint{0.600903in}{1.863151in}}%
\pgfpathlineto{\pgfqpoint{0.595348in}{1.857817in}}%
\pgfpathlineto{\pgfqpoint{0.595548in}{1.853584in}}%
\pgfpathlineto{\pgfqpoint{0.599282in}{1.851636in}}%
\pgfpathlineto{\pgfqpoint{0.597616in}{1.841914in}}%
\pgfpathlineto{\pgfqpoint{0.583582in}{1.827974in}}%
\pgfpathlineto{\pgfqpoint{0.585552in}{1.821819in}}%
\pgfpathlineto{\pgfqpoint{0.585013in}{1.811079in}}%
\pgfpathlineto{\pgfqpoint{0.579759in}{1.803398in}}%
\pgfpathlineto{\pgfqpoint{0.579595in}{1.795500in}}%
\pgfpathlineto{\pgfqpoint{0.574289in}{1.797104in}}%
\pgfpathclose%
\pgfusepath{fill}%
\end{pgfscope}%
\begin{pgfscope}%
\pgfpathrectangle{\pgfqpoint{0.100000in}{0.100000in}}{\pgfqpoint{3.007045in}{1.925000in}}%
\pgfusepath{clip}%
\pgfsetbuttcap%
\pgfsetmiterjoin%
\definecolor{currentfill}{rgb}{0.435294,0.690965,0.842599}%
\pgfsetfillcolor{currentfill}%
\pgfsetlinewidth{0.000000pt}%
\definecolor{currentstroke}{rgb}{0.000000,0.000000,0.000000}%
\pgfsetstrokecolor{currentstroke}%
\pgfsetstrokeopacity{0.000000}%
\pgfsetdash{}{0pt}%
\pgfpathmoveto{\pgfqpoint{2.196663in}{1.378800in}}%
\pgfpathlineto{\pgfqpoint{2.232094in}{1.382445in}}%
\pgfpathlineto{\pgfqpoint{2.229609in}{1.405210in}}%
\pgfpathlineto{\pgfqpoint{2.252299in}{1.407809in}}%
\pgfpathlineto{\pgfqpoint{2.254932in}{1.384910in}}%
\pgfpathlineto{\pgfqpoint{2.272022in}{1.386856in}}%
\pgfpathlineto{\pgfqpoint{2.275068in}{1.363856in}}%
\pgfpathlineto{\pgfqpoint{2.269373in}{1.363358in}}%
\pgfpathlineto{\pgfqpoint{2.223806in}{1.358262in}}%
\pgfpathlineto{\pgfqpoint{2.201376in}{1.356174in}}%
\pgfpathlineto{\pgfqpoint{2.200252in}{1.367456in}}%
\pgfpathlineto{\pgfqpoint{2.193342in}{1.366778in}}%
\pgfpathlineto{\pgfqpoint{2.196663in}{1.378800in}}%
\pgfpathclose%
\pgfusepath{fill}%
\end{pgfscope}%
\begin{pgfscope}%
\pgfpathrectangle{\pgfqpoint{0.100000in}{0.100000in}}{\pgfqpoint{3.007045in}{1.925000in}}%
\pgfusepath{clip}%
\pgfsetbuttcap%
\pgfsetmiterjoin%
\definecolor{currentfill}{rgb}{0.460392,0.704744,0.848012}%
\pgfsetfillcolor{currentfill}%
\pgfsetlinewidth{0.000000pt}%
\definecolor{currentstroke}{rgb}{0.000000,0.000000,0.000000}%
\pgfsetstrokecolor{currentstroke}%
\pgfsetstrokeopacity{0.000000}%
\pgfsetdash{}{0pt}%
\pgfpathmoveto{\pgfqpoint{1.102971in}{1.695747in}}%
\pgfpathlineto{\pgfqpoint{1.119016in}{1.693144in}}%
\pgfpathlineto{\pgfqpoint{1.113428in}{1.662137in}}%
\pgfpathlineto{\pgfqpoint{1.102024in}{1.663968in}}%
\pgfpathlineto{\pgfqpoint{1.073596in}{1.668769in}}%
\pgfpathlineto{\pgfqpoint{1.067509in}{1.667877in}}%
\pgfpathlineto{\pgfqpoint{1.037941in}{1.673119in}}%
\pgfpathlineto{\pgfqpoint{1.036887in}{1.681674in}}%
\pgfpathlineto{\pgfqpoint{1.040593in}{1.686492in}}%
\pgfpathlineto{\pgfqpoint{1.037077in}{1.690684in}}%
\pgfpathlineto{\pgfqpoint{1.033247in}{1.689867in}}%
\pgfpathlineto{\pgfqpoint{1.030764in}{1.699834in}}%
\pgfpathlineto{\pgfqpoint{1.027516in}{1.705647in}}%
\pgfpathlineto{\pgfqpoint{1.025028in}{1.717016in}}%
\pgfpathlineto{\pgfqpoint{1.019449in}{1.720433in}}%
\pgfpathlineto{\pgfqpoint{1.019738in}{1.725093in}}%
\pgfpathlineto{\pgfqpoint{1.025046in}{1.724117in}}%
\pgfpathlineto{\pgfqpoint{1.026102in}{1.729804in}}%
\pgfpathlineto{\pgfqpoint{1.043222in}{1.727442in}}%
\pgfpathlineto{\pgfqpoint{1.053818in}{1.723481in}}%
\pgfpathlineto{\pgfqpoint{1.059670in}{1.716732in}}%
\pgfpathlineto{\pgfqpoint{1.066660in}{1.704559in}}%
\pgfpathlineto{\pgfqpoint{1.070809in}{1.700125in}}%
\pgfpathlineto{\pgfqpoint{1.093536in}{1.701272in}}%
\pgfpathlineto{\pgfqpoint{1.099659in}{1.696257in}}%
\pgfpathlineto{\pgfqpoint{1.102971in}{1.695747in}}%
\pgfpathclose%
\pgfusepath{fill}%
\end{pgfscope}%
\begin{pgfscope}%
\pgfpathrectangle{\pgfqpoint{0.100000in}{0.100000in}}{\pgfqpoint{3.007045in}{1.925000in}}%
\pgfusepath{clip}%
\pgfsetbuttcap%
\pgfsetmiterjoin%
\definecolor{currentfill}{rgb}{0.417086,0.680631,0.838231}%
\pgfsetfillcolor{currentfill}%
\pgfsetlinewidth{0.000000pt}%
\definecolor{currentstroke}{rgb}{0.000000,0.000000,0.000000}%
\pgfsetstrokecolor{currentstroke}%
\pgfsetstrokeopacity{0.000000}%
\pgfsetdash{}{0pt}%
\pgfpathmoveto{\pgfqpoint{2.802639in}{1.349500in}}%
\pgfpathlineto{\pgfqpoint{2.804907in}{1.347247in}}%
\pgfpathlineto{\pgfqpoint{2.799417in}{1.334517in}}%
\pgfpathlineto{\pgfqpoint{2.789776in}{1.325534in}}%
\pgfpathlineto{\pgfqpoint{2.791685in}{1.322963in}}%
\pgfpathlineto{\pgfqpoint{2.777561in}{1.327350in}}%
\pgfpathlineto{\pgfqpoint{2.769299in}{1.335490in}}%
\pgfpathlineto{\pgfqpoint{2.764535in}{1.335742in}}%
\pgfpathlineto{\pgfqpoint{2.762277in}{1.343620in}}%
\pgfpathlineto{\pgfqpoint{2.755066in}{1.346497in}}%
\pgfpathlineto{\pgfqpoint{2.753885in}{1.357653in}}%
\pgfpathlineto{\pgfqpoint{2.756767in}{1.359194in}}%
\pgfpathlineto{\pgfqpoint{2.758253in}{1.365099in}}%
\pgfpathlineto{\pgfqpoint{2.753651in}{1.370464in}}%
\pgfpathlineto{\pgfqpoint{2.762147in}{1.386718in}}%
\pgfpathlineto{\pgfqpoint{2.763125in}{1.394125in}}%
\pgfpathlineto{\pgfqpoint{2.768627in}{1.400117in}}%
\pgfpathlineto{\pgfqpoint{2.794033in}{1.391447in}}%
\pgfpathlineto{\pgfqpoint{2.803517in}{1.405975in}}%
\pgfpathlineto{\pgfqpoint{2.805188in}{1.401167in}}%
\pgfpathlineto{\pgfqpoint{2.810869in}{1.395086in}}%
\pgfpathlineto{\pgfqpoint{2.812809in}{1.384940in}}%
\pgfpathlineto{\pgfqpoint{2.810062in}{1.361931in}}%
\pgfpathlineto{\pgfqpoint{2.803807in}{1.360108in}}%
\pgfpathlineto{\pgfqpoint{2.802639in}{1.349500in}}%
\pgfpathclose%
\pgfusepath{fill}%
\end{pgfscope}%
\begin{pgfscope}%
\pgfpathrectangle{\pgfqpoint{0.100000in}{0.100000in}}{\pgfqpoint{3.007045in}{1.925000in}}%
\pgfusepath{clip}%
\pgfsetbuttcap%
\pgfsetmiterjoin%
\definecolor{currentfill}{rgb}{0.396909,0.666851,0.830358}%
\pgfsetfillcolor{currentfill}%
\pgfsetlinewidth{0.000000pt}%
\definecolor{currentstroke}{rgb}{0.000000,0.000000,0.000000}%
\pgfsetstrokecolor{currentstroke}%
\pgfsetstrokeopacity{0.000000}%
\pgfsetdash{}{0pt}%
\pgfpathmoveto{\pgfqpoint{2.427990in}{0.752079in}}%
\pgfpathlineto{\pgfqpoint{2.441826in}{0.762195in}}%
\pgfpathlineto{\pgfqpoint{2.441164in}{0.768340in}}%
\pgfpathlineto{\pgfqpoint{2.436633in}{0.771428in}}%
\pgfpathlineto{\pgfqpoint{2.433317in}{0.776989in}}%
\pgfpathlineto{\pgfqpoint{2.432968in}{0.782406in}}%
\pgfpathlineto{\pgfqpoint{2.437607in}{0.788428in}}%
\pgfpathlineto{\pgfqpoint{2.441806in}{0.790953in}}%
\pgfpathlineto{\pgfqpoint{2.443204in}{0.795028in}}%
\pgfpathlineto{\pgfqpoint{2.448640in}{0.798533in}}%
\pgfpathlineto{\pgfqpoint{2.449695in}{0.794403in}}%
\pgfpathlineto{\pgfqpoint{2.454423in}{0.792007in}}%
\pgfpathlineto{\pgfqpoint{2.461072in}{0.785247in}}%
\pgfpathlineto{\pgfqpoint{2.464577in}{0.779288in}}%
\pgfpathlineto{\pgfqpoint{2.464508in}{0.773648in}}%
\pgfpathlineto{\pgfqpoint{2.470222in}{0.769671in}}%
\pgfpathlineto{\pgfqpoint{2.467299in}{0.764175in}}%
\pgfpathlineto{\pgfqpoint{2.469397in}{0.761860in}}%
\pgfpathlineto{\pgfqpoint{2.463033in}{0.756433in}}%
\pgfpathlineto{\pgfqpoint{2.460691in}{0.751658in}}%
\pgfpathlineto{\pgfqpoint{2.457386in}{0.739684in}}%
\pgfpathlineto{\pgfqpoint{2.458380in}{0.737732in}}%
\pgfpathlineto{\pgfqpoint{2.451822in}{0.729156in}}%
\pgfpathlineto{\pgfqpoint{2.444853in}{0.725072in}}%
\pgfpathlineto{\pgfqpoint{2.434230in}{0.742118in}}%
\pgfpathlineto{\pgfqpoint{2.427990in}{0.752079in}}%
\pgfpathclose%
\pgfusepath{fill}%
\end{pgfscope}%
\begin{pgfscope}%
\pgfpathrectangle{\pgfqpoint{0.100000in}{0.100000in}}{\pgfqpoint{3.007045in}{1.925000in}}%
\pgfusepath{clip}%
\pgfsetbuttcap%
\pgfsetmiterjoin%
\definecolor{currentfill}{rgb}{0.280892,0.587620,0.785083}%
\pgfsetfillcolor{currentfill}%
\pgfsetlinewidth{0.000000pt}%
\definecolor{currentstroke}{rgb}{0.000000,0.000000,0.000000}%
\pgfsetstrokecolor{currentstroke}%
\pgfsetstrokeopacity{0.000000}%
\pgfsetdash{}{0pt}%
\pgfpathmoveto{\pgfqpoint{2.113075in}{0.848655in}}%
\pgfpathlineto{\pgfqpoint{2.110261in}{0.847530in}}%
\pgfpathlineto{\pgfqpoint{2.110699in}{0.840860in}}%
\pgfpathlineto{\pgfqpoint{2.103193in}{0.839402in}}%
\pgfpathlineto{\pgfqpoint{2.085917in}{0.838289in}}%
\pgfpathlineto{\pgfqpoint{2.085442in}{0.845945in}}%
\pgfpathlineto{\pgfqpoint{2.082243in}{0.851523in}}%
\pgfpathlineto{\pgfqpoint{2.081112in}{0.868758in}}%
\pgfpathlineto{\pgfqpoint{2.078219in}{0.868579in}}%
\pgfpathlineto{\pgfqpoint{2.077608in}{0.878428in}}%
\pgfpathlineto{\pgfqpoint{2.085939in}{0.878995in}}%
\pgfpathlineto{\pgfqpoint{2.106178in}{0.880380in}}%
\pgfpathlineto{\pgfqpoint{2.106779in}{0.871379in}}%
\pgfpathlineto{\pgfqpoint{2.110591in}{0.871650in}}%
\pgfpathlineto{\pgfqpoint{2.112919in}{0.865058in}}%
\pgfpathlineto{\pgfqpoint{2.113075in}{0.848655in}}%
\pgfpathclose%
\pgfusepath{fill}%
\end{pgfscope}%
\begin{pgfscope}%
\pgfpathrectangle{\pgfqpoint{0.100000in}{0.100000in}}{\pgfqpoint{3.007045in}{1.925000in}}%
\pgfusepath{clip}%
\pgfsetbuttcap%
\pgfsetmiterjoin%
\definecolor{currentfill}{rgb}{0.460392,0.704744,0.848012}%
\pgfsetfillcolor{currentfill}%
\pgfsetlinewidth{0.000000pt}%
\definecolor{currentstroke}{rgb}{0.000000,0.000000,0.000000}%
\pgfsetstrokecolor{currentstroke}%
\pgfsetstrokeopacity{0.000000}%
\pgfsetdash{}{0pt}%
\pgfpathmoveto{\pgfqpoint{2.179696in}{0.840159in}}%
\pgfpathlineto{\pgfqpoint{2.144570in}{0.838481in}}%
\pgfpathlineto{\pgfqpoint{2.144807in}{0.847920in}}%
\pgfpathlineto{\pgfqpoint{2.145830in}{0.876417in}}%
\pgfpathlineto{\pgfqpoint{2.139583in}{0.883616in}}%
\pgfpathlineto{\pgfqpoint{2.151351in}{0.884373in}}%
\pgfpathlineto{\pgfqpoint{2.150055in}{0.903459in}}%
\pgfpathlineto{\pgfqpoint{2.147248in}{0.909640in}}%
\pgfpathlineto{\pgfqpoint{2.149635in}{0.914318in}}%
\pgfpathlineto{\pgfqpoint{2.163243in}{0.916924in}}%
\pgfpathlineto{\pgfqpoint{2.167138in}{0.916537in}}%
\pgfpathlineto{\pgfqpoint{2.171317in}{0.911958in}}%
\pgfpathlineto{\pgfqpoint{2.184378in}{0.916884in}}%
\pgfpathlineto{\pgfqpoint{2.189759in}{0.915550in}}%
\pgfpathlineto{\pgfqpoint{2.190733in}{0.912451in}}%
\pgfpathlineto{\pgfqpoint{2.192088in}{0.890541in}}%
\pgfpathlineto{\pgfqpoint{2.193121in}{0.887325in}}%
\pgfpathlineto{\pgfqpoint{2.193931in}{0.875265in}}%
\pgfpathlineto{\pgfqpoint{2.189352in}{0.871261in}}%
\pgfpathlineto{\pgfqpoint{2.182587in}{0.873244in}}%
\pgfpathlineto{\pgfqpoint{2.183086in}{0.867245in}}%
\pgfpathlineto{\pgfqpoint{2.178278in}{0.857446in}}%
\pgfpathlineto{\pgfqpoint{2.179696in}{0.840159in}}%
\pgfpathclose%
\pgfusepath{fill}%
\end{pgfscope}%
\begin{pgfscope}%
\pgfpathrectangle{\pgfqpoint{0.100000in}{0.100000in}}{\pgfqpoint{3.007045in}{1.925000in}}%
\pgfusepath{clip}%
\pgfsetbuttcap%
\pgfsetmiterjoin%
\definecolor{currentfill}{rgb}{0.311157,0.608289,0.796894}%
\pgfsetfillcolor{currentfill}%
\pgfsetlinewidth{0.000000pt}%
\definecolor{currentstroke}{rgb}{0.000000,0.000000,0.000000}%
\pgfsetstrokecolor{currentstroke}%
\pgfsetstrokeopacity{0.000000}%
\pgfsetdash{}{0pt}%
\pgfpathmoveto{\pgfqpoint{1.662731in}{0.409312in}}%
\pgfpathlineto{\pgfqpoint{1.664160in}{0.412781in}}%
\pgfpathlineto{\pgfqpoint{1.665098in}{0.416805in}}%
\pgfpathlineto{\pgfqpoint{1.670154in}{0.417855in}}%
\pgfpathlineto{\pgfqpoint{1.678639in}{0.424954in}}%
\pgfpathlineto{\pgfqpoint{1.680357in}{0.422649in}}%
\pgfpathlineto{\pgfqpoint{1.670606in}{0.415983in}}%
\pgfpathlineto{\pgfqpoint{1.662731in}{0.409312in}}%
\pgfpathclose%
\pgfusepath{fill}%
\end{pgfscope}%
\begin{pgfscope}%
\pgfpathrectangle{\pgfqpoint{0.100000in}{0.100000in}}{\pgfqpoint{3.007045in}{1.925000in}}%
\pgfusepath{clip}%
\pgfsetbuttcap%
\pgfsetmiterjoin%
\definecolor{currentfill}{rgb}{0.311157,0.608289,0.796894}%
\pgfsetfillcolor{currentfill}%
\pgfsetlinewidth{0.000000pt}%
\definecolor{currentstroke}{rgb}{0.000000,0.000000,0.000000}%
\pgfsetstrokecolor{currentstroke}%
\pgfsetstrokeopacity{0.000000}%
\pgfsetdash{}{0pt}%
\pgfpathmoveto{\pgfqpoint{1.694633in}{0.447645in}}%
\pgfpathlineto{\pgfqpoint{1.685694in}{0.443169in}}%
\pgfpathlineto{\pgfqpoint{1.680452in}{0.447923in}}%
\pgfpathlineto{\pgfqpoint{1.679297in}{0.453438in}}%
\pgfpathlineto{\pgfqpoint{1.674645in}{0.451462in}}%
\pgfpathlineto{\pgfqpoint{1.678471in}{0.443997in}}%
\pgfpathlineto{\pgfqpoint{1.690125in}{0.434156in}}%
\pgfpathlineto{\pgfqpoint{1.679523in}{0.428823in}}%
\pgfpathlineto{\pgfqpoint{1.676716in}{0.426224in}}%
\pgfpathlineto{\pgfqpoint{1.667536in}{0.431953in}}%
\pgfpathlineto{\pgfqpoint{1.661661in}{0.438978in}}%
\pgfpathlineto{\pgfqpoint{1.655495in}{0.438831in}}%
\pgfpathlineto{\pgfqpoint{1.651860in}{0.442260in}}%
\pgfpathlineto{\pgfqpoint{1.645296in}{0.442654in}}%
\pgfpathlineto{\pgfqpoint{1.632363in}{0.431928in}}%
\pgfpathlineto{\pgfqpoint{1.622101in}{0.436398in}}%
\pgfpathlineto{\pgfqpoint{1.621822in}{0.440642in}}%
\pgfpathlineto{\pgfqpoint{1.613769in}{0.442713in}}%
\pgfpathlineto{\pgfqpoint{1.609200in}{0.451305in}}%
\pgfpathlineto{\pgfqpoint{1.621451in}{0.460671in}}%
\pgfpathlineto{\pgfqpoint{1.611272in}{0.473936in}}%
\pgfpathlineto{\pgfqpoint{1.619767in}{0.480550in}}%
\pgfpathlineto{\pgfqpoint{1.641992in}{0.498317in}}%
\pgfpathlineto{\pgfqpoint{1.644405in}{0.511356in}}%
\pgfpathlineto{\pgfqpoint{1.648009in}{0.514439in}}%
\pgfpathlineto{\pgfqpoint{1.663618in}{0.514403in}}%
\pgfpathlineto{\pgfqpoint{1.668257in}{0.510605in}}%
\pgfpathlineto{\pgfqpoint{1.681469in}{0.494214in}}%
\pgfpathlineto{\pgfqpoint{1.676720in}{0.488443in}}%
\pgfpathlineto{\pgfqpoint{1.695745in}{0.469116in}}%
\pgfpathlineto{\pgfqpoint{1.694633in}{0.447645in}}%
\pgfpathclose%
\pgfusepath{fill}%
\end{pgfscope}%
\begin{pgfscope}%
\pgfpathrectangle{\pgfqpoint{0.100000in}{0.100000in}}{\pgfqpoint{3.007045in}{1.925000in}}%
\pgfusepath{clip}%
\pgfsetbuttcap%
\pgfsetmiterjoin%
\definecolor{currentfill}{rgb}{0.396909,0.666851,0.830358}%
\pgfsetfillcolor{currentfill}%
\pgfsetlinewidth{0.000000pt}%
\definecolor{currentstroke}{rgb}{0.000000,0.000000,0.000000}%
\pgfsetstrokecolor{currentstroke}%
\pgfsetstrokeopacity{0.000000}%
\pgfsetdash{}{0pt}%
\pgfpathmoveto{\pgfqpoint{1.985200in}{0.741858in}}%
\pgfpathlineto{\pgfqpoint{1.970228in}{0.741380in}}%
\pgfpathlineto{\pgfqpoint{1.968862in}{0.741315in}}%
\pgfpathlineto{\pgfqpoint{1.968042in}{0.766663in}}%
\pgfpathlineto{\pgfqpoint{1.938599in}{0.766214in}}%
\pgfpathlineto{\pgfqpoint{1.939283in}{0.759647in}}%
\pgfpathlineto{\pgfqpoint{1.931371in}{0.753155in}}%
\pgfpathlineto{\pgfqpoint{1.930951in}{0.750384in}}%
\pgfpathlineto{\pgfqpoint{1.921911in}{0.757915in}}%
\pgfpathlineto{\pgfqpoint{1.918474in}{0.757862in}}%
\pgfpathlineto{\pgfqpoint{1.917305in}{0.764130in}}%
\pgfpathlineto{\pgfqpoint{1.920677in}{0.772374in}}%
\pgfpathlineto{\pgfqpoint{1.918485in}{0.794539in}}%
\pgfpathlineto{\pgfqpoint{1.911801in}{0.803402in}}%
\pgfpathlineto{\pgfqpoint{1.910795in}{0.808969in}}%
\pgfpathlineto{\pgfqpoint{1.918185in}{0.809351in}}%
\pgfpathlineto{\pgfqpoint{1.923831in}{0.809696in}}%
\pgfpathlineto{\pgfqpoint{1.924611in}{0.820832in}}%
\pgfpathlineto{\pgfqpoint{1.924362in}{0.838176in}}%
\pgfpathlineto{\pgfqpoint{1.951736in}{0.838426in}}%
\pgfpathlineto{\pgfqpoint{1.951971in}{0.821985in}}%
\pgfpathlineto{\pgfqpoint{1.961337in}{0.821725in}}%
\pgfpathlineto{\pgfqpoint{1.962924in}{0.816378in}}%
\pgfpathlineto{\pgfqpoint{1.966388in}{0.815925in}}%
\pgfpathlineto{\pgfqpoint{1.967563in}{0.808897in}}%
\pgfpathlineto{\pgfqpoint{1.974688in}{0.804739in}}%
\pgfpathlineto{\pgfqpoint{1.984377in}{0.804898in}}%
\pgfpathlineto{\pgfqpoint{1.980611in}{0.811051in}}%
\pgfpathlineto{\pgfqpoint{1.984723in}{0.815682in}}%
\pgfpathlineto{\pgfqpoint{1.993621in}{0.816052in}}%
\pgfpathlineto{\pgfqpoint{1.998299in}{0.813452in}}%
\pgfpathlineto{\pgfqpoint{1.996950in}{0.809911in}}%
\pgfpathlineto{\pgfqpoint{1.986772in}{0.806255in}}%
\pgfpathlineto{\pgfqpoint{1.991139in}{0.803150in}}%
\pgfpathlineto{\pgfqpoint{1.988015in}{0.799119in}}%
\pgfpathlineto{\pgfqpoint{1.992182in}{0.792843in}}%
\pgfpathlineto{\pgfqpoint{1.984613in}{0.793228in}}%
\pgfpathlineto{\pgfqpoint{1.980623in}{0.785127in}}%
\pgfpathlineto{\pgfqpoint{1.985316in}{0.780979in}}%
\pgfpathlineto{\pgfqpoint{1.980431in}{0.778978in}}%
\pgfpathlineto{\pgfqpoint{1.985442in}{0.766767in}}%
\pgfpathlineto{\pgfqpoint{1.985739in}{0.761415in}}%
\pgfpathlineto{\pgfqpoint{1.988710in}{0.756370in}}%
\pgfpathlineto{\pgfqpoint{1.988976in}{0.750911in}}%
\pgfpathlineto{\pgfqpoint{1.985200in}{0.741858in}}%
\pgfpathclose%
\pgfusepath{fill}%
\end{pgfscope}%
\begin{pgfscope}%
\pgfpathrectangle{\pgfqpoint{0.100000in}{0.100000in}}{\pgfqpoint{3.007045in}{1.925000in}}%
\pgfusepath{clip}%
\pgfsetbuttcap%
\pgfsetmiterjoin%
\definecolor{currentfill}{rgb}{0.745713,0.845752,0.926490}%
\pgfsetfillcolor{currentfill}%
\pgfsetlinewidth{0.000000pt}%
\definecolor{currentstroke}{rgb}{0.000000,0.000000,0.000000}%
\pgfsetstrokecolor{currentstroke}%
\pgfsetstrokeopacity{0.000000}%
\pgfsetdash{}{0pt}%
\pgfpathmoveto{\pgfqpoint{2.793875in}{1.572201in}}%
\pgfpathlineto{\pgfqpoint{2.763396in}{1.560589in}}%
\pgfpathlineto{\pgfqpoint{2.763075in}{1.564033in}}%
\pgfpathlineto{\pgfqpoint{2.755118in}{1.563156in}}%
\pgfpathlineto{\pgfqpoint{2.747693in}{1.569195in}}%
\pgfpathlineto{\pgfqpoint{2.750775in}{1.572935in}}%
\pgfpathlineto{\pgfqpoint{2.747323in}{1.582366in}}%
\pgfpathlineto{\pgfqpoint{2.735921in}{1.578293in}}%
\pgfpathlineto{\pgfqpoint{2.735305in}{1.582994in}}%
\pgfpathlineto{\pgfqpoint{2.718437in}{1.632224in}}%
\pgfpathlineto{\pgfqpoint{2.713963in}{1.634075in}}%
\pgfpathlineto{\pgfqpoint{2.740305in}{1.639901in}}%
\pgfpathlineto{\pgfqpoint{2.776771in}{1.649915in}}%
\pgfpathlineto{\pgfqpoint{2.778407in}{1.644079in}}%
\pgfpathlineto{\pgfqpoint{2.777816in}{1.638392in}}%
\pgfpathlineto{\pgfqpoint{2.780372in}{1.636703in}}%
\pgfpathlineto{\pgfqpoint{2.780749in}{1.624141in}}%
\pgfpathlineto{\pgfqpoint{2.786419in}{1.617481in}}%
\pgfpathlineto{\pgfqpoint{2.787188in}{1.608949in}}%
\pgfpathlineto{\pgfqpoint{2.789757in}{1.602899in}}%
\pgfpathlineto{\pgfqpoint{2.787394in}{1.596652in}}%
\pgfpathlineto{\pgfqpoint{2.787488in}{1.586749in}}%
\pgfpathlineto{\pgfqpoint{2.793875in}{1.572201in}}%
\pgfpathclose%
\pgfusepath{fill}%
\end{pgfscope}%
\begin{pgfscope}%
\pgfpathrectangle{\pgfqpoint{0.100000in}{0.100000in}}{\pgfqpoint{3.007045in}{1.925000in}}%
\pgfusepath{clip}%
\pgfsetbuttcap%
\pgfsetmiterjoin%
\definecolor{currentfill}{rgb}{0.447843,0.697855,0.845306}%
\pgfsetfillcolor{currentfill}%
\pgfsetlinewidth{0.000000pt}%
\definecolor{currentstroke}{rgb}{0.000000,0.000000,0.000000}%
\pgfsetstrokecolor{currentstroke}%
\pgfsetstrokeopacity{0.000000}%
\pgfsetdash{}{0pt}%
\pgfpathmoveto{\pgfqpoint{1.415290in}{0.621980in}}%
\pgfpathlineto{\pgfqpoint{1.386813in}{0.624056in}}%
\pgfpathlineto{\pgfqpoint{1.391806in}{0.690752in}}%
\pgfpathlineto{\pgfqpoint{1.396350in}{0.690465in}}%
\pgfpathlineto{\pgfqpoint{1.398836in}{0.719387in}}%
\pgfpathlineto{\pgfqpoint{1.370186in}{0.721452in}}%
\pgfpathlineto{\pgfqpoint{1.372150in}{0.750256in}}%
\pgfpathlineto{\pgfqpoint{1.379483in}{0.749728in}}%
\pgfpathlineto{\pgfqpoint{1.429539in}{0.746331in}}%
\pgfpathlineto{\pgfqpoint{1.427496in}{0.717282in}}%
\pgfpathlineto{\pgfqpoint{1.425000in}{0.688410in}}%
\pgfpathlineto{\pgfqpoint{1.420501in}{0.688714in}}%
\pgfpathlineto{\pgfqpoint{1.417736in}{0.651743in}}%
\pgfpathlineto{\pgfqpoint{1.415290in}{0.621980in}}%
\pgfpathclose%
\pgfusepath{fill}%
\end{pgfscope}%
\begin{pgfscope}%
\pgfpathrectangle{\pgfqpoint{0.100000in}{0.100000in}}{\pgfqpoint{3.007045in}{1.925000in}}%
\pgfusepath{clip}%
\pgfsetbuttcap%
\pgfsetmiterjoin%
\definecolor{currentfill}{rgb}{0.093272,0.396878,0.673664}%
\pgfsetfillcolor{currentfill}%
\pgfsetlinewidth{0.000000pt}%
\definecolor{currentstroke}{rgb}{0.000000,0.000000,0.000000}%
\pgfsetstrokecolor{currentstroke}%
\pgfsetstrokeopacity{0.000000}%
\pgfsetdash{}{0pt}%
\pgfpathmoveto{\pgfqpoint{0.713739in}{0.318670in}}%
\pgfpathlineto{\pgfqpoint{0.713522in}{0.320731in}}%
\pgfpathlineto{\pgfqpoint{0.715798in}{0.320350in}}%
\pgfpathlineto{\pgfqpoint{0.717849in}{0.316514in}}%
\pgfpathlineto{\pgfqpoint{0.713739in}{0.318670in}}%
\pgfpathclose%
\pgfusepath{fill}%
\end{pgfscope}%
\begin{pgfscope}%
\pgfpathrectangle{\pgfqpoint{0.100000in}{0.100000in}}{\pgfqpoint{3.007045in}{1.925000in}}%
\pgfusepath{clip}%
\pgfsetbuttcap%
\pgfsetmiterjoin%
\definecolor{currentfill}{rgb}{0.093272,0.396878,0.673664}%
\pgfsetfillcolor{currentfill}%
\pgfsetlinewidth{0.000000pt}%
\definecolor{currentstroke}{rgb}{0.000000,0.000000,0.000000}%
\pgfsetstrokecolor{currentstroke}%
\pgfsetstrokeopacity{0.000000}%
\pgfsetdash{}{0pt}%
\pgfpathmoveto{\pgfqpoint{0.707133in}{0.321121in}}%
\pgfpathlineto{\pgfqpoint{0.708642in}{0.322633in}}%
\pgfpathlineto{\pgfqpoint{0.712152in}{0.322886in}}%
\pgfpathlineto{\pgfqpoint{0.711451in}{0.321404in}}%
\pgfpathlineto{\pgfqpoint{0.707133in}{0.321121in}}%
\pgfpathclose%
\pgfusepath{fill}%
\end{pgfscope}%
\begin{pgfscope}%
\pgfpathrectangle{\pgfqpoint{0.100000in}{0.100000in}}{\pgfqpoint{3.007045in}{1.925000in}}%
\pgfusepath{clip}%
\pgfsetbuttcap%
\pgfsetmiterjoin%
\definecolor{currentfill}{rgb}{0.093272,0.396878,0.673664}%
\pgfsetfillcolor{currentfill}%
\pgfsetlinewidth{0.000000pt}%
\definecolor{currentstroke}{rgb}{0.000000,0.000000,0.000000}%
\pgfsetstrokecolor{currentstroke}%
\pgfsetstrokeopacity{0.000000}%
\pgfsetdash{}{0pt}%
\pgfpathmoveto{\pgfqpoint{0.701239in}{0.348319in}}%
\pgfpathlineto{\pgfqpoint{0.697922in}{0.351049in}}%
\pgfpathlineto{\pgfqpoint{0.698993in}{0.352386in}}%
\pgfpathlineto{\pgfqpoint{0.701818in}{0.353160in}}%
\pgfpathlineto{\pgfqpoint{0.703476in}{0.352730in}}%
\pgfpathlineto{\pgfqpoint{0.704495in}{0.354085in}}%
\pgfpathlineto{\pgfqpoint{0.707132in}{0.354637in}}%
\pgfpathlineto{\pgfqpoint{0.708693in}{0.353375in}}%
\pgfpathlineto{\pgfqpoint{0.712359in}{0.352736in}}%
\pgfpathlineto{\pgfqpoint{0.711910in}{0.351797in}}%
\pgfpathlineto{\pgfqpoint{0.708260in}{0.352473in}}%
\pgfpathlineto{\pgfqpoint{0.704182in}{0.352191in}}%
\pgfpathlineto{\pgfqpoint{0.703882in}{0.349772in}}%
\pgfpathlineto{\pgfqpoint{0.701239in}{0.348319in}}%
\pgfpathclose%
\pgfusepath{fill}%
\end{pgfscope}%
\begin{pgfscope}%
\pgfpathrectangle{\pgfqpoint{0.100000in}{0.100000in}}{\pgfqpoint{3.007045in}{1.925000in}}%
\pgfusepath{clip}%
\pgfsetbuttcap%
\pgfsetmiterjoin%
\definecolor{currentfill}{rgb}{0.093272,0.396878,0.673664}%
\pgfsetfillcolor{currentfill}%
\pgfsetlinewidth{0.000000pt}%
\definecolor{currentstroke}{rgb}{0.000000,0.000000,0.000000}%
\pgfsetstrokecolor{currentstroke}%
\pgfsetstrokeopacity{0.000000}%
\pgfsetdash{}{0pt}%
\pgfpathmoveto{\pgfqpoint{0.758615in}{0.341736in}}%
\pgfpathlineto{\pgfqpoint{0.757407in}{0.339850in}}%
\pgfpathlineto{\pgfqpoint{0.759513in}{0.339536in}}%
\pgfpathlineto{\pgfqpoint{0.759838in}{0.338128in}}%
\pgfpathlineto{\pgfqpoint{0.761603in}{0.337668in}}%
\pgfpathlineto{\pgfqpoint{0.762040in}{0.334943in}}%
\pgfpathlineto{\pgfqpoint{0.760463in}{0.333422in}}%
\pgfpathlineto{\pgfqpoint{0.758461in}{0.333128in}}%
\pgfpathlineto{\pgfqpoint{0.753516in}{0.335984in}}%
\pgfpathlineto{\pgfqpoint{0.749366in}{0.335186in}}%
\pgfpathlineto{\pgfqpoint{0.746236in}{0.338305in}}%
\pgfpathlineto{\pgfqpoint{0.745834in}{0.340586in}}%
\pgfpathlineto{\pgfqpoint{0.747476in}{0.341739in}}%
\pgfpathlineto{\pgfqpoint{0.750424in}{0.341197in}}%
\pgfpathlineto{\pgfqpoint{0.750868in}{0.342943in}}%
\pgfpathlineto{\pgfqpoint{0.753106in}{0.339546in}}%
\pgfpathlineto{\pgfqpoint{0.754901in}{0.343295in}}%
\pgfpathlineto{\pgfqpoint{0.758150in}{0.343053in}}%
\pgfpathlineto{\pgfqpoint{0.759750in}{0.345315in}}%
\pgfpathlineto{\pgfqpoint{0.760957in}{0.344040in}}%
\pgfpathlineto{\pgfqpoint{0.760988in}{0.342549in}}%
\pgfpathlineto{\pgfqpoint{0.758615in}{0.341736in}}%
\pgfpathclose%
\pgfusepath{fill}%
\end{pgfscope}%
\begin{pgfscope}%
\pgfpathrectangle{\pgfqpoint{0.100000in}{0.100000in}}{\pgfqpoint{3.007045in}{1.925000in}}%
\pgfusepath{clip}%
\pgfsetbuttcap%
\pgfsetmiterjoin%
\definecolor{currentfill}{rgb}{0.093272,0.396878,0.673664}%
\pgfsetfillcolor{currentfill}%
\pgfsetlinewidth{0.000000pt}%
\definecolor{currentstroke}{rgb}{0.000000,0.000000,0.000000}%
\pgfsetstrokecolor{currentstroke}%
\pgfsetstrokeopacity{0.000000}%
\pgfsetdash{}{0pt}%
\pgfpathmoveto{\pgfqpoint{0.751166in}{0.331121in}}%
\pgfpathlineto{\pgfqpoint{0.751716in}{0.328502in}}%
\pgfpathlineto{\pgfqpoint{0.749924in}{0.328523in}}%
\pgfpathlineto{\pgfqpoint{0.747420in}{0.325126in}}%
\pgfpathlineto{\pgfqpoint{0.750723in}{0.323377in}}%
\pgfpathlineto{\pgfqpoint{0.746585in}{0.321465in}}%
\pgfpathlineto{\pgfqpoint{0.744840in}{0.322182in}}%
\pgfpathlineto{\pgfqpoint{0.743772in}{0.324350in}}%
\pgfpathlineto{\pgfqpoint{0.742472in}{0.323661in}}%
\pgfpathlineto{\pgfqpoint{0.742188in}{0.321974in}}%
\pgfpathlineto{\pgfqpoint{0.740508in}{0.321472in}}%
\pgfpathlineto{\pgfqpoint{0.734035in}{0.323447in}}%
\pgfpathlineto{\pgfqpoint{0.734346in}{0.321507in}}%
\pgfpathlineto{\pgfqpoint{0.732208in}{0.320942in}}%
\pgfpathlineto{\pgfqpoint{0.729299in}{0.323471in}}%
\pgfpathlineto{\pgfqpoint{0.727715in}{0.323803in}}%
\pgfpathlineto{\pgfqpoint{0.723753in}{0.321601in}}%
\pgfpathlineto{\pgfqpoint{0.719755in}{0.321321in}}%
\pgfpathlineto{\pgfqpoint{0.720327in}{0.323273in}}%
\pgfpathlineto{\pgfqpoint{0.724725in}{0.324272in}}%
\pgfpathlineto{\pgfqpoint{0.724952in}{0.327121in}}%
\pgfpathlineto{\pgfqpoint{0.722436in}{0.327416in}}%
\pgfpathlineto{\pgfqpoint{0.719318in}{0.325777in}}%
\pgfpathlineto{\pgfqpoint{0.718473in}{0.330811in}}%
\pgfpathlineto{\pgfqpoint{0.720682in}{0.334691in}}%
\pgfpathlineto{\pgfqpoint{0.719629in}{0.337498in}}%
\pgfpathlineto{\pgfqpoint{0.724542in}{0.340506in}}%
\pgfpathlineto{\pgfqpoint{0.728409in}{0.340545in}}%
\pgfpathlineto{\pgfqpoint{0.730761in}{0.339863in}}%
\pgfpathlineto{\pgfqpoint{0.732294in}{0.338142in}}%
\pgfpathlineto{\pgfqpoint{0.730507in}{0.333695in}}%
\pgfpathlineto{\pgfqpoint{0.733490in}{0.335333in}}%
\pgfpathlineto{\pgfqpoint{0.734831in}{0.339974in}}%
\pgfpathlineto{\pgfqpoint{0.736842in}{0.341018in}}%
\pgfpathlineto{\pgfqpoint{0.738720in}{0.340689in}}%
\pgfpathlineto{\pgfqpoint{0.738975in}{0.337460in}}%
\pgfpathlineto{\pgfqpoint{0.740912in}{0.337016in}}%
\pgfpathlineto{\pgfqpoint{0.741053in}{0.339842in}}%
\pgfpathlineto{\pgfqpoint{0.745035in}{0.337912in}}%
\pgfpathlineto{\pgfqpoint{0.746149in}{0.335843in}}%
\pgfpathlineto{\pgfqpoint{0.748088in}{0.333888in}}%
\pgfpathlineto{\pgfqpoint{0.746213in}{0.332218in}}%
\pgfpathlineto{\pgfqpoint{0.751350in}{0.331886in}}%
\pgfpathlineto{\pgfqpoint{0.751166in}{0.331121in}}%
\pgfpathclose%
\pgfusepath{fill}%
\end{pgfscope}%
\begin{pgfscope}%
\pgfpathrectangle{\pgfqpoint{0.100000in}{0.100000in}}{\pgfqpoint{3.007045in}{1.925000in}}%
\pgfusepath{clip}%
\pgfsetbuttcap%
\pgfsetmiterjoin%
\definecolor{currentfill}{rgb}{0.093272,0.396878,0.673664}%
\pgfsetfillcolor{currentfill}%
\pgfsetlinewidth{0.000000pt}%
\definecolor{currentstroke}{rgb}{0.000000,0.000000,0.000000}%
\pgfsetstrokecolor{currentstroke}%
\pgfsetstrokeopacity{0.000000}%
\pgfsetdash{}{0pt}%
\pgfpathmoveto{\pgfqpoint{0.757017in}{0.355194in}}%
\pgfpathlineto{\pgfqpoint{0.755520in}{0.355499in}}%
\pgfpathlineto{\pgfqpoint{0.753460in}{0.353984in}}%
\pgfpathlineto{\pgfqpoint{0.748969in}{0.353910in}}%
\pgfpathlineto{\pgfqpoint{0.747754in}{0.354827in}}%
\pgfpathlineto{\pgfqpoint{0.745252in}{0.353754in}}%
\pgfpathlineto{\pgfqpoint{0.742936in}{0.353492in}}%
\pgfpathlineto{\pgfqpoint{0.741809in}{0.351678in}}%
\pgfpathlineto{\pgfqpoint{0.740287in}{0.352714in}}%
\pgfpathlineto{\pgfqpoint{0.739684in}{0.350603in}}%
\pgfpathlineto{\pgfqpoint{0.733752in}{0.349525in}}%
\pgfpathlineto{\pgfqpoint{0.731314in}{0.349960in}}%
\pgfpathlineto{\pgfqpoint{0.728847in}{0.351743in}}%
\pgfpathlineto{\pgfqpoint{0.725995in}{0.352347in}}%
\pgfpathlineto{\pgfqpoint{0.725319in}{0.351090in}}%
\pgfpathlineto{\pgfqpoint{0.721782in}{0.351198in}}%
\pgfpathlineto{\pgfqpoint{0.719594in}{0.350212in}}%
\pgfpathlineto{\pgfqpoint{0.719062in}{0.352698in}}%
\pgfpathlineto{\pgfqpoint{0.717117in}{0.351179in}}%
\pgfpathlineto{\pgfqpoint{0.714893in}{0.351866in}}%
\pgfpathlineto{\pgfqpoint{0.713962in}{0.350368in}}%
\pgfpathlineto{\pgfqpoint{0.712047in}{0.351741in}}%
\pgfpathlineto{\pgfqpoint{0.715201in}{0.352581in}}%
\pgfpathlineto{\pgfqpoint{0.717812in}{0.354377in}}%
\pgfpathlineto{\pgfqpoint{0.719341in}{0.353168in}}%
\pgfpathlineto{\pgfqpoint{0.722429in}{0.352828in}}%
\pgfpathlineto{\pgfqpoint{0.726759in}{0.358371in}}%
\pgfpathlineto{\pgfqpoint{0.728487in}{0.358327in}}%
\pgfpathlineto{\pgfqpoint{0.730107in}{0.357059in}}%
\pgfpathlineto{\pgfqpoint{0.731400in}{0.357708in}}%
\pgfpathlineto{\pgfqpoint{0.734357in}{0.355834in}}%
\pgfpathlineto{\pgfqpoint{0.737813in}{0.354884in}}%
\pgfpathlineto{\pgfqpoint{0.738983in}{0.355722in}}%
\pgfpathlineto{\pgfqpoint{0.740865in}{0.355528in}}%
\pgfpathlineto{\pgfqpoint{0.743549in}{0.359090in}}%
\pgfpathlineto{\pgfqpoint{0.748423in}{0.355398in}}%
\pgfpathlineto{\pgfqpoint{0.749665in}{0.357072in}}%
\pgfpathlineto{\pgfqpoint{0.753292in}{0.354357in}}%
\pgfpathlineto{\pgfqpoint{0.755001in}{0.356698in}}%
\pgfpathlineto{\pgfqpoint{0.757017in}{0.355194in}}%
\pgfpathclose%
\pgfusepath{fill}%
\end{pgfscope}%
\begin{pgfscope}%
\pgfpathrectangle{\pgfqpoint{0.100000in}{0.100000in}}{\pgfqpoint{3.007045in}{1.925000in}}%
\pgfusepath{clip}%
\pgfsetbuttcap%
\pgfsetmiterjoin%
\definecolor{currentfill}{rgb}{0.652211,0.806013,0.893764}%
\pgfsetfillcolor{currentfill}%
\pgfsetlinewidth{0.000000pt}%
\definecolor{currentstroke}{rgb}{0.000000,0.000000,0.000000}%
\pgfsetstrokecolor{currentstroke}%
\pgfsetstrokeopacity{0.000000}%
\pgfsetdash{}{0pt}%
\pgfpathmoveto{\pgfqpoint{1.020483in}{1.573439in}}%
\pgfpathlineto{\pgfqpoint{1.018543in}{1.560865in}}%
\pgfpathlineto{\pgfqpoint{1.015015in}{1.563234in}}%
\pgfpathlineto{\pgfqpoint{1.009746in}{1.577792in}}%
\pgfpathlineto{\pgfqpoint{1.006564in}{1.581699in}}%
\pgfpathlineto{\pgfqpoint{1.001653in}{1.579735in}}%
\pgfpathlineto{\pgfqpoint{0.987365in}{1.582354in}}%
\pgfpathlineto{\pgfqpoint{0.988238in}{1.587058in}}%
\pgfpathlineto{\pgfqpoint{0.975154in}{1.589701in}}%
\pgfpathlineto{\pgfqpoint{0.971016in}{1.593385in}}%
\pgfpathlineto{\pgfqpoint{0.972074in}{1.604894in}}%
\pgfpathlineto{\pgfqpoint{0.965491in}{1.606172in}}%
\pgfpathlineto{\pgfqpoint{0.966618in}{1.611886in}}%
\pgfpathlineto{\pgfqpoint{0.960968in}{1.612983in}}%
\pgfpathlineto{\pgfqpoint{0.964235in}{1.629990in}}%
\pgfpathlineto{\pgfqpoint{0.963238in}{1.636470in}}%
\pgfpathlineto{\pgfqpoint{0.959812in}{1.636263in}}%
\pgfpathlineto{\pgfqpoint{0.955356in}{1.640352in}}%
\pgfpathlineto{\pgfqpoint{0.957302in}{1.649942in}}%
\pgfpathlineto{\pgfqpoint{0.964271in}{1.652566in}}%
\pgfpathlineto{\pgfqpoint{0.966686in}{1.656395in}}%
\pgfpathlineto{\pgfqpoint{0.981446in}{1.653571in}}%
\pgfpathlineto{\pgfqpoint{0.988214in}{1.659319in}}%
\pgfpathlineto{\pgfqpoint{0.994682in}{1.658385in}}%
\pgfpathlineto{\pgfqpoint{0.997011in}{1.655494in}}%
\pgfpathlineto{\pgfqpoint{1.004464in}{1.652753in}}%
\pgfpathlineto{\pgfqpoint{1.010507in}{1.658210in}}%
\pgfpathlineto{\pgfqpoint{1.016570in}{1.660701in}}%
\pgfpathlineto{\pgfqpoint{1.019996in}{1.667445in}}%
\pgfpathlineto{\pgfqpoint{1.024131in}{1.670887in}}%
\pgfpathlineto{\pgfqpoint{1.025044in}{1.675119in}}%
\pgfpathlineto{\pgfqpoint{1.037941in}{1.673119in}}%
\pgfpathlineto{\pgfqpoint{1.067509in}{1.667877in}}%
\pgfpathlineto{\pgfqpoint{1.073596in}{1.668769in}}%
\pgfpathlineto{\pgfqpoint{1.102024in}{1.663968in}}%
\pgfpathlineto{\pgfqpoint{1.101102in}{1.658254in}}%
\pgfpathlineto{\pgfqpoint{1.102412in}{1.652202in}}%
\pgfpathlineto{\pgfqpoint{1.110938in}{1.650768in}}%
\pgfpathlineto{\pgfqpoint{1.109783in}{1.645289in}}%
\pgfpathlineto{\pgfqpoint{1.105971in}{1.645829in}}%
\pgfpathlineto{\pgfqpoint{1.101489in}{1.634717in}}%
\pgfpathlineto{\pgfqpoint{1.099750in}{1.623315in}}%
\pgfpathlineto{\pgfqpoint{1.094101in}{1.624217in}}%
\pgfpathlineto{\pgfqpoint{1.081871in}{1.620429in}}%
\pgfpathlineto{\pgfqpoint{1.079975in}{1.609049in}}%
\pgfpathlineto{\pgfqpoint{1.073953in}{1.610065in}}%
\pgfpathlineto{\pgfqpoint{1.072000in}{1.598524in}}%
\pgfpathlineto{\pgfqpoint{1.084238in}{1.596483in}}%
\pgfpathlineto{\pgfqpoint{1.082392in}{1.585392in}}%
\pgfpathlineto{\pgfqpoint{1.040422in}{1.591840in}}%
\pgfpathlineto{\pgfqpoint{1.024388in}{1.595276in}}%
\pgfpathlineto{\pgfqpoint{1.020483in}{1.573439in}}%
\pgfpathclose%
\pgfusepath{fill}%
\end{pgfscope}%
\begin{pgfscope}%
\pgfpathrectangle{\pgfqpoint{0.100000in}{0.100000in}}{\pgfqpoint{3.007045in}{1.925000in}}%
\pgfusepath{clip}%
\pgfsetbuttcap%
\pgfsetmiterjoin%
\definecolor{currentfill}{rgb}{0.417086,0.680631,0.838231}%
\pgfsetfillcolor{currentfill}%
\pgfsetlinewidth{0.000000pt}%
\definecolor{currentstroke}{rgb}{0.000000,0.000000,0.000000}%
\pgfsetstrokecolor{currentstroke}%
\pgfsetstrokeopacity{0.000000}%
\pgfsetdash{}{0pt}%
\pgfpathmoveto{\pgfqpoint{1.607157in}{1.202049in}}%
\pgfpathlineto{\pgfqpoint{1.595536in}{1.202468in}}%
\pgfpathlineto{\pgfqpoint{1.584352in}{1.202908in}}%
\pgfpathlineto{\pgfqpoint{1.586476in}{1.248770in}}%
\pgfpathlineto{\pgfqpoint{1.631190in}{1.247136in}}%
\pgfpathlineto{\pgfqpoint{1.630558in}{1.224197in}}%
\pgfpathlineto{\pgfqpoint{1.608008in}{1.224954in}}%
\pgfpathlineto{\pgfqpoint{1.607157in}{1.202049in}}%
\pgfpathclose%
\pgfusepath{fill}%
\end{pgfscope}%
\begin{pgfscope}%
\pgfpathrectangle{\pgfqpoint{0.100000in}{0.100000in}}{\pgfqpoint{3.007045in}{1.925000in}}%
\pgfusepath{clip}%
\pgfsetbuttcap%
\pgfsetmiterjoin%
\definecolor{currentfill}{rgb}{0.622684,0.793464,0.883429}%
\pgfsetfillcolor{currentfill}%
\pgfsetlinewidth{0.000000pt}%
\definecolor{currentstroke}{rgb}{0.000000,0.000000,0.000000}%
\pgfsetstrokecolor{currentstroke}%
\pgfsetstrokeopacity{0.000000}%
\pgfsetdash{}{0pt}%
\pgfpathmoveto{\pgfqpoint{0.596099in}{1.599150in}}%
\pgfpathlineto{\pgfqpoint{0.573938in}{1.605496in}}%
\pgfpathlineto{\pgfqpoint{0.530203in}{1.618670in}}%
\pgfpathlineto{\pgfqpoint{0.499509in}{1.627848in}}%
\pgfpathlineto{\pgfqpoint{0.504292in}{1.636653in}}%
\pgfpathlineto{\pgfqpoint{0.505425in}{1.642739in}}%
\pgfpathlineto{\pgfqpoint{0.512713in}{1.647749in}}%
\pgfpathlineto{\pgfqpoint{0.519173in}{1.654672in}}%
\pgfpathlineto{\pgfqpoint{0.521441in}{1.664987in}}%
\pgfpathlineto{\pgfqpoint{0.521156in}{1.673183in}}%
\pgfpathlineto{\pgfqpoint{0.523556in}{1.681083in}}%
\pgfpathlineto{\pgfqpoint{0.527922in}{1.686852in}}%
\pgfpathlineto{\pgfqpoint{0.530035in}{1.695577in}}%
\pgfpathlineto{\pgfqpoint{0.533873in}{1.701067in}}%
\pgfpathlineto{\pgfqpoint{0.563962in}{1.691905in}}%
\pgfpathlineto{\pgfqpoint{0.596303in}{1.682378in}}%
\pgfpathlineto{\pgfqpoint{0.594595in}{1.678719in}}%
\pgfpathlineto{\pgfqpoint{0.588656in}{1.658125in}}%
\pgfpathlineto{\pgfqpoint{0.605211in}{1.653481in}}%
\pgfpathlineto{\pgfqpoint{0.609918in}{1.648785in}}%
\pgfpathlineto{\pgfqpoint{0.608444in}{1.643504in}}%
\pgfpathlineto{\pgfqpoint{0.619536in}{1.640423in}}%
\pgfpathlineto{\pgfqpoint{0.613279in}{1.618283in}}%
\pgfpathlineto{\pgfqpoint{0.607867in}{1.619794in}}%
\pgfpathlineto{\pgfqpoint{0.603136in}{1.603204in}}%
\pgfpathlineto{\pgfqpoint{0.597590in}{1.604775in}}%
\pgfpathlineto{\pgfqpoint{0.596099in}{1.599150in}}%
\pgfpathclose%
\pgfusepath{fill}%
\end{pgfscope}%
\begin{pgfscope}%
\pgfpathrectangle{\pgfqpoint{0.100000in}{0.100000in}}{\pgfqpoint{3.007045in}{1.925000in}}%
\pgfusepath{clip}%
\pgfsetbuttcap%
\pgfsetmiterjoin%
\definecolor{currentfill}{rgb}{0.326290,0.618624,0.802799}%
\pgfsetfillcolor{currentfill}%
\pgfsetlinewidth{0.000000pt}%
\definecolor{currentstroke}{rgb}{0.000000,0.000000,0.000000}%
\pgfsetstrokecolor{currentstroke}%
\pgfsetstrokeopacity{0.000000}%
\pgfsetdash{}{0pt}%
\pgfpathmoveto{\pgfqpoint{1.446935in}{1.181960in}}%
\pgfpathlineto{\pgfqpoint{1.414739in}{1.184340in}}%
\pgfpathlineto{\pgfqpoint{1.414767in}{1.184726in}}%
\pgfpathlineto{\pgfqpoint{1.418635in}{1.235643in}}%
\pgfpathlineto{\pgfqpoint{1.455104in}{1.233038in}}%
\pgfpathlineto{\pgfqpoint{1.453477in}{1.210184in}}%
\pgfpathlineto{\pgfqpoint{1.449096in}{1.210470in}}%
\pgfpathlineto{\pgfqpoint{1.446935in}{1.181960in}}%
\pgfpathclose%
\pgfusepath{fill}%
\end{pgfscope}%
\begin{pgfscope}%
\pgfpathrectangle{\pgfqpoint{0.100000in}{0.100000in}}{\pgfqpoint{3.007045in}{1.925000in}}%
\pgfusepath{clip}%
\pgfsetbuttcap%
\pgfsetmiterjoin%
\definecolor{currentfill}{rgb}{0.784591,0.864237,0.939962}%
\pgfsetfillcolor{currentfill}%
\pgfsetlinewidth{0.000000pt}%
\definecolor{currentstroke}{rgb}{0.000000,0.000000,0.000000}%
\pgfsetstrokecolor{currentstroke}%
\pgfsetstrokeopacity{0.000000}%
\pgfsetdash{}{0pt}%
\pgfpathmoveto{\pgfqpoint{0.774157in}{0.904247in}}%
\pgfpathlineto{\pgfqpoint{0.778392in}{0.924882in}}%
\pgfpathlineto{\pgfqpoint{0.794371in}{1.003188in}}%
\pgfpathlineto{\pgfqpoint{0.798014in}{1.020963in}}%
\pgfpathlineto{\pgfqpoint{0.800968in}{1.030983in}}%
\pgfpathlineto{\pgfqpoint{0.800098in}{1.036517in}}%
\pgfpathlineto{\pgfqpoint{0.803014in}{1.039936in}}%
\pgfpathlineto{\pgfqpoint{0.807519in}{1.038421in}}%
\pgfpathlineto{\pgfqpoint{0.809522in}{1.041947in}}%
\pgfpathlineto{\pgfqpoint{0.818051in}{1.044246in}}%
\pgfpathlineto{\pgfqpoint{0.832280in}{1.046410in}}%
\pgfpathlineto{\pgfqpoint{0.839109in}{1.048947in}}%
\pgfpathlineto{\pgfqpoint{0.842909in}{1.052579in}}%
\pgfpathlineto{\pgfqpoint{0.842643in}{1.059879in}}%
\pgfpathlineto{\pgfqpoint{0.845589in}{1.070038in}}%
\pgfpathlineto{\pgfqpoint{0.854802in}{1.090140in}}%
\pgfpathlineto{\pgfqpoint{0.836260in}{1.093783in}}%
\pgfpathlineto{\pgfqpoint{0.843170in}{1.128894in}}%
\pgfpathlineto{\pgfqpoint{0.854337in}{1.126666in}}%
\pgfpathlineto{\pgfqpoint{0.870763in}{1.123009in}}%
\pgfpathlineto{\pgfqpoint{0.937370in}{1.110856in}}%
\pgfpathlineto{\pgfqpoint{0.958933in}{1.107125in}}%
\pgfpathlineto{\pgfqpoint{0.951769in}{1.095219in}}%
\pgfpathlineto{\pgfqpoint{0.944689in}{1.092843in}}%
\pgfpathlineto{\pgfqpoint{0.941624in}{1.085677in}}%
\pgfpathlineto{\pgfqpoint{0.932657in}{1.082202in}}%
\pgfpathlineto{\pgfqpoint{0.926450in}{1.083530in}}%
\pgfpathlineto{\pgfqpoint{0.923032in}{1.079857in}}%
\pgfpathlineto{\pgfqpoint{0.915481in}{1.078785in}}%
\pgfpathlineto{\pgfqpoint{0.947453in}{1.073147in}}%
\pgfpathlineto{\pgfqpoint{0.916273in}{0.895048in}}%
\pgfpathlineto{\pgfqpoint{0.911088in}{0.896114in}}%
\pgfpathlineto{\pgfqpoint{0.903247in}{0.903940in}}%
\pgfpathlineto{\pgfqpoint{0.896277in}{0.906751in}}%
\pgfpathlineto{\pgfqpoint{0.890889in}{0.911616in}}%
\pgfpathlineto{\pgfqpoint{0.883290in}{0.911640in}}%
\pgfpathlineto{\pgfqpoint{0.876178in}{0.916207in}}%
\pgfpathlineto{\pgfqpoint{0.868902in}{0.911853in}}%
\pgfpathlineto{\pgfqpoint{0.863233in}{0.897098in}}%
\pgfpathlineto{\pgfqpoint{0.875838in}{0.894759in}}%
\pgfpathlineto{\pgfqpoint{0.873121in}{0.885088in}}%
\pgfpathlineto{\pgfqpoint{0.860696in}{0.887387in}}%
\pgfpathlineto{\pgfqpoint{0.837638in}{0.894986in}}%
\pgfpathlineto{\pgfqpoint{0.829566in}{0.885369in}}%
\pgfpathlineto{\pgfqpoint{0.805861in}{0.897916in}}%
\pgfpathlineto{\pgfqpoint{0.774157in}{0.904247in}}%
\pgfpathclose%
\pgfusepath{fill}%
\end{pgfscope}%
\begin{pgfscope}%
\pgfpathrectangle{\pgfqpoint{0.100000in}{0.100000in}}{\pgfqpoint{3.007045in}{1.925000in}}%
\pgfusepath{clip}%
\pgfsetbuttcap%
\pgfsetmiterjoin%
\definecolor{currentfill}{rgb}{0.529412,0.742637,0.862899}%
\pgfsetfillcolor{currentfill}%
\pgfsetlinewidth{0.000000pt}%
\definecolor{currentstroke}{rgb}{0.000000,0.000000,0.000000}%
\pgfsetstrokecolor{currentstroke}%
\pgfsetstrokeopacity{0.000000}%
\pgfsetdash{}{0pt}%
\pgfpathmoveto{\pgfqpoint{1.330895in}{1.191503in}}%
\pgfpathlineto{\pgfqpoint{1.325845in}{1.145704in}}%
\pgfpathlineto{\pgfqpoint{1.308723in}{1.147516in}}%
\pgfpathlineto{\pgfqpoint{1.310620in}{1.164502in}}%
\pgfpathlineto{\pgfqpoint{1.245658in}{1.171547in}}%
\pgfpathlineto{\pgfqpoint{1.242494in}{1.143237in}}%
\pgfpathlineto{\pgfqpoint{1.209760in}{1.146848in}}%
\pgfpathlineto{\pgfqpoint{1.213877in}{1.153875in}}%
\pgfpathlineto{\pgfqpoint{1.213448in}{1.158253in}}%
\pgfpathlineto{\pgfqpoint{1.208767in}{1.163511in}}%
\pgfpathlineto{\pgfqpoint{1.200990in}{1.169461in}}%
\pgfpathlineto{\pgfqpoint{1.201403in}{1.171929in}}%
\pgfpathlineto{\pgfqpoint{1.202987in}{1.177221in}}%
\pgfpathlineto{\pgfqpoint{1.203807in}{1.188442in}}%
\pgfpathlineto{\pgfqpoint{1.216380in}{1.193870in}}%
\pgfpathlineto{\pgfqpoint{1.226581in}{1.206947in}}%
\pgfpathlineto{\pgfqpoint{1.220004in}{1.212317in}}%
\pgfpathlineto{\pgfqpoint{1.225797in}{1.217368in}}%
\pgfpathlineto{\pgfqpoint{1.232976in}{1.220918in}}%
\pgfpathlineto{\pgfqpoint{1.234437in}{1.230208in}}%
\pgfpathlineto{\pgfqpoint{1.237002in}{1.232833in}}%
\pgfpathlineto{\pgfqpoint{1.237983in}{1.247422in}}%
\pgfpathlineto{\pgfqpoint{1.267871in}{1.244110in}}%
\pgfpathlineto{\pgfqpoint{1.265951in}{1.226957in}}%
\pgfpathlineto{\pgfqpoint{1.311373in}{1.222190in}}%
\pgfpathlineto{\pgfqpoint{1.333705in}{1.220024in}}%
\pgfpathlineto{\pgfqpoint{1.330895in}{1.191503in}}%
\pgfpathclose%
\pgfusepath{fill}%
\end{pgfscope}%
\begin{pgfscope}%
\pgfpathrectangle{\pgfqpoint{0.100000in}{0.100000in}}{\pgfqpoint{3.007045in}{1.925000in}}%
\pgfusepath{clip}%
\pgfsetbuttcap%
\pgfsetmiterjoin%
\definecolor{currentfill}{rgb}{0.080969,0.381130,0.661361}%
\pgfsetfillcolor{currentfill}%
\pgfsetlinewidth{0.000000pt}%
\definecolor{currentstroke}{rgb}{0.000000,0.000000,0.000000}%
\pgfsetstrokecolor{currentstroke}%
\pgfsetstrokeopacity{0.000000}%
\pgfsetdash{}{0pt}%
\pgfpathmoveto{\pgfqpoint{1.514641in}{0.483384in}}%
\pgfpathlineto{\pgfqpoint{1.473940in}{0.485183in}}%
\pgfpathlineto{\pgfqpoint{1.475915in}{0.520991in}}%
\pgfpathlineto{\pgfqpoint{1.441878in}{0.523060in}}%
\pgfpathlineto{\pgfqpoint{1.444627in}{0.567276in}}%
\pgfpathlineto{\pgfqpoint{1.429685in}{0.568203in}}%
\pgfpathlineto{\pgfqpoint{1.431514in}{0.595996in}}%
\pgfpathlineto{\pgfqpoint{1.479753in}{0.593339in}}%
\pgfpathlineto{\pgfqpoint{1.478167in}{0.565413in}}%
\pgfpathlineto{\pgfqpoint{1.498977in}{0.564304in}}%
\pgfpathlineto{\pgfqpoint{1.498020in}{0.549906in}}%
\pgfpathlineto{\pgfqpoint{1.501924in}{0.549696in}}%
\pgfpathlineto{\pgfqpoint{1.501272in}{0.538689in}}%
\pgfpathlineto{\pgfqpoint{1.506378in}{0.538351in}}%
\pgfpathlineto{\pgfqpoint{1.505414in}{0.519674in}}%
\pgfpathlineto{\pgfqpoint{1.516499in}{0.519146in}}%
\pgfpathlineto{\pgfqpoint{1.514641in}{0.483384in}}%
\pgfpathclose%
\pgfusepath{fill}%
\end{pgfscope}%
\begin{pgfscope}%
\pgfpathrectangle{\pgfqpoint{0.100000in}{0.100000in}}{\pgfqpoint{3.007045in}{1.925000in}}%
\pgfusepath{clip}%
\pgfsetbuttcap%
\pgfsetmiterjoin%
\definecolor{currentfill}{rgb}{0.485490,0.718524,0.853426}%
\pgfsetfillcolor{currentfill}%
\pgfsetlinewidth{0.000000pt}%
\definecolor{currentstroke}{rgb}{0.000000,0.000000,0.000000}%
\pgfsetstrokecolor{currentstroke}%
\pgfsetstrokeopacity{0.000000}%
\pgfsetdash{}{0pt}%
\pgfpathmoveto{\pgfqpoint{0.989331in}{0.759721in}}%
\pgfpathlineto{\pgfqpoint{0.997334in}{0.810949in}}%
\pgfpathlineto{\pgfqpoint{1.054598in}{0.801868in}}%
\pgfpathlineto{\pgfqpoint{1.061462in}{0.800953in}}%
\pgfpathlineto{\pgfqpoint{1.062578in}{0.792196in}}%
\pgfpathlineto{\pgfqpoint{1.061362in}{0.785689in}}%
\pgfpathlineto{\pgfqpoint{1.064806in}{0.782977in}}%
\pgfpathlineto{\pgfqpoint{1.063832in}{0.777370in}}%
\pgfpathlineto{\pgfqpoint{1.065981in}{0.771962in}}%
\pgfpathlineto{\pgfqpoint{1.063758in}{0.762298in}}%
\pgfpathlineto{\pgfqpoint{1.070654in}{0.759703in}}%
\pgfpathlineto{\pgfqpoint{1.087735in}{0.757358in}}%
\pgfpathlineto{\pgfqpoint{1.080529in}{0.703340in}}%
\pgfpathlineto{\pgfqpoint{1.029591in}{0.710510in}}%
\pgfpathlineto{\pgfqpoint{1.025251in}{0.680923in}}%
\pgfpathlineto{\pgfqpoint{0.978016in}{0.688001in}}%
\pgfpathlineto{\pgfqpoint{0.989331in}{0.759721in}}%
\pgfpathclose%
\pgfusepath{fill}%
\end{pgfscope}%
\begin{pgfscope}%
\pgfpathrectangle{\pgfqpoint{0.100000in}{0.100000in}}{\pgfqpoint{3.007045in}{1.925000in}}%
\pgfusepath{clip}%
\pgfsetbuttcap%
\pgfsetmiterjoin%
\definecolor{currentfill}{rgb}{0.296025,0.597955,0.790988}%
\pgfsetfillcolor{currentfill}%
\pgfsetlinewidth{0.000000pt}%
\definecolor{currentstroke}{rgb}{0.000000,0.000000,0.000000}%
\pgfsetstrokecolor{currentstroke}%
\pgfsetstrokeopacity{0.000000}%
\pgfsetdash{}{0pt}%
\pgfpathmoveto{\pgfqpoint{1.669507in}{1.592201in}}%
\pgfpathlineto{\pgfqpoint{1.634891in}{1.593241in}}%
\pgfpathlineto{\pgfqpoint{1.587550in}{1.595331in}}%
\pgfpathlineto{\pgfqpoint{1.588665in}{1.618076in}}%
\pgfpathlineto{\pgfqpoint{1.633089in}{1.616328in}}%
\pgfpathlineto{\pgfqpoint{1.633892in}{1.639443in}}%
\pgfpathlineto{\pgfqpoint{1.649938in}{1.638890in}}%
\pgfpathlineto{\pgfqpoint{1.668280in}{1.638340in}}%
\pgfpathlineto{\pgfqpoint{1.667675in}{1.615294in}}%
\pgfpathlineto{\pgfqpoint{1.668146in}{1.596346in}}%
\pgfpathlineto{\pgfqpoint{1.669507in}{1.592201in}}%
\pgfpathclose%
\pgfusepath{fill}%
\end{pgfscope}%
\begin{pgfscope}%
\pgfpathrectangle{\pgfqpoint{0.100000in}{0.100000in}}{\pgfqpoint{3.007045in}{1.925000in}}%
\pgfusepath{clip}%
\pgfsetbuttcap%
\pgfsetmiterjoin%
\definecolor{currentfill}{rgb}{0.696501,0.824837,0.909266}%
\pgfsetfillcolor{currentfill}%
\pgfsetlinewidth{0.000000pt}%
\definecolor{currentstroke}{rgb}{0.000000,0.000000,0.000000}%
\pgfsetstrokecolor{currentstroke}%
\pgfsetstrokeopacity{0.000000}%
\pgfsetdash{}{0pt}%
\pgfpathmoveto{\pgfqpoint{2.826283in}{1.627777in}}%
\pgfpathlineto{\pgfqpoint{2.827912in}{1.634106in}}%
\pgfpathlineto{\pgfqpoint{2.822013in}{1.635422in}}%
\pgfpathlineto{\pgfqpoint{2.823531in}{1.641882in}}%
\pgfpathlineto{\pgfqpoint{2.815249in}{1.643795in}}%
\pgfpathlineto{\pgfqpoint{2.816481in}{1.647652in}}%
\pgfpathlineto{\pgfqpoint{2.812663in}{1.658624in}}%
\pgfpathlineto{\pgfqpoint{2.823785in}{1.661144in}}%
\pgfpathlineto{\pgfqpoint{2.860264in}{1.671267in}}%
\pgfpathlineto{\pgfqpoint{2.862575in}{1.687976in}}%
\pgfpathlineto{\pgfqpoint{2.873035in}{1.690583in}}%
\pgfpathlineto{\pgfqpoint{2.874149in}{1.694991in}}%
\pgfpathlineto{\pgfqpoint{2.885278in}{1.661827in}}%
\pgfpathlineto{\pgfqpoint{2.895122in}{1.630668in}}%
\pgfpathlineto{\pgfqpoint{2.894447in}{1.627269in}}%
\pgfpathlineto{\pgfqpoint{2.884731in}{1.625099in}}%
\pgfpathlineto{\pgfqpoint{2.884101in}{1.617605in}}%
\pgfpathlineto{\pgfqpoint{2.877796in}{1.621050in}}%
\pgfpathlineto{\pgfqpoint{2.875904in}{1.623095in}}%
\pgfpathlineto{\pgfqpoint{2.869747in}{1.621836in}}%
\pgfpathlineto{\pgfqpoint{2.868383in}{1.624903in}}%
\pgfpathlineto{\pgfqpoint{2.862905in}{1.625347in}}%
\pgfpathlineto{\pgfqpoint{2.858546in}{1.629287in}}%
\pgfpathlineto{\pgfqpoint{2.854814in}{1.623529in}}%
\pgfpathlineto{\pgfqpoint{2.849656in}{1.622164in}}%
\pgfpathlineto{\pgfqpoint{2.846880in}{1.617175in}}%
\pgfpathlineto{\pgfqpoint{2.848338in}{1.611817in}}%
\pgfpathlineto{\pgfqpoint{2.837477in}{1.608579in}}%
\pgfpathlineto{\pgfqpoint{2.834224in}{1.609178in}}%
\pgfpathlineto{\pgfqpoint{2.837240in}{1.625008in}}%
\pgfpathlineto{\pgfqpoint{2.826283in}{1.627777in}}%
\pgfpathclose%
\pgfusepath{fill}%
\end{pgfscope}%
\begin{pgfscope}%
\pgfpathrectangle{\pgfqpoint{0.100000in}{0.100000in}}{\pgfqpoint{3.007045in}{1.925000in}}%
\pgfusepath{clip}%
\pgfsetbuttcap%
\pgfsetmiterjoin%
\definecolor{currentfill}{rgb}{0.162907,0.476632,0.727059}%
\pgfsetfillcolor{currentfill}%
\pgfsetlinewidth{0.000000pt}%
\definecolor{currentstroke}{rgb}{0.000000,0.000000,0.000000}%
\pgfsetstrokecolor{currentstroke}%
\pgfsetstrokeopacity{0.000000}%
\pgfsetdash{}{0pt}%
\pgfpathmoveto{\pgfqpoint{1.610638in}{1.328045in}}%
\pgfpathlineto{\pgfqpoint{1.611668in}{1.362462in}}%
\pgfpathlineto{\pgfqpoint{1.657059in}{1.361181in}}%
\pgfpathlineto{\pgfqpoint{1.656900in}{1.355436in}}%
\pgfpathlineto{\pgfqpoint{1.656422in}{1.338221in}}%
\pgfpathlineto{\pgfqpoint{1.673497in}{1.337786in}}%
\pgfpathlineto{\pgfqpoint{1.672961in}{1.314877in}}%
\pgfpathlineto{\pgfqpoint{1.633082in}{1.315887in}}%
\pgfpathlineto{\pgfqpoint{1.633318in}{1.327373in}}%
\pgfpathlineto{\pgfqpoint{1.610638in}{1.328045in}}%
\pgfpathclose%
\pgfusepath{fill}%
\end{pgfscope}%
\begin{pgfscope}%
\pgfpathrectangle{\pgfqpoint{0.100000in}{0.100000in}}{\pgfqpoint{3.007045in}{1.925000in}}%
\pgfusepath{clip}%
\pgfsetbuttcap%
\pgfsetmiterjoin%
\definecolor{currentfill}{rgb}{0.441569,0.694410,0.843952}%
\pgfsetfillcolor{currentfill}%
\pgfsetlinewidth{0.000000pt}%
\definecolor{currentstroke}{rgb}{0.000000,0.000000,0.000000}%
\pgfsetstrokecolor{currentstroke}%
\pgfsetstrokeopacity{0.000000}%
\pgfsetdash{}{0pt}%
\pgfpathmoveto{\pgfqpoint{1.333705in}{1.220024in}}%
\pgfpathlineto{\pgfqpoint{1.311373in}{1.222190in}}%
\pgfpathlineto{\pgfqpoint{1.265951in}{1.226957in}}%
\pgfpathlineto{\pgfqpoint{1.267871in}{1.244110in}}%
\pgfpathlineto{\pgfqpoint{1.237983in}{1.247422in}}%
\pgfpathlineto{\pgfqpoint{1.233045in}{1.254595in}}%
\pgfpathlineto{\pgfqpoint{1.229691in}{1.263388in}}%
\pgfpathlineto{\pgfqpoint{1.222250in}{1.286251in}}%
\pgfpathlineto{\pgfqpoint{1.216879in}{1.294670in}}%
\pgfpathlineto{\pgfqpoint{1.217135in}{1.298862in}}%
\pgfpathlineto{\pgfqpoint{1.262254in}{1.293547in}}%
\pgfpathlineto{\pgfqpoint{1.294492in}{1.290219in}}%
\pgfpathlineto{\pgfqpoint{1.346503in}{1.285013in}}%
\pgfpathlineto{\pgfqpoint{1.343169in}{1.253667in}}%
\pgfpathlineto{\pgfqpoint{1.348876in}{1.253120in}}%
\pgfpathlineto{\pgfqpoint{1.345447in}{1.218910in}}%
\pgfpathlineto{\pgfqpoint{1.333705in}{1.220024in}}%
\pgfpathclose%
\pgfusepath{fill}%
\end{pgfscope}%
\begin{pgfscope}%
\pgfpathrectangle{\pgfqpoint{0.100000in}{0.100000in}}{\pgfqpoint{3.007045in}{1.925000in}}%
\pgfusepath{clip}%
\pgfsetbuttcap%
\pgfsetmiterjoin%
\definecolor{currentfill}{rgb}{0.491765,0.721968,0.854779}%
\pgfsetfillcolor{currentfill}%
\pgfsetlinewidth{0.000000pt}%
\definecolor{currentstroke}{rgb}{0.000000,0.000000,0.000000}%
\pgfsetstrokecolor{currentstroke}%
\pgfsetstrokeopacity{0.000000}%
\pgfsetdash{}{0pt}%
\pgfpathmoveto{\pgfqpoint{0.741011in}{1.724091in}}%
\pgfpathlineto{\pgfqpoint{0.746177in}{1.746042in}}%
\pgfpathlineto{\pgfqpoint{0.743350in}{1.746702in}}%
\pgfpathlineto{\pgfqpoint{0.742787in}{1.756014in}}%
\pgfpathlineto{\pgfqpoint{0.737197in}{1.757418in}}%
\pgfpathlineto{\pgfqpoint{0.740246in}{1.767188in}}%
\pgfpathlineto{\pgfqpoint{0.748808in}{1.770376in}}%
\pgfpathlineto{\pgfqpoint{0.752113in}{1.767443in}}%
\pgfpathlineto{\pgfqpoint{0.758923in}{1.767006in}}%
\pgfpathlineto{\pgfqpoint{0.760765in}{1.761473in}}%
\pgfpathlineto{\pgfqpoint{0.765426in}{1.755324in}}%
\pgfpathlineto{\pgfqpoint{0.765843in}{1.746677in}}%
\pgfpathlineto{\pgfqpoint{0.773003in}{1.745268in}}%
\pgfpathlineto{\pgfqpoint{0.774865in}{1.752771in}}%
\pgfpathlineto{\pgfqpoint{0.789165in}{1.749329in}}%
\pgfpathlineto{\pgfqpoint{0.794807in}{1.753888in}}%
\pgfpathlineto{\pgfqpoint{0.807745in}{1.750662in}}%
\pgfpathlineto{\pgfqpoint{0.812473in}{1.770492in}}%
\pgfpathlineto{\pgfqpoint{0.844253in}{1.762934in}}%
\pgfpathlineto{\pgfqpoint{0.873084in}{1.756371in}}%
\pgfpathlineto{\pgfqpoint{0.873285in}{1.749159in}}%
\pgfpathlineto{\pgfqpoint{0.884881in}{1.738405in}}%
\pgfpathlineto{\pgfqpoint{0.884196in}{1.733912in}}%
\pgfpathlineto{\pgfqpoint{0.895843in}{1.733369in}}%
\pgfpathlineto{\pgfqpoint{0.897388in}{1.731726in}}%
\pgfpathlineto{\pgfqpoint{0.891175in}{1.716853in}}%
\pgfpathlineto{\pgfqpoint{0.885687in}{1.708694in}}%
\pgfpathlineto{\pgfqpoint{0.885040in}{1.701682in}}%
\pgfpathlineto{\pgfqpoint{0.881292in}{1.694653in}}%
\pgfpathlineto{\pgfqpoint{0.883784in}{1.683111in}}%
\pgfpathlineto{\pgfqpoint{0.877655in}{1.681396in}}%
\pgfpathlineto{\pgfqpoint{0.874072in}{1.677313in}}%
\pgfpathlineto{\pgfqpoint{0.875599in}{1.669863in}}%
\pgfpathlineto{\pgfqpoint{0.871070in}{1.663297in}}%
\pgfpathlineto{\pgfqpoint{0.865364in}{1.658676in}}%
\pgfpathlineto{\pgfqpoint{0.859318in}{1.660383in}}%
\pgfpathlineto{\pgfqpoint{0.865574in}{1.646714in}}%
\pgfpathlineto{\pgfqpoint{0.860069in}{1.641241in}}%
\pgfpathlineto{\pgfqpoint{0.802276in}{1.654267in}}%
\pgfpathlineto{\pgfqpoint{0.792690in}{1.650559in}}%
\pgfpathlineto{\pgfqpoint{0.786641in}{1.653470in}}%
\pgfpathlineto{\pgfqpoint{0.786144in}{1.663045in}}%
\pgfpathlineto{\pgfqpoint{0.770539in}{1.666797in}}%
\pgfpathlineto{\pgfqpoint{0.780091in}{1.680597in}}%
\pgfpathlineto{\pgfqpoint{0.786067in}{1.686768in}}%
\pgfpathlineto{\pgfqpoint{0.784656in}{1.695189in}}%
\pgfpathlineto{\pgfqpoint{0.778737in}{1.702487in}}%
\pgfpathlineto{\pgfqpoint{0.774980in}{1.703762in}}%
\pgfpathlineto{\pgfqpoint{0.771689in}{1.716185in}}%
\pgfpathlineto{\pgfqpoint{0.741011in}{1.724091in}}%
\pgfpathclose%
\pgfusepath{fill}%
\end{pgfscope}%
\begin{pgfscope}%
\pgfpathrectangle{\pgfqpoint{0.100000in}{0.100000in}}{\pgfqpoint{3.007045in}{1.925000in}}%
\pgfusepath{clip}%
\pgfsetbuttcap%
\pgfsetmiterjoin%
\definecolor{currentfill}{rgb}{0.671895,0.814379,0.900654}%
\pgfsetfillcolor{currentfill}%
\pgfsetlinewidth{0.000000pt}%
\definecolor{currentstroke}{rgb}{0.000000,0.000000,0.000000}%
\pgfsetstrokecolor{currentstroke}%
\pgfsetstrokeopacity{0.000000}%
\pgfsetdash{}{0pt}%
\pgfpathmoveto{\pgfqpoint{2.492884in}{0.596677in}}%
\pgfpathlineto{\pgfqpoint{2.516161in}{0.600413in}}%
\pgfpathlineto{\pgfqpoint{2.520191in}{0.572396in}}%
\pgfpathlineto{\pgfqpoint{2.522149in}{0.556021in}}%
\pgfpathlineto{\pgfqpoint{2.513609in}{0.551975in}}%
\pgfpathlineto{\pgfqpoint{2.513059in}{0.555566in}}%
\pgfpathlineto{\pgfqpoint{2.501841in}{0.554053in}}%
\pgfpathlineto{\pgfqpoint{2.504526in}{0.536259in}}%
\pgfpathlineto{\pgfqpoint{2.496864in}{0.535136in}}%
\pgfpathlineto{\pgfqpoint{2.498437in}{0.523892in}}%
\pgfpathlineto{\pgfqpoint{2.494299in}{0.520921in}}%
\pgfpathlineto{\pgfqpoint{2.488304in}{0.521528in}}%
\pgfpathlineto{\pgfqpoint{2.486048in}{0.518807in}}%
\pgfpathlineto{\pgfqpoint{2.485328in}{0.522899in}}%
\pgfpathlineto{\pgfqpoint{2.480931in}{0.529259in}}%
\pgfpathlineto{\pgfqpoint{2.469105in}{0.528163in}}%
\pgfpathlineto{\pgfqpoint{2.464882in}{0.534167in}}%
\pgfpathlineto{\pgfqpoint{2.458365in}{0.538964in}}%
\pgfpathlineto{\pgfqpoint{2.455703in}{0.543435in}}%
\pgfpathlineto{\pgfqpoint{2.450937in}{0.544451in}}%
\pgfpathlineto{\pgfqpoint{2.444679in}{0.548491in}}%
\pgfpathlineto{\pgfqpoint{2.442205in}{0.558231in}}%
\pgfpathlineto{\pgfqpoint{2.444618in}{0.558566in}}%
\pgfpathlineto{\pgfqpoint{2.446412in}{0.569336in}}%
\pgfpathlineto{\pgfqpoint{2.470809in}{0.572873in}}%
\pgfpathlineto{\pgfqpoint{2.471783in}{0.578648in}}%
\pgfpathlineto{\pgfqpoint{2.476107in}{0.580682in}}%
\pgfpathlineto{\pgfqpoint{2.479736in}{0.576043in}}%
\pgfpathlineto{\pgfqpoint{2.485231in}{0.574838in}}%
\pgfpathlineto{\pgfqpoint{2.487697in}{0.577489in}}%
\pgfpathlineto{\pgfqpoint{2.488025in}{0.585155in}}%
\pgfpathlineto{\pgfqpoint{2.492884in}{0.596677in}}%
\pgfpathclose%
\pgfusepath{fill}%
\end{pgfscope}%
\begin{pgfscope}%
\pgfpathrectangle{\pgfqpoint{0.100000in}{0.100000in}}{\pgfqpoint{3.007045in}{1.925000in}}%
\pgfusepath{clip}%
\pgfsetbuttcap%
\pgfsetmiterjoin%
\definecolor{currentfill}{rgb}{0.346467,0.632403,0.810673}%
\pgfsetfillcolor{currentfill}%
\pgfsetlinewidth{0.000000pt}%
\definecolor{currentstroke}{rgb}{0.000000,0.000000,0.000000}%
\pgfsetstrokecolor{currentstroke}%
\pgfsetstrokeopacity{0.000000}%
\pgfsetdash{}{0pt}%
\pgfpathmoveto{\pgfqpoint{2.364743in}{1.174580in}}%
\pgfpathlineto{\pgfqpoint{2.370900in}{1.172765in}}%
\pgfpathlineto{\pgfqpoint{2.374431in}{1.146809in}}%
\pgfpathlineto{\pgfqpoint{2.360800in}{1.150692in}}%
\pgfpathlineto{\pgfqpoint{2.356171in}{1.148869in}}%
\pgfpathlineto{\pgfqpoint{2.354011in}{1.144847in}}%
\pgfpathlineto{\pgfqpoint{2.348520in}{1.145975in}}%
\pgfpathlineto{\pgfqpoint{2.342358in}{1.152328in}}%
\pgfpathlineto{\pgfqpoint{2.336776in}{1.153283in}}%
\pgfpathlineto{\pgfqpoint{2.331133in}{1.151744in}}%
\pgfpathlineto{\pgfqpoint{2.323384in}{1.154002in}}%
\pgfpathlineto{\pgfqpoint{2.326824in}{1.138484in}}%
\pgfpathlineto{\pgfqpoint{2.317016in}{1.135906in}}%
\pgfpathlineto{\pgfqpoint{2.308461in}{1.129091in}}%
\pgfpathlineto{\pgfqpoint{2.297453in}{1.137486in}}%
\pgfpathlineto{\pgfqpoint{2.296120in}{1.144067in}}%
\pgfpathlineto{\pgfqpoint{2.288955in}{1.139422in}}%
\pgfpathlineto{\pgfqpoint{2.283647in}{1.145438in}}%
\pgfpathlineto{\pgfqpoint{2.290373in}{1.148367in}}%
\pgfpathlineto{\pgfqpoint{2.294975in}{1.154537in}}%
\pgfpathlineto{\pgfqpoint{2.289964in}{1.156159in}}%
\pgfpathlineto{\pgfqpoint{2.291496in}{1.161362in}}%
\pgfpathlineto{\pgfqpoint{2.288006in}{1.165183in}}%
\pgfpathlineto{\pgfqpoint{2.291530in}{1.169095in}}%
\pgfpathlineto{\pgfqpoint{2.288706in}{1.196382in}}%
\pgfpathlineto{\pgfqpoint{2.288393in}{1.199350in}}%
\pgfpathlineto{\pgfqpoint{2.305334in}{1.201390in}}%
\pgfpathlineto{\pgfqpoint{2.314937in}{1.203868in}}%
\pgfpathlineto{\pgfqpoint{2.330635in}{1.204428in}}%
\pgfpathlineto{\pgfqpoint{2.331627in}{1.183748in}}%
\pgfpathlineto{\pgfqpoint{2.338836in}{1.184076in}}%
\pgfpathlineto{\pgfqpoint{2.340324in}{1.169258in}}%
\pgfpathlineto{\pgfqpoint{2.353578in}{1.170749in}}%
\pgfpathlineto{\pgfqpoint{2.364743in}{1.174580in}}%
\pgfpathclose%
\pgfusepath{fill}%
\end{pgfscope}%
\begin{pgfscope}%
\pgfpathrectangle{\pgfqpoint{0.100000in}{0.100000in}}{\pgfqpoint{3.007045in}{1.925000in}}%
\pgfusepath{clip}%
\pgfsetbuttcap%
\pgfsetmiterjoin%
\definecolor{currentfill}{rgb}{0.422745,0.684075,0.839892}%
\pgfsetfillcolor{currentfill}%
\pgfsetlinewidth{0.000000pt}%
\definecolor{currentstroke}{rgb}{0.000000,0.000000,0.000000}%
\pgfsetstrokecolor{currentstroke}%
\pgfsetstrokeopacity{0.000000}%
\pgfsetdash{}{0pt}%
\pgfpathmoveto{\pgfqpoint{2.439065in}{1.211016in}}%
\pgfpathlineto{\pgfqpoint{2.428341in}{1.210131in}}%
\pgfpathlineto{\pgfqpoint{2.427935in}{1.216104in}}%
\pgfpathlineto{\pgfqpoint{2.422257in}{1.215672in}}%
\pgfpathlineto{\pgfqpoint{2.410789in}{1.216745in}}%
\pgfpathlineto{\pgfqpoint{2.410154in}{1.226352in}}%
\pgfpathlineto{\pgfqpoint{2.407820in}{1.232038in}}%
\pgfpathlineto{\pgfqpoint{2.403956in}{1.231787in}}%
\pgfpathlineto{\pgfqpoint{2.403598in}{1.237933in}}%
\pgfpathlineto{\pgfqpoint{2.415203in}{1.238420in}}%
\pgfpathlineto{\pgfqpoint{2.416625in}{1.241076in}}%
\pgfpathlineto{\pgfqpoint{2.415184in}{1.255293in}}%
\pgfpathlineto{\pgfqpoint{2.441070in}{1.257982in}}%
\pgfpathlineto{\pgfqpoint{2.440547in}{1.262665in}}%
\pgfpathlineto{\pgfqpoint{2.457021in}{1.264583in}}%
\pgfpathlineto{\pgfqpoint{2.463037in}{1.262554in}}%
\pgfpathlineto{\pgfqpoint{2.465474in}{1.242785in}}%
\pgfpathlineto{\pgfqpoint{2.461646in}{1.242312in}}%
\pgfpathlineto{\pgfqpoint{2.462970in}{1.231712in}}%
\pgfpathlineto{\pgfqpoint{2.465782in}{1.225372in}}%
\pgfpathlineto{\pgfqpoint{2.456449in}{1.222465in}}%
\pgfpathlineto{\pgfqpoint{2.445563in}{1.221325in}}%
\pgfpathlineto{\pgfqpoint{2.445630in}{1.214029in}}%
\pgfpathlineto{\pgfqpoint{2.439738in}{1.214050in}}%
\pgfpathlineto{\pgfqpoint{2.439065in}{1.211016in}}%
\pgfpathclose%
\pgfusepath{fill}%
\end{pgfscope}%
\begin{pgfscope}%
\pgfpathrectangle{\pgfqpoint{0.100000in}{0.100000in}}{\pgfqpoint{3.007045in}{1.925000in}}%
\pgfusepath{clip}%
\pgfsetbuttcap%
\pgfsetmiterjoin%
\definecolor{currentfill}{rgb}{0.366644,0.646182,0.818547}%
\pgfsetfillcolor{currentfill}%
\pgfsetlinewidth{0.000000pt}%
\definecolor{currentstroke}{rgb}{0.000000,0.000000,0.000000}%
\pgfsetstrokecolor{currentstroke}%
\pgfsetstrokeopacity{0.000000}%
\pgfsetdash{}{0pt}%
\pgfpathmoveto{\pgfqpoint{1.704306in}{1.528001in}}%
\pgfpathlineto{\pgfqpoint{1.684152in}{1.528327in}}%
\pgfpathlineto{\pgfqpoint{1.684336in}{1.539811in}}%
\pgfpathlineto{\pgfqpoint{1.668319in}{1.540277in}}%
\pgfpathlineto{\pgfqpoint{1.668554in}{1.549944in}}%
\pgfpathlineto{\pgfqpoint{1.678758in}{1.549649in}}%
\pgfpathlineto{\pgfqpoint{1.679482in}{1.551623in}}%
\pgfpathlineto{\pgfqpoint{1.703819in}{1.551137in}}%
\pgfpathlineto{\pgfqpoint{1.701484in}{1.554430in}}%
\pgfpathlineto{\pgfqpoint{1.694140in}{1.556870in}}%
\pgfpathlineto{\pgfqpoint{1.686157in}{1.569986in}}%
\pgfpathlineto{\pgfqpoint{1.687099in}{1.572723in}}%
\pgfpathlineto{\pgfqpoint{1.695859in}{1.579141in}}%
\pgfpathlineto{\pgfqpoint{1.699144in}{1.583940in}}%
\pgfpathlineto{\pgfqpoint{1.700191in}{1.591568in}}%
\pgfpathlineto{\pgfqpoint{1.699675in}{1.597325in}}%
\pgfpathlineto{\pgfqpoint{1.714002in}{1.597084in}}%
\pgfpathlineto{\pgfqpoint{1.714476in}{1.591318in}}%
\pgfpathlineto{\pgfqpoint{1.714182in}{1.568146in}}%
\pgfpathlineto{\pgfqpoint{1.714573in}{1.556685in}}%
\pgfpathlineto{\pgfqpoint{1.720360in}{1.556564in}}%
\pgfpathlineto{\pgfqpoint{1.720167in}{1.545159in}}%
\pgfpathlineto{\pgfqpoint{1.717578in}{1.542068in}}%
\pgfpathlineto{\pgfqpoint{1.712446in}{1.545741in}}%
\pgfpathlineto{\pgfqpoint{1.704575in}{1.547315in}}%
\pgfpathlineto{\pgfqpoint{1.704306in}{1.528001in}}%
\pgfpathclose%
\pgfusepath{fill}%
\end{pgfscope}%
\begin{pgfscope}%
\pgfpathrectangle{\pgfqpoint{0.100000in}{0.100000in}}{\pgfqpoint{3.007045in}{1.925000in}}%
\pgfusepath{clip}%
\pgfsetbuttcap%
\pgfsetmiterjoin%
\definecolor{currentfill}{rgb}{0.175087,0.488812,0.733333}%
\pgfsetfillcolor{currentfill}%
\pgfsetlinewidth{0.000000pt}%
\definecolor{currentstroke}{rgb}{0.000000,0.000000,0.000000}%
\pgfsetstrokecolor{currentstroke}%
\pgfsetstrokeopacity{0.000000}%
\pgfsetdash{}{0pt}%
\pgfpathmoveto{\pgfqpoint{2.234453in}{1.308012in}}%
\pgfpathlineto{\pgfqpoint{2.233410in}{1.317320in}}%
\pgfpathlineto{\pgfqpoint{2.213486in}{1.315239in}}%
\pgfpathlineto{\pgfqpoint{2.211201in}{1.336530in}}%
\pgfpathlineto{\pgfqpoint{2.224488in}{1.337759in}}%
\pgfpathlineto{\pgfqpoint{2.225748in}{1.340539in}}%
\pgfpathlineto{\pgfqpoint{2.223806in}{1.358262in}}%
\pgfpathlineto{\pgfqpoint{2.269373in}{1.363358in}}%
\pgfpathlineto{\pgfqpoint{2.271732in}{1.342879in}}%
\pgfpathlineto{\pgfqpoint{2.253575in}{1.340858in}}%
\pgfpathlineto{\pgfqpoint{2.257315in}{1.308409in}}%
\pgfpathlineto{\pgfqpoint{2.240301in}{1.306723in}}%
\pgfpathlineto{\pgfqpoint{2.234453in}{1.308012in}}%
\pgfpathclose%
\pgfusepath{fill}%
\end{pgfscope}%
\begin{pgfscope}%
\pgfpathrectangle{\pgfqpoint{0.100000in}{0.100000in}}{\pgfqpoint{3.007045in}{1.925000in}}%
\pgfusepath{clip}%
\pgfsetbuttcap%
\pgfsetmiterjoin%
\definecolor{currentfill}{rgb}{0.760477,0.852026,0.931657}%
\pgfsetfillcolor{currentfill}%
\pgfsetlinewidth{0.000000pt}%
\definecolor{currentstroke}{rgb}{0.000000,0.000000,0.000000}%
\pgfsetstrokecolor{currentstroke}%
\pgfsetstrokeopacity{0.000000}%
\pgfsetdash{}{0pt}%
\pgfpathmoveto{\pgfqpoint{1.229691in}{1.263388in}}%
\pgfpathlineto{\pgfqpoint{1.233045in}{1.254595in}}%
\pgfpathlineto{\pgfqpoint{1.237983in}{1.247422in}}%
\pgfpathlineto{\pgfqpoint{1.237002in}{1.232833in}}%
\pgfpathlineto{\pgfqpoint{1.234437in}{1.230208in}}%
\pgfpathlineto{\pgfqpoint{1.232976in}{1.220918in}}%
\pgfpathlineto{\pgfqpoint{1.225797in}{1.217368in}}%
\pgfpathlineto{\pgfqpoint{1.220004in}{1.212317in}}%
\pgfpathlineto{\pgfqpoint{1.214794in}{1.212322in}}%
\pgfpathlineto{\pgfqpoint{1.212843in}{1.220333in}}%
\pgfpathlineto{\pgfqpoint{1.205530in}{1.228275in}}%
\pgfpathlineto{\pgfqpoint{1.186601in}{1.231472in}}%
\pgfpathlineto{\pgfqpoint{1.186728in}{1.236591in}}%
\pgfpathlineto{\pgfqpoint{1.189821in}{1.258740in}}%
\pgfpathlineto{\pgfqpoint{1.189706in}{1.265642in}}%
\pgfpathlineto{\pgfqpoint{1.192494in}{1.261100in}}%
\pgfpathlineto{\pgfqpoint{1.205670in}{1.255974in}}%
\pgfpathlineto{\pgfqpoint{1.213800in}{1.255082in}}%
\pgfpathlineto{\pgfqpoint{1.220648in}{1.256986in}}%
\pgfpathlineto{\pgfqpoint{1.222855in}{1.255180in}}%
\pgfpathlineto{\pgfqpoint{1.229691in}{1.263388in}}%
\pgfpathclose%
\pgfusepath{fill}%
\end{pgfscope}%
\begin{pgfscope}%
\pgfpathrectangle{\pgfqpoint{0.100000in}{0.100000in}}{\pgfqpoint{3.007045in}{1.925000in}}%
\pgfusepath{clip}%
\pgfsetbuttcap%
\pgfsetmiterjoin%
\definecolor{currentfill}{rgb}{0.256286,0.570012,0.775163}%
\pgfsetfillcolor{currentfill}%
\pgfsetlinewidth{0.000000pt}%
\definecolor{currentstroke}{rgb}{0.000000,0.000000,0.000000}%
\pgfsetstrokecolor{currentstroke}%
\pgfsetstrokeopacity{0.000000}%
\pgfsetdash{}{0pt}%
\pgfpathmoveto{\pgfqpoint{2.215737in}{1.157449in}}%
\pgfpathlineto{\pgfqpoint{2.213096in}{1.157168in}}%
\pgfpathlineto{\pgfqpoint{2.210701in}{1.176219in}}%
\pgfpathlineto{\pgfqpoint{2.198043in}{1.175099in}}%
\pgfpathlineto{\pgfqpoint{2.197194in}{1.183592in}}%
\pgfpathlineto{\pgfqpoint{2.194423in}{1.183326in}}%
\pgfpathlineto{\pgfqpoint{2.195596in}{1.188844in}}%
\pgfpathlineto{\pgfqpoint{2.193431in}{1.193828in}}%
\pgfpathlineto{\pgfqpoint{2.189712in}{1.229760in}}%
\pgfpathlineto{\pgfqpoint{2.212424in}{1.232038in}}%
\pgfpathlineto{\pgfqpoint{2.212181in}{1.234336in}}%
\pgfpathlineto{\pgfqpoint{2.231222in}{1.236425in}}%
\pgfpathlineto{\pgfqpoint{2.233016in}{1.218375in}}%
\pgfpathlineto{\pgfqpoint{2.247375in}{1.219993in}}%
\pgfpathlineto{\pgfqpoint{2.247429in}{1.209448in}}%
\pgfpathlineto{\pgfqpoint{2.262289in}{1.211080in}}%
\pgfpathlineto{\pgfqpoint{2.264298in}{1.193927in}}%
\pgfpathlineto{\pgfqpoint{2.264834in}{1.189201in}}%
\pgfpathlineto{\pgfqpoint{2.248039in}{1.187423in}}%
\pgfpathlineto{\pgfqpoint{2.245949in}{1.180376in}}%
\pgfpathlineto{\pgfqpoint{2.225671in}{1.177961in}}%
\pgfpathlineto{\pgfqpoint{2.227838in}{1.158721in}}%
\pgfpathlineto{\pgfqpoint{2.215737in}{1.157449in}}%
\pgfpathclose%
\pgfusepath{fill}%
\end{pgfscope}%
\begin{pgfscope}%
\pgfpathrectangle{\pgfqpoint{0.100000in}{0.100000in}}{\pgfqpoint{3.007045in}{1.925000in}}%
\pgfusepath{clip}%
\pgfsetbuttcap%
\pgfsetmiterjoin%
\definecolor{currentfill}{rgb}{0.154787,0.468512,0.722876}%
\pgfsetfillcolor{currentfill}%
\pgfsetlinewidth{0.000000pt}%
\definecolor{currentstroke}{rgb}{0.000000,0.000000,0.000000}%
\pgfsetstrokecolor{currentstroke}%
\pgfsetstrokeopacity{0.000000}%
\pgfsetdash{}{0pt}%
\pgfpathmoveto{\pgfqpoint{0.765905in}{0.863949in}}%
\pgfpathlineto{\pgfqpoint{0.754327in}{0.807329in}}%
\pgfpathlineto{\pgfqpoint{0.748143in}{0.777058in}}%
\pgfpathlineto{\pgfqpoint{0.673269in}{0.823873in}}%
\pgfpathlineto{\pgfqpoint{0.675440in}{0.831946in}}%
\pgfpathlineto{\pgfqpoint{0.681603in}{0.837269in}}%
\pgfpathlineto{\pgfqpoint{0.604721in}{0.848024in}}%
\pgfpathlineto{\pgfqpoint{0.611833in}{0.877391in}}%
\pgfpathlineto{\pgfqpoint{0.613032in}{0.877119in}}%
\pgfpathlineto{\pgfqpoint{0.618157in}{0.899792in}}%
\pgfpathlineto{\pgfqpoint{0.673193in}{0.887262in}}%
\pgfpathlineto{\pgfqpoint{0.691055in}{0.881658in}}%
\pgfpathlineto{\pgfqpoint{0.691629in}{0.872688in}}%
\pgfpathlineto{\pgfqpoint{0.687845in}{0.862115in}}%
\pgfpathlineto{\pgfqpoint{0.689447in}{0.856914in}}%
\pgfpathlineto{\pgfqpoint{0.710499in}{0.852063in}}%
\pgfpathlineto{\pgfqpoint{0.716592in}{0.880125in}}%
\pgfpathlineto{\pgfqpoint{0.733331in}{0.876625in}}%
\pgfpathlineto{\pgfqpoint{0.732137in}{0.870987in}}%
\pgfpathlineto{\pgfqpoint{0.765905in}{0.863949in}}%
\pgfpathclose%
\pgfusepath{fill}%
\end{pgfscope}%
\begin{pgfscope}%
\pgfpathrectangle{\pgfqpoint{0.100000in}{0.100000in}}{\pgfqpoint{3.007045in}{1.925000in}}%
\pgfusepath{clip}%
\pgfsetbuttcap%
\pgfsetroundjoin%
\pgfsetlinewidth{0.050187pt}%
\definecolor{currentstroke}{rgb}{1.000000,1.000000,1.000000}%
\pgfsetstrokecolor{currentstroke}%
\pgfsetdash{}{0pt}%
\pgfusepath{stroke}%
\end{pgfscope}%
\begin{pgfscope}%
\pgfpathrectangle{\pgfqpoint{0.100000in}{0.100000in}}{\pgfqpoint{3.007045in}{1.925000in}}%
\pgfusepath{clip}%
\pgfsetbuttcap%
\pgfsetroundjoin%
\pgfsetlinewidth{0.050187pt}%
\definecolor{currentstroke}{rgb}{1.000000,1.000000,1.000000}%
\pgfsetstrokecolor{currentstroke}%
\pgfsetdash{}{0pt}%
\pgfusepath{stroke}%
\end{pgfscope}%
\begin{pgfscope}%
\pgfpathrectangle{\pgfqpoint{0.100000in}{0.100000in}}{\pgfqpoint{3.007045in}{1.925000in}}%
\pgfusepath{clip}%
\pgfsetbuttcap%
\pgfsetroundjoin%
\pgfsetlinewidth{0.050187pt}%
\definecolor{currentstroke}{rgb}{1.000000,1.000000,1.000000}%
\pgfsetstrokecolor{currentstroke}%
\pgfsetdash{}{0pt}%
\pgfusepath{stroke}%
\end{pgfscope}%
\begin{pgfscope}%
\pgfpathrectangle{\pgfqpoint{0.100000in}{0.100000in}}{\pgfqpoint{3.007045in}{1.925000in}}%
\pgfusepath{clip}%
\pgfsetbuttcap%
\pgfsetroundjoin%
\pgfsetlinewidth{0.050187pt}%
\definecolor{currentstroke}{rgb}{1.000000,1.000000,1.000000}%
\pgfsetstrokecolor{currentstroke}%
\pgfsetdash{}{0pt}%
\pgfusepath{stroke}%
\end{pgfscope}%
\begin{pgfscope}%
\pgfpathrectangle{\pgfqpoint{0.100000in}{0.100000in}}{\pgfqpoint{3.007045in}{1.925000in}}%
\pgfusepath{clip}%
\pgfsetbuttcap%
\pgfsetroundjoin%
\pgfsetlinewidth{0.050187pt}%
\definecolor{currentstroke}{rgb}{1.000000,1.000000,1.000000}%
\pgfsetstrokecolor{currentstroke}%
\pgfsetdash{}{0pt}%
\pgfusepath{stroke}%
\end{pgfscope}%
\begin{pgfscope}%
\pgfpathrectangle{\pgfqpoint{0.100000in}{0.100000in}}{\pgfqpoint{3.007045in}{1.925000in}}%
\pgfusepath{clip}%
\pgfsetbuttcap%
\pgfsetroundjoin%
\pgfsetlinewidth{0.050187pt}%
\definecolor{currentstroke}{rgb}{1.000000,1.000000,1.000000}%
\pgfsetstrokecolor{currentstroke}%
\pgfsetdash{}{0pt}%
\pgfusepath{stroke}%
\end{pgfscope}%
\begin{pgfscope}%
\pgfpathrectangle{\pgfqpoint{0.100000in}{0.100000in}}{\pgfqpoint{3.007045in}{1.925000in}}%
\pgfusepath{clip}%
\pgfsetbuttcap%
\pgfsetroundjoin%
\pgfsetlinewidth{0.050187pt}%
\definecolor{currentstroke}{rgb}{1.000000,1.000000,1.000000}%
\pgfsetstrokecolor{currentstroke}%
\pgfsetdash{}{0pt}%
\pgfpathmoveto{\pgfqpoint{0.814503in}{1.911787in}}%
\pgfpathlineto{\pgfqpoint{0.798909in}{1.849997in}}%
\pgfpathlineto{\pgfqpoint{0.787821in}{1.805724in}}%
\pgfpathlineto{\pgfqpoint{0.774637in}{1.752465in}}%
\pgfpathlineto{\pgfqpoint{0.772049in}{1.739338in}}%
\pgfpathlineto{\pgfqpoint{0.773739in}{1.727298in}}%
\pgfpathlineto{\pgfqpoint{0.771499in}{1.716207in}}%
\pgfpathlineto{\pgfqpoint{0.679470in}{1.740222in}}%
\pgfpathlineto{\pgfqpoint{0.671157in}{1.737433in}}%
\pgfpathlineto{\pgfqpoint{0.664042in}{1.739853in}}%
\pgfpathlineto{\pgfqpoint{0.631104in}{1.740645in}}%
\pgfpathlineto{\pgfqpoint{0.619963in}{1.737504in}}%
\pgfpathlineto{\pgfqpoint{0.608900in}{1.738565in}}%
\pgfpathlineto{\pgfqpoint{0.604211in}{1.743481in}}%
\pgfpathlineto{\pgfqpoint{0.574722in}{1.742409in}}%
\pgfpathlineto{\pgfqpoint{0.570064in}{1.750443in}}%
\pgfpathlineto{\pgfqpoint{0.561223in}{1.754676in}}%
\pgfpathlineto{\pgfqpoint{0.548360in}{1.757255in}}%
\pgfpathlineto{\pgfqpoint{0.526157in}{1.753479in}}%
\pgfpathlineto{\pgfqpoint{0.517960in}{1.757179in}}%
\pgfpathlineto{\pgfqpoint{0.505323in}{1.767019in}}%
\pgfpathlineto{\pgfqpoint{0.509122in}{1.786306in}}%
\pgfpathlineto{\pgfqpoint{0.507790in}{1.796155in}}%
\pgfpathlineto{\pgfqpoint{0.497775in}{1.806612in}}%
\pgfpathlineto{\pgfqpoint{0.491359in}{1.805963in}}%
\pgfpathlineto{\pgfqpoint{0.486757in}{1.816540in}}%
\pgfpathlineto{\pgfqpoint{0.475856in}{1.820815in}}%
\pgfpathlineto{\pgfqpoint{0.467931in}{1.820270in}}%
\pgfpathlineto{\pgfqpoint{0.465368in}{1.831133in}}%
\pgfpathlineto{\pgfqpoint{0.473262in}{1.829977in}}%
\pgfpathlineto{\pgfqpoint{0.472635in}{1.845075in}}%
\pgfpathlineto{\pgfqpoint{0.469194in}{1.854040in}}%
\pgfpathlineto{\pgfqpoint{0.470711in}{1.865514in}}%
\pgfpathlineto{\pgfqpoint{0.474841in}{1.874144in}}%
\pgfpathlineto{\pgfqpoint{0.474608in}{1.890521in}}%
\pgfpathlineto{\pgfqpoint{0.472404in}{1.896463in}}%
\pgfpathlineto{\pgfqpoint{0.476212in}{1.915526in}}%
\pgfpathlineto{\pgfqpoint{0.475112in}{1.927812in}}%
\pgfpathlineto{\pgfqpoint{0.471305in}{1.933703in}}%
\pgfpathlineto{\pgfqpoint{0.471886in}{1.952963in}}%
\pgfpathlineto{\pgfqpoint{0.477424in}{1.967128in}}%
\pgfpathlineto{\pgfqpoint{0.483447in}{1.963661in}}%
\pgfpathlineto{\pgfqpoint{0.503389in}{1.943061in}}%
\pgfpathlineto{\pgfqpoint{0.527555in}{1.931787in}}%
\pgfpathlineto{\pgfqpoint{0.539928in}{1.930453in}}%
\pgfpathlineto{\pgfqpoint{0.551585in}{1.925654in}}%
\pgfpathlineto{\pgfqpoint{0.554894in}{1.909503in}}%
\pgfpathlineto{\pgfqpoint{0.544678in}{1.906443in}}%
\pgfpathlineto{\pgfqpoint{0.537908in}{1.893715in}}%
\pgfpathlineto{\pgfqpoint{0.545857in}{1.894422in}}%
\pgfpathlineto{\pgfqpoint{0.549062in}{1.900135in}}%
\pgfpathlineto{\pgfqpoint{0.560272in}{1.907280in}}%
\pgfpathlineto{\pgfqpoint{0.559693in}{1.896670in}}%
\pgfpathlineto{\pgfqpoint{0.552208in}{1.894959in}}%
\pgfpathlineto{\pgfqpoint{0.553451in}{1.881322in}}%
\pgfpathlineto{\pgfqpoint{0.544337in}{1.867661in}}%
\pgfpathlineto{\pgfqpoint{0.538503in}{1.874157in}}%
\pgfpathlineto{\pgfqpoint{0.521168in}{1.870625in}}%
\pgfpathlineto{\pgfqpoint{0.520256in}{1.862677in}}%
\pgfpathlineto{\pgfqpoint{0.526232in}{1.857836in}}%
\pgfpathlineto{\pgfqpoint{0.535349in}{1.857360in}}%
\pgfpathlineto{\pgfqpoint{0.547955in}{1.867706in}}%
\pgfpathlineto{\pgfqpoint{0.557932in}{1.868565in}}%
\pgfpathlineto{\pgfqpoint{0.558315in}{1.880017in}}%
\pgfpathlineto{\pgfqpoint{0.563457in}{1.896838in}}%
\pgfpathlineto{\pgfqpoint{0.575643in}{1.909348in}}%
\pgfpathlineto{\pgfqpoint{0.571580in}{1.919037in}}%
\pgfpathlineto{\pgfqpoint{0.574355in}{1.929469in}}%
\pgfpathlineto{\pgfqpoint{0.568949in}{1.939866in}}%
\pgfpathlineto{\pgfqpoint{0.576654in}{1.953161in}}%
\pgfpathlineto{\pgfqpoint{0.577808in}{1.961160in}}%
\pgfpathlineto{\pgfqpoint{0.571119in}{1.966417in}}%
\pgfpathlineto{\pgfqpoint{0.572232in}{1.979864in}}%
\pgfpathlineto{\pgfqpoint{0.652382in}{1.955513in}}%
\pgfpathlineto{\pgfqpoint{0.737492in}{1.931646in}}%
\pgfpathlineto{\pgfqpoint{0.814503in}{1.911787in}}%
\pgfusepath{stroke}%
\end{pgfscope}%
\begin{pgfscope}%
\pgfpathrectangle{\pgfqpoint{0.100000in}{0.100000in}}{\pgfqpoint{3.007045in}{1.925000in}}%
\pgfusepath{clip}%
\pgfsetbuttcap%
\pgfsetroundjoin%
\pgfsetlinewidth{0.050187pt}%
\definecolor{currentstroke}{rgb}{1.000000,1.000000,1.000000}%
\pgfsetstrokecolor{currentstroke}%
\pgfsetdash{}{0pt}%
\pgfpathmoveto{\pgfqpoint{0.559048in}{1.933344in}}%
\pgfpathlineto{\pgfqpoint{0.562994in}{1.918128in}}%
\pgfpathlineto{\pgfqpoint{0.569183in}{1.913074in}}%
\pgfpathlineto{\pgfqpoint{0.564245in}{1.906641in}}%
\pgfpathlineto{\pgfqpoint{0.559432in}{1.916068in}}%
\pgfpathlineto{\pgfqpoint{0.559048in}{1.933344in}}%
\pgfusepath{stroke}%
\end{pgfscope}%
\begin{pgfscope}%
\pgfpathrectangle{\pgfqpoint{0.100000in}{0.100000in}}{\pgfqpoint{3.007045in}{1.925000in}}%
\pgfusepath{clip}%
\pgfsetbuttcap%
\pgfsetroundjoin%
\pgfsetlinewidth{0.050187pt}%
\definecolor{currentstroke}{rgb}{1.000000,1.000000,1.000000}%
\pgfsetstrokecolor{currentstroke}%
\pgfsetdash{}{0pt}%
\pgfpathmoveto{\pgfqpoint{0.855834in}{1.901800in}}%
\pgfpathlineto{\pgfqpoint{0.941509in}{1.882574in}}%
\pgfpathlineto{\pgfqpoint{1.022196in}{1.866264in}}%
\pgfpathlineto{\pgfqpoint{1.084275in}{1.854890in}}%
\pgfpathlineto{\pgfqpoint{1.138397in}{1.845797in}}%
\pgfpathlineto{\pgfqpoint{1.192639in}{1.837451in}}%
\pgfpathlineto{\pgfqpoint{1.238831in}{1.830944in}}%
\pgfpathlineto{\pgfqpoint{1.285096in}{1.824977in}}%
\pgfpathlineto{\pgfqpoint{1.331427in}{1.819551in}}%
\pgfpathlineto{\pgfqpoint{1.375088in}{1.814940in}}%
\pgfpathlineto{\pgfqpoint{1.369083in}{1.748399in}}%
\pgfpathlineto{\pgfqpoint{1.360146in}{1.658478in}}%
\pgfpathlineto{\pgfqpoint{1.355458in}{1.612230in}}%
\pgfpathlineto{\pgfqpoint{1.348717in}{1.549903in}}%
\pgfpathlineto{\pgfqpoint{1.300990in}{1.555114in}}%
\pgfpathlineto{\pgfqpoint{1.246348in}{1.561317in}}%
\pgfpathlineto{\pgfqpoint{1.170478in}{1.571700in}}%
\pgfpathlineto{\pgfqpoint{1.136592in}{1.576534in}}%
\pgfpathlineto{\pgfqpoint{1.053106in}{1.589639in}}%
\pgfpathlineto{\pgfqpoint{1.024367in}{1.594897in}}%
\pgfpathlineto{\pgfqpoint{1.018377in}{1.560840in}}%
\pgfpathlineto{\pgfqpoint{1.015103in}{1.563270in}}%
\pgfpathlineto{\pgfqpoint{1.008949in}{1.579653in}}%
\pgfpathlineto{\pgfqpoint{1.001038in}{1.578644in}}%
\pgfpathlineto{\pgfqpoint{0.995375in}{1.570362in}}%
\pgfpathlineto{\pgfqpoint{0.983625in}{1.569254in}}%
\pgfpathlineto{\pgfqpoint{0.981264in}{1.573217in}}%
\pgfpathlineto{\pgfqpoint{0.970258in}{1.572737in}}%
\pgfpathlineto{\pgfqpoint{0.964724in}{1.576574in}}%
\pgfpathlineto{\pgfqpoint{0.956987in}{1.570638in}}%
\pgfpathlineto{\pgfqpoint{0.938194in}{1.575980in}}%
\pgfpathlineto{\pgfqpoint{0.929993in}{1.573111in}}%
\pgfpathlineto{\pgfqpoint{0.925581in}{1.585841in}}%
\pgfpathlineto{\pgfqpoint{0.925374in}{1.597953in}}%
\pgfpathlineto{\pgfqpoint{0.912420in}{1.606664in}}%
\pgfpathlineto{\pgfqpoint{0.914532in}{1.619218in}}%
\pgfpathlineto{\pgfqpoint{0.905308in}{1.639956in}}%
\pgfpathlineto{\pgfqpoint{0.905930in}{1.657584in}}%
\pgfpathlineto{\pgfqpoint{0.898002in}{1.666211in}}%
\pgfpathlineto{\pgfqpoint{0.890564in}{1.658578in}}%
\pgfpathlineto{\pgfqpoint{0.880439in}{1.654450in}}%
\pgfpathlineto{\pgfqpoint{0.872732in}{1.662959in}}%
\pgfpathlineto{\pgfqpoint{0.873985in}{1.676547in}}%
\pgfpathlineto{\pgfqpoint{0.883228in}{1.681901in}}%
\pgfpathlineto{\pgfqpoint{0.880771in}{1.690510in}}%
\pgfpathlineto{\pgfqpoint{0.896838in}{1.732179in}}%
\pgfpathlineto{\pgfqpoint{0.884169in}{1.733223in}}%
\pgfpathlineto{\pgfqpoint{0.882867in}{1.740700in}}%
\pgfpathlineto{\pgfqpoint{0.873638in}{1.747149in}}%
\pgfpathlineto{\pgfqpoint{0.874228in}{1.754357in}}%
\pgfpathlineto{\pgfqpoint{0.869360in}{1.759957in}}%
\pgfpathlineto{\pgfqpoint{0.861436in}{1.780289in}}%
\pgfpathlineto{\pgfqpoint{0.854194in}{1.784436in}}%
\pgfpathlineto{\pgfqpoint{0.849009in}{1.801984in}}%
\pgfpathlineto{\pgfqpoint{0.850232in}{1.813647in}}%
\pgfpathlineto{\pgfqpoint{0.840559in}{1.835130in}}%
\pgfpathlineto{\pgfqpoint{0.855834in}{1.901800in}}%
\pgfusepath{stroke}%
\end{pgfscope}%
\begin{pgfscope}%
\pgfpathrectangle{\pgfqpoint{0.100000in}{0.100000in}}{\pgfqpoint{3.007045in}{1.925000in}}%
\pgfusepath{clip}%
\pgfsetbuttcap%
\pgfsetroundjoin%
\pgfsetlinewidth{0.050187pt}%
\definecolor{currentstroke}{rgb}{1.000000,1.000000,1.000000}%
\pgfsetstrokecolor{currentstroke}%
\pgfsetdash{}{0pt}%
\pgfpathmoveto{\pgfqpoint{2.924097in}{1.558905in}}%
\pgfpathlineto{\pgfqpoint{2.922382in}{1.566196in}}%
\pgfpathlineto{\pgfqpoint{2.912852in}{1.572583in}}%
\pgfpathlineto{\pgfqpoint{2.887667in}{1.654962in}}%
\pgfpathlineto{\pgfqpoint{2.874093in}{1.694717in}}%
\pgfpathlineto{\pgfqpoint{2.885006in}{1.706170in}}%
\pgfpathlineto{\pgfqpoint{2.895614in}{1.734199in}}%
\pgfpathlineto{\pgfqpoint{2.900905in}{1.743156in}}%
\pgfpathlineto{\pgfqpoint{2.897157in}{1.746918in}}%
\pgfpathlineto{\pgfqpoint{2.895902in}{1.771799in}}%
\pgfpathlineto{\pgfqpoint{2.900635in}{1.779426in}}%
\pgfpathlineto{\pgfqpoint{2.898560in}{1.797155in}}%
\pgfpathlineto{\pgfqpoint{2.917451in}{1.855321in}}%
\pgfpathlineto{\pgfqpoint{2.925963in}{1.855639in}}%
\pgfpathlineto{\pgfqpoint{2.929497in}{1.845249in}}%
\pgfpathlineto{\pgfqpoint{2.937061in}{1.842266in}}%
\pgfpathlineto{\pgfqpoint{2.951351in}{1.854487in}}%
\pgfpathlineto{\pgfqpoint{2.962555in}{1.861705in}}%
\pgfpathlineto{\pgfqpoint{2.987243in}{1.848994in}}%
\pgfpathlineto{\pgfqpoint{3.009547in}{1.778417in}}%
\pgfpathlineto{\pgfqpoint{3.013786in}{1.761053in}}%
\pgfpathlineto{\pgfqpoint{3.023556in}{1.759017in}}%
\pgfpathlineto{\pgfqpoint{3.036806in}{1.747224in}}%
\pgfpathlineto{\pgfqpoint{3.036056in}{1.740353in}}%
\pgfpathlineto{\pgfqpoint{3.045081in}{1.732208in}}%
\pgfpathlineto{\pgfqpoint{3.052413in}{1.736881in}}%
\pgfpathlineto{\pgfqpoint{3.067756in}{1.718987in}}%
\pgfpathlineto{\pgfqpoint{3.060899in}{1.704668in}}%
\pgfpathlineto{\pgfqpoint{3.051704in}{1.704384in}}%
\pgfpathlineto{\pgfqpoint{3.044326in}{1.691601in}}%
\pgfpathlineto{\pgfqpoint{3.035390in}{1.689788in}}%
\pgfpathlineto{\pgfqpoint{3.028829in}{1.682982in}}%
\pgfpathlineto{\pgfqpoint{3.009234in}{1.675682in}}%
\pgfpathlineto{\pgfqpoint{2.997374in}{1.663918in}}%
\pgfpathlineto{\pgfqpoint{2.991350in}{1.672362in}}%
\pgfpathlineto{\pgfqpoint{2.985875in}{1.666310in}}%
\pgfpathlineto{\pgfqpoint{2.987374in}{1.641845in}}%
\pgfpathlineto{\pgfqpoint{2.983040in}{1.632155in}}%
\pgfpathlineto{\pgfqpoint{2.973585in}{1.634802in}}%
\pgfpathlineto{\pgfqpoint{2.972059in}{1.624866in}}%
\pgfpathlineto{\pgfqpoint{2.967973in}{1.620781in}}%
\pgfpathlineto{\pgfqpoint{2.959797in}{1.621777in}}%
\pgfpathlineto{\pgfqpoint{2.960312in}{1.612488in}}%
\pgfpathlineto{\pgfqpoint{2.947919in}{1.615126in}}%
\pgfpathlineto{\pgfqpoint{2.941151in}{1.602256in}}%
\pgfpathlineto{\pgfqpoint{2.943707in}{1.595522in}}%
\pgfpathlineto{\pgfqpoint{2.939681in}{1.584322in}}%
\pgfpathlineto{\pgfqpoint{2.933343in}{1.576084in}}%
\pgfpathlineto{\pgfqpoint{2.931748in}{1.558887in}}%
\pgfpathlineto{\pgfqpoint{2.924097in}{1.558905in}}%
\pgfusepath{stroke}%
\end{pgfscope}%
\begin{pgfscope}%
\pgfpathrectangle{\pgfqpoint{0.100000in}{0.100000in}}{\pgfqpoint{3.007045in}{1.925000in}}%
\pgfusepath{clip}%
\pgfsetbuttcap%
\pgfsetroundjoin%
\pgfsetlinewidth{0.050187pt}%
\definecolor{currentstroke}{rgb}{1.000000,1.000000,1.000000}%
\pgfsetstrokecolor{currentstroke}%
\pgfsetdash{}{0pt}%
\pgfpathmoveto{\pgfqpoint{3.012754in}{1.670658in}}%
\pgfpathlineto{\pgfqpoint{3.018349in}{1.676519in}}%
\pgfpathlineto{\pgfqpoint{3.023676in}{1.671010in}}%
\pgfpathlineto{\pgfqpoint{3.014108in}{1.663712in}}%
\pgfpathlineto{\pgfqpoint{3.012754in}{1.670658in}}%
\pgfusepath{stroke}%
\end{pgfscope}%
\begin{pgfscope}%
\pgfpathrectangle{\pgfqpoint{0.100000in}{0.100000in}}{\pgfqpoint{3.007045in}{1.925000in}}%
\pgfusepath{clip}%
\pgfsetbuttcap%
\pgfsetroundjoin%
\pgfsetlinewidth{0.050187pt}%
\definecolor{currentstroke}{rgb}{1.000000,1.000000,1.000000}%
\pgfsetstrokecolor{currentstroke}%
\pgfsetdash{}{0pt}%
\pgfpathmoveto{\pgfqpoint{1.355458in}{1.612230in}}%
\pgfpathlineto{\pgfqpoint{1.360146in}{1.658478in}}%
\pgfpathlineto{\pgfqpoint{1.369083in}{1.748399in}}%
\pgfpathlineto{\pgfqpoint{1.375088in}{1.814940in}}%
\pgfpathlineto{\pgfqpoint{1.424263in}{1.810326in}}%
\pgfpathlineto{\pgfqpoint{1.487174in}{1.805318in}}%
\pgfpathlineto{\pgfqpoint{1.544677in}{1.801614in}}%
\pgfpathlineto{\pgfqpoint{1.596745in}{1.798979in}}%
\pgfpathlineto{\pgfqpoint{1.674447in}{1.796314in}}%
\pgfpathlineto{\pgfqpoint{1.679750in}{1.774414in}}%
\pgfpathlineto{\pgfqpoint{1.677852in}{1.763948in}}%
\pgfpathlineto{\pgfqpoint{1.677227in}{1.742437in}}%
\pgfpathlineto{\pgfqpoint{1.680835in}{1.726326in}}%
\pgfpathlineto{\pgfqpoint{1.689100in}{1.702558in}}%
\pgfpathlineto{\pgfqpoint{1.689081in}{1.672635in}}%
\pgfpathlineto{\pgfqpoint{1.690611in}{1.637873in}}%
\pgfpathlineto{\pgfqpoint{1.692711in}{1.628504in}}%
\pgfpathlineto{\pgfqpoint{1.698854in}{1.618219in}}%
\pgfpathlineto{\pgfqpoint{1.700880in}{1.602193in}}%
\pgfpathlineto{\pgfqpoint{1.700005in}{1.591493in}}%
\pgfpathlineto{\pgfqpoint{1.634865in}{1.592905in}}%
\pgfpathlineto{\pgfqpoint{1.587476in}{1.595307in}}%
\pgfpathlineto{\pgfqpoint{1.518008in}{1.599005in}}%
\pgfpathlineto{\pgfqpoint{1.449501in}{1.603871in}}%
\pgfpathlineto{\pgfqpoint{1.403872in}{1.607552in}}%
\pgfpathlineto{\pgfqpoint{1.355458in}{1.612230in}}%
\pgfusepath{stroke}%
\end{pgfscope}%
\begin{pgfscope}%
\pgfpathrectangle{\pgfqpoint{0.100000in}{0.100000in}}{\pgfqpoint{3.007045in}{1.925000in}}%
\pgfusepath{clip}%
\pgfsetbuttcap%
\pgfsetroundjoin%
\pgfsetlinewidth{0.050187pt}%
\definecolor{currentstroke}{rgb}{1.000000,1.000000,1.000000}%
\pgfsetstrokecolor{currentstroke}%
\pgfsetdash{}{0pt}%
\pgfpathmoveto{\pgfqpoint{1.335768in}{1.418681in}}%
\pgfpathlineto{\pgfqpoint{1.343289in}{1.496206in}}%
\pgfpathlineto{\pgfqpoint{1.348717in}{1.549903in}}%
\pgfpathlineto{\pgfqpoint{1.355458in}{1.612230in}}%
\pgfpathlineto{\pgfqpoint{1.403872in}{1.607552in}}%
\pgfpathlineto{\pgfqpoint{1.449501in}{1.603871in}}%
\pgfpathlineto{\pgfqpoint{1.518008in}{1.599005in}}%
\pgfpathlineto{\pgfqpoint{1.587476in}{1.595307in}}%
\pgfpathlineto{\pgfqpoint{1.634865in}{1.592905in}}%
\pgfpathlineto{\pgfqpoint{1.700005in}{1.591493in}}%
\pgfpathlineto{\pgfqpoint{1.695595in}{1.578623in}}%
\pgfpathlineto{\pgfqpoint{1.686792in}{1.568520in}}%
\pgfpathlineto{\pgfqpoint{1.693533in}{1.556888in}}%
\pgfpathlineto{\pgfqpoint{1.700966in}{1.554402in}}%
\pgfpathlineto{\pgfqpoint{1.704494in}{1.547720in}}%
\pgfpathlineto{\pgfqpoint{1.703684in}{1.498934in}}%
\pgfpathlineto{\pgfqpoint{1.702328in}{1.430275in}}%
\pgfpathlineto{\pgfqpoint{1.697315in}{1.412242in}}%
\pgfpathlineto{\pgfqpoint{1.701799in}{1.402266in}}%
\pgfpathlineto{\pgfqpoint{1.692796in}{1.380831in}}%
\pgfpathlineto{\pgfqpoint{1.702281in}{1.363552in}}%
\pgfpathlineto{\pgfqpoint{1.694222in}{1.364874in}}%
\pgfpathlineto{\pgfqpoint{1.688724in}{1.375625in}}%
\pgfpathlineto{\pgfqpoint{1.669102in}{1.382994in}}%
\pgfpathlineto{\pgfqpoint{1.656727in}{1.389477in}}%
\pgfpathlineto{\pgfqpoint{1.635962in}{1.390016in}}%
\pgfpathlineto{\pgfqpoint{1.628754in}{1.384110in}}%
\pgfpathlineto{\pgfqpoint{1.605250in}{1.395738in}}%
\pgfpathlineto{\pgfqpoint{1.603447in}{1.399416in}}%
\pgfpathlineto{\pgfqpoint{1.521419in}{1.403298in}}%
\pgfpathlineto{\pgfqpoint{1.471583in}{1.406155in}}%
\pgfpathlineto{\pgfqpoint{1.430419in}{1.409379in}}%
\pgfpathlineto{\pgfqpoint{1.362400in}{1.415805in}}%
\pgfpathlineto{\pgfqpoint{1.335768in}{1.418681in}}%
\pgfusepath{stroke}%
\end{pgfscope}%
\begin{pgfscope}%
\pgfpathrectangle{\pgfqpoint{0.100000in}{0.100000in}}{\pgfqpoint{3.007045in}{1.925000in}}%
\pgfusepath{clip}%
\pgfsetbuttcap%
\pgfsetroundjoin%
\pgfsetlinewidth{0.050187pt}%
\definecolor{currentstroke}{rgb}{1.000000,1.000000,1.000000}%
\pgfsetstrokecolor{currentstroke}%
\pgfsetdash{}{0pt}%
\pgfpathmoveto{\pgfqpoint{1.322867in}{1.287410in}}%
\pgfpathlineto{\pgfqpoint{1.279137in}{1.291403in}}%
\pgfpathlineto{\pgfqpoint{1.183815in}{1.303168in}}%
\pgfpathlineto{\pgfqpoint{1.131961in}{1.310611in}}%
\pgfpathlineto{\pgfqpoint{1.076345in}{1.318594in}}%
\pgfpathlineto{\pgfqpoint{1.029497in}{1.326091in}}%
\pgfpathlineto{\pgfqpoint{0.978076in}{1.334895in}}%
\pgfpathlineto{\pgfqpoint{0.989767in}{1.399751in}}%
\pgfpathlineto{\pgfqpoint{1.001610in}{1.466254in}}%
\pgfpathlineto{\pgfqpoint{1.018377in}{1.560840in}}%
\pgfpathlineto{\pgfqpoint{1.024367in}{1.594897in}}%
\pgfpathlineto{\pgfqpoint{1.053106in}{1.589639in}}%
\pgfpathlineto{\pgfqpoint{1.136592in}{1.576534in}}%
\pgfpathlineto{\pgfqpoint{1.170478in}{1.571700in}}%
\pgfpathlineto{\pgfqpoint{1.246348in}{1.561317in}}%
\pgfpathlineto{\pgfqpoint{1.300990in}{1.555114in}}%
\pgfpathlineto{\pgfqpoint{1.348717in}{1.549903in}}%
\pgfpathlineto{\pgfqpoint{1.343289in}{1.496206in}}%
\pgfpathlineto{\pgfqpoint{1.335768in}{1.418681in}}%
\pgfpathlineto{\pgfqpoint{1.329313in}{1.352796in}}%
\pgfpathlineto{\pgfqpoint{1.322867in}{1.287410in}}%
\pgfusepath{stroke}%
\end{pgfscope}%
\begin{pgfscope}%
\pgfpathrectangle{\pgfqpoint{0.100000in}{0.100000in}}{\pgfqpoint{3.007045in}{1.925000in}}%
\pgfusepath{clip}%
\pgfsetbuttcap%
\pgfsetroundjoin%
\pgfsetlinewidth{0.050187pt}%
\definecolor{currentstroke}{rgb}{1.000000,1.000000,1.000000}%
\pgfsetstrokecolor{currentstroke}%
\pgfsetdash{}{0pt}%
\pgfpathmoveto{\pgfqpoint{2.122408in}{1.376872in}}%
\pgfpathlineto{\pgfqpoint{2.066992in}{1.372935in}}%
\pgfpathlineto{\pgfqpoint{1.984391in}{1.369415in}}%
\pgfpathlineto{\pgfqpoint{1.981252in}{1.377759in}}%
\pgfpathlineto{\pgfqpoint{1.962935in}{1.383996in}}%
\pgfpathlineto{\pgfqpoint{1.958894in}{1.395780in}}%
\pgfpathlineto{\pgfqpoint{1.957202in}{1.410359in}}%
\pgfpathlineto{\pgfqpoint{1.961330in}{1.417827in}}%
\pgfpathlineto{\pgfqpoint{1.954811in}{1.424995in}}%
\pgfpathlineto{\pgfqpoint{1.953238in}{1.433544in}}%
\pgfpathlineto{\pgfqpoint{1.951137in}{1.452456in}}%
\pgfpathlineto{\pgfqpoint{1.944889in}{1.462729in}}%
\pgfpathlineto{\pgfqpoint{1.933793in}{1.468495in}}%
\pgfpathlineto{\pgfqpoint{1.921718in}{1.478009in}}%
\pgfpathlineto{\pgfqpoint{1.915477in}{1.489266in}}%
\pgfpathlineto{\pgfqpoint{1.904277in}{1.493800in}}%
\pgfpathlineto{\pgfqpoint{1.897693in}{1.501178in}}%
\pgfpathlineto{\pgfqpoint{1.889718in}{1.502430in}}%
\pgfpathlineto{\pgfqpoint{1.875478in}{1.513385in}}%
\pgfpathlineto{\pgfqpoint{1.877791in}{1.525991in}}%
\pgfpathlineto{\pgfqpoint{1.877345in}{1.549965in}}%
\pgfpathlineto{\pgfqpoint{1.881737in}{1.556564in}}%
\pgfpathlineto{\pgfqpoint{1.877791in}{1.566528in}}%
\pgfpathlineto{\pgfqpoint{1.870840in}{1.568455in}}%
\pgfpathlineto{\pgfqpoint{1.871415in}{1.577207in}}%
\pgfpathlineto{\pgfqpoint{1.880038in}{1.591032in}}%
\pgfpathlineto{\pgfqpoint{1.897116in}{1.601964in}}%
\pgfpathlineto{\pgfqpoint{1.896052in}{1.640844in}}%
\pgfpathlineto{\pgfqpoint{1.904597in}{1.646684in}}%
\pgfpathlineto{\pgfqpoint{1.912684in}{1.642793in}}%
\pgfpathlineto{\pgfqpoint{1.929157in}{1.648489in}}%
\pgfpathlineto{\pgfqpoint{1.960149in}{1.662803in}}%
\pgfpathlineto{\pgfqpoint{1.964187in}{1.658371in}}%
\pgfpathlineto{\pgfqpoint{1.958317in}{1.638296in}}%
\pgfpathlineto{\pgfqpoint{1.967050in}{1.642702in}}%
\pgfpathlineto{\pgfqpoint{1.982007in}{1.638302in}}%
\pgfpathlineto{\pgfqpoint{1.991199in}{1.634631in}}%
\pgfpathlineto{\pgfqpoint{1.996352in}{1.623871in}}%
\pgfpathlineto{\pgfqpoint{2.043506in}{1.613716in}}%
\pgfpathlineto{\pgfqpoint{2.057600in}{1.606750in}}%
\pgfpathlineto{\pgfqpoint{2.071931in}{1.606832in}}%
\pgfpathlineto{\pgfqpoint{2.086656in}{1.603958in}}%
\pgfpathlineto{\pgfqpoint{2.096212in}{1.594142in}}%
\pgfpathlineto{\pgfqpoint{2.105344in}{1.589291in}}%
\pgfpathlineto{\pgfqpoint{2.107021in}{1.575301in}}%
\pgfpathlineto{\pgfqpoint{2.104324in}{1.566517in}}%
\pgfpathlineto{\pgfqpoint{2.114501in}{1.565867in}}%
\pgfpathlineto{\pgfqpoint{2.111072in}{1.555679in}}%
\pgfpathlineto{\pgfqpoint{2.114334in}{1.552060in}}%
\pgfpathlineto{\pgfqpoint{2.117577in}{1.542443in}}%
\pgfpathlineto{\pgfqpoint{2.107659in}{1.537351in}}%
\pgfpathlineto{\pgfqpoint{2.101900in}{1.523164in}}%
\pgfpathlineto{\pgfqpoint{2.100093in}{1.513132in}}%
\pgfpathlineto{\pgfqpoint{2.105618in}{1.511415in}}%
\pgfpathlineto{\pgfqpoint{2.112691in}{1.518938in}}%
\pgfpathlineto{\pgfqpoint{2.118705in}{1.531974in}}%
\pgfpathlineto{\pgfqpoint{2.126842in}{1.536502in}}%
\pgfpathlineto{\pgfqpoint{2.132957in}{1.530613in}}%
\pgfpathlineto{\pgfqpoint{2.126938in}{1.512787in}}%
\pgfpathlineto{\pgfqpoint{2.125050in}{1.498949in}}%
\pgfpathlineto{\pgfqpoint{2.126828in}{1.489001in}}%
\pgfpathlineto{\pgfqpoint{2.121239in}{1.483374in}}%
\pgfpathlineto{\pgfqpoint{2.118447in}{1.469622in}}%
\pgfpathlineto{\pgfqpoint{2.120728in}{1.455170in}}%
\pgfpathlineto{\pgfqpoint{2.114132in}{1.433796in}}%
\pgfpathlineto{\pgfqpoint{2.114271in}{1.423115in}}%
\pgfpathlineto{\pgfqpoint{2.119483in}{1.399971in}}%
\pgfpathlineto{\pgfqpoint{2.122861in}{1.396001in}}%
\pgfpathlineto{\pgfqpoint{2.122408in}{1.376872in}}%
\pgfusepath{stroke}%
\end{pgfscope}%
\begin{pgfscope}%
\pgfpathrectangle{\pgfqpoint{0.100000in}{0.100000in}}{\pgfqpoint{3.007045in}{1.925000in}}%
\pgfusepath{clip}%
\pgfsetbuttcap%
\pgfsetroundjoin%
\pgfsetlinewidth{0.050187pt}%
\definecolor{currentstroke}{rgb}{1.000000,1.000000,1.000000}%
\pgfsetstrokecolor{currentstroke}%
\pgfsetdash{}{0pt}%
\pgfpathmoveto{\pgfqpoint{2.143186in}{1.564449in}}%
\pgfpathlineto{\pgfqpoint{2.144423in}{1.555218in}}%
\pgfpathlineto{\pgfqpoint{2.133091in}{1.530893in}}%
\pgfpathlineto{\pgfqpoint{2.128055in}{1.537939in}}%
\pgfpathlineto{\pgfqpoint{2.143186in}{1.564449in}}%
\pgfusepath{stroke}%
\end{pgfscope}%
\begin{pgfscope}%
\pgfpathrectangle{\pgfqpoint{0.100000in}{0.100000in}}{\pgfqpoint{3.007045in}{1.925000in}}%
\pgfusepath{clip}%
\pgfsetbuttcap%
\pgfsetroundjoin%
\pgfsetlinewidth{0.050187pt}%
\definecolor{currentstroke}{rgb}{1.000000,1.000000,1.000000}%
\pgfsetstrokecolor{currentstroke}%
\pgfsetdash{}{0pt}%
\pgfpathmoveto{\pgfqpoint{0.771499in}{1.716207in}}%
\pgfpathlineto{\pgfqpoint{0.773739in}{1.727298in}}%
\pgfpathlineto{\pgfqpoint{0.772049in}{1.739338in}}%
\pgfpathlineto{\pgfqpoint{0.774637in}{1.752465in}}%
\pgfpathlineto{\pgfqpoint{0.787821in}{1.805724in}}%
\pgfpathlineto{\pgfqpoint{0.798909in}{1.849997in}}%
\pgfpathlineto{\pgfqpoint{0.814503in}{1.911787in}}%
\pgfpathlineto{\pgfqpoint{0.855834in}{1.901800in}}%
\pgfpathlineto{\pgfqpoint{0.840559in}{1.835130in}}%
\pgfpathlineto{\pgfqpoint{0.850232in}{1.813647in}}%
\pgfpathlineto{\pgfqpoint{0.849009in}{1.801984in}}%
\pgfpathlineto{\pgfqpoint{0.854194in}{1.784436in}}%
\pgfpathlineto{\pgfqpoint{0.861436in}{1.780289in}}%
\pgfpathlineto{\pgfqpoint{0.869360in}{1.759957in}}%
\pgfpathlineto{\pgfqpoint{0.874228in}{1.754357in}}%
\pgfpathlineto{\pgfqpoint{0.873638in}{1.747149in}}%
\pgfpathlineto{\pgfqpoint{0.882867in}{1.740700in}}%
\pgfpathlineto{\pgfqpoint{0.884169in}{1.733223in}}%
\pgfpathlineto{\pgfqpoint{0.896838in}{1.732179in}}%
\pgfpathlineto{\pgfqpoint{0.880771in}{1.690510in}}%
\pgfpathlineto{\pgfqpoint{0.883228in}{1.681901in}}%
\pgfpathlineto{\pgfqpoint{0.873985in}{1.676547in}}%
\pgfpathlineto{\pgfqpoint{0.872732in}{1.662959in}}%
\pgfpathlineto{\pgfqpoint{0.880439in}{1.654450in}}%
\pgfpathlineto{\pgfqpoint{0.890564in}{1.658578in}}%
\pgfpathlineto{\pgfqpoint{0.898002in}{1.666211in}}%
\pgfpathlineto{\pgfqpoint{0.905930in}{1.657584in}}%
\pgfpathlineto{\pgfqpoint{0.905308in}{1.639956in}}%
\pgfpathlineto{\pgfqpoint{0.914532in}{1.619218in}}%
\pgfpathlineto{\pgfqpoint{0.912420in}{1.606664in}}%
\pgfpathlineto{\pgfqpoint{0.925374in}{1.597953in}}%
\pgfpathlineto{\pgfqpoint{0.925581in}{1.585841in}}%
\pgfpathlineto{\pgfqpoint{0.929993in}{1.573111in}}%
\pgfpathlineto{\pgfqpoint{0.938194in}{1.575980in}}%
\pgfpathlineto{\pgfqpoint{0.956987in}{1.570638in}}%
\pgfpathlineto{\pgfqpoint{0.964724in}{1.576574in}}%
\pgfpathlineto{\pgfqpoint{0.970258in}{1.572737in}}%
\pgfpathlineto{\pgfqpoint{0.981264in}{1.573217in}}%
\pgfpathlineto{\pgfqpoint{0.983625in}{1.569254in}}%
\pgfpathlineto{\pgfqpoint{0.995375in}{1.570362in}}%
\pgfpathlineto{\pgfqpoint{1.001038in}{1.578644in}}%
\pgfpathlineto{\pgfqpoint{1.008949in}{1.579653in}}%
\pgfpathlineto{\pgfqpoint{1.015103in}{1.563270in}}%
\pgfpathlineto{\pgfqpoint{1.018377in}{1.560840in}}%
\pgfpathlineto{\pgfqpoint{1.001610in}{1.466254in}}%
\pgfpathlineto{\pgfqpoint{0.989767in}{1.399751in}}%
\pgfpathlineto{\pgfqpoint{0.896364in}{1.417782in}}%
\pgfpathlineto{\pgfqpoint{0.845912in}{1.427818in}}%
\pgfpathlineto{\pgfqpoint{0.798720in}{1.438195in}}%
\pgfpathlineto{\pgfqpoint{0.754611in}{1.448181in}}%
\pgfpathlineto{\pgfqpoint{0.703563in}{1.460492in}}%
\pgfpathlineto{\pgfqpoint{0.729893in}{1.568511in}}%
\pgfpathlineto{\pgfqpoint{0.731313in}{1.576405in}}%
\pgfpathlineto{\pgfqpoint{0.743022in}{1.597090in}}%
\pgfpathlineto{\pgfqpoint{0.730898in}{1.609522in}}%
\pgfpathlineto{\pgfqpoint{0.733416in}{1.621731in}}%
\pgfpathlineto{\pgfqpoint{0.738031in}{1.624787in}}%
\pgfpathlineto{\pgfqpoint{0.746274in}{1.637367in}}%
\pgfpathlineto{\pgfqpoint{0.758289in}{1.646070in}}%
\pgfpathlineto{\pgfqpoint{0.758946in}{1.652510in}}%
\pgfpathlineto{\pgfqpoint{0.766198in}{1.658944in}}%
\pgfpathlineto{\pgfqpoint{0.772219in}{1.670988in}}%
\pgfpathlineto{\pgfqpoint{0.784570in}{1.683729in}}%
\pgfpathlineto{\pgfqpoint{0.783693in}{1.696314in}}%
\pgfpathlineto{\pgfqpoint{0.775249in}{1.703304in}}%
\pgfpathlineto{\pgfqpoint{0.771499in}{1.716207in}}%
\pgfusepath{stroke}%
\end{pgfscope}%
\begin{pgfscope}%
\pgfpathrectangle{\pgfqpoint{0.100000in}{0.100000in}}{\pgfqpoint{3.007045in}{1.925000in}}%
\pgfusepath{clip}%
\pgfsetbuttcap%
\pgfsetroundjoin%
\pgfsetlinewidth{0.050187pt}%
\definecolor{currentstroke}{rgb}{1.000000,1.000000,1.000000}%
\pgfsetstrokecolor{currentstroke}%
\pgfsetdash{}{0pt}%
\pgfpathmoveto{\pgfqpoint{2.816139in}{1.505410in}}%
\pgfpathlineto{\pgfqpoint{2.812258in}{1.517661in}}%
\pgfpathlineto{\pgfqpoint{2.805056in}{1.554825in}}%
\pgfpathlineto{\pgfqpoint{2.795745in}{1.566258in}}%
\pgfpathlineto{\pgfqpoint{2.787577in}{1.586820in}}%
\pgfpathlineto{\pgfqpoint{2.790259in}{1.602064in}}%
\pgfpathlineto{\pgfqpoint{2.788108in}{1.613300in}}%
\pgfpathlineto{\pgfqpoint{2.781130in}{1.624335in}}%
\pgfpathlineto{\pgfqpoint{2.780843in}{1.636492in}}%
\pgfpathlineto{\pgfqpoint{2.776791in}{1.649602in}}%
\pgfpathlineto{\pgfqpoint{2.813053in}{1.658527in}}%
\pgfpathlineto{\pgfqpoint{2.860154in}{1.671241in}}%
\pgfpathlineto{\pgfqpoint{2.862034in}{1.663950in}}%
\pgfpathlineto{\pgfqpoint{2.859035in}{1.652344in}}%
\pgfpathlineto{\pgfqpoint{2.866075in}{1.643057in}}%
\pgfpathlineto{\pgfqpoint{2.862336in}{1.631290in}}%
\pgfpathlineto{\pgfqpoint{2.847477in}{1.616498in}}%
\pgfpathlineto{\pgfqpoint{2.851562in}{1.605386in}}%
\pgfpathlineto{\pgfqpoint{2.848535in}{1.581900in}}%
\pgfpathlineto{\pgfqpoint{2.844297in}{1.568658in}}%
\pgfpathlineto{\pgfqpoint{2.849703in}{1.531000in}}%
\pgfpathlineto{\pgfqpoint{2.848975in}{1.517751in}}%
\pgfpathlineto{\pgfqpoint{2.854206in}{1.513499in}}%
\pgfpathlineto{\pgfqpoint{2.816139in}{1.505410in}}%
\pgfusepath{stroke}%
\end{pgfscope}%
\begin{pgfscope}%
\pgfpathrectangle{\pgfqpoint{0.100000in}{0.100000in}}{\pgfqpoint{3.007045in}{1.925000in}}%
\pgfusepath{clip}%
\pgfsetbuttcap%
\pgfsetroundjoin%
\pgfsetlinewidth{0.050187pt}%
\definecolor{currentstroke}{rgb}{1.000000,1.000000,1.000000}%
\pgfsetstrokecolor{currentstroke}%
\pgfsetdash{}{0pt}%
\pgfpathmoveto{\pgfqpoint{1.702328in}{1.430275in}}%
\pgfpathlineto{\pgfqpoint{1.703684in}{1.498934in}}%
\pgfpathlineto{\pgfqpoint{1.704494in}{1.547720in}}%
\pgfpathlineto{\pgfqpoint{1.700966in}{1.554402in}}%
\pgfpathlineto{\pgfqpoint{1.693533in}{1.556888in}}%
\pgfpathlineto{\pgfqpoint{1.686792in}{1.568520in}}%
\pgfpathlineto{\pgfqpoint{1.695595in}{1.578623in}}%
\pgfpathlineto{\pgfqpoint{1.700005in}{1.591493in}}%
\pgfpathlineto{\pgfqpoint{1.700880in}{1.602193in}}%
\pgfpathlineto{\pgfqpoint{1.698854in}{1.618219in}}%
\pgfpathlineto{\pgfqpoint{1.692711in}{1.628504in}}%
\pgfpathlineto{\pgfqpoint{1.690611in}{1.637873in}}%
\pgfpathlineto{\pgfqpoint{1.689081in}{1.672635in}}%
\pgfpathlineto{\pgfqpoint{1.689100in}{1.702558in}}%
\pgfpathlineto{\pgfqpoint{1.680835in}{1.726326in}}%
\pgfpathlineto{\pgfqpoint{1.677227in}{1.742437in}}%
\pgfpathlineto{\pgfqpoint{1.677852in}{1.763948in}}%
\pgfpathlineto{\pgfqpoint{1.679750in}{1.774414in}}%
\pgfpathlineto{\pgfqpoint{1.674447in}{1.796314in}}%
\pgfpathlineto{\pgfqpoint{1.710550in}{1.795591in}}%
\pgfpathlineto{\pgfqpoint{1.765389in}{1.795120in}}%
\pgfpathlineto{\pgfqpoint{1.765689in}{1.820011in}}%
\pgfpathlineto{\pgfqpoint{1.779648in}{1.817270in}}%
\pgfpathlineto{\pgfqpoint{1.786338in}{1.786920in}}%
\pgfpathlineto{\pgfqpoint{1.791270in}{1.776008in}}%
\pgfpathlineto{\pgfqpoint{1.803535in}{1.775685in}}%
\pgfpathlineto{\pgfqpoint{1.806280in}{1.771983in}}%
\pgfpathlineto{\pgfqpoint{1.823388in}{1.770342in}}%
\pgfpathlineto{\pgfqpoint{1.826264in}{1.762818in}}%
\pgfpathlineto{\pgfqpoint{1.838054in}{1.764509in}}%
\pgfpathlineto{\pgfqpoint{1.847211in}{1.771547in}}%
\pgfpathlineto{\pgfqpoint{1.863003in}{1.771284in}}%
\pgfpathlineto{\pgfqpoint{1.872775in}{1.765624in}}%
\pgfpathlineto{\pgfqpoint{1.873898in}{1.760316in}}%
\pgfpathlineto{\pgfqpoint{1.883196in}{1.759205in}}%
\pgfpathlineto{\pgfqpoint{1.889264in}{1.744727in}}%
\pgfpathlineto{\pgfqpoint{1.893179in}{1.753629in}}%
\pgfpathlineto{\pgfqpoint{1.903859in}{1.754177in}}%
\pgfpathlineto{\pgfqpoint{1.906561in}{1.747244in}}%
\pgfpathlineto{\pgfqpoint{1.918579in}{1.744079in}}%
\pgfpathlineto{\pgfqpoint{1.925206in}{1.734093in}}%
\pgfpathlineto{\pgfqpoint{1.939904in}{1.737194in}}%
\pgfpathlineto{\pgfqpoint{1.956078in}{1.749448in}}%
\pgfpathlineto{\pgfqpoint{1.961975in}{1.738646in}}%
\pgfpathlineto{\pgfqpoint{1.988513in}{1.741602in}}%
\pgfpathlineto{\pgfqpoint{1.999857in}{1.734188in}}%
\pgfpathlineto{\pgfqpoint{2.006460in}{1.736835in}}%
\pgfpathlineto{\pgfqpoint{2.011755in}{1.732661in}}%
\pgfpathlineto{\pgfqpoint{1.996048in}{1.722759in}}%
\pgfpathlineto{\pgfqpoint{1.973627in}{1.713935in}}%
\pgfpathlineto{\pgfqpoint{1.951422in}{1.696299in}}%
\pgfpathlineto{\pgfqpoint{1.932171in}{1.673105in}}%
\pgfpathlineto{\pgfqpoint{1.917597in}{1.659394in}}%
\pgfpathlineto{\pgfqpoint{1.904832in}{1.649969in}}%
\pgfpathlineto{\pgfqpoint{1.896052in}{1.640844in}}%
\pgfpathlineto{\pgfqpoint{1.897116in}{1.601964in}}%
\pgfpathlineto{\pgfqpoint{1.880038in}{1.591032in}}%
\pgfpathlineto{\pgfqpoint{1.871415in}{1.577207in}}%
\pgfpathlineto{\pgfqpoint{1.870840in}{1.568455in}}%
\pgfpathlineto{\pgfqpoint{1.877791in}{1.566528in}}%
\pgfpathlineto{\pgfqpoint{1.881737in}{1.556564in}}%
\pgfpathlineto{\pgfqpoint{1.877345in}{1.549965in}}%
\pgfpathlineto{\pgfqpoint{1.877791in}{1.525991in}}%
\pgfpathlineto{\pgfqpoint{1.875478in}{1.513385in}}%
\pgfpathlineto{\pgfqpoint{1.889718in}{1.502430in}}%
\pgfpathlineto{\pgfqpoint{1.897693in}{1.501178in}}%
\pgfpathlineto{\pgfqpoint{1.904277in}{1.493800in}}%
\pgfpathlineto{\pgfqpoint{1.915477in}{1.489266in}}%
\pgfpathlineto{\pgfqpoint{1.921718in}{1.478009in}}%
\pgfpathlineto{\pgfqpoint{1.933793in}{1.468495in}}%
\pgfpathlineto{\pgfqpoint{1.944889in}{1.462729in}}%
\pgfpathlineto{\pgfqpoint{1.951137in}{1.452456in}}%
\pgfpathlineto{\pgfqpoint{1.953238in}{1.433544in}}%
\pgfpathlineto{\pgfqpoint{1.894352in}{1.431406in}}%
\pgfpathlineto{\pgfqpoint{1.844156in}{1.430356in}}%
\pgfpathlineto{\pgfqpoint{1.798422in}{1.429684in}}%
\pgfpathlineto{\pgfqpoint{1.750041in}{1.429758in}}%
\pgfpathlineto{\pgfqpoint{1.702328in}{1.430275in}}%
\pgfusepath{stroke}%
\end{pgfscope}%
\begin{pgfscope}%
\pgfpathrectangle{\pgfqpoint{0.100000in}{0.100000in}}{\pgfqpoint{3.007045in}{1.925000in}}%
\pgfusepath{clip}%
\pgfsetbuttcap%
\pgfsetroundjoin%
\pgfsetlinewidth{0.050187pt}%
\definecolor{currentstroke}{rgb}{1.000000,1.000000,1.000000}%
\pgfsetstrokecolor{currentstroke}%
\pgfsetdash{}{0pt}%
\pgfpathmoveto{\pgfqpoint{0.365302in}{1.558630in}}%
\pgfpathlineto{\pgfqpoint{0.360639in}{1.567212in}}%
\pgfpathlineto{\pgfqpoint{0.363617in}{1.589239in}}%
\pgfpathlineto{\pgfqpoint{0.369364in}{1.600750in}}%
\pgfpathlineto{\pgfqpoint{0.366491in}{1.616323in}}%
\pgfpathlineto{\pgfqpoint{0.372487in}{1.622871in}}%
\pgfpathlineto{\pgfqpoint{0.383425in}{1.640522in}}%
\pgfpathlineto{\pgfqpoint{0.392700in}{1.651178in}}%
\pgfpathlineto{\pgfqpoint{0.406257in}{1.674384in}}%
\pgfpathlineto{\pgfqpoint{0.416655in}{1.699664in}}%
\pgfpathlineto{\pgfqpoint{0.427694in}{1.723373in}}%
\pgfpathlineto{\pgfqpoint{0.429925in}{1.733275in}}%
\pgfpathlineto{\pgfqpoint{0.445153in}{1.761603in}}%
\pgfpathlineto{\pgfqpoint{0.448085in}{1.774045in}}%
\pgfpathlineto{\pgfqpoint{0.454568in}{1.787110in}}%
\pgfpathlineto{\pgfqpoint{0.456886in}{1.803047in}}%
\pgfpathlineto{\pgfqpoint{0.469263in}{1.810819in}}%
\pgfpathlineto{\pgfqpoint{0.483940in}{1.814736in}}%
\pgfpathlineto{\pgfqpoint{0.491359in}{1.805963in}}%
\pgfpathlineto{\pgfqpoint{0.497775in}{1.806612in}}%
\pgfpathlineto{\pgfqpoint{0.507790in}{1.796155in}}%
\pgfpathlineto{\pgfqpoint{0.509122in}{1.786306in}}%
\pgfpathlineto{\pgfqpoint{0.505323in}{1.767019in}}%
\pgfpathlineto{\pgfqpoint{0.517960in}{1.757179in}}%
\pgfpathlineto{\pgfqpoint{0.526157in}{1.753479in}}%
\pgfpathlineto{\pgfqpoint{0.548360in}{1.757255in}}%
\pgfpathlineto{\pgfqpoint{0.561223in}{1.754676in}}%
\pgfpathlineto{\pgfqpoint{0.570064in}{1.750443in}}%
\pgfpathlineto{\pgfqpoint{0.574722in}{1.742409in}}%
\pgfpathlineto{\pgfqpoint{0.604211in}{1.743481in}}%
\pgfpathlineto{\pgfqpoint{0.608900in}{1.738565in}}%
\pgfpathlineto{\pgfqpoint{0.619963in}{1.737504in}}%
\pgfpathlineto{\pgfqpoint{0.631104in}{1.740645in}}%
\pgfpathlineto{\pgfqpoint{0.664042in}{1.739853in}}%
\pgfpathlineto{\pgfqpoint{0.671157in}{1.737433in}}%
\pgfpathlineto{\pgfqpoint{0.679470in}{1.740222in}}%
\pgfpathlineto{\pgfqpoint{0.771499in}{1.716207in}}%
\pgfpathlineto{\pgfqpoint{0.775249in}{1.703304in}}%
\pgfpathlineto{\pgfqpoint{0.783693in}{1.696314in}}%
\pgfpathlineto{\pgfqpoint{0.784570in}{1.683729in}}%
\pgfpathlineto{\pgfqpoint{0.772219in}{1.670988in}}%
\pgfpathlineto{\pgfqpoint{0.766198in}{1.658944in}}%
\pgfpathlineto{\pgfqpoint{0.758946in}{1.652510in}}%
\pgfpathlineto{\pgfqpoint{0.758289in}{1.646070in}}%
\pgfpathlineto{\pgfqpoint{0.746274in}{1.637367in}}%
\pgfpathlineto{\pgfqpoint{0.738031in}{1.624787in}}%
\pgfpathlineto{\pgfqpoint{0.733416in}{1.621731in}}%
\pgfpathlineto{\pgfqpoint{0.730898in}{1.609522in}}%
\pgfpathlineto{\pgfqpoint{0.743022in}{1.597090in}}%
\pgfpathlineto{\pgfqpoint{0.731313in}{1.576405in}}%
\pgfpathlineto{\pgfqpoint{0.729893in}{1.568511in}}%
\pgfpathlineto{\pgfqpoint{0.703563in}{1.460492in}}%
\pgfpathlineto{\pgfqpoint{0.648178in}{1.474684in}}%
\pgfpathlineto{\pgfqpoint{0.594748in}{1.488461in}}%
\pgfpathlineto{\pgfqpoint{0.562516in}{1.497430in}}%
\pgfpathlineto{\pgfqpoint{0.521101in}{1.509228in}}%
\pgfpathlineto{\pgfqpoint{0.455041in}{1.530020in}}%
\pgfpathlineto{\pgfqpoint{0.383230in}{1.552378in}}%
\pgfpathlineto{\pgfqpoint{0.365302in}{1.558630in}}%
\pgfusepath{stroke}%
\end{pgfscope}%
\begin{pgfscope}%
\pgfpathrectangle{\pgfqpoint{0.100000in}{0.100000in}}{\pgfqpoint{3.007045in}{1.925000in}}%
\pgfusepath{clip}%
\pgfsetbuttcap%
\pgfsetroundjoin%
\pgfsetlinewidth{0.050187pt}%
\definecolor{currentstroke}{rgb}{1.000000,1.000000,1.000000}%
\pgfsetstrokecolor{currentstroke}%
\pgfsetdash{}{0pt}%
\pgfpathmoveto{\pgfqpoint{2.854206in}{1.513499in}}%
\pgfpathlineto{\pgfqpoint{2.848975in}{1.517751in}}%
\pgfpathlineto{\pgfqpoint{2.849703in}{1.531000in}}%
\pgfpathlineto{\pgfqpoint{2.844297in}{1.568658in}}%
\pgfpathlineto{\pgfqpoint{2.848535in}{1.581900in}}%
\pgfpathlineto{\pgfqpoint{2.851562in}{1.605386in}}%
\pgfpathlineto{\pgfqpoint{2.847477in}{1.616498in}}%
\pgfpathlineto{\pgfqpoint{2.862336in}{1.631290in}}%
\pgfpathlineto{\pgfqpoint{2.866075in}{1.643057in}}%
\pgfpathlineto{\pgfqpoint{2.859035in}{1.652344in}}%
\pgfpathlineto{\pgfqpoint{2.862034in}{1.663950in}}%
\pgfpathlineto{\pgfqpoint{2.860154in}{1.671241in}}%
\pgfpathlineto{\pgfqpoint{2.861768in}{1.686854in}}%
\pgfpathlineto{\pgfqpoint{2.864787in}{1.691674in}}%
\pgfpathlineto{\pgfqpoint{2.874093in}{1.694717in}}%
\pgfpathlineto{\pgfqpoint{2.887667in}{1.654962in}}%
\pgfpathlineto{\pgfqpoint{2.912852in}{1.572583in}}%
\pgfpathlineto{\pgfqpoint{2.922382in}{1.566196in}}%
\pgfpathlineto{\pgfqpoint{2.924097in}{1.558905in}}%
\pgfpathlineto{\pgfqpoint{2.929126in}{1.555958in}}%
\pgfpathlineto{\pgfqpoint{2.928744in}{1.542735in}}%
\pgfpathlineto{\pgfqpoint{2.923415in}{1.542522in}}%
\pgfpathlineto{\pgfqpoint{2.912617in}{1.534288in}}%
\pgfpathlineto{\pgfqpoint{2.909503in}{1.526026in}}%
\pgfpathlineto{\pgfqpoint{2.880603in}{1.518876in}}%
\pgfpathlineto{\pgfqpoint{2.854206in}{1.513499in}}%
\pgfusepath{stroke}%
\end{pgfscope}%
\begin{pgfscope}%
\pgfpathrectangle{\pgfqpoint{0.100000in}{0.100000in}}{\pgfqpoint{3.007045in}{1.925000in}}%
\pgfusepath{clip}%
\pgfsetbuttcap%
\pgfsetroundjoin%
\pgfsetlinewidth{0.050187pt}%
\definecolor{currentstroke}{rgb}{1.000000,1.000000,1.000000}%
\pgfsetstrokecolor{currentstroke}%
\pgfsetdash{}{0pt}%
\pgfpathmoveto{\pgfqpoint{1.950519in}{1.226895in}}%
\pgfpathlineto{\pgfqpoint{1.935262in}{1.242004in}}%
\pgfpathlineto{\pgfqpoint{1.886493in}{1.239191in}}%
\pgfpathlineto{\pgfqpoint{1.810432in}{1.236685in}}%
\pgfpathlineto{\pgfqpoint{1.733914in}{1.237878in}}%
\pgfpathlineto{\pgfqpoint{1.728543in}{1.247244in}}%
\pgfpathlineto{\pgfqpoint{1.730736in}{1.256440in}}%
\pgfpathlineto{\pgfqpoint{1.729695in}{1.272194in}}%
\pgfpathlineto{\pgfqpoint{1.725651in}{1.287453in}}%
\pgfpathlineto{\pgfqpoint{1.726121in}{1.295511in}}%
\pgfpathlineto{\pgfqpoint{1.717525in}{1.304616in}}%
\pgfpathlineto{\pgfqpoint{1.719395in}{1.317284in}}%
\pgfpathlineto{\pgfqpoint{1.706209in}{1.342308in}}%
\pgfpathlineto{\pgfqpoint{1.702281in}{1.363552in}}%
\pgfpathlineto{\pgfqpoint{1.692796in}{1.380831in}}%
\pgfpathlineto{\pgfqpoint{1.701799in}{1.402266in}}%
\pgfpathlineto{\pgfqpoint{1.697315in}{1.412242in}}%
\pgfpathlineto{\pgfqpoint{1.702328in}{1.430275in}}%
\pgfpathlineto{\pgfqpoint{1.750041in}{1.429758in}}%
\pgfpathlineto{\pgfqpoint{1.798422in}{1.429684in}}%
\pgfpathlineto{\pgfqpoint{1.844156in}{1.430356in}}%
\pgfpathlineto{\pgfqpoint{1.894352in}{1.431406in}}%
\pgfpathlineto{\pgfqpoint{1.953238in}{1.433544in}}%
\pgfpathlineto{\pgfqpoint{1.954811in}{1.424995in}}%
\pgfpathlineto{\pgfqpoint{1.961330in}{1.417827in}}%
\pgfpathlineto{\pgfqpoint{1.957202in}{1.410359in}}%
\pgfpathlineto{\pgfqpoint{1.958894in}{1.395780in}}%
\pgfpathlineto{\pgfqpoint{1.962935in}{1.383996in}}%
\pgfpathlineto{\pgfqpoint{1.981252in}{1.377759in}}%
\pgfpathlineto{\pgfqpoint{1.984391in}{1.369415in}}%
\pgfpathlineto{\pgfqpoint{1.994442in}{1.360047in}}%
\pgfpathlineto{\pgfqpoint{1.998543in}{1.350356in}}%
\pgfpathlineto{\pgfqpoint{2.008727in}{1.343855in}}%
\pgfpathlineto{\pgfqpoint{2.010321in}{1.336029in}}%
\pgfpathlineto{\pgfqpoint{2.008338in}{1.324180in}}%
\pgfpathlineto{\pgfqpoint{2.003154in}{1.320629in}}%
\pgfpathlineto{\pgfqpoint{2.001591in}{1.309350in}}%
\pgfpathlineto{\pgfqpoint{1.986693in}{1.300392in}}%
\pgfpathlineto{\pgfqpoint{1.967275in}{1.295485in}}%
\pgfpathlineto{\pgfqpoint{1.965496in}{1.284201in}}%
\pgfpathlineto{\pgfqpoint{1.972987in}{1.276140in}}%
\pgfpathlineto{\pgfqpoint{1.973292in}{1.266004in}}%
\pgfpathlineto{\pgfqpoint{1.967249in}{1.258037in}}%
\pgfpathlineto{\pgfqpoint{1.964076in}{1.246200in}}%
\pgfpathlineto{\pgfqpoint{1.953571in}{1.242282in}}%
\pgfpathlineto{\pgfqpoint{1.954240in}{1.229091in}}%
\pgfpathlineto{\pgfqpoint{1.950519in}{1.226895in}}%
\pgfusepath{stroke}%
\end{pgfscope}%
\begin{pgfscope}%
\pgfpathrectangle{\pgfqpoint{0.100000in}{0.100000in}}{\pgfqpoint{3.007045in}{1.925000in}}%
\pgfusepath{clip}%
\pgfsetbuttcap%
\pgfsetroundjoin%
\pgfsetlinewidth{0.050187pt}%
\definecolor{currentstroke}{rgb}{1.000000,1.000000,1.000000}%
\pgfsetstrokecolor{currentstroke}%
\pgfsetdash{}{0pt}%
\pgfpathmoveto{\pgfqpoint{2.897124in}{1.475380in}}%
\pgfpathlineto{\pgfqpoint{2.882673in}{1.473100in}}%
\pgfpathlineto{\pgfqpoint{2.816297in}{1.458017in}}%
\pgfpathlineto{\pgfqpoint{2.815137in}{1.459774in}}%
\pgfpathlineto{\pgfqpoint{2.816139in}{1.505410in}}%
\pgfpathlineto{\pgfqpoint{2.854206in}{1.513499in}}%
\pgfpathlineto{\pgfqpoint{2.880603in}{1.518876in}}%
\pgfpathlineto{\pgfqpoint{2.909503in}{1.526026in}}%
\pgfpathlineto{\pgfqpoint{2.912617in}{1.534288in}}%
\pgfpathlineto{\pgfqpoint{2.923415in}{1.542522in}}%
\pgfpathlineto{\pgfqpoint{2.928744in}{1.542735in}}%
\pgfpathlineto{\pgfqpoint{2.935751in}{1.530721in}}%
\pgfpathlineto{\pgfqpoint{2.929393in}{1.513184in}}%
\pgfpathlineto{\pgfqpoint{2.928460in}{1.502901in}}%
\pgfpathlineto{\pgfqpoint{2.941328in}{1.503867in}}%
\pgfpathlineto{\pgfqpoint{2.947164in}{1.498934in}}%
\pgfpathlineto{\pgfqpoint{2.960252in}{1.478771in}}%
\pgfpathlineto{\pgfqpoint{2.966756in}{1.476318in}}%
\pgfpathlineto{\pgfqpoint{2.977652in}{1.477228in}}%
\pgfpathlineto{\pgfqpoint{2.985234in}{1.484080in}}%
\pgfpathlineto{\pgfqpoint{2.990274in}{1.477965in}}%
\pgfpathlineto{\pgfqpoint{2.958579in}{1.461239in}}%
\pgfpathlineto{\pgfqpoint{2.957589in}{1.473241in}}%
\pgfpathlineto{\pgfqpoint{2.943191in}{1.454588in}}%
\pgfpathlineto{\pgfqpoint{2.938129in}{1.451375in}}%
\pgfpathlineto{\pgfqpoint{2.931068in}{1.462137in}}%
\pgfpathlineto{\pgfqpoint{2.929135in}{1.463612in}}%
\pgfpathlineto{\pgfqpoint{2.922562in}{1.467094in}}%
\pgfpathlineto{\pgfqpoint{2.916830in}{1.481216in}}%
\pgfpathlineto{\pgfqpoint{2.897124in}{1.475380in}}%
\pgfusepath{stroke}%
\end{pgfscope}%
\begin{pgfscope}%
\pgfpathrectangle{\pgfqpoint{0.100000in}{0.100000in}}{\pgfqpoint{3.007045in}{1.925000in}}%
\pgfusepath{clip}%
\pgfsetbuttcap%
\pgfsetroundjoin%
\pgfsetlinewidth{0.050187pt}%
\definecolor{currentstroke}{rgb}{1.000000,1.000000,1.000000}%
\pgfsetstrokecolor{currentstroke}%
\pgfsetdash{}{0pt}%
\pgfpathmoveto{\pgfqpoint{1.416825in}{1.212637in}}%
\pgfpathlineto{\pgfqpoint{1.422117in}{1.278255in}}%
\pgfpathlineto{\pgfqpoint{1.392146in}{1.280688in}}%
\pgfpathlineto{\pgfqpoint{1.322867in}{1.287410in}}%
\pgfpathlineto{\pgfqpoint{1.329313in}{1.352796in}}%
\pgfpathlineto{\pgfqpoint{1.335768in}{1.418681in}}%
\pgfpathlineto{\pgfqpoint{1.362400in}{1.415805in}}%
\pgfpathlineto{\pgfqpoint{1.430419in}{1.409379in}}%
\pgfpathlineto{\pgfqpoint{1.471583in}{1.406155in}}%
\pgfpathlineto{\pgfqpoint{1.521419in}{1.403298in}}%
\pgfpathlineto{\pgfqpoint{1.603447in}{1.399416in}}%
\pgfpathlineto{\pgfqpoint{1.605250in}{1.395738in}}%
\pgfpathlineto{\pgfqpoint{1.628754in}{1.384110in}}%
\pgfpathlineto{\pgfqpoint{1.635962in}{1.390016in}}%
\pgfpathlineto{\pgfqpoint{1.656727in}{1.389477in}}%
\pgfpathlineto{\pgfqpoint{1.669102in}{1.382994in}}%
\pgfpathlineto{\pgfqpoint{1.688724in}{1.375625in}}%
\pgfpathlineto{\pgfqpoint{1.694222in}{1.364874in}}%
\pgfpathlineto{\pgfqpoint{1.702281in}{1.363552in}}%
\pgfpathlineto{\pgfqpoint{1.706209in}{1.342308in}}%
\pgfpathlineto{\pgfqpoint{1.719395in}{1.317284in}}%
\pgfpathlineto{\pgfqpoint{1.717525in}{1.304616in}}%
\pgfpathlineto{\pgfqpoint{1.726121in}{1.295511in}}%
\pgfpathlineto{\pgfqpoint{1.725651in}{1.287453in}}%
\pgfpathlineto{\pgfqpoint{1.729695in}{1.272194in}}%
\pgfpathlineto{\pgfqpoint{1.730736in}{1.256440in}}%
\pgfpathlineto{\pgfqpoint{1.728543in}{1.247244in}}%
\pgfpathlineto{\pgfqpoint{1.733914in}{1.237878in}}%
\pgfpathlineto{\pgfqpoint{1.741281in}{1.220848in}}%
\pgfpathlineto{\pgfqpoint{1.748332in}{1.213916in}}%
\pgfpathlineto{\pgfqpoint{1.756737in}{1.198896in}}%
\pgfpathlineto{\pgfqpoint{1.732915in}{1.198649in}}%
\pgfpathlineto{\pgfqpoint{1.681409in}{1.199446in}}%
\pgfpathlineto{\pgfqpoint{1.624502in}{1.201190in}}%
\pgfpathlineto{\pgfqpoint{1.567259in}{1.203386in}}%
\pgfpathlineto{\pgfqpoint{1.483096in}{1.207983in}}%
\pgfpathlineto{\pgfqpoint{1.416825in}{1.212637in}}%
\pgfusepath{stroke}%
\end{pgfscope}%
\begin{pgfscope}%
\pgfpathrectangle{\pgfqpoint{0.100000in}{0.100000in}}{\pgfqpoint{3.007045in}{1.925000in}}%
\pgfusepath{clip}%
\pgfsetbuttcap%
\pgfsetroundjoin%
\pgfsetlinewidth{0.050187pt}%
\definecolor{currentstroke}{rgb}{1.000000,1.000000,1.000000}%
\pgfsetstrokecolor{currentstroke}%
\pgfsetdash{}{0pt}%
\pgfpathmoveto{\pgfqpoint{2.512669in}{1.410567in}}%
\pgfpathlineto{\pgfqpoint{2.539023in}{1.435695in}}%
\pgfpathlineto{\pgfqpoint{2.542299in}{1.444627in}}%
\pgfpathlineto{\pgfqpoint{2.550021in}{1.452274in}}%
\pgfpathlineto{\pgfqpoint{2.544226in}{1.463407in}}%
\pgfpathlineto{\pgfqpoint{2.536962in}{1.469919in}}%
\pgfpathlineto{\pgfqpoint{2.534862in}{1.481459in}}%
\pgfpathlineto{\pgfqpoint{2.561903in}{1.493308in}}%
\pgfpathlineto{\pgfqpoint{2.584291in}{1.497056in}}%
\pgfpathlineto{\pgfqpoint{2.596321in}{1.497306in}}%
\pgfpathlineto{\pgfqpoint{2.605485in}{1.492779in}}%
\pgfpathlineto{\pgfqpoint{2.614418in}{1.496822in}}%
\pgfpathlineto{\pgfqpoint{2.636206in}{1.501353in}}%
\pgfpathlineto{\pgfqpoint{2.643734in}{1.507201in}}%
\pgfpathlineto{\pgfqpoint{2.654897in}{1.520147in}}%
\pgfpathlineto{\pgfqpoint{2.665052in}{1.525868in}}%
\pgfpathlineto{\pgfqpoint{2.665769in}{1.531355in}}%
\pgfpathlineto{\pgfqpoint{2.660439in}{1.543873in}}%
\pgfpathlineto{\pgfqpoint{2.664292in}{1.551240in}}%
\pgfpathlineto{\pgfqpoint{2.659104in}{1.559161in}}%
\pgfpathlineto{\pgfqpoint{2.651159in}{1.559714in}}%
\pgfpathlineto{\pgfqpoint{2.671043in}{1.583641in}}%
\pgfpathlineto{\pgfqpoint{2.673405in}{1.592759in}}%
\pgfpathlineto{\pgfqpoint{2.689054in}{1.616015in}}%
\pgfpathlineto{\pgfqpoint{2.703581in}{1.628599in}}%
\pgfpathlineto{\pgfqpoint{2.713557in}{1.633854in}}%
\pgfpathlineto{\pgfqpoint{2.746198in}{1.641250in}}%
\pgfpathlineto{\pgfqpoint{2.776791in}{1.649602in}}%
\pgfpathlineto{\pgfqpoint{2.780843in}{1.636492in}}%
\pgfpathlineto{\pgfqpoint{2.781130in}{1.624335in}}%
\pgfpathlineto{\pgfqpoint{2.788108in}{1.613300in}}%
\pgfpathlineto{\pgfqpoint{2.790259in}{1.602064in}}%
\pgfpathlineto{\pgfqpoint{2.787577in}{1.586820in}}%
\pgfpathlineto{\pgfqpoint{2.795745in}{1.566258in}}%
\pgfpathlineto{\pgfqpoint{2.805056in}{1.554825in}}%
\pgfpathlineto{\pgfqpoint{2.812258in}{1.517661in}}%
\pgfpathlineto{\pgfqpoint{2.816139in}{1.505410in}}%
\pgfpathlineto{\pgfqpoint{2.815137in}{1.459774in}}%
\pgfpathlineto{\pgfqpoint{2.816297in}{1.458017in}}%
\pgfpathlineto{\pgfqpoint{2.824776in}{1.408935in}}%
\pgfpathlineto{\pgfqpoint{2.829531in}{1.404463in}}%
\pgfpathlineto{\pgfqpoint{2.819303in}{1.394526in}}%
\pgfpathlineto{\pgfqpoint{2.824350in}{1.388822in}}%
\pgfpathlineto{\pgfqpoint{2.819914in}{1.380196in}}%
\pgfpathlineto{\pgfqpoint{2.819952in}{1.376524in}}%
\pgfpathlineto{\pgfqpoint{2.814406in}{1.373210in}}%
\pgfpathlineto{\pgfqpoint{2.811706in}{1.365884in}}%
\pgfpathlineto{\pgfqpoint{2.812563in}{1.386028in}}%
\pgfpathlineto{\pgfqpoint{2.795375in}{1.390462in}}%
\pgfpathlineto{\pgfqpoint{2.768496in}{1.399631in}}%
\pgfpathlineto{\pgfqpoint{2.764695in}{1.404147in}}%
\pgfpathlineto{\pgfqpoint{2.753463in}{1.405229in}}%
\pgfpathlineto{\pgfqpoint{2.747002in}{1.411951in}}%
\pgfpathlineto{\pgfqpoint{2.743516in}{1.425338in}}%
\pgfpathlineto{\pgfqpoint{2.734357in}{1.427003in}}%
\pgfpathlineto{\pgfqpoint{2.728231in}{1.434026in}}%
\pgfpathlineto{\pgfqpoint{2.650253in}{1.418305in}}%
\pgfpathlineto{\pgfqpoint{2.612985in}{1.410597in}}%
\pgfpathlineto{\pgfqpoint{2.556401in}{1.400270in}}%
\pgfpathlineto{\pgfqpoint{2.515654in}{1.393396in}}%
\pgfpathlineto{\pgfqpoint{2.512669in}{1.410567in}}%
\pgfusepath{stroke}%
\end{pgfscope}%
\begin{pgfscope}%
\pgfpathrectangle{\pgfqpoint{0.100000in}{0.100000in}}{\pgfqpoint{3.007045in}{1.925000in}}%
\pgfusepath{clip}%
\pgfsetbuttcap%
\pgfsetroundjoin%
\pgfsetlinewidth{0.050187pt}%
\definecolor{currentstroke}{rgb}{1.000000,1.000000,1.000000}%
\pgfsetstrokecolor{currentstroke}%
\pgfsetdash{}{0pt}%
\pgfpathmoveto{\pgfqpoint{2.825722in}{1.361793in}}%
\pgfpathlineto{\pgfqpoint{2.813662in}{1.358036in}}%
\pgfpathlineto{\pgfqpoint{2.811634in}{1.361490in}}%
\pgfpathlineto{\pgfqpoint{2.815505in}{1.373084in}}%
\pgfpathlineto{\pgfqpoint{2.822673in}{1.374226in}}%
\pgfpathlineto{\pgfqpoint{2.822032in}{1.377876in}}%
\pgfpathlineto{\pgfqpoint{2.828469in}{1.383355in}}%
\pgfpathlineto{\pgfqpoint{2.847078in}{1.387719in}}%
\pgfpathlineto{\pgfqpoint{2.855363in}{1.394327in}}%
\pgfpathlineto{\pgfqpoint{2.873974in}{1.399833in}}%
\pgfpathlineto{\pgfqpoint{2.882462in}{1.397818in}}%
\pgfpathlineto{\pgfqpoint{2.889602in}{1.406695in}}%
\pgfpathlineto{\pgfqpoint{2.900394in}{1.407868in}}%
\pgfpathlineto{\pgfqpoint{2.881983in}{1.390553in}}%
\pgfpathlineto{\pgfqpoint{2.825722in}{1.361793in}}%
\pgfusepath{stroke}%
\end{pgfscope}%
\begin{pgfscope}%
\pgfpathrectangle{\pgfqpoint{0.100000in}{0.100000in}}{\pgfqpoint{3.007045in}{1.925000in}}%
\pgfusepath{clip}%
\pgfsetbuttcap%
\pgfsetroundjoin%
\pgfsetlinewidth{0.050187pt}%
\definecolor{currentstroke}{rgb}{1.000000,1.000000,1.000000}%
\pgfsetstrokecolor{currentstroke}%
\pgfsetdash{}{0pt}%
\pgfpathmoveto{\pgfqpoint{2.554781in}{1.247456in}}%
\pgfpathlineto{\pgfqpoint{2.502630in}{1.238834in}}%
\pgfpathlineto{\pgfqpoint{2.493170in}{1.298432in}}%
\pgfpathlineto{\pgfqpoint{2.479125in}{1.386284in}}%
\pgfpathlineto{\pgfqpoint{2.512669in}{1.410567in}}%
\pgfpathlineto{\pgfqpoint{2.515654in}{1.393396in}}%
\pgfpathlineto{\pgfqpoint{2.556401in}{1.400270in}}%
\pgfpathlineto{\pgfqpoint{2.612985in}{1.410597in}}%
\pgfpathlineto{\pgfqpoint{2.650253in}{1.418305in}}%
\pgfpathlineto{\pgfqpoint{2.728231in}{1.434026in}}%
\pgfpathlineto{\pgfqpoint{2.734357in}{1.427003in}}%
\pgfpathlineto{\pgfqpoint{2.743516in}{1.425338in}}%
\pgfpathlineto{\pgfqpoint{2.747002in}{1.411951in}}%
\pgfpathlineto{\pgfqpoint{2.753463in}{1.405229in}}%
\pgfpathlineto{\pgfqpoint{2.764695in}{1.404147in}}%
\pgfpathlineto{\pgfqpoint{2.768496in}{1.399631in}}%
\pgfpathlineto{\pgfqpoint{2.764637in}{1.396146in}}%
\pgfpathlineto{\pgfqpoint{2.761160in}{1.383818in}}%
\pgfpathlineto{\pgfqpoint{2.753012in}{1.369966in}}%
\pgfpathlineto{\pgfqpoint{2.758481in}{1.363942in}}%
\pgfpathlineto{\pgfqpoint{2.754006in}{1.357488in}}%
\pgfpathlineto{\pgfqpoint{2.756531in}{1.343334in}}%
\pgfpathlineto{\pgfqpoint{2.764590in}{1.335904in}}%
\pgfpathlineto{\pgfqpoint{2.783714in}{1.323833in}}%
\pgfpathlineto{\pgfqpoint{2.768314in}{1.306801in}}%
\pgfpathlineto{\pgfqpoint{2.768098in}{1.300337in}}%
\pgfpathlineto{\pgfqpoint{2.755560in}{1.292000in}}%
\pgfpathlineto{\pgfqpoint{2.741696in}{1.290435in}}%
\pgfpathlineto{\pgfqpoint{2.738266in}{1.283189in}}%
\pgfpathlineto{\pgfqpoint{2.699704in}{1.274845in}}%
\pgfpathlineto{\pgfqpoint{2.654706in}{1.265782in}}%
\pgfpathlineto{\pgfqpoint{2.623776in}{1.260244in}}%
\pgfpathlineto{\pgfqpoint{2.554781in}{1.247456in}}%
\pgfusepath{stroke}%
\end{pgfscope}%
\begin{pgfscope}%
\pgfpathrectangle{\pgfqpoint{0.100000in}{0.100000in}}{\pgfqpoint{3.007045in}{1.925000in}}%
\pgfusepath{clip}%
\pgfsetbuttcap%
\pgfsetroundjoin%
\pgfsetlinewidth{0.050187pt}%
\definecolor{currentstroke}{rgb}{1.000000,1.000000,1.000000}%
\pgfsetstrokecolor{currentstroke}%
\pgfsetdash{}{0pt}%
\pgfpathmoveto{\pgfqpoint{2.816297in}{1.458017in}}%
\pgfpathlineto{\pgfqpoint{2.882673in}{1.473100in}}%
\pgfpathlineto{\pgfqpoint{2.897124in}{1.475380in}}%
\pgfpathlineto{\pgfqpoint{2.906688in}{1.437776in}}%
\pgfpathlineto{\pgfqpoint{2.905174in}{1.431042in}}%
\pgfpathlineto{\pgfqpoint{2.874451in}{1.419148in}}%
\pgfpathlineto{\pgfqpoint{2.856110in}{1.414991in}}%
\pgfpathlineto{\pgfqpoint{2.848316in}{1.405665in}}%
\pgfpathlineto{\pgfqpoint{2.824350in}{1.388822in}}%
\pgfpathlineto{\pgfqpoint{2.819303in}{1.394526in}}%
\pgfpathlineto{\pgfqpoint{2.829531in}{1.404463in}}%
\pgfpathlineto{\pgfqpoint{2.824776in}{1.408935in}}%
\pgfpathlineto{\pgfqpoint{2.816297in}{1.458017in}}%
\pgfusepath{stroke}%
\end{pgfscope}%
\begin{pgfscope}%
\pgfpathrectangle{\pgfqpoint{0.100000in}{0.100000in}}{\pgfqpoint{3.007045in}{1.925000in}}%
\pgfusepath{clip}%
\pgfsetbuttcap%
\pgfsetroundjoin%
\pgfsetlinewidth{0.050187pt}%
\definecolor{currentstroke}{rgb}{1.000000,1.000000,1.000000}%
\pgfsetstrokecolor{currentstroke}%
\pgfsetdash{}{0pt}%
\pgfpathmoveto{\pgfqpoint{2.897124in}{1.475380in}}%
\pgfpathlineto{\pgfqpoint{2.916830in}{1.481216in}}%
\pgfpathlineto{\pgfqpoint{2.922562in}{1.467094in}}%
\pgfpathlineto{\pgfqpoint{2.929135in}{1.463612in}}%
\pgfpathlineto{\pgfqpoint{2.921024in}{1.457664in}}%
\pgfpathlineto{\pgfqpoint{2.922047in}{1.440194in}}%
\pgfpathlineto{\pgfqpoint{2.905174in}{1.431042in}}%
\pgfpathlineto{\pgfqpoint{2.906688in}{1.437776in}}%
\pgfpathlineto{\pgfqpoint{2.897124in}{1.475380in}}%
\pgfusepath{stroke}%
\end{pgfscope}%
\begin{pgfscope}%
\pgfpathrectangle{\pgfqpoint{0.100000in}{0.100000in}}{\pgfqpoint{3.007045in}{1.925000in}}%
\pgfusepath{clip}%
\pgfsetbuttcap%
\pgfsetroundjoin%
\pgfsetlinewidth{0.050187pt}%
\definecolor{currentstroke}{rgb}{1.000000,1.000000,1.000000}%
\pgfsetstrokecolor{currentstroke}%
\pgfsetdash{}{0pt}%
\pgfpathmoveto{\pgfqpoint{2.753338in}{1.285842in}}%
\pgfpathlineto{\pgfqpoint{2.755560in}{1.292000in}}%
\pgfpathlineto{\pgfqpoint{2.768098in}{1.300337in}}%
\pgfpathlineto{\pgfqpoint{2.768314in}{1.306801in}}%
\pgfpathlineto{\pgfqpoint{2.783714in}{1.323833in}}%
\pgfpathlineto{\pgfqpoint{2.764590in}{1.335904in}}%
\pgfpathlineto{\pgfqpoint{2.756531in}{1.343334in}}%
\pgfpathlineto{\pgfqpoint{2.754006in}{1.357488in}}%
\pgfpathlineto{\pgfqpoint{2.758481in}{1.363942in}}%
\pgfpathlineto{\pgfqpoint{2.753012in}{1.369966in}}%
\pgfpathlineto{\pgfqpoint{2.761160in}{1.383818in}}%
\pgfpathlineto{\pgfqpoint{2.764637in}{1.396146in}}%
\pgfpathlineto{\pgfqpoint{2.768496in}{1.399631in}}%
\pgfpathlineto{\pgfqpoint{2.795375in}{1.390462in}}%
\pgfpathlineto{\pgfqpoint{2.812563in}{1.386028in}}%
\pgfpathlineto{\pgfqpoint{2.811706in}{1.365884in}}%
\pgfpathlineto{\pgfqpoint{2.801265in}{1.350609in}}%
\pgfpathlineto{\pgfqpoint{2.809869in}{1.348362in}}%
\pgfpathlineto{\pgfqpoint{2.818801in}{1.341807in}}%
\pgfpathlineto{\pgfqpoint{2.819328in}{1.323878in}}%
\pgfpathlineto{\pgfqpoint{2.816624in}{1.311186in}}%
\pgfpathlineto{\pgfqpoint{2.818423in}{1.300760in}}%
\pgfpathlineto{\pgfqpoint{2.802896in}{1.265581in}}%
\pgfpathlineto{\pgfqpoint{2.797316in}{1.249154in}}%
\pgfpathlineto{\pgfqpoint{2.789524in}{1.257138in}}%
\pgfpathlineto{\pgfqpoint{2.779192in}{1.255778in}}%
\pgfpathlineto{\pgfqpoint{2.753333in}{1.270718in}}%
\pgfpathlineto{\pgfqpoint{2.750687in}{1.278719in}}%
\pgfpathlineto{\pgfqpoint{2.753338in}{1.285842in}}%
\pgfusepath{stroke}%
\end{pgfscope}%
\begin{pgfscope}%
\pgfpathrectangle{\pgfqpoint{0.100000in}{0.100000in}}{\pgfqpoint{3.007045in}{1.925000in}}%
\pgfusepath{clip}%
\pgfsetbuttcap%
\pgfsetroundjoin%
\pgfsetlinewidth{0.050187pt}%
\definecolor{currentstroke}{rgb}{1.000000,1.000000,1.000000}%
\pgfsetstrokecolor{currentstroke}%
\pgfsetdash{}{0pt}%
\pgfpathmoveto{\pgfqpoint{2.134480in}{1.068255in}}%
\pgfpathlineto{\pgfqpoint{2.133271in}{1.084575in}}%
\pgfpathlineto{\pgfqpoint{2.137970in}{1.092672in}}%
\pgfpathlineto{\pgfqpoint{2.134873in}{1.096640in}}%
\pgfpathlineto{\pgfqpoint{2.146167in}{1.109630in}}%
\pgfpathlineto{\pgfqpoint{2.145520in}{1.112760in}}%
\pgfpathlineto{\pgfqpoint{2.156492in}{1.135423in}}%
\pgfpathlineto{\pgfqpoint{2.154275in}{1.146352in}}%
\pgfpathlineto{\pgfqpoint{2.146318in}{1.157795in}}%
\pgfpathlineto{\pgfqpoint{2.151841in}{1.171716in}}%
\pgfpathlineto{\pgfqpoint{2.144682in}{1.263329in}}%
\pgfpathlineto{\pgfqpoint{2.139527in}{1.327598in}}%
\pgfpathlineto{\pgfqpoint{2.146648in}{1.322273in}}%
\pgfpathlineto{\pgfqpoint{2.154593in}{1.322417in}}%
\pgfpathlineto{\pgfqpoint{2.173359in}{1.333287in}}%
\pgfpathlineto{\pgfqpoint{2.230918in}{1.338658in}}%
\pgfpathlineto{\pgfqpoint{2.273522in}{1.343148in}}%
\pgfpathlineto{\pgfqpoint{2.273898in}{1.338979in}}%
\pgfpathlineto{\pgfqpoint{2.283623in}{1.250907in}}%
\pgfpathlineto{\pgfqpoint{2.291994in}{1.168959in}}%
\pgfpathlineto{\pgfqpoint{2.288386in}{1.165112in}}%
\pgfpathlineto{\pgfqpoint{2.293928in}{1.155563in}}%
\pgfpathlineto{\pgfqpoint{2.293900in}{1.148681in}}%
\pgfpathlineto{\pgfqpoint{2.285991in}{1.146952in}}%
\pgfpathlineto{\pgfqpoint{2.277140in}{1.140317in}}%
\pgfpathlineto{\pgfqpoint{2.271148in}{1.142929in}}%
\pgfpathlineto{\pgfqpoint{2.262180in}{1.138682in}}%
\pgfpathlineto{\pgfqpoint{2.264955in}{1.130153in}}%
\pgfpathlineto{\pgfqpoint{2.255736in}{1.121584in}}%
\pgfpathlineto{\pgfqpoint{2.253191in}{1.111671in}}%
\pgfpathlineto{\pgfqpoint{2.246853in}{1.110045in}}%
\pgfpathlineto{\pgfqpoint{2.242123in}{1.102538in}}%
\pgfpathlineto{\pgfqpoint{2.242742in}{1.094978in}}%
\pgfpathlineto{\pgfqpoint{2.237177in}{1.089662in}}%
\pgfpathlineto{\pgfqpoint{2.228803in}{1.090482in}}%
\pgfpathlineto{\pgfqpoint{2.222438in}{1.098640in}}%
\pgfpathlineto{\pgfqpoint{2.211653in}{1.090784in}}%
\pgfpathlineto{\pgfqpoint{2.212410in}{1.083920in}}%
\pgfpathlineto{\pgfqpoint{2.200414in}{1.079910in}}%
\pgfpathlineto{\pgfqpoint{2.197401in}{1.084958in}}%
\pgfpathlineto{\pgfqpoint{2.185723in}{1.079234in}}%
\pgfpathlineto{\pgfqpoint{2.181457in}{1.071040in}}%
\pgfpathlineto{\pgfqpoint{2.167393in}{1.079442in}}%
\pgfpathlineto{\pgfqpoint{2.145103in}{1.073876in}}%
\pgfpathlineto{\pgfqpoint{2.134480in}{1.068255in}}%
\pgfusepath{stroke}%
\end{pgfscope}%
\begin{pgfscope}%
\pgfpathrectangle{\pgfqpoint{0.100000in}{0.100000in}}{\pgfqpoint{3.007045in}{1.925000in}}%
\pgfusepath{clip}%
\pgfsetbuttcap%
\pgfsetroundjoin%
\pgfsetlinewidth{0.050187pt}%
\definecolor{currentstroke}{rgb}{1.000000,1.000000,1.000000}%
\pgfsetstrokecolor{currentstroke}%
\pgfsetdash{}{0pt}%
\pgfpathmoveto{\pgfqpoint{0.562516in}{1.497430in}}%
\pgfpathlineto{\pgfqpoint{0.594748in}{1.488461in}}%
\pgfpathlineto{\pgfqpoint{0.648178in}{1.474684in}}%
\pgfpathlineto{\pgfqpoint{0.703563in}{1.460492in}}%
\pgfpathlineto{\pgfqpoint{0.754611in}{1.448181in}}%
\pgfpathlineto{\pgfqpoint{0.798720in}{1.438195in}}%
\pgfpathlineto{\pgfqpoint{0.845912in}{1.427818in}}%
\pgfpathlineto{\pgfqpoint{0.832277in}{1.363472in}}%
\pgfpathlineto{\pgfqpoint{0.820130in}{1.306329in}}%
\pgfpathlineto{\pgfqpoint{0.800203in}{1.214130in}}%
\pgfpathlineto{\pgfqpoint{0.785243in}{1.144539in}}%
\pgfpathlineto{\pgfqpoint{0.777160in}{1.105704in}}%
\pgfpathlineto{\pgfqpoint{0.766797in}{1.055300in}}%
\pgfpathlineto{\pgfqpoint{0.759627in}{1.045075in}}%
\pgfpathlineto{\pgfqpoint{0.753855in}{1.044733in}}%
\pgfpathlineto{\pgfqpoint{0.749739in}{1.053659in}}%
\pgfpathlineto{\pgfqpoint{0.740288in}{1.056900in}}%
\pgfpathlineto{\pgfqpoint{0.730103in}{1.055764in}}%
\pgfpathlineto{\pgfqpoint{0.727212in}{1.048463in}}%
\pgfpathlineto{\pgfqpoint{0.726911in}{1.023110in}}%
\pgfpathlineto{\pgfqpoint{0.723861in}{1.017321in}}%
\pgfpathlineto{\pgfqpoint{0.725604in}{0.996967in}}%
\pgfpathlineto{\pgfqpoint{0.719235in}{0.983407in}}%
\pgfpathlineto{\pgfqpoint{0.667589in}{1.062934in}}%
\pgfpathlineto{\pgfqpoint{0.616726in}{1.140213in}}%
\pgfpathlineto{\pgfqpoint{0.590250in}{1.180736in}}%
\pgfpathlineto{\pgfqpoint{0.568163in}{1.215713in}}%
\pgfpathlineto{\pgfqpoint{0.540333in}{1.258980in}}%
\pgfpathlineto{\pgfqpoint{0.508830in}{1.307514in}}%
\pgfpathlineto{\pgfqpoint{0.521782in}{1.353580in}}%
\pgfpathlineto{\pgfqpoint{0.547847in}{1.445968in}}%
\pgfpathlineto{\pgfqpoint{0.562516in}{1.497430in}}%
\pgfusepath{stroke}%
\end{pgfscope}%
\begin{pgfscope}%
\pgfpathrectangle{\pgfqpoint{0.100000in}{0.100000in}}{\pgfqpoint{3.007045in}{1.925000in}}%
\pgfusepath{clip}%
\pgfsetbuttcap%
\pgfsetroundjoin%
\pgfsetlinewidth{0.050187pt}%
\definecolor{currentstroke}{rgb}{1.000000,1.000000,1.000000}%
\pgfsetstrokecolor{currentstroke}%
\pgfsetdash{}{0pt}%
\pgfpathmoveto{\pgfqpoint{0.845912in}{1.427818in}}%
\pgfpathlineto{\pgfqpoint{0.896364in}{1.417782in}}%
\pgfpathlineto{\pgfqpoint{0.989767in}{1.399751in}}%
\pgfpathlineto{\pgfqpoint{0.978076in}{1.334895in}}%
\pgfpathlineto{\pgfqpoint{1.029497in}{1.326091in}}%
\pgfpathlineto{\pgfqpoint{1.076345in}{1.318594in}}%
\pgfpathlineto{\pgfqpoint{1.068205in}{1.267320in}}%
\pgfpathlineto{\pgfqpoint{1.059579in}{1.212028in}}%
\pgfpathlineto{\pgfqpoint{1.048031in}{1.139428in}}%
\pgfpathlineto{\pgfqpoint{1.047732in}{1.133342in}}%
\pgfpathlineto{\pgfqpoint{1.035744in}{1.058103in}}%
\pgfpathlineto{\pgfqpoint{0.986397in}{1.065751in}}%
\pgfpathlineto{\pgfqpoint{0.961251in}{1.070801in}}%
\pgfpathlineto{\pgfqpoint{0.870361in}{1.086778in}}%
\pgfpathlineto{\pgfqpoint{0.836134in}{1.093528in}}%
\pgfpathlineto{\pgfqpoint{0.777160in}{1.105704in}}%
\pgfpathlineto{\pgfqpoint{0.785243in}{1.144539in}}%
\pgfpathlineto{\pgfqpoint{0.800203in}{1.214130in}}%
\pgfpathlineto{\pgfqpoint{0.820130in}{1.306329in}}%
\pgfpathlineto{\pgfqpoint{0.832277in}{1.363472in}}%
\pgfpathlineto{\pgfqpoint{0.845912in}{1.427818in}}%
\pgfusepath{stroke}%
\end{pgfscope}%
\begin{pgfscope}%
\pgfpathrectangle{\pgfqpoint{0.100000in}{0.100000in}}{\pgfqpoint{3.007045in}{1.925000in}}%
\pgfusepath{clip}%
\pgfsetbuttcap%
\pgfsetroundjoin%
\pgfsetlinewidth{0.050187pt}%
\definecolor{currentstroke}{rgb}{1.000000,1.000000,1.000000}%
\pgfsetstrokecolor{currentstroke}%
\pgfsetdash{}{0pt}%
\pgfpathmoveto{\pgfqpoint{0.365302in}{1.558630in}}%
\pgfpathlineto{\pgfqpoint{0.383230in}{1.552378in}}%
\pgfpathlineto{\pgfqpoint{0.455041in}{1.530020in}}%
\pgfpathlineto{\pgfqpoint{0.521101in}{1.509228in}}%
\pgfpathlineto{\pgfqpoint{0.562516in}{1.497430in}}%
\pgfpathlineto{\pgfqpoint{0.547847in}{1.445968in}}%
\pgfpathlineto{\pgfqpoint{0.521782in}{1.353580in}}%
\pgfpathlineto{\pgfqpoint{0.508830in}{1.307514in}}%
\pgfpathlineto{\pgfqpoint{0.540333in}{1.258980in}}%
\pgfpathlineto{\pgfqpoint{0.568163in}{1.215713in}}%
\pgfpathlineto{\pgfqpoint{0.590250in}{1.180736in}}%
\pgfpathlineto{\pgfqpoint{0.616726in}{1.140213in}}%
\pgfpathlineto{\pgfqpoint{0.667589in}{1.062934in}}%
\pgfpathlineto{\pgfqpoint{0.719235in}{0.983407in}}%
\pgfpathlineto{\pgfqpoint{0.717162in}{0.975520in}}%
\pgfpathlineto{\pgfqpoint{0.723391in}{0.962960in}}%
\pgfpathlineto{\pgfqpoint{0.724623in}{0.945778in}}%
\pgfpathlineto{\pgfqpoint{0.733671in}{0.937444in}}%
\pgfpathlineto{\pgfqpoint{0.734029in}{0.930728in}}%
\pgfpathlineto{\pgfqpoint{0.717829in}{0.923106in}}%
\pgfpathlineto{\pgfqpoint{0.710111in}{0.915475in}}%
\pgfpathlineto{\pgfqpoint{0.707704in}{0.898624in}}%
\pgfpathlineto{\pgfqpoint{0.703798in}{0.889434in}}%
\pgfpathlineto{\pgfqpoint{0.695567in}{0.881688in}}%
\pgfpathlineto{\pgfqpoint{0.687380in}{0.861545in}}%
\pgfpathlineto{\pgfqpoint{0.698891in}{0.851052in}}%
\pgfpathlineto{\pgfqpoint{0.697405in}{0.842403in}}%
\pgfpathlineto{\pgfqpoint{0.687967in}{0.836384in}}%
\pgfpathlineto{\pgfqpoint{0.681459in}{0.837446in}}%
\pgfpathlineto{\pgfqpoint{0.604646in}{0.848106in}}%
\pgfpathlineto{\pgfqpoint{0.547913in}{0.856154in}}%
\pgfpathlineto{\pgfqpoint{0.550394in}{0.865314in}}%
\pgfpathlineto{\pgfqpoint{0.546700in}{0.880521in}}%
\pgfpathlineto{\pgfqpoint{0.546328in}{0.895860in}}%
\pgfpathlineto{\pgfqpoint{0.543914in}{0.904837in}}%
\pgfpathlineto{\pgfqpoint{0.536479in}{0.917671in}}%
\pgfpathlineto{\pgfqpoint{0.515092in}{0.947282in}}%
\pgfpathlineto{\pgfqpoint{0.508088in}{0.950910in}}%
\pgfpathlineto{\pgfqpoint{0.499063in}{0.950853in}}%
\pgfpathlineto{\pgfqpoint{0.501124in}{0.960194in}}%
\pgfpathlineto{\pgfqpoint{0.496851in}{0.971875in}}%
\pgfpathlineto{\pgfqpoint{0.475860in}{0.977681in}}%
\pgfpathlineto{\pgfqpoint{0.463064in}{0.988432in}}%
\pgfpathlineto{\pgfqpoint{0.462003in}{0.995009in}}%
\pgfpathlineto{\pgfqpoint{0.447270in}{1.011295in}}%
\pgfpathlineto{\pgfqpoint{0.433245in}{1.014411in}}%
\pgfpathlineto{\pgfqpoint{0.420251in}{1.022687in}}%
\pgfpathlineto{\pgfqpoint{0.403161in}{1.025551in}}%
\pgfpathlineto{\pgfqpoint{0.395859in}{1.036589in}}%
\pgfpathlineto{\pgfqpoint{0.402794in}{1.054061in}}%
\pgfpathlineto{\pgfqpoint{0.400673in}{1.057990in}}%
\pgfpathlineto{\pgfqpoint{0.406447in}{1.072555in}}%
\pgfpathlineto{\pgfqpoint{0.396175in}{1.080306in}}%
\pgfpathlineto{\pgfqpoint{0.399508in}{1.094359in}}%
\pgfpathlineto{\pgfqpoint{0.394021in}{1.097951in}}%
\pgfpathlineto{\pgfqpoint{0.389283in}{1.111233in}}%
\pgfpathlineto{\pgfqpoint{0.383566in}{1.115284in}}%
\pgfpathlineto{\pgfqpoint{0.383136in}{1.124877in}}%
\pgfpathlineto{\pgfqpoint{0.378657in}{1.131642in}}%
\pgfpathlineto{\pgfqpoint{0.371908in}{1.154426in}}%
\pgfpathlineto{\pgfqpoint{0.364523in}{1.165339in}}%
\pgfpathlineto{\pgfqpoint{0.366128in}{1.183864in}}%
\pgfpathlineto{\pgfqpoint{0.374812in}{1.185728in}}%
\pgfpathlineto{\pgfqpoint{0.380467in}{1.195776in}}%
\pgfpathlineto{\pgfqpoint{0.377050in}{1.206674in}}%
\pgfpathlineto{\pgfqpoint{0.367814in}{1.208493in}}%
\pgfpathlineto{\pgfqpoint{0.363206in}{1.213591in}}%
\pgfpathlineto{\pgfqpoint{0.355744in}{1.232340in}}%
\pgfpathlineto{\pgfqpoint{0.359231in}{1.239073in}}%
\pgfpathlineto{\pgfqpoint{0.356757in}{1.251611in}}%
\pgfpathlineto{\pgfqpoint{0.362235in}{1.267849in}}%
\pgfpathlineto{\pgfqpoint{0.368652in}{1.261869in}}%
\pgfpathlineto{\pgfqpoint{0.365740in}{1.254809in}}%
\pgfpathlineto{\pgfqpoint{0.376861in}{1.243618in}}%
\pgfpathlineto{\pgfqpoint{0.376151in}{1.260237in}}%
\pgfpathlineto{\pgfqpoint{0.371386in}{1.264694in}}%
\pgfpathlineto{\pgfqpoint{0.376833in}{1.279306in}}%
\pgfpathlineto{\pgfqpoint{0.397434in}{1.276979in}}%
\pgfpathlineto{\pgfqpoint{0.394662in}{1.282413in}}%
\pgfpathlineto{\pgfqpoint{0.381065in}{1.281878in}}%
\pgfpathlineto{\pgfqpoint{0.374596in}{1.290108in}}%
\pgfpathlineto{\pgfqpoint{0.366455in}{1.282800in}}%
\pgfpathlineto{\pgfqpoint{0.362134in}{1.270584in}}%
\pgfpathlineto{\pgfqpoint{0.350603in}{1.287022in}}%
\pgfpathlineto{\pgfqpoint{0.346162in}{1.290014in}}%
\pgfpathlineto{\pgfqpoint{0.347854in}{1.307904in}}%
\pgfpathlineto{\pgfqpoint{0.344340in}{1.318454in}}%
\pgfpathlineto{\pgfqpoint{0.337968in}{1.328366in}}%
\pgfpathlineto{\pgfqpoint{0.324921in}{1.358735in}}%
\pgfpathlineto{\pgfqpoint{0.329187in}{1.365451in}}%
\pgfpathlineto{\pgfqpoint{0.329137in}{1.386675in}}%
\pgfpathlineto{\pgfqpoint{0.336178in}{1.398512in}}%
\pgfpathlineto{\pgfqpoint{0.337793in}{1.417013in}}%
\pgfpathlineto{\pgfqpoint{0.331153in}{1.438209in}}%
\pgfpathlineto{\pgfqpoint{0.322344in}{1.451701in}}%
\pgfpathlineto{\pgfqpoint{0.323915in}{1.463876in}}%
\pgfpathlineto{\pgfqpoint{0.348653in}{1.493342in}}%
\pgfpathlineto{\pgfqpoint{0.349881in}{1.503394in}}%
\pgfpathlineto{\pgfqpoint{0.361028in}{1.522568in}}%
\pgfpathlineto{\pgfqpoint{0.362575in}{1.540742in}}%
\pgfpathlineto{\pgfqpoint{0.358991in}{1.545379in}}%
\pgfpathlineto{\pgfqpoint{0.365302in}{1.558630in}}%
\pgfusepath{stroke}%
\end{pgfscope}%
\begin{pgfscope}%
\pgfpathrectangle{\pgfqpoint{0.100000in}{0.100000in}}{\pgfqpoint{3.007045in}{1.925000in}}%
\pgfusepath{clip}%
\pgfsetbuttcap%
\pgfsetroundjoin%
\pgfsetlinewidth{0.050187pt}%
\definecolor{currentstroke}{rgb}{1.000000,1.000000,1.000000}%
\pgfsetstrokecolor{currentstroke}%
\pgfsetdash{}{0pt}%
\pgfpathmoveto{\pgfqpoint{2.291994in}{1.168959in}}%
\pgfpathlineto{\pgfqpoint{2.283623in}{1.250907in}}%
\pgfpathlineto{\pgfqpoint{2.273898in}{1.338979in}}%
\pgfpathlineto{\pgfqpoint{2.337587in}{1.348464in}}%
\pgfpathlineto{\pgfqpoint{2.354491in}{1.344065in}}%
\pgfpathlineto{\pgfqpoint{2.362598in}{1.339289in}}%
\pgfpathlineto{\pgfqpoint{2.372757in}{1.340614in}}%
\pgfpathlineto{\pgfqpoint{2.386151in}{1.332707in}}%
\pgfpathlineto{\pgfqpoint{2.411100in}{1.344475in}}%
\pgfpathlineto{\pgfqpoint{2.424868in}{1.344868in}}%
\pgfpathlineto{\pgfqpoint{2.440946in}{1.362805in}}%
\pgfpathlineto{\pgfqpoint{2.457305in}{1.373716in}}%
\pgfpathlineto{\pgfqpoint{2.479125in}{1.386284in}}%
\pgfpathlineto{\pgfqpoint{2.493170in}{1.298432in}}%
\pgfpathlineto{\pgfqpoint{2.486644in}{1.292790in}}%
\pgfpathlineto{\pgfqpoint{2.490858in}{1.287608in}}%
\pgfpathlineto{\pgfqpoint{2.492536in}{1.276247in}}%
\pgfpathlineto{\pgfqpoint{2.489267in}{1.265577in}}%
\pgfpathlineto{\pgfqpoint{2.488835in}{1.249980in}}%
\pgfpathlineto{\pgfqpoint{2.485764in}{1.229680in}}%
\pgfpathlineto{\pgfqpoint{2.470230in}{1.211575in}}%
\pgfpathlineto{\pgfqpoint{2.463690in}{1.207732in}}%
\pgfpathlineto{\pgfqpoint{2.458605in}{1.211041in}}%
\pgfpathlineto{\pgfqpoint{2.446020in}{1.193784in}}%
\pgfpathlineto{\pgfqpoint{2.447199in}{1.177168in}}%
\pgfpathlineto{\pgfqpoint{2.433183in}{1.181164in}}%
\pgfpathlineto{\pgfqpoint{2.426546in}{1.164570in}}%
\pgfpathlineto{\pgfqpoint{2.429901in}{1.152811in}}%
\pgfpathlineto{\pgfqpoint{2.424649in}{1.151074in}}%
\pgfpathlineto{\pgfqpoint{2.423914in}{1.141782in}}%
\pgfpathlineto{\pgfqpoint{2.411023in}{1.138034in}}%
\pgfpathlineto{\pgfqpoint{2.404325in}{1.145536in}}%
\pgfpathlineto{\pgfqpoint{2.395706in}{1.148445in}}%
\pgfpathlineto{\pgfqpoint{2.393680in}{1.156052in}}%
\pgfpathlineto{\pgfqpoint{2.385887in}{1.154714in}}%
\pgfpathlineto{\pgfqpoint{2.380782in}{1.147717in}}%
\pgfpathlineto{\pgfqpoint{2.373478in}{1.145257in}}%
\pgfpathlineto{\pgfqpoint{2.360576in}{1.150208in}}%
\pgfpathlineto{\pgfqpoint{2.353440in}{1.144317in}}%
\pgfpathlineto{\pgfqpoint{2.343298in}{1.151287in}}%
\pgfpathlineto{\pgfqpoint{2.323842in}{1.153426in}}%
\pgfpathlineto{\pgfqpoint{2.317973in}{1.166088in}}%
\pgfpathlineto{\pgfqpoint{2.310538in}{1.171698in}}%
\pgfpathlineto{\pgfqpoint{2.303330in}{1.168107in}}%
\pgfpathlineto{\pgfqpoint{2.291994in}{1.168959in}}%
\pgfusepath{stroke}%
\end{pgfscope}%
\begin{pgfscope}%
\pgfpathrectangle{\pgfqpoint{0.100000in}{0.100000in}}{\pgfqpoint{3.007045in}{1.925000in}}%
\pgfusepath{clip}%
\pgfsetbuttcap%
\pgfsetroundjoin%
\pgfsetlinewidth{0.050187pt}%
\definecolor{currentstroke}{rgb}{1.000000,1.000000,1.000000}%
\pgfsetstrokecolor{currentstroke}%
\pgfsetdash{}{0pt}%
\pgfpathmoveto{\pgfqpoint{2.081137in}{1.010475in}}%
\pgfpathlineto{\pgfqpoint{2.073675in}{1.016537in}}%
\pgfpathlineto{\pgfqpoint{2.067590in}{1.013653in}}%
\pgfpathlineto{\pgfqpoint{2.059812in}{1.028177in}}%
\pgfpathlineto{\pgfqpoint{2.063786in}{1.037307in}}%
\pgfpathlineto{\pgfqpoint{2.058058in}{1.047582in}}%
\pgfpathlineto{\pgfqpoint{2.057751in}{1.055672in}}%
\pgfpathlineto{\pgfqpoint{2.050012in}{1.058555in}}%
\pgfpathlineto{\pgfqpoint{2.046423in}{1.064650in}}%
\pgfpathlineto{\pgfqpoint{2.031291in}{1.072263in}}%
\pgfpathlineto{\pgfqpoint{2.018150in}{1.081642in}}%
\pgfpathlineto{\pgfqpoint{2.012044in}{1.088723in}}%
\pgfpathlineto{\pgfqpoint{2.011918in}{1.097359in}}%
\pgfpathlineto{\pgfqpoint{2.020088in}{1.113940in}}%
\pgfpathlineto{\pgfqpoint{2.022602in}{1.126623in}}%
\pgfpathlineto{\pgfqpoint{2.015945in}{1.133789in}}%
\pgfpathlineto{\pgfqpoint{2.007129in}{1.136489in}}%
\pgfpathlineto{\pgfqpoint{1.996433in}{1.130585in}}%
\pgfpathlineto{\pgfqpoint{1.991906in}{1.140721in}}%
\pgfpathlineto{\pgfqpoint{1.989621in}{1.154462in}}%
\pgfpathlineto{\pgfqpoint{1.973858in}{1.166711in}}%
\pgfpathlineto{\pgfqpoint{1.970708in}{1.172146in}}%
\pgfpathlineto{\pgfqpoint{1.956311in}{1.184451in}}%
\pgfpathlineto{\pgfqpoint{1.951798in}{1.193397in}}%
\pgfpathlineto{\pgfqpoint{1.947730in}{1.211138in}}%
\pgfpathlineto{\pgfqpoint{1.950519in}{1.226895in}}%
\pgfpathlineto{\pgfqpoint{1.954240in}{1.229091in}}%
\pgfpathlineto{\pgfqpoint{1.953571in}{1.242282in}}%
\pgfpathlineto{\pgfqpoint{1.964076in}{1.246200in}}%
\pgfpathlineto{\pgfqpoint{1.967249in}{1.258037in}}%
\pgfpathlineto{\pgfqpoint{1.973292in}{1.266004in}}%
\pgfpathlineto{\pgfqpoint{1.972987in}{1.276140in}}%
\pgfpathlineto{\pgfqpoint{1.965496in}{1.284201in}}%
\pgfpathlineto{\pgfqpoint{1.967275in}{1.295485in}}%
\pgfpathlineto{\pgfqpoint{1.986693in}{1.300392in}}%
\pgfpathlineto{\pgfqpoint{2.001591in}{1.309350in}}%
\pgfpathlineto{\pgfqpoint{2.003154in}{1.320629in}}%
\pgfpathlineto{\pgfqpoint{2.008338in}{1.324180in}}%
\pgfpathlineto{\pgfqpoint{2.010321in}{1.336029in}}%
\pgfpathlineto{\pgfqpoint{2.008727in}{1.343855in}}%
\pgfpathlineto{\pgfqpoint{1.998543in}{1.350356in}}%
\pgfpathlineto{\pgfqpoint{1.994442in}{1.360047in}}%
\pgfpathlineto{\pgfqpoint{1.984391in}{1.369415in}}%
\pgfpathlineto{\pgfqpoint{2.066992in}{1.372935in}}%
\pgfpathlineto{\pgfqpoint{2.122408in}{1.376872in}}%
\pgfpathlineto{\pgfqpoint{2.121395in}{1.365215in}}%
\pgfpathlineto{\pgfqpoint{2.130839in}{1.349136in}}%
\pgfpathlineto{\pgfqpoint{2.134804in}{1.335398in}}%
\pgfpathlineto{\pgfqpoint{2.139527in}{1.327598in}}%
\pgfpathlineto{\pgfqpoint{2.144682in}{1.263329in}}%
\pgfpathlineto{\pgfqpoint{2.151841in}{1.171716in}}%
\pgfpathlineto{\pgfqpoint{2.146318in}{1.157795in}}%
\pgfpathlineto{\pgfqpoint{2.154275in}{1.146352in}}%
\pgfpathlineto{\pgfqpoint{2.156492in}{1.135423in}}%
\pgfpathlineto{\pgfqpoint{2.145520in}{1.112760in}}%
\pgfpathlineto{\pgfqpoint{2.146167in}{1.109630in}}%
\pgfpathlineto{\pgfqpoint{2.134873in}{1.096640in}}%
\pgfpathlineto{\pgfqpoint{2.137970in}{1.092672in}}%
\pgfpathlineto{\pgfqpoint{2.133271in}{1.084575in}}%
\pgfpathlineto{\pgfqpoint{2.134480in}{1.068255in}}%
\pgfpathlineto{\pgfqpoint{2.128773in}{1.058240in}}%
\pgfpathlineto{\pgfqpoint{2.133415in}{1.046407in}}%
\pgfpathlineto{\pgfqpoint{2.113967in}{1.039976in}}%
\pgfpathlineto{\pgfqpoint{2.112175in}{1.032983in}}%
\pgfpathlineto{\pgfqpoint{2.117486in}{1.024119in}}%
\pgfpathlineto{\pgfqpoint{2.115042in}{1.018342in}}%
\pgfpathlineto{\pgfqpoint{2.094189in}{1.025479in}}%
\pgfpathlineto{\pgfqpoint{2.087308in}{1.026197in}}%
\pgfpathlineto{\pgfqpoint{2.078730in}{1.015347in}}%
\pgfpathlineto{\pgfqpoint{2.081137in}{1.010475in}}%
\pgfusepath{stroke}%
\end{pgfscope}%
\begin{pgfscope}%
\pgfpathrectangle{\pgfqpoint{0.100000in}{0.100000in}}{\pgfqpoint{3.007045in}{1.925000in}}%
\pgfusepath{clip}%
\pgfsetbuttcap%
\pgfsetroundjoin%
\pgfsetlinewidth{0.050187pt}%
\definecolor{currentstroke}{rgb}{1.000000,1.000000,1.000000}%
\pgfsetstrokecolor{currentstroke}%
\pgfsetdash{}{0pt}%
\pgfpathmoveto{\pgfqpoint{2.688153in}{1.209860in}}%
\pgfpathlineto{\pgfqpoint{2.682414in}{1.218382in}}%
\pgfpathlineto{\pgfqpoint{2.685652in}{1.223151in}}%
\pgfpathlineto{\pgfqpoint{2.693583in}{1.217795in}}%
\pgfpathlineto{\pgfqpoint{2.688153in}{1.209860in}}%
\pgfusepath{stroke}%
\end{pgfscope}%
\begin{pgfscope}%
\pgfpathrectangle{\pgfqpoint{0.100000in}{0.100000in}}{\pgfqpoint{3.007045in}{1.925000in}}%
\pgfusepath{clip}%
\pgfsetbuttcap%
\pgfsetroundjoin%
\pgfsetlinewidth{0.050187pt}%
\definecolor{currentstroke}{rgb}{1.000000,1.000000,1.000000}%
\pgfsetstrokecolor{currentstroke}%
\pgfsetdash{}{0pt}%
\pgfpathmoveto{\pgfqpoint{2.738266in}{1.283189in}}%
\pgfpathlineto{\pgfqpoint{2.741696in}{1.290435in}}%
\pgfpathlineto{\pgfqpoint{2.755560in}{1.292000in}}%
\pgfpathlineto{\pgfqpoint{2.753338in}{1.285842in}}%
\pgfpathlineto{\pgfqpoint{2.748764in}{1.277976in}}%
\pgfpathlineto{\pgfqpoint{2.751863in}{1.268604in}}%
\pgfpathlineto{\pgfqpoint{2.764095in}{1.257374in}}%
\pgfpathlineto{\pgfqpoint{2.766934in}{1.245549in}}%
\pgfpathlineto{\pgfqpoint{2.781027in}{1.230823in}}%
\pgfpathlineto{\pgfqpoint{2.786556in}{1.231453in}}%
\pgfpathlineto{\pgfqpoint{2.793443in}{1.209347in}}%
\pgfpathlineto{\pgfqpoint{2.792313in}{1.209127in}}%
\pgfpathlineto{\pgfqpoint{2.791057in}{1.208880in}}%
\pgfpathlineto{\pgfqpoint{2.760342in}{1.203000in}}%
\pgfpathlineto{\pgfqpoint{2.743908in}{1.261445in}}%
\pgfpathlineto{\pgfqpoint{2.738266in}{1.283189in}}%
\pgfusepath{stroke}%
\end{pgfscope}%
\begin{pgfscope}%
\pgfpathrectangle{\pgfqpoint{0.100000in}{0.100000in}}{\pgfqpoint{3.007045in}{1.925000in}}%
\pgfusepath{clip}%
\pgfsetbuttcap%
\pgfsetroundjoin%
\pgfsetlinewidth{0.050187pt}%
\definecolor{currentstroke}{rgb}{1.000000,1.000000,1.000000}%
\pgfsetstrokecolor{currentstroke}%
\pgfsetdash{}{0pt}%
\pgfpathmoveto{\pgfqpoint{2.451373in}{1.085086in}}%
\pgfpathlineto{\pgfqpoint{2.441667in}{1.085443in}}%
\pgfpathlineto{\pgfqpoint{2.432723in}{1.091604in}}%
\pgfpathlineto{\pgfqpoint{2.420641in}{1.110328in}}%
\pgfpathlineto{\pgfqpoint{2.410356in}{1.120244in}}%
\pgfpathlineto{\pgfqpoint{2.413044in}{1.127903in}}%
\pgfpathlineto{\pgfqpoint{2.411023in}{1.138034in}}%
\pgfpathlineto{\pgfqpoint{2.423914in}{1.141782in}}%
\pgfpathlineto{\pgfqpoint{2.424649in}{1.151074in}}%
\pgfpathlineto{\pgfqpoint{2.429901in}{1.152811in}}%
\pgfpathlineto{\pgfqpoint{2.426546in}{1.164570in}}%
\pgfpathlineto{\pgfqpoint{2.433183in}{1.181164in}}%
\pgfpathlineto{\pgfqpoint{2.447199in}{1.177168in}}%
\pgfpathlineto{\pgfqpoint{2.446020in}{1.193784in}}%
\pgfpathlineto{\pgfqpoint{2.458605in}{1.211041in}}%
\pgfpathlineto{\pgfqpoint{2.463690in}{1.207732in}}%
\pgfpathlineto{\pgfqpoint{2.470230in}{1.211575in}}%
\pgfpathlineto{\pgfqpoint{2.485764in}{1.229680in}}%
\pgfpathlineto{\pgfqpoint{2.488835in}{1.249980in}}%
\pgfpathlineto{\pgfqpoint{2.489267in}{1.265577in}}%
\pgfpathlineto{\pgfqpoint{2.492536in}{1.276247in}}%
\pgfpathlineto{\pgfqpoint{2.490858in}{1.287608in}}%
\pgfpathlineto{\pgfqpoint{2.486644in}{1.292790in}}%
\pgfpathlineto{\pgfqpoint{2.493170in}{1.298432in}}%
\pgfpathlineto{\pgfqpoint{2.502630in}{1.238834in}}%
\pgfpathlineto{\pgfqpoint{2.554781in}{1.247456in}}%
\pgfpathlineto{\pgfqpoint{2.560186in}{1.213439in}}%
\pgfpathlineto{\pgfqpoint{2.579067in}{1.235892in}}%
\pgfpathlineto{\pgfqpoint{2.583515in}{1.233653in}}%
\pgfpathlineto{\pgfqpoint{2.590060in}{1.246663in}}%
\pgfpathlineto{\pgfqpoint{2.599126in}{1.242584in}}%
\pgfpathlineto{\pgfqpoint{2.607031in}{1.243360in}}%
\pgfpathlineto{\pgfqpoint{2.612246in}{1.252404in}}%
\pgfpathlineto{\pgfqpoint{2.619817in}{1.257435in}}%
\pgfpathlineto{\pgfqpoint{2.630342in}{1.253004in}}%
\pgfpathlineto{\pgfqpoint{2.637267in}{1.254535in}}%
\pgfpathlineto{\pgfqpoint{2.647188in}{1.237365in}}%
\pgfpathlineto{\pgfqpoint{2.644319in}{1.224366in}}%
\pgfpathlineto{\pgfqpoint{2.618406in}{1.239002in}}%
\pgfpathlineto{\pgfqpoint{2.613553in}{1.226987in}}%
\pgfpathlineto{\pgfqpoint{2.615154in}{1.221459in}}%
\pgfpathlineto{\pgfqpoint{2.604125in}{1.202772in}}%
\pgfpathlineto{\pgfqpoint{2.598920in}{1.199055in}}%
\pgfpathlineto{\pgfqpoint{2.596570in}{1.190815in}}%
\pgfpathlineto{\pgfqpoint{2.587696in}{1.191677in}}%
\pgfpathlineto{\pgfqpoint{2.581324in}{1.169185in}}%
\pgfpathlineto{\pgfqpoint{2.577757in}{1.164024in}}%
\pgfpathlineto{\pgfqpoint{2.568590in}{1.165745in}}%
\pgfpathlineto{\pgfqpoint{2.559215in}{1.172839in}}%
\pgfpathlineto{\pgfqpoint{2.558890in}{1.161951in}}%
\pgfpathlineto{\pgfqpoint{2.549814in}{1.143653in}}%
\pgfpathlineto{\pgfqpoint{2.547556in}{1.130626in}}%
\pgfpathlineto{\pgfqpoint{2.540593in}{1.121950in}}%
\pgfpathlineto{\pgfqpoint{2.535313in}{1.108096in}}%
\pgfpathlineto{\pgfqpoint{2.535053in}{1.095270in}}%
\pgfpathlineto{\pgfqpoint{2.526900in}{1.092874in}}%
\pgfpathlineto{\pgfqpoint{2.517650in}{1.085614in}}%
\pgfpathlineto{\pgfqpoint{2.508744in}{1.084231in}}%
\pgfpathlineto{\pgfqpoint{2.506692in}{1.078086in}}%
\pgfpathlineto{\pgfqpoint{2.492341in}{1.071792in}}%
\pgfpathlineto{\pgfqpoint{2.484322in}{1.077155in}}%
\pgfpathlineto{\pgfqpoint{2.475358in}{1.066973in}}%
\pgfpathlineto{\pgfqpoint{2.459950in}{1.069974in}}%
\pgfpathlineto{\pgfqpoint{2.450500in}{1.080661in}}%
\pgfpathlineto{\pgfqpoint{2.451373in}{1.085086in}}%
\pgfusepath{stroke}%
\end{pgfscope}%
\begin{pgfscope}%
\pgfpathrectangle{\pgfqpoint{0.100000in}{0.100000in}}{\pgfqpoint{3.007045in}{1.925000in}}%
\pgfusepath{clip}%
\pgfsetbuttcap%
\pgfsetroundjoin%
\pgfsetlinewidth{0.050187pt}%
\definecolor{currentstroke}{rgb}{1.000000,1.000000,1.000000}%
\pgfsetstrokecolor{currentstroke}%
\pgfsetdash{}{0pt}%
\pgfpathmoveto{\pgfqpoint{2.791057in}{1.208880in}}%
\pgfpathlineto{\pgfqpoint{2.790664in}{1.196870in}}%
\pgfpathlineto{\pgfqpoint{2.786042in}{1.190966in}}%
\pgfpathlineto{\pgfqpoint{2.783092in}{1.177859in}}%
\pgfpathlineto{\pgfqpoint{2.769772in}{1.171817in}}%
\pgfpathlineto{\pgfqpoint{2.758597in}{1.170057in}}%
\pgfpathlineto{\pgfqpoint{2.761853in}{1.178677in}}%
\pgfpathlineto{\pgfqpoint{2.744646in}{1.185900in}}%
\pgfpathlineto{\pgfqpoint{2.730667in}{1.194944in}}%
\pgfpathlineto{\pgfqpoint{2.739321in}{1.208462in}}%
\pgfpathlineto{\pgfqpoint{2.732385in}{1.218971in}}%
\pgfpathlineto{\pgfqpoint{2.733202in}{1.228035in}}%
\pgfpathlineto{\pgfqpoint{2.728187in}{1.230522in}}%
\pgfpathlineto{\pgfqpoint{2.724117in}{1.245248in}}%
\pgfpathlineto{\pgfqpoint{2.733741in}{1.265210in}}%
\pgfpathlineto{\pgfqpoint{2.726510in}{1.268470in}}%
\pgfpathlineto{\pgfqpoint{2.724637in}{1.258622in}}%
\pgfpathlineto{\pgfqpoint{2.714315in}{1.255880in}}%
\pgfpathlineto{\pgfqpoint{2.715401in}{1.223478in}}%
\pgfpathlineto{\pgfqpoint{2.713513in}{1.213050in}}%
\pgfpathlineto{\pgfqpoint{2.718690in}{1.198187in}}%
\pgfpathlineto{\pgfqpoint{2.726654in}{1.191036in}}%
\pgfpathlineto{\pgfqpoint{2.723043in}{1.186546in}}%
\pgfpathlineto{\pgfqpoint{2.731172in}{1.179985in}}%
\pgfpathlineto{\pgfqpoint{2.734111in}{1.169336in}}%
\pgfpathlineto{\pgfqpoint{2.719210in}{1.178155in}}%
\pgfpathlineto{\pgfqpoint{2.709780in}{1.177009in}}%
\pgfpathlineto{\pgfqpoint{2.702458in}{1.186080in}}%
\pgfpathlineto{\pgfqpoint{2.695002in}{1.186964in}}%
\pgfpathlineto{\pgfqpoint{2.684407in}{1.182420in}}%
\pgfpathlineto{\pgfqpoint{2.680297in}{1.188082in}}%
\pgfpathlineto{\pgfqpoint{2.685698in}{1.199967in}}%
\pgfpathlineto{\pgfqpoint{2.688153in}{1.209860in}}%
\pgfpathlineto{\pgfqpoint{2.693583in}{1.217795in}}%
\pgfpathlineto{\pgfqpoint{2.685652in}{1.223151in}}%
\pgfpathlineto{\pgfqpoint{2.682414in}{1.218382in}}%
\pgfpathlineto{\pgfqpoint{2.674491in}{1.223222in}}%
\pgfpathlineto{\pgfqpoint{2.660462in}{1.226448in}}%
\pgfpathlineto{\pgfqpoint{2.661734in}{1.233539in}}%
\pgfpathlineto{\pgfqpoint{2.655375in}{1.237653in}}%
\pgfpathlineto{\pgfqpoint{2.647188in}{1.237365in}}%
\pgfpathlineto{\pgfqpoint{2.637267in}{1.254535in}}%
\pgfpathlineto{\pgfqpoint{2.630342in}{1.253004in}}%
\pgfpathlineto{\pgfqpoint{2.619817in}{1.257435in}}%
\pgfpathlineto{\pgfqpoint{2.612246in}{1.252404in}}%
\pgfpathlineto{\pgfqpoint{2.607031in}{1.243360in}}%
\pgfpathlineto{\pgfqpoint{2.599126in}{1.242584in}}%
\pgfpathlineto{\pgfqpoint{2.590060in}{1.246663in}}%
\pgfpathlineto{\pgfqpoint{2.583515in}{1.233653in}}%
\pgfpathlineto{\pgfqpoint{2.579067in}{1.235892in}}%
\pgfpathlineto{\pgfqpoint{2.560186in}{1.213439in}}%
\pgfpathlineto{\pgfqpoint{2.554781in}{1.247456in}}%
\pgfpathlineto{\pgfqpoint{2.623776in}{1.260244in}}%
\pgfpathlineto{\pgfqpoint{2.654706in}{1.265782in}}%
\pgfpathlineto{\pgfqpoint{2.699704in}{1.274845in}}%
\pgfpathlineto{\pgfqpoint{2.738266in}{1.283189in}}%
\pgfpathlineto{\pgfqpoint{2.743908in}{1.261445in}}%
\pgfpathlineto{\pgfqpoint{2.760342in}{1.203000in}}%
\pgfpathlineto{\pgfqpoint{2.791057in}{1.208880in}}%
\pgfusepath{stroke}%
\end{pgfscope}%
\begin{pgfscope}%
\pgfpathrectangle{\pgfqpoint{0.100000in}{0.100000in}}{\pgfqpoint{3.007045in}{1.925000in}}%
\pgfusepath{clip}%
\pgfsetbuttcap%
\pgfsetroundjoin%
\pgfsetlinewidth{0.050187pt}%
\definecolor{currentstroke}{rgb}{1.000000,1.000000,1.000000}%
\pgfsetstrokecolor{currentstroke}%
\pgfsetdash{}{0pt}%
\pgfpathmoveto{\pgfqpoint{1.402207in}{1.014889in}}%
\pgfpathlineto{\pgfqpoint{1.351869in}{1.019693in}}%
\pgfpathlineto{\pgfqpoint{1.299648in}{1.024321in}}%
\pgfpathlineto{\pgfqpoint{1.239303in}{1.030616in}}%
\pgfpathlineto{\pgfqpoint{1.149648in}{1.041279in}}%
\pgfpathlineto{\pgfqpoint{1.117842in}{1.046179in}}%
\pgfpathlineto{\pgfqpoint{1.035744in}{1.058103in}}%
\pgfpathlineto{\pgfqpoint{1.047732in}{1.133342in}}%
\pgfpathlineto{\pgfqpoint{1.048031in}{1.139428in}}%
\pgfpathlineto{\pgfqpoint{1.059579in}{1.212028in}}%
\pgfpathlineto{\pgfqpoint{1.068205in}{1.267320in}}%
\pgfpathlineto{\pgfqpoint{1.076345in}{1.318594in}}%
\pgfpathlineto{\pgfqpoint{1.131961in}{1.310611in}}%
\pgfpathlineto{\pgfqpoint{1.183815in}{1.303168in}}%
\pgfpathlineto{\pgfqpoint{1.279137in}{1.291403in}}%
\pgfpathlineto{\pgfqpoint{1.322867in}{1.287410in}}%
\pgfpathlineto{\pgfqpoint{1.392146in}{1.280688in}}%
\pgfpathlineto{\pgfqpoint{1.422117in}{1.278255in}}%
\pgfpathlineto{\pgfqpoint{1.416825in}{1.212637in}}%
\pgfpathlineto{\pgfqpoint{1.412045in}{1.149457in}}%
\pgfpathlineto{\pgfqpoint{1.408197in}{1.098013in}}%
\pgfpathlineto{\pgfqpoint{1.402207in}{1.014889in}}%
\pgfusepath{stroke}%
\end{pgfscope}%
\begin{pgfscope}%
\pgfpathrectangle{\pgfqpoint{0.100000in}{0.100000in}}{\pgfqpoint{3.007045in}{1.925000in}}%
\pgfusepath{clip}%
\pgfsetbuttcap%
\pgfsetroundjoin%
\pgfsetlinewidth{0.050187pt}%
\definecolor{currentstroke}{rgb}{1.000000,1.000000,1.000000}%
\pgfsetstrokecolor{currentstroke}%
\pgfsetdash{}{0pt}%
\pgfpathmoveto{\pgfqpoint{2.068136in}{0.977587in}}%
\pgfpathlineto{\pgfqpoint{2.069833in}{0.985194in}}%
\pgfpathlineto{\pgfqpoint{2.078628in}{0.983482in}}%
\pgfpathlineto{\pgfqpoint{2.081137in}{1.010475in}}%
\pgfpathlineto{\pgfqpoint{2.078730in}{1.015347in}}%
\pgfpathlineto{\pgfqpoint{2.087308in}{1.026197in}}%
\pgfpathlineto{\pgfqpoint{2.094189in}{1.025479in}}%
\pgfpathlineto{\pgfqpoint{2.115042in}{1.018342in}}%
\pgfpathlineto{\pgfqpoint{2.117486in}{1.024119in}}%
\pgfpathlineto{\pgfqpoint{2.112175in}{1.032983in}}%
\pgfpathlineto{\pgfqpoint{2.113967in}{1.039976in}}%
\pgfpathlineto{\pgfqpoint{2.133415in}{1.046407in}}%
\pgfpathlineto{\pgfqpoint{2.128773in}{1.058240in}}%
\pgfpathlineto{\pgfqpoint{2.134480in}{1.068255in}}%
\pgfpathlineto{\pgfqpoint{2.145103in}{1.073876in}}%
\pgfpathlineto{\pgfqpoint{2.167393in}{1.079442in}}%
\pgfpathlineto{\pgfqpoint{2.181457in}{1.071040in}}%
\pgfpathlineto{\pgfqpoint{2.185723in}{1.079234in}}%
\pgfpathlineto{\pgfqpoint{2.197401in}{1.084958in}}%
\pgfpathlineto{\pgfqpoint{2.200414in}{1.079910in}}%
\pgfpathlineto{\pgfqpoint{2.212410in}{1.083920in}}%
\pgfpathlineto{\pgfqpoint{2.211653in}{1.090784in}}%
\pgfpathlineto{\pgfqpoint{2.222438in}{1.098640in}}%
\pgfpathlineto{\pgfqpoint{2.228803in}{1.090482in}}%
\pgfpathlineto{\pgfqpoint{2.237177in}{1.089662in}}%
\pgfpathlineto{\pgfqpoint{2.242742in}{1.094978in}}%
\pgfpathlineto{\pgfqpoint{2.242123in}{1.102538in}}%
\pgfpathlineto{\pgfqpoint{2.246853in}{1.110045in}}%
\pgfpathlineto{\pgfqpoint{2.253191in}{1.111671in}}%
\pgfpathlineto{\pgfqpoint{2.255736in}{1.121584in}}%
\pgfpathlineto{\pgfqpoint{2.264955in}{1.130153in}}%
\pgfpathlineto{\pgfqpoint{2.262180in}{1.138682in}}%
\pgfpathlineto{\pgfqpoint{2.271148in}{1.142929in}}%
\pgfpathlineto{\pgfqpoint{2.277140in}{1.140317in}}%
\pgfpathlineto{\pgfqpoint{2.285991in}{1.146952in}}%
\pgfpathlineto{\pgfqpoint{2.293900in}{1.148681in}}%
\pgfpathlineto{\pgfqpoint{2.293928in}{1.155563in}}%
\pgfpathlineto{\pgfqpoint{2.288386in}{1.165112in}}%
\pgfpathlineto{\pgfqpoint{2.291994in}{1.168959in}}%
\pgfpathlineto{\pgfqpoint{2.303330in}{1.168107in}}%
\pgfpathlineto{\pgfqpoint{2.310538in}{1.171698in}}%
\pgfpathlineto{\pgfqpoint{2.317973in}{1.166088in}}%
\pgfpathlineto{\pgfqpoint{2.323842in}{1.153426in}}%
\pgfpathlineto{\pgfqpoint{2.343298in}{1.151287in}}%
\pgfpathlineto{\pgfqpoint{2.353440in}{1.144317in}}%
\pgfpathlineto{\pgfqpoint{2.360576in}{1.150208in}}%
\pgfpathlineto{\pgfqpoint{2.373478in}{1.145257in}}%
\pgfpathlineto{\pgfqpoint{2.380782in}{1.147717in}}%
\pgfpathlineto{\pgfqpoint{2.385887in}{1.154714in}}%
\pgfpathlineto{\pgfqpoint{2.393680in}{1.156052in}}%
\pgfpathlineto{\pgfqpoint{2.395706in}{1.148445in}}%
\pgfpathlineto{\pgfqpoint{2.404325in}{1.145536in}}%
\pgfpathlineto{\pgfqpoint{2.411023in}{1.138034in}}%
\pgfpathlineto{\pgfqpoint{2.413044in}{1.127903in}}%
\pgfpathlineto{\pgfqpoint{2.410356in}{1.120244in}}%
\pgfpathlineto{\pgfqpoint{2.420641in}{1.110328in}}%
\pgfpathlineto{\pgfqpoint{2.432723in}{1.091604in}}%
\pgfpathlineto{\pgfqpoint{2.441667in}{1.085443in}}%
\pgfpathlineto{\pgfqpoint{2.451373in}{1.085086in}}%
\pgfpathlineto{\pgfqpoint{2.433451in}{1.064522in}}%
\pgfpathlineto{\pgfqpoint{2.415810in}{1.052075in}}%
\pgfpathlineto{\pgfqpoint{2.409317in}{1.042191in}}%
\pgfpathlineto{\pgfqpoint{2.409427in}{1.036826in}}%
\pgfpathlineto{\pgfqpoint{2.399872in}{1.032715in}}%
\pgfpathlineto{\pgfqpoint{2.397134in}{1.024978in}}%
\pgfpathlineto{\pgfqpoint{2.377229in}{1.017186in}}%
\pgfpathlineto{\pgfqpoint{2.370173in}{1.012125in}}%
\pgfpathlineto{\pgfqpoint{2.369218in}{1.011045in}}%
\pgfpathlineto{\pgfqpoint{2.311961in}{1.005616in}}%
\pgfpathlineto{\pgfqpoint{2.277384in}{1.002708in}}%
\pgfpathlineto{\pgfqpoint{2.220644in}{0.999497in}}%
\pgfpathlineto{\pgfqpoint{2.149956in}{0.992504in}}%
\pgfpathlineto{\pgfqpoint{2.138278in}{0.994121in}}%
\pgfpathlineto{\pgfqpoint{2.140708in}{0.982200in}}%
\pgfpathlineto{\pgfqpoint{2.068136in}{0.977587in}}%
\pgfusepath{stroke}%
\end{pgfscope}%
\begin{pgfscope}%
\pgfpathrectangle{\pgfqpoint{0.100000in}{0.100000in}}{\pgfqpoint{3.007045in}{1.925000in}}%
\pgfusepath{clip}%
\pgfsetbuttcap%
\pgfsetroundjoin%
\pgfsetlinewidth{0.050187pt}%
\definecolor{currentstroke}{rgb}{1.000000,1.000000,1.000000}%
\pgfsetstrokecolor{currentstroke}%
\pgfsetdash{}{0pt}%
\pgfpathmoveto{\pgfqpoint{1.402207in}{1.014889in}}%
\pgfpathlineto{\pgfqpoint{1.408197in}{1.098013in}}%
\pgfpathlineto{\pgfqpoint{1.412045in}{1.149457in}}%
\pgfpathlineto{\pgfqpoint{1.416825in}{1.212637in}}%
\pgfpathlineto{\pgfqpoint{1.483096in}{1.207983in}}%
\pgfpathlineto{\pgfqpoint{1.567259in}{1.203386in}}%
\pgfpathlineto{\pgfqpoint{1.624502in}{1.201190in}}%
\pgfpathlineto{\pgfqpoint{1.681409in}{1.199446in}}%
\pgfpathlineto{\pgfqpoint{1.732915in}{1.198649in}}%
\pgfpathlineto{\pgfqpoint{1.756737in}{1.198896in}}%
\pgfpathlineto{\pgfqpoint{1.767223in}{1.190339in}}%
\pgfpathlineto{\pgfqpoint{1.775436in}{1.192065in}}%
\pgfpathlineto{\pgfqpoint{1.775693in}{1.184601in}}%
\pgfpathlineto{\pgfqpoint{1.769593in}{1.171690in}}%
\pgfpathlineto{\pgfqpoint{1.770255in}{1.163533in}}%
\pgfpathlineto{\pgfqpoint{1.776095in}{1.158154in}}%
\pgfpathlineto{\pgfqpoint{1.781459in}{1.146947in}}%
\pgfpathlineto{\pgfqpoint{1.792341in}{1.140513in}}%
\pgfpathlineto{\pgfqpoint{1.791978in}{1.098269in}}%
\pgfpathlineto{\pgfqpoint{1.792298in}{1.001125in}}%
\pgfpathlineto{\pgfqpoint{1.719359in}{1.001461in}}%
\pgfpathlineto{\pgfqpoint{1.642551in}{1.002805in}}%
\pgfpathlineto{\pgfqpoint{1.586009in}{1.004800in}}%
\pgfpathlineto{\pgfqpoint{1.538812in}{1.006621in}}%
\pgfpathlineto{\pgfqpoint{1.459299in}{1.011309in}}%
\pgfpathlineto{\pgfqpoint{1.402207in}{1.014889in}}%
\pgfusepath{stroke}%
\end{pgfscope}%
\begin{pgfscope}%
\pgfpathrectangle{\pgfqpoint{0.100000in}{0.100000in}}{\pgfqpoint{3.007045in}{1.925000in}}%
\pgfusepath{clip}%
\pgfsetbuttcap%
\pgfsetroundjoin%
\pgfsetlinewidth{0.050187pt}%
\definecolor{currentstroke}{rgb}{1.000000,1.000000,1.000000}%
\pgfsetstrokecolor{currentstroke}%
\pgfsetdash{}{0pt}%
\pgfpathmoveto{\pgfqpoint{2.475422in}{1.025896in}}%
\pgfpathlineto{\pgfqpoint{2.394637in}{1.014547in}}%
\pgfpathlineto{\pgfqpoint{2.370173in}{1.012125in}}%
\pgfpathlineto{\pgfqpoint{2.377229in}{1.017186in}}%
\pgfpathlineto{\pgfqpoint{2.397134in}{1.024978in}}%
\pgfpathlineto{\pgfqpoint{2.399872in}{1.032715in}}%
\pgfpathlineto{\pgfqpoint{2.409427in}{1.036826in}}%
\pgfpathlineto{\pgfqpoint{2.409317in}{1.042191in}}%
\pgfpathlineto{\pgfqpoint{2.415810in}{1.052075in}}%
\pgfpathlineto{\pgfqpoint{2.433451in}{1.064522in}}%
\pgfpathlineto{\pgfqpoint{2.451373in}{1.085086in}}%
\pgfpathlineto{\pgfqpoint{2.450500in}{1.080661in}}%
\pgfpathlineto{\pgfqpoint{2.459950in}{1.069974in}}%
\pgfpathlineto{\pgfqpoint{2.475358in}{1.066973in}}%
\pgfpathlineto{\pgfqpoint{2.484322in}{1.077155in}}%
\pgfpathlineto{\pgfqpoint{2.492341in}{1.071792in}}%
\pgfpathlineto{\pgfqpoint{2.506692in}{1.078086in}}%
\pgfpathlineto{\pgfqpoint{2.508744in}{1.084231in}}%
\pgfpathlineto{\pgfqpoint{2.517650in}{1.085614in}}%
\pgfpathlineto{\pgfqpoint{2.526900in}{1.092874in}}%
\pgfpathlineto{\pgfqpoint{2.535053in}{1.095270in}}%
\pgfpathlineto{\pgfqpoint{2.535313in}{1.108096in}}%
\pgfpathlineto{\pgfqpoint{2.540593in}{1.121950in}}%
\pgfpathlineto{\pgfqpoint{2.547556in}{1.130626in}}%
\pgfpathlineto{\pgfqpoint{2.549814in}{1.143653in}}%
\pgfpathlineto{\pgfqpoint{2.558890in}{1.161951in}}%
\pgfpathlineto{\pgfqpoint{2.559215in}{1.172839in}}%
\pgfpathlineto{\pgfqpoint{2.568590in}{1.165745in}}%
\pgfpathlineto{\pgfqpoint{2.577757in}{1.164024in}}%
\pgfpathlineto{\pgfqpoint{2.581324in}{1.169185in}}%
\pgfpathlineto{\pgfqpoint{2.587696in}{1.191677in}}%
\pgfpathlineto{\pgfqpoint{2.596570in}{1.190815in}}%
\pgfpathlineto{\pgfqpoint{2.598920in}{1.199055in}}%
\pgfpathlineto{\pgfqpoint{2.604125in}{1.202772in}}%
\pgfpathlineto{\pgfqpoint{2.615154in}{1.221459in}}%
\pgfpathlineto{\pgfqpoint{2.613553in}{1.226987in}}%
\pgfpathlineto{\pgfqpoint{2.618406in}{1.239002in}}%
\pgfpathlineto{\pgfqpoint{2.644319in}{1.224366in}}%
\pgfpathlineto{\pgfqpoint{2.647188in}{1.237365in}}%
\pgfpathlineto{\pgfqpoint{2.655375in}{1.237653in}}%
\pgfpathlineto{\pgfqpoint{2.661734in}{1.233539in}}%
\pgfpathlineto{\pgfqpoint{2.660462in}{1.226448in}}%
\pgfpathlineto{\pgfqpoint{2.674491in}{1.223222in}}%
\pgfpathlineto{\pgfqpoint{2.682414in}{1.218382in}}%
\pgfpathlineto{\pgfqpoint{2.688153in}{1.209860in}}%
\pgfpathlineto{\pgfqpoint{2.685698in}{1.199967in}}%
\pgfpathlineto{\pgfqpoint{2.680740in}{1.199156in}}%
\pgfpathlineto{\pgfqpoint{2.677865in}{1.184228in}}%
\pgfpathlineto{\pgfqpoint{2.684163in}{1.178391in}}%
\pgfpathlineto{\pgfqpoint{2.693030in}{1.183109in}}%
\pgfpathlineto{\pgfqpoint{2.701257in}{1.173146in}}%
\pgfpathlineto{\pgfqpoint{2.719638in}{1.171355in}}%
\pgfpathlineto{\pgfqpoint{2.722805in}{1.165624in}}%
\pgfpathlineto{\pgfqpoint{2.739815in}{1.160050in}}%
\pgfpathlineto{\pgfqpoint{2.737716in}{1.153473in}}%
\pgfpathlineto{\pgfqpoint{2.739929in}{1.134884in}}%
\pgfpathlineto{\pgfqpoint{2.746795in}{1.127865in}}%
\pgfpathlineto{\pgfqpoint{2.736664in}{1.124902in}}%
\pgfpathlineto{\pgfqpoint{2.738007in}{1.116967in}}%
\pgfpathlineto{\pgfqpoint{2.748815in}{1.110238in}}%
\pgfpathlineto{\pgfqpoint{2.749789in}{1.103601in}}%
\pgfpathlineto{\pgfqpoint{2.743687in}{1.098614in}}%
\pgfpathlineto{\pgfqpoint{2.731375in}{1.110429in}}%
\pgfpathlineto{\pgfqpoint{2.730158in}{1.101808in}}%
\pgfpathlineto{\pgfqpoint{2.740465in}{1.097709in}}%
\pgfpathlineto{\pgfqpoint{2.741496in}{1.093469in}}%
\pgfpathlineto{\pgfqpoint{2.755633in}{1.099072in}}%
\pgfpathlineto{\pgfqpoint{2.766466in}{1.100555in}}%
\pgfpathlineto{\pgfqpoint{2.777563in}{1.078165in}}%
\pgfpathlineto{\pgfqpoint{2.776325in}{1.077922in}}%
\pgfpathlineto{\pgfqpoint{2.771311in}{1.076886in}}%
\pgfpathlineto{\pgfqpoint{2.769833in}{1.076577in}}%
\pgfpathlineto{\pgfqpoint{2.768856in}{1.076386in}}%
\pgfpathlineto{\pgfqpoint{2.702785in}{1.062681in}}%
\pgfpathlineto{\pgfqpoint{2.643688in}{1.050585in}}%
\pgfpathlineto{\pgfqpoint{2.561981in}{1.036510in}}%
\pgfpathlineto{\pgfqpoint{2.492587in}{1.027345in}}%
\pgfpathlineto{\pgfqpoint{2.475422in}{1.025896in}}%
\pgfusepath{stroke}%
\end{pgfscope}%
\begin{pgfscope}%
\pgfpathrectangle{\pgfqpoint{0.100000in}{0.100000in}}{\pgfqpoint{3.007045in}{1.925000in}}%
\pgfusepath{clip}%
\pgfsetbuttcap%
\pgfsetroundjoin%
\pgfsetlinewidth{0.050187pt}%
\definecolor{currentstroke}{rgb}{1.000000,1.000000,1.000000}%
\pgfsetstrokecolor{currentstroke}%
\pgfsetdash{}{0pt}%
\pgfpathmoveto{\pgfqpoint{2.769772in}{1.171817in}}%
\pgfpathlineto{\pgfqpoint{2.783092in}{1.177859in}}%
\pgfpathlineto{\pgfqpoint{2.772455in}{1.146743in}}%
\pgfpathlineto{\pgfqpoint{2.765421in}{1.130265in}}%
\pgfpathlineto{\pgfqpoint{2.760353in}{1.139590in}}%
\pgfpathlineto{\pgfqpoint{2.769365in}{1.161915in}}%
\pgfpathlineto{\pgfqpoint{2.769772in}{1.171817in}}%
\pgfusepath{stroke}%
\end{pgfscope}%
\begin{pgfscope}%
\pgfpathrectangle{\pgfqpoint{0.100000in}{0.100000in}}{\pgfqpoint{3.007045in}{1.925000in}}%
\pgfusepath{clip}%
\pgfsetbuttcap%
\pgfsetroundjoin%
\pgfsetlinewidth{0.050187pt}%
\definecolor{currentstroke}{rgb}{1.000000,1.000000,1.000000}%
\pgfsetstrokecolor{currentstroke}%
\pgfsetdash{}{0pt}%
\pgfpathmoveto{\pgfqpoint{2.068136in}{0.977587in}}%
\pgfpathlineto{\pgfqpoint{2.064919in}{0.977122in}}%
\pgfpathlineto{\pgfqpoint{2.061886in}{0.976909in}}%
\pgfpathlineto{\pgfqpoint{2.063184in}{0.967596in}}%
\pgfpathlineto{\pgfqpoint{2.055358in}{0.958208in}}%
\pgfpathlineto{\pgfqpoint{2.053816in}{0.943518in}}%
\pgfpathlineto{\pgfqpoint{2.018831in}{0.940931in}}%
\pgfpathlineto{\pgfqpoint{2.021881in}{0.947828in}}%
\pgfpathlineto{\pgfqpoint{2.034495in}{0.960424in}}%
\pgfpathlineto{\pgfqpoint{2.034844in}{0.967726in}}%
\pgfpathlineto{\pgfqpoint{2.029259in}{0.974639in}}%
\pgfpathlineto{\pgfqpoint{1.977188in}{0.971886in}}%
\pgfpathlineto{\pgfqpoint{1.886150in}{0.968996in}}%
\pgfpathlineto{\pgfqpoint{1.832874in}{0.968025in}}%
\pgfpathlineto{\pgfqpoint{1.792603in}{0.967662in}}%
\pgfpathlineto{\pgfqpoint{1.792298in}{1.001125in}}%
\pgfpathlineto{\pgfqpoint{1.791978in}{1.098269in}}%
\pgfpathlineto{\pgfqpoint{1.792341in}{1.140513in}}%
\pgfpathlineto{\pgfqpoint{1.781459in}{1.146947in}}%
\pgfpathlineto{\pgfqpoint{1.776095in}{1.158154in}}%
\pgfpathlineto{\pgfqpoint{1.770255in}{1.163533in}}%
\pgfpathlineto{\pgfqpoint{1.769593in}{1.171690in}}%
\pgfpathlineto{\pgfqpoint{1.775693in}{1.184601in}}%
\pgfpathlineto{\pgfqpoint{1.775436in}{1.192065in}}%
\pgfpathlineto{\pgfqpoint{1.767223in}{1.190339in}}%
\pgfpathlineto{\pgfqpoint{1.756737in}{1.198896in}}%
\pgfpathlineto{\pgfqpoint{1.748332in}{1.213916in}}%
\pgfpathlineto{\pgfqpoint{1.741281in}{1.220848in}}%
\pgfpathlineto{\pgfqpoint{1.733914in}{1.237878in}}%
\pgfpathlineto{\pgfqpoint{1.810432in}{1.236685in}}%
\pgfpathlineto{\pgfqpoint{1.886493in}{1.239191in}}%
\pgfpathlineto{\pgfqpoint{1.935262in}{1.242004in}}%
\pgfpathlineto{\pgfqpoint{1.950519in}{1.226895in}}%
\pgfpathlineto{\pgfqpoint{1.947730in}{1.211138in}}%
\pgfpathlineto{\pgfqpoint{1.951798in}{1.193397in}}%
\pgfpathlineto{\pgfqpoint{1.956311in}{1.184451in}}%
\pgfpathlineto{\pgfqpoint{1.970708in}{1.172146in}}%
\pgfpathlineto{\pgfqpoint{1.973858in}{1.166711in}}%
\pgfpathlineto{\pgfqpoint{1.989621in}{1.154462in}}%
\pgfpathlineto{\pgfqpoint{1.991906in}{1.140721in}}%
\pgfpathlineto{\pgfqpoint{1.996433in}{1.130585in}}%
\pgfpathlineto{\pgfqpoint{2.007129in}{1.136489in}}%
\pgfpathlineto{\pgfqpoint{2.015945in}{1.133789in}}%
\pgfpathlineto{\pgfqpoint{2.022602in}{1.126623in}}%
\pgfpathlineto{\pgfqpoint{2.020088in}{1.113940in}}%
\pgfpathlineto{\pgfqpoint{2.011918in}{1.097359in}}%
\pgfpathlineto{\pgfqpoint{2.012044in}{1.088723in}}%
\pgfpathlineto{\pgfqpoint{2.018150in}{1.081642in}}%
\pgfpathlineto{\pgfqpoint{2.031291in}{1.072263in}}%
\pgfpathlineto{\pgfqpoint{2.046423in}{1.064650in}}%
\pgfpathlineto{\pgfqpoint{2.050012in}{1.058555in}}%
\pgfpathlineto{\pgfqpoint{2.057751in}{1.055672in}}%
\pgfpathlineto{\pgfqpoint{2.058058in}{1.047582in}}%
\pgfpathlineto{\pgfqpoint{2.063786in}{1.037307in}}%
\pgfpathlineto{\pgfqpoint{2.059812in}{1.028177in}}%
\pgfpathlineto{\pgfqpoint{2.067590in}{1.013653in}}%
\pgfpathlineto{\pgfqpoint{2.073675in}{1.016537in}}%
\pgfpathlineto{\pgfqpoint{2.081137in}{1.010475in}}%
\pgfpathlineto{\pgfqpoint{2.078628in}{0.983482in}}%
\pgfpathlineto{\pgfqpoint{2.069833in}{0.985194in}}%
\pgfpathlineto{\pgfqpoint{2.068136in}{0.977587in}}%
\pgfusepath{stroke}%
\end{pgfscope}%
\begin{pgfscope}%
\pgfpathrectangle{\pgfqpoint{0.100000in}{0.100000in}}{\pgfqpoint{3.007045in}{1.925000in}}%
\pgfusepath{clip}%
\pgfsetbuttcap%
\pgfsetroundjoin%
\pgfsetlinewidth{0.050187pt}%
\definecolor{currentstroke}{rgb}{1.000000,1.000000,1.000000}%
\pgfsetstrokecolor{currentstroke}%
\pgfsetdash{}{0pt}%
\pgfpathmoveto{\pgfqpoint{0.719235in}{0.983407in}}%
\pgfpathlineto{\pgfqpoint{0.725604in}{0.996967in}}%
\pgfpathlineto{\pgfqpoint{0.723861in}{1.017321in}}%
\pgfpathlineto{\pgfqpoint{0.726911in}{1.023110in}}%
\pgfpathlineto{\pgfqpoint{0.727212in}{1.048463in}}%
\pgfpathlineto{\pgfqpoint{0.730103in}{1.055764in}}%
\pgfpathlineto{\pgfqpoint{0.740288in}{1.056900in}}%
\pgfpathlineto{\pgfqpoint{0.749739in}{1.053659in}}%
\pgfpathlineto{\pgfqpoint{0.753855in}{1.044733in}}%
\pgfpathlineto{\pgfqpoint{0.759627in}{1.045075in}}%
\pgfpathlineto{\pgfqpoint{0.766797in}{1.055300in}}%
\pgfpathlineto{\pgfqpoint{0.777160in}{1.105704in}}%
\pgfpathlineto{\pgfqpoint{0.836134in}{1.093528in}}%
\pgfpathlineto{\pgfqpoint{0.870361in}{1.086778in}}%
\pgfpathlineto{\pgfqpoint{0.961251in}{1.070801in}}%
\pgfpathlineto{\pgfqpoint{0.986397in}{1.065751in}}%
\pgfpathlineto{\pgfqpoint{1.035744in}{1.058103in}}%
\pgfpathlineto{\pgfqpoint{1.025626in}{0.992952in}}%
\pgfpathlineto{\pgfqpoint{1.015102in}{0.924993in}}%
\pgfpathlineto{\pgfqpoint{1.002983in}{0.848537in}}%
\pgfpathlineto{\pgfqpoint{0.989353in}{0.760774in}}%
\pgfpathlineto{\pgfqpoint{0.978343in}{0.688693in}}%
\pgfpathlineto{\pgfqpoint{0.899477in}{0.701217in}}%
\pgfpathlineto{\pgfqpoint{0.864826in}{0.707151in}}%
\pgfpathlineto{\pgfqpoint{0.849347in}{0.716418in}}%
\pgfpathlineto{\pgfqpoint{0.748420in}{0.777313in}}%
\pgfpathlineto{\pgfqpoint{0.672680in}{0.823525in}}%
\pgfpathlineto{\pgfqpoint{0.675202in}{0.831715in}}%
\pgfpathlineto{\pgfqpoint{0.681459in}{0.837446in}}%
\pgfpathlineto{\pgfqpoint{0.687967in}{0.836384in}}%
\pgfpathlineto{\pgfqpoint{0.697405in}{0.842403in}}%
\pgfpathlineto{\pgfqpoint{0.698891in}{0.851052in}}%
\pgfpathlineto{\pgfqpoint{0.687380in}{0.861545in}}%
\pgfpathlineto{\pgfqpoint{0.695567in}{0.881688in}}%
\pgfpathlineto{\pgfqpoint{0.703798in}{0.889434in}}%
\pgfpathlineto{\pgfqpoint{0.707704in}{0.898624in}}%
\pgfpathlineto{\pgfqpoint{0.710111in}{0.915475in}}%
\pgfpathlineto{\pgfqpoint{0.717829in}{0.923106in}}%
\pgfpathlineto{\pgfqpoint{0.734029in}{0.930728in}}%
\pgfpathlineto{\pgfqpoint{0.733671in}{0.937444in}}%
\pgfpathlineto{\pgfqpoint{0.724623in}{0.945778in}}%
\pgfpathlineto{\pgfqpoint{0.723391in}{0.962960in}}%
\pgfpathlineto{\pgfqpoint{0.717162in}{0.975520in}}%
\pgfpathlineto{\pgfqpoint{0.719235in}{0.983407in}}%
\pgfusepath{stroke}%
\end{pgfscope}%
\begin{pgfscope}%
\pgfpathrectangle{\pgfqpoint{0.100000in}{0.100000in}}{\pgfqpoint{3.007045in}{1.925000in}}%
\pgfusepath{clip}%
\pgfsetbuttcap%
\pgfsetroundjoin%
\pgfsetlinewidth{0.050187pt}%
\definecolor{currentstroke}{rgb}{1.000000,1.000000,1.000000}%
\pgfsetstrokecolor{currentstroke}%
\pgfsetdash{}{0pt}%
\pgfpathmoveto{\pgfqpoint{1.801141in}{0.779025in}}%
\pgfpathlineto{\pgfqpoint{1.786521in}{0.783547in}}%
\pgfpathlineto{\pgfqpoint{1.767788in}{0.797893in}}%
\pgfpathlineto{\pgfqpoint{1.759467in}{0.800976in}}%
\pgfpathlineto{\pgfqpoint{1.754180in}{0.794780in}}%
\pgfpathlineto{\pgfqpoint{1.747503in}{0.794467in}}%
\pgfpathlineto{\pgfqpoint{1.739055in}{0.799728in}}%
\pgfpathlineto{\pgfqpoint{1.725799in}{0.792988in}}%
\pgfpathlineto{\pgfqpoint{1.721080in}{0.796307in}}%
\pgfpathlineto{\pgfqpoint{1.708036in}{0.788443in}}%
\pgfpathlineto{\pgfqpoint{1.694275in}{0.789894in}}%
\pgfpathlineto{\pgfqpoint{1.683661in}{0.795004in}}%
\pgfpathlineto{\pgfqpoint{1.676227in}{0.793081in}}%
\pgfpathlineto{\pgfqpoint{1.664329in}{0.800308in}}%
\pgfpathlineto{\pgfqpoint{1.657711in}{0.787562in}}%
\pgfpathlineto{\pgfqpoint{1.650942in}{0.798523in}}%
\pgfpathlineto{\pgfqpoint{1.637572in}{0.794271in}}%
\pgfpathlineto{\pgfqpoint{1.625879in}{0.804682in}}%
\pgfpathlineto{\pgfqpoint{1.615661in}{0.796290in}}%
\pgfpathlineto{\pgfqpoint{1.610544in}{0.804002in}}%
\pgfpathlineto{\pgfqpoint{1.603167in}{0.806507in}}%
\pgfpathlineto{\pgfqpoint{1.598675in}{0.813942in}}%
\pgfpathlineto{\pgfqpoint{1.589028in}{0.816058in}}%
\pgfpathlineto{\pgfqpoint{1.583461in}{0.810461in}}%
\pgfpathlineto{\pgfqpoint{1.574003in}{0.817714in}}%
\pgfpathlineto{\pgfqpoint{1.569594in}{0.816058in}}%
\pgfpathlineto{\pgfqpoint{1.553918in}{0.821935in}}%
\pgfpathlineto{\pgfqpoint{1.544100in}{0.822590in}}%
\pgfpathlineto{\pgfqpoint{1.539706in}{0.835071in}}%
\pgfpathlineto{\pgfqpoint{1.531836in}{0.833517in}}%
\pgfpathlineto{\pgfqpoint{1.522821in}{0.836596in}}%
\pgfpathlineto{\pgfqpoint{1.516885in}{0.834817in}}%
\pgfpathlineto{\pgfqpoint{1.504154in}{0.848834in}}%
\pgfpathlineto{\pgfqpoint{1.500607in}{0.847917in}}%
\pgfpathlineto{\pgfqpoint{1.503828in}{0.904762in}}%
\pgfpathlineto{\pgfqpoint{1.507338in}{0.975120in}}%
\pgfpathlineto{\pgfqpoint{1.449726in}{0.978343in}}%
\pgfpathlineto{\pgfqpoint{1.392875in}{0.982641in}}%
\pgfpathlineto{\pgfqpoint{1.348953in}{0.986455in}}%
\pgfpathlineto{\pgfqpoint{1.351869in}{1.019693in}}%
\pgfpathlineto{\pgfqpoint{1.402207in}{1.014889in}}%
\pgfpathlineto{\pgfqpoint{1.459299in}{1.011309in}}%
\pgfpathlineto{\pgfqpoint{1.538812in}{1.006621in}}%
\pgfpathlineto{\pgfqpoint{1.586009in}{1.004800in}}%
\pgfpathlineto{\pgfqpoint{1.642551in}{1.002805in}}%
\pgfpathlineto{\pgfqpoint{1.719359in}{1.001461in}}%
\pgfpathlineto{\pgfqpoint{1.792298in}{1.001125in}}%
\pgfpathlineto{\pgfqpoint{1.792603in}{0.967662in}}%
\pgfpathlineto{\pgfqpoint{1.796697in}{0.942452in}}%
\pgfpathlineto{\pgfqpoint{1.803058in}{0.895889in}}%
\pgfpathlineto{\pgfqpoint{1.801746in}{0.816372in}}%
\pgfpathlineto{\pgfqpoint{1.801141in}{0.779025in}}%
\pgfusepath{stroke}%
\end{pgfscope}%
\begin{pgfscope}%
\pgfpathrectangle{\pgfqpoint{0.100000in}{0.100000in}}{\pgfqpoint{3.007045in}{1.925000in}}%
\pgfusepath{clip}%
\pgfsetbuttcap%
\pgfsetroundjoin%
\pgfsetlinewidth{0.050187pt}%
\definecolor{currentstroke}{rgb}{1.000000,1.000000,1.000000}%
\pgfsetstrokecolor{currentstroke}%
\pgfsetdash{}{0pt}%
\pgfpathmoveto{\pgfqpoint{2.348542in}{0.902516in}}%
\pgfpathlineto{\pgfqpoint{2.348585in}{0.917250in}}%
\pgfpathlineto{\pgfqpoint{2.361388in}{0.922908in}}%
\pgfpathlineto{\pgfqpoint{2.361925in}{0.931952in}}%
\pgfpathlineto{\pgfqpoint{2.373389in}{0.943000in}}%
\pgfpathlineto{\pgfqpoint{2.387718in}{0.945263in}}%
\pgfpathlineto{\pgfqpoint{2.400316in}{0.957439in}}%
\pgfpathlineto{\pgfqpoint{2.413472in}{0.962881in}}%
\pgfpathlineto{\pgfqpoint{2.423346in}{0.979220in}}%
\pgfpathlineto{\pgfqpoint{2.432339in}{0.978033in}}%
\pgfpathlineto{\pgfqpoint{2.451345in}{0.992920in}}%
\pgfpathlineto{\pgfqpoint{2.461381in}{0.993200in}}%
\pgfpathlineto{\pgfqpoint{2.465620in}{1.004543in}}%
\pgfpathlineto{\pgfqpoint{2.473597in}{1.012440in}}%
\pgfpathlineto{\pgfqpoint{2.475422in}{1.025896in}}%
\pgfpathlineto{\pgfqpoint{2.492587in}{1.027345in}}%
\pgfpathlineto{\pgfqpoint{2.561981in}{1.036510in}}%
\pgfpathlineto{\pgfqpoint{2.643688in}{1.050585in}}%
\pgfpathlineto{\pgfqpoint{2.702785in}{1.062681in}}%
\pgfpathlineto{\pgfqpoint{2.768856in}{1.076386in}}%
\pgfpathlineto{\pgfqpoint{2.787736in}{1.050465in}}%
\pgfpathlineto{\pgfqpoint{2.777509in}{1.052120in}}%
\pgfpathlineto{\pgfqpoint{2.764382in}{1.048922in}}%
\pgfpathlineto{\pgfqpoint{2.751446in}{1.035712in}}%
\pgfpathlineto{\pgfqpoint{2.742149in}{1.036662in}}%
\pgfpathlineto{\pgfqpoint{2.740973in}{1.028816in}}%
\pgfpathlineto{\pgfqpoint{2.757777in}{1.035021in}}%
\pgfpathlineto{\pgfqpoint{2.774686in}{1.037579in}}%
\pgfpathlineto{\pgfqpoint{2.780965in}{1.034175in}}%
\pgfpathlineto{\pgfqpoint{2.789414in}{1.038056in}}%
\pgfpathlineto{\pgfqpoint{2.793774in}{1.035336in}}%
\pgfpathlineto{\pgfqpoint{2.797633in}{1.022389in}}%
\pgfpathlineto{\pgfqpoint{2.789602in}{1.018379in}}%
\pgfpathlineto{\pgfqpoint{2.784090in}{1.002591in}}%
\pgfpathlineto{\pgfqpoint{2.778311in}{0.996443in}}%
\pgfpathlineto{\pgfqpoint{2.760595in}{0.997788in}}%
\pgfpathlineto{\pgfqpoint{2.761547in}{1.007059in}}%
\pgfpathlineto{\pgfqpoint{2.751872in}{1.003009in}}%
\pgfpathlineto{\pgfqpoint{2.749767in}{0.995277in}}%
\pgfpathlineto{\pgfqpoint{2.756399in}{0.987268in}}%
\pgfpathlineto{\pgfqpoint{2.758652in}{0.973674in}}%
\pgfpathlineto{\pgfqpoint{2.747800in}{0.965929in}}%
\pgfpathlineto{\pgfqpoint{2.759537in}{0.963205in}}%
\pgfpathlineto{\pgfqpoint{2.764848in}{0.968900in}}%
\pgfpathlineto{\pgfqpoint{2.776591in}{0.969594in}}%
\pgfpathlineto{\pgfqpoint{2.770556in}{0.957295in}}%
\pgfpathlineto{\pgfqpoint{2.763165in}{0.951369in}}%
\pgfpathlineto{\pgfqpoint{2.740807in}{0.945436in}}%
\pgfpathlineto{\pgfqpoint{2.717851in}{0.924616in}}%
\pgfpathlineto{\pgfqpoint{2.703697in}{0.904102in}}%
\pgfpathlineto{\pgfqpoint{2.703613in}{0.895771in}}%
\pgfpathlineto{\pgfqpoint{2.697959in}{0.884275in}}%
\pgfpathlineto{\pgfqpoint{2.668941in}{0.876711in}}%
\pgfpathlineto{\pgfqpoint{2.599931in}{0.926180in}}%
\pgfpathlineto{\pgfqpoint{2.539158in}{0.917251in}}%
\pgfpathlineto{\pgfqpoint{2.538649in}{0.925496in}}%
\pgfpathlineto{\pgfqpoint{2.529411in}{0.934781in}}%
\pgfpathlineto{\pgfqpoint{2.522418in}{0.937054in}}%
\pgfpathlineto{\pgfqpoint{2.456388in}{0.930193in}}%
\pgfpathlineto{\pgfqpoint{2.441211in}{0.925045in}}%
\pgfpathlineto{\pgfqpoint{2.413807in}{0.911413in}}%
\pgfpathlineto{\pgfqpoint{2.390124in}{0.907640in}}%
\pgfpathlineto{\pgfqpoint{2.348542in}{0.902516in}}%
\pgfusepath{stroke}%
\end{pgfscope}%
\begin{pgfscope}%
\pgfpathrectangle{\pgfqpoint{0.100000in}{0.100000in}}{\pgfqpoint{3.007045in}{1.925000in}}%
\pgfusepath{clip}%
\pgfsetbuttcap%
\pgfsetroundjoin%
\pgfsetlinewidth{0.050187pt}%
\definecolor{currentstroke}{rgb}{1.000000,1.000000,1.000000}%
\pgfsetstrokecolor{currentstroke}%
\pgfsetdash{}{0pt}%
\pgfpathmoveto{\pgfqpoint{2.348542in}{0.902516in}}%
\pgfpathlineto{\pgfqpoint{2.279433in}{0.894926in}}%
\pgfpathlineto{\pgfqpoint{2.216213in}{0.889241in}}%
\pgfpathlineto{\pgfqpoint{2.151235in}{0.885039in}}%
\pgfpathlineto{\pgfqpoint{2.140081in}{0.883422in}}%
\pgfpathlineto{\pgfqpoint{2.096280in}{0.880053in}}%
\pgfpathlineto{\pgfqpoint{2.026124in}{0.875959in}}%
\pgfpathlineto{\pgfqpoint{2.038606in}{0.886320in}}%
\pgfpathlineto{\pgfqpoint{2.035758in}{0.897226in}}%
\pgfpathlineto{\pgfqpoint{2.038517in}{0.910446in}}%
\pgfpathlineto{\pgfqpoint{2.042732in}{0.916670in}}%
\pgfpathlineto{\pgfqpoint{2.042574in}{0.925329in}}%
\pgfpathlineto{\pgfqpoint{2.053797in}{0.930775in}}%
\pgfpathlineto{\pgfqpoint{2.053816in}{0.943518in}}%
\pgfpathlineto{\pgfqpoint{2.055358in}{0.958208in}}%
\pgfpathlineto{\pgfqpoint{2.063184in}{0.967596in}}%
\pgfpathlineto{\pgfqpoint{2.061886in}{0.976909in}}%
\pgfpathlineto{\pgfqpoint{2.064919in}{0.977122in}}%
\pgfpathlineto{\pgfqpoint{2.068136in}{0.977587in}}%
\pgfpathlineto{\pgfqpoint{2.140708in}{0.982200in}}%
\pgfpathlineto{\pgfqpoint{2.138278in}{0.994121in}}%
\pgfpathlineto{\pgfqpoint{2.149956in}{0.992504in}}%
\pgfpathlineto{\pgfqpoint{2.220644in}{0.999497in}}%
\pgfpathlineto{\pgfqpoint{2.277384in}{1.002708in}}%
\pgfpathlineto{\pgfqpoint{2.311961in}{1.005616in}}%
\pgfpathlineto{\pgfqpoint{2.369218in}{1.011045in}}%
\pgfpathlineto{\pgfqpoint{2.370173in}{1.012125in}}%
\pgfpathlineto{\pgfqpoint{2.394637in}{1.014547in}}%
\pgfpathlineto{\pgfqpoint{2.475422in}{1.025896in}}%
\pgfpathlineto{\pgfqpoint{2.473597in}{1.012440in}}%
\pgfpathlineto{\pgfqpoint{2.465620in}{1.004543in}}%
\pgfpathlineto{\pgfqpoint{2.461381in}{0.993200in}}%
\pgfpathlineto{\pgfqpoint{2.451345in}{0.992920in}}%
\pgfpathlineto{\pgfqpoint{2.432339in}{0.978033in}}%
\pgfpathlineto{\pgfqpoint{2.423346in}{0.979220in}}%
\pgfpathlineto{\pgfqpoint{2.413472in}{0.962881in}}%
\pgfpathlineto{\pgfqpoint{2.400316in}{0.957439in}}%
\pgfpathlineto{\pgfqpoint{2.387718in}{0.945263in}}%
\pgfpathlineto{\pgfqpoint{2.373389in}{0.943000in}}%
\pgfpathlineto{\pgfqpoint{2.361925in}{0.931952in}}%
\pgfpathlineto{\pgfqpoint{2.361388in}{0.922908in}}%
\pgfpathlineto{\pgfqpoint{2.348585in}{0.917250in}}%
\pgfpathlineto{\pgfqpoint{2.348542in}{0.902516in}}%
\pgfusepath{stroke}%
\end{pgfscope}%
\begin{pgfscope}%
\pgfpathrectangle{\pgfqpoint{0.100000in}{0.100000in}}{\pgfqpoint{3.007045in}{1.925000in}}%
\pgfusepath{clip}%
\pgfsetbuttcap%
\pgfsetroundjoin%
\pgfsetlinewidth{0.050187pt}%
\definecolor{currentstroke}{rgb}{1.000000,1.000000,1.000000}%
\pgfsetstrokecolor{currentstroke}%
\pgfsetdash{}{0pt}%
\pgfpathmoveto{\pgfqpoint{1.122887in}{0.697915in}}%
\pgfpathlineto{\pgfqpoint{1.117958in}{0.705835in}}%
\pgfpathlineto{\pgfqpoint{1.119998in}{0.712658in}}%
\pgfpathlineto{\pgfqpoint{1.219064in}{0.701043in}}%
\pgfpathlineto{\pgfqpoint{1.319441in}{0.691038in}}%
\pgfpathlineto{\pgfqpoint{1.322368in}{0.724970in}}%
\pgfpathlineto{\pgfqpoint{1.337455in}{0.871739in}}%
\pgfpathlineto{\pgfqpoint{1.347370in}{0.986543in}}%
\pgfpathlineto{\pgfqpoint{1.348953in}{0.986455in}}%
\pgfpathlineto{\pgfqpoint{1.449726in}{0.978343in}}%
\pgfpathlineto{\pgfqpoint{1.507338in}{0.975120in}}%
\pgfpathlineto{\pgfqpoint{1.500607in}{0.847917in}}%
\pgfpathlineto{\pgfqpoint{1.504154in}{0.848834in}}%
\pgfpathlineto{\pgfqpoint{1.516885in}{0.834817in}}%
\pgfpathlineto{\pgfqpoint{1.522821in}{0.836596in}}%
\pgfpathlineto{\pgfqpoint{1.531836in}{0.833517in}}%
\pgfpathlineto{\pgfqpoint{1.539706in}{0.835071in}}%
\pgfpathlineto{\pgfqpoint{1.544100in}{0.822590in}}%
\pgfpathlineto{\pgfqpoint{1.553918in}{0.821935in}}%
\pgfpathlineto{\pgfqpoint{1.569594in}{0.816058in}}%
\pgfpathlineto{\pgfqpoint{1.574003in}{0.817714in}}%
\pgfpathlineto{\pgfqpoint{1.583461in}{0.810461in}}%
\pgfpathlineto{\pgfqpoint{1.589028in}{0.816058in}}%
\pgfpathlineto{\pgfqpoint{1.598675in}{0.813942in}}%
\pgfpathlineto{\pgfqpoint{1.603167in}{0.806507in}}%
\pgfpathlineto{\pgfqpoint{1.610544in}{0.804002in}}%
\pgfpathlineto{\pgfqpoint{1.615661in}{0.796290in}}%
\pgfpathlineto{\pgfqpoint{1.625879in}{0.804682in}}%
\pgfpathlineto{\pgfqpoint{1.637572in}{0.794271in}}%
\pgfpathlineto{\pgfqpoint{1.650942in}{0.798523in}}%
\pgfpathlineto{\pgfqpoint{1.657711in}{0.787562in}}%
\pgfpathlineto{\pgfqpoint{1.664329in}{0.800308in}}%
\pgfpathlineto{\pgfqpoint{1.676227in}{0.793081in}}%
\pgfpathlineto{\pgfqpoint{1.683661in}{0.795004in}}%
\pgfpathlineto{\pgfqpoint{1.694275in}{0.789894in}}%
\pgfpathlineto{\pgfqpoint{1.708036in}{0.788443in}}%
\pgfpathlineto{\pgfqpoint{1.721080in}{0.796307in}}%
\pgfpathlineto{\pgfqpoint{1.725799in}{0.792988in}}%
\pgfpathlineto{\pgfqpoint{1.739055in}{0.799728in}}%
\pgfpathlineto{\pgfqpoint{1.747503in}{0.794467in}}%
\pgfpathlineto{\pgfqpoint{1.754180in}{0.794780in}}%
\pgfpathlineto{\pgfqpoint{1.759467in}{0.800976in}}%
\pgfpathlineto{\pgfqpoint{1.767788in}{0.797893in}}%
\pgfpathlineto{\pgfqpoint{1.786521in}{0.783547in}}%
\pgfpathlineto{\pgfqpoint{1.801141in}{0.779025in}}%
\pgfpathlineto{\pgfqpoint{1.807003in}{0.773486in}}%
\pgfpathlineto{\pgfqpoint{1.814342in}{0.776503in}}%
\pgfpathlineto{\pgfqpoint{1.825462in}{0.774193in}}%
\pgfpathlineto{\pgfqpoint{1.826608in}{0.670746in}}%
\pgfpathlineto{\pgfqpoint{1.834331in}{0.664193in}}%
\pgfpathlineto{\pgfqpoint{1.840521in}{0.652119in}}%
\pgfpathlineto{\pgfqpoint{1.838345in}{0.644041in}}%
\pgfpathlineto{\pgfqpoint{1.843206in}{0.640621in}}%
\pgfpathlineto{\pgfqpoint{1.847157in}{0.625724in}}%
\pgfpathlineto{\pgfqpoint{1.854722in}{0.617746in}}%
\pgfpathlineto{\pgfqpoint{1.856418in}{0.600809in}}%
\pgfpathlineto{\pgfqpoint{1.855102in}{0.593637in}}%
\pgfpathlineto{\pgfqpoint{1.844817in}{0.574659in}}%
\pgfpathlineto{\pgfqpoint{1.843635in}{0.561013in}}%
\pgfpathlineto{\pgfqpoint{1.847123in}{0.558165in}}%
\pgfpathlineto{\pgfqpoint{1.847427in}{0.546210in}}%
\pgfpathlineto{\pgfqpoint{1.843881in}{0.538697in}}%
\pgfpathlineto{\pgfqpoint{1.838366in}{0.535867in}}%
\pgfpathlineto{\pgfqpoint{1.832966in}{0.526048in}}%
\pgfpathlineto{\pgfqpoint{1.839848in}{0.516539in}}%
\pgfpathlineto{\pgfqpoint{1.826489in}{0.516351in}}%
\pgfpathlineto{\pgfqpoint{1.790766in}{0.500008in}}%
\pgfpathlineto{\pgfqpoint{1.797587in}{0.509789in}}%
\pgfpathlineto{\pgfqpoint{1.789328in}{0.515062in}}%
\pgfpathlineto{\pgfqpoint{1.787602in}{0.524028in}}%
\pgfpathlineto{\pgfqpoint{1.771483in}{0.508411in}}%
\pgfpathlineto{\pgfqpoint{1.778637in}{0.497737in}}%
\pgfpathlineto{\pgfqpoint{1.768432in}{0.484145in}}%
\pgfpathlineto{\pgfqpoint{1.762940in}{0.484430in}}%
\pgfpathlineto{\pgfqpoint{1.757774in}{0.469628in}}%
\pgfpathlineto{\pgfqpoint{1.741435in}{0.457985in}}%
\pgfpathlineto{\pgfqpoint{1.726171in}{0.453789in}}%
\pgfpathlineto{\pgfqpoint{1.699548in}{0.442873in}}%
\pgfpathlineto{\pgfqpoint{1.696789in}{0.448962in}}%
\pgfpathlineto{\pgfqpoint{1.680205in}{0.443357in}}%
\pgfpathlineto{\pgfqpoint{1.690405in}{0.433814in}}%
\pgfpathlineto{\pgfqpoint{1.674316in}{0.425524in}}%
\pgfpathlineto{\pgfqpoint{1.656962in}{0.413006in}}%
\pgfpathlineto{\pgfqpoint{1.652794in}{0.418825in}}%
\pgfpathlineto{\pgfqpoint{1.638592in}{0.410104in}}%
\pgfpathlineto{\pgfqpoint{1.652512in}{0.406792in}}%
\pgfpathlineto{\pgfqpoint{1.642016in}{0.393080in}}%
\pgfpathlineto{\pgfqpoint{1.636893in}{0.397161in}}%
\pgfpathlineto{\pgfqpoint{1.630012in}{0.390606in}}%
\pgfpathlineto{\pgfqpoint{1.638588in}{0.384881in}}%
\pgfpathlineto{\pgfqpoint{1.633518in}{0.376512in}}%
\pgfpathlineto{\pgfqpoint{1.627286in}{0.356702in}}%
\pgfpathlineto{\pgfqpoint{1.618297in}{0.337622in}}%
\pgfpathlineto{\pgfqpoint{1.622332in}{0.325116in}}%
\pgfpathlineto{\pgfqpoint{1.629192in}{0.295630in}}%
\pgfpathlineto{\pgfqpoint{1.629809in}{0.283697in}}%
\pgfpathlineto{\pgfqpoint{1.640407in}{0.268045in}}%
\pgfpathlineto{\pgfqpoint{1.632235in}{0.268959in}}%
\pgfpathlineto{\pgfqpoint{1.624297in}{0.261063in}}%
\pgfpathlineto{\pgfqpoint{1.615000in}{0.267269in}}%
\pgfpathlineto{\pgfqpoint{1.611663in}{0.273460in}}%
\pgfpathlineto{\pgfqpoint{1.598432in}{0.276340in}}%
\pgfpathlineto{\pgfqpoint{1.578220in}{0.276693in}}%
\pgfpathlineto{\pgfqpoint{1.563329in}{0.288418in}}%
\pgfpathlineto{\pgfqpoint{1.549807in}{0.290376in}}%
\pgfpathlineto{\pgfqpoint{1.541604in}{0.299694in}}%
\pgfpathlineto{\pgfqpoint{1.524424in}{0.303444in}}%
\pgfpathlineto{\pgfqpoint{1.515030in}{0.333403in}}%
\pgfpathlineto{\pgfqpoint{1.505425in}{0.345405in}}%
\pgfpathlineto{\pgfqpoint{1.507057in}{0.356809in}}%
\pgfpathlineto{\pgfqpoint{1.501101in}{0.365143in}}%
\pgfpathlineto{\pgfqpoint{1.504839in}{0.376534in}}%
\pgfpathlineto{\pgfqpoint{1.501755in}{0.384882in}}%
\pgfpathlineto{\pgfqpoint{1.492103in}{0.388664in}}%
\pgfpathlineto{\pgfqpoint{1.483078in}{0.398289in}}%
\pgfpathlineto{\pgfqpoint{1.479790in}{0.411177in}}%
\pgfpathlineto{\pgfqpoint{1.471250in}{0.422889in}}%
\pgfpathlineto{\pgfqpoint{1.459883in}{0.432001in}}%
\pgfpathlineto{\pgfqpoint{1.449644in}{0.458142in}}%
\pgfpathlineto{\pgfqpoint{1.441377in}{0.486731in}}%
\pgfpathlineto{\pgfqpoint{1.434594in}{0.498034in}}%
\pgfpathlineto{\pgfqpoint{1.422834in}{0.507551in}}%
\pgfpathlineto{\pgfqpoint{1.419895in}{0.514459in}}%
\pgfpathlineto{\pgfqpoint{1.408884in}{0.518748in}}%
\pgfpathlineto{\pgfqpoint{1.398016in}{0.537050in}}%
\pgfpathlineto{\pgfqpoint{1.388091in}{0.535645in}}%
\pgfpathlineto{\pgfqpoint{1.378014in}{0.540216in}}%
\pgfpathlineto{\pgfqpoint{1.363728in}{0.539332in}}%
\pgfpathlineto{\pgfqpoint{1.349216in}{0.546876in}}%
\pgfpathlineto{\pgfqpoint{1.345123in}{0.539702in}}%
\pgfpathlineto{\pgfqpoint{1.328163in}{0.539522in}}%
\pgfpathlineto{\pgfqpoint{1.319529in}{0.525922in}}%
\pgfpathlineto{\pgfqpoint{1.312019in}{0.509082in}}%
\pgfpathlineto{\pgfqpoint{1.296050in}{0.491006in}}%
\pgfpathlineto{\pgfqpoint{1.277957in}{0.498939in}}%
\pgfpathlineto{\pgfqpoint{1.275390in}{0.504181in}}%
\pgfpathlineto{\pgfqpoint{1.264395in}{0.508180in}}%
\pgfpathlineto{\pgfqpoint{1.261017in}{0.513650in}}%
\pgfpathlineto{\pgfqpoint{1.246421in}{0.519171in}}%
\pgfpathlineto{\pgfqpoint{1.238256in}{0.530458in}}%
\pgfpathlineto{\pgfqpoint{1.228720in}{0.535904in}}%
\pgfpathlineto{\pgfqpoint{1.220515in}{0.545406in}}%
\pgfpathlineto{\pgfqpoint{1.214124in}{0.561500in}}%
\pgfpathlineto{\pgfqpoint{1.214839in}{0.583492in}}%
\pgfpathlineto{\pgfqpoint{1.207341in}{0.594605in}}%
\pgfpathlineto{\pgfqpoint{1.206478in}{0.606641in}}%
\pgfpathlineto{\pgfqpoint{1.189842in}{0.624662in}}%
\pgfpathlineto{\pgfqpoint{1.180162in}{0.628509in}}%
\pgfpathlineto{\pgfqpoint{1.169890in}{0.645324in}}%
\pgfpathlineto{\pgfqpoint{1.161136in}{0.652005in}}%
\pgfpathlineto{\pgfqpoint{1.150014in}{0.668274in}}%
\pgfpathlineto{\pgfqpoint{1.138634in}{0.675314in}}%
\pgfpathlineto{\pgfqpoint{1.131182in}{0.693355in}}%
\pgfpathlineto{\pgfqpoint{1.122887in}{0.697915in}}%
\pgfpathlineto{\pgfqpoint{1.122887in}{0.697915in}}%
\pgfusepath{stroke}%
\end{pgfscope}%
\begin{pgfscope}%
\pgfpathrectangle{\pgfqpoint{0.100000in}{0.100000in}}{\pgfqpoint{3.007045in}{1.925000in}}%
\pgfusepath{clip}%
\pgfsetbuttcap%
\pgfsetroundjoin%
\pgfsetlinewidth{0.050187pt}%
\definecolor{currentstroke}{rgb}{1.000000,1.000000,1.000000}%
\pgfsetstrokecolor{currentstroke}%
\pgfsetdash{}{0pt}%
\pgfpathmoveto{\pgfqpoint{1.035744in}{1.058103in}}%
\pgfpathlineto{\pgfqpoint{1.117842in}{1.046179in}}%
\pgfpathlineto{\pgfqpoint{1.149648in}{1.041279in}}%
\pgfpathlineto{\pgfqpoint{1.239303in}{1.030616in}}%
\pgfpathlineto{\pgfqpoint{1.299648in}{1.024321in}}%
\pgfpathlineto{\pgfqpoint{1.351869in}{1.019693in}}%
\pgfpathlineto{\pgfqpoint{1.348953in}{0.986455in}}%
\pgfpathlineto{\pgfqpoint{1.347370in}{0.986543in}}%
\pgfpathlineto{\pgfqpoint{1.342587in}{0.929495in}}%
\pgfpathlineto{\pgfqpoint{1.337455in}{0.871739in}}%
\pgfpathlineto{\pgfqpoint{1.331519in}{0.811277in}}%
\pgfpathlineto{\pgfqpoint{1.322368in}{0.724970in}}%
\pgfpathlineto{\pgfqpoint{1.319441in}{0.691038in}}%
\pgfpathlineto{\pgfqpoint{1.265644in}{0.696466in}}%
\pgfpathlineto{\pgfqpoint{1.219064in}{0.701043in}}%
\pgfpathlineto{\pgfqpoint{1.154643in}{0.708360in}}%
\pgfpathlineto{\pgfqpoint{1.119998in}{0.712658in}}%
\pgfpathlineto{\pgfqpoint{1.117958in}{0.705835in}}%
\pgfpathlineto{\pgfqpoint{1.122887in}{0.697915in}}%
\pgfpathlineto{\pgfqpoint{1.081250in}{0.703323in}}%
\pgfpathlineto{\pgfqpoint{1.029878in}{0.710696in}}%
\pgfpathlineto{\pgfqpoint{1.025202in}{0.681640in}}%
\pgfpathlineto{\pgfqpoint{0.978343in}{0.688693in}}%
\pgfpathlineto{\pgfqpoint{0.989353in}{0.760774in}}%
\pgfpathlineto{\pgfqpoint{1.002983in}{0.848537in}}%
\pgfpathlineto{\pgfqpoint{1.015102in}{0.924993in}}%
\pgfpathlineto{\pgfqpoint{1.025626in}{0.992952in}}%
\pgfpathlineto{\pgfqpoint{1.035744in}{1.058103in}}%
\pgfusepath{stroke}%
\end{pgfscope}%
\begin{pgfscope}%
\pgfpathrectangle{\pgfqpoint{0.100000in}{0.100000in}}{\pgfqpoint{3.007045in}{1.925000in}}%
\pgfusepath{clip}%
\pgfsetbuttcap%
\pgfsetroundjoin%
\pgfsetlinewidth{0.050187pt}%
\definecolor{currentstroke}{rgb}{1.000000,1.000000,1.000000}%
\pgfsetstrokecolor{currentstroke}%
\pgfsetdash{}{0pt}%
\pgfpathmoveto{\pgfqpoint{2.341115in}{0.635787in}}%
\pgfpathlineto{\pgfqpoint{2.262793in}{0.627088in}}%
\pgfpathlineto{\pgfqpoint{2.193752in}{0.621728in}}%
\pgfpathlineto{\pgfqpoint{2.192889in}{0.613268in}}%
\pgfpathlineto{\pgfqpoint{2.199230in}{0.605215in}}%
\pgfpathlineto{\pgfqpoint{2.206971in}{0.600484in}}%
\pgfpathlineto{\pgfqpoint{2.206853in}{0.588027in}}%
\pgfpathlineto{\pgfqpoint{2.204803in}{0.579705in}}%
\pgfpathlineto{\pgfqpoint{2.197971in}{0.573707in}}%
\pgfpathlineto{\pgfqpoint{2.188449in}{0.574342in}}%
\pgfpathlineto{\pgfqpoint{2.179446in}{0.581774in}}%
\pgfpathlineto{\pgfqpoint{2.177842in}{0.594998in}}%
\pgfpathlineto{\pgfqpoint{2.171137in}{0.602689in}}%
\pgfpathlineto{\pgfqpoint{2.166573in}{0.575162in}}%
\pgfpathlineto{\pgfqpoint{2.151065in}{0.577778in}}%
\pgfpathlineto{\pgfqpoint{2.145672in}{0.625913in}}%
\pgfpathlineto{\pgfqpoint{2.139832in}{0.676640in}}%
\pgfpathlineto{\pgfqpoint{2.142001in}{0.749828in}}%
\pgfpathlineto{\pgfqpoint{2.143347in}{0.800051in}}%
\pgfpathlineto{\pgfqpoint{2.146210in}{0.876679in}}%
\pgfpathlineto{\pgfqpoint{2.140081in}{0.883422in}}%
\pgfpathlineto{\pgfqpoint{2.151235in}{0.885039in}}%
\pgfpathlineto{\pgfqpoint{2.216213in}{0.889241in}}%
\pgfpathlineto{\pgfqpoint{2.279433in}{0.894926in}}%
\pgfpathlineto{\pgfqpoint{2.296066in}{0.836665in}}%
\pgfpathlineto{\pgfqpoint{2.307410in}{0.793901in}}%
\pgfpathlineto{\pgfqpoint{2.317578in}{0.758099in}}%
\pgfpathlineto{\pgfqpoint{2.323459in}{0.743683in}}%
\pgfpathlineto{\pgfqpoint{2.332718in}{0.730267in}}%
\pgfpathlineto{\pgfqpoint{2.331222in}{0.723440in}}%
\pgfpathlineto{\pgfqpoint{2.337837in}{0.720114in}}%
\pgfpathlineto{\pgfqpoint{2.330020in}{0.709765in}}%
\pgfpathlineto{\pgfqpoint{2.327380in}{0.691335in}}%
\pgfpathlineto{\pgfqpoint{2.335019in}{0.669779in}}%
\pgfpathlineto{\pgfqpoint{2.333958in}{0.648089in}}%
\pgfpathlineto{\pgfqpoint{2.341115in}{0.635787in}}%
\pgfusepath{stroke}%
\end{pgfscope}%
\begin{pgfscope}%
\pgfpathrectangle{\pgfqpoint{0.100000in}{0.100000in}}{\pgfqpoint{3.007045in}{1.925000in}}%
\pgfusepath{clip}%
\pgfsetbuttcap%
\pgfsetroundjoin%
\pgfsetlinewidth{0.050187pt}%
\definecolor{currentstroke}{rgb}{1.000000,1.000000,1.000000}%
\pgfsetstrokecolor{currentstroke}%
\pgfsetdash{}{0pt}%
\pgfpathmoveto{\pgfqpoint{2.151065in}{0.577778in}}%
\pgfpathlineto{\pgfqpoint{2.135151in}{0.573224in}}%
\pgfpathlineto{\pgfqpoint{2.123289in}{0.576148in}}%
\pgfpathlineto{\pgfqpoint{2.101259in}{0.569141in}}%
\pgfpathlineto{\pgfqpoint{2.084645in}{0.560117in}}%
\pgfpathlineto{\pgfqpoint{2.077828in}{0.576424in}}%
\pgfpathlineto{\pgfqpoint{2.070801in}{0.583259in}}%
\pgfpathlineto{\pgfqpoint{2.067411in}{0.591438in}}%
\pgfpathlineto{\pgfqpoint{2.072380in}{0.613576in}}%
\pgfpathlineto{\pgfqpoint{2.025296in}{0.610677in}}%
\pgfpathlineto{\pgfqpoint{1.964244in}{0.608030in}}%
\pgfpathlineto{\pgfqpoint{1.967876in}{0.613540in}}%
\pgfpathlineto{\pgfqpoint{1.963386in}{0.623948in}}%
\pgfpathlineto{\pgfqpoint{1.967860in}{0.626069in}}%
\pgfpathlineto{\pgfqpoint{1.966867in}{0.636068in}}%
\pgfpathlineto{\pgfqpoint{1.974665in}{0.645765in}}%
\pgfpathlineto{\pgfqpoint{1.978931in}{0.664590in}}%
\pgfpathlineto{\pgfqpoint{1.993202in}{0.676993in}}%
\pgfpathlineto{\pgfqpoint{1.987913in}{0.689038in}}%
\pgfpathlineto{\pgfqpoint{1.998033in}{0.695971in}}%
\pgfpathlineto{\pgfqpoint{1.989303in}{0.708523in}}%
\pgfpathlineto{\pgfqpoint{1.982985in}{0.737257in}}%
\pgfpathlineto{\pgfqpoint{1.985381in}{0.742474in}}%
\pgfpathlineto{\pgfqpoint{1.989164in}{0.752482in}}%
\pgfpathlineto{\pgfqpoint{1.985643in}{0.762985in}}%
\pgfpathlineto{\pgfqpoint{1.987365in}{0.767763in}}%
\pgfpathlineto{\pgfqpoint{1.981171in}{0.785815in}}%
\pgfpathlineto{\pgfqpoint{1.999458in}{0.818229in}}%
\pgfpathlineto{\pgfqpoint{2.000308in}{0.826857in}}%
\pgfpathlineto{\pgfqpoint{2.009117in}{0.833130in}}%
\pgfpathlineto{\pgfqpoint{2.018516in}{0.853839in}}%
\pgfpathlineto{\pgfqpoint{2.018070in}{0.862257in}}%
\pgfpathlineto{\pgfqpoint{2.029777in}{0.870862in}}%
\pgfpathlineto{\pgfqpoint{2.026124in}{0.875959in}}%
\pgfpathlineto{\pgfqpoint{2.096280in}{0.880053in}}%
\pgfpathlineto{\pgfqpoint{2.140081in}{0.883422in}}%
\pgfpathlineto{\pgfqpoint{2.146210in}{0.876679in}}%
\pgfpathlineto{\pgfqpoint{2.143347in}{0.800051in}}%
\pgfpathlineto{\pgfqpoint{2.142001in}{0.749828in}}%
\pgfpathlineto{\pgfqpoint{2.139832in}{0.676640in}}%
\pgfpathlineto{\pgfqpoint{2.145672in}{0.625913in}}%
\pgfpathlineto{\pgfqpoint{2.151065in}{0.577778in}}%
\pgfusepath{stroke}%
\end{pgfscope}%
\begin{pgfscope}%
\pgfpathrectangle{\pgfqpoint{0.100000in}{0.100000in}}{\pgfqpoint{3.007045in}{1.925000in}}%
\pgfusepath{clip}%
\pgfsetbuttcap%
\pgfsetroundjoin%
\pgfsetlinewidth{0.050187pt}%
\definecolor{currentstroke}{rgb}{1.000000,1.000000,1.000000}%
\pgfsetstrokecolor{currentstroke}%
\pgfsetdash{}{0pt}%
\pgfpathmoveto{\pgfqpoint{2.279433in}{0.894926in}}%
\pgfpathlineto{\pgfqpoint{2.348542in}{0.902516in}}%
\pgfpathlineto{\pgfqpoint{2.390124in}{0.907640in}}%
\pgfpathlineto{\pgfqpoint{2.413807in}{0.911413in}}%
\pgfpathlineto{\pgfqpoint{2.403997in}{0.895940in}}%
\pgfpathlineto{\pgfqpoint{2.404003in}{0.888645in}}%
\pgfpathlineto{\pgfqpoint{2.414101in}{0.884711in}}%
\pgfpathlineto{\pgfqpoint{2.420968in}{0.878359in}}%
\pgfpathlineto{\pgfqpoint{2.431346in}{0.877574in}}%
\pgfpathlineto{\pgfqpoint{2.441046in}{0.859694in}}%
\pgfpathlineto{\pgfqpoint{2.451568in}{0.847085in}}%
\pgfpathlineto{\pgfqpoint{2.470255in}{0.836488in}}%
\pgfpathlineto{\pgfqpoint{2.475566in}{0.828036in}}%
\pgfpathlineto{\pgfqpoint{2.492230in}{0.818895in}}%
\pgfpathlineto{\pgfqpoint{2.493483in}{0.812266in}}%
\pgfpathlineto{\pgfqpoint{2.507001in}{0.799084in}}%
\pgfpathlineto{\pgfqpoint{2.517356in}{0.795331in}}%
\pgfpathlineto{\pgfqpoint{2.524689in}{0.782884in}}%
\pgfpathlineto{\pgfqpoint{2.527935in}{0.768894in}}%
\pgfpathlineto{\pgfqpoint{2.538690in}{0.763486in}}%
\pgfpathlineto{\pgfqpoint{2.547349in}{0.748448in}}%
\pgfpathlineto{\pgfqpoint{2.550141in}{0.737389in}}%
\pgfpathlineto{\pgfqpoint{2.562829in}{0.732681in}}%
\pgfpathlineto{\pgfqpoint{2.552182in}{0.712295in}}%
\pgfpathlineto{\pgfqpoint{2.548176in}{0.700248in}}%
\pgfpathlineto{\pgfqpoint{2.550790in}{0.694600in}}%
\pgfpathlineto{\pgfqpoint{2.546225in}{0.685215in}}%
\pgfpathlineto{\pgfqpoint{2.544285in}{0.672239in}}%
\pgfpathlineto{\pgfqpoint{2.536167in}{0.669822in}}%
\pgfpathlineto{\pgfqpoint{2.540454in}{0.657942in}}%
\pgfpathlineto{\pgfqpoint{2.540180in}{0.642869in}}%
\pgfpathlineto{\pgfqpoint{2.525751in}{0.643434in}}%
\pgfpathlineto{\pgfqpoint{2.515700in}{0.646184in}}%
\pgfpathlineto{\pgfqpoint{2.511350in}{0.641072in}}%
\pgfpathlineto{\pgfqpoint{2.514596in}{0.628996in}}%
\pgfpathlineto{\pgfqpoint{2.513890in}{0.614955in}}%
\pgfpathlineto{\pgfqpoint{2.507549in}{0.613881in}}%
\pgfpathlineto{\pgfqpoint{2.502419in}{0.626994in}}%
\pgfpathlineto{\pgfqpoint{2.450185in}{0.623513in}}%
\pgfpathlineto{\pgfqpoint{2.384322in}{0.619919in}}%
\pgfpathlineto{\pgfqpoint{2.351105in}{0.617581in}}%
\pgfpathlineto{\pgfqpoint{2.341115in}{0.635787in}}%
\pgfpathlineto{\pgfqpoint{2.333958in}{0.648089in}}%
\pgfpathlineto{\pgfqpoint{2.335019in}{0.669779in}}%
\pgfpathlineto{\pgfqpoint{2.327380in}{0.691335in}}%
\pgfpathlineto{\pgfqpoint{2.330020in}{0.709765in}}%
\pgfpathlineto{\pgfqpoint{2.337837in}{0.720114in}}%
\pgfpathlineto{\pgfqpoint{2.331222in}{0.723440in}}%
\pgfpathlineto{\pgfqpoint{2.332718in}{0.730267in}}%
\pgfpathlineto{\pgfqpoint{2.323459in}{0.743683in}}%
\pgfpathlineto{\pgfqpoint{2.317578in}{0.758099in}}%
\pgfpathlineto{\pgfqpoint{2.307410in}{0.793901in}}%
\pgfpathlineto{\pgfqpoint{2.296066in}{0.836665in}}%
\pgfpathlineto{\pgfqpoint{2.279433in}{0.894926in}}%
\pgfusepath{stroke}%
\end{pgfscope}%
\begin{pgfscope}%
\pgfpathrectangle{\pgfqpoint{0.100000in}{0.100000in}}{\pgfqpoint{3.007045in}{1.925000in}}%
\pgfusepath{clip}%
\pgfsetbuttcap%
\pgfsetroundjoin%
\pgfsetlinewidth{0.050187pt}%
\definecolor{currentstroke}{rgb}{1.000000,1.000000,1.000000}%
\pgfsetstrokecolor{currentstroke}%
\pgfsetdash{}{0pt}%
\pgfpathmoveto{\pgfqpoint{2.413807in}{0.911413in}}%
\pgfpathlineto{\pgfqpoint{2.441211in}{0.925045in}}%
\pgfpathlineto{\pgfqpoint{2.456388in}{0.930193in}}%
\pgfpathlineto{\pgfqpoint{2.522418in}{0.937054in}}%
\pgfpathlineto{\pgfqpoint{2.529411in}{0.934781in}}%
\pgfpathlineto{\pgfqpoint{2.538649in}{0.925496in}}%
\pgfpathlineto{\pgfqpoint{2.539158in}{0.917251in}}%
\pgfpathlineto{\pgfqpoint{2.599931in}{0.926180in}}%
\pgfpathlineto{\pgfqpoint{2.668941in}{0.876711in}}%
\pgfpathlineto{\pgfqpoint{2.655999in}{0.863222in}}%
\pgfpathlineto{\pgfqpoint{2.638241in}{0.831896in}}%
\pgfpathlineto{\pgfqpoint{2.643296in}{0.825164in}}%
\pgfpathlineto{\pgfqpoint{2.633823in}{0.812096in}}%
\pgfpathlineto{\pgfqpoint{2.624400in}{0.810617in}}%
\pgfpathlineto{\pgfqpoint{2.624365in}{0.802765in}}%
\pgfpathlineto{\pgfqpoint{2.617549in}{0.794548in}}%
\pgfpathlineto{\pgfqpoint{2.609066in}{0.792881in}}%
\pgfpathlineto{\pgfqpoint{2.610917in}{0.785581in}}%
\pgfpathlineto{\pgfqpoint{2.606183in}{0.779976in}}%
\pgfpathlineto{\pgfqpoint{2.594860in}{0.775134in}}%
\pgfpathlineto{\pgfqpoint{2.581058in}{0.764120in}}%
\pgfpathlineto{\pgfqpoint{2.583663in}{0.756995in}}%
\pgfpathlineto{\pgfqpoint{2.574977in}{0.752521in}}%
\pgfpathlineto{\pgfqpoint{2.567651in}{0.757225in}}%
\pgfpathlineto{\pgfqpoint{2.562294in}{0.736773in}}%
\pgfpathlineto{\pgfqpoint{2.550141in}{0.737389in}}%
\pgfpathlineto{\pgfqpoint{2.547349in}{0.748448in}}%
\pgfpathlineto{\pgfqpoint{2.538690in}{0.763486in}}%
\pgfpathlineto{\pgfqpoint{2.527935in}{0.768894in}}%
\pgfpathlineto{\pgfqpoint{2.524689in}{0.782884in}}%
\pgfpathlineto{\pgfqpoint{2.517356in}{0.795331in}}%
\pgfpathlineto{\pgfqpoint{2.507001in}{0.799084in}}%
\pgfpathlineto{\pgfqpoint{2.493483in}{0.812266in}}%
\pgfpathlineto{\pgfqpoint{2.492230in}{0.818895in}}%
\pgfpathlineto{\pgfqpoint{2.475566in}{0.828036in}}%
\pgfpathlineto{\pgfqpoint{2.470255in}{0.836488in}}%
\pgfpathlineto{\pgfqpoint{2.451568in}{0.847085in}}%
\pgfpathlineto{\pgfqpoint{2.441046in}{0.859694in}}%
\pgfpathlineto{\pgfqpoint{2.431346in}{0.877574in}}%
\pgfpathlineto{\pgfqpoint{2.420968in}{0.878359in}}%
\pgfpathlineto{\pgfqpoint{2.414101in}{0.884711in}}%
\pgfpathlineto{\pgfqpoint{2.404003in}{0.888645in}}%
\pgfpathlineto{\pgfqpoint{2.403997in}{0.895940in}}%
\pgfpathlineto{\pgfqpoint{2.413807in}{0.911413in}}%
\pgfusepath{stroke}%
\end{pgfscope}%
\begin{pgfscope}%
\pgfpathrectangle{\pgfqpoint{0.100000in}{0.100000in}}{\pgfqpoint{3.007045in}{1.925000in}}%
\pgfusepath{clip}%
\pgfsetbuttcap%
\pgfsetroundjoin%
\pgfsetlinewidth{0.050187pt}%
\definecolor{currentstroke}{rgb}{1.000000,1.000000,1.000000}%
\pgfsetstrokecolor{currentstroke}%
\pgfsetdash{}{0pt}%
\pgfpathmoveto{\pgfqpoint{1.792603in}{0.967662in}}%
\pgfpathlineto{\pgfqpoint{1.832874in}{0.968025in}}%
\pgfpathlineto{\pgfqpoint{1.886150in}{0.968996in}}%
\pgfpathlineto{\pgfqpoint{1.977188in}{0.971886in}}%
\pgfpathlineto{\pgfqpoint{2.029259in}{0.974639in}}%
\pgfpathlineto{\pgfqpoint{2.034844in}{0.967726in}}%
\pgfpathlineto{\pgfqpoint{2.034495in}{0.960424in}}%
\pgfpathlineto{\pgfqpoint{2.021881in}{0.947828in}}%
\pgfpathlineto{\pgfqpoint{2.018831in}{0.940931in}}%
\pgfpathlineto{\pgfqpoint{2.053816in}{0.943518in}}%
\pgfpathlineto{\pgfqpoint{2.053797in}{0.930775in}}%
\pgfpathlineto{\pgfqpoint{2.042574in}{0.925329in}}%
\pgfpathlineto{\pgfqpoint{2.042732in}{0.916670in}}%
\pgfpathlineto{\pgfqpoint{2.038517in}{0.910446in}}%
\pgfpathlineto{\pgfqpoint{2.035758in}{0.897226in}}%
\pgfpathlineto{\pgfqpoint{2.038606in}{0.886320in}}%
\pgfpathlineto{\pgfqpoint{2.026124in}{0.875959in}}%
\pgfpathlineto{\pgfqpoint{2.029777in}{0.870862in}}%
\pgfpathlineto{\pgfqpoint{2.018070in}{0.862257in}}%
\pgfpathlineto{\pgfqpoint{2.018516in}{0.853839in}}%
\pgfpathlineto{\pgfqpoint{2.009117in}{0.833130in}}%
\pgfpathlineto{\pgfqpoint{2.000308in}{0.826857in}}%
\pgfpathlineto{\pgfqpoint{1.999458in}{0.818229in}}%
\pgfpathlineto{\pgfqpoint{1.981171in}{0.785815in}}%
\pgfpathlineto{\pgfqpoint{1.987365in}{0.767763in}}%
\pgfpathlineto{\pgfqpoint{1.985643in}{0.762985in}}%
\pgfpathlineto{\pgfqpoint{1.989164in}{0.752482in}}%
\pgfpathlineto{\pgfqpoint{1.985381in}{0.742474in}}%
\pgfpathlineto{\pgfqpoint{1.935375in}{0.740410in}}%
\pgfpathlineto{\pgfqpoint{1.870456in}{0.739340in}}%
\pgfpathlineto{\pgfqpoint{1.825680in}{0.738936in}}%
\pgfpathlineto{\pgfqpoint{1.825462in}{0.774193in}}%
\pgfpathlineto{\pgfqpoint{1.814342in}{0.776503in}}%
\pgfpathlineto{\pgfqpoint{1.807003in}{0.773486in}}%
\pgfpathlineto{\pgfqpoint{1.801141in}{0.779025in}}%
\pgfpathlineto{\pgfqpoint{1.801746in}{0.816372in}}%
\pgfpathlineto{\pgfqpoint{1.803058in}{0.895889in}}%
\pgfpathlineto{\pgfqpoint{1.796697in}{0.942452in}}%
\pgfpathlineto{\pgfqpoint{1.792603in}{0.967662in}}%
\pgfusepath{stroke}%
\end{pgfscope}%
\begin{pgfscope}%
\pgfpathrectangle{\pgfqpoint{0.100000in}{0.100000in}}{\pgfqpoint{3.007045in}{1.925000in}}%
\pgfusepath{clip}%
\pgfsetbuttcap%
\pgfsetroundjoin%
\pgfsetlinewidth{0.050187pt}%
\definecolor{currentstroke}{rgb}{1.000000,1.000000,1.000000}%
\pgfsetstrokecolor{currentstroke}%
\pgfsetdash{}{0pt}%
\pgfpathmoveto{\pgfqpoint{1.825680in}{0.738936in}}%
\pgfpathlineto{\pgfqpoint{1.870456in}{0.739340in}}%
\pgfpathlineto{\pgfqpoint{1.935375in}{0.740410in}}%
\pgfpathlineto{\pgfqpoint{1.985381in}{0.742474in}}%
\pgfpathlineto{\pgfqpoint{1.982985in}{0.737257in}}%
\pgfpathlineto{\pgfqpoint{1.989303in}{0.708523in}}%
\pgfpathlineto{\pgfqpoint{1.998033in}{0.695971in}}%
\pgfpathlineto{\pgfqpoint{1.987913in}{0.689038in}}%
\pgfpathlineto{\pgfqpoint{1.993202in}{0.676993in}}%
\pgfpathlineto{\pgfqpoint{1.978931in}{0.664590in}}%
\pgfpathlineto{\pgfqpoint{1.974665in}{0.645765in}}%
\pgfpathlineto{\pgfqpoint{1.966867in}{0.636068in}}%
\pgfpathlineto{\pgfqpoint{1.967860in}{0.626069in}}%
\pgfpathlineto{\pgfqpoint{1.963386in}{0.623948in}}%
\pgfpathlineto{\pgfqpoint{1.967876in}{0.613540in}}%
\pgfpathlineto{\pgfqpoint{1.964244in}{0.608030in}}%
\pgfpathlineto{\pgfqpoint{2.025296in}{0.610677in}}%
\pgfpathlineto{\pgfqpoint{2.072380in}{0.613576in}}%
\pgfpathlineto{\pgfqpoint{2.067411in}{0.591438in}}%
\pgfpathlineto{\pgfqpoint{2.070801in}{0.583259in}}%
\pgfpathlineto{\pgfqpoint{2.077828in}{0.576424in}}%
\pgfpathlineto{\pgfqpoint{2.084645in}{0.560117in}}%
\pgfpathlineto{\pgfqpoint{2.063113in}{0.563874in}}%
\pgfpathlineto{\pgfqpoint{2.055170in}{0.570050in}}%
\pgfpathlineto{\pgfqpoint{2.045706in}{0.570344in}}%
\pgfpathlineto{\pgfqpoint{2.035757in}{0.556820in}}%
\pgfpathlineto{\pgfqpoint{2.037740in}{0.550661in}}%
\pgfpathlineto{\pgfqpoint{2.069524in}{0.546935in}}%
\pgfpathlineto{\pgfqpoint{2.077845in}{0.539855in}}%
\pgfpathlineto{\pgfqpoint{2.091386in}{0.536213in}}%
\pgfpathlineto{\pgfqpoint{2.085472in}{0.527832in}}%
\pgfpathlineto{\pgfqpoint{2.075509in}{0.520522in}}%
\pgfpathlineto{\pgfqpoint{2.101808in}{0.504085in}}%
\pgfpathlineto{\pgfqpoint{2.114058in}{0.501516in}}%
\pgfpathlineto{\pgfqpoint{2.119152in}{0.488080in}}%
\pgfpathlineto{\pgfqpoint{2.107618in}{0.485609in}}%
\pgfpathlineto{\pgfqpoint{2.085940in}{0.502507in}}%
\pgfpathlineto{\pgfqpoint{2.077465in}{0.504856in}}%
\pgfpathlineto{\pgfqpoint{2.073363in}{0.511532in}}%
\pgfpathlineto{\pgfqpoint{2.060791in}{0.508792in}}%
\pgfpathlineto{\pgfqpoint{2.056855in}{0.500190in}}%
\pgfpathlineto{\pgfqpoint{2.059348in}{0.490604in}}%
\pgfpathlineto{\pgfqpoint{2.050884in}{0.484948in}}%
\pgfpathlineto{\pgfqpoint{2.043160in}{0.498883in}}%
\pgfpathlineto{\pgfqpoint{2.029631in}{0.494728in}}%
\pgfpathlineto{\pgfqpoint{2.024562in}{0.486414in}}%
\pgfpathlineto{\pgfqpoint{2.014953in}{0.488779in}}%
\pgfpathlineto{\pgfqpoint{2.008810in}{0.499279in}}%
\pgfpathlineto{\pgfqpoint{1.992513in}{0.502746in}}%
\pgfpathlineto{\pgfqpoint{1.989436in}{0.508210in}}%
\pgfpathlineto{\pgfqpoint{1.972456in}{0.517727in}}%
\pgfpathlineto{\pgfqpoint{1.968241in}{0.526049in}}%
\pgfpathlineto{\pgfqpoint{1.954018in}{0.522648in}}%
\pgfpathlineto{\pgfqpoint{1.955825in}{0.530267in}}%
\pgfpathlineto{\pgfqpoint{1.938148in}{0.522451in}}%
\pgfpathlineto{\pgfqpoint{1.942896in}{0.514343in}}%
\pgfpathlineto{\pgfqpoint{1.929245in}{0.509548in}}%
\pgfpathlineto{\pgfqpoint{1.911163in}{0.512183in}}%
\pgfpathlineto{\pgfqpoint{1.874567in}{0.524719in}}%
\pgfpathlineto{\pgfqpoint{1.846328in}{0.522231in}}%
\pgfpathlineto{\pgfqpoint{1.843881in}{0.538697in}}%
\pgfpathlineto{\pgfqpoint{1.847427in}{0.546210in}}%
\pgfpathlineto{\pgfqpoint{1.847123in}{0.558165in}}%
\pgfpathlineto{\pgfqpoint{1.843635in}{0.561013in}}%
\pgfpathlineto{\pgfqpoint{1.844817in}{0.574659in}}%
\pgfpathlineto{\pgfqpoint{1.855102in}{0.593637in}}%
\pgfpathlineto{\pgfqpoint{1.856418in}{0.600809in}}%
\pgfpathlineto{\pgfqpoint{1.854722in}{0.617746in}}%
\pgfpathlineto{\pgfqpoint{1.847157in}{0.625724in}}%
\pgfpathlineto{\pgfqpoint{1.843206in}{0.640621in}}%
\pgfpathlineto{\pgfqpoint{1.838345in}{0.644041in}}%
\pgfpathlineto{\pgfqpoint{1.840521in}{0.652119in}}%
\pgfpathlineto{\pgfqpoint{1.834331in}{0.664193in}}%
\pgfpathlineto{\pgfqpoint{1.826608in}{0.670746in}}%
\pgfpathlineto{\pgfqpoint{1.825680in}{0.738936in}}%
\pgfusepath{stroke}%
\end{pgfscope}%
\begin{pgfscope}%
\pgfpathrectangle{\pgfqpoint{0.100000in}{0.100000in}}{\pgfqpoint{3.007045in}{1.925000in}}%
\pgfusepath{clip}%
\pgfsetbuttcap%
\pgfsetroundjoin%
\pgfsetlinewidth{0.050187pt}%
\definecolor{currentstroke}{rgb}{1.000000,1.000000,1.000000}%
\pgfsetstrokecolor{currentstroke}%
\pgfsetdash{}{0pt}%
\pgfpathmoveto{\pgfqpoint{1.945436in}{0.513722in}}%
\pgfpathlineto{\pgfqpoint{1.951913in}{0.517575in}}%
\pgfpathlineto{\pgfqpoint{1.959776in}{0.513019in}}%
\pgfpathlineto{\pgfqpoint{1.955391in}{0.506751in}}%
\pgfpathlineto{\pgfqpoint{1.945436in}{0.513722in}}%
\pgfusepath{stroke}%
\end{pgfscope}%
\begin{pgfscope}%
\pgfpathrectangle{\pgfqpoint{0.100000in}{0.100000in}}{\pgfqpoint{3.007045in}{1.925000in}}%
\pgfusepath{clip}%
\pgfsetbuttcap%
\pgfsetroundjoin%
\pgfsetlinewidth{0.050187pt}%
\definecolor{currentstroke}{rgb}{1.000000,1.000000,1.000000}%
\pgfsetstrokecolor{currentstroke}%
\pgfsetdash{}{0pt}%
\pgfpathmoveto{\pgfqpoint{2.206853in}{0.588027in}}%
\pgfpathlineto{\pgfqpoint{2.206971in}{0.600484in}}%
\pgfpathlineto{\pgfqpoint{2.199230in}{0.605215in}}%
\pgfpathlineto{\pgfqpoint{2.192889in}{0.613268in}}%
\pgfpathlineto{\pgfqpoint{2.193752in}{0.621728in}}%
\pgfpathlineto{\pgfqpoint{2.262793in}{0.627088in}}%
\pgfpathlineto{\pgfqpoint{2.341115in}{0.635787in}}%
\pgfpathlineto{\pgfqpoint{2.351105in}{0.617581in}}%
\pgfpathlineto{\pgfqpoint{2.384322in}{0.619919in}}%
\pgfpathlineto{\pgfqpoint{2.450185in}{0.623513in}}%
\pgfpathlineto{\pgfqpoint{2.502419in}{0.626994in}}%
\pgfpathlineto{\pgfqpoint{2.507549in}{0.613881in}}%
\pgfpathlineto{\pgfqpoint{2.513890in}{0.614955in}}%
\pgfpathlineto{\pgfqpoint{2.514596in}{0.628996in}}%
\pgfpathlineto{\pgfqpoint{2.511350in}{0.641072in}}%
\pgfpathlineto{\pgfqpoint{2.515700in}{0.646184in}}%
\pgfpathlineto{\pgfqpoint{2.525751in}{0.643434in}}%
\pgfpathlineto{\pgfqpoint{2.540180in}{0.642869in}}%
\pgfpathlineto{\pgfqpoint{2.542399in}{0.632080in}}%
\pgfpathlineto{\pgfqpoint{2.546837in}{0.625915in}}%
\pgfpathlineto{\pgfqpoint{2.550303in}{0.612426in}}%
\pgfpathlineto{\pgfqpoint{2.561042in}{0.591552in}}%
\pgfpathlineto{\pgfqpoint{2.561096in}{0.585906in}}%
\pgfpathlineto{\pgfqpoint{2.576968in}{0.561396in}}%
\pgfpathlineto{\pgfqpoint{2.578503in}{0.556332in}}%
\pgfpathlineto{\pgfqpoint{2.602314in}{0.521715in}}%
\pgfpathlineto{\pgfqpoint{2.598541in}{0.521156in}}%
\pgfpathlineto{\pgfqpoint{2.608579in}{0.496534in}}%
\pgfpathlineto{\pgfqpoint{2.636155in}{0.453729in}}%
\pgfpathlineto{\pgfqpoint{2.652165in}{0.422256in}}%
\pgfpathlineto{\pgfqpoint{2.657516in}{0.416746in}}%
\pgfpathlineto{\pgfqpoint{2.666685in}{0.396991in}}%
\pgfpathlineto{\pgfqpoint{2.669866in}{0.365359in}}%
\pgfpathlineto{\pgfqpoint{2.671132in}{0.341707in}}%
\pgfpathlineto{\pgfqpoint{2.669603in}{0.326563in}}%
\pgfpathlineto{\pgfqpoint{2.664687in}{0.315737in}}%
\pgfpathlineto{\pgfqpoint{2.666971in}{0.301540in}}%
\pgfpathlineto{\pgfqpoint{2.661678in}{0.290303in}}%
\pgfpathlineto{\pgfqpoint{2.645962in}{0.281088in}}%
\pgfpathlineto{\pgfqpoint{2.635740in}{0.281761in}}%
\pgfpathlineto{\pgfqpoint{2.629103in}{0.276944in}}%
\pgfpathlineto{\pgfqpoint{2.627201in}{0.289805in}}%
\pgfpathlineto{\pgfqpoint{2.616214in}{0.293200in}}%
\pgfpathlineto{\pgfqpoint{2.606364in}{0.311145in}}%
\pgfpathlineto{\pgfqpoint{2.605252in}{0.319317in}}%
\pgfpathlineto{\pgfqpoint{2.587624in}{0.324280in}}%
\pgfpathlineto{\pgfqpoint{2.576265in}{0.323205in}}%
\pgfpathlineto{\pgfqpoint{2.569838in}{0.335063in}}%
\pgfpathlineto{\pgfqpoint{2.562480in}{0.356375in}}%
\pgfpathlineto{\pgfqpoint{2.552268in}{0.360695in}}%
\pgfpathlineto{\pgfqpoint{2.546732in}{0.372908in}}%
\pgfpathlineto{\pgfqpoint{2.533193in}{0.380106in}}%
\pgfpathlineto{\pgfqpoint{2.525367in}{0.389110in}}%
\pgfpathlineto{\pgfqpoint{2.511867in}{0.413587in}}%
\pgfpathlineto{\pgfqpoint{2.510279in}{0.430815in}}%
\pgfpathlineto{\pgfqpoint{2.517636in}{0.441838in}}%
\pgfpathlineto{\pgfqpoint{2.516901in}{0.449497in}}%
\pgfpathlineto{\pgfqpoint{2.508399in}{0.450316in}}%
\pgfpathlineto{\pgfqpoint{2.501297in}{0.455644in}}%
\pgfpathlineto{\pgfqpoint{2.497407in}{0.449138in}}%
\pgfpathlineto{\pgfqpoint{2.504224in}{0.443837in}}%
\pgfpathlineto{\pgfqpoint{2.499262in}{0.434305in}}%
\pgfpathlineto{\pgfqpoint{2.491236in}{0.442217in}}%
\pgfpathlineto{\pgfqpoint{2.492167in}{0.464156in}}%
\pgfpathlineto{\pgfqpoint{2.496039in}{0.481956in}}%
\pgfpathlineto{\pgfqpoint{2.494048in}{0.512497in}}%
\pgfpathlineto{\pgfqpoint{2.486048in}{0.519772in}}%
\pgfpathlineto{\pgfqpoint{2.482024in}{0.529107in}}%
\pgfpathlineto{\pgfqpoint{2.468242in}{0.528899in}}%
\pgfpathlineto{\pgfqpoint{2.454618in}{0.544259in}}%
\pgfpathlineto{\pgfqpoint{2.445483in}{0.548863in}}%
\pgfpathlineto{\pgfqpoint{2.442774in}{0.558590in}}%
\pgfpathlineto{\pgfqpoint{2.433840in}{0.562006in}}%
\pgfpathlineto{\pgfqpoint{2.426435in}{0.572746in}}%
\pgfpathlineto{\pgfqpoint{2.406875in}{0.581513in}}%
\pgfpathlineto{\pgfqpoint{2.391668in}{0.581737in}}%
\pgfpathlineto{\pgfqpoint{2.385048in}{0.578370in}}%
\pgfpathlineto{\pgfqpoint{2.386693in}{0.567859in}}%
\pgfpathlineto{\pgfqpoint{2.379790in}{0.568357in}}%
\pgfpathlineto{\pgfqpoint{2.358555in}{0.553661in}}%
\pgfpathlineto{\pgfqpoint{2.333048in}{0.547836in}}%
\pgfpathlineto{\pgfqpoint{2.332629in}{0.555043in}}%
\pgfpathlineto{\pgfqpoint{2.326976in}{0.562087in}}%
\pgfpathlineto{\pgfqpoint{2.311810in}{0.571811in}}%
\pgfpathlineto{\pgfqpoint{2.290041in}{0.581781in}}%
\pgfpathlineto{\pgfqpoint{2.266395in}{0.587059in}}%
\pgfpathlineto{\pgfqpoint{2.261956in}{0.594236in}}%
\pgfpathlineto{\pgfqpoint{2.253431in}{0.588250in}}%
\pgfpathlineto{\pgfqpoint{2.243165in}{0.586930in}}%
\pgfpathlineto{\pgfqpoint{2.207429in}{0.577514in}}%
\pgfpathlineto{\pgfqpoint{2.206853in}{0.588027in}}%
\pgfusepath{stroke}%
\end{pgfscope}%
\begin{pgfscope}%
\pgfpathrectangle{\pgfqpoint{0.100000in}{0.100000in}}{\pgfqpoint{3.007045in}{1.925000in}}%
\pgfusepath{clip}%
\pgfsetbuttcap%
\pgfsetroundjoin%
\pgfsetlinewidth{0.050187pt}%
\definecolor{currentstroke}{rgb}{1.000000,1.000000,1.000000}%
\pgfsetstrokecolor{currentstroke}%
\pgfsetdash{}{0pt}%
\pgfpathmoveto{\pgfqpoint{2.605783in}{0.522207in}}%
\pgfpathlineto{\pgfqpoint{2.615029in}{0.511452in}}%
\pgfpathlineto{\pgfqpoint{2.604349in}{0.510775in}}%
\pgfpathlineto{\pgfqpoint{2.605783in}{0.522207in}}%
\pgfusepath{stroke}%
\end{pgfscope}%
\begin{pgfscope}%
\pgfpathrectangle{\pgfqpoint{0.100000in}{0.100000in}}{\pgfqpoint{3.007045in}{1.925000in}}%
\pgfusepath{clip}%
\pgfsetbuttcap%
\pgfsetroundjoin%
\pgfsetlinewidth{0.050187pt}%
\definecolor{currentstroke}{rgb}{1.000000,1.000000,1.000000}%
\pgfsetstrokecolor{currentstroke}%
\pgfsetdash{}{0pt}%
\pgfpathmoveto{\pgfqpoint{2.061492in}{1.750249in}}%
\pgfpathlineto{\pgfqpoint{2.056476in}{1.740485in}}%
\pgfpathlineto{\pgfqpoint{2.044508in}{1.734787in}}%
\pgfpathlineto{\pgfqpoint{2.039325in}{1.727117in}}%
\pgfpathlineto{\pgfqpoint{2.033253in}{1.732641in}}%
\pgfpathlineto{\pgfqpoint{2.061492in}{1.750249in}}%
\pgfusepath{stroke}%
\end{pgfscope}%
\begin{pgfscope}%
\pgfpathrectangle{\pgfqpoint{0.100000in}{0.100000in}}{\pgfqpoint{3.007045in}{1.925000in}}%
\pgfusepath{clip}%
\pgfsetbuttcap%
\pgfsetroundjoin%
\pgfsetlinewidth{0.050187pt}%
\definecolor{currentstroke}{rgb}{1.000000,1.000000,1.000000}%
\pgfsetstrokecolor{currentstroke}%
\pgfsetdash{}{0pt}%
\pgfpathmoveto{\pgfqpoint{2.065561in}{1.691489in}}%
\pgfpathlineto{\pgfqpoint{2.077807in}{1.702912in}}%
\pgfpathlineto{\pgfqpoint{2.096694in}{1.705904in}}%
\pgfpathlineto{\pgfqpoint{2.091492in}{1.697959in}}%
\pgfpathlineto{\pgfqpoint{2.078543in}{1.686470in}}%
\pgfpathlineto{\pgfqpoint{2.070973in}{1.671723in}}%
\pgfpathlineto{\pgfqpoint{2.067767in}{1.679722in}}%
\pgfpathlineto{\pgfqpoint{2.062075in}{1.680870in}}%
\pgfpathlineto{\pgfqpoint{2.061503in}{1.688091in}}%
\pgfpathlineto{\pgfqpoint{2.065561in}{1.691489in}}%
\pgfusepath{stroke}%
\end{pgfscope}%
\begin{pgfscope}%
\pgfpathrectangle{\pgfqpoint{0.100000in}{0.100000in}}{\pgfqpoint{3.007045in}{1.925000in}}%
\pgfusepath{clip}%
\pgfsetbuttcap%
\pgfsetroundjoin%
\pgfsetlinewidth{0.050187pt}%
\definecolor{currentstroke}{rgb}{1.000000,1.000000,1.000000}%
\pgfsetstrokecolor{currentstroke}%
\pgfsetdash{}{0pt}%
\pgfpathmoveto{\pgfqpoint{2.114334in}{1.552060in}}%
\pgfpathlineto{\pgfqpoint{2.111072in}{1.555679in}}%
\pgfpathlineto{\pgfqpoint{2.114501in}{1.565867in}}%
\pgfpathlineto{\pgfqpoint{2.104324in}{1.566517in}}%
\pgfpathlineto{\pgfqpoint{2.107021in}{1.575301in}}%
\pgfpathlineto{\pgfqpoint{2.105344in}{1.589291in}}%
\pgfpathlineto{\pgfqpoint{2.096212in}{1.594142in}}%
\pgfpathlineto{\pgfqpoint{2.086656in}{1.603958in}}%
\pgfpathlineto{\pgfqpoint{2.071931in}{1.606832in}}%
\pgfpathlineto{\pgfqpoint{2.057600in}{1.606750in}}%
\pgfpathlineto{\pgfqpoint{2.043506in}{1.613716in}}%
\pgfpathlineto{\pgfqpoint{1.996352in}{1.623871in}}%
\pgfpathlineto{\pgfqpoint{1.991199in}{1.634631in}}%
\pgfpathlineto{\pgfqpoint{1.982007in}{1.638302in}}%
\pgfpathlineto{\pgfqpoint{1.999369in}{1.646540in}}%
\pgfpathlineto{\pgfqpoint{2.009177in}{1.656818in}}%
\pgfpathlineto{\pgfqpoint{2.027446in}{1.659605in}}%
\pgfpathlineto{\pgfqpoint{2.038687in}{1.670074in}}%
\pgfpathlineto{\pgfqpoint{2.044585in}{1.670493in}}%
\pgfpathlineto{\pgfqpoint{2.049086in}{1.677958in}}%
\pgfpathlineto{\pgfqpoint{2.060059in}{1.686803in}}%
\pgfpathlineto{\pgfqpoint{2.061015in}{1.680552in}}%
\pgfpathlineto{\pgfqpoint{2.065959in}{1.679261in}}%
\pgfpathlineto{\pgfqpoint{2.069679in}{1.671805in}}%
\pgfpathlineto{\pgfqpoint{2.070349in}{1.659095in}}%
\pgfpathlineto{\pgfqpoint{2.081529in}{1.666691in}}%
\pgfpathlineto{\pgfqpoint{2.094557in}{1.668277in}}%
\pgfpathlineto{\pgfqpoint{2.105677in}{1.664298in}}%
\pgfpathlineto{\pgfqpoint{2.120759in}{1.643609in}}%
\pgfpathlineto{\pgfqpoint{2.137230in}{1.646915in}}%
\pgfpathlineto{\pgfqpoint{2.143923in}{1.641378in}}%
\pgfpathlineto{\pgfqpoint{2.154687in}{1.640884in}}%
\pgfpathlineto{\pgfqpoint{2.161839in}{1.650814in}}%
\pgfpathlineto{\pgfqpoint{2.175410in}{1.659592in}}%
\pgfpathlineto{\pgfqpoint{2.188452in}{1.662342in}}%
\pgfpathlineto{\pgfqpoint{2.204637in}{1.662628in}}%
\pgfpathlineto{\pgfqpoint{2.216463in}{1.669409in}}%
\pgfpathlineto{\pgfqpoint{2.226115in}{1.666285in}}%
\pgfpathlineto{\pgfqpoint{2.228171in}{1.651944in}}%
\pgfpathlineto{\pgfqpoint{2.248877in}{1.649698in}}%
\pgfpathlineto{\pgfqpoint{2.260125in}{1.656411in}}%
\pgfpathlineto{\pgfqpoint{2.267921in}{1.641288in}}%
\pgfpathlineto{\pgfqpoint{2.282798in}{1.623800in}}%
\pgfpathlineto{\pgfqpoint{2.261976in}{1.623901in}}%
\pgfpathlineto{\pgfqpoint{2.255397in}{1.621740in}}%
\pgfpathlineto{\pgfqpoint{2.246379in}{1.624549in}}%
\pgfpathlineto{\pgfqpoint{2.245772in}{1.612443in}}%
\pgfpathlineto{\pgfqpoint{2.229399in}{1.621904in}}%
\pgfpathlineto{\pgfqpoint{2.208341in}{1.624795in}}%
\pgfpathlineto{\pgfqpoint{2.202544in}{1.615581in}}%
\pgfpathlineto{\pgfqpoint{2.174992in}{1.611101in}}%
\pgfpathlineto{\pgfqpoint{2.171822in}{1.603295in}}%
\pgfpathlineto{\pgfqpoint{2.152650in}{1.600953in}}%
\pgfpathlineto{\pgfqpoint{2.146854in}{1.593019in}}%
\pgfpathlineto{\pgfqpoint{2.136713in}{1.590891in}}%
\pgfpathlineto{\pgfqpoint{2.128613in}{1.572076in}}%
\pgfpathlineto{\pgfqpoint{2.118357in}{1.553854in}}%
\pgfpathlineto{\pgfqpoint{2.114334in}{1.552060in}}%
\pgfusepath{stroke}%
\end{pgfscope}%
\begin{pgfscope}%
\pgfpathrectangle{\pgfqpoint{0.100000in}{0.100000in}}{\pgfqpoint{3.007045in}{1.925000in}}%
\pgfusepath{clip}%
\pgfsetbuttcap%
\pgfsetroundjoin%
\pgfsetlinewidth{0.050187pt}%
\definecolor{currentstroke}{rgb}{1.000000,1.000000,1.000000}%
\pgfsetstrokecolor{currentstroke}%
\pgfsetdash{}{0pt}%
\pgfpathmoveto{\pgfqpoint{2.173359in}{1.333287in}}%
\pgfpathlineto{\pgfqpoint{2.183137in}{1.343579in}}%
\pgfpathlineto{\pgfqpoint{2.187591in}{1.358500in}}%
\pgfpathlineto{\pgfqpoint{2.192889in}{1.367149in}}%
\pgfpathlineto{\pgfqpoint{2.196137in}{1.378919in}}%
\pgfpathlineto{\pgfqpoint{2.197147in}{1.402387in}}%
\pgfpathlineto{\pgfqpoint{2.192256in}{1.424878in}}%
\pgfpathlineto{\pgfqpoint{2.176094in}{1.459319in}}%
\pgfpathlineto{\pgfqpoint{2.180471in}{1.470147in}}%
\pgfpathlineto{\pgfqpoint{2.174769in}{1.485115in}}%
\pgfpathlineto{\pgfqpoint{2.184550in}{1.505816in}}%
\pgfpathlineto{\pgfqpoint{2.182956in}{1.528897in}}%
\pgfpathlineto{\pgfqpoint{2.189769in}{1.531802in}}%
\pgfpathlineto{\pgfqpoint{2.190625in}{1.542784in}}%
\pgfpathlineto{\pgfqpoint{2.202718in}{1.549833in}}%
\pgfpathlineto{\pgfqpoint{2.209558in}{1.548695in}}%
\pgfpathlineto{\pgfqpoint{2.211480in}{1.536917in}}%
\pgfpathlineto{\pgfqpoint{2.216819in}{1.536455in}}%
\pgfpathlineto{\pgfqpoint{2.221689in}{1.553444in}}%
\pgfpathlineto{\pgfqpoint{2.220034in}{1.566657in}}%
\pgfpathlineto{\pgfqpoint{2.223201in}{1.574245in}}%
\pgfpathlineto{\pgfqpoint{2.240322in}{1.582083in}}%
\pgfpathlineto{\pgfqpoint{2.232520in}{1.584872in}}%
\pgfpathlineto{\pgfqpoint{2.229979in}{1.591605in}}%
\pgfpathlineto{\pgfqpoint{2.235599in}{1.603430in}}%
\pgfpathlineto{\pgfqpoint{2.246683in}{1.607514in}}%
\pgfpathlineto{\pgfqpoint{2.259554in}{1.600537in}}%
\pgfpathlineto{\pgfqpoint{2.271689in}{1.600450in}}%
\pgfpathlineto{\pgfqpoint{2.277325in}{1.592301in}}%
\pgfpathlineto{\pgfqpoint{2.285823in}{1.592867in}}%
\pgfpathlineto{\pgfqpoint{2.312058in}{1.581538in}}%
\pgfpathlineto{\pgfqpoint{2.317300in}{1.570571in}}%
\pgfpathlineto{\pgfqpoint{2.311575in}{1.566773in}}%
\pgfpathlineto{\pgfqpoint{2.313339in}{1.558555in}}%
\pgfpathlineto{\pgfqpoint{2.318994in}{1.554899in}}%
\pgfpathlineto{\pgfqpoint{2.322146in}{1.544796in}}%
\pgfpathlineto{\pgfqpoint{2.321705in}{1.520181in}}%
\pgfpathlineto{\pgfqpoint{2.314238in}{1.514301in}}%
\pgfpathlineto{\pgfqpoint{2.312562in}{1.501372in}}%
\pgfpathlineto{\pgfqpoint{2.298712in}{1.489423in}}%
\pgfpathlineto{\pgfqpoint{2.299534in}{1.474960in}}%
\pgfpathlineto{\pgfqpoint{2.311673in}{1.469893in}}%
\pgfpathlineto{\pgfqpoint{2.325368in}{1.487939in}}%
\pgfpathlineto{\pgfqpoint{2.326500in}{1.494484in}}%
\pgfpathlineto{\pgfqpoint{2.343579in}{1.505363in}}%
\pgfpathlineto{\pgfqpoint{2.354440in}{1.500324in}}%
\pgfpathlineto{\pgfqpoint{2.361256in}{1.488936in}}%
\pgfpathlineto{\pgfqpoint{2.372251in}{1.449395in}}%
\pgfpathlineto{\pgfqpoint{2.378077in}{1.436886in}}%
\pgfpathlineto{\pgfqpoint{2.376253in}{1.431715in}}%
\pgfpathlineto{\pgfqpoint{2.376430in}{1.414106in}}%
\pgfpathlineto{\pgfqpoint{2.365832in}{1.415796in}}%
\pgfpathlineto{\pgfqpoint{2.359845in}{1.402637in}}%
\pgfpathlineto{\pgfqpoint{2.359015in}{1.393692in}}%
\pgfpathlineto{\pgfqpoint{2.351009in}{1.387949in}}%
\pgfpathlineto{\pgfqpoint{2.349230in}{1.370503in}}%
\pgfpathlineto{\pgfqpoint{2.337587in}{1.348464in}}%
\pgfpathlineto{\pgfqpoint{2.273898in}{1.338979in}}%
\pgfpathlineto{\pgfqpoint{2.273522in}{1.343148in}}%
\pgfpathlineto{\pgfqpoint{2.230918in}{1.338658in}}%
\pgfpathlineto{\pgfqpoint{2.173359in}{1.333287in}}%
\pgfusepath{stroke}%
\end{pgfscope}%
\begin{pgfscope}%
\pgfpathrectangle{\pgfqpoint{0.100000in}{0.100000in}}{\pgfqpoint{3.007045in}{1.925000in}}%
\pgfusepath{clip}%
\pgfsetbuttcap%
\pgfsetroundjoin%
\pgfsetlinewidth{0.050187pt}%
\definecolor{currentstroke}{rgb}{1.000000,1.000000,1.000000}%
\pgfsetstrokecolor{currentstroke}%
\pgfsetdash{}{0pt}%
\pgfpathmoveto{\pgfqpoint{0.000000in}{0.000000in}}%
\pgfusepath{stroke}%
\end{pgfscope}%
\begin{pgfscope}%
\pgfpathrectangle{\pgfqpoint{0.100000in}{0.100000in}}{\pgfqpoint{3.007045in}{1.925000in}}%
\pgfusepath{clip}%
\pgfsetbuttcap%
\pgfsetroundjoin%
\pgfsetlinewidth{0.050187pt}%
\definecolor{currentstroke}{rgb}{1.000000,1.000000,1.000000}%
\pgfsetstrokecolor{currentstroke}%
\pgfsetdash{}{0pt}%
\pgfusepath{stroke}%
\end{pgfscope}%
\begin{pgfscope}%
\pgfpathrectangle{\pgfqpoint{0.100000in}{0.100000in}}{\pgfqpoint{3.007045in}{1.925000in}}%
\pgfusepath{clip}%
\pgfsetbuttcap%
\pgfsetroundjoin%
\pgfsetlinewidth{0.050187pt}%
\definecolor{currentstroke}{rgb}{1.000000,1.000000,1.000000}%
\pgfsetstrokecolor{currentstroke}%
\pgfsetdash{}{0pt}%
\pgfusepath{stroke}%
\end{pgfscope}%
\begin{pgfscope}%
\pgfpathrectangle{\pgfqpoint{0.100000in}{0.100000in}}{\pgfqpoint{3.007045in}{1.925000in}}%
\pgfusepath{clip}%
\pgfsetbuttcap%
\pgfsetroundjoin%
\pgfsetlinewidth{0.050187pt}%
\definecolor{currentstroke}{rgb}{1.000000,1.000000,1.000000}%
\pgfsetstrokecolor{currentstroke}%
\pgfsetdash{}{0pt}%
\pgfusepath{stroke}%
\end{pgfscope}%
\begin{pgfscope}%
\pgfpathrectangle{\pgfqpoint{0.100000in}{0.100000in}}{\pgfqpoint{3.007045in}{1.925000in}}%
\pgfusepath{clip}%
\pgfsetbuttcap%
\pgfsetroundjoin%
\pgfsetlinewidth{0.050187pt}%
\definecolor{currentstroke}{rgb}{1.000000,1.000000,1.000000}%
\pgfsetstrokecolor{currentstroke}%
\pgfsetdash{}{0pt}%
\pgfusepath{stroke}%
\end{pgfscope}%
\begin{pgfscope}%
\pgfpathrectangle{\pgfqpoint{0.100000in}{0.100000in}}{\pgfqpoint{3.007045in}{1.925000in}}%
\pgfusepath{clip}%
\pgfsetbuttcap%
\pgfsetroundjoin%
\pgfsetlinewidth{0.050187pt}%
\definecolor{currentstroke}{rgb}{1.000000,1.000000,1.000000}%
\pgfsetstrokecolor{currentstroke}%
\pgfsetdash{}{0pt}%
\pgfusepath{stroke}%
\end{pgfscope}%
\begin{pgfscope}%
\pgfpathrectangle{\pgfqpoint{0.100000in}{0.100000in}}{\pgfqpoint{3.007045in}{1.925000in}}%
\pgfusepath{clip}%
\pgfsetbuttcap%
\pgfsetroundjoin%
\pgfsetlinewidth{0.050187pt}%
\definecolor{currentstroke}{rgb}{1.000000,1.000000,1.000000}%
\pgfsetstrokecolor{currentstroke}%
\pgfsetdash{}{0pt}%
\pgfusepath{stroke}%
\end{pgfscope}%
\begin{pgfscope}%
\pgfpathrectangle{\pgfqpoint{0.100000in}{0.100000in}}{\pgfqpoint{3.007045in}{1.925000in}}%
\pgfusepath{clip}%
\pgfsetbuttcap%
\pgfsetroundjoin%
\pgfsetlinewidth{0.050187pt}%
\definecolor{currentstroke}{rgb}{1.000000,1.000000,1.000000}%
\pgfsetstrokecolor{currentstroke}%
\pgfsetdash{}{0pt}%
\pgfusepath{stroke}%
\end{pgfscope}%
\begin{pgfscope}%
\pgfpathrectangle{\pgfqpoint{0.100000in}{0.100000in}}{\pgfqpoint{3.007045in}{1.925000in}}%
\pgfusepath{clip}%
\pgfsetbuttcap%
\pgfsetroundjoin%
\pgfsetlinewidth{0.050187pt}%
\definecolor{currentstroke}{rgb}{1.000000,1.000000,1.000000}%
\pgfsetstrokecolor{currentstroke}%
\pgfsetdash{}{0pt}%
\pgfusepath{stroke}%
\end{pgfscope}%
\begin{pgfscope}%
\pgfpathrectangle{\pgfqpoint{0.100000in}{0.100000in}}{\pgfqpoint{3.007045in}{1.925000in}}%
\pgfusepath{clip}%
\pgfsetbuttcap%
\pgfsetroundjoin%
\pgfsetlinewidth{0.050187pt}%
\definecolor{currentstroke}{rgb}{1.000000,1.000000,1.000000}%
\pgfsetstrokecolor{currentstroke}%
\pgfsetdash{}{0pt}%
\pgfusepath{stroke}%
\end{pgfscope}%
\begin{pgfscope}%
\pgfpathrectangle{\pgfqpoint{0.100000in}{0.100000in}}{\pgfqpoint{3.007045in}{1.925000in}}%
\pgfusepath{clip}%
\pgfsetbuttcap%
\pgfsetroundjoin%
\pgfsetlinewidth{0.050187pt}%
\definecolor{currentstroke}{rgb}{1.000000,1.000000,1.000000}%
\pgfsetstrokecolor{currentstroke}%
\pgfsetdash{}{0pt}%
\pgfusepath{stroke}%
\end{pgfscope}%
\begin{pgfscope}%
\pgfpathrectangle{\pgfqpoint{0.100000in}{0.100000in}}{\pgfqpoint{3.007045in}{1.925000in}}%
\pgfusepath{clip}%
\pgfsetbuttcap%
\pgfsetroundjoin%
\pgfsetlinewidth{0.050187pt}%
\definecolor{currentstroke}{rgb}{1.000000,1.000000,1.000000}%
\pgfsetstrokecolor{currentstroke}%
\pgfsetdash{}{0pt}%
\pgfusepath{stroke}%
\end{pgfscope}%
\begin{pgfscope}%
\pgfpathrectangle{\pgfqpoint{0.100000in}{0.100000in}}{\pgfqpoint{3.007045in}{1.925000in}}%
\pgfusepath{clip}%
\pgfsetbuttcap%
\pgfsetroundjoin%
\pgfsetlinewidth{0.050187pt}%
\definecolor{currentstroke}{rgb}{1.000000,1.000000,1.000000}%
\pgfsetstrokecolor{currentstroke}%
\pgfsetdash{}{0pt}%
\pgfusepath{stroke}%
\end{pgfscope}%
\begin{pgfscope}%
\pgfpathrectangle{\pgfqpoint{0.100000in}{0.100000in}}{\pgfqpoint{3.007045in}{1.925000in}}%
\pgfusepath{clip}%
\pgfsetbuttcap%
\pgfsetroundjoin%
\pgfsetlinewidth{0.050187pt}%
\definecolor{currentstroke}{rgb}{1.000000,1.000000,1.000000}%
\pgfsetstrokecolor{currentstroke}%
\pgfsetdash{}{0pt}%
\pgfusepath{stroke}%
\end{pgfscope}%
\begin{pgfscope}%
\pgfpathrectangle{\pgfqpoint{0.100000in}{0.100000in}}{\pgfqpoint{3.007045in}{1.925000in}}%
\pgfusepath{clip}%
\pgfsetbuttcap%
\pgfsetroundjoin%
\pgfsetlinewidth{0.050187pt}%
\definecolor{currentstroke}{rgb}{1.000000,1.000000,1.000000}%
\pgfsetstrokecolor{currentstroke}%
\pgfsetdash{}{0pt}%
\pgfusepath{stroke}%
\end{pgfscope}%
\begin{pgfscope}%
\pgfpathrectangle{\pgfqpoint{0.100000in}{0.100000in}}{\pgfqpoint{3.007045in}{1.925000in}}%
\pgfusepath{clip}%
\pgfsetbuttcap%
\pgfsetroundjoin%
\pgfsetlinewidth{0.050187pt}%
\definecolor{currentstroke}{rgb}{1.000000,1.000000,1.000000}%
\pgfsetstrokecolor{currentstroke}%
\pgfsetdash{}{0pt}%
\pgfusepath{stroke}%
\end{pgfscope}%
\begin{pgfscope}%
\pgfpathrectangle{\pgfqpoint{0.100000in}{0.100000in}}{\pgfqpoint{3.007045in}{1.925000in}}%
\pgfusepath{clip}%
\pgfsetbuttcap%
\pgfsetroundjoin%
\pgfsetlinewidth{0.050187pt}%
\definecolor{currentstroke}{rgb}{1.000000,1.000000,1.000000}%
\pgfsetstrokecolor{currentstroke}%
\pgfsetdash{}{0pt}%
\pgfusepath{stroke}%
\end{pgfscope}%
\begin{pgfscope}%
\pgfpathrectangle{\pgfqpoint{0.100000in}{0.100000in}}{\pgfqpoint{3.007045in}{1.925000in}}%
\pgfusepath{clip}%
\pgfsetbuttcap%
\pgfsetroundjoin%
\pgfsetlinewidth{0.050187pt}%
\definecolor{currentstroke}{rgb}{1.000000,1.000000,1.000000}%
\pgfsetstrokecolor{currentstroke}%
\pgfsetdash{}{0pt}%
\pgfusepath{stroke}%
\end{pgfscope}%
\begin{pgfscope}%
\pgfpathrectangle{\pgfqpoint{0.100000in}{0.100000in}}{\pgfqpoint{3.007045in}{1.925000in}}%
\pgfusepath{clip}%
\pgfsetbuttcap%
\pgfsetroundjoin%
\pgfsetlinewidth{0.050187pt}%
\definecolor{currentstroke}{rgb}{1.000000,1.000000,1.000000}%
\pgfsetstrokecolor{currentstroke}%
\pgfsetdash{}{0pt}%
\pgfusepath{stroke}%
\end{pgfscope}%
\begin{pgfscope}%
\pgfpathrectangle{\pgfqpoint{0.100000in}{0.100000in}}{\pgfqpoint{3.007045in}{1.925000in}}%
\pgfusepath{clip}%
\pgfsetbuttcap%
\pgfsetroundjoin%
\pgfsetlinewidth{0.050187pt}%
\definecolor{currentstroke}{rgb}{1.000000,1.000000,1.000000}%
\pgfsetstrokecolor{currentstroke}%
\pgfsetdash{}{0pt}%
\pgfusepath{stroke}%
\end{pgfscope}%
\begin{pgfscope}%
\pgfpathrectangle{\pgfqpoint{0.100000in}{0.100000in}}{\pgfqpoint{3.007045in}{1.925000in}}%
\pgfusepath{clip}%
\pgfsetbuttcap%
\pgfsetroundjoin%
\pgfsetlinewidth{0.050187pt}%
\definecolor{currentstroke}{rgb}{1.000000,1.000000,1.000000}%
\pgfsetstrokecolor{currentstroke}%
\pgfsetdash{}{0pt}%
\pgfusepath{stroke}%
\end{pgfscope}%
\begin{pgfscope}%
\pgfpathrectangle{\pgfqpoint{0.100000in}{0.100000in}}{\pgfqpoint{3.007045in}{1.925000in}}%
\pgfusepath{clip}%
\pgfsetbuttcap%
\pgfsetroundjoin%
\pgfsetlinewidth{0.050187pt}%
\definecolor{currentstroke}{rgb}{1.000000,1.000000,1.000000}%
\pgfsetstrokecolor{currentstroke}%
\pgfsetdash{}{0pt}%
\pgfusepath{stroke}%
\end{pgfscope}%
\begin{pgfscope}%
\pgfpathrectangle{\pgfqpoint{0.100000in}{0.100000in}}{\pgfqpoint{3.007045in}{1.925000in}}%
\pgfusepath{clip}%
\pgfsetbuttcap%
\pgfsetroundjoin%
\pgfsetlinewidth{0.050187pt}%
\definecolor{currentstroke}{rgb}{1.000000,1.000000,1.000000}%
\pgfsetstrokecolor{currentstroke}%
\pgfsetdash{}{0pt}%
\pgfusepath{stroke}%
\end{pgfscope}%
\begin{pgfscope}%
\pgfpathrectangle{\pgfqpoint{0.100000in}{0.100000in}}{\pgfqpoint{3.007045in}{1.925000in}}%
\pgfusepath{clip}%
\pgfsetbuttcap%
\pgfsetroundjoin%
\pgfsetlinewidth{0.050187pt}%
\definecolor{currentstroke}{rgb}{1.000000,1.000000,1.000000}%
\pgfsetstrokecolor{currentstroke}%
\pgfsetdash{}{0pt}%
\pgfusepath{stroke}%
\end{pgfscope}%
\begin{pgfscope}%
\pgfpathrectangle{\pgfqpoint{0.100000in}{0.100000in}}{\pgfqpoint{3.007045in}{1.925000in}}%
\pgfusepath{clip}%
\pgfsetbuttcap%
\pgfsetroundjoin%
\pgfsetlinewidth{0.050187pt}%
\definecolor{currentstroke}{rgb}{1.000000,1.000000,1.000000}%
\pgfsetstrokecolor{currentstroke}%
\pgfsetdash{}{0pt}%
\pgfusepath{stroke}%
\end{pgfscope}%
\begin{pgfscope}%
\pgfpathrectangle{\pgfqpoint{0.100000in}{0.100000in}}{\pgfqpoint{3.007045in}{1.925000in}}%
\pgfusepath{clip}%
\pgfsetbuttcap%
\pgfsetroundjoin%
\pgfsetlinewidth{0.050187pt}%
\definecolor{currentstroke}{rgb}{1.000000,1.000000,1.000000}%
\pgfsetstrokecolor{currentstroke}%
\pgfsetdash{}{0pt}%
\pgfusepath{stroke}%
\end{pgfscope}%
\begin{pgfscope}%
\pgfpathrectangle{\pgfqpoint{0.100000in}{0.100000in}}{\pgfqpoint{3.007045in}{1.925000in}}%
\pgfusepath{clip}%
\pgfsetbuttcap%
\pgfsetroundjoin%
\pgfsetlinewidth{0.050187pt}%
\definecolor{currentstroke}{rgb}{1.000000,1.000000,1.000000}%
\pgfsetstrokecolor{currentstroke}%
\pgfsetdash{}{0pt}%
\pgfusepath{stroke}%
\end{pgfscope}%
\begin{pgfscope}%
\pgfpathrectangle{\pgfqpoint{0.100000in}{0.100000in}}{\pgfqpoint{3.007045in}{1.925000in}}%
\pgfusepath{clip}%
\pgfsetbuttcap%
\pgfsetroundjoin%
\pgfsetlinewidth{0.050187pt}%
\definecolor{currentstroke}{rgb}{1.000000,1.000000,1.000000}%
\pgfsetstrokecolor{currentstroke}%
\pgfsetdash{}{0pt}%
\pgfusepath{stroke}%
\end{pgfscope}%
\begin{pgfscope}%
\pgfpathrectangle{\pgfqpoint{2.857045in}{0.350000in}}{\pgfqpoint{0.300705in}{0.962500in}}%
\pgfusepath{clip}%
\pgfsetbuttcap%
\pgfsetmiterjoin%
\definecolor{currentfill}{rgb}{0.790496,0.868174,0.941930}%
\pgfsetfillcolor{currentfill}%
\pgfsetlinewidth{0.000000pt}%
\definecolor{currentstroke}{rgb}{0.000000,0.000000,0.000000}%
\pgfsetstrokecolor{currentstroke}%
\pgfsetstrokeopacity{0.000000}%
\pgfsetdash{}{0pt}%
\pgfpathmoveto{\pgfqpoint{3.107615in}{0.350000in}}%
\pgfpathlineto{\pgfqpoint{3.157750in}{0.350000in}}%
\pgfpathlineto{\pgfqpoint{3.157750in}{0.446250in}}%
\pgfpathlineto{\pgfqpoint{3.107615in}{0.446250in}}%
\pgfpathlineto{\pgfqpoint{3.107615in}{0.350000in}}%
\pgfpathclose%
\pgfusepath{fill}%
\end{pgfscope}%
\begin{pgfscope}%
\pgfpathrectangle{\pgfqpoint{2.857045in}{0.350000in}}{\pgfqpoint{0.300705in}{0.962500in}}%
\pgfusepath{clip}%
\pgfsetbuttcap%
\pgfsetmiterjoin%
\definecolor{currentfill}{rgb}{0.760477,0.852026,0.931657}%
\pgfsetfillcolor{currentfill}%
\pgfsetlinewidth{0.000000pt}%
\definecolor{currentstroke}{rgb}{0.000000,0.000000,0.000000}%
\pgfsetstrokecolor{currentstroke}%
\pgfsetstrokeopacity{0.000000}%
\pgfsetdash{}{0pt}%
\pgfpathmoveto{\pgfqpoint{3.105780in}{0.359625in}}%
\pgfpathlineto{\pgfqpoint{3.157750in}{0.359625in}}%
\pgfpathlineto{\pgfqpoint{3.157750in}{0.455875in}}%
\pgfpathlineto{\pgfqpoint{3.105780in}{0.455875in}}%
\pgfpathlineto{\pgfqpoint{3.105780in}{0.359625in}}%
\pgfpathclose%
\pgfusepath{fill}%
\end{pgfscope}%
\begin{pgfscope}%
\pgfpathrectangle{\pgfqpoint{2.857045in}{0.350000in}}{\pgfqpoint{0.300705in}{0.962500in}}%
\pgfusepath{clip}%
\pgfsetbuttcap%
\pgfsetmiterjoin%
\definecolor{currentfill}{rgb}{0.721107,0.835294,0.917878}%
\pgfsetfillcolor{currentfill}%
\pgfsetlinewidth{0.000000pt}%
\definecolor{currentstroke}{rgb}{0.000000,0.000000,0.000000}%
\pgfsetstrokecolor{currentstroke}%
\pgfsetstrokeopacity{0.000000}%
\pgfsetdash{}{0pt}%
\pgfpathmoveto{\pgfqpoint{3.103965in}{0.369250in}}%
\pgfpathlineto{\pgfqpoint{3.157750in}{0.369250in}}%
\pgfpathlineto{\pgfqpoint{3.157750in}{0.465500in}}%
\pgfpathlineto{\pgfqpoint{3.103965in}{0.465500in}}%
\pgfpathlineto{\pgfqpoint{3.103965in}{0.369250in}}%
\pgfpathclose%
\pgfusepath{fill}%
\end{pgfscope}%
\begin{pgfscope}%
\pgfpathrectangle{\pgfqpoint{2.857045in}{0.350000in}}{\pgfqpoint{0.300705in}{0.962500in}}%
\pgfusepath{clip}%
\pgfsetbuttcap%
\pgfsetmiterjoin%
\definecolor{currentfill}{rgb}{0.691580,0.822745,0.907543}%
\pgfsetfillcolor{currentfill}%
\pgfsetlinewidth{0.000000pt}%
\definecolor{currentstroke}{rgb}{0.000000,0.000000,0.000000}%
\pgfsetstrokecolor{currentstroke}%
\pgfsetstrokeopacity{0.000000}%
\pgfsetdash{}{0pt}%
\pgfpathmoveto{\pgfqpoint{3.102756in}{0.378875in}}%
\pgfpathlineto{\pgfqpoint{3.157750in}{0.378875in}}%
\pgfpathlineto{\pgfqpoint{3.157750in}{0.475125in}}%
\pgfpathlineto{\pgfqpoint{3.102756in}{0.475125in}}%
\pgfpathlineto{\pgfqpoint{3.102756in}{0.378875in}}%
\pgfpathclose%
\pgfusepath{fill}%
\end{pgfscope}%
\begin{pgfscope}%
\pgfpathrectangle{\pgfqpoint{2.857045in}{0.350000in}}{\pgfqpoint{0.300705in}{0.962500in}}%
\pgfusepath{clip}%
\pgfsetbuttcap%
\pgfsetmiterjoin%
\definecolor{currentfill}{rgb}{0.676817,0.816471,0.902376}%
\pgfsetfillcolor{currentfill}%
\pgfsetlinewidth{0.000000pt}%
\definecolor{currentstroke}{rgb}{0.000000,0.000000,0.000000}%
\pgfsetstrokecolor{currentstroke}%
\pgfsetstrokeopacity{0.000000}%
\pgfsetdash{}{0pt}%
\pgfpathmoveto{\pgfqpoint{3.102170in}{0.388500in}}%
\pgfpathlineto{\pgfqpoint{3.157750in}{0.388500in}}%
\pgfpathlineto{\pgfqpoint{3.157750in}{0.484750in}}%
\pgfpathlineto{\pgfqpoint{3.102170in}{0.484750in}}%
\pgfpathlineto{\pgfqpoint{3.102170in}{0.388500in}}%
\pgfpathclose%
\pgfusepath{fill}%
\end{pgfscope}%
\begin{pgfscope}%
\pgfpathrectangle{\pgfqpoint{2.857045in}{0.350000in}}{\pgfqpoint{0.300705in}{0.962500in}}%
\pgfusepath{clip}%
\pgfsetbuttcap%
\pgfsetmiterjoin%
\definecolor{currentfill}{rgb}{0.671895,0.814379,0.900654}%
\pgfsetfillcolor{currentfill}%
\pgfsetlinewidth{0.000000pt}%
\definecolor{currentstroke}{rgb}{0.000000,0.000000,0.000000}%
\pgfsetstrokecolor{currentstroke}%
\pgfsetstrokeopacity{0.000000}%
\pgfsetdash{}{0pt}%
\pgfpathmoveto{\pgfqpoint{3.101868in}{0.398125in}}%
\pgfpathlineto{\pgfqpoint{3.157750in}{0.398125in}}%
\pgfpathlineto{\pgfqpoint{3.157750in}{0.494375in}}%
\pgfpathlineto{\pgfqpoint{3.101868in}{0.494375in}}%
\pgfpathlineto{\pgfqpoint{3.101868in}{0.398125in}}%
\pgfpathclose%
\pgfusepath{fill}%
\end{pgfscope}%
\begin{pgfscope}%
\pgfpathrectangle{\pgfqpoint{2.857045in}{0.350000in}}{\pgfqpoint{0.300705in}{0.962500in}}%
\pgfusepath{clip}%
\pgfsetbuttcap%
\pgfsetmiterjoin%
\definecolor{currentfill}{rgb}{0.657132,0.808105,0.895486}%
\pgfsetfillcolor{currentfill}%
\pgfsetlinewidth{0.000000pt}%
\definecolor{currentstroke}{rgb}{0.000000,0.000000,0.000000}%
\pgfsetstrokecolor{currentstroke}%
\pgfsetstrokeopacity{0.000000}%
\pgfsetdash{}{0pt}%
\pgfpathmoveto{\pgfqpoint{3.101144in}{0.407750in}}%
\pgfpathlineto{\pgfqpoint{3.157750in}{0.407750in}}%
\pgfpathlineto{\pgfqpoint{3.157750in}{0.504000in}}%
\pgfpathlineto{\pgfqpoint{3.101144in}{0.504000in}}%
\pgfpathlineto{\pgfqpoint{3.101144in}{0.407750in}}%
\pgfpathclose%
\pgfusepath{fill}%
\end{pgfscope}%
\begin{pgfscope}%
\pgfpathrectangle{\pgfqpoint{2.857045in}{0.350000in}}{\pgfqpoint{0.300705in}{0.962500in}}%
\pgfusepath{clip}%
\pgfsetbuttcap%
\pgfsetmiterjoin%
\definecolor{currentfill}{rgb}{0.637447,0.799739,0.888597}%
\pgfsetfillcolor{currentfill}%
\pgfsetlinewidth{0.000000pt}%
\definecolor{currentstroke}{rgb}{0.000000,0.000000,0.000000}%
\pgfsetstrokecolor{currentstroke}%
\pgfsetstrokeopacity{0.000000}%
\pgfsetdash{}{0pt}%
\pgfpathmoveto{\pgfqpoint{3.100283in}{0.417375in}}%
\pgfpathlineto{\pgfqpoint{3.157750in}{0.417375in}}%
\pgfpathlineto{\pgfqpoint{3.157750in}{0.513625in}}%
\pgfpathlineto{\pgfqpoint{3.100283in}{0.513625in}}%
\pgfpathlineto{\pgfqpoint{3.100283in}{0.417375in}}%
\pgfpathclose%
\pgfusepath{fill}%
\end{pgfscope}%
\begin{pgfscope}%
\pgfpathrectangle{\pgfqpoint{2.857045in}{0.350000in}}{\pgfqpoint{0.300705in}{0.962500in}}%
\pgfusepath{clip}%
\pgfsetbuttcap%
\pgfsetmiterjoin%
\definecolor{currentfill}{rgb}{0.632526,0.797647,0.886874}%
\pgfsetfillcolor{currentfill}%
\pgfsetlinewidth{0.000000pt}%
\definecolor{currentstroke}{rgb}{0.000000,0.000000,0.000000}%
\pgfsetstrokecolor{currentstroke}%
\pgfsetstrokeopacity{0.000000}%
\pgfsetdash{}{0pt}%
\pgfpathmoveto{\pgfqpoint{3.100079in}{0.427000in}}%
\pgfpathlineto{\pgfqpoint{3.157750in}{0.427000in}}%
\pgfpathlineto{\pgfqpoint{3.157750in}{0.523250in}}%
\pgfpathlineto{\pgfqpoint{3.100079in}{0.523250in}}%
\pgfpathlineto{\pgfqpoint{3.100079in}{0.427000in}}%
\pgfpathclose%
\pgfusepath{fill}%
\end{pgfscope}%
\begin{pgfscope}%
\pgfpathrectangle{\pgfqpoint{2.857045in}{0.350000in}}{\pgfqpoint{0.300705in}{0.962500in}}%
\pgfusepath{clip}%
\pgfsetbuttcap%
\pgfsetmiterjoin%
\definecolor{currentfill}{rgb}{0.632526,0.797647,0.886874}%
\pgfsetfillcolor{currentfill}%
\pgfsetlinewidth{0.000000pt}%
\definecolor{currentstroke}{rgb}{0.000000,0.000000,0.000000}%
\pgfsetstrokecolor{currentstroke}%
\pgfsetstrokeopacity{0.000000}%
\pgfsetdash{}{0pt}%
\pgfpathmoveto{\pgfqpoint{3.100079in}{0.436625in}}%
\pgfpathlineto{\pgfqpoint{3.157750in}{0.436625in}}%
\pgfpathlineto{\pgfqpoint{3.157750in}{0.532875in}}%
\pgfpathlineto{\pgfqpoint{3.100079in}{0.532875in}}%
\pgfpathlineto{\pgfqpoint{3.100079in}{0.436625in}}%
\pgfpathclose%
\pgfusepath{fill}%
\end{pgfscope}%
\begin{pgfscope}%
\pgfpathrectangle{\pgfqpoint{2.857045in}{0.350000in}}{\pgfqpoint{0.300705in}{0.962500in}}%
\pgfusepath{clip}%
\pgfsetbuttcap%
\pgfsetmiterjoin%
\definecolor{currentfill}{rgb}{0.632526,0.797647,0.886874}%
\pgfsetfillcolor{currentfill}%
\pgfsetlinewidth{0.000000pt}%
\definecolor{currentstroke}{rgb}{0.000000,0.000000,0.000000}%
\pgfsetstrokecolor{currentstroke}%
\pgfsetstrokeopacity{0.000000}%
\pgfsetdash{}{0pt}%
\pgfpathmoveto{\pgfqpoint{3.099996in}{0.446250in}}%
\pgfpathlineto{\pgfqpoint{3.157750in}{0.446250in}}%
\pgfpathlineto{\pgfqpoint{3.157750in}{0.542500in}}%
\pgfpathlineto{\pgfqpoint{3.099996in}{0.542500in}}%
\pgfpathlineto{\pgfqpoint{3.099996in}{0.446250in}}%
\pgfpathclose%
\pgfusepath{fill}%
\end{pgfscope}%
\begin{pgfscope}%
\pgfpathrectangle{\pgfqpoint{2.857045in}{0.350000in}}{\pgfqpoint{0.300705in}{0.962500in}}%
\pgfusepath{clip}%
\pgfsetbuttcap%
\pgfsetmiterjoin%
\definecolor{currentfill}{rgb}{0.632526,0.797647,0.886874}%
\pgfsetfillcolor{currentfill}%
\pgfsetlinewidth{0.000000pt}%
\definecolor{currentstroke}{rgb}{0.000000,0.000000,0.000000}%
\pgfsetstrokecolor{currentstroke}%
\pgfsetstrokeopacity{0.000000}%
\pgfsetdash{}{0pt}%
\pgfpathmoveto{\pgfqpoint{3.099996in}{0.455875in}}%
\pgfpathlineto{\pgfqpoint{3.157750in}{0.455875in}}%
\pgfpathlineto{\pgfqpoint{3.157750in}{0.552125in}}%
\pgfpathlineto{\pgfqpoint{3.099996in}{0.552125in}}%
\pgfpathlineto{\pgfqpoint{3.099996in}{0.455875in}}%
\pgfpathclose%
\pgfusepath{fill}%
\end{pgfscope}%
\begin{pgfscope}%
\pgfpathrectangle{\pgfqpoint{2.857045in}{0.350000in}}{\pgfqpoint{0.300705in}{0.962500in}}%
\pgfusepath{clip}%
\pgfsetbuttcap%
\pgfsetmiterjoin%
\definecolor{currentfill}{rgb}{0.627605,0.795556,0.885152}%
\pgfsetfillcolor{currentfill}%
\pgfsetlinewidth{0.000000pt}%
\definecolor{currentstroke}{rgb}{0.000000,0.000000,0.000000}%
\pgfsetstrokecolor{currentstroke}%
\pgfsetstrokeopacity{0.000000}%
\pgfsetdash{}{0pt}%
\pgfpathmoveto{\pgfqpoint{3.099947in}{0.465500in}}%
\pgfpathlineto{\pgfqpoint{3.157750in}{0.465500in}}%
\pgfpathlineto{\pgfqpoint{3.157750in}{0.561750in}}%
\pgfpathlineto{\pgfqpoint{3.099947in}{0.561750in}}%
\pgfpathlineto{\pgfqpoint{3.099947in}{0.465500in}}%
\pgfpathclose%
\pgfusepath{fill}%
\end{pgfscope}%
\begin{pgfscope}%
\pgfpathrectangle{\pgfqpoint{2.857045in}{0.350000in}}{\pgfqpoint{0.300705in}{0.962500in}}%
\pgfusepath{clip}%
\pgfsetbuttcap%
\pgfsetmiterjoin%
\definecolor{currentfill}{rgb}{0.592157,0.777086,0.876432}%
\pgfsetfillcolor{currentfill}%
\pgfsetlinewidth{0.000000pt}%
\definecolor{currentstroke}{rgb}{0.000000,0.000000,0.000000}%
\pgfsetstrokecolor{currentstroke}%
\pgfsetstrokeopacity{0.000000}%
\pgfsetdash{}{0pt}%
\pgfpathmoveto{\pgfqpoint{3.098467in}{0.475125in}}%
\pgfpathlineto{\pgfqpoint{3.157750in}{0.475125in}}%
\pgfpathlineto{\pgfqpoint{3.157750in}{0.571375in}}%
\pgfpathlineto{\pgfqpoint{3.098467in}{0.571375in}}%
\pgfpathlineto{\pgfqpoint{3.098467in}{0.475125in}}%
\pgfpathclose%
\pgfusepath{fill}%
\end{pgfscope}%
\begin{pgfscope}%
\pgfpathrectangle{\pgfqpoint{2.857045in}{0.350000in}}{\pgfqpoint{0.300705in}{0.962500in}}%
\pgfusepath{clip}%
\pgfsetbuttcap%
\pgfsetmiterjoin%
\definecolor{currentfill}{rgb}{0.585882,0.773641,0.875079}%
\pgfsetfillcolor{currentfill}%
\pgfsetlinewidth{0.000000pt}%
\definecolor{currentstroke}{rgb}{0.000000,0.000000,0.000000}%
\pgfsetstrokecolor{currentstroke}%
\pgfsetstrokeopacity{0.000000}%
\pgfsetdash{}{0pt}%
\pgfpathmoveto{\pgfqpoint{3.098376in}{0.484750in}}%
\pgfpathlineto{\pgfqpoint{3.157750in}{0.484750in}}%
\pgfpathlineto{\pgfqpoint{3.157750in}{0.581000in}}%
\pgfpathlineto{\pgfqpoint{3.098376in}{0.581000in}}%
\pgfpathlineto{\pgfqpoint{3.098376in}{0.484750in}}%
\pgfpathclose%
\pgfusepath{fill}%
\end{pgfscope}%
\begin{pgfscope}%
\pgfpathrectangle{\pgfqpoint{2.857045in}{0.350000in}}{\pgfqpoint{0.300705in}{0.962500in}}%
\pgfusepath{clip}%
\pgfsetbuttcap%
\pgfsetmiterjoin%
\definecolor{currentfill}{rgb}{0.579608,0.770196,0.873725}%
\pgfsetfillcolor{currentfill}%
\pgfsetlinewidth{0.000000pt}%
\definecolor{currentstroke}{rgb}{0.000000,0.000000,0.000000}%
\pgfsetstrokecolor{currentstroke}%
\pgfsetstrokeopacity{0.000000}%
\pgfsetdash{}{0pt}%
\pgfpathmoveto{\pgfqpoint{3.098186in}{0.494375in}}%
\pgfpathlineto{\pgfqpoint{3.157750in}{0.494375in}}%
\pgfpathlineto{\pgfqpoint{3.157750in}{0.590625in}}%
\pgfpathlineto{\pgfqpoint{3.098186in}{0.590625in}}%
\pgfpathlineto{\pgfqpoint{3.098186in}{0.494375in}}%
\pgfpathclose%
\pgfusepath{fill}%
\end{pgfscope}%
\begin{pgfscope}%
\pgfpathrectangle{\pgfqpoint{2.857045in}{0.350000in}}{\pgfqpoint{0.300705in}{0.962500in}}%
\pgfusepath{clip}%
\pgfsetbuttcap%
\pgfsetmiterjoin%
\definecolor{currentfill}{rgb}{0.567059,0.763306,0.871019}%
\pgfsetfillcolor{currentfill}%
\pgfsetlinewidth{0.000000pt}%
\definecolor{currentstroke}{rgb}{0.000000,0.000000,0.000000}%
\pgfsetstrokecolor{currentstroke}%
\pgfsetstrokeopacity{0.000000}%
\pgfsetdash{}{0pt}%
\pgfpathmoveto{\pgfqpoint{3.097756in}{0.504000in}}%
\pgfpathlineto{\pgfqpoint{3.157750in}{0.504000in}}%
\pgfpathlineto{\pgfqpoint{3.157750in}{0.600250in}}%
\pgfpathlineto{\pgfqpoint{3.097756in}{0.600250in}}%
\pgfpathlineto{\pgfqpoint{3.097756in}{0.504000in}}%
\pgfpathclose%
\pgfusepath{fill}%
\end{pgfscope}%
\begin{pgfscope}%
\pgfpathrectangle{\pgfqpoint{2.857045in}{0.350000in}}{\pgfqpoint{0.300705in}{0.962500in}}%
\pgfusepath{clip}%
\pgfsetbuttcap%
\pgfsetmiterjoin%
\definecolor{currentfill}{rgb}{0.567059,0.763306,0.871019}%
\pgfsetfillcolor{currentfill}%
\pgfsetlinewidth{0.000000pt}%
\definecolor{currentstroke}{rgb}{0.000000,0.000000,0.000000}%
\pgfsetstrokecolor{currentstroke}%
\pgfsetstrokeopacity{0.000000}%
\pgfsetdash{}{0pt}%
\pgfpathmoveto{\pgfqpoint{3.097609in}{0.513625in}}%
\pgfpathlineto{\pgfqpoint{3.157750in}{0.513625in}}%
\pgfpathlineto{\pgfqpoint{3.157750in}{0.609875in}}%
\pgfpathlineto{\pgfqpoint{3.097609in}{0.609875in}}%
\pgfpathlineto{\pgfqpoint{3.097609in}{0.513625in}}%
\pgfpathclose%
\pgfusepath{fill}%
\end{pgfscope}%
\begin{pgfscope}%
\pgfpathrectangle{\pgfqpoint{2.857045in}{0.350000in}}{\pgfqpoint{0.300705in}{0.962500in}}%
\pgfusepath{clip}%
\pgfsetbuttcap%
\pgfsetmiterjoin%
\definecolor{currentfill}{rgb}{0.554510,0.756417,0.868312}%
\pgfsetfillcolor{currentfill}%
\pgfsetlinewidth{0.000000pt}%
\definecolor{currentstroke}{rgb}{0.000000,0.000000,0.000000}%
\pgfsetstrokecolor{currentstroke}%
\pgfsetstrokeopacity{0.000000}%
\pgfsetdash{}{0pt}%
\pgfpathmoveto{\pgfqpoint{3.097293in}{0.523250in}}%
\pgfpathlineto{\pgfqpoint{3.157750in}{0.523250in}}%
\pgfpathlineto{\pgfqpoint{3.157750in}{0.619500in}}%
\pgfpathlineto{\pgfqpoint{3.097293in}{0.619500in}}%
\pgfpathlineto{\pgfqpoint{3.097293in}{0.523250in}}%
\pgfpathclose%
\pgfusepath{fill}%
\end{pgfscope}%
\begin{pgfscope}%
\pgfpathrectangle{\pgfqpoint{2.857045in}{0.350000in}}{\pgfqpoint{0.300705in}{0.962500in}}%
\pgfusepath{clip}%
\pgfsetbuttcap%
\pgfsetmiterjoin%
\definecolor{currentfill}{rgb}{0.554510,0.756417,0.868312}%
\pgfsetfillcolor{currentfill}%
\pgfsetlinewidth{0.000000pt}%
\definecolor{currentstroke}{rgb}{0.000000,0.000000,0.000000}%
\pgfsetstrokecolor{currentstroke}%
\pgfsetstrokeopacity{0.000000}%
\pgfsetdash{}{0pt}%
\pgfpathmoveto{\pgfqpoint{3.097268in}{0.532875in}}%
\pgfpathlineto{\pgfqpoint{3.157750in}{0.532875in}}%
\pgfpathlineto{\pgfqpoint{3.157750in}{0.629125in}}%
\pgfpathlineto{\pgfqpoint{3.097268in}{0.629125in}}%
\pgfpathlineto{\pgfqpoint{3.097268in}{0.532875in}}%
\pgfpathclose%
\pgfusepath{fill}%
\end{pgfscope}%
\begin{pgfscope}%
\pgfpathrectangle{\pgfqpoint{2.857045in}{0.350000in}}{\pgfqpoint{0.300705in}{0.962500in}}%
\pgfusepath{clip}%
\pgfsetbuttcap%
\pgfsetmiterjoin%
\definecolor{currentfill}{rgb}{0.548235,0.752972,0.866959}%
\pgfsetfillcolor{currentfill}%
\pgfsetlinewidth{0.000000pt}%
\definecolor{currentstroke}{rgb}{0.000000,0.000000,0.000000}%
\pgfsetstrokecolor{currentstroke}%
\pgfsetstrokeopacity{0.000000}%
\pgfsetdash{}{0pt}%
\pgfpathmoveto{\pgfqpoint{3.097104in}{0.542500in}}%
\pgfpathlineto{\pgfqpoint{3.157750in}{0.542500in}}%
\pgfpathlineto{\pgfqpoint{3.157750in}{0.638750in}}%
\pgfpathlineto{\pgfqpoint{3.097104in}{0.638750in}}%
\pgfpathlineto{\pgfqpoint{3.097104in}{0.542500in}}%
\pgfpathclose%
\pgfusepath{fill}%
\end{pgfscope}%
\begin{pgfscope}%
\pgfpathrectangle{\pgfqpoint{2.857045in}{0.350000in}}{\pgfqpoint{0.300705in}{0.962500in}}%
\pgfusepath{clip}%
\pgfsetbuttcap%
\pgfsetmiterjoin%
\definecolor{currentfill}{rgb}{0.541961,0.749527,0.865606}%
\pgfsetfillcolor{currentfill}%
\pgfsetlinewidth{0.000000pt}%
\definecolor{currentstroke}{rgb}{0.000000,0.000000,0.000000}%
\pgfsetstrokecolor{currentstroke}%
\pgfsetstrokeopacity{0.000000}%
\pgfsetdash{}{0pt}%
\pgfpathmoveto{\pgfqpoint{3.096865in}{0.552125in}}%
\pgfpathlineto{\pgfqpoint{3.157750in}{0.552125in}}%
\pgfpathlineto{\pgfqpoint{3.157750in}{0.648375in}}%
\pgfpathlineto{\pgfqpoint{3.096865in}{0.648375in}}%
\pgfpathlineto{\pgfqpoint{3.096865in}{0.552125in}}%
\pgfpathclose%
\pgfusepath{fill}%
\end{pgfscope}%
\begin{pgfscope}%
\pgfpathrectangle{\pgfqpoint{2.857045in}{0.350000in}}{\pgfqpoint{0.300705in}{0.962500in}}%
\pgfusepath{clip}%
\pgfsetbuttcap%
\pgfsetmiterjoin%
\definecolor{currentfill}{rgb}{0.535686,0.746082,0.864252}%
\pgfsetfillcolor{currentfill}%
\pgfsetlinewidth{0.000000pt}%
\definecolor{currentstroke}{rgb}{0.000000,0.000000,0.000000}%
\pgfsetstrokecolor{currentstroke}%
\pgfsetstrokeopacity{0.000000}%
\pgfsetdash{}{0pt}%
\pgfpathmoveto{\pgfqpoint{3.096667in}{0.561750in}}%
\pgfpathlineto{\pgfqpoint{3.157750in}{0.561750in}}%
\pgfpathlineto{\pgfqpoint{3.157750in}{0.658000in}}%
\pgfpathlineto{\pgfqpoint{3.096667in}{0.658000in}}%
\pgfpathlineto{\pgfqpoint{3.096667in}{0.561750in}}%
\pgfpathclose%
\pgfusepath{fill}%
\end{pgfscope}%
\begin{pgfscope}%
\pgfpathrectangle{\pgfqpoint{2.857045in}{0.350000in}}{\pgfqpoint{0.300705in}{0.962500in}}%
\pgfusepath{clip}%
\pgfsetbuttcap%
\pgfsetmiterjoin%
\definecolor{currentfill}{rgb}{0.535686,0.746082,0.864252}%
\pgfsetfillcolor{currentfill}%
\pgfsetlinewidth{0.000000pt}%
\definecolor{currentstroke}{rgb}{0.000000,0.000000,0.000000}%
\pgfsetstrokecolor{currentstroke}%
\pgfsetstrokeopacity{0.000000}%
\pgfsetdash{}{0pt}%
\pgfpathmoveto{\pgfqpoint{3.096667in}{0.571375in}}%
\pgfpathlineto{\pgfqpoint{3.157750in}{0.571375in}}%
\pgfpathlineto{\pgfqpoint{3.157750in}{0.667625in}}%
\pgfpathlineto{\pgfqpoint{3.096667in}{0.667625in}}%
\pgfpathlineto{\pgfqpoint{3.096667in}{0.571375in}}%
\pgfpathclose%
\pgfusepath{fill}%
\end{pgfscope}%
\begin{pgfscope}%
\pgfpathrectangle{\pgfqpoint{2.857045in}{0.350000in}}{\pgfqpoint{0.300705in}{0.962500in}}%
\pgfusepath{clip}%
\pgfsetbuttcap%
\pgfsetmiterjoin%
\definecolor{currentfill}{rgb}{0.535686,0.746082,0.864252}%
\pgfsetfillcolor{currentfill}%
\pgfsetlinewidth{0.000000pt}%
\definecolor{currentstroke}{rgb}{0.000000,0.000000,0.000000}%
\pgfsetstrokecolor{currentstroke}%
\pgfsetstrokeopacity{0.000000}%
\pgfsetdash{}{0pt}%
\pgfpathmoveto{\pgfqpoint{3.096667in}{0.581000in}}%
\pgfpathlineto{\pgfqpoint{3.157750in}{0.581000in}}%
\pgfpathlineto{\pgfqpoint{3.157750in}{0.677250in}}%
\pgfpathlineto{\pgfqpoint{3.096667in}{0.677250in}}%
\pgfpathlineto{\pgfqpoint{3.096667in}{0.581000in}}%
\pgfpathclose%
\pgfusepath{fill}%
\end{pgfscope}%
\begin{pgfscope}%
\pgfpathrectangle{\pgfqpoint{2.857045in}{0.350000in}}{\pgfqpoint{0.300705in}{0.962500in}}%
\pgfusepath{clip}%
\pgfsetbuttcap%
\pgfsetmiterjoin%
\definecolor{currentfill}{rgb}{0.535686,0.746082,0.864252}%
\pgfsetfillcolor{currentfill}%
\pgfsetlinewidth{0.000000pt}%
\definecolor{currentstroke}{rgb}{0.000000,0.000000,0.000000}%
\pgfsetstrokecolor{currentstroke}%
\pgfsetstrokeopacity{0.000000}%
\pgfsetdash{}{0pt}%
\pgfpathmoveto{\pgfqpoint{3.096667in}{0.590625in}}%
\pgfpathlineto{\pgfqpoint{3.157750in}{0.590625in}}%
\pgfpathlineto{\pgfqpoint{3.157750in}{0.686875in}}%
\pgfpathlineto{\pgfqpoint{3.096667in}{0.686875in}}%
\pgfpathlineto{\pgfqpoint{3.096667in}{0.590625in}}%
\pgfpathclose%
\pgfusepath{fill}%
\end{pgfscope}%
\begin{pgfscope}%
\pgfpathrectangle{\pgfqpoint{2.857045in}{0.350000in}}{\pgfqpoint{0.300705in}{0.962500in}}%
\pgfusepath{clip}%
\pgfsetbuttcap%
\pgfsetmiterjoin%
\definecolor{currentfill}{rgb}{0.535686,0.746082,0.864252}%
\pgfsetfillcolor{currentfill}%
\pgfsetlinewidth{0.000000pt}%
\definecolor{currentstroke}{rgb}{0.000000,0.000000,0.000000}%
\pgfsetstrokecolor{currentstroke}%
\pgfsetstrokeopacity{0.000000}%
\pgfsetdash{}{0pt}%
\pgfpathmoveto{\pgfqpoint{3.096513in}{0.600250in}}%
\pgfpathlineto{\pgfqpoint{3.157750in}{0.600250in}}%
\pgfpathlineto{\pgfqpoint{3.157750in}{0.696500in}}%
\pgfpathlineto{\pgfqpoint{3.096513in}{0.696500in}}%
\pgfpathlineto{\pgfqpoint{3.096513in}{0.600250in}}%
\pgfpathclose%
\pgfusepath{fill}%
\end{pgfscope}%
\begin{pgfscope}%
\pgfpathrectangle{\pgfqpoint{2.857045in}{0.350000in}}{\pgfqpoint{0.300705in}{0.962500in}}%
\pgfusepath{clip}%
\pgfsetbuttcap%
\pgfsetmiterjoin%
\definecolor{currentfill}{rgb}{0.529412,0.742637,0.862899}%
\pgfsetfillcolor{currentfill}%
\pgfsetlinewidth{0.000000pt}%
\definecolor{currentstroke}{rgb}{0.000000,0.000000,0.000000}%
\pgfsetstrokecolor{currentstroke}%
\pgfsetstrokeopacity{0.000000}%
\pgfsetdash{}{0pt}%
\pgfpathmoveto{\pgfqpoint{3.096381in}{0.609875in}}%
\pgfpathlineto{\pgfqpoint{3.157750in}{0.609875in}}%
\pgfpathlineto{\pgfqpoint{3.157750in}{0.706125in}}%
\pgfpathlineto{\pgfqpoint{3.096381in}{0.706125in}}%
\pgfpathlineto{\pgfqpoint{3.096381in}{0.609875in}}%
\pgfpathclose%
\pgfusepath{fill}%
\end{pgfscope}%
\begin{pgfscope}%
\pgfpathrectangle{\pgfqpoint{2.857045in}{0.350000in}}{\pgfqpoint{0.300705in}{0.962500in}}%
\pgfusepath{clip}%
\pgfsetbuttcap%
\pgfsetmiterjoin%
\definecolor{currentfill}{rgb}{0.529412,0.742637,0.862899}%
\pgfsetfillcolor{currentfill}%
\pgfsetlinewidth{0.000000pt}%
\definecolor{currentstroke}{rgb}{0.000000,0.000000,0.000000}%
\pgfsetstrokecolor{currentstroke}%
\pgfsetstrokeopacity{0.000000}%
\pgfsetdash{}{0pt}%
\pgfpathmoveto{\pgfqpoint{3.096347in}{0.619500in}}%
\pgfpathlineto{\pgfqpoint{3.157750in}{0.619500in}}%
\pgfpathlineto{\pgfqpoint{3.157750in}{0.715750in}}%
\pgfpathlineto{\pgfqpoint{3.096347in}{0.715750in}}%
\pgfpathlineto{\pgfqpoint{3.096347in}{0.619500in}}%
\pgfpathclose%
\pgfusepath{fill}%
\end{pgfscope}%
\begin{pgfscope}%
\pgfpathrectangle{\pgfqpoint{2.857045in}{0.350000in}}{\pgfqpoint{0.300705in}{0.962500in}}%
\pgfusepath{clip}%
\pgfsetbuttcap%
\pgfsetmiterjoin%
\definecolor{currentfill}{rgb}{0.529412,0.742637,0.862899}%
\pgfsetfillcolor{currentfill}%
\pgfsetlinewidth{0.000000pt}%
\definecolor{currentstroke}{rgb}{0.000000,0.000000,0.000000}%
\pgfsetstrokecolor{currentstroke}%
\pgfsetstrokeopacity{0.000000}%
\pgfsetdash{}{0pt}%
\pgfpathmoveto{\pgfqpoint{3.096347in}{0.629125in}}%
\pgfpathlineto{\pgfqpoint{3.157750in}{0.629125in}}%
\pgfpathlineto{\pgfqpoint{3.157750in}{0.725375in}}%
\pgfpathlineto{\pgfqpoint{3.096347in}{0.725375in}}%
\pgfpathlineto{\pgfqpoint{3.096347in}{0.629125in}}%
\pgfpathclose%
\pgfusepath{fill}%
\end{pgfscope}%
\begin{pgfscope}%
\pgfpathrectangle{\pgfqpoint{2.857045in}{0.350000in}}{\pgfqpoint{0.300705in}{0.962500in}}%
\pgfusepath{clip}%
\pgfsetbuttcap%
\pgfsetmiterjoin%
\definecolor{currentfill}{rgb}{0.516863,0.735748,0.860192}%
\pgfsetfillcolor{currentfill}%
\pgfsetlinewidth{0.000000pt}%
\definecolor{currentstroke}{rgb}{0.000000,0.000000,0.000000}%
\pgfsetstrokecolor{currentstroke}%
\pgfsetstrokeopacity{0.000000}%
\pgfsetdash{}{0pt}%
\pgfpathmoveto{\pgfqpoint{3.095995in}{0.638750in}}%
\pgfpathlineto{\pgfqpoint{3.157750in}{0.638750in}}%
\pgfpathlineto{\pgfqpoint{3.157750in}{0.735000in}}%
\pgfpathlineto{\pgfqpoint{3.095995in}{0.735000in}}%
\pgfpathlineto{\pgfqpoint{3.095995in}{0.638750in}}%
\pgfpathclose%
\pgfusepath{fill}%
\end{pgfscope}%
\begin{pgfscope}%
\pgfpathrectangle{\pgfqpoint{2.857045in}{0.350000in}}{\pgfqpoint{0.300705in}{0.962500in}}%
\pgfusepath{clip}%
\pgfsetbuttcap%
\pgfsetmiterjoin%
\definecolor{currentfill}{rgb}{0.510588,0.732303,0.858839}%
\pgfsetfillcolor{currentfill}%
\pgfsetlinewidth{0.000000pt}%
\definecolor{currentstroke}{rgb}{0.000000,0.000000,0.000000}%
\pgfsetstrokecolor{currentstroke}%
\pgfsetstrokeopacity{0.000000}%
\pgfsetdash{}{0pt}%
\pgfpathmoveto{\pgfqpoint{3.095716in}{0.648375in}}%
\pgfpathlineto{\pgfqpoint{3.157750in}{0.648375in}}%
\pgfpathlineto{\pgfqpoint{3.157750in}{0.744625in}}%
\pgfpathlineto{\pgfqpoint{3.095716in}{0.744625in}}%
\pgfpathlineto{\pgfqpoint{3.095716in}{0.648375in}}%
\pgfpathclose%
\pgfusepath{fill}%
\end{pgfscope}%
\begin{pgfscope}%
\pgfpathrectangle{\pgfqpoint{2.857045in}{0.350000in}}{\pgfqpoint{0.300705in}{0.962500in}}%
\pgfusepath{clip}%
\pgfsetbuttcap%
\pgfsetmiterjoin%
\definecolor{currentfill}{rgb}{0.504314,0.728858,0.857486}%
\pgfsetfillcolor{currentfill}%
\pgfsetlinewidth{0.000000pt}%
\definecolor{currentstroke}{rgb}{0.000000,0.000000,0.000000}%
\pgfsetstrokecolor{currentstroke}%
\pgfsetstrokeopacity{0.000000}%
\pgfsetdash{}{0pt}%
\pgfpathmoveto{\pgfqpoint{3.095387in}{0.658000in}}%
\pgfpathlineto{\pgfqpoint{3.157750in}{0.658000in}}%
\pgfpathlineto{\pgfqpoint{3.157750in}{0.754250in}}%
\pgfpathlineto{\pgfqpoint{3.095387in}{0.754250in}}%
\pgfpathlineto{\pgfqpoint{3.095387in}{0.658000in}}%
\pgfpathclose%
\pgfusepath{fill}%
\end{pgfscope}%
\begin{pgfscope}%
\pgfpathrectangle{\pgfqpoint{2.857045in}{0.350000in}}{\pgfqpoint{0.300705in}{0.962500in}}%
\pgfusepath{clip}%
\pgfsetbuttcap%
\pgfsetmiterjoin%
\definecolor{currentfill}{rgb}{0.491765,0.721968,0.854779}%
\pgfsetfillcolor{currentfill}%
\pgfsetlinewidth{0.000000pt}%
\definecolor{currentstroke}{rgb}{0.000000,0.000000,0.000000}%
\pgfsetstrokecolor{currentstroke}%
\pgfsetstrokeopacity{0.000000}%
\pgfsetdash{}{0pt}%
\pgfpathmoveto{\pgfqpoint{3.095060in}{0.667625in}}%
\pgfpathlineto{\pgfqpoint{3.157750in}{0.667625in}}%
\pgfpathlineto{\pgfqpoint{3.157750in}{0.763875in}}%
\pgfpathlineto{\pgfqpoint{3.095060in}{0.763875in}}%
\pgfpathlineto{\pgfqpoint{3.095060in}{0.667625in}}%
\pgfpathclose%
\pgfusepath{fill}%
\end{pgfscope}%
\begin{pgfscope}%
\pgfpathrectangle{\pgfqpoint{2.857045in}{0.350000in}}{\pgfqpoint{0.300705in}{0.962500in}}%
\pgfusepath{clip}%
\pgfsetbuttcap%
\pgfsetmiterjoin%
\definecolor{currentfill}{rgb}{0.485490,0.718524,0.853426}%
\pgfsetfillcolor{currentfill}%
\pgfsetlinewidth{0.000000pt}%
\definecolor{currentstroke}{rgb}{0.000000,0.000000,0.000000}%
\pgfsetstrokecolor{currentstroke}%
\pgfsetstrokeopacity{0.000000}%
\pgfsetdash{}{0pt}%
\pgfpathmoveto{\pgfqpoint{3.094912in}{0.677250in}}%
\pgfpathlineto{\pgfqpoint{3.157750in}{0.677250in}}%
\pgfpathlineto{\pgfqpoint{3.157750in}{0.773500in}}%
\pgfpathlineto{\pgfqpoint{3.094912in}{0.773500in}}%
\pgfpathlineto{\pgfqpoint{3.094912in}{0.677250in}}%
\pgfpathclose%
\pgfusepath{fill}%
\end{pgfscope}%
\begin{pgfscope}%
\pgfpathrectangle{\pgfqpoint{2.857045in}{0.350000in}}{\pgfqpoint{0.300705in}{0.962500in}}%
\pgfusepath{clip}%
\pgfsetbuttcap%
\pgfsetmiterjoin%
\definecolor{currentfill}{rgb}{0.485490,0.718524,0.853426}%
\pgfsetfillcolor{currentfill}%
\pgfsetlinewidth{0.000000pt}%
\definecolor{currentstroke}{rgb}{0.000000,0.000000,0.000000}%
\pgfsetstrokecolor{currentstroke}%
\pgfsetstrokeopacity{0.000000}%
\pgfsetdash{}{0pt}%
\pgfpathmoveto{\pgfqpoint{3.094713in}{0.686875in}}%
\pgfpathlineto{\pgfqpoint{3.157750in}{0.686875in}}%
\pgfpathlineto{\pgfqpoint{3.157750in}{0.783125in}}%
\pgfpathlineto{\pgfqpoint{3.094713in}{0.783125in}}%
\pgfpathlineto{\pgfqpoint{3.094713in}{0.686875in}}%
\pgfpathclose%
\pgfusepath{fill}%
\end{pgfscope}%
\begin{pgfscope}%
\pgfpathrectangle{\pgfqpoint{2.857045in}{0.350000in}}{\pgfqpoint{0.300705in}{0.962500in}}%
\pgfusepath{clip}%
\pgfsetbuttcap%
\pgfsetmiterjoin%
\definecolor{currentfill}{rgb}{0.479216,0.715079,0.852072}%
\pgfsetfillcolor{currentfill}%
\pgfsetlinewidth{0.000000pt}%
\definecolor{currentstroke}{rgb}{0.000000,0.000000,0.000000}%
\pgfsetstrokecolor{currentstroke}%
\pgfsetstrokeopacity{0.000000}%
\pgfsetdash{}{0pt}%
\pgfpathmoveto{\pgfqpoint{3.094672in}{0.696500in}}%
\pgfpathlineto{\pgfqpoint{3.157750in}{0.696500in}}%
\pgfpathlineto{\pgfqpoint{3.157750in}{0.792750in}}%
\pgfpathlineto{\pgfqpoint{3.094672in}{0.792750in}}%
\pgfpathlineto{\pgfqpoint{3.094672in}{0.696500in}}%
\pgfpathclose%
\pgfusepath{fill}%
\end{pgfscope}%
\begin{pgfscope}%
\pgfpathrectangle{\pgfqpoint{2.857045in}{0.350000in}}{\pgfqpoint{0.300705in}{0.962500in}}%
\pgfusepath{clip}%
\pgfsetbuttcap%
\pgfsetmiterjoin%
\definecolor{currentfill}{rgb}{0.479216,0.715079,0.852072}%
\pgfsetfillcolor{currentfill}%
\pgfsetlinewidth{0.000000pt}%
\definecolor{currentstroke}{rgb}{0.000000,0.000000,0.000000}%
\pgfsetstrokecolor{currentstroke}%
\pgfsetstrokeopacity{0.000000}%
\pgfsetdash{}{0pt}%
\pgfpathmoveto{\pgfqpoint{3.094672in}{0.706125in}}%
\pgfpathlineto{\pgfqpoint{3.157750in}{0.706125in}}%
\pgfpathlineto{\pgfqpoint{3.157750in}{0.802375in}}%
\pgfpathlineto{\pgfqpoint{3.094672in}{0.802375in}}%
\pgfpathlineto{\pgfqpoint{3.094672in}{0.706125in}}%
\pgfpathclose%
\pgfusepath{fill}%
\end{pgfscope}%
\begin{pgfscope}%
\pgfpathrectangle{\pgfqpoint{2.857045in}{0.350000in}}{\pgfqpoint{0.300705in}{0.962500in}}%
\pgfusepath{clip}%
\pgfsetbuttcap%
\pgfsetmiterjoin%
\definecolor{currentfill}{rgb}{0.460392,0.704744,0.848012}%
\pgfsetfillcolor{currentfill}%
\pgfsetlinewidth{0.000000pt}%
\definecolor{currentstroke}{rgb}{0.000000,0.000000,0.000000}%
\pgfsetstrokecolor{currentstroke}%
\pgfsetstrokeopacity{0.000000}%
\pgfsetdash{}{0pt}%
\pgfpathmoveto{\pgfqpoint{3.094015in}{0.715750in}}%
\pgfpathlineto{\pgfqpoint{3.157750in}{0.715750in}}%
\pgfpathlineto{\pgfqpoint{3.157750in}{0.812000in}}%
\pgfpathlineto{\pgfqpoint{3.094015in}{0.812000in}}%
\pgfpathlineto{\pgfqpoint{3.094015in}{0.715750in}}%
\pgfpathclose%
\pgfusepath{fill}%
\end{pgfscope}%
\begin{pgfscope}%
\pgfpathrectangle{\pgfqpoint{2.857045in}{0.350000in}}{\pgfqpoint{0.300705in}{0.962500in}}%
\pgfusepath{clip}%
\pgfsetbuttcap%
\pgfsetmiterjoin%
\definecolor{currentfill}{rgb}{0.460392,0.704744,0.848012}%
\pgfsetfillcolor{currentfill}%
\pgfsetlinewidth{0.000000pt}%
\definecolor{currentstroke}{rgb}{0.000000,0.000000,0.000000}%
\pgfsetstrokecolor{currentstroke}%
\pgfsetstrokeopacity{0.000000}%
\pgfsetdash{}{0pt}%
\pgfpathmoveto{\pgfqpoint{3.093973in}{0.725375in}}%
\pgfpathlineto{\pgfqpoint{3.157750in}{0.725375in}}%
\pgfpathlineto{\pgfqpoint{3.157750in}{0.821625in}}%
\pgfpathlineto{\pgfqpoint{3.093973in}{0.821625in}}%
\pgfpathlineto{\pgfqpoint{3.093973in}{0.725375in}}%
\pgfpathclose%
\pgfusepath{fill}%
\end{pgfscope}%
\begin{pgfscope}%
\pgfpathrectangle{\pgfqpoint{2.857045in}{0.350000in}}{\pgfqpoint{0.300705in}{0.962500in}}%
\pgfusepath{clip}%
\pgfsetbuttcap%
\pgfsetmiterjoin%
\definecolor{currentfill}{rgb}{0.460392,0.704744,0.848012}%
\pgfsetfillcolor{currentfill}%
\pgfsetlinewidth{0.000000pt}%
\definecolor{currentstroke}{rgb}{0.000000,0.000000,0.000000}%
\pgfsetstrokecolor{currentstroke}%
\pgfsetstrokeopacity{0.000000}%
\pgfsetdash{}{0pt}%
\pgfpathmoveto{\pgfqpoint{3.093972in}{0.735000in}}%
\pgfpathlineto{\pgfqpoint{3.157750in}{0.735000in}}%
\pgfpathlineto{\pgfqpoint{3.157750in}{0.831250in}}%
\pgfpathlineto{\pgfqpoint{3.093972in}{0.831250in}}%
\pgfpathlineto{\pgfqpoint{3.093972in}{0.735000in}}%
\pgfpathclose%
\pgfusepath{fill}%
\end{pgfscope}%
\begin{pgfscope}%
\pgfpathrectangle{\pgfqpoint{2.857045in}{0.350000in}}{\pgfqpoint{0.300705in}{0.962500in}}%
\pgfusepath{clip}%
\pgfsetbuttcap%
\pgfsetmiterjoin%
\definecolor{currentfill}{rgb}{0.460392,0.704744,0.848012}%
\pgfsetfillcolor{currentfill}%
\pgfsetlinewidth{0.000000pt}%
\definecolor{currentstroke}{rgb}{0.000000,0.000000,0.000000}%
\pgfsetstrokecolor{currentstroke}%
\pgfsetstrokeopacity{0.000000}%
\pgfsetdash{}{0pt}%
\pgfpathmoveto{\pgfqpoint{3.093896in}{0.744625in}}%
\pgfpathlineto{\pgfqpoint{3.157750in}{0.744625in}}%
\pgfpathlineto{\pgfqpoint{3.157750in}{0.840875in}}%
\pgfpathlineto{\pgfqpoint{3.093896in}{0.840875in}}%
\pgfpathlineto{\pgfqpoint{3.093896in}{0.744625in}}%
\pgfpathclose%
\pgfusepath{fill}%
\end{pgfscope}%
\begin{pgfscope}%
\pgfpathrectangle{\pgfqpoint{2.857045in}{0.350000in}}{\pgfqpoint{0.300705in}{0.962500in}}%
\pgfusepath{clip}%
\pgfsetbuttcap%
\pgfsetmiterjoin%
\definecolor{currentfill}{rgb}{0.441569,0.694410,0.843952}%
\pgfsetfillcolor{currentfill}%
\pgfsetlinewidth{0.000000pt}%
\definecolor{currentstroke}{rgb}{0.000000,0.000000,0.000000}%
\pgfsetstrokecolor{currentstroke}%
\pgfsetstrokeopacity{0.000000}%
\pgfsetdash{}{0pt}%
\pgfpathmoveto{\pgfqpoint{3.093315in}{0.754250in}}%
\pgfpathlineto{\pgfqpoint{3.157750in}{0.754250in}}%
\pgfpathlineto{\pgfqpoint{3.157750in}{0.850500in}}%
\pgfpathlineto{\pgfqpoint{3.093315in}{0.850500in}}%
\pgfpathlineto{\pgfqpoint{3.093315in}{0.754250in}}%
\pgfpathclose%
\pgfusepath{fill}%
\end{pgfscope}%
\begin{pgfscope}%
\pgfpathrectangle{\pgfqpoint{2.857045in}{0.350000in}}{\pgfqpoint{0.300705in}{0.962500in}}%
\pgfusepath{clip}%
\pgfsetbuttcap%
\pgfsetmiterjoin%
\definecolor{currentfill}{rgb}{0.435294,0.690965,0.842599}%
\pgfsetfillcolor{currentfill}%
\pgfsetlinewidth{0.000000pt}%
\definecolor{currentstroke}{rgb}{0.000000,0.000000,0.000000}%
\pgfsetstrokecolor{currentstroke}%
\pgfsetstrokeopacity{0.000000}%
\pgfsetdash{}{0pt}%
\pgfpathmoveto{\pgfqpoint{3.093140in}{0.763875in}}%
\pgfpathlineto{\pgfqpoint{3.157750in}{0.763875in}}%
\pgfpathlineto{\pgfqpoint{3.157750in}{0.860125in}}%
\pgfpathlineto{\pgfqpoint{3.093140in}{0.860125in}}%
\pgfpathlineto{\pgfqpoint{3.093140in}{0.763875in}}%
\pgfpathclose%
\pgfusepath{fill}%
\end{pgfscope}%
\begin{pgfscope}%
\pgfpathrectangle{\pgfqpoint{2.857045in}{0.350000in}}{\pgfqpoint{0.300705in}{0.962500in}}%
\pgfusepath{clip}%
\pgfsetbuttcap%
\pgfsetmiterjoin%
\definecolor{currentfill}{rgb}{0.435294,0.690965,0.842599}%
\pgfsetfillcolor{currentfill}%
\pgfsetlinewidth{0.000000pt}%
\definecolor{currentstroke}{rgb}{0.000000,0.000000,0.000000}%
\pgfsetstrokecolor{currentstroke}%
\pgfsetstrokeopacity{0.000000}%
\pgfsetdash{}{0pt}%
\pgfpathmoveto{\pgfqpoint{3.093114in}{0.773500in}}%
\pgfpathlineto{\pgfqpoint{3.157750in}{0.773500in}}%
\pgfpathlineto{\pgfqpoint{3.157750in}{0.869750in}}%
\pgfpathlineto{\pgfqpoint{3.093114in}{0.869750in}}%
\pgfpathlineto{\pgfqpoint{3.093114in}{0.773500in}}%
\pgfpathclose%
\pgfusepath{fill}%
\end{pgfscope}%
\begin{pgfscope}%
\pgfpathrectangle{\pgfqpoint{2.857045in}{0.350000in}}{\pgfqpoint{0.300705in}{0.962500in}}%
\pgfusepath{clip}%
\pgfsetbuttcap%
\pgfsetmiterjoin%
\definecolor{currentfill}{rgb}{0.429020,0.687520,0.841246}%
\pgfsetfillcolor{currentfill}%
\pgfsetlinewidth{0.000000pt}%
\definecolor{currentstroke}{rgb}{0.000000,0.000000,0.000000}%
\pgfsetstrokecolor{currentstroke}%
\pgfsetstrokeopacity{0.000000}%
\pgfsetdash{}{0pt}%
\pgfpathmoveto{\pgfqpoint{3.092877in}{0.783125in}}%
\pgfpathlineto{\pgfqpoint{3.157750in}{0.783125in}}%
\pgfpathlineto{\pgfqpoint{3.157750in}{0.879375in}}%
\pgfpathlineto{\pgfqpoint{3.092877in}{0.879375in}}%
\pgfpathlineto{\pgfqpoint{3.092877in}{0.783125in}}%
\pgfpathclose%
\pgfusepath{fill}%
\end{pgfscope}%
\begin{pgfscope}%
\pgfpathrectangle{\pgfqpoint{2.857045in}{0.350000in}}{\pgfqpoint{0.300705in}{0.962500in}}%
\pgfusepath{clip}%
\pgfsetbuttcap%
\pgfsetmiterjoin%
\definecolor{currentfill}{rgb}{0.429020,0.687520,0.841246}%
\pgfsetfillcolor{currentfill}%
\pgfsetlinewidth{0.000000pt}%
\definecolor{currentstroke}{rgb}{0.000000,0.000000,0.000000}%
\pgfsetstrokecolor{currentstroke}%
\pgfsetstrokeopacity{0.000000}%
\pgfsetdash{}{0pt}%
\pgfpathmoveto{\pgfqpoint{3.092877in}{0.792750in}}%
\pgfpathlineto{\pgfqpoint{3.157750in}{0.792750in}}%
\pgfpathlineto{\pgfqpoint{3.157750in}{0.889000in}}%
\pgfpathlineto{\pgfqpoint{3.092877in}{0.889000in}}%
\pgfpathlineto{\pgfqpoint{3.092877in}{0.792750in}}%
\pgfpathclose%
\pgfusepath{fill}%
\end{pgfscope}%
\begin{pgfscope}%
\pgfpathrectangle{\pgfqpoint{2.857045in}{0.350000in}}{\pgfqpoint{0.300705in}{0.962500in}}%
\pgfusepath{clip}%
\pgfsetbuttcap%
\pgfsetmiterjoin%
\definecolor{currentfill}{rgb}{0.429020,0.687520,0.841246}%
\pgfsetfillcolor{currentfill}%
\pgfsetlinewidth{0.000000pt}%
\definecolor{currentstroke}{rgb}{0.000000,0.000000,0.000000}%
\pgfsetstrokecolor{currentstroke}%
\pgfsetstrokeopacity{0.000000}%
\pgfsetdash{}{0pt}%
\pgfpathmoveto{\pgfqpoint{3.092877in}{0.802375in}}%
\pgfpathlineto{\pgfqpoint{3.157750in}{0.802375in}}%
\pgfpathlineto{\pgfqpoint{3.157750in}{0.898625in}}%
\pgfpathlineto{\pgfqpoint{3.092877in}{0.898625in}}%
\pgfpathlineto{\pgfqpoint{3.092877in}{0.802375in}}%
\pgfpathclose%
\pgfusepath{fill}%
\end{pgfscope}%
\begin{pgfscope}%
\pgfpathrectangle{\pgfqpoint{2.857045in}{0.350000in}}{\pgfqpoint{0.300705in}{0.962500in}}%
\pgfusepath{clip}%
\pgfsetbuttcap%
\pgfsetmiterjoin%
\definecolor{currentfill}{rgb}{0.429020,0.687520,0.841246}%
\pgfsetfillcolor{currentfill}%
\pgfsetlinewidth{0.000000pt}%
\definecolor{currentstroke}{rgb}{0.000000,0.000000,0.000000}%
\pgfsetstrokecolor{currentstroke}%
\pgfsetstrokeopacity{0.000000}%
\pgfsetdash{}{0pt}%
\pgfpathmoveto{\pgfqpoint{3.092877in}{0.812000in}}%
\pgfpathlineto{\pgfqpoint{3.157750in}{0.812000in}}%
\pgfpathlineto{\pgfqpoint{3.157750in}{0.908250in}}%
\pgfpathlineto{\pgfqpoint{3.092877in}{0.908250in}}%
\pgfpathlineto{\pgfqpoint{3.092877in}{0.812000in}}%
\pgfpathclose%
\pgfusepath{fill}%
\end{pgfscope}%
\begin{pgfscope}%
\pgfpathrectangle{\pgfqpoint{2.857045in}{0.350000in}}{\pgfqpoint{0.300705in}{0.962500in}}%
\pgfusepath{clip}%
\pgfsetbuttcap%
\pgfsetmiterjoin%
\definecolor{currentfill}{rgb}{0.429020,0.687520,0.841246}%
\pgfsetfillcolor{currentfill}%
\pgfsetlinewidth{0.000000pt}%
\definecolor{currentstroke}{rgb}{0.000000,0.000000,0.000000}%
\pgfsetstrokecolor{currentstroke}%
\pgfsetstrokeopacity{0.000000}%
\pgfsetdash{}{0pt}%
\pgfpathmoveto{\pgfqpoint{3.092877in}{0.821625in}}%
\pgfpathlineto{\pgfqpoint{3.157750in}{0.821625in}}%
\pgfpathlineto{\pgfqpoint{3.157750in}{0.917875in}}%
\pgfpathlineto{\pgfqpoint{3.092877in}{0.917875in}}%
\pgfpathlineto{\pgfqpoint{3.092877in}{0.821625in}}%
\pgfpathclose%
\pgfusepath{fill}%
\end{pgfscope}%
\begin{pgfscope}%
\pgfpathrectangle{\pgfqpoint{2.857045in}{0.350000in}}{\pgfqpoint{0.300705in}{0.962500in}}%
\pgfusepath{clip}%
\pgfsetbuttcap%
\pgfsetmiterjoin%
\definecolor{currentfill}{rgb}{0.429020,0.687520,0.841246}%
\pgfsetfillcolor{currentfill}%
\pgfsetlinewidth{0.000000pt}%
\definecolor{currentstroke}{rgb}{0.000000,0.000000,0.000000}%
\pgfsetstrokecolor{currentstroke}%
\pgfsetstrokeopacity{0.000000}%
\pgfsetdash{}{0pt}%
\pgfpathmoveto{\pgfqpoint{3.092877in}{0.831250in}}%
\pgfpathlineto{\pgfqpoint{3.157750in}{0.831250in}}%
\pgfpathlineto{\pgfqpoint{3.157750in}{0.927500in}}%
\pgfpathlineto{\pgfqpoint{3.092877in}{0.927500in}}%
\pgfpathlineto{\pgfqpoint{3.092877in}{0.831250in}}%
\pgfpathclose%
\pgfusepath{fill}%
\end{pgfscope}%
\begin{pgfscope}%
\pgfpathrectangle{\pgfqpoint{2.857045in}{0.350000in}}{\pgfqpoint{0.300705in}{0.962500in}}%
\pgfusepath{clip}%
\pgfsetbuttcap%
\pgfsetmiterjoin%
\definecolor{currentfill}{rgb}{0.429020,0.687520,0.841246}%
\pgfsetfillcolor{currentfill}%
\pgfsetlinewidth{0.000000pt}%
\definecolor{currentstroke}{rgb}{0.000000,0.000000,0.000000}%
\pgfsetstrokecolor{currentstroke}%
\pgfsetstrokeopacity{0.000000}%
\pgfsetdash{}{0pt}%
\pgfpathmoveto{\pgfqpoint{3.092809in}{0.840875in}}%
\pgfpathlineto{\pgfqpoint{3.157750in}{0.840875in}}%
\pgfpathlineto{\pgfqpoint{3.157750in}{0.937125in}}%
\pgfpathlineto{\pgfqpoint{3.092809in}{0.937125in}}%
\pgfpathlineto{\pgfqpoint{3.092809in}{0.840875in}}%
\pgfpathclose%
\pgfusepath{fill}%
\end{pgfscope}%
\begin{pgfscope}%
\pgfpathrectangle{\pgfqpoint{2.857045in}{0.350000in}}{\pgfqpoint{0.300705in}{0.962500in}}%
\pgfusepath{clip}%
\pgfsetbuttcap%
\pgfsetmiterjoin%
\definecolor{currentfill}{rgb}{0.422745,0.684075,0.839892}%
\pgfsetfillcolor{currentfill}%
\pgfsetlinewidth{0.000000pt}%
\definecolor{currentstroke}{rgb}{0.000000,0.000000,0.000000}%
\pgfsetstrokecolor{currentstroke}%
\pgfsetstrokeopacity{0.000000}%
\pgfsetdash{}{0pt}%
\pgfpathmoveto{\pgfqpoint{3.092707in}{0.850500in}}%
\pgfpathlineto{\pgfqpoint{3.157750in}{0.850500in}}%
\pgfpathlineto{\pgfqpoint{3.157750in}{0.946750in}}%
\pgfpathlineto{\pgfqpoint{3.092707in}{0.946750in}}%
\pgfpathlineto{\pgfqpoint{3.092707in}{0.850500in}}%
\pgfpathclose%
\pgfusepath{fill}%
\end{pgfscope}%
\begin{pgfscope}%
\pgfpathrectangle{\pgfqpoint{2.857045in}{0.350000in}}{\pgfqpoint{0.300705in}{0.962500in}}%
\pgfusepath{clip}%
\pgfsetbuttcap%
\pgfsetmiterjoin%
\definecolor{currentfill}{rgb}{0.422745,0.684075,0.839892}%
\pgfsetfillcolor{currentfill}%
\pgfsetlinewidth{0.000000pt}%
\definecolor{currentstroke}{rgb}{0.000000,0.000000,0.000000}%
\pgfsetstrokecolor{currentstroke}%
\pgfsetstrokeopacity{0.000000}%
\pgfsetdash{}{0pt}%
\pgfpathmoveto{\pgfqpoint{3.092541in}{0.860125in}}%
\pgfpathlineto{\pgfqpoint{3.157750in}{0.860125in}}%
\pgfpathlineto{\pgfqpoint{3.157750in}{0.956375in}}%
\pgfpathlineto{\pgfqpoint{3.092541in}{0.956375in}}%
\pgfpathlineto{\pgfqpoint{3.092541in}{0.860125in}}%
\pgfpathclose%
\pgfusepath{fill}%
\end{pgfscope}%
\begin{pgfscope}%
\pgfpathrectangle{\pgfqpoint{2.857045in}{0.350000in}}{\pgfqpoint{0.300705in}{0.962500in}}%
\pgfusepath{clip}%
\pgfsetbuttcap%
\pgfsetmiterjoin%
\definecolor{currentfill}{rgb}{0.422745,0.684075,0.839892}%
\pgfsetfillcolor{currentfill}%
\pgfsetlinewidth{0.000000pt}%
\definecolor{currentstroke}{rgb}{0.000000,0.000000,0.000000}%
\pgfsetstrokecolor{currentstroke}%
\pgfsetstrokeopacity{0.000000}%
\pgfsetdash{}{0pt}%
\pgfpathmoveto{\pgfqpoint{3.092541in}{0.869750in}}%
\pgfpathlineto{\pgfqpoint{3.157750in}{0.869750in}}%
\pgfpathlineto{\pgfqpoint{3.157750in}{0.966000in}}%
\pgfpathlineto{\pgfqpoint{3.092541in}{0.966000in}}%
\pgfpathlineto{\pgfqpoint{3.092541in}{0.869750in}}%
\pgfpathclose%
\pgfusepath{fill}%
\end{pgfscope}%
\begin{pgfscope}%
\pgfpathrectangle{\pgfqpoint{2.857045in}{0.350000in}}{\pgfqpoint{0.300705in}{0.962500in}}%
\pgfusepath{clip}%
\pgfsetbuttcap%
\pgfsetmiterjoin%
\definecolor{currentfill}{rgb}{0.417086,0.680631,0.838231}%
\pgfsetfillcolor{currentfill}%
\pgfsetlinewidth{0.000000pt}%
\definecolor{currentstroke}{rgb}{0.000000,0.000000,0.000000}%
\pgfsetstrokecolor{currentstroke}%
\pgfsetstrokeopacity{0.000000}%
\pgfsetdash{}{0pt}%
\pgfpathmoveto{\pgfqpoint{3.092485in}{0.879375in}}%
\pgfpathlineto{\pgfqpoint{3.157750in}{0.879375in}}%
\pgfpathlineto{\pgfqpoint{3.157750in}{0.975625in}}%
\pgfpathlineto{\pgfqpoint{3.092485in}{0.975625in}}%
\pgfpathlineto{\pgfqpoint{3.092485in}{0.879375in}}%
\pgfpathclose%
\pgfusepath{fill}%
\end{pgfscope}%
\begin{pgfscope}%
\pgfpathrectangle{\pgfqpoint{2.857045in}{0.350000in}}{\pgfqpoint{0.300705in}{0.962500in}}%
\pgfusepath{clip}%
\pgfsetbuttcap%
\pgfsetmiterjoin%
\definecolor{currentfill}{rgb}{0.417086,0.680631,0.838231}%
\pgfsetfillcolor{currentfill}%
\pgfsetlinewidth{0.000000pt}%
\definecolor{currentstroke}{rgb}{0.000000,0.000000,0.000000}%
\pgfsetstrokecolor{currentstroke}%
\pgfsetstrokeopacity{0.000000}%
\pgfsetdash{}{0pt}%
\pgfpathmoveto{\pgfqpoint{3.092350in}{0.889000in}}%
\pgfpathlineto{\pgfqpoint{3.157750in}{0.889000in}}%
\pgfpathlineto{\pgfqpoint{3.157750in}{0.985250in}}%
\pgfpathlineto{\pgfqpoint{3.092350in}{0.985250in}}%
\pgfpathlineto{\pgfqpoint{3.092350in}{0.889000in}}%
\pgfpathclose%
\pgfusepath{fill}%
\end{pgfscope}%
\begin{pgfscope}%
\pgfpathrectangle{\pgfqpoint{2.857045in}{0.350000in}}{\pgfqpoint{0.300705in}{0.962500in}}%
\pgfusepath{clip}%
\pgfsetbuttcap%
\pgfsetmiterjoin%
\definecolor{currentfill}{rgb}{0.417086,0.680631,0.838231}%
\pgfsetfillcolor{currentfill}%
\pgfsetlinewidth{0.000000pt}%
\definecolor{currentstroke}{rgb}{0.000000,0.000000,0.000000}%
\pgfsetstrokecolor{currentstroke}%
\pgfsetstrokeopacity{0.000000}%
\pgfsetdash{}{0pt}%
\pgfpathmoveto{\pgfqpoint{3.092295in}{0.898625in}}%
\pgfpathlineto{\pgfqpoint{3.157750in}{0.898625in}}%
\pgfpathlineto{\pgfqpoint{3.157750in}{0.994875in}}%
\pgfpathlineto{\pgfqpoint{3.092295in}{0.994875in}}%
\pgfpathlineto{\pgfqpoint{3.092295in}{0.898625in}}%
\pgfpathclose%
\pgfusepath{fill}%
\end{pgfscope}%
\begin{pgfscope}%
\pgfpathrectangle{\pgfqpoint{2.857045in}{0.350000in}}{\pgfqpoint{0.300705in}{0.962500in}}%
\pgfusepath{clip}%
\pgfsetbuttcap%
\pgfsetmiterjoin%
\definecolor{currentfill}{rgb}{0.417086,0.680631,0.838231}%
\pgfsetfillcolor{currentfill}%
\pgfsetlinewidth{0.000000pt}%
\definecolor{currentstroke}{rgb}{0.000000,0.000000,0.000000}%
\pgfsetstrokecolor{currentstroke}%
\pgfsetstrokeopacity{0.000000}%
\pgfsetdash{}{0pt}%
\pgfpathmoveto{\pgfqpoint{3.092295in}{0.908250in}}%
\pgfpathlineto{\pgfqpoint{3.157750in}{0.908250in}}%
\pgfpathlineto{\pgfqpoint{3.157750in}{1.004500in}}%
\pgfpathlineto{\pgfqpoint{3.092295in}{1.004500in}}%
\pgfpathlineto{\pgfqpoint{3.092295in}{0.908250in}}%
\pgfpathclose%
\pgfusepath{fill}%
\end{pgfscope}%
\begin{pgfscope}%
\pgfpathrectangle{\pgfqpoint{2.857045in}{0.350000in}}{\pgfqpoint{0.300705in}{0.962500in}}%
\pgfusepath{clip}%
\pgfsetbuttcap%
\pgfsetmiterjoin%
\definecolor{currentfill}{rgb}{0.406997,0.673741,0.834295}%
\pgfsetfillcolor{currentfill}%
\pgfsetlinewidth{0.000000pt}%
\definecolor{currentstroke}{rgb}{0.000000,0.000000,0.000000}%
\pgfsetstrokecolor{currentstroke}%
\pgfsetstrokeopacity{0.000000}%
\pgfsetdash{}{0pt}%
\pgfpathmoveto{\pgfqpoint{3.092038in}{0.917875in}}%
\pgfpathlineto{\pgfqpoint{3.157750in}{0.917875in}}%
\pgfpathlineto{\pgfqpoint{3.157750in}{1.014125in}}%
\pgfpathlineto{\pgfqpoint{3.092038in}{1.014125in}}%
\pgfpathlineto{\pgfqpoint{3.092038in}{0.917875in}}%
\pgfpathclose%
\pgfusepath{fill}%
\end{pgfscope}%
\begin{pgfscope}%
\pgfpathrectangle{\pgfqpoint{2.857045in}{0.350000in}}{\pgfqpoint{0.300705in}{0.962500in}}%
\pgfusepath{clip}%
\pgfsetbuttcap%
\pgfsetmiterjoin%
\definecolor{currentfill}{rgb}{0.406997,0.673741,0.834295}%
\pgfsetfillcolor{currentfill}%
\pgfsetlinewidth{0.000000pt}%
\definecolor{currentstroke}{rgb}{0.000000,0.000000,0.000000}%
\pgfsetstrokecolor{currentstroke}%
\pgfsetstrokeopacity{0.000000}%
\pgfsetdash{}{0pt}%
\pgfpathmoveto{\pgfqpoint{3.091863in}{0.927500in}}%
\pgfpathlineto{\pgfqpoint{3.157750in}{0.927500in}}%
\pgfpathlineto{\pgfqpoint{3.157750in}{1.023750in}}%
\pgfpathlineto{\pgfqpoint{3.091863in}{1.023750in}}%
\pgfpathlineto{\pgfqpoint{3.091863in}{0.927500in}}%
\pgfpathclose%
\pgfusepath{fill}%
\end{pgfscope}%
\begin{pgfscope}%
\pgfpathrectangle{\pgfqpoint{2.857045in}{0.350000in}}{\pgfqpoint{0.300705in}{0.962500in}}%
\pgfusepath{clip}%
\pgfsetbuttcap%
\pgfsetmiterjoin%
\definecolor{currentfill}{rgb}{0.401953,0.670296,0.832326}%
\pgfsetfillcolor{currentfill}%
\pgfsetlinewidth{0.000000pt}%
\definecolor{currentstroke}{rgb}{0.000000,0.000000,0.000000}%
\pgfsetstrokecolor{currentstroke}%
\pgfsetstrokeopacity{0.000000}%
\pgfsetdash{}{0pt}%
\pgfpathmoveto{\pgfqpoint{3.091782in}{0.937125in}}%
\pgfpathlineto{\pgfqpoint{3.157750in}{0.937125in}}%
\pgfpathlineto{\pgfqpoint{3.157750in}{1.033375in}}%
\pgfpathlineto{\pgfqpoint{3.091782in}{1.033375in}}%
\pgfpathlineto{\pgfqpoint{3.091782in}{0.937125in}}%
\pgfpathclose%
\pgfusepath{fill}%
\end{pgfscope}%
\begin{pgfscope}%
\pgfpathrectangle{\pgfqpoint{2.857045in}{0.350000in}}{\pgfqpoint{0.300705in}{0.962500in}}%
\pgfusepath{clip}%
\pgfsetbuttcap%
\pgfsetmiterjoin%
\definecolor{currentfill}{rgb}{0.396909,0.666851,0.830358}%
\pgfsetfillcolor{currentfill}%
\pgfsetlinewidth{0.000000pt}%
\definecolor{currentstroke}{rgb}{0.000000,0.000000,0.000000}%
\pgfsetstrokecolor{currentstroke}%
\pgfsetstrokeopacity{0.000000}%
\pgfsetdash{}{0pt}%
\pgfpathmoveto{\pgfqpoint{3.091583in}{0.946750in}}%
\pgfpathlineto{\pgfqpoint{3.157750in}{0.946750in}}%
\pgfpathlineto{\pgfqpoint{3.157750in}{1.043000in}}%
\pgfpathlineto{\pgfqpoint{3.091583in}{1.043000in}}%
\pgfpathlineto{\pgfqpoint{3.091583in}{0.946750in}}%
\pgfpathclose%
\pgfusepath{fill}%
\end{pgfscope}%
\begin{pgfscope}%
\pgfpathrectangle{\pgfqpoint{2.857045in}{0.350000in}}{\pgfqpoint{0.300705in}{0.962500in}}%
\pgfusepath{clip}%
\pgfsetbuttcap%
\pgfsetmiterjoin%
\definecolor{currentfill}{rgb}{0.396909,0.666851,0.830358}%
\pgfsetfillcolor{currentfill}%
\pgfsetlinewidth{0.000000pt}%
\definecolor{currentstroke}{rgb}{0.000000,0.000000,0.000000}%
\pgfsetstrokecolor{currentstroke}%
\pgfsetstrokeopacity{0.000000}%
\pgfsetdash{}{0pt}%
\pgfpathmoveto{\pgfqpoint{3.091572in}{0.956375in}}%
\pgfpathlineto{\pgfqpoint{3.157750in}{0.956375in}}%
\pgfpathlineto{\pgfqpoint{3.157750in}{1.052625in}}%
\pgfpathlineto{\pgfqpoint{3.091572in}{1.052625in}}%
\pgfpathlineto{\pgfqpoint{3.091572in}{0.956375in}}%
\pgfpathclose%
\pgfusepath{fill}%
\end{pgfscope}%
\begin{pgfscope}%
\pgfpathrectangle{\pgfqpoint{2.857045in}{0.350000in}}{\pgfqpoint{0.300705in}{0.962500in}}%
\pgfusepath{clip}%
\pgfsetbuttcap%
\pgfsetmiterjoin%
\definecolor{currentfill}{rgb}{0.391865,0.663406,0.828389}%
\pgfsetfillcolor{currentfill}%
\pgfsetlinewidth{0.000000pt}%
\definecolor{currentstroke}{rgb}{0.000000,0.000000,0.000000}%
\pgfsetstrokecolor{currentstroke}%
\pgfsetstrokeopacity{0.000000}%
\pgfsetdash{}{0pt}%
\pgfpathmoveto{\pgfqpoint{3.091289in}{0.966000in}}%
\pgfpathlineto{\pgfqpoint{3.157750in}{0.966000in}}%
\pgfpathlineto{\pgfqpoint{3.157750in}{1.062250in}}%
\pgfpathlineto{\pgfqpoint{3.091289in}{1.062250in}}%
\pgfpathlineto{\pgfqpoint{3.091289in}{0.966000in}}%
\pgfpathclose%
\pgfusepath{fill}%
\end{pgfscope}%
\begin{pgfscope}%
\pgfpathrectangle{\pgfqpoint{2.857045in}{0.350000in}}{\pgfqpoint{0.300705in}{0.962500in}}%
\pgfusepath{clip}%
\pgfsetbuttcap%
\pgfsetmiterjoin%
\definecolor{currentfill}{rgb}{0.381776,0.656517,0.824452}%
\pgfsetfillcolor{currentfill}%
\pgfsetlinewidth{0.000000pt}%
\definecolor{currentstroke}{rgb}{0.000000,0.000000,0.000000}%
\pgfsetstrokecolor{currentstroke}%
\pgfsetstrokeopacity{0.000000}%
\pgfsetdash{}{0pt}%
\pgfpathmoveto{\pgfqpoint{3.090894in}{0.975625in}}%
\pgfpathlineto{\pgfqpoint{3.157750in}{0.975625in}}%
\pgfpathlineto{\pgfqpoint{3.157750in}{1.071875in}}%
\pgfpathlineto{\pgfqpoint{3.090894in}{1.071875in}}%
\pgfpathlineto{\pgfqpoint{3.090894in}{0.975625in}}%
\pgfpathclose%
\pgfusepath{fill}%
\end{pgfscope}%
\begin{pgfscope}%
\pgfpathrectangle{\pgfqpoint{2.857045in}{0.350000in}}{\pgfqpoint{0.300705in}{0.962500in}}%
\pgfusepath{clip}%
\pgfsetbuttcap%
\pgfsetmiterjoin%
\definecolor{currentfill}{rgb}{0.381776,0.656517,0.824452}%
\pgfsetfillcolor{currentfill}%
\pgfsetlinewidth{0.000000pt}%
\definecolor{currentstroke}{rgb}{0.000000,0.000000,0.000000}%
\pgfsetstrokecolor{currentstroke}%
\pgfsetstrokeopacity{0.000000}%
\pgfsetdash{}{0pt}%
\pgfpathmoveto{\pgfqpoint{3.090744in}{0.985250in}}%
\pgfpathlineto{\pgfqpoint{3.157750in}{0.985250in}}%
\pgfpathlineto{\pgfqpoint{3.157750in}{1.081500in}}%
\pgfpathlineto{\pgfqpoint{3.090744in}{1.081500in}}%
\pgfpathlineto{\pgfqpoint{3.090744in}{0.985250in}}%
\pgfpathclose%
\pgfusepath{fill}%
\end{pgfscope}%
\begin{pgfscope}%
\pgfpathrectangle{\pgfqpoint{2.857045in}{0.350000in}}{\pgfqpoint{0.300705in}{0.962500in}}%
\pgfusepath{clip}%
\pgfsetbuttcap%
\pgfsetmiterjoin%
\definecolor{currentfill}{rgb}{0.376732,0.653072,0.822484}%
\pgfsetfillcolor{currentfill}%
\pgfsetlinewidth{0.000000pt}%
\definecolor{currentstroke}{rgb}{0.000000,0.000000,0.000000}%
\pgfsetstrokecolor{currentstroke}%
\pgfsetstrokeopacity{0.000000}%
\pgfsetdash{}{0pt}%
\pgfpathmoveto{\pgfqpoint{3.090572in}{0.994875in}}%
\pgfpathlineto{\pgfqpoint{3.157750in}{0.994875in}}%
\pgfpathlineto{\pgfqpoint{3.157750in}{1.091125in}}%
\pgfpathlineto{\pgfqpoint{3.090572in}{1.091125in}}%
\pgfpathlineto{\pgfqpoint{3.090572in}{0.994875in}}%
\pgfpathclose%
\pgfusepath{fill}%
\end{pgfscope}%
\begin{pgfscope}%
\pgfpathrectangle{\pgfqpoint{2.857045in}{0.350000in}}{\pgfqpoint{0.300705in}{0.962500in}}%
\pgfusepath{clip}%
\pgfsetbuttcap%
\pgfsetmiterjoin%
\definecolor{currentfill}{rgb}{0.376732,0.653072,0.822484}%
\pgfsetfillcolor{currentfill}%
\pgfsetlinewidth{0.000000pt}%
\definecolor{currentstroke}{rgb}{0.000000,0.000000,0.000000}%
\pgfsetstrokecolor{currentstroke}%
\pgfsetstrokeopacity{0.000000}%
\pgfsetdash{}{0pt}%
\pgfpathmoveto{\pgfqpoint{3.090572in}{1.004500in}}%
\pgfpathlineto{\pgfqpoint{3.157750in}{1.004500in}}%
\pgfpathlineto{\pgfqpoint{3.157750in}{1.100750in}}%
\pgfpathlineto{\pgfqpoint{3.090572in}{1.100750in}}%
\pgfpathlineto{\pgfqpoint{3.090572in}{1.004500in}}%
\pgfpathclose%
\pgfusepath{fill}%
\end{pgfscope}%
\begin{pgfscope}%
\pgfpathrectangle{\pgfqpoint{2.857045in}{0.350000in}}{\pgfqpoint{0.300705in}{0.962500in}}%
\pgfusepath{clip}%
\pgfsetbuttcap%
\pgfsetmiterjoin%
\definecolor{currentfill}{rgb}{0.371688,0.649627,0.820515}%
\pgfsetfillcolor{currentfill}%
\pgfsetlinewidth{0.000000pt}%
\definecolor{currentstroke}{rgb}{0.000000,0.000000,0.000000}%
\pgfsetstrokecolor{currentstroke}%
\pgfsetstrokeopacity{0.000000}%
\pgfsetdash{}{0pt}%
\pgfpathmoveto{\pgfqpoint{3.090370in}{1.014125in}}%
\pgfpathlineto{\pgfqpoint{3.157750in}{1.014125in}}%
\pgfpathlineto{\pgfqpoint{3.157750in}{1.110375in}}%
\pgfpathlineto{\pgfqpoint{3.090370in}{1.110375in}}%
\pgfpathlineto{\pgfqpoint{3.090370in}{1.014125in}}%
\pgfpathclose%
\pgfusepath{fill}%
\end{pgfscope}%
\begin{pgfscope}%
\pgfpathrectangle{\pgfqpoint{2.857045in}{0.350000in}}{\pgfqpoint{0.300705in}{0.962500in}}%
\pgfusepath{clip}%
\pgfsetbuttcap%
\pgfsetmiterjoin%
\definecolor{currentfill}{rgb}{0.371688,0.649627,0.820515}%
\pgfsetfillcolor{currentfill}%
\pgfsetlinewidth{0.000000pt}%
\definecolor{currentstroke}{rgb}{0.000000,0.000000,0.000000}%
\pgfsetstrokecolor{currentstroke}%
\pgfsetstrokeopacity{0.000000}%
\pgfsetdash{}{0pt}%
\pgfpathmoveto{\pgfqpoint{3.090370in}{1.023750in}}%
\pgfpathlineto{\pgfqpoint{3.157750in}{1.023750in}}%
\pgfpathlineto{\pgfqpoint{3.157750in}{1.120000in}}%
\pgfpathlineto{\pgfqpoint{3.090370in}{1.120000in}}%
\pgfpathlineto{\pgfqpoint{3.090370in}{1.023750in}}%
\pgfpathclose%
\pgfusepath{fill}%
\end{pgfscope}%
\begin{pgfscope}%
\pgfpathrectangle{\pgfqpoint{2.857045in}{0.350000in}}{\pgfqpoint{0.300705in}{0.962500in}}%
\pgfusepath{clip}%
\pgfsetbuttcap%
\pgfsetmiterjoin%
\definecolor{currentfill}{rgb}{0.366644,0.646182,0.818547}%
\pgfsetfillcolor{currentfill}%
\pgfsetlinewidth{0.000000pt}%
\definecolor{currentstroke}{rgb}{0.000000,0.000000,0.000000}%
\pgfsetstrokecolor{currentstroke}%
\pgfsetstrokeopacity{0.000000}%
\pgfsetdash{}{0pt}%
\pgfpathmoveto{\pgfqpoint{3.090080in}{1.033375in}}%
\pgfpathlineto{\pgfqpoint{3.157750in}{1.033375in}}%
\pgfpathlineto{\pgfqpoint{3.157750in}{1.129625in}}%
\pgfpathlineto{\pgfqpoint{3.090080in}{1.129625in}}%
\pgfpathlineto{\pgfqpoint{3.090080in}{1.033375in}}%
\pgfpathclose%
\pgfusepath{fill}%
\end{pgfscope}%
\begin{pgfscope}%
\pgfpathrectangle{\pgfqpoint{2.857045in}{0.350000in}}{\pgfqpoint{0.300705in}{0.962500in}}%
\pgfusepath{clip}%
\pgfsetbuttcap%
\pgfsetmiterjoin%
\definecolor{currentfill}{rgb}{0.361599,0.642737,0.816578}%
\pgfsetfillcolor{currentfill}%
\pgfsetlinewidth{0.000000pt}%
\definecolor{currentstroke}{rgb}{0.000000,0.000000,0.000000}%
\pgfsetstrokecolor{currentstroke}%
\pgfsetstrokeopacity{0.000000}%
\pgfsetdash{}{0pt}%
\pgfpathmoveto{\pgfqpoint{3.089945in}{1.043000in}}%
\pgfpathlineto{\pgfqpoint{3.157750in}{1.043000in}}%
\pgfpathlineto{\pgfqpoint{3.157750in}{1.139250in}}%
\pgfpathlineto{\pgfqpoint{3.089945in}{1.139250in}}%
\pgfpathlineto{\pgfqpoint{3.089945in}{1.043000in}}%
\pgfpathclose%
\pgfusepath{fill}%
\end{pgfscope}%
\begin{pgfscope}%
\pgfpathrectangle{\pgfqpoint{2.857045in}{0.350000in}}{\pgfqpoint{0.300705in}{0.962500in}}%
\pgfusepath{clip}%
\pgfsetbuttcap%
\pgfsetmiterjoin%
\definecolor{currentfill}{rgb}{0.361599,0.642737,0.816578}%
\pgfsetfillcolor{currentfill}%
\pgfsetlinewidth{0.000000pt}%
\definecolor{currentstroke}{rgb}{0.000000,0.000000,0.000000}%
\pgfsetstrokecolor{currentstroke}%
\pgfsetstrokeopacity{0.000000}%
\pgfsetdash{}{0pt}%
\pgfpathmoveto{\pgfqpoint{3.089932in}{1.052625in}}%
\pgfpathlineto{\pgfqpoint{3.157750in}{1.052625in}}%
\pgfpathlineto{\pgfqpoint{3.157750in}{1.148875in}}%
\pgfpathlineto{\pgfqpoint{3.089932in}{1.148875in}}%
\pgfpathlineto{\pgfqpoint{3.089932in}{1.052625in}}%
\pgfpathclose%
\pgfusepath{fill}%
\end{pgfscope}%
\begin{pgfscope}%
\pgfpathrectangle{\pgfqpoint{2.857045in}{0.350000in}}{\pgfqpoint{0.300705in}{0.962500in}}%
\pgfusepath{clip}%
\pgfsetbuttcap%
\pgfsetmiterjoin%
\definecolor{currentfill}{rgb}{0.351511,0.635848,0.812641}%
\pgfsetfillcolor{currentfill}%
\pgfsetlinewidth{0.000000pt}%
\definecolor{currentstroke}{rgb}{0.000000,0.000000,0.000000}%
\pgfsetstrokecolor{currentstroke}%
\pgfsetstrokeopacity{0.000000}%
\pgfsetdash{}{0pt}%
\pgfpathmoveto{\pgfqpoint{3.089626in}{1.062250in}}%
\pgfpathlineto{\pgfqpoint{3.157750in}{1.062250in}}%
\pgfpathlineto{\pgfqpoint{3.157750in}{1.158500in}}%
\pgfpathlineto{\pgfqpoint{3.089626in}{1.158500in}}%
\pgfpathlineto{\pgfqpoint{3.089626in}{1.062250in}}%
\pgfpathclose%
\pgfusepath{fill}%
\end{pgfscope}%
\begin{pgfscope}%
\pgfpathrectangle{\pgfqpoint{2.857045in}{0.350000in}}{\pgfqpoint{0.300705in}{0.962500in}}%
\pgfusepath{clip}%
\pgfsetbuttcap%
\pgfsetmiterjoin%
\definecolor{currentfill}{rgb}{0.351511,0.635848,0.812641}%
\pgfsetfillcolor{currentfill}%
\pgfsetlinewidth{0.000000pt}%
\definecolor{currentstroke}{rgb}{0.000000,0.000000,0.000000}%
\pgfsetstrokecolor{currentstroke}%
\pgfsetstrokeopacity{0.000000}%
\pgfsetdash{}{0pt}%
\pgfpathmoveto{\pgfqpoint{3.089486in}{1.071875in}}%
\pgfpathlineto{\pgfqpoint{3.157750in}{1.071875in}}%
\pgfpathlineto{\pgfqpoint{3.157750in}{1.168125in}}%
\pgfpathlineto{\pgfqpoint{3.089486in}{1.168125in}}%
\pgfpathlineto{\pgfqpoint{3.089486in}{1.071875in}}%
\pgfpathclose%
\pgfusepath{fill}%
\end{pgfscope}%
\begin{pgfscope}%
\pgfpathrectangle{\pgfqpoint{2.857045in}{0.350000in}}{\pgfqpoint{0.300705in}{0.962500in}}%
\pgfusepath{clip}%
\pgfsetbuttcap%
\pgfsetmiterjoin%
\definecolor{currentfill}{rgb}{0.346467,0.632403,0.810673}%
\pgfsetfillcolor{currentfill}%
\pgfsetlinewidth{0.000000pt}%
\definecolor{currentstroke}{rgb}{0.000000,0.000000,0.000000}%
\pgfsetstrokecolor{currentstroke}%
\pgfsetstrokeopacity{0.000000}%
\pgfsetdash{}{0pt}%
\pgfpathmoveto{\pgfqpoint{3.089241in}{1.081500in}}%
\pgfpathlineto{\pgfqpoint{3.157750in}{1.081500in}}%
\pgfpathlineto{\pgfqpoint{3.157750in}{1.177750in}}%
\pgfpathlineto{\pgfqpoint{3.089241in}{1.177750in}}%
\pgfpathlineto{\pgfqpoint{3.089241in}{1.081500in}}%
\pgfpathclose%
\pgfusepath{fill}%
\end{pgfscope}%
\begin{pgfscope}%
\pgfpathrectangle{\pgfqpoint{2.857045in}{0.350000in}}{\pgfqpoint{0.300705in}{0.962500in}}%
\pgfusepath{clip}%
\pgfsetbuttcap%
\pgfsetmiterjoin%
\definecolor{currentfill}{rgb}{0.336378,0.625513,0.806736}%
\pgfsetfillcolor{currentfill}%
\pgfsetlinewidth{0.000000pt}%
\definecolor{currentstroke}{rgb}{0.000000,0.000000,0.000000}%
\pgfsetstrokecolor{currentstroke}%
\pgfsetstrokeopacity{0.000000}%
\pgfsetdash{}{0pt}%
\pgfpathmoveto{\pgfqpoint{3.088864in}{1.091125in}}%
\pgfpathlineto{\pgfqpoint{3.157750in}{1.091125in}}%
\pgfpathlineto{\pgfqpoint{3.157750in}{1.187375in}}%
\pgfpathlineto{\pgfqpoint{3.088864in}{1.187375in}}%
\pgfpathlineto{\pgfqpoint{3.088864in}{1.091125in}}%
\pgfpathclose%
\pgfusepath{fill}%
\end{pgfscope}%
\begin{pgfscope}%
\pgfpathrectangle{\pgfqpoint{2.857045in}{0.350000in}}{\pgfqpoint{0.300705in}{0.962500in}}%
\pgfusepath{clip}%
\pgfsetbuttcap%
\pgfsetmiterjoin%
\definecolor{currentfill}{rgb}{0.336378,0.625513,0.806736}%
\pgfsetfillcolor{currentfill}%
\pgfsetlinewidth{0.000000pt}%
\definecolor{currentstroke}{rgb}{0.000000,0.000000,0.000000}%
\pgfsetstrokecolor{currentstroke}%
\pgfsetstrokeopacity{0.000000}%
\pgfsetdash{}{0pt}%
\pgfpathmoveto{\pgfqpoint{3.088766in}{1.100750in}}%
\pgfpathlineto{\pgfqpoint{3.157750in}{1.100750in}}%
\pgfpathlineto{\pgfqpoint{3.157750in}{1.197000in}}%
\pgfpathlineto{\pgfqpoint{3.088766in}{1.197000in}}%
\pgfpathlineto{\pgfqpoint{3.088766in}{1.100750in}}%
\pgfpathclose%
\pgfusepath{fill}%
\end{pgfscope}%
\begin{pgfscope}%
\pgfpathrectangle{\pgfqpoint{2.857045in}{0.350000in}}{\pgfqpoint{0.300705in}{0.962500in}}%
\pgfusepath{clip}%
\pgfsetbuttcap%
\pgfsetmiterjoin%
\definecolor{currentfill}{rgb}{0.326290,0.618624,0.802799}%
\pgfsetfillcolor{currentfill}%
\pgfsetlinewidth{0.000000pt}%
\definecolor{currentstroke}{rgb}{0.000000,0.000000,0.000000}%
\pgfsetstrokecolor{currentstroke}%
\pgfsetstrokeopacity{0.000000}%
\pgfsetdash{}{0pt}%
\pgfpathmoveto{\pgfqpoint{3.088423in}{1.110375in}}%
\pgfpathlineto{\pgfqpoint{3.157750in}{1.110375in}}%
\pgfpathlineto{\pgfqpoint{3.157750in}{1.206625in}}%
\pgfpathlineto{\pgfqpoint{3.088423in}{1.206625in}}%
\pgfpathlineto{\pgfqpoint{3.088423in}{1.110375in}}%
\pgfpathclose%
\pgfusepath{fill}%
\end{pgfscope}%
\begin{pgfscope}%
\pgfpathrectangle{\pgfqpoint{2.857045in}{0.350000in}}{\pgfqpoint{0.300705in}{0.962500in}}%
\pgfusepath{clip}%
\pgfsetbuttcap%
\pgfsetmiterjoin%
\definecolor{currentfill}{rgb}{0.311157,0.608289,0.796894}%
\pgfsetfillcolor{currentfill}%
\pgfsetlinewidth{0.000000pt}%
\definecolor{currentstroke}{rgb}{0.000000,0.000000,0.000000}%
\pgfsetstrokecolor{currentstroke}%
\pgfsetstrokeopacity{0.000000}%
\pgfsetdash{}{0pt}%
\pgfpathmoveto{\pgfqpoint{3.087813in}{1.120000in}}%
\pgfpathlineto{\pgfqpoint{3.157750in}{1.120000in}}%
\pgfpathlineto{\pgfqpoint{3.157750in}{1.216250in}}%
\pgfpathlineto{\pgfqpoint{3.087813in}{1.216250in}}%
\pgfpathlineto{\pgfqpoint{3.087813in}{1.120000in}}%
\pgfpathclose%
\pgfusepath{fill}%
\end{pgfscope}%
\begin{pgfscope}%
\pgfpathrectangle{\pgfqpoint{2.857045in}{0.350000in}}{\pgfqpoint{0.300705in}{0.962500in}}%
\pgfusepath{clip}%
\pgfsetbuttcap%
\pgfsetmiterjoin%
\definecolor{currentfill}{rgb}{0.311157,0.608289,0.796894}%
\pgfsetfillcolor{currentfill}%
\pgfsetlinewidth{0.000000pt}%
\definecolor{currentstroke}{rgb}{0.000000,0.000000,0.000000}%
\pgfsetstrokecolor{currentstroke}%
\pgfsetstrokeopacity{0.000000}%
\pgfsetdash{}{0pt}%
\pgfpathmoveto{\pgfqpoint{3.087661in}{1.129625in}}%
\pgfpathlineto{\pgfqpoint{3.157750in}{1.129625in}}%
\pgfpathlineto{\pgfqpoint{3.157750in}{1.225875in}}%
\pgfpathlineto{\pgfqpoint{3.087661in}{1.225875in}}%
\pgfpathlineto{\pgfqpoint{3.087661in}{1.129625in}}%
\pgfpathclose%
\pgfusepath{fill}%
\end{pgfscope}%
\begin{pgfscope}%
\pgfpathrectangle{\pgfqpoint{2.857045in}{0.350000in}}{\pgfqpoint{0.300705in}{0.962500in}}%
\pgfusepath{clip}%
\pgfsetbuttcap%
\pgfsetmiterjoin%
\definecolor{currentfill}{rgb}{0.311157,0.608289,0.796894}%
\pgfsetfillcolor{currentfill}%
\pgfsetlinewidth{0.000000pt}%
\definecolor{currentstroke}{rgb}{0.000000,0.000000,0.000000}%
\pgfsetstrokecolor{currentstroke}%
\pgfsetstrokeopacity{0.000000}%
\pgfsetdash{}{0pt}%
\pgfpathmoveto{\pgfqpoint{3.087661in}{1.139250in}}%
\pgfpathlineto{\pgfqpoint{3.157750in}{1.139250in}}%
\pgfpathlineto{\pgfqpoint{3.157750in}{1.235500in}}%
\pgfpathlineto{\pgfqpoint{3.087661in}{1.235500in}}%
\pgfpathlineto{\pgfqpoint{3.087661in}{1.139250in}}%
\pgfpathclose%
\pgfusepath{fill}%
\end{pgfscope}%
\begin{pgfscope}%
\pgfpathrectangle{\pgfqpoint{2.857045in}{0.350000in}}{\pgfqpoint{0.300705in}{0.962500in}}%
\pgfusepath{clip}%
\pgfsetbuttcap%
\pgfsetmiterjoin%
\definecolor{currentfill}{rgb}{0.296025,0.597955,0.790988}%
\pgfsetfillcolor{currentfill}%
\pgfsetlinewidth{0.000000pt}%
\definecolor{currentstroke}{rgb}{0.000000,0.000000,0.000000}%
\pgfsetstrokecolor{currentstroke}%
\pgfsetstrokeopacity{0.000000}%
\pgfsetdash{}{0pt}%
\pgfpathmoveto{\pgfqpoint{3.087211in}{1.148875in}}%
\pgfpathlineto{\pgfqpoint{3.157750in}{1.148875in}}%
\pgfpathlineto{\pgfqpoint{3.157750in}{1.245125in}}%
\pgfpathlineto{\pgfqpoint{3.087211in}{1.245125in}}%
\pgfpathlineto{\pgfqpoint{3.087211in}{1.148875in}}%
\pgfpathclose%
\pgfusepath{fill}%
\end{pgfscope}%
\begin{pgfscope}%
\pgfpathrectangle{\pgfqpoint{2.857045in}{0.350000in}}{\pgfqpoint{0.300705in}{0.962500in}}%
\pgfusepath{clip}%
\pgfsetbuttcap%
\pgfsetmiterjoin%
\definecolor{currentfill}{rgb}{0.290980,0.594510,0.789020}%
\pgfsetfillcolor{currentfill}%
\pgfsetlinewidth{0.000000pt}%
\definecolor{currentstroke}{rgb}{0.000000,0.000000,0.000000}%
\pgfsetstrokecolor{currentstroke}%
\pgfsetstrokeopacity{0.000000}%
\pgfsetdash{}{0pt}%
\pgfpathmoveto{\pgfqpoint{3.086798in}{1.158500in}}%
\pgfpathlineto{\pgfqpoint{3.157750in}{1.158500in}}%
\pgfpathlineto{\pgfqpoint{3.157750in}{1.254750in}}%
\pgfpathlineto{\pgfqpoint{3.086798in}{1.254750in}}%
\pgfpathlineto{\pgfqpoint{3.086798in}{1.158500in}}%
\pgfpathclose%
\pgfusepath{fill}%
\end{pgfscope}%
\begin{pgfscope}%
\pgfpathrectangle{\pgfqpoint{2.857045in}{0.350000in}}{\pgfqpoint{0.300705in}{0.962500in}}%
\pgfusepath{clip}%
\pgfsetbuttcap%
\pgfsetmiterjoin%
\definecolor{currentfill}{rgb}{0.270804,0.580730,0.781146}%
\pgfsetfillcolor{currentfill}%
\pgfsetlinewidth{0.000000pt}%
\definecolor{currentstroke}{rgb}{0.000000,0.000000,0.000000}%
\pgfsetstrokecolor{currentstroke}%
\pgfsetstrokeopacity{0.000000}%
\pgfsetdash{}{0pt}%
\pgfpathmoveto{\pgfqpoint{3.085933in}{1.168125in}}%
\pgfpathlineto{\pgfqpoint{3.157750in}{1.168125in}}%
\pgfpathlineto{\pgfqpoint{3.157750in}{1.264375in}}%
\pgfpathlineto{\pgfqpoint{3.085933in}{1.264375in}}%
\pgfpathlineto{\pgfqpoint{3.085933in}{1.168125in}}%
\pgfpathclose%
\pgfusepath{fill}%
\end{pgfscope}%
\begin{pgfscope}%
\pgfpathrectangle{\pgfqpoint{2.857045in}{0.350000in}}{\pgfqpoint{0.300705in}{0.962500in}}%
\pgfusepath{clip}%
\pgfsetbuttcap%
\pgfsetmiterjoin%
\definecolor{currentfill}{rgb}{0.256286,0.570012,0.775163}%
\pgfsetfillcolor{currentfill}%
\pgfsetlinewidth{0.000000pt}%
\definecolor{currentstroke}{rgb}{0.000000,0.000000,0.000000}%
\pgfsetstrokecolor{currentstroke}%
\pgfsetstrokeopacity{0.000000}%
\pgfsetdash{}{0pt}%
\pgfpathmoveto{\pgfqpoint{3.085385in}{1.177750in}}%
\pgfpathlineto{\pgfqpoint{3.157750in}{1.177750in}}%
\pgfpathlineto{\pgfqpoint{3.157750in}{1.274000in}}%
\pgfpathlineto{\pgfqpoint{3.085385in}{1.274000in}}%
\pgfpathlineto{\pgfqpoint{3.085385in}{1.177750in}}%
\pgfpathclose%
\pgfusepath{fill}%
\end{pgfscope}%
\begin{pgfscope}%
\pgfpathrectangle{\pgfqpoint{2.857045in}{0.350000in}}{\pgfqpoint{0.300705in}{0.962500in}}%
\pgfusepath{clip}%
\pgfsetbuttcap%
\pgfsetmiterjoin%
\definecolor{currentfill}{rgb}{0.256286,0.570012,0.775163}%
\pgfsetfillcolor{currentfill}%
\pgfsetlinewidth{0.000000pt}%
\definecolor{currentstroke}{rgb}{0.000000,0.000000,0.000000}%
\pgfsetstrokecolor{currentstroke}%
\pgfsetstrokeopacity{0.000000}%
\pgfsetdash{}{0pt}%
\pgfpathmoveto{\pgfqpoint{3.085357in}{1.187375in}}%
\pgfpathlineto{\pgfqpoint{3.157750in}{1.187375in}}%
\pgfpathlineto{\pgfqpoint{3.157750in}{1.283625in}}%
\pgfpathlineto{\pgfqpoint{3.085357in}{1.283625in}}%
\pgfpathlineto{\pgfqpoint{3.085357in}{1.187375in}}%
\pgfpathclose%
\pgfusepath{fill}%
\end{pgfscope}%
\begin{pgfscope}%
\pgfpathrectangle{\pgfqpoint{2.857045in}{0.350000in}}{\pgfqpoint{0.300705in}{0.962500in}}%
\pgfusepath{clip}%
\pgfsetbuttcap%
\pgfsetmiterjoin%
\definecolor{currentfill}{rgb}{0.244106,0.557832,0.768889}%
\pgfsetfillcolor{currentfill}%
\pgfsetlinewidth{0.000000pt}%
\definecolor{currentstroke}{rgb}{0.000000,0.000000,0.000000}%
\pgfsetstrokecolor{currentstroke}%
\pgfsetstrokeopacity{0.000000}%
\pgfsetdash{}{0pt}%
\pgfpathmoveto{\pgfqpoint{3.084778in}{1.197000in}}%
\pgfpathlineto{\pgfqpoint{3.157750in}{1.197000in}}%
\pgfpathlineto{\pgfqpoint{3.157750in}{1.293250in}}%
\pgfpathlineto{\pgfqpoint{3.084778in}{1.293250in}}%
\pgfpathlineto{\pgfqpoint{3.084778in}{1.197000in}}%
\pgfpathclose%
\pgfusepath{fill}%
\end{pgfscope}%
\begin{pgfscope}%
\pgfpathrectangle{\pgfqpoint{2.857045in}{0.350000in}}{\pgfqpoint{0.300705in}{0.962500in}}%
\pgfusepath{clip}%
\pgfsetbuttcap%
\pgfsetmiterjoin%
\definecolor{currentfill}{rgb}{0.223806,0.537532,0.758431}%
\pgfsetfillcolor{currentfill}%
\pgfsetlinewidth{0.000000pt}%
\definecolor{currentstroke}{rgb}{0.000000,0.000000,0.000000}%
\pgfsetstrokecolor{currentstroke}%
\pgfsetstrokeopacity{0.000000}%
\pgfsetdash{}{0pt}%
\pgfpathmoveto{\pgfqpoint{3.083688in}{1.206625in}}%
\pgfpathlineto{\pgfqpoint{3.157750in}{1.206625in}}%
\pgfpathlineto{\pgfqpoint{3.157750in}{1.302875in}}%
\pgfpathlineto{\pgfqpoint{3.083688in}{1.302875in}}%
\pgfpathlineto{\pgfqpoint{3.083688in}{1.206625in}}%
\pgfpathclose%
\pgfusepath{fill}%
\end{pgfscope}%
\begin{pgfscope}%
\pgfpathrectangle{\pgfqpoint{2.857045in}{0.350000in}}{\pgfqpoint{0.300705in}{0.962500in}}%
\pgfusepath{clip}%
\pgfsetbuttcap%
\pgfsetmiterjoin%
\definecolor{currentfill}{rgb}{0.223806,0.537532,0.758431}%
\pgfsetfillcolor{currentfill}%
\pgfsetlinewidth{0.000000pt}%
\definecolor{currentstroke}{rgb}{0.000000,0.000000,0.000000}%
\pgfsetstrokecolor{currentstroke}%
\pgfsetstrokeopacity{0.000000}%
\pgfsetdash{}{0pt}%
\pgfpathmoveto{\pgfqpoint{3.083619in}{1.216250in}}%
\pgfpathlineto{\pgfqpoint{3.157750in}{1.216250in}}%
\pgfpathlineto{\pgfqpoint{3.157750in}{1.312500in}}%
\pgfpathlineto{\pgfqpoint{3.083619in}{1.312500in}}%
\pgfpathlineto{\pgfqpoint{3.083619in}{1.216250in}}%
\pgfpathclose%
\pgfusepath{fill}%
\end{pgfscope}%
\begin{pgfscope}%
\pgfpathrectangle{\pgfqpoint{2.857045in}{0.350000in}}{\pgfqpoint{0.300705in}{0.962500in}}%
\pgfusepath{clip}%
\pgfsetbuttcap%
\pgfsetmiterjoin%
\definecolor{currentfill}{rgb}{0.215686,0.529412,0.754248}%
\pgfsetfillcolor{currentfill}%
\pgfsetlinewidth{0.000000pt}%
\definecolor{currentstroke}{rgb}{0.000000,0.000000,0.000000}%
\pgfsetstrokecolor{currentstroke}%
\pgfsetstrokeopacity{0.000000}%
\pgfsetdash{}{0pt}%
\pgfpathmoveto{\pgfqpoint{3.083105in}{1.225875in}}%
\pgfpathlineto{\pgfqpoint{3.157750in}{1.225875in}}%
\pgfpathlineto{\pgfqpoint{3.157750in}{1.322125in}}%
\pgfpathlineto{\pgfqpoint{3.083105in}{1.322125in}}%
\pgfpathlineto{\pgfqpoint{3.083105in}{1.225875in}}%
\pgfpathclose%
\pgfusepath{fill}%
\end{pgfscope}%
\begin{pgfscope}%
\pgfpathrectangle{\pgfqpoint{2.857045in}{0.350000in}}{\pgfqpoint{0.300705in}{0.962500in}}%
\pgfusepath{clip}%
\pgfsetbuttcap%
\pgfsetmiterjoin%
\definecolor{currentfill}{rgb}{0.207566,0.521292,0.750065}%
\pgfsetfillcolor{currentfill}%
\pgfsetlinewidth{0.000000pt}%
\definecolor{currentstroke}{rgb}{0.000000,0.000000,0.000000}%
\pgfsetstrokecolor{currentstroke}%
\pgfsetstrokeopacity{0.000000}%
\pgfsetdash{}{0pt}%
\pgfpathmoveto{\pgfqpoint{3.082791in}{1.235500in}}%
\pgfpathlineto{\pgfqpoint{3.157750in}{1.235500in}}%
\pgfpathlineto{\pgfqpoint{3.157750in}{1.331750in}}%
\pgfpathlineto{\pgfqpoint{3.082791in}{1.331750in}}%
\pgfpathlineto{\pgfqpoint{3.082791in}{1.235500in}}%
\pgfpathclose%
\pgfusepath{fill}%
\end{pgfscope}%
\begin{pgfscope}%
\pgfpathrectangle{\pgfqpoint{2.857045in}{0.350000in}}{\pgfqpoint{0.300705in}{0.962500in}}%
\pgfusepath{clip}%
\pgfsetbuttcap%
\pgfsetmiterjoin%
\definecolor{currentfill}{rgb}{0.187266,0.500992,0.739608}%
\pgfsetfillcolor{currentfill}%
\pgfsetlinewidth{0.000000pt}%
\definecolor{currentstroke}{rgb}{0.000000,0.000000,0.000000}%
\pgfsetstrokecolor{currentstroke}%
\pgfsetstrokeopacity{0.000000}%
\pgfsetdash{}{0pt}%
\pgfpathmoveto{\pgfqpoint{3.081559in}{1.245125in}}%
\pgfpathlineto{\pgfqpoint{3.157750in}{1.245125in}}%
\pgfpathlineto{\pgfqpoint{3.157750in}{1.341375in}}%
\pgfpathlineto{\pgfqpoint{3.081559in}{1.341375in}}%
\pgfpathlineto{\pgfqpoint{3.081559in}{1.245125in}}%
\pgfpathclose%
\pgfusepath{fill}%
\end{pgfscope}%
\begin{pgfscope}%
\pgfpathrectangle{\pgfqpoint{2.857045in}{0.350000in}}{\pgfqpoint{0.300705in}{0.962500in}}%
\pgfusepath{clip}%
\pgfsetbuttcap%
\pgfsetmiterjoin%
\definecolor{currentfill}{rgb}{0.166967,0.480692,0.729150}%
\pgfsetfillcolor{currentfill}%
\pgfsetlinewidth{0.000000pt}%
\definecolor{currentstroke}{rgb}{0.000000,0.000000,0.000000}%
\pgfsetstrokecolor{currentstroke}%
\pgfsetstrokeopacity{0.000000}%
\pgfsetdash{}{0pt}%
\pgfpathmoveto{\pgfqpoint{3.080544in}{1.254750in}}%
\pgfpathlineto{\pgfqpoint{3.157750in}{1.254750in}}%
\pgfpathlineto{\pgfqpoint{3.157750in}{1.351000in}}%
\pgfpathlineto{\pgfqpoint{3.080544in}{1.351000in}}%
\pgfpathlineto{\pgfqpoint{3.080544in}{1.254750in}}%
\pgfpathclose%
\pgfusepath{fill}%
\end{pgfscope}%
\begin{pgfscope}%
\pgfpathrectangle{\pgfqpoint{2.857045in}{0.350000in}}{\pgfqpoint{0.300705in}{0.962500in}}%
\pgfusepath{clip}%
\pgfsetbuttcap%
\pgfsetmiterjoin%
\definecolor{currentfill}{rgb}{0.166967,0.480692,0.729150}%
\pgfsetfillcolor{currentfill}%
\pgfsetlinewidth{0.000000pt}%
\definecolor{currentstroke}{rgb}{0.000000,0.000000,0.000000}%
\pgfsetstrokecolor{currentstroke}%
\pgfsetstrokeopacity{0.000000}%
\pgfsetdash{}{0pt}%
\pgfpathmoveto{\pgfqpoint{3.080544in}{1.264375in}}%
\pgfpathlineto{\pgfqpoint{3.157750in}{1.264375in}}%
\pgfpathlineto{\pgfqpoint{3.157750in}{1.360625in}}%
\pgfpathlineto{\pgfqpoint{3.080544in}{1.360625in}}%
\pgfpathlineto{\pgfqpoint{3.080544in}{1.264375in}}%
\pgfpathclose%
\pgfusepath{fill}%
\end{pgfscope}%
\begin{pgfscope}%
\pgfpathrectangle{\pgfqpoint{2.857045in}{0.350000in}}{\pgfqpoint{0.300705in}{0.962500in}}%
\pgfusepath{clip}%
\pgfsetbuttcap%
\pgfsetmiterjoin%
\definecolor{currentfill}{rgb}{0.130427,0.444152,0.710327}%
\pgfsetfillcolor{currentfill}%
\pgfsetlinewidth{0.000000pt}%
\definecolor{currentstroke}{rgb}{0.000000,0.000000,0.000000}%
\pgfsetstrokecolor{currentstroke}%
\pgfsetstrokeopacity{0.000000}%
\pgfsetdash{}{0pt}%
\pgfpathmoveto{\pgfqpoint{3.078510in}{1.274000in}}%
\pgfpathlineto{\pgfqpoint{3.157750in}{1.274000in}}%
\pgfpathlineto{\pgfqpoint{3.157750in}{1.370250in}}%
\pgfpathlineto{\pgfqpoint{3.078510in}{1.370250in}}%
\pgfpathlineto{\pgfqpoint{3.078510in}{1.274000in}}%
\pgfpathclose%
\pgfusepath{fill}%
\end{pgfscope}%
\begin{pgfscope}%
\pgfpathrectangle{\pgfqpoint{2.857045in}{0.350000in}}{\pgfqpoint{0.300705in}{0.962500in}}%
\pgfusepath{clip}%
\pgfsetbuttcap%
\pgfsetmiterjoin%
\definecolor{currentfill}{rgb}{0.087120,0.389004,0.667512}%
\pgfsetfillcolor{currentfill}%
\pgfsetlinewidth{0.000000pt}%
\definecolor{currentstroke}{rgb}{0.000000,0.000000,0.000000}%
\pgfsetstrokecolor{currentstroke}%
\pgfsetstrokeopacity{0.000000}%
\pgfsetdash{}{0pt}%
\pgfpathmoveto{\pgfqpoint{3.075409in}{1.283625in}}%
\pgfpathlineto{\pgfqpoint{3.157750in}{1.283625in}}%
\pgfpathlineto{\pgfqpoint{3.157750in}{1.379875in}}%
\pgfpathlineto{\pgfqpoint{3.075409in}{1.379875in}}%
\pgfpathlineto{\pgfqpoint{3.075409in}{1.283625in}}%
\pgfpathclose%
\pgfusepath{fill}%
\end{pgfscope}%
\begin{pgfscope}%
\pgfpathrectangle{\pgfqpoint{2.857045in}{0.350000in}}{\pgfqpoint{0.300705in}{0.962500in}}%
\pgfusepath{clip}%
\pgfsetbuttcap%
\pgfsetmiterjoin%
\definecolor{currentfill}{rgb}{0.040984,0.329950,0.621376}%
\pgfsetfillcolor{currentfill}%
\pgfsetlinewidth{0.000000pt}%
\definecolor{currentstroke}{rgb}{0.000000,0.000000,0.000000}%
\pgfsetstrokecolor{currentstroke}%
\pgfsetstrokeopacity{0.000000}%
\pgfsetdash{}{0pt}%
\pgfpathmoveto{\pgfqpoint{3.072243in}{1.293250in}}%
\pgfpathlineto{\pgfqpoint{3.157750in}{1.293250in}}%
\pgfpathlineto{\pgfqpoint{3.157750in}{1.389500in}}%
\pgfpathlineto{\pgfqpoint{3.072243in}{1.389500in}}%
\pgfpathlineto{\pgfqpoint{3.072243in}{1.293250in}}%
\pgfpathclose%
\pgfusepath{fill}%
\end{pgfscope}%
\begin{pgfscope}%
\pgfpathrectangle{\pgfqpoint{2.857045in}{0.350000in}}{\pgfqpoint{0.300705in}{0.962500in}}%
\pgfusepath{clip}%
\pgfsetbuttcap%
\pgfsetmiterjoin%
\definecolor{currentfill}{rgb}{0.031373,0.188235,0.419608}%
\pgfsetfillcolor{currentfill}%
\pgfsetlinewidth{0.000000pt}%
\definecolor{currentstroke}{rgb}{0.000000,0.000000,0.000000}%
\pgfsetstrokecolor{currentstroke}%
\pgfsetstrokeopacity{0.000000}%
\pgfsetdash{}{0pt}%
\pgfpathmoveto{\pgfqpoint{3.064343in}{1.302875in}}%
\pgfpathlineto{\pgfqpoint{3.157750in}{1.302875in}}%
\pgfpathlineto{\pgfqpoint{3.157750in}{1.399125in}}%
\pgfpathlineto{\pgfqpoint{3.064343in}{1.399125in}}%
\pgfpathlineto{\pgfqpoint{3.064343in}{1.302875in}}%
\pgfpathclose%
\pgfusepath{fill}%
\end{pgfscope}%
\begin{pgfscope}%
\definecolor{textcolor}{rgb}{0.000000,0.000000,0.000000}%
\pgfsetstrokecolor{textcolor}%
\pgfsetfillcolor{textcolor}%
\pgftext[x=2.977327in,y=1.360625in,left,base]{\color{textcolor}\setmainfont{Lato}\rmfamily\fontsize{7.000000}{8.400000}\selectfont 31.1\%}%
\end{pgfscope}%
\begin{pgfscope}%
\definecolor{textcolor}{rgb}{0.000000,0.000000,0.000000}%
\pgfsetstrokecolor{textcolor}%
\pgfsetfillcolor{textcolor}%
\pgftext[x=2.977327in,y=0.234500in,left,base]{\color{textcolor}\setmainfont{Lato}\rmfamily\fontsize{7.000000}{8.400000}\selectfont 12.3\%}%
\end{pgfscope}%
\end{pgfpicture}%
\makeatother%
\endgroup%
 \hspace{-3mm} \input{data/over64pop.pgf}

\vspace{-3mm}
\footnotesize{Source: American Community Survey, Dorn}


\newpage
\begin{minipage}{0.76\textwidth}

\subsection*{\color{black!70} \seriffont Demographics and Household Formation}

\small The Census Bureau estimates that the US population is 327.2 million in 2018 and reports population growth of 0.6 percent over the past year. By age, 22.9 percent are under the age of 18 and 16.1 percent age 65 or older. In 1989, the US population was 246.8 million, with 25.7 percent under 18 and 12.4 percent 65 or older.\\

The rate of household formation since 1989 can offer a high-level overview of some major demographic and economic developments. From 1989 to 1994, \\

This section should capture 1) population, 2) population growth, 3) aging, 4) increased education.\\

\vspace{2mm}

\noindent \normalsize \textbf{Household Formation by Type}\\
\footnotesize{\textit{one-year moving average of annual growth rates}}\\
\noindent \hspace*{-2mm} \begin{tikzpicture}
	\begin{axis}[\bbar{y}{0}, \dateaxisticks ytick={-2, -1, 0, 1, 2},
		xticklabel={`\short{\year}}, clip=false, 
		legend style={at={(0.95, 1.13)}}]
	\rbars
	\sbar{magenta!90!blue}{date}{Renter}{data/hhform.csv}
	\sbar{yellow!60!orange}{date}{Owner}{data/hhform.csv}
	\stdline{black}{date}{pop}{data/hhform.csv}
	\legend{Rented, Owned, Population Growth};
	\end{axis}
\end{tikzpicture}\\
\footnotesize{Source: Census Bureau}

\end{minipage}

\newpage

\begin{minipage}{0.76\textwidth}

\subsection*{\color{black!70} \seriffont Income to Persons}
\small This section looks at income received by people, by type of income, adjusted for inflation using the PCE implicit price deflator. Income is divided into labor income (see\cbox{green!80!black}), which is measured as compensation of employees, capital income (see\cbox{orange!50!yellow}), measured as the sum of proprietor income, rental income, and dividend and interest income, and welfare income  (see\cbox{blue!80!white}), which is measured as transfers to persons less contributions to social insurance. 
\vspace{4mm}

\noindent \normalsize \textbf{Personal Income}\\
\footnotesize{\textit{percentage point contribution to real personal income growth}}\\
\noindent \hspace*{-2mm} \begin{tikzpicture}
	\begin{axis}[\bbar{y}{0}, \dateaxisticks ytick={-10, -5, 0, 5, 10},
		xticklabel={`\short{\year}}, clip=false, yticklabel style={text width=1.5em}, 
		legend style={at={(0.95, 1.13)}}]
	\rbars
	\sbar{green!80!black}{date}{A033RC}{data/pi.csv}
	\sbar{orange!50!yellow}{date}{CAPITAL}{data/pi.csv}
	\sbar{blue!80!white}{date}{TRANSFER}{data/pi.csv}
	\legend{Labor, Capital$^1$, Welfare$^2$};
	\stdnode{3.4cm}{1.1cm}{\scriptsize $^1$ Includes proprietor, rent, and asset income\\ \scriptsize $^2$ Current transfer receipts minus contributions\\\scriptsize \ \ \ \ to government social insurance}
	\end{axis}
\end{tikzpicture}\\
\footnotesize{Source: Bureau of Economic Analysis}


\vspace{4mm}

\small Some descriptive text here. Perhaps mention the total amount for real personal income and the share from labor, capital, and welfare in 1989, 2000, and 2019. Otherwise, at least mention the main three sources and how they contribute to change in personal income in the latest data.\\

\end{minipage}

\noindent \normalsize \textbf{Personal Income by Source}\\
\footnotesize{\textit{percentage point contribution to real personal income growth \hspace{22mm} moving averages}\\ \vspace{4mm}
\noindent \rowcolors{1}{}{black!5} \setlength{\tabcolsep}{3.1pt} \color{black!90}
		{\renewcommand{\arraystretch}{1.55}
		 \begin{tabular}{p{42mm} R{6.7mm} R{6.7mm} R{6.7mm} R{6.7mm} R{6.7mm} 
		   R{7.8mm} R{6.7mm} R{6.7mm} }
			 &   2020  Q3 & `20  Q2 & `20  Q1 & `19  Q4 & `19  Q3 &3-year&10-year&30-year\\  \hspace{2mm}Personal  income &-13.22&38.00&2.80&1.99&1.13&4.11&3.31&2.89\\  \hspace{-1mm}\cbox{green!75!black}  Labor &8.88&-16.28&1.56&1.79&0.14&0.85&1.30&1.46\\  \hspace{4mm}  Wages  and  salaries &7.64&-13.90&1.58&1.63&0.09&0.73&1.12&1.18\\  \hspace{4mm}  Supplements &1.24&-2.38&-0.02&0.16&0.05&0.11&0.18&0.28\\  \hspace{-1mm}\cbox{orange!40!yellow}Capital &3.81&-6.06&0.03&0.35&0.71&0.35&1.02&0.77\\  \hspace{4mm}  Proprietors'  income &5.12&-4.49&0.06&0.31&0.93&0.27&0.28&0.30\\  \hspace{4mm}  Rental  income &0.05&-0.07&0.09&0.06&0.01&0.09&0.22&0.19\\  \hspace{4mm}  Personal  interest  income &-0.42&-0.86&-0.41&0.11&-0.14&-0.05&0.12&0.02\\  \hspace{4mm}  Personal  dividend  income &-0.94&-0.63&0.29&-0.13&-0.08&0.04&0.41&0.25\\  \hspace{-1mm}\cbox{blue!80!white}Welfare &-25.90&60.34&1.21&-0.16&0.29&2.91&0.98&0.66\\  \hspace{4mm}  Social  security &-0.09&0.27&0.47&0.10&0.09&0.18&0.16&0.16\\  \hspace{4mm}  Medicare &0.20&0.54&0.09&0.11&0.16&0.20&0.14&0.16\\  \hspace{4mm}  Medicaid &0.30&1.13&0.06&-0.11&0.11&0.16&0.14&0.15\\  \hspace{4mm}  Unemployment  insurance &-5.99&25.00&0.33&0.00&-0.00&1.48&0.38&0.16\\  \hspace{4mm}  Veterans'  benefits &0.04&0.10&0.08&0.04&0.04&0.05&0.05&0.02\\  \hspace{4mm}  Less  welfare  contributions &-0.90&1.38&-0.30&-0.18&0.02&-0.13&-0.16&-0.19\\ 
			 \hline
		\end{tabular}
		}	\\
		
\vspace{-6mm}
\footnotesize{Source: Bureau of Economic Analysis}


\newpage

\begin{minipage}{0.76\textwidth}

\normalsize

Capital Income \\

Welfare Income \\

\vspace{4mm}

[\textbf{Breakout section on income of the aged}] \\

\small Income of the aged is looking like a very important section. I hadn't realized the extent to which demographics are rapidly putting downward pressure on the employment rate. Something like four percent of the population is shifting from work age to retirement age.\\

It's important here to point out that social security retirement income is the solution, not the problem. There has been a reduction in private retirement benefits in the form of defined benefit pension plans and a shift towards reliance on 401k and IRA plans. But social security has never missed a payment. In contrast, hundreds of thousands of people are pushed in bankruptcy each year by medical bills. By not extending social insurance to younger people, a larger portion of pre-retirement-age people have their savings wiped out by common life events like having children or a period of illness. As a result, more pressure is put on social insurance for retirement. \\

Ideally I would like to look at the demographics (employment adjustment for age), and then replicate the income of the aged calculation on an annual basis to show how many households are kept out of poverty by social security retirement income. \\

\end{minipage}

\newpage 

\begin{minipage}{0.76\textwidth}
\subsection*{\color{black!70} \seriffont Household Expenditures}

\small This section covers household spending on goods (see\cbox{red}), services excluding housing and utilities (see\cbox{blue!75!white}), and shelter (see\cbox{green!85!blue}, calculated as housing services and utilities combined with residential fixed investment). \input{text/pce1.txt}
\vspace{3mm}

\noindent \normalsize \textbf{Consumer Spending and Residential Investment}\\
\footnotesize{\textit{percentage point contribution to GDP growth}}\\
\noindent \hspace*{-2mm} \begin{tikzpicture}
	\begin{axis}[\bbar{y}{0}, \dateaxisticks ytick={-5, 0, 5},
		xticklabel={`\short{\year}}, clip=false, 
		legend style={at={(0.95, 1.13)}}]
	\rbars
	\sbar{red}{date}{DGDSRY}{data/pce.csv}
	\sbar{blue!75!white}{date}{OTHSER}{data/pce.csv}
	\sbar{green!85!blue}{date}{HOUSING}{data/pce.csv}
	\legend{Goods, Services, Shelter$^1$};
	\stdnode{2.2cm}{0.3cm}{\footnotesize $^1$ Includes residential fixed investment}
	\end{axis}
\end{tikzpicture}\\
\footnotesize{Source: Bureau of Economic Analysis}\\

\small

\input{text/pce2.txt} \\

\end{minipage}

\noindent \normalsize \textbf{Consumer Spending and Residential Investment}\\
\footnotesize{\textit{percentage point contribution to real GDP growth \hspace{36mm} moving averages}\\ \vspace{4mm}
\noindent \rowcolors{1}{}{black!5} \setlength{\tabcolsep}{3.1pt} \color{black!90}
		{\renewcommand{\arraystretch}{1.55}
		 \begin{tabular}{p{2.0mm} p{38mm} R{6.7mm} R{6.7mm} R{6.7mm} R{6.7mm} R{6.7mm} 
		   R{7.8mm} R{6.7mm} R{6.7mm} }
			 & &  2024  Q2 & `24  Q1 & `23  Q4 & `23  Q2 &1-year&10-year&30-year\\  &  Consumer  Spending &1.57&0.98&2.20&0.55&1.71&1.87&1.83\\  \cbox{red}  &  Goods &0.55&-0.51&0.67&0.11&0.45&0.89&0.83\\  &  \hspace{1mm}  Motor  Vehicles  \&  Parts &0.12&-0.40&-0.05&-0.27&-0.09&0.08&0.08\\  &  \hspace{1mm}  Furniture  \&  HH  Equipment &0.11&0.00&0.03&0.00&0.06&0.09&0.09\\  &  \hspace{1mm}  Recreational  Durable  Goods &0.11&-0.06&0.18&0.27&0.16&0.23&0.23\\  &  \hspace{1mm}  Groceries &0.09&-0.01&0.04&0.05&0.06&0.13&0.09\\  &  \hspace{1mm}  Clothes  \&  Shoes &-0.06&0.06&0.06&-0.14&0.04&0.07&0.08\\  \cbox{blue!75!white}  &  Services  Excluding  Shelter &0.84&1.37&1.52&0.36&1.10&0.86&0.82\\  &  \hspace{1mm}  Health  Care  Services &0.45&0.77&0.83&0.27&0.59&0.39&0.30\\  &  \hspace{1mm}  Transportation &0.09&0.01&0.09&0.08&0.05&0.05&0.05\\  &  \hspace{1mm}  Recreational &0.11&0.07&0.01&0.04&0.06&0.06&0.06\\  &  \hspace{1mm}  Food  \&  Accommodations &0.05&-0.17&0.32&-0.05&0.13&0.14&0.10\\  &  \hspace{1mm}  Financial  \&  Insurance &0.08&0.26&-0.04&0.21&0.09&0.02&0.10\\  \cbox{green!85!blue}  &  Shelter  Including  Investment &0.13&0.71&0.13&-0.01&0.40&0.20&0.21\\  &  \hspace{1mm}  Housing  Services  \&  Utilities   &0.18&0.12&0.02&0.08&0.17&0.12&0.19\\  &  \hspace{1mm}  Residential  Fixed  Investment &-0.05&0.59&0.11&-0.09&0.23&0.08&0.02\\ 
			 \hline
		\end{tabular}
		}	\\
		
\vspace{-6mm}
\footnotesize{Source: Bureau of Economic Analysis}

\newpage 

\begin{minipage}{0.76\textwidth}

\small \input{text/pcegrowth.txt}\\

\vspace{2mm}

\noindent \normalsize \textbf{Consumer Spending Growth}\\
\footnotesize{\textit{annual growth, per capita real personal consumption expenditures, percent}}\\
\noindent \hspace*{-2mm} \begin{tikzpicture}
	\begin{axis}[\bbar{y}{0}, \dateaxisticks ytick={-2, 0, 2, 4}, 
		xticklabel={`\short{\year}}, clip=false, height=4.0cm]
	\rbars
	\stdline{green!80!black}{date}{value2}{data/pcegrowth.csv}
	\end{axis}
\end{tikzpicture}\\
\footnotesize{Source: Bureau of Economic Analysis} \\

\vspace{2mm}

\normalsize

[Top quintile consumer spending share of gross pre-tax income and bottom 80 percent share] \\

\vspace{4mm}

\small Changes to consumer spending (see {\color{black}\textbf{---}}) are largely the result of changes to income (see\cbox{yellow!50!orange}) and changes to the rate at which income is saved (see\cbox{magenta!90!white}). Changes to other outlays (see\cbox{blue!60!white}) reflect changes in interest payments, fines and fees, and charitable giving. \\

\input{text/pcedecomp.txt}\\


\vspace{4mm} 

\noindent \normalsize \textbf{Contributions to Consumer Spending}\\
\footnotesize{\textit{percentage point contribution to real per capita PCE growth, one-year moving average}}\\
\noindent \hspace*{-3mm} \begin{tikzpicture}
	\begin{axis}[\bbar{y}{0}, \dateaxisticks ytick={-4, -2, 0, 2, 4, 6},
		xticklabel={`\short{\year}}, clip=false, 
		legend style={at={(0.95, 1.13)}}]
	\rbars
	\sbar{yellow!50!orange}{date}{A067RC}{data/pcedecomp.csv}
	\sbar{magenta!90!white}{date}{SAVING}{data/pcedecomp.csv}
	\sbar{blue!60!white}{date}{OTHER}{data/pcedecomp.csv}
	\stdline{black}{date}{DPCERC}{data/pcedecomp.csv}
	\legend{Income, Saving, Other Outlays, Total};
	\end{axis}
\end{tikzpicture}\\
\footnotesize{Source: Bureau of Economic Analysis}


\end{minipage}

\newpage

\begin{minipage}{0.76\textwidth}

\subsection*{\color{black!70} \seriffont Household Balance Sheets}

\subsubsection*{\color{black!70} \seriffont Liabilities}

\vspace{1mm}

\small \input{text/hhdebt1.txt} \\

\input{text/hhdebt2.txt}\\

\vspace{1mm}

\noindent \normalsize \textbf{Household and Nonprofit Debt}\\
\footnotesize{\textit{by type, as share of disposable personal income}}\\
\noindent \hspace*{-2mm} \begin{tikzpicture}
	\begin{axis}[\bbar{y}{0}, \dateaxisticks ytick={0, 50, 100}, ymin=2,
		xticklabel={`\short{\year}}, clip=false, legend style={at={(0.95, 1.13)}}]
	\rbars
	\sbar{blue!60!violet}{date}{Mortgages}{data/hhdebt.csv}	
	\sbar{magenta}{date}{Consumer Credit}{data/hhdebt.csv}
	\sbar{orange!80!yellow}{date}{Other}{data/hhdebt.csv}	
	\legend{Home Mortgages, Consumer Credit, Other};
	\end{axis}
\end{tikzpicture}\\
\footnotesize{Source: Federal Reserve and Bureau of Economic Analysis}\\

\vspace{5mm}

\small \input{text/hhcdebt3.txt} \\

\vspace{1mm}


\normalsize \textbf{Mortgages and Consumer Credit}

\footnotesize{\textit{share of disposable personal income, percent}} \\

\input{text/hhcdebt2.txt}

\vspace{1mm}

\footnotesize{Source: Federal Reserve Bank of New York and Bureau of Economic Analysis}


\end{minipage}

\newpage


\begin{minipage}{0.76\textwidth}

\small Trends in household debt over the past three years, measured in both the US Financial Accounts and the New York Fed Consumer Credit Panel, show consumer credit growing in line with income while mortgage debt falls relative to income. The two series below, Mortgage Debt Total and Consumer Credit, are comparable between both data sources. Discrepancies arise because the Financial Accounts include debt of nonprofit institutions and the Consumer Credit Panel does not include persons without a social security number.\\

\input{text/hhcdebt4.txt} \\

\input{text/hhcdebt5.txt} \\

\vspace{4mm}

\noindent \normalsize \textbf{Household Debt Outstanding}\\
\footnotesize{\textit{trillions of US Dollars \hspace{46mm} share of disposable personal income}\\ \vspace{4mm}
\noindent \rowcolors{1}{}{black!5} \setlength{\tabcolsep}{3.1pt} \color{black!90}
		{\renewcommand{\arraystretch}{1.55}
		 \begin{tabular}{p{36mm} R{11.8mm} R{11.8mm}  R{8.0mm} R{7.4mm} R{7.4mm} 
		   R{7.4mm} R{7.4mm}}
			 & 2021  Q1 & 2020  Q4 & `21  Q1 & `20  Q4 & `18  Q1 & `13  Q1 & `03  Q1 \\  Financial  Accounts  Total* & \$17.24T & \$17.06T &88.2&98.7&100.9&112.0&108.5\\  \hspace{2mm}  \cbox{blue!60!violet}  Mortgage  Debt  Total & \$11.04T & \$10.92T &56.5&63.2&65.0&76.7&74.8\\  \hspace{2mm}  \cbox{magenta}  Consumer  Credit & \$4.16T & \$4.19T &21.3&24.2&24.9&23.6&24.0\\  \hspace{2mm}  \cbox{orange!80!yellow}  Other & \$2.04T & \$1.94T &10.4&11.2&11.0&11.7&9.7\\  Consumer  Credit  Panel  Total & \$14.64T & \$14.56T &74.9&84.2&86.2&90.8&87.2\\  \hspace{2mm}  Mortgage  Debt  Total & \$10.50T & \$10.39T &53.7&60.1&61.2&68.6&62.5\\  \hspace{4mm}  Mortgage & \$10.16T & \$10.04T &52.0&58.1&58.3&64.1&59.6\\  \hspace{4mm}  Home  Equity  Revolving & \$0.34T & \$0.35T &1.7&2.0&2.8&4.5&2.9\\  \hspace{2mm}  Consumer  Credit & \$4.15T & \$4.17T &21.2&24.1&25.0&22.2&24.7\\  \hspace{4mm}  \cbox{blue!60!cyan}  Auto  Loan & \$1.38T & \$1.37T &7.1&8.0&8.0&6.4&7.7\\  \hspace{4mm}  \cbox{red}  Credit  Card & \$0.77T & \$0.82T &3.9&4.7&5.3&5.3&8.3\\  \hspace{4mm}  \cbox{green!80!blue}  Student  Loan & \$1.58T & \$1.55T &8.1&9.0&9.2&8.0&2.9\\  \hspace{4mm}  Other & \$0.41T & \$0.42T &2.1&2.4&2.5&2.5&5.8\\ 
			 \hline
		\end{tabular}
		}	\\}
		
\vspace{-2mm}
\footnotesize{Source: Federal Reserve, Federal Reserve Bank of New York, Bureau of Economic Analysis}\\

\vspace{6mm}

\end{minipage}

\newpage

\begin{minipage}{0.76\textwidth}


\subsubsection*{\color{black!70} \seriffont Assets}

\small The return on total household assets has fallen, as measured by disposable income as a share of household assets.  \\ 

\vspace{2mm}

\noindent \normalsize \textbf{Return on Household Assets}\\
\footnotesize{\textit{disposable personal income as share of household and nonprofit total assets, percent}}\\
\noindent \hspace*{-2mm} \begin{tikzpicture}
	\begin{axis}[\bbar{y}{0}, \dateaxisticks ytick={12, 14, 16}, enlarge y limits={0.1},
		xticklabel={`\short{\year}}, clip=false, height=4.4cm]
	\rbars
	\stdline{red}{date}{DPIsh}{data/dpish.csv}
	\end{axis}
\end{tikzpicture}\\
\footnotesize{Source: Federal Reserve, Bureau of Economic Analysis} \\

\vspace{6mm}

\small Text about growth in various types of household assets. I divided the Fed Z.1 household and nonprofit balance sheet data into: business equity--directly and indirectly held corporate equities + proprietor equity in noncorporate businesses; nonfinancial assets, which are primarily composed of real estate; deposits--which include cash as well as checking and saving accounts and money market funds; and other financial assets, which are primarily debt securities and loans and claims on insurance companies. \\

\vspace{2mm}

\noindent \normalsize \textbf{Contributions to Real Growth in Assets}\\
\footnotesize{\textit{contribution to one-year percent change in assets, adjusted by PCE price deflator}}\\
\noindent \hspace*{-3mm} \begin{tikzpicture}
	\begin{axis}[\bbar{y}{0}, \dateaxisticks ytick={-10, 0, 10}, height=5.8cm, 
		xticklabel={`\short{\year}}, yticklabel style={text width=1.5em}, clip=false, 
		legend style={at={(0.95, 1.13)}}]
	\rbars
	\sbar{blue!70!violet}{date}{FL152010005.Q}{data/hh_asset_growth.csv}
	\sbar{cyan}{date}{CASHDEP}{data/hh_asset_growth.csv}
	\sbar{orange!70!white}{date}{EQ}{data/hh_asset_growth.csv}
	\sbar{magenta!80!purple}{date}{OTHFA}{data/hh_asset_growth.csv}
	\legend{Nonfinancial, Cash \& Deposits, Business Equity, Other Financial};
	\end{axis}
\end{tikzpicture}\\
\footnotesize{Source: Federal Reserve, Bureau of Economic Analysis}\\

\end{minipage}

\newpage

\begin{minipage}{0.76\textwidth}

\small Rate of growth for both real net worth and real after tax income. \\

\vspace{3mm}

\noindent \normalsize \textbf{Net Worth and After-Tax Income Growth}\\
\footnotesize{\textit{one-year percent change in net worth and after-tax income, adjusted by PCE price deflator}}\\
\noindent \hspace*{-3mm} \begin{tikzpicture}
	\begin{axis}[\bbar{y}{0}, \dateaxisticks ytick={-10, 0, 10}, height=5.8cm, 
		xticklabel={`\short{\year}}, yticklabel style={text width=1.5em}, clip=false, 
		legend style={at={(0.95, 1.13)}}]
	\rbars
	\sbar{cyan!40!white}{date}{FL152090005.Q}{data/rdpi_nw.csv}
	\stdline{blue!50!violet}{date}{RDPI}{data/rdpi_nw.csv}
	\legend{Net Worth, After-Tax Income};
	\end{axis}
\end{tikzpicture}\\
\footnotesize{Source: Federal Reserve, Bureau of Economic Analysis}\\


\end{minipage}

\newpage

\begin{minipage}{0.76\textwidth}

\normalsize

\subsubsection*{\color{black!70} \seriffont Housing}

\vspace{2mm}

\small Some data here on the US total and regional change in the value of residential homes during and since the housing bubble. \\

\vspace{2mm}

\noindent \normalsize \textbf{Owner-Occupied Housing Wealth} \hspace{10mm} \small \cbox{lime!70!green!80!white} Unit Price \hspace{3mm} \cbox{blue!50!violet} Inventory\\
\footnotesize{\textit{contributions to one-year percent change in regional owner-occupied housing wealth}}\\
\noindent \hspace*{-2mm} \begin{tikzpicture}
	\begin{axis}[\bbar{y}{0}, \dateaxisticks ytick={-10, 0, 10, 20}, 
		ymin = -15, ymax=25,
		xmin=2001-07-01, width=6.7cm, yticklabel style={text width=1.5em},
		xticklabel={`\short{\year}}]
	\fill[color=black!10] (axis cs:{2007-12-01},\pgfkeysvalueof{/pgfplots/ymin}) rectangle 
		(axis cs:{2009-07-01}, \pgfkeysvalueof{/pgfplots/ymax});
	\sbar{blue!50!violet}{date}{Midwest_Q}{data/val_ooh.csv}	
	\sbar{lime!70!green!80!white}{date}{Midwest_P}{data/val_ooh.csv}
	\node[right] at (axis cs:2001-05-01,23) {\footnotesize Midwest};
	\end{axis}
\end{tikzpicture} \hfill
\begin{tikzpicture}
	\begin{axis}[\bbar{y}{0}, \dateaxisticks ytick={-10, 0, 10, 20}, 
		ymin = -15, ymax=25,
		xmin=2001-07-01, width=6.7cm, yticklabel style={text width=1.5em},
		xticklabel={`\short{\year}}]
	\fill[color=black!10] (axis cs:{2007-12-01},\pgfkeysvalueof{/pgfplots/ymin}) rectangle 
		(axis cs:{2009-07-01}, \pgfkeysvalueof{/pgfplots/ymax});
	\sbar{blue!50!violet}{date}{Northeast_Q}{data/val_ooh.csv}	
	\sbar{lime!70!green!80!white}{date}{Northeast_P}{data/val_ooh.csv}	
	\node[right] at (axis cs:2001-05-01,23) {\footnotesize Northeast};
	\end{axis}
\end{tikzpicture}\\
\noindent \hspace*{-2mm} \begin{tikzpicture}
	\begin{axis}[\bbar{y}{0}, \dateaxisticks ytick={-10, 0, 10, 20}, 
		ymin = -15, ymax=25,
		xmin=2001-07-01, width=6.7cm, yticklabel style={text width=1.5em},
		xticklabel={`\short{\year}}]
	\fill[color=black!10] (axis cs:{2007-12-01},\pgfkeysvalueof{/pgfplots/ymin}) rectangle 
		(axis cs:{2009-07-01}, \pgfkeysvalueof{/pgfplots/ymax});
	\sbar{blue!50!violet}{date}{West_Q}{data/val_ooh.csv}	
	\sbar{lime!70!green!80!white}{date}{West_P}{data/val_ooh.csv}
	\node[right] at (axis cs:2001-05-01,23) {\footnotesize West};
	\end{axis}
\end{tikzpicture} \hfill
\begin{tikzpicture}
	\begin{axis}[\bbar{y}{0}, \dateaxisticks ytick={-10, 0, 10, 20}, 
		ymin = -15, ymax=25,
		xmin=2001-07-01, width=6.7cm, yticklabel style={text width=1.5em},
		xticklabel={`\short{\year}}]
	\fill[color=black!10] (axis cs:{2007-12-01},\pgfkeysvalueof{/pgfplots/ymin}) rectangle 
		(axis cs:{2009-07-01}, \pgfkeysvalueof{/pgfplots/ymax});
	\sbar{blue!50!violet}{date}{South_Q}{data/val_ooh.csv}	
	\sbar{lime!70!green!80!white}{date}{South_P}{data/val_ooh.csv}	
	\node[right] at (axis cs:2001-05-01,23) {\footnotesize South};
	\end{axis}
\end{tikzpicture}\\
\footnotesize{Source: Federal Reserve and Census Bureau}\\

\vspace{8mm}

\small \input{text/permits.txt}\\

\vspace{2mm}


\noindent \normalsize \textbf{Residential Construction}\\
\footnotesize{\textit{building permits issued, in thousands}}\\
\noindent \hspace*{-2mm} \begin{tikzpicture}
	\begin{axis}[\bbar{y}{0}, \dateaxisticks ytick={500, 1000, 1500, 2000, 2500}, 
		xticklabel={`\short{\year}}, clip=false]
	\rbars
	\stdline{blue}{date}{APERMITS}{data/permits.csv}	
	\end{axis}
\end{tikzpicture}\\
\footnotesize{Source: Census Bureau} \\

\end{minipage}

\newpage

\begin{minipage}{0.76\textwidth}

\normalsize

Housing permits/starts \\

Geographic location of housing permits \\

Households; owners' equity in real estate as a percentage of household real estate, Level (HOEREPHRE)\\

\vspace{4mm}

\small The Federal Housing Finance Agency (FHFA) housing price index \href{https://www.fhfa.gov/DataTools/Downloads/Pages/House-Price-Index-Datasets.aspx}{data} look useful primarily because they offer geographic specificity. Look into ways to use these. Ideally, I want to know about the ratio of housing prices to rental equivalent. For now, the chart below is more or less a placeholder, though I may keep it or some variation.\\


\vspace{2mm}

\noindent \normalsize \textbf{House Price Index}\\
\footnotesize{\textit{one-year percent change}}\\
\noindent \hspace*{-2mm} \begin{tikzpicture}
	\begin{axis}[\bbar{y}{0}, \dateaxisticks ytick={-10, -5, 0, 5, 10}, 
		yticklabel style={text width=1.5em},
		xticklabel={`\short{\year}}, width=11.8cm]
	\rbars
	\thickline{blue!70!cyan}{date}{index_sa}{data/hpi.csv}	
	\end{axis}
\end{tikzpicture}\\
\footnotesize{Source: Federal Housing Finance Agency} \\


\vspace{8mm}

\subsubsection*{\color{black!70} \seriffont Saving}

\small The portion of after-tax income that is not spent by households is considered personal saving, from an economic accounting perspective. Personal saving as a share of disposable personal income is referred to as the personal saving rate. Households use savings to handle unexpected expenses or cover expenses when income falls. However, economists also point out that aggregate personal saving is a direct reduction in corporate profits, as it represents income to persons that was at some point a business expense, but that does not get returned to businesses as revenue through consumer spending. \\

\input{text/psavert.txt}\\

\vspace{2mm}

\noindent \normalsize \textbf{Personal Saving Rate}\\
\footnotesize{\textit{personal saving as a share of disposable personal income}}\\
\noindent \hspace*{-2mm} \begin{tikzpicture}
	\begin{axis}[\bbar{y}{0}, \dateaxisticks ytick={4, 8, 12}, enlarge y limits={0.1},
		xticklabel={`\short{\year}}, height=4.4cm, width=12.6cm, clip=false]
	\rbars
	\stdline{red}{date}{VALUE}{data/psavert.csv}
	\input{text/psavert_node.txt}
	\end{axis}
\end{tikzpicture}\\
\footnotesize{Source: Bureau of Economic Analysis} \\


\end{minipage}

\newpage

\subsection*{\color{black!70} \seriffont Poverty}

\begin{minipage}{0.76\textwidth}
\small
Include data on number of people in poverty and the official poverty rate. Perhaps include a chart showing the official poverty rate over time. Perhaps also try to capture some concepts around methodology (SPM for example) and about relative poverty.\\

\end{minipage}

\noindent \normalsize \textbf{Share of local population in bottom third of housing-adjusted income, 2018}\\
\footnotesize{\textit{Share of commuting zone householders with after-housing-expense annual income below \$13,573}}

\vspace{-3mm}
\hspace{-15mm} \input{/home/brian/Documents/ACS/acs_map.pgf}

\vspace{-5mm}
\footnotesize{Source: American Community Survey}\\

\vspace{6mm}

\begin{minipage}{0.76\textwidth}

\normalsize

Income and Expenses by Age and Number of Children\\

Poverty rates and amount of poverty in millions of people\\




\end{minipage}

\newpage

\begin{minipage}{0.76\textwidth}

\section*{\color{darkgray}\LARGE \seriffont Businesses}
\small The factories, offices, and equipment that workers use to produce goods and services are all important to the economy. This section looks at the loosely defined business sector, with data covering business investment, retail sales, industrial production, corporate profits, and the financial activities of businesses.

\subsection*{\color{black!70} \seriffont Fixed Investment}
\small When businesses purchase items with a useful life of more than one year it is considered and investment in fixed assets, which is an exchange of assets rather than an expense. Investments in fixed assets that make workers more productive, by definition, allow businesses to produce goods and services using less effort from people. Business gross investments in fixed assets are grouped broadly as structures (see\cbox{yellow!50!orange}), equipment (see\cbox{cyan!60!white}), and intellectual property products (see\cbox{violet}). 
\vspace{5mm}

\noindent \normalsize \textbf{Business Fixed Investment}\\
\footnotesize{\textit{percentage point contribution to GDP growth}}\\
\noindent \hspace*{-2mm} \begin{tikzpicture}
	\begin{axis}[\bbar{y}{0}, \dateaxisticks ytick={-2, 0, 2},
		xticklabel={`\short{\year}}, clip=false, 
		legend style={at={(0.95, 1.13)}}]
	\rbars
	\sbar{yellow!50!orange}{date}{A009RY}{data/businv.csv}
	\sbar{cyan!60!white}{date}{Y033RY}{data/businv.csv}
	\sbar{violet}{date}{Y001RY}{data/businv.csv}
	\legend{Structures, Equipment, Intellectual Property Products};
	\end{axis}
\end{tikzpicture}\\
\footnotesize{Source: Bureau of Economic Analysis}

\vspace{5mm}

\small

&   2023  Q3 & `23  Q2 & `23  Q1 & `22  Q3 & `21  Q3 &1-year&10-year&30-year\\ Total&0.21&0.98&0.76&0.62&-0.15&0.55&0.54&0.60\\  \hspace{-2mm}\cbox{yellow!50!orange}Structures &0.33&0.46&0.77&-0.03&-0.12&0.43&0.05&0.03\\  \hspace{-2mm}\cbox{cyan!60!white}Equipment &-0.22&0.38&-0.21&0.28&-0.40&-0.08&0.16&0.31\\  \hspace{4mm}  Information  Processing &-0.12&-0.11&-0.02&0.09&-0.09&-0.16&0.10&0.20\\  \hspace{6mm}  Computers  \&  Peripherals &-0.11&0.02&-0.05&0.09&-0.03&-0.09&0.02&0.10\\  \hspace{4mm}  Industrial  Equipment &-0.06&-0.06&0.04&-0.07&0.08&-0.01&0.02&0.03\\  \hspace{4mm}  Transportation  Equipment &-0.02&0.54&-0.14&0.30&-0.41&0.13&0.02&0.05\\  \hspace{-2mm}\cbox{violet}Intellectual  Property  Products &0.10&0.15&0.20&0.37&0.37&0.19&0.34&0.25\\  \hspace{4mm}  Software &0.17&0.13&0.16&0.28&0.18&0.19&0.21&0.15\\  \hspace{4mm}  Research  \&  Development &-0.06&0.00&0.04&0.05&0.14&0.00&0.12&0.09\\  \\

\end{minipage}

\noindent \normalsize \textbf{Business Investment}\\
\footnotesize{\textit{percentage point contribution to real GDP growth \hspace{36mm} moving averages}\\ \vspace{4mm}
\noindent \rowcolors{1}{}{black!5} \setlength{\tabcolsep}{3.1pt} \color{black!90}
		{\renewcommand{\arraystretch}{1.55}
		 \begin{tabular}{p{42mm} R{6.7mm} R{6.7mm} R{6.7mm} R{6.7mm} R{6.7mm} 
		   R{7.8mm} R{6.7mm} R{6.7mm} }
			 &   2023  Q3 & `23  Q2 & `23  Q1 & `22  Q3 & `21  Q3 &1-year&10-year&30-year\\ Total&0.21&0.98&0.76&0.62&-0.15&0.55&0.54&0.60\\  \hspace{-2mm}\cbox{yellow!50!orange}Structures &0.33&0.46&0.77&-0.03&-0.12&0.43&0.05&0.03\\  \hspace{-2mm}\cbox{cyan!60!white}Equipment &-0.22&0.38&-0.21&0.28&-0.40&-0.08&0.16&0.31\\  \hspace{4mm}  Information  Processing &-0.12&-0.11&-0.02&0.09&-0.09&-0.16&0.10&0.20\\  \hspace{6mm}  Computers  \&  Peripherals &-0.11&0.02&-0.05&0.09&-0.03&-0.09&0.02&0.10\\  \hspace{4mm}  Industrial  Equipment &-0.06&-0.06&0.04&-0.07&0.08&-0.01&0.02&0.03\\  \hspace{4mm}  Transportation  Equipment &-0.02&0.54&-0.14&0.30&-0.41&0.13&0.02&0.05\\  \hspace{-2mm}\cbox{violet}Intellectual  Property  Products &0.10&0.15&0.20&0.37&0.37&0.19&0.34&0.25\\  \hspace{4mm}  Software &0.17&0.13&0.16&0.28&0.18&0.19&0.21&0.15\\  \hspace{4mm}  Research  \&  Development &-0.06&0.00&0.04&0.05&0.14&0.00&0.12&0.09\\ \hline
		\end{tabular}
		}	\\
		
\vspace{-6mm}
\footnotesize{Source: Bureau of Economic Analysis}

\newpage

\begin{minipage}{0.35\textwidth}

\small The productive investments of businesses are also measured by the new orders for core capital goods. The category excludes the more volatile aircraft orders as well as defense-related orders, and is derived from a Census Bureau \href{https://www.census.gov/manufacturing/m3/index.html}{survey} of shipments, inventories, and orders.  \\

\begin{tikzpicture}\begin{axis}[\bbar{y}{0}, \dateaxisticks ytick={4, 6, 8}, width=6.4cm, height=5.4cm,ymin=3.0, clip=false,xtick={{1992-01-01}, {1995-01-01}, {2000-01-01}, {2005-01-01}, {2010-01-01}, {2015-01-01}, {2021-02-01}}, xticklabels={`92, `95, `00, `05, `10, `15, Feb}, minor xtick={}]\rebars\thickline{purple!50!violet}{date}{value}{data/dgno.csv}\input{text/dgno_node.txt}\end{axis}\end{tikzpicture}
\end{minipage} \hspace{6mm}
\begin{minipage}{0.44\textwidth}

\noindent \normalsize \textbf{New Orders for Core Capital Goods}\\
\footnotesize{\textit{nondefense capital goods ex-aircraft, share of GDP}}\\
\noindent \hspace*{-2mm} \begin{tikzpicture}
	\begin{axis}[\bbar{y}{0}, \dateaxisticks ytick={4, 6, 8}, width=7.0cm, height=5.0cm,
		ymin=2.9, xtick={{1992-01-01}, {1995-01-01}, {2000-01-01}, {2005-01-01}, 
				{2010-01-01}, {2015-01-01}, {2019-10-01}},
		minor xtick={}, 
        xticklabels={`92, `95, `00, `05, `10, `15, Q3},]
	\rbars
	\thickline{purple!50!violet}{date}{value}{data/dgno.csv}
	\end{axis}
\end{tikzpicture}\\
\footnotesize{Source: Census Bureau} \\


\end{minipage}

\newpage 

\begin{minipage}{0.76\textwidth}



\subsection*{\color{black!70} \seriffont Corporate Profits}

\small The national accounts include detailed information on corporate profits, which are an important determinant in the business cycle.  \\

\vspace{2mm}

\noindent \normalsize \textbf{Destination of Corporate Profits}\\
\footnotesize{\textit{share of net national income}}\\
\noindent \hspace*{-2mm} \begin{tikzpicture}
	\begin{axis}[\bbar{y}{0}, \dateaxisticks ytick={0, 5, 10, 15}, ymin=0.2,
		xticklabel={`\short{\year}}, clip=false, legend style={at={(0.95, 1.13)}}]
	\rbars
	\sbar{blue!70!purple}{date}{DIV}{data/cprof.csv}
	\sbar{cyan!50!white}{date}{RE}{data/cprof.csv}
	\sbar{red!80!orange}{date}{TAX}{data/cprof.csv}	
	\legend{Net Dividends, Retained Earnings, Corporate Taxes};
	\end{axis}
\end{tikzpicture}\\
\footnotesize{Source: Bureau of Economic Analysis}\\

\vspace{4mm}

\small Aggregate corporate savings (corporate profits less dividends and corporate profit tax) are the result of net investment and nonbusiness saving. Investment is a source of aggregate profit because it is revenue for one party but not an expense for the other. Nonbusiness saving, which includes household, government, and rest of world saving, necessarily reduces aggregate corporate profits because it is money that did not return to businesses as revenue. \\

\vspace{2mm}

\noindent \normalsize \textbf{Sources of Corporate Saving}\\
\footnotesize{\textit{contribution to corporate saving, as share of gross national income}}\\
\noindent \hspace*{-2mm} \begin{tikzpicture}
	\begin{axis}[\bbar{y}{0}, \dateaxisticks ytick={-5, 0, 5, 10, 15}, 
		xticklabel={`\short{\year}}, clip=false, legend style={at={(0.95, 1.13)}}]
	\rbars
	\sbar{blue!70!green}{date}{HH Saving}{data/cprof2.csv}
	\sbar{yellow}{date}{Gov Saving}{data/cprof2.csv}
	\sbar{orange}{date}{ROW Saving}{data/cprof2.csv}
	\sbar{green!90!blue}{date}{Investment}{data/cprof2.csv}	
	\legend{Household Saving, Government Saving, ROW Saving, Investment};
	\end{axis}
\end{tikzpicture}\\
\footnotesize{Source: Bureau of Economic Analysis}\\

\end{minipage}

\newpage

\begin{minipage}{0.76\textwidth}

\subsection*{\color{black!70} \seriffont Business Debt}

\small \input{text/busdebtgdp.txt} \\

\vspace{2mm}

\noindent \normalsize \textbf{Nonfinancial Business Debt}\\
\footnotesize{\textit{by type, as share of GDP}}\\
\noindent \hspace*{-2mm} \begin{tikzpicture}
	\begin{axis}[\bbar{y}{0}, \dateaxisticks ytick={0, 25, 50, 75}, ymin=2,
		xticklabel={`\short{\year}}, clip=false, legend style={at={(0.95, 1.13)}}]
	\rbars
	\sbar{violet}{date}{Bank Loans and Mortgages}{data/busdebtgdp2.csv}	
	\sbar{magenta!70!purple}{date}{Debt Securities}{data/busdebtgdp2.csv}	
	\sbar{blue}{date}{Nonbank Loans}{data/busdebtgdp2.csv}	
	\legend{Mortgages and Bank Loans, Corporate Debt Securities, Nonbank Loans};
	\end{axis}
\end{tikzpicture}\\
\footnotesize{Source: Federal Reserve and Bureau of Economic Analysis}\\

\vspace{4mm}

\small The debt of the domestic financial sector includes agency and government-sponsored enterprise (GSE) backed securities, corporate and foreign bonds, loans, and open market paper. The long-term increase in financial sector debt reflects the emergence and growth of various asset-backed securities. In addition to home mortgage-backed securities, the domestic financial sector issues debt securities based on commercial mortgages, auto loans, credit card, student debt, and even restaurant revenue. \\

\input{text/findebtgdp.txt} \\

\vspace{2mm}

\noindent \normalsize \textbf{Financial Sector Debt}\\
\footnotesize{\textit{by type, as share of GDP}}\\
\noindent \hspace*{-2mm} \begin{tikzpicture}
	\begin{axis}[\bbar{y}{0}, \dateaxisticks ytick={0, 50,  100, 150}, ymin=2,
		xticklabel={`\short{\year}}, clip=false, legend style={at={(0.95, 1.13)}}]
	\rbars
	\sbar{orange!70!yellow}{date}{Financial Loans}{data/busdebtgdp2.csv}	
	\sbar{red!60!purple}{date}{Agency MBS}{data/busdebtgdp2.csv}
	\sbar{blue!40!violet}{date}{Other}{data/busdebtgdp2.csv}
	\legend{Loans, Agency- and GSE-backed Securities, Other Debt Securities};
	\end{axis}
\end{tikzpicture}\\
\footnotesize{Source: Federal Reserve and Bureau of Economic Analysis}\\

\end{minipage}

\newpage

\subsection*{\color{black!70} \seriffont Industrial Production}

\begin{minipage}{0.62\textwidth}

\small  & & Dec  `20 & Nov  `20 & Oct  `20 & Dec  `19 &   Dec  `20 &   Nov  `20 &   Oct  `20 &   Dec  `19 \\  &  \hspace{-1mm}Total  index &-3.6&-5.4&-5.0&-0.8&-3.6&-5.4&-5.0&-0.8\\  &  \hspace{1mm}Manufacturing &-2.1&-2.6&-2.6&-0.9&-2.8&-3.5&-3.4&-1.2\\  \cbox{blue!60!black}  &  \hspace{3mm}Durable  manufacturing &-1.1&-1.6&-1.4&-0.5&-3.0&-4.2&-3.6&-1.4\\    &  \hspace{5mm}Motor  vehicles  \&  parts &0.2&0.0&0.4&-0.5&3.6&0.1&7.0&-9.1\\  \cbox{blue!20!cyan!80!white}  &  \hspace{3mm}Nondurable  manufacturing &-0.8&-0.9&-1.0&-0.3&-2.2&-2.4&-2.6&-0.7\\  \cbox{orange!20!yellow}  &  \hspace{1mm}Mining &-1.6&-1.7&-2.0&0.1&-12.3&-12.9&-15.8&1.0\\  \cbox{green!80!blue}  &  \hspace{1mm}Utilities &0.3&-0.9&-0.2&-0.0&2.7&-8.8&-1.6&-0.1\\  \cbox{violet!60!black}  &  \hspace{1mm}Consumer  goods &0.1&-0.7&-0.1&-0.3&0.4&-2.3&-0.2&-1.3\\    &  \hspace{3mm}Consumer  durables &0.2&0.0&0.3&-0.3&3.5&0.6&4.7&-4.5\\    &  \hspace{5mm}Automotive  products &0.2&0.1&0.3&-0.2&6.2&1.9&9.5&-6.9\\    &  \hspace{3mm}Consumer  nondurables &-0.1&-0.7&-0.3&-0.1&-0.4&-3.1&-1.5&-0.3\\    &  \hspace{5mm}Foods  and  tobacco &0.1&0.3&0.1&0.2&0.6&3.0&1.0&2.6\\    &  \hspace{5mm}Chemical  products &-0.0&-0.2&-0.1&-0.1&-0.0&-3.0&-2.0&-2.1\\    &  \hspace{5mm}Consumer  energy  products &-0.1&-0.6&-0.2&-0.1&-1.6&-12.9&-4.1&-1.7\\  \cbox{magenta}  &  \hspace{1mm}Equipment  \&  nonindustrial  supplies &-1.4&-1.7&-1.7&-0.2&-5.4&-6.7&-6.6&-0.8\\    &  \hspace{3mm}Equipment &-0.9&-1.0&-1.0&-0.2&-7.3&-8.3&-8.7&-1.2\\    &  \hspace{5mm}Industrial  equipment &-0.1&-0.2&-0.2&-0.1&-5.2&-7.5&-7.1&-3.3\\    &  \hspace{3mm}Nonindustrial  supplies &-0.5&-0.7&-0.7&-0.1&-3.7&-5.2&-4.8&-0.4\\    &  \hspace{5mm}Construction  supplies &-0.1&-0.1&-0.2&0.0&-1.6&-2.4&-3.1&0.0\\    &  \hspace{5mm}Business  supplies &-0.4&-0.6&-0.5&-0.1&-5.1&-7.0&-6.0&-0.7\\  \cbox{orange!70!yellow}  &  \hspace{1mm}Materials &-2.2&-3.0&-3.1&-0.3&-4.9&-6.6&-6.8&-0.6\\    &  \hspace{3mm}Consumer  parts &0.1&0.0&0.1&-0.2&2.0&0.1&4.1&-8.6\\    &  \hspace{3mm}Equipment  parts &-0.2&-0.2&-0.2&0.0&-3.9&-4.8&-4.6&0.8\\    &  \hspace{3mm}Chemical  materials &-0.1&-0.0&-0.2&-0.1&-1.5&-0.6&-2.4&-2.1\\    &  \hspace{3mm}Energy  materials &-1.3&-1.9&-2.0&0.3&-7.5&-11.3&-11.6&1.8\\  \\

\input{text/indpro2.txt}

\end{minipage}\hspace{7mm}
\begin{minipage}{0.35\textwidth}

\noindent \normalsize \textbf{Industrial Production}\\
\footnotesize{\textit{index, 2012=100}}\\
\noindent \hspace*{-2mm} \begin{tikzpicture}
	\begin{axis}[\bbar{y}{0}, \dateaxisticks ytick={60, 80, 100}, 
		enlarge y limits={0.05}, legend cell align={left},
		xtick={{1989-01-01}, {1995-01-01}, {2000-01-01}, {2005-01-01}, 
				{2010-01-01}, {2015-01-01}, {2019-10-01}},
		minor xtick={}, 
        xticklabels={`89, `95, `00, `05, `10, `15, Oct},
		clip=false, height=4.7cm, width=6.4cm,
		legend style={fill=white, legend columns=1, at={(1.02, 0.33)}}]
	\rbars
	\thickline{red}{date}{Manufacturing}{data/indpro.csv}
	\stdline{blue!90!black}{date}{Total index}{data/indpro.csv}
	\legend{Manufacturing, Total Index};
	\end{axis}
\end{tikzpicture}\\
\footnotesize{Source: Federal Reserve} \\

\end{minipage}\\

\begin{minipage}{0.76\textwidth}

\vspace{4mm} 

\noindent \normalsize \textbf{Industrial Production Growth}\\
\footnotesize{\textit{percentage point contribution to one-year growth of total index \hspace{12mm} moving averages}\\ 
\noindent \rowcolors{1}{}{black!5} \setlength{\tabcolsep}{3.1pt} \color{black!90}
		{\renewcommand{\arraystretch}{1.55}
		 \begin{tabular}{p{2mm} p{47mm} R{7.3mm} R{7.3mm} R{7.3mm} R{7.6mm} R{7.3mm} 
		   R{7.3mm} R{7.3mm} }
			  & & Dec  `20 & Nov  `20 & Oct  `20 & Dec  `19 &   Dec  `20 &   Nov  `20 &   Oct  `20 &   Dec  `19 \\  &  \hspace{-1mm}Total  index &-3.6&-5.4&-5.0&-0.8&-3.6&-5.4&-5.0&-0.8\\  &  \hspace{1mm}Manufacturing &-2.1&-2.6&-2.6&-0.9&-2.8&-3.5&-3.4&-1.2\\  \cbox{blue!60!black}  &  \hspace{3mm}Durable  manufacturing &-1.1&-1.6&-1.4&-0.5&-3.0&-4.2&-3.6&-1.4\\    &  \hspace{5mm}Motor  vehicles  \&  parts &0.2&0.0&0.4&-0.5&3.6&0.1&7.0&-9.1\\  \cbox{blue!20!cyan!80!white}  &  \hspace{3mm}Nondurable  manufacturing &-0.8&-0.9&-1.0&-0.3&-2.2&-2.4&-2.6&-0.7\\  \cbox{orange!20!yellow}  &  \hspace{1mm}Mining &-1.6&-1.7&-2.0&0.1&-12.3&-12.9&-15.8&1.0\\  \cbox{green!80!blue}  &  \hspace{1mm}Utilities &0.3&-0.9&-0.2&-0.0&2.7&-8.8&-1.6&-0.1\\  \cbox{violet!60!black}  &  \hspace{1mm}Consumer  goods &0.1&-0.7&-0.1&-0.3&0.4&-2.3&-0.2&-1.3\\    &  \hspace{3mm}Consumer  durables &0.2&0.0&0.3&-0.3&3.5&0.6&4.7&-4.5\\    &  \hspace{5mm}Automotive  products &0.2&0.1&0.3&-0.2&6.2&1.9&9.5&-6.9\\    &  \hspace{3mm}Consumer  nondurables &-0.1&-0.7&-0.3&-0.1&-0.4&-3.1&-1.5&-0.3\\    &  \hspace{5mm}Foods  and  tobacco &0.1&0.3&0.1&0.2&0.6&3.0&1.0&2.6\\    &  \hspace{5mm}Chemical  products &-0.0&-0.2&-0.1&-0.1&-0.0&-3.0&-2.0&-2.1\\    &  \hspace{5mm}Consumer  energy  products &-0.1&-0.6&-0.2&-0.1&-1.6&-12.9&-4.1&-1.7\\  \cbox{magenta}  &  \hspace{1mm}Equipment  \&  nonindustrial  supplies &-1.4&-1.7&-1.7&-0.2&-5.4&-6.7&-6.6&-0.8\\    &  \hspace{3mm}Equipment &-0.9&-1.0&-1.0&-0.2&-7.3&-8.3&-8.7&-1.2\\    &  \hspace{5mm}Industrial  equipment &-0.1&-0.2&-0.2&-0.1&-5.2&-7.5&-7.1&-3.3\\    &  \hspace{3mm}Nonindustrial  supplies &-0.5&-0.7&-0.7&-0.1&-3.7&-5.2&-4.8&-0.4\\    &  \hspace{5mm}Construction  supplies &-0.1&-0.1&-0.2&0.0&-1.6&-2.4&-3.1&0.0\\    &  \hspace{5mm}Business  supplies &-0.4&-0.6&-0.5&-0.1&-5.1&-7.0&-6.0&-0.7\\  \cbox{orange!70!yellow}  &  \hspace{1mm}Materials &-2.2&-3.0&-3.1&-0.3&-4.9&-6.6&-6.8&-0.6\\    &  \hspace{3mm}Consumer  parts &0.1&0.0&0.1&-0.2&2.0&0.1&4.1&-8.6\\    &  \hspace{3mm}Equipment  parts &-0.2&-0.2&-0.2&0.0&-3.9&-4.8&-4.6&0.8\\    &  \hspace{3mm}Chemical  materials &-0.1&-0.0&-0.2&-0.1&-1.5&-0.6&-2.4&-2.1\\    &  \hspace{3mm}Energy  materials &-1.3&-1.9&-2.0&0.3&-7.5&-11.3&-11.6&1.8\\ \hline
		\end{tabular}}	\\
		
\footnotesize{Source: Federal Reserve}}

\end{minipage}

\newpage

\begin{minipage}{0.76\textwidth}

\vspace{3mm}

\small Market group data show the lack of growth in the production of consumer goods, equipment, and nonindustrial supplies over the past decade.\\

\vspace{1mm}

\noindent \normalsize \textbf{Industrial Production Growth, Market Group}\\
\footnotesize{\textit{percentage point contribution to one-year growth}}\\
\noindent \hspace*{-2mm} \begin{tikzpicture}
	\begin{axis}[\bbar{y}{0}, \dateaxisticks ytick={-15, -10, -5, 0, 5, 10},
		xticklabel={`\short{\year}}, clip=false, height=5.6cm,
		yticklabel style={text width=1.5em}, legend cell align={left},
		legend style={legend columns=1, at={(0.53, 0.4)}}]
	\rbars
	\sbar{violet!60!black}{date}{Consumer goods}{data/indprogr.csv}
	\sbar{magenta}{date}{ENS}{data/indprogr.csv}
	\sbar{orange!70!yellow}{date}{Materials}{data/indprogr.csv}
	\legend{Consumer Goods, Equipment \& Nonindustrial Supplies, Materials};
	\end{axis}
\end{tikzpicture}\\
\footnotesize{Source: Federal Reserve}\\

\vspace{3mm}

\small Industry group data show a change in the composition of new industrial activity, towards mining and away from manufacturing.\\

\vspace{1mm}

\noindent \normalsize \textbf{Industrial Production Growth, Industry Group}\\
\footnotesize{\textit{percentage point contribution to one-year growth}}\\
\noindent \hspace*{-2mm} \begin{tikzpicture}
	\begin{axis}[\bbar{y}{0}, \dateaxisticks ytick={-15, -10, -5, 0, 5, 10},
		xticklabel={`\short{\year}}, clip=false, height=5.6cm,
		yticklabel style={text width=1.5em}, legend cell align={left},
		legend style={legend columns=1, at={(0.43, 0.48)}}]
	\rbars
	\sbar{blue!60!black}{date}{Durable manufacturing}{data/indprogr2.csv}
	\sbar{blue!20!cyan!80!white}{date}{Nondurable manufacturing}{data/indprogr2.csv}
	\sbar{orange!20!yellow}{date}{Mining}{data/indprogr2.csv}
	\sbar{green!80!blue}{date}{Electric and gas utilities}{data/indprogr2.csv}
	\legend{Durable Manufacturing, Nondurable Manufacturing, Mining, Utilities};
	\end{axis}
\end{tikzpicture}\\
\footnotesize{Source: Federal Reserve}\\

\vspace{3mm}

\small The most recent slowdown has been broad-based. The monthly data are shown in detail below.\\

\vspace{1mm}

\noindent \normalsize \textbf{Recent data in detail}\\

\vspace{-1mm}

\noindent \normalsize \hspace{14mm} Market Group \hspace{38mm} Industry Group \footnotesize 

\noindent \hspace*{-2mm} \begin{tikzpicture}
	\begin{axis}[\bbar{y}{0}, \dateaxisticks ytick={-4, -2, 0, 2, 4, 6}, 
		clip=false, width=6.5cm, height=4.6cm,
		xtick={{2015-01-01}, {2016-01-01}, {2017-01-01}, 
				{2018-01-01}, {2019-01-01}, {2019-10-01}},
        minor xtick={}, enlarge y limits=0.06, 
        enlarge x limits={0.04}, 
        xticklabels={`15, `16, `17, `18, `19, Oct}]
	\sbar{violet!60!black}{date}{Consumer goods}{data/indprogr_rec.csv}
	\sbar{magenta}{date}{ENS}{data/indprogr_rec.csv}
	\sbar{orange!70!yellow}{date}{Materials}{data/indprogr_rec.csv}
	\end{axis}
\end{tikzpicture}
\hfill
\begin{tikzpicture}
	\begin{axis}[\bbar{y}{0}, \dateaxisticks ytick={-4, -2, 0, 2, 4, 6}, 
		clip=false, width=6.5cm, height=4.6cm,
		xtick={{2015-01-01}, {2016-01-01}, {2017-01-01}, 
				{2018-01-01}, {2019-01-01}, {2019-10-01}},
        minor xtick={}, enlarge y limits=0.06, 
        enlarge x limits={0.04},
        xticklabels={`15, `16, `17, `18, `19, Oct}]
	\sbar{blue!60!black}{date}{Durable manufacturing}{data/indprogr_rec2.csv}
	\sbar{blue!20!cyan!80!white}{date}{Nondurable manufacturing}{data/indprogr2_rec.csv}
	\sbar{orange!20!yellow}{date}{Mining}{data/indprogr_rec2.csv}
	\sbar{green!80!blue}{date}{Electric and gas utilities}{data/indprogr2_rec.csv}
	\end{axis}
\end{tikzpicture}
\\
\footnotesize{Source: Federal Reserve} \\

\end{minipage}

\newpage

\begin{minipage}{0.31\textwidth}
\small \input{text/ip_comp1.txt}\\

\input{text/ip_comp2.txt}

\end{minipage} \hspace{5mm}
\begin{minipage}{0.37\textwidth}
\noindent \normalsize \textbf{Industrial Production and Capacity}\\
\footnotesize{\textit{total three-year growth, percent}}\\ 
\noindent \hspace*{-5mm} \begin{tikzpicture}
  	\begin{axis}[\barplotnogrid axis y line=left, \barylab{3.3cm}{1.5ex}
    	width=5.5cm, bar width=1.7ex, height=10.0cm,
    	enlarge y limits={abs=0.32cm}, enlarge x limits=0.15, \bbar{x}{0},
		yticklabels from table={\ip}{name}, 
		yticklabel style={font=\scriptsize},
		nodes near coords style={/pgf/number format/.cd, fixed zerofill,
			precision=1, assume math mode},
		legend style={text=black!70, at={(0,1.06)}, anchor=north, legend columns=-1, 
				fill=none, draw=none,
		        /tikz/every even column/.append style={column sep=0.4cm}}]
  	\addplot[fill=cyan!90!blue, draw=none] 
  		table [y expr=-\coordindex, x index=2] {\ip};
  	\addplot[fill=green!60!lime, draw=none] 
  		table [y expr=-\coordindex, x index=1] {\ip};
 	\legend{Capacity, Production}
  	\end{axis}
\end{tikzpicture}\\
\footnotesize{Source: Federal Reserve} \\
\end{minipage}

\vspace{5mm}

\begin{minipage}{0.76\textwidth}

\small The Federal Reserve's monthly industrial production \href{https://www.federalreserve.gov/releases/g17/}{report} also measures the economy's total industrial capacity. The extent to which the economy is using its industrial capacity is called capacity utilization, and calculated as industrial production as a share of total industrial capacity. \input{text/tcu.txt}\\

\noindent \normalsize \textbf{Capacity Utilization}\\
\footnotesize{\textit{industrial production as a share of total capacity, percent}}\\ 
\noindent \hspace*{-2mm} \begin{tikzpicture}
	\begin{axis}[\bbar{y}{0}, \dateaxisticks ytick={70, 80}, 
		xticklabel={`\short{\year}}, yticklabel style={text width=1.5em}]
	\rbars
	\thickline{blue!80!black}{date}{Total index}{data/tcu.csv}
	\stdline{blue!40!cyan}{date}{Manufacturing}{data/tcu.csv}
	\stdnode{9.0cm}{1.3cm}{\scriptsize \color{blue!40!cyan}\textbf{Manufacturing}}
	\stdnode{8.4cm}{2.2cm}{\scriptsize \color{blue!80!black}\textbf{Total index}}
	\end{axis}
\end{tikzpicture}\\
\footnotesize{Source: Census Bureau} 

\end{minipage}

\newpage

\begin{minipage}{0.76\textwidth}

\vspace{5mm}

\normalsize

\subsection*{\color{darkgray} \seriffont Retail Sales}

\small \input{text/marts.txt} \\

\vspace{2mm}

\noindent \normalsize \textbf{Retail Sales and Food Services}\\
\footnotesize{\textit{annual growth, percent; nonstore is 3-month moving average}}\\ 
\noindent \hspace*{-2mm} \begin{tikzpicture}
	\begin{axis}[\bbar{y}{0}, \dateaxisticks ytick={-10, 0, 10, 20}, 
		xticklabel={`\short{\year}}, yticklabel style={text width=1.5em},
		width=11.0cm, height=5.0cm]
	\rbars
	\stdline{blue!70!black}{date}{NS_3M}{data/marts.csv}
	\stdline{green!90!blue}{date}{44X72}{data/marts.csv}
	\stdnode{6.1cm}{0.45cm}{\scriptsize \color{green!90!blue}\textbf{Total}}
	\stdnode{7.8cm}{2.65cm}{\scriptsize \color{blue!70!black}\textbf{Nonstore}}
	\end{axis}
\end{tikzpicture}\\
\footnotesize{Source: Census Bureau} \\

\vspace{6mm}

\normalsize


Free cash flow \\

Balance sheets \\

Inventories \\

[Box on tech industry]\\

\end{minipage}

\newpage

\begin{minipage}{0.76\textwidth}

\section*{\color{darkgray}\LARGE \seriffont Government}

\normalsize

\small Public institutions are collectively referred to as the public-sector or the government. In the United States, the government has the authority to spend, tax, and create money, as well as to regulate private sector activities. The government also enforces policies that determine the ownership of property. These activities are all extremely important in determining the production and distribution in the economy.\\



\subsection*{\color{black!70} \seriffont Government Consumption and Investment}
\small &   2020  Q1 & `19  Q4 & `19  Q3 & `19  Q2 & `19  Q1 &3-year&10-year&30-year\\ Total&0.20&0.44&0.30&0.82&0.50&0.30&-0.01&0.23\\  \hspace{1mm}Federal  total &0.13&0.22&0.22&0.53&0.14&0.18&-0.02&0.07\\  \hspace{1mm}\cbox{blue!60!black}National  defense &0.05&0.17&0.09&0.13&0.29&0.13&-0.04&0.01\\  \hspace{7mm}Consumption  expenditures &0.12&0.09&0.01&0.13&0.25&0.08&-0.03&0.01\\  \hspace{7mm}Gross  investment &-0.07&0.08&0.08&-0.01&0.04&0.04&-0.01&-0.00\\  \hspace{1mm}\cbox{green!85!black}Nondefense &0.09&0.05&0.13&0.40&-0.15&0.06&0.03&0.06\\  \hspace{7mm}Consumption  expenditures &0.09&0.05&0.10&0.36&-0.16&0.05&0.02&0.04\\  \hspace{7mm}Gross  investment &0.00&0.00&0.03&0.04&0.01&0.01&0.01&0.02\\  \hspace{-2mm}\cbox{purple!70!magenta}State  \&  local  total &0.06&0.22&0.08&0.29&0.36&0.12&0.01&0.16\\  \hspace{5mm}Consumption  expenditures &-0.10&0.11&0.15&0.07&0.10&0.07&0.01&0.12\\  \hspace{5mm}Gross  investment &0.16&0.11&-0.08&0.23&0.26&0.05&-0.00&0.03\\   \\
\vspace{2mm}

\noindent \normalsize \textbf{Government Consumption and Investment}\\
\footnotesize{\textit{percentage point contribution to GDP growth}}\\
\noindent \hspace*{-2mm} \begin{tikzpicture}
	\begin{axis}[\bbar{y}{0}, \dateaxisticks ytick={-1, 0, 1},
		xticklabel={`\short{\year}}, clip=false, 
		legend style={at={(0.95, 1.13)}}]
	\rbars
	\sbar{blue!60!black}{date}{A824RY}{data/gov.csv}
	\sbar{green!85!black}{date}{A825RY}{data/gov.csv}
	\sbar{purple!70!magenta}{date}{A829RY}{data/gov.csv}
	\legend{Defense, Federal Non-Defense, State and Local};
	\end{axis}
\end{tikzpicture}\\
\footnotesize{Source: Bureau of Economic Analysis}
\vspace{10mm}


\normalsize Table here.

\end{minipage}

\newpage

\begin{minipage}{0.76\textwidth}

\small Government expenditures provide services and income to people. Government receipts remove demand from the economy. When government expenditures exceed receipts, it is referred to as a government deficit, and corresponds to a private sector surplus. The size of the government deficit relative to GDP gives insight into the extent to which the government is stimulating the economy by increasing household income and corporate profits.\\

Individual text here on the size of deficits at a federal and state and local level, as well as recent changes. \\

\vspace{2mm}

\textbf{Receipts and Expenditures as Share of GDP}
\vspace{2mm}

\noindent \normalsize \hspace{20mm} Federal \hspace{44mm} State and Local \footnotesize 
\vspace{1mm}

\noindent \hspace*{-2mm} \begin{tikzpicture}
	\begin{axis}[\bbar{y}{0}, \dateaxisticks ytick={15, 20, 25}, 
		clip=false, width=6.7cm, 
		xtick={{1989-01-01}, {2000-01-01}, {2010-01-01}, {2019-04-01}},
        minor xtick={}, enlarge y limits={lower, 0.2}, 
        enlarge x limits={0.04},
        xticklabels={`89, `00, `10, `19 Q2}, 
		legend style={at={(0.95, 1.13)}}]
	\rbars
	\stdline{blue!90!cyan}{date}{W005RC}{data/fedgdp.csv}
	\stdline{red}{date}{W013RC}{data/fedgdp.csv}
	\legend{Receipts, Expenditures};
	\end{axis}
\end{tikzpicture}
\hfill
\begin{tikzpicture}
	\begin{axis}[\bbar{y}{0}, \dateaxisticks ytick={12, 14, 16}, 
		clip=false, width=6.7cm, 
		xtick={{1989-01-01}, {2000-01-01}, {2010-01-01}, {2019-04-01}},
        minor xtick={}, 
        xticklabels={`89, `00, `10, `19 Q2}, enlarge y limits={lower, 0.2}, 
        enlarge x limits={0.04}]
	\rbars
	\stdline{blue!90!cyan}{date}{W023RC}{data/slggdp.csv}
	\stdline{red}{date}{W024RC}{data/slggdp.csv}
	\end{axis}
\end{tikzpicture}
\\
\footnotesize{Source: Bureau of Economic Analysis} \\

\vspace{4mm}

\normalsize
Outlays on interest as share of GDP \\

Federal\\

State \\

Local \\

Balance sheets \\

\end{minipage}


\newpage
\begin{minipage}{0.76\textwidth}
\section*{\color{darkgray}\LARGE \seriffont International Transactions}
\small Transactions between the US and the rest of the world are recorded in the balance of payments as either current account transactions (which measure income) or capital and financial account transactions (which measure change in ownership of assets). This section details imbalances in international transactions, changes in trade by goods and by partner, international investment positions, and exchange rates. 

\subsection*{\color{black!70} \seriffont Balance of Payments}
\small The current account balance can be decomposed based on the balance on individual categories. Four major categories are the balance on trade in goods (see\cbox{yellow!80!white}), the balance on trade in services (see\cbox{cyan!50!white}), the balance on primary income (such as wages or income from assets, referred to here as income [see\cbox{blue!80!cyan}]), and secondary income (such as remittances and taxes, referred to here as transfers [see\cbox{green!80!lime!75!black}]). &   2020  Q2 & `20  Q1 & `19  Q4 & `19  Q3 & `19  Q2 & `19  Q1 &3-year&10-year\\  Current  receipts &nan&16.87&17.59&17.83&18.02&18.08&nan&nan\\  \hspace{1mm}Exports &9.26&11.31&11.57&11.63&11.79&11.95&11.78&12.59\\  \hspace{3mm}Goods &5.84&7.42&7.49&7.55&7.65&7.86&7.68&8.45\\  \hspace{5mm}Durable &3.24&4.35&4.45&4.51&4.57&4.77&4.63&5.21\\  \hspace{5mm}Nondurable &2.60&3.07&3.04&3.04&3.08&3.09&3.05&3.24\\  \hspace{3mm}Services &3.42&3.89&4.08&4.08&4.14&4.09&4.10&4.15\\  \hspace{1mm}Income  receipts &nan&4.89&5.36&5.48&5.55&5.44&nan&nan\\  \hspace{1mm}Transfer  receipts &0.72&0.67&0.66&0.72&0.68&0.69&0.73&0.74\\  Current  payments &nan&18.84&19.59&20.19&20.53&20.60&nan&nan\\  \hspace{1mm}Imports &12.07&13.60&14.10&14.56&14.81&14.87&14.61&15.64\\  \hspace{3mm}Goods &9.98&11.03&11.31&11.77&11.98&12.08&11.88&12.86\\  \hspace{5mm}Durable &6.21&7.05&7.31&7.62&7.70&7.89&7.71&7.92\\  \hspace{5mm}Nondurable &3.77&3.98&4.01&4.15&4.29&4.19&4.18&4.95\\  \hspace{3mm}Services &2.09&2.57&2.78&2.80&2.83&2.79&2.73&2.77\\  \hspace{1mm}Income  payments &nan&3.76&4.07&4.18&4.28&4.27&nan&nan\\  \hspace{1mm}Transfer  payments &1.60&1.47&1.42&1.44&1.44&1.46&1.46&1.42\\  Current  account  balance &nan&-1.96&-2.00&-2.36&-2.51&-2.52&nan&nan\\  

\vspace{5mm}

\noindent \normalsize \textbf{Current Account Balance}\\
\footnotesize{\textit{balance on individual current account component, as percent of GDP}}\\
\noindent \hspace*{-2mm} \begin{tikzpicture}
	\begin{axis}[\bbar{y}{0}, \dateaxisticks ytick={-5, -2, 0, 2},
		xticklabel={`\short{\year}}, clip=false, 
		legend style={at={(0.95, 1.13)}}]
	\rbars
	\sbar{yellow!80!white}{date}{GOODS}{data/cab.csv}
	\sbar{cyan!50!white}{date}{SERVICES}{data/cab.csv}
	\sbar{blue!80!cyan}{date}{INCOME}{data/cab.csv}
	\sbar{green!80!lime!75!black}{date}{TRANSFERS}{data/cab.csv}
	\stdline{black}{date}{A124RC}{data/cab.csv}
	\legend{Goods, Services, Income, Transfers, Total};
	\end{axis}
\end{tikzpicture}\\
\footnotesize{Source: Bureau of Economic Analysis}

\vspace{8mm}


\normalsize [Capital account balance]


\end{minipage}

\newpage

\begin{minipage}{0.76\textwidth}

\subsection*{\color{black!70} \seriffont Trade}
\small The trade balance (exports of goods\hspace*{-0.5mm}\cbox{green!70!white} and services\hspace*{-0.5mm}\cbox{green!50!black} minus imports of goods\hspace*{-0.5mm}\cbox{cyan!70!white} and services\hspace*{-0.5mm}\cbox{blue!70!black}) acts as an adjustment to consumption and investment in GDP calculations. As the US runs a persistent trade deficit, trade will generally subtract from GDP growth. In the income approach, the expanded trade deficit reduced nominal compensation of employees (extensive margin through outsourcing, intensive margin through lower wages from labor market slack) and reduced prices. \\

 \input{text/trade.txt} 
\vspace{5mm}

\noindent \normalsize \textbf{International Trade}\\
\footnotesize{\textit{percentage point contribution to GDP growth}}\\
\noindent \hspace*{-2mm} \begin{tikzpicture}
	\begin{axis}[\bbar{y}{0}, \dateaxisticks ytick={-2, 0, 2, 5},
		xticklabel={`\short{\year}}, clip=false, 
		legend style={at={(0.95, 1.13)}}]
	\rbars
	\sbar{green!70!white}{date}{A253RY}{data/nx.csv}
	\sbar{green!50!black}{date}{A646RY}{data/nx.csv}
	\sbar{cyan!70!white}{date}{A255RY}{data/nx.csv}
	\sbar{blue!70!black}{date}{A656RY}{data/nx.csv}	
	\legend{Goods Exports, Services Exports, Goods Imports, Services Imports};
	\end{axis}
\end{tikzpicture}\\
\footnotesize{Source: Bureau of Economic Analysis} \\

\vspace{8mm}

\small &   2020  Q2 & `20  Q1 & `19  Q2 &2016& 2012  --13 & 2005  --06 & 1998  --99 & 1989  --93 \\  Exports  of  goods  and  services &9.23&11.31&11.79&11.88&13.54&10.33&10.41&9.42\\  Exports  of  goods &5.82&7.42&7.65&7.70&9.34&7.32&7.52&6.84\\  \hspace{2mm}Foods,  feeds,  and  beverages &0.64&0.61&0.63&0.70&0.82&0.46&0.50&0.60\\  \hspace{2mm}Industrial  supplies  \&  materials &1.90&2.48&2.48&2.07&2.96&1.92&1.55&1.65\\  \hspace{4mm}Petroleum  and  products &0.50&0.94&0.92&0.53&0.90&0.28&0.11&0.12\\  \hspace{2mm}Capital  goods,  except  automotive &2.03&2.45&2.54&2.77&3.22&2.84&3.27&2.61\\  \hspace{2mm}Automotive  vehicles,  \&  parts &0.32&0.70&0.76&0.80&0.91&0.77&0.79&0.67\\  \hspace{2mm}Consumer  goods,  ex.  food  \&  auto &0.69&0.87&0.96&1.03&1.12&0.91&0.86&0.74\\  \hspace{4mm}Durable  goods &0.26&0.44&0.51&0.56&0.61&0.50&0.44&0.39\\  \hspace{4mm}Nondurable  goods &0.42&0.44&0.45&0.47&0.51&0.41&0.42&0.35\\  Exports  of  services &3.41&3.89&4.14&4.18&4.19&3.02&2.90&2.58\\  \hspace{2mm}Transport &0.18&0.38&0.43&0.44&0.52&0.41&0.48&0.59\\  \hspace{2mm}Travel &0.27&0.72&0.92&1.03&1.03&0.77&0.95&0.90\\  \hspace{2mm}Intellectual  property  charges &0.59&0.55&0.55&0.60&0.77&0.59&0.44&0.29\\  \hspace{2mm}Other  business  services &2.15&2.04&2.04&1.92&1.67&1.04&0.85&0.60\\  Imports  of  goods  and  services &11.98&13.60&14.81&14.61&16.76&15.89&12.63&10.38\\  Imports  of  goods &9.90&11.03&11.98&11.85&13.95&13.44&10.59&8.45\\  \hspace{2mm}Foods,  feeds,  and  beverages &0.76&0.72&0.72&0.70&0.69&0.54&0.46&0.43\\  \hspace{2mm}Industrial  supplies  \&  materials &1.82&2.23&2.51&2.33&4.26&4.24&2.22&2.16\\  \hspace{4mm}Petroleum  and  products &0.42&0.85&1.06&0.85&2.50&2.15&0.65&0.87\\  \hspace{2mm}Capital  goods,  except  automotive &3.00&3.00&3.20&3.17&3.37&3.00&3.03&2.04\\  \hspace{2mm}Automotive  vehicles,  \&  parts &0.84&1.63&1.80&1.87&1.84&1.84&1.74&1.46\\  \hspace{2mm}Consumer  goods,  ex.  food  \&  auto &2.88&2.80&3.10&3.12&3.19&3.20&2.47&1.83\\  \hspace{4mm}Durable  goods &1.25&1.28&1.53&1.63&1.71&1.75&1.29&0.97\\  \hspace{4mm}Nondurable  goods &1.63&1.51&1.56&1.49&1.48&1.46&1.18&0.86\\  Imports  of  services &2.09&2.57&2.83&2.77&2.81&2.45&2.04&1.93\\  \hspace{2mm}Transport &0.24&0.42&0.50&0.49&0.53&0.57&0.54&0.55\\  \hspace{2mm}Travel &0.05&0.46&0.65&0.58&0.60&0.61&0.63&0.61\\  \hspace{2mm}Intellectual  property  charges &0.19&0.19&0.20&0.22&0.24&0.19&0.13&0.06\\  \hspace{2mm}Other  business  services &1.43&1.31&1.28&1.28&1.24&0.83&0.54&0.38\\ \\

\vspace{2mm}


\noindent \normalsize \textbf{Imports and Exports, Nonpetroleum}\\
\footnotesize{\textit{includes goods and services, but excludes petroleum products, share of GDP}}\\
\noindent \hspace*{-2mm} \begin{tikzpicture}
	\begin{axis}[\bbar{y}{0}, \dateaxisticks ytick={10, 12, 14}, enlarge y limits={0.1},
		xticklabel={`\short{\year}}, clip=false, height=5.0cm,
		legend style={legend columns=1, at={(0.28, 0.84)}}]
	\rbars
	\thickline{blue!90!cyan}{date}{EX}{data/eximgdp.csv}
	\thickline{green!60!teal!80!black}{date}{IM}{data/eximgdp.csv}
	\legend{Exports, Imports};
	\end{axis}
\end{tikzpicture}\\
\footnotesize{Source: Bureau of Economic Analysis} \\


\end{minipage}

\newpage


\begin{minipage}{0.76\textwidth}

\small Changes to the trade balance come from a myriad of potential sources, such as changes in demand or relative supply of other countries, changes in exchange rates, changes in preferences for categories of goods, changes in trade policy, and changes in domestic demand. The following table captures the nominal value of major categories of goods and services as a share of nominal gross domestic product at various points over the past 30 years.  \\

\end{minipage}

\noindent \normalsize \textbf{Exports and Imports by Type}\\
\footnotesize{\textit{percentage point share of GDP \hspace{40mm} period averages}\\ 
\noindent \rowcolors{1}{}{black!5} \setlength{\tabcolsep}{3.1pt} \color{black!90}
		{\renewcommand{\arraystretch}{1.55}
		 \begin{tabular}{p{42mm} R{7.1mm} R{7.1mm} R{7.1mm} R{7.1mm} R{7.1mm} 
		   R{7.1mm} R{7.1mm} R{7.1mm} }
			 &   2020  Q2 & `20  Q1 & `19  Q2 &2016& 2012  --13 & 2005  --06 & 1998  --99 & 1989  --93 \\  Exports  of  goods  and  services &9.23&11.31&11.79&11.88&13.54&10.33&10.41&9.42\\  Exports  of  goods &5.82&7.42&7.65&7.70&9.34&7.32&7.52&6.84\\  \hspace{2mm}Foods,  feeds,  and  beverages &0.64&0.61&0.63&0.70&0.82&0.46&0.50&0.60\\  \hspace{2mm}Industrial  supplies  \&  materials &1.90&2.48&2.48&2.07&2.96&1.92&1.55&1.65\\  \hspace{4mm}Petroleum  and  products &0.50&0.94&0.92&0.53&0.90&0.28&0.11&0.12\\  \hspace{2mm}Capital  goods,  except  automotive &2.03&2.45&2.54&2.77&3.22&2.84&3.27&2.61\\  \hspace{2mm}Automotive  vehicles,  \&  parts &0.32&0.70&0.76&0.80&0.91&0.77&0.79&0.67\\  \hspace{2mm}Consumer  goods,  ex.  food  \&  auto &0.69&0.87&0.96&1.03&1.12&0.91&0.86&0.74\\  \hspace{4mm}Durable  goods &0.26&0.44&0.51&0.56&0.61&0.50&0.44&0.39\\  \hspace{4mm}Nondurable  goods &0.42&0.44&0.45&0.47&0.51&0.41&0.42&0.35\\  Exports  of  services &3.41&3.89&4.14&4.18&4.19&3.02&2.90&2.58\\  \hspace{2mm}Transport &0.18&0.38&0.43&0.44&0.52&0.41&0.48&0.59\\  \hspace{2mm}Travel &0.27&0.72&0.92&1.03&1.03&0.77&0.95&0.90\\  \hspace{2mm}Intellectual  property  charges &0.59&0.55&0.55&0.60&0.77&0.59&0.44&0.29\\  \hspace{2mm}Other  business  services &2.15&2.04&2.04&1.92&1.67&1.04&0.85&0.60\\  Imports  of  goods  and  services &11.98&13.60&14.81&14.61&16.76&15.89&12.63&10.38\\  Imports  of  goods &9.90&11.03&11.98&11.85&13.95&13.44&10.59&8.45\\  \hspace{2mm}Foods,  feeds,  and  beverages &0.76&0.72&0.72&0.70&0.69&0.54&0.46&0.43\\  \hspace{2mm}Industrial  supplies  \&  materials &1.82&2.23&2.51&2.33&4.26&4.24&2.22&2.16\\  \hspace{4mm}Petroleum  and  products &0.42&0.85&1.06&0.85&2.50&2.15&0.65&0.87\\  \hspace{2mm}Capital  goods,  except  automotive &3.00&3.00&3.20&3.17&3.37&3.00&3.03&2.04\\  \hspace{2mm}Automotive  vehicles,  \&  parts &0.84&1.63&1.80&1.87&1.84&1.84&1.74&1.46\\  \hspace{2mm}Consumer  goods,  ex.  food  \&  auto &2.88&2.80&3.10&3.12&3.19&3.20&2.47&1.83\\  \hspace{4mm}Durable  goods &1.25&1.28&1.53&1.63&1.71&1.75&1.29&0.97\\  \hspace{4mm}Nondurable  goods &1.63&1.51&1.56&1.49&1.48&1.46&1.18&0.86\\  Imports  of  services &2.09&2.57&2.83&2.77&2.81&2.45&2.04&1.93\\  \hspace{2mm}Transport &0.24&0.42&0.50&0.49&0.53&0.57&0.54&0.55\\  \hspace{2mm}Travel &0.05&0.46&0.65&0.58&0.60&0.61&0.63&0.61\\  \hspace{2mm}Intellectual  property  charges &0.19&0.19&0.20&0.22&0.24&0.19&0.13&0.06\\  \hspace{2mm}Other  business  services &1.43&1.31&1.28&1.28&1.24&0.83&0.54&0.38\\ \hline
		\end{tabular}}	\\
		
\vspace{-2mm}
\footnotesize{Source: Bureau of Economic Analysis}}



\newpage


\begin{minipage}{0.76\textwidth}

\small Goods can be produced domestically or imported or some combination of the two. The import share of the total US demand for goods, measured as US produced goods and imported goods less exported goods, is also referred to as ``import penetration''. This measure has risen considerably over the past thirty years. The majority of the long-term increase has been concentrated in consumer goods, while the decrease since 2011 has come primarily from petroleum and products. \\

\input{text/goodsimpsh.txt} \\

\noindent \normalsize \textbf{Import Share of Goods}\\
\footnotesize{\textit{import share of domestic-produced and imported goods less exports, by category}}\\
\noindent \hspace*{-2mm} \begin{tikzpicture}
	\begin{axis}[\bbar{y}{0}, \dateaxisticks ytick={0, 10, 20, 30, 40}, ymin=2,
		xticklabel={`\short{\year}}, clip=false, legend style={at={(0.95, 1.13)}}]
	\rbars
	\sbar{cyan!40!white}{date}{Consumer}{data/goodsimpsh.csv}	
	\sbar{blue!50!cyan}{date}{Capital}{data/goodsimpsh.csv}	
	\sbar{purple}{date}{B648RC}{data/goodsimpsh.csv}	
	\legend{Consumer, Capital/Supplies/Other, Petroleum and Products};
	\end{axis}
\end{tikzpicture}\\
\footnotesize{Source: Bureau of Economic Analysis}\\


\vspace{5mm}

\normalsize 

Trade in Goods \\

Trade in Services \\

Trade balance \\

[One page table to capture lots of external sector items as contribution to GDP growth (where possible) or otherwise as a share of GDP]\\

Direct and Portfolio Investment -- related here and to IIP below: the total value of domestic holdings of foreign assets is much smaller than the total value of foreign holdings of domestic assets, but, the return on foreign assets is so much higher than the return on domestic assets that the the US has positive net income from abroad.\\

International Investment Position \\

\end{minipage}

\newpage

\begin{minipage}{0.76\textwidth}

\subsection*{\color{black!70} \seriffont Exchange Rates}

\small Text here about exchange rates with selected other major currencies. The first chart shows the amount of Japanese Yen (JPY), British Pounds (GBP), Euros (EUR), and Canadian Dollars (CAD) required to buy one US Dollar (USD).\\ 

\vspace{2mm}

\noindent \normalsize \textbf{Selected Exchange Rates}\\
\footnotesize{\textit{units of foreign currency required to purchase one US dollar}}\\
\noindent \hspace*{-2mm} \begin{tikzpicture}
	\begin{axis}[\bbar{y}{1}, \dateaxisticks ytick={0.5, 1, 1.5}, enlarge y limits={0.1},
		xticklabel={`\short{\year}}, clip=false, height=5.0cm]
	\rbars
	\stdline{blue!90!cyan}{date}{RXI_N.B.UK}{data/fx1.csv}
	\stdline{green!90!blue}{date}{RXI_N.B.CA}{data/fx1.csv}
	\stdline{red}{date}{RXI_N.B.JA}{data/fx1.csv}
	\stdline{cyan!90!white}{date}{RXI_N.B.EU}{data/fx1.csv}
	\stdnode{10.8cm}{0.6cm}{\footnotesize \color{blue!90!cyan}GBP}
	\stdnode{5.5cm}{3.0cm}{\footnotesize \color{green!90!blue}CAD}
	\stdnode{4.1cm}{1.1cm}{\footnotesize \color{cyan!90!white}EUR}
	\stdnode{1.5cm}{1.24cm}{\footnotesize \color{red}JPY (100)}
	\end{axis}
\end{tikzpicture}\\
\footnotesize{Source: Federal Reserve} \\

\vspace{4mm}

\small Text here about other exchange rates. This next chart covers the Mexican Peso (MXN), the Brazilian Real (BRL), the Chinese Yuan (CNY), and the Singapore Dollar (SGP).\\

\vspace{2mm}

\noindent \normalsize \textbf{Selected Exchange Rates, Continued}\\
\footnotesize{\textit{units of foreign currency required to purchase one US dollar}}\\
\noindent \hspace*{-2mm} \begin{tikzpicture}
	\begin{axis}[\bbar{y}{1}, \dateaxisticks ytick={1, 10, 20, 30}, enlarge y limits={0.1},
		xticklabel={`\short{\year}}, clip=false, height=5.0cm]
	\rbars
	\stdline{blue!40!cyan}{date}{RXI_N.B.MX}{data/fx2.csv}
	\stdline{green!70!lime}{date}{RXI_N.B.BZ}{data/fx2.csv}
	\stdline{orange!80!red}{date}{RXI_N.B.SD}{data/fx2.csv}
	\stdline{blue!60!black}{date}{RXI_N.B.CH}{data/fx2.csv}
	\stdnode{9.0cm}{2.1cm}{\footnotesize \color{blue!40!cyan}MXN}
	\stdnode{4.1cm}{0.6cm}{\footnotesize \color{green!70!lime}BRL}
	\stdnode{11.0cm}{1.7cm}{\footnotesize \color{orange}SGP}
	\stdnode{0.1cm}{0.4cm}{\footnotesize \color{blue!60!black}CNY}
	\end{axis}
\end{tikzpicture}\\
\footnotesize{Source: Federal Reserve} \\

\vspace{4mm}

\small Trade-weighted dollar indices discussed here. Major currencies index goes back further, while the broad index starts in 1994.\\

\vspace{2mm}

\noindent \normalsize \textbf{Trade-Weighted USD Indices}\\
\footnotesize{\textit{index, 1973=100}}\\
\noindent \hspace*{-2mm} \begin{tikzpicture}
	\begin{axis}[\bbar{y}{1}, \dateaxisticks ytick={60, 80, 100, 120}, 
		enlarge y limits={0.1},
		xticklabel={`\short{\year}}, clip=false, height=5.0cm,
		legend style={fill=white, at={(0.5, 1.0)}}]
	\rbars
	\stdline{blue!60!black}{date}{V0.JRXWTFN_N.B}{data/fx_idx.csv}
	\stdline{magenta}{date}{V0.JRXWTFB_N.B}{data/fx_idx.csv}
	\stdnode{8.9cm}{2.1cm}{\footnotesize \color{magenta}Broad}
	\stdnode{4.0cm}{1.1cm}{\footnotesize \color{blue!60!black}Major}
	\end{axis}
\end{tikzpicture}\\
\footnotesize{Source: Federal Reserve} \\

\end{minipage}

\newpage

\begin{minipage}{0.76\textwidth}

\section*{\color{darkgray}\LARGE \seriffont Labor Markets}

\small Labor is the primary source of income for US households and essential to the production of goods and services. In labor markets, unlike other markets, wages (the price of labor) tend not to be cut in response to a decrease in demand; businesses instead employ fewer workers and/or cut hours. \\

Gross labor income (compensation of employees in the national accounts), which captures both employment and wages, \input{text/gli.txt} \\

\noindent \normalsize \textbf{Gross Labor Income Growth}\\
\footnotesize{\textit{percentage point contribution to gross labor income growth}}\\
\noindent \hspace*{-2mm} \begin{tikzpicture}
	\begin{axis}[\bbar{y}{0}, \dateaxisticks ytick={-10, -5, 0, 5, 10},
		xticklabel={`\short{\year}}, clip=false,  yticklabel style={text width=1.5em},
		legend style={at={(0.95, 1.13)}}]
	\rbars
	\sbar{green!80!lime!90!white}{date}{wage}{data/gli.csv}
	\sbar{teal!80!blue!85!white}{date}{work}{data/gli.csv}
	\legend{Rate of Compensation, Hours of Work};
	\end{axis}
\end{tikzpicture}\\
\footnotesize{Source: Author's Calculations} \\

\vspace{6mm}

\small More text here.

\end{minipage}

\newpage

\begin{minipage}{0.76\textwidth}

\subsection*{\color{black!70} \seriffont Employment}


\small \input{text/epop_text.txt} \\

\noindent \normalsize \textbf{Employment Rate}\\
\footnotesize{\textit{employed share of age 25-54 population}}\\
\noindent \hspace*{-2mm} \begin{tikzpicture}
	\begin{axis}[\bbar{y}{0}, \dateaxisticks ytick={75, 80}, enlarge y limits={0.1},
		xticklabel={`\short{\year}}, clip=false, height=5.0cm]
	\rbars
	\stdline{blue!90!cyan}{date}{PA_EPOP}{data/epop.csv}
	\node[above, align=right] at (axis cs:2018-06-01,80.9) {\scriptsize \input{text/epop.txt}};
	\end{axis}
\end{tikzpicture}\\
\footnotesize{Source: Bureau of Labor Statistics} \\

\vspace{4mm}

\small The monthly establishment survey enables tracking of non-farm payrolls. In September 2019, the US economy added 136,000 jobs. In 2019 Q2, the US added an average of 146,000 jobs per month, compared to 205,000 in 2019 Q1 and an annual average of 205,000 in 2018. \\


\normalsize [Quarterly employment growth with dot for latest monthly value]\\

\end{minipage}


\newpage


\begin{minipage}{0.76\textwidth} 

\subsection*{\color{black!70} \seriffont Unemployment}

\vspace{2mm}

\small The conventional ``unemployment rate'' is measured as the number of people who do not have a job and looked for one during a reference week, divided by the labor force, which includes the unemployed and those with jobs.\\

\input{text/unemp.txt}\\

\vspace{2mm}

\noindent \normalsize \textbf{Unemployment Rate}\\
\footnotesize{\textit{unemployed share of labor force}}\\
\noindent \hspace*{-2mm} \begin{tikzpicture}
	\begin{axis}[\bbar{y}{0}, \dateaxisticks ytick={0, 5, 10, 15}, enlarge y limits={0.1},
		xticklabel={`\short{\year}}, clip=false, height=5.0cm,
		legend style={fill=white, at={(0.5, 1.0)}}]
	\rbars
	\stdline{blue!90!cyan}{date}{White}{data/unemp.csv}
	\stdline{green!90!blue}{date}{Hispanic}{data/unemp.csv}
	\stdline{red}{date}{Black}{data/unemp.csv}
	\legend{White, Hispanic, Black};
	\end{axis}
\end{tikzpicture}\\
\footnotesize{Source: Bureau of Labor Statistics} \\

\vspace{3mm}

\subsubsection*{\color{black!70} \seriffont Reasons for unemployment}

\small There are multiple reasons for unemployment. During the trough of a business cycle, most unemployed are those who lost a job, for example from layoffs, or had a temporary job end (see\cbox{blue!30!cyan}). In general, many of the unemployed are re-entrants to the labor market, meaning they were out of the labor force prior but are looking for a job again (see\cbox{orange!80!yellow}). Some are new-entrants who are looking for their first job (see\cbox{red}). A small portion are also those who left a job voluntarily and are looking for a new one (see\cbox{yellow}). \\

\input{text/unemp_reason.txt} \\

\vspace{2mm}

\noindent \normalsize \textbf{Unemployment by Reason}\\
\footnotesize{\textit{unemployed share of labor force, by reason for unemployment}}\\
\noindent \hspace*{-3mm} \begin{tikzpicture}
	\begin{axis}[\bbar{y}{0}, \dateaxisticks ytick={0, 2, 4, 6, 8, 10}, 
		height=5.8cm, ymin=0,
		xticklabel={`\short{\year}}, clip=false, 
		legend style={fill=white, legend columns=2, at={(0.53, 1.0)}}]
	\rbars
	\sbar{blue!30!cyan}{date}{Job Loser}{data/unemp_reason.csv}
	\sbar{yellow}{date}{Job Leaver}{data/unemp_reason.csv}
	\sbar{orange!80!yellow}{date}{Re-entrant}{data/unemp_reason.csv}
	\sbar{red}{date}{New entrant}{data/unemp_reason.csv}
	\legend{Job Loser, Job Leaver, Re-entrant, New Entrant};
	\end{axis}
\end{tikzpicture}\\
\footnotesize{Source: Bureau of Labor Statistics}\\

\end{minipage}


\newpage

\vspace{4mm}

\normalsize

Unemployment by duration \\

\vspace{4mm}

\begin{minipage}{0.76\textwidth}

\small Summary text about local area estimates of unemployment. Will need to think about tables that show highlights, because there are too many MSAs to list all data. Something that captures diffusion would be nice. Perhaps I can list how many metro areas had the unemployment rate fall over the past year, and then talk about how many unemployed people that actually means--so that population is taken into consideration in some meaningful way.\\

\vspace{2mm}

\noindent \normalsize \textbf{Change in Unemployment Rate by Metro Area}\\
\footnotesize{\textit{one-year change, in percentage points, \input{text/unemp_map_date.txt}}}\\

\vspace{-2mm}

\hspace*{-15mm} %% Creator: Matplotlib, PGF backend
%%
%% To include the figure in your LaTeX document, write
%%   \input{<filename>.pgf}
%%
%% Make sure the required packages are loaded in your preamble
%%   \usepackage{pgf}
%%
%% Figures using additional raster images can only be included by \input if
%% they are in the same directory as the main LaTeX file. For loading figures
%% from other directories you can use the `import` package
%%   \usepackage{import}
%%
%% and then include the figures with
%%   \import{<path to file>}{<filename>.pgf}
%%
%% Matplotlib used the following preamble
%%   \usepackage{fontspec}
%%   \setmainfont{DejaVuSerif.ttf}[Path=\detokenize{/home/brian/miniconda3/lib/python3.8/site-packages/matplotlib/mpl-data/fonts/ttf/}]
%%   \setsansfont{DejaVuSans.ttf}[Path=\detokenize{/home/brian/miniconda3/lib/python3.8/site-packages/matplotlib/mpl-data/fonts/ttf/}]
%%   \setmonofont{DejaVuSansMono.ttf}[Path=\detokenize{/home/brian/miniconda3/lib/python3.8/site-packages/matplotlib/mpl-data/fonts/ttf/}]
%%
\begingroup%
\makeatletter%
\begin{pgfpicture}%
\pgfpathrectangle{\pgfpointorigin}{\pgfqpoint{5.237500in}{5.940271in}}%
\pgfusepath{use as bounding box, clip}%
\begin{pgfscope}%
\pgfsetbuttcap%
\pgfsetmiterjoin%
\pgfsetlinewidth{0.000000pt}%
\definecolor{currentstroke}{rgb}{1.000000,1.000000,1.000000}%
\pgfsetstrokecolor{currentstroke}%
\pgfsetstrokeopacity{0.000000}%
\pgfsetdash{}{0pt}%
\pgfpathmoveto{\pgfqpoint{0.000000in}{0.000000in}}%
\pgfpathlineto{\pgfqpoint{5.237500in}{0.000000in}}%
\pgfpathlineto{\pgfqpoint{5.237500in}{5.940271in}}%
\pgfpathlineto{\pgfqpoint{0.000000in}{5.940271in}}%
\pgfpathclose%
\pgfusepath{}%
\end{pgfscope}%
\begin{pgfscope}%
\pgfpathrectangle{\pgfqpoint{0.100000in}{2.413063in}}{\pgfqpoint{5.037500in}{3.427208in}}%
\pgfusepath{clip}%
\pgfsetbuttcap%
\pgfsetmiterjoin%
\definecolor{currentfill}{rgb}{1.000000,1.000000,1.000000}%
\pgfsetfillcolor{currentfill}%
\pgfsetlinewidth{0.501875pt}%
\definecolor{currentstroke}{rgb}{0.827451,0.827451,0.827451}%
\pgfsetstrokecolor{currentstroke}%
\pgfsetdash{}{0pt}%
\pgfpathmoveto{\pgfqpoint{1.635246in}{3.264091in}}%
\pgfpathlineto{\pgfqpoint{1.620137in}{3.264339in}}%
\pgfpathlineto{\pgfqpoint{1.609854in}{3.275720in}}%
\pgfpathlineto{\pgfqpoint{1.602861in}{3.305947in}}%
\pgfpathlineto{\pgfqpoint{1.619112in}{3.309734in}}%
\pgfpathlineto{\pgfqpoint{1.635621in}{3.305661in}}%
\pgfpathlineto{\pgfqpoint{1.652422in}{3.293753in}}%
\pgfpathlineto{\pgfqpoint{1.652414in}{3.277285in}}%
\pgfpathclose%
\pgfusepath{stroke,fill}%
\end{pgfscope}%
\begin{pgfscope}%
\pgfpathrectangle{\pgfqpoint{0.100000in}{2.413063in}}{\pgfqpoint{5.037500in}{3.427208in}}%
\pgfusepath{clip}%
\pgfsetbuttcap%
\pgfsetmiterjoin%
\definecolor{currentfill}{rgb}{1.000000,1.000000,1.000000}%
\pgfsetfillcolor{currentfill}%
\pgfsetlinewidth{0.501875pt}%
\definecolor{currentstroke}{rgb}{0.827451,0.827451,0.827451}%
\pgfsetstrokecolor{currentstroke}%
\pgfsetdash{}{0pt}%
\pgfpathmoveto{\pgfqpoint{1.727210in}{3.084640in}}%
\pgfpathlineto{\pgfqpoint{1.710308in}{3.091076in}}%
\pgfpathlineto{\pgfqpoint{1.710230in}{3.103701in}}%
\pgfpathlineto{\pgfqpoint{1.689811in}{3.114378in}}%
\pgfpathlineto{\pgfqpoint{1.692298in}{3.141261in}}%
\pgfpathlineto{\pgfqpoint{1.698212in}{3.153947in}}%
\pgfpathlineto{\pgfqpoint{1.710698in}{3.142271in}}%
\pgfpathlineto{\pgfqpoint{1.731004in}{3.144222in}}%
\pgfpathlineto{\pgfqpoint{1.725876in}{3.107065in}}%
\pgfpathclose%
\pgfusepath{stroke,fill}%
\end{pgfscope}%
\begin{pgfscope}%
\pgfpathrectangle{\pgfqpoint{0.100000in}{2.413063in}}{\pgfqpoint{5.037500in}{3.427208in}}%
\pgfusepath{clip}%
\pgfsetbuttcap%
\pgfsetmiterjoin%
\definecolor{currentfill}{rgb}{1.000000,1.000000,1.000000}%
\pgfsetfillcolor{currentfill}%
\pgfsetlinewidth{0.501875pt}%
\definecolor{currentstroke}{rgb}{0.827451,0.827451,0.827451}%
\pgfsetstrokecolor{currentstroke}%
\pgfsetdash{}{0pt}%
\pgfpathmoveto{\pgfqpoint{1.798159in}{3.001762in}}%
\pgfpathlineto{\pgfqpoint{1.776708in}{3.005415in}}%
\pgfpathlineto{\pgfqpoint{1.764847in}{3.023779in}}%
\pgfpathlineto{\pgfqpoint{1.763172in}{3.038878in}}%
\pgfpathclose%
\pgfusepath{stroke,fill}%
\end{pgfscope}%
\begin{pgfscope}%
\pgfpathrectangle{\pgfqpoint{0.100000in}{2.413063in}}{\pgfqpoint{5.037500in}{3.427208in}}%
\pgfusepath{clip}%
\pgfsetbuttcap%
\pgfsetmiterjoin%
\definecolor{currentfill}{rgb}{1.000000,1.000000,1.000000}%
\pgfsetfillcolor{currentfill}%
\pgfsetlinewidth{0.501875pt}%
\definecolor{currentstroke}{rgb}{0.827451,0.827451,0.827451}%
\pgfsetstrokecolor{currentstroke}%
\pgfsetdash{}{0pt}%
\pgfpathmoveto{\pgfqpoint{1.754202in}{3.004965in}}%
\pgfpathlineto{\pgfqpoint{1.765448in}{2.995835in}}%
\pgfpathlineto{\pgfqpoint{1.767890in}{2.980139in}}%
\pgfpathlineto{\pgfqpoint{1.745251in}{2.981006in}}%
\pgfpathclose%
\pgfusepath{stroke,fill}%
\end{pgfscope}%
\begin{pgfscope}%
\pgfpathrectangle{\pgfqpoint{0.100000in}{2.413063in}}{\pgfqpoint{5.037500in}{3.427208in}}%
\pgfusepath{clip}%
\pgfsetbuttcap%
\pgfsetmiterjoin%
\definecolor{currentfill}{rgb}{1.000000,1.000000,1.000000}%
\pgfsetfillcolor{currentfill}%
\pgfsetlinewidth{0.501875pt}%
\definecolor{currentstroke}{rgb}{0.827451,0.827451,0.827451}%
\pgfsetstrokecolor{currentstroke}%
\pgfsetdash{}{0pt}%
\pgfpathmoveto{\pgfqpoint{1.805849in}{2.917287in}}%
\pgfpathlineto{\pgfqpoint{1.783257in}{2.925408in}}%
\pgfpathlineto{\pgfqpoint{1.789731in}{2.945097in}}%
\pgfpathlineto{\pgfqpoint{1.780262in}{2.965159in}}%
\pgfpathlineto{\pgfqpoint{1.783236in}{2.978951in}}%
\pgfpathlineto{\pgfqpoint{1.801416in}{2.984207in}}%
\pgfpathlineto{\pgfqpoint{1.801066in}{2.962112in}}%
\pgfpathlineto{\pgfqpoint{1.822569in}{2.949856in}}%
\pgfpathlineto{\pgfqpoint{1.834868in}{2.920225in}}%
\pgfpathlineto{\pgfqpoint{1.821162in}{2.908814in}}%
\pgfpathclose%
\pgfusepath{stroke,fill}%
\end{pgfscope}%
\begin{pgfscope}%
\pgfpathrectangle{\pgfqpoint{0.100000in}{2.413063in}}{\pgfqpoint{5.037500in}{3.427208in}}%
\pgfusepath{clip}%
\pgfsetbuttcap%
\pgfsetmiterjoin%
\definecolor{currentfill}{rgb}{1.000000,1.000000,1.000000}%
\pgfsetfillcolor{currentfill}%
\pgfsetlinewidth{0.501875pt}%
\definecolor{currentstroke}{rgb}{0.827451,0.827451,0.827451}%
\pgfsetstrokecolor{currentstroke}%
\pgfsetdash{}{0pt}%
\pgfpathmoveto{\pgfqpoint{1.724255in}{2.716939in}}%
\pgfpathlineto{\pgfqpoint{1.713491in}{2.739173in}}%
\pgfpathlineto{\pgfqpoint{1.718257in}{2.754128in}}%
\pgfpathlineto{\pgfqpoint{1.739369in}{2.774829in}}%
\pgfpathlineto{\pgfqpoint{1.741367in}{2.790698in}}%
\pgfpathlineto{\pgfqpoint{1.755183in}{2.825830in}}%
\pgfpathlineto{\pgfqpoint{1.791521in}{2.831833in}}%
\pgfpathlineto{\pgfqpoint{1.798476in}{2.852398in}}%
\pgfpathlineto{\pgfqpoint{1.814497in}{2.844687in}}%
\pgfpathlineto{\pgfqpoint{1.847051in}{2.786970in}}%
\pgfpathlineto{\pgfqpoint{1.847462in}{2.768126in}}%
\pgfpathlineto{\pgfqpoint{1.838260in}{2.751817in}}%
\pgfpathlineto{\pgfqpoint{1.849936in}{2.714370in}}%
\pgfpathlineto{\pgfqpoint{1.841778in}{2.708333in}}%
\pgfpathlineto{\pgfqpoint{1.813989in}{2.710088in}}%
\pgfpathlineto{\pgfqpoint{1.780572in}{2.721257in}}%
\pgfpathlineto{\pgfqpoint{1.752487in}{2.725868in}}%
\pgfpathlineto{\pgfqpoint{1.744986in}{2.719316in}}%
\pgfpathclose%
\pgfusepath{stroke,fill}%
\end{pgfscope}%
\begin{pgfscope}%
\pgfpathrectangle{\pgfqpoint{0.100000in}{2.413063in}}{\pgfqpoint{5.037500in}{3.427208in}}%
\pgfusepath{clip}%
\pgfsetbuttcap%
\pgfsetmiterjoin%
\definecolor{currentfill}{rgb}{1.000000,1.000000,1.000000}%
\pgfsetfillcolor{currentfill}%
\pgfsetlinewidth{0.501875pt}%
\definecolor{currentstroke}{rgb}{0.827451,0.827451,0.827451}%
\pgfsetstrokecolor{currentstroke}%
\pgfsetdash{}{0pt}%
\pgfpathmoveto{\pgfqpoint{0.951624in}{2.987308in}}%
\pgfpathlineto{\pgfqpoint{0.949465in}{2.985407in}}%
\pgfpathlineto{\pgfqpoint{0.939556in}{2.987940in}}%
\pgfpathlineto{\pgfqpoint{0.940296in}{2.992929in}}%
\pgfpathlineto{\pgfqpoint{0.945905in}{2.995212in}}%
\pgfpathlineto{\pgfqpoint{0.953902in}{3.006708in}}%
\pgfpathlineto{\pgfqpoint{0.955231in}{3.014478in}}%
\pgfpathlineto{\pgfqpoint{0.959726in}{3.016515in}}%
\pgfpathlineto{\pgfqpoint{0.967041in}{3.016522in}}%
\pgfpathlineto{\pgfqpoint{0.974687in}{3.041738in}}%
\pgfpathlineto{\pgfqpoint{0.977683in}{3.043812in}}%
\pgfpathlineto{\pgfqpoint{0.974996in}{3.048741in}}%
\pgfpathlineto{\pgfqpoint{0.968708in}{3.043755in}}%
\pgfpathlineto{\pgfqpoint{0.950231in}{3.050477in}}%
\pgfpathlineto{\pgfqpoint{0.939203in}{3.061802in}}%
\pgfpathlineto{\pgfqpoint{0.944011in}{3.066969in}}%
\pgfpathlineto{\pgfqpoint{0.941824in}{3.074257in}}%
\pgfpathlineto{\pgfqpoint{0.943172in}{3.082867in}}%
\pgfpathlineto{\pgfqpoint{0.940800in}{3.092932in}}%
\pgfpathlineto{\pgfqpoint{0.940421in}{3.103009in}}%
\pgfpathlineto{\pgfqpoint{0.951035in}{3.101323in}}%
\pgfpathlineto{\pgfqpoint{0.953137in}{3.104293in}}%
\pgfpathlineto{\pgfqpoint{0.958416in}{3.104005in}}%
\pgfpathlineto{\pgfqpoint{0.964496in}{3.093258in}}%
\pgfpathlineto{\pgfqpoint{0.970359in}{3.085389in}}%
\pgfpathlineto{\pgfqpoint{0.965663in}{3.080268in}}%
\pgfpathlineto{\pgfqpoint{0.973783in}{3.078303in}}%
\pgfpathlineto{\pgfqpoint{0.975632in}{3.080815in}}%
\pgfpathlineto{\pgfqpoint{0.971290in}{3.091008in}}%
\pgfpathlineto{\pgfqpoint{0.962148in}{3.099455in}}%
\pgfpathlineto{\pgfqpoint{0.963266in}{3.102995in}}%
\pgfpathlineto{\pgfqpoint{0.955970in}{3.111941in}}%
\pgfpathlineto{\pgfqpoint{0.963421in}{3.113757in}}%
\pgfpathlineto{\pgfqpoint{0.956965in}{3.132911in}}%
\pgfpathlineto{\pgfqpoint{0.962214in}{3.135566in}}%
\pgfpathlineto{\pgfqpoint{0.963494in}{3.140158in}}%
\pgfpathlineto{\pgfqpoint{0.960079in}{3.145613in}}%
\pgfpathlineto{\pgfqpoint{0.967127in}{3.146979in}}%
\pgfpathlineto{\pgfqpoint{0.968391in}{3.151105in}}%
\pgfpathlineto{\pgfqpoint{0.976797in}{3.144912in}}%
\pgfpathlineto{\pgfqpoint{0.978664in}{3.151627in}}%
\pgfpathlineto{\pgfqpoint{0.983081in}{3.154581in}}%
\pgfpathlineto{\pgfqpoint{1.004574in}{3.158691in}}%
\pgfpathlineto{\pgfqpoint{1.011172in}{3.157238in}}%
\pgfpathlineto{\pgfqpoint{1.016700in}{3.165394in}}%
\pgfpathlineto{\pgfqpoint{1.029815in}{3.170744in}}%
\pgfpathlineto{\pgfqpoint{1.039917in}{3.168715in}}%
\pgfpathlineto{\pgfqpoint{1.044526in}{3.161612in}}%
\pgfpathlineto{\pgfqpoint{1.044284in}{3.150327in}}%
\pgfpathlineto{\pgfqpoint{1.049454in}{3.147769in}}%
\pgfpathlineto{\pgfqpoint{1.059999in}{3.148126in}}%
\pgfpathlineto{\pgfqpoint{1.073506in}{3.151062in}}%
\pgfpathlineto{\pgfqpoint{1.073340in}{3.147453in}}%
\pgfpathlineto{\pgfqpoint{1.090413in}{3.137164in}}%
\pgfpathlineto{\pgfqpoint{1.102354in}{3.139923in}}%
\pgfpathlineto{\pgfqpoint{1.105870in}{3.143494in}}%
\pgfpathlineto{\pgfqpoint{1.109634in}{3.152639in}}%
\pgfpathlineto{\pgfqpoint{1.114303in}{3.158496in}}%
\pgfpathlineto{\pgfqpoint{1.114719in}{3.167272in}}%
\pgfpathlineto{\pgfqpoint{1.110445in}{3.170724in}}%
\pgfpathlineto{\pgfqpoint{1.116774in}{3.173843in}}%
\pgfpathlineto{\pgfqpoint{1.127410in}{3.169156in}}%
\pgfpathlineto{\pgfqpoint{1.130874in}{3.172050in}}%
\pgfpathlineto{\pgfqpoint{1.131483in}{3.184466in}}%
\pgfpathlineto{\pgfqpoint{1.124609in}{3.181871in}}%
\pgfpathlineto{\pgfqpoint{1.119841in}{3.186174in}}%
\pgfpathlineto{\pgfqpoint{1.112102in}{3.188855in}}%
\pgfpathlineto{\pgfqpoint{1.099816in}{3.188408in}}%
\pgfpathlineto{\pgfqpoint{1.088046in}{3.192487in}}%
\pgfpathlineto{\pgfqpoint{1.087455in}{3.202649in}}%
\pgfpathlineto{\pgfqpoint{1.076928in}{3.213444in}}%
\pgfpathlineto{\pgfqpoint{1.063579in}{3.217975in}}%
\pgfpathlineto{\pgfqpoint{1.051926in}{3.238872in}}%
\pgfpathlineto{\pgfqpoint{1.052882in}{3.246605in}}%
\pgfpathlineto{\pgfqpoint{1.059156in}{3.250131in}}%
\pgfpathlineto{\pgfqpoint{1.058232in}{3.257452in}}%
\pgfpathlineto{\pgfqpoint{1.059689in}{3.264734in}}%
\pgfpathlineto{\pgfqpoint{1.064654in}{3.259749in}}%
\pgfpathlineto{\pgfqpoint{1.071209in}{3.261829in}}%
\pgfpathlineto{\pgfqpoint{1.070881in}{3.268154in}}%
\pgfpathlineto{\pgfqpoint{1.061792in}{3.280854in}}%
\pgfpathlineto{\pgfqpoint{1.059132in}{3.293954in}}%
\pgfpathlineto{\pgfqpoint{1.066739in}{3.294953in}}%
\pgfpathlineto{\pgfqpoint{1.078126in}{3.294149in}}%
\pgfpathlineto{\pgfqpoint{1.081711in}{3.289767in}}%
\pgfpathlineto{\pgfqpoint{1.091918in}{3.291335in}}%
\pgfpathlineto{\pgfqpoint{1.101406in}{3.289277in}}%
\pgfpathlineto{\pgfqpoint{1.107819in}{3.285658in}}%
\pgfpathlineto{\pgfqpoint{1.112258in}{3.287312in}}%
\pgfpathlineto{\pgfqpoint{1.124045in}{3.283696in}}%
\pgfpathlineto{\pgfqpoint{1.124153in}{3.280493in}}%
\pgfpathlineto{\pgfqpoint{1.133129in}{3.281575in}}%
\pgfpathlineto{\pgfqpoint{1.145165in}{3.273762in}}%
\pgfpathlineto{\pgfqpoint{1.144698in}{3.268577in}}%
\pgfpathlineto{\pgfqpoint{1.138971in}{3.266182in}}%
\pgfpathlineto{\pgfqpoint{1.132884in}{3.260488in}}%
\pgfpathlineto{\pgfqpoint{1.132148in}{3.252669in}}%
\pgfpathlineto{\pgfqpoint{1.145260in}{3.242696in}}%
\pgfpathlineto{\pgfqpoint{1.143163in}{3.237918in}}%
\pgfpathlineto{\pgfqpoint{1.152493in}{3.234247in}}%
\pgfpathlineto{\pgfqpoint{1.154784in}{3.228253in}}%
\pgfpathlineto{\pgfqpoint{1.166624in}{3.233016in}}%
\pgfpathlineto{\pgfqpoint{1.171839in}{3.226774in}}%
\pgfpathlineto{\pgfqpoint{1.174245in}{3.230552in}}%
\pgfpathlineto{\pgfqpoint{1.167159in}{3.242850in}}%
\pgfpathlineto{\pgfqpoint{1.169677in}{3.246534in}}%
\pgfpathlineto{\pgfqpoint{1.170793in}{3.256248in}}%
\pgfpathlineto{\pgfqpoint{1.167971in}{3.260522in}}%
\pgfpathlineto{\pgfqpoint{1.170088in}{3.266265in}}%
\pgfpathlineto{\pgfqpoint{1.177026in}{3.265725in}}%
\pgfpathlineto{\pgfqpoint{1.175590in}{3.255953in}}%
\pgfpathlineto{\pgfqpoint{1.171161in}{3.252693in}}%
\pgfpathlineto{\pgfqpoint{1.171475in}{3.239546in}}%
\pgfpathlineto{\pgfqpoint{1.179422in}{3.237975in}}%
\pgfpathlineto{\pgfqpoint{1.179519in}{3.228201in}}%
\pgfpathlineto{\pgfqpoint{1.188198in}{3.221859in}}%
\pgfpathlineto{\pgfqpoint{1.192188in}{3.225858in}}%
\pgfpathlineto{\pgfqpoint{1.187827in}{3.238028in}}%
\pgfpathlineto{\pgfqpoint{1.183318in}{3.241107in}}%
\pgfpathlineto{\pgfqpoint{1.178882in}{3.239081in}}%
\pgfpathlineto{\pgfqpoint{1.174978in}{3.241667in}}%
\pgfpathlineto{\pgfqpoint{1.175659in}{3.252913in}}%
\pgfpathlineto{\pgfqpoint{1.182173in}{3.257512in}}%
\pgfpathlineto{\pgfqpoint{1.188584in}{3.257089in}}%
\pgfpathlineto{\pgfqpoint{1.185928in}{3.263452in}}%
\pgfpathlineto{\pgfqpoint{1.176318in}{3.268326in}}%
\pgfpathlineto{\pgfqpoint{1.163592in}{3.287746in}}%
\pgfpathlineto{\pgfqpoint{1.169943in}{3.297060in}}%
\pgfpathlineto{\pgfqpoint{1.173478in}{3.305832in}}%
\pgfpathlineto{\pgfqpoint{1.173756in}{3.323448in}}%
\pgfpathlineto{\pgfqpoint{1.172395in}{3.338309in}}%
\pgfpathlineto{\pgfqpoint{1.168512in}{3.348067in}}%
\pgfpathlineto{\pgfqpoint{1.169286in}{3.357824in}}%
\pgfpathlineto{\pgfqpoint{1.179218in}{3.366286in}}%
\pgfpathlineto{\pgfqpoint{1.189095in}{3.376341in}}%
\pgfpathlineto{\pgfqpoint{1.213857in}{3.355796in}}%
\pgfpathlineto{\pgfqpoint{1.230047in}{3.354282in}}%
\pgfpathlineto{\pgfqpoint{1.243410in}{3.359398in}}%
\pgfpathlineto{\pgfqpoint{1.255140in}{3.366135in}}%
\pgfpathlineto{\pgfqpoint{1.257633in}{3.369344in}}%
\pgfpathlineto{\pgfqpoint{1.286589in}{3.376664in}}%
\pgfpathlineto{\pgfqpoint{1.289104in}{3.371302in}}%
\pgfpathlineto{\pgfqpoint{1.298944in}{3.366914in}}%
\pgfpathlineto{\pgfqpoint{1.317722in}{3.368261in}}%
\pgfpathlineto{\pgfqpoint{1.321152in}{3.367062in}}%
\pgfpathlineto{\pgfqpoint{1.329592in}{3.370264in}}%
\pgfpathlineto{\pgfqpoint{1.332985in}{3.364727in}}%
\pgfpathlineto{\pgfqpoint{1.341089in}{3.359668in}}%
\pgfpathlineto{\pgfqpoint{1.345143in}{3.360240in}}%
\pgfpathlineto{\pgfqpoint{1.352990in}{3.354162in}}%
\pgfpathlineto{\pgfqpoint{1.366250in}{3.355279in}}%
\pgfpathlineto{\pgfqpoint{1.379495in}{3.360200in}}%
\pgfpathlineto{\pgfqpoint{1.390311in}{3.344464in}}%
\pgfpathlineto{\pgfqpoint{1.389472in}{3.340932in}}%
\pgfpathlineto{\pgfqpoint{1.377537in}{3.336874in}}%
\pgfpathlineto{\pgfqpoint{1.380726in}{3.330737in}}%
\pgfpathlineto{\pgfqpoint{1.391713in}{3.337196in}}%
\pgfpathlineto{\pgfqpoint{1.396199in}{3.335631in}}%
\pgfpathlineto{\pgfqpoint{1.398107in}{3.327905in}}%
\pgfpathlineto{\pgfqpoint{1.391548in}{3.324638in}}%
\pgfpathlineto{\pgfqpoint{1.396233in}{3.314872in}}%
\pgfpathlineto{\pgfqpoint{1.403035in}{3.316522in}}%
\pgfpathlineto{\pgfqpoint{1.412907in}{3.311482in}}%
\pgfpathlineto{\pgfqpoint{1.422347in}{3.297492in}}%
\pgfpathlineto{\pgfqpoint{1.414040in}{3.293608in}}%
\pgfpathlineto{\pgfqpoint{1.421379in}{3.283300in}}%
\pgfpathlineto{\pgfqpoint{1.415196in}{3.280891in}}%
\pgfpathlineto{\pgfqpoint{1.422888in}{3.271101in}}%
\pgfpathlineto{\pgfqpoint{1.432268in}{3.272144in}}%
\pgfpathlineto{\pgfqpoint{1.437208in}{3.267739in}}%
\pgfpathlineto{\pgfqpoint{1.449701in}{3.261244in}}%
\pgfpathlineto{\pgfqpoint{1.465817in}{3.238189in}}%
\pgfpathlineto{\pgfqpoint{1.465453in}{3.232675in}}%
\pgfpathlineto{\pgfqpoint{1.489622in}{3.215013in}}%
\pgfpathlineto{\pgfqpoint{1.490143in}{3.208766in}}%
\pgfpathlineto{\pgfqpoint{1.496584in}{3.199419in}}%
\pgfpathlineto{\pgfqpoint{1.516372in}{3.194747in}}%
\pgfpathlineto{\pgfqpoint{1.523710in}{3.191424in}}%
\pgfpathlineto{\pgfqpoint{1.530615in}{3.181120in}}%
\pgfpathlineto{\pgfqpoint{1.531504in}{3.172094in}}%
\pgfpathlineto{\pgfqpoint{1.537437in}{3.164738in}}%
\pgfpathlineto{\pgfqpoint{1.539220in}{3.156441in}}%
\pgfpathlineto{\pgfqpoint{1.544780in}{3.153411in}}%
\pgfpathlineto{\pgfqpoint{1.517233in}{3.103693in}}%
\pgfpathlineto{\pgfqpoint{1.460054in}{3.000474in}}%
\pgfpathlineto{\pgfqpoint{1.376162in}{2.849003in}}%
\pgfpathlineto{\pgfqpoint{1.343317in}{2.789686in}}%
\pgfpathlineto{\pgfqpoint{1.350313in}{2.781720in}}%
\pgfpathlineto{\pgfqpoint{1.353395in}{2.784262in}}%
\pgfpathlineto{\pgfqpoint{1.359643in}{2.774988in}}%
\pgfpathlineto{\pgfqpoint{1.368318in}{2.777678in}}%
\pgfpathlineto{\pgfqpoint{1.379832in}{2.772122in}}%
\pgfpathlineto{\pgfqpoint{1.372374in}{2.763439in}}%
\pgfpathlineto{\pgfqpoint{1.378134in}{2.751722in}}%
\pgfpathlineto{\pgfqpoint{1.376943in}{2.746017in}}%
\pgfpathlineto{\pgfqpoint{1.386129in}{2.715933in}}%
\pgfpathlineto{\pgfqpoint{1.382242in}{2.702252in}}%
\pgfpathlineto{\pgfqpoint{1.399793in}{2.705640in}}%
\pgfpathlineto{\pgfqpoint{1.404102in}{2.703420in}}%
\pgfpathlineto{\pgfqpoint{1.408780in}{2.707029in}}%
\pgfpathlineto{\pgfqpoint{1.412108in}{2.713850in}}%
\pgfpathlineto{\pgfqpoint{1.416874in}{2.717760in}}%
\pgfpathlineto{\pgfqpoint{1.437213in}{2.717265in}}%
\pgfpathlineto{\pgfqpoint{1.441809in}{2.704137in}}%
\pgfpathlineto{\pgfqpoint{1.437865in}{2.692613in}}%
\pgfpathlineto{\pgfqpoint{1.442347in}{2.688811in}}%
\pgfpathlineto{\pgfqpoint{1.444703in}{2.682307in}}%
\pgfpathlineto{\pgfqpoint{1.444189in}{2.670075in}}%
\pgfpathlineto{\pgfqpoint{1.450169in}{2.661305in}}%
\pgfpathlineto{\pgfqpoint{1.453448in}{2.647561in}}%
\pgfpathlineto{\pgfqpoint{1.451765in}{2.640138in}}%
\pgfpathlineto{\pgfqpoint{1.453795in}{2.632614in}}%
\pgfpathlineto{\pgfqpoint{1.456722in}{2.589045in}}%
\pgfpathlineto{\pgfqpoint{1.452528in}{2.585613in}}%
\pgfpathlineto{\pgfqpoint{1.458062in}{2.580871in}}%
\pgfpathlineto{\pgfqpoint{1.453713in}{2.575103in}}%
\pgfpathlineto{\pgfqpoint{1.457635in}{2.570400in}}%
\pgfpathlineto{\pgfqpoint{1.455109in}{2.562374in}}%
\pgfpathlineto{\pgfqpoint{1.460758in}{2.560197in}}%
\pgfpathlineto{\pgfqpoint{1.468024in}{2.547989in}}%
\pgfpathlineto{\pgfqpoint{1.473424in}{2.544311in}}%
\pgfpathlineto{\pgfqpoint{1.474837in}{2.538933in}}%
\pgfpathlineto{\pgfqpoint{1.477965in}{2.536604in}}%
\pgfpathlineto{\pgfqpoint{1.477313in}{2.532139in}}%
\pgfpathlineto{\pgfqpoint{1.484056in}{2.528904in}}%
\pgfpathlineto{\pgfqpoint{1.482391in}{2.520313in}}%
\pgfpathlineto{\pgfqpoint{1.476587in}{2.515813in}}%
\pgfpathlineto{\pgfqpoint{1.473950in}{2.507451in}}%
\pgfpathlineto{\pgfqpoint{1.473250in}{2.496254in}}%
\pgfpathlineto{\pgfqpoint{1.470025in}{2.495183in}}%
\pgfpathlineto{\pgfqpoint{1.459486in}{2.485368in}}%
\pgfpathlineto{\pgfqpoint{1.450119in}{2.482777in}}%
\pgfpathlineto{\pgfqpoint{1.445648in}{2.486735in}}%
\pgfpathlineto{\pgfqpoint{1.449968in}{2.497240in}}%
\pgfpathlineto{\pgfqpoint{1.448648in}{2.502397in}}%
\pgfpathlineto{\pgfqpoint{1.454837in}{2.505010in}}%
\pgfpathlineto{\pgfqpoint{1.460625in}{2.520208in}}%
\pgfpathlineto{\pgfqpoint{1.458180in}{2.533190in}}%
\pgfpathlineto{\pgfqpoint{1.454269in}{2.536255in}}%
\pgfpathlineto{\pgfqpoint{1.441186in}{2.535262in}}%
\pgfpathlineto{\pgfqpoint{1.443823in}{2.531871in}}%
\pgfpathlineto{\pgfqpoint{1.434354in}{2.522137in}}%
\pgfpathlineto{\pgfqpoint{1.431303in}{2.527549in}}%
\pgfpathlineto{\pgfqpoint{1.432097in}{2.534620in}}%
\pgfpathlineto{\pgfqpoint{1.437533in}{2.534490in}}%
\pgfpathlineto{\pgfqpoint{1.442372in}{2.539684in}}%
\pgfpathlineto{\pgfqpoint{1.445236in}{2.547269in}}%
\pgfpathlineto{\pgfqpoint{1.456026in}{2.545128in}}%
\pgfpathlineto{\pgfqpoint{1.447388in}{2.549180in}}%
\pgfpathlineto{\pgfqpoint{1.448148in}{2.554607in}}%
\pgfpathlineto{\pgfqpoint{1.444553in}{2.557058in}}%
\pgfpathlineto{\pgfqpoint{1.443082in}{2.564452in}}%
\pgfpathlineto{\pgfqpoint{1.445633in}{2.568303in}}%
\pgfpathlineto{\pgfqpoint{1.439548in}{2.580265in}}%
\pgfpathlineto{\pgfqpoint{1.440061in}{2.587875in}}%
\pgfpathlineto{\pgfqpoint{1.432930in}{2.594651in}}%
\pgfpathlineto{\pgfqpoint{1.429073in}{2.600133in}}%
\pgfpathlineto{\pgfqpoint{1.434400in}{2.605358in}}%
\pgfpathlineto{\pgfqpoint{1.436374in}{2.613167in}}%
\pgfpathlineto{\pgfqpoint{1.434612in}{2.618294in}}%
\pgfpathlineto{\pgfqpoint{1.436673in}{2.624409in}}%
\pgfpathlineto{\pgfqpoint{1.434046in}{2.644346in}}%
\pgfpathlineto{\pgfqpoint{1.430119in}{2.653391in}}%
\pgfpathlineto{\pgfqpoint{1.425627in}{2.656790in}}%
\pgfpathlineto{\pgfqpoint{1.427539in}{2.668556in}}%
\pgfpathlineto{\pgfqpoint{1.426190in}{2.677745in}}%
\pgfpathlineto{\pgfqpoint{1.428633in}{2.683640in}}%
\pgfpathlineto{\pgfqpoint{1.423578in}{2.684465in}}%
\pgfpathlineto{\pgfqpoint{1.422223in}{2.670149in}}%
\pgfpathlineto{\pgfqpoint{1.416543in}{2.654178in}}%
\pgfpathlineto{\pgfqpoint{1.411982in}{2.657815in}}%
\pgfpathlineto{\pgfqpoint{1.411372in}{2.663064in}}%
\pgfpathlineto{\pgfqpoint{1.405792in}{2.670234in}}%
\pgfpathlineto{\pgfqpoint{1.408627in}{2.674929in}}%
\pgfpathlineto{\pgfqpoint{1.407746in}{2.685497in}}%
\pgfpathlineto{\pgfqpoint{1.400524in}{2.681780in}}%
\pgfpathlineto{\pgfqpoint{1.401941in}{2.676048in}}%
\pgfpathlineto{\pgfqpoint{1.400581in}{2.668771in}}%
\pgfpathlineto{\pgfqpoint{1.392474in}{2.668784in}}%
\pgfpathlineto{\pgfqpoint{1.388840in}{2.673295in}}%
\pgfpathlineto{\pgfqpoint{1.383356in}{2.674156in}}%
\pgfpathlineto{\pgfqpoint{1.379665in}{2.679012in}}%
\pgfpathlineto{\pgfqpoint{1.373315in}{2.692347in}}%
\pgfpathlineto{\pgfqpoint{1.371257in}{2.702701in}}%
\pgfpathlineto{\pgfqpoint{1.372574in}{2.707102in}}%
\pgfpathlineto{\pgfqpoint{1.369501in}{2.716794in}}%
\pgfpathlineto{\pgfqpoint{1.364408in}{2.722405in}}%
\pgfpathlineto{\pgfqpoint{1.355600in}{2.736743in}}%
\pgfpathlineto{\pgfqpoint{1.349211in}{2.749198in}}%
\pgfpathlineto{\pgfqpoint{1.355730in}{2.749385in}}%
\pgfpathlineto{\pgfqpoint{1.359566in}{2.752000in}}%
\pgfpathlineto{\pgfqpoint{1.360320in}{2.760094in}}%
\pgfpathlineto{\pgfqpoint{1.341875in}{2.761071in}}%
\pgfpathlineto{\pgfqpoint{1.326187in}{2.778596in}}%
\pgfpathlineto{\pgfqpoint{1.330312in}{2.783205in}}%
\pgfpathlineto{\pgfqpoint{1.322726in}{2.783992in}}%
\pgfpathlineto{\pgfqpoint{1.307659in}{2.800285in}}%
\pgfpathlineto{\pgfqpoint{1.283789in}{2.808454in}}%
\pgfpathlineto{\pgfqpoint{1.280575in}{2.818029in}}%
\pgfpathlineto{\pgfqpoint{1.275622in}{2.823016in}}%
\pgfpathlineto{\pgfqpoint{1.275397in}{2.827699in}}%
\pgfpathlineto{\pgfqpoint{1.271671in}{2.831176in}}%
\pgfpathlineto{\pgfqpoint{1.276561in}{2.836988in}}%
\pgfpathlineto{\pgfqpoint{1.265409in}{2.837549in}}%
\pgfpathlineto{\pgfqpoint{1.262574in}{2.845070in}}%
\pgfpathlineto{\pgfqpoint{1.257499in}{2.848272in}}%
\pgfpathlineto{\pgfqpoint{1.267663in}{2.852137in}}%
\pgfpathlineto{\pgfqpoint{1.262377in}{2.859031in}}%
\pgfpathlineto{\pgfqpoint{1.252649in}{2.861924in}}%
\pgfpathlineto{\pgfqpoint{1.254162in}{2.874658in}}%
\pgfpathlineto{\pgfqpoint{1.251373in}{2.879166in}}%
\pgfpathlineto{\pgfqpoint{1.246350in}{2.882341in}}%
\pgfpathlineto{\pgfqpoint{1.242055in}{2.879156in}}%
\pgfpathlineto{\pgfqpoint{1.239517in}{2.882151in}}%
\pgfpathlineto{\pgfqpoint{1.232116in}{2.883009in}}%
\pgfpathlineto{\pgfqpoint{1.232712in}{2.894564in}}%
\pgfpathlineto{\pgfqpoint{1.221982in}{2.886760in}}%
\pgfpathlineto{\pgfqpoint{1.222208in}{2.870676in}}%
\pgfpathlineto{\pgfqpoint{1.210691in}{2.867592in}}%
\pgfpathlineto{\pgfqpoint{1.203169in}{2.861169in}}%
\pgfpathlineto{\pgfqpoint{1.197123in}{2.859816in}}%
\pgfpathlineto{\pgfqpoint{1.190581in}{2.866289in}}%
\pgfpathlineto{\pgfqpoint{1.184755in}{2.865431in}}%
\pgfpathlineto{\pgfqpoint{1.185835in}{2.871344in}}%
\pgfpathlineto{\pgfqpoint{1.177554in}{2.868160in}}%
\pgfpathlineto{\pgfqpoint{1.169881in}{2.867802in}}%
\pgfpathlineto{\pgfqpoint{1.160900in}{2.864574in}}%
\pgfpathlineto{\pgfqpoint{1.142675in}{2.866391in}}%
\pgfpathlineto{\pgfqpoint{1.136823in}{2.865196in}}%
\pgfpathlineto{\pgfqpoint{1.132601in}{2.870513in}}%
\pgfpathlineto{\pgfqpoint{1.124808in}{2.871044in}}%
\pgfpathlineto{\pgfqpoint{1.123187in}{2.877934in}}%
\pgfpathlineto{\pgfqpoint{1.127832in}{2.881688in}}%
\pgfpathlineto{\pgfqpoint{1.133281in}{2.881335in}}%
\pgfpathlineto{\pgfqpoint{1.137829in}{2.876217in}}%
\pgfpathlineto{\pgfqpoint{1.140824in}{2.884078in}}%
\pgfpathlineto{\pgfqpoint{1.136645in}{2.892787in}}%
\pgfpathlineto{\pgfqpoint{1.146104in}{2.900453in}}%
\pgfpathlineto{\pgfqpoint{1.155583in}{2.903278in}}%
\pgfpathlineto{\pgfqpoint{1.162183in}{2.907887in}}%
\pgfpathlineto{\pgfqpoint{1.166457in}{2.912851in}}%
\pgfpathlineto{\pgfqpoint{1.168843in}{2.920989in}}%
\pgfpathlineto{\pgfqpoint{1.176380in}{2.918774in}}%
\pgfpathlineto{\pgfqpoint{1.192721in}{2.919912in}}%
\pgfpathlineto{\pgfqpoint{1.196229in}{2.910989in}}%
\pgfpathlineto{\pgfqpoint{1.201960in}{2.910600in}}%
\pgfpathlineto{\pgfqpoint{1.212392in}{2.896942in}}%
\pgfpathlineto{\pgfqpoint{1.212411in}{2.901921in}}%
\pgfpathlineto{\pgfqpoint{1.208767in}{2.903540in}}%
\pgfpathlineto{\pgfqpoint{1.202098in}{2.919845in}}%
\pgfpathlineto{\pgfqpoint{1.205343in}{2.922060in}}%
\pgfpathlineto{\pgfqpoint{1.196436in}{2.930567in}}%
\pgfpathlineto{\pgfqpoint{1.188475in}{2.932223in}}%
\pgfpathlineto{\pgfqpoint{1.180890in}{2.929282in}}%
\pgfpathlineto{\pgfqpoint{1.167898in}{2.931573in}}%
\pgfpathlineto{\pgfqpoint{1.161737in}{2.927452in}}%
\pgfpathlineto{\pgfqpoint{1.147708in}{2.924268in}}%
\pgfpathlineto{\pgfqpoint{1.146425in}{2.919686in}}%
\pgfpathlineto{\pgfqpoint{1.135646in}{2.918437in}}%
\pgfpathlineto{\pgfqpoint{1.132880in}{2.913199in}}%
\pgfpathlineto{\pgfqpoint{1.126714in}{2.909361in}}%
\pgfpathlineto{\pgfqpoint{1.119266in}{2.910021in}}%
\pgfpathlineto{\pgfqpoint{1.115548in}{2.906310in}}%
\pgfpathlineto{\pgfqpoint{1.110856in}{2.906485in}}%
\pgfpathlineto{\pgfqpoint{1.106372in}{2.909824in}}%
\pgfpathlineto{\pgfqpoint{1.098533in}{2.909228in}}%
\pgfpathlineto{\pgfqpoint{1.096687in}{2.906150in}}%
\pgfpathlineto{\pgfqpoint{1.088453in}{2.909309in}}%
\pgfpathlineto{\pgfqpoint{1.081462in}{2.901155in}}%
\pgfpathlineto{\pgfqpoint{1.088166in}{2.893265in}}%
\pgfpathlineto{\pgfqpoint{1.090745in}{2.886777in}}%
\pgfpathlineto{\pgfqpoint{1.088367in}{2.881281in}}%
\pgfpathlineto{\pgfqpoint{1.078842in}{2.877488in}}%
\pgfpathlineto{\pgfqpoint{1.073324in}{2.880575in}}%
\pgfpathlineto{\pgfqpoint{1.066378in}{2.878904in}}%
\pgfpathlineto{\pgfqpoint{1.065541in}{2.874087in}}%
\pgfpathlineto{\pgfqpoint{1.057086in}{2.875791in}}%
\pgfpathlineto{\pgfqpoint{1.058958in}{2.870978in}}%
\pgfpathlineto{\pgfqpoint{1.046487in}{2.869963in}}%
\pgfpathlineto{\pgfqpoint{1.038414in}{2.876072in}}%
\pgfpathlineto{\pgfqpoint{1.034102in}{2.872187in}}%
\pgfpathlineto{\pgfqpoint{1.022628in}{2.876323in}}%
\pgfpathlineto{\pgfqpoint{1.019770in}{2.872134in}}%
\pgfpathlineto{\pgfqpoint{1.014082in}{2.870270in}}%
\pgfpathlineto{\pgfqpoint{1.009518in}{2.875182in}}%
\pgfpathlineto{\pgfqpoint{0.993915in}{2.873961in}}%
\pgfpathlineto{\pgfqpoint{0.993931in}{2.866717in}}%
\pgfpathlineto{\pgfqpoint{0.987663in}{2.864800in}}%
\pgfpathlineto{\pgfqpoint{0.982786in}{2.868523in}}%
\pgfpathlineto{\pgfqpoint{0.976340in}{2.866454in}}%
\pgfpathlineto{\pgfqpoint{0.967998in}{2.865961in}}%
\pgfpathlineto{\pgfqpoint{0.955569in}{2.871199in}}%
\pgfpathlineto{\pgfqpoint{0.948453in}{2.865798in}}%
\pgfpathlineto{\pgfqpoint{0.938273in}{2.873016in}}%
\pgfpathlineto{\pgfqpoint{0.934945in}{2.870281in}}%
\pgfpathlineto{\pgfqpoint{0.932608in}{2.863211in}}%
\pgfpathlineto{\pgfqpoint{0.926065in}{2.859066in}}%
\pgfpathlineto{\pgfqpoint{0.915395in}{2.863811in}}%
\pgfpathlineto{\pgfqpoint{0.897718in}{2.875532in}}%
\pgfpathlineto{\pgfqpoint{0.892566in}{2.876574in}}%
\pgfpathlineto{\pgfqpoint{0.889659in}{2.874131in}}%
\pgfpathlineto{\pgfqpoint{0.881884in}{2.875792in}}%
\pgfpathlineto{\pgfqpoint{0.878091in}{2.879440in}}%
\pgfpathlineto{\pgfqpoint{0.870326in}{2.881406in}}%
\pgfpathlineto{\pgfqpoint{0.859720in}{2.881520in}}%
\pgfpathlineto{\pgfqpoint{0.855509in}{2.885871in}}%
\pgfpathlineto{\pgfqpoint{0.863579in}{2.890374in}}%
\pgfpathlineto{\pgfqpoint{0.861696in}{2.894819in}}%
\pgfpathlineto{\pgfqpoint{0.856152in}{2.893821in}}%
\pgfpathlineto{\pgfqpoint{0.853382in}{2.890327in}}%
\pgfpathlineto{\pgfqpoint{0.840401in}{2.885316in}}%
\pgfpathlineto{\pgfqpoint{0.833837in}{2.887021in}}%
\pgfpathlineto{\pgfqpoint{0.831075in}{2.897193in}}%
\pgfpathlineto{\pgfqpoint{0.822887in}{2.892819in}}%
\pgfpathlineto{\pgfqpoint{0.820507in}{2.905712in}}%
\pgfpathlineto{\pgfqpoint{0.816627in}{2.902389in}}%
\pgfpathlineto{\pgfqpoint{0.817276in}{2.897428in}}%
\pgfpathlineto{\pgfqpoint{0.808354in}{2.898738in}}%
\pgfpathlineto{\pgfqpoint{0.817635in}{2.907383in}}%
\pgfpathlineto{\pgfqpoint{0.827041in}{2.902246in}}%
\pgfpathlineto{\pgfqpoint{0.829057in}{2.904825in}}%
\pgfpathlineto{\pgfqpoint{0.838952in}{2.901734in}}%
\pgfpathlineto{\pgfqpoint{0.838334in}{2.905253in}}%
\pgfpathlineto{\pgfqpoint{0.865329in}{2.906418in}}%
\pgfpathlineto{\pgfqpoint{0.878499in}{2.900929in}}%
\pgfpathlineto{\pgfqpoint{0.883595in}{2.895513in}}%
\pgfpathlineto{\pgfqpoint{0.882083in}{2.890785in}}%
\pgfpathlineto{\pgfqpoint{0.887200in}{2.888996in}}%
\pgfpathlineto{\pgfqpoint{0.899915in}{2.896458in}}%
\pgfpathlineto{\pgfqpoint{0.916364in}{2.896936in}}%
\pgfpathlineto{\pgfqpoint{0.931303in}{2.892438in}}%
\pgfpathlineto{\pgfqpoint{0.939764in}{2.893193in}}%
\pgfpathlineto{\pgfqpoint{0.942916in}{2.886215in}}%
\pgfpathlineto{\pgfqpoint{0.947648in}{2.893908in}}%
\pgfpathlineto{\pgfqpoint{0.950558in}{2.895574in}}%
\pgfpathlineto{\pgfqpoint{0.964023in}{2.898918in}}%
\pgfpathlineto{\pgfqpoint{0.969028in}{2.896874in}}%
\pgfpathlineto{\pgfqpoint{0.974439in}{2.899301in}}%
\pgfpathlineto{\pgfqpoint{0.980895in}{2.898032in}}%
\pgfpathlineto{\pgfqpoint{0.982418in}{2.900859in}}%
\pgfpathlineto{\pgfqpoint{0.995888in}{2.913831in}}%
\pgfpathlineto{\pgfqpoint{1.001592in}{2.915755in}}%
\pgfpathlineto{\pgfqpoint{1.004963in}{2.922756in}}%
\pgfpathlineto{\pgfqpoint{1.022854in}{2.928179in}}%
\pgfpathlineto{\pgfqpoint{1.025094in}{2.932315in}}%
\pgfpathlineto{\pgfqpoint{0.999738in}{2.938698in}}%
\pgfpathlineto{\pgfqpoint{1.000735in}{2.945564in}}%
\pgfpathlineto{\pgfqpoint{0.998128in}{2.954632in}}%
\pgfpathlineto{\pgfqpoint{0.992224in}{2.953878in}}%
\pgfpathlineto{\pgfqpoint{0.988207in}{2.942111in}}%
\pgfpathlineto{\pgfqpoint{0.983545in}{2.941141in}}%
\pgfpathlineto{\pgfqpoint{0.980510in}{2.944950in}}%
\pgfpathlineto{\pgfqpoint{0.984050in}{2.961614in}}%
\pgfpathlineto{\pgfqpoint{0.980525in}{2.968597in}}%
\pgfpathlineto{\pgfqpoint{0.973696in}{2.976829in}}%
\pgfpathlineto{\pgfqpoint{0.976978in}{2.983110in}}%
\pgfpathlineto{\pgfqpoint{0.962567in}{2.983336in}}%
\pgfpathlineto{\pgfqpoint{0.962896in}{2.985773in}}%
\pgfpathlineto{\pgfqpoint{0.954297in}{2.989187in}}%
\pgfpathclose%
\pgfusepath{stroke,fill}%
\end{pgfscope}%
\begin{pgfscope}%
\pgfpathrectangle{\pgfqpoint{0.100000in}{2.413063in}}{\pgfqpoint{5.037500in}{3.427208in}}%
\pgfusepath{clip}%
\pgfsetbuttcap%
\pgfsetmiterjoin%
\definecolor{currentfill}{rgb}{1.000000,1.000000,1.000000}%
\pgfsetfillcolor{currentfill}%
\pgfsetlinewidth{0.501875pt}%
\definecolor{currentstroke}{rgb}{0.827451,0.827451,0.827451}%
\pgfsetstrokecolor{currentstroke}%
\pgfsetdash{}{0pt}%
\pgfpathmoveto{\pgfqpoint{0.925909in}{3.108677in}}%
\pgfpathlineto{\pgfqpoint{0.924127in}{3.105546in}}%
\pgfpathlineto{\pgfqpoint{0.928841in}{3.098878in}}%
\pgfpathlineto{\pgfqpoint{0.921389in}{3.092640in}}%
\pgfpathlineto{\pgfqpoint{0.921597in}{3.087275in}}%
\pgfpathlineto{\pgfqpoint{0.918227in}{3.086009in}}%
\pgfpathlineto{\pgfqpoint{0.906646in}{3.094010in}}%
\pgfpathlineto{\pgfqpoint{0.898390in}{3.111524in}}%
\pgfpathlineto{\pgfqpoint{0.897838in}{3.116585in}}%
\pgfpathlineto{\pgfqpoint{0.899960in}{3.122642in}}%
\pgfpathlineto{\pgfqpoint{0.908905in}{3.113376in}}%
\pgfpathlineto{\pgfqpoint{0.911642in}{3.115164in}}%
\pgfpathlineto{\pgfqpoint{0.919230in}{3.113463in}}%
\pgfpathclose%
\pgfusepath{stroke,fill}%
\end{pgfscope}%
\begin{pgfscope}%
\pgfpathrectangle{\pgfqpoint{0.100000in}{2.413063in}}{\pgfqpoint{5.037500in}{3.427208in}}%
\pgfusepath{clip}%
\pgfsetbuttcap%
\pgfsetmiterjoin%
\definecolor{currentfill}{rgb}{1.000000,1.000000,1.000000}%
\pgfsetfillcolor{currentfill}%
\pgfsetlinewidth{0.501875pt}%
\definecolor{currentstroke}{rgb}{0.827451,0.827451,0.827451}%
\pgfsetstrokecolor{currentstroke}%
\pgfsetdash{}{0pt}%
\pgfpathmoveto{\pgfqpoint{0.788555in}{2.905777in}}%
\pgfpathlineto{\pgfqpoint{0.780416in}{2.904455in}}%
\pgfpathlineto{\pgfqpoint{0.774525in}{2.907178in}}%
\pgfpathlineto{\pgfqpoint{0.771950in}{2.911611in}}%
\pgfpathlineto{\pgfqpoint{0.774868in}{2.917970in}}%
\pgfpathlineto{\pgfqpoint{0.781368in}{2.916288in}}%
\pgfpathlineto{\pgfqpoint{0.792051in}{2.920061in}}%
\pgfpathlineto{\pgfqpoint{0.795793in}{2.914849in}}%
\pgfpathlineto{\pgfqpoint{0.799421in}{2.915812in}}%
\pgfpathlineto{\pgfqpoint{0.808203in}{2.912569in}}%
\pgfpathlineto{\pgfqpoint{0.811998in}{2.908449in}}%
\pgfpathlineto{\pgfqpoint{0.807367in}{2.897674in}}%
\pgfpathlineto{\pgfqpoint{0.802761in}{2.895390in}}%
\pgfpathlineto{\pgfqpoint{0.798621in}{2.896621in}}%
\pgfpathclose%
\pgfusepath{stroke,fill}%
\end{pgfscope}%
\begin{pgfscope}%
\pgfpathrectangle{\pgfqpoint{0.100000in}{2.413063in}}{\pgfqpoint{5.037500in}{3.427208in}}%
\pgfusepath{clip}%
\pgfsetbuttcap%
\pgfsetmiterjoin%
\definecolor{currentfill}{rgb}{1.000000,1.000000,1.000000}%
\pgfsetfillcolor{currentfill}%
\pgfsetlinewidth{0.501875pt}%
\definecolor{currentstroke}{rgb}{0.827451,0.827451,0.827451}%
\pgfsetstrokecolor{currentstroke}%
\pgfsetdash{}{0pt}%
\pgfpathmoveto{\pgfqpoint{0.715775in}{2.915567in}}%
\pgfpathlineto{\pgfqpoint{0.704241in}{2.921069in}}%
\pgfpathlineto{\pgfqpoint{0.693568in}{2.922705in}}%
\pgfpathlineto{\pgfqpoint{0.690204in}{2.925048in}}%
\pgfpathlineto{\pgfqpoint{0.694035in}{2.929660in}}%
\pgfpathlineto{\pgfqpoint{0.700833in}{2.923899in}}%
\pgfpathlineto{\pgfqpoint{0.706535in}{2.924693in}}%
\pgfpathlineto{\pgfqpoint{0.715807in}{2.928683in}}%
\pgfpathlineto{\pgfqpoint{0.716557in}{2.933822in}}%
\pgfpathlineto{\pgfqpoint{0.727946in}{2.931278in}}%
\pgfpathlineto{\pgfqpoint{0.725279in}{2.923541in}}%
\pgfpathlineto{\pgfqpoint{0.732535in}{2.922975in}}%
\pgfpathlineto{\pgfqpoint{0.726447in}{2.913772in}}%
\pgfpathlineto{\pgfqpoint{0.720552in}{2.916755in}}%
\pgfpathclose%
\pgfusepath{stroke,fill}%
\end{pgfscope}%
\begin{pgfscope}%
\pgfpathrectangle{\pgfqpoint{0.100000in}{2.413063in}}{\pgfqpoint{5.037500in}{3.427208in}}%
\pgfusepath{clip}%
\pgfsetbuttcap%
\pgfsetmiterjoin%
\definecolor{currentfill}{rgb}{1.000000,1.000000,1.000000}%
\pgfsetfillcolor{currentfill}%
\pgfsetlinewidth{0.501875pt}%
\definecolor{currentstroke}{rgb}{0.827451,0.827451,0.827451}%
\pgfsetstrokecolor{currentstroke}%
\pgfsetdash{}{0pt}%
\pgfpathmoveto{\pgfqpoint{0.695511in}{2.934587in}}%
\pgfpathlineto{\pgfqpoint{0.691272in}{2.932282in}}%
\pgfpathlineto{\pgfqpoint{0.679721in}{2.935825in}}%
\pgfpathlineto{\pgfqpoint{0.671090in}{2.933773in}}%
\pgfpathlineto{\pgfqpoint{0.666705in}{2.939602in}}%
\pgfpathlineto{\pgfqpoint{0.668618in}{2.942271in}}%
\pgfpathlineto{\pgfqpoint{0.675159in}{2.942717in}}%
\pgfpathlineto{\pgfqpoint{0.680162in}{2.941042in}}%
\pgfpathlineto{\pgfqpoint{0.685614in}{2.943693in}}%
\pgfpathlineto{\pgfqpoint{0.693972in}{2.940147in}}%
\pgfpathclose%
\pgfusepath{stroke,fill}%
\end{pgfscope}%
\begin{pgfscope}%
\pgfpathrectangle{\pgfqpoint{0.100000in}{2.413063in}}{\pgfqpoint{5.037500in}{3.427208in}}%
\pgfusepath{clip}%
\pgfsetbuttcap%
\pgfsetmiterjoin%
\definecolor{currentfill}{rgb}{1.000000,1.000000,1.000000}%
\pgfsetfillcolor{currentfill}%
\pgfsetlinewidth{0.501875pt}%
\definecolor{currentstroke}{rgb}{0.827451,0.827451,0.827451}%
\pgfsetstrokecolor{currentstroke}%
\pgfsetdash{}{0pt}%
\pgfpathmoveto{\pgfqpoint{0.559354in}{3.020376in}}%
\pgfpathlineto{\pgfqpoint{0.559047in}{3.014660in}}%
\pgfpathlineto{\pgfqpoint{0.550615in}{3.011068in}}%
\pgfpathlineto{\pgfqpoint{0.542915in}{3.020060in}}%
\pgfpathlineto{\pgfqpoint{0.551550in}{3.023056in}}%
\pgfpathclose%
\pgfusepath{stroke,fill}%
\end{pgfscope}%
\begin{pgfscope}%
\pgfpathrectangle{\pgfqpoint{0.100000in}{2.413063in}}{\pgfqpoint{5.037500in}{3.427208in}}%
\pgfusepath{clip}%
\pgfsetbuttcap%
\pgfsetmiterjoin%
\definecolor{currentfill}{rgb}{1.000000,1.000000,1.000000}%
\pgfsetfillcolor{currentfill}%
\pgfsetlinewidth{0.501875pt}%
\definecolor{currentstroke}{rgb}{0.827451,0.827451,0.827451}%
\pgfsetstrokecolor{currentstroke}%
\pgfsetdash{}{0pt}%
\pgfpathmoveto{\pgfqpoint{0.489807in}{3.052507in}}%
\pgfpathlineto{\pgfqpoint{0.491597in}{3.054992in}}%
\pgfpathlineto{\pgfqpoint{0.501941in}{3.054564in}}%
\pgfpathlineto{\pgfqpoint{0.503755in}{3.058261in}}%
\pgfpathlineto{\pgfqpoint{0.508293in}{3.055035in}}%
\pgfpathlineto{\pgfqpoint{0.505673in}{3.048828in}}%
\pgfpathlineto{\pgfqpoint{0.501818in}{3.048302in}}%
\pgfpathlineto{\pgfqpoint{0.492605in}{3.050250in}}%
\pgfpathclose%
\pgfusepath{stroke,fill}%
\end{pgfscope}%
\begin{pgfscope}%
\pgfpathrectangle{\pgfqpoint{0.100000in}{2.413063in}}{\pgfqpoint{5.037500in}{3.427208in}}%
\pgfusepath{clip}%
\pgfsetbuttcap%
\pgfsetmiterjoin%
\definecolor{currentfill}{rgb}{1.000000,1.000000,1.000000}%
\pgfsetfillcolor{currentfill}%
\pgfsetlinewidth{0.501875pt}%
\definecolor{currentstroke}{rgb}{0.827451,0.827451,0.827451}%
\pgfsetstrokecolor{currentstroke}%
\pgfsetdash{}{0pt}%
\pgfpathmoveto{\pgfqpoint{0.475139in}{3.071744in}}%
\pgfpathlineto{\pgfqpoint{0.474512in}{3.077342in}}%
\pgfpathlineto{\pgfqpoint{0.479908in}{3.076928in}}%
\pgfpathlineto{\pgfqpoint{0.479533in}{3.084790in}}%
\pgfpathlineto{\pgfqpoint{0.484961in}{3.080547in}}%
\pgfpathlineto{\pgfqpoint{0.483083in}{3.074414in}}%
\pgfpathclose%
\pgfusepath{stroke,fill}%
\end{pgfscope}%
\begin{pgfscope}%
\pgfpathrectangle{\pgfqpoint{0.100000in}{2.413063in}}{\pgfqpoint{5.037500in}{3.427208in}}%
\pgfusepath{clip}%
\pgfsetbuttcap%
\pgfsetmiterjoin%
\definecolor{currentfill}{rgb}{1.000000,1.000000,1.000000}%
\pgfsetfillcolor{currentfill}%
\pgfsetlinewidth{0.501875pt}%
\definecolor{currentstroke}{rgb}{0.827451,0.827451,0.827451}%
\pgfsetstrokecolor{currentstroke}%
\pgfsetdash{}{0pt}%
\pgfpathmoveto{\pgfqpoint{0.950612in}{3.271270in}}%
\pgfpathlineto{\pgfqpoint{0.942005in}{3.273128in}}%
\pgfpathlineto{\pgfqpoint{0.940155in}{3.278880in}}%
\pgfpathlineto{\pgfqpoint{0.943114in}{3.284630in}}%
\pgfpathlineto{\pgfqpoint{0.949739in}{3.287633in}}%
\pgfpathlineto{\pgfqpoint{0.957325in}{3.272874in}}%
\pgfpathlineto{\pgfqpoint{0.964205in}{3.272219in}}%
\pgfpathlineto{\pgfqpoint{0.969338in}{3.267664in}}%
\pgfpathlineto{\pgfqpoint{0.969238in}{3.261398in}}%
\pgfpathlineto{\pgfqpoint{0.965484in}{3.258394in}}%
\pgfpathlineto{\pgfqpoint{0.971588in}{3.239861in}}%
\pgfpathlineto{\pgfqpoint{0.978267in}{3.233060in}}%
\pgfpathlineto{\pgfqpoint{0.971438in}{3.230902in}}%
\pgfpathlineto{\pgfqpoint{0.965912in}{3.238533in}}%
\pgfpathlineto{\pgfqpoint{0.953900in}{3.236170in}}%
\pgfpathlineto{\pgfqpoint{0.957635in}{3.243721in}}%
\pgfpathlineto{\pgfqpoint{0.955896in}{3.246729in}}%
\pgfpathlineto{\pgfqpoint{0.956916in}{3.254600in}}%
\pgfpathlineto{\pgfqpoint{0.954536in}{3.268829in}}%
\pgfpathclose%
\pgfusepath{stroke,fill}%
\end{pgfscope}%
\begin{pgfscope}%
\pgfpathrectangle{\pgfqpoint{0.100000in}{2.413063in}}{\pgfqpoint{5.037500in}{3.427208in}}%
\pgfusepath{clip}%
\pgfsetbuttcap%
\pgfsetmiterjoin%
\definecolor{currentfill}{rgb}{1.000000,1.000000,1.000000}%
\pgfsetfillcolor{currentfill}%
\pgfsetlinewidth{0.501875pt}%
\definecolor{currentstroke}{rgb}{0.827451,0.827451,0.827451}%
\pgfsetstrokecolor{currentstroke}%
\pgfsetdash{}{0pt}%
\pgfpathmoveto{\pgfqpoint{1.253018in}{2.847315in}}%
\pgfpathlineto{\pgfqpoint{1.240541in}{2.847455in}}%
\pgfpathlineto{\pgfqpoint{1.241502in}{2.853719in}}%
\pgfpathlineto{\pgfqpoint{1.246496in}{2.855959in}}%
\pgfpathclose%
\pgfusepath{stroke,fill}%
\end{pgfscope}%
\begin{pgfscope}%
\pgfpathrectangle{\pgfqpoint{0.100000in}{2.413063in}}{\pgfqpoint{5.037500in}{3.427208in}}%
\pgfusepath{clip}%
\pgfsetbuttcap%
\pgfsetmiterjoin%
\definecolor{currentfill}{rgb}{1.000000,1.000000,1.000000}%
\pgfsetfillcolor{currentfill}%
\pgfsetlinewidth{0.501875pt}%
\definecolor{currentstroke}{rgb}{0.827451,0.827451,0.827451}%
\pgfsetstrokecolor{currentstroke}%
\pgfsetdash{}{0pt}%
\pgfpathmoveto{\pgfqpoint{1.235704in}{2.852987in}}%
\pgfpathlineto{\pgfqpoint{1.226764in}{2.850825in}}%
\pgfpathlineto{\pgfqpoint{1.218270in}{2.847120in}}%
\pgfpathlineto{\pgfqpoint{1.215647in}{2.851485in}}%
\pgfpathlineto{\pgfqpoint{1.230255in}{2.854439in}}%
\pgfpathclose%
\pgfusepath{stroke,fill}%
\end{pgfscope}%
\begin{pgfscope}%
\pgfpathrectangle{\pgfqpoint{0.100000in}{2.413063in}}{\pgfqpoint{5.037500in}{3.427208in}}%
\pgfusepath{clip}%
\pgfsetbuttcap%
\pgfsetmiterjoin%
\definecolor{currentfill}{rgb}{1.000000,1.000000,1.000000}%
\pgfsetfillcolor{currentfill}%
\pgfsetlinewidth{0.501875pt}%
\definecolor{currentstroke}{rgb}{0.827451,0.827451,0.827451}%
\pgfsetstrokecolor{currentstroke}%
\pgfsetdash{}{0pt}%
\pgfpathmoveto{\pgfqpoint{1.098678in}{2.848553in}}%
\pgfpathlineto{\pgfqpoint{1.099020in}{2.844222in}}%
\pgfpathlineto{\pgfqpoint{1.093868in}{2.841529in}}%
\pgfpathlineto{\pgfqpoint{1.079494in}{2.847704in}}%
\pgfpathlineto{\pgfqpoint{1.077265in}{2.845220in}}%
\pgfpathlineto{\pgfqpoint{1.073788in}{2.856971in}}%
\pgfpathlineto{\pgfqpoint{1.080989in}{2.858606in}}%
\pgfpathlineto{\pgfqpoint{1.084318in}{2.853777in}}%
\pgfpathlineto{\pgfqpoint{1.091547in}{2.859822in}}%
\pgfpathclose%
\pgfusepath{stroke,fill}%
\end{pgfscope}%
\begin{pgfscope}%
\pgfpathrectangle{\pgfqpoint{0.100000in}{2.413063in}}{\pgfqpoint{5.037500in}{3.427208in}}%
\pgfusepath{clip}%
\pgfsetbuttcap%
\pgfsetmiterjoin%
\definecolor{currentfill}{rgb}{1.000000,1.000000,1.000000}%
\pgfsetfillcolor{currentfill}%
\pgfsetlinewidth{0.501875pt}%
\definecolor{currentstroke}{rgb}{0.827451,0.827451,0.827451}%
\pgfsetstrokecolor{currentstroke}%
\pgfsetdash{}{0pt}%
\pgfpathmoveto{\pgfqpoint{1.422926in}{2.608438in}}%
\pgfpathlineto{\pgfqpoint{1.410872in}{2.603950in}}%
\pgfpathlineto{\pgfqpoint{1.404488in}{2.603681in}}%
\pgfpathlineto{\pgfqpoint{1.409684in}{2.613202in}}%
\pgfpathlineto{\pgfqpoint{1.414124in}{2.616752in}}%
\pgfpathlineto{\pgfqpoint{1.413981in}{2.626611in}}%
\pgfpathlineto{\pgfqpoint{1.418807in}{2.644409in}}%
\pgfpathlineto{\pgfqpoint{1.423252in}{2.647789in}}%
\pgfpathlineto{\pgfqpoint{1.433513in}{2.642895in}}%
\pgfpathlineto{\pgfqpoint{1.432015in}{2.640091in}}%
\pgfpathlineto{\pgfqpoint{1.432792in}{2.626619in}}%
\pgfpathlineto{\pgfqpoint{1.429465in}{2.622982in}}%
\pgfpathlineto{\pgfqpoint{1.428037in}{2.613413in}}%
\pgfpathclose%
\pgfusepath{stroke,fill}%
\end{pgfscope}%
\begin{pgfscope}%
\pgfpathrectangle{\pgfqpoint{0.100000in}{2.413063in}}{\pgfqpoint{5.037500in}{3.427208in}}%
\pgfusepath{clip}%
\pgfsetbuttcap%
\pgfsetmiterjoin%
\definecolor{currentfill}{rgb}{1.000000,1.000000,1.000000}%
\pgfsetfillcolor{currentfill}%
\pgfsetlinewidth{0.501875pt}%
\definecolor{currentstroke}{rgb}{0.827451,0.827451,0.827451}%
\pgfsetstrokecolor{currentstroke}%
\pgfsetdash{}{0pt}%
\pgfpathmoveto{\pgfqpoint{1.396233in}{2.652557in}}%
\pgfpathlineto{\pgfqpoint{1.403294in}{2.640364in}}%
\pgfpathlineto{\pgfqpoint{1.401985in}{2.637455in}}%
\pgfpathlineto{\pgfqpoint{1.411430in}{2.634498in}}%
\pgfpathlineto{\pgfqpoint{1.408718in}{2.623290in}}%
\pgfpathlineto{\pgfqpoint{1.404606in}{2.623808in}}%
\pgfpathlineto{\pgfqpoint{1.393800in}{2.643346in}}%
\pgfpathlineto{\pgfqpoint{1.395336in}{2.634553in}}%
\pgfpathlineto{\pgfqpoint{1.393496in}{2.629488in}}%
\pgfpathlineto{\pgfqpoint{1.388920in}{2.627235in}}%
\pgfpathlineto{\pgfqpoint{1.386318in}{2.629663in}}%
\pgfpathlineto{\pgfqpoint{1.384866in}{2.640922in}}%
\pgfpathlineto{\pgfqpoint{1.385723in}{2.648224in}}%
\pgfpathlineto{\pgfqpoint{1.382675in}{2.651411in}}%
\pgfpathlineto{\pgfqpoint{1.390726in}{2.660634in}}%
\pgfpathlineto{\pgfqpoint{1.396009in}{2.663371in}}%
\pgfpathlineto{\pgfqpoint{1.397768in}{2.659775in}}%
\pgfpathlineto{\pgfqpoint{1.403280in}{2.662558in}}%
\pgfpathlineto{\pgfqpoint{1.407273in}{2.655001in}}%
\pgfpathlineto{\pgfqpoint{1.415909in}{2.645386in}}%
\pgfpathlineto{\pgfqpoint{1.414730in}{2.641031in}}%
\pgfpathlineto{\pgfqpoint{1.409935in}{2.636286in}}%
\pgfpathlineto{\pgfqpoint{1.405501in}{2.638522in}}%
\pgfpathclose%
\pgfusepath{stroke,fill}%
\end{pgfscope}%
\begin{pgfscope}%
\pgfpathrectangle{\pgfqpoint{0.100000in}{2.413063in}}{\pgfqpoint{5.037500in}{3.427208in}}%
\pgfusepath{clip}%
\pgfsetbuttcap%
\pgfsetmiterjoin%
\definecolor{currentfill}{rgb}{1.000000,1.000000,1.000000}%
\pgfsetfillcolor{currentfill}%
\pgfsetlinewidth{0.501875pt}%
\definecolor{currentstroke}{rgb}{0.827451,0.827451,0.827451}%
\pgfsetstrokecolor{currentstroke}%
\pgfsetdash{}{0pt}%
\pgfpathmoveto{\pgfqpoint{1.045813in}{2.825442in}}%
\pgfpathlineto{\pgfqpoint{1.037066in}{2.824304in}}%
\pgfpathlineto{\pgfqpoint{1.024746in}{2.819568in}}%
\pgfpathlineto{\pgfqpoint{1.021873in}{2.822141in}}%
\pgfpathlineto{\pgfqpoint{1.034741in}{2.825049in}}%
\pgfpathlineto{\pgfqpoint{1.031292in}{2.827587in}}%
\pgfpathlineto{\pgfqpoint{1.023313in}{2.829445in}}%
\pgfpathlineto{\pgfqpoint{1.021183in}{2.835391in}}%
\pgfpathlineto{\pgfqpoint{1.025198in}{2.840514in}}%
\pgfpathlineto{\pgfqpoint{1.028578in}{2.852270in}}%
\pgfpathlineto{\pgfqpoint{1.040928in}{2.854460in}}%
\pgfpathlineto{\pgfqpoint{1.046907in}{2.849381in}}%
\pgfpathlineto{\pgfqpoint{1.052523in}{2.854297in}}%
\pgfpathlineto{\pgfqpoint{1.059567in}{2.853091in}}%
\pgfpathlineto{\pgfqpoint{1.064629in}{2.844012in}}%
\pgfpathlineto{\pgfqpoint{1.068916in}{2.852081in}}%
\pgfpathlineto{\pgfqpoint{1.081083in}{2.833205in}}%
\pgfpathlineto{\pgfqpoint{1.076688in}{2.831696in}}%
\pgfpathlineto{\pgfqpoint{1.073990in}{2.827604in}}%
\pgfpathlineto{\pgfqpoint{1.079087in}{2.823870in}}%
\pgfpathlineto{\pgfqpoint{1.071184in}{2.820369in}}%
\pgfpathlineto{\pgfqpoint{1.065454in}{2.822297in}}%
\pgfpathlineto{\pgfqpoint{1.054630in}{2.828686in}}%
\pgfpathclose%
\pgfusepath{stroke,fill}%
\end{pgfscope}%
\begin{pgfscope}%
\pgfpathrectangle{\pgfqpoint{0.100000in}{2.413063in}}{\pgfqpoint{5.037500in}{3.427208in}}%
\pgfusepath{clip}%
\pgfsetbuttcap%
\pgfsetmiterjoin%
\definecolor{currentfill}{rgb}{1.000000,1.000000,1.000000}%
\pgfsetfillcolor{currentfill}%
\pgfsetlinewidth{0.501875pt}%
\definecolor{currentstroke}{rgb}{0.827451,0.827451,0.827451}%
\pgfsetstrokecolor{currentstroke}%
\pgfsetdash{}{0pt}%
\pgfpathmoveto{\pgfqpoint{1.388888in}{2.572423in}}%
\pgfpathlineto{\pgfqpoint{1.387153in}{2.589856in}}%
\pgfpathlineto{\pgfqpoint{1.391828in}{2.596257in}}%
\pgfpathlineto{\pgfqpoint{1.386830in}{2.596702in}}%
\pgfpathlineto{\pgfqpoint{1.388657in}{2.609151in}}%
\pgfpathlineto{\pgfqpoint{1.390811in}{2.616542in}}%
\pgfpathlineto{\pgfqpoint{1.388939in}{2.626471in}}%
\pgfpathlineto{\pgfqpoint{1.393517in}{2.628394in}}%
\pgfpathlineto{\pgfqpoint{1.395011in}{2.630962in}}%
\pgfpathlineto{\pgfqpoint{1.399522in}{2.629988in}}%
\pgfpathlineto{\pgfqpoint{1.403426in}{2.622032in}}%
\pgfpathlineto{\pgfqpoint{1.404151in}{2.614744in}}%
\pgfpathlineto{\pgfqpoint{1.399375in}{2.592915in}}%
\pgfpathclose%
\pgfusepath{stroke,fill}%
\end{pgfscope}%
\begin{pgfscope}%
\pgfpathrectangle{\pgfqpoint{0.100000in}{2.413063in}}{\pgfqpoint{5.037500in}{3.427208in}}%
\pgfusepath{clip}%
\pgfsetbuttcap%
\pgfsetmiterjoin%
\definecolor{currentfill}{rgb}{1.000000,1.000000,1.000000}%
\pgfsetfillcolor{currentfill}%
\pgfsetlinewidth{0.501875pt}%
\definecolor{currentstroke}{rgb}{0.827451,0.827451,0.827451}%
\pgfsetstrokecolor{currentstroke}%
\pgfsetdash{}{0pt}%
\pgfpathmoveto{\pgfqpoint{1.387850in}{2.625640in}}%
\pgfpathlineto{\pgfqpoint{1.387611in}{2.617140in}}%
\pgfpathlineto{\pgfqpoint{1.383691in}{2.613281in}}%
\pgfpathlineto{\pgfqpoint{1.379160in}{2.614707in}}%
\pgfpathlineto{\pgfqpoint{1.384887in}{2.627054in}}%
\pgfpathclose%
\pgfusepath{stroke,fill}%
\end{pgfscope}%
\begin{pgfscope}%
\pgfpathrectangle{\pgfqpoint{0.100000in}{2.413063in}}{\pgfqpoint{5.037500in}{3.427208in}}%
\pgfusepath{clip}%
\pgfsetbuttcap%
\pgfsetmiterjoin%
\definecolor{currentfill}{rgb}{1.000000,1.000000,1.000000}%
\pgfsetfillcolor{currentfill}%
\pgfsetlinewidth{0.501875pt}%
\definecolor{currentstroke}{rgb}{0.827451,0.827451,0.827451}%
\pgfsetstrokecolor{currentstroke}%
\pgfsetdash{}{0pt}%
\pgfpathmoveto{\pgfqpoint{1.435950in}{2.587901in}}%
\pgfpathlineto{\pgfqpoint{1.432607in}{2.573676in}}%
\pgfpathlineto{\pgfqpoint{1.424819in}{2.569179in}}%
\pgfpathlineto{\pgfqpoint{1.414600in}{2.573322in}}%
\pgfpathlineto{\pgfqpoint{1.417416in}{2.578850in}}%
\pgfpathlineto{\pgfqpoint{1.417359in}{2.582678in}}%
\pgfpathlineto{\pgfqpoint{1.421158in}{2.589182in}}%
\pgfpathlineto{\pgfqpoint{1.418488in}{2.587873in}}%
\pgfpathlineto{\pgfqpoint{1.420434in}{2.591049in}}%
\pgfpathlineto{\pgfqpoint{1.416736in}{2.598265in}}%
\pgfpathlineto{\pgfqpoint{1.420933in}{2.599596in}}%
\pgfpathlineto{\pgfqpoint{1.430648in}{2.591152in}}%
\pgfpathclose%
\pgfusepath{stroke,fill}%
\end{pgfscope}%
\begin{pgfscope}%
\pgfpathrectangle{\pgfqpoint{0.100000in}{2.413063in}}{\pgfqpoint{5.037500in}{3.427208in}}%
\pgfusepath{clip}%
\pgfsetbuttcap%
\pgfsetmiterjoin%
\definecolor{currentfill}{rgb}{1.000000,1.000000,1.000000}%
\pgfsetfillcolor{currentfill}%
\pgfsetlinewidth{0.501875pt}%
\definecolor{currentstroke}{rgb}{0.827451,0.827451,0.827451}%
\pgfsetstrokecolor{currentstroke}%
\pgfsetdash{}{0pt}%
\pgfpathmoveto{\pgfqpoint{1.401856in}{2.562748in}}%
\pgfpathlineto{\pgfqpoint{1.397450in}{2.561035in}}%
\pgfpathlineto{\pgfqpoint{1.399966in}{2.576046in}}%
\pgfpathlineto{\pgfqpoint{1.405253in}{2.579549in}}%
\pgfpathlineto{\pgfqpoint{1.403964in}{2.590797in}}%
\pgfpathlineto{\pgfqpoint{1.406113in}{2.595784in}}%
\pgfpathlineto{\pgfqpoint{1.410363in}{2.597653in}}%
\pgfpathlineto{\pgfqpoint{1.414824in}{2.592761in}}%
\pgfpathlineto{\pgfqpoint{1.418653in}{2.586122in}}%
\pgfpathlineto{\pgfqpoint{1.406344in}{2.573117in}}%
\pgfpathclose%
\pgfusepath{stroke,fill}%
\end{pgfscope}%
\begin{pgfscope}%
\pgfpathrectangle{\pgfqpoint{0.100000in}{2.413063in}}{\pgfqpoint{5.037500in}{3.427208in}}%
\pgfusepath{clip}%
\pgfsetbuttcap%
\pgfsetmiterjoin%
\definecolor{currentfill}{rgb}{1.000000,1.000000,1.000000}%
\pgfsetfillcolor{currentfill}%
\pgfsetlinewidth{0.501875pt}%
\definecolor{currentstroke}{rgb}{0.827451,0.827451,0.827451}%
\pgfsetstrokecolor{currentstroke}%
\pgfsetdash{}{0pt}%
\pgfpathmoveto{\pgfqpoint{1.437904in}{2.578393in}}%
\pgfpathlineto{\pgfqpoint{1.440063in}{2.568035in}}%
\pgfpathlineto{\pgfqpoint{1.430417in}{2.569220in}}%
\pgfpathclose%
\pgfusepath{stroke,fill}%
\end{pgfscope}%
\begin{pgfscope}%
\pgfpathrectangle{\pgfqpoint{0.100000in}{2.413063in}}{\pgfqpoint{5.037500in}{3.427208in}}%
\pgfusepath{clip}%
\pgfsetbuttcap%
\pgfsetmiterjoin%
\definecolor{currentfill}{rgb}{1.000000,1.000000,1.000000}%
\pgfsetfillcolor{currentfill}%
\pgfsetlinewidth{0.501875pt}%
\definecolor{currentstroke}{rgb}{0.827451,0.827451,0.827451}%
\pgfsetstrokecolor{currentstroke}%
\pgfsetdash{}{0pt}%
\pgfpathmoveto{\pgfqpoint{1.442508in}{2.562566in}}%
\pgfpathlineto{\pgfqpoint{1.444088in}{2.556175in}}%
\pgfpathlineto{\pgfqpoint{1.447114in}{2.554131in}}%
\pgfpathlineto{\pgfqpoint{1.446563in}{2.548116in}}%
\pgfpathlineto{\pgfqpoint{1.442336in}{2.546093in}}%
\pgfpathlineto{\pgfqpoint{1.438904in}{2.555072in}}%
\pgfpathclose%
\pgfusepath{stroke,fill}%
\end{pgfscope}%
\begin{pgfscope}%
\pgfpathrectangle{\pgfqpoint{0.100000in}{2.413063in}}{\pgfqpoint{5.037500in}{3.427208in}}%
\pgfusepath{clip}%
\pgfsetbuttcap%
\pgfsetmiterjoin%
\definecolor{currentfill}{rgb}{1.000000,1.000000,1.000000}%
\pgfsetfillcolor{currentfill}%
\pgfsetlinewidth{0.501875pt}%
\definecolor{currentstroke}{rgb}{0.827451,0.827451,0.827451}%
\pgfsetstrokecolor{currentstroke}%
\pgfsetdash{}{0pt}%
\pgfpathmoveto{\pgfqpoint{1.434457in}{2.565017in}}%
\pgfpathlineto{\pgfqpoint{1.431834in}{2.557594in}}%
\pgfpathlineto{\pgfqpoint{1.427759in}{2.557867in}}%
\pgfpathlineto{\pgfqpoint{1.425189in}{2.563988in}}%
\pgfpathlineto{\pgfqpoint{1.429376in}{2.566887in}}%
\pgfpathclose%
\pgfusepath{stroke,fill}%
\end{pgfscope}%
\begin{pgfscope}%
\pgfpathrectangle{\pgfqpoint{0.100000in}{2.413063in}}{\pgfqpoint{5.037500in}{3.427208in}}%
\pgfusepath{clip}%
\pgfsetbuttcap%
\pgfsetmiterjoin%
\definecolor{currentfill}{rgb}{1.000000,1.000000,1.000000}%
\pgfsetfillcolor{currentfill}%
\pgfsetlinewidth{0.501875pt}%
\definecolor{currentstroke}{rgb}{0.827451,0.827451,0.827451}%
\pgfsetstrokecolor{currentstroke}%
\pgfsetdash{}{0pt}%
\pgfpathmoveto{\pgfqpoint{1.423464in}{2.526748in}}%
\pgfpathlineto{\pgfqpoint{1.427100in}{2.522771in}}%
\pgfpathlineto{\pgfqpoint{1.427901in}{2.514391in}}%
\pgfpathlineto{\pgfqpoint{1.429814in}{2.511956in}}%
\pgfpathlineto{\pgfqpoint{1.426689in}{2.504374in}}%
\pgfpathlineto{\pgfqpoint{1.422577in}{2.500104in}}%
\pgfpathlineto{\pgfqpoint{1.420388in}{2.492184in}}%
\pgfpathlineto{\pgfqpoint{1.415893in}{2.504527in}}%
\pgfpathlineto{\pgfqpoint{1.415148in}{2.515734in}}%
\pgfpathlineto{\pgfqpoint{1.412633in}{2.521313in}}%
\pgfpathlineto{\pgfqpoint{1.403970in}{2.525836in}}%
\pgfpathlineto{\pgfqpoint{1.412096in}{2.535268in}}%
\pgfpathlineto{\pgfqpoint{1.406679in}{2.539299in}}%
\pgfpathlineto{\pgfqpoint{1.411451in}{2.543316in}}%
\pgfpathlineto{\pgfqpoint{1.417787in}{2.558739in}}%
\pgfpathlineto{\pgfqpoint{1.411220in}{2.564177in}}%
\pgfpathlineto{\pgfqpoint{1.413750in}{2.569397in}}%
\pgfpathlineto{\pgfqpoint{1.422238in}{2.564766in}}%
\pgfpathlineto{\pgfqpoint{1.423098in}{2.557458in}}%
\pgfpathlineto{\pgfqpoint{1.419886in}{2.553296in}}%
\pgfpathlineto{\pgfqpoint{1.426250in}{2.547666in}}%
\pgfpathlineto{\pgfqpoint{1.428254in}{2.538896in}}%
\pgfpathlineto{\pgfqpoint{1.427982in}{2.526595in}}%
\pgfpathclose%
\pgfusepath{stroke,fill}%
\end{pgfscope}%
\begin{pgfscope}%
\pgfpathrectangle{\pgfqpoint{0.100000in}{2.413063in}}{\pgfqpoint{5.037500in}{3.427208in}}%
\pgfusepath{clip}%
\pgfsetbuttcap%
\pgfsetmiterjoin%
\definecolor{currentfill}{rgb}{1.000000,1.000000,1.000000}%
\pgfsetfillcolor{currentfill}%
\pgfsetlinewidth{0.501875pt}%
\definecolor{currentstroke}{rgb}{0.827451,0.827451,0.827451}%
\pgfsetstrokecolor{currentstroke}%
\pgfsetdash{}{0pt}%
\pgfpathmoveto{\pgfqpoint{1.439005in}{2.557624in}}%
\pgfpathlineto{\pgfqpoint{1.437026in}{2.553358in}}%
\pgfpathlineto{\pgfqpoint{1.438885in}{2.546429in}}%
\pgfpathlineto{\pgfqpoint{1.434125in}{2.545847in}}%
\pgfpathlineto{\pgfqpoint{1.429134in}{2.549054in}}%
\pgfpathlineto{\pgfqpoint{1.430754in}{2.556120in}}%
\pgfpathclose%
\pgfusepath{stroke,fill}%
\end{pgfscope}%
\begin{pgfscope}%
\pgfpathrectangle{\pgfqpoint{0.100000in}{2.413063in}}{\pgfqpoint{5.037500in}{3.427208in}}%
\pgfusepath{clip}%
\pgfsetbuttcap%
\pgfsetmiterjoin%
\definecolor{currentfill}{rgb}{1.000000,1.000000,1.000000}%
\pgfsetfillcolor{currentfill}%
\pgfsetlinewidth{0.501875pt}%
\definecolor{currentstroke}{rgb}{0.827451,0.827451,0.827451}%
\pgfsetstrokecolor{currentstroke}%
\pgfsetdash{}{0pt}%
\pgfpathmoveto{\pgfqpoint{1.416821in}{2.558107in}}%
\pgfpathlineto{\pgfqpoint{1.406314in}{2.553395in}}%
\pgfpathlineto{\pgfqpoint{1.402743in}{2.555090in}}%
\pgfpathlineto{\pgfqpoint{1.409908in}{2.561307in}}%
\pgfpathclose%
\pgfusepath{stroke,fill}%
\end{pgfscope}%
\begin{pgfscope}%
\pgfpathrectangle{\pgfqpoint{0.100000in}{2.413063in}}{\pgfqpoint{5.037500in}{3.427208in}}%
\pgfusepath{clip}%
\pgfsetbuttcap%
\pgfsetmiterjoin%
\definecolor{currentfill}{rgb}{1.000000,1.000000,1.000000}%
\pgfsetfillcolor{currentfill}%
\pgfsetlinewidth{0.501875pt}%
\definecolor{currentstroke}{rgb}{0.827451,0.827451,0.827451}%
\pgfsetstrokecolor{currentstroke}%
\pgfsetdash{}{0pt}%
\pgfpathmoveto{\pgfqpoint{1.439121in}{2.512137in}}%
\pgfpathlineto{\pgfqpoint{1.436616in}{2.519232in}}%
\pgfpathlineto{\pgfqpoint{1.441914in}{2.520977in}}%
\pgfpathlineto{\pgfqpoint{1.443518in}{2.528629in}}%
\pgfpathlineto{\pgfqpoint{1.449259in}{2.528637in}}%
\pgfpathlineto{\pgfqpoint{1.449235in}{2.533965in}}%
\pgfpathlineto{\pgfqpoint{1.457214in}{2.532705in}}%
\pgfpathlineto{\pgfqpoint{1.458574in}{2.518262in}}%
\pgfpathlineto{\pgfqpoint{1.453946in}{2.509054in}}%
\pgfpathlineto{\pgfqpoint{1.447079in}{2.503302in}}%
\pgfpathclose%
\pgfusepath{stroke,fill}%
\end{pgfscope}%
\begin{pgfscope}%
\pgfpathrectangle{\pgfqpoint{0.100000in}{2.413063in}}{\pgfqpoint{5.037500in}{3.427208in}}%
\pgfusepath{clip}%
\pgfsetbuttcap%
\pgfsetmiterjoin%
\definecolor{currentfill}{rgb}{1.000000,1.000000,1.000000}%
\pgfsetfillcolor{currentfill}%
\pgfsetlinewidth{0.501875pt}%
\definecolor{currentstroke}{rgb}{0.827451,0.827451,0.827451}%
\pgfsetstrokecolor{currentstroke}%
\pgfsetdash{}{0pt}%
\pgfpathmoveto{\pgfqpoint{1.436019in}{2.517594in}}%
\pgfpathlineto{\pgfqpoint{1.438281in}{2.510876in}}%
\pgfpathlineto{\pgfqpoint{1.432702in}{2.510059in}}%
\pgfpathclose%
\pgfusepath{stroke,fill}%
\end{pgfscope}%
\begin{pgfscope}%
\pgfpathrectangle{\pgfqpoint{0.100000in}{2.413063in}}{\pgfqpoint{5.037500in}{3.427208in}}%
\pgfusepath{clip}%
\pgfsetbuttcap%
\pgfsetmiterjoin%
\definecolor{currentfill}{rgb}{1.000000,1.000000,1.000000}%
\pgfsetfillcolor{currentfill}%
\pgfsetlinewidth{0.501875pt}%
\definecolor{currentstroke}{rgb}{0.827451,0.827451,0.827451}%
\pgfsetstrokecolor{currentstroke}%
\pgfsetdash{}{0pt}%
\pgfpathmoveto{\pgfqpoint{1.440722in}{2.507713in}}%
\pgfpathlineto{\pgfqpoint{1.439944in}{2.499149in}}%
\pgfpathlineto{\pgfqpoint{1.434058in}{2.499858in}}%
\pgfpathlineto{\pgfqpoint{1.435962in}{2.503996in}}%
\pgfpathclose%
\pgfusepath{stroke,fill}%
\end{pgfscope}%
\begin{pgfscope}%
\pgfpathrectangle{\pgfqpoint{0.100000in}{2.413063in}}{\pgfqpoint{5.037500in}{3.427208in}}%
\pgfusepath{clip}%
\pgfsetbuttcap%
\pgfsetroundjoin%
\pgfsetlinewidth{0.501875pt}%
\definecolor{currentstroke}{rgb}{0.827451,0.827451,0.827451}%
\pgfsetstrokecolor{currentstroke}%
\pgfsetdash{}{0pt}%
\pgfusepath{stroke}%
\end{pgfscope}%
\begin{pgfscope}%
\pgfpathrectangle{\pgfqpoint{0.100000in}{2.413063in}}{\pgfqpoint{5.037500in}{3.427208in}}%
\pgfusepath{clip}%
\pgfsetbuttcap%
\pgfsetroundjoin%
\pgfsetlinewidth{0.501875pt}%
\definecolor{currentstroke}{rgb}{0.827451,0.827451,0.827451}%
\pgfsetstrokecolor{currentstroke}%
\pgfsetdash{}{0pt}%
\pgfusepath{stroke}%
\end{pgfscope}%
\begin{pgfscope}%
\pgfpathrectangle{\pgfqpoint{0.100000in}{2.413063in}}{\pgfqpoint{5.037500in}{3.427208in}}%
\pgfusepath{clip}%
\pgfsetbuttcap%
\pgfsetroundjoin%
\pgfsetlinewidth{0.501875pt}%
\definecolor{currentstroke}{rgb}{0.827451,0.827451,0.827451}%
\pgfsetstrokecolor{currentstroke}%
\pgfsetdash{}{0pt}%
\pgfusepath{stroke}%
\end{pgfscope}%
\begin{pgfscope}%
\pgfpathrectangle{\pgfqpoint{0.100000in}{2.413063in}}{\pgfqpoint{5.037500in}{3.427208in}}%
\pgfusepath{clip}%
\pgfsetbuttcap%
\pgfsetroundjoin%
\pgfsetlinewidth{0.501875pt}%
\definecolor{currentstroke}{rgb}{0.827451,0.827451,0.827451}%
\pgfsetstrokecolor{currentstroke}%
\pgfsetdash{}{0pt}%
\pgfusepath{stroke}%
\end{pgfscope}%
\begin{pgfscope}%
\pgfpathrectangle{\pgfqpoint{0.100000in}{2.413063in}}{\pgfqpoint{5.037500in}{3.427208in}}%
\pgfusepath{clip}%
\pgfsetbuttcap%
\pgfsetroundjoin%
\pgfsetlinewidth{0.501875pt}%
\definecolor{currentstroke}{rgb}{0.827451,0.827451,0.827451}%
\pgfsetstrokecolor{currentstroke}%
\pgfsetdash{}{0pt}%
\pgfusepath{stroke}%
\end{pgfscope}%
\begin{pgfscope}%
\pgfpathrectangle{\pgfqpoint{0.100000in}{2.413063in}}{\pgfqpoint{5.037500in}{3.427208in}}%
\pgfusepath{clip}%
\pgfsetbuttcap%
\pgfsetroundjoin%
\pgfsetlinewidth{0.501875pt}%
\definecolor{currentstroke}{rgb}{0.827451,0.827451,0.827451}%
\pgfsetstrokecolor{currentstroke}%
\pgfsetdash{}{0pt}%
\pgfusepath{stroke}%
\end{pgfscope}%
\begin{pgfscope}%
\pgfpathrectangle{\pgfqpoint{0.100000in}{2.413063in}}{\pgfqpoint{5.037500in}{3.427208in}}%
\pgfusepath{clip}%
\pgfsetbuttcap%
\pgfsetroundjoin%
\pgfsetlinewidth{0.501875pt}%
\definecolor{currentstroke}{rgb}{0.827451,0.827451,0.827451}%
\pgfsetstrokecolor{currentstroke}%
\pgfsetdash{}{0pt}%
\pgfpathmoveto{\pgfqpoint{1.129685in}{5.642352in}}%
\pgfpathlineto{\pgfqpoint{1.102423in}{5.534332in}}%
\pgfpathlineto{\pgfqpoint{1.083039in}{5.456934in}}%
\pgfpathlineto{\pgfqpoint{1.059991in}{5.363826in}}%
\pgfpathlineto{\pgfqpoint{1.055466in}{5.340878in}}%
\pgfpathlineto{\pgfqpoint{1.058421in}{5.319829in}}%
\pgfpathlineto{\pgfqpoint{1.054504in}{5.300440in}}%
\pgfpathlineto{\pgfqpoint{0.893620in}{5.342423in}}%
\pgfpathlineto{\pgfqpoint{0.879086in}{5.337547in}}%
\pgfpathlineto{\pgfqpoint{0.866649in}{5.341778in}}%
\pgfpathlineto{\pgfqpoint{0.809066in}{5.343162in}}%
\pgfpathlineto{\pgfqpoint{0.789590in}{5.337671in}}%
\pgfpathlineto{\pgfqpoint{0.770250in}{5.339526in}}%
\pgfpathlineto{\pgfqpoint{0.762052in}{5.348120in}}%
\pgfpathlineto{\pgfqpoint{0.710499in}{5.346247in}}%
\pgfpathlineto{\pgfqpoint{0.702357in}{5.360292in}}%
\pgfpathlineto{\pgfqpoint{0.686901in}{5.367691in}}%
\pgfpathlineto{\pgfqpoint{0.664413in}{5.372200in}}%
\pgfpathlineto{\pgfqpoint{0.625598in}{5.365598in}}%
\pgfpathlineto{\pgfqpoint{0.611269in}{5.372066in}}%
\pgfpathlineto{\pgfqpoint{0.589177in}{5.389269in}}%
\pgfpathlineto{\pgfqpoint{0.595818in}{5.422986in}}%
\pgfpathlineto{\pgfqpoint{0.593490in}{5.440205in}}%
\pgfpathlineto{\pgfqpoint{0.575981in}{5.458485in}}%
\pgfpathlineto{\pgfqpoint{0.564765in}{5.457351in}}%
\pgfpathlineto{\pgfqpoint{0.556720in}{5.475841in}}%
\pgfpathlineto{\pgfqpoint{0.537662in}{5.483315in}}%
\pgfpathlineto{\pgfqpoint{0.523809in}{5.482363in}}%
\pgfpathlineto{\pgfqpoint{0.519327in}{5.501353in}}%
\pgfpathlineto{\pgfqpoint{0.533128in}{5.499332in}}%
\pgfpathlineto{\pgfqpoint{0.532031in}{5.525726in}}%
\pgfpathlineto{\pgfqpoint{0.526017in}{5.541398in}}%
\pgfpathlineto{\pgfqpoint{0.528668in}{5.561458in}}%
\pgfpathlineto{\pgfqpoint{0.535889in}{5.576545in}}%
\pgfpathlineto{\pgfqpoint{0.535482in}{5.605175in}}%
\pgfpathlineto{\pgfqpoint{0.531627in}{5.615562in}}%
\pgfpathlineto{\pgfqpoint{0.538284in}{5.648888in}}%
\pgfpathlineto{\pgfqpoint{0.536362in}{5.670367in}}%
\pgfpathlineto{\pgfqpoint{0.529707in}{5.680665in}}%
\pgfpathlineto{\pgfqpoint{0.530722in}{5.714337in}}%
\pgfpathlineto{\pgfqpoint{0.540403in}{5.739099in}}%
\pgfpathlineto{\pgfqpoint{0.550932in}{5.733039in}}%
\pgfpathlineto{\pgfqpoint{0.585796in}{5.697025in}}%
\pgfpathlineto{\pgfqpoint{0.628042in}{5.677317in}}%
\pgfpathlineto{\pgfqpoint{0.649673in}{5.674985in}}%
\pgfpathlineto{\pgfqpoint{0.670052in}{5.666594in}}%
\pgfpathlineto{\pgfqpoint{0.675837in}{5.638359in}}%
\pgfpathlineto{\pgfqpoint{0.657977in}{5.633010in}}%
\pgfpathlineto{\pgfqpoint{0.646142in}{5.610758in}}%
\pgfpathlineto{\pgfqpoint{0.660038in}{5.611994in}}%
\pgfpathlineto{\pgfqpoint{0.665641in}{5.621983in}}%
\pgfpathlineto{\pgfqpoint{0.685238in}{5.634473in}}%
\pgfpathlineto{\pgfqpoint{0.684226in}{5.615924in}}%
\pgfpathlineto{\pgfqpoint{0.671141in}{5.612934in}}%
\pgfpathlineto{\pgfqpoint{0.673314in}{5.589093in}}%
\pgfpathlineto{\pgfqpoint{0.657380in}{5.565210in}}%
\pgfpathlineto{\pgfqpoint{0.647182in}{5.576567in}}%
\pgfpathlineto{\pgfqpoint{0.616877in}{5.570392in}}%
\pgfpathlineto{\pgfqpoint{0.615283in}{5.556499in}}%
\pgfpathlineto{\pgfqpoint{0.625730in}{5.548035in}}%
\pgfpathlineto{\pgfqpoint{0.641668in}{5.547203in}}%
\pgfpathlineto{\pgfqpoint{0.663706in}{5.565290in}}%
\pgfpathlineto{\pgfqpoint{0.681148in}{5.566792in}}%
\pgfpathlineto{\pgfqpoint{0.681817in}{5.586812in}}%
\pgfpathlineto{\pgfqpoint{0.690807in}{5.616218in}}%
\pgfpathlineto{\pgfqpoint{0.712110in}{5.638088in}}%
\pgfpathlineto{\pgfqpoint{0.705007in}{5.655027in}}%
\pgfpathlineto{\pgfqpoint{0.709859in}{5.673264in}}%
\pgfpathlineto{\pgfqpoint{0.700407in}{5.691439in}}%
\pgfpathlineto{\pgfqpoint{0.713877in}{5.714682in}}%
\pgfpathlineto{\pgfqpoint{0.715895in}{5.728665in}}%
\pgfpathlineto{\pgfqpoint{0.704200in}{5.737856in}}%
\pgfpathlineto{\pgfqpoint{0.706147in}{5.761364in}}%
\pgfpathlineto{\pgfqpoint{0.846264in}{5.718794in}}%
\pgfpathlineto{\pgfqpoint{0.995055in}{5.677070in}}%
\pgfpathlineto{\pgfqpoint{1.129685in}{5.642352in}}%
\pgfusepath{stroke}%
\end{pgfscope}%
\begin{pgfscope}%
\pgfpathrectangle{\pgfqpoint{0.100000in}{2.413063in}}{\pgfqpoint{5.037500in}{3.427208in}}%
\pgfusepath{clip}%
\pgfsetbuttcap%
\pgfsetroundjoin%
\pgfsetlinewidth{0.501875pt}%
\definecolor{currentstroke}{rgb}{0.827451,0.827451,0.827451}%
\pgfsetstrokecolor{currentstroke}%
\pgfsetdash{}{0pt}%
\pgfpathmoveto{\pgfqpoint{0.683098in}{5.680037in}}%
\pgfpathlineto{\pgfqpoint{0.689997in}{5.653438in}}%
\pgfpathlineto{\pgfqpoint{0.700816in}{5.644603in}}%
\pgfpathlineto{\pgfqpoint{0.692183in}{5.633355in}}%
\pgfpathlineto{\pgfqpoint{0.683769in}{5.649836in}}%
\pgfpathlineto{\pgfqpoint{0.683098in}{5.680037in}}%
\pgfusepath{stroke}%
\end{pgfscope}%
\begin{pgfscope}%
\pgfpathrectangle{\pgfqpoint{0.100000in}{2.413063in}}{\pgfqpoint{5.037500in}{3.427208in}}%
\pgfusepath{clip}%
\pgfsetbuttcap%
\pgfsetroundjoin%
\pgfsetlinewidth{0.501875pt}%
\definecolor{currentstroke}{rgb}{0.827451,0.827451,0.827451}%
\pgfsetstrokecolor{currentstroke}%
\pgfsetdash{}{0pt}%
\pgfpathmoveto{\pgfqpoint{1.201939in}{5.624893in}}%
\pgfpathlineto{\pgfqpoint{1.351716in}{5.591281in}}%
\pgfpathlineto{\pgfqpoint{1.492772in}{5.562770in}}%
\pgfpathlineto{\pgfqpoint{1.601299in}{5.542885in}}%
\pgfpathlineto{\pgfqpoint{1.695915in}{5.526989in}}%
\pgfpathlineto{\pgfqpoint{1.790740in}{5.512398in}}%
\pgfpathlineto{\pgfqpoint{1.871493in}{5.501022in}}%
\pgfpathlineto{\pgfqpoint{1.952373in}{5.490591in}}%
\pgfpathlineto{\pgfqpoint{2.033369in}{5.481105in}}%
\pgfpathlineto{\pgfqpoint{2.109697in}{5.473044in}}%
\pgfpathlineto{\pgfqpoint{2.099200in}{5.356717in}}%
\pgfpathlineto{\pgfqpoint{2.083575in}{5.199518in}}%
\pgfpathlineto{\pgfqpoint{2.075380in}{5.118668in}}%
\pgfpathlineto{\pgfqpoint{2.063595in}{5.009707in}}%
\pgfpathlineto{\pgfqpoint{1.980159in}{5.018817in}}%
\pgfpathlineto{\pgfqpoint{1.884635in}{5.029661in}}%
\pgfpathlineto{\pgfqpoint{1.751998in}{5.047814in}}%
\pgfpathlineto{\pgfqpoint{1.692759in}{5.056264in}}%
\pgfpathlineto{\pgfqpoint{1.546810in}{5.079174in}}%
\pgfpathlineto{\pgfqpoint{1.496569in}{5.088365in}}%
\pgfpathlineto{\pgfqpoint{1.486097in}{5.028827in}}%
\pgfpathlineto{\pgfqpoint{1.480372in}{5.033075in}}%
\pgfpathlineto{\pgfqpoint{1.469613in}{5.061717in}}%
\pgfpathlineto{\pgfqpoint{1.455784in}{5.059953in}}%
\pgfpathlineto{\pgfqpoint{1.445885in}{5.045474in}}%
\pgfpathlineto{\pgfqpoint{1.425343in}{5.043537in}}%
\pgfpathlineto{\pgfqpoint{1.421216in}{5.050465in}}%
\pgfpathlineto{\pgfqpoint{1.401974in}{5.049626in}}%
\pgfpathlineto{\pgfqpoint{1.392301in}{5.056334in}}%
\pgfpathlineto{\pgfqpoint{1.378775in}{5.045956in}}%
\pgfpathlineto{\pgfqpoint{1.345921in}{5.055295in}}%
\pgfpathlineto{\pgfqpoint{1.331583in}{5.050280in}}%
\pgfpathlineto{\pgfqpoint{1.323870in}{5.072534in}}%
\pgfpathlineto{\pgfqpoint{1.323509in}{5.093709in}}%
\pgfpathlineto{\pgfqpoint{1.300862in}{5.108938in}}%
\pgfpathlineto{\pgfqpoint{1.304555in}{5.130885in}}%
\pgfpathlineto{\pgfqpoint{1.288430in}{5.167138in}}%
\pgfpathlineto{\pgfqpoint{1.289516in}{5.197955in}}%
\pgfpathlineto{\pgfqpoint{1.275657in}{5.213038in}}%
\pgfpathlineto{\pgfqpoint{1.262653in}{5.199694in}}%
\pgfpathlineto{\pgfqpoint{1.244953in}{5.192477in}}%
\pgfpathlineto{\pgfqpoint{1.231480in}{5.207352in}}%
\pgfpathlineto{\pgfqpoint{1.233671in}{5.231107in}}%
\pgfpathlineto{\pgfqpoint{1.249829in}{5.240466in}}%
\pgfpathlineto{\pgfqpoint{1.245534in}{5.255516in}}%
\pgfpathlineto{\pgfqpoint{1.273621in}{5.328361in}}%
\pgfpathlineto{\pgfqpoint{1.251474in}{5.330187in}}%
\pgfpathlineto{\pgfqpoint{1.249197in}{5.343259in}}%
\pgfpathlineto{\pgfqpoint{1.233064in}{5.354532in}}%
\pgfpathlineto{\pgfqpoint{1.234095in}{5.367134in}}%
\pgfpathlineto{\pgfqpoint{1.225585in}{5.376924in}}%
\pgfpathlineto{\pgfqpoint{1.211732in}{5.412469in}}%
\pgfpathlineto{\pgfqpoint{1.199072in}{5.419718in}}%
\pgfpathlineto{\pgfqpoint{1.190007in}{5.450394in}}%
\pgfpathlineto{\pgfqpoint{1.192146in}{5.470785in}}%
\pgfpathlineto{\pgfqpoint{1.175235in}{5.508340in}}%
\pgfpathlineto{\pgfqpoint{1.201939in}{5.624893in}}%
\pgfusepath{stroke}%
\end{pgfscope}%
\begin{pgfscope}%
\pgfpathrectangle{\pgfqpoint{0.100000in}{2.413063in}}{\pgfqpoint{5.037500in}{3.427208in}}%
\pgfusepath{clip}%
\pgfsetbuttcap%
\pgfsetroundjoin%
\pgfsetlinewidth{0.501875pt}%
\definecolor{currentstroke}{rgb}{0.827451,0.827451,0.827451}%
\pgfsetstrokecolor{currentstroke}%
\pgfsetdash{}{0pt}%
\pgfpathmoveto{\pgfqpoint{4.817670in}{5.025444in}}%
\pgfpathlineto{\pgfqpoint{4.814672in}{5.038191in}}%
\pgfpathlineto{\pgfqpoint{4.798011in}{5.049357in}}%
\pgfpathlineto{\pgfqpoint{4.753984in}{5.193372in}}%
\pgfpathlineto{\pgfqpoint{4.730253in}{5.262870in}}%
\pgfpathlineto{\pgfqpoint{4.749331in}{5.282892in}}%
\pgfpathlineto{\pgfqpoint{4.767877in}{5.331893in}}%
\pgfpathlineto{\pgfqpoint{4.777126in}{5.347553in}}%
\pgfpathlineto{\pgfqpoint{4.770574in}{5.354129in}}%
\pgfpathlineto{\pgfqpoint{4.768379in}{5.397626in}}%
\pgfpathlineto{\pgfqpoint{4.776654in}{5.410960in}}%
\pgfpathlineto{\pgfqpoint{4.773027in}{5.441953in}}%
\pgfpathlineto{\pgfqpoint{4.806052in}{5.543639in}}%
\pgfpathlineto{\pgfqpoint{4.820933in}{5.544194in}}%
\pgfpathlineto{\pgfqpoint{4.827110in}{5.526031in}}%
\pgfpathlineto{\pgfqpoint{4.840334in}{5.520815in}}%
\pgfpathlineto{\pgfqpoint{4.865315in}{5.542180in}}%
\pgfpathlineto{\pgfqpoint{4.884903in}{5.554799in}}%
\pgfpathlineto{\pgfqpoint{4.928062in}{5.532577in}}%
\pgfpathlineto{\pgfqpoint{4.967053in}{5.409195in}}%
\pgfpathlineto{\pgfqpoint{4.974465in}{5.378840in}}%
\pgfpathlineto{\pgfqpoint{4.991544in}{5.375280in}}%
\pgfpathlineto{\pgfqpoint{5.014707in}{5.354663in}}%
\pgfpathlineto{\pgfqpoint{5.013396in}{5.342651in}}%
\pgfpathlineto{\pgfqpoint{5.029175in}{5.328413in}}%
\pgfpathlineto{\pgfqpoint{5.041992in}{5.336582in}}%
\pgfpathlineto{\pgfqpoint{5.068814in}{5.305301in}}%
\pgfpathlineto{\pgfqpoint{5.056826in}{5.280267in}}%
\pgfpathlineto{\pgfqpoint{5.040752in}{5.279772in}}%
\pgfpathlineto{\pgfqpoint{5.027855in}{5.257424in}}%
\pgfpathlineto{\pgfqpoint{5.012232in}{5.254255in}}%
\pgfpathlineto{\pgfqpoint{5.000763in}{5.242355in}}%
\pgfpathlineto{\pgfqpoint{4.966507in}{5.229594in}}%
\pgfpathlineto{\pgfqpoint{4.945773in}{5.209029in}}%
\pgfpathlineto{\pgfqpoint{4.935241in}{5.223790in}}%
\pgfpathlineto{\pgfqpoint{4.925671in}{5.213211in}}%
\pgfpathlineto{\pgfqpoint{4.928292in}{5.170441in}}%
\pgfpathlineto{\pgfqpoint{4.920714in}{5.153500in}}%
\pgfpathlineto{\pgfqpoint{4.904185in}{5.158127in}}%
\pgfpathlineto{\pgfqpoint{4.901518in}{5.140759in}}%
\pgfpathlineto{\pgfqpoint{4.894375in}{5.133617in}}%
\pgfpathlineto{\pgfqpoint{4.880081in}{5.135357in}}%
\pgfpathlineto{\pgfqpoint{4.880982in}{5.119118in}}%
\pgfpathlineto{\pgfqpoint{4.859316in}{5.123730in}}%
\pgfpathlineto{\pgfqpoint{4.847484in}{5.101231in}}%
\pgfpathlineto{\pgfqpoint{4.851952in}{5.089459in}}%
\pgfpathlineto{\pgfqpoint{4.844914in}{5.069879in}}%
\pgfpathlineto{\pgfqpoint{4.833835in}{5.055478in}}%
\pgfpathlineto{\pgfqpoint{4.831046in}{5.025414in}}%
\pgfpathlineto{\pgfqpoint{4.817670in}{5.025444in}}%
\pgfusepath{stroke}%
\end{pgfscope}%
\begin{pgfscope}%
\pgfpathrectangle{\pgfqpoint{0.100000in}{2.413063in}}{\pgfqpoint{5.037500in}{3.427208in}}%
\pgfusepath{clip}%
\pgfsetbuttcap%
\pgfsetroundjoin%
\pgfsetlinewidth{0.501875pt}%
\definecolor{currentstroke}{rgb}{0.827451,0.827451,0.827451}%
\pgfsetstrokecolor{currentstroke}%
\pgfsetdash{}{0pt}%
\pgfpathmoveto{\pgfqpoint{4.972660in}{5.220812in}}%
\pgfpathlineto{\pgfqpoint{4.982442in}{5.231057in}}%
\pgfpathlineto{\pgfqpoint{4.991754in}{5.221427in}}%
\pgfpathlineto{\pgfqpoint{4.975027in}{5.208668in}}%
\pgfpathlineto{\pgfqpoint{4.972660in}{5.220812in}}%
\pgfusepath{stroke}%
\end{pgfscope}%
\begin{pgfscope}%
\pgfpathrectangle{\pgfqpoint{0.100000in}{2.413063in}}{\pgfqpoint{5.037500in}{3.427208in}}%
\pgfusepath{clip}%
\pgfsetbuttcap%
\pgfsetroundjoin%
\pgfsetlinewidth{0.501875pt}%
\definecolor{currentstroke}{rgb}{0.827451,0.827451,0.827451}%
\pgfsetstrokecolor{currentstroke}%
\pgfsetdash{}{0pt}%
\pgfpathmoveto{\pgfqpoint{2.075380in}{5.118668in}}%
\pgfpathlineto{\pgfqpoint{2.083575in}{5.199518in}}%
\pgfpathlineto{\pgfqpoint{2.099200in}{5.356717in}}%
\pgfpathlineto{\pgfqpoint{2.109697in}{5.473044in}}%
\pgfpathlineto{\pgfqpoint{2.195665in}{5.464979in}}%
\pgfpathlineto{\pgfqpoint{2.305646in}{5.456223in}}%
\pgfpathlineto{\pgfqpoint{2.406173in}{5.449747in}}%
\pgfpathlineto{\pgfqpoint{2.497197in}{5.445142in}}%
\pgfpathlineto{\pgfqpoint{2.633036in}{5.440482in}}%
\pgfpathlineto{\pgfqpoint{2.642307in}{5.402197in}}%
\pgfpathlineto{\pgfqpoint{2.638988in}{5.383901in}}%
\pgfpathlineto{\pgfqpoint{2.637896in}{5.346294in}}%
\pgfpathlineto{\pgfqpoint{2.644203in}{5.318130in}}%
\pgfpathlineto{\pgfqpoint{2.658653in}{5.276579in}}%
\pgfpathlineto{\pgfqpoint{2.658619in}{5.224267in}}%
\pgfpathlineto{\pgfqpoint{2.661293in}{5.163497in}}%
\pgfpathlineto{\pgfqpoint{2.664964in}{5.147119in}}%
\pgfpathlineto{\pgfqpoint{2.675704in}{5.129137in}}%
\pgfpathlineto{\pgfqpoint{2.679246in}{5.101121in}}%
\pgfpathlineto{\pgfqpoint{2.677717in}{5.082416in}}%
\pgfpathlineto{\pgfqpoint{2.563838in}{5.084883in}}%
\pgfpathlineto{\pgfqpoint{2.480994in}{5.089082in}}%
\pgfpathlineto{\pgfqpoint{2.359549in}{5.095547in}}%
\pgfpathlineto{\pgfqpoint{2.239785in}{5.104054in}}%
\pgfpathlineto{\pgfqpoint{2.160017in}{5.110489in}}%
\pgfpathlineto{\pgfqpoint{2.075380in}{5.118668in}}%
\pgfusepath{stroke}%
\end{pgfscope}%
\begin{pgfscope}%
\pgfpathrectangle{\pgfqpoint{0.100000in}{2.413063in}}{\pgfqpoint{5.037500in}{3.427208in}}%
\pgfusepath{clip}%
\pgfsetbuttcap%
\pgfsetroundjoin%
\pgfsetlinewidth{0.501875pt}%
\definecolor{currentstroke}{rgb}{0.827451,0.827451,0.827451}%
\pgfsetstrokecolor{currentstroke}%
\pgfsetdash{}{0pt}%
\pgfpathmoveto{\pgfqpoint{2.040958in}{4.780305in}}%
\pgfpathlineto{\pgfqpoint{2.054105in}{4.915834in}}%
\pgfpathlineto{\pgfqpoint{2.063595in}{5.009707in}}%
\pgfpathlineto{\pgfqpoint{2.075380in}{5.118668in}}%
\pgfpathlineto{\pgfqpoint{2.160017in}{5.110489in}}%
\pgfpathlineto{\pgfqpoint{2.239785in}{5.104054in}}%
\pgfpathlineto{\pgfqpoint{2.359549in}{5.095547in}}%
\pgfpathlineto{\pgfqpoint{2.480994in}{5.089082in}}%
\pgfpathlineto{\pgfqpoint{2.563838in}{5.084883in}}%
\pgfpathlineto{\pgfqpoint{2.677717in}{5.082416in}}%
\pgfpathlineto{\pgfqpoint{2.670006in}{5.059917in}}%
\pgfpathlineto{\pgfqpoint{2.654617in}{5.042254in}}%
\pgfpathlineto{\pgfqpoint{2.666402in}{5.021918in}}%
\pgfpathlineto{\pgfqpoint{2.679397in}{5.017573in}}%
\pgfpathlineto{\pgfqpoint{2.685563in}{5.005892in}}%
\pgfpathlineto{\pgfqpoint{2.684148in}{4.920604in}}%
\pgfpathlineto{\pgfqpoint{2.681778in}{4.800574in}}%
\pgfpathlineto{\pgfqpoint{2.673013in}{4.769048in}}%
\pgfpathlineto{\pgfqpoint{2.680853in}{4.751610in}}%
\pgfpathlineto{\pgfqpoint{2.665113in}{4.714137in}}%
\pgfpathlineto{\pgfqpoint{2.681695in}{4.683929in}}%
\pgfpathlineto{\pgfqpoint{2.667607in}{4.686241in}}%
\pgfpathlineto{\pgfqpoint{2.657995in}{4.705035in}}%
\pgfpathlineto{\pgfqpoint{2.623692in}{4.717918in}}%
\pgfpathlineto{\pgfqpoint{2.602058in}{4.729251in}}%
\pgfpathlineto{\pgfqpoint{2.565756in}{4.730193in}}%
\pgfpathlineto{\pgfqpoint{2.553155in}{4.719869in}}%
\pgfpathlineto{\pgfqpoint{2.512066in}{4.740198in}}%
\pgfpathlineto{\pgfqpoint{2.508914in}{4.746627in}}%
\pgfpathlineto{\pgfqpoint{2.365512in}{4.753414in}}%
\pgfpathlineto{\pgfqpoint{2.278390in}{4.758408in}}%
\pgfpathlineto{\pgfqpoint{2.206426in}{4.764043in}}%
\pgfpathlineto{\pgfqpoint{2.087517in}{4.775278in}}%
\pgfpathlineto{\pgfqpoint{2.040958in}{4.780305in}}%
\pgfusepath{stroke}%
\end{pgfscope}%
\begin{pgfscope}%
\pgfpathrectangle{\pgfqpoint{0.100000in}{2.413063in}}{\pgfqpoint{5.037500in}{3.427208in}}%
\pgfusepath{clip}%
\pgfsetbuttcap%
\pgfsetroundjoin%
\pgfsetlinewidth{0.501875pt}%
\definecolor{currentstroke}{rgb}{0.827451,0.827451,0.827451}%
\pgfsetstrokecolor{currentstroke}%
\pgfsetdash{}{0pt}%
\pgfpathmoveto{\pgfqpoint{2.018405in}{4.550818in}}%
\pgfpathlineto{\pgfqpoint{1.941955in}{4.557799in}}%
\pgfpathlineto{\pgfqpoint{1.775315in}{4.578367in}}%
\pgfpathlineto{\pgfqpoint{1.684663in}{4.591377in}}%
\pgfpathlineto{\pgfqpoint{1.587436in}{4.605334in}}%
\pgfpathlineto{\pgfqpoint{1.505537in}{4.618440in}}%
\pgfpathlineto{\pgfqpoint{1.415642in}{4.633831in}}%
\pgfpathlineto{\pgfqpoint{1.436081in}{4.747212in}}%
\pgfpathlineto{\pgfqpoint{1.456784in}{4.863472in}}%
\pgfpathlineto{\pgfqpoint{1.486097in}{5.028827in}}%
\pgfpathlineto{\pgfqpoint{1.496569in}{5.088365in}}%
\pgfpathlineto{\pgfqpoint{1.546810in}{5.079174in}}%
\pgfpathlineto{\pgfqpoint{1.692759in}{5.056264in}}%
\pgfpathlineto{\pgfqpoint{1.751998in}{5.047814in}}%
\pgfpathlineto{\pgfqpoint{1.884635in}{5.029661in}}%
\pgfpathlineto{\pgfqpoint{1.980159in}{5.018817in}}%
\pgfpathlineto{\pgfqpoint{2.063595in}{5.009707in}}%
\pgfpathlineto{\pgfqpoint{2.054105in}{4.915834in}}%
\pgfpathlineto{\pgfqpoint{2.040958in}{4.780305in}}%
\pgfpathlineto{\pgfqpoint{2.029673in}{4.665125in}}%
\pgfpathlineto{\pgfqpoint{2.018405in}{4.550818in}}%
\pgfusepath{stroke}%
\end{pgfscope}%
\begin{pgfscope}%
\pgfpathrectangle{\pgfqpoint{0.100000in}{2.413063in}}{\pgfqpoint{5.037500in}{3.427208in}}%
\pgfusepath{clip}%
\pgfsetbuttcap%
\pgfsetroundjoin%
\pgfsetlinewidth{0.501875pt}%
\definecolor{currentstroke}{rgb}{0.827451,0.827451,0.827451}%
\pgfsetstrokecolor{currentstroke}%
\pgfsetdash{}{0pt}%
\pgfpathmoveto{\pgfqpoint{3.416161in}{4.707215in}}%
\pgfpathlineto{\pgfqpoint{3.319282in}{4.700333in}}%
\pgfpathlineto{\pgfqpoint{3.174878in}{4.694179in}}%
\pgfpathlineto{\pgfqpoint{3.169391in}{4.708766in}}%
\pgfpathlineto{\pgfqpoint{3.137369in}{4.719670in}}%
\pgfpathlineto{\pgfqpoint{3.130306in}{4.740271in}}%
\pgfpathlineto{\pgfqpoint{3.127348in}{4.765756in}}%
\pgfpathlineto{\pgfqpoint{3.134564in}{4.778814in}}%
\pgfpathlineto{\pgfqpoint{3.123167in}{4.791344in}}%
\pgfpathlineto{\pgfqpoint{3.120418in}{4.806289in}}%
\pgfpathlineto{\pgfqpoint{3.116744in}{4.839351in}}%
\pgfpathlineto{\pgfqpoint{3.105822in}{4.857310in}}%
\pgfpathlineto{\pgfqpoint{3.086424in}{4.867390in}}%
\pgfpathlineto{\pgfqpoint{3.065315in}{4.884023in}}%
\pgfpathlineto{\pgfqpoint{3.054403in}{4.903703in}}%
\pgfpathlineto{\pgfqpoint{3.034824in}{4.911628in}}%
\pgfpathlineto{\pgfqpoint{3.023314in}{4.924526in}}%
\pgfpathlineto{\pgfqpoint{3.009371in}{4.926715in}}%
\pgfpathlineto{\pgfqpoint{2.984477in}{4.945866in}}%
\pgfpathlineto{\pgfqpoint{2.988521in}{4.967905in}}%
\pgfpathlineto{\pgfqpoint{2.987741in}{5.009815in}}%
\pgfpathlineto{\pgfqpoint{2.995419in}{5.021352in}}%
\pgfpathlineto{\pgfqpoint{2.988521in}{5.038771in}}%
\pgfpathlineto{\pgfqpoint{2.976370in}{5.042141in}}%
\pgfpathlineto{\pgfqpoint{2.977374in}{5.057440in}}%
\pgfpathlineto{\pgfqpoint{2.992450in}{5.081609in}}%
\pgfpathlineto{\pgfqpoint{3.022305in}{5.100721in}}%
\pgfpathlineto{\pgfqpoint{3.020445in}{5.168690in}}%
\pgfpathlineto{\pgfqpoint{3.035383in}{5.178900in}}%
\pgfpathlineto{\pgfqpoint{3.049520in}{5.172098in}}%
\pgfpathlineto{\pgfqpoint{3.078318in}{5.182055in}}%
\pgfpathlineto{\pgfqpoint{3.132500in}{5.207078in}}%
\pgfpathlineto{\pgfqpoint{3.139558in}{5.199330in}}%
\pgfpathlineto{\pgfqpoint{3.129296in}{5.164236in}}%
\pgfpathlineto{\pgfqpoint{3.144563in}{5.171938in}}%
\pgfpathlineto{\pgfqpoint{3.170712in}{5.164247in}}%
\pgfpathlineto{\pgfqpoint{3.186780in}{5.157828in}}%
\pgfpathlineto{\pgfqpoint{3.195789in}{5.139018in}}%
\pgfpathlineto{\pgfqpoint{3.278223in}{5.121266in}}%
\pgfpathlineto{\pgfqpoint{3.302863in}{5.109087in}}%
\pgfpathlineto{\pgfqpoint{3.327916in}{5.109231in}}%
\pgfpathlineto{\pgfqpoint{3.353658in}{5.104206in}}%
\pgfpathlineto{\pgfqpoint{3.370364in}{5.087046in}}%
\pgfpathlineto{\pgfqpoint{3.386328in}{5.078566in}}%
\pgfpathlineto{\pgfqpoint{3.389261in}{5.054109in}}%
\pgfpathlineto{\pgfqpoint{3.384545in}{5.038752in}}%
\pgfpathlineto{\pgfqpoint{3.402337in}{5.037617in}}%
\pgfpathlineto{\pgfqpoint{3.396343in}{5.019806in}}%
\pgfpathlineto{\pgfqpoint{3.402044in}{5.013479in}}%
\pgfpathlineto{\pgfqpoint{3.407714in}{4.996666in}}%
\pgfpathlineto{\pgfqpoint{3.390375in}{4.987764in}}%
\pgfpathlineto{\pgfqpoint{3.380307in}{4.962963in}}%
\pgfpathlineto{\pgfqpoint{3.377148in}{4.945425in}}%
\pgfpathlineto{\pgfqpoint{3.386807in}{4.942424in}}%
\pgfpathlineto{\pgfqpoint{3.399173in}{4.955574in}}%
\pgfpathlineto{\pgfqpoint{3.409687in}{4.978365in}}%
\pgfpathlineto{\pgfqpoint{3.423912in}{4.986280in}}%
\pgfpathlineto{\pgfqpoint{3.434602in}{4.975984in}}%
\pgfpathlineto{\pgfqpoint{3.424080in}{4.944821in}}%
\pgfpathlineto{\pgfqpoint{3.420778in}{4.920629in}}%
\pgfpathlineto{\pgfqpoint{3.423887in}{4.903239in}}%
\pgfpathlineto{\pgfqpoint{3.414116in}{4.893401in}}%
\pgfpathlineto{\pgfqpoint{3.409236in}{4.869361in}}%
\pgfpathlineto{\pgfqpoint{3.413223in}{4.844095in}}%
\pgfpathlineto{\pgfqpoint{3.401692in}{4.806729in}}%
\pgfpathlineto{\pgfqpoint{3.401934in}{4.788058in}}%
\pgfpathlineto{\pgfqpoint{3.411046in}{4.747597in}}%
\pgfpathlineto{\pgfqpoint{3.416952in}{4.740656in}}%
\pgfpathlineto{\pgfqpoint{3.416161in}{4.707215in}}%
\pgfusepath{stroke}%
\end{pgfscope}%
\begin{pgfscope}%
\pgfpathrectangle{\pgfqpoint{0.100000in}{2.413063in}}{\pgfqpoint{5.037500in}{3.427208in}}%
\pgfusepath{clip}%
\pgfsetbuttcap%
\pgfsetroundjoin%
\pgfsetlinewidth{0.501875pt}%
\definecolor{currentstroke}{rgb}{0.827451,0.827451,0.827451}%
\pgfsetstrokecolor{currentstroke}%
\pgfsetdash{}{0pt}%
\pgfpathmoveto{\pgfqpoint{3.452484in}{5.035138in}}%
\pgfpathlineto{\pgfqpoint{3.454646in}{5.018999in}}%
\pgfpathlineto{\pgfqpoint{3.434836in}{4.976475in}}%
\pgfpathlineto{\pgfqpoint{3.426032in}{4.988793in}}%
\pgfpathlineto{\pgfqpoint{3.452484in}{5.035138in}}%
\pgfusepath{stroke}%
\end{pgfscope}%
\begin{pgfscope}%
\pgfpathrectangle{\pgfqpoint{0.100000in}{2.413063in}}{\pgfqpoint{5.037500in}{3.427208in}}%
\pgfusepath{clip}%
\pgfsetbuttcap%
\pgfsetroundjoin%
\pgfsetlinewidth{0.501875pt}%
\definecolor{currentstroke}{rgb}{0.827451,0.827451,0.827451}%
\pgfsetstrokecolor{currentstroke}%
\pgfsetdash{}{0pt}%
\pgfpathmoveto{\pgfqpoint{1.054504in}{5.300440in}}%
\pgfpathlineto{\pgfqpoint{1.058421in}{5.319829in}}%
\pgfpathlineto{\pgfqpoint{1.055466in}{5.340878in}}%
\pgfpathlineto{\pgfqpoint{1.059991in}{5.363826in}}%
\pgfpathlineto{\pgfqpoint{1.083039in}{5.456934in}}%
\pgfpathlineto{\pgfqpoint{1.102423in}{5.534332in}}%
\pgfpathlineto{\pgfqpoint{1.129685in}{5.642352in}}%
\pgfpathlineto{\pgfqpoint{1.201939in}{5.624893in}}%
\pgfpathlineto{\pgfqpoint{1.175235in}{5.508340in}}%
\pgfpathlineto{\pgfqpoint{1.192146in}{5.470785in}}%
\pgfpathlineto{\pgfqpoint{1.190007in}{5.450394in}}%
\pgfpathlineto{\pgfqpoint{1.199072in}{5.419718in}}%
\pgfpathlineto{\pgfqpoint{1.211732in}{5.412469in}}%
\pgfpathlineto{\pgfqpoint{1.225585in}{5.376924in}}%
\pgfpathlineto{\pgfqpoint{1.234095in}{5.367134in}}%
\pgfpathlineto{\pgfqpoint{1.233064in}{5.354532in}}%
\pgfpathlineto{\pgfqpoint{1.249197in}{5.343259in}}%
\pgfpathlineto{\pgfqpoint{1.251474in}{5.330187in}}%
\pgfpathlineto{\pgfqpoint{1.273621in}{5.328361in}}%
\pgfpathlineto{\pgfqpoint{1.245534in}{5.255516in}}%
\pgfpathlineto{\pgfqpoint{1.249829in}{5.240466in}}%
\pgfpathlineto{\pgfqpoint{1.233671in}{5.231107in}}%
\pgfpathlineto{\pgfqpoint{1.231480in}{5.207352in}}%
\pgfpathlineto{\pgfqpoint{1.244953in}{5.192477in}}%
\pgfpathlineto{\pgfqpoint{1.262653in}{5.199694in}}%
\pgfpathlineto{\pgfqpoint{1.275657in}{5.213038in}}%
\pgfpathlineto{\pgfqpoint{1.289516in}{5.197955in}}%
\pgfpathlineto{\pgfqpoint{1.288430in}{5.167138in}}%
\pgfpathlineto{\pgfqpoint{1.304555in}{5.130885in}}%
\pgfpathlineto{\pgfqpoint{1.300862in}{5.108938in}}%
\pgfpathlineto{\pgfqpoint{1.323509in}{5.093709in}}%
\pgfpathlineto{\pgfqpoint{1.323870in}{5.072534in}}%
\pgfpathlineto{\pgfqpoint{1.331583in}{5.050280in}}%
\pgfpathlineto{\pgfqpoint{1.345921in}{5.055295in}}%
\pgfpathlineto{\pgfqpoint{1.378775in}{5.045956in}}%
\pgfpathlineto{\pgfqpoint{1.392301in}{5.056334in}}%
\pgfpathlineto{\pgfqpoint{1.401974in}{5.049626in}}%
\pgfpathlineto{\pgfqpoint{1.421216in}{5.050465in}}%
\pgfpathlineto{\pgfqpoint{1.425343in}{5.043537in}}%
\pgfpathlineto{\pgfqpoint{1.445885in}{5.045474in}}%
\pgfpathlineto{\pgfqpoint{1.455784in}{5.059953in}}%
\pgfpathlineto{\pgfqpoint{1.469613in}{5.061717in}}%
\pgfpathlineto{\pgfqpoint{1.480372in}{5.033075in}}%
\pgfpathlineto{\pgfqpoint{1.486097in}{5.028827in}}%
\pgfpathlineto{\pgfqpoint{1.456784in}{4.863472in}}%
\pgfpathlineto{\pgfqpoint{1.436081in}{4.747212in}}%
\pgfpathlineto{\pgfqpoint{1.272793in}{4.778734in}}%
\pgfpathlineto{\pgfqpoint{1.184593in}{4.796279in}}%
\pgfpathlineto{\pgfqpoint{1.102093in}{4.814420in}}%
\pgfpathlineto{\pgfqpoint{1.024981in}{4.831878in}}%
\pgfpathlineto{\pgfqpoint{0.935738in}{4.853400in}}%
\pgfpathlineto{\pgfqpoint{0.981770in}{5.042238in}}%
\pgfpathlineto{\pgfqpoint{0.984252in}{5.056039in}}%
\pgfpathlineto{\pgfqpoint{1.004722in}{5.092200in}}%
\pgfpathlineto{\pgfqpoint{0.983526in}{5.113934in}}%
\pgfpathlineto{\pgfqpoint{0.987927in}{5.135277in}}%
\pgfpathlineto{\pgfqpoint{0.995996in}{5.140619in}}%
\pgfpathlineto{\pgfqpoint{1.010406in}{5.162611in}}%
\pgfpathlineto{\pgfqpoint{1.031412in}{5.177827in}}%
\pgfpathlineto{\pgfqpoint{1.032559in}{5.189084in}}%
\pgfpathlineto{\pgfqpoint{1.045237in}{5.200333in}}%
\pgfpathlineto{\pgfqpoint{1.055763in}{5.221387in}}%
\pgfpathlineto{\pgfqpoint{1.077356in}{5.243662in}}%
\pgfpathlineto{\pgfqpoint{1.075823in}{5.265662in}}%
\pgfpathlineto{\pgfqpoint{1.061060in}{5.277883in}}%
\pgfpathlineto{\pgfqpoint{1.054504in}{5.300440in}}%
\pgfusepath{stroke}%
\end{pgfscope}%
\begin{pgfscope}%
\pgfpathrectangle{\pgfqpoint{0.100000in}{2.413063in}}{\pgfqpoint{5.037500in}{3.427208in}}%
\pgfusepath{clip}%
\pgfsetbuttcap%
\pgfsetroundjoin%
\pgfsetlinewidth{0.501875pt}%
\definecolor{currentstroke}{rgb}{0.827451,0.827451,0.827451}%
\pgfsetstrokecolor{currentstroke}%
\pgfsetdash{}{0pt}%
\pgfpathmoveto{\pgfqpoint{4.628939in}{4.931925in}}%
\pgfpathlineto{\pgfqpoint{4.622154in}{4.953342in}}%
\pgfpathlineto{\pgfqpoint{4.609564in}{5.018312in}}%
\pgfpathlineto{\pgfqpoint{4.593285in}{5.038300in}}%
\pgfpathlineto{\pgfqpoint{4.579007in}{5.074245in}}%
\pgfpathlineto{\pgfqpoint{4.583695in}{5.100896in}}%
\pgfpathlineto{\pgfqpoint{4.579935in}{5.120538in}}%
\pgfpathlineto{\pgfqpoint{4.567736in}{5.139830in}}%
\pgfpathlineto{\pgfqpoint{4.567234in}{5.161082in}}%
\pgfpathlineto{\pgfqpoint{4.560150in}{5.184002in}}%
\pgfpathlineto{\pgfqpoint{4.623544in}{5.199604in}}%
\pgfpathlineto{\pgfqpoint{4.705886in}{5.221830in}}%
\pgfpathlineto{\pgfqpoint{4.709172in}{5.209085in}}%
\pgfpathlineto{\pgfqpoint{4.703929in}{5.188795in}}%
\pgfpathlineto{\pgfqpoint{4.716237in}{5.172560in}}%
\pgfpathlineto{\pgfqpoint{4.709700in}{5.151988in}}%
\pgfpathlineto{\pgfqpoint{4.683724in}{5.126129in}}%
\pgfpathlineto{\pgfqpoint{4.690864in}{5.106703in}}%
\pgfpathlineto{\pgfqpoint{4.685572in}{5.065644in}}%
\pgfpathlineto{\pgfqpoint{4.678164in}{5.042494in}}%
\pgfpathlineto{\pgfqpoint{4.687616in}{4.976662in}}%
\pgfpathlineto{\pgfqpoint{4.686342in}{4.953500in}}%
\pgfpathlineto{\pgfqpoint{4.695486in}{4.946066in}}%
\pgfpathlineto{\pgfqpoint{4.628939in}{4.931925in}}%
\pgfusepath{stroke}%
\end{pgfscope}%
\begin{pgfscope}%
\pgfpathrectangle{\pgfqpoint{0.100000in}{2.413063in}}{\pgfqpoint{5.037500in}{3.427208in}}%
\pgfusepath{clip}%
\pgfsetbuttcap%
\pgfsetroundjoin%
\pgfsetlinewidth{0.501875pt}%
\definecolor{currentstroke}{rgb}{0.827451,0.827451,0.827451}%
\pgfsetstrokecolor{currentstroke}%
\pgfsetdash{}{0pt}%
\pgfpathmoveto{\pgfqpoint{2.681778in}{4.800574in}}%
\pgfpathlineto{\pgfqpoint{2.684148in}{4.920604in}}%
\pgfpathlineto{\pgfqpoint{2.685563in}{5.005892in}}%
\pgfpathlineto{\pgfqpoint{2.679397in}{5.017573in}}%
\pgfpathlineto{\pgfqpoint{2.666402in}{5.021918in}}%
\pgfpathlineto{\pgfqpoint{2.654617in}{5.042254in}}%
\pgfpathlineto{\pgfqpoint{2.670006in}{5.059917in}}%
\pgfpathlineto{\pgfqpoint{2.677717in}{5.082416in}}%
\pgfpathlineto{\pgfqpoint{2.679246in}{5.101121in}}%
\pgfpathlineto{\pgfqpoint{2.675704in}{5.129137in}}%
\pgfpathlineto{\pgfqpoint{2.664964in}{5.147119in}}%
\pgfpathlineto{\pgfqpoint{2.661293in}{5.163497in}}%
\pgfpathlineto{\pgfqpoint{2.658619in}{5.224267in}}%
\pgfpathlineto{\pgfqpoint{2.658653in}{5.276579in}}%
\pgfpathlineto{\pgfqpoint{2.644203in}{5.318130in}}%
\pgfpathlineto{\pgfqpoint{2.637896in}{5.346294in}}%
\pgfpathlineto{\pgfqpoint{2.638988in}{5.383901in}}%
\pgfpathlineto{\pgfqpoint{2.642307in}{5.402197in}}%
\pgfpathlineto{\pgfqpoint{2.633036in}{5.440482in}}%
\pgfpathlineto{\pgfqpoint{2.696152in}{5.439219in}}%
\pgfpathlineto{\pgfqpoint{2.792021in}{5.438395in}}%
\pgfpathlineto{\pgfqpoint{2.792545in}{5.481909in}}%
\pgfpathlineto{\pgfqpoint{2.816948in}{5.477118in}}%
\pgfpathlineto{\pgfqpoint{2.828643in}{5.424060in}}%
\pgfpathlineto{\pgfqpoint{2.837265in}{5.404983in}}%
\pgfpathlineto{\pgfqpoint{2.858707in}{5.404420in}}%
\pgfpathlineto{\pgfqpoint{2.863506in}{5.397946in}}%
\pgfpathlineto{\pgfqpoint{2.893414in}{5.395079in}}%
\pgfpathlineto{\pgfqpoint{2.898441in}{5.381925in}}%
\pgfpathlineto{\pgfqpoint{2.919053in}{5.384882in}}%
\pgfpathlineto{\pgfqpoint{2.935062in}{5.397185in}}%
\pgfpathlineto{\pgfqpoint{2.962670in}{5.396725in}}%
\pgfpathlineto{\pgfqpoint{2.979752in}{5.386831in}}%
\pgfpathlineto{\pgfqpoint{2.981716in}{5.377552in}}%
\pgfpathlineto{\pgfqpoint{2.997970in}{5.375609in}}%
\pgfpathlineto{\pgfqpoint{3.008578in}{5.350299in}}%
\pgfpathlineto{\pgfqpoint{3.015422in}{5.365861in}}%
\pgfpathlineto{\pgfqpoint{3.034094in}{5.366818in}}%
\pgfpathlineto{\pgfqpoint{3.038817in}{5.354698in}}%
\pgfpathlineto{\pgfqpoint{3.059826in}{5.349166in}}%
\pgfpathlineto{\pgfqpoint{3.071412in}{5.331709in}}%
\pgfpathlineto{\pgfqpoint{3.097108in}{5.337129in}}%
\pgfpathlineto{\pgfqpoint{3.125382in}{5.358552in}}%
\pgfpathlineto{\pgfqpoint{3.135691in}{5.339667in}}%
\pgfpathlineto{\pgfqpoint{3.182085in}{5.344836in}}%
\pgfpathlineto{\pgfqpoint{3.201917in}{5.331874in}}%
\pgfpathlineto{\pgfqpoint{3.213460in}{5.336502in}}%
\pgfpathlineto{\pgfqpoint{3.222716in}{5.329205in}}%
\pgfpathlineto{\pgfqpoint{3.195257in}{5.311893in}}%
\pgfpathlineto{\pgfqpoint{3.156062in}{5.296468in}}%
\pgfpathlineto{\pgfqpoint{3.117242in}{5.265636in}}%
\pgfpathlineto{\pgfqpoint{3.083587in}{5.225090in}}%
\pgfpathlineto{\pgfqpoint{3.058111in}{5.201119in}}%
\pgfpathlineto{\pgfqpoint{3.035795in}{5.184642in}}%
\pgfpathlineto{\pgfqpoint{3.020445in}{5.168690in}}%
\pgfpathlineto{\pgfqpoint{3.022305in}{5.100721in}}%
\pgfpathlineto{\pgfqpoint{2.992450in}{5.081609in}}%
\pgfpathlineto{\pgfqpoint{2.977374in}{5.057440in}}%
\pgfpathlineto{\pgfqpoint{2.976370in}{5.042141in}}%
\pgfpathlineto{\pgfqpoint{2.988521in}{5.038771in}}%
\pgfpathlineto{\pgfqpoint{2.995419in}{5.021352in}}%
\pgfpathlineto{\pgfqpoint{2.987741in}{5.009815in}}%
\pgfpathlineto{\pgfqpoint{2.988521in}{4.967905in}}%
\pgfpathlineto{\pgfqpoint{2.984477in}{4.945866in}}%
\pgfpathlineto{\pgfqpoint{3.009371in}{4.926715in}}%
\pgfpathlineto{\pgfqpoint{3.023314in}{4.924526in}}%
\pgfpathlineto{\pgfqpoint{3.034824in}{4.911628in}}%
\pgfpathlineto{\pgfqpoint{3.054403in}{4.903703in}}%
\pgfpathlineto{\pgfqpoint{3.065315in}{4.884023in}}%
\pgfpathlineto{\pgfqpoint{3.086424in}{4.867390in}}%
\pgfpathlineto{\pgfqpoint{3.105822in}{4.857310in}}%
\pgfpathlineto{\pgfqpoint{3.116744in}{4.839351in}}%
\pgfpathlineto{\pgfqpoint{3.120418in}{4.806289in}}%
\pgfpathlineto{\pgfqpoint{3.017472in}{4.802551in}}%
\pgfpathlineto{\pgfqpoint{2.929720in}{4.800716in}}%
\pgfpathlineto{\pgfqpoint{2.849769in}{4.799541in}}%
\pgfpathlineto{\pgfqpoint{2.765190in}{4.799670in}}%
\pgfpathlineto{\pgfqpoint{2.681778in}{4.800574in}}%
\pgfusepath{stroke}%
\end{pgfscope}%
\begin{pgfscope}%
\pgfpathrectangle{\pgfqpoint{0.100000in}{2.413063in}}{\pgfqpoint{5.037500in}{3.427208in}}%
\pgfusepath{clip}%
\pgfsetbuttcap%
\pgfsetroundjoin%
\pgfsetlinewidth{0.501875pt}%
\definecolor{currentstroke}{rgb}{0.827451,0.827451,0.827451}%
\pgfsetstrokecolor{currentstroke}%
\pgfsetdash{}{0pt}%
\pgfpathmoveto{\pgfqpoint{0.344393in}{5.024965in}}%
\pgfpathlineto{\pgfqpoint{0.336240in}{5.039967in}}%
\pgfpathlineto{\pgfqpoint{0.341446in}{5.078475in}}%
\pgfpathlineto{\pgfqpoint{0.351493in}{5.098599in}}%
\pgfpathlineto{\pgfqpoint{0.346470in}{5.125823in}}%
\pgfpathlineto{\pgfqpoint{0.356952in}{5.137270in}}%
\pgfpathlineto{\pgfqpoint{0.376075in}{5.168128in}}%
\pgfpathlineto{\pgfqpoint{0.392289in}{5.186755in}}%
\pgfpathlineto{\pgfqpoint{0.415990in}{5.227325in}}%
\pgfpathlineto{\pgfqpoint{0.434168in}{5.271519in}}%
\pgfpathlineto{\pgfqpoint{0.453465in}{5.312967in}}%
\pgfpathlineto{\pgfqpoint{0.457366in}{5.330279in}}%
\pgfpathlineto{\pgfqpoint{0.483987in}{5.379801in}}%
\pgfpathlineto{\pgfqpoint{0.489113in}{5.401551in}}%
\pgfpathlineto{\pgfqpoint{0.500447in}{5.424392in}}%
\pgfpathlineto{\pgfqpoint{0.504499in}{5.452254in}}%
\pgfpathlineto{\pgfqpoint{0.526137in}{5.465840in}}%
\pgfpathlineto{\pgfqpoint{0.551794in}{5.472687in}}%
\pgfpathlineto{\pgfqpoint{0.564765in}{5.457351in}}%
\pgfpathlineto{\pgfqpoint{0.575981in}{5.458485in}}%
\pgfpathlineto{\pgfqpoint{0.593490in}{5.440205in}}%
\pgfpathlineto{\pgfqpoint{0.595818in}{5.422986in}}%
\pgfpathlineto{\pgfqpoint{0.589177in}{5.389269in}}%
\pgfpathlineto{\pgfqpoint{0.611269in}{5.372066in}}%
\pgfpathlineto{\pgfqpoint{0.625598in}{5.365598in}}%
\pgfpathlineto{\pgfqpoint{0.664413in}{5.372200in}}%
\pgfpathlineto{\pgfqpoint{0.686901in}{5.367691in}}%
\pgfpathlineto{\pgfqpoint{0.702357in}{5.360292in}}%
\pgfpathlineto{\pgfqpoint{0.710499in}{5.346247in}}%
\pgfpathlineto{\pgfqpoint{0.762052in}{5.348120in}}%
\pgfpathlineto{\pgfqpoint{0.770250in}{5.339526in}}%
\pgfpathlineto{\pgfqpoint{0.789590in}{5.337671in}}%
\pgfpathlineto{\pgfqpoint{0.809066in}{5.343162in}}%
\pgfpathlineto{\pgfqpoint{0.866649in}{5.341778in}}%
\pgfpathlineto{\pgfqpoint{0.879086in}{5.337547in}}%
\pgfpathlineto{\pgfqpoint{0.893620in}{5.342423in}}%
\pgfpathlineto{\pgfqpoint{1.054504in}{5.300440in}}%
\pgfpathlineto{\pgfqpoint{1.061060in}{5.277883in}}%
\pgfpathlineto{\pgfqpoint{1.075823in}{5.265662in}}%
\pgfpathlineto{\pgfqpoint{1.077356in}{5.243662in}}%
\pgfpathlineto{\pgfqpoint{1.055763in}{5.221387in}}%
\pgfpathlineto{\pgfqpoint{1.045237in}{5.200333in}}%
\pgfpathlineto{\pgfqpoint{1.032559in}{5.189084in}}%
\pgfpathlineto{\pgfqpoint{1.031412in}{5.177827in}}%
\pgfpathlineto{\pgfqpoint{1.010406in}{5.162611in}}%
\pgfpathlineto{\pgfqpoint{0.995996in}{5.140619in}}%
\pgfpathlineto{\pgfqpoint{0.987927in}{5.135277in}}%
\pgfpathlineto{\pgfqpoint{0.983526in}{5.113934in}}%
\pgfpathlineto{\pgfqpoint{1.004722in}{5.092200in}}%
\pgfpathlineto{\pgfqpoint{0.984252in}{5.056039in}}%
\pgfpathlineto{\pgfqpoint{0.981770in}{5.042238in}}%
\pgfpathlineto{\pgfqpoint{0.935738in}{4.853400in}}%
\pgfpathlineto{\pgfqpoint{0.838915in}{4.878210in}}%
\pgfpathlineto{\pgfqpoint{0.745509in}{4.902295in}}%
\pgfpathlineto{\pgfqpoint{0.689161in}{4.917975in}}%
\pgfpathlineto{\pgfqpoint{0.616760in}{4.938599in}}%
\pgfpathlineto{\pgfqpoint{0.501273in}{4.974949in}}%
\pgfpathlineto{\pgfqpoint{0.375733in}{5.014034in}}%
\pgfpathlineto{\pgfqpoint{0.344393in}{5.024965in}}%
\pgfusepath{stroke}%
\end{pgfscope}%
\begin{pgfscope}%
\pgfpathrectangle{\pgfqpoint{0.100000in}{2.413063in}}{\pgfqpoint{5.037500in}{3.427208in}}%
\pgfusepath{clip}%
\pgfsetbuttcap%
\pgfsetroundjoin%
\pgfsetlinewidth{0.501875pt}%
\definecolor{currentstroke}{rgb}{0.827451,0.827451,0.827451}%
\pgfsetstrokecolor{currentstroke}%
\pgfsetdash{}{0pt}%
\pgfpathmoveto{\pgfqpoint{4.695486in}{4.946066in}}%
\pgfpathlineto{\pgfqpoint{4.686342in}{4.953500in}}%
\pgfpathlineto{\pgfqpoint{4.687616in}{4.976662in}}%
\pgfpathlineto{\pgfqpoint{4.678164in}{5.042494in}}%
\pgfpathlineto{\pgfqpoint{4.685572in}{5.065644in}}%
\pgfpathlineto{\pgfqpoint{4.690864in}{5.106703in}}%
\pgfpathlineto{\pgfqpoint{4.683724in}{5.126129in}}%
\pgfpathlineto{\pgfqpoint{4.709700in}{5.151988in}}%
\pgfpathlineto{\pgfqpoint{4.716237in}{5.172560in}}%
\pgfpathlineto{\pgfqpoint{4.703929in}{5.188795in}}%
\pgfpathlineto{\pgfqpoint{4.709172in}{5.209085in}}%
\pgfpathlineto{\pgfqpoint{4.705886in}{5.221830in}}%
\pgfpathlineto{\pgfqpoint{4.708708in}{5.249125in}}%
\pgfpathlineto{\pgfqpoint{4.713985in}{5.257552in}}%
\pgfpathlineto{\pgfqpoint{4.730253in}{5.262870in}}%
\pgfpathlineto{\pgfqpoint{4.753984in}{5.193372in}}%
\pgfpathlineto{\pgfqpoint{4.798011in}{5.049357in}}%
\pgfpathlineto{\pgfqpoint{4.814672in}{5.038191in}}%
\pgfpathlineto{\pgfqpoint{4.817670in}{5.025444in}}%
\pgfpathlineto{\pgfqpoint{4.826463in}{5.020294in}}%
\pgfpathlineto{\pgfqpoint{4.825794in}{4.997176in}}%
\pgfpathlineto{\pgfqpoint{4.816477in}{4.996805in}}%
\pgfpathlineto{\pgfqpoint{4.797601in}{4.982409in}}%
\pgfpathlineto{\pgfqpoint{4.792157in}{4.967966in}}%
\pgfpathlineto{\pgfqpoint{4.741634in}{4.955466in}}%
\pgfpathlineto{\pgfqpoint{4.695486in}{4.946066in}}%
\pgfusepath{stroke}%
\end{pgfscope}%
\begin{pgfscope}%
\pgfpathrectangle{\pgfqpoint{0.100000in}{2.413063in}}{\pgfqpoint{5.037500in}{3.427208in}}%
\pgfusepath{clip}%
\pgfsetbuttcap%
\pgfsetroundjoin%
\pgfsetlinewidth{0.501875pt}%
\definecolor{currentstroke}{rgb}{0.827451,0.827451,0.827451}%
\pgfsetstrokecolor{currentstroke}%
\pgfsetdash{}{0pt}%
\pgfpathmoveto{\pgfqpoint{3.115663in}{4.445025in}}%
\pgfpathlineto{\pgfqpoint{3.088992in}{4.471439in}}%
\pgfpathlineto{\pgfqpoint{3.003734in}{4.466521in}}%
\pgfpathlineto{\pgfqpoint{2.870765in}{4.462140in}}%
\pgfpathlineto{\pgfqpoint{2.736996in}{4.464227in}}%
\pgfpathlineto{\pgfqpoint{2.727607in}{4.480599in}}%
\pgfpathlineto{\pgfqpoint{2.731440in}{4.496676in}}%
\pgfpathlineto{\pgfqpoint{2.729620in}{4.524217in}}%
\pgfpathlineto{\pgfqpoint{2.722550in}{4.550893in}}%
\pgfpathlineto{\pgfqpoint{2.723372in}{4.564980in}}%
\pgfpathlineto{\pgfqpoint{2.708345in}{4.580897in}}%
\pgfpathlineto{\pgfqpoint{2.711614in}{4.603044in}}%
\pgfpathlineto{\pgfqpoint{2.688563in}{4.646790in}}%
\pgfpathlineto{\pgfqpoint{2.681695in}{4.683929in}}%
\pgfpathlineto{\pgfqpoint{2.665113in}{4.714137in}}%
\pgfpathlineto{\pgfqpoint{2.680853in}{4.751610in}}%
\pgfpathlineto{\pgfqpoint{2.673013in}{4.769048in}}%
\pgfpathlineto{\pgfqpoint{2.681778in}{4.800574in}}%
\pgfpathlineto{\pgfqpoint{2.765190in}{4.799670in}}%
\pgfpathlineto{\pgfqpoint{2.849769in}{4.799541in}}%
\pgfpathlineto{\pgfqpoint{2.929720in}{4.800716in}}%
\pgfpathlineto{\pgfqpoint{3.017472in}{4.802551in}}%
\pgfpathlineto{\pgfqpoint{3.120418in}{4.806289in}}%
\pgfpathlineto{\pgfqpoint{3.123167in}{4.791344in}}%
\pgfpathlineto{\pgfqpoint{3.134564in}{4.778814in}}%
\pgfpathlineto{\pgfqpoint{3.127348in}{4.765756in}}%
\pgfpathlineto{\pgfqpoint{3.130306in}{4.740271in}}%
\pgfpathlineto{\pgfqpoint{3.137369in}{4.719670in}}%
\pgfpathlineto{\pgfqpoint{3.169391in}{4.708766in}}%
\pgfpathlineto{\pgfqpoint{3.174878in}{4.694179in}}%
\pgfpathlineto{\pgfqpoint{3.192451in}{4.677802in}}%
\pgfpathlineto{\pgfqpoint{3.199619in}{4.660860in}}%
\pgfpathlineto{\pgfqpoint{3.217424in}{4.649495in}}%
\pgfpathlineto{\pgfqpoint{3.220210in}{4.635813in}}%
\pgfpathlineto{\pgfqpoint{3.216743in}{4.615099in}}%
\pgfpathlineto{\pgfqpoint{3.207681in}{4.608891in}}%
\pgfpathlineto{\pgfqpoint{3.204948in}{4.589174in}}%
\pgfpathlineto{\pgfqpoint{3.178904in}{4.573514in}}%
\pgfpathlineto{\pgfqpoint{3.144957in}{4.564935in}}%
\pgfpathlineto{\pgfqpoint{3.141846in}{4.545209in}}%
\pgfpathlineto{\pgfqpoint{3.154942in}{4.531116in}}%
\pgfpathlineto{\pgfqpoint{3.155476in}{4.513396in}}%
\pgfpathlineto{\pgfqpoint{3.144911in}{4.499468in}}%
\pgfpathlineto{\pgfqpoint{3.139365in}{4.478775in}}%
\pgfpathlineto{\pgfqpoint{3.120999in}{4.471925in}}%
\pgfpathlineto{\pgfqpoint{3.122170in}{4.448866in}}%
\pgfpathlineto{\pgfqpoint{3.115663in}{4.445025in}}%
\pgfusepath{stroke}%
\end{pgfscope}%
\begin{pgfscope}%
\pgfpathrectangle{\pgfqpoint{0.100000in}{2.413063in}}{\pgfqpoint{5.037500in}{3.427208in}}%
\pgfusepath{clip}%
\pgfsetbuttcap%
\pgfsetroundjoin%
\pgfsetlinewidth{0.501875pt}%
\definecolor{currentstroke}{rgb}{0.827451,0.827451,0.827451}%
\pgfsetstrokecolor{currentstroke}%
\pgfsetdash{}{0pt}%
\pgfpathmoveto{\pgfqpoint{4.770516in}{4.879426in}}%
\pgfpathlineto{\pgfqpoint{4.745254in}{4.875440in}}%
\pgfpathlineto{\pgfqpoint{4.629215in}{4.849072in}}%
\pgfpathlineto{\pgfqpoint{4.627187in}{4.852145in}}%
\pgfpathlineto{\pgfqpoint{4.628939in}{4.931925in}}%
\pgfpathlineto{\pgfqpoint{4.695486in}{4.946066in}}%
\pgfpathlineto{\pgfqpoint{4.741634in}{4.955466in}}%
\pgfpathlineto{\pgfqpoint{4.792157in}{4.967966in}}%
\pgfpathlineto{\pgfqpoint{4.797601in}{4.982409in}}%
\pgfpathlineto{\pgfqpoint{4.816477in}{4.996805in}}%
\pgfpathlineto{\pgfqpoint{4.825794in}{4.997176in}}%
\pgfpathlineto{\pgfqpoint{4.838043in}{4.976174in}}%
\pgfpathlineto{\pgfqpoint{4.826929in}{4.945516in}}%
\pgfpathlineto{\pgfqpoint{4.825298in}{4.927539in}}%
\pgfpathlineto{\pgfqpoint{4.847794in}{4.929227in}}%
\pgfpathlineto{\pgfqpoint{4.857997in}{4.920604in}}%
\pgfpathlineto{\pgfqpoint{4.880876in}{4.885354in}}%
\pgfpathlineto{\pgfqpoint{4.892247in}{4.881067in}}%
\pgfpathlineto{\pgfqpoint{4.911295in}{4.882657in}}%
\pgfpathlineto{\pgfqpoint{4.924550in}{4.894636in}}%
\pgfpathlineto{\pgfqpoint{4.933361in}{4.883945in}}%
\pgfpathlineto{\pgfqpoint{4.877952in}{4.854706in}}%
\pgfpathlineto{\pgfqpoint{4.876220in}{4.875687in}}%
\pgfpathlineto{\pgfqpoint{4.851050in}{4.843078in}}%
\pgfpathlineto{\pgfqpoint{4.842200in}{4.837460in}}%
\pgfpathlineto{\pgfqpoint{4.829857in}{4.856276in}}%
\pgfpathlineto{\pgfqpoint{4.826478in}{4.858855in}}%
\pgfpathlineto{\pgfqpoint{4.814987in}{4.864941in}}%
\pgfpathlineto{\pgfqpoint{4.804966in}{4.889628in}}%
\pgfpathlineto{\pgfqpoint{4.770516in}{4.879426in}}%
\pgfusepath{stroke}%
\end{pgfscope}%
\begin{pgfscope}%
\pgfpathrectangle{\pgfqpoint{0.100000in}{2.413063in}}{\pgfqpoint{5.037500in}{3.427208in}}%
\pgfusepath{clip}%
\pgfsetbuttcap%
\pgfsetroundjoin%
\pgfsetlinewidth{0.501875pt}%
\definecolor{currentstroke}{rgb}{0.827451,0.827451,0.827451}%
\pgfsetstrokecolor{currentstroke}%
\pgfsetdash{}{0pt}%
\pgfpathmoveto{\pgfqpoint{2.182662in}{4.420100in}}%
\pgfpathlineto{\pgfqpoint{2.191913in}{4.534814in}}%
\pgfpathlineto{\pgfqpoint{2.139519in}{4.539067in}}%
\pgfpathlineto{\pgfqpoint{2.018405in}{4.550818in}}%
\pgfpathlineto{\pgfqpoint{2.029673in}{4.665125in}}%
\pgfpathlineto{\pgfqpoint{2.040958in}{4.780305in}}%
\pgfpathlineto{\pgfqpoint{2.087517in}{4.775278in}}%
\pgfpathlineto{\pgfqpoint{2.206426in}{4.764043in}}%
\pgfpathlineto{\pgfqpoint{2.278390in}{4.758408in}}%
\pgfpathlineto{\pgfqpoint{2.365512in}{4.753414in}}%
\pgfpathlineto{\pgfqpoint{2.508914in}{4.746627in}}%
\pgfpathlineto{\pgfqpoint{2.512066in}{4.740198in}}%
\pgfpathlineto{\pgfqpoint{2.553155in}{4.719869in}}%
\pgfpathlineto{\pgfqpoint{2.565756in}{4.730193in}}%
\pgfpathlineto{\pgfqpoint{2.602058in}{4.729251in}}%
\pgfpathlineto{\pgfqpoint{2.623692in}{4.717918in}}%
\pgfpathlineto{\pgfqpoint{2.657995in}{4.705035in}}%
\pgfpathlineto{\pgfqpoint{2.667607in}{4.686241in}}%
\pgfpathlineto{\pgfqpoint{2.681695in}{4.683929in}}%
\pgfpathlineto{\pgfqpoint{2.688563in}{4.646790in}}%
\pgfpathlineto{\pgfqpoint{2.711614in}{4.603044in}}%
\pgfpathlineto{\pgfqpoint{2.708345in}{4.580897in}}%
\pgfpathlineto{\pgfqpoint{2.723372in}{4.564980in}}%
\pgfpathlineto{\pgfqpoint{2.722550in}{4.550893in}}%
\pgfpathlineto{\pgfqpoint{2.729620in}{4.524217in}}%
\pgfpathlineto{\pgfqpoint{2.731440in}{4.496676in}}%
\pgfpathlineto{\pgfqpoint{2.727607in}{4.480599in}}%
\pgfpathlineto{\pgfqpoint{2.736996in}{4.464227in}}%
\pgfpathlineto{\pgfqpoint{2.749875in}{4.434454in}}%
\pgfpathlineto{\pgfqpoint{2.762201in}{4.422337in}}%
\pgfpathlineto{\pgfqpoint{2.776895in}{4.396079in}}%
\pgfpathlineto{\pgfqpoint{2.735250in}{4.395647in}}%
\pgfpathlineto{\pgfqpoint{2.645206in}{4.397040in}}%
\pgfpathlineto{\pgfqpoint{2.545721in}{4.400088in}}%
\pgfpathlineto{\pgfqpoint{2.445651in}{4.403928in}}%
\pgfpathlineto{\pgfqpoint{2.298517in}{4.411965in}}%
\pgfpathlineto{\pgfqpoint{2.182662in}{4.420100in}}%
\pgfusepath{stroke}%
\end{pgfscope}%
\begin{pgfscope}%
\pgfpathrectangle{\pgfqpoint{0.100000in}{2.413063in}}{\pgfqpoint{5.037500in}{3.427208in}}%
\pgfusepath{clip}%
\pgfsetbuttcap%
\pgfsetroundjoin%
\pgfsetlinewidth{0.501875pt}%
\definecolor{currentstroke}{rgb}{0.827451,0.827451,0.827451}%
\pgfsetstrokecolor{currentstroke}%
\pgfsetdash{}{0pt}%
\pgfpathmoveto{\pgfqpoint{4.098414in}{4.766121in}}%
\pgfpathlineto{\pgfqpoint{4.144485in}{4.810050in}}%
\pgfpathlineto{\pgfqpoint{4.150212in}{4.825665in}}%
\pgfpathlineto{\pgfqpoint{4.163712in}{4.839033in}}%
\pgfpathlineto{\pgfqpoint{4.153582in}{4.858496in}}%
\pgfpathlineto{\pgfqpoint{4.140881in}{4.869880in}}%
\pgfpathlineto{\pgfqpoint{4.137210in}{4.890054in}}%
\pgfpathlineto{\pgfqpoint{4.184483in}{4.910768in}}%
\pgfpathlineto{\pgfqpoint{4.223622in}{4.917320in}}%
\pgfpathlineto{\pgfqpoint{4.244653in}{4.917758in}}%
\pgfpathlineto{\pgfqpoint{4.260673in}{4.909843in}}%
\pgfpathlineto{\pgfqpoint{4.276290in}{4.916911in}}%
\pgfpathlineto{\pgfqpoint{4.314380in}{4.924832in}}%
\pgfpathlineto{\pgfqpoint{4.327541in}{4.935056in}}%
\pgfpathlineto{\pgfqpoint{4.347056in}{4.957689in}}%
\pgfpathlineto{\pgfqpoint{4.364809in}{4.967690in}}%
\pgfpathlineto{\pgfqpoint{4.366061in}{4.977283in}}%
\pgfpathlineto{\pgfqpoint{4.356744in}{4.999166in}}%
\pgfpathlineto{\pgfqpoint{4.363480in}{5.012045in}}%
\pgfpathlineto{\pgfqpoint{4.354410in}{5.025892in}}%
\pgfpathlineto{\pgfqpoint{4.340522in}{5.026859in}}%
\pgfpathlineto{\pgfqpoint{4.375282in}{5.068688in}}%
\pgfpathlineto{\pgfqpoint{4.379411in}{5.084629in}}%
\pgfpathlineto{\pgfqpoint{4.406769in}{5.125284in}}%
\pgfpathlineto{\pgfqpoint{4.432165in}{5.147284in}}%
\pgfpathlineto{\pgfqpoint{4.449604in}{5.156471in}}%
\pgfpathlineto{\pgfqpoint{4.506667in}{5.169401in}}%
\pgfpathlineto{\pgfqpoint{4.560150in}{5.184002in}}%
\pgfpathlineto{\pgfqpoint{4.567234in}{5.161082in}}%
\pgfpathlineto{\pgfqpoint{4.567736in}{5.139830in}}%
\pgfpathlineto{\pgfqpoint{4.579935in}{5.120538in}}%
\pgfpathlineto{\pgfqpoint{4.583695in}{5.100896in}}%
\pgfpathlineto{\pgfqpoint{4.579007in}{5.074245in}}%
\pgfpathlineto{\pgfqpoint{4.593285in}{5.038300in}}%
\pgfpathlineto{\pgfqpoint{4.609564in}{5.018312in}}%
\pgfpathlineto{\pgfqpoint{4.622154in}{4.953342in}}%
\pgfpathlineto{\pgfqpoint{4.628939in}{4.931925in}}%
\pgfpathlineto{\pgfqpoint{4.627187in}{4.852145in}}%
\pgfpathlineto{\pgfqpoint{4.629215in}{4.849072in}}%
\pgfpathlineto{\pgfqpoint{4.644037in}{4.763268in}}%
\pgfpathlineto{\pgfqpoint{4.652350in}{4.755449in}}%
\pgfpathlineto{\pgfqpoint{4.634470in}{4.738078in}}%
\pgfpathlineto{\pgfqpoint{4.643292in}{4.728107in}}%
\pgfpathlineto{\pgfqpoint{4.635538in}{4.713027in}}%
\pgfpathlineto{\pgfqpoint{4.635604in}{4.706606in}}%
\pgfpathlineto{\pgfqpoint{4.625910in}{4.700813in}}%
\pgfpathlineto{\pgfqpoint{4.621189in}{4.688007in}}%
\pgfpathlineto{\pgfqpoint{4.622686in}{4.723221in}}%
\pgfpathlineto{\pgfqpoint{4.592639in}{4.730974in}}%
\pgfpathlineto{\pgfqpoint{4.545650in}{4.747002in}}%
\pgfpathlineto{\pgfqpoint{4.539004in}{4.754897in}}%
\pgfpathlineto{\pgfqpoint{4.519368in}{4.756789in}}%
\pgfpathlineto{\pgfqpoint{4.508073in}{4.768540in}}%
\pgfpathlineto{\pgfqpoint{4.501979in}{4.791943in}}%
\pgfpathlineto{\pgfqpoint{4.485967in}{4.794854in}}%
\pgfpathlineto{\pgfqpoint{4.475258in}{4.807131in}}%
\pgfpathlineto{\pgfqpoint{4.338937in}{4.779648in}}%
\pgfpathlineto{\pgfqpoint{4.273785in}{4.766174in}}%
\pgfpathlineto{\pgfqpoint{4.174866in}{4.748119in}}%
\pgfpathlineto{\pgfqpoint{4.103631in}{4.736102in}}%
\pgfpathlineto{\pgfqpoint{4.098414in}{4.766121in}}%
\pgfusepath{stroke}%
\end{pgfscope}%
\begin{pgfscope}%
\pgfpathrectangle{\pgfqpoint{0.100000in}{2.413063in}}{\pgfqpoint{5.037500in}{3.427208in}}%
\pgfusepath{clip}%
\pgfsetbuttcap%
\pgfsetroundjoin%
\pgfsetlinewidth{0.501875pt}%
\definecolor{currentstroke}{rgb}{0.827451,0.827451,0.827451}%
\pgfsetstrokecolor{currentstroke}%
\pgfsetdash{}{0pt}%
\pgfpathmoveto{\pgfqpoint{4.645692in}{4.680855in}}%
\pgfpathlineto{\pgfqpoint{4.624609in}{4.674287in}}%
\pgfpathlineto{\pgfqpoint{4.621062in}{4.680325in}}%
\pgfpathlineto{\pgfqpoint{4.627829in}{4.700594in}}%
\pgfpathlineto{\pgfqpoint{4.640362in}{4.702589in}}%
\pgfpathlineto{\pgfqpoint{4.639240in}{4.708971in}}%
\pgfpathlineto{\pgfqpoint{4.650493in}{4.718549in}}%
\pgfpathlineto{\pgfqpoint{4.683026in}{4.726179in}}%
\pgfpathlineto{\pgfqpoint{4.697509in}{4.737730in}}%
\pgfpathlineto{\pgfqpoint{4.730045in}{4.747355in}}%
\pgfpathlineto{\pgfqpoint{4.744884in}{4.743833in}}%
\pgfpathlineto{\pgfqpoint{4.757367in}{4.759352in}}%
\pgfpathlineto{\pgfqpoint{4.776234in}{4.761403in}}%
\pgfpathlineto{\pgfqpoint{4.744046in}{4.731132in}}%
\pgfpathlineto{\pgfqpoint{4.645692in}{4.680855in}}%
\pgfusepath{stroke}%
\end{pgfscope}%
\begin{pgfscope}%
\pgfpathrectangle{\pgfqpoint{0.100000in}{2.413063in}}{\pgfqpoint{5.037500in}{3.427208in}}%
\pgfusepath{clip}%
\pgfsetbuttcap%
\pgfsetroundjoin%
\pgfsetlinewidth{0.501875pt}%
\definecolor{currentstroke}{rgb}{0.827451,0.827451,0.827451}%
\pgfsetstrokecolor{currentstroke}%
\pgfsetdash{}{0pt}%
\pgfpathmoveto{\pgfqpoint{4.172033in}{4.480971in}}%
\pgfpathlineto{\pgfqpoint{4.080863in}{4.465897in}}%
\pgfpathlineto{\pgfqpoint{4.064326in}{4.570087in}}%
\pgfpathlineto{\pgfqpoint{4.039771in}{4.723670in}}%
\pgfpathlineto{\pgfqpoint{4.098414in}{4.766121in}}%
\pgfpathlineto{\pgfqpoint{4.103631in}{4.736102in}}%
\pgfpathlineto{\pgfqpoint{4.174866in}{4.748119in}}%
\pgfpathlineto{\pgfqpoint{4.273785in}{4.766174in}}%
\pgfpathlineto{\pgfqpoint{4.338937in}{4.779648in}}%
\pgfpathlineto{\pgfqpoint{4.475258in}{4.807131in}}%
\pgfpathlineto{\pgfqpoint{4.485967in}{4.794854in}}%
\pgfpathlineto{\pgfqpoint{4.501979in}{4.791943in}}%
\pgfpathlineto{\pgfqpoint{4.508073in}{4.768540in}}%
\pgfpathlineto{\pgfqpoint{4.519368in}{4.756789in}}%
\pgfpathlineto{\pgfqpoint{4.539004in}{4.754897in}}%
\pgfpathlineto{\pgfqpoint{4.545650in}{4.747002in}}%
\pgfpathlineto{\pgfqpoint{4.538903in}{4.740911in}}%
\pgfpathlineto{\pgfqpoint{4.532824in}{4.719359in}}%
\pgfpathlineto{\pgfqpoint{4.518581in}{4.695143in}}%
\pgfpathlineto{\pgfqpoint{4.528142in}{4.684611in}}%
\pgfpathlineto{\pgfqpoint{4.520318in}{4.673328in}}%
\pgfpathlineto{\pgfqpoint{4.524732in}{4.648584in}}%
\pgfpathlineto{\pgfqpoint{4.538821in}{4.635595in}}%
\pgfpathlineto{\pgfqpoint{4.572254in}{4.614493in}}%
\pgfpathlineto{\pgfqpoint{4.545330in}{4.584718in}}%
\pgfpathlineto{\pgfqpoint{4.544953in}{4.573417in}}%
\pgfpathlineto{\pgfqpoint{4.523034in}{4.558842in}}%
\pgfpathlineto{\pgfqpoint{4.498797in}{4.556107in}}%
\pgfpathlineto{\pgfqpoint{4.492802in}{4.543439in}}%
\pgfpathlineto{\pgfqpoint{4.425387in}{4.528852in}}%
\pgfpathlineto{\pgfqpoint{4.346722in}{4.513009in}}%
\pgfpathlineto{\pgfqpoint{4.292651in}{4.503327in}}%
\pgfpathlineto{\pgfqpoint{4.172033in}{4.480971in}}%
\pgfusepath{stroke}%
\end{pgfscope}%
\begin{pgfscope}%
\pgfpathrectangle{\pgfqpoint{0.100000in}{2.413063in}}{\pgfqpoint{5.037500in}{3.427208in}}%
\pgfusepath{clip}%
\pgfsetbuttcap%
\pgfsetroundjoin%
\pgfsetlinewidth{0.501875pt}%
\definecolor{currentstroke}{rgb}{0.827451,0.827451,0.827451}%
\pgfsetstrokecolor{currentstroke}%
\pgfsetdash{}{0pt}%
\pgfpathmoveto{\pgfqpoint{4.629215in}{4.849072in}}%
\pgfpathlineto{\pgfqpoint{4.745254in}{4.875440in}}%
\pgfpathlineto{\pgfqpoint{4.770516in}{4.879426in}}%
\pgfpathlineto{\pgfqpoint{4.787236in}{4.813688in}}%
\pgfpathlineto{\pgfqpoint{4.784590in}{4.801916in}}%
\pgfpathlineto{\pgfqpoint{4.730879in}{4.781122in}}%
\pgfpathlineto{\pgfqpoint{4.698815in}{4.773854in}}%
\pgfpathlineto{\pgfqpoint{4.685191in}{4.757551in}}%
\pgfpathlineto{\pgfqpoint{4.643292in}{4.728107in}}%
\pgfpathlineto{\pgfqpoint{4.634470in}{4.738078in}}%
\pgfpathlineto{\pgfqpoint{4.652350in}{4.755449in}}%
\pgfpathlineto{\pgfqpoint{4.644037in}{4.763268in}}%
\pgfpathlineto{\pgfqpoint{4.629215in}{4.849072in}}%
\pgfusepath{stroke}%
\end{pgfscope}%
\begin{pgfscope}%
\pgfpathrectangle{\pgfqpoint{0.100000in}{2.413063in}}{\pgfqpoint{5.037500in}{3.427208in}}%
\pgfusepath{clip}%
\pgfsetbuttcap%
\pgfsetroundjoin%
\pgfsetlinewidth{0.501875pt}%
\definecolor{currentstroke}{rgb}{0.827451,0.827451,0.827451}%
\pgfsetstrokecolor{currentstroke}%
\pgfsetdash{}{0pt}%
\pgfpathmoveto{\pgfqpoint{4.770516in}{4.879426in}}%
\pgfpathlineto{\pgfqpoint{4.804966in}{4.889628in}}%
\pgfpathlineto{\pgfqpoint{4.814987in}{4.864941in}}%
\pgfpathlineto{\pgfqpoint{4.826478in}{4.858855in}}%
\pgfpathlineto{\pgfqpoint{4.812298in}{4.848456in}}%
\pgfpathlineto{\pgfqpoint{4.814086in}{4.817915in}}%
\pgfpathlineto{\pgfqpoint{4.784590in}{4.801916in}}%
\pgfpathlineto{\pgfqpoint{4.787236in}{4.813688in}}%
\pgfpathlineto{\pgfqpoint{4.770516in}{4.879426in}}%
\pgfusepath{stroke}%
\end{pgfscope}%
\begin{pgfscope}%
\pgfpathrectangle{\pgfqpoint{0.100000in}{2.413063in}}{\pgfqpoint{5.037500in}{3.427208in}}%
\pgfusepath{clip}%
\pgfsetbuttcap%
\pgfsetroundjoin%
\pgfsetlinewidth{0.501875pt}%
\definecolor{currentstroke}{rgb}{0.827451,0.827451,0.827451}%
\pgfsetstrokecolor{currentstroke}%
\pgfsetdash{}{0pt}%
\pgfpathmoveto{\pgfqpoint{4.519150in}{4.548078in}}%
\pgfpathlineto{\pgfqpoint{4.523034in}{4.558842in}}%
\pgfpathlineto{\pgfqpoint{4.544953in}{4.573417in}}%
\pgfpathlineto{\pgfqpoint{4.545330in}{4.584718in}}%
\pgfpathlineto{\pgfqpoint{4.572254in}{4.614493in}}%
\pgfpathlineto{\pgfqpoint{4.538821in}{4.635595in}}%
\pgfpathlineto{\pgfqpoint{4.524732in}{4.648584in}}%
\pgfpathlineto{\pgfqpoint{4.520318in}{4.673328in}}%
\pgfpathlineto{\pgfqpoint{4.528142in}{4.684611in}}%
\pgfpathlineto{\pgfqpoint{4.518581in}{4.695143in}}%
\pgfpathlineto{\pgfqpoint{4.532824in}{4.719359in}}%
\pgfpathlineto{\pgfqpoint{4.538903in}{4.740911in}}%
\pgfpathlineto{\pgfqpoint{4.545650in}{4.747002in}}%
\pgfpathlineto{\pgfqpoint{4.592639in}{4.730974in}}%
\pgfpathlineto{\pgfqpoint{4.622686in}{4.723221in}}%
\pgfpathlineto{\pgfqpoint{4.621189in}{4.688007in}}%
\pgfpathlineto{\pgfqpoint{4.602937in}{4.661303in}}%
\pgfpathlineto{\pgfqpoint{4.617977in}{4.657374in}}%
\pgfpathlineto{\pgfqpoint{4.633592in}{4.645915in}}%
\pgfpathlineto{\pgfqpoint{4.634513in}{4.614571in}}%
\pgfpathlineto{\pgfqpoint{4.629787in}{4.592383in}}%
\pgfpathlineto{\pgfqpoint{4.632932in}{4.574157in}}%
\pgfpathlineto{\pgfqpoint{4.605787in}{4.512656in}}%
\pgfpathlineto{\pgfqpoint{4.596032in}{4.483939in}}%
\pgfpathlineto{\pgfqpoint{4.582411in}{4.497898in}}%
\pgfpathlineto{\pgfqpoint{4.564348in}{4.495519in}}%
\pgfpathlineto{\pgfqpoint{4.519141in}{4.521637in}}%
\pgfpathlineto{\pgfqpoint{4.514516in}{4.535624in}}%
\pgfpathlineto{\pgfqpoint{4.519150in}{4.548078in}}%
\pgfusepath{stroke}%
\end{pgfscope}%
\begin{pgfscope}%
\pgfpathrectangle{\pgfqpoint{0.100000in}{2.413063in}}{\pgfqpoint{5.037500in}{3.427208in}}%
\pgfusepath{clip}%
\pgfsetbuttcap%
\pgfsetroundjoin%
\pgfsetlinewidth{0.501875pt}%
\definecolor{currentstroke}{rgb}{0.827451,0.827451,0.827451}%
\pgfsetstrokecolor{currentstroke}%
\pgfsetdash{}{0pt}%
\pgfpathmoveto{\pgfqpoint{3.437265in}{4.167692in}}%
\pgfpathlineto{\pgfqpoint{3.435151in}{4.196223in}}%
\pgfpathlineto{\pgfqpoint{3.443366in}{4.210378in}}%
\pgfpathlineto{\pgfqpoint{3.437951in}{4.217314in}}%
\pgfpathlineto{\pgfqpoint{3.457695in}{4.240023in}}%
\pgfpathlineto{\pgfqpoint{3.456564in}{4.245496in}}%
\pgfpathlineto{\pgfqpoint{3.475746in}{4.285114in}}%
\pgfpathlineto{\pgfqpoint{3.471869in}{4.304221in}}%
\pgfpathlineto{\pgfqpoint{3.457959in}{4.324225in}}%
\pgfpathlineto{\pgfqpoint{3.467615in}{4.348563in}}%
\pgfpathlineto{\pgfqpoint{3.455099in}{4.508720in}}%
\pgfpathlineto{\pgfqpoint{3.446087in}{4.621075in}}%
\pgfpathlineto{\pgfqpoint{3.458537in}{4.611765in}}%
\pgfpathlineto{\pgfqpoint{3.472426in}{4.612017in}}%
\pgfpathlineto{\pgfqpoint{3.505233in}{4.631021in}}%
\pgfpathlineto{\pgfqpoint{3.605857in}{4.640409in}}%
\pgfpathlineto{\pgfqpoint{3.680336in}{4.648259in}}%
\pgfpathlineto{\pgfqpoint{3.680994in}{4.640971in}}%
\pgfpathlineto{\pgfqpoint{3.697996in}{4.487004in}}%
\pgfpathlineto{\pgfqpoint{3.712630in}{4.343743in}}%
\pgfpathlineto{\pgfqpoint{3.706322in}{4.337017in}}%
\pgfpathlineto{\pgfqpoint{3.716011in}{4.320323in}}%
\pgfpathlineto{\pgfqpoint{3.715962in}{4.308292in}}%
\pgfpathlineto{\pgfqpoint{3.702135in}{4.305269in}}%
\pgfpathlineto{\pgfqpoint{3.686663in}{4.293671in}}%
\pgfpathlineto{\pgfqpoint{3.676188in}{4.298237in}}%
\pgfpathlineto{\pgfqpoint{3.660509in}{4.290813in}}%
\pgfpathlineto{\pgfqpoint{3.665360in}{4.275902in}}%
\pgfpathlineto{\pgfqpoint{3.649244in}{4.260922in}}%
\pgfpathlineto{\pgfqpoint{3.644795in}{4.243592in}}%
\pgfpathlineto{\pgfqpoint{3.633714in}{4.240750in}}%
\pgfpathlineto{\pgfqpoint{3.625445in}{4.227626in}}%
\pgfpathlineto{\pgfqpoint{3.626527in}{4.214409in}}%
\pgfpathlineto{\pgfqpoint{3.616798in}{4.205115in}}%
\pgfpathlineto{\pgfqpoint{3.602160in}{4.206550in}}%
\pgfpathlineto{\pgfqpoint{3.591032in}{4.220810in}}%
\pgfpathlineto{\pgfqpoint{3.572178in}{4.207078in}}%
\pgfpathlineto{\pgfqpoint{3.573500in}{4.195078in}}%
\pgfpathlineto{\pgfqpoint{3.552531in}{4.188067in}}%
\pgfpathlineto{\pgfqpoint{3.547262in}{4.196892in}}%
\pgfpathlineto{\pgfqpoint{3.526847in}{4.186886in}}%
\pgfpathlineto{\pgfqpoint{3.519389in}{4.172561in}}%
\pgfpathlineto{\pgfqpoint{3.494803in}{4.187250in}}%
\pgfpathlineto{\pgfqpoint{3.455836in}{4.177519in}}%
\pgfpathlineto{\pgfqpoint{3.437265in}{4.167692in}}%
\pgfusepath{stroke}%
\end{pgfscope}%
\begin{pgfscope}%
\pgfpathrectangle{\pgfqpoint{0.100000in}{2.413063in}}{\pgfqpoint{5.037500in}{3.427208in}}%
\pgfusepath{clip}%
\pgfsetbuttcap%
\pgfsetroundjoin%
\pgfsetlinewidth{0.501875pt}%
\definecolor{currentstroke}{rgb}{0.827451,0.827451,0.827451}%
\pgfsetstrokecolor{currentstroke}%
\pgfsetdash{}{0pt}%
\pgfpathmoveto{\pgfqpoint{0.689161in}{4.917975in}}%
\pgfpathlineto{\pgfqpoint{0.745509in}{4.902295in}}%
\pgfpathlineto{\pgfqpoint{0.838915in}{4.878210in}}%
\pgfpathlineto{\pgfqpoint{0.935738in}{4.853400in}}%
\pgfpathlineto{\pgfqpoint{1.024981in}{4.831878in}}%
\pgfpathlineto{\pgfqpoint{1.102093in}{4.814420in}}%
\pgfpathlineto{\pgfqpoint{1.184593in}{4.796279in}}%
\pgfpathlineto{\pgfqpoint{1.160756in}{4.683789in}}%
\pgfpathlineto{\pgfqpoint{1.139521in}{4.583892in}}%
\pgfpathlineto{\pgfqpoint{1.104685in}{4.422710in}}%
\pgfpathlineto{\pgfqpoint{1.078531in}{4.301052in}}%
\pgfpathlineto{\pgfqpoint{1.064402in}{4.233160in}}%
\pgfpathlineto{\pgfqpoint{1.046284in}{4.145043in}}%
\pgfpathlineto{\pgfqpoint{1.033749in}{4.127168in}}%
\pgfpathlineto{\pgfqpoint{1.023660in}{4.126570in}}%
\pgfpathlineto{\pgfqpoint{1.016464in}{4.142175in}}%
\pgfpathlineto{\pgfqpoint{0.999941in}{4.147841in}}%
\pgfpathlineto{\pgfqpoint{0.982136in}{4.145855in}}%
\pgfpathlineto{\pgfqpoint{0.977083in}{4.133092in}}%
\pgfpathlineto{\pgfqpoint{0.976556in}{4.088769in}}%
\pgfpathlineto{\pgfqpoint{0.971224in}{4.078649in}}%
\pgfpathlineto{\pgfqpoint{0.974271in}{4.043066in}}%
\pgfpathlineto{\pgfqpoint{0.963137in}{4.019360in}}%
\pgfpathlineto{\pgfqpoint{0.872849in}{4.158390in}}%
\pgfpathlineto{\pgfqpoint{0.783931in}{4.293488in}}%
\pgfpathlineto{\pgfqpoint{0.737645in}{4.364331in}}%
\pgfpathlineto{\pgfqpoint{0.699033in}{4.425478in}}%
\pgfpathlineto{\pgfqpoint{0.650381in}{4.501117in}}%
\pgfpathlineto{\pgfqpoint{0.595308in}{4.585963in}}%
\pgfpathlineto{\pgfqpoint{0.617950in}{4.666497in}}%
\pgfpathlineto{\pgfqpoint{0.663516in}{4.828009in}}%
\pgfpathlineto{\pgfqpoint{0.689161in}{4.917975in}}%
\pgfusepath{stroke}%
\end{pgfscope}%
\begin{pgfscope}%
\pgfpathrectangle{\pgfqpoint{0.100000in}{2.413063in}}{\pgfqpoint{5.037500in}{3.427208in}}%
\pgfusepath{clip}%
\pgfsetbuttcap%
\pgfsetroundjoin%
\pgfsetlinewidth{0.501875pt}%
\definecolor{currentstroke}{rgb}{0.827451,0.827451,0.827451}%
\pgfsetstrokecolor{currentstroke}%
\pgfsetdash{}{0pt}%
\pgfpathmoveto{\pgfqpoint{1.184593in}{4.796279in}}%
\pgfpathlineto{\pgfqpoint{1.272793in}{4.778734in}}%
\pgfpathlineto{\pgfqpoint{1.436081in}{4.747212in}}%
\pgfpathlineto{\pgfqpoint{1.415642in}{4.633831in}}%
\pgfpathlineto{\pgfqpoint{1.505537in}{4.618440in}}%
\pgfpathlineto{\pgfqpoint{1.587436in}{4.605334in}}%
\pgfpathlineto{\pgfqpoint{1.573205in}{4.515697in}}%
\pgfpathlineto{\pgfqpoint{1.558125in}{4.419035in}}%
\pgfpathlineto{\pgfqpoint{1.537936in}{4.292117in}}%
\pgfpathlineto{\pgfqpoint{1.537414in}{4.281477in}}%
\pgfpathlineto{\pgfqpoint{1.516457in}{4.149943in}}%
\pgfpathlineto{\pgfqpoint{1.430188in}{4.163315in}}%
\pgfpathlineto{\pgfqpoint{1.386229in}{4.172143in}}%
\pgfpathlineto{\pgfqpoint{1.227336in}{4.200074in}}%
\pgfpathlineto{\pgfqpoint{1.167500in}{4.211874in}}%
\pgfpathlineto{\pgfqpoint{1.064402in}{4.233160in}}%
\pgfpathlineto{\pgfqpoint{1.078531in}{4.301052in}}%
\pgfpathlineto{\pgfqpoint{1.104685in}{4.422710in}}%
\pgfpathlineto{\pgfqpoint{1.139521in}{4.583892in}}%
\pgfpathlineto{\pgfqpoint{1.160756in}{4.683789in}}%
\pgfpathlineto{\pgfqpoint{1.184593in}{4.796279in}}%
\pgfusepath{stroke}%
\end{pgfscope}%
\begin{pgfscope}%
\pgfpathrectangle{\pgfqpoint{0.100000in}{2.413063in}}{\pgfqpoint{5.037500in}{3.427208in}}%
\pgfusepath{clip}%
\pgfsetbuttcap%
\pgfsetroundjoin%
\pgfsetlinewidth{0.501875pt}%
\definecolor{currentstroke}{rgb}{0.827451,0.827451,0.827451}%
\pgfsetstrokecolor{currentstroke}%
\pgfsetdash{}{0pt}%
\pgfpathmoveto{\pgfqpoint{0.344393in}{5.024965in}}%
\pgfpathlineto{\pgfqpoint{0.375733in}{5.014034in}}%
\pgfpathlineto{\pgfqpoint{0.501273in}{4.974949in}}%
\pgfpathlineto{\pgfqpoint{0.616760in}{4.938599in}}%
\pgfpathlineto{\pgfqpoint{0.689161in}{4.917975in}}%
\pgfpathlineto{\pgfqpoint{0.663516in}{4.828009in}}%
\pgfpathlineto{\pgfqpoint{0.617950in}{4.666497in}}%
\pgfpathlineto{\pgfqpoint{0.595308in}{4.585963in}}%
\pgfpathlineto{\pgfqpoint{0.650381in}{4.501117in}}%
\pgfpathlineto{\pgfqpoint{0.699033in}{4.425478in}}%
\pgfpathlineto{\pgfqpoint{0.737645in}{4.364331in}}%
\pgfpathlineto{\pgfqpoint{0.783931in}{4.293488in}}%
\pgfpathlineto{\pgfqpoint{0.872849in}{4.158390in}}%
\pgfpathlineto{\pgfqpoint{0.963137in}{4.019360in}}%
\pgfpathlineto{\pgfqpoint{0.959513in}{4.005573in}}%
\pgfpathlineto{\pgfqpoint{0.970403in}{3.983615in}}%
\pgfpathlineto{\pgfqpoint{0.972557in}{3.953579in}}%
\pgfpathlineto{\pgfqpoint{0.988373in}{3.939008in}}%
\pgfpathlineto{\pgfqpoint{0.988999in}{3.927268in}}%
\pgfpathlineto{\pgfqpoint{0.960679in}{3.913943in}}%
\pgfpathlineto{\pgfqpoint{0.947187in}{3.900602in}}%
\pgfpathlineto{\pgfqpoint{0.942979in}{3.871143in}}%
\pgfpathlineto{\pgfqpoint{0.936150in}{3.855078in}}%
\pgfpathlineto{\pgfqpoint{0.921760in}{3.841535in}}%
\pgfpathlineto{\pgfqpoint{0.907448in}{3.806322in}}%
\pgfpathlineto{\pgfqpoint{0.927572in}{3.787979in}}%
\pgfpathlineto{\pgfqpoint{0.924974in}{3.772858in}}%
\pgfpathlineto{\pgfqpoint{0.908473in}{3.762335in}}%
\pgfpathlineto{\pgfqpoint{0.897097in}{3.764192in}}%
\pgfpathlineto{\pgfqpoint{0.762813in}{3.782827in}}%
\pgfpathlineto{\pgfqpoint{0.663632in}{3.796898in}}%
\pgfpathlineto{\pgfqpoint{0.667969in}{3.812911in}}%
\pgfpathlineto{\pgfqpoint{0.661512in}{3.839496in}}%
\pgfpathlineto{\pgfqpoint{0.660861in}{3.866312in}}%
\pgfpathlineto{\pgfqpoint{0.656642in}{3.882005in}}%
\pgfpathlineto{\pgfqpoint{0.643643in}{3.904441in}}%
\pgfpathlineto{\pgfqpoint{0.606255in}{3.956207in}}%
\pgfpathlineto{\pgfqpoint{0.594010in}{3.962550in}}%
\pgfpathlineto{\pgfqpoint{0.578232in}{3.962450in}}%
\pgfpathlineto{\pgfqpoint{0.581837in}{3.978780in}}%
\pgfpathlineto{\pgfqpoint{0.574366in}{3.999201in}}%
\pgfpathlineto{\pgfqpoint{0.537669in}{4.009351in}}%
\pgfpathlineto{\pgfqpoint{0.515300in}{4.028145in}}%
\pgfpathlineto{\pgfqpoint{0.513445in}{4.039644in}}%
\pgfpathlineto{\pgfqpoint{0.487688in}{4.068115in}}%
\pgfpathlineto{\pgfqpoint{0.463170in}{4.073562in}}%
\pgfpathlineto{\pgfqpoint{0.440454in}{4.088031in}}%
\pgfpathlineto{\pgfqpoint{0.410577in}{4.093037in}}%
\pgfpathlineto{\pgfqpoint{0.397812in}{4.112333in}}%
\pgfpathlineto{\pgfqpoint{0.409935in}{4.142878in}}%
\pgfpathlineto{\pgfqpoint{0.406227in}{4.149746in}}%
\pgfpathlineto{\pgfqpoint{0.416322in}{4.175209in}}%
\pgfpathlineto{\pgfqpoint{0.398364in}{4.188760in}}%
\pgfpathlineto{\pgfqpoint{0.404191in}{4.213326in}}%
\pgfpathlineto{\pgfqpoint{0.394598in}{4.219606in}}%
\pgfpathlineto{\pgfqpoint{0.386316in}{4.242826in}}%
\pgfpathlineto{\pgfqpoint{0.376322in}{4.249907in}}%
\pgfpathlineto{\pgfqpoint{0.375570in}{4.266679in}}%
\pgfpathlineto{\pgfqpoint{0.367740in}{4.278505in}}%
\pgfpathlineto{\pgfqpoint{0.355940in}{4.318337in}}%
\pgfpathlineto{\pgfqpoint{0.343030in}{4.337414in}}%
\pgfpathlineto{\pgfqpoint{0.345836in}{4.369800in}}%
\pgfpathlineto{\pgfqpoint{0.361018in}{4.373059in}}%
\pgfpathlineto{\pgfqpoint{0.370904in}{4.390624in}}%
\pgfpathlineto{\pgfqpoint{0.364931in}{4.409676in}}%
\pgfpathlineto{\pgfqpoint{0.348784in}{4.412856in}}%
\pgfpathlineto{\pgfqpoint{0.340727in}{4.421767in}}%
\pgfpathlineto{\pgfqpoint{0.327682in}{4.454544in}}%
\pgfpathlineto{\pgfqpoint{0.333779in}{4.466316in}}%
\pgfpathlineto{\pgfqpoint{0.329454in}{4.488234in}}%
\pgfpathlineto{\pgfqpoint{0.339031in}{4.516622in}}%
\pgfpathlineto{\pgfqpoint{0.350249in}{4.506168in}}%
\pgfpathlineto{\pgfqpoint{0.345158in}{4.493826in}}%
\pgfpathlineto{\pgfqpoint{0.364600in}{4.474261in}}%
\pgfpathlineto{\pgfqpoint{0.363359in}{4.503314in}}%
\pgfpathlineto{\pgfqpoint{0.355028in}{4.511106in}}%
\pgfpathlineto{\pgfqpoint{0.364550in}{4.536651in}}%
\pgfpathlineto{\pgfqpoint{0.400566in}{4.532583in}}%
\pgfpathlineto{\pgfqpoint{0.395719in}{4.542082in}}%
\pgfpathlineto{\pgfqpoint{0.371948in}{4.541147in}}%
\pgfpathlineto{\pgfqpoint{0.360641in}{4.555535in}}%
\pgfpathlineto{\pgfqpoint{0.346407in}{4.542758in}}%
\pgfpathlineto{\pgfqpoint{0.338854in}{4.521403in}}%
\pgfpathlineto{\pgfqpoint{0.318696in}{4.550139in}}%
\pgfpathlineto{\pgfqpoint{0.310932in}{4.555371in}}%
\pgfpathlineto{\pgfqpoint{0.313889in}{4.586646in}}%
\pgfpathlineto{\pgfqpoint{0.307746in}{4.605089in}}%
\pgfpathlineto{\pgfqpoint{0.296607in}{4.622417in}}%
\pgfpathlineto{\pgfqpoint{0.273798in}{4.675509in}}%
\pgfpathlineto{\pgfqpoint{0.281255in}{4.687249in}}%
\pgfpathlineto{\pgfqpoint{0.281168in}{4.724352in}}%
\pgfpathlineto{\pgfqpoint{0.293478in}{4.745047in}}%
\pgfpathlineto{\pgfqpoint{0.296301in}{4.777389in}}%
\pgfpathlineto{\pgfqpoint{0.284693in}{4.814445in}}%
\pgfpathlineto{\pgfqpoint{0.269293in}{4.838032in}}%
\pgfpathlineto{\pgfqpoint{0.272040in}{4.859316in}}%
\pgfpathlineto{\pgfqpoint{0.315286in}{4.910828in}}%
\pgfpathlineto{\pgfqpoint{0.317434in}{4.928400in}}%
\pgfpathlineto{\pgfqpoint{0.336921in}{4.961920in}}%
\pgfpathlineto{\pgfqpoint{0.339624in}{4.993692in}}%
\pgfpathlineto{\pgfqpoint{0.333359in}{5.001799in}}%
\pgfpathlineto{\pgfqpoint{0.344393in}{5.024965in}}%
\pgfusepath{stroke}%
\end{pgfscope}%
\begin{pgfscope}%
\pgfpathrectangle{\pgfqpoint{0.100000in}{2.413063in}}{\pgfqpoint{5.037500in}{3.427208in}}%
\pgfusepath{clip}%
\pgfsetbuttcap%
\pgfsetroundjoin%
\pgfsetlinewidth{0.501875pt}%
\definecolor{currentstroke}{rgb}{0.827451,0.827451,0.827451}%
\pgfsetstrokecolor{currentstroke}%
\pgfsetdash{}{0pt}%
\pgfpathmoveto{\pgfqpoint{3.712630in}{4.343743in}}%
\pgfpathlineto{\pgfqpoint{3.697996in}{4.487004in}}%
\pgfpathlineto{\pgfqpoint{3.680994in}{4.640971in}}%
\pgfpathlineto{\pgfqpoint{3.792336in}{4.657553in}}%
\pgfpathlineto{\pgfqpoint{3.821886in}{4.649862in}}%
\pgfpathlineto{\pgfqpoint{3.836059in}{4.641513in}}%
\pgfpathlineto{\pgfqpoint{3.853819in}{4.643829in}}%
\pgfpathlineto{\pgfqpoint{3.877234in}{4.630006in}}%
\pgfpathlineto{\pgfqpoint{3.920850in}{4.650579in}}%
\pgfpathlineto{\pgfqpoint{3.944919in}{4.651266in}}%
\pgfpathlineto{\pgfqpoint{3.973027in}{4.682624in}}%
\pgfpathlineto{\pgfqpoint{4.001625in}{4.701697in}}%
\pgfpathlineto{\pgfqpoint{4.039771in}{4.723670in}}%
\pgfpathlineto{\pgfqpoint{4.064326in}{4.570087in}}%
\pgfpathlineto{\pgfqpoint{4.052917in}{4.560224in}}%
\pgfpathlineto{\pgfqpoint{4.060283in}{4.551164in}}%
\pgfpathlineto{\pgfqpoint{4.063217in}{4.531303in}}%
\pgfpathlineto{\pgfqpoint{4.057501in}{4.512649in}}%
\pgfpathlineto{\pgfqpoint{4.056746in}{4.485383in}}%
\pgfpathlineto{\pgfqpoint{4.051378in}{4.449895in}}%
\pgfpathlineto{\pgfqpoint{4.024221in}{4.418243in}}%
\pgfpathlineto{\pgfqpoint{4.012789in}{4.411525in}}%
\pgfpathlineto{\pgfqpoint{4.003899in}{4.417309in}}%
\pgfpathlineto{\pgfqpoint{3.981897in}{4.387141in}}%
\pgfpathlineto{\pgfqpoint{3.983958in}{4.358094in}}%
\pgfpathlineto{\pgfqpoint{3.959456in}{4.365080in}}%
\pgfpathlineto{\pgfqpoint{3.947853in}{4.336069in}}%
\pgfpathlineto{\pgfqpoint{3.953718in}{4.315512in}}%
\pgfpathlineto{\pgfqpoint{3.944537in}{4.312475in}}%
\pgfpathlineto{\pgfqpoint{3.943252in}{4.296232in}}%
\pgfpathlineto{\pgfqpoint{3.920717in}{4.289680in}}%
\pgfpathlineto{\pgfqpoint{3.909006in}{4.302795in}}%
\pgfpathlineto{\pgfqpoint{3.893938in}{4.307880in}}%
\pgfpathlineto{\pgfqpoint{3.890398in}{4.321178in}}%
\pgfpathlineto{\pgfqpoint{3.876773in}{4.318839in}}%
\pgfpathlineto{\pgfqpoint{3.867848in}{4.306608in}}%
\pgfpathlineto{\pgfqpoint{3.855079in}{4.302307in}}%
\pgfpathlineto{\pgfqpoint{3.832524in}{4.310962in}}%
\pgfpathlineto{\pgfqpoint{3.820049in}{4.300664in}}%
\pgfpathlineto{\pgfqpoint{3.802320in}{4.312848in}}%
\pgfpathlineto{\pgfqpoint{3.768306in}{4.316588in}}%
\pgfpathlineto{\pgfqpoint{3.758047in}{4.338723in}}%
\pgfpathlineto{\pgfqpoint{3.745048in}{4.348530in}}%
\pgfpathlineto{\pgfqpoint{3.732447in}{4.342252in}}%
\pgfpathlineto{\pgfqpoint{3.712630in}{4.343743in}}%
\pgfusepath{stroke}%
\end{pgfscope}%
\begin{pgfscope}%
\pgfpathrectangle{\pgfqpoint{0.100000in}{2.413063in}}{\pgfqpoint{5.037500in}{3.427208in}}%
\pgfusepath{clip}%
\pgfsetbuttcap%
\pgfsetroundjoin%
\pgfsetlinewidth{0.501875pt}%
\definecolor{currentstroke}{rgb}{0.827451,0.827451,0.827451}%
\pgfsetstrokecolor{currentstroke}%
\pgfsetdash{}{0pt}%
\pgfpathmoveto{\pgfqpoint{3.344010in}{4.066681in}}%
\pgfpathlineto{\pgfqpoint{3.330964in}{4.077278in}}%
\pgfpathlineto{\pgfqpoint{3.320328in}{4.072236in}}%
\pgfpathlineto{\pgfqpoint{3.306729in}{4.097628in}}%
\pgfpathlineto{\pgfqpoint{3.313676in}{4.113588in}}%
\pgfpathlineto{\pgfqpoint{3.303664in}{4.131552in}}%
\pgfpathlineto{\pgfqpoint{3.303126in}{4.145694in}}%
\pgfpathlineto{\pgfqpoint{3.289597in}{4.150734in}}%
\pgfpathlineto{\pgfqpoint{3.283323in}{4.161390in}}%
\pgfpathlineto{\pgfqpoint{3.256870in}{4.174698in}}%
\pgfpathlineto{\pgfqpoint{3.233897in}{4.191095in}}%
\pgfpathlineto{\pgfqpoint{3.223222in}{4.203474in}}%
\pgfpathlineto{\pgfqpoint{3.223001in}{4.218571in}}%
\pgfpathlineto{\pgfqpoint{3.237285in}{4.247559in}}%
\pgfpathlineto{\pgfqpoint{3.241679in}{4.269731in}}%
\pgfpathlineto{\pgfqpoint{3.230042in}{4.282258in}}%
\pgfpathlineto{\pgfqpoint{3.214629in}{4.286979in}}%
\pgfpathlineto{\pgfqpoint{3.195930in}{4.276657in}}%
\pgfpathlineto{\pgfqpoint{3.188016in}{4.294377in}}%
\pgfpathlineto{\pgfqpoint{3.184022in}{4.318399in}}%
\pgfpathlineto{\pgfqpoint{3.156465in}{4.339812in}}%
\pgfpathlineto{\pgfqpoint{3.150959in}{4.349315in}}%
\pgfpathlineto{\pgfqpoint{3.125789in}{4.370826in}}%
\pgfpathlineto{\pgfqpoint{3.117900in}{4.386465in}}%
\pgfpathlineto{\pgfqpoint{3.110789in}{4.417479in}}%
\pgfpathlineto{\pgfqpoint{3.115663in}{4.445025in}}%
\pgfpathlineto{\pgfqpoint{3.122170in}{4.448866in}}%
\pgfpathlineto{\pgfqpoint{3.120999in}{4.471925in}}%
\pgfpathlineto{\pgfqpoint{3.139365in}{4.478775in}}%
\pgfpathlineto{\pgfqpoint{3.144911in}{4.499468in}}%
\pgfpathlineto{\pgfqpoint{3.155476in}{4.513396in}}%
\pgfpathlineto{\pgfqpoint{3.154942in}{4.531116in}}%
\pgfpathlineto{\pgfqpoint{3.141846in}{4.545209in}}%
\pgfpathlineto{\pgfqpoint{3.144957in}{4.564935in}}%
\pgfpathlineto{\pgfqpoint{3.178904in}{4.573514in}}%
\pgfpathlineto{\pgfqpoint{3.204948in}{4.589174in}}%
\pgfpathlineto{\pgfqpoint{3.207681in}{4.608891in}}%
\pgfpathlineto{\pgfqpoint{3.216743in}{4.615099in}}%
\pgfpathlineto{\pgfqpoint{3.220210in}{4.635813in}}%
\pgfpathlineto{\pgfqpoint{3.217424in}{4.649495in}}%
\pgfpathlineto{\pgfqpoint{3.199619in}{4.660860in}}%
\pgfpathlineto{\pgfqpoint{3.192451in}{4.677802in}}%
\pgfpathlineto{\pgfqpoint{3.174878in}{4.694179in}}%
\pgfpathlineto{\pgfqpoint{3.319282in}{4.700333in}}%
\pgfpathlineto{\pgfqpoint{3.416161in}{4.707215in}}%
\pgfpathlineto{\pgfqpoint{3.414389in}{4.686836in}}%
\pgfpathlineto{\pgfqpoint{3.430899in}{4.658727in}}%
\pgfpathlineto{\pgfqpoint{3.437831in}{4.634711in}}%
\pgfpathlineto{\pgfqpoint{3.446087in}{4.621075in}}%
\pgfpathlineto{\pgfqpoint{3.455099in}{4.508720in}}%
\pgfpathlineto{\pgfqpoint{3.467615in}{4.348563in}}%
\pgfpathlineto{\pgfqpoint{3.457959in}{4.324225in}}%
\pgfpathlineto{\pgfqpoint{3.471869in}{4.304221in}}%
\pgfpathlineto{\pgfqpoint{3.475746in}{4.285114in}}%
\pgfpathlineto{\pgfqpoint{3.456564in}{4.245496in}}%
\pgfpathlineto{\pgfqpoint{3.457695in}{4.240023in}}%
\pgfpathlineto{\pgfqpoint{3.437951in}{4.217314in}}%
\pgfpathlineto{\pgfqpoint{3.443366in}{4.210378in}}%
\pgfpathlineto{\pgfqpoint{3.435151in}{4.196223in}}%
\pgfpathlineto{\pgfqpoint{3.437265in}{4.167692in}}%
\pgfpathlineto{\pgfqpoint{3.427288in}{4.150184in}}%
\pgfpathlineto{\pgfqpoint{3.435402in}{4.129497in}}%
\pgfpathlineto{\pgfqpoint{3.401404in}{4.118255in}}%
\pgfpathlineto{\pgfqpoint{3.398271in}{4.106030in}}%
\pgfpathlineto{\pgfqpoint{3.407555in}{4.090534in}}%
\pgfpathlineto{\pgfqpoint{3.403283in}{4.080434in}}%
\pgfpathlineto{\pgfqpoint{3.366827in}{4.092910in}}%
\pgfpathlineto{\pgfqpoint{3.354799in}{4.094166in}}%
\pgfpathlineto{\pgfqpoint{3.339802in}{4.075199in}}%
\pgfpathlineto{\pgfqpoint{3.344010in}{4.066681in}}%
\pgfusepath{stroke}%
\end{pgfscope}%
\begin{pgfscope}%
\pgfpathrectangle{\pgfqpoint{0.100000in}{2.413063in}}{\pgfqpoint{5.037500in}{3.427208in}}%
\pgfusepath{clip}%
\pgfsetbuttcap%
\pgfsetroundjoin%
\pgfsetlinewidth{0.501875pt}%
\definecolor{currentstroke}{rgb}{0.827451,0.827451,0.827451}%
\pgfsetstrokecolor{currentstroke}%
\pgfsetdash{}{0pt}%
\pgfpathmoveto{\pgfqpoint{4.405194in}{4.415246in}}%
\pgfpathlineto{\pgfqpoint{4.395161in}{4.430143in}}%
\pgfpathlineto{\pgfqpoint{4.400822in}{4.438480in}}%
\pgfpathlineto{\pgfqpoint{4.414687in}{4.429117in}}%
\pgfpathlineto{\pgfqpoint{4.405194in}{4.415246in}}%
\pgfusepath{stroke}%
\end{pgfscope}%
\begin{pgfscope}%
\pgfpathrectangle{\pgfqpoint{0.100000in}{2.413063in}}{\pgfqpoint{5.037500in}{3.427208in}}%
\pgfusepath{clip}%
\pgfsetbuttcap%
\pgfsetroundjoin%
\pgfsetlinewidth{0.501875pt}%
\definecolor{currentstroke}{rgb}{0.827451,0.827451,0.827451}%
\pgfsetstrokecolor{currentstroke}%
\pgfsetdash{}{0pt}%
\pgfpathmoveto{\pgfqpoint{4.492802in}{4.543439in}}%
\pgfpathlineto{\pgfqpoint{4.498797in}{4.556107in}}%
\pgfpathlineto{\pgfqpoint{4.523034in}{4.558842in}}%
\pgfpathlineto{\pgfqpoint{4.519150in}{4.548078in}}%
\pgfpathlineto{\pgfqpoint{4.511154in}{4.534325in}}%
\pgfpathlineto{\pgfqpoint{4.516572in}{4.517941in}}%
\pgfpathlineto{\pgfqpoint{4.537956in}{4.498309in}}%
\pgfpathlineto{\pgfqpoint{4.542919in}{4.477636in}}%
\pgfpathlineto{\pgfqpoint{4.567556in}{4.451893in}}%
\pgfpathlineto{\pgfqpoint{4.577221in}{4.452995in}}%
\pgfpathlineto{\pgfqpoint{4.589261in}{4.414349in}}%
\pgfpathlineto{\pgfqpoint{4.587286in}{4.413964in}}%
\pgfpathlineto{\pgfqpoint{4.585091in}{4.413533in}}%
\pgfpathlineto{\pgfqpoint{4.531394in}{4.403253in}}%
\pgfpathlineto{\pgfqpoint{4.502665in}{4.505425in}}%
\pgfpathlineto{\pgfqpoint{4.492802in}{4.543439in}}%
\pgfusepath{stroke}%
\end{pgfscope}%
\begin{pgfscope}%
\pgfpathrectangle{\pgfqpoint{0.100000in}{2.413063in}}{\pgfqpoint{5.037500in}{3.427208in}}%
\pgfusepath{clip}%
\pgfsetbuttcap%
\pgfsetroundjoin%
\pgfsetlinewidth{0.501875pt}%
\definecolor{currentstroke}{rgb}{0.827451,0.827451,0.827451}%
\pgfsetstrokecolor{currentstroke}%
\pgfsetdash{}{0pt}%
\pgfpathmoveto{\pgfqpoint{3.991255in}{4.197116in}}%
\pgfpathlineto{\pgfqpoint{3.974287in}{4.197740in}}%
\pgfpathlineto{\pgfqpoint{3.958652in}{4.208510in}}%
\pgfpathlineto{\pgfqpoint{3.937530in}{4.241244in}}%
\pgfpathlineto{\pgfqpoint{3.919550in}{4.258580in}}%
\pgfpathlineto{\pgfqpoint{3.924249in}{4.271969in}}%
\pgfpathlineto{\pgfqpoint{3.920717in}{4.289680in}}%
\pgfpathlineto{\pgfqpoint{3.943252in}{4.296232in}}%
\pgfpathlineto{\pgfqpoint{3.944537in}{4.312475in}}%
\pgfpathlineto{\pgfqpoint{3.953718in}{4.315512in}}%
\pgfpathlineto{\pgfqpoint{3.947853in}{4.336069in}}%
\pgfpathlineto{\pgfqpoint{3.959456in}{4.365080in}}%
\pgfpathlineto{\pgfqpoint{3.983958in}{4.358094in}}%
\pgfpathlineto{\pgfqpoint{3.981897in}{4.387141in}}%
\pgfpathlineto{\pgfqpoint{4.003899in}{4.417309in}}%
\pgfpathlineto{\pgfqpoint{4.012789in}{4.411525in}}%
\pgfpathlineto{\pgfqpoint{4.024221in}{4.418243in}}%
\pgfpathlineto{\pgfqpoint{4.051378in}{4.449895in}}%
\pgfpathlineto{\pgfqpoint{4.056746in}{4.485383in}}%
\pgfpathlineto{\pgfqpoint{4.057501in}{4.512649in}}%
\pgfpathlineto{\pgfqpoint{4.063217in}{4.531303in}}%
\pgfpathlineto{\pgfqpoint{4.060283in}{4.551164in}}%
\pgfpathlineto{\pgfqpoint{4.052917in}{4.560224in}}%
\pgfpathlineto{\pgfqpoint{4.064326in}{4.570087in}}%
\pgfpathlineto{\pgfqpoint{4.080863in}{4.465897in}}%
\pgfpathlineto{\pgfqpoint{4.172033in}{4.480971in}}%
\pgfpathlineto{\pgfqpoint{4.181483in}{4.421503in}}%
\pgfpathlineto{\pgfqpoint{4.214489in}{4.460755in}}%
\pgfpathlineto{\pgfqpoint{4.222266in}{4.456841in}}%
\pgfpathlineto{\pgfqpoint{4.233707in}{4.479585in}}%
\pgfpathlineto{\pgfqpoint{4.249557in}{4.472454in}}%
\pgfpathlineto{\pgfqpoint{4.263377in}{4.473810in}}%
\pgfpathlineto{\pgfqpoint{4.272494in}{4.489620in}}%
\pgfpathlineto{\pgfqpoint{4.285729in}{4.498416in}}%
\pgfpathlineto{\pgfqpoint{4.304128in}{4.490670in}}%
\pgfpathlineto{\pgfqpoint{4.316235in}{4.493347in}}%
\pgfpathlineto{\pgfqpoint{4.333579in}{4.463330in}}%
\pgfpathlineto{\pgfqpoint{4.328563in}{4.440605in}}%
\pgfpathlineto{\pgfqpoint{4.283263in}{4.466191in}}%
\pgfpathlineto{\pgfqpoint{4.274778in}{4.445187in}}%
\pgfpathlineto{\pgfqpoint{4.277577in}{4.435523in}}%
\pgfpathlineto{\pgfqpoint{4.258297in}{4.402854in}}%
\pgfpathlineto{\pgfqpoint{4.249198in}{4.396356in}}%
\pgfpathlineto{\pgfqpoint{4.245088in}{4.381950in}}%
\pgfpathlineto{\pgfqpoint{4.229575in}{4.383457in}}%
\pgfpathlineto{\pgfqpoint{4.218435in}{4.344138in}}%
\pgfpathlineto{\pgfqpoint{4.212199in}{4.335115in}}%
\pgfpathlineto{\pgfqpoint{4.196175in}{4.338124in}}%
\pgfpathlineto{\pgfqpoint{4.179784in}{4.350525in}}%
\pgfpathlineto{\pgfqpoint{4.179217in}{4.331491in}}%
\pgfpathlineto{\pgfqpoint{4.163350in}{4.299502in}}%
\pgfpathlineto{\pgfqpoint{4.159403in}{4.276729in}}%
\pgfpathlineto{\pgfqpoint{4.147229in}{4.261562in}}%
\pgfpathlineto{\pgfqpoint{4.137999in}{4.237341in}}%
\pgfpathlineto{\pgfqpoint{4.137545in}{4.214919in}}%
\pgfpathlineto{\pgfqpoint{4.123291in}{4.210732in}}%
\pgfpathlineto{\pgfqpoint{4.107121in}{4.198039in}}%
\pgfpathlineto{\pgfqpoint{4.091551in}{4.195621in}}%
\pgfpathlineto{\pgfqpoint{4.087963in}{4.184878in}}%
\pgfpathlineto{\pgfqpoint{4.062876in}{4.173876in}}%
\pgfpathlineto{\pgfqpoint{4.048856in}{4.183250in}}%
\pgfpathlineto{\pgfqpoint{4.033186in}{4.165450in}}%
\pgfpathlineto{\pgfqpoint{4.006250in}{4.170697in}}%
\pgfpathlineto{\pgfqpoint{3.989729in}{4.189379in}}%
\pgfpathlineto{\pgfqpoint{3.991255in}{4.197116in}}%
\pgfusepath{stroke}%
\end{pgfscope}%
\begin{pgfscope}%
\pgfpathrectangle{\pgfqpoint{0.100000in}{2.413063in}}{\pgfqpoint{5.037500in}{3.427208in}}%
\pgfusepath{clip}%
\pgfsetbuttcap%
\pgfsetroundjoin%
\pgfsetlinewidth{0.501875pt}%
\definecolor{currentstroke}{rgb}{0.827451,0.827451,0.827451}%
\pgfsetstrokecolor{currentstroke}%
\pgfsetdash{}{0pt}%
\pgfpathmoveto{\pgfqpoint{4.585091in}{4.413533in}}%
\pgfpathlineto{\pgfqpoint{4.584403in}{4.392537in}}%
\pgfpathlineto{\pgfqpoint{4.576322in}{4.382215in}}%
\pgfpathlineto{\pgfqpoint{4.571165in}{4.359301in}}%
\pgfpathlineto{\pgfqpoint{4.547880in}{4.348739in}}%
\pgfpathlineto{\pgfqpoint{4.528343in}{4.345661in}}%
\pgfpathlineto{\pgfqpoint{4.534036in}{4.360731in}}%
\pgfpathlineto{\pgfqpoint{4.503955in}{4.373358in}}%
\pgfpathlineto{\pgfqpoint{4.479516in}{4.389169in}}%
\pgfpathlineto{\pgfqpoint{4.494646in}{4.412802in}}%
\pgfpathlineto{\pgfqpoint{4.482520in}{4.431173in}}%
\pgfpathlineto{\pgfqpoint{4.483948in}{4.447018in}}%
\pgfpathlineto{\pgfqpoint{4.475180in}{4.451368in}}%
\pgfpathlineto{\pgfqpoint{4.468066in}{4.477111in}}%
\pgfpathlineto{\pgfqpoint{4.484890in}{4.512009in}}%
\pgfpathlineto{\pgfqpoint{4.472250in}{4.517708in}}%
\pgfpathlineto{\pgfqpoint{4.468974in}{4.500491in}}%
\pgfpathlineto{\pgfqpoint{4.450931in}{4.495698in}}%
\pgfpathlineto{\pgfqpoint{4.452828in}{4.439052in}}%
\pgfpathlineto{\pgfqpoint{4.449529in}{4.420822in}}%
\pgfpathlineto{\pgfqpoint{4.458579in}{4.394839in}}%
\pgfpathlineto{\pgfqpoint{4.472502in}{4.382338in}}%
\pgfpathlineto{\pgfqpoint{4.466188in}{4.374488in}}%
\pgfpathlineto{\pgfqpoint{4.480400in}{4.363018in}}%
\pgfpathlineto{\pgfqpoint{4.485537in}{4.344401in}}%
\pgfpathlineto{\pgfqpoint{4.459488in}{4.359818in}}%
\pgfpathlineto{\pgfqpoint{4.443002in}{4.357815in}}%
\pgfpathlineto{\pgfqpoint{4.430202in}{4.373674in}}%
\pgfpathlineto{\pgfqpoint{4.417168in}{4.375219in}}%
\pgfpathlineto{\pgfqpoint{4.398644in}{4.367275in}}%
\pgfpathlineto{\pgfqpoint{4.391460in}{4.377174in}}%
\pgfpathlineto{\pgfqpoint{4.400902in}{4.397950in}}%
\pgfpathlineto{\pgfqpoint{4.405194in}{4.415246in}}%
\pgfpathlineto{\pgfqpoint{4.414687in}{4.429117in}}%
\pgfpathlineto{\pgfqpoint{4.400822in}{4.438480in}}%
\pgfpathlineto{\pgfqpoint{4.395161in}{4.430143in}}%
\pgfpathlineto{\pgfqpoint{4.381310in}{4.438606in}}%
\pgfpathlineto{\pgfqpoint{4.356785in}{4.444244in}}%
\pgfpathlineto{\pgfqpoint{4.359009in}{4.456642in}}%
\pgfpathlineto{\pgfqpoint{4.347892in}{4.463833in}}%
\pgfpathlineto{\pgfqpoint{4.333579in}{4.463330in}}%
\pgfpathlineto{\pgfqpoint{4.316235in}{4.493347in}}%
\pgfpathlineto{\pgfqpoint{4.304128in}{4.490670in}}%
\pgfpathlineto{\pgfqpoint{4.285729in}{4.498416in}}%
\pgfpathlineto{\pgfqpoint{4.272494in}{4.489620in}}%
\pgfpathlineto{\pgfqpoint{4.263377in}{4.473810in}}%
\pgfpathlineto{\pgfqpoint{4.249557in}{4.472454in}}%
\pgfpathlineto{\pgfqpoint{4.233707in}{4.479585in}}%
\pgfpathlineto{\pgfqpoint{4.222266in}{4.456841in}}%
\pgfpathlineto{\pgfqpoint{4.214489in}{4.460755in}}%
\pgfpathlineto{\pgfqpoint{4.181483in}{4.421503in}}%
\pgfpathlineto{\pgfqpoint{4.172033in}{4.480971in}}%
\pgfpathlineto{\pgfqpoint{4.292651in}{4.503327in}}%
\pgfpathlineto{\pgfqpoint{4.346722in}{4.513009in}}%
\pgfpathlineto{\pgfqpoint{4.425387in}{4.528852in}}%
\pgfpathlineto{\pgfqpoint{4.492802in}{4.543439in}}%
\pgfpathlineto{\pgfqpoint{4.502665in}{4.505425in}}%
\pgfpathlineto{\pgfqpoint{4.531394in}{4.403253in}}%
\pgfpathlineto{\pgfqpoint{4.585091in}{4.413533in}}%
\pgfusepath{stroke}%
\end{pgfscope}%
\begin{pgfscope}%
\pgfpathrectangle{\pgfqpoint{0.100000in}{2.413063in}}{\pgfqpoint{5.037500in}{3.427208in}}%
\pgfusepath{clip}%
\pgfsetbuttcap%
\pgfsetroundjoin%
\pgfsetlinewidth{0.501875pt}%
\definecolor{currentstroke}{rgb}{0.827451,0.827451,0.827451}%
\pgfsetstrokecolor{currentstroke}%
\pgfsetdash{}{0pt}%
\pgfpathmoveto{\pgfqpoint{2.157107in}{4.074398in}}%
\pgfpathlineto{\pgfqpoint{2.069106in}{4.082796in}}%
\pgfpathlineto{\pgfqpoint{1.977813in}{4.090887in}}%
\pgfpathlineto{\pgfqpoint{1.872319in}{4.101891in}}%
\pgfpathlineto{\pgfqpoint{1.715584in}{4.120532in}}%
\pgfpathlineto{\pgfqpoint{1.659981in}{4.129100in}}%
\pgfpathlineto{\pgfqpoint{1.516457in}{4.149943in}}%
\pgfpathlineto{\pgfqpoint{1.537414in}{4.281477in}}%
\pgfpathlineto{\pgfqpoint{1.537936in}{4.292117in}}%
\pgfpathlineto{\pgfqpoint{1.558125in}{4.419035in}}%
\pgfpathlineto{\pgfqpoint{1.573205in}{4.515697in}}%
\pgfpathlineto{\pgfqpoint{1.587436in}{4.605334in}}%
\pgfpathlineto{\pgfqpoint{1.684663in}{4.591377in}}%
\pgfpathlineto{\pgfqpoint{1.775315in}{4.578367in}}%
\pgfpathlineto{\pgfqpoint{1.941955in}{4.557799in}}%
\pgfpathlineto{\pgfqpoint{2.018405in}{4.550818in}}%
\pgfpathlineto{\pgfqpoint{2.139519in}{4.539067in}}%
\pgfpathlineto{\pgfqpoint{2.191913in}{4.534814in}}%
\pgfpathlineto{\pgfqpoint{2.182662in}{4.420100in}}%
\pgfpathlineto{\pgfqpoint{2.174305in}{4.309649in}}%
\pgfpathlineto{\pgfqpoint{2.167579in}{4.219715in}}%
\pgfpathlineto{\pgfqpoint{2.157107in}{4.074398in}}%
\pgfusepath{stroke}%
\end{pgfscope}%
\begin{pgfscope}%
\pgfpathrectangle{\pgfqpoint{0.100000in}{2.413063in}}{\pgfqpoint{5.037500in}{3.427208in}}%
\pgfusepath{clip}%
\pgfsetbuttcap%
\pgfsetroundjoin%
\pgfsetlinewidth{0.501875pt}%
\definecolor{currentstroke}{rgb}{0.827451,0.827451,0.827451}%
\pgfsetstrokecolor{currentstroke}%
\pgfsetdash{}{0pt}%
\pgfpathmoveto{\pgfqpoint{3.321282in}{4.009187in}}%
\pgfpathlineto{\pgfqpoint{3.324248in}{4.022485in}}%
\pgfpathlineto{\pgfqpoint{3.339624in}{4.019492in}}%
\pgfpathlineto{\pgfqpoint{3.344010in}{4.066681in}}%
\pgfpathlineto{\pgfqpoint{3.339802in}{4.075199in}}%
\pgfpathlineto{\pgfqpoint{3.354799in}{4.094166in}}%
\pgfpathlineto{\pgfqpoint{3.366827in}{4.092910in}}%
\pgfpathlineto{\pgfqpoint{3.403283in}{4.080434in}}%
\pgfpathlineto{\pgfqpoint{3.407555in}{4.090534in}}%
\pgfpathlineto{\pgfqpoint{3.398271in}{4.106030in}}%
\pgfpathlineto{\pgfqpoint{3.401404in}{4.118255in}}%
\pgfpathlineto{\pgfqpoint{3.435402in}{4.129497in}}%
\pgfpathlineto{\pgfqpoint{3.427288in}{4.150184in}}%
\pgfpathlineto{\pgfqpoint{3.437265in}{4.167692in}}%
\pgfpathlineto{\pgfqpoint{3.455836in}{4.177519in}}%
\pgfpathlineto{\pgfqpoint{3.494803in}{4.187250in}}%
\pgfpathlineto{\pgfqpoint{3.519389in}{4.172561in}}%
\pgfpathlineto{\pgfqpoint{3.526847in}{4.186886in}}%
\pgfpathlineto{\pgfqpoint{3.547262in}{4.196892in}}%
\pgfpathlineto{\pgfqpoint{3.552531in}{4.188067in}}%
\pgfpathlineto{\pgfqpoint{3.573500in}{4.195078in}}%
\pgfpathlineto{\pgfqpoint{3.572178in}{4.207078in}}%
\pgfpathlineto{\pgfqpoint{3.591032in}{4.220810in}}%
\pgfpathlineto{\pgfqpoint{3.602160in}{4.206550in}}%
\pgfpathlineto{\pgfqpoint{3.616798in}{4.205115in}}%
\pgfpathlineto{\pgfqpoint{3.626527in}{4.214409in}}%
\pgfpathlineto{\pgfqpoint{3.625445in}{4.227626in}}%
\pgfpathlineto{\pgfqpoint{3.633714in}{4.240750in}}%
\pgfpathlineto{\pgfqpoint{3.644795in}{4.243592in}}%
\pgfpathlineto{\pgfqpoint{3.649244in}{4.260922in}}%
\pgfpathlineto{\pgfqpoint{3.665360in}{4.275902in}}%
\pgfpathlineto{\pgfqpoint{3.660509in}{4.290813in}}%
\pgfpathlineto{\pgfqpoint{3.676188in}{4.298237in}}%
\pgfpathlineto{\pgfqpoint{3.686663in}{4.293671in}}%
\pgfpathlineto{\pgfqpoint{3.702135in}{4.305269in}}%
\pgfpathlineto{\pgfqpoint{3.715962in}{4.308292in}}%
\pgfpathlineto{\pgfqpoint{3.716011in}{4.320323in}}%
\pgfpathlineto{\pgfqpoint{3.706322in}{4.337017in}}%
\pgfpathlineto{\pgfqpoint{3.712630in}{4.343743in}}%
\pgfpathlineto{\pgfqpoint{3.732447in}{4.342252in}}%
\pgfpathlineto{\pgfqpoint{3.745048in}{4.348530in}}%
\pgfpathlineto{\pgfqpoint{3.758047in}{4.338723in}}%
\pgfpathlineto{\pgfqpoint{3.768306in}{4.316588in}}%
\pgfpathlineto{\pgfqpoint{3.802320in}{4.312848in}}%
\pgfpathlineto{\pgfqpoint{3.820049in}{4.300664in}}%
\pgfpathlineto{\pgfqpoint{3.832524in}{4.310962in}}%
\pgfpathlineto{\pgfqpoint{3.855079in}{4.302307in}}%
\pgfpathlineto{\pgfqpoint{3.867848in}{4.306608in}}%
\pgfpathlineto{\pgfqpoint{3.876773in}{4.318839in}}%
\pgfpathlineto{\pgfqpoint{3.890398in}{4.321178in}}%
\pgfpathlineto{\pgfqpoint{3.893938in}{4.307880in}}%
\pgfpathlineto{\pgfqpoint{3.909006in}{4.302795in}}%
\pgfpathlineto{\pgfqpoint{3.920717in}{4.289680in}}%
\pgfpathlineto{\pgfqpoint{3.924249in}{4.271969in}}%
\pgfpathlineto{\pgfqpoint{3.919550in}{4.258580in}}%
\pgfpathlineto{\pgfqpoint{3.937530in}{4.241244in}}%
\pgfpathlineto{\pgfqpoint{3.958652in}{4.208510in}}%
\pgfpathlineto{\pgfqpoint{3.974287in}{4.197740in}}%
\pgfpathlineto{\pgfqpoint{3.991255in}{4.197116in}}%
\pgfpathlineto{\pgfqpoint{3.959924in}{4.161166in}}%
\pgfpathlineto{\pgfqpoint{3.929084in}{4.139406in}}%
\pgfpathlineto{\pgfqpoint{3.917734in}{4.122126in}}%
\pgfpathlineto{\pgfqpoint{3.917926in}{4.112748in}}%
\pgfpathlineto{\pgfqpoint{3.901222in}{4.105560in}}%
\pgfpathlineto{\pgfqpoint{3.896435in}{4.092036in}}%
\pgfpathlineto{\pgfqpoint{3.861637in}{4.078413in}}%
\pgfpathlineto{\pgfqpoint{3.849302in}{4.069566in}}%
\pgfpathlineto{\pgfqpoint{3.847633in}{4.067678in}}%
\pgfpathlineto{\pgfqpoint{3.747536in}{4.058186in}}%
\pgfpathlineto{\pgfqpoint{3.687088in}{4.053103in}}%
\pgfpathlineto{\pgfqpoint{3.587896in}{4.047489in}}%
\pgfpathlineto{\pgfqpoint{3.464320in}{4.035264in}}%
\pgfpathlineto{\pgfqpoint{3.443904in}{4.038092in}}%
\pgfpathlineto{\pgfqpoint{3.448152in}{4.017250in}}%
\pgfpathlineto{\pgfqpoint{3.321282in}{4.009187in}}%
\pgfusepath{stroke}%
\end{pgfscope}%
\begin{pgfscope}%
\pgfpathrectangle{\pgfqpoint{0.100000in}{2.413063in}}{\pgfqpoint{5.037500in}{3.427208in}}%
\pgfusepath{clip}%
\pgfsetbuttcap%
\pgfsetroundjoin%
\pgfsetlinewidth{0.501875pt}%
\definecolor{currentstroke}{rgb}{0.827451,0.827451,0.827451}%
\pgfsetstrokecolor{currentstroke}%
\pgfsetdash{}{0pt}%
\pgfpathmoveto{\pgfqpoint{2.157107in}{4.074398in}}%
\pgfpathlineto{\pgfqpoint{2.167579in}{4.219715in}}%
\pgfpathlineto{\pgfqpoint{2.174305in}{4.309649in}}%
\pgfpathlineto{\pgfqpoint{2.182662in}{4.420100in}}%
\pgfpathlineto{\pgfqpoint{2.298517in}{4.411965in}}%
\pgfpathlineto{\pgfqpoint{2.445651in}{4.403928in}}%
\pgfpathlineto{\pgfqpoint{2.545721in}{4.400088in}}%
\pgfpathlineto{\pgfqpoint{2.645206in}{4.397040in}}%
\pgfpathlineto{\pgfqpoint{2.735250in}{4.395647in}}%
\pgfpathlineto{\pgfqpoint{2.776895in}{4.396079in}}%
\pgfpathlineto{\pgfqpoint{2.795226in}{4.381120in}}%
\pgfpathlineto{\pgfqpoint{2.809584in}{4.384135in}}%
\pgfpathlineto{\pgfqpoint{2.810034in}{4.371087in}}%
\pgfpathlineto{\pgfqpoint{2.799370in}{4.348517in}}%
\pgfpathlineto{\pgfqpoint{2.800527in}{4.334257in}}%
\pgfpathlineto{\pgfqpoint{2.810737in}{4.324854in}}%
\pgfpathlineto{\pgfqpoint{2.820114in}{4.305262in}}%
\pgfpathlineto{\pgfqpoint{2.839138in}{4.294013in}}%
\pgfpathlineto{\pgfqpoint{2.838503in}{4.220162in}}%
\pgfpathlineto{\pgfqpoint{2.839063in}{4.050335in}}%
\pgfpathlineto{\pgfqpoint{2.711550in}{4.050923in}}%
\pgfpathlineto{\pgfqpoint{2.577276in}{4.053273in}}%
\pgfpathlineto{\pgfqpoint{2.478429in}{4.056760in}}%
\pgfpathlineto{\pgfqpoint{2.395919in}{4.059944in}}%
\pgfpathlineto{\pgfqpoint{2.256916in}{4.068140in}}%
\pgfpathlineto{\pgfqpoint{2.157107in}{4.074398in}}%
\pgfusepath{stroke}%
\end{pgfscope}%
\begin{pgfscope}%
\pgfpathrectangle{\pgfqpoint{0.100000in}{2.413063in}}{\pgfqpoint{5.037500in}{3.427208in}}%
\pgfusepath{clip}%
\pgfsetbuttcap%
\pgfsetroundjoin%
\pgfsetlinewidth{0.501875pt}%
\definecolor{currentstroke}{rgb}{0.827451,0.827451,0.827451}%
\pgfsetstrokecolor{currentstroke}%
\pgfsetdash{}{0pt}%
\pgfpathmoveto{\pgfqpoint{4.033298in}{4.093639in}}%
\pgfpathlineto{\pgfqpoint{3.892070in}{4.073801in}}%
\pgfpathlineto{\pgfqpoint{3.849302in}{4.069566in}}%
\pgfpathlineto{\pgfqpoint{3.861637in}{4.078413in}}%
\pgfpathlineto{\pgfqpoint{3.896435in}{4.092036in}}%
\pgfpathlineto{\pgfqpoint{3.901222in}{4.105560in}}%
\pgfpathlineto{\pgfqpoint{3.917926in}{4.112748in}}%
\pgfpathlineto{\pgfqpoint{3.917734in}{4.122126in}}%
\pgfpathlineto{\pgfqpoint{3.929084in}{4.139406in}}%
\pgfpathlineto{\pgfqpoint{3.959924in}{4.161166in}}%
\pgfpathlineto{\pgfqpoint{3.991255in}{4.197116in}}%
\pgfpathlineto{\pgfqpoint{3.989729in}{4.189379in}}%
\pgfpathlineto{\pgfqpoint{4.006250in}{4.170697in}}%
\pgfpathlineto{\pgfqpoint{4.033186in}{4.165450in}}%
\pgfpathlineto{\pgfqpoint{4.048856in}{4.183250in}}%
\pgfpathlineto{\pgfqpoint{4.062876in}{4.173876in}}%
\pgfpathlineto{\pgfqpoint{4.087963in}{4.184878in}}%
\pgfpathlineto{\pgfqpoint{4.091551in}{4.195621in}}%
\pgfpathlineto{\pgfqpoint{4.107121in}{4.198039in}}%
\pgfpathlineto{\pgfqpoint{4.123291in}{4.210732in}}%
\pgfpathlineto{\pgfqpoint{4.137545in}{4.214919in}}%
\pgfpathlineto{\pgfqpoint{4.137999in}{4.237341in}}%
\pgfpathlineto{\pgfqpoint{4.147229in}{4.261562in}}%
\pgfpathlineto{\pgfqpoint{4.159403in}{4.276729in}}%
\pgfpathlineto{\pgfqpoint{4.163350in}{4.299502in}}%
\pgfpathlineto{\pgfqpoint{4.179217in}{4.331491in}}%
\pgfpathlineto{\pgfqpoint{4.179784in}{4.350525in}}%
\pgfpathlineto{\pgfqpoint{4.196175in}{4.338124in}}%
\pgfpathlineto{\pgfqpoint{4.212199in}{4.335115in}}%
\pgfpathlineto{\pgfqpoint{4.218435in}{4.344138in}}%
\pgfpathlineto{\pgfqpoint{4.229575in}{4.383457in}}%
\pgfpathlineto{\pgfqpoint{4.245088in}{4.381950in}}%
\pgfpathlineto{\pgfqpoint{4.249198in}{4.396356in}}%
\pgfpathlineto{\pgfqpoint{4.258297in}{4.402854in}}%
\pgfpathlineto{\pgfqpoint{4.277577in}{4.435523in}}%
\pgfpathlineto{\pgfqpoint{4.274778in}{4.445187in}}%
\pgfpathlineto{\pgfqpoint{4.283263in}{4.466191in}}%
\pgfpathlineto{\pgfqpoint{4.328563in}{4.440605in}}%
\pgfpathlineto{\pgfqpoint{4.333579in}{4.463330in}}%
\pgfpathlineto{\pgfqpoint{4.347892in}{4.463833in}}%
\pgfpathlineto{\pgfqpoint{4.359009in}{4.456642in}}%
\pgfpathlineto{\pgfqpoint{4.356785in}{4.444244in}}%
\pgfpathlineto{\pgfqpoint{4.381310in}{4.438606in}}%
\pgfpathlineto{\pgfqpoint{4.395161in}{4.430143in}}%
\pgfpathlineto{\pgfqpoint{4.405194in}{4.415246in}}%
\pgfpathlineto{\pgfqpoint{4.400902in}{4.397950in}}%
\pgfpathlineto{\pgfqpoint{4.392234in}{4.396533in}}%
\pgfpathlineto{\pgfqpoint{4.387208in}{4.370435in}}%
\pgfpathlineto{\pgfqpoint{4.398219in}{4.360231in}}%
\pgfpathlineto{\pgfqpoint{4.413719in}{4.368479in}}%
\pgfpathlineto{\pgfqpoint{4.428102in}{4.351062in}}%
\pgfpathlineto{\pgfqpoint{4.460236in}{4.347931in}}%
\pgfpathlineto{\pgfqpoint{4.465772in}{4.337912in}}%
\pgfpathlineto{\pgfqpoint{4.495510in}{4.328168in}}%
\pgfpathlineto{\pgfqpoint{4.491840in}{4.316670in}}%
\pgfpathlineto{\pgfqpoint{4.495709in}{4.284173in}}%
\pgfpathlineto{\pgfqpoint{4.507712in}{4.271902in}}%
\pgfpathlineto{\pgfqpoint{4.490000in}{4.266722in}}%
\pgfpathlineto{\pgfqpoint{4.492348in}{4.252849in}}%
\pgfpathlineto{\pgfqpoint{4.511242in}{4.241087in}}%
\pgfpathlineto{\pgfqpoint{4.512946in}{4.229484in}}%
\pgfpathlineto{\pgfqpoint{4.502278in}{4.220765in}}%
\pgfpathlineto{\pgfqpoint{4.480755in}{4.241421in}}%
\pgfpathlineto{\pgfqpoint{4.478627in}{4.226350in}}%
\pgfpathlineto{\pgfqpoint{4.496645in}{4.219183in}}%
\pgfpathlineto{\pgfqpoint{4.498448in}{4.211772in}}%
\pgfpathlineto{\pgfqpoint{4.523162in}{4.221567in}}%
\pgfpathlineto{\pgfqpoint{4.542100in}{4.224158in}}%
\pgfpathlineto{\pgfqpoint{4.561500in}{4.185016in}}%
\pgfpathlineto{\pgfqpoint{4.559336in}{4.184593in}}%
\pgfpathlineto{\pgfqpoint{4.550569in}{4.182781in}}%
\pgfpathlineto{\pgfqpoint{4.547986in}{4.182241in}}%
\pgfpathlineto{\pgfqpoint{4.546278in}{4.181907in}}%
\pgfpathlineto{\pgfqpoint{4.430773in}{4.157948in}}%
\pgfpathlineto{\pgfqpoint{4.327461in}{4.136802in}}%
\pgfpathlineto{\pgfqpoint{4.184620in}{4.112195in}}%
\pgfpathlineto{\pgfqpoint{4.063306in}{4.096172in}}%
\pgfpathlineto{\pgfqpoint{4.033298in}{4.093639in}}%
\pgfusepath{stroke}%
\end{pgfscope}%
\begin{pgfscope}%
\pgfpathrectangle{\pgfqpoint{0.100000in}{2.413063in}}{\pgfqpoint{5.037500in}{3.427208in}}%
\pgfusepath{clip}%
\pgfsetbuttcap%
\pgfsetroundjoin%
\pgfsetlinewidth{0.501875pt}%
\definecolor{currentstroke}{rgb}{0.827451,0.827451,0.827451}%
\pgfsetstrokecolor{currentstroke}%
\pgfsetdash{}{0pt}%
\pgfpathmoveto{\pgfqpoint{4.547880in}{4.348739in}}%
\pgfpathlineto{\pgfqpoint{4.571165in}{4.359301in}}%
\pgfpathlineto{\pgfqpoint{4.552570in}{4.304904in}}%
\pgfpathlineto{\pgfqpoint{4.540273in}{4.276098in}}%
\pgfpathlineto{\pgfqpoint{4.531414in}{4.292400in}}%
\pgfpathlineto{\pgfqpoint{4.547168in}{4.331428in}}%
\pgfpathlineto{\pgfqpoint{4.547880in}{4.348739in}}%
\pgfusepath{stroke}%
\end{pgfscope}%
\begin{pgfscope}%
\pgfpathrectangle{\pgfqpoint{0.100000in}{2.413063in}}{\pgfqpoint{5.037500in}{3.427208in}}%
\pgfusepath{clip}%
\pgfsetbuttcap%
\pgfsetroundjoin%
\pgfsetlinewidth{0.501875pt}%
\definecolor{currentstroke}{rgb}{0.827451,0.827451,0.827451}%
\pgfsetstrokecolor{currentstroke}%
\pgfsetdash{}{0pt}%
\pgfpathmoveto{\pgfqpoint{3.321282in}{4.009187in}}%
\pgfpathlineto{\pgfqpoint{3.315658in}{4.008373in}}%
\pgfpathlineto{\pgfqpoint{3.310355in}{4.008001in}}%
\pgfpathlineto{\pgfqpoint{3.312625in}{3.991720in}}%
\pgfpathlineto{\pgfqpoint{3.298944in}{3.975308in}}%
\pgfpathlineto{\pgfqpoint{3.296248in}{3.949628in}}%
\pgfpathlineto{\pgfqpoint{3.235087in}{3.945104in}}%
\pgfpathlineto{\pgfqpoint{3.240418in}{3.957162in}}%
\pgfpathlineto{\pgfqpoint{3.262470in}{3.979183in}}%
\pgfpathlineto{\pgfqpoint{3.263080in}{3.991948in}}%
\pgfpathlineto{\pgfqpoint{3.253318in}{4.004033in}}%
\pgfpathlineto{\pgfqpoint{3.162287in}{3.999219in}}%
\pgfpathlineto{\pgfqpoint{3.003135in}{3.994167in}}%
\pgfpathlineto{\pgfqpoint{2.909997in}{3.992469in}}%
\pgfpathlineto{\pgfqpoint{2.839596in}{3.991836in}}%
\pgfpathlineto{\pgfqpoint{2.839063in}{4.050335in}}%
\pgfpathlineto{\pgfqpoint{2.838503in}{4.220162in}}%
\pgfpathlineto{\pgfqpoint{2.839138in}{4.294013in}}%
\pgfpathlineto{\pgfqpoint{2.820114in}{4.305262in}}%
\pgfpathlineto{\pgfqpoint{2.810737in}{4.324854in}}%
\pgfpathlineto{\pgfqpoint{2.800527in}{4.334257in}}%
\pgfpathlineto{\pgfqpoint{2.799370in}{4.348517in}}%
\pgfpathlineto{\pgfqpoint{2.810034in}{4.371087in}}%
\pgfpathlineto{\pgfqpoint{2.809584in}{4.384135in}}%
\pgfpathlineto{\pgfqpoint{2.795226in}{4.381120in}}%
\pgfpathlineto{\pgfqpoint{2.776895in}{4.396079in}}%
\pgfpathlineto{\pgfqpoint{2.762201in}{4.422337in}}%
\pgfpathlineto{\pgfqpoint{2.749875in}{4.434454in}}%
\pgfpathlineto{\pgfqpoint{2.736996in}{4.464227in}}%
\pgfpathlineto{\pgfqpoint{2.870765in}{4.462140in}}%
\pgfpathlineto{\pgfqpoint{3.003734in}{4.466521in}}%
\pgfpathlineto{\pgfqpoint{3.088992in}{4.471439in}}%
\pgfpathlineto{\pgfqpoint{3.115663in}{4.445025in}}%
\pgfpathlineto{\pgfqpoint{3.110789in}{4.417479in}}%
\pgfpathlineto{\pgfqpoint{3.117900in}{4.386465in}}%
\pgfpathlineto{\pgfqpoint{3.125789in}{4.370826in}}%
\pgfpathlineto{\pgfqpoint{3.150959in}{4.349315in}}%
\pgfpathlineto{\pgfqpoint{3.156465in}{4.339812in}}%
\pgfpathlineto{\pgfqpoint{3.184022in}{4.318399in}}%
\pgfpathlineto{\pgfqpoint{3.188016in}{4.294377in}}%
\pgfpathlineto{\pgfqpoint{3.195930in}{4.276657in}}%
\pgfpathlineto{\pgfqpoint{3.214629in}{4.286979in}}%
\pgfpathlineto{\pgfqpoint{3.230042in}{4.282258in}}%
\pgfpathlineto{\pgfqpoint{3.241679in}{4.269731in}}%
\pgfpathlineto{\pgfqpoint{3.237285in}{4.247559in}}%
\pgfpathlineto{\pgfqpoint{3.223001in}{4.218571in}}%
\pgfpathlineto{\pgfqpoint{3.223222in}{4.203474in}}%
\pgfpathlineto{\pgfqpoint{3.233897in}{4.191095in}}%
\pgfpathlineto{\pgfqpoint{3.256870in}{4.174698in}}%
\pgfpathlineto{\pgfqpoint{3.283323in}{4.161390in}}%
\pgfpathlineto{\pgfqpoint{3.289597in}{4.150734in}}%
\pgfpathlineto{\pgfqpoint{3.303126in}{4.145694in}}%
\pgfpathlineto{\pgfqpoint{3.303664in}{4.131552in}}%
\pgfpathlineto{\pgfqpoint{3.313676in}{4.113588in}}%
\pgfpathlineto{\pgfqpoint{3.306729in}{4.097628in}}%
\pgfpathlineto{\pgfqpoint{3.320328in}{4.072236in}}%
\pgfpathlineto{\pgfqpoint{3.330964in}{4.077278in}}%
\pgfpathlineto{\pgfqpoint{3.344010in}{4.066681in}}%
\pgfpathlineto{\pgfqpoint{3.339624in}{4.019492in}}%
\pgfpathlineto{\pgfqpoint{3.324248in}{4.022485in}}%
\pgfpathlineto{\pgfqpoint{3.321282in}{4.009187in}}%
\pgfusepath{stroke}%
\end{pgfscope}%
\begin{pgfscope}%
\pgfpathrectangle{\pgfqpoint{0.100000in}{2.413063in}}{\pgfqpoint{5.037500in}{3.427208in}}%
\pgfusepath{clip}%
\pgfsetbuttcap%
\pgfsetroundjoin%
\pgfsetlinewidth{0.501875pt}%
\definecolor{currentstroke}{rgb}{0.827451,0.827451,0.827451}%
\pgfsetstrokecolor{currentstroke}%
\pgfsetdash{}{0pt}%
\pgfpathmoveto{\pgfqpoint{0.963137in}{4.019360in}}%
\pgfpathlineto{\pgfqpoint{0.974271in}{4.043066in}}%
\pgfpathlineto{\pgfqpoint{0.971224in}{4.078649in}}%
\pgfpathlineto{\pgfqpoint{0.976556in}{4.088769in}}%
\pgfpathlineto{\pgfqpoint{0.977083in}{4.133092in}}%
\pgfpathlineto{\pgfqpoint{0.982136in}{4.145855in}}%
\pgfpathlineto{\pgfqpoint{0.999941in}{4.147841in}}%
\pgfpathlineto{\pgfqpoint{1.016464in}{4.142175in}}%
\pgfpathlineto{\pgfqpoint{1.023660in}{4.126570in}}%
\pgfpathlineto{\pgfqpoint{1.033749in}{4.127168in}}%
\pgfpathlineto{\pgfqpoint{1.046284in}{4.145043in}}%
\pgfpathlineto{\pgfqpoint{1.064402in}{4.233160in}}%
\pgfpathlineto{\pgfqpoint{1.167500in}{4.211874in}}%
\pgfpathlineto{\pgfqpoint{1.227336in}{4.200074in}}%
\pgfpathlineto{\pgfqpoint{1.386229in}{4.172143in}}%
\pgfpathlineto{\pgfqpoint{1.430188in}{4.163315in}}%
\pgfpathlineto{\pgfqpoint{1.516457in}{4.149943in}}%
\pgfpathlineto{\pgfqpoint{1.498768in}{4.036047in}}%
\pgfpathlineto{\pgfqpoint{1.480371in}{3.917241in}}%
\pgfpathlineto{\pgfqpoint{1.459184in}{3.783582in}}%
\pgfpathlineto{\pgfqpoint{1.435357in}{3.630155in}}%
\pgfpathlineto{\pgfqpoint{1.416109in}{3.504143in}}%
\pgfpathlineto{\pgfqpoint{1.278236in}{3.526036in}}%
\pgfpathlineto{\pgfqpoint{1.217659in}{3.536411in}}%
\pgfpathlineto{\pgfqpoint{1.190598in}{3.552612in}}%
\pgfpathlineto{\pgfqpoint{1.014159in}{3.659067in}}%
\pgfpathlineto{\pgfqpoint{0.881749in}{3.739855in}}%
\pgfpathlineto{\pgfqpoint{0.886159in}{3.754174in}}%
\pgfpathlineto{\pgfqpoint{0.897097in}{3.764192in}}%
\pgfpathlineto{\pgfqpoint{0.908473in}{3.762335in}}%
\pgfpathlineto{\pgfqpoint{0.924974in}{3.772858in}}%
\pgfpathlineto{\pgfqpoint{0.927572in}{3.787979in}}%
\pgfpathlineto{\pgfqpoint{0.907448in}{3.806322in}}%
\pgfpathlineto{\pgfqpoint{0.921760in}{3.841535in}}%
\pgfpathlineto{\pgfqpoint{0.936150in}{3.855078in}}%
\pgfpathlineto{\pgfqpoint{0.942979in}{3.871143in}}%
\pgfpathlineto{\pgfqpoint{0.947187in}{3.900602in}}%
\pgfpathlineto{\pgfqpoint{0.960679in}{3.913943in}}%
\pgfpathlineto{\pgfqpoint{0.988999in}{3.927268in}}%
\pgfpathlineto{\pgfqpoint{0.988373in}{3.939008in}}%
\pgfpathlineto{\pgfqpoint{0.972557in}{3.953579in}}%
\pgfpathlineto{\pgfqpoint{0.970403in}{3.983615in}}%
\pgfpathlineto{\pgfqpoint{0.959513in}{4.005573in}}%
\pgfpathlineto{\pgfqpoint{0.963137in}{4.019360in}}%
\pgfusepath{stroke}%
\end{pgfscope}%
\begin{pgfscope}%
\pgfpathrectangle{\pgfqpoint{0.100000in}{2.413063in}}{\pgfqpoint{5.037500in}{3.427208in}}%
\pgfusepath{clip}%
\pgfsetbuttcap%
\pgfsetroundjoin%
\pgfsetlinewidth{0.501875pt}%
\definecolor{currentstroke}{rgb}{0.827451,0.827451,0.827451}%
\pgfsetstrokecolor{currentstroke}%
\pgfsetdash{}{0pt}%
\pgfpathmoveto{\pgfqpoint{2.854521in}{3.662061in}}%
\pgfpathlineto{\pgfqpoint{2.828963in}{3.669966in}}%
\pgfpathlineto{\pgfqpoint{2.796214in}{3.695046in}}%
\pgfpathlineto{\pgfqpoint{2.781668in}{3.700436in}}%
\pgfpathlineto{\pgfqpoint{2.772424in}{3.689604in}}%
\pgfpathlineto{\pgfqpoint{2.760751in}{3.689057in}}%
\pgfpathlineto{\pgfqpoint{2.745983in}{3.698253in}}%
\pgfpathlineto{\pgfqpoint{2.722809in}{3.686470in}}%
\pgfpathlineto{\pgfqpoint{2.714559in}{3.692273in}}%
\pgfpathlineto{\pgfqpoint{2.691756in}{3.678526in}}%
\pgfpathlineto{\pgfqpoint{2.667699in}{3.681062in}}%
\pgfpathlineto{\pgfqpoint{2.649143in}{3.689996in}}%
\pgfpathlineto{\pgfqpoint{2.636148in}{3.686633in}}%
\pgfpathlineto{\pgfqpoint{2.615348in}{3.699267in}}%
\pgfpathlineto{\pgfqpoint{2.603777in}{3.676985in}}%
\pgfpathlineto{\pgfqpoint{2.591944in}{3.696148in}}%
\pgfpathlineto{\pgfqpoint{2.568572in}{3.688713in}}%
\pgfpathlineto{\pgfqpoint{2.548129in}{3.706915in}}%
\pgfpathlineto{\pgfqpoint{2.530266in}{3.692244in}}%
\pgfpathlineto{\pgfqpoint{2.521320in}{3.705726in}}%
\pgfpathlineto{\pgfqpoint{2.508425in}{3.710104in}}%
\pgfpathlineto{\pgfqpoint{2.500572in}{3.723102in}}%
\pgfpathlineto{\pgfqpoint{2.483707in}{3.726801in}}%
\pgfpathlineto{\pgfqpoint{2.473975in}{3.717018in}}%
\pgfpathlineto{\pgfqpoint{2.457440in}{3.729697in}}%
\pgfpathlineto{\pgfqpoint{2.449732in}{3.726802in}}%
\pgfpathlineto{\pgfqpoint{2.422328in}{3.737076in}}%
\pgfpathlineto{\pgfqpoint{2.405164in}{3.738222in}}%
\pgfpathlineto{\pgfqpoint{2.397482in}{3.760041in}}%
\pgfpathlineto{\pgfqpoint{2.383723in}{3.757323in}}%
\pgfpathlineto{\pgfqpoint{2.367963in}{3.762707in}}%
\pgfpathlineto{\pgfqpoint{2.357586in}{3.759596in}}%
\pgfpathlineto{\pgfqpoint{2.335330in}{3.784101in}}%
\pgfpathlineto{\pgfqpoint{2.329128in}{3.782498in}}%
\pgfpathlineto{\pgfqpoint{2.334761in}{3.881873in}}%
\pgfpathlineto{\pgfqpoint{2.340896in}{4.004874in}}%
\pgfpathlineto{\pgfqpoint{2.240180in}{4.010507in}}%
\pgfpathlineto{\pgfqpoint{2.140792in}{4.018021in}}%
\pgfpathlineto{\pgfqpoint{2.064008in}{4.024689in}}%
\pgfpathlineto{\pgfqpoint{2.069106in}{4.082796in}}%
\pgfpathlineto{\pgfqpoint{2.157107in}{4.074398in}}%
\pgfpathlineto{\pgfqpoint{2.256916in}{4.068140in}}%
\pgfpathlineto{\pgfqpoint{2.395919in}{4.059944in}}%
\pgfpathlineto{\pgfqpoint{2.478429in}{4.056760in}}%
\pgfpathlineto{\pgfqpoint{2.577276in}{4.053273in}}%
\pgfpathlineto{\pgfqpoint{2.711550in}{4.050923in}}%
\pgfpathlineto{\pgfqpoint{2.839063in}{4.050335in}}%
\pgfpathlineto{\pgfqpoint{2.839596in}{3.991836in}}%
\pgfpathlineto{\pgfqpoint{2.846754in}{3.947764in}}%
\pgfpathlineto{\pgfqpoint{2.857873in}{3.866362in}}%
\pgfpathlineto{\pgfqpoint{2.855579in}{3.727350in}}%
\pgfpathlineto{\pgfqpoint{2.854521in}{3.662061in}}%
\pgfusepath{stroke}%
\end{pgfscope}%
\begin{pgfscope}%
\pgfpathrectangle{\pgfqpoint{0.100000in}{2.413063in}}{\pgfqpoint{5.037500in}{3.427208in}}%
\pgfusepath{clip}%
\pgfsetbuttcap%
\pgfsetroundjoin%
\pgfsetlinewidth{0.501875pt}%
\definecolor{currentstroke}{rgb}{0.827451,0.827451,0.827451}%
\pgfsetstrokecolor{currentstroke}%
\pgfsetdash{}{0pt}%
\pgfpathmoveto{\pgfqpoint{3.811487in}{3.877948in}}%
\pgfpathlineto{\pgfqpoint{3.811562in}{3.903705in}}%
\pgfpathlineto{\pgfqpoint{3.833945in}{3.913596in}}%
\pgfpathlineto{\pgfqpoint{3.834884in}{3.929407in}}%
\pgfpathlineto{\pgfqpoint{3.854924in}{3.948722in}}%
\pgfpathlineto{\pgfqpoint{3.879975in}{3.952678in}}%
\pgfpathlineto{\pgfqpoint{3.901997in}{3.973964in}}%
\pgfpathlineto{\pgfqpoint{3.924997in}{3.983478in}}%
\pgfpathlineto{\pgfqpoint{3.942259in}{4.012040in}}%
\pgfpathlineto{\pgfqpoint{3.957980in}{4.009966in}}%
\pgfpathlineto{\pgfqpoint{3.991206in}{4.035991in}}%
\pgfpathlineto{\pgfqpoint{4.008752in}{4.036482in}}%
\pgfpathlineto{\pgfqpoint{4.016161in}{4.056311in}}%
\pgfpathlineto{\pgfqpoint{4.030107in}{4.070117in}}%
\pgfpathlineto{\pgfqpoint{4.033298in}{4.093639in}}%
\pgfpathlineto{\pgfqpoint{4.063306in}{4.096172in}}%
\pgfpathlineto{\pgfqpoint{4.184620in}{4.112195in}}%
\pgfpathlineto{\pgfqpoint{4.327461in}{4.136802in}}%
\pgfpathlineto{\pgfqpoint{4.430773in}{4.157948in}}%
\pgfpathlineto{\pgfqpoint{4.546278in}{4.181907in}}%
\pgfpathlineto{\pgfqpoint{4.579284in}{4.136592in}}%
\pgfpathlineto{\pgfqpoint{4.561406in}{4.139485in}}%
\pgfpathlineto{\pgfqpoint{4.538456in}{4.133894in}}%
\pgfpathlineto{\pgfqpoint{4.515842in}{4.110800in}}%
\pgfpathlineto{\pgfqpoint{4.499590in}{4.112461in}}%
\pgfpathlineto{\pgfqpoint{4.497534in}{4.098745in}}%
\pgfpathlineto{\pgfqpoint{4.526910in}{4.109592in}}%
\pgfpathlineto{\pgfqpoint{4.556471in}{4.114065in}}%
\pgfpathlineto{\pgfqpoint{4.567447in}{4.108113in}}%
\pgfpathlineto{\pgfqpoint{4.582218in}{4.114897in}}%
\pgfpathlineto{\pgfqpoint{4.589841in}{4.110144in}}%
\pgfpathlineto{\pgfqpoint{4.596587in}{4.087509in}}%
\pgfpathlineto{\pgfqpoint{4.582547in}{4.080499in}}%
\pgfpathlineto{\pgfqpoint{4.572911in}{4.052898in}}%
\pgfpathlineto{\pgfqpoint{4.562808in}{4.042151in}}%
\pgfpathlineto{\pgfqpoint{4.531837in}{4.044502in}}%
\pgfpathlineto{\pgfqpoint{4.533501in}{4.060709in}}%
\pgfpathlineto{\pgfqpoint{4.516587in}{4.053630in}}%
\pgfpathlineto{\pgfqpoint{4.512908in}{4.040112in}}%
\pgfpathlineto{\pgfqpoint{4.524502in}{4.026111in}}%
\pgfpathlineto{\pgfqpoint{4.528440in}{4.002346in}}%
\pgfpathlineto{\pgfqpoint{4.509469in}{3.988806in}}%
\pgfpathlineto{\pgfqpoint{4.529986in}{3.984043in}}%
\pgfpathlineto{\pgfqpoint{4.539271in}{3.994000in}}%
\pgfpathlineto{\pgfqpoint{4.559800in}{3.995213in}}%
\pgfpathlineto{\pgfqpoint{4.549250in}{3.973712in}}%
\pgfpathlineto{\pgfqpoint{4.536329in}{3.963352in}}%
\pgfpathlineto{\pgfqpoint{4.497243in}{3.952980in}}%
\pgfpathlineto{\pgfqpoint{4.457112in}{3.916583in}}%
\pgfpathlineto{\pgfqpoint{4.432367in}{3.880721in}}%
\pgfpathlineto{\pgfqpoint{4.432221in}{3.866156in}}%
\pgfpathlineto{\pgfqpoint{4.422337in}{3.846058in}}%
\pgfpathlineto{\pgfqpoint{4.371607in}{3.832835in}}%
\pgfpathlineto{\pgfqpoint{4.250965in}{3.919317in}}%
\pgfpathlineto{\pgfqpoint{4.144721in}{3.903707in}}%
\pgfpathlineto{\pgfqpoint{4.143831in}{3.918121in}}%
\pgfpathlineto{\pgfqpoint{4.127682in}{3.934353in}}%
\pgfpathlineto{\pgfqpoint{4.115457in}{3.938326in}}%
\pgfpathlineto{\pgfqpoint{4.000022in}{3.926332in}}%
\pgfpathlineto{\pgfqpoint{3.973490in}{3.917333in}}%
\pgfpathlineto{\pgfqpoint{3.925582in}{3.893502in}}%
\pgfpathlineto{\pgfqpoint{3.884181in}{3.886905in}}%
\pgfpathlineto{\pgfqpoint{3.811487in}{3.877948in}}%
\pgfusepath{stroke}%
\end{pgfscope}%
\begin{pgfscope}%
\pgfpathrectangle{\pgfqpoint{0.100000in}{2.413063in}}{\pgfqpoint{5.037500in}{3.427208in}}%
\pgfusepath{clip}%
\pgfsetbuttcap%
\pgfsetroundjoin%
\pgfsetlinewidth{0.501875pt}%
\definecolor{currentstroke}{rgb}{0.827451,0.827451,0.827451}%
\pgfsetstrokecolor{currentstroke}%
\pgfsetdash{}{0pt}%
\pgfpathmoveto{\pgfqpoint{3.811487in}{3.877948in}}%
\pgfpathlineto{\pgfqpoint{3.690671in}{3.864679in}}%
\pgfpathlineto{\pgfqpoint{3.580149in}{3.854740in}}%
\pgfpathlineto{\pgfqpoint{3.466555in}{3.847394in}}%
\pgfpathlineto{\pgfqpoint{3.447056in}{3.844568in}}%
\pgfpathlineto{\pgfqpoint{3.370483in}{3.838678in}}%
\pgfpathlineto{\pgfqpoint{3.247837in}{3.831521in}}%
\pgfpathlineto{\pgfqpoint{3.269658in}{3.849634in}}%
\pgfpathlineto{\pgfqpoint{3.264679in}{3.868699in}}%
\pgfpathlineto{\pgfqpoint{3.269502in}{3.891811in}}%
\pgfpathlineto{\pgfqpoint{3.276871in}{3.902691in}}%
\pgfpathlineto{\pgfqpoint{3.276593in}{3.917829in}}%
\pgfpathlineto{\pgfqpoint{3.296214in}{3.927350in}}%
\pgfpathlineto{\pgfqpoint{3.296248in}{3.949628in}}%
\pgfpathlineto{\pgfqpoint{3.298944in}{3.975308in}}%
\pgfpathlineto{\pgfqpoint{3.312625in}{3.991720in}}%
\pgfpathlineto{\pgfqpoint{3.310355in}{4.008001in}}%
\pgfpathlineto{\pgfqpoint{3.315658in}{4.008373in}}%
\pgfpathlineto{\pgfqpoint{3.321282in}{4.009187in}}%
\pgfpathlineto{\pgfqpoint{3.448152in}{4.017250in}}%
\pgfpathlineto{\pgfqpoint{3.443904in}{4.038092in}}%
\pgfpathlineto{\pgfqpoint{3.464320in}{4.035264in}}%
\pgfpathlineto{\pgfqpoint{3.587896in}{4.047489in}}%
\pgfpathlineto{\pgfqpoint{3.687088in}{4.053103in}}%
\pgfpathlineto{\pgfqpoint{3.747536in}{4.058186in}}%
\pgfpathlineto{\pgfqpoint{3.847633in}{4.067678in}}%
\pgfpathlineto{\pgfqpoint{3.849302in}{4.069566in}}%
\pgfpathlineto{\pgfqpoint{3.892070in}{4.073801in}}%
\pgfpathlineto{\pgfqpoint{4.033298in}{4.093639in}}%
\pgfpathlineto{\pgfqpoint{4.030107in}{4.070117in}}%
\pgfpathlineto{\pgfqpoint{4.016161in}{4.056311in}}%
\pgfpathlineto{\pgfqpoint{4.008752in}{4.036482in}}%
\pgfpathlineto{\pgfqpoint{3.991206in}{4.035991in}}%
\pgfpathlineto{\pgfqpoint{3.957980in}{4.009966in}}%
\pgfpathlineto{\pgfqpoint{3.942259in}{4.012040in}}%
\pgfpathlineto{\pgfqpoint{3.924997in}{3.983478in}}%
\pgfpathlineto{\pgfqpoint{3.901997in}{3.973964in}}%
\pgfpathlineto{\pgfqpoint{3.879975in}{3.952678in}}%
\pgfpathlineto{\pgfqpoint{3.854924in}{3.948722in}}%
\pgfpathlineto{\pgfqpoint{3.834884in}{3.929407in}}%
\pgfpathlineto{\pgfqpoint{3.833945in}{3.913596in}}%
\pgfpathlineto{\pgfqpoint{3.811562in}{3.903705in}}%
\pgfpathlineto{\pgfqpoint{3.811487in}{3.877948in}}%
\pgfusepath{stroke}%
\end{pgfscope}%
\begin{pgfscope}%
\pgfpathrectangle{\pgfqpoint{0.100000in}{2.413063in}}{\pgfqpoint{5.037500in}{3.427208in}}%
\pgfusepath{clip}%
\pgfsetbuttcap%
\pgfsetroundjoin%
\pgfsetlinewidth{0.501875pt}%
\definecolor{currentstroke}{rgb}{0.827451,0.827451,0.827451}%
\pgfsetstrokecolor{currentstroke}%
\pgfsetdash{}{0pt}%
\pgfpathmoveto{\pgfqpoint{1.668801in}{3.520265in}}%
\pgfpathlineto{\pgfqpoint{1.660184in}{3.534111in}}%
\pgfpathlineto{\pgfqpoint{1.663749in}{3.546038in}}%
\pgfpathlineto{\pgfqpoint{1.836936in}{3.525734in}}%
\pgfpathlineto{\pgfqpoint{2.012416in}{3.508241in}}%
\pgfpathlineto{\pgfqpoint{2.017532in}{3.567562in}}%
\pgfpathlineto{\pgfqpoint{2.043908in}{3.824143in}}%
\pgfpathlineto{\pgfqpoint{2.061241in}{4.024844in}}%
\pgfpathlineto{\pgfqpoint{2.064008in}{4.024689in}}%
\pgfpathlineto{\pgfqpoint{2.240180in}{4.010507in}}%
\pgfpathlineto{\pgfqpoint{2.340896in}{4.004874in}}%
\pgfpathlineto{\pgfqpoint{2.329128in}{3.782498in}}%
\pgfpathlineto{\pgfqpoint{2.335330in}{3.784101in}}%
\pgfpathlineto{\pgfqpoint{2.357586in}{3.759596in}}%
\pgfpathlineto{\pgfqpoint{2.367963in}{3.762707in}}%
\pgfpathlineto{\pgfqpoint{2.383723in}{3.757323in}}%
\pgfpathlineto{\pgfqpoint{2.397482in}{3.760041in}}%
\pgfpathlineto{\pgfqpoint{2.405164in}{3.738222in}}%
\pgfpathlineto{\pgfqpoint{2.422328in}{3.737076in}}%
\pgfpathlineto{\pgfqpoint{2.449732in}{3.726802in}}%
\pgfpathlineto{\pgfqpoint{2.457440in}{3.729697in}}%
\pgfpathlineto{\pgfqpoint{2.473975in}{3.717018in}}%
\pgfpathlineto{\pgfqpoint{2.483707in}{3.726801in}}%
\pgfpathlineto{\pgfqpoint{2.500572in}{3.723102in}}%
\pgfpathlineto{\pgfqpoint{2.508425in}{3.710104in}}%
\pgfpathlineto{\pgfqpoint{2.521320in}{3.705726in}}%
\pgfpathlineto{\pgfqpoint{2.530266in}{3.692244in}}%
\pgfpathlineto{\pgfqpoint{2.548129in}{3.706915in}}%
\pgfpathlineto{\pgfqpoint{2.568572in}{3.688713in}}%
\pgfpathlineto{\pgfqpoint{2.591944in}{3.696148in}}%
\pgfpathlineto{\pgfqpoint{2.603777in}{3.676985in}}%
\pgfpathlineto{\pgfqpoint{2.615348in}{3.699267in}}%
\pgfpathlineto{\pgfqpoint{2.636148in}{3.686633in}}%
\pgfpathlineto{\pgfqpoint{2.649143in}{3.689996in}}%
\pgfpathlineto{\pgfqpoint{2.667699in}{3.681062in}}%
\pgfpathlineto{\pgfqpoint{2.691756in}{3.678526in}}%
\pgfpathlineto{\pgfqpoint{2.714559in}{3.692273in}}%
\pgfpathlineto{\pgfqpoint{2.722809in}{3.686470in}}%
\pgfpathlineto{\pgfqpoint{2.745983in}{3.698253in}}%
\pgfpathlineto{\pgfqpoint{2.760751in}{3.689057in}}%
\pgfpathlineto{\pgfqpoint{2.772424in}{3.689604in}}%
\pgfpathlineto{\pgfqpoint{2.781668in}{3.700436in}}%
\pgfpathlineto{\pgfqpoint{2.796214in}{3.695046in}}%
\pgfpathlineto{\pgfqpoint{2.828963in}{3.669966in}}%
\pgfpathlineto{\pgfqpoint{2.854521in}{3.662061in}}%
\pgfpathlineto{\pgfqpoint{2.864771in}{3.652378in}}%
\pgfpathlineto{\pgfqpoint{2.877599in}{3.657651in}}%
\pgfpathlineto{\pgfqpoint{2.897040in}{3.653613in}}%
\pgfpathlineto{\pgfqpoint{2.899043in}{3.472768in}}%
\pgfpathlineto{\pgfqpoint{2.912544in}{3.461313in}}%
\pgfpathlineto{\pgfqpoint{2.923366in}{3.440204in}}%
\pgfpathlineto{\pgfqpoint{2.919562in}{3.426082in}}%
\pgfpathlineto{\pgfqpoint{2.928060in}{3.420103in}}%
\pgfpathlineto{\pgfqpoint{2.934966in}{3.394061in}}%
\pgfpathlineto{\pgfqpoint{2.948193in}{3.380114in}}%
\pgfpathlineto{\pgfqpoint{2.951156in}{3.350504in}}%
\pgfpathlineto{\pgfqpoint{2.948856in}{3.337967in}}%
\pgfpathlineto{\pgfqpoint{2.930876in}{3.304788in}}%
\pgfpathlineto{\pgfqpoint{2.928809in}{3.280933in}}%
\pgfpathlineto{\pgfqpoint{2.934908in}{3.275954in}}%
\pgfpathlineto{\pgfqpoint{2.935438in}{3.255055in}}%
\pgfpathlineto{\pgfqpoint{2.929239in}{3.241920in}}%
\pgfpathlineto{\pgfqpoint{2.919598in}{3.236972in}}%
\pgfpathlineto{\pgfqpoint{2.910158in}{3.219808in}}%
\pgfpathlineto{\pgfqpoint{2.922190in}{3.203183in}}%
\pgfpathlineto{\pgfqpoint{2.898834in}{3.202856in}}%
\pgfpathlineto{\pgfqpoint{2.836385in}{3.174284in}}%
\pgfpathlineto{\pgfqpoint{2.848308in}{3.191383in}}%
\pgfpathlineto{\pgfqpoint{2.833870in}{3.200602in}}%
\pgfpathlineto{\pgfqpoint{2.830853in}{3.216277in}}%
\pgfpathlineto{\pgfqpoint{2.802674in}{3.188975in}}%
\pgfpathlineto{\pgfqpoint{2.815180in}{3.170314in}}%
\pgfpathlineto{\pgfqpoint{2.797340in}{3.146553in}}%
\pgfpathlineto{\pgfqpoint{2.787739in}{3.147050in}}%
\pgfpathlineto{\pgfqpoint{2.778707in}{3.121174in}}%
\pgfpathlineto{\pgfqpoint{2.750144in}{3.100819in}}%
\pgfpathlineto{\pgfqpoint{2.723459in}{3.093484in}}%
\pgfpathlineto{\pgfqpoint{2.676917in}{3.074401in}}%
\pgfpathlineto{\pgfqpoint{2.672094in}{3.085046in}}%
\pgfpathlineto{\pgfqpoint{2.643101in}{3.075248in}}%
\pgfpathlineto{\pgfqpoint{2.660934in}{3.058565in}}%
\pgfpathlineto{\pgfqpoint{2.632807in}{3.044071in}}%
\pgfpathlineto{\pgfqpoint{2.602468in}{3.022188in}}%
\pgfpathlineto{\pgfqpoint{2.595182in}{3.032361in}}%
\pgfpathlineto{\pgfqpoint{2.570354in}{3.017114in}}%
\pgfpathlineto{\pgfqpoint{2.594689in}{3.011325in}}%
\pgfpathlineto{\pgfqpoint{2.576341in}{2.987353in}}%
\pgfpathlineto{\pgfqpoint{2.567385in}{2.994487in}}%
\pgfpathlineto{\pgfqpoint{2.555355in}{2.983028in}}%
\pgfpathlineto{\pgfqpoint{2.570347in}{2.973019in}}%
\pgfpathlineto{\pgfqpoint{2.561483in}{2.958390in}}%
\pgfpathlineto{\pgfqpoint{2.550589in}{2.923757in}}%
\pgfpathlineto{\pgfqpoint{2.534875in}{2.890402in}}%
\pgfpathlineto{\pgfqpoint{2.541929in}{2.868538in}}%
\pgfpathlineto{\pgfqpoint{2.553920in}{2.816991in}}%
\pgfpathlineto{\pgfqpoint{2.555001in}{2.796129in}}%
\pgfpathlineto{\pgfqpoint{2.573528in}{2.768767in}}%
\pgfpathlineto{\pgfqpoint{2.559241in}{2.770366in}}%
\pgfpathlineto{\pgfqpoint{2.545364in}{2.756562in}}%
\pgfpathlineto{\pgfqpoint{2.529111in}{2.767411in}}%
\pgfpathlineto{\pgfqpoint{2.523277in}{2.778234in}}%
\pgfpathlineto{\pgfqpoint{2.500146in}{2.783269in}}%
\pgfpathlineto{\pgfqpoint{2.464813in}{2.783886in}}%
\pgfpathlineto{\pgfqpoint{2.438779in}{2.804383in}}%
\pgfpathlineto{\pgfqpoint{2.415141in}{2.807806in}}%
\pgfpathlineto{\pgfqpoint{2.400800in}{2.824095in}}%
\pgfpathlineto{\pgfqpoint{2.370766in}{2.830651in}}%
\pgfpathlineto{\pgfqpoint{2.354344in}{2.883026in}}%
\pgfpathlineto{\pgfqpoint{2.337552in}{2.904007in}}%
\pgfpathlineto{\pgfqpoint{2.340405in}{2.923943in}}%
\pgfpathlineto{\pgfqpoint{2.329993in}{2.938513in}}%
\pgfpathlineto{\pgfqpoint{2.336527in}{2.958428in}}%
\pgfpathlineto{\pgfqpoint{2.331137in}{2.973021in}}%
\pgfpathlineto{\pgfqpoint{2.314263in}{2.979632in}}%
\pgfpathlineto{\pgfqpoint{2.298486in}{2.996460in}}%
\pgfpathlineto{\pgfqpoint{2.292736in}{3.018989in}}%
\pgfpathlineto{\pgfqpoint{2.277807in}{3.039464in}}%
\pgfpathlineto{\pgfqpoint{2.257935in}{3.055395in}}%
\pgfpathlineto{\pgfqpoint{2.240037in}{3.101094in}}%
\pgfpathlineto{\pgfqpoint{2.225584in}{3.151073in}}%
\pgfpathlineto{\pgfqpoint{2.213725in}{3.170834in}}%
\pgfpathlineto{\pgfqpoint{2.193167in}{3.187471in}}%
\pgfpathlineto{\pgfqpoint{2.188029in}{3.199547in}}%
\pgfpathlineto{\pgfqpoint{2.168780in}{3.207045in}}%
\pgfpathlineto{\pgfqpoint{2.149781in}{3.239041in}}%
\pgfpathlineto{\pgfqpoint{2.132430in}{3.236585in}}%
\pgfpathlineto{\pgfqpoint{2.114812in}{3.244576in}}%
\pgfpathlineto{\pgfqpoint{2.089838in}{3.243031in}}%
\pgfpathlineto{\pgfqpoint{2.064467in}{3.256219in}}%
\pgfpathlineto{\pgfqpoint{2.057313in}{3.243677in}}%
\pgfpathlineto{\pgfqpoint{2.027664in}{3.243363in}}%
\pgfpathlineto{\pgfqpoint{2.012569in}{3.219586in}}%
\pgfpathlineto{\pgfqpoint{1.999441in}{3.190148in}}%
\pgfpathlineto{\pgfqpoint{1.971522in}{3.158547in}}%
\pgfpathlineto{\pgfqpoint{1.939894in}{3.172415in}}%
\pgfpathlineto{\pgfqpoint{1.935405in}{3.181580in}}%
\pgfpathlineto{\pgfqpoint{1.916184in}{3.188571in}}%
\pgfpathlineto{\pgfqpoint{1.910279in}{3.198133in}}%
\pgfpathlineto{\pgfqpoint{1.884762in}{3.207785in}}%
\pgfpathlineto{\pgfqpoint{1.870487in}{3.227517in}}%
\pgfpathlineto{\pgfqpoint{1.853817in}{3.237037in}}%
\pgfpathlineto{\pgfqpoint{1.839474in}{3.253650in}}%
\pgfpathlineto{\pgfqpoint{1.828300in}{3.281784in}}%
\pgfpathlineto{\pgfqpoint{1.829550in}{3.320230in}}%
\pgfpathlineto{\pgfqpoint{1.816442in}{3.339658in}}%
\pgfpathlineto{\pgfqpoint{1.814934in}{3.360700in}}%
\pgfpathlineto{\pgfqpoint{1.785850in}{3.392204in}}%
\pgfpathlineto{\pgfqpoint{1.768928in}{3.398929in}}%
\pgfpathlineto{\pgfqpoint{1.750970in}{3.428326in}}%
\pgfpathlineto{\pgfqpoint{1.735668in}{3.440005in}}%
\pgfpathlineto{\pgfqpoint{1.716224in}{3.468446in}}%
\pgfpathlineto{\pgfqpoint{1.696330in}{3.480754in}}%
\pgfpathlineto{\pgfqpoint{1.683301in}{3.512292in}}%
\pgfpathlineto{\pgfqpoint{1.668801in}{3.520265in}}%
\pgfpathlineto{\pgfqpoint{1.668801in}{3.520265in}}%
\pgfusepath{stroke}%
\end{pgfscope}%
\begin{pgfscope}%
\pgfpathrectangle{\pgfqpoint{0.100000in}{2.413063in}}{\pgfqpoint{5.037500in}{3.427208in}}%
\pgfusepath{clip}%
\pgfsetbuttcap%
\pgfsetroundjoin%
\pgfsetlinewidth{0.501875pt}%
\definecolor{currentstroke}{rgb}{0.827451,0.827451,0.827451}%
\pgfsetstrokecolor{currentstroke}%
\pgfsetdash{}{0pt}%
\pgfpathmoveto{\pgfqpoint{1.516457in}{4.149943in}}%
\pgfpathlineto{\pgfqpoint{1.659981in}{4.129100in}}%
\pgfpathlineto{\pgfqpoint{1.715584in}{4.120532in}}%
\pgfpathlineto{\pgfqpoint{1.872319in}{4.101891in}}%
\pgfpathlineto{\pgfqpoint{1.977813in}{4.090887in}}%
\pgfpathlineto{\pgfqpoint{2.069106in}{4.082796in}}%
\pgfpathlineto{\pgfqpoint{2.064008in}{4.024689in}}%
\pgfpathlineto{\pgfqpoint{2.061241in}{4.024844in}}%
\pgfpathlineto{\pgfqpoint{2.052879in}{3.925113in}}%
\pgfpathlineto{\pgfqpoint{2.043908in}{3.824143in}}%
\pgfpathlineto{\pgfqpoint{2.033530in}{3.718444in}}%
\pgfpathlineto{\pgfqpoint{2.017532in}{3.567562in}}%
\pgfpathlineto{\pgfqpoint{2.012416in}{3.508241in}}%
\pgfpathlineto{\pgfqpoint{1.918368in}{3.517731in}}%
\pgfpathlineto{\pgfqpoint{1.836936in}{3.525734in}}%
\pgfpathlineto{\pgfqpoint{1.724316in}{3.538525in}}%
\pgfpathlineto{\pgfqpoint{1.663749in}{3.546038in}}%
\pgfpathlineto{\pgfqpoint{1.660184in}{3.534111in}}%
\pgfpathlineto{\pgfqpoint{1.668801in}{3.520265in}}%
\pgfpathlineto{\pgfqpoint{1.596010in}{3.529718in}}%
\pgfpathlineto{\pgfqpoint{1.506202in}{3.542608in}}%
\pgfpathlineto{\pgfqpoint{1.498028in}{3.491813in}}%
\pgfpathlineto{\pgfqpoint{1.416109in}{3.504143in}}%
\pgfpathlineto{\pgfqpoint{1.435357in}{3.630155in}}%
\pgfpathlineto{\pgfqpoint{1.459184in}{3.783582in}}%
\pgfpathlineto{\pgfqpoint{1.480371in}{3.917241in}}%
\pgfpathlineto{\pgfqpoint{1.498768in}{4.036047in}}%
\pgfpathlineto{\pgfqpoint{1.516457in}{4.149943in}}%
\pgfusepath{stroke}%
\end{pgfscope}%
\begin{pgfscope}%
\pgfpathrectangle{\pgfqpoint{0.100000in}{2.413063in}}{\pgfqpoint{5.037500in}{3.427208in}}%
\pgfusepath{clip}%
\pgfsetbuttcap%
\pgfsetroundjoin%
\pgfsetlinewidth{0.501875pt}%
\definecolor{currentstroke}{rgb}{0.827451,0.827451,0.827451}%
\pgfsetstrokecolor{currentstroke}%
\pgfsetdash{}{0pt}%
\pgfpathmoveto{\pgfqpoint{3.798503in}{3.411653in}}%
\pgfpathlineto{\pgfqpoint{3.661580in}{3.396445in}}%
\pgfpathlineto{\pgfqpoint{3.540884in}{3.387076in}}%
\pgfpathlineto{\pgfqpoint{3.539374in}{3.372285in}}%
\pgfpathlineto{\pgfqpoint{3.550461in}{3.358206in}}%
\pgfpathlineto{\pgfqpoint{3.563992in}{3.349936in}}%
\pgfpathlineto{\pgfqpoint{3.563787in}{3.328159in}}%
\pgfpathlineto{\pgfqpoint{3.560202in}{3.313610in}}%
\pgfpathlineto{\pgfqpoint{3.548259in}{3.303125in}}%
\pgfpathlineto{\pgfqpoint{3.531613in}{3.304235in}}%
\pgfpathlineto{\pgfqpoint{3.515874in}{3.317227in}}%
\pgfpathlineto{\pgfqpoint{3.513069in}{3.340345in}}%
\pgfpathlineto{\pgfqpoint{3.501347in}{3.353791in}}%
\pgfpathlineto{\pgfqpoint{3.493369in}{3.305669in}}%
\pgfpathlineto{\pgfqpoint{3.466258in}{3.310242in}}%
\pgfpathlineto{\pgfqpoint{3.456830in}{3.394390in}}%
\pgfpathlineto{\pgfqpoint{3.446620in}{3.483072in}}%
\pgfpathlineto{\pgfqpoint{3.450412in}{3.611019in}}%
\pgfpathlineto{\pgfqpoint{3.452766in}{3.698819in}}%
\pgfpathlineto{\pgfqpoint{3.457771in}{3.832779in}}%
\pgfpathlineto{\pgfqpoint{3.447056in}{3.844568in}}%
\pgfpathlineto{\pgfqpoint{3.466555in}{3.847394in}}%
\pgfpathlineto{\pgfqpoint{3.580149in}{3.854740in}}%
\pgfpathlineto{\pgfqpoint{3.690671in}{3.864679in}}%
\pgfpathlineto{\pgfqpoint{3.719749in}{3.762827in}}%
\pgfpathlineto{\pgfqpoint{3.739580in}{3.688066in}}%
\pgfpathlineto{\pgfqpoint{3.757355in}{3.625478in}}%
\pgfpathlineto{\pgfqpoint{3.767636in}{3.600276in}}%
\pgfpathlineto{\pgfqpoint{3.783823in}{3.576822in}}%
\pgfpathlineto{\pgfqpoint{3.781209in}{3.564887in}}%
\pgfpathlineto{\pgfqpoint{3.792772in}{3.559073in}}%
\pgfpathlineto{\pgfqpoint{3.779107in}{3.540981in}}%
\pgfpathlineto{\pgfqpoint{3.774492in}{3.508762in}}%
\pgfpathlineto{\pgfqpoint{3.787846in}{3.471078in}}%
\pgfpathlineto{\pgfqpoint{3.785992in}{3.433159in}}%
\pgfpathlineto{\pgfqpoint{3.798503in}{3.411653in}}%
\pgfusepath{stroke}%
\end{pgfscope}%
\begin{pgfscope}%
\pgfpathrectangle{\pgfqpoint{0.100000in}{2.413063in}}{\pgfqpoint{5.037500in}{3.427208in}}%
\pgfusepath{clip}%
\pgfsetbuttcap%
\pgfsetroundjoin%
\pgfsetlinewidth{0.501875pt}%
\definecolor{currentstroke}{rgb}{0.827451,0.827451,0.827451}%
\pgfsetstrokecolor{currentstroke}%
\pgfsetdash{}{0pt}%
\pgfpathmoveto{\pgfqpoint{3.466258in}{3.310242in}}%
\pgfpathlineto{\pgfqpoint{3.438437in}{3.302281in}}%
\pgfpathlineto{\pgfqpoint{3.417699in}{3.307393in}}%
\pgfpathlineto{\pgfqpoint{3.379188in}{3.295143in}}%
\pgfpathlineto{\pgfqpoint{3.350143in}{3.279367in}}%
\pgfpathlineto{\pgfqpoint{3.338225in}{3.307874in}}%
\pgfpathlineto{\pgfqpoint{3.325940in}{3.319823in}}%
\pgfpathlineto{\pgfqpoint{3.320014in}{3.334122in}}%
\pgfpathlineto{\pgfqpoint{3.328701in}{3.372823in}}%
\pgfpathlineto{\pgfqpoint{3.246390in}{3.367755in}}%
\pgfpathlineto{\pgfqpoint{3.139657in}{3.363127in}}%
\pgfpathlineto{\pgfqpoint{3.146007in}{3.372761in}}%
\pgfpathlineto{\pgfqpoint{3.138158in}{3.390955in}}%
\pgfpathlineto{\pgfqpoint{3.145979in}{3.394663in}}%
\pgfpathlineto{\pgfqpoint{3.144244in}{3.412144in}}%
\pgfpathlineto{\pgfqpoint{3.157876in}{3.429096in}}%
\pgfpathlineto{\pgfqpoint{3.165334in}{3.462006in}}%
\pgfpathlineto{\pgfqpoint{3.190283in}{3.483688in}}%
\pgfpathlineto{\pgfqpoint{3.181036in}{3.504745in}}%
\pgfpathlineto{\pgfqpoint{3.198727in}{3.516866in}}%
\pgfpathlineto{\pgfqpoint{3.183467in}{3.538809in}}%
\pgfpathlineto{\pgfqpoint{3.172421in}{3.589042in}}%
\pgfpathlineto{\pgfqpoint{3.176610in}{3.598163in}}%
\pgfpathlineto{\pgfqpoint{3.183224in}{3.615659in}}%
\pgfpathlineto{\pgfqpoint{3.177068in}{3.634020in}}%
\pgfpathlineto{\pgfqpoint{3.180079in}{3.642373in}}%
\pgfpathlineto{\pgfqpoint{3.169249in}{3.673932in}}%
\pgfpathlineto{\pgfqpoint{3.201219in}{3.730598in}}%
\pgfpathlineto{\pgfqpoint{3.202705in}{3.745680in}}%
\pgfpathlineto{\pgfqpoint{3.218105in}{3.756647in}}%
\pgfpathlineto{\pgfqpoint{3.234537in}{3.792851in}}%
\pgfpathlineto{\pgfqpoint{3.233756in}{3.807568in}}%
\pgfpathlineto{\pgfqpoint{3.254223in}{3.822611in}}%
\pgfpathlineto{\pgfqpoint{3.247837in}{3.831521in}}%
\pgfpathlineto{\pgfqpoint{3.370483in}{3.838678in}}%
\pgfpathlineto{\pgfqpoint{3.447056in}{3.844568in}}%
\pgfpathlineto{\pgfqpoint{3.457771in}{3.832779in}}%
\pgfpathlineto{\pgfqpoint{3.452766in}{3.698819in}}%
\pgfpathlineto{\pgfqpoint{3.450412in}{3.611019in}}%
\pgfpathlineto{\pgfqpoint{3.446620in}{3.483072in}}%
\pgfpathlineto{\pgfqpoint{3.456830in}{3.394390in}}%
\pgfpathlineto{\pgfqpoint{3.466258in}{3.310242in}}%
\pgfusepath{stroke}%
\end{pgfscope}%
\begin{pgfscope}%
\pgfpathrectangle{\pgfqpoint{0.100000in}{2.413063in}}{\pgfqpoint{5.037500in}{3.427208in}}%
\pgfusepath{clip}%
\pgfsetbuttcap%
\pgfsetroundjoin%
\pgfsetlinewidth{0.501875pt}%
\definecolor{currentstroke}{rgb}{0.827451,0.827451,0.827451}%
\pgfsetstrokecolor{currentstroke}%
\pgfsetdash{}{0pt}%
\pgfpathmoveto{\pgfqpoint{3.690671in}{3.864679in}}%
\pgfpathlineto{\pgfqpoint{3.811487in}{3.877948in}}%
\pgfpathlineto{\pgfqpoint{3.884181in}{3.886905in}}%
\pgfpathlineto{\pgfqpoint{3.925582in}{3.893502in}}%
\pgfpathlineto{\pgfqpoint{3.908432in}{3.866452in}}%
\pgfpathlineto{\pgfqpoint{3.908443in}{3.853698in}}%
\pgfpathlineto{\pgfqpoint{3.926096in}{3.846821in}}%
\pgfpathlineto{\pgfqpoint{3.938102in}{3.835717in}}%
\pgfpathlineto{\pgfqpoint{3.956245in}{3.834345in}}%
\pgfpathlineto{\pgfqpoint{3.973202in}{3.803087in}}%
\pgfpathlineto{\pgfqpoint{3.991596in}{3.781044in}}%
\pgfpathlineto{\pgfqpoint{4.024265in}{3.762518in}}%
\pgfpathlineto{\pgfqpoint{4.033550in}{3.747742in}}%
\pgfpathlineto{\pgfqpoint{4.062682in}{3.731761in}}%
\pgfpathlineto{\pgfqpoint{4.064872in}{3.720172in}}%
\pgfpathlineto{\pgfqpoint{4.088505in}{3.697128in}}%
\pgfpathlineto{\pgfqpoint{4.106607in}{3.690567in}}%
\pgfpathlineto{\pgfqpoint{4.119426in}{3.668808in}}%
\pgfpathlineto{\pgfqpoint{4.125101in}{3.644350in}}%
\pgfpathlineto{\pgfqpoint{4.143903in}{3.634896in}}%
\pgfpathlineto{\pgfqpoint{4.159040in}{3.608607in}}%
\pgfpathlineto{\pgfqpoint{4.163922in}{3.589272in}}%
\pgfpathlineto{\pgfqpoint{4.186104in}{3.581043in}}%
\pgfpathlineto{\pgfqpoint{4.167490in}{3.545404in}}%
\pgfpathlineto{\pgfqpoint{4.160485in}{3.524342in}}%
\pgfpathlineto{\pgfqpoint{4.165056in}{3.514470in}}%
\pgfpathlineto{\pgfqpoint{4.157076in}{3.498062in}}%
\pgfpathlineto{\pgfqpoint{4.153683in}{3.475378in}}%
\pgfpathlineto{\pgfqpoint{4.139492in}{3.471153in}}%
\pgfpathlineto{\pgfqpoint{4.146986in}{3.450384in}}%
\pgfpathlineto{\pgfqpoint{4.146508in}{3.424034in}}%
\pgfpathlineto{\pgfqpoint{4.121284in}{3.425022in}}%
\pgfpathlineto{\pgfqpoint{4.103711in}{3.429829in}}%
\pgfpathlineto{\pgfqpoint{4.096108in}{3.420892in}}%
\pgfpathlineto{\pgfqpoint{4.101781in}{3.399780in}}%
\pgfpathlineto{\pgfqpoint{4.100548in}{3.375235in}}%
\pgfpathlineto{\pgfqpoint{4.089462in}{3.373357in}}%
\pgfpathlineto{\pgfqpoint{4.080494in}{3.396281in}}%
\pgfpathlineto{\pgfqpoint{3.989179in}{3.390195in}}%
\pgfpathlineto{\pgfqpoint{3.874037in}{3.383913in}}%
\pgfpathlineto{\pgfqpoint{3.815967in}{3.379825in}}%
\pgfpathlineto{\pgfqpoint{3.798503in}{3.411653in}}%
\pgfpathlineto{\pgfqpoint{3.785992in}{3.433159in}}%
\pgfpathlineto{\pgfqpoint{3.787846in}{3.471078in}}%
\pgfpathlineto{\pgfqpoint{3.774492in}{3.508762in}}%
\pgfpathlineto{\pgfqpoint{3.779107in}{3.540981in}}%
\pgfpathlineto{\pgfqpoint{3.792772in}{3.559073in}}%
\pgfpathlineto{\pgfqpoint{3.781209in}{3.564887in}}%
\pgfpathlineto{\pgfqpoint{3.783823in}{3.576822in}}%
\pgfpathlineto{\pgfqpoint{3.767636in}{3.600276in}}%
\pgfpathlineto{\pgfqpoint{3.757355in}{3.625478in}}%
\pgfpathlineto{\pgfqpoint{3.739580in}{3.688066in}}%
\pgfpathlineto{\pgfqpoint{3.719749in}{3.762827in}}%
\pgfpathlineto{\pgfqpoint{3.690671in}{3.864679in}}%
\pgfusepath{stroke}%
\end{pgfscope}%
\begin{pgfscope}%
\pgfpathrectangle{\pgfqpoint{0.100000in}{2.413063in}}{\pgfqpoint{5.037500in}{3.427208in}}%
\pgfusepath{clip}%
\pgfsetbuttcap%
\pgfsetroundjoin%
\pgfsetlinewidth{0.501875pt}%
\definecolor{currentstroke}{rgb}{0.827451,0.827451,0.827451}%
\pgfsetstrokecolor{currentstroke}%
\pgfsetdash{}{0pt}%
\pgfpathmoveto{\pgfqpoint{3.925582in}{3.893502in}}%
\pgfpathlineto{\pgfqpoint{3.973490in}{3.917333in}}%
\pgfpathlineto{\pgfqpoint{4.000022in}{3.926332in}}%
\pgfpathlineto{\pgfqpoint{4.115457in}{3.938326in}}%
\pgfpathlineto{\pgfqpoint{4.127682in}{3.934353in}}%
\pgfpathlineto{\pgfqpoint{4.143831in}{3.918121in}}%
\pgfpathlineto{\pgfqpoint{4.144721in}{3.903707in}}%
\pgfpathlineto{\pgfqpoint{4.250965in}{3.919317in}}%
\pgfpathlineto{\pgfqpoint{4.371607in}{3.832835in}}%
\pgfpathlineto{\pgfqpoint{4.348983in}{3.809254in}}%
\pgfpathlineto{\pgfqpoint{4.317938in}{3.754490in}}%
\pgfpathlineto{\pgfqpoint{4.326775in}{3.742720in}}%
\pgfpathlineto{\pgfqpoint{4.310214in}{3.719876in}}%
\pgfpathlineto{\pgfqpoint{4.293741in}{3.717290in}}%
\pgfpathlineto{\pgfqpoint{4.293680in}{3.703562in}}%
\pgfpathlineto{\pgfqpoint{4.281764in}{3.689198in}}%
\pgfpathlineto{\pgfqpoint{4.266934in}{3.686285in}}%
\pgfpathlineto{\pgfqpoint{4.270170in}{3.673521in}}%
\pgfpathlineto{\pgfqpoint{4.261894in}{3.663723in}}%
\pgfpathlineto{\pgfqpoint{4.242099in}{3.655259in}}%
\pgfpathlineto{\pgfqpoint{4.217970in}{3.636005in}}%
\pgfpathlineto{\pgfqpoint{4.222525in}{3.623548in}}%
\pgfpathlineto{\pgfqpoint{4.207340in}{3.615727in}}%
\pgfpathlineto{\pgfqpoint{4.194533in}{3.623950in}}%
\pgfpathlineto{\pgfqpoint{4.185167in}{3.588197in}}%
\pgfpathlineto{\pgfqpoint{4.163922in}{3.589272in}}%
\pgfpathlineto{\pgfqpoint{4.159040in}{3.608607in}}%
\pgfpathlineto{\pgfqpoint{4.143903in}{3.634896in}}%
\pgfpathlineto{\pgfqpoint{4.125101in}{3.644350in}}%
\pgfpathlineto{\pgfqpoint{4.119426in}{3.668808in}}%
\pgfpathlineto{\pgfqpoint{4.106607in}{3.690567in}}%
\pgfpathlineto{\pgfqpoint{4.088505in}{3.697128in}}%
\pgfpathlineto{\pgfqpoint{4.064872in}{3.720172in}}%
\pgfpathlineto{\pgfqpoint{4.062682in}{3.731761in}}%
\pgfpathlineto{\pgfqpoint{4.033550in}{3.747742in}}%
\pgfpathlineto{\pgfqpoint{4.024265in}{3.762518in}}%
\pgfpathlineto{\pgfqpoint{3.991596in}{3.781044in}}%
\pgfpathlineto{\pgfqpoint{3.973202in}{3.803087in}}%
\pgfpathlineto{\pgfqpoint{3.956245in}{3.834345in}}%
\pgfpathlineto{\pgfqpoint{3.938102in}{3.835717in}}%
\pgfpathlineto{\pgfqpoint{3.926096in}{3.846821in}}%
\pgfpathlineto{\pgfqpoint{3.908443in}{3.853698in}}%
\pgfpathlineto{\pgfqpoint{3.908432in}{3.866452in}}%
\pgfpathlineto{\pgfqpoint{3.925582in}{3.893502in}}%
\pgfusepath{stroke}%
\end{pgfscope}%
\begin{pgfscope}%
\pgfpathrectangle{\pgfqpoint{0.100000in}{2.413063in}}{\pgfqpoint{5.037500in}{3.427208in}}%
\pgfusepath{clip}%
\pgfsetbuttcap%
\pgfsetroundjoin%
\pgfsetlinewidth{0.501875pt}%
\definecolor{currentstroke}{rgb}{0.827451,0.827451,0.827451}%
\pgfsetstrokecolor{currentstroke}%
\pgfsetdash{}{0pt}%
\pgfpathmoveto{\pgfqpoint{2.839596in}{3.991836in}}%
\pgfpathlineto{\pgfqpoint{2.909997in}{3.992469in}}%
\pgfpathlineto{\pgfqpoint{3.003135in}{3.994167in}}%
\pgfpathlineto{\pgfqpoint{3.162287in}{3.999219in}}%
\pgfpathlineto{\pgfqpoint{3.253318in}{4.004033in}}%
\pgfpathlineto{\pgfqpoint{3.263080in}{3.991948in}}%
\pgfpathlineto{\pgfqpoint{3.262470in}{3.979183in}}%
\pgfpathlineto{\pgfqpoint{3.240418in}{3.957162in}}%
\pgfpathlineto{\pgfqpoint{3.235087in}{3.945104in}}%
\pgfpathlineto{\pgfqpoint{3.296248in}{3.949628in}}%
\pgfpathlineto{\pgfqpoint{3.296214in}{3.927350in}}%
\pgfpathlineto{\pgfqpoint{3.276593in}{3.917829in}}%
\pgfpathlineto{\pgfqpoint{3.276871in}{3.902691in}}%
\pgfpathlineto{\pgfqpoint{3.269502in}{3.891811in}}%
\pgfpathlineto{\pgfqpoint{3.264679in}{3.868699in}}%
\pgfpathlineto{\pgfqpoint{3.269658in}{3.849634in}}%
\pgfpathlineto{\pgfqpoint{3.247837in}{3.831521in}}%
\pgfpathlineto{\pgfqpoint{3.254223in}{3.822611in}}%
\pgfpathlineto{\pgfqpoint{3.233756in}{3.807568in}}%
\pgfpathlineto{\pgfqpoint{3.234537in}{3.792851in}}%
\pgfpathlineto{\pgfqpoint{3.218105in}{3.756647in}}%
\pgfpathlineto{\pgfqpoint{3.202705in}{3.745680in}}%
\pgfpathlineto{\pgfqpoint{3.201219in}{3.730598in}}%
\pgfpathlineto{\pgfqpoint{3.169249in}{3.673932in}}%
\pgfpathlineto{\pgfqpoint{3.180079in}{3.642373in}}%
\pgfpathlineto{\pgfqpoint{3.177068in}{3.634020in}}%
\pgfpathlineto{\pgfqpoint{3.183224in}{3.615659in}}%
\pgfpathlineto{\pgfqpoint{3.176610in}{3.598163in}}%
\pgfpathlineto{\pgfqpoint{3.089189in}{3.594554in}}%
\pgfpathlineto{\pgfqpoint{2.975699in}{3.592684in}}%
\pgfpathlineto{\pgfqpoint{2.897420in}{3.591978in}}%
\pgfpathlineto{\pgfqpoint{2.897040in}{3.653613in}}%
\pgfpathlineto{\pgfqpoint{2.877599in}{3.657651in}}%
\pgfpathlineto{\pgfqpoint{2.864771in}{3.652378in}}%
\pgfpathlineto{\pgfqpoint{2.854521in}{3.662061in}}%
\pgfpathlineto{\pgfqpoint{2.855579in}{3.727350in}}%
\pgfpathlineto{\pgfqpoint{2.857873in}{3.866362in}}%
\pgfpathlineto{\pgfqpoint{2.846754in}{3.947764in}}%
\pgfpathlineto{\pgfqpoint{2.839596in}{3.991836in}}%
\pgfusepath{stroke}%
\end{pgfscope}%
\begin{pgfscope}%
\pgfpathrectangle{\pgfqpoint{0.100000in}{2.413063in}}{\pgfqpoint{5.037500in}{3.427208in}}%
\pgfusepath{clip}%
\pgfsetbuttcap%
\pgfsetroundjoin%
\pgfsetlinewidth{0.501875pt}%
\definecolor{currentstroke}{rgb}{0.827451,0.827451,0.827451}%
\pgfsetstrokecolor{currentstroke}%
\pgfsetdash{}{0pt}%
\pgfpathmoveto{\pgfqpoint{2.897420in}{3.591978in}}%
\pgfpathlineto{\pgfqpoint{2.975699in}{3.592684in}}%
\pgfpathlineto{\pgfqpoint{3.089189in}{3.594554in}}%
\pgfpathlineto{\pgfqpoint{3.176610in}{3.598163in}}%
\pgfpathlineto{\pgfqpoint{3.172421in}{3.589042in}}%
\pgfpathlineto{\pgfqpoint{3.183467in}{3.538809in}}%
\pgfpathlineto{\pgfqpoint{3.198727in}{3.516866in}}%
\pgfpathlineto{\pgfqpoint{3.181036in}{3.504745in}}%
\pgfpathlineto{\pgfqpoint{3.190283in}{3.483688in}}%
\pgfpathlineto{\pgfqpoint{3.165334in}{3.462006in}}%
\pgfpathlineto{\pgfqpoint{3.157876in}{3.429096in}}%
\pgfpathlineto{\pgfqpoint{3.144244in}{3.412144in}}%
\pgfpathlineto{\pgfqpoint{3.145979in}{3.394663in}}%
\pgfpathlineto{\pgfqpoint{3.138158in}{3.390955in}}%
\pgfpathlineto{\pgfqpoint{3.146007in}{3.372761in}}%
\pgfpathlineto{\pgfqpoint{3.139657in}{3.363127in}}%
\pgfpathlineto{\pgfqpoint{3.246390in}{3.367755in}}%
\pgfpathlineto{\pgfqpoint{3.328701in}{3.372823in}}%
\pgfpathlineto{\pgfqpoint{3.320014in}{3.334122in}}%
\pgfpathlineto{\pgfqpoint{3.325940in}{3.319823in}}%
\pgfpathlineto{\pgfqpoint{3.338225in}{3.307874in}}%
\pgfpathlineto{\pgfqpoint{3.350143in}{3.279367in}}%
\pgfpathlineto{\pgfqpoint{3.312500in}{3.285934in}}%
\pgfpathlineto{\pgfqpoint{3.298615in}{3.296732in}}%
\pgfpathlineto{\pgfqpoint{3.282070in}{3.297246in}}%
\pgfpathlineto{\pgfqpoint{3.264677in}{3.273602in}}%
\pgfpathlineto{\pgfqpoint{3.268143in}{3.262835in}}%
\pgfpathlineto{\pgfqpoint{3.323709in}{3.256321in}}%
\pgfpathlineto{\pgfqpoint{3.338256in}{3.243945in}}%
\pgfpathlineto{\pgfqpoint{3.361928in}{3.237578in}}%
\pgfpathlineto{\pgfqpoint{3.351589in}{3.222926in}}%
\pgfpathlineto{\pgfqpoint{3.334171in}{3.210148in}}%
\pgfpathlineto{\pgfqpoint{3.380146in}{3.181411in}}%
\pgfpathlineto{\pgfqpoint{3.401562in}{3.176921in}}%
\pgfpathlineto{\pgfqpoint{3.410468in}{3.153432in}}%
\pgfpathlineto{\pgfqpoint{3.390304in}{3.149111in}}%
\pgfpathlineto{\pgfqpoint{3.352407in}{3.178654in}}%
\pgfpathlineto{\pgfqpoint{3.337590in}{3.182760in}}%
\pgfpathlineto{\pgfqpoint{3.330420in}{3.194431in}}%
\pgfpathlineto{\pgfqpoint{3.308442in}{3.189641in}}%
\pgfpathlineto{\pgfqpoint{3.301561in}{3.174602in}}%
\pgfpathlineto{\pgfqpoint{3.305919in}{3.157845in}}%
\pgfpathlineto{\pgfqpoint{3.291123in}{3.147957in}}%
\pgfpathlineto{\pgfqpoint{3.277619in}{3.172318in}}%
\pgfpathlineto{\pgfqpoint{3.253967in}{3.165054in}}%
\pgfpathlineto{\pgfqpoint{3.245105in}{3.150519in}}%
\pgfpathlineto{\pgfqpoint{3.228308in}{3.154653in}}%
\pgfpathlineto{\pgfqpoint{3.217569in}{3.173009in}}%
\pgfpathlineto{\pgfqpoint{3.189078in}{3.179071in}}%
\pgfpathlineto{\pgfqpoint{3.183698in}{3.188622in}}%
\pgfpathlineto{\pgfqpoint{3.154014in}{3.205261in}}%
\pgfpathlineto{\pgfqpoint{3.146645in}{3.219809in}}%
\pgfpathlineto{\pgfqpoint{3.121782in}{3.213864in}}%
\pgfpathlineto{\pgfqpoint{3.124940in}{3.227183in}}%
\pgfpathlineto{\pgfqpoint{3.094036in}{3.213520in}}%
\pgfpathlineto{\pgfqpoint{3.102338in}{3.199344in}}%
\pgfpathlineto{\pgfqpoint{3.078472in}{3.190963in}}%
\pgfpathlineto{\pgfqpoint{3.046862in}{3.195569in}}%
\pgfpathlineto{\pgfqpoint{2.982884in}{3.217484in}}%
\pgfpathlineto{\pgfqpoint{2.933518in}{3.213134in}}%
\pgfpathlineto{\pgfqpoint{2.929239in}{3.241920in}}%
\pgfpathlineto{\pgfqpoint{2.935438in}{3.255055in}}%
\pgfpathlineto{\pgfqpoint{2.934908in}{3.275954in}}%
\pgfpathlineto{\pgfqpoint{2.928809in}{3.280933in}}%
\pgfpathlineto{\pgfqpoint{2.930876in}{3.304788in}}%
\pgfpathlineto{\pgfqpoint{2.948856in}{3.337967in}}%
\pgfpathlineto{\pgfqpoint{2.951156in}{3.350504in}}%
\pgfpathlineto{\pgfqpoint{2.948193in}{3.380114in}}%
\pgfpathlineto{\pgfqpoint{2.934966in}{3.394061in}}%
\pgfpathlineto{\pgfqpoint{2.928060in}{3.420103in}}%
\pgfpathlineto{\pgfqpoint{2.919562in}{3.426082in}}%
\pgfpathlineto{\pgfqpoint{2.923366in}{3.440204in}}%
\pgfpathlineto{\pgfqpoint{2.912544in}{3.461313in}}%
\pgfpathlineto{\pgfqpoint{2.899043in}{3.472768in}}%
\pgfpathlineto{\pgfqpoint{2.897420in}{3.591978in}}%
\pgfusepath{stroke}%
\end{pgfscope}%
\begin{pgfscope}%
\pgfpathrectangle{\pgfqpoint{0.100000in}{2.413063in}}{\pgfqpoint{5.037500in}{3.427208in}}%
\pgfusepath{clip}%
\pgfsetbuttcap%
\pgfsetroundjoin%
\pgfsetlinewidth{0.501875pt}%
\definecolor{currentstroke}{rgb}{0.827451,0.827451,0.827451}%
\pgfsetstrokecolor{currentstroke}%
\pgfsetdash{}{0pt}%
\pgfpathmoveto{\pgfqpoint{3.106777in}{3.198259in}}%
\pgfpathlineto{\pgfqpoint{3.118102in}{3.204994in}}%
\pgfpathlineto{\pgfqpoint{3.131847in}{3.197031in}}%
\pgfpathlineto{\pgfqpoint{3.124181in}{3.186073in}}%
\pgfpathlineto{\pgfqpoint{3.106777in}{3.198259in}}%
\pgfusepath{stroke}%
\end{pgfscope}%
\begin{pgfscope}%
\pgfpathrectangle{\pgfqpoint{0.100000in}{2.413063in}}{\pgfqpoint{5.037500in}{3.427208in}}%
\pgfusepath{clip}%
\pgfsetbuttcap%
\pgfsetroundjoin%
\pgfsetlinewidth{0.501875pt}%
\definecolor{currentstroke}{rgb}{0.827451,0.827451,0.827451}%
\pgfsetstrokecolor{currentstroke}%
\pgfsetdash{}{0pt}%
\pgfpathmoveto{\pgfqpoint{3.563787in}{3.328159in}}%
\pgfpathlineto{\pgfqpoint{3.563992in}{3.349936in}}%
\pgfpathlineto{\pgfqpoint{3.550461in}{3.358206in}}%
\pgfpathlineto{\pgfqpoint{3.539374in}{3.372285in}}%
\pgfpathlineto{\pgfqpoint{3.540884in}{3.387076in}}%
\pgfpathlineto{\pgfqpoint{3.661580in}{3.396445in}}%
\pgfpathlineto{\pgfqpoint{3.798503in}{3.411653in}}%
\pgfpathlineto{\pgfqpoint{3.815967in}{3.379825in}}%
\pgfpathlineto{\pgfqpoint{3.874037in}{3.383913in}}%
\pgfpathlineto{\pgfqpoint{3.989179in}{3.390195in}}%
\pgfpathlineto{\pgfqpoint{4.080494in}{3.396281in}}%
\pgfpathlineto{\pgfqpoint{4.089462in}{3.373357in}}%
\pgfpathlineto{\pgfqpoint{4.100548in}{3.375235in}}%
\pgfpathlineto{\pgfqpoint{4.101781in}{3.399780in}}%
\pgfpathlineto{\pgfqpoint{4.096108in}{3.420892in}}%
\pgfpathlineto{\pgfqpoint{4.103711in}{3.429829in}}%
\pgfpathlineto{\pgfqpoint{4.121284in}{3.425022in}}%
\pgfpathlineto{\pgfqpoint{4.146508in}{3.424034in}}%
\pgfpathlineto{\pgfqpoint{4.150386in}{3.405172in}}%
\pgfpathlineto{\pgfqpoint{4.158145in}{3.394394in}}%
\pgfpathlineto{\pgfqpoint{4.164205in}{3.370814in}}%
\pgfpathlineto{\pgfqpoint{4.182979in}{3.334321in}}%
\pgfpathlineto{\pgfqpoint{4.183073in}{3.324452in}}%
\pgfpathlineto{\pgfqpoint{4.210820in}{3.281602in}}%
\pgfpathlineto{\pgfqpoint{4.213504in}{3.272750in}}%
\pgfpathlineto{\pgfqpoint{4.255129in}{3.212232in}}%
\pgfpathlineto{\pgfqpoint{4.248535in}{3.211255in}}%
\pgfpathlineto{\pgfqpoint{4.266083in}{3.168210in}}%
\pgfpathlineto{\pgfqpoint{4.314290in}{3.093379in}}%
\pgfpathlineto{\pgfqpoint{4.342279in}{3.038358in}}%
\pgfpathlineto{\pgfqpoint{4.351634in}{3.028725in}}%
\pgfpathlineto{\pgfqpoint{4.367664in}{2.994190in}}%
\pgfpathlineto{\pgfqpoint{4.373225in}{2.938891in}}%
\pgfpathlineto{\pgfqpoint{4.375438in}{2.897542in}}%
\pgfpathlineto{\pgfqpoint{4.372765in}{2.871067in}}%
\pgfpathlineto{\pgfqpoint{4.364170in}{2.852141in}}%
\pgfpathlineto{\pgfqpoint{4.368164in}{2.827323in}}%
\pgfpathlineto{\pgfqpoint{4.358910in}{2.807678in}}%
\pgfpathlineto{\pgfqpoint{4.331435in}{2.791570in}}%
\pgfpathlineto{\pgfqpoint{4.313566in}{2.792746in}}%
\pgfpathlineto{\pgfqpoint{4.301962in}{2.784325in}}%
\pgfpathlineto{\pgfqpoint{4.298638in}{2.806808in}}%
\pgfpathlineto{\pgfqpoint{4.279430in}{2.812743in}}%
\pgfpathlineto{\pgfqpoint{4.262210in}{2.844115in}}%
\pgfpathlineto{\pgfqpoint{4.260267in}{2.858401in}}%
\pgfpathlineto{\pgfqpoint{4.229450in}{2.867076in}}%
\pgfpathlineto{\pgfqpoint{4.209592in}{2.865198in}}%
\pgfpathlineto{\pgfqpoint{4.198356in}{2.885927in}}%
\pgfpathlineto{\pgfqpoint{4.185492in}{2.923185in}}%
\pgfpathlineto{\pgfqpoint{4.167640in}{2.930738in}}%
\pgfpathlineto{\pgfqpoint{4.157962in}{2.952089in}}%
\pgfpathlineto{\pgfqpoint{4.134293in}{2.964672in}}%
\pgfpathlineto{\pgfqpoint{4.120612in}{2.980412in}}%
\pgfpathlineto{\pgfqpoint{4.097012in}{3.023203in}}%
\pgfpathlineto{\pgfqpoint{4.094236in}{3.053321in}}%
\pgfpathlineto{\pgfqpoint{4.107097in}{3.072591in}}%
\pgfpathlineto{\pgfqpoint{4.105811in}{3.085982in}}%
\pgfpathlineto{\pgfqpoint{4.090949in}{3.087413in}}%
\pgfpathlineto{\pgfqpoint{4.078532in}{3.096726in}}%
\pgfpathlineto{\pgfqpoint{4.071732in}{3.085353in}}%
\pgfpathlineto{\pgfqpoint{4.083649in}{3.076086in}}%
\pgfpathlineto{\pgfqpoint{4.074976in}{3.059422in}}%
\pgfpathlineto{\pgfqpoint{4.060944in}{3.073253in}}%
\pgfpathlineto{\pgfqpoint{4.062571in}{3.111608in}}%
\pgfpathlineto{\pgfqpoint{4.069340in}{3.142725in}}%
\pgfpathlineto{\pgfqpoint{4.065861in}{3.196118in}}%
\pgfpathlineto{\pgfqpoint{4.051874in}{3.208836in}}%
\pgfpathlineto{\pgfqpoint{4.044840in}{3.225155in}}%
\pgfpathlineto{\pgfqpoint{4.020746in}{3.224791in}}%
\pgfpathlineto{\pgfqpoint{3.996928in}{3.251644in}}%
\pgfpathlineto{\pgfqpoint{3.980959in}{3.259692in}}%
\pgfpathlineto{\pgfqpoint{3.976224in}{3.276697in}}%
\pgfpathlineto{\pgfqpoint{3.960604in}{3.282669in}}%
\pgfpathlineto{\pgfqpoint{3.947660in}{3.301445in}}%
\pgfpathlineto{\pgfqpoint{3.913464in}{3.316770in}}%
\pgfpathlineto{\pgfqpoint{3.886879in}{3.317162in}}%
\pgfpathlineto{\pgfqpoint{3.875306in}{3.311276in}}%
\pgfpathlineto{\pgfqpoint{3.878182in}{3.292901in}}%
\pgfpathlineto{\pgfqpoint{3.866114in}{3.293771in}}%
\pgfpathlineto{\pgfqpoint{3.828991in}{3.268080in}}%
\pgfpathlineto{\pgfqpoint{3.784400in}{3.257897in}}%
\pgfpathlineto{\pgfqpoint{3.783668in}{3.270496in}}%
\pgfpathlineto{\pgfqpoint{3.773786in}{3.282810in}}%
\pgfpathlineto{\pgfqpoint{3.747272in}{3.299810in}}%
\pgfpathlineto{\pgfqpoint{3.709215in}{3.317240in}}%
\pgfpathlineto{\pgfqpoint{3.667877in}{3.326466in}}%
\pgfpathlineto{\pgfqpoint{3.660118in}{3.339013in}}%
\pgfpathlineto{\pgfqpoint{3.645213in}{3.328549in}}%
\pgfpathlineto{\pgfqpoint{3.627266in}{3.326240in}}%
\pgfpathlineto{\pgfqpoint{3.564793in}{3.309780in}}%
\pgfpathlineto{\pgfqpoint{3.563787in}{3.328159in}}%
\pgfusepath{stroke}%
\end{pgfscope}%
\begin{pgfscope}%
\pgfpathrectangle{\pgfqpoint{0.100000in}{2.413063in}}{\pgfqpoint{5.037500in}{3.427208in}}%
\pgfusepath{clip}%
\pgfsetbuttcap%
\pgfsetroundjoin%
\pgfsetlinewidth{0.501875pt}%
\definecolor{currentstroke}{rgb}{0.827451,0.827451,0.827451}%
\pgfsetstrokecolor{currentstroke}%
\pgfsetdash{}{0pt}%
\pgfpathmoveto{\pgfqpoint{4.261195in}{3.213093in}}%
\pgfpathlineto{\pgfqpoint{4.277358in}{3.194291in}}%
\pgfpathlineto{\pgfqpoint{4.258688in}{3.193108in}}%
\pgfpathlineto{\pgfqpoint{4.261195in}{3.213093in}}%
\pgfusepath{stroke}%
\end{pgfscope}%
\begin{pgfscope}%
\pgfpathrectangle{\pgfqpoint{0.100000in}{2.413063in}}{\pgfqpoint{5.037500in}{3.427208in}}%
\pgfusepath{clip}%
\pgfsetbuttcap%
\pgfsetroundjoin%
\pgfsetlinewidth{0.501875pt}%
\definecolor{currentstroke}{rgb}{0.827451,0.827451,0.827451}%
\pgfsetstrokecolor{currentstroke}%
\pgfsetdash{}{0pt}%
\pgfpathmoveto{\pgfqpoint{3.309667in}{5.359952in}}%
\pgfpathlineto{\pgfqpoint{3.300898in}{5.342883in}}%
\pgfpathlineto{\pgfqpoint{3.279976in}{5.332921in}}%
\pgfpathlineto{\pgfqpoint{3.270914in}{5.319513in}}%
\pgfpathlineto{\pgfqpoint{3.260299in}{5.329170in}}%
\pgfpathlineto{\pgfqpoint{3.309667in}{5.359952in}}%
\pgfusepath{stroke}%
\end{pgfscope}%
\begin{pgfscope}%
\pgfpathrectangle{\pgfqpoint{0.100000in}{2.413063in}}{\pgfqpoint{5.037500in}{3.427208in}}%
\pgfusepath{clip}%
\pgfsetbuttcap%
\pgfsetroundjoin%
\pgfsetlinewidth{0.501875pt}%
\definecolor{currentstroke}{rgb}{0.827451,0.827451,0.827451}%
\pgfsetstrokecolor{currentstroke}%
\pgfsetdash{}{0pt}%
\pgfpathmoveto{\pgfqpoint{3.316781in}{5.257228in}}%
\pgfpathlineto{\pgfqpoint{3.338188in}{5.277198in}}%
\pgfpathlineto{\pgfqpoint{3.371207in}{5.282428in}}%
\pgfpathlineto{\pgfqpoint{3.362113in}{5.268539in}}%
\pgfpathlineto{\pgfqpoint{3.339476in}{5.248453in}}%
\pgfpathlineto{\pgfqpoint{3.326242in}{5.222673in}}%
\pgfpathlineto{\pgfqpoint{3.320637in}{5.236656in}}%
\pgfpathlineto{\pgfqpoint{3.310685in}{5.238663in}}%
\pgfpathlineto{\pgfqpoint{3.309685in}{5.251287in}}%
\pgfpathlineto{\pgfqpoint{3.316781in}{5.257228in}}%
\pgfusepath{stroke}%
\end{pgfscope}%
\begin{pgfscope}%
\pgfpathrectangle{\pgfqpoint{0.100000in}{2.413063in}}{\pgfqpoint{5.037500in}{3.427208in}}%
\pgfusepath{clip}%
\pgfsetbuttcap%
\pgfsetroundjoin%
\pgfsetlinewidth{0.501875pt}%
\definecolor{currentstroke}{rgb}{0.827451,0.827451,0.827451}%
\pgfsetstrokecolor{currentstroke}%
\pgfsetdash{}{0pt}%
\pgfpathmoveto{\pgfqpoint{3.402044in}{5.013479in}}%
\pgfpathlineto{\pgfqpoint{3.396343in}{5.019806in}}%
\pgfpathlineto{\pgfqpoint{3.402337in}{5.037617in}}%
\pgfpathlineto{\pgfqpoint{3.384545in}{5.038752in}}%
\pgfpathlineto{\pgfqpoint{3.389261in}{5.054109in}}%
\pgfpathlineto{\pgfqpoint{3.386328in}{5.078566in}}%
\pgfpathlineto{\pgfqpoint{3.370364in}{5.087046in}}%
\pgfpathlineto{\pgfqpoint{3.353658in}{5.104206in}}%
\pgfpathlineto{\pgfqpoint{3.327916in}{5.109231in}}%
\pgfpathlineto{\pgfqpoint{3.302863in}{5.109087in}}%
\pgfpathlineto{\pgfqpoint{3.278223in}{5.121266in}}%
\pgfpathlineto{\pgfqpoint{3.195789in}{5.139018in}}%
\pgfpathlineto{\pgfqpoint{3.186780in}{5.157828in}}%
\pgfpathlineto{\pgfqpoint{3.170712in}{5.164247in}}%
\pgfpathlineto{\pgfqpoint{3.201064in}{5.178648in}}%
\pgfpathlineto{\pgfqpoint{3.218211in}{5.196616in}}%
\pgfpathlineto{\pgfqpoint{3.250147in}{5.201489in}}%
\pgfpathlineto{\pgfqpoint{3.269800in}{5.219789in}}%
\pgfpathlineto{\pgfqpoint{3.280110in}{5.220522in}}%
\pgfpathlineto{\pgfqpoint{3.287979in}{5.233574in}}%
\pgfpathlineto{\pgfqpoint{3.307161in}{5.249035in}}%
\pgfpathlineto{\pgfqpoint{3.308833in}{5.238107in}}%
\pgfpathlineto{\pgfqpoint{3.317476in}{5.235850in}}%
\pgfpathlineto{\pgfqpoint{3.323979in}{5.222817in}}%
\pgfpathlineto{\pgfqpoint{3.325151in}{5.200597in}}%
\pgfpathlineto{\pgfqpoint{3.344696in}{5.213876in}}%
\pgfpathlineto{\pgfqpoint{3.367471in}{5.216649in}}%
\pgfpathlineto{\pgfqpoint{3.386912in}{5.209693in}}%
\pgfpathlineto{\pgfqpoint{3.413276in}{5.173524in}}%
\pgfpathlineto{\pgfqpoint{3.442071in}{5.179304in}}%
\pgfpathlineto{\pgfqpoint{3.453772in}{5.169624in}}%
\pgfpathlineto{\pgfqpoint{3.472590in}{5.168760in}}%
\pgfpathlineto{\pgfqpoint{3.485093in}{5.186120in}}%
\pgfpathlineto{\pgfqpoint{3.508817in}{5.201466in}}%
\pgfpathlineto{\pgfqpoint{3.531619in}{5.206272in}}%
\pgfpathlineto{\pgfqpoint{3.559912in}{5.206773in}}%
\pgfpathlineto{\pgfqpoint{3.580587in}{5.218627in}}%
\pgfpathlineto{\pgfqpoint{3.597460in}{5.213167in}}%
\pgfpathlineto{\pgfqpoint{3.601054in}{5.188095in}}%
\pgfpathlineto{\pgfqpoint{3.637253in}{5.184169in}}%
\pgfpathlineto{\pgfqpoint{3.656916in}{5.195905in}}%
\pgfpathlineto{\pgfqpoint{3.670546in}{5.169467in}}%
\pgfpathlineto{\pgfqpoint{3.696553in}{5.138895in}}%
\pgfpathlineto{\pgfqpoint{3.660152in}{5.139072in}}%
\pgfpathlineto{\pgfqpoint{3.648651in}{5.135292in}}%
\pgfpathlineto{\pgfqpoint{3.632886in}{5.140204in}}%
\pgfpathlineto{\pgfqpoint{3.631825in}{5.119039in}}%
\pgfpathlineto{\pgfqpoint{3.603200in}{5.135579in}}%
\pgfpathlineto{\pgfqpoint{3.566388in}{5.140633in}}%
\pgfpathlineto{\pgfqpoint{3.556253in}{5.124526in}}%
\pgfpathlineto{\pgfqpoint{3.508088in}{5.116693in}}%
\pgfpathlineto{\pgfqpoint{3.502546in}{5.103048in}}%
\pgfpathlineto{\pgfqpoint{3.469029in}{5.098953in}}%
\pgfpathlineto{\pgfqpoint{3.458896in}{5.085082in}}%
\pgfpathlineto{\pgfqpoint{3.441168in}{5.081363in}}%
\pgfpathlineto{\pgfqpoint{3.427007in}{5.048470in}}%
\pgfpathlineto{\pgfqpoint{3.409079in}{5.016614in}}%
\pgfpathlineto{\pgfqpoint{3.402044in}{5.013479in}}%
\pgfusepath{stroke}%
\end{pgfscope}%
\begin{pgfscope}%
\pgfpathrectangle{\pgfqpoint{0.100000in}{2.413063in}}{\pgfqpoint{5.037500in}{3.427208in}}%
\pgfusepath{clip}%
\pgfsetbuttcap%
\pgfsetroundjoin%
\pgfsetlinewidth{0.501875pt}%
\definecolor{currentstroke}{rgb}{0.827451,0.827451,0.827451}%
\pgfsetstrokecolor{currentstroke}%
\pgfsetdash{}{0pt}%
\pgfpathmoveto{\pgfqpoint{3.505233in}{4.631021in}}%
\pgfpathlineto{\pgfqpoint{3.522326in}{4.649013in}}%
\pgfpathlineto{\pgfqpoint{3.530113in}{4.675098in}}%
\pgfpathlineto{\pgfqpoint{3.539374in}{4.690218in}}%
\pgfpathlineto{\pgfqpoint{3.545053in}{4.710794in}}%
\pgfpathlineto{\pgfqpoint{3.546819in}{4.751821in}}%
\pgfpathlineto{\pgfqpoint{3.538268in}{4.791140in}}%
\pgfpathlineto{\pgfqpoint{3.510013in}{4.851348in}}%
\pgfpathlineto{\pgfqpoint{3.517665in}{4.870279in}}%
\pgfpathlineto{\pgfqpoint{3.507697in}{4.896446in}}%
\pgfpathlineto{\pgfqpoint{3.524796in}{4.932634in}}%
\pgfpathlineto{\pgfqpoint{3.522010in}{4.972984in}}%
\pgfpathlineto{\pgfqpoint{3.533920in}{4.978063in}}%
\pgfpathlineto{\pgfqpoint{3.535416in}{4.997262in}}%
\pgfpathlineto{\pgfqpoint{3.556558in}{5.009585in}}%
\pgfpathlineto{\pgfqpoint{3.568515in}{5.007596in}}%
\pgfpathlineto{\pgfqpoint{3.571876in}{4.987005in}}%
\pgfpathlineto{\pgfqpoint{3.581208in}{4.986198in}}%
\pgfpathlineto{\pgfqpoint{3.589723in}{5.015898in}}%
\pgfpathlineto{\pgfqpoint{3.586829in}{5.038997in}}%
\pgfpathlineto{\pgfqpoint{3.592365in}{5.052261in}}%
\pgfpathlineto{\pgfqpoint{3.622298in}{5.065964in}}%
\pgfpathlineto{\pgfqpoint{3.608658in}{5.070841in}}%
\pgfpathlineto{\pgfqpoint{3.604215in}{5.082612in}}%
\pgfpathlineto{\pgfqpoint{3.614041in}{5.103283in}}%
\pgfpathlineto{\pgfqpoint{3.633417in}{5.110422in}}%
\pgfpathlineto{\pgfqpoint{3.655918in}{5.098226in}}%
\pgfpathlineto{\pgfqpoint{3.677133in}{5.098074in}}%
\pgfpathlineto{\pgfqpoint{3.686985in}{5.083827in}}%
\pgfpathlineto{\pgfqpoint{3.701842in}{5.084817in}}%
\pgfpathlineto{\pgfqpoint{3.747705in}{5.065012in}}%
\pgfpathlineto{\pgfqpoint{3.756869in}{5.045840in}}%
\pgfpathlineto{\pgfqpoint{3.746861in}{5.039200in}}%
\pgfpathlineto{\pgfqpoint{3.749945in}{5.024832in}}%
\pgfpathlineto{\pgfqpoint{3.759831in}{5.018442in}}%
\pgfpathlineto{\pgfqpoint{3.765341in}{5.000779in}}%
\pgfpathlineto{\pgfqpoint{3.764570in}{4.957748in}}%
\pgfpathlineto{\pgfqpoint{3.751517in}{4.947469in}}%
\pgfpathlineto{\pgfqpoint{3.748587in}{4.924866in}}%
\pgfpathlineto{\pgfqpoint{3.724374in}{4.903977in}}%
\pgfpathlineto{\pgfqpoint{3.725811in}{4.878692in}}%
\pgfpathlineto{\pgfqpoint{3.747033in}{4.869835in}}%
\pgfpathlineto{\pgfqpoint{3.770974in}{4.901382in}}%
\pgfpathlineto{\pgfqpoint{3.772953in}{4.912824in}}%
\pgfpathlineto{\pgfqpoint{3.802811in}{4.931842in}}%
\pgfpathlineto{\pgfqpoint{3.821798in}{4.923033in}}%
\pgfpathlineto{\pgfqpoint{3.833713in}{4.903124in}}%
\pgfpathlineto{\pgfqpoint{3.852934in}{4.834001in}}%
\pgfpathlineto{\pgfqpoint{3.863121in}{4.812132in}}%
\pgfpathlineto{\pgfqpoint{3.859932in}{4.803091in}}%
\pgfpathlineto{\pgfqpoint{3.860241in}{4.772307in}}%
\pgfpathlineto{\pgfqpoint{3.841712in}{4.775262in}}%
\pgfpathlineto{\pgfqpoint{3.831247in}{4.752258in}}%
\pgfpathlineto{\pgfqpoint{3.829796in}{4.736619in}}%
\pgfpathlineto{\pgfqpoint{3.815800in}{4.726580in}}%
\pgfpathlineto{\pgfqpoint{3.812690in}{4.696081in}}%
\pgfpathlineto{\pgfqpoint{3.792336in}{4.657553in}}%
\pgfpathlineto{\pgfqpoint{3.680994in}{4.640971in}}%
\pgfpathlineto{\pgfqpoint{3.680336in}{4.648259in}}%
\pgfpathlineto{\pgfqpoint{3.605857in}{4.640409in}}%
\pgfpathlineto{\pgfqpoint{3.505233in}{4.631021in}}%
\pgfusepath{stroke}%
\end{pgfscope}%
\begin{pgfscope}%
\pgfpathrectangle{\pgfqpoint{0.100000in}{2.413063in}}{\pgfqpoint{5.037500in}{3.427208in}}%
\pgfusepath{clip}%
\pgfsetbuttcap%
\pgfsetroundjoin%
\pgfsetlinewidth{0.501875pt}%
\definecolor{currentstroke}{rgb}{0.827451,0.827451,0.827451}%
\pgfsetstrokecolor{currentstroke}%
\pgfsetdash{}{0pt}%
\pgfpathmoveto{\pgfqpoint{0.000000in}{0.000000in}}%
\pgfusepath{stroke}%
\end{pgfscope}%
\begin{pgfscope}%
\pgfpathrectangle{\pgfqpoint{0.100000in}{2.413063in}}{\pgfqpoint{5.037500in}{3.427208in}}%
\pgfusepath{clip}%
\pgfsetbuttcap%
\pgfsetroundjoin%
\pgfsetlinewidth{0.501875pt}%
\definecolor{currentstroke}{rgb}{0.827451,0.827451,0.827451}%
\pgfsetstrokecolor{currentstroke}%
\pgfsetdash{}{0pt}%
\pgfusepath{stroke}%
\end{pgfscope}%
\begin{pgfscope}%
\pgfpathrectangle{\pgfqpoint{0.100000in}{2.413063in}}{\pgfqpoint{5.037500in}{3.427208in}}%
\pgfusepath{clip}%
\pgfsetbuttcap%
\pgfsetroundjoin%
\pgfsetlinewidth{0.501875pt}%
\definecolor{currentstroke}{rgb}{0.827451,0.827451,0.827451}%
\pgfsetstrokecolor{currentstroke}%
\pgfsetdash{}{0pt}%
\pgfusepath{stroke}%
\end{pgfscope}%
\begin{pgfscope}%
\pgfpathrectangle{\pgfqpoint{0.100000in}{2.413063in}}{\pgfqpoint{5.037500in}{3.427208in}}%
\pgfusepath{clip}%
\pgfsetbuttcap%
\pgfsetroundjoin%
\pgfsetlinewidth{0.501875pt}%
\definecolor{currentstroke}{rgb}{0.827451,0.827451,0.827451}%
\pgfsetstrokecolor{currentstroke}%
\pgfsetdash{}{0pt}%
\pgfusepath{stroke}%
\end{pgfscope}%
\begin{pgfscope}%
\pgfpathrectangle{\pgfqpoint{0.100000in}{2.413063in}}{\pgfqpoint{5.037500in}{3.427208in}}%
\pgfusepath{clip}%
\pgfsetbuttcap%
\pgfsetroundjoin%
\pgfsetlinewidth{0.501875pt}%
\definecolor{currentstroke}{rgb}{0.827451,0.827451,0.827451}%
\pgfsetstrokecolor{currentstroke}%
\pgfsetdash{}{0pt}%
\pgfusepath{stroke}%
\end{pgfscope}%
\begin{pgfscope}%
\pgfpathrectangle{\pgfqpoint{0.100000in}{2.413063in}}{\pgfqpoint{5.037500in}{3.427208in}}%
\pgfusepath{clip}%
\pgfsetbuttcap%
\pgfsetroundjoin%
\pgfsetlinewidth{0.501875pt}%
\definecolor{currentstroke}{rgb}{0.827451,0.827451,0.827451}%
\pgfsetstrokecolor{currentstroke}%
\pgfsetdash{}{0pt}%
\pgfusepath{stroke}%
\end{pgfscope}%
\begin{pgfscope}%
\pgfpathrectangle{\pgfqpoint{0.100000in}{2.413063in}}{\pgfqpoint{5.037500in}{3.427208in}}%
\pgfusepath{clip}%
\pgfsetbuttcap%
\pgfsetroundjoin%
\pgfsetlinewidth{0.501875pt}%
\definecolor{currentstroke}{rgb}{0.827451,0.827451,0.827451}%
\pgfsetstrokecolor{currentstroke}%
\pgfsetdash{}{0pt}%
\pgfusepath{stroke}%
\end{pgfscope}%
\begin{pgfscope}%
\pgfpathrectangle{\pgfqpoint{0.100000in}{2.413063in}}{\pgfqpoint{5.037500in}{3.427208in}}%
\pgfusepath{clip}%
\pgfsetbuttcap%
\pgfsetroundjoin%
\pgfsetlinewidth{0.501875pt}%
\definecolor{currentstroke}{rgb}{0.827451,0.827451,0.827451}%
\pgfsetstrokecolor{currentstroke}%
\pgfsetdash{}{0pt}%
\pgfusepath{stroke}%
\end{pgfscope}%
\begin{pgfscope}%
\pgfpathrectangle{\pgfqpoint{0.100000in}{2.413063in}}{\pgfqpoint{5.037500in}{3.427208in}}%
\pgfusepath{clip}%
\pgfsetbuttcap%
\pgfsetroundjoin%
\pgfsetlinewidth{0.501875pt}%
\definecolor{currentstroke}{rgb}{0.827451,0.827451,0.827451}%
\pgfsetstrokecolor{currentstroke}%
\pgfsetdash{}{0pt}%
\pgfusepath{stroke}%
\end{pgfscope}%
\begin{pgfscope}%
\pgfpathrectangle{\pgfqpoint{0.100000in}{2.413063in}}{\pgfqpoint{5.037500in}{3.427208in}}%
\pgfusepath{clip}%
\pgfsetbuttcap%
\pgfsetroundjoin%
\pgfsetlinewidth{0.501875pt}%
\definecolor{currentstroke}{rgb}{0.827451,0.827451,0.827451}%
\pgfsetstrokecolor{currentstroke}%
\pgfsetdash{}{0pt}%
\pgfusepath{stroke}%
\end{pgfscope}%
\begin{pgfscope}%
\pgfpathrectangle{\pgfqpoint{0.100000in}{2.413063in}}{\pgfqpoint{5.037500in}{3.427208in}}%
\pgfusepath{clip}%
\pgfsetbuttcap%
\pgfsetroundjoin%
\pgfsetlinewidth{0.501875pt}%
\definecolor{currentstroke}{rgb}{0.827451,0.827451,0.827451}%
\pgfsetstrokecolor{currentstroke}%
\pgfsetdash{}{0pt}%
\pgfusepath{stroke}%
\end{pgfscope}%
\begin{pgfscope}%
\pgfpathrectangle{\pgfqpoint{0.100000in}{2.413063in}}{\pgfqpoint{5.037500in}{3.427208in}}%
\pgfusepath{clip}%
\pgfsetbuttcap%
\pgfsetroundjoin%
\pgfsetlinewidth{0.501875pt}%
\definecolor{currentstroke}{rgb}{0.827451,0.827451,0.827451}%
\pgfsetstrokecolor{currentstroke}%
\pgfsetdash{}{0pt}%
\pgfusepath{stroke}%
\end{pgfscope}%
\begin{pgfscope}%
\pgfpathrectangle{\pgfqpoint{0.100000in}{2.413063in}}{\pgfqpoint{5.037500in}{3.427208in}}%
\pgfusepath{clip}%
\pgfsetbuttcap%
\pgfsetroundjoin%
\pgfsetlinewidth{0.501875pt}%
\definecolor{currentstroke}{rgb}{0.827451,0.827451,0.827451}%
\pgfsetstrokecolor{currentstroke}%
\pgfsetdash{}{0pt}%
\pgfusepath{stroke}%
\end{pgfscope}%
\begin{pgfscope}%
\pgfpathrectangle{\pgfqpoint{0.100000in}{2.413063in}}{\pgfqpoint{5.037500in}{3.427208in}}%
\pgfusepath{clip}%
\pgfsetbuttcap%
\pgfsetroundjoin%
\pgfsetlinewidth{0.501875pt}%
\definecolor{currentstroke}{rgb}{0.827451,0.827451,0.827451}%
\pgfsetstrokecolor{currentstroke}%
\pgfsetdash{}{0pt}%
\pgfusepath{stroke}%
\end{pgfscope}%
\begin{pgfscope}%
\pgfpathrectangle{\pgfqpoint{0.100000in}{2.413063in}}{\pgfqpoint{5.037500in}{3.427208in}}%
\pgfusepath{clip}%
\pgfsetbuttcap%
\pgfsetroundjoin%
\pgfsetlinewidth{0.501875pt}%
\definecolor{currentstroke}{rgb}{0.827451,0.827451,0.827451}%
\pgfsetstrokecolor{currentstroke}%
\pgfsetdash{}{0pt}%
\pgfusepath{stroke}%
\end{pgfscope}%
\begin{pgfscope}%
\pgfpathrectangle{\pgfqpoint{0.100000in}{2.413063in}}{\pgfqpoint{5.037500in}{3.427208in}}%
\pgfusepath{clip}%
\pgfsetbuttcap%
\pgfsetroundjoin%
\pgfsetlinewidth{0.501875pt}%
\definecolor{currentstroke}{rgb}{0.827451,0.827451,0.827451}%
\pgfsetstrokecolor{currentstroke}%
\pgfsetdash{}{0pt}%
\pgfusepath{stroke}%
\end{pgfscope}%
\begin{pgfscope}%
\pgfpathrectangle{\pgfqpoint{0.100000in}{2.413063in}}{\pgfqpoint{5.037500in}{3.427208in}}%
\pgfusepath{clip}%
\pgfsetbuttcap%
\pgfsetroundjoin%
\pgfsetlinewidth{0.501875pt}%
\definecolor{currentstroke}{rgb}{0.827451,0.827451,0.827451}%
\pgfsetstrokecolor{currentstroke}%
\pgfsetdash{}{0pt}%
\pgfusepath{stroke}%
\end{pgfscope}%
\begin{pgfscope}%
\pgfpathrectangle{\pgfqpoint{0.100000in}{2.413063in}}{\pgfqpoint{5.037500in}{3.427208in}}%
\pgfusepath{clip}%
\pgfsetbuttcap%
\pgfsetroundjoin%
\pgfsetlinewidth{0.501875pt}%
\definecolor{currentstroke}{rgb}{0.827451,0.827451,0.827451}%
\pgfsetstrokecolor{currentstroke}%
\pgfsetdash{}{0pt}%
\pgfusepath{stroke}%
\end{pgfscope}%
\begin{pgfscope}%
\pgfpathrectangle{\pgfqpoint{0.100000in}{2.413063in}}{\pgfqpoint{5.037500in}{3.427208in}}%
\pgfusepath{clip}%
\pgfsetbuttcap%
\pgfsetroundjoin%
\pgfsetlinewidth{0.501875pt}%
\definecolor{currentstroke}{rgb}{0.827451,0.827451,0.827451}%
\pgfsetstrokecolor{currentstroke}%
\pgfsetdash{}{0pt}%
\pgfusepath{stroke}%
\end{pgfscope}%
\begin{pgfscope}%
\pgfpathrectangle{\pgfqpoint{0.100000in}{2.413063in}}{\pgfqpoint{5.037500in}{3.427208in}}%
\pgfusepath{clip}%
\pgfsetbuttcap%
\pgfsetroundjoin%
\pgfsetlinewidth{0.501875pt}%
\definecolor{currentstroke}{rgb}{0.827451,0.827451,0.827451}%
\pgfsetstrokecolor{currentstroke}%
\pgfsetdash{}{0pt}%
\pgfusepath{stroke}%
\end{pgfscope}%
\begin{pgfscope}%
\pgfpathrectangle{\pgfqpoint{0.100000in}{2.413063in}}{\pgfqpoint{5.037500in}{3.427208in}}%
\pgfusepath{clip}%
\pgfsetbuttcap%
\pgfsetroundjoin%
\pgfsetlinewidth{0.501875pt}%
\definecolor{currentstroke}{rgb}{0.827451,0.827451,0.827451}%
\pgfsetstrokecolor{currentstroke}%
\pgfsetdash{}{0pt}%
\pgfusepath{stroke}%
\end{pgfscope}%
\begin{pgfscope}%
\pgfpathrectangle{\pgfqpoint{0.100000in}{2.413063in}}{\pgfqpoint{5.037500in}{3.427208in}}%
\pgfusepath{clip}%
\pgfsetbuttcap%
\pgfsetroundjoin%
\pgfsetlinewidth{0.501875pt}%
\definecolor{currentstroke}{rgb}{0.827451,0.827451,0.827451}%
\pgfsetstrokecolor{currentstroke}%
\pgfsetdash{}{0pt}%
\pgfusepath{stroke}%
\end{pgfscope}%
\begin{pgfscope}%
\pgfpathrectangle{\pgfqpoint{0.100000in}{2.413063in}}{\pgfqpoint{5.037500in}{3.427208in}}%
\pgfusepath{clip}%
\pgfsetbuttcap%
\pgfsetroundjoin%
\pgfsetlinewidth{0.501875pt}%
\definecolor{currentstroke}{rgb}{0.827451,0.827451,0.827451}%
\pgfsetstrokecolor{currentstroke}%
\pgfsetdash{}{0pt}%
\pgfusepath{stroke}%
\end{pgfscope}%
\begin{pgfscope}%
\pgfpathrectangle{\pgfqpoint{0.100000in}{2.413063in}}{\pgfqpoint{5.037500in}{3.427208in}}%
\pgfusepath{clip}%
\pgfsetbuttcap%
\pgfsetroundjoin%
\pgfsetlinewidth{0.501875pt}%
\definecolor{currentstroke}{rgb}{0.827451,0.827451,0.827451}%
\pgfsetstrokecolor{currentstroke}%
\pgfsetdash{}{0pt}%
\pgfusepath{stroke}%
\end{pgfscope}%
\begin{pgfscope}%
\pgfpathrectangle{\pgfqpoint{0.100000in}{2.413063in}}{\pgfqpoint{5.037500in}{3.427208in}}%
\pgfusepath{clip}%
\pgfsetbuttcap%
\pgfsetroundjoin%
\pgfsetlinewidth{0.501875pt}%
\definecolor{currentstroke}{rgb}{0.827451,0.827451,0.827451}%
\pgfsetstrokecolor{currentstroke}%
\pgfsetdash{}{0pt}%
\pgfusepath{stroke}%
\end{pgfscope}%
\begin{pgfscope}%
\pgfpathrectangle{\pgfqpoint{0.100000in}{2.413063in}}{\pgfqpoint{5.037500in}{3.427208in}}%
\pgfusepath{clip}%
\pgfsetbuttcap%
\pgfsetroundjoin%
\pgfsetlinewidth{0.501875pt}%
\definecolor{currentstroke}{rgb}{0.827451,0.827451,0.827451}%
\pgfsetstrokecolor{currentstroke}%
\pgfsetdash{}{0pt}%
\pgfusepath{stroke}%
\end{pgfscope}%
\begin{pgfscope}%
\pgfpathrectangle{\pgfqpoint{0.100000in}{2.413063in}}{\pgfqpoint{5.037500in}{3.427208in}}%
\pgfusepath{clip}%
\pgfsetbuttcap%
\pgfsetroundjoin%
\pgfsetlinewidth{0.501875pt}%
\definecolor{currentstroke}{rgb}{0.827451,0.827451,0.827451}%
\pgfsetstrokecolor{currentstroke}%
\pgfsetdash{}{0pt}%
\pgfusepath{stroke}%
\end{pgfscope}%
\begin{pgfscope}%
\pgfpathrectangle{\pgfqpoint{0.100000in}{2.413063in}}{\pgfqpoint{5.037500in}{3.427208in}}%
\pgfusepath{clip}%
\pgfsetbuttcap%
\pgfsetroundjoin%
\pgfsetlinewidth{0.501875pt}%
\definecolor{currentstroke}{rgb}{0.827451,0.827451,0.827451}%
\pgfsetstrokecolor{currentstroke}%
\pgfsetdash{}{0pt}%
\pgfusepath{stroke}%
\end{pgfscope}%
\begin{pgfscope}%
\pgfpathrectangle{\pgfqpoint{0.100000in}{2.413063in}}{\pgfqpoint{5.037500in}{3.427208in}}%
\pgfusepath{clip}%
\pgfsetrectcap%
\pgfsetroundjoin%
\pgfsetlinewidth{1.505625pt}%
\definecolor{currentstroke}{rgb}{0.000000,0.000000,1.000000}%
\pgfsetstrokecolor{currentstroke}%
\pgfsetstrokeopacity{0.500000}%
\pgfsetdash{}{0pt}%
\pgfpathmoveto{\pgfqpoint{3.684003in}{3.713619in}}%
\pgfusepath{stroke}%
\end{pgfscope}%
\begin{pgfscope}%
\pgfpathrectangle{\pgfqpoint{0.100000in}{2.413063in}}{\pgfqpoint{5.037500in}{3.427208in}}%
\pgfusepath{clip}%
\pgfsetbuttcap%
\pgfsetroundjoin%
\definecolor{currentfill}{rgb}{0.000000,0.000000,1.000000}%
\pgfsetfillcolor{currentfill}%
\pgfsetfillopacity{0.500000}%
\pgfsetlinewidth{0.250937pt}%
\definecolor{currentstroke}{rgb}{0.000000,0.000000,0.000000}%
\pgfsetstrokecolor{currentstroke}%
\pgfsetstrokeopacity{0.500000}%
\pgfsetdash{}{0pt}%
\pgfsys@defobject{currentmarker}{\pgfqpoint{-0.013889in}{-0.013889in}}{\pgfqpoint{0.013889in}{0.013889in}}{%
\pgfpathmoveto{\pgfqpoint{0.000000in}{-0.013889in}}%
\pgfpathcurveto{\pgfqpoint{0.003683in}{-0.013889in}}{\pgfqpoint{0.007216in}{-0.012425in}}{\pgfqpoint{0.009821in}{-0.009821in}}%
\pgfpathcurveto{\pgfqpoint{0.012425in}{-0.007216in}}{\pgfqpoint{0.013889in}{-0.003683in}}{\pgfqpoint{0.013889in}{0.000000in}}%
\pgfpathcurveto{\pgfqpoint{0.013889in}{0.003683in}}{\pgfqpoint{0.012425in}{0.007216in}}{\pgfqpoint{0.009821in}{0.009821in}}%
\pgfpathcurveto{\pgfqpoint{0.007216in}{0.012425in}}{\pgfqpoint{0.003683in}{0.013889in}}{\pgfqpoint{0.000000in}{0.013889in}}%
\pgfpathcurveto{\pgfqpoint{-0.003683in}{0.013889in}}{\pgfqpoint{-0.007216in}{0.012425in}}{\pgfqpoint{-0.009821in}{0.009821in}}%
\pgfpathcurveto{\pgfqpoint{-0.012425in}{0.007216in}}{\pgfqpoint{-0.013889in}{0.003683in}}{\pgfqpoint{-0.013889in}{0.000000in}}%
\pgfpathcurveto{\pgfqpoint{-0.013889in}{-0.003683in}}{\pgfqpoint{-0.012425in}{-0.007216in}}{\pgfqpoint{-0.009821in}{-0.009821in}}%
\pgfpathcurveto{\pgfqpoint{-0.007216in}{-0.012425in}}{\pgfqpoint{-0.003683in}{-0.013889in}}{\pgfqpoint{0.000000in}{-0.013889in}}%
\pgfpathclose%
\pgfusepath{stroke,fill}%
}%
\begin{pgfscope}%
\pgfsys@transformshift{3.684003in}{3.713619in}%
\pgfsys@useobject{currentmarker}{}%
\end{pgfscope}%
\end{pgfscope}%
\begin{pgfscope}%
\pgfpathrectangle{\pgfqpoint{0.100000in}{2.413063in}}{\pgfqpoint{5.037500in}{3.427208in}}%
\pgfusepath{clip}%
\pgfsetrectcap%
\pgfsetroundjoin%
\pgfsetlinewidth{1.505625pt}%
\definecolor{currentstroke}{rgb}{0.501961,0.501961,0.501961}%
\pgfsetstrokecolor{currentstroke}%
\pgfsetstrokeopacity{0.500000}%
\pgfsetdash{}{0pt}%
\pgfpathmoveto{\pgfqpoint{3.728741in}{3.597179in}}%
\pgfusepath{stroke}%
\end{pgfscope}%
\begin{pgfscope}%
\pgfpathrectangle{\pgfqpoint{0.100000in}{2.413063in}}{\pgfqpoint{5.037500in}{3.427208in}}%
\pgfusepath{clip}%
\pgfsetbuttcap%
\pgfsetroundjoin%
\definecolor{currentfill}{rgb}{0.501961,0.501961,0.501961}%
\pgfsetfillcolor{currentfill}%
\pgfsetfillopacity{0.500000}%
\pgfsetlinewidth{0.250937pt}%
\definecolor{currentstroke}{rgb}{0.000000,0.000000,0.000000}%
\pgfsetstrokecolor{currentstroke}%
\pgfsetstrokeopacity{0.500000}%
\pgfsetdash{}{0pt}%
\pgfsys@defobject{currentmarker}{\pgfqpoint{-0.013889in}{-0.013889in}}{\pgfqpoint{0.013889in}{0.013889in}}{%
\pgfpathmoveto{\pgfqpoint{0.000000in}{-0.013889in}}%
\pgfpathcurveto{\pgfqpoint{0.003683in}{-0.013889in}}{\pgfqpoint{0.007216in}{-0.012425in}}{\pgfqpoint{0.009821in}{-0.009821in}}%
\pgfpathcurveto{\pgfqpoint{0.012425in}{-0.007216in}}{\pgfqpoint{0.013889in}{-0.003683in}}{\pgfqpoint{0.013889in}{0.000000in}}%
\pgfpathcurveto{\pgfqpoint{0.013889in}{0.003683in}}{\pgfqpoint{0.012425in}{0.007216in}}{\pgfqpoint{0.009821in}{0.009821in}}%
\pgfpathcurveto{\pgfqpoint{0.007216in}{0.012425in}}{\pgfqpoint{0.003683in}{0.013889in}}{\pgfqpoint{0.000000in}{0.013889in}}%
\pgfpathcurveto{\pgfqpoint{-0.003683in}{0.013889in}}{\pgfqpoint{-0.007216in}{0.012425in}}{\pgfqpoint{-0.009821in}{0.009821in}}%
\pgfpathcurveto{\pgfqpoint{-0.012425in}{0.007216in}}{\pgfqpoint{-0.013889in}{0.003683in}}{\pgfqpoint{-0.013889in}{0.000000in}}%
\pgfpathcurveto{\pgfqpoint{-0.013889in}{-0.003683in}}{\pgfqpoint{-0.012425in}{-0.007216in}}{\pgfqpoint{-0.009821in}{-0.009821in}}%
\pgfpathcurveto{\pgfqpoint{-0.007216in}{-0.012425in}}{\pgfqpoint{-0.003683in}{-0.013889in}}{\pgfqpoint{0.000000in}{-0.013889in}}%
\pgfpathclose%
\pgfusepath{stroke,fill}%
}%
\begin{pgfscope}%
\pgfsys@transformshift{3.728741in}{3.597179in}%
\pgfsys@useobject{currentmarker}{}%
\end{pgfscope}%
\end{pgfscope}%
\begin{pgfscope}%
\pgfpathrectangle{\pgfqpoint{0.100000in}{2.413063in}}{\pgfqpoint{5.037500in}{3.427208in}}%
\pgfusepath{clip}%
\pgfsetrectcap%
\pgfsetroundjoin%
\pgfsetlinewidth{1.505625pt}%
\definecolor{currentstroke}{rgb}{0.000000,0.000000,1.000000}%
\pgfsetstrokecolor{currentstroke}%
\pgfsetstrokeopacity{0.500000}%
\pgfsetdash{}{0pt}%
\pgfpathmoveto{\pgfqpoint{3.593628in}{3.684664in}}%
\pgfusepath{stroke}%
\end{pgfscope}%
\begin{pgfscope}%
\pgfpathrectangle{\pgfqpoint{0.100000in}{2.413063in}}{\pgfqpoint{5.037500in}{3.427208in}}%
\pgfusepath{clip}%
\pgfsetbuttcap%
\pgfsetroundjoin%
\definecolor{currentfill}{rgb}{0.000000,0.000000,1.000000}%
\pgfsetfillcolor{currentfill}%
\pgfsetfillopacity{0.500000}%
\pgfsetlinewidth{0.250937pt}%
\definecolor{currentstroke}{rgb}{0.000000,0.000000,0.000000}%
\pgfsetstrokecolor{currentstroke}%
\pgfsetstrokeopacity{0.500000}%
\pgfsetdash{}{0pt}%
\pgfsys@defobject{currentmarker}{\pgfqpoint{-0.011111in}{-0.011111in}}{\pgfqpoint{0.011111in}{0.011111in}}{%
\pgfpathmoveto{\pgfqpoint{0.000000in}{-0.011111in}}%
\pgfpathcurveto{\pgfqpoint{0.002947in}{-0.011111in}}{\pgfqpoint{0.005773in}{-0.009940in}}{\pgfqpoint{0.007857in}{-0.007857in}}%
\pgfpathcurveto{\pgfqpoint{0.009940in}{-0.005773in}}{\pgfqpoint{0.011111in}{-0.002947in}}{\pgfqpoint{0.011111in}{0.000000in}}%
\pgfpathcurveto{\pgfqpoint{0.011111in}{0.002947in}}{\pgfqpoint{0.009940in}{0.005773in}}{\pgfqpoint{0.007857in}{0.007857in}}%
\pgfpathcurveto{\pgfqpoint{0.005773in}{0.009940in}}{\pgfqpoint{0.002947in}{0.011111in}}{\pgfqpoint{0.000000in}{0.011111in}}%
\pgfpathcurveto{\pgfqpoint{-0.002947in}{0.011111in}}{\pgfqpoint{-0.005773in}{0.009940in}}{\pgfqpoint{-0.007857in}{0.007857in}}%
\pgfpathcurveto{\pgfqpoint{-0.009940in}{0.005773in}}{\pgfqpoint{-0.011111in}{0.002947in}}{\pgfqpoint{-0.011111in}{0.000000in}}%
\pgfpathcurveto{\pgfqpoint{-0.011111in}{-0.002947in}}{\pgfqpoint{-0.009940in}{-0.005773in}}{\pgfqpoint{-0.007857in}{-0.007857in}}%
\pgfpathcurveto{\pgfqpoint{-0.005773in}{-0.009940in}}{\pgfqpoint{-0.002947in}{-0.011111in}}{\pgfqpoint{0.000000in}{-0.011111in}}%
\pgfpathclose%
\pgfusepath{stroke,fill}%
}%
\begin{pgfscope}%
\pgfsys@transformshift{3.593628in}{3.684664in}%
\pgfsys@useobject{currentmarker}{}%
\end{pgfscope}%
\end{pgfscope}%
\begin{pgfscope}%
\pgfpathrectangle{\pgfqpoint{0.100000in}{2.413063in}}{\pgfqpoint{5.037500in}{3.427208in}}%
\pgfusepath{clip}%
\pgfsetrectcap%
\pgfsetroundjoin%
\pgfsetlinewidth{1.505625pt}%
\definecolor{currentstroke}{rgb}{0.501961,0.501961,0.501961}%
\pgfsetstrokecolor{currentstroke}%
\pgfsetstrokeopacity{0.500000}%
\pgfsetdash{}{0pt}%
\pgfpathmoveto{\pgfqpoint{3.519290in}{3.335074in}}%
\pgfusepath{stroke}%
\end{pgfscope}%
\begin{pgfscope}%
\pgfpathrectangle{\pgfqpoint{0.100000in}{2.413063in}}{\pgfqpoint{5.037500in}{3.427208in}}%
\pgfusepath{clip}%
\pgfsetbuttcap%
\pgfsetroundjoin%
\definecolor{currentfill}{rgb}{0.501961,0.501961,0.501961}%
\pgfsetfillcolor{currentfill}%
\pgfsetfillopacity{0.500000}%
\pgfsetlinewidth{0.250937pt}%
\definecolor{currentstroke}{rgb}{0.000000,0.000000,0.000000}%
\pgfsetstrokecolor{currentstroke}%
\pgfsetstrokeopacity{0.500000}%
\pgfsetdash{}{0pt}%
\pgfsys@defobject{currentmarker}{\pgfqpoint{-0.013889in}{-0.013889in}}{\pgfqpoint{0.013889in}{0.013889in}}{%
\pgfpathmoveto{\pgfqpoint{0.000000in}{-0.013889in}}%
\pgfpathcurveto{\pgfqpoint{0.003683in}{-0.013889in}}{\pgfqpoint{0.007216in}{-0.012425in}}{\pgfqpoint{0.009821in}{-0.009821in}}%
\pgfpathcurveto{\pgfqpoint{0.012425in}{-0.007216in}}{\pgfqpoint{0.013889in}{-0.003683in}}{\pgfqpoint{0.013889in}{0.000000in}}%
\pgfpathcurveto{\pgfqpoint{0.013889in}{0.003683in}}{\pgfqpoint{0.012425in}{0.007216in}}{\pgfqpoint{0.009821in}{0.009821in}}%
\pgfpathcurveto{\pgfqpoint{0.007216in}{0.012425in}}{\pgfqpoint{0.003683in}{0.013889in}}{\pgfqpoint{0.000000in}{0.013889in}}%
\pgfpathcurveto{\pgfqpoint{-0.003683in}{0.013889in}}{\pgfqpoint{-0.007216in}{0.012425in}}{\pgfqpoint{-0.009821in}{0.009821in}}%
\pgfpathcurveto{\pgfqpoint{-0.012425in}{0.007216in}}{\pgfqpoint{-0.013889in}{0.003683in}}{\pgfqpoint{-0.013889in}{0.000000in}}%
\pgfpathcurveto{\pgfqpoint{-0.013889in}{-0.003683in}}{\pgfqpoint{-0.012425in}{-0.007216in}}{\pgfqpoint{-0.009821in}{-0.009821in}}%
\pgfpathcurveto{\pgfqpoint{-0.007216in}{-0.012425in}}{\pgfqpoint{-0.003683in}{-0.013889in}}{\pgfqpoint{0.000000in}{-0.013889in}}%
\pgfpathclose%
\pgfusepath{stroke,fill}%
}%
\begin{pgfscope}%
\pgfsys@transformshift{3.519290in}{3.335074in}%
\pgfsys@useobject{currentmarker}{}%
\end{pgfscope}%
\end{pgfscope}%
\begin{pgfscope}%
\pgfpathrectangle{\pgfqpoint{0.100000in}{2.413063in}}{\pgfqpoint{5.037500in}{3.427208in}}%
\pgfusepath{clip}%
\pgfsetrectcap%
\pgfsetroundjoin%
\pgfsetlinewidth{1.505625pt}%
\definecolor{currentstroke}{rgb}{0.501961,0.501961,0.501961}%
\pgfsetstrokecolor{currentstroke}%
\pgfsetstrokeopacity{0.500000}%
\pgfsetdash{}{0pt}%
\pgfpathmoveto{\pgfqpoint{3.557956in}{3.809517in}}%
\pgfusepath{stroke}%
\end{pgfscope}%
\begin{pgfscope}%
\pgfpathrectangle{\pgfqpoint{0.100000in}{2.413063in}}{\pgfqpoint{5.037500in}{3.427208in}}%
\pgfusepath{clip}%
\pgfsetbuttcap%
\pgfsetroundjoin%
\definecolor{currentfill}{rgb}{0.501961,0.501961,0.501961}%
\pgfsetfillcolor{currentfill}%
\pgfsetfillopacity{0.500000}%
\pgfsetlinewidth{0.250937pt}%
\definecolor{currentstroke}{rgb}{0.000000,0.000000,0.000000}%
\pgfsetstrokecolor{currentstroke}%
\pgfsetstrokeopacity{0.500000}%
\pgfsetdash{}{0pt}%
\pgfsys@defobject{currentmarker}{\pgfqpoint{-0.013889in}{-0.013889in}}{\pgfqpoint{0.013889in}{0.013889in}}{%
\pgfpathmoveto{\pgfqpoint{0.000000in}{-0.013889in}}%
\pgfpathcurveto{\pgfqpoint{0.003683in}{-0.013889in}}{\pgfqpoint{0.007216in}{-0.012425in}}{\pgfqpoint{0.009821in}{-0.009821in}}%
\pgfpathcurveto{\pgfqpoint{0.012425in}{-0.007216in}}{\pgfqpoint{0.013889in}{-0.003683in}}{\pgfqpoint{0.013889in}{0.000000in}}%
\pgfpathcurveto{\pgfqpoint{0.013889in}{0.003683in}}{\pgfqpoint{0.012425in}{0.007216in}}{\pgfqpoint{0.009821in}{0.009821in}}%
\pgfpathcurveto{\pgfqpoint{0.007216in}{0.012425in}}{\pgfqpoint{0.003683in}{0.013889in}}{\pgfqpoint{0.000000in}{0.013889in}}%
\pgfpathcurveto{\pgfqpoint{-0.003683in}{0.013889in}}{\pgfqpoint{-0.007216in}{0.012425in}}{\pgfqpoint{-0.009821in}{0.009821in}}%
\pgfpathcurveto{\pgfqpoint{-0.012425in}{0.007216in}}{\pgfqpoint{-0.013889in}{0.003683in}}{\pgfqpoint{-0.013889in}{0.000000in}}%
\pgfpathcurveto{\pgfqpoint{-0.013889in}{-0.003683in}}{\pgfqpoint{-0.012425in}{-0.007216in}}{\pgfqpoint{-0.009821in}{-0.009821in}}%
\pgfpathcurveto{\pgfqpoint{-0.007216in}{-0.012425in}}{\pgfqpoint{-0.003683in}{-0.013889in}}{\pgfqpoint{0.000000in}{-0.013889in}}%
\pgfpathclose%
\pgfusepath{stroke,fill}%
}%
\begin{pgfscope}%
\pgfsys@transformshift{3.557956in}{3.809517in}%
\pgfsys@useobject{currentmarker}{}%
\end{pgfscope}%
\end{pgfscope}%
\begin{pgfscope}%
\pgfpathrectangle{\pgfqpoint{0.100000in}{2.413063in}}{\pgfqpoint{5.037500in}{3.427208in}}%
\pgfusepath{clip}%
\pgfsetrectcap%
\pgfsetroundjoin%
\pgfsetlinewidth{1.505625pt}%
\definecolor{currentstroke}{rgb}{0.501961,0.501961,0.501961}%
\pgfsetstrokecolor{currentstroke}%
\pgfsetstrokeopacity{0.500000}%
\pgfsetdash{}{0pt}%
\pgfpathmoveto{\pgfqpoint{3.757252in}{3.433137in}}%
\pgfusepath{stroke}%
\end{pgfscope}%
\begin{pgfscope}%
\pgfpathrectangle{\pgfqpoint{0.100000in}{2.413063in}}{\pgfqpoint{5.037500in}{3.427208in}}%
\pgfusepath{clip}%
\pgfsetbuttcap%
\pgfsetroundjoin%
\definecolor{currentfill}{rgb}{0.501961,0.501961,0.501961}%
\pgfsetfillcolor{currentfill}%
\pgfsetfillopacity{0.500000}%
\pgfsetlinewidth{0.250937pt}%
\definecolor{currentstroke}{rgb}{0.000000,0.000000,0.000000}%
\pgfsetstrokecolor{currentstroke}%
\pgfsetstrokeopacity{0.500000}%
\pgfsetdash{}{0pt}%
\pgfsys@defobject{currentmarker}{\pgfqpoint{-0.013889in}{-0.013889in}}{\pgfqpoint{0.013889in}{0.013889in}}{%
\pgfpathmoveto{\pgfqpoint{0.000000in}{-0.013889in}}%
\pgfpathcurveto{\pgfqpoint{0.003683in}{-0.013889in}}{\pgfqpoint{0.007216in}{-0.012425in}}{\pgfqpoint{0.009821in}{-0.009821in}}%
\pgfpathcurveto{\pgfqpoint{0.012425in}{-0.007216in}}{\pgfqpoint{0.013889in}{-0.003683in}}{\pgfqpoint{0.013889in}{0.000000in}}%
\pgfpathcurveto{\pgfqpoint{0.013889in}{0.003683in}}{\pgfqpoint{0.012425in}{0.007216in}}{\pgfqpoint{0.009821in}{0.009821in}}%
\pgfpathcurveto{\pgfqpoint{0.007216in}{0.012425in}}{\pgfqpoint{0.003683in}{0.013889in}}{\pgfqpoint{0.000000in}{0.013889in}}%
\pgfpathcurveto{\pgfqpoint{-0.003683in}{0.013889in}}{\pgfqpoint{-0.007216in}{0.012425in}}{\pgfqpoint{-0.009821in}{0.009821in}}%
\pgfpathcurveto{\pgfqpoint{-0.012425in}{0.007216in}}{\pgfqpoint{-0.013889in}{0.003683in}}{\pgfqpoint{-0.013889in}{0.000000in}}%
\pgfpathcurveto{\pgfqpoint{-0.013889in}{-0.003683in}}{\pgfqpoint{-0.012425in}{-0.007216in}}{\pgfqpoint{-0.009821in}{-0.009821in}}%
\pgfpathcurveto{\pgfqpoint{-0.007216in}{-0.012425in}}{\pgfqpoint{-0.003683in}{-0.013889in}}{\pgfqpoint{0.000000in}{-0.013889in}}%
\pgfpathclose%
\pgfusepath{stroke,fill}%
}%
\begin{pgfscope}%
\pgfsys@transformshift{3.757252in}{3.433137in}%
\pgfsys@useobject{currentmarker}{}%
\end{pgfscope}%
\end{pgfscope}%
\begin{pgfscope}%
\pgfpathrectangle{\pgfqpoint{0.100000in}{2.413063in}}{\pgfqpoint{5.037500in}{3.427208in}}%
\pgfusepath{clip}%
\pgfsetrectcap%
\pgfsetroundjoin%
\pgfsetlinewidth{1.505625pt}%
\definecolor{currentstroke}{rgb}{0.501961,0.501961,0.501961}%
\pgfsetstrokecolor{currentstroke}%
\pgfsetstrokeopacity{0.500000}%
\pgfsetdash{}{0pt}%
\pgfpathmoveto{\pgfqpoint{3.497670in}{3.824813in}}%
\pgfusepath{stroke}%
\end{pgfscope}%
\begin{pgfscope}%
\pgfpathrectangle{\pgfqpoint{0.100000in}{2.413063in}}{\pgfqpoint{5.037500in}{3.427208in}}%
\pgfusepath{clip}%
\pgfsetbuttcap%
\pgfsetroundjoin%
\definecolor{currentfill}{rgb}{0.501961,0.501961,0.501961}%
\pgfsetfillcolor{currentfill}%
\pgfsetfillopacity{0.500000}%
\pgfsetlinewidth{0.250937pt}%
\definecolor{currentstroke}{rgb}{0.000000,0.000000,0.000000}%
\pgfsetstrokecolor{currentstroke}%
\pgfsetstrokeopacity{0.500000}%
\pgfsetdash{}{0pt}%
\pgfsys@defobject{currentmarker}{\pgfqpoint{-0.013889in}{-0.013889in}}{\pgfqpoint{0.013889in}{0.013889in}}{%
\pgfpathmoveto{\pgfqpoint{0.000000in}{-0.013889in}}%
\pgfpathcurveto{\pgfqpoint{0.003683in}{-0.013889in}}{\pgfqpoint{0.007216in}{-0.012425in}}{\pgfqpoint{0.009821in}{-0.009821in}}%
\pgfpathcurveto{\pgfqpoint{0.012425in}{-0.007216in}}{\pgfqpoint{0.013889in}{-0.003683in}}{\pgfqpoint{0.013889in}{0.000000in}}%
\pgfpathcurveto{\pgfqpoint{0.013889in}{0.003683in}}{\pgfqpoint{0.012425in}{0.007216in}}{\pgfqpoint{0.009821in}{0.009821in}}%
\pgfpathcurveto{\pgfqpoint{0.007216in}{0.012425in}}{\pgfqpoint{0.003683in}{0.013889in}}{\pgfqpoint{0.000000in}{0.013889in}}%
\pgfpathcurveto{\pgfqpoint{-0.003683in}{0.013889in}}{\pgfqpoint{-0.007216in}{0.012425in}}{\pgfqpoint{-0.009821in}{0.009821in}}%
\pgfpathcurveto{\pgfqpoint{-0.012425in}{0.007216in}}{\pgfqpoint{-0.013889in}{0.003683in}}{\pgfqpoint{-0.013889in}{0.000000in}}%
\pgfpathcurveto{\pgfqpoint{-0.013889in}{-0.003683in}}{\pgfqpoint{-0.012425in}{-0.007216in}}{\pgfqpoint{-0.009821in}{-0.009821in}}%
\pgfpathcurveto{\pgfqpoint{-0.007216in}{-0.012425in}}{\pgfqpoint{-0.003683in}{-0.013889in}}{\pgfqpoint{0.000000in}{-0.013889in}}%
\pgfpathclose%
\pgfusepath{stroke,fill}%
}%
\begin{pgfscope}%
\pgfsys@transformshift{3.497670in}{3.824813in}%
\pgfsys@useobject{currentmarker}{}%
\end{pgfscope}%
\end{pgfscope}%
\begin{pgfscope}%
\pgfpathrectangle{\pgfqpoint{0.100000in}{2.413063in}}{\pgfqpoint{5.037500in}{3.427208in}}%
\pgfusepath{clip}%
\pgfsetrectcap%
\pgfsetroundjoin%
\pgfsetlinewidth{1.505625pt}%
\definecolor{currentstroke}{rgb}{0.000000,0.000000,1.000000}%
\pgfsetstrokecolor{currentstroke}%
\pgfsetstrokeopacity{0.500000}%
\pgfsetdash{}{0pt}%
\pgfpathmoveto{\pgfqpoint{3.664388in}{3.748720in}}%
\pgfusepath{stroke}%
\end{pgfscope}%
\begin{pgfscope}%
\pgfpathrectangle{\pgfqpoint{0.100000in}{2.413063in}}{\pgfqpoint{5.037500in}{3.427208in}}%
\pgfusepath{clip}%
\pgfsetbuttcap%
\pgfsetroundjoin%
\definecolor{currentfill}{rgb}{0.000000,0.000000,1.000000}%
\pgfsetfillcolor{currentfill}%
\pgfsetfillopacity{0.500000}%
\pgfsetlinewidth{0.250937pt}%
\definecolor{currentstroke}{rgb}{0.000000,0.000000,0.000000}%
\pgfsetstrokecolor{currentstroke}%
\pgfsetstrokeopacity{0.500000}%
\pgfsetdash{}{0pt}%
\pgfsys@defobject{currentmarker}{\pgfqpoint{-0.013889in}{-0.013889in}}{\pgfqpoint{0.013889in}{0.013889in}}{%
\pgfpathmoveto{\pgfqpoint{0.000000in}{-0.013889in}}%
\pgfpathcurveto{\pgfqpoint{0.003683in}{-0.013889in}}{\pgfqpoint{0.007216in}{-0.012425in}}{\pgfqpoint{0.009821in}{-0.009821in}}%
\pgfpathcurveto{\pgfqpoint{0.012425in}{-0.007216in}}{\pgfqpoint{0.013889in}{-0.003683in}}{\pgfqpoint{0.013889in}{0.000000in}}%
\pgfpathcurveto{\pgfqpoint{0.013889in}{0.003683in}}{\pgfqpoint{0.012425in}{0.007216in}}{\pgfqpoint{0.009821in}{0.009821in}}%
\pgfpathcurveto{\pgfqpoint{0.007216in}{0.012425in}}{\pgfqpoint{0.003683in}{0.013889in}}{\pgfqpoint{0.000000in}{0.013889in}}%
\pgfpathcurveto{\pgfqpoint{-0.003683in}{0.013889in}}{\pgfqpoint{-0.007216in}{0.012425in}}{\pgfqpoint{-0.009821in}{0.009821in}}%
\pgfpathcurveto{\pgfqpoint{-0.012425in}{0.007216in}}{\pgfqpoint{-0.013889in}{0.003683in}}{\pgfqpoint{-0.013889in}{0.000000in}}%
\pgfpathcurveto{\pgfqpoint{-0.013889in}{-0.003683in}}{\pgfqpoint{-0.012425in}{-0.007216in}}{\pgfqpoint{-0.009821in}{-0.009821in}}%
\pgfpathcurveto{\pgfqpoint{-0.007216in}{-0.012425in}}{\pgfqpoint{-0.003683in}{-0.013889in}}{\pgfqpoint{0.000000in}{-0.013889in}}%
\pgfpathclose%
\pgfusepath{stroke,fill}%
}%
\begin{pgfscope}%
\pgfsys@transformshift{3.664388in}{3.748720in}%
\pgfsys@useobject{currentmarker}{}%
\end{pgfscope}%
\end{pgfscope}%
\begin{pgfscope}%
\pgfpathrectangle{\pgfqpoint{0.100000in}{2.413063in}}{\pgfqpoint{5.037500in}{3.427208in}}%
\pgfusepath{clip}%
\pgfsetrectcap%
\pgfsetroundjoin%
\pgfsetlinewidth{1.505625pt}%
\definecolor{currentstroke}{rgb}{0.501961,0.501961,0.501961}%
\pgfsetstrokecolor{currentstroke}%
\pgfsetstrokeopacity{0.500000}%
\pgfsetdash{}{0pt}%
\pgfpathmoveto{\pgfqpoint{3.601503in}{3.825717in}}%
\pgfusepath{stroke}%
\end{pgfscope}%
\begin{pgfscope}%
\pgfpathrectangle{\pgfqpoint{0.100000in}{2.413063in}}{\pgfqpoint{5.037500in}{3.427208in}}%
\pgfusepath{clip}%
\pgfsetbuttcap%
\pgfsetroundjoin%
\definecolor{currentfill}{rgb}{0.501961,0.501961,0.501961}%
\pgfsetfillcolor{currentfill}%
\pgfsetfillopacity{0.500000}%
\pgfsetlinewidth{0.250937pt}%
\definecolor{currentstroke}{rgb}{0.000000,0.000000,0.000000}%
\pgfsetstrokecolor{currentstroke}%
\pgfsetstrokeopacity{0.500000}%
\pgfsetdash{}{0pt}%
\pgfsys@defobject{currentmarker}{\pgfqpoint{-0.013889in}{-0.013889in}}{\pgfqpoint{0.013889in}{0.013889in}}{%
\pgfpathmoveto{\pgfqpoint{0.000000in}{-0.013889in}}%
\pgfpathcurveto{\pgfqpoint{0.003683in}{-0.013889in}}{\pgfqpoint{0.007216in}{-0.012425in}}{\pgfqpoint{0.009821in}{-0.009821in}}%
\pgfpathcurveto{\pgfqpoint{0.012425in}{-0.007216in}}{\pgfqpoint{0.013889in}{-0.003683in}}{\pgfqpoint{0.013889in}{0.000000in}}%
\pgfpathcurveto{\pgfqpoint{0.013889in}{0.003683in}}{\pgfqpoint{0.012425in}{0.007216in}}{\pgfqpoint{0.009821in}{0.009821in}}%
\pgfpathcurveto{\pgfqpoint{0.007216in}{0.012425in}}{\pgfqpoint{0.003683in}{0.013889in}}{\pgfqpoint{0.000000in}{0.013889in}}%
\pgfpathcurveto{\pgfqpoint{-0.003683in}{0.013889in}}{\pgfqpoint{-0.007216in}{0.012425in}}{\pgfqpoint{-0.009821in}{0.009821in}}%
\pgfpathcurveto{\pgfqpoint{-0.012425in}{0.007216in}}{\pgfqpoint{-0.013889in}{0.003683in}}{\pgfqpoint{-0.013889in}{0.000000in}}%
\pgfpathcurveto{\pgfqpoint{-0.013889in}{-0.003683in}}{\pgfqpoint{-0.012425in}{-0.007216in}}{\pgfqpoint{-0.009821in}{-0.009821in}}%
\pgfpathcurveto{\pgfqpoint{-0.007216in}{-0.012425in}}{\pgfqpoint{-0.003683in}{-0.013889in}}{\pgfqpoint{0.000000in}{-0.013889in}}%
\pgfpathclose%
\pgfusepath{stroke,fill}%
}%
\begin{pgfscope}%
\pgfsys@transformshift{3.601503in}{3.825717in}%
\pgfsys@useobject{currentmarker}{}%
\end{pgfscope}%
\end{pgfscope}%
\begin{pgfscope}%
\pgfpathrectangle{\pgfqpoint{0.100000in}{2.413063in}}{\pgfqpoint{5.037500in}{3.427208in}}%
\pgfusepath{clip}%
\pgfsetrectcap%
\pgfsetroundjoin%
\pgfsetlinewidth{1.505625pt}%
\definecolor{currentstroke}{rgb}{0.000000,0.000000,1.000000}%
\pgfsetstrokecolor{currentstroke}%
\pgfsetstrokeopacity{0.500000}%
\pgfsetdash{}{0pt}%
\pgfpathmoveto{\pgfqpoint{3.499471in}{3.347832in}}%
\pgfusepath{stroke}%
\end{pgfscope}%
\begin{pgfscope}%
\pgfpathrectangle{\pgfqpoint{0.100000in}{2.413063in}}{\pgfqpoint{5.037500in}{3.427208in}}%
\pgfusepath{clip}%
\pgfsetbuttcap%
\pgfsetroundjoin%
\definecolor{currentfill}{rgb}{0.000000,0.000000,1.000000}%
\pgfsetfillcolor{currentfill}%
\pgfsetfillopacity{0.500000}%
\pgfsetlinewidth{0.250937pt}%
\definecolor{currentstroke}{rgb}{0.000000,0.000000,0.000000}%
\pgfsetstrokecolor{currentstroke}%
\pgfsetstrokeopacity{0.500000}%
\pgfsetdash{}{0pt}%
\pgfsys@defobject{currentmarker}{\pgfqpoint{-0.022222in}{-0.022222in}}{\pgfqpoint{0.022222in}{0.022222in}}{%
\pgfpathmoveto{\pgfqpoint{0.000000in}{-0.022222in}}%
\pgfpathcurveto{\pgfqpoint{0.005893in}{-0.022222in}}{\pgfqpoint{0.011546in}{-0.019881in}}{\pgfqpoint{0.015713in}{-0.015713in}}%
\pgfpathcurveto{\pgfqpoint{0.019881in}{-0.011546in}}{\pgfqpoint{0.022222in}{-0.005893in}}{\pgfqpoint{0.022222in}{0.000000in}}%
\pgfpathcurveto{\pgfqpoint{0.022222in}{0.005893in}}{\pgfqpoint{0.019881in}{0.011546in}}{\pgfqpoint{0.015713in}{0.015713in}}%
\pgfpathcurveto{\pgfqpoint{0.011546in}{0.019881in}}{\pgfqpoint{0.005893in}{0.022222in}}{\pgfqpoint{0.000000in}{0.022222in}}%
\pgfpathcurveto{\pgfqpoint{-0.005893in}{0.022222in}}{\pgfqpoint{-0.011546in}{0.019881in}}{\pgfqpoint{-0.015713in}{0.015713in}}%
\pgfpathcurveto{\pgfqpoint{-0.019881in}{0.011546in}}{\pgfqpoint{-0.022222in}{0.005893in}}{\pgfqpoint{-0.022222in}{0.000000in}}%
\pgfpathcurveto{\pgfqpoint{-0.022222in}{-0.005893in}}{\pgfqpoint{-0.019881in}{-0.011546in}}{\pgfqpoint{-0.015713in}{-0.015713in}}%
\pgfpathcurveto{\pgfqpoint{-0.011546in}{-0.019881in}}{\pgfqpoint{-0.005893in}{-0.022222in}}{\pgfqpoint{0.000000in}{-0.022222in}}%
\pgfpathclose%
\pgfusepath{stroke,fill}%
}%
\begin{pgfscope}%
\pgfsys@transformshift{3.499471in}{3.347832in}%
\pgfsys@useobject{currentmarker}{}%
\end{pgfscope}%
\end{pgfscope}%
\begin{pgfscope}%
\pgfpathrectangle{\pgfqpoint{0.100000in}{2.413063in}}{\pgfqpoint{5.037500in}{3.427208in}}%
\pgfusepath{clip}%
\pgfsetrectcap%
\pgfsetroundjoin%
\pgfsetlinewidth{1.505625pt}%
\definecolor{currentstroke}{rgb}{0.000000,0.000000,1.000000}%
\pgfsetstrokecolor{currentstroke}%
\pgfsetstrokeopacity{0.500000}%
\pgfsetdash{}{0pt}%
\pgfpathmoveto{\pgfqpoint{3.654580in}{3.556172in}}%
\pgfusepath{stroke}%
\end{pgfscope}%
\begin{pgfscope}%
\pgfpathrectangle{\pgfqpoint{0.100000in}{2.413063in}}{\pgfqpoint{5.037500in}{3.427208in}}%
\pgfusepath{clip}%
\pgfsetbuttcap%
\pgfsetroundjoin%
\definecolor{currentfill}{rgb}{0.000000,0.000000,1.000000}%
\pgfsetfillcolor{currentfill}%
\pgfsetfillopacity{0.500000}%
\pgfsetlinewidth{0.250937pt}%
\definecolor{currentstroke}{rgb}{0.000000,0.000000,0.000000}%
\pgfsetstrokecolor{currentstroke}%
\pgfsetstrokeopacity{0.500000}%
\pgfsetdash{}{0pt}%
\pgfsys@defobject{currentmarker}{\pgfqpoint{-0.027778in}{-0.027778in}}{\pgfqpoint{0.027778in}{0.027778in}}{%
\pgfpathmoveto{\pgfqpoint{0.000000in}{-0.027778in}}%
\pgfpathcurveto{\pgfqpoint{0.007367in}{-0.027778in}}{\pgfqpoint{0.014433in}{-0.024851in}}{\pgfqpoint{0.019642in}{-0.019642in}}%
\pgfpathcurveto{\pgfqpoint{0.024851in}{-0.014433in}}{\pgfqpoint{0.027778in}{-0.007367in}}{\pgfqpoint{0.027778in}{0.000000in}}%
\pgfpathcurveto{\pgfqpoint{0.027778in}{0.007367in}}{\pgfqpoint{0.024851in}{0.014433in}}{\pgfqpoint{0.019642in}{0.019642in}}%
\pgfpathcurveto{\pgfqpoint{0.014433in}{0.024851in}}{\pgfqpoint{0.007367in}{0.027778in}}{\pgfqpoint{0.000000in}{0.027778in}}%
\pgfpathcurveto{\pgfqpoint{-0.007367in}{0.027778in}}{\pgfqpoint{-0.014433in}{0.024851in}}{\pgfqpoint{-0.019642in}{0.019642in}}%
\pgfpathcurveto{\pgfqpoint{-0.024851in}{0.014433in}}{\pgfqpoint{-0.027778in}{0.007367in}}{\pgfqpoint{-0.027778in}{0.000000in}}%
\pgfpathcurveto{\pgfqpoint{-0.027778in}{-0.007367in}}{\pgfqpoint{-0.024851in}{-0.014433in}}{\pgfqpoint{-0.019642in}{-0.019642in}}%
\pgfpathcurveto{\pgfqpoint{-0.014433in}{-0.024851in}}{\pgfqpoint{-0.007367in}{-0.027778in}}{\pgfqpoint{0.000000in}{-0.027778in}}%
\pgfpathclose%
\pgfusepath{stroke,fill}%
}%
\begin{pgfscope}%
\pgfsys@transformshift{3.654580in}{3.556172in}%
\pgfsys@useobject{currentmarker}{}%
\end{pgfscope}%
\end{pgfscope}%
\begin{pgfscope}%
\pgfpathrectangle{\pgfqpoint{0.100000in}{2.413063in}}{\pgfqpoint{5.037500in}{3.427208in}}%
\pgfusepath{clip}%
\pgfsetrectcap%
\pgfsetroundjoin%
\pgfsetlinewidth{1.505625pt}%
\definecolor{currentstroke}{rgb}{0.000000,0.000000,1.000000}%
\pgfsetstrokecolor{currentstroke}%
\pgfsetstrokeopacity{0.500000}%
\pgfsetdash{}{0pt}%
\pgfpathmoveto{\pgfqpoint{3.530356in}{3.649700in}}%
\pgfusepath{stroke}%
\end{pgfscope}%
\begin{pgfscope}%
\pgfpathrectangle{\pgfqpoint{0.100000in}{2.413063in}}{\pgfqpoint{5.037500in}{3.427208in}}%
\pgfusepath{clip}%
\pgfsetbuttcap%
\pgfsetroundjoin%
\definecolor{currentfill}{rgb}{0.000000,0.000000,1.000000}%
\pgfsetfillcolor{currentfill}%
\pgfsetfillopacity{0.500000}%
\pgfsetlinewidth{0.250937pt}%
\definecolor{currentstroke}{rgb}{0.000000,0.000000,0.000000}%
\pgfsetstrokecolor{currentstroke}%
\pgfsetstrokeopacity{0.500000}%
\pgfsetdash{}{0pt}%
\pgfsys@defobject{currentmarker}{\pgfqpoint{-0.022222in}{-0.022222in}}{\pgfqpoint{0.022222in}{0.022222in}}{%
\pgfpathmoveto{\pgfqpoint{0.000000in}{-0.022222in}}%
\pgfpathcurveto{\pgfqpoint{0.005893in}{-0.022222in}}{\pgfqpoint{0.011546in}{-0.019881in}}{\pgfqpoint{0.015713in}{-0.015713in}}%
\pgfpathcurveto{\pgfqpoint{0.019881in}{-0.011546in}}{\pgfqpoint{0.022222in}{-0.005893in}}{\pgfqpoint{0.022222in}{0.000000in}}%
\pgfpathcurveto{\pgfqpoint{0.022222in}{0.005893in}}{\pgfqpoint{0.019881in}{0.011546in}}{\pgfqpoint{0.015713in}{0.015713in}}%
\pgfpathcurveto{\pgfqpoint{0.011546in}{0.019881in}}{\pgfqpoint{0.005893in}{0.022222in}}{\pgfqpoint{0.000000in}{0.022222in}}%
\pgfpathcurveto{\pgfqpoint{-0.005893in}{0.022222in}}{\pgfqpoint{-0.011546in}{0.019881in}}{\pgfqpoint{-0.015713in}{0.015713in}}%
\pgfpathcurveto{\pgfqpoint{-0.019881in}{0.011546in}}{\pgfqpoint{-0.022222in}{0.005893in}}{\pgfqpoint{-0.022222in}{0.000000in}}%
\pgfpathcurveto{\pgfqpoint{-0.022222in}{-0.005893in}}{\pgfqpoint{-0.019881in}{-0.011546in}}{\pgfqpoint{-0.015713in}{-0.015713in}}%
\pgfpathcurveto{\pgfqpoint{-0.011546in}{-0.019881in}}{\pgfqpoint{-0.005893in}{-0.022222in}}{\pgfqpoint{0.000000in}{-0.022222in}}%
\pgfpathclose%
\pgfusepath{stroke,fill}%
}%
\begin{pgfscope}%
\pgfsys@transformshift{3.530356in}{3.649700in}%
\pgfsys@useobject{currentmarker}{}%
\end{pgfscope}%
\end{pgfscope}%
\begin{pgfscope}%
\pgfpathrectangle{\pgfqpoint{0.100000in}{2.413063in}}{\pgfqpoint{5.037500in}{3.427208in}}%
\pgfusepath{clip}%
\pgfsetrectcap%
\pgfsetroundjoin%
\pgfsetlinewidth{1.505625pt}%
\definecolor{currentstroke}{rgb}{0.000000,0.000000,1.000000}%
\pgfsetstrokecolor{currentstroke}%
\pgfsetstrokeopacity{0.500000}%
\pgfsetdash{}{0pt}%
\pgfpathmoveto{\pgfqpoint{1.206521in}{2.920030in}}%
\pgfusepath{stroke}%
\end{pgfscope}%
\begin{pgfscope}%
\pgfpathrectangle{\pgfqpoint{0.100000in}{2.413063in}}{\pgfqpoint{5.037500in}{3.427208in}}%
\pgfusepath{clip}%
\pgfsetbuttcap%
\pgfsetroundjoin%
\definecolor{currentfill}{rgb}{0.000000,0.000000,1.000000}%
\pgfsetfillcolor{currentfill}%
\pgfsetfillopacity{0.500000}%
\pgfsetlinewidth{0.250937pt}%
\definecolor{currentstroke}{rgb}{0.000000,0.000000,0.000000}%
\pgfsetstrokecolor{currentstroke}%
\pgfsetstrokeopacity{0.500000}%
\pgfsetdash{}{0pt}%
\pgfsys@defobject{currentmarker}{\pgfqpoint{-0.011111in}{-0.011111in}}{\pgfqpoint{0.011111in}{0.011111in}}{%
\pgfpathmoveto{\pgfqpoint{0.000000in}{-0.011111in}}%
\pgfpathcurveto{\pgfqpoint{0.002947in}{-0.011111in}}{\pgfqpoint{0.005773in}{-0.009940in}}{\pgfqpoint{0.007857in}{-0.007857in}}%
\pgfpathcurveto{\pgfqpoint{0.009940in}{-0.005773in}}{\pgfqpoint{0.011111in}{-0.002947in}}{\pgfqpoint{0.011111in}{0.000000in}}%
\pgfpathcurveto{\pgfqpoint{0.011111in}{0.002947in}}{\pgfqpoint{0.009940in}{0.005773in}}{\pgfqpoint{0.007857in}{0.007857in}}%
\pgfpathcurveto{\pgfqpoint{0.005773in}{0.009940in}}{\pgfqpoint{0.002947in}{0.011111in}}{\pgfqpoint{0.000000in}{0.011111in}}%
\pgfpathcurveto{\pgfqpoint{-0.002947in}{0.011111in}}{\pgfqpoint{-0.005773in}{0.009940in}}{\pgfqpoint{-0.007857in}{0.007857in}}%
\pgfpathcurveto{\pgfqpoint{-0.009940in}{0.005773in}}{\pgfqpoint{-0.011111in}{0.002947in}}{\pgfqpoint{-0.011111in}{0.000000in}}%
\pgfpathcurveto{\pgfqpoint{-0.011111in}{-0.002947in}}{\pgfqpoint{-0.009940in}{-0.005773in}}{\pgfqpoint{-0.007857in}{-0.007857in}}%
\pgfpathcurveto{\pgfqpoint{-0.005773in}{-0.009940in}}{\pgfqpoint{-0.002947in}{-0.011111in}}{\pgfqpoint{0.000000in}{-0.011111in}}%
\pgfpathclose%
\pgfusepath{stroke,fill}%
}%
\begin{pgfscope}%
\pgfsys@transformshift{1.206521in}{2.920030in}%
\pgfsys@useobject{currentmarker}{}%
\end{pgfscope}%
\end{pgfscope}%
\begin{pgfscope}%
\pgfpathrectangle{\pgfqpoint{0.100000in}{2.413063in}}{\pgfqpoint{5.037500in}{3.427208in}}%
\pgfusepath{clip}%
\pgfsetrectcap%
\pgfsetroundjoin%
\pgfsetlinewidth{1.505625pt}%
\definecolor{currentstroke}{rgb}{0.000000,0.000000,1.000000}%
\pgfsetstrokecolor{currentstroke}%
\pgfsetstrokeopacity{0.500000}%
\pgfsetdash{}{0pt}%
\pgfpathmoveto{\pgfqpoint{1.330071in}{3.026783in}}%
\pgfusepath{stroke}%
\end{pgfscope}%
\begin{pgfscope}%
\pgfpathrectangle{\pgfqpoint{0.100000in}{2.413063in}}{\pgfqpoint{5.037500in}{3.427208in}}%
\pgfusepath{clip}%
\pgfsetbuttcap%
\pgfsetroundjoin%
\definecolor{currentfill}{rgb}{0.000000,0.000000,1.000000}%
\pgfsetfillcolor{currentfill}%
\pgfsetfillopacity{0.500000}%
\pgfsetlinewidth{0.250937pt}%
\definecolor{currentstroke}{rgb}{0.000000,0.000000,0.000000}%
\pgfsetstrokecolor{currentstroke}%
\pgfsetstrokeopacity{0.500000}%
\pgfsetdash{}{0pt}%
\pgfsys@defobject{currentmarker}{\pgfqpoint{-0.005556in}{-0.005556in}}{\pgfqpoint{0.005556in}{0.005556in}}{%
\pgfpathmoveto{\pgfqpoint{0.000000in}{-0.005556in}}%
\pgfpathcurveto{\pgfqpoint{0.001473in}{-0.005556in}}{\pgfqpoint{0.002887in}{-0.004970in}}{\pgfqpoint{0.003928in}{-0.003928in}}%
\pgfpathcurveto{\pgfqpoint{0.004970in}{-0.002887in}}{\pgfqpoint{0.005556in}{-0.001473in}}{\pgfqpoint{0.005556in}{0.000000in}}%
\pgfpathcurveto{\pgfqpoint{0.005556in}{0.001473in}}{\pgfqpoint{0.004970in}{0.002887in}}{\pgfqpoint{0.003928in}{0.003928in}}%
\pgfpathcurveto{\pgfqpoint{0.002887in}{0.004970in}}{\pgfqpoint{0.001473in}{0.005556in}}{\pgfqpoint{0.000000in}{0.005556in}}%
\pgfpathcurveto{\pgfqpoint{-0.001473in}{0.005556in}}{\pgfqpoint{-0.002887in}{0.004970in}}{\pgfqpoint{-0.003928in}{0.003928in}}%
\pgfpathcurveto{\pgfqpoint{-0.004970in}{0.002887in}}{\pgfqpoint{-0.005556in}{0.001473in}}{\pgfqpoint{-0.005556in}{0.000000in}}%
\pgfpathcurveto{\pgfqpoint{-0.005556in}{-0.001473in}}{\pgfqpoint{-0.004970in}{-0.002887in}}{\pgfqpoint{-0.003928in}{-0.003928in}}%
\pgfpathcurveto{\pgfqpoint{-0.002887in}{-0.004970in}}{\pgfqpoint{-0.001473in}{-0.005556in}}{\pgfqpoint{0.000000in}{-0.005556in}}%
\pgfpathclose%
\pgfusepath{stroke,fill}%
}%
\begin{pgfscope}%
\pgfsys@transformshift{1.330071in}{3.026783in}%
\pgfsys@useobject{currentmarker}{}%
\end{pgfscope}%
\end{pgfscope}%
\begin{pgfscope}%
\pgfpathrectangle{\pgfqpoint{0.100000in}{2.413063in}}{\pgfqpoint{5.037500in}{3.427208in}}%
\pgfusepath{clip}%
\pgfsetrectcap%
\pgfsetroundjoin%
\pgfsetlinewidth{1.505625pt}%
\definecolor{currentstroke}{rgb}{0.678431,1.000000,0.184314}%
\pgfsetstrokecolor{currentstroke}%
\pgfsetstrokeopacity{0.500000}%
\pgfsetdash{}{0pt}%
\pgfpathmoveto{\pgfqpoint{1.242579in}{3.986452in}}%
\pgfusepath{stroke}%
\end{pgfscope}%
\begin{pgfscope}%
\pgfpathrectangle{\pgfqpoint{0.100000in}{2.413063in}}{\pgfqpoint{5.037500in}{3.427208in}}%
\pgfusepath{clip}%
\pgfsetbuttcap%
\pgfsetroundjoin%
\definecolor{currentfill}{rgb}{0.678431,1.000000,0.184314}%
\pgfsetfillcolor{currentfill}%
\pgfsetfillopacity{0.500000}%
\pgfsetlinewidth{0.250937pt}%
\definecolor{currentstroke}{rgb}{0.000000,0.000000,0.000000}%
\pgfsetstrokecolor{currentstroke}%
\pgfsetstrokeopacity{0.500000}%
\pgfsetdash{}{0pt}%
\pgfsys@defobject{currentmarker}{\pgfqpoint{-0.058333in}{-0.058333in}}{\pgfqpoint{0.058333in}{0.058333in}}{%
\pgfpathmoveto{\pgfqpoint{0.000000in}{-0.058333in}}%
\pgfpathcurveto{\pgfqpoint{0.015470in}{-0.058333in}}{\pgfqpoint{0.030309in}{-0.052187in}}{\pgfqpoint{0.041248in}{-0.041248in}}%
\pgfpathcurveto{\pgfqpoint{0.052187in}{-0.030309in}}{\pgfqpoint{0.058333in}{-0.015470in}}{\pgfqpoint{0.058333in}{0.000000in}}%
\pgfpathcurveto{\pgfqpoint{0.058333in}{0.015470in}}{\pgfqpoint{0.052187in}{0.030309in}}{\pgfqpoint{0.041248in}{0.041248in}}%
\pgfpathcurveto{\pgfqpoint{0.030309in}{0.052187in}}{\pgfqpoint{0.015470in}{0.058333in}}{\pgfqpoint{0.000000in}{0.058333in}}%
\pgfpathcurveto{\pgfqpoint{-0.015470in}{0.058333in}}{\pgfqpoint{-0.030309in}{0.052187in}}{\pgfqpoint{-0.041248in}{0.041248in}}%
\pgfpathcurveto{\pgfqpoint{-0.052187in}{0.030309in}}{\pgfqpoint{-0.058333in}{0.015470in}}{\pgfqpoint{-0.058333in}{0.000000in}}%
\pgfpathcurveto{\pgfqpoint{-0.058333in}{-0.015470in}}{\pgfqpoint{-0.052187in}{-0.030309in}}{\pgfqpoint{-0.041248in}{-0.041248in}}%
\pgfpathcurveto{\pgfqpoint{-0.030309in}{-0.052187in}}{\pgfqpoint{-0.015470in}{-0.058333in}}{\pgfqpoint{0.000000in}{-0.058333in}}%
\pgfpathclose%
\pgfusepath{stroke,fill}%
}%
\begin{pgfscope}%
\pgfsys@transformshift{1.242579in}{3.986452in}%
\pgfsys@useobject{currentmarker}{}%
\end{pgfscope}%
\end{pgfscope}%
\begin{pgfscope}%
\pgfpathrectangle{\pgfqpoint{0.100000in}{2.413063in}}{\pgfqpoint{5.037500in}{3.427208in}}%
\pgfusepath{clip}%
\pgfsetrectcap%
\pgfsetroundjoin%
\pgfsetlinewidth{1.505625pt}%
\definecolor{currentstroke}{rgb}{0.678431,1.000000,0.184314}%
\pgfsetstrokecolor{currentstroke}%
\pgfsetstrokeopacity{0.500000}%
\pgfsetdash{}{0pt}%
\pgfpathmoveto{\pgfqpoint{0.975939in}{3.954572in}}%
\pgfusepath{stroke}%
\end{pgfscope}%
\begin{pgfscope}%
\pgfpathrectangle{\pgfqpoint{0.100000in}{2.413063in}}{\pgfqpoint{5.037500in}{3.427208in}}%
\pgfusepath{clip}%
\pgfsetbuttcap%
\pgfsetroundjoin%
\definecolor{currentfill}{rgb}{0.678431,1.000000,0.184314}%
\pgfsetfillcolor{currentfill}%
\pgfsetfillopacity{0.500000}%
\pgfsetlinewidth{0.250937pt}%
\definecolor{currentstroke}{rgb}{0.000000,0.000000,0.000000}%
\pgfsetstrokecolor{currentstroke}%
\pgfsetstrokeopacity{0.500000}%
\pgfsetdash{}{0pt}%
\pgfsys@defobject{currentmarker}{\pgfqpoint{-0.063889in}{-0.063889in}}{\pgfqpoint{0.063889in}{0.063889in}}{%
\pgfpathmoveto{\pgfqpoint{0.000000in}{-0.063889in}}%
\pgfpathcurveto{\pgfqpoint{0.016944in}{-0.063889in}}{\pgfqpoint{0.033195in}{-0.057157in}}{\pgfqpoint{0.045176in}{-0.045176in}}%
\pgfpathcurveto{\pgfqpoint{0.057157in}{-0.033195in}}{\pgfqpoint{0.063889in}{-0.016944in}}{\pgfqpoint{0.063889in}{0.000000in}}%
\pgfpathcurveto{\pgfqpoint{0.063889in}{0.016944in}}{\pgfqpoint{0.057157in}{0.033195in}}{\pgfqpoint{0.045176in}{0.045176in}}%
\pgfpathcurveto{\pgfqpoint{0.033195in}{0.057157in}}{\pgfqpoint{0.016944in}{0.063889in}}{\pgfqpoint{0.000000in}{0.063889in}}%
\pgfpathcurveto{\pgfqpoint{-0.016944in}{0.063889in}}{\pgfqpoint{-0.033195in}{0.057157in}}{\pgfqpoint{-0.045176in}{0.045176in}}%
\pgfpathcurveto{\pgfqpoint{-0.057157in}{0.033195in}}{\pgfqpoint{-0.063889in}{0.016944in}}{\pgfqpoint{-0.063889in}{0.000000in}}%
\pgfpathcurveto{\pgfqpoint{-0.063889in}{-0.016944in}}{\pgfqpoint{-0.057157in}{-0.033195in}}{\pgfqpoint{-0.045176in}{-0.045176in}}%
\pgfpathcurveto{\pgfqpoint{-0.033195in}{-0.057157in}}{\pgfqpoint{-0.016944in}{-0.063889in}}{\pgfqpoint{0.000000in}{-0.063889in}}%
\pgfpathclose%
\pgfusepath{stroke,fill}%
}%
\begin{pgfscope}%
\pgfsys@transformshift{0.975939in}{3.954572in}%
\pgfsys@useobject{currentmarker}{}%
\end{pgfscope}%
\end{pgfscope}%
\begin{pgfscope}%
\pgfpathrectangle{\pgfqpoint{0.100000in}{2.413063in}}{\pgfqpoint{5.037500in}{3.427208in}}%
\pgfusepath{clip}%
\pgfsetrectcap%
\pgfsetroundjoin%
\pgfsetlinewidth{1.505625pt}%
\definecolor{currentstroke}{rgb}{0.678431,1.000000,0.184314}%
\pgfsetstrokecolor{currentstroke}%
\pgfsetstrokeopacity{0.500000}%
\pgfsetdash{}{0pt}%
\pgfpathmoveto{\pgfqpoint{1.165310in}{3.795127in}}%
\pgfusepath{stroke}%
\end{pgfscope}%
\begin{pgfscope}%
\pgfpathrectangle{\pgfqpoint{0.100000in}{2.413063in}}{\pgfqpoint{5.037500in}{3.427208in}}%
\pgfusepath{clip}%
\pgfsetbuttcap%
\pgfsetroundjoin%
\definecolor{currentfill}{rgb}{0.678431,1.000000,0.184314}%
\pgfsetfillcolor{currentfill}%
\pgfsetfillopacity{0.500000}%
\pgfsetlinewidth{0.250937pt}%
\definecolor{currentstroke}{rgb}{0.000000,0.000000,0.000000}%
\pgfsetstrokecolor{currentstroke}%
\pgfsetstrokeopacity{0.500000}%
\pgfsetdash{}{0pt}%
\pgfsys@defobject{currentmarker}{\pgfqpoint{-0.041667in}{-0.041667in}}{\pgfqpoint{0.041667in}{0.041667in}}{%
\pgfpathmoveto{\pgfqpoint{0.000000in}{-0.041667in}}%
\pgfpathcurveto{\pgfqpoint{0.011050in}{-0.041667in}}{\pgfqpoint{0.021649in}{-0.037276in}}{\pgfqpoint{0.029463in}{-0.029463in}}%
\pgfpathcurveto{\pgfqpoint{0.037276in}{-0.021649in}}{\pgfqpoint{0.041667in}{-0.011050in}}{\pgfqpoint{0.041667in}{0.000000in}}%
\pgfpathcurveto{\pgfqpoint{0.041667in}{0.011050in}}{\pgfqpoint{0.037276in}{0.021649in}}{\pgfqpoint{0.029463in}{0.029463in}}%
\pgfpathcurveto{\pgfqpoint{0.021649in}{0.037276in}}{\pgfqpoint{0.011050in}{0.041667in}}{\pgfqpoint{0.000000in}{0.041667in}}%
\pgfpathcurveto{\pgfqpoint{-0.011050in}{0.041667in}}{\pgfqpoint{-0.021649in}{0.037276in}}{\pgfqpoint{-0.029463in}{0.029463in}}%
\pgfpathcurveto{\pgfqpoint{-0.037276in}{0.021649in}}{\pgfqpoint{-0.041667in}{0.011050in}}{\pgfqpoint{-0.041667in}{0.000000in}}%
\pgfpathcurveto{\pgfqpoint{-0.041667in}{-0.011050in}}{\pgfqpoint{-0.037276in}{-0.021649in}}{\pgfqpoint{-0.029463in}{-0.029463in}}%
\pgfpathcurveto{\pgfqpoint{-0.021649in}{-0.037276in}}{\pgfqpoint{-0.011050in}{-0.041667in}}{\pgfqpoint{0.000000in}{-0.041667in}}%
\pgfpathclose%
\pgfusepath{stroke,fill}%
}%
\begin{pgfscope}%
\pgfsys@transformshift{1.165310in}{3.795127in}%
\pgfsys@useobject{currentmarker}{}%
\end{pgfscope}%
\end{pgfscope}%
\begin{pgfscope}%
\pgfpathrectangle{\pgfqpoint{0.100000in}{2.413063in}}{\pgfqpoint{5.037500in}{3.427208in}}%
\pgfusepath{clip}%
\pgfsetrectcap%
\pgfsetroundjoin%
\pgfsetlinewidth{1.505625pt}%
\definecolor{currentstroke}{rgb}{0.678431,1.000000,0.184314}%
\pgfsetstrokecolor{currentstroke}%
\pgfsetstrokeopacity{0.500000}%
\pgfsetdash{}{0pt}%
\pgfpathmoveto{\pgfqpoint{1.152245in}{3.926148in}}%
\pgfusepath{stroke}%
\end{pgfscope}%
\begin{pgfscope}%
\pgfpathrectangle{\pgfqpoint{0.100000in}{2.413063in}}{\pgfqpoint{5.037500in}{3.427208in}}%
\pgfusepath{clip}%
\pgfsetbuttcap%
\pgfsetroundjoin%
\definecolor{currentfill}{rgb}{0.678431,1.000000,0.184314}%
\pgfsetfillcolor{currentfill}%
\pgfsetfillopacity{0.500000}%
\pgfsetlinewidth{0.250937pt}%
\definecolor{currentstroke}{rgb}{0.000000,0.000000,0.000000}%
\pgfsetstrokecolor{currentstroke}%
\pgfsetstrokeopacity{0.500000}%
\pgfsetdash{}{0pt}%
\pgfsys@defobject{currentmarker}{\pgfqpoint{-0.052778in}{-0.052778in}}{\pgfqpoint{0.052778in}{0.052778in}}{%
\pgfpathmoveto{\pgfqpoint{0.000000in}{-0.052778in}}%
\pgfpathcurveto{\pgfqpoint{0.013997in}{-0.052778in}}{\pgfqpoint{0.027422in}{-0.047217in}}{\pgfqpoint{0.037320in}{-0.037320in}}%
\pgfpathcurveto{\pgfqpoint{0.047217in}{-0.027422in}}{\pgfqpoint{0.052778in}{-0.013997in}}{\pgfqpoint{0.052778in}{0.000000in}}%
\pgfpathcurveto{\pgfqpoint{0.052778in}{0.013997in}}{\pgfqpoint{0.047217in}{0.027422in}}{\pgfqpoint{0.037320in}{0.037320in}}%
\pgfpathcurveto{\pgfqpoint{0.027422in}{0.047217in}}{\pgfqpoint{0.013997in}{0.052778in}}{\pgfqpoint{0.000000in}{0.052778in}}%
\pgfpathcurveto{\pgfqpoint{-0.013997in}{0.052778in}}{\pgfqpoint{-0.027422in}{0.047217in}}{\pgfqpoint{-0.037320in}{0.037320in}}%
\pgfpathcurveto{\pgfqpoint{-0.047217in}{0.027422in}}{\pgfqpoint{-0.052778in}{0.013997in}}{\pgfqpoint{-0.052778in}{0.000000in}}%
\pgfpathcurveto{\pgfqpoint{-0.052778in}{-0.013997in}}{\pgfqpoint{-0.047217in}{-0.027422in}}{\pgfqpoint{-0.037320in}{-0.037320in}}%
\pgfpathcurveto{\pgfqpoint{-0.027422in}{-0.047217in}}{\pgfqpoint{-0.013997in}{-0.052778in}}{\pgfqpoint{0.000000in}{-0.052778in}}%
\pgfpathclose%
\pgfusepath{stroke,fill}%
}%
\begin{pgfscope}%
\pgfsys@transformshift{1.152245in}{3.926148in}%
\pgfsys@useobject{currentmarker}{}%
\end{pgfscope}%
\end{pgfscope}%
\begin{pgfscope}%
\pgfpathrectangle{\pgfqpoint{0.100000in}{2.413063in}}{\pgfqpoint{5.037500in}{3.427208in}}%
\pgfusepath{clip}%
\pgfsetrectcap%
\pgfsetroundjoin%
\pgfsetlinewidth{1.505625pt}%
\definecolor{currentstroke}{rgb}{0.678431,1.000000,0.184314}%
\pgfsetstrokecolor{currentstroke}%
\pgfsetstrokeopacity{0.500000}%
\pgfsetdash{}{0pt}%
\pgfpathmoveto{\pgfqpoint{1.297020in}{3.548381in}}%
\pgfusepath{stroke}%
\end{pgfscope}%
\begin{pgfscope}%
\pgfpathrectangle{\pgfqpoint{0.100000in}{2.413063in}}{\pgfqpoint{5.037500in}{3.427208in}}%
\pgfusepath{clip}%
\pgfsetbuttcap%
\pgfsetroundjoin%
\definecolor{currentfill}{rgb}{0.678431,1.000000,0.184314}%
\pgfsetfillcolor{currentfill}%
\pgfsetfillopacity{0.500000}%
\pgfsetlinewidth{0.250937pt}%
\definecolor{currentstroke}{rgb}{0.000000,0.000000,0.000000}%
\pgfsetstrokecolor{currentstroke}%
\pgfsetstrokeopacity{0.500000}%
\pgfsetdash{}{0pt}%
\pgfsys@defobject{currentmarker}{\pgfqpoint{-0.072222in}{-0.072222in}}{\pgfqpoint{0.072222in}{0.072222in}}{%
\pgfpathmoveto{\pgfqpoint{0.000000in}{-0.072222in}}%
\pgfpathcurveto{\pgfqpoint{0.019154in}{-0.072222in}}{\pgfqpoint{0.037525in}{-0.064612in}}{\pgfqpoint{0.051069in}{-0.051069in}}%
\pgfpathcurveto{\pgfqpoint{0.064612in}{-0.037525in}}{\pgfqpoint{0.072222in}{-0.019154in}}{\pgfqpoint{0.072222in}{0.000000in}}%
\pgfpathcurveto{\pgfqpoint{0.072222in}{0.019154in}}{\pgfqpoint{0.064612in}{0.037525in}}{\pgfqpoint{0.051069in}{0.051069in}}%
\pgfpathcurveto{\pgfqpoint{0.037525in}{0.064612in}}{\pgfqpoint{0.019154in}{0.072222in}}{\pgfqpoint{0.000000in}{0.072222in}}%
\pgfpathcurveto{\pgfqpoint{-0.019154in}{0.072222in}}{\pgfqpoint{-0.037525in}{0.064612in}}{\pgfqpoint{-0.051069in}{0.051069in}}%
\pgfpathcurveto{\pgfqpoint{-0.064612in}{0.037525in}}{\pgfqpoint{-0.072222in}{0.019154in}}{\pgfqpoint{-0.072222in}{0.000000in}}%
\pgfpathcurveto{\pgfqpoint{-0.072222in}{-0.019154in}}{\pgfqpoint{-0.064612in}{-0.037525in}}{\pgfqpoint{-0.051069in}{-0.051069in}}%
\pgfpathcurveto{\pgfqpoint{-0.037525in}{-0.064612in}}{\pgfqpoint{-0.019154in}{-0.072222in}}{\pgfqpoint{0.000000in}{-0.072222in}}%
\pgfpathclose%
\pgfusepath{stroke,fill}%
}%
\begin{pgfscope}%
\pgfsys@transformshift{1.297020in}{3.548381in}%
\pgfsys@useobject{currentmarker}{}%
\end{pgfscope}%
\end{pgfscope}%
\begin{pgfscope}%
\pgfpathrectangle{\pgfqpoint{0.100000in}{2.413063in}}{\pgfqpoint{5.037500in}{3.427208in}}%
\pgfusepath{clip}%
\pgfsetrectcap%
\pgfsetroundjoin%
\pgfsetlinewidth{1.505625pt}%
\definecolor{currentstroke}{rgb}{0.678431,1.000000,0.184314}%
\pgfsetstrokecolor{currentstroke}%
\pgfsetstrokeopacity{0.500000}%
\pgfsetdash{}{0pt}%
\pgfpathmoveto{\pgfqpoint{1.249803in}{3.635149in}}%
\pgfusepath{stroke}%
\end{pgfscope}%
\begin{pgfscope}%
\pgfpathrectangle{\pgfqpoint{0.100000in}{2.413063in}}{\pgfqpoint{5.037500in}{3.427208in}}%
\pgfusepath{clip}%
\pgfsetbuttcap%
\pgfsetroundjoin%
\definecolor{currentfill}{rgb}{0.678431,1.000000,0.184314}%
\pgfsetfillcolor{currentfill}%
\pgfsetfillopacity{0.500000}%
\pgfsetlinewidth{0.250937pt}%
\definecolor{currentstroke}{rgb}{0.000000,0.000000,0.000000}%
\pgfsetstrokecolor{currentstroke}%
\pgfsetstrokeopacity{0.500000}%
\pgfsetdash{}{0pt}%
\pgfsys@defobject{currentmarker}{\pgfqpoint{-0.038889in}{-0.038889in}}{\pgfqpoint{0.038889in}{0.038889in}}{%
\pgfpathmoveto{\pgfqpoint{0.000000in}{-0.038889in}}%
\pgfpathcurveto{\pgfqpoint{0.010313in}{-0.038889in}}{\pgfqpoint{0.020206in}{-0.034791in}}{\pgfqpoint{0.027499in}{-0.027499in}}%
\pgfpathcurveto{\pgfqpoint{0.034791in}{-0.020206in}}{\pgfqpoint{0.038889in}{-0.010313in}}{\pgfqpoint{0.038889in}{0.000000in}}%
\pgfpathcurveto{\pgfqpoint{0.038889in}{0.010313in}}{\pgfqpoint{0.034791in}{0.020206in}}{\pgfqpoint{0.027499in}{0.027499in}}%
\pgfpathcurveto{\pgfqpoint{0.020206in}{0.034791in}}{\pgfqpoint{0.010313in}{0.038889in}}{\pgfqpoint{0.000000in}{0.038889in}}%
\pgfpathcurveto{\pgfqpoint{-0.010313in}{0.038889in}}{\pgfqpoint{-0.020206in}{0.034791in}}{\pgfqpoint{-0.027499in}{0.027499in}}%
\pgfpathcurveto{\pgfqpoint{-0.034791in}{0.020206in}}{\pgfqpoint{-0.038889in}{0.010313in}}{\pgfqpoint{-0.038889in}{0.000000in}}%
\pgfpathcurveto{\pgfqpoint{-0.038889in}{-0.010313in}}{\pgfqpoint{-0.034791in}{-0.020206in}}{\pgfqpoint{-0.027499in}{-0.027499in}}%
\pgfpathcurveto{\pgfqpoint{-0.020206in}{-0.034791in}}{\pgfqpoint{-0.010313in}{-0.038889in}}{\pgfqpoint{0.000000in}{-0.038889in}}%
\pgfpathclose%
\pgfusepath{stroke,fill}%
}%
\begin{pgfscope}%
\pgfsys@transformshift{1.249803in}{3.635149in}%
\pgfsys@useobject{currentmarker}{}%
\end{pgfscope}%
\end{pgfscope}%
\begin{pgfscope}%
\pgfpathrectangle{\pgfqpoint{0.100000in}{2.413063in}}{\pgfqpoint{5.037500in}{3.427208in}}%
\pgfusepath{clip}%
\pgfsetrectcap%
\pgfsetroundjoin%
\pgfsetlinewidth{1.505625pt}%
\definecolor{currentstroke}{rgb}{0.678431,1.000000,0.184314}%
\pgfsetstrokecolor{currentstroke}%
\pgfsetstrokeopacity{0.500000}%
\pgfsetdash{}{0pt}%
\pgfpathmoveto{\pgfqpoint{0.919145in}{3.752754in}}%
\pgfusepath{stroke}%
\end{pgfscope}%
\begin{pgfscope}%
\pgfpathrectangle{\pgfqpoint{0.100000in}{2.413063in}}{\pgfqpoint{5.037500in}{3.427208in}}%
\pgfusepath{clip}%
\pgfsetbuttcap%
\pgfsetroundjoin%
\definecolor{currentfill}{rgb}{0.678431,1.000000,0.184314}%
\pgfsetfillcolor{currentfill}%
\pgfsetfillopacity{0.500000}%
\pgfsetlinewidth{0.250937pt}%
\definecolor{currentstroke}{rgb}{0.000000,0.000000,0.000000}%
\pgfsetstrokecolor{currentstroke}%
\pgfsetstrokeopacity{0.500000}%
\pgfsetdash{}{0pt}%
\pgfsys@defobject{currentmarker}{\pgfqpoint{-0.152778in}{-0.152778in}}{\pgfqpoint{0.152778in}{0.152778in}}{%
\pgfpathmoveto{\pgfqpoint{0.000000in}{-0.152778in}}%
\pgfpathcurveto{\pgfqpoint{0.040517in}{-0.152778in}}{\pgfqpoint{0.079380in}{-0.136680in}}{\pgfqpoint{0.108030in}{-0.108030in}}%
\pgfpathcurveto{\pgfqpoint{0.136680in}{-0.079380in}}{\pgfqpoint{0.152778in}{-0.040517in}}{\pgfqpoint{0.152778in}{0.000000in}}%
\pgfpathcurveto{\pgfqpoint{0.152778in}{0.040517in}}{\pgfqpoint{0.136680in}{0.079380in}}{\pgfqpoint{0.108030in}{0.108030in}}%
\pgfpathcurveto{\pgfqpoint{0.079380in}{0.136680in}}{\pgfqpoint{0.040517in}{0.152778in}}{\pgfqpoint{0.000000in}{0.152778in}}%
\pgfpathcurveto{\pgfqpoint{-0.040517in}{0.152778in}}{\pgfqpoint{-0.079380in}{0.136680in}}{\pgfqpoint{-0.108030in}{0.108030in}}%
\pgfpathcurveto{\pgfqpoint{-0.136680in}{0.079380in}}{\pgfqpoint{-0.152778in}{0.040517in}}{\pgfqpoint{-0.152778in}{0.000000in}}%
\pgfpathcurveto{\pgfqpoint{-0.152778in}{-0.040517in}}{\pgfqpoint{-0.136680in}{-0.079380in}}{\pgfqpoint{-0.108030in}{-0.108030in}}%
\pgfpathcurveto{\pgfqpoint{-0.079380in}{-0.136680in}}{\pgfqpoint{-0.040517in}{-0.152778in}}{\pgfqpoint{0.000000in}{-0.152778in}}%
\pgfpathclose%
\pgfusepath{stroke,fill}%
}%
\begin{pgfscope}%
\pgfsys@transformshift{0.919145in}{3.752754in}%
\pgfsys@useobject{currentmarker}{}%
\end{pgfscope}%
\end{pgfscope}%
\begin{pgfscope}%
\pgfpathrectangle{\pgfqpoint{0.100000in}{2.413063in}}{\pgfqpoint{5.037500in}{3.427208in}}%
\pgfusepath{clip}%
\pgfsetrectcap%
\pgfsetroundjoin%
\pgfsetlinewidth{1.505625pt}%
\definecolor{currentstroke}{rgb}{0.678431,1.000000,0.184314}%
\pgfsetstrokecolor{currentstroke}%
\pgfsetstrokeopacity{0.500000}%
\pgfsetdash{}{0pt}%
\pgfpathmoveto{\pgfqpoint{2.882663in}{3.942918in}}%
\pgfusepath{stroke}%
\end{pgfscope}%
\begin{pgfscope}%
\pgfpathrectangle{\pgfqpoint{0.100000in}{2.413063in}}{\pgfqpoint{5.037500in}{3.427208in}}%
\pgfusepath{clip}%
\pgfsetbuttcap%
\pgfsetroundjoin%
\definecolor{currentfill}{rgb}{0.678431,1.000000,0.184314}%
\pgfsetfillcolor{currentfill}%
\pgfsetfillopacity{0.500000}%
\pgfsetlinewidth{0.250937pt}%
\definecolor{currentstroke}{rgb}{0.000000,0.000000,0.000000}%
\pgfsetstrokecolor{currentstroke}%
\pgfsetstrokeopacity{0.500000}%
\pgfsetdash{}{0pt}%
\pgfsys@defobject{currentmarker}{\pgfqpoint{-0.019444in}{-0.019444in}}{\pgfqpoint{0.019444in}{0.019444in}}{%
\pgfpathmoveto{\pgfqpoint{0.000000in}{-0.019444in}}%
\pgfpathcurveto{\pgfqpoint{0.005157in}{-0.019444in}}{\pgfqpoint{0.010103in}{-0.017396in}}{\pgfqpoint{0.013749in}{-0.013749in}}%
\pgfpathcurveto{\pgfqpoint{0.017396in}{-0.010103in}}{\pgfqpoint{0.019444in}{-0.005157in}}{\pgfqpoint{0.019444in}{0.000000in}}%
\pgfpathcurveto{\pgfqpoint{0.019444in}{0.005157in}}{\pgfqpoint{0.017396in}{0.010103in}}{\pgfqpoint{0.013749in}{0.013749in}}%
\pgfpathcurveto{\pgfqpoint{0.010103in}{0.017396in}}{\pgfqpoint{0.005157in}{0.019444in}}{\pgfqpoint{0.000000in}{0.019444in}}%
\pgfpathcurveto{\pgfqpoint{-0.005157in}{0.019444in}}{\pgfqpoint{-0.010103in}{0.017396in}}{\pgfqpoint{-0.013749in}{0.013749in}}%
\pgfpathcurveto{\pgfqpoint{-0.017396in}{0.010103in}}{\pgfqpoint{-0.019444in}{0.005157in}}{\pgfqpoint{-0.019444in}{0.000000in}}%
\pgfpathcurveto{\pgfqpoint{-0.019444in}{-0.005157in}}{\pgfqpoint{-0.017396in}{-0.010103in}}{\pgfqpoint{-0.013749in}{-0.013749in}}%
\pgfpathcurveto{\pgfqpoint{-0.010103in}{-0.017396in}}{\pgfqpoint{-0.005157in}{-0.019444in}}{\pgfqpoint{0.000000in}{-0.019444in}}%
\pgfpathclose%
\pgfusepath{stroke,fill}%
}%
\begin{pgfscope}%
\pgfsys@transformshift{2.882663in}{3.942918in}%
\pgfsys@useobject{currentmarker}{}%
\end{pgfscope}%
\end{pgfscope}%
\begin{pgfscope}%
\pgfpathrectangle{\pgfqpoint{0.100000in}{2.413063in}}{\pgfqpoint{5.037500in}{3.427208in}}%
\pgfusepath{clip}%
\pgfsetrectcap%
\pgfsetroundjoin%
\pgfsetlinewidth{1.505625pt}%
\definecolor{currentstroke}{rgb}{0.678431,1.000000,0.184314}%
\pgfsetstrokecolor{currentstroke}%
\pgfsetstrokeopacity{0.500000}%
\pgfsetdash{}{0pt}%
\pgfpathmoveto{\pgfqpoint{2.858226in}{3.864827in}}%
\pgfusepath{stroke}%
\end{pgfscope}%
\begin{pgfscope}%
\pgfpathrectangle{\pgfqpoint{0.100000in}{2.413063in}}{\pgfqpoint{5.037500in}{3.427208in}}%
\pgfusepath{clip}%
\pgfsetbuttcap%
\pgfsetroundjoin%
\definecolor{currentfill}{rgb}{0.678431,1.000000,0.184314}%
\pgfsetfillcolor{currentfill}%
\pgfsetfillopacity{0.500000}%
\pgfsetlinewidth{0.250937pt}%
\definecolor{currentstroke}{rgb}{0.000000,0.000000,0.000000}%
\pgfsetstrokecolor{currentstroke}%
\pgfsetstrokeopacity{0.500000}%
\pgfsetdash{}{0pt}%
\pgfsys@defobject{currentmarker}{\pgfqpoint{-0.033333in}{-0.033333in}}{\pgfqpoint{0.033333in}{0.033333in}}{%
\pgfpathmoveto{\pgfqpoint{0.000000in}{-0.033333in}}%
\pgfpathcurveto{\pgfqpoint{0.008840in}{-0.033333in}}{\pgfqpoint{0.017319in}{-0.029821in}}{\pgfqpoint{0.023570in}{-0.023570in}}%
\pgfpathcurveto{\pgfqpoint{0.029821in}{-0.017319in}}{\pgfqpoint{0.033333in}{-0.008840in}}{\pgfqpoint{0.033333in}{0.000000in}}%
\pgfpathcurveto{\pgfqpoint{0.033333in}{0.008840in}}{\pgfqpoint{0.029821in}{0.017319in}}{\pgfqpoint{0.023570in}{0.023570in}}%
\pgfpathcurveto{\pgfqpoint{0.017319in}{0.029821in}}{\pgfqpoint{0.008840in}{0.033333in}}{\pgfqpoint{0.000000in}{0.033333in}}%
\pgfpathcurveto{\pgfqpoint{-0.008840in}{0.033333in}}{\pgfqpoint{-0.017319in}{0.029821in}}{\pgfqpoint{-0.023570in}{0.023570in}}%
\pgfpathcurveto{\pgfqpoint{-0.029821in}{0.017319in}}{\pgfqpoint{-0.033333in}{0.008840in}}{\pgfqpoint{-0.033333in}{0.000000in}}%
\pgfpathcurveto{\pgfqpoint{-0.033333in}{-0.008840in}}{\pgfqpoint{-0.029821in}{-0.017319in}}{\pgfqpoint{-0.023570in}{-0.023570in}}%
\pgfpathcurveto{\pgfqpoint{-0.017319in}{-0.029821in}}{\pgfqpoint{-0.008840in}{-0.033333in}}{\pgfqpoint{0.000000in}{-0.033333in}}%
\pgfpathclose%
\pgfusepath{stroke,fill}%
}%
\begin{pgfscope}%
\pgfsys@transformshift{2.858226in}{3.864827in}%
\pgfsys@useobject{currentmarker}{}%
\end{pgfscope}%
\end{pgfscope}%
\begin{pgfscope}%
\pgfpathrectangle{\pgfqpoint{0.100000in}{2.413063in}}{\pgfqpoint{5.037500in}{3.427208in}}%
\pgfusepath{clip}%
\pgfsetrectcap%
\pgfsetroundjoin%
\pgfsetlinewidth{1.505625pt}%
\definecolor{currentstroke}{rgb}{0.678431,1.000000,0.184314}%
\pgfsetstrokecolor{currentstroke}%
\pgfsetstrokeopacity{0.500000}%
\pgfsetdash{}{0pt}%
\pgfpathmoveto{\pgfqpoint{2.990171in}{3.758232in}}%
\pgfusepath{stroke}%
\end{pgfscope}%
\begin{pgfscope}%
\pgfpathrectangle{\pgfqpoint{0.100000in}{2.413063in}}{\pgfqpoint{5.037500in}{3.427208in}}%
\pgfusepath{clip}%
\pgfsetbuttcap%
\pgfsetroundjoin%
\definecolor{currentfill}{rgb}{0.678431,1.000000,0.184314}%
\pgfsetfillcolor{currentfill}%
\pgfsetfillopacity{0.500000}%
\pgfsetlinewidth{0.250937pt}%
\definecolor{currentstroke}{rgb}{0.000000,0.000000,0.000000}%
\pgfsetstrokecolor{currentstroke}%
\pgfsetstrokeopacity{0.500000}%
\pgfsetdash{}{0pt}%
\pgfsys@defobject{currentmarker}{\pgfqpoint{-0.011111in}{-0.011111in}}{\pgfqpoint{0.011111in}{0.011111in}}{%
\pgfpathmoveto{\pgfqpoint{0.000000in}{-0.011111in}}%
\pgfpathcurveto{\pgfqpoint{0.002947in}{-0.011111in}}{\pgfqpoint{0.005773in}{-0.009940in}}{\pgfqpoint{0.007857in}{-0.007857in}}%
\pgfpathcurveto{\pgfqpoint{0.009940in}{-0.005773in}}{\pgfqpoint{0.011111in}{-0.002947in}}{\pgfqpoint{0.011111in}{0.000000in}}%
\pgfpathcurveto{\pgfqpoint{0.011111in}{0.002947in}}{\pgfqpoint{0.009940in}{0.005773in}}{\pgfqpoint{0.007857in}{0.007857in}}%
\pgfpathcurveto{\pgfqpoint{0.005773in}{0.009940in}}{\pgfqpoint{0.002947in}{0.011111in}}{\pgfqpoint{0.000000in}{0.011111in}}%
\pgfpathcurveto{\pgfqpoint{-0.002947in}{0.011111in}}{\pgfqpoint{-0.005773in}{0.009940in}}{\pgfqpoint{-0.007857in}{0.007857in}}%
\pgfpathcurveto{\pgfqpoint{-0.009940in}{0.005773in}}{\pgfqpoint{-0.011111in}{0.002947in}}{\pgfqpoint{-0.011111in}{0.000000in}}%
\pgfpathcurveto{\pgfqpoint{-0.011111in}{-0.002947in}}{\pgfqpoint{-0.009940in}{-0.005773in}}{\pgfqpoint{-0.007857in}{-0.007857in}}%
\pgfpathcurveto{\pgfqpoint{-0.005773in}{-0.009940in}}{\pgfqpoint{-0.002947in}{-0.011111in}}{\pgfqpoint{0.000000in}{-0.011111in}}%
\pgfpathclose%
\pgfusepath{stroke,fill}%
}%
\begin{pgfscope}%
\pgfsys@transformshift{2.990171in}{3.758232in}%
\pgfsys@useobject{currentmarker}{}%
\end{pgfscope}%
\end{pgfscope}%
\begin{pgfscope}%
\pgfpathrectangle{\pgfqpoint{0.100000in}{2.413063in}}{\pgfqpoint{5.037500in}{3.427208in}}%
\pgfusepath{clip}%
\pgfsetrectcap%
\pgfsetroundjoin%
\pgfsetlinewidth{1.505625pt}%
\definecolor{currentstroke}{rgb}{0.678431,1.000000,0.184314}%
\pgfsetstrokecolor{currentstroke}%
\pgfsetstrokeopacity{0.500000}%
\pgfsetdash{}{0pt}%
\pgfpathmoveto{\pgfqpoint{3.205657in}{3.926711in}}%
\pgfusepath{stroke}%
\end{pgfscope}%
\begin{pgfscope}%
\pgfpathrectangle{\pgfqpoint{0.100000in}{2.413063in}}{\pgfqpoint{5.037500in}{3.427208in}}%
\pgfusepath{clip}%
\pgfsetbuttcap%
\pgfsetroundjoin%
\definecolor{currentfill}{rgb}{0.678431,1.000000,0.184314}%
\pgfsetfillcolor{currentfill}%
\pgfsetfillopacity{0.500000}%
\pgfsetlinewidth{0.250937pt}%
\definecolor{currentstroke}{rgb}{0.000000,0.000000,0.000000}%
\pgfsetstrokecolor{currentstroke}%
\pgfsetstrokeopacity{0.500000}%
\pgfsetdash{}{0pt}%
\pgfsys@defobject{currentmarker}{\pgfqpoint{-0.019444in}{-0.019444in}}{\pgfqpoint{0.019444in}{0.019444in}}{%
\pgfpathmoveto{\pgfqpoint{0.000000in}{-0.019444in}}%
\pgfpathcurveto{\pgfqpoint{0.005157in}{-0.019444in}}{\pgfqpoint{0.010103in}{-0.017396in}}{\pgfqpoint{0.013749in}{-0.013749in}}%
\pgfpathcurveto{\pgfqpoint{0.017396in}{-0.010103in}}{\pgfqpoint{0.019444in}{-0.005157in}}{\pgfqpoint{0.019444in}{0.000000in}}%
\pgfpathcurveto{\pgfqpoint{0.019444in}{0.005157in}}{\pgfqpoint{0.017396in}{0.010103in}}{\pgfqpoint{0.013749in}{0.013749in}}%
\pgfpathcurveto{\pgfqpoint{0.010103in}{0.017396in}}{\pgfqpoint{0.005157in}{0.019444in}}{\pgfqpoint{0.000000in}{0.019444in}}%
\pgfpathcurveto{\pgfqpoint{-0.005157in}{0.019444in}}{\pgfqpoint{-0.010103in}{0.017396in}}{\pgfqpoint{-0.013749in}{0.013749in}}%
\pgfpathcurveto{\pgfqpoint{-0.017396in}{0.010103in}}{\pgfqpoint{-0.019444in}{0.005157in}}{\pgfqpoint{-0.019444in}{0.000000in}}%
\pgfpathcurveto{\pgfqpoint{-0.019444in}{-0.005157in}}{\pgfqpoint{-0.017396in}{-0.010103in}}{\pgfqpoint{-0.013749in}{-0.013749in}}%
\pgfpathcurveto{\pgfqpoint{-0.010103in}{-0.017396in}}{\pgfqpoint{-0.005157in}{-0.019444in}}{\pgfqpoint{0.000000in}{-0.019444in}}%
\pgfpathclose%
\pgfusepath{stroke,fill}%
}%
\begin{pgfscope}%
\pgfsys@transformshift{3.205657in}{3.926711in}%
\pgfsys@useobject{currentmarker}{}%
\end{pgfscope}%
\end{pgfscope}%
\begin{pgfscope}%
\pgfpathrectangle{\pgfqpoint{0.100000in}{2.413063in}}{\pgfqpoint{5.037500in}{3.427208in}}%
\pgfusepath{clip}%
\pgfsetrectcap%
\pgfsetroundjoin%
\pgfsetlinewidth{1.505625pt}%
\definecolor{currentstroke}{rgb}{0.678431,1.000000,0.184314}%
\pgfsetstrokecolor{currentstroke}%
\pgfsetstrokeopacity{0.500000}%
\pgfsetdash{}{0pt}%
\pgfpathmoveto{\pgfqpoint{3.061256in}{3.794532in}}%
\pgfusepath{stroke}%
\end{pgfscope}%
\begin{pgfscope}%
\pgfpathrectangle{\pgfqpoint{0.100000in}{2.413063in}}{\pgfqpoint{5.037500in}{3.427208in}}%
\pgfusepath{clip}%
\pgfsetbuttcap%
\pgfsetroundjoin%
\definecolor{currentfill}{rgb}{0.678431,1.000000,0.184314}%
\pgfsetfillcolor{currentfill}%
\pgfsetfillopacity{0.500000}%
\pgfsetlinewidth{0.250937pt}%
\definecolor{currentstroke}{rgb}{0.000000,0.000000,0.000000}%
\pgfsetstrokecolor{currentstroke}%
\pgfsetstrokeopacity{0.500000}%
\pgfsetdash{}{0pt}%
\pgfsys@defobject{currentmarker}{\pgfqpoint{-0.016667in}{-0.016667in}}{\pgfqpoint{0.016667in}{0.016667in}}{%
\pgfpathmoveto{\pgfqpoint{0.000000in}{-0.016667in}}%
\pgfpathcurveto{\pgfqpoint{0.004420in}{-0.016667in}}{\pgfqpoint{0.008660in}{-0.014911in}}{\pgfqpoint{0.011785in}{-0.011785in}}%
\pgfpathcurveto{\pgfqpoint{0.014911in}{-0.008660in}}{\pgfqpoint{0.016667in}{-0.004420in}}{\pgfqpoint{0.016667in}{0.000000in}}%
\pgfpathcurveto{\pgfqpoint{0.016667in}{0.004420in}}{\pgfqpoint{0.014911in}{0.008660in}}{\pgfqpoint{0.011785in}{0.011785in}}%
\pgfpathcurveto{\pgfqpoint{0.008660in}{0.014911in}}{\pgfqpoint{0.004420in}{0.016667in}}{\pgfqpoint{0.000000in}{0.016667in}}%
\pgfpathcurveto{\pgfqpoint{-0.004420in}{0.016667in}}{\pgfqpoint{-0.008660in}{0.014911in}}{\pgfqpoint{-0.011785in}{0.011785in}}%
\pgfpathcurveto{\pgfqpoint{-0.014911in}{0.008660in}}{\pgfqpoint{-0.016667in}{0.004420in}}{\pgfqpoint{-0.016667in}{0.000000in}}%
\pgfpathcurveto{\pgfqpoint{-0.016667in}{-0.004420in}}{\pgfqpoint{-0.014911in}{-0.008660in}}{\pgfqpoint{-0.011785in}{-0.011785in}}%
\pgfpathcurveto{\pgfqpoint{-0.008660in}{-0.014911in}}{\pgfqpoint{-0.004420in}{-0.016667in}}{\pgfqpoint{0.000000in}{-0.016667in}}%
\pgfpathclose%
\pgfusepath{stroke,fill}%
}%
\begin{pgfscope}%
\pgfsys@transformshift{3.061256in}{3.794532in}%
\pgfsys@useobject{currentmarker}{}%
\end{pgfscope}%
\end{pgfscope}%
\begin{pgfscope}%
\pgfpathrectangle{\pgfqpoint{0.100000in}{2.413063in}}{\pgfqpoint{5.037500in}{3.427208in}}%
\pgfusepath{clip}%
\pgfsetrectcap%
\pgfsetroundjoin%
\pgfsetlinewidth{1.505625pt}%
\definecolor{currentstroke}{rgb}{0.678431,1.000000,0.184314}%
\pgfsetstrokecolor{currentstroke}%
\pgfsetstrokeopacity{0.500000}%
\pgfsetdash{}{0pt}%
\pgfpathmoveto{\pgfqpoint{3.090389in}{3.735577in}}%
\pgfusepath{stroke}%
\end{pgfscope}%
\begin{pgfscope}%
\pgfpathrectangle{\pgfqpoint{0.100000in}{2.413063in}}{\pgfqpoint{5.037500in}{3.427208in}}%
\pgfusepath{clip}%
\pgfsetbuttcap%
\pgfsetroundjoin%
\definecolor{currentfill}{rgb}{0.678431,1.000000,0.184314}%
\pgfsetfillcolor{currentfill}%
\pgfsetfillopacity{0.500000}%
\pgfsetlinewidth{0.250937pt}%
\definecolor{currentstroke}{rgb}{0.000000,0.000000,0.000000}%
\pgfsetstrokecolor{currentstroke}%
\pgfsetstrokeopacity{0.500000}%
\pgfsetdash{}{0pt}%
\pgfsys@defobject{currentmarker}{\pgfqpoint{-0.030556in}{-0.030556in}}{\pgfqpoint{0.030556in}{0.030556in}}{%
\pgfpathmoveto{\pgfqpoint{0.000000in}{-0.030556in}}%
\pgfpathcurveto{\pgfqpoint{0.008103in}{-0.030556in}}{\pgfqpoint{0.015876in}{-0.027336in}}{\pgfqpoint{0.021606in}{-0.021606in}}%
\pgfpathcurveto{\pgfqpoint{0.027336in}{-0.015876in}}{\pgfqpoint{0.030556in}{-0.008103in}}{\pgfqpoint{0.030556in}{0.000000in}}%
\pgfpathcurveto{\pgfqpoint{0.030556in}{0.008103in}}{\pgfqpoint{0.027336in}{0.015876in}}{\pgfqpoint{0.021606in}{0.021606in}}%
\pgfpathcurveto{\pgfqpoint{0.015876in}{0.027336in}}{\pgfqpoint{0.008103in}{0.030556in}}{\pgfqpoint{0.000000in}{0.030556in}}%
\pgfpathcurveto{\pgfqpoint{-0.008103in}{0.030556in}}{\pgfqpoint{-0.015876in}{0.027336in}}{\pgfqpoint{-0.021606in}{0.021606in}}%
\pgfpathcurveto{\pgfqpoint{-0.027336in}{0.015876in}}{\pgfqpoint{-0.030556in}{0.008103in}}{\pgfqpoint{-0.030556in}{0.000000in}}%
\pgfpathcurveto{\pgfqpoint{-0.030556in}{-0.008103in}}{\pgfqpoint{-0.027336in}{-0.015876in}}{\pgfqpoint{-0.021606in}{-0.021606in}}%
\pgfpathcurveto{\pgfqpoint{-0.015876in}{-0.027336in}}{\pgfqpoint{-0.008103in}{-0.030556in}}{\pgfqpoint{0.000000in}{-0.030556in}}%
\pgfpathclose%
\pgfusepath{stroke,fill}%
}%
\begin{pgfscope}%
\pgfsys@transformshift{3.090389in}{3.735577in}%
\pgfsys@useobject{currentmarker}{}%
\end{pgfscope}%
\end{pgfscope}%
\begin{pgfscope}%
\pgfpathrectangle{\pgfqpoint{0.100000in}{2.413063in}}{\pgfqpoint{5.037500in}{3.427208in}}%
\pgfusepath{clip}%
\pgfsetrectcap%
\pgfsetroundjoin%
\pgfsetlinewidth{1.505625pt}%
\definecolor{currentstroke}{rgb}{0.501961,0.501961,0.501961}%
\pgfsetstrokecolor{currentstroke}%
\pgfsetstrokeopacity{0.500000}%
\pgfsetdash{}{0pt}%
\pgfpathmoveto{\pgfqpoint{0.570500in}{4.160029in}}%
\pgfusepath{stroke}%
\end{pgfscope}%
\begin{pgfscope}%
\pgfpathrectangle{\pgfqpoint{0.100000in}{2.413063in}}{\pgfqpoint{5.037500in}{3.427208in}}%
\pgfusepath{clip}%
\pgfsetbuttcap%
\pgfsetroundjoin%
\definecolor{currentfill}{rgb}{0.501961,0.501961,0.501961}%
\pgfsetfillcolor{currentfill}%
\pgfsetfillopacity{0.500000}%
\pgfsetlinewidth{0.250937pt}%
\definecolor{currentstroke}{rgb}{0.000000,0.000000,0.000000}%
\pgfsetstrokecolor{currentstroke}%
\pgfsetstrokeopacity{0.500000}%
\pgfsetdash{}{0pt}%
\pgfsys@defobject{currentmarker}{\pgfqpoint{-0.013889in}{-0.013889in}}{\pgfqpoint{0.013889in}{0.013889in}}{%
\pgfpathmoveto{\pgfqpoint{0.000000in}{-0.013889in}}%
\pgfpathcurveto{\pgfqpoint{0.003683in}{-0.013889in}}{\pgfqpoint{0.007216in}{-0.012425in}}{\pgfqpoint{0.009821in}{-0.009821in}}%
\pgfpathcurveto{\pgfqpoint{0.012425in}{-0.007216in}}{\pgfqpoint{0.013889in}{-0.003683in}}{\pgfqpoint{0.013889in}{0.000000in}}%
\pgfpathcurveto{\pgfqpoint{0.013889in}{0.003683in}}{\pgfqpoint{0.012425in}{0.007216in}}{\pgfqpoint{0.009821in}{0.009821in}}%
\pgfpathcurveto{\pgfqpoint{0.007216in}{0.012425in}}{\pgfqpoint{0.003683in}{0.013889in}}{\pgfqpoint{0.000000in}{0.013889in}}%
\pgfpathcurveto{\pgfqpoint{-0.003683in}{0.013889in}}{\pgfqpoint{-0.007216in}{0.012425in}}{\pgfqpoint{-0.009821in}{0.009821in}}%
\pgfpathcurveto{\pgfqpoint{-0.012425in}{0.007216in}}{\pgfqpoint{-0.013889in}{0.003683in}}{\pgfqpoint{-0.013889in}{0.000000in}}%
\pgfpathcurveto{\pgfqpoint{-0.013889in}{-0.003683in}}{\pgfqpoint{-0.012425in}{-0.007216in}}{\pgfqpoint{-0.009821in}{-0.009821in}}%
\pgfpathcurveto{\pgfqpoint{-0.007216in}{-0.012425in}}{\pgfqpoint{-0.003683in}{-0.013889in}}{\pgfqpoint{0.000000in}{-0.013889in}}%
\pgfpathclose%
\pgfusepath{stroke,fill}%
}%
\begin{pgfscope}%
\pgfsys@transformshift{0.570500in}{4.160029in}%
\pgfsys@useobject{currentmarker}{}%
\end{pgfscope}%
\end{pgfscope}%
\begin{pgfscope}%
\pgfpathrectangle{\pgfqpoint{0.100000in}{2.413063in}}{\pgfqpoint{5.037500in}{3.427208in}}%
\pgfusepath{clip}%
\pgfsetrectcap%
\pgfsetroundjoin%
\pgfsetlinewidth{1.505625pt}%
\definecolor{currentstroke}{rgb}{0.000000,0.000000,1.000000}%
\pgfsetstrokecolor{currentstroke}%
\pgfsetstrokeopacity{0.500000}%
\pgfsetdash{}{0pt}%
\pgfpathmoveto{\pgfqpoint{0.461714in}{4.713248in}}%
\pgfusepath{stroke}%
\end{pgfscope}%
\begin{pgfscope}%
\pgfpathrectangle{\pgfqpoint{0.100000in}{2.413063in}}{\pgfqpoint{5.037500in}{3.427208in}}%
\pgfusepath{clip}%
\pgfsetbuttcap%
\pgfsetroundjoin%
\definecolor{currentfill}{rgb}{0.000000,0.000000,1.000000}%
\pgfsetfillcolor{currentfill}%
\pgfsetfillopacity{0.500000}%
\pgfsetlinewidth{0.250937pt}%
\definecolor{currentstroke}{rgb}{0.000000,0.000000,0.000000}%
\pgfsetstrokecolor{currentstroke}%
\pgfsetstrokeopacity{0.500000}%
\pgfsetdash{}{0pt}%
\pgfsys@defobject{currentmarker}{\pgfqpoint{-0.005556in}{-0.005556in}}{\pgfqpoint{0.005556in}{0.005556in}}{%
\pgfpathmoveto{\pgfqpoint{0.000000in}{-0.005556in}}%
\pgfpathcurveto{\pgfqpoint{0.001473in}{-0.005556in}}{\pgfqpoint{0.002887in}{-0.004970in}}{\pgfqpoint{0.003928in}{-0.003928in}}%
\pgfpathcurveto{\pgfqpoint{0.004970in}{-0.002887in}}{\pgfqpoint{0.005556in}{-0.001473in}}{\pgfqpoint{0.005556in}{0.000000in}}%
\pgfpathcurveto{\pgfqpoint{0.005556in}{0.001473in}}{\pgfqpoint{0.004970in}{0.002887in}}{\pgfqpoint{0.003928in}{0.003928in}}%
\pgfpathcurveto{\pgfqpoint{0.002887in}{0.004970in}}{\pgfqpoint{0.001473in}{0.005556in}}{\pgfqpoint{0.000000in}{0.005556in}}%
\pgfpathcurveto{\pgfqpoint{-0.001473in}{0.005556in}}{\pgfqpoint{-0.002887in}{0.004970in}}{\pgfqpoint{-0.003928in}{0.003928in}}%
\pgfpathcurveto{\pgfqpoint{-0.004970in}{0.002887in}}{\pgfqpoint{-0.005556in}{0.001473in}}{\pgfqpoint{-0.005556in}{0.000000in}}%
\pgfpathcurveto{\pgfqpoint{-0.005556in}{-0.001473in}}{\pgfqpoint{-0.004970in}{-0.002887in}}{\pgfqpoint{-0.003928in}{-0.003928in}}%
\pgfpathcurveto{\pgfqpoint{-0.002887in}{-0.004970in}}{\pgfqpoint{-0.001473in}{-0.005556in}}{\pgfqpoint{0.000000in}{-0.005556in}}%
\pgfpathclose%
\pgfusepath{stroke,fill}%
}%
\begin{pgfscope}%
\pgfsys@transformshift{0.461714in}{4.713248in}%
\pgfsys@useobject{currentmarker}{}%
\end{pgfscope}%
\end{pgfscope}%
\begin{pgfscope}%
\pgfpathrectangle{\pgfqpoint{0.100000in}{2.413063in}}{\pgfqpoint{5.037500in}{3.427208in}}%
\pgfusepath{clip}%
\pgfsetrectcap%
\pgfsetroundjoin%
\pgfsetlinewidth{1.505625pt}%
\definecolor{currentstroke}{rgb}{0.678431,1.000000,0.184314}%
\pgfsetstrokecolor{currentstroke}%
\pgfsetstrokeopacity{0.500000}%
\pgfsetdash{}{0pt}%
\pgfpathmoveto{\pgfqpoint{0.818988in}{3.790241in}}%
\pgfusepath{stroke}%
\end{pgfscope}%
\begin{pgfscope}%
\pgfpathrectangle{\pgfqpoint{0.100000in}{2.413063in}}{\pgfqpoint{5.037500in}{3.427208in}}%
\pgfusepath{clip}%
\pgfsetbuttcap%
\pgfsetroundjoin%
\definecolor{currentfill}{rgb}{0.678431,1.000000,0.184314}%
\pgfsetfillcolor{currentfill}%
\pgfsetfillopacity{0.500000}%
\pgfsetlinewidth{0.250937pt}%
\definecolor{currentstroke}{rgb}{0.000000,0.000000,0.000000}%
\pgfsetstrokecolor{currentstroke}%
\pgfsetstrokeopacity{0.500000}%
\pgfsetdash{}{0pt}%
\pgfsys@defobject{currentmarker}{\pgfqpoint{-0.172222in}{-0.172222in}}{\pgfqpoint{0.172222in}{0.172222in}}{%
\pgfpathmoveto{\pgfqpoint{0.000000in}{-0.172222in}}%
\pgfpathcurveto{\pgfqpoint{0.045674in}{-0.172222in}}{\pgfqpoint{0.089483in}{-0.154076in}}{\pgfqpoint{0.121780in}{-0.121780in}}%
\pgfpathcurveto{\pgfqpoint{0.154076in}{-0.089483in}}{\pgfqpoint{0.172222in}{-0.045674in}}{\pgfqpoint{0.172222in}{0.000000in}}%
\pgfpathcurveto{\pgfqpoint{0.172222in}{0.045674in}}{\pgfqpoint{0.154076in}{0.089483in}}{\pgfqpoint{0.121780in}{0.121780in}}%
\pgfpathcurveto{\pgfqpoint{0.089483in}{0.154076in}}{\pgfqpoint{0.045674in}{0.172222in}}{\pgfqpoint{0.000000in}{0.172222in}}%
\pgfpathcurveto{\pgfqpoint{-0.045674in}{0.172222in}}{\pgfqpoint{-0.089483in}{0.154076in}}{\pgfqpoint{-0.121780in}{0.121780in}}%
\pgfpathcurveto{\pgfqpoint{-0.154076in}{0.089483in}}{\pgfqpoint{-0.172222in}{0.045674in}}{\pgfqpoint{-0.172222in}{0.000000in}}%
\pgfpathcurveto{\pgfqpoint{-0.172222in}{-0.045674in}}{\pgfqpoint{-0.154076in}{-0.089483in}}{\pgfqpoint{-0.121780in}{-0.121780in}}%
\pgfpathcurveto{\pgfqpoint{-0.089483in}{-0.154076in}}{\pgfqpoint{-0.045674in}{-0.172222in}}{\pgfqpoint{0.000000in}{-0.172222in}}%
\pgfpathclose%
\pgfusepath{stroke,fill}%
}%
\begin{pgfscope}%
\pgfsys@transformshift{0.818988in}{3.790241in}%
\pgfsys@useobject{currentmarker}{}%
\end{pgfscope}%
\end{pgfscope}%
\begin{pgfscope}%
\pgfpathrectangle{\pgfqpoint{0.100000in}{2.413063in}}{\pgfqpoint{5.037500in}{3.427208in}}%
\pgfusepath{clip}%
\pgfsetrectcap%
\pgfsetroundjoin%
\pgfsetlinewidth{1.505625pt}%
\definecolor{currentstroke}{rgb}{0.678431,1.000000,0.184314}%
\pgfsetstrokecolor{currentstroke}%
\pgfsetstrokeopacity{0.500000}%
\pgfsetdash{}{0pt}%
\pgfpathmoveto{\pgfqpoint{0.549937in}{4.327788in}}%
\pgfusepath{stroke}%
\end{pgfscope}%
\begin{pgfscope}%
\pgfpathrectangle{\pgfqpoint{0.100000in}{2.413063in}}{\pgfqpoint{5.037500in}{3.427208in}}%
\pgfusepath{clip}%
\pgfsetbuttcap%
\pgfsetroundjoin%
\definecolor{currentfill}{rgb}{0.678431,1.000000,0.184314}%
\pgfsetfillcolor{currentfill}%
\pgfsetfillopacity{0.500000}%
\pgfsetlinewidth{0.250937pt}%
\definecolor{currentstroke}{rgb}{0.000000,0.000000,0.000000}%
\pgfsetstrokecolor{currentstroke}%
\pgfsetstrokeopacity{0.500000}%
\pgfsetdash{}{0pt}%
\pgfsys@defobject{currentmarker}{\pgfqpoint{-0.011111in}{-0.011111in}}{\pgfqpoint{0.011111in}{0.011111in}}{%
\pgfpathmoveto{\pgfqpoint{0.000000in}{-0.011111in}}%
\pgfpathcurveto{\pgfqpoint{0.002947in}{-0.011111in}}{\pgfqpoint{0.005773in}{-0.009940in}}{\pgfqpoint{0.007857in}{-0.007857in}}%
\pgfpathcurveto{\pgfqpoint{0.009940in}{-0.005773in}}{\pgfqpoint{0.011111in}{-0.002947in}}{\pgfqpoint{0.011111in}{0.000000in}}%
\pgfpathcurveto{\pgfqpoint{0.011111in}{0.002947in}}{\pgfqpoint{0.009940in}{0.005773in}}{\pgfqpoint{0.007857in}{0.007857in}}%
\pgfpathcurveto{\pgfqpoint{0.005773in}{0.009940in}}{\pgfqpoint{0.002947in}{0.011111in}}{\pgfqpoint{0.000000in}{0.011111in}}%
\pgfpathcurveto{\pgfqpoint{-0.002947in}{0.011111in}}{\pgfqpoint{-0.005773in}{0.009940in}}{\pgfqpoint{-0.007857in}{0.007857in}}%
\pgfpathcurveto{\pgfqpoint{-0.009940in}{0.005773in}}{\pgfqpoint{-0.011111in}{0.002947in}}{\pgfqpoint{-0.011111in}{0.000000in}}%
\pgfpathcurveto{\pgfqpoint{-0.011111in}{-0.002947in}}{\pgfqpoint{-0.009940in}{-0.005773in}}{\pgfqpoint{-0.007857in}{-0.007857in}}%
\pgfpathcurveto{\pgfqpoint{-0.005773in}{-0.009940in}}{\pgfqpoint{-0.002947in}{-0.011111in}}{\pgfqpoint{0.000000in}{-0.011111in}}%
\pgfpathclose%
\pgfusepath{stroke,fill}%
}%
\begin{pgfscope}%
\pgfsys@transformshift{0.549937in}{4.327788in}%
\pgfsys@useobject{currentmarker}{}%
\end{pgfscope}%
\end{pgfscope}%
\begin{pgfscope}%
\pgfpathrectangle{\pgfqpoint{0.100000in}{2.413063in}}{\pgfqpoint{5.037500in}{3.427208in}}%
\pgfusepath{clip}%
\pgfsetrectcap%
\pgfsetroundjoin%
\pgfsetlinewidth{1.505625pt}%
\definecolor{currentstroke}{rgb}{0.678431,1.000000,0.184314}%
\pgfsetstrokecolor{currentstroke}%
\pgfsetstrokeopacity{0.500000}%
\pgfsetdash{}{0pt}%
\pgfpathmoveto{\pgfqpoint{0.543143in}{4.281573in}}%
\pgfusepath{stroke}%
\end{pgfscope}%
\begin{pgfscope}%
\pgfpathrectangle{\pgfqpoint{0.100000in}{2.413063in}}{\pgfqpoint{5.037500in}{3.427208in}}%
\pgfusepath{clip}%
\pgfsetbuttcap%
\pgfsetroundjoin%
\definecolor{currentfill}{rgb}{0.678431,1.000000,0.184314}%
\pgfsetfillcolor{currentfill}%
\pgfsetfillopacity{0.500000}%
\pgfsetlinewidth{0.250937pt}%
\definecolor{currentstroke}{rgb}{0.000000,0.000000,0.000000}%
\pgfsetstrokecolor{currentstroke}%
\pgfsetstrokeopacity{0.500000}%
\pgfsetdash{}{0pt}%
\pgfsys@defobject{currentmarker}{\pgfqpoint{-0.022222in}{-0.022222in}}{\pgfqpoint{0.022222in}{0.022222in}}{%
\pgfpathmoveto{\pgfqpoint{0.000000in}{-0.022222in}}%
\pgfpathcurveto{\pgfqpoint{0.005893in}{-0.022222in}}{\pgfqpoint{0.011546in}{-0.019881in}}{\pgfqpoint{0.015713in}{-0.015713in}}%
\pgfpathcurveto{\pgfqpoint{0.019881in}{-0.011546in}}{\pgfqpoint{0.022222in}{-0.005893in}}{\pgfqpoint{0.022222in}{0.000000in}}%
\pgfpathcurveto{\pgfqpoint{0.022222in}{0.005893in}}{\pgfqpoint{0.019881in}{0.011546in}}{\pgfqpoint{0.015713in}{0.015713in}}%
\pgfpathcurveto{\pgfqpoint{0.011546in}{0.019881in}}{\pgfqpoint{0.005893in}{0.022222in}}{\pgfqpoint{0.000000in}{0.022222in}}%
\pgfpathcurveto{\pgfqpoint{-0.005893in}{0.022222in}}{\pgfqpoint{-0.011546in}{0.019881in}}{\pgfqpoint{-0.015713in}{0.015713in}}%
\pgfpathcurveto{\pgfqpoint{-0.019881in}{0.011546in}}{\pgfqpoint{-0.022222in}{0.005893in}}{\pgfqpoint{-0.022222in}{0.000000in}}%
\pgfpathcurveto{\pgfqpoint{-0.022222in}{-0.005893in}}{\pgfqpoint{-0.019881in}{-0.011546in}}{\pgfqpoint{-0.015713in}{-0.015713in}}%
\pgfpathcurveto{\pgfqpoint{-0.011546in}{-0.019881in}}{\pgfqpoint{-0.005893in}{-0.022222in}}{\pgfqpoint{0.000000in}{-0.022222in}}%
\pgfpathclose%
\pgfusepath{stroke,fill}%
}%
\begin{pgfscope}%
\pgfsys@transformshift{0.543143in}{4.281573in}%
\pgfsys@useobject{currentmarker}{}%
\end{pgfscope}%
\end{pgfscope}%
\begin{pgfscope}%
\pgfpathrectangle{\pgfqpoint{0.100000in}{2.413063in}}{\pgfqpoint{5.037500in}{3.427208in}}%
\pgfusepath{clip}%
\pgfsetrectcap%
\pgfsetroundjoin%
\pgfsetlinewidth{1.505625pt}%
\definecolor{currentstroke}{rgb}{0.000000,0.000000,1.000000}%
\pgfsetstrokecolor{currentstroke}%
\pgfsetstrokeopacity{0.500000}%
\pgfsetdash{}{0pt}%
\pgfpathmoveto{\pgfqpoint{0.602428in}{3.993927in}}%
\pgfusepath{stroke}%
\end{pgfscope}%
\begin{pgfscope}%
\pgfpathrectangle{\pgfqpoint{0.100000in}{2.413063in}}{\pgfqpoint{5.037500in}{3.427208in}}%
\pgfusepath{clip}%
\pgfsetbuttcap%
\pgfsetroundjoin%
\definecolor{currentfill}{rgb}{0.000000,0.000000,1.000000}%
\pgfsetfillcolor{currentfill}%
\pgfsetfillopacity{0.500000}%
\pgfsetlinewidth{0.250937pt}%
\definecolor{currentstroke}{rgb}{0.000000,0.000000,0.000000}%
\pgfsetstrokecolor{currentstroke}%
\pgfsetstrokeopacity{0.500000}%
\pgfsetdash{}{0pt}%
\pgfsys@defobject{currentmarker}{\pgfqpoint{-0.047222in}{-0.047222in}}{\pgfqpoint{0.047222in}{0.047222in}}{%
\pgfpathmoveto{\pgfqpoint{0.000000in}{-0.047222in}}%
\pgfpathcurveto{\pgfqpoint{0.012523in}{-0.047222in}}{\pgfqpoint{0.024536in}{-0.042247in}}{\pgfqpoint{0.033391in}{-0.033391in}}%
\pgfpathcurveto{\pgfqpoint{0.042247in}{-0.024536in}}{\pgfqpoint{0.047222in}{-0.012523in}}{\pgfqpoint{0.047222in}{0.000000in}}%
\pgfpathcurveto{\pgfqpoint{0.047222in}{0.012523in}}{\pgfqpoint{0.042247in}{0.024536in}}{\pgfqpoint{0.033391in}{0.033391in}}%
\pgfpathcurveto{\pgfqpoint{0.024536in}{0.042247in}}{\pgfqpoint{0.012523in}{0.047222in}}{\pgfqpoint{0.000000in}{0.047222in}}%
\pgfpathcurveto{\pgfqpoint{-0.012523in}{0.047222in}}{\pgfqpoint{-0.024536in}{0.042247in}}{\pgfqpoint{-0.033391in}{0.033391in}}%
\pgfpathcurveto{\pgfqpoint{-0.042247in}{0.024536in}}{\pgfqpoint{-0.047222in}{0.012523in}}{\pgfqpoint{-0.047222in}{0.000000in}}%
\pgfpathcurveto{\pgfqpoint{-0.047222in}{-0.012523in}}{\pgfqpoint{-0.042247in}{-0.024536in}}{\pgfqpoint{-0.033391in}{-0.033391in}}%
\pgfpathcurveto{\pgfqpoint{-0.024536in}{-0.042247in}}{\pgfqpoint{-0.012523in}{-0.047222in}}{\pgfqpoint{0.000000in}{-0.047222in}}%
\pgfpathclose%
\pgfusepath{stroke,fill}%
}%
\begin{pgfscope}%
\pgfsys@transformshift{0.602428in}{3.993927in}%
\pgfsys@useobject{currentmarker}{}%
\end{pgfscope}%
\end{pgfscope}%
\begin{pgfscope}%
\pgfpathrectangle{\pgfqpoint{0.100000in}{2.413063in}}{\pgfqpoint{5.037500in}{3.427208in}}%
\pgfusepath{clip}%
\pgfsetrectcap%
\pgfsetroundjoin%
\pgfsetlinewidth{1.505625pt}%
\definecolor{currentstroke}{rgb}{0.678431,1.000000,0.184314}%
\pgfsetstrokecolor{currentstroke}%
\pgfsetstrokeopacity{0.500000}%
\pgfsetdash{}{0pt}%
\pgfpathmoveto{\pgfqpoint{0.557875in}{4.378461in}}%
\pgfusepath{stroke}%
\end{pgfscope}%
\begin{pgfscope}%
\pgfpathrectangle{\pgfqpoint{0.100000in}{2.413063in}}{\pgfqpoint{5.037500in}{3.427208in}}%
\pgfusepath{clip}%
\pgfsetbuttcap%
\pgfsetroundjoin%
\definecolor{currentfill}{rgb}{0.678431,1.000000,0.184314}%
\pgfsetfillcolor{currentfill}%
\pgfsetfillopacity{0.500000}%
\pgfsetlinewidth{0.250937pt}%
\definecolor{currentstroke}{rgb}{0.000000,0.000000,0.000000}%
\pgfsetstrokecolor{currentstroke}%
\pgfsetstrokeopacity{0.500000}%
\pgfsetdash{}{0pt}%
\pgfsys@defobject{currentmarker}{\pgfqpoint{-0.011111in}{-0.011111in}}{\pgfqpoint{0.011111in}{0.011111in}}{%
\pgfpathmoveto{\pgfqpoint{0.000000in}{-0.011111in}}%
\pgfpathcurveto{\pgfqpoint{0.002947in}{-0.011111in}}{\pgfqpoint{0.005773in}{-0.009940in}}{\pgfqpoint{0.007857in}{-0.007857in}}%
\pgfpathcurveto{\pgfqpoint{0.009940in}{-0.005773in}}{\pgfqpoint{0.011111in}{-0.002947in}}{\pgfqpoint{0.011111in}{0.000000in}}%
\pgfpathcurveto{\pgfqpoint{0.011111in}{0.002947in}}{\pgfqpoint{0.009940in}{0.005773in}}{\pgfqpoint{0.007857in}{0.007857in}}%
\pgfpathcurveto{\pgfqpoint{0.005773in}{0.009940in}}{\pgfqpoint{0.002947in}{0.011111in}}{\pgfqpoint{0.000000in}{0.011111in}}%
\pgfpathcurveto{\pgfqpoint{-0.002947in}{0.011111in}}{\pgfqpoint{-0.005773in}{0.009940in}}{\pgfqpoint{-0.007857in}{0.007857in}}%
\pgfpathcurveto{\pgfqpoint{-0.009940in}{0.005773in}}{\pgfqpoint{-0.011111in}{0.002947in}}{\pgfqpoint{-0.011111in}{0.000000in}}%
\pgfpathcurveto{\pgfqpoint{-0.011111in}{-0.002947in}}{\pgfqpoint{-0.009940in}{-0.005773in}}{\pgfqpoint{-0.007857in}{-0.007857in}}%
\pgfpathcurveto{\pgfqpoint{-0.005773in}{-0.009940in}}{\pgfqpoint{-0.002947in}{-0.011111in}}{\pgfqpoint{0.000000in}{-0.011111in}}%
\pgfpathclose%
\pgfusepath{stroke,fill}%
}%
\begin{pgfscope}%
\pgfsys@transformshift{0.557875in}{4.378461in}%
\pgfsys@useobject{currentmarker}{}%
\end{pgfscope}%
\end{pgfscope}%
\begin{pgfscope}%
\pgfpathrectangle{\pgfqpoint{0.100000in}{2.413063in}}{\pgfqpoint{5.037500in}{3.427208in}}%
\pgfusepath{clip}%
\pgfsetrectcap%
\pgfsetroundjoin%
\pgfsetlinewidth{1.505625pt}%
\definecolor{currentstroke}{rgb}{0.678431,1.000000,0.184314}%
\pgfsetstrokecolor{currentstroke}%
\pgfsetstrokeopacity{0.500000}%
\pgfsetdash{}{0pt}%
\pgfpathmoveto{\pgfqpoint{0.499312in}{4.410808in}}%
\pgfusepath{stroke}%
\end{pgfscope}%
\begin{pgfscope}%
\pgfpathrectangle{\pgfqpoint{0.100000in}{2.413063in}}{\pgfqpoint{5.037500in}{3.427208in}}%
\pgfusepath{clip}%
\pgfsetbuttcap%
\pgfsetroundjoin%
\definecolor{currentfill}{rgb}{0.678431,1.000000,0.184314}%
\pgfsetfillcolor{currentfill}%
\pgfsetfillopacity{0.500000}%
\pgfsetlinewidth{0.250937pt}%
\definecolor{currentstroke}{rgb}{0.000000,0.000000,0.000000}%
\pgfsetstrokecolor{currentstroke}%
\pgfsetstrokeopacity{0.500000}%
\pgfsetdash{}{0pt}%
\pgfsys@defobject{currentmarker}{\pgfqpoint{-0.008333in}{-0.008333in}}{\pgfqpoint{0.008333in}{0.008333in}}{%
\pgfpathmoveto{\pgfqpoint{0.000000in}{-0.008333in}}%
\pgfpathcurveto{\pgfqpoint{0.002210in}{-0.008333in}}{\pgfqpoint{0.004330in}{-0.007455in}}{\pgfqpoint{0.005893in}{-0.005893in}}%
\pgfpathcurveto{\pgfqpoint{0.007455in}{-0.004330in}}{\pgfqpoint{0.008333in}{-0.002210in}}{\pgfqpoint{0.008333in}{0.000000in}}%
\pgfpathcurveto{\pgfqpoint{0.008333in}{0.002210in}}{\pgfqpoint{0.007455in}{0.004330in}}{\pgfqpoint{0.005893in}{0.005893in}}%
\pgfpathcurveto{\pgfqpoint{0.004330in}{0.007455in}}{\pgfqpoint{0.002210in}{0.008333in}}{\pgfqpoint{0.000000in}{0.008333in}}%
\pgfpathcurveto{\pgfqpoint{-0.002210in}{0.008333in}}{\pgfqpoint{-0.004330in}{0.007455in}}{\pgfqpoint{-0.005893in}{0.005893in}}%
\pgfpathcurveto{\pgfqpoint{-0.007455in}{0.004330in}}{\pgfqpoint{-0.008333in}{0.002210in}}{\pgfqpoint{-0.008333in}{0.000000in}}%
\pgfpathcurveto{\pgfqpoint{-0.008333in}{-0.002210in}}{\pgfqpoint{-0.007455in}{-0.004330in}}{\pgfqpoint{-0.005893in}{-0.005893in}}%
\pgfpathcurveto{\pgfqpoint{-0.004330in}{-0.007455in}}{\pgfqpoint{-0.002210in}{-0.008333in}}{\pgfqpoint{0.000000in}{-0.008333in}}%
\pgfpathclose%
\pgfusepath{stroke,fill}%
}%
\begin{pgfscope}%
\pgfsys@transformshift{0.499312in}{4.410808in}%
\pgfsys@useobject{currentmarker}{}%
\end{pgfscope}%
\end{pgfscope}%
\begin{pgfscope}%
\pgfpathrectangle{\pgfqpoint{0.100000in}{2.413063in}}{\pgfqpoint{5.037500in}{3.427208in}}%
\pgfusepath{clip}%
\pgfsetrectcap%
\pgfsetroundjoin%
\pgfsetlinewidth{1.505625pt}%
\definecolor{currentstroke}{rgb}{0.501961,0.501961,0.501961}%
\pgfsetstrokecolor{currentstroke}%
\pgfsetstrokeopacity{0.500000}%
\pgfsetdash{}{0pt}%
\pgfpathmoveto{\pgfqpoint{0.465142in}{4.461082in}}%
\pgfusepath{stroke}%
\end{pgfscope}%
\begin{pgfscope}%
\pgfpathrectangle{\pgfqpoint{0.100000in}{2.413063in}}{\pgfqpoint{5.037500in}{3.427208in}}%
\pgfusepath{clip}%
\pgfsetbuttcap%
\pgfsetroundjoin%
\definecolor{currentfill}{rgb}{0.501961,0.501961,0.501961}%
\pgfsetfillcolor{currentfill}%
\pgfsetfillopacity{0.500000}%
\pgfsetlinewidth{0.250937pt}%
\definecolor{currentstroke}{rgb}{0.000000,0.000000,0.000000}%
\pgfsetstrokecolor{currentstroke}%
\pgfsetstrokeopacity{0.500000}%
\pgfsetdash{}{0pt}%
\pgfsys@defobject{currentmarker}{\pgfqpoint{-0.013889in}{-0.013889in}}{\pgfqpoint{0.013889in}{0.013889in}}{%
\pgfpathmoveto{\pgfqpoint{0.000000in}{-0.013889in}}%
\pgfpathcurveto{\pgfqpoint{0.003683in}{-0.013889in}}{\pgfqpoint{0.007216in}{-0.012425in}}{\pgfqpoint{0.009821in}{-0.009821in}}%
\pgfpathcurveto{\pgfqpoint{0.012425in}{-0.007216in}}{\pgfqpoint{0.013889in}{-0.003683in}}{\pgfqpoint{0.013889in}{0.000000in}}%
\pgfpathcurveto{\pgfqpoint{0.013889in}{0.003683in}}{\pgfqpoint{0.012425in}{0.007216in}}{\pgfqpoint{0.009821in}{0.009821in}}%
\pgfpathcurveto{\pgfqpoint{0.007216in}{0.012425in}}{\pgfqpoint{0.003683in}{0.013889in}}{\pgfqpoint{0.000000in}{0.013889in}}%
\pgfpathcurveto{\pgfqpoint{-0.003683in}{0.013889in}}{\pgfqpoint{-0.007216in}{0.012425in}}{\pgfqpoint{-0.009821in}{0.009821in}}%
\pgfpathcurveto{\pgfqpoint{-0.012425in}{0.007216in}}{\pgfqpoint{-0.013889in}{0.003683in}}{\pgfqpoint{-0.013889in}{0.000000in}}%
\pgfpathcurveto{\pgfqpoint{-0.013889in}{-0.003683in}}{\pgfqpoint{-0.012425in}{-0.007216in}}{\pgfqpoint{-0.009821in}{-0.009821in}}%
\pgfpathcurveto{\pgfqpoint{-0.007216in}{-0.012425in}}{\pgfqpoint{-0.003683in}{-0.013889in}}{\pgfqpoint{0.000000in}{-0.013889in}}%
\pgfpathclose%
\pgfusepath{stroke,fill}%
}%
\begin{pgfscope}%
\pgfsys@transformshift{0.465142in}{4.461082in}%
\pgfsys@useobject{currentmarker}{}%
\end{pgfscope}%
\end{pgfscope}%
\begin{pgfscope}%
\pgfpathrectangle{\pgfqpoint{0.100000in}{2.413063in}}{\pgfqpoint{5.037500in}{3.427208in}}%
\pgfusepath{clip}%
\pgfsetrectcap%
\pgfsetroundjoin%
\pgfsetlinewidth{1.505625pt}%
\definecolor{currentstroke}{rgb}{0.000000,0.000000,1.000000}%
\pgfsetstrokecolor{currentstroke}%
\pgfsetstrokeopacity{0.500000}%
\pgfsetdash{}{0pt}%
\pgfpathmoveto{\pgfqpoint{0.375033in}{4.567443in}}%
\pgfusepath{stroke}%
\end{pgfscope}%
\begin{pgfscope}%
\pgfpathrectangle{\pgfqpoint{0.100000in}{2.413063in}}{\pgfqpoint{5.037500in}{3.427208in}}%
\pgfusepath{clip}%
\pgfsetbuttcap%
\pgfsetroundjoin%
\definecolor{currentfill}{rgb}{0.000000,0.000000,1.000000}%
\pgfsetfillcolor{currentfill}%
\pgfsetfillopacity{0.500000}%
\pgfsetlinewidth{0.250937pt}%
\definecolor{currentstroke}{rgb}{0.000000,0.000000,0.000000}%
\pgfsetstrokecolor{currentstroke}%
\pgfsetstrokeopacity{0.500000}%
\pgfsetdash{}{0pt}%
\pgfsys@defobject{currentmarker}{\pgfqpoint{-0.027778in}{-0.027778in}}{\pgfqpoint{0.027778in}{0.027778in}}{%
\pgfpathmoveto{\pgfqpoint{0.000000in}{-0.027778in}}%
\pgfpathcurveto{\pgfqpoint{0.007367in}{-0.027778in}}{\pgfqpoint{0.014433in}{-0.024851in}}{\pgfqpoint{0.019642in}{-0.019642in}}%
\pgfpathcurveto{\pgfqpoint{0.024851in}{-0.014433in}}{\pgfqpoint{0.027778in}{-0.007367in}}{\pgfqpoint{0.027778in}{0.000000in}}%
\pgfpathcurveto{\pgfqpoint{0.027778in}{0.007367in}}{\pgfqpoint{0.024851in}{0.014433in}}{\pgfqpoint{0.019642in}{0.019642in}}%
\pgfpathcurveto{\pgfqpoint{0.014433in}{0.024851in}}{\pgfqpoint{0.007367in}{0.027778in}}{\pgfqpoint{0.000000in}{0.027778in}}%
\pgfpathcurveto{\pgfqpoint{-0.007367in}{0.027778in}}{\pgfqpoint{-0.014433in}{0.024851in}}{\pgfqpoint{-0.019642in}{0.019642in}}%
\pgfpathcurveto{\pgfqpoint{-0.024851in}{0.014433in}}{\pgfqpoint{-0.027778in}{0.007367in}}{\pgfqpoint{-0.027778in}{0.000000in}}%
\pgfpathcurveto{\pgfqpoint{-0.027778in}{-0.007367in}}{\pgfqpoint{-0.024851in}{-0.014433in}}{\pgfqpoint{-0.019642in}{-0.019642in}}%
\pgfpathcurveto{\pgfqpoint{-0.014433in}{-0.024851in}}{\pgfqpoint{-0.007367in}{-0.027778in}}{\pgfqpoint{0.000000in}{-0.027778in}}%
\pgfpathclose%
\pgfusepath{stroke,fill}%
}%
\begin{pgfscope}%
\pgfsys@transformshift{0.375033in}{4.567443in}%
\pgfsys@useobject{currentmarker}{}%
\end{pgfscope}%
\end{pgfscope}%
\begin{pgfscope}%
\pgfpathrectangle{\pgfqpoint{0.100000in}{2.413063in}}{\pgfqpoint{5.037500in}{3.427208in}}%
\pgfusepath{clip}%
\pgfsetrectcap%
\pgfsetroundjoin%
\pgfsetlinewidth{1.505625pt}%
\definecolor{currentstroke}{rgb}{0.000000,0.000000,1.000000}%
\pgfsetstrokecolor{currentstroke}%
\pgfsetstrokeopacity{0.500000}%
\pgfsetdash{}{0pt}%
\pgfpathmoveto{\pgfqpoint{0.520176in}{4.032959in}}%
\pgfusepath{stroke}%
\end{pgfscope}%
\begin{pgfscope}%
\pgfpathrectangle{\pgfqpoint{0.100000in}{2.413063in}}{\pgfqpoint{5.037500in}{3.427208in}}%
\pgfusepath{clip}%
\pgfsetbuttcap%
\pgfsetroundjoin%
\definecolor{currentfill}{rgb}{0.000000,0.000000,1.000000}%
\pgfsetfillcolor{currentfill}%
\pgfsetfillopacity{0.500000}%
\pgfsetlinewidth{0.250937pt}%
\definecolor{currentstroke}{rgb}{0.000000,0.000000,0.000000}%
\pgfsetstrokecolor{currentstroke}%
\pgfsetstrokeopacity{0.500000}%
\pgfsetdash{}{0pt}%
\pgfsys@defobject{currentmarker}{\pgfqpoint{-0.016667in}{-0.016667in}}{\pgfqpoint{0.016667in}{0.016667in}}{%
\pgfpathmoveto{\pgfqpoint{0.000000in}{-0.016667in}}%
\pgfpathcurveto{\pgfqpoint{0.004420in}{-0.016667in}}{\pgfqpoint{0.008660in}{-0.014911in}}{\pgfqpoint{0.011785in}{-0.011785in}}%
\pgfpathcurveto{\pgfqpoint{0.014911in}{-0.008660in}}{\pgfqpoint{0.016667in}{-0.004420in}}{\pgfqpoint{0.016667in}{0.000000in}}%
\pgfpathcurveto{\pgfqpoint{0.016667in}{0.004420in}}{\pgfqpoint{0.014911in}{0.008660in}}{\pgfqpoint{0.011785in}{0.011785in}}%
\pgfpathcurveto{\pgfqpoint{0.008660in}{0.014911in}}{\pgfqpoint{0.004420in}{0.016667in}}{\pgfqpoint{0.000000in}{0.016667in}}%
\pgfpathcurveto{\pgfqpoint{-0.004420in}{0.016667in}}{\pgfqpoint{-0.008660in}{0.014911in}}{\pgfqpoint{-0.011785in}{0.011785in}}%
\pgfpathcurveto{\pgfqpoint{-0.014911in}{0.008660in}}{\pgfqpoint{-0.016667in}{0.004420in}}{\pgfqpoint{-0.016667in}{0.000000in}}%
\pgfpathcurveto{\pgfqpoint{-0.016667in}{-0.004420in}}{\pgfqpoint{-0.014911in}{-0.008660in}}{\pgfqpoint{-0.011785in}{-0.011785in}}%
\pgfpathcurveto{\pgfqpoint{-0.008660in}{-0.014911in}}{\pgfqpoint{-0.004420in}{-0.016667in}}{\pgfqpoint{0.000000in}{-0.016667in}}%
\pgfpathclose%
\pgfusepath{stroke,fill}%
}%
\begin{pgfscope}%
\pgfsys@transformshift{0.520176in}{4.032959in}%
\pgfsys@useobject{currentmarker}{}%
\end{pgfscope}%
\end{pgfscope}%
\begin{pgfscope}%
\pgfpathrectangle{\pgfqpoint{0.100000in}{2.413063in}}{\pgfqpoint{5.037500in}{3.427208in}}%
\pgfusepath{clip}%
\pgfsetrectcap%
\pgfsetroundjoin%
\pgfsetlinewidth{1.505625pt}%
\definecolor{currentstroke}{rgb}{0.000000,0.000000,1.000000}%
\pgfsetstrokecolor{currentstroke}%
\pgfsetstrokeopacity{0.500000}%
\pgfsetdash{}{0pt}%
\pgfpathmoveto{\pgfqpoint{0.444132in}{4.822021in}}%
\pgfusepath{stroke}%
\end{pgfscope}%
\begin{pgfscope}%
\pgfpathrectangle{\pgfqpoint{0.100000in}{2.413063in}}{\pgfqpoint{5.037500in}{3.427208in}}%
\pgfusepath{clip}%
\pgfsetbuttcap%
\pgfsetroundjoin%
\definecolor{currentfill}{rgb}{0.000000,0.000000,1.000000}%
\pgfsetfillcolor{currentfill}%
\pgfsetfillopacity{0.500000}%
\pgfsetlinewidth{0.250937pt}%
\definecolor{currentstroke}{rgb}{0.000000,0.000000,0.000000}%
\pgfsetstrokecolor{currentstroke}%
\pgfsetstrokeopacity{0.500000}%
\pgfsetdash{}{0pt}%
\pgfsys@defobject{currentmarker}{\pgfqpoint{-0.008333in}{-0.008333in}}{\pgfqpoint{0.008333in}{0.008333in}}{%
\pgfpathmoveto{\pgfqpoint{0.000000in}{-0.008333in}}%
\pgfpathcurveto{\pgfqpoint{0.002210in}{-0.008333in}}{\pgfqpoint{0.004330in}{-0.007455in}}{\pgfqpoint{0.005893in}{-0.005893in}}%
\pgfpathcurveto{\pgfqpoint{0.007455in}{-0.004330in}}{\pgfqpoint{0.008333in}{-0.002210in}}{\pgfqpoint{0.008333in}{0.000000in}}%
\pgfpathcurveto{\pgfqpoint{0.008333in}{0.002210in}}{\pgfqpoint{0.007455in}{0.004330in}}{\pgfqpoint{0.005893in}{0.005893in}}%
\pgfpathcurveto{\pgfqpoint{0.004330in}{0.007455in}}{\pgfqpoint{0.002210in}{0.008333in}}{\pgfqpoint{0.000000in}{0.008333in}}%
\pgfpathcurveto{\pgfqpoint{-0.002210in}{0.008333in}}{\pgfqpoint{-0.004330in}{0.007455in}}{\pgfqpoint{-0.005893in}{0.005893in}}%
\pgfpathcurveto{\pgfqpoint{-0.007455in}{0.004330in}}{\pgfqpoint{-0.008333in}{0.002210in}}{\pgfqpoint{-0.008333in}{0.000000in}}%
\pgfpathcurveto{\pgfqpoint{-0.008333in}{-0.002210in}}{\pgfqpoint{-0.007455in}{-0.004330in}}{\pgfqpoint{-0.005893in}{-0.005893in}}%
\pgfpathcurveto{\pgfqpoint{-0.004330in}{-0.007455in}}{\pgfqpoint{-0.002210in}{-0.008333in}}{\pgfqpoint{0.000000in}{-0.008333in}}%
\pgfpathclose%
\pgfusepath{stroke,fill}%
}%
\begin{pgfscope}%
\pgfsys@transformshift{0.444132in}{4.822021in}%
\pgfsys@useobject{currentmarker}{}%
\end{pgfscope}%
\end{pgfscope}%
\begin{pgfscope}%
\pgfpathrectangle{\pgfqpoint{0.100000in}{2.413063in}}{\pgfqpoint{5.037500in}{3.427208in}}%
\pgfusepath{clip}%
\pgfsetrectcap%
\pgfsetroundjoin%
\pgfsetlinewidth{1.505625pt}%
\definecolor{currentstroke}{rgb}{0.000000,0.000000,1.000000}%
\pgfsetstrokecolor{currentstroke}%
\pgfsetstrokeopacity{0.500000}%
\pgfsetdash{}{0pt}%
\pgfpathmoveto{\pgfqpoint{0.678250in}{3.962355in}}%
\pgfusepath{stroke}%
\end{pgfscope}%
\begin{pgfscope}%
\pgfpathrectangle{\pgfqpoint{0.100000in}{2.413063in}}{\pgfqpoint{5.037500in}{3.427208in}}%
\pgfusepath{clip}%
\pgfsetbuttcap%
\pgfsetroundjoin%
\definecolor{currentfill}{rgb}{0.000000,0.000000,1.000000}%
\pgfsetfillcolor{currentfill}%
\pgfsetfillopacity{0.500000}%
\pgfsetlinewidth{0.250937pt}%
\definecolor{currentstroke}{rgb}{0.000000,0.000000,0.000000}%
\pgfsetstrokecolor{currentstroke}%
\pgfsetstrokeopacity{0.500000}%
\pgfsetdash{}{0pt}%
\pgfsys@defobject{currentmarker}{\pgfqpoint{-0.038889in}{-0.038889in}}{\pgfqpoint{0.038889in}{0.038889in}}{%
\pgfpathmoveto{\pgfqpoint{0.000000in}{-0.038889in}}%
\pgfpathcurveto{\pgfqpoint{0.010313in}{-0.038889in}}{\pgfqpoint{0.020206in}{-0.034791in}}{\pgfqpoint{0.027499in}{-0.027499in}}%
\pgfpathcurveto{\pgfqpoint{0.034791in}{-0.020206in}}{\pgfqpoint{0.038889in}{-0.010313in}}{\pgfqpoint{0.038889in}{0.000000in}}%
\pgfpathcurveto{\pgfqpoint{0.038889in}{0.010313in}}{\pgfqpoint{0.034791in}{0.020206in}}{\pgfqpoint{0.027499in}{0.027499in}}%
\pgfpathcurveto{\pgfqpoint{0.020206in}{0.034791in}}{\pgfqpoint{0.010313in}{0.038889in}}{\pgfqpoint{0.000000in}{0.038889in}}%
\pgfpathcurveto{\pgfqpoint{-0.010313in}{0.038889in}}{\pgfqpoint{-0.020206in}{0.034791in}}{\pgfqpoint{-0.027499in}{0.027499in}}%
\pgfpathcurveto{\pgfqpoint{-0.034791in}{0.020206in}}{\pgfqpoint{-0.038889in}{0.010313in}}{\pgfqpoint{-0.038889in}{0.000000in}}%
\pgfpathcurveto{\pgfqpoint{-0.038889in}{-0.010313in}}{\pgfqpoint{-0.034791in}{-0.020206in}}{\pgfqpoint{-0.027499in}{-0.027499in}}%
\pgfpathcurveto{\pgfqpoint{-0.020206in}{-0.034791in}}{\pgfqpoint{-0.010313in}{-0.038889in}}{\pgfqpoint{0.000000in}{-0.038889in}}%
\pgfpathclose%
\pgfusepath{stroke,fill}%
}%
\begin{pgfscope}%
\pgfsys@transformshift{0.678250in}{3.962355in}%
\pgfsys@useobject{currentmarker}{}%
\end{pgfscope}%
\end{pgfscope}%
\begin{pgfscope}%
\pgfpathrectangle{\pgfqpoint{0.100000in}{2.413063in}}{\pgfqpoint{5.037500in}{3.427208in}}%
\pgfusepath{clip}%
\pgfsetrectcap%
\pgfsetroundjoin%
\pgfsetlinewidth{1.505625pt}%
\definecolor{currentstroke}{rgb}{0.000000,0.000000,1.000000}%
\pgfsetstrokecolor{currentstroke}%
\pgfsetstrokeopacity{0.500000}%
\pgfsetdash{}{0pt}%
\pgfpathmoveto{\pgfqpoint{0.452839in}{4.577966in}}%
\pgfusepath{stroke}%
\end{pgfscope}%
\begin{pgfscope}%
\pgfpathrectangle{\pgfqpoint{0.100000in}{2.413063in}}{\pgfqpoint{5.037500in}{3.427208in}}%
\pgfusepath{clip}%
\pgfsetbuttcap%
\pgfsetroundjoin%
\definecolor{currentfill}{rgb}{0.000000,0.000000,1.000000}%
\pgfsetfillcolor{currentfill}%
\pgfsetfillopacity{0.500000}%
\pgfsetlinewidth{0.250937pt}%
\definecolor{currentstroke}{rgb}{0.000000,0.000000,0.000000}%
\pgfsetstrokecolor{currentstroke}%
\pgfsetstrokeopacity{0.500000}%
\pgfsetdash{}{0pt}%
\pgfsys@defobject{currentmarker}{\pgfqpoint{-0.027778in}{-0.027778in}}{\pgfqpoint{0.027778in}{0.027778in}}{%
\pgfpathmoveto{\pgfqpoint{0.000000in}{-0.027778in}}%
\pgfpathcurveto{\pgfqpoint{0.007367in}{-0.027778in}}{\pgfqpoint{0.014433in}{-0.024851in}}{\pgfqpoint{0.019642in}{-0.019642in}}%
\pgfpathcurveto{\pgfqpoint{0.024851in}{-0.014433in}}{\pgfqpoint{0.027778in}{-0.007367in}}{\pgfqpoint{0.027778in}{0.000000in}}%
\pgfpathcurveto{\pgfqpoint{0.027778in}{0.007367in}}{\pgfqpoint{0.024851in}{0.014433in}}{\pgfqpoint{0.019642in}{0.019642in}}%
\pgfpathcurveto{\pgfqpoint{0.014433in}{0.024851in}}{\pgfqpoint{0.007367in}{0.027778in}}{\pgfqpoint{0.000000in}{0.027778in}}%
\pgfpathcurveto{\pgfqpoint{-0.007367in}{0.027778in}}{\pgfqpoint{-0.014433in}{0.024851in}}{\pgfqpoint{-0.019642in}{0.019642in}}%
\pgfpathcurveto{\pgfqpoint{-0.024851in}{0.014433in}}{\pgfqpoint{-0.027778in}{0.007367in}}{\pgfqpoint{-0.027778in}{0.000000in}}%
\pgfpathcurveto{\pgfqpoint{-0.027778in}{-0.007367in}}{\pgfqpoint{-0.024851in}{-0.014433in}}{\pgfqpoint{-0.019642in}{-0.019642in}}%
\pgfpathcurveto{\pgfqpoint{-0.014433in}{-0.024851in}}{\pgfqpoint{-0.007367in}{-0.027778in}}{\pgfqpoint{0.000000in}{-0.027778in}}%
\pgfpathclose%
\pgfusepath{stroke,fill}%
}%
\begin{pgfscope}%
\pgfsys@transformshift{0.452839in}{4.577966in}%
\pgfsys@useobject{currentmarker}{}%
\end{pgfscope}%
\end{pgfscope}%
\begin{pgfscope}%
\pgfpathrectangle{\pgfqpoint{0.100000in}{2.413063in}}{\pgfqpoint{5.037500in}{3.427208in}}%
\pgfusepath{clip}%
\pgfsetrectcap%
\pgfsetroundjoin%
\pgfsetlinewidth{1.505625pt}%
\definecolor{currentstroke}{rgb}{0.678431,1.000000,0.184314}%
\pgfsetstrokecolor{currentstroke}%
\pgfsetstrokeopacity{0.500000}%
\pgfsetdash{}{0pt}%
\pgfpathmoveto{\pgfqpoint{0.375477in}{4.371832in}}%
\pgfusepath{stroke}%
\end{pgfscope}%
\begin{pgfscope}%
\pgfpathrectangle{\pgfqpoint{0.100000in}{2.413063in}}{\pgfqpoint{5.037500in}{3.427208in}}%
\pgfusepath{clip}%
\pgfsetbuttcap%
\pgfsetroundjoin%
\definecolor{currentfill}{rgb}{0.678431,1.000000,0.184314}%
\pgfsetfillcolor{currentfill}%
\pgfsetfillopacity{0.500000}%
\pgfsetlinewidth{0.250937pt}%
\definecolor{currentstroke}{rgb}{0.000000,0.000000,0.000000}%
\pgfsetstrokecolor{currentstroke}%
\pgfsetstrokeopacity{0.500000}%
\pgfsetdash{}{0pt}%
\pgfsys@defobject{currentmarker}{\pgfqpoint{-0.019444in}{-0.019444in}}{\pgfqpoint{0.019444in}{0.019444in}}{%
\pgfpathmoveto{\pgfqpoint{0.000000in}{-0.019444in}}%
\pgfpathcurveto{\pgfqpoint{0.005157in}{-0.019444in}}{\pgfqpoint{0.010103in}{-0.017396in}}{\pgfqpoint{0.013749in}{-0.013749in}}%
\pgfpathcurveto{\pgfqpoint{0.017396in}{-0.010103in}}{\pgfqpoint{0.019444in}{-0.005157in}}{\pgfqpoint{0.019444in}{0.000000in}}%
\pgfpathcurveto{\pgfqpoint{0.019444in}{0.005157in}}{\pgfqpoint{0.017396in}{0.010103in}}{\pgfqpoint{0.013749in}{0.013749in}}%
\pgfpathcurveto{\pgfqpoint{0.010103in}{0.017396in}}{\pgfqpoint{0.005157in}{0.019444in}}{\pgfqpoint{0.000000in}{0.019444in}}%
\pgfpathcurveto{\pgfqpoint{-0.005157in}{0.019444in}}{\pgfqpoint{-0.010103in}{0.017396in}}{\pgfqpoint{-0.013749in}{0.013749in}}%
\pgfpathcurveto{\pgfqpoint{-0.017396in}{0.010103in}}{\pgfqpoint{-0.019444in}{0.005157in}}{\pgfqpoint{-0.019444in}{0.000000in}}%
\pgfpathcurveto{\pgfqpoint{-0.019444in}{-0.005157in}}{\pgfqpoint{-0.017396in}{-0.010103in}}{\pgfqpoint{-0.013749in}{-0.013749in}}%
\pgfpathcurveto{\pgfqpoint{-0.010103in}{-0.017396in}}{\pgfqpoint{-0.005157in}{-0.019444in}}{\pgfqpoint{0.000000in}{-0.019444in}}%
\pgfpathclose%
\pgfusepath{stroke,fill}%
}%
\begin{pgfscope}%
\pgfsys@transformshift{0.375477in}{4.371832in}%
\pgfsys@useobject{currentmarker}{}%
\end{pgfscope}%
\end{pgfscope}%
\begin{pgfscope}%
\pgfpathrectangle{\pgfqpoint{0.100000in}{2.413063in}}{\pgfqpoint{5.037500in}{3.427208in}}%
\pgfusepath{clip}%
\pgfsetrectcap%
\pgfsetroundjoin%
\pgfsetlinewidth{1.505625pt}%
\definecolor{currentstroke}{rgb}{0.000000,0.000000,1.000000}%
\pgfsetstrokecolor{currentstroke}%
\pgfsetstrokeopacity{0.500000}%
\pgfsetdash{}{0pt}%
\pgfpathmoveto{\pgfqpoint{0.665426in}{3.818137in}}%
\pgfusepath{stroke}%
\end{pgfscope}%
\begin{pgfscope}%
\pgfpathrectangle{\pgfqpoint{0.100000in}{2.413063in}}{\pgfqpoint{5.037500in}{3.427208in}}%
\pgfusepath{clip}%
\pgfsetbuttcap%
\pgfsetroundjoin%
\definecolor{currentfill}{rgb}{0.000000,0.000000,1.000000}%
\pgfsetfillcolor{currentfill}%
\pgfsetfillopacity{0.500000}%
\pgfsetlinewidth{0.250937pt}%
\definecolor{currentstroke}{rgb}{0.000000,0.000000,0.000000}%
\pgfsetstrokecolor{currentstroke}%
\pgfsetstrokeopacity{0.500000}%
\pgfsetdash{}{0pt}%
\pgfsys@defobject{currentmarker}{\pgfqpoint{-0.033333in}{-0.033333in}}{\pgfqpoint{0.033333in}{0.033333in}}{%
\pgfpathmoveto{\pgfqpoint{0.000000in}{-0.033333in}}%
\pgfpathcurveto{\pgfqpoint{0.008840in}{-0.033333in}}{\pgfqpoint{0.017319in}{-0.029821in}}{\pgfqpoint{0.023570in}{-0.023570in}}%
\pgfpathcurveto{\pgfqpoint{0.029821in}{-0.017319in}}{\pgfqpoint{0.033333in}{-0.008840in}}{\pgfqpoint{0.033333in}{0.000000in}}%
\pgfpathcurveto{\pgfqpoint{0.033333in}{0.008840in}}{\pgfqpoint{0.029821in}{0.017319in}}{\pgfqpoint{0.023570in}{0.023570in}}%
\pgfpathcurveto{\pgfqpoint{0.017319in}{0.029821in}}{\pgfqpoint{0.008840in}{0.033333in}}{\pgfqpoint{0.000000in}{0.033333in}}%
\pgfpathcurveto{\pgfqpoint{-0.008840in}{0.033333in}}{\pgfqpoint{-0.017319in}{0.029821in}}{\pgfqpoint{-0.023570in}{0.023570in}}%
\pgfpathcurveto{\pgfqpoint{-0.029821in}{0.017319in}}{\pgfqpoint{-0.033333in}{0.008840in}}{\pgfqpoint{-0.033333in}{0.000000in}}%
\pgfpathcurveto{\pgfqpoint{-0.033333in}{-0.008840in}}{\pgfqpoint{-0.029821in}{-0.017319in}}{\pgfqpoint{-0.023570in}{-0.023570in}}%
\pgfpathcurveto{\pgfqpoint{-0.017319in}{-0.029821in}}{\pgfqpoint{-0.008840in}{-0.033333in}}{\pgfqpoint{0.000000in}{-0.033333in}}%
\pgfpathclose%
\pgfusepath{stroke,fill}%
}%
\begin{pgfscope}%
\pgfsys@transformshift{0.665426in}{3.818137in}%
\pgfsys@useobject{currentmarker}{}%
\end{pgfscope}%
\end{pgfscope}%
\begin{pgfscope}%
\pgfpathrectangle{\pgfqpoint{0.100000in}{2.413063in}}{\pgfqpoint{5.037500in}{3.427208in}}%
\pgfusepath{clip}%
\pgfsetrectcap%
\pgfsetroundjoin%
\pgfsetlinewidth{1.505625pt}%
\definecolor{currentstroke}{rgb}{0.000000,0.000000,1.000000}%
\pgfsetstrokecolor{currentstroke}%
\pgfsetstrokeopacity{0.500000}%
\pgfsetdash{}{0pt}%
\pgfpathmoveto{\pgfqpoint{0.345769in}{4.514080in}}%
\pgfusepath{stroke}%
\end{pgfscope}%
\begin{pgfscope}%
\pgfpathrectangle{\pgfqpoint{0.100000in}{2.413063in}}{\pgfqpoint{5.037500in}{3.427208in}}%
\pgfusepath{clip}%
\pgfsetbuttcap%
\pgfsetroundjoin%
\definecolor{currentfill}{rgb}{0.000000,0.000000,1.000000}%
\pgfsetfillcolor{currentfill}%
\pgfsetfillopacity{0.500000}%
\pgfsetlinewidth{0.250937pt}%
\definecolor{currentstroke}{rgb}{0.000000,0.000000,0.000000}%
\pgfsetstrokecolor{currentstroke}%
\pgfsetstrokeopacity{0.500000}%
\pgfsetdash{}{0pt}%
\pgfsys@defobject{currentmarker}{\pgfqpoint{-0.030556in}{-0.030556in}}{\pgfqpoint{0.030556in}{0.030556in}}{%
\pgfpathmoveto{\pgfqpoint{0.000000in}{-0.030556in}}%
\pgfpathcurveto{\pgfqpoint{0.008103in}{-0.030556in}}{\pgfqpoint{0.015876in}{-0.027336in}}{\pgfqpoint{0.021606in}{-0.021606in}}%
\pgfpathcurveto{\pgfqpoint{0.027336in}{-0.015876in}}{\pgfqpoint{0.030556in}{-0.008103in}}{\pgfqpoint{0.030556in}{0.000000in}}%
\pgfpathcurveto{\pgfqpoint{0.030556in}{0.008103in}}{\pgfqpoint{0.027336in}{0.015876in}}{\pgfqpoint{0.021606in}{0.021606in}}%
\pgfpathcurveto{\pgfqpoint{0.015876in}{0.027336in}}{\pgfqpoint{0.008103in}{0.030556in}}{\pgfqpoint{0.000000in}{0.030556in}}%
\pgfpathcurveto{\pgfqpoint{-0.008103in}{0.030556in}}{\pgfqpoint{-0.015876in}{0.027336in}}{\pgfqpoint{-0.021606in}{0.021606in}}%
\pgfpathcurveto{\pgfqpoint{-0.027336in}{0.015876in}}{\pgfqpoint{-0.030556in}{0.008103in}}{\pgfqpoint{-0.030556in}{0.000000in}}%
\pgfpathcurveto{\pgfqpoint{-0.030556in}{-0.008103in}}{\pgfqpoint{-0.027336in}{-0.015876in}}{\pgfqpoint{-0.021606in}{-0.021606in}}%
\pgfpathcurveto{\pgfqpoint{-0.015876in}{-0.027336in}}{\pgfqpoint{-0.008103in}{-0.030556in}}{\pgfqpoint{0.000000in}{-0.030556in}}%
\pgfpathclose%
\pgfusepath{stroke,fill}%
}%
\begin{pgfscope}%
\pgfsys@transformshift{0.345769in}{4.514080in}%
\pgfsys@useobject{currentmarker}{}%
\end{pgfscope}%
\end{pgfscope}%
\begin{pgfscope}%
\pgfpathrectangle{\pgfqpoint{0.100000in}{2.413063in}}{\pgfqpoint{5.037500in}{3.427208in}}%
\pgfusepath{clip}%
\pgfsetrectcap%
\pgfsetroundjoin%
\pgfsetlinewidth{1.505625pt}%
\definecolor{currentstroke}{rgb}{0.000000,0.000000,1.000000}%
\pgfsetstrokecolor{currentstroke}%
\pgfsetstrokeopacity{0.500000}%
\pgfsetdash{}{0pt}%
\pgfpathmoveto{\pgfqpoint{0.376878in}{4.451092in}}%
\pgfusepath{stroke}%
\end{pgfscope}%
\begin{pgfscope}%
\pgfpathrectangle{\pgfqpoint{0.100000in}{2.413063in}}{\pgfqpoint{5.037500in}{3.427208in}}%
\pgfusepath{clip}%
\pgfsetbuttcap%
\pgfsetroundjoin%
\definecolor{currentfill}{rgb}{0.000000,0.000000,1.000000}%
\pgfsetfillcolor{currentfill}%
\pgfsetfillopacity{0.500000}%
\pgfsetlinewidth{0.250937pt}%
\definecolor{currentstroke}{rgb}{0.000000,0.000000,0.000000}%
\pgfsetstrokecolor{currentstroke}%
\pgfsetstrokeopacity{0.500000}%
\pgfsetdash{}{0pt}%
\pgfsys@defobject{currentmarker}{\pgfqpoint{-0.016667in}{-0.016667in}}{\pgfqpoint{0.016667in}{0.016667in}}{%
\pgfpathmoveto{\pgfqpoint{0.000000in}{-0.016667in}}%
\pgfpathcurveto{\pgfqpoint{0.004420in}{-0.016667in}}{\pgfqpoint{0.008660in}{-0.014911in}}{\pgfqpoint{0.011785in}{-0.011785in}}%
\pgfpathcurveto{\pgfqpoint{0.014911in}{-0.008660in}}{\pgfqpoint{0.016667in}{-0.004420in}}{\pgfqpoint{0.016667in}{0.000000in}}%
\pgfpathcurveto{\pgfqpoint{0.016667in}{0.004420in}}{\pgfqpoint{0.014911in}{0.008660in}}{\pgfqpoint{0.011785in}{0.011785in}}%
\pgfpathcurveto{\pgfqpoint{0.008660in}{0.014911in}}{\pgfqpoint{0.004420in}{0.016667in}}{\pgfqpoint{0.000000in}{0.016667in}}%
\pgfpathcurveto{\pgfqpoint{-0.004420in}{0.016667in}}{\pgfqpoint{-0.008660in}{0.014911in}}{\pgfqpoint{-0.011785in}{0.011785in}}%
\pgfpathcurveto{\pgfqpoint{-0.014911in}{0.008660in}}{\pgfqpoint{-0.016667in}{0.004420in}}{\pgfqpoint{-0.016667in}{0.000000in}}%
\pgfpathcurveto{\pgfqpoint{-0.016667in}{-0.004420in}}{\pgfqpoint{-0.014911in}{-0.008660in}}{\pgfqpoint{-0.011785in}{-0.011785in}}%
\pgfpathcurveto{\pgfqpoint{-0.008660in}{-0.014911in}}{\pgfqpoint{-0.004420in}{-0.016667in}}{\pgfqpoint{0.000000in}{-0.016667in}}%
\pgfpathclose%
\pgfusepath{stroke,fill}%
}%
\begin{pgfscope}%
\pgfsys@transformshift{0.376878in}{4.451092in}%
\pgfsys@useobject{currentmarker}{}%
\end{pgfscope}%
\end{pgfscope}%
\begin{pgfscope}%
\pgfpathrectangle{\pgfqpoint{0.100000in}{2.413063in}}{\pgfqpoint{5.037500in}{3.427208in}}%
\pgfusepath{clip}%
\pgfsetrectcap%
\pgfsetroundjoin%
\pgfsetlinewidth{1.505625pt}%
\definecolor{currentstroke}{rgb}{0.000000,0.000000,1.000000}%
\pgfsetstrokecolor{currentstroke}%
\pgfsetstrokeopacity{0.500000}%
\pgfsetdash{}{0pt}%
\pgfpathmoveto{\pgfqpoint{0.418989in}{4.191586in}}%
\pgfusepath{stroke}%
\end{pgfscope}%
\begin{pgfscope}%
\pgfpathrectangle{\pgfqpoint{0.100000in}{2.413063in}}{\pgfqpoint{5.037500in}{3.427208in}}%
\pgfusepath{clip}%
\pgfsetbuttcap%
\pgfsetroundjoin%
\definecolor{currentfill}{rgb}{0.000000,0.000000,1.000000}%
\pgfsetfillcolor{currentfill}%
\pgfsetfillopacity{0.500000}%
\pgfsetlinewidth{0.250937pt}%
\definecolor{currentstroke}{rgb}{0.000000,0.000000,0.000000}%
\pgfsetstrokecolor{currentstroke}%
\pgfsetstrokeopacity{0.500000}%
\pgfsetdash{}{0pt}%
\pgfsys@defobject{currentmarker}{\pgfqpoint{-0.019444in}{-0.019444in}}{\pgfqpoint{0.019444in}{0.019444in}}{%
\pgfpathmoveto{\pgfqpoint{0.000000in}{-0.019444in}}%
\pgfpathcurveto{\pgfqpoint{0.005157in}{-0.019444in}}{\pgfqpoint{0.010103in}{-0.017396in}}{\pgfqpoint{0.013749in}{-0.013749in}}%
\pgfpathcurveto{\pgfqpoint{0.017396in}{-0.010103in}}{\pgfqpoint{0.019444in}{-0.005157in}}{\pgfqpoint{0.019444in}{0.000000in}}%
\pgfpathcurveto{\pgfqpoint{0.019444in}{0.005157in}}{\pgfqpoint{0.017396in}{0.010103in}}{\pgfqpoint{0.013749in}{0.013749in}}%
\pgfpathcurveto{\pgfqpoint{0.010103in}{0.017396in}}{\pgfqpoint{0.005157in}{0.019444in}}{\pgfqpoint{0.000000in}{0.019444in}}%
\pgfpathcurveto{\pgfqpoint{-0.005157in}{0.019444in}}{\pgfqpoint{-0.010103in}{0.017396in}}{\pgfqpoint{-0.013749in}{0.013749in}}%
\pgfpathcurveto{\pgfqpoint{-0.017396in}{0.010103in}}{\pgfqpoint{-0.019444in}{0.005157in}}{\pgfqpoint{-0.019444in}{0.000000in}}%
\pgfpathcurveto{\pgfqpoint{-0.019444in}{-0.005157in}}{\pgfqpoint{-0.017396in}{-0.010103in}}{\pgfqpoint{-0.013749in}{-0.013749in}}%
\pgfpathcurveto{\pgfqpoint{-0.010103in}{-0.017396in}}{\pgfqpoint{-0.005157in}{-0.019444in}}{\pgfqpoint{0.000000in}{-0.019444in}}%
\pgfpathclose%
\pgfusepath{stroke,fill}%
}%
\begin{pgfscope}%
\pgfsys@transformshift{0.418989in}{4.191586in}%
\pgfsys@useobject{currentmarker}{}%
\end{pgfscope}%
\end{pgfscope}%
\begin{pgfscope}%
\pgfpathrectangle{\pgfqpoint{0.100000in}{2.413063in}}{\pgfqpoint{5.037500in}{3.427208in}}%
\pgfusepath{clip}%
\pgfsetrectcap%
\pgfsetroundjoin%
\pgfsetlinewidth{1.505625pt}%
\definecolor{currentstroke}{rgb}{0.000000,0.000000,1.000000}%
\pgfsetstrokecolor{currentstroke}%
\pgfsetstrokeopacity{0.500000}%
\pgfsetdash{}{0pt}%
\pgfpathmoveto{\pgfqpoint{0.352770in}{4.414777in}}%
\pgfusepath{stroke}%
\end{pgfscope}%
\begin{pgfscope}%
\pgfpathrectangle{\pgfqpoint{0.100000in}{2.413063in}}{\pgfqpoint{5.037500in}{3.427208in}}%
\pgfusepath{clip}%
\pgfsetbuttcap%
\pgfsetroundjoin%
\definecolor{currentfill}{rgb}{0.000000,0.000000,1.000000}%
\pgfsetfillcolor{currentfill}%
\pgfsetfillopacity{0.500000}%
\pgfsetlinewidth{0.250937pt}%
\definecolor{currentstroke}{rgb}{0.000000,0.000000,0.000000}%
\pgfsetstrokecolor{currentstroke}%
\pgfsetstrokeopacity{0.500000}%
\pgfsetdash{}{0pt}%
\pgfsys@defobject{currentmarker}{\pgfqpoint{-0.005556in}{-0.005556in}}{\pgfqpoint{0.005556in}{0.005556in}}{%
\pgfpathmoveto{\pgfqpoint{0.000000in}{-0.005556in}}%
\pgfpathcurveto{\pgfqpoint{0.001473in}{-0.005556in}}{\pgfqpoint{0.002887in}{-0.004970in}}{\pgfqpoint{0.003928in}{-0.003928in}}%
\pgfpathcurveto{\pgfqpoint{0.004970in}{-0.002887in}}{\pgfqpoint{0.005556in}{-0.001473in}}{\pgfqpoint{0.005556in}{0.000000in}}%
\pgfpathcurveto{\pgfqpoint{0.005556in}{0.001473in}}{\pgfqpoint{0.004970in}{0.002887in}}{\pgfqpoint{0.003928in}{0.003928in}}%
\pgfpathcurveto{\pgfqpoint{0.002887in}{0.004970in}}{\pgfqpoint{0.001473in}{0.005556in}}{\pgfqpoint{0.000000in}{0.005556in}}%
\pgfpathcurveto{\pgfqpoint{-0.001473in}{0.005556in}}{\pgfqpoint{-0.002887in}{0.004970in}}{\pgfqpoint{-0.003928in}{0.003928in}}%
\pgfpathcurveto{\pgfqpoint{-0.004970in}{0.002887in}}{\pgfqpoint{-0.005556in}{0.001473in}}{\pgfqpoint{-0.005556in}{0.000000in}}%
\pgfpathcurveto{\pgfqpoint{-0.005556in}{-0.001473in}}{\pgfqpoint{-0.004970in}{-0.002887in}}{\pgfqpoint{-0.003928in}{-0.003928in}}%
\pgfpathcurveto{\pgfqpoint{-0.002887in}{-0.004970in}}{\pgfqpoint{-0.001473in}{-0.005556in}}{\pgfqpoint{0.000000in}{-0.005556in}}%
\pgfpathclose%
\pgfusepath{stroke,fill}%
}%
\begin{pgfscope}%
\pgfsys@transformshift{0.352770in}{4.414777in}%
\pgfsys@useobject{currentmarker}{}%
\end{pgfscope}%
\end{pgfscope}%
\begin{pgfscope}%
\pgfpathrectangle{\pgfqpoint{0.100000in}{2.413063in}}{\pgfqpoint{5.037500in}{3.427208in}}%
\pgfusepath{clip}%
\pgfsetrectcap%
\pgfsetroundjoin%
\pgfsetlinewidth{1.505625pt}%
\definecolor{currentstroke}{rgb}{0.000000,0.000000,1.000000}%
\pgfsetstrokecolor{currentstroke}%
\pgfsetstrokeopacity{0.500000}%
\pgfsetdash{}{0pt}%
\pgfpathmoveto{\pgfqpoint{0.428730in}{4.149174in}}%
\pgfusepath{stroke}%
\end{pgfscope}%
\begin{pgfscope}%
\pgfpathrectangle{\pgfqpoint{0.100000in}{2.413063in}}{\pgfqpoint{5.037500in}{3.427208in}}%
\pgfusepath{clip}%
\pgfsetbuttcap%
\pgfsetroundjoin%
\definecolor{currentfill}{rgb}{0.000000,0.000000,1.000000}%
\pgfsetfillcolor{currentfill}%
\pgfsetfillopacity{0.500000}%
\pgfsetlinewidth{0.250937pt}%
\definecolor{currentstroke}{rgb}{0.000000,0.000000,0.000000}%
\pgfsetstrokecolor{currentstroke}%
\pgfsetstrokeopacity{0.500000}%
\pgfsetdash{}{0pt}%
\pgfsys@defobject{currentmarker}{\pgfqpoint{-0.008333in}{-0.008333in}}{\pgfqpoint{0.008333in}{0.008333in}}{%
\pgfpathmoveto{\pgfqpoint{0.000000in}{-0.008333in}}%
\pgfpathcurveto{\pgfqpoint{0.002210in}{-0.008333in}}{\pgfqpoint{0.004330in}{-0.007455in}}{\pgfqpoint{0.005893in}{-0.005893in}}%
\pgfpathcurveto{\pgfqpoint{0.007455in}{-0.004330in}}{\pgfqpoint{0.008333in}{-0.002210in}}{\pgfqpoint{0.008333in}{0.000000in}}%
\pgfpathcurveto{\pgfqpoint{0.008333in}{0.002210in}}{\pgfqpoint{0.007455in}{0.004330in}}{\pgfqpoint{0.005893in}{0.005893in}}%
\pgfpathcurveto{\pgfqpoint{0.004330in}{0.007455in}}{\pgfqpoint{0.002210in}{0.008333in}}{\pgfqpoint{0.000000in}{0.008333in}}%
\pgfpathcurveto{\pgfqpoint{-0.002210in}{0.008333in}}{\pgfqpoint{-0.004330in}{0.007455in}}{\pgfqpoint{-0.005893in}{0.005893in}}%
\pgfpathcurveto{\pgfqpoint{-0.007455in}{0.004330in}}{\pgfqpoint{-0.008333in}{0.002210in}}{\pgfqpoint{-0.008333in}{0.000000in}}%
\pgfpathcurveto{\pgfqpoint{-0.008333in}{-0.002210in}}{\pgfqpoint{-0.007455in}{-0.004330in}}{\pgfqpoint{-0.005893in}{-0.005893in}}%
\pgfpathcurveto{\pgfqpoint{-0.004330in}{-0.007455in}}{\pgfqpoint{-0.002210in}{-0.008333in}}{\pgfqpoint{0.000000in}{-0.008333in}}%
\pgfpathclose%
\pgfusepath{stroke,fill}%
}%
\begin{pgfscope}%
\pgfsys@transformshift{0.428730in}{4.149174in}%
\pgfsys@useobject{currentmarker}{}%
\end{pgfscope}%
\end{pgfscope}%
\begin{pgfscope}%
\pgfpathrectangle{\pgfqpoint{0.100000in}{2.413063in}}{\pgfqpoint{5.037500in}{3.427208in}}%
\pgfusepath{clip}%
\pgfsetrectcap%
\pgfsetroundjoin%
\pgfsetlinewidth{1.505625pt}%
\definecolor{currentstroke}{rgb}{0.000000,0.000000,1.000000}%
\pgfsetstrokecolor{currentstroke}%
\pgfsetstrokeopacity{0.500000}%
\pgfsetdash{}{0pt}%
\pgfpathmoveto{\pgfqpoint{0.342987in}{4.594739in}}%
\pgfusepath{stroke}%
\end{pgfscope}%
\begin{pgfscope}%
\pgfpathrectangle{\pgfqpoint{0.100000in}{2.413063in}}{\pgfqpoint{5.037500in}{3.427208in}}%
\pgfusepath{clip}%
\pgfsetbuttcap%
\pgfsetroundjoin%
\definecolor{currentfill}{rgb}{0.000000,0.000000,1.000000}%
\pgfsetfillcolor{currentfill}%
\pgfsetfillopacity{0.500000}%
\pgfsetlinewidth{0.250937pt}%
\definecolor{currentstroke}{rgb}{0.000000,0.000000,0.000000}%
\pgfsetstrokecolor{currentstroke}%
\pgfsetstrokeopacity{0.500000}%
\pgfsetdash{}{0pt}%
\pgfsys@defobject{currentmarker}{\pgfqpoint{-0.025000in}{-0.025000in}}{\pgfqpoint{0.025000in}{0.025000in}}{%
\pgfpathmoveto{\pgfqpoint{0.000000in}{-0.025000in}}%
\pgfpathcurveto{\pgfqpoint{0.006630in}{-0.025000in}}{\pgfqpoint{0.012989in}{-0.022366in}}{\pgfqpoint{0.017678in}{-0.017678in}}%
\pgfpathcurveto{\pgfqpoint{0.022366in}{-0.012989in}}{\pgfqpoint{0.025000in}{-0.006630in}}{\pgfqpoint{0.025000in}{0.000000in}}%
\pgfpathcurveto{\pgfqpoint{0.025000in}{0.006630in}}{\pgfqpoint{0.022366in}{0.012989in}}{\pgfqpoint{0.017678in}{0.017678in}}%
\pgfpathcurveto{\pgfqpoint{0.012989in}{0.022366in}}{\pgfqpoint{0.006630in}{0.025000in}}{\pgfqpoint{0.000000in}{0.025000in}}%
\pgfpathcurveto{\pgfqpoint{-0.006630in}{0.025000in}}{\pgfqpoint{-0.012989in}{0.022366in}}{\pgfqpoint{-0.017678in}{0.017678in}}%
\pgfpathcurveto{\pgfqpoint{-0.022366in}{0.012989in}}{\pgfqpoint{-0.025000in}{0.006630in}}{\pgfqpoint{-0.025000in}{0.000000in}}%
\pgfpathcurveto{\pgfqpoint{-0.025000in}{-0.006630in}}{\pgfqpoint{-0.022366in}{-0.012989in}}{\pgfqpoint{-0.017678in}{-0.017678in}}%
\pgfpathcurveto{\pgfqpoint{-0.012989in}{-0.022366in}}{\pgfqpoint{-0.006630in}{-0.025000in}}{\pgfqpoint{0.000000in}{-0.025000in}}%
\pgfpathclose%
\pgfusepath{stroke,fill}%
}%
\begin{pgfscope}%
\pgfsys@transformshift{0.342987in}{4.594739in}%
\pgfsys@useobject{currentmarker}{}%
\end{pgfscope}%
\end{pgfscope}%
\begin{pgfscope}%
\pgfpathrectangle{\pgfqpoint{0.100000in}{2.413063in}}{\pgfqpoint{5.037500in}{3.427208in}}%
\pgfusepath{clip}%
\pgfsetrectcap%
\pgfsetroundjoin%
\pgfsetlinewidth{1.505625pt}%
\definecolor{currentstroke}{rgb}{0.000000,0.000000,1.000000}%
\pgfsetstrokecolor{currentstroke}%
\pgfsetstrokeopacity{0.500000}%
\pgfsetdash{}{0pt}%
\pgfpathmoveto{\pgfqpoint{0.449905in}{4.503858in}}%
\pgfusepath{stroke}%
\end{pgfscope}%
\begin{pgfscope}%
\pgfpathrectangle{\pgfqpoint{0.100000in}{2.413063in}}{\pgfqpoint{5.037500in}{3.427208in}}%
\pgfusepath{clip}%
\pgfsetbuttcap%
\pgfsetroundjoin%
\definecolor{currentfill}{rgb}{0.000000,0.000000,1.000000}%
\pgfsetfillcolor{currentfill}%
\pgfsetfillopacity{0.500000}%
\pgfsetlinewidth{0.250937pt}%
\definecolor{currentstroke}{rgb}{0.000000,0.000000,0.000000}%
\pgfsetstrokecolor{currentstroke}%
\pgfsetstrokeopacity{0.500000}%
\pgfsetdash{}{0pt}%
\pgfsys@defobject{currentmarker}{\pgfqpoint{-0.011111in}{-0.011111in}}{\pgfqpoint{0.011111in}{0.011111in}}{%
\pgfpathmoveto{\pgfqpoint{0.000000in}{-0.011111in}}%
\pgfpathcurveto{\pgfqpoint{0.002947in}{-0.011111in}}{\pgfqpoint{0.005773in}{-0.009940in}}{\pgfqpoint{0.007857in}{-0.007857in}}%
\pgfpathcurveto{\pgfqpoint{0.009940in}{-0.005773in}}{\pgfqpoint{0.011111in}{-0.002947in}}{\pgfqpoint{0.011111in}{0.000000in}}%
\pgfpathcurveto{\pgfqpoint{0.011111in}{0.002947in}}{\pgfqpoint{0.009940in}{0.005773in}}{\pgfqpoint{0.007857in}{0.007857in}}%
\pgfpathcurveto{\pgfqpoint{0.005773in}{0.009940in}}{\pgfqpoint{0.002947in}{0.011111in}}{\pgfqpoint{0.000000in}{0.011111in}}%
\pgfpathcurveto{\pgfqpoint{-0.002947in}{0.011111in}}{\pgfqpoint{-0.005773in}{0.009940in}}{\pgfqpoint{-0.007857in}{0.007857in}}%
\pgfpathcurveto{\pgfqpoint{-0.009940in}{0.005773in}}{\pgfqpoint{-0.011111in}{0.002947in}}{\pgfqpoint{-0.011111in}{0.000000in}}%
\pgfpathcurveto{\pgfqpoint{-0.011111in}{-0.002947in}}{\pgfqpoint{-0.009940in}{-0.005773in}}{\pgfqpoint{-0.007857in}{-0.007857in}}%
\pgfpathcurveto{\pgfqpoint{-0.005773in}{-0.009940in}}{\pgfqpoint{-0.002947in}{-0.011111in}}{\pgfqpoint{0.000000in}{-0.011111in}}%
\pgfpathclose%
\pgfusepath{stroke,fill}%
}%
\begin{pgfscope}%
\pgfsys@transformshift{0.449905in}{4.503858in}%
\pgfsys@useobject{currentmarker}{}%
\end{pgfscope}%
\end{pgfscope}%
\begin{pgfscope}%
\pgfpathrectangle{\pgfqpoint{0.100000in}{2.413063in}}{\pgfqpoint{5.037500in}{3.427208in}}%
\pgfusepath{clip}%
\pgfsetrectcap%
\pgfsetroundjoin%
\pgfsetlinewidth{1.505625pt}%
\definecolor{currentstroke}{rgb}{0.000000,0.000000,1.000000}%
\pgfsetstrokecolor{currentstroke}%
\pgfsetstrokeopacity{0.500000}%
\pgfsetdash{}{0pt}%
\pgfpathmoveto{\pgfqpoint{0.370959in}{4.545432in}}%
\pgfusepath{stroke}%
\end{pgfscope}%
\begin{pgfscope}%
\pgfpathrectangle{\pgfqpoint{0.100000in}{2.413063in}}{\pgfqpoint{5.037500in}{3.427208in}}%
\pgfusepath{clip}%
\pgfsetbuttcap%
\pgfsetroundjoin%
\definecolor{currentfill}{rgb}{0.000000,0.000000,1.000000}%
\pgfsetfillcolor{currentfill}%
\pgfsetfillopacity{0.500000}%
\pgfsetlinewidth{0.250937pt}%
\definecolor{currentstroke}{rgb}{0.000000,0.000000,0.000000}%
\pgfsetstrokecolor{currentstroke}%
\pgfsetstrokeopacity{0.500000}%
\pgfsetdash{}{0pt}%
\pgfsys@defobject{currentmarker}{\pgfqpoint{-0.044444in}{-0.044444in}}{\pgfqpoint{0.044444in}{0.044444in}}{%
\pgfpathmoveto{\pgfqpoint{0.000000in}{-0.044444in}}%
\pgfpathcurveto{\pgfqpoint{0.011787in}{-0.044444in}}{\pgfqpoint{0.023092in}{-0.039761in}}{\pgfqpoint{0.031427in}{-0.031427in}}%
\pgfpathcurveto{\pgfqpoint{0.039761in}{-0.023092in}}{\pgfqpoint{0.044444in}{-0.011787in}}{\pgfqpoint{0.044444in}{0.000000in}}%
\pgfpathcurveto{\pgfqpoint{0.044444in}{0.011787in}}{\pgfqpoint{0.039761in}{0.023092in}}{\pgfqpoint{0.031427in}{0.031427in}}%
\pgfpathcurveto{\pgfqpoint{0.023092in}{0.039761in}}{\pgfqpoint{0.011787in}{0.044444in}}{\pgfqpoint{0.000000in}{0.044444in}}%
\pgfpathcurveto{\pgfqpoint{-0.011787in}{0.044444in}}{\pgfqpoint{-0.023092in}{0.039761in}}{\pgfqpoint{-0.031427in}{0.031427in}}%
\pgfpathcurveto{\pgfqpoint{-0.039761in}{0.023092in}}{\pgfqpoint{-0.044444in}{0.011787in}}{\pgfqpoint{-0.044444in}{0.000000in}}%
\pgfpathcurveto{\pgfqpoint{-0.044444in}{-0.011787in}}{\pgfqpoint{-0.039761in}{-0.023092in}}{\pgfqpoint{-0.031427in}{-0.031427in}}%
\pgfpathcurveto{\pgfqpoint{-0.023092in}{-0.039761in}}{\pgfqpoint{-0.011787in}{-0.044444in}}{\pgfqpoint{0.000000in}{-0.044444in}}%
\pgfpathclose%
\pgfusepath{stroke,fill}%
}%
\begin{pgfscope}%
\pgfsys@transformshift{0.370959in}{4.545432in}%
\pgfsys@useobject{currentmarker}{}%
\end{pgfscope}%
\end{pgfscope}%
\begin{pgfscope}%
\pgfpathrectangle{\pgfqpoint{0.100000in}{2.413063in}}{\pgfqpoint{5.037500in}{3.427208in}}%
\pgfusepath{clip}%
\pgfsetrectcap%
\pgfsetroundjoin%
\pgfsetlinewidth{1.505625pt}%
\definecolor{currentstroke}{rgb}{0.678431,1.000000,0.184314}%
\pgfsetstrokecolor{currentstroke}%
\pgfsetstrokeopacity{0.500000}%
\pgfsetdash{}{0pt}%
\pgfpathmoveto{\pgfqpoint{0.574887in}{4.273145in}}%
\pgfusepath{stroke}%
\end{pgfscope}%
\begin{pgfscope}%
\pgfpathrectangle{\pgfqpoint{0.100000in}{2.413063in}}{\pgfqpoint{5.037500in}{3.427208in}}%
\pgfusepath{clip}%
\pgfsetbuttcap%
\pgfsetroundjoin%
\definecolor{currentfill}{rgb}{0.678431,1.000000,0.184314}%
\pgfsetfillcolor{currentfill}%
\pgfsetfillopacity{0.500000}%
\pgfsetlinewidth{0.250937pt}%
\definecolor{currentstroke}{rgb}{0.000000,0.000000,0.000000}%
\pgfsetstrokecolor{currentstroke}%
\pgfsetstrokeopacity{0.500000}%
\pgfsetdash{}{0pt}%
\pgfsys@defobject{currentmarker}{\pgfqpoint{-0.047222in}{-0.047222in}}{\pgfqpoint{0.047222in}{0.047222in}}{%
\pgfpathmoveto{\pgfqpoint{0.000000in}{-0.047222in}}%
\pgfpathcurveto{\pgfqpoint{0.012523in}{-0.047222in}}{\pgfqpoint{0.024536in}{-0.042247in}}{\pgfqpoint{0.033391in}{-0.033391in}}%
\pgfpathcurveto{\pgfqpoint{0.042247in}{-0.024536in}}{\pgfqpoint{0.047222in}{-0.012523in}}{\pgfqpoint{0.047222in}{0.000000in}}%
\pgfpathcurveto{\pgfqpoint{0.047222in}{0.012523in}}{\pgfqpoint{0.042247in}{0.024536in}}{\pgfqpoint{0.033391in}{0.033391in}}%
\pgfpathcurveto{\pgfqpoint{0.024536in}{0.042247in}}{\pgfqpoint{0.012523in}{0.047222in}}{\pgfqpoint{0.000000in}{0.047222in}}%
\pgfpathcurveto{\pgfqpoint{-0.012523in}{0.047222in}}{\pgfqpoint{-0.024536in}{0.042247in}}{\pgfqpoint{-0.033391in}{0.033391in}}%
\pgfpathcurveto{\pgfqpoint{-0.042247in}{0.024536in}}{\pgfqpoint{-0.047222in}{0.012523in}}{\pgfqpoint{-0.047222in}{0.000000in}}%
\pgfpathcurveto{\pgfqpoint{-0.047222in}{-0.012523in}}{\pgfqpoint{-0.042247in}{-0.024536in}}{\pgfqpoint{-0.033391in}{-0.033391in}}%
\pgfpathcurveto{\pgfqpoint{-0.024536in}{-0.042247in}}{\pgfqpoint{-0.012523in}{-0.047222in}}{\pgfqpoint{0.000000in}{-0.047222in}}%
\pgfpathclose%
\pgfusepath{stroke,fill}%
}%
\begin{pgfscope}%
\pgfsys@transformshift{0.574887in}{4.273145in}%
\pgfsys@useobject{currentmarker}{}%
\end{pgfscope}%
\end{pgfscope}%
\begin{pgfscope}%
\pgfpathrectangle{\pgfqpoint{0.100000in}{2.413063in}}{\pgfqpoint{5.037500in}{3.427208in}}%
\pgfusepath{clip}%
\pgfsetrectcap%
\pgfsetroundjoin%
\pgfsetlinewidth{1.505625pt}%
\definecolor{currentstroke}{rgb}{0.678431,1.000000,0.184314}%
\pgfsetstrokecolor{currentstroke}%
\pgfsetstrokeopacity{0.500000}%
\pgfsetdash{}{0pt}%
\pgfpathmoveto{\pgfqpoint{0.460859in}{4.642756in}}%
\pgfusepath{stroke}%
\end{pgfscope}%
\begin{pgfscope}%
\pgfpathrectangle{\pgfqpoint{0.100000in}{2.413063in}}{\pgfqpoint{5.037500in}{3.427208in}}%
\pgfusepath{clip}%
\pgfsetbuttcap%
\pgfsetroundjoin%
\definecolor{currentfill}{rgb}{0.678431,1.000000,0.184314}%
\pgfsetfillcolor{currentfill}%
\pgfsetfillopacity{0.500000}%
\pgfsetlinewidth{0.250937pt}%
\definecolor{currentstroke}{rgb}{0.000000,0.000000,0.000000}%
\pgfsetstrokecolor{currentstroke}%
\pgfsetstrokeopacity{0.500000}%
\pgfsetdash{}{0pt}%
\pgfsys@defobject{currentmarker}{\pgfqpoint{-0.011111in}{-0.011111in}}{\pgfqpoint{0.011111in}{0.011111in}}{%
\pgfpathmoveto{\pgfqpoint{0.000000in}{-0.011111in}}%
\pgfpathcurveto{\pgfqpoint{0.002947in}{-0.011111in}}{\pgfqpoint{0.005773in}{-0.009940in}}{\pgfqpoint{0.007857in}{-0.007857in}}%
\pgfpathcurveto{\pgfqpoint{0.009940in}{-0.005773in}}{\pgfqpoint{0.011111in}{-0.002947in}}{\pgfqpoint{0.011111in}{0.000000in}}%
\pgfpathcurveto{\pgfqpoint{0.011111in}{0.002947in}}{\pgfqpoint{0.009940in}{0.005773in}}{\pgfqpoint{0.007857in}{0.007857in}}%
\pgfpathcurveto{\pgfqpoint{0.005773in}{0.009940in}}{\pgfqpoint{0.002947in}{0.011111in}}{\pgfqpoint{0.000000in}{0.011111in}}%
\pgfpathcurveto{\pgfqpoint{-0.002947in}{0.011111in}}{\pgfqpoint{-0.005773in}{0.009940in}}{\pgfqpoint{-0.007857in}{0.007857in}}%
\pgfpathcurveto{\pgfqpoint{-0.009940in}{0.005773in}}{\pgfqpoint{-0.011111in}{0.002947in}}{\pgfqpoint{-0.011111in}{0.000000in}}%
\pgfpathcurveto{\pgfqpoint{-0.011111in}{-0.002947in}}{\pgfqpoint{-0.009940in}{-0.005773in}}{\pgfqpoint{-0.007857in}{-0.007857in}}%
\pgfpathcurveto{\pgfqpoint{-0.005773in}{-0.009940in}}{\pgfqpoint{-0.002947in}{-0.011111in}}{\pgfqpoint{0.000000in}{-0.011111in}}%
\pgfpathclose%
\pgfusepath{stroke,fill}%
}%
\begin{pgfscope}%
\pgfsys@transformshift{0.460859in}{4.642756in}%
\pgfsys@useobject{currentmarker}{}%
\end{pgfscope}%
\end{pgfscope}%
\begin{pgfscope}%
\pgfpathrectangle{\pgfqpoint{0.100000in}{2.413063in}}{\pgfqpoint{5.037500in}{3.427208in}}%
\pgfusepath{clip}%
\pgfsetrectcap%
\pgfsetroundjoin%
\pgfsetlinewidth{1.505625pt}%
\definecolor{currentstroke}{rgb}{0.000000,0.000000,1.000000}%
\pgfsetstrokecolor{currentstroke}%
\pgfsetstrokeopacity{0.500000}%
\pgfsetdash{}{0pt}%
\pgfpathmoveto{\pgfqpoint{1.900192in}{4.448961in}}%
\pgfusepath{stroke}%
\end{pgfscope}%
\begin{pgfscope}%
\pgfpathrectangle{\pgfqpoint{0.100000in}{2.413063in}}{\pgfqpoint{5.037500in}{3.427208in}}%
\pgfusepath{clip}%
\pgfsetbuttcap%
\pgfsetroundjoin%
\definecolor{currentfill}{rgb}{0.000000,0.000000,1.000000}%
\pgfsetfillcolor{currentfill}%
\pgfsetfillopacity{0.500000}%
\pgfsetlinewidth{0.250937pt}%
\definecolor{currentstroke}{rgb}{0.000000,0.000000,0.000000}%
\pgfsetstrokecolor{currentstroke}%
\pgfsetstrokeopacity{0.500000}%
\pgfsetdash{}{0pt}%
\pgfsys@defobject{currentmarker}{\pgfqpoint{-0.033333in}{-0.033333in}}{\pgfqpoint{0.033333in}{0.033333in}}{%
\pgfpathmoveto{\pgfqpoint{0.000000in}{-0.033333in}}%
\pgfpathcurveto{\pgfqpoint{0.008840in}{-0.033333in}}{\pgfqpoint{0.017319in}{-0.029821in}}{\pgfqpoint{0.023570in}{-0.023570in}}%
\pgfpathcurveto{\pgfqpoint{0.029821in}{-0.017319in}}{\pgfqpoint{0.033333in}{-0.008840in}}{\pgfqpoint{0.033333in}{0.000000in}}%
\pgfpathcurveto{\pgfqpoint{0.033333in}{0.008840in}}{\pgfqpoint{0.029821in}{0.017319in}}{\pgfqpoint{0.023570in}{0.023570in}}%
\pgfpathcurveto{\pgfqpoint{0.017319in}{0.029821in}}{\pgfqpoint{0.008840in}{0.033333in}}{\pgfqpoint{0.000000in}{0.033333in}}%
\pgfpathcurveto{\pgfqpoint{-0.008840in}{0.033333in}}{\pgfqpoint{-0.017319in}{0.029821in}}{\pgfqpoint{-0.023570in}{0.023570in}}%
\pgfpathcurveto{\pgfqpoint{-0.029821in}{0.017319in}}{\pgfqpoint{-0.033333in}{0.008840in}}{\pgfqpoint{-0.033333in}{0.000000in}}%
\pgfpathcurveto{\pgfqpoint{-0.033333in}{-0.008840in}}{\pgfqpoint{-0.029821in}{-0.017319in}}{\pgfqpoint{-0.023570in}{-0.023570in}}%
\pgfpathcurveto{\pgfqpoint{-0.017319in}{-0.029821in}}{\pgfqpoint{-0.008840in}{-0.033333in}}{\pgfqpoint{0.000000in}{-0.033333in}}%
\pgfpathclose%
\pgfusepath{stroke,fill}%
}%
\begin{pgfscope}%
\pgfsys@transformshift{1.900192in}{4.448961in}%
\pgfsys@useobject{currentmarker}{}%
\end{pgfscope}%
\end{pgfscope}%
\begin{pgfscope}%
\pgfpathrectangle{\pgfqpoint{0.100000in}{2.413063in}}{\pgfqpoint{5.037500in}{3.427208in}}%
\pgfusepath{clip}%
\pgfsetrectcap%
\pgfsetroundjoin%
\pgfsetlinewidth{1.505625pt}%
\definecolor{currentstroke}{rgb}{0.000000,0.000000,1.000000}%
\pgfsetstrokecolor{currentstroke}%
\pgfsetstrokeopacity{0.500000}%
\pgfsetdash{}{0pt}%
\pgfpathmoveto{\pgfqpoint{1.924544in}{4.309425in}}%
\pgfusepath{stroke}%
\end{pgfscope}%
\begin{pgfscope}%
\pgfpathrectangle{\pgfqpoint{0.100000in}{2.413063in}}{\pgfqpoint{5.037500in}{3.427208in}}%
\pgfusepath{clip}%
\pgfsetbuttcap%
\pgfsetroundjoin%
\definecolor{currentfill}{rgb}{0.000000,0.000000,1.000000}%
\pgfsetfillcolor{currentfill}%
\pgfsetfillopacity{0.500000}%
\pgfsetlinewidth{0.250937pt}%
\definecolor{currentstroke}{rgb}{0.000000,0.000000,0.000000}%
\pgfsetstrokecolor{currentstroke}%
\pgfsetstrokeopacity{0.500000}%
\pgfsetdash{}{0pt}%
\pgfsys@defobject{currentmarker}{\pgfqpoint{-0.041667in}{-0.041667in}}{\pgfqpoint{0.041667in}{0.041667in}}{%
\pgfpathmoveto{\pgfqpoint{0.000000in}{-0.041667in}}%
\pgfpathcurveto{\pgfqpoint{0.011050in}{-0.041667in}}{\pgfqpoint{0.021649in}{-0.037276in}}{\pgfqpoint{0.029463in}{-0.029463in}}%
\pgfpathcurveto{\pgfqpoint{0.037276in}{-0.021649in}}{\pgfqpoint{0.041667in}{-0.011050in}}{\pgfqpoint{0.041667in}{0.000000in}}%
\pgfpathcurveto{\pgfqpoint{0.041667in}{0.011050in}}{\pgfqpoint{0.037276in}{0.021649in}}{\pgfqpoint{0.029463in}{0.029463in}}%
\pgfpathcurveto{\pgfqpoint{0.021649in}{0.037276in}}{\pgfqpoint{0.011050in}{0.041667in}}{\pgfqpoint{0.000000in}{0.041667in}}%
\pgfpathcurveto{\pgfqpoint{-0.011050in}{0.041667in}}{\pgfqpoint{-0.021649in}{0.037276in}}{\pgfqpoint{-0.029463in}{0.029463in}}%
\pgfpathcurveto{\pgfqpoint{-0.037276in}{0.021649in}}{\pgfqpoint{-0.041667in}{0.011050in}}{\pgfqpoint{-0.041667in}{0.000000in}}%
\pgfpathcurveto{\pgfqpoint{-0.041667in}{-0.011050in}}{\pgfqpoint{-0.037276in}{-0.021649in}}{\pgfqpoint{-0.029463in}{-0.029463in}}%
\pgfpathcurveto{\pgfqpoint{-0.021649in}{-0.037276in}}{\pgfqpoint{-0.011050in}{-0.041667in}}{\pgfqpoint{0.000000in}{-0.041667in}}%
\pgfpathclose%
\pgfusepath{stroke,fill}%
}%
\begin{pgfscope}%
\pgfsys@transformshift{1.924544in}{4.309425in}%
\pgfsys@useobject{currentmarker}{}%
\end{pgfscope}%
\end{pgfscope}%
\begin{pgfscope}%
\pgfpathrectangle{\pgfqpoint{0.100000in}{2.413063in}}{\pgfqpoint{5.037500in}{3.427208in}}%
\pgfusepath{clip}%
\pgfsetrectcap%
\pgfsetroundjoin%
\pgfsetlinewidth{1.505625pt}%
\definecolor{currentstroke}{rgb}{0.000000,0.000000,1.000000}%
\pgfsetstrokecolor{currentstroke}%
\pgfsetstrokeopacity{0.500000}%
\pgfsetdash{}{0pt}%
\pgfpathmoveto{\pgfqpoint{1.921764in}{4.414584in}}%
\pgfusepath{stroke}%
\end{pgfscope}%
\begin{pgfscope}%
\pgfpathrectangle{\pgfqpoint{0.100000in}{2.413063in}}{\pgfqpoint{5.037500in}{3.427208in}}%
\pgfusepath{clip}%
\pgfsetbuttcap%
\pgfsetroundjoin%
\definecolor{currentfill}{rgb}{0.000000,0.000000,1.000000}%
\pgfsetfillcolor{currentfill}%
\pgfsetfillopacity{0.500000}%
\pgfsetlinewidth{0.250937pt}%
\definecolor{currentstroke}{rgb}{0.000000,0.000000,0.000000}%
\pgfsetstrokecolor{currentstroke}%
\pgfsetstrokeopacity{0.500000}%
\pgfsetdash{}{0pt}%
\pgfsys@defobject{currentmarker}{\pgfqpoint{-0.052778in}{-0.052778in}}{\pgfqpoint{0.052778in}{0.052778in}}{%
\pgfpathmoveto{\pgfqpoint{0.000000in}{-0.052778in}}%
\pgfpathcurveto{\pgfqpoint{0.013997in}{-0.052778in}}{\pgfqpoint{0.027422in}{-0.047217in}}{\pgfqpoint{0.037320in}{-0.037320in}}%
\pgfpathcurveto{\pgfqpoint{0.047217in}{-0.027422in}}{\pgfqpoint{0.052778in}{-0.013997in}}{\pgfqpoint{0.052778in}{0.000000in}}%
\pgfpathcurveto{\pgfqpoint{0.052778in}{0.013997in}}{\pgfqpoint{0.047217in}{0.027422in}}{\pgfqpoint{0.037320in}{0.037320in}}%
\pgfpathcurveto{\pgfqpoint{0.027422in}{0.047217in}}{\pgfqpoint{0.013997in}{0.052778in}}{\pgfqpoint{0.000000in}{0.052778in}}%
\pgfpathcurveto{\pgfqpoint{-0.013997in}{0.052778in}}{\pgfqpoint{-0.027422in}{0.047217in}}{\pgfqpoint{-0.037320in}{0.037320in}}%
\pgfpathcurveto{\pgfqpoint{-0.047217in}{0.027422in}}{\pgfqpoint{-0.052778in}{0.013997in}}{\pgfqpoint{-0.052778in}{0.000000in}}%
\pgfpathcurveto{\pgfqpoint{-0.052778in}{-0.013997in}}{\pgfqpoint{-0.047217in}{-0.027422in}}{\pgfqpoint{-0.037320in}{-0.037320in}}%
\pgfpathcurveto{\pgfqpoint{-0.027422in}{-0.047217in}}{\pgfqpoint{-0.013997in}{-0.052778in}}{\pgfqpoint{0.000000in}{-0.052778in}}%
\pgfpathclose%
\pgfusepath{stroke,fill}%
}%
\begin{pgfscope}%
\pgfsys@transformshift{1.921764in}{4.414584in}%
\pgfsys@useobject{currentmarker}{}%
\end{pgfscope}%
\end{pgfscope}%
\begin{pgfscope}%
\pgfpathrectangle{\pgfqpoint{0.100000in}{2.413063in}}{\pgfqpoint{5.037500in}{3.427208in}}%
\pgfusepath{clip}%
\pgfsetrectcap%
\pgfsetroundjoin%
\pgfsetlinewidth{1.505625pt}%
\definecolor{currentstroke}{rgb}{0.000000,0.000000,1.000000}%
\pgfsetstrokecolor{currentstroke}%
\pgfsetstrokeopacity{0.500000}%
\pgfsetdash{}{0pt}%
\pgfpathmoveto{\pgfqpoint{1.924872in}{4.508293in}}%
\pgfusepath{stroke}%
\end{pgfscope}%
\begin{pgfscope}%
\pgfpathrectangle{\pgfqpoint{0.100000in}{2.413063in}}{\pgfqpoint{5.037500in}{3.427208in}}%
\pgfusepath{clip}%
\pgfsetbuttcap%
\pgfsetroundjoin%
\definecolor{currentfill}{rgb}{0.000000,0.000000,1.000000}%
\pgfsetfillcolor{currentfill}%
\pgfsetfillopacity{0.500000}%
\pgfsetlinewidth{0.250937pt}%
\definecolor{currentstroke}{rgb}{0.000000,0.000000,0.000000}%
\pgfsetstrokecolor{currentstroke}%
\pgfsetstrokeopacity{0.500000}%
\pgfsetdash{}{0pt}%
\pgfsys@defobject{currentmarker}{\pgfqpoint{-0.038889in}{-0.038889in}}{\pgfqpoint{0.038889in}{0.038889in}}{%
\pgfpathmoveto{\pgfqpoint{0.000000in}{-0.038889in}}%
\pgfpathcurveto{\pgfqpoint{0.010313in}{-0.038889in}}{\pgfqpoint{0.020206in}{-0.034791in}}{\pgfqpoint{0.027499in}{-0.027499in}}%
\pgfpathcurveto{\pgfqpoint{0.034791in}{-0.020206in}}{\pgfqpoint{0.038889in}{-0.010313in}}{\pgfqpoint{0.038889in}{0.000000in}}%
\pgfpathcurveto{\pgfqpoint{0.038889in}{0.010313in}}{\pgfqpoint{0.034791in}{0.020206in}}{\pgfqpoint{0.027499in}{0.027499in}}%
\pgfpathcurveto{\pgfqpoint{0.020206in}{0.034791in}}{\pgfqpoint{0.010313in}{0.038889in}}{\pgfqpoint{0.000000in}{0.038889in}}%
\pgfpathcurveto{\pgfqpoint{-0.010313in}{0.038889in}}{\pgfqpoint{-0.020206in}{0.034791in}}{\pgfqpoint{-0.027499in}{0.027499in}}%
\pgfpathcurveto{\pgfqpoint{-0.034791in}{0.020206in}}{\pgfqpoint{-0.038889in}{0.010313in}}{\pgfqpoint{-0.038889in}{0.000000in}}%
\pgfpathcurveto{\pgfqpoint{-0.038889in}{-0.010313in}}{\pgfqpoint{-0.034791in}{-0.020206in}}{\pgfqpoint{-0.027499in}{-0.027499in}}%
\pgfpathcurveto{\pgfqpoint{-0.020206in}{-0.034791in}}{\pgfqpoint{-0.010313in}{-0.038889in}}{\pgfqpoint{0.000000in}{-0.038889in}}%
\pgfpathclose%
\pgfusepath{stroke,fill}%
}%
\begin{pgfscope}%
\pgfsys@transformshift{1.924872in}{4.508293in}%
\pgfsys@useobject{currentmarker}{}%
\end{pgfscope}%
\end{pgfscope}%
\begin{pgfscope}%
\pgfpathrectangle{\pgfqpoint{0.100000in}{2.413063in}}{\pgfqpoint{5.037500in}{3.427208in}}%
\pgfusepath{clip}%
\pgfsetrectcap%
\pgfsetroundjoin%
\pgfsetlinewidth{1.505625pt}%
\definecolor{currentstroke}{rgb}{0.000000,0.000000,1.000000}%
\pgfsetstrokecolor{currentstroke}%
\pgfsetstrokeopacity{0.500000}%
\pgfsetdash{}{0pt}%
\pgfpathmoveto{\pgfqpoint{1.597107in}{4.378471in}}%
\pgfusepath{stroke}%
\end{pgfscope}%
\begin{pgfscope}%
\pgfpathrectangle{\pgfqpoint{0.100000in}{2.413063in}}{\pgfqpoint{5.037500in}{3.427208in}}%
\pgfusepath{clip}%
\pgfsetbuttcap%
\pgfsetroundjoin%
\definecolor{currentfill}{rgb}{0.000000,0.000000,1.000000}%
\pgfsetfillcolor{currentfill}%
\pgfsetfillopacity{0.500000}%
\pgfsetlinewidth{0.250937pt}%
\definecolor{currentstroke}{rgb}{0.000000,0.000000,0.000000}%
\pgfsetstrokecolor{currentstroke}%
\pgfsetstrokeopacity{0.500000}%
\pgfsetdash{}{0pt}%
\pgfsys@defobject{currentmarker}{\pgfqpoint{-0.038889in}{-0.038889in}}{\pgfqpoint{0.038889in}{0.038889in}}{%
\pgfpathmoveto{\pgfqpoint{0.000000in}{-0.038889in}}%
\pgfpathcurveto{\pgfqpoint{0.010313in}{-0.038889in}}{\pgfqpoint{0.020206in}{-0.034791in}}{\pgfqpoint{0.027499in}{-0.027499in}}%
\pgfpathcurveto{\pgfqpoint{0.034791in}{-0.020206in}}{\pgfqpoint{0.038889in}{-0.010313in}}{\pgfqpoint{0.038889in}{0.000000in}}%
\pgfpathcurveto{\pgfqpoint{0.038889in}{0.010313in}}{\pgfqpoint{0.034791in}{0.020206in}}{\pgfqpoint{0.027499in}{0.027499in}}%
\pgfpathcurveto{\pgfqpoint{0.020206in}{0.034791in}}{\pgfqpoint{0.010313in}{0.038889in}}{\pgfqpoint{0.000000in}{0.038889in}}%
\pgfpathcurveto{\pgfqpoint{-0.010313in}{0.038889in}}{\pgfqpoint{-0.020206in}{0.034791in}}{\pgfqpoint{-0.027499in}{0.027499in}}%
\pgfpathcurveto{\pgfqpoint{-0.034791in}{0.020206in}}{\pgfqpoint{-0.038889in}{0.010313in}}{\pgfqpoint{-0.038889in}{0.000000in}}%
\pgfpathcurveto{\pgfqpoint{-0.038889in}{-0.010313in}}{\pgfqpoint{-0.034791in}{-0.020206in}}{\pgfqpoint{-0.027499in}{-0.027499in}}%
\pgfpathcurveto{\pgfqpoint{-0.020206in}{-0.034791in}}{\pgfqpoint{-0.010313in}{-0.038889in}}{\pgfqpoint{0.000000in}{-0.038889in}}%
\pgfpathclose%
\pgfusepath{stroke,fill}%
}%
\begin{pgfscope}%
\pgfsys@transformshift{1.597107in}{4.378471in}%
\pgfsys@useobject{currentmarker}{}%
\end{pgfscope}%
\end{pgfscope}%
\begin{pgfscope}%
\pgfpathrectangle{\pgfqpoint{0.100000in}{2.413063in}}{\pgfqpoint{5.037500in}{3.427208in}}%
\pgfusepath{clip}%
\pgfsetrectcap%
\pgfsetroundjoin%
\pgfsetlinewidth{1.505625pt}%
\definecolor{currentstroke}{rgb}{0.000000,0.000000,1.000000}%
\pgfsetstrokecolor{currentstroke}%
\pgfsetstrokeopacity{0.500000}%
\pgfsetdash{}{0pt}%
\pgfpathmoveto{\pgfqpoint{1.954425in}{4.490288in}}%
\pgfusepath{stroke}%
\end{pgfscope}%
\begin{pgfscope}%
\pgfpathrectangle{\pgfqpoint{0.100000in}{2.413063in}}{\pgfqpoint{5.037500in}{3.427208in}}%
\pgfusepath{clip}%
\pgfsetbuttcap%
\pgfsetroundjoin%
\definecolor{currentfill}{rgb}{0.000000,0.000000,1.000000}%
\pgfsetfillcolor{currentfill}%
\pgfsetfillopacity{0.500000}%
\pgfsetlinewidth{0.250937pt}%
\definecolor{currentstroke}{rgb}{0.000000,0.000000,0.000000}%
\pgfsetstrokecolor{currentstroke}%
\pgfsetstrokeopacity{0.500000}%
\pgfsetdash{}{0pt}%
\pgfsys@defobject{currentmarker}{\pgfqpoint{-0.050000in}{-0.050000in}}{\pgfqpoint{0.050000in}{0.050000in}}{%
\pgfpathmoveto{\pgfqpoint{0.000000in}{-0.050000in}}%
\pgfpathcurveto{\pgfqpoint{0.013260in}{-0.050000in}}{\pgfqpoint{0.025979in}{-0.044732in}}{\pgfqpoint{0.035355in}{-0.035355in}}%
\pgfpathcurveto{\pgfqpoint{0.044732in}{-0.025979in}}{\pgfqpoint{0.050000in}{-0.013260in}}{\pgfqpoint{0.050000in}{0.000000in}}%
\pgfpathcurveto{\pgfqpoint{0.050000in}{0.013260in}}{\pgfqpoint{0.044732in}{0.025979in}}{\pgfqpoint{0.035355in}{0.035355in}}%
\pgfpathcurveto{\pgfqpoint{0.025979in}{0.044732in}}{\pgfqpoint{0.013260in}{0.050000in}}{\pgfqpoint{0.000000in}{0.050000in}}%
\pgfpathcurveto{\pgfqpoint{-0.013260in}{0.050000in}}{\pgfqpoint{-0.025979in}{0.044732in}}{\pgfqpoint{-0.035355in}{0.035355in}}%
\pgfpathcurveto{\pgfqpoint{-0.044732in}{0.025979in}}{\pgfqpoint{-0.050000in}{0.013260in}}{\pgfqpoint{-0.050000in}{0.000000in}}%
\pgfpathcurveto{\pgfqpoint{-0.050000in}{-0.013260in}}{\pgfqpoint{-0.044732in}{-0.025979in}}{\pgfqpoint{-0.035355in}{-0.035355in}}%
\pgfpathcurveto{\pgfqpoint{-0.025979in}{-0.044732in}}{\pgfqpoint{-0.013260in}{-0.050000in}}{\pgfqpoint{0.000000in}{-0.050000in}}%
\pgfpathclose%
\pgfusepath{stroke,fill}%
}%
\begin{pgfscope}%
\pgfsys@transformshift{1.954425in}{4.490288in}%
\pgfsys@useobject{currentmarker}{}%
\end{pgfscope}%
\end{pgfscope}%
\begin{pgfscope}%
\pgfpathrectangle{\pgfqpoint{0.100000in}{2.413063in}}{\pgfqpoint{5.037500in}{3.427208in}}%
\pgfusepath{clip}%
\pgfsetrectcap%
\pgfsetroundjoin%
\pgfsetlinewidth{1.505625pt}%
\definecolor{currentstroke}{rgb}{0.000000,0.000000,1.000000}%
\pgfsetstrokecolor{currentstroke}%
\pgfsetstrokeopacity{0.500000}%
\pgfsetdash{}{0pt}%
\pgfpathmoveto{\pgfqpoint{1.936783in}{4.240996in}}%
\pgfusepath{stroke}%
\end{pgfscope}%
\begin{pgfscope}%
\pgfpathrectangle{\pgfqpoint{0.100000in}{2.413063in}}{\pgfqpoint{5.037500in}{3.427208in}}%
\pgfusepath{clip}%
\pgfsetbuttcap%
\pgfsetroundjoin%
\definecolor{currentfill}{rgb}{0.000000,0.000000,1.000000}%
\pgfsetfillcolor{currentfill}%
\pgfsetfillopacity{0.500000}%
\pgfsetlinewidth{0.250937pt}%
\definecolor{currentstroke}{rgb}{0.000000,0.000000,0.000000}%
\pgfsetstrokecolor{currentstroke}%
\pgfsetstrokeopacity{0.500000}%
\pgfsetdash{}{0pt}%
\pgfsys@defobject{currentmarker}{\pgfqpoint{-0.077778in}{-0.077778in}}{\pgfqpoint{0.077778in}{0.077778in}}{%
\pgfpathmoveto{\pgfqpoint{0.000000in}{-0.077778in}}%
\pgfpathcurveto{\pgfqpoint{0.020627in}{-0.077778in}}{\pgfqpoint{0.040412in}{-0.069583in}}{\pgfqpoint{0.054997in}{-0.054997in}}%
\pgfpathcurveto{\pgfqpoint{0.069583in}{-0.040412in}}{\pgfqpoint{0.077778in}{-0.020627in}}{\pgfqpoint{0.077778in}{0.000000in}}%
\pgfpathcurveto{\pgfqpoint{0.077778in}{0.020627in}}{\pgfqpoint{0.069583in}{0.040412in}}{\pgfqpoint{0.054997in}{0.054997in}}%
\pgfpathcurveto{\pgfqpoint{0.040412in}{0.069583in}}{\pgfqpoint{0.020627in}{0.077778in}}{\pgfqpoint{0.000000in}{0.077778in}}%
\pgfpathcurveto{\pgfqpoint{-0.020627in}{0.077778in}}{\pgfqpoint{-0.040412in}{0.069583in}}{\pgfqpoint{-0.054997in}{0.054997in}}%
\pgfpathcurveto{\pgfqpoint{-0.069583in}{0.040412in}}{\pgfqpoint{-0.077778in}{0.020627in}}{\pgfqpoint{-0.077778in}{0.000000in}}%
\pgfpathcurveto{\pgfqpoint{-0.077778in}{-0.020627in}}{\pgfqpoint{-0.069583in}{-0.040412in}}{\pgfqpoint{-0.054997in}{-0.054997in}}%
\pgfpathcurveto{\pgfqpoint{-0.040412in}{-0.069583in}}{\pgfqpoint{-0.020627in}{-0.077778in}}{\pgfqpoint{0.000000in}{-0.077778in}}%
\pgfpathclose%
\pgfusepath{stroke,fill}%
}%
\begin{pgfscope}%
\pgfsys@transformshift{1.936783in}{4.240996in}%
\pgfsys@useobject{currentmarker}{}%
\end{pgfscope}%
\end{pgfscope}%
\begin{pgfscope}%
\pgfpathrectangle{\pgfqpoint{0.100000in}{2.413063in}}{\pgfqpoint{5.037500in}{3.427208in}}%
\pgfusepath{clip}%
\pgfsetrectcap%
\pgfsetroundjoin%
\pgfsetlinewidth{1.505625pt}%
\definecolor{currentstroke}{rgb}{0.000000,0.000000,1.000000}%
\pgfsetstrokecolor{currentstroke}%
\pgfsetstrokeopacity{0.500000}%
\pgfsetdash{}{0pt}%
\pgfpathmoveto{\pgfqpoint{4.676566in}{4.756149in}}%
\pgfusepath{stroke}%
\end{pgfscope}%
\begin{pgfscope}%
\pgfpathrectangle{\pgfqpoint{0.100000in}{2.413063in}}{\pgfqpoint{5.037500in}{3.427208in}}%
\pgfusepath{clip}%
\pgfsetbuttcap%
\pgfsetroundjoin%
\definecolor{currentfill}{rgb}{0.000000,0.000000,1.000000}%
\pgfsetfillcolor{currentfill}%
\pgfsetfillopacity{0.500000}%
\pgfsetlinewidth{0.250937pt}%
\definecolor{currentstroke}{rgb}{0.000000,0.000000,0.000000}%
\pgfsetstrokecolor{currentstroke}%
\pgfsetstrokeopacity{0.500000}%
\pgfsetdash{}{0pt}%
\pgfsys@defobject{currentmarker}{\pgfqpoint{-0.041667in}{-0.041667in}}{\pgfqpoint{0.041667in}{0.041667in}}{%
\pgfpathmoveto{\pgfqpoint{0.000000in}{-0.041667in}}%
\pgfpathcurveto{\pgfqpoint{0.011050in}{-0.041667in}}{\pgfqpoint{0.021649in}{-0.037276in}}{\pgfqpoint{0.029463in}{-0.029463in}}%
\pgfpathcurveto{\pgfqpoint{0.037276in}{-0.021649in}}{\pgfqpoint{0.041667in}{-0.011050in}}{\pgfqpoint{0.041667in}{0.000000in}}%
\pgfpathcurveto{\pgfqpoint{0.041667in}{0.011050in}}{\pgfqpoint{0.037276in}{0.021649in}}{\pgfqpoint{0.029463in}{0.029463in}}%
\pgfpathcurveto{\pgfqpoint{0.021649in}{0.037276in}}{\pgfqpoint{0.011050in}{0.041667in}}{\pgfqpoint{0.000000in}{0.041667in}}%
\pgfpathcurveto{\pgfqpoint{-0.011050in}{0.041667in}}{\pgfqpoint{-0.021649in}{0.037276in}}{\pgfqpoint{-0.029463in}{0.029463in}}%
\pgfpathcurveto{\pgfqpoint{-0.037276in}{0.021649in}}{\pgfqpoint{-0.041667in}{0.011050in}}{\pgfqpoint{-0.041667in}{0.000000in}}%
\pgfpathcurveto{\pgfqpoint{-0.041667in}{-0.011050in}}{\pgfqpoint{-0.037276in}{-0.021649in}}{\pgfqpoint{-0.029463in}{-0.029463in}}%
\pgfpathcurveto{\pgfqpoint{-0.021649in}{-0.037276in}}{\pgfqpoint{-0.011050in}{-0.041667in}}{\pgfqpoint{0.000000in}{-0.041667in}}%
\pgfpathclose%
\pgfusepath{stroke,fill}%
}%
\begin{pgfscope}%
\pgfsys@transformshift{4.676566in}{4.756149in}%
\pgfsys@useobject{currentmarker}{}%
\end{pgfscope}%
\end{pgfscope}%
\begin{pgfscope}%
\pgfpathrectangle{\pgfqpoint{0.100000in}{2.413063in}}{\pgfqpoint{5.037500in}{3.427208in}}%
\pgfusepath{clip}%
\pgfsetrectcap%
\pgfsetroundjoin%
\pgfsetlinewidth{1.505625pt}%
\definecolor{currentstroke}{rgb}{0.000000,0.000000,1.000000}%
\pgfsetstrokecolor{currentstroke}%
\pgfsetstrokeopacity{0.500000}%
\pgfsetdash{}{0pt}%
\pgfpathmoveto{\pgfqpoint{4.649399in}{4.776556in}}%
\pgfusepath{stroke}%
\end{pgfscope}%
\begin{pgfscope}%
\pgfpathrectangle{\pgfqpoint{0.100000in}{2.413063in}}{\pgfqpoint{5.037500in}{3.427208in}}%
\pgfusepath{clip}%
\pgfsetbuttcap%
\pgfsetroundjoin%
\definecolor{currentfill}{rgb}{0.000000,0.000000,1.000000}%
\pgfsetfillcolor{currentfill}%
\pgfsetfillopacity{0.500000}%
\pgfsetlinewidth{0.250937pt}%
\definecolor{currentstroke}{rgb}{0.000000,0.000000,0.000000}%
\pgfsetstrokecolor{currentstroke}%
\pgfsetstrokeopacity{0.500000}%
\pgfsetdash{}{0pt}%
\pgfsys@defobject{currentmarker}{\pgfqpoint{-0.030556in}{-0.030556in}}{\pgfqpoint{0.030556in}{0.030556in}}{%
\pgfpathmoveto{\pgfqpoint{0.000000in}{-0.030556in}}%
\pgfpathcurveto{\pgfqpoint{0.008103in}{-0.030556in}}{\pgfqpoint{0.015876in}{-0.027336in}}{\pgfqpoint{0.021606in}{-0.021606in}}%
\pgfpathcurveto{\pgfqpoint{0.027336in}{-0.015876in}}{\pgfqpoint{0.030556in}{-0.008103in}}{\pgfqpoint{0.030556in}{0.000000in}}%
\pgfpathcurveto{\pgfqpoint{0.030556in}{0.008103in}}{\pgfqpoint{0.027336in}{0.015876in}}{\pgfqpoint{0.021606in}{0.021606in}}%
\pgfpathcurveto{\pgfqpoint{0.015876in}{0.027336in}}{\pgfqpoint{0.008103in}{0.030556in}}{\pgfqpoint{0.000000in}{0.030556in}}%
\pgfpathcurveto{\pgfqpoint{-0.008103in}{0.030556in}}{\pgfqpoint{-0.015876in}{0.027336in}}{\pgfqpoint{-0.021606in}{0.021606in}}%
\pgfpathcurveto{\pgfqpoint{-0.027336in}{0.015876in}}{\pgfqpoint{-0.030556in}{0.008103in}}{\pgfqpoint{-0.030556in}{0.000000in}}%
\pgfpathcurveto{\pgfqpoint{-0.030556in}{-0.008103in}}{\pgfqpoint{-0.027336in}{-0.015876in}}{\pgfqpoint{-0.021606in}{-0.021606in}}%
\pgfpathcurveto{\pgfqpoint{-0.015876in}{-0.027336in}}{\pgfqpoint{-0.008103in}{-0.030556in}}{\pgfqpoint{0.000000in}{-0.030556in}}%
\pgfpathclose%
\pgfusepath{stroke,fill}%
}%
\begin{pgfscope}%
\pgfsys@transformshift{4.649399in}{4.776556in}%
\pgfsys@useobject{currentmarker}{}%
\end{pgfscope}%
\end{pgfscope}%
\begin{pgfscope}%
\pgfpathrectangle{\pgfqpoint{0.100000in}{2.413063in}}{\pgfqpoint{5.037500in}{3.427208in}}%
\pgfusepath{clip}%
\pgfsetrectcap%
\pgfsetroundjoin%
\pgfsetlinewidth{1.505625pt}%
\definecolor{currentstroke}{rgb}{0.000000,0.000000,1.000000}%
\pgfsetstrokecolor{currentstroke}%
\pgfsetstrokeopacity{0.500000}%
\pgfsetdash{}{0pt}%
\pgfpathmoveto{\pgfqpoint{4.703787in}{4.833704in}}%
\pgfusepath{stroke}%
\end{pgfscope}%
\begin{pgfscope}%
\pgfpathrectangle{\pgfqpoint{0.100000in}{2.413063in}}{\pgfqpoint{5.037500in}{3.427208in}}%
\pgfusepath{clip}%
\pgfsetbuttcap%
\pgfsetroundjoin%
\definecolor{currentfill}{rgb}{0.000000,0.000000,1.000000}%
\pgfsetfillcolor{currentfill}%
\pgfsetfillopacity{0.500000}%
\pgfsetlinewidth{0.250937pt}%
\definecolor{currentstroke}{rgb}{0.000000,0.000000,0.000000}%
\pgfsetstrokecolor{currentstroke}%
\pgfsetstrokeopacity{0.500000}%
\pgfsetdash{}{0pt}%
\pgfsys@defobject{currentmarker}{\pgfqpoint{-0.038889in}{-0.038889in}}{\pgfqpoint{0.038889in}{0.038889in}}{%
\pgfpathmoveto{\pgfqpoint{0.000000in}{-0.038889in}}%
\pgfpathcurveto{\pgfqpoint{0.010313in}{-0.038889in}}{\pgfqpoint{0.020206in}{-0.034791in}}{\pgfqpoint{0.027499in}{-0.027499in}}%
\pgfpathcurveto{\pgfqpoint{0.034791in}{-0.020206in}}{\pgfqpoint{0.038889in}{-0.010313in}}{\pgfqpoint{0.038889in}{0.000000in}}%
\pgfpathcurveto{\pgfqpoint{0.038889in}{0.010313in}}{\pgfqpoint{0.034791in}{0.020206in}}{\pgfqpoint{0.027499in}{0.027499in}}%
\pgfpathcurveto{\pgfqpoint{0.020206in}{0.034791in}}{\pgfqpoint{0.010313in}{0.038889in}}{\pgfqpoint{0.000000in}{0.038889in}}%
\pgfpathcurveto{\pgfqpoint{-0.010313in}{0.038889in}}{\pgfqpoint{-0.020206in}{0.034791in}}{\pgfqpoint{-0.027499in}{0.027499in}}%
\pgfpathcurveto{\pgfqpoint{-0.034791in}{0.020206in}}{\pgfqpoint{-0.038889in}{0.010313in}}{\pgfqpoint{-0.038889in}{0.000000in}}%
\pgfpathcurveto{\pgfqpoint{-0.038889in}{-0.010313in}}{\pgfqpoint{-0.034791in}{-0.020206in}}{\pgfqpoint{-0.027499in}{-0.027499in}}%
\pgfpathcurveto{\pgfqpoint{-0.020206in}{-0.034791in}}{\pgfqpoint{-0.010313in}{-0.038889in}}{\pgfqpoint{0.000000in}{-0.038889in}}%
\pgfpathclose%
\pgfusepath{stroke,fill}%
}%
\begin{pgfscope}%
\pgfsys@transformshift{4.703787in}{4.833704in}%
\pgfsys@useobject{currentmarker}{}%
\end{pgfscope}%
\end{pgfscope}%
\begin{pgfscope}%
\pgfpathrectangle{\pgfqpoint{0.100000in}{2.413063in}}{\pgfqpoint{5.037500in}{3.427208in}}%
\pgfusepath{clip}%
\pgfsetrectcap%
\pgfsetroundjoin%
\pgfsetlinewidth{1.505625pt}%
\definecolor{currentstroke}{rgb}{0.000000,0.000000,1.000000}%
\pgfsetstrokecolor{currentstroke}%
\pgfsetstrokeopacity{0.500000}%
\pgfsetdash{}{0pt}%
\pgfpathmoveto{\pgfqpoint{4.696218in}{4.777739in}}%
\pgfusepath{stroke}%
\end{pgfscope}%
\begin{pgfscope}%
\pgfpathrectangle{\pgfqpoint{0.100000in}{2.413063in}}{\pgfqpoint{5.037500in}{3.427208in}}%
\pgfusepath{clip}%
\pgfsetbuttcap%
\pgfsetroundjoin%
\definecolor{currentfill}{rgb}{0.000000,0.000000,1.000000}%
\pgfsetfillcolor{currentfill}%
\pgfsetfillopacity{0.500000}%
\pgfsetlinewidth{0.250937pt}%
\definecolor{currentstroke}{rgb}{0.000000,0.000000,0.000000}%
\pgfsetstrokecolor{currentstroke}%
\pgfsetstrokeopacity{0.500000}%
\pgfsetdash{}{0pt}%
\pgfsys@defobject{currentmarker}{\pgfqpoint{-0.036111in}{-0.036111in}}{\pgfqpoint{0.036111in}{0.036111in}}{%
\pgfpathmoveto{\pgfqpoint{0.000000in}{-0.036111in}}%
\pgfpathcurveto{\pgfqpoint{0.009577in}{-0.036111in}}{\pgfqpoint{0.018763in}{-0.032306in}}{\pgfqpoint{0.025534in}{-0.025534in}}%
\pgfpathcurveto{\pgfqpoint{0.032306in}{-0.018763in}}{\pgfqpoint{0.036111in}{-0.009577in}}{\pgfqpoint{0.036111in}{0.000000in}}%
\pgfpathcurveto{\pgfqpoint{0.036111in}{0.009577in}}{\pgfqpoint{0.032306in}{0.018763in}}{\pgfqpoint{0.025534in}{0.025534in}}%
\pgfpathcurveto{\pgfqpoint{0.018763in}{0.032306in}}{\pgfqpoint{0.009577in}{0.036111in}}{\pgfqpoint{0.000000in}{0.036111in}}%
\pgfpathcurveto{\pgfqpoint{-0.009577in}{0.036111in}}{\pgfqpoint{-0.018763in}{0.032306in}}{\pgfqpoint{-0.025534in}{0.025534in}}%
\pgfpathcurveto{\pgfqpoint{-0.032306in}{0.018763in}}{\pgfqpoint{-0.036111in}{0.009577in}}{\pgfqpoint{-0.036111in}{0.000000in}}%
\pgfpathcurveto{\pgfqpoint{-0.036111in}{-0.009577in}}{\pgfqpoint{-0.032306in}{-0.018763in}}{\pgfqpoint{-0.025534in}{-0.025534in}}%
\pgfpathcurveto{\pgfqpoint{-0.018763in}{-0.032306in}}{\pgfqpoint{-0.009577in}{-0.036111in}}{\pgfqpoint{0.000000in}{-0.036111in}}%
\pgfpathclose%
\pgfusepath{stroke,fill}%
}%
\begin{pgfscope}%
\pgfsys@transformshift{4.696218in}{4.777739in}%
\pgfsys@useobject{currentmarker}{}%
\end{pgfscope}%
\end{pgfscope}%
\begin{pgfscope}%
\pgfpathrectangle{\pgfqpoint{0.100000in}{2.413063in}}{\pgfqpoint{5.037500in}{3.427208in}}%
\pgfusepath{clip}%
\pgfsetrectcap%
\pgfsetroundjoin%
\pgfsetlinewidth{1.505625pt}%
\definecolor{currentstroke}{rgb}{0.000000,0.000000,1.000000}%
\pgfsetstrokecolor{currentstroke}%
\pgfsetstrokeopacity{0.500000}%
\pgfsetdash{}{0pt}%
\pgfpathmoveto{\pgfqpoint{4.761265in}{4.819862in}}%
\pgfusepath{stroke}%
\end{pgfscope}%
\begin{pgfscope}%
\pgfpathrectangle{\pgfqpoint{0.100000in}{2.413063in}}{\pgfqpoint{5.037500in}{3.427208in}}%
\pgfusepath{clip}%
\pgfsetbuttcap%
\pgfsetroundjoin%
\definecolor{currentfill}{rgb}{0.000000,0.000000,1.000000}%
\pgfsetfillcolor{currentfill}%
\pgfsetfillopacity{0.500000}%
\pgfsetlinewidth{0.250937pt}%
\definecolor{currentstroke}{rgb}{0.000000,0.000000,0.000000}%
\pgfsetstrokecolor{currentstroke}%
\pgfsetstrokeopacity{0.500000}%
\pgfsetdash{}{0pt}%
\pgfsys@defobject{currentmarker}{\pgfqpoint{-0.050000in}{-0.050000in}}{\pgfqpoint{0.050000in}{0.050000in}}{%
\pgfpathmoveto{\pgfqpoint{0.000000in}{-0.050000in}}%
\pgfpathcurveto{\pgfqpoint{0.013260in}{-0.050000in}}{\pgfqpoint{0.025979in}{-0.044732in}}{\pgfqpoint{0.035355in}{-0.035355in}}%
\pgfpathcurveto{\pgfqpoint{0.044732in}{-0.025979in}}{\pgfqpoint{0.050000in}{-0.013260in}}{\pgfqpoint{0.050000in}{0.000000in}}%
\pgfpathcurveto{\pgfqpoint{0.050000in}{0.013260in}}{\pgfqpoint{0.044732in}{0.025979in}}{\pgfqpoint{0.035355in}{0.035355in}}%
\pgfpathcurveto{\pgfqpoint{0.025979in}{0.044732in}}{\pgfqpoint{0.013260in}{0.050000in}}{\pgfqpoint{0.000000in}{0.050000in}}%
\pgfpathcurveto{\pgfqpoint{-0.013260in}{0.050000in}}{\pgfqpoint{-0.025979in}{0.044732in}}{\pgfqpoint{-0.035355in}{0.035355in}}%
\pgfpathcurveto{\pgfqpoint{-0.044732in}{0.025979in}}{\pgfqpoint{-0.050000in}{0.013260in}}{\pgfqpoint{-0.050000in}{0.000000in}}%
\pgfpathcurveto{\pgfqpoint{-0.050000in}{-0.013260in}}{\pgfqpoint{-0.044732in}{-0.025979in}}{\pgfqpoint{-0.035355in}{-0.035355in}}%
\pgfpathcurveto{\pgfqpoint{-0.025979in}{-0.044732in}}{\pgfqpoint{-0.013260in}{-0.050000in}}{\pgfqpoint{0.000000in}{-0.050000in}}%
\pgfpathclose%
\pgfusepath{stroke,fill}%
}%
\begin{pgfscope}%
\pgfsys@transformshift{4.761265in}{4.819862in}%
\pgfsys@useobject{currentmarker}{}%
\end{pgfscope}%
\end{pgfscope}%
\begin{pgfscope}%
\pgfpathrectangle{\pgfqpoint{0.100000in}{2.413063in}}{\pgfqpoint{5.037500in}{3.427208in}}%
\pgfusepath{clip}%
\pgfsetrectcap%
\pgfsetroundjoin%
\pgfsetlinewidth{1.505625pt}%
\definecolor{currentstroke}{rgb}{0.000000,0.000000,1.000000}%
\pgfsetstrokecolor{currentstroke}%
\pgfsetstrokeopacity{0.500000}%
\pgfsetdash{}{0pt}%
\pgfpathmoveto{\pgfqpoint{4.678703in}{4.803082in}}%
\pgfusepath{stroke}%
\end{pgfscope}%
\begin{pgfscope}%
\pgfpathrectangle{\pgfqpoint{0.100000in}{2.413063in}}{\pgfqpoint{5.037500in}{3.427208in}}%
\pgfusepath{clip}%
\pgfsetbuttcap%
\pgfsetroundjoin%
\definecolor{currentfill}{rgb}{0.000000,0.000000,1.000000}%
\pgfsetfillcolor{currentfill}%
\pgfsetfillopacity{0.500000}%
\pgfsetlinewidth{0.250937pt}%
\definecolor{currentstroke}{rgb}{0.000000,0.000000,0.000000}%
\pgfsetstrokecolor{currentstroke}%
\pgfsetstrokeopacity{0.500000}%
\pgfsetdash{}{0pt}%
\pgfsys@defobject{currentmarker}{\pgfqpoint{-0.047222in}{-0.047222in}}{\pgfqpoint{0.047222in}{0.047222in}}{%
\pgfpathmoveto{\pgfqpoint{0.000000in}{-0.047222in}}%
\pgfpathcurveto{\pgfqpoint{0.012523in}{-0.047222in}}{\pgfqpoint{0.024536in}{-0.042247in}}{\pgfqpoint{0.033391in}{-0.033391in}}%
\pgfpathcurveto{\pgfqpoint{0.042247in}{-0.024536in}}{\pgfqpoint{0.047222in}{-0.012523in}}{\pgfqpoint{0.047222in}{0.000000in}}%
\pgfpathcurveto{\pgfqpoint{0.047222in}{0.012523in}}{\pgfqpoint{0.042247in}{0.024536in}}{\pgfqpoint{0.033391in}{0.033391in}}%
\pgfpathcurveto{\pgfqpoint{0.024536in}{0.042247in}}{\pgfqpoint{0.012523in}{0.047222in}}{\pgfqpoint{0.000000in}{0.047222in}}%
\pgfpathcurveto{\pgfqpoint{-0.012523in}{0.047222in}}{\pgfqpoint{-0.024536in}{0.042247in}}{\pgfqpoint{-0.033391in}{0.033391in}}%
\pgfpathcurveto{\pgfqpoint{-0.042247in}{0.024536in}}{\pgfqpoint{-0.047222in}{0.012523in}}{\pgfqpoint{-0.047222in}{0.000000in}}%
\pgfpathcurveto{\pgfqpoint{-0.047222in}{-0.012523in}}{\pgfqpoint{-0.042247in}{-0.024536in}}{\pgfqpoint{-0.033391in}{-0.033391in}}%
\pgfpathcurveto{\pgfqpoint{-0.024536in}{-0.042247in}}{\pgfqpoint{-0.012523in}{-0.047222in}}{\pgfqpoint{0.000000in}{-0.047222in}}%
\pgfpathclose%
\pgfusepath{stroke,fill}%
}%
\begin{pgfscope}%
\pgfsys@transformshift{4.678703in}{4.803082in}%
\pgfsys@useobject{currentmarker}{}%
\end{pgfscope}%
\end{pgfscope}%
\begin{pgfscope}%
\pgfpathrectangle{\pgfqpoint{0.100000in}{2.413063in}}{\pgfqpoint{5.037500in}{3.427208in}}%
\pgfusepath{clip}%
\pgfsetrectcap%
\pgfsetroundjoin%
\pgfsetlinewidth{1.505625pt}%
\definecolor{currentstroke}{rgb}{0.000000,0.000000,1.000000}%
\pgfsetstrokecolor{currentstroke}%
\pgfsetstrokeopacity{0.500000}%
\pgfsetdash{}{0pt}%
\pgfpathmoveto{\pgfqpoint{4.529746in}{4.484780in}}%
\pgfusepath{stroke}%
\end{pgfscope}%
\begin{pgfscope}%
\pgfpathrectangle{\pgfqpoint{0.100000in}{2.413063in}}{\pgfqpoint{5.037500in}{3.427208in}}%
\pgfusepath{clip}%
\pgfsetbuttcap%
\pgfsetroundjoin%
\definecolor{currentfill}{rgb}{0.000000,0.000000,1.000000}%
\pgfsetfillcolor{currentfill}%
\pgfsetfillopacity{0.500000}%
\pgfsetlinewidth{0.250937pt}%
\definecolor{currentstroke}{rgb}{0.000000,0.000000,0.000000}%
\pgfsetstrokecolor{currentstroke}%
\pgfsetstrokeopacity{0.500000}%
\pgfsetdash{}{0pt}%
\pgfsys@defobject{currentmarker}{\pgfqpoint{-0.027778in}{-0.027778in}}{\pgfqpoint{0.027778in}{0.027778in}}{%
\pgfpathmoveto{\pgfqpoint{0.000000in}{-0.027778in}}%
\pgfpathcurveto{\pgfqpoint{0.007367in}{-0.027778in}}{\pgfqpoint{0.014433in}{-0.024851in}}{\pgfqpoint{0.019642in}{-0.019642in}}%
\pgfpathcurveto{\pgfqpoint{0.024851in}{-0.014433in}}{\pgfqpoint{0.027778in}{-0.007367in}}{\pgfqpoint{0.027778in}{0.000000in}}%
\pgfpathcurveto{\pgfqpoint{0.027778in}{0.007367in}}{\pgfqpoint{0.024851in}{0.014433in}}{\pgfqpoint{0.019642in}{0.019642in}}%
\pgfpathcurveto{\pgfqpoint{0.014433in}{0.024851in}}{\pgfqpoint{0.007367in}{0.027778in}}{\pgfqpoint{0.000000in}{0.027778in}}%
\pgfpathcurveto{\pgfqpoint{-0.007367in}{0.027778in}}{\pgfqpoint{-0.014433in}{0.024851in}}{\pgfqpoint{-0.019642in}{0.019642in}}%
\pgfpathcurveto{\pgfqpoint{-0.024851in}{0.014433in}}{\pgfqpoint{-0.027778in}{0.007367in}}{\pgfqpoint{-0.027778in}{0.000000in}}%
\pgfpathcurveto{\pgfqpoint{-0.027778in}{-0.007367in}}{\pgfqpoint{-0.024851in}{-0.014433in}}{\pgfqpoint{-0.019642in}{-0.019642in}}%
\pgfpathcurveto{\pgfqpoint{-0.014433in}{-0.024851in}}{\pgfqpoint{-0.007367in}{-0.027778in}}{\pgfqpoint{0.000000in}{-0.027778in}}%
\pgfpathclose%
\pgfusepath{stroke,fill}%
}%
\begin{pgfscope}%
\pgfsys@transformshift{4.529746in}{4.484780in}%
\pgfsys@useobject{currentmarker}{}%
\end{pgfscope}%
\end{pgfscope}%
\begin{pgfscope}%
\pgfpathrectangle{\pgfqpoint{0.100000in}{2.413063in}}{\pgfqpoint{5.037500in}{3.427208in}}%
\pgfusepath{clip}%
\pgfsetrectcap%
\pgfsetroundjoin%
\pgfsetlinewidth{1.505625pt}%
\definecolor{currentstroke}{rgb}{0.678431,1.000000,0.184314}%
\pgfsetstrokecolor{currentstroke}%
\pgfsetstrokeopacity{0.500000}%
\pgfsetdash{}{0pt}%
\pgfpathmoveto{\pgfqpoint{4.542388in}{4.394206in}}%
\pgfusepath{stroke}%
\end{pgfscope}%
\begin{pgfscope}%
\pgfpathrectangle{\pgfqpoint{0.100000in}{2.413063in}}{\pgfqpoint{5.037500in}{3.427208in}}%
\pgfusepath{clip}%
\pgfsetbuttcap%
\pgfsetroundjoin%
\definecolor{currentfill}{rgb}{0.678431,1.000000,0.184314}%
\pgfsetfillcolor{currentfill}%
\pgfsetfillopacity{0.500000}%
\pgfsetlinewidth{0.250937pt}%
\definecolor{currentstroke}{rgb}{0.000000,0.000000,0.000000}%
\pgfsetstrokecolor{currentstroke}%
\pgfsetstrokeopacity{0.500000}%
\pgfsetdash{}{0pt}%
\pgfsys@defobject{currentmarker}{\pgfqpoint{-0.005556in}{-0.005556in}}{\pgfqpoint{0.005556in}{0.005556in}}{%
\pgfpathmoveto{\pgfqpoint{0.000000in}{-0.005556in}}%
\pgfpathcurveto{\pgfqpoint{0.001473in}{-0.005556in}}{\pgfqpoint{0.002887in}{-0.004970in}}{\pgfqpoint{0.003928in}{-0.003928in}}%
\pgfpathcurveto{\pgfqpoint{0.004970in}{-0.002887in}}{\pgfqpoint{0.005556in}{-0.001473in}}{\pgfqpoint{0.005556in}{0.000000in}}%
\pgfpathcurveto{\pgfqpoint{0.005556in}{0.001473in}}{\pgfqpoint{0.004970in}{0.002887in}}{\pgfqpoint{0.003928in}{0.003928in}}%
\pgfpathcurveto{\pgfqpoint{0.002887in}{0.004970in}}{\pgfqpoint{0.001473in}{0.005556in}}{\pgfqpoint{0.000000in}{0.005556in}}%
\pgfpathcurveto{\pgfqpoint{-0.001473in}{0.005556in}}{\pgfqpoint{-0.002887in}{0.004970in}}{\pgfqpoint{-0.003928in}{0.003928in}}%
\pgfpathcurveto{\pgfqpoint{-0.004970in}{0.002887in}}{\pgfqpoint{-0.005556in}{0.001473in}}{\pgfqpoint{-0.005556in}{0.000000in}}%
\pgfpathcurveto{\pgfqpoint{-0.005556in}{-0.001473in}}{\pgfqpoint{-0.004970in}{-0.002887in}}{\pgfqpoint{-0.003928in}{-0.003928in}}%
\pgfpathcurveto{\pgfqpoint{-0.002887in}{-0.004970in}}{\pgfqpoint{-0.001473in}{-0.005556in}}{\pgfqpoint{0.000000in}{-0.005556in}}%
\pgfpathclose%
\pgfusepath{stroke,fill}%
}%
\begin{pgfscope}%
\pgfsys@transformshift{4.542388in}{4.394206in}%
\pgfsys@useobject{currentmarker}{}%
\end{pgfscope}%
\end{pgfscope}%
\begin{pgfscope}%
\pgfpathrectangle{\pgfqpoint{0.100000in}{2.413063in}}{\pgfqpoint{5.037500in}{3.427208in}}%
\pgfusepath{clip}%
\pgfsetrectcap%
\pgfsetroundjoin%
\pgfsetlinewidth{1.505625pt}%
\definecolor{currentstroke}{rgb}{0.000000,0.000000,1.000000}%
\pgfsetstrokecolor{currentstroke}%
\pgfsetstrokeopacity{0.500000}%
\pgfsetdash{}{0pt}%
\pgfpathmoveto{\pgfqpoint{4.403529in}{4.427445in}}%
\pgfusepath{stroke}%
\end{pgfscope}%
\begin{pgfscope}%
\pgfpathrectangle{\pgfqpoint{0.100000in}{2.413063in}}{\pgfqpoint{5.037500in}{3.427208in}}%
\pgfusepath{clip}%
\pgfsetbuttcap%
\pgfsetroundjoin%
\definecolor{currentfill}{rgb}{0.000000,0.000000,1.000000}%
\pgfsetfillcolor{currentfill}%
\pgfsetfillopacity{0.500000}%
\pgfsetlinewidth{0.250937pt}%
\definecolor{currentstroke}{rgb}{0.000000,0.000000,0.000000}%
\pgfsetstrokecolor{currentstroke}%
\pgfsetstrokeopacity{0.500000}%
\pgfsetdash{}{0pt}%
\pgfsys@defobject{currentmarker}{\pgfqpoint{-0.022222in}{-0.022222in}}{\pgfqpoint{0.022222in}{0.022222in}}{%
\pgfpathmoveto{\pgfqpoint{0.000000in}{-0.022222in}}%
\pgfpathcurveto{\pgfqpoint{0.005893in}{-0.022222in}}{\pgfqpoint{0.011546in}{-0.019881in}}{\pgfqpoint{0.015713in}{-0.015713in}}%
\pgfpathcurveto{\pgfqpoint{0.019881in}{-0.011546in}}{\pgfqpoint{0.022222in}{-0.005893in}}{\pgfqpoint{0.022222in}{0.000000in}}%
\pgfpathcurveto{\pgfqpoint{0.022222in}{0.005893in}}{\pgfqpoint{0.019881in}{0.011546in}}{\pgfqpoint{0.015713in}{0.015713in}}%
\pgfpathcurveto{\pgfqpoint{0.011546in}{0.019881in}}{\pgfqpoint{0.005893in}{0.022222in}}{\pgfqpoint{0.000000in}{0.022222in}}%
\pgfpathcurveto{\pgfqpoint{-0.005893in}{0.022222in}}{\pgfqpoint{-0.011546in}{0.019881in}}{\pgfqpoint{-0.015713in}{0.015713in}}%
\pgfpathcurveto{\pgfqpoint{-0.019881in}{0.011546in}}{\pgfqpoint{-0.022222in}{0.005893in}}{\pgfqpoint{-0.022222in}{0.000000in}}%
\pgfpathcurveto{\pgfqpoint{-0.022222in}{-0.005893in}}{\pgfqpoint{-0.019881in}{-0.011546in}}{\pgfqpoint{-0.015713in}{-0.015713in}}%
\pgfpathcurveto{\pgfqpoint{-0.011546in}{-0.019881in}}{\pgfqpoint{-0.005893in}{-0.022222in}}{\pgfqpoint{0.000000in}{-0.022222in}}%
\pgfpathclose%
\pgfusepath{stroke,fill}%
}%
\begin{pgfscope}%
\pgfsys@transformshift{4.403529in}{4.427445in}%
\pgfsys@useobject{currentmarker}{}%
\end{pgfscope}%
\end{pgfscope}%
\begin{pgfscope}%
\pgfpathrectangle{\pgfqpoint{0.100000in}{2.413063in}}{\pgfqpoint{5.037500in}{3.427208in}}%
\pgfusepath{clip}%
\pgfsetrectcap%
\pgfsetroundjoin%
\pgfsetlinewidth{1.505625pt}%
\definecolor{currentstroke}{rgb}{0.000000,0.000000,1.000000}%
\pgfsetstrokecolor{currentstroke}%
\pgfsetstrokeopacity{0.500000}%
\pgfsetdash{}{0pt}%
\pgfpathmoveto{\pgfqpoint{4.170525in}{2.940820in}}%
\pgfusepath{stroke}%
\end{pgfscope}%
\begin{pgfscope}%
\pgfpathrectangle{\pgfqpoint{0.100000in}{2.413063in}}{\pgfqpoint{5.037500in}{3.427208in}}%
\pgfusepath{clip}%
\pgfsetbuttcap%
\pgfsetroundjoin%
\definecolor{currentfill}{rgb}{0.000000,0.000000,1.000000}%
\pgfsetfillcolor{currentfill}%
\pgfsetfillopacity{0.500000}%
\pgfsetlinewidth{0.250937pt}%
\definecolor{currentstroke}{rgb}{0.000000,0.000000,0.000000}%
\pgfsetstrokecolor{currentstroke}%
\pgfsetstrokeopacity{0.500000}%
\pgfsetdash{}{0pt}%
\pgfsys@defobject{currentmarker}{\pgfqpoint{-0.008333in}{-0.008333in}}{\pgfqpoint{0.008333in}{0.008333in}}{%
\pgfpathmoveto{\pgfqpoint{0.000000in}{-0.008333in}}%
\pgfpathcurveto{\pgfqpoint{0.002210in}{-0.008333in}}{\pgfqpoint{0.004330in}{-0.007455in}}{\pgfqpoint{0.005893in}{-0.005893in}}%
\pgfpathcurveto{\pgfqpoint{0.007455in}{-0.004330in}}{\pgfqpoint{0.008333in}{-0.002210in}}{\pgfqpoint{0.008333in}{0.000000in}}%
\pgfpathcurveto{\pgfqpoint{0.008333in}{0.002210in}}{\pgfqpoint{0.007455in}{0.004330in}}{\pgfqpoint{0.005893in}{0.005893in}}%
\pgfpathcurveto{\pgfqpoint{0.004330in}{0.007455in}}{\pgfqpoint{0.002210in}{0.008333in}}{\pgfqpoint{0.000000in}{0.008333in}}%
\pgfpathcurveto{\pgfqpoint{-0.002210in}{0.008333in}}{\pgfqpoint{-0.004330in}{0.007455in}}{\pgfqpoint{-0.005893in}{0.005893in}}%
\pgfpathcurveto{\pgfqpoint{-0.007455in}{0.004330in}}{\pgfqpoint{-0.008333in}{0.002210in}}{\pgfqpoint{-0.008333in}{0.000000in}}%
\pgfpathcurveto{\pgfqpoint{-0.008333in}{-0.002210in}}{\pgfqpoint{-0.007455in}{-0.004330in}}{\pgfqpoint{-0.005893in}{-0.005893in}}%
\pgfpathcurveto{\pgfqpoint{-0.004330in}{-0.007455in}}{\pgfqpoint{-0.002210in}{-0.008333in}}{\pgfqpoint{0.000000in}{-0.008333in}}%
\pgfpathclose%
\pgfusepath{stroke,fill}%
}%
\begin{pgfscope}%
\pgfsys@transformshift{4.170525in}{2.940820in}%
\pgfsys@useobject{currentmarker}{}%
\end{pgfscope}%
\end{pgfscope}%
\begin{pgfscope}%
\pgfpathrectangle{\pgfqpoint{0.100000in}{2.413063in}}{\pgfqpoint{5.037500in}{3.427208in}}%
\pgfusepath{clip}%
\pgfsetrectcap%
\pgfsetroundjoin%
\pgfsetlinewidth{1.505625pt}%
\definecolor{currentstroke}{rgb}{0.501961,0.501961,0.501961}%
\pgfsetstrokecolor{currentstroke}%
\pgfsetstrokeopacity{0.500000}%
\pgfsetdash{}{0pt}%
\pgfpathmoveto{\pgfqpoint{3.645536in}{3.368139in}}%
\pgfusepath{stroke}%
\end{pgfscope}%
\begin{pgfscope}%
\pgfpathrectangle{\pgfqpoint{0.100000in}{2.413063in}}{\pgfqpoint{5.037500in}{3.427208in}}%
\pgfusepath{clip}%
\pgfsetbuttcap%
\pgfsetroundjoin%
\definecolor{currentfill}{rgb}{0.501961,0.501961,0.501961}%
\pgfsetfillcolor{currentfill}%
\pgfsetfillopacity{0.500000}%
\pgfsetlinewidth{0.250937pt}%
\definecolor{currentstroke}{rgb}{0.000000,0.000000,0.000000}%
\pgfsetstrokecolor{currentstroke}%
\pgfsetstrokeopacity{0.500000}%
\pgfsetdash{}{0pt}%
\pgfsys@defobject{currentmarker}{\pgfqpoint{-0.013889in}{-0.013889in}}{\pgfqpoint{0.013889in}{0.013889in}}{%
\pgfpathmoveto{\pgfqpoint{0.000000in}{-0.013889in}}%
\pgfpathcurveto{\pgfqpoint{0.003683in}{-0.013889in}}{\pgfqpoint{0.007216in}{-0.012425in}}{\pgfqpoint{0.009821in}{-0.009821in}}%
\pgfpathcurveto{\pgfqpoint{0.012425in}{-0.007216in}}{\pgfqpoint{0.013889in}{-0.003683in}}{\pgfqpoint{0.013889in}{0.000000in}}%
\pgfpathcurveto{\pgfqpoint{0.013889in}{0.003683in}}{\pgfqpoint{0.012425in}{0.007216in}}{\pgfqpoint{0.009821in}{0.009821in}}%
\pgfpathcurveto{\pgfqpoint{0.007216in}{0.012425in}}{\pgfqpoint{0.003683in}{0.013889in}}{\pgfqpoint{0.000000in}{0.013889in}}%
\pgfpathcurveto{\pgfqpoint{-0.003683in}{0.013889in}}{\pgfqpoint{-0.007216in}{0.012425in}}{\pgfqpoint{-0.009821in}{0.009821in}}%
\pgfpathcurveto{\pgfqpoint{-0.012425in}{0.007216in}}{\pgfqpoint{-0.013889in}{0.003683in}}{\pgfqpoint{-0.013889in}{0.000000in}}%
\pgfpathcurveto{\pgfqpoint{-0.013889in}{-0.003683in}}{\pgfqpoint{-0.012425in}{-0.007216in}}{\pgfqpoint{-0.009821in}{-0.009821in}}%
\pgfpathcurveto{\pgfqpoint{-0.007216in}{-0.012425in}}{\pgfqpoint{-0.003683in}{-0.013889in}}{\pgfqpoint{0.000000in}{-0.013889in}}%
\pgfpathclose%
\pgfusepath{stroke,fill}%
}%
\begin{pgfscope}%
\pgfsys@transformshift{3.645536in}{3.368139in}%
\pgfsys@useobject{currentmarker}{}%
\end{pgfscope}%
\end{pgfscope}%
\begin{pgfscope}%
\pgfpathrectangle{\pgfqpoint{0.100000in}{2.413063in}}{\pgfqpoint{5.037500in}{3.427208in}}%
\pgfusepath{clip}%
\pgfsetrectcap%
\pgfsetroundjoin%
\pgfsetlinewidth{1.505625pt}%
\definecolor{currentstroke}{rgb}{0.501961,0.501961,0.501961}%
\pgfsetstrokecolor{currentstroke}%
\pgfsetstrokeopacity{0.500000}%
\pgfsetdash{}{0pt}%
\pgfpathmoveto{\pgfqpoint{3.644545in}{3.326395in}}%
\pgfusepath{stroke}%
\end{pgfscope}%
\begin{pgfscope}%
\pgfpathrectangle{\pgfqpoint{0.100000in}{2.413063in}}{\pgfqpoint{5.037500in}{3.427208in}}%
\pgfusepath{clip}%
\pgfsetbuttcap%
\pgfsetroundjoin%
\definecolor{currentfill}{rgb}{0.501961,0.501961,0.501961}%
\pgfsetfillcolor{currentfill}%
\pgfsetfillopacity{0.500000}%
\pgfsetlinewidth{0.250937pt}%
\definecolor{currentstroke}{rgb}{0.000000,0.000000,0.000000}%
\pgfsetstrokecolor{currentstroke}%
\pgfsetstrokeopacity{0.500000}%
\pgfsetdash{}{0pt}%
\pgfsys@defobject{currentmarker}{\pgfqpoint{-0.013889in}{-0.013889in}}{\pgfqpoint{0.013889in}{0.013889in}}{%
\pgfpathmoveto{\pgfqpoint{0.000000in}{-0.013889in}}%
\pgfpathcurveto{\pgfqpoint{0.003683in}{-0.013889in}}{\pgfqpoint{0.007216in}{-0.012425in}}{\pgfqpoint{0.009821in}{-0.009821in}}%
\pgfpathcurveto{\pgfqpoint{0.012425in}{-0.007216in}}{\pgfqpoint{0.013889in}{-0.003683in}}{\pgfqpoint{0.013889in}{0.000000in}}%
\pgfpathcurveto{\pgfqpoint{0.013889in}{0.003683in}}{\pgfqpoint{0.012425in}{0.007216in}}{\pgfqpoint{0.009821in}{0.009821in}}%
\pgfpathcurveto{\pgfqpoint{0.007216in}{0.012425in}}{\pgfqpoint{0.003683in}{0.013889in}}{\pgfqpoint{0.000000in}{0.013889in}}%
\pgfpathcurveto{\pgfqpoint{-0.003683in}{0.013889in}}{\pgfqpoint{-0.007216in}{0.012425in}}{\pgfqpoint{-0.009821in}{0.009821in}}%
\pgfpathcurveto{\pgfqpoint{-0.012425in}{0.007216in}}{\pgfqpoint{-0.013889in}{0.003683in}}{\pgfqpoint{-0.013889in}{0.000000in}}%
\pgfpathcurveto{\pgfqpoint{-0.013889in}{-0.003683in}}{\pgfqpoint{-0.012425in}{-0.007216in}}{\pgfqpoint{-0.009821in}{-0.009821in}}%
\pgfpathcurveto{\pgfqpoint{-0.007216in}{-0.012425in}}{\pgfqpoint{-0.003683in}{-0.013889in}}{\pgfqpoint{0.000000in}{-0.013889in}}%
\pgfpathclose%
\pgfusepath{stroke,fill}%
}%
\begin{pgfscope}%
\pgfsys@transformshift{3.644545in}{3.326395in}%
\pgfsys@useobject{currentmarker}{}%
\end{pgfscope}%
\end{pgfscope}%
\begin{pgfscope}%
\pgfpathrectangle{\pgfqpoint{0.100000in}{2.413063in}}{\pgfqpoint{5.037500in}{3.427208in}}%
\pgfusepath{clip}%
\pgfsetrectcap%
\pgfsetroundjoin%
\pgfsetlinewidth{1.505625pt}%
\definecolor{currentstroke}{rgb}{0.000000,0.000000,1.000000}%
\pgfsetstrokecolor{currentstroke}%
\pgfsetstrokeopacity{0.500000}%
\pgfsetdash{}{0pt}%
\pgfpathmoveto{\pgfqpoint{4.204682in}{3.218061in}}%
\pgfusepath{stroke}%
\end{pgfscope}%
\begin{pgfscope}%
\pgfpathrectangle{\pgfqpoint{0.100000in}{2.413063in}}{\pgfqpoint{5.037500in}{3.427208in}}%
\pgfusepath{clip}%
\pgfsetbuttcap%
\pgfsetroundjoin%
\definecolor{currentfill}{rgb}{0.000000,0.000000,1.000000}%
\pgfsetfillcolor{currentfill}%
\pgfsetfillopacity{0.500000}%
\pgfsetlinewidth{0.250937pt}%
\definecolor{currentstroke}{rgb}{0.000000,0.000000,0.000000}%
\pgfsetstrokecolor{currentstroke}%
\pgfsetstrokeopacity{0.500000}%
\pgfsetdash{}{0pt}%
\pgfsys@defobject{currentmarker}{\pgfqpoint{-0.008333in}{-0.008333in}}{\pgfqpoint{0.008333in}{0.008333in}}{%
\pgfpathmoveto{\pgfqpoint{0.000000in}{-0.008333in}}%
\pgfpathcurveto{\pgfqpoint{0.002210in}{-0.008333in}}{\pgfqpoint{0.004330in}{-0.007455in}}{\pgfqpoint{0.005893in}{-0.005893in}}%
\pgfpathcurveto{\pgfqpoint{0.007455in}{-0.004330in}}{\pgfqpoint{0.008333in}{-0.002210in}}{\pgfqpoint{0.008333in}{0.000000in}}%
\pgfpathcurveto{\pgfqpoint{0.008333in}{0.002210in}}{\pgfqpoint{0.007455in}{0.004330in}}{\pgfqpoint{0.005893in}{0.005893in}}%
\pgfpathcurveto{\pgfqpoint{0.004330in}{0.007455in}}{\pgfqpoint{0.002210in}{0.008333in}}{\pgfqpoint{0.000000in}{0.008333in}}%
\pgfpathcurveto{\pgfqpoint{-0.002210in}{0.008333in}}{\pgfqpoint{-0.004330in}{0.007455in}}{\pgfqpoint{-0.005893in}{0.005893in}}%
\pgfpathcurveto{\pgfqpoint{-0.007455in}{0.004330in}}{\pgfqpoint{-0.008333in}{0.002210in}}{\pgfqpoint{-0.008333in}{0.000000in}}%
\pgfpathcurveto{\pgfqpoint{-0.008333in}{-0.002210in}}{\pgfqpoint{-0.007455in}{-0.004330in}}{\pgfqpoint{-0.005893in}{-0.005893in}}%
\pgfpathcurveto{\pgfqpoint{-0.004330in}{-0.007455in}}{\pgfqpoint{-0.002210in}{-0.008333in}}{\pgfqpoint{0.000000in}{-0.008333in}}%
\pgfpathclose%
\pgfusepath{stroke,fill}%
}%
\begin{pgfscope}%
\pgfsys@transformshift{4.204682in}{3.218061in}%
\pgfsys@useobject{currentmarker}{}%
\end{pgfscope}%
\end{pgfscope}%
\begin{pgfscope}%
\pgfpathrectangle{\pgfqpoint{0.100000in}{2.413063in}}{\pgfqpoint{5.037500in}{3.427208in}}%
\pgfusepath{clip}%
\pgfsetrectcap%
\pgfsetroundjoin%
\pgfsetlinewidth{1.505625pt}%
\definecolor{currentstroke}{rgb}{0.000000,0.000000,1.000000}%
\pgfsetstrokecolor{currentstroke}%
\pgfsetstrokeopacity{0.500000}%
\pgfsetdash{}{0pt}%
\pgfpathmoveto{\pgfqpoint{4.085003in}{3.289230in}}%
\pgfusepath{stroke}%
\end{pgfscope}%
\begin{pgfscope}%
\pgfpathrectangle{\pgfqpoint{0.100000in}{2.413063in}}{\pgfqpoint{5.037500in}{3.427208in}}%
\pgfusepath{clip}%
\pgfsetbuttcap%
\pgfsetroundjoin%
\definecolor{currentfill}{rgb}{0.000000,0.000000,1.000000}%
\pgfsetfillcolor{currentfill}%
\pgfsetfillopacity{0.500000}%
\pgfsetlinewidth{0.250937pt}%
\definecolor{currentstroke}{rgb}{0.000000,0.000000,0.000000}%
\pgfsetstrokecolor{currentstroke}%
\pgfsetstrokeopacity{0.500000}%
\pgfsetdash{}{0pt}%
\pgfsys@defobject{currentmarker}{\pgfqpoint{-0.008333in}{-0.008333in}}{\pgfqpoint{0.008333in}{0.008333in}}{%
\pgfpathmoveto{\pgfqpoint{0.000000in}{-0.008333in}}%
\pgfpathcurveto{\pgfqpoint{0.002210in}{-0.008333in}}{\pgfqpoint{0.004330in}{-0.007455in}}{\pgfqpoint{0.005893in}{-0.005893in}}%
\pgfpathcurveto{\pgfqpoint{0.007455in}{-0.004330in}}{\pgfqpoint{0.008333in}{-0.002210in}}{\pgfqpoint{0.008333in}{0.000000in}}%
\pgfpathcurveto{\pgfqpoint{0.008333in}{0.002210in}}{\pgfqpoint{0.007455in}{0.004330in}}{\pgfqpoint{0.005893in}{0.005893in}}%
\pgfpathcurveto{\pgfqpoint{0.004330in}{0.007455in}}{\pgfqpoint{0.002210in}{0.008333in}}{\pgfqpoint{0.000000in}{0.008333in}}%
\pgfpathcurveto{\pgfqpoint{-0.002210in}{0.008333in}}{\pgfqpoint{-0.004330in}{0.007455in}}{\pgfqpoint{-0.005893in}{0.005893in}}%
\pgfpathcurveto{\pgfqpoint{-0.007455in}{0.004330in}}{\pgfqpoint{-0.008333in}{0.002210in}}{\pgfqpoint{-0.008333in}{0.000000in}}%
\pgfpathcurveto{\pgfqpoint{-0.008333in}{-0.002210in}}{\pgfqpoint{-0.007455in}{-0.004330in}}{\pgfqpoint{-0.005893in}{-0.005893in}}%
\pgfpathcurveto{\pgfqpoint{-0.004330in}{-0.007455in}}{\pgfqpoint{-0.002210in}{-0.008333in}}{\pgfqpoint{0.000000in}{-0.008333in}}%
\pgfpathclose%
\pgfusepath{stroke,fill}%
}%
\begin{pgfscope}%
\pgfsys@transformshift{4.085003in}{3.289230in}%
\pgfsys@useobject{currentmarker}{}%
\end{pgfscope}%
\end{pgfscope}%
\begin{pgfscope}%
\pgfpathrectangle{\pgfqpoint{0.100000in}{2.413063in}}{\pgfqpoint{5.037500in}{3.427208in}}%
\pgfusepath{clip}%
\pgfsetrectcap%
\pgfsetroundjoin%
\pgfsetlinewidth{1.505625pt}%
\definecolor{currentstroke}{rgb}{0.501961,0.501961,0.501961}%
\pgfsetstrokecolor{currentstroke}%
\pgfsetstrokeopacity{0.500000}%
\pgfsetdash{}{0pt}%
\pgfpathmoveto{\pgfqpoint{4.073308in}{3.187567in}}%
\pgfusepath{stroke}%
\end{pgfscope}%
\begin{pgfscope}%
\pgfpathrectangle{\pgfqpoint{0.100000in}{2.413063in}}{\pgfqpoint{5.037500in}{3.427208in}}%
\pgfusepath{clip}%
\pgfsetbuttcap%
\pgfsetroundjoin%
\definecolor{currentfill}{rgb}{0.501961,0.501961,0.501961}%
\pgfsetfillcolor{currentfill}%
\pgfsetfillopacity{0.500000}%
\pgfsetlinewidth{0.250937pt}%
\definecolor{currentstroke}{rgb}{0.000000,0.000000,0.000000}%
\pgfsetstrokecolor{currentstroke}%
\pgfsetstrokeopacity{0.500000}%
\pgfsetdash{}{0pt}%
\pgfsys@defobject{currentmarker}{\pgfqpoint{-0.013889in}{-0.013889in}}{\pgfqpoint{0.013889in}{0.013889in}}{%
\pgfpathmoveto{\pgfqpoint{0.000000in}{-0.013889in}}%
\pgfpathcurveto{\pgfqpoint{0.003683in}{-0.013889in}}{\pgfqpoint{0.007216in}{-0.012425in}}{\pgfqpoint{0.009821in}{-0.009821in}}%
\pgfpathcurveto{\pgfqpoint{0.012425in}{-0.007216in}}{\pgfqpoint{0.013889in}{-0.003683in}}{\pgfqpoint{0.013889in}{0.000000in}}%
\pgfpathcurveto{\pgfqpoint{0.013889in}{0.003683in}}{\pgfqpoint{0.012425in}{0.007216in}}{\pgfqpoint{0.009821in}{0.009821in}}%
\pgfpathcurveto{\pgfqpoint{0.007216in}{0.012425in}}{\pgfqpoint{0.003683in}{0.013889in}}{\pgfqpoint{0.000000in}{0.013889in}}%
\pgfpathcurveto{\pgfqpoint{-0.003683in}{0.013889in}}{\pgfqpoint{-0.007216in}{0.012425in}}{\pgfqpoint{-0.009821in}{0.009821in}}%
\pgfpathcurveto{\pgfqpoint{-0.012425in}{0.007216in}}{\pgfqpoint{-0.013889in}{0.003683in}}{\pgfqpoint{-0.013889in}{0.000000in}}%
\pgfpathcurveto{\pgfqpoint{-0.013889in}{-0.003683in}}{\pgfqpoint{-0.012425in}{-0.007216in}}{\pgfqpoint{-0.009821in}{-0.009821in}}%
\pgfpathcurveto{\pgfqpoint{-0.007216in}{-0.012425in}}{\pgfqpoint{-0.003683in}{-0.013889in}}{\pgfqpoint{0.000000in}{-0.013889in}}%
\pgfpathclose%
\pgfusepath{stroke,fill}%
}%
\begin{pgfscope}%
\pgfsys@transformshift{4.073308in}{3.187567in}%
\pgfsys@useobject{currentmarker}{}%
\end{pgfscope}%
\end{pgfscope}%
\begin{pgfscope}%
\pgfpathrectangle{\pgfqpoint{0.100000in}{2.413063in}}{\pgfqpoint{5.037500in}{3.427208in}}%
\pgfusepath{clip}%
\pgfsetrectcap%
\pgfsetroundjoin%
\pgfsetlinewidth{1.505625pt}%
\definecolor{currentstroke}{rgb}{0.000000,0.000000,1.000000}%
\pgfsetstrokecolor{currentstroke}%
\pgfsetstrokeopacity{0.500000}%
\pgfsetdash{}{0pt}%
\pgfpathmoveto{\pgfqpoint{4.140568in}{3.377345in}}%
\pgfusepath{stroke}%
\end{pgfscope}%
\begin{pgfscope}%
\pgfpathrectangle{\pgfqpoint{0.100000in}{2.413063in}}{\pgfqpoint{5.037500in}{3.427208in}}%
\pgfusepath{clip}%
\pgfsetbuttcap%
\pgfsetroundjoin%
\definecolor{currentfill}{rgb}{0.000000,0.000000,1.000000}%
\pgfsetfillcolor{currentfill}%
\pgfsetfillopacity{0.500000}%
\pgfsetlinewidth{0.250937pt}%
\definecolor{currentstroke}{rgb}{0.000000,0.000000,0.000000}%
\pgfsetstrokecolor{currentstroke}%
\pgfsetstrokeopacity{0.500000}%
\pgfsetdash{}{0pt}%
\pgfsys@defobject{currentmarker}{\pgfqpoint{-0.005556in}{-0.005556in}}{\pgfqpoint{0.005556in}{0.005556in}}{%
\pgfpathmoveto{\pgfqpoint{0.000000in}{-0.005556in}}%
\pgfpathcurveto{\pgfqpoint{0.001473in}{-0.005556in}}{\pgfqpoint{0.002887in}{-0.004970in}}{\pgfqpoint{0.003928in}{-0.003928in}}%
\pgfpathcurveto{\pgfqpoint{0.004970in}{-0.002887in}}{\pgfqpoint{0.005556in}{-0.001473in}}{\pgfqpoint{0.005556in}{0.000000in}}%
\pgfpathcurveto{\pgfqpoint{0.005556in}{0.001473in}}{\pgfqpoint{0.004970in}{0.002887in}}{\pgfqpoint{0.003928in}{0.003928in}}%
\pgfpathcurveto{\pgfqpoint{0.002887in}{0.004970in}}{\pgfqpoint{0.001473in}{0.005556in}}{\pgfqpoint{0.000000in}{0.005556in}}%
\pgfpathcurveto{\pgfqpoint{-0.001473in}{0.005556in}}{\pgfqpoint{-0.002887in}{0.004970in}}{\pgfqpoint{-0.003928in}{0.003928in}}%
\pgfpathcurveto{\pgfqpoint{-0.004970in}{0.002887in}}{\pgfqpoint{-0.005556in}{0.001473in}}{\pgfqpoint{-0.005556in}{0.000000in}}%
\pgfpathcurveto{\pgfqpoint{-0.005556in}{-0.001473in}}{\pgfqpoint{-0.004970in}{-0.002887in}}{\pgfqpoint{-0.003928in}{-0.003928in}}%
\pgfpathcurveto{\pgfqpoint{-0.002887in}{-0.004970in}}{\pgfqpoint{-0.001473in}{-0.005556in}}{\pgfqpoint{0.000000in}{-0.005556in}}%
\pgfpathclose%
\pgfusepath{stroke,fill}%
}%
\begin{pgfscope}%
\pgfsys@transformshift{4.140568in}{3.377345in}%
\pgfsys@useobject{currentmarker}{}%
\end{pgfscope}%
\end{pgfscope}%
\begin{pgfscope}%
\pgfpathrectangle{\pgfqpoint{0.100000in}{2.413063in}}{\pgfqpoint{5.037500in}{3.427208in}}%
\pgfusepath{clip}%
\pgfsetrectcap%
\pgfsetroundjoin%
\pgfsetlinewidth{1.505625pt}%
\definecolor{currentstroke}{rgb}{0.000000,0.000000,1.000000}%
\pgfsetstrokecolor{currentstroke}%
\pgfsetstrokeopacity{0.500000}%
\pgfsetdash{}{0pt}%
\pgfpathmoveto{\pgfqpoint{4.149644in}{3.107988in}}%
\pgfusepath{stroke}%
\end{pgfscope}%
\begin{pgfscope}%
\pgfpathrectangle{\pgfqpoint{0.100000in}{2.413063in}}{\pgfqpoint{5.037500in}{3.427208in}}%
\pgfusepath{clip}%
\pgfsetbuttcap%
\pgfsetroundjoin%
\definecolor{currentfill}{rgb}{0.000000,0.000000,1.000000}%
\pgfsetfillcolor{currentfill}%
\pgfsetfillopacity{0.500000}%
\pgfsetlinewidth{0.250937pt}%
\definecolor{currentstroke}{rgb}{0.000000,0.000000,0.000000}%
\pgfsetstrokecolor{currentstroke}%
\pgfsetstrokeopacity{0.500000}%
\pgfsetdash{}{0pt}%
\pgfsys@defobject{currentmarker}{\pgfqpoint{-0.016667in}{-0.016667in}}{\pgfqpoint{0.016667in}{0.016667in}}{%
\pgfpathmoveto{\pgfqpoint{0.000000in}{-0.016667in}}%
\pgfpathcurveto{\pgfqpoint{0.004420in}{-0.016667in}}{\pgfqpoint{0.008660in}{-0.014911in}}{\pgfqpoint{0.011785in}{-0.011785in}}%
\pgfpathcurveto{\pgfqpoint{0.014911in}{-0.008660in}}{\pgfqpoint{0.016667in}{-0.004420in}}{\pgfqpoint{0.016667in}{0.000000in}}%
\pgfpathcurveto{\pgfqpoint{0.016667in}{0.004420in}}{\pgfqpoint{0.014911in}{0.008660in}}{\pgfqpoint{0.011785in}{0.011785in}}%
\pgfpathcurveto{\pgfqpoint{0.008660in}{0.014911in}}{\pgfqpoint{0.004420in}{0.016667in}}{\pgfqpoint{0.000000in}{0.016667in}}%
\pgfpathcurveto{\pgfqpoint{-0.004420in}{0.016667in}}{\pgfqpoint{-0.008660in}{0.014911in}}{\pgfqpoint{-0.011785in}{0.011785in}}%
\pgfpathcurveto{\pgfqpoint{-0.014911in}{0.008660in}}{\pgfqpoint{-0.016667in}{0.004420in}}{\pgfqpoint{-0.016667in}{0.000000in}}%
\pgfpathcurveto{\pgfqpoint{-0.016667in}{-0.004420in}}{\pgfqpoint{-0.014911in}{-0.008660in}}{\pgfqpoint{-0.011785in}{-0.011785in}}%
\pgfpathcurveto{\pgfqpoint{-0.008660in}{-0.014911in}}{\pgfqpoint{-0.004420in}{-0.016667in}}{\pgfqpoint{0.000000in}{-0.016667in}}%
\pgfpathclose%
\pgfusepath{stroke,fill}%
}%
\begin{pgfscope}%
\pgfsys@transformshift{4.149644in}{3.107988in}%
\pgfsys@useobject{currentmarker}{}%
\end{pgfscope}%
\end{pgfscope}%
\begin{pgfscope}%
\pgfpathrectangle{\pgfqpoint{0.100000in}{2.413063in}}{\pgfqpoint{5.037500in}{3.427208in}}%
\pgfusepath{clip}%
\pgfsetrectcap%
\pgfsetroundjoin%
\pgfsetlinewidth{1.505625pt}%
\definecolor{currentstroke}{rgb}{0.000000,0.000000,1.000000}%
\pgfsetstrokecolor{currentstroke}%
\pgfsetstrokeopacity{0.500000}%
\pgfsetdash{}{0pt}%
\pgfpathmoveto{\pgfqpoint{4.372433in}{2.872717in}}%
\pgfusepath{stroke}%
\end{pgfscope}%
\begin{pgfscope}%
\pgfpathrectangle{\pgfqpoint{0.100000in}{2.413063in}}{\pgfqpoint{5.037500in}{3.427208in}}%
\pgfusepath{clip}%
\pgfsetbuttcap%
\pgfsetroundjoin%
\definecolor{currentfill}{rgb}{0.000000,0.000000,1.000000}%
\pgfsetfillcolor{currentfill}%
\pgfsetfillopacity{0.500000}%
\pgfsetlinewidth{0.250937pt}%
\definecolor{currentstroke}{rgb}{0.000000,0.000000,0.000000}%
\pgfsetstrokecolor{currentstroke}%
\pgfsetstrokeopacity{0.500000}%
\pgfsetdash{}{0pt}%
\pgfsys@defobject{currentmarker}{\pgfqpoint{-0.005556in}{-0.005556in}}{\pgfqpoint{0.005556in}{0.005556in}}{%
\pgfpathmoveto{\pgfqpoint{0.000000in}{-0.005556in}}%
\pgfpathcurveto{\pgfqpoint{0.001473in}{-0.005556in}}{\pgfqpoint{0.002887in}{-0.004970in}}{\pgfqpoint{0.003928in}{-0.003928in}}%
\pgfpathcurveto{\pgfqpoint{0.004970in}{-0.002887in}}{\pgfqpoint{0.005556in}{-0.001473in}}{\pgfqpoint{0.005556in}{0.000000in}}%
\pgfpathcurveto{\pgfqpoint{0.005556in}{0.001473in}}{\pgfqpoint{0.004970in}{0.002887in}}{\pgfqpoint{0.003928in}{0.003928in}}%
\pgfpathcurveto{\pgfqpoint{0.002887in}{0.004970in}}{\pgfqpoint{0.001473in}{0.005556in}}{\pgfqpoint{0.000000in}{0.005556in}}%
\pgfpathcurveto{\pgfqpoint{-0.001473in}{0.005556in}}{\pgfqpoint{-0.002887in}{0.004970in}}{\pgfqpoint{-0.003928in}{0.003928in}}%
\pgfpathcurveto{\pgfqpoint{-0.004970in}{0.002887in}}{\pgfqpoint{-0.005556in}{0.001473in}}{\pgfqpoint{-0.005556in}{0.000000in}}%
\pgfpathcurveto{\pgfqpoint{-0.005556in}{-0.001473in}}{\pgfqpoint{-0.004970in}{-0.002887in}}{\pgfqpoint{-0.003928in}{-0.003928in}}%
\pgfpathcurveto{\pgfqpoint{-0.002887in}{-0.004970in}}{\pgfqpoint{-0.001473in}{-0.005556in}}{\pgfqpoint{0.000000in}{-0.005556in}}%
\pgfpathclose%
\pgfusepath{stroke,fill}%
}%
\begin{pgfscope}%
\pgfsys@transformshift{4.372433in}{2.872717in}%
\pgfsys@useobject{currentmarker}{}%
\end{pgfscope}%
\end{pgfscope}%
\begin{pgfscope}%
\pgfpathrectangle{\pgfqpoint{0.100000in}{2.413063in}}{\pgfqpoint{5.037500in}{3.427208in}}%
\pgfusepath{clip}%
\pgfsetrectcap%
\pgfsetroundjoin%
\pgfsetlinewidth{1.505625pt}%
\definecolor{currentstroke}{rgb}{0.501961,0.501961,0.501961}%
\pgfsetstrokecolor{currentstroke}%
\pgfsetstrokeopacity{0.500000}%
\pgfsetdash{}{0pt}%
\pgfpathmoveto{\pgfqpoint{4.197819in}{2.889530in}}%
\pgfusepath{stroke}%
\end{pgfscope}%
\begin{pgfscope}%
\pgfpathrectangle{\pgfqpoint{0.100000in}{2.413063in}}{\pgfqpoint{5.037500in}{3.427208in}}%
\pgfusepath{clip}%
\pgfsetbuttcap%
\pgfsetroundjoin%
\definecolor{currentfill}{rgb}{0.501961,0.501961,0.501961}%
\pgfsetfillcolor{currentfill}%
\pgfsetfillopacity{0.500000}%
\pgfsetlinewidth{0.250937pt}%
\definecolor{currentstroke}{rgb}{0.000000,0.000000,0.000000}%
\pgfsetstrokecolor{currentstroke}%
\pgfsetstrokeopacity{0.500000}%
\pgfsetdash{}{0pt}%
\pgfsys@defobject{currentmarker}{\pgfqpoint{-0.013889in}{-0.013889in}}{\pgfqpoint{0.013889in}{0.013889in}}{%
\pgfpathmoveto{\pgfqpoint{0.000000in}{-0.013889in}}%
\pgfpathcurveto{\pgfqpoint{0.003683in}{-0.013889in}}{\pgfqpoint{0.007216in}{-0.012425in}}{\pgfqpoint{0.009821in}{-0.009821in}}%
\pgfpathcurveto{\pgfqpoint{0.012425in}{-0.007216in}}{\pgfqpoint{0.013889in}{-0.003683in}}{\pgfqpoint{0.013889in}{0.000000in}}%
\pgfpathcurveto{\pgfqpoint{0.013889in}{0.003683in}}{\pgfqpoint{0.012425in}{0.007216in}}{\pgfqpoint{0.009821in}{0.009821in}}%
\pgfpathcurveto{\pgfqpoint{0.007216in}{0.012425in}}{\pgfqpoint{0.003683in}{0.013889in}}{\pgfqpoint{0.000000in}{0.013889in}}%
\pgfpathcurveto{\pgfqpoint{-0.003683in}{0.013889in}}{\pgfqpoint{-0.007216in}{0.012425in}}{\pgfqpoint{-0.009821in}{0.009821in}}%
\pgfpathcurveto{\pgfqpoint{-0.012425in}{0.007216in}}{\pgfqpoint{-0.013889in}{0.003683in}}{\pgfqpoint{-0.013889in}{0.000000in}}%
\pgfpathcurveto{\pgfqpoint{-0.013889in}{-0.003683in}}{\pgfqpoint{-0.012425in}{-0.007216in}}{\pgfqpoint{-0.009821in}{-0.009821in}}%
\pgfpathcurveto{\pgfqpoint{-0.007216in}{-0.012425in}}{\pgfqpoint{-0.003683in}{-0.013889in}}{\pgfqpoint{0.000000in}{-0.013889in}}%
\pgfpathclose%
\pgfusepath{stroke,fill}%
}%
\begin{pgfscope}%
\pgfsys@transformshift{4.197819in}{2.889530in}%
\pgfsys@useobject{currentmarker}{}%
\end{pgfscope}%
\end{pgfscope}%
\begin{pgfscope}%
\pgfpathrectangle{\pgfqpoint{0.100000in}{2.413063in}}{\pgfqpoint{5.037500in}{3.427208in}}%
\pgfusepath{clip}%
\pgfsetrectcap%
\pgfsetroundjoin%
\pgfsetlinewidth{1.505625pt}%
\definecolor{currentstroke}{rgb}{0.501961,0.501961,0.501961}%
\pgfsetstrokecolor{currentstroke}%
\pgfsetstrokeopacity{0.500000}%
\pgfsetdash{}{0pt}%
\pgfpathmoveto{\pgfqpoint{4.136698in}{2.988100in}}%
\pgfusepath{stroke}%
\end{pgfscope}%
\begin{pgfscope}%
\pgfpathrectangle{\pgfqpoint{0.100000in}{2.413063in}}{\pgfqpoint{5.037500in}{3.427208in}}%
\pgfusepath{clip}%
\pgfsetbuttcap%
\pgfsetroundjoin%
\definecolor{currentfill}{rgb}{0.501961,0.501961,0.501961}%
\pgfsetfillcolor{currentfill}%
\pgfsetfillopacity{0.500000}%
\pgfsetlinewidth{0.250937pt}%
\definecolor{currentstroke}{rgb}{0.000000,0.000000,0.000000}%
\pgfsetstrokecolor{currentstroke}%
\pgfsetstrokeopacity{0.500000}%
\pgfsetdash{}{0pt}%
\pgfsys@defobject{currentmarker}{\pgfqpoint{-0.013889in}{-0.013889in}}{\pgfqpoint{0.013889in}{0.013889in}}{%
\pgfpathmoveto{\pgfqpoint{0.000000in}{-0.013889in}}%
\pgfpathcurveto{\pgfqpoint{0.003683in}{-0.013889in}}{\pgfqpoint{0.007216in}{-0.012425in}}{\pgfqpoint{0.009821in}{-0.009821in}}%
\pgfpathcurveto{\pgfqpoint{0.012425in}{-0.007216in}}{\pgfqpoint{0.013889in}{-0.003683in}}{\pgfqpoint{0.013889in}{0.000000in}}%
\pgfpathcurveto{\pgfqpoint{0.013889in}{0.003683in}}{\pgfqpoint{0.012425in}{0.007216in}}{\pgfqpoint{0.009821in}{0.009821in}}%
\pgfpathcurveto{\pgfqpoint{0.007216in}{0.012425in}}{\pgfqpoint{0.003683in}{0.013889in}}{\pgfqpoint{0.000000in}{0.013889in}}%
\pgfpathcurveto{\pgfqpoint{-0.003683in}{0.013889in}}{\pgfqpoint{-0.007216in}{0.012425in}}{\pgfqpoint{-0.009821in}{0.009821in}}%
\pgfpathcurveto{\pgfqpoint{-0.012425in}{0.007216in}}{\pgfqpoint{-0.013889in}{0.003683in}}{\pgfqpoint{-0.013889in}{0.000000in}}%
\pgfpathcurveto{\pgfqpoint{-0.013889in}{-0.003683in}}{\pgfqpoint{-0.012425in}{-0.007216in}}{\pgfqpoint{-0.009821in}{-0.009821in}}%
\pgfpathcurveto{\pgfqpoint{-0.007216in}{-0.012425in}}{\pgfqpoint{-0.003683in}{-0.013889in}}{\pgfqpoint{0.000000in}{-0.013889in}}%
\pgfpathclose%
\pgfusepath{stroke,fill}%
}%
\begin{pgfscope}%
\pgfsys@transformshift{4.136698in}{2.988100in}%
\pgfsys@useobject{currentmarker}{}%
\end{pgfscope}%
\end{pgfscope}%
\begin{pgfscope}%
\pgfpathrectangle{\pgfqpoint{0.100000in}{2.413063in}}{\pgfqpoint{5.037500in}{3.427208in}}%
\pgfusepath{clip}%
\pgfsetrectcap%
\pgfsetroundjoin%
\pgfsetlinewidth{1.505625pt}%
\definecolor{currentstroke}{rgb}{0.000000,0.000000,1.000000}%
\pgfsetstrokecolor{currentstroke}%
\pgfsetstrokeopacity{0.500000}%
\pgfsetdash{}{0pt}%
\pgfpathmoveto{\pgfqpoint{4.111216in}{3.238146in}}%
\pgfusepath{stroke}%
\end{pgfscope}%
\begin{pgfscope}%
\pgfpathrectangle{\pgfqpoint{0.100000in}{2.413063in}}{\pgfqpoint{5.037500in}{3.427208in}}%
\pgfusepath{clip}%
\pgfsetbuttcap%
\pgfsetroundjoin%
\definecolor{currentfill}{rgb}{0.000000,0.000000,1.000000}%
\pgfsetfillcolor{currentfill}%
\pgfsetfillopacity{0.500000}%
\pgfsetlinewidth{0.250937pt}%
\definecolor{currentstroke}{rgb}{0.000000,0.000000,0.000000}%
\pgfsetstrokecolor{currentstroke}%
\pgfsetstrokeopacity{0.500000}%
\pgfsetdash{}{0pt}%
\pgfsys@defobject{currentmarker}{\pgfqpoint{-0.008333in}{-0.008333in}}{\pgfqpoint{0.008333in}{0.008333in}}{%
\pgfpathmoveto{\pgfqpoint{0.000000in}{-0.008333in}}%
\pgfpathcurveto{\pgfqpoint{0.002210in}{-0.008333in}}{\pgfqpoint{0.004330in}{-0.007455in}}{\pgfqpoint{0.005893in}{-0.005893in}}%
\pgfpathcurveto{\pgfqpoint{0.007455in}{-0.004330in}}{\pgfqpoint{0.008333in}{-0.002210in}}{\pgfqpoint{0.008333in}{0.000000in}}%
\pgfpathcurveto{\pgfqpoint{0.008333in}{0.002210in}}{\pgfqpoint{0.007455in}{0.004330in}}{\pgfqpoint{0.005893in}{0.005893in}}%
\pgfpathcurveto{\pgfqpoint{0.004330in}{0.007455in}}{\pgfqpoint{0.002210in}{0.008333in}}{\pgfqpoint{0.000000in}{0.008333in}}%
\pgfpathcurveto{\pgfqpoint{-0.002210in}{0.008333in}}{\pgfqpoint{-0.004330in}{0.007455in}}{\pgfqpoint{-0.005893in}{0.005893in}}%
\pgfpathcurveto{\pgfqpoint{-0.007455in}{0.004330in}}{\pgfqpoint{-0.008333in}{0.002210in}}{\pgfqpoint{-0.008333in}{0.000000in}}%
\pgfpathcurveto{\pgfqpoint{-0.008333in}{-0.002210in}}{\pgfqpoint{-0.007455in}{-0.004330in}}{\pgfqpoint{-0.005893in}{-0.005893in}}%
\pgfpathcurveto{\pgfqpoint{-0.004330in}{-0.007455in}}{\pgfqpoint{-0.002210in}{-0.008333in}}{\pgfqpoint{0.000000in}{-0.008333in}}%
\pgfpathclose%
\pgfusepath{stroke,fill}%
}%
\begin{pgfscope}%
\pgfsys@transformshift{4.111216in}{3.238146in}%
\pgfsys@useobject{currentmarker}{}%
\end{pgfscope}%
\end{pgfscope}%
\begin{pgfscope}%
\pgfpathrectangle{\pgfqpoint{0.100000in}{2.413063in}}{\pgfqpoint{5.037500in}{3.427208in}}%
\pgfusepath{clip}%
\pgfsetrectcap%
\pgfsetroundjoin%
\pgfsetlinewidth{1.505625pt}%
\definecolor{currentstroke}{rgb}{0.000000,0.000000,1.000000}%
\pgfsetstrokecolor{currentstroke}%
\pgfsetstrokeopacity{0.500000}%
\pgfsetdash{}{0pt}%
\pgfpathmoveto{\pgfqpoint{4.199267in}{3.174835in}}%
\pgfusepath{stroke}%
\end{pgfscope}%
\begin{pgfscope}%
\pgfpathrectangle{\pgfqpoint{0.100000in}{2.413063in}}{\pgfqpoint{5.037500in}{3.427208in}}%
\pgfusepath{clip}%
\pgfsetbuttcap%
\pgfsetroundjoin%
\definecolor{currentfill}{rgb}{0.000000,0.000000,1.000000}%
\pgfsetfillcolor{currentfill}%
\pgfsetfillopacity{0.500000}%
\pgfsetlinewidth{0.250937pt}%
\definecolor{currentstroke}{rgb}{0.000000,0.000000,0.000000}%
\pgfsetstrokecolor{currentstroke}%
\pgfsetstrokeopacity{0.500000}%
\pgfsetdash{}{0pt}%
\pgfsys@defobject{currentmarker}{\pgfqpoint{-0.025000in}{-0.025000in}}{\pgfqpoint{0.025000in}{0.025000in}}{%
\pgfpathmoveto{\pgfqpoint{0.000000in}{-0.025000in}}%
\pgfpathcurveto{\pgfqpoint{0.006630in}{-0.025000in}}{\pgfqpoint{0.012989in}{-0.022366in}}{\pgfqpoint{0.017678in}{-0.017678in}}%
\pgfpathcurveto{\pgfqpoint{0.022366in}{-0.012989in}}{\pgfqpoint{0.025000in}{-0.006630in}}{\pgfqpoint{0.025000in}{0.000000in}}%
\pgfpathcurveto{\pgfqpoint{0.025000in}{0.006630in}}{\pgfqpoint{0.022366in}{0.012989in}}{\pgfqpoint{0.017678in}{0.017678in}}%
\pgfpathcurveto{\pgfqpoint{0.012989in}{0.022366in}}{\pgfqpoint{0.006630in}{0.025000in}}{\pgfqpoint{0.000000in}{0.025000in}}%
\pgfpathcurveto{\pgfqpoint{-0.006630in}{0.025000in}}{\pgfqpoint{-0.012989in}{0.022366in}}{\pgfqpoint{-0.017678in}{0.017678in}}%
\pgfpathcurveto{\pgfqpoint{-0.022366in}{0.012989in}}{\pgfqpoint{-0.025000in}{0.006630in}}{\pgfqpoint{-0.025000in}{0.000000in}}%
\pgfpathcurveto{\pgfqpoint{-0.025000in}{-0.006630in}}{\pgfqpoint{-0.022366in}{-0.012989in}}{\pgfqpoint{-0.017678in}{-0.017678in}}%
\pgfpathcurveto{\pgfqpoint{-0.012989in}{-0.022366in}}{\pgfqpoint{-0.006630in}{-0.025000in}}{\pgfqpoint{0.000000in}{-0.025000in}}%
\pgfpathclose%
\pgfusepath{stroke,fill}%
}%
\begin{pgfscope}%
\pgfsys@transformshift{4.199267in}{3.174835in}%
\pgfsys@useobject{currentmarker}{}%
\end{pgfscope}%
\end{pgfscope}%
\begin{pgfscope}%
\pgfpathrectangle{\pgfqpoint{0.100000in}{2.413063in}}{\pgfqpoint{5.037500in}{3.427208in}}%
\pgfusepath{clip}%
\pgfsetrectcap%
\pgfsetroundjoin%
\pgfsetlinewidth{1.505625pt}%
\definecolor{currentstroke}{rgb}{0.501961,0.501961,0.501961}%
\pgfsetstrokecolor{currentstroke}%
\pgfsetstrokeopacity{0.500000}%
\pgfsetdash{}{0pt}%
\pgfpathmoveto{\pgfqpoint{4.283427in}{3.127572in}}%
\pgfusepath{stroke}%
\end{pgfscope}%
\begin{pgfscope}%
\pgfpathrectangle{\pgfqpoint{0.100000in}{2.413063in}}{\pgfqpoint{5.037500in}{3.427208in}}%
\pgfusepath{clip}%
\pgfsetbuttcap%
\pgfsetroundjoin%
\definecolor{currentfill}{rgb}{0.501961,0.501961,0.501961}%
\pgfsetfillcolor{currentfill}%
\pgfsetfillopacity{0.500000}%
\pgfsetlinewidth{0.250937pt}%
\definecolor{currentstroke}{rgb}{0.000000,0.000000,0.000000}%
\pgfsetstrokecolor{currentstroke}%
\pgfsetstrokeopacity{0.500000}%
\pgfsetdash{}{0pt}%
\pgfsys@defobject{currentmarker}{\pgfqpoint{-0.013889in}{-0.013889in}}{\pgfqpoint{0.013889in}{0.013889in}}{%
\pgfpathmoveto{\pgfqpoint{0.000000in}{-0.013889in}}%
\pgfpathcurveto{\pgfqpoint{0.003683in}{-0.013889in}}{\pgfqpoint{0.007216in}{-0.012425in}}{\pgfqpoint{0.009821in}{-0.009821in}}%
\pgfpathcurveto{\pgfqpoint{0.012425in}{-0.007216in}}{\pgfqpoint{0.013889in}{-0.003683in}}{\pgfqpoint{0.013889in}{0.000000in}}%
\pgfpathcurveto{\pgfqpoint{0.013889in}{0.003683in}}{\pgfqpoint{0.012425in}{0.007216in}}{\pgfqpoint{0.009821in}{0.009821in}}%
\pgfpathcurveto{\pgfqpoint{0.007216in}{0.012425in}}{\pgfqpoint{0.003683in}{0.013889in}}{\pgfqpoint{0.000000in}{0.013889in}}%
\pgfpathcurveto{\pgfqpoint{-0.003683in}{0.013889in}}{\pgfqpoint{-0.007216in}{0.012425in}}{\pgfqpoint{-0.009821in}{0.009821in}}%
\pgfpathcurveto{\pgfqpoint{-0.012425in}{0.007216in}}{\pgfqpoint{-0.013889in}{0.003683in}}{\pgfqpoint{-0.013889in}{0.000000in}}%
\pgfpathcurveto{\pgfqpoint{-0.013889in}{-0.003683in}}{\pgfqpoint{-0.012425in}{-0.007216in}}{\pgfqpoint{-0.009821in}{-0.009821in}}%
\pgfpathcurveto{\pgfqpoint{-0.007216in}{-0.012425in}}{\pgfqpoint{-0.003683in}{-0.013889in}}{\pgfqpoint{0.000000in}{-0.013889in}}%
\pgfpathclose%
\pgfusepath{stroke,fill}%
}%
\begin{pgfscope}%
\pgfsys@transformshift{4.283427in}{3.127572in}%
\pgfsys@useobject{currentmarker}{}%
\end{pgfscope}%
\end{pgfscope}%
\begin{pgfscope}%
\pgfpathrectangle{\pgfqpoint{0.100000in}{2.413063in}}{\pgfqpoint{5.037500in}{3.427208in}}%
\pgfusepath{clip}%
\pgfsetrectcap%
\pgfsetroundjoin%
\pgfsetlinewidth{1.505625pt}%
\definecolor{currentstroke}{rgb}{0.501961,0.501961,0.501961}%
\pgfsetstrokecolor{currentstroke}%
\pgfsetstrokeopacity{0.500000}%
\pgfsetdash{}{0pt}%
\pgfpathmoveto{\pgfqpoint{3.743850in}{3.307362in}}%
\pgfusepath{stroke}%
\end{pgfscope}%
\begin{pgfscope}%
\pgfpathrectangle{\pgfqpoint{0.100000in}{2.413063in}}{\pgfqpoint{5.037500in}{3.427208in}}%
\pgfusepath{clip}%
\pgfsetbuttcap%
\pgfsetroundjoin%
\definecolor{currentfill}{rgb}{0.501961,0.501961,0.501961}%
\pgfsetfillcolor{currentfill}%
\pgfsetfillopacity{0.500000}%
\pgfsetlinewidth{0.250937pt}%
\definecolor{currentstroke}{rgb}{0.000000,0.000000,0.000000}%
\pgfsetstrokecolor{currentstroke}%
\pgfsetstrokeopacity{0.500000}%
\pgfsetdash{}{0pt}%
\pgfsys@defobject{currentmarker}{\pgfqpoint{-0.013889in}{-0.013889in}}{\pgfqpoint{0.013889in}{0.013889in}}{%
\pgfpathmoveto{\pgfqpoint{0.000000in}{-0.013889in}}%
\pgfpathcurveto{\pgfqpoint{0.003683in}{-0.013889in}}{\pgfqpoint{0.007216in}{-0.012425in}}{\pgfqpoint{0.009821in}{-0.009821in}}%
\pgfpathcurveto{\pgfqpoint{0.012425in}{-0.007216in}}{\pgfqpoint{0.013889in}{-0.003683in}}{\pgfqpoint{0.013889in}{0.000000in}}%
\pgfpathcurveto{\pgfqpoint{0.013889in}{0.003683in}}{\pgfqpoint{0.012425in}{0.007216in}}{\pgfqpoint{0.009821in}{0.009821in}}%
\pgfpathcurveto{\pgfqpoint{0.007216in}{0.012425in}}{\pgfqpoint{0.003683in}{0.013889in}}{\pgfqpoint{0.000000in}{0.013889in}}%
\pgfpathcurveto{\pgfqpoint{-0.003683in}{0.013889in}}{\pgfqpoint{-0.007216in}{0.012425in}}{\pgfqpoint{-0.009821in}{0.009821in}}%
\pgfpathcurveto{\pgfqpoint{-0.012425in}{0.007216in}}{\pgfqpoint{-0.013889in}{0.003683in}}{\pgfqpoint{-0.013889in}{0.000000in}}%
\pgfpathcurveto{\pgfqpoint{-0.013889in}{-0.003683in}}{\pgfqpoint{-0.012425in}{-0.007216in}}{\pgfqpoint{-0.009821in}{-0.009821in}}%
\pgfpathcurveto{\pgfqpoint{-0.007216in}{-0.012425in}}{\pgfqpoint{-0.003683in}{-0.013889in}}{\pgfqpoint{0.000000in}{-0.013889in}}%
\pgfpathclose%
\pgfusepath{stroke,fill}%
}%
\begin{pgfscope}%
\pgfsys@transformshift{3.743850in}{3.307362in}%
\pgfsys@useobject{currentmarker}{}%
\end{pgfscope}%
\end{pgfscope}%
\begin{pgfscope}%
\pgfpathrectangle{\pgfqpoint{0.100000in}{2.413063in}}{\pgfqpoint{5.037500in}{3.427208in}}%
\pgfusepath{clip}%
\pgfsetrectcap%
\pgfsetroundjoin%
\pgfsetlinewidth{1.505625pt}%
\definecolor{currentstroke}{rgb}{0.000000,0.000000,1.000000}%
\pgfsetstrokecolor{currentstroke}%
\pgfsetstrokeopacity{0.500000}%
\pgfsetdash{}{0pt}%
\pgfpathmoveto{\pgfqpoint{3.584571in}{3.322865in}}%
\pgfusepath{stroke}%
\end{pgfscope}%
\begin{pgfscope}%
\pgfpathrectangle{\pgfqpoint{0.100000in}{2.413063in}}{\pgfqpoint{5.037500in}{3.427208in}}%
\pgfusepath{clip}%
\pgfsetbuttcap%
\pgfsetroundjoin%
\definecolor{currentfill}{rgb}{0.000000,0.000000,1.000000}%
\pgfsetfillcolor{currentfill}%
\pgfsetfillopacity{0.500000}%
\pgfsetlinewidth{0.250937pt}%
\definecolor{currentstroke}{rgb}{0.000000,0.000000,0.000000}%
\pgfsetstrokecolor{currentstroke}%
\pgfsetstrokeopacity{0.500000}%
\pgfsetdash{}{0pt}%
\pgfsys@defobject{currentmarker}{\pgfqpoint{-0.008333in}{-0.008333in}}{\pgfqpoint{0.008333in}{0.008333in}}{%
\pgfpathmoveto{\pgfqpoint{0.000000in}{-0.008333in}}%
\pgfpathcurveto{\pgfqpoint{0.002210in}{-0.008333in}}{\pgfqpoint{0.004330in}{-0.007455in}}{\pgfqpoint{0.005893in}{-0.005893in}}%
\pgfpathcurveto{\pgfqpoint{0.007455in}{-0.004330in}}{\pgfqpoint{0.008333in}{-0.002210in}}{\pgfqpoint{0.008333in}{0.000000in}}%
\pgfpathcurveto{\pgfqpoint{0.008333in}{0.002210in}}{\pgfqpoint{0.007455in}{0.004330in}}{\pgfqpoint{0.005893in}{0.005893in}}%
\pgfpathcurveto{\pgfqpoint{0.004330in}{0.007455in}}{\pgfqpoint{0.002210in}{0.008333in}}{\pgfqpoint{0.000000in}{0.008333in}}%
\pgfpathcurveto{\pgfqpoint{-0.002210in}{0.008333in}}{\pgfqpoint{-0.004330in}{0.007455in}}{\pgfqpoint{-0.005893in}{0.005893in}}%
\pgfpathcurveto{\pgfqpoint{-0.007455in}{0.004330in}}{\pgfqpoint{-0.008333in}{0.002210in}}{\pgfqpoint{-0.008333in}{0.000000in}}%
\pgfpathcurveto{\pgfqpoint{-0.008333in}{-0.002210in}}{\pgfqpoint{-0.007455in}{-0.004330in}}{\pgfqpoint{-0.005893in}{-0.005893in}}%
\pgfpathcurveto{\pgfqpoint{-0.004330in}{-0.007455in}}{\pgfqpoint{-0.002210in}{-0.008333in}}{\pgfqpoint{0.000000in}{-0.008333in}}%
\pgfpathclose%
\pgfusepath{stroke,fill}%
}%
\begin{pgfscope}%
\pgfsys@transformshift{3.584571in}{3.322865in}%
\pgfsys@useobject{currentmarker}{}%
\end{pgfscope}%
\end{pgfscope}%
\begin{pgfscope}%
\pgfpathrectangle{\pgfqpoint{0.100000in}{2.413063in}}{\pgfqpoint{5.037500in}{3.427208in}}%
\pgfusepath{clip}%
\pgfsetrectcap%
\pgfsetroundjoin%
\pgfsetlinewidth{1.505625pt}%
\definecolor{currentstroke}{rgb}{0.501961,0.501961,0.501961}%
\pgfsetstrokecolor{currentstroke}%
\pgfsetstrokeopacity{0.500000}%
\pgfsetdash{}{0pt}%
\pgfpathmoveto{\pgfqpoint{4.327246in}{3.046750in}}%
\pgfusepath{stroke}%
\end{pgfscope}%
\begin{pgfscope}%
\pgfpathrectangle{\pgfqpoint{0.100000in}{2.413063in}}{\pgfqpoint{5.037500in}{3.427208in}}%
\pgfusepath{clip}%
\pgfsetbuttcap%
\pgfsetroundjoin%
\definecolor{currentfill}{rgb}{0.501961,0.501961,0.501961}%
\pgfsetfillcolor{currentfill}%
\pgfsetfillopacity{0.500000}%
\pgfsetlinewidth{0.250937pt}%
\definecolor{currentstroke}{rgb}{0.000000,0.000000,0.000000}%
\pgfsetstrokecolor{currentstroke}%
\pgfsetstrokeopacity{0.500000}%
\pgfsetdash{}{0pt}%
\pgfsys@defobject{currentmarker}{\pgfqpoint{-0.013889in}{-0.013889in}}{\pgfqpoint{0.013889in}{0.013889in}}{%
\pgfpathmoveto{\pgfqpoint{0.000000in}{-0.013889in}}%
\pgfpathcurveto{\pgfqpoint{0.003683in}{-0.013889in}}{\pgfqpoint{0.007216in}{-0.012425in}}{\pgfqpoint{0.009821in}{-0.009821in}}%
\pgfpathcurveto{\pgfqpoint{0.012425in}{-0.007216in}}{\pgfqpoint{0.013889in}{-0.003683in}}{\pgfqpoint{0.013889in}{0.000000in}}%
\pgfpathcurveto{\pgfqpoint{0.013889in}{0.003683in}}{\pgfqpoint{0.012425in}{0.007216in}}{\pgfqpoint{0.009821in}{0.009821in}}%
\pgfpathcurveto{\pgfqpoint{0.007216in}{0.012425in}}{\pgfqpoint{0.003683in}{0.013889in}}{\pgfqpoint{0.000000in}{0.013889in}}%
\pgfpathcurveto{\pgfqpoint{-0.003683in}{0.013889in}}{\pgfqpoint{-0.007216in}{0.012425in}}{\pgfqpoint{-0.009821in}{0.009821in}}%
\pgfpathcurveto{\pgfqpoint{-0.012425in}{0.007216in}}{\pgfqpoint{-0.013889in}{0.003683in}}{\pgfqpoint{-0.013889in}{0.000000in}}%
\pgfpathcurveto{\pgfqpoint{-0.013889in}{-0.003683in}}{\pgfqpoint{-0.012425in}{-0.007216in}}{\pgfqpoint{-0.009821in}{-0.009821in}}%
\pgfpathcurveto{\pgfqpoint{-0.007216in}{-0.012425in}}{\pgfqpoint{-0.003683in}{-0.013889in}}{\pgfqpoint{0.000000in}{-0.013889in}}%
\pgfpathclose%
\pgfusepath{stroke,fill}%
}%
\begin{pgfscope}%
\pgfsys@transformshift{4.327246in}{3.046750in}%
\pgfsys@useobject{currentmarker}{}%
\end{pgfscope}%
\end{pgfscope}%
\begin{pgfscope}%
\pgfpathrectangle{\pgfqpoint{0.100000in}{2.413063in}}{\pgfqpoint{5.037500in}{3.427208in}}%
\pgfusepath{clip}%
\pgfsetrectcap%
\pgfsetroundjoin%
\pgfsetlinewidth{1.505625pt}%
\definecolor{currentstroke}{rgb}{0.501961,0.501961,0.501961}%
\pgfsetstrokecolor{currentstroke}%
\pgfsetstrokeopacity{0.500000}%
\pgfsetdash{}{0pt}%
\pgfpathmoveto{\pgfqpoint{4.158371in}{2.977590in}}%
\pgfusepath{stroke}%
\end{pgfscope}%
\begin{pgfscope}%
\pgfpathrectangle{\pgfqpoint{0.100000in}{2.413063in}}{\pgfqpoint{5.037500in}{3.427208in}}%
\pgfusepath{clip}%
\pgfsetbuttcap%
\pgfsetroundjoin%
\definecolor{currentfill}{rgb}{0.501961,0.501961,0.501961}%
\pgfsetfillcolor{currentfill}%
\pgfsetfillopacity{0.500000}%
\pgfsetlinewidth{0.250937pt}%
\definecolor{currentstroke}{rgb}{0.000000,0.000000,0.000000}%
\pgfsetstrokecolor{currentstroke}%
\pgfsetstrokeopacity{0.500000}%
\pgfsetdash{}{0pt}%
\pgfsys@defobject{currentmarker}{\pgfqpoint{-0.013889in}{-0.013889in}}{\pgfqpoint{0.013889in}{0.013889in}}{%
\pgfpathmoveto{\pgfqpoint{0.000000in}{-0.013889in}}%
\pgfpathcurveto{\pgfqpoint{0.003683in}{-0.013889in}}{\pgfqpoint{0.007216in}{-0.012425in}}{\pgfqpoint{0.009821in}{-0.009821in}}%
\pgfpathcurveto{\pgfqpoint{0.012425in}{-0.007216in}}{\pgfqpoint{0.013889in}{-0.003683in}}{\pgfqpoint{0.013889in}{0.000000in}}%
\pgfpathcurveto{\pgfqpoint{0.013889in}{0.003683in}}{\pgfqpoint{0.012425in}{0.007216in}}{\pgfqpoint{0.009821in}{0.009821in}}%
\pgfpathcurveto{\pgfqpoint{0.007216in}{0.012425in}}{\pgfqpoint{0.003683in}{0.013889in}}{\pgfqpoint{0.000000in}{0.013889in}}%
\pgfpathcurveto{\pgfqpoint{-0.003683in}{0.013889in}}{\pgfqpoint{-0.007216in}{0.012425in}}{\pgfqpoint{-0.009821in}{0.009821in}}%
\pgfpathcurveto{\pgfqpoint{-0.012425in}{0.007216in}}{\pgfqpoint{-0.013889in}{0.003683in}}{\pgfqpoint{-0.013889in}{0.000000in}}%
\pgfpathcurveto{\pgfqpoint{-0.013889in}{-0.003683in}}{\pgfqpoint{-0.012425in}{-0.007216in}}{\pgfqpoint{-0.009821in}{-0.009821in}}%
\pgfpathcurveto{\pgfqpoint{-0.007216in}{-0.012425in}}{\pgfqpoint{-0.003683in}{-0.013889in}}{\pgfqpoint{0.000000in}{-0.013889in}}%
\pgfpathclose%
\pgfusepath{stroke,fill}%
}%
\begin{pgfscope}%
\pgfsys@transformshift{4.158371in}{2.977590in}%
\pgfsys@useobject{currentmarker}{}%
\end{pgfscope}%
\end{pgfscope}%
\begin{pgfscope}%
\pgfpathrectangle{\pgfqpoint{0.100000in}{2.413063in}}{\pgfqpoint{5.037500in}{3.427208in}}%
\pgfusepath{clip}%
\pgfsetrectcap%
\pgfsetroundjoin%
\pgfsetlinewidth{1.505625pt}%
\definecolor{currentstroke}{rgb}{0.501961,0.501961,0.501961}%
\pgfsetstrokecolor{currentstroke}%
\pgfsetstrokeopacity{0.500000}%
\pgfsetdash{}{0pt}%
\pgfpathmoveto{\pgfqpoint{4.305082in}{3.105300in}}%
\pgfusepath{stroke}%
\end{pgfscope}%
\begin{pgfscope}%
\pgfpathrectangle{\pgfqpoint{0.100000in}{2.413063in}}{\pgfqpoint{5.037500in}{3.427208in}}%
\pgfusepath{clip}%
\pgfsetbuttcap%
\pgfsetroundjoin%
\definecolor{currentfill}{rgb}{0.501961,0.501961,0.501961}%
\pgfsetfillcolor{currentfill}%
\pgfsetfillopacity{0.500000}%
\pgfsetlinewidth{0.250937pt}%
\definecolor{currentstroke}{rgb}{0.000000,0.000000,0.000000}%
\pgfsetstrokecolor{currentstroke}%
\pgfsetstrokeopacity{0.500000}%
\pgfsetdash{}{0pt}%
\pgfsys@defobject{currentmarker}{\pgfqpoint{-0.013889in}{-0.013889in}}{\pgfqpoint{0.013889in}{0.013889in}}{%
\pgfpathmoveto{\pgfqpoint{0.000000in}{-0.013889in}}%
\pgfpathcurveto{\pgfqpoint{0.003683in}{-0.013889in}}{\pgfqpoint{0.007216in}{-0.012425in}}{\pgfqpoint{0.009821in}{-0.009821in}}%
\pgfpathcurveto{\pgfqpoint{0.012425in}{-0.007216in}}{\pgfqpoint{0.013889in}{-0.003683in}}{\pgfqpoint{0.013889in}{0.000000in}}%
\pgfpathcurveto{\pgfqpoint{0.013889in}{0.003683in}}{\pgfqpoint{0.012425in}{0.007216in}}{\pgfqpoint{0.009821in}{0.009821in}}%
\pgfpathcurveto{\pgfqpoint{0.007216in}{0.012425in}}{\pgfqpoint{0.003683in}{0.013889in}}{\pgfqpoint{0.000000in}{0.013889in}}%
\pgfpathcurveto{\pgfqpoint{-0.003683in}{0.013889in}}{\pgfqpoint{-0.007216in}{0.012425in}}{\pgfqpoint{-0.009821in}{0.009821in}}%
\pgfpathcurveto{\pgfqpoint{-0.012425in}{0.007216in}}{\pgfqpoint{-0.013889in}{0.003683in}}{\pgfqpoint{-0.013889in}{0.000000in}}%
\pgfpathcurveto{\pgfqpoint{-0.013889in}{-0.003683in}}{\pgfqpoint{-0.012425in}{-0.007216in}}{\pgfqpoint{-0.009821in}{-0.009821in}}%
\pgfpathcurveto{\pgfqpoint{-0.007216in}{-0.012425in}}{\pgfqpoint{-0.003683in}{-0.013889in}}{\pgfqpoint{0.000000in}{-0.013889in}}%
\pgfpathclose%
\pgfusepath{stroke,fill}%
}%
\begin{pgfscope}%
\pgfsys@transformshift{4.305082in}{3.105300in}%
\pgfsys@useobject{currentmarker}{}%
\end{pgfscope}%
\end{pgfscope}%
\begin{pgfscope}%
\pgfpathrectangle{\pgfqpoint{0.100000in}{2.413063in}}{\pgfqpoint{5.037500in}{3.427208in}}%
\pgfusepath{clip}%
\pgfsetrectcap%
\pgfsetroundjoin%
\pgfsetlinewidth{1.505625pt}%
\definecolor{currentstroke}{rgb}{0.501961,0.501961,0.501961}%
\pgfsetstrokecolor{currentstroke}%
\pgfsetstrokeopacity{0.500000}%
\pgfsetdash{}{0pt}%
\pgfpathmoveto{\pgfqpoint{4.211198in}{3.052586in}}%
\pgfusepath{stroke}%
\end{pgfscope}%
\begin{pgfscope}%
\pgfpathrectangle{\pgfqpoint{0.100000in}{2.413063in}}{\pgfqpoint{5.037500in}{3.427208in}}%
\pgfusepath{clip}%
\pgfsetbuttcap%
\pgfsetroundjoin%
\definecolor{currentfill}{rgb}{0.501961,0.501961,0.501961}%
\pgfsetfillcolor{currentfill}%
\pgfsetfillopacity{0.500000}%
\pgfsetlinewidth{0.250937pt}%
\definecolor{currentstroke}{rgb}{0.000000,0.000000,0.000000}%
\pgfsetstrokecolor{currentstroke}%
\pgfsetstrokeopacity{0.500000}%
\pgfsetdash{}{0pt}%
\pgfsys@defobject{currentmarker}{\pgfqpoint{-0.013889in}{-0.013889in}}{\pgfqpoint{0.013889in}{0.013889in}}{%
\pgfpathmoveto{\pgfqpoint{0.000000in}{-0.013889in}}%
\pgfpathcurveto{\pgfqpoint{0.003683in}{-0.013889in}}{\pgfqpoint{0.007216in}{-0.012425in}}{\pgfqpoint{0.009821in}{-0.009821in}}%
\pgfpathcurveto{\pgfqpoint{0.012425in}{-0.007216in}}{\pgfqpoint{0.013889in}{-0.003683in}}{\pgfqpoint{0.013889in}{0.000000in}}%
\pgfpathcurveto{\pgfqpoint{0.013889in}{0.003683in}}{\pgfqpoint{0.012425in}{0.007216in}}{\pgfqpoint{0.009821in}{0.009821in}}%
\pgfpathcurveto{\pgfqpoint{0.007216in}{0.012425in}}{\pgfqpoint{0.003683in}{0.013889in}}{\pgfqpoint{0.000000in}{0.013889in}}%
\pgfpathcurveto{\pgfqpoint{-0.003683in}{0.013889in}}{\pgfqpoint{-0.007216in}{0.012425in}}{\pgfqpoint{-0.009821in}{0.009821in}}%
\pgfpathcurveto{\pgfqpoint{-0.012425in}{0.007216in}}{\pgfqpoint{-0.013889in}{0.003683in}}{\pgfqpoint{-0.013889in}{0.000000in}}%
\pgfpathcurveto{\pgfqpoint{-0.013889in}{-0.003683in}}{\pgfqpoint{-0.012425in}{-0.007216in}}{\pgfqpoint{-0.009821in}{-0.009821in}}%
\pgfpathcurveto{\pgfqpoint{-0.007216in}{-0.012425in}}{\pgfqpoint{-0.003683in}{-0.013889in}}{\pgfqpoint{0.000000in}{-0.013889in}}%
\pgfpathclose%
\pgfusepath{stroke,fill}%
}%
\begin{pgfscope}%
\pgfsys@transformshift{4.211198in}{3.052586in}%
\pgfsys@useobject{currentmarker}{}%
\end{pgfscope}%
\end{pgfscope}%
\begin{pgfscope}%
\pgfpathrectangle{\pgfqpoint{0.100000in}{2.413063in}}{\pgfqpoint{5.037500in}{3.427208in}}%
\pgfusepath{clip}%
\pgfsetrectcap%
\pgfsetroundjoin%
\pgfsetlinewidth{1.505625pt}%
\definecolor{currentstroke}{rgb}{0.000000,0.000000,1.000000}%
\pgfsetstrokecolor{currentstroke}%
\pgfsetstrokeopacity{0.500000}%
\pgfsetdash{}{0pt}%
\pgfpathmoveto{\pgfqpoint{3.877566in}{3.354757in}}%
\pgfusepath{stroke}%
\end{pgfscope}%
\begin{pgfscope}%
\pgfpathrectangle{\pgfqpoint{0.100000in}{2.413063in}}{\pgfqpoint{5.037500in}{3.427208in}}%
\pgfusepath{clip}%
\pgfsetbuttcap%
\pgfsetroundjoin%
\definecolor{currentfill}{rgb}{0.000000,0.000000,1.000000}%
\pgfsetfillcolor{currentfill}%
\pgfsetfillopacity{0.500000}%
\pgfsetlinewidth{0.250937pt}%
\definecolor{currentstroke}{rgb}{0.000000,0.000000,0.000000}%
\pgfsetstrokecolor{currentstroke}%
\pgfsetstrokeopacity{0.500000}%
\pgfsetdash{}{0pt}%
\pgfsys@defobject{currentmarker}{\pgfqpoint{-0.011111in}{-0.011111in}}{\pgfqpoint{0.011111in}{0.011111in}}{%
\pgfpathmoveto{\pgfqpoint{0.000000in}{-0.011111in}}%
\pgfpathcurveto{\pgfqpoint{0.002947in}{-0.011111in}}{\pgfqpoint{0.005773in}{-0.009940in}}{\pgfqpoint{0.007857in}{-0.007857in}}%
\pgfpathcurveto{\pgfqpoint{0.009940in}{-0.005773in}}{\pgfqpoint{0.011111in}{-0.002947in}}{\pgfqpoint{0.011111in}{0.000000in}}%
\pgfpathcurveto{\pgfqpoint{0.011111in}{0.002947in}}{\pgfqpoint{0.009940in}{0.005773in}}{\pgfqpoint{0.007857in}{0.007857in}}%
\pgfpathcurveto{\pgfqpoint{0.005773in}{0.009940in}}{\pgfqpoint{0.002947in}{0.011111in}}{\pgfqpoint{0.000000in}{0.011111in}}%
\pgfpathcurveto{\pgfqpoint{-0.002947in}{0.011111in}}{\pgfqpoint{-0.005773in}{0.009940in}}{\pgfqpoint{-0.007857in}{0.007857in}}%
\pgfpathcurveto{\pgfqpoint{-0.009940in}{0.005773in}}{\pgfqpoint{-0.011111in}{0.002947in}}{\pgfqpoint{-0.011111in}{0.000000in}}%
\pgfpathcurveto{\pgfqpoint{-0.011111in}{-0.002947in}}{\pgfqpoint{-0.009940in}{-0.005773in}}{\pgfqpoint{-0.007857in}{-0.007857in}}%
\pgfpathcurveto{\pgfqpoint{-0.005773in}{-0.009940in}}{\pgfqpoint{-0.002947in}{-0.011111in}}{\pgfqpoint{0.000000in}{-0.011111in}}%
\pgfpathclose%
\pgfusepath{stroke,fill}%
}%
\begin{pgfscope}%
\pgfsys@transformshift{3.877566in}{3.354757in}%
\pgfsys@useobject{currentmarker}{}%
\end{pgfscope}%
\end{pgfscope}%
\begin{pgfscope}%
\pgfpathrectangle{\pgfqpoint{0.100000in}{2.413063in}}{\pgfqpoint{5.037500in}{3.427208in}}%
\pgfusepath{clip}%
\pgfsetrectcap%
\pgfsetroundjoin%
\pgfsetlinewidth{1.505625pt}%
\definecolor{currentstroke}{rgb}{0.501961,0.501961,0.501961}%
\pgfsetstrokecolor{currentstroke}%
\pgfsetstrokeopacity{0.500000}%
\pgfsetdash{}{0pt}%
\pgfpathmoveto{\pgfqpoint{4.099020in}{3.089954in}}%
\pgfusepath{stroke}%
\end{pgfscope}%
\begin{pgfscope}%
\pgfpathrectangle{\pgfqpoint{0.100000in}{2.413063in}}{\pgfqpoint{5.037500in}{3.427208in}}%
\pgfusepath{clip}%
\pgfsetbuttcap%
\pgfsetroundjoin%
\definecolor{currentfill}{rgb}{0.501961,0.501961,0.501961}%
\pgfsetfillcolor{currentfill}%
\pgfsetfillopacity{0.500000}%
\pgfsetlinewidth{0.250937pt}%
\definecolor{currentstroke}{rgb}{0.000000,0.000000,0.000000}%
\pgfsetstrokecolor{currentstroke}%
\pgfsetstrokeopacity{0.500000}%
\pgfsetdash{}{0pt}%
\pgfsys@defobject{currentmarker}{\pgfqpoint{-0.013889in}{-0.013889in}}{\pgfqpoint{0.013889in}{0.013889in}}{%
\pgfpathmoveto{\pgfqpoint{0.000000in}{-0.013889in}}%
\pgfpathcurveto{\pgfqpoint{0.003683in}{-0.013889in}}{\pgfqpoint{0.007216in}{-0.012425in}}{\pgfqpoint{0.009821in}{-0.009821in}}%
\pgfpathcurveto{\pgfqpoint{0.012425in}{-0.007216in}}{\pgfqpoint{0.013889in}{-0.003683in}}{\pgfqpoint{0.013889in}{0.000000in}}%
\pgfpathcurveto{\pgfqpoint{0.013889in}{0.003683in}}{\pgfqpoint{0.012425in}{0.007216in}}{\pgfqpoint{0.009821in}{0.009821in}}%
\pgfpathcurveto{\pgfqpoint{0.007216in}{0.012425in}}{\pgfqpoint{0.003683in}{0.013889in}}{\pgfqpoint{0.000000in}{0.013889in}}%
\pgfpathcurveto{\pgfqpoint{-0.003683in}{0.013889in}}{\pgfqpoint{-0.007216in}{0.012425in}}{\pgfqpoint{-0.009821in}{0.009821in}}%
\pgfpathcurveto{\pgfqpoint{-0.012425in}{0.007216in}}{\pgfqpoint{-0.013889in}{0.003683in}}{\pgfqpoint{-0.013889in}{0.000000in}}%
\pgfpathcurveto{\pgfqpoint{-0.013889in}{-0.003683in}}{\pgfqpoint{-0.012425in}{-0.007216in}}{\pgfqpoint{-0.009821in}{-0.009821in}}%
\pgfpathcurveto{\pgfqpoint{-0.007216in}{-0.012425in}}{\pgfqpoint{-0.003683in}{-0.013889in}}{\pgfqpoint{0.000000in}{-0.013889in}}%
\pgfpathclose%
\pgfusepath{stroke,fill}%
}%
\begin{pgfscope}%
\pgfsys@transformshift{4.099020in}{3.089954in}%
\pgfsys@useobject{currentmarker}{}%
\end{pgfscope}%
\end{pgfscope}%
\begin{pgfscope}%
\pgfpathrectangle{\pgfqpoint{0.100000in}{2.413063in}}{\pgfqpoint{5.037500in}{3.427208in}}%
\pgfusepath{clip}%
\pgfsetrectcap%
\pgfsetroundjoin%
\pgfsetlinewidth{1.505625pt}%
\definecolor{currentstroke}{rgb}{0.501961,0.501961,0.501961}%
\pgfsetstrokecolor{currentstroke}%
\pgfsetstrokeopacity{0.500000}%
\pgfsetdash{}{0pt}%
\pgfpathmoveto{\pgfqpoint{4.134085in}{3.213055in}}%
\pgfusepath{stroke}%
\end{pgfscope}%
\begin{pgfscope}%
\pgfpathrectangle{\pgfqpoint{0.100000in}{2.413063in}}{\pgfqpoint{5.037500in}{3.427208in}}%
\pgfusepath{clip}%
\pgfsetbuttcap%
\pgfsetroundjoin%
\definecolor{currentfill}{rgb}{0.501961,0.501961,0.501961}%
\pgfsetfillcolor{currentfill}%
\pgfsetfillopacity{0.500000}%
\pgfsetlinewidth{0.250937pt}%
\definecolor{currentstroke}{rgb}{0.000000,0.000000,0.000000}%
\pgfsetstrokecolor{currentstroke}%
\pgfsetstrokeopacity{0.500000}%
\pgfsetdash{}{0pt}%
\pgfsys@defobject{currentmarker}{\pgfqpoint{-0.013889in}{-0.013889in}}{\pgfqpoint{0.013889in}{0.013889in}}{%
\pgfpathmoveto{\pgfqpoint{0.000000in}{-0.013889in}}%
\pgfpathcurveto{\pgfqpoint{0.003683in}{-0.013889in}}{\pgfqpoint{0.007216in}{-0.012425in}}{\pgfqpoint{0.009821in}{-0.009821in}}%
\pgfpathcurveto{\pgfqpoint{0.012425in}{-0.007216in}}{\pgfqpoint{0.013889in}{-0.003683in}}{\pgfqpoint{0.013889in}{0.000000in}}%
\pgfpathcurveto{\pgfqpoint{0.013889in}{0.003683in}}{\pgfqpoint{0.012425in}{0.007216in}}{\pgfqpoint{0.009821in}{0.009821in}}%
\pgfpathcurveto{\pgfqpoint{0.007216in}{0.012425in}}{\pgfqpoint{0.003683in}{0.013889in}}{\pgfqpoint{0.000000in}{0.013889in}}%
\pgfpathcurveto{\pgfqpoint{-0.003683in}{0.013889in}}{\pgfqpoint{-0.007216in}{0.012425in}}{\pgfqpoint{-0.009821in}{0.009821in}}%
\pgfpathcurveto{\pgfqpoint{-0.012425in}{0.007216in}}{\pgfqpoint{-0.013889in}{0.003683in}}{\pgfqpoint{-0.013889in}{0.000000in}}%
\pgfpathcurveto{\pgfqpoint{-0.013889in}{-0.003683in}}{\pgfqpoint{-0.012425in}{-0.007216in}}{\pgfqpoint{-0.009821in}{-0.009821in}}%
\pgfpathcurveto{\pgfqpoint{-0.007216in}{-0.012425in}}{\pgfqpoint{-0.003683in}{-0.013889in}}{\pgfqpoint{0.000000in}{-0.013889in}}%
\pgfpathclose%
\pgfusepath{stroke,fill}%
}%
\begin{pgfscope}%
\pgfsys@transformshift{4.134085in}{3.213055in}%
\pgfsys@useobject{currentmarker}{}%
\end{pgfscope}%
\end{pgfscope}%
\begin{pgfscope}%
\pgfpathrectangle{\pgfqpoint{0.100000in}{2.413063in}}{\pgfqpoint{5.037500in}{3.427208in}}%
\pgfusepath{clip}%
\pgfsetrectcap%
\pgfsetroundjoin%
\pgfsetlinewidth{1.505625pt}%
\definecolor{currentstroke}{rgb}{0.678431,1.000000,0.184314}%
\pgfsetstrokecolor{currentstroke}%
\pgfsetstrokeopacity{0.500000}%
\pgfsetdash{}{0pt}%
\pgfpathmoveto{\pgfqpoint{3.874269in}{3.487855in}}%
\pgfusepath{stroke}%
\end{pgfscope}%
\begin{pgfscope}%
\pgfpathrectangle{\pgfqpoint{0.100000in}{2.413063in}}{\pgfqpoint{5.037500in}{3.427208in}}%
\pgfusepath{clip}%
\pgfsetbuttcap%
\pgfsetroundjoin%
\definecolor{currentfill}{rgb}{0.678431,1.000000,0.184314}%
\pgfsetfillcolor{currentfill}%
\pgfsetfillopacity{0.500000}%
\pgfsetlinewidth{0.250937pt}%
\definecolor{currentstroke}{rgb}{0.000000,0.000000,0.000000}%
\pgfsetstrokecolor{currentstroke}%
\pgfsetstrokeopacity{0.500000}%
\pgfsetdash{}{0pt}%
\pgfsys@defobject{currentmarker}{\pgfqpoint{-0.019444in}{-0.019444in}}{\pgfqpoint{0.019444in}{0.019444in}}{%
\pgfpathmoveto{\pgfqpoint{0.000000in}{-0.019444in}}%
\pgfpathcurveto{\pgfqpoint{0.005157in}{-0.019444in}}{\pgfqpoint{0.010103in}{-0.017396in}}{\pgfqpoint{0.013749in}{-0.013749in}}%
\pgfpathcurveto{\pgfqpoint{0.017396in}{-0.010103in}}{\pgfqpoint{0.019444in}{-0.005157in}}{\pgfqpoint{0.019444in}{0.000000in}}%
\pgfpathcurveto{\pgfqpoint{0.019444in}{0.005157in}}{\pgfqpoint{0.017396in}{0.010103in}}{\pgfqpoint{0.013749in}{0.013749in}}%
\pgfpathcurveto{\pgfqpoint{0.010103in}{0.017396in}}{\pgfqpoint{0.005157in}{0.019444in}}{\pgfqpoint{0.000000in}{0.019444in}}%
\pgfpathcurveto{\pgfqpoint{-0.005157in}{0.019444in}}{\pgfqpoint{-0.010103in}{0.017396in}}{\pgfqpoint{-0.013749in}{0.013749in}}%
\pgfpathcurveto{\pgfqpoint{-0.017396in}{0.010103in}}{\pgfqpoint{-0.019444in}{0.005157in}}{\pgfqpoint{-0.019444in}{0.000000in}}%
\pgfpathcurveto{\pgfqpoint{-0.019444in}{-0.005157in}}{\pgfqpoint{-0.017396in}{-0.010103in}}{\pgfqpoint{-0.013749in}{-0.013749in}}%
\pgfpathcurveto{\pgfqpoint{-0.010103in}{-0.017396in}}{\pgfqpoint{-0.005157in}{-0.019444in}}{\pgfqpoint{0.000000in}{-0.019444in}}%
\pgfpathclose%
\pgfusepath{stroke,fill}%
}%
\begin{pgfscope}%
\pgfsys@transformshift{3.874269in}{3.487855in}%
\pgfsys@useobject{currentmarker}{}%
\end{pgfscope}%
\end{pgfscope}%
\begin{pgfscope}%
\pgfpathrectangle{\pgfqpoint{0.100000in}{2.413063in}}{\pgfqpoint{5.037500in}{3.427208in}}%
\pgfusepath{clip}%
\pgfsetrectcap%
\pgfsetroundjoin%
\pgfsetlinewidth{1.505625pt}%
\definecolor{currentstroke}{rgb}{0.678431,1.000000,0.184314}%
\pgfsetstrokecolor{currentstroke}%
\pgfsetstrokeopacity{0.500000}%
\pgfsetdash{}{0pt}%
\pgfpathmoveto{\pgfqpoint{3.914523in}{3.770303in}}%
\pgfusepath{stroke}%
\end{pgfscope}%
\begin{pgfscope}%
\pgfpathrectangle{\pgfqpoint{0.100000in}{2.413063in}}{\pgfqpoint{5.037500in}{3.427208in}}%
\pgfusepath{clip}%
\pgfsetbuttcap%
\pgfsetroundjoin%
\definecolor{currentfill}{rgb}{0.678431,1.000000,0.184314}%
\pgfsetfillcolor{currentfill}%
\pgfsetfillopacity{0.500000}%
\pgfsetlinewidth{0.250937pt}%
\definecolor{currentstroke}{rgb}{0.000000,0.000000,0.000000}%
\pgfsetstrokecolor{currentstroke}%
\pgfsetstrokeopacity{0.500000}%
\pgfsetdash{}{0pt}%
\pgfsys@defobject{currentmarker}{\pgfqpoint{-0.016667in}{-0.016667in}}{\pgfqpoint{0.016667in}{0.016667in}}{%
\pgfpathmoveto{\pgfqpoint{0.000000in}{-0.016667in}}%
\pgfpathcurveto{\pgfqpoint{0.004420in}{-0.016667in}}{\pgfqpoint{0.008660in}{-0.014911in}}{\pgfqpoint{0.011785in}{-0.011785in}}%
\pgfpathcurveto{\pgfqpoint{0.014911in}{-0.008660in}}{\pgfqpoint{0.016667in}{-0.004420in}}{\pgfqpoint{0.016667in}{0.000000in}}%
\pgfpathcurveto{\pgfqpoint{0.016667in}{0.004420in}}{\pgfqpoint{0.014911in}{0.008660in}}{\pgfqpoint{0.011785in}{0.011785in}}%
\pgfpathcurveto{\pgfqpoint{0.008660in}{0.014911in}}{\pgfqpoint{0.004420in}{0.016667in}}{\pgfqpoint{0.000000in}{0.016667in}}%
\pgfpathcurveto{\pgfqpoint{-0.004420in}{0.016667in}}{\pgfqpoint{-0.008660in}{0.014911in}}{\pgfqpoint{-0.011785in}{0.011785in}}%
\pgfpathcurveto{\pgfqpoint{-0.014911in}{0.008660in}}{\pgfqpoint{-0.016667in}{0.004420in}}{\pgfqpoint{-0.016667in}{0.000000in}}%
\pgfpathcurveto{\pgfqpoint{-0.016667in}{-0.004420in}}{\pgfqpoint{-0.014911in}{-0.008660in}}{\pgfqpoint{-0.011785in}{-0.011785in}}%
\pgfpathcurveto{\pgfqpoint{-0.008660in}{-0.014911in}}{\pgfqpoint{-0.004420in}{-0.016667in}}{\pgfqpoint{0.000000in}{-0.016667in}}%
\pgfpathclose%
\pgfusepath{stroke,fill}%
}%
\begin{pgfscope}%
\pgfsys@transformshift{3.914523in}{3.770303in}%
\pgfsys@useobject{currentmarker}{}%
\end{pgfscope}%
\end{pgfscope}%
\begin{pgfscope}%
\pgfpathrectangle{\pgfqpoint{0.100000in}{2.413063in}}{\pgfqpoint{5.037500in}{3.427208in}}%
\pgfusepath{clip}%
\pgfsetrectcap%
\pgfsetroundjoin%
\pgfsetlinewidth{1.505625pt}%
\definecolor{currentstroke}{rgb}{0.678431,1.000000,0.184314}%
\pgfsetstrokecolor{currentstroke}%
\pgfsetstrokeopacity{0.500000}%
\pgfsetdash{}{0pt}%
\pgfpathmoveto{\pgfqpoint{3.821936in}{3.734931in}}%
\pgfusepath{stroke}%
\end{pgfscope}%
\begin{pgfscope}%
\pgfpathrectangle{\pgfqpoint{0.100000in}{2.413063in}}{\pgfqpoint{5.037500in}{3.427208in}}%
\pgfusepath{clip}%
\pgfsetbuttcap%
\pgfsetroundjoin%
\definecolor{currentfill}{rgb}{0.678431,1.000000,0.184314}%
\pgfsetfillcolor{currentfill}%
\pgfsetfillopacity{0.500000}%
\pgfsetlinewidth{0.250937pt}%
\definecolor{currentstroke}{rgb}{0.000000,0.000000,0.000000}%
\pgfsetstrokecolor{currentstroke}%
\pgfsetstrokeopacity{0.500000}%
\pgfsetdash{}{0pt}%
\pgfsys@defobject{currentmarker}{\pgfqpoint{-0.016667in}{-0.016667in}}{\pgfqpoint{0.016667in}{0.016667in}}{%
\pgfpathmoveto{\pgfqpoint{0.000000in}{-0.016667in}}%
\pgfpathcurveto{\pgfqpoint{0.004420in}{-0.016667in}}{\pgfqpoint{0.008660in}{-0.014911in}}{\pgfqpoint{0.011785in}{-0.011785in}}%
\pgfpathcurveto{\pgfqpoint{0.014911in}{-0.008660in}}{\pgfqpoint{0.016667in}{-0.004420in}}{\pgfqpoint{0.016667in}{0.000000in}}%
\pgfpathcurveto{\pgfqpoint{0.016667in}{0.004420in}}{\pgfqpoint{0.014911in}{0.008660in}}{\pgfqpoint{0.011785in}{0.011785in}}%
\pgfpathcurveto{\pgfqpoint{0.008660in}{0.014911in}}{\pgfqpoint{0.004420in}{0.016667in}}{\pgfqpoint{0.000000in}{0.016667in}}%
\pgfpathcurveto{\pgfqpoint{-0.004420in}{0.016667in}}{\pgfqpoint{-0.008660in}{0.014911in}}{\pgfqpoint{-0.011785in}{0.011785in}}%
\pgfpathcurveto{\pgfqpoint{-0.014911in}{0.008660in}}{\pgfqpoint{-0.016667in}{0.004420in}}{\pgfqpoint{-0.016667in}{0.000000in}}%
\pgfpathcurveto{\pgfqpoint{-0.016667in}{-0.004420in}}{\pgfqpoint{-0.014911in}{-0.008660in}}{\pgfqpoint{-0.011785in}{-0.011785in}}%
\pgfpathcurveto{\pgfqpoint{-0.008660in}{-0.014911in}}{\pgfqpoint{-0.004420in}{-0.016667in}}{\pgfqpoint{0.000000in}{-0.016667in}}%
\pgfpathclose%
\pgfusepath{stroke,fill}%
}%
\begin{pgfscope}%
\pgfsys@transformshift{3.821936in}{3.734931in}%
\pgfsys@useobject{currentmarker}{}%
\end{pgfscope}%
\end{pgfscope}%
\begin{pgfscope}%
\pgfpathrectangle{\pgfqpoint{0.100000in}{2.413063in}}{\pgfqpoint{5.037500in}{3.427208in}}%
\pgfusepath{clip}%
\pgfsetrectcap%
\pgfsetroundjoin%
\pgfsetlinewidth{1.505625pt}%
\definecolor{currentstroke}{rgb}{0.678431,1.000000,0.184314}%
\pgfsetstrokecolor{currentstroke}%
\pgfsetstrokeopacity{0.500000}%
\pgfsetdash{}{0pt}%
\pgfpathmoveto{\pgfqpoint{4.056797in}{3.733205in}}%
\pgfusepath{stroke}%
\end{pgfscope}%
\begin{pgfscope}%
\pgfpathrectangle{\pgfqpoint{0.100000in}{2.413063in}}{\pgfqpoint{5.037500in}{3.427208in}}%
\pgfusepath{clip}%
\pgfsetbuttcap%
\pgfsetroundjoin%
\definecolor{currentfill}{rgb}{0.678431,1.000000,0.184314}%
\pgfsetfillcolor{currentfill}%
\pgfsetfillopacity{0.500000}%
\pgfsetlinewidth{0.250937pt}%
\definecolor{currentstroke}{rgb}{0.000000,0.000000,0.000000}%
\pgfsetstrokecolor{currentstroke}%
\pgfsetstrokeopacity{0.500000}%
\pgfsetdash{}{0pt}%
\pgfsys@defobject{currentmarker}{\pgfqpoint{-0.011111in}{-0.011111in}}{\pgfqpoint{0.011111in}{0.011111in}}{%
\pgfpathmoveto{\pgfqpoint{0.000000in}{-0.011111in}}%
\pgfpathcurveto{\pgfqpoint{0.002947in}{-0.011111in}}{\pgfqpoint{0.005773in}{-0.009940in}}{\pgfqpoint{0.007857in}{-0.007857in}}%
\pgfpathcurveto{\pgfqpoint{0.009940in}{-0.005773in}}{\pgfqpoint{0.011111in}{-0.002947in}}{\pgfqpoint{0.011111in}{0.000000in}}%
\pgfpathcurveto{\pgfqpoint{0.011111in}{0.002947in}}{\pgfqpoint{0.009940in}{0.005773in}}{\pgfqpoint{0.007857in}{0.007857in}}%
\pgfpathcurveto{\pgfqpoint{0.005773in}{0.009940in}}{\pgfqpoint{0.002947in}{0.011111in}}{\pgfqpoint{0.000000in}{0.011111in}}%
\pgfpathcurveto{\pgfqpoint{-0.002947in}{0.011111in}}{\pgfqpoint{-0.005773in}{0.009940in}}{\pgfqpoint{-0.007857in}{0.007857in}}%
\pgfpathcurveto{\pgfqpoint{-0.009940in}{0.005773in}}{\pgfqpoint{-0.011111in}{0.002947in}}{\pgfqpoint{-0.011111in}{0.000000in}}%
\pgfpathcurveto{\pgfqpoint{-0.011111in}{-0.002947in}}{\pgfqpoint{-0.009940in}{-0.005773in}}{\pgfqpoint{-0.007857in}{-0.007857in}}%
\pgfpathcurveto{\pgfqpoint{-0.005773in}{-0.009940in}}{\pgfqpoint{-0.002947in}{-0.011111in}}{\pgfqpoint{0.000000in}{-0.011111in}}%
\pgfpathclose%
\pgfusepath{stroke,fill}%
}%
\begin{pgfscope}%
\pgfsys@transformshift{4.056797in}{3.733205in}%
\pgfsys@useobject{currentmarker}{}%
\end{pgfscope}%
\end{pgfscope}%
\begin{pgfscope}%
\pgfpathrectangle{\pgfqpoint{0.100000in}{2.413063in}}{\pgfqpoint{5.037500in}{3.427208in}}%
\pgfusepath{clip}%
\pgfsetrectcap%
\pgfsetroundjoin%
\pgfsetlinewidth{1.505625pt}%
\definecolor{currentstroke}{rgb}{0.678431,1.000000,0.184314}%
\pgfsetstrokecolor{currentstroke}%
\pgfsetstrokeopacity{0.500000}%
\pgfsetdash{}{0pt}%
\pgfpathmoveto{\pgfqpoint{4.142799in}{3.473859in}}%
\pgfusepath{stroke}%
\end{pgfscope}%
\begin{pgfscope}%
\pgfpathrectangle{\pgfqpoint{0.100000in}{2.413063in}}{\pgfqpoint{5.037500in}{3.427208in}}%
\pgfusepath{clip}%
\pgfsetbuttcap%
\pgfsetroundjoin%
\definecolor{currentfill}{rgb}{0.678431,1.000000,0.184314}%
\pgfsetfillcolor{currentfill}%
\pgfsetfillopacity{0.500000}%
\pgfsetlinewidth{0.250937pt}%
\definecolor{currentstroke}{rgb}{0.000000,0.000000,0.000000}%
\pgfsetstrokecolor{currentstroke}%
\pgfsetstrokeopacity{0.500000}%
\pgfsetdash{}{0pt}%
\pgfsys@defobject{currentmarker}{\pgfqpoint{-0.022222in}{-0.022222in}}{\pgfqpoint{0.022222in}{0.022222in}}{%
\pgfpathmoveto{\pgfqpoint{0.000000in}{-0.022222in}}%
\pgfpathcurveto{\pgfqpoint{0.005893in}{-0.022222in}}{\pgfqpoint{0.011546in}{-0.019881in}}{\pgfqpoint{0.015713in}{-0.015713in}}%
\pgfpathcurveto{\pgfqpoint{0.019881in}{-0.011546in}}{\pgfqpoint{0.022222in}{-0.005893in}}{\pgfqpoint{0.022222in}{0.000000in}}%
\pgfpathcurveto{\pgfqpoint{0.022222in}{0.005893in}}{\pgfqpoint{0.019881in}{0.011546in}}{\pgfqpoint{0.015713in}{0.015713in}}%
\pgfpathcurveto{\pgfqpoint{0.011546in}{0.019881in}}{\pgfqpoint{0.005893in}{0.022222in}}{\pgfqpoint{0.000000in}{0.022222in}}%
\pgfpathcurveto{\pgfqpoint{-0.005893in}{0.022222in}}{\pgfqpoint{-0.011546in}{0.019881in}}{\pgfqpoint{-0.015713in}{0.015713in}}%
\pgfpathcurveto{\pgfqpoint{-0.019881in}{0.011546in}}{\pgfqpoint{-0.022222in}{0.005893in}}{\pgfqpoint{-0.022222in}{0.000000in}}%
\pgfpathcurveto{\pgfqpoint{-0.022222in}{-0.005893in}}{\pgfqpoint{-0.019881in}{-0.011546in}}{\pgfqpoint{-0.015713in}{-0.015713in}}%
\pgfpathcurveto{\pgfqpoint{-0.011546in}{-0.019881in}}{\pgfqpoint{-0.005893in}{-0.022222in}}{\pgfqpoint{0.000000in}{-0.022222in}}%
\pgfpathclose%
\pgfusepath{stroke,fill}%
}%
\begin{pgfscope}%
\pgfsys@transformshift{4.142799in}{3.473859in}%
\pgfsys@useobject{currentmarker}{}%
\end{pgfscope}%
\end{pgfscope}%
\begin{pgfscope}%
\pgfpathrectangle{\pgfqpoint{0.100000in}{2.413063in}}{\pgfqpoint{5.037500in}{3.427208in}}%
\pgfusepath{clip}%
\pgfsetrectcap%
\pgfsetroundjoin%
\pgfsetlinewidth{1.505625pt}%
\definecolor{currentstroke}{rgb}{0.678431,1.000000,0.184314}%
\pgfsetstrokecolor{currentstroke}%
\pgfsetstrokeopacity{0.500000}%
\pgfsetdash{}{0pt}%
\pgfpathmoveto{\pgfqpoint{3.781250in}{3.580214in}}%
\pgfusepath{stroke}%
\end{pgfscope}%
\begin{pgfscope}%
\pgfpathrectangle{\pgfqpoint{0.100000in}{2.413063in}}{\pgfqpoint{5.037500in}{3.427208in}}%
\pgfusepath{clip}%
\pgfsetbuttcap%
\pgfsetroundjoin%
\definecolor{currentfill}{rgb}{0.678431,1.000000,0.184314}%
\pgfsetfillcolor{currentfill}%
\pgfsetfillopacity{0.500000}%
\pgfsetlinewidth{0.250937pt}%
\definecolor{currentstroke}{rgb}{0.000000,0.000000,0.000000}%
\pgfsetstrokecolor{currentstroke}%
\pgfsetstrokeopacity{0.500000}%
\pgfsetdash{}{0pt}%
\pgfsys@defobject{currentmarker}{\pgfqpoint{-0.019444in}{-0.019444in}}{\pgfqpoint{0.019444in}{0.019444in}}{%
\pgfpathmoveto{\pgfqpoint{0.000000in}{-0.019444in}}%
\pgfpathcurveto{\pgfqpoint{0.005157in}{-0.019444in}}{\pgfqpoint{0.010103in}{-0.017396in}}{\pgfqpoint{0.013749in}{-0.013749in}}%
\pgfpathcurveto{\pgfqpoint{0.017396in}{-0.010103in}}{\pgfqpoint{0.019444in}{-0.005157in}}{\pgfqpoint{0.019444in}{0.000000in}}%
\pgfpathcurveto{\pgfqpoint{0.019444in}{0.005157in}}{\pgfqpoint{0.017396in}{0.010103in}}{\pgfqpoint{0.013749in}{0.013749in}}%
\pgfpathcurveto{\pgfqpoint{0.010103in}{0.017396in}}{\pgfqpoint{0.005157in}{0.019444in}}{\pgfqpoint{0.000000in}{0.019444in}}%
\pgfpathcurveto{\pgfqpoint{-0.005157in}{0.019444in}}{\pgfqpoint{-0.010103in}{0.017396in}}{\pgfqpoint{-0.013749in}{0.013749in}}%
\pgfpathcurveto{\pgfqpoint{-0.017396in}{0.010103in}}{\pgfqpoint{-0.019444in}{0.005157in}}{\pgfqpoint{-0.019444in}{0.000000in}}%
\pgfpathcurveto{\pgfqpoint{-0.019444in}{-0.005157in}}{\pgfqpoint{-0.017396in}{-0.010103in}}{\pgfqpoint{-0.013749in}{-0.013749in}}%
\pgfpathcurveto{\pgfqpoint{-0.010103in}{-0.017396in}}{\pgfqpoint{-0.005157in}{-0.019444in}}{\pgfqpoint{0.000000in}{-0.019444in}}%
\pgfpathclose%
\pgfusepath{stroke,fill}%
}%
\begin{pgfscope}%
\pgfsys@transformshift{3.781250in}{3.580214in}%
\pgfsys@useobject{currentmarker}{}%
\end{pgfscope}%
\end{pgfscope}%
\begin{pgfscope}%
\pgfpathrectangle{\pgfqpoint{0.100000in}{2.413063in}}{\pgfqpoint{5.037500in}{3.427208in}}%
\pgfusepath{clip}%
\pgfsetrectcap%
\pgfsetroundjoin%
\pgfsetlinewidth{1.505625pt}%
\definecolor{currentstroke}{rgb}{0.678431,1.000000,0.184314}%
\pgfsetstrokecolor{currentstroke}%
\pgfsetstrokeopacity{0.500000}%
\pgfsetdash{}{0pt}%
\pgfpathmoveto{\pgfqpoint{3.763715in}{3.841616in}}%
\pgfusepath{stroke}%
\end{pgfscope}%
\begin{pgfscope}%
\pgfpathrectangle{\pgfqpoint{0.100000in}{2.413063in}}{\pgfqpoint{5.037500in}{3.427208in}}%
\pgfusepath{clip}%
\pgfsetbuttcap%
\pgfsetroundjoin%
\definecolor{currentfill}{rgb}{0.678431,1.000000,0.184314}%
\pgfsetfillcolor{currentfill}%
\pgfsetfillopacity{0.500000}%
\pgfsetlinewidth{0.250937pt}%
\definecolor{currentstroke}{rgb}{0.000000,0.000000,0.000000}%
\pgfsetstrokecolor{currentstroke}%
\pgfsetstrokeopacity{0.500000}%
\pgfsetdash{}{0pt}%
\pgfsys@defobject{currentmarker}{\pgfqpoint{-0.075000in}{-0.075000in}}{\pgfqpoint{0.075000in}{0.075000in}}{%
\pgfpathmoveto{\pgfqpoint{0.000000in}{-0.075000in}}%
\pgfpathcurveto{\pgfqpoint{0.019890in}{-0.075000in}}{\pgfqpoint{0.038968in}{-0.067098in}}{\pgfqpoint{0.053033in}{-0.053033in}}%
\pgfpathcurveto{\pgfqpoint{0.067098in}{-0.038968in}}{\pgfqpoint{0.075000in}{-0.019890in}}{\pgfqpoint{0.075000in}{0.000000in}}%
\pgfpathcurveto{\pgfqpoint{0.075000in}{0.019890in}}{\pgfqpoint{0.067098in}{0.038968in}}{\pgfqpoint{0.053033in}{0.053033in}}%
\pgfpathcurveto{\pgfqpoint{0.038968in}{0.067098in}}{\pgfqpoint{0.019890in}{0.075000in}}{\pgfqpoint{0.000000in}{0.075000in}}%
\pgfpathcurveto{\pgfqpoint{-0.019890in}{0.075000in}}{\pgfqpoint{-0.038968in}{0.067098in}}{\pgfqpoint{-0.053033in}{0.053033in}}%
\pgfpathcurveto{\pgfqpoint{-0.067098in}{0.038968in}}{\pgfqpoint{-0.075000in}{0.019890in}}{\pgfqpoint{-0.075000in}{0.000000in}}%
\pgfpathcurveto{\pgfqpoint{-0.075000in}{-0.019890in}}{\pgfqpoint{-0.067098in}{-0.038968in}}{\pgfqpoint{-0.053033in}{-0.053033in}}%
\pgfpathcurveto{\pgfqpoint{-0.038968in}{-0.067098in}}{\pgfqpoint{-0.019890in}{-0.075000in}}{\pgfqpoint{0.000000in}{-0.075000in}}%
\pgfpathclose%
\pgfusepath{stroke,fill}%
}%
\begin{pgfscope}%
\pgfsys@transformshift{3.763715in}{3.841616in}%
\pgfsys@useobject{currentmarker}{}%
\end{pgfscope}%
\end{pgfscope}%
\begin{pgfscope}%
\pgfpathrectangle{\pgfqpoint{0.100000in}{2.413063in}}{\pgfqpoint{5.037500in}{3.427208in}}%
\pgfusepath{clip}%
\pgfsetrectcap%
\pgfsetroundjoin%
\pgfsetlinewidth{1.505625pt}%
\definecolor{currentstroke}{rgb}{0.678431,1.000000,0.184314}%
\pgfsetstrokecolor{currentstroke}%
\pgfsetstrokeopacity{0.500000}%
\pgfsetdash{}{0pt}%
\pgfpathmoveto{\pgfqpoint{3.868196in}{3.804393in}}%
\pgfusepath{stroke}%
\end{pgfscope}%
\begin{pgfscope}%
\pgfpathrectangle{\pgfqpoint{0.100000in}{2.413063in}}{\pgfqpoint{5.037500in}{3.427208in}}%
\pgfusepath{clip}%
\pgfsetbuttcap%
\pgfsetroundjoin%
\definecolor{currentfill}{rgb}{0.678431,1.000000,0.184314}%
\pgfsetfillcolor{currentfill}%
\pgfsetfillopacity{0.500000}%
\pgfsetlinewidth{0.250937pt}%
\definecolor{currentstroke}{rgb}{0.000000,0.000000,0.000000}%
\pgfsetstrokecolor{currentstroke}%
\pgfsetstrokeopacity{0.500000}%
\pgfsetdash{}{0pt}%
\pgfsys@defobject{currentmarker}{\pgfqpoint{-0.016667in}{-0.016667in}}{\pgfqpoint{0.016667in}{0.016667in}}{%
\pgfpathmoveto{\pgfqpoint{0.000000in}{-0.016667in}}%
\pgfpathcurveto{\pgfqpoint{0.004420in}{-0.016667in}}{\pgfqpoint{0.008660in}{-0.014911in}}{\pgfqpoint{0.011785in}{-0.011785in}}%
\pgfpathcurveto{\pgfqpoint{0.014911in}{-0.008660in}}{\pgfqpoint{0.016667in}{-0.004420in}}{\pgfqpoint{0.016667in}{0.000000in}}%
\pgfpathcurveto{\pgfqpoint{0.016667in}{0.004420in}}{\pgfqpoint{0.014911in}{0.008660in}}{\pgfqpoint{0.011785in}{0.011785in}}%
\pgfpathcurveto{\pgfqpoint{0.008660in}{0.014911in}}{\pgfqpoint{0.004420in}{0.016667in}}{\pgfqpoint{0.000000in}{0.016667in}}%
\pgfpathcurveto{\pgfqpoint{-0.004420in}{0.016667in}}{\pgfqpoint{-0.008660in}{0.014911in}}{\pgfqpoint{-0.011785in}{0.011785in}}%
\pgfpathcurveto{\pgfqpoint{-0.014911in}{0.008660in}}{\pgfqpoint{-0.016667in}{0.004420in}}{\pgfqpoint{-0.016667in}{0.000000in}}%
\pgfpathcurveto{\pgfqpoint{-0.016667in}{-0.004420in}}{\pgfqpoint{-0.014911in}{-0.008660in}}{\pgfqpoint{-0.011785in}{-0.011785in}}%
\pgfpathcurveto{\pgfqpoint{-0.008660in}{-0.014911in}}{\pgfqpoint{-0.004420in}{-0.016667in}}{\pgfqpoint{0.000000in}{-0.016667in}}%
\pgfpathclose%
\pgfusepath{stroke,fill}%
}%
\begin{pgfscope}%
\pgfsys@transformshift{3.868196in}{3.804393in}%
\pgfsys@useobject{currentmarker}{}%
\end{pgfscope}%
\end{pgfscope}%
\begin{pgfscope}%
\pgfpathrectangle{\pgfqpoint{0.100000in}{2.413063in}}{\pgfqpoint{5.037500in}{3.427208in}}%
\pgfusepath{clip}%
\pgfsetrectcap%
\pgfsetroundjoin%
\pgfsetlinewidth{1.505625pt}%
\definecolor{currentstroke}{rgb}{0.678431,1.000000,0.184314}%
\pgfsetstrokecolor{currentstroke}%
\pgfsetstrokeopacity{0.500000}%
\pgfsetdash{}{0pt}%
\pgfpathmoveto{\pgfqpoint{4.120605in}{3.552479in}}%
\pgfusepath{stroke}%
\end{pgfscope}%
\begin{pgfscope}%
\pgfpathrectangle{\pgfqpoint{0.100000in}{2.413063in}}{\pgfqpoint{5.037500in}{3.427208in}}%
\pgfusepath{clip}%
\pgfsetbuttcap%
\pgfsetroundjoin%
\definecolor{currentfill}{rgb}{0.678431,1.000000,0.184314}%
\pgfsetfillcolor{currentfill}%
\pgfsetfillopacity{0.500000}%
\pgfsetlinewidth{0.250937pt}%
\definecolor{currentstroke}{rgb}{0.000000,0.000000,0.000000}%
\pgfsetstrokecolor{currentstroke}%
\pgfsetstrokeopacity{0.500000}%
\pgfsetdash{}{0pt}%
\pgfsys@defobject{currentmarker}{\pgfqpoint{-0.027778in}{-0.027778in}}{\pgfqpoint{0.027778in}{0.027778in}}{%
\pgfpathmoveto{\pgfqpoint{0.000000in}{-0.027778in}}%
\pgfpathcurveto{\pgfqpoint{0.007367in}{-0.027778in}}{\pgfqpoint{0.014433in}{-0.024851in}}{\pgfqpoint{0.019642in}{-0.019642in}}%
\pgfpathcurveto{\pgfqpoint{0.024851in}{-0.014433in}}{\pgfqpoint{0.027778in}{-0.007367in}}{\pgfqpoint{0.027778in}{0.000000in}}%
\pgfpathcurveto{\pgfqpoint{0.027778in}{0.007367in}}{\pgfqpoint{0.024851in}{0.014433in}}{\pgfqpoint{0.019642in}{0.019642in}}%
\pgfpathcurveto{\pgfqpoint{0.014433in}{0.024851in}}{\pgfqpoint{0.007367in}{0.027778in}}{\pgfqpoint{0.000000in}{0.027778in}}%
\pgfpathcurveto{\pgfqpoint{-0.007367in}{0.027778in}}{\pgfqpoint{-0.014433in}{0.024851in}}{\pgfqpoint{-0.019642in}{0.019642in}}%
\pgfpathcurveto{\pgfqpoint{-0.024851in}{0.014433in}}{\pgfqpoint{-0.027778in}{0.007367in}}{\pgfqpoint{-0.027778in}{0.000000in}}%
\pgfpathcurveto{\pgfqpoint{-0.027778in}{-0.007367in}}{\pgfqpoint{-0.024851in}{-0.014433in}}{\pgfqpoint{-0.019642in}{-0.019642in}}%
\pgfpathcurveto{\pgfqpoint{-0.014433in}{-0.024851in}}{\pgfqpoint{-0.007367in}{-0.027778in}}{\pgfqpoint{0.000000in}{-0.027778in}}%
\pgfpathclose%
\pgfusepath{stroke,fill}%
}%
\begin{pgfscope}%
\pgfsys@transformshift{4.120605in}{3.552479in}%
\pgfsys@useobject{currentmarker}{}%
\end{pgfscope}%
\end{pgfscope}%
\begin{pgfscope}%
\pgfpathrectangle{\pgfqpoint{0.100000in}{2.413063in}}{\pgfqpoint{5.037500in}{3.427208in}}%
\pgfusepath{clip}%
\pgfsetrectcap%
\pgfsetroundjoin%
\pgfsetlinewidth{1.505625pt}%
\definecolor{currentstroke}{rgb}{0.678431,1.000000,0.184314}%
\pgfsetstrokecolor{currentstroke}%
\pgfsetstrokeopacity{0.500000}%
\pgfsetdash{}{0pt}%
\pgfpathmoveto{\pgfqpoint{3.907380in}{3.639417in}}%
\pgfusepath{stroke}%
\end{pgfscope}%
\begin{pgfscope}%
\pgfpathrectangle{\pgfqpoint{0.100000in}{2.413063in}}{\pgfqpoint{5.037500in}{3.427208in}}%
\pgfusepath{clip}%
\pgfsetbuttcap%
\pgfsetroundjoin%
\definecolor{currentfill}{rgb}{0.678431,1.000000,0.184314}%
\pgfsetfillcolor{currentfill}%
\pgfsetfillopacity{0.500000}%
\pgfsetlinewidth{0.250937pt}%
\definecolor{currentstroke}{rgb}{0.000000,0.000000,0.000000}%
\pgfsetstrokecolor{currentstroke}%
\pgfsetstrokeopacity{0.500000}%
\pgfsetdash{}{0pt}%
\pgfsys@defobject{currentmarker}{\pgfqpoint{-0.019444in}{-0.019444in}}{\pgfqpoint{0.019444in}{0.019444in}}{%
\pgfpathmoveto{\pgfqpoint{0.000000in}{-0.019444in}}%
\pgfpathcurveto{\pgfqpoint{0.005157in}{-0.019444in}}{\pgfqpoint{0.010103in}{-0.017396in}}{\pgfqpoint{0.013749in}{-0.013749in}}%
\pgfpathcurveto{\pgfqpoint{0.017396in}{-0.010103in}}{\pgfqpoint{0.019444in}{-0.005157in}}{\pgfqpoint{0.019444in}{0.000000in}}%
\pgfpathcurveto{\pgfqpoint{0.019444in}{0.005157in}}{\pgfqpoint{0.017396in}{0.010103in}}{\pgfqpoint{0.013749in}{0.013749in}}%
\pgfpathcurveto{\pgfqpoint{0.010103in}{0.017396in}}{\pgfqpoint{0.005157in}{0.019444in}}{\pgfqpoint{0.000000in}{0.019444in}}%
\pgfpathcurveto{\pgfqpoint{-0.005157in}{0.019444in}}{\pgfqpoint{-0.010103in}{0.017396in}}{\pgfqpoint{-0.013749in}{0.013749in}}%
\pgfpathcurveto{\pgfqpoint{-0.017396in}{0.010103in}}{\pgfqpoint{-0.019444in}{0.005157in}}{\pgfqpoint{-0.019444in}{0.000000in}}%
\pgfpathcurveto{\pgfqpoint{-0.019444in}{-0.005157in}}{\pgfqpoint{-0.017396in}{-0.010103in}}{\pgfqpoint{-0.013749in}{-0.013749in}}%
\pgfpathcurveto{\pgfqpoint{-0.010103in}{-0.017396in}}{\pgfqpoint{-0.005157in}{-0.019444in}}{\pgfqpoint{0.000000in}{-0.019444in}}%
\pgfpathclose%
\pgfusepath{stroke,fill}%
}%
\begin{pgfscope}%
\pgfsys@transformshift{3.907380in}{3.639417in}%
\pgfsys@useobject{currentmarker}{}%
\end{pgfscope}%
\end{pgfscope}%
\begin{pgfscope}%
\pgfpathrectangle{\pgfqpoint{0.100000in}{2.413063in}}{\pgfqpoint{5.037500in}{3.427208in}}%
\pgfusepath{clip}%
\pgfsetrectcap%
\pgfsetroundjoin%
\pgfsetlinewidth{1.505625pt}%
\definecolor{currentstroke}{rgb}{0.678431,1.000000,0.184314}%
\pgfsetstrokecolor{currentstroke}%
\pgfsetstrokeopacity{0.500000}%
\pgfsetdash{}{0pt}%
\pgfpathmoveto{\pgfqpoint{3.741599in}{3.784942in}}%
\pgfusepath{stroke}%
\end{pgfscope}%
\begin{pgfscope}%
\pgfpathrectangle{\pgfqpoint{0.100000in}{2.413063in}}{\pgfqpoint{5.037500in}{3.427208in}}%
\pgfusepath{clip}%
\pgfsetbuttcap%
\pgfsetroundjoin%
\definecolor{currentfill}{rgb}{0.678431,1.000000,0.184314}%
\pgfsetfillcolor{currentfill}%
\pgfsetfillopacity{0.500000}%
\pgfsetlinewidth{0.250937pt}%
\definecolor{currentstroke}{rgb}{0.000000,0.000000,0.000000}%
\pgfsetstrokecolor{currentstroke}%
\pgfsetstrokeopacity{0.500000}%
\pgfsetdash{}{0pt}%
\pgfsys@defobject{currentmarker}{\pgfqpoint{-0.050000in}{-0.050000in}}{\pgfqpoint{0.050000in}{0.050000in}}{%
\pgfpathmoveto{\pgfqpoint{0.000000in}{-0.050000in}}%
\pgfpathcurveto{\pgfqpoint{0.013260in}{-0.050000in}}{\pgfqpoint{0.025979in}{-0.044732in}}{\pgfqpoint{0.035355in}{-0.035355in}}%
\pgfpathcurveto{\pgfqpoint{0.044732in}{-0.025979in}}{\pgfqpoint{0.050000in}{-0.013260in}}{\pgfqpoint{0.050000in}{0.000000in}}%
\pgfpathcurveto{\pgfqpoint{0.050000in}{0.013260in}}{\pgfqpoint{0.044732in}{0.025979in}}{\pgfqpoint{0.035355in}{0.035355in}}%
\pgfpathcurveto{\pgfqpoint{0.025979in}{0.044732in}}{\pgfqpoint{0.013260in}{0.050000in}}{\pgfqpoint{0.000000in}{0.050000in}}%
\pgfpathcurveto{\pgfqpoint{-0.013260in}{0.050000in}}{\pgfqpoint{-0.025979in}{0.044732in}}{\pgfqpoint{-0.035355in}{0.035355in}}%
\pgfpathcurveto{\pgfqpoint{-0.044732in}{0.025979in}}{\pgfqpoint{-0.050000in}{0.013260in}}{\pgfqpoint{-0.050000in}{0.000000in}}%
\pgfpathcurveto{\pgfqpoint{-0.050000in}{-0.013260in}}{\pgfqpoint{-0.044732in}{-0.025979in}}{\pgfqpoint{-0.035355in}{-0.035355in}}%
\pgfpathcurveto{\pgfqpoint{-0.025979in}{-0.044732in}}{\pgfqpoint{-0.013260in}{-0.050000in}}{\pgfqpoint{0.000000in}{-0.050000in}}%
\pgfpathclose%
\pgfusepath{stroke,fill}%
}%
\begin{pgfscope}%
\pgfsys@transformshift{3.741599in}{3.784942in}%
\pgfsys@useobject{currentmarker}{}%
\end{pgfscope}%
\end{pgfscope}%
\begin{pgfscope}%
\pgfpathrectangle{\pgfqpoint{0.100000in}{2.413063in}}{\pgfqpoint{5.037500in}{3.427208in}}%
\pgfusepath{clip}%
\pgfsetrectcap%
\pgfsetroundjoin%
\pgfsetlinewidth{1.505625pt}%
\definecolor{currentstroke}{rgb}{0.678431,1.000000,0.184314}%
\pgfsetstrokecolor{currentstroke}%
\pgfsetstrokeopacity{0.500000}%
\pgfsetdash{}{0pt}%
\pgfpathmoveto{\pgfqpoint{4.164798in}{3.586837in}}%
\pgfusepath{stroke}%
\end{pgfscope}%
\begin{pgfscope}%
\pgfpathrectangle{\pgfqpoint{0.100000in}{2.413063in}}{\pgfqpoint{5.037500in}{3.427208in}}%
\pgfusepath{clip}%
\pgfsetbuttcap%
\pgfsetroundjoin%
\definecolor{currentfill}{rgb}{0.678431,1.000000,0.184314}%
\pgfsetfillcolor{currentfill}%
\pgfsetfillopacity{0.500000}%
\pgfsetlinewidth{0.250937pt}%
\definecolor{currentstroke}{rgb}{0.000000,0.000000,0.000000}%
\pgfsetstrokecolor{currentstroke}%
\pgfsetstrokeopacity{0.500000}%
\pgfsetdash{}{0pt}%
\pgfsys@defobject{currentmarker}{\pgfqpoint{-0.013889in}{-0.013889in}}{\pgfqpoint{0.013889in}{0.013889in}}{%
\pgfpathmoveto{\pgfqpoint{0.000000in}{-0.013889in}}%
\pgfpathcurveto{\pgfqpoint{0.003683in}{-0.013889in}}{\pgfqpoint{0.007216in}{-0.012425in}}{\pgfqpoint{0.009821in}{-0.009821in}}%
\pgfpathcurveto{\pgfqpoint{0.012425in}{-0.007216in}}{\pgfqpoint{0.013889in}{-0.003683in}}{\pgfqpoint{0.013889in}{0.000000in}}%
\pgfpathcurveto{\pgfqpoint{0.013889in}{0.003683in}}{\pgfqpoint{0.012425in}{0.007216in}}{\pgfqpoint{0.009821in}{0.009821in}}%
\pgfpathcurveto{\pgfqpoint{0.007216in}{0.012425in}}{\pgfqpoint{0.003683in}{0.013889in}}{\pgfqpoint{0.000000in}{0.013889in}}%
\pgfpathcurveto{\pgfqpoint{-0.003683in}{0.013889in}}{\pgfqpoint{-0.007216in}{0.012425in}}{\pgfqpoint{-0.009821in}{0.009821in}}%
\pgfpathcurveto{\pgfqpoint{-0.012425in}{0.007216in}}{\pgfqpoint{-0.013889in}{0.003683in}}{\pgfqpoint{-0.013889in}{0.000000in}}%
\pgfpathcurveto{\pgfqpoint{-0.013889in}{-0.003683in}}{\pgfqpoint{-0.012425in}{-0.007216in}}{\pgfqpoint{-0.009821in}{-0.009821in}}%
\pgfpathcurveto{\pgfqpoint{-0.007216in}{-0.012425in}}{\pgfqpoint{-0.003683in}{-0.013889in}}{\pgfqpoint{0.000000in}{-0.013889in}}%
\pgfpathclose%
\pgfusepath{stroke,fill}%
}%
\begin{pgfscope}%
\pgfsys@transformshift{4.164798in}{3.586837in}%
\pgfsys@useobject{currentmarker}{}%
\end{pgfscope}%
\end{pgfscope}%
\begin{pgfscope}%
\pgfpathrectangle{\pgfqpoint{0.100000in}{2.413063in}}{\pgfqpoint{5.037500in}{3.427208in}}%
\pgfusepath{clip}%
\pgfsetrectcap%
\pgfsetroundjoin%
\pgfsetlinewidth{1.505625pt}%
\definecolor{currentstroke}{rgb}{0.678431,1.000000,0.184314}%
\pgfsetstrokecolor{currentstroke}%
\pgfsetstrokeopacity{0.500000}%
\pgfsetdash{}{0pt}%
\pgfpathmoveto{\pgfqpoint{3.971514in}{3.412687in}}%
\pgfusepath{stroke}%
\end{pgfscope}%
\begin{pgfscope}%
\pgfpathrectangle{\pgfqpoint{0.100000in}{2.413063in}}{\pgfqpoint{5.037500in}{3.427208in}}%
\pgfusepath{clip}%
\pgfsetbuttcap%
\pgfsetroundjoin%
\definecolor{currentfill}{rgb}{0.678431,1.000000,0.184314}%
\pgfsetfillcolor{currentfill}%
\pgfsetfillopacity{0.500000}%
\pgfsetlinewidth{0.250937pt}%
\definecolor{currentstroke}{rgb}{0.000000,0.000000,0.000000}%
\pgfsetstrokecolor{currentstroke}%
\pgfsetstrokeopacity{0.500000}%
\pgfsetdash{}{0pt}%
\pgfsys@defobject{currentmarker}{\pgfqpoint{-0.025000in}{-0.025000in}}{\pgfqpoint{0.025000in}{0.025000in}}{%
\pgfpathmoveto{\pgfqpoint{0.000000in}{-0.025000in}}%
\pgfpathcurveto{\pgfqpoint{0.006630in}{-0.025000in}}{\pgfqpoint{0.012989in}{-0.022366in}}{\pgfqpoint{0.017678in}{-0.017678in}}%
\pgfpathcurveto{\pgfqpoint{0.022366in}{-0.012989in}}{\pgfqpoint{0.025000in}{-0.006630in}}{\pgfqpoint{0.025000in}{0.000000in}}%
\pgfpathcurveto{\pgfqpoint{0.025000in}{0.006630in}}{\pgfqpoint{0.022366in}{0.012989in}}{\pgfqpoint{0.017678in}{0.017678in}}%
\pgfpathcurveto{\pgfqpoint{0.012989in}{0.022366in}}{\pgfqpoint{0.006630in}{0.025000in}}{\pgfqpoint{0.000000in}{0.025000in}}%
\pgfpathcurveto{\pgfqpoint{-0.006630in}{0.025000in}}{\pgfqpoint{-0.012989in}{0.022366in}}{\pgfqpoint{-0.017678in}{0.017678in}}%
\pgfpathcurveto{\pgfqpoint{-0.022366in}{0.012989in}}{\pgfqpoint{-0.025000in}{0.006630in}}{\pgfqpoint{-0.025000in}{0.000000in}}%
\pgfpathcurveto{\pgfqpoint{-0.025000in}{-0.006630in}}{\pgfqpoint{-0.022366in}{-0.012989in}}{\pgfqpoint{-0.017678in}{-0.017678in}}%
\pgfpathcurveto{\pgfqpoint{-0.012989in}{-0.022366in}}{\pgfqpoint{-0.006630in}{-0.025000in}}{\pgfqpoint{0.000000in}{-0.025000in}}%
\pgfpathclose%
\pgfusepath{stroke,fill}%
}%
\begin{pgfscope}%
\pgfsys@transformshift{3.971514in}{3.412687in}%
\pgfsys@useobject{currentmarker}{}%
\end{pgfscope}%
\end{pgfscope}%
\begin{pgfscope}%
\pgfpathrectangle{\pgfqpoint{0.100000in}{2.413063in}}{\pgfqpoint{5.037500in}{3.427208in}}%
\pgfusepath{clip}%
\pgfsetrectcap%
\pgfsetroundjoin%
\pgfsetlinewidth{1.505625pt}%
\definecolor{currentstroke}{rgb}{0.678431,1.000000,0.184314}%
\pgfsetstrokecolor{currentstroke}%
\pgfsetstrokeopacity{0.500000}%
\pgfsetdash{}{0pt}%
\pgfpathmoveto{\pgfqpoint{3.911534in}{3.611641in}}%
\pgfusepath{stroke}%
\end{pgfscope}%
\begin{pgfscope}%
\pgfpathrectangle{\pgfqpoint{0.100000in}{2.413063in}}{\pgfqpoint{5.037500in}{3.427208in}}%
\pgfusepath{clip}%
\pgfsetbuttcap%
\pgfsetroundjoin%
\definecolor{currentfill}{rgb}{0.678431,1.000000,0.184314}%
\pgfsetfillcolor{currentfill}%
\pgfsetfillopacity{0.500000}%
\pgfsetlinewidth{0.250937pt}%
\definecolor{currentstroke}{rgb}{0.000000,0.000000,0.000000}%
\pgfsetstrokecolor{currentstroke}%
\pgfsetstrokeopacity{0.500000}%
\pgfsetdash{}{0pt}%
\pgfsys@defobject{currentmarker}{\pgfqpoint{-0.022222in}{-0.022222in}}{\pgfqpoint{0.022222in}{0.022222in}}{%
\pgfpathmoveto{\pgfqpoint{0.000000in}{-0.022222in}}%
\pgfpathcurveto{\pgfqpoint{0.005893in}{-0.022222in}}{\pgfqpoint{0.011546in}{-0.019881in}}{\pgfqpoint{0.015713in}{-0.015713in}}%
\pgfpathcurveto{\pgfqpoint{0.019881in}{-0.011546in}}{\pgfqpoint{0.022222in}{-0.005893in}}{\pgfqpoint{0.022222in}{0.000000in}}%
\pgfpathcurveto{\pgfqpoint{0.022222in}{0.005893in}}{\pgfqpoint{0.019881in}{0.011546in}}{\pgfqpoint{0.015713in}{0.015713in}}%
\pgfpathcurveto{\pgfqpoint{0.011546in}{0.019881in}}{\pgfqpoint{0.005893in}{0.022222in}}{\pgfqpoint{0.000000in}{0.022222in}}%
\pgfpathcurveto{\pgfqpoint{-0.005893in}{0.022222in}}{\pgfqpoint{-0.011546in}{0.019881in}}{\pgfqpoint{-0.015713in}{0.015713in}}%
\pgfpathcurveto{\pgfqpoint{-0.019881in}{0.011546in}}{\pgfqpoint{-0.022222in}{0.005893in}}{\pgfqpoint{-0.022222in}{0.000000in}}%
\pgfpathcurveto{\pgfqpoint{-0.022222in}{-0.005893in}}{\pgfqpoint{-0.019881in}{-0.011546in}}{\pgfqpoint{-0.015713in}{-0.015713in}}%
\pgfpathcurveto{\pgfqpoint{-0.011546in}{-0.019881in}}{\pgfqpoint{-0.005893in}{-0.022222in}}{\pgfqpoint{0.000000in}{-0.022222in}}%
\pgfpathclose%
\pgfusepath{stroke,fill}%
}%
\begin{pgfscope}%
\pgfsys@transformshift{3.911534in}{3.611641in}%
\pgfsys@useobject{currentmarker}{}%
\end{pgfscope}%
\end{pgfscope}%
\begin{pgfscope}%
\pgfpathrectangle{\pgfqpoint{0.100000in}{2.413063in}}{\pgfqpoint{5.037500in}{3.427208in}}%
\pgfusepath{clip}%
\pgfsetrectcap%
\pgfsetroundjoin%
\pgfsetlinewidth{1.505625pt}%
\definecolor{currentstroke}{rgb}{0.000000,0.000000,1.000000}%
\pgfsetstrokecolor{currentstroke}%
\pgfsetstrokeopacity{0.500000}%
\pgfsetdash{}{0pt}%
\pgfpathmoveto{\pgfqpoint{1.800904in}{2.961134in}}%
\pgfusepath{stroke}%
\end{pgfscope}%
\begin{pgfscope}%
\pgfpathrectangle{\pgfqpoint{0.100000in}{2.413063in}}{\pgfqpoint{5.037500in}{3.427208in}}%
\pgfusepath{clip}%
\pgfsetbuttcap%
\pgfsetroundjoin%
\definecolor{currentfill}{rgb}{0.000000,0.000000,1.000000}%
\pgfsetfillcolor{currentfill}%
\pgfsetfillopacity{0.500000}%
\pgfsetlinewidth{0.250937pt}%
\definecolor{currentstroke}{rgb}{0.000000,0.000000,0.000000}%
\pgfsetstrokecolor{currentstroke}%
\pgfsetstrokeopacity{0.500000}%
\pgfsetdash{}{0pt}%
\pgfsys@defobject{currentmarker}{\pgfqpoint{-0.125000in}{-0.125000in}}{\pgfqpoint{0.125000in}{0.125000in}}{%
\pgfpathmoveto{\pgfqpoint{0.000000in}{-0.125000in}}%
\pgfpathcurveto{\pgfqpoint{0.033150in}{-0.125000in}}{\pgfqpoint{0.064947in}{-0.111829in}}{\pgfqpoint{0.088388in}{-0.088388in}}%
\pgfpathcurveto{\pgfqpoint{0.111829in}{-0.064947in}}{\pgfqpoint{0.125000in}{-0.033150in}}{\pgfqpoint{0.125000in}{0.000000in}}%
\pgfpathcurveto{\pgfqpoint{0.125000in}{0.033150in}}{\pgfqpoint{0.111829in}{0.064947in}}{\pgfqpoint{0.088388in}{0.088388in}}%
\pgfpathcurveto{\pgfqpoint{0.064947in}{0.111829in}}{\pgfqpoint{0.033150in}{0.125000in}}{\pgfqpoint{0.000000in}{0.125000in}}%
\pgfpathcurveto{\pgfqpoint{-0.033150in}{0.125000in}}{\pgfqpoint{-0.064947in}{0.111829in}}{\pgfqpoint{-0.088388in}{0.088388in}}%
\pgfpathcurveto{\pgfqpoint{-0.111829in}{0.064947in}}{\pgfqpoint{-0.125000in}{0.033150in}}{\pgfqpoint{-0.125000in}{0.000000in}}%
\pgfpathcurveto{\pgfqpoint{-0.125000in}{-0.033150in}}{\pgfqpoint{-0.111829in}{-0.064947in}}{\pgfqpoint{-0.088388in}{-0.088388in}}%
\pgfpathcurveto{\pgfqpoint{-0.064947in}{-0.111829in}}{\pgfqpoint{-0.033150in}{-0.125000in}}{\pgfqpoint{0.000000in}{-0.125000in}}%
\pgfpathclose%
\pgfusepath{stroke,fill}%
}%
\begin{pgfscope}%
\pgfsys@transformshift{1.800904in}{2.961134in}%
\pgfsys@useobject{currentmarker}{}%
\end{pgfscope}%
\end{pgfscope}%
\begin{pgfscope}%
\pgfpathrectangle{\pgfqpoint{0.100000in}{2.413063in}}{\pgfqpoint{5.037500in}{3.427208in}}%
\pgfusepath{clip}%
\pgfsetrectcap%
\pgfsetroundjoin%
\pgfsetlinewidth{1.505625pt}%
\definecolor{currentstroke}{rgb}{0.000000,0.000000,1.000000}%
\pgfsetstrokecolor{currentstroke}%
\pgfsetstrokeopacity{0.500000}%
\pgfsetdash{}{0pt}%
\pgfpathmoveto{\pgfqpoint{1.714465in}{3.099522in}}%
\pgfusepath{stroke}%
\end{pgfscope}%
\begin{pgfscope}%
\pgfpathrectangle{\pgfqpoint{0.100000in}{2.413063in}}{\pgfqpoint{5.037500in}{3.427208in}}%
\pgfusepath{clip}%
\pgfsetbuttcap%
\pgfsetroundjoin%
\definecolor{currentfill}{rgb}{0.000000,0.000000,1.000000}%
\pgfsetfillcolor{currentfill}%
\pgfsetfillopacity{0.500000}%
\pgfsetlinewidth{0.250937pt}%
\definecolor{currentstroke}{rgb}{0.000000,0.000000,0.000000}%
\pgfsetstrokecolor{currentstroke}%
\pgfsetstrokeopacity{0.500000}%
\pgfsetdash{}{0pt}%
\pgfsys@defobject{currentmarker}{\pgfqpoint{-0.077778in}{-0.077778in}}{\pgfqpoint{0.077778in}{0.077778in}}{%
\pgfpathmoveto{\pgfqpoint{0.000000in}{-0.077778in}}%
\pgfpathcurveto{\pgfqpoint{0.020627in}{-0.077778in}}{\pgfqpoint{0.040412in}{-0.069583in}}{\pgfqpoint{0.054997in}{-0.054997in}}%
\pgfpathcurveto{\pgfqpoint{0.069583in}{-0.040412in}}{\pgfqpoint{0.077778in}{-0.020627in}}{\pgfqpoint{0.077778in}{0.000000in}}%
\pgfpathcurveto{\pgfqpoint{0.077778in}{0.020627in}}{\pgfqpoint{0.069583in}{0.040412in}}{\pgfqpoint{0.054997in}{0.054997in}}%
\pgfpathcurveto{\pgfqpoint{0.040412in}{0.069583in}}{\pgfqpoint{0.020627in}{0.077778in}}{\pgfqpoint{0.000000in}{0.077778in}}%
\pgfpathcurveto{\pgfqpoint{-0.020627in}{0.077778in}}{\pgfqpoint{-0.040412in}{0.069583in}}{\pgfqpoint{-0.054997in}{0.054997in}}%
\pgfpathcurveto{\pgfqpoint{-0.069583in}{0.040412in}}{\pgfqpoint{-0.077778in}{0.020627in}}{\pgfqpoint{-0.077778in}{0.000000in}}%
\pgfpathcurveto{\pgfqpoint{-0.077778in}{-0.020627in}}{\pgfqpoint{-0.069583in}{-0.040412in}}{\pgfqpoint{-0.054997in}{-0.054997in}}%
\pgfpathcurveto{\pgfqpoint{-0.040412in}{-0.069583in}}{\pgfqpoint{-0.020627in}{-0.077778in}}{\pgfqpoint{0.000000in}{-0.077778in}}%
\pgfpathclose%
\pgfusepath{stroke,fill}%
}%
\begin{pgfscope}%
\pgfsys@transformshift{1.714465in}{3.099522in}%
\pgfsys@useobject{currentmarker}{}%
\end{pgfscope}%
\end{pgfscope}%
\begin{pgfscope}%
\pgfpathrectangle{\pgfqpoint{0.100000in}{2.413063in}}{\pgfqpoint{5.037500in}{3.427208in}}%
\pgfusepath{clip}%
\pgfsetrectcap%
\pgfsetroundjoin%
\pgfsetlinewidth{1.505625pt}%
\definecolor{currentstroke}{rgb}{0.678431,1.000000,0.184314}%
\pgfsetstrokecolor{currentstroke}%
\pgfsetstrokeopacity{0.500000}%
\pgfsetdash{}{0pt}%
\pgfpathmoveto{\pgfqpoint{1.043203in}{5.017798in}}%
\pgfusepath{stroke}%
\end{pgfscope}%
\begin{pgfscope}%
\pgfpathrectangle{\pgfqpoint{0.100000in}{2.413063in}}{\pgfqpoint{5.037500in}{3.427208in}}%
\pgfusepath{clip}%
\pgfsetbuttcap%
\pgfsetroundjoin%
\definecolor{currentfill}{rgb}{0.678431,1.000000,0.184314}%
\pgfsetfillcolor{currentfill}%
\pgfsetfillopacity{0.500000}%
\pgfsetlinewidth{0.250937pt}%
\definecolor{currentstroke}{rgb}{0.000000,0.000000,0.000000}%
\pgfsetstrokecolor{currentstroke}%
\pgfsetstrokeopacity{0.500000}%
\pgfsetdash{}{0pt}%
\pgfsys@defobject{currentmarker}{\pgfqpoint{-0.013889in}{-0.013889in}}{\pgfqpoint{0.013889in}{0.013889in}}{%
\pgfpathmoveto{\pgfqpoint{0.000000in}{-0.013889in}}%
\pgfpathcurveto{\pgfqpoint{0.003683in}{-0.013889in}}{\pgfqpoint{0.007216in}{-0.012425in}}{\pgfqpoint{0.009821in}{-0.009821in}}%
\pgfpathcurveto{\pgfqpoint{0.012425in}{-0.007216in}}{\pgfqpoint{0.013889in}{-0.003683in}}{\pgfqpoint{0.013889in}{0.000000in}}%
\pgfpathcurveto{\pgfqpoint{0.013889in}{0.003683in}}{\pgfqpoint{0.012425in}{0.007216in}}{\pgfqpoint{0.009821in}{0.009821in}}%
\pgfpathcurveto{\pgfqpoint{0.007216in}{0.012425in}}{\pgfqpoint{0.003683in}{0.013889in}}{\pgfqpoint{0.000000in}{0.013889in}}%
\pgfpathcurveto{\pgfqpoint{-0.003683in}{0.013889in}}{\pgfqpoint{-0.007216in}{0.012425in}}{\pgfqpoint{-0.009821in}{0.009821in}}%
\pgfpathcurveto{\pgfqpoint{-0.012425in}{0.007216in}}{\pgfqpoint{-0.013889in}{0.003683in}}{\pgfqpoint{-0.013889in}{0.000000in}}%
\pgfpathcurveto{\pgfqpoint{-0.013889in}{-0.003683in}}{\pgfqpoint{-0.012425in}{-0.007216in}}{\pgfqpoint{-0.009821in}{-0.009821in}}%
\pgfpathcurveto{\pgfqpoint{-0.007216in}{-0.012425in}}{\pgfqpoint{-0.003683in}{-0.013889in}}{\pgfqpoint{0.000000in}{-0.013889in}}%
\pgfpathclose%
\pgfusepath{stroke,fill}%
}%
\begin{pgfscope}%
\pgfsys@transformshift{1.043203in}{5.017798in}%
\pgfsys@useobject{currentmarker}{}%
\end{pgfscope}%
\end{pgfscope}%
\begin{pgfscope}%
\pgfpathrectangle{\pgfqpoint{0.100000in}{2.413063in}}{\pgfqpoint{5.037500in}{3.427208in}}%
\pgfusepath{clip}%
\pgfsetrectcap%
\pgfsetroundjoin%
\pgfsetlinewidth{1.505625pt}%
\definecolor{currentstroke}{rgb}{0.678431,1.000000,0.184314}%
\pgfsetstrokecolor{currentstroke}%
\pgfsetstrokeopacity{0.500000}%
\pgfsetdash{}{0pt}%
\pgfpathmoveto{\pgfqpoint{1.111792in}{5.487791in}}%
\pgfusepath{stroke}%
\end{pgfscope}%
\begin{pgfscope}%
\pgfpathrectangle{\pgfqpoint{0.100000in}{2.413063in}}{\pgfqpoint{5.037500in}{3.427208in}}%
\pgfusepath{clip}%
\pgfsetbuttcap%
\pgfsetroundjoin%
\definecolor{currentfill}{rgb}{0.678431,1.000000,0.184314}%
\pgfsetfillcolor{currentfill}%
\pgfsetfillopacity{0.500000}%
\pgfsetlinewidth{0.250937pt}%
\definecolor{currentstroke}{rgb}{0.000000,0.000000,0.000000}%
\pgfsetstrokecolor{currentstroke}%
\pgfsetstrokeopacity{0.500000}%
\pgfsetdash{}{0pt}%
\pgfsys@defobject{currentmarker}{\pgfqpoint{-0.030556in}{-0.030556in}}{\pgfqpoint{0.030556in}{0.030556in}}{%
\pgfpathmoveto{\pgfqpoint{0.000000in}{-0.030556in}}%
\pgfpathcurveto{\pgfqpoint{0.008103in}{-0.030556in}}{\pgfqpoint{0.015876in}{-0.027336in}}{\pgfqpoint{0.021606in}{-0.021606in}}%
\pgfpathcurveto{\pgfqpoint{0.027336in}{-0.015876in}}{\pgfqpoint{0.030556in}{-0.008103in}}{\pgfqpoint{0.030556in}{0.000000in}}%
\pgfpathcurveto{\pgfqpoint{0.030556in}{0.008103in}}{\pgfqpoint{0.027336in}{0.015876in}}{\pgfqpoint{0.021606in}{0.021606in}}%
\pgfpathcurveto{\pgfqpoint{0.015876in}{0.027336in}}{\pgfqpoint{0.008103in}{0.030556in}}{\pgfqpoint{0.000000in}{0.030556in}}%
\pgfpathcurveto{\pgfqpoint{-0.008103in}{0.030556in}}{\pgfqpoint{-0.015876in}{0.027336in}}{\pgfqpoint{-0.021606in}{0.021606in}}%
\pgfpathcurveto{\pgfqpoint{-0.027336in}{0.015876in}}{\pgfqpoint{-0.030556in}{0.008103in}}{\pgfqpoint{-0.030556in}{0.000000in}}%
\pgfpathcurveto{\pgfqpoint{-0.030556in}{-0.008103in}}{\pgfqpoint{-0.027336in}{-0.015876in}}{\pgfqpoint{-0.021606in}{-0.021606in}}%
\pgfpathcurveto{\pgfqpoint{-0.015876in}{-0.027336in}}{\pgfqpoint{-0.008103in}{-0.030556in}}{\pgfqpoint{0.000000in}{-0.030556in}}%
\pgfpathclose%
\pgfusepath{stroke,fill}%
}%
\begin{pgfscope}%
\pgfsys@transformshift{1.111792in}{5.487791in}%
\pgfsys@useobject{currentmarker}{}%
\end{pgfscope}%
\end{pgfscope}%
\begin{pgfscope}%
\pgfpathrectangle{\pgfqpoint{0.100000in}{2.413063in}}{\pgfqpoint{5.037500in}{3.427208in}}%
\pgfusepath{clip}%
\pgfsetrectcap%
\pgfsetroundjoin%
\pgfsetlinewidth{1.505625pt}%
\definecolor{currentstroke}{rgb}{0.678431,1.000000,0.184314}%
\pgfsetstrokecolor{currentstroke}%
\pgfsetstrokeopacity{0.500000}%
\pgfsetdash{}{0pt}%
\pgfpathmoveto{\pgfqpoint{1.384741in}{4.932340in}}%
\pgfusepath{stroke}%
\end{pgfscope}%
\begin{pgfscope}%
\pgfpathrectangle{\pgfqpoint{0.100000in}{2.413063in}}{\pgfqpoint{5.037500in}{3.427208in}}%
\pgfusepath{clip}%
\pgfsetbuttcap%
\pgfsetroundjoin%
\definecolor{currentfill}{rgb}{0.678431,1.000000,0.184314}%
\pgfsetfillcolor{currentfill}%
\pgfsetfillopacity{0.500000}%
\pgfsetlinewidth{0.250937pt}%
\definecolor{currentstroke}{rgb}{0.000000,0.000000,0.000000}%
\pgfsetstrokecolor{currentstroke}%
\pgfsetstrokeopacity{0.500000}%
\pgfsetdash{}{0pt}%
\pgfsys@defobject{currentmarker}{\pgfqpoint{-0.013889in}{-0.013889in}}{\pgfqpoint{0.013889in}{0.013889in}}{%
\pgfpathmoveto{\pgfqpoint{0.000000in}{-0.013889in}}%
\pgfpathcurveto{\pgfqpoint{0.003683in}{-0.013889in}}{\pgfqpoint{0.007216in}{-0.012425in}}{\pgfqpoint{0.009821in}{-0.009821in}}%
\pgfpathcurveto{\pgfqpoint{0.012425in}{-0.007216in}}{\pgfqpoint{0.013889in}{-0.003683in}}{\pgfqpoint{0.013889in}{0.000000in}}%
\pgfpathcurveto{\pgfqpoint{0.013889in}{0.003683in}}{\pgfqpoint{0.012425in}{0.007216in}}{\pgfqpoint{0.009821in}{0.009821in}}%
\pgfpathcurveto{\pgfqpoint{0.007216in}{0.012425in}}{\pgfqpoint{0.003683in}{0.013889in}}{\pgfqpoint{0.000000in}{0.013889in}}%
\pgfpathcurveto{\pgfqpoint{-0.003683in}{0.013889in}}{\pgfqpoint{-0.007216in}{0.012425in}}{\pgfqpoint{-0.009821in}{0.009821in}}%
\pgfpathcurveto{\pgfqpoint{-0.012425in}{0.007216in}}{\pgfqpoint{-0.013889in}{0.003683in}}{\pgfqpoint{-0.013889in}{0.000000in}}%
\pgfpathcurveto{\pgfqpoint{-0.013889in}{-0.003683in}}{\pgfqpoint{-0.012425in}{-0.007216in}}{\pgfqpoint{-0.009821in}{-0.009821in}}%
\pgfpathcurveto{\pgfqpoint{-0.007216in}{-0.012425in}}{\pgfqpoint{-0.003683in}{-0.013889in}}{\pgfqpoint{0.000000in}{-0.013889in}}%
\pgfpathclose%
\pgfusepath{stroke,fill}%
}%
\begin{pgfscope}%
\pgfsys@transformshift{1.384741in}{4.932340in}%
\pgfsys@useobject{currentmarker}{}%
\end{pgfscope}%
\end{pgfscope}%
\begin{pgfscope}%
\pgfpathrectangle{\pgfqpoint{0.100000in}{2.413063in}}{\pgfqpoint{5.037500in}{3.427208in}}%
\pgfusepath{clip}%
\pgfsetrectcap%
\pgfsetroundjoin%
\pgfsetlinewidth{1.505625pt}%
\definecolor{currentstroke}{rgb}{0.678431,1.000000,0.184314}%
\pgfsetstrokecolor{currentstroke}%
\pgfsetstrokeopacity{0.500000}%
\pgfsetdash{}{0pt}%
\pgfpathmoveto{\pgfqpoint{1.058529in}{5.349824in}}%
\pgfusepath{stroke}%
\end{pgfscope}%
\begin{pgfscope}%
\pgfpathrectangle{\pgfqpoint{0.100000in}{2.413063in}}{\pgfqpoint{5.037500in}{3.427208in}}%
\pgfusepath{clip}%
\pgfsetbuttcap%
\pgfsetroundjoin%
\definecolor{currentfill}{rgb}{0.678431,1.000000,0.184314}%
\pgfsetfillcolor{currentfill}%
\pgfsetfillopacity{0.500000}%
\pgfsetlinewidth{0.250937pt}%
\definecolor{currentstroke}{rgb}{0.000000,0.000000,0.000000}%
\pgfsetstrokecolor{currentstroke}%
\pgfsetstrokeopacity{0.500000}%
\pgfsetdash{}{0pt}%
\pgfsys@defobject{currentmarker}{\pgfqpoint{-0.022222in}{-0.022222in}}{\pgfqpoint{0.022222in}{0.022222in}}{%
\pgfpathmoveto{\pgfqpoint{0.000000in}{-0.022222in}}%
\pgfpathcurveto{\pgfqpoint{0.005893in}{-0.022222in}}{\pgfqpoint{0.011546in}{-0.019881in}}{\pgfqpoint{0.015713in}{-0.015713in}}%
\pgfpathcurveto{\pgfqpoint{0.019881in}{-0.011546in}}{\pgfqpoint{0.022222in}{-0.005893in}}{\pgfqpoint{0.022222in}{0.000000in}}%
\pgfpathcurveto{\pgfqpoint{0.022222in}{0.005893in}}{\pgfqpoint{0.019881in}{0.011546in}}{\pgfqpoint{0.015713in}{0.015713in}}%
\pgfpathcurveto{\pgfqpoint{0.011546in}{0.019881in}}{\pgfqpoint{0.005893in}{0.022222in}}{\pgfqpoint{0.000000in}{0.022222in}}%
\pgfpathcurveto{\pgfqpoint{-0.005893in}{0.022222in}}{\pgfqpoint{-0.011546in}{0.019881in}}{\pgfqpoint{-0.015713in}{0.015713in}}%
\pgfpathcurveto{\pgfqpoint{-0.019881in}{0.011546in}}{\pgfqpoint{-0.022222in}{0.005893in}}{\pgfqpoint{-0.022222in}{0.000000in}}%
\pgfpathcurveto{\pgfqpoint{-0.022222in}{-0.005893in}}{\pgfqpoint{-0.019881in}{-0.011546in}}{\pgfqpoint{-0.015713in}{-0.015713in}}%
\pgfpathcurveto{\pgfqpoint{-0.011546in}{-0.019881in}}{\pgfqpoint{-0.005893in}{-0.022222in}}{\pgfqpoint{0.000000in}{-0.022222in}}%
\pgfpathclose%
\pgfusepath{stroke,fill}%
}%
\begin{pgfscope}%
\pgfsys@transformshift{1.058529in}{5.349824in}%
\pgfsys@useobject{currentmarker}{}%
\end{pgfscope}%
\end{pgfscope}%
\begin{pgfscope}%
\pgfpathrectangle{\pgfqpoint{0.100000in}{2.413063in}}{\pgfqpoint{5.037500in}{3.427208in}}%
\pgfusepath{clip}%
\pgfsetrectcap%
\pgfsetroundjoin%
\pgfsetlinewidth{1.505625pt}%
\definecolor{currentstroke}{rgb}{0.678431,1.000000,0.184314}%
\pgfsetstrokecolor{currentstroke}%
\pgfsetstrokeopacity{0.500000}%
\pgfsetdash{}{0pt}%
\pgfpathmoveto{\pgfqpoint{1.338071in}{4.867867in}}%
\pgfusepath{stroke}%
\end{pgfscope}%
\begin{pgfscope}%
\pgfpathrectangle{\pgfqpoint{0.100000in}{2.413063in}}{\pgfqpoint{5.037500in}{3.427208in}}%
\pgfusepath{clip}%
\pgfsetbuttcap%
\pgfsetroundjoin%
\definecolor{currentfill}{rgb}{0.678431,1.000000,0.184314}%
\pgfsetfillcolor{currentfill}%
\pgfsetfillopacity{0.500000}%
\pgfsetlinewidth{0.250937pt}%
\definecolor{currentstroke}{rgb}{0.000000,0.000000,0.000000}%
\pgfsetstrokecolor{currentstroke}%
\pgfsetstrokeopacity{0.500000}%
\pgfsetdash{}{0pt}%
\pgfsys@defobject{currentmarker}{\pgfqpoint{-0.019444in}{-0.019444in}}{\pgfqpoint{0.019444in}{0.019444in}}{%
\pgfpathmoveto{\pgfqpoint{0.000000in}{-0.019444in}}%
\pgfpathcurveto{\pgfqpoint{0.005157in}{-0.019444in}}{\pgfqpoint{0.010103in}{-0.017396in}}{\pgfqpoint{0.013749in}{-0.013749in}}%
\pgfpathcurveto{\pgfqpoint{0.017396in}{-0.010103in}}{\pgfqpoint{0.019444in}{-0.005157in}}{\pgfqpoint{0.019444in}{0.000000in}}%
\pgfpathcurveto{\pgfqpoint{0.019444in}{0.005157in}}{\pgfqpoint{0.017396in}{0.010103in}}{\pgfqpoint{0.013749in}{0.013749in}}%
\pgfpathcurveto{\pgfqpoint{0.010103in}{0.017396in}}{\pgfqpoint{0.005157in}{0.019444in}}{\pgfqpoint{0.000000in}{0.019444in}}%
\pgfpathcurveto{\pgfqpoint{-0.005157in}{0.019444in}}{\pgfqpoint{-0.010103in}{0.017396in}}{\pgfqpoint{-0.013749in}{0.013749in}}%
\pgfpathcurveto{\pgfqpoint{-0.017396in}{0.010103in}}{\pgfqpoint{-0.019444in}{0.005157in}}{\pgfqpoint{-0.019444in}{0.000000in}}%
\pgfpathcurveto{\pgfqpoint{-0.019444in}{-0.005157in}}{\pgfqpoint{-0.017396in}{-0.010103in}}{\pgfqpoint{-0.013749in}{-0.013749in}}%
\pgfpathcurveto{\pgfqpoint{-0.010103in}{-0.017396in}}{\pgfqpoint{-0.005157in}{-0.019444in}}{\pgfqpoint{0.000000in}{-0.019444in}}%
\pgfpathclose%
\pgfusepath{stroke,fill}%
}%
\begin{pgfscope}%
\pgfsys@transformshift{1.338071in}{4.867867in}%
\pgfsys@useobject{currentmarker}{}%
\end{pgfscope}%
\end{pgfscope}%
\begin{pgfscope}%
\pgfpathrectangle{\pgfqpoint{0.100000in}{2.413063in}}{\pgfqpoint{5.037500in}{3.427208in}}%
\pgfusepath{clip}%
\pgfsetrectcap%
\pgfsetroundjoin%
\pgfsetlinewidth{1.505625pt}%
\definecolor{currentstroke}{rgb}{0.000000,0.000000,1.000000}%
\pgfsetstrokecolor{currentstroke}%
\pgfsetstrokeopacity{0.500000}%
\pgfsetdash{}{0pt}%
\pgfpathmoveto{\pgfqpoint{3.329976in}{4.467971in}}%
\pgfusepath{stroke}%
\end{pgfscope}%
\begin{pgfscope}%
\pgfpathrectangle{\pgfqpoint{0.100000in}{2.413063in}}{\pgfqpoint{5.037500in}{3.427208in}}%
\pgfusepath{clip}%
\pgfsetbuttcap%
\pgfsetroundjoin%
\definecolor{currentfill}{rgb}{0.000000,0.000000,1.000000}%
\pgfsetfillcolor{currentfill}%
\pgfsetfillopacity{0.500000}%
\pgfsetlinewidth{0.250937pt}%
\definecolor{currentstroke}{rgb}{0.000000,0.000000,0.000000}%
\pgfsetstrokecolor{currentstroke}%
\pgfsetstrokeopacity{0.500000}%
\pgfsetdash{}{0pt}%
\pgfsys@defobject{currentmarker}{\pgfqpoint{-0.011111in}{-0.011111in}}{\pgfqpoint{0.011111in}{0.011111in}}{%
\pgfpathmoveto{\pgfqpoint{0.000000in}{-0.011111in}}%
\pgfpathcurveto{\pgfqpoint{0.002947in}{-0.011111in}}{\pgfqpoint{0.005773in}{-0.009940in}}{\pgfqpoint{0.007857in}{-0.007857in}}%
\pgfpathcurveto{\pgfqpoint{0.009940in}{-0.005773in}}{\pgfqpoint{0.011111in}{-0.002947in}}{\pgfqpoint{0.011111in}{0.000000in}}%
\pgfpathcurveto{\pgfqpoint{0.011111in}{0.002947in}}{\pgfqpoint{0.009940in}{0.005773in}}{\pgfqpoint{0.007857in}{0.007857in}}%
\pgfpathcurveto{\pgfqpoint{0.005773in}{0.009940in}}{\pgfqpoint{0.002947in}{0.011111in}}{\pgfqpoint{0.000000in}{0.011111in}}%
\pgfpathcurveto{\pgfqpoint{-0.002947in}{0.011111in}}{\pgfqpoint{-0.005773in}{0.009940in}}{\pgfqpoint{-0.007857in}{0.007857in}}%
\pgfpathcurveto{\pgfqpoint{-0.009940in}{0.005773in}}{\pgfqpoint{-0.011111in}{0.002947in}}{\pgfqpoint{-0.011111in}{0.000000in}}%
\pgfpathcurveto{\pgfqpoint{-0.011111in}{-0.002947in}}{\pgfqpoint{-0.009940in}{-0.005773in}}{\pgfqpoint{-0.007857in}{-0.007857in}}%
\pgfpathcurveto{\pgfqpoint{-0.005773in}{-0.009940in}}{\pgfqpoint{-0.002947in}{-0.011111in}}{\pgfqpoint{0.000000in}{-0.011111in}}%
\pgfpathclose%
\pgfusepath{stroke,fill}%
}%
\begin{pgfscope}%
\pgfsys@transformshift{3.329976in}{4.467971in}%
\pgfsys@useobject{currentmarker}{}%
\end{pgfscope}%
\end{pgfscope}%
\begin{pgfscope}%
\pgfpathrectangle{\pgfqpoint{0.100000in}{2.413063in}}{\pgfqpoint{5.037500in}{3.427208in}}%
\pgfusepath{clip}%
\pgfsetrectcap%
\pgfsetroundjoin%
\pgfsetlinewidth{1.505625pt}%
\definecolor{currentstroke}{rgb}{0.000000,0.000000,1.000000}%
\pgfsetstrokecolor{currentstroke}%
\pgfsetstrokeopacity{0.500000}%
\pgfsetdash{}{0pt}%
\pgfpathmoveto{\pgfqpoint{3.330628in}{4.151165in}}%
\pgfusepath{stroke}%
\end{pgfscope}%
\begin{pgfscope}%
\pgfpathrectangle{\pgfqpoint{0.100000in}{2.413063in}}{\pgfqpoint{5.037500in}{3.427208in}}%
\pgfusepath{clip}%
\pgfsetbuttcap%
\pgfsetroundjoin%
\definecolor{currentfill}{rgb}{0.000000,0.000000,1.000000}%
\pgfsetfillcolor{currentfill}%
\pgfsetfillopacity{0.500000}%
\pgfsetlinewidth{0.250937pt}%
\definecolor{currentstroke}{rgb}{0.000000,0.000000,0.000000}%
\pgfsetstrokecolor{currentstroke}%
\pgfsetstrokeopacity{0.500000}%
\pgfsetdash{}{0pt}%
\pgfsys@defobject{currentmarker}{\pgfqpoint{-0.016667in}{-0.016667in}}{\pgfqpoint{0.016667in}{0.016667in}}{%
\pgfpathmoveto{\pgfqpoint{0.000000in}{-0.016667in}}%
\pgfpathcurveto{\pgfqpoint{0.004420in}{-0.016667in}}{\pgfqpoint{0.008660in}{-0.014911in}}{\pgfqpoint{0.011785in}{-0.011785in}}%
\pgfpathcurveto{\pgfqpoint{0.014911in}{-0.008660in}}{\pgfqpoint{0.016667in}{-0.004420in}}{\pgfqpoint{0.016667in}{0.000000in}}%
\pgfpathcurveto{\pgfqpoint{0.016667in}{0.004420in}}{\pgfqpoint{0.014911in}{0.008660in}}{\pgfqpoint{0.011785in}{0.011785in}}%
\pgfpathcurveto{\pgfqpoint{0.008660in}{0.014911in}}{\pgfqpoint{0.004420in}{0.016667in}}{\pgfqpoint{0.000000in}{0.016667in}}%
\pgfpathcurveto{\pgfqpoint{-0.004420in}{0.016667in}}{\pgfqpoint{-0.008660in}{0.014911in}}{\pgfqpoint{-0.011785in}{0.011785in}}%
\pgfpathcurveto{\pgfqpoint{-0.014911in}{0.008660in}}{\pgfqpoint{-0.016667in}{0.004420in}}{\pgfqpoint{-0.016667in}{0.000000in}}%
\pgfpathcurveto{\pgfqpoint{-0.016667in}{-0.004420in}}{\pgfqpoint{-0.014911in}{-0.008660in}}{\pgfqpoint{-0.011785in}{-0.011785in}}%
\pgfpathcurveto{\pgfqpoint{-0.008660in}{-0.014911in}}{\pgfqpoint{-0.004420in}{-0.016667in}}{\pgfqpoint{0.000000in}{-0.016667in}}%
\pgfpathclose%
\pgfusepath{stroke,fill}%
}%
\begin{pgfscope}%
\pgfsys@transformshift{3.330628in}{4.151165in}%
\pgfsys@useobject{currentmarker}{}%
\end{pgfscope}%
\end{pgfscope}%
\begin{pgfscope}%
\pgfpathrectangle{\pgfqpoint{0.100000in}{2.413063in}}{\pgfqpoint{5.037500in}{3.427208in}}%
\pgfusepath{clip}%
\pgfsetrectcap%
\pgfsetroundjoin%
\pgfsetlinewidth{1.505625pt}%
\definecolor{currentstroke}{rgb}{0.000000,0.000000,1.000000}%
\pgfsetstrokecolor{currentstroke}%
\pgfsetstrokeopacity{0.500000}%
\pgfsetdash{}{0pt}%
\pgfpathmoveto{\pgfqpoint{3.398646in}{4.431619in}}%
\pgfusepath{stroke}%
\end{pgfscope}%
\begin{pgfscope}%
\pgfpathrectangle{\pgfqpoint{0.100000in}{2.413063in}}{\pgfqpoint{5.037500in}{3.427208in}}%
\pgfusepath{clip}%
\pgfsetbuttcap%
\pgfsetroundjoin%
\definecolor{currentfill}{rgb}{0.000000,0.000000,1.000000}%
\pgfsetfillcolor{currentfill}%
\pgfsetfillopacity{0.500000}%
\pgfsetlinewidth{0.250937pt}%
\definecolor{currentstroke}{rgb}{0.000000,0.000000,0.000000}%
\pgfsetstrokecolor{currentstroke}%
\pgfsetstrokeopacity{0.500000}%
\pgfsetdash{}{0pt}%
\pgfsys@defobject{currentmarker}{\pgfqpoint{-0.011111in}{-0.011111in}}{\pgfqpoint{0.011111in}{0.011111in}}{%
\pgfpathmoveto{\pgfqpoint{0.000000in}{-0.011111in}}%
\pgfpathcurveto{\pgfqpoint{0.002947in}{-0.011111in}}{\pgfqpoint{0.005773in}{-0.009940in}}{\pgfqpoint{0.007857in}{-0.007857in}}%
\pgfpathcurveto{\pgfqpoint{0.009940in}{-0.005773in}}{\pgfqpoint{0.011111in}{-0.002947in}}{\pgfqpoint{0.011111in}{0.000000in}}%
\pgfpathcurveto{\pgfqpoint{0.011111in}{0.002947in}}{\pgfqpoint{0.009940in}{0.005773in}}{\pgfqpoint{0.007857in}{0.007857in}}%
\pgfpathcurveto{\pgfqpoint{0.005773in}{0.009940in}}{\pgfqpoint{0.002947in}{0.011111in}}{\pgfqpoint{0.000000in}{0.011111in}}%
\pgfpathcurveto{\pgfqpoint{-0.002947in}{0.011111in}}{\pgfqpoint{-0.005773in}{0.009940in}}{\pgfqpoint{-0.007857in}{0.007857in}}%
\pgfpathcurveto{\pgfqpoint{-0.009940in}{0.005773in}}{\pgfqpoint{-0.011111in}{0.002947in}}{\pgfqpoint{-0.011111in}{0.000000in}}%
\pgfpathcurveto{\pgfqpoint{-0.011111in}{-0.002947in}}{\pgfqpoint{-0.009940in}{-0.005773in}}{\pgfqpoint{-0.007857in}{-0.007857in}}%
\pgfpathcurveto{\pgfqpoint{-0.005773in}{-0.009940in}}{\pgfqpoint{-0.002947in}{-0.011111in}}{\pgfqpoint{0.000000in}{-0.011111in}}%
\pgfpathclose%
\pgfusepath{stroke,fill}%
}%
\begin{pgfscope}%
\pgfsys@transformshift{3.398646in}{4.431619in}%
\pgfsys@useobject{currentmarker}{}%
\end{pgfscope}%
\end{pgfscope}%
\begin{pgfscope}%
\pgfpathrectangle{\pgfqpoint{0.100000in}{2.413063in}}{\pgfqpoint{5.037500in}{3.427208in}}%
\pgfusepath{clip}%
\pgfsetrectcap%
\pgfsetroundjoin%
\pgfsetlinewidth{1.505625pt}%
\definecolor{currentstroke}{rgb}{0.000000,0.000000,1.000000}%
\pgfsetstrokecolor{currentstroke}%
\pgfsetstrokeopacity{0.500000}%
\pgfsetdash{}{0pt}%
\pgfpathmoveto{\pgfqpoint{3.436563in}{4.637868in}}%
\pgfusepath{stroke}%
\end{pgfscope}%
\begin{pgfscope}%
\pgfpathrectangle{\pgfqpoint{0.100000in}{2.413063in}}{\pgfqpoint{5.037500in}{3.427208in}}%
\pgfusepath{clip}%
\pgfsetbuttcap%
\pgfsetroundjoin%
\definecolor{currentfill}{rgb}{0.000000,0.000000,1.000000}%
\pgfsetfillcolor{currentfill}%
\pgfsetfillopacity{0.500000}%
\pgfsetlinewidth{0.250937pt}%
\definecolor{currentstroke}{rgb}{0.000000,0.000000,0.000000}%
\pgfsetstrokecolor{currentstroke}%
\pgfsetstrokeopacity{0.500000}%
\pgfsetdash{}{0pt}%
\pgfsys@defobject{currentmarker}{\pgfqpoint{-0.030556in}{-0.030556in}}{\pgfqpoint{0.030556in}{0.030556in}}{%
\pgfpathmoveto{\pgfqpoint{0.000000in}{-0.030556in}}%
\pgfpathcurveto{\pgfqpoint{0.008103in}{-0.030556in}}{\pgfqpoint{0.015876in}{-0.027336in}}{\pgfqpoint{0.021606in}{-0.021606in}}%
\pgfpathcurveto{\pgfqpoint{0.027336in}{-0.015876in}}{\pgfqpoint{0.030556in}{-0.008103in}}{\pgfqpoint{0.030556in}{0.000000in}}%
\pgfpathcurveto{\pgfqpoint{0.030556in}{0.008103in}}{\pgfqpoint{0.027336in}{0.015876in}}{\pgfqpoint{0.021606in}{0.021606in}}%
\pgfpathcurveto{\pgfqpoint{0.015876in}{0.027336in}}{\pgfqpoint{0.008103in}{0.030556in}}{\pgfqpoint{0.000000in}{0.030556in}}%
\pgfpathcurveto{\pgfqpoint{-0.008103in}{0.030556in}}{\pgfqpoint{-0.015876in}{0.027336in}}{\pgfqpoint{-0.021606in}{0.021606in}}%
\pgfpathcurveto{\pgfqpoint{-0.027336in}{0.015876in}}{\pgfqpoint{-0.030556in}{0.008103in}}{\pgfqpoint{-0.030556in}{0.000000in}}%
\pgfpathcurveto{\pgfqpoint{-0.030556in}{-0.008103in}}{\pgfqpoint{-0.027336in}{-0.015876in}}{\pgfqpoint{-0.021606in}{-0.021606in}}%
\pgfpathcurveto{\pgfqpoint{-0.015876in}{-0.027336in}}{\pgfqpoint{-0.008103in}{-0.030556in}}{\pgfqpoint{0.000000in}{-0.030556in}}%
\pgfpathclose%
\pgfusepath{stroke,fill}%
}%
\begin{pgfscope}%
\pgfsys@transformshift{3.436563in}{4.637868in}%
\pgfsys@useobject{currentmarker}{}%
\end{pgfscope}%
\end{pgfscope}%
\begin{pgfscope}%
\pgfpathrectangle{\pgfqpoint{0.100000in}{2.413063in}}{\pgfqpoint{5.037500in}{3.427208in}}%
\pgfusepath{clip}%
\pgfsetrectcap%
\pgfsetroundjoin%
\pgfsetlinewidth{1.505625pt}%
\definecolor{currentstroke}{rgb}{0.501961,0.501961,0.501961}%
\pgfsetstrokecolor{currentstroke}%
\pgfsetstrokeopacity{0.500000}%
\pgfsetdash{}{0pt}%
\pgfpathmoveto{\pgfqpoint{3.452382in}{4.436821in}}%
\pgfusepath{stroke}%
\end{pgfscope}%
\begin{pgfscope}%
\pgfpathrectangle{\pgfqpoint{0.100000in}{2.413063in}}{\pgfqpoint{5.037500in}{3.427208in}}%
\pgfusepath{clip}%
\pgfsetbuttcap%
\pgfsetroundjoin%
\definecolor{currentfill}{rgb}{0.501961,0.501961,0.501961}%
\pgfsetfillcolor{currentfill}%
\pgfsetfillopacity{0.500000}%
\pgfsetlinewidth{0.250937pt}%
\definecolor{currentstroke}{rgb}{0.000000,0.000000,0.000000}%
\pgfsetstrokecolor{currentstroke}%
\pgfsetstrokeopacity{0.500000}%
\pgfsetdash{}{0pt}%
\pgfsys@defobject{currentmarker}{\pgfqpoint{-0.013889in}{-0.013889in}}{\pgfqpoint{0.013889in}{0.013889in}}{%
\pgfpathmoveto{\pgfqpoint{0.000000in}{-0.013889in}}%
\pgfpathcurveto{\pgfqpoint{0.003683in}{-0.013889in}}{\pgfqpoint{0.007216in}{-0.012425in}}{\pgfqpoint{0.009821in}{-0.009821in}}%
\pgfpathcurveto{\pgfqpoint{0.012425in}{-0.007216in}}{\pgfqpoint{0.013889in}{-0.003683in}}{\pgfqpoint{0.013889in}{0.000000in}}%
\pgfpathcurveto{\pgfqpoint{0.013889in}{0.003683in}}{\pgfqpoint{0.012425in}{0.007216in}}{\pgfqpoint{0.009821in}{0.009821in}}%
\pgfpathcurveto{\pgfqpoint{0.007216in}{0.012425in}}{\pgfqpoint{0.003683in}{0.013889in}}{\pgfqpoint{0.000000in}{0.013889in}}%
\pgfpathcurveto{\pgfqpoint{-0.003683in}{0.013889in}}{\pgfqpoint{-0.007216in}{0.012425in}}{\pgfqpoint{-0.009821in}{0.009821in}}%
\pgfpathcurveto{\pgfqpoint{-0.012425in}{0.007216in}}{\pgfqpoint{-0.013889in}{0.003683in}}{\pgfqpoint{-0.013889in}{0.000000in}}%
\pgfpathcurveto{\pgfqpoint{-0.013889in}{-0.003683in}}{\pgfqpoint{-0.012425in}{-0.007216in}}{\pgfqpoint{-0.009821in}{-0.009821in}}%
\pgfpathcurveto{\pgfqpoint{-0.007216in}{-0.012425in}}{\pgfqpoint{-0.003683in}{-0.013889in}}{\pgfqpoint{0.000000in}{-0.013889in}}%
\pgfpathclose%
\pgfusepath{stroke,fill}%
}%
\begin{pgfscope}%
\pgfsys@transformshift{3.452382in}{4.436821in}%
\pgfsys@useobject{currentmarker}{}%
\end{pgfscope}%
\end{pgfscope}%
\begin{pgfscope}%
\pgfpathrectangle{\pgfqpoint{0.100000in}{2.413063in}}{\pgfqpoint{5.037500in}{3.427208in}}%
\pgfusepath{clip}%
\pgfsetrectcap%
\pgfsetroundjoin%
\pgfsetlinewidth{1.505625pt}%
\definecolor{currentstroke}{rgb}{0.678431,1.000000,0.184314}%
\pgfsetstrokecolor{currentstroke}%
\pgfsetstrokeopacity{0.500000}%
\pgfsetdash{}{0pt}%
\pgfpathmoveto{\pgfqpoint{3.185573in}{4.580908in}}%
\pgfusepath{stroke}%
\end{pgfscope}%
\begin{pgfscope}%
\pgfpathrectangle{\pgfqpoint{0.100000in}{2.413063in}}{\pgfqpoint{5.037500in}{3.427208in}}%
\pgfusepath{clip}%
\pgfsetbuttcap%
\pgfsetroundjoin%
\definecolor{currentfill}{rgb}{0.678431,1.000000,0.184314}%
\pgfsetfillcolor{currentfill}%
\pgfsetfillopacity{0.500000}%
\pgfsetlinewidth{0.250937pt}%
\definecolor{currentstroke}{rgb}{0.000000,0.000000,0.000000}%
\pgfsetstrokecolor{currentstroke}%
\pgfsetstrokeopacity{0.500000}%
\pgfsetdash{}{0pt}%
\pgfsys@defobject{currentmarker}{\pgfqpoint{-0.011111in}{-0.011111in}}{\pgfqpoint{0.011111in}{0.011111in}}{%
\pgfpathmoveto{\pgfqpoint{0.000000in}{-0.011111in}}%
\pgfpathcurveto{\pgfqpoint{0.002947in}{-0.011111in}}{\pgfqpoint{0.005773in}{-0.009940in}}{\pgfqpoint{0.007857in}{-0.007857in}}%
\pgfpathcurveto{\pgfqpoint{0.009940in}{-0.005773in}}{\pgfqpoint{0.011111in}{-0.002947in}}{\pgfqpoint{0.011111in}{0.000000in}}%
\pgfpathcurveto{\pgfqpoint{0.011111in}{0.002947in}}{\pgfqpoint{0.009940in}{0.005773in}}{\pgfqpoint{0.007857in}{0.007857in}}%
\pgfpathcurveto{\pgfqpoint{0.005773in}{0.009940in}}{\pgfqpoint{0.002947in}{0.011111in}}{\pgfqpoint{0.000000in}{0.011111in}}%
\pgfpathcurveto{\pgfqpoint{-0.002947in}{0.011111in}}{\pgfqpoint{-0.005773in}{0.009940in}}{\pgfqpoint{-0.007857in}{0.007857in}}%
\pgfpathcurveto{\pgfqpoint{-0.009940in}{0.005773in}}{\pgfqpoint{-0.011111in}{0.002947in}}{\pgfqpoint{-0.011111in}{0.000000in}}%
\pgfpathcurveto{\pgfqpoint{-0.011111in}{-0.002947in}}{\pgfqpoint{-0.009940in}{-0.005773in}}{\pgfqpoint{-0.007857in}{-0.007857in}}%
\pgfpathcurveto{\pgfqpoint{-0.005773in}{-0.009940in}}{\pgfqpoint{-0.002947in}{-0.011111in}}{\pgfqpoint{0.000000in}{-0.011111in}}%
\pgfpathclose%
\pgfusepath{stroke,fill}%
}%
\begin{pgfscope}%
\pgfsys@transformshift{3.185573in}{4.580908in}%
\pgfsys@useobject{currentmarker}{}%
\end{pgfscope}%
\end{pgfscope}%
\begin{pgfscope}%
\pgfpathrectangle{\pgfqpoint{0.100000in}{2.413063in}}{\pgfqpoint{5.037500in}{3.427208in}}%
\pgfusepath{clip}%
\pgfsetrectcap%
\pgfsetroundjoin%
\pgfsetlinewidth{1.505625pt}%
\definecolor{currentstroke}{rgb}{0.000000,0.000000,1.000000}%
\pgfsetstrokecolor{currentstroke}%
\pgfsetstrokeopacity{0.500000}%
\pgfsetdash{}{0pt}%
\pgfpathmoveto{\pgfqpoint{3.338431in}{4.396084in}}%
\pgfusepath{stroke}%
\end{pgfscope}%
\begin{pgfscope}%
\pgfpathrectangle{\pgfqpoint{0.100000in}{2.413063in}}{\pgfqpoint{5.037500in}{3.427208in}}%
\pgfusepath{clip}%
\pgfsetbuttcap%
\pgfsetroundjoin%
\definecolor{currentfill}{rgb}{0.000000,0.000000,1.000000}%
\pgfsetfillcolor{currentfill}%
\pgfsetfillopacity{0.500000}%
\pgfsetlinewidth{0.250937pt}%
\definecolor{currentstroke}{rgb}{0.000000,0.000000,0.000000}%
\pgfsetstrokecolor{currentstroke}%
\pgfsetstrokeopacity{0.500000}%
\pgfsetdash{}{0pt}%
\pgfsys@defobject{currentmarker}{\pgfqpoint{-0.033333in}{-0.033333in}}{\pgfqpoint{0.033333in}{0.033333in}}{%
\pgfpathmoveto{\pgfqpoint{0.000000in}{-0.033333in}}%
\pgfpathcurveto{\pgfqpoint{0.008840in}{-0.033333in}}{\pgfqpoint{0.017319in}{-0.029821in}}{\pgfqpoint{0.023570in}{-0.023570in}}%
\pgfpathcurveto{\pgfqpoint{0.029821in}{-0.017319in}}{\pgfqpoint{0.033333in}{-0.008840in}}{\pgfqpoint{0.033333in}{0.000000in}}%
\pgfpathcurveto{\pgfqpoint{0.033333in}{0.008840in}}{\pgfqpoint{0.029821in}{0.017319in}}{\pgfqpoint{0.023570in}{0.023570in}}%
\pgfpathcurveto{\pgfqpoint{0.017319in}{0.029821in}}{\pgfqpoint{0.008840in}{0.033333in}}{\pgfqpoint{0.000000in}{0.033333in}}%
\pgfpathcurveto{\pgfqpoint{-0.008840in}{0.033333in}}{\pgfqpoint{-0.017319in}{0.029821in}}{\pgfqpoint{-0.023570in}{0.023570in}}%
\pgfpathcurveto{\pgfqpoint{-0.029821in}{0.017319in}}{\pgfqpoint{-0.033333in}{0.008840in}}{\pgfqpoint{-0.033333in}{0.000000in}}%
\pgfpathcurveto{\pgfqpoint{-0.033333in}{-0.008840in}}{\pgfqpoint{-0.029821in}{-0.017319in}}{\pgfqpoint{-0.023570in}{-0.023570in}}%
\pgfpathcurveto{\pgfqpoint{-0.017319in}{-0.029821in}}{\pgfqpoint{-0.008840in}{-0.033333in}}{\pgfqpoint{0.000000in}{-0.033333in}}%
\pgfpathclose%
\pgfusepath{stroke,fill}%
}%
\begin{pgfscope}%
\pgfsys@transformshift{3.338431in}{4.396084in}%
\pgfsys@useobject{currentmarker}{}%
\end{pgfscope}%
\end{pgfscope}%
\begin{pgfscope}%
\pgfpathrectangle{\pgfqpoint{0.100000in}{2.413063in}}{\pgfqpoint{5.037500in}{3.427208in}}%
\pgfusepath{clip}%
\pgfsetrectcap%
\pgfsetroundjoin%
\pgfsetlinewidth{1.505625pt}%
\definecolor{currentstroke}{rgb}{0.000000,0.000000,1.000000}%
\pgfsetstrokecolor{currentstroke}%
\pgfsetstrokeopacity{0.500000}%
\pgfsetdash{}{0pt}%
\pgfpathmoveto{\pgfqpoint{3.423139in}{4.549440in}}%
\pgfusepath{stroke}%
\end{pgfscope}%
\begin{pgfscope}%
\pgfpathrectangle{\pgfqpoint{0.100000in}{2.413063in}}{\pgfqpoint{5.037500in}{3.427208in}}%
\pgfusepath{clip}%
\pgfsetbuttcap%
\pgfsetroundjoin%
\definecolor{currentfill}{rgb}{0.000000,0.000000,1.000000}%
\pgfsetfillcolor{currentfill}%
\pgfsetfillopacity{0.500000}%
\pgfsetlinewidth{0.250937pt}%
\definecolor{currentstroke}{rgb}{0.000000,0.000000,0.000000}%
\pgfsetstrokecolor{currentstroke}%
\pgfsetstrokeopacity{0.500000}%
\pgfsetdash{}{0pt}%
\pgfsys@defobject{currentmarker}{\pgfqpoint{-0.005556in}{-0.005556in}}{\pgfqpoint{0.005556in}{0.005556in}}{%
\pgfpathmoveto{\pgfqpoint{0.000000in}{-0.005556in}}%
\pgfpathcurveto{\pgfqpoint{0.001473in}{-0.005556in}}{\pgfqpoint{0.002887in}{-0.004970in}}{\pgfqpoint{0.003928in}{-0.003928in}}%
\pgfpathcurveto{\pgfqpoint{0.004970in}{-0.002887in}}{\pgfqpoint{0.005556in}{-0.001473in}}{\pgfqpoint{0.005556in}{0.000000in}}%
\pgfpathcurveto{\pgfqpoint{0.005556in}{0.001473in}}{\pgfqpoint{0.004970in}{0.002887in}}{\pgfqpoint{0.003928in}{0.003928in}}%
\pgfpathcurveto{\pgfqpoint{0.002887in}{0.004970in}}{\pgfqpoint{0.001473in}{0.005556in}}{\pgfqpoint{0.000000in}{0.005556in}}%
\pgfpathcurveto{\pgfqpoint{-0.001473in}{0.005556in}}{\pgfqpoint{-0.002887in}{0.004970in}}{\pgfqpoint{-0.003928in}{0.003928in}}%
\pgfpathcurveto{\pgfqpoint{-0.004970in}{0.002887in}}{\pgfqpoint{-0.005556in}{0.001473in}}{\pgfqpoint{-0.005556in}{0.000000in}}%
\pgfpathcurveto{\pgfqpoint{-0.005556in}{-0.001473in}}{\pgfqpoint{-0.004970in}{-0.002887in}}{\pgfqpoint{-0.003928in}{-0.003928in}}%
\pgfpathcurveto{\pgfqpoint{-0.002887in}{-0.004970in}}{\pgfqpoint{-0.001473in}{-0.005556in}}{\pgfqpoint{0.000000in}{-0.005556in}}%
\pgfpathclose%
\pgfusepath{stroke,fill}%
}%
\begin{pgfscope}%
\pgfsys@transformshift{3.423139in}{4.549440in}%
\pgfsys@useobject{currentmarker}{}%
\end{pgfscope}%
\end{pgfscope}%
\begin{pgfscope}%
\pgfpathrectangle{\pgfqpoint{0.100000in}{2.413063in}}{\pgfqpoint{5.037500in}{3.427208in}}%
\pgfusepath{clip}%
\pgfsetrectcap%
\pgfsetroundjoin%
\pgfsetlinewidth{1.505625pt}%
\definecolor{currentstroke}{rgb}{0.000000,0.000000,1.000000}%
\pgfsetstrokecolor{currentstroke}%
\pgfsetstrokeopacity{0.500000}%
\pgfsetdash{}{0pt}%
\pgfpathmoveto{\pgfqpoint{3.276435in}{4.490082in}}%
\pgfusepath{stroke}%
\end{pgfscope}%
\begin{pgfscope}%
\pgfpathrectangle{\pgfqpoint{0.100000in}{2.413063in}}{\pgfqpoint{5.037500in}{3.427208in}}%
\pgfusepath{clip}%
\pgfsetbuttcap%
\pgfsetroundjoin%
\definecolor{currentfill}{rgb}{0.000000,0.000000,1.000000}%
\pgfsetfillcolor{currentfill}%
\pgfsetfillopacity{0.500000}%
\pgfsetlinewidth{0.250937pt}%
\definecolor{currentstroke}{rgb}{0.000000,0.000000,0.000000}%
\pgfsetstrokecolor{currentstroke}%
\pgfsetstrokeopacity{0.500000}%
\pgfsetdash{}{0pt}%
\pgfsys@defobject{currentmarker}{\pgfqpoint{-0.011111in}{-0.011111in}}{\pgfqpoint{0.011111in}{0.011111in}}{%
\pgfpathmoveto{\pgfqpoint{0.000000in}{-0.011111in}}%
\pgfpathcurveto{\pgfqpoint{0.002947in}{-0.011111in}}{\pgfqpoint{0.005773in}{-0.009940in}}{\pgfqpoint{0.007857in}{-0.007857in}}%
\pgfpathcurveto{\pgfqpoint{0.009940in}{-0.005773in}}{\pgfqpoint{0.011111in}{-0.002947in}}{\pgfqpoint{0.011111in}{0.000000in}}%
\pgfpathcurveto{\pgfqpoint{0.011111in}{0.002947in}}{\pgfqpoint{0.009940in}{0.005773in}}{\pgfqpoint{0.007857in}{0.007857in}}%
\pgfpathcurveto{\pgfqpoint{0.005773in}{0.009940in}}{\pgfqpoint{0.002947in}{0.011111in}}{\pgfqpoint{0.000000in}{0.011111in}}%
\pgfpathcurveto{\pgfqpoint{-0.002947in}{0.011111in}}{\pgfqpoint{-0.005773in}{0.009940in}}{\pgfqpoint{-0.007857in}{0.007857in}}%
\pgfpathcurveto{\pgfqpoint{-0.009940in}{0.005773in}}{\pgfqpoint{-0.011111in}{0.002947in}}{\pgfqpoint{-0.011111in}{0.000000in}}%
\pgfpathcurveto{\pgfqpoint{-0.011111in}{-0.002947in}}{\pgfqpoint{-0.009940in}{-0.005773in}}{\pgfqpoint{-0.007857in}{-0.007857in}}%
\pgfpathcurveto{\pgfqpoint{-0.005773in}{-0.009940in}}{\pgfqpoint{-0.002947in}{-0.011111in}}{\pgfqpoint{0.000000in}{-0.011111in}}%
\pgfpathclose%
\pgfusepath{stroke,fill}%
}%
\begin{pgfscope}%
\pgfsys@transformshift{3.276435in}{4.490082in}%
\pgfsys@useobject{currentmarker}{}%
\end{pgfscope}%
\end{pgfscope}%
\begin{pgfscope}%
\pgfpathrectangle{\pgfqpoint{0.100000in}{2.413063in}}{\pgfqpoint{5.037500in}{3.427208in}}%
\pgfusepath{clip}%
\pgfsetrectcap%
\pgfsetroundjoin%
\pgfsetlinewidth{1.505625pt}%
\definecolor{currentstroke}{rgb}{0.000000,0.000000,1.000000}%
\pgfsetstrokecolor{currentstroke}%
\pgfsetstrokeopacity{0.500000}%
\pgfsetdash{}{0pt}%
\pgfpathmoveto{\pgfqpoint{3.307789in}{4.674204in}}%
\pgfusepath{stroke}%
\end{pgfscope}%
\begin{pgfscope}%
\pgfpathrectangle{\pgfqpoint{0.100000in}{2.413063in}}{\pgfqpoint{5.037500in}{3.427208in}}%
\pgfusepath{clip}%
\pgfsetbuttcap%
\pgfsetroundjoin%
\definecolor{currentfill}{rgb}{0.000000,0.000000,1.000000}%
\pgfsetfillcolor{currentfill}%
\pgfsetfillopacity{0.500000}%
\pgfsetlinewidth{0.250937pt}%
\definecolor{currentstroke}{rgb}{0.000000,0.000000,0.000000}%
\pgfsetstrokecolor{currentstroke}%
\pgfsetstrokeopacity{0.500000}%
\pgfsetdash{}{0pt}%
\pgfsys@defobject{currentmarker}{\pgfqpoint{-0.036111in}{-0.036111in}}{\pgfqpoint{0.036111in}{0.036111in}}{%
\pgfpathmoveto{\pgfqpoint{0.000000in}{-0.036111in}}%
\pgfpathcurveto{\pgfqpoint{0.009577in}{-0.036111in}}{\pgfqpoint{0.018763in}{-0.032306in}}{\pgfqpoint{0.025534in}{-0.025534in}}%
\pgfpathcurveto{\pgfqpoint{0.032306in}{-0.018763in}}{\pgfqpoint{0.036111in}{-0.009577in}}{\pgfqpoint{0.036111in}{0.000000in}}%
\pgfpathcurveto{\pgfqpoint{0.036111in}{0.009577in}}{\pgfqpoint{0.032306in}{0.018763in}}{\pgfqpoint{0.025534in}{0.025534in}}%
\pgfpathcurveto{\pgfqpoint{0.018763in}{0.032306in}}{\pgfqpoint{0.009577in}{0.036111in}}{\pgfqpoint{0.000000in}{0.036111in}}%
\pgfpathcurveto{\pgfqpoint{-0.009577in}{0.036111in}}{\pgfqpoint{-0.018763in}{0.032306in}}{\pgfqpoint{-0.025534in}{0.025534in}}%
\pgfpathcurveto{\pgfqpoint{-0.032306in}{0.018763in}}{\pgfqpoint{-0.036111in}{0.009577in}}{\pgfqpoint{-0.036111in}{0.000000in}}%
\pgfpathcurveto{\pgfqpoint{-0.036111in}{-0.009577in}}{\pgfqpoint{-0.032306in}{-0.018763in}}{\pgfqpoint{-0.025534in}{-0.025534in}}%
\pgfpathcurveto{\pgfqpoint{-0.018763in}{-0.032306in}}{\pgfqpoint{-0.009577in}{-0.036111in}}{\pgfqpoint{0.000000in}{-0.036111in}}%
\pgfpathclose%
\pgfusepath{stroke,fill}%
}%
\begin{pgfscope}%
\pgfsys@transformshift{3.307789in}{4.674204in}%
\pgfsys@useobject{currentmarker}{}%
\end{pgfscope}%
\end{pgfscope}%
\begin{pgfscope}%
\pgfpathrectangle{\pgfqpoint{0.100000in}{2.413063in}}{\pgfqpoint{5.037500in}{3.427208in}}%
\pgfusepath{clip}%
\pgfsetrectcap%
\pgfsetroundjoin%
\pgfsetlinewidth{1.505625pt}%
\definecolor{currentstroke}{rgb}{0.000000,0.000000,1.000000}%
\pgfsetstrokecolor{currentstroke}%
\pgfsetstrokeopacity{0.500000}%
\pgfsetdash{}{0pt}%
\pgfpathmoveto{\pgfqpoint{3.277687in}{4.386909in}}%
\pgfusepath{stroke}%
\end{pgfscope}%
\begin{pgfscope}%
\pgfpathrectangle{\pgfqpoint{0.100000in}{2.413063in}}{\pgfqpoint{5.037500in}{3.427208in}}%
\pgfusepath{clip}%
\pgfsetbuttcap%
\pgfsetroundjoin%
\definecolor{currentfill}{rgb}{0.000000,0.000000,1.000000}%
\pgfsetfillcolor{currentfill}%
\pgfsetfillopacity{0.500000}%
\pgfsetlinewidth{0.250937pt}%
\definecolor{currentstroke}{rgb}{0.000000,0.000000,0.000000}%
\pgfsetstrokecolor{currentstroke}%
\pgfsetstrokeopacity{0.500000}%
\pgfsetdash{}{0pt}%
\pgfsys@defobject{currentmarker}{\pgfqpoint{-0.016667in}{-0.016667in}}{\pgfqpoint{0.016667in}{0.016667in}}{%
\pgfpathmoveto{\pgfqpoint{0.000000in}{-0.016667in}}%
\pgfpathcurveto{\pgfqpoint{0.004420in}{-0.016667in}}{\pgfqpoint{0.008660in}{-0.014911in}}{\pgfqpoint{0.011785in}{-0.011785in}}%
\pgfpathcurveto{\pgfqpoint{0.014911in}{-0.008660in}}{\pgfqpoint{0.016667in}{-0.004420in}}{\pgfqpoint{0.016667in}{0.000000in}}%
\pgfpathcurveto{\pgfqpoint{0.016667in}{0.004420in}}{\pgfqpoint{0.014911in}{0.008660in}}{\pgfqpoint{0.011785in}{0.011785in}}%
\pgfpathcurveto{\pgfqpoint{0.008660in}{0.014911in}}{\pgfqpoint{0.004420in}{0.016667in}}{\pgfqpoint{0.000000in}{0.016667in}}%
\pgfpathcurveto{\pgfqpoint{-0.004420in}{0.016667in}}{\pgfqpoint{-0.008660in}{0.014911in}}{\pgfqpoint{-0.011785in}{0.011785in}}%
\pgfpathcurveto{\pgfqpoint{-0.014911in}{0.008660in}}{\pgfqpoint{-0.016667in}{0.004420in}}{\pgfqpoint{-0.016667in}{0.000000in}}%
\pgfpathcurveto{\pgfqpoint{-0.016667in}{-0.004420in}}{\pgfqpoint{-0.014911in}{-0.008660in}}{\pgfqpoint{-0.011785in}{-0.011785in}}%
\pgfpathcurveto{\pgfqpoint{-0.008660in}{-0.014911in}}{\pgfqpoint{-0.004420in}{-0.016667in}}{\pgfqpoint{0.000000in}{-0.016667in}}%
\pgfpathclose%
\pgfusepath{stroke,fill}%
}%
\begin{pgfscope}%
\pgfsys@transformshift{3.277687in}{4.386909in}%
\pgfsys@useobject{currentmarker}{}%
\end{pgfscope}%
\end{pgfscope}%
\begin{pgfscope}%
\pgfpathrectangle{\pgfqpoint{0.100000in}{2.413063in}}{\pgfqpoint{5.037500in}{3.427208in}}%
\pgfusepath{clip}%
\pgfsetrectcap%
\pgfsetroundjoin%
\pgfsetlinewidth{1.505625pt}%
\definecolor{currentstroke}{rgb}{0.678431,1.000000,0.184314}%
\pgfsetstrokecolor{currentstroke}%
\pgfsetstrokeopacity{0.500000}%
\pgfsetdash{}{0pt}%
\pgfpathmoveto{\pgfqpoint{3.558808in}{4.335360in}}%
\pgfusepath{stroke}%
\end{pgfscope}%
\begin{pgfscope}%
\pgfpathrectangle{\pgfqpoint{0.100000in}{2.413063in}}{\pgfqpoint{5.037500in}{3.427208in}}%
\pgfusepath{clip}%
\pgfsetbuttcap%
\pgfsetroundjoin%
\definecolor{currentfill}{rgb}{0.678431,1.000000,0.184314}%
\pgfsetfillcolor{currentfill}%
\pgfsetfillopacity{0.500000}%
\pgfsetlinewidth{0.250937pt}%
\definecolor{currentstroke}{rgb}{0.000000,0.000000,0.000000}%
\pgfsetstrokecolor{currentstroke}%
\pgfsetstrokeopacity{0.500000}%
\pgfsetdash{}{0pt}%
\pgfsys@defobject{currentmarker}{\pgfqpoint{-0.044444in}{-0.044444in}}{\pgfqpoint{0.044444in}{0.044444in}}{%
\pgfpathmoveto{\pgfqpoint{0.000000in}{-0.044444in}}%
\pgfpathcurveto{\pgfqpoint{0.011787in}{-0.044444in}}{\pgfqpoint{0.023092in}{-0.039761in}}{\pgfqpoint{0.031427in}{-0.031427in}}%
\pgfpathcurveto{\pgfqpoint{0.039761in}{-0.023092in}}{\pgfqpoint{0.044444in}{-0.011787in}}{\pgfqpoint{0.044444in}{0.000000in}}%
\pgfpathcurveto{\pgfqpoint{0.044444in}{0.011787in}}{\pgfqpoint{0.039761in}{0.023092in}}{\pgfqpoint{0.031427in}{0.031427in}}%
\pgfpathcurveto{\pgfqpoint{0.023092in}{0.039761in}}{\pgfqpoint{0.011787in}{0.044444in}}{\pgfqpoint{0.000000in}{0.044444in}}%
\pgfpathcurveto{\pgfqpoint{-0.011787in}{0.044444in}}{\pgfqpoint{-0.023092in}{0.039761in}}{\pgfqpoint{-0.031427in}{0.031427in}}%
\pgfpathcurveto{\pgfqpoint{-0.039761in}{0.023092in}}{\pgfqpoint{-0.044444in}{0.011787in}}{\pgfqpoint{-0.044444in}{0.000000in}}%
\pgfpathcurveto{\pgfqpoint{-0.044444in}{-0.011787in}}{\pgfqpoint{-0.039761in}{-0.023092in}}{\pgfqpoint{-0.031427in}{-0.031427in}}%
\pgfpathcurveto{\pgfqpoint{-0.023092in}{-0.039761in}}{\pgfqpoint{-0.011787in}{-0.044444in}}{\pgfqpoint{0.000000in}{-0.044444in}}%
\pgfpathclose%
\pgfusepath{stroke,fill}%
}%
\begin{pgfscope}%
\pgfsys@transformshift{3.558808in}{4.335360in}%
\pgfsys@useobject{currentmarker}{}%
\end{pgfscope}%
\end{pgfscope}%
\begin{pgfscope}%
\pgfpathrectangle{\pgfqpoint{0.100000in}{2.413063in}}{\pgfqpoint{5.037500in}{3.427208in}}%
\pgfusepath{clip}%
\pgfsetrectcap%
\pgfsetroundjoin%
\pgfsetlinewidth{1.505625pt}%
\definecolor{currentstroke}{rgb}{0.678431,1.000000,0.184314}%
\pgfsetstrokecolor{currentstroke}%
\pgfsetstrokeopacity{0.500000}%
\pgfsetdash{}{0pt}%
\pgfpathmoveto{\pgfqpoint{3.612876in}{4.344575in}}%
\pgfusepath{stroke}%
\end{pgfscope}%
\begin{pgfscope}%
\pgfpathrectangle{\pgfqpoint{0.100000in}{2.413063in}}{\pgfqpoint{5.037500in}{3.427208in}}%
\pgfusepath{clip}%
\pgfsetbuttcap%
\pgfsetroundjoin%
\definecolor{currentfill}{rgb}{0.678431,1.000000,0.184314}%
\pgfsetfillcolor{currentfill}%
\pgfsetfillopacity{0.500000}%
\pgfsetlinewidth{0.250937pt}%
\definecolor{currentstroke}{rgb}{0.000000,0.000000,0.000000}%
\pgfsetstrokecolor{currentstroke}%
\pgfsetstrokeopacity{0.500000}%
\pgfsetdash{}{0pt}%
\pgfsys@defobject{currentmarker}{\pgfqpoint{-0.030556in}{-0.030556in}}{\pgfqpoint{0.030556in}{0.030556in}}{%
\pgfpathmoveto{\pgfqpoint{0.000000in}{-0.030556in}}%
\pgfpathcurveto{\pgfqpoint{0.008103in}{-0.030556in}}{\pgfqpoint{0.015876in}{-0.027336in}}{\pgfqpoint{0.021606in}{-0.021606in}}%
\pgfpathcurveto{\pgfqpoint{0.027336in}{-0.015876in}}{\pgfqpoint{0.030556in}{-0.008103in}}{\pgfqpoint{0.030556in}{0.000000in}}%
\pgfpathcurveto{\pgfqpoint{0.030556in}{0.008103in}}{\pgfqpoint{0.027336in}{0.015876in}}{\pgfqpoint{0.021606in}{0.021606in}}%
\pgfpathcurveto{\pgfqpoint{0.015876in}{0.027336in}}{\pgfqpoint{0.008103in}{0.030556in}}{\pgfqpoint{0.000000in}{0.030556in}}%
\pgfpathcurveto{\pgfqpoint{-0.008103in}{0.030556in}}{\pgfqpoint{-0.015876in}{0.027336in}}{\pgfqpoint{-0.021606in}{0.021606in}}%
\pgfpathcurveto{\pgfqpoint{-0.027336in}{0.015876in}}{\pgfqpoint{-0.030556in}{0.008103in}}{\pgfqpoint{-0.030556in}{0.000000in}}%
\pgfpathcurveto{\pgfqpoint{-0.030556in}{-0.008103in}}{\pgfqpoint{-0.027336in}{-0.015876in}}{\pgfqpoint{-0.021606in}{-0.021606in}}%
\pgfpathcurveto{\pgfqpoint{-0.015876in}{-0.027336in}}{\pgfqpoint{-0.008103in}{-0.030556in}}{\pgfqpoint{0.000000in}{-0.030556in}}%
\pgfpathclose%
\pgfusepath{stroke,fill}%
}%
\begin{pgfscope}%
\pgfsys@transformshift{3.612876in}{4.344575in}%
\pgfsys@useobject{currentmarker}{}%
\end{pgfscope}%
\end{pgfscope}%
\begin{pgfscope}%
\pgfpathrectangle{\pgfqpoint{0.100000in}{2.413063in}}{\pgfqpoint{5.037500in}{3.427208in}}%
\pgfusepath{clip}%
\pgfsetrectcap%
\pgfsetroundjoin%
\pgfsetlinewidth{1.505625pt}%
\definecolor{currentstroke}{rgb}{0.678431,1.000000,0.184314}%
\pgfsetstrokecolor{currentstroke}%
\pgfsetstrokeopacity{0.500000}%
\pgfsetdash{}{0pt}%
\pgfpathmoveto{\pgfqpoint{3.579642in}{4.628409in}}%
\pgfusepath{stroke}%
\end{pgfscope}%
\begin{pgfscope}%
\pgfpathrectangle{\pgfqpoint{0.100000in}{2.413063in}}{\pgfqpoint{5.037500in}{3.427208in}}%
\pgfusepath{clip}%
\pgfsetbuttcap%
\pgfsetroundjoin%
\definecolor{currentfill}{rgb}{0.678431,1.000000,0.184314}%
\pgfsetfillcolor{currentfill}%
\pgfsetfillopacity{0.500000}%
\pgfsetlinewidth{0.250937pt}%
\definecolor{currentstroke}{rgb}{0.000000,0.000000,0.000000}%
\pgfsetstrokecolor{currentstroke}%
\pgfsetstrokeopacity{0.500000}%
\pgfsetdash{}{0pt}%
\pgfsys@defobject{currentmarker}{\pgfqpoint{-0.050000in}{-0.050000in}}{\pgfqpoint{0.050000in}{0.050000in}}{%
\pgfpathmoveto{\pgfqpoint{0.000000in}{-0.050000in}}%
\pgfpathcurveto{\pgfqpoint{0.013260in}{-0.050000in}}{\pgfqpoint{0.025979in}{-0.044732in}}{\pgfqpoint{0.035355in}{-0.035355in}}%
\pgfpathcurveto{\pgfqpoint{0.044732in}{-0.025979in}}{\pgfqpoint{0.050000in}{-0.013260in}}{\pgfqpoint{0.050000in}{0.000000in}}%
\pgfpathcurveto{\pgfqpoint{0.050000in}{0.013260in}}{\pgfqpoint{0.044732in}{0.025979in}}{\pgfqpoint{0.035355in}{0.035355in}}%
\pgfpathcurveto{\pgfqpoint{0.025979in}{0.044732in}}{\pgfqpoint{0.013260in}{0.050000in}}{\pgfqpoint{0.000000in}{0.050000in}}%
\pgfpathcurveto{\pgfqpoint{-0.013260in}{0.050000in}}{\pgfqpoint{-0.025979in}{0.044732in}}{\pgfqpoint{-0.035355in}{0.035355in}}%
\pgfpathcurveto{\pgfqpoint{-0.044732in}{0.025979in}}{\pgfqpoint{-0.050000in}{0.013260in}}{\pgfqpoint{-0.050000in}{0.000000in}}%
\pgfpathcurveto{\pgfqpoint{-0.050000in}{-0.013260in}}{\pgfqpoint{-0.044732in}{-0.025979in}}{\pgfqpoint{-0.035355in}{-0.035355in}}%
\pgfpathcurveto{\pgfqpoint{-0.025979in}{-0.044732in}}{\pgfqpoint{-0.013260in}{-0.050000in}}{\pgfqpoint{0.000000in}{-0.050000in}}%
\pgfpathclose%
\pgfusepath{stroke,fill}%
}%
\begin{pgfscope}%
\pgfsys@transformshift{3.579642in}{4.628409in}%
\pgfsys@useobject{currentmarker}{}%
\end{pgfscope}%
\end{pgfscope}%
\begin{pgfscope}%
\pgfpathrectangle{\pgfqpoint{0.100000in}{2.413063in}}{\pgfqpoint{5.037500in}{3.427208in}}%
\pgfusepath{clip}%
\pgfsetrectcap%
\pgfsetroundjoin%
\pgfsetlinewidth{1.505625pt}%
\definecolor{currentstroke}{rgb}{0.678431,1.000000,0.184314}%
\pgfsetstrokecolor{currentstroke}%
\pgfsetstrokeopacity{0.500000}%
\pgfsetdash{}{0pt}%
\pgfpathmoveto{\pgfqpoint{3.479193in}{4.190553in}}%
\pgfusepath{stroke}%
\end{pgfscope}%
\begin{pgfscope}%
\pgfpathrectangle{\pgfqpoint{0.100000in}{2.413063in}}{\pgfqpoint{5.037500in}{3.427208in}}%
\pgfusepath{clip}%
\pgfsetbuttcap%
\pgfsetroundjoin%
\definecolor{currentfill}{rgb}{0.678431,1.000000,0.184314}%
\pgfsetfillcolor{currentfill}%
\pgfsetfillopacity{0.500000}%
\pgfsetlinewidth{0.250937pt}%
\definecolor{currentstroke}{rgb}{0.000000,0.000000,0.000000}%
\pgfsetstrokecolor{currentstroke}%
\pgfsetstrokeopacity{0.500000}%
\pgfsetdash{}{0pt}%
\pgfsys@defobject{currentmarker}{\pgfqpoint{-0.036111in}{-0.036111in}}{\pgfqpoint{0.036111in}{0.036111in}}{%
\pgfpathmoveto{\pgfqpoint{0.000000in}{-0.036111in}}%
\pgfpathcurveto{\pgfqpoint{0.009577in}{-0.036111in}}{\pgfqpoint{0.018763in}{-0.032306in}}{\pgfqpoint{0.025534in}{-0.025534in}}%
\pgfpathcurveto{\pgfqpoint{0.032306in}{-0.018763in}}{\pgfqpoint{0.036111in}{-0.009577in}}{\pgfqpoint{0.036111in}{0.000000in}}%
\pgfpathcurveto{\pgfqpoint{0.036111in}{0.009577in}}{\pgfqpoint{0.032306in}{0.018763in}}{\pgfqpoint{0.025534in}{0.025534in}}%
\pgfpathcurveto{\pgfqpoint{0.018763in}{0.032306in}}{\pgfqpoint{0.009577in}{0.036111in}}{\pgfqpoint{0.000000in}{0.036111in}}%
\pgfpathcurveto{\pgfqpoint{-0.009577in}{0.036111in}}{\pgfqpoint{-0.018763in}{0.032306in}}{\pgfqpoint{-0.025534in}{0.025534in}}%
\pgfpathcurveto{\pgfqpoint{-0.032306in}{0.018763in}}{\pgfqpoint{-0.036111in}{0.009577in}}{\pgfqpoint{-0.036111in}{0.000000in}}%
\pgfpathcurveto{\pgfqpoint{-0.036111in}{-0.009577in}}{\pgfqpoint{-0.032306in}{-0.018763in}}{\pgfqpoint{-0.025534in}{-0.025534in}}%
\pgfpathcurveto{\pgfqpoint{-0.018763in}{-0.032306in}}{\pgfqpoint{-0.009577in}{-0.036111in}}{\pgfqpoint{0.000000in}{-0.036111in}}%
\pgfpathclose%
\pgfusepath{stroke,fill}%
}%
\begin{pgfscope}%
\pgfsys@transformshift{3.479193in}{4.190553in}%
\pgfsys@useobject{currentmarker}{}%
\end{pgfscope}%
\end{pgfscope}%
\begin{pgfscope}%
\pgfpathrectangle{\pgfqpoint{0.100000in}{2.413063in}}{\pgfqpoint{5.037500in}{3.427208in}}%
\pgfusepath{clip}%
\pgfsetrectcap%
\pgfsetroundjoin%
\pgfsetlinewidth{1.505625pt}%
\definecolor{currentstroke}{rgb}{0.678431,1.000000,0.184314}%
\pgfsetstrokecolor{currentstroke}%
\pgfsetstrokeopacity{0.500000}%
\pgfsetdash{}{0pt}%
\pgfpathmoveto{\pgfqpoint{3.658913in}{4.566924in}}%
\pgfusepath{stroke}%
\end{pgfscope}%
\begin{pgfscope}%
\pgfpathrectangle{\pgfqpoint{0.100000in}{2.413063in}}{\pgfqpoint{5.037500in}{3.427208in}}%
\pgfusepath{clip}%
\pgfsetbuttcap%
\pgfsetroundjoin%
\definecolor{currentfill}{rgb}{0.678431,1.000000,0.184314}%
\pgfsetfillcolor{currentfill}%
\pgfsetfillopacity{0.500000}%
\pgfsetlinewidth{0.250937pt}%
\definecolor{currentstroke}{rgb}{0.000000,0.000000,0.000000}%
\pgfsetstrokecolor{currentstroke}%
\pgfsetstrokeopacity{0.500000}%
\pgfsetdash{}{0pt}%
\pgfsys@defobject{currentmarker}{\pgfqpoint{-0.041667in}{-0.041667in}}{\pgfqpoint{0.041667in}{0.041667in}}{%
\pgfpathmoveto{\pgfqpoint{0.000000in}{-0.041667in}}%
\pgfpathcurveto{\pgfqpoint{0.011050in}{-0.041667in}}{\pgfqpoint{0.021649in}{-0.037276in}}{\pgfqpoint{0.029463in}{-0.029463in}}%
\pgfpathcurveto{\pgfqpoint{0.037276in}{-0.021649in}}{\pgfqpoint{0.041667in}{-0.011050in}}{\pgfqpoint{0.041667in}{0.000000in}}%
\pgfpathcurveto{\pgfqpoint{0.041667in}{0.011050in}}{\pgfqpoint{0.037276in}{0.021649in}}{\pgfqpoint{0.029463in}{0.029463in}}%
\pgfpathcurveto{\pgfqpoint{0.021649in}{0.037276in}}{\pgfqpoint{0.011050in}{0.041667in}}{\pgfqpoint{0.000000in}{0.041667in}}%
\pgfpathcurveto{\pgfqpoint{-0.011050in}{0.041667in}}{\pgfqpoint{-0.021649in}{0.037276in}}{\pgfqpoint{-0.029463in}{0.029463in}}%
\pgfpathcurveto{\pgfqpoint{-0.037276in}{0.021649in}}{\pgfqpoint{-0.041667in}{0.011050in}}{\pgfqpoint{-0.041667in}{0.000000in}}%
\pgfpathcurveto{\pgfqpoint{-0.041667in}{-0.011050in}}{\pgfqpoint{-0.037276in}{-0.021649in}}{\pgfqpoint{-0.029463in}{-0.029463in}}%
\pgfpathcurveto{\pgfqpoint{-0.021649in}{-0.037276in}}{\pgfqpoint{-0.011050in}{-0.041667in}}{\pgfqpoint{0.000000in}{-0.041667in}}%
\pgfpathclose%
\pgfusepath{stroke,fill}%
}%
\begin{pgfscope}%
\pgfsys@transformshift{3.658913in}{4.566924in}%
\pgfsys@useobject{currentmarker}{}%
\end{pgfscope}%
\end{pgfscope}%
\begin{pgfscope}%
\pgfpathrectangle{\pgfqpoint{0.100000in}{2.413063in}}{\pgfqpoint{5.037500in}{3.427208in}}%
\pgfusepath{clip}%
\pgfsetrectcap%
\pgfsetroundjoin%
\pgfsetlinewidth{1.505625pt}%
\definecolor{currentstroke}{rgb}{0.678431,1.000000,0.184314}%
\pgfsetstrokecolor{currentstroke}%
\pgfsetstrokeopacity{0.500000}%
\pgfsetdash{}{0pt}%
\pgfpathmoveto{\pgfqpoint{3.623626in}{4.450288in}}%
\pgfusepath{stroke}%
\end{pgfscope}%
\begin{pgfscope}%
\pgfpathrectangle{\pgfqpoint{0.100000in}{2.413063in}}{\pgfqpoint{5.037500in}{3.427208in}}%
\pgfusepath{clip}%
\pgfsetbuttcap%
\pgfsetroundjoin%
\definecolor{currentfill}{rgb}{0.678431,1.000000,0.184314}%
\pgfsetfillcolor{currentfill}%
\pgfsetfillopacity{0.500000}%
\pgfsetlinewidth{0.250937pt}%
\definecolor{currentstroke}{rgb}{0.000000,0.000000,0.000000}%
\pgfsetstrokecolor{currentstroke}%
\pgfsetstrokeopacity{0.500000}%
\pgfsetdash{}{0pt}%
\pgfsys@defobject{currentmarker}{\pgfqpoint{-0.036111in}{-0.036111in}}{\pgfqpoint{0.036111in}{0.036111in}}{%
\pgfpathmoveto{\pgfqpoint{0.000000in}{-0.036111in}}%
\pgfpathcurveto{\pgfqpoint{0.009577in}{-0.036111in}}{\pgfqpoint{0.018763in}{-0.032306in}}{\pgfqpoint{0.025534in}{-0.025534in}}%
\pgfpathcurveto{\pgfqpoint{0.032306in}{-0.018763in}}{\pgfqpoint{0.036111in}{-0.009577in}}{\pgfqpoint{0.036111in}{0.000000in}}%
\pgfpathcurveto{\pgfqpoint{0.036111in}{0.009577in}}{\pgfqpoint{0.032306in}{0.018763in}}{\pgfqpoint{0.025534in}{0.025534in}}%
\pgfpathcurveto{\pgfqpoint{0.018763in}{0.032306in}}{\pgfqpoint{0.009577in}{0.036111in}}{\pgfqpoint{0.000000in}{0.036111in}}%
\pgfpathcurveto{\pgfqpoint{-0.009577in}{0.036111in}}{\pgfqpoint{-0.018763in}{0.032306in}}{\pgfqpoint{-0.025534in}{0.025534in}}%
\pgfpathcurveto{\pgfqpoint{-0.032306in}{0.018763in}}{\pgfqpoint{-0.036111in}{0.009577in}}{\pgfqpoint{-0.036111in}{0.000000in}}%
\pgfpathcurveto{\pgfqpoint{-0.036111in}{-0.009577in}}{\pgfqpoint{-0.032306in}{-0.018763in}}{\pgfqpoint{-0.025534in}{-0.025534in}}%
\pgfpathcurveto{\pgfqpoint{-0.018763in}{-0.032306in}}{\pgfqpoint{-0.009577in}{-0.036111in}}{\pgfqpoint{0.000000in}{-0.036111in}}%
\pgfpathclose%
\pgfusepath{stroke,fill}%
}%
\begin{pgfscope}%
\pgfsys@transformshift{3.623626in}{4.450288in}%
\pgfsys@useobject{currentmarker}{}%
\end{pgfscope}%
\end{pgfscope}%
\begin{pgfscope}%
\pgfpathrectangle{\pgfqpoint{0.100000in}{2.413063in}}{\pgfqpoint{5.037500in}{3.427208in}}%
\pgfusepath{clip}%
\pgfsetrectcap%
\pgfsetroundjoin%
\pgfsetlinewidth{1.505625pt}%
\definecolor{currentstroke}{rgb}{0.678431,1.000000,0.184314}%
\pgfsetstrokecolor{currentstroke}%
\pgfsetstrokeopacity{0.500000}%
\pgfsetdash{}{0pt}%
\pgfpathmoveto{\pgfqpoint{3.585492in}{4.407463in}}%
\pgfusepath{stroke}%
\end{pgfscope}%
\begin{pgfscope}%
\pgfpathrectangle{\pgfqpoint{0.100000in}{2.413063in}}{\pgfqpoint{5.037500in}{3.427208in}}%
\pgfusepath{clip}%
\pgfsetbuttcap%
\pgfsetroundjoin%
\definecolor{currentfill}{rgb}{0.678431,1.000000,0.184314}%
\pgfsetfillcolor{currentfill}%
\pgfsetfillopacity{0.500000}%
\pgfsetlinewidth{0.250937pt}%
\definecolor{currentstroke}{rgb}{0.000000,0.000000,0.000000}%
\pgfsetstrokecolor{currentstroke}%
\pgfsetstrokeopacity{0.500000}%
\pgfsetdash{}{0pt}%
\pgfsys@defobject{currentmarker}{\pgfqpoint{-0.036111in}{-0.036111in}}{\pgfqpoint{0.036111in}{0.036111in}}{%
\pgfpathmoveto{\pgfqpoint{0.000000in}{-0.036111in}}%
\pgfpathcurveto{\pgfqpoint{0.009577in}{-0.036111in}}{\pgfqpoint{0.018763in}{-0.032306in}}{\pgfqpoint{0.025534in}{-0.025534in}}%
\pgfpathcurveto{\pgfqpoint{0.032306in}{-0.018763in}}{\pgfqpoint{0.036111in}{-0.009577in}}{\pgfqpoint{0.036111in}{0.000000in}}%
\pgfpathcurveto{\pgfqpoint{0.036111in}{0.009577in}}{\pgfqpoint{0.032306in}{0.018763in}}{\pgfqpoint{0.025534in}{0.025534in}}%
\pgfpathcurveto{\pgfqpoint{0.018763in}{0.032306in}}{\pgfqpoint{0.009577in}{0.036111in}}{\pgfqpoint{0.000000in}{0.036111in}}%
\pgfpathcurveto{\pgfqpoint{-0.009577in}{0.036111in}}{\pgfqpoint{-0.018763in}{0.032306in}}{\pgfqpoint{-0.025534in}{0.025534in}}%
\pgfpathcurveto{\pgfqpoint{-0.032306in}{0.018763in}}{\pgfqpoint{-0.036111in}{0.009577in}}{\pgfqpoint{-0.036111in}{0.000000in}}%
\pgfpathcurveto{\pgfqpoint{-0.036111in}{-0.009577in}}{\pgfqpoint{-0.032306in}{-0.018763in}}{\pgfqpoint{-0.025534in}{-0.025534in}}%
\pgfpathcurveto{\pgfqpoint{-0.018763in}{-0.032306in}}{\pgfqpoint{-0.009577in}{-0.036111in}}{\pgfqpoint{0.000000in}{-0.036111in}}%
\pgfpathclose%
\pgfusepath{stroke,fill}%
}%
\begin{pgfscope}%
\pgfsys@transformshift{3.585492in}{4.407463in}%
\pgfsys@useobject{currentmarker}{}%
\end{pgfscope}%
\end{pgfscope}%
\begin{pgfscope}%
\pgfpathrectangle{\pgfqpoint{0.100000in}{2.413063in}}{\pgfqpoint{5.037500in}{3.427208in}}%
\pgfusepath{clip}%
\pgfsetrectcap%
\pgfsetroundjoin%
\pgfsetlinewidth{1.505625pt}%
\definecolor{currentstroke}{rgb}{0.678431,1.000000,0.184314}%
\pgfsetstrokecolor{currentstroke}%
\pgfsetstrokeopacity{0.500000}%
\pgfsetdash{}{0pt}%
\pgfpathmoveto{\pgfqpoint{3.579613in}{4.489985in}}%
\pgfusepath{stroke}%
\end{pgfscope}%
\begin{pgfscope}%
\pgfpathrectangle{\pgfqpoint{0.100000in}{2.413063in}}{\pgfqpoint{5.037500in}{3.427208in}}%
\pgfusepath{clip}%
\pgfsetbuttcap%
\pgfsetroundjoin%
\definecolor{currentfill}{rgb}{0.678431,1.000000,0.184314}%
\pgfsetfillcolor{currentfill}%
\pgfsetfillopacity{0.500000}%
\pgfsetlinewidth{0.250937pt}%
\definecolor{currentstroke}{rgb}{0.000000,0.000000,0.000000}%
\pgfsetstrokecolor{currentstroke}%
\pgfsetstrokeopacity{0.500000}%
\pgfsetdash{}{0pt}%
\pgfsys@defobject{currentmarker}{\pgfqpoint{-0.027778in}{-0.027778in}}{\pgfqpoint{0.027778in}{0.027778in}}{%
\pgfpathmoveto{\pgfqpoint{0.000000in}{-0.027778in}}%
\pgfpathcurveto{\pgfqpoint{0.007367in}{-0.027778in}}{\pgfqpoint{0.014433in}{-0.024851in}}{\pgfqpoint{0.019642in}{-0.019642in}}%
\pgfpathcurveto{\pgfqpoint{0.024851in}{-0.014433in}}{\pgfqpoint{0.027778in}{-0.007367in}}{\pgfqpoint{0.027778in}{0.000000in}}%
\pgfpathcurveto{\pgfqpoint{0.027778in}{0.007367in}}{\pgfqpoint{0.024851in}{0.014433in}}{\pgfqpoint{0.019642in}{0.019642in}}%
\pgfpathcurveto{\pgfqpoint{0.014433in}{0.024851in}}{\pgfqpoint{0.007367in}{0.027778in}}{\pgfqpoint{0.000000in}{0.027778in}}%
\pgfpathcurveto{\pgfqpoint{-0.007367in}{0.027778in}}{\pgfqpoint{-0.014433in}{0.024851in}}{\pgfqpoint{-0.019642in}{0.019642in}}%
\pgfpathcurveto{\pgfqpoint{-0.024851in}{0.014433in}}{\pgfqpoint{-0.027778in}{0.007367in}}{\pgfqpoint{-0.027778in}{0.000000in}}%
\pgfpathcurveto{\pgfqpoint{-0.027778in}{-0.007367in}}{\pgfqpoint{-0.024851in}{-0.014433in}}{\pgfqpoint{-0.019642in}{-0.019642in}}%
\pgfpathcurveto{\pgfqpoint{-0.014433in}{-0.024851in}}{\pgfqpoint{-0.007367in}{-0.027778in}}{\pgfqpoint{0.000000in}{-0.027778in}}%
\pgfpathclose%
\pgfusepath{stroke,fill}%
}%
\begin{pgfscope}%
\pgfsys@transformshift{3.579613in}{4.489985in}%
\pgfsys@useobject{currentmarker}{}%
\end{pgfscope}%
\end{pgfscope}%
\begin{pgfscope}%
\pgfpathrectangle{\pgfqpoint{0.100000in}{2.413063in}}{\pgfqpoint{5.037500in}{3.427208in}}%
\pgfusepath{clip}%
\pgfsetrectcap%
\pgfsetroundjoin%
\pgfsetlinewidth{1.505625pt}%
\definecolor{currentstroke}{rgb}{0.678431,1.000000,0.184314}%
\pgfsetstrokecolor{currentstroke}%
\pgfsetstrokeopacity{0.500000}%
\pgfsetdash{}{0pt}%
\pgfpathmoveto{\pgfqpoint{3.515644in}{4.475920in}}%
\pgfusepath{stroke}%
\end{pgfscope}%
\begin{pgfscope}%
\pgfpathrectangle{\pgfqpoint{0.100000in}{2.413063in}}{\pgfqpoint{5.037500in}{3.427208in}}%
\pgfusepath{clip}%
\pgfsetbuttcap%
\pgfsetroundjoin%
\definecolor{currentfill}{rgb}{0.678431,1.000000,0.184314}%
\pgfsetfillcolor{currentfill}%
\pgfsetfillopacity{0.500000}%
\pgfsetlinewidth{0.250937pt}%
\definecolor{currentstroke}{rgb}{0.000000,0.000000,0.000000}%
\pgfsetstrokecolor{currentstroke}%
\pgfsetstrokeopacity{0.500000}%
\pgfsetdash{}{0pt}%
\pgfsys@defobject{currentmarker}{\pgfqpoint{-0.036111in}{-0.036111in}}{\pgfqpoint{0.036111in}{0.036111in}}{%
\pgfpathmoveto{\pgfqpoint{0.000000in}{-0.036111in}}%
\pgfpathcurveto{\pgfqpoint{0.009577in}{-0.036111in}}{\pgfqpoint{0.018763in}{-0.032306in}}{\pgfqpoint{0.025534in}{-0.025534in}}%
\pgfpathcurveto{\pgfqpoint{0.032306in}{-0.018763in}}{\pgfqpoint{0.036111in}{-0.009577in}}{\pgfqpoint{0.036111in}{0.000000in}}%
\pgfpathcurveto{\pgfqpoint{0.036111in}{0.009577in}}{\pgfqpoint{0.032306in}{0.018763in}}{\pgfqpoint{0.025534in}{0.025534in}}%
\pgfpathcurveto{\pgfqpoint{0.018763in}{0.032306in}}{\pgfqpoint{0.009577in}{0.036111in}}{\pgfqpoint{0.000000in}{0.036111in}}%
\pgfpathcurveto{\pgfqpoint{-0.009577in}{0.036111in}}{\pgfqpoint{-0.018763in}{0.032306in}}{\pgfqpoint{-0.025534in}{0.025534in}}%
\pgfpathcurveto{\pgfqpoint{-0.032306in}{0.018763in}}{\pgfqpoint{-0.036111in}{0.009577in}}{\pgfqpoint{-0.036111in}{0.000000in}}%
\pgfpathcurveto{\pgfqpoint{-0.036111in}{-0.009577in}}{\pgfqpoint{-0.032306in}{-0.018763in}}{\pgfqpoint{-0.025534in}{-0.025534in}}%
\pgfpathcurveto{\pgfqpoint{-0.018763in}{-0.032306in}}{\pgfqpoint{-0.009577in}{-0.036111in}}{\pgfqpoint{0.000000in}{-0.036111in}}%
\pgfpathclose%
\pgfusepath{stroke,fill}%
}%
\begin{pgfscope}%
\pgfsys@transformshift{3.515644in}{4.475920in}%
\pgfsys@useobject{currentmarker}{}%
\end{pgfscope}%
\end{pgfscope}%
\begin{pgfscope}%
\pgfpathrectangle{\pgfqpoint{0.100000in}{2.413063in}}{\pgfqpoint{5.037500in}{3.427208in}}%
\pgfusepath{clip}%
\pgfsetrectcap%
\pgfsetroundjoin%
\pgfsetlinewidth{1.505625pt}%
\definecolor{currentstroke}{rgb}{0.678431,1.000000,0.184314}%
\pgfsetstrokecolor{currentstroke}%
\pgfsetstrokeopacity{0.500000}%
\pgfsetdash{}{0pt}%
\pgfpathmoveto{\pgfqpoint{3.500672in}{4.623903in}}%
\pgfusepath{stroke}%
\end{pgfscope}%
\begin{pgfscope}%
\pgfpathrectangle{\pgfqpoint{0.100000in}{2.413063in}}{\pgfqpoint{5.037500in}{3.427208in}}%
\pgfusepath{clip}%
\pgfsetbuttcap%
\pgfsetroundjoin%
\definecolor{currentfill}{rgb}{0.678431,1.000000,0.184314}%
\pgfsetfillcolor{currentfill}%
\pgfsetfillopacity{0.500000}%
\pgfsetlinewidth{0.250937pt}%
\definecolor{currentstroke}{rgb}{0.000000,0.000000,0.000000}%
\pgfsetstrokecolor{currentstroke}%
\pgfsetstrokeopacity{0.500000}%
\pgfsetdash{}{0pt}%
\pgfsys@defobject{currentmarker}{\pgfqpoint{-0.058333in}{-0.058333in}}{\pgfqpoint{0.058333in}{0.058333in}}{%
\pgfpathmoveto{\pgfqpoint{0.000000in}{-0.058333in}}%
\pgfpathcurveto{\pgfqpoint{0.015470in}{-0.058333in}}{\pgfqpoint{0.030309in}{-0.052187in}}{\pgfqpoint{0.041248in}{-0.041248in}}%
\pgfpathcurveto{\pgfqpoint{0.052187in}{-0.030309in}}{\pgfqpoint{0.058333in}{-0.015470in}}{\pgfqpoint{0.058333in}{0.000000in}}%
\pgfpathcurveto{\pgfqpoint{0.058333in}{0.015470in}}{\pgfqpoint{0.052187in}{0.030309in}}{\pgfqpoint{0.041248in}{0.041248in}}%
\pgfpathcurveto{\pgfqpoint{0.030309in}{0.052187in}}{\pgfqpoint{0.015470in}{0.058333in}}{\pgfqpoint{0.000000in}{0.058333in}}%
\pgfpathcurveto{\pgfqpoint{-0.015470in}{0.058333in}}{\pgfqpoint{-0.030309in}{0.052187in}}{\pgfqpoint{-0.041248in}{0.041248in}}%
\pgfpathcurveto{\pgfqpoint{-0.052187in}{0.030309in}}{\pgfqpoint{-0.058333in}{0.015470in}}{\pgfqpoint{-0.058333in}{0.000000in}}%
\pgfpathcurveto{\pgfqpoint{-0.058333in}{-0.015470in}}{\pgfqpoint{-0.052187in}{-0.030309in}}{\pgfqpoint{-0.041248in}{-0.041248in}}%
\pgfpathcurveto{\pgfqpoint{-0.030309in}{-0.052187in}}{\pgfqpoint{-0.015470in}{-0.058333in}}{\pgfqpoint{0.000000in}{-0.058333in}}%
\pgfpathclose%
\pgfusepath{stroke,fill}%
}%
\begin{pgfscope}%
\pgfsys@transformshift{3.500672in}{4.623903in}%
\pgfsys@useobject{currentmarker}{}%
\end{pgfscope}%
\end{pgfscope}%
\begin{pgfscope}%
\pgfpathrectangle{\pgfqpoint{0.100000in}{2.413063in}}{\pgfqpoint{5.037500in}{3.427208in}}%
\pgfusepath{clip}%
\pgfsetrectcap%
\pgfsetroundjoin%
\pgfsetlinewidth{1.505625pt}%
\definecolor{currentstroke}{rgb}{0.678431,1.000000,0.184314}%
\pgfsetstrokecolor{currentstroke}%
\pgfsetstrokeopacity{0.500000}%
\pgfsetdash{}{0pt}%
\pgfpathmoveto{\pgfqpoint{3.648286in}{4.463096in}}%
\pgfusepath{stroke}%
\end{pgfscope}%
\begin{pgfscope}%
\pgfpathrectangle{\pgfqpoint{0.100000in}{2.413063in}}{\pgfqpoint{5.037500in}{3.427208in}}%
\pgfusepath{clip}%
\pgfsetbuttcap%
\pgfsetroundjoin%
\definecolor{currentfill}{rgb}{0.678431,1.000000,0.184314}%
\pgfsetfillcolor{currentfill}%
\pgfsetfillopacity{0.500000}%
\pgfsetlinewidth{0.250937pt}%
\definecolor{currentstroke}{rgb}{0.000000,0.000000,0.000000}%
\pgfsetstrokecolor{currentstroke}%
\pgfsetstrokeopacity{0.500000}%
\pgfsetdash{}{0pt}%
\pgfsys@defobject{currentmarker}{\pgfqpoint{-0.050000in}{-0.050000in}}{\pgfqpoint{0.050000in}{0.050000in}}{%
\pgfpathmoveto{\pgfqpoint{0.000000in}{-0.050000in}}%
\pgfpathcurveto{\pgfqpoint{0.013260in}{-0.050000in}}{\pgfqpoint{0.025979in}{-0.044732in}}{\pgfqpoint{0.035355in}{-0.035355in}}%
\pgfpathcurveto{\pgfqpoint{0.044732in}{-0.025979in}}{\pgfqpoint{0.050000in}{-0.013260in}}{\pgfqpoint{0.050000in}{0.000000in}}%
\pgfpathcurveto{\pgfqpoint{0.050000in}{0.013260in}}{\pgfqpoint{0.044732in}{0.025979in}}{\pgfqpoint{0.035355in}{0.035355in}}%
\pgfpathcurveto{\pgfqpoint{0.025979in}{0.044732in}}{\pgfqpoint{0.013260in}{0.050000in}}{\pgfqpoint{0.000000in}{0.050000in}}%
\pgfpathcurveto{\pgfqpoint{-0.013260in}{0.050000in}}{\pgfqpoint{-0.025979in}{0.044732in}}{\pgfqpoint{-0.035355in}{0.035355in}}%
\pgfpathcurveto{\pgfqpoint{-0.044732in}{0.025979in}}{\pgfqpoint{-0.050000in}{0.013260in}}{\pgfqpoint{-0.050000in}{0.000000in}}%
\pgfpathcurveto{\pgfqpoint{-0.050000in}{-0.013260in}}{\pgfqpoint{-0.044732in}{-0.025979in}}{\pgfqpoint{-0.035355in}{-0.035355in}}%
\pgfpathcurveto{\pgfqpoint{-0.025979in}{-0.044732in}}{\pgfqpoint{-0.013260in}{-0.050000in}}{\pgfqpoint{0.000000in}{-0.050000in}}%
\pgfpathclose%
\pgfusepath{stroke,fill}%
}%
\begin{pgfscope}%
\pgfsys@transformshift{3.648286in}{4.463096in}%
\pgfsys@useobject{currentmarker}{}%
\end{pgfscope}%
\end{pgfscope}%
\begin{pgfscope}%
\pgfpathrectangle{\pgfqpoint{0.100000in}{2.413063in}}{\pgfqpoint{5.037500in}{3.427208in}}%
\pgfusepath{clip}%
\pgfsetrectcap%
\pgfsetroundjoin%
\pgfsetlinewidth{1.505625pt}%
\definecolor{currentstroke}{rgb}{0.678431,1.000000,0.184314}%
\pgfsetstrokecolor{currentstroke}%
\pgfsetstrokeopacity{0.500000}%
\pgfsetdash{}{0pt}%
\pgfpathmoveto{\pgfqpoint{3.556209in}{4.626271in}}%
\pgfusepath{stroke}%
\end{pgfscope}%
\begin{pgfscope}%
\pgfpathrectangle{\pgfqpoint{0.100000in}{2.413063in}}{\pgfqpoint{5.037500in}{3.427208in}}%
\pgfusepath{clip}%
\pgfsetbuttcap%
\pgfsetroundjoin%
\definecolor{currentfill}{rgb}{0.678431,1.000000,0.184314}%
\pgfsetfillcolor{currentfill}%
\pgfsetfillopacity{0.500000}%
\pgfsetlinewidth{0.250937pt}%
\definecolor{currentstroke}{rgb}{0.000000,0.000000,0.000000}%
\pgfsetstrokecolor{currentstroke}%
\pgfsetstrokeopacity{0.500000}%
\pgfsetdash{}{0pt}%
\pgfsys@defobject{currentmarker}{\pgfqpoint{-0.041667in}{-0.041667in}}{\pgfqpoint{0.041667in}{0.041667in}}{%
\pgfpathmoveto{\pgfqpoint{0.000000in}{-0.041667in}}%
\pgfpathcurveto{\pgfqpoint{0.011050in}{-0.041667in}}{\pgfqpoint{0.021649in}{-0.037276in}}{\pgfqpoint{0.029463in}{-0.029463in}}%
\pgfpathcurveto{\pgfqpoint{0.037276in}{-0.021649in}}{\pgfqpoint{0.041667in}{-0.011050in}}{\pgfqpoint{0.041667in}{0.000000in}}%
\pgfpathcurveto{\pgfqpoint{0.041667in}{0.011050in}}{\pgfqpoint{0.037276in}{0.021649in}}{\pgfqpoint{0.029463in}{0.029463in}}%
\pgfpathcurveto{\pgfqpoint{0.021649in}{0.037276in}}{\pgfqpoint{0.011050in}{0.041667in}}{\pgfqpoint{0.000000in}{0.041667in}}%
\pgfpathcurveto{\pgfqpoint{-0.011050in}{0.041667in}}{\pgfqpoint{-0.021649in}{0.037276in}}{\pgfqpoint{-0.029463in}{0.029463in}}%
\pgfpathcurveto{\pgfqpoint{-0.037276in}{0.021649in}}{\pgfqpoint{-0.041667in}{0.011050in}}{\pgfqpoint{-0.041667in}{0.000000in}}%
\pgfpathcurveto{\pgfqpoint{-0.041667in}{-0.011050in}}{\pgfqpoint{-0.037276in}{-0.021649in}}{\pgfqpoint{-0.029463in}{-0.029463in}}%
\pgfpathcurveto{\pgfqpoint{-0.021649in}{-0.037276in}}{\pgfqpoint{-0.011050in}{-0.041667in}}{\pgfqpoint{0.000000in}{-0.041667in}}%
\pgfpathclose%
\pgfusepath{stroke,fill}%
}%
\begin{pgfscope}%
\pgfsys@transformshift{3.556209in}{4.626271in}%
\pgfsys@useobject{currentmarker}{}%
\end{pgfscope}%
\end{pgfscope}%
\begin{pgfscope}%
\pgfpathrectangle{\pgfqpoint{0.100000in}{2.413063in}}{\pgfqpoint{5.037500in}{3.427208in}}%
\pgfusepath{clip}%
\pgfsetrectcap%
\pgfsetroundjoin%
\pgfsetlinewidth{1.505625pt}%
\definecolor{currentstroke}{rgb}{0.678431,1.000000,0.184314}%
\pgfsetstrokecolor{currentstroke}%
\pgfsetstrokeopacity{0.500000}%
\pgfsetdash{}{0pt}%
\pgfpathmoveto{\pgfqpoint{3.477710in}{4.362820in}}%
\pgfusepath{stroke}%
\end{pgfscope}%
\begin{pgfscope}%
\pgfpathrectangle{\pgfqpoint{0.100000in}{2.413063in}}{\pgfqpoint{5.037500in}{3.427208in}}%
\pgfusepath{clip}%
\pgfsetbuttcap%
\pgfsetroundjoin%
\definecolor{currentfill}{rgb}{0.678431,1.000000,0.184314}%
\pgfsetfillcolor{currentfill}%
\pgfsetfillopacity{0.500000}%
\pgfsetlinewidth{0.250937pt}%
\definecolor{currentstroke}{rgb}{0.000000,0.000000,0.000000}%
\pgfsetstrokecolor{currentstroke}%
\pgfsetstrokeopacity{0.500000}%
\pgfsetdash{}{0pt}%
\pgfsys@defobject{currentmarker}{\pgfqpoint{-0.063889in}{-0.063889in}}{\pgfqpoint{0.063889in}{0.063889in}}{%
\pgfpathmoveto{\pgfqpoint{0.000000in}{-0.063889in}}%
\pgfpathcurveto{\pgfqpoint{0.016944in}{-0.063889in}}{\pgfqpoint{0.033195in}{-0.057157in}}{\pgfqpoint{0.045176in}{-0.045176in}}%
\pgfpathcurveto{\pgfqpoint{0.057157in}{-0.033195in}}{\pgfqpoint{0.063889in}{-0.016944in}}{\pgfqpoint{0.063889in}{0.000000in}}%
\pgfpathcurveto{\pgfqpoint{0.063889in}{0.016944in}}{\pgfqpoint{0.057157in}{0.033195in}}{\pgfqpoint{0.045176in}{0.045176in}}%
\pgfpathcurveto{\pgfqpoint{0.033195in}{0.057157in}}{\pgfqpoint{0.016944in}{0.063889in}}{\pgfqpoint{0.000000in}{0.063889in}}%
\pgfpathcurveto{\pgfqpoint{-0.016944in}{0.063889in}}{\pgfqpoint{-0.033195in}{0.057157in}}{\pgfqpoint{-0.045176in}{0.045176in}}%
\pgfpathcurveto{\pgfqpoint{-0.057157in}{0.033195in}}{\pgfqpoint{-0.063889in}{0.016944in}}{\pgfqpoint{-0.063889in}{0.000000in}}%
\pgfpathcurveto{\pgfqpoint{-0.063889in}{-0.016944in}}{\pgfqpoint{-0.057157in}{-0.033195in}}{\pgfqpoint{-0.045176in}{-0.045176in}}%
\pgfpathcurveto{\pgfqpoint{-0.033195in}{-0.057157in}}{\pgfqpoint{-0.016944in}{-0.063889in}}{\pgfqpoint{0.000000in}{-0.063889in}}%
\pgfpathclose%
\pgfusepath{stroke,fill}%
}%
\begin{pgfscope}%
\pgfsys@transformshift{3.477710in}{4.362820in}%
\pgfsys@useobject{currentmarker}{}%
\end{pgfscope}%
\end{pgfscope}%
\begin{pgfscope}%
\pgfpathrectangle{\pgfqpoint{0.100000in}{2.413063in}}{\pgfqpoint{5.037500in}{3.427208in}}%
\pgfusepath{clip}%
\pgfsetrectcap%
\pgfsetroundjoin%
\pgfsetlinewidth{1.505625pt}%
\definecolor{currentstroke}{rgb}{0.501961,0.501961,0.501961}%
\pgfsetstrokecolor{currentstroke}%
\pgfsetstrokeopacity{0.500000}%
\pgfsetdash{}{0pt}%
\pgfpathmoveto{\pgfqpoint{2.922572in}{4.630525in}}%
\pgfusepath{stroke}%
\end{pgfscope}%
\begin{pgfscope}%
\pgfpathrectangle{\pgfqpoint{0.100000in}{2.413063in}}{\pgfqpoint{5.037500in}{3.427208in}}%
\pgfusepath{clip}%
\pgfsetbuttcap%
\pgfsetroundjoin%
\definecolor{currentfill}{rgb}{0.501961,0.501961,0.501961}%
\pgfsetfillcolor{currentfill}%
\pgfsetfillopacity{0.500000}%
\pgfsetlinewidth{0.250937pt}%
\definecolor{currentstroke}{rgb}{0.000000,0.000000,0.000000}%
\pgfsetstrokecolor{currentstroke}%
\pgfsetstrokeopacity{0.500000}%
\pgfsetdash{}{0pt}%
\pgfsys@defobject{currentmarker}{\pgfqpoint{-0.013889in}{-0.013889in}}{\pgfqpoint{0.013889in}{0.013889in}}{%
\pgfpathmoveto{\pgfqpoint{0.000000in}{-0.013889in}}%
\pgfpathcurveto{\pgfqpoint{0.003683in}{-0.013889in}}{\pgfqpoint{0.007216in}{-0.012425in}}{\pgfqpoint{0.009821in}{-0.009821in}}%
\pgfpathcurveto{\pgfqpoint{0.012425in}{-0.007216in}}{\pgfqpoint{0.013889in}{-0.003683in}}{\pgfqpoint{0.013889in}{0.000000in}}%
\pgfpathcurveto{\pgfqpoint{0.013889in}{0.003683in}}{\pgfqpoint{0.012425in}{0.007216in}}{\pgfqpoint{0.009821in}{0.009821in}}%
\pgfpathcurveto{\pgfqpoint{0.007216in}{0.012425in}}{\pgfqpoint{0.003683in}{0.013889in}}{\pgfqpoint{0.000000in}{0.013889in}}%
\pgfpathcurveto{\pgfqpoint{-0.003683in}{0.013889in}}{\pgfqpoint{-0.007216in}{0.012425in}}{\pgfqpoint{-0.009821in}{0.009821in}}%
\pgfpathcurveto{\pgfqpoint{-0.012425in}{0.007216in}}{\pgfqpoint{-0.013889in}{0.003683in}}{\pgfqpoint{-0.013889in}{0.000000in}}%
\pgfpathcurveto{\pgfqpoint{-0.013889in}{-0.003683in}}{\pgfqpoint{-0.012425in}{-0.007216in}}{\pgfqpoint{-0.009821in}{-0.009821in}}%
\pgfpathcurveto{\pgfqpoint{-0.007216in}{-0.012425in}}{\pgfqpoint{-0.003683in}{-0.013889in}}{\pgfqpoint{0.000000in}{-0.013889in}}%
\pgfpathclose%
\pgfusepath{stroke,fill}%
}%
\begin{pgfscope}%
\pgfsys@transformshift{2.922572in}{4.630525in}%
\pgfsys@useobject{currentmarker}{}%
\end{pgfscope}%
\end{pgfscope}%
\begin{pgfscope}%
\pgfpathrectangle{\pgfqpoint{0.100000in}{2.413063in}}{\pgfqpoint{5.037500in}{3.427208in}}%
\pgfusepath{clip}%
\pgfsetrectcap%
\pgfsetroundjoin%
\pgfsetlinewidth{1.505625pt}%
\definecolor{currentstroke}{rgb}{0.501961,0.501961,0.501961}%
\pgfsetstrokecolor{currentstroke}%
\pgfsetstrokeopacity{0.500000}%
\pgfsetdash{}{0pt}%
\pgfpathmoveto{\pgfqpoint{3.089457in}{4.628984in}}%
\pgfusepath{stroke}%
\end{pgfscope}%
\begin{pgfscope}%
\pgfpathrectangle{\pgfqpoint{0.100000in}{2.413063in}}{\pgfqpoint{5.037500in}{3.427208in}}%
\pgfusepath{clip}%
\pgfsetbuttcap%
\pgfsetroundjoin%
\definecolor{currentfill}{rgb}{0.501961,0.501961,0.501961}%
\pgfsetfillcolor{currentfill}%
\pgfsetfillopacity{0.500000}%
\pgfsetlinewidth{0.250937pt}%
\definecolor{currentstroke}{rgb}{0.000000,0.000000,0.000000}%
\pgfsetstrokecolor{currentstroke}%
\pgfsetstrokeopacity{0.500000}%
\pgfsetdash{}{0pt}%
\pgfsys@defobject{currentmarker}{\pgfqpoint{-0.013889in}{-0.013889in}}{\pgfqpoint{0.013889in}{0.013889in}}{%
\pgfpathmoveto{\pgfqpoint{0.000000in}{-0.013889in}}%
\pgfpathcurveto{\pgfqpoint{0.003683in}{-0.013889in}}{\pgfqpoint{0.007216in}{-0.012425in}}{\pgfqpoint{0.009821in}{-0.009821in}}%
\pgfpathcurveto{\pgfqpoint{0.012425in}{-0.007216in}}{\pgfqpoint{0.013889in}{-0.003683in}}{\pgfqpoint{0.013889in}{0.000000in}}%
\pgfpathcurveto{\pgfqpoint{0.013889in}{0.003683in}}{\pgfqpoint{0.012425in}{0.007216in}}{\pgfqpoint{0.009821in}{0.009821in}}%
\pgfpathcurveto{\pgfqpoint{0.007216in}{0.012425in}}{\pgfqpoint{0.003683in}{0.013889in}}{\pgfqpoint{0.000000in}{0.013889in}}%
\pgfpathcurveto{\pgfqpoint{-0.003683in}{0.013889in}}{\pgfqpoint{-0.007216in}{0.012425in}}{\pgfqpoint{-0.009821in}{0.009821in}}%
\pgfpathcurveto{\pgfqpoint{-0.012425in}{0.007216in}}{\pgfqpoint{-0.013889in}{0.003683in}}{\pgfqpoint{-0.013889in}{0.000000in}}%
\pgfpathcurveto{\pgfqpoint{-0.013889in}{-0.003683in}}{\pgfqpoint{-0.012425in}{-0.007216in}}{\pgfqpoint{-0.009821in}{-0.009821in}}%
\pgfpathcurveto{\pgfqpoint{-0.007216in}{-0.012425in}}{\pgfqpoint{-0.003683in}{-0.013889in}}{\pgfqpoint{0.000000in}{-0.013889in}}%
\pgfpathclose%
\pgfusepath{stroke,fill}%
}%
\begin{pgfscope}%
\pgfsys@transformshift{3.089457in}{4.628984in}%
\pgfsys@useobject{currentmarker}{}%
\end{pgfscope}%
\end{pgfscope}%
\begin{pgfscope}%
\pgfpathrectangle{\pgfqpoint{0.100000in}{2.413063in}}{\pgfqpoint{5.037500in}{3.427208in}}%
\pgfusepath{clip}%
\pgfsetrectcap%
\pgfsetroundjoin%
\pgfsetlinewidth{1.505625pt}%
\definecolor{currentstroke}{rgb}{0.501961,0.501961,0.501961}%
\pgfsetstrokecolor{currentstroke}%
\pgfsetstrokeopacity{0.500000}%
\pgfsetdash{}{0pt}%
\pgfpathmoveto{\pgfqpoint{2.924486in}{4.580320in}}%
\pgfusepath{stroke}%
\end{pgfscope}%
\begin{pgfscope}%
\pgfpathrectangle{\pgfqpoint{0.100000in}{2.413063in}}{\pgfqpoint{5.037500in}{3.427208in}}%
\pgfusepath{clip}%
\pgfsetbuttcap%
\pgfsetroundjoin%
\definecolor{currentfill}{rgb}{0.501961,0.501961,0.501961}%
\pgfsetfillcolor{currentfill}%
\pgfsetfillopacity{0.500000}%
\pgfsetlinewidth{0.250937pt}%
\definecolor{currentstroke}{rgb}{0.000000,0.000000,0.000000}%
\pgfsetstrokecolor{currentstroke}%
\pgfsetstrokeopacity{0.500000}%
\pgfsetdash{}{0pt}%
\pgfsys@defobject{currentmarker}{\pgfqpoint{-0.013889in}{-0.013889in}}{\pgfqpoint{0.013889in}{0.013889in}}{%
\pgfpathmoveto{\pgfqpoint{0.000000in}{-0.013889in}}%
\pgfpathcurveto{\pgfqpoint{0.003683in}{-0.013889in}}{\pgfqpoint{0.007216in}{-0.012425in}}{\pgfqpoint{0.009821in}{-0.009821in}}%
\pgfpathcurveto{\pgfqpoint{0.012425in}{-0.007216in}}{\pgfqpoint{0.013889in}{-0.003683in}}{\pgfqpoint{0.013889in}{0.000000in}}%
\pgfpathcurveto{\pgfqpoint{0.013889in}{0.003683in}}{\pgfqpoint{0.012425in}{0.007216in}}{\pgfqpoint{0.009821in}{0.009821in}}%
\pgfpathcurveto{\pgfqpoint{0.007216in}{0.012425in}}{\pgfqpoint{0.003683in}{0.013889in}}{\pgfqpoint{0.000000in}{0.013889in}}%
\pgfpathcurveto{\pgfqpoint{-0.003683in}{0.013889in}}{\pgfqpoint{-0.007216in}{0.012425in}}{\pgfqpoint{-0.009821in}{0.009821in}}%
\pgfpathcurveto{\pgfqpoint{-0.012425in}{0.007216in}}{\pgfqpoint{-0.013889in}{0.003683in}}{\pgfqpoint{-0.013889in}{0.000000in}}%
\pgfpathcurveto{\pgfqpoint{-0.013889in}{-0.003683in}}{\pgfqpoint{-0.012425in}{-0.007216in}}{\pgfqpoint{-0.009821in}{-0.009821in}}%
\pgfpathcurveto{\pgfqpoint{-0.007216in}{-0.012425in}}{\pgfqpoint{-0.003683in}{-0.013889in}}{\pgfqpoint{0.000000in}{-0.013889in}}%
\pgfpathclose%
\pgfusepath{stroke,fill}%
}%
\begin{pgfscope}%
\pgfsys@transformshift{2.924486in}{4.580320in}%
\pgfsys@useobject{currentmarker}{}%
\end{pgfscope}%
\end{pgfscope}%
\begin{pgfscope}%
\pgfpathrectangle{\pgfqpoint{0.100000in}{2.413063in}}{\pgfqpoint{5.037500in}{3.427208in}}%
\pgfusepath{clip}%
\pgfsetrectcap%
\pgfsetroundjoin%
\pgfsetlinewidth{1.505625pt}%
\definecolor{currentstroke}{rgb}{0.501961,0.501961,0.501961}%
\pgfsetstrokecolor{currentstroke}%
\pgfsetstrokeopacity{0.500000}%
\pgfsetdash{}{0pt}%
\pgfpathmoveto{\pgfqpoint{3.172689in}{4.693064in}}%
\pgfusepath{stroke}%
\end{pgfscope}%
\begin{pgfscope}%
\pgfpathrectangle{\pgfqpoint{0.100000in}{2.413063in}}{\pgfqpoint{5.037500in}{3.427208in}}%
\pgfusepath{clip}%
\pgfsetbuttcap%
\pgfsetroundjoin%
\definecolor{currentfill}{rgb}{0.501961,0.501961,0.501961}%
\pgfsetfillcolor{currentfill}%
\pgfsetfillopacity{0.500000}%
\pgfsetlinewidth{0.250937pt}%
\definecolor{currentstroke}{rgb}{0.000000,0.000000,0.000000}%
\pgfsetstrokecolor{currentstroke}%
\pgfsetstrokeopacity{0.500000}%
\pgfsetdash{}{0pt}%
\pgfsys@defobject{currentmarker}{\pgfqpoint{-0.013889in}{-0.013889in}}{\pgfqpoint{0.013889in}{0.013889in}}{%
\pgfpathmoveto{\pgfqpoint{0.000000in}{-0.013889in}}%
\pgfpathcurveto{\pgfqpoint{0.003683in}{-0.013889in}}{\pgfqpoint{0.007216in}{-0.012425in}}{\pgfqpoint{0.009821in}{-0.009821in}}%
\pgfpathcurveto{\pgfqpoint{0.012425in}{-0.007216in}}{\pgfqpoint{0.013889in}{-0.003683in}}{\pgfqpoint{0.013889in}{0.000000in}}%
\pgfpathcurveto{\pgfqpoint{0.013889in}{0.003683in}}{\pgfqpoint{0.012425in}{0.007216in}}{\pgfqpoint{0.009821in}{0.009821in}}%
\pgfpathcurveto{\pgfqpoint{0.007216in}{0.012425in}}{\pgfqpoint{0.003683in}{0.013889in}}{\pgfqpoint{0.000000in}{0.013889in}}%
\pgfpathcurveto{\pgfqpoint{-0.003683in}{0.013889in}}{\pgfqpoint{-0.007216in}{0.012425in}}{\pgfqpoint{-0.009821in}{0.009821in}}%
\pgfpathcurveto{\pgfqpoint{-0.012425in}{0.007216in}}{\pgfqpoint{-0.013889in}{0.003683in}}{\pgfqpoint{-0.013889in}{0.000000in}}%
\pgfpathcurveto{\pgfqpoint{-0.013889in}{-0.003683in}}{\pgfqpoint{-0.012425in}{-0.007216in}}{\pgfqpoint{-0.009821in}{-0.009821in}}%
\pgfpathcurveto{\pgfqpoint{-0.007216in}{-0.012425in}}{\pgfqpoint{-0.003683in}{-0.013889in}}{\pgfqpoint{0.000000in}{-0.013889in}}%
\pgfpathclose%
\pgfusepath{stroke,fill}%
}%
\begin{pgfscope}%
\pgfsys@transformshift{3.172689in}{4.693064in}%
\pgfsys@useobject{currentmarker}{}%
\end{pgfscope}%
\end{pgfscope}%
\begin{pgfscope}%
\pgfpathrectangle{\pgfqpoint{0.100000in}{2.413063in}}{\pgfqpoint{5.037500in}{3.427208in}}%
\pgfusepath{clip}%
\pgfsetrectcap%
\pgfsetroundjoin%
\pgfsetlinewidth{1.505625pt}%
\definecolor{currentstroke}{rgb}{0.000000,0.000000,1.000000}%
\pgfsetstrokecolor{currentstroke}%
\pgfsetstrokeopacity{0.500000}%
\pgfsetdash{}{0pt}%
\pgfpathmoveto{\pgfqpoint{3.102876in}{4.593191in}}%
\pgfusepath{stroke}%
\end{pgfscope}%
\begin{pgfscope}%
\pgfpathrectangle{\pgfqpoint{0.100000in}{2.413063in}}{\pgfqpoint{5.037500in}{3.427208in}}%
\pgfusepath{clip}%
\pgfsetbuttcap%
\pgfsetroundjoin%
\definecolor{currentfill}{rgb}{0.000000,0.000000,1.000000}%
\pgfsetfillcolor{currentfill}%
\pgfsetfillopacity{0.500000}%
\pgfsetlinewidth{0.250937pt}%
\definecolor{currentstroke}{rgb}{0.000000,0.000000,0.000000}%
\pgfsetstrokecolor{currentstroke}%
\pgfsetstrokeopacity{0.500000}%
\pgfsetdash{}{0pt}%
\pgfsys@defobject{currentmarker}{\pgfqpoint{-0.008333in}{-0.008333in}}{\pgfqpoint{0.008333in}{0.008333in}}{%
\pgfpathmoveto{\pgfqpoint{0.000000in}{-0.008333in}}%
\pgfpathcurveto{\pgfqpoint{0.002210in}{-0.008333in}}{\pgfqpoint{0.004330in}{-0.007455in}}{\pgfqpoint{0.005893in}{-0.005893in}}%
\pgfpathcurveto{\pgfqpoint{0.007455in}{-0.004330in}}{\pgfqpoint{0.008333in}{-0.002210in}}{\pgfqpoint{0.008333in}{0.000000in}}%
\pgfpathcurveto{\pgfqpoint{0.008333in}{0.002210in}}{\pgfqpoint{0.007455in}{0.004330in}}{\pgfqpoint{0.005893in}{0.005893in}}%
\pgfpathcurveto{\pgfqpoint{0.004330in}{0.007455in}}{\pgfqpoint{0.002210in}{0.008333in}}{\pgfqpoint{0.000000in}{0.008333in}}%
\pgfpathcurveto{\pgfqpoint{-0.002210in}{0.008333in}}{\pgfqpoint{-0.004330in}{0.007455in}}{\pgfqpoint{-0.005893in}{0.005893in}}%
\pgfpathcurveto{\pgfqpoint{-0.007455in}{0.004330in}}{\pgfqpoint{-0.008333in}{0.002210in}}{\pgfqpoint{-0.008333in}{0.000000in}}%
\pgfpathcurveto{\pgfqpoint{-0.008333in}{-0.002210in}}{\pgfqpoint{-0.007455in}{-0.004330in}}{\pgfqpoint{-0.005893in}{-0.005893in}}%
\pgfpathcurveto{\pgfqpoint{-0.004330in}{-0.007455in}}{\pgfqpoint{-0.002210in}{-0.008333in}}{\pgfqpoint{0.000000in}{-0.008333in}}%
\pgfpathclose%
\pgfusepath{stroke,fill}%
}%
\begin{pgfscope}%
\pgfsys@transformshift{3.102876in}{4.593191in}%
\pgfsys@useobject{currentmarker}{}%
\end{pgfscope}%
\end{pgfscope}%
\begin{pgfscope}%
\pgfpathrectangle{\pgfqpoint{0.100000in}{2.413063in}}{\pgfqpoint{5.037500in}{3.427208in}}%
\pgfusepath{clip}%
\pgfsetrectcap%
\pgfsetroundjoin%
\pgfsetlinewidth{1.505625pt}%
\definecolor{currentstroke}{rgb}{0.678431,1.000000,0.184314}%
\pgfsetstrokecolor{currentstroke}%
\pgfsetstrokeopacity{0.500000}%
\pgfsetdash{}{0pt}%
\pgfpathmoveto{\pgfqpoint{2.684559in}{4.684742in}}%
\pgfusepath{stroke}%
\end{pgfscope}%
\begin{pgfscope}%
\pgfpathrectangle{\pgfqpoint{0.100000in}{2.413063in}}{\pgfqpoint{5.037500in}{3.427208in}}%
\pgfusepath{clip}%
\pgfsetbuttcap%
\pgfsetroundjoin%
\definecolor{currentfill}{rgb}{0.678431,1.000000,0.184314}%
\pgfsetfillcolor{currentfill}%
\pgfsetfillopacity{0.500000}%
\pgfsetlinewidth{0.250937pt}%
\definecolor{currentstroke}{rgb}{0.000000,0.000000,0.000000}%
\pgfsetstrokecolor{currentstroke}%
\pgfsetstrokeopacity{0.500000}%
\pgfsetdash{}{0pt}%
\pgfsys@defobject{currentmarker}{\pgfqpoint{-0.008333in}{-0.008333in}}{\pgfqpoint{0.008333in}{0.008333in}}{%
\pgfpathmoveto{\pgfqpoint{0.000000in}{-0.008333in}}%
\pgfpathcurveto{\pgfqpoint{0.002210in}{-0.008333in}}{\pgfqpoint{0.004330in}{-0.007455in}}{\pgfqpoint{0.005893in}{-0.005893in}}%
\pgfpathcurveto{\pgfqpoint{0.007455in}{-0.004330in}}{\pgfqpoint{0.008333in}{-0.002210in}}{\pgfqpoint{0.008333in}{0.000000in}}%
\pgfpathcurveto{\pgfqpoint{0.008333in}{0.002210in}}{\pgfqpoint{0.007455in}{0.004330in}}{\pgfqpoint{0.005893in}{0.005893in}}%
\pgfpathcurveto{\pgfqpoint{0.004330in}{0.007455in}}{\pgfqpoint{0.002210in}{0.008333in}}{\pgfqpoint{0.000000in}{0.008333in}}%
\pgfpathcurveto{\pgfqpoint{-0.002210in}{0.008333in}}{\pgfqpoint{-0.004330in}{0.007455in}}{\pgfqpoint{-0.005893in}{0.005893in}}%
\pgfpathcurveto{\pgfqpoint{-0.007455in}{0.004330in}}{\pgfqpoint{-0.008333in}{0.002210in}}{\pgfqpoint{-0.008333in}{0.000000in}}%
\pgfpathcurveto{\pgfqpoint{-0.008333in}{-0.002210in}}{\pgfqpoint{-0.007455in}{-0.004330in}}{\pgfqpoint{-0.005893in}{-0.005893in}}%
\pgfpathcurveto{\pgfqpoint{-0.004330in}{-0.007455in}}{\pgfqpoint{-0.002210in}{-0.008333in}}{\pgfqpoint{0.000000in}{-0.008333in}}%
\pgfpathclose%
\pgfusepath{stroke,fill}%
}%
\begin{pgfscope}%
\pgfsys@transformshift{2.684559in}{4.684742in}%
\pgfsys@useobject{currentmarker}{}%
\end{pgfscope}%
\end{pgfscope}%
\begin{pgfscope}%
\pgfpathrectangle{\pgfqpoint{0.100000in}{2.413063in}}{\pgfqpoint{5.037500in}{3.427208in}}%
\pgfusepath{clip}%
\pgfsetrectcap%
\pgfsetroundjoin%
\pgfsetlinewidth{1.505625pt}%
\definecolor{currentstroke}{rgb}{0.501961,0.501961,0.501961}%
\pgfsetstrokecolor{currentstroke}%
\pgfsetstrokeopacity{0.500000}%
\pgfsetdash{}{0pt}%
\pgfpathmoveto{\pgfqpoint{3.030598in}{4.687321in}}%
\pgfusepath{stroke}%
\end{pgfscope}%
\begin{pgfscope}%
\pgfpathrectangle{\pgfqpoint{0.100000in}{2.413063in}}{\pgfqpoint{5.037500in}{3.427208in}}%
\pgfusepath{clip}%
\pgfsetbuttcap%
\pgfsetroundjoin%
\definecolor{currentfill}{rgb}{0.501961,0.501961,0.501961}%
\pgfsetfillcolor{currentfill}%
\pgfsetfillopacity{0.500000}%
\pgfsetlinewidth{0.250937pt}%
\definecolor{currentstroke}{rgb}{0.000000,0.000000,0.000000}%
\pgfsetstrokecolor{currentstroke}%
\pgfsetstrokeopacity{0.500000}%
\pgfsetdash{}{0pt}%
\pgfsys@defobject{currentmarker}{\pgfqpoint{-0.013889in}{-0.013889in}}{\pgfqpoint{0.013889in}{0.013889in}}{%
\pgfpathmoveto{\pgfqpoint{0.000000in}{-0.013889in}}%
\pgfpathcurveto{\pgfqpoint{0.003683in}{-0.013889in}}{\pgfqpoint{0.007216in}{-0.012425in}}{\pgfqpoint{0.009821in}{-0.009821in}}%
\pgfpathcurveto{\pgfqpoint{0.012425in}{-0.007216in}}{\pgfqpoint{0.013889in}{-0.003683in}}{\pgfqpoint{0.013889in}{0.000000in}}%
\pgfpathcurveto{\pgfqpoint{0.013889in}{0.003683in}}{\pgfqpoint{0.012425in}{0.007216in}}{\pgfqpoint{0.009821in}{0.009821in}}%
\pgfpathcurveto{\pgfqpoint{0.007216in}{0.012425in}}{\pgfqpoint{0.003683in}{0.013889in}}{\pgfqpoint{0.000000in}{0.013889in}}%
\pgfpathcurveto{\pgfqpoint{-0.003683in}{0.013889in}}{\pgfqpoint{-0.007216in}{0.012425in}}{\pgfqpoint{-0.009821in}{0.009821in}}%
\pgfpathcurveto{\pgfqpoint{-0.012425in}{0.007216in}}{\pgfqpoint{-0.013889in}{0.003683in}}{\pgfqpoint{-0.013889in}{0.000000in}}%
\pgfpathcurveto{\pgfqpoint{-0.013889in}{-0.003683in}}{\pgfqpoint{-0.012425in}{-0.007216in}}{\pgfqpoint{-0.009821in}{-0.009821in}}%
\pgfpathcurveto{\pgfqpoint{-0.007216in}{-0.012425in}}{\pgfqpoint{-0.003683in}{-0.013889in}}{\pgfqpoint{0.000000in}{-0.013889in}}%
\pgfpathclose%
\pgfusepath{stroke,fill}%
}%
\begin{pgfscope}%
\pgfsys@transformshift{3.030598in}{4.687321in}%
\pgfsys@useobject{currentmarker}{}%
\end{pgfscope}%
\end{pgfscope}%
\begin{pgfscope}%
\pgfpathrectangle{\pgfqpoint{0.100000in}{2.413063in}}{\pgfqpoint{5.037500in}{3.427208in}}%
\pgfusepath{clip}%
\pgfsetrectcap%
\pgfsetroundjoin%
\pgfsetlinewidth{1.505625pt}%
\definecolor{currentstroke}{rgb}{0.678431,1.000000,0.184314}%
\pgfsetstrokecolor{currentstroke}%
\pgfsetstrokeopacity{0.500000}%
\pgfsetdash{}{0pt}%
\pgfpathmoveto{\pgfqpoint{2.783006in}{4.277745in}}%
\pgfusepath{stroke}%
\end{pgfscope}%
\begin{pgfscope}%
\pgfpathrectangle{\pgfqpoint{0.100000in}{2.413063in}}{\pgfqpoint{5.037500in}{3.427208in}}%
\pgfusepath{clip}%
\pgfsetbuttcap%
\pgfsetroundjoin%
\definecolor{currentfill}{rgb}{0.678431,1.000000,0.184314}%
\pgfsetfillcolor{currentfill}%
\pgfsetfillopacity{0.500000}%
\pgfsetlinewidth{0.250937pt}%
\definecolor{currentstroke}{rgb}{0.000000,0.000000,0.000000}%
\pgfsetstrokecolor{currentstroke}%
\pgfsetstrokeopacity{0.500000}%
\pgfsetdash{}{0pt}%
\pgfsys@defobject{currentmarker}{\pgfqpoint{-0.011111in}{-0.011111in}}{\pgfqpoint{0.011111in}{0.011111in}}{%
\pgfpathmoveto{\pgfqpoint{0.000000in}{-0.011111in}}%
\pgfpathcurveto{\pgfqpoint{0.002947in}{-0.011111in}}{\pgfqpoint{0.005773in}{-0.009940in}}{\pgfqpoint{0.007857in}{-0.007857in}}%
\pgfpathcurveto{\pgfqpoint{0.009940in}{-0.005773in}}{\pgfqpoint{0.011111in}{-0.002947in}}{\pgfqpoint{0.011111in}{0.000000in}}%
\pgfpathcurveto{\pgfqpoint{0.011111in}{0.002947in}}{\pgfqpoint{0.009940in}{0.005773in}}{\pgfqpoint{0.007857in}{0.007857in}}%
\pgfpathcurveto{\pgfqpoint{0.005773in}{0.009940in}}{\pgfqpoint{0.002947in}{0.011111in}}{\pgfqpoint{0.000000in}{0.011111in}}%
\pgfpathcurveto{\pgfqpoint{-0.002947in}{0.011111in}}{\pgfqpoint{-0.005773in}{0.009940in}}{\pgfqpoint{-0.007857in}{0.007857in}}%
\pgfpathcurveto{\pgfqpoint{-0.009940in}{0.005773in}}{\pgfqpoint{-0.011111in}{0.002947in}}{\pgfqpoint{-0.011111in}{0.000000in}}%
\pgfpathcurveto{\pgfqpoint{-0.011111in}{-0.002947in}}{\pgfqpoint{-0.009940in}{-0.005773in}}{\pgfqpoint{-0.007857in}{-0.007857in}}%
\pgfpathcurveto{\pgfqpoint{-0.005773in}{-0.009940in}}{\pgfqpoint{-0.002947in}{-0.011111in}}{\pgfqpoint{0.000000in}{-0.011111in}}%
\pgfpathclose%
\pgfusepath{stroke,fill}%
}%
\begin{pgfscope}%
\pgfsys@transformshift{2.783006in}{4.277745in}%
\pgfsys@useobject{currentmarker}{}%
\end{pgfscope}%
\end{pgfscope}%
\begin{pgfscope}%
\pgfpathrectangle{\pgfqpoint{0.100000in}{2.413063in}}{\pgfqpoint{5.037500in}{3.427208in}}%
\pgfusepath{clip}%
\pgfsetrectcap%
\pgfsetroundjoin%
\pgfsetlinewidth{1.505625pt}%
\definecolor{currentstroke}{rgb}{0.678431,1.000000,0.184314}%
\pgfsetstrokecolor{currentstroke}%
\pgfsetstrokeopacity{0.500000}%
\pgfsetdash{}{0pt}%
\pgfpathmoveto{\pgfqpoint{2.663858in}{4.303304in}}%
\pgfusepath{stroke}%
\end{pgfscope}%
\begin{pgfscope}%
\pgfpathrectangle{\pgfqpoint{0.100000in}{2.413063in}}{\pgfqpoint{5.037500in}{3.427208in}}%
\pgfusepath{clip}%
\pgfsetbuttcap%
\pgfsetroundjoin%
\definecolor{currentfill}{rgb}{0.678431,1.000000,0.184314}%
\pgfsetfillcolor{currentfill}%
\pgfsetfillopacity{0.500000}%
\pgfsetlinewidth{0.250937pt}%
\definecolor{currentstroke}{rgb}{0.000000,0.000000,0.000000}%
\pgfsetstrokecolor{currentstroke}%
\pgfsetstrokeopacity{0.500000}%
\pgfsetdash{}{0pt}%
\pgfsys@defobject{currentmarker}{\pgfqpoint{-0.008333in}{-0.008333in}}{\pgfqpoint{0.008333in}{0.008333in}}{%
\pgfpathmoveto{\pgfqpoint{0.000000in}{-0.008333in}}%
\pgfpathcurveto{\pgfqpoint{0.002210in}{-0.008333in}}{\pgfqpoint{0.004330in}{-0.007455in}}{\pgfqpoint{0.005893in}{-0.005893in}}%
\pgfpathcurveto{\pgfqpoint{0.007455in}{-0.004330in}}{\pgfqpoint{0.008333in}{-0.002210in}}{\pgfqpoint{0.008333in}{0.000000in}}%
\pgfpathcurveto{\pgfqpoint{0.008333in}{0.002210in}}{\pgfqpoint{0.007455in}{0.004330in}}{\pgfqpoint{0.005893in}{0.005893in}}%
\pgfpathcurveto{\pgfqpoint{0.004330in}{0.007455in}}{\pgfqpoint{0.002210in}{0.008333in}}{\pgfqpoint{0.000000in}{0.008333in}}%
\pgfpathcurveto{\pgfqpoint{-0.002210in}{0.008333in}}{\pgfqpoint{-0.004330in}{0.007455in}}{\pgfqpoint{-0.005893in}{0.005893in}}%
\pgfpathcurveto{\pgfqpoint{-0.007455in}{0.004330in}}{\pgfqpoint{-0.008333in}{0.002210in}}{\pgfqpoint{-0.008333in}{0.000000in}}%
\pgfpathcurveto{\pgfqpoint{-0.008333in}{-0.002210in}}{\pgfqpoint{-0.007455in}{-0.004330in}}{\pgfqpoint{-0.005893in}{-0.005893in}}%
\pgfpathcurveto{\pgfqpoint{-0.004330in}{-0.007455in}}{\pgfqpoint{-0.002210in}{-0.008333in}}{\pgfqpoint{0.000000in}{-0.008333in}}%
\pgfpathclose%
\pgfusepath{stroke,fill}%
}%
\begin{pgfscope}%
\pgfsys@transformshift{2.663858in}{4.303304in}%
\pgfsys@useobject{currentmarker}{}%
\end{pgfscope}%
\end{pgfscope}%
\begin{pgfscope}%
\pgfpathrectangle{\pgfqpoint{0.100000in}{2.413063in}}{\pgfqpoint{5.037500in}{3.427208in}}%
\pgfusepath{clip}%
\pgfsetrectcap%
\pgfsetroundjoin%
\pgfsetlinewidth{1.505625pt}%
\definecolor{currentstroke}{rgb}{0.678431,1.000000,0.184314}%
\pgfsetstrokecolor{currentstroke}%
\pgfsetstrokeopacity{0.500000}%
\pgfsetdash{}{0pt}%
\pgfpathmoveto{\pgfqpoint{2.743539in}{4.286819in}}%
\pgfusepath{stroke}%
\end{pgfscope}%
\begin{pgfscope}%
\pgfpathrectangle{\pgfqpoint{0.100000in}{2.413063in}}{\pgfqpoint{5.037500in}{3.427208in}}%
\pgfusepath{clip}%
\pgfsetbuttcap%
\pgfsetroundjoin%
\definecolor{currentfill}{rgb}{0.678431,1.000000,0.184314}%
\pgfsetfillcolor{currentfill}%
\pgfsetfillopacity{0.500000}%
\pgfsetlinewidth{0.250937pt}%
\definecolor{currentstroke}{rgb}{0.000000,0.000000,0.000000}%
\pgfsetstrokecolor{currentstroke}%
\pgfsetstrokeopacity{0.500000}%
\pgfsetdash{}{0pt}%
\pgfsys@defobject{currentmarker}{\pgfqpoint{-0.019444in}{-0.019444in}}{\pgfqpoint{0.019444in}{0.019444in}}{%
\pgfpathmoveto{\pgfqpoint{0.000000in}{-0.019444in}}%
\pgfpathcurveto{\pgfqpoint{0.005157in}{-0.019444in}}{\pgfqpoint{0.010103in}{-0.017396in}}{\pgfqpoint{0.013749in}{-0.013749in}}%
\pgfpathcurveto{\pgfqpoint{0.017396in}{-0.010103in}}{\pgfqpoint{0.019444in}{-0.005157in}}{\pgfqpoint{0.019444in}{0.000000in}}%
\pgfpathcurveto{\pgfqpoint{0.019444in}{0.005157in}}{\pgfqpoint{0.017396in}{0.010103in}}{\pgfqpoint{0.013749in}{0.013749in}}%
\pgfpathcurveto{\pgfqpoint{0.010103in}{0.017396in}}{\pgfqpoint{0.005157in}{0.019444in}}{\pgfqpoint{0.000000in}{0.019444in}}%
\pgfpathcurveto{\pgfqpoint{-0.005157in}{0.019444in}}{\pgfqpoint{-0.010103in}{0.017396in}}{\pgfqpoint{-0.013749in}{0.013749in}}%
\pgfpathcurveto{\pgfqpoint{-0.017396in}{0.010103in}}{\pgfqpoint{-0.019444in}{0.005157in}}{\pgfqpoint{-0.019444in}{0.000000in}}%
\pgfpathcurveto{\pgfqpoint{-0.019444in}{-0.005157in}}{\pgfqpoint{-0.017396in}{-0.010103in}}{\pgfqpoint{-0.013749in}{-0.013749in}}%
\pgfpathcurveto{\pgfqpoint{-0.010103in}{-0.017396in}}{\pgfqpoint{-0.005157in}{-0.019444in}}{\pgfqpoint{0.000000in}{-0.019444in}}%
\pgfpathclose%
\pgfusepath{stroke,fill}%
}%
\begin{pgfscope}%
\pgfsys@transformshift{2.743539in}{4.286819in}%
\pgfsys@useobject{currentmarker}{}%
\end{pgfscope}%
\end{pgfscope}%
\begin{pgfscope}%
\pgfpathrectangle{\pgfqpoint{0.100000in}{2.413063in}}{\pgfqpoint{5.037500in}{3.427208in}}%
\pgfusepath{clip}%
\pgfsetrectcap%
\pgfsetroundjoin%
\pgfsetlinewidth{1.505625pt}%
\definecolor{currentstroke}{rgb}{0.501961,0.501961,0.501961}%
\pgfsetstrokecolor{currentstroke}%
\pgfsetstrokeopacity{0.500000}%
\pgfsetdash{}{0pt}%
\pgfpathmoveto{\pgfqpoint{2.591100in}{4.133088in}}%
\pgfusepath{stroke}%
\end{pgfscope}%
\begin{pgfscope}%
\pgfpathrectangle{\pgfqpoint{0.100000in}{2.413063in}}{\pgfqpoint{5.037500in}{3.427208in}}%
\pgfusepath{clip}%
\pgfsetbuttcap%
\pgfsetroundjoin%
\definecolor{currentfill}{rgb}{0.501961,0.501961,0.501961}%
\pgfsetfillcolor{currentfill}%
\pgfsetfillopacity{0.500000}%
\pgfsetlinewidth{0.250937pt}%
\definecolor{currentstroke}{rgb}{0.000000,0.000000,0.000000}%
\pgfsetstrokecolor{currentstroke}%
\pgfsetstrokeopacity{0.500000}%
\pgfsetdash{}{0pt}%
\pgfsys@defobject{currentmarker}{\pgfqpoint{-0.013889in}{-0.013889in}}{\pgfqpoint{0.013889in}{0.013889in}}{%
\pgfpathmoveto{\pgfqpoint{0.000000in}{-0.013889in}}%
\pgfpathcurveto{\pgfqpoint{0.003683in}{-0.013889in}}{\pgfqpoint{0.007216in}{-0.012425in}}{\pgfqpoint{0.009821in}{-0.009821in}}%
\pgfpathcurveto{\pgfqpoint{0.012425in}{-0.007216in}}{\pgfqpoint{0.013889in}{-0.003683in}}{\pgfqpoint{0.013889in}{0.000000in}}%
\pgfpathcurveto{\pgfqpoint{0.013889in}{0.003683in}}{\pgfqpoint{0.012425in}{0.007216in}}{\pgfqpoint{0.009821in}{0.009821in}}%
\pgfpathcurveto{\pgfqpoint{0.007216in}{0.012425in}}{\pgfqpoint{0.003683in}{0.013889in}}{\pgfqpoint{0.000000in}{0.013889in}}%
\pgfpathcurveto{\pgfqpoint{-0.003683in}{0.013889in}}{\pgfqpoint{-0.007216in}{0.012425in}}{\pgfqpoint{-0.009821in}{0.009821in}}%
\pgfpathcurveto{\pgfqpoint{-0.012425in}{0.007216in}}{\pgfqpoint{-0.013889in}{0.003683in}}{\pgfqpoint{-0.013889in}{0.000000in}}%
\pgfpathcurveto{\pgfqpoint{-0.013889in}{-0.003683in}}{\pgfqpoint{-0.012425in}{-0.007216in}}{\pgfqpoint{-0.009821in}{-0.009821in}}%
\pgfpathcurveto{\pgfqpoint{-0.007216in}{-0.012425in}}{\pgfqpoint{-0.003683in}{-0.013889in}}{\pgfqpoint{0.000000in}{-0.013889in}}%
\pgfpathclose%
\pgfusepath{stroke,fill}%
}%
\begin{pgfscope}%
\pgfsys@transformshift{2.591100in}{4.133088in}%
\pgfsys@useobject{currentmarker}{}%
\end{pgfscope}%
\end{pgfscope}%
\begin{pgfscope}%
\pgfpathrectangle{\pgfqpoint{0.100000in}{2.413063in}}{\pgfqpoint{5.037500in}{3.427208in}}%
\pgfusepath{clip}%
\pgfsetrectcap%
\pgfsetroundjoin%
\pgfsetlinewidth{1.505625pt}%
\definecolor{currentstroke}{rgb}{0.678431,1.000000,0.184314}%
\pgfsetstrokecolor{currentstroke}%
\pgfsetstrokeopacity{0.500000}%
\pgfsetdash{}{0pt}%
\pgfpathmoveto{\pgfqpoint{3.590438in}{4.086531in}}%
\pgfusepath{stroke}%
\end{pgfscope}%
\begin{pgfscope}%
\pgfpathrectangle{\pgfqpoint{0.100000in}{2.413063in}}{\pgfqpoint{5.037500in}{3.427208in}}%
\pgfusepath{clip}%
\pgfsetbuttcap%
\pgfsetroundjoin%
\definecolor{currentfill}{rgb}{0.678431,1.000000,0.184314}%
\pgfsetfillcolor{currentfill}%
\pgfsetfillopacity{0.500000}%
\pgfsetlinewidth{0.250937pt}%
\definecolor{currentstroke}{rgb}{0.000000,0.000000,0.000000}%
\pgfsetstrokecolor{currentstroke}%
\pgfsetstrokeopacity{0.500000}%
\pgfsetdash{}{0pt}%
\pgfsys@defobject{currentmarker}{\pgfqpoint{-0.025000in}{-0.025000in}}{\pgfqpoint{0.025000in}{0.025000in}}{%
\pgfpathmoveto{\pgfqpoint{0.000000in}{-0.025000in}}%
\pgfpathcurveto{\pgfqpoint{0.006630in}{-0.025000in}}{\pgfqpoint{0.012989in}{-0.022366in}}{\pgfqpoint{0.017678in}{-0.017678in}}%
\pgfpathcurveto{\pgfqpoint{0.022366in}{-0.012989in}}{\pgfqpoint{0.025000in}{-0.006630in}}{\pgfqpoint{0.025000in}{0.000000in}}%
\pgfpathcurveto{\pgfqpoint{0.025000in}{0.006630in}}{\pgfqpoint{0.022366in}{0.012989in}}{\pgfqpoint{0.017678in}{0.017678in}}%
\pgfpathcurveto{\pgfqpoint{0.012989in}{0.022366in}}{\pgfqpoint{0.006630in}{0.025000in}}{\pgfqpoint{0.000000in}{0.025000in}}%
\pgfpathcurveto{\pgfqpoint{-0.006630in}{0.025000in}}{\pgfqpoint{-0.012989in}{0.022366in}}{\pgfqpoint{-0.017678in}{0.017678in}}%
\pgfpathcurveto{\pgfqpoint{-0.022366in}{0.012989in}}{\pgfqpoint{-0.025000in}{0.006630in}}{\pgfqpoint{-0.025000in}{0.000000in}}%
\pgfpathcurveto{\pgfqpoint{-0.025000in}{-0.006630in}}{\pgfqpoint{-0.022366in}{-0.012989in}}{\pgfqpoint{-0.017678in}{-0.017678in}}%
\pgfpathcurveto{\pgfqpoint{-0.012989in}{-0.022366in}}{\pgfqpoint{-0.006630in}{-0.025000in}}{\pgfqpoint{0.000000in}{-0.025000in}}%
\pgfpathclose%
\pgfusepath{stroke,fill}%
}%
\begin{pgfscope}%
\pgfsys@transformshift{3.590438in}{4.086531in}%
\pgfsys@useobject{currentmarker}{}%
\end{pgfscope}%
\end{pgfscope}%
\begin{pgfscope}%
\pgfpathrectangle{\pgfqpoint{0.100000in}{2.413063in}}{\pgfqpoint{5.037500in}{3.427208in}}%
\pgfusepath{clip}%
\pgfsetrectcap%
\pgfsetroundjoin%
\pgfsetlinewidth{1.505625pt}%
\definecolor{currentstroke}{rgb}{0.678431,1.000000,0.184314}%
\pgfsetstrokecolor{currentstroke}%
\pgfsetstrokeopacity{0.500000}%
\pgfsetdash{}{0pt}%
\pgfpathmoveto{\pgfqpoint{3.635835in}{4.172410in}}%
\pgfusepath{stroke}%
\end{pgfscope}%
\begin{pgfscope}%
\pgfpathrectangle{\pgfqpoint{0.100000in}{2.413063in}}{\pgfqpoint{5.037500in}{3.427208in}}%
\pgfusepath{clip}%
\pgfsetbuttcap%
\pgfsetroundjoin%
\definecolor{currentfill}{rgb}{0.678431,1.000000,0.184314}%
\pgfsetfillcolor{currentfill}%
\pgfsetfillopacity{0.500000}%
\pgfsetlinewidth{0.250937pt}%
\definecolor{currentstroke}{rgb}{0.000000,0.000000,0.000000}%
\pgfsetstrokecolor{currentstroke}%
\pgfsetstrokeopacity{0.500000}%
\pgfsetdash{}{0pt}%
\pgfsys@defobject{currentmarker}{\pgfqpoint{-0.005556in}{-0.005556in}}{\pgfqpoint{0.005556in}{0.005556in}}{%
\pgfpathmoveto{\pgfqpoint{0.000000in}{-0.005556in}}%
\pgfpathcurveto{\pgfqpoint{0.001473in}{-0.005556in}}{\pgfqpoint{0.002887in}{-0.004970in}}{\pgfqpoint{0.003928in}{-0.003928in}}%
\pgfpathcurveto{\pgfqpoint{0.004970in}{-0.002887in}}{\pgfqpoint{0.005556in}{-0.001473in}}{\pgfqpoint{0.005556in}{0.000000in}}%
\pgfpathcurveto{\pgfqpoint{0.005556in}{0.001473in}}{\pgfqpoint{0.004970in}{0.002887in}}{\pgfqpoint{0.003928in}{0.003928in}}%
\pgfpathcurveto{\pgfqpoint{0.002887in}{0.004970in}}{\pgfqpoint{0.001473in}{0.005556in}}{\pgfqpoint{0.000000in}{0.005556in}}%
\pgfpathcurveto{\pgfqpoint{-0.001473in}{0.005556in}}{\pgfqpoint{-0.002887in}{0.004970in}}{\pgfqpoint{-0.003928in}{0.003928in}}%
\pgfpathcurveto{\pgfqpoint{-0.004970in}{0.002887in}}{\pgfqpoint{-0.005556in}{0.001473in}}{\pgfqpoint{-0.005556in}{0.000000in}}%
\pgfpathcurveto{\pgfqpoint{-0.005556in}{-0.001473in}}{\pgfqpoint{-0.004970in}{-0.002887in}}{\pgfqpoint{-0.003928in}{-0.003928in}}%
\pgfpathcurveto{\pgfqpoint{-0.002887in}{-0.004970in}}{\pgfqpoint{-0.001473in}{-0.005556in}}{\pgfqpoint{0.000000in}{-0.005556in}}%
\pgfpathclose%
\pgfusepath{stroke,fill}%
}%
\begin{pgfscope}%
\pgfsys@transformshift{3.635835in}{4.172410in}%
\pgfsys@useobject{currentmarker}{}%
\end{pgfscope}%
\end{pgfscope}%
\begin{pgfscope}%
\pgfpathrectangle{\pgfqpoint{0.100000in}{2.413063in}}{\pgfqpoint{5.037500in}{3.427208in}}%
\pgfusepath{clip}%
\pgfsetrectcap%
\pgfsetroundjoin%
\pgfsetlinewidth{1.505625pt}%
\definecolor{currentstroke}{rgb}{0.501961,0.501961,0.501961}%
\pgfsetstrokecolor{currentstroke}%
\pgfsetstrokeopacity{0.500000}%
\pgfsetdash{}{0pt}%
\pgfpathmoveto{\pgfqpoint{3.754570in}{4.226123in}}%
\pgfusepath{stroke}%
\end{pgfscope}%
\begin{pgfscope}%
\pgfpathrectangle{\pgfqpoint{0.100000in}{2.413063in}}{\pgfqpoint{5.037500in}{3.427208in}}%
\pgfusepath{clip}%
\pgfsetbuttcap%
\pgfsetroundjoin%
\definecolor{currentfill}{rgb}{0.501961,0.501961,0.501961}%
\pgfsetfillcolor{currentfill}%
\pgfsetfillopacity{0.500000}%
\pgfsetlinewidth{0.250937pt}%
\definecolor{currentstroke}{rgb}{0.000000,0.000000,0.000000}%
\pgfsetstrokecolor{currentstroke}%
\pgfsetstrokeopacity{0.500000}%
\pgfsetdash{}{0pt}%
\pgfsys@defobject{currentmarker}{\pgfqpoint{-0.013889in}{-0.013889in}}{\pgfqpoint{0.013889in}{0.013889in}}{%
\pgfpathmoveto{\pgfqpoint{0.000000in}{-0.013889in}}%
\pgfpathcurveto{\pgfqpoint{0.003683in}{-0.013889in}}{\pgfqpoint{0.007216in}{-0.012425in}}{\pgfqpoint{0.009821in}{-0.009821in}}%
\pgfpathcurveto{\pgfqpoint{0.012425in}{-0.007216in}}{\pgfqpoint{0.013889in}{-0.003683in}}{\pgfqpoint{0.013889in}{0.000000in}}%
\pgfpathcurveto{\pgfqpoint{0.013889in}{0.003683in}}{\pgfqpoint{0.012425in}{0.007216in}}{\pgfqpoint{0.009821in}{0.009821in}}%
\pgfpathcurveto{\pgfqpoint{0.007216in}{0.012425in}}{\pgfqpoint{0.003683in}{0.013889in}}{\pgfqpoint{0.000000in}{0.013889in}}%
\pgfpathcurveto{\pgfqpoint{-0.003683in}{0.013889in}}{\pgfqpoint{-0.007216in}{0.012425in}}{\pgfqpoint{-0.009821in}{0.009821in}}%
\pgfpathcurveto{\pgfqpoint{-0.012425in}{0.007216in}}{\pgfqpoint{-0.013889in}{0.003683in}}{\pgfqpoint{-0.013889in}{0.000000in}}%
\pgfpathcurveto{\pgfqpoint{-0.013889in}{-0.003683in}}{\pgfqpoint{-0.012425in}{-0.007216in}}{\pgfqpoint{-0.009821in}{-0.009821in}}%
\pgfpathcurveto{\pgfqpoint{-0.007216in}{-0.012425in}}{\pgfqpoint{-0.003683in}{-0.013889in}}{\pgfqpoint{0.000000in}{-0.013889in}}%
\pgfpathclose%
\pgfusepath{stroke,fill}%
}%
\begin{pgfscope}%
\pgfsys@transformshift{3.754570in}{4.226123in}%
\pgfsys@useobject{currentmarker}{}%
\end{pgfscope}%
\end{pgfscope}%
\begin{pgfscope}%
\pgfpathrectangle{\pgfqpoint{0.100000in}{2.413063in}}{\pgfqpoint{5.037500in}{3.427208in}}%
\pgfusepath{clip}%
\pgfsetrectcap%
\pgfsetroundjoin%
\pgfsetlinewidth{1.505625pt}%
\definecolor{currentstroke}{rgb}{0.678431,1.000000,0.184314}%
\pgfsetstrokecolor{currentstroke}%
\pgfsetstrokeopacity{0.500000}%
\pgfsetdash{}{0pt}%
\pgfpathmoveto{\pgfqpoint{3.638327in}{4.237522in}}%
\pgfusepath{stroke}%
\end{pgfscope}%
\begin{pgfscope}%
\pgfpathrectangle{\pgfqpoint{0.100000in}{2.413063in}}{\pgfqpoint{5.037500in}{3.427208in}}%
\pgfusepath{clip}%
\pgfsetbuttcap%
\pgfsetroundjoin%
\definecolor{currentfill}{rgb}{0.678431,1.000000,0.184314}%
\pgfsetfillcolor{currentfill}%
\pgfsetfillopacity{0.500000}%
\pgfsetlinewidth{0.250937pt}%
\definecolor{currentstroke}{rgb}{0.000000,0.000000,0.000000}%
\pgfsetstrokecolor{currentstroke}%
\pgfsetstrokeopacity{0.500000}%
\pgfsetdash{}{0pt}%
\pgfsys@defobject{currentmarker}{\pgfqpoint{-0.005556in}{-0.005556in}}{\pgfqpoint{0.005556in}{0.005556in}}{%
\pgfpathmoveto{\pgfqpoint{0.000000in}{-0.005556in}}%
\pgfpathcurveto{\pgfqpoint{0.001473in}{-0.005556in}}{\pgfqpoint{0.002887in}{-0.004970in}}{\pgfqpoint{0.003928in}{-0.003928in}}%
\pgfpathcurveto{\pgfqpoint{0.004970in}{-0.002887in}}{\pgfqpoint{0.005556in}{-0.001473in}}{\pgfqpoint{0.005556in}{0.000000in}}%
\pgfpathcurveto{\pgfqpoint{0.005556in}{0.001473in}}{\pgfqpoint{0.004970in}{0.002887in}}{\pgfqpoint{0.003928in}{0.003928in}}%
\pgfpathcurveto{\pgfqpoint{0.002887in}{0.004970in}}{\pgfqpoint{0.001473in}{0.005556in}}{\pgfqpoint{0.000000in}{0.005556in}}%
\pgfpathcurveto{\pgfqpoint{-0.001473in}{0.005556in}}{\pgfqpoint{-0.002887in}{0.004970in}}{\pgfqpoint{-0.003928in}{0.003928in}}%
\pgfpathcurveto{\pgfqpoint{-0.004970in}{0.002887in}}{\pgfqpoint{-0.005556in}{0.001473in}}{\pgfqpoint{-0.005556in}{0.000000in}}%
\pgfpathcurveto{\pgfqpoint{-0.005556in}{-0.001473in}}{\pgfqpoint{-0.004970in}{-0.002887in}}{\pgfqpoint{-0.003928in}{-0.003928in}}%
\pgfpathcurveto{\pgfqpoint{-0.002887in}{-0.004970in}}{\pgfqpoint{-0.001473in}{-0.005556in}}{\pgfqpoint{0.000000in}{-0.005556in}}%
\pgfpathclose%
\pgfusepath{stroke,fill}%
}%
\begin{pgfscope}%
\pgfsys@transformshift{3.638327in}{4.237522in}%
\pgfsys@useobject{currentmarker}{}%
\end{pgfscope}%
\end{pgfscope}%
\begin{pgfscope}%
\pgfpathrectangle{\pgfqpoint{0.100000in}{2.413063in}}{\pgfqpoint{5.037500in}{3.427208in}}%
\pgfusepath{clip}%
\pgfsetrectcap%
\pgfsetroundjoin%
\pgfsetlinewidth{1.505625pt}%
\definecolor{currentstroke}{rgb}{0.678431,1.000000,0.184314}%
\pgfsetstrokecolor{currentstroke}%
\pgfsetstrokeopacity{0.500000}%
\pgfsetdash{}{0pt}%
\pgfpathmoveto{\pgfqpoint{3.521226in}{4.170926in}}%
\pgfusepath{stroke}%
\end{pgfscope}%
\begin{pgfscope}%
\pgfpathrectangle{\pgfqpoint{0.100000in}{2.413063in}}{\pgfqpoint{5.037500in}{3.427208in}}%
\pgfusepath{clip}%
\pgfsetbuttcap%
\pgfsetroundjoin%
\definecolor{currentfill}{rgb}{0.678431,1.000000,0.184314}%
\pgfsetfillcolor{currentfill}%
\pgfsetfillopacity{0.500000}%
\pgfsetlinewidth{0.250937pt}%
\definecolor{currentstroke}{rgb}{0.000000,0.000000,0.000000}%
\pgfsetstrokecolor{currentstroke}%
\pgfsetstrokeopacity{0.500000}%
\pgfsetdash{}{0pt}%
\pgfsys@defobject{currentmarker}{\pgfqpoint{-0.011111in}{-0.011111in}}{\pgfqpoint{0.011111in}{0.011111in}}{%
\pgfpathmoveto{\pgfqpoint{0.000000in}{-0.011111in}}%
\pgfpathcurveto{\pgfqpoint{0.002947in}{-0.011111in}}{\pgfqpoint{0.005773in}{-0.009940in}}{\pgfqpoint{0.007857in}{-0.007857in}}%
\pgfpathcurveto{\pgfqpoint{0.009940in}{-0.005773in}}{\pgfqpoint{0.011111in}{-0.002947in}}{\pgfqpoint{0.011111in}{0.000000in}}%
\pgfpathcurveto{\pgfqpoint{0.011111in}{0.002947in}}{\pgfqpoint{0.009940in}{0.005773in}}{\pgfqpoint{0.007857in}{0.007857in}}%
\pgfpathcurveto{\pgfqpoint{0.005773in}{0.009940in}}{\pgfqpoint{0.002947in}{0.011111in}}{\pgfqpoint{0.000000in}{0.011111in}}%
\pgfpathcurveto{\pgfqpoint{-0.002947in}{0.011111in}}{\pgfqpoint{-0.005773in}{0.009940in}}{\pgfqpoint{-0.007857in}{0.007857in}}%
\pgfpathcurveto{\pgfqpoint{-0.009940in}{0.005773in}}{\pgfqpoint{-0.011111in}{0.002947in}}{\pgfqpoint{-0.011111in}{0.000000in}}%
\pgfpathcurveto{\pgfqpoint{-0.011111in}{-0.002947in}}{\pgfqpoint{-0.009940in}{-0.005773in}}{\pgfqpoint{-0.007857in}{-0.007857in}}%
\pgfpathcurveto{\pgfqpoint{-0.005773in}{-0.009940in}}{\pgfqpoint{-0.002947in}{-0.011111in}}{\pgfqpoint{0.000000in}{-0.011111in}}%
\pgfpathclose%
\pgfusepath{stroke,fill}%
}%
\begin{pgfscope}%
\pgfsys@transformshift{3.521226in}{4.170926in}%
\pgfsys@useobject{currentmarker}{}%
\end{pgfscope}%
\end{pgfscope}%
\begin{pgfscope}%
\pgfpathrectangle{\pgfqpoint{0.100000in}{2.413063in}}{\pgfqpoint{5.037500in}{3.427208in}}%
\pgfusepath{clip}%
\pgfsetrectcap%
\pgfsetroundjoin%
\pgfsetlinewidth{1.505625pt}%
\definecolor{currentstroke}{rgb}{0.678431,1.000000,0.184314}%
\pgfsetstrokecolor{currentstroke}%
\pgfsetstrokeopacity{0.500000}%
\pgfsetdash{}{0pt}%
\pgfpathmoveto{\pgfqpoint{3.058052in}{3.383452in}}%
\pgfusepath{stroke}%
\end{pgfscope}%
\begin{pgfscope}%
\pgfpathrectangle{\pgfqpoint{0.100000in}{2.413063in}}{\pgfqpoint{5.037500in}{3.427208in}}%
\pgfusepath{clip}%
\pgfsetbuttcap%
\pgfsetroundjoin%
\definecolor{currentfill}{rgb}{0.678431,1.000000,0.184314}%
\pgfsetfillcolor{currentfill}%
\pgfsetfillopacity{0.500000}%
\pgfsetlinewidth{0.250937pt}%
\definecolor{currentstroke}{rgb}{0.000000,0.000000,0.000000}%
\pgfsetstrokecolor{currentstroke}%
\pgfsetstrokeopacity{0.500000}%
\pgfsetdash{}{0pt}%
\pgfsys@defobject{currentmarker}{\pgfqpoint{-0.058333in}{-0.058333in}}{\pgfqpoint{0.058333in}{0.058333in}}{%
\pgfpathmoveto{\pgfqpoint{0.000000in}{-0.058333in}}%
\pgfpathcurveto{\pgfqpoint{0.015470in}{-0.058333in}}{\pgfqpoint{0.030309in}{-0.052187in}}{\pgfqpoint{0.041248in}{-0.041248in}}%
\pgfpathcurveto{\pgfqpoint{0.052187in}{-0.030309in}}{\pgfqpoint{0.058333in}{-0.015470in}}{\pgfqpoint{0.058333in}{0.000000in}}%
\pgfpathcurveto{\pgfqpoint{0.058333in}{0.015470in}}{\pgfqpoint{0.052187in}{0.030309in}}{\pgfqpoint{0.041248in}{0.041248in}}%
\pgfpathcurveto{\pgfqpoint{0.030309in}{0.052187in}}{\pgfqpoint{0.015470in}{0.058333in}}{\pgfqpoint{0.000000in}{0.058333in}}%
\pgfpathcurveto{\pgfqpoint{-0.015470in}{0.058333in}}{\pgfqpoint{-0.030309in}{0.052187in}}{\pgfqpoint{-0.041248in}{0.041248in}}%
\pgfpathcurveto{\pgfqpoint{-0.052187in}{0.030309in}}{\pgfqpoint{-0.058333in}{0.015470in}}{\pgfqpoint{-0.058333in}{0.000000in}}%
\pgfpathcurveto{\pgfqpoint{-0.058333in}{-0.015470in}}{\pgfqpoint{-0.052187in}{-0.030309in}}{\pgfqpoint{-0.041248in}{-0.041248in}}%
\pgfpathcurveto{\pgfqpoint{-0.030309in}{-0.052187in}}{\pgfqpoint{-0.015470in}{-0.058333in}}{\pgfqpoint{0.000000in}{-0.058333in}}%
\pgfpathclose%
\pgfusepath{stroke,fill}%
}%
\begin{pgfscope}%
\pgfsys@transformshift{3.058052in}{3.383452in}%
\pgfsys@useobject{currentmarker}{}%
\end{pgfscope}%
\end{pgfscope}%
\begin{pgfscope}%
\pgfpathrectangle{\pgfqpoint{0.100000in}{2.413063in}}{\pgfqpoint{5.037500in}{3.427208in}}%
\pgfusepath{clip}%
\pgfsetrectcap%
\pgfsetroundjoin%
\pgfsetlinewidth{1.505625pt}%
\definecolor{currentstroke}{rgb}{0.678431,1.000000,0.184314}%
\pgfsetstrokecolor{currentstroke}%
\pgfsetstrokeopacity{0.500000}%
\pgfsetdash{}{0pt}%
\pgfpathmoveto{\pgfqpoint{3.189939in}{3.301015in}}%
\pgfusepath{stroke}%
\end{pgfscope}%
\begin{pgfscope}%
\pgfpathrectangle{\pgfqpoint{0.100000in}{2.413063in}}{\pgfqpoint{5.037500in}{3.427208in}}%
\pgfusepath{clip}%
\pgfsetbuttcap%
\pgfsetroundjoin%
\definecolor{currentfill}{rgb}{0.678431,1.000000,0.184314}%
\pgfsetfillcolor{currentfill}%
\pgfsetfillopacity{0.500000}%
\pgfsetlinewidth{0.250937pt}%
\definecolor{currentstroke}{rgb}{0.000000,0.000000,0.000000}%
\pgfsetstrokecolor{currentstroke}%
\pgfsetstrokeopacity{0.500000}%
\pgfsetdash{}{0pt}%
\pgfsys@defobject{currentmarker}{\pgfqpoint{-0.025000in}{-0.025000in}}{\pgfqpoint{0.025000in}{0.025000in}}{%
\pgfpathmoveto{\pgfqpoint{0.000000in}{-0.025000in}}%
\pgfpathcurveto{\pgfqpoint{0.006630in}{-0.025000in}}{\pgfqpoint{0.012989in}{-0.022366in}}{\pgfqpoint{0.017678in}{-0.017678in}}%
\pgfpathcurveto{\pgfqpoint{0.022366in}{-0.012989in}}{\pgfqpoint{0.025000in}{-0.006630in}}{\pgfqpoint{0.025000in}{0.000000in}}%
\pgfpathcurveto{\pgfqpoint{0.025000in}{0.006630in}}{\pgfqpoint{0.022366in}{0.012989in}}{\pgfqpoint{0.017678in}{0.017678in}}%
\pgfpathcurveto{\pgfqpoint{0.012989in}{0.022366in}}{\pgfqpoint{0.006630in}{0.025000in}}{\pgfqpoint{0.000000in}{0.025000in}}%
\pgfpathcurveto{\pgfqpoint{-0.006630in}{0.025000in}}{\pgfqpoint{-0.012989in}{0.022366in}}{\pgfqpoint{-0.017678in}{0.017678in}}%
\pgfpathcurveto{\pgfqpoint{-0.022366in}{0.012989in}}{\pgfqpoint{-0.025000in}{0.006630in}}{\pgfqpoint{-0.025000in}{0.000000in}}%
\pgfpathcurveto{\pgfqpoint{-0.025000in}{-0.006630in}}{\pgfqpoint{-0.022366in}{-0.012989in}}{\pgfqpoint{-0.017678in}{-0.017678in}}%
\pgfpathcurveto{\pgfqpoint{-0.012989in}{-0.022366in}}{\pgfqpoint{-0.006630in}{-0.025000in}}{\pgfqpoint{0.000000in}{-0.025000in}}%
\pgfpathclose%
\pgfusepath{stroke,fill}%
}%
\begin{pgfscope}%
\pgfsys@transformshift{3.189939in}{3.301015in}%
\pgfsys@useobject{currentmarker}{}%
\end{pgfscope}%
\end{pgfscope}%
\begin{pgfscope}%
\pgfpathrectangle{\pgfqpoint{0.100000in}{2.413063in}}{\pgfqpoint{5.037500in}{3.427208in}}%
\pgfusepath{clip}%
\pgfsetrectcap%
\pgfsetroundjoin%
\pgfsetlinewidth{1.505625pt}%
\definecolor{currentstroke}{rgb}{0.678431,1.000000,0.184314}%
\pgfsetstrokecolor{currentstroke}%
\pgfsetstrokeopacity{0.500000}%
\pgfsetdash{}{0pt}%
\pgfpathmoveto{\pgfqpoint{3.449980in}{4.605174in}}%
\pgfusepath{stroke}%
\end{pgfscope}%
\begin{pgfscope}%
\pgfpathrectangle{\pgfqpoint{0.100000in}{2.413063in}}{\pgfqpoint{5.037500in}{3.427208in}}%
\pgfusepath{clip}%
\pgfsetbuttcap%
\pgfsetroundjoin%
\definecolor{currentfill}{rgb}{0.678431,1.000000,0.184314}%
\pgfsetfillcolor{currentfill}%
\pgfsetfillopacity{0.500000}%
\pgfsetlinewidth{0.250937pt}%
\definecolor{currentstroke}{rgb}{0.000000,0.000000,0.000000}%
\pgfsetstrokecolor{currentstroke}%
\pgfsetstrokeopacity{0.500000}%
\pgfsetdash{}{0pt}%
\pgfsys@defobject{currentmarker}{\pgfqpoint{-0.016667in}{-0.016667in}}{\pgfqpoint{0.016667in}{0.016667in}}{%
\pgfpathmoveto{\pgfqpoint{0.000000in}{-0.016667in}}%
\pgfpathcurveto{\pgfqpoint{0.004420in}{-0.016667in}}{\pgfqpoint{0.008660in}{-0.014911in}}{\pgfqpoint{0.011785in}{-0.011785in}}%
\pgfpathcurveto{\pgfqpoint{0.014911in}{-0.008660in}}{\pgfqpoint{0.016667in}{-0.004420in}}{\pgfqpoint{0.016667in}{0.000000in}}%
\pgfpathcurveto{\pgfqpoint{0.016667in}{0.004420in}}{\pgfqpoint{0.014911in}{0.008660in}}{\pgfqpoint{0.011785in}{0.011785in}}%
\pgfpathcurveto{\pgfqpoint{0.008660in}{0.014911in}}{\pgfqpoint{0.004420in}{0.016667in}}{\pgfqpoint{0.000000in}{0.016667in}}%
\pgfpathcurveto{\pgfqpoint{-0.004420in}{0.016667in}}{\pgfqpoint{-0.008660in}{0.014911in}}{\pgfqpoint{-0.011785in}{0.011785in}}%
\pgfpathcurveto{\pgfqpoint{-0.014911in}{0.008660in}}{\pgfqpoint{-0.016667in}{0.004420in}}{\pgfqpoint{-0.016667in}{0.000000in}}%
\pgfpathcurveto{\pgfqpoint{-0.016667in}{-0.004420in}}{\pgfqpoint{-0.014911in}{-0.008660in}}{\pgfqpoint{-0.011785in}{-0.011785in}}%
\pgfpathcurveto{\pgfqpoint{-0.008660in}{-0.014911in}}{\pgfqpoint{-0.004420in}{-0.016667in}}{\pgfqpoint{0.000000in}{-0.016667in}}%
\pgfpathclose%
\pgfusepath{stroke,fill}%
}%
\begin{pgfscope}%
\pgfsys@transformshift{3.449980in}{4.605174in}%
\pgfsys@useobject{currentmarker}{}%
\end{pgfscope}%
\end{pgfscope}%
\begin{pgfscope}%
\pgfpathrectangle{\pgfqpoint{0.100000in}{2.413063in}}{\pgfqpoint{5.037500in}{3.427208in}}%
\pgfusepath{clip}%
\pgfsetrectcap%
\pgfsetroundjoin%
\pgfsetlinewidth{1.505625pt}%
\definecolor{currentstroke}{rgb}{0.678431,1.000000,0.184314}%
\pgfsetstrokecolor{currentstroke}%
\pgfsetstrokeopacity{0.500000}%
\pgfsetdash{}{0pt}%
\pgfpathmoveto{\pgfqpoint{3.238245in}{3.203433in}}%
\pgfusepath{stroke}%
\end{pgfscope}%
\begin{pgfscope}%
\pgfpathrectangle{\pgfqpoint{0.100000in}{2.413063in}}{\pgfqpoint{5.037500in}{3.427208in}}%
\pgfusepath{clip}%
\pgfsetbuttcap%
\pgfsetroundjoin%
\definecolor{currentfill}{rgb}{0.678431,1.000000,0.184314}%
\pgfsetfillcolor{currentfill}%
\pgfsetfillopacity{0.500000}%
\pgfsetlinewidth{0.250937pt}%
\definecolor{currentstroke}{rgb}{0.000000,0.000000,0.000000}%
\pgfsetstrokecolor{currentstroke}%
\pgfsetstrokeopacity{0.500000}%
\pgfsetdash{}{0pt}%
\pgfsys@defobject{currentmarker}{\pgfqpoint{-0.013889in}{-0.013889in}}{\pgfqpoint{0.013889in}{0.013889in}}{%
\pgfpathmoveto{\pgfqpoint{0.000000in}{-0.013889in}}%
\pgfpathcurveto{\pgfqpoint{0.003683in}{-0.013889in}}{\pgfqpoint{0.007216in}{-0.012425in}}{\pgfqpoint{0.009821in}{-0.009821in}}%
\pgfpathcurveto{\pgfqpoint{0.012425in}{-0.007216in}}{\pgfqpoint{0.013889in}{-0.003683in}}{\pgfqpoint{0.013889in}{0.000000in}}%
\pgfpathcurveto{\pgfqpoint{0.013889in}{0.003683in}}{\pgfqpoint{0.012425in}{0.007216in}}{\pgfqpoint{0.009821in}{0.009821in}}%
\pgfpathcurveto{\pgfqpoint{0.007216in}{0.012425in}}{\pgfqpoint{0.003683in}{0.013889in}}{\pgfqpoint{0.000000in}{0.013889in}}%
\pgfpathcurveto{\pgfqpoint{-0.003683in}{0.013889in}}{\pgfqpoint{-0.007216in}{0.012425in}}{\pgfqpoint{-0.009821in}{0.009821in}}%
\pgfpathcurveto{\pgfqpoint{-0.012425in}{0.007216in}}{\pgfqpoint{-0.013889in}{0.003683in}}{\pgfqpoint{-0.013889in}{0.000000in}}%
\pgfpathcurveto{\pgfqpoint{-0.013889in}{-0.003683in}}{\pgfqpoint{-0.012425in}{-0.007216in}}{\pgfqpoint{-0.009821in}{-0.009821in}}%
\pgfpathcurveto{\pgfqpoint{-0.007216in}{-0.012425in}}{\pgfqpoint{-0.003683in}{-0.013889in}}{\pgfqpoint{0.000000in}{-0.013889in}}%
\pgfpathclose%
\pgfusepath{stroke,fill}%
}%
\begin{pgfscope}%
\pgfsys@transformshift{3.238245in}{3.203433in}%
\pgfsys@useobject{currentmarker}{}%
\end{pgfscope}%
\end{pgfscope}%
\begin{pgfscope}%
\pgfpathrectangle{\pgfqpoint{0.100000in}{2.413063in}}{\pgfqpoint{5.037500in}{3.427208in}}%
\pgfusepath{clip}%
\pgfsetrectcap%
\pgfsetroundjoin%
\pgfsetlinewidth{1.505625pt}%
\definecolor{currentstroke}{rgb}{0.678431,1.000000,0.184314}%
\pgfsetstrokecolor{currentstroke}%
\pgfsetstrokeopacity{0.500000}%
\pgfsetdash{}{0pt}%
\pgfpathmoveto{\pgfqpoint{3.104026in}{3.271375in}}%
\pgfusepath{stroke}%
\end{pgfscope}%
\begin{pgfscope}%
\pgfpathrectangle{\pgfqpoint{0.100000in}{2.413063in}}{\pgfqpoint{5.037500in}{3.427208in}}%
\pgfusepath{clip}%
\pgfsetbuttcap%
\pgfsetroundjoin%
\definecolor{currentfill}{rgb}{0.678431,1.000000,0.184314}%
\pgfsetfillcolor{currentfill}%
\pgfsetfillopacity{0.500000}%
\pgfsetlinewidth{0.250937pt}%
\definecolor{currentstroke}{rgb}{0.000000,0.000000,0.000000}%
\pgfsetstrokecolor{currentstroke}%
\pgfsetstrokeopacity{0.500000}%
\pgfsetdash{}{0pt}%
\pgfsys@defobject{currentmarker}{\pgfqpoint{-0.030556in}{-0.030556in}}{\pgfqpoint{0.030556in}{0.030556in}}{%
\pgfpathmoveto{\pgfqpoint{0.000000in}{-0.030556in}}%
\pgfpathcurveto{\pgfqpoint{0.008103in}{-0.030556in}}{\pgfqpoint{0.015876in}{-0.027336in}}{\pgfqpoint{0.021606in}{-0.021606in}}%
\pgfpathcurveto{\pgfqpoint{0.027336in}{-0.015876in}}{\pgfqpoint{0.030556in}{-0.008103in}}{\pgfqpoint{0.030556in}{0.000000in}}%
\pgfpathcurveto{\pgfqpoint{0.030556in}{0.008103in}}{\pgfqpoint{0.027336in}{0.015876in}}{\pgfqpoint{0.021606in}{0.021606in}}%
\pgfpathcurveto{\pgfqpoint{0.015876in}{0.027336in}}{\pgfqpoint{0.008103in}{0.030556in}}{\pgfqpoint{0.000000in}{0.030556in}}%
\pgfpathcurveto{\pgfqpoint{-0.008103in}{0.030556in}}{\pgfqpoint{-0.015876in}{0.027336in}}{\pgfqpoint{-0.021606in}{0.021606in}}%
\pgfpathcurveto{\pgfqpoint{-0.027336in}{0.015876in}}{\pgfqpoint{-0.030556in}{0.008103in}}{\pgfqpoint{-0.030556in}{0.000000in}}%
\pgfpathcurveto{\pgfqpoint{-0.030556in}{-0.008103in}}{\pgfqpoint{-0.027336in}{-0.015876in}}{\pgfqpoint{-0.021606in}{-0.021606in}}%
\pgfpathcurveto{\pgfqpoint{-0.015876in}{-0.027336in}}{\pgfqpoint{-0.008103in}{-0.030556in}}{\pgfqpoint{0.000000in}{-0.030556in}}%
\pgfpathclose%
\pgfusepath{stroke,fill}%
}%
\begin{pgfscope}%
\pgfsys@transformshift{3.104026in}{3.271375in}%
\pgfsys@useobject{currentmarker}{}%
\end{pgfscope}%
\end{pgfscope}%
\begin{pgfscope}%
\pgfpathrectangle{\pgfqpoint{0.100000in}{2.413063in}}{\pgfqpoint{5.037500in}{3.427208in}}%
\pgfusepath{clip}%
\pgfsetrectcap%
\pgfsetroundjoin%
\pgfsetlinewidth{1.505625pt}%
\definecolor{currentstroke}{rgb}{0.678431,1.000000,0.184314}%
\pgfsetstrokecolor{currentstroke}%
\pgfsetstrokeopacity{0.500000}%
\pgfsetdash{}{0pt}%
\pgfpathmoveto{\pgfqpoint{2.983531in}{3.268508in}}%
\pgfusepath{stroke}%
\end{pgfscope}%
\begin{pgfscope}%
\pgfpathrectangle{\pgfqpoint{0.100000in}{2.413063in}}{\pgfqpoint{5.037500in}{3.427208in}}%
\pgfusepath{clip}%
\pgfsetbuttcap%
\pgfsetroundjoin%
\definecolor{currentfill}{rgb}{0.678431,1.000000,0.184314}%
\pgfsetfillcolor{currentfill}%
\pgfsetfillopacity{0.500000}%
\pgfsetlinewidth{0.250937pt}%
\definecolor{currentstroke}{rgb}{0.000000,0.000000,0.000000}%
\pgfsetstrokecolor{currentstroke}%
\pgfsetstrokeopacity{0.500000}%
\pgfsetdash{}{0pt}%
\pgfsys@defobject{currentmarker}{\pgfqpoint{-0.008333in}{-0.008333in}}{\pgfqpoint{0.008333in}{0.008333in}}{%
\pgfpathmoveto{\pgfqpoint{0.000000in}{-0.008333in}}%
\pgfpathcurveto{\pgfqpoint{0.002210in}{-0.008333in}}{\pgfqpoint{0.004330in}{-0.007455in}}{\pgfqpoint{0.005893in}{-0.005893in}}%
\pgfpathcurveto{\pgfqpoint{0.007455in}{-0.004330in}}{\pgfqpoint{0.008333in}{-0.002210in}}{\pgfqpoint{0.008333in}{0.000000in}}%
\pgfpathcurveto{\pgfqpoint{0.008333in}{0.002210in}}{\pgfqpoint{0.007455in}{0.004330in}}{\pgfqpoint{0.005893in}{0.005893in}}%
\pgfpathcurveto{\pgfqpoint{0.004330in}{0.007455in}}{\pgfqpoint{0.002210in}{0.008333in}}{\pgfqpoint{0.000000in}{0.008333in}}%
\pgfpathcurveto{\pgfqpoint{-0.002210in}{0.008333in}}{\pgfqpoint{-0.004330in}{0.007455in}}{\pgfqpoint{-0.005893in}{0.005893in}}%
\pgfpathcurveto{\pgfqpoint{-0.007455in}{0.004330in}}{\pgfqpoint{-0.008333in}{0.002210in}}{\pgfqpoint{-0.008333in}{0.000000in}}%
\pgfpathcurveto{\pgfqpoint{-0.008333in}{-0.002210in}}{\pgfqpoint{-0.007455in}{-0.004330in}}{\pgfqpoint{-0.005893in}{-0.005893in}}%
\pgfpathcurveto{\pgfqpoint{-0.004330in}{-0.007455in}}{\pgfqpoint{-0.002210in}{-0.008333in}}{\pgfqpoint{0.000000in}{-0.008333in}}%
\pgfpathclose%
\pgfusepath{stroke,fill}%
}%
\begin{pgfscope}%
\pgfsys@transformshift{2.983531in}{3.268508in}%
\pgfsys@useobject{currentmarker}{}%
\end{pgfscope}%
\end{pgfscope}%
\begin{pgfscope}%
\pgfpathrectangle{\pgfqpoint{0.100000in}{2.413063in}}{\pgfqpoint{5.037500in}{3.427208in}}%
\pgfusepath{clip}%
\pgfsetrectcap%
\pgfsetroundjoin%
\pgfsetlinewidth{1.505625pt}%
\definecolor{currentstroke}{rgb}{0.678431,1.000000,0.184314}%
\pgfsetstrokecolor{currentstroke}%
\pgfsetstrokeopacity{0.500000}%
\pgfsetdash{}{0pt}%
\pgfpathmoveto{\pgfqpoint{3.239760in}{4.208703in}}%
\pgfusepath{stroke}%
\end{pgfscope}%
\begin{pgfscope}%
\pgfpathrectangle{\pgfqpoint{0.100000in}{2.413063in}}{\pgfqpoint{5.037500in}{3.427208in}}%
\pgfusepath{clip}%
\pgfsetbuttcap%
\pgfsetroundjoin%
\definecolor{currentfill}{rgb}{0.678431,1.000000,0.184314}%
\pgfsetfillcolor{currentfill}%
\pgfsetfillopacity{0.500000}%
\pgfsetlinewidth{0.250937pt}%
\definecolor{currentstroke}{rgb}{0.000000,0.000000,0.000000}%
\pgfsetstrokecolor{currentstroke}%
\pgfsetstrokeopacity{0.500000}%
\pgfsetdash{}{0pt}%
\pgfsys@defobject{currentmarker}{\pgfqpoint{-0.050000in}{-0.050000in}}{\pgfqpoint{0.050000in}{0.050000in}}{%
\pgfpathmoveto{\pgfqpoint{0.000000in}{-0.050000in}}%
\pgfpathcurveto{\pgfqpoint{0.013260in}{-0.050000in}}{\pgfqpoint{0.025979in}{-0.044732in}}{\pgfqpoint{0.035355in}{-0.035355in}}%
\pgfpathcurveto{\pgfqpoint{0.044732in}{-0.025979in}}{\pgfqpoint{0.050000in}{-0.013260in}}{\pgfqpoint{0.050000in}{0.000000in}}%
\pgfpathcurveto{\pgfqpoint{0.050000in}{0.013260in}}{\pgfqpoint{0.044732in}{0.025979in}}{\pgfqpoint{0.035355in}{0.035355in}}%
\pgfpathcurveto{\pgfqpoint{0.025979in}{0.044732in}}{\pgfqpoint{0.013260in}{0.050000in}}{\pgfqpoint{0.000000in}{0.050000in}}%
\pgfpathcurveto{\pgfqpoint{-0.013260in}{0.050000in}}{\pgfqpoint{-0.025979in}{0.044732in}}{\pgfqpoint{-0.035355in}{0.035355in}}%
\pgfpathcurveto{\pgfqpoint{-0.044732in}{0.025979in}}{\pgfqpoint{-0.050000in}{0.013260in}}{\pgfqpoint{-0.050000in}{0.000000in}}%
\pgfpathcurveto{\pgfqpoint{-0.050000in}{-0.013260in}}{\pgfqpoint{-0.044732in}{-0.025979in}}{\pgfqpoint{-0.035355in}{-0.035355in}}%
\pgfpathcurveto{\pgfqpoint{-0.025979in}{-0.044732in}}{\pgfqpoint{-0.013260in}{-0.050000in}}{\pgfqpoint{0.000000in}{-0.050000in}}%
\pgfpathclose%
\pgfusepath{stroke,fill}%
}%
\begin{pgfscope}%
\pgfsys@transformshift{3.239760in}{4.208703in}%
\pgfsys@useobject{currentmarker}{}%
\end{pgfscope}%
\end{pgfscope}%
\begin{pgfscope}%
\pgfpathrectangle{\pgfqpoint{0.100000in}{2.413063in}}{\pgfqpoint{5.037500in}{3.427208in}}%
\pgfusepath{clip}%
\pgfsetrectcap%
\pgfsetroundjoin%
\pgfsetlinewidth{1.505625pt}%
\definecolor{currentstroke}{rgb}{0.000000,0.000000,1.000000}%
\pgfsetstrokecolor{currentstroke}%
\pgfsetstrokeopacity{0.500000}%
\pgfsetdash{}{0pt}%
\pgfpathmoveto{\pgfqpoint{3.301808in}{3.248013in}}%
\pgfusepath{stroke}%
\end{pgfscope}%
\begin{pgfscope}%
\pgfpathrectangle{\pgfqpoint{0.100000in}{2.413063in}}{\pgfqpoint{5.037500in}{3.427208in}}%
\pgfusepath{clip}%
\pgfsetbuttcap%
\pgfsetroundjoin%
\definecolor{currentfill}{rgb}{0.000000,0.000000,1.000000}%
\pgfsetfillcolor{currentfill}%
\pgfsetfillopacity{0.500000}%
\pgfsetlinewidth{0.250937pt}%
\definecolor{currentstroke}{rgb}{0.000000,0.000000,0.000000}%
\pgfsetstrokecolor{currentstroke}%
\pgfsetstrokeopacity{0.500000}%
\pgfsetdash{}{0pt}%
\pgfsys@defobject{currentmarker}{\pgfqpoint{-0.008333in}{-0.008333in}}{\pgfqpoint{0.008333in}{0.008333in}}{%
\pgfpathmoveto{\pgfqpoint{0.000000in}{-0.008333in}}%
\pgfpathcurveto{\pgfqpoint{0.002210in}{-0.008333in}}{\pgfqpoint{0.004330in}{-0.007455in}}{\pgfqpoint{0.005893in}{-0.005893in}}%
\pgfpathcurveto{\pgfqpoint{0.007455in}{-0.004330in}}{\pgfqpoint{0.008333in}{-0.002210in}}{\pgfqpoint{0.008333in}{0.000000in}}%
\pgfpathcurveto{\pgfqpoint{0.008333in}{0.002210in}}{\pgfqpoint{0.007455in}{0.004330in}}{\pgfqpoint{0.005893in}{0.005893in}}%
\pgfpathcurveto{\pgfqpoint{0.004330in}{0.007455in}}{\pgfqpoint{0.002210in}{0.008333in}}{\pgfqpoint{0.000000in}{0.008333in}}%
\pgfpathcurveto{\pgfqpoint{-0.002210in}{0.008333in}}{\pgfqpoint{-0.004330in}{0.007455in}}{\pgfqpoint{-0.005893in}{0.005893in}}%
\pgfpathcurveto{\pgfqpoint{-0.007455in}{0.004330in}}{\pgfqpoint{-0.008333in}{0.002210in}}{\pgfqpoint{-0.008333in}{0.000000in}}%
\pgfpathcurveto{\pgfqpoint{-0.008333in}{-0.002210in}}{\pgfqpoint{-0.007455in}{-0.004330in}}{\pgfqpoint{-0.005893in}{-0.005893in}}%
\pgfpathcurveto{\pgfqpoint{-0.004330in}{-0.007455in}}{\pgfqpoint{-0.002210in}{-0.008333in}}{\pgfqpoint{0.000000in}{-0.008333in}}%
\pgfpathclose%
\pgfusepath{stroke,fill}%
}%
\begin{pgfscope}%
\pgfsys@transformshift{3.301808in}{3.248013in}%
\pgfsys@useobject{currentmarker}{}%
\end{pgfscope}%
\end{pgfscope}%
\begin{pgfscope}%
\pgfpathrectangle{\pgfqpoint{0.100000in}{2.413063in}}{\pgfqpoint{5.037500in}{3.427208in}}%
\pgfusepath{clip}%
\pgfsetrectcap%
\pgfsetroundjoin%
\pgfsetlinewidth{1.505625pt}%
\definecolor{currentstroke}{rgb}{0.678431,1.000000,0.184314}%
\pgfsetstrokecolor{currentstroke}%
\pgfsetstrokeopacity{0.500000}%
\pgfsetdash{}{0pt}%
\pgfpathmoveto{\pgfqpoint{2.924780in}{3.534245in}}%
\pgfusepath{stroke}%
\end{pgfscope}%
\begin{pgfscope}%
\pgfpathrectangle{\pgfqpoint{0.100000in}{2.413063in}}{\pgfqpoint{5.037500in}{3.427208in}}%
\pgfusepath{clip}%
\pgfsetbuttcap%
\pgfsetroundjoin%
\definecolor{currentfill}{rgb}{0.678431,1.000000,0.184314}%
\pgfsetfillcolor{currentfill}%
\pgfsetfillopacity{0.500000}%
\pgfsetlinewidth{0.250937pt}%
\definecolor{currentstroke}{rgb}{0.000000,0.000000,0.000000}%
\pgfsetstrokecolor{currentstroke}%
\pgfsetstrokeopacity{0.500000}%
\pgfsetdash{}{0pt}%
\pgfsys@defobject{currentmarker}{\pgfqpoint{-0.030556in}{-0.030556in}}{\pgfqpoint{0.030556in}{0.030556in}}{%
\pgfpathmoveto{\pgfqpoint{0.000000in}{-0.030556in}}%
\pgfpathcurveto{\pgfqpoint{0.008103in}{-0.030556in}}{\pgfqpoint{0.015876in}{-0.027336in}}{\pgfqpoint{0.021606in}{-0.021606in}}%
\pgfpathcurveto{\pgfqpoint{0.027336in}{-0.015876in}}{\pgfqpoint{0.030556in}{-0.008103in}}{\pgfqpoint{0.030556in}{0.000000in}}%
\pgfpathcurveto{\pgfqpoint{0.030556in}{0.008103in}}{\pgfqpoint{0.027336in}{0.015876in}}{\pgfqpoint{0.021606in}{0.021606in}}%
\pgfpathcurveto{\pgfqpoint{0.015876in}{0.027336in}}{\pgfqpoint{0.008103in}{0.030556in}}{\pgfqpoint{0.000000in}{0.030556in}}%
\pgfpathcurveto{\pgfqpoint{-0.008103in}{0.030556in}}{\pgfqpoint{-0.015876in}{0.027336in}}{\pgfqpoint{-0.021606in}{0.021606in}}%
\pgfpathcurveto{\pgfqpoint{-0.027336in}{0.015876in}}{\pgfqpoint{-0.030556in}{0.008103in}}{\pgfqpoint{-0.030556in}{0.000000in}}%
\pgfpathcurveto{\pgfqpoint{-0.030556in}{-0.008103in}}{\pgfqpoint{-0.027336in}{-0.015876in}}{\pgfqpoint{-0.021606in}{-0.021606in}}%
\pgfpathcurveto{\pgfqpoint{-0.015876in}{-0.027336in}}{\pgfqpoint{-0.008103in}{-0.030556in}}{\pgfqpoint{0.000000in}{-0.030556in}}%
\pgfpathclose%
\pgfusepath{stroke,fill}%
}%
\begin{pgfscope}%
\pgfsys@transformshift{2.924780in}{3.534245in}%
\pgfsys@useobject{currentmarker}{}%
\end{pgfscope}%
\end{pgfscope}%
\begin{pgfscope}%
\pgfpathrectangle{\pgfqpoint{0.100000in}{2.413063in}}{\pgfqpoint{5.037500in}{3.427208in}}%
\pgfusepath{clip}%
\pgfsetrectcap%
\pgfsetroundjoin%
\pgfsetlinewidth{1.505625pt}%
\definecolor{currentstroke}{rgb}{0.000000,0.000000,1.000000}%
\pgfsetstrokecolor{currentstroke}%
\pgfsetstrokeopacity{0.500000}%
\pgfsetdash{}{0pt}%
\pgfpathmoveto{\pgfqpoint{4.927826in}{5.258593in}}%
\pgfusepath{stroke}%
\end{pgfscope}%
\begin{pgfscope}%
\pgfpathrectangle{\pgfqpoint{0.100000in}{2.413063in}}{\pgfqpoint{5.037500in}{3.427208in}}%
\pgfusepath{clip}%
\pgfsetbuttcap%
\pgfsetroundjoin%
\definecolor{currentfill}{rgb}{0.000000,0.000000,1.000000}%
\pgfsetfillcolor{currentfill}%
\pgfsetfillopacity{0.500000}%
\pgfsetlinewidth{0.250937pt}%
\definecolor{currentstroke}{rgb}{0.000000,0.000000,0.000000}%
\pgfsetstrokecolor{currentstroke}%
\pgfsetstrokeopacity{0.500000}%
\pgfsetdash{}{0pt}%
\pgfsys@defobject{currentmarker}{\pgfqpoint{-0.041667in}{-0.041667in}}{\pgfqpoint{0.041667in}{0.041667in}}{%
\pgfpathmoveto{\pgfqpoint{0.000000in}{-0.041667in}}%
\pgfpathcurveto{\pgfqpoint{0.011050in}{-0.041667in}}{\pgfqpoint{0.021649in}{-0.037276in}}{\pgfqpoint{0.029463in}{-0.029463in}}%
\pgfpathcurveto{\pgfqpoint{0.037276in}{-0.021649in}}{\pgfqpoint{0.041667in}{-0.011050in}}{\pgfqpoint{0.041667in}{0.000000in}}%
\pgfpathcurveto{\pgfqpoint{0.041667in}{0.011050in}}{\pgfqpoint{0.037276in}{0.021649in}}{\pgfqpoint{0.029463in}{0.029463in}}%
\pgfpathcurveto{\pgfqpoint{0.021649in}{0.037276in}}{\pgfqpoint{0.011050in}{0.041667in}}{\pgfqpoint{0.000000in}{0.041667in}}%
\pgfpathcurveto{\pgfqpoint{-0.011050in}{0.041667in}}{\pgfqpoint{-0.021649in}{0.037276in}}{\pgfqpoint{-0.029463in}{0.029463in}}%
\pgfpathcurveto{\pgfqpoint{-0.037276in}{0.021649in}}{\pgfqpoint{-0.041667in}{0.011050in}}{\pgfqpoint{-0.041667in}{0.000000in}}%
\pgfpathcurveto{\pgfqpoint{-0.041667in}{-0.011050in}}{\pgfqpoint{-0.037276in}{-0.021649in}}{\pgfqpoint{-0.029463in}{-0.029463in}}%
\pgfpathcurveto{\pgfqpoint{-0.021649in}{-0.037276in}}{\pgfqpoint{-0.011050in}{-0.041667in}}{\pgfqpoint{0.000000in}{-0.041667in}}%
\pgfpathclose%
\pgfusepath{stroke,fill}%
}%
\begin{pgfscope}%
\pgfsys@transformshift{4.927826in}{5.258593in}%
\pgfsys@useobject{currentmarker}{}%
\end{pgfscope}%
\end{pgfscope}%
\begin{pgfscope}%
\pgfpathrectangle{\pgfqpoint{0.100000in}{2.413063in}}{\pgfqpoint{5.037500in}{3.427208in}}%
\pgfusepath{clip}%
\pgfsetrectcap%
\pgfsetroundjoin%
\pgfsetlinewidth{1.505625pt}%
\definecolor{currentstroke}{rgb}{0.000000,0.000000,1.000000}%
\pgfsetstrokecolor{currentstroke}%
\pgfsetstrokeopacity{0.500000}%
\pgfsetdash{}{0pt}%
\pgfpathmoveto{\pgfqpoint{4.836271in}{5.147841in}}%
\pgfusepath{stroke}%
\end{pgfscope}%
\begin{pgfscope}%
\pgfpathrectangle{\pgfqpoint{0.100000in}{2.413063in}}{\pgfqpoint{5.037500in}{3.427208in}}%
\pgfusepath{clip}%
\pgfsetbuttcap%
\pgfsetroundjoin%
\definecolor{currentfill}{rgb}{0.000000,0.000000,1.000000}%
\pgfsetfillcolor{currentfill}%
\pgfsetfillopacity{0.500000}%
\pgfsetlinewidth{0.250937pt}%
\definecolor{currentstroke}{rgb}{0.000000,0.000000,0.000000}%
\pgfsetstrokecolor{currentstroke}%
\pgfsetstrokeopacity{0.500000}%
\pgfsetdash{}{0pt}%
\pgfsys@defobject{currentmarker}{\pgfqpoint{-0.050000in}{-0.050000in}}{\pgfqpoint{0.050000in}{0.050000in}}{%
\pgfpathmoveto{\pgfqpoint{0.000000in}{-0.050000in}}%
\pgfpathcurveto{\pgfqpoint{0.013260in}{-0.050000in}}{\pgfqpoint{0.025979in}{-0.044732in}}{\pgfqpoint{0.035355in}{-0.035355in}}%
\pgfpathcurveto{\pgfqpoint{0.044732in}{-0.025979in}}{\pgfqpoint{0.050000in}{-0.013260in}}{\pgfqpoint{0.050000in}{0.000000in}}%
\pgfpathcurveto{\pgfqpoint{0.050000in}{0.013260in}}{\pgfqpoint{0.044732in}{0.025979in}}{\pgfqpoint{0.035355in}{0.035355in}}%
\pgfpathcurveto{\pgfqpoint{0.025979in}{0.044732in}}{\pgfqpoint{0.013260in}{0.050000in}}{\pgfqpoint{0.000000in}{0.050000in}}%
\pgfpathcurveto{\pgfqpoint{-0.013260in}{0.050000in}}{\pgfqpoint{-0.025979in}{0.044732in}}{\pgfqpoint{-0.035355in}{0.035355in}}%
\pgfpathcurveto{\pgfqpoint{-0.044732in}{0.025979in}}{\pgfqpoint{-0.050000in}{0.013260in}}{\pgfqpoint{-0.050000in}{0.000000in}}%
\pgfpathcurveto{\pgfqpoint{-0.050000in}{-0.013260in}}{\pgfqpoint{-0.044732in}{-0.025979in}}{\pgfqpoint{-0.035355in}{-0.035355in}}%
\pgfpathcurveto{\pgfqpoint{-0.025979in}{-0.044732in}}{\pgfqpoint{-0.013260in}{-0.050000in}}{\pgfqpoint{0.000000in}{-0.050000in}}%
\pgfpathclose%
\pgfusepath{stroke,fill}%
}%
\begin{pgfscope}%
\pgfsys@transformshift{4.836271in}{5.147841in}%
\pgfsys@useobject{currentmarker}{}%
\end{pgfscope}%
\end{pgfscope}%
\begin{pgfscope}%
\pgfpathrectangle{\pgfqpoint{0.100000in}{2.413063in}}{\pgfqpoint{5.037500in}{3.427208in}}%
\pgfusepath{clip}%
\pgfsetrectcap%
\pgfsetroundjoin%
\pgfsetlinewidth{1.505625pt}%
\definecolor{currentstroke}{rgb}{0.000000,0.000000,1.000000}%
\pgfsetstrokecolor{currentstroke}%
\pgfsetstrokeopacity{0.500000}%
\pgfsetdash{}{0pt}%
\pgfpathmoveto{\pgfqpoint{4.846720in}{5.098047in}}%
\pgfusepath{stroke}%
\end{pgfscope}%
\begin{pgfscope}%
\pgfpathrectangle{\pgfqpoint{0.100000in}{2.413063in}}{\pgfqpoint{5.037500in}{3.427208in}}%
\pgfusepath{clip}%
\pgfsetbuttcap%
\pgfsetroundjoin%
\definecolor{currentfill}{rgb}{0.000000,0.000000,1.000000}%
\pgfsetfillcolor{currentfill}%
\pgfsetfillopacity{0.500000}%
\pgfsetlinewidth{0.250937pt}%
\definecolor{currentstroke}{rgb}{0.000000,0.000000,0.000000}%
\pgfsetstrokecolor{currentstroke}%
\pgfsetstrokeopacity{0.500000}%
\pgfsetdash{}{0pt}%
\pgfsys@defobject{currentmarker}{\pgfqpoint{-0.041667in}{-0.041667in}}{\pgfqpoint{0.041667in}{0.041667in}}{%
\pgfpathmoveto{\pgfqpoint{0.000000in}{-0.041667in}}%
\pgfpathcurveto{\pgfqpoint{0.011050in}{-0.041667in}}{\pgfqpoint{0.021649in}{-0.037276in}}{\pgfqpoint{0.029463in}{-0.029463in}}%
\pgfpathcurveto{\pgfqpoint{0.037276in}{-0.021649in}}{\pgfqpoint{0.041667in}{-0.011050in}}{\pgfqpoint{0.041667in}{0.000000in}}%
\pgfpathcurveto{\pgfqpoint{0.041667in}{0.011050in}}{\pgfqpoint{0.037276in}{0.021649in}}{\pgfqpoint{0.029463in}{0.029463in}}%
\pgfpathcurveto{\pgfqpoint{0.021649in}{0.037276in}}{\pgfqpoint{0.011050in}{0.041667in}}{\pgfqpoint{0.000000in}{0.041667in}}%
\pgfpathcurveto{\pgfqpoint{-0.011050in}{0.041667in}}{\pgfqpoint{-0.021649in}{0.037276in}}{\pgfqpoint{-0.029463in}{0.029463in}}%
\pgfpathcurveto{\pgfqpoint{-0.037276in}{0.021649in}}{\pgfqpoint{-0.041667in}{0.011050in}}{\pgfqpoint{-0.041667in}{0.000000in}}%
\pgfpathcurveto{\pgfqpoint{-0.041667in}{-0.011050in}}{\pgfqpoint{-0.037276in}{-0.021649in}}{\pgfqpoint{-0.029463in}{-0.029463in}}%
\pgfpathcurveto{\pgfqpoint{-0.021649in}{-0.037276in}}{\pgfqpoint{-0.011050in}{-0.041667in}}{\pgfqpoint{0.000000in}{-0.041667in}}%
\pgfpathclose%
\pgfusepath{stroke,fill}%
}%
\begin{pgfscope}%
\pgfsys@transformshift{4.846720in}{5.098047in}%
\pgfsys@useobject{currentmarker}{}%
\end{pgfscope}%
\end{pgfscope}%
\begin{pgfscope}%
\pgfpathrectangle{\pgfqpoint{0.100000in}{2.413063in}}{\pgfqpoint{5.037500in}{3.427208in}}%
\pgfusepath{clip}%
\pgfsetrectcap%
\pgfsetroundjoin%
\pgfsetlinewidth{1.505625pt}%
\definecolor{currentstroke}{rgb}{0.000000,0.000000,1.000000}%
\pgfsetstrokecolor{currentstroke}%
\pgfsetstrokeopacity{0.500000}%
\pgfsetdash{}{0pt}%
\pgfpathmoveto{\pgfqpoint{4.431773in}{4.479688in}}%
\pgfusepath{stroke}%
\end{pgfscope}%
\begin{pgfscope}%
\pgfpathrectangle{\pgfqpoint{0.100000in}{2.413063in}}{\pgfqpoint{5.037500in}{3.427208in}}%
\pgfusepath{clip}%
\pgfsetbuttcap%
\pgfsetroundjoin%
\definecolor{currentfill}{rgb}{0.000000,0.000000,1.000000}%
\pgfsetfillcolor{currentfill}%
\pgfsetfillopacity{0.500000}%
\pgfsetlinewidth{0.250937pt}%
\definecolor{currentstroke}{rgb}{0.000000,0.000000,0.000000}%
\pgfsetstrokecolor{currentstroke}%
\pgfsetstrokeopacity{0.500000}%
\pgfsetdash{}{0pt}%
\pgfsys@defobject{currentmarker}{\pgfqpoint{-0.022222in}{-0.022222in}}{\pgfqpoint{0.022222in}{0.022222in}}{%
\pgfpathmoveto{\pgfqpoint{0.000000in}{-0.022222in}}%
\pgfpathcurveto{\pgfqpoint{0.005893in}{-0.022222in}}{\pgfqpoint{0.011546in}{-0.019881in}}{\pgfqpoint{0.015713in}{-0.015713in}}%
\pgfpathcurveto{\pgfqpoint{0.019881in}{-0.011546in}}{\pgfqpoint{0.022222in}{-0.005893in}}{\pgfqpoint{0.022222in}{0.000000in}}%
\pgfpathcurveto{\pgfqpoint{0.022222in}{0.005893in}}{\pgfqpoint{0.019881in}{0.011546in}}{\pgfqpoint{0.015713in}{0.015713in}}%
\pgfpathcurveto{\pgfqpoint{0.011546in}{0.019881in}}{\pgfqpoint{0.005893in}{0.022222in}}{\pgfqpoint{0.000000in}{0.022222in}}%
\pgfpathcurveto{\pgfqpoint{-0.005893in}{0.022222in}}{\pgfqpoint{-0.011546in}{0.019881in}}{\pgfqpoint{-0.015713in}{0.015713in}}%
\pgfpathcurveto{\pgfqpoint{-0.019881in}{0.011546in}}{\pgfqpoint{-0.022222in}{0.005893in}}{\pgfqpoint{-0.022222in}{0.000000in}}%
\pgfpathcurveto{\pgfqpoint{-0.022222in}{-0.005893in}}{\pgfqpoint{-0.019881in}{-0.011546in}}{\pgfqpoint{-0.015713in}{-0.015713in}}%
\pgfpathcurveto{\pgfqpoint{-0.011546in}{-0.019881in}}{\pgfqpoint{-0.005893in}{-0.022222in}}{\pgfqpoint{0.000000in}{-0.022222in}}%
\pgfpathclose%
\pgfusepath{stroke,fill}%
}%
\begin{pgfscope}%
\pgfsys@transformshift{4.431773in}{4.479688in}%
\pgfsys@useobject{currentmarker}{}%
\end{pgfscope}%
\end{pgfscope}%
\begin{pgfscope}%
\pgfpathrectangle{\pgfqpoint{0.100000in}{2.413063in}}{\pgfqpoint{5.037500in}{3.427208in}}%
\pgfusepath{clip}%
\pgfsetrectcap%
\pgfsetroundjoin%
\pgfsetlinewidth{1.505625pt}%
\definecolor{currentstroke}{rgb}{0.000000,0.000000,1.000000}%
\pgfsetstrokecolor{currentstroke}%
\pgfsetstrokeopacity{0.500000}%
\pgfsetdash{}{0pt}%
\pgfpathmoveto{\pgfqpoint{4.463839in}{4.369843in}}%
\pgfusepath{stroke}%
\end{pgfscope}%
\begin{pgfscope}%
\pgfpathrectangle{\pgfqpoint{0.100000in}{2.413063in}}{\pgfqpoint{5.037500in}{3.427208in}}%
\pgfusepath{clip}%
\pgfsetbuttcap%
\pgfsetroundjoin%
\definecolor{currentfill}{rgb}{0.000000,0.000000,1.000000}%
\pgfsetfillcolor{currentfill}%
\pgfsetfillopacity{0.500000}%
\pgfsetlinewidth{0.250937pt}%
\definecolor{currentstroke}{rgb}{0.000000,0.000000,0.000000}%
\pgfsetstrokecolor{currentstroke}%
\pgfsetstrokeopacity{0.500000}%
\pgfsetdash{}{0pt}%
\pgfsys@defobject{currentmarker}{\pgfqpoint{-0.019444in}{-0.019444in}}{\pgfqpoint{0.019444in}{0.019444in}}{%
\pgfpathmoveto{\pgfqpoint{0.000000in}{-0.019444in}}%
\pgfpathcurveto{\pgfqpoint{0.005157in}{-0.019444in}}{\pgfqpoint{0.010103in}{-0.017396in}}{\pgfqpoint{0.013749in}{-0.013749in}}%
\pgfpathcurveto{\pgfqpoint{0.017396in}{-0.010103in}}{\pgfqpoint{0.019444in}{-0.005157in}}{\pgfqpoint{0.019444in}{0.000000in}}%
\pgfpathcurveto{\pgfqpoint{0.019444in}{0.005157in}}{\pgfqpoint{0.017396in}{0.010103in}}{\pgfqpoint{0.013749in}{0.013749in}}%
\pgfpathcurveto{\pgfqpoint{0.010103in}{0.017396in}}{\pgfqpoint{0.005157in}{0.019444in}}{\pgfqpoint{0.000000in}{0.019444in}}%
\pgfpathcurveto{\pgfqpoint{-0.005157in}{0.019444in}}{\pgfqpoint{-0.010103in}{0.017396in}}{\pgfqpoint{-0.013749in}{0.013749in}}%
\pgfpathcurveto{\pgfqpoint{-0.017396in}{0.010103in}}{\pgfqpoint{-0.019444in}{0.005157in}}{\pgfqpoint{-0.019444in}{0.000000in}}%
\pgfpathcurveto{\pgfqpoint{-0.019444in}{-0.005157in}}{\pgfqpoint{-0.017396in}{-0.010103in}}{\pgfqpoint{-0.013749in}{-0.013749in}}%
\pgfpathcurveto{\pgfqpoint{-0.010103in}{-0.017396in}}{\pgfqpoint{-0.005157in}{-0.019444in}}{\pgfqpoint{0.000000in}{-0.019444in}}%
\pgfpathclose%
\pgfusepath{stroke,fill}%
}%
\begin{pgfscope}%
\pgfsys@transformshift{4.463839in}{4.369843in}%
\pgfsys@useobject{currentmarker}{}%
\end{pgfscope}%
\end{pgfscope}%
\begin{pgfscope}%
\pgfpathrectangle{\pgfqpoint{0.100000in}{2.413063in}}{\pgfqpoint{5.037500in}{3.427208in}}%
\pgfusepath{clip}%
\pgfsetrectcap%
\pgfsetroundjoin%
\pgfsetlinewidth{1.505625pt}%
\definecolor{currentstroke}{rgb}{0.678431,1.000000,0.184314}%
\pgfsetstrokecolor{currentstroke}%
\pgfsetstrokeopacity{0.500000}%
\pgfsetdash{}{0pt}%
\pgfpathmoveto{\pgfqpoint{4.236084in}{4.484394in}}%
\pgfusepath{stroke}%
\end{pgfscope}%
\begin{pgfscope}%
\pgfpathrectangle{\pgfqpoint{0.100000in}{2.413063in}}{\pgfqpoint{5.037500in}{3.427208in}}%
\pgfusepath{clip}%
\pgfsetbuttcap%
\pgfsetroundjoin%
\definecolor{currentfill}{rgb}{0.678431,1.000000,0.184314}%
\pgfsetfillcolor{currentfill}%
\pgfsetfillopacity{0.500000}%
\pgfsetlinewidth{0.250937pt}%
\definecolor{currentstroke}{rgb}{0.000000,0.000000,0.000000}%
\pgfsetstrokecolor{currentstroke}%
\pgfsetstrokeopacity{0.500000}%
\pgfsetdash{}{0pt}%
\pgfsys@defobject{currentmarker}{\pgfqpoint{-0.019444in}{-0.019444in}}{\pgfqpoint{0.019444in}{0.019444in}}{%
\pgfpathmoveto{\pgfqpoint{0.000000in}{-0.019444in}}%
\pgfpathcurveto{\pgfqpoint{0.005157in}{-0.019444in}}{\pgfqpoint{0.010103in}{-0.017396in}}{\pgfqpoint{0.013749in}{-0.013749in}}%
\pgfpathcurveto{\pgfqpoint{0.017396in}{-0.010103in}}{\pgfqpoint{0.019444in}{-0.005157in}}{\pgfqpoint{0.019444in}{0.000000in}}%
\pgfpathcurveto{\pgfqpoint{0.019444in}{0.005157in}}{\pgfqpoint{0.017396in}{0.010103in}}{\pgfqpoint{0.013749in}{0.013749in}}%
\pgfpathcurveto{\pgfqpoint{0.010103in}{0.017396in}}{\pgfqpoint{0.005157in}{0.019444in}}{\pgfqpoint{0.000000in}{0.019444in}}%
\pgfpathcurveto{\pgfqpoint{-0.005157in}{0.019444in}}{\pgfqpoint{-0.010103in}{0.017396in}}{\pgfqpoint{-0.013749in}{0.013749in}}%
\pgfpathcurveto{\pgfqpoint{-0.017396in}{0.010103in}}{\pgfqpoint{-0.019444in}{0.005157in}}{\pgfqpoint{-0.019444in}{0.000000in}}%
\pgfpathcurveto{\pgfqpoint{-0.019444in}{-0.005157in}}{\pgfqpoint{-0.017396in}{-0.010103in}}{\pgfqpoint{-0.013749in}{-0.013749in}}%
\pgfpathcurveto{\pgfqpoint{-0.010103in}{-0.017396in}}{\pgfqpoint{-0.005157in}{-0.019444in}}{\pgfqpoint{0.000000in}{-0.019444in}}%
\pgfpathclose%
\pgfusepath{stroke,fill}%
}%
\begin{pgfscope}%
\pgfsys@transformshift{4.236084in}{4.484394in}%
\pgfsys@useobject{currentmarker}{}%
\end{pgfscope}%
\end{pgfscope}%
\begin{pgfscope}%
\pgfpathrectangle{\pgfqpoint{0.100000in}{2.413063in}}{\pgfqpoint{5.037500in}{3.427208in}}%
\pgfusepath{clip}%
\pgfsetrectcap%
\pgfsetroundjoin%
\pgfsetlinewidth{1.505625pt}%
\definecolor{currentstroke}{rgb}{0.678431,1.000000,0.184314}%
\pgfsetstrokecolor{currentstroke}%
\pgfsetstrokeopacity{0.500000}%
\pgfsetdash{}{0pt}%
\pgfpathmoveto{\pgfqpoint{4.327155in}{4.500101in}}%
\pgfusepath{stroke}%
\end{pgfscope}%
\begin{pgfscope}%
\pgfpathrectangle{\pgfqpoint{0.100000in}{2.413063in}}{\pgfqpoint{5.037500in}{3.427208in}}%
\pgfusepath{clip}%
\pgfsetbuttcap%
\pgfsetroundjoin%
\definecolor{currentfill}{rgb}{0.678431,1.000000,0.184314}%
\pgfsetfillcolor{currentfill}%
\pgfsetfillopacity{0.500000}%
\pgfsetlinewidth{0.250937pt}%
\definecolor{currentstroke}{rgb}{0.000000,0.000000,0.000000}%
\pgfsetstrokecolor{currentstroke}%
\pgfsetstrokeopacity{0.500000}%
\pgfsetdash{}{0pt}%
\pgfsys@defobject{currentmarker}{\pgfqpoint{-0.005556in}{-0.005556in}}{\pgfqpoint{0.005556in}{0.005556in}}{%
\pgfpathmoveto{\pgfqpoint{0.000000in}{-0.005556in}}%
\pgfpathcurveto{\pgfqpoint{0.001473in}{-0.005556in}}{\pgfqpoint{0.002887in}{-0.004970in}}{\pgfqpoint{0.003928in}{-0.003928in}}%
\pgfpathcurveto{\pgfqpoint{0.004970in}{-0.002887in}}{\pgfqpoint{0.005556in}{-0.001473in}}{\pgfqpoint{0.005556in}{0.000000in}}%
\pgfpathcurveto{\pgfqpoint{0.005556in}{0.001473in}}{\pgfqpoint{0.004970in}{0.002887in}}{\pgfqpoint{0.003928in}{0.003928in}}%
\pgfpathcurveto{\pgfqpoint{0.002887in}{0.004970in}}{\pgfqpoint{0.001473in}{0.005556in}}{\pgfqpoint{0.000000in}{0.005556in}}%
\pgfpathcurveto{\pgfqpoint{-0.001473in}{0.005556in}}{\pgfqpoint{-0.002887in}{0.004970in}}{\pgfqpoint{-0.003928in}{0.003928in}}%
\pgfpathcurveto{\pgfqpoint{-0.004970in}{0.002887in}}{\pgfqpoint{-0.005556in}{0.001473in}}{\pgfqpoint{-0.005556in}{0.000000in}}%
\pgfpathcurveto{\pgfqpoint{-0.005556in}{-0.001473in}}{\pgfqpoint{-0.004970in}{-0.002887in}}{\pgfqpoint{-0.003928in}{-0.003928in}}%
\pgfpathcurveto{\pgfqpoint{-0.002887in}{-0.004970in}}{\pgfqpoint{-0.001473in}{-0.005556in}}{\pgfqpoint{0.000000in}{-0.005556in}}%
\pgfpathclose%
\pgfusepath{stroke,fill}%
}%
\begin{pgfscope}%
\pgfsys@transformshift{4.327155in}{4.500101in}%
\pgfsys@useobject{currentmarker}{}%
\end{pgfscope}%
\end{pgfscope}%
\begin{pgfscope}%
\pgfpathrectangle{\pgfqpoint{0.100000in}{2.413063in}}{\pgfqpoint{5.037500in}{3.427208in}}%
\pgfusepath{clip}%
\pgfsetrectcap%
\pgfsetroundjoin%
\pgfsetlinewidth{1.505625pt}%
\definecolor{currentstroke}{rgb}{0.000000,0.000000,1.000000}%
\pgfsetstrokecolor{currentstroke}%
\pgfsetstrokeopacity{0.500000}%
\pgfsetdash{}{0pt}%
\pgfpathmoveto{\pgfqpoint{4.906699in}{4.874236in}}%
\pgfusepath{stroke}%
\end{pgfscope}%
\begin{pgfscope}%
\pgfpathrectangle{\pgfqpoint{0.100000in}{2.413063in}}{\pgfqpoint{5.037500in}{3.427208in}}%
\pgfusepath{clip}%
\pgfsetbuttcap%
\pgfsetroundjoin%
\definecolor{currentfill}{rgb}{0.000000,0.000000,1.000000}%
\pgfsetfillcolor{currentfill}%
\pgfsetfillopacity{0.500000}%
\pgfsetlinewidth{0.250937pt}%
\definecolor{currentstroke}{rgb}{0.000000,0.000000,0.000000}%
\pgfsetstrokecolor{currentstroke}%
\pgfsetstrokeopacity{0.500000}%
\pgfsetdash{}{0pt}%
\pgfsys@defobject{currentmarker}{\pgfqpoint{-0.019444in}{-0.019444in}}{\pgfqpoint{0.019444in}{0.019444in}}{%
\pgfpathmoveto{\pgfqpoint{0.000000in}{-0.019444in}}%
\pgfpathcurveto{\pgfqpoint{0.005157in}{-0.019444in}}{\pgfqpoint{0.010103in}{-0.017396in}}{\pgfqpoint{0.013749in}{-0.013749in}}%
\pgfpathcurveto{\pgfqpoint{0.017396in}{-0.010103in}}{\pgfqpoint{0.019444in}{-0.005157in}}{\pgfqpoint{0.019444in}{0.000000in}}%
\pgfpathcurveto{\pgfqpoint{0.019444in}{0.005157in}}{\pgfqpoint{0.017396in}{0.010103in}}{\pgfqpoint{0.013749in}{0.013749in}}%
\pgfpathcurveto{\pgfqpoint{0.010103in}{0.017396in}}{\pgfqpoint{0.005157in}{0.019444in}}{\pgfqpoint{0.000000in}{0.019444in}}%
\pgfpathcurveto{\pgfqpoint{-0.005157in}{0.019444in}}{\pgfqpoint{-0.010103in}{0.017396in}}{\pgfqpoint{-0.013749in}{0.013749in}}%
\pgfpathcurveto{\pgfqpoint{-0.017396in}{0.010103in}}{\pgfqpoint{-0.019444in}{0.005157in}}{\pgfqpoint{-0.019444in}{0.000000in}}%
\pgfpathcurveto{\pgfqpoint{-0.019444in}{-0.005157in}}{\pgfqpoint{-0.017396in}{-0.010103in}}{\pgfqpoint{-0.013749in}{-0.013749in}}%
\pgfpathcurveto{\pgfqpoint{-0.010103in}{-0.017396in}}{\pgfqpoint{-0.005157in}{-0.019444in}}{\pgfqpoint{0.000000in}{-0.019444in}}%
\pgfpathclose%
\pgfusepath{stroke,fill}%
}%
\begin{pgfscope}%
\pgfsys@transformshift{4.906699in}{4.874236in}%
\pgfsys@useobject{currentmarker}{}%
\end{pgfscope}%
\end{pgfscope}%
\begin{pgfscope}%
\pgfpathrectangle{\pgfqpoint{0.100000in}{2.413063in}}{\pgfqpoint{5.037500in}{3.427208in}}%
\pgfusepath{clip}%
\pgfsetrectcap%
\pgfsetroundjoin%
\pgfsetlinewidth{1.505625pt}%
\definecolor{currentstroke}{rgb}{0.000000,0.000000,1.000000}%
\pgfsetstrokecolor{currentstroke}%
\pgfsetstrokeopacity{0.500000}%
\pgfsetdash{}{0pt}%
\pgfpathmoveto{\pgfqpoint{4.820957in}{4.935309in}}%
\pgfusepath{stroke}%
\end{pgfscope}%
\begin{pgfscope}%
\pgfpathrectangle{\pgfqpoint{0.100000in}{2.413063in}}{\pgfqpoint{5.037500in}{3.427208in}}%
\pgfusepath{clip}%
\pgfsetbuttcap%
\pgfsetroundjoin%
\definecolor{currentfill}{rgb}{0.000000,0.000000,1.000000}%
\pgfsetfillcolor{currentfill}%
\pgfsetfillopacity{0.500000}%
\pgfsetlinewidth{0.250937pt}%
\definecolor{currentstroke}{rgb}{0.000000,0.000000,0.000000}%
\pgfsetstrokecolor{currentstroke}%
\pgfsetstrokeopacity{0.500000}%
\pgfsetdash{}{0pt}%
\pgfsys@defobject{currentmarker}{\pgfqpoint{-0.025000in}{-0.025000in}}{\pgfqpoint{0.025000in}{0.025000in}}{%
\pgfpathmoveto{\pgfqpoint{0.000000in}{-0.025000in}}%
\pgfpathcurveto{\pgfqpoint{0.006630in}{-0.025000in}}{\pgfqpoint{0.012989in}{-0.022366in}}{\pgfqpoint{0.017678in}{-0.017678in}}%
\pgfpathcurveto{\pgfqpoint{0.022366in}{-0.012989in}}{\pgfqpoint{0.025000in}{-0.006630in}}{\pgfqpoint{0.025000in}{0.000000in}}%
\pgfpathcurveto{\pgfqpoint{0.025000in}{0.006630in}}{\pgfqpoint{0.022366in}{0.012989in}}{\pgfqpoint{0.017678in}{0.017678in}}%
\pgfpathcurveto{\pgfqpoint{0.012989in}{0.022366in}}{\pgfqpoint{0.006630in}{0.025000in}}{\pgfqpoint{0.000000in}{0.025000in}}%
\pgfpathcurveto{\pgfqpoint{-0.006630in}{0.025000in}}{\pgfqpoint{-0.012989in}{0.022366in}}{\pgfqpoint{-0.017678in}{0.017678in}}%
\pgfpathcurveto{\pgfqpoint{-0.022366in}{0.012989in}}{\pgfqpoint{-0.025000in}{0.006630in}}{\pgfqpoint{-0.025000in}{0.000000in}}%
\pgfpathcurveto{\pgfqpoint{-0.025000in}{-0.006630in}}{\pgfqpoint{-0.022366in}{-0.012989in}}{\pgfqpoint{-0.017678in}{-0.017678in}}%
\pgfpathcurveto{\pgfqpoint{-0.012989in}{-0.022366in}}{\pgfqpoint{-0.006630in}{-0.025000in}}{\pgfqpoint{0.000000in}{-0.025000in}}%
\pgfpathclose%
\pgfusepath{stroke,fill}%
}%
\begin{pgfscope}%
\pgfsys@transformshift{4.820957in}{4.935309in}%
\pgfsys@useobject{currentmarker}{}%
\end{pgfscope}%
\end{pgfscope}%
\begin{pgfscope}%
\pgfpathrectangle{\pgfqpoint{0.100000in}{2.413063in}}{\pgfqpoint{5.037500in}{3.427208in}}%
\pgfusepath{clip}%
\pgfsetrectcap%
\pgfsetroundjoin%
\pgfsetlinewidth{1.505625pt}%
\definecolor{currentstroke}{rgb}{0.000000,0.000000,1.000000}%
\pgfsetstrokecolor{currentstroke}%
\pgfsetstrokeopacity{0.500000}%
\pgfsetdash{}{0pt}%
\pgfpathmoveto{\pgfqpoint{4.758476in}{4.938361in}}%
\pgfusepath{stroke}%
\end{pgfscope}%
\begin{pgfscope}%
\pgfpathrectangle{\pgfqpoint{0.100000in}{2.413063in}}{\pgfqpoint{5.037500in}{3.427208in}}%
\pgfusepath{clip}%
\pgfsetbuttcap%
\pgfsetroundjoin%
\definecolor{currentfill}{rgb}{0.000000,0.000000,1.000000}%
\pgfsetfillcolor{currentfill}%
\pgfsetfillopacity{0.500000}%
\pgfsetlinewidth{0.250937pt}%
\definecolor{currentstroke}{rgb}{0.000000,0.000000,0.000000}%
\pgfsetstrokecolor{currentstroke}%
\pgfsetstrokeopacity{0.500000}%
\pgfsetdash{}{0pt}%
\pgfsys@defobject{currentmarker}{\pgfqpoint{-0.030556in}{-0.030556in}}{\pgfqpoint{0.030556in}{0.030556in}}{%
\pgfpathmoveto{\pgfqpoint{0.000000in}{-0.030556in}}%
\pgfpathcurveto{\pgfqpoint{0.008103in}{-0.030556in}}{\pgfqpoint{0.015876in}{-0.027336in}}{\pgfqpoint{0.021606in}{-0.021606in}}%
\pgfpathcurveto{\pgfqpoint{0.027336in}{-0.015876in}}{\pgfqpoint{0.030556in}{-0.008103in}}{\pgfqpoint{0.030556in}{0.000000in}}%
\pgfpathcurveto{\pgfqpoint{0.030556in}{0.008103in}}{\pgfqpoint{0.027336in}{0.015876in}}{\pgfqpoint{0.021606in}{0.021606in}}%
\pgfpathcurveto{\pgfqpoint{0.015876in}{0.027336in}}{\pgfqpoint{0.008103in}{0.030556in}}{\pgfqpoint{0.000000in}{0.030556in}}%
\pgfpathcurveto{\pgfqpoint{-0.008103in}{0.030556in}}{\pgfqpoint{-0.015876in}{0.027336in}}{\pgfqpoint{-0.021606in}{0.021606in}}%
\pgfpathcurveto{\pgfqpoint{-0.027336in}{0.015876in}}{\pgfqpoint{-0.030556in}{0.008103in}}{\pgfqpoint{-0.030556in}{0.000000in}}%
\pgfpathcurveto{\pgfqpoint{-0.030556in}{-0.008103in}}{\pgfqpoint{-0.027336in}{-0.015876in}}{\pgfqpoint{-0.021606in}{-0.021606in}}%
\pgfpathcurveto{\pgfqpoint{-0.015876in}{-0.027336in}}{\pgfqpoint{-0.008103in}{-0.030556in}}{\pgfqpoint{0.000000in}{-0.030556in}}%
\pgfpathclose%
\pgfusepath{stroke,fill}%
}%
\begin{pgfscope}%
\pgfsys@transformshift{4.758476in}{4.938361in}%
\pgfsys@useobject{currentmarker}{}%
\end{pgfscope}%
\end{pgfscope}%
\begin{pgfscope}%
\pgfpathrectangle{\pgfqpoint{0.100000in}{2.413063in}}{\pgfqpoint{5.037500in}{3.427208in}}%
\pgfusepath{clip}%
\pgfsetrectcap%
\pgfsetroundjoin%
\pgfsetlinewidth{1.505625pt}%
\definecolor{currentstroke}{rgb}{0.000000,0.000000,1.000000}%
\pgfsetstrokecolor{currentstroke}%
\pgfsetstrokeopacity{0.500000}%
\pgfsetdash{}{0pt}%
\pgfpathmoveto{\pgfqpoint{4.853127in}{4.857502in}}%
\pgfusepath{stroke}%
\end{pgfscope}%
\begin{pgfscope}%
\pgfpathrectangle{\pgfqpoint{0.100000in}{2.413063in}}{\pgfqpoint{5.037500in}{3.427208in}}%
\pgfusepath{clip}%
\pgfsetbuttcap%
\pgfsetroundjoin%
\definecolor{currentfill}{rgb}{0.000000,0.000000,1.000000}%
\pgfsetfillcolor{currentfill}%
\pgfsetfillopacity{0.500000}%
\pgfsetlinewidth{0.250937pt}%
\definecolor{currentstroke}{rgb}{0.000000,0.000000,0.000000}%
\pgfsetstrokecolor{currentstroke}%
\pgfsetstrokeopacity{0.500000}%
\pgfsetdash{}{0pt}%
\pgfsys@defobject{currentmarker}{\pgfqpoint{-0.030556in}{-0.030556in}}{\pgfqpoint{0.030556in}{0.030556in}}{%
\pgfpathmoveto{\pgfqpoint{0.000000in}{-0.030556in}}%
\pgfpathcurveto{\pgfqpoint{0.008103in}{-0.030556in}}{\pgfqpoint{0.015876in}{-0.027336in}}{\pgfqpoint{0.021606in}{-0.021606in}}%
\pgfpathcurveto{\pgfqpoint{0.027336in}{-0.015876in}}{\pgfqpoint{0.030556in}{-0.008103in}}{\pgfqpoint{0.030556in}{0.000000in}}%
\pgfpathcurveto{\pgfqpoint{0.030556in}{0.008103in}}{\pgfqpoint{0.027336in}{0.015876in}}{\pgfqpoint{0.021606in}{0.021606in}}%
\pgfpathcurveto{\pgfqpoint{0.015876in}{0.027336in}}{\pgfqpoint{0.008103in}{0.030556in}}{\pgfqpoint{0.000000in}{0.030556in}}%
\pgfpathcurveto{\pgfqpoint{-0.008103in}{0.030556in}}{\pgfqpoint{-0.015876in}{0.027336in}}{\pgfqpoint{-0.021606in}{0.021606in}}%
\pgfpathcurveto{\pgfqpoint{-0.027336in}{0.015876in}}{\pgfqpoint{-0.030556in}{0.008103in}}{\pgfqpoint{-0.030556in}{0.000000in}}%
\pgfpathcurveto{\pgfqpoint{-0.030556in}{-0.008103in}}{\pgfqpoint{-0.027336in}{-0.015876in}}{\pgfqpoint{-0.021606in}{-0.021606in}}%
\pgfpathcurveto{\pgfqpoint{-0.015876in}{-0.027336in}}{\pgfqpoint{-0.008103in}{-0.030556in}}{\pgfqpoint{0.000000in}{-0.030556in}}%
\pgfpathclose%
\pgfusepath{stroke,fill}%
}%
\begin{pgfscope}%
\pgfsys@transformshift{4.853127in}{4.857502in}%
\pgfsys@useobject{currentmarker}{}%
\end{pgfscope}%
\end{pgfscope}%
\begin{pgfscope}%
\pgfpathrectangle{\pgfqpoint{0.100000in}{2.413063in}}{\pgfqpoint{5.037500in}{3.427208in}}%
\pgfusepath{clip}%
\pgfsetrectcap%
\pgfsetroundjoin%
\pgfsetlinewidth{1.505625pt}%
\definecolor{currentstroke}{rgb}{0.000000,0.000000,1.000000}%
\pgfsetstrokecolor{currentstroke}%
\pgfsetstrokeopacity{0.500000}%
\pgfsetdash{}{0pt}%
\pgfpathmoveto{\pgfqpoint{4.638079in}{4.899009in}}%
\pgfusepath{stroke}%
\end{pgfscope}%
\begin{pgfscope}%
\pgfpathrectangle{\pgfqpoint{0.100000in}{2.413063in}}{\pgfqpoint{5.037500in}{3.427208in}}%
\pgfusepath{clip}%
\pgfsetbuttcap%
\pgfsetroundjoin%
\definecolor{currentfill}{rgb}{0.000000,0.000000,1.000000}%
\pgfsetfillcolor{currentfill}%
\pgfsetfillopacity{0.500000}%
\pgfsetlinewidth{0.250937pt}%
\definecolor{currentstroke}{rgb}{0.000000,0.000000,0.000000}%
\pgfsetstrokecolor{currentstroke}%
\pgfsetstrokeopacity{0.500000}%
\pgfsetdash{}{0pt}%
\pgfsys@defobject{currentmarker}{\pgfqpoint{-0.030556in}{-0.030556in}}{\pgfqpoint{0.030556in}{0.030556in}}{%
\pgfpathmoveto{\pgfqpoint{0.000000in}{-0.030556in}}%
\pgfpathcurveto{\pgfqpoint{0.008103in}{-0.030556in}}{\pgfqpoint{0.015876in}{-0.027336in}}{\pgfqpoint{0.021606in}{-0.021606in}}%
\pgfpathcurveto{\pgfqpoint{0.027336in}{-0.015876in}}{\pgfqpoint{0.030556in}{-0.008103in}}{\pgfqpoint{0.030556in}{0.000000in}}%
\pgfpathcurveto{\pgfqpoint{0.030556in}{0.008103in}}{\pgfqpoint{0.027336in}{0.015876in}}{\pgfqpoint{0.021606in}{0.021606in}}%
\pgfpathcurveto{\pgfqpoint{0.015876in}{0.027336in}}{\pgfqpoint{0.008103in}{0.030556in}}{\pgfqpoint{0.000000in}{0.030556in}}%
\pgfpathcurveto{\pgfqpoint{-0.008103in}{0.030556in}}{\pgfqpoint{-0.015876in}{0.027336in}}{\pgfqpoint{-0.021606in}{0.021606in}}%
\pgfpathcurveto{\pgfqpoint{-0.027336in}{0.015876in}}{\pgfqpoint{-0.030556in}{0.008103in}}{\pgfqpoint{-0.030556in}{0.000000in}}%
\pgfpathcurveto{\pgfqpoint{-0.030556in}{-0.008103in}}{\pgfqpoint{-0.027336in}{-0.015876in}}{\pgfqpoint{-0.021606in}{-0.021606in}}%
\pgfpathcurveto{\pgfqpoint{-0.015876in}{-0.027336in}}{\pgfqpoint{-0.008103in}{-0.030556in}}{\pgfqpoint{0.000000in}{-0.030556in}}%
\pgfpathclose%
\pgfusepath{stroke,fill}%
}%
\begin{pgfscope}%
\pgfsys@transformshift{4.638079in}{4.899009in}%
\pgfsys@useobject{currentmarker}{}%
\end{pgfscope}%
\end{pgfscope}%
\begin{pgfscope}%
\pgfpathrectangle{\pgfqpoint{0.100000in}{2.413063in}}{\pgfqpoint{5.037500in}{3.427208in}}%
\pgfusepath{clip}%
\pgfsetrectcap%
\pgfsetroundjoin%
\pgfsetlinewidth{1.505625pt}%
\definecolor{currentstroke}{rgb}{0.000000,0.000000,1.000000}%
\pgfsetstrokecolor{currentstroke}%
\pgfsetstrokeopacity{0.500000}%
\pgfsetdash{}{0pt}%
\pgfpathmoveto{\pgfqpoint{4.702051in}{4.873445in}}%
\pgfusepath{stroke}%
\end{pgfscope}%
\begin{pgfscope}%
\pgfpathrectangle{\pgfqpoint{0.100000in}{2.413063in}}{\pgfqpoint{5.037500in}{3.427208in}}%
\pgfusepath{clip}%
\pgfsetbuttcap%
\pgfsetroundjoin%
\definecolor{currentfill}{rgb}{0.000000,0.000000,1.000000}%
\pgfsetfillcolor{currentfill}%
\pgfsetfillopacity{0.500000}%
\pgfsetlinewidth{0.250937pt}%
\definecolor{currentstroke}{rgb}{0.000000,0.000000,0.000000}%
\pgfsetstrokecolor{currentstroke}%
\pgfsetstrokeopacity{0.500000}%
\pgfsetdash{}{0pt}%
\pgfsys@defobject{currentmarker}{\pgfqpoint{-0.030556in}{-0.030556in}}{\pgfqpoint{0.030556in}{0.030556in}}{%
\pgfpathmoveto{\pgfqpoint{0.000000in}{-0.030556in}}%
\pgfpathcurveto{\pgfqpoint{0.008103in}{-0.030556in}}{\pgfqpoint{0.015876in}{-0.027336in}}{\pgfqpoint{0.021606in}{-0.021606in}}%
\pgfpathcurveto{\pgfqpoint{0.027336in}{-0.015876in}}{\pgfqpoint{0.030556in}{-0.008103in}}{\pgfqpoint{0.030556in}{0.000000in}}%
\pgfpathcurveto{\pgfqpoint{0.030556in}{0.008103in}}{\pgfqpoint{0.027336in}{0.015876in}}{\pgfqpoint{0.021606in}{0.021606in}}%
\pgfpathcurveto{\pgfqpoint{0.015876in}{0.027336in}}{\pgfqpoint{0.008103in}{0.030556in}}{\pgfqpoint{0.000000in}{0.030556in}}%
\pgfpathcurveto{\pgfqpoint{-0.008103in}{0.030556in}}{\pgfqpoint{-0.015876in}{0.027336in}}{\pgfqpoint{-0.021606in}{0.021606in}}%
\pgfpathcurveto{\pgfqpoint{-0.027336in}{0.015876in}}{\pgfqpoint{-0.030556in}{0.008103in}}{\pgfqpoint{-0.030556in}{0.000000in}}%
\pgfpathcurveto{\pgfqpoint{-0.030556in}{-0.008103in}}{\pgfqpoint{-0.027336in}{-0.015876in}}{\pgfqpoint{-0.021606in}{-0.021606in}}%
\pgfpathcurveto{\pgfqpoint{-0.015876in}{-0.027336in}}{\pgfqpoint{-0.008103in}{-0.030556in}}{\pgfqpoint{0.000000in}{-0.030556in}}%
\pgfpathclose%
\pgfusepath{stroke,fill}%
}%
\begin{pgfscope}%
\pgfsys@transformshift{4.702051in}{4.873445in}%
\pgfsys@useobject{currentmarker}{}%
\end{pgfscope}%
\end{pgfscope}%
\begin{pgfscope}%
\pgfpathrectangle{\pgfqpoint{0.100000in}{2.413063in}}{\pgfqpoint{5.037500in}{3.427208in}}%
\pgfusepath{clip}%
\pgfsetrectcap%
\pgfsetroundjoin%
\pgfsetlinewidth{1.505625pt}%
\definecolor{currentstroke}{rgb}{0.000000,0.000000,1.000000}%
\pgfsetstrokecolor{currentstroke}%
\pgfsetstrokeopacity{0.500000}%
\pgfsetdash{}{0pt}%
\pgfpathmoveto{\pgfqpoint{4.762625in}{4.908152in}}%
\pgfusepath{stroke}%
\end{pgfscope}%
\begin{pgfscope}%
\pgfpathrectangle{\pgfqpoint{0.100000in}{2.413063in}}{\pgfqpoint{5.037500in}{3.427208in}}%
\pgfusepath{clip}%
\pgfsetbuttcap%
\pgfsetroundjoin%
\definecolor{currentfill}{rgb}{0.000000,0.000000,1.000000}%
\pgfsetfillcolor{currentfill}%
\pgfsetfillopacity{0.500000}%
\pgfsetlinewidth{0.250937pt}%
\definecolor{currentstroke}{rgb}{0.000000,0.000000,0.000000}%
\pgfsetstrokecolor{currentstroke}%
\pgfsetstrokeopacity{0.500000}%
\pgfsetdash{}{0pt}%
\pgfsys@defobject{currentmarker}{\pgfqpoint{-0.022222in}{-0.022222in}}{\pgfqpoint{0.022222in}{0.022222in}}{%
\pgfpathmoveto{\pgfqpoint{0.000000in}{-0.022222in}}%
\pgfpathcurveto{\pgfqpoint{0.005893in}{-0.022222in}}{\pgfqpoint{0.011546in}{-0.019881in}}{\pgfqpoint{0.015713in}{-0.015713in}}%
\pgfpathcurveto{\pgfqpoint{0.019881in}{-0.011546in}}{\pgfqpoint{0.022222in}{-0.005893in}}{\pgfqpoint{0.022222in}{0.000000in}}%
\pgfpathcurveto{\pgfqpoint{0.022222in}{0.005893in}}{\pgfqpoint{0.019881in}{0.011546in}}{\pgfqpoint{0.015713in}{0.015713in}}%
\pgfpathcurveto{\pgfqpoint{0.011546in}{0.019881in}}{\pgfqpoint{0.005893in}{0.022222in}}{\pgfqpoint{0.000000in}{0.022222in}}%
\pgfpathcurveto{\pgfqpoint{-0.005893in}{0.022222in}}{\pgfqpoint{-0.011546in}{0.019881in}}{\pgfqpoint{-0.015713in}{0.015713in}}%
\pgfpathcurveto{\pgfqpoint{-0.019881in}{0.011546in}}{\pgfqpoint{-0.022222in}{0.005893in}}{\pgfqpoint{-0.022222in}{0.000000in}}%
\pgfpathcurveto{\pgfqpoint{-0.022222in}{-0.005893in}}{\pgfqpoint{-0.019881in}{-0.011546in}}{\pgfqpoint{-0.015713in}{-0.015713in}}%
\pgfpathcurveto{\pgfqpoint{-0.011546in}{-0.019881in}}{\pgfqpoint{-0.005893in}{-0.022222in}}{\pgfqpoint{0.000000in}{-0.022222in}}%
\pgfpathclose%
\pgfusepath{stroke,fill}%
}%
\begin{pgfscope}%
\pgfsys@transformshift{4.762625in}{4.908152in}%
\pgfsys@useobject{currentmarker}{}%
\end{pgfscope}%
\end{pgfscope}%
\begin{pgfscope}%
\pgfpathrectangle{\pgfqpoint{0.100000in}{2.413063in}}{\pgfqpoint{5.037500in}{3.427208in}}%
\pgfusepath{clip}%
\pgfsetrectcap%
\pgfsetroundjoin%
\pgfsetlinewidth{1.505625pt}%
\definecolor{currentstroke}{rgb}{0.000000,0.000000,1.000000}%
\pgfsetstrokecolor{currentstroke}%
\pgfsetstrokeopacity{0.500000}%
\pgfsetdash{}{0pt}%
\pgfpathmoveto{\pgfqpoint{3.763378in}{4.717009in}}%
\pgfusepath{stroke}%
\end{pgfscope}%
\begin{pgfscope}%
\pgfpathrectangle{\pgfqpoint{0.100000in}{2.413063in}}{\pgfqpoint{5.037500in}{3.427208in}}%
\pgfusepath{clip}%
\pgfsetbuttcap%
\pgfsetroundjoin%
\definecolor{currentfill}{rgb}{0.000000,0.000000,1.000000}%
\pgfsetfillcolor{currentfill}%
\pgfsetfillopacity{0.500000}%
\pgfsetlinewidth{0.250937pt}%
\definecolor{currentstroke}{rgb}{0.000000,0.000000,0.000000}%
\pgfsetstrokecolor{currentstroke}%
\pgfsetstrokeopacity{0.500000}%
\pgfsetdash{}{0pt}%
\pgfsys@defobject{currentmarker}{\pgfqpoint{-0.036111in}{-0.036111in}}{\pgfqpoint{0.036111in}{0.036111in}}{%
\pgfpathmoveto{\pgfqpoint{0.000000in}{-0.036111in}}%
\pgfpathcurveto{\pgfqpoint{0.009577in}{-0.036111in}}{\pgfqpoint{0.018763in}{-0.032306in}}{\pgfqpoint{0.025534in}{-0.025534in}}%
\pgfpathcurveto{\pgfqpoint{0.032306in}{-0.018763in}}{\pgfqpoint{0.036111in}{-0.009577in}}{\pgfqpoint{0.036111in}{0.000000in}}%
\pgfpathcurveto{\pgfqpoint{0.036111in}{0.009577in}}{\pgfqpoint{0.032306in}{0.018763in}}{\pgfqpoint{0.025534in}{0.025534in}}%
\pgfpathcurveto{\pgfqpoint{0.018763in}{0.032306in}}{\pgfqpoint{0.009577in}{0.036111in}}{\pgfqpoint{0.000000in}{0.036111in}}%
\pgfpathcurveto{\pgfqpoint{-0.009577in}{0.036111in}}{\pgfqpoint{-0.018763in}{0.032306in}}{\pgfqpoint{-0.025534in}{0.025534in}}%
\pgfpathcurveto{\pgfqpoint{-0.032306in}{0.018763in}}{\pgfqpoint{-0.036111in}{0.009577in}}{\pgfqpoint{-0.036111in}{0.000000in}}%
\pgfpathcurveto{\pgfqpoint{-0.036111in}{-0.009577in}}{\pgfqpoint{-0.032306in}{-0.018763in}}{\pgfqpoint{-0.025534in}{-0.025534in}}%
\pgfpathcurveto{\pgfqpoint{-0.018763in}{-0.032306in}}{\pgfqpoint{-0.009577in}{-0.036111in}}{\pgfqpoint{0.000000in}{-0.036111in}}%
\pgfpathclose%
\pgfusepath{stroke,fill}%
}%
\begin{pgfscope}%
\pgfsys@transformshift{3.763378in}{4.717009in}%
\pgfsys@useobject{currentmarker}{}%
\end{pgfscope}%
\end{pgfscope}%
\begin{pgfscope}%
\pgfpathrectangle{\pgfqpoint{0.100000in}{2.413063in}}{\pgfqpoint{5.037500in}{3.427208in}}%
\pgfusepath{clip}%
\pgfsetrectcap%
\pgfsetroundjoin%
\pgfsetlinewidth{1.505625pt}%
\definecolor{currentstroke}{rgb}{0.000000,0.000000,1.000000}%
\pgfsetstrokecolor{currentstroke}%
\pgfsetstrokeopacity{0.500000}%
\pgfsetdash{}{0pt}%
\pgfpathmoveto{\pgfqpoint{3.639727in}{4.708532in}}%
\pgfusepath{stroke}%
\end{pgfscope}%
\begin{pgfscope}%
\pgfpathrectangle{\pgfqpoint{0.100000in}{2.413063in}}{\pgfqpoint{5.037500in}{3.427208in}}%
\pgfusepath{clip}%
\pgfsetbuttcap%
\pgfsetroundjoin%
\definecolor{currentfill}{rgb}{0.000000,0.000000,1.000000}%
\pgfsetfillcolor{currentfill}%
\pgfsetfillopacity{0.500000}%
\pgfsetlinewidth{0.250937pt}%
\definecolor{currentstroke}{rgb}{0.000000,0.000000,0.000000}%
\pgfsetstrokecolor{currentstroke}%
\pgfsetstrokeopacity{0.500000}%
\pgfsetdash{}{0pt}%
\pgfsys@defobject{currentmarker}{\pgfqpoint{-0.058333in}{-0.058333in}}{\pgfqpoint{0.058333in}{0.058333in}}{%
\pgfpathmoveto{\pgfqpoint{0.000000in}{-0.058333in}}%
\pgfpathcurveto{\pgfqpoint{0.015470in}{-0.058333in}}{\pgfqpoint{0.030309in}{-0.052187in}}{\pgfqpoint{0.041248in}{-0.041248in}}%
\pgfpathcurveto{\pgfqpoint{0.052187in}{-0.030309in}}{\pgfqpoint{0.058333in}{-0.015470in}}{\pgfqpoint{0.058333in}{0.000000in}}%
\pgfpathcurveto{\pgfqpoint{0.058333in}{0.015470in}}{\pgfqpoint{0.052187in}{0.030309in}}{\pgfqpoint{0.041248in}{0.041248in}}%
\pgfpathcurveto{\pgfqpoint{0.030309in}{0.052187in}}{\pgfqpoint{0.015470in}{0.058333in}}{\pgfqpoint{0.000000in}{0.058333in}}%
\pgfpathcurveto{\pgfqpoint{-0.015470in}{0.058333in}}{\pgfqpoint{-0.030309in}{0.052187in}}{\pgfqpoint{-0.041248in}{0.041248in}}%
\pgfpathcurveto{\pgfqpoint{-0.052187in}{0.030309in}}{\pgfqpoint{-0.058333in}{0.015470in}}{\pgfqpoint{-0.058333in}{0.000000in}}%
\pgfpathcurveto{\pgfqpoint{-0.058333in}{-0.015470in}}{\pgfqpoint{-0.052187in}{-0.030309in}}{\pgfqpoint{-0.041248in}{-0.041248in}}%
\pgfpathcurveto{\pgfqpoint{-0.030309in}{-0.052187in}}{\pgfqpoint{-0.015470in}{-0.058333in}}{\pgfqpoint{0.000000in}{-0.058333in}}%
\pgfpathclose%
\pgfusepath{stroke,fill}%
}%
\begin{pgfscope}%
\pgfsys@transformshift{3.639727in}{4.708532in}%
\pgfsys@useobject{currentmarker}{}%
\end{pgfscope}%
\end{pgfscope}%
\begin{pgfscope}%
\pgfpathrectangle{\pgfqpoint{0.100000in}{2.413063in}}{\pgfqpoint{5.037500in}{3.427208in}}%
\pgfusepath{clip}%
\pgfsetrectcap%
\pgfsetroundjoin%
\pgfsetlinewidth{1.505625pt}%
\definecolor{currentstroke}{rgb}{0.000000,0.000000,1.000000}%
\pgfsetstrokecolor{currentstroke}%
\pgfsetstrokeopacity{0.500000}%
\pgfsetdash{}{0pt}%
\pgfpathmoveto{\pgfqpoint{3.731354in}{4.867293in}}%
\pgfusepath{stroke}%
\end{pgfscope}%
\begin{pgfscope}%
\pgfpathrectangle{\pgfqpoint{0.100000in}{2.413063in}}{\pgfqpoint{5.037500in}{3.427208in}}%
\pgfusepath{clip}%
\pgfsetbuttcap%
\pgfsetroundjoin%
\definecolor{currentfill}{rgb}{0.000000,0.000000,1.000000}%
\pgfsetfillcolor{currentfill}%
\pgfsetfillopacity{0.500000}%
\pgfsetlinewidth{0.250937pt}%
\definecolor{currentstroke}{rgb}{0.000000,0.000000,0.000000}%
\pgfsetstrokecolor{currentstroke}%
\pgfsetstrokeopacity{0.500000}%
\pgfsetdash{}{0pt}%
\pgfsys@defobject{currentmarker}{\pgfqpoint{-0.038889in}{-0.038889in}}{\pgfqpoint{0.038889in}{0.038889in}}{%
\pgfpathmoveto{\pgfqpoint{0.000000in}{-0.038889in}}%
\pgfpathcurveto{\pgfqpoint{0.010313in}{-0.038889in}}{\pgfqpoint{0.020206in}{-0.034791in}}{\pgfqpoint{0.027499in}{-0.027499in}}%
\pgfpathcurveto{\pgfqpoint{0.034791in}{-0.020206in}}{\pgfqpoint{0.038889in}{-0.010313in}}{\pgfqpoint{0.038889in}{0.000000in}}%
\pgfpathcurveto{\pgfqpoint{0.038889in}{0.010313in}}{\pgfqpoint{0.034791in}{0.020206in}}{\pgfqpoint{0.027499in}{0.027499in}}%
\pgfpathcurveto{\pgfqpoint{0.020206in}{0.034791in}}{\pgfqpoint{0.010313in}{0.038889in}}{\pgfqpoint{0.000000in}{0.038889in}}%
\pgfpathcurveto{\pgfqpoint{-0.010313in}{0.038889in}}{\pgfqpoint{-0.020206in}{0.034791in}}{\pgfqpoint{-0.027499in}{0.027499in}}%
\pgfpathcurveto{\pgfqpoint{-0.034791in}{0.020206in}}{\pgfqpoint{-0.038889in}{0.010313in}}{\pgfqpoint{-0.038889in}{0.000000in}}%
\pgfpathcurveto{\pgfqpoint{-0.038889in}{-0.010313in}}{\pgfqpoint{-0.034791in}{-0.020206in}}{\pgfqpoint{-0.027499in}{-0.027499in}}%
\pgfpathcurveto{\pgfqpoint{-0.020206in}{-0.034791in}}{\pgfqpoint{-0.010313in}{-0.038889in}}{\pgfqpoint{0.000000in}{-0.038889in}}%
\pgfpathclose%
\pgfusepath{stroke,fill}%
}%
\begin{pgfscope}%
\pgfsys@transformshift{3.731354in}{4.867293in}%
\pgfsys@useobject{currentmarker}{}%
\end{pgfscope}%
\end{pgfscope}%
\begin{pgfscope}%
\pgfpathrectangle{\pgfqpoint{0.100000in}{2.413063in}}{\pgfqpoint{5.037500in}{3.427208in}}%
\pgfusepath{clip}%
\pgfsetrectcap%
\pgfsetroundjoin%
\pgfsetlinewidth{1.505625pt}%
\definecolor{currentstroke}{rgb}{0.000000,0.000000,1.000000}%
\pgfsetstrokecolor{currentstroke}%
\pgfsetstrokeopacity{0.500000}%
\pgfsetdash{}{0pt}%
\pgfpathmoveto{\pgfqpoint{3.819255in}{4.733545in}}%
\pgfusepath{stroke}%
\end{pgfscope}%
\begin{pgfscope}%
\pgfpathrectangle{\pgfqpoint{0.100000in}{2.413063in}}{\pgfqpoint{5.037500in}{3.427208in}}%
\pgfusepath{clip}%
\pgfsetbuttcap%
\pgfsetroundjoin%
\definecolor{currentfill}{rgb}{0.000000,0.000000,1.000000}%
\pgfsetfillcolor{currentfill}%
\pgfsetfillopacity{0.500000}%
\pgfsetlinewidth{0.250937pt}%
\definecolor{currentstroke}{rgb}{0.000000,0.000000,0.000000}%
\pgfsetstrokecolor{currentstroke}%
\pgfsetstrokeopacity{0.500000}%
\pgfsetdash{}{0pt}%
\pgfsys@defobject{currentmarker}{\pgfqpoint{-0.016667in}{-0.016667in}}{\pgfqpoint{0.016667in}{0.016667in}}{%
\pgfpathmoveto{\pgfqpoint{0.000000in}{-0.016667in}}%
\pgfpathcurveto{\pgfqpoint{0.004420in}{-0.016667in}}{\pgfqpoint{0.008660in}{-0.014911in}}{\pgfqpoint{0.011785in}{-0.011785in}}%
\pgfpathcurveto{\pgfqpoint{0.014911in}{-0.008660in}}{\pgfqpoint{0.016667in}{-0.004420in}}{\pgfqpoint{0.016667in}{0.000000in}}%
\pgfpathcurveto{\pgfqpoint{0.016667in}{0.004420in}}{\pgfqpoint{0.014911in}{0.008660in}}{\pgfqpoint{0.011785in}{0.011785in}}%
\pgfpathcurveto{\pgfqpoint{0.008660in}{0.014911in}}{\pgfqpoint{0.004420in}{0.016667in}}{\pgfqpoint{0.000000in}{0.016667in}}%
\pgfpathcurveto{\pgfqpoint{-0.004420in}{0.016667in}}{\pgfqpoint{-0.008660in}{0.014911in}}{\pgfqpoint{-0.011785in}{0.011785in}}%
\pgfpathcurveto{\pgfqpoint{-0.014911in}{0.008660in}}{\pgfqpoint{-0.016667in}{0.004420in}}{\pgfqpoint{-0.016667in}{0.000000in}}%
\pgfpathcurveto{\pgfqpoint{-0.016667in}{-0.004420in}}{\pgfqpoint{-0.014911in}{-0.008660in}}{\pgfqpoint{-0.011785in}{-0.011785in}}%
\pgfpathcurveto{\pgfqpoint{-0.008660in}{-0.014911in}}{\pgfqpoint{-0.004420in}{-0.016667in}}{\pgfqpoint{0.000000in}{-0.016667in}}%
\pgfpathclose%
\pgfusepath{stroke,fill}%
}%
\begin{pgfscope}%
\pgfsys@transformshift{3.819255in}{4.733545in}%
\pgfsys@useobject{currentmarker}{}%
\end{pgfscope}%
\end{pgfscope}%
\begin{pgfscope}%
\pgfpathrectangle{\pgfqpoint{0.100000in}{2.413063in}}{\pgfqpoint{5.037500in}{3.427208in}}%
\pgfusepath{clip}%
\pgfsetrectcap%
\pgfsetroundjoin%
\pgfsetlinewidth{1.505625pt}%
\definecolor{currentstroke}{rgb}{0.000000,0.000000,1.000000}%
\pgfsetstrokecolor{currentstroke}%
\pgfsetstrokeopacity{0.500000}%
\pgfsetdash{}{0pt}%
\pgfpathmoveto{\pgfqpoint{3.753328in}{4.802843in}}%
\pgfusepath{stroke}%
\end{pgfscope}%
\begin{pgfscope}%
\pgfpathrectangle{\pgfqpoint{0.100000in}{2.413063in}}{\pgfqpoint{5.037500in}{3.427208in}}%
\pgfusepath{clip}%
\pgfsetbuttcap%
\pgfsetroundjoin%
\definecolor{currentfill}{rgb}{0.000000,0.000000,1.000000}%
\pgfsetfillcolor{currentfill}%
\pgfsetfillopacity{0.500000}%
\pgfsetlinewidth{0.250937pt}%
\definecolor{currentstroke}{rgb}{0.000000,0.000000,0.000000}%
\pgfsetstrokecolor{currentstroke}%
\pgfsetstrokeopacity{0.500000}%
\pgfsetdash{}{0pt}%
\pgfsys@defobject{currentmarker}{\pgfqpoint{-0.066667in}{-0.066667in}}{\pgfqpoint{0.066667in}{0.066667in}}{%
\pgfpathmoveto{\pgfqpoint{0.000000in}{-0.066667in}}%
\pgfpathcurveto{\pgfqpoint{0.017680in}{-0.066667in}}{\pgfqpoint{0.034639in}{-0.059642in}}{\pgfqpoint{0.047140in}{-0.047140in}}%
\pgfpathcurveto{\pgfqpoint{0.059642in}{-0.034639in}}{\pgfqpoint{0.066667in}{-0.017680in}}{\pgfqpoint{0.066667in}{0.000000in}}%
\pgfpathcurveto{\pgfqpoint{0.066667in}{0.017680in}}{\pgfqpoint{0.059642in}{0.034639in}}{\pgfqpoint{0.047140in}{0.047140in}}%
\pgfpathcurveto{\pgfqpoint{0.034639in}{0.059642in}}{\pgfqpoint{0.017680in}{0.066667in}}{\pgfqpoint{0.000000in}{0.066667in}}%
\pgfpathcurveto{\pgfqpoint{-0.017680in}{0.066667in}}{\pgfqpoint{-0.034639in}{0.059642in}}{\pgfqpoint{-0.047140in}{0.047140in}}%
\pgfpathcurveto{\pgfqpoint{-0.059642in}{0.034639in}}{\pgfqpoint{-0.066667in}{0.017680in}}{\pgfqpoint{-0.066667in}{0.000000in}}%
\pgfpathcurveto{\pgfqpoint{-0.066667in}{-0.017680in}}{\pgfqpoint{-0.059642in}{-0.034639in}}{\pgfqpoint{-0.047140in}{-0.047140in}}%
\pgfpathcurveto{\pgfqpoint{-0.034639in}{-0.059642in}}{\pgfqpoint{-0.017680in}{-0.066667in}}{\pgfqpoint{0.000000in}{-0.066667in}}%
\pgfpathclose%
\pgfusepath{stroke,fill}%
}%
\begin{pgfscope}%
\pgfsys@transformshift{3.753328in}{4.802843in}%
\pgfsys@useobject{currentmarker}{}%
\end{pgfscope}%
\end{pgfscope}%
\begin{pgfscope}%
\pgfpathrectangle{\pgfqpoint{0.100000in}{2.413063in}}{\pgfqpoint{5.037500in}{3.427208in}}%
\pgfusepath{clip}%
\pgfsetrectcap%
\pgfsetroundjoin%
\pgfsetlinewidth{1.505625pt}%
\definecolor{currentstroke}{rgb}{0.000000,0.000000,1.000000}%
\pgfsetstrokecolor{currentstroke}%
\pgfsetstrokeopacity{0.500000}%
\pgfsetdash{}{0pt}%
\pgfpathmoveto{\pgfqpoint{3.590943in}{4.778086in}}%
\pgfusepath{stroke}%
\end{pgfscope}%
\begin{pgfscope}%
\pgfpathrectangle{\pgfqpoint{0.100000in}{2.413063in}}{\pgfqpoint{5.037500in}{3.427208in}}%
\pgfusepath{clip}%
\pgfsetbuttcap%
\pgfsetroundjoin%
\definecolor{currentfill}{rgb}{0.000000,0.000000,1.000000}%
\pgfsetfillcolor{currentfill}%
\pgfsetfillopacity{0.500000}%
\pgfsetlinewidth{0.250937pt}%
\definecolor{currentstroke}{rgb}{0.000000,0.000000,0.000000}%
\pgfsetstrokecolor{currentstroke}%
\pgfsetstrokeopacity{0.500000}%
\pgfsetdash{}{0pt}%
\pgfsys@defobject{currentmarker}{\pgfqpoint{-0.036111in}{-0.036111in}}{\pgfqpoint{0.036111in}{0.036111in}}{%
\pgfpathmoveto{\pgfqpoint{0.000000in}{-0.036111in}}%
\pgfpathcurveto{\pgfqpoint{0.009577in}{-0.036111in}}{\pgfqpoint{0.018763in}{-0.032306in}}{\pgfqpoint{0.025534in}{-0.025534in}}%
\pgfpathcurveto{\pgfqpoint{0.032306in}{-0.018763in}}{\pgfqpoint{0.036111in}{-0.009577in}}{\pgfqpoint{0.036111in}{0.000000in}}%
\pgfpathcurveto{\pgfqpoint{0.036111in}{0.009577in}}{\pgfqpoint{0.032306in}{0.018763in}}{\pgfqpoint{0.025534in}{0.025534in}}%
\pgfpathcurveto{\pgfqpoint{0.018763in}{0.032306in}}{\pgfqpoint{0.009577in}{0.036111in}}{\pgfqpoint{0.000000in}{0.036111in}}%
\pgfpathcurveto{\pgfqpoint{-0.009577in}{0.036111in}}{\pgfqpoint{-0.018763in}{0.032306in}}{\pgfqpoint{-0.025534in}{0.025534in}}%
\pgfpathcurveto{\pgfqpoint{-0.032306in}{0.018763in}}{\pgfqpoint{-0.036111in}{0.009577in}}{\pgfqpoint{-0.036111in}{0.000000in}}%
\pgfpathcurveto{\pgfqpoint{-0.036111in}{-0.009577in}}{\pgfqpoint{-0.032306in}{-0.018763in}}{\pgfqpoint{-0.025534in}{-0.025534in}}%
\pgfpathcurveto{\pgfqpoint{-0.018763in}{-0.032306in}}{\pgfqpoint{-0.009577in}{-0.036111in}}{\pgfqpoint{0.000000in}{-0.036111in}}%
\pgfpathclose%
\pgfusepath{stroke,fill}%
}%
\begin{pgfscope}%
\pgfsys@transformshift{3.590943in}{4.778086in}%
\pgfsys@useobject{currentmarker}{}%
\end{pgfscope}%
\end{pgfscope}%
\begin{pgfscope}%
\pgfpathrectangle{\pgfqpoint{0.100000in}{2.413063in}}{\pgfqpoint{5.037500in}{3.427208in}}%
\pgfusepath{clip}%
\pgfsetrectcap%
\pgfsetroundjoin%
\pgfsetlinewidth{1.505625pt}%
\definecolor{currentstroke}{rgb}{0.000000,0.000000,1.000000}%
\pgfsetstrokecolor{currentstroke}%
\pgfsetstrokeopacity{0.500000}%
\pgfsetdash{}{0pt}%
\pgfpathmoveto{\pgfqpoint{3.706510in}{4.707536in}}%
\pgfusepath{stroke}%
\end{pgfscope}%
\begin{pgfscope}%
\pgfpathrectangle{\pgfqpoint{0.100000in}{2.413063in}}{\pgfqpoint{5.037500in}{3.427208in}}%
\pgfusepath{clip}%
\pgfsetbuttcap%
\pgfsetroundjoin%
\definecolor{currentfill}{rgb}{0.000000,0.000000,1.000000}%
\pgfsetfillcolor{currentfill}%
\pgfsetfillopacity{0.500000}%
\pgfsetlinewidth{0.250937pt}%
\definecolor{currentstroke}{rgb}{0.000000,0.000000,0.000000}%
\pgfsetstrokecolor{currentstroke}%
\pgfsetstrokeopacity{0.500000}%
\pgfsetdash{}{0pt}%
\pgfsys@defobject{currentmarker}{\pgfqpoint{-0.047222in}{-0.047222in}}{\pgfqpoint{0.047222in}{0.047222in}}{%
\pgfpathmoveto{\pgfqpoint{0.000000in}{-0.047222in}}%
\pgfpathcurveto{\pgfqpoint{0.012523in}{-0.047222in}}{\pgfqpoint{0.024536in}{-0.042247in}}{\pgfqpoint{0.033391in}{-0.033391in}}%
\pgfpathcurveto{\pgfqpoint{0.042247in}{-0.024536in}}{\pgfqpoint{0.047222in}{-0.012523in}}{\pgfqpoint{0.047222in}{0.000000in}}%
\pgfpathcurveto{\pgfqpoint{0.047222in}{0.012523in}}{\pgfqpoint{0.042247in}{0.024536in}}{\pgfqpoint{0.033391in}{0.033391in}}%
\pgfpathcurveto{\pgfqpoint{0.024536in}{0.042247in}}{\pgfqpoint{0.012523in}{0.047222in}}{\pgfqpoint{0.000000in}{0.047222in}}%
\pgfpathcurveto{\pgfqpoint{-0.012523in}{0.047222in}}{\pgfqpoint{-0.024536in}{0.042247in}}{\pgfqpoint{-0.033391in}{0.033391in}}%
\pgfpathcurveto{\pgfqpoint{-0.042247in}{0.024536in}}{\pgfqpoint{-0.047222in}{0.012523in}}{\pgfqpoint{-0.047222in}{0.000000in}}%
\pgfpathcurveto{\pgfqpoint{-0.047222in}{-0.012523in}}{\pgfqpoint{-0.042247in}{-0.024536in}}{\pgfqpoint{-0.033391in}{-0.033391in}}%
\pgfpathcurveto{\pgfqpoint{-0.024536in}{-0.042247in}}{\pgfqpoint{-0.012523in}{-0.047222in}}{\pgfqpoint{0.000000in}{-0.047222in}}%
\pgfpathclose%
\pgfusepath{stroke,fill}%
}%
\begin{pgfscope}%
\pgfsys@transformshift{3.706510in}{4.707536in}%
\pgfsys@useobject{currentmarker}{}%
\end{pgfscope}%
\end{pgfscope}%
\begin{pgfscope}%
\pgfpathrectangle{\pgfqpoint{0.100000in}{2.413063in}}{\pgfqpoint{5.037500in}{3.427208in}}%
\pgfusepath{clip}%
\pgfsetrectcap%
\pgfsetroundjoin%
\pgfsetlinewidth{1.505625pt}%
\definecolor{currentstroke}{rgb}{0.000000,0.000000,1.000000}%
\pgfsetstrokecolor{currentstroke}%
\pgfsetstrokeopacity{0.500000}%
\pgfsetdash{}{0pt}%
\pgfpathmoveto{\pgfqpoint{3.605727in}{4.701726in}}%
\pgfusepath{stroke}%
\end{pgfscope}%
\begin{pgfscope}%
\pgfpathrectangle{\pgfqpoint{0.100000in}{2.413063in}}{\pgfqpoint{5.037500in}{3.427208in}}%
\pgfusepath{clip}%
\pgfsetbuttcap%
\pgfsetroundjoin%
\definecolor{currentfill}{rgb}{0.000000,0.000000,1.000000}%
\pgfsetfillcolor{currentfill}%
\pgfsetfillopacity{0.500000}%
\pgfsetlinewidth{0.250937pt}%
\definecolor{currentstroke}{rgb}{0.000000,0.000000,0.000000}%
\pgfsetstrokecolor{currentstroke}%
\pgfsetstrokeopacity{0.500000}%
\pgfsetdash{}{0pt}%
\pgfsys@defobject{currentmarker}{\pgfqpoint{-0.036111in}{-0.036111in}}{\pgfqpoint{0.036111in}{0.036111in}}{%
\pgfpathmoveto{\pgfqpoint{0.000000in}{-0.036111in}}%
\pgfpathcurveto{\pgfqpoint{0.009577in}{-0.036111in}}{\pgfqpoint{0.018763in}{-0.032306in}}{\pgfqpoint{0.025534in}{-0.025534in}}%
\pgfpathcurveto{\pgfqpoint{0.032306in}{-0.018763in}}{\pgfqpoint{0.036111in}{-0.009577in}}{\pgfqpoint{0.036111in}{0.000000in}}%
\pgfpathcurveto{\pgfqpoint{0.036111in}{0.009577in}}{\pgfqpoint{0.032306in}{0.018763in}}{\pgfqpoint{0.025534in}{0.025534in}}%
\pgfpathcurveto{\pgfqpoint{0.018763in}{0.032306in}}{\pgfqpoint{0.009577in}{0.036111in}}{\pgfqpoint{0.000000in}{0.036111in}}%
\pgfpathcurveto{\pgfqpoint{-0.009577in}{0.036111in}}{\pgfqpoint{-0.018763in}{0.032306in}}{\pgfqpoint{-0.025534in}{0.025534in}}%
\pgfpathcurveto{\pgfqpoint{-0.032306in}{0.018763in}}{\pgfqpoint{-0.036111in}{0.009577in}}{\pgfqpoint{-0.036111in}{0.000000in}}%
\pgfpathcurveto{\pgfqpoint{-0.036111in}{-0.009577in}}{\pgfqpoint{-0.032306in}{-0.018763in}}{\pgfqpoint{-0.025534in}{-0.025534in}}%
\pgfpathcurveto{\pgfqpoint{-0.018763in}{-0.032306in}}{\pgfqpoint{-0.009577in}{-0.036111in}}{\pgfqpoint{0.000000in}{-0.036111in}}%
\pgfpathclose%
\pgfusepath{stroke,fill}%
}%
\begin{pgfscope}%
\pgfsys@transformshift{3.605727in}{4.701726in}%
\pgfsys@useobject{currentmarker}{}%
\end{pgfscope}%
\end{pgfscope}%
\begin{pgfscope}%
\pgfpathrectangle{\pgfqpoint{0.100000in}{2.413063in}}{\pgfqpoint{5.037500in}{3.427208in}}%
\pgfusepath{clip}%
\pgfsetrectcap%
\pgfsetroundjoin%
\pgfsetlinewidth{1.505625pt}%
\definecolor{currentstroke}{rgb}{0.000000,0.000000,1.000000}%
\pgfsetstrokecolor{currentstroke}%
\pgfsetstrokeopacity{0.500000}%
\pgfsetdash{}{0pt}%
\pgfpathmoveto{\pgfqpoint{3.687384in}{4.761983in}}%
\pgfusepath{stroke}%
\end{pgfscope}%
\begin{pgfscope}%
\pgfpathrectangle{\pgfqpoint{0.100000in}{2.413063in}}{\pgfqpoint{5.037500in}{3.427208in}}%
\pgfusepath{clip}%
\pgfsetbuttcap%
\pgfsetroundjoin%
\definecolor{currentfill}{rgb}{0.000000,0.000000,1.000000}%
\pgfsetfillcolor{currentfill}%
\pgfsetfillopacity{0.500000}%
\pgfsetlinewidth{0.250937pt}%
\definecolor{currentstroke}{rgb}{0.000000,0.000000,0.000000}%
\pgfsetstrokecolor{currentstroke}%
\pgfsetstrokeopacity{0.500000}%
\pgfsetdash{}{0pt}%
\pgfsys@defobject{currentmarker}{\pgfqpoint{-0.036111in}{-0.036111in}}{\pgfqpoint{0.036111in}{0.036111in}}{%
\pgfpathmoveto{\pgfqpoint{0.000000in}{-0.036111in}}%
\pgfpathcurveto{\pgfqpoint{0.009577in}{-0.036111in}}{\pgfqpoint{0.018763in}{-0.032306in}}{\pgfqpoint{0.025534in}{-0.025534in}}%
\pgfpathcurveto{\pgfqpoint{0.032306in}{-0.018763in}}{\pgfqpoint{0.036111in}{-0.009577in}}{\pgfqpoint{0.036111in}{0.000000in}}%
\pgfpathcurveto{\pgfqpoint{0.036111in}{0.009577in}}{\pgfqpoint{0.032306in}{0.018763in}}{\pgfqpoint{0.025534in}{0.025534in}}%
\pgfpathcurveto{\pgfqpoint{0.018763in}{0.032306in}}{\pgfqpoint{0.009577in}{0.036111in}}{\pgfqpoint{0.000000in}{0.036111in}}%
\pgfpathcurveto{\pgfqpoint{-0.009577in}{0.036111in}}{\pgfqpoint{-0.018763in}{0.032306in}}{\pgfqpoint{-0.025534in}{0.025534in}}%
\pgfpathcurveto{\pgfqpoint{-0.032306in}{0.018763in}}{\pgfqpoint{-0.036111in}{0.009577in}}{\pgfqpoint{-0.036111in}{0.000000in}}%
\pgfpathcurveto{\pgfqpoint{-0.036111in}{-0.009577in}}{\pgfqpoint{-0.032306in}{-0.018763in}}{\pgfqpoint{-0.025534in}{-0.025534in}}%
\pgfpathcurveto{\pgfqpoint{-0.018763in}{-0.032306in}}{\pgfqpoint{-0.009577in}{-0.036111in}}{\pgfqpoint{0.000000in}{-0.036111in}}%
\pgfpathclose%
\pgfusepath{stroke,fill}%
}%
\begin{pgfscope}%
\pgfsys@transformshift{3.687384in}{4.761983in}%
\pgfsys@useobject{currentmarker}{}%
\end{pgfscope}%
\end{pgfscope}%
\begin{pgfscope}%
\pgfpathrectangle{\pgfqpoint{0.100000in}{2.413063in}}{\pgfqpoint{5.037500in}{3.427208in}}%
\pgfusepath{clip}%
\pgfsetrectcap%
\pgfsetroundjoin%
\pgfsetlinewidth{1.505625pt}%
\definecolor{currentstroke}{rgb}{0.000000,0.000000,1.000000}%
\pgfsetstrokecolor{currentstroke}%
\pgfsetstrokeopacity{0.500000}%
\pgfsetdash{}{0pt}%
\pgfpathmoveto{\pgfqpoint{3.701307in}{4.866114in}}%
\pgfusepath{stroke}%
\end{pgfscope}%
\begin{pgfscope}%
\pgfpathrectangle{\pgfqpoint{0.100000in}{2.413063in}}{\pgfqpoint{5.037500in}{3.427208in}}%
\pgfusepath{clip}%
\pgfsetbuttcap%
\pgfsetroundjoin%
\definecolor{currentfill}{rgb}{0.000000,0.000000,1.000000}%
\pgfsetfillcolor{currentfill}%
\pgfsetfillopacity{0.500000}%
\pgfsetlinewidth{0.250937pt}%
\definecolor{currentstroke}{rgb}{0.000000,0.000000,0.000000}%
\pgfsetstrokecolor{currentstroke}%
\pgfsetstrokeopacity{0.500000}%
\pgfsetdash{}{0pt}%
\pgfsys@defobject{currentmarker}{\pgfqpoint{-0.022222in}{-0.022222in}}{\pgfqpoint{0.022222in}{0.022222in}}{%
\pgfpathmoveto{\pgfqpoint{0.000000in}{-0.022222in}}%
\pgfpathcurveto{\pgfqpoint{0.005893in}{-0.022222in}}{\pgfqpoint{0.011546in}{-0.019881in}}{\pgfqpoint{0.015713in}{-0.015713in}}%
\pgfpathcurveto{\pgfqpoint{0.019881in}{-0.011546in}}{\pgfqpoint{0.022222in}{-0.005893in}}{\pgfqpoint{0.022222in}{0.000000in}}%
\pgfpathcurveto{\pgfqpoint{0.022222in}{0.005893in}}{\pgfqpoint{0.019881in}{0.011546in}}{\pgfqpoint{0.015713in}{0.015713in}}%
\pgfpathcurveto{\pgfqpoint{0.011546in}{0.019881in}}{\pgfqpoint{0.005893in}{0.022222in}}{\pgfqpoint{0.000000in}{0.022222in}}%
\pgfpathcurveto{\pgfqpoint{-0.005893in}{0.022222in}}{\pgfqpoint{-0.011546in}{0.019881in}}{\pgfqpoint{-0.015713in}{0.015713in}}%
\pgfpathcurveto{\pgfqpoint{-0.019881in}{0.011546in}}{\pgfqpoint{-0.022222in}{0.005893in}}{\pgfqpoint{-0.022222in}{0.000000in}}%
\pgfpathcurveto{\pgfqpoint{-0.022222in}{-0.005893in}}{\pgfqpoint{-0.019881in}{-0.011546in}}{\pgfqpoint{-0.015713in}{-0.015713in}}%
\pgfpathcurveto{\pgfqpoint{-0.011546in}{-0.019881in}}{\pgfqpoint{-0.005893in}{-0.022222in}}{\pgfqpoint{0.000000in}{-0.022222in}}%
\pgfpathclose%
\pgfusepath{stroke,fill}%
}%
\begin{pgfscope}%
\pgfsys@transformshift{3.701307in}{4.866114in}%
\pgfsys@useobject{currentmarker}{}%
\end{pgfscope}%
\end{pgfscope}%
\begin{pgfscope}%
\pgfpathrectangle{\pgfqpoint{0.100000in}{2.413063in}}{\pgfqpoint{5.037500in}{3.427208in}}%
\pgfusepath{clip}%
\pgfsetrectcap%
\pgfsetroundjoin%
\pgfsetlinewidth{1.505625pt}%
\definecolor{currentstroke}{rgb}{0.000000,0.000000,1.000000}%
\pgfsetstrokecolor{currentstroke}%
\pgfsetstrokeopacity{0.500000}%
\pgfsetdash{}{0pt}%
\pgfpathmoveto{\pgfqpoint{3.796884in}{4.680144in}}%
\pgfusepath{stroke}%
\end{pgfscope}%
\begin{pgfscope}%
\pgfpathrectangle{\pgfqpoint{0.100000in}{2.413063in}}{\pgfqpoint{5.037500in}{3.427208in}}%
\pgfusepath{clip}%
\pgfsetbuttcap%
\pgfsetroundjoin%
\definecolor{currentfill}{rgb}{0.000000,0.000000,1.000000}%
\pgfsetfillcolor{currentfill}%
\pgfsetfillopacity{0.500000}%
\pgfsetlinewidth{0.250937pt}%
\definecolor{currentstroke}{rgb}{0.000000,0.000000,0.000000}%
\pgfsetstrokecolor{currentstroke}%
\pgfsetstrokeopacity{0.500000}%
\pgfsetdash{}{0pt}%
\pgfsys@defobject{currentmarker}{\pgfqpoint{-0.041667in}{-0.041667in}}{\pgfqpoint{0.041667in}{0.041667in}}{%
\pgfpathmoveto{\pgfqpoint{0.000000in}{-0.041667in}}%
\pgfpathcurveto{\pgfqpoint{0.011050in}{-0.041667in}}{\pgfqpoint{0.021649in}{-0.037276in}}{\pgfqpoint{0.029463in}{-0.029463in}}%
\pgfpathcurveto{\pgfqpoint{0.037276in}{-0.021649in}}{\pgfqpoint{0.041667in}{-0.011050in}}{\pgfqpoint{0.041667in}{0.000000in}}%
\pgfpathcurveto{\pgfqpoint{0.041667in}{0.011050in}}{\pgfqpoint{0.037276in}{0.021649in}}{\pgfqpoint{0.029463in}{0.029463in}}%
\pgfpathcurveto{\pgfqpoint{0.021649in}{0.037276in}}{\pgfqpoint{0.011050in}{0.041667in}}{\pgfqpoint{0.000000in}{0.041667in}}%
\pgfpathcurveto{\pgfqpoint{-0.011050in}{0.041667in}}{\pgfqpoint{-0.021649in}{0.037276in}}{\pgfqpoint{-0.029463in}{0.029463in}}%
\pgfpathcurveto{\pgfqpoint{-0.037276in}{0.021649in}}{\pgfqpoint{-0.041667in}{0.011050in}}{\pgfqpoint{-0.041667in}{0.000000in}}%
\pgfpathcurveto{\pgfqpoint{-0.041667in}{-0.011050in}}{\pgfqpoint{-0.037276in}{-0.021649in}}{\pgfqpoint{-0.029463in}{-0.029463in}}%
\pgfpathcurveto{\pgfqpoint{-0.021649in}{-0.037276in}}{\pgfqpoint{-0.011050in}{-0.041667in}}{\pgfqpoint{0.000000in}{-0.041667in}}%
\pgfpathclose%
\pgfusepath{stroke,fill}%
}%
\begin{pgfscope}%
\pgfsys@transformshift{3.796884in}{4.680144in}%
\pgfsys@useobject{currentmarker}{}%
\end{pgfscope}%
\end{pgfscope}%
\begin{pgfscope}%
\pgfpathrectangle{\pgfqpoint{0.100000in}{2.413063in}}{\pgfqpoint{5.037500in}{3.427208in}}%
\pgfusepath{clip}%
\pgfsetrectcap%
\pgfsetroundjoin%
\pgfsetlinewidth{1.505625pt}%
\definecolor{currentstroke}{rgb}{0.000000,0.000000,1.000000}%
\pgfsetstrokecolor{currentstroke}%
\pgfsetstrokeopacity{0.500000}%
\pgfsetdash{}{0pt}%
\pgfpathmoveto{\pgfqpoint{3.539146in}{4.804347in}}%
\pgfusepath{stroke}%
\end{pgfscope}%
\begin{pgfscope}%
\pgfpathrectangle{\pgfqpoint{0.100000in}{2.413063in}}{\pgfqpoint{5.037500in}{3.427208in}}%
\pgfusepath{clip}%
\pgfsetbuttcap%
\pgfsetroundjoin%
\definecolor{currentfill}{rgb}{0.000000,0.000000,1.000000}%
\pgfsetfillcolor{currentfill}%
\pgfsetfillopacity{0.500000}%
\pgfsetlinewidth{0.250937pt}%
\definecolor{currentstroke}{rgb}{0.000000,0.000000,0.000000}%
\pgfsetstrokecolor{currentstroke}%
\pgfsetstrokeopacity{0.500000}%
\pgfsetdash{}{0pt}%
\pgfsys@defobject{currentmarker}{\pgfqpoint{-0.072222in}{-0.072222in}}{\pgfqpoint{0.072222in}{0.072222in}}{%
\pgfpathmoveto{\pgfqpoint{0.000000in}{-0.072222in}}%
\pgfpathcurveto{\pgfqpoint{0.019154in}{-0.072222in}}{\pgfqpoint{0.037525in}{-0.064612in}}{\pgfqpoint{0.051069in}{-0.051069in}}%
\pgfpathcurveto{\pgfqpoint{0.064612in}{-0.037525in}}{\pgfqpoint{0.072222in}{-0.019154in}}{\pgfqpoint{0.072222in}{0.000000in}}%
\pgfpathcurveto{\pgfqpoint{0.072222in}{0.019154in}}{\pgfqpoint{0.064612in}{0.037525in}}{\pgfqpoint{0.051069in}{0.051069in}}%
\pgfpathcurveto{\pgfqpoint{0.037525in}{0.064612in}}{\pgfqpoint{0.019154in}{0.072222in}}{\pgfqpoint{0.000000in}{0.072222in}}%
\pgfpathcurveto{\pgfqpoint{-0.019154in}{0.072222in}}{\pgfqpoint{-0.037525in}{0.064612in}}{\pgfqpoint{-0.051069in}{0.051069in}}%
\pgfpathcurveto{\pgfqpoint{-0.064612in}{0.037525in}}{\pgfqpoint{-0.072222in}{0.019154in}}{\pgfqpoint{-0.072222in}{0.000000in}}%
\pgfpathcurveto{\pgfqpoint{-0.072222in}{-0.019154in}}{\pgfqpoint{-0.064612in}{-0.037525in}}{\pgfqpoint{-0.051069in}{-0.051069in}}%
\pgfpathcurveto{\pgfqpoint{-0.037525in}{-0.064612in}}{\pgfqpoint{-0.019154in}{-0.072222in}}{\pgfqpoint{0.000000in}{-0.072222in}}%
\pgfpathclose%
\pgfusepath{stroke,fill}%
}%
\begin{pgfscope}%
\pgfsys@transformshift{3.539146in}{4.804347in}%
\pgfsys@useobject{currentmarker}{}%
\end{pgfscope}%
\end{pgfscope}%
\begin{pgfscope}%
\pgfpathrectangle{\pgfqpoint{0.100000in}{2.413063in}}{\pgfqpoint{5.037500in}{3.427208in}}%
\pgfusepath{clip}%
\pgfsetrectcap%
\pgfsetroundjoin%
\pgfsetlinewidth{1.505625pt}%
\definecolor{currentstroke}{rgb}{0.000000,0.000000,1.000000}%
\pgfsetstrokecolor{currentstroke}%
\pgfsetstrokeopacity{0.500000}%
\pgfsetdash{}{0pt}%
\pgfpathmoveto{\pgfqpoint{3.554230in}{4.643033in}}%
\pgfusepath{stroke}%
\end{pgfscope}%
\begin{pgfscope}%
\pgfpathrectangle{\pgfqpoint{0.100000in}{2.413063in}}{\pgfqpoint{5.037500in}{3.427208in}}%
\pgfusepath{clip}%
\pgfsetbuttcap%
\pgfsetroundjoin%
\definecolor{currentfill}{rgb}{0.000000,0.000000,1.000000}%
\pgfsetfillcolor{currentfill}%
\pgfsetfillopacity{0.500000}%
\pgfsetlinewidth{0.250937pt}%
\definecolor{currentstroke}{rgb}{0.000000,0.000000,0.000000}%
\pgfsetstrokecolor{currentstroke}%
\pgfsetstrokeopacity{0.500000}%
\pgfsetdash{}{0pt}%
\pgfsys@defobject{currentmarker}{\pgfqpoint{-0.036111in}{-0.036111in}}{\pgfqpoint{0.036111in}{0.036111in}}{%
\pgfpathmoveto{\pgfqpoint{0.000000in}{-0.036111in}}%
\pgfpathcurveto{\pgfqpoint{0.009577in}{-0.036111in}}{\pgfqpoint{0.018763in}{-0.032306in}}{\pgfqpoint{0.025534in}{-0.025534in}}%
\pgfpathcurveto{\pgfqpoint{0.032306in}{-0.018763in}}{\pgfqpoint{0.036111in}{-0.009577in}}{\pgfqpoint{0.036111in}{0.000000in}}%
\pgfpathcurveto{\pgfqpoint{0.036111in}{0.009577in}}{\pgfqpoint{0.032306in}{0.018763in}}{\pgfqpoint{0.025534in}{0.025534in}}%
\pgfpathcurveto{\pgfqpoint{0.018763in}{0.032306in}}{\pgfqpoint{0.009577in}{0.036111in}}{\pgfqpoint{0.000000in}{0.036111in}}%
\pgfpathcurveto{\pgfqpoint{-0.009577in}{0.036111in}}{\pgfqpoint{-0.018763in}{0.032306in}}{\pgfqpoint{-0.025534in}{0.025534in}}%
\pgfpathcurveto{\pgfqpoint{-0.032306in}{0.018763in}}{\pgfqpoint{-0.036111in}{0.009577in}}{\pgfqpoint{-0.036111in}{0.000000in}}%
\pgfpathcurveto{\pgfqpoint{-0.036111in}{-0.009577in}}{\pgfqpoint{-0.032306in}{-0.018763in}}{\pgfqpoint{-0.025534in}{-0.025534in}}%
\pgfpathcurveto{\pgfqpoint{-0.018763in}{-0.032306in}}{\pgfqpoint{-0.009577in}{-0.036111in}}{\pgfqpoint{0.000000in}{-0.036111in}}%
\pgfpathclose%
\pgfusepath{stroke,fill}%
}%
\begin{pgfscope}%
\pgfsys@transformshift{3.554230in}{4.643033in}%
\pgfsys@useobject{currentmarker}{}%
\end{pgfscope}%
\end{pgfscope}%
\begin{pgfscope}%
\pgfpathrectangle{\pgfqpoint{0.100000in}{2.413063in}}{\pgfqpoint{5.037500in}{3.427208in}}%
\pgfusepath{clip}%
\pgfsetrectcap%
\pgfsetroundjoin%
\pgfsetlinewidth{1.505625pt}%
\definecolor{currentstroke}{rgb}{0.000000,0.000000,1.000000}%
\pgfsetstrokecolor{currentstroke}%
\pgfsetstrokeopacity{0.500000}%
\pgfsetdash{}{0pt}%
\pgfpathmoveto{\pgfqpoint{3.728805in}{4.846680in}}%
\pgfusepath{stroke}%
\end{pgfscope}%
\begin{pgfscope}%
\pgfpathrectangle{\pgfqpoint{0.100000in}{2.413063in}}{\pgfqpoint{5.037500in}{3.427208in}}%
\pgfusepath{clip}%
\pgfsetbuttcap%
\pgfsetroundjoin%
\definecolor{currentfill}{rgb}{0.000000,0.000000,1.000000}%
\pgfsetfillcolor{currentfill}%
\pgfsetfillopacity{0.500000}%
\pgfsetlinewidth{0.250937pt}%
\definecolor{currentstroke}{rgb}{0.000000,0.000000,0.000000}%
\pgfsetstrokecolor{currentstroke}%
\pgfsetstrokeopacity{0.500000}%
\pgfsetdash{}{0pt}%
\pgfsys@defobject{currentmarker}{\pgfqpoint{-0.052778in}{-0.052778in}}{\pgfqpoint{0.052778in}{0.052778in}}{%
\pgfpathmoveto{\pgfqpoint{0.000000in}{-0.052778in}}%
\pgfpathcurveto{\pgfqpoint{0.013997in}{-0.052778in}}{\pgfqpoint{0.027422in}{-0.047217in}}{\pgfqpoint{0.037320in}{-0.037320in}}%
\pgfpathcurveto{\pgfqpoint{0.047217in}{-0.027422in}}{\pgfqpoint{0.052778in}{-0.013997in}}{\pgfqpoint{0.052778in}{0.000000in}}%
\pgfpathcurveto{\pgfqpoint{0.052778in}{0.013997in}}{\pgfqpoint{0.047217in}{0.027422in}}{\pgfqpoint{0.037320in}{0.037320in}}%
\pgfpathcurveto{\pgfqpoint{0.027422in}{0.047217in}}{\pgfqpoint{0.013997in}{0.052778in}}{\pgfqpoint{0.000000in}{0.052778in}}%
\pgfpathcurveto{\pgfqpoint{-0.013997in}{0.052778in}}{\pgfqpoint{-0.027422in}{0.047217in}}{\pgfqpoint{-0.037320in}{0.037320in}}%
\pgfpathcurveto{\pgfqpoint{-0.047217in}{0.027422in}}{\pgfqpoint{-0.052778in}{0.013997in}}{\pgfqpoint{-0.052778in}{0.000000in}}%
\pgfpathcurveto{\pgfqpoint{-0.052778in}{-0.013997in}}{\pgfqpoint{-0.047217in}{-0.027422in}}{\pgfqpoint{-0.037320in}{-0.037320in}}%
\pgfpathcurveto{\pgfqpoint{-0.027422in}{-0.047217in}}{\pgfqpoint{-0.013997in}{-0.052778in}}{\pgfqpoint{0.000000in}{-0.052778in}}%
\pgfpathclose%
\pgfusepath{stroke,fill}%
}%
\begin{pgfscope}%
\pgfsys@transformshift{3.728805in}{4.846680in}%
\pgfsys@useobject{currentmarker}{}%
\end{pgfscope}%
\end{pgfscope}%
\begin{pgfscope}%
\pgfpathrectangle{\pgfqpoint{0.100000in}{2.413063in}}{\pgfqpoint{5.037500in}{3.427208in}}%
\pgfusepath{clip}%
\pgfsetrectcap%
\pgfsetroundjoin%
\pgfsetlinewidth{1.505625pt}%
\definecolor{currentstroke}{rgb}{0.678431,1.000000,0.184314}%
\pgfsetstrokecolor{currentstroke}%
\pgfsetstrokeopacity{0.500000}%
\pgfsetdash{}{0pt}%
\pgfpathmoveto{\pgfqpoint{3.032951in}{5.182346in}}%
\pgfusepath{stroke}%
\end{pgfscope}%
\begin{pgfscope}%
\pgfpathrectangle{\pgfqpoint{0.100000in}{2.413063in}}{\pgfqpoint{5.037500in}{3.427208in}}%
\pgfusepath{clip}%
\pgfsetbuttcap%
\pgfsetroundjoin%
\definecolor{currentfill}{rgb}{0.678431,1.000000,0.184314}%
\pgfsetfillcolor{currentfill}%
\pgfsetfillopacity{0.500000}%
\pgfsetlinewidth{0.250937pt}%
\definecolor{currentstroke}{rgb}{0.000000,0.000000,0.000000}%
\pgfsetstrokecolor{currentstroke}%
\pgfsetstrokeopacity{0.500000}%
\pgfsetdash{}{0pt}%
\pgfsys@defobject{currentmarker}{\pgfqpoint{-0.022222in}{-0.022222in}}{\pgfqpoint{0.022222in}{0.022222in}}{%
\pgfpathmoveto{\pgfqpoint{0.000000in}{-0.022222in}}%
\pgfpathcurveto{\pgfqpoint{0.005893in}{-0.022222in}}{\pgfqpoint{0.011546in}{-0.019881in}}{\pgfqpoint{0.015713in}{-0.015713in}}%
\pgfpathcurveto{\pgfqpoint{0.019881in}{-0.011546in}}{\pgfqpoint{0.022222in}{-0.005893in}}{\pgfqpoint{0.022222in}{0.000000in}}%
\pgfpathcurveto{\pgfqpoint{0.022222in}{0.005893in}}{\pgfqpoint{0.019881in}{0.011546in}}{\pgfqpoint{0.015713in}{0.015713in}}%
\pgfpathcurveto{\pgfqpoint{0.011546in}{0.019881in}}{\pgfqpoint{0.005893in}{0.022222in}}{\pgfqpoint{0.000000in}{0.022222in}}%
\pgfpathcurveto{\pgfqpoint{-0.005893in}{0.022222in}}{\pgfqpoint{-0.011546in}{0.019881in}}{\pgfqpoint{-0.015713in}{0.015713in}}%
\pgfpathcurveto{\pgfqpoint{-0.019881in}{0.011546in}}{\pgfqpoint{-0.022222in}{0.005893in}}{\pgfqpoint{-0.022222in}{0.000000in}}%
\pgfpathcurveto{\pgfqpoint{-0.022222in}{-0.005893in}}{\pgfqpoint{-0.019881in}{-0.011546in}}{\pgfqpoint{-0.015713in}{-0.015713in}}%
\pgfpathcurveto{\pgfqpoint{-0.011546in}{-0.019881in}}{\pgfqpoint{-0.005893in}{-0.022222in}}{\pgfqpoint{0.000000in}{-0.022222in}}%
\pgfpathclose%
\pgfusepath{stroke,fill}%
}%
\begin{pgfscope}%
\pgfsys@transformshift{3.032951in}{5.182346in}%
\pgfsys@useobject{currentmarker}{}%
\end{pgfscope}%
\end{pgfscope}%
\begin{pgfscope}%
\pgfpathrectangle{\pgfqpoint{0.100000in}{2.413063in}}{\pgfqpoint{5.037500in}{3.427208in}}%
\pgfusepath{clip}%
\pgfsetrectcap%
\pgfsetroundjoin%
\pgfsetlinewidth{1.505625pt}%
\definecolor{currentstroke}{rgb}{0.501961,0.501961,0.501961}%
\pgfsetstrokecolor{currentstroke}%
\pgfsetstrokeopacity{0.500000}%
\pgfsetdash{}{0pt}%
\pgfpathmoveto{\pgfqpoint{2.887116in}{4.876746in}}%
\pgfusepath{stroke}%
\end{pgfscope}%
\begin{pgfscope}%
\pgfpathrectangle{\pgfqpoint{0.100000in}{2.413063in}}{\pgfqpoint{5.037500in}{3.427208in}}%
\pgfusepath{clip}%
\pgfsetbuttcap%
\pgfsetroundjoin%
\definecolor{currentfill}{rgb}{0.501961,0.501961,0.501961}%
\pgfsetfillcolor{currentfill}%
\pgfsetfillopacity{0.500000}%
\pgfsetlinewidth{0.250937pt}%
\definecolor{currentstroke}{rgb}{0.000000,0.000000,0.000000}%
\pgfsetstrokecolor{currentstroke}%
\pgfsetstrokeopacity{0.500000}%
\pgfsetdash{}{0pt}%
\pgfsys@defobject{currentmarker}{\pgfqpoint{-0.013889in}{-0.013889in}}{\pgfqpoint{0.013889in}{0.013889in}}{%
\pgfpathmoveto{\pgfqpoint{0.000000in}{-0.013889in}}%
\pgfpathcurveto{\pgfqpoint{0.003683in}{-0.013889in}}{\pgfqpoint{0.007216in}{-0.012425in}}{\pgfqpoint{0.009821in}{-0.009821in}}%
\pgfpathcurveto{\pgfqpoint{0.012425in}{-0.007216in}}{\pgfqpoint{0.013889in}{-0.003683in}}{\pgfqpoint{0.013889in}{0.000000in}}%
\pgfpathcurveto{\pgfqpoint{0.013889in}{0.003683in}}{\pgfqpoint{0.012425in}{0.007216in}}{\pgfqpoint{0.009821in}{0.009821in}}%
\pgfpathcurveto{\pgfqpoint{0.007216in}{0.012425in}}{\pgfqpoint{0.003683in}{0.013889in}}{\pgfqpoint{0.000000in}{0.013889in}}%
\pgfpathcurveto{\pgfqpoint{-0.003683in}{0.013889in}}{\pgfqpoint{-0.007216in}{0.012425in}}{\pgfqpoint{-0.009821in}{0.009821in}}%
\pgfpathcurveto{\pgfqpoint{-0.012425in}{0.007216in}}{\pgfqpoint{-0.013889in}{0.003683in}}{\pgfqpoint{-0.013889in}{0.000000in}}%
\pgfpathcurveto{\pgfqpoint{-0.013889in}{-0.003683in}}{\pgfqpoint{-0.012425in}{-0.007216in}}{\pgfqpoint{-0.009821in}{-0.009821in}}%
\pgfpathcurveto{\pgfqpoint{-0.007216in}{-0.012425in}}{\pgfqpoint{-0.003683in}{-0.013889in}}{\pgfqpoint{0.000000in}{-0.013889in}}%
\pgfpathclose%
\pgfusepath{stroke,fill}%
}%
\begin{pgfscope}%
\pgfsys@transformshift{2.887116in}{4.876746in}%
\pgfsys@useobject{currentmarker}{}%
\end{pgfscope}%
\end{pgfscope}%
\begin{pgfscope}%
\pgfpathrectangle{\pgfqpoint{0.100000in}{2.413063in}}{\pgfqpoint{5.037500in}{3.427208in}}%
\pgfusepath{clip}%
\pgfsetrectcap%
\pgfsetroundjoin%
\pgfsetlinewidth{1.505625pt}%
\definecolor{currentstroke}{rgb}{0.678431,1.000000,0.184314}%
\pgfsetstrokecolor{currentstroke}%
\pgfsetstrokeopacity{0.500000}%
\pgfsetdash{}{0pt}%
\pgfpathmoveto{\pgfqpoint{2.946176in}{4.971799in}}%
\pgfusepath{stroke}%
\end{pgfscope}%
\begin{pgfscope}%
\pgfpathrectangle{\pgfqpoint{0.100000in}{2.413063in}}{\pgfqpoint{5.037500in}{3.427208in}}%
\pgfusepath{clip}%
\pgfsetbuttcap%
\pgfsetroundjoin%
\definecolor{currentfill}{rgb}{0.678431,1.000000,0.184314}%
\pgfsetfillcolor{currentfill}%
\pgfsetfillopacity{0.500000}%
\pgfsetlinewidth{0.250937pt}%
\definecolor{currentstroke}{rgb}{0.000000,0.000000,0.000000}%
\pgfsetstrokecolor{currentstroke}%
\pgfsetstrokeopacity{0.500000}%
\pgfsetdash{}{0pt}%
\pgfsys@defobject{currentmarker}{\pgfqpoint{-0.008333in}{-0.008333in}}{\pgfqpoint{0.008333in}{0.008333in}}{%
\pgfpathmoveto{\pgfqpoint{0.000000in}{-0.008333in}}%
\pgfpathcurveto{\pgfqpoint{0.002210in}{-0.008333in}}{\pgfqpoint{0.004330in}{-0.007455in}}{\pgfqpoint{0.005893in}{-0.005893in}}%
\pgfpathcurveto{\pgfqpoint{0.007455in}{-0.004330in}}{\pgfqpoint{0.008333in}{-0.002210in}}{\pgfqpoint{0.008333in}{0.000000in}}%
\pgfpathcurveto{\pgfqpoint{0.008333in}{0.002210in}}{\pgfqpoint{0.007455in}{0.004330in}}{\pgfqpoint{0.005893in}{0.005893in}}%
\pgfpathcurveto{\pgfqpoint{0.004330in}{0.007455in}}{\pgfqpoint{0.002210in}{0.008333in}}{\pgfqpoint{0.000000in}{0.008333in}}%
\pgfpathcurveto{\pgfqpoint{-0.002210in}{0.008333in}}{\pgfqpoint{-0.004330in}{0.007455in}}{\pgfqpoint{-0.005893in}{0.005893in}}%
\pgfpathcurveto{\pgfqpoint{-0.007455in}{0.004330in}}{\pgfqpoint{-0.008333in}{0.002210in}}{\pgfqpoint{-0.008333in}{0.000000in}}%
\pgfpathcurveto{\pgfqpoint{-0.008333in}{-0.002210in}}{\pgfqpoint{-0.007455in}{-0.004330in}}{\pgfqpoint{-0.005893in}{-0.005893in}}%
\pgfpathcurveto{\pgfqpoint{-0.004330in}{-0.007455in}}{\pgfqpoint{-0.002210in}{-0.008333in}}{\pgfqpoint{0.000000in}{-0.008333in}}%
\pgfpathclose%
\pgfusepath{stroke,fill}%
}%
\begin{pgfscope}%
\pgfsys@transformshift{2.946176in}{4.971799in}%
\pgfsys@useobject{currentmarker}{}%
\end{pgfscope}%
\end{pgfscope}%
\begin{pgfscope}%
\pgfpathrectangle{\pgfqpoint{0.100000in}{2.413063in}}{\pgfqpoint{5.037500in}{3.427208in}}%
\pgfusepath{clip}%
\pgfsetrectcap%
\pgfsetroundjoin%
\pgfsetlinewidth{1.505625pt}%
\definecolor{currentstroke}{rgb}{0.678431,1.000000,0.184314}%
\pgfsetstrokecolor{currentstroke}%
\pgfsetstrokeopacity{0.500000}%
\pgfsetdash{}{0pt}%
\pgfpathmoveto{\pgfqpoint{3.014958in}{4.863032in}}%
\pgfusepath{stroke}%
\end{pgfscope}%
\begin{pgfscope}%
\pgfpathrectangle{\pgfqpoint{0.100000in}{2.413063in}}{\pgfqpoint{5.037500in}{3.427208in}}%
\pgfusepath{clip}%
\pgfsetbuttcap%
\pgfsetroundjoin%
\definecolor{currentfill}{rgb}{0.678431,1.000000,0.184314}%
\pgfsetfillcolor{currentfill}%
\pgfsetfillopacity{0.500000}%
\pgfsetlinewidth{0.250937pt}%
\definecolor{currentstroke}{rgb}{0.000000,0.000000,0.000000}%
\pgfsetstrokecolor{currentstroke}%
\pgfsetstrokeopacity{0.500000}%
\pgfsetdash{}{0pt}%
\pgfsys@defobject{currentmarker}{\pgfqpoint{-0.011111in}{-0.011111in}}{\pgfqpoint{0.011111in}{0.011111in}}{%
\pgfpathmoveto{\pgfqpoint{0.000000in}{-0.011111in}}%
\pgfpathcurveto{\pgfqpoint{0.002947in}{-0.011111in}}{\pgfqpoint{0.005773in}{-0.009940in}}{\pgfqpoint{0.007857in}{-0.007857in}}%
\pgfpathcurveto{\pgfqpoint{0.009940in}{-0.005773in}}{\pgfqpoint{0.011111in}{-0.002947in}}{\pgfqpoint{0.011111in}{0.000000in}}%
\pgfpathcurveto{\pgfqpoint{0.011111in}{0.002947in}}{\pgfqpoint{0.009940in}{0.005773in}}{\pgfqpoint{0.007857in}{0.007857in}}%
\pgfpathcurveto{\pgfqpoint{0.005773in}{0.009940in}}{\pgfqpoint{0.002947in}{0.011111in}}{\pgfqpoint{0.000000in}{0.011111in}}%
\pgfpathcurveto{\pgfqpoint{-0.002947in}{0.011111in}}{\pgfqpoint{-0.005773in}{0.009940in}}{\pgfqpoint{-0.007857in}{0.007857in}}%
\pgfpathcurveto{\pgfqpoint{-0.009940in}{0.005773in}}{\pgfqpoint{-0.011111in}{0.002947in}}{\pgfqpoint{-0.011111in}{0.000000in}}%
\pgfpathcurveto{\pgfqpoint{-0.011111in}{-0.002947in}}{\pgfqpoint{-0.009940in}{-0.005773in}}{\pgfqpoint{-0.007857in}{-0.007857in}}%
\pgfpathcurveto{\pgfqpoint{-0.005773in}{-0.009940in}}{\pgfqpoint{-0.002947in}{-0.011111in}}{\pgfqpoint{0.000000in}{-0.011111in}}%
\pgfpathclose%
\pgfusepath{stroke,fill}%
}%
\begin{pgfscope}%
\pgfsys@transformshift{3.014958in}{4.863032in}%
\pgfsys@useobject{currentmarker}{}%
\end{pgfscope}%
\end{pgfscope}%
\begin{pgfscope}%
\pgfpathrectangle{\pgfqpoint{0.100000in}{2.413063in}}{\pgfqpoint{5.037500in}{3.427208in}}%
\pgfusepath{clip}%
\pgfsetrectcap%
\pgfsetroundjoin%
\pgfsetlinewidth{1.505625pt}%
\definecolor{currentstroke}{rgb}{0.678431,1.000000,0.184314}%
\pgfsetstrokecolor{currentstroke}%
\pgfsetstrokeopacity{0.500000}%
\pgfsetdash{}{0pt}%
\pgfpathmoveto{\pgfqpoint{2.871956in}{5.038443in}}%
\pgfusepath{stroke}%
\end{pgfscope}%
\begin{pgfscope}%
\pgfpathrectangle{\pgfqpoint{0.100000in}{2.413063in}}{\pgfqpoint{5.037500in}{3.427208in}}%
\pgfusepath{clip}%
\pgfsetbuttcap%
\pgfsetroundjoin%
\definecolor{currentfill}{rgb}{0.678431,1.000000,0.184314}%
\pgfsetfillcolor{currentfill}%
\pgfsetfillopacity{0.500000}%
\pgfsetlinewidth{0.250937pt}%
\definecolor{currentstroke}{rgb}{0.000000,0.000000,0.000000}%
\pgfsetstrokecolor{currentstroke}%
\pgfsetstrokeopacity{0.500000}%
\pgfsetdash{}{0pt}%
\pgfsys@defobject{currentmarker}{\pgfqpoint{-0.011111in}{-0.011111in}}{\pgfqpoint{0.011111in}{0.011111in}}{%
\pgfpathmoveto{\pgfqpoint{0.000000in}{-0.011111in}}%
\pgfpathcurveto{\pgfqpoint{0.002947in}{-0.011111in}}{\pgfqpoint{0.005773in}{-0.009940in}}{\pgfqpoint{0.007857in}{-0.007857in}}%
\pgfpathcurveto{\pgfqpoint{0.009940in}{-0.005773in}}{\pgfqpoint{0.011111in}{-0.002947in}}{\pgfqpoint{0.011111in}{0.000000in}}%
\pgfpathcurveto{\pgfqpoint{0.011111in}{0.002947in}}{\pgfqpoint{0.009940in}{0.005773in}}{\pgfqpoint{0.007857in}{0.007857in}}%
\pgfpathcurveto{\pgfqpoint{0.005773in}{0.009940in}}{\pgfqpoint{0.002947in}{0.011111in}}{\pgfqpoint{0.000000in}{0.011111in}}%
\pgfpathcurveto{\pgfqpoint{-0.002947in}{0.011111in}}{\pgfqpoint{-0.005773in}{0.009940in}}{\pgfqpoint{-0.007857in}{0.007857in}}%
\pgfpathcurveto{\pgfqpoint{-0.009940in}{0.005773in}}{\pgfqpoint{-0.011111in}{0.002947in}}{\pgfqpoint{-0.011111in}{0.000000in}}%
\pgfpathcurveto{\pgfqpoint{-0.011111in}{-0.002947in}}{\pgfqpoint{-0.009940in}{-0.005773in}}{\pgfqpoint{-0.007857in}{-0.007857in}}%
\pgfpathcurveto{\pgfqpoint{-0.005773in}{-0.009940in}}{\pgfqpoint{-0.002947in}{-0.011111in}}{\pgfqpoint{0.000000in}{-0.011111in}}%
\pgfpathclose%
\pgfusepath{stroke,fill}%
}%
\begin{pgfscope}%
\pgfsys@transformshift{2.871956in}{5.038443in}%
\pgfsys@useobject{currentmarker}{}%
\end{pgfscope}%
\end{pgfscope}%
\begin{pgfscope}%
\pgfpathrectangle{\pgfqpoint{0.100000in}{2.413063in}}{\pgfqpoint{5.037500in}{3.427208in}}%
\pgfusepath{clip}%
\pgfsetrectcap%
\pgfsetroundjoin%
\pgfsetlinewidth{1.505625pt}%
\definecolor{currentstroke}{rgb}{0.678431,1.000000,0.184314}%
\pgfsetstrokecolor{currentstroke}%
\pgfsetstrokeopacity{0.500000}%
\pgfsetdash{}{0pt}%
\pgfpathmoveto{\pgfqpoint{3.397181in}{3.302431in}}%
\pgfusepath{stroke}%
\end{pgfscope}%
\begin{pgfscope}%
\pgfpathrectangle{\pgfqpoint{0.100000in}{2.413063in}}{\pgfqpoint{5.037500in}{3.427208in}}%
\pgfusepath{clip}%
\pgfsetbuttcap%
\pgfsetroundjoin%
\definecolor{currentfill}{rgb}{0.678431,1.000000,0.184314}%
\pgfsetfillcolor{currentfill}%
\pgfsetfillopacity{0.500000}%
\pgfsetlinewidth{0.250937pt}%
\definecolor{currentstroke}{rgb}{0.000000,0.000000,0.000000}%
\pgfsetstrokecolor{currentstroke}%
\pgfsetstrokeopacity{0.500000}%
\pgfsetdash{}{0pt}%
\pgfsys@defobject{currentmarker}{\pgfqpoint{-0.055556in}{-0.055556in}}{\pgfqpoint{0.055556in}{0.055556in}}{%
\pgfpathmoveto{\pgfqpoint{0.000000in}{-0.055556in}}%
\pgfpathcurveto{\pgfqpoint{0.014734in}{-0.055556in}}{\pgfqpoint{0.028866in}{-0.049702in}}{\pgfqpoint{0.039284in}{-0.039284in}}%
\pgfpathcurveto{\pgfqpoint{0.049702in}{-0.028866in}}{\pgfqpoint{0.055556in}{-0.014734in}}{\pgfqpoint{0.055556in}{0.000000in}}%
\pgfpathcurveto{\pgfqpoint{0.055556in}{0.014734in}}{\pgfqpoint{0.049702in}{0.028866in}}{\pgfqpoint{0.039284in}{0.039284in}}%
\pgfpathcurveto{\pgfqpoint{0.028866in}{0.049702in}}{\pgfqpoint{0.014734in}{0.055556in}}{\pgfqpoint{0.000000in}{0.055556in}}%
\pgfpathcurveto{\pgfqpoint{-0.014734in}{0.055556in}}{\pgfqpoint{-0.028866in}{0.049702in}}{\pgfqpoint{-0.039284in}{0.039284in}}%
\pgfpathcurveto{\pgfqpoint{-0.049702in}{0.028866in}}{\pgfqpoint{-0.055556in}{0.014734in}}{\pgfqpoint{-0.055556in}{0.000000in}}%
\pgfpathcurveto{\pgfqpoint{-0.055556in}{-0.014734in}}{\pgfqpoint{-0.049702in}{-0.028866in}}{\pgfqpoint{-0.039284in}{-0.039284in}}%
\pgfpathcurveto{\pgfqpoint{-0.028866in}{-0.049702in}}{\pgfqpoint{-0.014734in}{-0.055556in}}{\pgfqpoint{0.000000in}{-0.055556in}}%
\pgfpathclose%
\pgfusepath{stroke,fill}%
}%
\begin{pgfscope}%
\pgfsys@transformshift{3.397181in}{3.302431in}%
\pgfsys@useobject{currentmarker}{}%
\end{pgfscope}%
\end{pgfscope}%
\begin{pgfscope}%
\pgfpathrectangle{\pgfqpoint{0.100000in}{2.413063in}}{\pgfqpoint{5.037500in}{3.427208in}}%
\pgfusepath{clip}%
\pgfsetrectcap%
\pgfsetroundjoin%
\pgfsetlinewidth{1.505625pt}%
\definecolor{currentstroke}{rgb}{0.678431,1.000000,0.184314}%
\pgfsetstrokecolor{currentstroke}%
\pgfsetstrokeopacity{0.500000}%
\pgfsetdash{}{0pt}%
\pgfpathmoveto{\pgfqpoint{3.370369in}{3.412562in}}%
\pgfusepath{stroke}%
\end{pgfscope}%
\begin{pgfscope}%
\pgfpathrectangle{\pgfqpoint{0.100000in}{2.413063in}}{\pgfqpoint{5.037500in}{3.427208in}}%
\pgfusepath{clip}%
\pgfsetbuttcap%
\pgfsetroundjoin%
\definecolor{currentfill}{rgb}{0.678431,1.000000,0.184314}%
\pgfsetfillcolor{currentfill}%
\pgfsetfillopacity{0.500000}%
\pgfsetlinewidth{0.250937pt}%
\definecolor{currentstroke}{rgb}{0.000000,0.000000,0.000000}%
\pgfsetstrokecolor{currentstroke}%
\pgfsetstrokeopacity{0.500000}%
\pgfsetdash{}{0pt}%
\pgfsys@defobject{currentmarker}{\pgfqpoint{-0.044444in}{-0.044444in}}{\pgfqpoint{0.044444in}{0.044444in}}{%
\pgfpathmoveto{\pgfqpoint{0.000000in}{-0.044444in}}%
\pgfpathcurveto{\pgfqpoint{0.011787in}{-0.044444in}}{\pgfqpoint{0.023092in}{-0.039761in}}{\pgfqpoint{0.031427in}{-0.031427in}}%
\pgfpathcurveto{\pgfqpoint{0.039761in}{-0.023092in}}{\pgfqpoint{0.044444in}{-0.011787in}}{\pgfqpoint{0.044444in}{0.000000in}}%
\pgfpathcurveto{\pgfqpoint{0.044444in}{0.011787in}}{\pgfqpoint{0.039761in}{0.023092in}}{\pgfqpoint{0.031427in}{0.031427in}}%
\pgfpathcurveto{\pgfqpoint{0.023092in}{0.039761in}}{\pgfqpoint{0.011787in}{0.044444in}}{\pgfqpoint{0.000000in}{0.044444in}}%
\pgfpathcurveto{\pgfqpoint{-0.011787in}{0.044444in}}{\pgfqpoint{-0.023092in}{0.039761in}}{\pgfqpoint{-0.031427in}{0.031427in}}%
\pgfpathcurveto{\pgfqpoint{-0.039761in}{0.023092in}}{\pgfqpoint{-0.044444in}{0.011787in}}{\pgfqpoint{-0.044444in}{0.000000in}}%
\pgfpathcurveto{\pgfqpoint{-0.044444in}{-0.011787in}}{\pgfqpoint{-0.039761in}{-0.023092in}}{\pgfqpoint{-0.031427in}{-0.031427in}}%
\pgfpathcurveto{\pgfqpoint{-0.023092in}{-0.039761in}}{\pgfqpoint{-0.011787in}{-0.044444in}}{\pgfqpoint{0.000000in}{-0.044444in}}%
\pgfpathclose%
\pgfusepath{stroke,fill}%
}%
\begin{pgfscope}%
\pgfsys@transformshift{3.370369in}{3.412562in}%
\pgfsys@useobject{currentmarker}{}%
\end{pgfscope}%
\end{pgfscope}%
\begin{pgfscope}%
\pgfpathrectangle{\pgfqpoint{0.100000in}{2.413063in}}{\pgfqpoint{5.037500in}{3.427208in}}%
\pgfusepath{clip}%
\pgfsetrectcap%
\pgfsetroundjoin%
\pgfsetlinewidth{1.505625pt}%
\definecolor{currentstroke}{rgb}{0.678431,1.000000,0.184314}%
\pgfsetstrokecolor{currentstroke}%
\pgfsetstrokeopacity{0.500000}%
\pgfsetdash{}{0pt}%
\pgfpathmoveto{\pgfqpoint{3.277318in}{3.533073in}}%
\pgfusepath{stroke}%
\end{pgfscope}%
\begin{pgfscope}%
\pgfpathrectangle{\pgfqpoint{0.100000in}{2.413063in}}{\pgfqpoint{5.037500in}{3.427208in}}%
\pgfusepath{clip}%
\pgfsetbuttcap%
\pgfsetroundjoin%
\definecolor{currentfill}{rgb}{0.678431,1.000000,0.184314}%
\pgfsetfillcolor{currentfill}%
\pgfsetfillopacity{0.500000}%
\pgfsetlinewidth{0.250937pt}%
\definecolor{currentstroke}{rgb}{0.000000,0.000000,0.000000}%
\pgfsetstrokecolor{currentstroke}%
\pgfsetstrokeopacity{0.500000}%
\pgfsetdash{}{0pt}%
\pgfsys@defobject{currentmarker}{\pgfqpoint{-0.041667in}{-0.041667in}}{\pgfqpoint{0.041667in}{0.041667in}}{%
\pgfpathmoveto{\pgfqpoint{0.000000in}{-0.041667in}}%
\pgfpathcurveto{\pgfqpoint{0.011050in}{-0.041667in}}{\pgfqpoint{0.021649in}{-0.037276in}}{\pgfqpoint{0.029463in}{-0.029463in}}%
\pgfpathcurveto{\pgfqpoint{0.037276in}{-0.021649in}}{\pgfqpoint{0.041667in}{-0.011050in}}{\pgfqpoint{0.041667in}{0.000000in}}%
\pgfpathcurveto{\pgfqpoint{0.041667in}{0.011050in}}{\pgfqpoint{0.037276in}{0.021649in}}{\pgfqpoint{0.029463in}{0.029463in}}%
\pgfpathcurveto{\pgfqpoint{0.021649in}{0.037276in}}{\pgfqpoint{0.011050in}{0.041667in}}{\pgfqpoint{0.000000in}{0.041667in}}%
\pgfpathcurveto{\pgfqpoint{-0.011050in}{0.041667in}}{\pgfqpoint{-0.021649in}{0.037276in}}{\pgfqpoint{-0.029463in}{0.029463in}}%
\pgfpathcurveto{\pgfqpoint{-0.037276in}{0.021649in}}{\pgfqpoint{-0.041667in}{0.011050in}}{\pgfqpoint{-0.041667in}{0.000000in}}%
\pgfpathcurveto{\pgfqpoint{-0.041667in}{-0.011050in}}{\pgfqpoint{-0.037276in}{-0.021649in}}{\pgfqpoint{-0.029463in}{-0.029463in}}%
\pgfpathcurveto{\pgfqpoint{-0.021649in}{-0.037276in}}{\pgfqpoint{-0.011050in}{-0.041667in}}{\pgfqpoint{0.000000in}{-0.041667in}}%
\pgfpathclose%
\pgfusepath{stroke,fill}%
}%
\begin{pgfscope}%
\pgfsys@transformshift{3.277318in}{3.533073in}%
\pgfsys@useobject{currentmarker}{}%
\end{pgfscope}%
\end{pgfscope}%
\begin{pgfscope}%
\pgfpathrectangle{\pgfqpoint{0.100000in}{2.413063in}}{\pgfqpoint{5.037500in}{3.427208in}}%
\pgfusepath{clip}%
\pgfsetrectcap%
\pgfsetroundjoin%
\pgfsetlinewidth{1.505625pt}%
\definecolor{currentstroke}{rgb}{0.678431,1.000000,0.184314}%
\pgfsetstrokecolor{currentstroke}%
\pgfsetstrokeopacity{0.500000}%
\pgfsetdash{}{0pt}%
\pgfpathmoveto{\pgfqpoint{3.306143in}{4.100990in}}%
\pgfusepath{stroke}%
\end{pgfscope}%
\begin{pgfscope}%
\pgfpathrectangle{\pgfqpoint{0.100000in}{2.413063in}}{\pgfqpoint{5.037500in}{3.427208in}}%
\pgfusepath{clip}%
\pgfsetbuttcap%
\pgfsetroundjoin%
\definecolor{currentfill}{rgb}{0.678431,1.000000,0.184314}%
\pgfsetfillcolor{currentfill}%
\pgfsetfillopacity{0.500000}%
\pgfsetlinewidth{0.250937pt}%
\definecolor{currentstroke}{rgb}{0.000000,0.000000,0.000000}%
\pgfsetstrokecolor{currentstroke}%
\pgfsetstrokeopacity{0.500000}%
\pgfsetdash{}{0pt}%
\pgfsys@defobject{currentmarker}{\pgfqpoint{-0.019444in}{-0.019444in}}{\pgfqpoint{0.019444in}{0.019444in}}{%
\pgfpathmoveto{\pgfqpoint{0.000000in}{-0.019444in}}%
\pgfpathcurveto{\pgfqpoint{0.005157in}{-0.019444in}}{\pgfqpoint{0.010103in}{-0.017396in}}{\pgfqpoint{0.013749in}{-0.013749in}}%
\pgfpathcurveto{\pgfqpoint{0.017396in}{-0.010103in}}{\pgfqpoint{0.019444in}{-0.005157in}}{\pgfqpoint{0.019444in}{0.000000in}}%
\pgfpathcurveto{\pgfqpoint{0.019444in}{0.005157in}}{\pgfqpoint{0.017396in}{0.010103in}}{\pgfqpoint{0.013749in}{0.013749in}}%
\pgfpathcurveto{\pgfqpoint{0.010103in}{0.017396in}}{\pgfqpoint{0.005157in}{0.019444in}}{\pgfqpoint{0.000000in}{0.019444in}}%
\pgfpathcurveto{\pgfqpoint{-0.005157in}{0.019444in}}{\pgfqpoint{-0.010103in}{0.017396in}}{\pgfqpoint{-0.013749in}{0.013749in}}%
\pgfpathcurveto{\pgfqpoint{-0.017396in}{0.010103in}}{\pgfqpoint{-0.019444in}{0.005157in}}{\pgfqpoint{-0.019444in}{0.000000in}}%
\pgfpathcurveto{\pgfqpoint{-0.019444in}{-0.005157in}}{\pgfqpoint{-0.017396in}{-0.010103in}}{\pgfqpoint{-0.013749in}{-0.013749in}}%
\pgfpathcurveto{\pgfqpoint{-0.010103in}{-0.017396in}}{\pgfqpoint{-0.005157in}{-0.019444in}}{\pgfqpoint{0.000000in}{-0.019444in}}%
\pgfpathclose%
\pgfusepath{stroke,fill}%
}%
\begin{pgfscope}%
\pgfsys@transformshift{3.306143in}{4.100990in}%
\pgfsys@useobject{currentmarker}{}%
\end{pgfscope}%
\end{pgfscope}%
\begin{pgfscope}%
\pgfpathrectangle{\pgfqpoint{0.100000in}{2.413063in}}{\pgfqpoint{5.037500in}{3.427208in}}%
\pgfusepath{clip}%
\pgfsetrectcap%
\pgfsetroundjoin%
\pgfsetlinewidth{1.505625pt}%
\definecolor{currentstroke}{rgb}{0.678431,1.000000,0.184314}%
\pgfsetstrokecolor{currentstroke}%
\pgfsetstrokeopacity{0.500000}%
\pgfsetdash{}{0pt}%
\pgfpathmoveto{\pgfqpoint{3.042850in}{4.278911in}}%
\pgfusepath{stroke}%
\end{pgfscope}%
\begin{pgfscope}%
\pgfpathrectangle{\pgfqpoint{0.100000in}{2.413063in}}{\pgfqpoint{5.037500in}{3.427208in}}%
\pgfusepath{clip}%
\pgfsetbuttcap%
\pgfsetroundjoin%
\definecolor{currentfill}{rgb}{0.678431,1.000000,0.184314}%
\pgfsetfillcolor{currentfill}%
\pgfsetfillopacity{0.500000}%
\pgfsetlinewidth{0.250937pt}%
\definecolor{currentstroke}{rgb}{0.000000,0.000000,0.000000}%
\pgfsetstrokecolor{currentstroke}%
\pgfsetstrokeopacity{0.500000}%
\pgfsetdash{}{0pt}%
\pgfsys@defobject{currentmarker}{\pgfqpoint{-0.013889in}{-0.013889in}}{\pgfqpoint{0.013889in}{0.013889in}}{%
\pgfpathmoveto{\pgfqpoint{0.000000in}{-0.013889in}}%
\pgfpathcurveto{\pgfqpoint{0.003683in}{-0.013889in}}{\pgfqpoint{0.007216in}{-0.012425in}}{\pgfqpoint{0.009821in}{-0.009821in}}%
\pgfpathcurveto{\pgfqpoint{0.012425in}{-0.007216in}}{\pgfqpoint{0.013889in}{-0.003683in}}{\pgfqpoint{0.013889in}{0.000000in}}%
\pgfpathcurveto{\pgfqpoint{0.013889in}{0.003683in}}{\pgfqpoint{0.012425in}{0.007216in}}{\pgfqpoint{0.009821in}{0.009821in}}%
\pgfpathcurveto{\pgfqpoint{0.007216in}{0.012425in}}{\pgfqpoint{0.003683in}{0.013889in}}{\pgfqpoint{0.000000in}{0.013889in}}%
\pgfpathcurveto{\pgfqpoint{-0.003683in}{0.013889in}}{\pgfqpoint{-0.007216in}{0.012425in}}{\pgfqpoint{-0.009821in}{0.009821in}}%
\pgfpathcurveto{\pgfqpoint{-0.012425in}{0.007216in}}{\pgfqpoint{-0.013889in}{0.003683in}}{\pgfqpoint{-0.013889in}{0.000000in}}%
\pgfpathcurveto{\pgfqpoint{-0.013889in}{-0.003683in}}{\pgfqpoint{-0.012425in}{-0.007216in}}{\pgfqpoint{-0.009821in}{-0.009821in}}%
\pgfpathcurveto{\pgfqpoint{-0.007216in}{-0.012425in}}{\pgfqpoint{-0.003683in}{-0.013889in}}{\pgfqpoint{0.000000in}{-0.013889in}}%
\pgfpathclose%
\pgfusepath{stroke,fill}%
}%
\begin{pgfscope}%
\pgfsys@transformshift{3.042850in}{4.278911in}%
\pgfsys@useobject{currentmarker}{}%
\end{pgfscope}%
\end{pgfscope}%
\begin{pgfscope}%
\pgfpathrectangle{\pgfqpoint{0.100000in}{2.413063in}}{\pgfqpoint{5.037500in}{3.427208in}}%
\pgfusepath{clip}%
\pgfsetrectcap%
\pgfsetroundjoin%
\pgfsetlinewidth{1.505625pt}%
\definecolor{currentstroke}{rgb}{0.678431,1.000000,0.184314}%
\pgfsetstrokecolor{currentstroke}%
\pgfsetstrokeopacity{0.500000}%
\pgfsetdash{}{0pt}%
\pgfpathmoveto{\pgfqpoint{3.058630in}{4.236242in}}%
\pgfusepath{stroke}%
\end{pgfscope}%
\begin{pgfscope}%
\pgfpathrectangle{\pgfqpoint{0.100000in}{2.413063in}}{\pgfqpoint{5.037500in}{3.427208in}}%
\pgfusepath{clip}%
\pgfsetbuttcap%
\pgfsetroundjoin%
\definecolor{currentfill}{rgb}{0.678431,1.000000,0.184314}%
\pgfsetfillcolor{currentfill}%
\pgfsetfillopacity{0.500000}%
\pgfsetlinewidth{0.250937pt}%
\definecolor{currentstroke}{rgb}{0.000000,0.000000,0.000000}%
\pgfsetstrokecolor{currentstroke}%
\pgfsetstrokeopacity{0.500000}%
\pgfsetdash{}{0pt}%
\pgfsys@defobject{currentmarker}{\pgfqpoint{-0.019444in}{-0.019444in}}{\pgfqpoint{0.019444in}{0.019444in}}{%
\pgfpathmoveto{\pgfqpoint{0.000000in}{-0.019444in}}%
\pgfpathcurveto{\pgfqpoint{0.005157in}{-0.019444in}}{\pgfqpoint{0.010103in}{-0.017396in}}{\pgfqpoint{0.013749in}{-0.013749in}}%
\pgfpathcurveto{\pgfqpoint{0.017396in}{-0.010103in}}{\pgfqpoint{0.019444in}{-0.005157in}}{\pgfqpoint{0.019444in}{0.000000in}}%
\pgfpathcurveto{\pgfqpoint{0.019444in}{0.005157in}}{\pgfqpoint{0.017396in}{0.010103in}}{\pgfqpoint{0.013749in}{0.013749in}}%
\pgfpathcurveto{\pgfqpoint{0.010103in}{0.017396in}}{\pgfqpoint{0.005157in}{0.019444in}}{\pgfqpoint{0.000000in}{0.019444in}}%
\pgfpathcurveto{\pgfqpoint{-0.005157in}{0.019444in}}{\pgfqpoint{-0.010103in}{0.017396in}}{\pgfqpoint{-0.013749in}{0.013749in}}%
\pgfpathcurveto{\pgfqpoint{-0.017396in}{0.010103in}}{\pgfqpoint{-0.019444in}{0.005157in}}{\pgfqpoint{-0.019444in}{0.000000in}}%
\pgfpathcurveto{\pgfqpoint{-0.019444in}{-0.005157in}}{\pgfqpoint{-0.017396in}{-0.010103in}}{\pgfqpoint{-0.013749in}{-0.013749in}}%
\pgfpathcurveto{\pgfqpoint{-0.010103in}{-0.017396in}}{\pgfqpoint{-0.005157in}{-0.019444in}}{\pgfqpoint{0.000000in}{-0.019444in}}%
\pgfpathclose%
\pgfusepath{stroke,fill}%
}%
\begin{pgfscope}%
\pgfsys@transformshift{3.058630in}{4.236242in}%
\pgfsys@useobject{currentmarker}{}%
\end{pgfscope}%
\end{pgfscope}%
\begin{pgfscope}%
\pgfpathrectangle{\pgfqpoint{0.100000in}{2.413063in}}{\pgfqpoint{5.037500in}{3.427208in}}%
\pgfusepath{clip}%
\pgfsetrectcap%
\pgfsetroundjoin%
\pgfsetlinewidth{1.505625pt}%
\definecolor{currentstroke}{rgb}{0.678431,1.000000,0.184314}%
\pgfsetstrokecolor{currentstroke}%
\pgfsetstrokeopacity{0.500000}%
\pgfsetdash{}{0pt}%
\pgfpathmoveto{\pgfqpoint{2.848864in}{4.060427in}}%
\pgfusepath{stroke}%
\end{pgfscope}%
\begin{pgfscope}%
\pgfpathrectangle{\pgfqpoint{0.100000in}{2.413063in}}{\pgfqpoint{5.037500in}{3.427208in}}%
\pgfusepath{clip}%
\pgfsetbuttcap%
\pgfsetroundjoin%
\definecolor{currentfill}{rgb}{0.678431,1.000000,0.184314}%
\pgfsetfillcolor{currentfill}%
\pgfsetfillopacity{0.500000}%
\pgfsetlinewidth{0.250937pt}%
\definecolor{currentstroke}{rgb}{0.000000,0.000000,0.000000}%
\pgfsetstrokecolor{currentstroke}%
\pgfsetstrokeopacity{0.500000}%
\pgfsetdash{}{0pt}%
\pgfsys@defobject{currentmarker}{\pgfqpoint{-0.019444in}{-0.019444in}}{\pgfqpoint{0.019444in}{0.019444in}}{%
\pgfpathmoveto{\pgfqpoint{0.000000in}{-0.019444in}}%
\pgfpathcurveto{\pgfqpoint{0.005157in}{-0.019444in}}{\pgfqpoint{0.010103in}{-0.017396in}}{\pgfqpoint{0.013749in}{-0.013749in}}%
\pgfpathcurveto{\pgfqpoint{0.017396in}{-0.010103in}}{\pgfqpoint{0.019444in}{-0.005157in}}{\pgfqpoint{0.019444in}{0.000000in}}%
\pgfpathcurveto{\pgfqpoint{0.019444in}{0.005157in}}{\pgfqpoint{0.017396in}{0.010103in}}{\pgfqpoint{0.013749in}{0.013749in}}%
\pgfpathcurveto{\pgfqpoint{0.010103in}{0.017396in}}{\pgfqpoint{0.005157in}{0.019444in}}{\pgfqpoint{0.000000in}{0.019444in}}%
\pgfpathcurveto{\pgfqpoint{-0.005157in}{0.019444in}}{\pgfqpoint{-0.010103in}{0.017396in}}{\pgfqpoint{-0.013749in}{0.013749in}}%
\pgfpathcurveto{\pgfqpoint{-0.017396in}{0.010103in}}{\pgfqpoint{-0.019444in}{0.005157in}}{\pgfqpoint{-0.019444in}{0.000000in}}%
\pgfpathcurveto{\pgfqpoint{-0.019444in}{-0.005157in}}{\pgfqpoint{-0.017396in}{-0.010103in}}{\pgfqpoint{-0.013749in}{-0.013749in}}%
\pgfpathcurveto{\pgfqpoint{-0.010103in}{-0.017396in}}{\pgfqpoint{-0.005157in}{-0.019444in}}{\pgfqpoint{0.000000in}{-0.019444in}}%
\pgfpathclose%
\pgfusepath{stroke,fill}%
}%
\begin{pgfscope}%
\pgfsys@transformshift{2.848864in}{4.060427in}%
\pgfsys@useobject{currentmarker}{}%
\end{pgfscope}%
\end{pgfscope}%
\begin{pgfscope}%
\pgfpathrectangle{\pgfqpoint{0.100000in}{2.413063in}}{\pgfqpoint{5.037500in}{3.427208in}}%
\pgfusepath{clip}%
\pgfsetrectcap%
\pgfsetroundjoin%
\pgfsetlinewidth{1.505625pt}%
\definecolor{currentstroke}{rgb}{0.678431,1.000000,0.184314}%
\pgfsetstrokecolor{currentstroke}%
\pgfsetstrokeopacity{0.500000}%
\pgfsetdash{}{0pt}%
\pgfpathmoveto{\pgfqpoint{2.843184in}{4.290769in}}%
\pgfusepath{stroke}%
\end{pgfscope}%
\begin{pgfscope}%
\pgfpathrectangle{\pgfqpoint{0.100000in}{2.413063in}}{\pgfqpoint{5.037500in}{3.427208in}}%
\pgfusepath{clip}%
\pgfsetbuttcap%
\pgfsetroundjoin%
\definecolor{currentfill}{rgb}{0.678431,1.000000,0.184314}%
\pgfsetfillcolor{currentfill}%
\pgfsetfillopacity{0.500000}%
\pgfsetlinewidth{0.250937pt}%
\definecolor{currentstroke}{rgb}{0.000000,0.000000,0.000000}%
\pgfsetstrokecolor{currentstroke}%
\pgfsetstrokeopacity{0.500000}%
\pgfsetdash{}{0pt}%
\pgfsys@defobject{currentmarker}{\pgfqpoint{-0.013889in}{-0.013889in}}{\pgfqpoint{0.013889in}{0.013889in}}{%
\pgfpathmoveto{\pgfqpoint{0.000000in}{-0.013889in}}%
\pgfpathcurveto{\pgfqpoint{0.003683in}{-0.013889in}}{\pgfqpoint{0.007216in}{-0.012425in}}{\pgfqpoint{0.009821in}{-0.009821in}}%
\pgfpathcurveto{\pgfqpoint{0.012425in}{-0.007216in}}{\pgfqpoint{0.013889in}{-0.003683in}}{\pgfqpoint{0.013889in}{0.000000in}}%
\pgfpathcurveto{\pgfqpoint{0.013889in}{0.003683in}}{\pgfqpoint{0.012425in}{0.007216in}}{\pgfqpoint{0.009821in}{0.009821in}}%
\pgfpathcurveto{\pgfqpoint{0.007216in}{0.012425in}}{\pgfqpoint{0.003683in}{0.013889in}}{\pgfqpoint{0.000000in}{0.013889in}}%
\pgfpathcurveto{\pgfqpoint{-0.003683in}{0.013889in}}{\pgfqpoint{-0.007216in}{0.012425in}}{\pgfqpoint{-0.009821in}{0.009821in}}%
\pgfpathcurveto{\pgfqpoint{-0.012425in}{0.007216in}}{\pgfqpoint{-0.013889in}{0.003683in}}{\pgfqpoint{-0.013889in}{0.000000in}}%
\pgfpathcurveto{\pgfqpoint{-0.013889in}{-0.003683in}}{\pgfqpoint{-0.012425in}{-0.007216in}}{\pgfqpoint{-0.009821in}{-0.009821in}}%
\pgfpathcurveto{\pgfqpoint{-0.007216in}{-0.012425in}}{\pgfqpoint{-0.003683in}{-0.013889in}}{\pgfqpoint{0.000000in}{-0.013889in}}%
\pgfpathclose%
\pgfusepath{stroke,fill}%
}%
\begin{pgfscope}%
\pgfsys@transformshift{2.843184in}{4.290769in}%
\pgfsys@useobject{currentmarker}{}%
\end{pgfscope}%
\end{pgfscope}%
\begin{pgfscope}%
\pgfpathrectangle{\pgfqpoint{0.100000in}{2.413063in}}{\pgfqpoint{5.037500in}{3.427208in}}%
\pgfusepath{clip}%
\pgfsetrectcap%
\pgfsetroundjoin%
\pgfsetlinewidth{1.505625pt}%
\definecolor{currentstroke}{rgb}{0.678431,1.000000,0.184314}%
\pgfsetstrokecolor{currentstroke}%
\pgfsetstrokeopacity{0.500000}%
\pgfsetdash{}{0pt}%
\pgfpathmoveto{\pgfqpoint{2.817703in}{4.369465in}}%
\pgfusepath{stroke}%
\end{pgfscope}%
\begin{pgfscope}%
\pgfpathrectangle{\pgfqpoint{0.100000in}{2.413063in}}{\pgfqpoint{5.037500in}{3.427208in}}%
\pgfusepath{clip}%
\pgfsetbuttcap%
\pgfsetroundjoin%
\definecolor{currentfill}{rgb}{0.678431,1.000000,0.184314}%
\pgfsetfillcolor{currentfill}%
\pgfsetfillopacity{0.500000}%
\pgfsetlinewidth{0.250937pt}%
\definecolor{currentstroke}{rgb}{0.000000,0.000000,0.000000}%
\pgfsetstrokecolor{currentstroke}%
\pgfsetstrokeopacity{0.500000}%
\pgfsetdash{}{0pt}%
\pgfsys@defobject{currentmarker}{\pgfqpoint{-0.019444in}{-0.019444in}}{\pgfqpoint{0.019444in}{0.019444in}}{%
\pgfpathmoveto{\pgfqpoint{0.000000in}{-0.019444in}}%
\pgfpathcurveto{\pgfqpoint{0.005157in}{-0.019444in}}{\pgfqpoint{0.010103in}{-0.017396in}}{\pgfqpoint{0.013749in}{-0.013749in}}%
\pgfpathcurveto{\pgfqpoint{0.017396in}{-0.010103in}}{\pgfqpoint{0.019444in}{-0.005157in}}{\pgfqpoint{0.019444in}{0.000000in}}%
\pgfpathcurveto{\pgfqpoint{0.019444in}{0.005157in}}{\pgfqpoint{0.017396in}{0.010103in}}{\pgfqpoint{0.013749in}{0.013749in}}%
\pgfpathcurveto{\pgfqpoint{0.010103in}{0.017396in}}{\pgfqpoint{0.005157in}{0.019444in}}{\pgfqpoint{0.000000in}{0.019444in}}%
\pgfpathcurveto{\pgfqpoint{-0.005157in}{0.019444in}}{\pgfqpoint{-0.010103in}{0.017396in}}{\pgfqpoint{-0.013749in}{0.013749in}}%
\pgfpathcurveto{\pgfqpoint{-0.017396in}{0.010103in}}{\pgfqpoint{-0.019444in}{0.005157in}}{\pgfqpoint{-0.019444in}{0.000000in}}%
\pgfpathcurveto{\pgfqpoint{-0.019444in}{-0.005157in}}{\pgfqpoint{-0.017396in}{-0.010103in}}{\pgfqpoint{-0.013749in}{-0.013749in}}%
\pgfpathcurveto{\pgfqpoint{-0.010103in}{-0.017396in}}{\pgfqpoint{-0.005157in}{-0.019444in}}{\pgfqpoint{0.000000in}{-0.019444in}}%
\pgfpathclose%
\pgfusepath{stroke,fill}%
}%
\begin{pgfscope}%
\pgfsys@transformshift{2.817703in}{4.369465in}%
\pgfsys@useobject{currentmarker}{}%
\end{pgfscope}%
\end{pgfscope}%
\begin{pgfscope}%
\pgfpathrectangle{\pgfqpoint{0.100000in}{2.413063in}}{\pgfqpoint{5.037500in}{3.427208in}}%
\pgfusepath{clip}%
\pgfsetrectcap%
\pgfsetroundjoin%
\pgfsetlinewidth{1.505625pt}%
\definecolor{currentstroke}{rgb}{0.501961,0.501961,0.501961}%
\pgfsetstrokecolor{currentstroke}%
\pgfsetstrokeopacity{0.500000}%
\pgfsetdash{}{0pt}%
\pgfpathmoveto{\pgfqpoint{3.235918in}{4.249441in}}%
\pgfusepath{stroke}%
\end{pgfscope}%
\begin{pgfscope}%
\pgfpathrectangle{\pgfqpoint{0.100000in}{2.413063in}}{\pgfqpoint{5.037500in}{3.427208in}}%
\pgfusepath{clip}%
\pgfsetbuttcap%
\pgfsetroundjoin%
\definecolor{currentfill}{rgb}{0.501961,0.501961,0.501961}%
\pgfsetfillcolor{currentfill}%
\pgfsetfillopacity{0.500000}%
\pgfsetlinewidth{0.250937pt}%
\definecolor{currentstroke}{rgb}{0.000000,0.000000,0.000000}%
\pgfsetstrokecolor{currentstroke}%
\pgfsetstrokeopacity{0.500000}%
\pgfsetdash{}{0pt}%
\pgfsys@defobject{currentmarker}{\pgfqpoint{-0.013889in}{-0.013889in}}{\pgfqpoint{0.013889in}{0.013889in}}{%
\pgfpathmoveto{\pgfqpoint{0.000000in}{-0.013889in}}%
\pgfpathcurveto{\pgfqpoint{0.003683in}{-0.013889in}}{\pgfqpoint{0.007216in}{-0.012425in}}{\pgfqpoint{0.009821in}{-0.009821in}}%
\pgfpathcurveto{\pgfqpoint{0.012425in}{-0.007216in}}{\pgfqpoint{0.013889in}{-0.003683in}}{\pgfqpoint{0.013889in}{0.000000in}}%
\pgfpathcurveto{\pgfqpoint{0.013889in}{0.003683in}}{\pgfqpoint{0.012425in}{0.007216in}}{\pgfqpoint{0.009821in}{0.009821in}}%
\pgfpathcurveto{\pgfqpoint{0.007216in}{0.012425in}}{\pgfqpoint{0.003683in}{0.013889in}}{\pgfqpoint{0.000000in}{0.013889in}}%
\pgfpathcurveto{\pgfqpoint{-0.003683in}{0.013889in}}{\pgfqpoint{-0.007216in}{0.012425in}}{\pgfqpoint{-0.009821in}{0.009821in}}%
\pgfpathcurveto{\pgfqpoint{-0.012425in}{0.007216in}}{\pgfqpoint{-0.013889in}{0.003683in}}{\pgfqpoint{-0.013889in}{0.000000in}}%
\pgfpathcurveto{\pgfqpoint{-0.013889in}{-0.003683in}}{\pgfqpoint{-0.012425in}{-0.007216in}}{\pgfqpoint{-0.009821in}{-0.009821in}}%
\pgfpathcurveto{\pgfqpoint{-0.007216in}{-0.012425in}}{\pgfqpoint{-0.003683in}{-0.013889in}}{\pgfqpoint{0.000000in}{-0.013889in}}%
\pgfpathclose%
\pgfusepath{stroke,fill}%
}%
\begin{pgfscope}%
\pgfsys@transformshift{3.235918in}{4.249441in}%
\pgfsys@useobject{currentmarker}{}%
\end{pgfscope}%
\end{pgfscope}%
\begin{pgfscope}%
\pgfpathrectangle{\pgfqpoint{0.100000in}{2.413063in}}{\pgfqpoint{5.037500in}{3.427208in}}%
\pgfusepath{clip}%
\pgfsetrectcap%
\pgfsetroundjoin%
\pgfsetlinewidth{1.505625pt}%
\definecolor{currentstroke}{rgb}{0.678431,1.000000,0.184314}%
\pgfsetstrokecolor{currentstroke}%
\pgfsetstrokeopacity{0.500000}%
\pgfsetdash{}{0pt}%
\pgfpathmoveto{\pgfqpoint{2.960251in}{4.076874in}}%
\pgfusepath{stroke}%
\end{pgfscope}%
\begin{pgfscope}%
\pgfpathrectangle{\pgfqpoint{0.100000in}{2.413063in}}{\pgfqpoint{5.037500in}{3.427208in}}%
\pgfusepath{clip}%
\pgfsetbuttcap%
\pgfsetroundjoin%
\definecolor{currentfill}{rgb}{0.678431,1.000000,0.184314}%
\pgfsetfillcolor{currentfill}%
\pgfsetfillopacity{0.500000}%
\pgfsetlinewidth{0.250937pt}%
\definecolor{currentstroke}{rgb}{0.000000,0.000000,0.000000}%
\pgfsetstrokecolor{currentstroke}%
\pgfsetstrokeopacity{0.500000}%
\pgfsetdash{}{0pt}%
\pgfsys@defobject{currentmarker}{\pgfqpoint{-0.027778in}{-0.027778in}}{\pgfqpoint{0.027778in}{0.027778in}}{%
\pgfpathmoveto{\pgfqpoint{0.000000in}{-0.027778in}}%
\pgfpathcurveto{\pgfqpoint{0.007367in}{-0.027778in}}{\pgfqpoint{0.014433in}{-0.024851in}}{\pgfqpoint{0.019642in}{-0.019642in}}%
\pgfpathcurveto{\pgfqpoint{0.024851in}{-0.014433in}}{\pgfqpoint{0.027778in}{-0.007367in}}{\pgfqpoint{0.027778in}{0.000000in}}%
\pgfpathcurveto{\pgfqpoint{0.027778in}{0.007367in}}{\pgfqpoint{0.024851in}{0.014433in}}{\pgfqpoint{0.019642in}{0.019642in}}%
\pgfpathcurveto{\pgfqpoint{0.014433in}{0.024851in}}{\pgfqpoint{0.007367in}{0.027778in}}{\pgfqpoint{0.000000in}{0.027778in}}%
\pgfpathcurveto{\pgfqpoint{-0.007367in}{0.027778in}}{\pgfqpoint{-0.014433in}{0.024851in}}{\pgfqpoint{-0.019642in}{0.019642in}}%
\pgfpathcurveto{\pgfqpoint{-0.024851in}{0.014433in}}{\pgfqpoint{-0.027778in}{0.007367in}}{\pgfqpoint{-0.027778in}{0.000000in}}%
\pgfpathcurveto{\pgfqpoint{-0.027778in}{-0.007367in}}{\pgfqpoint{-0.024851in}{-0.014433in}}{\pgfqpoint{-0.019642in}{-0.019642in}}%
\pgfpathcurveto{\pgfqpoint{-0.014433in}{-0.024851in}}{\pgfqpoint{-0.007367in}{-0.027778in}}{\pgfqpoint{0.000000in}{-0.027778in}}%
\pgfpathclose%
\pgfusepath{stroke,fill}%
}%
\begin{pgfscope}%
\pgfsys@transformshift{2.960251in}{4.076874in}%
\pgfsys@useobject{currentmarker}{}%
\end{pgfscope}%
\end{pgfscope}%
\begin{pgfscope}%
\pgfpathrectangle{\pgfqpoint{0.100000in}{2.413063in}}{\pgfqpoint{5.037500in}{3.427208in}}%
\pgfusepath{clip}%
\pgfsetrectcap%
\pgfsetroundjoin%
\pgfsetlinewidth{1.505625pt}%
\definecolor{currentstroke}{rgb}{0.678431,1.000000,0.184314}%
\pgfsetstrokecolor{currentstroke}%
\pgfsetstrokeopacity{0.500000}%
\pgfsetdash{}{0pt}%
\pgfpathmoveto{\pgfqpoint{1.716761in}{5.145212in}}%
\pgfusepath{stroke}%
\end{pgfscope}%
\begin{pgfscope}%
\pgfpathrectangle{\pgfqpoint{0.100000in}{2.413063in}}{\pgfqpoint{5.037500in}{3.427208in}}%
\pgfusepath{clip}%
\pgfsetbuttcap%
\pgfsetroundjoin%
\definecolor{currentfill}{rgb}{0.678431,1.000000,0.184314}%
\pgfsetfillcolor{currentfill}%
\pgfsetfillopacity{0.500000}%
\pgfsetlinewidth{0.250937pt}%
\definecolor{currentstroke}{rgb}{0.000000,0.000000,0.000000}%
\pgfsetstrokecolor{currentstroke}%
\pgfsetstrokeopacity{0.500000}%
\pgfsetdash{}{0pt}%
\pgfsys@defobject{currentmarker}{\pgfqpoint{-0.041667in}{-0.041667in}}{\pgfqpoint{0.041667in}{0.041667in}}{%
\pgfpathmoveto{\pgfqpoint{0.000000in}{-0.041667in}}%
\pgfpathcurveto{\pgfqpoint{0.011050in}{-0.041667in}}{\pgfqpoint{0.021649in}{-0.037276in}}{\pgfqpoint{0.029463in}{-0.029463in}}%
\pgfpathcurveto{\pgfqpoint{0.037276in}{-0.021649in}}{\pgfqpoint{0.041667in}{-0.011050in}}{\pgfqpoint{0.041667in}{0.000000in}}%
\pgfpathcurveto{\pgfqpoint{0.041667in}{0.011050in}}{\pgfqpoint{0.037276in}{0.021649in}}{\pgfqpoint{0.029463in}{0.029463in}}%
\pgfpathcurveto{\pgfqpoint{0.021649in}{0.037276in}}{\pgfqpoint{0.011050in}{0.041667in}}{\pgfqpoint{0.000000in}{0.041667in}}%
\pgfpathcurveto{\pgfqpoint{-0.011050in}{0.041667in}}{\pgfqpoint{-0.021649in}{0.037276in}}{\pgfqpoint{-0.029463in}{0.029463in}}%
\pgfpathcurveto{\pgfqpoint{-0.037276in}{0.021649in}}{\pgfqpoint{-0.041667in}{0.011050in}}{\pgfqpoint{-0.041667in}{0.000000in}}%
\pgfpathcurveto{\pgfqpoint{-0.041667in}{-0.011050in}}{\pgfqpoint{-0.037276in}{-0.021649in}}{\pgfqpoint{-0.029463in}{-0.029463in}}%
\pgfpathcurveto{\pgfqpoint{-0.021649in}{-0.037276in}}{\pgfqpoint{-0.011050in}{-0.041667in}}{\pgfqpoint{0.000000in}{-0.041667in}}%
\pgfpathclose%
\pgfusepath{stroke,fill}%
}%
\begin{pgfscope}%
\pgfsys@transformshift{1.716761in}{5.145212in}%
\pgfsys@useobject{currentmarker}{}%
\end{pgfscope}%
\end{pgfscope}%
\begin{pgfscope}%
\pgfpathrectangle{\pgfqpoint{0.100000in}{2.413063in}}{\pgfqpoint{5.037500in}{3.427208in}}%
\pgfusepath{clip}%
\pgfsetrectcap%
\pgfsetroundjoin%
\pgfsetlinewidth{1.505625pt}%
\definecolor{currentstroke}{rgb}{0.678431,1.000000,0.184314}%
\pgfsetstrokecolor{currentstroke}%
\pgfsetstrokeopacity{0.500000}%
\pgfsetdash{}{0pt}%
\pgfpathmoveto{\pgfqpoint{1.529298in}{5.378464in}}%
\pgfusepath{stroke}%
\end{pgfscope}%
\begin{pgfscope}%
\pgfpathrectangle{\pgfqpoint{0.100000in}{2.413063in}}{\pgfqpoint{5.037500in}{3.427208in}}%
\pgfusepath{clip}%
\pgfsetbuttcap%
\pgfsetroundjoin%
\definecolor{currentfill}{rgb}{0.678431,1.000000,0.184314}%
\pgfsetfillcolor{currentfill}%
\pgfsetfillopacity{0.500000}%
\pgfsetlinewidth{0.250937pt}%
\definecolor{currentstroke}{rgb}{0.000000,0.000000,0.000000}%
\pgfsetstrokecolor{currentstroke}%
\pgfsetstrokeopacity{0.500000}%
\pgfsetdash{}{0pt}%
\pgfsys@defobject{currentmarker}{\pgfqpoint{-0.044444in}{-0.044444in}}{\pgfqpoint{0.044444in}{0.044444in}}{%
\pgfpathmoveto{\pgfqpoint{0.000000in}{-0.044444in}}%
\pgfpathcurveto{\pgfqpoint{0.011787in}{-0.044444in}}{\pgfqpoint{0.023092in}{-0.039761in}}{\pgfqpoint{0.031427in}{-0.031427in}}%
\pgfpathcurveto{\pgfqpoint{0.039761in}{-0.023092in}}{\pgfqpoint{0.044444in}{-0.011787in}}{\pgfqpoint{0.044444in}{0.000000in}}%
\pgfpathcurveto{\pgfqpoint{0.044444in}{0.011787in}}{\pgfqpoint{0.039761in}{0.023092in}}{\pgfqpoint{0.031427in}{0.031427in}}%
\pgfpathcurveto{\pgfqpoint{0.023092in}{0.039761in}}{\pgfqpoint{0.011787in}{0.044444in}}{\pgfqpoint{0.000000in}{0.044444in}}%
\pgfpathcurveto{\pgfqpoint{-0.011787in}{0.044444in}}{\pgfqpoint{-0.023092in}{0.039761in}}{\pgfqpoint{-0.031427in}{0.031427in}}%
\pgfpathcurveto{\pgfqpoint{-0.039761in}{0.023092in}}{\pgfqpoint{-0.044444in}{0.011787in}}{\pgfqpoint{-0.044444in}{0.000000in}}%
\pgfpathcurveto{\pgfqpoint{-0.044444in}{-0.011787in}}{\pgfqpoint{-0.039761in}{-0.023092in}}{\pgfqpoint{-0.031427in}{-0.031427in}}%
\pgfpathcurveto{\pgfqpoint{-0.023092in}{-0.039761in}}{\pgfqpoint{-0.011787in}{-0.044444in}}{\pgfqpoint{0.000000in}{-0.044444in}}%
\pgfpathclose%
\pgfusepath{stroke,fill}%
}%
\begin{pgfscope}%
\pgfsys@transformshift{1.529298in}{5.378464in}%
\pgfsys@useobject{currentmarker}{}%
\end{pgfscope}%
\end{pgfscope}%
\begin{pgfscope}%
\pgfpathrectangle{\pgfqpoint{0.100000in}{2.413063in}}{\pgfqpoint{5.037500in}{3.427208in}}%
\pgfusepath{clip}%
\pgfsetrectcap%
\pgfsetroundjoin%
\pgfsetlinewidth{1.505625pt}%
\definecolor{currentstroke}{rgb}{0.678431,1.000000,0.184314}%
\pgfsetstrokecolor{currentstroke}%
\pgfsetstrokeopacity{0.500000}%
\pgfsetdash{}{0pt}%
\pgfpathmoveto{\pgfqpoint{1.305298in}{5.347237in}}%
\pgfusepath{stroke}%
\end{pgfscope}%
\begin{pgfscope}%
\pgfpathrectangle{\pgfqpoint{0.100000in}{2.413063in}}{\pgfqpoint{5.037500in}{3.427208in}}%
\pgfusepath{clip}%
\pgfsetbuttcap%
\pgfsetroundjoin%
\definecolor{currentfill}{rgb}{0.678431,1.000000,0.184314}%
\pgfsetfillcolor{currentfill}%
\pgfsetfillopacity{0.500000}%
\pgfsetlinewidth{0.250937pt}%
\definecolor{currentstroke}{rgb}{0.000000,0.000000,0.000000}%
\pgfsetstrokecolor{currentstroke}%
\pgfsetstrokeopacity{0.500000}%
\pgfsetdash{}{0pt}%
\pgfsys@defobject{currentmarker}{\pgfqpoint{-0.044444in}{-0.044444in}}{\pgfqpoint{0.044444in}{0.044444in}}{%
\pgfpathmoveto{\pgfqpoint{0.000000in}{-0.044444in}}%
\pgfpathcurveto{\pgfqpoint{0.011787in}{-0.044444in}}{\pgfqpoint{0.023092in}{-0.039761in}}{\pgfqpoint{0.031427in}{-0.031427in}}%
\pgfpathcurveto{\pgfqpoint{0.039761in}{-0.023092in}}{\pgfqpoint{0.044444in}{-0.011787in}}{\pgfqpoint{0.044444in}{0.000000in}}%
\pgfpathcurveto{\pgfqpoint{0.044444in}{0.011787in}}{\pgfqpoint{0.039761in}{0.023092in}}{\pgfqpoint{0.031427in}{0.031427in}}%
\pgfpathcurveto{\pgfqpoint{0.023092in}{0.039761in}}{\pgfqpoint{0.011787in}{0.044444in}}{\pgfqpoint{0.000000in}{0.044444in}}%
\pgfpathcurveto{\pgfqpoint{-0.011787in}{0.044444in}}{\pgfqpoint{-0.023092in}{0.039761in}}{\pgfqpoint{-0.031427in}{0.031427in}}%
\pgfpathcurveto{\pgfqpoint{-0.039761in}{0.023092in}}{\pgfqpoint{-0.044444in}{0.011787in}}{\pgfqpoint{-0.044444in}{0.000000in}}%
\pgfpathcurveto{\pgfqpoint{-0.044444in}{-0.011787in}}{\pgfqpoint{-0.039761in}{-0.023092in}}{\pgfqpoint{-0.031427in}{-0.031427in}}%
\pgfpathcurveto{\pgfqpoint{-0.023092in}{-0.039761in}}{\pgfqpoint{-0.011787in}{-0.044444in}}{\pgfqpoint{0.000000in}{-0.044444in}}%
\pgfpathclose%
\pgfusepath{stroke,fill}%
}%
\begin{pgfscope}%
\pgfsys@transformshift{1.305298in}{5.347237in}%
\pgfsys@useobject{currentmarker}{}%
\end{pgfscope}%
\end{pgfscope}%
\begin{pgfscope}%
\pgfpathrectangle{\pgfqpoint{0.100000in}{2.413063in}}{\pgfqpoint{5.037500in}{3.427208in}}%
\pgfusepath{clip}%
\pgfsetrectcap%
\pgfsetroundjoin%
\pgfsetlinewidth{1.505625pt}%
\definecolor{currentstroke}{rgb}{0.678431,1.000000,0.184314}%
\pgfsetstrokecolor{currentstroke}%
\pgfsetstrokeopacity{0.500000}%
\pgfsetdash{}{0pt}%
\pgfpathmoveto{\pgfqpoint{2.513569in}{4.507898in}}%
\pgfusepath{stroke}%
\end{pgfscope}%
\begin{pgfscope}%
\pgfpathrectangle{\pgfqpoint{0.100000in}{2.413063in}}{\pgfqpoint{5.037500in}{3.427208in}}%
\pgfusepath{clip}%
\pgfsetbuttcap%
\pgfsetroundjoin%
\definecolor{currentfill}{rgb}{0.678431,1.000000,0.184314}%
\pgfsetfillcolor{currentfill}%
\pgfsetfillopacity{0.500000}%
\pgfsetlinewidth{0.250937pt}%
\definecolor{currentstroke}{rgb}{0.000000,0.000000,0.000000}%
\pgfsetstrokecolor{currentstroke}%
\pgfsetstrokeopacity{0.500000}%
\pgfsetdash{}{0pt}%
\pgfsys@defobject{currentmarker}{\pgfqpoint{-0.047222in}{-0.047222in}}{\pgfqpoint{0.047222in}{0.047222in}}{%
\pgfpathmoveto{\pgfqpoint{0.000000in}{-0.047222in}}%
\pgfpathcurveto{\pgfqpoint{0.012523in}{-0.047222in}}{\pgfqpoint{0.024536in}{-0.042247in}}{\pgfqpoint{0.033391in}{-0.033391in}}%
\pgfpathcurveto{\pgfqpoint{0.042247in}{-0.024536in}}{\pgfqpoint{0.047222in}{-0.012523in}}{\pgfqpoint{0.047222in}{0.000000in}}%
\pgfpathcurveto{\pgfqpoint{0.047222in}{0.012523in}}{\pgfqpoint{0.042247in}{0.024536in}}{\pgfqpoint{0.033391in}{0.033391in}}%
\pgfpathcurveto{\pgfqpoint{0.024536in}{0.042247in}}{\pgfqpoint{0.012523in}{0.047222in}}{\pgfqpoint{0.000000in}{0.047222in}}%
\pgfpathcurveto{\pgfqpoint{-0.012523in}{0.047222in}}{\pgfqpoint{-0.024536in}{0.042247in}}{\pgfqpoint{-0.033391in}{0.033391in}}%
\pgfpathcurveto{\pgfqpoint{-0.042247in}{0.024536in}}{\pgfqpoint{-0.047222in}{0.012523in}}{\pgfqpoint{-0.047222in}{0.000000in}}%
\pgfpathcurveto{\pgfqpoint{-0.047222in}{-0.012523in}}{\pgfqpoint{-0.042247in}{-0.024536in}}{\pgfqpoint{-0.033391in}{-0.033391in}}%
\pgfpathcurveto{\pgfqpoint{-0.024536in}{-0.042247in}}{\pgfqpoint{-0.012523in}{-0.047222in}}{\pgfqpoint{0.000000in}{-0.047222in}}%
\pgfpathclose%
\pgfusepath{stroke,fill}%
}%
\begin{pgfscope}%
\pgfsys@transformshift{2.513569in}{4.507898in}%
\pgfsys@useobject{currentmarker}{}%
\end{pgfscope}%
\end{pgfscope}%
\begin{pgfscope}%
\pgfpathrectangle{\pgfqpoint{0.100000in}{2.413063in}}{\pgfqpoint{5.037500in}{3.427208in}}%
\pgfusepath{clip}%
\pgfsetrectcap%
\pgfsetroundjoin%
\pgfsetlinewidth{1.505625pt}%
\definecolor{currentstroke}{rgb}{0.678431,1.000000,0.184314}%
\pgfsetstrokecolor{currentstroke}%
\pgfsetstrokeopacity{0.500000}%
\pgfsetdash{}{0pt}%
\pgfpathmoveto{\pgfqpoint{2.658694in}{4.489578in}}%
\pgfusepath{stroke}%
\end{pgfscope}%
\begin{pgfscope}%
\pgfpathrectangle{\pgfqpoint{0.100000in}{2.413063in}}{\pgfqpoint{5.037500in}{3.427208in}}%
\pgfusepath{clip}%
\pgfsetbuttcap%
\pgfsetroundjoin%
\definecolor{currentfill}{rgb}{0.678431,1.000000,0.184314}%
\pgfsetfillcolor{currentfill}%
\pgfsetfillopacity{0.500000}%
\pgfsetlinewidth{0.250937pt}%
\definecolor{currentstroke}{rgb}{0.000000,0.000000,0.000000}%
\pgfsetstrokecolor{currentstroke}%
\pgfsetstrokeopacity{0.500000}%
\pgfsetdash{}{0pt}%
\pgfsys@defobject{currentmarker}{\pgfqpoint{-0.033333in}{-0.033333in}}{\pgfqpoint{0.033333in}{0.033333in}}{%
\pgfpathmoveto{\pgfqpoint{0.000000in}{-0.033333in}}%
\pgfpathcurveto{\pgfqpoint{0.008840in}{-0.033333in}}{\pgfqpoint{0.017319in}{-0.029821in}}{\pgfqpoint{0.023570in}{-0.023570in}}%
\pgfpathcurveto{\pgfqpoint{0.029821in}{-0.017319in}}{\pgfqpoint{0.033333in}{-0.008840in}}{\pgfqpoint{0.033333in}{0.000000in}}%
\pgfpathcurveto{\pgfqpoint{0.033333in}{0.008840in}}{\pgfqpoint{0.029821in}{0.017319in}}{\pgfqpoint{0.023570in}{0.023570in}}%
\pgfpathcurveto{\pgfqpoint{0.017319in}{0.029821in}}{\pgfqpoint{0.008840in}{0.033333in}}{\pgfqpoint{0.000000in}{0.033333in}}%
\pgfpathcurveto{\pgfqpoint{-0.008840in}{0.033333in}}{\pgfqpoint{-0.017319in}{0.029821in}}{\pgfqpoint{-0.023570in}{0.023570in}}%
\pgfpathcurveto{\pgfqpoint{-0.029821in}{0.017319in}}{\pgfqpoint{-0.033333in}{0.008840in}}{\pgfqpoint{-0.033333in}{0.000000in}}%
\pgfpathcurveto{\pgfqpoint{-0.033333in}{-0.008840in}}{\pgfqpoint{-0.029821in}{-0.017319in}}{\pgfqpoint{-0.023570in}{-0.023570in}}%
\pgfpathcurveto{\pgfqpoint{-0.017319in}{-0.029821in}}{\pgfqpoint{-0.008840in}{-0.033333in}}{\pgfqpoint{0.000000in}{-0.033333in}}%
\pgfpathclose%
\pgfusepath{stroke,fill}%
}%
\begin{pgfscope}%
\pgfsys@transformshift{2.658694in}{4.489578in}%
\pgfsys@useobject{currentmarker}{}%
\end{pgfscope}%
\end{pgfscope}%
\begin{pgfscope}%
\pgfpathrectangle{\pgfqpoint{0.100000in}{2.413063in}}{\pgfqpoint{5.037500in}{3.427208in}}%
\pgfusepath{clip}%
\pgfsetrectcap%
\pgfsetroundjoin%
\pgfsetlinewidth{1.505625pt}%
\definecolor{currentstroke}{rgb}{0.678431,1.000000,0.184314}%
\pgfsetstrokecolor{currentstroke}%
\pgfsetstrokeopacity{0.500000}%
\pgfsetdash{}{0pt}%
\pgfpathmoveto{\pgfqpoint{2.722888in}{4.541515in}}%
\pgfusepath{stroke}%
\end{pgfscope}%
\begin{pgfscope}%
\pgfpathrectangle{\pgfqpoint{0.100000in}{2.413063in}}{\pgfqpoint{5.037500in}{3.427208in}}%
\pgfusepath{clip}%
\pgfsetbuttcap%
\pgfsetroundjoin%
\definecolor{currentfill}{rgb}{0.678431,1.000000,0.184314}%
\pgfsetfillcolor{currentfill}%
\pgfsetfillopacity{0.500000}%
\pgfsetlinewidth{0.250937pt}%
\definecolor{currentstroke}{rgb}{0.000000,0.000000,0.000000}%
\pgfsetstrokecolor{currentstroke}%
\pgfsetstrokeopacity{0.500000}%
\pgfsetdash{}{0pt}%
\pgfsys@defobject{currentmarker}{\pgfqpoint{-0.030556in}{-0.030556in}}{\pgfqpoint{0.030556in}{0.030556in}}{%
\pgfpathmoveto{\pgfqpoint{0.000000in}{-0.030556in}}%
\pgfpathcurveto{\pgfqpoint{0.008103in}{-0.030556in}}{\pgfqpoint{0.015876in}{-0.027336in}}{\pgfqpoint{0.021606in}{-0.021606in}}%
\pgfpathcurveto{\pgfqpoint{0.027336in}{-0.015876in}}{\pgfqpoint{0.030556in}{-0.008103in}}{\pgfqpoint{0.030556in}{0.000000in}}%
\pgfpathcurveto{\pgfqpoint{0.030556in}{0.008103in}}{\pgfqpoint{0.027336in}{0.015876in}}{\pgfqpoint{0.021606in}{0.021606in}}%
\pgfpathcurveto{\pgfqpoint{0.015876in}{0.027336in}}{\pgfqpoint{0.008103in}{0.030556in}}{\pgfqpoint{0.000000in}{0.030556in}}%
\pgfpathcurveto{\pgfqpoint{-0.008103in}{0.030556in}}{\pgfqpoint{-0.015876in}{0.027336in}}{\pgfqpoint{-0.021606in}{0.021606in}}%
\pgfpathcurveto{\pgfqpoint{-0.027336in}{0.015876in}}{\pgfqpoint{-0.030556in}{0.008103in}}{\pgfqpoint{-0.030556in}{0.000000in}}%
\pgfpathcurveto{\pgfqpoint{-0.030556in}{-0.008103in}}{\pgfqpoint{-0.027336in}{-0.015876in}}{\pgfqpoint{-0.021606in}{-0.021606in}}%
\pgfpathcurveto{\pgfqpoint{-0.015876in}{-0.027336in}}{\pgfqpoint{-0.008103in}{-0.030556in}}{\pgfqpoint{0.000000in}{-0.030556in}}%
\pgfpathclose%
\pgfusepath{stroke,fill}%
}%
\begin{pgfscope}%
\pgfsys@transformshift{2.722888in}{4.541515in}%
\pgfsys@useobject{currentmarker}{}%
\end{pgfscope}%
\end{pgfscope}%
\begin{pgfscope}%
\pgfpathrectangle{\pgfqpoint{0.100000in}{2.413063in}}{\pgfqpoint{5.037500in}{3.427208in}}%
\pgfusepath{clip}%
\pgfsetrectcap%
\pgfsetroundjoin%
\pgfsetlinewidth{1.505625pt}%
\definecolor{currentstroke}{rgb}{0.678431,1.000000,0.184314}%
\pgfsetstrokecolor{currentstroke}%
\pgfsetstrokeopacity{0.500000}%
\pgfsetdash{}{0pt}%
\pgfpathmoveto{\pgfqpoint{0.620199in}{4.599268in}}%
\pgfusepath{stroke}%
\end{pgfscope}%
\begin{pgfscope}%
\pgfpathrectangle{\pgfqpoint{0.100000in}{2.413063in}}{\pgfqpoint{5.037500in}{3.427208in}}%
\pgfusepath{clip}%
\pgfsetbuttcap%
\pgfsetroundjoin%
\definecolor{currentfill}{rgb}{0.678431,1.000000,0.184314}%
\pgfsetfillcolor{currentfill}%
\pgfsetfillopacity{0.500000}%
\pgfsetlinewidth{0.250937pt}%
\definecolor{currentstroke}{rgb}{0.000000,0.000000,0.000000}%
\pgfsetstrokecolor{currentstroke}%
\pgfsetstrokeopacity{0.500000}%
\pgfsetdash{}{0pt}%
\pgfsys@defobject{currentmarker}{\pgfqpoint{-0.011111in}{-0.011111in}}{\pgfqpoint{0.011111in}{0.011111in}}{%
\pgfpathmoveto{\pgfqpoint{0.000000in}{-0.011111in}}%
\pgfpathcurveto{\pgfqpoint{0.002947in}{-0.011111in}}{\pgfqpoint{0.005773in}{-0.009940in}}{\pgfqpoint{0.007857in}{-0.007857in}}%
\pgfpathcurveto{\pgfqpoint{0.009940in}{-0.005773in}}{\pgfqpoint{0.011111in}{-0.002947in}}{\pgfqpoint{0.011111in}{0.000000in}}%
\pgfpathcurveto{\pgfqpoint{0.011111in}{0.002947in}}{\pgfqpoint{0.009940in}{0.005773in}}{\pgfqpoint{0.007857in}{0.007857in}}%
\pgfpathcurveto{\pgfqpoint{0.005773in}{0.009940in}}{\pgfqpoint{0.002947in}{0.011111in}}{\pgfqpoint{0.000000in}{0.011111in}}%
\pgfpathcurveto{\pgfqpoint{-0.002947in}{0.011111in}}{\pgfqpoint{-0.005773in}{0.009940in}}{\pgfqpoint{-0.007857in}{0.007857in}}%
\pgfpathcurveto{\pgfqpoint{-0.009940in}{0.005773in}}{\pgfqpoint{-0.011111in}{0.002947in}}{\pgfqpoint{-0.011111in}{0.000000in}}%
\pgfpathcurveto{\pgfqpoint{-0.011111in}{-0.002947in}}{\pgfqpoint{-0.009940in}{-0.005773in}}{\pgfqpoint{-0.007857in}{-0.007857in}}%
\pgfpathcurveto{\pgfqpoint{-0.005773in}{-0.009940in}}{\pgfqpoint{-0.002947in}{-0.011111in}}{\pgfqpoint{0.000000in}{-0.011111in}}%
\pgfpathclose%
\pgfusepath{stroke,fill}%
}%
\begin{pgfscope}%
\pgfsys@transformshift{0.620199in}{4.599268in}%
\pgfsys@useobject{currentmarker}{}%
\end{pgfscope}%
\end{pgfscope}%
\begin{pgfscope}%
\pgfpathrectangle{\pgfqpoint{0.100000in}{2.413063in}}{\pgfqpoint{5.037500in}{3.427208in}}%
\pgfusepath{clip}%
\pgfsetrectcap%
\pgfsetroundjoin%
\pgfsetlinewidth{1.505625pt}%
\definecolor{currentstroke}{rgb}{0.000000,0.000000,1.000000}%
\pgfsetstrokecolor{currentstroke}%
\pgfsetstrokeopacity{0.500000}%
\pgfsetdash{}{0pt}%
\pgfpathmoveto{\pgfqpoint{0.944001in}{4.161574in}}%
\pgfusepath{stroke}%
\end{pgfscope}%
\begin{pgfscope}%
\pgfpathrectangle{\pgfqpoint{0.100000in}{2.413063in}}{\pgfqpoint{5.037500in}{3.427208in}}%
\pgfusepath{clip}%
\pgfsetbuttcap%
\pgfsetroundjoin%
\definecolor{currentfill}{rgb}{0.000000,0.000000,1.000000}%
\pgfsetfillcolor{currentfill}%
\pgfsetfillopacity{0.500000}%
\pgfsetlinewidth{0.250937pt}%
\definecolor{currentstroke}{rgb}{0.000000,0.000000,0.000000}%
\pgfsetstrokecolor{currentstroke}%
\pgfsetstrokeopacity{0.500000}%
\pgfsetdash{}{0pt}%
\pgfsys@defobject{currentmarker}{\pgfqpoint{-0.066667in}{-0.066667in}}{\pgfqpoint{0.066667in}{0.066667in}}{%
\pgfpathmoveto{\pgfqpoint{0.000000in}{-0.066667in}}%
\pgfpathcurveto{\pgfqpoint{0.017680in}{-0.066667in}}{\pgfqpoint{0.034639in}{-0.059642in}}{\pgfqpoint{0.047140in}{-0.047140in}}%
\pgfpathcurveto{\pgfqpoint{0.059642in}{-0.034639in}}{\pgfqpoint{0.066667in}{-0.017680in}}{\pgfqpoint{0.066667in}{0.000000in}}%
\pgfpathcurveto{\pgfqpoint{0.066667in}{0.017680in}}{\pgfqpoint{0.059642in}{0.034639in}}{\pgfqpoint{0.047140in}{0.047140in}}%
\pgfpathcurveto{\pgfqpoint{0.034639in}{0.059642in}}{\pgfqpoint{0.017680in}{0.066667in}}{\pgfqpoint{0.000000in}{0.066667in}}%
\pgfpathcurveto{\pgfqpoint{-0.017680in}{0.066667in}}{\pgfqpoint{-0.034639in}{0.059642in}}{\pgfqpoint{-0.047140in}{0.047140in}}%
\pgfpathcurveto{\pgfqpoint{-0.059642in}{0.034639in}}{\pgfqpoint{-0.066667in}{0.017680in}}{\pgfqpoint{-0.066667in}{0.000000in}}%
\pgfpathcurveto{\pgfqpoint{-0.066667in}{-0.017680in}}{\pgfqpoint{-0.059642in}{-0.034639in}}{\pgfqpoint{-0.047140in}{-0.047140in}}%
\pgfpathcurveto{\pgfqpoint{-0.034639in}{-0.059642in}}{\pgfqpoint{-0.017680in}{-0.066667in}}{\pgfqpoint{0.000000in}{-0.066667in}}%
\pgfpathclose%
\pgfusepath{stroke,fill}%
}%
\begin{pgfscope}%
\pgfsys@transformshift{0.944001in}{4.161574in}%
\pgfsys@useobject{currentmarker}{}%
\end{pgfscope}%
\end{pgfscope}%
\begin{pgfscope}%
\pgfpathrectangle{\pgfqpoint{0.100000in}{2.413063in}}{\pgfqpoint{5.037500in}{3.427208in}}%
\pgfusepath{clip}%
\pgfsetrectcap%
\pgfsetroundjoin%
\pgfsetlinewidth{1.505625pt}%
\definecolor{currentstroke}{rgb}{0.501961,0.501961,0.501961}%
\pgfsetstrokecolor{currentstroke}%
\pgfsetstrokeopacity{0.500000}%
\pgfsetdash{}{0pt}%
\pgfpathmoveto{\pgfqpoint{0.627569in}{4.640906in}}%
\pgfusepath{stroke}%
\end{pgfscope}%
\begin{pgfscope}%
\pgfpathrectangle{\pgfqpoint{0.100000in}{2.413063in}}{\pgfqpoint{5.037500in}{3.427208in}}%
\pgfusepath{clip}%
\pgfsetbuttcap%
\pgfsetroundjoin%
\definecolor{currentfill}{rgb}{0.501961,0.501961,0.501961}%
\pgfsetfillcolor{currentfill}%
\pgfsetfillopacity{0.500000}%
\pgfsetlinewidth{0.250937pt}%
\definecolor{currentstroke}{rgb}{0.000000,0.000000,0.000000}%
\pgfsetstrokecolor{currentstroke}%
\pgfsetstrokeopacity{0.500000}%
\pgfsetdash{}{0pt}%
\pgfsys@defobject{currentmarker}{\pgfqpoint{-0.013889in}{-0.013889in}}{\pgfqpoint{0.013889in}{0.013889in}}{%
\pgfpathmoveto{\pgfqpoint{0.000000in}{-0.013889in}}%
\pgfpathcurveto{\pgfqpoint{0.003683in}{-0.013889in}}{\pgfqpoint{0.007216in}{-0.012425in}}{\pgfqpoint{0.009821in}{-0.009821in}}%
\pgfpathcurveto{\pgfqpoint{0.012425in}{-0.007216in}}{\pgfqpoint{0.013889in}{-0.003683in}}{\pgfqpoint{0.013889in}{0.000000in}}%
\pgfpathcurveto{\pgfqpoint{0.013889in}{0.003683in}}{\pgfqpoint{0.012425in}{0.007216in}}{\pgfqpoint{0.009821in}{0.009821in}}%
\pgfpathcurveto{\pgfqpoint{0.007216in}{0.012425in}}{\pgfqpoint{0.003683in}{0.013889in}}{\pgfqpoint{0.000000in}{0.013889in}}%
\pgfpathcurveto{\pgfqpoint{-0.003683in}{0.013889in}}{\pgfqpoint{-0.007216in}{0.012425in}}{\pgfqpoint{-0.009821in}{0.009821in}}%
\pgfpathcurveto{\pgfqpoint{-0.012425in}{0.007216in}}{\pgfqpoint{-0.013889in}{0.003683in}}{\pgfqpoint{-0.013889in}{0.000000in}}%
\pgfpathcurveto{\pgfqpoint{-0.013889in}{-0.003683in}}{\pgfqpoint{-0.012425in}{-0.007216in}}{\pgfqpoint{-0.009821in}{-0.009821in}}%
\pgfpathcurveto{\pgfqpoint{-0.007216in}{-0.012425in}}{\pgfqpoint{-0.003683in}{-0.013889in}}{\pgfqpoint{0.000000in}{-0.013889in}}%
\pgfpathclose%
\pgfusepath{stroke,fill}%
}%
\begin{pgfscope}%
\pgfsys@transformshift{0.627569in}{4.640906in}%
\pgfsys@useobject{currentmarker}{}%
\end{pgfscope}%
\end{pgfscope}%
\begin{pgfscope}%
\pgfpathrectangle{\pgfqpoint{0.100000in}{2.413063in}}{\pgfqpoint{5.037500in}{3.427208in}}%
\pgfusepath{clip}%
\pgfsetrectcap%
\pgfsetroundjoin%
\pgfsetlinewidth{1.505625pt}%
\definecolor{currentstroke}{rgb}{0.501961,0.501961,0.501961}%
\pgfsetstrokecolor{currentstroke}%
\pgfsetstrokeopacity{0.500000}%
\pgfsetdash{}{0pt}%
\pgfpathmoveto{\pgfqpoint{4.810896in}{5.032692in}}%
\pgfusepath{stroke}%
\end{pgfscope}%
\begin{pgfscope}%
\pgfpathrectangle{\pgfqpoint{0.100000in}{2.413063in}}{\pgfqpoint{5.037500in}{3.427208in}}%
\pgfusepath{clip}%
\pgfsetbuttcap%
\pgfsetroundjoin%
\definecolor{currentfill}{rgb}{0.501961,0.501961,0.501961}%
\pgfsetfillcolor{currentfill}%
\pgfsetfillopacity{0.500000}%
\pgfsetlinewidth{0.250937pt}%
\definecolor{currentstroke}{rgb}{0.000000,0.000000,0.000000}%
\pgfsetstrokecolor{currentstroke}%
\pgfsetstrokeopacity{0.500000}%
\pgfsetdash{}{0pt}%
\pgfsys@defobject{currentmarker}{\pgfqpoint{-0.013889in}{-0.013889in}}{\pgfqpoint{0.013889in}{0.013889in}}{%
\pgfpathmoveto{\pgfqpoint{0.000000in}{-0.013889in}}%
\pgfpathcurveto{\pgfqpoint{0.003683in}{-0.013889in}}{\pgfqpoint{0.007216in}{-0.012425in}}{\pgfqpoint{0.009821in}{-0.009821in}}%
\pgfpathcurveto{\pgfqpoint{0.012425in}{-0.007216in}}{\pgfqpoint{0.013889in}{-0.003683in}}{\pgfqpoint{0.013889in}{0.000000in}}%
\pgfpathcurveto{\pgfqpoint{0.013889in}{0.003683in}}{\pgfqpoint{0.012425in}{0.007216in}}{\pgfqpoint{0.009821in}{0.009821in}}%
\pgfpathcurveto{\pgfqpoint{0.007216in}{0.012425in}}{\pgfqpoint{0.003683in}{0.013889in}}{\pgfqpoint{0.000000in}{0.013889in}}%
\pgfpathcurveto{\pgfqpoint{-0.003683in}{0.013889in}}{\pgfqpoint{-0.007216in}{0.012425in}}{\pgfqpoint{-0.009821in}{0.009821in}}%
\pgfpathcurveto{\pgfqpoint{-0.012425in}{0.007216in}}{\pgfqpoint{-0.013889in}{0.003683in}}{\pgfqpoint{-0.013889in}{0.000000in}}%
\pgfpathcurveto{\pgfqpoint{-0.013889in}{-0.003683in}}{\pgfqpoint{-0.012425in}{-0.007216in}}{\pgfqpoint{-0.009821in}{-0.009821in}}%
\pgfpathcurveto{\pgfqpoint{-0.007216in}{-0.012425in}}{\pgfqpoint{-0.003683in}{-0.013889in}}{\pgfqpoint{0.000000in}{-0.013889in}}%
\pgfpathclose%
\pgfusepath{stroke,fill}%
}%
\begin{pgfscope}%
\pgfsys@transformshift{4.810896in}{5.032692in}%
\pgfsys@useobject{currentmarker}{}%
\end{pgfscope}%
\end{pgfscope}%
\begin{pgfscope}%
\pgfpathrectangle{\pgfqpoint{0.100000in}{2.413063in}}{\pgfqpoint{5.037500in}{3.427208in}}%
\pgfusepath{clip}%
\pgfsetrectcap%
\pgfsetroundjoin%
\pgfsetlinewidth{1.505625pt}%
\definecolor{currentstroke}{rgb}{0.501961,0.501961,0.501961}%
\pgfsetstrokecolor{currentstroke}%
\pgfsetstrokeopacity{0.500000}%
\pgfsetdash{}{0pt}%
\pgfpathmoveto{\pgfqpoint{4.769641in}{4.997446in}}%
\pgfusepath{stroke}%
\end{pgfscope}%
\begin{pgfscope}%
\pgfpathrectangle{\pgfqpoint{0.100000in}{2.413063in}}{\pgfqpoint{5.037500in}{3.427208in}}%
\pgfusepath{clip}%
\pgfsetbuttcap%
\pgfsetroundjoin%
\definecolor{currentfill}{rgb}{0.501961,0.501961,0.501961}%
\pgfsetfillcolor{currentfill}%
\pgfsetfillopacity{0.500000}%
\pgfsetlinewidth{0.250937pt}%
\definecolor{currentstroke}{rgb}{0.000000,0.000000,0.000000}%
\pgfsetstrokecolor{currentstroke}%
\pgfsetstrokeopacity{0.500000}%
\pgfsetdash{}{0pt}%
\pgfsys@defobject{currentmarker}{\pgfqpoint{-0.013889in}{-0.013889in}}{\pgfqpoint{0.013889in}{0.013889in}}{%
\pgfpathmoveto{\pgfqpoint{0.000000in}{-0.013889in}}%
\pgfpathcurveto{\pgfqpoint{0.003683in}{-0.013889in}}{\pgfqpoint{0.007216in}{-0.012425in}}{\pgfqpoint{0.009821in}{-0.009821in}}%
\pgfpathcurveto{\pgfqpoint{0.012425in}{-0.007216in}}{\pgfqpoint{0.013889in}{-0.003683in}}{\pgfqpoint{0.013889in}{0.000000in}}%
\pgfpathcurveto{\pgfqpoint{0.013889in}{0.003683in}}{\pgfqpoint{0.012425in}{0.007216in}}{\pgfqpoint{0.009821in}{0.009821in}}%
\pgfpathcurveto{\pgfqpoint{0.007216in}{0.012425in}}{\pgfqpoint{0.003683in}{0.013889in}}{\pgfqpoint{0.000000in}{0.013889in}}%
\pgfpathcurveto{\pgfqpoint{-0.003683in}{0.013889in}}{\pgfqpoint{-0.007216in}{0.012425in}}{\pgfqpoint{-0.009821in}{0.009821in}}%
\pgfpathcurveto{\pgfqpoint{-0.012425in}{0.007216in}}{\pgfqpoint{-0.013889in}{0.003683in}}{\pgfqpoint{-0.013889in}{0.000000in}}%
\pgfpathcurveto{\pgfqpoint{-0.013889in}{-0.003683in}}{\pgfqpoint{-0.012425in}{-0.007216in}}{\pgfqpoint{-0.009821in}{-0.009821in}}%
\pgfpathcurveto{\pgfqpoint{-0.007216in}{-0.012425in}}{\pgfqpoint{-0.003683in}{-0.013889in}}{\pgfqpoint{0.000000in}{-0.013889in}}%
\pgfpathclose%
\pgfusepath{stroke,fill}%
}%
\begin{pgfscope}%
\pgfsys@transformshift{4.769641in}{4.997446in}%
\pgfsys@useobject{currentmarker}{}%
\end{pgfscope}%
\end{pgfscope}%
\begin{pgfscope}%
\pgfpathrectangle{\pgfqpoint{0.100000in}{2.413063in}}{\pgfqpoint{5.037500in}{3.427208in}}%
\pgfusepath{clip}%
\pgfsetrectcap%
\pgfsetroundjoin%
\pgfsetlinewidth{1.505625pt}%
\definecolor{currentstroke}{rgb}{0.000000,0.000000,1.000000}%
\pgfsetstrokecolor{currentstroke}%
\pgfsetstrokeopacity{0.500000}%
\pgfsetdash{}{0pt}%
\pgfpathmoveto{\pgfqpoint{4.823756in}{5.021108in}}%
\pgfusepath{stroke}%
\end{pgfscope}%
\begin{pgfscope}%
\pgfpathrectangle{\pgfqpoint{0.100000in}{2.413063in}}{\pgfqpoint{5.037500in}{3.427208in}}%
\pgfusepath{clip}%
\pgfsetbuttcap%
\pgfsetroundjoin%
\definecolor{currentfill}{rgb}{0.000000,0.000000,1.000000}%
\pgfsetfillcolor{currentfill}%
\pgfsetfillopacity{0.500000}%
\pgfsetlinewidth{0.250937pt}%
\definecolor{currentstroke}{rgb}{0.000000,0.000000,0.000000}%
\pgfsetstrokecolor{currentstroke}%
\pgfsetstrokeopacity{0.500000}%
\pgfsetdash{}{0pt}%
\pgfsys@defobject{currentmarker}{\pgfqpoint{-0.005556in}{-0.005556in}}{\pgfqpoint{0.005556in}{0.005556in}}{%
\pgfpathmoveto{\pgfqpoint{0.000000in}{-0.005556in}}%
\pgfpathcurveto{\pgfqpoint{0.001473in}{-0.005556in}}{\pgfqpoint{0.002887in}{-0.004970in}}{\pgfqpoint{0.003928in}{-0.003928in}}%
\pgfpathcurveto{\pgfqpoint{0.004970in}{-0.002887in}}{\pgfqpoint{0.005556in}{-0.001473in}}{\pgfqpoint{0.005556in}{0.000000in}}%
\pgfpathcurveto{\pgfqpoint{0.005556in}{0.001473in}}{\pgfqpoint{0.004970in}{0.002887in}}{\pgfqpoint{0.003928in}{0.003928in}}%
\pgfpathcurveto{\pgfqpoint{0.002887in}{0.004970in}}{\pgfqpoint{0.001473in}{0.005556in}}{\pgfqpoint{0.000000in}{0.005556in}}%
\pgfpathcurveto{\pgfqpoint{-0.001473in}{0.005556in}}{\pgfqpoint{-0.002887in}{0.004970in}}{\pgfqpoint{-0.003928in}{0.003928in}}%
\pgfpathcurveto{\pgfqpoint{-0.004970in}{0.002887in}}{\pgfqpoint{-0.005556in}{0.001473in}}{\pgfqpoint{-0.005556in}{0.000000in}}%
\pgfpathcurveto{\pgfqpoint{-0.005556in}{-0.001473in}}{\pgfqpoint{-0.004970in}{-0.002887in}}{\pgfqpoint{-0.003928in}{-0.003928in}}%
\pgfpathcurveto{\pgfqpoint{-0.002887in}{-0.004970in}}{\pgfqpoint{-0.001473in}{-0.005556in}}{\pgfqpoint{0.000000in}{-0.005556in}}%
\pgfpathclose%
\pgfusepath{stroke,fill}%
}%
\begin{pgfscope}%
\pgfsys@transformshift{4.823756in}{5.021108in}%
\pgfsys@useobject{currentmarker}{}%
\end{pgfscope}%
\end{pgfscope}%
\begin{pgfscope}%
\pgfpathrectangle{\pgfqpoint{0.100000in}{2.413063in}}{\pgfqpoint{5.037500in}{3.427208in}}%
\pgfusepath{clip}%
\pgfsetrectcap%
\pgfsetroundjoin%
\pgfsetlinewidth{1.505625pt}%
\definecolor{currentstroke}{rgb}{0.000000,0.000000,1.000000}%
\pgfsetstrokecolor{currentstroke}%
\pgfsetstrokeopacity{0.500000}%
\pgfsetdash{}{0pt}%
\pgfpathmoveto{\pgfqpoint{4.620381in}{4.529404in}}%
\pgfusepath{stroke}%
\end{pgfscope}%
\begin{pgfscope}%
\pgfpathrectangle{\pgfqpoint{0.100000in}{2.413063in}}{\pgfqpoint{5.037500in}{3.427208in}}%
\pgfusepath{clip}%
\pgfsetbuttcap%
\pgfsetroundjoin%
\definecolor{currentfill}{rgb}{0.000000,0.000000,1.000000}%
\pgfsetfillcolor{currentfill}%
\pgfsetfillopacity{0.500000}%
\pgfsetlinewidth{0.250937pt}%
\definecolor{currentstroke}{rgb}{0.000000,0.000000,0.000000}%
\pgfsetstrokecolor{currentstroke}%
\pgfsetstrokeopacity{0.500000}%
\pgfsetdash{}{0pt}%
\pgfsys@defobject{currentmarker}{\pgfqpoint{-0.069444in}{-0.069444in}}{\pgfqpoint{0.069444in}{0.069444in}}{%
\pgfpathmoveto{\pgfqpoint{0.000000in}{-0.069444in}}%
\pgfpathcurveto{\pgfqpoint{0.018417in}{-0.069444in}}{\pgfqpoint{0.036082in}{-0.062127in}}{\pgfqpoint{0.049105in}{-0.049105in}}%
\pgfpathcurveto{\pgfqpoint{0.062127in}{-0.036082in}}{\pgfqpoint{0.069444in}{-0.018417in}}{\pgfqpoint{0.069444in}{0.000000in}}%
\pgfpathcurveto{\pgfqpoint{0.069444in}{0.018417in}}{\pgfqpoint{0.062127in}{0.036082in}}{\pgfqpoint{0.049105in}{0.049105in}}%
\pgfpathcurveto{\pgfqpoint{0.036082in}{0.062127in}}{\pgfqpoint{0.018417in}{0.069444in}}{\pgfqpoint{0.000000in}{0.069444in}}%
\pgfpathcurveto{\pgfqpoint{-0.018417in}{0.069444in}}{\pgfqpoint{-0.036082in}{0.062127in}}{\pgfqpoint{-0.049105in}{0.049105in}}%
\pgfpathcurveto{\pgfqpoint{-0.062127in}{0.036082in}}{\pgfqpoint{-0.069444in}{0.018417in}}{\pgfqpoint{-0.069444in}{0.000000in}}%
\pgfpathcurveto{\pgfqpoint{-0.069444in}{-0.018417in}}{\pgfqpoint{-0.062127in}{-0.036082in}}{\pgfqpoint{-0.049105in}{-0.049105in}}%
\pgfpathcurveto{\pgfqpoint{-0.036082in}{-0.062127in}}{\pgfqpoint{-0.018417in}{-0.069444in}}{\pgfqpoint{0.000000in}{-0.069444in}}%
\pgfpathclose%
\pgfusepath{stroke,fill}%
}%
\begin{pgfscope}%
\pgfsys@transformshift{4.620381in}{4.529404in}%
\pgfsys@useobject{currentmarker}{}%
\end{pgfscope}%
\end{pgfscope}%
\begin{pgfscope}%
\pgfpathrectangle{\pgfqpoint{0.100000in}{2.413063in}}{\pgfqpoint{5.037500in}{3.427208in}}%
\pgfusepath{clip}%
\pgfsetrectcap%
\pgfsetroundjoin%
\pgfsetlinewidth{1.505625pt}%
\definecolor{currentstroke}{rgb}{0.678431,1.000000,0.184314}%
\pgfsetstrokecolor{currentstroke}%
\pgfsetstrokeopacity{0.500000}%
\pgfsetdash{}{0pt}%
\pgfpathmoveto{\pgfqpoint{4.609445in}{4.516655in}}%
\pgfusepath{stroke}%
\end{pgfscope}%
\begin{pgfscope}%
\pgfpathrectangle{\pgfqpoint{0.100000in}{2.413063in}}{\pgfqpoint{5.037500in}{3.427208in}}%
\pgfusepath{clip}%
\pgfsetbuttcap%
\pgfsetroundjoin%
\definecolor{currentfill}{rgb}{0.678431,1.000000,0.184314}%
\pgfsetfillcolor{currentfill}%
\pgfsetfillopacity{0.500000}%
\pgfsetlinewidth{0.250937pt}%
\definecolor{currentstroke}{rgb}{0.000000,0.000000,0.000000}%
\pgfsetstrokecolor{currentstroke}%
\pgfsetstrokeopacity{0.500000}%
\pgfsetdash{}{0pt}%
\pgfsys@defobject{currentmarker}{\pgfqpoint{-0.025000in}{-0.025000in}}{\pgfqpoint{0.025000in}{0.025000in}}{%
\pgfpathmoveto{\pgfqpoint{0.000000in}{-0.025000in}}%
\pgfpathcurveto{\pgfqpoint{0.006630in}{-0.025000in}}{\pgfqpoint{0.012989in}{-0.022366in}}{\pgfqpoint{0.017678in}{-0.017678in}}%
\pgfpathcurveto{\pgfqpoint{0.022366in}{-0.012989in}}{\pgfqpoint{0.025000in}{-0.006630in}}{\pgfqpoint{0.025000in}{0.000000in}}%
\pgfpathcurveto{\pgfqpoint{0.025000in}{0.006630in}}{\pgfqpoint{0.022366in}{0.012989in}}{\pgfqpoint{0.017678in}{0.017678in}}%
\pgfpathcurveto{\pgfqpoint{0.012989in}{0.022366in}}{\pgfqpoint{0.006630in}{0.025000in}}{\pgfqpoint{0.000000in}{0.025000in}}%
\pgfpathcurveto{\pgfqpoint{-0.006630in}{0.025000in}}{\pgfqpoint{-0.012989in}{0.022366in}}{\pgfqpoint{-0.017678in}{0.017678in}}%
\pgfpathcurveto{\pgfqpoint{-0.022366in}{0.012989in}}{\pgfqpoint{-0.025000in}{0.006630in}}{\pgfqpoint{-0.025000in}{0.000000in}}%
\pgfpathcurveto{\pgfqpoint{-0.025000in}{-0.006630in}}{\pgfqpoint{-0.022366in}{-0.012989in}}{\pgfqpoint{-0.017678in}{-0.017678in}}%
\pgfpathcurveto{\pgfqpoint{-0.012989in}{-0.022366in}}{\pgfqpoint{-0.006630in}{-0.025000in}}{\pgfqpoint{0.000000in}{-0.025000in}}%
\pgfpathclose%
\pgfusepath{stroke,fill}%
}%
\begin{pgfscope}%
\pgfsys@transformshift{4.609445in}{4.516655in}%
\pgfsys@useobject{currentmarker}{}%
\end{pgfscope}%
\end{pgfscope}%
\begin{pgfscope}%
\pgfpathrectangle{\pgfqpoint{0.100000in}{2.413063in}}{\pgfqpoint{5.037500in}{3.427208in}}%
\pgfusepath{clip}%
\pgfsetrectcap%
\pgfsetroundjoin%
\pgfsetlinewidth{1.505625pt}%
\definecolor{currentstroke}{rgb}{0.000000,0.000000,1.000000}%
\pgfsetstrokecolor{currentstroke}%
\pgfsetstrokeopacity{0.500000}%
\pgfsetdash{}{0pt}%
\pgfpathmoveto{\pgfqpoint{4.570899in}{4.618831in}}%
\pgfusepath{stroke}%
\end{pgfscope}%
\begin{pgfscope}%
\pgfpathrectangle{\pgfqpoint{0.100000in}{2.413063in}}{\pgfqpoint{5.037500in}{3.427208in}}%
\pgfusepath{clip}%
\pgfsetbuttcap%
\pgfsetroundjoin%
\definecolor{currentfill}{rgb}{0.000000,0.000000,1.000000}%
\pgfsetfillcolor{currentfill}%
\pgfsetfillopacity{0.500000}%
\pgfsetlinewidth{0.250937pt}%
\definecolor{currentstroke}{rgb}{0.000000,0.000000,0.000000}%
\pgfsetstrokecolor{currentstroke}%
\pgfsetstrokeopacity{0.500000}%
\pgfsetdash{}{0pt}%
\pgfsys@defobject{currentmarker}{\pgfqpoint{-0.027778in}{-0.027778in}}{\pgfqpoint{0.027778in}{0.027778in}}{%
\pgfpathmoveto{\pgfqpoint{0.000000in}{-0.027778in}}%
\pgfpathcurveto{\pgfqpoint{0.007367in}{-0.027778in}}{\pgfqpoint{0.014433in}{-0.024851in}}{\pgfqpoint{0.019642in}{-0.019642in}}%
\pgfpathcurveto{\pgfqpoint{0.024851in}{-0.014433in}}{\pgfqpoint{0.027778in}{-0.007367in}}{\pgfqpoint{0.027778in}{0.000000in}}%
\pgfpathcurveto{\pgfqpoint{0.027778in}{0.007367in}}{\pgfqpoint{0.024851in}{0.014433in}}{\pgfqpoint{0.019642in}{0.019642in}}%
\pgfpathcurveto{\pgfqpoint{0.014433in}{0.024851in}}{\pgfqpoint{0.007367in}{0.027778in}}{\pgfqpoint{0.000000in}{0.027778in}}%
\pgfpathcurveto{\pgfqpoint{-0.007367in}{0.027778in}}{\pgfqpoint{-0.014433in}{0.024851in}}{\pgfqpoint{-0.019642in}{0.019642in}}%
\pgfpathcurveto{\pgfqpoint{-0.024851in}{0.014433in}}{\pgfqpoint{-0.027778in}{0.007367in}}{\pgfqpoint{-0.027778in}{0.000000in}}%
\pgfpathcurveto{\pgfqpoint{-0.027778in}{-0.007367in}}{\pgfqpoint{-0.024851in}{-0.014433in}}{\pgfqpoint{-0.019642in}{-0.019642in}}%
\pgfpathcurveto{\pgfqpoint{-0.014433in}{-0.024851in}}{\pgfqpoint{-0.007367in}{-0.027778in}}{\pgfqpoint{0.000000in}{-0.027778in}}%
\pgfpathclose%
\pgfusepath{stroke,fill}%
}%
\begin{pgfscope}%
\pgfsys@transformshift{4.570899in}{4.618831in}%
\pgfsys@useobject{currentmarker}{}%
\end{pgfscope}%
\end{pgfscope}%
\begin{pgfscope}%
\pgfpathrectangle{\pgfqpoint{0.100000in}{2.413063in}}{\pgfqpoint{5.037500in}{3.427208in}}%
\pgfusepath{clip}%
\pgfsetrectcap%
\pgfsetroundjoin%
\pgfsetlinewidth{1.505625pt}%
\definecolor{currentstroke}{rgb}{0.000000,0.000000,1.000000}%
\pgfsetstrokecolor{currentstroke}%
\pgfsetstrokeopacity{0.500000}%
\pgfsetdash{}{0pt}%
\pgfpathmoveto{\pgfqpoint{4.567371in}{4.530224in}}%
\pgfusepath{stroke}%
\end{pgfscope}%
\begin{pgfscope}%
\pgfpathrectangle{\pgfqpoint{0.100000in}{2.413063in}}{\pgfqpoint{5.037500in}{3.427208in}}%
\pgfusepath{clip}%
\pgfsetbuttcap%
\pgfsetroundjoin%
\definecolor{currentfill}{rgb}{0.000000,0.000000,1.000000}%
\pgfsetfillcolor{currentfill}%
\pgfsetfillopacity{0.500000}%
\pgfsetlinewidth{0.250937pt}%
\definecolor{currentstroke}{rgb}{0.000000,0.000000,0.000000}%
\pgfsetstrokecolor{currentstroke}%
\pgfsetstrokeopacity{0.500000}%
\pgfsetdash{}{0pt}%
\pgfsys@defobject{currentmarker}{\pgfqpoint{-0.033333in}{-0.033333in}}{\pgfqpoint{0.033333in}{0.033333in}}{%
\pgfpathmoveto{\pgfqpoint{0.000000in}{-0.033333in}}%
\pgfpathcurveto{\pgfqpoint{0.008840in}{-0.033333in}}{\pgfqpoint{0.017319in}{-0.029821in}}{\pgfqpoint{0.023570in}{-0.023570in}}%
\pgfpathcurveto{\pgfqpoint{0.029821in}{-0.017319in}}{\pgfqpoint{0.033333in}{-0.008840in}}{\pgfqpoint{0.033333in}{0.000000in}}%
\pgfpathcurveto{\pgfqpoint{0.033333in}{0.008840in}}{\pgfqpoint{0.029821in}{0.017319in}}{\pgfqpoint{0.023570in}{0.023570in}}%
\pgfpathcurveto{\pgfqpoint{0.017319in}{0.029821in}}{\pgfqpoint{0.008840in}{0.033333in}}{\pgfqpoint{0.000000in}{0.033333in}}%
\pgfpathcurveto{\pgfqpoint{-0.008840in}{0.033333in}}{\pgfqpoint{-0.017319in}{0.029821in}}{\pgfqpoint{-0.023570in}{0.023570in}}%
\pgfpathcurveto{\pgfqpoint{-0.029821in}{0.017319in}}{\pgfqpoint{-0.033333in}{0.008840in}}{\pgfqpoint{-0.033333in}{0.000000in}}%
\pgfpathcurveto{\pgfqpoint{-0.033333in}{-0.008840in}}{\pgfqpoint{-0.029821in}{-0.017319in}}{\pgfqpoint{-0.023570in}{-0.023570in}}%
\pgfpathcurveto{\pgfqpoint{-0.017319in}{-0.029821in}}{\pgfqpoint{-0.008840in}{-0.033333in}}{\pgfqpoint{0.000000in}{-0.033333in}}%
\pgfpathclose%
\pgfusepath{stroke,fill}%
}%
\begin{pgfscope}%
\pgfsys@transformshift{4.567371in}{4.530224in}%
\pgfsys@useobject{currentmarker}{}%
\end{pgfscope}%
\end{pgfscope}%
\begin{pgfscope}%
\pgfpathrectangle{\pgfqpoint{0.100000in}{2.413063in}}{\pgfqpoint{5.037500in}{3.427208in}}%
\pgfusepath{clip}%
\pgfsetrectcap%
\pgfsetroundjoin%
\pgfsetlinewidth{1.505625pt}%
\definecolor{currentstroke}{rgb}{0.000000,0.000000,1.000000}%
\pgfsetstrokecolor{currentstroke}%
\pgfsetstrokeopacity{0.500000}%
\pgfsetdash{}{0pt}%
\pgfpathmoveto{\pgfqpoint{1.706681in}{3.900122in}}%
\pgfusepath{stroke}%
\end{pgfscope}%
\begin{pgfscope}%
\pgfpathrectangle{\pgfqpoint{0.100000in}{2.413063in}}{\pgfqpoint{5.037500in}{3.427208in}}%
\pgfusepath{clip}%
\pgfsetbuttcap%
\pgfsetroundjoin%
\definecolor{currentfill}{rgb}{0.000000,0.000000,1.000000}%
\pgfsetfillcolor{currentfill}%
\pgfsetfillopacity{0.500000}%
\pgfsetlinewidth{0.250937pt}%
\definecolor{currentstroke}{rgb}{0.000000,0.000000,0.000000}%
\pgfsetstrokecolor{currentstroke}%
\pgfsetstrokeopacity{0.500000}%
\pgfsetdash{}{0pt}%
\pgfsys@defobject{currentmarker}{\pgfqpoint{-0.008333in}{-0.008333in}}{\pgfqpoint{0.008333in}{0.008333in}}{%
\pgfpathmoveto{\pgfqpoint{0.000000in}{-0.008333in}}%
\pgfpathcurveto{\pgfqpoint{0.002210in}{-0.008333in}}{\pgfqpoint{0.004330in}{-0.007455in}}{\pgfqpoint{0.005893in}{-0.005893in}}%
\pgfpathcurveto{\pgfqpoint{0.007455in}{-0.004330in}}{\pgfqpoint{0.008333in}{-0.002210in}}{\pgfqpoint{0.008333in}{0.000000in}}%
\pgfpathcurveto{\pgfqpoint{0.008333in}{0.002210in}}{\pgfqpoint{0.007455in}{0.004330in}}{\pgfqpoint{0.005893in}{0.005893in}}%
\pgfpathcurveto{\pgfqpoint{0.004330in}{0.007455in}}{\pgfqpoint{0.002210in}{0.008333in}}{\pgfqpoint{0.000000in}{0.008333in}}%
\pgfpathcurveto{\pgfqpoint{-0.002210in}{0.008333in}}{\pgfqpoint{-0.004330in}{0.007455in}}{\pgfqpoint{-0.005893in}{0.005893in}}%
\pgfpathcurveto{\pgfqpoint{-0.007455in}{0.004330in}}{\pgfqpoint{-0.008333in}{0.002210in}}{\pgfqpoint{-0.008333in}{0.000000in}}%
\pgfpathcurveto{\pgfqpoint{-0.008333in}{-0.002210in}}{\pgfqpoint{-0.007455in}{-0.004330in}}{\pgfqpoint{-0.005893in}{-0.005893in}}%
\pgfpathcurveto{\pgfqpoint{-0.004330in}{-0.007455in}}{\pgfqpoint{-0.002210in}{-0.008333in}}{\pgfqpoint{0.000000in}{-0.008333in}}%
\pgfpathclose%
\pgfusepath{stroke,fill}%
}%
\begin{pgfscope}%
\pgfsys@transformshift{1.706681in}{3.900122in}%
\pgfsys@useobject{currentmarker}{}%
\end{pgfscope}%
\end{pgfscope}%
\begin{pgfscope}%
\pgfpathrectangle{\pgfqpoint{0.100000in}{2.413063in}}{\pgfqpoint{5.037500in}{3.427208in}}%
\pgfusepath{clip}%
\pgfsetrectcap%
\pgfsetroundjoin%
\pgfsetlinewidth{1.505625pt}%
\definecolor{currentstroke}{rgb}{0.501961,0.501961,0.501961}%
\pgfsetstrokecolor{currentstroke}%
\pgfsetstrokeopacity{0.500000}%
\pgfsetdash{}{0pt}%
\pgfpathmoveto{\pgfqpoint{1.588407in}{4.108047in}}%
\pgfusepath{stroke}%
\end{pgfscope}%
\begin{pgfscope}%
\pgfpathrectangle{\pgfqpoint{0.100000in}{2.413063in}}{\pgfqpoint{5.037500in}{3.427208in}}%
\pgfusepath{clip}%
\pgfsetbuttcap%
\pgfsetroundjoin%
\definecolor{currentfill}{rgb}{0.501961,0.501961,0.501961}%
\pgfsetfillcolor{currentfill}%
\pgfsetfillopacity{0.500000}%
\pgfsetlinewidth{0.250937pt}%
\definecolor{currentstroke}{rgb}{0.000000,0.000000,0.000000}%
\pgfsetstrokecolor{currentstroke}%
\pgfsetstrokeopacity{0.500000}%
\pgfsetdash{}{0pt}%
\pgfsys@defobject{currentmarker}{\pgfqpoint{-0.013889in}{-0.013889in}}{\pgfqpoint{0.013889in}{0.013889in}}{%
\pgfpathmoveto{\pgfqpoint{0.000000in}{-0.013889in}}%
\pgfpathcurveto{\pgfqpoint{0.003683in}{-0.013889in}}{\pgfqpoint{0.007216in}{-0.012425in}}{\pgfqpoint{0.009821in}{-0.009821in}}%
\pgfpathcurveto{\pgfqpoint{0.012425in}{-0.007216in}}{\pgfqpoint{0.013889in}{-0.003683in}}{\pgfqpoint{0.013889in}{0.000000in}}%
\pgfpathcurveto{\pgfqpoint{0.013889in}{0.003683in}}{\pgfqpoint{0.012425in}{0.007216in}}{\pgfqpoint{0.009821in}{0.009821in}}%
\pgfpathcurveto{\pgfqpoint{0.007216in}{0.012425in}}{\pgfqpoint{0.003683in}{0.013889in}}{\pgfqpoint{0.000000in}{0.013889in}}%
\pgfpathcurveto{\pgfqpoint{-0.003683in}{0.013889in}}{\pgfqpoint{-0.007216in}{0.012425in}}{\pgfqpoint{-0.009821in}{0.009821in}}%
\pgfpathcurveto{\pgfqpoint{-0.012425in}{0.007216in}}{\pgfqpoint{-0.013889in}{0.003683in}}{\pgfqpoint{-0.013889in}{0.000000in}}%
\pgfpathcurveto{\pgfqpoint{-0.013889in}{-0.003683in}}{\pgfqpoint{-0.012425in}{-0.007216in}}{\pgfqpoint{-0.009821in}{-0.009821in}}%
\pgfpathcurveto{\pgfqpoint{-0.007216in}{-0.012425in}}{\pgfqpoint{-0.003683in}{-0.013889in}}{\pgfqpoint{0.000000in}{-0.013889in}}%
\pgfpathclose%
\pgfusepath{stroke,fill}%
}%
\begin{pgfscope}%
\pgfsys@transformshift{1.588407in}{4.108047in}%
\pgfsys@useobject{currentmarker}{}%
\end{pgfscope}%
\end{pgfscope}%
\begin{pgfscope}%
\pgfpathrectangle{\pgfqpoint{0.100000in}{2.413063in}}{\pgfqpoint{5.037500in}{3.427208in}}%
\pgfusepath{clip}%
\pgfsetrectcap%
\pgfsetroundjoin%
\pgfsetlinewidth{1.505625pt}%
\definecolor{currentstroke}{rgb}{0.678431,1.000000,0.184314}%
\pgfsetstrokecolor{currentstroke}%
\pgfsetstrokeopacity{0.500000}%
\pgfsetdash{}{0pt}%
\pgfpathmoveto{\pgfqpoint{1.653203in}{3.583998in}}%
\pgfusepath{stroke}%
\end{pgfscope}%
\begin{pgfscope}%
\pgfpathrectangle{\pgfqpoint{0.100000in}{2.413063in}}{\pgfqpoint{5.037500in}{3.427208in}}%
\pgfusepath{clip}%
\pgfsetbuttcap%
\pgfsetroundjoin%
\definecolor{currentfill}{rgb}{0.678431,1.000000,0.184314}%
\pgfsetfillcolor{currentfill}%
\pgfsetfillopacity{0.500000}%
\pgfsetlinewidth{0.250937pt}%
\definecolor{currentstroke}{rgb}{0.000000,0.000000,0.000000}%
\pgfsetstrokecolor{currentstroke}%
\pgfsetstrokeopacity{0.500000}%
\pgfsetdash{}{0pt}%
\pgfsys@defobject{currentmarker}{\pgfqpoint{-0.016667in}{-0.016667in}}{\pgfqpoint{0.016667in}{0.016667in}}{%
\pgfpathmoveto{\pgfqpoint{0.000000in}{-0.016667in}}%
\pgfpathcurveto{\pgfqpoint{0.004420in}{-0.016667in}}{\pgfqpoint{0.008660in}{-0.014911in}}{\pgfqpoint{0.011785in}{-0.011785in}}%
\pgfpathcurveto{\pgfqpoint{0.014911in}{-0.008660in}}{\pgfqpoint{0.016667in}{-0.004420in}}{\pgfqpoint{0.016667in}{0.000000in}}%
\pgfpathcurveto{\pgfqpoint{0.016667in}{0.004420in}}{\pgfqpoint{0.014911in}{0.008660in}}{\pgfqpoint{0.011785in}{0.011785in}}%
\pgfpathcurveto{\pgfqpoint{0.008660in}{0.014911in}}{\pgfqpoint{0.004420in}{0.016667in}}{\pgfqpoint{0.000000in}{0.016667in}}%
\pgfpathcurveto{\pgfqpoint{-0.004420in}{0.016667in}}{\pgfqpoint{-0.008660in}{0.014911in}}{\pgfqpoint{-0.011785in}{0.011785in}}%
\pgfpathcurveto{\pgfqpoint{-0.014911in}{0.008660in}}{\pgfqpoint{-0.016667in}{0.004420in}}{\pgfqpoint{-0.016667in}{0.000000in}}%
\pgfpathcurveto{\pgfqpoint{-0.016667in}{-0.004420in}}{\pgfqpoint{-0.014911in}{-0.008660in}}{\pgfqpoint{-0.011785in}{-0.011785in}}%
\pgfpathcurveto{\pgfqpoint{-0.008660in}{-0.014911in}}{\pgfqpoint{-0.004420in}{-0.016667in}}{\pgfqpoint{0.000000in}{-0.016667in}}%
\pgfpathclose%
\pgfusepath{stroke,fill}%
}%
\begin{pgfscope}%
\pgfsys@transformshift{1.653203in}{3.583998in}%
\pgfsys@useobject{currentmarker}{}%
\end{pgfscope}%
\end{pgfscope}%
\begin{pgfscope}%
\pgfpathrectangle{\pgfqpoint{0.100000in}{2.413063in}}{\pgfqpoint{5.037500in}{3.427208in}}%
\pgfusepath{clip}%
\pgfsetrectcap%
\pgfsetroundjoin%
\pgfsetlinewidth{1.505625pt}%
\definecolor{currentstroke}{rgb}{0.000000,0.000000,1.000000}%
\pgfsetstrokecolor{currentstroke}%
\pgfsetstrokeopacity{0.500000}%
\pgfsetdash{}{0pt}%
\pgfpathmoveto{\pgfqpoint{1.781876in}{3.960833in}}%
\pgfusepath{stroke}%
\end{pgfscope}%
\begin{pgfscope}%
\pgfpathrectangle{\pgfqpoint{0.100000in}{2.413063in}}{\pgfqpoint{5.037500in}{3.427208in}}%
\pgfusepath{clip}%
\pgfsetbuttcap%
\pgfsetroundjoin%
\definecolor{currentfill}{rgb}{0.000000,0.000000,1.000000}%
\pgfsetfillcolor{currentfill}%
\pgfsetfillopacity{0.500000}%
\pgfsetlinewidth{0.250937pt}%
\definecolor{currentstroke}{rgb}{0.000000,0.000000,0.000000}%
\pgfsetstrokecolor{currentstroke}%
\pgfsetstrokeopacity{0.500000}%
\pgfsetdash{}{0pt}%
\pgfsys@defobject{currentmarker}{\pgfqpoint{-0.022222in}{-0.022222in}}{\pgfqpoint{0.022222in}{0.022222in}}{%
\pgfpathmoveto{\pgfqpoint{0.000000in}{-0.022222in}}%
\pgfpathcurveto{\pgfqpoint{0.005893in}{-0.022222in}}{\pgfqpoint{0.011546in}{-0.019881in}}{\pgfqpoint{0.015713in}{-0.015713in}}%
\pgfpathcurveto{\pgfqpoint{0.019881in}{-0.011546in}}{\pgfqpoint{0.022222in}{-0.005893in}}{\pgfqpoint{0.022222in}{0.000000in}}%
\pgfpathcurveto{\pgfqpoint{0.022222in}{0.005893in}}{\pgfqpoint{0.019881in}{0.011546in}}{\pgfqpoint{0.015713in}{0.015713in}}%
\pgfpathcurveto{\pgfqpoint{0.011546in}{0.019881in}}{\pgfqpoint{0.005893in}{0.022222in}}{\pgfqpoint{0.000000in}{0.022222in}}%
\pgfpathcurveto{\pgfqpoint{-0.005893in}{0.022222in}}{\pgfqpoint{-0.011546in}{0.019881in}}{\pgfqpoint{-0.015713in}{0.015713in}}%
\pgfpathcurveto{\pgfqpoint{-0.019881in}{0.011546in}}{\pgfqpoint{-0.022222in}{0.005893in}}{\pgfqpoint{-0.022222in}{0.000000in}}%
\pgfpathcurveto{\pgfqpoint{-0.022222in}{-0.005893in}}{\pgfqpoint{-0.019881in}{-0.011546in}}{\pgfqpoint{-0.015713in}{-0.015713in}}%
\pgfpathcurveto{\pgfqpoint{-0.011546in}{-0.019881in}}{\pgfqpoint{-0.005893in}{-0.022222in}}{\pgfqpoint{0.000000in}{-0.022222in}}%
\pgfpathclose%
\pgfusepath{stroke,fill}%
}%
\begin{pgfscope}%
\pgfsys@transformshift{1.781876in}{3.960833in}%
\pgfsys@useobject{currentmarker}{}%
\end{pgfscope}%
\end{pgfscope}%
\begin{pgfscope}%
\pgfpathrectangle{\pgfqpoint{0.100000in}{2.413063in}}{\pgfqpoint{5.037500in}{3.427208in}}%
\pgfusepath{clip}%
\pgfsetrectcap%
\pgfsetroundjoin%
\pgfsetlinewidth{1.505625pt}%
\definecolor{currentstroke}{rgb}{0.678431,1.000000,0.184314}%
\pgfsetstrokecolor{currentstroke}%
\pgfsetstrokeopacity{0.500000}%
\pgfsetdash{}{0pt}%
\pgfpathmoveto{\pgfqpoint{4.590541in}{4.911408in}}%
\pgfusepath{stroke}%
\end{pgfscope}%
\begin{pgfscope}%
\pgfpathrectangle{\pgfqpoint{0.100000in}{2.413063in}}{\pgfqpoint{5.037500in}{3.427208in}}%
\pgfusepath{clip}%
\pgfsetbuttcap%
\pgfsetroundjoin%
\definecolor{currentfill}{rgb}{0.678431,1.000000,0.184314}%
\pgfsetfillcolor{currentfill}%
\pgfsetfillopacity{0.500000}%
\pgfsetlinewidth{0.250937pt}%
\definecolor{currentstroke}{rgb}{0.000000,0.000000,0.000000}%
\pgfsetstrokecolor{currentstroke}%
\pgfsetstrokeopacity{0.500000}%
\pgfsetdash{}{0pt}%
\pgfsys@defobject{currentmarker}{\pgfqpoint{-0.027778in}{-0.027778in}}{\pgfqpoint{0.027778in}{0.027778in}}{%
\pgfpathmoveto{\pgfqpoint{0.000000in}{-0.027778in}}%
\pgfpathcurveto{\pgfqpoint{0.007367in}{-0.027778in}}{\pgfqpoint{0.014433in}{-0.024851in}}{\pgfqpoint{0.019642in}{-0.019642in}}%
\pgfpathcurveto{\pgfqpoint{0.024851in}{-0.014433in}}{\pgfqpoint{0.027778in}{-0.007367in}}{\pgfqpoint{0.027778in}{0.000000in}}%
\pgfpathcurveto{\pgfqpoint{0.027778in}{0.007367in}}{\pgfqpoint{0.024851in}{0.014433in}}{\pgfqpoint{0.019642in}{0.019642in}}%
\pgfpathcurveto{\pgfqpoint{0.014433in}{0.024851in}}{\pgfqpoint{0.007367in}{0.027778in}}{\pgfqpoint{0.000000in}{0.027778in}}%
\pgfpathcurveto{\pgfqpoint{-0.007367in}{0.027778in}}{\pgfqpoint{-0.014433in}{0.024851in}}{\pgfqpoint{-0.019642in}{0.019642in}}%
\pgfpathcurveto{\pgfqpoint{-0.024851in}{0.014433in}}{\pgfqpoint{-0.027778in}{0.007367in}}{\pgfqpoint{-0.027778in}{0.000000in}}%
\pgfpathcurveto{\pgfqpoint{-0.027778in}{-0.007367in}}{\pgfqpoint{-0.024851in}{-0.014433in}}{\pgfqpoint{-0.019642in}{-0.019642in}}%
\pgfpathcurveto{\pgfqpoint{-0.014433in}{-0.024851in}}{\pgfqpoint{-0.007367in}{-0.027778in}}{\pgfqpoint{0.000000in}{-0.027778in}}%
\pgfpathclose%
\pgfusepath{stroke,fill}%
}%
\begin{pgfscope}%
\pgfsys@transformshift{4.590541in}{4.911408in}%
\pgfsys@useobject{currentmarker}{}%
\end{pgfscope}%
\end{pgfscope}%
\begin{pgfscope}%
\pgfpathrectangle{\pgfqpoint{0.100000in}{2.413063in}}{\pgfqpoint{5.037500in}{3.427208in}}%
\pgfusepath{clip}%
\pgfsetrectcap%
\pgfsetroundjoin%
\pgfsetlinewidth{1.505625pt}%
\definecolor{currentstroke}{rgb}{0.678431,1.000000,0.184314}%
\pgfsetstrokecolor{currentstroke}%
\pgfsetstrokeopacity{0.500000}%
\pgfsetdash{}{0pt}%
\pgfpathmoveto{\pgfqpoint{4.425126in}{4.808565in}}%
\pgfusepath{stroke}%
\end{pgfscope}%
\begin{pgfscope}%
\pgfpathrectangle{\pgfqpoint{0.100000in}{2.413063in}}{\pgfqpoint{5.037500in}{3.427208in}}%
\pgfusepath{clip}%
\pgfsetbuttcap%
\pgfsetroundjoin%
\definecolor{currentfill}{rgb}{0.678431,1.000000,0.184314}%
\pgfsetfillcolor{currentfill}%
\pgfsetfillopacity{0.500000}%
\pgfsetlinewidth{0.250937pt}%
\definecolor{currentstroke}{rgb}{0.000000,0.000000,0.000000}%
\pgfsetstrokecolor{currentstroke}%
\pgfsetstrokeopacity{0.500000}%
\pgfsetdash{}{0pt}%
\pgfsys@defobject{currentmarker}{\pgfqpoint{-0.041667in}{-0.041667in}}{\pgfqpoint{0.041667in}{0.041667in}}{%
\pgfpathmoveto{\pgfqpoint{0.000000in}{-0.041667in}}%
\pgfpathcurveto{\pgfqpoint{0.011050in}{-0.041667in}}{\pgfqpoint{0.021649in}{-0.037276in}}{\pgfqpoint{0.029463in}{-0.029463in}}%
\pgfpathcurveto{\pgfqpoint{0.037276in}{-0.021649in}}{\pgfqpoint{0.041667in}{-0.011050in}}{\pgfqpoint{0.041667in}{0.000000in}}%
\pgfpathcurveto{\pgfqpoint{0.041667in}{0.011050in}}{\pgfqpoint{0.037276in}{0.021649in}}{\pgfqpoint{0.029463in}{0.029463in}}%
\pgfpathcurveto{\pgfqpoint{0.021649in}{0.037276in}}{\pgfqpoint{0.011050in}{0.041667in}}{\pgfqpoint{0.000000in}{0.041667in}}%
\pgfpathcurveto{\pgfqpoint{-0.011050in}{0.041667in}}{\pgfqpoint{-0.021649in}{0.037276in}}{\pgfqpoint{-0.029463in}{0.029463in}}%
\pgfpathcurveto{\pgfqpoint{-0.037276in}{0.021649in}}{\pgfqpoint{-0.041667in}{0.011050in}}{\pgfqpoint{-0.041667in}{0.000000in}}%
\pgfpathcurveto{\pgfqpoint{-0.041667in}{-0.011050in}}{\pgfqpoint{-0.037276in}{-0.021649in}}{\pgfqpoint{-0.029463in}{-0.029463in}}%
\pgfpathcurveto{\pgfqpoint{-0.021649in}{-0.037276in}}{\pgfqpoint{-0.011050in}{-0.041667in}}{\pgfqpoint{0.000000in}{-0.041667in}}%
\pgfpathclose%
\pgfusepath{stroke,fill}%
}%
\begin{pgfscope}%
\pgfsys@transformshift{4.425126in}{4.808565in}%
\pgfsys@useobject{currentmarker}{}%
\end{pgfscope}%
\end{pgfscope}%
\begin{pgfscope}%
\pgfpathrectangle{\pgfqpoint{0.100000in}{2.413063in}}{\pgfqpoint{5.037500in}{3.427208in}}%
\pgfusepath{clip}%
\pgfsetrectcap%
\pgfsetroundjoin%
\pgfsetlinewidth{1.505625pt}%
\definecolor{currentstroke}{rgb}{0.678431,1.000000,0.184314}%
\pgfsetstrokecolor{currentstroke}%
\pgfsetstrokeopacity{0.500000}%
\pgfsetdash{}{0pt}%
\pgfpathmoveto{\pgfqpoint{4.160198in}{4.849420in}}%
\pgfusepath{stroke}%
\end{pgfscope}%
\begin{pgfscope}%
\pgfpathrectangle{\pgfqpoint{0.100000in}{2.413063in}}{\pgfqpoint{5.037500in}{3.427208in}}%
\pgfusepath{clip}%
\pgfsetbuttcap%
\pgfsetroundjoin%
\definecolor{currentfill}{rgb}{0.678431,1.000000,0.184314}%
\pgfsetfillcolor{currentfill}%
\pgfsetfillopacity{0.500000}%
\pgfsetlinewidth{0.250937pt}%
\definecolor{currentstroke}{rgb}{0.000000,0.000000,0.000000}%
\pgfsetstrokecolor{currentstroke}%
\pgfsetstrokeopacity{0.500000}%
\pgfsetdash{}{0pt}%
\pgfsys@defobject{currentmarker}{\pgfqpoint{-0.030556in}{-0.030556in}}{\pgfqpoint{0.030556in}{0.030556in}}{%
\pgfpathmoveto{\pgfqpoint{0.000000in}{-0.030556in}}%
\pgfpathcurveto{\pgfqpoint{0.008103in}{-0.030556in}}{\pgfqpoint{0.015876in}{-0.027336in}}{\pgfqpoint{0.021606in}{-0.021606in}}%
\pgfpathcurveto{\pgfqpoint{0.027336in}{-0.015876in}}{\pgfqpoint{0.030556in}{-0.008103in}}{\pgfqpoint{0.030556in}{0.000000in}}%
\pgfpathcurveto{\pgfqpoint{0.030556in}{0.008103in}}{\pgfqpoint{0.027336in}{0.015876in}}{\pgfqpoint{0.021606in}{0.021606in}}%
\pgfpathcurveto{\pgfqpoint{0.015876in}{0.027336in}}{\pgfqpoint{0.008103in}{0.030556in}}{\pgfqpoint{0.000000in}{0.030556in}}%
\pgfpathcurveto{\pgfqpoint{-0.008103in}{0.030556in}}{\pgfqpoint{-0.015876in}{0.027336in}}{\pgfqpoint{-0.021606in}{0.021606in}}%
\pgfpathcurveto{\pgfqpoint{-0.027336in}{0.015876in}}{\pgfqpoint{-0.030556in}{0.008103in}}{\pgfqpoint{-0.030556in}{0.000000in}}%
\pgfpathcurveto{\pgfqpoint{-0.030556in}{-0.008103in}}{\pgfqpoint{-0.027336in}{-0.015876in}}{\pgfqpoint{-0.021606in}{-0.021606in}}%
\pgfpathcurveto{\pgfqpoint{-0.015876in}{-0.027336in}}{\pgfqpoint{-0.008103in}{-0.030556in}}{\pgfqpoint{0.000000in}{-0.030556in}}%
\pgfpathclose%
\pgfusepath{stroke,fill}%
}%
\begin{pgfscope}%
\pgfsys@transformshift{4.160198in}{4.849420in}%
\pgfsys@useobject{currentmarker}{}%
\end{pgfscope}%
\end{pgfscope}%
\begin{pgfscope}%
\pgfpathrectangle{\pgfqpoint{0.100000in}{2.413063in}}{\pgfqpoint{5.037500in}{3.427208in}}%
\pgfusepath{clip}%
\pgfsetrectcap%
\pgfsetroundjoin%
\pgfsetlinewidth{1.505625pt}%
\definecolor{currentstroke}{rgb}{0.678431,1.000000,0.184314}%
\pgfsetstrokecolor{currentstroke}%
\pgfsetstrokeopacity{0.500000}%
\pgfsetdash{}{0pt}%
\pgfpathmoveto{\pgfqpoint{4.350454in}{4.792186in}}%
\pgfusepath{stroke}%
\end{pgfscope}%
\begin{pgfscope}%
\pgfpathrectangle{\pgfqpoint{0.100000in}{2.413063in}}{\pgfqpoint{5.037500in}{3.427208in}}%
\pgfusepath{clip}%
\pgfsetbuttcap%
\pgfsetroundjoin%
\definecolor{currentfill}{rgb}{0.678431,1.000000,0.184314}%
\pgfsetfillcolor{currentfill}%
\pgfsetfillopacity{0.500000}%
\pgfsetlinewidth{0.250937pt}%
\definecolor{currentstroke}{rgb}{0.000000,0.000000,0.000000}%
\pgfsetstrokecolor{currentstroke}%
\pgfsetstrokeopacity{0.500000}%
\pgfsetdash{}{0pt}%
\pgfsys@defobject{currentmarker}{\pgfqpoint{-0.027778in}{-0.027778in}}{\pgfqpoint{0.027778in}{0.027778in}}{%
\pgfpathmoveto{\pgfqpoint{0.000000in}{-0.027778in}}%
\pgfpathcurveto{\pgfqpoint{0.007367in}{-0.027778in}}{\pgfqpoint{0.014433in}{-0.024851in}}{\pgfqpoint{0.019642in}{-0.019642in}}%
\pgfpathcurveto{\pgfqpoint{0.024851in}{-0.014433in}}{\pgfqpoint{0.027778in}{-0.007367in}}{\pgfqpoint{0.027778in}{0.000000in}}%
\pgfpathcurveto{\pgfqpoint{0.027778in}{0.007367in}}{\pgfqpoint{0.024851in}{0.014433in}}{\pgfqpoint{0.019642in}{0.019642in}}%
\pgfpathcurveto{\pgfqpoint{0.014433in}{0.024851in}}{\pgfqpoint{0.007367in}{0.027778in}}{\pgfqpoint{0.000000in}{0.027778in}}%
\pgfpathcurveto{\pgfqpoint{-0.007367in}{0.027778in}}{\pgfqpoint{-0.014433in}{0.024851in}}{\pgfqpoint{-0.019642in}{0.019642in}}%
\pgfpathcurveto{\pgfqpoint{-0.024851in}{0.014433in}}{\pgfqpoint{-0.027778in}{0.007367in}}{\pgfqpoint{-0.027778in}{0.000000in}}%
\pgfpathcurveto{\pgfqpoint{-0.027778in}{-0.007367in}}{\pgfqpoint{-0.024851in}{-0.014433in}}{\pgfqpoint{-0.019642in}{-0.019642in}}%
\pgfpathcurveto{\pgfqpoint{-0.014433in}{-0.024851in}}{\pgfqpoint{-0.007367in}{-0.027778in}}{\pgfqpoint{0.000000in}{-0.027778in}}%
\pgfpathclose%
\pgfusepath{stroke,fill}%
}%
\begin{pgfscope}%
\pgfsys@transformshift{4.350454in}{4.792186in}%
\pgfsys@useobject{currentmarker}{}%
\end{pgfscope}%
\end{pgfscope}%
\begin{pgfscope}%
\pgfpathrectangle{\pgfqpoint{0.100000in}{2.413063in}}{\pgfqpoint{5.037500in}{3.427208in}}%
\pgfusepath{clip}%
\pgfsetrectcap%
\pgfsetroundjoin%
\pgfsetlinewidth{1.505625pt}%
\definecolor{currentstroke}{rgb}{0.678431,1.000000,0.184314}%
\pgfsetstrokecolor{currentstroke}%
\pgfsetstrokeopacity{0.500000}%
\pgfsetdash{}{0pt}%
\pgfpathmoveto{\pgfqpoint{4.581959in}{4.987532in}}%
\pgfusepath{stroke}%
\end{pgfscope}%
\begin{pgfscope}%
\pgfpathrectangle{\pgfqpoint{0.100000in}{2.413063in}}{\pgfqpoint{5.037500in}{3.427208in}}%
\pgfusepath{clip}%
\pgfsetbuttcap%
\pgfsetroundjoin%
\definecolor{currentfill}{rgb}{0.678431,1.000000,0.184314}%
\pgfsetfillcolor{currentfill}%
\pgfsetfillopacity{0.500000}%
\pgfsetlinewidth{0.250937pt}%
\definecolor{currentstroke}{rgb}{0.000000,0.000000,0.000000}%
\pgfsetstrokecolor{currentstroke}%
\pgfsetstrokeopacity{0.500000}%
\pgfsetdash{}{0pt}%
\pgfsys@defobject{currentmarker}{\pgfqpoint{-0.047222in}{-0.047222in}}{\pgfqpoint{0.047222in}{0.047222in}}{%
\pgfpathmoveto{\pgfqpoint{0.000000in}{-0.047222in}}%
\pgfpathcurveto{\pgfqpoint{0.012523in}{-0.047222in}}{\pgfqpoint{0.024536in}{-0.042247in}}{\pgfqpoint{0.033391in}{-0.033391in}}%
\pgfpathcurveto{\pgfqpoint{0.042247in}{-0.024536in}}{\pgfqpoint{0.047222in}{-0.012523in}}{\pgfqpoint{0.047222in}{0.000000in}}%
\pgfpathcurveto{\pgfqpoint{0.047222in}{0.012523in}}{\pgfqpoint{0.042247in}{0.024536in}}{\pgfqpoint{0.033391in}{0.033391in}}%
\pgfpathcurveto{\pgfqpoint{0.024536in}{0.042247in}}{\pgfqpoint{0.012523in}{0.047222in}}{\pgfqpoint{0.000000in}{0.047222in}}%
\pgfpathcurveto{\pgfqpoint{-0.012523in}{0.047222in}}{\pgfqpoint{-0.024536in}{0.042247in}}{\pgfqpoint{-0.033391in}{0.033391in}}%
\pgfpathcurveto{\pgfqpoint{-0.042247in}{0.024536in}}{\pgfqpoint{-0.047222in}{0.012523in}}{\pgfqpoint{-0.047222in}{0.000000in}}%
\pgfpathcurveto{\pgfqpoint{-0.047222in}{-0.012523in}}{\pgfqpoint{-0.042247in}{-0.024536in}}{\pgfqpoint{-0.033391in}{-0.033391in}}%
\pgfpathcurveto{\pgfqpoint{-0.024536in}{-0.042247in}}{\pgfqpoint{-0.012523in}{-0.047222in}}{\pgfqpoint{0.000000in}{-0.047222in}}%
\pgfpathclose%
\pgfusepath{stroke,fill}%
}%
\begin{pgfscope}%
\pgfsys@transformshift{4.581959in}{4.987532in}%
\pgfsys@useobject{currentmarker}{}%
\end{pgfscope}%
\end{pgfscope}%
\begin{pgfscope}%
\pgfpathrectangle{\pgfqpoint{0.100000in}{2.413063in}}{\pgfqpoint{5.037500in}{3.427208in}}%
\pgfusepath{clip}%
\pgfsetrectcap%
\pgfsetroundjoin%
\pgfsetlinewidth{1.505625pt}%
\definecolor{currentstroke}{rgb}{0.678431,1.000000,0.184314}%
\pgfsetstrokecolor{currentstroke}%
\pgfsetstrokeopacity{0.500000}%
\pgfsetdash{}{0pt}%
\pgfpathmoveto{\pgfqpoint{4.368363in}{4.837031in}}%
\pgfusepath{stroke}%
\end{pgfscope}%
\begin{pgfscope}%
\pgfpathrectangle{\pgfqpoint{0.100000in}{2.413063in}}{\pgfqpoint{5.037500in}{3.427208in}}%
\pgfusepath{clip}%
\pgfsetbuttcap%
\pgfsetroundjoin%
\definecolor{currentfill}{rgb}{0.678431,1.000000,0.184314}%
\pgfsetfillcolor{currentfill}%
\pgfsetfillopacity{0.500000}%
\pgfsetlinewidth{0.250937pt}%
\definecolor{currentstroke}{rgb}{0.000000,0.000000,0.000000}%
\pgfsetstrokecolor{currentstroke}%
\pgfsetstrokeopacity{0.500000}%
\pgfsetdash{}{0pt}%
\pgfsys@defobject{currentmarker}{\pgfqpoint{-0.025000in}{-0.025000in}}{\pgfqpoint{0.025000in}{0.025000in}}{%
\pgfpathmoveto{\pgfqpoint{0.000000in}{-0.025000in}}%
\pgfpathcurveto{\pgfqpoint{0.006630in}{-0.025000in}}{\pgfqpoint{0.012989in}{-0.022366in}}{\pgfqpoint{0.017678in}{-0.017678in}}%
\pgfpathcurveto{\pgfqpoint{0.022366in}{-0.012989in}}{\pgfqpoint{0.025000in}{-0.006630in}}{\pgfqpoint{0.025000in}{0.000000in}}%
\pgfpathcurveto{\pgfqpoint{0.025000in}{0.006630in}}{\pgfqpoint{0.022366in}{0.012989in}}{\pgfqpoint{0.017678in}{0.017678in}}%
\pgfpathcurveto{\pgfqpoint{0.012989in}{0.022366in}}{\pgfqpoint{0.006630in}{0.025000in}}{\pgfqpoint{0.000000in}{0.025000in}}%
\pgfpathcurveto{\pgfqpoint{-0.006630in}{0.025000in}}{\pgfqpoint{-0.012989in}{0.022366in}}{\pgfqpoint{-0.017678in}{0.017678in}}%
\pgfpathcurveto{\pgfqpoint{-0.022366in}{0.012989in}}{\pgfqpoint{-0.025000in}{0.006630in}}{\pgfqpoint{-0.025000in}{0.000000in}}%
\pgfpathcurveto{\pgfqpoint{-0.025000in}{-0.006630in}}{\pgfqpoint{-0.022366in}{-0.012989in}}{\pgfqpoint{-0.017678in}{-0.017678in}}%
\pgfpathcurveto{\pgfqpoint{-0.012989in}{-0.022366in}}{\pgfqpoint{-0.006630in}{-0.025000in}}{\pgfqpoint{0.000000in}{-0.025000in}}%
\pgfpathclose%
\pgfusepath{stroke,fill}%
}%
\begin{pgfscope}%
\pgfsys@transformshift{4.368363in}{4.837031in}%
\pgfsys@useobject{currentmarker}{}%
\end{pgfscope}%
\end{pgfscope}%
\begin{pgfscope}%
\pgfpathrectangle{\pgfqpoint{0.100000in}{2.413063in}}{\pgfqpoint{5.037500in}{3.427208in}}%
\pgfusepath{clip}%
\pgfsetrectcap%
\pgfsetroundjoin%
\pgfsetlinewidth{1.505625pt}%
\definecolor{currentstroke}{rgb}{0.678431,1.000000,0.184314}%
\pgfsetstrokecolor{currentstroke}%
\pgfsetstrokeopacity{0.500000}%
\pgfsetdash{}{0pt}%
\pgfpathmoveto{\pgfqpoint{4.589191in}{4.825473in}}%
\pgfusepath{stroke}%
\end{pgfscope}%
\begin{pgfscope}%
\pgfpathrectangle{\pgfqpoint{0.100000in}{2.413063in}}{\pgfqpoint{5.037500in}{3.427208in}}%
\pgfusepath{clip}%
\pgfsetbuttcap%
\pgfsetroundjoin%
\definecolor{currentfill}{rgb}{0.678431,1.000000,0.184314}%
\pgfsetfillcolor{currentfill}%
\pgfsetfillopacity{0.500000}%
\pgfsetlinewidth{0.250937pt}%
\definecolor{currentstroke}{rgb}{0.000000,0.000000,0.000000}%
\pgfsetstrokecolor{currentstroke}%
\pgfsetstrokeopacity{0.500000}%
\pgfsetdash{}{0pt}%
\pgfsys@defobject{currentmarker}{\pgfqpoint{-0.022222in}{-0.022222in}}{\pgfqpoint{0.022222in}{0.022222in}}{%
\pgfpathmoveto{\pgfqpoint{0.000000in}{-0.022222in}}%
\pgfpathcurveto{\pgfqpoint{0.005893in}{-0.022222in}}{\pgfqpoint{0.011546in}{-0.019881in}}{\pgfqpoint{0.015713in}{-0.015713in}}%
\pgfpathcurveto{\pgfqpoint{0.019881in}{-0.011546in}}{\pgfqpoint{0.022222in}{-0.005893in}}{\pgfqpoint{0.022222in}{0.000000in}}%
\pgfpathcurveto{\pgfqpoint{0.022222in}{0.005893in}}{\pgfqpoint{0.019881in}{0.011546in}}{\pgfqpoint{0.015713in}{0.015713in}}%
\pgfpathcurveto{\pgfqpoint{0.011546in}{0.019881in}}{\pgfqpoint{0.005893in}{0.022222in}}{\pgfqpoint{0.000000in}{0.022222in}}%
\pgfpathcurveto{\pgfqpoint{-0.005893in}{0.022222in}}{\pgfqpoint{-0.011546in}{0.019881in}}{\pgfqpoint{-0.015713in}{0.015713in}}%
\pgfpathcurveto{\pgfqpoint{-0.019881in}{0.011546in}}{\pgfqpoint{-0.022222in}{0.005893in}}{\pgfqpoint{-0.022222in}{0.000000in}}%
\pgfpathcurveto{\pgfqpoint{-0.022222in}{-0.005893in}}{\pgfqpoint{-0.019881in}{-0.011546in}}{\pgfqpoint{-0.015713in}{-0.015713in}}%
\pgfpathcurveto{\pgfqpoint{-0.011546in}{-0.019881in}}{\pgfqpoint{-0.005893in}{-0.022222in}}{\pgfqpoint{0.000000in}{-0.022222in}}%
\pgfpathclose%
\pgfusepath{stroke,fill}%
}%
\begin{pgfscope}%
\pgfsys@transformshift{4.589191in}{4.825473in}%
\pgfsys@useobject{currentmarker}{}%
\end{pgfscope}%
\end{pgfscope}%
\begin{pgfscope}%
\pgfpathrectangle{\pgfqpoint{0.100000in}{2.413063in}}{\pgfqpoint{5.037500in}{3.427208in}}%
\pgfusepath{clip}%
\pgfsetrectcap%
\pgfsetroundjoin%
\pgfsetlinewidth{1.505625pt}%
\definecolor{currentstroke}{rgb}{0.000000,0.000000,1.000000}%
\pgfsetstrokecolor{currentstroke}%
\pgfsetstrokeopacity{0.500000}%
\pgfsetdash{}{0pt}%
\pgfpathmoveto{\pgfqpoint{4.622153in}{4.691381in}}%
\pgfusepath{stroke}%
\end{pgfscope}%
\begin{pgfscope}%
\pgfpathrectangle{\pgfqpoint{0.100000in}{2.413063in}}{\pgfqpoint{5.037500in}{3.427208in}}%
\pgfusepath{clip}%
\pgfsetbuttcap%
\pgfsetroundjoin%
\definecolor{currentfill}{rgb}{0.000000,0.000000,1.000000}%
\pgfsetfillcolor{currentfill}%
\pgfsetfillopacity{0.500000}%
\pgfsetlinewidth{0.250937pt}%
\definecolor{currentstroke}{rgb}{0.000000,0.000000,0.000000}%
\pgfsetstrokecolor{currentstroke}%
\pgfsetstrokeopacity{0.500000}%
\pgfsetdash{}{0pt}%
\pgfsys@defobject{currentmarker}{\pgfqpoint{-0.066667in}{-0.066667in}}{\pgfqpoint{0.066667in}{0.066667in}}{%
\pgfpathmoveto{\pgfqpoint{0.000000in}{-0.066667in}}%
\pgfpathcurveto{\pgfqpoint{0.017680in}{-0.066667in}}{\pgfqpoint{0.034639in}{-0.059642in}}{\pgfqpoint{0.047140in}{-0.047140in}}%
\pgfpathcurveto{\pgfqpoint{0.059642in}{-0.034639in}}{\pgfqpoint{0.066667in}{-0.017680in}}{\pgfqpoint{0.066667in}{0.000000in}}%
\pgfpathcurveto{\pgfqpoint{0.066667in}{0.017680in}}{\pgfqpoint{0.059642in}{0.034639in}}{\pgfqpoint{0.047140in}{0.047140in}}%
\pgfpathcurveto{\pgfqpoint{0.034639in}{0.059642in}}{\pgfqpoint{0.017680in}{0.066667in}}{\pgfqpoint{0.000000in}{0.066667in}}%
\pgfpathcurveto{\pgfqpoint{-0.017680in}{0.066667in}}{\pgfqpoint{-0.034639in}{0.059642in}}{\pgfqpoint{-0.047140in}{0.047140in}}%
\pgfpathcurveto{\pgfqpoint{-0.059642in}{0.034639in}}{\pgfqpoint{-0.066667in}{0.017680in}}{\pgfqpoint{-0.066667in}{0.000000in}}%
\pgfpathcurveto{\pgfqpoint{-0.066667in}{-0.017680in}}{\pgfqpoint{-0.059642in}{-0.034639in}}{\pgfqpoint{-0.047140in}{-0.047140in}}%
\pgfpathcurveto{\pgfqpoint{-0.034639in}{-0.059642in}}{\pgfqpoint{-0.017680in}{-0.066667in}}{\pgfqpoint{0.000000in}{-0.066667in}}%
\pgfpathclose%
\pgfusepath{stroke,fill}%
}%
\begin{pgfscope}%
\pgfsys@transformshift{4.622153in}{4.691381in}%
\pgfsys@useobject{currentmarker}{}%
\end{pgfscope}%
\end{pgfscope}%
\begin{pgfscope}%
\pgfpathrectangle{\pgfqpoint{0.100000in}{2.413063in}}{\pgfqpoint{5.037500in}{3.427208in}}%
\pgfusepath{clip}%
\pgfsetrectcap%
\pgfsetroundjoin%
\pgfsetlinewidth{1.505625pt}%
\definecolor{currentstroke}{rgb}{0.678431,1.000000,0.184314}%
\pgfsetstrokecolor{currentstroke}%
\pgfsetstrokeopacity{0.500000}%
\pgfsetdash{}{0pt}%
\pgfpathmoveto{\pgfqpoint{4.259261in}{4.899660in}}%
\pgfusepath{stroke}%
\end{pgfscope}%
\begin{pgfscope}%
\pgfpathrectangle{\pgfqpoint{0.100000in}{2.413063in}}{\pgfqpoint{5.037500in}{3.427208in}}%
\pgfusepath{clip}%
\pgfsetbuttcap%
\pgfsetroundjoin%
\definecolor{currentfill}{rgb}{0.678431,1.000000,0.184314}%
\pgfsetfillcolor{currentfill}%
\pgfsetfillopacity{0.500000}%
\pgfsetlinewidth{0.250937pt}%
\definecolor{currentstroke}{rgb}{0.000000,0.000000,0.000000}%
\pgfsetstrokecolor{currentstroke}%
\pgfsetstrokeopacity{0.500000}%
\pgfsetdash{}{0pt}%
\pgfsys@defobject{currentmarker}{\pgfqpoint{-0.033333in}{-0.033333in}}{\pgfqpoint{0.033333in}{0.033333in}}{%
\pgfpathmoveto{\pgfqpoint{0.000000in}{-0.033333in}}%
\pgfpathcurveto{\pgfqpoint{0.008840in}{-0.033333in}}{\pgfqpoint{0.017319in}{-0.029821in}}{\pgfqpoint{0.023570in}{-0.023570in}}%
\pgfpathcurveto{\pgfqpoint{0.029821in}{-0.017319in}}{\pgfqpoint{0.033333in}{-0.008840in}}{\pgfqpoint{0.033333in}{0.000000in}}%
\pgfpathcurveto{\pgfqpoint{0.033333in}{0.008840in}}{\pgfqpoint{0.029821in}{0.017319in}}{\pgfqpoint{0.023570in}{0.023570in}}%
\pgfpathcurveto{\pgfqpoint{0.017319in}{0.029821in}}{\pgfqpoint{0.008840in}{0.033333in}}{\pgfqpoint{0.000000in}{0.033333in}}%
\pgfpathcurveto{\pgfqpoint{-0.008840in}{0.033333in}}{\pgfqpoint{-0.017319in}{0.029821in}}{\pgfqpoint{-0.023570in}{0.023570in}}%
\pgfpathcurveto{\pgfqpoint{-0.029821in}{0.017319in}}{\pgfqpoint{-0.033333in}{0.008840in}}{\pgfqpoint{-0.033333in}{0.000000in}}%
\pgfpathcurveto{\pgfqpoint{-0.033333in}{-0.008840in}}{\pgfqpoint{-0.029821in}{-0.017319in}}{\pgfqpoint{-0.023570in}{-0.023570in}}%
\pgfpathcurveto{\pgfqpoint{-0.017319in}{-0.029821in}}{\pgfqpoint{-0.008840in}{-0.033333in}}{\pgfqpoint{0.000000in}{-0.033333in}}%
\pgfpathclose%
\pgfusepath{stroke,fill}%
}%
\begin{pgfscope}%
\pgfsys@transformshift{4.259261in}{4.899660in}%
\pgfsys@useobject{currentmarker}{}%
\end{pgfscope}%
\end{pgfscope}%
\begin{pgfscope}%
\pgfpathrectangle{\pgfqpoint{0.100000in}{2.413063in}}{\pgfqpoint{5.037500in}{3.427208in}}%
\pgfusepath{clip}%
\pgfsetrectcap%
\pgfsetroundjoin%
\pgfsetlinewidth{1.505625pt}%
\definecolor{currentstroke}{rgb}{0.678431,1.000000,0.184314}%
\pgfsetstrokecolor{currentstroke}%
\pgfsetstrokeopacity{0.500000}%
\pgfsetdash{}{0pt}%
\pgfpathmoveto{\pgfqpoint{4.383016in}{4.911810in}}%
\pgfusepath{stroke}%
\end{pgfscope}%
\begin{pgfscope}%
\pgfpathrectangle{\pgfqpoint{0.100000in}{2.413063in}}{\pgfqpoint{5.037500in}{3.427208in}}%
\pgfusepath{clip}%
\pgfsetbuttcap%
\pgfsetroundjoin%
\definecolor{currentfill}{rgb}{0.678431,1.000000,0.184314}%
\pgfsetfillcolor{currentfill}%
\pgfsetfillopacity{0.500000}%
\pgfsetlinewidth{0.250937pt}%
\definecolor{currentstroke}{rgb}{0.000000,0.000000,0.000000}%
\pgfsetstrokecolor{currentstroke}%
\pgfsetstrokeopacity{0.500000}%
\pgfsetdash{}{0pt}%
\pgfsys@defobject{currentmarker}{\pgfqpoint{-0.033333in}{-0.033333in}}{\pgfqpoint{0.033333in}{0.033333in}}{%
\pgfpathmoveto{\pgfqpoint{0.000000in}{-0.033333in}}%
\pgfpathcurveto{\pgfqpoint{0.008840in}{-0.033333in}}{\pgfqpoint{0.017319in}{-0.029821in}}{\pgfqpoint{0.023570in}{-0.023570in}}%
\pgfpathcurveto{\pgfqpoint{0.029821in}{-0.017319in}}{\pgfqpoint{0.033333in}{-0.008840in}}{\pgfqpoint{0.033333in}{0.000000in}}%
\pgfpathcurveto{\pgfqpoint{0.033333in}{0.008840in}}{\pgfqpoint{0.029821in}{0.017319in}}{\pgfqpoint{0.023570in}{0.023570in}}%
\pgfpathcurveto{\pgfqpoint{0.017319in}{0.029821in}}{\pgfqpoint{0.008840in}{0.033333in}}{\pgfqpoint{0.000000in}{0.033333in}}%
\pgfpathcurveto{\pgfqpoint{-0.008840in}{0.033333in}}{\pgfqpoint{-0.017319in}{0.029821in}}{\pgfqpoint{-0.023570in}{0.023570in}}%
\pgfpathcurveto{\pgfqpoint{-0.029821in}{0.017319in}}{\pgfqpoint{-0.033333in}{0.008840in}}{\pgfqpoint{-0.033333in}{0.000000in}}%
\pgfpathcurveto{\pgfqpoint{-0.033333in}{-0.008840in}}{\pgfqpoint{-0.029821in}{-0.017319in}}{\pgfqpoint{-0.023570in}{-0.023570in}}%
\pgfpathcurveto{\pgfqpoint{-0.017319in}{-0.029821in}}{\pgfqpoint{-0.008840in}{-0.033333in}}{\pgfqpoint{0.000000in}{-0.033333in}}%
\pgfpathclose%
\pgfusepath{stroke,fill}%
}%
\begin{pgfscope}%
\pgfsys@transformshift{4.383016in}{4.911810in}%
\pgfsys@useobject{currentmarker}{}%
\end{pgfscope}%
\end{pgfscope}%
\begin{pgfscope}%
\pgfpathrectangle{\pgfqpoint{0.100000in}{2.413063in}}{\pgfqpoint{5.037500in}{3.427208in}}%
\pgfusepath{clip}%
\pgfsetrectcap%
\pgfsetroundjoin%
\pgfsetlinewidth{1.505625pt}%
\definecolor{currentstroke}{rgb}{0.678431,1.000000,0.184314}%
\pgfsetstrokecolor{currentstroke}%
\pgfsetstrokeopacity{0.500000}%
\pgfsetdash{}{0pt}%
\pgfpathmoveto{\pgfqpoint{4.457128in}{4.934032in}}%
\pgfusepath{stroke}%
\end{pgfscope}%
\begin{pgfscope}%
\pgfpathrectangle{\pgfqpoint{0.100000in}{2.413063in}}{\pgfqpoint{5.037500in}{3.427208in}}%
\pgfusepath{clip}%
\pgfsetbuttcap%
\pgfsetroundjoin%
\definecolor{currentfill}{rgb}{0.678431,1.000000,0.184314}%
\pgfsetfillcolor{currentfill}%
\pgfsetfillopacity{0.500000}%
\pgfsetlinewidth{0.250937pt}%
\definecolor{currentstroke}{rgb}{0.000000,0.000000,0.000000}%
\pgfsetstrokecolor{currentstroke}%
\pgfsetstrokeopacity{0.500000}%
\pgfsetdash{}{0pt}%
\pgfsys@defobject{currentmarker}{\pgfqpoint{-0.033333in}{-0.033333in}}{\pgfqpoint{0.033333in}{0.033333in}}{%
\pgfpathmoveto{\pgfqpoint{0.000000in}{-0.033333in}}%
\pgfpathcurveto{\pgfqpoint{0.008840in}{-0.033333in}}{\pgfqpoint{0.017319in}{-0.029821in}}{\pgfqpoint{0.023570in}{-0.023570in}}%
\pgfpathcurveto{\pgfqpoint{0.029821in}{-0.017319in}}{\pgfqpoint{0.033333in}{-0.008840in}}{\pgfqpoint{0.033333in}{0.000000in}}%
\pgfpathcurveto{\pgfqpoint{0.033333in}{0.008840in}}{\pgfqpoint{0.029821in}{0.017319in}}{\pgfqpoint{0.023570in}{0.023570in}}%
\pgfpathcurveto{\pgfqpoint{0.017319in}{0.029821in}}{\pgfqpoint{0.008840in}{0.033333in}}{\pgfqpoint{0.000000in}{0.033333in}}%
\pgfpathcurveto{\pgfqpoint{-0.008840in}{0.033333in}}{\pgfqpoint{-0.017319in}{0.029821in}}{\pgfqpoint{-0.023570in}{0.023570in}}%
\pgfpathcurveto{\pgfqpoint{-0.029821in}{0.017319in}}{\pgfqpoint{-0.033333in}{0.008840in}}{\pgfqpoint{-0.033333in}{0.000000in}}%
\pgfpathcurveto{\pgfqpoint{-0.033333in}{-0.008840in}}{\pgfqpoint{-0.029821in}{-0.017319in}}{\pgfqpoint{-0.023570in}{-0.023570in}}%
\pgfpathcurveto{\pgfqpoint{-0.017319in}{-0.029821in}}{\pgfqpoint{-0.008840in}{-0.033333in}}{\pgfqpoint{0.000000in}{-0.033333in}}%
\pgfpathclose%
\pgfusepath{stroke,fill}%
}%
\begin{pgfscope}%
\pgfsys@transformshift{4.457128in}{4.934032in}%
\pgfsys@useobject{currentmarker}{}%
\end{pgfscope}%
\end{pgfscope}%
\begin{pgfscope}%
\pgfpathrectangle{\pgfqpoint{0.100000in}{2.413063in}}{\pgfqpoint{5.037500in}{3.427208in}}%
\pgfusepath{clip}%
\pgfsetrectcap%
\pgfsetroundjoin%
\pgfsetlinewidth{1.505625pt}%
\definecolor{currentstroke}{rgb}{0.678431,1.000000,0.184314}%
\pgfsetstrokecolor{currentstroke}%
\pgfsetstrokeopacity{0.500000}%
\pgfsetdash{}{0pt}%
\pgfpathmoveto{\pgfqpoint{4.380229in}{5.020625in}}%
\pgfusepath{stroke}%
\end{pgfscope}%
\begin{pgfscope}%
\pgfpathrectangle{\pgfqpoint{0.100000in}{2.413063in}}{\pgfqpoint{5.037500in}{3.427208in}}%
\pgfusepath{clip}%
\pgfsetbuttcap%
\pgfsetroundjoin%
\definecolor{currentfill}{rgb}{0.678431,1.000000,0.184314}%
\pgfsetfillcolor{currentfill}%
\pgfsetfillopacity{0.500000}%
\pgfsetlinewidth{0.250937pt}%
\definecolor{currentstroke}{rgb}{0.000000,0.000000,0.000000}%
\pgfsetstrokecolor{currentstroke}%
\pgfsetstrokeopacity{0.500000}%
\pgfsetdash{}{0pt}%
\pgfsys@defobject{currentmarker}{\pgfqpoint{-0.086111in}{-0.086111in}}{\pgfqpoint{0.086111in}{0.086111in}}{%
\pgfpathmoveto{\pgfqpoint{0.000000in}{-0.086111in}}%
\pgfpathcurveto{\pgfqpoint{0.022837in}{-0.086111in}}{\pgfqpoint{0.044742in}{-0.077038in}}{\pgfqpoint{0.060890in}{-0.060890in}}%
\pgfpathcurveto{\pgfqpoint{0.077038in}{-0.044742in}}{\pgfqpoint{0.086111in}{-0.022837in}}{\pgfqpoint{0.086111in}{0.000000in}}%
\pgfpathcurveto{\pgfqpoint{0.086111in}{0.022837in}}{\pgfqpoint{0.077038in}{0.044742in}}{\pgfqpoint{0.060890in}{0.060890in}}%
\pgfpathcurveto{\pgfqpoint{0.044742in}{0.077038in}}{\pgfqpoint{0.022837in}{0.086111in}}{\pgfqpoint{0.000000in}{0.086111in}}%
\pgfpathcurveto{\pgfqpoint{-0.022837in}{0.086111in}}{\pgfqpoint{-0.044742in}{0.077038in}}{\pgfqpoint{-0.060890in}{0.060890in}}%
\pgfpathcurveto{\pgfqpoint{-0.077038in}{0.044742in}}{\pgfqpoint{-0.086111in}{0.022837in}}{\pgfqpoint{-0.086111in}{0.000000in}}%
\pgfpathcurveto{\pgfqpoint{-0.086111in}{-0.022837in}}{\pgfqpoint{-0.077038in}{-0.044742in}}{\pgfqpoint{-0.060890in}{-0.060890in}}%
\pgfpathcurveto{\pgfqpoint{-0.044742in}{-0.077038in}}{\pgfqpoint{-0.022837in}{-0.086111in}}{\pgfqpoint{0.000000in}{-0.086111in}}%
\pgfpathclose%
\pgfusepath{stroke,fill}%
}%
\begin{pgfscope}%
\pgfsys@transformshift{4.380229in}{5.020625in}%
\pgfsys@useobject{currentmarker}{}%
\end{pgfscope}%
\end{pgfscope}%
\begin{pgfscope}%
\pgfpathrectangle{\pgfqpoint{0.100000in}{2.413063in}}{\pgfqpoint{5.037500in}{3.427208in}}%
\pgfusepath{clip}%
\pgfsetrectcap%
\pgfsetroundjoin%
\pgfsetlinewidth{1.505625pt}%
\definecolor{currentstroke}{rgb}{0.501961,0.501961,0.501961}%
\pgfsetstrokecolor{currentstroke}%
\pgfsetstrokeopacity{0.500000}%
\pgfsetdash{}{0pt}%
\pgfpathmoveto{\pgfqpoint{3.967863in}{3.969128in}}%
\pgfusepath{stroke}%
\end{pgfscope}%
\begin{pgfscope}%
\pgfpathrectangle{\pgfqpoint{0.100000in}{2.413063in}}{\pgfqpoint{5.037500in}{3.427208in}}%
\pgfusepath{clip}%
\pgfsetbuttcap%
\pgfsetroundjoin%
\definecolor{currentfill}{rgb}{0.501961,0.501961,0.501961}%
\pgfsetfillcolor{currentfill}%
\pgfsetfillopacity{0.500000}%
\pgfsetlinewidth{0.250937pt}%
\definecolor{currentstroke}{rgb}{0.000000,0.000000,0.000000}%
\pgfsetstrokecolor{currentstroke}%
\pgfsetstrokeopacity{0.500000}%
\pgfsetdash{}{0pt}%
\pgfsys@defobject{currentmarker}{\pgfqpoint{-0.013889in}{-0.013889in}}{\pgfqpoint{0.013889in}{0.013889in}}{%
\pgfpathmoveto{\pgfqpoint{0.000000in}{-0.013889in}}%
\pgfpathcurveto{\pgfqpoint{0.003683in}{-0.013889in}}{\pgfqpoint{0.007216in}{-0.012425in}}{\pgfqpoint{0.009821in}{-0.009821in}}%
\pgfpathcurveto{\pgfqpoint{0.012425in}{-0.007216in}}{\pgfqpoint{0.013889in}{-0.003683in}}{\pgfqpoint{0.013889in}{0.000000in}}%
\pgfpathcurveto{\pgfqpoint{0.013889in}{0.003683in}}{\pgfqpoint{0.012425in}{0.007216in}}{\pgfqpoint{0.009821in}{0.009821in}}%
\pgfpathcurveto{\pgfqpoint{0.007216in}{0.012425in}}{\pgfqpoint{0.003683in}{0.013889in}}{\pgfqpoint{0.000000in}{0.013889in}}%
\pgfpathcurveto{\pgfqpoint{-0.003683in}{0.013889in}}{\pgfqpoint{-0.007216in}{0.012425in}}{\pgfqpoint{-0.009821in}{0.009821in}}%
\pgfpathcurveto{\pgfqpoint{-0.012425in}{0.007216in}}{\pgfqpoint{-0.013889in}{0.003683in}}{\pgfqpoint{-0.013889in}{0.000000in}}%
\pgfpathcurveto{\pgfqpoint{-0.013889in}{-0.003683in}}{\pgfqpoint{-0.012425in}{-0.007216in}}{\pgfqpoint{-0.009821in}{-0.009821in}}%
\pgfpathcurveto{\pgfqpoint{-0.007216in}{-0.012425in}}{\pgfqpoint{-0.003683in}{-0.013889in}}{\pgfqpoint{0.000000in}{-0.013889in}}%
\pgfpathclose%
\pgfusepath{stroke,fill}%
}%
\begin{pgfscope}%
\pgfsys@transformshift{3.967863in}{3.969128in}%
\pgfsys@useobject{currentmarker}{}%
\end{pgfscope}%
\end{pgfscope}%
\begin{pgfscope}%
\pgfpathrectangle{\pgfqpoint{0.100000in}{2.413063in}}{\pgfqpoint{5.037500in}{3.427208in}}%
\pgfusepath{clip}%
\pgfsetrectcap%
\pgfsetroundjoin%
\pgfsetlinewidth{1.505625pt}%
\definecolor{currentstroke}{rgb}{0.501961,0.501961,0.501961}%
\pgfsetstrokecolor{currentstroke}%
\pgfsetstrokeopacity{0.500000}%
\pgfsetdash{}{0pt}%
\pgfpathmoveto{\pgfqpoint{4.246950in}{4.070213in}}%
\pgfusepath{stroke}%
\end{pgfscope}%
\begin{pgfscope}%
\pgfpathrectangle{\pgfqpoint{0.100000in}{2.413063in}}{\pgfqpoint{5.037500in}{3.427208in}}%
\pgfusepath{clip}%
\pgfsetbuttcap%
\pgfsetroundjoin%
\definecolor{currentfill}{rgb}{0.501961,0.501961,0.501961}%
\pgfsetfillcolor{currentfill}%
\pgfsetfillopacity{0.500000}%
\pgfsetlinewidth{0.250937pt}%
\definecolor{currentstroke}{rgb}{0.000000,0.000000,0.000000}%
\pgfsetstrokecolor{currentstroke}%
\pgfsetstrokeopacity{0.500000}%
\pgfsetdash{}{0pt}%
\pgfsys@defobject{currentmarker}{\pgfqpoint{-0.013889in}{-0.013889in}}{\pgfqpoint{0.013889in}{0.013889in}}{%
\pgfpathmoveto{\pgfqpoint{0.000000in}{-0.013889in}}%
\pgfpathcurveto{\pgfqpoint{0.003683in}{-0.013889in}}{\pgfqpoint{0.007216in}{-0.012425in}}{\pgfqpoint{0.009821in}{-0.009821in}}%
\pgfpathcurveto{\pgfqpoint{0.012425in}{-0.007216in}}{\pgfqpoint{0.013889in}{-0.003683in}}{\pgfqpoint{0.013889in}{0.000000in}}%
\pgfpathcurveto{\pgfqpoint{0.013889in}{0.003683in}}{\pgfqpoint{0.012425in}{0.007216in}}{\pgfqpoint{0.009821in}{0.009821in}}%
\pgfpathcurveto{\pgfqpoint{0.007216in}{0.012425in}}{\pgfqpoint{0.003683in}{0.013889in}}{\pgfqpoint{0.000000in}{0.013889in}}%
\pgfpathcurveto{\pgfqpoint{-0.003683in}{0.013889in}}{\pgfqpoint{-0.007216in}{0.012425in}}{\pgfqpoint{-0.009821in}{0.009821in}}%
\pgfpathcurveto{\pgfqpoint{-0.012425in}{0.007216in}}{\pgfqpoint{-0.013889in}{0.003683in}}{\pgfqpoint{-0.013889in}{0.000000in}}%
\pgfpathcurveto{\pgfqpoint{-0.013889in}{-0.003683in}}{\pgfqpoint{-0.012425in}{-0.007216in}}{\pgfqpoint{-0.009821in}{-0.009821in}}%
\pgfpathcurveto{\pgfqpoint{-0.007216in}{-0.012425in}}{\pgfqpoint{-0.003683in}{-0.013889in}}{\pgfqpoint{0.000000in}{-0.013889in}}%
\pgfpathclose%
\pgfusepath{stroke,fill}%
}%
\begin{pgfscope}%
\pgfsys@transformshift{4.246950in}{4.070213in}%
\pgfsys@useobject{currentmarker}{}%
\end{pgfscope}%
\end{pgfscope}%
\begin{pgfscope}%
\pgfpathrectangle{\pgfqpoint{0.100000in}{2.413063in}}{\pgfqpoint{5.037500in}{3.427208in}}%
\pgfusepath{clip}%
\pgfsetrectcap%
\pgfsetroundjoin%
\pgfsetlinewidth{1.505625pt}%
\definecolor{currentstroke}{rgb}{0.000000,0.000000,1.000000}%
\pgfsetstrokecolor{currentstroke}%
\pgfsetstrokeopacity{0.500000}%
\pgfsetdash{}{0pt}%
\pgfpathmoveto{\pgfqpoint{4.133311in}{3.949918in}}%
\pgfusepath{stroke}%
\end{pgfscope}%
\begin{pgfscope}%
\pgfpathrectangle{\pgfqpoint{0.100000in}{2.413063in}}{\pgfqpoint{5.037500in}{3.427208in}}%
\pgfusepath{clip}%
\pgfsetbuttcap%
\pgfsetroundjoin%
\definecolor{currentfill}{rgb}{0.000000,0.000000,1.000000}%
\pgfsetfillcolor{currentfill}%
\pgfsetfillopacity{0.500000}%
\pgfsetlinewidth{0.250937pt}%
\definecolor{currentstroke}{rgb}{0.000000,0.000000,0.000000}%
\pgfsetstrokecolor{currentstroke}%
\pgfsetstrokeopacity{0.500000}%
\pgfsetdash{}{0pt}%
\pgfsys@defobject{currentmarker}{\pgfqpoint{-0.005556in}{-0.005556in}}{\pgfqpoint{0.005556in}{0.005556in}}{%
\pgfpathmoveto{\pgfqpoint{0.000000in}{-0.005556in}}%
\pgfpathcurveto{\pgfqpoint{0.001473in}{-0.005556in}}{\pgfqpoint{0.002887in}{-0.004970in}}{\pgfqpoint{0.003928in}{-0.003928in}}%
\pgfpathcurveto{\pgfqpoint{0.004970in}{-0.002887in}}{\pgfqpoint{0.005556in}{-0.001473in}}{\pgfqpoint{0.005556in}{0.000000in}}%
\pgfpathcurveto{\pgfqpoint{0.005556in}{0.001473in}}{\pgfqpoint{0.004970in}{0.002887in}}{\pgfqpoint{0.003928in}{0.003928in}}%
\pgfpathcurveto{\pgfqpoint{0.002887in}{0.004970in}}{\pgfqpoint{0.001473in}{0.005556in}}{\pgfqpoint{0.000000in}{0.005556in}}%
\pgfpathcurveto{\pgfqpoint{-0.001473in}{0.005556in}}{\pgfqpoint{-0.002887in}{0.004970in}}{\pgfqpoint{-0.003928in}{0.003928in}}%
\pgfpathcurveto{\pgfqpoint{-0.004970in}{0.002887in}}{\pgfqpoint{-0.005556in}{0.001473in}}{\pgfqpoint{-0.005556in}{0.000000in}}%
\pgfpathcurveto{\pgfqpoint{-0.005556in}{-0.001473in}}{\pgfqpoint{-0.004970in}{-0.002887in}}{\pgfqpoint{-0.003928in}{-0.003928in}}%
\pgfpathcurveto{\pgfqpoint{-0.002887in}{-0.004970in}}{\pgfqpoint{-0.001473in}{-0.005556in}}{\pgfqpoint{0.000000in}{-0.005556in}}%
\pgfpathclose%
\pgfusepath{stroke,fill}%
}%
\begin{pgfscope}%
\pgfsys@transformshift{4.133311in}{3.949918in}%
\pgfsys@useobject{currentmarker}{}%
\end{pgfscope}%
\end{pgfscope}%
\begin{pgfscope}%
\pgfpathrectangle{\pgfqpoint{0.100000in}{2.413063in}}{\pgfqpoint{5.037500in}{3.427208in}}%
\pgfusepath{clip}%
\pgfsetrectcap%
\pgfsetroundjoin%
\pgfsetlinewidth{1.505625pt}%
\definecolor{currentstroke}{rgb}{0.678431,1.000000,0.184314}%
\pgfsetstrokecolor{currentstroke}%
\pgfsetstrokeopacity{0.500000}%
\pgfsetdash{}{0pt}%
\pgfpathmoveto{\pgfqpoint{4.298489in}{4.067381in}}%
\pgfusepath{stroke}%
\end{pgfscope}%
\begin{pgfscope}%
\pgfpathrectangle{\pgfqpoint{0.100000in}{2.413063in}}{\pgfqpoint{5.037500in}{3.427208in}}%
\pgfusepath{clip}%
\pgfsetbuttcap%
\pgfsetroundjoin%
\definecolor{currentfill}{rgb}{0.678431,1.000000,0.184314}%
\pgfsetfillcolor{currentfill}%
\pgfsetfillopacity{0.500000}%
\pgfsetlinewidth{0.250937pt}%
\definecolor{currentstroke}{rgb}{0.000000,0.000000,0.000000}%
\pgfsetstrokecolor{currentstroke}%
\pgfsetstrokeopacity{0.500000}%
\pgfsetdash{}{0pt}%
\pgfsys@defobject{currentmarker}{\pgfqpoint{-0.008333in}{-0.008333in}}{\pgfqpoint{0.008333in}{0.008333in}}{%
\pgfpathmoveto{\pgfqpoint{0.000000in}{-0.008333in}}%
\pgfpathcurveto{\pgfqpoint{0.002210in}{-0.008333in}}{\pgfqpoint{0.004330in}{-0.007455in}}{\pgfqpoint{0.005893in}{-0.005893in}}%
\pgfpathcurveto{\pgfqpoint{0.007455in}{-0.004330in}}{\pgfqpoint{0.008333in}{-0.002210in}}{\pgfqpoint{0.008333in}{0.000000in}}%
\pgfpathcurveto{\pgfqpoint{0.008333in}{0.002210in}}{\pgfqpoint{0.007455in}{0.004330in}}{\pgfqpoint{0.005893in}{0.005893in}}%
\pgfpathcurveto{\pgfqpoint{0.004330in}{0.007455in}}{\pgfqpoint{0.002210in}{0.008333in}}{\pgfqpoint{0.000000in}{0.008333in}}%
\pgfpathcurveto{\pgfqpoint{-0.002210in}{0.008333in}}{\pgfqpoint{-0.004330in}{0.007455in}}{\pgfqpoint{-0.005893in}{0.005893in}}%
\pgfpathcurveto{\pgfqpoint{-0.007455in}{0.004330in}}{\pgfqpoint{-0.008333in}{0.002210in}}{\pgfqpoint{-0.008333in}{0.000000in}}%
\pgfpathcurveto{\pgfqpoint{-0.008333in}{-0.002210in}}{\pgfqpoint{-0.007455in}{-0.004330in}}{\pgfqpoint{-0.005893in}{-0.005893in}}%
\pgfpathcurveto{\pgfqpoint{-0.004330in}{-0.007455in}}{\pgfqpoint{-0.002210in}{-0.008333in}}{\pgfqpoint{0.000000in}{-0.008333in}}%
\pgfpathclose%
\pgfusepath{stroke,fill}%
}%
\begin{pgfscope}%
\pgfsys@transformshift{4.298489in}{4.067381in}%
\pgfsys@useobject{currentmarker}{}%
\end{pgfscope}%
\end{pgfscope}%
\begin{pgfscope}%
\pgfpathrectangle{\pgfqpoint{0.100000in}{2.413063in}}{\pgfqpoint{5.037500in}{3.427208in}}%
\pgfusepath{clip}%
\pgfsetrectcap%
\pgfsetroundjoin%
\pgfsetlinewidth{1.505625pt}%
\definecolor{currentstroke}{rgb}{0.000000,0.000000,1.000000}%
\pgfsetstrokecolor{currentstroke}%
\pgfsetstrokeopacity{0.500000}%
\pgfsetdash{}{0pt}%
\pgfpathmoveto{\pgfqpoint{4.319477in}{3.960847in}}%
\pgfusepath{stroke}%
\end{pgfscope}%
\begin{pgfscope}%
\pgfpathrectangle{\pgfqpoint{0.100000in}{2.413063in}}{\pgfqpoint{5.037500in}{3.427208in}}%
\pgfusepath{clip}%
\pgfsetbuttcap%
\pgfsetroundjoin%
\definecolor{currentfill}{rgb}{0.000000,0.000000,1.000000}%
\pgfsetfillcolor{currentfill}%
\pgfsetfillopacity{0.500000}%
\pgfsetlinewidth{0.250937pt}%
\definecolor{currentstroke}{rgb}{0.000000,0.000000,0.000000}%
\pgfsetstrokecolor{currentstroke}%
\pgfsetstrokeopacity{0.500000}%
\pgfsetdash{}{0pt}%
\pgfsys@defobject{currentmarker}{\pgfqpoint{-0.013889in}{-0.013889in}}{\pgfqpoint{0.013889in}{0.013889in}}{%
\pgfpathmoveto{\pgfqpoint{0.000000in}{-0.013889in}}%
\pgfpathcurveto{\pgfqpoint{0.003683in}{-0.013889in}}{\pgfqpoint{0.007216in}{-0.012425in}}{\pgfqpoint{0.009821in}{-0.009821in}}%
\pgfpathcurveto{\pgfqpoint{0.012425in}{-0.007216in}}{\pgfqpoint{0.013889in}{-0.003683in}}{\pgfqpoint{0.013889in}{0.000000in}}%
\pgfpathcurveto{\pgfqpoint{0.013889in}{0.003683in}}{\pgfqpoint{0.012425in}{0.007216in}}{\pgfqpoint{0.009821in}{0.009821in}}%
\pgfpathcurveto{\pgfqpoint{0.007216in}{0.012425in}}{\pgfqpoint{0.003683in}{0.013889in}}{\pgfqpoint{0.000000in}{0.013889in}}%
\pgfpathcurveto{\pgfqpoint{-0.003683in}{0.013889in}}{\pgfqpoint{-0.007216in}{0.012425in}}{\pgfqpoint{-0.009821in}{0.009821in}}%
\pgfpathcurveto{\pgfqpoint{-0.012425in}{0.007216in}}{\pgfqpoint{-0.013889in}{0.003683in}}{\pgfqpoint{-0.013889in}{0.000000in}}%
\pgfpathcurveto{\pgfqpoint{-0.013889in}{-0.003683in}}{\pgfqpoint{-0.012425in}{-0.007216in}}{\pgfqpoint{-0.009821in}{-0.009821in}}%
\pgfpathcurveto{\pgfqpoint{-0.007216in}{-0.012425in}}{\pgfqpoint{-0.003683in}{-0.013889in}}{\pgfqpoint{0.000000in}{-0.013889in}}%
\pgfpathclose%
\pgfusepath{stroke,fill}%
}%
\begin{pgfscope}%
\pgfsys@transformshift{4.319477in}{3.960847in}%
\pgfsys@useobject{currentmarker}{}%
\end{pgfscope}%
\end{pgfscope}%
\begin{pgfscope}%
\pgfpathrectangle{\pgfqpoint{0.100000in}{2.413063in}}{\pgfqpoint{5.037500in}{3.427208in}}%
\pgfusepath{clip}%
\pgfsetrectcap%
\pgfsetroundjoin%
\pgfsetlinewidth{1.505625pt}%
\definecolor{currentstroke}{rgb}{0.501961,0.501961,0.501961}%
\pgfsetstrokecolor{currentstroke}%
\pgfsetstrokeopacity{0.500000}%
\pgfsetdash{}{0pt}%
\pgfpathmoveto{\pgfqpoint{4.394677in}{4.013655in}}%
\pgfusepath{stroke}%
\end{pgfscope}%
\begin{pgfscope}%
\pgfpathrectangle{\pgfqpoint{0.100000in}{2.413063in}}{\pgfqpoint{5.037500in}{3.427208in}}%
\pgfusepath{clip}%
\pgfsetbuttcap%
\pgfsetroundjoin%
\definecolor{currentfill}{rgb}{0.501961,0.501961,0.501961}%
\pgfsetfillcolor{currentfill}%
\pgfsetfillopacity{0.500000}%
\pgfsetlinewidth{0.250937pt}%
\definecolor{currentstroke}{rgb}{0.000000,0.000000,0.000000}%
\pgfsetstrokecolor{currentstroke}%
\pgfsetstrokeopacity{0.500000}%
\pgfsetdash{}{0pt}%
\pgfsys@defobject{currentmarker}{\pgfqpoint{-0.013889in}{-0.013889in}}{\pgfqpoint{0.013889in}{0.013889in}}{%
\pgfpathmoveto{\pgfqpoint{0.000000in}{-0.013889in}}%
\pgfpathcurveto{\pgfqpoint{0.003683in}{-0.013889in}}{\pgfqpoint{0.007216in}{-0.012425in}}{\pgfqpoint{0.009821in}{-0.009821in}}%
\pgfpathcurveto{\pgfqpoint{0.012425in}{-0.007216in}}{\pgfqpoint{0.013889in}{-0.003683in}}{\pgfqpoint{0.013889in}{0.000000in}}%
\pgfpathcurveto{\pgfqpoint{0.013889in}{0.003683in}}{\pgfqpoint{0.012425in}{0.007216in}}{\pgfqpoint{0.009821in}{0.009821in}}%
\pgfpathcurveto{\pgfqpoint{0.007216in}{0.012425in}}{\pgfqpoint{0.003683in}{0.013889in}}{\pgfqpoint{0.000000in}{0.013889in}}%
\pgfpathcurveto{\pgfqpoint{-0.003683in}{0.013889in}}{\pgfqpoint{-0.007216in}{0.012425in}}{\pgfqpoint{-0.009821in}{0.009821in}}%
\pgfpathcurveto{\pgfqpoint{-0.012425in}{0.007216in}}{\pgfqpoint{-0.013889in}{0.003683in}}{\pgfqpoint{-0.013889in}{0.000000in}}%
\pgfpathcurveto{\pgfqpoint{-0.013889in}{-0.003683in}}{\pgfqpoint{-0.012425in}{-0.007216in}}{\pgfqpoint{-0.009821in}{-0.009821in}}%
\pgfpathcurveto{\pgfqpoint{-0.007216in}{-0.012425in}}{\pgfqpoint{-0.003683in}{-0.013889in}}{\pgfqpoint{0.000000in}{-0.013889in}}%
\pgfpathclose%
\pgfusepath{stroke,fill}%
}%
\begin{pgfscope}%
\pgfsys@transformshift{4.394677in}{4.013655in}%
\pgfsys@useobject{currentmarker}{}%
\end{pgfscope}%
\end{pgfscope}%
\begin{pgfscope}%
\pgfpathrectangle{\pgfqpoint{0.100000in}{2.413063in}}{\pgfqpoint{5.037500in}{3.427208in}}%
\pgfusepath{clip}%
\pgfsetrectcap%
\pgfsetroundjoin%
\pgfsetlinewidth{1.505625pt}%
\definecolor{currentstroke}{rgb}{0.000000,0.000000,1.000000}%
\pgfsetstrokecolor{currentstroke}%
\pgfsetstrokeopacity{0.500000}%
\pgfsetdash{}{0pt}%
\pgfpathmoveto{\pgfqpoint{4.214866in}{4.062032in}}%
\pgfusepath{stroke}%
\end{pgfscope}%
\begin{pgfscope}%
\pgfpathrectangle{\pgfqpoint{0.100000in}{2.413063in}}{\pgfqpoint{5.037500in}{3.427208in}}%
\pgfusepath{clip}%
\pgfsetbuttcap%
\pgfsetroundjoin%
\definecolor{currentfill}{rgb}{0.000000,0.000000,1.000000}%
\pgfsetfillcolor{currentfill}%
\pgfsetfillopacity{0.500000}%
\pgfsetlinewidth{0.250937pt}%
\definecolor{currentstroke}{rgb}{0.000000,0.000000,0.000000}%
\pgfsetstrokecolor{currentstroke}%
\pgfsetstrokeopacity{0.500000}%
\pgfsetdash{}{0pt}%
\pgfsys@defobject{currentmarker}{\pgfqpoint{-0.011111in}{-0.011111in}}{\pgfqpoint{0.011111in}{0.011111in}}{%
\pgfpathmoveto{\pgfqpoint{0.000000in}{-0.011111in}}%
\pgfpathcurveto{\pgfqpoint{0.002947in}{-0.011111in}}{\pgfqpoint{0.005773in}{-0.009940in}}{\pgfqpoint{0.007857in}{-0.007857in}}%
\pgfpathcurveto{\pgfqpoint{0.009940in}{-0.005773in}}{\pgfqpoint{0.011111in}{-0.002947in}}{\pgfqpoint{0.011111in}{0.000000in}}%
\pgfpathcurveto{\pgfqpoint{0.011111in}{0.002947in}}{\pgfqpoint{0.009940in}{0.005773in}}{\pgfqpoint{0.007857in}{0.007857in}}%
\pgfpathcurveto{\pgfqpoint{0.005773in}{0.009940in}}{\pgfqpoint{0.002947in}{0.011111in}}{\pgfqpoint{0.000000in}{0.011111in}}%
\pgfpathcurveto{\pgfqpoint{-0.002947in}{0.011111in}}{\pgfqpoint{-0.005773in}{0.009940in}}{\pgfqpoint{-0.007857in}{0.007857in}}%
\pgfpathcurveto{\pgfqpoint{-0.009940in}{0.005773in}}{\pgfqpoint{-0.011111in}{0.002947in}}{\pgfqpoint{-0.011111in}{0.000000in}}%
\pgfpathcurveto{\pgfqpoint{-0.011111in}{-0.002947in}}{\pgfqpoint{-0.009940in}{-0.005773in}}{\pgfqpoint{-0.007857in}{-0.007857in}}%
\pgfpathcurveto{\pgfqpoint{-0.005773in}{-0.009940in}}{\pgfqpoint{-0.002947in}{-0.011111in}}{\pgfqpoint{0.000000in}{-0.011111in}}%
\pgfpathclose%
\pgfusepath{stroke,fill}%
}%
\begin{pgfscope}%
\pgfsys@transformshift{4.214866in}{4.062032in}%
\pgfsys@useobject{currentmarker}{}%
\end{pgfscope}%
\end{pgfscope}%
\begin{pgfscope}%
\pgfpathrectangle{\pgfqpoint{0.100000in}{2.413063in}}{\pgfqpoint{5.037500in}{3.427208in}}%
\pgfusepath{clip}%
\pgfsetrectcap%
\pgfsetroundjoin%
\pgfsetlinewidth{1.505625pt}%
\definecolor{currentstroke}{rgb}{0.501961,0.501961,0.501961}%
\pgfsetstrokecolor{currentstroke}%
\pgfsetstrokeopacity{0.500000}%
\pgfsetdash{}{0pt}%
\pgfpathmoveto{\pgfqpoint{4.447469in}{4.050603in}}%
\pgfusepath{stroke}%
\end{pgfscope}%
\begin{pgfscope}%
\pgfpathrectangle{\pgfqpoint{0.100000in}{2.413063in}}{\pgfqpoint{5.037500in}{3.427208in}}%
\pgfusepath{clip}%
\pgfsetbuttcap%
\pgfsetroundjoin%
\definecolor{currentfill}{rgb}{0.501961,0.501961,0.501961}%
\pgfsetfillcolor{currentfill}%
\pgfsetfillopacity{0.500000}%
\pgfsetlinewidth{0.250937pt}%
\definecolor{currentstroke}{rgb}{0.000000,0.000000,0.000000}%
\pgfsetstrokecolor{currentstroke}%
\pgfsetstrokeopacity{0.500000}%
\pgfsetdash{}{0pt}%
\pgfsys@defobject{currentmarker}{\pgfqpoint{-0.013889in}{-0.013889in}}{\pgfqpoint{0.013889in}{0.013889in}}{%
\pgfpathmoveto{\pgfqpoint{0.000000in}{-0.013889in}}%
\pgfpathcurveto{\pgfqpoint{0.003683in}{-0.013889in}}{\pgfqpoint{0.007216in}{-0.012425in}}{\pgfqpoint{0.009821in}{-0.009821in}}%
\pgfpathcurveto{\pgfqpoint{0.012425in}{-0.007216in}}{\pgfqpoint{0.013889in}{-0.003683in}}{\pgfqpoint{0.013889in}{0.000000in}}%
\pgfpathcurveto{\pgfqpoint{0.013889in}{0.003683in}}{\pgfqpoint{0.012425in}{0.007216in}}{\pgfqpoint{0.009821in}{0.009821in}}%
\pgfpathcurveto{\pgfqpoint{0.007216in}{0.012425in}}{\pgfqpoint{0.003683in}{0.013889in}}{\pgfqpoint{0.000000in}{0.013889in}}%
\pgfpathcurveto{\pgfqpoint{-0.003683in}{0.013889in}}{\pgfqpoint{-0.007216in}{0.012425in}}{\pgfqpoint{-0.009821in}{0.009821in}}%
\pgfpathcurveto{\pgfqpoint{-0.012425in}{0.007216in}}{\pgfqpoint{-0.013889in}{0.003683in}}{\pgfqpoint{-0.013889in}{0.000000in}}%
\pgfpathcurveto{\pgfqpoint{-0.013889in}{-0.003683in}}{\pgfqpoint{-0.012425in}{-0.007216in}}{\pgfqpoint{-0.009821in}{-0.009821in}}%
\pgfpathcurveto{\pgfqpoint{-0.007216in}{-0.012425in}}{\pgfqpoint{-0.003683in}{-0.013889in}}{\pgfqpoint{0.000000in}{-0.013889in}}%
\pgfpathclose%
\pgfusepath{stroke,fill}%
}%
\begin{pgfscope}%
\pgfsys@transformshift{4.447469in}{4.050603in}%
\pgfsys@useobject{currentmarker}{}%
\end{pgfscope}%
\end{pgfscope}%
\begin{pgfscope}%
\pgfpathrectangle{\pgfqpoint{0.100000in}{2.413063in}}{\pgfqpoint{5.037500in}{3.427208in}}%
\pgfusepath{clip}%
\pgfsetrectcap%
\pgfsetroundjoin%
\pgfsetlinewidth{1.505625pt}%
\definecolor{currentstroke}{rgb}{0.678431,1.000000,0.184314}%
\pgfsetstrokecolor{currentstroke}%
\pgfsetstrokeopacity{0.500000}%
\pgfsetdash{}{0pt}%
\pgfpathmoveto{\pgfqpoint{4.078163in}{4.000492in}}%
\pgfusepath{stroke}%
\end{pgfscope}%
\begin{pgfscope}%
\pgfpathrectangle{\pgfqpoint{0.100000in}{2.413063in}}{\pgfqpoint{5.037500in}{3.427208in}}%
\pgfusepath{clip}%
\pgfsetbuttcap%
\pgfsetroundjoin%
\definecolor{currentfill}{rgb}{0.678431,1.000000,0.184314}%
\pgfsetfillcolor{currentfill}%
\pgfsetfillopacity{0.500000}%
\pgfsetlinewidth{0.250937pt}%
\definecolor{currentstroke}{rgb}{0.000000,0.000000,0.000000}%
\pgfsetstrokecolor{currentstroke}%
\pgfsetstrokeopacity{0.500000}%
\pgfsetdash{}{0pt}%
\pgfsys@defobject{currentmarker}{\pgfqpoint{-0.005556in}{-0.005556in}}{\pgfqpoint{0.005556in}{0.005556in}}{%
\pgfpathmoveto{\pgfqpoint{0.000000in}{-0.005556in}}%
\pgfpathcurveto{\pgfqpoint{0.001473in}{-0.005556in}}{\pgfqpoint{0.002887in}{-0.004970in}}{\pgfqpoint{0.003928in}{-0.003928in}}%
\pgfpathcurveto{\pgfqpoint{0.004970in}{-0.002887in}}{\pgfqpoint{0.005556in}{-0.001473in}}{\pgfqpoint{0.005556in}{0.000000in}}%
\pgfpathcurveto{\pgfqpoint{0.005556in}{0.001473in}}{\pgfqpoint{0.004970in}{0.002887in}}{\pgfqpoint{0.003928in}{0.003928in}}%
\pgfpathcurveto{\pgfqpoint{0.002887in}{0.004970in}}{\pgfqpoint{0.001473in}{0.005556in}}{\pgfqpoint{0.000000in}{0.005556in}}%
\pgfpathcurveto{\pgfqpoint{-0.001473in}{0.005556in}}{\pgfqpoint{-0.002887in}{0.004970in}}{\pgfqpoint{-0.003928in}{0.003928in}}%
\pgfpathcurveto{\pgfqpoint{-0.004970in}{0.002887in}}{\pgfqpoint{-0.005556in}{0.001473in}}{\pgfqpoint{-0.005556in}{0.000000in}}%
\pgfpathcurveto{\pgfqpoint{-0.005556in}{-0.001473in}}{\pgfqpoint{-0.004970in}{-0.002887in}}{\pgfqpoint{-0.003928in}{-0.003928in}}%
\pgfpathcurveto{\pgfqpoint{-0.002887in}{-0.004970in}}{\pgfqpoint{-0.001473in}{-0.005556in}}{\pgfqpoint{0.000000in}{-0.005556in}}%
\pgfpathclose%
\pgfusepath{stroke,fill}%
}%
\begin{pgfscope}%
\pgfsys@transformshift{4.078163in}{4.000492in}%
\pgfsys@useobject{currentmarker}{}%
\end{pgfscope}%
\end{pgfscope}%
\begin{pgfscope}%
\pgfpathrectangle{\pgfqpoint{0.100000in}{2.413063in}}{\pgfqpoint{5.037500in}{3.427208in}}%
\pgfusepath{clip}%
\pgfsetrectcap%
\pgfsetroundjoin%
\pgfsetlinewidth{1.505625pt}%
\definecolor{currentstroke}{rgb}{0.678431,1.000000,0.184314}%
\pgfsetstrokecolor{currentstroke}%
\pgfsetstrokeopacity{0.500000}%
\pgfsetdash{}{0pt}%
\pgfpathmoveto{\pgfqpoint{4.460629in}{3.952203in}}%
\pgfusepath{stroke}%
\end{pgfscope}%
\begin{pgfscope}%
\pgfpathrectangle{\pgfqpoint{0.100000in}{2.413063in}}{\pgfqpoint{5.037500in}{3.427208in}}%
\pgfusepath{clip}%
\pgfsetbuttcap%
\pgfsetroundjoin%
\definecolor{currentfill}{rgb}{0.678431,1.000000,0.184314}%
\pgfsetfillcolor{currentfill}%
\pgfsetfillopacity{0.500000}%
\pgfsetlinewidth{0.250937pt}%
\definecolor{currentstroke}{rgb}{0.000000,0.000000,0.000000}%
\pgfsetstrokecolor{currentstroke}%
\pgfsetstrokeopacity{0.500000}%
\pgfsetdash{}{0pt}%
\pgfsys@defobject{currentmarker}{\pgfqpoint{-0.016667in}{-0.016667in}}{\pgfqpoint{0.016667in}{0.016667in}}{%
\pgfpathmoveto{\pgfqpoint{0.000000in}{-0.016667in}}%
\pgfpathcurveto{\pgfqpoint{0.004420in}{-0.016667in}}{\pgfqpoint{0.008660in}{-0.014911in}}{\pgfqpoint{0.011785in}{-0.011785in}}%
\pgfpathcurveto{\pgfqpoint{0.014911in}{-0.008660in}}{\pgfqpoint{0.016667in}{-0.004420in}}{\pgfqpoint{0.016667in}{0.000000in}}%
\pgfpathcurveto{\pgfqpoint{0.016667in}{0.004420in}}{\pgfqpoint{0.014911in}{0.008660in}}{\pgfqpoint{0.011785in}{0.011785in}}%
\pgfpathcurveto{\pgfqpoint{0.008660in}{0.014911in}}{\pgfqpoint{0.004420in}{0.016667in}}{\pgfqpoint{0.000000in}{0.016667in}}%
\pgfpathcurveto{\pgfqpoint{-0.004420in}{0.016667in}}{\pgfqpoint{-0.008660in}{0.014911in}}{\pgfqpoint{-0.011785in}{0.011785in}}%
\pgfpathcurveto{\pgfqpoint{-0.014911in}{0.008660in}}{\pgfqpoint{-0.016667in}{0.004420in}}{\pgfqpoint{-0.016667in}{0.000000in}}%
\pgfpathcurveto{\pgfqpoint{-0.016667in}{-0.004420in}}{\pgfqpoint{-0.014911in}{-0.008660in}}{\pgfqpoint{-0.011785in}{-0.011785in}}%
\pgfpathcurveto{\pgfqpoint{-0.008660in}{-0.014911in}}{\pgfqpoint{-0.004420in}{-0.016667in}}{\pgfqpoint{0.000000in}{-0.016667in}}%
\pgfpathclose%
\pgfusepath{stroke,fill}%
}%
\begin{pgfscope}%
\pgfsys@transformshift{4.460629in}{3.952203in}%
\pgfsys@useobject{currentmarker}{}%
\end{pgfscope}%
\end{pgfscope}%
\begin{pgfscope}%
\pgfpathrectangle{\pgfqpoint{0.100000in}{2.413063in}}{\pgfqpoint{5.037500in}{3.427208in}}%
\pgfusepath{clip}%
\pgfsetrectcap%
\pgfsetroundjoin%
\pgfsetlinewidth{1.505625pt}%
\definecolor{currentstroke}{rgb}{0.678431,1.000000,0.184314}%
\pgfsetstrokecolor{currentstroke}%
\pgfsetstrokeopacity{0.500000}%
\pgfsetdash{}{0pt}%
\pgfpathmoveto{\pgfqpoint{4.488531in}{3.999440in}}%
\pgfusepath{stroke}%
\end{pgfscope}%
\begin{pgfscope}%
\pgfpathrectangle{\pgfqpoint{0.100000in}{2.413063in}}{\pgfqpoint{5.037500in}{3.427208in}}%
\pgfusepath{clip}%
\pgfsetbuttcap%
\pgfsetroundjoin%
\definecolor{currentfill}{rgb}{0.678431,1.000000,0.184314}%
\pgfsetfillcolor{currentfill}%
\pgfsetfillopacity{0.500000}%
\pgfsetlinewidth{0.250937pt}%
\definecolor{currentstroke}{rgb}{0.000000,0.000000,0.000000}%
\pgfsetstrokecolor{currentstroke}%
\pgfsetstrokeopacity{0.500000}%
\pgfsetdash{}{0pt}%
\pgfsys@defobject{currentmarker}{\pgfqpoint{-0.008333in}{-0.008333in}}{\pgfqpoint{0.008333in}{0.008333in}}{%
\pgfpathmoveto{\pgfqpoint{0.000000in}{-0.008333in}}%
\pgfpathcurveto{\pgfqpoint{0.002210in}{-0.008333in}}{\pgfqpoint{0.004330in}{-0.007455in}}{\pgfqpoint{0.005893in}{-0.005893in}}%
\pgfpathcurveto{\pgfqpoint{0.007455in}{-0.004330in}}{\pgfqpoint{0.008333in}{-0.002210in}}{\pgfqpoint{0.008333in}{0.000000in}}%
\pgfpathcurveto{\pgfqpoint{0.008333in}{0.002210in}}{\pgfqpoint{0.007455in}{0.004330in}}{\pgfqpoint{0.005893in}{0.005893in}}%
\pgfpathcurveto{\pgfqpoint{0.004330in}{0.007455in}}{\pgfqpoint{0.002210in}{0.008333in}}{\pgfqpoint{0.000000in}{0.008333in}}%
\pgfpathcurveto{\pgfqpoint{-0.002210in}{0.008333in}}{\pgfqpoint{-0.004330in}{0.007455in}}{\pgfqpoint{-0.005893in}{0.005893in}}%
\pgfpathcurveto{\pgfqpoint{-0.007455in}{0.004330in}}{\pgfqpoint{-0.008333in}{0.002210in}}{\pgfqpoint{-0.008333in}{0.000000in}}%
\pgfpathcurveto{\pgfqpoint{-0.008333in}{-0.002210in}}{\pgfqpoint{-0.007455in}{-0.004330in}}{\pgfqpoint{-0.005893in}{-0.005893in}}%
\pgfpathcurveto{\pgfqpoint{-0.004330in}{-0.007455in}}{\pgfqpoint{-0.002210in}{-0.008333in}}{\pgfqpoint{0.000000in}{-0.008333in}}%
\pgfpathclose%
\pgfusepath{stroke,fill}%
}%
\begin{pgfscope}%
\pgfsys@transformshift{4.488531in}{3.999440in}%
\pgfsys@useobject{currentmarker}{}%
\end{pgfscope}%
\end{pgfscope}%
\begin{pgfscope}%
\pgfpathrectangle{\pgfqpoint{0.100000in}{2.413063in}}{\pgfqpoint{5.037500in}{3.427208in}}%
\pgfusepath{clip}%
\pgfsetrectcap%
\pgfsetroundjoin%
\pgfsetlinewidth{1.505625pt}%
\definecolor{currentstroke}{rgb}{0.501961,0.501961,0.501961}%
\pgfsetstrokecolor{currentstroke}%
\pgfsetstrokeopacity{0.500000}%
\pgfsetdash{}{0pt}%
\pgfpathmoveto{\pgfqpoint{4.326728in}{4.047444in}}%
\pgfusepath{stroke}%
\end{pgfscope}%
\begin{pgfscope}%
\pgfpathrectangle{\pgfqpoint{0.100000in}{2.413063in}}{\pgfqpoint{5.037500in}{3.427208in}}%
\pgfusepath{clip}%
\pgfsetbuttcap%
\pgfsetroundjoin%
\definecolor{currentfill}{rgb}{0.501961,0.501961,0.501961}%
\pgfsetfillcolor{currentfill}%
\pgfsetfillopacity{0.500000}%
\pgfsetlinewidth{0.250937pt}%
\definecolor{currentstroke}{rgb}{0.000000,0.000000,0.000000}%
\pgfsetstrokecolor{currentstroke}%
\pgfsetstrokeopacity{0.500000}%
\pgfsetdash{}{0pt}%
\pgfsys@defobject{currentmarker}{\pgfqpoint{-0.013889in}{-0.013889in}}{\pgfqpoint{0.013889in}{0.013889in}}{%
\pgfpathmoveto{\pgfqpoint{0.000000in}{-0.013889in}}%
\pgfpathcurveto{\pgfqpoint{0.003683in}{-0.013889in}}{\pgfqpoint{0.007216in}{-0.012425in}}{\pgfqpoint{0.009821in}{-0.009821in}}%
\pgfpathcurveto{\pgfqpoint{0.012425in}{-0.007216in}}{\pgfqpoint{0.013889in}{-0.003683in}}{\pgfqpoint{0.013889in}{0.000000in}}%
\pgfpathcurveto{\pgfqpoint{0.013889in}{0.003683in}}{\pgfqpoint{0.012425in}{0.007216in}}{\pgfqpoint{0.009821in}{0.009821in}}%
\pgfpathcurveto{\pgfqpoint{0.007216in}{0.012425in}}{\pgfqpoint{0.003683in}{0.013889in}}{\pgfqpoint{0.000000in}{0.013889in}}%
\pgfpathcurveto{\pgfqpoint{-0.003683in}{0.013889in}}{\pgfqpoint{-0.007216in}{0.012425in}}{\pgfqpoint{-0.009821in}{0.009821in}}%
\pgfpathcurveto{\pgfqpoint{-0.012425in}{0.007216in}}{\pgfqpoint{-0.013889in}{0.003683in}}{\pgfqpoint{-0.013889in}{0.000000in}}%
\pgfpathcurveto{\pgfqpoint{-0.013889in}{-0.003683in}}{\pgfqpoint{-0.012425in}{-0.007216in}}{\pgfqpoint{-0.009821in}{-0.009821in}}%
\pgfpathcurveto{\pgfqpoint{-0.007216in}{-0.012425in}}{\pgfqpoint{-0.003683in}{-0.013889in}}{\pgfqpoint{0.000000in}{-0.013889in}}%
\pgfpathclose%
\pgfusepath{stroke,fill}%
}%
\begin{pgfscope}%
\pgfsys@transformshift{4.326728in}{4.047444in}%
\pgfsys@useobject{currentmarker}{}%
\end{pgfscope}%
\end{pgfscope}%
\begin{pgfscope}%
\pgfpathrectangle{\pgfqpoint{0.100000in}{2.413063in}}{\pgfqpoint{5.037500in}{3.427208in}}%
\pgfusepath{clip}%
\pgfsetrectcap%
\pgfsetroundjoin%
\pgfsetlinewidth{1.505625pt}%
\definecolor{currentstroke}{rgb}{0.000000,0.000000,1.000000}%
\pgfsetstrokecolor{currentstroke}%
\pgfsetstrokeopacity{0.500000}%
\pgfsetdash{}{0pt}%
\pgfpathmoveto{\pgfqpoint{4.400786in}{4.080282in}}%
\pgfusepath{stroke}%
\end{pgfscope}%
\begin{pgfscope}%
\pgfpathrectangle{\pgfqpoint{0.100000in}{2.413063in}}{\pgfqpoint{5.037500in}{3.427208in}}%
\pgfusepath{clip}%
\pgfsetbuttcap%
\pgfsetroundjoin%
\definecolor{currentfill}{rgb}{0.000000,0.000000,1.000000}%
\pgfsetfillcolor{currentfill}%
\pgfsetfillopacity{0.500000}%
\pgfsetlinewidth{0.250937pt}%
\definecolor{currentstroke}{rgb}{0.000000,0.000000,0.000000}%
\pgfsetstrokecolor{currentstroke}%
\pgfsetstrokeopacity{0.500000}%
\pgfsetdash{}{0pt}%
\pgfsys@defobject{currentmarker}{\pgfqpoint{-0.025000in}{-0.025000in}}{\pgfqpoint{0.025000in}{0.025000in}}{%
\pgfpathmoveto{\pgfqpoint{0.000000in}{-0.025000in}}%
\pgfpathcurveto{\pgfqpoint{0.006630in}{-0.025000in}}{\pgfqpoint{0.012989in}{-0.022366in}}{\pgfqpoint{0.017678in}{-0.017678in}}%
\pgfpathcurveto{\pgfqpoint{0.022366in}{-0.012989in}}{\pgfqpoint{0.025000in}{-0.006630in}}{\pgfqpoint{0.025000in}{0.000000in}}%
\pgfpathcurveto{\pgfqpoint{0.025000in}{0.006630in}}{\pgfqpoint{0.022366in}{0.012989in}}{\pgfqpoint{0.017678in}{0.017678in}}%
\pgfpathcurveto{\pgfqpoint{0.012989in}{0.022366in}}{\pgfqpoint{0.006630in}{0.025000in}}{\pgfqpoint{0.000000in}{0.025000in}}%
\pgfpathcurveto{\pgfqpoint{-0.006630in}{0.025000in}}{\pgfqpoint{-0.012989in}{0.022366in}}{\pgfqpoint{-0.017678in}{0.017678in}}%
\pgfpathcurveto{\pgfqpoint{-0.022366in}{0.012989in}}{\pgfqpoint{-0.025000in}{0.006630in}}{\pgfqpoint{-0.025000in}{0.000000in}}%
\pgfpathcurveto{\pgfqpoint{-0.025000in}{-0.006630in}}{\pgfqpoint{-0.022366in}{-0.012989in}}{\pgfqpoint{-0.017678in}{-0.017678in}}%
\pgfpathcurveto{\pgfqpoint{-0.012989in}{-0.022366in}}{\pgfqpoint{-0.006630in}{-0.025000in}}{\pgfqpoint{0.000000in}{-0.025000in}}%
\pgfpathclose%
\pgfusepath{stroke,fill}%
}%
\begin{pgfscope}%
\pgfsys@transformshift{4.400786in}{4.080282in}%
\pgfsys@useobject{currentmarker}{}%
\end{pgfscope}%
\end{pgfscope}%
\begin{pgfscope}%
\pgfpathrectangle{\pgfqpoint{0.100000in}{2.413063in}}{\pgfqpoint{5.037500in}{3.427208in}}%
\pgfusepath{clip}%
\pgfsetrectcap%
\pgfsetroundjoin%
\pgfsetlinewidth{1.505625pt}%
\definecolor{currentstroke}{rgb}{0.678431,1.000000,0.184314}%
\pgfsetstrokecolor{currentstroke}%
\pgfsetstrokeopacity{0.500000}%
\pgfsetdash{}{0pt}%
\pgfpathmoveto{\pgfqpoint{4.424090in}{3.883005in}}%
\pgfusepath{stroke}%
\end{pgfscope}%
\begin{pgfscope}%
\pgfpathrectangle{\pgfqpoint{0.100000in}{2.413063in}}{\pgfqpoint{5.037500in}{3.427208in}}%
\pgfusepath{clip}%
\pgfsetbuttcap%
\pgfsetroundjoin%
\definecolor{currentfill}{rgb}{0.678431,1.000000,0.184314}%
\pgfsetfillcolor{currentfill}%
\pgfsetfillopacity{0.500000}%
\pgfsetlinewidth{0.250937pt}%
\definecolor{currentstroke}{rgb}{0.000000,0.000000,0.000000}%
\pgfsetstrokecolor{currentstroke}%
\pgfsetstrokeopacity{0.500000}%
\pgfsetdash{}{0pt}%
\pgfsys@defobject{currentmarker}{\pgfqpoint{-0.008333in}{-0.008333in}}{\pgfqpoint{0.008333in}{0.008333in}}{%
\pgfpathmoveto{\pgfqpoint{0.000000in}{-0.008333in}}%
\pgfpathcurveto{\pgfqpoint{0.002210in}{-0.008333in}}{\pgfqpoint{0.004330in}{-0.007455in}}{\pgfqpoint{0.005893in}{-0.005893in}}%
\pgfpathcurveto{\pgfqpoint{0.007455in}{-0.004330in}}{\pgfqpoint{0.008333in}{-0.002210in}}{\pgfqpoint{0.008333in}{0.000000in}}%
\pgfpathcurveto{\pgfqpoint{0.008333in}{0.002210in}}{\pgfqpoint{0.007455in}{0.004330in}}{\pgfqpoint{0.005893in}{0.005893in}}%
\pgfpathcurveto{\pgfqpoint{0.004330in}{0.007455in}}{\pgfqpoint{0.002210in}{0.008333in}}{\pgfqpoint{0.000000in}{0.008333in}}%
\pgfpathcurveto{\pgfqpoint{-0.002210in}{0.008333in}}{\pgfqpoint{-0.004330in}{0.007455in}}{\pgfqpoint{-0.005893in}{0.005893in}}%
\pgfpathcurveto{\pgfqpoint{-0.007455in}{0.004330in}}{\pgfqpoint{-0.008333in}{0.002210in}}{\pgfqpoint{-0.008333in}{0.000000in}}%
\pgfpathcurveto{\pgfqpoint{-0.008333in}{-0.002210in}}{\pgfqpoint{-0.007455in}{-0.004330in}}{\pgfqpoint{-0.005893in}{-0.005893in}}%
\pgfpathcurveto{\pgfqpoint{-0.004330in}{-0.007455in}}{\pgfqpoint{-0.002210in}{-0.008333in}}{\pgfqpoint{0.000000in}{-0.008333in}}%
\pgfpathclose%
\pgfusepath{stroke,fill}%
}%
\begin{pgfscope}%
\pgfsys@transformshift{4.424090in}{3.883005in}%
\pgfsys@useobject{currentmarker}{}%
\end{pgfscope}%
\end{pgfscope}%
\begin{pgfscope}%
\pgfpathrectangle{\pgfqpoint{0.100000in}{2.413063in}}{\pgfqpoint{5.037500in}{3.427208in}}%
\pgfusepath{clip}%
\pgfsetrectcap%
\pgfsetroundjoin%
\pgfsetlinewidth{1.505625pt}%
\definecolor{currentstroke}{rgb}{0.501961,0.501961,0.501961}%
\pgfsetstrokecolor{currentstroke}%
\pgfsetstrokeopacity{0.500000}%
\pgfsetdash{}{0pt}%
\pgfpathmoveto{\pgfqpoint{4.172799in}{4.058209in}}%
\pgfusepath{stroke}%
\end{pgfscope}%
\begin{pgfscope}%
\pgfpathrectangle{\pgfqpoint{0.100000in}{2.413063in}}{\pgfqpoint{5.037500in}{3.427208in}}%
\pgfusepath{clip}%
\pgfsetbuttcap%
\pgfsetroundjoin%
\definecolor{currentfill}{rgb}{0.501961,0.501961,0.501961}%
\pgfsetfillcolor{currentfill}%
\pgfsetfillopacity{0.500000}%
\pgfsetlinewidth{0.250937pt}%
\definecolor{currentstroke}{rgb}{0.000000,0.000000,0.000000}%
\pgfsetstrokecolor{currentstroke}%
\pgfsetstrokeopacity{0.500000}%
\pgfsetdash{}{0pt}%
\pgfsys@defobject{currentmarker}{\pgfqpoint{-0.013889in}{-0.013889in}}{\pgfqpoint{0.013889in}{0.013889in}}{%
\pgfpathmoveto{\pgfqpoint{0.000000in}{-0.013889in}}%
\pgfpathcurveto{\pgfqpoint{0.003683in}{-0.013889in}}{\pgfqpoint{0.007216in}{-0.012425in}}{\pgfqpoint{0.009821in}{-0.009821in}}%
\pgfpathcurveto{\pgfqpoint{0.012425in}{-0.007216in}}{\pgfqpoint{0.013889in}{-0.003683in}}{\pgfqpoint{0.013889in}{0.000000in}}%
\pgfpathcurveto{\pgfqpoint{0.013889in}{0.003683in}}{\pgfqpoint{0.012425in}{0.007216in}}{\pgfqpoint{0.009821in}{0.009821in}}%
\pgfpathcurveto{\pgfqpoint{0.007216in}{0.012425in}}{\pgfqpoint{0.003683in}{0.013889in}}{\pgfqpoint{0.000000in}{0.013889in}}%
\pgfpathcurveto{\pgfqpoint{-0.003683in}{0.013889in}}{\pgfqpoint{-0.007216in}{0.012425in}}{\pgfqpoint{-0.009821in}{0.009821in}}%
\pgfpathcurveto{\pgfqpoint{-0.012425in}{0.007216in}}{\pgfqpoint{-0.013889in}{0.003683in}}{\pgfqpoint{-0.013889in}{0.000000in}}%
\pgfpathcurveto{\pgfqpoint{-0.013889in}{-0.003683in}}{\pgfqpoint{-0.012425in}{-0.007216in}}{\pgfqpoint{-0.009821in}{-0.009821in}}%
\pgfpathcurveto{\pgfqpoint{-0.007216in}{-0.012425in}}{\pgfqpoint{-0.003683in}{-0.013889in}}{\pgfqpoint{0.000000in}{-0.013889in}}%
\pgfpathclose%
\pgfusepath{stroke,fill}%
}%
\begin{pgfscope}%
\pgfsys@transformshift{4.172799in}{4.058209in}%
\pgfsys@useobject{currentmarker}{}%
\end{pgfscope}%
\end{pgfscope}%
\begin{pgfscope}%
\pgfpathrectangle{\pgfqpoint{0.100000in}{2.413063in}}{\pgfqpoint{5.037500in}{3.427208in}}%
\pgfusepath{clip}%
\pgfsetrectcap%
\pgfsetroundjoin%
\pgfsetlinewidth{1.505625pt}%
\definecolor{currentstroke}{rgb}{0.000000,0.000000,1.000000}%
\pgfsetstrokecolor{currentstroke}%
\pgfsetstrokeopacity{0.500000}%
\pgfsetdash{}{0pt}%
\pgfpathmoveto{\pgfqpoint{2.344274in}{5.197479in}}%
\pgfusepath{stroke}%
\end{pgfscope}%
\begin{pgfscope}%
\pgfpathrectangle{\pgfqpoint{0.100000in}{2.413063in}}{\pgfqpoint{5.037500in}{3.427208in}}%
\pgfusepath{clip}%
\pgfsetbuttcap%
\pgfsetroundjoin%
\definecolor{currentfill}{rgb}{0.000000,0.000000,1.000000}%
\pgfsetfillcolor{currentfill}%
\pgfsetfillopacity{0.500000}%
\pgfsetlinewidth{0.250937pt}%
\definecolor{currentstroke}{rgb}{0.000000,0.000000,0.000000}%
\pgfsetstrokecolor{currentstroke}%
\pgfsetstrokeopacity{0.500000}%
\pgfsetdash{}{0pt}%
\pgfsys@defobject{currentmarker}{\pgfqpoint{-0.008333in}{-0.008333in}}{\pgfqpoint{0.008333in}{0.008333in}}{%
\pgfpathmoveto{\pgfqpoint{0.000000in}{-0.008333in}}%
\pgfpathcurveto{\pgfqpoint{0.002210in}{-0.008333in}}{\pgfqpoint{0.004330in}{-0.007455in}}{\pgfqpoint{0.005893in}{-0.005893in}}%
\pgfpathcurveto{\pgfqpoint{0.007455in}{-0.004330in}}{\pgfqpoint{0.008333in}{-0.002210in}}{\pgfqpoint{0.008333in}{0.000000in}}%
\pgfpathcurveto{\pgfqpoint{0.008333in}{0.002210in}}{\pgfqpoint{0.007455in}{0.004330in}}{\pgfqpoint{0.005893in}{0.005893in}}%
\pgfpathcurveto{\pgfqpoint{0.004330in}{0.007455in}}{\pgfqpoint{0.002210in}{0.008333in}}{\pgfqpoint{0.000000in}{0.008333in}}%
\pgfpathcurveto{\pgfqpoint{-0.002210in}{0.008333in}}{\pgfqpoint{-0.004330in}{0.007455in}}{\pgfqpoint{-0.005893in}{0.005893in}}%
\pgfpathcurveto{\pgfqpoint{-0.007455in}{0.004330in}}{\pgfqpoint{-0.008333in}{0.002210in}}{\pgfqpoint{-0.008333in}{0.000000in}}%
\pgfpathcurveto{\pgfqpoint{-0.008333in}{-0.002210in}}{\pgfqpoint{-0.007455in}{-0.004330in}}{\pgfqpoint{-0.005893in}{-0.005893in}}%
\pgfpathcurveto{\pgfqpoint{-0.004330in}{-0.007455in}}{\pgfqpoint{-0.002210in}{-0.008333in}}{\pgfqpoint{0.000000in}{-0.008333in}}%
\pgfpathclose%
\pgfusepath{stroke,fill}%
}%
\begin{pgfscope}%
\pgfsys@transformshift{2.344274in}{5.197479in}%
\pgfsys@useobject{currentmarker}{}%
\end{pgfscope}%
\end{pgfscope}%
\begin{pgfscope}%
\pgfpathrectangle{\pgfqpoint{0.100000in}{2.413063in}}{\pgfqpoint{5.037500in}{3.427208in}}%
\pgfusepath{clip}%
\pgfsetrectcap%
\pgfsetroundjoin%
\pgfsetlinewidth{1.505625pt}%
\definecolor{currentstroke}{rgb}{0.501961,0.501961,0.501961}%
\pgfsetstrokecolor{currentstroke}%
\pgfsetstrokeopacity{0.500000}%
\pgfsetdash{}{0pt}%
\pgfpathmoveto{\pgfqpoint{2.661894in}{5.192252in}}%
\pgfusepath{stroke}%
\end{pgfscope}%
\begin{pgfscope}%
\pgfpathrectangle{\pgfqpoint{0.100000in}{2.413063in}}{\pgfqpoint{5.037500in}{3.427208in}}%
\pgfusepath{clip}%
\pgfsetbuttcap%
\pgfsetroundjoin%
\definecolor{currentfill}{rgb}{0.501961,0.501961,0.501961}%
\pgfsetfillcolor{currentfill}%
\pgfsetfillopacity{0.500000}%
\pgfsetlinewidth{0.250937pt}%
\definecolor{currentstroke}{rgb}{0.000000,0.000000,0.000000}%
\pgfsetstrokecolor{currentstroke}%
\pgfsetstrokeopacity{0.500000}%
\pgfsetdash{}{0pt}%
\pgfsys@defobject{currentmarker}{\pgfqpoint{-0.013889in}{-0.013889in}}{\pgfqpoint{0.013889in}{0.013889in}}{%
\pgfpathmoveto{\pgfqpoint{0.000000in}{-0.013889in}}%
\pgfpathcurveto{\pgfqpoint{0.003683in}{-0.013889in}}{\pgfqpoint{0.007216in}{-0.012425in}}{\pgfqpoint{0.009821in}{-0.009821in}}%
\pgfpathcurveto{\pgfqpoint{0.012425in}{-0.007216in}}{\pgfqpoint{0.013889in}{-0.003683in}}{\pgfqpoint{0.013889in}{0.000000in}}%
\pgfpathcurveto{\pgfqpoint{0.013889in}{0.003683in}}{\pgfqpoint{0.012425in}{0.007216in}}{\pgfqpoint{0.009821in}{0.009821in}}%
\pgfpathcurveto{\pgfqpoint{0.007216in}{0.012425in}}{\pgfqpoint{0.003683in}{0.013889in}}{\pgfqpoint{0.000000in}{0.013889in}}%
\pgfpathcurveto{\pgfqpoint{-0.003683in}{0.013889in}}{\pgfqpoint{-0.007216in}{0.012425in}}{\pgfqpoint{-0.009821in}{0.009821in}}%
\pgfpathcurveto{\pgfqpoint{-0.012425in}{0.007216in}}{\pgfqpoint{-0.013889in}{0.003683in}}{\pgfqpoint{-0.013889in}{0.000000in}}%
\pgfpathcurveto{\pgfqpoint{-0.013889in}{-0.003683in}}{\pgfqpoint{-0.012425in}{-0.007216in}}{\pgfqpoint{-0.009821in}{-0.009821in}}%
\pgfpathcurveto{\pgfqpoint{-0.007216in}{-0.012425in}}{\pgfqpoint{-0.003683in}{-0.013889in}}{\pgfqpoint{0.000000in}{-0.013889in}}%
\pgfpathclose%
\pgfusepath{stroke,fill}%
}%
\begin{pgfscope}%
\pgfsys@transformshift{2.661894in}{5.192252in}%
\pgfsys@useobject{currentmarker}{}%
\end{pgfscope}%
\end{pgfscope}%
\begin{pgfscope}%
\pgfpathrectangle{\pgfqpoint{0.100000in}{2.413063in}}{\pgfqpoint{5.037500in}{3.427208in}}%
\pgfusepath{clip}%
\pgfsetrectcap%
\pgfsetroundjoin%
\pgfsetlinewidth{1.505625pt}%
\definecolor{currentstroke}{rgb}{0.000000,0.000000,1.000000}%
\pgfsetstrokecolor{currentstroke}%
\pgfsetstrokeopacity{0.500000}%
\pgfsetdash{}{0pt}%
\pgfpathmoveto{\pgfqpoint{2.643208in}{5.312659in}}%
\pgfusepath{stroke}%
\end{pgfscope}%
\begin{pgfscope}%
\pgfpathrectangle{\pgfqpoint{0.100000in}{2.413063in}}{\pgfqpoint{5.037500in}{3.427208in}}%
\pgfusepath{clip}%
\pgfsetbuttcap%
\pgfsetroundjoin%
\definecolor{currentfill}{rgb}{0.000000,0.000000,1.000000}%
\pgfsetfillcolor{currentfill}%
\pgfsetfillopacity{0.500000}%
\pgfsetlinewidth{0.250937pt}%
\definecolor{currentstroke}{rgb}{0.000000,0.000000,0.000000}%
\pgfsetstrokecolor{currentstroke}%
\pgfsetstrokeopacity{0.500000}%
\pgfsetdash{}{0pt}%
\pgfsys@defobject{currentmarker}{\pgfqpoint{-0.005556in}{-0.005556in}}{\pgfqpoint{0.005556in}{0.005556in}}{%
\pgfpathmoveto{\pgfqpoint{0.000000in}{-0.005556in}}%
\pgfpathcurveto{\pgfqpoint{0.001473in}{-0.005556in}}{\pgfqpoint{0.002887in}{-0.004970in}}{\pgfqpoint{0.003928in}{-0.003928in}}%
\pgfpathcurveto{\pgfqpoint{0.004970in}{-0.002887in}}{\pgfqpoint{0.005556in}{-0.001473in}}{\pgfqpoint{0.005556in}{0.000000in}}%
\pgfpathcurveto{\pgfqpoint{0.005556in}{0.001473in}}{\pgfqpoint{0.004970in}{0.002887in}}{\pgfqpoint{0.003928in}{0.003928in}}%
\pgfpathcurveto{\pgfqpoint{0.002887in}{0.004970in}}{\pgfqpoint{0.001473in}{0.005556in}}{\pgfqpoint{0.000000in}{0.005556in}}%
\pgfpathcurveto{\pgfqpoint{-0.001473in}{0.005556in}}{\pgfqpoint{-0.002887in}{0.004970in}}{\pgfqpoint{-0.003928in}{0.003928in}}%
\pgfpathcurveto{\pgfqpoint{-0.004970in}{0.002887in}}{\pgfqpoint{-0.005556in}{0.001473in}}{\pgfqpoint{-0.005556in}{0.000000in}}%
\pgfpathcurveto{\pgfqpoint{-0.005556in}{-0.001473in}}{\pgfqpoint{-0.004970in}{-0.002887in}}{\pgfqpoint{-0.003928in}{-0.003928in}}%
\pgfpathcurveto{\pgfqpoint{-0.002887in}{-0.004970in}}{\pgfqpoint{-0.001473in}{-0.005556in}}{\pgfqpoint{0.000000in}{-0.005556in}}%
\pgfpathclose%
\pgfusepath{stroke,fill}%
}%
\begin{pgfscope}%
\pgfsys@transformshift{2.643208in}{5.312659in}%
\pgfsys@useobject{currentmarker}{}%
\end{pgfscope}%
\end{pgfscope}%
\begin{pgfscope}%
\pgfpathrectangle{\pgfqpoint{0.100000in}{2.413063in}}{\pgfqpoint{5.037500in}{3.427208in}}%
\pgfusepath{clip}%
\pgfsetrectcap%
\pgfsetroundjoin%
\pgfsetlinewidth{1.505625pt}%
\definecolor{currentstroke}{rgb}{0.678431,1.000000,0.184314}%
\pgfsetstrokecolor{currentstroke}%
\pgfsetstrokeopacity{0.500000}%
\pgfsetdash{}{0pt}%
\pgfpathmoveto{\pgfqpoint{3.970658in}{4.607541in}}%
\pgfusepath{stroke}%
\end{pgfscope}%
\begin{pgfscope}%
\pgfpathrectangle{\pgfqpoint{0.100000in}{2.413063in}}{\pgfqpoint{5.037500in}{3.427208in}}%
\pgfusepath{clip}%
\pgfsetbuttcap%
\pgfsetroundjoin%
\definecolor{currentfill}{rgb}{0.678431,1.000000,0.184314}%
\pgfsetfillcolor{currentfill}%
\pgfsetfillopacity{0.500000}%
\pgfsetlinewidth{0.250937pt}%
\definecolor{currentstroke}{rgb}{0.000000,0.000000,0.000000}%
\pgfsetstrokecolor{currentstroke}%
\pgfsetstrokeopacity{0.500000}%
\pgfsetdash{}{0pt}%
\pgfsys@defobject{currentmarker}{\pgfqpoint{-0.019444in}{-0.019444in}}{\pgfqpoint{0.019444in}{0.019444in}}{%
\pgfpathmoveto{\pgfqpoint{0.000000in}{-0.019444in}}%
\pgfpathcurveto{\pgfqpoint{0.005157in}{-0.019444in}}{\pgfqpoint{0.010103in}{-0.017396in}}{\pgfqpoint{0.013749in}{-0.013749in}}%
\pgfpathcurveto{\pgfqpoint{0.017396in}{-0.010103in}}{\pgfqpoint{0.019444in}{-0.005157in}}{\pgfqpoint{0.019444in}{0.000000in}}%
\pgfpathcurveto{\pgfqpoint{0.019444in}{0.005157in}}{\pgfqpoint{0.017396in}{0.010103in}}{\pgfqpoint{0.013749in}{0.013749in}}%
\pgfpathcurveto{\pgfqpoint{0.010103in}{0.017396in}}{\pgfqpoint{0.005157in}{0.019444in}}{\pgfqpoint{0.000000in}{0.019444in}}%
\pgfpathcurveto{\pgfqpoint{-0.005157in}{0.019444in}}{\pgfqpoint{-0.010103in}{0.017396in}}{\pgfqpoint{-0.013749in}{0.013749in}}%
\pgfpathcurveto{\pgfqpoint{-0.017396in}{0.010103in}}{\pgfqpoint{-0.019444in}{0.005157in}}{\pgfqpoint{-0.019444in}{0.000000in}}%
\pgfpathcurveto{\pgfqpoint{-0.019444in}{-0.005157in}}{\pgfqpoint{-0.017396in}{-0.010103in}}{\pgfqpoint{-0.013749in}{-0.013749in}}%
\pgfpathcurveto{\pgfqpoint{-0.010103in}{-0.017396in}}{\pgfqpoint{-0.005157in}{-0.019444in}}{\pgfqpoint{0.000000in}{-0.019444in}}%
\pgfpathclose%
\pgfusepath{stroke,fill}%
}%
\begin{pgfscope}%
\pgfsys@transformshift{3.970658in}{4.607541in}%
\pgfsys@useobject{currentmarker}{}%
\end{pgfscope}%
\end{pgfscope}%
\begin{pgfscope}%
\pgfpathrectangle{\pgfqpoint{0.100000in}{2.413063in}}{\pgfqpoint{5.037500in}{3.427208in}}%
\pgfusepath{clip}%
\pgfsetrectcap%
\pgfsetroundjoin%
\pgfsetlinewidth{1.505625pt}%
\definecolor{currentstroke}{rgb}{0.678431,1.000000,0.184314}%
\pgfsetstrokecolor{currentstroke}%
\pgfsetstrokeopacity{0.500000}%
\pgfsetdash{}{0pt}%
\pgfpathmoveto{\pgfqpoint{3.987573in}{4.576983in}}%
\pgfusepath{stroke}%
\end{pgfscope}%
\begin{pgfscope}%
\pgfpathrectangle{\pgfqpoint{0.100000in}{2.413063in}}{\pgfqpoint{5.037500in}{3.427208in}}%
\pgfusepath{clip}%
\pgfsetbuttcap%
\pgfsetroundjoin%
\definecolor{currentfill}{rgb}{0.678431,1.000000,0.184314}%
\pgfsetfillcolor{currentfill}%
\pgfsetfillopacity{0.500000}%
\pgfsetlinewidth{0.250937pt}%
\definecolor{currentstroke}{rgb}{0.000000,0.000000,0.000000}%
\pgfsetstrokecolor{currentstroke}%
\pgfsetstrokeopacity{0.500000}%
\pgfsetdash{}{0pt}%
\pgfsys@defobject{currentmarker}{\pgfqpoint{-0.030556in}{-0.030556in}}{\pgfqpoint{0.030556in}{0.030556in}}{%
\pgfpathmoveto{\pgfqpoint{0.000000in}{-0.030556in}}%
\pgfpathcurveto{\pgfqpoint{0.008103in}{-0.030556in}}{\pgfqpoint{0.015876in}{-0.027336in}}{\pgfqpoint{0.021606in}{-0.021606in}}%
\pgfpathcurveto{\pgfqpoint{0.027336in}{-0.015876in}}{\pgfqpoint{0.030556in}{-0.008103in}}{\pgfqpoint{0.030556in}{0.000000in}}%
\pgfpathcurveto{\pgfqpoint{0.030556in}{0.008103in}}{\pgfqpoint{0.027336in}{0.015876in}}{\pgfqpoint{0.021606in}{0.021606in}}%
\pgfpathcurveto{\pgfqpoint{0.015876in}{0.027336in}}{\pgfqpoint{0.008103in}{0.030556in}}{\pgfqpoint{0.000000in}{0.030556in}}%
\pgfpathcurveto{\pgfqpoint{-0.008103in}{0.030556in}}{\pgfqpoint{-0.015876in}{0.027336in}}{\pgfqpoint{-0.021606in}{0.021606in}}%
\pgfpathcurveto{\pgfqpoint{-0.027336in}{0.015876in}}{\pgfqpoint{-0.030556in}{0.008103in}}{\pgfqpoint{-0.030556in}{0.000000in}}%
\pgfpathcurveto{\pgfqpoint{-0.030556in}{-0.008103in}}{\pgfqpoint{-0.027336in}{-0.015876in}}{\pgfqpoint{-0.021606in}{-0.021606in}}%
\pgfpathcurveto{\pgfqpoint{-0.015876in}{-0.027336in}}{\pgfqpoint{-0.008103in}{-0.030556in}}{\pgfqpoint{0.000000in}{-0.030556in}}%
\pgfpathclose%
\pgfusepath{stroke,fill}%
}%
\begin{pgfscope}%
\pgfsys@transformshift{3.987573in}{4.576983in}%
\pgfsys@useobject{currentmarker}{}%
\end{pgfscope}%
\end{pgfscope}%
\begin{pgfscope}%
\pgfpathrectangle{\pgfqpoint{0.100000in}{2.413063in}}{\pgfqpoint{5.037500in}{3.427208in}}%
\pgfusepath{clip}%
\pgfsetrectcap%
\pgfsetroundjoin%
\pgfsetlinewidth{1.505625pt}%
\definecolor{currentstroke}{rgb}{0.678431,1.000000,0.184314}%
\pgfsetstrokecolor{currentstroke}%
\pgfsetstrokeopacity{0.500000}%
\pgfsetdash{}{0pt}%
\pgfpathmoveto{\pgfqpoint{3.739192in}{4.346646in}}%
\pgfusepath{stroke}%
\end{pgfscope}%
\begin{pgfscope}%
\pgfpathrectangle{\pgfqpoint{0.100000in}{2.413063in}}{\pgfqpoint{5.037500in}{3.427208in}}%
\pgfusepath{clip}%
\pgfsetbuttcap%
\pgfsetroundjoin%
\definecolor{currentfill}{rgb}{0.678431,1.000000,0.184314}%
\pgfsetfillcolor{currentfill}%
\pgfsetfillopacity{0.500000}%
\pgfsetlinewidth{0.250937pt}%
\definecolor{currentstroke}{rgb}{0.000000,0.000000,0.000000}%
\pgfsetstrokecolor{currentstroke}%
\pgfsetstrokeopacity{0.500000}%
\pgfsetdash{}{0pt}%
\pgfsys@defobject{currentmarker}{\pgfqpoint{-0.016667in}{-0.016667in}}{\pgfqpoint{0.016667in}{0.016667in}}{%
\pgfpathmoveto{\pgfqpoint{0.000000in}{-0.016667in}}%
\pgfpathcurveto{\pgfqpoint{0.004420in}{-0.016667in}}{\pgfqpoint{0.008660in}{-0.014911in}}{\pgfqpoint{0.011785in}{-0.011785in}}%
\pgfpathcurveto{\pgfqpoint{0.014911in}{-0.008660in}}{\pgfqpoint{0.016667in}{-0.004420in}}{\pgfqpoint{0.016667in}{0.000000in}}%
\pgfpathcurveto{\pgfqpoint{0.016667in}{0.004420in}}{\pgfqpoint{0.014911in}{0.008660in}}{\pgfqpoint{0.011785in}{0.011785in}}%
\pgfpathcurveto{\pgfqpoint{0.008660in}{0.014911in}}{\pgfqpoint{0.004420in}{0.016667in}}{\pgfqpoint{0.000000in}{0.016667in}}%
\pgfpathcurveto{\pgfqpoint{-0.004420in}{0.016667in}}{\pgfqpoint{-0.008660in}{0.014911in}}{\pgfqpoint{-0.011785in}{0.011785in}}%
\pgfpathcurveto{\pgfqpoint{-0.014911in}{0.008660in}}{\pgfqpoint{-0.016667in}{0.004420in}}{\pgfqpoint{-0.016667in}{0.000000in}}%
\pgfpathcurveto{\pgfqpoint{-0.016667in}{-0.004420in}}{\pgfqpoint{-0.014911in}{-0.008660in}}{\pgfqpoint{-0.011785in}{-0.011785in}}%
\pgfpathcurveto{\pgfqpoint{-0.008660in}{-0.014911in}}{\pgfqpoint{-0.004420in}{-0.016667in}}{\pgfqpoint{0.000000in}{-0.016667in}}%
\pgfpathclose%
\pgfusepath{stroke,fill}%
}%
\begin{pgfscope}%
\pgfsys@transformshift{3.739192in}{4.346646in}%
\pgfsys@useobject{currentmarker}{}%
\end{pgfscope}%
\end{pgfscope}%
\begin{pgfscope}%
\pgfpathrectangle{\pgfqpoint{0.100000in}{2.413063in}}{\pgfqpoint{5.037500in}{3.427208in}}%
\pgfusepath{clip}%
\pgfsetrectcap%
\pgfsetroundjoin%
\pgfsetlinewidth{1.505625pt}%
\definecolor{currentstroke}{rgb}{0.000000,0.000000,1.000000}%
\pgfsetstrokecolor{currentstroke}%
\pgfsetstrokeopacity{0.500000}%
\pgfsetdash{}{0pt}%
\pgfpathmoveto{\pgfqpoint{3.948532in}{4.653429in}}%
\pgfusepath{stroke}%
\end{pgfscope}%
\begin{pgfscope}%
\pgfpathrectangle{\pgfqpoint{0.100000in}{2.413063in}}{\pgfqpoint{5.037500in}{3.427208in}}%
\pgfusepath{clip}%
\pgfsetbuttcap%
\pgfsetroundjoin%
\definecolor{currentfill}{rgb}{0.000000,0.000000,1.000000}%
\pgfsetfillcolor{currentfill}%
\pgfsetfillopacity{0.500000}%
\pgfsetlinewidth{0.250937pt}%
\definecolor{currentstroke}{rgb}{0.000000,0.000000,0.000000}%
\pgfsetstrokecolor{currentstroke}%
\pgfsetstrokeopacity{0.500000}%
\pgfsetdash{}{0pt}%
\pgfsys@defobject{currentmarker}{\pgfqpoint{-0.019444in}{-0.019444in}}{\pgfqpoint{0.019444in}{0.019444in}}{%
\pgfpathmoveto{\pgfqpoint{0.000000in}{-0.019444in}}%
\pgfpathcurveto{\pgfqpoint{0.005157in}{-0.019444in}}{\pgfqpoint{0.010103in}{-0.017396in}}{\pgfqpoint{0.013749in}{-0.013749in}}%
\pgfpathcurveto{\pgfqpoint{0.017396in}{-0.010103in}}{\pgfqpoint{0.019444in}{-0.005157in}}{\pgfqpoint{0.019444in}{0.000000in}}%
\pgfpathcurveto{\pgfqpoint{0.019444in}{0.005157in}}{\pgfqpoint{0.017396in}{0.010103in}}{\pgfqpoint{0.013749in}{0.013749in}}%
\pgfpathcurveto{\pgfqpoint{0.010103in}{0.017396in}}{\pgfqpoint{0.005157in}{0.019444in}}{\pgfqpoint{0.000000in}{0.019444in}}%
\pgfpathcurveto{\pgfqpoint{-0.005157in}{0.019444in}}{\pgfqpoint{-0.010103in}{0.017396in}}{\pgfqpoint{-0.013749in}{0.013749in}}%
\pgfpathcurveto{\pgfqpoint{-0.017396in}{0.010103in}}{\pgfqpoint{-0.019444in}{0.005157in}}{\pgfqpoint{-0.019444in}{0.000000in}}%
\pgfpathcurveto{\pgfqpoint{-0.019444in}{-0.005157in}}{\pgfqpoint{-0.017396in}{-0.010103in}}{\pgfqpoint{-0.013749in}{-0.013749in}}%
\pgfpathcurveto{\pgfqpoint{-0.010103in}{-0.017396in}}{\pgfqpoint{-0.005157in}{-0.019444in}}{\pgfqpoint{0.000000in}{-0.019444in}}%
\pgfpathclose%
\pgfusepath{stroke,fill}%
}%
\begin{pgfscope}%
\pgfsys@transformshift{3.948532in}{4.653429in}%
\pgfsys@useobject{currentmarker}{}%
\end{pgfscope}%
\end{pgfscope}%
\begin{pgfscope}%
\pgfpathrectangle{\pgfqpoint{0.100000in}{2.413063in}}{\pgfqpoint{5.037500in}{3.427208in}}%
\pgfusepath{clip}%
\pgfsetrectcap%
\pgfsetroundjoin%
\pgfsetlinewidth{1.505625pt}%
\definecolor{currentstroke}{rgb}{0.678431,1.000000,0.184314}%
\pgfsetstrokecolor{currentstroke}%
\pgfsetstrokeopacity{0.500000}%
\pgfsetdash{}{0pt}%
\pgfpathmoveto{\pgfqpoint{3.860194in}{4.461577in}}%
\pgfusepath{stroke}%
\end{pgfscope}%
\begin{pgfscope}%
\pgfpathrectangle{\pgfqpoint{0.100000in}{2.413063in}}{\pgfqpoint{5.037500in}{3.427208in}}%
\pgfusepath{clip}%
\pgfsetbuttcap%
\pgfsetroundjoin%
\definecolor{currentfill}{rgb}{0.678431,1.000000,0.184314}%
\pgfsetfillcolor{currentfill}%
\pgfsetfillopacity{0.500000}%
\pgfsetlinewidth{0.250937pt}%
\definecolor{currentstroke}{rgb}{0.000000,0.000000,0.000000}%
\pgfsetstrokecolor{currentstroke}%
\pgfsetstrokeopacity{0.500000}%
\pgfsetdash{}{0pt}%
\pgfsys@defobject{currentmarker}{\pgfqpoint{-0.016667in}{-0.016667in}}{\pgfqpoint{0.016667in}{0.016667in}}{%
\pgfpathmoveto{\pgfqpoint{0.000000in}{-0.016667in}}%
\pgfpathcurveto{\pgfqpoint{0.004420in}{-0.016667in}}{\pgfqpoint{0.008660in}{-0.014911in}}{\pgfqpoint{0.011785in}{-0.011785in}}%
\pgfpathcurveto{\pgfqpoint{0.014911in}{-0.008660in}}{\pgfqpoint{0.016667in}{-0.004420in}}{\pgfqpoint{0.016667in}{0.000000in}}%
\pgfpathcurveto{\pgfqpoint{0.016667in}{0.004420in}}{\pgfqpoint{0.014911in}{0.008660in}}{\pgfqpoint{0.011785in}{0.011785in}}%
\pgfpathcurveto{\pgfqpoint{0.008660in}{0.014911in}}{\pgfqpoint{0.004420in}{0.016667in}}{\pgfqpoint{0.000000in}{0.016667in}}%
\pgfpathcurveto{\pgfqpoint{-0.004420in}{0.016667in}}{\pgfqpoint{-0.008660in}{0.014911in}}{\pgfqpoint{-0.011785in}{0.011785in}}%
\pgfpathcurveto{\pgfqpoint{-0.014911in}{0.008660in}}{\pgfqpoint{-0.016667in}{0.004420in}}{\pgfqpoint{-0.016667in}{0.000000in}}%
\pgfpathcurveto{\pgfqpoint{-0.016667in}{-0.004420in}}{\pgfqpoint{-0.014911in}{-0.008660in}}{\pgfqpoint{-0.011785in}{-0.011785in}}%
\pgfpathcurveto{\pgfqpoint{-0.008660in}{-0.014911in}}{\pgfqpoint{-0.004420in}{-0.016667in}}{\pgfqpoint{0.000000in}{-0.016667in}}%
\pgfpathclose%
\pgfusepath{stroke,fill}%
}%
\begin{pgfscope}%
\pgfsys@transformshift{3.860194in}{4.461577in}%
\pgfsys@useobject{currentmarker}{}%
\end{pgfscope}%
\end{pgfscope}%
\begin{pgfscope}%
\pgfpathrectangle{\pgfqpoint{0.100000in}{2.413063in}}{\pgfqpoint{5.037500in}{3.427208in}}%
\pgfusepath{clip}%
\pgfsetrectcap%
\pgfsetroundjoin%
\pgfsetlinewidth{1.505625pt}%
\definecolor{currentstroke}{rgb}{0.678431,1.000000,0.184314}%
\pgfsetstrokecolor{currentstroke}%
\pgfsetstrokeopacity{0.500000}%
\pgfsetdash{}{0pt}%
\pgfpathmoveto{\pgfqpoint{3.758682in}{4.425173in}}%
\pgfusepath{stroke}%
\end{pgfscope}%
\begin{pgfscope}%
\pgfpathrectangle{\pgfqpoint{0.100000in}{2.413063in}}{\pgfqpoint{5.037500in}{3.427208in}}%
\pgfusepath{clip}%
\pgfsetbuttcap%
\pgfsetroundjoin%
\definecolor{currentfill}{rgb}{0.678431,1.000000,0.184314}%
\pgfsetfillcolor{currentfill}%
\pgfsetfillopacity{0.500000}%
\pgfsetlinewidth{0.250937pt}%
\definecolor{currentstroke}{rgb}{0.000000,0.000000,0.000000}%
\pgfsetstrokecolor{currentstroke}%
\pgfsetstrokeopacity{0.500000}%
\pgfsetdash{}{0pt}%
\pgfsys@defobject{currentmarker}{\pgfqpoint{-0.016667in}{-0.016667in}}{\pgfqpoint{0.016667in}{0.016667in}}{%
\pgfpathmoveto{\pgfqpoint{0.000000in}{-0.016667in}}%
\pgfpathcurveto{\pgfqpoint{0.004420in}{-0.016667in}}{\pgfqpoint{0.008660in}{-0.014911in}}{\pgfqpoint{0.011785in}{-0.011785in}}%
\pgfpathcurveto{\pgfqpoint{0.014911in}{-0.008660in}}{\pgfqpoint{0.016667in}{-0.004420in}}{\pgfqpoint{0.016667in}{0.000000in}}%
\pgfpathcurveto{\pgfqpoint{0.016667in}{0.004420in}}{\pgfqpoint{0.014911in}{0.008660in}}{\pgfqpoint{0.011785in}{0.011785in}}%
\pgfpathcurveto{\pgfqpoint{0.008660in}{0.014911in}}{\pgfqpoint{0.004420in}{0.016667in}}{\pgfqpoint{0.000000in}{0.016667in}}%
\pgfpathcurveto{\pgfqpoint{-0.004420in}{0.016667in}}{\pgfqpoint{-0.008660in}{0.014911in}}{\pgfqpoint{-0.011785in}{0.011785in}}%
\pgfpathcurveto{\pgfqpoint{-0.014911in}{0.008660in}}{\pgfqpoint{-0.016667in}{0.004420in}}{\pgfqpoint{-0.016667in}{0.000000in}}%
\pgfpathcurveto{\pgfqpoint{-0.016667in}{-0.004420in}}{\pgfqpoint{-0.014911in}{-0.008660in}}{\pgfqpoint{-0.011785in}{-0.011785in}}%
\pgfpathcurveto{\pgfqpoint{-0.008660in}{-0.014911in}}{\pgfqpoint{-0.004420in}{-0.016667in}}{\pgfqpoint{0.000000in}{-0.016667in}}%
\pgfpathclose%
\pgfusepath{stroke,fill}%
}%
\begin{pgfscope}%
\pgfsys@transformshift{3.758682in}{4.425173in}%
\pgfsys@useobject{currentmarker}{}%
\end{pgfscope}%
\end{pgfscope}%
\begin{pgfscope}%
\pgfpathrectangle{\pgfqpoint{0.100000in}{2.413063in}}{\pgfqpoint{5.037500in}{3.427208in}}%
\pgfusepath{clip}%
\pgfsetrectcap%
\pgfsetroundjoin%
\pgfsetlinewidth{1.505625pt}%
\definecolor{currentstroke}{rgb}{0.678431,1.000000,0.184314}%
\pgfsetstrokecolor{currentstroke}%
\pgfsetstrokeopacity{0.500000}%
\pgfsetdash{}{0pt}%
\pgfpathmoveto{\pgfqpoint{3.752726in}{4.538555in}}%
\pgfusepath{stroke}%
\end{pgfscope}%
\begin{pgfscope}%
\pgfpathrectangle{\pgfqpoint{0.100000in}{2.413063in}}{\pgfqpoint{5.037500in}{3.427208in}}%
\pgfusepath{clip}%
\pgfsetbuttcap%
\pgfsetroundjoin%
\definecolor{currentfill}{rgb}{0.678431,1.000000,0.184314}%
\pgfsetfillcolor{currentfill}%
\pgfsetfillopacity{0.500000}%
\pgfsetlinewidth{0.250937pt}%
\definecolor{currentstroke}{rgb}{0.000000,0.000000,0.000000}%
\pgfsetstrokecolor{currentstroke}%
\pgfsetstrokeopacity{0.500000}%
\pgfsetdash{}{0pt}%
\pgfsys@defobject{currentmarker}{\pgfqpoint{-0.016667in}{-0.016667in}}{\pgfqpoint{0.016667in}{0.016667in}}{%
\pgfpathmoveto{\pgfqpoint{0.000000in}{-0.016667in}}%
\pgfpathcurveto{\pgfqpoint{0.004420in}{-0.016667in}}{\pgfqpoint{0.008660in}{-0.014911in}}{\pgfqpoint{0.011785in}{-0.011785in}}%
\pgfpathcurveto{\pgfqpoint{0.014911in}{-0.008660in}}{\pgfqpoint{0.016667in}{-0.004420in}}{\pgfqpoint{0.016667in}{0.000000in}}%
\pgfpathcurveto{\pgfqpoint{0.016667in}{0.004420in}}{\pgfqpoint{0.014911in}{0.008660in}}{\pgfqpoint{0.011785in}{0.011785in}}%
\pgfpathcurveto{\pgfqpoint{0.008660in}{0.014911in}}{\pgfqpoint{0.004420in}{0.016667in}}{\pgfqpoint{0.000000in}{0.016667in}}%
\pgfpathcurveto{\pgfqpoint{-0.004420in}{0.016667in}}{\pgfqpoint{-0.008660in}{0.014911in}}{\pgfqpoint{-0.011785in}{0.011785in}}%
\pgfpathcurveto{\pgfqpoint{-0.014911in}{0.008660in}}{\pgfqpoint{-0.016667in}{0.004420in}}{\pgfqpoint{-0.016667in}{0.000000in}}%
\pgfpathcurveto{\pgfqpoint{-0.016667in}{-0.004420in}}{\pgfqpoint{-0.014911in}{-0.008660in}}{\pgfqpoint{-0.011785in}{-0.011785in}}%
\pgfpathcurveto{\pgfqpoint{-0.008660in}{-0.014911in}}{\pgfqpoint{-0.004420in}{-0.016667in}}{\pgfqpoint{0.000000in}{-0.016667in}}%
\pgfpathclose%
\pgfusepath{stroke,fill}%
}%
\begin{pgfscope}%
\pgfsys@transformshift{3.752726in}{4.538555in}%
\pgfsys@useobject{currentmarker}{}%
\end{pgfscope}%
\end{pgfscope}%
\begin{pgfscope}%
\pgfpathrectangle{\pgfqpoint{0.100000in}{2.413063in}}{\pgfqpoint{5.037500in}{3.427208in}}%
\pgfusepath{clip}%
\pgfsetrectcap%
\pgfsetroundjoin%
\pgfsetlinewidth{1.505625pt}%
\definecolor{currentstroke}{rgb}{0.678431,1.000000,0.184314}%
\pgfsetstrokecolor{currentstroke}%
\pgfsetstrokeopacity{0.500000}%
\pgfsetdash{}{0pt}%
\pgfpathmoveto{\pgfqpoint{3.890081in}{4.558161in}}%
\pgfusepath{stroke}%
\end{pgfscope}%
\begin{pgfscope}%
\pgfpathrectangle{\pgfqpoint{0.100000in}{2.413063in}}{\pgfqpoint{5.037500in}{3.427208in}}%
\pgfusepath{clip}%
\pgfsetbuttcap%
\pgfsetroundjoin%
\definecolor{currentfill}{rgb}{0.678431,1.000000,0.184314}%
\pgfsetfillcolor{currentfill}%
\pgfsetfillopacity{0.500000}%
\pgfsetlinewidth{0.250937pt}%
\definecolor{currentstroke}{rgb}{0.000000,0.000000,0.000000}%
\pgfsetstrokecolor{currentstroke}%
\pgfsetstrokeopacity{0.500000}%
\pgfsetdash{}{0pt}%
\pgfsys@defobject{currentmarker}{\pgfqpoint{-0.019444in}{-0.019444in}}{\pgfqpoint{0.019444in}{0.019444in}}{%
\pgfpathmoveto{\pgfqpoint{0.000000in}{-0.019444in}}%
\pgfpathcurveto{\pgfqpoint{0.005157in}{-0.019444in}}{\pgfqpoint{0.010103in}{-0.017396in}}{\pgfqpoint{0.013749in}{-0.013749in}}%
\pgfpathcurveto{\pgfqpoint{0.017396in}{-0.010103in}}{\pgfqpoint{0.019444in}{-0.005157in}}{\pgfqpoint{0.019444in}{0.000000in}}%
\pgfpathcurveto{\pgfqpoint{0.019444in}{0.005157in}}{\pgfqpoint{0.017396in}{0.010103in}}{\pgfqpoint{0.013749in}{0.013749in}}%
\pgfpathcurveto{\pgfqpoint{0.010103in}{0.017396in}}{\pgfqpoint{0.005157in}{0.019444in}}{\pgfqpoint{0.000000in}{0.019444in}}%
\pgfpathcurveto{\pgfqpoint{-0.005157in}{0.019444in}}{\pgfqpoint{-0.010103in}{0.017396in}}{\pgfqpoint{-0.013749in}{0.013749in}}%
\pgfpathcurveto{\pgfqpoint{-0.017396in}{0.010103in}}{\pgfqpoint{-0.019444in}{0.005157in}}{\pgfqpoint{-0.019444in}{0.000000in}}%
\pgfpathcurveto{\pgfqpoint{-0.019444in}{-0.005157in}}{\pgfqpoint{-0.017396in}{-0.010103in}}{\pgfqpoint{-0.013749in}{-0.013749in}}%
\pgfpathcurveto{\pgfqpoint{-0.010103in}{-0.017396in}}{\pgfqpoint{-0.005157in}{-0.019444in}}{\pgfqpoint{0.000000in}{-0.019444in}}%
\pgfpathclose%
\pgfusepath{stroke,fill}%
}%
\begin{pgfscope}%
\pgfsys@transformshift{3.890081in}{4.558161in}%
\pgfsys@useobject{currentmarker}{}%
\end{pgfscope}%
\end{pgfscope}%
\begin{pgfscope}%
\pgfpathrectangle{\pgfqpoint{0.100000in}{2.413063in}}{\pgfqpoint{5.037500in}{3.427208in}}%
\pgfusepath{clip}%
\pgfsetrectcap%
\pgfsetroundjoin%
\pgfsetlinewidth{1.505625pt}%
\definecolor{currentstroke}{rgb}{0.678431,1.000000,0.184314}%
\pgfsetstrokecolor{currentstroke}%
\pgfsetstrokeopacity{0.500000}%
\pgfsetdash{}{0pt}%
\pgfpathmoveto{\pgfqpoint{3.789983in}{4.448154in}}%
\pgfusepath{stroke}%
\end{pgfscope}%
\begin{pgfscope}%
\pgfpathrectangle{\pgfqpoint{0.100000in}{2.413063in}}{\pgfqpoint{5.037500in}{3.427208in}}%
\pgfusepath{clip}%
\pgfsetbuttcap%
\pgfsetroundjoin%
\definecolor{currentfill}{rgb}{0.678431,1.000000,0.184314}%
\pgfsetfillcolor{currentfill}%
\pgfsetfillopacity{0.500000}%
\pgfsetlinewidth{0.250937pt}%
\definecolor{currentstroke}{rgb}{0.000000,0.000000,0.000000}%
\pgfsetstrokecolor{currentstroke}%
\pgfsetstrokeopacity{0.500000}%
\pgfsetdash{}{0pt}%
\pgfsys@defobject{currentmarker}{\pgfqpoint{-0.025000in}{-0.025000in}}{\pgfqpoint{0.025000in}{0.025000in}}{%
\pgfpathmoveto{\pgfqpoint{0.000000in}{-0.025000in}}%
\pgfpathcurveto{\pgfqpoint{0.006630in}{-0.025000in}}{\pgfqpoint{0.012989in}{-0.022366in}}{\pgfqpoint{0.017678in}{-0.017678in}}%
\pgfpathcurveto{\pgfqpoint{0.022366in}{-0.012989in}}{\pgfqpoint{0.025000in}{-0.006630in}}{\pgfqpoint{0.025000in}{0.000000in}}%
\pgfpathcurveto{\pgfqpoint{0.025000in}{0.006630in}}{\pgfqpoint{0.022366in}{0.012989in}}{\pgfqpoint{0.017678in}{0.017678in}}%
\pgfpathcurveto{\pgfqpoint{0.012989in}{0.022366in}}{\pgfqpoint{0.006630in}{0.025000in}}{\pgfqpoint{0.000000in}{0.025000in}}%
\pgfpathcurveto{\pgfqpoint{-0.006630in}{0.025000in}}{\pgfqpoint{-0.012989in}{0.022366in}}{\pgfqpoint{-0.017678in}{0.017678in}}%
\pgfpathcurveto{\pgfqpoint{-0.022366in}{0.012989in}}{\pgfqpoint{-0.025000in}{0.006630in}}{\pgfqpoint{-0.025000in}{0.000000in}}%
\pgfpathcurveto{\pgfqpoint{-0.025000in}{-0.006630in}}{\pgfqpoint{-0.022366in}{-0.012989in}}{\pgfqpoint{-0.017678in}{-0.017678in}}%
\pgfpathcurveto{\pgfqpoint{-0.012989in}{-0.022366in}}{\pgfqpoint{-0.006630in}{-0.025000in}}{\pgfqpoint{0.000000in}{-0.025000in}}%
\pgfpathclose%
\pgfusepath{stroke,fill}%
}%
\begin{pgfscope}%
\pgfsys@transformshift{3.789983in}{4.448154in}%
\pgfsys@useobject{currentmarker}{}%
\end{pgfscope}%
\end{pgfscope}%
\begin{pgfscope}%
\pgfpathrectangle{\pgfqpoint{0.100000in}{2.413063in}}{\pgfqpoint{5.037500in}{3.427208in}}%
\pgfusepath{clip}%
\pgfsetrectcap%
\pgfsetroundjoin%
\pgfsetlinewidth{1.505625pt}%
\definecolor{currentstroke}{rgb}{0.678431,1.000000,0.184314}%
\pgfsetstrokecolor{currentstroke}%
\pgfsetstrokeopacity{0.500000}%
\pgfsetdash{}{0pt}%
\pgfpathmoveto{\pgfqpoint{3.790448in}{4.651915in}}%
\pgfusepath{stroke}%
\end{pgfscope}%
\begin{pgfscope}%
\pgfpathrectangle{\pgfqpoint{0.100000in}{2.413063in}}{\pgfqpoint{5.037500in}{3.427208in}}%
\pgfusepath{clip}%
\pgfsetbuttcap%
\pgfsetroundjoin%
\definecolor{currentfill}{rgb}{0.678431,1.000000,0.184314}%
\pgfsetfillcolor{currentfill}%
\pgfsetfillopacity{0.500000}%
\pgfsetlinewidth{0.250937pt}%
\definecolor{currentstroke}{rgb}{0.000000,0.000000,0.000000}%
\pgfsetstrokecolor{currentstroke}%
\pgfsetstrokeopacity{0.500000}%
\pgfsetdash{}{0pt}%
\pgfsys@defobject{currentmarker}{\pgfqpoint{-0.013889in}{-0.013889in}}{\pgfqpoint{0.013889in}{0.013889in}}{%
\pgfpathmoveto{\pgfqpoint{0.000000in}{-0.013889in}}%
\pgfpathcurveto{\pgfqpoint{0.003683in}{-0.013889in}}{\pgfqpoint{0.007216in}{-0.012425in}}{\pgfqpoint{0.009821in}{-0.009821in}}%
\pgfpathcurveto{\pgfqpoint{0.012425in}{-0.007216in}}{\pgfqpoint{0.013889in}{-0.003683in}}{\pgfqpoint{0.013889in}{0.000000in}}%
\pgfpathcurveto{\pgfqpoint{0.013889in}{0.003683in}}{\pgfqpoint{0.012425in}{0.007216in}}{\pgfqpoint{0.009821in}{0.009821in}}%
\pgfpathcurveto{\pgfqpoint{0.007216in}{0.012425in}}{\pgfqpoint{0.003683in}{0.013889in}}{\pgfqpoint{0.000000in}{0.013889in}}%
\pgfpathcurveto{\pgfqpoint{-0.003683in}{0.013889in}}{\pgfqpoint{-0.007216in}{0.012425in}}{\pgfqpoint{-0.009821in}{0.009821in}}%
\pgfpathcurveto{\pgfqpoint{-0.012425in}{0.007216in}}{\pgfqpoint{-0.013889in}{0.003683in}}{\pgfqpoint{-0.013889in}{0.000000in}}%
\pgfpathcurveto{\pgfqpoint{-0.013889in}{-0.003683in}}{\pgfqpoint{-0.012425in}{-0.007216in}}{\pgfqpoint{-0.009821in}{-0.009821in}}%
\pgfpathcurveto{\pgfqpoint{-0.007216in}{-0.012425in}}{\pgfqpoint{-0.003683in}{-0.013889in}}{\pgfqpoint{0.000000in}{-0.013889in}}%
\pgfpathclose%
\pgfusepath{stroke,fill}%
}%
\begin{pgfscope}%
\pgfsys@transformshift{3.790448in}{4.651915in}%
\pgfsys@useobject{currentmarker}{}%
\end{pgfscope}%
\end{pgfscope}%
\begin{pgfscope}%
\pgfpathrectangle{\pgfqpoint{0.100000in}{2.413063in}}{\pgfqpoint{5.037500in}{3.427208in}}%
\pgfusepath{clip}%
\pgfsetrectcap%
\pgfsetroundjoin%
\pgfsetlinewidth{1.505625pt}%
\definecolor{currentstroke}{rgb}{0.678431,1.000000,0.184314}%
\pgfsetstrokecolor{currentstroke}%
\pgfsetstrokeopacity{0.500000}%
\pgfsetdash{}{0pt}%
\pgfpathmoveto{\pgfqpoint{4.062468in}{4.544330in}}%
\pgfusepath{stroke}%
\end{pgfscope}%
\begin{pgfscope}%
\pgfpathrectangle{\pgfqpoint{0.100000in}{2.413063in}}{\pgfqpoint{5.037500in}{3.427208in}}%
\pgfusepath{clip}%
\pgfsetbuttcap%
\pgfsetroundjoin%
\definecolor{currentfill}{rgb}{0.678431,1.000000,0.184314}%
\pgfsetfillcolor{currentfill}%
\pgfsetfillopacity{0.500000}%
\pgfsetlinewidth{0.250937pt}%
\definecolor{currentstroke}{rgb}{0.000000,0.000000,0.000000}%
\pgfsetstrokecolor{currentstroke}%
\pgfsetstrokeopacity{0.500000}%
\pgfsetdash{}{0pt}%
\pgfsys@defobject{currentmarker}{\pgfqpoint{-0.033333in}{-0.033333in}}{\pgfqpoint{0.033333in}{0.033333in}}{%
\pgfpathmoveto{\pgfqpoint{0.000000in}{-0.033333in}}%
\pgfpathcurveto{\pgfqpoint{0.008840in}{-0.033333in}}{\pgfqpoint{0.017319in}{-0.029821in}}{\pgfqpoint{0.023570in}{-0.023570in}}%
\pgfpathcurveto{\pgfqpoint{0.029821in}{-0.017319in}}{\pgfqpoint{0.033333in}{-0.008840in}}{\pgfqpoint{0.033333in}{0.000000in}}%
\pgfpathcurveto{\pgfqpoint{0.033333in}{0.008840in}}{\pgfqpoint{0.029821in}{0.017319in}}{\pgfqpoint{0.023570in}{0.023570in}}%
\pgfpathcurveto{\pgfqpoint{0.017319in}{0.029821in}}{\pgfqpoint{0.008840in}{0.033333in}}{\pgfqpoint{0.000000in}{0.033333in}}%
\pgfpathcurveto{\pgfqpoint{-0.008840in}{0.033333in}}{\pgfqpoint{-0.017319in}{0.029821in}}{\pgfqpoint{-0.023570in}{0.023570in}}%
\pgfpathcurveto{\pgfqpoint{-0.029821in}{0.017319in}}{\pgfqpoint{-0.033333in}{0.008840in}}{\pgfqpoint{-0.033333in}{0.000000in}}%
\pgfpathcurveto{\pgfqpoint{-0.033333in}{-0.008840in}}{\pgfqpoint{-0.029821in}{-0.017319in}}{\pgfqpoint{-0.023570in}{-0.023570in}}%
\pgfpathcurveto{\pgfqpoint{-0.017319in}{-0.029821in}}{\pgfqpoint{-0.008840in}{-0.033333in}}{\pgfqpoint{0.000000in}{-0.033333in}}%
\pgfpathclose%
\pgfusepath{stroke,fill}%
}%
\begin{pgfscope}%
\pgfsys@transformshift{4.062468in}{4.544330in}%
\pgfsys@useobject{currentmarker}{}%
\end{pgfscope}%
\end{pgfscope}%
\begin{pgfscope}%
\pgfpathrectangle{\pgfqpoint{0.100000in}{2.413063in}}{\pgfqpoint{5.037500in}{3.427208in}}%
\pgfusepath{clip}%
\pgfsetrectcap%
\pgfsetroundjoin%
\pgfsetlinewidth{1.505625pt}%
\definecolor{currentstroke}{rgb}{0.678431,1.000000,0.184314}%
\pgfsetstrokecolor{currentstroke}%
\pgfsetstrokeopacity{0.500000}%
\pgfsetdash{}{0pt}%
\pgfpathmoveto{\pgfqpoint{4.044939in}{4.620954in}}%
\pgfusepath{stroke}%
\end{pgfscope}%
\begin{pgfscope}%
\pgfpathrectangle{\pgfqpoint{0.100000in}{2.413063in}}{\pgfqpoint{5.037500in}{3.427208in}}%
\pgfusepath{clip}%
\pgfsetbuttcap%
\pgfsetroundjoin%
\definecolor{currentfill}{rgb}{0.678431,1.000000,0.184314}%
\pgfsetfillcolor{currentfill}%
\pgfsetfillopacity{0.500000}%
\pgfsetlinewidth{0.250937pt}%
\definecolor{currentstroke}{rgb}{0.000000,0.000000,0.000000}%
\pgfsetstrokecolor{currentstroke}%
\pgfsetstrokeopacity{0.500000}%
\pgfsetdash{}{0pt}%
\pgfsys@defobject{currentmarker}{\pgfqpoint{-0.033333in}{-0.033333in}}{\pgfqpoint{0.033333in}{0.033333in}}{%
\pgfpathmoveto{\pgfqpoint{0.000000in}{-0.033333in}}%
\pgfpathcurveto{\pgfqpoint{0.008840in}{-0.033333in}}{\pgfqpoint{0.017319in}{-0.029821in}}{\pgfqpoint{0.023570in}{-0.023570in}}%
\pgfpathcurveto{\pgfqpoint{0.029821in}{-0.017319in}}{\pgfqpoint{0.033333in}{-0.008840in}}{\pgfqpoint{0.033333in}{0.000000in}}%
\pgfpathcurveto{\pgfqpoint{0.033333in}{0.008840in}}{\pgfqpoint{0.029821in}{0.017319in}}{\pgfqpoint{0.023570in}{0.023570in}}%
\pgfpathcurveto{\pgfqpoint{0.017319in}{0.029821in}}{\pgfqpoint{0.008840in}{0.033333in}}{\pgfqpoint{0.000000in}{0.033333in}}%
\pgfpathcurveto{\pgfqpoint{-0.008840in}{0.033333in}}{\pgfqpoint{-0.017319in}{0.029821in}}{\pgfqpoint{-0.023570in}{0.023570in}}%
\pgfpathcurveto{\pgfqpoint{-0.029821in}{0.017319in}}{\pgfqpoint{-0.033333in}{0.008840in}}{\pgfqpoint{-0.033333in}{0.000000in}}%
\pgfpathcurveto{\pgfqpoint{-0.033333in}{-0.008840in}}{\pgfqpoint{-0.029821in}{-0.017319in}}{\pgfqpoint{-0.023570in}{-0.023570in}}%
\pgfpathcurveto{\pgfqpoint{-0.017319in}{-0.029821in}}{\pgfqpoint{-0.008840in}{-0.033333in}}{\pgfqpoint{0.000000in}{-0.033333in}}%
\pgfpathclose%
\pgfusepath{stroke,fill}%
}%
\begin{pgfscope}%
\pgfsys@transformshift{4.044939in}{4.620954in}%
\pgfsys@useobject{currentmarker}{}%
\end{pgfscope}%
\end{pgfscope}%
\begin{pgfscope}%
\pgfpathrectangle{\pgfqpoint{0.100000in}{2.413063in}}{\pgfqpoint{5.037500in}{3.427208in}}%
\pgfusepath{clip}%
\pgfsetrectcap%
\pgfsetroundjoin%
\pgfsetlinewidth{1.505625pt}%
\definecolor{currentstroke}{rgb}{0.678431,1.000000,0.184314}%
\pgfsetstrokecolor{currentstroke}%
\pgfsetstrokeopacity{0.500000}%
\pgfsetdash{}{0pt}%
\pgfpathmoveto{\pgfqpoint{2.481932in}{3.780796in}}%
\pgfusepath{stroke}%
\end{pgfscope}%
\begin{pgfscope}%
\pgfpathrectangle{\pgfqpoint{0.100000in}{2.413063in}}{\pgfqpoint{5.037500in}{3.427208in}}%
\pgfusepath{clip}%
\pgfsetbuttcap%
\pgfsetroundjoin%
\definecolor{currentfill}{rgb}{0.678431,1.000000,0.184314}%
\pgfsetfillcolor{currentfill}%
\pgfsetfillopacity{0.500000}%
\pgfsetlinewidth{0.250937pt}%
\definecolor{currentstroke}{rgb}{0.000000,0.000000,0.000000}%
\pgfsetstrokecolor{currentstroke}%
\pgfsetstrokeopacity{0.500000}%
\pgfsetdash{}{0pt}%
\pgfsys@defobject{currentmarker}{\pgfqpoint{-0.027778in}{-0.027778in}}{\pgfqpoint{0.027778in}{0.027778in}}{%
\pgfpathmoveto{\pgfqpoint{0.000000in}{-0.027778in}}%
\pgfpathcurveto{\pgfqpoint{0.007367in}{-0.027778in}}{\pgfqpoint{0.014433in}{-0.024851in}}{\pgfqpoint{0.019642in}{-0.019642in}}%
\pgfpathcurveto{\pgfqpoint{0.024851in}{-0.014433in}}{\pgfqpoint{0.027778in}{-0.007367in}}{\pgfqpoint{0.027778in}{0.000000in}}%
\pgfpathcurveto{\pgfqpoint{0.027778in}{0.007367in}}{\pgfqpoint{0.024851in}{0.014433in}}{\pgfqpoint{0.019642in}{0.019642in}}%
\pgfpathcurveto{\pgfqpoint{0.014433in}{0.024851in}}{\pgfqpoint{0.007367in}{0.027778in}}{\pgfqpoint{0.000000in}{0.027778in}}%
\pgfpathcurveto{\pgfqpoint{-0.007367in}{0.027778in}}{\pgfqpoint{-0.014433in}{0.024851in}}{\pgfqpoint{-0.019642in}{0.019642in}}%
\pgfpathcurveto{\pgfqpoint{-0.024851in}{0.014433in}}{\pgfqpoint{-0.027778in}{0.007367in}}{\pgfqpoint{-0.027778in}{0.000000in}}%
\pgfpathcurveto{\pgfqpoint{-0.027778in}{-0.007367in}}{\pgfqpoint{-0.024851in}{-0.014433in}}{\pgfqpoint{-0.019642in}{-0.019642in}}%
\pgfpathcurveto{\pgfqpoint{-0.014433in}{-0.024851in}}{\pgfqpoint{-0.007367in}{-0.027778in}}{\pgfqpoint{0.000000in}{-0.027778in}}%
\pgfpathclose%
\pgfusepath{stroke,fill}%
}%
\begin{pgfscope}%
\pgfsys@transformshift{2.481932in}{3.780796in}%
\pgfsys@useobject{currentmarker}{}%
\end{pgfscope}%
\end{pgfscope}%
\begin{pgfscope}%
\pgfpathrectangle{\pgfqpoint{0.100000in}{2.413063in}}{\pgfqpoint{5.037500in}{3.427208in}}%
\pgfusepath{clip}%
\pgfsetrectcap%
\pgfsetroundjoin%
\pgfsetlinewidth{1.505625pt}%
\definecolor{currentstroke}{rgb}{0.678431,1.000000,0.184314}%
\pgfsetstrokecolor{currentstroke}%
\pgfsetstrokeopacity{0.500000}%
\pgfsetdash{}{0pt}%
\pgfpathmoveto{\pgfqpoint{2.567661in}{3.877827in}}%
\pgfusepath{stroke}%
\end{pgfscope}%
\begin{pgfscope}%
\pgfpathrectangle{\pgfqpoint{0.100000in}{2.413063in}}{\pgfqpoint{5.037500in}{3.427208in}}%
\pgfusepath{clip}%
\pgfsetbuttcap%
\pgfsetroundjoin%
\definecolor{currentfill}{rgb}{0.678431,1.000000,0.184314}%
\pgfsetfillcolor{currentfill}%
\pgfsetfillopacity{0.500000}%
\pgfsetlinewidth{0.250937pt}%
\definecolor{currentstroke}{rgb}{0.000000,0.000000,0.000000}%
\pgfsetstrokecolor{currentstroke}%
\pgfsetstrokeopacity{0.500000}%
\pgfsetdash{}{0pt}%
\pgfsys@defobject{currentmarker}{\pgfqpoint{-0.025000in}{-0.025000in}}{\pgfqpoint{0.025000in}{0.025000in}}{%
\pgfpathmoveto{\pgfqpoint{0.000000in}{-0.025000in}}%
\pgfpathcurveto{\pgfqpoint{0.006630in}{-0.025000in}}{\pgfqpoint{0.012989in}{-0.022366in}}{\pgfqpoint{0.017678in}{-0.017678in}}%
\pgfpathcurveto{\pgfqpoint{0.022366in}{-0.012989in}}{\pgfqpoint{0.025000in}{-0.006630in}}{\pgfqpoint{0.025000in}{0.000000in}}%
\pgfpathcurveto{\pgfqpoint{0.025000in}{0.006630in}}{\pgfqpoint{0.022366in}{0.012989in}}{\pgfqpoint{0.017678in}{0.017678in}}%
\pgfpathcurveto{\pgfqpoint{0.012989in}{0.022366in}}{\pgfqpoint{0.006630in}{0.025000in}}{\pgfqpoint{0.000000in}{0.025000in}}%
\pgfpathcurveto{\pgfqpoint{-0.006630in}{0.025000in}}{\pgfqpoint{-0.012989in}{0.022366in}}{\pgfqpoint{-0.017678in}{0.017678in}}%
\pgfpathcurveto{\pgfqpoint{-0.022366in}{0.012989in}}{\pgfqpoint{-0.025000in}{0.006630in}}{\pgfqpoint{-0.025000in}{0.000000in}}%
\pgfpathcurveto{\pgfqpoint{-0.025000in}{-0.006630in}}{\pgfqpoint{-0.022366in}{-0.012989in}}{\pgfqpoint{-0.017678in}{-0.017678in}}%
\pgfpathcurveto{\pgfqpoint{-0.012989in}{-0.022366in}}{\pgfqpoint{-0.006630in}{-0.025000in}}{\pgfqpoint{0.000000in}{-0.025000in}}%
\pgfpathclose%
\pgfusepath{stroke,fill}%
}%
\begin{pgfscope}%
\pgfsys@transformshift{2.567661in}{3.877827in}%
\pgfsys@useobject{currentmarker}{}%
\end{pgfscope}%
\end{pgfscope}%
\begin{pgfscope}%
\pgfpathrectangle{\pgfqpoint{0.100000in}{2.413063in}}{\pgfqpoint{5.037500in}{3.427208in}}%
\pgfusepath{clip}%
\pgfsetrectcap%
\pgfsetroundjoin%
\pgfsetlinewidth{1.505625pt}%
\definecolor{currentstroke}{rgb}{0.678431,1.000000,0.184314}%
\pgfsetstrokecolor{currentstroke}%
\pgfsetstrokeopacity{0.500000}%
\pgfsetdash{}{0pt}%
\pgfpathmoveto{\pgfqpoint{2.711698in}{3.953793in}}%
\pgfusepath{stroke}%
\end{pgfscope}%
\begin{pgfscope}%
\pgfpathrectangle{\pgfqpoint{0.100000in}{2.413063in}}{\pgfqpoint{5.037500in}{3.427208in}}%
\pgfusepath{clip}%
\pgfsetbuttcap%
\pgfsetroundjoin%
\definecolor{currentfill}{rgb}{0.678431,1.000000,0.184314}%
\pgfsetfillcolor{currentfill}%
\pgfsetfillopacity{0.500000}%
\pgfsetlinewidth{0.250937pt}%
\definecolor{currentstroke}{rgb}{0.000000,0.000000,0.000000}%
\pgfsetstrokecolor{currentstroke}%
\pgfsetstrokeopacity{0.500000}%
\pgfsetdash{}{0pt}%
\pgfsys@defobject{currentmarker}{\pgfqpoint{-0.025000in}{-0.025000in}}{\pgfqpoint{0.025000in}{0.025000in}}{%
\pgfpathmoveto{\pgfqpoint{0.000000in}{-0.025000in}}%
\pgfpathcurveto{\pgfqpoint{0.006630in}{-0.025000in}}{\pgfqpoint{0.012989in}{-0.022366in}}{\pgfqpoint{0.017678in}{-0.017678in}}%
\pgfpathcurveto{\pgfqpoint{0.022366in}{-0.012989in}}{\pgfqpoint{0.025000in}{-0.006630in}}{\pgfqpoint{0.025000in}{0.000000in}}%
\pgfpathcurveto{\pgfqpoint{0.025000in}{0.006630in}}{\pgfqpoint{0.022366in}{0.012989in}}{\pgfqpoint{0.017678in}{0.017678in}}%
\pgfpathcurveto{\pgfqpoint{0.012989in}{0.022366in}}{\pgfqpoint{0.006630in}{0.025000in}}{\pgfqpoint{0.000000in}{0.025000in}}%
\pgfpathcurveto{\pgfqpoint{-0.006630in}{0.025000in}}{\pgfqpoint{-0.012989in}{0.022366in}}{\pgfqpoint{-0.017678in}{0.017678in}}%
\pgfpathcurveto{\pgfqpoint{-0.022366in}{0.012989in}}{\pgfqpoint{-0.025000in}{0.006630in}}{\pgfqpoint{-0.025000in}{0.000000in}}%
\pgfpathcurveto{\pgfqpoint{-0.025000in}{-0.006630in}}{\pgfqpoint{-0.022366in}{-0.012989in}}{\pgfqpoint{-0.017678in}{-0.017678in}}%
\pgfpathcurveto{\pgfqpoint{-0.012989in}{-0.022366in}}{\pgfqpoint{-0.006630in}{-0.025000in}}{\pgfqpoint{0.000000in}{-0.025000in}}%
\pgfpathclose%
\pgfusepath{stroke,fill}%
}%
\begin{pgfscope}%
\pgfsys@transformshift{2.711698in}{3.953793in}%
\pgfsys@useobject{currentmarker}{}%
\end{pgfscope}%
\end{pgfscope}%
\begin{pgfscope}%
\pgfpathrectangle{\pgfqpoint{0.100000in}{2.413063in}}{\pgfqpoint{5.037500in}{3.427208in}}%
\pgfusepath{clip}%
\pgfsetrectcap%
\pgfsetroundjoin%
\pgfsetlinewidth{1.505625pt}%
\definecolor{currentstroke}{rgb}{0.000000,0.000000,1.000000}%
\pgfsetstrokecolor{currentstroke}%
\pgfsetstrokeopacity{0.500000}%
\pgfsetdash{}{0pt}%
\pgfpathmoveto{\pgfqpoint{0.526793in}{5.285996in}}%
\pgfusepath{stroke}%
\end{pgfscope}%
\begin{pgfscope}%
\pgfpathrectangle{\pgfqpoint{0.100000in}{2.413063in}}{\pgfqpoint{5.037500in}{3.427208in}}%
\pgfusepath{clip}%
\pgfsetbuttcap%
\pgfsetroundjoin%
\definecolor{currentfill}{rgb}{0.000000,0.000000,1.000000}%
\pgfsetfillcolor{currentfill}%
\pgfsetfillopacity{0.500000}%
\pgfsetlinewidth{0.250937pt}%
\definecolor{currentstroke}{rgb}{0.000000,0.000000,0.000000}%
\pgfsetstrokecolor{currentstroke}%
\pgfsetstrokeopacity{0.500000}%
\pgfsetdash{}{0pt}%
\pgfsys@defobject{currentmarker}{\pgfqpoint{-0.016667in}{-0.016667in}}{\pgfqpoint{0.016667in}{0.016667in}}{%
\pgfpathmoveto{\pgfqpoint{0.000000in}{-0.016667in}}%
\pgfpathcurveto{\pgfqpoint{0.004420in}{-0.016667in}}{\pgfqpoint{0.008660in}{-0.014911in}}{\pgfqpoint{0.011785in}{-0.011785in}}%
\pgfpathcurveto{\pgfqpoint{0.014911in}{-0.008660in}}{\pgfqpoint{0.016667in}{-0.004420in}}{\pgfqpoint{0.016667in}{0.000000in}}%
\pgfpathcurveto{\pgfqpoint{0.016667in}{0.004420in}}{\pgfqpoint{0.014911in}{0.008660in}}{\pgfqpoint{0.011785in}{0.011785in}}%
\pgfpathcurveto{\pgfqpoint{0.008660in}{0.014911in}}{\pgfqpoint{0.004420in}{0.016667in}}{\pgfqpoint{0.000000in}{0.016667in}}%
\pgfpathcurveto{\pgfqpoint{-0.004420in}{0.016667in}}{\pgfqpoint{-0.008660in}{0.014911in}}{\pgfqpoint{-0.011785in}{0.011785in}}%
\pgfpathcurveto{\pgfqpoint{-0.014911in}{0.008660in}}{\pgfqpoint{-0.016667in}{0.004420in}}{\pgfqpoint{-0.016667in}{0.000000in}}%
\pgfpathcurveto{\pgfqpoint{-0.016667in}{-0.004420in}}{\pgfqpoint{-0.014911in}{-0.008660in}}{\pgfqpoint{-0.011785in}{-0.011785in}}%
\pgfpathcurveto{\pgfqpoint{-0.008660in}{-0.014911in}}{\pgfqpoint{-0.004420in}{-0.016667in}}{\pgfqpoint{0.000000in}{-0.016667in}}%
\pgfpathclose%
\pgfusepath{stroke,fill}%
}%
\begin{pgfscope}%
\pgfsys@transformshift{0.526793in}{5.285996in}%
\pgfsys@useobject{currentmarker}{}%
\end{pgfscope}%
\end{pgfscope}%
\begin{pgfscope}%
\pgfpathrectangle{\pgfqpoint{0.100000in}{2.413063in}}{\pgfqpoint{5.037500in}{3.427208in}}%
\pgfusepath{clip}%
\pgfsetrectcap%
\pgfsetroundjoin%
\pgfsetlinewidth{1.505625pt}%
\definecolor{currentstroke}{rgb}{0.000000,0.000000,1.000000}%
\pgfsetstrokecolor{currentstroke}%
\pgfsetstrokeopacity{0.500000}%
\pgfsetdash{}{0pt}%
\pgfpathmoveto{\pgfqpoint{0.648561in}{5.178411in}}%
\pgfusepath{stroke}%
\end{pgfscope}%
\begin{pgfscope}%
\pgfpathrectangle{\pgfqpoint{0.100000in}{2.413063in}}{\pgfqpoint{5.037500in}{3.427208in}}%
\pgfusepath{clip}%
\pgfsetbuttcap%
\pgfsetroundjoin%
\definecolor{currentfill}{rgb}{0.000000,0.000000,1.000000}%
\pgfsetfillcolor{currentfill}%
\pgfsetfillopacity{0.500000}%
\pgfsetlinewidth{0.250937pt}%
\definecolor{currentstroke}{rgb}{0.000000,0.000000,0.000000}%
\pgfsetstrokecolor{currentstroke}%
\pgfsetstrokeopacity{0.500000}%
\pgfsetdash{}{0pt}%
\pgfsys@defobject{currentmarker}{\pgfqpoint{-0.016667in}{-0.016667in}}{\pgfqpoint{0.016667in}{0.016667in}}{%
\pgfpathmoveto{\pgfqpoint{0.000000in}{-0.016667in}}%
\pgfpathcurveto{\pgfqpoint{0.004420in}{-0.016667in}}{\pgfqpoint{0.008660in}{-0.014911in}}{\pgfqpoint{0.011785in}{-0.011785in}}%
\pgfpathcurveto{\pgfqpoint{0.014911in}{-0.008660in}}{\pgfqpoint{0.016667in}{-0.004420in}}{\pgfqpoint{0.016667in}{0.000000in}}%
\pgfpathcurveto{\pgfqpoint{0.016667in}{0.004420in}}{\pgfqpoint{0.014911in}{0.008660in}}{\pgfqpoint{0.011785in}{0.011785in}}%
\pgfpathcurveto{\pgfqpoint{0.008660in}{0.014911in}}{\pgfqpoint{0.004420in}{0.016667in}}{\pgfqpoint{0.000000in}{0.016667in}}%
\pgfpathcurveto{\pgfqpoint{-0.004420in}{0.016667in}}{\pgfqpoint{-0.008660in}{0.014911in}}{\pgfqpoint{-0.011785in}{0.011785in}}%
\pgfpathcurveto{\pgfqpoint{-0.014911in}{0.008660in}}{\pgfqpoint{-0.016667in}{0.004420in}}{\pgfqpoint{-0.016667in}{0.000000in}}%
\pgfpathcurveto{\pgfqpoint{-0.016667in}{-0.004420in}}{\pgfqpoint{-0.014911in}{-0.008660in}}{\pgfqpoint{-0.011785in}{-0.011785in}}%
\pgfpathcurveto{\pgfqpoint{-0.008660in}{-0.014911in}}{\pgfqpoint{-0.004420in}{-0.016667in}}{\pgfqpoint{0.000000in}{-0.016667in}}%
\pgfpathclose%
\pgfusepath{stroke,fill}%
}%
\begin{pgfscope}%
\pgfsys@transformshift{0.648561in}{5.178411in}%
\pgfsys@useobject{currentmarker}{}%
\end{pgfscope}%
\end{pgfscope}%
\begin{pgfscope}%
\pgfpathrectangle{\pgfqpoint{0.100000in}{2.413063in}}{\pgfqpoint{5.037500in}{3.427208in}}%
\pgfusepath{clip}%
\pgfsetrectcap%
\pgfsetroundjoin%
\pgfsetlinewidth{1.505625pt}%
\definecolor{currentstroke}{rgb}{0.000000,0.000000,1.000000}%
\pgfsetstrokecolor{currentstroke}%
\pgfsetstrokeopacity{0.500000}%
\pgfsetdash{}{0pt}%
\pgfpathmoveto{\pgfqpoint{0.512006in}{5.281994in}}%
\pgfusepath{stroke}%
\end{pgfscope}%
\begin{pgfscope}%
\pgfpathrectangle{\pgfqpoint{0.100000in}{2.413063in}}{\pgfqpoint{5.037500in}{3.427208in}}%
\pgfusepath{clip}%
\pgfsetbuttcap%
\pgfsetroundjoin%
\definecolor{currentfill}{rgb}{0.000000,0.000000,1.000000}%
\pgfsetfillcolor{currentfill}%
\pgfsetfillopacity{0.500000}%
\pgfsetlinewidth{0.250937pt}%
\definecolor{currentstroke}{rgb}{0.000000,0.000000,0.000000}%
\pgfsetstrokecolor{currentstroke}%
\pgfsetstrokeopacity{0.500000}%
\pgfsetdash{}{0pt}%
\pgfsys@defobject{currentmarker}{\pgfqpoint{-0.013889in}{-0.013889in}}{\pgfqpoint{0.013889in}{0.013889in}}{%
\pgfpathmoveto{\pgfqpoint{0.000000in}{-0.013889in}}%
\pgfpathcurveto{\pgfqpoint{0.003683in}{-0.013889in}}{\pgfqpoint{0.007216in}{-0.012425in}}{\pgfqpoint{0.009821in}{-0.009821in}}%
\pgfpathcurveto{\pgfqpoint{0.012425in}{-0.007216in}}{\pgfqpoint{0.013889in}{-0.003683in}}{\pgfqpoint{0.013889in}{0.000000in}}%
\pgfpathcurveto{\pgfqpoint{0.013889in}{0.003683in}}{\pgfqpoint{0.012425in}{0.007216in}}{\pgfqpoint{0.009821in}{0.009821in}}%
\pgfpathcurveto{\pgfqpoint{0.007216in}{0.012425in}}{\pgfqpoint{0.003683in}{0.013889in}}{\pgfqpoint{0.000000in}{0.013889in}}%
\pgfpathcurveto{\pgfqpoint{-0.003683in}{0.013889in}}{\pgfqpoint{-0.007216in}{0.012425in}}{\pgfqpoint{-0.009821in}{0.009821in}}%
\pgfpathcurveto{\pgfqpoint{-0.012425in}{0.007216in}}{\pgfqpoint{-0.013889in}{0.003683in}}{\pgfqpoint{-0.013889in}{0.000000in}}%
\pgfpathcurveto{\pgfqpoint{-0.013889in}{-0.003683in}}{\pgfqpoint{-0.012425in}{-0.007216in}}{\pgfqpoint{-0.009821in}{-0.009821in}}%
\pgfpathcurveto{\pgfqpoint{-0.007216in}{-0.012425in}}{\pgfqpoint{-0.003683in}{-0.013889in}}{\pgfqpoint{0.000000in}{-0.013889in}}%
\pgfpathclose%
\pgfusepath{stroke,fill}%
}%
\begin{pgfscope}%
\pgfsys@transformshift{0.512006in}{5.281994in}%
\pgfsys@useobject{currentmarker}{}%
\end{pgfscope}%
\end{pgfscope}%
\begin{pgfscope}%
\pgfpathrectangle{\pgfqpoint{0.100000in}{2.413063in}}{\pgfqpoint{5.037500in}{3.427208in}}%
\pgfusepath{clip}%
\pgfsetrectcap%
\pgfsetroundjoin%
\pgfsetlinewidth{1.505625pt}%
\definecolor{currentstroke}{rgb}{0.000000,0.000000,1.000000}%
\pgfsetstrokecolor{currentstroke}%
\pgfsetstrokeopacity{0.500000}%
\pgfsetdash{}{0pt}%
\pgfpathmoveto{\pgfqpoint{0.504276in}{5.228196in}}%
\pgfusepath{stroke}%
\end{pgfscope}%
\begin{pgfscope}%
\pgfpathrectangle{\pgfqpoint{0.100000in}{2.413063in}}{\pgfqpoint{5.037500in}{3.427208in}}%
\pgfusepath{clip}%
\pgfsetbuttcap%
\pgfsetroundjoin%
\definecolor{currentfill}{rgb}{0.000000,0.000000,1.000000}%
\pgfsetfillcolor{currentfill}%
\pgfsetfillopacity{0.500000}%
\pgfsetlinewidth{0.250937pt}%
\definecolor{currentstroke}{rgb}{0.000000,0.000000,0.000000}%
\pgfsetstrokecolor{currentstroke}%
\pgfsetstrokeopacity{0.500000}%
\pgfsetdash{}{0pt}%
\pgfsys@defobject{currentmarker}{\pgfqpoint{-0.025000in}{-0.025000in}}{\pgfqpoint{0.025000in}{0.025000in}}{%
\pgfpathmoveto{\pgfqpoint{0.000000in}{-0.025000in}}%
\pgfpathcurveto{\pgfqpoint{0.006630in}{-0.025000in}}{\pgfqpoint{0.012989in}{-0.022366in}}{\pgfqpoint{0.017678in}{-0.017678in}}%
\pgfpathcurveto{\pgfqpoint{0.022366in}{-0.012989in}}{\pgfqpoint{0.025000in}{-0.006630in}}{\pgfqpoint{0.025000in}{0.000000in}}%
\pgfpathcurveto{\pgfqpoint{0.025000in}{0.006630in}}{\pgfqpoint{0.022366in}{0.012989in}}{\pgfqpoint{0.017678in}{0.017678in}}%
\pgfpathcurveto{\pgfqpoint{0.012989in}{0.022366in}}{\pgfqpoint{0.006630in}{0.025000in}}{\pgfqpoint{0.000000in}{0.025000in}}%
\pgfpathcurveto{\pgfqpoint{-0.006630in}{0.025000in}}{\pgfqpoint{-0.012989in}{0.022366in}}{\pgfqpoint{-0.017678in}{0.017678in}}%
\pgfpathcurveto{\pgfqpoint{-0.022366in}{0.012989in}}{\pgfqpoint{-0.025000in}{0.006630in}}{\pgfqpoint{-0.025000in}{0.000000in}}%
\pgfpathcurveto{\pgfqpoint{-0.025000in}{-0.006630in}}{\pgfqpoint{-0.022366in}{-0.012989in}}{\pgfqpoint{-0.017678in}{-0.017678in}}%
\pgfpathcurveto{\pgfqpoint{-0.012989in}{-0.022366in}}{\pgfqpoint{-0.006630in}{-0.025000in}}{\pgfqpoint{0.000000in}{-0.025000in}}%
\pgfpathclose%
\pgfusepath{stroke,fill}%
}%
\begin{pgfscope}%
\pgfsys@transformshift{0.504276in}{5.228196in}%
\pgfsys@useobject{currentmarker}{}%
\end{pgfscope}%
\end{pgfscope}%
\begin{pgfscope}%
\pgfpathrectangle{\pgfqpoint{0.100000in}{2.413063in}}{\pgfqpoint{5.037500in}{3.427208in}}%
\pgfusepath{clip}%
\pgfsetrectcap%
\pgfsetroundjoin%
\pgfsetlinewidth{1.505625pt}%
\definecolor{currentstroke}{rgb}{0.000000,0.000000,1.000000}%
\pgfsetstrokecolor{currentstroke}%
\pgfsetstrokeopacity{0.500000}%
\pgfsetdash{}{0pt}%
\pgfpathmoveto{\pgfqpoint{0.431607in}{5.050000in}}%
\pgfusepath{stroke}%
\end{pgfscope}%
\begin{pgfscope}%
\pgfpathrectangle{\pgfqpoint{0.100000in}{2.413063in}}{\pgfqpoint{5.037500in}{3.427208in}}%
\pgfusepath{clip}%
\pgfsetbuttcap%
\pgfsetroundjoin%
\definecolor{currentfill}{rgb}{0.000000,0.000000,1.000000}%
\pgfsetfillcolor{currentfill}%
\pgfsetfillopacity{0.500000}%
\pgfsetlinewidth{0.250937pt}%
\definecolor{currentstroke}{rgb}{0.000000,0.000000,0.000000}%
\pgfsetstrokecolor{currentstroke}%
\pgfsetstrokeopacity{0.500000}%
\pgfsetdash{}{0pt}%
\pgfsys@defobject{currentmarker}{\pgfqpoint{-0.022222in}{-0.022222in}}{\pgfqpoint{0.022222in}{0.022222in}}{%
\pgfpathmoveto{\pgfqpoint{0.000000in}{-0.022222in}}%
\pgfpathcurveto{\pgfqpoint{0.005893in}{-0.022222in}}{\pgfqpoint{0.011546in}{-0.019881in}}{\pgfqpoint{0.015713in}{-0.015713in}}%
\pgfpathcurveto{\pgfqpoint{0.019881in}{-0.011546in}}{\pgfqpoint{0.022222in}{-0.005893in}}{\pgfqpoint{0.022222in}{0.000000in}}%
\pgfpathcurveto{\pgfqpoint{0.022222in}{0.005893in}}{\pgfqpoint{0.019881in}{0.011546in}}{\pgfqpoint{0.015713in}{0.015713in}}%
\pgfpathcurveto{\pgfqpoint{0.011546in}{0.019881in}}{\pgfqpoint{0.005893in}{0.022222in}}{\pgfqpoint{0.000000in}{0.022222in}}%
\pgfpathcurveto{\pgfqpoint{-0.005893in}{0.022222in}}{\pgfqpoint{-0.011546in}{0.019881in}}{\pgfqpoint{-0.015713in}{0.015713in}}%
\pgfpathcurveto{\pgfqpoint{-0.019881in}{0.011546in}}{\pgfqpoint{-0.022222in}{0.005893in}}{\pgfqpoint{-0.022222in}{0.000000in}}%
\pgfpathcurveto{\pgfqpoint{-0.022222in}{-0.005893in}}{\pgfqpoint{-0.019881in}{-0.011546in}}{\pgfqpoint{-0.015713in}{-0.015713in}}%
\pgfpathcurveto{\pgfqpoint{-0.011546in}{-0.019881in}}{\pgfqpoint{-0.005893in}{-0.022222in}}{\pgfqpoint{0.000000in}{-0.022222in}}%
\pgfpathclose%
\pgfusepath{stroke,fill}%
}%
\begin{pgfscope}%
\pgfsys@transformshift{0.431607in}{5.050000in}%
\pgfsys@useobject{currentmarker}{}%
\end{pgfscope}%
\end{pgfscope}%
\begin{pgfscope}%
\pgfpathrectangle{\pgfqpoint{0.100000in}{2.413063in}}{\pgfqpoint{5.037500in}{3.427208in}}%
\pgfusepath{clip}%
\pgfsetrectcap%
\pgfsetroundjoin%
\pgfsetlinewidth{1.505625pt}%
\definecolor{currentstroke}{rgb}{0.000000,0.000000,1.000000}%
\pgfsetstrokecolor{currentstroke}%
\pgfsetstrokeopacity{0.500000}%
\pgfsetdash{}{0pt}%
\pgfpathmoveto{\pgfqpoint{0.464601in}{5.025783in}}%
\pgfusepath{stroke}%
\end{pgfscope}%
\begin{pgfscope}%
\pgfpathrectangle{\pgfqpoint{0.100000in}{2.413063in}}{\pgfqpoint{5.037500in}{3.427208in}}%
\pgfusepath{clip}%
\pgfsetbuttcap%
\pgfsetroundjoin%
\definecolor{currentfill}{rgb}{0.000000,0.000000,1.000000}%
\pgfsetfillcolor{currentfill}%
\pgfsetfillopacity{0.500000}%
\pgfsetlinewidth{0.250937pt}%
\definecolor{currentstroke}{rgb}{0.000000,0.000000,0.000000}%
\pgfsetstrokecolor{currentstroke}%
\pgfsetstrokeopacity{0.500000}%
\pgfsetdash{}{0pt}%
\pgfsys@defobject{currentmarker}{\pgfqpoint{-0.016667in}{-0.016667in}}{\pgfqpoint{0.016667in}{0.016667in}}{%
\pgfpathmoveto{\pgfqpoint{0.000000in}{-0.016667in}}%
\pgfpathcurveto{\pgfqpoint{0.004420in}{-0.016667in}}{\pgfqpoint{0.008660in}{-0.014911in}}{\pgfqpoint{0.011785in}{-0.011785in}}%
\pgfpathcurveto{\pgfqpoint{0.014911in}{-0.008660in}}{\pgfqpoint{0.016667in}{-0.004420in}}{\pgfqpoint{0.016667in}{0.000000in}}%
\pgfpathcurveto{\pgfqpoint{0.016667in}{0.004420in}}{\pgfqpoint{0.014911in}{0.008660in}}{\pgfqpoint{0.011785in}{0.011785in}}%
\pgfpathcurveto{\pgfqpoint{0.008660in}{0.014911in}}{\pgfqpoint{0.004420in}{0.016667in}}{\pgfqpoint{0.000000in}{0.016667in}}%
\pgfpathcurveto{\pgfqpoint{-0.004420in}{0.016667in}}{\pgfqpoint{-0.008660in}{0.014911in}}{\pgfqpoint{-0.011785in}{0.011785in}}%
\pgfpathcurveto{\pgfqpoint{-0.014911in}{0.008660in}}{\pgfqpoint{-0.016667in}{0.004420in}}{\pgfqpoint{-0.016667in}{0.000000in}}%
\pgfpathcurveto{\pgfqpoint{-0.016667in}{-0.004420in}}{\pgfqpoint{-0.014911in}{-0.008660in}}{\pgfqpoint{-0.011785in}{-0.011785in}}%
\pgfpathcurveto{\pgfqpoint{-0.008660in}{-0.014911in}}{\pgfqpoint{-0.004420in}{-0.016667in}}{\pgfqpoint{0.000000in}{-0.016667in}}%
\pgfpathclose%
\pgfusepath{stroke,fill}%
}%
\begin{pgfscope}%
\pgfsys@transformshift{0.464601in}{5.025783in}%
\pgfsys@useobject{currentmarker}{}%
\end{pgfscope}%
\end{pgfscope}%
\begin{pgfscope}%
\pgfpathrectangle{\pgfqpoint{0.100000in}{2.413063in}}{\pgfqpoint{5.037500in}{3.427208in}}%
\pgfusepath{clip}%
\pgfsetrectcap%
\pgfsetroundjoin%
\pgfsetlinewidth{1.505625pt}%
\definecolor{currentstroke}{rgb}{0.000000,0.000000,1.000000}%
\pgfsetstrokecolor{currentstroke}%
\pgfsetstrokeopacity{0.500000}%
\pgfsetdash{}{0pt}%
\pgfpathmoveto{\pgfqpoint{0.591363in}{5.372886in}}%
\pgfusepath{stroke}%
\end{pgfscope}%
\begin{pgfscope}%
\pgfpathrectangle{\pgfqpoint{0.100000in}{2.413063in}}{\pgfqpoint{5.037500in}{3.427208in}}%
\pgfusepath{clip}%
\pgfsetbuttcap%
\pgfsetroundjoin%
\definecolor{currentfill}{rgb}{0.000000,0.000000,1.000000}%
\pgfsetfillcolor{currentfill}%
\pgfsetfillopacity{0.500000}%
\pgfsetlinewidth{0.250937pt}%
\definecolor{currentstroke}{rgb}{0.000000,0.000000,0.000000}%
\pgfsetstrokecolor{currentstroke}%
\pgfsetstrokeopacity{0.500000}%
\pgfsetdash{}{0pt}%
\pgfsys@defobject{currentmarker}{\pgfqpoint{-0.016667in}{-0.016667in}}{\pgfqpoint{0.016667in}{0.016667in}}{%
\pgfpathmoveto{\pgfqpoint{0.000000in}{-0.016667in}}%
\pgfpathcurveto{\pgfqpoint{0.004420in}{-0.016667in}}{\pgfqpoint{0.008660in}{-0.014911in}}{\pgfqpoint{0.011785in}{-0.011785in}}%
\pgfpathcurveto{\pgfqpoint{0.014911in}{-0.008660in}}{\pgfqpoint{0.016667in}{-0.004420in}}{\pgfqpoint{0.016667in}{0.000000in}}%
\pgfpathcurveto{\pgfqpoint{0.016667in}{0.004420in}}{\pgfqpoint{0.014911in}{0.008660in}}{\pgfqpoint{0.011785in}{0.011785in}}%
\pgfpathcurveto{\pgfqpoint{0.008660in}{0.014911in}}{\pgfqpoint{0.004420in}{0.016667in}}{\pgfqpoint{0.000000in}{0.016667in}}%
\pgfpathcurveto{\pgfqpoint{-0.004420in}{0.016667in}}{\pgfqpoint{-0.008660in}{0.014911in}}{\pgfqpoint{-0.011785in}{0.011785in}}%
\pgfpathcurveto{\pgfqpoint{-0.014911in}{0.008660in}}{\pgfqpoint{-0.016667in}{0.004420in}}{\pgfqpoint{-0.016667in}{0.000000in}}%
\pgfpathcurveto{\pgfqpoint{-0.016667in}{-0.004420in}}{\pgfqpoint{-0.014911in}{-0.008660in}}{\pgfqpoint{-0.011785in}{-0.011785in}}%
\pgfpathcurveto{\pgfqpoint{-0.008660in}{-0.014911in}}{\pgfqpoint{-0.004420in}{-0.016667in}}{\pgfqpoint{0.000000in}{-0.016667in}}%
\pgfpathclose%
\pgfusepath{stroke,fill}%
}%
\begin{pgfscope}%
\pgfsys@transformshift{0.591363in}{5.372886in}%
\pgfsys@useobject{currentmarker}{}%
\end{pgfscope}%
\end{pgfscope}%
\begin{pgfscope}%
\pgfpathrectangle{\pgfqpoint{0.100000in}{2.413063in}}{\pgfqpoint{5.037500in}{3.427208in}}%
\pgfusepath{clip}%
\pgfsetrectcap%
\pgfsetroundjoin%
\pgfsetlinewidth{1.505625pt}%
\definecolor{currentstroke}{rgb}{0.000000,0.000000,1.000000}%
\pgfsetstrokecolor{currentstroke}%
\pgfsetstrokeopacity{0.500000}%
\pgfsetdash{}{0pt}%
\pgfpathmoveto{\pgfqpoint{0.543139in}{5.317548in}}%
\pgfusepath{stroke}%
\end{pgfscope}%
\begin{pgfscope}%
\pgfpathrectangle{\pgfqpoint{0.100000in}{2.413063in}}{\pgfqpoint{5.037500in}{3.427208in}}%
\pgfusepath{clip}%
\pgfsetbuttcap%
\pgfsetroundjoin%
\definecolor{currentfill}{rgb}{0.000000,0.000000,1.000000}%
\pgfsetfillcolor{currentfill}%
\pgfsetfillopacity{0.500000}%
\pgfsetlinewidth{0.250937pt}%
\definecolor{currentstroke}{rgb}{0.000000,0.000000,0.000000}%
\pgfsetstrokecolor{currentstroke}%
\pgfsetstrokeopacity{0.500000}%
\pgfsetdash{}{0pt}%
\pgfsys@defobject{currentmarker}{\pgfqpoint{-0.013889in}{-0.013889in}}{\pgfqpoint{0.013889in}{0.013889in}}{%
\pgfpathmoveto{\pgfqpoint{0.000000in}{-0.013889in}}%
\pgfpathcurveto{\pgfqpoint{0.003683in}{-0.013889in}}{\pgfqpoint{0.007216in}{-0.012425in}}{\pgfqpoint{0.009821in}{-0.009821in}}%
\pgfpathcurveto{\pgfqpoint{0.012425in}{-0.007216in}}{\pgfqpoint{0.013889in}{-0.003683in}}{\pgfqpoint{0.013889in}{0.000000in}}%
\pgfpathcurveto{\pgfqpoint{0.013889in}{0.003683in}}{\pgfqpoint{0.012425in}{0.007216in}}{\pgfqpoint{0.009821in}{0.009821in}}%
\pgfpathcurveto{\pgfqpoint{0.007216in}{0.012425in}}{\pgfqpoint{0.003683in}{0.013889in}}{\pgfqpoint{0.000000in}{0.013889in}}%
\pgfpathcurveto{\pgfqpoint{-0.003683in}{0.013889in}}{\pgfqpoint{-0.007216in}{0.012425in}}{\pgfqpoint{-0.009821in}{0.009821in}}%
\pgfpathcurveto{\pgfqpoint{-0.012425in}{0.007216in}}{\pgfqpoint{-0.013889in}{0.003683in}}{\pgfqpoint{-0.013889in}{0.000000in}}%
\pgfpathcurveto{\pgfqpoint{-0.013889in}{-0.003683in}}{\pgfqpoint{-0.012425in}{-0.007216in}}{\pgfqpoint{-0.009821in}{-0.009821in}}%
\pgfpathcurveto{\pgfqpoint{-0.007216in}{-0.012425in}}{\pgfqpoint{-0.003683in}{-0.013889in}}{\pgfqpoint{0.000000in}{-0.013889in}}%
\pgfpathclose%
\pgfusepath{stroke,fill}%
}%
\begin{pgfscope}%
\pgfsys@transformshift{0.543139in}{5.317548in}%
\pgfsys@useobject{currentmarker}{}%
\end{pgfscope}%
\end{pgfscope}%
\begin{pgfscope}%
\pgfpathrectangle{\pgfqpoint{0.100000in}{2.413063in}}{\pgfqpoint{5.037500in}{3.427208in}}%
\pgfusepath{clip}%
\pgfsetrectcap%
\pgfsetroundjoin%
\pgfsetlinewidth{1.505625pt}%
\definecolor{currentstroke}{rgb}{0.678431,1.000000,0.184314}%
\pgfsetstrokecolor{currentstroke}%
\pgfsetstrokeopacity{0.500000}%
\pgfsetdash{}{0pt}%
\pgfpathmoveto{\pgfqpoint{4.498915in}{4.648262in}}%
\pgfusepath{stroke}%
\end{pgfscope}%
\begin{pgfscope}%
\pgfpathrectangle{\pgfqpoint{0.100000in}{2.413063in}}{\pgfqpoint{5.037500in}{3.427208in}}%
\pgfusepath{clip}%
\pgfsetbuttcap%
\pgfsetroundjoin%
\definecolor{currentfill}{rgb}{0.678431,1.000000,0.184314}%
\pgfsetfillcolor{currentfill}%
\pgfsetfillopacity{0.500000}%
\pgfsetlinewidth{0.250937pt}%
\definecolor{currentstroke}{rgb}{0.000000,0.000000,0.000000}%
\pgfsetstrokecolor{currentstroke}%
\pgfsetstrokeopacity{0.500000}%
\pgfsetdash{}{0pt}%
\pgfsys@defobject{currentmarker}{\pgfqpoint{-0.011111in}{-0.011111in}}{\pgfqpoint{0.011111in}{0.011111in}}{%
\pgfpathmoveto{\pgfqpoint{0.000000in}{-0.011111in}}%
\pgfpathcurveto{\pgfqpoint{0.002947in}{-0.011111in}}{\pgfqpoint{0.005773in}{-0.009940in}}{\pgfqpoint{0.007857in}{-0.007857in}}%
\pgfpathcurveto{\pgfqpoint{0.009940in}{-0.005773in}}{\pgfqpoint{0.011111in}{-0.002947in}}{\pgfqpoint{0.011111in}{0.000000in}}%
\pgfpathcurveto{\pgfqpoint{0.011111in}{0.002947in}}{\pgfqpoint{0.009940in}{0.005773in}}{\pgfqpoint{0.007857in}{0.007857in}}%
\pgfpathcurveto{\pgfqpoint{0.005773in}{0.009940in}}{\pgfqpoint{0.002947in}{0.011111in}}{\pgfqpoint{0.000000in}{0.011111in}}%
\pgfpathcurveto{\pgfqpoint{-0.002947in}{0.011111in}}{\pgfqpoint{-0.005773in}{0.009940in}}{\pgfqpoint{-0.007857in}{0.007857in}}%
\pgfpathcurveto{\pgfqpoint{-0.009940in}{0.005773in}}{\pgfqpoint{-0.011111in}{0.002947in}}{\pgfqpoint{-0.011111in}{0.000000in}}%
\pgfpathcurveto{\pgfqpoint{-0.011111in}{-0.002947in}}{\pgfqpoint{-0.009940in}{-0.005773in}}{\pgfqpoint{-0.007857in}{-0.007857in}}%
\pgfpathcurveto{\pgfqpoint{-0.005773in}{-0.009940in}}{\pgfqpoint{-0.002947in}{-0.011111in}}{\pgfqpoint{0.000000in}{-0.011111in}}%
\pgfpathclose%
\pgfusepath{stroke,fill}%
}%
\begin{pgfscope}%
\pgfsys@transformshift{4.498915in}{4.648262in}%
\pgfsys@useobject{currentmarker}{}%
\end{pgfscope}%
\end{pgfscope}%
\begin{pgfscope}%
\pgfpathrectangle{\pgfqpoint{0.100000in}{2.413063in}}{\pgfqpoint{5.037500in}{3.427208in}}%
\pgfusepath{clip}%
\pgfsetrectcap%
\pgfsetroundjoin%
\pgfsetlinewidth{1.505625pt}%
\definecolor{currentstroke}{rgb}{0.678431,1.000000,0.184314}%
\pgfsetstrokecolor{currentstroke}%
\pgfsetstrokeopacity{0.500000}%
\pgfsetdash{}{0pt}%
\pgfpathmoveto{\pgfqpoint{4.250037in}{4.588307in}}%
\pgfusepath{stroke}%
\end{pgfscope}%
\begin{pgfscope}%
\pgfpathrectangle{\pgfqpoint{0.100000in}{2.413063in}}{\pgfqpoint{5.037500in}{3.427208in}}%
\pgfusepath{clip}%
\pgfsetbuttcap%
\pgfsetroundjoin%
\definecolor{currentfill}{rgb}{0.678431,1.000000,0.184314}%
\pgfsetfillcolor{currentfill}%
\pgfsetfillopacity{0.500000}%
\pgfsetlinewidth{0.250937pt}%
\definecolor{currentstroke}{rgb}{0.000000,0.000000,0.000000}%
\pgfsetstrokecolor{currentstroke}%
\pgfsetstrokeopacity{0.500000}%
\pgfsetdash{}{0pt}%
\pgfsys@defobject{currentmarker}{\pgfqpoint{-0.027778in}{-0.027778in}}{\pgfqpoint{0.027778in}{0.027778in}}{%
\pgfpathmoveto{\pgfqpoint{0.000000in}{-0.027778in}}%
\pgfpathcurveto{\pgfqpoint{0.007367in}{-0.027778in}}{\pgfqpoint{0.014433in}{-0.024851in}}{\pgfqpoint{0.019642in}{-0.019642in}}%
\pgfpathcurveto{\pgfqpoint{0.024851in}{-0.014433in}}{\pgfqpoint{0.027778in}{-0.007367in}}{\pgfqpoint{0.027778in}{0.000000in}}%
\pgfpathcurveto{\pgfqpoint{0.027778in}{0.007367in}}{\pgfqpoint{0.024851in}{0.014433in}}{\pgfqpoint{0.019642in}{0.019642in}}%
\pgfpathcurveto{\pgfqpoint{0.014433in}{0.024851in}}{\pgfqpoint{0.007367in}{0.027778in}}{\pgfqpoint{0.000000in}{0.027778in}}%
\pgfpathcurveto{\pgfqpoint{-0.007367in}{0.027778in}}{\pgfqpoint{-0.014433in}{0.024851in}}{\pgfqpoint{-0.019642in}{0.019642in}}%
\pgfpathcurveto{\pgfqpoint{-0.024851in}{0.014433in}}{\pgfqpoint{-0.027778in}{0.007367in}}{\pgfqpoint{-0.027778in}{0.000000in}}%
\pgfpathcurveto{\pgfqpoint{-0.027778in}{-0.007367in}}{\pgfqpoint{-0.024851in}{-0.014433in}}{\pgfqpoint{-0.019642in}{-0.019642in}}%
\pgfpathcurveto{\pgfqpoint{-0.014433in}{-0.024851in}}{\pgfqpoint{-0.007367in}{-0.027778in}}{\pgfqpoint{0.000000in}{-0.027778in}}%
\pgfpathclose%
\pgfusepath{stroke,fill}%
}%
\begin{pgfscope}%
\pgfsys@transformshift{4.250037in}{4.588307in}%
\pgfsys@useobject{currentmarker}{}%
\end{pgfscope}%
\end{pgfscope}%
\begin{pgfscope}%
\pgfpathrectangle{\pgfqpoint{0.100000in}{2.413063in}}{\pgfqpoint{5.037500in}{3.427208in}}%
\pgfusepath{clip}%
\pgfsetrectcap%
\pgfsetroundjoin%
\pgfsetlinewidth{1.505625pt}%
\definecolor{currentstroke}{rgb}{0.678431,1.000000,0.184314}%
\pgfsetstrokecolor{currentstroke}%
\pgfsetstrokeopacity{0.500000}%
\pgfsetdash{}{0pt}%
\pgfpathmoveto{\pgfqpoint{4.405482in}{4.675732in}}%
\pgfusepath{stroke}%
\end{pgfscope}%
\begin{pgfscope}%
\pgfpathrectangle{\pgfqpoint{0.100000in}{2.413063in}}{\pgfqpoint{5.037500in}{3.427208in}}%
\pgfusepath{clip}%
\pgfsetbuttcap%
\pgfsetroundjoin%
\definecolor{currentfill}{rgb}{0.678431,1.000000,0.184314}%
\pgfsetfillcolor{currentfill}%
\pgfsetfillopacity{0.500000}%
\pgfsetlinewidth{0.250937pt}%
\definecolor{currentstroke}{rgb}{0.000000,0.000000,0.000000}%
\pgfsetstrokecolor{currentstroke}%
\pgfsetstrokeopacity{0.500000}%
\pgfsetdash{}{0pt}%
\pgfsys@defobject{currentmarker}{\pgfqpoint{-0.038889in}{-0.038889in}}{\pgfqpoint{0.038889in}{0.038889in}}{%
\pgfpathmoveto{\pgfqpoint{0.000000in}{-0.038889in}}%
\pgfpathcurveto{\pgfqpoint{0.010313in}{-0.038889in}}{\pgfqpoint{0.020206in}{-0.034791in}}{\pgfqpoint{0.027499in}{-0.027499in}}%
\pgfpathcurveto{\pgfqpoint{0.034791in}{-0.020206in}}{\pgfqpoint{0.038889in}{-0.010313in}}{\pgfqpoint{0.038889in}{0.000000in}}%
\pgfpathcurveto{\pgfqpoint{0.038889in}{0.010313in}}{\pgfqpoint{0.034791in}{0.020206in}}{\pgfqpoint{0.027499in}{0.027499in}}%
\pgfpathcurveto{\pgfqpoint{0.020206in}{0.034791in}}{\pgfqpoint{0.010313in}{0.038889in}}{\pgfqpoint{0.000000in}{0.038889in}}%
\pgfpathcurveto{\pgfqpoint{-0.010313in}{0.038889in}}{\pgfqpoint{-0.020206in}{0.034791in}}{\pgfqpoint{-0.027499in}{0.027499in}}%
\pgfpathcurveto{\pgfqpoint{-0.034791in}{0.020206in}}{\pgfqpoint{-0.038889in}{0.010313in}}{\pgfqpoint{-0.038889in}{0.000000in}}%
\pgfpathcurveto{\pgfqpoint{-0.038889in}{-0.010313in}}{\pgfqpoint{-0.034791in}{-0.020206in}}{\pgfqpoint{-0.027499in}{-0.027499in}}%
\pgfpathcurveto{\pgfqpoint{-0.020206in}{-0.034791in}}{\pgfqpoint{-0.010313in}{-0.038889in}}{\pgfqpoint{0.000000in}{-0.038889in}}%
\pgfpathclose%
\pgfusepath{stroke,fill}%
}%
\begin{pgfscope}%
\pgfsys@transformshift{4.405482in}{4.675732in}%
\pgfsys@useobject{currentmarker}{}%
\end{pgfscope}%
\end{pgfscope}%
\begin{pgfscope}%
\pgfpathrectangle{\pgfqpoint{0.100000in}{2.413063in}}{\pgfqpoint{5.037500in}{3.427208in}}%
\pgfusepath{clip}%
\pgfsetrectcap%
\pgfsetroundjoin%
\pgfsetlinewidth{1.505625pt}%
\definecolor{currentstroke}{rgb}{0.678431,1.000000,0.184314}%
\pgfsetstrokecolor{currentstroke}%
\pgfsetstrokeopacity{0.500000}%
\pgfsetdash{}{0pt}%
\pgfpathmoveto{\pgfqpoint{4.325892in}{4.534885in}}%
\pgfusepath{stroke}%
\end{pgfscope}%
\begin{pgfscope}%
\pgfpathrectangle{\pgfqpoint{0.100000in}{2.413063in}}{\pgfqpoint{5.037500in}{3.427208in}}%
\pgfusepath{clip}%
\pgfsetbuttcap%
\pgfsetroundjoin%
\definecolor{currentfill}{rgb}{0.678431,1.000000,0.184314}%
\pgfsetfillcolor{currentfill}%
\pgfsetfillopacity{0.500000}%
\pgfsetlinewidth{0.250937pt}%
\definecolor{currentstroke}{rgb}{0.000000,0.000000,0.000000}%
\pgfsetstrokecolor{currentstroke}%
\pgfsetstrokeopacity{0.500000}%
\pgfsetdash{}{0pt}%
\pgfsys@defobject{currentmarker}{\pgfqpoint{-0.030556in}{-0.030556in}}{\pgfqpoint{0.030556in}{0.030556in}}{%
\pgfpathmoveto{\pgfqpoint{0.000000in}{-0.030556in}}%
\pgfpathcurveto{\pgfqpoint{0.008103in}{-0.030556in}}{\pgfqpoint{0.015876in}{-0.027336in}}{\pgfqpoint{0.021606in}{-0.021606in}}%
\pgfpathcurveto{\pgfqpoint{0.027336in}{-0.015876in}}{\pgfqpoint{0.030556in}{-0.008103in}}{\pgfqpoint{0.030556in}{0.000000in}}%
\pgfpathcurveto{\pgfqpoint{0.030556in}{0.008103in}}{\pgfqpoint{0.027336in}{0.015876in}}{\pgfqpoint{0.021606in}{0.021606in}}%
\pgfpathcurveto{\pgfqpoint{0.015876in}{0.027336in}}{\pgfqpoint{0.008103in}{0.030556in}}{\pgfqpoint{0.000000in}{0.030556in}}%
\pgfpathcurveto{\pgfqpoint{-0.008103in}{0.030556in}}{\pgfqpoint{-0.015876in}{0.027336in}}{\pgfqpoint{-0.021606in}{0.021606in}}%
\pgfpathcurveto{\pgfqpoint{-0.027336in}{0.015876in}}{\pgfqpoint{-0.030556in}{0.008103in}}{\pgfqpoint{-0.030556in}{0.000000in}}%
\pgfpathcurveto{\pgfqpoint{-0.030556in}{-0.008103in}}{\pgfqpoint{-0.027336in}{-0.015876in}}{\pgfqpoint{-0.021606in}{-0.021606in}}%
\pgfpathcurveto{\pgfqpoint{-0.015876in}{-0.027336in}}{\pgfqpoint{-0.008103in}{-0.030556in}}{\pgfqpoint{0.000000in}{-0.030556in}}%
\pgfpathclose%
\pgfusepath{stroke,fill}%
}%
\begin{pgfscope}%
\pgfsys@transformshift{4.325892in}{4.534885in}%
\pgfsys@useobject{currentmarker}{}%
\end{pgfscope}%
\end{pgfscope}%
\begin{pgfscope}%
\pgfpathrectangle{\pgfqpoint{0.100000in}{2.413063in}}{\pgfqpoint{5.037500in}{3.427208in}}%
\pgfusepath{clip}%
\pgfsetrectcap%
\pgfsetroundjoin%
\pgfsetlinewidth{1.505625pt}%
\definecolor{currentstroke}{rgb}{0.678431,1.000000,0.184314}%
\pgfsetstrokecolor{currentstroke}%
\pgfsetstrokeopacity{0.500000}%
\pgfsetdash{}{0pt}%
\pgfpathmoveto{\pgfqpoint{4.513781in}{4.698398in}}%
\pgfusepath{stroke}%
\end{pgfscope}%
\begin{pgfscope}%
\pgfpathrectangle{\pgfqpoint{0.100000in}{2.413063in}}{\pgfqpoint{5.037500in}{3.427208in}}%
\pgfusepath{clip}%
\pgfsetbuttcap%
\pgfsetroundjoin%
\definecolor{currentfill}{rgb}{0.678431,1.000000,0.184314}%
\pgfsetfillcolor{currentfill}%
\pgfsetfillopacity{0.500000}%
\pgfsetlinewidth{0.250937pt}%
\definecolor{currentstroke}{rgb}{0.000000,0.000000,0.000000}%
\pgfsetstrokecolor{currentstroke}%
\pgfsetstrokeopacity{0.500000}%
\pgfsetdash{}{0pt}%
\pgfsys@defobject{currentmarker}{\pgfqpoint{-0.008333in}{-0.008333in}}{\pgfqpoint{0.008333in}{0.008333in}}{%
\pgfpathmoveto{\pgfqpoint{0.000000in}{-0.008333in}}%
\pgfpathcurveto{\pgfqpoint{0.002210in}{-0.008333in}}{\pgfqpoint{0.004330in}{-0.007455in}}{\pgfqpoint{0.005893in}{-0.005893in}}%
\pgfpathcurveto{\pgfqpoint{0.007455in}{-0.004330in}}{\pgfqpoint{0.008333in}{-0.002210in}}{\pgfqpoint{0.008333in}{0.000000in}}%
\pgfpathcurveto{\pgfqpoint{0.008333in}{0.002210in}}{\pgfqpoint{0.007455in}{0.004330in}}{\pgfqpoint{0.005893in}{0.005893in}}%
\pgfpathcurveto{\pgfqpoint{0.004330in}{0.007455in}}{\pgfqpoint{0.002210in}{0.008333in}}{\pgfqpoint{0.000000in}{0.008333in}}%
\pgfpathcurveto{\pgfqpoint{-0.002210in}{0.008333in}}{\pgfqpoint{-0.004330in}{0.007455in}}{\pgfqpoint{-0.005893in}{0.005893in}}%
\pgfpathcurveto{\pgfqpoint{-0.007455in}{0.004330in}}{\pgfqpoint{-0.008333in}{0.002210in}}{\pgfqpoint{-0.008333in}{0.000000in}}%
\pgfpathcurveto{\pgfqpoint{-0.008333in}{-0.002210in}}{\pgfqpoint{-0.007455in}{-0.004330in}}{\pgfqpoint{-0.005893in}{-0.005893in}}%
\pgfpathcurveto{\pgfqpoint{-0.004330in}{-0.007455in}}{\pgfqpoint{-0.002210in}{-0.008333in}}{\pgfqpoint{0.000000in}{-0.008333in}}%
\pgfpathclose%
\pgfusepath{stroke,fill}%
}%
\begin{pgfscope}%
\pgfsys@transformshift{4.513781in}{4.698398in}%
\pgfsys@useobject{currentmarker}{}%
\end{pgfscope}%
\end{pgfscope}%
\begin{pgfscope}%
\pgfpathrectangle{\pgfqpoint{0.100000in}{2.413063in}}{\pgfqpoint{5.037500in}{3.427208in}}%
\pgfusepath{clip}%
\pgfsetrectcap%
\pgfsetroundjoin%
\pgfsetlinewidth{1.505625pt}%
\definecolor{currentstroke}{rgb}{0.501961,0.501961,0.501961}%
\pgfsetstrokecolor{currentstroke}%
\pgfsetstrokeopacity{0.500000}%
\pgfsetdash{}{0pt}%
\pgfpathmoveto{\pgfqpoint{4.073915in}{4.745853in}}%
\pgfusepath{stroke}%
\end{pgfscope}%
\begin{pgfscope}%
\pgfpathrectangle{\pgfqpoint{0.100000in}{2.413063in}}{\pgfqpoint{5.037500in}{3.427208in}}%
\pgfusepath{clip}%
\pgfsetbuttcap%
\pgfsetroundjoin%
\definecolor{currentfill}{rgb}{0.501961,0.501961,0.501961}%
\pgfsetfillcolor{currentfill}%
\pgfsetfillopacity{0.500000}%
\pgfsetlinewidth{0.250937pt}%
\definecolor{currentstroke}{rgb}{0.000000,0.000000,0.000000}%
\pgfsetstrokecolor{currentstroke}%
\pgfsetstrokeopacity{0.500000}%
\pgfsetdash{}{0pt}%
\pgfsys@defobject{currentmarker}{\pgfqpoint{-0.013889in}{-0.013889in}}{\pgfqpoint{0.013889in}{0.013889in}}{%
\pgfpathmoveto{\pgfqpoint{0.000000in}{-0.013889in}}%
\pgfpathcurveto{\pgfqpoint{0.003683in}{-0.013889in}}{\pgfqpoint{0.007216in}{-0.012425in}}{\pgfqpoint{0.009821in}{-0.009821in}}%
\pgfpathcurveto{\pgfqpoint{0.012425in}{-0.007216in}}{\pgfqpoint{0.013889in}{-0.003683in}}{\pgfqpoint{0.013889in}{0.000000in}}%
\pgfpathcurveto{\pgfqpoint{0.013889in}{0.003683in}}{\pgfqpoint{0.012425in}{0.007216in}}{\pgfqpoint{0.009821in}{0.009821in}}%
\pgfpathcurveto{\pgfqpoint{0.007216in}{0.012425in}}{\pgfqpoint{0.003683in}{0.013889in}}{\pgfqpoint{0.000000in}{0.013889in}}%
\pgfpathcurveto{\pgfqpoint{-0.003683in}{0.013889in}}{\pgfqpoint{-0.007216in}{0.012425in}}{\pgfqpoint{-0.009821in}{0.009821in}}%
\pgfpathcurveto{\pgfqpoint{-0.012425in}{0.007216in}}{\pgfqpoint{-0.013889in}{0.003683in}}{\pgfqpoint{-0.013889in}{0.000000in}}%
\pgfpathcurveto{\pgfqpoint{-0.013889in}{-0.003683in}}{\pgfqpoint{-0.012425in}{-0.007216in}}{\pgfqpoint{-0.009821in}{-0.009821in}}%
\pgfpathcurveto{\pgfqpoint{-0.007216in}{-0.012425in}}{\pgfqpoint{-0.003683in}{-0.013889in}}{\pgfqpoint{0.000000in}{-0.013889in}}%
\pgfpathclose%
\pgfusepath{stroke,fill}%
}%
\begin{pgfscope}%
\pgfsys@transformshift{4.073915in}{4.745853in}%
\pgfsys@useobject{currentmarker}{}%
\end{pgfscope}%
\end{pgfscope}%
\begin{pgfscope}%
\pgfpathrectangle{\pgfqpoint{0.100000in}{2.413063in}}{\pgfqpoint{5.037500in}{3.427208in}}%
\pgfusepath{clip}%
\pgfsetrectcap%
\pgfsetroundjoin%
\pgfsetlinewidth{1.505625pt}%
\definecolor{currentstroke}{rgb}{0.678431,1.000000,0.184314}%
\pgfsetstrokecolor{currentstroke}%
\pgfsetstrokeopacity{0.500000}%
\pgfsetdash{}{0pt}%
\pgfpathmoveto{\pgfqpoint{4.365499in}{4.529766in}}%
\pgfusepath{stroke}%
\end{pgfscope}%
\begin{pgfscope}%
\pgfpathrectangle{\pgfqpoint{0.100000in}{2.413063in}}{\pgfqpoint{5.037500in}{3.427208in}}%
\pgfusepath{clip}%
\pgfsetbuttcap%
\pgfsetroundjoin%
\definecolor{currentfill}{rgb}{0.678431,1.000000,0.184314}%
\pgfsetfillcolor{currentfill}%
\pgfsetfillopacity{0.500000}%
\pgfsetlinewidth{0.250937pt}%
\definecolor{currentstroke}{rgb}{0.000000,0.000000,0.000000}%
\pgfsetstrokecolor{currentstroke}%
\pgfsetstrokeopacity{0.500000}%
\pgfsetdash{}{0pt}%
\pgfsys@defobject{currentmarker}{\pgfqpoint{-0.016667in}{-0.016667in}}{\pgfqpoint{0.016667in}{0.016667in}}{%
\pgfpathmoveto{\pgfqpoint{0.000000in}{-0.016667in}}%
\pgfpathcurveto{\pgfqpoint{0.004420in}{-0.016667in}}{\pgfqpoint{0.008660in}{-0.014911in}}{\pgfqpoint{0.011785in}{-0.011785in}}%
\pgfpathcurveto{\pgfqpoint{0.014911in}{-0.008660in}}{\pgfqpoint{0.016667in}{-0.004420in}}{\pgfqpoint{0.016667in}{0.000000in}}%
\pgfpathcurveto{\pgfqpoint{0.016667in}{0.004420in}}{\pgfqpoint{0.014911in}{0.008660in}}{\pgfqpoint{0.011785in}{0.011785in}}%
\pgfpathcurveto{\pgfqpoint{0.008660in}{0.014911in}}{\pgfqpoint{0.004420in}{0.016667in}}{\pgfqpoint{0.000000in}{0.016667in}}%
\pgfpathcurveto{\pgfqpoint{-0.004420in}{0.016667in}}{\pgfqpoint{-0.008660in}{0.014911in}}{\pgfqpoint{-0.011785in}{0.011785in}}%
\pgfpathcurveto{\pgfqpoint{-0.014911in}{0.008660in}}{\pgfqpoint{-0.016667in}{0.004420in}}{\pgfqpoint{-0.016667in}{0.000000in}}%
\pgfpathcurveto{\pgfqpoint{-0.016667in}{-0.004420in}}{\pgfqpoint{-0.014911in}{-0.008660in}}{\pgfqpoint{-0.011785in}{-0.011785in}}%
\pgfpathcurveto{\pgfqpoint{-0.008660in}{-0.014911in}}{\pgfqpoint{-0.004420in}{-0.016667in}}{\pgfqpoint{0.000000in}{-0.016667in}}%
\pgfpathclose%
\pgfusepath{stroke,fill}%
}%
\begin{pgfscope}%
\pgfsys@transformshift{4.365499in}{4.529766in}%
\pgfsys@useobject{currentmarker}{}%
\end{pgfscope}%
\end{pgfscope}%
\begin{pgfscope}%
\pgfpathrectangle{\pgfqpoint{0.100000in}{2.413063in}}{\pgfqpoint{5.037500in}{3.427208in}}%
\pgfusepath{clip}%
\pgfsetrectcap%
\pgfsetroundjoin%
\pgfsetlinewidth{1.505625pt}%
\definecolor{currentstroke}{rgb}{0.678431,1.000000,0.184314}%
\pgfsetstrokecolor{currentstroke}%
\pgfsetstrokeopacity{0.500000}%
\pgfsetdash{}{0pt}%
\pgfpathmoveto{\pgfqpoint{4.385486in}{4.584896in}}%
\pgfusepath{stroke}%
\end{pgfscope}%
\begin{pgfscope}%
\pgfpathrectangle{\pgfqpoint{0.100000in}{2.413063in}}{\pgfqpoint{5.037500in}{3.427208in}}%
\pgfusepath{clip}%
\pgfsetbuttcap%
\pgfsetroundjoin%
\definecolor{currentfill}{rgb}{0.678431,1.000000,0.184314}%
\pgfsetfillcolor{currentfill}%
\pgfsetfillopacity{0.500000}%
\pgfsetlinewidth{0.250937pt}%
\definecolor{currentstroke}{rgb}{0.000000,0.000000,0.000000}%
\pgfsetstrokecolor{currentstroke}%
\pgfsetstrokeopacity{0.500000}%
\pgfsetdash{}{0pt}%
\pgfsys@defobject{currentmarker}{\pgfqpoint{-0.011111in}{-0.011111in}}{\pgfqpoint{0.011111in}{0.011111in}}{%
\pgfpathmoveto{\pgfqpoint{0.000000in}{-0.011111in}}%
\pgfpathcurveto{\pgfqpoint{0.002947in}{-0.011111in}}{\pgfqpoint{0.005773in}{-0.009940in}}{\pgfqpoint{0.007857in}{-0.007857in}}%
\pgfpathcurveto{\pgfqpoint{0.009940in}{-0.005773in}}{\pgfqpoint{0.011111in}{-0.002947in}}{\pgfqpoint{0.011111in}{0.000000in}}%
\pgfpathcurveto{\pgfqpoint{0.011111in}{0.002947in}}{\pgfqpoint{0.009940in}{0.005773in}}{\pgfqpoint{0.007857in}{0.007857in}}%
\pgfpathcurveto{\pgfqpoint{0.005773in}{0.009940in}}{\pgfqpoint{0.002947in}{0.011111in}}{\pgfqpoint{0.000000in}{0.011111in}}%
\pgfpathcurveto{\pgfqpoint{-0.002947in}{0.011111in}}{\pgfqpoint{-0.005773in}{0.009940in}}{\pgfqpoint{-0.007857in}{0.007857in}}%
\pgfpathcurveto{\pgfqpoint{-0.009940in}{0.005773in}}{\pgfqpoint{-0.011111in}{0.002947in}}{\pgfqpoint{-0.011111in}{0.000000in}}%
\pgfpathcurveto{\pgfqpoint{-0.011111in}{-0.002947in}}{\pgfqpoint{-0.009940in}{-0.005773in}}{\pgfqpoint{-0.007857in}{-0.007857in}}%
\pgfpathcurveto{\pgfqpoint{-0.005773in}{-0.009940in}}{\pgfqpoint{-0.002947in}{-0.011111in}}{\pgfqpoint{0.000000in}{-0.011111in}}%
\pgfpathclose%
\pgfusepath{stroke,fill}%
}%
\begin{pgfscope}%
\pgfsys@transformshift{4.385486in}{4.584896in}%
\pgfsys@useobject{currentmarker}{}%
\end{pgfscope}%
\end{pgfscope}%
\begin{pgfscope}%
\pgfpathrectangle{\pgfqpoint{0.100000in}{2.413063in}}{\pgfqpoint{5.037500in}{3.427208in}}%
\pgfusepath{clip}%
\pgfsetrectcap%
\pgfsetroundjoin%
\pgfsetlinewidth{1.505625pt}%
\definecolor{currentstroke}{rgb}{0.678431,1.000000,0.184314}%
\pgfsetstrokecolor{currentstroke}%
\pgfsetstrokeopacity{0.500000}%
\pgfsetdash{}{0pt}%
\pgfpathmoveto{\pgfqpoint{4.208507in}{4.558288in}}%
\pgfusepath{stroke}%
\end{pgfscope}%
\begin{pgfscope}%
\pgfpathrectangle{\pgfqpoint{0.100000in}{2.413063in}}{\pgfqpoint{5.037500in}{3.427208in}}%
\pgfusepath{clip}%
\pgfsetbuttcap%
\pgfsetroundjoin%
\definecolor{currentfill}{rgb}{0.678431,1.000000,0.184314}%
\pgfsetfillcolor{currentfill}%
\pgfsetfillopacity{0.500000}%
\pgfsetlinewidth{0.250937pt}%
\definecolor{currentstroke}{rgb}{0.000000,0.000000,0.000000}%
\pgfsetstrokecolor{currentstroke}%
\pgfsetstrokeopacity{0.500000}%
\pgfsetdash{}{0pt}%
\pgfsys@defobject{currentmarker}{\pgfqpoint{-0.025000in}{-0.025000in}}{\pgfqpoint{0.025000in}{0.025000in}}{%
\pgfpathmoveto{\pgfqpoint{0.000000in}{-0.025000in}}%
\pgfpathcurveto{\pgfqpoint{0.006630in}{-0.025000in}}{\pgfqpoint{0.012989in}{-0.022366in}}{\pgfqpoint{0.017678in}{-0.017678in}}%
\pgfpathcurveto{\pgfqpoint{0.022366in}{-0.012989in}}{\pgfqpoint{0.025000in}{-0.006630in}}{\pgfqpoint{0.025000in}{0.000000in}}%
\pgfpathcurveto{\pgfqpoint{0.025000in}{0.006630in}}{\pgfqpoint{0.022366in}{0.012989in}}{\pgfqpoint{0.017678in}{0.017678in}}%
\pgfpathcurveto{\pgfqpoint{0.012989in}{0.022366in}}{\pgfqpoint{0.006630in}{0.025000in}}{\pgfqpoint{0.000000in}{0.025000in}}%
\pgfpathcurveto{\pgfqpoint{-0.006630in}{0.025000in}}{\pgfqpoint{-0.012989in}{0.022366in}}{\pgfqpoint{-0.017678in}{0.017678in}}%
\pgfpathcurveto{\pgfqpoint{-0.022366in}{0.012989in}}{\pgfqpoint{-0.025000in}{0.006630in}}{\pgfqpoint{-0.025000in}{0.000000in}}%
\pgfpathcurveto{\pgfqpoint{-0.025000in}{-0.006630in}}{\pgfqpoint{-0.022366in}{-0.012989in}}{\pgfqpoint{-0.017678in}{-0.017678in}}%
\pgfpathcurveto{\pgfqpoint{-0.012989in}{-0.022366in}}{\pgfqpoint{-0.006630in}{-0.025000in}}{\pgfqpoint{0.000000in}{-0.025000in}}%
\pgfpathclose%
\pgfusepath{stroke,fill}%
}%
\begin{pgfscope}%
\pgfsys@transformshift{4.208507in}{4.558288in}%
\pgfsys@useobject{currentmarker}{}%
\end{pgfscope}%
\end{pgfscope}%
\begin{pgfscope}%
\pgfpathrectangle{\pgfqpoint{0.100000in}{2.413063in}}{\pgfqpoint{5.037500in}{3.427208in}}%
\pgfusepath{clip}%
\pgfsetrectcap%
\pgfsetroundjoin%
\pgfsetlinewidth{1.505625pt}%
\definecolor{currentstroke}{rgb}{0.678431,1.000000,0.184314}%
\pgfsetstrokecolor{currentstroke}%
\pgfsetstrokeopacity{0.500000}%
\pgfsetdash{}{0pt}%
\pgfpathmoveto{\pgfqpoint{4.440822in}{4.569431in}}%
\pgfusepath{stroke}%
\end{pgfscope}%
\begin{pgfscope}%
\pgfpathrectangle{\pgfqpoint{0.100000in}{2.413063in}}{\pgfqpoint{5.037500in}{3.427208in}}%
\pgfusepath{clip}%
\pgfsetbuttcap%
\pgfsetroundjoin%
\definecolor{currentfill}{rgb}{0.678431,1.000000,0.184314}%
\pgfsetfillcolor{currentfill}%
\pgfsetfillopacity{0.500000}%
\pgfsetlinewidth{0.250937pt}%
\definecolor{currentstroke}{rgb}{0.000000,0.000000,0.000000}%
\pgfsetstrokecolor{currentstroke}%
\pgfsetstrokeopacity{0.500000}%
\pgfsetdash{}{0pt}%
\pgfsys@defobject{currentmarker}{\pgfqpoint{-0.005556in}{-0.005556in}}{\pgfqpoint{0.005556in}{0.005556in}}{%
\pgfpathmoveto{\pgfqpoint{0.000000in}{-0.005556in}}%
\pgfpathcurveto{\pgfqpoint{0.001473in}{-0.005556in}}{\pgfqpoint{0.002887in}{-0.004970in}}{\pgfqpoint{0.003928in}{-0.003928in}}%
\pgfpathcurveto{\pgfqpoint{0.004970in}{-0.002887in}}{\pgfqpoint{0.005556in}{-0.001473in}}{\pgfqpoint{0.005556in}{0.000000in}}%
\pgfpathcurveto{\pgfqpoint{0.005556in}{0.001473in}}{\pgfqpoint{0.004970in}{0.002887in}}{\pgfqpoint{0.003928in}{0.003928in}}%
\pgfpathcurveto{\pgfqpoint{0.002887in}{0.004970in}}{\pgfqpoint{0.001473in}{0.005556in}}{\pgfqpoint{0.000000in}{0.005556in}}%
\pgfpathcurveto{\pgfqpoint{-0.001473in}{0.005556in}}{\pgfqpoint{-0.002887in}{0.004970in}}{\pgfqpoint{-0.003928in}{0.003928in}}%
\pgfpathcurveto{\pgfqpoint{-0.004970in}{0.002887in}}{\pgfqpoint{-0.005556in}{0.001473in}}{\pgfqpoint{-0.005556in}{0.000000in}}%
\pgfpathcurveto{\pgfqpoint{-0.005556in}{-0.001473in}}{\pgfqpoint{-0.004970in}{-0.002887in}}{\pgfqpoint{-0.003928in}{-0.003928in}}%
\pgfpathcurveto{\pgfqpoint{-0.002887in}{-0.004970in}}{\pgfqpoint{-0.001473in}{-0.005556in}}{\pgfqpoint{0.000000in}{-0.005556in}}%
\pgfpathclose%
\pgfusepath{stroke,fill}%
}%
\begin{pgfscope}%
\pgfsys@transformshift{4.440822in}{4.569431in}%
\pgfsys@useobject{currentmarker}{}%
\end{pgfscope}%
\end{pgfscope}%
\begin{pgfscope}%
\pgfpathrectangle{\pgfqpoint{0.100000in}{2.413063in}}{\pgfqpoint{5.037500in}{3.427208in}}%
\pgfusepath{clip}%
\pgfsetrectcap%
\pgfsetroundjoin%
\pgfsetlinewidth{1.505625pt}%
\definecolor{currentstroke}{rgb}{0.678431,1.000000,0.184314}%
\pgfsetstrokecolor{currentstroke}%
\pgfsetstrokeopacity{0.500000}%
\pgfsetdash{}{0pt}%
\pgfpathmoveto{\pgfqpoint{4.423020in}{4.602009in}}%
\pgfusepath{stroke}%
\end{pgfscope}%
\begin{pgfscope}%
\pgfpathrectangle{\pgfqpoint{0.100000in}{2.413063in}}{\pgfqpoint{5.037500in}{3.427208in}}%
\pgfusepath{clip}%
\pgfsetbuttcap%
\pgfsetroundjoin%
\definecolor{currentfill}{rgb}{0.678431,1.000000,0.184314}%
\pgfsetfillcolor{currentfill}%
\pgfsetfillopacity{0.500000}%
\pgfsetlinewidth{0.250937pt}%
\definecolor{currentstroke}{rgb}{0.000000,0.000000,0.000000}%
\pgfsetstrokecolor{currentstroke}%
\pgfsetstrokeopacity{0.500000}%
\pgfsetdash{}{0pt}%
\pgfsys@defobject{currentmarker}{\pgfqpoint{-0.019444in}{-0.019444in}}{\pgfqpoint{0.019444in}{0.019444in}}{%
\pgfpathmoveto{\pgfqpoint{0.000000in}{-0.019444in}}%
\pgfpathcurveto{\pgfqpoint{0.005157in}{-0.019444in}}{\pgfqpoint{0.010103in}{-0.017396in}}{\pgfqpoint{0.013749in}{-0.013749in}}%
\pgfpathcurveto{\pgfqpoint{0.017396in}{-0.010103in}}{\pgfqpoint{0.019444in}{-0.005157in}}{\pgfqpoint{0.019444in}{0.000000in}}%
\pgfpathcurveto{\pgfqpoint{0.019444in}{0.005157in}}{\pgfqpoint{0.017396in}{0.010103in}}{\pgfqpoint{0.013749in}{0.013749in}}%
\pgfpathcurveto{\pgfqpoint{0.010103in}{0.017396in}}{\pgfqpoint{0.005157in}{0.019444in}}{\pgfqpoint{0.000000in}{0.019444in}}%
\pgfpathcurveto{\pgfqpoint{-0.005157in}{0.019444in}}{\pgfqpoint{-0.010103in}{0.017396in}}{\pgfqpoint{-0.013749in}{0.013749in}}%
\pgfpathcurveto{\pgfqpoint{-0.017396in}{0.010103in}}{\pgfqpoint{-0.019444in}{0.005157in}}{\pgfqpoint{-0.019444in}{0.000000in}}%
\pgfpathcurveto{\pgfqpoint{-0.019444in}{-0.005157in}}{\pgfqpoint{-0.017396in}{-0.010103in}}{\pgfqpoint{-0.013749in}{-0.013749in}}%
\pgfpathcurveto{\pgfqpoint{-0.010103in}{-0.017396in}}{\pgfqpoint{-0.005157in}{-0.019444in}}{\pgfqpoint{0.000000in}{-0.019444in}}%
\pgfpathclose%
\pgfusepath{stroke,fill}%
}%
\begin{pgfscope}%
\pgfsys@transformshift{4.423020in}{4.602009in}%
\pgfsys@useobject{currentmarker}{}%
\end{pgfscope}%
\end{pgfscope}%
\begin{pgfscope}%
\pgfpathrectangle{\pgfqpoint{0.100000in}{2.413063in}}{\pgfqpoint{5.037500in}{3.427208in}}%
\pgfusepath{clip}%
\pgfsetrectcap%
\pgfsetroundjoin%
\pgfsetlinewidth{1.505625pt}%
\definecolor{currentstroke}{rgb}{0.000000,0.000000,1.000000}%
\pgfsetstrokecolor{currentstroke}%
\pgfsetstrokeopacity{0.500000}%
\pgfsetdash{}{0pt}%
\pgfpathmoveto{\pgfqpoint{4.541403in}{4.580980in}}%
\pgfusepath{stroke}%
\end{pgfscope}%
\begin{pgfscope}%
\pgfpathrectangle{\pgfqpoint{0.100000in}{2.413063in}}{\pgfqpoint{5.037500in}{3.427208in}}%
\pgfusepath{clip}%
\pgfsetbuttcap%
\pgfsetroundjoin%
\definecolor{currentfill}{rgb}{0.000000,0.000000,1.000000}%
\pgfsetfillcolor{currentfill}%
\pgfsetfillopacity{0.500000}%
\pgfsetlinewidth{0.250937pt}%
\definecolor{currentstroke}{rgb}{0.000000,0.000000,0.000000}%
\pgfsetstrokecolor{currentstroke}%
\pgfsetstrokeopacity{0.500000}%
\pgfsetdash{}{0pt}%
\pgfsys@defobject{currentmarker}{\pgfqpoint{-0.011111in}{-0.011111in}}{\pgfqpoint{0.011111in}{0.011111in}}{%
\pgfpathmoveto{\pgfqpoint{0.000000in}{-0.011111in}}%
\pgfpathcurveto{\pgfqpoint{0.002947in}{-0.011111in}}{\pgfqpoint{0.005773in}{-0.009940in}}{\pgfqpoint{0.007857in}{-0.007857in}}%
\pgfpathcurveto{\pgfqpoint{0.009940in}{-0.005773in}}{\pgfqpoint{0.011111in}{-0.002947in}}{\pgfqpoint{0.011111in}{0.000000in}}%
\pgfpathcurveto{\pgfqpoint{0.011111in}{0.002947in}}{\pgfqpoint{0.009940in}{0.005773in}}{\pgfqpoint{0.007857in}{0.007857in}}%
\pgfpathcurveto{\pgfqpoint{0.005773in}{0.009940in}}{\pgfqpoint{0.002947in}{0.011111in}}{\pgfqpoint{0.000000in}{0.011111in}}%
\pgfpathcurveto{\pgfqpoint{-0.002947in}{0.011111in}}{\pgfqpoint{-0.005773in}{0.009940in}}{\pgfqpoint{-0.007857in}{0.007857in}}%
\pgfpathcurveto{\pgfqpoint{-0.009940in}{0.005773in}}{\pgfqpoint{-0.011111in}{0.002947in}}{\pgfqpoint{-0.011111in}{0.000000in}}%
\pgfpathcurveto{\pgfqpoint{-0.011111in}{-0.002947in}}{\pgfqpoint{-0.009940in}{-0.005773in}}{\pgfqpoint{-0.007857in}{-0.007857in}}%
\pgfpathcurveto{\pgfqpoint{-0.005773in}{-0.009940in}}{\pgfqpoint{-0.002947in}{-0.011111in}}{\pgfqpoint{0.000000in}{-0.011111in}}%
\pgfpathclose%
\pgfusepath{stroke,fill}%
}%
\begin{pgfscope}%
\pgfsys@transformshift{4.541403in}{4.580980in}%
\pgfsys@useobject{currentmarker}{}%
\end{pgfscope}%
\end{pgfscope}%
\begin{pgfscope}%
\pgfpathrectangle{\pgfqpoint{0.100000in}{2.413063in}}{\pgfqpoint{5.037500in}{3.427208in}}%
\pgfusepath{clip}%
\pgfsetrectcap%
\pgfsetroundjoin%
\pgfsetlinewidth{1.505625pt}%
\definecolor{currentstroke}{rgb}{0.678431,1.000000,0.184314}%
\pgfsetstrokecolor{currentstroke}%
\pgfsetstrokeopacity{0.500000}%
\pgfsetdash{}{0pt}%
\pgfpathmoveto{\pgfqpoint{4.113913in}{4.555385in}}%
\pgfusepath{stroke}%
\end{pgfscope}%
\begin{pgfscope}%
\pgfpathrectangle{\pgfqpoint{0.100000in}{2.413063in}}{\pgfqpoint{5.037500in}{3.427208in}}%
\pgfusepath{clip}%
\pgfsetbuttcap%
\pgfsetroundjoin%
\definecolor{currentfill}{rgb}{0.678431,1.000000,0.184314}%
\pgfsetfillcolor{currentfill}%
\pgfsetfillopacity{0.500000}%
\pgfsetlinewidth{0.250937pt}%
\definecolor{currentstroke}{rgb}{0.000000,0.000000,0.000000}%
\pgfsetstrokecolor{currentstroke}%
\pgfsetstrokeopacity{0.500000}%
\pgfsetdash{}{0pt}%
\pgfsys@defobject{currentmarker}{\pgfqpoint{-0.011111in}{-0.011111in}}{\pgfqpoint{0.011111in}{0.011111in}}{%
\pgfpathmoveto{\pgfqpoint{0.000000in}{-0.011111in}}%
\pgfpathcurveto{\pgfqpoint{0.002947in}{-0.011111in}}{\pgfqpoint{0.005773in}{-0.009940in}}{\pgfqpoint{0.007857in}{-0.007857in}}%
\pgfpathcurveto{\pgfqpoint{0.009940in}{-0.005773in}}{\pgfqpoint{0.011111in}{-0.002947in}}{\pgfqpoint{0.011111in}{0.000000in}}%
\pgfpathcurveto{\pgfqpoint{0.011111in}{0.002947in}}{\pgfqpoint{0.009940in}{0.005773in}}{\pgfqpoint{0.007857in}{0.007857in}}%
\pgfpathcurveto{\pgfqpoint{0.005773in}{0.009940in}}{\pgfqpoint{0.002947in}{0.011111in}}{\pgfqpoint{0.000000in}{0.011111in}}%
\pgfpathcurveto{\pgfqpoint{-0.002947in}{0.011111in}}{\pgfqpoint{-0.005773in}{0.009940in}}{\pgfqpoint{-0.007857in}{0.007857in}}%
\pgfpathcurveto{\pgfqpoint{-0.009940in}{0.005773in}}{\pgfqpoint{-0.011111in}{0.002947in}}{\pgfqpoint{-0.011111in}{0.000000in}}%
\pgfpathcurveto{\pgfqpoint{-0.011111in}{-0.002947in}}{\pgfqpoint{-0.009940in}{-0.005773in}}{\pgfqpoint{-0.007857in}{-0.007857in}}%
\pgfpathcurveto{\pgfqpoint{-0.005773in}{-0.009940in}}{\pgfqpoint{-0.002947in}{-0.011111in}}{\pgfqpoint{0.000000in}{-0.011111in}}%
\pgfpathclose%
\pgfusepath{stroke,fill}%
}%
\begin{pgfscope}%
\pgfsys@transformshift{4.113913in}{4.555385in}%
\pgfsys@useobject{currentmarker}{}%
\end{pgfscope}%
\end{pgfscope}%
\begin{pgfscope}%
\pgfpathrectangle{\pgfqpoint{0.100000in}{2.413063in}}{\pgfqpoint{5.037500in}{3.427208in}}%
\pgfusepath{clip}%
\pgfsetrectcap%
\pgfsetroundjoin%
\pgfsetlinewidth{1.505625pt}%
\definecolor{currentstroke}{rgb}{0.678431,1.000000,0.184314}%
\pgfsetstrokecolor{currentstroke}%
\pgfsetstrokeopacity{0.500000}%
\pgfsetdash{}{0pt}%
\pgfpathmoveto{\pgfqpoint{4.466278in}{4.609780in}}%
\pgfusepath{stroke}%
\end{pgfscope}%
\begin{pgfscope}%
\pgfpathrectangle{\pgfqpoint{0.100000in}{2.413063in}}{\pgfqpoint{5.037500in}{3.427208in}}%
\pgfusepath{clip}%
\pgfsetbuttcap%
\pgfsetroundjoin%
\definecolor{currentfill}{rgb}{0.678431,1.000000,0.184314}%
\pgfsetfillcolor{currentfill}%
\pgfsetfillopacity{0.500000}%
\pgfsetlinewidth{0.250937pt}%
\definecolor{currentstroke}{rgb}{0.000000,0.000000,0.000000}%
\pgfsetstrokecolor{currentstroke}%
\pgfsetstrokeopacity{0.500000}%
\pgfsetdash{}{0pt}%
\pgfsys@defobject{currentmarker}{\pgfqpoint{-0.005556in}{-0.005556in}}{\pgfqpoint{0.005556in}{0.005556in}}{%
\pgfpathmoveto{\pgfqpoint{0.000000in}{-0.005556in}}%
\pgfpathcurveto{\pgfqpoint{0.001473in}{-0.005556in}}{\pgfqpoint{0.002887in}{-0.004970in}}{\pgfqpoint{0.003928in}{-0.003928in}}%
\pgfpathcurveto{\pgfqpoint{0.004970in}{-0.002887in}}{\pgfqpoint{0.005556in}{-0.001473in}}{\pgfqpoint{0.005556in}{0.000000in}}%
\pgfpathcurveto{\pgfqpoint{0.005556in}{0.001473in}}{\pgfqpoint{0.004970in}{0.002887in}}{\pgfqpoint{0.003928in}{0.003928in}}%
\pgfpathcurveto{\pgfqpoint{0.002887in}{0.004970in}}{\pgfqpoint{0.001473in}{0.005556in}}{\pgfqpoint{0.000000in}{0.005556in}}%
\pgfpathcurveto{\pgfqpoint{-0.001473in}{0.005556in}}{\pgfqpoint{-0.002887in}{0.004970in}}{\pgfqpoint{-0.003928in}{0.003928in}}%
\pgfpathcurveto{\pgfqpoint{-0.004970in}{0.002887in}}{\pgfqpoint{-0.005556in}{0.001473in}}{\pgfqpoint{-0.005556in}{0.000000in}}%
\pgfpathcurveto{\pgfqpoint{-0.005556in}{-0.001473in}}{\pgfqpoint{-0.004970in}{-0.002887in}}{\pgfqpoint{-0.003928in}{-0.003928in}}%
\pgfpathcurveto{\pgfqpoint{-0.002887in}{-0.004970in}}{\pgfqpoint{-0.001473in}{-0.005556in}}{\pgfqpoint{0.000000in}{-0.005556in}}%
\pgfpathclose%
\pgfusepath{stroke,fill}%
}%
\begin{pgfscope}%
\pgfsys@transformshift{4.466278in}{4.609780in}%
\pgfsys@useobject{currentmarker}{}%
\end{pgfscope}%
\end{pgfscope}%
\begin{pgfscope}%
\pgfpathrectangle{\pgfqpoint{0.100000in}{2.413063in}}{\pgfqpoint{5.037500in}{3.427208in}}%
\pgfusepath{clip}%
\pgfsetrectcap%
\pgfsetroundjoin%
\pgfsetlinewidth{1.505625pt}%
\definecolor{currentstroke}{rgb}{0.678431,1.000000,0.184314}%
\pgfsetstrokecolor{currentstroke}%
\pgfsetstrokeopacity{0.500000}%
\pgfsetdash{}{0pt}%
\pgfpathmoveto{\pgfqpoint{4.462981in}{4.735541in}}%
\pgfusepath{stroke}%
\end{pgfscope}%
\begin{pgfscope}%
\pgfpathrectangle{\pgfqpoint{0.100000in}{2.413063in}}{\pgfqpoint{5.037500in}{3.427208in}}%
\pgfusepath{clip}%
\pgfsetbuttcap%
\pgfsetroundjoin%
\definecolor{currentfill}{rgb}{0.678431,1.000000,0.184314}%
\pgfsetfillcolor{currentfill}%
\pgfsetfillopacity{0.500000}%
\pgfsetlinewidth{0.250937pt}%
\definecolor{currentstroke}{rgb}{0.000000,0.000000,0.000000}%
\pgfsetstrokecolor{currentstroke}%
\pgfsetstrokeopacity{0.500000}%
\pgfsetdash{}{0pt}%
\pgfsys@defobject{currentmarker}{\pgfqpoint{-0.016667in}{-0.016667in}}{\pgfqpoint{0.016667in}{0.016667in}}{%
\pgfpathmoveto{\pgfqpoint{0.000000in}{-0.016667in}}%
\pgfpathcurveto{\pgfqpoint{0.004420in}{-0.016667in}}{\pgfqpoint{0.008660in}{-0.014911in}}{\pgfqpoint{0.011785in}{-0.011785in}}%
\pgfpathcurveto{\pgfqpoint{0.014911in}{-0.008660in}}{\pgfqpoint{0.016667in}{-0.004420in}}{\pgfqpoint{0.016667in}{0.000000in}}%
\pgfpathcurveto{\pgfqpoint{0.016667in}{0.004420in}}{\pgfqpoint{0.014911in}{0.008660in}}{\pgfqpoint{0.011785in}{0.011785in}}%
\pgfpathcurveto{\pgfqpoint{0.008660in}{0.014911in}}{\pgfqpoint{0.004420in}{0.016667in}}{\pgfqpoint{0.000000in}{0.016667in}}%
\pgfpathcurveto{\pgfqpoint{-0.004420in}{0.016667in}}{\pgfqpoint{-0.008660in}{0.014911in}}{\pgfqpoint{-0.011785in}{0.011785in}}%
\pgfpathcurveto{\pgfqpoint{-0.014911in}{0.008660in}}{\pgfqpoint{-0.016667in}{0.004420in}}{\pgfqpoint{-0.016667in}{0.000000in}}%
\pgfpathcurveto{\pgfqpoint{-0.016667in}{-0.004420in}}{\pgfqpoint{-0.014911in}{-0.008660in}}{\pgfqpoint{-0.011785in}{-0.011785in}}%
\pgfpathcurveto{\pgfqpoint{-0.008660in}{-0.014911in}}{\pgfqpoint{-0.004420in}{-0.016667in}}{\pgfqpoint{0.000000in}{-0.016667in}}%
\pgfpathclose%
\pgfusepath{stroke,fill}%
}%
\begin{pgfscope}%
\pgfsys@transformshift{4.462981in}{4.735541in}%
\pgfsys@useobject{currentmarker}{}%
\end{pgfscope}%
\end{pgfscope}%
\begin{pgfscope}%
\pgfpathrectangle{\pgfqpoint{0.100000in}{2.413063in}}{\pgfqpoint{5.037500in}{3.427208in}}%
\pgfusepath{clip}%
\pgfsetrectcap%
\pgfsetroundjoin%
\pgfsetlinewidth{1.505625pt}%
\definecolor{currentstroke}{rgb}{0.678431,1.000000,0.184314}%
\pgfsetstrokecolor{currentstroke}%
\pgfsetstrokeopacity{0.500000}%
\pgfsetdash{}{0pt}%
\pgfpathmoveto{\pgfqpoint{4.290048in}{4.628094in}}%
\pgfusepath{stroke}%
\end{pgfscope}%
\begin{pgfscope}%
\pgfpathrectangle{\pgfqpoint{0.100000in}{2.413063in}}{\pgfqpoint{5.037500in}{3.427208in}}%
\pgfusepath{clip}%
\pgfsetbuttcap%
\pgfsetroundjoin%
\definecolor{currentfill}{rgb}{0.678431,1.000000,0.184314}%
\pgfsetfillcolor{currentfill}%
\pgfsetfillopacity{0.500000}%
\pgfsetlinewidth{0.250937pt}%
\definecolor{currentstroke}{rgb}{0.000000,0.000000,0.000000}%
\pgfsetstrokecolor{currentstroke}%
\pgfsetstrokeopacity{0.500000}%
\pgfsetdash{}{0pt}%
\pgfsys@defobject{currentmarker}{\pgfqpoint{-0.013889in}{-0.013889in}}{\pgfqpoint{0.013889in}{0.013889in}}{%
\pgfpathmoveto{\pgfqpoint{0.000000in}{-0.013889in}}%
\pgfpathcurveto{\pgfqpoint{0.003683in}{-0.013889in}}{\pgfqpoint{0.007216in}{-0.012425in}}{\pgfqpoint{0.009821in}{-0.009821in}}%
\pgfpathcurveto{\pgfqpoint{0.012425in}{-0.007216in}}{\pgfqpoint{0.013889in}{-0.003683in}}{\pgfqpoint{0.013889in}{0.000000in}}%
\pgfpathcurveto{\pgfqpoint{0.013889in}{0.003683in}}{\pgfqpoint{0.012425in}{0.007216in}}{\pgfqpoint{0.009821in}{0.009821in}}%
\pgfpathcurveto{\pgfqpoint{0.007216in}{0.012425in}}{\pgfqpoint{0.003683in}{0.013889in}}{\pgfqpoint{0.000000in}{0.013889in}}%
\pgfpathcurveto{\pgfqpoint{-0.003683in}{0.013889in}}{\pgfqpoint{-0.007216in}{0.012425in}}{\pgfqpoint{-0.009821in}{0.009821in}}%
\pgfpathcurveto{\pgfqpoint{-0.012425in}{0.007216in}}{\pgfqpoint{-0.013889in}{0.003683in}}{\pgfqpoint{-0.013889in}{0.000000in}}%
\pgfpathcurveto{\pgfqpoint{-0.013889in}{-0.003683in}}{\pgfqpoint{-0.012425in}{-0.007216in}}{\pgfqpoint{-0.009821in}{-0.009821in}}%
\pgfpathcurveto{\pgfqpoint{-0.007216in}{-0.012425in}}{\pgfqpoint{-0.003683in}{-0.013889in}}{\pgfqpoint{0.000000in}{-0.013889in}}%
\pgfpathclose%
\pgfusepath{stroke,fill}%
}%
\begin{pgfscope}%
\pgfsys@transformshift{4.290048in}{4.628094in}%
\pgfsys@useobject{currentmarker}{}%
\end{pgfscope}%
\end{pgfscope}%
\begin{pgfscope}%
\pgfpathrectangle{\pgfqpoint{0.100000in}{2.413063in}}{\pgfqpoint{5.037500in}{3.427208in}}%
\pgfusepath{clip}%
\pgfsetrectcap%
\pgfsetroundjoin%
\pgfsetlinewidth{1.505625pt}%
\definecolor{currentstroke}{rgb}{0.678431,1.000000,0.184314}%
\pgfsetstrokecolor{currentstroke}%
\pgfsetstrokeopacity{0.500000}%
\pgfsetdash{}{0pt}%
\pgfpathmoveto{\pgfqpoint{4.353156in}{4.693901in}}%
\pgfusepath{stroke}%
\end{pgfscope}%
\begin{pgfscope}%
\pgfpathrectangle{\pgfqpoint{0.100000in}{2.413063in}}{\pgfqpoint{5.037500in}{3.427208in}}%
\pgfusepath{clip}%
\pgfsetbuttcap%
\pgfsetroundjoin%
\definecolor{currentfill}{rgb}{0.678431,1.000000,0.184314}%
\pgfsetfillcolor{currentfill}%
\pgfsetfillopacity{0.500000}%
\pgfsetlinewidth{0.250937pt}%
\definecolor{currentstroke}{rgb}{0.000000,0.000000,0.000000}%
\pgfsetstrokecolor{currentstroke}%
\pgfsetstrokeopacity{0.500000}%
\pgfsetdash{}{0pt}%
\pgfsys@defobject{currentmarker}{\pgfqpoint{-0.030556in}{-0.030556in}}{\pgfqpoint{0.030556in}{0.030556in}}{%
\pgfpathmoveto{\pgfqpoint{0.000000in}{-0.030556in}}%
\pgfpathcurveto{\pgfqpoint{0.008103in}{-0.030556in}}{\pgfqpoint{0.015876in}{-0.027336in}}{\pgfqpoint{0.021606in}{-0.021606in}}%
\pgfpathcurveto{\pgfqpoint{0.027336in}{-0.015876in}}{\pgfqpoint{0.030556in}{-0.008103in}}{\pgfqpoint{0.030556in}{0.000000in}}%
\pgfpathcurveto{\pgfqpoint{0.030556in}{0.008103in}}{\pgfqpoint{0.027336in}{0.015876in}}{\pgfqpoint{0.021606in}{0.021606in}}%
\pgfpathcurveto{\pgfqpoint{0.015876in}{0.027336in}}{\pgfqpoint{0.008103in}{0.030556in}}{\pgfqpoint{0.000000in}{0.030556in}}%
\pgfpathcurveto{\pgfqpoint{-0.008103in}{0.030556in}}{\pgfqpoint{-0.015876in}{0.027336in}}{\pgfqpoint{-0.021606in}{0.021606in}}%
\pgfpathcurveto{\pgfqpoint{-0.027336in}{0.015876in}}{\pgfqpoint{-0.030556in}{0.008103in}}{\pgfqpoint{-0.030556in}{0.000000in}}%
\pgfpathcurveto{\pgfqpoint{-0.030556in}{-0.008103in}}{\pgfqpoint{-0.027336in}{-0.015876in}}{\pgfqpoint{-0.021606in}{-0.021606in}}%
\pgfpathcurveto{\pgfqpoint{-0.015876in}{-0.027336in}}{\pgfqpoint{-0.008103in}{-0.030556in}}{\pgfqpoint{0.000000in}{-0.030556in}}%
\pgfpathclose%
\pgfusepath{stroke,fill}%
}%
\begin{pgfscope}%
\pgfsys@transformshift{4.353156in}{4.693901in}%
\pgfsys@useobject{currentmarker}{}%
\end{pgfscope}%
\end{pgfscope}%
\begin{pgfscope}%
\pgfpathrectangle{\pgfqpoint{0.100000in}{2.413063in}}{\pgfqpoint{5.037500in}{3.427208in}}%
\pgfusepath{clip}%
\pgfsetrectcap%
\pgfsetroundjoin%
\pgfsetlinewidth{1.505625pt}%
\definecolor{currentstroke}{rgb}{0.678431,1.000000,0.184314}%
\pgfsetstrokecolor{currentstroke}%
\pgfsetstrokeopacity{0.500000}%
\pgfsetdash{}{0pt}%
\pgfpathmoveto{\pgfqpoint{4.406146in}{4.553374in}}%
\pgfusepath{stroke}%
\end{pgfscope}%
\begin{pgfscope}%
\pgfpathrectangle{\pgfqpoint{0.100000in}{2.413063in}}{\pgfqpoint{5.037500in}{3.427208in}}%
\pgfusepath{clip}%
\pgfsetbuttcap%
\pgfsetroundjoin%
\definecolor{currentfill}{rgb}{0.678431,1.000000,0.184314}%
\pgfsetfillcolor{currentfill}%
\pgfsetfillopacity{0.500000}%
\pgfsetlinewidth{0.250937pt}%
\definecolor{currentstroke}{rgb}{0.000000,0.000000,0.000000}%
\pgfsetstrokecolor{currentstroke}%
\pgfsetstrokeopacity{0.500000}%
\pgfsetdash{}{0pt}%
\pgfsys@defobject{currentmarker}{\pgfqpoint{-0.013889in}{-0.013889in}}{\pgfqpoint{0.013889in}{0.013889in}}{%
\pgfpathmoveto{\pgfqpoint{0.000000in}{-0.013889in}}%
\pgfpathcurveto{\pgfqpoint{0.003683in}{-0.013889in}}{\pgfqpoint{0.007216in}{-0.012425in}}{\pgfqpoint{0.009821in}{-0.009821in}}%
\pgfpathcurveto{\pgfqpoint{0.012425in}{-0.007216in}}{\pgfqpoint{0.013889in}{-0.003683in}}{\pgfqpoint{0.013889in}{0.000000in}}%
\pgfpathcurveto{\pgfqpoint{0.013889in}{0.003683in}}{\pgfqpoint{0.012425in}{0.007216in}}{\pgfqpoint{0.009821in}{0.009821in}}%
\pgfpathcurveto{\pgfqpoint{0.007216in}{0.012425in}}{\pgfqpoint{0.003683in}{0.013889in}}{\pgfqpoint{0.000000in}{0.013889in}}%
\pgfpathcurveto{\pgfqpoint{-0.003683in}{0.013889in}}{\pgfqpoint{-0.007216in}{0.012425in}}{\pgfqpoint{-0.009821in}{0.009821in}}%
\pgfpathcurveto{\pgfqpoint{-0.012425in}{0.007216in}}{\pgfqpoint{-0.013889in}{0.003683in}}{\pgfqpoint{-0.013889in}{0.000000in}}%
\pgfpathcurveto{\pgfqpoint{-0.013889in}{-0.003683in}}{\pgfqpoint{-0.012425in}{-0.007216in}}{\pgfqpoint{-0.009821in}{-0.009821in}}%
\pgfpathcurveto{\pgfqpoint{-0.007216in}{-0.012425in}}{\pgfqpoint{-0.003683in}{-0.013889in}}{\pgfqpoint{0.000000in}{-0.013889in}}%
\pgfpathclose%
\pgfusepath{stroke,fill}%
}%
\begin{pgfscope}%
\pgfsys@transformshift{4.406146in}{4.553374in}%
\pgfsys@useobject{currentmarker}{}%
\end{pgfscope}%
\end{pgfscope}%
\begin{pgfscope}%
\pgfpathrectangle{\pgfqpoint{0.100000in}{2.413063in}}{\pgfqpoint{5.037500in}{3.427208in}}%
\pgfusepath{clip}%
\pgfsetrectcap%
\pgfsetroundjoin%
\pgfsetlinewidth{1.505625pt}%
\definecolor{currentstroke}{rgb}{0.000000,0.000000,1.000000}%
\pgfsetstrokecolor{currentstroke}%
\pgfsetstrokeopacity{0.500000}%
\pgfsetdash{}{0pt}%
\pgfpathmoveto{\pgfqpoint{4.807749in}{4.867735in}}%
\pgfusepath{stroke}%
\end{pgfscope}%
\begin{pgfscope}%
\pgfpathrectangle{\pgfqpoint{0.100000in}{2.413063in}}{\pgfqpoint{5.037500in}{3.427208in}}%
\pgfusepath{clip}%
\pgfsetbuttcap%
\pgfsetroundjoin%
\definecolor{currentfill}{rgb}{0.000000,0.000000,1.000000}%
\pgfsetfillcolor{currentfill}%
\pgfsetfillopacity{0.500000}%
\pgfsetlinewidth{0.250937pt}%
\definecolor{currentstroke}{rgb}{0.000000,0.000000,0.000000}%
\pgfsetstrokecolor{currentstroke}%
\pgfsetstrokeopacity{0.500000}%
\pgfsetdash{}{0pt}%
\pgfsys@defobject{currentmarker}{\pgfqpoint{-0.013889in}{-0.013889in}}{\pgfqpoint{0.013889in}{0.013889in}}{%
\pgfpathmoveto{\pgfqpoint{0.000000in}{-0.013889in}}%
\pgfpathcurveto{\pgfqpoint{0.003683in}{-0.013889in}}{\pgfqpoint{0.007216in}{-0.012425in}}{\pgfqpoint{0.009821in}{-0.009821in}}%
\pgfpathcurveto{\pgfqpoint{0.012425in}{-0.007216in}}{\pgfqpoint{0.013889in}{-0.003683in}}{\pgfqpoint{0.013889in}{0.000000in}}%
\pgfpathcurveto{\pgfqpoint{0.013889in}{0.003683in}}{\pgfqpoint{0.012425in}{0.007216in}}{\pgfqpoint{0.009821in}{0.009821in}}%
\pgfpathcurveto{\pgfqpoint{0.007216in}{0.012425in}}{\pgfqpoint{0.003683in}{0.013889in}}{\pgfqpoint{0.000000in}{0.013889in}}%
\pgfpathcurveto{\pgfqpoint{-0.003683in}{0.013889in}}{\pgfqpoint{-0.007216in}{0.012425in}}{\pgfqpoint{-0.009821in}{0.009821in}}%
\pgfpathcurveto{\pgfqpoint{-0.012425in}{0.007216in}}{\pgfqpoint{-0.013889in}{0.003683in}}{\pgfqpoint{-0.013889in}{0.000000in}}%
\pgfpathcurveto{\pgfqpoint{-0.013889in}{-0.003683in}}{\pgfqpoint{-0.012425in}{-0.007216in}}{\pgfqpoint{-0.009821in}{-0.009821in}}%
\pgfpathcurveto{\pgfqpoint{-0.007216in}{-0.012425in}}{\pgfqpoint{-0.003683in}{-0.013889in}}{\pgfqpoint{0.000000in}{-0.013889in}}%
\pgfpathclose%
\pgfusepath{stroke,fill}%
}%
\begin{pgfscope}%
\pgfsys@transformshift{4.807749in}{4.867735in}%
\pgfsys@useobject{currentmarker}{}%
\end{pgfscope}%
\end{pgfscope}%
\begin{pgfscope}%
\pgfpathrectangle{\pgfqpoint{0.100000in}{2.413063in}}{\pgfqpoint{5.037500in}{3.427208in}}%
\pgfusepath{clip}%
\pgfsetrectcap%
\pgfsetroundjoin%
\pgfsetlinewidth{1.505625pt}%
\definecolor{currentstroke}{rgb}{0.000000,0.000000,1.000000}%
\pgfsetstrokecolor{currentstroke}%
\pgfsetstrokeopacity{0.500000}%
\pgfsetdash{}{0pt}%
\pgfpathmoveto{\pgfqpoint{4.263845in}{3.685403in}}%
\pgfusepath{stroke}%
\end{pgfscope}%
\begin{pgfscope}%
\pgfpathrectangle{\pgfqpoint{0.100000in}{2.413063in}}{\pgfqpoint{5.037500in}{3.427208in}}%
\pgfusepath{clip}%
\pgfsetbuttcap%
\pgfsetroundjoin%
\definecolor{currentfill}{rgb}{0.000000,0.000000,1.000000}%
\pgfsetfillcolor{currentfill}%
\pgfsetfillopacity{0.500000}%
\pgfsetlinewidth{0.250937pt}%
\definecolor{currentstroke}{rgb}{0.000000,0.000000,0.000000}%
\pgfsetstrokecolor{currentstroke}%
\pgfsetstrokeopacity{0.500000}%
\pgfsetdash{}{0pt}%
\pgfsys@defobject{currentmarker}{\pgfqpoint{-0.025000in}{-0.025000in}}{\pgfqpoint{0.025000in}{0.025000in}}{%
\pgfpathmoveto{\pgfqpoint{0.000000in}{-0.025000in}}%
\pgfpathcurveto{\pgfqpoint{0.006630in}{-0.025000in}}{\pgfqpoint{0.012989in}{-0.022366in}}{\pgfqpoint{0.017678in}{-0.017678in}}%
\pgfpathcurveto{\pgfqpoint{0.022366in}{-0.012989in}}{\pgfqpoint{0.025000in}{-0.006630in}}{\pgfqpoint{0.025000in}{0.000000in}}%
\pgfpathcurveto{\pgfqpoint{0.025000in}{0.006630in}}{\pgfqpoint{0.022366in}{0.012989in}}{\pgfqpoint{0.017678in}{0.017678in}}%
\pgfpathcurveto{\pgfqpoint{0.012989in}{0.022366in}}{\pgfqpoint{0.006630in}{0.025000in}}{\pgfqpoint{0.000000in}{0.025000in}}%
\pgfpathcurveto{\pgfqpoint{-0.006630in}{0.025000in}}{\pgfqpoint{-0.012989in}{0.022366in}}{\pgfqpoint{-0.017678in}{0.017678in}}%
\pgfpathcurveto{\pgfqpoint{-0.022366in}{0.012989in}}{\pgfqpoint{-0.025000in}{0.006630in}}{\pgfqpoint{-0.025000in}{0.000000in}}%
\pgfpathcurveto{\pgfqpoint{-0.025000in}{-0.006630in}}{\pgfqpoint{-0.022366in}{-0.012989in}}{\pgfqpoint{-0.017678in}{-0.017678in}}%
\pgfpathcurveto{\pgfqpoint{-0.012989in}{-0.022366in}}{\pgfqpoint{-0.006630in}{-0.025000in}}{\pgfqpoint{0.000000in}{-0.025000in}}%
\pgfpathclose%
\pgfusepath{stroke,fill}%
}%
\begin{pgfscope}%
\pgfsys@transformshift{4.263845in}{3.685403in}%
\pgfsys@useobject{currentmarker}{}%
\end{pgfscope}%
\end{pgfscope}%
\begin{pgfscope}%
\pgfpathrectangle{\pgfqpoint{0.100000in}{2.413063in}}{\pgfqpoint{5.037500in}{3.427208in}}%
\pgfusepath{clip}%
\pgfsetrectcap%
\pgfsetroundjoin%
\pgfsetlinewidth{1.505625pt}%
\definecolor{currentstroke}{rgb}{0.000000,0.000000,1.000000}%
\pgfsetstrokecolor{currentstroke}%
\pgfsetstrokeopacity{0.500000}%
\pgfsetdash{}{0pt}%
\pgfpathmoveto{\pgfqpoint{4.137182in}{3.807175in}}%
\pgfusepath{stroke}%
\end{pgfscope}%
\begin{pgfscope}%
\pgfpathrectangle{\pgfqpoint{0.100000in}{2.413063in}}{\pgfqpoint{5.037500in}{3.427208in}}%
\pgfusepath{clip}%
\pgfsetbuttcap%
\pgfsetroundjoin%
\definecolor{currentfill}{rgb}{0.000000,0.000000,1.000000}%
\pgfsetfillcolor{currentfill}%
\pgfsetfillopacity{0.500000}%
\pgfsetlinewidth{0.250937pt}%
\definecolor{currentstroke}{rgb}{0.000000,0.000000,0.000000}%
\pgfsetstrokecolor{currentstroke}%
\pgfsetstrokeopacity{0.500000}%
\pgfsetdash{}{0pt}%
\pgfsys@defobject{currentmarker}{\pgfqpoint{-0.025000in}{-0.025000in}}{\pgfqpoint{0.025000in}{0.025000in}}{%
\pgfpathmoveto{\pgfqpoint{0.000000in}{-0.025000in}}%
\pgfpathcurveto{\pgfqpoint{0.006630in}{-0.025000in}}{\pgfqpoint{0.012989in}{-0.022366in}}{\pgfqpoint{0.017678in}{-0.017678in}}%
\pgfpathcurveto{\pgfqpoint{0.022366in}{-0.012989in}}{\pgfqpoint{0.025000in}{-0.006630in}}{\pgfqpoint{0.025000in}{0.000000in}}%
\pgfpathcurveto{\pgfqpoint{0.025000in}{0.006630in}}{\pgfqpoint{0.022366in}{0.012989in}}{\pgfqpoint{0.017678in}{0.017678in}}%
\pgfpathcurveto{\pgfqpoint{0.012989in}{0.022366in}}{\pgfqpoint{0.006630in}{0.025000in}}{\pgfqpoint{0.000000in}{0.025000in}}%
\pgfpathcurveto{\pgfqpoint{-0.006630in}{0.025000in}}{\pgfqpoint{-0.012989in}{0.022366in}}{\pgfqpoint{-0.017678in}{0.017678in}}%
\pgfpathcurveto{\pgfqpoint{-0.022366in}{0.012989in}}{\pgfqpoint{-0.025000in}{0.006630in}}{\pgfqpoint{-0.025000in}{0.000000in}}%
\pgfpathcurveto{\pgfqpoint{-0.025000in}{-0.006630in}}{\pgfqpoint{-0.022366in}{-0.012989in}}{\pgfqpoint{-0.017678in}{-0.017678in}}%
\pgfpathcurveto{\pgfqpoint{-0.012989in}{-0.022366in}}{\pgfqpoint{-0.006630in}{-0.025000in}}{\pgfqpoint{0.000000in}{-0.025000in}}%
\pgfpathclose%
\pgfusepath{stroke,fill}%
}%
\begin{pgfscope}%
\pgfsys@transformshift{4.137182in}{3.807175in}%
\pgfsys@useobject{currentmarker}{}%
\end{pgfscope}%
\end{pgfscope}%
\begin{pgfscope}%
\pgfpathrectangle{\pgfqpoint{0.100000in}{2.413063in}}{\pgfqpoint{5.037500in}{3.427208in}}%
\pgfusepath{clip}%
\pgfsetrectcap%
\pgfsetroundjoin%
\pgfsetlinewidth{1.505625pt}%
\definecolor{currentstroke}{rgb}{0.000000,0.000000,1.000000}%
\pgfsetstrokecolor{currentstroke}%
\pgfsetstrokeopacity{0.500000}%
\pgfsetdash{}{0pt}%
\pgfpathmoveto{\pgfqpoint{4.253237in}{3.849127in}}%
\pgfusepath{stroke}%
\end{pgfscope}%
\begin{pgfscope}%
\pgfpathrectangle{\pgfqpoint{0.100000in}{2.413063in}}{\pgfqpoint{5.037500in}{3.427208in}}%
\pgfusepath{clip}%
\pgfsetbuttcap%
\pgfsetroundjoin%
\definecolor{currentfill}{rgb}{0.000000,0.000000,1.000000}%
\pgfsetfillcolor{currentfill}%
\pgfsetfillopacity{0.500000}%
\pgfsetlinewidth{0.250937pt}%
\definecolor{currentstroke}{rgb}{0.000000,0.000000,0.000000}%
\pgfsetstrokecolor{currentstroke}%
\pgfsetstrokeopacity{0.500000}%
\pgfsetdash{}{0pt}%
\pgfsys@defobject{currentmarker}{\pgfqpoint{-0.022222in}{-0.022222in}}{\pgfqpoint{0.022222in}{0.022222in}}{%
\pgfpathmoveto{\pgfqpoint{0.000000in}{-0.022222in}}%
\pgfpathcurveto{\pgfqpoint{0.005893in}{-0.022222in}}{\pgfqpoint{0.011546in}{-0.019881in}}{\pgfqpoint{0.015713in}{-0.015713in}}%
\pgfpathcurveto{\pgfqpoint{0.019881in}{-0.011546in}}{\pgfqpoint{0.022222in}{-0.005893in}}{\pgfqpoint{0.022222in}{0.000000in}}%
\pgfpathcurveto{\pgfqpoint{0.022222in}{0.005893in}}{\pgfqpoint{0.019881in}{0.011546in}}{\pgfqpoint{0.015713in}{0.015713in}}%
\pgfpathcurveto{\pgfqpoint{0.011546in}{0.019881in}}{\pgfqpoint{0.005893in}{0.022222in}}{\pgfqpoint{0.000000in}{0.022222in}}%
\pgfpathcurveto{\pgfqpoint{-0.005893in}{0.022222in}}{\pgfqpoint{-0.011546in}{0.019881in}}{\pgfqpoint{-0.015713in}{0.015713in}}%
\pgfpathcurveto{\pgfqpoint{-0.019881in}{0.011546in}}{\pgfqpoint{-0.022222in}{0.005893in}}{\pgfqpoint{-0.022222in}{0.000000in}}%
\pgfpathcurveto{\pgfqpoint{-0.022222in}{-0.005893in}}{\pgfqpoint{-0.019881in}{-0.011546in}}{\pgfqpoint{-0.015713in}{-0.015713in}}%
\pgfpathcurveto{\pgfqpoint{-0.011546in}{-0.019881in}}{\pgfqpoint{-0.005893in}{-0.022222in}}{\pgfqpoint{0.000000in}{-0.022222in}}%
\pgfpathclose%
\pgfusepath{stroke,fill}%
}%
\begin{pgfscope}%
\pgfsys@transformshift{4.253237in}{3.849127in}%
\pgfsys@useobject{currentmarker}{}%
\end{pgfscope}%
\end{pgfscope}%
\begin{pgfscope}%
\pgfpathrectangle{\pgfqpoint{0.100000in}{2.413063in}}{\pgfqpoint{5.037500in}{3.427208in}}%
\pgfusepath{clip}%
\pgfsetrectcap%
\pgfsetroundjoin%
\pgfsetlinewidth{1.505625pt}%
\definecolor{currentstroke}{rgb}{0.000000,0.000000,1.000000}%
\pgfsetstrokecolor{currentstroke}%
\pgfsetstrokeopacity{0.500000}%
\pgfsetdash{}{0pt}%
\pgfpathmoveto{\pgfqpoint{3.975994in}{3.842770in}}%
\pgfusepath{stroke}%
\end{pgfscope}%
\begin{pgfscope}%
\pgfpathrectangle{\pgfqpoint{0.100000in}{2.413063in}}{\pgfqpoint{5.037500in}{3.427208in}}%
\pgfusepath{clip}%
\pgfsetbuttcap%
\pgfsetroundjoin%
\definecolor{currentfill}{rgb}{0.000000,0.000000,1.000000}%
\pgfsetfillcolor{currentfill}%
\pgfsetfillopacity{0.500000}%
\pgfsetlinewidth{0.250937pt}%
\definecolor{currentstroke}{rgb}{0.000000,0.000000,0.000000}%
\pgfsetstrokecolor{currentstroke}%
\pgfsetstrokeopacity{0.500000}%
\pgfsetdash{}{0pt}%
\pgfsys@defobject{currentmarker}{\pgfqpoint{-0.019444in}{-0.019444in}}{\pgfqpoint{0.019444in}{0.019444in}}{%
\pgfpathmoveto{\pgfqpoint{0.000000in}{-0.019444in}}%
\pgfpathcurveto{\pgfqpoint{0.005157in}{-0.019444in}}{\pgfqpoint{0.010103in}{-0.017396in}}{\pgfqpoint{0.013749in}{-0.013749in}}%
\pgfpathcurveto{\pgfqpoint{0.017396in}{-0.010103in}}{\pgfqpoint{0.019444in}{-0.005157in}}{\pgfqpoint{0.019444in}{0.000000in}}%
\pgfpathcurveto{\pgfqpoint{0.019444in}{0.005157in}}{\pgfqpoint{0.017396in}{0.010103in}}{\pgfqpoint{0.013749in}{0.013749in}}%
\pgfpathcurveto{\pgfqpoint{0.010103in}{0.017396in}}{\pgfqpoint{0.005157in}{0.019444in}}{\pgfqpoint{0.000000in}{0.019444in}}%
\pgfpathcurveto{\pgfqpoint{-0.005157in}{0.019444in}}{\pgfqpoint{-0.010103in}{0.017396in}}{\pgfqpoint{-0.013749in}{0.013749in}}%
\pgfpathcurveto{\pgfqpoint{-0.017396in}{0.010103in}}{\pgfqpoint{-0.019444in}{0.005157in}}{\pgfqpoint{-0.019444in}{0.000000in}}%
\pgfpathcurveto{\pgfqpoint{-0.019444in}{-0.005157in}}{\pgfqpoint{-0.017396in}{-0.010103in}}{\pgfqpoint{-0.013749in}{-0.013749in}}%
\pgfpathcurveto{\pgfqpoint{-0.010103in}{-0.017396in}}{\pgfqpoint{-0.005157in}{-0.019444in}}{\pgfqpoint{0.000000in}{-0.019444in}}%
\pgfpathclose%
\pgfusepath{stroke,fill}%
}%
\begin{pgfscope}%
\pgfsys@transformshift{3.975994in}{3.842770in}%
\pgfsys@useobject{currentmarker}{}%
\end{pgfscope}%
\end{pgfscope}%
\begin{pgfscope}%
\pgfpathrectangle{\pgfqpoint{0.100000in}{2.413063in}}{\pgfqpoint{5.037500in}{3.427208in}}%
\pgfusepath{clip}%
\pgfsetrectcap%
\pgfsetroundjoin%
\pgfsetlinewidth{1.505625pt}%
\definecolor{currentstroke}{rgb}{0.000000,0.000000,1.000000}%
\pgfsetstrokecolor{currentstroke}%
\pgfsetstrokeopacity{0.500000}%
\pgfsetdash{}{0pt}%
\pgfpathmoveto{\pgfqpoint{3.994269in}{3.885455in}}%
\pgfusepath{stroke}%
\end{pgfscope}%
\begin{pgfscope}%
\pgfpathrectangle{\pgfqpoint{0.100000in}{2.413063in}}{\pgfqpoint{5.037500in}{3.427208in}}%
\pgfusepath{clip}%
\pgfsetbuttcap%
\pgfsetroundjoin%
\definecolor{currentfill}{rgb}{0.000000,0.000000,1.000000}%
\pgfsetfillcolor{currentfill}%
\pgfsetfillopacity{0.500000}%
\pgfsetlinewidth{0.250937pt}%
\definecolor{currentstroke}{rgb}{0.000000,0.000000,0.000000}%
\pgfsetstrokecolor{currentstroke}%
\pgfsetstrokeopacity{0.500000}%
\pgfsetdash{}{0pt}%
\pgfsys@defobject{currentmarker}{\pgfqpoint{-0.019444in}{-0.019444in}}{\pgfqpoint{0.019444in}{0.019444in}}{%
\pgfpathmoveto{\pgfqpoint{0.000000in}{-0.019444in}}%
\pgfpathcurveto{\pgfqpoint{0.005157in}{-0.019444in}}{\pgfqpoint{0.010103in}{-0.017396in}}{\pgfqpoint{0.013749in}{-0.013749in}}%
\pgfpathcurveto{\pgfqpoint{0.017396in}{-0.010103in}}{\pgfqpoint{0.019444in}{-0.005157in}}{\pgfqpoint{0.019444in}{0.000000in}}%
\pgfpathcurveto{\pgfqpoint{0.019444in}{0.005157in}}{\pgfqpoint{0.017396in}{0.010103in}}{\pgfqpoint{0.013749in}{0.013749in}}%
\pgfpathcurveto{\pgfqpoint{0.010103in}{0.017396in}}{\pgfqpoint{0.005157in}{0.019444in}}{\pgfqpoint{0.000000in}{0.019444in}}%
\pgfpathcurveto{\pgfqpoint{-0.005157in}{0.019444in}}{\pgfqpoint{-0.010103in}{0.017396in}}{\pgfqpoint{-0.013749in}{0.013749in}}%
\pgfpathcurveto{\pgfqpoint{-0.017396in}{0.010103in}}{\pgfqpoint{-0.019444in}{0.005157in}}{\pgfqpoint{-0.019444in}{0.000000in}}%
\pgfpathcurveto{\pgfqpoint{-0.019444in}{-0.005157in}}{\pgfqpoint{-0.017396in}{-0.010103in}}{\pgfqpoint{-0.013749in}{-0.013749in}}%
\pgfpathcurveto{\pgfqpoint{-0.010103in}{-0.017396in}}{\pgfqpoint{-0.005157in}{-0.019444in}}{\pgfqpoint{0.000000in}{-0.019444in}}%
\pgfpathclose%
\pgfusepath{stroke,fill}%
}%
\begin{pgfscope}%
\pgfsys@transformshift{3.994269in}{3.885455in}%
\pgfsys@useobject{currentmarker}{}%
\end{pgfscope}%
\end{pgfscope}%
\begin{pgfscope}%
\pgfpathrectangle{\pgfqpoint{0.100000in}{2.413063in}}{\pgfqpoint{5.037500in}{3.427208in}}%
\pgfusepath{clip}%
\pgfsetrectcap%
\pgfsetroundjoin%
\pgfsetlinewidth{1.505625pt}%
\definecolor{currentstroke}{rgb}{0.000000,0.000000,1.000000}%
\pgfsetstrokecolor{currentstroke}%
\pgfsetstrokeopacity{0.500000}%
\pgfsetdash{}{0pt}%
\pgfpathmoveto{\pgfqpoint{4.196108in}{3.607309in}}%
\pgfusepath{stroke}%
\end{pgfscope}%
\begin{pgfscope}%
\pgfpathrectangle{\pgfqpoint{0.100000in}{2.413063in}}{\pgfqpoint{5.037500in}{3.427208in}}%
\pgfusepath{clip}%
\pgfsetbuttcap%
\pgfsetroundjoin%
\definecolor{currentfill}{rgb}{0.000000,0.000000,1.000000}%
\pgfsetfillcolor{currentfill}%
\pgfsetfillopacity{0.500000}%
\pgfsetlinewidth{0.250937pt}%
\definecolor{currentstroke}{rgb}{0.000000,0.000000,0.000000}%
\pgfsetstrokecolor{currentstroke}%
\pgfsetstrokeopacity{0.500000}%
\pgfsetdash{}{0pt}%
\pgfsys@defobject{currentmarker}{\pgfqpoint{-0.019444in}{-0.019444in}}{\pgfqpoint{0.019444in}{0.019444in}}{%
\pgfpathmoveto{\pgfqpoint{0.000000in}{-0.019444in}}%
\pgfpathcurveto{\pgfqpoint{0.005157in}{-0.019444in}}{\pgfqpoint{0.010103in}{-0.017396in}}{\pgfqpoint{0.013749in}{-0.013749in}}%
\pgfpathcurveto{\pgfqpoint{0.017396in}{-0.010103in}}{\pgfqpoint{0.019444in}{-0.005157in}}{\pgfqpoint{0.019444in}{0.000000in}}%
\pgfpathcurveto{\pgfqpoint{0.019444in}{0.005157in}}{\pgfqpoint{0.017396in}{0.010103in}}{\pgfqpoint{0.013749in}{0.013749in}}%
\pgfpathcurveto{\pgfqpoint{0.010103in}{0.017396in}}{\pgfqpoint{0.005157in}{0.019444in}}{\pgfqpoint{0.000000in}{0.019444in}}%
\pgfpathcurveto{\pgfqpoint{-0.005157in}{0.019444in}}{\pgfqpoint{-0.010103in}{0.017396in}}{\pgfqpoint{-0.013749in}{0.013749in}}%
\pgfpathcurveto{\pgfqpoint{-0.017396in}{0.010103in}}{\pgfqpoint{-0.019444in}{0.005157in}}{\pgfqpoint{-0.019444in}{0.000000in}}%
\pgfpathcurveto{\pgfqpoint{-0.019444in}{-0.005157in}}{\pgfqpoint{-0.017396in}{-0.010103in}}{\pgfqpoint{-0.013749in}{-0.013749in}}%
\pgfpathcurveto{\pgfqpoint{-0.010103in}{-0.017396in}}{\pgfqpoint{-0.005157in}{-0.019444in}}{\pgfqpoint{0.000000in}{-0.019444in}}%
\pgfpathclose%
\pgfusepath{stroke,fill}%
}%
\begin{pgfscope}%
\pgfsys@transformshift{4.196108in}{3.607309in}%
\pgfsys@useobject{currentmarker}{}%
\end{pgfscope}%
\end{pgfscope}%
\begin{pgfscope}%
\pgfpathrectangle{\pgfqpoint{0.100000in}{2.413063in}}{\pgfqpoint{5.037500in}{3.427208in}}%
\pgfusepath{clip}%
\pgfsetrectcap%
\pgfsetroundjoin%
\pgfsetlinewidth{1.505625pt}%
\definecolor{currentstroke}{rgb}{0.000000,0.000000,1.000000}%
\pgfsetstrokecolor{currentstroke}%
\pgfsetstrokeopacity{0.500000}%
\pgfsetdash{}{0pt}%
\pgfpathmoveto{\pgfqpoint{4.346076in}{3.806299in}}%
\pgfusepath{stroke}%
\end{pgfscope}%
\begin{pgfscope}%
\pgfpathrectangle{\pgfqpoint{0.100000in}{2.413063in}}{\pgfqpoint{5.037500in}{3.427208in}}%
\pgfusepath{clip}%
\pgfsetbuttcap%
\pgfsetroundjoin%
\definecolor{currentfill}{rgb}{0.000000,0.000000,1.000000}%
\pgfsetfillcolor{currentfill}%
\pgfsetfillopacity{0.500000}%
\pgfsetlinewidth{0.250937pt}%
\definecolor{currentstroke}{rgb}{0.000000,0.000000,0.000000}%
\pgfsetstrokecolor{currentstroke}%
\pgfsetstrokeopacity{0.500000}%
\pgfsetdash{}{0pt}%
\pgfsys@defobject{currentmarker}{\pgfqpoint{-0.016667in}{-0.016667in}}{\pgfqpoint{0.016667in}{0.016667in}}{%
\pgfpathmoveto{\pgfqpoint{0.000000in}{-0.016667in}}%
\pgfpathcurveto{\pgfqpoint{0.004420in}{-0.016667in}}{\pgfqpoint{0.008660in}{-0.014911in}}{\pgfqpoint{0.011785in}{-0.011785in}}%
\pgfpathcurveto{\pgfqpoint{0.014911in}{-0.008660in}}{\pgfqpoint{0.016667in}{-0.004420in}}{\pgfqpoint{0.016667in}{0.000000in}}%
\pgfpathcurveto{\pgfqpoint{0.016667in}{0.004420in}}{\pgfqpoint{0.014911in}{0.008660in}}{\pgfqpoint{0.011785in}{0.011785in}}%
\pgfpathcurveto{\pgfqpoint{0.008660in}{0.014911in}}{\pgfqpoint{0.004420in}{0.016667in}}{\pgfqpoint{0.000000in}{0.016667in}}%
\pgfpathcurveto{\pgfqpoint{-0.004420in}{0.016667in}}{\pgfqpoint{-0.008660in}{0.014911in}}{\pgfqpoint{-0.011785in}{0.011785in}}%
\pgfpathcurveto{\pgfqpoint{-0.014911in}{0.008660in}}{\pgfqpoint{-0.016667in}{0.004420in}}{\pgfqpoint{-0.016667in}{0.000000in}}%
\pgfpathcurveto{\pgfqpoint{-0.016667in}{-0.004420in}}{\pgfqpoint{-0.014911in}{-0.008660in}}{\pgfqpoint{-0.011785in}{-0.011785in}}%
\pgfpathcurveto{\pgfqpoint{-0.008660in}{-0.014911in}}{\pgfqpoint{-0.004420in}{-0.016667in}}{\pgfqpoint{0.000000in}{-0.016667in}}%
\pgfpathclose%
\pgfusepath{stroke,fill}%
}%
\begin{pgfscope}%
\pgfsys@transformshift{4.346076in}{3.806299in}%
\pgfsys@useobject{currentmarker}{}%
\end{pgfscope}%
\end{pgfscope}%
\begin{pgfscope}%
\pgfpathrectangle{\pgfqpoint{0.100000in}{2.413063in}}{\pgfqpoint{5.037500in}{3.427208in}}%
\pgfusepath{clip}%
\pgfsetrectcap%
\pgfsetroundjoin%
\pgfsetlinewidth{1.505625pt}%
\definecolor{currentstroke}{rgb}{0.000000,0.000000,1.000000}%
\pgfsetstrokecolor{currentstroke}%
\pgfsetstrokeopacity{0.500000}%
\pgfsetdash{}{0pt}%
\pgfpathmoveto{\pgfqpoint{4.036398in}{3.902925in}}%
\pgfusepath{stroke}%
\end{pgfscope}%
\begin{pgfscope}%
\pgfpathrectangle{\pgfqpoint{0.100000in}{2.413063in}}{\pgfqpoint{5.037500in}{3.427208in}}%
\pgfusepath{clip}%
\pgfsetbuttcap%
\pgfsetroundjoin%
\definecolor{currentfill}{rgb}{0.000000,0.000000,1.000000}%
\pgfsetfillcolor{currentfill}%
\pgfsetfillopacity{0.500000}%
\pgfsetlinewidth{0.250937pt}%
\definecolor{currentstroke}{rgb}{0.000000,0.000000,0.000000}%
\pgfsetstrokecolor{currentstroke}%
\pgfsetstrokeopacity{0.500000}%
\pgfsetdash{}{0pt}%
\pgfsys@defobject{currentmarker}{\pgfqpoint{-0.030556in}{-0.030556in}}{\pgfqpoint{0.030556in}{0.030556in}}{%
\pgfpathmoveto{\pgfqpoint{0.000000in}{-0.030556in}}%
\pgfpathcurveto{\pgfqpoint{0.008103in}{-0.030556in}}{\pgfqpoint{0.015876in}{-0.027336in}}{\pgfqpoint{0.021606in}{-0.021606in}}%
\pgfpathcurveto{\pgfqpoint{0.027336in}{-0.015876in}}{\pgfqpoint{0.030556in}{-0.008103in}}{\pgfqpoint{0.030556in}{0.000000in}}%
\pgfpathcurveto{\pgfqpoint{0.030556in}{0.008103in}}{\pgfqpoint{0.027336in}{0.015876in}}{\pgfqpoint{0.021606in}{0.021606in}}%
\pgfpathcurveto{\pgfqpoint{0.015876in}{0.027336in}}{\pgfqpoint{0.008103in}{0.030556in}}{\pgfqpoint{0.000000in}{0.030556in}}%
\pgfpathcurveto{\pgfqpoint{-0.008103in}{0.030556in}}{\pgfqpoint{-0.015876in}{0.027336in}}{\pgfqpoint{-0.021606in}{0.021606in}}%
\pgfpathcurveto{\pgfqpoint{-0.027336in}{0.015876in}}{\pgfqpoint{-0.030556in}{0.008103in}}{\pgfqpoint{-0.030556in}{0.000000in}}%
\pgfpathcurveto{\pgfqpoint{-0.030556in}{-0.008103in}}{\pgfqpoint{-0.027336in}{-0.015876in}}{\pgfqpoint{-0.021606in}{-0.021606in}}%
\pgfpathcurveto{\pgfqpoint{-0.015876in}{-0.027336in}}{\pgfqpoint{-0.008103in}{-0.030556in}}{\pgfqpoint{0.000000in}{-0.030556in}}%
\pgfpathclose%
\pgfusepath{stroke,fill}%
}%
\begin{pgfscope}%
\pgfsys@transformshift{4.036398in}{3.902925in}%
\pgfsys@useobject{currentmarker}{}%
\end{pgfscope}%
\end{pgfscope}%
\begin{pgfscope}%
\pgfpathrectangle{\pgfqpoint{0.100000in}{2.413063in}}{\pgfqpoint{5.037500in}{3.427208in}}%
\pgfusepath{clip}%
\pgfsetrectcap%
\pgfsetroundjoin%
\pgfsetlinewidth{1.505625pt}%
\definecolor{currentstroke}{rgb}{0.000000,0.000000,1.000000}%
\pgfsetstrokecolor{currentstroke}%
\pgfsetstrokeopacity{0.500000}%
\pgfsetdash{}{0pt}%
\pgfpathmoveto{\pgfqpoint{4.204247in}{3.808425in}}%
\pgfusepath{stroke}%
\end{pgfscope}%
\begin{pgfscope}%
\pgfpathrectangle{\pgfqpoint{0.100000in}{2.413063in}}{\pgfqpoint{5.037500in}{3.427208in}}%
\pgfusepath{clip}%
\pgfsetbuttcap%
\pgfsetroundjoin%
\definecolor{currentfill}{rgb}{0.000000,0.000000,1.000000}%
\pgfsetfillcolor{currentfill}%
\pgfsetfillopacity{0.500000}%
\pgfsetlinewidth{0.250937pt}%
\definecolor{currentstroke}{rgb}{0.000000,0.000000,0.000000}%
\pgfsetstrokecolor{currentstroke}%
\pgfsetstrokeopacity{0.500000}%
\pgfsetdash{}{0pt}%
\pgfsys@defobject{currentmarker}{\pgfqpoint{-0.016667in}{-0.016667in}}{\pgfqpoint{0.016667in}{0.016667in}}{%
\pgfpathmoveto{\pgfqpoint{0.000000in}{-0.016667in}}%
\pgfpathcurveto{\pgfqpoint{0.004420in}{-0.016667in}}{\pgfqpoint{0.008660in}{-0.014911in}}{\pgfqpoint{0.011785in}{-0.011785in}}%
\pgfpathcurveto{\pgfqpoint{0.014911in}{-0.008660in}}{\pgfqpoint{0.016667in}{-0.004420in}}{\pgfqpoint{0.016667in}{0.000000in}}%
\pgfpathcurveto{\pgfqpoint{0.016667in}{0.004420in}}{\pgfqpoint{0.014911in}{0.008660in}}{\pgfqpoint{0.011785in}{0.011785in}}%
\pgfpathcurveto{\pgfqpoint{0.008660in}{0.014911in}}{\pgfqpoint{0.004420in}{0.016667in}}{\pgfqpoint{0.000000in}{0.016667in}}%
\pgfpathcurveto{\pgfqpoint{-0.004420in}{0.016667in}}{\pgfqpoint{-0.008660in}{0.014911in}}{\pgfqpoint{-0.011785in}{0.011785in}}%
\pgfpathcurveto{\pgfqpoint{-0.014911in}{0.008660in}}{\pgfqpoint{-0.016667in}{0.004420in}}{\pgfqpoint{-0.016667in}{0.000000in}}%
\pgfpathcurveto{\pgfqpoint{-0.016667in}{-0.004420in}}{\pgfqpoint{-0.014911in}{-0.008660in}}{\pgfqpoint{-0.011785in}{-0.011785in}}%
\pgfpathcurveto{\pgfqpoint{-0.008660in}{-0.014911in}}{\pgfqpoint{-0.004420in}{-0.016667in}}{\pgfqpoint{0.000000in}{-0.016667in}}%
\pgfpathclose%
\pgfusepath{stroke,fill}%
}%
\begin{pgfscope}%
\pgfsys@transformshift{4.204247in}{3.808425in}%
\pgfsys@useobject{currentmarker}{}%
\end{pgfscope}%
\end{pgfscope}%
\begin{pgfscope}%
\pgfpathrectangle{\pgfqpoint{0.100000in}{2.413063in}}{\pgfqpoint{5.037500in}{3.427208in}}%
\pgfusepath{clip}%
\pgfsetrectcap%
\pgfsetroundjoin%
\pgfsetlinewidth{1.505625pt}%
\definecolor{currentstroke}{rgb}{0.678431,1.000000,0.184314}%
\pgfsetstrokecolor{currentstroke}%
\pgfsetstrokeopacity{0.500000}%
\pgfsetdash{}{0pt}%
\pgfpathmoveto{\pgfqpoint{2.121809in}{4.897686in}}%
\pgfusepath{stroke}%
\end{pgfscope}%
\begin{pgfscope}%
\pgfpathrectangle{\pgfqpoint{0.100000in}{2.413063in}}{\pgfqpoint{5.037500in}{3.427208in}}%
\pgfusepath{clip}%
\pgfsetbuttcap%
\pgfsetroundjoin%
\definecolor{currentfill}{rgb}{0.678431,1.000000,0.184314}%
\pgfsetfillcolor{currentfill}%
\pgfsetfillopacity{0.500000}%
\pgfsetlinewidth{0.250937pt}%
\definecolor{currentstroke}{rgb}{0.000000,0.000000,0.000000}%
\pgfsetstrokecolor{currentstroke}%
\pgfsetstrokeopacity{0.500000}%
\pgfsetdash{}{0pt}%
\pgfsys@defobject{currentmarker}{\pgfqpoint{-0.022222in}{-0.022222in}}{\pgfqpoint{0.022222in}{0.022222in}}{%
\pgfpathmoveto{\pgfqpoint{0.000000in}{-0.022222in}}%
\pgfpathcurveto{\pgfqpoint{0.005893in}{-0.022222in}}{\pgfqpoint{0.011546in}{-0.019881in}}{\pgfqpoint{0.015713in}{-0.015713in}}%
\pgfpathcurveto{\pgfqpoint{0.019881in}{-0.011546in}}{\pgfqpoint{0.022222in}{-0.005893in}}{\pgfqpoint{0.022222in}{0.000000in}}%
\pgfpathcurveto{\pgfqpoint{0.022222in}{0.005893in}}{\pgfqpoint{0.019881in}{0.011546in}}{\pgfqpoint{0.015713in}{0.015713in}}%
\pgfpathcurveto{\pgfqpoint{0.011546in}{0.019881in}}{\pgfqpoint{0.005893in}{0.022222in}}{\pgfqpoint{0.000000in}{0.022222in}}%
\pgfpathcurveto{\pgfqpoint{-0.005893in}{0.022222in}}{\pgfqpoint{-0.011546in}{0.019881in}}{\pgfqpoint{-0.015713in}{0.015713in}}%
\pgfpathcurveto{\pgfqpoint{-0.019881in}{0.011546in}}{\pgfqpoint{-0.022222in}{0.005893in}}{\pgfqpoint{-0.022222in}{0.000000in}}%
\pgfpathcurveto{\pgfqpoint{-0.022222in}{-0.005893in}}{\pgfqpoint{-0.019881in}{-0.011546in}}{\pgfqpoint{-0.015713in}{-0.015713in}}%
\pgfpathcurveto{\pgfqpoint{-0.011546in}{-0.019881in}}{\pgfqpoint{-0.005893in}{-0.022222in}}{\pgfqpoint{0.000000in}{-0.022222in}}%
\pgfpathclose%
\pgfusepath{stroke,fill}%
}%
\begin{pgfscope}%
\pgfsys@transformshift{2.121809in}{4.897686in}%
\pgfsys@useobject{currentmarker}{}%
\end{pgfscope}%
\end{pgfscope}%
\begin{pgfscope}%
\pgfpathrectangle{\pgfqpoint{0.100000in}{2.413063in}}{\pgfqpoint{5.037500in}{3.427208in}}%
\pgfusepath{clip}%
\pgfsetrectcap%
\pgfsetroundjoin%
\pgfsetlinewidth{1.505625pt}%
\definecolor{currentstroke}{rgb}{0.678431,1.000000,0.184314}%
\pgfsetstrokecolor{currentstroke}%
\pgfsetstrokeopacity{0.500000}%
\pgfsetdash{}{0pt}%
\pgfpathmoveto{\pgfqpoint{2.661793in}{4.806727in}}%
\pgfusepath{stroke}%
\end{pgfscope}%
\begin{pgfscope}%
\pgfpathrectangle{\pgfqpoint{0.100000in}{2.413063in}}{\pgfqpoint{5.037500in}{3.427208in}}%
\pgfusepath{clip}%
\pgfsetbuttcap%
\pgfsetroundjoin%
\definecolor{currentfill}{rgb}{0.678431,1.000000,0.184314}%
\pgfsetfillcolor{currentfill}%
\pgfsetfillopacity{0.500000}%
\pgfsetlinewidth{0.250937pt}%
\definecolor{currentstroke}{rgb}{0.000000,0.000000,0.000000}%
\pgfsetstrokecolor{currentstroke}%
\pgfsetstrokeopacity{0.500000}%
\pgfsetdash{}{0pt}%
\pgfsys@defobject{currentmarker}{\pgfqpoint{-0.016667in}{-0.016667in}}{\pgfqpoint{0.016667in}{0.016667in}}{%
\pgfpathmoveto{\pgfqpoint{0.000000in}{-0.016667in}}%
\pgfpathcurveto{\pgfqpoint{0.004420in}{-0.016667in}}{\pgfqpoint{0.008660in}{-0.014911in}}{\pgfqpoint{0.011785in}{-0.011785in}}%
\pgfpathcurveto{\pgfqpoint{0.014911in}{-0.008660in}}{\pgfqpoint{0.016667in}{-0.004420in}}{\pgfqpoint{0.016667in}{0.000000in}}%
\pgfpathcurveto{\pgfqpoint{0.016667in}{0.004420in}}{\pgfqpoint{0.014911in}{0.008660in}}{\pgfqpoint{0.011785in}{0.011785in}}%
\pgfpathcurveto{\pgfqpoint{0.008660in}{0.014911in}}{\pgfqpoint{0.004420in}{0.016667in}}{\pgfqpoint{0.000000in}{0.016667in}}%
\pgfpathcurveto{\pgfqpoint{-0.004420in}{0.016667in}}{\pgfqpoint{-0.008660in}{0.014911in}}{\pgfqpoint{-0.011785in}{0.011785in}}%
\pgfpathcurveto{\pgfqpoint{-0.014911in}{0.008660in}}{\pgfqpoint{-0.016667in}{0.004420in}}{\pgfqpoint{-0.016667in}{0.000000in}}%
\pgfpathcurveto{\pgfqpoint{-0.016667in}{-0.004420in}}{\pgfqpoint{-0.014911in}{-0.008660in}}{\pgfqpoint{-0.011785in}{-0.011785in}}%
\pgfpathcurveto{\pgfqpoint{-0.008660in}{-0.014911in}}{\pgfqpoint{-0.004420in}{-0.016667in}}{\pgfqpoint{0.000000in}{-0.016667in}}%
\pgfpathclose%
\pgfusepath{stroke,fill}%
}%
\begin{pgfscope}%
\pgfsys@transformshift{2.661793in}{4.806727in}%
\pgfsys@useobject{currentmarker}{}%
\end{pgfscope}%
\end{pgfscope}%
\begin{pgfscope}%
\pgfpathrectangle{\pgfqpoint{0.100000in}{2.413063in}}{\pgfqpoint{5.037500in}{3.427208in}}%
\pgfusepath{clip}%
\pgfsetrectcap%
\pgfsetroundjoin%
\pgfsetlinewidth{1.505625pt}%
\definecolor{currentstroke}{rgb}{0.678431,1.000000,0.184314}%
\pgfsetstrokecolor{currentstroke}%
\pgfsetstrokeopacity{0.500000}%
\pgfsetdash{}{0pt}%
\pgfpathmoveto{\pgfqpoint{3.718137in}{3.874012in}}%
\pgfusepath{stroke}%
\end{pgfscope}%
\begin{pgfscope}%
\pgfpathrectangle{\pgfqpoint{0.100000in}{2.413063in}}{\pgfqpoint{5.037500in}{3.427208in}}%
\pgfusepath{clip}%
\pgfsetbuttcap%
\pgfsetroundjoin%
\definecolor{currentfill}{rgb}{0.678431,1.000000,0.184314}%
\pgfsetfillcolor{currentfill}%
\pgfsetfillopacity{0.500000}%
\pgfsetlinewidth{0.250937pt}%
\definecolor{currentstroke}{rgb}{0.000000,0.000000,0.000000}%
\pgfsetstrokecolor{currentstroke}%
\pgfsetstrokeopacity{0.500000}%
\pgfsetdash{}{0pt}%
\pgfsys@defobject{currentmarker}{\pgfqpoint{-0.013889in}{-0.013889in}}{\pgfqpoint{0.013889in}{0.013889in}}{%
\pgfpathmoveto{\pgfqpoint{0.000000in}{-0.013889in}}%
\pgfpathcurveto{\pgfqpoint{0.003683in}{-0.013889in}}{\pgfqpoint{0.007216in}{-0.012425in}}{\pgfqpoint{0.009821in}{-0.009821in}}%
\pgfpathcurveto{\pgfqpoint{0.012425in}{-0.007216in}}{\pgfqpoint{0.013889in}{-0.003683in}}{\pgfqpoint{0.013889in}{0.000000in}}%
\pgfpathcurveto{\pgfqpoint{0.013889in}{0.003683in}}{\pgfqpoint{0.012425in}{0.007216in}}{\pgfqpoint{0.009821in}{0.009821in}}%
\pgfpathcurveto{\pgfqpoint{0.007216in}{0.012425in}}{\pgfqpoint{0.003683in}{0.013889in}}{\pgfqpoint{0.000000in}{0.013889in}}%
\pgfpathcurveto{\pgfqpoint{-0.003683in}{0.013889in}}{\pgfqpoint{-0.007216in}{0.012425in}}{\pgfqpoint{-0.009821in}{0.009821in}}%
\pgfpathcurveto{\pgfqpoint{-0.012425in}{0.007216in}}{\pgfqpoint{-0.013889in}{0.003683in}}{\pgfqpoint{-0.013889in}{0.000000in}}%
\pgfpathcurveto{\pgfqpoint{-0.013889in}{-0.003683in}}{\pgfqpoint{-0.012425in}{-0.007216in}}{\pgfqpoint{-0.009821in}{-0.009821in}}%
\pgfpathcurveto{\pgfqpoint{-0.007216in}{-0.012425in}}{\pgfqpoint{-0.003683in}{-0.013889in}}{\pgfqpoint{0.000000in}{-0.013889in}}%
\pgfpathclose%
\pgfusepath{stroke,fill}%
}%
\begin{pgfscope}%
\pgfsys@transformshift{3.718137in}{3.874012in}%
\pgfsys@useobject{currentmarker}{}%
\end{pgfscope}%
\end{pgfscope}%
\begin{pgfscope}%
\pgfpathrectangle{\pgfqpoint{0.100000in}{2.413063in}}{\pgfqpoint{5.037500in}{3.427208in}}%
\pgfusepath{clip}%
\pgfsetrectcap%
\pgfsetroundjoin%
\pgfsetlinewidth{1.505625pt}%
\definecolor{currentstroke}{rgb}{0.501961,0.501961,0.501961}%
\pgfsetstrokecolor{currentstroke}%
\pgfsetstrokeopacity{0.500000}%
\pgfsetdash{}{0pt}%
\pgfpathmoveto{\pgfqpoint{3.511014in}{4.025917in}}%
\pgfusepath{stroke}%
\end{pgfscope}%
\begin{pgfscope}%
\pgfpathrectangle{\pgfqpoint{0.100000in}{2.413063in}}{\pgfqpoint{5.037500in}{3.427208in}}%
\pgfusepath{clip}%
\pgfsetbuttcap%
\pgfsetroundjoin%
\definecolor{currentfill}{rgb}{0.501961,0.501961,0.501961}%
\pgfsetfillcolor{currentfill}%
\pgfsetfillopacity{0.500000}%
\pgfsetlinewidth{0.250937pt}%
\definecolor{currentstroke}{rgb}{0.000000,0.000000,0.000000}%
\pgfsetstrokecolor{currentstroke}%
\pgfsetstrokeopacity{0.500000}%
\pgfsetdash{}{0pt}%
\pgfsys@defobject{currentmarker}{\pgfqpoint{-0.013889in}{-0.013889in}}{\pgfqpoint{0.013889in}{0.013889in}}{%
\pgfpathmoveto{\pgfqpoint{0.000000in}{-0.013889in}}%
\pgfpathcurveto{\pgfqpoint{0.003683in}{-0.013889in}}{\pgfqpoint{0.007216in}{-0.012425in}}{\pgfqpoint{0.009821in}{-0.009821in}}%
\pgfpathcurveto{\pgfqpoint{0.012425in}{-0.007216in}}{\pgfqpoint{0.013889in}{-0.003683in}}{\pgfqpoint{0.013889in}{0.000000in}}%
\pgfpathcurveto{\pgfqpoint{0.013889in}{0.003683in}}{\pgfqpoint{0.012425in}{0.007216in}}{\pgfqpoint{0.009821in}{0.009821in}}%
\pgfpathcurveto{\pgfqpoint{0.007216in}{0.012425in}}{\pgfqpoint{0.003683in}{0.013889in}}{\pgfqpoint{0.000000in}{0.013889in}}%
\pgfpathcurveto{\pgfqpoint{-0.003683in}{0.013889in}}{\pgfqpoint{-0.007216in}{0.012425in}}{\pgfqpoint{-0.009821in}{0.009821in}}%
\pgfpathcurveto{\pgfqpoint{-0.012425in}{0.007216in}}{\pgfqpoint{-0.013889in}{0.003683in}}{\pgfqpoint{-0.013889in}{0.000000in}}%
\pgfpathcurveto{\pgfqpoint{-0.013889in}{-0.003683in}}{\pgfqpoint{-0.012425in}{-0.007216in}}{\pgfqpoint{-0.009821in}{-0.009821in}}%
\pgfpathcurveto{\pgfqpoint{-0.007216in}{-0.012425in}}{\pgfqpoint{-0.003683in}{-0.013889in}}{\pgfqpoint{0.000000in}{-0.013889in}}%
\pgfpathclose%
\pgfusepath{stroke,fill}%
}%
\begin{pgfscope}%
\pgfsys@transformshift{3.511014in}{4.025917in}%
\pgfsys@useobject{currentmarker}{}%
\end{pgfscope}%
\end{pgfscope}%
\begin{pgfscope}%
\pgfpathrectangle{\pgfqpoint{0.100000in}{2.413063in}}{\pgfqpoint{5.037500in}{3.427208in}}%
\pgfusepath{clip}%
\pgfsetrectcap%
\pgfsetroundjoin%
\pgfsetlinewidth{1.505625pt}%
\definecolor{currentstroke}{rgb}{0.501961,0.501961,0.501961}%
\pgfsetstrokecolor{currentstroke}%
\pgfsetstrokeopacity{0.500000}%
\pgfsetdash{}{0pt}%
\pgfpathmoveto{\pgfqpoint{3.757359in}{3.891528in}}%
\pgfusepath{stroke}%
\end{pgfscope}%
\begin{pgfscope}%
\pgfpathrectangle{\pgfqpoint{0.100000in}{2.413063in}}{\pgfqpoint{5.037500in}{3.427208in}}%
\pgfusepath{clip}%
\pgfsetbuttcap%
\pgfsetroundjoin%
\definecolor{currentfill}{rgb}{0.501961,0.501961,0.501961}%
\pgfsetfillcolor{currentfill}%
\pgfsetfillopacity{0.500000}%
\pgfsetlinewidth{0.250937pt}%
\definecolor{currentstroke}{rgb}{0.000000,0.000000,0.000000}%
\pgfsetstrokecolor{currentstroke}%
\pgfsetstrokeopacity{0.500000}%
\pgfsetdash{}{0pt}%
\pgfsys@defobject{currentmarker}{\pgfqpoint{-0.013889in}{-0.013889in}}{\pgfqpoint{0.013889in}{0.013889in}}{%
\pgfpathmoveto{\pgfqpoint{0.000000in}{-0.013889in}}%
\pgfpathcurveto{\pgfqpoint{0.003683in}{-0.013889in}}{\pgfqpoint{0.007216in}{-0.012425in}}{\pgfqpoint{0.009821in}{-0.009821in}}%
\pgfpathcurveto{\pgfqpoint{0.012425in}{-0.007216in}}{\pgfqpoint{0.013889in}{-0.003683in}}{\pgfqpoint{0.013889in}{0.000000in}}%
\pgfpathcurveto{\pgfqpoint{0.013889in}{0.003683in}}{\pgfqpoint{0.012425in}{0.007216in}}{\pgfqpoint{0.009821in}{0.009821in}}%
\pgfpathcurveto{\pgfqpoint{0.007216in}{0.012425in}}{\pgfqpoint{0.003683in}{0.013889in}}{\pgfqpoint{0.000000in}{0.013889in}}%
\pgfpathcurveto{\pgfqpoint{-0.003683in}{0.013889in}}{\pgfqpoint{-0.007216in}{0.012425in}}{\pgfqpoint{-0.009821in}{0.009821in}}%
\pgfpathcurveto{\pgfqpoint{-0.012425in}{0.007216in}}{\pgfqpoint{-0.013889in}{0.003683in}}{\pgfqpoint{-0.013889in}{0.000000in}}%
\pgfpathcurveto{\pgfqpoint{-0.013889in}{-0.003683in}}{\pgfqpoint{-0.012425in}{-0.007216in}}{\pgfqpoint{-0.009821in}{-0.009821in}}%
\pgfpathcurveto{\pgfqpoint{-0.007216in}{-0.012425in}}{\pgfqpoint{-0.003683in}{-0.013889in}}{\pgfqpoint{0.000000in}{-0.013889in}}%
\pgfpathclose%
\pgfusepath{stroke,fill}%
}%
\begin{pgfscope}%
\pgfsys@transformshift{3.757359in}{3.891528in}%
\pgfsys@useobject{currentmarker}{}%
\end{pgfscope}%
\end{pgfscope}%
\begin{pgfscope}%
\pgfpathrectangle{\pgfqpoint{0.100000in}{2.413063in}}{\pgfqpoint{5.037500in}{3.427208in}}%
\pgfusepath{clip}%
\pgfsetrectcap%
\pgfsetroundjoin%
\pgfsetlinewidth{1.505625pt}%
\definecolor{currentstroke}{rgb}{0.501961,0.501961,0.501961}%
\pgfsetstrokecolor{currentstroke}%
\pgfsetstrokeopacity{0.500000}%
\pgfsetdash{}{0pt}%
\pgfpathmoveto{\pgfqpoint{3.383445in}{3.910583in}}%
\pgfusepath{stroke}%
\end{pgfscope}%
\begin{pgfscope}%
\pgfpathrectangle{\pgfqpoint{0.100000in}{2.413063in}}{\pgfqpoint{5.037500in}{3.427208in}}%
\pgfusepath{clip}%
\pgfsetbuttcap%
\pgfsetroundjoin%
\definecolor{currentfill}{rgb}{0.501961,0.501961,0.501961}%
\pgfsetfillcolor{currentfill}%
\pgfsetfillopacity{0.500000}%
\pgfsetlinewidth{0.250937pt}%
\definecolor{currentstroke}{rgb}{0.000000,0.000000,0.000000}%
\pgfsetstrokecolor{currentstroke}%
\pgfsetstrokeopacity{0.500000}%
\pgfsetdash{}{0pt}%
\pgfsys@defobject{currentmarker}{\pgfqpoint{-0.013889in}{-0.013889in}}{\pgfqpoint{0.013889in}{0.013889in}}{%
\pgfpathmoveto{\pgfqpoint{0.000000in}{-0.013889in}}%
\pgfpathcurveto{\pgfqpoint{0.003683in}{-0.013889in}}{\pgfqpoint{0.007216in}{-0.012425in}}{\pgfqpoint{0.009821in}{-0.009821in}}%
\pgfpathcurveto{\pgfqpoint{0.012425in}{-0.007216in}}{\pgfqpoint{0.013889in}{-0.003683in}}{\pgfqpoint{0.013889in}{0.000000in}}%
\pgfpathcurveto{\pgfqpoint{0.013889in}{0.003683in}}{\pgfqpoint{0.012425in}{0.007216in}}{\pgfqpoint{0.009821in}{0.009821in}}%
\pgfpathcurveto{\pgfqpoint{0.007216in}{0.012425in}}{\pgfqpoint{0.003683in}{0.013889in}}{\pgfqpoint{0.000000in}{0.013889in}}%
\pgfpathcurveto{\pgfqpoint{-0.003683in}{0.013889in}}{\pgfqpoint{-0.007216in}{0.012425in}}{\pgfqpoint{-0.009821in}{0.009821in}}%
\pgfpathcurveto{\pgfqpoint{-0.012425in}{0.007216in}}{\pgfqpoint{-0.013889in}{0.003683in}}{\pgfqpoint{-0.013889in}{0.000000in}}%
\pgfpathcurveto{\pgfqpoint{-0.013889in}{-0.003683in}}{\pgfqpoint{-0.012425in}{-0.007216in}}{\pgfqpoint{-0.009821in}{-0.009821in}}%
\pgfpathcurveto{\pgfqpoint{-0.007216in}{-0.012425in}}{\pgfqpoint{-0.003683in}{-0.013889in}}{\pgfqpoint{0.000000in}{-0.013889in}}%
\pgfpathclose%
\pgfusepath{stroke,fill}%
}%
\begin{pgfscope}%
\pgfsys@transformshift{3.383445in}{3.910583in}%
\pgfsys@useobject{currentmarker}{}%
\end{pgfscope}%
\end{pgfscope}%
\begin{pgfscope}%
\pgfpathrectangle{\pgfqpoint{0.100000in}{2.413063in}}{\pgfqpoint{5.037500in}{3.427208in}}%
\pgfusepath{clip}%
\pgfsetrectcap%
\pgfsetroundjoin%
\pgfsetlinewidth{1.505625pt}%
\definecolor{currentstroke}{rgb}{0.678431,1.000000,0.184314}%
\pgfsetstrokecolor{currentstroke}%
\pgfsetstrokeopacity{0.500000}%
\pgfsetdash{}{0pt}%
\pgfpathmoveto{\pgfqpoint{3.975097in}{4.053085in}}%
\pgfusepath{stroke}%
\end{pgfscope}%
\begin{pgfscope}%
\pgfpathrectangle{\pgfqpoint{0.100000in}{2.413063in}}{\pgfqpoint{5.037500in}{3.427208in}}%
\pgfusepath{clip}%
\pgfsetbuttcap%
\pgfsetroundjoin%
\definecolor{currentfill}{rgb}{0.678431,1.000000,0.184314}%
\pgfsetfillcolor{currentfill}%
\pgfsetfillopacity{0.500000}%
\pgfsetlinewidth{0.250937pt}%
\definecolor{currentstroke}{rgb}{0.000000,0.000000,0.000000}%
\pgfsetstrokecolor{currentstroke}%
\pgfsetstrokeopacity{0.500000}%
\pgfsetdash{}{0pt}%
\pgfsys@defobject{currentmarker}{\pgfqpoint{-0.005556in}{-0.005556in}}{\pgfqpoint{0.005556in}{0.005556in}}{%
\pgfpathmoveto{\pgfqpoint{0.000000in}{-0.005556in}}%
\pgfpathcurveto{\pgfqpoint{0.001473in}{-0.005556in}}{\pgfqpoint{0.002887in}{-0.004970in}}{\pgfqpoint{0.003928in}{-0.003928in}}%
\pgfpathcurveto{\pgfqpoint{0.004970in}{-0.002887in}}{\pgfqpoint{0.005556in}{-0.001473in}}{\pgfqpoint{0.005556in}{0.000000in}}%
\pgfpathcurveto{\pgfqpoint{0.005556in}{0.001473in}}{\pgfqpoint{0.004970in}{0.002887in}}{\pgfqpoint{0.003928in}{0.003928in}}%
\pgfpathcurveto{\pgfqpoint{0.002887in}{0.004970in}}{\pgfqpoint{0.001473in}{0.005556in}}{\pgfqpoint{0.000000in}{0.005556in}}%
\pgfpathcurveto{\pgfqpoint{-0.001473in}{0.005556in}}{\pgfqpoint{-0.002887in}{0.004970in}}{\pgfqpoint{-0.003928in}{0.003928in}}%
\pgfpathcurveto{\pgfqpoint{-0.004970in}{0.002887in}}{\pgfqpoint{-0.005556in}{0.001473in}}{\pgfqpoint{-0.005556in}{0.000000in}}%
\pgfpathcurveto{\pgfqpoint{-0.005556in}{-0.001473in}}{\pgfqpoint{-0.004970in}{-0.002887in}}{\pgfqpoint{-0.003928in}{-0.003928in}}%
\pgfpathcurveto{\pgfqpoint{-0.002887in}{-0.004970in}}{\pgfqpoint{-0.001473in}{-0.005556in}}{\pgfqpoint{0.000000in}{-0.005556in}}%
\pgfpathclose%
\pgfusepath{stroke,fill}%
}%
\begin{pgfscope}%
\pgfsys@transformshift{3.975097in}{4.053085in}%
\pgfsys@useobject{currentmarker}{}%
\end{pgfscope}%
\end{pgfscope}%
\begin{pgfscope}%
\pgfpathrectangle{\pgfqpoint{0.100000in}{2.413063in}}{\pgfqpoint{5.037500in}{3.427208in}}%
\pgfusepath{clip}%
\pgfsetrectcap%
\pgfsetroundjoin%
\pgfsetlinewidth{1.505625pt}%
\definecolor{currentstroke}{rgb}{0.678431,1.000000,0.184314}%
\pgfsetstrokecolor{currentstroke}%
\pgfsetstrokeopacity{0.500000}%
\pgfsetdash{}{0pt}%
\pgfpathmoveto{\pgfqpoint{3.985695in}{4.087467in}}%
\pgfusepath{stroke}%
\end{pgfscope}%
\begin{pgfscope}%
\pgfpathrectangle{\pgfqpoint{0.100000in}{2.413063in}}{\pgfqpoint{5.037500in}{3.427208in}}%
\pgfusepath{clip}%
\pgfsetbuttcap%
\pgfsetroundjoin%
\definecolor{currentfill}{rgb}{0.678431,1.000000,0.184314}%
\pgfsetfillcolor{currentfill}%
\pgfsetfillopacity{0.500000}%
\pgfsetlinewidth{0.250937pt}%
\definecolor{currentstroke}{rgb}{0.000000,0.000000,0.000000}%
\pgfsetstrokecolor{currentstroke}%
\pgfsetstrokeopacity{0.500000}%
\pgfsetdash{}{0pt}%
\pgfsys@defobject{currentmarker}{\pgfqpoint{-0.008333in}{-0.008333in}}{\pgfqpoint{0.008333in}{0.008333in}}{%
\pgfpathmoveto{\pgfqpoint{0.000000in}{-0.008333in}}%
\pgfpathcurveto{\pgfqpoint{0.002210in}{-0.008333in}}{\pgfqpoint{0.004330in}{-0.007455in}}{\pgfqpoint{0.005893in}{-0.005893in}}%
\pgfpathcurveto{\pgfqpoint{0.007455in}{-0.004330in}}{\pgfqpoint{0.008333in}{-0.002210in}}{\pgfqpoint{0.008333in}{0.000000in}}%
\pgfpathcurveto{\pgfqpoint{0.008333in}{0.002210in}}{\pgfqpoint{0.007455in}{0.004330in}}{\pgfqpoint{0.005893in}{0.005893in}}%
\pgfpathcurveto{\pgfqpoint{0.004330in}{0.007455in}}{\pgfqpoint{0.002210in}{0.008333in}}{\pgfqpoint{0.000000in}{0.008333in}}%
\pgfpathcurveto{\pgfqpoint{-0.002210in}{0.008333in}}{\pgfqpoint{-0.004330in}{0.007455in}}{\pgfqpoint{-0.005893in}{0.005893in}}%
\pgfpathcurveto{\pgfqpoint{-0.007455in}{0.004330in}}{\pgfqpoint{-0.008333in}{0.002210in}}{\pgfqpoint{-0.008333in}{0.000000in}}%
\pgfpathcurveto{\pgfqpoint{-0.008333in}{-0.002210in}}{\pgfqpoint{-0.007455in}{-0.004330in}}{\pgfqpoint{-0.005893in}{-0.005893in}}%
\pgfpathcurveto{\pgfqpoint{-0.004330in}{-0.007455in}}{\pgfqpoint{-0.002210in}{-0.008333in}}{\pgfqpoint{0.000000in}{-0.008333in}}%
\pgfpathclose%
\pgfusepath{stroke,fill}%
}%
\begin{pgfscope}%
\pgfsys@transformshift{3.985695in}{4.087467in}%
\pgfsys@useobject{currentmarker}{}%
\end{pgfscope}%
\end{pgfscope}%
\begin{pgfscope}%
\pgfpathrectangle{\pgfqpoint{0.100000in}{2.413063in}}{\pgfqpoint{5.037500in}{3.427208in}}%
\pgfusepath{clip}%
\pgfsetrectcap%
\pgfsetroundjoin%
\pgfsetlinewidth{1.505625pt}%
\definecolor{currentstroke}{rgb}{0.678431,1.000000,0.184314}%
\pgfsetstrokecolor{currentstroke}%
\pgfsetstrokeopacity{0.500000}%
\pgfsetdash{}{0pt}%
\pgfpathmoveto{\pgfqpoint{3.952422in}{4.077493in}}%
\pgfusepath{stroke}%
\end{pgfscope}%
\begin{pgfscope}%
\pgfpathrectangle{\pgfqpoint{0.100000in}{2.413063in}}{\pgfqpoint{5.037500in}{3.427208in}}%
\pgfusepath{clip}%
\pgfsetbuttcap%
\pgfsetroundjoin%
\definecolor{currentfill}{rgb}{0.678431,1.000000,0.184314}%
\pgfsetfillcolor{currentfill}%
\pgfsetfillopacity{0.500000}%
\pgfsetlinewidth{0.250937pt}%
\definecolor{currentstroke}{rgb}{0.000000,0.000000,0.000000}%
\pgfsetstrokecolor{currentstroke}%
\pgfsetstrokeopacity{0.500000}%
\pgfsetdash{}{0pt}%
\pgfsys@defobject{currentmarker}{\pgfqpoint{-0.008333in}{-0.008333in}}{\pgfqpoint{0.008333in}{0.008333in}}{%
\pgfpathmoveto{\pgfqpoint{0.000000in}{-0.008333in}}%
\pgfpathcurveto{\pgfqpoint{0.002210in}{-0.008333in}}{\pgfqpoint{0.004330in}{-0.007455in}}{\pgfqpoint{0.005893in}{-0.005893in}}%
\pgfpathcurveto{\pgfqpoint{0.007455in}{-0.004330in}}{\pgfqpoint{0.008333in}{-0.002210in}}{\pgfqpoint{0.008333in}{0.000000in}}%
\pgfpathcurveto{\pgfqpoint{0.008333in}{0.002210in}}{\pgfqpoint{0.007455in}{0.004330in}}{\pgfqpoint{0.005893in}{0.005893in}}%
\pgfpathcurveto{\pgfqpoint{0.004330in}{0.007455in}}{\pgfqpoint{0.002210in}{0.008333in}}{\pgfqpoint{0.000000in}{0.008333in}}%
\pgfpathcurveto{\pgfqpoint{-0.002210in}{0.008333in}}{\pgfqpoint{-0.004330in}{0.007455in}}{\pgfqpoint{-0.005893in}{0.005893in}}%
\pgfpathcurveto{\pgfqpoint{-0.007455in}{0.004330in}}{\pgfqpoint{-0.008333in}{0.002210in}}{\pgfqpoint{-0.008333in}{0.000000in}}%
\pgfpathcurveto{\pgfqpoint{-0.008333in}{-0.002210in}}{\pgfqpoint{-0.007455in}{-0.004330in}}{\pgfqpoint{-0.005893in}{-0.005893in}}%
\pgfpathcurveto{\pgfqpoint{-0.004330in}{-0.007455in}}{\pgfqpoint{-0.002210in}{-0.008333in}}{\pgfqpoint{0.000000in}{-0.008333in}}%
\pgfpathclose%
\pgfusepath{stroke,fill}%
}%
\begin{pgfscope}%
\pgfsys@transformshift{3.952422in}{4.077493in}%
\pgfsys@useobject{currentmarker}{}%
\end{pgfscope}%
\end{pgfscope}%
\begin{pgfscope}%
\pgfpathrectangle{\pgfqpoint{0.100000in}{2.413063in}}{\pgfqpoint{5.037500in}{3.427208in}}%
\pgfusepath{clip}%
\pgfsetrectcap%
\pgfsetroundjoin%
\pgfsetlinewidth{1.505625pt}%
\definecolor{currentstroke}{rgb}{0.501961,0.501961,0.501961}%
\pgfsetstrokecolor{currentstroke}%
\pgfsetstrokeopacity{0.500000}%
\pgfsetdash{}{0pt}%
\pgfpathmoveto{\pgfqpoint{3.835675in}{3.993728in}}%
\pgfusepath{stroke}%
\end{pgfscope}%
\begin{pgfscope}%
\pgfpathrectangle{\pgfqpoint{0.100000in}{2.413063in}}{\pgfqpoint{5.037500in}{3.427208in}}%
\pgfusepath{clip}%
\pgfsetbuttcap%
\pgfsetroundjoin%
\definecolor{currentfill}{rgb}{0.501961,0.501961,0.501961}%
\pgfsetfillcolor{currentfill}%
\pgfsetfillopacity{0.500000}%
\pgfsetlinewidth{0.250937pt}%
\definecolor{currentstroke}{rgb}{0.000000,0.000000,0.000000}%
\pgfsetstrokecolor{currentstroke}%
\pgfsetstrokeopacity{0.500000}%
\pgfsetdash{}{0pt}%
\pgfsys@defobject{currentmarker}{\pgfqpoint{-0.013889in}{-0.013889in}}{\pgfqpoint{0.013889in}{0.013889in}}{%
\pgfpathmoveto{\pgfqpoint{0.000000in}{-0.013889in}}%
\pgfpathcurveto{\pgfqpoint{0.003683in}{-0.013889in}}{\pgfqpoint{0.007216in}{-0.012425in}}{\pgfqpoint{0.009821in}{-0.009821in}}%
\pgfpathcurveto{\pgfqpoint{0.012425in}{-0.007216in}}{\pgfqpoint{0.013889in}{-0.003683in}}{\pgfqpoint{0.013889in}{0.000000in}}%
\pgfpathcurveto{\pgfqpoint{0.013889in}{0.003683in}}{\pgfqpoint{0.012425in}{0.007216in}}{\pgfqpoint{0.009821in}{0.009821in}}%
\pgfpathcurveto{\pgfqpoint{0.007216in}{0.012425in}}{\pgfqpoint{0.003683in}{0.013889in}}{\pgfqpoint{0.000000in}{0.013889in}}%
\pgfpathcurveto{\pgfqpoint{-0.003683in}{0.013889in}}{\pgfqpoint{-0.007216in}{0.012425in}}{\pgfqpoint{-0.009821in}{0.009821in}}%
\pgfpathcurveto{\pgfqpoint{-0.012425in}{0.007216in}}{\pgfqpoint{-0.013889in}{0.003683in}}{\pgfqpoint{-0.013889in}{0.000000in}}%
\pgfpathcurveto{\pgfqpoint{-0.013889in}{-0.003683in}}{\pgfqpoint{-0.012425in}{-0.007216in}}{\pgfqpoint{-0.009821in}{-0.009821in}}%
\pgfpathcurveto{\pgfqpoint{-0.007216in}{-0.012425in}}{\pgfqpoint{-0.003683in}{-0.013889in}}{\pgfqpoint{0.000000in}{-0.013889in}}%
\pgfpathclose%
\pgfusepath{stroke,fill}%
}%
\begin{pgfscope}%
\pgfsys@transformshift{3.835675in}{3.993728in}%
\pgfsys@useobject{currentmarker}{}%
\end{pgfscope}%
\end{pgfscope}%
\begin{pgfscope}%
\pgfpathrectangle{\pgfqpoint{0.100000in}{2.413063in}}{\pgfqpoint{5.037500in}{3.427208in}}%
\pgfusepath{clip}%
\pgfsetrectcap%
\pgfsetroundjoin%
\pgfsetlinewidth{1.505625pt}%
\definecolor{currentstroke}{rgb}{0.501961,0.501961,0.501961}%
\pgfsetstrokecolor{currentstroke}%
\pgfsetstrokeopacity{0.500000}%
\pgfsetdash{}{0pt}%
\pgfpathmoveto{\pgfqpoint{3.270873in}{3.849879in}}%
\pgfusepath{stroke}%
\end{pgfscope}%
\begin{pgfscope}%
\pgfpathrectangle{\pgfqpoint{0.100000in}{2.413063in}}{\pgfqpoint{5.037500in}{3.427208in}}%
\pgfusepath{clip}%
\pgfsetbuttcap%
\pgfsetroundjoin%
\definecolor{currentfill}{rgb}{0.501961,0.501961,0.501961}%
\pgfsetfillcolor{currentfill}%
\pgfsetfillopacity{0.500000}%
\pgfsetlinewidth{0.250937pt}%
\definecolor{currentstroke}{rgb}{0.000000,0.000000,0.000000}%
\pgfsetstrokecolor{currentstroke}%
\pgfsetstrokeopacity{0.500000}%
\pgfsetdash{}{0pt}%
\pgfsys@defobject{currentmarker}{\pgfqpoint{-0.013889in}{-0.013889in}}{\pgfqpoint{0.013889in}{0.013889in}}{%
\pgfpathmoveto{\pgfqpoint{0.000000in}{-0.013889in}}%
\pgfpathcurveto{\pgfqpoint{0.003683in}{-0.013889in}}{\pgfqpoint{0.007216in}{-0.012425in}}{\pgfqpoint{0.009821in}{-0.009821in}}%
\pgfpathcurveto{\pgfqpoint{0.012425in}{-0.007216in}}{\pgfqpoint{0.013889in}{-0.003683in}}{\pgfqpoint{0.013889in}{0.000000in}}%
\pgfpathcurveto{\pgfqpoint{0.013889in}{0.003683in}}{\pgfqpoint{0.012425in}{0.007216in}}{\pgfqpoint{0.009821in}{0.009821in}}%
\pgfpathcurveto{\pgfqpoint{0.007216in}{0.012425in}}{\pgfqpoint{0.003683in}{0.013889in}}{\pgfqpoint{0.000000in}{0.013889in}}%
\pgfpathcurveto{\pgfqpoint{-0.003683in}{0.013889in}}{\pgfqpoint{-0.007216in}{0.012425in}}{\pgfqpoint{-0.009821in}{0.009821in}}%
\pgfpathcurveto{\pgfqpoint{-0.012425in}{0.007216in}}{\pgfqpoint{-0.013889in}{0.003683in}}{\pgfqpoint{-0.013889in}{0.000000in}}%
\pgfpathcurveto{\pgfqpoint{-0.013889in}{-0.003683in}}{\pgfqpoint{-0.012425in}{-0.007216in}}{\pgfqpoint{-0.009821in}{-0.009821in}}%
\pgfpathcurveto{\pgfqpoint{-0.007216in}{-0.012425in}}{\pgfqpoint{-0.003683in}{-0.013889in}}{\pgfqpoint{0.000000in}{-0.013889in}}%
\pgfpathclose%
\pgfusepath{stroke,fill}%
}%
\begin{pgfscope}%
\pgfsys@transformshift{3.270873in}{3.849879in}%
\pgfsys@useobject{currentmarker}{}%
\end{pgfscope}%
\end{pgfscope}%
\begin{pgfscope}%
\pgfpathrectangle{\pgfqpoint{0.100000in}{2.413063in}}{\pgfqpoint{5.037500in}{3.427208in}}%
\pgfusepath{clip}%
\pgfsetrectcap%
\pgfsetroundjoin%
\pgfsetlinewidth{1.505625pt}%
\definecolor{currentstroke}{rgb}{0.678431,1.000000,0.184314}%
\pgfsetstrokecolor{currentstroke}%
\pgfsetstrokeopacity{0.500000}%
\pgfsetdash{}{0pt}%
\pgfpathmoveto{\pgfqpoint{3.889904in}{4.030032in}}%
\pgfusepath{stroke}%
\end{pgfscope}%
\begin{pgfscope}%
\pgfpathrectangle{\pgfqpoint{0.100000in}{2.413063in}}{\pgfqpoint{5.037500in}{3.427208in}}%
\pgfusepath{clip}%
\pgfsetbuttcap%
\pgfsetroundjoin%
\definecolor{currentfill}{rgb}{0.678431,1.000000,0.184314}%
\pgfsetfillcolor{currentfill}%
\pgfsetfillopacity{0.500000}%
\pgfsetlinewidth{0.250937pt}%
\definecolor{currentstroke}{rgb}{0.000000,0.000000,0.000000}%
\pgfsetstrokecolor{currentstroke}%
\pgfsetstrokeopacity{0.500000}%
\pgfsetdash{}{0pt}%
\pgfsys@defobject{currentmarker}{\pgfqpoint{-0.005556in}{-0.005556in}}{\pgfqpoint{0.005556in}{0.005556in}}{%
\pgfpathmoveto{\pgfqpoint{0.000000in}{-0.005556in}}%
\pgfpathcurveto{\pgfqpoint{0.001473in}{-0.005556in}}{\pgfqpoint{0.002887in}{-0.004970in}}{\pgfqpoint{0.003928in}{-0.003928in}}%
\pgfpathcurveto{\pgfqpoint{0.004970in}{-0.002887in}}{\pgfqpoint{0.005556in}{-0.001473in}}{\pgfqpoint{0.005556in}{0.000000in}}%
\pgfpathcurveto{\pgfqpoint{0.005556in}{0.001473in}}{\pgfqpoint{0.004970in}{0.002887in}}{\pgfqpoint{0.003928in}{0.003928in}}%
\pgfpathcurveto{\pgfqpoint{0.002887in}{0.004970in}}{\pgfqpoint{0.001473in}{0.005556in}}{\pgfqpoint{0.000000in}{0.005556in}}%
\pgfpathcurveto{\pgfqpoint{-0.001473in}{0.005556in}}{\pgfqpoint{-0.002887in}{0.004970in}}{\pgfqpoint{-0.003928in}{0.003928in}}%
\pgfpathcurveto{\pgfqpoint{-0.004970in}{0.002887in}}{\pgfqpoint{-0.005556in}{0.001473in}}{\pgfqpoint{-0.005556in}{0.000000in}}%
\pgfpathcurveto{\pgfqpoint{-0.005556in}{-0.001473in}}{\pgfqpoint{-0.004970in}{-0.002887in}}{\pgfqpoint{-0.003928in}{-0.003928in}}%
\pgfpathcurveto{\pgfqpoint{-0.002887in}{-0.004970in}}{\pgfqpoint{-0.001473in}{-0.005556in}}{\pgfqpoint{0.000000in}{-0.005556in}}%
\pgfpathclose%
\pgfusepath{stroke,fill}%
}%
\begin{pgfscope}%
\pgfsys@transformshift{3.889904in}{4.030032in}%
\pgfsys@useobject{currentmarker}{}%
\end{pgfscope}%
\end{pgfscope}%
\begin{pgfscope}%
\pgfpathrectangle{\pgfqpoint{0.100000in}{2.413063in}}{\pgfqpoint{5.037500in}{3.427208in}}%
\pgfusepath{clip}%
\pgfsetrectcap%
\pgfsetroundjoin%
\pgfsetlinewidth{1.505625pt}%
\definecolor{currentstroke}{rgb}{0.000000,0.000000,1.000000}%
\pgfsetstrokecolor{currentstroke}%
\pgfsetstrokeopacity{0.500000}%
\pgfsetdash{}{0pt}%
\pgfpathmoveto{\pgfqpoint{3.568755in}{3.988672in}}%
\pgfusepath{stroke}%
\end{pgfscope}%
\begin{pgfscope}%
\pgfpathrectangle{\pgfqpoint{0.100000in}{2.413063in}}{\pgfqpoint{5.037500in}{3.427208in}}%
\pgfusepath{clip}%
\pgfsetbuttcap%
\pgfsetroundjoin%
\definecolor{currentfill}{rgb}{0.000000,0.000000,1.000000}%
\pgfsetfillcolor{currentfill}%
\pgfsetfillopacity{0.500000}%
\pgfsetlinewidth{0.250937pt}%
\definecolor{currentstroke}{rgb}{0.000000,0.000000,0.000000}%
\pgfsetstrokecolor{currentstroke}%
\pgfsetstrokeopacity{0.500000}%
\pgfsetdash{}{0pt}%
\pgfsys@defobject{currentmarker}{\pgfqpoint{-0.005556in}{-0.005556in}}{\pgfqpoint{0.005556in}{0.005556in}}{%
\pgfpathmoveto{\pgfqpoint{0.000000in}{-0.005556in}}%
\pgfpathcurveto{\pgfqpoint{0.001473in}{-0.005556in}}{\pgfqpoint{0.002887in}{-0.004970in}}{\pgfqpoint{0.003928in}{-0.003928in}}%
\pgfpathcurveto{\pgfqpoint{0.004970in}{-0.002887in}}{\pgfqpoint{0.005556in}{-0.001473in}}{\pgfqpoint{0.005556in}{0.000000in}}%
\pgfpathcurveto{\pgfqpoint{0.005556in}{0.001473in}}{\pgfqpoint{0.004970in}{0.002887in}}{\pgfqpoint{0.003928in}{0.003928in}}%
\pgfpathcurveto{\pgfqpoint{0.002887in}{0.004970in}}{\pgfqpoint{0.001473in}{0.005556in}}{\pgfqpoint{0.000000in}{0.005556in}}%
\pgfpathcurveto{\pgfqpoint{-0.001473in}{0.005556in}}{\pgfqpoint{-0.002887in}{0.004970in}}{\pgfqpoint{-0.003928in}{0.003928in}}%
\pgfpathcurveto{\pgfqpoint{-0.004970in}{0.002887in}}{\pgfqpoint{-0.005556in}{0.001473in}}{\pgfqpoint{-0.005556in}{0.000000in}}%
\pgfpathcurveto{\pgfqpoint{-0.005556in}{-0.001473in}}{\pgfqpoint{-0.004970in}{-0.002887in}}{\pgfqpoint{-0.003928in}{-0.003928in}}%
\pgfpathcurveto{\pgfqpoint{-0.002887in}{-0.004970in}}{\pgfqpoint{-0.001473in}{-0.005556in}}{\pgfqpoint{0.000000in}{-0.005556in}}%
\pgfpathclose%
\pgfusepath{stroke,fill}%
}%
\begin{pgfscope}%
\pgfsys@transformshift{3.568755in}{3.988672in}%
\pgfsys@useobject{currentmarker}{}%
\end{pgfscope}%
\end{pgfscope}%
\begin{pgfscope}%
\pgfpathrectangle{\pgfqpoint{0.100000in}{2.413063in}}{\pgfqpoint{5.037500in}{3.427208in}}%
\pgfusepath{clip}%
\pgfsetrectcap%
\pgfsetroundjoin%
\pgfsetlinewidth{1.505625pt}%
\definecolor{currentstroke}{rgb}{0.000000,0.000000,1.000000}%
\pgfsetstrokecolor{currentstroke}%
\pgfsetstrokeopacity{0.500000}%
\pgfsetdash{}{0pt}%
\pgfpathmoveto{\pgfqpoint{2.341376in}{3.536703in}}%
\pgfusepath{stroke}%
\end{pgfscope}%
\begin{pgfscope}%
\pgfpathrectangle{\pgfqpoint{0.100000in}{2.413063in}}{\pgfqpoint{5.037500in}{3.427208in}}%
\pgfusepath{clip}%
\pgfsetbuttcap%
\pgfsetroundjoin%
\definecolor{currentfill}{rgb}{0.000000,0.000000,1.000000}%
\pgfsetfillcolor{currentfill}%
\pgfsetfillopacity{0.500000}%
\pgfsetlinewidth{0.250937pt}%
\definecolor{currentstroke}{rgb}{0.000000,0.000000,0.000000}%
\pgfsetstrokecolor{currentstroke}%
\pgfsetstrokeopacity{0.500000}%
\pgfsetdash{}{0pt}%
\pgfsys@defobject{currentmarker}{\pgfqpoint{-0.019444in}{-0.019444in}}{\pgfqpoint{0.019444in}{0.019444in}}{%
\pgfpathmoveto{\pgfqpoint{0.000000in}{-0.019444in}}%
\pgfpathcurveto{\pgfqpoint{0.005157in}{-0.019444in}}{\pgfqpoint{0.010103in}{-0.017396in}}{\pgfqpoint{0.013749in}{-0.013749in}}%
\pgfpathcurveto{\pgfqpoint{0.017396in}{-0.010103in}}{\pgfqpoint{0.019444in}{-0.005157in}}{\pgfqpoint{0.019444in}{0.000000in}}%
\pgfpathcurveto{\pgfqpoint{0.019444in}{0.005157in}}{\pgfqpoint{0.017396in}{0.010103in}}{\pgfqpoint{0.013749in}{0.013749in}}%
\pgfpathcurveto{\pgfqpoint{0.010103in}{0.017396in}}{\pgfqpoint{0.005157in}{0.019444in}}{\pgfqpoint{0.000000in}{0.019444in}}%
\pgfpathcurveto{\pgfqpoint{-0.005157in}{0.019444in}}{\pgfqpoint{-0.010103in}{0.017396in}}{\pgfqpoint{-0.013749in}{0.013749in}}%
\pgfpathcurveto{\pgfqpoint{-0.017396in}{0.010103in}}{\pgfqpoint{-0.019444in}{0.005157in}}{\pgfqpoint{-0.019444in}{0.000000in}}%
\pgfpathcurveto{\pgfqpoint{-0.019444in}{-0.005157in}}{\pgfqpoint{-0.017396in}{-0.010103in}}{\pgfqpoint{-0.013749in}{-0.013749in}}%
\pgfpathcurveto{\pgfqpoint{-0.010103in}{-0.017396in}}{\pgfqpoint{-0.005157in}{-0.019444in}}{\pgfqpoint{0.000000in}{-0.019444in}}%
\pgfpathclose%
\pgfusepath{stroke,fill}%
}%
\begin{pgfscope}%
\pgfsys@transformshift{2.341376in}{3.536703in}%
\pgfsys@useobject{currentmarker}{}%
\end{pgfscope}%
\end{pgfscope}%
\begin{pgfscope}%
\pgfpathrectangle{\pgfqpoint{0.100000in}{2.413063in}}{\pgfqpoint{5.037500in}{3.427208in}}%
\pgfusepath{clip}%
\pgfsetrectcap%
\pgfsetroundjoin%
\pgfsetlinewidth{1.505625pt}%
\definecolor{currentstroke}{rgb}{0.000000,0.000000,1.000000}%
\pgfsetstrokecolor{currentstroke}%
\pgfsetstrokeopacity{0.500000}%
\pgfsetdash{}{0pt}%
\pgfpathmoveto{\pgfqpoint{2.160335in}{3.868102in}}%
\pgfusepath{stroke}%
\end{pgfscope}%
\begin{pgfscope}%
\pgfpathrectangle{\pgfqpoint{0.100000in}{2.413063in}}{\pgfqpoint{5.037500in}{3.427208in}}%
\pgfusepath{clip}%
\pgfsetbuttcap%
\pgfsetroundjoin%
\definecolor{currentfill}{rgb}{0.000000,0.000000,1.000000}%
\pgfsetfillcolor{currentfill}%
\pgfsetfillopacity{0.500000}%
\pgfsetlinewidth{0.250937pt}%
\definecolor{currentstroke}{rgb}{0.000000,0.000000,0.000000}%
\pgfsetstrokecolor{currentstroke}%
\pgfsetstrokeopacity{0.500000}%
\pgfsetdash{}{0pt}%
\pgfsys@defobject{currentmarker}{\pgfqpoint{-0.016667in}{-0.016667in}}{\pgfqpoint{0.016667in}{0.016667in}}{%
\pgfpathmoveto{\pgfqpoint{0.000000in}{-0.016667in}}%
\pgfpathcurveto{\pgfqpoint{0.004420in}{-0.016667in}}{\pgfqpoint{0.008660in}{-0.014911in}}{\pgfqpoint{0.011785in}{-0.011785in}}%
\pgfpathcurveto{\pgfqpoint{0.014911in}{-0.008660in}}{\pgfqpoint{0.016667in}{-0.004420in}}{\pgfqpoint{0.016667in}{0.000000in}}%
\pgfpathcurveto{\pgfqpoint{0.016667in}{0.004420in}}{\pgfqpoint{0.014911in}{0.008660in}}{\pgfqpoint{0.011785in}{0.011785in}}%
\pgfpathcurveto{\pgfqpoint{0.008660in}{0.014911in}}{\pgfqpoint{0.004420in}{0.016667in}}{\pgfqpoint{0.000000in}{0.016667in}}%
\pgfpathcurveto{\pgfqpoint{-0.004420in}{0.016667in}}{\pgfqpoint{-0.008660in}{0.014911in}}{\pgfqpoint{-0.011785in}{0.011785in}}%
\pgfpathcurveto{\pgfqpoint{-0.014911in}{0.008660in}}{\pgfqpoint{-0.016667in}{0.004420in}}{\pgfqpoint{-0.016667in}{0.000000in}}%
\pgfpathcurveto{\pgfqpoint{-0.016667in}{-0.004420in}}{\pgfqpoint{-0.014911in}{-0.008660in}}{\pgfqpoint{-0.011785in}{-0.011785in}}%
\pgfpathcurveto{\pgfqpoint{-0.008660in}{-0.014911in}}{\pgfqpoint{-0.004420in}{-0.016667in}}{\pgfqpoint{0.000000in}{-0.016667in}}%
\pgfpathclose%
\pgfusepath{stroke,fill}%
}%
\begin{pgfscope}%
\pgfsys@transformshift{2.160335in}{3.868102in}%
\pgfsys@useobject{currentmarker}{}%
\end{pgfscope}%
\end{pgfscope}%
\begin{pgfscope}%
\pgfpathrectangle{\pgfqpoint{0.100000in}{2.413063in}}{\pgfqpoint{5.037500in}{3.427208in}}%
\pgfusepath{clip}%
\pgfsetrectcap%
\pgfsetroundjoin%
\pgfsetlinewidth{1.505625pt}%
\definecolor{currentstroke}{rgb}{0.000000,0.000000,1.000000}%
\pgfsetstrokecolor{currentstroke}%
\pgfsetstrokeopacity{0.500000}%
\pgfsetdash{}{0pt}%
\pgfpathmoveto{\pgfqpoint{2.528183in}{3.276102in}}%
\pgfusepath{stroke}%
\end{pgfscope}%
\begin{pgfscope}%
\pgfpathrectangle{\pgfqpoint{0.100000in}{2.413063in}}{\pgfqpoint{5.037500in}{3.427208in}}%
\pgfusepath{clip}%
\pgfsetbuttcap%
\pgfsetroundjoin%
\definecolor{currentfill}{rgb}{0.000000,0.000000,1.000000}%
\pgfsetfillcolor{currentfill}%
\pgfsetfillopacity{0.500000}%
\pgfsetlinewidth{0.250937pt}%
\definecolor{currentstroke}{rgb}{0.000000,0.000000,0.000000}%
\pgfsetstrokecolor{currentstroke}%
\pgfsetstrokeopacity{0.500000}%
\pgfsetdash{}{0pt}%
\pgfsys@defobject{currentmarker}{\pgfqpoint{-0.013889in}{-0.013889in}}{\pgfqpoint{0.013889in}{0.013889in}}{%
\pgfpathmoveto{\pgfqpoint{0.000000in}{-0.013889in}}%
\pgfpathcurveto{\pgfqpoint{0.003683in}{-0.013889in}}{\pgfqpoint{0.007216in}{-0.012425in}}{\pgfqpoint{0.009821in}{-0.009821in}}%
\pgfpathcurveto{\pgfqpoint{0.012425in}{-0.007216in}}{\pgfqpoint{0.013889in}{-0.003683in}}{\pgfqpoint{0.013889in}{0.000000in}}%
\pgfpathcurveto{\pgfqpoint{0.013889in}{0.003683in}}{\pgfqpoint{0.012425in}{0.007216in}}{\pgfqpoint{0.009821in}{0.009821in}}%
\pgfpathcurveto{\pgfqpoint{0.007216in}{0.012425in}}{\pgfqpoint{0.003683in}{0.013889in}}{\pgfqpoint{0.000000in}{0.013889in}}%
\pgfpathcurveto{\pgfqpoint{-0.003683in}{0.013889in}}{\pgfqpoint{-0.007216in}{0.012425in}}{\pgfqpoint{-0.009821in}{0.009821in}}%
\pgfpathcurveto{\pgfqpoint{-0.012425in}{0.007216in}}{\pgfqpoint{-0.013889in}{0.003683in}}{\pgfqpoint{-0.013889in}{0.000000in}}%
\pgfpathcurveto{\pgfqpoint{-0.013889in}{-0.003683in}}{\pgfqpoint{-0.012425in}{-0.007216in}}{\pgfqpoint{-0.009821in}{-0.009821in}}%
\pgfpathcurveto{\pgfqpoint{-0.007216in}{-0.012425in}}{\pgfqpoint{-0.003683in}{-0.013889in}}{\pgfqpoint{0.000000in}{-0.013889in}}%
\pgfpathclose%
\pgfusepath{stroke,fill}%
}%
\begin{pgfscope}%
\pgfsys@transformshift{2.528183in}{3.276102in}%
\pgfsys@useobject{currentmarker}{}%
\end{pgfscope}%
\end{pgfscope}%
\begin{pgfscope}%
\pgfpathrectangle{\pgfqpoint{0.100000in}{2.413063in}}{\pgfqpoint{5.037500in}{3.427208in}}%
\pgfusepath{clip}%
\pgfsetrectcap%
\pgfsetroundjoin%
\pgfsetlinewidth{1.505625pt}%
\definecolor{currentstroke}{rgb}{0.000000,0.000000,1.000000}%
\pgfsetstrokecolor{currentstroke}%
\pgfsetstrokeopacity{0.500000}%
\pgfsetdash{}{0pt}%
\pgfpathmoveto{\pgfqpoint{2.894684in}{3.250822in}}%
\pgfusepath{stroke}%
\end{pgfscope}%
\begin{pgfscope}%
\pgfpathrectangle{\pgfqpoint{0.100000in}{2.413063in}}{\pgfqpoint{5.037500in}{3.427208in}}%
\pgfusepath{clip}%
\pgfsetbuttcap%
\pgfsetroundjoin%
\definecolor{currentfill}{rgb}{0.000000,0.000000,1.000000}%
\pgfsetfillcolor{currentfill}%
\pgfsetfillopacity{0.500000}%
\pgfsetlinewidth{0.250937pt}%
\definecolor{currentstroke}{rgb}{0.000000,0.000000,0.000000}%
\pgfsetstrokecolor{currentstroke}%
\pgfsetstrokeopacity{0.500000}%
\pgfsetdash{}{0pt}%
\pgfsys@defobject{currentmarker}{\pgfqpoint{-0.050000in}{-0.050000in}}{\pgfqpoint{0.050000in}{0.050000in}}{%
\pgfpathmoveto{\pgfqpoint{0.000000in}{-0.050000in}}%
\pgfpathcurveto{\pgfqpoint{0.013260in}{-0.050000in}}{\pgfqpoint{0.025979in}{-0.044732in}}{\pgfqpoint{0.035355in}{-0.035355in}}%
\pgfpathcurveto{\pgfqpoint{0.044732in}{-0.025979in}}{\pgfqpoint{0.050000in}{-0.013260in}}{\pgfqpoint{0.050000in}{0.000000in}}%
\pgfpathcurveto{\pgfqpoint{0.050000in}{0.013260in}}{\pgfqpoint{0.044732in}{0.025979in}}{\pgfqpoint{0.035355in}{0.035355in}}%
\pgfpathcurveto{\pgfqpoint{0.025979in}{0.044732in}}{\pgfqpoint{0.013260in}{0.050000in}}{\pgfqpoint{0.000000in}{0.050000in}}%
\pgfpathcurveto{\pgfqpoint{-0.013260in}{0.050000in}}{\pgfqpoint{-0.025979in}{0.044732in}}{\pgfqpoint{-0.035355in}{0.035355in}}%
\pgfpathcurveto{\pgfqpoint{-0.044732in}{0.025979in}}{\pgfqpoint{-0.050000in}{0.013260in}}{\pgfqpoint{-0.050000in}{0.000000in}}%
\pgfpathcurveto{\pgfqpoint{-0.050000in}{-0.013260in}}{\pgfqpoint{-0.044732in}{-0.025979in}}{\pgfqpoint{-0.035355in}{-0.035355in}}%
\pgfpathcurveto{\pgfqpoint{-0.025979in}{-0.044732in}}{\pgfqpoint{-0.013260in}{-0.050000in}}{\pgfqpoint{0.000000in}{-0.050000in}}%
\pgfpathclose%
\pgfusepath{stroke,fill}%
}%
\begin{pgfscope}%
\pgfsys@transformshift{2.894684in}{3.250822in}%
\pgfsys@useobject{currentmarker}{}%
\end{pgfscope}%
\end{pgfscope}%
\begin{pgfscope}%
\pgfpathrectangle{\pgfqpoint{0.100000in}{2.413063in}}{\pgfqpoint{5.037500in}{3.427208in}}%
\pgfusepath{clip}%
\pgfsetrectcap%
\pgfsetroundjoin%
\pgfsetlinewidth{1.505625pt}%
\definecolor{currentstroke}{rgb}{0.000000,0.000000,1.000000}%
\pgfsetstrokecolor{currentstroke}%
\pgfsetstrokeopacity{0.500000}%
\pgfsetdash{}{0pt}%
\pgfpathmoveto{\pgfqpoint{2.539796in}{2.764801in}}%
\pgfusepath{stroke}%
\end{pgfscope}%
\begin{pgfscope}%
\pgfpathrectangle{\pgfqpoint{0.100000in}{2.413063in}}{\pgfqpoint{5.037500in}{3.427208in}}%
\pgfusepath{clip}%
\pgfsetbuttcap%
\pgfsetroundjoin%
\definecolor{currentfill}{rgb}{0.000000,0.000000,1.000000}%
\pgfsetfillcolor{currentfill}%
\pgfsetfillopacity{0.500000}%
\pgfsetlinewidth{0.250937pt}%
\definecolor{currentstroke}{rgb}{0.000000,0.000000,0.000000}%
\pgfsetstrokecolor{currentstroke}%
\pgfsetstrokeopacity{0.500000}%
\pgfsetdash{}{0pt}%
\pgfsys@defobject{currentmarker}{\pgfqpoint{-0.036111in}{-0.036111in}}{\pgfqpoint{0.036111in}{0.036111in}}{%
\pgfpathmoveto{\pgfqpoint{0.000000in}{-0.036111in}}%
\pgfpathcurveto{\pgfqpoint{0.009577in}{-0.036111in}}{\pgfqpoint{0.018763in}{-0.032306in}}{\pgfqpoint{0.025534in}{-0.025534in}}%
\pgfpathcurveto{\pgfqpoint{0.032306in}{-0.018763in}}{\pgfqpoint{0.036111in}{-0.009577in}}{\pgfqpoint{0.036111in}{0.000000in}}%
\pgfpathcurveto{\pgfqpoint{0.036111in}{0.009577in}}{\pgfqpoint{0.032306in}{0.018763in}}{\pgfqpoint{0.025534in}{0.025534in}}%
\pgfpathcurveto{\pgfqpoint{0.018763in}{0.032306in}}{\pgfqpoint{0.009577in}{0.036111in}}{\pgfqpoint{0.000000in}{0.036111in}}%
\pgfpathcurveto{\pgfqpoint{-0.009577in}{0.036111in}}{\pgfqpoint{-0.018763in}{0.032306in}}{\pgfqpoint{-0.025534in}{0.025534in}}%
\pgfpathcurveto{\pgfqpoint{-0.032306in}{0.018763in}}{\pgfqpoint{-0.036111in}{0.009577in}}{\pgfqpoint{-0.036111in}{0.000000in}}%
\pgfpathcurveto{\pgfqpoint{-0.036111in}{-0.009577in}}{\pgfqpoint{-0.032306in}{-0.018763in}}{\pgfqpoint{-0.025534in}{-0.025534in}}%
\pgfpathcurveto{\pgfqpoint{-0.018763in}{-0.032306in}}{\pgfqpoint{-0.009577in}{-0.036111in}}{\pgfqpoint{0.000000in}{-0.036111in}}%
\pgfpathclose%
\pgfusepath{stroke,fill}%
}%
\begin{pgfscope}%
\pgfsys@transformshift{2.539796in}{2.764801in}%
\pgfsys@useobject{currentmarker}{}%
\end{pgfscope}%
\end{pgfscope}%
\begin{pgfscope}%
\pgfpathrectangle{\pgfqpoint{0.100000in}{2.413063in}}{\pgfqpoint{5.037500in}{3.427208in}}%
\pgfusepath{clip}%
\pgfsetrectcap%
\pgfsetroundjoin%
\pgfsetlinewidth{1.505625pt}%
\definecolor{currentstroke}{rgb}{0.000000,0.000000,1.000000}%
\pgfsetstrokecolor{currentstroke}%
\pgfsetstrokeopacity{0.500000}%
\pgfsetdash{}{0pt}%
\pgfpathmoveto{\pgfqpoint{2.671238in}{3.314144in}}%
\pgfusepath{stroke}%
\end{pgfscope}%
\begin{pgfscope}%
\pgfpathrectangle{\pgfqpoint{0.100000in}{2.413063in}}{\pgfqpoint{5.037500in}{3.427208in}}%
\pgfusepath{clip}%
\pgfsetbuttcap%
\pgfsetroundjoin%
\definecolor{currentfill}{rgb}{0.000000,0.000000,1.000000}%
\pgfsetfillcolor{currentfill}%
\pgfsetfillopacity{0.500000}%
\pgfsetlinewidth{0.250937pt}%
\definecolor{currentstroke}{rgb}{0.000000,0.000000,0.000000}%
\pgfsetstrokecolor{currentstroke}%
\pgfsetstrokeopacity{0.500000}%
\pgfsetdash{}{0pt}%
\pgfsys@defobject{currentmarker}{\pgfqpoint{-0.022222in}{-0.022222in}}{\pgfqpoint{0.022222in}{0.022222in}}{%
\pgfpathmoveto{\pgfqpoint{0.000000in}{-0.022222in}}%
\pgfpathcurveto{\pgfqpoint{0.005893in}{-0.022222in}}{\pgfqpoint{0.011546in}{-0.019881in}}{\pgfqpoint{0.015713in}{-0.015713in}}%
\pgfpathcurveto{\pgfqpoint{0.019881in}{-0.011546in}}{\pgfqpoint{0.022222in}{-0.005893in}}{\pgfqpoint{0.022222in}{0.000000in}}%
\pgfpathcurveto{\pgfqpoint{0.022222in}{0.005893in}}{\pgfqpoint{0.019881in}{0.011546in}}{\pgfqpoint{0.015713in}{0.015713in}}%
\pgfpathcurveto{\pgfqpoint{0.011546in}{0.019881in}}{\pgfqpoint{0.005893in}{0.022222in}}{\pgfqpoint{0.000000in}{0.022222in}}%
\pgfpathcurveto{\pgfqpoint{-0.005893in}{0.022222in}}{\pgfqpoint{-0.011546in}{0.019881in}}{\pgfqpoint{-0.015713in}{0.015713in}}%
\pgfpathcurveto{\pgfqpoint{-0.019881in}{0.011546in}}{\pgfqpoint{-0.022222in}{0.005893in}}{\pgfqpoint{-0.022222in}{0.000000in}}%
\pgfpathcurveto{\pgfqpoint{-0.022222in}{-0.005893in}}{\pgfqpoint{-0.019881in}{-0.011546in}}{\pgfqpoint{-0.015713in}{-0.015713in}}%
\pgfpathcurveto{\pgfqpoint{-0.011546in}{-0.019881in}}{\pgfqpoint{-0.005893in}{-0.022222in}}{\pgfqpoint{0.000000in}{-0.022222in}}%
\pgfpathclose%
\pgfusepath{stroke,fill}%
}%
\begin{pgfscope}%
\pgfsys@transformshift{2.671238in}{3.314144in}%
\pgfsys@useobject{currentmarker}{}%
\end{pgfscope}%
\end{pgfscope}%
\begin{pgfscope}%
\pgfpathrectangle{\pgfqpoint{0.100000in}{2.413063in}}{\pgfqpoint{5.037500in}{3.427208in}}%
\pgfusepath{clip}%
\pgfsetrectcap%
\pgfsetroundjoin%
\pgfsetlinewidth{1.505625pt}%
\definecolor{currentstroke}{rgb}{0.000000,0.000000,1.000000}%
\pgfsetstrokecolor{currentstroke}%
\pgfsetstrokeopacity{0.500000}%
\pgfsetdash{}{0pt}%
\pgfpathmoveto{\pgfqpoint{2.555575in}{2.986438in}}%
\pgfusepath{stroke}%
\end{pgfscope}%
\begin{pgfscope}%
\pgfpathrectangle{\pgfqpoint{0.100000in}{2.413063in}}{\pgfqpoint{5.037500in}{3.427208in}}%
\pgfusepath{clip}%
\pgfsetbuttcap%
\pgfsetroundjoin%
\definecolor{currentfill}{rgb}{0.000000,0.000000,1.000000}%
\pgfsetfillcolor{currentfill}%
\pgfsetfillopacity{0.500000}%
\pgfsetlinewidth{0.250937pt}%
\definecolor{currentstroke}{rgb}{0.000000,0.000000,0.000000}%
\pgfsetstrokecolor{currentstroke}%
\pgfsetstrokeopacity{0.500000}%
\pgfsetdash{}{0pt}%
\pgfsys@defobject{currentmarker}{\pgfqpoint{-0.044444in}{-0.044444in}}{\pgfqpoint{0.044444in}{0.044444in}}{%
\pgfpathmoveto{\pgfqpoint{0.000000in}{-0.044444in}}%
\pgfpathcurveto{\pgfqpoint{0.011787in}{-0.044444in}}{\pgfqpoint{0.023092in}{-0.039761in}}{\pgfqpoint{0.031427in}{-0.031427in}}%
\pgfpathcurveto{\pgfqpoint{0.039761in}{-0.023092in}}{\pgfqpoint{0.044444in}{-0.011787in}}{\pgfqpoint{0.044444in}{0.000000in}}%
\pgfpathcurveto{\pgfqpoint{0.044444in}{0.011787in}}{\pgfqpoint{0.039761in}{0.023092in}}{\pgfqpoint{0.031427in}{0.031427in}}%
\pgfpathcurveto{\pgfqpoint{0.023092in}{0.039761in}}{\pgfqpoint{0.011787in}{0.044444in}}{\pgfqpoint{0.000000in}{0.044444in}}%
\pgfpathcurveto{\pgfqpoint{-0.011787in}{0.044444in}}{\pgfqpoint{-0.023092in}{0.039761in}}{\pgfqpoint{-0.031427in}{0.031427in}}%
\pgfpathcurveto{\pgfqpoint{-0.039761in}{0.023092in}}{\pgfqpoint{-0.044444in}{0.011787in}}{\pgfqpoint{-0.044444in}{0.000000in}}%
\pgfpathcurveto{\pgfqpoint{-0.044444in}{-0.011787in}}{\pgfqpoint{-0.039761in}{-0.023092in}}{\pgfqpoint{-0.031427in}{-0.031427in}}%
\pgfpathcurveto{\pgfqpoint{-0.023092in}{-0.039761in}}{\pgfqpoint{-0.011787in}{-0.044444in}}{\pgfqpoint{0.000000in}{-0.044444in}}%
\pgfpathclose%
\pgfusepath{stroke,fill}%
}%
\begin{pgfscope}%
\pgfsys@transformshift{2.555575in}{2.986438in}%
\pgfsys@useobject{currentmarker}{}%
\end{pgfscope}%
\end{pgfscope}%
\begin{pgfscope}%
\pgfpathrectangle{\pgfqpoint{0.100000in}{2.413063in}}{\pgfqpoint{5.037500in}{3.427208in}}%
\pgfusepath{clip}%
\pgfsetrectcap%
\pgfsetroundjoin%
\pgfsetlinewidth{1.505625pt}%
\definecolor{currentstroke}{rgb}{0.000000,0.000000,1.000000}%
\pgfsetstrokecolor{currentstroke}%
\pgfsetstrokeopacity{0.500000}%
\pgfsetdash{}{0pt}%
\pgfpathmoveto{\pgfqpoint{2.629147in}{3.564594in}}%
\pgfusepath{stroke}%
\end{pgfscope}%
\begin{pgfscope}%
\pgfpathrectangle{\pgfqpoint{0.100000in}{2.413063in}}{\pgfqpoint{5.037500in}{3.427208in}}%
\pgfusepath{clip}%
\pgfsetbuttcap%
\pgfsetroundjoin%
\definecolor{currentfill}{rgb}{0.000000,0.000000,1.000000}%
\pgfsetfillcolor{currentfill}%
\pgfsetfillopacity{0.500000}%
\pgfsetlinewidth{0.250937pt}%
\definecolor{currentstroke}{rgb}{0.000000,0.000000,0.000000}%
\pgfsetstrokecolor{currentstroke}%
\pgfsetstrokeopacity{0.500000}%
\pgfsetdash{}{0pt}%
\pgfsys@defobject{currentmarker}{\pgfqpoint{-0.019444in}{-0.019444in}}{\pgfqpoint{0.019444in}{0.019444in}}{%
\pgfpathmoveto{\pgfqpoint{0.000000in}{-0.019444in}}%
\pgfpathcurveto{\pgfqpoint{0.005157in}{-0.019444in}}{\pgfqpoint{0.010103in}{-0.017396in}}{\pgfqpoint{0.013749in}{-0.013749in}}%
\pgfpathcurveto{\pgfqpoint{0.017396in}{-0.010103in}}{\pgfqpoint{0.019444in}{-0.005157in}}{\pgfqpoint{0.019444in}{0.000000in}}%
\pgfpathcurveto{\pgfqpoint{0.019444in}{0.005157in}}{\pgfqpoint{0.017396in}{0.010103in}}{\pgfqpoint{0.013749in}{0.013749in}}%
\pgfpathcurveto{\pgfqpoint{0.010103in}{0.017396in}}{\pgfqpoint{0.005157in}{0.019444in}}{\pgfqpoint{0.000000in}{0.019444in}}%
\pgfpathcurveto{\pgfqpoint{-0.005157in}{0.019444in}}{\pgfqpoint{-0.010103in}{0.017396in}}{\pgfqpoint{-0.013749in}{0.013749in}}%
\pgfpathcurveto{\pgfqpoint{-0.017396in}{0.010103in}}{\pgfqpoint{-0.019444in}{0.005157in}}{\pgfqpoint{-0.019444in}{0.000000in}}%
\pgfpathcurveto{\pgfqpoint{-0.019444in}{-0.005157in}}{\pgfqpoint{-0.017396in}{-0.010103in}}{\pgfqpoint{-0.013749in}{-0.013749in}}%
\pgfpathcurveto{\pgfqpoint{-0.010103in}{-0.017396in}}{\pgfqpoint{-0.005157in}{-0.019444in}}{\pgfqpoint{0.000000in}{-0.019444in}}%
\pgfpathclose%
\pgfusepath{stroke,fill}%
}%
\begin{pgfscope}%
\pgfsys@transformshift{2.629147in}{3.564594in}%
\pgfsys@useobject{currentmarker}{}%
\end{pgfscope}%
\end{pgfscope}%
\begin{pgfscope}%
\pgfpathrectangle{\pgfqpoint{0.100000in}{2.413063in}}{\pgfqpoint{5.037500in}{3.427208in}}%
\pgfusepath{clip}%
\pgfsetrectcap%
\pgfsetroundjoin%
\pgfsetlinewidth{1.505625pt}%
\definecolor{currentstroke}{rgb}{0.000000,0.000000,1.000000}%
\pgfsetstrokecolor{currentstroke}%
\pgfsetstrokeopacity{0.500000}%
\pgfsetdash{}{0pt}%
\pgfpathmoveto{\pgfqpoint{1.673527in}{3.516590in}}%
\pgfusepath{stroke}%
\end{pgfscope}%
\begin{pgfscope}%
\pgfpathrectangle{\pgfqpoint{0.100000in}{2.413063in}}{\pgfqpoint{5.037500in}{3.427208in}}%
\pgfusepath{clip}%
\pgfsetbuttcap%
\pgfsetroundjoin%
\definecolor{currentfill}{rgb}{0.000000,0.000000,1.000000}%
\pgfsetfillcolor{currentfill}%
\pgfsetfillopacity{0.500000}%
\pgfsetlinewidth{0.250937pt}%
\definecolor{currentstroke}{rgb}{0.000000,0.000000,0.000000}%
\pgfsetstrokecolor{currentstroke}%
\pgfsetstrokeopacity{0.500000}%
\pgfsetdash{}{0pt}%
\pgfsys@defobject{currentmarker}{\pgfqpoint{-0.033333in}{-0.033333in}}{\pgfqpoint{0.033333in}{0.033333in}}{%
\pgfpathmoveto{\pgfqpoint{0.000000in}{-0.033333in}}%
\pgfpathcurveto{\pgfqpoint{0.008840in}{-0.033333in}}{\pgfqpoint{0.017319in}{-0.029821in}}{\pgfqpoint{0.023570in}{-0.023570in}}%
\pgfpathcurveto{\pgfqpoint{0.029821in}{-0.017319in}}{\pgfqpoint{0.033333in}{-0.008840in}}{\pgfqpoint{0.033333in}{0.000000in}}%
\pgfpathcurveto{\pgfqpoint{0.033333in}{0.008840in}}{\pgfqpoint{0.029821in}{0.017319in}}{\pgfqpoint{0.023570in}{0.023570in}}%
\pgfpathcurveto{\pgfqpoint{0.017319in}{0.029821in}}{\pgfqpoint{0.008840in}{0.033333in}}{\pgfqpoint{0.000000in}{0.033333in}}%
\pgfpathcurveto{\pgfqpoint{-0.008840in}{0.033333in}}{\pgfqpoint{-0.017319in}{0.029821in}}{\pgfqpoint{-0.023570in}{0.023570in}}%
\pgfpathcurveto{\pgfqpoint{-0.029821in}{0.017319in}}{\pgfqpoint{-0.033333in}{0.008840in}}{\pgfqpoint{-0.033333in}{0.000000in}}%
\pgfpathcurveto{\pgfqpoint{-0.033333in}{-0.008840in}}{\pgfqpoint{-0.029821in}{-0.017319in}}{\pgfqpoint{-0.023570in}{-0.023570in}}%
\pgfpathcurveto{\pgfqpoint{-0.017319in}{-0.029821in}}{\pgfqpoint{-0.008840in}{-0.033333in}}{\pgfqpoint{0.000000in}{-0.033333in}}%
\pgfpathclose%
\pgfusepath{stroke,fill}%
}%
\begin{pgfscope}%
\pgfsys@transformshift{1.673527in}{3.516590in}%
\pgfsys@useobject{currentmarker}{}%
\end{pgfscope}%
\end{pgfscope}%
\begin{pgfscope}%
\pgfpathrectangle{\pgfqpoint{0.100000in}{2.413063in}}{\pgfqpoint{5.037500in}{3.427208in}}%
\pgfusepath{clip}%
\pgfsetrectcap%
\pgfsetroundjoin%
\pgfsetlinewidth{1.505625pt}%
\definecolor{currentstroke}{rgb}{0.000000,0.000000,1.000000}%
\pgfsetstrokecolor{currentstroke}%
\pgfsetstrokeopacity{0.500000}%
\pgfsetdash{}{0pt}%
\pgfpathmoveto{\pgfqpoint{2.766900in}{3.212317in}}%
\pgfusepath{stroke}%
\end{pgfscope}%
\begin{pgfscope}%
\pgfpathrectangle{\pgfqpoint{0.100000in}{2.413063in}}{\pgfqpoint{5.037500in}{3.427208in}}%
\pgfusepath{clip}%
\pgfsetbuttcap%
\pgfsetroundjoin%
\definecolor{currentfill}{rgb}{0.000000,0.000000,1.000000}%
\pgfsetfillcolor{currentfill}%
\pgfsetfillopacity{0.500000}%
\pgfsetlinewidth{0.250937pt}%
\definecolor{currentstroke}{rgb}{0.000000,0.000000,0.000000}%
\pgfsetstrokecolor{currentstroke}%
\pgfsetstrokeopacity{0.500000}%
\pgfsetdash{}{0pt}%
\pgfsys@defobject{currentmarker}{\pgfqpoint{-0.033333in}{-0.033333in}}{\pgfqpoint{0.033333in}{0.033333in}}{%
\pgfpathmoveto{\pgfqpoint{0.000000in}{-0.033333in}}%
\pgfpathcurveto{\pgfqpoint{0.008840in}{-0.033333in}}{\pgfqpoint{0.017319in}{-0.029821in}}{\pgfqpoint{0.023570in}{-0.023570in}}%
\pgfpathcurveto{\pgfqpoint{0.029821in}{-0.017319in}}{\pgfqpoint{0.033333in}{-0.008840in}}{\pgfqpoint{0.033333in}{0.000000in}}%
\pgfpathcurveto{\pgfqpoint{0.033333in}{0.008840in}}{\pgfqpoint{0.029821in}{0.017319in}}{\pgfqpoint{0.023570in}{0.023570in}}%
\pgfpathcurveto{\pgfqpoint{0.017319in}{0.029821in}}{\pgfqpoint{0.008840in}{0.033333in}}{\pgfqpoint{0.000000in}{0.033333in}}%
\pgfpathcurveto{\pgfqpoint{-0.008840in}{0.033333in}}{\pgfqpoint{-0.017319in}{0.029821in}}{\pgfqpoint{-0.023570in}{0.023570in}}%
\pgfpathcurveto{\pgfqpoint{-0.029821in}{0.017319in}}{\pgfqpoint{-0.033333in}{0.008840in}}{\pgfqpoint{-0.033333in}{0.000000in}}%
\pgfpathcurveto{\pgfqpoint{-0.033333in}{-0.008840in}}{\pgfqpoint{-0.029821in}{-0.017319in}}{\pgfqpoint{-0.023570in}{-0.023570in}}%
\pgfpathcurveto{\pgfqpoint{-0.017319in}{-0.029821in}}{\pgfqpoint{-0.008840in}{-0.033333in}}{\pgfqpoint{0.000000in}{-0.033333in}}%
\pgfpathclose%
\pgfusepath{stroke,fill}%
}%
\begin{pgfscope}%
\pgfsys@transformshift{2.766900in}{3.212317in}%
\pgfsys@useobject{currentmarker}{}%
\end{pgfscope}%
\end{pgfscope}%
\begin{pgfscope}%
\pgfpathrectangle{\pgfqpoint{0.100000in}{2.413063in}}{\pgfqpoint{5.037500in}{3.427208in}}%
\pgfusepath{clip}%
\pgfsetrectcap%
\pgfsetroundjoin%
\pgfsetlinewidth{1.505625pt}%
\definecolor{currentstroke}{rgb}{0.000000,0.000000,1.000000}%
\pgfsetstrokecolor{currentstroke}%
\pgfsetstrokeopacity{0.500000}%
\pgfsetdash{}{0pt}%
\pgfpathmoveto{\pgfqpoint{2.532736in}{3.374442in}}%
\pgfusepath{stroke}%
\end{pgfscope}%
\begin{pgfscope}%
\pgfpathrectangle{\pgfqpoint{0.100000in}{2.413063in}}{\pgfqpoint{5.037500in}{3.427208in}}%
\pgfusepath{clip}%
\pgfsetbuttcap%
\pgfsetroundjoin%
\definecolor{currentfill}{rgb}{0.000000,0.000000,1.000000}%
\pgfsetfillcolor{currentfill}%
\pgfsetfillopacity{0.500000}%
\pgfsetlinewidth{0.250937pt}%
\definecolor{currentstroke}{rgb}{0.000000,0.000000,0.000000}%
\pgfsetstrokecolor{currentstroke}%
\pgfsetstrokeopacity{0.500000}%
\pgfsetdash{}{0pt}%
\pgfsys@defobject{currentmarker}{\pgfqpoint{-0.027778in}{-0.027778in}}{\pgfqpoint{0.027778in}{0.027778in}}{%
\pgfpathmoveto{\pgfqpoint{0.000000in}{-0.027778in}}%
\pgfpathcurveto{\pgfqpoint{0.007367in}{-0.027778in}}{\pgfqpoint{0.014433in}{-0.024851in}}{\pgfqpoint{0.019642in}{-0.019642in}}%
\pgfpathcurveto{\pgfqpoint{0.024851in}{-0.014433in}}{\pgfqpoint{0.027778in}{-0.007367in}}{\pgfqpoint{0.027778in}{0.000000in}}%
\pgfpathcurveto{\pgfqpoint{0.027778in}{0.007367in}}{\pgfqpoint{0.024851in}{0.014433in}}{\pgfqpoint{0.019642in}{0.019642in}}%
\pgfpathcurveto{\pgfqpoint{0.014433in}{0.024851in}}{\pgfqpoint{0.007367in}{0.027778in}}{\pgfqpoint{0.000000in}{0.027778in}}%
\pgfpathcurveto{\pgfqpoint{-0.007367in}{0.027778in}}{\pgfqpoint{-0.014433in}{0.024851in}}{\pgfqpoint{-0.019642in}{0.019642in}}%
\pgfpathcurveto{\pgfqpoint{-0.024851in}{0.014433in}}{\pgfqpoint{-0.027778in}{0.007367in}}{\pgfqpoint{-0.027778in}{0.000000in}}%
\pgfpathcurveto{\pgfqpoint{-0.027778in}{-0.007367in}}{\pgfqpoint{-0.024851in}{-0.014433in}}{\pgfqpoint{-0.019642in}{-0.019642in}}%
\pgfpathcurveto{\pgfqpoint{-0.014433in}{-0.024851in}}{\pgfqpoint{-0.007367in}{-0.027778in}}{\pgfqpoint{0.000000in}{-0.027778in}}%
\pgfpathclose%
\pgfusepath{stroke,fill}%
}%
\begin{pgfscope}%
\pgfsys@transformshift{2.532736in}{3.374442in}%
\pgfsys@useobject{currentmarker}{}%
\end{pgfscope}%
\end{pgfscope}%
\begin{pgfscope}%
\pgfpathrectangle{\pgfqpoint{0.100000in}{2.413063in}}{\pgfqpoint{5.037500in}{3.427208in}}%
\pgfusepath{clip}%
\pgfsetrectcap%
\pgfsetroundjoin%
\pgfsetlinewidth{1.505625pt}%
\definecolor{currentstroke}{rgb}{0.000000,0.000000,1.000000}%
\pgfsetstrokecolor{currentstroke}%
\pgfsetstrokeopacity{0.500000}%
\pgfsetdash{}{0pt}%
\pgfpathmoveto{\pgfqpoint{2.334733in}{2.960286in}}%
\pgfusepath{stroke}%
\end{pgfscope}%
\begin{pgfscope}%
\pgfpathrectangle{\pgfqpoint{0.100000in}{2.413063in}}{\pgfqpoint{5.037500in}{3.427208in}}%
\pgfusepath{clip}%
\pgfsetbuttcap%
\pgfsetroundjoin%
\definecolor{currentfill}{rgb}{0.000000,0.000000,1.000000}%
\pgfsetfillcolor{currentfill}%
\pgfsetfillopacity{0.500000}%
\pgfsetlinewidth{0.250937pt}%
\definecolor{currentstroke}{rgb}{0.000000,0.000000,0.000000}%
\pgfsetstrokecolor{currentstroke}%
\pgfsetstrokeopacity{0.500000}%
\pgfsetdash{}{0pt}%
\pgfsys@defobject{currentmarker}{\pgfqpoint{-0.036111in}{-0.036111in}}{\pgfqpoint{0.036111in}{0.036111in}}{%
\pgfpathmoveto{\pgfqpoint{0.000000in}{-0.036111in}}%
\pgfpathcurveto{\pgfqpoint{0.009577in}{-0.036111in}}{\pgfqpoint{0.018763in}{-0.032306in}}{\pgfqpoint{0.025534in}{-0.025534in}}%
\pgfpathcurveto{\pgfqpoint{0.032306in}{-0.018763in}}{\pgfqpoint{0.036111in}{-0.009577in}}{\pgfqpoint{0.036111in}{0.000000in}}%
\pgfpathcurveto{\pgfqpoint{0.036111in}{0.009577in}}{\pgfqpoint{0.032306in}{0.018763in}}{\pgfqpoint{0.025534in}{0.025534in}}%
\pgfpathcurveto{\pgfqpoint{0.018763in}{0.032306in}}{\pgfqpoint{0.009577in}{0.036111in}}{\pgfqpoint{0.000000in}{0.036111in}}%
\pgfpathcurveto{\pgfqpoint{-0.009577in}{0.036111in}}{\pgfqpoint{-0.018763in}{0.032306in}}{\pgfqpoint{-0.025534in}{0.025534in}}%
\pgfpathcurveto{\pgfqpoint{-0.032306in}{0.018763in}}{\pgfqpoint{-0.036111in}{0.009577in}}{\pgfqpoint{-0.036111in}{0.000000in}}%
\pgfpathcurveto{\pgfqpoint{-0.036111in}{-0.009577in}}{\pgfqpoint{-0.032306in}{-0.018763in}}{\pgfqpoint{-0.025534in}{-0.025534in}}%
\pgfpathcurveto{\pgfqpoint{-0.018763in}{-0.032306in}}{\pgfqpoint{-0.009577in}{-0.036111in}}{\pgfqpoint{0.000000in}{-0.036111in}}%
\pgfpathclose%
\pgfusepath{stroke,fill}%
}%
\begin{pgfscope}%
\pgfsys@transformshift{2.334733in}{2.960286in}%
\pgfsys@useobject{currentmarker}{}%
\end{pgfscope}%
\end{pgfscope}%
\begin{pgfscope}%
\pgfpathrectangle{\pgfqpoint{0.100000in}{2.413063in}}{\pgfqpoint{5.037500in}{3.427208in}}%
\pgfusepath{clip}%
\pgfsetrectcap%
\pgfsetroundjoin%
\pgfsetlinewidth{1.505625pt}%
\definecolor{currentstroke}{rgb}{0.000000,0.000000,1.000000}%
\pgfsetstrokecolor{currentstroke}%
\pgfsetstrokeopacity{0.500000}%
\pgfsetdash{}{0pt}%
\pgfpathmoveto{\pgfqpoint{2.829493in}{3.530973in}}%
\pgfusepath{stroke}%
\end{pgfscope}%
\begin{pgfscope}%
\pgfpathrectangle{\pgfqpoint{0.100000in}{2.413063in}}{\pgfqpoint{5.037500in}{3.427208in}}%
\pgfusepath{clip}%
\pgfsetbuttcap%
\pgfsetroundjoin%
\definecolor{currentfill}{rgb}{0.000000,0.000000,1.000000}%
\pgfsetfillcolor{currentfill}%
\pgfsetfillopacity{0.500000}%
\pgfsetlinewidth{0.250937pt}%
\definecolor{currentstroke}{rgb}{0.000000,0.000000,0.000000}%
\pgfsetstrokecolor{currentstroke}%
\pgfsetstrokeopacity{0.500000}%
\pgfsetdash{}{0pt}%
\pgfsys@defobject{currentmarker}{\pgfqpoint{-0.033333in}{-0.033333in}}{\pgfqpoint{0.033333in}{0.033333in}}{%
\pgfpathmoveto{\pgfqpoint{0.000000in}{-0.033333in}}%
\pgfpathcurveto{\pgfqpoint{0.008840in}{-0.033333in}}{\pgfqpoint{0.017319in}{-0.029821in}}{\pgfqpoint{0.023570in}{-0.023570in}}%
\pgfpathcurveto{\pgfqpoint{0.029821in}{-0.017319in}}{\pgfqpoint{0.033333in}{-0.008840in}}{\pgfqpoint{0.033333in}{0.000000in}}%
\pgfpathcurveto{\pgfqpoint{0.033333in}{0.008840in}}{\pgfqpoint{0.029821in}{0.017319in}}{\pgfqpoint{0.023570in}{0.023570in}}%
\pgfpathcurveto{\pgfqpoint{0.017319in}{0.029821in}}{\pgfqpoint{0.008840in}{0.033333in}}{\pgfqpoint{0.000000in}{0.033333in}}%
\pgfpathcurveto{\pgfqpoint{-0.008840in}{0.033333in}}{\pgfqpoint{-0.017319in}{0.029821in}}{\pgfqpoint{-0.023570in}{0.023570in}}%
\pgfpathcurveto{\pgfqpoint{-0.029821in}{0.017319in}}{\pgfqpoint{-0.033333in}{0.008840in}}{\pgfqpoint{-0.033333in}{0.000000in}}%
\pgfpathcurveto{\pgfqpoint{-0.033333in}{-0.008840in}}{\pgfqpoint{-0.029821in}{-0.017319in}}{\pgfqpoint{-0.023570in}{-0.023570in}}%
\pgfpathcurveto{\pgfqpoint{-0.017319in}{-0.029821in}}{\pgfqpoint{-0.008840in}{-0.033333in}}{\pgfqpoint{0.000000in}{-0.033333in}}%
\pgfpathclose%
\pgfusepath{stroke,fill}%
}%
\begin{pgfscope}%
\pgfsys@transformshift{2.829493in}{3.530973in}%
\pgfsys@useobject{currentmarker}{}%
\end{pgfscope}%
\end{pgfscope}%
\begin{pgfscope}%
\pgfpathrectangle{\pgfqpoint{0.100000in}{2.413063in}}{\pgfqpoint{5.037500in}{3.427208in}}%
\pgfusepath{clip}%
\pgfsetrectcap%
\pgfsetroundjoin%
\pgfsetlinewidth{1.505625pt}%
\definecolor{currentstroke}{rgb}{0.000000,0.000000,1.000000}%
\pgfsetstrokecolor{currentstroke}%
\pgfsetstrokeopacity{0.500000}%
\pgfsetdash{}{0pt}%
\pgfpathmoveto{\pgfqpoint{2.144143in}{3.680550in}}%
\pgfusepath{stroke}%
\end{pgfscope}%
\begin{pgfscope}%
\pgfpathrectangle{\pgfqpoint{0.100000in}{2.413063in}}{\pgfqpoint{5.037500in}{3.427208in}}%
\pgfusepath{clip}%
\pgfsetbuttcap%
\pgfsetroundjoin%
\definecolor{currentfill}{rgb}{0.000000,0.000000,1.000000}%
\pgfsetfillcolor{currentfill}%
\pgfsetfillopacity{0.500000}%
\pgfsetlinewidth{0.250937pt}%
\definecolor{currentstroke}{rgb}{0.000000,0.000000,0.000000}%
\pgfsetstrokecolor{currentstroke}%
\pgfsetstrokeopacity{0.500000}%
\pgfsetdash{}{0pt}%
\pgfsys@defobject{currentmarker}{\pgfqpoint{-0.025000in}{-0.025000in}}{\pgfqpoint{0.025000in}{0.025000in}}{%
\pgfpathmoveto{\pgfqpoint{0.000000in}{-0.025000in}}%
\pgfpathcurveto{\pgfqpoint{0.006630in}{-0.025000in}}{\pgfqpoint{0.012989in}{-0.022366in}}{\pgfqpoint{0.017678in}{-0.017678in}}%
\pgfpathcurveto{\pgfqpoint{0.022366in}{-0.012989in}}{\pgfqpoint{0.025000in}{-0.006630in}}{\pgfqpoint{0.025000in}{0.000000in}}%
\pgfpathcurveto{\pgfqpoint{0.025000in}{0.006630in}}{\pgfqpoint{0.022366in}{0.012989in}}{\pgfqpoint{0.017678in}{0.017678in}}%
\pgfpathcurveto{\pgfqpoint{0.012989in}{0.022366in}}{\pgfqpoint{0.006630in}{0.025000in}}{\pgfqpoint{0.000000in}{0.025000in}}%
\pgfpathcurveto{\pgfqpoint{-0.006630in}{0.025000in}}{\pgfqpoint{-0.012989in}{0.022366in}}{\pgfqpoint{-0.017678in}{0.017678in}}%
\pgfpathcurveto{\pgfqpoint{-0.022366in}{0.012989in}}{\pgfqpoint{-0.025000in}{0.006630in}}{\pgfqpoint{-0.025000in}{0.000000in}}%
\pgfpathcurveto{\pgfqpoint{-0.025000in}{-0.006630in}}{\pgfqpoint{-0.022366in}{-0.012989in}}{\pgfqpoint{-0.017678in}{-0.017678in}}%
\pgfpathcurveto{\pgfqpoint{-0.012989in}{-0.022366in}}{\pgfqpoint{-0.006630in}{-0.025000in}}{\pgfqpoint{0.000000in}{-0.025000in}}%
\pgfpathclose%
\pgfusepath{stroke,fill}%
}%
\begin{pgfscope}%
\pgfsys@transformshift{2.144143in}{3.680550in}%
\pgfsys@useobject{currentmarker}{}%
\end{pgfscope}%
\end{pgfscope}%
\begin{pgfscope}%
\pgfpathrectangle{\pgfqpoint{0.100000in}{2.413063in}}{\pgfqpoint{5.037500in}{3.427208in}}%
\pgfusepath{clip}%
\pgfsetrectcap%
\pgfsetroundjoin%
\pgfsetlinewidth{1.505625pt}%
\definecolor{currentstroke}{rgb}{0.000000,0.000000,1.000000}%
\pgfsetstrokecolor{currentstroke}%
\pgfsetstrokeopacity{0.500000}%
\pgfsetdash{}{0pt}%
\pgfpathmoveto{\pgfqpoint{2.462352in}{2.801522in}}%
\pgfusepath{stroke}%
\end{pgfscope}%
\begin{pgfscope}%
\pgfpathrectangle{\pgfqpoint{0.100000in}{2.413063in}}{\pgfqpoint{5.037500in}{3.427208in}}%
\pgfusepath{clip}%
\pgfsetbuttcap%
\pgfsetroundjoin%
\definecolor{currentfill}{rgb}{0.000000,0.000000,1.000000}%
\pgfsetfillcolor{currentfill}%
\pgfsetfillopacity{0.500000}%
\pgfsetlinewidth{0.250937pt}%
\definecolor{currentstroke}{rgb}{0.000000,0.000000,0.000000}%
\pgfsetstrokecolor{currentstroke}%
\pgfsetstrokeopacity{0.500000}%
\pgfsetdash{}{0pt}%
\pgfsys@defobject{currentmarker}{\pgfqpoint{-0.025000in}{-0.025000in}}{\pgfqpoint{0.025000in}{0.025000in}}{%
\pgfpathmoveto{\pgfqpoint{0.000000in}{-0.025000in}}%
\pgfpathcurveto{\pgfqpoint{0.006630in}{-0.025000in}}{\pgfqpoint{0.012989in}{-0.022366in}}{\pgfqpoint{0.017678in}{-0.017678in}}%
\pgfpathcurveto{\pgfqpoint{0.022366in}{-0.012989in}}{\pgfqpoint{0.025000in}{-0.006630in}}{\pgfqpoint{0.025000in}{0.000000in}}%
\pgfpathcurveto{\pgfqpoint{0.025000in}{0.006630in}}{\pgfqpoint{0.022366in}{0.012989in}}{\pgfqpoint{0.017678in}{0.017678in}}%
\pgfpathcurveto{\pgfqpoint{0.012989in}{0.022366in}}{\pgfqpoint{0.006630in}{0.025000in}}{\pgfqpoint{0.000000in}{0.025000in}}%
\pgfpathcurveto{\pgfqpoint{-0.006630in}{0.025000in}}{\pgfqpoint{-0.012989in}{0.022366in}}{\pgfqpoint{-0.017678in}{0.017678in}}%
\pgfpathcurveto{\pgfqpoint{-0.022366in}{0.012989in}}{\pgfqpoint{-0.025000in}{0.006630in}}{\pgfqpoint{-0.025000in}{0.000000in}}%
\pgfpathcurveto{\pgfqpoint{-0.025000in}{-0.006630in}}{\pgfqpoint{-0.022366in}{-0.012989in}}{\pgfqpoint{-0.017678in}{-0.017678in}}%
\pgfpathcurveto{\pgfqpoint{-0.012989in}{-0.022366in}}{\pgfqpoint{-0.006630in}{-0.025000in}}{\pgfqpoint{0.000000in}{-0.025000in}}%
\pgfpathclose%
\pgfusepath{stroke,fill}%
}%
\begin{pgfscope}%
\pgfsys@transformshift{2.462352in}{2.801522in}%
\pgfsys@useobject{currentmarker}{}%
\end{pgfscope}%
\end{pgfscope}%
\begin{pgfscope}%
\pgfpathrectangle{\pgfqpoint{0.100000in}{2.413063in}}{\pgfqpoint{5.037500in}{3.427208in}}%
\pgfusepath{clip}%
\pgfsetrectcap%
\pgfsetroundjoin%
\pgfsetlinewidth{1.505625pt}%
\definecolor{currentstroke}{rgb}{0.000000,0.000000,1.000000}%
\pgfsetstrokecolor{currentstroke}%
\pgfsetstrokeopacity{0.500000}%
\pgfsetdash{}{0pt}%
\pgfpathmoveto{\pgfqpoint{2.108493in}{3.499680in}}%
\pgfusepath{stroke}%
\end{pgfscope}%
\begin{pgfscope}%
\pgfpathrectangle{\pgfqpoint{0.100000in}{2.413063in}}{\pgfqpoint{5.037500in}{3.427208in}}%
\pgfusepath{clip}%
\pgfsetbuttcap%
\pgfsetroundjoin%
\definecolor{currentfill}{rgb}{0.000000,0.000000,1.000000}%
\pgfsetfillcolor{currentfill}%
\pgfsetfillopacity{0.500000}%
\pgfsetlinewidth{0.250937pt}%
\definecolor{currentstroke}{rgb}{0.000000,0.000000,0.000000}%
\pgfsetstrokecolor{currentstroke}%
\pgfsetstrokeopacity{0.500000}%
\pgfsetdash{}{0pt}%
\pgfsys@defobject{currentmarker}{\pgfqpoint{-0.055556in}{-0.055556in}}{\pgfqpoint{0.055556in}{0.055556in}}{%
\pgfpathmoveto{\pgfqpoint{0.000000in}{-0.055556in}}%
\pgfpathcurveto{\pgfqpoint{0.014734in}{-0.055556in}}{\pgfqpoint{0.028866in}{-0.049702in}}{\pgfqpoint{0.039284in}{-0.039284in}}%
\pgfpathcurveto{\pgfqpoint{0.049702in}{-0.028866in}}{\pgfqpoint{0.055556in}{-0.014734in}}{\pgfqpoint{0.055556in}{0.000000in}}%
\pgfpathcurveto{\pgfqpoint{0.055556in}{0.014734in}}{\pgfqpoint{0.049702in}{0.028866in}}{\pgfqpoint{0.039284in}{0.039284in}}%
\pgfpathcurveto{\pgfqpoint{0.028866in}{0.049702in}}{\pgfqpoint{0.014734in}{0.055556in}}{\pgfqpoint{0.000000in}{0.055556in}}%
\pgfpathcurveto{\pgfqpoint{-0.014734in}{0.055556in}}{\pgfqpoint{-0.028866in}{0.049702in}}{\pgfqpoint{-0.039284in}{0.039284in}}%
\pgfpathcurveto{\pgfqpoint{-0.049702in}{0.028866in}}{\pgfqpoint{-0.055556in}{0.014734in}}{\pgfqpoint{-0.055556in}{0.000000in}}%
\pgfpathcurveto{\pgfqpoint{-0.055556in}{-0.014734in}}{\pgfqpoint{-0.049702in}{-0.028866in}}{\pgfqpoint{-0.039284in}{-0.039284in}}%
\pgfpathcurveto{\pgfqpoint{-0.028866in}{-0.049702in}}{\pgfqpoint{-0.014734in}{-0.055556in}}{\pgfqpoint{0.000000in}{-0.055556in}}%
\pgfpathclose%
\pgfusepath{stroke,fill}%
}%
\begin{pgfscope}%
\pgfsys@transformshift{2.108493in}{3.499680in}%
\pgfsys@useobject{currentmarker}{}%
\end{pgfscope}%
\end{pgfscope}%
\begin{pgfscope}%
\pgfpathrectangle{\pgfqpoint{0.100000in}{2.413063in}}{\pgfqpoint{5.037500in}{3.427208in}}%
\pgfusepath{clip}%
\pgfsetrectcap%
\pgfsetroundjoin%
\pgfsetlinewidth{1.505625pt}%
\definecolor{currentstroke}{rgb}{0.000000,0.000000,1.000000}%
\pgfsetstrokecolor{currentstroke}%
\pgfsetstrokeopacity{0.500000}%
\pgfsetdash{}{0pt}%
\pgfpathmoveto{\pgfqpoint{2.078653in}{3.484405in}}%
\pgfusepath{stroke}%
\end{pgfscope}%
\begin{pgfscope}%
\pgfpathrectangle{\pgfqpoint{0.100000in}{2.413063in}}{\pgfqpoint{5.037500in}{3.427208in}}%
\pgfusepath{clip}%
\pgfsetbuttcap%
\pgfsetroundjoin%
\definecolor{currentfill}{rgb}{0.000000,0.000000,1.000000}%
\pgfsetfillcolor{currentfill}%
\pgfsetfillopacity{0.500000}%
\pgfsetlinewidth{0.250937pt}%
\definecolor{currentstroke}{rgb}{0.000000,0.000000,0.000000}%
\pgfsetstrokecolor{currentstroke}%
\pgfsetstrokeopacity{0.500000}%
\pgfsetdash{}{0pt}%
\pgfsys@defobject{currentmarker}{\pgfqpoint{-0.083333in}{-0.083333in}}{\pgfqpoint{0.083333in}{0.083333in}}{%
\pgfpathmoveto{\pgfqpoint{0.000000in}{-0.083333in}}%
\pgfpathcurveto{\pgfqpoint{0.022100in}{-0.083333in}}{\pgfqpoint{0.043298in}{-0.074553in}}{\pgfqpoint{0.058926in}{-0.058926in}}%
\pgfpathcurveto{\pgfqpoint{0.074553in}{-0.043298in}}{\pgfqpoint{0.083333in}{-0.022100in}}{\pgfqpoint{0.083333in}{0.000000in}}%
\pgfpathcurveto{\pgfqpoint{0.083333in}{0.022100in}}{\pgfqpoint{0.074553in}{0.043298in}}{\pgfqpoint{0.058926in}{0.058926in}}%
\pgfpathcurveto{\pgfqpoint{0.043298in}{0.074553in}}{\pgfqpoint{0.022100in}{0.083333in}}{\pgfqpoint{0.000000in}{0.083333in}}%
\pgfpathcurveto{\pgfqpoint{-0.022100in}{0.083333in}}{\pgfqpoint{-0.043298in}{0.074553in}}{\pgfqpoint{-0.058926in}{0.058926in}}%
\pgfpathcurveto{\pgfqpoint{-0.074553in}{0.043298in}}{\pgfqpoint{-0.083333in}{0.022100in}}{\pgfqpoint{-0.083333in}{0.000000in}}%
\pgfpathcurveto{\pgfqpoint{-0.083333in}{-0.022100in}}{\pgfqpoint{-0.074553in}{-0.043298in}}{\pgfqpoint{-0.058926in}{-0.058926in}}%
\pgfpathcurveto{\pgfqpoint{-0.043298in}{-0.074553in}}{\pgfqpoint{-0.022100in}{-0.083333in}}{\pgfqpoint{0.000000in}{-0.083333in}}%
\pgfpathclose%
\pgfusepath{stroke,fill}%
}%
\begin{pgfscope}%
\pgfsys@transformshift{2.078653in}{3.484405in}%
\pgfsys@useobject{currentmarker}{}%
\end{pgfscope}%
\end{pgfscope}%
\begin{pgfscope}%
\pgfpathrectangle{\pgfqpoint{0.100000in}{2.413063in}}{\pgfqpoint{5.037500in}{3.427208in}}%
\pgfusepath{clip}%
\pgfsetrectcap%
\pgfsetroundjoin%
\pgfsetlinewidth{1.505625pt}%
\definecolor{currentstroke}{rgb}{0.000000,0.000000,1.000000}%
\pgfsetstrokecolor{currentstroke}%
\pgfsetstrokeopacity{0.500000}%
\pgfsetdash{}{0pt}%
\pgfpathmoveto{\pgfqpoint{2.265570in}{3.426888in}}%
\pgfusepath{stroke}%
\end{pgfscope}%
\begin{pgfscope}%
\pgfpathrectangle{\pgfqpoint{0.100000in}{2.413063in}}{\pgfqpoint{5.037500in}{3.427208in}}%
\pgfusepath{clip}%
\pgfsetbuttcap%
\pgfsetroundjoin%
\definecolor{currentfill}{rgb}{0.000000,0.000000,1.000000}%
\pgfsetfillcolor{currentfill}%
\pgfsetfillopacity{0.500000}%
\pgfsetlinewidth{0.250937pt}%
\definecolor{currentstroke}{rgb}{0.000000,0.000000,0.000000}%
\pgfsetstrokecolor{currentstroke}%
\pgfsetstrokeopacity{0.500000}%
\pgfsetdash{}{0pt}%
\pgfsys@defobject{currentmarker}{\pgfqpoint{-0.022222in}{-0.022222in}}{\pgfqpoint{0.022222in}{0.022222in}}{%
\pgfpathmoveto{\pgfqpoint{0.000000in}{-0.022222in}}%
\pgfpathcurveto{\pgfqpoint{0.005893in}{-0.022222in}}{\pgfqpoint{0.011546in}{-0.019881in}}{\pgfqpoint{0.015713in}{-0.015713in}}%
\pgfpathcurveto{\pgfqpoint{0.019881in}{-0.011546in}}{\pgfqpoint{0.022222in}{-0.005893in}}{\pgfqpoint{0.022222in}{0.000000in}}%
\pgfpathcurveto{\pgfqpoint{0.022222in}{0.005893in}}{\pgfqpoint{0.019881in}{0.011546in}}{\pgfqpoint{0.015713in}{0.015713in}}%
\pgfpathcurveto{\pgfqpoint{0.011546in}{0.019881in}}{\pgfqpoint{0.005893in}{0.022222in}}{\pgfqpoint{0.000000in}{0.022222in}}%
\pgfpathcurveto{\pgfqpoint{-0.005893in}{0.022222in}}{\pgfqpoint{-0.011546in}{0.019881in}}{\pgfqpoint{-0.015713in}{0.015713in}}%
\pgfpathcurveto{\pgfqpoint{-0.019881in}{0.011546in}}{\pgfqpoint{-0.022222in}{0.005893in}}{\pgfqpoint{-0.022222in}{0.000000in}}%
\pgfpathcurveto{\pgfqpoint{-0.022222in}{-0.005893in}}{\pgfqpoint{-0.019881in}{-0.011546in}}{\pgfqpoint{-0.015713in}{-0.015713in}}%
\pgfpathcurveto{\pgfqpoint{-0.011546in}{-0.019881in}}{\pgfqpoint{-0.005893in}{-0.022222in}}{\pgfqpoint{0.000000in}{-0.022222in}}%
\pgfpathclose%
\pgfusepath{stroke,fill}%
}%
\begin{pgfscope}%
\pgfsys@transformshift{2.265570in}{3.426888in}%
\pgfsys@useobject{currentmarker}{}%
\end{pgfscope}%
\end{pgfscope}%
\begin{pgfscope}%
\pgfpathrectangle{\pgfqpoint{0.100000in}{2.413063in}}{\pgfqpoint{5.037500in}{3.427208in}}%
\pgfusepath{clip}%
\pgfsetrectcap%
\pgfsetroundjoin%
\pgfsetlinewidth{1.505625pt}%
\definecolor{currentstroke}{rgb}{0.000000,0.000000,1.000000}%
\pgfsetstrokecolor{currentstroke}%
\pgfsetstrokeopacity{0.500000}%
\pgfsetdash{}{0pt}%
\pgfpathmoveto{\pgfqpoint{2.448846in}{3.180064in}}%
\pgfusepath{stroke}%
\end{pgfscope}%
\begin{pgfscope}%
\pgfpathrectangle{\pgfqpoint{0.100000in}{2.413063in}}{\pgfqpoint{5.037500in}{3.427208in}}%
\pgfusepath{clip}%
\pgfsetbuttcap%
\pgfsetroundjoin%
\definecolor{currentfill}{rgb}{0.000000,0.000000,1.000000}%
\pgfsetfillcolor{currentfill}%
\pgfsetfillopacity{0.500000}%
\pgfsetlinewidth{0.250937pt}%
\definecolor{currentstroke}{rgb}{0.000000,0.000000,0.000000}%
\pgfsetstrokecolor{currentstroke}%
\pgfsetstrokeopacity{0.500000}%
\pgfsetdash{}{0pt}%
\pgfsys@defobject{currentmarker}{\pgfqpoint{-0.025000in}{-0.025000in}}{\pgfqpoint{0.025000in}{0.025000in}}{%
\pgfpathmoveto{\pgfqpoint{0.000000in}{-0.025000in}}%
\pgfpathcurveto{\pgfqpoint{0.006630in}{-0.025000in}}{\pgfqpoint{0.012989in}{-0.022366in}}{\pgfqpoint{0.017678in}{-0.017678in}}%
\pgfpathcurveto{\pgfqpoint{0.022366in}{-0.012989in}}{\pgfqpoint{0.025000in}{-0.006630in}}{\pgfqpoint{0.025000in}{0.000000in}}%
\pgfpathcurveto{\pgfqpoint{0.025000in}{0.006630in}}{\pgfqpoint{0.022366in}{0.012989in}}{\pgfqpoint{0.017678in}{0.017678in}}%
\pgfpathcurveto{\pgfqpoint{0.012989in}{0.022366in}}{\pgfqpoint{0.006630in}{0.025000in}}{\pgfqpoint{0.000000in}{0.025000in}}%
\pgfpathcurveto{\pgfqpoint{-0.006630in}{0.025000in}}{\pgfqpoint{-0.012989in}{0.022366in}}{\pgfqpoint{-0.017678in}{0.017678in}}%
\pgfpathcurveto{\pgfqpoint{-0.022366in}{0.012989in}}{\pgfqpoint{-0.025000in}{0.006630in}}{\pgfqpoint{-0.025000in}{0.000000in}}%
\pgfpathcurveto{\pgfqpoint{-0.025000in}{-0.006630in}}{\pgfqpoint{-0.022366in}{-0.012989in}}{\pgfqpoint{-0.017678in}{-0.017678in}}%
\pgfpathcurveto{\pgfqpoint{-0.012989in}{-0.022366in}}{\pgfqpoint{-0.006630in}{-0.025000in}}{\pgfqpoint{0.000000in}{-0.025000in}}%
\pgfpathclose%
\pgfusepath{stroke,fill}%
}%
\begin{pgfscope}%
\pgfsys@transformshift{2.448846in}{3.180064in}%
\pgfsys@useobject{currentmarker}{}%
\end{pgfscope}%
\end{pgfscope}%
\begin{pgfscope}%
\pgfpathrectangle{\pgfqpoint{0.100000in}{2.413063in}}{\pgfqpoint{5.037500in}{3.427208in}}%
\pgfusepath{clip}%
\pgfsetrectcap%
\pgfsetroundjoin%
\pgfsetlinewidth{1.505625pt}%
\definecolor{currentstroke}{rgb}{0.000000,0.000000,1.000000}%
\pgfsetstrokecolor{currentstroke}%
\pgfsetstrokeopacity{0.500000}%
\pgfsetdash{}{0pt}%
\pgfpathmoveto{\pgfqpoint{2.649125in}{3.663744in}}%
\pgfusepath{stroke}%
\end{pgfscope}%
\begin{pgfscope}%
\pgfpathrectangle{\pgfqpoint{0.100000in}{2.413063in}}{\pgfqpoint{5.037500in}{3.427208in}}%
\pgfusepath{clip}%
\pgfsetbuttcap%
\pgfsetroundjoin%
\definecolor{currentfill}{rgb}{0.000000,0.000000,1.000000}%
\pgfsetfillcolor{currentfill}%
\pgfsetfillopacity{0.500000}%
\pgfsetlinewidth{0.250937pt}%
\definecolor{currentstroke}{rgb}{0.000000,0.000000,0.000000}%
\pgfsetstrokecolor{currentstroke}%
\pgfsetstrokeopacity{0.500000}%
\pgfsetdash{}{0pt}%
\pgfsys@defobject{currentmarker}{\pgfqpoint{-0.016667in}{-0.016667in}}{\pgfqpoint{0.016667in}{0.016667in}}{%
\pgfpathmoveto{\pgfqpoint{0.000000in}{-0.016667in}}%
\pgfpathcurveto{\pgfqpoint{0.004420in}{-0.016667in}}{\pgfqpoint{0.008660in}{-0.014911in}}{\pgfqpoint{0.011785in}{-0.011785in}}%
\pgfpathcurveto{\pgfqpoint{0.014911in}{-0.008660in}}{\pgfqpoint{0.016667in}{-0.004420in}}{\pgfqpoint{0.016667in}{0.000000in}}%
\pgfpathcurveto{\pgfqpoint{0.016667in}{0.004420in}}{\pgfqpoint{0.014911in}{0.008660in}}{\pgfqpoint{0.011785in}{0.011785in}}%
\pgfpathcurveto{\pgfqpoint{0.008660in}{0.014911in}}{\pgfqpoint{0.004420in}{0.016667in}}{\pgfqpoint{0.000000in}{0.016667in}}%
\pgfpathcurveto{\pgfqpoint{-0.004420in}{0.016667in}}{\pgfqpoint{-0.008660in}{0.014911in}}{\pgfqpoint{-0.011785in}{0.011785in}}%
\pgfpathcurveto{\pgfqpoint{-0.014911in}{0.008660in}}{\pgfqpoint{-0.016667in}{0.004420in}}{\pgfqpoint{-0.016667in}{0.000000in}}%
\pgfpathcurveto{\pgfqpoint{-0.016667in}{-0.004420in}}{\pgfqpoint{-0.014911in}{-0.008660in}}{\pgfqpoint{-0.011785in}{-0.011785in}}%
\pgfpathcurveto{\pgfqpoint{-0.008660in}{-0.014911in}}{\pgfqpoint{-0.004420in}{-0.016667in}}{\pgfqpoint{0.000000in}{-0.016667in}}%
\pgfpathclose%
\pgfusepath{stroke,fill}%
}%
\begin{pgfscope}%
\pgfsys@transformshift{2.649125in}{3.663744in}%
\pgfsys@useobject{currentmarker}{}%
\end{pgfscope}%
\end{pgfscope}%
\begin{pgfscope}%
\pgfpathrectangle{\pgfqpoint{0.100000in}{2.413063in}}{\pgfqpoint{5.037500in}{3.427208in}}%
\pgfusepath{clip}%
\pgfsetrectcap%
\pgfsetroundjoin%
\pgfsetlinewidth{1.505625pt}%
\definecolor{currentstroke}{rgb}{0.000000,0.000000,1.000000}%
\pgfsetstrokecolor{currentstroke}%
\pgfsetstrokeopacity{0.500000}%
\pgfsetdash{}{0pt}%
\pgfpathmoveto{\pgfqpoint{2.896083in}{3.638458in}}%
\pgfusepath{stroke}%
\end{pgfscope}%
\begin{pgfscope}%
\pgfpathrectangle{\pgfqpoint{0.100000in}{2.413063in}}{\pgfqpoint{5.037500in}{3.427208in}}%
\pgfusepath{clip}%
\pgfsetbuttcap%
\pgfsetroundjoin%
\definecolor{currentfill}{rgb}{0.000000,0.000000,1.000000}%
\pgfsetfillcolor{currentfill}%
\pgfsetfillopacity{0.500000}%
\pgfsetlinewidth{0.250937pt}%
\definecolor{currentstroke}{rgb}{0.000000,0.000000,0.000000}%
\pgfsetstrokecolor{currentstroke}%
\pgfsetstrokeopacity{0.500000}%
\pgfsetdash{}{0pt}%
\pgfsys@defobject{currentmarker}{\pgfqpoint{-0.008333in}{-0.008333in}}{\pgfqpoint{0.008333in}{0.008333in}}{%
\pgfpathmoveto{\pgfqpoint{0.000000in}{-0.008333in}}%
\pgfpathcurveto{\pgfqpoint{0.002210in}{-0.008333in}}{\pgfqpoint{0.004330in}{-0.007455in}}{\pgfqpoint{0.005893in}{-0.005893in}}%
\pgfpathcurveto{\pgfqpoint{0.007455in}{-0.004330in}}{\pgfqpoint{0.008333in}{-0.002210in}}{\pgfqpoint{0.008333in}{0.000000in}}%
\pgfpathcurveto{\pgfqpoint{0.008333in}{0.002210in}}{\pgfqpoint{0.007455in}{0.004330in}}{\pgfqpoint{0.005893in}{0.005893in}}%
\pgfpathcurveto{\pgfqpoint{0.004330in}{0.007455in}}{\pgfqpoint{0.002210in}{0.008333in}}{\pgfqpoint{0.000000in}{0.008333in}}%
\pgfpathcurveto{\pgfqpoint{-0.002210in}{0.008333in}}{\pgfqpoint{-0.004330in}{0.007455in}}{\pgfqpoint{-0.005893in}{0.005893in}}%
\pgfpathcurveto{\pgfqpoint{-0.007455in}{0.004330in}}{\pgfqpoint{-0.008333in}{0.002210in}}{\pgfqpoint{-0.008333in}{0.000000in}}%
\pgfpathcurveto{\pgfqpoint{-0.008333in}{-0.002210in}}{\pgfqpoint{-0.007455in}{-0.004330in}}{\pgfqpoint{-0.005893in}{-0.005893in}}%
\pgfpathcurveto{\pgfqpoint{-0.004330in}{-0.007455in}}{\pgfqpoint{-0.002210in}{-0.008333in}}{\pgfqpoint{0.000000in}{-0.008333in}}%
\pgfpathclose%
\pgfusepath{stroke,fill}%
}%
\begin{pgfscope}%
\pgfsys@transformshift{2.896083in}{3.638458in}%
\pgfsys@useobject{currentmarker}{}%
\end{pgfscope}%
\end{pgfscope}%
\begin{pgfscope}%
\pgfpathrectangle{\pgfqpoint{0.100000in}{2.413063in}}{\pgfqpoint{5.037500in}{3.427208in}}%
\pgfusepath{clip}%
\pgfsetrectcap%
\pgfsetroundjoin%
\pgfsetlinewidth{1.505625pt}%
\definecolor{currentstroke}{rgb}{0.000000,0.000000,1.000000}%
\pgfsetstrokecolor{currentstroke}%
\pgfsetstrokeopacity{0.500000}%
\pgfsetdash{}{0pt}%
\pgfpathmoveto{\pgfqpoint{2.774645in}{3.513664in}}%
\pgfusepath{stroke}%
\end{pgfscope}%
\begin{pgfscope}%
\pgfpathrectangle{\pgfqpoint{0.100000in}{2.413063in}}{\pgfqpoint{5.037500in}{3.427208in}}%
\pgfusepath{clip}%
\pgfsetbuttcap%
\pgfsetroundjoin%
\definecolor{currentfill}{rgb}{0.000000,0.000000,1.000000}%
\pgfsetfillcolor{currentfill}%
\pgfsetfillopacity{0.500000}%
\pgfsetlinewidth{0.250937pt}%
\definecolor{currentstroke}{rgb}{0.000000,0.000000,0.000000}%
\pgfsetstrokecolor{currentstroke}%
\pgfsetstrokeopacity{0.500000}%
\pgfsetdash{}{0pt}%
\pgfsys@defobject{currentmarker}{\pgfqpoint{-0.016667in}{-0.016667in}}{\pgfqpoint{0.016667in}{0.016667in}}{%
\pgfpathmoveto{\pgfqpoint{0.000000in}{-0.016667in}}%
\pgfpathcurveto{\pgfqpoint{0.004420in}{-0.016667in}}{\pgfqpoint{0.008660in}{-0.014911in}}{\pgfqpoint{0.011785in}{-0.011785in}}%
\pgfpathcurveto{\pgfqpoint{0.014911in}{-0.008660in}}{\pgfqpoint{0.016667in}{-0.004420in}}{\pgfqpoint{0.016667in}{0.000000in}}%
\pgfpathcurveto{\pgfqpoint{0.016667in}{0.004420in}}{\pgfqpoint{0.014911in}{0.008660in}}{\pgfqpoint{0.011785in}{0.011785in}}%
\pgfpathcurveto{\pgfqpoint{0.008660in}{0.014911in}}{\pgfqpoint{0.004420in}{0.016667in}}{\pgfqpoint{0.000000in}{0.016667in}}%
\pgfpathcurveto{\pgfqpoint{-0.004420in}{0.016667in}}{\pgfqpoint{-0.008660in}{0.014911in}}{\pgfqpoint{-0.011785in}{0.011785in}}%
\pgfpathcurveto{\pgfqpoint{-0.014911in}{0.008660in}}{\pgfqpoint{-0.016667in}{0.004420in}}{\pgfqpoint{-0.016667in}{0.000000in}}%
\pgfpathcurveto{\pgfqpoint{-0.016667in}{-0.004420in}}{\pgfqpoint{-0.014911in}{-0.008660in}}{\pgfqpoint{-0.011785in}{-0.011785in}}%
\pgfpathcurveto{\pgfqpoint{-0.008660in}{-0.014911in}}{\pgfqpoint{-0.004420in}{-0.016667in}}{\pgfqpoint{0.000000in}{-0.016667in}}%
\pgfpathclose%
\pgfusepath{stroke,fill}%
}%
\begin{pgfscope}%
\pgfsys@transformshift{2.774645in}{3.513664in}%
\pgfsys@useobject{currentmarker}{}%
\end{pgfscope}%
\end{pgfscope}%
\begin{pgfscope}%
\pgfpathrectangle{\pgfqpoint{0.100000in}{2.413063in}}{\pgfqpoint{5.037500in}{3.427208in}}%
\pgfusepath{clip}%
\pgfsetrectcap%
\pgfsetroundjoin%
\pgfsetlinewidth{1.505625pt}%
\definecolor{currentstroke}{rgb}{0.000000,0.000000,1.000000}%
\pgfsetstrokecolor{currentstroke}%
\pgfsetstrokeopacity{0.500000}%
\pgfsetdash{}{0pt}%
\pgfpathmoveto{\pgfqpoint{2.598833in}{3.103133in}}%
\pgfusepath{stroke}%
\end{pgfscope}%
\begin{pgfscope}%
\pgfpathrectangle{\pgfqpoint{0.100000in}{2.413063in}}{\pgfqpoint{5.037500in}{3.427208in}}%
\pgfusepath{clip}%
\pgfsetbuttcap%
\pgfsetroundjoin%
\definecolor{currentfill}{rgb}{0.000000,0.000000,1.000000}%
\pgfsetfillcolor{currentfill}%
\pgfsetfillopacity{0.500000}%
\pgfsetlinewidth{0.250937pt}%
\definecolor{currentstroke}{rgb}{0.000000,0.000000,0.000000}%
\pgfsetstrokecolor{currentstroke}%
\pgfsetstrokeopacity{0.500000}%
\pgfsetdash{}{0pt}%
\pgfsys@defobject{currentmarker}{\pgfqpoint{-0.044444in}{-0.044444in}}{\pgfqpoint{0.044444in}{0.044444in}}{%
\pgfpathmoveto{\pgfqpoint{0.000000in}{-0.044444in}}%
\pgfpathcurveto{\pgfqpoint{0.011787in}{-0.044444in}}{\pgfqpoint{0.023092in}{-0.039761in}}{\pgfqpoint{0.031427in}{-0.031427in}}%
\pgfpathcurveto{\pgfqpoint{0.039761in}{-0.023092in}}{\pgfqpoint{0.044444in}{-0.011787in}}{\pgfqpoint{0.044444in}{0.000000in}}%
\pgfpathcurveto{\pgfqpoint{0.044444in}{0.011787in}}{\pgfqpoint{0.039761in}{0.023092in}}{\pgfqpoint{0.031427in}{0.031427in}}%
\pgfpathcurveto{\pgfqpoint{0.023092in}{0.039761in}}{\pgfqpoint{0.011787in}{0.044444in}}{\pgfqpoint{0.000000in}{0.044444in}}%
\pgfpathcurveto{\pgfqpoint{-0.011787in}{0.044444in}}{\pgfqpoint{-0.023092in}{0.039761in}}{\pgfqpoint{-0.031427in}{0.031427in}}%
\pgfpathcurveto{\pgfqpoint{-0.039761in}{0.023092in}}{\pgfqpoint{-0.044444in}{0.011787in}}{\pgfqpoint{-0.044444in}{0.000000in}}%
\pgfpathcurveto{\pgfqpoint{-0.044444in}{-0.011787in}}{\pgfqpoint{-0.039761in}{-0.023092in}}{\pgfqpoint{-0.031427in}{-0.031427in}}%
\pgfpathcurveto{\pgfqpoint{-0.023092in}{-0.039761in}}{\pgfqpoint{-0.011787in}{-0.044444in}}{\pgfqpoint{0.000000in}{-0.044444in}}%
\pgfpathclose%
\pgfusepath{stroke,fill}%
}%
\begin{pgfscope}%
\pgfsys@transformshift{2.598833in}{3.103133in}%
\pgfsys@useobject{currentmarker}{}%
\end{pgfscope}%
\end{pgfscope}%
\begin{pgfscope}%
\pgfpathrectangle{\pgfqpoint{0.100000in}{2.413063in}}{\pgfqpoint{5.037500in}{3.427208in}}%
\pgfusepath{clip}%
\pgfsetrectcap%
\pgfsetroundjoin%
\pgfsetlinewidth{1.505625pt}%
\definecolor{currentstroke}{rgb}{0.000000,0.000000,1.000000}%
\pgfsetstrokecolor{currentstroke}%
\pgfsetstrokeopacity{0.500000}%
\pgfsetdash{}{0pt}%
\pgfpathmoveto{\pgfqpoint{2.591727in}{3.423086in}}%
\pgfusepath{stroke}%
\end{pgfscope}%
\begin{pgfscope}%
\pgfpathrectangle{\pgfqpoint{0.100000in}{2.413063in}}{\pgfqpoint{5.037500in}{3.427208in}}%
\pgfusepath{clip}%
\pgfsetbuttcap%
\pgfsetroundjoin%
\definecolor{currentfill}{rgb}{0.000000,0.000000,1.000000}%
\pgfsetfillcolor{currentfill}%
\pgfsetfillopacity{0.500000}%
\pgfsetlinewidth{0.250937pt}%
\definecolor{currentstroke}{rgb}{0.000000,0.000000,0.000000}%
\pgfsetstrokecolor{currentstroke}%
\pgfsetstrokeopacity{0.500000}%
\pgfsetdash{}{0pt}%
\pgfsys@defobject{currentmarker}{\pgfqpoint{-0.013889in}{-0.013889in}}{\pgfqpoint{0.013889in}{0.013889in}}{%
\pgfpathmoveto{\pgfqpoint{0.000000in}{-0.013889in}}%
\pgfpathcurveto{\pgfqpoint{0.003683in}{-0.013889in}}{\pgfqpoint{0.007216in}{-0.012425in}}{\pgfqpoint{0.009821in}{-0.009821in}}%
\pgfpathcurveto{\pgfqpoint{0.012425in}{-0.007216in}}{\pgfqpoint{0.013889in}{-0.003683in}}{\pgfqpoint{0.013889in}{0.000000in}}%
\pgfpathcurveto{\pgfqpoint{0.013889in}{0.003683in}}{\pgfqpoint{0.012425in}{0.007216in}}{\pgfqpoint{0.009821in}{0.009821in}}%
\pgfpathcurveto{\pgfqpoint{0.007216in}{0.012425in}}{\pgfqpoint{0.003683in}{0.013889in}}{\pgfqpoint{0.000000in}{0.013889in}}%
\pgfpathcurveto{\pgfqpoint{-0.003683in}{0.013889in}}{\pgfqpoint{-0.007216in}{0.012425in}}{\pgfqpoint{-0.009821in}{0.009821in}}%
\pgfpathcurveto{\pgfqpoint{-0.012425in}{0.007216in}}{\pgfqpoint{-0.013889in}{0.003683in}}{\pgfqpoint{-0.013889in}{0.000000in}}%
\pgfpathcurveto{\pgfqpoint{-0.013889in}{-0.003683in}}{\pgfqpoint{-0.012425in}{-0.007216in}}{\pgfqpoint{-0.009821in}{-0.009821in}}%
\pgfpathcurveto{\pgfqpoint{-0.007216in}{-0.012425in}}{\pgfqpoint{-0.003683in}{-0.013889in}}{\pgfqpoint{0.000000in}{-0.013889in}}%
\pgfpathclose%
\pgfusepath{stroke,fill}%
}%
\begin{pgfscope}%
\pgfsys@transformshift{2.591727in}{3.423086in}%
\pgfsys@useobject{currentmarker}{}%
\end{pgfscope}%
\end{pgfscope}%
\begin{pgfscope}%
\pgfpathrectangle{\pgfqpoint{0.100000in}{2.413063in}}{\pgfqpoint{5.037500in}{3.427208in}}%
\pgfusepath{clip}%
\pgfsetrectcap%
\pgfsetroundjoin%
\pgfsetlinewidth{1.505625pt}%
\definecolor{currentstroke}{rgb}{0.000000,0.000000,1.000000}%
\pgfsetstrokecolor{currentstroke}%
\pgfsetstrokeopacity{0.500000}%
\pgfsetdash{}{0pt}%
\pgfpathmoveto{\pgfqpoint{2.469056in}{3.700922in}}%
\pgfusepath{stroke}%
\end{pgfscope}%
\begin{pgfscope}%
\pgfpathrectangle{\pgfqpoint{0.100000in}{2.413063in}}{\pgfqpoint{5.037500in}{3.427208in}}%
\pgfusepath{clip}%
\pgfsetbuttcap%
\pgfsetroundjoin%
\definecolor{currentfill}{rgb}{0.000000,0.000000,1.000000}%
\pgfsetfillcolor{currentfill}%
\pgfsetfillopacity{0.500000}%
\pgfsetlinewidth{0.250937pt}%
\definecolor{currentstroke}{rgb}{0.000000,0.000000,0.000000}%
\pgfsetstrokecolor{currentstroke}%
\pgfsetstrokeopacity{0.500000}%
\pgfsetdash{}{0pt}%
\pgfsys@defobject{currentmarker}{\pgfqpoint{-0.027778in}{-0.027778in}}{\pgfqpoint{0.027778in}{0.027778in}}{%
\pgfpathmoveto{\pgfqpoint{0.000000in}{-0.027778in}}%
\pgfpathcurveto{\pgfqpoint{0.007367in}{-0.027778in}}{\pgfqpoint{0.014433in}{-0.024851in}}{\pgfqpoint{0.019642in}{-0.019642in}}%
\pgfpathcurveto{\pgfqpoint{0.024851in}{-0.014433in}}{\pgfqpoint{0.027778in}{-0.007367in}}{\pgfqpoint{0.027778in}{0.000000in}}%
\pgfpathcurveto{\pgfqpoint{0.027778in}{0.007367in}}{\pgfqpoint{0.024851in}{0.014433in}}{\pgfqpoint{0.019642in}{0.019642in}}%
\pgfpathcurveto{\pgfqpoint{0.014433in}{0.024851in}}{\pgfqpoint{0.007367in}{0.027778in}}{\pgfqpoint{0.000000in}{0.027778in}}%
\pgfpathcurveto{\pgfqpoint{-0.007367in}{0.027778in}}{\pgfqpoint{-0.014433in}{0.024851in}}{\pgfqpoint{-0.019642in}{0.019642in}}%
\pgfpathcurveto{\pgfqpoint{-0.024851in}{0.014433in}}{\pgfqpoint{-0.027778in}{0.007367in}}{\pgfqpoint{-0.027778in}{0.000000in}}%
\pgfpathcurveto{\pgfqpoint{-0.027778in}{-0.007367in}}{\pgfqpoint{-0.024851in}{-0.014433in}}{\pgfqpoint{-0.019642in}{-0.019642in}}%
\pgfpathcurveto{\pgfqpoint{-0.014433in}{-0.024851in}}{\pgfqpoint{-0.007367in}{-0.027778in}}{\pgfqpoint{0.000000in}{-0.027778in}}%
\pgfpathclose%
\pgfusepath{stroke,fill}%
}%
\begin{pgfscope}%
\pgfsys@transformshift{2.469056in}{3.700922in}%
\pgfsys@useobject{currentmarker}{}%
\end{pgfscope}%
\end{pgfscope}%
\begin{pgfscope}%
\pgfpathrectangle{\pgfqpoint{0.100000in}{2.413063in}}{\pgfqpoint{5.037500in}{3.427208in}}%
\pgfusepath{clip}%
\pgfsetrectcap%
\pgfsetroundjoin%
\pgfsetlinewidth{1.505625pt}%
\definecolor{currentstroke}{rgb}{0.678431,1.000000,0.184314}%
\pgfsetstrokecolor{currentstroke}%
\pgfsetstrokeopacity{0.500000}%
\pgfsetdash{}{0pt}%
\pgfpathmoveto{\pgfqpoint{1.364191in}{4.729322in}}%
\pgfusepath{stroke}%
\end{pgfscope}%
\begin{pgfscope}%
\pgfpathrectangle{\pgfqpoint{0.100000in}{2.413063in}}{\pgfqpoint{5.037500in}{3.427208in}}%
\pgfusepath{clip}%
\pgfsetbuttcap%
\pgfsetroundjoin%
\definecolor{currentfill}{rgb}{0.678431,1.000000,0.184314}%
\pgfsetfillcolor{currentfill}%
\pgfsetfillopacity{0.500000}%
\pgfsetlinewidth{0.250937pt}%
\definecolor{currentstroke}{rgb}{0.000000,0.000000,0.000000}%
\pgfsetstrokecolor{currentstroke}%
\pgfsetstrokeopacity{0.500000}%
\pgfsetdash{}{0pt}%
\pgfsys@defobject{currentmarker}{\pgfqpoint{-0.016667in}{-0.016667in}}{\pgfqpoint{0.016667in}{0.016667in}}{%
\pgfpathmoveto{\pgfqpoint{0.000000in}{-0.016667in}}%
\pgfpathcurveto{\pgfqpoint{0.004420in}{-0.016667in}}{\pgfqpoint{0.008660in}{-0.014911in}}{\pgfqpoint{0.011785in}{-0.011785in}}%
\pgfpathcurveto{\pgfqpoint{0.014911in}{-0.008660in}}{\pgfqpoint{0.016667in}{-0.004420in}}{\pgfqpoint{0.016667in}{0.000000in}}%
\pgfpathcurveto{\pgfqpoint{0.016667in}{0.004420in}}{\pgfqpoint{0.014911in}{0.008660in}}{\pgfqpoint{0.011785in}{0.011785in}}%
\pgfpathcurveto{\pgfqpoint{0.008660in}{0.014911in}}{\pgfqpoint{0.004420in}{0.016667in}}{\pgfqpoint{0.000000in}{0.016667in}}%
\pgfpathcurveto{\pgfqpoint{-0.004420in}{0.016667in}}{\pgfqpoint{-0.008660in}{0.014911in}}{\pgfqpoint{-0.011785in}{0.011785in}}%
\pgfpathcurveto{\pgfqpoint{-0.014911in}{0.008660in}}{\pgfqpoint{-0.016667in}{0.004420in}}{\pgfqpoint{-0.016667in}{0.000000in}}%
\pgfpathcurveto{\pgfqpoint{-0.016667in}{-0.004420in}}{\pgfqpoint{-0.014911in}{-0.008660in}}{\pgfqpoint{-0.011785in}{-0.011785in}}%
\pgfpathcurveto{\pgfqpoint{-0.008660in}{-0.014911in}}{\pgfqpoint{-0.004420in}{-0.016667in}}{\pgfqpoint{0.000000in}{-0.016667in}}%
\pgfpathclose%
\pgfusepath{stroke,fill}%
}%
\begin{pgfscope}%
\pgfsys@transformshift{1.364191in}{4.729322in}%
\pgfsys@useobject{currentmarker}{}%
\end{pgfscope}%
\end{pgfscope}%
\begin{pgfscope}%
\pgfpathrectangle{\pgfqpoint{0.100000in}{2.413063in}}{\pgfqpoint{5.037500in}{3.427208in}}%
\pgfusepath{clip}%
\pgfsetrectcap%
\pgfsetroundjoin%
\pgfsetlinewidth{1.505625pt}%
\definecolor{currentstroke}{rgb}{0.678431,1.000000,0.184314}%
\pgfsetstrokecolor{currentstroke}%
\pgfsetstrokeopacity{0.500000}%
\pgfsetdash{}{0pt}%
\pgfpathmoveto{\pgfqpoint{1.341563in}{4.673966in}}%
\pgfusepath{stroke}%
\end{pgfscope}%
\begin{pgfscope}%
\pgfpathrectangle{\pgfqpoint{0.100000in}{2.413063in}}{\pgfqpoint{5.037500in}{3.427208in}}%
\pgfusepath{clip}%
\pgfsetbuttcap%
\pgfsetroundjoin%
\definecolor{currentfill}{rgb}{0.678431,1.000000,0.184314}%
\pgfsetfillcolor{currentfill}%
\pgfsetfillopacity{0.500000}%
\pgfsetlinewidth{0.250937pt}%
\definecolor{currentstroke}{rgb}{0.000000,0.000000,0.000000}%
\pgfsetstrokecolor{currentstroke}%
\pgfsetstrokeopacity{0.500000}%
\pgfsetdash{}{0pt}%
\pgfsys@defobject{currentmarker}{\pgfqpoint{-0.025000in}{-0.025000in}}{\pgfqpoint{0.025000in}{0.025000in}}{%
\pgfpathmoveto{\pgfqpoint{0.000000in}{-0.025000in}}%
\pgfpathcurveto{\pgfqpoint{0.006630in}{-0.025000in}}{\pgfqpoint{0.012989in}{-0.022366in}}{\pgfqpoint{0.017678in}{-0.017678in}}%
\pgfpathcurveto{\pgfqpoint{0.022366in}{-0.012989in}}{\pgfqpoint{0.025000in}{-0.006630in}}{\pgfqpoint{0.025000in}{0.000000in}}%
\pgfpathcurveto{\pgfqpoint{0.025000in}{0.006630in}}{\pgfqpoint{0.022366in}{0.012989in}}{\pgfqpoint{0.017678in}{0.017678in}}%
\pgfpathcurveto{\pgfqpoint{0.012989in}{0.022366in}}{\pgfqpoint{0.006630in}{0.025000in}}{\pgfqpoint{0.000000in}{0.025000in}}%
\pgfpathcurveto{\pgfqpoint{-0.006630in}{0.025000in}}{\pgfqpoint{-0.012989in}{0.022366in}}{\pgfqpoint{-0.017678in}{0.017678in}}%
\pgfpathcurveto{\pgfqpoint{-0.022366in}{0.012989in}}{\pgfqpoint{-0.025000in}{0.006630in}}{\pgfqpoint{-0.025000in}{0.000000in}}%
\pgfpathcurveto{\pgfqpoint{-0.025000in}{-0.006630in}}{\pgfqpoint{-0.022366in}{-0.012989in}}{\pgfqpoint{-0.017678in}{-0.017678in}}%
\pgfpathcurveto{\pgfqpoint{-0.012989in}{-0.022366in}}{\pgfqpoint{-0.006630in}{-0.025000in}}{\pgfqpoint{0.000000in}{-0.025000in}}%
\pgfpathclose%
\pgfusepath{stroke,fill}%
}%
\begin{pgfscope}%
\pgfsys@transformshift{1.341563in}{4.673966in}%
\pgfsys@useobject{currentmarker}{}%
\end{pgfscope}%
\end{pgfscope}%
\begin{pgfscope}%
\pgfpathrectangle{\pgfqpoint{0.100000in}{2.413063in}}{\pgfqpoint{5.037500in}{3.427208in}}%
\pgfusepath{clip}%
\pgfsetrectcap%
\pgfsetroundjoin%
\pgfsetlinewidth{1.505625pt}%
\definecolor{currentstroke}{rgb}{0.678431,1.000000,0.184314}%
\pgfsetstrokecolor{currentstroke}%
\pgfsetstrokeopacity{0.500000}%
\pgfsetdash{}{0pt}%
\pgfpathmoveto{\pgfqpoint{1.347653in}{4.556899in}}%
\pgfusepath{stroke}%
\end{pgfscope}%
\begin{pgfscope}%
\pgfpathrectangle{\pgfqpoint{0.100000in}{2.413063in}}{\pgfqpoint{5.037500in}{3.427208in}}%
\pgfusepath{clip}%
\pgfsetbuttcap%
\pgfsetroundjoin%
\definecolor{currentfill}{rgb}{0.678431,1.000000,0.184314}%
\pgfsetfillcolor{currentfill}%
\pgfsetfillopacity{0.500000}%
\pgfsetlinewidth{0.250937pt}%
\definecolor{currentstroke}{rgb}{0.000000,0.000000,0.000000}%
\pgfsetstrokecolor{currentstroke}%
\pgfsetstrokeopacity{0.500000}%
\pgfsetdash{}{0pt}%
\pgfsys@defobject{currentmarker}{\pgfqpoint{-0.022222in}{-0.022222in}}{\pgfqpoint{0.022222in}{0.022222in}}{%
\pgfpathmoveto{\pgfqpoint{0.000000in}{-0.022222in}}%
\pgfpathcurveto{\pgfqpoint{0.005893in}{-0.022222in}}{\pgfqpoint{0.011546in}{-0.019881in}}{\pgfqpoint{0.015713in}{-0.015713in}}%
\pgfpathcurveto{\pgfqpoint{0.019881in}{-0.011546in}}{\pgfqpoint{0.022222in}{-0.005893in}}{\pgfqpoint{0.022222in}{0.000000in}}%
\pgfpathcurveto{\pgfqpoint{0.022222in}{0.005893in}}{\pgfqpoint{0.019881in}{0.011546in}}{\pgfqpoint{0.015713in}{0.015713in}}%
\pgfpathcurveto{\pgfqpoint{0.011546in}{0.019881in}}{\pgfqpoint{0.005893in}{0.022222in}}{\pgfqpoint{0.000000in}{0.022222in}}%
\pgfpathcurveto{\pgfqpoint{-0.005893in}{0.022222in}}{\pgfqpoint{-0.011546in}{0.019881in}}{\pgfqpoint{-0.015713in}{0.015713in}}%
\pgfpathcurveto{\pgfqpoint{-0.019881in}{0.011546in}}{\pgfqpoint{-0.022222in}{0.005893in}}{\pgfqpoint{-0.022222in}{0.000000in}}%
\pgfpathcurveto{\pgfqpoint{-0.022222in}{-0.005893in}}{\pgfqpoint{-0.019881in}{-0.011546in}}{\pgfqpoint{-0.015713in}{-0.015713in}}%
\pgfpathcurveto{\pgfqpoint{-0.011546in}{-0.019881in}}{\pgfqpoint{-0.005893in}{-0.022222in}}{\pgfqpoint{0.000000in}{-0.022222in}}%
\pgfpathclose%
\pgfusepath{stroke,fill}%
}%
\begin{pgfscope}%
\pgfsys@transformshift{1.347653in}{4.556899in}%
\pgfsys@useobject{currentmarker}{}%
\end{pgfscope}%
\end{pgfscope}%
\begin{pgfscope}%
\pgfpathrectangle{\pgfqpoint{0.100000in}{2.413063in}}{\pgfqpoint{5.037500in}{3.427208in}}%
\pgfusepath{clip}%
\pgfsetrectcap%
\pgfsetroundjoin%
\pgfsetlinewidth{1.505625pt}%
\definecolor{currentstroke}{rgb}{0.678431,1.000000,0.184314}%
\pgfsetstrokecolor{currentstroke}%
\pgfsetstrokeopacity{0.500000}%
\pgfsetdash{}{0pt}%
\pgfpathmoveto{\pgfqpoint{1.108330in}{4.236603in}}%
\pgfusepath{stroke}%
\end{pgfscope}%
\begin{pgfscope}%
\pgfpathrectangle{\pgfqpoint{0.100000in}{2.413063in}}{\pgfqpoint{5.037500in}{3.427208in}}%
\pgfusepath{clip}%
\pgfsetbuttcap%
\pgfsetroundjoin%
\definecolor{currentfill}{rgb}{0.678431,1.000000,0.184314}%
\pgfsetfillcolor{currentfill}%
\pgfsetfillopacity{0.500000}%
\pgfsetlinewidth{0.250937pt}%
\definecolor{currentstroke}{rgb}{0.000000,0.000000,0.000000}%
\pgfsetstrokecolor{currentstroke}%
\pgfsetstrokeopacity{0.500000}%
\pgfsetdash{}{0pt}%
\pgfsys@defobject{currentmarker}{\pgfqpoint{-0.027778in}{-0.027778in}}{\pgfqpoint{0.027778in}{0.027778in}}{%
\pgfpathmoveto{\pgfqpoint{0.000000in}{-0.027778in}}%
\pgfpathcurveto{\pgfqpoint{0.007367in}{-0.027778in}}{\pgfqpoint{0.014433in}{-0.024851in}}{\pgfqpoint{0.019642in}{-0.019642in}}%
\pgfpathcurveto{\pgfqpoint{0.024851in}{-0.014433in}}{\pgfqpoint{0.027778in}{-0.007367in}}{\pgfqpoint{0.027778in}{0.000000in}}%
\pgfpathcurveto{\pgfqpoint{0.027778in}{0.007367in}}{\pgfqpoint{0.024851in}{0.014433in}}{\pgfqpoint{0.019642in}{0.019642in}}%
\pgfpathcurveto{\pgfqpoint{0.014433in}{0.024851in}}{\pgfqpoint{0.007367in}{0.027778in}}{\pgfqpoint{0.000000in}{0.027778in}}%
\pgfpathcurveto{\pgfqpoint{-0.007367in}{0.027778in}}{\pgfqpoint{-0.014433in}{0.024851in}}{\pgfqpoint{-0.019642in}{0.019642in}}%
\pgfpathcurveto{\pgfqpoint{-0.024851in}{0.014433in}}{\pgfqpoint{-0.027778in}{0.007367in}}{\pgfqpoint{-0.027778in}{0.000000in}}%
\pgfpathcurveto{\pgfqpoint{-0.027778in}{-0.007367in}}{\pgfqpoint{-0.024851in}{-0.014433in}}{\pgfqpoint{-0.019642in}{-0.019642in}}%
\pgfpathcurveto{\pgfqpoint{-0.014433in}{-0.024851in}}{\pgfqpoint{-0.007367in}{-0.027778in}}{\pgfqpoint{0.000000in}{-0.027778in}}%
\pgfpathclose%
\pgfusepath{stroke,fill}%
}%
\begin{pgfscope}%
\pgfsys@transformshift{1.108330in}{4.236603in}%
\pgfsys@useobject{currentmarker}{}%
\end{pgfscope}%
\end{pgfscope}%
\begin{pgfscope}%
\pgfpathrectangle{\pgfqpoint{0.100000in}{2.413063in}}{\pgfqpoint{5.037500in}{3.427208in}}%
\pgfusepath{clip}%
\pgfsetrectcap%
\pgfsetroundjoin%
\pgfsetlinewidth{1.505625pt}%
\definecolor{currentstroke}{rgb}{0.678431,1.000000,0.184314}%
\pgfsetstrokecolor{currentstroke}%
\pgfsetstrokeopacity{0.500000}%
\pgfsetdash{}{0pt}%
\pgfpathmoveto{\pgfqpoint{1.338959in}{4.620996in}}%
\pgfusepath{stroke}%
\end{pgfscope}%
\begin{pgfscope}%
\pgfpathrectangle{\pgfqpoint{0.100000in}{2.413063in}}{\pgfqpoint{5.037500in}{3.427208in}}%
\pgfusepath{clip}%
\pgfsetbuttcap%
\pgfsetroundjoin%
\definecolor{currentfill}{rgb}{0.678431,1.000000,0.184314}%
\pgfsetfillcolor{currentfill}%
\pgfsetfillopacity{0.500000}%
\pgfsetlinewidth{0.250937pt}%
\definecolor{currentstroke}{rgb}{0.000000,0.000000,0.000000}%
\pgfsetstrokecolor{currentstroke}%
\pgfsetstrokeopacity{0.500000}%
\pgfsetdash{}{0pt}%
\pgfsys@defobject{currentmarker}{\pgfqpoint{-0.022222in}{-0.022222in}}{\pgfqpoint{0.022222in}{0.022222in}}{%
\pgfpathmoveto{\pgfqpoint{0.000000in}{-0.022222in}}%
\pgfpathcurveto{\pgfqpoint{0.005893in}{-0.022222in}}{\pgfqpoint{0.011546in}{-0.019881in}}{\pgfqpoint{0.015713in}{-0.015713in}}%
\pgfpathcurveto{\pgfqpoint{0.019881in}{-0.011546in}}{\pgfqpoint{0.022222in}{-0.005893in}}{\pgfqpoint{0.022222in}{0.000000in}}%
\pgfpathcurveto{\pgfqpoint{0.022222in}{0.005893in}}{\pgfqpoint{0.019881in}{0.011546in}}{\pgfqpoint{0.015713in}{0.015713in}}%
\pgfpathcurveto{\pgfqpoint{0.011546in}{0.019881in}}{\pgfqpoint{0.005893in}{0.022222in}}{\pgfqpoint{0.000000in}{0.022222in}}%
\pgfpathcurveto{\pgfqpoint{-0.005893in}{0.022222in}}{\pgfqpoint{-0.011546in}{0.019881in}}{\pgfqpoint{-0.015713in}{0.015713in}}%
\pgfpathcurveto{\pgfqpoint{-0.019881in}{0.011546in}}{\pgfqpoint{-0.022222in}{0.005893in}}{\pgfqpoint{-0.022222in}{0.000000in}}%
\pgfpathcurveto{\pgfqpoint{-0.022222in}{-0.005893in}}{\pgfqpoint{-0.019881in}{-0.011546in}}{\pgfqpoint{-0.015713in}{-0.015713in}}%
\pgfpathcurveto{\pgfqpoint{-0.011546in}{-0.019881in}}{\pgfqpoint{-0.005893in}{-0.022222in}}{\pgfqpoint{0.000000in}{-0.022222in}}%
\pgfpathclose%
\pgfusepath{stroke,fill}%
}%
\begin{pgfscope}%
\pgfsys@transformshift{1.338959in}{4.620996in}%
\pgfsys@useobject{currentmarker}{}%
\end{pgfscope}%
\end{pgfscope}%
\begin{pgfscope}%
\pgfpathrectangle{\pgfqpoint{0.100000in}{2.413063in}}{\pgfqpoint{5.037500in}{3.427208in}}%
\pgfusepath{clip}%
\pgfsetrectcap%
\pgfsetroundjoin%
\pgfsetlinewidth{1.505625pt}%
\definecolor{currentstroke}{rgb}{0.501961,0.501961,0.501961}%
\pgfsetstrokecolor{currentstroke}%
\pgfsetstrokeopacity{0.500000}%
\pgfsetdash{}{0pt}%
\pgfpathmoveto{\pgfqpoint{4.585415in}{5.126583in}}%
\pgfusepath{stroke}%
\end{pgfscope}%
\begin{pgfscope}%
\pgfpathrectangle{\pgfqpoint{0.100000in}{2.413063in}}{\pgfqpoint{5.037500in}{3.427208in}}%
\pgfusepath{clip}%
\pgfsetbuttcap%
\pgfsetroundjoin%
\definecolor{currentfill}{rgb}{0.501961,0.501961,0.501961}%
\pgfsetfillcolor{currentfill}%
\pgfsetfillopacity{0.500000}%
\pgfsetlinewidth{0.250937pt}%
\definecolor{currentstroke}{rgb}{0.000000,0.000000,0.000000}%
\pgfsetstrokecolor{currentstroke}%
\pgfsetstrokeopacity{0.500000}%
\pgfsetdash{}{0pt}%
\pgfsys@defobject{currentmarker}{\pgfqpoint{-0.013889in}{-0.013889in}}{\pgfqpoint{0.013889in}{0.013889in}}{%
\pgfpathmoveto{\pgfqpoint{0.000000in}{-0.013889in}}%
\pgfpathcurveto{\pgfqpoint{0.003683in}{-0.013889in}}{\pgfqpoint{0.007216in}{-0.012425in}}{\pgfqpoint{0.009821in}{-0.009821in}}%
\pgfpathcurveto{\pgfqpoint{0.012425in}{-0.007216in}}{\pgfqpoint{0.013889in}{-0.003683in}}{\pgfqpoint{0.013889in}{0.000000in}}%
\pgfpathcurveto{\pgfqpoint{0.013889in}{0.003683in}}{\pgfqpoint{0.012425in}{0.007216in}}{\pgfqpoint{0.009821in}{0.009821in}}%
\pgfpathcurveto{\pgfqpoint{0.007216in}{0.012425in}}{\pgfqpoint{0.003683in}{0.013889in}}{\pgfqpoint{0.000000in}{0.013889in}}%
\pgfpathcurveto{\pgfqpoint{-0.003683in}{0.013889in}}{\pgfqpoint{-0.007216in}{0.012425in}}{\pgfqpoint{-0.009821in}{0.009821in}}%
\pgfpathcurveto{\pgfqpoint{-0.012425in}{0.007216in}}{\pgfqpoint{-0.013889in}{0.003683in}}{\pgfqpoint{-0.013889in}{0.000000in}}%
\pgfpathcurveto{\pgfqpoint{-0.013889in}{-0.003683in}}{\pgfqpoint{-0.012425in}{-0.007216in}}{\pgfqpoint{-0.009821in}{-0.009821in}}%
\pgfpathcurveto{\pgfqpoint{-0.007216in}{-0.012425in}}{\pgfqpoint{-0.003683in}{-0.013889in}}{\pgfqpoint{0.000000in}{-0.013889in}}%
\pgfpathclose%
\pgfusepath{stroke,fill}%
}%
\begin{pgfscope}%
\pgfsys@transformshift{4.585415in}{5.126583in}%
\pgfsys@useobject{currentmarker}{}%
\end{pgfscope}%
\end{pgfscope}%
\begin{pgfscope}%
\pgfpathrectangle{\pgfqpoint{0.100000in}{2.413063in}}{\pgfqpoint{5.037500in}{3.427208in}}%
\pgfusepath{clip}%
\pgfsetrectcap%
\pgfsetroundjoin%
\pgfsetlinewidth{1.505625pt}%
\definecolor{currentstroke}{rgb}{0.678431,1.000000,0.184314}%
\pgfsetstrokecolor{currentstroke}%
\pgfsetstrokeopacity{0.500000}%
\pgfsetdash{}{0pt}%
\pgfpathmoveto{\pgfqpoint{4.136391in}{4.184211in}}%
\pgfusepath{stroke}%
\end{pgfscope}%
\begin{pgfscope}%
\pgfpathrectangle{\pgfqpoint{0.100000in}{2.413063in}}{\pgfqpoint{5.037500in}{3.427208in}}%
\pgfusepath{clip}%
\pgfsetbuttcap%
\pgfsetroundjoin%
\definecolor{currentfill}{rgb}{0.678431,1.000000,0.184314}%
\pgfsetfillcolor{currentfill}%
\pgfsetfillopacity{0.500000}%
\pgfsetlinewidth{0.250937pt}%
\definecolor{currentstroke}{rgb}{0.000000,0.000000,0.000000}%
\pgfsetstrokecolor{currentstroke}%
\pgfsetstrokeopacity{0.500000}%
\pgfsetdash{}{0pt}%
\pgfsys@defobject{currentmarker}{\pgfqpoint{-0.027778in}{-0.027778in}}{\pgfqpoint{0.027778in}{0.027778in}}{%
\pgfpathmoveto{\pgfqpoint{0.000000in}{-0.027778in}}%
\pgfpathcurveto{\pgfqpoint{0.007367in}{-0.027778in}}{\pgfqpoint{0.014433in}{-0.024851in}}{\pgfqpoint{0.019642in}{-0.019642in}}%
\pgfpathcurveto{\pgfqpoint{0.024851in}{-0.014433in}}{\pgfqpoint{0.027778in}{-0.007367in}}{\pgfqpoint{0.027778in}{0.000000in}}%
\pgfpathcurveto{\pgfqpoint{0.027778in}{0.007367in}}{\pgfqpoint{0.024851in}{0.014433in}}{\pgfqpoint{0.019642in}{0.019642in}}%
\pgfpathcurveto{\pgfqpoint{0.014433in}{0.024851in}}{\pgfqpoint{0.007367in}{0.027778in}}{\pgfqpoint{0.000000in}{0.027778in}}%
\pgfpathcurveto{\pgfqpoint{-0.007367in}{0.027778in}}{\pgfqpoint{-0.014433in}{0.024851in}}{\pgfqpoint{-0.019642in}{0.019642in}}%
\pgfpathcurveto{\pgfqpoint{-0.024851in}{0.014433in}}{\pgfqpoint{-0.027778in}{0.007367in}}{\pgfqpoint{-0.027778in}{0.000000in}}%
\pgfpathcurveto{\pgfqpoint{-0.027778in}{-0.007367in}}{\pgfqpoint{-0.024851in}{-0.014433in}}{\pgfqpoint{-0.019642in}{-0.019642in}}%
\pgfpathcurveto{\pgfqpoint{-0.014433in}{-0.024851in}}{\pgfqpoint{-0.007367in}{-0.027778in}}{\pgfqpoint{0.000000in}{-0.027778in}}%
\pgfpathclose%
\pgfusepath{stroke,fill}%
}%
\begin{pgfscope}%
\pgfsys@transformshift{4.136391in}{4.184211in}%
\pgfsys@useobject{currentmarker}{}%
\end{pgfscope}%
\end{pgfscope}%
\begin{pgfscope}%
\pgfpathrectangle{\pgfqpoint{0.100000in}{2.413063in}}{\pgfqpoint{5.037500in}{3.427208in}}%
\pgfusepath{clip}%
\pgfsetrectcap%
\pgfsetroundjoin%
\pgfsetlinewidth{1.505625pt}%
\definecolor{currentstroke}{rgb}{0.000000,0.000000,1.000000}%
\pgfsetstrokecolor{currentstroke}%
\pgfsetstrokeopacity{0.500000}%
\pgfsetdash{}{0pt}%
\pgfpathmoveto{\pgfqpoint{4.294817in}{4.305064in}}%
\pgfusepath{stroke}%
\end{pgfscope}%
\begin{pgfscope}%
\pgfpathrectangle{\pgfqpoint{0.100000in}{2.413063in}}{\pgfqpoint{5.037500in}{3.427208in}}%
\pgfusepath{clip}%
\pgfsetbuttcap%
\pgfsetroundjoin%
\definecolor{currentfill}{rgb}{0.000000,0.000000,1.000000}%
\pgfsetfillcolor{currentfill}%
\pgfsetfillopacity{0.500000}%
\pgfsetlinewidth{0.250937pt}%
\definecolor{currentstroke}{rgb}{0.000000,0.000000,0.000000}%
\pgfsetstrokecolor{currentstroke}%
\pgfsetstrokeopacity{0.500000}%
\pgfsetdash{}{0pt}%
\pgfsys@defobject{currentmarker}{\pgfqpoint{-0.011111in}{-0.011111in}}{\pgfqpoint{0.011111in}{0.011111in}}{%
\pgfpathmoveto{\pgfqpoint{0.000000in}{-0.011111in}}%
\pgfpathcurveto{\pgfqpoint{0.002947in}{-0.011111in}}{\pgfqpoint{0.005773in}{-0.009940in}}{\pgfqpoint{0.007857in}{-0.007857in}}%
\pgfpathcurveto{\pgfqpoint{0.009940in}{-0.005773in}}{\pgfqpoint{0.011111in}{-0.002947in}}{\pgfqpoint{0.011111in}{0.000000in}}%
\pgfpathcurveto{\pgfqpoint{0.011111in}{0.002947in}}{\pgfqpoint{0.009940in}{0.005773in}}{\pgfqpoint{0.007857in}{0.007857in}}%
\pgfpathcurveto{\pgfqpoint{0.005773in}{0.009940in}}{\pgfqpoint{0.002947in}{0.011111in}}{\pgfqpoint{0.000000in}{0.011111in}}%
\pgfpathcurveto{\pgfqpoint{-0.002947in}{0.011111in}}{\pgfqpoint{-0.005773in}{0.009940in}}{\pgfqpoint{-0.007857in}{0.007857in}}%
\pgfpathcurveto{\pgfqpoint{-0.009940in}{0.005773in}}{\pgfqpoint{-0.011111in}{0.002947in}}{\pgfqpoint{-0.011111in}{0.000000in}}%
\pgfpathcurveto{\pgfqpoint{-0.011111in}{-0.002947in}}{\pgfqpoint{-0.009940in}{-0.005773in}}{\pgfqpoint{-0.007857in}{-0.007857in}}%
\pgfpathcurveto{\pgfqpoint{-0.005773in}{-0.009940in}}{\pgfqpoint{-0.002947in}{-0.011111in}}{\pgfqpoint{0.000000in}{-0.011111in}}%
\pgfpathclose%
\pgfusepath{stroke,fill}%
}%
\begin{pgfscope}%
\pgfsys@transformshift{4.294817in}{4.305064in}%
\pgfsys@useobject{currentmarker}{}%
\end{pgfscope}%
\end{pgfscope}%
\begin{pgfscope}%
\pgfpathrectangle{\pgfqpoint{0.100000in}{2.413063in}}{\pgfqpoint{5.037500in}{3.427208in}}%
\pgfusepath{clip}%
\pgfsetrectcap%
\pgfsetroundjoin%
\pgfsetlinewidth{1.505625pt}%
\definecolor{currentstroke}{rgb}{0.000000,0.000000,1.000000}%
\pgfsetstrokecolor{currentstroke}%
\pgfsetstrokeopacity{0.500000}%
\pgfsetdash{}{0pt}%
\pgfpathmoveto{\pgfqpoint{4.251269in}{4.346316in}}%
\pgfusepath{stroke}%
\end{pgfscope}%
\begin{pgfscope}%
\pgfpathrectangle{\pgfqpoint{0.100000in}{2.413063in}}{\pgfqpoint{5.037500in}{3.427208in}}%
\pgfusepath{clip}%
\pgfsetbuttcap%
\pgfsetroundjoin%
\definecolor{currentfill}{rgb}{0.000000,0.000000,1.000000}%
\pgfsetfillcolor{currentfill}%
\pgfsetfillopacity{0.500000}%
\pgfsetlinewidth{0.250937pt}%
\definecolor{currentstroke}{rgb}{0.000000,0.000000,0.000000}%
\pgfsetstrokecolor{currentstroke}%
\pgfsetstrokeopacity{0.500000}%
\pgfsetdash{}{0pt}%
\pgfsys@defobject{currentmarker}{\pgfqpoint{-0.008333in}{-0.008333in}}{\pgfqpoint{0.008333in}{0.008333in}}{%
\pgfpathmoveto{\pgfqpoint{0.000000in}{-0.008333in}}%
\pgfpathcurveto{\pgfqpoint{0.002210in}{-0.008333in}}{\pgfqpoint{0.004330in}{-0.007455in}}{\pgfqpoint{0.005893in}{-0.005893in}}%
\pgfpathcurveto{\pgfqpoint{0.007455in}{-0.004330in}}{\pgfqpoint{0.008333in}{-0.002210in}}{\pgfqpoint{0.008333in}{0.000000in}}%
\pgfpathcurveto{\pgfqpoint{0.008333in}{0.002210in}}{\pgfqpoint{0.007455in}{0.004330in}}{\pgfqpoint{0.005893in}{0.005893in}}%
\pgfpathcurveto{\pgfqpoint{0.004330in}{0.007455in}}{\pgfqpoint{0.002210in}{0.008333in}}{\pgfqpoint{0.000000in}{0.008333in}}%
\pgfpathcurveto{\pgfqpoint{-0.002210in}{0.008333in}}{\pgfqpoint{-0.004330in}{0.007455in}}{\pgfqpoint{-0.005893in}{0.005893in}}%
\pgfpathcurveto{\pgfqpoint{-0.007455in}{0.004330in}}{\pgfqpoint{-0.008333in}{0.002210in}}{\pgfqpoint{-0.008333in}{0.000000in}}%
\pgfpathcurveto{\pgfqpoint{-0.008333in}{-0.002210in}}{\pgfqpoint{-0.007455in}{-0.004330in}}{\pgfqpoint{-0.005893in}{-0.005893in}}%
\pgfpathcurveto{\pgfqpoint{-0.004330in}{-0.007455in}}{\pgfqpoint{-0.002210in}{-0.008333in}}{\pgfqpoint{0.000000in}{-0.008333in}}%
\pgfpathclose%
\pgfusepath{stroke,fill}%
}%
\begin{pgfscope}%
\pgfsys@transformshift{4.251269in}{4.346316in}%
\pgfsys@useobject{currentmarker}{}%
\end{pgfscope}%
\end{pgfscope}%
\begin{pgfscope}%
\pgfpathrectangle{\pgfqpoint{0.100000in}{2.413063in}}{\pgfqpoint{5.037500in}{3.427208in}}%
\pgfusepath{clip}%
\pgfsetrectcap%
\pgfsetroundjoin%
\pgfsetlinewidth{1.505625pt}%
\definecolor{currentstroke}{rgb}{0.501961,0.501961,0.501961}%
\pgfsetstrokecolor{currentstroke}%
\pgfsetstrokeopacity{0.500000}%
\pgfsetdash{}{0pt}%
\pgfpathmoveto{\pgfqpoint{4.244154in}{4.222410in}}%
\pgfusepath{stroke}%
\end{pgfscope}%
\begin{pgfscope}%
\pgfpathrectangle{\pgfqpoint{0.100000in}{2.413063in}}{\pgfqpoint{5.037500in}{3.427208in}}%
\pgfusepath{clip}%
\pgfsetbuttcap%
\pgfsetroundjoin%
\definecolor{currentfill}{rgb}{0.501961,0.501961,0.501961}%
\pgfsetfillcolor{currentfill}%
\pgfsetfillopacity{0.500000}%
\pgfsetlinewidth{0.250937pt}%
\definecolor{currentstroke}{rgb}{0.000000,0.000000,0.000000}%
\pgfsetstrokecolor{currentstroke}%
\pgfsetstrokeopacity{0.500000}%
\pgfsetdash{}{0pt}%
\pgfsys@defobject{currentmarker}{\pgfqpoint{-0.013889in}{-0.013889in}}{\pgfqpoint{0.013889in}{0.013889in}}{%
\pgfpathmoveto{\pgfqpoint{0.000000in}{-0.013889in}}%
\pgfpathcurveto{\pgfqpoint{0.003683in}{-0.013889in}}{\pgfqpoint{0.007216in}{-0.012425in}}{\pgfqpoint{0.009821in}{-0.009821in}}%
\pgfpathcurveto{\pgfqpoint{0.012425in}{-0.007216in}}{\pgfqpoint{0.013889in}{-0.003683in}}{\pgfqpoint{0.013889in}{0.000000in}}%
\pgfpathcurveto{\pgfqpoint{0.013889in}{0.003683in}}{\pgfqpoint{0.012425in}{0.007216in}}{\pgfqpoint{0.009821in}{0.009821in}}%
\pgfpathcurveto{\pgfqpoint{0.007216in}{0.012425in}}{\pgfqpoint{0.003683in}{0.013889in}}{\pgfqpoint{0.000000in}{0.013889in}}%
\pgfpathcurveto{\pgfqpoint{-0.003683in}{0.013889in}}{\pgfqpoint{-0.007216in}{0.012425in}}{\pgfqpoint{-0.009821in}{0.009821in}}%
\pgfpathcurveto{\pgfqpoint{-0.012425in}{0.007216in}}{\pgfqpoint{-0.013889in}{0.003683in}}{\pgfqpoint{-0.013889in}{0.000000in}}%
\pgfpathcurveto{\pgfqpoint{-0.013889in}{-0.003683in}}{\pgfqpoint{-0.012425in}{-0.007216in}}{\pgfqpoint{-0.009821in}{-0.009821in}}%
\pgfpathcurveto{\pgfqpoint{-0.007216in}{-0.012425in}}{\pgfqpoint{-0.003683in}{-0.013889in}}{\pgfqpoint{0.000000in}{-0.013889in}}%
\pgfpathclose%
\pgfusepath{stroke,fill}%
}%
\begin{pgfscope}%
\pgfsys@transformshift{4.244154in}{4.222410in}%
\pgfsys@useobject{currentmarker}{}%
\end{pgfscope}%
\end{pgfscope}%
\begin{pgfscope}%
\pgfpathrectangle{\pgfqpoint{0.100000in}{2.413063in}}{\pgfqpoint{5.037500in}{3.427208in}}%
\pgfusepath{clip}%
\pgfsetrectcap%
\pgfsetroundjoin%
\pgfsetlinewidth{1.505625pt}%
\definecolor{currentstroke}{rgb}{0.000000,0.000000,1.000000}%
\pgfsetstrokecolor{currentstroke}%
\pgfsetstrokeopacity{0.500000}%
\pgfsetdash{}{0pt}%
\pgfpathmoveto{\pgfqpoint{4.398579in}{4.267220in}}%
\pgfusepath{stroke}%
\end{pgfscope}%
\begin{pgfscope}%
\pgfpathrectangle{\pgfqpoint{0.100000in}{2.413063in}}{\pgfqpoint{5.037500in}{3.427208in}}%
\pgfusepath{clip}%
\pgfsetbuttcap%
\pgfsetroundjoin%
\definecolor{currentfill}{rgb}{0.000000,0.000000,1.000000}%
\pgfsetfillcolor{currentfill}%
\pgfsetfillopacity{0.500000}%
\pgfsetlinewidth{0.250937pt}%
\definecolor{currentstroke}{rgb}{0.000000,0.000000,0.000000}%
\pgfsetstrokecolor{currentstroke}%
\pgfsetstrokeopacity{0.500000}%
\pgfsetdash{}{0pt}%
\pgfsys@defobject{currentmarker}{\pgfqpoint{-0.022222in}{-0.022222in}}{\pgfqpoint{0.022222in}{0.022222in}}{%
\pgfpathmoveto{\pgfqpoint{0.000000in}{-0.022222in}}%
\pgfpathcurveto{\pgfqpoint{0.005893in}{-0.022222in}}{\pgfqpoint{0.011546in}{-0.019881in}}{\pgfqpoint{0.015713in}{-0.015713in}}%
\pgfpathcurveto{\pgfqpoint{0.019881in}{-0.011546in}}{\pgfqpoint{0.022222in}{-0.005893in}}{\pgfqpoint{0.022222in}{0.000000in}}%
\pgfpathcurveto{\pgfqpoint{0.022222in}{0.005893in}}{\pgfqpoint{0.019881in}{0.011546in}}{\pgfqpoint{0.015713in}{0.015713in}}%
\pgfpathcurveto{\pgfqpoint{0.011546in}{0.019881in}}{\pgfqpoint{0.005893in}{0.022222in}}{\pgfqpoint{0.000000in}{0.022222in}}%
\pgfpathcurveto{\pgfqpoint{-0.005893in}{0.022222in}}{\pgfqpoint{-0.011546in}{0.019881in}}{\pgfqpoint{-0.015713in}{0.015713in}}%
\pgfpathcurveto{\pgfqpoint{-0.019881in}{0.011546in}}{\pgfqpoint{-0.022222in}{0.005893in}}{\pgfqpoint{-0.022222in}{0.000000in}}%
\pgfpathcurveto{\pgfqpoint{-0.022222in}{-0.005893in}}{\pgfqpoint{-0.019881in}{-0.011546in}}{\pgfqpoint{-0.015713in}{-0.015713in}}%
\pgfpathcurveto{\pgfqpoint{-0.011546in}{-0.019881in}}{\pgfqpoint{-0.005893in}{-0.022222in}}{\pgfqpoint{0.000000in}{-0.022222in}}%
\pgfpathclose%
\pgfusepath{stroke,fill}%
}%
\begin{pgfscope}%
\pgfsys@transformshift{4.398579in}{4.267220in}%
\pgfsys@useobject{currentmarker}{}%
\end{pgfscope}%
\end{pgfscope}%
\begin{pgfscope}%
\pgfpathrectangle{\pgfqpoint{0.100000in}{2.413063in}}{\pgfqpoint{5.037500in}{3.427208in}}%
\pgfusepath{clip}%
\pgfsetrectcap%
\pgfsetroundjoin%
\pgfsetlinewidth{1.505625pt}%
\definecolor{currentstroke}{rgb}{0.000000,0.000000,1.000000}%
\pgfsetstrokecolor{currentstroke}%
\pgfsetstrokeopacity{0.500000}%
\pgfsetdash{}{0pt}%
\pgfpathmoveto{\pgfqpoint{4.178350in}{4.195940in}}%
\pgfusepath{stroke}%
\end{pgfscope}%
\begin{pgfscope}%
\pgfpathrectangle{\pgfqpoint{0.100000in}{2.413063in}}{\pgfqpoint{5.037500in}{3.427208in}}%
\pgfusepath{clip}%
\pgfsetbuttcap%
\pgfsetroundjoin%
\definecolor{currentfill}{rgb}{0.000000,0.000000,1.000000}%
\pgfsetfillcolor{currentfill}%
\pgfsetfillopacity{0.500000}%
\pgfsetlinewidth{0.250937pt}%
\definecolor{currentstroke}{rgb}{0.000000,0.000000,0.000000}%
\pgfsetstrokecolor{currentstroke}%
\pgfsetstrokeopacity{0.500000}%
\pgfsetdash{}{0pt}%
\pgfsys@defobject{currentmarker}{\pgfqpoint{-0.005556in}{-0.005556in}}{\pgfqpoint{0.005556in}{0.005556in}}{%
\pgfpathmoveto{\pgfqpoint{0.000000in}{-0.005556in}}%
\pgfpathcurveto{\pgfqpoint{0.001473in}{-0.005556in}}{\pgfqpoint{0.002887in}{-0.004970in}}{\pgfqpoint{0.003928in}{-0.003928in}}%
\pgfpathcurveto{\pgfqpoint{0.004970in}{-0.002887in}}{\pgfqpoint{0.005556in}{-0.001473in}}{\pgfqpoint{0.005556in}{0.000000in}}%
\pgfpathcurveto{\pgfqpoint{0.005556in}{0.001473in}}{\pgfqpoint{0.004970in}{0.002887in}}{\pgfqpoint{0.003928in}{0.003928in}}%
\pgfpathcurveto{\pgfqpoint{0.002887in}{0.004970in}}{\pgfqpoint{0.001473in}{0.005556in}}{\pgfqpoint{0.000000in}{0.005556in}}%
\pgfpathcurveto{\pgfqpoint{-0.001473in}{0.005556in}}{\pgfqpoint{-0.002887in}{0.004970in}}{\pgfqpoint{-0.003928in}{0.003928in}}%
\pgfpathcurveto{\pgfqpoint{-0.004970in}{0.002887in}}{\pgfqpoint{-0.005556in}{0.001473in}}{\pgfqpoint{-0.005556in}{0.000000in}}%
\pgfpathcurveto{\pgfqpoint{-0.005556in}{-0.001473in}}{\pgfqpoint{-0.004970in}{-0.002887in}}{\pgfqpoint{-0.003928in}{-0.003928in}}%
\pgfpathcurveto{\pgfqpoint{-0.002887in}{-0.004970in}}{\pgfqpoint{-0.001473in}{-0.005556in}}{\pgfqpoint{0.000000in}{-0.005556in}}%
\pgfpathclose%
\pgfusepath{stroke,fill}%
}%
\begin{pgfscope}%
\pgfsys@transformshift{4.178350in}{4.195940in}%
\pgfsys@useobject{currentmarker}{}%
\end{pgfscope}%
\end{pgfscope}%
\begin{pgfscope}%
\pgfpathrectangle{\pgfqpoint{0.100000in}{2.413063in}}{\pgfqpoint{5.037500in}{3.427208in}}%
\pgfusepath{clip}%
\pgfsetrectcap%
\pgfsetroundjoin%
\pgfsetlinewidth{1.505625pt}%
\definecolor{currentstroke}{rgb}{0.000000,0.000000,1.000000}%
\pgfsetstrokecolor{currentstroke}%
\pgfsetstrokeopacity{0.500000}%
\pgfsetdash{}{0pt}%
\pgfpathmoveto{\pgfqpoint{4.239205in}{4.309104in}}%
\pgfusepath{stroke}%
\end{pgfscope}%
\begin{pgfscope}%
\pgfpathrectangle{\pgfqpoint{0.100000in}{2.413063in}}{\pgfqpoint{5.037500in}{3.427208in}}%
\pgfusepath{clip}%
\pgfsetbuttcap%
\pgfsetroundjoin%
\definecolor{currentfill}{rgb}{0.000000,0.000000,1.000000}%
\pgfsetfillcolor{currentfill}%
\pgfsetfillopacity{0.500000}%
\pgfsetlinewidth{0.250937pt}%
\definecolor{currentstroke}{rgb}{0.000000,0.000000,0.000000}%
\pgfsetstrokecolor{currentstroke}%
\pgfsetstrokeopacity{0.500000}%
\pgfsetdash{}{0pt}%
\pgfsys@defobject{currentmarker}{\pgfqpoint{-0.008333in}{-0.008333in}}{\pgfqpoint{0.008333in}{0.008333in}}{%
\pgfpathmoveto{\pgfqpoint{0.000000in}{-0.008333in}}%
\pgfpathcurveto{\pgfqpoint{0.002210in}{-0.008333in}}{\pgfqpoint{0.004330in}{-0.007455in}}{\pgfqpoint{0.005893in}{-0.005893in}}%
\pgfpathcurveto{\pgfqpoint{0.007455in}{-0.004330in}}{\pgfqpoint{0.008333in}{-0.002210in}}{\pgfqpoint{0.008333in}{0.000000in}}%
\pgfpathcurveto{\pgfqpoint{0.008333in}{0.002210in}}{\pgfqpoint{0.007455in}{0.004330in}}{\pgfqpoint{0.005893in}{0.005893in}}%
\pgfpathcurveto{\pgfqpoint{0.004330in}{0.007455in}}{\pgfqpoint{0.002210in}{0.008333in}}{\pgfqpoint{0.000000in}{0.008333in}}%
\pgfpathcurveto{\pgfqpoint{-0.002210in}{0.008333in}}{\pgfqpoint{-0.004330in}{0.007455in}}{\pgfqpoint{-0.005893in}{0.005893in}}%
\pgfpathcurveto{\pgfqpoint{-0.007455in}{0.004330in}}{\pgfqpoint{-0.008333in}{0.002210in}}{\pgfqpoint{-0.008333in}{0.000000in}}%
\pgfpathcurveto{\pgfqpoint{-0.008333in}{-0.002210in}}{\pgfqpoint{-0.007455in}{-0.004330in}}{\pgfqpoint{-0.005893in}{-0.005893in}}%
\pgfpathcurveto{\pgfqpoint{-0.004330in}{-0.007455in}}{\pgfqpoint{-0.002210in}{-0.008333in}}{\pgfqpoint{0.000000in}{-0.008333in}}%
\pgfpathclose%
\pgfusepath{stroke,fill}%
}%
\begin{pgfscope}%
\pgfsys@transformshift{4.239205in}{4.309104in}%
\pgfsys@useobject{currentmarker}{}%
\end{pgfscope}%
\end{pgfscope}%
\begin{pgfscope}%
\pgfpathrectangle{\pgfqpoint{0.100000in}{2.413063in}}{\pgfqpoint{5.037500in}{3.427208in}}%
\pgfusepath{clip}%
\pgfsetrectcap%
\pgfsetroundjoin%
\pgfsetlinewidth{1.505625pt}%
\definecolor{currentstroke}{rgb}{0.000000,0.000000,1.000000}%
\pgfsetstrokecolor{currentstroke}%
\pgfsetstrokeopacity{0.500000}%
\pgfsetdash{}{0pt}%
\pgfpathmoveto{\pgfqpoint{4.545369in}{4.216571in}}%
\pgfusepath{stroke}%
\end{pgfscope}%
\begin{pgfscope}%
\pgfpathrectangle{\pgfqpoint{0.100000in}{2.413063in}}{\pgfqpoint{5.037500in}{3.427208in}}%
\pgfusepath{clip}%
\pgfsetbuttcap%
\pgfsetroundjoin%
\definecolor{currentfill}{rgb}{0.000000,0.000000,1.000000}%
\pgfsetfillcolor{currentfill}%
\pgfsetfillopacity{0.500000}%
\pgfsetlinewidth{0.250937pt}%
\definecolor{currentstroke}{rgb}{0.000000,0.000000,0.000000}%
\pgfsetstrokecolor{currentstroke}%
\pgfsetstrokeopacity{0.500000}%
\pgfsetdash{}{0pt}%
\pgfsys@defobject{currentmarker}{\pgfqpoint{-0.016667in}{-0.016667in}}{\pgfqpoint{0.016667in}{0.016667in}}{%
\pgfpathmoveto{\pgfqpoint{0.000000in}{-0.016667in}}%
\pgfpathcurveto{\pgfqpoint{0.004420in}{-0.016667in}}{\pgfqpoint{0.008660in}{-0.014911in}}{\pgfqpoint{0.011785in}{-0.011785in}}%
\pgfpathcurveto{\pgfqpoint{0.014911in}{-0.008660in}}{\pgfqpoint{0.016667in}{-0.004420in}}{\pgfqpoint{0.016667in}{0.000000in}}%
\pgfpathcurveto{\pgfqpoint{0.016667in}{0.004420in}}{\pgfqpoint{0.014911in}{0.008660in}}{\pgfqpoint{0.011785in}{0.011785in}}%
\pgfpathcurveto{\pgfqpoint{0.008660in}{0.014911in}}{\pgfqpoint{0.004420in}{0.016667in}}{\pgfqpoint{0.000000in}{0.016667in}}%
\pgfpathcurveto{\pgfqpoint{-0.004420in}{0.016667in}}{\pgfqpoint{-0.008660in}{0.014911in}}{\pgfqpoint{-0.011785in}{0.011785in}}%
\pgfpathcurveto{\pgfqpoint{-0.014911in}{0.008660in}}{\pgfqpoint{-0.016667in}{0.004420in}}{\pgfqpoint{-0.016667in}{0.000000in}}%
\pgfpathcurveto{\pgfqpoint{-0.016667in}{-0.004420in}}{\pgfqpoint{-0.014911in}{-0.008660in}}{\pgfqpoint{-0.011785in}{-0.011785in}}%
\pgfpathcurveto{\pgfqpoint{-0.008660in}{-0.014911in}}{\pgfqpoint{-0.004420in}{-0.016667in}}{\pgfqpoint{0.000000in}{-0.016667in}}%
\pgfpathclose%
\pgfusepath{stroke,fill}%
}%
\begin{pgfscope}%
\pgfsys@transformshift{4.545369in}{4.216571in}%
\pgfsys@useobject{currentmarker}{}%
\end{pgfscope}%
\end{pgfscope}%
\begin{pgfscope}%
\pgfpathrectangle{\pgfqpoint{0.100000in}{2.413063in}}{\pgfqpoint{5.037500in}{3.427208in}}%
\pgfusepath{clip}%
\pgfsetrectcap%
\pgfsetroundjoin%
\pgfsetlinewidth{1.505625pt}%
\definecolor{currentstroke}{rgb}{0.000000,0.000000,1.000000}%
\pgfsetstrokecolor{currentstroke}%
\pgfsetstrokeopacity{0.500000}%
\pgfsetdash{}{0pt}%
\pgfpathmoveto{\pgfqpoint{4.298077in}{4.441049in}}%
\pgfusepath{stroke}%
\end{pgfscope}%
\begin{pgfscope}%
\pgfpathrectangle{\pgfqpoint{0.100000in}{2.413063in}}{\pgfqpoint{5.037500in}{3.427208in}}%
\pgfusepath{clip}%
\pgfsetbuttcap%
\pgfsetroundjoin%
\definecolor{currentfill}{rgb}{0.000000,0.000000,1.000000}%
\pgfsetfillcolor{currentfill}%
\pgfsetfillopacity{0.500000}%
\pgfsetlinewidth{0.250937pt}%
\definecolor{currentstroke}{rgb}{0.000000,0.000000,0.000000}%
\pgfsetstrokecolor{currentstroke}%
\pgfsetstrokeopacity{0.500000}%
\pgfsetdash{}{0pt}%
\pgfsys@defobject{currentmarker}{\pgfqpoint{-0.008333in}{-0.008333in}}{\pgfqpoint{0.008333in}{0.008333in}}{%
\pgfpathmoveto{\pgfqpoint{0.000000in}{-0.008333in}}%
\pgfpathcurveto{\pgfqpoint{0.002210in}{-0.008333in}}{\pgfqpoint{0.004330in}{-0.007455in}}{\pgfqpoint{0.005893in}{-0.005893in}}%
\pgfpathcurveto{\pgfqpoint{0.007455in}{-0.004330in}}{\pgfqpoint{0.008333in}{-0.002210in}}{\pgfqpoint{0.008333in}{0.000000in}}%
\pgfpathcurveto{\pgfqpoint{0.008333in}{0.002210in}}{\pgfqpoint{0.007455in}{0.004330in}}{\pgfqpoint{0.005893in}{0.005893in}}%
\pgfpathcurveto{\pgfqpoint{0.004330in}{0.007455in}}{\pgfqpoint{0.002210in}{0.008333in}}{\pgfqpoint{0.000000in}{0.008333in}}%
\pgfpathcurveto{\pgfqpoint{-0.002210in}{0.008333in}}{\pgfqpoint{-0.004330in}{0.007455in}}{\pgfqpoint{-0.005893in}{0.005893in}}%
\pgfpathcurveto{\pgfqpoint{-0.007455in}{0.004330in}}{\pgfqpoint{-0.008333in}{0.002210in}}{\pgfqpoint{-0.008333in}{0.000000in}}%
\pgfpathcurveto{\pgfqpoint{-0.008333in}{-0.002210in}}{\pgfqpoint{-0.007455in}{-0.004330in}}{\pgfqpoint{-0.005893in}{-0.005893in}}%
\pgfpathcurveto{\pgfqpoint{-0.004330in}{-0.007455in}}{\pgfqpoint{-0.002210in}{-0.008333in}}{\pgfqpoint{0.000000in}{-0.008333in}}%
\pgfpathclose%
\pgfusepath{stroke,fill}%
}%
\begin{pgfscope}%
\pgfsys@transformshift{4.298077in}{4.441049in}%
\pgfsys@useobject{currentmarker}{}%
\end{pgfscope}%
\end{pgfscope}%
\begin{pgfscope}%
\pgfpathrectangle{\pgfqpoint{0.100000in}{2.413063in}}{\pgfqpoint{5.037500in}{3.427208in}}%
\pgfusepath{clip}%
\pgfsetrectcap%
\pgfsetroundjoin%
\pgfsetlinewidth{1.505625pt}%
\definecolor{currentstroke}{rgb}{0.678431,1.000000,0.184314}%
\pgfsetstrokecolor{currentstroke}%
\pgfsetstrokeopacity{0.500000}%
\pgfsetdash{}{0pt}%
\pgfpathmoveto{\pgfqpoint{0.718567in}{5.727400in}}%
\pgfusepath{stroke}%
\end{pgfscope}%
\begin{pgfscope}%
\pgfpathrectangle{\pgfqpoint{0.100000in}{2.413063in}}{\pgfqpoint{5.037500in}{3.427208in}}%
\pgfusepath{clip}%
\pgfsetbuttcap%
\pgfsetroundjoin%
\definecolor{currentfill}{rgb}{0.678431,1.000000,0.184314}%
\pgfsetfillcolor{currentfill}%
\pgfsetfillopacity{0.500000}%
\pgfsetlinewidth{0.250937pt}%
\definecolor{currentstroke}{rgb}{0.000000,0.000000,0.000000}%
\pgfsetstrokecolor{currentstroke}%
\pgfsetstrokeopacity{0.500000}%
\pgfsetdash{}{0pt}%
\pgfsys@defobject{currentmarker}{\pgfqpoint{-0.016667in}{-0.016667in}}{\pgfqpoint{0.016667in}{0.016667in}}{%
\pgfpathmoveto{\pgfqpoint{0.000000in}{-0.016667in}}%
\pgfpathcurveto{\pgfqpoint{0.004420in}{-0.016667in}}{\pgfqpoint{0.008660in}{-0.014911in}}{\pgfqpoint{0.011785in}{-0.011785in}}%
\pgfpathcurveto{\pgfqpoint{0.014911in}{-0.008660in}}{\pgfqpoint{0.016667in}{-0.004420in}}{\pgfqpoint{0.016667in}{0.000000in}}%
\pgfpathcurveto{\pgfqpoint{0.016667in}{0.004420in}}{\pgfqpoint{0.014911in}{0.008660in}}{\pgfqpoint{0.011785in}{0.011785in}}%
\pgfpathcurveto{\pgfqpoint{0.008660in}{0.014911in}}{\pgfqpoint{0.004420in}{0.016667in}}{\pgfqpoint{0.000000in}{0.016667in}}%
\pgfpathcurveto{\pgfqpoint{-0.004420in}{0.016667in}}{\pgfqpoint{-0.008660in}{0.014911in}}{\pgfqpoint{-0.011785in}{0.011785in}}%
\pgfpathcurveto{\pgfqpoint{-0.014911in}{0.008660in}}{\pgfqpoint{-0.016667in}{0.004420in}}{\pgfqpoint{-0.016667in}{0.000000in}}%
\pgfpathcurveto{\pgfqpoint{-0.016667in}{-0.004420in}}{\pgfqpoint{-0.014911in}{-0.008660in}}{\pgfqpoint{-0.011785in}{-0.011785in}}%
\pgfpathcurveto{\pgfqpoint{-0.008660in}{-0.014911in}}{\pgfqpoint{-0.004420in}{-0.016667in}}{\pgfqpoint{0.000000in}{-0.016667in}}%
\pgfpathclose%
\pgfusepath{stroke,fill}%
}%
\begin{pgfscope}%
\pgfsys@transformshift{0.718567in}{5.727400in}%
\pgfsys@useobject{currentmarker}{}%
\end{pgfscope}%
\end{pgfscope}%
\begin{pgfscope}%
\pgfpathrectangle{\pgfqpoint{0.100000in}{2.413063in}}{\pgfqpoint{5.037500in}{3.427208in}}%
\pgfusepath{clip}%
\pgfsetrectcap%
\pgfsetroundjoin%
\pgfsetlinewidth{1.505625pt}%
\definecolor{currentstroke}{rgb}{0.678431,1.000000,0.184314}%
\pgfsetstrokecolor{currentstroke}%
\pgfsetstrokeopacity{0.500000}%
\pgfsetdash{}{0pt}%
\pgfpathmoveto{\pgfqpoint{0.655467in}{5.597815in}}%
\pgfusepath{stroke}%
\end{pgfscope}%
\begin{pgfscope}%
\pgfpathrectangle{\pgfqpoint{0.100000in}{2.413063in}}{\pgfqpoint{5.037500in}{3.427208in}}%
\pgfusepath{clip}%
\pgfsetbuttcap%
\pgfsetroundjoin%
\definecolor{currentfill}{rgb}{0.678431,1.000000,0.184314}%
\pgfsetfillcolor{currentfill}%
\pgfsetfillopacity{0.500000}%
\pgfsetlinewidth{0.250937pt}%
\definecolor{currentstroke}{rgb}{0.000000,0.000000,0.000000}%
\pgfsetstrokecolor{currentstroke}%
\pgfsetstrokeopacity{0.500000}%
\pgfsetdash{}{0pt}%
\pgfsys@defobject{currentmarker}{\pgfqpoint{-0.016667in}{-0.016667in}}{\pgfqpoint{0.016667in}{0.016667in}}{%
\pgfpathmoveto{\pgfqpoint{0.000000in}{-0.016667in}}%
\pgfpathcurveto{\pgfqpoint{0.004420in}{-0.016667in}}{\pgfqpoint{0.008660in}{-0.014911in}}{\pgfqpoint{0.011785in}{-0.011785in}}%
\pgfpathcurveto{\pgfqpoint{0.014911in}{-0.008660in}}{\pgfqpoint{0.016667in}{-0.004420in}}{\pgfqpoint{0.016667in}{0.000000in}}%
\pgfpathcurveto{\pgfqpoint{0.016667in}{0.004420in}}{\pgfqpoint{0.014911in}{0.008660in}}{\pgfqpoint{0.011785in}{0.011785in}}%
\pgfpathcurveto{\pgfqpoint{0.008660in}{0.014911in}}{\pgfqpoint{0.004420in}{0.016667in}}{\pgfqpoint{0.000000in}{0.016667in}}%
\pgfpathcurveto{\pgfqpoint{-0.004420in}{0.016667in}}{\pgfqpoint{-0.008660in}{0.014911in}}{\pgfqpoint{-0.011785in}{0.011785in}}%
\pgfpathcurveto{\pgfqpoint{-0.014911in}{0.008660in}}{\pgfqpoint{-0.016667in}{0.004420in}}{\pgfqpoint{-0.016667in}{0.000000in}}%
\pgfpathcurveto{\pgfqpoint{-0.016667in}{-0.004420in}}{\pgfqpoint{-0.014911in}{-0.008660in}}{\pgfqpoint{-0.011785in}{-0.011785in}}%
\pgfpathcurveto{\pgfqpoint{-0.008660in}{-0.014911in}}{\pgfqpoint{-0.004420in}{-0.016667in}}{\pgfqpoint{0.000000in}{-0.016667in}}%
\pgfpathclose%
\pgfusepath{stroke,fill}%
}%
\begin{pgfscope}%
\pgfsys@transformshift{0.655467in}{5.597815in}%
\pgfsys@useobject{currentmarker}{}%
\end{pgfscope}%
\end{pgfscope}%
\begin{pgfscope}%
\pgfpathrectangle{\pgfqpoint{0.100000in}{2.413063in}}{\pgfqpoint{5.037500in}{3.427208in}}%
\pgfusepath{clip}%
\pgfsetrectcap%
\pgfsetroundjoin%
\pgfsetlinewidth{1.505625pt}%
\definecolor{currentstroke}{rgb}{0.678431,1.000000,0.184314}%
\pgfsetstrokecolor{currentstroke}%
\pgfsetstrokeopacity{0.500000}%
\pgfsetdash{}{0pt}%
\pgfpathmoveto{\pgfqpoint{0.888019in}{5.369454in}}%
\pgfusepath{stroke}%
\end{pgfscope}%
\begin{pgfscope}%
\pgfpathrectangle{\pgfqpoint{0.100000in}{2.413063in}}{\pgfqpoint{5.037500in}{3.427208in}}%
\pgfusepath{clip}%
\pgfsetbuttcap%
\pgfsetroundjoin%
\definecolor{currentfill}{rgb}{0.678431,1.000000,0.184314}%
\pgfsetfillcolor{currentfill}%
\pgfsetfillopacity{0.500000}%
\pgfsetlinewidth{0.250937pt}%
\definecolor{currentstroke}{rgb}{0.000000,0.000000,0.000000}%
\pgfsetstrokecolor{currentstroke}%
\pgfsetstrokeopacity{0.500000}%
\pgfsetdash{}{0pt}%
\pgfsys@defobject{currentmarker}{\pgfqpoint{-0.033333in}{-0.033333in}}{\pgfqpoint{0.033333in}{0.033333in}}{%
\pgfpathmoveto{\pgfqpoint{0.000000in}{-0.033333in}}%
\pgfpathcurveto{\pgfqpoint{0.008840in}{-0.033333in}}{\pgfqpoint{0.017319in}{-0.029821in}}{\pgfqpoint{0.023570in}{-0.023570in}}%
\pgfpathcurveto{\pgfqpoint{0.029821in}{-0.017319in}}{\pgfqpoint{0.033333in}{-0.008840in}}{\pgfqpoint{0.033333in}{0.000000in}}%
\pgfpathcurveto{\pgfqpoint{0.033333in}{0.008840in}}{\pgfqpoint{0.029821in}{0.017319in}}{\pgfqpoint{0.023570in}{0.023570in}}%
\pgfpathcurveto{\pgfqpoint{0.017319in}{0.029821in}}{\pgfqpoint{0.008840in}{0.033333in}}{\pgfqpoint{0.000000in}{0.033333in}}%
\pgfpathcurveto{\pgfqpoint{-0.008840in}{0.033333in}}{\pgfqpoint{-0.017319in}{0.029821in}}{\pgfqpoint{-0.023570in}{0.023570in}}%
\pgfpathcurveto{\pgfqpoint{-0.029821in}{0.017319in}}{\pgfqpoint{-0.033333in}{0.008840in}}{\pgfqpoint{-0.033333in}{0.000000in}}%
\pgfpathcurveto{\pgfqpoint{-0.033333in}{-0.008840in}}{\pgfqpoint{-0.029821in}{-0.017319in}}{\pgfqpoint{-0.023570in}{-0.023570in}}%
\pgfpathcurveto{\pgfqpoint{-0.017319in}{-0.029821in}}{\pgfqpoint{-0.008840in}{-0.033333in}}{\pgfqpoint{0.000000in}{-0.033333in}}%
\pgfpathclose%
\pgfusepath{stroke,fill}%
}%
\begin{pgfscope}%
\pgfsys@transformshift{0.888019in}{5.369454in}%
\pgfsys@useobject{currentmarker}{}%
\end{pgfscope}%
\end{pgfscope}%
\begin{pgfscope}%
\pgfpathrectangle{\pgfqpoint{0.100000in}{2.413063in}}{\pgfqpoint{5.037500in}{3.427208in}}%
\pgfusepath{clip}%
\pgfsetrectcap%
\pgfsetroundjoin%
\pgfsetlinewidth{1.505625pt}%
\definecolor{currentstroke}{rgb}{0.678431,1.000000,0.184314}%
\pgfsetstrokecolor{currentstroke}%
\pgfsetstrokeopacity{0.500000}%
\pgfsetdash{}{0pt}%
\pgfpathmoveto{\pgfqpoint{0.591430in}{5.448578in}}%
\pgfusepath{stroke}%
\end{pgfscope}%
\begin{pgfscope}%
\pgfpathrectangle{\pgfqpoint{0.100000in}{2.413063in}}{\pgfqpoint{5.037500in}{3.427208in}}%
\pgfusepath{clip}%
\pgfsetbuttcap%
\pgfsetroundjoin%
\definecolor{currentfill}{rgb}{0.678431,1.000000,0.184314}%
\pgfsetfillcolor{currentfill}%
\pgfsetfillopacity{0.500000}%
\pgfsetlinewidth{0.250937pt}%
\definecolor{currentstroke}{rgb}{0.000000,0.000000,0.000000}%
\pgfsetstrokecolor{currentstroke}%
\pgfsetstrokeopacity{0.500000}%
\pgfsetdash{}{0pt}%
\pgfsys@defobject{currentmarker}{\pgfqpoint{-0.022222in}{-0.022222in}}{\pgfqpoint{0.022222in}{0.022222in}}{%
\pgfpathmoveto{\pgfqpoint{0.000000in}{-0.022222in}}%
\pgfpathcurveto{\pgfqpoint{0.005893in}{-0.022222in}}{\pgfqpoint{0.011546in}{-0.019881in}}{\pgfqpoint{0.015713in}{-0.015713in}}%
\pgfpathcurveto{\pgfqpoint{0.019881in}{-0.011546in}}{\pgfqpoint{0.022222in}{-0.005893in}}{\pgfqpoint{0.022222in}{0.000000in}}%
\pgfpathcurveto{\pgfqpoint{0.022222in}{0.005893in}}{\pgfqpoint{0.019881in}{0.011546in}}{\pgfqpoint{0.015713in}{0.015713in}}%
\pgfpathcurveto{\pgfqpoint{0.011546in}{0.019881in}}{\pgfqpoint{0.005893in}{0.022222in}}{\pgfqpoint{0.000000in}{0.022222in}}%
\pgfpathcurveto{\pgfqpoint{-0.005893in}{0.022222in}}{\pgfqpoint{-0.011546in}{0.019881in}}{\pgfqpoint{-0.015713in}{0.015713in}}%
\pgfpathcurveto{\pgfqpoint{-0.019881in}{0.011546in}}{\pgfqpoint{-0.022222in}{0.005893in}}{\pgfqpoint{-0.022222in}{0.000000in}}%
\pgfpathcurveto{\pgfqpoint{-0.022222in}{-0.005893in}}{\pgfqpoint{-0.019881in}{-0.011546in}}{\pgfqpoint{-0.015713in}{-0.015713in}}%
\pgfpathcurveto{\pgfqpoint{-0.011546in}{-0.019881in}}{\pgfqpoint{-0.005893in}{-0.022222in}}{\pgfqpoint{0.000000in}{-0.022222in}}%
\pgfpathclose%
\pgfusepath{stroke,fill}%
}%
\begin{pgfscope}%
\pgfsys@transformshift{0.591430in}{5.448578in}%
\pgfsys@useobject{currentmarker}{}%
\end{pgfscope}%
\end{pgfscope}%
\begin{pgfscope}%
\pgfpathrectangle{\pgfqpoint{0.100000in}{2.413063in}}{\pgfqpoint{5.037500in}{3.427208in}}%
\pgfusepath{clip}%
\pgfsetrectcap%
\pgfsetroundjoin%
\pgfsetlinewidth{1.505625pt}%
\definecolor{currentstroke}{rgb}{0.678431,1.000000,0.184314}%
\pgfsetstrokecolor{currentstroke}%
\pgfsetstrokeopacity{0.500000}%
\pgfsetdash{}{0pt}%
\pgfpathmoveto{\pgfqpoint{0.718215in}{5.686601in}}%
\pgfusepath{stroke}%
\end{pgfscope}%
\begin{pgfscope}%
\pgfpathrectangle{\pgfqpoint{0.100000in}{2.413063in}}{\pgfqpoint{5.037500in}{3.427208in}}%
\pgfusepath{clip}%
\pgfsetbuttcap%
\pgfsetroundjoin%
\definecolor{currentfill}{rgb}{0.678431,1.000000,0.184314}%
\pgfsetfillcolor{currentfill}%
\pgfsetfillopacity{0.500000}%
\pgfsetlinewidth{0.250937pt}%
\definecolor{currentstroke}{rgb}{0.000000,0.000000,0.000000}%
\pgfsetstrokecolor{currentstroke}%
\pgfsetstrokeopacity{0.500000}%
\pgfsetdash{}{0pt}%
\pgfsys@defobject{currentmarker}{\pgfqpoint{-0.022222in}{-0.022222in}}{\pgfqpoint{0.022222in}{0.022222in}}{%
\pgfpathmoveto{\pgfqpoint{0.000000in}{-0.022222in}}%
\pgfpathcurveto{\pgfqpoint{0.005893in}{-0.022222in}}{\pgfqpoint{0.011546in}{-0.019881in}}{\pgfqpoint{0.015713in}{-0.015713in}}%
\pgfpathcurveto{\pgfqpoint{0.019881in}{-0.011546in}}{\pgfqpoint{0.022222in}{-0.005893in}}{\pgfqpoint{0.022222in}{0.000000in}}%
\pgfpathcurveto{\pgfqpoint{0.022222in}{0.005893in}}{\pgfqpoint{0.019881in}{0.011546in}}{\pgfqpoint{0.015713in}{0.015713in}}%
\pgfpathcurveto{\pgfqpoint{0.011546in}{0.019881in}}{\pgfqpoint{0.005893in}{0.022222in}}{\pgfqpoint{0.000000in}{0.022222in}}%
\pgfpathcurveto{\pgfqpoint{-0.005893in}{0.022222in}}{\pgfqpoint{-0.011546in}{0.019881in}}{\pgfqpoint{-0.015713in}{0.015713in}}%
\pgfpathcurveto{\pgfqpoint{-0.019881in}{0.011546in}}{\pgfqpoint{-0.022222in}{0.005893in}}{\pgfqpoint{-0.022222in}{0.000000in}}%
\pgfpathcurveto{\pgfqpoint{-0.022222in}{-0.005893in}}{\pgfqpoint{-0.019881in}{-0.011546in}}{\pgfqpoint{-0.015713in}{-0.015713in}}%
\pgfpathcurveto{\pgfqpoint{-0.011546in}{-0.019881in}}{\pgfqpoint{-0.005893in}{-0.022222in}}{\pgfqpoint{0.000000in}{-0.022222in}}%
\pgfpathclose%
\pgfusepath{stroke,fill}%
}%
\begin{pgfscope}%
\pgfsys@transformshift{0.718215in}{5.686601in}%
\pgfsys@useobject{currentmarker}{}%
\end{pgfscope}%
\end{pgfscope}%
\begin{pgfscope}%
\pgfpathrectangle{\pgfqpoint{0.100000in}{2.413063in}}{\pgfqpoint{5.037500in}{3.427208in}}%
\pgfusepath{clip}%
\pgfsetrectcap%
\pgfsetroundjoin%
\pgfsetlinewidth{1.505625pt}%
\definecolor{currentstroke}{rgb}{0.678431,1.000000,0.184314}%
\pgfsetstrokecolor{currentstroke}%
\pgfsetstrokeopacity{0.500000}%
\pgfsetdash{}{0pt}%
\pgfpathmoveto{\pgfqpoint{0.627763in}{5.547031in}}%
\pgfusepath{stroke}%
\end{pgfscope}%
\begin{pgfscope}%
\pgfpathrectangle{\pgfqpoint{0.100000in}{2.413063in}}{\pgfqpoint{5.037500in}{3.427208in}}%
\pgfusepath{clip}%
\pgfsetbuttcap%
\pgfsetroundjoin%
\definecolor{currentfill}{rgb}{0.678431,1.000000,0.184314}%
\pgfsetfillcolor{currentfill}%
\pgfsetfillopacity{0.500000}%
\pgfsetlinewidth{0.250937pt}%
\definecolor{currentstroke}{rgb}{0.000000,0.000000,0.000000}%
\pgfsetstrokecolor{currentstroke}%
\pgfsetstrokeopacity{0.500000}%
\pgfsetdash{}{0pt}%
\pgfsys@defobject{currentmarker}{\pgfqpoint{-0.019444in}{-0.019444in}}{\pgfqpoint{0.019444in}{0.019444in}}{%
\pgfpathmoveto{\pgfqpoint{0.000000in}{-0.019444in}}%
\pgfpathcurveto{\pgfqpoint{0.005157in}{-0.019444in}}{\pgfqpoint{0.010103in}{-0.017396in}}{\pgfqpoint{0.013749in}{-0.013749in}}%
\pgfpathcurveto{\pgfqpoint{0.017396in}{-0.010103in}}{\pgfqpoint{0.019444in}{-0.005157in}}{\pgfqpoint{0.019444in}{0.000000in}}%
\pgfpathcurveto{\pgfqpoint{0.019444in}{0.005157in}}{\pgfqpoint{0.017396in}{0.010103in}}{\pgfqpoint{0.013749in}{0.013749in}}%
\pgfpathcurveto{\pgfqpoint{0.010103in}{0.017396in}}{\pgfqpoint{0.005157in}{0.019444in}}{\pgfqpoint{0.000000in}{0.019444in}}%
\pgfpathcurveto{\pgfqpoint{-0.005157in}{0.019444in}}{\pgfqpoint{-0.010103in}{0.017396in}}{\pgfqpoint{-0.013749in}{0.013749in}}%
\pgfpathcurveto{\pgfqpoint{-0.017396in}{0.010103in}}{\pgfqpoint{-0.019444in}{0.005157in}}{\pgfqpoint{-0.019444in}{0.000000in}}%
\pgfpathcurveto{\pgfqpoint{-0.019444in}{-0.005157in}}{\pgfqpoint{-0.017396in}{-0.010103in}}{\pgfqpoint{-0.013749in}{-0.013749in}}%
\pgfpathcurveto{\pgfqpoint{-0.010103in}{-0.017396in}}{\pgfqpoint{-0.005157in}{-0.019444in}}{\pgfqpoint{0.000000in}{-0.019444in}}%
\pgfpathclose%
\pgfusepath{stroke,fill}%
}%
\begin{pgfscope}%
\pgfsys@transformshift{0.627763in}{5.547031in}%
\pgfsys@useobject{currentmarker}{}%
\end{pgfscope}%
\end{pgfscope}%
\begin{pgfscope}%
\pgfpathrectangle{\pgfqpoint{0.100000in}{2.413063in}}{\pgfqpoint{5.037500in}{3.427208in}}%
\pgfusepath{clip}%
\pgfsetrectcap%
\pgfsetroundjoin%
\pgfsetlinewidth{1.505625pt}%
\definecolor{currentstroke}{rgb}{0.000000,0.000000,1.000000}%
\pgfsetstrokecolor{currentstroke}%
\pgfsetstrokeopacity{0.500000}%
\pgfsetdash{}{0pt}%
\pgfpathmoveto{\pgfqpoint{0.689747in}{5.595764in}}%
\pgfusepath{stroke}%
\end{pgfscope}%
\begin{pgfscope}%
\pgfpathrectangle{\pgfqpoint{0.100000in}{2.413063in}}{\pgfqpoint{5.037500in}{3.427208in}}%
\pgfusepath{clip}%
\pgfsetbuttcap%
\pgfsetroundjoin%
\definecolor{currentfill}{rgb}{0.000000,0.000000,1.000000}%
\pgfsetfillcolor{currentfill}%
\pgfsetfillopacity{0.500000}%
\pgfsetlinewidth{0.250937pt}%
\definecolor{currentstroke}{rgb}{0.000000,0.000000,0.000000}%
\pgfsetstrokecolor{currentstroke}%
\pgfsetstrokeopacity{0.500000}%
\pgfsetdash{}{0pt}%
\pgfsys@defobject{currentmarker}{\pgfqpoint{-0.019444in}{-0.019444in}}{\pgfqpoint{0.019444in}{0.019444in}}{%
\pgfpathmoveto{\pgfqpoint{0.000000in}{-0.019444in}}%
\pgfpathcurveto{\pgfqpoint{0.005157in}{-0.019444in}}{\pgfqpoint{0.010103in}{-0.017396in}}{\pgfqpoint{0.013749in}{-0.013749in}}%
\pgfpathcurveto{\pgfqpoint{0.017396in}{-0.010103in}}{\pgfqpoint{0.019444in}{-0.005157in}}{\pgfqpoint{0.019444in}{0.000000in}}%
\pgfpathcurveto{\pgfqpoint{0.019444in}{0.005157in}}{\pgfqpoint{0.017396in}{0.010103in}}{\pgfqpoint{0.013749in}{0.013749in}}%
\pgfpathcurveto{\pgfqpoint{0.010103in}{0.017396in}}{\pgfqpoint{0.005157in}{0.019444in}}{\pgfqpoint{0.000000in}{0.019444in}}%
\pgfpathcurveto{\pgfqpoint{-0.005157in}{0.019444in}}{\pgfqpoint{-0.010103in}{0.017396in}}{\pgfqpoint{-0.013749in}{0.013749in}}%
\pgfpathcurveto{\pgfqpoint{-0.017396in}{0.010103in}}{\pgfqpoint{-0.019444in}{0.005157in}}{\pgfqpoint{-0.019444in}{0.000000in}}%
\pgfpathcurveto{\pgfqpoint{-0.019444in}{-0.005157in}}{\pgfqpoint{-0.017396in}{-0.010103in}}{\pgfqpoint{-0.013749in}{-0.013749in}}%
\pgfpathcurveto{\pgfqpoint{-0.010103in}{-0.017396in}}{\pgfqpoint{-0.005157in}{-0.019444in}}{\pgfqpoint{0.000000in}{-0.019444in}}%
\pgfpathclose%
\pgfusepath{stroke,fill}%
}%
\begin{pgfscope}%
\pgfsys@transformshift{0.689747in}{5.595764in}%
\pgfsys@useobject{currentmarker}{}%
\end{pgfscope}%
\end{pgfscope}%
\begin{pgfscope}%
\pgfpathrectangle{\pgfqpoint{0.100000in}{2.413063in}}{\pgfqpoint{5.037500in}{3.427208in}}%
\pgfusepath{clip}%
\pgfsetrectcap%
\pgfsetroundjoin%
\pgfsetlinewidth{1.505625pt}%
\definecolor{currentstroke}{rgb}{0.678431,1.000000,0.184314}%
\pgfsetstrokecolor{currentstroke}%
\pgfsetstrokeopacity{0.500000}%
\pgfsetdash{}{0pt}%
\pgfpathmoveto{\pgfqpoint{1.062204in}{5.497835in}}%
\pgfusepath{stroke}%
\end{pgfscope}%
\begin{pgfscope}%
\pgfpathrectangle{\pgfqpoint{0.100000in}{2.413063in}}{\pgfqpoint{5.037500in}{3.427208in}}%
\pgfusepath{clip}%
\pgfsetbuttcap%
\pgfsetroundjoin%
\definecolor{currentfill}{rgb}{0.678431,1.000000,0.184314}%
\pgfsetfillcolor{currentfill}%
\pgfsetfillopacity{0.500000}%
\pgfsetlinewidth{0.250937pt}%
\definecolor{currentstroke}{rgb}{0.000000,0.000000,0.000000}%
\pgfsetstrokecolor{currentstroke}%
\pgfsetstrokeopacity{0.500000}%
\pgfsetdash{}{0pt}%
\pgfsys@defobject{currentmarker}{\pgfqpoint{-0.030556in}{-0.030556in}}{\pgfqpoint{0.030556in}{0.030556in}}{%
\pgfpathmoveto{\pgfqpoint{0.000000in}{-0.030556in}}%
\pgfpathcurveto{\pgfqpoint{0.008103in}{-0.030556in}}{\pgfqpoint{0.015876in}{-0.027336in}}{\pgfqpoint{0.021606in}{-0.021606in}}%
\pgfpathcurveto{\pgfqpoint{0.027336in}{-0.015876in}}{\pgfqpoint{0.030556in}{-0.008103in}}{\pgfqpoint{0.030556in}{0.000000in}}%
\pgfpathcurveto{\pgfqpoint{0.030556in}{0.008103in}}{\pgfqpoint{0.027336in}{0.015876in}}{\pgfqpoint{0.021606in}{0.021606in}}%
\pgfpathcurveto{\pgfqpoint{0.015876in}{0.027336in}}{\pgfqpoint{0.008103in}{0.030556in}}{\pgfqpoint{0.000000in}{0.030556in}}%
\pgfpathcurveto{\pgfqpoint{-0.008103in}{0.030556in}}{\pgfqpoint{-0.015876in}{0.027336in}}{\pgfqpoint{-0.021606in}{0.021606in}}%
\pgfpathcurveto{\pgfqpoint{-0.027336in}{0.015876in}}{\pgfqpoint{-0.030556in}{0.008103in}}{\pgfqpoint{-0.030556in}{0.000000in}}%
\pgfpathcurveto{\pgfqpoint{-0.030556in}{-0.008103in}}{\pgfqpoint{-0.027336in}{-0.015876in}}{\pgfqpoint{-0.021606in}{-0.021606in}}%
\pgfpathcurveto{\pgfqpoint{-0.015876in}{-0.027336in}}{\pgfqpoint{-0.008103in}{-0.030556in}}{\pgfqpoint{0.000000in}{-0.030556in}}%
\pgfpathclose%
\pgfusepath{stroke,fill}%
}%
\begin{pgfscope}%
\pgfsys@transformshift{1.062204in}{5.497835in}%
\pgfsys@useobject{currentmarker}{}%
\end{pgfscope}%
\end{pgfscope}%
\begin{pgfscope}%
\pgfpathrectangle{\pgfqpoint{0.100000in}{2.413063in}}{\pgfqpoint{5.037500in}{3.427208in}}%
\pgfusepath{clip}%
\pgfsetrectcap%
\pgfsetroundjoin%
\pgfsetlinewidth{1.505625pt}%
\definecolor{currentstroke}{rgb}{0.678431,1.000000,0.184314}%
\pgfsetstrokecolor{currentstroke}%
\pgfsetstrokeopacity{0.500000}%
\pgfsetdash{}{0pt}%
\pgfpathmoveto{\pgfqpoint{0.946329in}{5.336301in}}%
\pgfusepath{stroke}%
\end{pgfscope}%
\begin{pgfscope}%
\pgfpathrectangle{\pgfqpoint{0.100000in}{2.413063in}}{\pgfqpoint{5.037500in}{3.427208in}}%
\pgfusepath{clip}%
\pgfsetbuttcap%
\pgfsetroundjoin%
\definecolor{currentfill}{rgb}{0.678431,1.000000,0.184314}%
\pgfsetfillcolor{currentfill}%
\pgfsetfillopacity{0.500000}%
\pgfsetlinewidth{0.250937pt}%
\definecolor{currentstroke}{rgb}{0.000000,0.000000,0.000000}%
\pgfsetstrokecolor{currentstroke}%
\pgfsetstrokeopacity{0.500000}%
\pgfsetdash{}{0pt}%
\pgfsys@defobject{currentmarker}{\pgfqpoint{-0.030556in}{-0.030556in}}{\pgfqpoint{0.030556in}{0.030556in}}{%
\pgfpathmoveto{\pgfqpoint{0.000000in}{-0.030556in}}%
\pgfpathcurveto{\pgfqpoint{0.008103in}{-0.030556in}}{\pgfqpoint{0.015876in}{-0.027336in}}{\pgfqpoint{0.021606in}{-0.021606in}}%
\pgfpathcurveto{\pgfqpoint{0.027336in}{-0.015876in}}{\pgfqpoint{0.030556in}{-0.008103in}}{\pgfqpoint{0.030556in}{0.000000in}}%
\pgfpathcurveto{\pgfqpoint{0.030556in}{0.008103in}}{\pgfqpoint{0.027336in}{0.015876in}}{\pgfqpoint{0.021606in}{0.021606in}}%
\pgfpathcurveto{\pgfqpoint{0.015876in}{0.027336in}}{\pgfqpoint{0.008103in}{0.030556in}}{\pgfqpoint{0.000000in}{0.030556in}}%
\pgfpathcurveto{\pgfqpoint{-0.008103in}{0.030556in}}{\pgfqpoint{-0.015876in}{0.027336in}}{\pgfqpoint{-0.021606in}{0.021606in}}%
\pgfpathcurveto{\pgfqpoint{-0.027336in}{0.015876in}}{\pgfqpoint{-0.030556in}{0.008103in}}{\pgfqpoint{-0.030556in}{0.000000in}}%
\pgfpathcurveto{\pgfqpoint{-0.030556in}{-0.008103in}}{\pgfqpoint{-0.027336in}{-0.015876in}}{\pgfqpoint{-0.021606in}{-0.021606in}}%
\pgfpathcurveto{\pgfqpoint{-0.015876in}{-0.027336in}}{\pgfqpoint{-0.008103in}{-0.030556in}}{\pgfqpoint{0.000000in}{-0.030556in}}%
\pgfpathclose%
\pgfusepath{stroke,fill}%
}%
\begin{pgfscope}%
\pgfsys@transformshift{0.946329in}{5.336301in}%
\pgfsys@useobject{currentmarker}{}%
\end{pgfscope}%
\end{pgfscope}%
\begin{pgfscope}%
\pgfpathrectangle{\pgfqpoint{0.100000in}{2.413063in}}{\pgfqpoint{5.037500in}{3.427208in}}%
\pgfusepath{clip}%
\pgfsetrectcap%
\pgfsetroundjoin%
\pgfsetlinewidth{1.505625pt}%
\definecolor{currentstroke}{rgb}{0.678431,1.000000,0.184314}%
\pgfsetstrokecolor{currentstroke}%
\pgfsetstrokeopacity{0.500000}%
\pgfsetdash{}{0pt}%
\pgfpathmoveto{\pgfqpoint{0.835996in}{5.530263in}}%
\pgfusepath{stroke}%
\end{pgfscope}%
\begin{pgfscope}%
\pgfpathrectangle{\pgfqpoint{0.100000in}{2.413063in}}{\pgfqpoint{5.037500in}{3.427208in}}%
\pgfusepath{clip}%
\pgfsetbuttcap%
\pgfsetroundjoin%
\definecolor{currentfill}{rgb}{0.678431,1.000000,0.184314}%
\pgfsetfillcolor{currentfill}%
\pgfsetfillopacity{0.500000}%
\pgfsetlinewidth{0.250937pt}%
\definecolor{currentstroke}{rgb}{0.000000,0.000000,0.000000}%
\pgfsetstrokecolor{currentstroke}%
\pgfsetstrokeopacity{0.500000}%
\pgfsetdash{}{0pt}%
\pgfsys@defobject{currentmarker}{\pgfqpoint{-0.030556in}{-0.030556in}}{\pgfqpoint{0.030556in}{0.030556in}}{%
\pgfpathmoveto{\pgfqpoint{0.000000in}{-0.030556in}}%
\pgfpathcurveto{\pgfqpoint{0.008103in}{-0.030556in}}{\pgfqpoint{0.015876in}{-0.027336in}}{\pgfqpoint{0.021606in}{-0.021606in}}%
\pgfpathcurveto{\pgfqpoint{0.027336in}{-0.015876in}}{\pgfqpoint{0.030556in}{-0.008103in}}{\pgfqpoint{0.030556in}{0.000000in}}%
\pgfpathcurveto{\pgfqpoint{0.030556in}{0.008103in}}{\pgfqpoint{0.027336in}{0.015876in}}{\pgfqpoint{0.021606in}{0.021606in}}%
\pgfpathcurveto{\pgfqpoint{0.015876in}{0.027336in}}{\pgfqpoint{0.008103in}{0.030556in}}{\pgfqpoint{0.000000in}{0.030556in}}%
\pgfpathcurveto{\pgfqpoint{-0.008103in}{0.030556in}}{\pgfqpoint{-0.015876in}{0.027336in}}{\pgfqpoint{-0.021606in}{0.021606in}}%
\pgfpathcurveto{\pgfqpoint{-0.027336in}{0.015876in}}{\pgfqpoint{-0.030556in}{0.008103in}}{\pgfqpoint{-0.030556in}{0.000000in}}%
\pgfpathcurveto{\pgfqpoint{-0.030556in}{-0.008103in}}{\pgfqpoint{-0.027336in}{-0.015876in}}{\pgfqpoint{-0.021606in}{-0.021606in}}%
\pgfpathcurveto{\pgfqpoint{-0.015876in}{-0.027336in}}{\pgfqpoint{-0.008103in}{-0.030556in}}{\pgfqpoint{0.000000in}{-0.030556in}}%
\pgfpathclose%
\pgfusepath{stroke,fill}%
}%
\begin{pgfscope}%
\pgfsys@transformshift{0.835996in}{5.530263in}%
\pgfsys@useobject{currentmarker}{}%
\end{pgfscope}%
\end{pgfscope}%
\begin{pgfscope}%
\pgfpathrectangle{\pgfqpoint{0.100000in}{2.413063in}}{\pgfqpoint{5.037500in}{3.427208in}}%
\pgfusepath{clip}%
\pgfsetrectcap%
\pgfsetroundjoin%
\pgfsetlinewidth{1.505625pt}%
\definecolor{currentstroke}{rgb}{0.678431,1.000000,0.184314}%
\pgfsetstrokecolor{currentstroke}%
\pgfsetstrokeopacity{0.500000}%
\pgfsetdash{}{0pt}%
\pgfpathmoveto{\pgfqpoint{0.794333in}{5.442772in}}%
\pgfusepath{stroke}%
\end{pgfscope}%
\begin{pgfscope}%
\pgfpathrectangle{\pgfqpoint{0.100000in}{2.413063in}}{\pgfqpoint{5.037500in}{3.427208in}}%
\pgfusepath{clip}%
\pgfsetbuttcap%
\pgfsetroundjoin%
\definecolor{currentfill}{rgb}{0.678431,1.000000,0.184314}%
\pgfsetfillcolor{currentfill}%
\pgfsetfillopacity{0.500000}%
\pgfsetlinewidth{0.250937pt}%
\definecolor{currentstroke}{rgb}{0.000000,0.000000,0.000000}%
\pgfsetstrokecolor{currentstroke}%
\pgfsetstrokeopacity{0.500000}%
\pgfsetdash{}{0pt}%
\pgfsys@defobject{currentmarker}{\pgfqpoint{-0.052778in}{-0.052778in}}{\pgfqpoint{0.052778in}{0.052778in}}{%
\pgfpathmoveto{\pgfqpoint{0.000000in}{-0.052778in}}%
\pgfpathcurveto{\pgfqpoint{0.013997in}{-0.052778in}}{\pgfqpoint{0.027422in}{-0.047217in}}{\pgfqpoint{0.037320in}{-0.037320in}}%
\pgfpathcurveto{\pgfqpoint{0.047217in}{-0.027422in}}{\pgfqpoint{0.052778in}{-0.013997in}}{\pgfqpoint{0.052778in}{0.000000in}}%
\pgfpathcurveto{\pgfqpoint{0.052778in}{0.013997in}}{\pgfqpoint{0.047217in}{0.027422in}}{\pgfqpoint{0.037320in}{0.037320in}}%
\pgfpathcurveto{\pgfqpoint{0.027422in}{0.047217in}}{\pgfqpoint{0.013997in}{0.052778in}}{\pgfqpoint{0.000000in}{0.052778in}}%
\pgfpathcurveto{\pgfqpoint{-0.013997in}{0.052778in}}{\pgfqpoint{-0.027422in}{0.047217in}}{\pgfqpoint{-0.037320in}{0.037320in}}%
\pgfpathcurveto{\pgfqpoint{-0.047217in}{0.027422in}}{\pgfqpoint{-0.052778in}{0.013997in}}{\pgfqpoint{-0.052778in}{0.000000in}}%
\pgfpathcurveto{\pgfqpoint{-0.052778in}{-0.013997in}}{\pgfqpoint{-0.047217in}{-0.027422in}}{\pgfqpoint{-0.037320in}{-0.037320in}}%
\pgfpathcurveto{\pgfqpoint{-0.027422in}{-0.047217in}}{\pgfqpoint{-0.013997in}{-0.052778in}}{\pgfqpoint{0.000000in}{-0.052778in}}%
\pgfpathclose%
\pgfusepath{stroke,fill}%
}%
\begin{pgfscope}%
\pgfsys@transformshift{0.794333in}{5.442772in}%
\pgfsys@useobject{currentmarker}{}%
\end{pgfscope}%
\end{pgfscope}%
\begin{pgfscope}%
\pgfpathrectangle{\pgfqpoint{0.100000in}{2.413063in}}{\pgfqpoint{5.037500in}{3.427208in}}%
\pgfusepath{clip}%
\pgfsetrectcap%
\pgfsetroundjoin%
\pgfsetlinewidth{1.505625pt}%
\definecolor{currentstroke}{rgb}{0.678431,1.000000,0.184314}%
\pgfsetstrokecolor{currentstroke}%
\pgfsetstrokeopacity{0.500000}%
\pgfsetdash{}{0pt}%
\pgfpathmoveto{\pgfqpoint{4.056648in}{4.235611in}}%
\pgfusepath{stroke}%
\end{pgfscope}%
\begin{pgfscope}%
\pgfpathrectangle{\pgfqpoint{0.100000in}{2.413063in}}{\pgfqpoint{5.037500in}{3.427208in}}%
\pgfusepath{clip}%
\pgfsetbuttcap%
\pgfsetroundjoin%
\definecolor{currentfill}{rgb}{0.678431,1.000000,0.184314}%
\pgfsetfillcolor{currentfill}%
\pgfsetfillopacity{0.500000}%
\pgfsetlinewidth{0.250937pt}%
\definecolor{currentstroke}{rgb}{0.000000,0.000000,0.000000}%
\pgfsetstrokecolor{currentstroke}%
\pgfsetstrokeopacity{0.500000}%
\pgfsetdash{}{0pt}%
\pgfsys@defobject{currentmarker}{\pgfqpoint{-0.047222in}{-0.047222in}}{\pgfqpoint{0.047222in}{0.047222in}}{%
\pgfpathmoveto{\pgfqpoint{0.000000in}{-0.047222in}}%
\pgfpathcurveto{\pgfqpoint{0.012523in}{-0.047222in}}{\pgfqpoint{0.024536in}{-0.042247in}}{\pgfqpoint{0.033391in}{-0.033391in}}%
\pgfpathcurveto{\pgfqpoint{0.042247in}{-0.024536in}}{\pgfqpoint{0.047222in}{-0.012523in}}{\pgfqpoint{0.047222in}{0.000000in}}%
\pgfpathcurveto{\pgfqpoint{0.047222in}{0.012523in}}{\pgfqpoint{0.042247in}{0.024536in}}{\pgfqpoint{0.033391in}{0.033391in}}%
\pgfpathcurveto{\pgfqpoint{0.024536in}{0.042247in}}{\pgfqpoint{0.012523in}{0.047222in}}{\pgfqpoint{0.000000in}{0.047222in}}%
\pgfpathcurveto{\pgfqpoint{-0.012523in}{0.047222in}}{\pgfqpoint{-0.024536in}{0.042247in}}{\pgfqpoint{-0.033391in}{0.033391in}}%
\pgfpathcurveto{\pgfqpoint{-0.042247in}{0.024536in}}{\pgfqpoint{-0.047222in}{0.012523in}}{\pgfqpoint{-0.047222in}{0.000000in}}%
\pgfpathcurveto{\pgfqpoint{-0.047222in}{-0.012523in}}{\pgfqpoint{-0.042247in}{-0.024536in}}{\pgfqpoint{-0.033391in}{-0.033391in}}%
\pgfpathcurveto{\pgfqpoint{-0.024536in}{-0.042247in}}{\pgfqpoint{-0.012523in}{-0.047222in}}{\pgfqpoint{0.000000in}{-0.047222in}}%
\pgfpathclose%
\pgfusepath{stroke,fill}%
}%
\begin{pgfscope}%
\pgfsys@transformshift{4.056648in}{4.235611in}%
\pgfsys@useobject{currentmarker}{}%
\end{pgfscope}%
\end{pgfscope}%
\begin{pgfscope}%
\pgfpathrectangle{\pgfqpoint{0.100000in}{2.413063in}}{\pgfqpoint{5.037500in}{3.427208in}}%
\pgfusepath{clip}%
\pgfsetrectcap%
\pgfsetroundjoin%
\pgfsetlinewidth{1.505625pt}%
\definecolor{currentstroke}{rgb}{0.678431,1.000000,0.184314}%
\pgfsetstrokecolor{currentstroke}%
\pgfsetstrokeopacity{0.500000}%
\pgfsetdash{}{0pt}%
\pgfpathmoveto{\pgfqpoint{4.006916in}{4.294787in}}%
\pgfusepath{stroke}%
\end{pgfscope}%
\begin{pgfscope}%
\pgfpathrectangle{\pgfqpoint{0.100000in}{2.413063in}}{\pgfqpoint{5.037500in}{3.427208in}}%
\pgfusepath{clip}%
\pgfsetbuttcap%
\pgfsetroundjoin%
\definecolor{currentfill}{rgb}{0.678431,1.000000,0.184314}%
\pgfsetfillcolor{currentfill}%
\pgfsetfillopacity{0.500000}%
\pgfsetlinewidth{0.250937pt}%
\definecolor{currentstroke}{rgb}{0.000000,0.000000,0.000000}%
\pgfsetstrokecolor{currentstroke}%
\pgfsetstrokeopacity{0.500000}%
\pgfsetdash{}{0pt}%
\pgfsys@defobject{currentmarker}{\pgfqpoint{-0.038889in}{-0.038889in}}{\pgfqpoint{0.038889in}{0.038889in}}{%
\pgfpathmoveto{\pgfqpoint{0.000000in}{-0.038889in}}%
\pgfpathcurveto{\pgfqpoint{0.010313in}{-0.038889in}}{\pgfqpoint{0.020206in}{-0.034791in}}{\pgfqpoint{0.027499in}{-0.027499in}}%
\pgfpathcurveto{\pgfqpoint{0.034791in}{-0.020206in}}{\pgfqpoint{0.038889in}{-0.010313in}}{\pgfqpoint{0.038889in}{0.000000in}}%
\pgfpathcurveto{\pgfqpoint{0.038889in}{0.010313in}}{\pgfqpoint{0.034791in}{0.020206in}}{\pgfqpoint{0.027499in}{0.027499in}}%
\pgfpathcurveto{\pgfqpoint{0.020206in}{0.034791in}}{\pgfqpoint{0.010313in}{0.038889in}}{\pgfqpoint{0.000000in}{0.038889in}}%
\pgfpathcurveto{\pgfqpoint{-0.010313in}{0.038889in}}{\pgfqpoint{-0.020206in}{0.034791in}}{\pgfqpoint{-0.027499in}{0.027499in}}%
\pgfpathcurveto{\pgfqpoint{-0.034791in}{0.020206in}}{\pgfqpoint{-0.038889in}{0.010313in}}{\pgfqpoint{-0.038889in}{0.000000in}}%
\pgfpathcurveto{\pgfqpoint{-0.038889in}{-0.010313in}}{\pgfqpoint{-0.034791in}{-0.020206in}}{\pgfqpoint{-0.027499in}{-0.027499in}}%
\pgfpathcurveto{\pgfqpoint{-0.020206in}{-0.034791in}}{\pgfqpoint{-0.010313in}{-0.038889in}}{\pgfqpoint{0.000000in}{-0.038889in}}%
\pgfpathclose%
\pgfusepath{stroke,fill}%
}%
\begin{pgfscope}%
\pgfsys@transformshift{4.006916in}{4.294787in}%
\pgfsys@useobject{currentmarker}{}%
\end{pgfscope}%
\end{pgfscope}%
\begin{pgfscope}%
\pgfpathrectangle{\pgfqpoint{0.100000in}{2.413063in}}{\pgfqpoint{5.037500in}{3.427208in}}%
\pgfusepath{clip}%
\pgfsetrectcap%
\pgfsetroundjoin%
\pgfsetlinewidth{1.505625pt}%
\definecolor{currentstroke}{rgb}{0.678431,1.000000,0.184314}%
\pgfsetstrokecolor{currentstroke}%
\pgfsetstrokeopacity{0.500000}%
\pgfsetdash{}{0pt}%
\pgfpathmoveto{\pgfqpoint{3.933253in}{4.292196in}}%
\pgfusepath{stroke}%
\end{pgfscope}%
\begin{pgfscope}%
\pgfpathrectangle{\pgfqpoint{0.100000in}{2.413063in}}{\pgfqpoint{5.037500in}{3.427208in}}%
\pgfusepath{clip}%
\pgfsetbuttcap%
\pgfsetroundjoin%
\definecolor{currentfill}{rgb}{0.678431,1.000000,0.184314}%
\pgfsetfillcolor{currentfill}%
\pgfsetfillopacity{0.500000}%
\pgfsetlinewidth{0.250937pt}%
\definecolor{currentstroke}{rgb}{0.000000,0.000000,0.000000}%
\pgfsetstrokecolor{currentstroke}%
\pgfsetstrokeopacity{0.500000}%
\pgfsetdash{}{0pt}%
\pgfsys@defobject{currentmarker}{\pgfqpoint{-0.033333in}{-0.033333in}}{\pgfqpoint{0.033333in}{0.033333in}}{%
\pgfpathmoveto{\pgfqpoint{0.000000in}{-0.033333in}}%
\pgfpathcurveto{\pgfqpoint{0.008840in}{-0.033333in}}{\pgfqpoint{0.017319in}{-0.029821in}}{\pgfqpoint{0.023570in}{-0.023570in}}%
\pgfpathcurveto{\pgfqpoint{0.029821in}{-0.017319in}}{\pgfqpoint{0.033333in}{-0.008840in}}{\pgfqpoint{0.033333in}{0.000000in}}%
\pgfpathcurveto{\pgfqpoint{0.033333in}{0.008840in}}{\pgfqpoint{0.029821in}{0.017319in}}{\pgfqpoint{0.023570in}{0.023570in}}%
\pgfpathcurveto{\pgfqpoint{0.017319in}{0.029821in}}{\pgfqpoint{0.008840in}{0.033333in}}{\pgfqpoint{0.000000in}{0.033333in}}%
\pgfpathcurveto{\pgfqpoint{-0.008840in}{0.033333in}}{\pgfqpoint{-0.017319in}{0.029821in}}{\pgfqpoint{-0.023570in}{0.023570in}}%
\pgfpathcurveto{\pgfqpoint{-0.029821in}{0.017319in}}{\pgfqpoint{-0.033333in}{0.008840in}}{\pgfqpoint{-0.033333in}{0.000000in}}%
\pgfpathcurveto{\pgfqpoint{-0.033333in}{-0.008840in}}{\pgfqpoint{-0.029821in}{-0.017319in}}{\pgfqpoint{-0.023570in}{-0.023570in}}%
\pgfpathcurveto{\pgfqpoint{-0.017319in}{-0.029821in}}{\pgfqpoint{-0.008840in}{-0.033333in}}{\pgfqpoint{0.000000in}{-0.033333in}}%
\pgfpathclose%
\pgfusepath{stroke,fill}%
}%
\begin{pgfscope}%
\pgfsys@transformshift{3.933253in}{4.292196in}%
\pgfsys@useobject{currentmarker}{}%
\end{pgfscope}%
\end{pgfscope}%
\begin{pgfscope}%
\pgfpathrectangle{\pgfqpoint{0.100000in}{2.413063in}}{\pgfqpoint{5.037500in}{3.427208in}}%
\pgfusepath{clip}%
\pgfsetrectcap%
\pgfsetroundjoin%
\pgfsetlinewidth{1.505625pt}%
\definecolor{currentstroke}{rgb}{0.678431,1.000000,0.184314}%
\pgfsetstrokecolor{currentstroke}%
\pgfsetstrokeopacity{0.500000}%
\pgfsetdash{}{0pt}%
\pgfpathmoveto{\pgfqpoint{4.132275in}{4.463641in}}%
\pgfusepath{stroke}%
\end{pgfscope}%
\begin{pgfscope}%
\pgfpathrectangle{\pgfqpoint{0.100000in}{2.413063in}}{\pgfqpoint{5.037500in}{3.427208in}}%
\pgfusepath{clip}%
\pgfsetbuttcap%
\pgfsetroundjoin%
\definecolor{currentfill}{rgb}{0.678431,1.000000,0.184314}%
\pgfsetfillcolor{currentfill}%
\pgfsetfillopacity{0.500000}%
\pgfsetlinewidth{0.250937pt}%
\definecolor{currentstroke}{rgb}{0.000000,0.000000,0.000000}%
\pgfsetstrokecolor{currentstroke}%
\pgfsetstrokeopacity{0.500000}%
\pgfsetdash{}{0pt}%
\pgfsys@defobject{currentmarker}{\pgfqpoint{-0.025000in}{-0.025000in}}{\pgfqpoint{0.025000in}{0.025000in}}{%
\pgfpathmoveto{\pgfqpoint{0.000000in}{-0.025000in}}%
\pgfpathcurveto{\pgfqpoint{0.006630in}{-0.025000in}}{\pgfqpoint{0.012989in}{-0.022366in}}{\pgfqpoint{0.017678in}{-0.017678in}}%
\pgfpathcurveto{\pgfqpoint{0.022366in}{-0.012989in}}{\pgfqpoint{0.025000in}{-0.006630in}}{\pgfqpoint{0.025000in}{0.000000in}}%
\pgfpathcurveto{\pgfqpoint{0.025000in}{0.006630in}}{\pgfqpoint{0.022366in}{0.012989in}}{\pgfqpoint{0.017678in}{0.017678in}}%
\pgfpathcurveto{\pgfqpoint{0.012989in}{0.022366in}}{\pgfqpoint{0.006630in}{0.025000in}}{\pgfqpoint{0.000000in}{0.025000in}}%
\pgfpathcurveto{\pgfqpoint{-0.006630in}{0.025000in}}{\pgfqpoint{-0.012989in}{0.022366in}}{\pgfqpoint{-0.017678in}{0.017678in}}%
\pgfpathcurveto{\pgfqpoint{-0.022366in}{0.012989in}}{\pgfqpoint{-0.025000in}{0.006630in}}{\pgfqpoint{-0.025000in}{0.000000in}}%
\pgfpathcurveto{\pgfqpoint{-0.025000in}{-0.006630in}}{\pgfqpoint{-0.022366in}{-0.012989in}}{\pgfqpoint{-0.017678in}{-0.017678in}}%
\pgfpathcurveto{\pgfqpoint{-0.012989in}{-0.022366in}}{\pgfqpoint{-0.006630in}{-0.025000in}}{\pgfqpoint{0.000000in}{-0.025000in}}%
\pgfpathclose%
\pgfusepath{stroke,fill}%
}%
\begin{pgfscope}%
\pgfsys@transformshift{4.132275in}{4.463641in}%
\pgfsys@useobject{currentmarker}{}%
\end{pgfscope}%
\end{pgfscope}%
\begin{pgfscope}%
\pgfpathrectangle{\pgfqpoint{0.100000in}{2.413063in}}{\pgfqpoint{5.037500in}{3.427208in}}%
\pgfusepath{clip}%
\pgfsetrectcap%
\pgfsetroundjoin%
\pgfsetlinewidth{1.505625pt}%
\definecolor{currentstroke}{rgb}{0.678431,1.000000,0.184314}%
\pgfsetstrokecolor{currentstroke}%
\pgfsetstrokeopacity{0.500000}%
\pgfsetdash{}{0pt}%
\pgfpathmoveto{\pgfqpoint{3.998684in}{4.400590in}}%
\pgfusepath{stroke}%
\end{pgfscope}%
\begin{pgfscope}%
\pgfpathrectangle{\pgfqpoint{0.100000in}{2.413063in}}{\pgfqpoint{5.037500in}{3.427208in}}%
\pgfusepath{clip}%
\pgfsetbuttcap%
\pgfsetroundjoin%
\definecolor{currentfill}{rgb}{0.678431,1.000000,0.184314}%
\pgfsetfillcolor{currentfill}%
\pgfsetfillopacity{0.500000}%
\pgfsetlinewidth{0.250937pt}%
\definecolor{currentstroke}{rgb}{0.000000,0.000000,0.000000}%
\pgfsetstrokecolor{currentstroke}%
\pgfsetstrokeopacity{0.500000}%
\pgfsetdash{}{0pt}%
\pgfsys@defobject{currentmarker}{\pgfqpoint{-0.044444in}{-0.044444in}}{\pgfqpoint{0.044444in}{0.044444in}}{%
\pgfpathmoveto{\pgfqpoint{0.000000in}{-0.044444in}}%
\pgfpathcurveto{\pgfqpoint{0.011787in}{-0.044444in}}{\pgfqpoint{0.023092in}{-0.039761in}}{\pgfqpoint{0.031427in}{-0.031427in}}%
\pgfpathcurveto{\pgfqpoint{0.039761in}{-0.023092in}}{\pgfqpoint{0.044444in}{-0.011787in}}{\pgfqpoint{0.044444in}{0.000000in}}%
\pgfpathcurveto{\pgfqpoint{0.044444in}{0.011787in}}{\pgfqpoint{0.039761in}{0.023092in}}{\pgfqpoint{0.031427in}{0.031427in}}%
\pgfpathcurveto{\pgfqpoint{0.023092in}{0.039761in}}{\pgfqpoint{0.011787in}{0.044444in}}{\pgfqpoint{0.000000in}{0.044444in}}%
\pgfpathcurveto{\pgfqpoint{-0.011787in}{0.044444in}}{\pgfqpoint{-0.023092in}{0.039761in}}{\pgfqpoint{-0.031427in}{0.031427in}}%
\pgfpathcurveto{\pgfqpoint{-0.039761in}{0.023092in}}{\pgfqpoint{-0.044444in}{0.011787in}}{\pgfqpoint{-0.044444in}{0.000000in}}%
\pgfpathcurveto{\pgfqpoint{-0.044444in}{-0.011787in}}{\pgfqpoint{-0.039761in}{-0.023092in}}{\pgfqpoint{-0.031427in}{-0.031427in}}%
\pgfpathcurveto{\pgfqpoint{-0.023092in}{-0.039761in}}{\pgfqpoint{-0.011787in}{-0.044444in}}{\pgfqpoint{0.000000in}{-0.044444in}}%
\pgfpathclose%
\pgfusepath{stroke,fill}%
}%
\begin{pgfscope}%
\pgfsys@transformshift{3.998684in}{4.400590in}%
\pgfsys@useobject{currentmarker}{}%
\end{pgfscope}%
\end{pgfscope}%
\begin{pgfscope}%
\pgfpathrectangle{\pgfqpoint{0.100000in}{2.413063in}}{\pgfqpoint{5.037500in}{3.427208in}}%
\pgfusepath{clip}%
\pgfsetrectcap%
\pgfsetroundjoin%
\pgfsetlinewidth{1.505625pt}%
\definecolor{currentstroke}{rgb}{0.678431,1.000000,0.184314}%
\pgfsetstrokecolor{currentstroke}%
\pgfsetstrokeopacity{0.500000}%
\pgfsetdash{}{0pt}%
\pgfpathmoveto{\pgfqpoint{4.057487in}{4.502137in}}%
\pgfusepath{stroke}%
\end{pgfscope}%
\begin{pgfscope}%
\pgfpathrectangle{\pgfqpoint{0.100000in}{2.413063in}}{\pgfqpoint{5.037500in}{3.427208in}}%
\pgfusepath{clip}%
\pgfsetbuttcap%
\pgfsetroundjoin%
\definecolor{currentfill}{rgb}{0.678431,1.000000,0.184314}%
\pgfsetfillcolor{currentfill}%
\pgfsetfillopacity{0.500000}%
\pgfsetlinewidth{0.250937pt}%
\definecolor{currentstroke}{rgb}{0.000000,0.000000,0.000000}%
\pgfsetstrokecolor{currentstroke}%
\pgfsetstrokeopacity{0.500000}%
\pgfsetdash{}{0pt}%
\pgfsys@defobject{currentmarker}{\pgfqpoint{-0.061111in}{-0.061111in}}{\pgfqpoint{0.061111in}{0.061111in}}{%
\pgfpathmoveto{\pgfqpoint{0.000000in}{-0.061111in}}%
\pgfpathcurveto{\pgfqpoint{0.016207in}{-0.061111in}}{\pgfqpoint{0.031752in}{-0.054672in}}{\pgfqpoint{0.043212in}{-0.043212in}}%
\pgfpathcurveto{\pgfqpoint{0.054672in}{-0.031752in}}{\pgfqpoint{0.061111in}{-0.016207in}}{\pgfqpoint{0.061111in}{0.000000in}}%
\pgfpathcurveto{\pgfqpoint{0.061111in}{0.016207in}}{\pgfqpoint{0.054672in}{0.031752in}}{\pgfqpoint{0.043212in}{0.043212in}}%
\pgfpathcurveto{\pgfqpoint{0.031752in}{0.054672in}}{\pgfqpoint{0.016207in}{0.061111in}}{\pgfqpoint{0.000000in}{0.061111in}}%
\pgfpathcurveto{\pgfqpoint{-0.016207in}{0.061111in}}{\pgfqpoint{-0.031752in}{0.054672in}}{\pgfqpoint{-0.043212in}{0.043212in}}%
\pgfpathcurveto{\pgfqpoint{-0.054672in}{0.031752in}}{\pgfqpoint{-0.061111in}{0.016207in}}{\pgfqpoint{-0.061111in}{0.000000in}}%
\pgfpathcurveto{\pgfqpoint{-0.061111in}{-0.016207in}}{\pgfqpoint{-0.054672in}{-0.031752in}}{\pgfqpoint{-0.043212in}{-0.043212in}}%
\pgfpathcurveto{\pgfqpoint{-0.031752in}{-0.054672in}}{\pgfqpoint{-0.016207in}{-0.061111in}}{\pgfqpoint{0.000000in}{-0.061111in}}%
\pgfpathclose%
\pgfusepath{stroke,fill}%
}%
\begin{pgfscope}%
\pgfsys@transformshift{4.057487in}{4.502137in}%
\pgfsys@useobject{currentmarker}{}%
\end{pgfscope}%
\end{pgfscope}%
\begin{pgfscope}%
\pgfpathrectangle{\pgfqpoint{0.100000in}{2.413063in}}{\pgfqpoint{5.037500in}{3.427208in}}%
\pgfusepath{clip}%
\pgfsetrectcap%
\pgfsetroundjoin%
\pgfsetlinewidth{1.505625pt}%
\definecolor{currentstroke}{rgb}{0.678431,1.000000,0.184314}%
\pgfsetstrokecolor{currentstroke}%
\pgfsetstrokeopacity{0.500000}%
\pgfsetdash{}{0pt}%
\pgfpathmoveto{\pgfqpoint{3.349641in}{4.907379in}}%
\pgfusepath{stroke}%
\end{pgfscope}%
\begin{pgfscope}%
\pgfpathrectangle{\pgfqpoint{0.100000in}{2.413063in}}{\pgfqpoint{5.037500in}{3.427208in}}%
\pgfusepath{clip}%
\pgfsetbuttcap%
\pgfsetroundjoin%
\definecolor{currentfill}{rgb}{0.678431,1.000000,0.184314}%
\pgfsetfillcolor{currentfill}%
\pgfsetfillopacity{0.500000}%
\pgfsetlinewidth{0.250937pt}%
\definecolor{currentstroke}{rgb}{0.000000,0.000000,0.000000}%
\pgfsetstrokecolor{currentstroke}%
\pgfsetstrokeopacity{0.500000}%
\pgfsetdash{}{0pt}%
\pgfsys@defobject{currentmarker}{\pgfqpoint{-0.025000in}{-0.025000in}}{\pgfqpoint{0.025000in}{0.025000in}}{%
\pgfpathmoveto{\pgfqpoint{0.000000in}{-0.025000in}}%
\pgfpathcurveto{\pgfqpoint{0.006630in}{-0.025000in}}{\pgfqpoint{0.012989in}{-0.022366in}}{\pgfqpoint{0.017678in}{-0.017678in}}%
\pgfpathcurveto{\pgfqpoint{0.022366in}{-0.012989in}}{\pgfqpoint{0.025000in}{-0.006630in}}{\pgfqpoint{0.025000in}{0.000000in}}%
\pgfpathcurveto{\pgfqpoint{0.025000in}{0.006630in}}{\pgfqpoint{0.022366in}{0.012989in}}{\pgfqpoint{0.017678in}{0.017678in}}%
\pgfpathcurveto{\pgfqpoint{0.012989in}{0.022366in}}{\pgfqpoint{0.006630in}{0.025000in}}{\pgfqpoint{0.000000in}{0.025000in}}%
\pgfpathcurveto{\pgfqpoint{-0.006630in}{0.025000in}}{\pgfqpoint{-0.012989in}{0.022366in}}{\pgfqpoint{-0.017678in}{0.017678in}}%
\pgfpathcurveto{\pgfqpoint{-0.022366in}{0.012989in}}{\pgfqpoint{-0.025000in}{0.006630in}}{\pgfqpoint{-0.025000in}{0.000000in}}%
\pgfpathcurveto{\pgfqpoint{-0.025000in}{-0.006630in}}{\pgfqpoint{-0.022366in}{-0.012989in}}{\pgfqpoint{-0.017678in}{-0.017678in}}%
\pgfpathcurveto{\pgfqpoint{-0.012989in}{-0.022366in}}{\pgfqpoint{-0.006630in}{-0.025000in}}{\pgfqpoint{0.000000in}{-0.025000in}}%
\pgfpathclose%
\pgfusepath{stroke,fill}%
}%
\begin{pgfscope}%
\pgfsys@transformshift{3.349641in}{4.907379in}%
\pgfsys@useobject{currentmarker}{}%
\end{pgfscope}%
\end{pgfscope}%
\begin{pgfscope}%
\pgfpathrectangle{\pgfqpoint{0.100000in}{2.413063in}}{\pgfqpoint{5.037500in}{3.427208in}}%
\pgfusepath{clip}%
\pgfsetrectcap%
\pgfsetroundjoin%
\pgfsetlinewidth{1.505625pt}%
\definecolor{currentstroke}{rgb}{0.678431,1.000000,0.184314}%
\pgfsetstrokecolor{currentstroke}%
\pgfsetstrokeopacity{0.500000}%
\pgfsetdash{}{0pt}%
\pgfpathmoveto{\pgfqpoint{3.091567in}{4.956771in}}%
\pgfusepath{stroke}%
\end{pgfscope}%
\begin{pgfscope}%
\pgfpathrectangle{\pgfqpoint{0.100000in}{2.413063in}}{\pgfqpoint{5.037500in}{3.427208in}}%
\pgfusepath{clip}%
\pgfsetbuttcap%
\pgfsetroundjoin%
\definecolor{currentfill}{rgb}{0.678431,1.000000,0.184314}%
\pgfsetfillcolor{currentfill}%
\pgfsetfillopacity{0.500000}%
\pgfsetlinewidth{0.250937pt}%
\definecolor{currentstroke}{rgb}{0.000000,0.000000,0.000000}%
\pgfsetstrokecolor{currentstroke}%
\pgfsetstrokeopacity{0.500000}%
\pgfsetdash{}{0pt}%
\pgfsys@defobject{currentmarker}{\pgfqpoint{-0.036111in}{-0.036111in}}{\pgfqpoint{0.036111in}{0.036111in}}{%
\pgfpathmoveto{\pgfqpoint{0.000000in}{-0.036111in}}%
\pgfpathcurveto{\pgfqpoint{0.009577in}{-0.036111in}}{\pgfqpoint{0.018763in}{-0.032306in}}{\pgfqpoint{0.025534in}{-0.025534in}}%
\pgfpathcurveto{\pgfqpoint{0.032306in}{-0.018763in}}{\pgfqpoint{0.036111in}{-0.009577in}}{\pgfqpoint{0.036111in}{0.000000in}}%
\pgfpathcurveto{\pgfqpoint{0.036111in}{0.009577in}}{\pgfqpoint{0.032306in}{0.018763in}}{\pgfqpoint{0.025534in}{0.025534in}}%
\pgfpathcurveto{\pgfqpoint{0.018763in}{0.032306in}}{\pgfqpoint{0.009577in}{0.036111in}}{\pgfqpoint{0.000000in}{0.036111in}}%
\pgfpathcurveto{\pgfqpoint{-0.009577in}{0.036111in}}{\pgfqpoint{-0.018763in}{0.032306in}}{\pgfqpoint{-0.025534in}{0.025534in}}%
\pgfpathcurveto{\pgfqpoint{-0.032306in}{0.018763in}}{\pgfqpoint{-0.036111in}{0.009577in}}{\pgfqpoint{-0.036111in}{0.000000in}}%
\pgfpathcurveto{\pgfqpoint{-0.036111in}{-0.009577in}}{\pgfqpoint{-0.032306in}{-0.018763in}}{\pgfqpoint{-0.025534in}{-0.025534in}}%
\pgfpathcurveto{\pgfqpoint{-0.018763in}{-0.032306in}}{\pgfqpoint{-0.009577in}{-0.036111in}}{\pgfqpoint{0.000000in}{-0.036111in}}%
\pgfpathclose%
\pgfusepath{stroke,fill}%
}%
\begin{pgfscope}%
\pgfsys@transformshift{3.091567in}{4.956771in}%
\pgfsys@useobject{currentmarker}{}%
\end{pgfscope}%
\end{pgfscope}%
\begin{pgfscope}%
\pgfpathrectangle{\pgfqpoint{0.100000in}{2.413063in}}{\pgfqpoint{5.037500in}{3.427208in}}%
\pgfusepath{clip}%
\pgfsetrectcap%
\pgfsetroundjoin%
\pgfsetlinewidth{1.505625pt}%
\definecolor{currentstroke}{rgb}{0.678431,1.000000,0.184314}%
\pgfsetstrokecolor{currentstroke}%
\pgfsetstrokeopacity{0.500000}%
\pgfsetdash{}{0pt}%
\pgfpathmoveto{\pgfqpoint{3.350467in}{4.851087in}}%
\pgfusepath{stroke}%
\end{pgfscope}%
\begin{pgfscope}%
\pgfpathrectangle{\pgfqpoint{0.100000in}{2.413063in}}{\pgfqpoint{5.037500in}{3.427208in}}%
\pgfusepath{clip}%
\pgfsetbuttcap%
\pgfsetroundjoin%
\definecolor{currentfill}{rgb}{0.678431,1.000000,0.184314}%
\pgfsetfillcolor{currentfill}%
\pgfsetfillopacity{0.500000}%
\pgfsetlinewidth{0.250937pt}%
\definecolor{currentstroke}{rgb}{0.000000,0.000000,0.000000}%
\pgfsetstrokecolor{currentstroke}%
\pgfsetstrokeopacity{0.500000}%
\pgfsetdash{}{0pt}%
\pgfsys@defobject{currentmarker}{\pgfqpoint{-0.027778in}{-0.027778in}}{\pgfqpoint{0.027778in}{0.027778in}}{%
\pgfpathmoveto{\pgfqpoint{0.000000in}{-0.027778in}}%
\pgfpathcurveto{\pgfqpoint{0.007367in}{-0.027778in}}{\pgfqpoint{0.014433in}{-0.024851in}}{\pgfqpoint{0.019642in}{-0.019642in}}%
\pgfpathcurveto{\pgfqpoint{0.024851in}{-0.014433in}}{\pgfqpoint{0.027778in}{-0.007367in}}{\pgfqpoint{0.027778in}{0.000000in}}%
\pgfpathcurveto{\pgfqpoint{0.027778in}{0.007367in}}{\pgfqpoint{0.024851in}{0.014433in}}{\pgfqpoint{0.019642in}{0.019642in}}%
\pgfpathcurveto{\pgfqpoint{0.014433in}{0.024851in}}{\pgfqpoint{0.007367in}{0.027778in}}{\pgfqpoint{0.000000in}{0.027778in}}%
\pgfpathcurveto{\pgfqpoint{-0.007367in}{0.027778in}}{\pgfqpoint{-0.014433in}{0.024851in}}{\pgfqpoint{-0.019642in}{0.019642in}}%
\pgfpathcurveto{\pgfqpoint{-0.024851in}{0.014433in}}{\pgfqpoint{-0.027778in}{0.007367in}}{\pgfqpoint{-0.027778in}{0.000000in}}%
\pgfpathcurveto{\pgfqpoint{-0.027778in}{-0.007367in}}{\pgfqpoint{-0.024851in}{-0.014433in}}{\pgfqpoint{-0.019642in}{-0.019642in}}%
\pgfpathcurveto{\pgfqpoint{-0.014433in}{-0.024851in}}{\pgfqpoint{-0.007367in}{-0.027778in}}{\pgfqpoint{0.000000in}{-0.027778in}}%
\pgfpathclose%
\pgfusepath{stroke,fill}%
}%
\begin{pgfscope}%
\pgfsys@transformshift{3.350467in}{4.851087in}%
\pgfsys@useobject{currentmarker}{}%
\end{pgfscope}%
\end{pgfscope}%
\begin{pgfscope}%
\pgfpathrectangle{\pgfqpoint{0.100000in}{2.413063in}}{\pgfqpoint{5.037500in}{3.427208in}}%
\pgfusepath{clip}%
\pgfsetrectcap%
\pgfsetroundjoin%
\pgfsetlinewidth{1.505625pt}%
\definecolor{currentstroke}{rgb}{0.678431,1.000000,0.184314}%
\pgfsetstrokecolor{currentstroke}%
\pgfsetstrokeopacity{0.500000}%
\pgfsetdash{}{0pt}%
\pgfpathmoveto{\pgfqpoint{3.378797in}{4.940736in}}%
\pgfusepath{stroke}%
\end{pgfscope}%
\begin{pgfscope}%
\pgfpathrectangle{\pgfqpoint{0.100000in}{2.413063in}}{\pgfqpoint{5.037500in}{3.427208in}}%
\pgfusepath{clip}%
\pgfsetbuttcap%
\pgfsetroundjoin%
\definecolor{currentfill}{rgb}{0.678431,1.000000,0.184314}%
\pgfsetfillcolor{currentfill}%
\pgfsetfillopacity{0.500000}%
\pgfsetlinewidth{0.250937pt}%
\definecolor{currentstroke}{rgb}{0.000000,0.000000,0.000000}%
\pgfsetstrokecolor{currentstroke}%
\pgfsetstrokeopacity{0.500000}%
\pgfsetdash{}{0pt}%
\pgfsys@defobject{currentmarker}{\pgfqpoint{-0.027778in}{-0.027778in}}{\pgfqpoint{0.027778in}{0.027778in}}{%
\pgfpathmoveto{\pgfqpoint{0.000000in}{-0.027778in}}%
\pgfpathcurveto{\pgfqpoint{0.007367in}{-0.027778in}}{\pgfqpoint{0.014433in}{-0.024851in}}{\pgfqpoint{0.019642in}{-0.019642in}}%
\pgfpathcurveto{\pgfqpoint{0.024851in}{-0.014433in}}{\pgfqpoint{0.027778in}{-0.007367in}}{\pgfqpoint{0.027778in}{0.000000in}}%
\pgfpathcurveto{\pgfqpoint{0.027778in}{0.007367in}}{\pgfqpoint{0.024851in}{0.014433in}}{\pgfqpoint{0.019642in}{0.019642in}}%
\pgfpathcurveto{\pgfqpoint{0.014433in}{0.024851in}}{\pgfqpoint{0.007367in}{0.027778in}}{\pgfqpoint{0.000000in}{0.027778in}}%
\pgfpathcurveto{\pgfqpoint{-0.007367in}{0.027778in}}{\pgfqpoint{-0.014433in}{0.024851in}}{\pgfqpoint{-0.019642in}{0.019642in}}%
\pgfpathcurveto{\pgfqpoint{-0.024851in}{0.014433in}}{\pgfqpoint{-0.027778in}{0.007367in}}{\pgfqpoint{-0.027778in}{0.000000in}}%
\pgfpathcurveto{\pgfqpoint{-0.027778in}{-0.007367in}}{\pgfqpoint{-0.024851in}{-0.014433in}}{\pgfqpoint{-0.019642in}{-0.019642in}}%
\pgfpathcurveto{\pgfqpoint{-0.014433in}{-0.024851in}}{\pgfqpoint{-0.007367in}{-0.027778in}}{\pgfqpoint{0.000000in}{-0.027778in}}%
\pgfpathclose%
\pgfusepath{stroke,fill}%
}%
\begin{pgfscope}%
\pgfsys@transformshift{3.378797in}{4.940736in}%
\pgfsys@useobject{currentmarker}{}%
\end{pgfscope}%
\end{pgfscope}%
\begin{pgfscope}%
\pgfpathrectangle{\pgfqpoint{0.100000in}{2.413063in}}{\pgfqpoint{5.037500in}{3.427208in}}%
\pgfusepath{clip}%
\pgfsetrectcap%
\pgfsetroundjoin%
\pgfsetlinewidth{1.505625pt}%
\definecolor{currentstroke}{rgb}{0.678431,1.000000,0.184314}%
\pgfsetstrokecolor{currentstroke}%
\pgfsetstrokeopacity{0.500000}%
\pgfsetdash{}{0pt}%
\pgfpathmoveto{\pgfqpoint{3.313193in}{4.725863in}}%
\pgfusepath{stroke}%
\end{pgfscope}%
\begin{pgfscope}%
\pgfpathrectangle{\pgfqpoint{0.100000in}{2.413063in}}{\pgfqpoint{5.037500in}{3.427208in}}%
\pgfusepath{clip}%
\pgfsetbuttcap%
\pgfsetroundjoin%
\definecolor{currentfill}{rgb}{0.678431,1.000000,0.184314}%
\pgfsetfillcolor{currentfill}%
\pgfsetfillopacity{0.500000}%
\pgfsetlinewidth{0.250937pt}%
\definecolor{currentstroke}{rgb}{0.000000,0.000000,0.000000}%
\pgfsetstrokecolor{currentstroke}%
\pgfsetstrokeopacity{0.500000}%
\pgfsetdash{}{0pt}%
\pgfsys@defobject{currentmarker}{\pgfqpoint{-0.027778in}{-0.027778in}}{\pgfqpoint{0.027778in}{0.027778in}}{%
\pgfpathmoveto{\pgfqpoint{0.000000in}{-0.027778in}}%
\pgfpathcurveto{\pgfqpoint{0.007367in}{-0.027778in}}{\pgfqpoint{0.014433in}{-0.024851in}}{\pgfqpoint{0.019642in}{-0.019642in}}%
\pgfpathcurveto{\pgfqpoint{0.024851in}{-0.014433in}}{\pgfqpoint{0.027778in}{-0.007367in}}{\pgfqpoint{0.027778in}{0.000000in}}%
\pgfpathcurveto{\pgfqpoint{0.027778in}{0.007367in}}{\pgfqpoint{0.024851in}{0.014433in}}{\pgfqpoint{0.019642in}{0.019642in}}%
\pgfpathcurveto{\pgfqpoint{0.014433in}{0.024851in}}{\pgfqpoint{0.007367in}{0.027778in}}{\pgfqpoint{0.000000in}{0.027778in}}%
\pgfpathcurveto{\pgfqpoint{-0.007367in}{0.027778in}}{\pgfqpoint{-0.014433in}{0.024851in}}{\pgfqpoint{-0.019642in}{0.019642in}}%
\pgfpathcurveto{\pgfqpoint{-0.024851in}{0.014433in}}{\pgfqpoint{-0.027778in}{0.007367in}}{\pgfqpoint{-0.027778in}{0.000000in}}%
\pgfpathcurveto{\pgfqpoint{-0.027778in}{-0.007367in}}{\pgfqpoint{-0.024851in}{-0.014433in}}{\pgfqpoint{-0.019642in}{-0.019642in}}%
\pgfpathcurveto{\pgfqpoint{-0.014433in}{-0.024851in}}{\pgfqpoint{-0.007367in}{-0.027778in}}{\pgfqpoint{0.000000in}{-0.027778in}}%
\pgfpathclose%
\pgfusepath{stroke,fill}%
}%
\begin{pgfscope}%
\pgfsys@transformshift{3.313193in}{4.725863in}%
\pgfsys@useobject{currentmarker}{}%
\end{pgfscope}%
\end{pgfscope}%
\begin{pgfscope}%
\pgfpathrectangle{\pgfqpoint{0.100000in}{2.413063in}}{\pgfqpoint{5.037500in}{3.427208in}}%
\pgfusepath{clip}%
\pgfsetrectcap%
\pgfsetroundjoin%
\pgfsetlinewidth{1.505625pt}%
\definecolor{currentstroke}{rgb}{0.678431,1.000000,0.184314}%
\pgfsetstrokecolor{currentstroke}%
\pgfsetstrokeopacity{0.500000}%
\pgfsetdash{}{0pt}%
\pgfpathmoveto{\pgfqpoint{3.117641in}{4.840924in}}%
\pgfusepath{stroke}%
\end{pgfscope}%
\begin{pgfscope}%
\pgfpathrectangle{\pgfqpoint{0.100000in}{2.413063in}}{\pgfqpoint{5.037500in}{3.427208in}}%
\pgfusepath{clip}%
\pgfsetbuttcap%
\pgfsetroundjoin%
\definecolor{currentfill}{rgb}{0.678431,1.000000,0.184314}%
\pgfsetfillcolor{currentfill}%
\pgfsetfillopacity{0.500000}%
\pgfsetlinewidth{0.250937pt}%
\definecolor{currentstroke}{rgb}{0.000000,0.000000,0.000000}%
\pgfsetstrokecolor{currentstroke}%
\pgfsetstrokeopacity{0.500000}%
\pgfsetdash{}{0pt}%
\pgfsys@defobject{currentmarker}{\pgfqpoint{-0.027778in}{-0.027778in}}{\pgfqpoint{0.027778in}{0.027778in}}{%
\pgfpathmoveto{\pgfqpoint{0.000000in}{-0.027778in}}%
\pgfpathcurveto{\pgfqpoint{0.007367in}{-0.027778in}}{\pgfqpoint{0.014433in}{-0.024851in}}{\pgfqpoint{0.019642in}{-0.019642in}}%
\pgfpathcurveto{\pgfqpoint{0.024851in}{-0.014433in}}{\pgfqpoint{0.027778in}{-0.007367in}}{\pgfqpoint{0.027778in}{0.000000in}}%
\pgfpathcurveto{\pgfqpoint{0.027778in}{0.007367in}}{\pgfqpoint{0.024851in}{0.014433in}}{\pgfqpoint{0.019642in}{0.019642in}}%
\pgfpathcurveto{\pgfqpoint{0.014433in}{0.024851in}}{\pgfqpoint{0.007367in}{0.027778in}}{\pgfqpoint{0.000000in}{0.027778in}}%
\pgfpathcurveto{\pgfqpoint{-0.007367in}{0.027778in}}{\pgfqpoint{-0.014433in}{0.024851in}}{\pgfqpoint{-0.019642in}{0.019642in}}%
\pgfpathcurveto{\pgfqpoint{-0.024851in}{0.014433in}}{\pgfqpoint{-0.027778in}{0.007367in}}{\pgfqpoint{-0.027778in}{0.000000in}}%
\pgfpathcurveto{\pgfqpoint{-0.027778in}{-0.007367in}}{\pgfqpoint{-0.024851in}{-0.014433in}}{\pgfqpoint{-0.019642in}{-0.019642in}}%
\pgfpathcurveto{\pgfqpoint{-0.014433in}{-0.024851in}}{\pgfqpoint{-0.007367in}{-0.027778in}}{\pgfqpoint{0.000000in}{-0.027778in}}%
\pgfpathclose%
\pgfusepath{stroke,fill}%
}%
\begin{pgfscope}%
\pgfsys@transformshift{3.117641in}{4.840924in}%
\pgfsys@useobject{currentmarker}{}%
\end{pgfscope}%
\end{pgfscope}%
\begin{pgfscope}%
\pgfpathrectangle{\pgfqpoint{0.100000in}{2.413063in}}{\pgfqpoint{5.037500in}{3.427208in}}%
\pgfusepath{clip}%
\pgfsetrectcap%
\pgfsetroundjoin%
\pgfsetlinewidth{1.505625pt}%
\definecolor{currentstroke}{rgb}{0.678431,1.000000,0.184314}%
\pgfsetstrokecolor{currentstroke}%
\pgfsetstrokeopacity{0.500000}%
\pgfsetdash{}{0pt}%
\pgfpathmoveto{\pgfqpoint{3.277383in}{4.765152in}}%
\pgfusepath{stroke}%
\end{pgfscope}%
\begin{pgfscope}%
\pgfpathrectangle{\pgfqpoint{0.100000in}{2.413063in}}{\pgfqpoint{5.037500in}{3.427208in}}%
\pgfusepath{clip}%
\pgfsetbuttcap%
\pgfsetroundjoin%
\definecolor{currentfill}{rgb}{0.678431,1.000000,0.184314}%
\pgfsetfillcolor{currentfill}%
\pgfsetfillopacity{0.500000}%
\pgfsetlinewidth{0.250937pt}%
\definecolor{currentstroke}{rgb}{0.000000,0.000000,0.000000}%
\pgfsetstrokecolor{currentstroke}%
\pgfsetstrokeopacity{0.500000}%
\pgfsetdash{}{0pt}%
\pgfsys@defobject{currentmarker}{\pgfqpoint{-0.019444in}{-0.019444in}}{\pgfqpoint{0.019444in}{0.019444in}}{%
\pgfpathmoveto{\pgfqpoint{0.000000in}{-0.019444in}}%
\pgfpathcurveto{\pgfqpoint{0.005157in}{-0.019444in}}{\pgfqpoint{0.010103in}{-0.017396in}}{\pgfqpoint{0.013749in}{-0.013749in}}%
\pgfpathcurveto{\pgfqpoint{0.017396in}{-0.010103in}}{\pgfqpoint{0.019444in}{-0.005157in}}{\pgfqpoint{0.019444in}{0.000000in}}%
\pgfpathcurveto{\pgfqpoint{0.019444in}{0.005157in}}{\pgfqpoint{0.017396in}{0.010103in}}{\pgfqpoint{0.013749in}{0.013749in}}%
\pgfpathcurveto{\pgfqpoint{0.010103in}{0.017396in}}{\pgfqpoint{0.005157in}{0.019444in}}{\pgfqpoint{0.000000in}{0.019444in}}%
\pgfpathcurveto{\pgfqpoint{-0.005157in}{0.019444in}}{\pgfqpoint{-0.010103in}{0.017396in}}{\pgfqpoint{-0.013749in}{0.013749in}}%
\pgfpathcurveto{\pgfqpoint{-0.017396in}{0.010103in}}{\pgfqpoint{-0.019444in}{0.005157in}}{\pgfqpoint{-0.019444in}{0.000000in}}%
\pgfpathcurveto{\pgfqpoint{-0.019444in}{-0.005157in}}{\pgfqpoint{-0.017396in}{-0.010103in}}{\pgfqpoint{-0.013749in}{-0.013749in}}%
\pgfpathcurveto{\pgfqpoint{-0.010103in}{-0.017396in}}{\pgfqpoint{-0.005157in}{-0.019444in}}{\pgfqpoint{0.000000in}{-0.019444in}}%
\pgfpathclose%
\pgfusepath{stroke,fill}%
}%
\begin{pgfscope}%
\pgfsys@transformshift{3.277383in}{4.765152in}%
\pgfsys@useobject{currentmarker}{}%
\end{pgfscope}%
\end{pgfscope}%
\begin{pgfscope}%
\pgfpathrectangle{\pgfqpoint{0.100000in}{2.413063in}}{\pgfqpoint{5.037500in}{3.427208in}}%
\pgfusepath{clip}%
\pgfsetrectcap%
\pgfsetroundjoin%
\pgfsetlinewidth{1.505625pt}%
\definecolor{currentstroke}{rgb}{0.678431,1.000000,0.184314}%
\pgfsetstrokecolor{currentstroke}%
\pgfsetstrokeopacity{0.500000}%
\pgfsetdash{}{0pt}%
\pgfpathmoveto{\pgfqpoint{3.400651in}{4.769173in}}%
\pgfusepath{stroke}%
\end{pgfscope}%
\begin{pgfscope}%
\pgfpathrectangle{\pgfqpoint{0.100000in}{2.413063in}}{\pgfqpoint{5.037500in}{3.427208in}}%
\pgfusepath{clip}%
\pgfsetbuttcap%
\pgfsetroundjoin%
\definecolor{currentfill}{rgb}{0.678431,1.000000,0.184314}%
\pgfsetfillcolor{currentfill}%
\pgfsetfillopacity{0.500000}%
\pgfsetlinewidth{0.250937pt}%
\definecolor{currentstroke}{rgb}{0.000000,0.000000,0.000000}%
\pgfsetstrokecolor{currentstroke}%
\pgfsetstrokeopacity{0.500000}%
\pgfsetdash{}{0pt}%
\pgfsys@defobject{currentmarker}{\pgfqpoint{-0.019444in}{-0.019444in}}{\pgfqpoint{0.019444in}{0.019444in}}{%
\pgfpathmoveto{\pgfqpoint{0.000000in}{-0.019444in}}%
\pgfpathcurveto{\pgfqpoint{0.005157in}{-0.019444in}}{\pgfqpoint{0.010103in}{-0.017396in}}{\pgfqpoint{0.013749in}{-0.013749in}}%
\pgfpathcurveto{\pgfqpoint{0.017396in}{-0.010103in}}{\pgfqpoint{0.019444in}{-0.005157in}}{\pgfqpoint{0.019444in}{0.000000in}}%
\pgfpathcurveto{\pgfqpoint{0.019444in}{0.005157in}}{\pgfqpoint{0.017396in}{0.010103in}}{\pgfqpoint{0.013749in}{0.013749in}}%
\pgfpathcurveto{\pgfqpoint{0.010103in}{0.017396in}}{\pgfqpoint{0.005157in}{0.019444in}}{\pgfqpoint{0.000000in}{0.019444in}}%
\pgfpathcurveto{\pgfqpoint{-0.005157in}{0.019444in}}{\pgfqpoint{-0.010103in}{0.017396in}}{\pgfqpoint{-0.013749in}{0.013749in}}%
\pgfpathcurveto{\pgfqpoint{-0.017396in}{0.010103in}}{\pgfqpoint{-0.019444in}{0.005157in}}{\pgfqpoint{-0.019444in}{0.000000in}}%
\pgfpathcurveto{\pgfqpoint{-0.019444in}{-0.005157in}}{\pgfqpoint{-0.017396in}{-0.010103in}}{\pgfqpoint{-0.013749in}{-0.013749in}}%
\pgfpathcurveto{\pgfqpoint{-0.010103in}{-0.017396in}}{\pgfqpoint{-0.005157in}{-0.019444in}}{\pgfqpoint{0.000000in}{-0.019444in}}%
\pgfpathclose%
\pgfusepath{stroke,fill}%
}%
\begin{pgfscope}%
\pgfsys@transformshift{3.400651in}{4.769173in}%
\pgfsys@useobject{currentmarker}{}%
\end{pgfscope}%
\end{pgfscope}%
\begin{pgfscope}%
\pgfpathrectangle{\pgfqpoint{0.100000in}{2.413063in}}{\pgfqpoint{5.037500in}{3.427208in}}%
\pgfusepath{clip}%
\pgfsetrectcap%
\pgfsetroundjoin%
\pgfsetlinewidth{1.505625pt}%
\definecolor{currentstroke}{rgb}{0.678431,1.000000,0.184314}%
\pgfsetstrokecolor{currentstroke}%
\pgfsetstrokeopacity{0.500000}%
\pgfsetdash{}{0pt}%
\pgfpathmoveto{\pgfqpoint{3.340539in}{4.878855in}}%
\pgfusepath{stroke}%
\end{pgfscope}%
\begin{pgfscope}%
\pgfpathrectangle{\pgfqpoint{0.100000in}{2.413063in}}{\pgfqpoint{5.037500in}{3.427208in}}%
\pgfusepath{clip}%
\pgfsetbuttcap%
\pgfsetroundjoin%
\definecolor{currentfill}{rgb}{0.678431,1.000000,0.184314}%
\pgfsetfillcolor{currentfill}%
\pgfsetfillopacity{0.500000}%
\pgfsetlinewidth{0.250937pt}%
\definecolor{currentstroke}{rgb}{0.000000,0.000000,0.000000}%
\pgfsetstrokecolor{currentstroke}%
\pgfsetstrokeopacity{0.500000}%
\pgfsetdash{}{0pt}%
\pgfsys@defobject{currentmarker}{\pgfqpoint{-0.027778in}{-0.027778in}}{\pgfqpoint{0.027778in}{0.027778in}}{%
\pgfpathmoveto{\pgfqpoint{0.000000in}{-0.027778in}}%
\pgfpathcurveto{\pgfqpoint{0.007367in}{-0.027778in}}{\pgfqpoint{0.014433in}{-0.024851in}}{\pgfqpoint{0.019642in}{-0.019642in}}%
\pgfpathcurveto{\pgfqpoint{0.024851in}{-0.014433in}}{\pgfqpoint{0.027778in}{-0.007367in}}{\pgfqpoint{0.027778in}{0.000000in}}%
\pgfpathcurveto{\pgfqpoint{0.027778in}{0.007367in}}{\pgfqpoint{0.024851in}{0.014433in}}{\pgfqpoint{0.019642in}{0.019642in}}%
\pgfpathcurveto{\pgfqpoint{0.014433in}{0.024851in}}{\pgfqpoint{0.007367in}{0.027778in}}{\pgfqpoint{0.000000in}{0.027778in}}%
\pgfpathcurveto{\pgfqpoint{-0.007367in}{0.027778in}}{\pgfqpoint{-0.014433in}{0.024851in}}{\pgfqpoint{-0.019642in}{0.019642in}}%
\pgfpathcurveto{\pgfqpoint{-0.024851in}{0.014433in}}{\pgfqpoint{-0.027778in}{0.007367in}}{\pgfqpoint{-0.027778in}{0.000000in}}%
\pgfpathcurveto{\pgfqpoint{-0.027778in}{-0.007367in}}{\pgfqpoint{-0.024851in}{-0.014433in}}{\pgfqpoint{-0.019642in}{-0.019642in}}%
\pgfpathcurveto{\pgfqpoint{-0.014433in}{-0.024851in}}{\pgfqpoint{-0.007367in}{-0.027778in}}{\pgfqpoint{0.000000in}{-0.027778in}}%
\pgfpathclose%
\pgfusepath{stroke,fill}%
}%
\begin{pgfscope}%
\pgfsys@transformshift{3.340539in}{4.878855in}%
\pgfsys@useobject{currentmarker}{}%
\end{pgfscope}%
\end{pgfscope}%
\begin{pgfscope}%
\pgfpathrectangle{\pgfqpoint{0.100000in}{2.413063in}}{\pgfqpoint{5.037500in}{3.427208in}}%
\pgfusepath{clip}%
\pgfsetrectcap%
\pgfsetroundjoin%
\pgfsetlinewidth{1.505625pt}%
\definecolor{currentstroke}{rgb}{0.678431,1.000000,0.184314}%
\pgfsetstrokecolor{currentstroke}%
\pgfsetstrokeopacity{0.500000}%
\pgfsetdash{}{0pt}%
\pgfpathmoveto{\pgfqpoint{3.415252in}{4.734560in}}%
\pgfusepath{stroke}%
\end{pgfscope}%
\begin{pgfscope}%
\pgfpathrectangle{\pgfqpoint{0.100000in}{2.413063in}}{\pgfqpoint{5.037500in}{3.427208in}}%
\pgfusepath{clip}%
\pgfsetbuttcap%
\pgfsetroundjoin%
\definecolor{currentfill}{rgb}{0.678431,1.000000,0.184314}%
\pgfsetfillcolor{currentfill}%
\pgfsetfillopacity{0.500000}%
\pgfsetlinewidth{0.250937pt}%
\definecolor{currentstroke}{rgb}{0.000000,0.000000,0.000000}%
\pgfsetstrokecolor{currentstroke}%
\pgfsetstrokeopacity{0.500000}%
\pgfsetdash{}{0pt}%
\pgfsys@defobject{currentmarker}{\pgfqpoint{-0.030556in}{-0.030556in}}{\pgfqpoint{0.030556in}{0.030556in}}{%
\pgfpathmoveto{\pgfqpoint{0.000000in}{-0.030556in}}%
\pgfpathcurveto{\pgfqpoint{0.008103in}{-0.030556in}}{\pgfqpoint{0.015876in}{-0.027336in}}{\pgfqpoint{0.021606in}{-0.021606in}}%
\pgfpathcurveto{\pgfqpoint{0.027336in}{-0.015876in}}{\pgfqpoint{0.030556in}{-0.008103in}}{\pgfqpoint{0.030556in}{0.000000in}}%
\pgfpathcurveto{\pgfqpoint{0.030556in}{0.008103in}}{\pgfqpoint{0.027336in}{0.015876in}}{\pgfqpoint{0.021606in}{0.021606in}}%
\pgfpathcurveto{\pgfqpoint{0.015876in}{0.027336in}}{\pgfqpoint{0.008103in}{0.030556in}}{\pgfqpoint{0.000000in}{0.030556in}}%
\pgfpathcurveto{\pgfqpoint{-0.008103in}{0.030556in}}{\pgfqpoint{-0.015876in}{0.027336in}}{\pgfqpoint{-0.021606in}{0.021606in}}%
\pgfpathcurveto{\pgfqpoint{-0.027336in}{0.015876in}}{\pgfqpoint{-0.030556in}{0.008103in}}{\pgfqpoint{-0.030556in}{0.000000in}}%
\pgfpathcurveto{\pgfqpoint{-0.030556in}{-0.008103in}}{\pgfqpoint{-0.027336in}{-0.015876in}}{\pgfqpoint{-0.021606in}{-0.021606in}}%
\pgfpathcurveto{\pgfqpoint{-0.015876in}{-0.027336in}}{\pgfqpoint{-0.008103in}{-0.030556in}}{\pgfqpoint{0.000000in}{-0.030556in}}%
\pgfpathclose%
\pgfusepath{stroke,fill}%
}%
\begin{pgfscope}%
\pgfsys@transformshift{3.415252in}{4.734560in}%
\pgfsys@useobject{currentmarker}{}%
\end{pgfscope}%
\end{pgfscope}%
\begin{pgfscope}%
\pgfpathrectangle{\pgfqpoint{0.100000in}{2.413063in}}{\pgfqpoint{5.037500in}{3.427208in}}%
\pgfusepath{clip}%
\pgfsetrectcap%
\pgfsetroundjoin%
\pgfsetlinewidth{1.505625pt}%
\definecolor{currentstroke}{rgb}{0.678431,1.000000,0.184314}%
\pgfsetstrokecolor{currentstroke}%
\pgfsetstrokeopacity{0.500000}%
\pgfsetdash{}{0pt}%
\pgfpathmoveto{\pgfqpoint{3.411522in}{4.852958in}}%
\pgfusepath{stroke}%
\end{pgfscope}%
\begin{pgfscope}%
\pgfpathrectangle{\pgfqpoint{0.100000in}{2.413063in}}{\pgfqpoint{5.037500in}{3.427208in}}%
\pgfusepath{clip}%
\pgfsetbuttcap%
\pgfsetroundjoin%
\definecolor{currentfill}{rgb}{0.678431,1.000000,0.184314}%
\pgfsetfillcolor{currentfill}%
\pgfsetfillopacity{0.500000}%
\pgfsetlinewidth{0.250937pt}%
\definecolor{currentstroke}{rgb}{0.000000,0.000000,0.000000}%
\pgfsetstrokecolor{currentstroke}%
\pgfsetstrokeopacity{0.500000}%
\pgfsetdash{}{0pt}%
\pgfsys@defobject{currentmarker}{\pgfqpoint{-0.027778in}{-0.027778in}}{\pgfqpoint{0.027778in}{0.027778in}}{%
\pgfpathmoveto{\pgfqpoint{0.000000in}{-0.027778in}}%
\pgfpathcurveto{\pgfqpoint{0.007367in}{-0.027778in}}{\pgfqpoint{0.014433in}{-0.024851in}}{\pgfqpoint{0.019642in}{-0.019642in}}%
\pgfpathcurveto{\pgfqpoint{0.024851in}{-0.014433in}}{\pgfqpoint{0.027778in}{-0.007367in}}{\pgfqpoint{0.027778in}{0.000000in}}%
\pgfpathcurveto{\pgfqpoint{0.027778in}{0.007367in}}{\pgfqpoint{0.024851in}{0.014433in}}{\pgfqpoint{0.019642in}{0.019642in}}%
\pgfpathcurveto{\pgfqpoint{0.014433in}{0.024851in}}{\pgfqpoint{0.007367in}{0.027778in}}{\pgfqpoint{0.000000in}{0.027778in}}%
\pgfpathcurveto{\pgfqpoint{-0.007367in}{0.027778in}}{\pgfqpoint{-0.014433in}{0.024851in}}{\pgfqpoint{-0.019642in}{0.019642in}}%
\pgfpathcurveto{\pgfqpoint{-0.024851in}{0.014433in}}{\pgfqpoint{-0.027778in}{0.007367in}}{\pgfqpoint{-0.027778in}{0.000000in}}%
\pgfpathcurveto{\pgfqpoint{-0.027778in}{-0.007367in}}{\pgfqpoint{-0.024851in}{-0.014433in}}{\pgfqpoint{-0.019642in}{-0.019642in}}%
\pgfpathcurveto{\pgfqpoint{-0.014433in}{-0.024851in}}{\pgfqpoint{-0.007367in}{-0.027778in}}{\pgfqpoint{0.000000in}{-0.027778in}}%
\pgfpathclose%
\pgfusepath{stroke,fill}%
}%
\begin{pgfscope}%
\pgfsys@transformshift{3.411522in}{4.852958in}%
\pgfsys@useobject{currentmarker}{}%
\end{pgfscope}%
\end{pgfscope}%
\begin{pgfscope}%
\pgfpathrectangle{\pgfqpoint{0.100000in}{2.413063in}}{\pgfqpoint{5.037500in}{3.427208in}}%
\pgfusepath{clip}%
\pgfsetrectcap%
\pgfsetroundjoin%
\pgfsetlinewidth{1.505625pt}%
\definecolor{currentstroke}{rgb}{0.678431,1.000000,0.184314}%
\pgfsetstrokecolor{currentstroke}%
\pgfsetstrokeopacity{0.500000}%
\pgfsetdash{}{0pt}%
\pgfpathmoveto{\pgfqpoint{3.243806in}{4.981389in}}%
\pgfusepath{stroke}%
\end{pgfscope}%
\begin{pgfscope}%
\pgfpathrectangle{\pgfqpoint{0.100000in}{2.413063in}}{\pgfqpoint{5.037500in}{3.427208in}}%
\pgfusepath{clip}%
\pgfsetbuttcap%
\pgfsetroundjoin%
\definecolor{currentfill}{rgb}{0.678431,1.000000,0.184314}%
\pgfsetfillcolor{currentfill}%
\pgfsetfillopacity{0.500000}%
\pgfsetlinewidth{0.250937pt}%
\definecolor{currentstroke}{rgb}{0.000000,0.000000,0.000000}%
\pgfsetstrokecolor{currentstroke}%
\pgfsetstrokeopacity{0.500000}%
\pgfsetdash{}{0pt}%
\pgfsys@defobject{currentmarker}{\pgfqpoint{-0.030556in}{-0.030556in}}{\pgfqpoint{0.030556in}{0.030556in}}{%
\pgfpathmoveto{\pgfqpoint{0.000000in}{-0.030556in}}%
\pgfpathcurveto{\pgfqpoint{0.008103in}{-0.030556in}}{\pgfqpoint{0.015876in}{-0.027336in}}{\pgfqpoint{0.021606in}{-0.021606in}}%
\pgfpathcurveto{\pgfqpoint{0.027336in}{-0.015876in}}{\pgfqpoint{0.030556in}{-0.008103in}}{\pgfqpoint{0.030556in}{0.000000in}}%
\pgfpathcurveto{\pgfqpoint{0.030556in}{0.008103in}}{\pgfqpoint{0.027336in}{0.015876in}}{\pgfqpoint{0.021606in}{0.021606in}}%
\pgfpathcurveto{\pgfqpoint{0.015876in}{0.027336in}}{\pgfqpoint{0.008103in}{0.030556in}}{\pgfqpoint{0.000000in}{0.030556in}}%
\pgfpathcurveto{\pgfqpoint{-0.008103in}{0.030556in}}{\pgfqpoint{-0.015876in}{0.027336in}}{\pgfqpoint{-0.021606in}{0.021606in}}%
\pgfpathcurveto{\pgfqpoint{-0.027336in}{0.015876in}}{\pgfqpoint{-0.030556in}{0.008103in}}{\pgfqpoint{-0.030556in}{0.000000in}}%
\pgfpathcurveto{\pgfqpoint{-0.030556in}{-0.008103in}}{\pgfqpoint{-0.027336in}{-0.015876in}}{\pgfqpoint{-0.021606in}{-0.021606in}}%
\pgfpathcurveto{\pgfqpoint{-0.015876in}{-0.027336in}}{\pgfqpoint{-0.008103in}{-0.030556in}}{\pgfqpoint{0.000000in}{-0.030556in}}%
\pgfpathclose%
\pgfusepath{stroke,fill}%
}%
\begin{pgfscope}%
\pgfsys@transformshift{3.243806in}{4.981389in}%
\pgfsys@useobject{currentmarker}{}%
\end{pgfscope}%
\end{pgfscope}%
\begin{pgfscope}%
\pgfpathrectangle{\pgfqpoint{0.100000in}{2.413063in}}{\pgfqpoint{5.037500in}{3.427208in}}%
\pgfusepath{clip}%
\pgfsetrectcap%
\pgfsetroundjoin%
\pgfsetlinewidth{1.505625pt}%
\definecolor{currentstroke}{rgb}{0.678431,1.000000,0.184314}%
\pgfsetstrokecolor{currentstroke}%
\pgfsetstrokeopacity{0.500000}%
\pgfsetdash{}{0pt}%
\pgfpathmoveto{\pgfqpoint{1.849699in}{4.785986in}}%
\pgfusepath{stroke}%
\end{pgfscope}%
\begin{pgfscope}%
\pgfpathrectangle{\pgfqpoint{0.100000in}{2.413063in}}{\pgfqpoint{5.037500in}{3.427208in}}%
\pgfusepath{clip}%
\pgfsetbuttcap%
\pgfsetroundjoin%
\definecolor{currentfill}{rgb}{0.678431,1.000000,0.184314}%
\pgfsetfillcolor{currentfill}%
\pgfsetfillopacity{0.500000}%
\pgfsetlinewidth{0.250937pt}%
\definecolor{currentstroke}{rgb}{0.000000,0.000000,0.000000}%
\pgfsetstrokecolor{currentstroke}%
\pgfsetstrokeopacity{0.500000}%
\pgfsetdash{}{0pt}%
\pgfsys@defobject{currentmarker}{\pgfqpoint{-0.038889in}{-0.038889in}}{\pgfqpoint{0.038889in}{0.038889in}}{%
\pgfpathmoveto{\pgfqpoint{0.000000in}{-0.038889in}}%
\pgfpathcurveto{\pgfqpoint{0.010313in}{-0.038889in}}{\pgfqpoint{0.020206in}{-0.034791in}}{\pgfqpoint{0.027499in}{-0.027499in}}%
\pgfpathcurveto{\pgfqpoint{0.034791in}{-0.020206in}}{\pgfqpoint{0.038889in}{-0.010313in}}{\pgfqpoint{0.038889in}{0.000000in}}%
\pgfpathcurveto{\pgfqpoint{0.038889in}{0.010313in}}{\pgfqpoint{0.034791in}{0.020206in}}{\pgfqpoint{0.027499in}{0.027499in}}%
\pgfpathcurveto{\pgfqpoint{0.020206in}{0.034791in}}{\pgfqpoint{0.010313in}{0.038889in}}{\pgfqpoint{0.000000in}{0.038889in}}%
\pgfpathcurveto{\pgfqpoint{-0.010313in}{0.038889in}}{\pgfqpoint{-0.020206in}{0.034791in}}{\pgfqpoint{-0.027499in}{0.027499in}}%
\pgfpathcurveto{\pgfqpoint{-0.034791in}{0.020206in}}{\pgfqpoint{-0.038889in}{0.010313in}}{\pgfqpoint{-0.038889in}{0.000000in}}%
\pgfpathcurveto{\pgfqpoint{-0.038889in}{-0.010313in}}{\pgfqpoint{-0.034791in}{-0.020206in}}{\pgfqpoint{-0.027499in}{-0.027499in}}%
\pgfpathcurveto{\pgfqpoint{-0.020206in}{-0.034791in}}{\pgfqpoint{-0.010313in}{-0.038889in}}{\pgfqpoint{0.000000in}{-0.038889in}}%
\pgfpathclose%
\pgfusepath{stroke,fill}%
}%
\begin{pgfscope}%
\pgfsys@transformshift{1.849699in}{4.785986in}%
\pgfsys@useobject{currentmarker}{}%
\end{pgfscope}%
\end{pgfscope}%
\begin{pgfscope}%
\pgfpathrectangle{\pgfqpoint{0.100000in}{2.413063in}}{\pgfqpoint{5.037500in}{3.427208in}}%
\pgfusepath{clip}%
\pgfsetrectcap%
\pgfsetroundjoin%
\pgfsetlinewidth{1.505625pt}%
\definecolor{currentstroke}{rgb}{0.678431,1.000000,0.184314}%
\pgfsetstrokecolor{currentstroke}%
\pgfsetstrokeopacity{0.500000}%
\pgfsetdash{}{0pt}%
\pgfpathmoveto{\pgfqpoint{1.953642in}{4.573410in}}%
\pgfusepath{stroke}%
\end{pgfscope}%
\begin{pgfscope}%
\pgfpathrectangle{\pgfqpoint{0.100000in}{2.413063in}}{\pgfqpoint{5.037500in}{3.427208in}}%
\pgfusepath{clip}%
\pgfsetbuttcap%
\pgfsetroundjoin%
\definecolor{currentfill}{rgb}{0.678431,1.000000,0.184314}%
\pgfsetfillcolor{currentfill}%
\pgfsetfillopacity{0.500000}%
\pgfsetlinewidth{0.250937pt}%
\definecolor{currentstroke}{rgb}{0.000000,0.000000,0.000000}%
\pgfsetstrokecolor{currentstroke}%
\pgfsetstrokeopacity{0.500000}%
\pgfsetdash{}{0pt}%
\pgfsys@defobject{currentmarker}{\pgfqpoint{-0.044444in}{-0.044444in}}{\pgfqpoint{0.044444in}{0.044444in}}{%
\pgfpathmoveto{\pgfqpoint{0.000000in}{-0.044444in}}%
\pgfpathcurveto{\pgfqpoint{0.011787in}{-0.044444in}}{\pgfqpoint{0.023092in}{-0.039761in}}{\pgfqpoint{0.031427in}{-0.031427in}}%
\pgfpathcurveto{\pgfqpoint{0.039761in}{-0.023092in}}{\pgfqpoint{0.044444in}{-0.011787in}}{\pgfqpoint{0.044444in}{0.000000in}}%
\pgfpathcurveto{\pgfqpoint{0.044444in}{0.011787in}}{\pgfqpoint{0.039761in}{0.023092in}}{\pgfqpoint{0.031427in}{0.031427in}}%
\pgfpathcurveto{\pgfqpoint{0.023092in}{0.039761in}}{\pgfqpoint{0.011787in}{0.044444in}}{\pgfqpoint{0.000000in}{0.044444in}}%
\pgfpathcurveto{\pgfqpoint{-0.011787in}{0.044444in}}{\pgfqpoint{-0.023092in}{0.039761in}}{\pgfqpoint{-0.031427in}{0.031427in}}%
\pgfpathcurveto{\pgfqpoint{-0.039761in}{0.023092in}}{\pgfqpoint{-0.044444in}{0.011787in}}{\pgfqpoint{-0.044444in}{0.000000in}}%
\pgfpathcurveto{\pgfqpoint{-0.044444in}{-0.011787in}}{\pgfqpoint{-0.039761in}{-0.023092in}}{\pgfqpoint{-0.031427in}{-0.031427in}}%
\pgfpathcurveto{\pgfqpoint{-0.023092in}{-0.039761in}}{\pgfqpoint{-0.011787in}{-0.044444in}}{\pgfqpoint{0.000000in}{-0.044444in}}%
\pgfpathclose%
\pgfusepath{stroke,fill}%
}%
\begin{pgfscope}%
\pgfsys@transformshift{1.953642in}{4.573410in}%
\pgfsys@useobject{currentmarker}{}%
\end{pgfscope}%
\end{pgfscope}%
\begin{pgfscope}%
\pgfpathrectangle{\pgfqpoint{0.100000in}{2.413063in}}{\pgfqpoint{5.037500in}{3.427208in}}%
\pgfusepath{clip}%
\pgfsetrectcap%
\pgfsetroundjoin%
\pgfsetlinewidth{1.505625pt}%
\definecolor{currentstroke}{rgb}{0.501961,0.501961,0.501961}%
\pgfsetstrokecolor{currentstroke}%
\pgfsetstrokeopacity{0.500000}%
\pgfsetdash{}{0pt}%
\pgfpathmoveto{\pgfqpoint{2.769875in}{5.668910in}}%
\pgfusepath{stroke}%
\end{pgfscope}%
\begin{pgfscope}%
\pgfpathrectangle{\pgfqpoint{0.100000in}{2.413063in}}{\pgfqpoint{5.037500in}{3.427208in}}%
\pgfusepath{clip}%
\pgfsetbuttcap%
\pgfsetroundjoin%
\definecolor{currentfill}{rgb}{0.501961,0.501961,0.501961}%
\pgfsetfillcolor{currentfill}%
\pgfsetfillopacity{0.500000}%
\pgfsetlinewidth{0.250937pt}%
\definecolor{currentstroke}{rgb}{0.000000,0.000000,0.000000}%
\pgfsetstrokecolor{currentstroke}%
\pgfsetstrokeopacity{0.500000}%
\pgfsetdash{}{0pt}%
\pgfsys@defobject{currentmarker}{\pgfqpoint{-0.013889in}{-0.013889in}}{\pgfqpoint{0.013889in}{0.013889in}}{%
\pgfpathmoveto{\pgfqpoint{0.000000in}{-0.013889in}}%
\pgfpathcurveto{\pgfqpoint{0.003683in}{-0.013889in}}{\pgfqpoint{0.007216in}{-0.012425in}}{\pgfqpoint{0.009821in}{-0.009821in}}%
\pgfpathcurveto{\pgfqpoint{0.012425in}{-0.007216in}}{\pgfqpoint{0.013889in}{-0.003683in}}{\pgfqpoint{0.013889in}{0.000000in}}%
\pgfpathcurveto{\pgfqpoint{0.013889in}{0.003683in}}{\pgfqpoint{0.012425in}{0.007216in}}{\pgfqpoint{0.009821in}{0.009821in}}%
\pgfpathcurveto{\pgfqpoint{0.007216in}{0.012425in}}{\pgfqpoint{0.003683in}{0.013889in}}{\pgfqpoint{0.000000in}{0.013889in}}%
\pgfpathcurveto{\pgfqpoint{-0.003683in}{0.013889in}}{\pgfqpoint{-0.007216in}{0.012425in}}{\pgfqpoint{-0.009821in}{0.009821in}}%
\pgfpathcurveto{\pgfqpoint{-0.012425in}{0.007216in}}{\pgfqpoint{-0.013889in}{0.003683in}}{\pgfqpoint{-0.013889in}{0.000000in}}%
\pgfpathcurveto{\pgfqpoint{-0.013889in}{-0.003683in}}{\pgfqpoint{-0.012425in}{-0.007216in}}{\pgfqpoint{-0.009821in}{-0.009821in}}%
\pgfpathcurveto{\pgfqpoint{-0.007216in}{-0.012425in}}{\pgfqpoint{-0.003683in}{-0.013889in}}{\pgfqpoint{0.000000in}{-0.013889in}}%
\pgfpathclose%
\pgfusepath{stroke,fill}%
}%
\begin{pgfscope}%
\pgfsys@transformshift{2.769875in}{5.668910in}%
\pgfsys@useobject{currentmarker}{}%
\end{pgfscope}%
\end{pgfscope}%
\begin{pgfscope}%
\pgfpathrectangle{\pgfqpoint{0.100000in}{2.413063in}}{\pgfqpoint{5.037500in}{3.427208in}}%
\pgfusepath{clip}%
\pgfsetrectcap%
\pgfsetroundjoin%
\pgfsetlinewidth{1.505625pt}%
\definecolor{currentstroke}{rgb}{0.000000,0.000000,1.000000}%
\pgfsetstrokecolor{currentstroke}%
\pgfsetstrokeopacity{0.500000}%
\pgfsetdash{}{0pt}%
\pgfpathmoveto{\pgfqpoint{3.550687in}{5.668910in}}%
\pgfusepath{stroke}%
\end{pgfscope}%
\begin{pgfscope}%
\pgfpathrectangle{\pgfqpoint{0.100000in}{2.413063in}}{\pgfqpoint{5.037500in}{3.427208in}}%
\pgfusepath{clip}%
\pgfsetbuttcap%
\pgfsetroundjoin%
\definecolor{currentfill}{rgb}{0.000000,0.000000,1.000000}%
\pgfsetfillcolor{currentfill}%
\pgfsetfillopacity{0.500000}%
\pgfsetlinewidth{0.250937pt}%
\definecolor{currentstroke}{rgb}{0.000000,0.000000,0.000000}%
\pgfsetstrokecolor{currentstroke}%
\pgfsetstrokeopacity{0.500000}%
\pgfsetdash{}{0pt}%
\pgfsys@defobject{currentmarker}{\pgfqpoint{-0.083333in}{-0.083333in}}{\pgfqpoint{0.083333in}{0.083333in}}{%
\pgfpathmoveto{\pgfqpoint{0.000000in}{-0.083333in}}%
\pgfpathcurveto{\pgfqpoint{0.022100in}{-0.083333in}}{\pgfqpoint{0.043298in}{-0.074553in}}{\pgfqpoint{0.058926in}{-0.058926in}}%
\pgfpathcurveto{\pgfqpoint{0.074553in}{-0.043298in}}{\pgfqpoint{0.083333in}{-0.022100in}}{\pgfqpoint{0.083333in}{0.000000in}}%
\pgfpathcurveto{\pgfqpoint{0.083333in}{0.022100in}}{\pgfqpoint{0.074553in}{0.043298in}}{\pgfqpoint{0.058926in}{0.058926in}}%
\pgfpathcurveto{\pgfqpoint{0.043298in}{0.074553in}}{\pgfqpoint{0.022100in}{0.083333in}}{\pgfqpoint{0.000000in}{0.083333in}}%
\pgfpathcurveto{\pgfqpoint{-0.022100in}{0.083333in}}{\pgfqpoint{-0.043298in}{0.074553in}}{\pgfqpoint{-0.058926in}{0.058926in}}%
\pgfpathcurveto{\pgfqpoint{-0.074553in}{0.043298in}}{\pgfqpoint{-0.083333in}{0.022100in}}{\pgfqpoint{-0.083333in}{0.000000in}}%
\pgfpathcurveto{\pgfqpoint{-0.083333in}{-0.022100in}}{\pgfqpoint{-0.074553in}{-0.043298in}}{\pgfqpoint{-0.058926in}{-0.058926in}}%
\pgfpathcurveto{\pgfqpoint{-0.043298in}{-0.074553in}}{\pgfqpoint{-0.022100in}{-0.083333in}}{\pgfqpoint{0.000000in}{-0.083333in}}%
\pgfpathclose%
\pgfusepath{stroke,fill}%
}%
\begin{pgfscope}%
\pgfsys@transformshift{3.550687in}{5.668910in}%
\pgfsys@useobject{currentmarker}{}%
\end{pgfscope}%
\end{pgfscope}%
\begin{pgfscope}%
\pgfpathrectangle{\pgfqpoint{0.100000in}{2.413063in}}{\pgfqpoint{5.037500in}{3.427208in}}%
\pgfusepath{clip}%
\pgfsetrectcap%
\pgfsetroundjoin%
\pgfsetlinewidth{1.505625pt}%
\definecolor{currentstroke}{rgb}{0.000000,0.000000,1.000000}%
\pgfsetstrokecolor{currentstroke}%
\pgfsetstrokeopacity{0.500000}%
\pgfsetdash{}{0pt}%
\pgfpathmoveto{\pgfqpoint{3.550687in}{5.497550in}}%
\pgfusepath{stroke}%
\end{pgfscope}%
\begin{pgfscope}%
\pgfpathrectangle{\pgfqpoint{0.100000in}{2.413063in}}{\pgfqpoint{5.037500in}{3.427208in}}%
\pgfusepath{clip}%
\pgfsetbuttcap%
\pgfsetroundjoin%
\definecolor{currentfill}{rgb}{0.000000,0.000000,1.000000}%
\pgfsetfillcolor{currentfill}%
\pgfsetfillopacity{0.500000}%
\pgfsetlinewidth{0.250937pt}%
\definecolor{currentstroke}{rgb}{0.000000,0.000000,0.000000}%
\pgfsetstrokecolor{currentstroke}%
\pgfsetstrokeopacity{0.500000}%
\pgfsetdash{}{0pt}%
\pgfsys@defobject{currentmarker}{\pgfqpoint{-0.055556in}{-0.055556in}}{\pgfqpoint{0.055556in}{0.055556in}}{%
\pgfpathmoveto{\pgfqpoint{0.000000in}{-0.055556in}}%
\pgfpathcurveto{\pgfqpoint{0.014734in}{-0.055556in}}{\pgfqpoint{0.028866in}{-0.049702in}}{\pgfqpoint{0.039284in}{-0.039284in}}%
\pgfpathcurveto{\pgfqpoint{0.049702in}{-0.028866in}}{\pgfqpoint{0.055556in}{-0.014734in}}{\pgfqpoint{0.055556in}{0.000000in}}%
\pgfpathcurveto{\pgfqpoint{0.055556in}{0.014734in}}{\pgfqpoint{0.049702in}{0.028866in}}{\pgfqpoint{0.039284in}{0.039284in}}%
\pgfpathcurveto{\pgfqpoint{0.028866in}{0.049702in}}{\pgfqpoint{0.014734in}{0.055556in}}{\pgfqpoint{0.000000in}{0.055556in}}%
\pgfpathcurveto{\pgfqpoint{-0.014734in}{0.055556in}}{\pgfqpoint{-0.028866in}{0.049702in}}{\pgfqpoint{-0.039284in}{0.039284in}}%
\pgfpathcurveto{\pgfqpoint{-0.049702in}{0.028866in}}{\pgfqpoint{-0.055556in}{0.014734in}}{\pgfqpoint{-0.055556in}{0.000000in}}%
\pgfpathcurveto{\pgfqpoint{-0.055556in}{-0.014734in}}{\pgfqpoint{-0.049702in}{-0.028866in}}{\pgfqpoint{-0.039284in}{-0.039284in}}%
\pgfpathcurveto{\pgfqpoint{-0.028866in}{-0.049702in}}{\pgfqpoint{-0.014734in}{-0.055556in}}{\pgfqpoint{0.000000in}{-0.055556in}}%
\pgfpathclose%
\pgfusepath{stroke,fill}%
}%
\begin{pgfscope}%
\pgfsys@transformshift{3.550687in}{5.497550in}%
\pgfsys@useobject{currentmarker}{}%
\end{pgfscope}%
\end{pgfscope}%
\begin{pgfscope}%
\pgfpathrectangle{\pgfqpoint{0.100000in}{2.413063in}}{\pgfqpoint{5.037500in}{3.427208in}}%
\pgfusepath{clip}%
\pgfsetrectcap%
\pgfsetroundjoin%
\pgfsetlinewidth{1.505625pt}%
\definecolor{currentstroke}{rgb}{0.000000,0.000000,1.000000}%
\pgfsetstrokecolor{currentstroke}%
\pgfsetstrokeopacity{0.500000}%
\pgfsetdash{}{0pt}%
\pgfpathmoveto{\pgfqpoint{3.550687in}{5.360462in}}%
\pgfusepath{stroke}%
\end{pgfscope}%
\begin{pgfscope}%
\pgfpathrectangle{\pgfqpoint{0.100000in}{2.413063in}}{\pgfqpoint{5.037500in}{3.427208in}}%
\pgfusepath{clip}%
\pgfsetbuttcap%
\pgfsetroundjoin%
\definecolor{currentfill}{rgb}{0.000000,0.000000,1.000000}%
\pgfsetfillcolor{currentfill}%
\pgfsetfillopacity{0.500000}%
\pgfsetlinewidth{0.250937pt}%
\definecolor{currentstroke}{rgb}{0.000000,0.000000,0.000000}%
\pgfsetstrokecolor{currentstroke}%
\pgfsetstrokeopacity{0.500000}%
\pgfsetdash{}{0pt}%
\pgfsys@defobject{currentmarker}{\pgfqpoint{-0.027778in}{-0.027778in}}{\pgfqpoint{0.027778in}{0.027778in}}{%
\pgfpathmoveto{\pgfqpoint{0.000000in}{-0.027778in}}%
\pgfpathcurveto{\pgfqpoint{0.007367in}{-0.027778in}}{\pgfqpoint{0.014433in}{-0.024851in}}{\pgfqpoint{0.019642in}{-0.019642in}}%
\pgfpathcurveto{\pgfqpoint{0.024851in}{-0.014433in}}{\pgfqpoint{0.027778in}{-0.007367in}}{\pgfqpoint{0.027778in}{0.000000in}}%
\pgfpathcurveto{\pgfqpoint{0.027778in}{0.007367in}}{\pgfqpoint{0.024851in}{0.014433in}}{\pgfqpoint{0.019642in}{0.019642in}}%
\pgfpathcurveto{\pgfqpoint{0.014433in}{0.024851in}}{\pgfqpoint{0.007367in}{0.027778in}}{\pgfqpoint{0.000000in}{0.027778in}}%
\pgfpathcurveto{\pgfqpoint{-0.007367in}{0.027778in}}{\pgfqpoint{-0.014433in}{0.024851in}}{\pgfqpoint{-0.019642in}{0.019642in}}%
\pgfpathcurveto{\pgfqpoint{-0.024851in}{0.014433in}}{\pgfqpoint{-0.027778in}{0.007367in}}{\pgfqpoint{-0.027778in}{0.000000in}}%
\pgfpathcurveto{\pgfqpoint{-0.027778in}{-0.007367in}}{\pgfqpoint{-0.024851in}{-0.014433in}}{\pgfqpoint{-0.019642in}{-0.019642in}}%
\pgfpathcurveto{\pgfqpoint{-0.014433in}{-0.024851in}}{\pgfqpoint{-0.007367in}{-0.027778in}}{\pgfqpoint{0.000000in}{-0.027778in}}%
\pgfpathclose%
\pgfusepath{stroke,fill}%
}%
\begin{pgfscope}%
\pgfsys@transformshift{3.550687in}{5.360462in}%
\pgfsys@useobject{currentmarker}{}%
\end{pgfscope}%
\end{pgfscope}%
\begin{pgfscope}%
\pgfpathrectangle{\pgfqpoint{0.100000in}{2.413063in}}{\pgfqpoint{5.037500in}{3.427208in}}%
\pgfusepath{clip}%
\pgfsetrectcap%
\pgfsetroundjoin%
\pgfsetlinewidth{1.505625pt}%
\definecolor{currentstroke}{rgb}{0.678431,1.000000,0.184314}%
\pgfsetstrokecolor{currentstroke}%
\pgfsetstrokeopacity{0.500000}%
\pgfsetdash{}{0pt}%
\pgfpathmoveto{\pgfqpoint{4.205562in}{5.668910in}}%
\pgfusepath{stroke}%
\end{pgfscope}%
\begin{pgfscope}%
\pgfpathrectangle{\pgfqpoint{0.100000in}{2.413063in}}{\pgfqpoint{5.037500in}{3.427208in}}%
\pgfusepath{clip}%
\pgfsetbuttcap%
\pgfsetroundjoin%
\definecolor{currentfill}{rgb}{0.678431,1.000000,0.184314}%
\pgfsetfillcolor{currentfill}%
\pgfsetfillopacity{0.500000}%
\pgfsetlinewidth{0.250937pt}%
\definecolor{currentstroke}{rgb}{0.000000,0.000000,0.000000}%
\pgfsetstrokecolor{currentstroke}%
\pgfsetstrokeopacity{0.500000}%
\pgfsetdash{}{0pt}%
\pgfsys@defobject{currentmarker}{\pgfqpoint{-0.083333in}{-0.083333in}}{\pgfqpoint{0.083333in}{0.083333in}}{%
\pgfpathmoveto{\pgfqpoint{0.000000in}{-0.083333in}}%
\pgfpathcurveto{\pgfqpoint{0.022100in}{-0.083333in}}{\pgfqpoint{0.043298in}{-0.074553in}}{\pgfqpoint{0.058926in}{-0.058926in}}%
\pgfpathcurveto{\pgfqpoint{0.074553in}{-0.043298in}}{\pgfqpoint{0.083333in}{-0.022100in}}{\pgfqpoint{0.083333in}{0.000000in}}%
\pgfpathcurveto{\pgfqpoint{0.083333in}{0.022100in}}{\pgfqpoint{0.074553in}{0.043298in}}{\pgfqpoint{0.058926in}{0.058926in}}%
\pgfpathcurveto{\pgfqpoint{0.043298in}{0.074553in}}{\pgfqpoint{0.022100in}{0.083333in}}{\pgfqpoint{0.000000in}{0.083333in}}%
\pgfpathcurveto{\pgfqpoint{-0.022100in}{0.083333in}}{\pgfqpoint{-0.043298in}{0.074553in}}{\pgfqpoint{-0.058926in}{0.058926in}}%
\pgfpathcurveto{\pgfqpoint{-0.074553in}{0.043298in}}{\pgfqpoint{-0.083333in}{0.022100in}}{\pgfqpoint{-0.083333in}{0.000000in}}%
\pgfpathcurveto{\pgfqpoint{-0.083333in}{-0.022100in}}{\pgfqpoint{-0.074553in}{-0.043298in}}{\pgfqpoint{-0.058926in}{-0.058926in}}%
\pgfpathcurveto{\pgfqpoint{-0.043298in}{-0.074553in}}{\pgfqpoint{-0.022100in}{-0.083333in}}{\pgfqpoint{0.000000in}{-0.083333in}}%
\pgfpathclose%
\pgfusepath{stroke,fill}%
}%
\begin{pgfscope}%
\pgfsys@transformshift{4.205562in}{5.668910in}%
\pgfsys@useobject{currentmarker}{}%
\end{pgfscope}%
\end{pgfscope}%
\begin{pgfscope}%
\pgfpathrectangle{\pgfqpoint{0.100000in}{2.413063in}}{\pgfqpoint{5.037500in}{3.427208in}}%
\pgfusepath{clip}%
\pgfsetrectcap%
\pgfsetroundjoin%
\pgfsetlinewidth{1.505625pt}%
\definecolor{currentstroke}{rgb}{0.678431,1.000000,0.184314}%
\pgfsetstrokecolor{currentstroke}%
\pgfsetstrokeopacity{0.500000}%
\pgfsetdash{}{0pt}%
\pgfpathmoveto{\pgfqpoint{4.205562in}{5.497550in}}%
\pgfusepath{stroke}%
\end{pgfscope}%
\begin{pgfscope}%
\pgfpathrectangle{\pgfqpoint{0.100000in}{2.413063in}}{\pgfqpoint{5.037500in}{3.427208in}}%
\pgfusepath{clip}%
\pgfsetbuttcap%
\pgfsetroundjoin%
\definecolor{currentfill}{rgb}{0.678431,1.000000,0.184314}%
\pgfsetfillcolor{currentfill}%
\pgfsetfillopacity{0.500000}%
\pgfsetlinewidth{0.250937pt}%
\definecolor{currentstroke}{rgb}{0.000000,0.000000,0.000000}%
\pgfsetstrokecolor{currentstroke}%
\pgfsetstrokeopacity{0.500000}%
\pgfsetdash{}{0pt}%
\pgfsys@defobject{currentmarker}{\pgfqpoint{-0.055556in}{-0.055556in}}{\pgfqpoint{0.055556in}{0.055556in}}{%
\pgfpathmoveto{\pgfqpoint{0.000000in}{-0.055556in}}%
\pgfpathcurveto{\pgfqpoint{0.014734in}{-0.055556in}}{\pgfqpoint{0.028866in}{-0.049702in}}{\pgfqpoint{0.039284in}{-0.039284in}}%
\pgfpathcurveto{\pgfqpoint{0.049702in}{-0.028866in}}{\pgfqpoint{0.055556in}{-0.014734in}}{\pgfqpoint{0.055556in}{0.000000in}}%
\pgfpathcurveto{\pgfqpoint{0.055556in}{0.014734in}}{\pgfqpoint{0.049702in}{0.028866in}}{\pgfqpoint{0.039284in}{0.039284in}}%
\pgfpathcurveto{\pgfqpoint{0.028866in}{0.049702in}}{\pgfqpoint{0.014734in}{0.055556in}}{\pgfqpoint{0.000000in}{0.055556in}}%
\pgfpathcurveto{\pgfqpoint{-0.014734in}{0.055556in}}{\pgfqpoint{-0.028866in}{0.049702in}}{\pgfqpoint{-0.039284in}{0.039284in}}%
\pgfpathcurveto{\pgfqpoint{-0.049702in}{0.028866in}}{\pgfqpoint{-0.055556in}{0.014734in}}{\pgfqpoint{-0.055556in}{0.000000in}}%
\pgfpathcurveto{\pgfqpoint{-0.055556in}{-0.014734in}}{\pgfqpoint{-0.049702in}{-0.028866in}}{\pgfqpoint{-0.039284in}{-0.039284in}}%
\pgfpathcurveto{\pgfqpoint{-0.028866in}{-0.049702in}}{\pgfqpoint{-0.014734in}{-0.055556in}}{\pgfqpoint{0.000000in}{-0.055556in}}%
\pgfpathclose%
\pgfusepath{stroke,fill}%
}%
\begin{pgfscope}%
\pgfsys@transformshift{4.205562in}{5.497550in}%
\pgfsys@useobject{currentmarker}{}%
\end{pgfscope}%
\end{pgfscope}%
\begin{pgfscope}%
\pgfpathrectangle{\pgfqpoint{0.100000in}{2.413063in}}{\pgfqpoint{5.037500in}{3.427208in}}%
\pgfusepath{clip}%
\pgfsetrectcap%
\pgfsetroundjoin%
\pgfsetlinewidth{1.505625pt}%
\definecolor{currentstroke}{rgb}{0.678431,1.000000,0.184314}%
\pgfsetstrokecolor{currentstroke}%
\pgfsetstrokeopacity{0.500000}%
\pgfsetdash{}{0pt}%
\pgfpathmoveto{\pgfqpoint{4.205562in}{5.360462in}}%
\pgfusepath{stroke}%
\end{pgfscope}%
\begin{pgfscope}%
\pgfpathrectangle{\pgfqpoint{0.100000in}{2.413063in}}{\pgfqpoint{5.037500in}{3.427208in}}%
\pgfusepath{clip}%
\pgfsetbuttcap%
\pgfsetroundjoin%
\definecolor{currentfill}{rgb}{0.678431,1.000000,0.184314}%
\pgfsetfillcolor{currentfill}%
\pgfsetfillopacity{0.500000}%
\pgfsetlinewidth{0.250937pt}%
\definecolor{currentstroke}{rgb}{0.000000,0.000000,0.000000}%
\pgfsetstrokecolor{currentstroke}%
\pgfsetstrokeopacity{0.500000}%
\pgfsetdash{}{0pt}%
\pgfsys@defobject{currentmarker}{\pgfqpoint{-0.027778in}{-0.027778in}}{\pgfqpoint{0.027778in}{0.027778in}}{%
\pgfpathmoveto{\pgfqpoint{0.000000in}{-0.027778in}}%
\pgfpathcurveto{\pgfqpoint{0.007367in}{-0.027778in}}{\pgfqpoint{0.014433in}{-0.024851in}}{\pgfqpoint{0.019642in}{-0.019642in}}%
\pgfpathcurveto{\pgfqpoint{0.024851in}{-0.014433in}}{\pgfqpoint{0.027778in}{-0.007367in}}{\pgfqpoint{0.027778in}{0.000000in}}%
\pgfpathcurveto{\pgfqpoint{0.027778in}{0.007367in}}{\pgfqpoint{0.024851in}{0.014433in}}{\pgfqpoint{0.019642in}{0.019642in}}%
\pgfpathcurveto{\pgfqpoint{0.014433in}{0.024851in}}{\pgfqpoint{0.007367in}{0.027778in}}{\pgfqpoint{0.000000in}{0.027778in}}%
\pgfpathcurveto{\pgfqpoint{-0.007367in}{0.027778in}}{\pgfqpoint{-0.014433in}{0.024851in}}{\pgfqpoint{-0.019642in}{0.019642in}}%
\pgfpathcurveto{\pgfqpoint{-0.024851in}{0.014433in}}{\pgfqpoint{-0.027778in}{0.007367in}}{\pgfqpoint{-0.027778in}{0.000000in}}%
\pgfpathcurveto{\pgfqpoint{-0.027778in}{-0.007367in}}{\pgfqpoint{-0.024851in}{-0.014433in}}{\pgfqpoint{-0.019642in}{-0.019642in}}%
\pgfpathcurveto{\pgfqpoint{-0.014433in}{-0.024851in}}{\pgfqpoint{-0.007367in}{-0.027778in}}{\pgfqpoint{0.000000in}{-0.027778in}}%
\pgfpathclose%
\pgfusepath{stroke,fill}%
}%
\begin{pgfscope}%
\pgfsys@transformshift{4.205562in}{5.360462in}%
\pgfsys@useobject{currentmarker}{}%
\end{pgfscope}%
\end{pgfscope}%
\begin{pgfscope}%
\definecolor{textcolor}{rgb}{0.000000,0.000000,0.000000}%
\pgfsetstrokecolor{textcolor}%
\pgfsetfillcolor{textcolor}%
\pgftext[x=2.895812in,y=5.636352in,left,base]{\color{textcolor}\setmainfont{Lato}\rmfamily\fontsize{7.000000}{8.400000}\selectfont +/- 0.2pp}%
\end{pgfscope}%
\begin{pgfscope}%
\definecolor{textcolor}{rgb}{0.000000,0.000000,0.000000}%
\pgfsetstrokecolor{textcolor}%
\pgfsetfillcolor{textcolor}%
\pgftext[x=3.676625in,y=5.636352in,left,base]{\color{textcolor}\setmainfont{Lato}\rmfamily\fontsize{7.000000}{8.400000}\selectfont +3.0pp}%
\end{pgfscope}%
\begin{pgfscope}%
\definecolor{textcolor}{rgb}{0.000000,0.000000,0.000000}%
\pgfsetstrokecolor{textcolor}%
\pgfsetfillcolor{textcolor}%
\pgftext[x=3.676625in,y=5.483670in,left,base]{\color{textcolor}\setmainfont{Lato}\rmfamily\fontsize{7.000000}{8.400000}\selectfont +2.0pp}%
\end{pgfscope}%
\begin{pgfscope}%
\definecolor{textcolor}{rgb}{0.000000,0.000000,0.000000}%
\pgfsetstrokecolor{textcolor}%
\pgfsetfillcolor{textcolor}%
\pgftext[x=3.676625in,y=5.330988in,left,base]{\color{textcolor}\setmainfont{Lato}\rmfamily\fontsize{7.000000}{8.400000}\selectfont +1.0pp}%
\end{pgfscope}%
\begin{pgfscope}%
\definecolor{textcolor}{rgb}{0.000000,0.000000,0.000000}%
\pgfsetstrokecolor{textcolor}%
\pgfsetfillcolor{textcolor}%
\pgftext[x=4.331500in,y=5.636352in,left,base]{\color{textcolor}\setmainfont{Lato}\rmfamily\fontsize{7.000000}{8.400000}\selectfont -3.0pp}%
\end{pgfscope}%
\begin{pgfscope}%
\definecolor{textcolor}{rgb}{0.000000,0.000000,0.000000}%
\pgfsetstrokecolor{textcolor}%
\pgfsetfillcolor{textcolor}%
\pgftext[x=4.331500in,y=5.483670in,left,base]{\color{textcolor}\setmainfont{Lato}\rmfamily\fontsize{7.000000}{8.400000}\selectfont -2.0pp}%
\end{pgfscope}%
\begin{pgfscope}%
\definecolor{textcolor}{rgb}{0.000000,0.000000,0.000000}%
\pgfsetstrokecolor{textcolor}%
\pgfsetfillcolor{textcolor}%
\pgftext[x=4.331500in,y=5.330988in,left,base]{\color{textcolor}\setmainfont{Lato}\rmfamily\fontsize{7.000000}{8.400000}\selectfont -1.0pp}%
\end{pgfscope}%
\begin{pgfscope}%
\pgfpathrectangle{\pgfqpoint{0.100000in}{0.100000in}}{\pgfqpoint{5.037500in}{2.013333in}}%
\pgfusepath{clip}%
\pgfsetrectcap%
\pgfsetroundjoin%
\pgfsetlinewidth{1.505625pt}%
\definecolor{currentstroke}{rgb}{0.000000,0.000000,1.000000}%
\pgfsetstrokecolor{currentstroke}%
\pgfsetstrokeopacity{0.500000}%
\pgfsetdash{}{0pt}%
\pgfpathmoveto{\pgfqpoint{0.780062in}{2.012667in}}%
\pgfusepath{stroke}%
\end{pgfscope}%
\begin{pgfscope}%
\pgfpathrectangle{\pgfqpoint{0.100000in}{0.100000in}}{\pgfqpoint{5.037500in}{2.013333in}}%
\pgfusepath{clip}%
\pgfsetbuttcap%
\pgfsetroundjoin%
\definecolor{currentfill}{rgb}{0.000000,0.000000,1.000000}%
\pgfsetfillcolor{currentfill}%
\pgfsetfillopacity{0.500000}%
\pgfsetlinewidth{0.250937pt}%
\definecolor{currentstroke}{rgb}{0.000000,0.000000,0.000000}%
\pgfsetstrokecolor{currentstroke}%
\pgfsetstrokeopacity{0.500000}%
\pgfsetdash{}{0pt}%
\pgfsys@defobject{currentmarker}{\pgfqpoint{-0.066667in}{-0.066667in}}{\pgfqpoint{0.066667in}{0.066667in}}{%
\pgfpathmoveto{\pgfqpoint{0.000000in}{-0.066667in}}%
\pgfpathcurveto{\pgfqpoint{0.017680in}{-0.066667in}}{\pgfqpoint{0.034639in}{-0.059642in}}{\pgfqpoint{0.047140in}{-0.047140in}}%
\pgfpathcurveto{\pgfqpoint{0.059642in}{-0.034639in}}{\pgfqpoint{0.066667in}{-0.017680in}}{\pgfqpoint{0.066667in}{0.000000in}}%
\pgfpathcurveto{\pgfqpoint{0.066667in}{0.017680in}}{\pgfqpoint{0.059642in}{0.034639in}}{\pgfqpoint{0.047140in}{0.047140in}}%
\pgfpathcurveto{\pgfqpoint{0.034639in}{0.059642in}}{\pgfqpoint{0.017680in}{0.066667in}}{\pgfqpoint{0.000000in}{0.066667in}}%
\pgfpathcurveto{\pgfqpoint{-0.017680in}{0.066667in}}{\pgfqpoint{-0.034639in}{0.059642in}}{\pgfqpoint{-0.047140in}{0.047140in}}%
\pgfpathcurveto{\pgfqpoint{-0.059642in}{0.034639in}}{\pgfqpoint{-0.066667in}{0.017680in}}{\pgfqpoint{-0.066667in}{0.000000in}}%
\pgfpathcurveto{\pgfqpoint{-0.066667in}{-0.017680in}}{\pgfqpoint{-0.059642in}{-0.034639in}}{\pgfqpoint{-0.047140in}{-0.047140in}}%
\pgfpathcurveto{\pgfqpoint{-0.034639in}{-0.059642in}}{\pgfqpoint{-0.017680in}{-0.066667in}}{\pgfqpoint{0.000000in}{-0.066667in}}%
\pgfpathclose%
\pgfusepath{stroke,fill}%
}%
\begin{pgfscope}%
\pgfsys@transformshift{0.780062in}{2.012667in}%
\pgfsys@useobject{currentmarker}{}%
\end{pgfscope}%
\end{pgfscope}%
\begin{pgfscope}%
\pgfpathrectangle{\pgfqpoint{0.100000in}{0.100000in}}{\pgfqpoint{5.037500in}{2.013333in}}%
\pgfusepath{clip}%
\pgfsetrectcap%
\pgfsetroundjoin%
\pgfsetlinewidth{0.250937pt}%
\definecolor{currentstroke}{rgb}{0.862745,0.862745,0.862745}%
\pgfsetstrokecolor{currentstroke}%
\pgfsetdash{}{0pt}%
\pgfpathmoveto{\pgfqpoint{0.906000in}{1.932133in}}%
\pgfpathlineto{\pgfqpoint{4.180375in}{1.932133in}}%
\pgfusepath{stroke}%
\end{pgfscope}%
\begin{pgfscope}%
\pgfpathrectangle{\pgfqpoint{0.100000in}{0.100000in}}{\pgfqpoint{5.037500in}{2.013333in}}%
\pgfusepath{clip}%
\pgfsetrectcap%
\pgfsetroundjoin%
\pgfsetlinewidth{1.505625pt}%
\definecolor{currentstroke}{rgb}{0.000000,0.000000,1.000000}%
\pgfsetstrokecolor{currentstroke}%
\pgfsetstrokeopacity{0.500000}%
\pgfsetdash{}{0pt}%
\pgfpathmoveto{\pgfqpoint{0.780062in}{1.829636in}}%
\pgfusepath{stroke}%
\end{pgfscope}%
\begin{pgfscope}%
\pgfpathrectangle{\pgfqpoint{0.100000in}{0.100000in}}{\pgfqpoint{5.037500in}{2.013333in}}%
\pgfusepath{clip}%
\pgfsetbuttcap%
\pgfsetroundjoin%
\definecolor{currentfill}{rgb}{0.000000,0.000000,1.000000}%
\pgfsetfillcolor{currentfill}%
\pgfsetfillopacity{0.500000}%
\pgfsetlinewidth{0.250937pt}%
\definecolor{currentstroke}{rgb}{0.000000,0.000000,0.000000}%
\pgfsetstrokecolor{currentstroke}%
\pgfsetstrokeopacity{0.500000}%
\pgfsetdash{}{0pt}%
\pgfsys@defobject{currentmarker}{\pgfqpoint{-0.047222in}{-0.047222in}}{\pgfqpoint{0.047222in}{0.047222in}}{%
\pgfpathmoveto{\pgfqpoint{0.000000in}{-0.047222in}}%
\pgfpathcurveto{\pgfqpoint{0.012523in}{-0.047222in}}{\pgfqpoint{0.024536in}{-0.042247in}}{\pgfqpoint{0.033391in}{-0.033391in}}%
\pgfpathcurveto{\pgfqpoint{0.042247in}{-0.024536in}}{\pgfqpoint{0.047222in}{-0.012523in}}{\pgfqpoint{0.047222in}{0.000000in}}%
\pgfpathcurveto{\pgfqpoint{0.047222in}{0.012523in}}{\pgfqpoint{0.042247in}{0.024536in}}{\pgfqpoint{0.033391in}{0.033391in}}%
\pgfpathcurveto{\pgfqpoint{0.024536in}{0.042247in}}{\pgfqpoint{0.012523in}{0.047222in}}{\pgfqpoint{0.000000in}{0.047222in}}%
\pgfpathcurveto{\pgfqpoint{-0.012523in}{0.047222in}}{\pgfqpoint{-0.024536in}{0.042247in}}{\pgfqpoint{-0.033391in}{0.033391in}}%
\pgfpathcurveto{\pgfqpoint{-0.042247in}{0.024536in}}{\pgfqpoint{-0.047222in}{0.012523in}}{\pgfqpoint{-0.047222in}{0.000000in}}%
\pgfpathcurveto{\pgfqpoint{-0.047222in}{-0.012523in}}{\pgfqpoint{-0.042247in}{-0.024536in}}{\pgfqpoint{-0.033391in}{-0.033391in}}%
\pgfpathcurveto{\pgfqpoint{-0.024536in}{-0.042247in}}{\pgfqpoint{-0.012523in}{-0.047222in}}{\pgfqpoint{0.000000in}{-0.047222in}}%
\pgfpathclose%
\pgfusepath{stroke,fill}%
}%
\begin{pgfscope}%
\pgfsys@transformshift{0.780062in}{1.829636in}%
\pgfsys@useobject{currentmarker}{}%
\end{pgfscope}%
\end{pgfscope}%
\begin{pgfscope}%
\pgfpathrectangle{\pgfqpoint{0.100000in}{0.100000in}}{\pgfqpoint{5.037500in}{2.013333in}}%
\pgfusepath{clip}%
\pgfsetrectcap%
\pgfsetroundjoin%
\pgfsetlinewidth{0.250937pt}%
\definecolor{currentstroke}{rgb}{0.862745,0.862745,0.862745}%
\pgfsetstrokecolor{currentstroke}%
\pgfsetdash{}{0pt}%
\pgfpathmoveto{\pgfqpoint{0.906000in}{1.749103in}}%
\pgfpathlineto{\pgfqpoint{4.180375in}{1.749103in}}%
\pgfusepath{stroke}%
\end{pgfscope}%
\begin{pgfscope}%
\pgfpathrectangle{\pgfqpoint{0.100000in}{0.100000in}}{\pgfqpoint{5.037500in}{2.013333in}}%
\pgfusepath{clip}%
\pgfsetrectcap%
\pgfsetroundjoin%
\pgfsetlinewidth{1.505625pt}%
\definecolor{currentstroke}{rgb}{0.000000,0.000000,1.000000}%
\pgfsetstrokecolor{currentstroke}%
\pgfsetstrokeopacity{0.500000}%
\pgfsetdash{}{0pt}%
\pgfpathmoveto{\pgfqpoint{0.780062in}{1.646606in}}%
\pgfusepath{stroke}%
\end{pgfscope}%
\begin{pgfscope}%
\pgfpathrectangle{\pgfqpoint{0.100000in}{0.100000in}}{\pgfqpoint{5.037500in}{2.013333in}}%
\pgfusepath{clip}%
\pgfsetbuttcap%
\pgfsetroundjoin%
\definecolor{currentfill}{rgb}{0.000000,0.000000,1.000000}%
\pgfsetfillcolor{currentfill}%
\pgfsetfillopacity{0.500000}%
\pgfsetlinewidth{0.250937pt}%
\definecolor{currentstroke}{rgb}{0.000000,0.000000,0.000000}%
\pgfsetstrokecolor{currentstroke}%
\pgfsetstrokeopacity{0.500000}%
\pgfsetdash{}{0pt}%
\pgfsys@defobject{currentmarker}{\pgfqpoint{-0.030556in}{-0.030556in}}{\pgfqpoint{0.030556in}{0.030556in}}{%
\pgfpathmoveto{\pgfqpoint{0.000000in}{-0.030556in}}%
\pgfpathcurveto{\pgfqpoint{0.008103in}{-0.030556in}}{\pgfqpoint{0.015876in}{-0.027336in}}{\pgfqpoint{0.021606in}{-0.021606in}}%
\pgfpathcurveto{\pgfqpoint{0.027336in}{-0.015876in}}{\pgfqpoint{0.030556in}{-0.008103in}}{\pgfqpoint{0.030556in}{0.000000in}}%
\pgfpathcurveto{\pgfqpoint{0.030556in}{0.008103in}}{\pgfqpoint{0.027336in}{0.015876in}}{\pgfqpoint{0.021606in}{0.021606in}}%
\pgfpathcurveto{\pgfqpoint{0.015876in}{0.027336in}}{\pgfqpoint{0.008103in}{0.030556in}}{\pgfqpoint{0.000000in}{0.030556in}}%
\pgfpathcurveto{\pgfqpoint{-0.008103in}{0.030556in}}{\pgfqpoint{-0.015876in}{0.027336in}}{\pgfqpoint{-0.021606in}{0.021606in}}%
\pgfpathcurveto{\pgfqpoint{-0.027336in}{0.015876in}}{\pgfqpoint{-0.030556in}{0.008103in}}{\pgfqpoint{-0.030556in}{0.000000in}}%
\pgfpathcurveto{\pgfqpoint{-0.030556in}{-0.008103in}}{\pgfqpoint{-0.027336in}{-0.015876in}}{\pgfqpoint{-0.021606in}{-0.021606in}}%
\pgfpathcurveto{\pgfqpoint{-0.015876in}{-0.027336in}}{\pgfqpoint{-0.008103in}{-0.030556in}}{\pgfqpoint{0.000000in}{-0.030556in}}%
\pgfpathclose%
\pgfusepath{stroke,fill}%
}%
\begin{pgfscope}%
\pgfsys@transformshift{0.780062in}{1.646606in}%
\pgfsys@useobject{currentmarker}{}%
\end{pgfscope}%
\end{pgfscope}%
\begin{pgfscope}%
\pgfpathrectangle{\pgfqpoint{0.100000in}{0.100000in}}{\pgfqpoint{5.037500in}{2.013333in}}%
\pgfusepath{clip}%
\pgfsetrectcap%
\pgfsetroundjoin%
\pgfsetlinewidth{0.250937pt}%
\definecolor{currentstroke}{rgb}{0.862745,0.862745,0.862745}%
\pgfsetstrokecolor{currentstroke}%
\pgfsetdash{}{0pt}%
\pgfpathmoveto{\pgfqpoint{0.906000in}{1.566073in}}%
\pgfpathlineto{\pgfqpoint{4.180375in}{1.566073in}}%
\pgfusepath{stroke}%
\end{pgfscope}%
\begin{pgfscope}%
\pgfpathrectangle{\pgfqpoint{0.100000in}{0.100000in}}{\pgfqpoint{5.037500in}{2.013333in}}%
\pgfusepath{clip}%
\pgfsetrectcap%
\pgfsetroundjoin%
\pgfsetlinewidth{1.505625pt}%
\definecolor{currentstroke}{rgb}{0.000000,0.000000,1.000000}%
\pgfsetstrokecolor{currentstroke}%
\pgfsetstrokeopacity{0.500000}%
\pgfsetdash{}{0pt}%
\pgfpathmoveto{\pgfqpoint{0.780062in}{1.463576in}}%
\pgfusepath{stroke}%
\end{pgfscope}%
\begin{pgfscope}%
\pgfpathrectangle{\pgfqpoint{0.100000in}{0.100000in}}{\pgfqpoint{5.037500in}{2.013333in}}%
\pgfusepath{clip}%
\pgfsetbuttcap%
\pgfsetroundjoin%
\definecolor{currentfill}{rgb}{0.000000,0.000000,1.000000}%
\pgfsetfillcolor{currentfill}%
\pgfsetfillopacity{0.500000}%
\pgfsetlinewidth{0.250937pt}%
\definecolor{currentstroke}{rgb}{0.000000,0.000000,0.000000}%
\pgfsetstrokecolor{currentstroke}%
\pgfsetstrokeopacity{0.500000}%
\pgfsetdash{}{0pt}%
\pgfsys@defobject{currentmarker}{\pgfqpoint{-0.019444in}{-0.019444in}}{\pgfqpoint{0.019444in}{0.019444in}}{%
\pgfpathmoveto{\pgfqpoint{0.000000in}{-0.019444in}}%
\pgfpathcurveto{\pgfqpoint{0.005157in}{-0.019444in}}{\pgfqpoint{0.010103in}{-0.017396in}}{\pgfqpoint{0.013749in}{-0.013749in}}%
\pgfpathcurveto{\pgfqpoint{0.017396in}{-0.010103in}}{\pgfqpoint{0.019444in}{-0.005157in}}{\pgfqpoint{0.019444in}{0.000000in}}%
\pgfpathcurveto{\pgfqpoint{0.019444in}{0.005157in}}{\pgfqpoint{0.017396in}{0.010103in}}{\pgfqpoint{0.013749in}{0.013749in}}%
\pgfpathcurveto{\pgfqpoint{0.010103in}{0.017396in}}{\pgfqpoint{0.005157in}{0.019444in}}{\pgfqpoint{0.000000in}{0.019444in}}%
\pgfpathcurveto{\pgfqpoint{-0.005157in}{0.019444in}}{\pgfqpoint{-0.010103in}{0.017396in}}{\pgfqpoint{-0.013749in}{0.013749in}}%
\pgfpathcurveto{\pgfqpoint{-0.017396in}{0.010103in}}{\pgfqpoint{-0.019444in}{0.005157in}}{\pgfqpoint{-0.019444in}{0.000000in}}%
\pgfpathcurveto{\pgfqpoint{-0.019444in}{-0.005157in}}{\pgfqpoint{-0.017396in}{-0.010103in}}{\pgfqpoint{-0.013749in}{-0.013749in}}%
\pgfpathcurveto{\pgfqpoint{-0.010103in}{-0.017396in}}{\pgfqpoint{-0.005157in}{-0.019444in}}{\pgfqpoint{0.000000in}{-0.019444in}}%
\pgfpathclose%
\pgfusepath{stroke,fill}%
}%
\begin{pgfscope}%
\pgfsys@transformshift{0.780062in}{1.463576in}%
\pgfsys@useobject{currentmarker}{}%
\end{pgfscope}%
\end{pgfscope}%
\begin{pgfscope}%
\pgfpathrectangle{\pgfqpoint{0.100000in}{0.100000in}}{\pgfqpoint{5.037500in}{2.013333in}}%
\pgfusepath{clip}%
\pgfsetrectcap%
\pgfsetroundjoin%
\pgfsetlinewidth{0.250937pt}%
\definecolor{currentstroke}{rgb}{0.862745,0.862745,0.862745}%
\pgfsetstrokecolor{currentstroke}%
\pgfsetdash{}{0pt}%
\pgfpathmoveto{\pgfqpoint{0.906000in}{1.383042in}}%
\pgfpathlineto{\pgfqpoint{4.180375in}{1.383042in}}%
\pgfusepath{stroke}%
\end{pgfscope}%
\begin{pgfscope}%
\pgfpathrectangle{\pgfqpoint{0.100000in}{0.100000in}}{\pgfqpoint{5.037500in}{2.013333in}}%
\pgfusepath{clip}%
\pgfsetrectcap%
\pgfsetroundjoin%
\pgfsetlinewidth{1.505625pt}%
\definecolor{currentstroke}{rgb}{0.000000,0.000000,1.000000}%
\pgfsetstrokecolor{currentstroke}%
\pgfsetstrokeopacity{0.500000}%
\pgfsetdash{}{0pt}%
\pgfpathmoveto{\pgfqpoint{0.780062in}{1.280545in}}%
\pgfusepath{stroke}%
\end{pgfscope}%
\begin{pgfscope}%
\pgfpathrectangle{\pgfqpoint{0.100000in}{0.100000in}}{\pgfqpoint{5.037500in}{2.013333in}}%
\pgfusepath{clip}%
\pgfsetbuttcap%
\pgfsetroundjoin%
\definecolor{currentfill}{rgb}{0.000000,0.000000,1.000000}%
\pgfsetfillcolor{currentfill}%
\pgfsetfillopacity{0.500000}%
\pgfsetlinewidth{0.250937pt}%
\definecolor{currentstroke}{rgb}{0.000000,0.000000,0.000000}%
\pgfsetstrokecolor{currentstroke}%
\pgfsetstrokeopacity{0.500000}%
\pgfsetdash{}{0pt}%
\pgfsys@defobject{currentmarker}{\pgfqpoint{-0.033333in}{-0.033333in}}{\pgfqpoint{0.033333in}{0.033333in}}{%
\pgfpathmoveto{\pgfqpoint{0.000000in}{-0.033333in}}%
\pgfpathcurveto{\pgfqpoint{0.008840in}{-0.033333in}}{\pgfqpoint{0.017319in}{-0.029821in}}{\pgfqpoint{0.023570in}{-0.023570in}}%
\pgfpathcurveto{\pgfqpoint{0.029821in}{-0.017319in}}{\pgfqpoint{0.033333in}{-0.008840in}}{\pgfqpoint{0.033333in}{0.000000in}}%
\pgfpathcurveto{\pgfqpoint{0.033333in}{0.008840in}}{\pgfqpoint{0.029821in}{0.017319in}}{\pgfqpoint{0.023570in}{0.023570in}}%
\pgfpathcurveto{\pgfqpoint{0.017319in}{0.029821in}}{\pgfqpoint{0.008840in}{0.033333in}}{\pgfqpoint{0.000000in}{0.033333in}}%
\pgfpathcurveto{\pgfqpoint{-0.008840in}{0.033333in}}{\pgfqpoint{-0.017319in}{0.029821in}}{\pgfqpoint{-0.023570in}{0.023570in}}%
\pgfpathcurveto{\pgfqpoint{-0.029821in}{0.017319in}}{\pgfqpoint{-0.033333in}{0.008840in}}{\pgfqpoint{-0.033333in}{0.000000in}}%
\pgfpathcurveto{\pgfqpoint{-0.033333in}{-0.008840in}}{\pgfqpoint{-0.029821in}{-0.017319in}}{\pgfqpoint{-0.023570in}{-0.023570in}}%
\pgfpathcurveto{\pgfqpoint{-0.017319in}{-0.029821in}}{\pgfqpoint{-0.008840in}{-0.033333in}}{\pgfqpoint{0.000000in}{-0.033333in}}%
\pgfpathclose%
\pgfusepath{stroke,fill}%
}%
\begin{pgfscope}%
\pgfsys@transformshift{0.780062in}{1.280545in}%
\pgfsys@useobject{currentmarker}{}%
\end{pgfscope}%
\end{pgfscope}%
\begin{pgfscope}%
\pgfpathrectangle{\pgfqpoint{0.100000in}{0.100000in}}{\pgfqpoint{5.037500in}{2.013333in}}%
\pgfusepath{clip}%
\pgfsetrectcap%
\pgfsetroundjoin%
\pgfsetlinewidth{0.250937pt}%
\definecolor{currentstroke}{rgb}{0.862745,0.862745,0.862745}%
\pgfsetstrokecolor{currentstroke}%
\pgfsetdash{}{0pt}%
\pgfpathmoveto{\pgfqpoint{0.906000in}{1.200012in}}%
\pgfpathlineto{\pgfqpoint{4.180375in}{1.200012in}}%
\pgfusepath{stroke}%
\end{pgfscope}%
\begin{pgfscope}%
\pgfpathrectangle{\pgfqpoint{0.100000in}{0.100000in}}{\pgfqpoint{5.037500in}{2.013333in}}%
\pgfusepath{clip}%
\pgfsetrectcap%
\pgfsetroundjoin%
\pgfsetlinewidth{1.505625pt}%
\definecolor{currentstroke}{rgb}{0.000000,0.000000,1.000000}%
\pgfsetstrokecolor{currentstroke}%
\pgfsetstrokeopacity{0.500000}%
\pgfsetdash{}{0pt}%
\pgfpathmoveto{\pgfqpoint{0.780062in}{1.097515in}}%
\pgfusepath{stroke}%
\end{pgfscope}%
\begin{pgfscope}%
\pgfpathrectangle{\pgfqpoint{0.100000in}{0.100000in}}{\pgfqpoint{5.037500in}{2.013333in}}%
\pgfusepath{clip}%
\pgfsetbuttcap%
\pgfsetroundjoin%
\definecolor{currentfill}{rgb}{0.000000,0.000000,1.000000}%
\pgfsetfillcolor{currentfill}%
\pgfsetfillopacity{0.500000}%
\pgfsetlinewidth{0.250937pt}%
\definecolor{currentstroke}{rgb}{0.000000,0.000000,0.000000}%
\pgfsetstrokecolor{currentstroke}%
\pgfsetstrokeopacity{0.500000}%
\pgfsetdash{}{0pt}%
\pgfsys@defobject{currentmarker}{\pgfqpoint{-0.022222in}{-0.022222in}}{\pgfqpoint{0.022222in}{0.022222in}}{%
\pgfpathmoveto{\pgfqpoint{0.000000in}{-0.022222in}}%
\pgfpathcurveto{\pgfqpoint{0.005893in}{-0.022222in}}{\pgfqpoint{0.011546in}{-0.019881in}}{\pgfqpoint{0.015713in}{-0.015713in}}%
\pgfpathcurveto{\pgfqpoint{0.019881in}{-0.011546in}}{\pgfqpoint{0.022222in}{-0.005893in}}{\pgfqpoint{0.022222in}{0.000000in}}%
\pgfpathcurveto{\pgfqpoint{0.022222in}{0.005893in}}{\pgfqpoint{0.019881in}{0.011546in}}{\pgfqpoint{0.015713in}{0.015713in}}%
\pgfpathcurveto{\pgfqpoint{0.011546in}{0.019881in}}{\pgfqpoint{0.005893in}{0.022222in}}{\pgfqpoint{0.000000in}{0.022222in}}%
\pgfpathcurveto{\pgfqpoint{-0.005893in}{0.022222in}}{\pgfqpoint{-0.011546in}{0.019881in}}{\pgfqpoint{-0.015713in}{0.015713in}}%
\pgfpathcurveto{\pgfqpoint{-0.019881in}{0.011546in}}{\pgfqpoint{-0.022222in}{0.005893in}}{\pgfqpoint{-0.022222in}{0.000000in}}%
\pgfpathcurveto{\pgfqpoint{-0.022222in}{-0.005893in}}{\pgfqpoint{-0.019881in}{-0.011546in}}{\pgfqpoint{-0.015713in}{-0.015713in}}%
\pgfpathcurveto{\pgfqpoint{-0.011546in}{-0.019881in}}{\pgfqpoint{-0.005893in}{-0.022222in}}{\pgfqpoint{0.000000in}{-0.022222in}}%
\pgfpathclose%
\pgfusepath{stroke,fill}%
}%
\begin{pgfscope}%
\pgfsys@transformshift{0.780062in}{1.097515in}%
\pgfsys@useobject{currentmarker}{}%
\end{pgfscope}%
\end{pgfscope}%
\begin{pgfscope}%
\pgfpathrectangle{\pgfqpoint{0.100000in}{0.100000in}}{\pgfqpoint{5.037500in}{2.013333in}}%
\pgfusepath{clip}%
\pgfsetrectcap%
\pgfsetroundjoin%
\pgfsetlinewidth{0.250937pt}%
\definecolor{currentstroke}{rgb}{0.862745,0.862745,0.862745}%
\pgfsetstrokecolor{currentstroke}%
\pgfsetdash{}{0pt}%
\pgfpathmoveto{\pgfqpoint{0.906000in}{1.016982in}}%
\pgfpathlineto{\pgfqpoint{4.180375in}{1.016982in}}%
\pgfusepath{stroke}%
\end{pgfscope}%
\begin{pgfscope}%
\pgfpathrectangle{\pgfqpoint{0.100000in}{0.100000in}}{\pgfqpoint{5.037500in}{2.013333in}}%
\pgfusepath{clip}%
\pgfsetrectcap%
\pgfsetroundjoin%
\pgfsetlinewidth{1.505625pt}%
\definecolor{currentstroke}{rgb}{0.678431,1.000000,0.184314}%
\pgfsetstrokecolor{currentstroke}%
\pgfsetstrokeopacity{0.500000}%
\pgfsetdash{}{0pt}%
\pgfpathmoveto{\pgfqpoint{0.780062in}{0.914485in}}%
\pgfusepath{stroke}%
\end{pgfscope}%
\begin{pgfscope}%
\pgfpathrectangle{\pgfqpoint{0.100000in}{0.100000in}}{\pgfqpoint{5.037500in}{2.013333in}}%
\pgfusepath{clip}%
\pgfsetbuttcap%
\pgfsetroundjoin%
\definecolor{currentfill}{rgb}{0.678431,1.000000,0.184314}%
\pgfsetfillcolor{currentfill}%
\pgfsetfillopacity{0.500000}%
\pgfsetlinewidth{0.250937pt}%
\definecolor{currentstroke}{rgb}{0.000000,0.000000,0.000000}%
\pgfsetstrokecolor{currentstroke}%
\pgfsetstrokeopacity{0.500000}%
\pgfsetdash{}{0pt}%
\pgfsys@defobject{currentmarker}{\pgfqpoint{-0.016667in}{-0.016667in}}{\pgfqpoint{0.016667in}{0.016667in}}{%
\pgfpathmoveto{\pgfqpoint{0.000000in}{-0.016667in}}%
\pgfpathcurveto{\pgfqpoint{0.004420in}{-0.016667in}}{\pgfqpoint{0.008660in}{-0.014911in}}{\pgfqpoint{0.011785in}{-0.011785in}}%
\pgfpathcurveto{\pgfqpoint{0.014911in}{-0.008660in}}{\pgfqpoint{0.016667in}{-0.004420in}}{\pgfqpoint{0.016667in}{0.000000in}}%
\pgfpathcurveto{\pgfqpoint{0.016667in}{0.004420in}}{\pgfqpoint{0.014911in}{0.008660in}}{\pgfqpoint{0.011785in}{0.011785in}}%
\pgfpathcurveto{\pgfqpoint{0.008660in}{0.014911in}}{\pgfqpoint{0.004420in}{0.016667in}}{\pgfqpoint{0.000000in}{0.016667in}}%
\pgfpathcurveto{\pgfqpoint{-0.004420in}{0.016667in}}{\pgfqpoint{-0.008660in}{0.014911in}}{\pgfqpoint{-0.011785in}{0.011785in}}%
\pgfpathcurveto{\pgfqpoint{-0.014911in}{0.008660in}}{\pgfqpoint{-0.016667in}{0.004420in}}{\pgfqpoint{-0.016667in}{0.000000in}}%
\pgfpathcurveto{\pgfqpoint{-0.016667in}{-0.004420in}}{\pgfqpoint{-0.014911in}{-0.008660in}}{\pgfqpoint{-0.011785in}{-0.011785in}}%
\pgfpathcurveto{\pgfqpoint{-0.008660in}{-0.014911in}}{\pgfqpoint{-0.004420in}{-0.016667in}}{\pgfqpoint{0.000000in}{-0.016667in}}%
\pgfpathclose%
\pgfusepath{stroke,fill}%
}%
\begin{pgfscope}%
\pgfsys@transformshift{0.780062in}{0.914485in}%
\pgfsys@useobject{currentmarker}{}%
\end{pgfscope}%
\end{pgfscope}%
\begin{pgfscope}%
\pgfpathrectangle{\pgfqpoint{0.100000in}{0.100000in}}{\pgfqpoint{5.037500in}{2.013333in}}%
\pgfusepath{clip}%
\pgfsetrectcap%
\pgfsetroundjoin%
\pgfsetlinewidth{0.250937pt}%
\definecolor{currentstroke}{rgb}{0.862745,0.862745,0.862745}%
\pgfsetstrokecolor{currentstroke}%
\pgfsetdash{}{0pt}%
\pgfpathmoveto{\pgfqpoint{0.906000in}{0.833952in}}%
\pgfpathlineto{\pgfqpoint{4.180375in}{0.833952in}}%
\pgfusepath{stroke}%
\end{pgfscope}%
\begin{pgfscope}%
\pgfpathrectangle{\pgfqpoint{0.100000in}{0.100000in}}{\pgfqpoint{5.037500in}{2.013333in}}%
\pgfusepath{clip}%
\pgfsetrectcap%
\pgfsetroundjoin%
\pgfsetlinewidth{1.505625pt}%
\definecolor{currentstroke}{rgb}{0.000000,0.000000,1.000000}%
\pgfsetstrokecolor{currentstroke}%
\pgfsetstrokeopacity{0.500000}%
\pgfsetdash{}{0pt}%
\pgfpathmoveto{\pgfqpoint{0.780062in}{0.731455in}}%
\pgfusepath{stroke}%
\end{pgfscope}%
\begin{pgfscope}%
\pgfpathrectangle{\pgfqpoint{0.100000in}{0.100000in}}{\pgfqpoint{5.037500in}{2.013333in}}%
\pgfusepath{clip}%
\pgfsetbuttcap%
\pgfsetroundjoin%
\definecolor{currentfill}{rgb}{0.000000,0.000000,1.000000}%
\pgfsetfillcolor{currentfill}%
\pgfsetfillopacity{0.500000}%
\pgfsetlinewidth{0.250937pt}%
\definecolor{currentstroke}{rgb}{0.000000,0.000000,0.000000}%
\pgfsetstrokecolor{currentstroke}%
\pgfsetstrokeopacity{0.500000}%
\pgfsetdash{}{0pt}%
\pgfsys@defobject{currentmarker}{\pgfqpoint{-0.005556in}{-0.005556in}}{\pgfqpoint{0.005556in}{0.005556in}}{%
\pgfpathmoveto{\pgfqpoint{0.000000in}{-0.005556in}}%
\pgfpathcurveto{\pgfqpoint{0.001473in}{-0.005556in}}{\pgfqpoint{0.002887in}{-0.004970in}}{\pgfqpoint{0.003928in}{-0.003928in}}%
\pgfpathcurveto{\pgfqpoint{0.004970in}{-0.002887in}}{\pgfqpoint{0.005556in}{-0.001473in}}{\pgfqpoint{0.005556in}{0.000000in}}%
\pgfpathcurveto{\pgfqpoint{0.005556in}{0.001473in}}{\pgfqpoint{0.004970in}{0.002887in}}{\pgfqpoint{0.003928in}{0.003928in}}%
\pgfpathcurveto{\pgfqpoint{0.002887in}{0.004970in}}{\pgfqpoint{0.001473in}{0.005556in}}{\pgfqpoint{0.000000in}{0.005556in}}%
\pgfpathcurveto{\pgfqpoint{-0.001473in}{0.005556in}}{\pgfqpoint{-0.002887in}{0.004970in}}{\pgfqpoint{-0.003928in}{0.003928in}}%
\pgfpathcurveto{\pgfqpoint{-0.004970in}{0.002887in}}{\pgfqpoint{-0.005556in}{0.001473in}}{\pgfqpoint{-0.005556in}{0.000000in}}%
\pgfpathcurveto{\pgfqpoint{-0.005556in}{-0.001473in}}{\pgfqpoint{-0.004970in}{-0.002887in}}{\pgfqpoint{-0.003928in}{-0.003928in}}%
\pgfpathcurveto{\pgfqpoint{-0.002887in}{-0.004970in}}{\pgfqpoint{-0.001473in}{-0.005556in}}{\pgfqpoint{0.000000in}{-0.005556in}}%
\pgfpathclose%
\pgfusepath{stroke,fill}%
}%
\begin{pgfscope}%
\pgfsys@transformshift{0.780062in}{0.731455in}%
\pgfsys@useobject{currentmarker}{}%
\end{pgfscope}%
\end{pgfscope}%
\begin{pgfscope}%
\pgfpathrectangle{\pgfqpoint{0.100000in}{0.100000in}}{\pgfqpoint{5.037500in}{2.013333in}}%
\pgfusepath{clip}%
\pgfsetrectcap%
\pgfsetroundjoin%
\pgfsetlinewidth{0.250937pt}%
\definecolor{currentstroke}{rgb}{0.862745,0.862745,0.862745}%
\pgfsetstrokecolor{currentstroke}%
\pgfsetdash{}{0pt}%
\pgfpathmoveto{\pgfqpoint{0.906000in}{0.650921in}}%
\pgfpathlineto{\pgfqpoint{4.180375in}{0.650921in}}%
\pgfusepath{stroke}%
\end{pgfscope}%
\begin{pgfscope}%
\pgfpathrectangle{\pgfqpoint{0.100000in}{0.100000in}}{\pgfqpoint{5.037500in}{2.013333in}}%
\pgfusepath{clip}%
\pgfsetrectcap%
\pgfsetroundjoin%
\pgfsetlinewidth{1.505625pt}%
\definecolor{currentstroke}{rgb}{0.000000,0.000000,1.000000}%
\pgfsetstrokecolor{currentstroke}%
\pgfsetstrokeopacity{0.500000}%
\pgfsetdash{}{0pt}%
\pgfpathmoveto{\pgfqpoint{0.780062in}{0.548424in}}%
\pgfusepath{stroke}%
\end{pgfscope}%
\begin{pgfscope}%
\pgfpathrectangle{\pgfqpoint{0.100000in}{0.100000in}}{\pgfqpoint{5.037500in}{2.013333in}}%
\pgfusepath{clip}%
\pgfsetbuttcap%
\pgfsetroundjoin%
\definecolor{currentfill}{rgb}{0.000000,0.000000,1.000000}%
\pgfsetfillcolor{currentfill}%
\pgfsetfillopacity{0.500000}%
\pgfsetlinewidth{0.250937pt}%
\definecolor{currentstroke}{rgb}{0.000000,0.000000,0.000000}%
\pgfsetstrokecolor{currentstroke}%
\pgfsetstrokeopacity{0.500000}%
\pgfsetdash{}{0pt}%
\pgfsys@defobject{currentmarker}{\pgfqpoint{-0.011111in}{-0.011111in}}{\pgfqpoint{0.011111in}{0.011111in}}{%
\pgfpathmoveto{\pgfqpoint{0.000000in}{-0.011111in}}%
\pgfpathcurveto{\pgfqpoint{0.002947in}{-0.011111in}}{\pgfqpoint{0.005773in}{-0.009940in}}{\pgfqpoint{0.007857in}{-0.007857in}}%
\pgfpathcurveto{\pgfqpoint{0.009940in}{-0.005773in}}{\pgfqpoint{0.011111in}{-0.002947in}}{\pgfqpoint{0.011111in}{0.000000in}}%
\pgfpathcurveto{\pgfqpoint{0.011111in}{0.002947in}}{\pgfqpoint{0.009940in}{0.005773in}}{\pgfqpoint{0.007857in}{0.007857in}}%
\pgfpathcurveto{\pgfqpoint{0.005773in}{0.009940in}}{\pgfqpoint{0.002947in}{0.011111in}}{\pgfqpoint{0.000000in}{0.011111in}}%
\pgfpathcurveto{\pgfqpoint{-0.002947in}{0.011111in}}{\pgfqpoint{-0.005773in}{0.009940in}}{\pgfqpoint{-0.007857in}{0.007857in}}%
\pgfpathcurveto{\pgfqpoint{-0.009940in}{0.005773in}}{\pgfqpoint{-0.011111in}{0.002947in}}{\pgfqpoint{-0.011111in}{0.000000in}}%
\pgfpathcurveto{\pgfqpoint{-0.011111in}{-0.002947in}}{\pgfqpoint{-0.009940in}{-0.005773in}}{\pgfqpoint{-0.007857in}{-0.007857in}}%
\pgfpathcurveto{\pgfqpoint{-0.005773in}{-0.009940in}}{\pgfqpoint{-0.002947in}{-0.011111in}}{\pgfqpoint{0.000000in}{-0.011111in}}%
\pgfpathclose%
\pgfusepath{stroke,fill}%
}%
\begin{pgfscope}%
\pgfsys@transformshift{0.780062in}{0.548424in}%
\pgfsys@useobject{currentmarker}{}%
\end{pgfscope}%
\end{pgfscope}%
\begin{pgfscope}%
\pgfpathrectangle{\pgfqpoint{0.100000in}{0.100000in}}{\pgfqpoint{5.037500in}{2.013333in}}%
\pgfusepath{clip}%
\pgfsetrectcap%
\pgfsetroundjoin%
\pgfsetlinewidth{0.250937pt}%
\definecolor{currentstroke}{rgb}{0.862745,0.862745,0.862745}%
\pgfsetstrokecolor{currentstroke}%
\pgfsetdash{}{0pt}%
\pgfpathmoveto{\pgfqpoint{0.906000in}{0.467891in}}%
\pgfpathlineto{\pgfqpoint{4.180375in}{0.467891in}}%
\pgfusepath{stroke}%
\end{pgfscope}%
\begin{pgfscope}%
\pgfpathrectangle{\pgfqpoint{0.100000in}{0.100000in}}{\pgfqpoint{5.037500in}{2.013333in}}%
\pgfusepath{clip}%
\pgfsetrectcap%
\pgfsetroundjoin%
\pgfsetlinewidth{1.505625pt}%
\definecolor{currentstroke}{rgb}{0.000000,0.000000,1.000000}%
\pgfsetstrokecolor{currentstroke}%
\pgfsetstrokeopacity{0.500000}%
\pgfsetdash{}{0pt}%
\pgfpathmoveto{\pgfqpoint{0.780062in}{0.365394in}}%
\pgfusepath{stroke}%
\end{pgfscope}%
\begin{pgfscope}%
\pgfpathrectangle{\pgfqpoint{0.100000in}{0.100000in}}{\pgfqpoint{5.037500in}{2.013333in}}%
\pgfusepath{clip}%
\pgfsetbuttcap%
\pgfsetroundjoin%
\definecolor{currentfill}{rgb}{0.000000,0.000000,1.000000}%
\pgfsetfillcolor{currentfill}%
\pgfsetfillopacity{0.500000}%
\pgfsetlinewidth{0.250937pt}%
\definecolor{currentstroke}{rgb}{0.000000,0.000000,0.000000}%
\pgfsetstrokecolor{currentstroke}%
\pgfsetstrokeopacity{0.500000}%
\pgfsetdash{}{0pt}%
\pgfsys@defobject{currentmarker}{\pgfqpoint{-0.025000in}{-0.025000in}}{\pgfqpoint{0.025000in}{0.025000in}}{%
\pgfpathmoveto{\pgfqpoint{0.000000in}{-0.025000in}}%
\pgfpathcurveto{\pgfqpoint{0.006630in}{-0.025000in}}{\pgfqpoint{0.012989in}{-0.022366in}}{\pgfqpoint{0.017678in}{-0.017678in}}%
\pgfpathcurveto{\pgfqpoint{0.022366in}{-0.012989in}}{\pgfqpoint{0.025000in}{-0.006630in}}{\pgfqpoint{0.025000in}{0.000000in}}%
\pgfpathcurveto{\pgfqpoint{0.025000in}{0.006630in}}{\pgfqpoint{0.022366in}{0.012989in}}{\pgfqpoint{0.017678in}{0.017678in}}%
\pgfpathcurveto{\pgfqpoint{0.012989in}{0.022366in}}{\pgfqpoint{0.006630in}{0.025000in}}{\pgfqpoint{0.000000in}{0.025000in}}%
\pgfpathcurveto{\pgfqpoint{-0.006630in}{0.025000in}}{\pgfqpoint{-0.012989in}{0.022366in}}{\pgfqpoint{-0.017678in}{0.017678in}}%
\pgfpathcurveto{\pgfqpoint{-0.022366in}{0.012989in}}{\pgfqpoint{-0.025000in}{0.006630in}}{\pgfqpoint{-0.025000in}{0.000000in}}%
\pgfpathcurveto{\pgfqpoint{-0.025000in}{-0.006630in}}{\pgfqpoint{-0.022366in}{-0.012989in}}{\pgfqpoint{-0.017678in}{-0.017678in}}%
\pgfpathcurveto{\pgfqpoint{-0.012989in}{-0.022366in}}{\pgfqpoint{-0.006630in}{-0.025000in}}{\pgfqpoint{0.000000in}{-0.025000in}}%
\pgfpathclose%
\pgfusepath{stroke,fill}%
}%
\begin{pgfscope}%
\pgfsys@transformshift{0.780062in}{0.365394in}%
\pgfsys@useobject{currentmarker}{}%
\end{pgfscope}%
\end{pgfscope}%
\begin{pgfscope}%
\pgfpathrectangle{\pgfqpoint{0.100000in}{0.100000in}}{\pgfqpoint{5.037500in}{2.013333in}}%
\pgfusepath{clip}%
\pgfsetrectcap%
\pgfsetroundjoin%
\pgfsetlinewidth{0.250937pt}%
\definecolor{currentstroke}{rgb}{0.862745,0.862745,0.862745}%
\pgfsetstrokecolor{currentstroke}%
\pgfsetdash{}{0pt}%
\pgfpathmoveto{\pgfqpoint{0.906000in}{0.284861in}}%
\pgfpathlineto{\pgfqpoint{4.180375in}{0.284861in}}%
\pgfusepath{stroke}%
\end{pgfscope}%
\begin{pgfscope}%
\pgfpathrectangle{\pgfqpoint{0.100000in}{0.100000in}}{\pgfqpoint{5.037500in}{2.013333in}}%
\pgfusepath{clip}%
\pgfsetrectcap%
\pgfsetroundjoin%
\pgfsetlinewidth{1.505625pt}%
\definecolor{currentstroke}{rgb}{0.678431,1.000000,0.184314}%
\pgfsetstrokecolor{currentstroke}%
\pgfsetstrokeopacity{0.500000}%
\pgfsetdash{}{0pt}%
\pgfpathmoveto{\pgfqpoint{0.780062in}{0.182364in}}%
\pgfusepath{stroke}%
\end{pgfscope}%
\begin{pgfscope}%
\pgfpathrectangle{\pgfqpoint{0.100000in}{0.100000in}}{\pgfqpoint{5.037500in}{2.013333in}}%
\pgfusepath{clip}%
\pgfsetbuttcap%
\pgfsetroundjoin%
\definecolor{currentfill}{rgb}{0.678431,1.000000,0.184314}%
\pgfsetfillcolor{currentfill}%
\pgfsetfillopacity{0.500000}%
\pgfsetlinewidth{0.250937pt}%
\definecolor{currentstroke}{rgb}{0.000000,0.000000,0.000000}%
\pgfsetstrokecolor{currentstroke}%
\pgfsetstrokeopacity{0.500000}%
\pgfsetdash{}{0pt}%
\pgfsys@defobject{currentmarker}{\pgfqpoint{-0.041667in}{-0.041667in}}{\pgfqpoint{0.041667in}{0.041667in}}{%
\pgfpathmoveto{\pgfqpoint{0.000000in}{-0.041667in}}%
\pgfpathcurveto{\pgfqpoint{0.011050in}{-0.041667in}}{\pgfqpoint{0.021649in}{-0.037276in}}{\pgfqpoint{0.029463in}{-0.029463in}}%
\pgfpathcurveto{\pgfqpoint{0.037276in}{-0.021649in}}{\pgfqpoint{0.041667in}{-0.011050in}}{\pgfqpoint{0.041667in}{0.000000in}}%
\pgfpathcurveto{\pgfqpoint{0.041667in}{0.011050in}}{\pgfqpoint{0.037276in}{0.021649in}}{\pgfqpoint{0.029463in}{0.029463in}}%
\pgfpathcurveto{\pgfqpoint{0.021649in}{0.037276in}}{\pgfqpoint{0.011050in}{0.041667in}}{\pgfqpoint{0.000000in}{0.041667in}}%
\pgfpathcurveto{\pgfqpoint{-0.011050in}{0.041667in}}{\pgfqpoint{-0.021649in}{0.037276in}}{\pgfqpoint{-0.029463in}{0.029463in}}%
\pgfpathcurveto{\pgfqpoint{-0.037276in}{0.021649in}}{\pgfqpoint{-0.041667in}{0.011050in}}{\pgfqpoint{-0.041667in}{0.000000in}}%
\pgfpathcurveto{\pgfqpoint{-0.041667in}{-0.011050in}}{\pgfqpoint{-0.037276in}{-0.021649in}}{\pgfqpoint{-0.029463in}{-0.029463in}}%
\pgfpathcurveto{\pgfqpoint{-0.021649in}{-0.037276in}}{\pgfqpoint{-0.011050in}{-0.041667in}}{\pgfqpoint{0.000000in}{-0.041667in}}%
\pgfpathclose%
\pgfusepath{stroke,fill}%
}%
\begin{pgfscope}%
\pgfsys@transformshift{0.780062in}{0.182364in}%
\pgfsys@useobject{currentmarker}{}%
\end{pgfscope}%
\end{pgfscope}%
\begin{pgfscope}%
\pgfpathrectangle{\pgfqpoint{0.100000in}{0.100000in}}{\pgfqpoint{5.037500in}{2.013333in}}%
\pgfusepath{clip}%
\pgfsetrectcap%
\pgfsetroundjoin%
\pgfsetlinewidth{0.250937pt}%
\definecolor{currentstroke}{rgb}{0.862745,0.862745,0.862745}%
\pgfsetstrokecolor{currentstroke}%
\pgfsetdash{}{0pt}%
\pgfpathmoveto{\pgfqpoint{0.906000in}{0.101830in}}%
\pgfpathlineto{\pgfqpoint{4.180375in}{0.101830in}}%
\pgfusepath{stroke}%
\end{pgfscope}%
\begin{pgfscope}%
\definecolor{textcolor}{rgb}{0.000000,0.000000,0.000000}%
\pgfsetstrokecolor{textcolor}%
\pgfsetfillcolor{textcolor}%
\pgftext[x=2.266125in,y=5.634638in,left,base]{\color{textcolor}\setmainfont{Lato}\rmfamily\fontsize{8.000000}{9.600000}\bfseries\selectfont Legend:}%
\end{pgfscope}%
\begin{pgfscope}%
\definecolor{textcolor}{rgb}{0.000000,0.000000,0.000000}%
\pgfsetstrokecolor{textcolor}%
\pgfsetfillcolor{textcolor}%
\pgftext[x=0.906000in,y=1.972400in,left,base]{\color{textcolor}\setmainfont{Lato}\rmfamily\fontsize{8.000000}{9.600000}\selectfont New York, NY}%
\end{pgfscope}%
\begin{pgfscope}%
\definecolor{textcolor}{rgb}{0.000000,0.000000,0.000000}%
\pgfsetstrokecolor{textcolor}%
\pgfsetfillcolor{textcolor}%
\pgftext[x=1.963875in,y=1.972400in,left,base]{\color{textcolor}\setmainfont{Lato}\rmfamily\fontsize{8.000000}{9.600000}\selectfont 5.5}%
\end{pgfscope}%
\begin{pgfscope}%
\definecolor{textcolor}{rgb}{0.000000,0.000000,0.000000}%
\pgfsetstrokecolor{textcolor}%
\pgfsetfillcolor{textcolor}%
\pgftext[x=2.568375in,y=1.972400in,left,base]{\color{textcolor}\setmainfont{Lato}\rmfamily\fontsize{8.000000}{9.600000}\selectfont 3.1}%
\end{pgfscope}%
\begin{pgfscope}%
\definecolor{textcolor}{rgb}{0.000000,0.000000,0.000000}%
\pgfsetstrokecolor{textcolor}%
\pgfsetfillcolor{textcolor}%
\pgftext[x=0.654125in,y=1.972400in,right,base]{\color{textcolor}\setmainfont{Lato}\rmfamily\fontsize{8.000000}{9.600000}\selectfont +2.4}%
\end{pgfscope}%
\begin{pgfscope}%
\definecolor{textcolor}{rgb}{0.000000,0.000000,0.000000}%
\pgfsetstrokecolor{textcolor}%
\pgfsetfillcolor{textcolor}%
\pgftext[x=3.097312in,y=1.972400in,left,base]{\color{textcolor}\setmainfont{Lato}\rmfamily\fontsize{8.000000}{9.600000}\selectfont 9,626,000}%
\end{pgfscope}%
\begin{pgfscope}%
\definecolor{textcolor}{rgb}{0.000000,0.000000,0.000000}%
\pgfsetstrokecolor{textcolor}%
\pgfsetfillcolor{textcolor}%
\pgftext[x=4.180375in,y=1.972400in,right,base]{\color{textcolor}\setmainfont{Lato}\rmfamily\fontsize{8.000000}{9.600000}\selectfont -3.6}%
\end{pgfscope}%
\begin{pgfscope}%
\definecolor{textcolor}{rgb}{0.000000,0.000000,0.000000}%
\pgfsetstrokecolor{textcolor}%
\pgfsetfillcolor{textcolor}%
\pgftext[x=0.906000in,y=1.789370in,left,base]{\color{textcolor}\setmainfont{Lato}\rmfamily\fontsize{8.000000}{9.600000}\selectfont Los Angeles, CA}%
\end{pgfscope}%
\begin{pgfscope}%
\definecolor{textcolor}{rgb}{0.000000,0.000000,0.000000}%
\pgfsetstrokecolor{textcolor}%
\pgfsetfillcolor{textcolor}%
\pgftext[x=1.963875in,y=1.789370in,left,base]{\color{textcolor}\setmainfont{Lato}\rmfamily\fontsize{8.000000}{9.600000}\selectfont 5.6}%
\end{pgfscope}%
\begin{pgfscope}%
\definecolor{textcolor}{rgb}{0.000000,0.000000,0.000000}%
\pgfsetstrokecolor{textcolor}%
\pgfsetfillcolor{textcolor}%
\pgftext[x=2.568375in,y=1.789370in,left,base]{\color{textcolor}\setmainfont{Lato}\rmfamily\fontsize{8.000000}{9.600000}\selectfont 3.9}%
\end{pgfscope}%
\begin{pgfscope}%
\definecolor{textcolor}{rgb}{0.000000,0.000000,0.000000}%
\pgfsetstrokecolor{textcolor}%
\pgfsetfillcolor{textcolor}%
\pgftext[x=0.654125in,y=1.789370in,right,base]{\color{textcolor}\setmainfont{Lato}\rmfamily\fontsize{8.000000}{9.600000}\selectfont +1.7}%
\end{pgfscope}%
\begin{pgfscope}%
\definecolor{textcolor}{rgb}{0.000000,0.000000,0.000000}%
\pgfsetstrokecolor{textcolor}%
\pgfsetfillcolor{textcolor}%
\pgftext[x=3.097312in,y=1.789370in,left,base]{\color{textcolor}\setmainfont{Lato}\rmfamily\fontsize{8.000000}{9.600000}\selectfont 6,610,100}%
\end{pgfscope}%
\begin{pgfscope}%
\definecolor{textcolor}{rgb}{0.000000,0.000000,0.000000}%
\pgfsetstrokecolor{textcolor}%
\pgfsetfillcolor{textcolor}%
\pgftext[x=4.180375in,y=1.789370in,right,base]{\color{textcolor}\setmainfont{Lato}\rmfamily\fontsize{8.000000}{9.600000}\selectfont -3.2}%
\end{pgfscope}%
\begin{pgfscope}%
\definecolor{textcolor}{rgb}{0.000000,0.000000,0.000000}%
\pgfsetstrokecolor{textcolor}%
\pgfsetfillcolor{textcolor}%
\pgftext[x=0.906000in,y=1.606339in,left,base]{\color{textcolor}\setmainfont{Lato}\rmfamily\fontsize{8.000000}{9.600000}\selectfont Chicago, IL}%
\end{pgfscope}%
\begin{pgfscope}%
\definecolor{textcolor}{rgb}{0.000000,0.000000,0.000000}%
\pgfsetstrokecolor{textcolor}%
\pgfsetfillcolor{textcolor}%
\pgftext[x=1.963875in,y=1.606339in,left,base]{\color{textcolor}\setmainfont{Lato}\rmfamily\fontsize{8.000000}{9.600000}\selectfont 4.3}%
\end{pgfscope}%
\begin{pgfscope}%
\definecolor{textcolor}{rgb}{0.000000,0.000000,0.000000}%
\pgfsetstrokecolor{textcolor}%
\pgfsetfillcolor{textcolor}%
\pgftext[x=2.568375in,y=1.606339in,left,base]{\color{textcolor}\setmainfont{Lato}\rmfamily\fontsize{8.000000}{9.600000}\selectfont 3.2}%
\end{pgfscope}%
\begin{pgfscope}%
\definecolor{textcolor}{rgb}{0.000000,0.000000,0.000000}%
\pgfsetstrokecolor{textcolor}%
\pgfsetfillcolor{textcolor}%
\pgftext[x=0.654125in,y=1.606339in,right,base]{\color{textcolor}\setmainfont{Lato}\rmfamily\fontsize{8.000000}{9.600000}\selectfont +1.1}%
\end{pgfscope}%
\begin{pgfscope}%
\definecolor{textcolor}{rgb}{0.000000,0.000000,0.000000}%
\pgfsetstrokecolor{textcolor}%
\pgfsetfillcolor{textcolor}%
\pgftext[x=3.097312in,y=1.606339in,left,base]{\color{textcolor}\setmainfont{Lato}\rmfamily\fontsize{8.000000}{9.600000}\selectfont 4,832,400}%
\end{pgfscope}%
\begin{pgfscope}%
\definecolor{textcolor}{rgb}{0.000000,0.000000,0.000000}%
\pgfsetstrokecolor{textcolor}%
\pgfsetfillcolor{textcolor}%
\pgftext[x=4.180375in,y=1.606339in,right,base]{\color{textcolor}\setmainfont{Lato}\rmfamily\fontsize{8.000000}{9.600000}\selectfont 0.1}%
\end{pgfscope}%
\begin{pgfscope}%
\definecolor{textcolor}{rgb}{0.000000,0.000000,0.000000}%
\pgfsetstrokecolor{textcolor}%
\pgfsetfillcolor{textcolor}%
\pgftext[x=0.906000in,y=1.423309in,left,base]{\color{textcolor}\setmainfont{Lato}\rmfamily\fontsize{8.000000}{9.600000}\selectfont Dallas, TX}%
\end{pgfscope}%
\begin{pgfscope}%
\definecolor{textcolor}{rgb}{0.000000,0.000000,0.000000}%
\pgfsetstrokecolor{textcolor}%
\pgfsetfillcolor{textcolor}%
\pgftext[x=1.963875in,y=1.423309in,left,base]{\color{textcolor}\setmainfont{Lato}\rmfamily\fontsize{8.000000}{9.600000}\selectfont 3.6}%
\end{pgfscope}%
\begin{pgfscope}%
\definecolor{textcolor}{rgb}{0.000000,0.000000,0.000000}%
\pgfsetstrokecolor{textcolor}%
\pgfsetfillcolor{textcolor}%
\pgftext[x=2.568375in,y=1.423309in,left,base]{\color{textcolor}\setmainfont{Lato}\rmfamily\fontsize{8.000000}{9.600000}\selectfont 2.9}%
\end{pgfscope}%
\begin{pgfscope}%
\definecolor{textcolor}{rgb}{0.000000,0.000000,0.000000}%
\pgfsetstrokecolor{textcolor}%
\pgfsetfillcolor{textcolor}%
\pgftext[x=0.654125in,y=1.423309in,right,base]{\color{textcolor}\setmainfont{Lato}\rmfamily\fontsize{8.000000}{9.600000}\selectfont +0.7}%
\end{pgfscope}%
\begin{pgfscope}%
\definecolor{textcolor}{rgb}{0.000000,0.000000,0.000000}%
\pgfsetstrokecolor{textcolor}%
\pgfsetfillcolor{textcolor}%
\pgftext[x=3.097312in,y=1.423309in,left,base]{\color{textcolor}\setmainfont{Lato}\rmfamily\fontsize{8.000000}{9.600000}\selectfont 4,182,200}%
\end{pgfscope}%
\begin{pgfscope}%
\definecolor{textcolor}{rgb}{0.000000,0.000000,0.000000}%
\pgfsetstrokecolor{textcolor}%
\pgfsetfillcolor{textcolor}%
\pgftext[x=4.180375in,y=1.423309in,right,base]{\color{textcolor}\setmainfont{Lato}\rmfamily\fontsize{8.000000}{9.600000}\selectfont 3.4}%
\end{pgfscope}%
\begin{pgfscope}%
\definecolor{textcolor}{rgb}{0.000000,0.000000,0.000000}%
\pgfsetstrokecolor{textcolor}%
\pgfsetfillcolor{textcolor}%
\pgftext[x=0.906000in,y=1.240279in,left,base]{\color{textcolor}\setmainfont{Lato}\rmfamily\fontsize{8.000000}{9.600000}\selectfont Houston, TX}%
\end{pgfscope}%
\begin{pgfscope}%
\definecolor{textcolor}{rgb}{0.000000,0.000000,0.000000}%
\pgfsetstrokecolor{textcolor}%
\pgfsetfillcolor{textcolor}%
\pgftext[x=1.963875in,y=1.240279in,left,base]{\color{textcolor}\setmainfont{Lato}\rmfamily\fontsize{8.000000}{9.600000}\selectfont 4.8}%
\end{pgfscope}%
\begin{pgfscope}%
\definecolor{textcolor}{rgb}{0.000000,0.000000,0.000000}%
\pgfsetstrokecolor{textcolor}%
\pgfsetfillcolor{textcolor}%
\pgftext[x=2.568375in,y=1.240279in,left,base]{\color{textcolor}\setmainfont{Lato}\rmfamily\fontsize{8.000000}{9.600000}\selectfont 3.6}%
\end{pgfscope}%
\begin{pgfscope}%
\definecolor{textcolor}{rgb}{0.000000,0.000000,0.000000}%
\pgfsetstrokecolor{textcolor}%
\pgfsetfillcolor{textcolor}%
\pgftext[x=0.654125in,y=1.240279in,right,base]{\color{textcolor}\setmainfont{Lato}\rmfamily\fontsize{8.000000}{9.600000}\selectfont +1.2}%
\end{pgfscope}%
\begin{pgfscope}%
\definecolor{textcolor}{rgb}{0.000000,0.000000,0.000000}%
\pgfsetstrokecolor{textcolor}%
\pgfsetfillcolor{textcolor}%
\pgftext[x=3.097312in,y=1.240279in,left,base]{\color{textcolor}\setmainfont{Lato}\rmfamily\fontsize{8.000000}{9.600000}\selectfont 3,482,700}%
\end{pgfscope}%
\begin{pgfscope}%
\definecolor{textcolor}{rgb}{0.000000,0.000000,0.000000}%
\pgfsetstrokecolor{textcolor}%
\pgfsetfillcolor{textcolor}%
\pgftext[x=4.180375in,y=1.240279in,right,base]{\color{textcolor}\setmainfont{Lato}\rmfamily\fontsize{8.000000}{9.600000}\selectfont 0.6}%
\end{pgfscope}%
\begin{pgfscope}%
\definecolor{textcolor}{rgb}{0.000000,0.000000,0.000000}%
\pgfsetstrokecolor{textcolor}%
\pgfsetfillcolor{textcolor}%
\pgftext[x=0.906000in,y=1.057248in,left,base]{\color{textcolor}\setmainfont{Lato}\rmfamily\fontsize{8.000000}{9.600000}\selectfont Washington, DC}%
\end{pgfscope}%
\begin{pgfscope}%
\definecolor{textcolor}{rgb}{0.000000,0.000000,0.000000}%
\pgfsetstrokecolor{textcolor}%
\pgfsetfillcolor{textcolor}%
\pgftext[x=1.963875in,y=1.057248in,left,base]{\color{textcolor}\setmainfont{Lato}\rmfamily\fontsize{8.000000}{9.600000}\selectfont 3.3}%
\end{pgfscope}%
\begin{pgfscope}%
\definecolor{textcolor}{rgb}{0.000000,0.000000,0.000000}%
\pgfsetstrokecolor{textcolor}%
\pgfsetfillcolor{textcolor}%
\pgftext[x=2.568375in,y=1.057248in,left,base]{\color{textcolor}\setmainfont{Lato}\rmfamily\fontsize{8.000000}{9.600000}\selectfont 2.5}%
\end{pgfscope}%
\begin{pgfscope}%
\definecolor{textcolor}{rgb}{0.000000,0.000000,0.000000}%
\pgfsetstrokecolor{textcolor}%
\pgfsetfillcolor{textcolor}%
\pgftext[x=0.654125in,y=1.057248in,right,base]{\color{textcolor}\setmainfont{Lato}\rmfamily\fontsize{8.000000}{9.600000}\selectfont +0.8}%
\end{pgfscope}%
\begin{pgfscope}%
\definecolor{textcolor}{rgb}{0.000000,0.000000,0.000000}%
\pgfsetstrokecolor{textcolor}%
\pgfsetfillcolor{textcolor}%
\pgftext[x=3.097312in,y=1.057248in,left,base]{\color{textcolor}\setmainfont{Lato}\rmfamily\fontsize{8.000000}{9.600000}\selectfont 3,351,300}%
\end{pgfscope}%
\begin{pgfscope}%
\definecolor{textcolor}{rgb}{0.000000,0.000000,0.000000}%
\pgfsetstrokecolor{textcolor}%
\pgfsetfillcolor{textcolor}%
\pgftext[x=4.180375in,y=1.057248in,right,base]{\color{textcolor}\setmainfont{Lato}\rmfamily\fontsize{8.000000}{9.600000}\selectfont -4.3}%
\end{pgfscope}%
\begin{pgfscope}%
\definecolor{textcolor}{rgb}{0.000000,0.000000,0.000000}%
\pgfsetstrokecolor{textcolor}%
\pgfsetfillcolor{textcolor}%
\pgftext[x=0.906000in,y=0.874218in,left,base]{\color{textcolor}\setmainfont{Lato}\rmfamily\fontsize{8.000000}{9.600000}\selectfont Atlanta, GA}%
\end{pgfscope}%
\begin{pgfscope}%
\definecolor{textcolor}{rgb}{0.000000,0.000000,0.000000}%
\pgfsetstrokecolor{textcolor}%
\pgfsetfillcolor{textcolor}%
\pgftext[x=1.963875in,y=0.874218in,left,base]{\color{textcolor}\setmainfont{Lato}\rmfamily\fontsize{8.000000}{9.600000}\selectfont 2.3}%
\end{pgfscope}%
\begin{pgfscope}%
\definecolor{textcolor}{rgb}{0.000000,0.000000,0.000000}%
\pgfsetstrokecolor{textcolor}%
\pgfsetfillcolor{textcolor}%
\pgftext[x=2.568375in,y=0.874218in,left,base]{\color{textcolor}\setmainfont{Lato}\rmfamily\fontsize{8.000000}{9.600000}\selectfont 2.9}%
\end{pgfscope}%
\begin{pgfscope}%
\definecolor{textcolor}{rgb}{0.000000,0.000000,0.000000}%
\pgfsetstrokecolor{textcolor}%
\pgfsetfillcolor{textcolor}%
\pgftext[x=0.654125in,y=0.874218in,right,base]{\color{textcolor}\setmainfont{Lato}\rmfamily\fontsize{8.000000}{9.600000}\selectfont -0.6}%
\end{pgfscope}%
\begin{pgfscope}%
\definecolor{textcolor}{rgb}{0.000000,0.000000,0.000000}%
\pgfsetstrokecolor{textcolor}%
\pgfsetfillcolor{textcolor}%
\pgftext[x=3.097312in,y=0.874218in,left,base]{\color{textcolor}\setmainfont{Lato}\rmfamily\fontsize{8.000000}{9.600000}\selectfont 3,157,300}%
\end{pgfscope}%
\begin{pgfscope}%
\definecolor{textcolor}{rgb}{0.000000,0.000000,0.000000}%
\pgfsetstrokecolor{textcolor}%
\pgfsetfillcolor{textcolor}%
\pgftext[x=4.180375in,y=0.874218in,right,base]{\color{textcolor}\setmainfont{Lato}\rmfamily\fontsize{8.000000}{9.600000}\selectfont 0.9}%
\end{pgfscope}%
\begin{pgfscope}%
\definecolor{textcolor}{rgb}{0.000000,0.000000,0.000000}%
\pgfsetstrokecolor{textcolor}%
\pgfsetfillcolor{textcolor}%
\pgftext[x=0.906000in,y=0.691188in,left,base]{\color{textcolor}\setmainfont{Lato}\rmfamily\fontsize{8.000000}{9.600000}\selectfont Miami, FL}%
\end{pgfscope}%
\begin{pgfscope}%
\definecolor{textcolor}{rgb}{0.000000,0.000000,0.000000}%
\pgfsetstrokecolor{textcolor}%
\pgfsetfillcolor{textcolor}%
\pgftext[x=1.963875in,y=0.691188in,left,base]{\color{textcolor}\setmainfont{Lato}\rmfamily\fontsize{8.000000}{9.600000}\selectfont 2.6}%
\end{pgfscope}%
\begin{pgfscope}%
\definecolor{textcolor}{rgb}{0.000000,0.000000,0.000000}%
\pgfsetstrokecolor{textcolor}%
\pgfsetfillcolor{textcolor}%
\pgftext[x=2.568375in,y=0.691188in,left,base]{\color{textcolor}\setmainfont{Lato}\rmfamily\fontsize{8.000000}{9.600000}\selectfont 2.4}%
\end{pgfscope}%
\begin{pgfscope}%
\definecolor{textcolor}{rgb}{0.000000,0.000000,0.000000}%
\pgfsetstrokecolor{textcolor}%
\pgfsetfillcolor{textcolor}%
\pgftext[x=0.654125in,y=0.691188in,right,base]{\color{textcolor}\setmainfont{Lato}\rmfamily\fontsize{8.000000}{9.600000}\selectfont unch.}%
\end{pgfscope}%
\begin{pgfscope}%
\definecolor{textcolor}{rgb}{0.000000,0.000000,0.000000}%
\pgfsetstrokecolor{textcolor}%
\pgfsetfillcolor{textcolor}%
\pgftext[x=3.097312in,y=0.691188in,left,base]{\color{textcolor}\setmainfont{Lato}\rmfamily\fontsize{8.000000}{9.600000}\selectfont 3,132,800}%
\end{pgfscope}%
\begin{pgfscope}%
\definecolor{textcolor}{rgb}{0.000000,0.000000,0.000000}%
\pgfsetstrokecolor{textcolor}%
\pgfsetfillcolor{textcolor}%
\pgftext[x=4.180375in,y=0.691188in,right,base]{\color{textcolor}\setmainfont{Lato}\rmfamily\fontsize{8.000000}{9.600000}\selectfont -0.8}%
\end{pgfscope}%
\begin{pgfscope}%
\definecolor{textcolor}{rgb}{0.000000,0.000000,0.000000}%
\pgfsetstrokecolor{textcolor}%
\pgfsetfillcolor{textcolor}%
\pgftext[x=0.906000in,y=0.508158in,left,base]{\color{textcolor}\setmainfont{Lato}\rmfamily\fontsize{8.000000}{9.600000}\selectfont Philadelphia, PA}%
\end{pgfscope}%
\begin{pgfscope}%
\definecolor{textcolor}{rgb}{0.000000,0.000000,0.000000}%
\pgfsetstrokecolor{textcolor}%
\pgfsetfillcolor{textcolor}%
\pgftext[x=1.963875in,y=0.508158in,left,base]{\color{textcolor}\setmainfont{Lato}\rmfamily\fontsize{8.000000}{9.600000}\selectfont 4.2}%
\end{pgfscope}%
\begin{pgfscope}%
\definecolor{textcolor}{rgb}{0.000000,0.000000,0.000000}%
\pgfsetstrokecolor{textcolor}%
\pgfsetfillcolor{textcolor}%
\pgftext[x=2.568375in,y=0.508158in,left,base]{\color{textcolor}\setmainfont{Lato}\rmfamily\fontsize{8.000000}{9.600000}\selectfont 3.8}%
\end{pgfscope}%
\begin{pgfscope}%
\definecolor{textcolor}{rgb}{0.000000,0.000000,0.000000}%
\pgfsetstrokecolor{textcolor}%
\pgfsetfillcolor{textcolor}%
\pgftext[x=0.654125in,y=0.508158in,right,base]{\color{textcolor}\setmainfont{Lato}\rmfamily\fontsize{8.000000}{9.600000}\selectfont +0.4}%
\end{pgfscope}%
\begin{pgfscope}%
\definecolor{textcolor}{rgb}{0.000000,0.000000,0.000000}%
\pgfsetstrokecolor{textcolor}%
\pgfsetfillcolor{textcolor}%
\pgftext[x=3.097312in,y=0.508158in,left,base]{\color{textcolor}\setmainfont{Lato}\rmfamily\fontsize{8.000000}{9.600000}\selectfont 3,036,500}%
\end{pgfscope}%
\begin{pgfscope}%
\definecolor{textcolor}{rgb}{0.000000,0.000000,0.000000}%
\pgfsetstrokecolor{textcolor}%
\pgfsetfillcolor{textcolor}%
\pgftext[x=4.180375in,y=0.508158in,right,base]{\color{textcolor}\setmainfont{Lato}\rmfamily\fontsize{8.000000}{9.600000}\selectfont -4.1}%
\end{pgfscope}%
\begin{pgfscope}%
\definecolor{textcolor}{rgb}{0.000000,0.000000,0.000000}%
\pgfsetstrokecolor{textcolor}%
\pgfsetfillcolor{textcolor}%
\pgftext[x=0.906000in,y=0.325127in,left,base]{\color{textcolor}\setmainfont{Lato}\rmfamily\fontsize{8.000000}{9.600000}\selectfont Boston, MA}%
\end{pgfscope}%
\begin{pgfscope}%
\definecolor{textcolor}{rgb}{0.000000,0.000000,0.000000}%
\pgfsetstrokecolor{textcolor}%
\pgfsetfillcolor{textcolor}%
\pgftext[x=1.963875in,y=0.325127in,left,base]{\color{textcolor}\setmainfont{Lato}\rmfamily\fontsize{8.000000}{9.600000}\selectfont 3.1}%
\end{pgfscope}%
\begin{pgfscope}%
\definecolor{textcolor}{rgb}{0.000000,0.000000,0.000000}%
\pgfsetstrokecolor{textcolor}%
\pgfsetfillcolor{textcolor}%
\pgftext[x=2.568375in,y=0.325127in,left,base]{\color{textcolor}\setmainfont{Lato}\rmfamily\fontsize{8.000000}{9.600000}\selectfont 2.2}%
\end{pgfscope}%
\begin{pgfscope}%
\definecolor{textcolor}{rgb}{0.000000,0.000000,0.000000}%
\pgfsetstrokecolor{textcolor}%
\pgfsetfillcolor{textcolor}%
\pgftext[x=0.654125in,y=0.325127in,right,base]{\color{textcolor}\setmainfont{Lato}\rmfamily\fontsize{8.000000}{9.600000}\selectfont +0.9}%
\end{pgfscope}%
\begin{pgfscope}%
\definecolor{textcolor}{rgb}{0.000000,0.000000,0.000000}%
\pgfsetstrokecolor{textcolor}%
\pgfsetfillcolor{textcolor}%
\pgftext[x=3.097312in,y=0.325127in,left,base]{\color{textcolor}\setmainfont{Lato}\rmfamily\fontsize{8.000000}{9.600000}\selectfont 2,749,000}%
\end{pgfscope}%
\begin{pgfscope}%
\definecolor{textcolor}{rgb}{0.000000,0.000000,0.000000}%
\pgfsetstrokecolor{textcolor}%
\pgfsetfillcolor{textcolor}%
\pgftext[x=4.180375in,y=0.325127in,right,base]{\color{textcolor}\setmainfont{Lato}\rmfamily\fontsize{8.000000}{9.600000}\selectfont -2.2}%
\end{pgfscope}%
\begin{pgfscope}%
\definecolor{textcolor}{rgb}{0.000000,0.000000,0.000000}%
\pgfsetstrokecolor{textcolor}%
\pgfsetfillcolor{textcolor}%
\pgftext[x=0.906000in,y=0.142097in,left,base]{\color{textcolor}\setmainfont{Lato}\rmfamily\fontsize{8.000000}{9.600000}\selectfont Phoenix, AZ}%
\end{pgfscope}%
\begin{pgfscope}%
\definecolor{textcolor}{rgb}{0.000000,0.000000,0.000000}%
\pgfsetstrokecolor{textcolor}%
\pgfsetfillcolor{textcolor}%
\pgftext[x=1.963875in,y=0.142097in,left,base]{\color{textcolor}\setmainfont{Lato}\rmfamily\fontsize{8.000000}{9.600000}\selectfont 2.4}%
\end{pgfscope}%
\begin{pgfscope}%
\definecolor{textcolor}{rgb}{0.000000,0.000000,0.000000}%
\pgfsetstrokecolor{textcolor}%
\pgfsetfillcolor{textcolor}%
\pgftext[x=2.568375in,y=0.142097in,left,base]{\color{textcolor}\setmainfont{Lato}\rmfamily\fontsize{8.000000}{9.600000}\selectfont 3.9}%
\end{pgfscope}%
\begin{pgfscope}%
\definecolor{textcolor}{rgb}{0.000000,0.000000,0.000000}%
\pgfsetstrokecolor{textcolor}%
\pgfsetfillcolor{textcolor}%
\pgftext[x=0.654125in,y=0.142097in,right,base]{\color{textcolor}\setmainfont{Lato}\rmfamily\fontsize{8.000000}{9.600000}\selectfont -1.5}%
\end{pgfscope}%
\begin{pgfscope}%
\definecolor{textcolor}{rgb}{0.000000,0.000000,0.000000}%
\pgfsetstrokecolor{textcolor}%
\pgfsetfillcolor{textcolor}%
\pgftext[x=3.097312in,y=0.142097in,left,base]{\color{textcolor}\setmainfont{Lato}\rmfamily\fontsize{8.000000}{9.600000}\selectfont 2,601,900}%
\end{pgfscope}%
\begin{pgfscope}%
\definecolor{textcolor}{rgb}{0.000000,0.000000,0.000000}%
\pgfsetstrokecolor{textcolor}%
\pgfsetfillcolor{textcolor}%
\pgftext[x=4.180375in,y=0.142097in,right,base]{\color{textcolor}\setmainfont{Lato}\rmfamily\fontsize{8.000000}{9.600000}\selectfont 3.1}%
\end{pgfscope}%
\begin{pgfscope}%
\definecolor{textcolor}{rgb}{0.000000,0.000000,0.000000}%
\pgfsetstrokecolor{textcolor}%
\pgfsetfillcolor{textcolor}%
\pgftext[x=0.283182in,y=2.351273in,left,base]{\color{textcolor}\setmainfont{Lato}\rmfamily\fontsize{10.000000}{12.000000}\bfseries\selectfont Largest MSAs:}%
\end{pgfscope}%
\begin{pgfscope}%
\definecolor{textcolor}{rgb}{0.411765,0.411765,0.411765}%
\pgfsetstrokecolor{textcolor}%
\pgfsetfillcolor{textcolor}%
\pgftext[x=0.924318in,y=2.149939in,left,base]{\color{textcolor}\setmainfont{Lato}\rmfamily\fontsize{9.000000}{10.800000}\selectfont Core City \ \ \ \ \ \ \ \ Dec 21  \ \ \ \ \ \  Dec 19 \ \ \ \ \ \ Labor Force \ \ \ \ \ \ Pct Ch*}%
\end{pgfscope}%
\end{pgfpicture}%
\makeatother%
\endgroup%


\vspace{-2mm}

\footnotesize{Source: Bureau of Labor Statistics}\\


\end{minipage}


\newpage


\begin{minipage}{0.76\textwidth}

\subsection*{\color{black!70} \seriffont Labor Force Flows}

\vspace{2mm}

\small Among newly employed workers, the vast majority were considered to be out of the labor force the prior month, as opposed unemployed. \input{text/lf_flow.txt} \\

\vspace{2mm}

\noindent \normalsize \textbf{Newly Employed, Not Previously Looking For Work}\\
\footnotesize{\textit{share of newly employed that were not looking for work in the prior month}}\\
\noindent \hspace*{-2mm} \begin{tikzpicture}
	\begin{axis}[\bbar{y}{0}, \dateaxisticks ytick={60, 70}, 
		xticklabel={`\short{\year}}, clip=false, height=5.0cm]
	\rbars
	\stdline{lime!50!green!60!white}{date}{total}{data/lf_flow.csv}
	\stdline{green!60!teal!90!black}{date}{quarterly}{data/lf_flow_q.csv}
	\end{axis}
\end{tikzpicture}\\
\footnotesize{Source: Bureau of Labor Statistics} \\

\end{minipage}

\vspace{5mm}

\begin{minipage}{0.30\textwidth}
\small The great recession worsened job-finding prospects for those not in the labor force (NILF) due to disability or illness. Only over the past few years have these prospects recovered. \input{text/disflow.txt}
\end{minipage} \hspace{8mm} \begin{minipage}{0.38\textwidth}
\noindent \normalsize \textbf{Flow, Disability to Work}\\
\footnotesize{\textit{NILF disability/illness, share employed one year later}}\\
\noindent \hspace*{-2mm} \begin{tikzpicture}
	\begin{axis}[\bbar{y}{0}, \dateaxisticks ytick={6, 8}, 
		xticklabel={`\short{\year}}, height=5.2cm, width=7.4cm]
	\rbars
	\stdline{blue}{date}{Share}{data/disflow.csv}
	\end{axis}
\end{tikzpicture}\\
\footnotesize{Source: Author's Calculations}
\end{minipage}

\vspace{8mm}

\begin{minipage}{0.76\textwidth}


\normalsize

Part-time and full-time and hours worked \\

Job growth \\

Wage growth: \\

\hspace{4mm} [AHE and UWE both in various forms] \\

\hspace{4mm} [Either FRB Atlanta Wage Tracker or replication]\\

\end{minipage}


\newpage


\begin{minipage}{0.76\textwidth}

\subsection*{\color{black!70} \seriffont Wage Growth}

\small The usual wages of full-time workers can be measured at various points in the income distribution using the Current Population Survey. BLS reports these data by decile and quartile, with the most commonly used measure being the median usual weekly earnings. The first decile usual weekly earnings of full-time workers have increased rapidly over the past year, suggesting fewer people are working full-time for less than \$10 per hour.\\


\vspace{2mm}

\noindent \normalsize \textbf{Weekly Earnings, First Decile} 

\vspace{-2mm}
\footnotesize{\textit{full-time, wage and salary earners, age 16+}} \hspace{10mm} \small 	
	\colorline{lime!65!green}{ \ bd CPS} \hspace{3mm} \colorline{blue!65!black}{ \ BLS}\\
\noindent \hspace*{-2mm} \begin{tikzpicture}
	\begin{axis}[\bbar{y}{0}, \dateaxisticks ytick={200, 300, 400, 500}, enlarge y limits={0.15},
		clip=false, width=6.5cm,
		xtick={{1989-01-01}, {2000-01-01}, {2010-01-01}, {2019-04-01}},
        minor xtick={}, xticklabels={`89, `00, `10, `19 Q3}]
	\rbars
	\stdline{lime!65!green}{date}{All}{data/uwe_bd.csv}
	\stdline{blue!65!black}{date}{value}{data/uwe_bls.csv}
	\node[right] at (axis cs:1989-07-01,485) {\footnotesize Levels, USD};
	\end{axis}
\end{tikzpicture} \hfill \hspace{3mm}
\begin{tikzpicture}
	\begin{axis}[\bbar{y}{0}, \dateaxisticks ytick={-5, 0, 5}, enlarge y limits={0.15},
		clip=false, width=6.5cm, xmin=1989-12-15,
		xtick={{1990-01-01}, {2000-01-01}, {2010-01-01}, {2019-04-01}},
        minor xtick={}, xticklabels={`90, `00, `10, `19 Q3}]
	\rbars
	\stdline{lime!65!green}{date}{All}{data/uwe_bd_gr.csv}
	\stdline{blue!65!black}{date}{value}{data/uwe_bls_gr.csv}
	\node[right] at (axis cs:1990-01-01,9.0) {\footnotesize One-year growth, percent};
	\end{axis}
\end{tikzpicture}\\
\footnotesize{Source: Bureau of Labor Statistics and Author's Calculations} \\

\vspace{12mm}


\noindent \normalsize \textbf{Weekly Earnings Growth, First Decile and Median}\\
\footnotesize{\textit{full-time, wage and salary earners, age 16+, one-year growth, percent}} \\
\noindent \hspace*{-2mm} \begin{tikzpicture}
	\begin{axis}[\bbar{y}{0}, \dateaxisticks ytick={-2, 0, 2, 4, 6, 8}, 
		enlarge y limits={0.1}, legend style={at={(0.98, 1.13)}},
		xticklabel={`\short{\year}}, clip=false, height=5.0cm]
	\rbars
	\stdline{cyan!70!blue}{date}{p10_sa}{data/p10med.csv}
	\stdline{violet}{date}{p50_sa}{data/p10med.csv}
	\legend{First Decile, Median};
	\end{axis}
\end{tikzpicture}\\
\footnotesize{Source: Author's Calculations} \\


\end{minipage}

\newpage

\begin{minipage}{0.76\textwidth}


\small Some types of turnover in the labor market are healthy and mean people are better able find a new job if they do not like the one they have. The Bureau of Labor Statistics \href{https://www.bls.gov/news.release/pdf/jolts.pdf}{reports} the number of job openings, hires, and separations in several industry groups on a monthly basis. Within separations, these data distinguish voluntarily leaving of a job as ``quits''. 
\end{minipage}\\

\begin{minipage}{0.26\textwidth}
\small \input{text/jolts2.txt}
\end{minipage} \hspace{5mm}
\begin{minipage}{0.4\textwidth}
\noindent \normalsize \textbf{Job Turnover}\\
\footnotesize{\textit{job openings, hires, and quits, in millions}}\\
\noindent \hspace*{-2mm} \begin{tikzpicture}
	\begin{axis}[\bbar{y}{0}, \dateaxisticks ytick={2, 4, 6, 8}, 
		enlarge y limits={0.1}, 
		xticklabel={`\short{\year}}, height=5.6cm, width=8.0cm]
	\rbars
	\stdline{cyan}{date}{Hires}{data/jolts.csv}
	\stdline{red}{date}{Quits}{data/jolts.csv}
	\stdline{blue!80!black}{date}{Openings}{data/jolts.csv}
	\stdnode{3.3cm}{2.0cm}{\scriptsize \color{cyan}\textbf{Hires}}
	\stdnode{5.2cm}{3.38cm}{\scriptsize \color{blue!80!black}\textbf{Openings}}
	\stdnode{3.7cm}{0.2cm}{\scriptsize \color{red}\textbf{Quits}}
	\end{axis}
\end{tikzpicture}\\
\footnotesize{Source: Bureau of Labor Statistics}
\end{minipage}

\vspace{6mm}

\begin{minipage}{0.76\textwidth}

\normalsize

Quits \\ 

Openings \\

\vspace{6mm}

\small \input{text/icsa.txt}\\

\vspace{2mm}

\noindent \normalsize \textbf{New Unemployment Insurance Claims}\\
\footnotesize{\textit{initial claims, in thousands, seasonally adjusted, three-month moving average highlighted}}\\
\noindent \hspace*{-2mm} \begin{tikzpicture}
	\begin{axis}[\bbar{y}{0}, \dateaxisticks ytick={200, 400, 600}, enlarge y limits={0.1},
		xticklabel={`\short{\year}}, clip=false, height=5.0cm]
	\rbars
	\stdline{cyan!30}{date}{weekly}{data/icsa.csv}
	\stdline{blue!60!black}{date}{3M}{data/icsa.csv}
	\end{axis}
\end{tikzpicture}\\
\footnotesize{Source: Department of Labor} \\

\vspace{2mm}

\normalsize

Reasons for non-participation \\

Union membership \\

State- and sub-state-level analysis \\



\end{minipage}

\newpage

\begin{minipage}{0.76\textwidth}


\subsection*{\color{black!70} \seriffont Labor Productivity}

\small Labor productivity is \href{https://www.bls.gov/news.release/prod2.nr0.htm}{reported} by the Bureau of Labor Statistics and measured as real output per hour of work in the nonfarm business sector. Economic theory suggests that labor productivity is particularly important for long-term real economic growth. The measure captures the rate at which people, with all of the resources and equipment and infrastructure available to them, are able to produce goods and services with their work. An increase in labor productivity means real wages can increase without putting upward pressure on inflation. Alternatively, an increase in productivity means a society can meet its material needs with less work.\\

\input{text/lprod.txt}
\vspace{5mm}

\noindent \normalsize \textbf{Labor Productivity Growth}\\
\footnotesize{\textit{quarterly growth at seasonally adjusted annual rate, percent}}\\
\noindent \hspace*{-2mm} \begin{tikzpicture}
	\begin{axis}[\bbar{y}{0}, \dateaxisticks ytick={-5, 0, 5}, 
		xticklabel={`\short{\year}}, clip=false, height=4.0cm]
	\rbars
	\sbar{teal}{date}{value}{data/lprod.csv}
	\end{axis}
\end{tikzpicture}\\
\footnotesize{Source: Bureau of Labor Statistics} \\

\vspace{3mm}

\small There are two areas to investigate in understanding trends in productivity growth rates. The first is the theory that the level of business net investment in equipment and other capital goods, particularly relative to the size of the workforce, determines productivity growth. Such investment allows more goods and services to be produced by the same number amount of work. The second theory, sometimes called the Kaldor-Verdoorn Law, is that overall economic growth and capacity utilization determine productivity growth. In this scenario an economy facing real resource constraints is more likely to find ways to produce goods and services more efficiently.\\



\end{minipage}

\newpage

\begin{minipage}{0.76\textwidth}

\section*{\color{darkgray}\LARGE \seriffont Financial Markets}

\small The US equity markets and capital markets provide businesses and governments with funding for activities and fixed investments. \\

\vspace{2mm}



\subsection*{\color{black!70} \seriffont Equity Markets}

\hspace{4mm} [SP500] \\

\vspace{5mm}

\small \input{text/sp500rr.txt}  \\

\vspace{2mm}

\noindent \normalsize \textbf{S\&P 500 Real Return}\\
\footnotesize{\textit{trailing 20-year annualized real return}}\\ 
\noindent \hspace*{-2mm} \begin{tikzpicture}
	\begin{axis}[\bbar{y}{0}, \dateaxisticks ytick={4, 8, 12}, 
		xticklabel={`\short{\year}}, clip=false, 
		height=4.4cm, enlarge y limits={0.11}]
	\rbars
	\stdline{green!80!blue}{date}{Return}{data/sp500rr.csv}
	\end{axis}
\end{tikzpicture}\\
\footnotesize{Source: Shiller, Author's Calculations} \\

\vspace{4mm}

\small The Chicago Board Options Exchange uses S\&P 500 options data to identify expectations of future volatility. \\

\vspace{2mm}

\noindent \normalsize \textbf{S\&P 500 Volatility Index}\\
\footnotesize{\textit{index, monthly average shown}}\\ 
\noindent \hspace*{-2mm} \begin{tikzpicture}
	\begin{axis}[\bbar{y}{0}, \dateaxisticks ytick={0, 20, 40, 60}, 
		xticklabel={`\short{\year}}, clip=false, 
		height=4.6cm, enlarge y limits={0.11}]
	\rbars
	\stdline{magenta}{date}{value}{data/vix.csv}
	\end{axis}
\end{tikzpicture}\\
\footnotesize{Source: Chicago Board Options Exchange} \\

\vspace{4mm}

\normalsize

Valuations \\

\hspace{4mm} [PE Ratio] \\

\end{minipage}

\newpage

\begin{minipage}{0.76\textwidth}

\subsection*{\color{black!70} \seriffont Interest Rates}

\vspace{4mm}

\small The US Federal Reserve System (Fed) has a congressional \href{https://www.federalreserve.gov/faqs/money_12848.htm}{mandate} to promote price stability and maximum employment. In practice, a Fed committee determines the federal funds rate, which aims to influence interest rates in the broader economy. Fed monetary policy can be neutral or be used to stimulate or slow the economy.\\

Actual data here on recent moves by the Fed and the Fed funds rate. \\

\vspace{1mm}

\noindent \normalsize \textbf{Effective Fed Funds Rate}\\
\footnotesize{\textit{percent, monthly average except for latest value }}\\ 
\noindent \hspace*{-2mm} \begin{tikzpicture}
	\begin{axis}[\bbar{y}{0}, \dateaxisticks ytick={0, 5, 10}, 
		xticklabel={`\short{\year}}, clip=false]
	\rbars
	\stdline{blue!60!black}{date}{Fed Funds}{data/rates.csv}
	\end{axis}
\end{tikzpicture}\\
\footnotesize{Source: Federal Reserve} \\

\vspace{4mm}

\small Text here about Treasury yields. \\

\vspace{1mm}

\noindent \normalsize \textbf{Treasury Constant Maturity Yields}\\
\footnotesize{\textit{percent, monthly average except for latest value }}\\ 
\noindent \hspace*{-2mm} \begin{tikzpicture}
	\begin{axis}[\bbar{y}{0}, \dateaxisticks ytick={0, 5, 10}, ymin=0.05,
		xticklabel={`\short{\year}}, clip=false]
	\rbars
	\stdline{red!80!purple}{date}{Ten-year}{data/rates.csv}
	\stdline{orange!70!yellow}{date}{Two-year}{data/rates.csv}
	\stdnode{6.2cm}{0.65cm}{\scriptsize \color{orange!70!yellow}\textbf{Two-year}}
	\stdnode{8.7cm}{1.28cm}{\scriptsize \color{red!80!purple}\textbf{Ten-year}}
	\end{axis}
\end{tikzpicture}\\
\footnotesize{Source: Federal Reserve} \\

\vspace{4mm}

\normalsize

\hspace{4mm} [Fed liabilities] \\

\hspace{4mm} [Fed assets] \\

\hspace{4mm} [AAA and high-yield] \\

Yield curve \\

\end{minipage}

\newpage

\begin{minipage}{0.76\textwidth}

\subsection*{\color{black!70} \seriffont Money and Monetary Policy}

\vspace{4mm}

\small Text here on the money supply. \\

\vspace{2mm}

\noindent \normalsize \textbf{M2 and Institutional Money Funds}\\
\footnotesize{\textit{one-year percent change, monthly average}}\\ 
\noindent \hspace*{-2mm} \begin{tikzpicture}
	\begin{axis}[\bbar{y}{0}, \dateaxisticks ytick={-5, 0, 5, 10, 15}, 
		xticklabel={`\short{\year}}, clip=false]
	\rbars
	\thickline{green!80!blue}{date}{value}{data/M2imf.csv}
	\end{axis}
\end{tikzpicture}\\
\footnotesize{Source: Federal Reserve} \\

\end{minipage}

\newpage

\begin{minipage}{0.76\textwidth}

\section*{\color{darkgray}\LARGE \seriffont Prices}

\small \input{text/cpi_main.txt} \\

\vspace{2mm}

\noindent \normalsize \textbf{Consumer Price Index}\\
\footnotesize{\textit{annual growth, percent}}\\ 
\noindent \hspace*{-2mm} \begin{tikzpicture}
	\begin{axis}[\bbar{y}{0}, \dateaxisticks ytick={-2, 0, 2, 4, 6}, 
		xticklabel={`\short{\year}}, clip=false, 
		legend style={at={(0.98, 1.13)}},
		height=4.4cm, enlarge y limits={0.11}]
	\rbars
	\stdline{blue!60!cyan}{date}{value}{data/cpi.csv}
	\stdline{gray}{date}{value2}{data/cpi.csv}
	%\stdnode{2.4cm}{0.35cm}{\scriptsize \input{text/cpi.txt}}
	\legend{All-items, Core};
	\end{axis}
\end{tikzpicture}\\
\footnotesize{Source: Bureau of Labor Statistics} \\

\end{minipage}

\vspace{2mm}

\begin{minipage}{0.3\textwidth}
\small Housing has been the largest contributor to inflation since XXXXXX, XXXX. In October 2019, housing contributed X.X percentage points to overall CPI inflation, compared to a contribution of X.X percentage points in October 2018. Medical care added X.X percentage points to overall inflation in October 2019, compared to a contribution of X.X percent in October 2018. \\

In October 2019, Energy subtracted X.X percentage points from overall CPI inflation, compared to a contribution of X.X percent in October 2019. In total, six of the ten selected CPI components contributed positively to overall CPI inflation, while four subtracted from inflation and none were unchanged. 
\end{minipage} \hspace{6mm}
\begin{minipage}{0.36\textwidth}
\noindent \normalsize \textbf{Consumer Price Index}\\
\footnotesize{\textit{contribution to annual growth, percentage points}}\\ 
\noindent \hspace*{-7mm} \begin{tikzpicture}
  	\begin{axis}[\barplotnogrid axis y line=left, \barylab{3.3cm}{1.5ex}
    	width=5.0cm, bar width=1.6ex, height=7.8cm,
    	enlarge y limits={abs=0.35cm}, enlarge x limits=0.3, \bbar{x}{0},
		yticklabels from table={\cpi}{name}, 
		yticklabel style={font=\footnotesize, xshift=-4pt},
		nodes near coords style={/pgf/number format/.cd, fixed zerofill,
			precision=2, assume math mode},
		legend style={text=black!70, at={(-0.18,1.08)}, anchor=north, legend columns=-1, 
				fill=none, draw=none,
		        /tikz/every even column/.append style={column sep=0.4cm}}]
  	\addplot[fill=cyan!40, draw=none] 
  		table [y expr=-\coordindex, x index=2] {\cpi};
  	\addplot[fill=blue!70, draw=none] 
  		table [y expr=-\coordindex, x index=1] {\cpi};
 	\legend{\input{text/cpi_mo2.txt}, \input{text/cpi_mo1.txt}}
  	\end{axis}
\end{tikzpicture}\\
\footnotesize{Source: Bureau of Labor Statistics} \\
\end{minipage}

\newpage

\normalsize

CPI: \\

\hspace{4mm} [CPI-U growth - core, all-items, CPI-U-RS]\\

\hspace{4mm} [CPI-U components contribution - horizontal range chart] \\ 

PPI \\

XMPI \\

PCE \\

Expectations \\



\newpage

\begin{minipage}{0.76\textwidth}

\small \input{text/wti.txt}\\

\vspace{2mm}

\noindent \normalsize \textbf{Oil Price}\\
\footnotesize{\textit{USD, west Texas intermediate crude, monthly average}}\\ 
\noindent \hspace*{-2mm} \begin{tikzpicture}
	\begin{axis}[\bbar{y}{0}, \dateaxisticks ytick={0, 50, 100}, 
		xticklabel={`\short{\year}}, ymin=0]
	\rbars
	\stdline{red!80!purple}{date}{VALUE}{data/wti.csv}
	\end{axis}
\end{tikzpicture}\\
\footnotesize{Source: FRED} \\

\end{minipage}

\newpage

\section*{\color{darkgray}\LARGE \seriffont International Comparisons}

\normalsize

Demographics \\

Economic Activity \\

Labor Markets \\

Poverty \\


\newpage

\begin{minipage}{0.76\textwidth}
\section*{\color{darkgray}\LARGE \seriffont References}

\small

List of tables and sources along with some notes... \\

One option for this section is to have some json data that captures what original data goes into each series and also what types of calculations are done on the original data.

\subsection*{\color{black!70} {\seriffont Acknowledgments}}

Gabriel Mathy, Iordan Koulov, Lara Merling, Kevin Cashman, Rebecca Watts, Dean Baker, Eileen Appelbaum, John Schmitt, Yevgeniya Korniyenko, Magali Pinat, Rainer K\"ohler, Gersenda Varisco, Venkat Josyula, Tom Augspurger, Mike Sieferling, Matt Bruenig, and Ernie Tedeschi.

\end{minipage}

\end{document}