% % % % % % % % % % % % % % 
%
%	U.S. Chartbook
%	Brian W. Dew (brianwdew@gmail.com)
%	Updated: September 23, 2019
%	GitHub repo contains to do list (issues)
%   https://github.com/bdecon/US-chartbook
%
% % % % % % % % % % % % % %
\PassOptionsToPackage{table}{xcolor}
\documentclass{report}

%
% % % % % % Packages % % % % % % % % % 
%
	
	\usepackage[letterpaper, margin=1.2in]{geometry}
	\usepackage{microtype}
	\usepackage[default]{lato}
	\usepackage{pgfplots}
	\usepackage{xcolor}
	\usepackage{array}
	\usepackage{fontawesome}
	\usetikzlibrary{pgfplots.dateplot}

%
% % % % % Document Settings % % % % % % % 
%

	% Paragraph spacing
		\setlength{\parskip}{8pt}
		\setlength{\parindent}{0pt}
		
%
% % % % % Graph Settings % % % % % % % 
%
	
	% Color square
	\newcommand{\cbox}[1]{
		\begin{tikzpicture} \draw [#1, line width=6](0,0) -- (.2,0);  
		\end{tikzpicture}}
	
	% Last two digits of year
	\makeatletter
	\newcommand*\short[1]{\expandafter\@gobbletwo\number\numexpr#1\relax}
	\makeatother	
	
	% Column width and alignment
	\newcolumntype{R}[1]{>{\raggedleft\let\newline\\\arraybackslash\hspace{0pt}}m{#1}}	
	
	% Style for date plots
	\pgfplotsset{compat=newest, 
		scaled y ticks=false,
		axis line style={black!20}, 
		xtick style={black!20}, ytick style={draw=none},
		every tick label/.style={black!50, font=\scriptsize,
			/pgf/number format/assume math mode=true},
		width=13.0cm, height=4.6cm, 
		xticklabel style={align=left}, 
		yticklabel style={text width=0.85em, align=right},       
		axis x line*=bottom, x axis line style={black!50},
	    axis y line=left, y axis line style={opacity=0},
	    ymajorgrids, grid style={very thin, black!10},	        
	    every node near coord/.style={black!70},
	    legend style={legend columns=-1, draw=none, fill=none,
	    	/tikz/every even column/.append style={column sep=0.3cm}}}
	
	% stacked diverging bar
	\newcommand{\sbar}[4]{
		\addplot[ybar stacked, bar width=2.7pt, draw opacity=0, fill=#1] 
			table [x=#2, y=#3, col sep=comma]{#4};}
					
	% text node
	\newcommand{\stdnode}[3]{\node[below, align=left, shift=({#1,#2})]{#3};}	        
		        
	% Date (X) Axis Tick Marks, one tick per year, every even year labeled
	\newcommand{\dateaxisticks}{
		date coordinates in=x, axis line style={draw=none},
		xmax={2019-10-01},
		max space between ticks=40,	    
		xtick={{1990-01-01}, {1992-01-01}, {1994-01-01}, 
			{1996-01-01}, {1998-01-01}, {2000-01-01}, 
			{2002-01-01}, {2004-01-01}, {2006-01-01},
			{2008-01-01}, {2010-01-01}, {2012-01-01}, {2014-01-01},
		    {2016-01-01}, {2018-01-01}},
		minor xtick={{1989-01-01}, {1991-01-01}, {1993-01-01},
			{1995-01-01}, {1997-01-01}, {1999-01-01}, 
			{2001-01-01}, {2003-01-01}, {2005-01-01}, {2007-01-01},
		    {2009-01-01}, {2011-01-01}, {2013-01-01}, {2015-01-01},
		    {2017-01-01}, {2019-01-01}},
		enlarge y limits={0.04}, enlarge x limits={0.01},
		}
	
	% Solid bars at significant  x or y values
	\newcommand{\bbar}[2]{extra #1 ticks = {{#2}}, extra #1 tick labels = ,
		extra #1 tick style = {grid=major, grid style={thick, black!25}},}
		
	% Standard line
	\newcommand{\stdline}[4]{\addplot[very thick, no markers, color=#1] 
		table [x=#2, y=#3, col sep=comma] {#4};	}
		
	% Recession bars		
	\newcommand{\rbars}{
		\fill[color=black!10] (axis cs:{1990-07-01},\pgfkeysvalueof{/pgfplots/ymin}) rectangle 
			(axis cs:{1991-03-01}, \pgfkeysvalueof{/pgfplots/ymax});
		\fill[color=black!10] (axis cs:{2007-12-01},\pgfkeysvalueof{/pgfplots/ymin}) rectangle 
			(axis cs:{2009-07-01}, \pgfkeysvalueof{/pgfplots/ymax});
		\fill[color=black!10] (axis cs:{2001-03-01},\pgfkeysvalueof{/pgfplots/ymin}) rectangle 
			(axis cs:{2001-11-01}, \pgfkeysvalueof{/pgfplots/ymax});}
	
	\newfontfamily\seriffont{SourceSerifPro}	
			    		    
% % % % % % % %
%
%  Begin Document
%
% % % % % % % %		
\begin{document}

\chapter*{
		\textcolor{blue!70}{\rule[-1pt]{6pt}{20pt}}
		\textcolor{green!70!blue}{\rule[-1pt]{6pt}{32pt}} \ \color{darkgray} \seriffont US Chartbook}
\vspace*{-16mm}
\footnotesize \hspace{11mm} v0.0, \today \normalsize \\

\vspace{12mm}

\begin{minipage}{0.76\textwidth}
\subsection*{\color{black!70} {\seriffont Notes}}

{\color{red} \textbf{Very early stage draft}} -- Contents not considered reliable.\\


\subsection*{\color{black!70} {\seriffont Contact}}

\textbf{Brian Wilson Dew} \ 

{\color{gray} \faEnvelope} \ brian.w.dew@gmail.com \ 

{\color{gray} \faTwitter} \ @bd\_econ \

{\color{gray} \faGithub} \ \ bdecon\\

\end{minipage}
\thispagestyle{empty}

\newpage

\subsection*{\color{black!70} {\seriffont Contents}}

\begin{description}

\item {\seriffont Overall Economic Activity}

\item {\seriffont Overall Financial Activity}

\item {\seriffont Households}

\item {\seriffont Businesses}

\item {\seriffont Government}

\item {\seriffont External Sector}

\item {\seriffont Labor Markets}

\item {\seriffont Capital Markets}

\item {\seriffont Prices}

\item {\seriffont International Comparisons}

\item {\seriffont References}

\end{description}

\vspace{6mm}


\subsection*{\color{black!70} {\seriffont Ideas/Suggestions/To Do}}

\begin{minipage}{0.76\textwidth}

\small For now, focus on filling out the content. Once around 1/3 of the content is in place (~35 pages) start to look into ways to more efficiently create the document. \\

It will be nice to have a section showing the top five indicators: GDP growth, wages, epop, cpi inflation, 10-year treasury yields. \\

It will also be nice to have a section in that puts some context on numbers generally. The key example that I've tried to do before is to put a threshold on GDP growth that marks how much is needed for population growth and depreciation and then calculate how much one extra pp of growth (beyond the previous amount) is worth, per person. For example, if population growth is 0.6pp and depreciation is 0.8pp, then it would take 1.4pp to keep the same level of real per capita production. Beyond that, an extra percentage point of GDP might mean something like \$900 per person in additional goods and services. \\

Section listing recent updates and upcoming releases would be nice. \\

Beyond content, I still need to do/add: links to subsection, links to sources, links to data, links to code, date of last update, list of charts and numbering system, links between charts and references, marks for recent updates, and much much more. 
\end{minipage}


\newpage

\section*{\color{darkgray} \LARGE  \seriffont Overall Economic Activity}
\input{text/gdp.txt}

\begin{minipage}{0.76\textwidth}
\subsection*{\color{black!70}\seriffont Economic Growth}

\small GDP (see\cbox{red!95!black}) \input{text/gdp_gr.txt}
\vspace{5mm}

\noindent \normalsize \textbf{Real Gross Domestic Product Growth}\\
\footnotesize{\textit{quarterly growth at seasonally adjusted annual rate, percent}}\\
\noindent \hspace*{-2mm} \begin{tikzpicture}
	\begin{axis}[\bbar{y}{0}, \dateaxisticks ytick={-5, 0, 5}, 
		xticklabel={`\short{\year}}, clip=false, height=4.0cm]
	\rbars
	\sbar{red!95!black}{date}{A191RL}{data/gdp.csv}
	\node[above, align=left] at (axis cs:2017-01-01,-8.5) {\scriptsize \input{data/gdp.txt}};
	\end{axis}
\end{tikzpicture}\\
\footnotesize{Source: Bureau of Economic Analysis} 

\subsection*{\color{black!70} \seriffont Components of Growth}

\small The \textbf{expenditure approach} compiles GDP from the sum of spending on domestic goods and services. Major spending categories are consumer spending (see\cbox{yellow!80!orange}), private investment (gross spending on capital goods) and changes in private inventories (see\cbox{blue!70!black}), government spending and investment (see\cbox{cyan!50!white}), and net exports (see\cbox{green!60!black}) which is measured as foreign spending on US goods and services less US spending on goods and services produced by the rest of the world. 
\vspace{5mm}

\noindent \normalsize \textbf{Real GDP Growth by Expenditure Type}\\
\footnotesize{\textit{percentage point contribution to GDP growth}}\\
\noindent \hspace*{-2mm} \begin{tikzpicture}
	\begin{axis}[\bbar{y}{0}, \dateaxisticks ytick={-5, 0, 5},
		xticklabel={`\short{\year}}, clip=false, legend style={at={(0.95, 1.13)}}]
	\rbars
	\sbar{yellow!80!orange}{date}{DPCERY}{data/comp.csv}
	\sbar{blue!70!black}{date}{A006RY}{data/comp.csv}
	\sbar{cyan!50!white}{date}{A822RY}{data/comp.csv}
	\sbar{green!60!black}{date}{A019RY}{data/comp.csv}
	\stdnode{2.2cm}{0.4cm}{\footnotesize $^*$ Includes change in private inventories}
	\legend{Consumer Spending, Investment$^*$, Government, Net Exports};
	\end{axis}
\end{tikzpicture}\\
\footnotesize{Source: Bureau of Economic Analysis}
\end{minipage}

\newpage
\begin{minipage}{0.76\textwidth}
\small The \textbf{production approach} calculates GDP as the sum of gross value added--output minus inputs--in each sector. This identifies contributions from: goods-producing sectors combined with trade, transportation, and utilities (see\cbox{purple!70!blue}), finance, insurance, and real estate (see\cbox{red!90!white}), other service-providing sectors (see\cbox{blue!90!white}), and government (see\cbox{orange!80!white}).
\vspace{5mm}

\noindent \normalsize \textbf{Real GDP Growth by Industry Group}\\
\footnotesize{\textit{percentage point contribution to GDP growth}}\\
\noindent \hspace*{-2mm} \begin{tikzpicture}
	\begin{axis}[\bbar{y}{0}, \dateaxisticks ytick={-5, 0, 5},
		xticklabel={`\short{\year}}, clip=false, legend style={at={(0.95, 1.13)}}]
	\rbars
	\draw [dashed] (axis cs:{2004-10-01},\pgfkeysvalueof{/pgfplots/ymin}) -- (axis cs:{2004-10-01},
		\pgfkeysvalueof{/pgfplots/ymax});
	\draw [dashed] (axis cs:{1998-01-01},\pgfkeysvalueof{/pgfplots/ymin}) -- (axis cs:{1998-01-01},
		\pgfkeysvalueof{/pgfplots/ymax});
	\sbar{orange!80!white}{date}{Government}{data/gdpva.csv}
	\sbar{blue!90!white}{date}{Oth_Serv}{data/gdpva.csv}
	\sbar{red!90!white}{date}{FIRE}{data/gdpva.csv}
	\sbar{purple!70!blue}{date}{GoodsTTU}{data/gdpva.csv}
	\stdnode{0.65cm}{0.4cm}{\scriptsize historical data}
	\stdnode{4.1cm}{0.4cm}{\scriptsize annual data}
	\legend{Government, Other Services, FIRE, Goods and TTU};
	\end{axis}
\end{tikzpicture}\\
\footnotesize{Source: Bureau of Economic Analysis}
\vspace{6mm}

\small The \textbf{income approach} calculates GDP as the sum of market income to persons (in exchange for labor (see\cbox{magenta!90!blue}) or from returns on capital (see\cbox{yellow!60!orange})), indirect taxes such as sales taxes or tariffs (see\cbox{violet}), and depreciation (see\cbox{teal!60!white}). 
\vspace{5mm}

\noindent \normalsize \textbf{Real Gross Domestic Income Growth}\\
\footnotesize{\textit{percentage point contribution to GDI growth}}\\
\noindent \hspace*{-2mm} \begin{tikzpicture}
	\begin{axis}[\bbar{y}{0}, \dateaxisticks ytick={-5, 0, 5},
		xticklabel={`\short{\year}}, clip=false, 
		legend style={at={(0.95, 1.13)}}]
	\rbars
	\sbar{magenta!90!blue}{date}{A4002C}{data/gdi.csv}
	\sbar{yellow!60!orange}{date}{W271RC}{data/gdi.csv}
	\sbar{teal!60!white}{date}{A262RC}{data/gdi.csv}
	\sbar{violet}{date}{indirect}{data/gdi.csv}	
	\legend{Labor, Profit, Depreciation, Indirect Taxes};
	\end{axis}
\end{tikzpicture}\\
\footnotesize{Source: Bureau of Economic Analysis}
\vspace{6mm}

\small Changes to GDP can be assigned to changes in \textbf{household inputs}: population (see\cbox{lime}), employment rates (see\cbox{green!30!teal!90!black}), average hours worked (see\cbox{blue}), and total economy productivity (see\cbox{cyan!50!white}). 
\vspace{5mm}

\noindent \normalsize \textbf{Real GDP Growth by Inputs}\\
\footnotesize{\textit{percentage point contribution to GDP growth}}\\
\noindent \hspace*{-2mm} \begin{tikzpicture}
	\begin{axis}[\bbar{y}{0}, \dateaxisticks ytick={-5, 0, 5},
		xticklabel={`\short{\year}}, clip=false, 
		legend style={at={(0.95, 1.13)}}]
	\rbars
	\sbar{lime}{date}{pop_contr}{data/gdpjobs.csv}
	\sbar{green!30!teal!90!black}{date}{epop_contr}{data/gdpjobs.csv}
	\sbar{cyan!50!white}{date}{prod}{data/gdpjobs.csv}
	\sbar{blue}{date}{hours_contr}{data/gdpjobs.csv}	
	\legend{Population, Employment Rate, Productivity, Average Hours};
	\end{axis}
\end{tikzpicture}\\
\footnotesize{Source: Author's Calculations}
\end{minipage}

\newpage

\noindent \normalsize \textbf{Components of Economic Growth}\\
\footnotesize{\textit{percentage point contribution to real GDP/GDI growth \hspace{30mm} moving averages}\\ \vspace{4mm}
\hspace*{-2mm} \noindent \rowcolors{1}{}{black!5} \setlength{\tabcolsep}{3.5pt} \color{black!90}
		{\renewcommand{\arraystretch}{1.55}
		 \begin{tabular}{p{2mm} p{35.2mm} R{6.8mm} R{6.8mm} R{6.8mm} R{6.8mm} R{6.8mm} p{0mm} R{6.8mm} R{6.8mm} R{6.8mm} }
& & 2019 Q3 & '19 Q2 & '19 Q1 & '18 Q4 & '18 Q3 & & 3-year & 10-year & 30-year \\
\cbox{red!95!black} & \textbf{Gross Domestic Product} & 2.1 & 2.0 & 3.1 & 1.1 & 2.9 & & 2.5 &  2.3 & 2.5 \\
\cbox{yellow!80!orange} & \hspace{2mm} Consumer Spending & 1.97 & 3.03 & 0.78 & 0.97 & 2.34 & & 1.87 &  1.65 & 1.82 \\
& \hspace{4mm} Durable Goods & 0.57 & 0.87 & 0.02 & 0.09 & 0.25 & & 0.45 &  0.44 & 0.42 \\
& \hspace{4mm} Non-durable Goods  & 0.59 & 0.87 & 0.30 & 0.24 & 0.50 & & 0.41 &  0.33 & 0.34 \\
& \hspace{4mm} Services  & 0.80 & 1.29 & 0.46 & 0.65 & 1.59 & & 1.02 &  0.89 & 1.06 \\
\cbox{blue!70!black} & \hspace{2mm} Gross Investment & -0.01 & -1.16 & 1.09 & 0.53 & 2.27 & & 0.64 &  0.94 & 0.60 \\
& \hspace{4mm} Non-residential  & -0.36 & -0.14 & 0.60 & 0.64 & 0.29 & & 0.53 &  0.61 & 0.53 \\
& \hspace{4mm} Residential  & 0.18 & -0.11 & -0.04 & -0.18 & -0.16 & & 0.01 &  0.13 & 0.03 \\
& \hspace{4mm} Change in inventories  & 0.17 & -0.91 & 0.53 & 0.07 & 2.14 & & 0.10 &  0.20 & 0.04 \\
\cbox{cyan!50!white} & \hspace{2mm} Government  & 0.28 & 0.82 & 0.50 & -0.07 & 0.36 & & 0.29 &  -0.02 & 0.23 \\
& \hspace{4mm} Federal  & 0.22 & 0.53 & 0.14 & 0.07 & 0.19 & & 0.17 &  -0.01 & 0.07 \\
& \hspace{4mm} State and Local  & 0.06 & 0.29 & 0.36 & -0.14 & 0.17 & & 0.12 &  -0.01 & 0.16 \\
\cbox{green!60!black} & \hspace{2mm} Net Exports  & -0.11 & -0.68 & 0.73 & -0.35 & -2.05 & & -0.29 &  -0.26 & -0.16 \\
& \hspace{4mm} Exports  & 0.11 & -0.69 & 0.49 & 0.18 & -0.78 & & 0.24 &  0.48 & 0.49 \\
& \hspace{4mm} Imports  & -0.22 & 0.01 & 0.23 & -0.53 & -1.27 & & -0.53 &  -0.73 & -0.66 \\
& & & & & & & & & & \\
\cbox{purple!70!blue} & \hspace{2mm} Goods and TTU  & -- & 0.20 & 0.48 & 0.73 & 1.04 & & 0.78 &  0.72 & 0.90 \\
& \hspace{4mm} Manufacturing  & -- & 0.05 & -0.40 & 0.25 & 0.51 & & 0.24 &  0.21 & 0.33 \\
& \hspace{4mm} Construction  & -- & -0.01 & 0.16 & -0.14 & 0.03 & & 0.06 &  0.04 & -0.00 \\
& \hspace{4mm} Retail Trade  & -- & 0.01 & 0.46 & -0.14 & 0.16 & & 0.17 &  0.13 & 0.19 \\
\cbox{red!90!white} & \hspace{2mm} FIRE  & -- & 0.51 & 1.55 & -0.54 & 0.39 & & 0.40 &  0.41 & 0.49 \\
\cbox{blue!90!white} & \hspace{2mm} Other Services  & -- & 0.93 & 1.24 & 0.92 & 1.33 & & 1.18 &  0.97 & 0.89 \\
& \hspace{4mm} Education \& Healthcare  & -- & 0.06 & 0.37 & 0.24 & 0.27 & & 0.20 &  0.18 & 0.19 \\
& \hspace{4mm} Professional \& Business & -- & 0.78 & 0.85 & 0.31 & 0.73 & & 0.57 &  0.43 & 0.35 \\
& \hspace{4mm} Information  & -- & 0.22 & 0.08 & 0.25 & 0.26 & & 0.31 &  0.27 & 0.25 \\
\cbox{orange!80!white} & \hspace{2mm} Government  & -- & 0.37 & -0.19 & -0.02 & 0.12 & & 0.11 &  0.03 & 0.11 \\
& & & & & & & & & & \\
\cbox{lime} & \hspace{2mm} Population  & 0.68 & 0.57 & 0.55 & 0.66 & 0.70 & & 0.59 &  0.69 & 0.96 \\
\cbox{green!30!teal!90!black} & \hspace{2mm} Employment Rate  & 2.85 & -0.43 & 0.25 & 1.12 & 0.41 & & 0.80 &  0.58 & 0.05 \\
\cbox{blue} & \hspace{2mm} Average Hours & 0.93 & 0.53 & -0.11 & -0.23 & 0.12 & & 0.31 &  0.33 & 0.03 \\
\cbox{cyan!50!white} & \hspace{2mm} Productivity  & -2.34 & 1.35 & 2.40 & -0.47 & 1.70 & & 0.65 &  0.69 & 1.42 \\
& & & & & & & & & & \\& \textbf{Gross Domestic Income}  & 2.4 & 0.9 & 3.2 & 0.8 & 3.3 & & 2.3 &  2.4 & 2.5 \\
\cbox{magenta!90!blue} & \hspace{2mm} Labor  & 1.03 & 0.15 & 4.41 & 0.28 & 1.39 & & 1.41 &  1.15 & 1.29 \\
\cbox{yellow!60!orange} & \hspace{2mm} Profit  & 0.88 & 0.14 & -1.95 & -0.11 & 1.26 & & 0.23 &  0.78 & 0.65 \\
\cbox{teal!60!white} & \hspace{2mm} Depreciation  & 0.48 & 0.43 & 0.73 & 0.53 & 0.59 & & 0.46 &  0.34 & 0.42 \\
\cbox{violet} & \hspace{2mm} Indirect Taxes  & 0.01 & 0.16 & 0.06 & 0.07 & 0.05 & & 0.16 &  0.15 & 0.17 \\

		\end{tabular}
		}	\\
\footnotesize{Source: Bureau of Economic Analysis and Author's Calculations}
		
\newpage

\noindent \normalsize \textbf{Real GDP Growth by State}\\
\footnotesize{\textit{percentage point change in real GDP}}\\
\vspace{-2mm}
\hspace{-8mm} %% Creator: Matplotlib, PGF backend
%%
%% To include the figure in your LaTeX document, write
%%   \input{<filename>.pgf}
%%
%% Make sure the required packages are loaded in your preamble
%%   \usepackage{pgf}
%%
%% and, on pdftex
%%   \usepackage[utf8]{inputenc}\DeclareUnicodeCharacter{2212}{-}
%%
%% or, on luatex and xetex
%%   \usepackage{unicode-math}
%%
%% Figures using additional raster images can only be included by \input if
%% they are in the same directory as the main LaTeX file. For loading figures
%% from other directories you can use the `import` package
%%   \usepackage{import}
%%
%% and then include the figures with
%%   \import{<path to file>}{<filename>.pgf}
%%
%% Matplotlib used the following preamble
%%   \usepackage{fontspec}
%%   \setmainfont{DejaVuSerif.ttf}[Path=/home/brian/miniconda3/lib/python3.8/site-packages/matplotlib/mpl-data/fonts/ttf/]
%%   \setsansfont{DejaVuSans.ttf}[Path=/home/brian/miniconda3/lib/python3.8/site-packages/matplotlib/mpl-data/fonts/ttf/]
%%   \setmonofont{DejaVuSansMono.ttf}[Path=/home/brian/miniconda3/lib/python3.8/site-packages/matplotlib/mpl-data/fonts/ttf/]
%%
\begingroup%
\makeatletter%
\begin{pgfpicture}%
\pgfpathrectangle{\pgfpointorigin}{\pgfqpoint{6.482344in}{2.105000in}}%
\pgfusepath{use as bounding box, clip}%
\begin{pgfscope}%
\pgfsetbuttcap%
\pgfsetmiterjoin%
\pgfsetlinewidth{0.000000pt}%
\definecolor{currentstroke}{rgb}{1.000000,1.000000,1.000000}%
\pgfsetstrokecolor{currentstroke}%
\pgfsetstrokeopacity{0.000000}%
\pgfsetdash{}{0pt}%
\pgfpathmoveto{\pgfqpoint{0.000000in}{0.000000in}}%
\pgfpathlineto{\pgfqpoint{6.482344in}{0.000000in}}%
\pgfpathlineto{\pgfqpoint{6.482344in}{2.105000in}}%
\pgfpathlineto{\pgfqpoint{0.000000in}{2.105000in}}%
\pgfpathclose%
\pgfusepath{}%
\end{pgfscope}%
\begin{pgfscope}%
\pgfpathrectangle{\pgfqpoint{0.100000in}{0.100000in}}{\pgfqpoint{2.857344in}{1.829167in}}%
\pgfusepath{clip}%
\pgfsetbuttcap%
\pgfsetmiterjoin%
\definecolor{currentfill}{rgb}{0.933026,0.391311,0.271972}%
\pgfsetfillcolor{currentfill}%
\pgfsetlinewidth{0.000000pt}%
\definecolor{currentstroke}{rgb}{0.000000,0.000000,0.000000}%
\pgfsetstrokecolor{currentstroke}%
\pgfsetstrokeopacity{0.000000}%
\pgfsetdash{}{0pt}%
\pgfpathmoveto{\pgfqpoint{1.053726in}{0.528908in}}%
\pgfpathlineto{\pgfqpoint{1.045514in}{0.529042in}}%
\pgfpathlineto{\pgfqpoint{1.039924in}{0.535229in}}%
\pgfpathlineto{\pgfqpoint{1.036123in}{0.551658in}}%
\pgfpathlineto{\pgfqpoint{1.044957in}{0.553716in}}%
\pgfpathlineto{\pgfqpoint{1.053930in}{0.551503in}}%
\pgfpathlineto{\pgfqpoint{1.063062in}{0.545030in}}%
\pgfpathlineto{\pgfqpoint{1.063058in}{0.536080in}}%
\pgfpathclose%
\pgfusepath{fill}%
\end{pgfscope}%
\begin{pgfscope}%
\pgfpathrectangle{\pgfqpoint{0.100000in}{0.100000in}}{\pgfqpoint{2.857344in}{1.829167in}}%
\pgfusepath{clip}%
\pgfsetbuttcap%
\pgfsetmiterjoin%
\definecolor{currentfill}{rgb}{0.933026,0.391311,0.271972}%
\pgfsetfillcolor{currentfill}%
\pgfsetlinewidth{0.000000pt}%
\definecolor{currentstroke}{rgb}{0.000000,0.000000,0.000000}%
\pgfsetstrokecolor{currentstroke}%
\pgfsetstrokeopacity{0.000000}%
\pgfsetdash{}{0pt}%
\pgfpathmoveto{\pgfqpoint{1.103712in}{0.431369in}}%
\pgfpathlineto{\pgfqpoint{1.094525in}{0.434867in}}%
\pgfpathlineto{\pgfqpoint{1.094483in}{0.441730in}}%
\pgfpathlineto{\pgfqpoint{1.083384in}{0.447533in}}%
\pgfpathlineto{\pgfqpoint{1.084736in}{0.462145in}}%
\pgfpathlineto{\pgfqpoint{1.087951in}{0.469040in}}%
\pgfpathlineto{\pgfqpoint{1.094737in}{0.462694in}}%
\pgfpathlineto{\pgfqpoint{1.105774in}{0.463754in}}%
\pgfpathlineto{\pgfqpoint{1.102987in}{0.443558in}}%
\pgfpathclose%
\pgfusepath{fill}%
\end{pgfscope}%
\begin{pgfscope}%
\pgfpathrectangle{\pgfqpoint{0.100000in}{0.100000in}}{\pgfqpoint{2.857344in}{1.829167in}}%
\pgfusepath{clip}%
\pgfsetbuttcap%
\pgfsetmiterjoin%
\definecolor{currentfill}{rgb}{0.933026,0.391311,0.271972}%
\pgfsetfillcolor{currentfill}%
\pgfsetlinewidth{0.000000pt}%
\definecolor{currentstroke}{rgb}{0.000000,0.000000,0.000000}%
\pgfsetstrokecolor{currentstroke}%
\pgfsetstrokeopacity{0.000000}%
\pgfsetdash{}{0pt}%
\pgfpathmoveto{\pgfqpoint{1.142276in}{0.386322in}}%
\pgfpathlineto{\pgfqpoint{1.130617in}{0.388307in}}%
\pgfpathlineto{\pgfqpoint{1.124169in}{0.398288in}}%
\pgfpathlineto{\pgfqpoint{1.123259in}{0.406495in}}%
\pgfpathclose%
\pgfusepath{fill}%
\end{pgfscope}%
\begin{pgfscope}%
\pgfpathrectangle{\pgfqpoint{0.100000in}{0.100000in}}{\pgfqpoint{2.857344in}{1.829167in}}%
\pgfusepath{clip}%
\pgfsetbuttcap%
\pgfsetmiterjoin%
\definecolor{currentfill}{rgb}{0.933026,0.391311,0.271972}%
\pgfsetfillcolor{currentfill}%
\pgfsetlinewidth{0.000000pt}%
\definecolor{currentstroke}{rgb}{0.000000,0.000000,0.000000}%
\pgfsetstrokecolor{currentstroke}%
\pgfsetstrokeopacity{0.000000}%
\pgfsetdash{}{0pt}%
\pgfpathmoveto{\pgfqpoint{1.118383in}{0.388062in}}%
\pgfpathlineto{\pgfqpoint{1.124496in}{0.383100in}}%
\pgfpathlineto{\pgfqpoint{1.125823in}{0.374568in}}%
\pgfpathlineto{\pgfqpoint{1.113518in}{0.375040in}}%
\pgfpathclose%
\pgfusepath{fill}%
\end{pgfscope}%
\begin{pgfscope}%
\pgfpathrectangle{\pgfqpoint{0.100000in}{0.100000in}}{\pgfqpoint{2.857344in}{1.829167in}}%
\pgfusepath{clip}%
\pgfsetbuttcap%
\pgfsetmiterjoin%
\definecolor{currentfill}{rgb}{0.933026,0.391311,0.271972}%
\pgfsetfillcolor{currentfill}%
\pgfsetlinewidth{0.000000pt}%
\definecolor{currentstroke}{rgb}{0.000000,0.000000,0.000000}%
\pgfsetstrokecolor{currentstroke}%
\pgfsetstrokeopacity{0.000000}%
\pgfsetdash{}{0pt}%
\pgfpathmoveto{\pgfqpoint{1.146456in}{0.340406in}}%
\pgfpathlineto{\pgfqpoint{1.134176in}{0.344820in}}%
\pgfpathlineto{\pgfqpoint{1.137695in}{0.355522in}}%
\pgfpathlineto{\pgfqpoint{1.132548in}{0.366426in}}%
\pgfpathlineto{\pgfqpoint{1.134164in}{0.373922in}}%
\pgfpathlineto{\pgfqpoint{1.144046in}{0.376780in}}%
\pgfpathlineto{\pgfqpoint{1.143856in}{0.364770in}}%
\pgfpathlineto{\pgfqpoint{1.155544in}{0.358108in}}%
\pgfpathlineto{\pgfqpoint{1.162229in}{0.342003in}}%
\pgfpathlineto{\pgfqpoint{1.154779in}{0.335800in}}%
\pgfpathclose%
\pgfusepath{fill}%
\end{pgfscope}%
\begin{pgfscope}%
\pgfpathrectangle{\pgfqpoint{0.100000in}{0.100000in}}{\pgfqpoint{2.857344in}{1.829167in}}%
\pgfusepath{clip}%
\pgfsetbuttcap%
\pgfsetmiterjoin%
\definecolor{currentfill}{rgb}{0.933026,0.391311,0.271972}%
\pgfsetfillcolor{currentfill}%
\pgfsetlinewidth{0.000000pt}%
\definecolor{currentstroke}{rgb}{0.000000,0.000000,0.000000}%
\pgfsetstrokecolor{currentstroke}%
\pgfsetstrokeopacity{0.000000}%
\pgfsetdash{}{0pt}%
\pgfpathmoveto{\pgfqpoint{1.102106in}{0.231508in}}%
\pgfpathlineto{\pgfqpoint{1.096256in}{0.243594in}}%
\pgfpathlineto{\pgfqpoint{1.098846in}{0.251722in}}%
\pgfpathlineto{\pgfqpoint{1.110321in}{0.262974in}}%
\pgfpathlineto{\pgfqpoint{1.111407in}{0.271599in}}%
\pgfpathlineto{\pgfqpoint{1.118917in}{0.290695in}}%
\pgfpathlineto{\pgfqpoint{1.138668in}{0.293958in}}%
\pgfpathlineto{\pgfqpoint{1.142448in}{0.305136in}}%
\pgfpathlineto{\pgfqpoint{1.151156in}{0.300945in}}%
\pgfpathlineto{\pgfqpoint{1.168851in}{0.269573in}}%
\pgfpathlineto{\pgfqpoint{1.169074in}{0.259331in}}%
\pgfpathlineto{\pgfqpoint{1.164073in}{0.250466in}}%
\pgfpathlineto{\pgfqpoint{1.170419in}{0.230112in}}%
\pgfpathlineto{\pgfqpoint{1.165984in}{0.226831in}}%
\pgfpathlineto{\pgfqpoint{1.150880in}{0.227785in}}%
\pgfpathlineto{\pgfqpoint{1.132717in}{0.233855in}}%
\pgfpathlineto{\pgfqpoint{1.117451in}{0.236362in}}%
\pgfpathlineto{\pgfqpoint{1.113374in}{0.232800in}}%
\pgfpathclose%
\pgfusepath{fill}%
\end{pgfscope}%
\begin{pgfscope}%
\pgfpathrectangle{\pgfqpoint{0.100000in}{0.100000in}}{\pgfqpoint{2.857344in}{1.829167in}}%
\pgfusepath{clip}%
\pgfsetbuttcap%
\pgfsetmiterjoin%
\definecolor{currentfill}{rgb}{0.874740,0.949712,0.601615}%
\pgfsetfillcolor{currentfill}%
\pgfsetlinewidth{0.000000pt}%
\definecolor{currentstroke}{rgb}{0.000000,0.000000,0.000000}%
\pgfsetstrokecolor{currentstroke}%
\pgfsetstrokeopacity{0.000000}%
\pgfsetdash{}{0pt}%
\pgfpathmoveto{\pgfqpoint{0.778933in}{1.821590in}}%
\pgfpathlineto{\pgfqpoint{0.764115in}{1.762876in}}%
\pgfpathlineto{\pgfqpoint{0.753579in}{1.720807in}}%
\pgfpathlineto{\pgfqpoint{0.741052in}{1.670200in}}%
\pgfpathlineto{\pgfqpoint{0.738592in}{1.657726in}}%
\pgfpathlineto{\pgfqpoint{0.740198in}{1.646285in}}%
\pgfpathlineto{\pgfqpoint{0.738069in}{1.635747in}}%
\pgfpathlineto{\pgfqpoint{0.650622in}{1.658566in}}%
\pgfpathlineto{\pgfqpoint{0.642722in}{1.655916in}}%
\pgfpathlineto{\pgfqpoint{0.635962in}{1.658216in}}%
\pgfpathlineto{\pgfqpoint{0.604663in}{1.658968in}}%
\pgfpathlineto{\pgfqpoint{0.594077in}{1.655983in}}%
\pgfpathlineto{\pgfqpoint{0.583565in}{1.656991in}}%
\pgfpathlineto{\pgfqpoint{0.579110in}{1.661663in}}%
\pgfpathlineto{\pgfqpoint{0.551088in}{1.660644in}}%
\pgfpathlineto{\pgfqpoint{0.546663in}{1.668278in}}%
\pgfpathlineto{\pgfqpoint{0.538262in}{1.672300in}}%
\pgfpathlineto{\pgfqpoint{0.526039in}{1.674751in}}%
\pgfpathlineto{\pgfqpoint{0.504941in}{1.671163in}}%
\pgfpathlineto{\pgfqpoint{0.497153in}{1.674678in}}%
\pgfpathlineto{\pgfqpoint{0.485145in}{1.684029in}}%
\pgfpathlineto{\pgfqpoint{0.488754in}{1.702356in}}%
\pgfpathlineto{\pgfqpoint{0.487489in}{1.711715in}}%
\pgfpathlineto{\pgfqpoint{0.477972in}{1.721650in}}%
\pgfpathlineto{\pgfqpoint{0.471876in}{1.721034in}}%
\pgfpathlineto{\pgfqpoint{0.467503in}{1.731084in}}%
\pgfpathlineto{\pgfqpoint{0.457144in}{1.735146in}}%
\pgfpathlineto{\pgfqpoint{0.449614in}{1.734629in}}%
\pgfpathlineto{\pgfqpoint{0.447179in}{1.744951in}}%
\pgfpathlineto{\pgfqpoint{0.454680in}{1.743852in}}%
\pgfpathlineto{\pgfqpoint{0.454084in}{1.758199in}}%
\pgfpathlineto{\pgfqpoint{0.450815in}{1.766717in}}%
\pgfpathlineto{\pgfqpoint{0.452256in}{1.777620in}}%
\pgfpathlineto{\pgfqpoint{0.456180in}{1.785821in}}%
\pgfpathlineto{\pgfqpoint{0.455959in}{1.801382in}}%
\pgfpathlineto{\pgfqpoint{0.453864in}{1.807028in}}%
\pgfpathlineto{\pgfqpoint{0.457482in}{1.825143in}}%
\pgfpathlineto{\pgfqpoint{0.456438in}{1.836817in}}%
\pgfpathlineto{\pgfqpoint{0.452820in}{1.842414in}}%
\pgfpathlineto{\pgfqpoint{0.453372in}{1.860716in}}%
\pgfpathlineto{\pgfqpoint{0.458634in}{1.874176in}}%
\pgfpathlineto{\pgfqpoint{0.464357in}{1.870882in}}%
\pgfpathlineto{\pgfqpoint{0.483307in}{1.851307in}}%
\pgfpathlineto{\pgfqpoint{0.506269in}{1.840594in}}%
\pgfpathlineto{\pgfqpoint{0.518027in}{1.839327in}}%
\pgfpathlineto{\pgfqpoint{0.529104in}{1.834766in}}%
\pgfpathlineto{\pgfqpoint{0.532248in}{1.819419in}}%
\pgfpathlineto{\pgfqpoint{0.522540in}{1.816512in}}%
\pgfpathlineto{\pgfqpoint{0.516108in}{1.804417in}}%
\pgfpathlineto{\pgfqpoint{0.523661in}{1.805089in}}%
\pgfpathlineto{\pgfqpoint{0.526706in}{1.810518in}}%
\pgfpathlineto{\pgfqpoint{0.537358in}{1.817307in}}%
\pgfpathlineto{\pgfqpoint{0.536808in}{1.807225in}}%
\pgfpathlineto{\pgfqpoint{0.529696in}{1.805600in}}%
\pgfpathlineto{\pgfqpoint{0.530877in}{1.792641in}}%
\pgfpathlineto{\pgfqpoint{0.522216in}{1.779660in}}%
\pgfpathlineto{\pgfqpoint{0.516673in}{1.785833in}}%
\pgfpathlineto{\pgfqpoint{0.500201in}{1.782477in}}%
\pgfpathlineto{\pgfqpoint{0.499334in}{1.774925in}}%
\pgfpathlineto{\pgfqpoint{0.505013in}{1.770325in}}%
\pgfpathlineto{\pgfqpoint{0.513676in}{1.769872in}}%
\pgfpathlineto{\pgfqpoint{0.525654in}{1.779703in}}%
\pgfpathlineto{\pgfqpoint{0.535135in}{1.780520in}}%
\pgfpathlineto{\pgfqpoint{0.535499in}{1.791402in}}%
\pgfpathlineto{\pgfqpoint{0.540385in}{1.807385in}}%
\pgfpathlineto{\pgfqpoint{0.551964in}{1.819272in}}%
\pgfpathlineto{\pgfqpoint{0.548103in}{1.828479in}}%
\pgfpathlineto{\pgfqpoint{0.550740in}{1.838391in}}%
\pgfpathlineto{\pgfqpoint{0.545603in}{1.848271in}}%
\pgfpathlineto{\pgfqpoint{0.552924in}{1.860904in}}%
\pgfpathlineto{\pgfqpoint{0.554021in}{1.868504in}}%
\pgfpathlineto{\pgfqpoint{0.547665in}{1.873500in}}%
\pgfpathlineto{\pgfqpoint{0.548723in}{1.886278in}}%
\pgfpathlineto{\pgfqpoint{0.624882in}{1.863139in}}%
\pgfpathlineto{\pgfqpoint{0.705756in}{1.840460in}}%
\pgfpathclose%
\pgfusepath{fill}%
\end{pgfscope}%
\begin{pgfscope}%
\pgfpathrectangle{\pgfqpoint{0.100000in}{0.100000in}}{\pgfqpoint{2.857344in}{1.829167in}}%
\pgfusepath{clip}%
\pgfsetbuttcap%
\pgfsetmiterjoin%
\definecolor{currentfill}{rgb}{0.874740,0.949712,0.601615}%
\pgfsetfillcolor{currentfill}%
\pgfsetlinewidth{0.000000pt}%
\definecolor{currentstroke}{rgb}{0.000000,0.000000,0.000000}%
\pgfsetstrokecolor{currentstroke}%
\pgfsetstrokeopacity{0.000000}%
\pgfsetdash{}{0pt}%
\pgfpathmoveto{\pgfqpoint{0.536195in}{1.842073in}}%
\pgfpathlineto{\pgfqpoint{0.539945in}{1.827616in}}%
\pgfpathlineto{\pgfqpoint{0.545825in}{1.822813in}}%
\pgfpathlineto{\pgfqpoint{0.541133in}{1.816700in}}%
\pgfpathlineto{\pgfqpoint{0.536560in}{1.825658in}}%
\pgfpathclose%
\pgfusepath{fill}%
\end{pgfscope}%
\begin{pgfscope}%
\pgfpathrectangle{\pgfqpoint{0.100000in}{0.100000in}}{\pgfqpoint{2.857344in}{1.829167in}}%
\pgfusepath{clip}%
\pgfsetbuttcap%
\pgfsetmiterjoin%
\definecolor{currentfill}{rgb}{0.998385,0.949942,0.665052}%
\pgfsetfillcolor{currentfill}%
\pgfsetlinewidth{0.000000pt}%
\definecolor{currentstroke}{rgb}{0.000000,0.000000,0.000000}%
\pgfsetstrokecolor{currentstroke}%
\pgfsetstrokeopacity{0.000000}%
\pgfsetdash{}{0pt}%
\pgfpathmoveto{\pgfqpoint{0.818206in}{1.812100in}}%
\pgfpathlineto{\pgfqpoint{0.899616in}{1.793831in}}%
\pgfpathlineto{\pgfqpoint{0.976286in}{1.778333in}}%
\pgfpathlineto{\pgfqpoint{1.035275in}{1.767525in}}%
\pgfpathlineto{\pgfqpoint{1.086702in}{1.758885in}}%
\pgfpathlineto{\pgfqpoint{1.138243in}{1.750954in}}%
\pgfpathlineto{\pgfqpoint{1.182136in}{1.744771in}}%
\pgfpathlineto{\pgfqpoint{1.226097in}{1.739101in}}%
\pgfpathlineto{\pgfqpoint{1.270122in}{1.733945in}}%
\pgfpathlineto{\pgfqpoint{1.311610in}{1.729564in}}%
\pgfpathlineto{\pgfqpoint{1.305904in}{1.666336in}}%
\pgfpathlineto{\pgfqpoint{1.297411in}{1.580891in}}%
\pgfpathlineto{\pgfqpoint{1.292957in}{1.536946in}}%
\pgfpathlineto{\pgfqpoint{1.286551in}{1.477721in}}%
\pgfpathlineto{\pgfqpoint{1.241200in}{1.482673in}}%
\pgfpathlineto{\pgfqpoint{1.189279in}{1.488567in}}%
\pgfpathlineto{\pgfqpoint{1.117186in}{1.498434in}}%
\pgfpathlineto{\pgfqpoint{1.084987in}{1.503027in}}%
\pgfpathlineto{\pgfqpoint{1.005657in}{1.515479in}}%
\pgfpathlineto{\pgfqpoint{0.978349in}{1.520475in}}%
\pgfpathlineto{\pgfqpoint{0.972657in}{1.488114in}}%
\pgfpathlineto{\pgfqpoint{0.969546in}{1.490423in}}%
\pgfpathlineto{\pgfqpoint{0.963698in}{1.505991in}}%
\pgfpathlineto{\pgfqpoint{0.956181in}{1.505032in}}%
\pgfpathlineto{\pgfqpoint{0.950800in}{1.497162in}}%
\pgfpathlineto{\pgfqpoint{0.939635in}{1.496109in}}%
\pgfpathlineto{\pgfqpoint{0.937392in}{1.499875in}}%
\pgfpathlineto{\pgfqpoint{0.926933in}{1.499419in}}%
\pgfpathlineto{\pgfqpoint{0.921675in}{1.503065in}}%
\pgfpathlineto{\pgfqpoint{0.914323in}{1.497424in}}%
\pgfpathlineto{\pgfqpoint{0.896466in}{1.502500in}}%
\pgfpathlineto{\pgfqpoint{0.888673in}{1.499774in}}%
\pgfpathlineto{\pgfqpoint{0.884480in}{1.511871in}}%
\pgfpathlineto{\pgfqpoint{0.884284in}{1.523380in}}%
\pgfpathlineto{\pgfqpoint{0.871975in}{1.531657in}}%
\pgfpathlineto{\pgfqpoint{0.873982in}{1.543586in}}%
\pgfpathlineto{\pgfqpoint{0.865217in}{1.563291in}}%
\pgfpathlineto{\pgfqpoint{0.865808in}{1.580042in}}%
\pgfpathlineto{\pgfqpoint{0.858275in}{1.588240in}}%
\pgfpathlineto{\pgfqpoint{0.851206in}{1.580987in}}%
\pgfpathlineto{\pgfqpoint{0.841586in}{1.577064in}}%
\pgfpathlineto{\pgfqpoint{0.834263in}{1.585150in}}%
\pgfpathlineto{\pgfqpoint{0.835453in}{1.598061in}}%
\pgfpathlineto{\pgfqpoint{0.844236in}{1.603148in}}%
\pgfpathlineto{\pgfqpoint{0.841901in}{1.611328in}}%
\pgfpathlineto{\pgfqpoint{0.857168in}{1.650923in}}%
\pgfpathlineto{\pgfqpoint{0.845130in}{1.651915in}}%
\pgfpathlineto{\pgfqpoint{0.843893in}{1.659021in}}%
\pgfpathlineto{\pgfqpoint{0.835124in}{1.665148in}}%
\pgfpathlineto{\pgfqpoint{0.835684in}{1.671997in}}%
\pgfpathlineto{\pgfqpoint{0.831059in}{1.677319in}}%
\pgfpathlineto{\pgfqpoint{0.823529in}{1.696639in}}%
\pgfpathlineto{\pgfqpoint{0.816648in}{1.700579in}}%
\pgfpathlineto{\pgfqpoint{0.811720in}{1.717253in}}%
\pgfpathlineto{\pgfqpoint{0.812883in}{1.728336in}}%
\pgfpathlineto{\pgfqpoint{0.803691in}{1.748749in}}%
\pgfpathclose%
\pgfusepath{fill}%
\end{pgfscope}%
\begin{pgfscope}%
\pgfpathrectangle{\pgfqpoint{0.100000in}{0.100000in}}{\pgfqpoint{2.857344in}{1.829167in}}%
\pgfusepath{clip}%
\pgfsetbuttcap%
\pgfsetmiterjoin%
\definecolor{currentfill}{rgb}{0.997001,0.907036,0.593080}%
\pgfsetfillcolor{currentfill}%
\pgfsetlinewidth{0.000000pt}%
\definecolor{currentstroke}{rgb}{0.000000,0.000000,0.000000}%
\pgfsetstrokecolor{currentstroke}%
\pgfsetstrokeopacity{0.000000}%
\pgfsetdash{}{0pt}%
\pgfpathmoveto{\pgfqpoint{2.783503in}{1.486275in}}%
\pgfpathlineto{\pgfqpoint{2.781874in}{1.493203in}}%
\pgfpathlineto{\pgfqpoint{2.772818in}{1.499273in}}%
\pgfpathlineto{\pgfqpoint{2.748887in}{1.577551in}}%
\pgfpathlineto{\pgfqpoint{2.735989in}{1.615326in}}%
\pgfpathlineto{\pgfqpoint{2.746358in}{1.626209in}}%
\pgfpathlineto{\pgfqpoint{2.756439in}{1.652843in}}%
\pgfpathlineto{\pgfqpoint{2.761466in}{1.661354in}}%
\pgfpathlineto{\pgfqpoint{2.757905in}{1.664929in}}%
\pgfpathlineto{\pgfqpoint{2.756712in}{1.688571in}}%
\pgfpathlineto{\pgfqpoint{2.761210in}{1.695818in}}%
\pgfpathlineto{\pgfqpoint{2.759238in}{1.712665in}}%
\pgfpathlineto{\pgfqpoint{2.777189in}{1.767935in}}%
\pgfpathlineto{\pgfqpoint{2.785277in}{1.768237in}}%
\pgfpathlineto{\pgfqpoint{2.788634in}{1.758364in}}%
\pgfpathlineto{\pgfqpoint{2.795822in}{1.755529in}}%
\pgfpathlineto{\pgfqpoint{2.809400in}{1.767142in}}%
\pgfpathlineto{\pgfqpoint{2.820047in}{1.774001in}}%
\pgfpathlineto{\pgfqpoint{2.843506in}{1.761923in}}%
\pgfpathlineto{\pgfqpoint{2.864699in}{1.694859in}}%
\pgfpathlineto{\pgfqpoint{2.868728in}{1.678360in}}%
\pgfpathlineto{\pgfqpoint{2.878011in}{1.676425in}}%
\pgfpathlineto{\pgfqpoint{2.890601in}{1.665219in}}%
\pgfpathlineto{\pgfqpoint{2.889888in}{1.658690in}}%
\pgfpathlineto{\pgfqpoint{2.898465in}{1.650951in}}%
\pgfpathlineto{\pgfqpoint{2.905432in}{1.655391in}}%
\pgfpathlineto{\pgfqpoint{2.920010in}{1.638389in}}%
\pgfpathlineto{\pgfqpoint{2.913495in}{1.624782in}}%
\pgfpathlineto{\pgfqpoint{2.904757in}{1.624512in}}%
\pgfpathlineto{\pgfqpoint{2.897747in}{1.612366in}}%
\pgfpathlineto{\pgfqpoint{2.889256in}{1.610643in}}%
\pgfpathlineto{\pgfqpoint{2.883022in}{1.604175in}}%
\pgfpathlineto{\pgfqpoint{2.864402in}{1.597239in}}%
\pgfpathlineto{\pgfqpoint{2.853133in}{1.586061in}}%
\pgfpathlineto{\pgfqpoint{2.847408in}{1.594084in}}%
\pgfpathlineto{\pgfqpoint{2.842206in}{1.588334in}}%
\pgfpathlineto{\pgfqpoint{2.843631in}{1.565087in}}%
\pgfpathlineto{\pgfqpoint{2.839512in}{1.555879in}}%
\pgfpathlineto{\pgfqpoint{2.830528in}{1.558394in}}%
\pgfpathlineto{\pgfqpoint{2.829078in}{1.548953in}}%
\pgfpathlineto{\pgfqpoint{2.825196in}{1.545071in}}%
\pgfpathlineto{\pgfqpoint{2.817426in}{1.546017in}}%
\pgfpathlineto{\pgfqpoint{2.817916in}{1.537191in}}%
\pgfpathlineto{\pgfqpoint{2.806140in}{1.539698in}}%
\pgfpathlineto{\pgfqpoint{2.799709in}{1.527468in}}%
\pgfpathlineto{\pgfqpoint{2.802137in}{1.521070in}}%
\pgfpathlineto{\pgfqpoint{2.798312in}{1.510427in}}%
\pgfpathlineto{\pgfqpoint{2.792289in}{1.502600in}}%
\pgfpathlineto{\pgfqpoint{2.790774in}{1.486259in}}%
\pgfpathclose%
\pgfusepath{fill}%
\end{pgfscope}%
\begin{pgfscope}%
\pgfpathrectangle{\pgfqpoint{0.100000in}{0.100000in}}{\pgfqpoint{2.857344in}{1.829167in}}%
\pgfusepath{clip}%
\pgfsetbuttcap%
\pgfsetmiterjoin%
\definecolor{currentfill}{rgb}{0.997001,0.907036,0.593080}%
\pgfsetfillcolor{currentfill}%
\pgfsetlinewidth{0.000000pt}%
\definecolor{currentstroke}{rgb}{0.000000,0.000000,0.000000}%
\pgfsetstrokecolor{currentstroke}%
\pgfsetstrokeopacity{0.000000}%
\pgfsetdash{}{0pt}%
\pgfpathmoveto{\pgfqpoint{2.867747in}{1.592465in}}%
\pgfpathlineto{\pgfqpoint{2.873063in}{1.598034in}}%
\pgfpathlineto{\pgfqpoint{2.878125in}{1.592800in}}%
\pgfpathlineto{\pgfqpoint{2.869033in}{1.585865in}}%
\pgfpathclose%
\pgfusepath{fill}%
\end{pgfscope}%
\begin{pgfscope}%
\pgfpathrectangle{\pgfqpoint{0.100000in}{0.100000in}}{\pgfqpoint{2.857344in}{1.829167in}}%
\pgfusepath{clip}%
\pgfsetbuttcap%
\pgfsetmiterjoin%
\definecolor{currentfill}{rgb}{0.999769,0.992849,0.737024}%
\pgfsetfillcolor{currentfill}%
\pgfsetlinewidth{0.000000pt}%
\definecolor{currentstroke}{rgb}{0.000000,0.000000,0.000000}%
\pgfsetstrokecolor{currentstroke}%
\pgfsetstrokeopacity{0.000000}%
\pgfsetdash{}{0pt}%
\pgfpathmoveto{\pgfqpoint{1.292957in}{1.536946in}}%
\pgfpathlineto{\pgfqpoint{1.297411in}{1.580891in}}%
\pgfpathlineto{\pgfqpoint{1.305904in}{1.666336in}}%
\pgfpathlineto{\pgfqpoint{1.311610in}{1.729564in}}%
\pgfpathlineto{\pgfqpoint{1.358337in}{1.725180in}}%
\pgfpathlineto{\pgfqpoint{1.418116in}{1.720421in}}%
\pgfpathlineto{\pgfqpoint{1.472756in}{1.716901in}}%
\pgfpathlineto{\pgfqpoint{1.522231in}{1.714398in}}%
\pgfpathlineto{\pgfqpoint{1.596066in}{1.711865in}}%
\pgfpathlineto{\pgfqpoint{1.601105in}{1.691056in}}%
\pgfpathlineto{\pgfqpoint{1.599301in}{1.681111in}}%
\pgfpathlineto{\pgfqpoint{1.598707in}{1.660670in}}%
\pgfpathlineto{\pgfqpoint{1.602135in}{1.645362in}}%
\pgfpathlineto{\pgfqpoint{1.609989in}{1.622777in}}%
\pgfpathlineto{\pgfqpoint{1.609971in}{1.594344in}}%
\pgfpathlineto{\pgfqpoint{1.611425in}{1.561312in}}%
\pgfpathlineto{\pgfqpoint{1.613420in}{1.552410in}}%
\pgfpathlineto{\pgfqpoint{1.619257in}{1.542637in}}%
\pgfpathlineto{\pgfqpoint{1.621182in}{1.527408in}}%
\pgfpathlineto{\pgfqpoint{1.620351in}{1.517241in}}%
\pgfpathlineto{\pgfqpoint{1.558454in}{1.518582in}}%
\pgfpathlineto{\pgfqpoint{1.513424in}{1.520865in}}%
\pgfpathlineto{\pgfqpoint{1.447414in}{1.524379in}}%
\pgfpathlineto{\pgfqpoint{1.382318in}{1.529003in}}%
\pgfpathlineto{\pgfqpoint{1.338961in}{1.532501in}}%
\pgfpathclose%
\pgfusepath{fill}%
\end{pgfscope}%
\begin{pgfscope}%
\pgfpathrectangle{\pgfqpoint{0.100000in}{0.100000in}}{\pgfqpoint{2.857344in}{1.829167in}}%
\pgfusepath{clip}%
\pgfsetbuttcap%
\pgfsetmiterjoin%
\definecolor{currentfill}{rgb}{0.874740,0.949712,0.601615}%
\pgfsetfillcolor{currentfill}%
\pgfsetlinewidth{0.000000pt}%
\definecolor{currentstroke}{rgb}{0.000000,0.000000,0.000000}%
\pgfsetstrokecolor{currentstroke}%
\pgfsetstrokeopacity{0.000000}%
\pgfsetdash{}{0pt}%
\pgfpathmoveto{\pgfqpoint{1.274247in}{1.353032in}}%
\pgfpathlineto{\pgfqpoint{1.281393in}{1.426697in}}%
\pgfpathlineto{\pgfqpoint{1.286551in}{1.477721in}}%
\pgfpathlineto{\pgfqpoint{1.292957in}{1.536946in}}%
\pgfpathlineto{\pgfqpoint{1.338961in}{1.532501in}}%
\pgfpathlineto{\pgfqpoint{1.382318in}{1.529003in}}%
\pgfpathlineto{\pgfqpoint{1.447414in}{1.524379in}}%
\pgfpathlineto{\pgfqpoint{1.513424in}{1.520865in}}%
\pgfpathlineto{\pgfqpoint{1.558454in}{1.518582in}}%
\pgfpathlineto{\pgfqpoint{1.620351in}{1.517241in}}%
\pgfpathlineto{\pgfqpoint{1.616160in}{1.505012in}}%
\pgfpathlineto{\pgfqpoint{1.607796in}{1.495412in}}%
\pgfpathlineto{\pgfqpoint{1.614201in}{1.484359in}}%
\pgfpathlineto{\pgfqpoint{1.621264in}{1.481997in}}%
\pgfpathlineto{\pgfqpoint{1.624616in}{1.475648in}}%
\pgfpathlineto{\pgfqpoint{1.623847in}{1.429290in}}%
\pgfpathlineto{\pgfqpoint{1.622559in}{1.364049in}}%
\pgfpathlineto{\pgfqpoint{1.617795in}{1.346914in}}%
\pgfpathlineto{\pgfqpoint{1.622056in}{1.337435in}}%
\pgfpathlineto{\pgfqpoint{1.613501in}{1.317067in}}%
\pgfpathlineto{\pgfqpoint{1.622514in}{1.300647in}}%
\pgfpathlineto{\pgfqpoint{1.614856in}{1.301904in}}%
\pgfpathlineto{\pgfqpoint{1.609632in}{1.312120in}}%
\pgfpathlineto{\pgfqpoint{1.590986in}{1.319122in}}%
\pgfpathlineto{\pgfqpoint{1.579228in}{1.325282in}}%
\pgfpathlineto{\pgfqpoint{1.559496in}{1.325794in}}%
\pgfpathlineto{\pgfqpoint{1.552647in}{1.320183in}}%
\pgfpathlineto{\pgfqpoint{1.530313in}{1.331232in}}%
\pgfpathlineto{\pgfqpoint{1.528600in}{1.334727in}}%
\pgfpathlineto{\pgfqpoint{1.450656in}{1.338416in}}%
\pgfpathlineto{\pgfqpoint{1.403301in}{1.341130in}}%
\pgfpathlineto{\pgfqpoint{1.364186in}{1.344193in}}%
\pgfpathlineto{\pgfqpoint{1.299554in}{1.350300in}}%
\pgfpathclose%
\pgfusepath{fill}%
\end{pgfscope}%
\begin{pgfscope}%
\pgfpathrectangle{\pgfqpoint{0.100000in}{0.100000in}}{\pgfqpoint{2.857344in}{1.829167in}}%
\pgfusepath{clip}%
\pgfsetbuttcap%
\pgfsetmiterjoin%
\definecolor{currentfill}{rgb}{0.974856,0.557401,0.322722}%
\pgfsetfillcolor{currentfill}%
\pgfsetlinewidth{0.000000pt}%
\definecolor{currentstroke}{rgb}{0.000000,0.000000,0.000000}%
\pgfsetstrokecolor{currentstroke}%
\pgfsetstrokeopacity{0.000000}%
\pgfsetdash{}{0pt}%
\pgfpathmoveto{\pgfqpoint{1.261989in}{1.228296in}}%
\pgfpathlineto{\pgfqpoint{1.220435in}{1.232091in}}%
\pgfpathlineto{\pgfqpoint{1.129859in}{1.243270in}}%
\pgfpathlineto{\pgfqpoint{1.080586in}{1.250342in}}%
\pgfpathlineto{\pgfqpoint{1.027739in}{1.257928in}}%
\pgfpathlineto{\pgfqpoint{0.983224in}{1.265052in}}%
\pgfpathlineto{\pgfqpoint{0.934362in}{1.273417in}}%
\pgfpathlineto{\pgfqpoint{0.945471in}{1.335044in}}%
\pgfpathlineto{\pgfqpoint{0.956725in}{1.398237in}}%
\pgfpathlineto{\pgfqpoint{0.972657in}{1.488114in}}%
\pgfpathlineto{\pgfqpoint{0.978349in}{1.520475in}}%
\pgfpathlineto{\pgfqpoint{1.005657in}{1.515479in}}%
\pgfpathlineto{\pgfqpoint{1.084987in}{1.503027in}}%
\pgfpathlineto{\pgfqpoint{1.117186in}{1.498434in}}%
\pgfpathlineto{\pgfqpoint{1.189279in}{1.488567in}}%
\pgfpathlineto{\pgfqpoint{1.241200in}{1.482673in}}%
\pgfpathlineto{\pgfqpoint{1.286551in}{1.477721in}}%
\pgfpathlineto{\pgfqpoint{1.281393in}{1.426697in}}%
\pgfpathlineto{\pgfqpoint{1.274247in}{1.353032in}}%
\pgfpathlineto{\pgfqpoint{1.268113in}{1.290427in}}%
\pgfpathclose%
\pgfusepath{fill}%
\end{pgfscope}%
\begin{pgfscope}%
\pgfpathrectangle{\pgfqpoint{0.100000in}{0.100000in}}{\pgfqpoint{2.857344in}{1.829167in}}%
\pgfusepath{clip}%
\pgfsetbuttcap%
\pgfsetmiterjoin%
\definecolor{currentfill}{rgb}{0.998078,0.940408,0.649058}%
\pgfsetfillcolor{currentfill}%
\pgfsetlinewidth{0.000000pt}%
\definecolor{currentstroke}{rgb}{0.000000,0.000000,0.000000}%
\pgfsetstrokecolor{currentstroke}%
\pgfsetstrokeopacity{0.000000}%
\pgfsetdash{}{0pt}%
\pgfpathmoveto{\pgfqpoint{2.021726in}{1.313305in}}%
\pgfpathlineto{\pgfqpoint{1.969068in}{1.309564in}}%
\pgfpathlineto{\pgfqpoint{1.890579in}{1.306219in}}%
\pgfpathlineto{\pgfqpoint{1.887596in}{1.314147in}}%
\pgfpathlineto{\pgfqpoint{1.870191in}{1.320074in}}%
\pgfpathlineto{\pgfqpoint{1.866352in}{1.331272in}}%
\pgfpathlineto{\pgfqpoint{1.864744in}{1.345124in}}%
\pgfpathlineto{\pgfqpoint{1.868667in}{1.352221in}}%
\pgfpathlineto{\pgfqpoint{1.862472in}{1.359032in}}%
\pgfpathlineto{\pgfqpoint{1.860978in}{1.367156in}}%
\pgfpathlineto{\pgfqpoint{1.858981in}{1.385126in}}%
\pgfpathlineto{\pgfqpoint{1.853044in}{1.394887in}}%
\pgfpathlineto{\pgfqpoint{1.842500in}{1.400366in}}%
\pgfpathlineto{\pgfqpoint{1.831027in}{1.409407in}}%
\pgfpathlineto{\pgfqpoint{1.825096in}{1.420104in}}%
\pgfpathlineto{\pgfqpoint{1.814454in}{1.424411in}}%
\pgfpathlineto{\pgfqpoint{1.808198in}{1.431422in}}%
\pgfpathlineto{\pgfqpoint{1.800619in}{1.432612in}}%
\pgfpathlineto{\pgfqpoint{1.787088in}{1.443021in}}%
\pgfpathlineto{\pgfqpoint{1.789286in}{1.455000in}}%
\pgfpathlineto{\pgfqpoint{1.788862in}{1.477780in}}%
\pgfpathlineto{\pgfqpoint{1.793035in}{1.484051in}}%
\pgfpathlineto{\pgfqpoint{1.789286in}{1.493519in}}%
\pgfpathlineto{\pgfqpoint{1.782682in}{1.495350in}}%
\pgfpathlineto{\pgfqpoint{1.783227in}{1.503666in}}%
\pgfpathlineto{\pgfqpoint{1.791422in}{1.516803in}}%
\pgfpathlineto{\pgfqpoint{1.807649in}{1.527191in}}%
\pgfpathlineto{\pgfqpoint{1.806638in}{1.564135in}}%
\pgfpathlineto{\pgfqpoint{1.814758in}{1.569684in}}%
\pgfpathlineto{\pgfqpoint{1.822442in}{1.565988in}}%
\pgfpathlineto{\pgfqpoint{1.838095in}{1.571400in}}%
\pgfpathlineto{\pgfqpoint{1.867545in}{1.585001in}}%
\pgfpathlineto{\pgfqpoint{1.871381in}{1.580789in}}%
\pgfpathlineto{\pgfqpoint{1.865803in}{1.561714in}}%
\pgfpathlineto{\pgfqpoint{1.874101in}{1.565900in}}%
\pgfpathlineto{\pgfqpoint{1.888314in}{1.561720in}}%
\pgfpathlineto{\pgfqpoint{1.897048in}{1.558231in}}%
\pgfpathlineto{\pgfqpoint{1.901945in}{1.548007in}}%
\pgfpathlineto{\pgfqpoint{1.946751in}{1.538358in}}%
\pgfpathlineto{\pgfqpoint{1.960144in}{1.531739in}}%
\pgfpathlineto{\pgfqpoint{1.973761in}{1.531816in}}%
\pgfpathlineto{\pgfqpoint{1.987753in}{1.529086in}}%
\pgfpathlineto{\pgfqpoint{1.996834in}{1.519758in}}%
\pgfpathlineto{\pgfqpoint{2.005511in}{1.515149in}}%
\pgfpathlineto{\pgfqpoint{2.007105in}{1.501855in}}%
\pgfpathlineto{\pgfqpoint{2.004541in}{1.493508in}}%
\pgfpathlineto{\pgfqpoint{2.014212in}{1.492891in}}%
\pgfpathlineto{\pgfqpoint{2.010954in}{1.483210in}}%
\pgfpathlineto{\pgfqpoint{2.014053in}{1.479771in}}%
\pgfpathlineto{\pgfqpoint{2.017135in}{1.470633in}}%
\pgfpathlineto{\pgfqpoint{2.007710in}{1.465795in}}%
\pgfpathlineto{\pgfqpoint{2.002238in}{1.452314in}}%
\pgfpathlineto{\pgfqpoint{2.000521in}{1.442782in}}%
\pgfpathlineto{\pgfqpoint{2.005771in}{1.441150in}}%
\pgfpathlineto{\pgfqpoint{2.012492in}{1.448298in}}%
\pgfpathlineto{\pgfqpoint{2.018207in}{1.460686in}}%
\pgfpathlineto{\pgfqpoint{2.025939in}{1.464988in}}%
\pgfpathlineto{\pgfqpoint{2.031749in}{1.459392in}}%
\pgfpathlineto{\pgfqpoint{2.026030in}{1.442453in}}%
\pgfpathlineto{\pgfqpoint{2.024236in}{1.429304in}}%
\pgfpathlineto{\pgfqpoint{2.025926in}{1.419852in}}%
\pgfpathlineto{\pgfqpoint{2.020614in}{1.414504in}}%
\pgfpathlineto{\pgfqpoint{2.017962in}{1.401437in}}%
\pgfpathlineto{\pgfqpoint{2.020129in}{1.387704in}}%
\pgfpathlineto{\pgfqpoint{2.013861in}{1.367394in}}%
\pgfpathlineto{\pgfqpoint{2.013993in}{1.357246in}}%
\pgfpathlineto{\pgfqpoint{2.018946in}{1.335254in}}%
\pgfpathlineto{\pgfqpoint{2.022156in}{1.331481in}}%
\pgfpathclose%
\pgfusepath{fill}%
\end{pgfscope}%
\begin{pgfscope}%
\pgfpathrectangle{\pgfqpoint{0.100000in}{0.100000in}}{\pgfqpoint{2.857344in}{1.829167in}}%
\pgfusepath{clip}%
\pgfsetbuttcap%
\pgfsetmiterjoin%
\definecolor{currentfill}{rgb}{0.998078,0.940408,0.649058}%
\pgfsetfillcolor{currentfill}%
\pgfsetlinewidth{0.000000pt}%
\definecolor{currentstroke}{rgb}{0.000000,0.000000,0.000000}%
\pgfsetstrokecolor{currentstroke}%
\pgfsetstrokeopacity{0.000000}%
\pgfsetdash{}{0pt}%
\pgfpathmoveto{\pgfqpoint{2.041469in}{1.491544in}}%
\pgfpathlineto{\pgfqpoint{2.042644in}{1.482772in}}%
\pgfpathlineto{\pgfqpoint{2.031876in}{1.459658in}}%
\pgfpathlineto{\pgfqpoint{2.027091in}{1.466354in}}%
\pgfpathclose%
\pgfusepath{fill}%
\end{pgfscope}%
\begin{pgfscope}%
\pgfpathrectangle{\pgfqpoint{0.100000in}{0.100000in}}{\pgfqpoint{2.857344in}{1.829167in}}%
\pgfusepath{clip}%
\pgfsetbuttcap%
\pgfsetmiterjoin%
\definecolor{currentfill}{rgb}{0.901961,0.960784,0.596078}%
\pgfsetfillcolor{currentfill}%
\pgfsetlinewidth{0.000000pt}%
\definecolor{currentstroke}{rgb}{0.000000,0.000000,0.000000}%
\pgfsetstrokecolor{currentstroke}%
\pgfsetstrokeopacity{0.000000}%
\pgfsetdash{}{0pt}%
\pgfpathmoveto{\pgfqpoint{0.738069in}{1.635747in}}%
\pgfpathlineto{\pgfqpoint{0.740198in}{1.646285in}}%
\pgfpathlineto{\pgfqpoint{0.738592in}{1.657726in}}%
\pgfpathlineto{\pgfqpoint{0.741052in}{1.670200in}}%
\pgfpathlineto{\pgfqpoint{0.753579in}{1.720807in}}%
\pgfpathlineto{\pgfqpoint{0.764115in}{1.762876in}}%
\pgfpathlineto{\pgfqpoint{0.778933in}{1.821590in}}%
\pgfpathlineto{\pgfqpoint{0.818206in}{1.812100in}}%
\pgfpathlineto{\pgfqpoint{0.803691in}{1.748749in}}%
\pgfpathlineto{\pgfqpoint{0.812883in}{1.728336in}}%
\pgfpathlineto{\pgfqpoint{0.811720in}{1.717253in}}%
\pgfpathlineto{\pgfqpoint{0.816648in}{1.700579in}}%
\pgfpathlineto{\pgfqpoint{0.823529in}{1.696639in}}%
\pgfpathlineto{\pgfqpoint{0.831059in}{1.677319in}}%
\pgfpathlineto{\pgfqpoint{0.835684in}{1.671997in}}%
\pgfpathlineto{\pgfqpoint{0.835124in}{1.665148in}}%
\pgfpathlineto{\pgfqpoint{0.843893in}{1.659021in}}%
\pgfpathlineto{\pgfqpoint{0.845130in}{1.651915in}}%
\pgfpathlineto{\pgfqpoint{0.857168in}{1.650923in}}%
\pgfpathlineto{\pgfqpoint{0.841901in}{1.611328in}}%
\pgfpathlineto{\pgfqpoint{0.844236in}{1.603148in}}%
\pgfpathlineto{\pgfqpoint{0.835453in}{1.598061in}}%
\pgfpathlineto{\pgfqpoint{0.834263in}{1.585150in}}%
\pgfpathlineto{\pgfqpoint{0.841586in}{1.577064in}}%
\pgfpathlineto{\pgfqpoint{0.851206in}{1.580987in}}%
\pgfpathlineto{\pgfqpoint{0.858275in}{1.588240in}}%
\pgfpathlineto{\pgfqpoint{0.865808in}{1.580042in}}%
\pgfpathlineto{\pgfqpoint{0.865217in}{1.563291in}}%
\pgfpathlineto{\pgfqpoint{0.873982in}{1.543586in}}%
\pgfpathlineto{\pgfqpoint{0.871975in}{1.531657in}}%
\pgfpathlineto{\pgfqpoint{0.884284in}{1.523380in}}%
\pgfpathlineto{\pgfqpoint{0.884480in}{1.511871in}}%
\pgfpathlineto{\pgfqpoint{0.888673in}{1.499774in}}%
\pgfpathlineto{\pgfqpoint{0.896466in}{1.502500in}}%
\pgfpathlineto{\pgfqpoint{0.914323in}{1.497424in}}%
\pgfpathlineto{\pgfqpoint{0.921675in}{1.503065in}}%
\pgfpathlineto{\pgfqpoint{0.926933in}{1.499419in}}%
\pgfpathlineto{\pgfqpoint{0.937392in}{1.499875in}}%
\pgfpathlineto{\pgfqpoint{0.939635in}{1.496109in}}%
\pgfpathlineto{\pgfqpoint{0.950800in}{1.497162in}}%
\pgfpathlineto{\pgfqpoint{0.956181in}{1.505032in}}%
\pgfpathlineto{\pgfqpoint{0.963698in}{1.505991in}}%
\pgfpathlineto{\pgfqpoint{0.969546in}{1.490423in}}%
\pgfpathlineto{\pgfqpoint{0.972657in}{1.488114in}}%
\pgfpathlineto{\pgfqpoint{0.956725in}{1.398237in}}%
\pgfpathlineto{\pgfqpoint{0.945471in}{1.335044in}}%
\pgfpathlineto{\pgfqpoint{0.856718in}{1.352178in}}%
\pgfpathlineto{\pgfqpoint{0.808777in}{1.361714in}}%
\pgfpathlineto{\pgfqpoint{0.763936in}{1.371575in}}%
\pgfpathlineto{\pgfqpoint{0.722022in}{1.381064in}}%
\pgfpathlineto{\pgfqpoint{0.673515in}{1.392762in}}%
\pgfpathlineto{\pgfqpoint{0.698535in}{1.495404in}}%
\pgfpathlineto{\pgfqpoint{0.699884in}{1.502905in}}%
\pgfpathlineto{\pgfqpoint{0.711011in}{1.522559in}}%
\pgfpathlineto{\pgfqpoint{0.699490in}{1.534373in}}%
\pgfpathlineto{\pgfqpoint{0.701882in}{1.545974in}}%
\pgfpathlineto{\pgfqpoint{0.706268in}{1.548877in}}%
\pgfpathlineto{\pgfqpoint{0.714100in}{1.560831in}}%
\pgfpathlineto{\pgfqpoint{0.725517in}{1.569101in}}%
\pgfpathlineto{\pgfqpoint{0.726141in}{1.575220in}}%
\pgfpathlineto{\pgfqpoint{0.733032in}{1.581334in}}%
\pgfpathlineto{\pgfqpoint{0.738753in}{1.592778in}}%
\pgfpathlineto{\pgfqpoint{0.750490in}{1.604886in}}%
\pgfpathlineto{\pgfqpoint{0.749657in}{1.616844in}}%
\pgfpathlineto{\pgfqpoint{0.741632in}{1.623486in}}%
\pgfpathclose%
\pgfusepath{fill}%
\end{pgfscope}%
\begin{pgfscope}%
\pgfpathrectangle{\pgfqpoint{0.100000in}{0.100000in}}{\pgfqpoint{2.857344in}{1.829167in}}%
\pgfusepath{clip}%
\pgfsetbuttcap%
\pgfsetmiterjoin%
\definecolor{currentfill}{rgb}{0.996847,0.902268,0.585083}%
\pgfsetfillcolor{currentfill}%
\pgfsetlinewidth{0.000000pt}%
\definecolor{currentstroke}{rgb}{0.000000,0.000000,0.000000}%
\pgfsetstrokecolor{currentstroke}%
\pgfsetstrokeopacity{0.000000}%
\pgfsetdash{}{0pt}%
\pgfpathmoveto{\pgfqpoint{2.680920in}{1.435444in}}%
\pgfpathlineto{\pgfqpoint{2.677232in}{1.447085in}}%
\pgfpathlineto{\pgfqpoint{2.670389in}{1.482398in}}%
\pgfpathlineto{\pgfqpoint{2.661541in}{1.493263in}}%
\pgfpathlineto{\pgfqpoint{2.653780in}{1.512801in}}%
\pgfpathlineto{\pgfqpoint{2.656328in}{1.527286in}}%
\pgfpathlineto{\pgfqpoint{2.654285in}{1.537962in}}%
\pgfpathlineto{\pgfqpoint{2.647654in}{1.548449in}}%
\pgfpathlineto{\pgfqpoint{2.647381in}{1.560000in}}%
\pgfpathlineto{\pgfqpoint{2.643531in}{1.572458in}}%
\pgfpathlineto{\pgfqpoint{2.677988in}{1.580938in}}%
\pgfpathlineto{\pgfqpoint{2.722744in}{1.593019in}}%
\pgfpathlineto{\pgfqpoint{2.724530in}{1.586091in}}%
\pgfpathlineto{\pgfqpoint{2.721681in}{1.575063in}}%
\pgfpathlineto{\pgfqpoint{2.728370in}{1.566239in}}%
\pgfpathlineto{\pgfqpoint{2.724817in}{1.555057in}}%
\pgfpathlineto{\pgfqpoint{2.710698in}{1.541001in}}%
\pgfpathlineto{\pgfqpoint{2.714579in}{1.530443in}}%
\pgfpathlineto{\pgfqpoint{2.711703in}{1.508126in}}%
\pgfpathlineto{\pgfqpoint{2.707676in}{1.495543in}}%
\pgfpathlineto{\pgfqpoint{2.712813in}{1.459760in}}%
\pgfpathlineto{\pgfqpoint{2.712121in}{1.447171in}}%
\pgfpathlineto{\pgfqpoint{2.717091in}{1.443130in}}%
\pgfpathclose%
\pgfusepath{fill}%
\end{pgfscope}%
\begin{pgfscope}%
\pgfpathrectangle{\pgfqpoint{0.100000in}{0.100000in}}{\pgfqpoint{2.857344in}{1.829167in}}%
\pgfusepath{clip}%
\pgfsetbuttcap%
\pgfsetmiterjoin%
\definecolor{currentfill}{rgb}{0.999462,0.983314,0.721030}%
\pgfsetfillcolor{currentfill}%
\pgfsetlinewidth{0.000000pt}%
\definecolor{currentstroke}{rgb}{0.000000,0.000000,0.000000}%
\pgfsetstrokecolor{currentstroke}%
\pgfsetstrokeopacity{0.000000}%
\pgfsetdash{}{0pt}%
\pgfpathmoveto{\pgfqpoint{1.622559in}{1.364049in}}%
\pgfpathlineto{\pgfqpoint{1.623847in}{1.429290in}}%
\pgfpathlineto{\pgfqpoint{1.624616in}{1.475648in}}%
\pgfpathlineto{\pgfqpoint{1.621264in}{1.481997in}}%
\pgfpathlineto{\pgfqpoint{1.614201in}{1.484359in}}%
\pgfpathlineto{\pgfqpoint{1.607796in}{1.495412in}}%
\pgfpathlineto{\pgfqpoint{1.616160in}{1.505012in}}%
\pgfpathlineto{\pgfqpoint{1.620351in}{1.517241in}}%
\pgfpathlineto{\pgfqpoint{1.621182in}{1.527408in}}%
\pgfpathlineto{\pgfqpoint{1.619257in}{1.542637in}}%
\pgfpathlineto{\pgfqpoint{1.613420in}{1.552410in}}%
\pgfpathlineto{\pgfqpoint{1.611425in}{1.561312in}}%
\pgfpathlineto{\pgfqpoint{1.609971in}{1.594344in}}%
\pgfpathlineto{\pgfqpoint{1.609989in}{1.622777in}}%
\pgfpathlineto{\pgfqpoint{1.602135in}{1.645362in}}%
\pgfpathlineto{\pgfqpoint{1.598707in}{1.660670in}}%
\pgfpathlineto{\pgfqpoint{1.599301in}{1.681111in}}%
\pgfpathlineto{\pgfqpoint{1.601105in}{1.691056in}}%
\pgfpathlineto{\pgfqpoint{1.596066in}{1.711865in}}%
\pgfpathlineto{\pgfqpoint{1.630372in}{1.711178in}}%
\pgfpathlineto{\pgfqpoint{1.682480in}{1.710731in}}%
\pgfpathlineto{\pgfqpoint{1.682765in}{1.734382in}}%
\pgfpathlineto{\pgfqpoint{1.696029in}{1.731778in}}%
\pgfpathlineto{\pgfqpoint{1.702386in}{1.702939in}}%
\pgfpathlineto{\pgfqpoint{1.707072in}{1.692570in}}%
\pgfpathlineto{\pgfqpoint{1.718727in}{1.692264in}}%
\pgfpathlineto{\pgfqpoint{1.721335in}{1.688745in}}%
\pgfpathlineto{\pgfqpoint{1.737592in}{1.687187in}}%
\pgfpathlineto{\pgfqpoint{1.740324in}{1.680037in}}%
\pgfpathlineto{\pgfqpoint{1.751527in}{1.681644in}}%
\pgfpathlineto{\pgfqpoint{1.760229in}{1.688331in}}%
\pgfpathlineto{\pgfqpoint{1.775235in}{1.688081in}}%
\pgfpathlineto{\pgfqpoint{1.784520in}{1.682703in}}%
\pgfpathlineto{\pgfqpoint{1.785587in}{1.677660in}}%
\pgfpathlineto{\pgfqpoint{1.794422in}{1.676604in}}%
\pgfpathlineto{\pgfqpoint{1.800188in}{1.662847in}}%
\pgfpathlineto{\pgfqpoint{1.803908in}{1.671306in}}%
\pgfpathlineto{\pgfqpoint{1.814057in}{1.671826in}}%
\pgfpathlineto{\pgfqpoint{1.816624in}{1.665238in}}%
\pgfpathlineto{\pgfqpoint{1.828043in}{1.662231in}}%
\pgfpathlineto{\pgfqpoint{1.834341in}{1.652742in}}%
\pgfpathlineto{\pgfqpoint{1.848307in}{1.655689in}}%
\pgfpathlineto{\pgfqpoint{1.863676in}{1.667333in}}%
\pgfpathlineto{\pgfqpoint{1.869279in}{1.657068in}}%
\pgfpathlineto{\pgfqpoint{1.894496in}{1.659878in}}%
\pgfpathlineto{\pgfqpoint{1.905276in}{1.652833in}}%
\pgfpathlineto{\pgfqpoint{1.911550in}{1.655348in}}%
\pgfpathlineto{\pgfqpoint{1.916581in}{1.651382in}}%
\pgfpathlineto{\pgfqpoint{1.901656in}{1.641972in}}%
\pgfpathlineto{\pgfqpoint{1.880352in}{1.633588in}}%
\pgfpathlineto{\pgfqpoint{1.859252in}{1.616829in}}%
\pgfpathlineto{\pgfqpoint{1.840959in}{1.594791in}}%
\pgfpathlineto{\pgfqpoint{1.827111in}{1.581761in}}%
\pgfpathlineto{\pgfqpoint{1.814981in}{1.572806in}}%
\pgfpathlineto{\pgfqpoint{1.806638in}{1.564135in}}%
\pgfpathlineto{\pgfqpoint{1.807649in}{1.527191in}}%
\pgfpathlineto{\pgfqpoint{1.791422in}{1.516803in}}%
\pgfpathlineto{\pgfqpoint{1.783227in}{1.503666in}}%
\pgfpathlineto{\pgfqpoint{1.782682in}{1.495350in}}%
\pgfpathlineto{\pgfqpoint{1.789286in}{1.493519in}}%
\pgfpathlineto{\pgfqpoint{1.793035in}{1.484051in}}%
\pgfpathlineto{\pgfqpoint{1.788862in}{1.477780in}}%
\pgfpathlineto{\pgfqpoint{1.789286in}{1.455000in}}%
\pgfpathlineto{\pgfqpoint{1.787088in}{1.443021in}}%
\pgfpathlineto{\pgfqpoint{1.800619in}{1.432612in}}%
\pgfpathlineto{\pgfqpoint{1.808198in}{1.431422in}}%
\pgfpathlineto{\pgfqpoint{1.814454in}{1.424411in}}%
\pgfpathlineto{\pgfqpoint{1.825096in}{1.420104in}}%
\pgfpathlineto{\pgfqpoint{1.831027in}{1.409407in}}%
\pgfpathlineto{\pgfqpoint{1.842500in}{1.400366in}}%
\pgfpathlineto{\pgfqpoint{1.853044in}{1.394887in}}%
\pgfpathlineto{\pgfqpoint{1.858981in}{1.385126in}}%
\pgfpathlineto{\pgfqpoint{1.860978in}{1.367156in}}%
\pgfpathlineto{\pgfqpoint{1.805022in}{1.365124in}}%
\pgfpathlineto{\pgfqpoint{1.757325in}{1.364126in}}%
\pgfpathlineto{\pgfqpoint{1.713869in}{1.363487in}}%
\pgfpathlineto{\pgfqpoint{1.667897in}{1.363558in}}%
\pgfpathclose%
\pgfusepath{fill}%
\end{pgfscope}%
\begin{pgfscope}%
\pgfpathrectangle{\pgfqpoint{0.100000in}{0.100000in}}{\pgfqpoint{2.857344in}{1.829167in}}%
\pgfusepath{clip}%
\pgfsetbuttcap%
\pgfsetmiterjoin%
\definecolor{currentfill}{rgb}{0.999154,0.973779,0.705037}%
\pgfsetfillcolor{currentfill}%
\pgfsetlinewidth{0.000000pt}%
\definecolor{currentstroke}{rgb}{0.000000,0.000000,0.000000}%
\pgfsetstrokecolor{currentstroke}%
\pgfsetstrokeopacity{0.000000}%
\pgfsetdash{}{0pt}%
\pgfpathmoveto{\pgfqpoint{0.352095in}{1.486015in}}%
\pgfpathlineto{\pgfqpoint{0.347663in}{1.494169in}}%
\pgfpathlineto{\pgfqpoint{0.350493in}{1.515099in}}%
\pgfpathlineto{\pgfqpoint{0.355954in}{1.526038in}}%
\pgfpathlineto{\pgfqpoint{0.353224in}{1.540835in}}%
\pgfpathlineto{\pgfqpoint{0.358921in}{1.547057in}}%
\pgfpathlineto{\pgfqpoint{0.369315in}{1.563830in}}%
\pgfpathlineto{\pgfqpoint{0.378128in}{1.573954in}}%
\pgfpathlineto{\pgfqpoint{0.391011in}{1.596006in}}%
\pgfpathlineto{\pgfqpoint{0.400891in}{1.620027in}}%
\pgfpathlineto{\pgfqpoint{0.411380in}{1.642556in}}%
\pgfpathlineto{\pgfqpoint{0.413500in}{1.651965in}}%
\pgfpathlineto{\pgfqpoint{0.427970in}{1.678883in}}%
\pgfpathlineto{\pgfqpoint{0.430756in}{1.690705in}}%
\pgfpathlineto{\pgfqpoint{0.436917in}{1.703119in}}%
\pgfpathlineto{\pgfqpoint{0.439119in}{1.718264in}}%
\pgfpathlineto{\pgfqpoint{0.450880in}{1.725648in}}%
\pgfpathlineto{\pgfqpoint{0.464826in}{1.729370in}}%
\pgfpathlineto{\pgfqpoint{0.471876in}{1.721034in}}%
\pgfpathlineto{\pgfqpoint{0.477972in}{1.721650in}}%
\pgfpathlineto{\pgfqpoint{0.487489in}{1.711715in}}%
\pgfpathlineto{\pgfqpoint{0.488754in}{1.702356in}}%
\pgfpathlineto{\pgfqpoint{0.485145in}{1.684029in}}%
\pgfpathlineto{\pgfqpoint{0.497153in}{1.674678in}}%
\pgfpathlineto{\pgfqpoint{0.504941in}{1.671163in}}%
\pgfpathlineto{\pgfqpoint{0.526039in}{1.674751in}}%
\pgfpathlineto{\pgfqpoint{0.538262in}{1.672300in}}%
\pgfpathlineto{\pgfqpoint{0.546663in}{1.668278in}}%
\pgfpathlineto{\pgfqpoint{0.551088in}{1.660644in}}%
\pgfpathlineto{\pgfqpoint{0.579110in}{1.661663in}}%
\pgfpathlineto{\pgfqpoint{0.583565in}{1.656991in}}%
\pgfpathlineto{\pgfqpoint{0.594077in}{1.655983in}}%
\pgfpathlineto{\pgfqpoint{0.604663in}{1.658968in}}%
\pgfpathlineto{\pgfqpoint{0.635962in}{1.658216in}}%
\pgfpathlineto{\pgfqpoint{0.642722in}{1.655916in}}%
\pgfpathlineto{\pgfqpoint{0.650622in}{1.658566in}}%
\pgfpathlineto{\pgfqpoint{0.738069in}{1.635747in}}%
\pgfpathlineto{\pgfqpoint{0.741632in}{1.623486in}}%
\pgfpathlineto{\pgfqpoint{0.749657in}{1.616844in}}%
\pgfpathlineto{\pgfqpoint{0.750490in}{1.604886in}}%
\pgfpathlineto{\pgfqpoint{0.738753in}{1.592778in}}%
\pgfpathlineto{\pgfqpoint{0.733032in}{1.581334in}}%
\pgfpathlineto{\pgfqpoint{0.726141in}{1.575220in}}%
\pgfpathlineto{\pgfqpoint{0.725517in}{1.569101in}}%
\pgfpathlineto{\pgfqpoint{0.714100in}{1.560831in}}%
\pgfpathlineto{\pgfqpoint{0.706268in}{1.548877in}}%
\pgfpathlineto{\pgfqpoint{0.701882in}{1.545974in}}%
\pgfpathlineto{\pgfqpoint{0.699490in}{1.534373in}}%
\pgfpathlineto{\pgfqpoint{0.711011in}{1.522559in}}%
\pgfpathlineto{\pgfqpoint{0.699884in}{1.502905in}}%
\pgfpathlineto{\pgfqpoint{0.698535in}{1.495404in}}%
\pgfpathlineto{\pgfqpoint{0.673515in}{1.392762in}}%
\pgfpathlineto{\pgfqpoint{0.620888in}{1.406247in}}%
\pgfpathlineto{\pgfqpoint{0.570118in}{1.419338in}}%
\pgfpathlineto{\pgfqpoint{0.539490in}{1.427861in}}%
\pgfpathlineto{\pgfqpoint{0.500137in}{1.439071in}}%
\pgfpathlineto{\pgfqpoint{0.437366in}{1.458829in}}%
\pgfpathlineto{\pgfqpoint{0.369129in}{1.480073in}}%
\pgfpathclose%
\pgfusepath{fill}%
\end{pgfscope}%
\begin{pgfscope}%
\pgfpathrectangle{\pgfqpoint{0.100000in}{0.100000in}}{\pgfqpoint{2.857344in}{1.829167in}}%
\pgfusepath{clip}%
\pgfsetbuttcap%
\pgfsetmiterjoin%
\definecolor{currentfill}{rgb}{0.999000,0.969012,0.697040}%
\pgfsetfillcolor{currentfill}%
\pgfsetlinewidth{0.000000pt}%
\definecolor{currentstroke}{rgb}{0.000000,0.000000,0.000000}%
\pgfsetstrokecolor{currentstroke}%
\pgfsetstrokeopacity{0.000000}%
\pgfsetdash{}{0pt}%
\pgfpathmoveto{\pgfqpoint{2.717091in}{1.443130in}}%
\pgfpathlineto{\pgfqpoint{2.712121in}{1.447171in}}%
\pgfpathlineto{\pgfqpoint{2.712813in}{1.459760in}}%
\pgfpathlineto{\pgfqpoint{2.707676in}{1.495543in}}%
\pgfpathlineto{\pgfqpoint{2.711703in}{1.508126in}}%
\pgfpathlineto{\pgfqpoint{2.714579in}{1.530443in}}%
\pgfpathlineto{\pgfqpoint{2.710698in}{1.541001in}}%
\pgfpathlineto{\pgfqpoint{2.724817in}{1.555057in}}%
\pgfpathlineto{\pgfqpoint{2.728370in}{1.566239in}}%
\pgfpathlineto{\pgfqpoint{2.721681in}{1.575063in}}%
\pgfpathlineto{\pgfqpoint{2.724530in}{1.586091in}}%
\pgfpathlineto{\pgfqpoint{2.722744in}{1.593019in}}%
\pgfpathlineto{\pgfqpoint{2.724278in}{1.607855in}}%
\pgfpathlineto{\pgfqpoint{2.727146in}{1.612435in}}%
\pgfpathlineto{\pgfqpoint{2.735989in}{1.615326in}}%
\pgfpathlineto{\pgfqpoint{2.748887in}{1.577551in}}%
\pgfpathlineto{\pgfqpoint{2.772818in}{1.499273in}}%
\pgfpathlineto{\pgfqpoint{2.781874in}{1.493203in}}%
\pgfpathlineto{\pgfqpoint{2.783503in}{1.486275in}}%
\pgfpathlineto{\pgfqpoint{2.788283in}{1.483476in}}%
\pgfpathlineto{\pgfqpoint{2.787919in}{1.470910in}}%
\pgfpathlineto{\pgfqpoint{2.782855in}{1.470708in}}%
\pgfpathlineto{\pgfqpoint{2.772595in}{1.462884in}}%
\pgfpathlineto{\pgfqpoint{2.769636in}{1.455033in}}%
\pgfpathlineto{\pgfqpoint{2.742174in}{1.448239in}}%
\pgfpathclose%
\pgfusepath{fill}%
\end{pgfscope}%
\begin{pgfscope}%
\pgfpathrectangle{\pgfqpoint{0.100000in}{0.100000in}}{\pgfqpoint{2.857344in}{1.829167in}}%
\pgfusepath{clip}%
\pgfsetbuttcap%
\pgfsetmiterjoin%
\definecolor{currentfill}{rgb}{0.847520,0.938639,0.607151}%
\pgfsetfillcolor{currentfill}%
\pgfsetlinewidth{0.000000pt}%
\definecolor{currentstroke}{rgb}{0.000000,0.000000,0.000000}%
\pgfsetstrokecolor{currentstroke}%
\pgfsetstrokeopacity{0.000000}%
\pgfsetdash{}{0pt}%
\pgfpathmoveto{\pgfqpoint{1.858393in}{1.170794in}}%
\pgfpathlineto{\pgfqpoint{1.843896in}{1.185151in}}%
\pgfpathlineto{\pgfqpoint{1.797555in}{1.182478in}}%
\pgfpathlineto{\pgfqpoint{1.725281in}{1.180096in}}%
\pgfpathlineto{\pgfqpoint{1.652572in}{1.181231in}}%
\pgfpathlineto{\pgfqpoint{1.647469in}{1.190130in}}%
\pgfpathlineto{\pgfqpoint{1.649552in}{1.198868in}}%
\pgfpathlineto{\pgfqpoint{1.648563in}{1.213838in}}%
\pgfpathlineto{\pgfqpoint{1.644720in}{1.228337in}}%
\pgfpathlineto{\pgfqpoint{1.645167in}{1.235994in}}%
\pgfpathlineto{\pgfqpoint{1.636999in}{1.244645in}}%
\pgfpathlineto{\pgfqpoint{1.638776in}{1.256683in}}%
\pgfpathlineto{\pgfqpoint{1.626247in}{1.280461in}}%
\pgfpathlineto{\pgfqpoint{1.622514in}{1.300647in}}%
\pgfpathlineto{\pgfqpoint{1.613501in}{1.317067in}}%
\pgfpathlineto{\pgfqpoint{1.622056in}{1.337435in}}%
\pgfpathlineto{\pgfqpoint{1.617795in}{1.346914in}}%
\pgfpathlineto{\pgfqpoint{1.622559in}{1.364049in}}%
\pgfpathlineto{\pgfqpoint{1.667897in}{1.363558in}}%
\pgfpathlineto{\pgfqpoint{1.713869in}{1.363487in}}%
\pgfpathlineto{\pgfqpoint{1.757325in}{1.364126in}}%
\pgfpathlineto{\pgfqpoint{1.805022in}{1.365124in}}%
\pgfpathlineto{\pgfqpoint{1.860978in}{1.367156in}}%
\pgfpathlineto{\pgfqpoint{1.862472in}{1.359032in}}%
\pgfpathlineto{\pgfqpoint{1.868667in}{1.352221in}}%
\pgfpathlineto{\pgfqpoint{1.864744in}{1.345124in}}%
\pgfpathlineto{\pgfqpoint{1.866352in}{1.331272in}}%
\pgfpathlineto{\pgfqpoint{1.870191in}{1.320074in}}%
\pgfpathlineto{\pgfqpoint{1.887596in}{1.314147in}}%
\pgfpathlineto{\pgfqpoint{1.890579in}{1.306219in}}%
\pgfpathlineto{\pgfqpoint{1.900130in}{1.297317in}}%
\pgfpathlineto{\pgfqpoint{1.904026in}{1.288109in}}%
\pgfpathlineto{\pgfqpoint{1.913704in}{1.281931in}}%
\pgfpathlineto{\pgfqpoint{1.915219in}{1.274495in}}%
\pgfpathlineto{\pgfqpoint{1.913334in}{1.263236in}}%
\pgfpathlineto{\pgfqpoint{1.908409in}{1.259862in}}%
\pgfpathlineto{\pgfqpoint{1.906923in}{1.249144in}}%
\pgfpathlineto{\pgfqpoint{1.892767in}{1.240633in}}%
\pgfpathlineto{\pgfqpoint{1.874315in}{1.235970in}}%
\pgfpathlineto{\pgfqpoint{1.872625in}{1.225248in}}%
\pgfpathlineto{\pgfqpoint{1.879743in}{1.217587in}}%
\pgfpathlineto{\pgfqpoint{1.880033in}{1.207956in}}%
\pgfpathlineto{\pgfqpoint{1.874290in}{1.200386in}}%
\pgfpathlineto{\pgfqpoint{1.871276in}{1.189138in}}%
\pgfpathlineto{\pgfqpoint{1.861293in}{1.185415in}}%
\pgfpathlineto{\pgfqpoint{1.861930in}{1.172881in}}%
\pgfpathclose%
\pgfusepath{fill}%
\end{pgfscope}%
\begin{pgfscope}%
\pgfpathrectangle{\pgfqpoint{0.100000in}{0.100000in}}{\pgfqpoint{2.857344in}{1.829167in}}%
\pgfusepath{clip}%
\pgfsetbuttcap%
\pgfsetmiterjoin%
\definecolor{currentfill}{rgb}{0.999462,0.983314,0.721030}%
\pgfsetfillcolor{currentfill}%
\pgfsetlinewidth{0.000000pt}%
\definecolor{currentstroke}{rgb}{0.000000,0.000000,0.000000}%
\pgfsetstrokecolor{currentstroke}%
\pgfsetstrokeopacity{0.000000}%
\pgfsetdash{}{0pt}%
\pgfpathmoveto{\pgfqpoint{2.757873in}{1.406908in}}%
\pgfpathlineto{\pgfqpoint{2.744142in}{1.404742in}}%
\pgfpathlineto{\pgfqpoint{2.681070in}{1.390410in}}%
\pgfpathlineto{\pgfqpoint{2.679968in}{1.392080in}}%
\pgfpathlineto{\pgfqpoint{2.680920in}{1.435444in}}%
\pgfpathlineto{\pgfqpoint{2.717091in}{1.443130in}}%
\pgfpathlineto{\pgfqpoint{2.742174in}{1.448239in}}%
\pgfpathlineto{\pgfqpoint{2.769636in}{1.455033in}}%
\pgfpathlineto{\pgfqpoint{2.772595in}{1.462884in}}%
\pgfpathlineto{\pgfqpoint{2.782855in}{1.470708in}}%
\pgfpathlineto{\pgfqpoint{2.787919in}{1.470910in}}%
\pgfpathlineto{\pgfqpoint{2.794577in}{1.459495in}}%
\pgfpathlineto{\pgfqpoint{2.788536in}{1.442831in}}%
\pgfpathlineto{\pgfqpoint{2.787649in}{1.433059in}}%
\pgfpathlineto{\pgfqpoint{2.799877in}{1.433977in}}%
\pgfpathlineto{\pgfqpoint{2.805422in}{1.429290in}}%
\pgfpathlineto{\pgfqpoint{2.817858in}{1.410130in}}%
\pgfpathlineto{\pgfqpoint{2.824039in}{1.407800in}}%
\pgfpathlineto{\pgfqpoint{2.834392in}{1.408664in}}%
\pgfpathlineto{\pgfqpoint{2.841597in}{1.415175in}}%
\pgfpathlineto{\pgfqpoint{2.846386in}{1.409365in}}%
\pgfpathlineto{\pgfqpoint{2.816269in}{1.393472in}}%
\pgfpathlineto{\pgfqpoint{2.815328in}{1.404876in}}%
\pgfpathlineto{\pgfqpoint{2.801647in}{1.387152in}}%
\pgfpathlineto{\pgfqpoint{2.796836in}{1.384098in}}%
\pgfpathlineto{\pgfqpoint{2.790127in}{1.394325in}}%
\pgfpathlineto{\pgfqpoint{2.788291in}{1.395727in}}%
\pgfpathlineto{\pgfqpoint{2.782045in}{1.399035in}}%
\pgfpathlineto{\pgfqpoint{2.776598in}{1.412454in}}%
\pgfpathclose%
\pgfusepath{fill}%
\end{pgfscope}%
\begin{pgfscope}%
\pgfpathrectangle{\pgfqpoint{0.100000in}{0.100000in}}{\pgfqpoint{2.857344in}{1.829167in}}%
\pgfusepath{clip}%
\pgfsetbuttcap%
\pgfsetmiterjoin%
\definecolor{currentfill}{rgb}{0.975010,0.990004,0.710035}%
\pgfsetfillcolor{currentfill}%
\pgfsetlinewidth{0.000000pt}%
\definecolor{currentstroke}{rgb}{0.000000,0.000000,0.000000}%
\pgfsetstrokecolor{currentstroke}%
\pgfsetstrokeopacity{0.000000}%
\pgfsetdash{}{0pt}%
\pgfpathmoveto{\pgfqpoint{1.351269in}{1.157246in}}%
\pgfpathlineto{\pgfqpoint{1.356297in}{1.219598in}}%
\pgfpathlineto{\pgfqpoint{1.327819in}{1.221909in}}%
\pgfpathlineto{\pgfqpoint{1.261989in}{1.228296in}}%
\pgfpathlineto{\pgfqpoint{1.268113in}{1.290427in}}%
\pgfpathlineto{\pgfqpoint{1.274247in}{1.353032in}}%
\pgfpathlineto{\pgfqpoint{1.299554in}{1.350300in}}%
\pgfpathlineto{\pgfqpoint{1.364186in}{1.344193in}}%
\pgfpathlineto{\pgfqpoint{1.403301in}{1.341130in}}%
\pgfpathlineto{\pgfqpoint{1.450656in}{1.338416in}}%
\pgfpathlineto{\pgfqpoint{1.528600in}{1.334727in}}%
\pgfpathlineto{\pgfqpoint{1.530313in}{1.331232in}}%
\pgfpathlineto{\pgfqpoint{1.552647in}{1.320183in}}%
\pgfpathlineto{\pgfqpoint{1.559496in}{1.325794in}}%
\pgfpathlineto{\pgfqpoint{1.579228in}{1.325282in}}%
\pgfpathlineto{\pgfqpoint{1.590986in}{1.319122in}}%
\pgfpathlineto{\pgfqpoint{1.609632in}{1.312120in}}%
\pgfpathlineto{\pgfqpoint{1.614856in}{1.301904in}}%
\pgfpathlineto{\pgfqpoint{1.622514in}{1.300647in}}%
\pgfpathlineto{\pgfqpoint{1.626247in}{1.280461in}}%
\pgfpathlineto{\pgfqpoint{1.638776in}{1.256683in}}%
\pgfpathlineto{\pgfqpoint{1.636999in}{1.244645in}}%
\pgfpathlineto{\pgfqpoint{1.645167in}{1.235994in}}%
\pgfpathlineto{\pgfqpoint{1.644720in}{1.228337in}}%
\pgfpathlineto{\pgfqpoint{1.648563in}{1.213838in}}%
\pgfpathlineto{\pgfqpoint{1.649552in}{1.198868in}}%
\pgfpathlineto{\pgfqpoint{1.647469in}{1.190130in}}%
\pgfpathlineto{\pgfqpoint{1.652572in}{1.181231in}}%
\pgfpathlineto{\pgfqpoint{1.659572in}{1.165048in}}%
\pgfpathlineto{\pgfqpoint{1.666272in}{1.158462in}}%
\pgfpathlineto{\pgfqpoint{1.674259in}{1.144189in}}%
\pgfpathlineto{\pgfqpoint{1.651623in}{1.143955in}}%
\pgfpathlineto{\pgfqpoint{1.602681in}{1.144712in}}%
\pgfpathlineto{\pgfqpoint{1.548607in}{1.146369in}}%
\pgfpathlineto{\pgfqpoint{1.494214in}{1.148456in}}%
\pgfpathlineto{\pgfqpoint{1.414241in}{1.152824in}}%
\pgfpathclose%
\pgfusepath{fill}%
\end{pgfscope}%
\begin{pgfscope}%
\pgfpathrectangle{\pgfqpoint{0.100000in}{0.100000in}}{\pgfqpoint{2.857344in}{1.829167in}}%
\pgfusepath{clip}%
\pgfsetbuttcap%
\pgfsetmiterjoin%
\definecolor{currentfill}{rgb}{0.993618,0.755402,0.441753}%
\pgfsetfillcolor{currentfill}%
\pgfsetlinewidth{0.000000pt}%
\definecolor{currentstroke}{rgb}{0.000000,0.000000,0.000000}%
\pgfsetstrokecolor{currentstroke}%
\pgfsetstrokeopacity{0.000000}%
\pgfsetdash{}{0pt}%
\pgfpathmoveto{\pgfqpoint{2.392558in}{1.345322in}}%
\pgfpathlineto{\pgfqpoint{2.417600in}{1.369200in}}%
\pgfpathlineto{\pgfqpoint{2.420713in}{1.377687in}}%
\pgfpathlineto{\pgfqpoint{2.428050in}{1.384953in}}%
\pgfpathlineto{\pgfqpoint{2.422544in}{1.395532in}}%
\pgfpathlineto{\pgfqpoint{2.415641in}{1.401720in}}%
\pgfpathlineto{\pgfqpoint{2.413646in}{1.412685in}}%
\pgfpathlineto{\pgfqpoint{2.439340in}{1.423944in}}%
\pgfpathlineto{\pgfqpoint{2.460614in}{1.427505in}}%
\pgfpathlineto{\pgfqpoint{2.472045in}{1.427743in}}%
\pgfpathlineto{\pgfqpoint{2.480753in}{1.423441in}}%
\pgfpathlineto{\pgfqpoint{2.489241in}{1.427283in}}%
\pgfpathlineto{\pgfqpoint{2.509945in}{1.431588in}}%
\pgfpathlineto{\pgfqpoint{2.517098in}{1.437145in}}%
\pgfpathlineto{\pgfqpoint{2.527705in}{1.449447in}}%
\pgfpathlineto{\pgfqpoint{2.537355in}{1.454884in}}%
\pgfpathlineto{\pgfqpoint{2.538035in}{1.460097in}}%
\pgfpathlineto{\pgfqpoint{2.532971in}{1.471992in}}%
\pgfpathlineto{\pgfqpoint{2.536632in}{1.478992in}}%
\pgfpathlineto{\pgfqpoint{2.531703in}{1.486519in}}%
\pgfpathlineto{\pgfqpoint{2.524154in}{1.487044in}}%
\pgfpathlineto{\pgfqpoint{2.543047in}{1.509780in}}%
\pgfpathlineto{\pgfqpoint{2.545291in}{1.518444in}}%
\pgfpathlineto{\pgfqpoint{2.560162in}{1.540542in}}%
\pgfpathlineto{\pgfqpoint{2.573966in}{1.552500in}}%
\pgfpathlineto{\pgfqpoint{2.583444in}{1.557494in}}%
\pgfpathlineto{\pgfqpoint{2.614461in}{1.564521in}}%
\pgfpathlineto{\pgfqpoint{2.643531in}{1.572458in}}%
\pgfpathlineto{\pgfqpoint{2.647381in}{1.560000in}}%
\pgfpathlineto{\pgfqpoint{2.647654in}{1.548449in}}%
\pgfpathlineto{\pgfqpoint{2.654285in}{1.537962in}}%
\pgfpathlineto{\pgfqpoint{2.656328in}{1.527286in}}%
\pgfpathlineto{\pgfqpoint{2.653780in}{1.512801in}}%
\pgfpathlineto{\pgfqpoint{2.661541in}{1.493263in}}%
\pgfpathlineto{\pgfqpoint{2.670389in}{1.482398in}}%
\pgfpathlineto{\pgfqpoint{2.677232in}{1.447085in}}%
\pgfpathlineto{\pgfqpoint{2.680920in}{1.435444in}}%
\pgfpathlineto{\pgfqpoint{2.679968in}{1.392080in}}%
\pgfpathlineto{\pgfqpoint{2.681070in}{1.390410in}}%
\pgfpathlineto{\pgfqpoint{2.689127in}{1.343772in}}%
\pgfpathlineto{\pgfqpoint{2.693645in}{1.339522in}}%
\pgfpathlineto{\pgfqpoint{2.683927in}{1.330080in}}%
\pgfpathlineto{\pgfqpoint{2.688722in}{1.324660in}}%
\pgfpathlineto{\pgfqpoint{2.684507in}{1.316463in}}%
\pgfpathlineto{\pgfqpoint{2.684543in}{1.312974in}}%
\pgfpathlineto{\pgfqpoint{2.679274in}{1.309825in}}%
\pgfpathlineto{\pgfqpoint{2.676708in}{1.302864in}}%
\pgfpathlineto{\pgfqpoint{2.677522in}{1.322005in}}%
\pgfpathlineto{\pgfqpoint{2.661190in}{1.326218in}}%
\pgfpathlineto{\pgfqpoint{2.635649in}{1.334930in}}%
\pgfpathlineto{\pgfqpoint{2.632037in}{1.339222in}}%
\pgfpathlineto{\pgfqpoint{2.621364in}{1.340250in}}%
\pgfpathlineto{\pgfqpoint{2.615225in}{1.346637in}}%
\pgfpathlineto{\pgfqpoint{2.611912in}{1.359358in}}%
\pgfpathlineto{\pgfqpoint{2.603209in}{1.360940in}}%
\pgfpathlineto{\pgfqpoint{2.597389in}{1.367613in}}%
\pgfpathlineto{\pgfqpoint{2.523292in}{1.352675in}}%
\pgfpathlineto{\pgfqpoint{2.487880in}{1.345351in}}%
\pgfpathlineto{\pgfqpoint{2.434113in}{1.335538in}}%
\pgfpathlineto{\pgfqpoint{2.395394in}{1.329006in}}%
\pgfpathclose%
\pgfusepath{fill}%
\end{pgfscope}%
\begin{pgfscope}%
\pgfpathrectangle{\pgfqpoint{0.100000in}{0.100000in}}{\pgfqpoint{2.857344in}{1.829167in}}%
\pgfusepath{clip}%
\pgfsetbuttcap%
\pgfsetmiterjoin%
\definecolor{currentfill}{rgb}{0.993618,0.755402,0.441753}%
\pgfsetfillcolor{currentfill}%
\pgfsetlinewidth{0.000000pt}%
\definecolor{currentstroke}{rgb}{0.000000,0.000000,0.000000}%
\pgfsetstrokecolor{currentstroke}%
\pgfsetstrokeopacity{0.000000}%
\pgfsetdash{}{0pt}%
\pgfpathmoveto{\pgfqpoint{2.690026in}{1.298977in}}%
\pgfpathlineto{\pgfqpoint{2.678567in}{1.295407in}}%
\pgfpathlineto{\pgfqpoint{2.676639in}{1.298689in}}%
\pgfpathlineto{\pgfqpoint{2.680317in}{1.309706in}}%
\pgfpathlineto{\pgfqpoint{2.687129in}{1.310790in}}%
\pgfpathlineto{\pgfqpoint{2.686519in}{1.314259in}}%
\pgfpathlineto{\pgfqpoint{2.692636in}{1.319465in}}%
\pgfpathlineto{\pgfqpoint{2.710319in}{1.323612in}}%
\pgfpathlineto{\pgfqpoint{2.718191in}{1.329891in}}%
\pgfpathlineto{\pgfqpoint{2.735875in}{1.335123in}}%
\pgfpathlineto{\pgfqpoint{2.743941in}{1.333208in}}%
\pgfpathlineto{\pgfqpoint{2.750726in}{1.341643in}}%
\pgfpathlineto{\pgfqpoint{2.760981in}{1.342758in}}%
\pgfpathlineto{\pgfqpoint{2.743486in}{1.326304in}}%
\pgfpathclose%
\pgfusepath{fill}%
\end{pgfscope}%
\begin{pgfscope}%
\pgfpathrectangle{\pgfqpoint{0.100000in}{0.100000in}}{\pgfqpoint{2.857344in}{1.829167in}}%
\pgfusepath{clip}%
\pgfsetbuttcap%
\pgfsetmiterjoin%
\definecolor{currentfill}{rgb}{0.997463,0.921338,0.617070}%
\pgfsetfillcolor{currentfill}%
\pgfsetlinewidth{0.000000pt}%
\definecolor{currentstroke}{rgb}{0.000000,0.000000,0.000000}%
\pgfsetstrokecolor{currentstroke}%
\pgfsetstrokeopacity{0.000000}%
\pgfsetdash{}{0pt}%
\pgfpathmoveto{\pgfqpoint{2.432573in}{1.190332in}}%
\pgfpathlineto{\pgfqpoint{2.383019in}{1.182138in}}%
\pgfpathlineto{\pgfqpoint{2.374030in}{1.238770in}}%
\pgfpathlineto{\pgfqpoint{2.360683in}{1.322248in}}%
\pgfpathlineto{\pgfqpoint{2.392558in}{1.345322in}}%
\pgfpathlineto{\pgfqpoint{2.395394in}{1.329006in}}%
\pgfpathlineto{\pgfqpoint{2.434113in}{1.335538in}}%
\pgfpathlineto{\pgfqpoint{2.487880in}{1.345351in}}%
\pgfpathlineto{\pgfqpoint{2.523292in}{1.352675in}}%
\pgfpathlineto{\pgfqpoint{2.597389in}{1.367613in}}%
\pgfpathlineto{\pgfqpoint{2.603209in}{1.360940in}}%
\pgfpathlineto{\pgfqpoint{2.611912in}{1.359358in}}%
\pgfpathlineto{\pgfqpoint{2.615225in}{1.346637in}}%
\pgfpathlineto{\pgfqpoint{2.621364in}{1.340250in}}%
\pgfpathlineto{\pgfqpoint{2.632037in}{1.339222in}}%
\pgfpathlineto{\pgfqpoint{2.635649in}{1.334930in}}%
\pgfpathlineto{\pgfqpoint{2.631982in}{1.331620in}}%
\pgfpathlineto{\pgfqpoint{2.628678in}{1.319905in}}%
\pgfpathlineto{\pgfqpoint{2.620936in}{1.306743in}}%
\pgfpathlineto{\pgfqpoint{2.626133in}{1.301019in}}%
\pgfpathlineto{\pgfqpoint{2.621880in}{1.294886in}}%
\pgfpathlineto{\pgfqpoint{2.624279in}{1.281436in}}%
\pgfpathlineto{\pgfqpoint{2.631938in}{1.274376in}}%
\pgfpathlineto{\pgfqpoint{2.650110in}{1.262906in}}%
\pgfpathlineto{\pgfqpoint{2.635476in}{1.246723in}}%
\pgfpathlineto{\pgfqpoint{2.635270in}{1.240580in}}%
\pgfpathlineto{\pgfqpoint{2.623356in}{1.232658in}}%
\pgfpathlineto{\pgfqpoint{2.610183in}{1.231171in}}%
\pgfpathlineto{\pgfqpoint{2.606924in}{1.224286in}}%
\pgfpathlineto{\pgfqpoint{2.570281in}{1.216357in}}%
\pgfpathlineto{\pgfqpoint{2.527524in}{1.207746in}}%
\pgfpathlineto{\pgfqpoint{2.498134in}{1.202483in}}%
\pgfpathclose%
\pgfusepath{fill}%
\end{pgfscope}%
\begin{pgfscope}%
\pgfpathrectangle{\pgfqpoint{0.100000in}{0.100000in}}{\pgfqpoint{2.857344in}{1.829167in}}%
\pgfusepath{clip}%
\pgfsetbuttcap%
\pgfsetmiterjoin%
\definecolor{currentfill}{rgb}{0.999769,0.992849,0.737024}%
\pgfsetfillcolor{currentfill}%
\pgfsetlinewidth{0.000000pt}%
\definecolor{currentstroke}{rgb}{0.000000,0.000000,0.000000}%
\pgfsetstrokecolor{currentstroke}%
\pgfsetstrokeopacity{0.000000}%
\pgfsetdash{}{0pt}%
\pgfpathmoveto{\pgfqpoint{2.681070in}{1.390410in}}%
\pgfpathlineto{\pgfqpoint{2.744142in}{1.404742in}}%
\pgfpathlineto{\pgfqpoint{2.757873in}{1.406908in}}%
\pgfpathlineto{\pgfqpoint{2.766961in}{1.371177in}}%
\pgfpathlineto{\pgfqpoint{2.765523in}{1.364778in}}%
\pgfpathlineto{\pgfqpoint{2.736329in}{1.353476in}}%
\pgfpathlineto{\pgfqpoint{2.718901in}{1.349526in}}%
\pgfpathlineto{\pgfqpoint{2.711495in}{1.340664in}}%
\pgfpathlineto{\pgfqpoint{2.688722in}{1.324660in}}%
\pgfpathlineto{\pgfqpoint{2.683927in}{1.330080in}}%
\pgfpathlineto{\pgfqpoint{2.693645in}{1.339522in}}%
\pgfpathlineto{\pgfqpoint{2.689127in}{1.343772in}}%
\pgfpathclose%
\pgfusepath{fill}%
\end{pgfscope}%
\begin{pgfscope}%
\pgfpathrectangle{\pgfqpoint{0.100000in}{0.100000in}}{\pgfqpoint{2.857344in}{1.829167in}}%
\pgfusepath{clip}%
\pgfsetbuttcap%
\pgfsetmiterjoin%
\definecolor{currentfill}{rgb}{0.998231,0.945175,0.657055}%
\pgfsetfillcolor{currentfill}%
\pgfsetlinewidth{0.000000pt}%
\definecolor{currentstroke}{rgb}{0.000000,0.000000,0.000000}%
\pgfsetstrokecolor{currentstroke}%
\pgfsetstrokeopacity{0.000000}%
\pgfsetdash{}{0pt}%
\pgfpathmoveto{\pgfqpoint{2.757873in}{1.406908in}}%
\pgfpathlineto{\pgfqpoint{2.776598in}{1.412454in}}%
\pgfpathlineto{\pgfqpoint{2.782045in}{1.399035in}}%
\pgfpathlineto{\pgfqpoint{2.788291in}{1.395727in}}%
\pgfpathlineto{\pgfqpoint{2.780583in}{1.390075in}}%
\pgfpathlineto{\pgfqpoint{2.781555in}{1.373474in}}%
\pgfpathlineto{\pgfqpoint{2.765523in}{1.364778in}}%
\pgfpathlineto{\pgfqpoint{2.766961in}{1.371177in}}%
\pgfpathclose%
\pgfusepath{fill}%
\end{pgfscope}%
\begin{pgfscope}%
\pgfpathrectangle{\pgfqpoint{0.100000in}{0.100000in}}{\pgfqpoint{2.857344in}{1.829167in}}%
\pgfusepath{clip}%
\pgfsetbuttcap%
\pgfsetmiterjoin%
\definecolor{currentfill}{rgb}{0.998539,0.954710,0.673049}%
\pgfsetfillcolor{currentfill}%
\pgfsetlinewidth{0.000000pt}%
\definecolor{currentstroke}{rgb}{0.000000,0.000000,0.000000}%
\pgfsetstrokecolor{currentstroke}%
\pgfsetstrokeopacity{0.000000}%
\pgfsetdash{}{0pt}%
\pgfpathmoveto{\pgfqpoint{2.621245in}{1.226807in}}%
\pgfpathlineto{\pgfqpoint{2.623356in}{1.232658in}}%
\pgfpathlineto{\pgfqpoint{2.635270in}{1.240580in}}%
\pgfpathlineto{\pgfqpoint{2.635476in}{1.246723in}}%
\pgfpathlineto{\pgfqpoint{2.650110in}{1.262906in}}%
\pgfpathlineto{\pgfqpoint{2.631938in}{1.274376in}}%
\pgfpathlineto{\pgfqpoint{2.624279in}{1.281436in}}%
\pgfpathlineto{\pgfqpoint{2.621880in}{1.294886in}}%
\pgfpathlineto{\pgfqpoint{2.626133in}{1.301019in}}%
\pgfpathlineto{\pgfqpoint{2.620936in}{1.306743in}}%
\pgfpathlineto{\pgfqpoint{2.628678in}{1.319905in}}%
\pgfpathlineto{\pgfqpoint{2.631982in}{1.331620in}}%
\pgfpathlineto{\pgfqpoint{2.635649in}{1.334930in}}%
\pgfpathlineto{\pgfqpoint{2.661190in}{1.326218in}}%
\pgfpathlineto{\pgfqpoint{2.677522in}{1.322005in}}%
\pgfpathlineto{\pgfqpoint{2.676708in}{1.302864in}}%
\pgfpathlineto{\pgfqpoint{2.666787in}{1.288350in}}%
\pgfpathlineto{\pgfqpoint{2.674962in}{1.286214in}}%
\pgfpathlineto{\pgfqpoint{2.683449in}{1.279985in}}%
\pgfpathlineto{\pgfqpoint{2.683950in}{1.262949in}}%
\pgfpathlineto{\pgfqpoint{2.681381in}{1.250888in}}%
\pgfpathlineto{\pgfqpoint{2.683091in}{1.240982in}}%
\pgfpathlineto{\pgfqpoint{2.668336in}{1.207554in}}%
\pgfpathlineto{\pgfqpoint{2.663034in}{1.191945in}}%
\pgfpathlineto{\pgfqpoint{2.655630in}{1.199532in}}%
\pgfpathlineto{\pgfqpoint{2.645812in}{1.198239in}}%
\pgfpathlineto{\pgfqpoint{2.621241in}{1.212435in}}%
\pgfpathlineto{\pgfqpoint{2.618727in}{1.220038in}}%
\pgfpathclose%
\pgfusepath{fill}%
\end{pgfscope}%
\begin{pgfscope}%
\pgfpathrectangle{\pgfqpoint{0.100000in}{0.100000in}}{\pgfqpoint{2.857344in}{1.829167in}}%
\pgfusepath{clip}%
\pgfsetbuttcap%
\pgfsetmiterjoin%
\definecolor{currentfill}{rgb}{0.955786,0.982314,0.680046}%
\pgfsetfillcolor{currentfill}%
\pgfsetlinewidth{0.000000pt}%
\definecolor{currentstroke}{rgb}{0.000000,0.000000,0.000000}%
\pgfsetstrokecolor{currentstroke}%
\pgfsetstrokeopacity{0.000000}%
\pgfsetdash{}{0pt}%
\pgfpathmoveto{\pgfqpoint{2.033197in}{1.020052in}}%
\pgfpathlineto{\pgfqpoint{2.032048in}{1.035560in}}%
\pgfpathlineto{\pgfqpoint{2.036513in}{1.043254in}}%
\pgfpathlineto{\pgfqpoint{2.033570in}{1.047023in}}%
\pgfpathlineto{\pgfqpoint{2.044301in}{1.059367in}}%
\pgfpathlineto{\pgfqpoint{2.043686in}{1.062342in}}%
\pgfpathlineto{\pgfqpoint{2.054113in}{1.083876in}}%
\pgfpathlineto{\pgfqpoint{2.052005in}{1.094261in}}%
\pgfpathlineto{\pgfqpoint{2.044445in}{1.105134in}}%
\pgfpathlineto{\pgfqpoint{2.049693in}{1.118362in}}%
\pgfpathlineto{\pgfqpoint{2.042890in}{1.205414in}}%
\pgfpathlineto{\pgfqpoint{2.037992in}{1.266484in}}%
\pgfpathlineto{\pgfqpoint{2.044759in}{1.261424in}}%
\pgfpathlineto{\pgfqpoint{2.052309in}{1.261561in}}%
\pgfpathlineto{\pgfqpoint{2.070140in}{1.271890in}}%
\pgfpathlineto{\pgfqpoint{2.124833in}{1.276993in}}%
\pgfpathlineto{\pgfqpoint{2.165316in}{1.281260in}}%
\pgfpathlineto{\pgfqpoint{2.165674in}{1.277298in}}%
\pgfpathlineto{\pgfqpoint{2.174915in}{1.193611in}}%
\pgfpathlineto{\pgfqpoint{2.182869in}{1.115743in}}%
\pgfpathlineto{\pgfqpoint{2.179440in}{1.112087in}}%
\pgfpathlineto{\pgfqpoint{2.184707in}{1.103013in}}%
\pgfpathlineto{\pgfqpoint{2.184680in}{1.096474in}}%
\pgfpathlineto{\pgfqpoint{2.177165in}{1.094831in}}%
\pgfpathlineto{\pgfqpoint{2.168755in}{1.088527in}}%
\pgfpathlineto{\pgfqpoint{2.163061in}{1.091009in}}%
\pgfpathlineto{\pgfqpoint{2.154539in}{1.086973in}}%
\pgfpathlineto{\pgfqpoint{2.157176in}{1.078868in}}%
\pgfpathlineto{\pgfqpoint{2.148416in}{1.070726in}}%
\pgfpathlineto{\pgfqpoint{2.145998in}{1.061307in}}%
\pgfpathlineto{\pgfqpoint{2.139975in}{1.059762in}}%
\pgfpathlineto{\pgfqpoint{2.135480in}{1.052628in}}%
\pgfpathlineto{\pgfqpoint{2.136068in}{1.045445in}}%
\pgfpathlineto{\pgfqpoint{2.130781in}{1.040393in}}%
\pgfpathlineto{\pgfqpoint{2.122824in}{1.041173in}}%
\pgfpathlineto{\pgfqpoint{2.116776in}{1.048924in}}%
\pgfpathlineto{\pgfqpoint{2.106527in}{1.041459in}}%
\pgfpathlineto{\pgfqpoint{2.107246in}{1.034937in}}%
\pgfpathlineto{\pgfqpoint{2.095848in}{1.031127in}}%
\pgfpathlineto{\pgfqpoint{2.092985in}{1.035923in}}%
\pgfpathlineto{\pgfqpoint{2.081888in}{1.030484in}}%
\pgfpathlineto{\pgfqpoint{2.077835in}{1.022698in}}%
\pgfpathlineto{\pgfqpoint{2.064471in}{1.030682in}}%
\pgfpathlineto{\pgfqpoint{2.043291in}{1.025393in}}%
\pgfpathclose%
\pgfusepath{fill}%
\end{pgfscope}%
\begin{pgfscope}%
\pgfpathrectangle{\pgfqpoint{0.100000in}{0.100000in}}{\pgfqpoint{2.857344in}{1.829167in}}%
\pgfusepath{clip}%
\pgfsetbuttcap%
\pgfsetmiterjoin%
\definecolor{currentfill}{rgb}{0.997309,0.916571,0.609073}%
\pgfsetfillcolor{currentfill}%
\pgfsetlinewidth{0.000000pt}%
\definecolor{currentstroke}{rgb}{0.000000,0.000000,0.000000}%
\pgfsetstrokecolor{currentstroke}%
\pgfsetstrokeopacity{0.000000}%
\pgfsetdash{}{0pt}%
\pgfpathmoveto{\pgfqpoint{0.539490in}{1.427861in}}%
\pgfpathlineto{\pgfqpoint{0.570118in}{1.419338in}}%
\pgfpathlineto{\pgfqpoint{0.620888in}{1.406247in}}%
\pgfpathlineto{\pgfqpoint{0.673515in}{1.392762in}}%
\pgfpathlineto{\pgfqpoint{0.722022in}{1.381064in}}%
\pgfpathlineto{\pgfqpoint{0.763936in}{1.371575in}}%
\pgfpathlineto{\pgfqpoint{0.808777in}{1.361714in}}%
\pgfpathlineto{\pgfqpoint{0.795822in}{1.300572in}}%
\pgfpathlineto{\pgfqpoint{0.784279in}{1.246274in}}%
\pgfpathlineto{\pgfqpoint{0.765345in}{1.158664in}}%
\pgfpathlineto{\pgfqpoint{0.751129in}{1.092538in}}%
\pgfpathlineto{\pgfqpoint{0.743449in}{1.055636in}}%
\pgfpathlineto{\pgfqpoint{0.733601in}{1.007741in}}%
\pgfpathlineto{\pgfqpoint{0.726788in}{0.998025in}}%
\pgfpathlineto{\pgfqpoint{0.721304in}{0.997701in}}%
\pgfpathlineto{\pgfqpoint{0.717393in}{1.006182in}}%
\pgfpathlineto{\pgfqpoint{0.708412in}{1.009262in}}%
\pgfpathlineto{\pgfqpoint{0.698734in}{1.008183in}}%
\pgfpathlineto{\pgfqpoint{0.695988in}{1.001245in}}%
\pgfpathlineto{\pgfqpoint{0.695701in}{0.977154in}}%
\pgfpathlineto{\pgfqpoint{0.692803in}{0.971654in}}%
\pgfpathlineto{\pgfqpoint{0.694459in}{0.952313in}}%
\pgfpathlineto{\pgfqpoint{0.688408in}{0.939428in}}%
\pgfpathlineto{\pgfqpoint{0.639332in}{1.014996in}}%
\pgfpathlineto{\pgfqpoint{0.591002in}{1.088427in}}%
\pgfpathlineto{\pgfqpoint{0.565843in}{1.126933in}}%
\pgfpathlineto{\pgfqpoint{0.544856in}{1.160169in}}%
\pgfpathlineto{\pgfqpoint{0.518412in}{1.201282in}}%
\pgfpathlineto{\pgfqpoint{0.488477in}{1.247399in}}%
\pgfpathlineto{\pgfqpoint{0.500784in}{1.291173in}}%
\pgfpathlineto{\pgfqpoint{0.525551in}{1.378961in}}%
\pgfpathclose%
\pgfusepath{fill}%
\end{pgfscope}%
\begin{pgfscope}%
\pgfpathrectangle{\pgfqpoint{0.100000in}{0.100000in}}{\pgfqpoint{2.857344in}{1.829167in}}%
\pgfusepath{clip}%
\pgfsetbuttcap%
\pgfsetmiterjoin%
\definecolor{currentfill}{rgb}{0.693272,0.875894,0.638524}%
\pgfsetfillcolor{currentfill}%
\pgfsetlinewidth{0.000000pt}%
\definecolor{currentstroke}{rgb}{0.000000,0.000000,0.000000}%
\pgfsetstrokecolor{currentstroke}%
\pgfsetstrokeopacity{0.000000}%
\pgfsetdash{}{0pt}%
\pgfpathmoveto{\pgfqpoint{0.808777in}{1.361714in}}%
\pgfpathlineto{\pgfqpoint{0.856718in}{1.352178in}}%
\pgfpathlineto{\pgfqpoint{0.945471in}{1.335044in}}%
\pgfpathlineto{\pgfqpoint{0.934362in}{1.273417in}}%
\pgfpathlineto{\pgfqpoint{0.983224in}{1.265052in}}%
\pgfpathlineto{\pgfqpoint{1.027739in}{1.257928in}}%
\pgfpathlineto{\pgfqpoint{1.020004in}{1.209206in}}%
\pgfpathlineto{\pgfqpoint{1.011808in}{1.156667in}}%
\pgfpathlineto{\pgfqpoint{1.000834in}{1.087682in}}%
\pgfpathlineto{\pgfqpoint{1.000550in}{1.081899in}}%
\pgfpathlineto{\pgfqpoint{0.989159in}{1.010405in}}%
\pgfpathlineto{\pgfqpoint{0.942269in}{1.017673in}}%
\pgfpathlineto{\pgfqpoint{0.918375in}{1.022471in}}%
\pgfpathlineto{\pgfqpoint{0.832010in}{1.037653in}}%
\pgfpathlineto{\pgfqpoint{0.799487in}{1.044067in}}%
\pgfpathlineto{\pgfqpoint{0.743449in}{1.055636in}}%
\pgfpathlineto{\pgfqpoint{0.751129in}{1.092538in}}%
\pgfpathlineto{\pgfqpoint{0.765345in}{1.158664in}}%
\pgfpathlineto{\pgfqpoint{0.784279in}{1.246274in}}%
\pgfpathlineto{\pgfqpoint{0.795822in}{1.300572in}}%
\pgfpathclose%
\pgfusepath{fill}%
\end{pgfscope}%
\begin{pgfscope}%
\pgfpathrectangle{\pgfqpoint{0.100000in}{0.100000in}}{\pgfqpoint{2.857344in}{1.829167in}}%
\pgfusepath{clip}%
\pgfsetbuttcap%
\pgfsetmiterjoin%
\definecolor{currentfill}{rgb}{0.998385,0.949942,0.665052}%
\pgfsetfillcolor{currentfill}%
\pgfsetlinewidth{0.000000pt}%
\definecolor{currentstroke}{rgb}{0.000000,0.000000,0.000000}%
\pgfsetstrokecolor{currentstroke}%
\pgfsetstrokeopacity{0.000000}%
\pgfsetdash{}{0pt}%
\pgfpathmoveto{\pgfqpoint{0.352095in}{1.486015in}}%
\pgfpathlineto{\pgfqpoint{0.369129in}{1.480073in}}%
\pgfpathlineto{\pgfqpoint{0.437366in}{1.458829in}}%
\pgfpathlineto{\pgfqpoint{0.500137in}{1.439071in}}%
\pgfpathlineto{\pgfqpoint{0.539490in}{1.427861in}}%
\pgfpathlineto{\pgfqpoint{0.525551in}{1.378961in}}%
\pgfpathlineto{\pgfqpoint{0.500784in}{1.291173in}}%
\pgfpathlineto{\pgfqpoint{0.488477in}{1.247399in}}%
\pgfpathlineto{\pgfqpoint{0.518412in}{1.201282in}}%
\pgfpathlineto{\pgfqpoint{0.544856in}{1.160169in}}%
\pgfpathlineto{\pgfqpoint{0.565843in}{1.126933in}}%
\pgfpathlineto{\pgfqpoint{0.591002in}{1.088427in}}%
\pgfpathlineto{\pgfqpoint{0.639332in}{1.014996in}}%
\pgfpathlineto{\pgfqpoint{0.688408in}{0.939428in}}%
\pgfpathlineto{\pgfqpoint{0.686437in}{0.931934in}}%
\pgfpathlineto{\pgfqpoint{0.692357in}{0.919999in}}%
\pgfpathlineto{\pgfqpoint{0.693527in}{0.903672in}}%
\pgfpathlineto{\pgfqpoint{0.702124in}{0.895753in}}%
\pgfpathlineto{\pgfqpoint{0.702464in}{0.889371in}}%
\pgfpathlineto{\pgfqpoint{0.687071in}{0.882129in}}%
\pgfpathlineto{\pgfqpoint{0.679738in}{0.874878in}}%
\pgfpathlineto{\pgfqpoint{0.677451in}{0.858865in}}%
\pgfpathlineto{\pgfqpoint{0.673739in}{0.850133in}}%
\pgfpathlineto{\pgfqpoint{0.665917in}{0.842772in}}%
\pgfpathlineto{\pgfqpoint{0.658138in}{0.823633in}}%
\pgfpathlineto{\pgfqpoint{0.669076in}{0.813662in}}%
\pgfpathlineto{\pgfqpoint{0.667664in}{0.805443in}}%
\pgfpathlineto{\pgfqpoint{0.658695in}{0.799724in}}%
\pgfpathlineto{\pgfqpoint{0.652512in}{0.800733in}}%
\pgfpathlineto{\pgfqpoint{0.579523in}{0.810862in}}%
\pgfpathlineto{\pgfqpoint{0.525614in}{0.818510in}}%
\pgfpathlineto{\pgfqpoint{0.527971in}{0.827214in}}%
\pgfpathlineto{\pgfqpoint{0.524462in}{0.841664in}}%
\pgfpathlineto{\pgfqpoint{0.524108in}{0.856240in}}%
\pgfpathlineto{\pgfqpoint{0.521815in}{0.864769in}}%
\pgfpathlineto{\pgfqpoint{0.514749in}{0.876964in}}%
\pgfpathlineto{\pgfqpoint{0.494427in}{0.905101in}}%
\pgfpathlineto{\pgfqpoint{0.487772in}{0.908549in}}%
\pgfpathlineto{\pgfqpoint{0.479196in}{0.908494in}}%
\pgfpathlineto{\pgfqpoint{0.481155in}{0.917370in}}%
\pgfpathlineto{\pgfqpoint{0.477094in}{0.928470in}}%
\pgfpathlineto{\pgfqpoint{0.457148in}{0.933987in}}%
\pgfpathlineto{\pgfqpoint{0.444990in}{0.944202in}}%
\pgfpathlineto{\pgfqpoint{0.443981in}{0.950453in}}%
\pgfpathlineto{\pgfqpoint{0.429981in}{0.965927in}}%
\pgfpathlineto{\pgfqpoint{0.416655in}{0.968889in}}%
\pgfpathlineto{\pgfqpoint{0.404308in}{0.976753in}}%
\pgfpathlineto{\pgfqpoint{0.388068in}{0.979474in}}%
\pgfpathlineto{\pgfqpoint{0.381130in}{0.989962in}}%
\pgfpathlineto{\pgfqpoint{0.387719in}{1.006565in}}%
\pgfpathlineto{\pgfqpoint{0.385704in}{1.010298in}}%
\pgfpathlineto{\pgfqpoint{0.391191in}{1.024138in}}%
\pgfpathlineto{\pgfqpoint{0.381430in}{1.031503in}}%
\pgfpathlineto{\pgfqpoint{0.384597in}{1.044856in}}%
\pgfpathlineto{\pgfqpoint{0.379383in}{1.048269in}}%
\pgfpathlineto{\pgfqpoint{0.374882in}{1.060890in}}%
\pgfpathlineto{\pgfqpoint{0.369449in}{1.064739in}}%
\pgfpathlineto{\pgfqpoint{0.369041in}{1.073855in}}%
\pgfpathlineto{\pgfqpoint{0.364785in}{1.080283in}}%
\pgfpathlineto{\pgfqpoint{0.358371in}{1.101933in}}%
\pgfpathlineto{\pgfqpoint{0.351354in}{1.112303in}}%
\pgfpathlineto{\pgfqpoint{0.352879in}{1.129906in}}%
\pgfpathlineto{\pgfqpoint{0.361131in}{1.131677in}}%
\pgfpathlineto{\pgfqpoint{0.366505in}{1.141225in}}%
\pgfpathlineto{\pgfqpoint{0.363258in}{1.151580in}}%
\pgfpathlineto{\pgfqpoint{0.354481in}{1.153309in}}%
\pgfpathlineto{\pgfqpoint{0.350102in}{1.158152in}}%
\pgfpathlineto{\pgfqpoint{0.343012in}{1.175968in}}%
\pgfpathlineto{\pgfqpoint{0.346326in}{1.182366in}}%
\pgfpathlineto{\pgfqpoint{0.343975in}{1.194280in}}%
\pgfpathlineto{\pgfqpoint{0.349180in}{1.209710in}}%
\pgfpathlineto{\pgfqpoint{0.355278in}{1.204027in}}%
\pgfpathlineto{\pgfqpoint{0.352511in}{1.197319in}}%
\pgfpathlineto{\pgfqpoint{0.363078in}{1.186685in}}%
\pgfpathlineto{\pgfqpoint{0.362403in}{1.202476in}}%
\pgfpathlineto{\pgfqpoint{0.357875in}{1.206711in}}%
\pgfpathlineto{\pgfqpoint{0.363051in}{1.220596in}}%
\pgfpathlineto{\pgfqpoint{0.382627in}{1.218385in}}%
\pgfpathlineto{\pgfqpoint{0.379992in}{1.223548in}}%
\pgfpathlineto{\pgfqpoint{0.367072in}{1.223040in}}%
\pgfpathlineto{\pgfqpoint{0.360926in}{1.230860in}}%
\pgfpathlineto{\pgfqpoint{0.353190in}{1.223916in}}%
\pgfpathlineto{\pgfqpoint{0.349084in}{1.212308in}}%
\pgfpathlineto{\pgfqpoint{0.338127in}{1.227928in}}%
\pgfpathlineto{\pgfqpoint{0.333907in}{1.230771in}}%
\pgfpathlineto{\pgfqpoint{0.335515in}{1.247771in}}%
\pgfpathlineto{\pgfqpoint{0.332175in}{1.257795in}}%
\pgfpathlineto{\pgfqpoint{0.326121in}{1.267213in}}%
\pgfpathlineto{\pgfqpoint{0.313723in}{1.296071in}}%
\pgfpathlineto{\pgfqpoint{0.317777in}{1.302452in}}%
\pgfpathlineto{\pgfqpoint{0.317730in}{1.322619in}}%
\pgfpathlineto{\pgfqpoint{0.324421in}{1.333868in}}%
\pgfpathlineto{\pgfqpoint{0.325955in}{1.351447in}}%
\pgfpathlineto{\pgfqpoint{0.319645in}{1.371588in}}%
\pgfpathlineto{\pgfqpoint{0.311275in}{1.384409in}}%
\pgfpathlineto{\pgfqpoint{0.312768in}{1.395978in}}%
\pgfpathlineto{\pgfqpoint{0.336274in}{1.423977in}}%
\pgfpathlineto{\pgfqpoint{0.337442in}{1.433528in}}%
\pgfpathlineto{\pgfqpoint{0.348033in}{1.451747in}}%
\pgfpathlineto{\pgfqpoint{0.349503in}{1.469017in}}%
\pgfpathlineto{\pgfqpoint{0.346097in}{1.473423in}}%
\pgfpathclose%
\pgfusepath{fill}%
\end{pgfscope}%
\begin{pgfscope}%
\pgfpathrectangle{\pgfqpoint{0.100000in}{0.100000in}}{\pgfqpoint{2.857344in}{1.829167in}}%
\pgfusepath{clip}%
\pgfsetbuttcap%
\pgfsetmiterjoin%
\definecolor{currentfill}{rgb}{0.999308,0.978547,0.713033}%
\pgfsetfillcolor{currentfill}%
\pgfsetlinewidth{0.000000pt}%
\definecolor{currentstroke}{rgb}{0.000000,0.000000,0.000000}%
\pgfsetstrokecolor{currentstroke}%
\pgfsetstrokeopacity{0.000000}%
\pgfsetdash{}{0pt}%
\pgfpathmoveto{\pgfqpoint{2.182869in}{1.115743in}}%
\pgfpathlineto{\pgfqpoint{2.174915in}{1.193611in}}%
\pgfpathlineto{\pgfqpoint{2.165674in}{1.277298in}}%
\pgfpathlineto{\pgfqpoint{2.226192in}{1.286311in}}%
\pgfpathlineto{\pgfqpoint{2.242254in}{1.282131in}}%
\pgfpathlineto{\pgfqpoint{2.249958in}{1.277593in}}%
\pgfpathlineto{\pgfqpoint{2.259611in}{1.278852in}}%
\pgfpathlineto{\pgfqpoint{2.272338in}{1.271339in}}%
\pgfpathlineto{\pgfqpoint{2.296045in}{1.282521in}}%
\pgfpathlineto{\pgfqpoint{2.309128in}{1.282894in}}%
\pgfpathlineto{\pgfqpoint{2.324405in}{1.299938in}}%
\pgfpathlineto{\pgfqpoint{2.339950in}{1.310306in}}%
\pgfpathlineto{\pgfqpoint{2.360683in}{1.322248in}}%
\pgfpathlineto{\pgfqpoint{2.374030in}{1.238770in}}%
\pgfpathlineto{\pgfqpoint{2.367829in}{1.233409in}}%
\pgfpathlineto{\pgfqpoint{2.371832in}{1.228484in}}%
\pgfpathlineto{\pgfqpoint{2.373427in}{1.217689in}}%
\pgfpathlineto{\pgfqpoint{2.370321in}{1.207550in}}%
\pgfpathlineto{\pgfqpoint{2.369910in}{1.192730in}}%
\pgfpathlineto{\pgfqpoint{2.366992in}{1.173441in}}%
\pgfpathlineto{\pgfqpoint{2.352231in}{1.156237in}}%
\pgfpathlineto{\pgfqpoint{2.346018in}{1.152585in}}%
\pgfpathlineto{\pgfqpoint{2.341185in}{1.155729in}}%
\pgfpathlineto{\pgfqpoint{2.329227in}{1.139332in}}%
\pgfpathlineto{\pgfqpoint{2.330347in}{1.123543in}}%
\pgfpathlineto{\pgfqpoint{2.317029in}{1.127340in}}%
\pgfpathlineto{\pgfqpoint{2.310722in}{1.111572in}}%
\pgfpathlineto{\pgfqpoint{2.313910in}{1.100398in}}%
\pgfpathlineto{\pgfqpoint{2.308920in}{1.098747in}}%
\pgfpathlineto{\pgfqpoint{2.308222in}{1.089919in}}%
\pgfpathlineto{\pgfqpoint{2.295973in}{1.086357in}}%
\pgfpathlineto{\pgfqpoint{2.289608in}{1.093486in}}%
\pgfpathlineto{\pgfqpoint{2.281417in}{1.096250in}}%
\pgfpathlineto{\pgfqpoint{2.279493in}{1.103478in}}%
\pgfpathlineto{\pgfqpoint{2.272088in}{1.102206in}}%
\pgfpathlineto{\pgfqpoint{2.267236in}{1.095558in}}%
\pgfpathlineto{\pgfqpoint{2.260296in}{1.093220in}}%
\pgfpathlineto{\pgfqpoint{2.248036in}{1.097925in}}%
\pgfpathlineto{\pgfqpoint{2.241256in}{1.092327in}}%
\pgfpathlineto{\pgfqpoint{2.231619in}{1.098950in}}%
\pgfpathlineto{\pgfqpoint{2.213131in}{1.100983in}}%
\pgfpathlineto{\pgfqpoint{2.207555in}{1.113014in}}%
\pgfpathlineto{\pgfqpoint{2.200490in}{1.118345in}}%
\pgfpathlineto{\pgfqpoint{2.193640in}{1.114933in}}%
\pgfpathclose%
\pgfusepath{fill}%
\end{pgfscope}%
\begin{pgfscope}%
\pgfpathrectangle{\pgfqpoint{0.100000in}{0.100000in}}{\pgfqpoint{2.857344in}{1.829167in}}%
\pgfusepath{clip}%
\pgfsetbuttcap%
\pgfsetmiterjoin%
\definecolor{currentfill}{rgb}{0.999616,0.988082,0.729027}%
\pgfsetfillcolor{currentfill}%
\pgfsetlinewidth{0.000000pt}%
\definecolor{currentstroke}{rgb}{0.000000,0.000000,0.000000}%
\pgfsetstrokecolor{currentstroke}%
\pgfsetstrokeopacity{0.000000}%
\pgfsetdash{}{0pt}%
\pgfpathmoveto{\pgfqpoint{1.982509in}{0.965148in}}%
\pgfpathlineto{\pgfqpoint{1.975418in}{0.970908in}}%
\pgfpathlineto{\pgfqpoint{1.969637in}{0.968168in}}%
\pgfpathlineto{\pgfqpoint{1.962245in}{0.981969in}}%
\pgfpathlineto{\pgfqpoint{1.966021in}{0.990644in}}%
\pgfpathlineto{\pgfqpoint{1.960579in}{1.000408in}}%
\pgfpathlineto{\pgfqpoint{1.960287in}{1.008095in}}%
\pgfpathlineto{\pgfqpoint{1.952933in}{1.010834in}}%
\pgfpathlineto{\pgfqpoint{1.949523in}{1.016626in}}%
\pgfpathlineto{\pgfqpoint{1.935145in}{1.023860in}}%
\pgfpathlineto{\pgfqpoint{1.922658in}{1.032772in}}%
\pgfpathlineto{\pgfqpoint{1.916856in}{1.039501in}}%
\pgfpathlineto{\pgfqpoint{1.916736in}{1.047707in}}%
\pgfpathlineto{\pgfqpoint{1.924500in}{1.063463in}}%
\pgfpathlineto{\pgfqpoint{1.926888in}{1.075514in}}%
\pgfpathlineto{\pgfqpoint{1.920562in}{1.082323in}}%
\pgfpathlineto{\pgfqpoint{1.912185in}{1.084889in}}%
\pgfpathlineto{\pgfqpoint{1.902022in}{1.079279in}}%
\pgfpathlineto{\pgfqpoint{1.897720in}{1.088911in}}%
\pgfpathlineto{\pgfqpoint{1.895549in}{1.101967in}}%
\pgfpathlineto{\pgfqpoint{1.880571in}{1.113606in}}%
\pgfpathlineto{\pgfqpoint{1.877578in}{1.118771in}}%
\pgfpathlineto{\pgfqpoint{1.863897in}{1.130464in}}%
\pgfpathlineto{\pgfqpoint{1.859609in}{1.138964in}}%
\pgfpathlineto{\pgfqpoint{1.855744in}{1.155821in}}%
\pgfpathlineto{\pgfqpoint{1.858393in}{1.170794in}}%
\pgfpathlineto{\pgfqpoint{1.861930in}{1.172881in}}%
\pgfpathlineto{\pgfqpoint{1.861293in}{1.185415in}}%
\pgfpathlineto{\pgfqpoint{1.871276in}{1.189138in}}%
\pgfpathlineto{\pgfqpoint{1.874290in}{1.200386in}}%
\pgfpathlineto{\pgfqpoint{1.880033in}{1.207956in}}%
\pgfpathlineto{\pgfqpoint{1.879743in}{1.217587in}}%
\pgfpathlineto{\pgfqpoint{1.872625in}{1.225248in}}%
\pgfpathlineto{\pgfqpoint{1.874315in}{1.235970in}}%
\pgfpathlineto{\pgfqpoint{1.892767in}{1.240633in}}%
\pgfpathlineto{\pgfqpoint{1.906923in}{1.249144in}}%
\pgfpathlineto{\pgfqpoint{1.908409in}{1.259862in}}%
\pgfpathlineto{\pgfqpoint{1.913334in}{1.263236in}}%
\pgfpathlineto{\pgfqpoint{1.915219in}{1.274495in}}%
\pgfpathlineto{\pgfqpoint{1.913704in}{1.281931in}}%
\pgfpathlineto{\pgfqpoint{1.904026in}{1.288109in}}%
\pgfpathlineto{\pgfqpoint{1.900130in}{1.297317in}}%
\pgfpathlineto{\pgfqpoint{1.890579in}{1.306219in}}%
\pgfpathlineto{\pgfqpoint{1.969068in}{1.309564in}}%
\pgfpathlineto{\pgfqpoint{2.021726in}{1.313305in}}%
\pgfpathlineto{\pgfqpoint{2.020763in}{1.302228in}}%
\pgfpathlineto{\pgfqpoint{2.029737in}{1.286949in}}%
\pgfpathlineto{\pgfqpoint{2.033505in}{1.273895in}}%
\pgfpathlineto{\pgfqpoint{2.037992in}{1.266484in}}%
\pgfpathlineto{\pgfqpoint{2.042890in}{1.205414in}}%
\pgfpathlineto{\pgfqpoint{2.049693in}{1.118362in}}%
\pgfpathlineto{\pgfqpoint{2.044445in}{1.105134in}}%
\pgfpathlineto{\pgfqpoint{2.052005in}{1.094261in}}%
\pgfpathlineto{\pgfqpoint{2.054113in}{1.083876in}}%
\pgfpathlineto{\pgfqpoint{2.043686in}{1.062342in}}%
\pgfpathlineto{\pgfqpoint{2.044301in}{1.059367in}}%
\pgfpathlineto{\pgfqpoint{2.033570in}{1.047023in}}%
\pgfpathlineto{\pgfqpoint{2.036513in}{1.043254in}}%
\pgfpathlineto{\pgfqpoint{2.032048in}{1.035560in}}%
\pgfpathlineto{\pgfqpoint{2.033197in}{1.020052in}}%
\pgfpathlineto{\pgfqpoint{2.027774in}{1.010536in}}%
\pgfpathlineto{\pgfqpoint{2.032184in}{0.999291in}}%
\pgfpathlineto{\pgfqpoint{2.013705in}{0.993181in}}%
\pgfpathlineto{\pgfqpoint{2.012002in}{0.986536in}}%
\pgfpathlineto{\pgfqpoint{2.017048in}{0.978113in}}%
\pgfpathlineto{\pgfqpoint{2.014726in}{0.972624in}}%
\pgfpathlineto{\pgfqpoint{1.994911in}{0.979405in}}%
\pgfpathlineto{\pgfqpoint{1.988373in}{0.980087in}}%
\pgfpathlineto{\pgfqpoint{1.980222in}{0.969778in}}%
\pgfpathclose%
\pgfusepath{fill}%
\end{pgfscope}%
\begin{pgfscope}%
\pgfpathrectangle{\pgfqpoint{0.100000in}{0.100000in}}{\pgfqpoint{2.857344in}{1.829167in}}%
\pgfusepath{clip}%
\pgfsetbuttcap%
\pgfsetmiterjoin%
\definecolor{currentfill}{rgb}{0.967320,0.986928,0.698039}%
\pgfsetfillcolor{currentfill}%
\pgfsetlinewidth{0.000000pt}%
\definecolor{currentstroke}{rgb}{0.000000,0.000000,0.000000}%
\pgfsetstrokecolor{currentstroke}%
\pgfsetstrokeopacity{0.000000}%
\pgfsetdash{}{0pt}%
\pgfpathmoveto{\pgfqpoint{2.559306in}{1.154607in}}%
\pgfpathlineto{\pgfqpoint{2.553852in}{1.162705in}}%
\pgfpathlineto{\pgfqpoint{2.556929in}{1.167236in}}%
\pgfpathlineto{\pgfqpoint{2.564466in}{1.162147in}}%
\pgfpathclose%
\pgfusepath{fill}%
\end{pgfscope}%
\begin{pgfscope}%
\pgfpathrectangle{\pgfqpoint{0.100000in}{0.100000in}}{\pgfqpoint{2.857344in}{1.829167in}}%
\pgfusepath{clip}%
\pgfsetbuttcap%
\pgfsetmiterjoin%
\definecolor{currentfill}{rgb}{0.982699,0.993080,0.722030}%
\pgfsetfillcolor{currentfill}%
\pgfsetlinewidth{0.000000pt}%
\definecolor{currentstroke}{rgb}{0.000000,0.000000,0.000000}%
\pgfsetstrokecolor{currentstroke}%
\pgfsetstrokeopacity{0.000000}%
\pgfsetdash{}{0pt}%
\pgfpathmoveto{\pgfqpoint{2.606924in}{1.224286in}}%
\pgfpathlineto{\pgfqpoint{2.610183in}{1.231171in}}%
\pgfpathlineto{\pgfqpoint{2.623356in}{1.232658in}}%
\pgfpathlineto{\pgfqpoint{2.621245in}{1.226807in}}%
\pgfpathlineto{\pgfqpoint{2.616899in}{1.219332in}}%
\pgfpathlineto{\pgfqpoint{2.619844in}{1.210426in}}%
\pgfpathlineto{\pgfqpoint{2.631467in}{1.199756in}}%
\pgfpathlineto{\pgfqpoint{2.634165in}{1.188519in}}%
\pgfpathlineto{\pgfqpoint{2.647556in}{1.174527in}}%
\pgfpathlineto{\pgfqpoint{2.652809in}{1.175125in}}%
\pgfpathlineto{\pgfqpoint{2.659353in}{1.154120in}}%
\pgfpathlineto{\pgfqpoint{2.658280in}{1.153911in}}%
\pgfpathlineto{\pgfqpoint{2.657087in}{1.153676in}}%
\pgfpathlineto{\pgfqpoint{2.627900in}{1.148089in}}%
\pgfpathlineto{\pgfqpoint{2.612285in}{1.203624in}}%
\pgfpathclose%
\pgfusepath{fill}%
\end{pgfscope}%
\begin{pgfscope}%
\pgfpathrectangle{\pgfqpoint{0.100000in}{0.100000in}}{\pgfqpoint{2.857344in}{1.829167in}}%
\pgfusepath{clip}%
\pgfsetbuttcap%
\pgfsetmiterjoin%
\definecolor{currentfill}{rgb}{0.998078,0.940408,0.649058}%
\pgfsetfillcolor{currentfill}%
\pgfsetlinewidth{0.000000pt}%
\definecolor{currentstroke}{rgb}{0.000000,0.000000,0.000000}%
\pgfsetstrokecolor{currentstroke}%
\pgfsetstrokeopacity{0.000000}%
\pgfsetdash{}{0pt}%
\pgfpathmoveto{\pgfqpoint{2.334313in}{1.036045in}}%
\pgfpathlineto{\pgfqpoint{2.325090in}{1.036384in}}%
\pgfpathlineto{\pgfqpoint{2.316592in}{1.042238in}}%
\pgfpathlineto{\pgfqpoint{2.305111in}{1.060030in}}%
\pgfpathlineto{\pgfqpoint{2.295338in}{1.069453in}}%
\pgfpathlineto{\pgfqpoint{2.297892in}{1.076731in}}%
\pgfpathlineto{\pgfqpoint{2.295973in}{1.086357in}}%
\pgfpathlineto{\pgfqpoint{2.308222in}{1.089919in}}%
\pgfpathlineto{\pgfqpoint{2.308920in}{1.098747in}}%
\pgfpathlineto{\pgfqpoint{2.313910in}{1.100398in}}%
\pgfpathlineto{\pgfqpoint{2.310722in}{1.111572in}}%
\pgfpathlineto{\pgfqpoint{2.317029in}{1.127340in}}%
\pgfpathlineto{\pgfqpoint{2.330347in}{1.123543in}}%
\pgfpathlineto{\pgfqpoint{2.329227in}{1.139332in}}%
\pgfpathlineto{\pgfqpoint{2.341185in}{1.155729in}}%
\pgfpathlineto{\pgfqpoint{2.346018in}{1.152585in}}%
\pgfpathlineto{\pgfqpoint{2.352231in}{1.156237in}}%
\pgfpathlineto{\pgfqpoint{2.366992in}{1.173441in}}%
\pgfpathlineto{\pgfqpoint{2.369910in}{1.192730in}}%
\pgfpathlineto{\pgfqpoint{2.370321in}{1.207550in}}%
\pgfpathlineto{\pgfqpoint{2.373427in}{1.217689in}}%
\pgfpathlineto{\pgfqpoint{2.371832in}{1.228484in}}%
\pgfpathlineto{\pgfqpoint{2.367829in}{1.233409in}}%
\pgfpathlineto{\pgfqpoint{2.374030in}{1.238770in}}%
\pgfpathlineto{\pgfqpoint{2.383019in}{1.182138in}}%
\pgfpathlineto{\pgfqpoint{2.432573in}{1.190332in}}%
\pgfpathlineto{\pgfqpoint{2.437709in}{1.158008in}}%
\pgfpathlineto{\pgfqpoint{2.455650in}{1.179344in}}%
\pgfpathlineto{\pgfqpoint{2.459877in}{1.177216in}}%
\pgfpathlineto{\pgfqpoint{2.466096in}{1.189579in}}%
\pgfpathlineto{\pgfqpoint{2.474710in}{1.185702in}}%
\pgfpathlineto{\pgfqpoint{2.482222in}{1.186439in}}%
\pgfpathlineto{\pgfqpoint{2.487178in}{1.195033in}}%
\pgfpathlineto{\pgfqpoint{2.494371in}{1.199814in}}%
\pgfpathlineto{\pgfqpoint{2.504372in}{1.195604in}}%
\pgfpathlineto{\pgfqpoint{2.510953in}{1.197058in}}%
\pgfpathlineto{\pgfqpoint{2.520380in}{1.180743in}}%
\pgfpathlineto{\pgfqpoint{2.517654in}{1.168391in}}%
\pgfpathlineto{\pgfqpoint{2.493031in}{1.182298in}}%
\pgfpathlineto{\pgfqpoint{2.488419in}{1.170881in}}%
\pgfpathlineto{\pgfqpoint{2.489941in}{1.165629in}}%
\pgfpathlineto{\pgfqpoint{2.479461in}{1.147872in}}%
\pgfpathlineto{\pgfqpoint{2.474515in}{1.144340in}}%
\pgfpathlineto{\pgfqpoint{2.472282in}{1.136510in}}%
\pgfpathlineto{\pgfqpoint{2.463850in}{1.137329in}}%
\pgfpathlineto{\pgfqpoint{2.457795in}{1.115957in}}%
\pgfpathlineto{\pgfqpoint{2.454405in}{1.111053in}}%
\pgfpathlineto{\pgfqpoint{2.445695in}{1.112688in}}%
\pgfpathlineto{\pgfqpoint{2.436786in}{1.119429in}}%
\pgfpathlineto{\pgfqpoint{2.436478in}{1.109084in}}%
\pgfpathlineto{\pgfqpoint{2.427853in}{1.091696in}}%
\pgfpathlineto{\pgfqpoint{2.425708in}{1.079318in}}%
\pgfpathlineto{\pgfqpoint{2.419091in}{1.071074in}}%
\pgfpathlineto{\pgfqpoint{2.414074in}{1.057909in}}%
\pgfpathlineto{\pgfqpoint{2.413828in}{1.045722in}}%
\pgfpathlineto{\pgfqpoint{2.406080in}{1.043446in}}%
\pgfpathlineto{\pgfqpoint{2.397291in}{1.036546in}}%
\pgfpathlineto{\pgfqpoint{2.388828in}{1.035233in}}%
\pgfpathlineto{\pgfqpoint{2.386878in}{1.029393in}}%
\pgfpathlineto{\pgfqpoint{2.373242in}{1.023413in}}%
\pgfpathlineto{\pgfqpoint{2.365622in}{1.028508in}}%
\pgfpathlineto{\pgfqpoint{2.357104in}{1.018833in}}%
\pgfpathlineto{\pgfqpoint{2.342463in}{1.021685in}}%
\pgfpathlineto{\pgfqpoint{2.333483in}{1.031840in}}%
\pgfpathclose%
\pgfusepath{fill}%
\end{pgfscope}%
\begin{pgfscope}%
\pgfpathrectangle{\pgfqpoint{0.100000in}{0.100000in}}{\pgfqpoint{2.857344in}{1.829167in}}%
\pgfusepath{clip}%
\pgfsetbuttcap%
\pgfsetmiterjoin%
\definecolor{currentfill}{rgb}{0.986544,0.994617,0.728028}%
\pgfsetfillcolor{currentfill}%
\pgfsetlinewidth{0.000000pt}%
\definecolor{currentstroke}{rgb}{0.000000,0.000000,0.000000}%
\pgfsetstrokecolor{currentstroke}%
\pgfsetstrokeopacity{0.000000}%
\pgfsetdash{}{0pt}%
\pgfpathmoveto{\pgfqpoint{2.657087in}{1.153676in}}%
\pgfpathlineto{\pgfqpoint{2.656713in}{1.142264in}}%
\pgfpathlineto{\pgfqpoint{2.652321in}{1.136654in}}%
\pgfpathlineto{\pgfqpoint{2.649518in}{1.124199in}}%
\pgfpathlineto{\pgfqpoint{2.636861in}{1.118458in}}%
\pgfpathlineto{\pgfqpoint{2.626242in}{1.116785in}}%
\pgfpathlineto{\pgfqpoint{2.629336in}{1.124976in}}%
\pgfpathlineto{\pgfqpoint{2.612986in}{1.131840in}}%
\pgfpathlineto{\pgfqpoint{2.599703in}{1.140433in}}%
\pgfpathlineto{\pgfqpoint{2.607926in}{1.153279in}}%
\pgfpathlineto{\pgfqpoint{2.601336in}{1.163264in}}%
\pgfpathlineto{\pgfqpoint{2.602112in}{1.171877in}}%
\pgfpathlineto{\pgfqpoint{2.597346in}{1.174241in}}%
\pgfpathlineto{\pgfqpoint{2.593479in}{1.188234in}}%
\pgfpathlineto{\pgfqpoint{2.602624in}{1.207202in}}%
\pgfpathlineto{\pgfqpoint{2.595753in}{1.210300in}}%
\pgfpathlineto{\pgfqpoint{2.593973in}{1.200942in}}%
\pgfpathlineto{\pgfqpoint{2.584165in}{1.198336in}}%
\pgfpathlineto{\pgfqpoint{2.585197in}{1.167547in}}%
\pgfpathlineto{\pgfqpoint{2.583403in}{1.157638in}}%
\pgfpathlineto{\pgfqpoint{2.588323in}{1.143515in}}%
\pgfpathlineto{\pgfqpoint{2.595890in}{1.136721in}}%
\pgfpathlineto{\pgfqpoint{2.592458in}{1.132454in}}%
\pgfpathlineto{\pgfqpoint{2.600183in}{1.126220in}}%
\pgfpathlineto{\pgfqpoint{2.602976in}{1.116101in}}%
\pgfpathlineto{\pgfqpoint{2.588817in}{1.124480in}}%
\pgfpathlineto{\pgfqpoint{2.579856in}{1.123392in}}%
\pgfpathlineto{\pgfqpoint{2.572898in}{1.132011in}}%
\pgfpathlineto{\pgfqpoint{2.565814in}{1.132851in}}%
\pgfpathlineto{\pgfqpoint{2.555746in}{1.128534in}}%
\pgfpathlineto{\pgfqpoint{2.551841in}{1.133914in}}%
\pgfpathlineto{\pgfqpoint{2.556973in}{1.145207in}}%
\pgfpathlineto{\pgfqpoint{2.559306in}{1.154607in}}%
\pgfpathlineto{\pgfqpoint{2.564466in}{1.162147in}}%
\pgfpathlineto{\pgfqpoint{2.556929in}{1.167236in}}%
\pgfpathlineto{\pgfqpoint{2.553852in}{1.162705in}}%
\pgfpathlineto{\pgfqpoint{2.546324in}{1.167304in}}%
\pgfpathlineto{\pgfqpoint{2.532993in}{1.170369in}}%
\pgfpathlineto{\pgfqpoint{2.534202in}{1.177108in}}%
\pgfpathlineto{\pgfqpoint{2.528160in}{1.181016in}}%
\pgfpathlineto{\pgfqpoint{2.520380in}{1.180743in}}%
\pgfpathlineto{\pgfqpoint{2.510953in}{1.197058in}}%
\pgfpathlineto{\pgfqpoint{2.504372in}{1.195604in}}%
\pgfpathlineto{\pgfqpoint{2.494371in}{1.199814in}}%
\pgfpathlineto{\pgfqpoint{2.487178in}{1.195033in}}%
\pgfpathlineto{\pgfqpoint{2.482222in}{1.186439in}}%
\pgfpathlineto{\pgfqpoint{2.474710in}{1.185702in}}%
\pgfpathlineto{\pgfqpoint{2.466096in}{1.189579in}}%
\pgfpathlineto{\pgfqpoint{2.459877in}{1.177216in}}%
\pgfpathlineto{\pgfqpoint{2.455650in}{1.179344in}}%
\pgfpathlineto{\pgfqpoint{2.437709in}{1.158008in}}%
\pgfpathlineto{\pgfqpoint{2.432573in}{1.190332in}}%
\pgfpathlineto{\pgfqpoint{2.498134in}{1.202483in}}%
\pgfpathlineto{\pgfqpoint{2.527524in}{1.207746in}}%
\pgfpathlineto{\pgfqpoint{2.570281in}{1.216357in}}%
\pgfpathlineto{\pgfqpoint{2.606924in}{1.224286in}}%
\pgfpathlineto{\pgfqpoint{2.612285in}{1.203624in}}%
\pgfpathlineto{\pgfqpoint{2.627900in}{1.148089in}}%
\pgfpathclose%
\pgfusepath{fill}%
\end{pgfscope}%
\begin{pgfscope}%
\pgfpathrectangle{\pgfqpoint{0.100000in}{0.100000in}}{\pgfqpoint{2.857344in}{1.829167in}}%
\pgfusepath{clip}%
\pgfsetbuttcap%
\pgfsetmiterjoin%
\definecolor{currentfill}{rgb}{0.944252,0.977701,0.662053}%
\pgfsetfillcolor{currentfill}%
\pgfsetlinewidth{0.000000pt}%
\definecolor{currentstroke}{rgb}{0.000000,0.000000,0.000000}%
\pgfsetstrokecolor{currentstroke}%
\pgfsetstrokeopacity{0.000000}%
\pgfsetdash{}{0pt}%
\pgfpathmoveto{\pgfqpoint{1.337379in}{0.969343in}}%
\pgfpathlineto{\pgfqpoint{1.289547in}{0.973908in}}%
\pgfpathlineto{\pgfqpoint{1.239925in}{0.978305in}}%
\pgfpathlineto{\pgfqpoint{1.182585in}{0.984286in}}%
\pgfpathlineto{\pgfqpoint{1.097393in}{0.994419in}}%
\pgfpathlineto{\pgfqpoint{1.067171in}{0.999075in}}%
\pgfpathlineto{\pgfqpoint{0.989159in}{1.010405in}}%
\pgfpathlineto{\pgfqpoint{1.000550in}{1.081899in}}%
\pgfpathlineto{\pgfqpoint{1.000834in}{1.087682in}}%
\pgfpathlineto{\pgfqpoint{1.011808in}{1.156667in}}%
\pgfpathlineto{\pgfqpoint{1.020004in}{1.209206in}}%
\pgfpathlineto{\pgfqpoint{1.027739in}{1.257928in}}%
\pgfpathlineto{\pgfqpoint{1.080586in}{1.250342in}}%
\pgfpathlineto{\pgfqpoint{1.129859in}{1.243270in}}%
\pgfpathlineto{\pgfqpoint{1.220435in}{1.232091in}}%
\pgfpathlineto{\pgfqpoint{1.261989in}{1.228296in}}%
\pgfpathlineto{\pgfqpoint{1.327819in}{1.221909in}}%
\pgfpathlineto{\pgfqpoint{1.356297in}{1.219598in}}%
\pgfpathlineto{\pgfqpoint{1.351269in}{1.157246in}}%
\pgfpathlineto{\pgfqpoint{1.346726in}{1.097211in}}%
\pgfpathlineto{\pgfqpoint{1.343071in}{1.048329in}}%
\pgfpathclose%
\pgfusepath{fill}%
\end{pgfscope}%
\begin{pgfscope}%
\pgfpathrectangle{\pgfqpoint{0.100000in}{0.100000in}}{\pgfqpoint{2.857344in}{1.829167in}}%
\pgfusepath{clip}%
\pgfsetbuttcap%
\pgfsetmiterjoin%
\definecolor{currentfill}{rgb}{0.999769,0.992849,0.737024}%
\pgfsetfillcolor{currentfill}%
\pgfsetlinewidth{0.000000pt}%
\definecolor{currentstroke}{rgb}{0.000000,0.000000,0.000000}%
\pgfsetstrokecolor{currentstroke}%
\pgfsetstrokeopacity{0.000000}%
\pgfsetdash{}{0pt}%
\pgfpathmoveto{\pgfqpoint{1.970155in}{0.933898in}}%
\pgfpathlineto{\pgfqpoint{1.971767in}{0.941126in}}%
\pgfpathlineto{\pgfqpoint{1.980125in}{0.939499in}}%
\pgfpathlineto{\pgfqpoint{1.982509in}{0.965148in}}%
\pgfpathlineto{\pgfqpoint{1.980222in}{0.969778in}}%
\pgfpathlineto{\pgfqpoint{1.988373in}{0.980087in}}%
\pgfpathlineto{\pgfqpoint{1.994911in}{0.979405in}}%
\pgfpathlineto{\pgfqpoint{2.014726in}{0.972624in}}%
\pgfpathlineto{\pgfqpoint{2.017048in}{0.978113in}}%
\pgfpathlineto{\pgfqpoint{2.012002in}{0.986536in}}%
\pgfpathlineto{\pgfqpoint{2.013705in}{0.993181in}}%
\pgfpathlineto{\pgfqpoint{2.032184in}{0.999291in}}%
\pgfpathlineto{\pgfqpoint{2.027774in}{1.010536in}}%
\pgfpathlineto{\pgfqpoint{2.033197in}{1.020052in}}%
\pgfpathlineto{\pgfqpoint{2.043291in}{1.025393in}}%
\pgfpathlineto{\pgfqpoint{2.064471in}{1.030682in}}%
\pgfpathlineto{\pgfqpoint{2.077835in}{1.022698in}}%
\pgfpathlineto{\pgfqpoint{2.081888in}{1.030484in}}%
\pgfpathlineto{\pgfqpoint{2.092985in}{1.035923in}}%
\pgfpathlineto{\pgfqpoint{2.095848in}{1.031127in}}%
\pgfpathlineto{\pgfqpoint{2.107246in}{1.034937in}}%
\pgfpathlineto{\pgfqpoint{2.106527in}{1.041459in}}%
\pgfpathlineto{\pgfqpoint{2.116776in}{1.048924in}}%
\pgfpathlineto{\pgfqpoint{2.122824in}{1.041173in}}%
\pgfpathlineto{\pgfqpoint{2.130781in}{1.040393in}}%
\pgfpathlineto{\pgfqpoint{2.136068in}{1.045445in}}%
\pgfpathlineto{\pgfqpoint{2.135480in}{1.052628in}}%
\pgfpathlineto{\pgfqpoint{2.139975in}{1.059762in}}%
\pgfpathlineto{\pgfqpoint{2.145998in}{1.061307in}}%
\pgfpathlineto{\pgfqpoint{2.148416in}{1.070726in}}%
\pgfpathlineto{\pgfqpoint{2.157176in}{1.078868in}}%
\pgfpathlineto{\pgfqpoint{2.154539in}{1.086973in}}%
\pgfpathlineto{\pgfqpoint{2.163061in}{1.091009in}}%
\pgfpathlineto{\pgfqpoint{2.168755in}{1.088527in}}%
\pgfpathlineto{\pgfqpoint{2.177165in}{1.094831in}}%
\pgfpathlineto{\pgfqpoint{2.184680in}{1.096474in}}%
\pgfpathlineto{\pgfqpoint{2.184707in}{1.103013in}}%
\pgfpathlineto{\pgfqpoint{2.179440in}{1.112087in}}%
\pgfpathlineto{\pgfqpoint{2.182869in}{1.115743in}}%
\pgfpathlineto{\pgfqpoint{2.193640in}{1.114933in}}%
\pgfpathlineto{\pgfqpoint{2.200490in}{1.118345in}}%
\pgfpathlineto{\pgfqpoint{2.207555in}{1.113014in}}%
\pgfpathlineto{\pgfqpoint{2.213131in}{1.100983in}}%
\pgfpathlineto{\pgfqpoint{2.231619in}{1.098950in}}%
\pgfpathlineto{\pgfqpoint{2.241256in}{1.092327in}}%
\pgfpathlineto{\pgfqpoint{2.248036in}{1.097925in}}%
\pgfpathlineto{\pgfqpoint{2.260296in}{1.093220in}}%
\pgfpathlineto{\pgfqpoint{2.267236in}{1.095558in}}%
\pgfpathlineto{\pgfqpoint{2.272088in}{1.102206in}}%
\pgfpathlineto{\pgfqpoint{2.279493in}{1.103478in}}%
\pgfpathlineto{\pgfqpoint{2.281417in}{1.096250in}}%
\pgfpathlineto{\pgfqpoint{2.289608in}{1.093486in}}%
\pgfpathlineto{\pgfqpoint{2.295973in}{1.086357in}}%
\pgfpathlineto{\pgfqpoint{2.297892in}{1.076731in}}%
\pgfpathlineto{\pgfqpoint{2.295338in}{1.069453in}}%
\pgfpathlineto{\pgfqpoint{2.305111in}{1.060030in}}%
\pgfpathlineto{\pgfqpoint{2.316592in}{1.042238in}}%
\pgfpathlineto{\pgfqpoint{2.325090in}{1.036384in}}%
\pgfpathlineto{\pgfqpoint{2.334313in}{1.036045in}}%
\pgfpathlineto{\pgfqpoint{2.317283in}{1.016505in}}%
\pgfpathlineto{\pgfqpoint{2.300521in}{1.004677in}}%
\pgfpathlineto{\pgfqpoint{2.294351in}{0.995285in}}%
\pgfpathlineto{\pgfqpoint{2.294456in}{0.990187in}}%
\pgfpathlineto{\pgfqpoint{2.285376in}{0.986281in}}%
\pgfpathlineto{\pgfqpoint{2.282775in}{0.978930in}}%
\pgfpathlineto{\pgfqpoint{2.263860in}{0.971525in}}%
\pgfpathlineto{\pgfqpoint{2.257156in}{0.966716in}}%
\pgfpathlineto{\pgfqpoint{2.256248in}{0.965690in}}%
\pgfpathlineto{\pgfqpoint{2.201842in}{0.960531in}}%
\pgfpathlineto{\pgfqpoint{2.168986in}{0.957768in}}%
\pgfpathlineto{\pgfqpoint{2.115071in}{0.954717in}}%
\pgfpathlineto{\pgfqpoint{2.047902in}{0.948072in}}%
\pgfpathlineto{\pgfqpoint{2.036805in}{0.949609in}}%
\pgfpathlineto{\pgfqpoint{2.039114in}{0.938281in}}%
\pgfpathclose%
\pgfusepath{fill}%
\end{pgfscope}%
\begin{pgfscope}%
\pgfpathrectangle{\pgfqpoint{0.100000in}{0.100000in}}{\pgfqpoint{2.857344in}{1.829167in}}%
\pgfusepath{clip}%
\pgfsetbuttcap%
\pgfsetmiterjoin%
\definecolor{currentfill}{rgb}{0.978854,0.991542,0.716032}%
\pgfsetfillcolor{currentfill}%
\pgfsetlinewidth{0.000000pt}%
\definecolor{currentstroke}{rgb}{0.000000,0.000000,0.000000}%
\pgfsetstrokecolor{currentstroke}%
\pgfsetstrokeopacity{0.000000}%
\pgfsetdash{}{0pt}%
\pgfpathmoveto{\pgfqpoint{1.337379in}{0.969343in}}%
\pgfpathlineto{\pgfqpoint{1.343071in}{1.048329in}}%
\pgfpathlineto{\pgfqpoint{1.346726in}{1.097211in}}%
\pgfpathlineto{\pgfqpoint{1.351269in}{1.157246in}}%
\pgfpathlineto{\pgfqpoint{1.414241in}{1.152824in}}%
\pgfpathlineto{\pgfqpoint{1.494214in}{1.148456in}}%
\pgfpathlineto{\pgfqpoint{1.548607in}{1.146369in}}%
\pgfpathlineto{\pgfqpoint{1.602681in}{1.144712in}}%
\pgfpathlineto{\pgfqpoint{1.651623in}{1.143955in}}%
\pgfpathlineto{\pgfqpoint{1.674259in}{1.144189in}}%
\pgfpathlineto{\pgfqpoint{1.684222in}{1.136058in}}%
\pgfpathlineto{\pgfqpoint{1.692027in}{1.137698in}}%
\pgfpathlineto{\pgfqpoint{1.692271in}{1.130606in}}%
\pgfpathlineto{\pgfqpoint{1.686475in}{1.118338in}}%
\pgfpathlineto{\pgfqpoint{1.687104in}{1.110587in}}%
\pgfpathlineto{\pgfqpoint{1.692653in}{1.105476in}}%
\pgfpathlineto{\pgfqpoint{1.697750in}{1.094827in}}%
\pgfpathlineto{\pgfqpoint{1.708091in}{1.088713in}}%
\pgfpathlineto{\pgfqpoint{1.707745in}{1.048571in}}%
\pgfpathlineto{\pgfqpoint{1.708049in}{0.956264in}}%
\pgfpathlineto{\pgfqpoint{1.638741in}{0.956583in}}%
\pgfpathlineto{\pgfqpoint{1.565758in}{0.957860in}}%
\pgfpathlineto{\pgfqpoint{1.512030in}{0.959756in}}%
\pgfpathlineto{\pgfqpoint{1.467183in}{0.961486in}}%
\pgfpathlineto{\pgfqpoint{1.391629in}{0.965941in}}%
\pgfpathclose%
\pgfusepath{fill}%
\end{pgfscope}%
\begin{pgfscope}%
\pgfpathrectangle{\pgfqpoint{0.100000in}{0.100000in}}{\pgfqpoint{2.857344in}{1.829167in}}%
\pgfusepath{clip}%
\pgfsetbuttcap%
\pgfsetmiterjoin%
\definecolor{currentfill}{rgb}{0.978854,0.991542,0.716032}%
\pgfsetfillcolor{currentfill}%
\pgfsetlinewidth{0.000000pt}%
\definecolor{currentstroke}{rgb}{0.000000,0.000000,0.000000}%
\pgfsetstrokecolor{currentstroke}%
\pgfsetstrokeopacity{0.000000}%
\pgfsetdash{}{0pt}%
\pgfpathmoveto{\pgfqpoint{2.357165in}{0.979801in}}%
\pgfpathlineto{\pgfqpoint{2.280402in}{0.969018in}}%
\pgfpathlineto{\pgfqpoint{2.257156in}{0.966716in}}%
\pgfpathlineto{\pgfqpoint{2.263860in}{0.971525in}}%
\pgfpathlineto{\pgfqpoint{2.282775in}{0.978930in}}%
\pgfpathlineto{\pgfqpoint{2.285376in}{0.986281in}}%
\pgfpathlineto{\pgfqpoint{2.294456in}{0.990187in}}%
\pgfpathlineto{\pgfqpoint{2.294351in}{0.995285in}}%
\pgfpathlineto{\pgfqpoint{2.300521in}{1.004677in}}%
\pgfpathlineto{\pgfqpoint{2.317283in}{1.016505in}}%
\pgfpathlineto{\pgfqpoint{2.334313in}{1.036045in}}%
\pgfpathlineto{\pgfqpoint{2.333483in}{1.031840in}}%
\pgfpathlineto{\pgfqpoint{2.342463in}{1.021685in}}%
\pgfpathlineto{\pgfqpoint{2.357104in}{1.018833in}}%
\pgfpathlineto{\pgfqpoint{2.365622in}{1.028508in}}%
\pgfpathlineto{\pgfqpoint{2.373242in}{1.023413in}}%
\pgfpathlineto{\pgfqpoint{2.386878in}{1.029393in}}%
\pgfpathlineto{\pgfqpoint{2.388828in}{1.035233in}}%
\pgfpathlineto{\pgfqpoint{2.397291in}{1.036546in}}%
\pgfpathlineto{\pgfqpoint{2.406080in}{1.043446in}}%
\pgfpathlineto{\pgfqpoint{2.413828in}{1.045722in}}%
\pgfpathlineto{\pgfqpoint{2.414074in}{1.057909in}}%
\pgfpathlineto{\pgfqpoint{2.419091in}{1.071074in}}%
\pgfpathlineto{\pgfqpoint{2.425708in}{1.079318in}}%
\pgfpathlineto{\pgfqpoint{2.427853in}{1.091696in}}%
\pgfpathlineto{\pgfqpoint{2.436478in}{1.109084in}}%
\pgfpathlineto{\pgfqpoint{2.436786in}{1.119429in}}%
\pgfpathlineto{\pgfqpoint{2.445695in}{1.112688in}}%
\pgfpathlineto{\pgfqpoint{2.454405in}{1.111053in}}%
\pgfpathlineto{\pgfqpoint{2.457795in}{1.115957in}}%
\pgfpathlineto{\pgfqpoint{2.463850in}{1.137329in}}%
\pgfpathlineto{\pgfqpoint{2.472282in}{1.136510in}}%
\pgfpathlineto{\pgfqpoint{2.474515in}{1.144340in}}%
\pgfpathlineto{\pgfqpoint{2.479461in}{1.147872in}}%
\pgfpathlineto{\pgfqpoint{2.489941in}{1.165629in}}%
\pgfpathlineto{\pgfqpoint{2.488419in}{1.170881in}}%
\pgfpathlineto{\pgfqpoint{2.493031in}{1.182298in}}%
\pgfpathlineto{\pgfqpoint{2.517654in}{1.168391in}}%
\pgfpathlineto{\pgfqpoint{2.520380in}{1.180743in}}%
\pgfpathlineto{\pgfqpoint{2.528160in}{1.181016in}}%
\pgfpathlineto{\pgfqpoint{2.534202in}{1.177108in}}%
\pgfpathlineto{\pgfqpoint{2.532993in}{1.170369in}}%
\pgfpathlineto{\pgfqpoint{2.546324in}{1.167304in}}%
\pgfpathlineto{\pgfqpoint{2.553852in}{1.162705in}}%
\pgfpathlineto{\pgfqpoint{2.559306in}{1.154607in}}%
\pgfpathlineto{\pgfqpoint{2.556973in}{1.145207in}}%
\pgfpathlineto{\pgfqpoint{2.552262in}{1.144436in}}%
\pgfpathlineto{\pgfqpoint{2.549529in}{1.130251in}}%
\pgfpathlineto{\pgfqpoint{2.555515in}{1.124705in}}%
\pgfpathlineto{\pgfqpoint{2.563940in}{1.129188in}}%
\pgfpathlineto{\pgfqpoint{2.571757in}{1.119721in}}%
\pgfpathlineto{\pgfqpoint{2.589223in}{1.118019in}}%
\pgfpathlineto{\pgfqpoint{2.592232in}{1.112574in}}%
\pgfpathlineto{\pgfqpoint{2.608396in}{1.107277in}}%
\pgfpathlineto{\pgfqpoint{2.606401in}{1.101027in}}%
\pgfpathlineto{\pgfqpoint{2.608504in}{1.083364in}}%
\pgfpathlineto{\pgfqpoint{2.615028in}{1.076694in}}%
\pgfpathlineto{\pgfqpoint{2.605401in}{1.073879in}}%
\pgfpathlineto{\pgfqpoint{2.606678in}{1.066338in}}%
\pgfpathlineto{\pgfqpoint{2.616947in}{1.059945in}}%
\pgfpathlineto{\pgfqpoint{2.617873in}{1.053638in}}%
\pgfpathlineto{\pgfqpoint{2.612075in}{1.048899in}}%
\pgfpathlineto{\pgfqpoint{2.600376in}{1.060127in}}%
\pgfpathlineto{\pgfqpoint{2.599219in}{1.051935in}}%
\pgfpathlineto{\pgfqpoint{2.609013in}{1.048039in}}%
\pgfpathlineto{\pgfqpoint{2.609993in}{1.044011in}}%
\pgfpathlineto{\pgfqpoint{2.623426in}{1.049335in}}%
\pgfpathlineto{\pgfqpoint{2.633720in}{1.050743in}}%
\pgfpathlineto{\pgfqpoint{2.644264in}{1.029468in}}%
\pgfpathlineto{\pgfqpoint{2.643088in}{1.029238in}}%
\pgfpathlineto{\pgfqpoint{2.638323in}{1.028253in}}%
\pgfpathlineto{\pgfqpoint{2.636919in}{1.027960in}}%
\pgfpathlineto{\pgfqpoint{2.635991in}{1.027778in}}%
\pgfpathlineto{\pgfqpoint{2.573209in}{1.014755in}}%
\pgfpathlineto{\pgfqpoint{2.517054in}{1.003262in}}%
\pgfpathlineto{\pgfqpoint{2.439415in}{0.989887in}}%
\pgfpathlineto{\pgfqpoint{2.373476in}{0.981178in}}%
\pgfpathclose%
\pgfusepath{fill}%
\end{pgfscope}%
\begin{pgfscope}%
\pgfpathrectangle{\pgfqpoint{0.100000in}{0.100000in}}{\pgfqpoint{2.857344in}{1.829167in}}%
\pgfusepath{clip}%
\pgfsetbuttcap%
\pgfsetmiterjoin%
\definecolor{currentfill}{rgb}{0.978854,0.991542,0.716032}%
\pgfsetfillcolor{currentfill}%
\pgfsetlinewidth{0.000000pt}%
\definecolor{currentstroke}{rgb}{0.000000,0.000000,0.000000}%
\pgfsetstrokecolor{currentstroke}%
\pgfsetstrokeopacity{0.000000}%
\pgfsetdash{}{0pt}%
\pgfpathmoveto{\pgfqpoint{2.636861in}{1.118458in}}%
\pgfpathlineto{\pgfqpoint{2.649518in}{1.124199in}}%
\pgfpathlineto{\pgfqpoint{2.639411in}{1.094632in}}%
\pgfpathlineto{\pgfqpoint{2.632726in}{1.078975in}}%
\pgfpathlineto{\pgfqpoint{2.627911in}{1.087836in}}%
\pgfpathlineto{\pgfqpoint{2.636474in}{1.109049in}}%
\pgfpathclose%
\pgfusepath{fill}%
\end{pgfscope}%
\begin{pgfscope}%
\pgfpathrectangle{\pgfqpoint{0.100000in}{0.100000in}}{\pgfqpoint{2.857344in}{1.829167in}}%
\pgfusepath{clip}%
\pgfsetbuttcap%
\pgfsetmiterjoin%
\definecolor{currentfill}{rgb}{0.990388,0.996155,0.734025}%
\pgfsetfillcolor{currentfill}%
\pgfsetlinewidth{0.000000pt}%
\definecolor{currentstroke}{rgb}{0.000000,0.000000,0.000000}%
\pgfsetstrokecolor{currentstroke}%
\pgfsetstrokeopacity{0.000000}%
\pgfsetdash{}{0pt}%
\pgfpathmoveto{\pgfqpoint{1.970155in}{0.933898in}}%
\pgfpathlineto{\pgfqpoint{1.967099in}{0.933455in}}%
\pgfpathlineto{\pgfqpoint{1.964216in}{0.933253in}}%
\pgfpathlineto{\pgfqpoint{1.965450in}{0.924404in}}%
\pgfpathlineto{\pgfqpoint{1.958014in}{0.915483in}}%
\pgfpathlineto{\pgfqpoint{1.956548in}{0.901525in}}%
\pgfpathlineto{\pgfqpoint{1.923305in}{0.899066in}}%
\pgfpathlineto{\pgfqpoint{1.926203in}{0.905620in}}%
\pgfpathlineto{\pgfqpoint{1.938189in}{0.917589in}}%
\pgfpathlineto{\pgfqpoint{1.938520in}{0.924528in}}%
\pgfpathlineto{\pgfqpoint{1.933214in}{0.931097in}}%
\pgfpathlineto{\pgfqpoint{1.883735in}{0.928480in}}%
\pgfpathlineto{\pgfqpoint{1.797229in}{0.925734in}}%
\pgfpathlineto{\pgfqpoint{1.746605in}{0.924811in}}%
\pgfpathlineto{\pgfqpoint{1.708339in}{0.924467in}}%
\pgfpathlineto{\pgfqpoint{1.708049in}{0.956264in}}%
\pgfpathlineto{\pgfqpoint{1.707745in}{1.048571in}}%
\pgfpathlineto{\pgfqpoint{1.708091in}{1.088713in}}%
\pgfpathlineto{\pgfqpoint{1.697750in}{1.094827in}}%
\pgfpathlineto{\pgfqpoint{1.692653in}{1.105476in}}%
\pgfpathlineto{\pgfqpoint{1.687104in}{1.110587in}}%
\pgfpathlineto{\pgfqpoint{1.686475in}{1.118338in}}%
\pgfpathlineto{\pgfqpoint{1.692271in}{1.130606in}}%
\pgfpathlineto{\pgfqpoint{1.692027in}{1.137698in}}%
\pgfpathlineto{\pgfqpoint{1.684222in}{1.136058in}}%
\pgfpathlineto{\pgfqpoint{1.674259in}{1.144189in}}%
\pgfpathlineto{\pgfqpoint{1.666272in}{1.158462in}}%
\pgfpathlineto{\pgfqpoint{1.659572in}{1.165048in}}%
\pgfpathlineto{\pgfqpoint{1.652572in}{1.181231in}}%
\pgfpathlineto{\pgfqpoint{1.725281in}{1.180096in}}%
\pgfpathlineto{\pgfqpoint{1.797555in}{1.182478in}}%
\pgfpathlineto{\pgfqpoint{1.843896in}{1.185151in}}%
\pgfpathlineto{\pgfqpoint{1.858393in}{1.170794in}}%
\pgfpathlineto{\pgfqpoint{1.855744in}{1.155821in}}%
\pgfpathlineto{\pgfqpoint{1.859609in}{1.138964in}}%
\pgfpathlineto{\pgfqpoint{1.863897in}{1.130464in}}%
\pgfpathlineto{\pgfqpoint{1.877578in}{1.118771in}}%
\pgfpathlineto{\pgfqpoint{1.880571in}{1.113606in}}%
\pgfpathlineto{\pgfqpoint{1.895549in}{1.101967in}}%
\pgfpathlineto{\pgfqpoint{1.897720in}{1.088911in}}%
\pgfpathlineto{\pgfqpoint{1.902022in}{1.079279in}}%
\pgfpathlineto{\pgfqpoint{1.912185in}{1.084889in}}%
\pgfpathlineto{\pgfqpoint{1.920562in}{1.082323in}}%
\pgfpathlineto{\pgfqpoint{1.926888in}{1.075514in}}%
\pgfpathlineto{\pgfqpoint{1.924500in}{1.063463in}}%
\pgfpathlineto{\pgfqpoint{1.916736in}{1.047707in}}%
\pgfpathlineto{\pgfqpoint{1.916856in}{1.039501in}}%
\pgfpathlineto{\pgfqpoint{1.922658in}{1.032772in}}%
\pgfpathlineto{\pgfqpoint{1.935145in}{1.023860in}}%
\pgfpathlineto{\pgfqpoint{1.949523in}{1.016626in}}%
\pgfpathlineto{\pgfqpoint{1.952933in}{1.010834in}}%
\pgfpathlineto{\pgfqpoint{1.960287in}{1.008095in}}%
\pgfpathlineto{\pgfqpoint{1.960579in}{1.000408in}}%
\pgfpathlineto{\pgfqpoint{1.966021in}{0.990644in}}%
\pgfpathlineto{\pgfqpoint{1.962245in}{0.981969in}}%
\pgfpathlineto{\pgfqpoint{1.969637in}{0.968168in}}%
\pgfpathlineto{\pgfqpoint{1.975418in}{0.970908in}}%
\pgfpathlineto{\pgfqpoint{1.982509in}{0.965148in}}%
\pgfpathlineto{\pgfqpoint{1.980125in}{0.939499in}}%
\pgfpathlineto{\pgfqpoint{1.971767in}{0.941126in}}%
\pgfpathclose%
\pgfusepath{fill}%
\end{pgfscope}%
\begin{pgfscope}%
\pgfpathrectangle{\pgfqpoint{0.100000in}{0.100000in}}{\pgfqpoint{2.857344in}{1.829167in}}%
\pgfusepath{clip}%
\pgfsetbuttcap%
\pgfsetmiterjoin%
\definecolor{currentfill}{rgb}{0.913495,0.965398,0.614072}%
\pgfsetfillcolor{currentfill}%
\pgfsetlinewidth{0.000000pt}%
\definecolor{currentstroke}{rgb}{0.000000,0.000000,0.000000}%
\pgfsetstrokecolor{currentstroke}%
\pgfsetstrokeopacity{0.000000}%
\pgfsetdash{}{0pt}%
\pgfpathmoveto{\pgfqpoint{0.688408in}{0.939428in}}%
\pgfpathlineto{\pgfqpoint{0.694459in}{0.952313in}}%
\pgfpathlineto{\pgfqpoint{0.692803in}{0.971654in}}%
\pgfpathlineto{\pgfqpoint{0.695701in}{0.977154in}}%
\pgfpathlineto{\pgfqpoint{0.695988in}{1.001245in}}%
\pgfpathlineto{\pgfqpoint{0.698734in}{1.008183in}}%
\pgfpathlineto{\pgfqpoint{0.708412in}{1.009262in}}%
\pgfpathlineto{\pgfqpoint{0.717393in}{1.006182in}}%
\pgfpathlineto{\pgfqpoint{0.721304in}{0.997701in}}%
\pgfpathlineto{\pgfqpoint{0.726788in}{0.998025in}}%
\pgfpathlineto{\pgfqpoint{0.733601in}{1.007741in}}%
\pgfpathlineto{\pgfqpoint{0.743449in}{1.055636in}}%
\pgfpathlineto{\pgfqpoint{0.799487in}{1.044067in}}%
\pgfpathlineto{\pgfqpoint{0.832010in}{1.037653in}}%
\pgfpathlineto{\pgfqpoint{0.918375in}{1.022471in}}%
\pgfpathlineto{\pgfqpoint{0.942269in}{1.017673in}}%
\pgfpathlineto{\pgfqpoint{0.989159in}{1.010405in}}%
\pgfpathlineto{\pgfqpoint{0.979545in}{0.948497in}}%
\pgfpathlineto{\pgfqpoint{0.969545in}{0.883922in}}%
\pgfpathlineto{\pgfqpoint{0.958029in}{0.811272in}}%
\pgfpathlineto{\pgfqpoint{0.945078in}{0.727879in}}%
\pgfpathlineto{\pgfqpoint{0.934616in}{0.659386in}}%
\pgfpathlineto{\pgfqpoint{0.859677in}{0.671286in}}%
\pgfpathlineto{\pgfqpoint{0.826750in}{0.676925in}}%
\pgfpathlineto{\pgfqpoint{0.812042in}{0.685731in}}%
\pgfpathlineto{\pgfqpoint{0.716140in}{0.743594in}}%
\pgfpathlineto{\pgfqpoint{0.644170in}{0.787505in}}%
\pgfpathlineto{\pgfqpoint{0.646567in}{0.795288in}}%
\pgfpathlineto{\pgfqpoint{0.652512in}{0.800733in}}%
\pgfpathlineto{\pgfqpoint{0.658695in}{0.799724in}}%
\pgfpathlineto{\pgfqpoint{0.667664in}{0.805443in}}%
\pgfpathlineto{\pgfqpoint{0.669076in}{0.813662in}}%
\pgfpathlineto{\pgfqpoint{0.658138in}{0.823633in}}%
\pgfpathlineto{\pgfqpoint{0.665917in}{0.842772in}}%
\pgfpathlineto{\pgfqpoint{0.673739in}{0.850133in}}%
\pgfpathlineto{\pgfqpoint{0.677451in}{0.858865in}}%
\pgfpathlineto{\pgfqpoint{0.679738in}{0.874878in}}%
\pgfpathlineto{\pgfqpoint{0.687071in}{0.882129in}}%
\pgfpathlineto{\pgfqpoint{0.702464in}{0.889371in}}%
\pgfpathlineto{\pgfqpoint{0.702124in}{0.895753in}}%
\pgfpathlineto{\pgfqpoint{0.693527in}{0.903672in}}%
\pgfpathlineto{\pgfqpoint{0.692357in}{0.919999in}}%
\pgfpathlineto{\pgfqpoint{0.686437in}{0.931934in}}%
\pgfpathclose%
\pgfusepath{fill}%
\end{pgfscope}%
\begin{pgfscope}%
\pgfpathrectangle{\pgfqpoint{0.100000in}{0.100000in}}{\pgfqpoint{2.857344in}{1.829167in}}%
\pgfusepath{clip}%
\pgfsetbuttcap%
\pgfsetmiterjoin%
\definecolor{currentfill}{rgb}{0.994387,0.793849,0.474048}%
\pgfsetfillcolor{currentfill}%
\pgfsetlinewidth{0.000000pt}%
\definecolor{currentstroke}{rgb}{0.000000,0.000000,0.000000}%
\pgfsetstrokecolor{currentstroke}%
\pgfsetstrokeopacity{0.000000}%
\pgfsetdash{}{0pt}%
\pgfpathmoveto{\pgfqpoint{1.716452in}{0.745221in}}%
\pgfpathlineto{\pgfqpoint{1.702560in}{0.749517in}}%
\pgfpathlineto{\pgfqpoint{1.684760in}{0.763149in}}%
\pgfpathlineto{\pgfqpoint{1.676853in}{0.766079in}}%
\pgfpathlineto{\pgfqpoint{1.671829in}{0.760192in}}%
\pgfpathlineto{\pgfqpoint{1.665484in}{0.759894in}}%
\pgfpathlineto{\pgfqpoint{1.657457in}{0.764893in}}%
\pgfpathlineto{\pgfqpoint{1.644861in}{0.758488in}}%
\pgfpathlineto{\pgfqpoint{1.640377in}{0.761642in}}%
\pgfpathlineto{\pgfqpoint{1.627982in}{0.754170in}}%
\pgfpathlineto{\pgfqpoint{1.614906in}{0.755549in}}%
\pgfpathlineto{\pgfqpoint{1.604821in}{0.760404in}}%
\pgfpathlineto{\pgfqpoint{1.597757in}{0.758577in}}%
\pgfpathlineto{\pgfqpoint{1.586451in}{0.765444in}}%
\pgfpathlineto{\pgfqpoint{1.580162in}{0.753333in}}%
\pgfpathlineto{\pgfqpoint{1.573730in}{0.763748in}}%
\pgfpathlineto{\pgfqpoint{1.561027in}{0.759707in}}%
\pgfpathlineto{\pgfqpoint{1.549915in}{0.769601in}}%
\pgfpathlineto{\pgfqpoint{1.540206in}{0.761626in}}%
\pgfpathlineto{\pgfqpoint{1.535343in}{0.768954in}}%
\pgfpathlineto{\pgfqpoint{1.528334in}{0.771334in}}%
\pgfpathlineto{\pgfqpoint{1.524066in}{0.778399in}}%
\pgfpathlineto{\pgfqpoint{1.514899in}{0.780410in}}%
\pgfpathlineto{\pgfqpoint{1.509609in}{0.775092in}}%
\pgfpathlineto{\pgfqpoint{1.500622in}{0.781984in}}%
\pgfpathlineto{\pgfqpoint{1.496433in}{0.780410in}}%
\pgfpathlineto{\pgfqpoint{1.481537in}{0.785994in}}%
\pgfpathlineto{\pgfqpoint{1.472208in}{0.786617in}}%
\pgfpathlineto{\pgfqpoint{1.468032in}{0.798477in}}%
\pgfpathlineto{\pgfqpoint{1.460554in}{0.797000in}}%
\pgfpathlineto{\pgfqpoint{1.451988in}{0.799926in}}%
\pgfpathlineto{\pgfqpoint{1.446347in}{0.798235in}}%
\pgfpathlineto{\pgfqpoint{1.434250in}{0.811555in}}%
\pgfpathlineto{\pgfqpoint{1.430879in}{0.810683in}}%
\pgfpathlineto{\pgfqpoint{1.433941in}{0.864698in}}%
\pgfpathlineto{\pgfqpoint{1.437276in}{0.931553in}}%
\pgfpathlineto{\pgfqpoint{1.382532in}{0.934616in}}%
\pgfpathlineto{\pgfqpoint{1.328511in}{0.938700in}}%
\pgfpathlineto{\pgfqpoint{1.286775in}{0.942324in}}%
\pgfpathlineto{\pgfqpoint{1.289547in}{0.973908in}}%
\pgfpathlineto{\pgfqpoint{1.337379in}{0.969343in}}%
\pgfpathlineto{\pgfqpoint{1.391629in}{0.965941in}}%
\pgfpathlineto{\pgfqpoint{1.467183in}{0.961486in}}%
\pgfpathlineto{\pgfqpoint{1.512030in}{0.959756in}}%
\pgfpathlineto{\pgfqpoint{1.565758in}{0.957860in}}%
\pgfpathlineto{\pgfqpoint{1.638741in}{0.956583in}}%
\pgfpathlineto{\pgfqpoint{1.708049in}{0.956264in}}%
\pgfpathlineto{\pgfqpoint{1.708339in}{0.924467in}}%
\pgfpathlineto{\pgfqpoint{1.712230in}{0.900512in}}%
\pgfpathlineto{\pgfqpoint{1.718273in}{0.856267in}}%
\pgfpathlineto{\pgfqpoint{1.717027in}{0.780708in}}%
\pgfpathclose%
\pgfusepath{fill}%
\end{pgfscope}%
\begin{pgfscope}%
\pgfpathrectangle{\pgfqpoint{0.100000in}{0.100000in}}{\pgfqpoint{2.857344in}{1.829167in}}%
\pgfusepath{clip}%
\pgfsetbuttcap%
\pgfsetmiterjoin%
\definecolor{currentfill}{rgb}{0.959631,0.983852,0.686044}%
\pgfsetfillcolor{currentfill}%
\pgfsetlinewidth{0.000000pt}%
\definecolor{currentstroke}{rgb}{0.000000,0.000000,0.000000}%
\pgfsetstrokecolor{currentstroke}%
\pgfsetstrokeopacity{0.000000}%
\pgfsetdash{}{0pt}%
\pgfpathmoveto{\pgfqpoint{2.236602in}{0.862564in}}%
\pgfpathlineto{\pgfqpoint{2.236643in}{0.876564in}}%
\pgfpathlineto{\pgfqpoint{2.248808in}{0.881940in}}%
\pgfpathlineto{\pgfqpoint{2.249319in}{0.890534in}}%
\pgfpathlineto{\pgfqpoint{2.260212in}{0.901033in}}%
\pgfpathlineto{\pgfqpoint{2.273828in}{0.903183in}}%
\pgfpathlineto{\pgfqpoint{2.285798in}{0.914753in}}%
\pgfpathlineto{\pgfqpoint{2.298299in}{0.919924in}}%
\pgfpathlineto{\pgfqpoint{2.307682in}{0.935449in}}%
\pgfpathlineto{\pgfqpoint{2.316227in}{0.934322in}}%
\pgfpathlineto{\pgfqpoint{2.334286in}{0.948467in}}%
\pgfpathlineto{\pgfqpoint{2.343823in}{0.948734in}}%
\pgfpathlineto{\pgfqpoint{2.347851in}{0.959512in}}%
\pgfpathlineto{\pgfqpoint{2.355430in}{0.967016in}}%
\pgfpathlineto{\pgfqpoint{2.357165in}{0.979801in}}%
\pgfpathlineto{\pgfqpoint{2.373476in}{0.981178in}}%
\pgfpathlineto{\pgfqpoint{2.439415in}{0.989887in}}%
\pgfpathlineto{\pgfqpoint{2.517054in}{1.003262in}}%
\pgfpathlineto{\pgfqpoint{2.573209in}{1.014755in}}%
\pgfpathlineto{\pgfqpoint{2.635991in}{1.027778in}}%
\pgfpathlineto{\pgfqpoint{2.653931in}{1.003148in}}%
\pgfpathlineto{\pgfqpoint{2.644213in}{1.004720in}}%
\pgfpathlineto{\pgfqpoint{2.631739in}{1.001681in}}%
\pgfpathlineto{\pgfqpoint{2.619448in}{0.989129in}}%
\pgfpathlineto{\pgfqpoint{2.610614in}{0.990031in}}%
\pgfpathlineto{\pgfqpoint{2.609496in}{0.982576in}}%
\pgfpathlineto{\pgfqpoint{2.625463in}{0.988472in}}%
\pgfpathlineto{\pgfqpoint{2.641531in}{0.990903in}}%
\pgfpathlineto{\pgfqpoint{2.647497in}{0.987668in}}%
\pgfpathlineto{\pgfqpoint{2.655526in}{0.991356in}}%
\pgfpathlineto{\pgfqpoint{2.659669in}{0.988772in}}%
\pgfpathlineto{\pgfqpoint{2.663336in}{0.976469in}}%
\pgfpathlineto{\pgfqpoint{2.655704in}{0.972659in}}%
\pgfpathlineto{\pgfqpoint{2.650467in}{0.957656in}}%
\pgfpathlineto{\pgfqpoint{2.644975in}{0.951815in}}%
\pgfpathlineto{\pgfqpoint{2.628141in}{0.953093in}}%
\pgfpathlineto{\pgfqpoint{2.629046in}{0.961902in}}%
\pgfpathlineto{\pgfqpoint{2.619852in}{0.958054in}}%
\pgfpathlineto{\pgfqpoint{2.617852in}{0.950707in}}%
\pgfpathlineto{\pgfqpoint{2.624154in}{0.943097in}}%
\pgfpathlineto{\pgfqpoint{2.626295in}{0.930180in}}%
\pgfpathlineto{\pgfqpoint{2.615983in}{0.922820in}}%
\pgfpathlineto{\pgfqpoint{2.627135in}{0.920231in}}%
\pgfpathlineto{\pgfqpoint{2.632182in}{0.925643in}}%
\pgfpathlineto{\pgfqpoint{2.643341in}{0.926302in}}%
\pgfpathlineto{\pgfqpoint{2.637606in}{0.914616in}}%
\pgfpathlineto{\pgfqpoint{2.630583in}{0.908985in}}%
\pgfpathlineto{\pgfqpoint{2.609338in}{0.903347in}}%
\pgfpathlineto{\pgfqpoint{2.587525in}{0.883564in}}%
\pgfpathlineto{\pgfqpoint{2.574075in}{0.864071in}}%
\pgfpathlineto{\pgfqpoint{2.573996in}{0.856155in}}%
\pgfpathlineto{\pgfqpoint{2.568623in}{0.845231in}}%
\pgfpathlineto{\pgfqpoint{2.541050in}{0.838044in}}%
\pgfpathlineto{\pgfqpoint{2.475476in}{0.885050in}}%
\pgfpathlineto{\pgfqpoint{2.417728in}{0.876565in}}%
\pgfpathlineto{\pgfqpoint{2.417244in}{0.884400in}}%
\pgfpathlineto{\pgfqpoint{2.408467in}{0.893223in}}%
\pgfpathlineto{\pgfqpoint{2.401822in}{0.895382in}}%
\pgfpathlineto{\pgfqpoint{2.339078in}{0.888863in}}%
\pgfpathlineto{\pgfqpoint{2.324657in}{0.883972in}}%
\pgfpathlineto{\pgfqpoint{2.298617in}{0.871018in}}%
\pgfpathlineto{\pgfqpoint{2.276114in}{0.867433in}}%
\pgfpathclose%
\pgfusepath{fill}%
\end{pgfscope}%
\begin{pgfscope}%
\pgfpathrectangle{\pgfqpoint{0.100000in}{0.100000in}}{\pgfqpoint{2.857344in}{1.829167in}}%
\pgfusepath{clip}%
\pgfsetbuttcap%
\pgfsetmiterjoin%
\definecolor{currentfill}{rgb}{0.998539,0.954710,0.673049}%
\pgfsetfillcolor{currentfill}%
\pgfsetlinewidth{0.000000pt}%
\definecolor{currentstroke}{rgb}{0.000000,0.000000,0.000000}%
\pgfsetstrokecolor{currentstroke}%
\pgfsetstrokeopacity{0.000000}%
\pgfsetdash{}{0pt}%
\pgfpathmoveto{\pgfqpoint{2.236602in}{0.862564in}}%
\pgfpathlineto{\pgfqpoint{2.170934in}{0.855352in}}%
\pgfpathlineto{\pgfqpoint{2.110860in}{0.849950in}}%
\pgfpathlineto{\pgfqpoint{2.049117in}{0.845957in}}%
\pgfpathlineto{\pgfqpoint{2.038519in}{0.844421in}}%
\pgfpathlineto{\pgfqpoint{1.996898in}{0.841219in}}%
\pgfpathlineto{\pgfqpoint{1.930235in}{0.837329in}}%
\pgfpathlineto{\pgfqpoint{1.942095in}{0.847174in}}%
\pgfpathlineto{\pgfqpoint{1.939389in}{0.857537in}}%
\pgfpathlineto{\pgfqpoint{1.942011in}{0.870099in}}%
\pgfpathlineto{\pgfqpoint{1.946016in}{0.876013in}}%
\pgfpathlineto{\pgfqpoint{1.945865in}{0.884241in}}%
\pgfpathlineto{\pgfqpoint{1.956530in}{0.889416in}}%
\pgfpathlineto{\pgfqpoint{1.956548in}{0.901525in}}%
\pgfpathlineto{\pgfqpoint{1.958014in}{0.915483in}}%
\pgfpathlineto{\pgfqpoint{1.965450in}{0.924404in}}%
\pgfpathlineto{\pgfqpoint{1.964216in}{0.933253in}}%
\pgfpathlineto{\pgfqpoint{1.967099in}{0.933455in}}%
\pgfpathlineto{\pgfqpoint{1.970155in}{0.933898in}}%
\pgfpathlineto{\pgfqpoint{2.039114in}{0.938281in}}%
\pgfpathlineto{\pgfqpoint{2.036805in}{0.949609in}}%
\pgfpathlineto{\pgfqpoint{2.047902in}{0.948072in}}%
\pgfpathlineto{\pgfqpoint{2.115071in}{0.954717in}}%
\pgfpathlineto{\pgfqpoint{2.168986in}{0.957768in}}%
\pgfpathlineto{\pgfqpoint{2.201842in}{0.960531in}}%
\pgfpathlineto{\pgfqpoint{2.256248in}{0.965690in}}%
\pgfpathlineto{\pgfqpoint{2.257156in}{0.966716in}}%
\pgfpathlineto{\pgfqpoint{2.280402in}{0.969018in}}%
\pgfpathlineto{\pgfqpoint{2.357165in}{0.979801in}}%
\pgfpathlineto{\pgfqpoint{2.355430in}{0.967016in}}%
\pgfpathlineto{\pgfqpoint{2.347851in}{0.959512in}}%
\pgfpathlineto{\pgfqpoint{2.343823in}{0.948734in}}%
\pgfpathlineto{\pgfqpoint{2.334286in}{0.948467in}}%
\pgfpathlineto{\pgfqpoint{2.316227in}{0.934322in}}%
\pgfpathlineto{\pgfqpoint{2.307682in}{0.935449in}}%
\pgfpathlineto{\pgfqpoint{2.298299in}{0.919924in}}%
\pgfpathlineto{\pgfqpoint{2.285798in}{0.914753in}}%
\pgfpathlineto{\pgfqpoint{2.273828in}{0.903183in}}%
\pgfpathlineto{\pgfqpoint{2.260212in}{0.901033in}}%
\pgfpathlineto{\pgfqpoint{2.249319in}{0.890534in}}%
\pgfpathlineto{\pgfqpoint{2.248808in}{0.881940in}}%
\pgfpathlineto{\pgfqpoint{2.236643in}{0.876564in}}%
\pgfpathclose%
\pgfusepath{fill}%
\end{pgfscope}%
\begin{pgfscope}%
\pgfpathrectangle{\pgfqpoint{0.100000in}{0.100000in}}{\pgfqpoint{2.857344in}{1.829167in}}%
\pgfusepath{clip}%
\pgfsetbuttcap%
\pgfsetmiterjoin%
\definecolor{currentfill}{rgb}{0.994233,0.997693,0.740023}%
\pgfsetfillcolor{currentfill}%
\pgfsetlinewidth{0.000000pt}%
\definecolor{currentstroke}{rgb}{0.000000,0.000000,0.000000}%
\pgfsetstrokecolor{currentstroke}%
\pgfsetstrokeopacity{0.000000}%
\pgfsetdash{}{0pt}%
\pgfpathmoveto{\pgfqpoint{1.071964in}{0.668149in}}%
\pgfpathlineto{\pgfqpoint{1.067281in}{0.675675in}}%
\pgfpathlineto{\pgfqpoint{1.069219in}{0.682157in}}%
\pgfpathlineto{\pgfqpoint{1.102139in}{0.678074in}}%
\pgfpathlineto{\pgfqpoint{1.163353in}{0.671121in}}%
\pgfpathlineto{\pgfqpoint{1.207614in}{0.666772in}}%
\pgfpathlineto{\pgfqpoint{1.258733in}{0.661614in}}%
\pgfpathlineto{\pgfqpoint{1.261514in}{0.693857in}}%
\pgfpathlineto{\pgfqpoint{1.270210in}{0.775867in}}%
\pgfpathlineto{\pgfqpoint{1.275850in}{0.833319in}}%
\pgfpathlineto{\pgfqpoint{1.280726in}{0.888200in}}%
\pgfpathlineto{\pgfqpoint{1.285272in}{0.942408in}}%
\pgfpathlineto{\pgfqpoint{1.286775in}{0.942324in}}%
\pgfpathlineto{\pgfqpoint{1.328511in}{0.938700in}}%
\pgfpathlineto{\pgfqpoint{1.382532in}{0.934616in}}%
\pgfpathlineto{\pgfqpoint{1.437276in}{0.931553in}}%
\pgfpathlineto{\pgfqpoint{1.433941in}{0.864698in}}%
\pgfpathlineto{\pgfqpoint{1.430879in}{0.810683in}}%
\pgfpathlineto{\pgfqpoint{1.434250in}{0.811555in}}%
\pgfpathlineto{\pgfqpoint{1.446347in}{0.798235in}}%
\pgfpathlineto{\pgfqpoint{1.451988in}{0.799926in}}%
\pgfpathlineto{\pgfqpoint{1.460554in}{0.797000in}}%
\pgfpathlineto{\pgfqpoint{1.468032in}{0.798477in}}%
\pgfpathlineto{\pgfqpoint{1.472208in}{0.786617in}}%
\pgfpathlineto{\pgfqpoint{1.481537in}{0.785994in}}%
\pgfpathlineto{\pgfqpoint{1.496433in}{0.780410in}}%
\pgfpathlineto{\pgfqpoint{1.500622in}{0.781984in}}%
\pgfpathlineto{\pgfqpoint{1.509609in}{0.775092in}}%
\pgfpathlineto{\pgfqpoint{1.514899in}{0.780410in}}%
\pgfpathlineto{\pgfqpoint{1.524066in}{0.778399in}}%
\pgfpathlineto{\pgfqpoint{1.528334in}{0.771334in}}%
\pgfpathlineto{\pgfqpoint{1.535343in}{0.768954in}}%
\pgfpathlineto{\pgfqpoint{1.540206in}{0.761626in}}%
\pgfpathlineto{\pgfqpoint{1.549915in}{0.769601in}}%
\pgfpathlineto{\pgfqpoint{1.561027in}{0.759707in}}%
\pgfpathlineto{\pgfqpoint{1.573730in}{0.763748in}}%
\pgfpathlineto{\pgfqpoint{1.580162in}{0.753333in}}%
\pgfpathlineto{\pgfqpoint{1.586451in}{0.765444in}}%
\pgfpathlineto{\pgfqpoint{1.597757in}{0.758577in}}%
\pgfpathlineto{\pgfqpoint{1.604821in}{0.760404in}}%
\pgfpathlineto{\pgfqpoint{1.614906in}{0.755549in}}%
\pgfpathlineto{\pgfqpoint{1.627982in}{0.754170in}}%
\pgfpathlineto{\pgfqpoint{1.640377in}{0.761642in}}%
\pgfpathlineto{\pgfqpoint{1.644861in}{0.758488in}}%
\pgfpathlineto{\pgfqpoint{1.657457in}{0.764893in}}%
\pgfpathlineto{\pgfqpoint{1.665484in}{0.759894in}}%
\pgfpathlineto{\pgfqpoint{1.671829in}{0.760192in}}%
\pgfpathlineto{\pgfqpoint{1.676853in}{0.766079in}}%
\pgfpathlineto{\pgfqpoint{1.684760in}{0.763149in}}%
\pgfpathlineto{\pgfqpoint{1.702560in}{0.749517in}}%
\pgfpathlineto{\pgfqpoint{1.716452in}{0.745221in}}%
\pgfpathlineto{\pgfqpoint{1.722023in}{0.739958in}}%
\pgfpathlineto{\pgfqpoint{1.728996in}{0.742824in}}%
\pgfpathlineto{\pgfqpoint{1.739562in}{0.740629in}}%
\pgfpathlineto{\pgfqpoint{1.739769in}{0.707128in}}%
\pgfpathlineto{\pgfqpoint{1.740651in}{0.642332in}}%
\pgfpathlineto{\pgfqpoint{1.747989in}{0.636106in}}%
\pgfpathlineto{\pgfqpoint{1.753872in}{0.624633in}}%
\pgfpathlineto{\pgfqpoint{1.751804in}{0.616957in}}%
\pgfpathlineto{\pgfqpoint{1.756423in}{0.613707in}}%
\pgfpathlineto{\pgfqpoint{1.760177in}{0.599552in}}%
\pgfpathlineto{\pgfqpoint{1.767366in}{0.591971in}}%
\pgfpathlineto{\pgfqpoint{1.768977in}{0.575877in}}%
\pgfpathlineto{\pgfqpoint{1.767727in}{0.569062in}}%
\pgfpathlineto{\pgfqpoint{1.757954in}{0.551028in}}%
\pgfpathlineto{\pgfqpoint{1.756831in}{0.538062in}}%
\pgfpathlineto{\pgfqpoint{1.760145in}{0.535356in}}%
\pgfpathlineto{\pgfqpoint{1.760434in}{0.523996in}}%
\pgfpathlineto{\pgfqpoint{1.757064in}{0.516857in}}%
\pgfpathlineto{\pgfqpoint{1.751824in}{0.514168in}}%
\pgfpathlineto{\pgfqpoint{1.746693in}{0.504838in}}%
\pgfpathlineto{\pgfqpoint{1.753232in}{0.495802in}}%
\pgfpathlineto{\pgfqpoint{1.740538in}{0.495624in}}%
\pgfpathlineto{\pgfqpoint{1.706594in}{0.480094in}}%
\pgfpathlineto{\pgfqpoint{1.713075in}{0.489388in}}%
\pgfpathlineto{\pgfqpoint{1.705227in}{0.494399in}}%
\pgfpathlineto{\pgfqpoint{1.703587in}{0.502919in}}%
\pgfpathlineto{\pgfqpoint{1.688271in}{0.488079in}}%
\pgfpathlineto{\pgfqpoint{1.695068in}{0.477936in}}%
\pgfpathlineto{\pgfqpoint{1.685371in}{0.465021in}}%
\pgfpathlineto{\pgfqpoint{1.680153in}{0.465291in}}%
\pgfpathlineto{\pgfqpoint{1.675244in}{0.451227in}}%
\pgfpathlineto{\pgfqpoint{1.659719in}{0.440163in}}%
\pgfpathlineto{\pgfqpoint{1.645214in}{0.436176in}}%
\pgfpathlineto{\pgfqpoint{1.619917in}{0.425803in}}%
\pgfpathlineto{\pgfqpoint{1.617295in}{0.431590in}}%
\pgfpathlineto{\pgfqpoint{1.601537in}{0.426264in}}%
\pgfpathlineto{\pgfqpoint{1.611229in}{0.417196in}}%
\pgfpathlineto{\pgfqpoint{1.595941in}{0.409318in}}%
\pgfpathlineto{\pgfqpoint{1.579451in}{0.397424in}}%
\pgfpathlineto{\pgfqpoint{1.575490in}{0.402953in}}%
\pgfpathlineto{\pgfqpoint{1.561995in}{0.394666in}}%
\pgfpathlineto{\pgfqpoint{1.575222in}{0.391519in}}%
\pgfpathlineto{\pgfqpoint{1.565249in}{0.378489in}}%
\pgfpathlineto{\pgfqpoint{1.560381in}{0.382367in}}%
\pgfpathlineto{\pgfqpoint{1.553843in}{0.376138in}}%
\pgfpathlineto{\pgfqpoint{1.561992in}{0.370699in}}%
\pgfpathlineto{\pgfqpoint{1.557174in}{0.362747in}}%
\pgfpathlineto{\pgfqpoint{1.551252in}{0.343923in}}%
\pgfpathlineto{\pgfqpoint{1.542711in}{0.325793in}}%
\pgfpathlineto{\pgfqpoint{1.546545in}{0.313909in}}%
\pgfpathlineto{\pgfqpoint{1.553063in}{0.285891in}}%
\pgfpathlineto{\pgfqpoint{1.553650in}{0.274552in}}%
\pgfpathlineto{\pgfqpoint{1.563720in}{0.259679in}}%
\pgfpathlineto{\pgfqpoint{1.555955in}{0.260548in}}%
\pgfpathlineto{\pgfqpoint{1.548412in}{0.253045in}}%
\pgfpathlineto{\pgfqpoint{1.539578in}{0.258942in}}%
\pgfpathlineto{\pgfqpoint{1.536407in}{0.264825in}}%
\pgfpathlineto{\pgfqpoint{1.523834in}{0.267561in}}%
\pgfpathlineto{\pgfqpoint{1.504629in}{0.267897in}}%
\pgfpathlineto{\pgfqpoint{1.490479in}{0.279038in}}%
\pgfpathlineto{\pgfqpoint{1.477631in}{0.280899in}}%
\pgfpathlineto{\pgfqpoint{1.469836in}{0.289752in}}%
\pgfpathlineto{\pgfqpoint{1.453511in}{0.293315in}}%
\pgfpathlineto{\pgfqpoint{1.444585in}{0.321783in}}%
\pgfpathlineto{\pgfqpoint{1.435458in}{0.333188in}}%
\pgfpathlineto{\pgfqpoint{1.437008in}{0.344024in}}%
\pgfpathlineto{\pgfqpoint{1.431349in}{0.351943in}}%
\pgfpathlineto{\pgfqpoint{1.434901in}{0.362767in}}%
\pgfpathlineto{\pgfqpoint{1.431971in}{0.370699in}}%
\pgfpathlineto{\pgfqpoint{1.422799in}{0.374293in}}%
\pgfpathlineto{\pgfqpoint{1.414224in}{0.383439in}}%
\pgfpathlineto{\pgfqpoint{1.411099in}{0.395685in}}%
\pgfpathlineto{\pgfqpoint{1.402984in}{0.406814in}}%
\pgfpathlineto{\pgfqpoint{1.392183in}{0.415473in}}%
\pgfpathlineto{\pgfqpoint{1.382454in}{0.440312in}}%
\pgfpathlineto{\pgfqpoint{1.374599in}{0.467478in}}%
\pgfpathlineto{\pgfqpoint{1.368153in}{0.478219in}}%
\pgfpathlineto{\pgfqpoint{1.356979in}{0.487262in}}%
\pgfpathlineto{\pgfqpoint{1.354186in}{0.493825in}}%
\pgfpathlineto{\pgfqpoint{1.343723in}{0.497901in}}%
\pgfpathlineto{\pgfqpoint{1.333397in}{0.515292in}}%
\pgfpathlineto{\pgfqpoint{1.323966in}{0.513957in}}%
\pgfpathlineto{\pgfqpoint{1.314390in}{0.518301in}}%
\pgfpathlineto{\pgfqpoint{1.300816in}{0.517461in}}%
\pgfpathlineto{\pgfqpoint{1.287025in}{0.524629in}}%
\pgfpathlineto{\pgfqpoint{1.283137in}{0.517812in}}%
\pgfpathlineto{\pgfqpoint{1.267021in}{0.517641in}}%
\pgfpathlineto{\pgfqpoint{1.258816in}{0.504718in}}%
\pgfpathlineto{\pgfqpoint{1.251681in}{0.488717in}}%
\pgfpathlineto{\pgfqpoint{1.236506in}{0.471540in}}%
\pgfpathlineto{\pgfqpoint{1.219314in}{0.479078in}}%
\pgfpathlineto{\pgfqpoint{1.216875in}{0.484060in}}%
\pgfpathlineto{\pgfqpoint{1.206427in}{0.487860in}}%
\pgfpathlineto{\pgfqpoint{1.203218in}{0.493057in}}%
\pgfpathlineto{\pgfqpoint{1.189348in}{0.498303in}}%
\pgfpathlineto{\pgfqpoint{1.181589in}{0.509028in}}%
\pgfpathlineto{\pgfqpoint{1.172528in}{0.514203in}}%
\pgfpathlineto{\pgfqpoint{1.164732in}{0.523233in}}%
\pgfpathlineto{\pgfqpoint{1.158659in}{0.538525in}}%
\pgfpathlineto{\pgfqpoint{1.159338in}{0.559422in}}%
\pgfpathlineto{\pgfqpoint{1.152213in}{0.569982in}}%
\pgfpathlineto{\pgfqpoint{1.151394in}{0.581419in}}%
\pgfpathlineto{\pgfqpoint{1.135586in}{0.598542in}}%
\pgfpathlineto{\pgfqpoint{1.126387in}{0.602198in}}%
\pgfpathlineto{\pgfqpoint{1.116627in}{0.618176in}}%
\pgfpathlineto{\pgfqpoint{1.108309in}{0.624525in}}%
\pgfpathlineto{\pgfqpoint{1.097741in}{0.639983in}}%
\pgfpathlineto{\pgfqpoint{1.086927in}{0.646673in}}%
\pgfpathlineto{\pgfqpoint{1.079846in}{0.663815in}}%
\pgfpathclose%
\pgfusepath{fill}%
\end{pgfscope}%
\begin{pgfscope}%
\pgfpathrectangle{\pgfqpoint{0.100000in}{0.100000in}}{\pgfqpoint{2.857344in}{1.829167in}}%
\pgfusepath{clip}%
\pgfsetbuttcap%
\pgfsetmiterjoin%
\definecolor{currentfill}{rgb}{0.997001,0.907036,0.593080}%
\pgfsetfillcolor{currentfill}%
\pgfsetlinewidth{0.000000pt}%
\definecolor{currentstroke}{rgb}{0.000000,0.000000,0.000000}%
\pgfsetstrokecolor{currentstroke}%
\pgfsetstrokeopacity{0.000000}%
\pgfsetdash{}{0pt}%
\pgfpathmoveto{\pgfqpoint{0.989159in}{1.010405in}}%
\pgfpathlineto{\pgfqpoint{1.067171in}{0.999075in}}%
\pgfpathlineto{\pgfqpoint{1.097393in}{0.994419in}}%
\pgfpathlineto{\pgfqpoint{1.182585in}{0.984286in}}%
\pgfpathlineto{\pgfqpoint{1.239925in}{0.978305in}}%
\pgfpathlineto{\pgfqpoint{1.289547in}{0.973908in}}%
\pgfpathlineto{\pgfqpoint{1.286775in}{0.942324in}}%
\pgfpathlineto{\pgfqpoint{1.285272in}{0.942408in}}%
\pgfpathlineto{\pgfqpoint{1.280726in}{0.888200in}}%
\pgfpathlineto{\pgfqpoint{1.275850in}{0.833319in}}%
\pgfpathlineto{\pgfqpoint{1.270210in}{0.775867in}}%
\pgfpathlineto{\pgfqpoint{1.261514in}{0.693857in}}%
\pgfpathlineto{\pgfqpoint{1.258733in}{0.661614in}}%
\pgfpathlineto{\pgfqpoint{1.207614in}{0.666772in}}%
\pgfpathlineto{\pgfqpoint{1.163353in}{0.671121in}}%
\pgfpathlineto{\pgfqpoint{1.102139in}{0.678074in}}%
\pgfpathlineto{\pgfqpoint{1.069219in}{0.682157in}}%
\pgfpathlineto{\pgfqpoint{1.067281in}{0.675675in}}%
\pgfpathlineto{\pgfqpoint{1.071964in}{0.668149in}}%
\pgfpathlineto{\pgfqpoint{1.032399in}{0.673287in}}%
\pgfpathlineto{\pgfqpoint{0.983585in}{0.680293in}}%
\pgfpathlineto{\pgfqpoint{0.979142in}{0.652684in}}%
\pgfpathlineto{\pgfqpoint{0.934616in}{0.659386in}}%
\pgfpathlineto{\pgfqpoint{0.945078in}{0.727879in}}%
\pgfpathlineto{\pgfqpoint{0.958029in}{0.811272in}}%
\pgfpathlineto{\pgfqpoint{0.969545in}{0.883922in}}%
\pgfpathlineto{\pgfqpoint{0.979545in}{0.948497in}}%
\pgfpathclose%
\pgfusepath{fill}%
\end{pgfscope}%
\begin{pgfscope}%
\pgfpathrectangle{\pgfqpoint{0.100000in}{0.100000in}}{\pgfqpoint{2.857344in}{1.829167in}}%
\pgfusepath{clip}%
\pgfsetbuttcap%
\pgfsetmiterjoin%
\definecolor{currentfill}{rgb}{0.948097,0.979239,0.668051}%
\pgfsetfillcolor{currentfill}%
\pgfsetlinewidth{0.000000pt}%
\definecolor{currentstroke}{rgb}{0.000000,0.000000,0.000000}%
\pgfsetstrokecolor{currentstroke}%
\pgfsetstrokeopacity{0.000000}%
\pgfsetdash{}{0pt}%
\pgfpathmoveto{\pgfqpoint{2.229545in}{0.609114in}}%
\pgfpathlineto{\pgfqpoint{2.155121in}{0.600848in}}%
\pgfpathlineto{\pgfqpoint{2.089518in}{0.595755in}}%
\pgfpathlineto{\pgfqpoint{2.088697in}{0.587716in}}%
\pgfpathlineto{\pgfqpoint{2.094723in}{0.580063in}}%
\pgfpathlineto{\pgfqpoint{2.102078in}{0.575568in}}%
\pgfpathlineto{\pgfqpoint{2.101967in}{0.563732in}}%
\pgfpathlineto{\pgfqpoint{2.100018in}{0.555823in}}%
\pgfpathlineto{\pgfqpoint{2.093527in}{0.550125in}}%
\pgfpathlineto{\pgfqpoint{2.084479in}{0.550728in}}%
\pgfpathlineto{\pgfqpoint{2.075924in}{0.557789in}}%
\pgfpathlineto{\pgfqpoint{2.074399in}{0.570355in}}%
\pgfpathlineto{\pgfqpoint{2.068028in}{0.577663in}}%
\pgfpathlineto{\pgfqpoint{2.063691in}{0.551507in}}%
\pgfpathlineto{\pgfqpoint{2.048956in}{0.553993in}}%
\pgfpathlineto{\pgfqpoint{2.043831in}{0.599731in}}%
\pgfpathlineto{\pgfqpoint{2.038282in}{0.647933in}}%
\pgfpathlineto{\pgfqpoint{2.040343in}{0.717478in}}%
\pgfpathlineto{\pgfqpoint{2.041622in}{0.765200in}}%
\pgfpathlineto{\pgfqpoint{2.044343in}{0.838013in}}%
\pgfpathlineto{\pgfqpoint{2.038519in}{0.844421in}}%
\pgfpathlineto{\pgfqpoint{2.049117in}{0.845957in}}%
\pgfpathlineto{\pgfqpoint{2.110860in}{0.849950in}}%
\pgfpathlineto{\pgfqpoint{2.170934in}{0.855352in}}%
\pgfpathlineto{\pgfqpoint{2.186738in}{0.799991in}}%
\pgfpathlineto{\pgfqpoint{2.197517in}{0.759356in}}%
\pgfpathlineto{\pgfqpoint{2.207179in}{0.725336in}}%
\pgfpathlineto{\pgfqpoint{2.212767in}{0.711638in}}%
\pgfpathlineto{\pgfqpoint{2.221565in}{0.698890in}}%
\pgfpathlineto{\pgfqpoint{2.220144in}{0.692403in}}%
\pgfpathlineto{\pgfqpoint{2.226429in}{0.689242in}}%
\pgfpathlineto{\pgfqpoint{2.219002in}{0.679409in}}%
\pgfpathlineto{\pgfqpoint{2.216493in}{0.661896in}}%
\pgfpathlineto{\pgfqpoint{2.223752in}{0.641414in}}%
\pgfpathlineto{\pgfqpoint{2.222744in}{0.620803in}}%
\pgfpathclose%
\pgfusepath{fill}%
\end{pgfscope}%
\begin{pgfscope}%
\pgfpathrectangle{\pgfqpoint{0.100000in}{0.100000in}}{\pgfqpoint{2.857344in}{1.829167in}}%
\pgfusepath{clip}%
\pgfsetbuttcap%
\pgfsetmiterjoin%
\definecolor{currentfill}{rgb}{0.963476,0.985390,0.692042}%
\pgfsetfillcolor{currentfill}%
\pgfsetlinewidth{0.000000pt}%
\definecolor{currentstroke}{rgb}{0.000000,0.000000,0.000000}%
\pgfsetstrokecolor{currentstroke}%
\pgfsetstrokeopacity{0.000000}%
\pgfsetdash{}{0pt}%
\pgfpathmoveto{\pgfqpoint{2.048956in}{0.553993in}}%
\pgfpathlineto{\pgfqpoint{2.033834in}{0.549666in}}%
\pgfpathlineto{\pgfqpoint{2.022562in}{0.552444in}}%
\pgfpathlineto{\pgfqpoint{2.001630in}{0.545786in}}%
\pgfpathlineto{\pgfqpoint{1.985842in}{0.537211in}}%
\pgfpathlineto{\pgfqpoint{1.979365in}{0.552706in}}%
\pgfpathlineto{\pgfqpoint{1.972687in}{0.559201in}}%
\pgfpathlineto{\pgfqpoint{1.969466in}{0.566972in}}%
\pgfpathlineto{\pgfqpoint{1.974188in}{0.588008in}}%
\pgfpathlineto{\pgfqpoint{1.929448in}{0.585253in}}%
\pgfpathlineto{\pgfqpoint{1.871435in}{0.582738in}}%
\pgfpathlineto{\pgfqpoint{1.874886in}{0.587974in}}%
\pgfpathlineto{\pgfqpoint{1.870620in}{0.597864in}}%
\pgfpathlineto{\pgfqpoint{1.874871in}{0.599879in}}%
\pgfpathlineto{\pgfqpoint{1.873928in}{0.609381in}}%
\pgfpathlineto{\pgfqpoint{1.881337in}{0.618595in}}%
\pgfpathlineto{\pgfqpoint{1.885391in}{0.636483in}}%
\pgfpathlineto{\pgfqpoint{1.898952in}{0.648268in}}%
\pgfpathlineto{\pgfqpoint{1.893926in}{0.659713in}}%
\pgfpathlineto{\pgfqpoint{1.903542in}{0.666301in}}%
\pgfpathlineto{\pgfqpoint{1.895247in}{0.678228in}}%
\pgfpathlineto{\pgfqpoint{1.889243in}{0.705532in}}%
\pgfpathlineto{\pgfqpoint{1.891520in}{0.710489in}}%
\pgfpathlineto{\pgfqpoint{1.895115in}{0.719999in}}%
\pgfpathlineto{\pgfqpoint{1.891769in}{0.729979in}}%
\pgfpathlineto{\pgfqpoint{1.893406in}{0.734520in}}%
\pgfpathlineto{\pgfqpoint{1.887519in}{0.751673in}}%
\pgfpathlineto{\pgfqpoint{1.904896in}{0.782473in}}%
\pgfpathlineto{\pgfqpoint{1.905704in}{0.790671in}}%
\pgfpathlineto{\pgfqpoint{1.914075in}{0.796632in}}%
\pgfpathlineto{\pgfqpoint{1.923006in}{0.816311in}}%
\pgfpathlineto{\pgfqpoint{1.922582in}{0.824310in}}%
\pgfpathlineto{\pgfqpoint{1.933706in}{0.832486in}}%
\pgfpathlineto{\pgfqpoint{1.930235in}{0.837329in}}%
\pgfpathlineto{\pgfqpoint{1.996898in}{0.841219in}}%
\pgfpathlineto{\pgfqpoint{2.038519in}{0.844421in}}%
\pgfpathlineto{\pgfqpoint{2.044343in}{0.838013in}}%
\pgfpathlineto{\pgfqpoint{2.041622in}{0.765200in}}%
\pgfpathlineto{\pgfqpoint{2.040343in}{0.717478in}}%
\pgfpathlineto{\pgfqpoint{2.038282in}{0.647933in}}%
\pgfpathlineto{\pgfqpoint{2.043831in}{0.599731in}}%
\pgfpathclose%
\pgfusepath{fill}%
\end{pgfscope}%
\begin{pgfscope}%
\pgfpathrectangle{\pgfqpoint{0.100000in}{0.100000in}}{\pgfqpoint{2.857344in}{1.829167in}}%
\pgfusepath{clip}%
\pgfsetbuttcap%
\pgfsetmiterjoin%
\definecolor{currentfill}{rgb}{0.948097,0.979239,0.668051}%
\pgfsetfillcolor{currentfill}%
\pgfsetlinewidth{0.000000pt}%
\definecolor{currentstroke}{rgb}{0.000000,0.000000,0.000000}%
\pgfsetstrokecolor{currentstroke}%
\pgfsetstrokeopacity{0.000000}%
\pgfsetdash{}{0pt}%
\pgfpathmoveto{\pgfqpoint{2.170934in}{0.855352in}}%
\pgfpathlineto{\pgfqpoint{2.236602in}{0.862564in}}%
\pgfpathlineto{\pgfqpoint{2.276114in}{0.867433in}}%
\pgfpathlineto{\pgfqpoint{2.298617in}{0.871018in}}%
\pgfpathlineto{\pgfqpoint{2.289295in}{0.856315in}}%
\pgfpathlineto{\pgfqpoint{2.289301in}{0.849383in}}%
\pgfpathlineto{\pgfqpoint{2.298897in}{0.845645in}}%
\pgfpathlineto{\pgfqpoint{2.305422in}{0.839610in}}%
\pgfpathlineto{\pgfqpoint{2.315284in}{0.838864in}}%
\pgfpathlineto{\pgfqpoint{2.324501in}{0.821874in}}%
\pgfpathlineto{\pgfqpoint{2.334499in}{0.809893in}}%
\pgfpathlineto{\pgfqpoint{2.352255in}{0.799823in}}%
\pgfpathlineto{\pgfqpoint{2.357302in}{0.791792in}}%
\pgfpathlineto{\pgfqpoint{2.373136in}{0.783106in}}%
\pgfpathlineto{\pgfqpoint{2.374327in}{0.776807in}}%
\pgfpathlineto{\pgfqpoint{2.387172in}{0.764281in}}%
\pgfpathlineto{\pgfqpoint{2.397012in}{0.760715in}}%
\pgfpathlineto{\pgfqpoint{2.403979in}{0.748888in}}%
\pgfpathlineto{\pgfqpoint{2.407063in}{0.735594in}}%
\pgfpathlineto{\pgfqpoint{2.417283in}{0.730456in}}%
\pgfpathlineto{\pgfqpoint{2.425511in}{0.716166in}}%
\pgfpathlineto{\pgfqpoint{2.428164in}{0.705657in}}%
\pgfpathlineto{\pgfqpoint{2.440221in}{0.701184in}}%
\pgfpathlineto{\pgfqpoint{2.430104in}{0.681813in}}%
\pgfpathlineto{\pgfqpoint{2.426297in}{0.670365in}}%
\pgfpathlineto{\pgfqpoint{2.428781in}{0.664999in}}%
\pgfpathlineto{\pgfqpoint{2.424443in}{0.656081in}}%
\pgfpathlineto{\pgfqpoint{2.422599in}{0.643751in}}%
\pgfpathlineto{\pgfqpoint{2.414886in}{0.641454in}}%
\pgfpathlineto{\pgfqpoint{2.418959in}{0.630165in}}%
\pgfpathlineto{\pgfqpoint{2.418699in}{0.615843in}}%
\pgfpathlineto{\pgfqpoint{2.404989in}{0.616380in}}%
\pgfpathlineto{\pgfqpoint{2.395438in}{0.618993in}}%
\pgfpathlineto{\pgfqpoint{2.391305in}{0.614136in}}%
\pgfpathlineto{\pgfqpoint{2.394389in}{0.602660in}}%
\pgfpathlineto{\pgfqpoint{2.393718in}{0.589319in}}%
\pgfpathlineto{\pgfqpoint{2.387692in}{0.588298in}}%
\pgfpathlineto{\pgfqpoint{2.382818in}{0.600758in}}%
\pgfpathlineto{\pgfqpoint{2.333184in}{0.597451in}}%
\pgfpathlineto{\pgfqpoint{2.270600in}{0.594036in}}%
\pgfpathlineto{\pgfqpoint{2.239037in}{0.591814in}}%
\pgfpathlineto{\pgfqpoint{2.229545in}{0.609114in}}%
\pgfpathlineto{\pgfqpoint{2.222744in}{0.620803in}}%
\pgfpathlineto{\pgfqpoint{2.223752in}{0.641414in}}%
\pgfpathlineto{\pgfqpoint{2.216493in}{0.661896in}}%
\pgfpathlineto{\pgfqpoint{2.219002in}{0.679409in}}%
\pgfpathlineto{\pgfqpoint{2.226429in}{0.689242in}}%
\pgfpathlineto{\pgfqpoint{2.220144in}{0.692403in}}%
\pgfpathlineto{\pgfqpoint{2.221565in}{0.698890in}}%
\pgfpathlineto{\pgfqpoint{2.212767in}{0.711638in}}%
\pgfpathlineto{\pgfqpoint{2.207179in}{0.725336in}}%
\pgfpathlineto{\pgfqpoint{2.197517in}{0.759356in}}%
\pgfpathlineto{\pgfqpoint{2.186738in}{0.799991in}}%
\pgfpathclose%
\pgfusepath{fill}%
\end{pgfscope}%
\begin{pgfscope}%
\pgfpathrectangle{\pgfqpoint{0.100000in}{0.100000in}}{\pgfqpoint{2.857344in}{1.829167in}}%
\pgfusepath{clip}%
\pgfsetbuttcap%
\pgfsetmiterjoin%
\definecolor{currentfill}{rgb}{0.998539,0.954710,0.673049}%
\pgfsetfillcolor{currentfill}%
\pgfsetlinewidth{0.000000pt}%
\definecolor{currentstroke}{rgb}{0.000000,0.000000,0.000000}%
\pgfsetstrokecolor{currentstroke}%
\pgfsetstrokeopacity{0.000000}%
\pgfsetdash{}{0pt}%
\pgfpathmoveto{\pgfqpoint{2.298617in}{0.871018in}}%
\pgfpathlineto{\pgfqpoint{2.324657in}{0.883972in}}%
\pgfpathlineto{\pgfqpoint{2.339078in}{0.888863in}}%
\pgfpathlineto{\pgfqpoint{2.401822in}{0.895382in}}%
\pgfpathlineto{\pgfqpoint{2.408467in}{0.893223in}}%
\pgfpathlineto{\pgfqpoint{2.417244in}{0.884400in}}%
\pgfpathlineto{\pgfqpoint{2.417728in}{0.876565in}}%
\pgfpathlineto{\pgfqpoint{2.475476in}{0.885050in}}%
\pgfpathlineto{\pgfqpoint{2.541050in}{0.838044in}}%
\pgfpathlineto{\pgfqpoint{2.528753in}{0.825226in}}%
\pgfpathlineto{\pgfqpoint{2.511879in}{0.795460in}}%
\pgfpathlineto{\pgfqpoint{2.516682in}{0.789062in}}%
\pgfpathlineto{\pgfqpoint{2.507680in}{0.776646in}}%
\pgfpathlineto{\pgfqpoint{2.498727in}{0.775240in}}%
\pgfpathlineto{\pgfqpoint{2.498693in}{0.767778in}}%
\pgfpathlineto{\pgfqpoint{2.492216in}{0.759971in}}%
\pgfpathlineto{\pgfqpoint{2.484156in}{0.758387in}}%
\pgfpathlineto{\pgfqpoint{2.485915in}{0.751450in}}%
\pgfpathlineto{\pgfqpoint{2.481417in}{0.746124in}}%
\pgfpathlineto{\pgfqpoint{2.470657in}{0.741524in}}%
\pgfpathlineto{\pgfqpoint{2.457542in}{0.731058in}}%
\pgfpathlineto{\pgfqpoint{2.460018in}{0.724287in}}%
\pgfpathlineto{\pgfqpoint{2.451764in}{0.720036in}}%
\pgfpathlineto{\pgfqpoint{2.444803in}{0.724506in}}%
\pgfpathlineto{\pgfqpoint{2.439712in}{0.705073in}}%
\pgfpathlineto{\pgfqpoint{2.428164in}{0.705657in}}%
\pgfpathlineto{\pgfqpoint{2.425511in}{0.716166in}}%
\pgfpathlineto{\pgfqpoint{2.417283in}{0.730456in}}%
\pgfpathlineto{\pgfqpoint{2.407063in}{0.735594in}}%
\pgfpathlineto{\pgfqpoint{2.403979in}{0.748888in}}%
\pgfpathlineto{\pgfqpoint{2.397012in}{0.760715in}}%
\pgfpathlineto{\pgfqpoint{2.387172in}{0.764281in}}%
\pgfpathlineto{\pgfqpoint{2.374327in}{0.776807in}}%
\pgfpathlineto{\pgfqpoint{2.373136in}{0.783106in}}%
\pgfpathlineto{\pgfqpoint{2.357302in}{0.791792in}}%
\pgfpathlineto{\pgfqpoint{2.352255in}{0.799823in}}%
\pgfpathlineto{\pgfqpoint{2.334499in}{0.809893in}}%
\pgfpathlineto{\pgfqpoint{2.324501in}{0.821874in}}%
\pgfpathlineto{\pgfqpoint{2.315284in}{0.838864in}}%
\pgfpathlineto{\pgfqpoint{2.305422in}{0.839610in}}%
\pgfpathlineto{\pgfqpoint{2.298897in}{0.845645in}}%
\pgfpathlineto{\pgfqpoint{2.289301in}{0.849383in}}%
\pgfpathlineto{\pgfqpoint{2.289295in}{0.856315in}}%
\pgfpathclose%
\pgfusepath{fill}%
\end{pgfscope}%
\begin{pgfscope}%
\pgfpathrectangle{\pgfqpoint{0.100000in}{0.100000in}}{\pgfqpoint{2.857344in}{1.829167in}}%
\pgfusepath{clip}%
\pgfsetbuttcap%
\pgfsetmiterjoin%
\definecolor{currentfill}{rgb}{0.951942,0.980777,0.674048}%
\pgfsetfillcolor{currentfill}%
\pgfsetlinewidth{0.000000pt}%
\definecolor{currentstroke}{rgb}{0.000000,0.000000,0.000000}%
\pgfsetstrokecolor{currentstroke}%
\pgfsetstrokeopacity{0.000000}%
\pgfsetdash{}{0pt}%
\pgfpathmoveto{\pgfqpoint{1.708339in}{0.924467in}}%
\pgfpathlineto{\pgfqpoint{1.746605in}{0.924811in}}%
\pgfpathlineto{\pgfqpoint{1.797229in}{0.925734in}}%
\pgfpathlineto{\pgfqpoint{1.883735in}{0.928480in}}%
\pgfpathlineto{\pgfqpoint{1.933214in}{0.931097in}}%
\pgfpathlineto{\pgfqpoint{1.938520in}{0.924528in}}%
\pgfpathlineto{\pgfqpoint{1.938189in}{0.917589in}}%
\pgfpathlineto{\pgfqpoint{1.926203in}{0.905620in}}%
\pgfpathlineto{\pgfqpoint{1.923305in}{0.899066in}}%
\pgfpathlineto{\pgfqpoint{1.956548in}{0.901525in}}%
\pgfpathlineto{\pgfqpoint{1.956530in}{0.889416in}}%
\pgfpathlineto{\pgfqpoint{1.945865in}{0.884241in}}%
\pgfpathlineto{\pgfqpoint{1.946016in}{0.876013in}}%
\pgfpathlineto{\pgfqpoint{1.942011in}{0.870099in}}%
\pgfpathlineto{\pgfqpoint{1.939389in}{0.857537in}}%
\pgfpathlineto{\pgfqpoint{1.942095in}{0.847174in}}%
\pgfpathlineto{\pgfqpoint{1.930235in}{0.837329in}}%
\pgfpathlineto{\pgfqpoint{1.933706in}{0.832486in}}%
\pgfpathlineto{\pgfqpoint{1.922582in}{0.824310in}}%
\pgfpathlineto{\pgfqpoint{1.923006in}{0.816311in}}%
\pgfpathlineto{\pgfqpoint{1.914075in}{0.796632in}}%
\pgfpathlineto{\pgfqpoint{1.905704in}{0.790671in}}%
\pgfpathlineto{\pgfqpoint{1.904896in}{0.782473in}}%
\pgfpathlineto{\pgfqpoint{1.887519in}{0.751673in}}%
\pgfpathlineto{\pgfqpoint{1.893406in}{0.734520in}}%
\pgfpathlineto{\pgfqpoint{1.891769in}{0.729979in}}%
\pgfpathlineto{\pgfqpoint{1.895115in}{0.719999in}}%
\pgfpathlineto{\pgfqpoint{1.891520in}{0.710489in}}%
\pgfpathlineto{\pgfqpoint{1.844003in}{0.708528in}}%
\pgfpathlineto{\pgfqpoint{1.782317in}{0.707512in}}%
\pgfpathlineto{\pgfqpoint{1.739769in}{0.707128in}}%
\pgfpathlineto{\pgfqpoint{1.739562in}{0.740629in}}%
\pgfpathlineto{\pgfqpoint{1.728996in}{0.742824in}}%
\pgfpathlineto{\pgfqpoint{1.722023in}{0.739958in}}%
\pgfpathlineto{\pgfqpoint{1.716452in}{0.745221in}}%
\pgfpathlineto{\pgfqpoint{1.717027in}{0.780708in}}%
\pgfpathlineto{\pgfqpoint{1.718273in}{0.856267in}}%
\pgfpathlineto{\pgfqpoint{1.712230in}{0.900512in}}%
\pgfpathclose%
\pgfusepath{fill}%
\end{pgfscope}%
\begin{pgfscope}%
\pgfpathrectangle{\pgfqpoint{0.100000in}{0.100000in}}{\pgfqpoint{2.857344in}{1.829167in}}%
\pgfusepath{clip}%
\pgfsetbuttcap%
\pgfsetmiterjoin%
\definecolor{currentfill}{rgb}{0.994848,0.816917,0.493426}%
\pgfsetfillcolor{currentfill}%
\pgfsetlinewidth{0.000000pt}%
\definecolor{currentstroke}{rgb}{0.000000,0.000000,0.000000}%
\pgfsetstrokecolor{currentstroke}%
\pgfsetstrokeopacity{0.000000}%
\pgfsetdash{}{0pt}%
\pgfpathmoveto{\pgfqpoint{1.739769in}{0.707128in}}%
\pgfpathlineto{\pgfqpoint{1.782317in}{0.707512in}}%
\pgfpathlineto{\pgfqpoint{1.844003in}{0.708528in}}%
\pgfpathlineto{\pgfqpoint{1.891520in}{0.710489in}}%
\pgfpathlineto{\pgfqpoint{1.889243in}{0.705532in}}%
\pgfpathlineto{\pgfqpoint{1.895247in}{0.678228in}}%
\pgfpathlineto{\pgfqpoint{1.903542in}{0.666301in}}%
\pgfpathlineto{\pgfqpoint{1.893926in}{0.659713in}}%
\pgfpathlineto{\pgfqpoint{1.898952in}{0.648268in}}%
\pgfpathlineto{\pgfqpoint{1.885391in}{0.636483in}}%
\pgfpathlineto{\pgfqpoint{1.881337in}{0.618595in}}%
\pgfpathlineto{\pgfqpoint{1.873928in}{0.609381in}}%
\pgfpathlineto{\pgfqpoint{1.874871in}{0.599879in}}%
\pgfpathlineto{\pgfqpoint{1.870620in}{0.597864in}}%
\pgfpathlineto{\pgfqpoint{1.874886in}{0.587974in}}%
\pgfpathlineto{\pgfqpoint{1.871435in}{0.582738in}}%
\pgfpathlineto{\pgfqpoint{1.929448in}{0.585253in}}%
\pgfpathlineto{\pgfqpoint{1.974188in}{0.588008in}}%
\pgfpathlineto{\pgfqpoint{1.969466in}{0.566972in}}%
\pgfpathlineto{\pgfqpoint{1.972687in}{0.559201in}}%
\pgfpathlineto{\pgfqpoint{1.979365in}{0.552706in}}%
\pgfpathlineto{\pgfqpoint{1.985842in}{0.537211in}}%
\pgfpathlineto{\pgfqpoint{1.965382in}{0.540781in}}%
\pgfpathlineto{\pgfqpoint{1.957835in}{0.546650in}}%
\pgfpathlineto{\pgfqpoint{1.948842in}{0.546929in}}%
\pgfpathlineto{\pgfqpoint{1.939388in}{0.534077in}}%
\pgfpathlineto{\pgfqpoint{1.941272in}{0.528225in}}%
\pgfpathlineto{\pgfqpoint{1.971475in}{0.524685in}}%
\pgfpathlineto{\pgfqpoint{1.979381in}{0.517958in}}%
\pgfpathlineto{\pgfqpoint{1.992248in}{0.514497in}}%
\pgfpathlineto{\pgfqpoint{1.986629in}{0.506533in}}%
\pgfpathlineto{\pgfqpoint{1.977161in}{0.499587in}}%
\pgfpathlineto{\pgfqpoint{2.002151in}{0.483968in}}%
\pgfpathlineto{\pgfqpoint{2.013791in}{0.481527in}}%
\pgfpathlineto{\pgfqpoint{2.018631in}{0.468760in}}%
\pgfpathlineto{\pgfqpoint{2.007672in}{0.466412in}}%
\pgfpathlineto{\pgfqpoint{1.987073in}{0.482469in}}%
\pgfpathlineto{\pgfqpoint{1.979019in}{0.484701in}}%
\pgfpathlineto{\pgfqpoint{1.975122in}{0.491045in}}%
\pgfpathlineto{\pgfqpoint{1.963176in}{0.488441in}}%
\pgfpathlineto{\pgfqpoint{1.959436in}{0.480267in}}%
\pgfpathlineto{\pgfqpoint{1.961805in}{0.471159in}}%
\pgfpathlineto{\pgfqpoint{1.953762in}{0.465784in}}%
\pgfpathlineto{\pgfqpoint{1.946423in}{0.479026in}}%
\pgfpathlineto{\pgfqpoint{1.933567in}{0.475077in}}%
\pgfpathlineto{\pgfqpoint{1.928750in}{0.467177in}}%
\pgfpathlineto{\pgfqpoint{1.919620in}{0.469424in}}%
\pgfpathlineto{\pgfqpoint{1.913783in}{0.479401in}}%
\pgfpathlineto{\pgfqpoint{1.898297in}{0.482696in}}%
\pgfpathlineto{\pgfqpoint{1.895373in}{0.487888in}}%
\pgfpathlineto{\pgfqpoint{1.879238in}{0.496932in}}%
\pgfpathlineto{\pgfqpoint{1.875233in}{0.504839in}}%
\pgfpathlineto{\pgfqpoint{1.861719in}{0.501607in}}%
\pgfpathlineto{\pgfqpoint{1.863435in}{0.508847in}}%
\pgfpathlineto{\pgfqpoint{1.846638in}{0.501420in}}%
\pgfpathlineto{\pgfqpoint{1.851150in}{0.493715in}}%
\pgfpathlineto{\pgfqpoint{1.838178in}{0.489160in}}%
\pgfpathlineto{\pgfqpoint{1.820997in}{0.491664in}}%
\pgfpathlineto{\pgfqpoint{1.786222in}{0.503575in}}%
\pgfpathlineto{\pgfqpoint{1.759390in}{0.501211in}}%
\pgfpathlineto{\pgfqpoint{1.757064in}{0.516857in}}%
\pgfpathlineto{\pgfqpoint{1.760434in}{0.523996in}}%
\pgfpathlineto{\pgfqpoint{1.760145in}{0.535356in}}%
\pgfpathlineto{\pgfqpoint{1.756831in}{0.538062in}}%
\pgfpathlineto{\pgfqpoint{1.757954in}{0.551028in}}%
\pgfpathlineto{\pgfqpoint{1.767727in}{0.569062in}}%
\pgfpathlineto{\pgfqpoint{1.768977in}{0.575877in}}%
\pgfpathlineto{\pgfqpoint{1.767366in}{0.591971in}}%
\pgfpathlineto{\pgfqpoint{1.760177in}{0.599552in}}%
\pgfpathlineto{\pgfqpoint{1.756423in}{0.613707in}}%
\pgfpathlineto{\pgfqpoint{1.751804in}{0.616957in}}%
\pgfpathlineto{\pgfqpoint{1.753872in}{0.624633in}}%
\pgfpathlineto{\pgfqpoint{1.747989in}{0.636106in}}%
\pgfpathlineto{\pgfqpoint{1.740651in}{0.642332in}}%
\pgfpathclose%
\pgfusepath{fill}%
\end{pgfscope}%
\begin{pgfscope}%
\pgfpathrectangle{\pgfqpoint{0.100000in}{0.100000in}}{\pgfqpoint{2.857344in}{1.829167in}}%
\pgfusepath{clip}%
\pgfsetbuttcap%
\pgfsetmiterjoin%
\definecolor{currentfill}{rgb}{0.994848,0.816917,0.493426}%
\pgfsetfillcolor{currentfill}%
\pgfsetlinewidth{0.000000pt}%
\definecolor{currentstroke}{rgb}{0.000000,0.000000,0.000000}%
\pgfsetstrokecolor{currentstroke}%
\pgfsetstrokeopacity{0.000000}%
\pgfsetdash{}{0pt}%
\pgfpathmoveto{\pgfqpoint{1.853563in}{0.493125in}}%
\pgfpathlineto{\pgfqpoint{1.859719in}{0.496786in}}%
\pgfpathlineto{\pgfqpoint{1.867190in}{0.492458in}}%
\pgfpathlineto{\pgfqpoint{1.863023in}{0.486502in}}%
\pgfpathclose%
\pgfusepath{fill}%
\end{pgfscope}%
\begin{pgfscope}%
\pgfpathrectangle{\pgfqpoint{0.100000in}{0.100000in}}{\pgfqpoint{2.857344in}{1.829167in}}%
\pgfusepath{clip}%
\pgfsetbuttcap%
\pgfsetmiterjoin%
\definecolor{currentfill}{rgb}{0.999923,0.997616,0.745021}%
\pgfsetfillcolor{currentfill}%
\pgfsetlinewidth{0.000000pt}%
\definecolor{currentstroke}{rgb}{0.000000,0.000000,0.000000}%
\pgfsetstrokecolor{currentstroke}%
\pgfsetstrokeopacity{0.000000}%
\pgfsetdash{}{0pt}%
\pgfpathmoveto{\pgfqpoint{2.101967in}{0.563732in}}%
\pgfpathlineto{\pgfqpoint{2.102078in}{0.575568in}}%
\pgfpathlineto{\pgfqpoint{2.094723in}{0.580063in}}%
\pgfpathlineto{\pgfqpoint{2.088697in}{0.587716in}}%
\pgfpathlineto{\pgfqpoint{2.089518in}{0.595755in}}%
\pgfpathlineto{\pgfqpoint{2.155121in}{0.600848in}}%
\pgfpathlineto{\pgfqpoint{2.229545in}{0.609114in}}%
\pgfpathlineto{\pgfqpoint{2.239037in}{0.591814in}}%
\pgfpathlineto{\pgfqpoint{2.270600in}{0.594036in}}%
\pgfpathlineto{\pgfqpoint{2.333184in}{0.597451in}}%
\pgfpathlineto{\pgfqpoint{2.382818in}{0.600758in}}%
\pgfpathlineto{\pgfqpoint{2.387692in}{0.588298in}}%
\pgfpathlineto{\pgfqpoint{2.393718in}{0.589319in}}%
\pgfpathlineto{\pgfqpoint{2.394389in}{0.602660in}}%
\pgfpathlineto{\pgfqpoint{2.391305in}{0.614136in}}%
\pgfpathlineto{\pgfqpoint{2.395438in}{0.618993in}}%
\pgfpathlineto{\pgfqpoint{2.404989in}{0.616380in}}%
\pgfpathlineto{\pgfqpoint{2.418699in}{0.615843in}}%
\pgfpathlineto{\pgfqpoint{2.420807in}{0.605591in}}%
\pgfpathlineto{\pgfqpoint{2.425025in}{0.599733in}}%
\pgfpathlineto{\pgfqpoint{2.428318in}{0.586916in}}%
\pgfpathlineto{\pgfqpoint{2.438523in}{0.567081in}}%
\pgfpathlineto{\pgfqpoint{2.438574in}{0.561716in}}%
\pgfpathlineto{\pgfqpoint{2.453655in}{0.538426in}}%
\pgfpathlineto{\pgfqpoint{2.455114in}{0.533615in}}%
\pgfpathlineto{\pgfqpoint{2.477740in}{0.500721in}}%
\pgfpathlineto{\pgfqpoint{2.474155in}{0.500189in}}%
\pgfpathlineto{\pgfqpoint{2.483693in}{0.476793in}}%
\pgfpathlineto{\pgfqpoint{2.509896in}{0.436119in}}%
\pgfpathlineto{\pgfqpoint{2.525109in}{0.406213in}}%
\pgfpathlineto{\pgfqpoint{2.530194in}{0.400977in}}%
\pgfpathlineto{\pgfqpoint{2.538907in}{0.382206in}}%
\pgfpathlineto{\pgfqpoint{2.541929in}{0.352148in}}%
\pgfpathlineto{\pgfqpoint{2.543132in}{0.329674in}}%
\pgfpathlineto{\pgfqpoint{2.541679in}{0.315283in}}%
\pgfpathlineto{\pgfqpoint{2.537008in}{0.304996in}}%
\pgfpathlineto{\pgfqpoint{2.539178in}{0.291507in}}%
\pgfpathlineto{\pgfqpoint{2.534148in}{0.280829in}}%
\pgfpathlineto{\pgfqpoint{2.519215in}{0.272073in}}%
\pgfpathlineto{\pgfqpoint{2.509502in}{0.272712in}}%
\pgfpathlineto{\pgfqpoint{2.503195in}{0.268135in}}%
\pgfpathlineto{\pgfqpoint{2.501388in}{0.280356in}}%
\pgfpathlineto{\pgfqpoint{2.490948in}{0.283582in}}%
\pgfpathlineto{\pgfqpoint{2.481588in}{0.300634in}}%
\pgfpathlineto{\pgfqpoint{2.480532in}{0.308399in}}%
\pgfpathlineto{\pgfqpoint{2.463782in}{0.313114in}}%
\pgfpathlineto{\pgfqpoint{2.452988in}{0.312093in}}%
\pgfpathlineto{\pgfqpoint{2.446881in}{0.323360in}}%
\pgfpathlineto{\pgfqpoint{2.439889in}{0.343612in}}%
\pgfpathlineto{\pgfqpoint{2.430185in}{0.347717in}}%
\pgfpathlineto{\pgfqpoint{2.424925in}{0.359322in}}%
\pgfpathlineto{\pgfqpoint{2.412060in}{0.366161in}}%
\pgfpathlineto{\pgfqpoint{2.404624in}{0.374717in}}%
\pgfpathlineto{\pgfqpoint{2.391796in}{0.397975in}}%
\pgfpathlineto{\pgfqpoint{2.390287in}{0.414346in}}%
\pgfpathlineto{\pgfqpoint{2.397278in}{0.424820in}}%
\pgfpathlineto{\pgfqpoint{2.396579in}{0.432098in}}%
\pgfpathlineto{\pgfqpoint{2.388501in}{0.432876in}}%
\pgfpathlineto{\pgfqpoint{2.381752in}{0.437938in}}%
\pgfpathlineto{\pgfqpoint{2.378056in}{0.431757in}}%
\pgfpathlineto{\pgfqpoint{2.384533in}{0.426720in}}%
\pgfpathlineto{\pgfqpoint{2.379819in}{0.417662in}}%
\pgfpathlineto{\pgfqpoint{2.372192in}{0.425180in}}%
\pgfpathlineto{\pgfqpoint{2.373076in}{0.446027in}}%
\pgfpathlineto{\pgfqpoint{2.376755in}{0.462940in}}%
\pgfpathlineto{\pgfqpoint{2.374864in}{0.491962in}}%
\pgfpathlineto{\pgfqpoint{2.367262in}{0.498874in}}%
\pgfpathlineto{\pgfqpoint{2.363439in}{0.507744in}}%
\pgfpathlineto{\pgfqpoint{2.350342in}{0.507547in}}%
\pgfpathlineto{\pgfqpoint{2.337397in}{0.522142in}}%
\pgfpathlineto{\pgfqpoint{2.328717in}{0.526517in}}%
\pgfpathlineto{\pgfqpoint{2.326143in}{0.535760in}}%
\pgfpathlineto{\pgfqpoint{2.317653in}{0.539006in}}%
\pgfpathlineto{\pgfqpoint{2.310617in}{0.549211in}}%
\pgfpathlineto{\pgfqpoint{2.292030in}{0.557541in}}%
\pgfpathlineto{\pgfqpoint{2.277580in}{0.557754in}}%
\pgfpathlineto{\pgfqpoint{2.271290in}{0.554555in}}%
\pgfpathlineto{\pgfqpoint{2.272853in}{0.544567in}}%
\pgfpathlineto{\pgfqpoint{2.266294in}{0.545040in}}%
\pgfpathlineto{\pgfqpoint{2.246116in}{0.531076in}}%
\pgfpathlineto{\pgfqpoint{2.221879in}{0.525541in}}%
\pgfpathlineto{\pgfqpoint{2.221481in}{0.532389in}}%
\pgfpathlineto{\pgfqpoint{2.216110in}{0.539082in}}%
\pgfpathlineto{\pgfqpoint{2.201698in}{0.548323in}}%
\pgfpathlineto{\pgfqpoint{2.181013in}{0.557796in}}%
\pgfpathlineto{\pgfqpoint{2.158544in}{0.562811in}}%
\pgfpathlineto{\pgfqpoint{2.154327in}{0.569631in}}%
\pgfpathlineto{\pgfqpoint{2.146225in}{0.563943in}}%
\pgfpathlineto{\pgfqpoint{2.136470in}{0.562689in}}%
\pgfpathlineto{\pgfqpoint{2.102513in}{0.553742in}}%
\pgfpathclose%
\pgfusepath{fill}%
\end{pgfscope}%
\begin{pgfscope}%
\pgfpathrectangle{\pgfqpoint{0.100000in}{0.100000in}}{\pgfqpoint{2.857344in}{1.829167in}}%
\pgfusepath{clip}%
\pgfsetbuttcap%
\pgfsetmiterjoin%
\definecolor{currentfill}{rgb}{0.999923,0.997616,0.745021}%
\pgfsetfillcolor{currentfill}%
\pgfsetlinewidth{0.000000pt}%
\definecolor{currentstroke}{rgb}{0.000000,0.000000,0.000000}%
\pgfsetstrokecolor{currentstroke}%
\pgfsetstrokeopacity{0.000000}%
\pgfsetdash{}{0pt}%
\pgfpathmoveto{\pgfqpoint{2.481036in}{0.501188in}}%
\pgfpathlineto{\pgfqpoint{2.489822in}{0.490968in}}%
\pgfpathlineto{\pgfqpoint{2.479674in}{0.490325in}}%
\pgfpathclose%
\pgfusepath{fill}%
\end{pgfscope}%
\begin{pgfscope}%
\pgfpathrectangle{\pgfqpoint{0.100000in}{0.100000in}}{\pgfqpoint{2.857344in}{1.829167in}}%
\pgfusepath{clip}%
\pgfsetbuttcap%
\pgfsetmiterjoin%
\definecolor{currentfill}{rgb}{0.995617,0.855363,0.525721}%
\pgfsetfillcolor{currentfill}%
\pgfsetlinewidth{0.000000pt}%
\definecolor{currentstroke}{rgb}{0.000000,0.000000,0.000000}%
\pgfsetstrokecolor{currentstroke}%
\pgfsetstrokeopacity{0.000000}%
\pgfsetdash{}{0pt}%
\pgfpathmoveto{\pgfqpoint{1.963842in}{1.668094in}}%
\pgfpathlineto{\pgfqpoint{1.959076in}{1.658816in}}%
\pgfpathlineto{\pgfqpoint{1.947704in}{1.653401in}}%
\pgfpathlineto{\pgfqpoint{1.942778in}{1.646113in}}%
\pgfpathlineto{\pgfqpoint{1.937009in}{1.651363in}}%
\pgfpathclose%
\pgfusepath{fill}%
\end{pgfscope}%
\begin{pgfscope}%
\pgfpathrectangle{\pgfqpoint{0.100000in}{0.100000in}}{\pgfqpoint{2.857344in}{1.829167in}}%
\pgfusepath{clip}%
\pgfsetbuttcap%
\pgfsetmiterjoin%
\definecolor{currentfill}{rgb}{0.995617,0.855363,0.525721}%
\pgfsetfillcolor{currentfill}%
\pgfsetlinewidth{0.000000pt}%
\definecolor{currentstroke}{rgb}{0.000000,0.000000,0.000000}%
\pgfsetstrokecolor{currentstroke}%
\pgfsetstrokeopacity{0.000000}%
\pgfsetdash{}{0pt}%
\pgfpathmoveto{\pgfqpoint{1.967709in}{1.612259in}}%
\pgfpathlineto{\pgfqpoint{1.979344in}{1.623114in}}%
\pgfpathlineto{\pgfqpoint{1.997291in}{1.625956in}}%
\pgfpathlineto{\pgfqpoint{1.992348in}{1.618407in}}%
\pgfpathlineto{\pgfqpoint{1.980044in}{1.607490in}}%
\pgfpathlineto{\pgfqpoint{1.972851in}{1.593477in}}%
\pgfpathlineto{\pgfqpoint{1.969805in}{1.601077in}}%
\pgfpathlineto{\pgfqpoint{1.964396in}{1.602169in}}%
\pgfpathlineto{\pgfqpoint{1.963852in}{1.609030in}}%
\pgfpathclose%
\pgfusepath{fill}%
\end{pgfscope}%
\begin{pgfscope}%
\pgfpathrectangle{\pgfqpoint{0.100000in}{0.100000in}}{\pgfqpoint{2.857344in}{1.829167in}}%
\pgfusepath{clip}%
\pgfsetbuttcap%
\pgfsetmiterjoin%
\definecolor{currentfill}{rgb}{0.995617,0.855363,0.525721}%
\pgfsetfillcolor{currentfill}%
\pgfsetlinewidth{0.000000pt}%
\definecolor{currentstroke}{rgb}{0.000000,0.000000,0.000000}%
\pgfsetstrokecolor{currentstroke}%
\pgfsetstrokeopacity{0.000000}%
\pgfsetdash{}{0pt}%
\pgfpathmoveto{\pgfqpoint{2.014053in}{1.479771in}}%
\pgfpathlineto{\pgfqpoint{2.010954in}{1.483210in}}%
\pgfpathlineto{\pgfqpoint{2.014212in}{1.492891in}}%
\pgfpathlineto{\pgfqpoint{2.004541in}{1.493508in}}%
\pgfpathlineto{\pgfqpoint{2.007105in}{1.501855in}}%
\pgfpathlineto{\pgfqpoint{2.005511in}{1.515149in}}%
\pgfpathlineto{\pgfqpoint{1.996834in}{1.519758in}}%
\pgfpathlineto{\pgfqpoint{1.987753in}{1.529086in}}%
\pgfpathlineto{\pgfqpoint{1.973761in}{1.531816in}}%
\pgfpathlineto{\pgfqpoint{1.960144in}{1.531739in}}%
\pgfpathlineto{\pgfqpoint{1.946751in}{1.538358in}}%
\pgfpathlineto{\pgfqpoint{1.901945in}{1.548007in}}%
\pgfpathlineto{\pgfqpoint{1.897048in}{1.558231in}}%
\pgfpathlineto{\pgfqpoint{1.888314in}{1.561720in}}%
\pgfpathlineto{\pgfqpoint{1.904812in}{1.569548in}}%
\pgfpathlineto{\pgfqpoint{1.914132in}{1.579314in}}%
\pgfpathlineto{\pgfqpoint{1.931491in}{1.581963in}}%
\pgfpathlineto{\pgfqpoint{1.942173in}{1.591910in}}%
\pgfpathlineto{\pgfqpoint{1.947777in}{1.592308in}}%
\pgfpathlineto{\pgfqpoint{1.952054in}{1.599402in}}%
\pgfpathlineto{\pgfqpoint{1.962480in}{1.607806in}}%
\pgfpathlineto{\pgfqpoint{1.963389in}{1.601866in}}%
\pgfpathlineto{\pgfqpoint{1.968087in}{1.600639in}}%
\pgfpathlineto{\pgfqpoint{1.971621in}{1.593555in}}%
\pgfpathlineto{\pgfqpoint{1.972258in}{1.581478in}}%
\pgfpathlineto{\pgfqpoint{1.982882in}{1.588695in}}%
\pgfpathlineto{\pgfqpoint{1.995261in}{1.590203in}}%
\pgfpathlineto{\pgfqpoint{2.005828in}{1.586422in}}%
\pgfpathlineto{\pgfqpoint{2.020158in}{1.566762in}}%
\pgfpathlineto{\pgfqpoint{2.035809in}{1.569904in}}%
\pgfpathlineto{\pgfqpoint{2.042169in}{1.564643in}}%
\pgfpathlineto{\pgfqpoint{2.052398in}{1.564173in}}%
\pgfpathlineto{\pgfqpoint{2.059193in}{1.573609in}}%
\pgfpathlineto{\pgfqpoint{2.072088in}{1.581950in}}%
\pgfpathlineto{\pgfqpoint{2.084482in}{1.584563in}}%
\pgfpathlineto{\pgfqpoint{2.099860in}{1.584835in}}%
\pgfpathlineto{\pgfqpoint{2.111098in}{1.591278in}}%
\pgfpathlineto{\pgfqpoint{2.120270in}{1.588310in}}%
\pgfpathlineto{\pgfqpoint{2.122223in}{1.574683in}}%
\pgfpathlineto{\pgfqpoint{2.141899in}{1.572549in}}%
\pgfpathlineto{\pgfqpoint{2.152586in}{1.578928in}}%
\pgfpathlineto{\pgfqpoint{2.159994in}{1.564557in}}%
\pgfpathlineto{\pgfqpoint{2.174130in}{1.547940in}}%
\pgfpathlineto{\pgfqpoint{2.154345in}{1.548036in}}%
\pgfpathlineto{\pgfqpoint{2.148094in}{1.545982in}}%
\pgfpathlineto{\pgfqpoint{2.139525in}{1.548652in}}%
\pgfpathlineto{\pgfqpoint{2.138948in}{1.537148in}}%
\pgfpathlineto{\pgfqpoint{2.123390in}{1.546138in}}%
\pgfpathlineto{\pgfqpoint{2.103380in}{1.548885in}}%
\pgfpathlineto{\pgfqpoint{2.097872in}{1.540130in}}%
\pgfpathlineto{\pgfqpoint{2.071692in}{1.535873in}}%
\pgfpathlineto{\pgfqpoint{2.068680in}{1.528456in}}%
\pgfpathlineto{\pgfqpoint{2.050462in}{1.526230in}}%
\pgfpathlineto{\pgfqpoint{2.044954in}{1.518691in}}%
\pgfpathlineto{\pgfqpoint{2.035318in}{1.516669in}}%
\pgfpathlineto{\pgfqpoint{2.027621in}{1.498791in}}%
\pgfpathlineto{\pgfqpoint{2.017876in}{1.481476in}}%
\pgfpathclose%
\pgfusepath{fill}%
\end{pgfscope}%
\begin{pgfscope}%
\pgfpathrectangle{\pgfqpoint{0.100000in}{0.100000in}}{\pgfqpoint{2.857344in}{1.829167in}}%
\pgfusepath{clip}%
\pgfsetbuttcap%
\pgfsetmiterjoin%
\definecolor{currentfill}{rgb}{0.995617,0.855363,0.525721}%
\pgfsetfillcolor{currentfill}%
\pgfsetlinewidth{0.000000pt}%
\definecolor{currentstroke}{rgb}{0.000000,0.000000,0.000000}%
\pgfsetstrokecolor{currentstroke}%
\pgfsetstrokeopacity{0.000000}%
\pgfsetdash{}{0pt}%
\pgfpathmoveto{\pgfqpoint{2.070140in}{1.271890in}}%
\pgfpathlineto{\pgfqpoint{2.079431in}{1.281669in}}%
\pgfpathlineto{\pgfqpoint{2.083664in}{1.295848in}}%
\pgfpathlineto{\pgfqpoint{2.088697in}{1.304066in}}%
\pgfpathlineto{\pgfqpoint{2.091784in}{1.315250in}}%
\pgfpathlineto{\pgfqpoint{2.092744in}{1.337550in}}%
\pgfpathlineto{\pgfqpoint{2.088096in}{1.358921in}}%
\pgfpathlineto{\pgfqpoint{2.072738in}{1.391647in}}%
\pgfpathlineto{\pgfqpoint{2.076897in}{1.401937in}}%
\pgfpathlineto{\pgfqpoint{2.071479in}{1.416159in}}%
\pgfpathlineto{\pgfqpoint{2.080773in}{1.435829in}}%
\pgfpathlineto{\pgfqpoint{2.079259in}{1.457761in}}%
\pgfpathlineto{\pgfqpoint{2.085733in}{1.460522in}}%
\pgfpathlineto{\pgfqpoint{2.086546in}{1.470957in}}%
\pgfpathlineto{\pgfqpoint{2.098037in}{1.477655in}}%
\pgfpathlineto{\pgfqpoint{2.104536in}{1.476574in}}%
\pgfpathlineto{\pgfqpoint{2.106363in}{1.465382in}}%
\pgfpathlineto{\pgfqpoint{2.111436in}{1.464943in}}%
\pgfpathlineto{\pgfqpoint{2.116064in}{1.481086in}}%
\pgfpathlineto{\pgfqpoint{2.114491in}{1.493642in}}%
\pgfpathlineto{\pgfqpoint{2.117500in}{1.500851in}}%
\pgfpathlineto{\pgfqpoint{2.133770in}{1.508299in}}%
\pgfpathlineto{\pgfqpoint{2.126356in}{1.510950in}}%
\pgfpathlineto{\pgfqpoint{2.123941in}{1.517348in}}%
\pgfpathlineto{\pgfqpoint{2.129282in}{1.528584in}}%
\pgfpathlineto{\pgfqpoint{2.139814in}{1.532464in}}%
\pgfpathlineto{\pgfqpoint{2.152044in}{1.525835in}}%
\pgfpathlineto{\pgfqpoint{2.163575in}{1.525752in}}%
\pgfpathlineto{\pgfqpoint{2.168930in}{1.518009in}}%
\pgfpathlineto{\pgfqpoint{2.177005in}{1.518547in}}%
\pgfpathlineto{\pgfqpoint{2.201934in}{1.507782in}}%
\pgfpathlineto{\pgfqpoint{2.206915in}{1.497361in}}%
\pgfpathlineto{\pgfqpoint{2.201475in}{1.493752in}}%
\pgfpathlineto{\pgfqpoint{2.203151in}{1.485943in}}%
\pgfpathlineto{\pgfqpoint{2.208525in}{1.482469in}}%
\pgfpathlineto{\pgfqpoint{2.211520in}{1.472869in}}%
\pgfpathlineto{\pgfqpoint{2.211100in}{1.449480in}}%
\pgfpathlineto{\pgfqpoint{2.204006in}{1.443892in}}%
\pgfpathlineto{\pgfqpoint{2.202413in}{1.431607in}}%
\pgfpathlineto{\pgfqpoint{2.189252in}{1.420253in}}%
\pgfpathlineto{\pgfqpoint{2.190033in}{1.406510in}}%
\pgfpathlineto{\pgfqpoint{2.201568in}{1.401695in}}%
\pgfpathlineto{\pgfqpoint{2.214581in}{1.418842in}}%
\pgfpathlineto{\pgfqpoint{2.215657in}{1.425062in}}%
\pgfpathlineto{\pgfqpoint{2.231886in}{1.435399in}}%
\pgfpathlineto{\pgfqpoint{2.242206in}{1.430611in}}%
\pgfpathlineto{\pgfqpoint{2.248683in}{1.419789in}}%
\pgfpathlineto{\pgfqpoint{2.259130in}{1.382218in}}%
\pgfpathlineto{\pgfqpoint{2.264667in}{1.370331in}}%
\pgfpathlineto{\pgfqpoint{2.262933in}{1.365417in}}%
\pgfpathlineto{\pgfqpoint{2.263102in}{1.348685in}}%
\pgfpathlineto{\pgfqpoint{2.253030in}{1.350291in}}%
\pgfpathlineto{\pgfqpoint{2.247342in}{1.337787in}}%
\pgfpathlineto{\pgfqpoint{2.246554in}{1.329287in}}%
\pgfpathlineto{\pgfqpoint{2.238946in}{1.323830in}}%
\pgfpathlineto{\pgfqpoint{2.237256in}{1.307253in}}%
\pgfpathlineto{\pgfqpoint{2.226192in}{1.286311in}}%
\pgfpathlineto{\pgfqpoint{2.165674in}{1.277298in}}%
\pgfpathlineto{\pgfqpoint{2.165316in}{1.281260in}}%
\pgfpathlineto{\pgfqpoint{2.124833in}{1.276993in}}%
\pgfpathclose%
\pgfusepath{fill}%
\end{pgfscope}%
\begin{pgfscope}%
\pgfpathrectangle{\pgfqpoint{0.100000in}{0.100000in}}{\pgfqpoint{2.857344in}{1.829167in}}%
\pgfusepath{clip}%
\pgfsetbuttcap%
\pgfsetmiterjoin%
\definecolor{currentfill}{rgb}{0.997155,0.911803,0.601077}%
\pgfsetfillcolor{currentfill}%
\pgfsetlinewidth{0.000000pt}%
\definecolor{currentstroke}{rgb}{0.000000,0.000000,0.000000}%
\pgfsetstrokecolor{currentstroke}%
\pgfsetstrokeopacity{0.000000}%
\pgfsetdash{}{0pt}%
\pgfpathmoveto{\pgfqpoint{0.541676in}{0.431164in}}%
\pgfpathlineto{\pgfqpoint{0.540468in}{0.430101in}}%
\pgfpathlineto{\pgfqpoint{0.534923in}{0.431518in}}%
\pgfpathlineto{\pgfqpoint{0.535337in}{0.434309in}}%
\pgfpathlineto{\pgfqpoint{0.538475in}{0.435587in}}%
\pgfpathlineto{\pgfqpoint{0.542950in}{0.442019in}}%
\pgfpathlineto{\pgfqpoint{0.543693in}{0.446367in}}%
\pgfpathlineto{\pgfqpoint{0.546209in}{0.447506in}}%
\pgfpathlineto{\pgfqpoint{0.550301in}{0.447511in}}%
\pgfpathlineto{\pgfqpoint{0.554580in}{0.461619in}}%
\pgfpathlineto{\pgfqpoint{0.556256in}{0.462780in}}%
\pgfpathlineto{\pgfqpoint{0.554753in}{0.465538in}}%
\pgfpathlineto{\pgfqpoint{0.551234in}{0.462748in}}%
\pgfpathlineto{\pgfqpoint{0.540896in}{0.466509in}}%
\pgfpathlineto{\pgfqpoint{0.534725in}{0.472846in}}%
\pgfpathlineto{\pgfqpoint{0.537416in}{0.475737in}}%
\pgfpathlineto{\pgfqpoint{0.536192in}{0.479815in}}%
\pgfpathlineto{\pgfqpoint{0.536946in}{0.484632in}}%
\pgfpathlineto{\pgfqpoint{0.535619in}{0.490264in}}%
\pgfpathlineto{\pgfqpoint{0.535407in}{0.495903in}}%
\pgfpathlineto{\pgfqpoint{0.541346in}{0.494959in}}%
\pgfpathlineto{\pgfqpoint{0.542522in}{0.496621in}}%
\pgfpathlineto{\pgfqpoint{0.545476in}{0.496460in}}%
\pgfpathlineto{\pgfqpoint{0.548878in}{0.490446in}}%
\pgfpathlineto{\pgfqpoint{0.552158in}{0.486044in}}%
\pgfpathlineto{\pgfqpoint{0.549531in}{0.483178in}}%
\pgfpathlineto{\pgfqpoint{0.554074in}{0.482079in}}%
\pgfpathlineto{\pgfqpoint{0.555109in}{0.483484in}}%
\pgfpathlineto{\pgfqpoint{0.552679in}{0.489187in}}%
\pgfpathlineto{\pgfqpoint{0.547564in}{0.493914in}}%
\pgfpathlineto{\pgfqpoint{0.548189in}{0.495895in}}%
\pgfpathlineto{\pgfqpoint{0.544107in}{0.500900in}}%
\pgfpathlineto{\pgfqpoint{0.548276in}{0.501916in}}%
\pgfpathlineto{\pgfqpoint{0.544664in}{0.512633in}}%
\pgfpathlineto{\pgfqpoint{0.547601in}{0.514119in}}%
\pgfpathlineto{\pgfqpoint{0.548317in}{0.516688in}}%
\pgfpathlineto{\pgfqpoint{0.546406in}{0.519741in}}%
\pgfpathlineto{\pgfqpoint{0.550349in}{0.520505in}}%
\pgfpathlineto{\pgfqpoint{0.551057in}{0.522813in}}%
\pgfpathlineto{\pgfqpoint{0.555760in}{0.519348in}}%
\pgfpathlineto{\pgfqpoint{0.556805in}{0.523105in}}%
\pgfpathlineto{\pgfqpoint{0.559277in}{0.524759in}}%
\pgfpathlineto{\pgfqpoint{0.571303in}{0.527058in}}%
\pgfpathlineto{\pgfqpoint{0.574994in}{0.526245in}}%
\pgfpathlineto{\pgfqpoint{0.578087in}{0.530808in}}%
\pgfpathlineto{\pgfqpoint{0.585425in}{0.533802in}}%
\pgfpathlineto{\pgfqpoint{0.591078in}{0.532667in}}%
\pgfpathlineto{\pgfqpoint{0.593657in}{0.528692in}}%
\pgfpathlineto{\pgfqpoint{0.593521in}{0.522378in}}%
\pgfpathlineto{\pgfqpoint{0.596414in}{0.520947in}}%
\pgfpathlineto{\pgfqpoint{0.602314in}{0.521147in}}%
\pgfpathlineto{\pgfqpoint{0.609872in}{0.522789in}}%
\pgfpathlineto{\pgfqpoint{0.609779in}{0.520770in}}%
\pgfpathlineto{\pgfqpoint{0.619332in}{0.515013in}}%
\pgfpathlineto{\pgfqpoint{0.626013in}{0.516556in}}%
\pgfpathlineto{\pgfqpoint{0.627980in}{0.518555in}}%
\pgfpathlineto{\pgfqpoint{0.630086in}{0.523672in}}%
\pgfpathlineto{\pgfqpoint{0.632698in}{0.526949in}}%
\pgfpathlineto{\pgfqpoint{0.632932in}{0.531859in}}%
\pgfpathlineto{\pgfqpoint{0.630540in}{0.533791in}}%
\pgfpathlineto{\pgfqpoint{0.634081in}{0.535536in}}%
\pgfpathlineto{\pgfqpoint{0.640032in}{0.532914in}}%
\pgfpathlineto{\pgfqpoint{0.641970in}{0.534533in}}%
\pgfpathlineto{\pgfqpoint{0.642312in}{0.541480in}}%
\pgfpathlineto{\pgfqpoint{0.638465in}{0.540028in}}%
\pgfpathlineto{\pgfqpoint{0.635798in}{0.542435in}}%
\pgfpathlineto{\pgfqpoint{0.631467in}{0.543936in}}%
\pgfpathlineto{\pgfqpoint{0.624593in}{0.543686in}}%
\pgfpathlineto{\pgfqpoint{0.618007in}{0.545968in}}%
\pgfpathlineto{\pgfqpoint{0.617677in}{0.551654in}}%
\pgfpathlineto{\pgfqpoint{0.611787in}{0.557694in}}%
\pgfpathlineto{\pgfqpoint{0.604317in}{0.560229in}}%
\pgfpathlineto{\pgfqpoint{0.597797in}{0.571922in}}%
\pgfpathlineto{\pgfqpoint{0.598332in}{0.576248in}}%
\pgfpathlineto{\pgfqpoint{0.601843in}{0.578221in}}%
\pgfpathlineto{\pgfqpoint{0.601326in}{0.582317in}}%
\pgfpathlineto{\pgfqpoint{0.602141in}{0.586392in}}%
\pgfpathlineto{\pgfqpoint{0.604919in}{0.583603in}}%
\pgfpathlineto{\pgfqpoint{0.608586in}{0.584766in}}%
\pgfpathlineto{\pgfqpoint{0.608403in}{0.588306in}}%
\pgfpathlineto{\pgfqpoint{0.603318in}{0.595412in}}%
\pgfpathlineto{\pgfqpoint{0.601829in}{0.602741in}}%
\pgfpathlineto{\pgfqpoint{0.606085in}{0.603300in}}%
\pgfpathlineto{\pgfqpoint{0.612456in}{0.602850in}}%
\pgfpathlineto{\pgfqpoint{0.614463in}{0.600399in}}%
\pgfpathlineto{\pgfqpoint{0.620174in}{0.601276in}}%
\pgfpathlineto{\pgfqpoint{0.625482in}{0.600124in}}%
\pgfpathlineto{\pgfqpoint{0.629071in}{0.598099in}}%
\pgfpathlineto{\pgfqpoint{0.631554in}{0.599025in}}%
\pgfpathlineto{\pgfqpoint{0.638150in}{0.597001in}}%
\pgfpathlineto{\pgfqpoint{0.638210in}{0.595210in}}%
\pgfpathlineto{\pgfqpoint{0.643232in}{0.595815in}}%
\pgfpathlineto{\pgfqpoint{0.649967in}{0.591444in}}%
\pgfpathlineto{\pgfqpoint{0.649706in}{0.588542in}}%
\pgfpathlineto{\pgfqpoint{0.646501in}{0.587202in}}%
\pgfpathlineto{\pgfqpoint{0.643095in}{0.584016in}}%
\pgfpathlineto{\pgfqpoint{0.642684in}{0.579641in}}%
\pgfpathlineto{\pgfqpoint{0.650020in}{0.574061in}}%
\pgfpathlineto{\pgfqpoint{0.648847in}{0.571387in}}%
\pgfpathlineto{\pgfqpoint{0.654067in}{0.569334in}}%
\pgfpathlineto{\pgfqpoint{0.655349in}{0.565980in}}%
\pgfpathlineto{\pgfqpoint{0.661974in}{0.568645in}}%
\pgfpathlineto{\pgfqpoint{0.664892in}{0.565152in}}%
\pgfpathlineto{\pgfqpoint{0.666238in}{0.567266in}}%
\pgfpathlineto{\pgfqpoint{0.662273in}{0.574147in}}%
\pgfpathlineto{\pgfqpoint{0.663682in}{0.576208in}}%
\pgfpathlineto{\pgfqpoint{0.664307in}{0.581643in}}%
\pgfpathlineto{\pgfqpoint{0.662727in}{0.584035in}}%
\pgfpathlineto{\pgfqpoint{0.663912in}{0.587249in}}%
\pgfpathlineto{\pgfqpoint{0.667794in}{0.586947in}}%
\pgfpathlineto{\pgfqpoint{0.666991in}{0.581479in}}%
\pgfpathlineto{\pgfqpoint{0.664512in}{0.579655in}}%
\pgfpathlineto{\pgfqpoint{0.664688in}{0.572299in}}%
\pgfpathlineto{\pgfqpoint{0.669135in}{0.571420in}}%
\pgfpathlineto{\pgfqpoint{0.669189in}{0.565951in}}%
\pgfpathlineto{\pgfqpoint{0.674045in}{0.562402in}}%
\pgfpathlineto{\pgfqpoint{0.676278in}{0.564640in}}%
\pgfpathlineto{\pgfqpoint{0.673838in}{0.571449in}}%
\pgfpathlineto{\pgfqpoint{0.671314in}{0.573172in}}%
\pgfpathlineto{\pgfqpoint{0.668832in}{0.572039in}}%
\pgfpathlineto{\pgfqpoint{0.666648in}{0.573485in}}%
\pgfpathlineto{\pgfqpoint{0.667029in}{0.579778in}}%
\pgfpathlineto{\pgfqpoint{0.670674in}{0.582351in}}%
\pgfpathlineto{\pgfqpoint{0.674261in}{0.582114in}}%
\pgfpathlineto{\pgfqpoint{0.672775in}{0.585675in}}%
\pgfpathlineto{\pgfqpoint{0.667398in}{0.588402in}}%
\pgfpathlineto{\pgfqpoint{0.660277in}{0.599268in}}%
\pgfpathlineto{\pgfqpoint{0.663831in}{0.604479in}}%
\pgfpathlineto{\pgfqpoint{0.665809in}{0.609387in}}%
\pgfpathlineto{\pgfqpoint{0.665965in}{0.619244in}}%
\pgfpathlineto{\pgfqpoint{0.665203in}{0.627559in}}%
\pgfpathlineto{\pgfqpoint{0.663030in}{0.633019in}}%
\pgfpathlineto{\pgfqpoint{0.663463in}{0.638478in}}%
\pgfpathlineto{\pgfqpoint{0.669021in}{0.643213in}}%
\pgfpathlineto{\pgfqpoint{0.674547in}{0.648839in}}%
\pgfpathlineto{\pgfqpoint{0.688402in}{0.637344in}}%
\pgfpathlineto{\pgfqpoint{0.697461in}{0.636496in}}%
\pgfpathlineto{\pgfqpoint{0.704937in}{0.639359in}}%
\pgfpathlineto{\pgfqpoint{0.711501in}{0.643128in}}%
\pgfpathlineto{\pgfqpoint{0.712896in}{0.644924in}}%
\pgfpathlineto{\pgfqpoint{0.729098in}{0.649020in}}%
\pgfpathlineto{\pgfqpoint{0.730505in}{0.646020in}}%
\pgfpathlineto{\pgfqpoint{0.736010in}{0.643565in}}%
\pgfpathlineto{\pgfqpoint{0.746517in}{0.644318in}}%
\pgfpathlineto{\pgfqpoint{0.748436in}{0.643647in}}%
\pgfpathlineto{\pgfqpoint{0.753159in}{0.645439in}}%
\pgfpathlineto{\pgfqpoint{0.755057in}{0.642341in}}%
\pgfpathlineto{\pgfqpoint{0.759592in}{0.639510in}}%
\pgfpathlineto{\pgfqpoint{0.761860in}{0.639830in}}%
\pgfpathlineto{\pgfqpoint{0.766251in}{0.636429in}}%
\pgfpathlineto{\pgfqpoint{0.773670in}{0.637054in}}%
\pgfpathlineto{\pgfqpoint{0.781081in}{0.639808in}}%
\pgfpathlineto{\pgfqpoint{0.787133in}{0.631003in}}%
\pgfpathlineto{\pgfqpoint{0.786663in}{0.629027in}}%
\pgfpathlineto{\pgfqpoint{0.779985in}{0.626756in}}%
\pgfpathlineto{\pgfqpoint{0.781770in}{0.623322in}}%
\pgfpathlineto{\pgfqpoint{0.787917in}{0.626936in}}%
\pgfpathlineto{\pgfqpoint{0.790427in}{0.626061in}}%
\pgfpathlineto{\pgfqpoint{0.791495in}{0.621738in}}%
\pgfpathlineto{\pgfqpoint{0.787825in}{0.619910in}}%
\pgfpathlineto{\pgfqpoint{0.790446in}{0.614445in}}%
\pgfpathlineto{\pgfqpoint{0.794252in}{0.615369in}}%
\pgfpathlineto{\pgfqpoint{0.799776in}{0.612548in}}%
\pgfpathlineto{\pgfqpoint{0.805058in}{0.604721in}}%
\pgfpathlineto{\pgfqpoint{0.800410in}{0.602548in}}%
\pgfpathlineto{\pgfqpoint{0.804516in}{0.596780in}}%
\pgfpathlineto{\pgfqpoint{0.801057in}{0.595432in}}%
\pgfpathlineto{\pgfqpoint{0.805361in}{0.589954in}}%
\pgfpathlineto{\pgfqpoint{0.810609in}{0.590538in}}%
\pgfpathlineto{\pgfqpoint{0.813373in}{0.588073in}}%
\pgfpathlineto{\pgfqpoint{0.820363in}{0.584439in}}%
\pgfpathlineto{\pgfqpoint{0.829380in}{0.571539in}}%
\pgfpathlineto{\pgfqpoint{0.829177in}{0.568454in}}%
\pgfpathlineto{\pgfqpoint{0.842700in}{0.558572in}}%
\pgfpathlineto{\pgfqpoint{0.842991in}{0.555076in}}%
\pgfpathlineto{\pgfqpoint{0.846596in}{0.549847in}}%
\pgfpathlineto{\pgfqpoint{0.857667in}{0.547232in}}%
\pgfpathlineto{\pgfqpoint{0.861773in}{0.545373in}}%
\pgfpathlineto{\pgfqpoint{0.865637in}{0.539608in}}%
\pgfpathlineto{\pgfqpoint{0.866134in}{0.534558in}}%
\pgfpathlineto{\pgfqpoint{0.869454in}{0.530441in}}%
\pgfpathlineto{\pgfqpoint{0.870451in}{0.525799in}}%
\pgfpathlineto{\pgfqpoint{0.873563in}{0.524104in}}%
\pgfpathlineto{\pgfqpoint{0.858149in}{0.496285in}}%
\pgfpathlineto{\pgfqpoint{0.826156in}{0.438531in}}%
\pgfpathlineto{\pgfqpoint{0.779216in}{0.353779in}}%
\pgfpathlineto{\pgfqpoint{0.760838in}{0.320590in}}%
\pgfpathlineto{\pgfqpoint{0.764753in}{0.316132in}}%
\pgfpathlineto{\pgfqpoint{0.766477in}{0.317555in}}%
\pgfpathlineto{\pgfqpoint{0.769973in}{0.312365in}}%
\pgfpathlineto{\pgfqpoint{0.774827in}{0.313871in}}%
\pgfpathlineto{\pgfqpoint{0.781270in}{0.310762in}}%
\pgfpathlineto{\pgfqpoint{0.777097in}{0.305904in}}%
\pgfpathlineto{\pgfqpoint{0.780319in}{0.299348in}}%
\pgfpathlineto{\pgfqpoint{0.779653in}{0.296156in}}%
\pgfpathlineto{\pgfqpoint{0.784793in}{0.279323in}}%
\pgfpathlineto{\pgfqpoint{0.782618in}{0.271668in}}%
\pgfpathlineto{\pgfqpoint{0.792438in}{0.273563in}}%
\pgfpathlineto{\pgfqpoint{0.794849in}{0.272321in}}%
\pgfpathlineto{\pgfqpoint{0.797467in}{0.274341in}}%
\pgfpathlineto{\pgfqpoint{0.799329in}{0.278157in}}%
\pgfpathlineto{\pgfqpoint{0.801995in}{0.280345in}}%
\pgfpathlineto{\pgfqpoint{0.813375in}{0.280068in}}%
\pgfpathlineto{\pgfqpoint{0.815947in}{0.272723in}}%
\pgfpathlineto{\pgfqpoint{0.813740in}{0.266275in}}%
\pgfpathlineto{\pgfqpoint{0.816248in}{0.264147in}}%
\pgfpathlineto{\pgfqpoint{0.817567in}{0.260508in}}%
\pgfpathlineto{\pgfqpoint{0.817279in}{0.253664in}}%
\pgfpathlineto{\pgfqpoint{0.820625in}{0.248757in}}%
\pgfpathlineto{\pgfqpoint{0.822460in}{0.241067in}}%
\pgfpathlineto{\pgfqpoint{0.821518in}{0.236913in}}%
\pgfpathlineto{\pgfqpoint{0.822654in}{0.232704in}}%
\pgfpathlineto{\pgfqpoint{0.824292in}{0.208325in}}%
\pgfpathlineto{\pgfqpoint{0.821945in}{0.206405in}}%
\pgfpathlineto{\pgfqpoint{0.825041in}{0.203752in}}%
\pgfpathlineto{\pgfqpoint{0.822608in}{0.200525in}}%
\pgfpathlineto{\pgfqpoint{0.824803in}{0.197893in}}%
\pgfpathlineto{\pgfqpoint{0.823389in}{0.193402in}}%
\pgfpathlineto{\pgfqpoint{0.826550in}{0.192184in}}%
\pgfpathlineto{\pgfqpoint{0.830616in}{0.185354in}}%
\pgfpathlineto{\pgfqpoint{0.833637in}{0.183295in}}%
\pgfpathlineto{\pgfqpoint{0.834427in}{0.180287in}}%
\pgfpathlineto{\pgfqpoint{0.836178in}{0.178983in}}%
\pgfpathlineto{\pgfqpoint{0.835813in}{0.176485in}}%
\pgfpathlineto{\pgfqpoint{0.839585in}{0.174675in}}%
\pgfpathlineto{\pgfqpoint{0.838654in}{0.169868in}}%
\pgfpathlineto{\pgfqpoint{0.835406in}{0.167350in}}%
\pgfpathlineto{\pgfqpoint{0.833931in}{0.162671in}}%
\pgfpathlineto{\pgfqpoint{0.833539in}{0.156407in}}%
\pgfpathlineto{\pgfqpoint{0.831735in}{0.155807in}}%
\pgfpathlineto{\pgfqpoint{0.825838in}{0.150315in}}%
\pgfpathlineto{\pgfqpoint{0.820597in}{0.148866in}}%
\pgfpathlineto{\pgfqpoint{0.818095in}{0.151080in}}%
\pgfpathlineto{\pgfqpoint{0.820512in}{0.156958in}}%
\pgfpathlineto{\pgfqpoint{0.819774in}{0.159844in}}%
\pgfpathlineto{\pgfqpoint{0.823237in}{0.161305in}}%
\pgfpathlineto{\pgfqpoint{0.826475in}{0.169810in}}%
\pgfpathlineto{\pgfqpoint{0.825108in}{0.177073in}}%
\pgfpathlineto{\pgfqpoint{0.822919in}{0.178788in}}%
\pgfpathlineto{\pgfqpoint{0.815599in}{0.178233in}}%
\pgfpathlineto{\pgfqpoint{0.817074in}{0.176335in}}%
\pgfpathlineto{\pgfqpoint{0.811776in}{0.170888in}}%
\pgfpathlineto{\pgfqpoint{0.810069in}{0.173917in}}%
\pgfpathlineto{\pgfqpoint{0.810513in}{0.177873in}}%
\pgfpathlineto{\pgfqpoint{0.813555in}{0.177800in}}%
\pgfpathlineto{\pgfqpoint{0.816262in}{0.180707in}}%
\pgfpathlineto{\pgfqpoint{0.817865in}{0.184951in}}%
\pgfpathlineto{\pgfqpoint{0.823902in}{0.183753in}}%
\pgfpathlineto{\pgfqpoint{0.819069in}{0.186020in}}%
\pgfpathlineto{\pgfqpoint{0.819494in}{0.189057in}}%
\pgfpathlineto{\pgfqpoint{0.817483in}{0.190428in}}%
\pgfpathlineto{\pgfqpoint{0.816659in}{0.194565in}}%
\pgfpathlineto{\pgfqpoint{0.818087in}{0.196720in}}%
\pgfpathlineto{\pgfqpoint{0.814682in}{0.203413in}}%
\pgfpathlineto{\pgfqpoint{0.814969in}{0.207671in}}%
\pgfpathlineto{\pgfqpoint{0.810979in}{0.211462in}}%
\pgfpathlineto{\pgfqpoint{0.808821in}{0.214530in}}%
\pgfpathlineto{\pgfqpoint{0.811802in}{0.217453in}}%
\pgfpathlineto{\pgfqpoint{0.812906in}{0.221822in}}%
\pgfpathlineto{\pgfqpoint{0.811920in}{0.224691in}}%
\pgfpathlineto{\pgfqpoint{0.813074in}{0.228113in}}%
\pgfpathlineto{\pgfqpoint{0.811604in}{0.239268in}}%
\pgfpathlineto{\pgfqpoint{0.809406in}{0.244329in}}%
\pgfpathlineto{\pgfqpoint{0.806893in}{0.246231in}}%
\pgfpathlineto{\pgfqpoint{0.807963in}{0.252814in}}%
\pgfpathlineto{\pgfqpoint{0.807208in}{0.257956in}}%
\pgfpathlineto{\pgfqpoint{0.808575in}{0.261254in}}%
\pgfpathlineto{\pgfqpoint{0.805747in}{0.261716in}}%
\pgfpathlineto{\pgfqpoint{0.804989in}{0.253705in}}%
\pgfpathlineto{\pgfqpoint{0.801810in}{0.244769in}}%
\pgfpathlineto{\pgfqpoint{0.799258in}{0.246804in}}%
\pgfpathlineto{\pgfqpoint{0.798917in}{0.249741in}}%
\pgfpathlineto{\pgfqpoint{0.795795in}{0.253753in}}%
\pgfpathlineto{\pgfqpoint{0.797381in}{0.256380in}}%
\pgfpathlineto{\pgfqpoint{0.796888in}{0.262293in}}%
\pgfpathlineto{\pgfqpoint{0.792847in}{0.260213in}}%
\pgfpathlineto{\pgfqpoint{0.793640in}{0.257006in}}%
\pgfpathlineto{\pgfqpoint{0.792879in}{0.252934in}}%
\pgfpathlineto{\pgfqpoint{0.788343in}{0.252942in}}%
\pgfpathlineto{\pgfqpoint{0.786310in}{0.255466in}}%
\pgfpathlineto{\pgfqpoint{0.783241in}{0.255947in}}%
\pgfpathlineto{\pgfqpoint{0.781176in}{0.258664in}}%
\pgfpathlineto{\pgfqpoint{0.777623in}{0.266126in}}%
\pgfpathlineto{\pgfqpoint{0.776472in}{0.271919in}}%
\pgfpathlineto{\pgfqpoint{0.777209in}{0.274382in}}%
\pgfpathlineto{\pgfqpoint{0.775489in}{0.279805in}}%
\pgfpathlineto{\pgfqpoint{0.772639in}{0.282944in}}%
\pgfpathlineto{\pgfqpoint{0.767711in}{0.290967in}}%
\pgfpathlineto{\pgfqpoint{0.764136in}{0.297935in}}%
\pgfpathlineto{\pgfqpoint{0.767784in}{0.298040in}}%
\pgfpathlineto{\pgfqpoint{0.769930in}{0.299503in}}%
\pgfpathlineto{\pgfqpoint{0.770352in}{0.304032in}}%
\pgfpathlineto{\pgfqpoint{0.760031in}{0.304579in}}%
\pgfpathlineto{\pgfqpoint{0.751254in}{0.314384in}}%
\pgfpathlineto{\pgfqpoint{0.753562in}{0.316963in}}%
\pgfpathlineto{\pgfqpoint{0.749317in}{0.317404in}}%
\pgfpathlineto{\pgfqpoint{0.740887in}{0.326520in}}%
\pgfpathlineto{\pgfqpoint{0.727531in}{0.331091in}}%
\pgfpathlineto{\pgfqpoint{0.725732in}{0.336448in}}%
\pgfpathlineto{\pgfqpoint{0.722961in}{0.339239in}}%
\pgfpathlineto{\pgfqpoint{0.722835in}{0.341859in}}%
\pgfpathlineto{\pgfqpoint{0.720750in}{0.343805in}}%
\pgfpathlineto{\pgfqpoint{0.723486in}{0.347057in}}%
\pgfpathlineto{\pgfqpoint{0.717247in}{0.347370in}}%
\pgfpathlineto{\pgfqpoint{0.715660in}{0.351578in}}%
\pgfpathlineto{\pgfqpoint{0.712821in}{0.353370in}}%
\pgfpathlineto{\pgfqpoint{0.718508in}{0.355533in}}%
\pgfpathlineto{\pgfqpoint{0.715550in}{0.359390in}}%
\pgfpathlineto{\pgfqpoint{0.710107in}{0.361009in}}%
\pgfpathlineto{\pgfqpoint{0.710954in}{0.368134in}}%
\pgfpathlineto{\pgfqpoint{0.709393in}{0.370656in}}%
\pgfpathlineto{\pgfqpoint{0.706583in}{0.372432in}}%
\pgfpathlineto{\pgfqpoint{0.704180in}{0.370651in}}%
\pgfpathlineto{\pgfqpoint{0.702760in}{0.372326in}}%
\pgfpathlineto{\pgfqpoint{0.698618in}{0.372806in}}%
\pgfpathlineto{\pgfqpoint{0.698952in}{0.379272in}}%
\pgfpathlineto{\pgfqpoint{0.692948in}{0.374905in}}%
\pgfpathlineto{\pgfqpoint{0.693075in}{0.365906in}}%
\pgfpathlineto{\pgfqpoint{0.686630in}{0.364180in}}%
\pgfpathlineto{\pgfqpoint{0.682422in}{0.360586in}}%
\pgfpathlineto{\pgfqpoint{0.679039in}{0.359829in}}%
\pgfpathlineto{\pgfqpoint{0.675378in}{0.363451in}}%
\pgfpathlineto{\pgfqpoint{0.672119in}{0.362971in}}%
\pgfpathlineto{\pgfqpoint{0.672723in}{0.366280in}}%
\pgfpathlineto{\pgfqpoint{0.668089in}{0.364498in}}%
\pgfpathlineto{\pgfqpoint{0.663796in}{0.364298in}}%
\pgfpathlineto{\pgfqpoint{0.658771in}{0.362492in}}%
\pgfpathlineto{\pgfqpoint{0.648573in}{0.363508in}}%
\pgfpathlineto{\pgfqpoint{0.645299in}{0.362840in}}%
\pgfpathlineto{\pgfqpoint{0.642937in}{0.365814in}}%
\pgfpathlineto{\pgfqpoint{0.638577in}{0.366112in}}%
\pgfpathlineto{\pgfqpoint{0.637670in}{0.369967in}}%
\pgfpathlineto{\pgfqpoint{0.640268in}{0.372067in}}%
\pgfpathlineto{\pgfqpoint{0.643318in}{0.371869in}}%
\pgfpathlineto{\pgfqpoint{0.645862in}{0.369006in}}%
\pgfpathlineto{\pgfqpoint{0.647538in}{0.373405in}}%
\pgfpathlineto{\pgfqpoint{0.645200in}{0.378278in}}%
\pgfpathlineto{\pgfqpoint{0.650492in}{0.382567in}}%
\pgfpathlineto{\pgfqpoint{0.655796in}{0.384148in}}%
\pgfpathlineto{\pgfqpoint{0.659489in}{0.386726in}}%
\pgfpathlineto{\pgfqpoint{0.661881in}{0.389504in}}%
\pgfpathlineto{\pgfqpoint{0.663215in}{0.394057in}}%
\pgfpathlineto{\pgfqpoint{0.667432in}{0.392818in}}%
\pgfpathlineto{\pgfqpoint{0.676576in}{0.393454in}}%
\pgfpathlineto{\pgfqpoint{0.678538in}{0.388462in}}%
\pgfpathlineto{\pgfqpoint{0.681745in}{0.388245in}}%
\pgfpathlineto{\pgfqpoint{0.687582in}{0.380602in}}%
\pgfpathlineto{\pgfqpoint{0.687593in}{0.383388in}}%
\pgfpathlineto{\pgfqpoint{0.685554in}{0.384294in}}%
\pgfpathlineto{\pgfqpoint{0.681822in}{0.393417in}}%
\pgfpathlineto{\pgfqpoint{0.683638in}{0.394657in}}%
\pgfpathlineto{\pgfqpoint{0.678654in}{0.399416in}}%
\pgfpathlineto{\pgfqpoint{0.674200in}{0.400343in}}%
\pgfpathlineto{\pgfqpoint{0.669956in}{0.398697in}}%
\pgfpathlineto{\pgfqpoint{0.662687in}{0.399979in}}%
\pgfpathlineto{\pgfqpoint{0.659240in}{0.397673in}}%
\pgfpathlineto{\pgfqpoint{0.651390in}{0.395892in}}%
\pgfpathlineto{\pgfqpoint{0.650672in}{0.393328in}}%
\pgfpathlineto{\pgfqpoint{0.644641in}{0.392629in}}%
\pgfpathlineto{\pgfqpoint{0.643093in}{0.389698in}}%
\pgfpathlineto{\pgfqpoint{0.639643in}{0.387551in}}%
\pgfpathlineto{\pgfqpoint{0.635476in}{0.387920in}}%
\pgfpathlineto{\pgfqpoint{0.633395in}{0.385844in}}%
\pgfpathlineto{\pgfqpoint{0.630770in}{0.385942in}}%
\pgfpathlineto{\pgfqpoint{0.628261in}{0.387810in}}%
\pgfpathlineto{\pgfqpoint{0.623875in}{0.387477in}}%
\pgfpathlineto{\pgfqpoint{0.622842in}{0.385754in}}%
\pgfpathlineto{\pgfqpoint{0.618235in}{0.387522in}}%
\pgfpathlineto{\pgfqpoint{0.614324in}{0.382960in}}%
\pgfpathlineto{\pgfqpoint{0.618074in}{0.378545in}}%
\pgfpathlineto{\pgfqpoint{0.619517in}{0.374914in}}%
\pgfpathlineto{\pgfqpoint{0.618187in}{0.371839in}}%
\pgfpathlineto{\pgfqpoint{0.612857in}{0.369717in}}%
\pgfpathlineto{\pgfqpoint{0.609770in}{0.371444in}}%
\pgfpathlineto{\pgfqpoint{0.605883in}{0.370509in}}%
\pgfpathlineto{\pgfqpoint{0.605415in}{0.367814in}}%
\pgfpathlineto{\pgfqpoint{0.600684in}{0.368768in}}%
\pgfpathlineto{\pgfqpoint{0.601732in}{0.366075in}}%
\pgfpathlineto{\pgfqpoint{0.594754in}{0.365507in}}%
\pgfpathlineto{\pgfqpoint{0.590237in}{0.368925in}}%
\pgfpathlineto{\pgfqpoint{0.587824in}{0.366751in}}%
\pgfpathlineto{\pgfqpoint{0.581404in}{0.369065in}}%
\pgfpathlineto{\pgfqpoint{0.579805in}{0.366722in}}%
\pgfpathlineto{\pgfqpoint{0.576622in}{0.365679in}}%
\pgfpathlineto{\pgfqpoint{0.574069in}{0.368427in}}%
\pgfpathlineto{\pgfqpoint{0.565338in}{0.367744in}}%
\pgfpathlineto{\pgfqpoint{0.565348in}{0.363690in}}%
\pgfpathlineto{\pgfqpoint{0.561840in}{0.362618in}}%
\pgfpathlineto{\pgfqpoint{0.559112in}{0.364701in}}%
\pgfpathlineto{\pgfqpoint{0.555505in}{0.363543in}}%
\pgfpathlineto{\pgfqpoint{0.550837in}{0.363268in}}%
\pgfpathlineto{\pgfqpoint{0.543883in}{0.366198in}}%
\pgfpathlineto{\pgfqpoint{0.539901in}{0.363176in}}%
\pgfpathlineto{\pgfqpoint{0.534205in}{0.367215in}}%
\pgfpathlineto{\pgfqpoint{0.532343in}{0.365685in}}%
\pgfpathlineto{\pgfqpoint{0.531036in}{0.361729in}}%
\pgfpathlineto{\pgfqpoint{0.527374in}{0.359410in}}%
\pgfpathlineto{\pgfqpoint{0.521404in}{0.362065in}}%
\pgfpathlineto{\pgfqpoint{0.511514in}{0.368623in}}%
\pgfpathlineto{\pgfqpoint{0.508631in}{0.369206in}}%
\pgfpathlineto{\pgfqpoint{0.507004in}{0.367839in}}%
\pgfpathlineto{\pgfqpoint{0.502654in}{0.368768in}}%
\pgfpathlineto{\pgfqpoint{0.500532in}{0.370809in}}%
\pgfpathlineto{\pgfqpoint{0.496187in}{0.371910in}}%
\pgfpathlineto{\pgfqpoint{0.490253in}{0.371973in}}%
\pgfpathlineto{\pgfqpoint{0.487897in}{0.374408in}}%
\pgfpathlineto{\pgfqpoint{0.492412in}{0.376928in}}%
\pgfpathlineto{\pgfqpoint{0.491358in}{0.379414in}}%
\pgfpathlineto{\pgfqpoint{0.488256in}{0.378856in}}%
\pgfpathlineto{\pgfqpoint{0.486706in}{0.376901in}}%
\pgfpathlineto{\pgfqpoint{0.479443in}{0.374097in}}%
\pgfpathlineto{\pgfqpoint{0.475771in}{0.375051in}}%
\pgfpathlineto{\pgfqpoint{0.474225in}{0.380743in}}%
\pgfpathlineto{\pgfqpoint{0.469644in}{0.378295in}}%
\pgfpathlineto{\pgfqpoint{0.468312in}{0.385509in}}%
\pgfpathlineto{\pgfqpoint{0.466141in}{0.383650in}}%
\pgfpathlineto{\pgfqpoint{0.466504in}{0.380874in}}%
\pgfpathlineto{\pgfqpoint{0.461512in}{0.381607in}}%
\pgfpathlineto{\pgfqpoint{0.466705in}{0.386444in}}%
\pgfpathlineto{\pgfqpoint{0.471968in}{0.383570in}}%
\pgfpathlineto{\pgfqpoint{0.473096in}{0.385013in}}%
\pgfpathlineto{\pgfqpoint{0.478632in}{0.383284in}}%
\pgfpathlineto{\pgfqpoint{0.478287in}{0.385253in}}%
\pgfpathlineto{\pgfqpoint{0.493391in}{0.385905in}}%
\pgfpathlineto{\pgfqpoint{0.500760in}{0.382833in}}%
\pgfpathlineto{\pgfqpoint{0.503612in}{0.379803in}}%
\pgfpathlineto{\pgfqpoint{0.502765in}{0.377157in}}%
\pgfpathlineto{\pgfqpoint{0.505629in}{0.376156in}}%
\pgfpathlineto{\pgfqpoint{0.512743in}{0.380331in}}%
\pgfpathlineto{\pgfqpoint{0.521946in}{0.380599in}}%
\pgfpathlineto{\pgfqpoint{0.530305in}{0.378082in}}%
\pgfpathlineto{\pgfqpoint{0.535040in}{0.378504in}}%
\pgfpathlineto{\pgfqpoint{0.536803in}{0.374600in}}%
\pgfpathlineto{\pgfqpoint{0.539451in}{0.378905in}}%
\pgfpathlineto{\pgfqpoint{0.541079in}{0.379837in}}%
\pgfpathlineto{\pgfqpoint{0.548613in}{0.381708in}}%
\pgfpathlineto{\pgfqpoint{0.551413in}{0.380564in}}%
\pgfpathlineto{\pgfqpoint{0.554441in}{0.381922in}}%
\pgfpathlineto{\pgfqpoint{0.558053in}{0.381212in}}%
\pgfpathlineto{\pgfqpoint{0.558906in}{0.382794in}}%
\pgfpathlineto{\pgfqpoint{0.566442in}{0.390052in}}%
\pgfpathlineto{\pgfqpoint{0.569634in}{0.391128in}}%
\pgfpathlineto{\pgfqpoint{0.571520in}{0.395046in}}%
\pgfpathlineto{\pgfqpoint{0.581530in}{0.398080in}}%
\pgfpathlineto{\pgfqpoint{0.582784in}{0.400394in}}%
\pgfpathlineto{\pgfqpoint{0.568596in}{0.403966in}}%
\pgfpathlineto{\pgfqpoint{0.569154in}{0.407808in}}%
\pgfpathlineto{\pgfqpoint{0.567696in}{0.412881in}}%
\pgfpathlineto{\pgfqpoint{0.564392in}{0.412460in}}%
\pgfpathlineto{\pgfqpoint{0.562144in}{0.405875in}}%
\pgfpathlineto{\pgfqpoint{0.559536in}{0.405333in}}%
\pgfpathlineto{\pgfqpoint{0.557838in}{0.407464in}}%
\pgfpathlineto{\pgfqpoint{0.559819in}{0.416788in}}%
\pgfpathlineto{\pgfqpoint{0.557846in}{0.420695in}}%
\pgfpathlineto{\pgfqpoint{0.554025in}{0.425301in}}%
\pgfpathlineto{\pgfqpoint{0.555862in}{0.428816in}}%
\pgfpathlineto{\pgfqpoint{0.547799in}{0.428942in}}%
\pgfpathlineto{\pgfqpoint{0.547982in}{0.430306in}}%
\pgfpathlineto{\pgfqpoint{0.543171in}{0.432216in}}%
\pgfpathclose%
\pgfusepath{fill}%
\end{pgfscope}%
\begin{pgfscope}%
\pgfpathrectangle{\pgfqpoint{0.100000in}{0.100000in}}{\pgfqpoint{2.857344in}{1.829167in}}%
\pgfusepath{clip}%
\pgfsetbuttcap%
\pgfsetmiterjoin%
\definecolor{currentfill}{rgb}{0.997155,0.911803,0.601077}%
\pgfsetfillcolor{currentfill}%
\pgfsetlinewidth{0.000000pt}%
\definecolor{currentstroke}{rgb}{0.000000,0.000000,0.000000}%
\pgfsetstrokecolor{currentstroke}%
\pgfsetstrokeopacity{0.000000}%
\pgfsetdash{}{0pt}%
\pgfpathmoveto{\pgfqpoint{0.527287in}{0.499074in}}%
\pgfpathlineto{\pgfqpoint{0.526290in}{0.497322in}}%
\pgfpathlineto{\pgfqpoint{0.528928in}{0.493591in}}%
\pgfpathlineto{\pgfqpoint{0.524758in}{0.490101in}}%
\pgfpathlineto{\pgfqpoint{0.524874in}{0.487099in}}%
\pgfpathlineto{\pgfqpoint{0.522989in}{0.486391in}}%
\pgfpathlineto{\pgfqpoint{0.516509in}{0.490867in}}%
\pgfpathlineto{\pgfqpoint{0.511890in}{0.500667in}}%
\pgfpathlineto{\pgfqpoint{0.511580in}{0.503498in}}%
\pgfpathlineto{\pgfqpoint{0.512768in}{0.506888in}}%
\pgfpathlineto{\pgfqpoint{0.517773in}{0.501703in}}%
\pgfpathlineto{\pgfqpoint{0.519304in}{0.502703in}}%
\pgfpathlineto{\pgfqpoint{0.523550in}{0.501752in}}%
\pgfpathclose%
\pgfusepath{fill}%
\end{pgfscope}%
\begin{pgfscope}%
\pgfpathrectangle{\pgfqpoint{0.100000in}{0.100000in}}{\pgfqpoint{2.857344in}{1.829167in}}%
\pgfusepath{clip}%
\pgfsetbuttcap%
\pgfsetmiterjoin%
\definecolor{currentfill}{rgb}{0.997155,0.911803,0.601077}%
\pgfsetfillcolor{currentfill}%
\pgfsetlinewidth{0.000000pt}%
\definecolor{currentstroke}{rgb}{0.000000,0.000000,0.000000}%
\pgfsetstrokecolor{currentstroke}%
\pgfsetstrokeopacity{0.000000}%
\pgfsetdash{}{0pt}%
\pgfpathmoveto{\pgfqpoint{0.450434in}{0.385546in}}%
\pgfpathlineto{\pgfqpoint{0.445880in}{0.384806in}}%
\pgfpathlineto{\pgfqpoint{0.442584in}{0.386330in}}%
\pgfpathlineto{\pgfqpoint{0.441143in}{0.388810in}}%
\pgfpathlineto{\pgfqpoint{0.442776in}{0.392368in}}%
\pgfpathlineto{\pgfqpoint{0.446412in}{0.391427in}}%
\pgfpathlineto{\pgfqpoint{0.452390in}{0.393538in}}%
\pgfpathlineto{\pgfqpoint{0.454484in}{0.390622in}}%
\pgfpathlineto{\pgfqpoint{0.456513in}{0.391160in}}%
\pgfpathlineto{\pgfqpoint{0.461427in}{0.389346in}}%
\pgfpathlineto{\pgfqpoint{0.463551in}{0.387041in}}%
\pgfpathlineto{\pgfqpoint{0.460960in}{0.381012in}}%
\pgfpathlineto{\pgfqpoint{0.458382in}{0.379734in}}%
\pgfpathlineto{\pgfqpoint{0.456066in}{0.380423in}}%
\pgfpathclose%
\pgfusepath{fill}%
\end{pgfscope}%
\begin{pgfscope}%
\pgfpathrectangle{\pgfqpoint{0.100000in}{0.100000in}}{\pgfqpoint{2.857344in}{1.829167in}}%
\pgfusepath{clip}%
\pgfsetbuttcap%
\pgfsetmiterjoin%
\definecolor{currentfill}{rgb}{0.997155,0.911803,0.601077}%
\pgfsetfillcolor{currentfill}%
\pgfsetlinewidth{0.000000pt}%
\definecolor{currentstroke}{rgb}{0.000000,0.000000,0.000000}%
\pgfsetstrokecolor{currentstroke}%
\pgfsetstrokeopacity{0.000000}%
\pgfsetdash{}{0pt}%
\pgfpathmoveto{\pgfqpoint{0.409711in}{0.391024in}}%
\pgfpathlineto{\pgfqpoint{0.403258in}{0.394102in}}%
\pgfpathlineto{\pgfqpoint{0.397286in}{0.395017in}}%
\pgfpathlineto{\pgfqpoint{0.395404in}{0.396328in}}%
\pgfpathlineto{\pgfqpoint{0.397547in}{0.398909in}}%
\pgfpathlineto{\pgfqpoint{0.401351in}{0.395685in}}%
\pgfpathlineto{\pgfqpoint{0.404541in}{0.396130in}}%
\pgfpathlineto{\pgfqpoint{0.409730in}{0.398362in}}%
\pgfpathlineto{\pgfqpoint{0.410149in}{0.401237in}}%
\pgfpathlineto{\pgfqpoint{0.416521in}{0.399814in}}%
\pgfpathlineto{\pgfqpoint{0.415029in}{0.395485in}}%
\pgfpathlineto{\pgfqpoint{0.419089in}{0.395169in}}%
\pgfpathlineto{\pgfqpoint{0.415683in}{0.390019in}}%
\pgfpathlineto{\pgfqpoint{0.412384in}{0.391688in}}%
\pgfpathclose%
\pgfusepath{fill}%
\end{pgfscope}%
\begin{pgfscope}%
\pgfpathrectangle{\pgfqpoint{0.100000in}{0.100000in}}{\pgfqpoint{2.857344in}{1.829167in}}%
\pgfusepath{clip}%
\pgfsetbuttcap%
\pgfsetmiterjoin%
\definecolor{currentfill}{rgb}{0.997155,0.911803,0.601077}%
\pgfsetfillcolor{currentfill}%
\pgfsetlinewidth{0.000000pt}%
\definecolor{currentstroke}{rgb}{0.000000,0.000000,0.000000}%
\pgfsetstrokecolor{currentstroke}%
\pgfsetstrokeopacity{0.000000}%
\pgfsetdash{}{0pt}%
\pgfpathmoveto{\pgfqpoint{0.398373in}{0.401666in}}%
\pgfpathlineto{\pgfqpoint{0.396001in}{0.400376in}}%
\pgfpathlineto{\pgfqpoint{0.389538in}{0.402358in}}%
\pgfpathlineto{\pgfqpoint{0.384709in}{0.401210in}}%
\pgfpathlineto{\pgfqpoint{0.382256in}{0.404472in}}%
\pgfpathlineto{\pgfqpoint{0.383326in}{0.405965in}}%
\pgfpathlineto{\pgfqpoint{0.386986in}{0.406215in}}%
\pgfpathlineto{\pgfqpoint{0.389785in}{0.405277in}}%
\pgfpathlineto{\pgfqpoint{0.392835in}{0.406761in}}%
\pgfpathlineto{\pgfqpoint{0.397512in}{0.404777in}}%
\pgfpathclose%
\pgfusepath{fill}%
\end{pgfscope}%
\begin{pgfscope}%
\pgfpathrectangle{\pgfqpoint{0.100000in}{0.100000in}}{\pgfqpoint{2.857344in}{1.829167in}}%
\pgfusepath{clip}%
\pgfsetbuttcap%
\pgfsetmiterjoin%
\definecolor{currentfill}{rgb}{0.997155,0.911803,0.601077}%
\pgfsetfillcolor{currentfill}%
\pgfsetlinewidth{0.000000pt}%
\definecolor{currentstroke}{rgb}{0.000000,0.000000,0.000000}%
\pgfsetstrokecolor{currentstroke}%
\pgfsetstrokeopacity{0.000000}%
\pgfsetdash{}{0pt}%
\pgfpathmoveto{\pgfqpoint{0.322189in}{0.449667in}}%
\pgfpathlineto{\pgfqpoint{0.322018in}{0.446469in}}%
\pgfpathlineto{\pgfqpoint{0.317300in}{0.444459in}}%
\pgfpathlineto{\pgfqpoint{0.312991in}{0.449490in}}%
\pgfpathlineto{\pgfqpoint{0.317823in}{0.451166in}}%
\pgfpathclose%
\pgfusepath{fill}%
\end{pgfscope}%
\begin{pgfscope}%
\pgfpathrectangle{\pgfqpoint{0.100000in}{0.100000in}}{\pgfqpoint{2.857344in}{1.829167in}}%
\pgfusepath{clip}%
\pgfsetbuttcap%
\pgfsetmiterjoin%
\definecolor{currentfill}{rgb}{0.997155,0.911803,0.601077}%
\pgfsetfillcolor{currentfill}%
\pgfsetlinewidth{0.000000pt}%
\definecolor{currentstroke}{rgb}{0.000000,0.000000,0.000000}%
\pgfsetstrokecolor{currentstroke}%
\pgfsetstrokeopacity{0.000000}%
\pgfsetdash{}{0pt}%
\pgfpathmoveto{\pgfqpoint{0.283276in}{0.467645in}}%
\pgfpathlineto{\pgfqpoint{0.284278in}{0.469036in}}%
\pgfpathlineto{\pgfqpoint{0.290065in}{0.468796in}}%
\pgfpathlineto{\pgfqpoint{0.291080in}{0.470864in}}%
\pgfpathlineto{\pgfqpoint{0.293620in}{0.469059in}}%
\pgfpathlineto{\pgfqpoint{0.292153in}{0.465586in}}%
\pgfpathlineto{\pgfqpoint{0.289997in}{0.465292in}}%
\pgfpathlineto{\pgfqpoint{0.284842in}{0.466382in}}%
\pgfpathclose%
\pgfusepath{fill}%
\end{pgfscope}%
\begin{pgfscope}%
\pgfpathrectangle{\pgfqpoint{0.100000in}{0.100000in}}{\pgfqpoint{2.857344in}{1.829167in}}%
\pgfusepath{clip}%
\pgfsetbuttcap%
\pgfsetmiterjoin%
\definecolor{currentfill}{rgb}{0.997155,0.911803,0.601077}%
\pgfsetfillcolor{currentfill}%
\pgfsetlinewidth{0.000000pt}%
\definecolor{currentstroke}{rgb}{0.000000,0.000000,0.000000}%
\pgfsetstrokecolor{currentstroke}%
\pgfsetstrokeopacity{0.000000}%
\pgfsetdash{}{0pt}%
\pgfpathmoveto{\pgfqpoint{0.275069in}{0.478409in}}%
\pgfpathlineto{\pgfqpoint{0.274718in}{0.481541in}}%
\pgfpathlineto{\pgfqpoint{0.277738in}{0.481309in}}%
\pgfpathlineto{\pgfqpoint{0.277527in}{0.485708in}}%
\pgfpathlineto{\pgfqpoint{0.280565in}{0.483334in}}%
\pgfpathlineto{\pgfqpoint{0.279514in}{0.479903in}}%
\pgfpathclose%
\pgfusepath{fill}%
\end{pgfscope}%
\begin{pgfscope}%
\pgfpathrectangle{\pgfqpoint{0.100000in}{0.100000in}}{\pgfqpoint{2.857344in}{1.829167in}}%
\pgfusepath{clip}%
\pgfsetbuttcap%
\pgfsetmiterjoin%
\definecolor{currentfill}{rgb}{0.997155,0.911803,0.601077}%
\pgfsetfillcolor{currentfill}%
\pgfsetlinewidth{0.000000pt}%
\definecolor{currentstroke}{rgb}{0.000000,0.000000,0.000000}%
\pgfsetstrokecolor{currentstroke}%
\pgfsetstrokeopacity{0.000000}%
\pgfsetdash{}{0pt}%
\pgfpathmoveto{\pgfqpoint{0.541109in}{0.590049in}}%
\pgfpathlineto{\pgfqpoint{0.536293in}{0.591089in}}%
\pgfpathlineto{\pgfqpoint{0.535258in}{0.594307in}}%
\pgfpathlineto{\pgfqpoint{0.536914in}{0.597524in}}%
\pgfpathlineto{\pgfqpoint{0.540621in}{0.599205in}}%
\pgfpathlineto{\pgfqpoint{0.544865in}{0.590946in}}%
\pgfpathlineto{\pgfqpoint{0.548715in}{0.590580in}}%
\pgfpathlineto{\pgfqpoint{0.551587in}{0.588031in}}%
\pgfpathlineto{\pgfqpoint{0.551531in}{0.584525in}}%
\pgfpathlineto{\pgfqpoint{0.549431in}{0.582844in}}%
\pgfpathlineto{\pgfqpoint{0.552846in}{0.572475in}}%
\pgfpathlineto{\pgfqpoint{0.556583in}{0.568669in}}%
\pgfpathlineto{\pgfqpoint{0.552762in}{0.567462in}}%
\pgfpathlineto{\pgfqpoint{0.549670in}{0.571731in}}%
\pgfpathlineto{\pgfqpoint{0.542949in}{0.570410in}}%
\pgfpathlineto{\pgfqpoint{0.545039in}{0.574635in}}%
\pgfpathlineto{\pgfqpoint{0.544066in}{0.576318in}}%
\pgfpathlineto{\pgfqpoint{0.544636in}{0.580722in}}%
\pgfpathlineto{\pgfqpoint{0.543304in}{0.588683in}}%
\pgfpathclose%
\pgfusepath{fill}%
\end{pgfscope}%
\begin{pgfscope}%
\pgfpathrectangle{\pgfqpoint{0.100000in}{0.100000in}}{\pgfqpoint{2.857344in}{1.829167in}}%
\pgfusepath{clip}%
\pgfsetbuttcap%
\pgfsetmiterjoin%
\definecolor{currentfill}{rgb}{0.997155,0.911803,0.601077}%
\pgfsetfillcolor{currentfill}%
\pgfsetlinewidth{0.000000pt}%
\definecolor{currentstroke}{rgb}{0.000000,0.000000,0.000000}%
\pgfsetstrokecolor{currentstroke}%
\pgfsetstrokeopacity{0.000000}%
\pgfsetdash{}{0pt}%
\pgfpathmoveto{\pgfqpoint{0.710314in}{0.352834in}}%
\pgfpathlineto{\pgfqpoint{0.703332in}{0.352913in}}%
\pgfpathlineto{\pgfqpoint{0.703870in}{0.356418in}}%
\pgfpathlineto{\pgfqpoint{0.706665in}{0.357671in}}%
\pgfpathclose%
\pgfusepath{fill}%
\end{pgfscope}%
\begin{pgfscope}%
\pgfpathrectangle{\pgfqpoint{0.100000in}{0.100000in}}{\pgfqpoint{2.857344in}{1.829167in}}%
\pgfusepath{clip}%
\pgfsetbuttcap%
\pgfsetmiterjoin%
\definecolor{currentfill}{rgb}{0.997155,0.911803,0.601077}%
\pgfsetfillcolor{currentfill}%
\pgfsetlinewidth{0.000000pt}%
\definecolor{currentstroke}{rgb}{0.000000,0.000000,0.000000}%
\pgfsetstrokecolor{currentstroke}%
\pgfsetstrokeopacity{0.000000}%
\pgfsetdash{}{0pt}%
\pgfpathmoveto{\pgfqpoint{0.700626in}{0.356008in}}%
\pgfpathlineto{\pgfqpoint{0.695624in}{0.354798in}}%
\pgfpathlineto{\pgfqpoint{0.690871in}{0.352726in}}%
\pgfpathlineto{\pgfqpoint{0.689404in}{0.355168in}}%
\pgfpathlineto{\pgfqpoint{0.697577in}{0.356821in}}%
\pgfpathclose%
\pgfusepath{fill}%
\end{pgfscope}%
\begin{pgfscope}%
\pgfpathrectangle{\pgfqpoint{0.100000in}{0.100000in}}{\pgfqpoint{2.857344in}{1.829167in}}%
\pgfusepath{clip}%
\pgfsetbuttcap%
\pgfsetmiterjoin%
\definecolor{currentfill}{rgb}{0.997155,0.911803,0.601077}%
\pgfsetfillcolor{currentfill}%
\pgfsetlinewidth{0.000000pt}%
\definecolor{currentstroke}{rgb}{0.000000,0.000000,0.000000}%
\pgfsetstrokecolor{currentstroke}%
\pgfsetstrokeopacity{0.000000}%
\pgfsetdash{}{0pt}%
\pgfpathmoveto{\pgfqpoint{0.623956in}{0.353527in}}%
\pgfpathlineto{\pgfqpoint{0.624147in}{0.351104in}}%
\pgfpathlineto{\pgfqpoint{0.621265in}{0.349597in}}%
\pgfpathlineto{\pgfqpoint{0.613222in}{0.353052in}}%
\pgfpathlineto{\pgfqpoint{0.611975in}{0.351662in}}%
\pgfpathlineto{\pgfqpoint{0.610029in}{0.358237in}}%
\pgfpathlineto{\pgfqpoint{0.614059in}{0.359152in}}%
\pgfpathlineto{\pgfqpoint{0.615921in}{0.356450in}}%
\pgfpathlineto{\pgfqpoint{0.619966in}{0.359832in}}%
\pgfpathclose%
\pgfusepath{fill}%
\end{pgfscope}%
\begin{pgfscope}%
\pgfpathrectangle{\pgfqpoint{0.100000in}{0.100000in}}{\pgfqpoint{2.857344in}{1.829167in}}%
\pgfusepath{clip}%
\pgfsetbuttcap%
\pgfsetmiterjoin%
\definecolor{currentfill}{rgb}{0.997155,0.911803,0.601077}%
\pgfsetfillcolor{currentfill}%
\pgfsetlinewidth{0.000000pt}%
\definecolor{currentstroke}{rgb}{0.000000,0.000000,0.000000}%
\pgfsetstrokecolor{currentstroke}%
\pgfsetstrokeopacity{0.000000}%
\pgfsetdash{}{0pt}%
\pgfpathmoveto{\pgfqpoint{0.805382in}{0.219176in}}%
\pgfpathlineto{\pgfqpoint{0.798637in}{0.216665in}}%
\pgfpathlineto{\pgfqpoint{0.795065in}{0.216515in}}%
\pgfpathlineto{\pgfqpoint{0.797973in}{0.221842in}}%
\pgfpathlineto{\pgfqpoint{0.800457in}{0.223828in}}%
\pgfpathlineto{\pgfqpoint{0.800377in}{0.229345in}}%
\pgfpathlineto{\pgfqpoint{0.803077in}{0.239303in}}%
\pgfpathlineto{\pgfqpoint{0.805564in}{0.241194in}}%
\pgfpathlineto{\pgfqpoint{0.811306in}{0.238456in}}%
\pgfpathlineto{\pgfqpoint{0.810467in}{0.236887in}}%
\pgfpathlineto{\pgfqpoint{0.810902in}{0.229349in}}%
\pgfpathlineto{\pgfqpoint{0.809041in}{0.227314in}}%
\pgfpathlineto{\pgfqpoint{0.808241in}{0.221960in}}%
\pgfpathclose%
\pgfusepath{fill}%
\end{pgfscope}%
\begin{pgfscope}%
\pgfpathrectangle{\pgfqpoint{0.100000in}{0.100000in}}{\pgfqpoint{2.857344in}{1.829167in}}%
\pgfusepath{clip}%
\pgfsetbuttcap%
\pgfsetmiterjoin%
\definecolor{currentfill}{rgb}{0.997155,0.911803,0.601077}%
\pgfsetfillcolor{currentfill}%
\pgfsetlinewidth{0.000000pt}%
\definecolor{currentstroke}{rgb}{0.000000,0.000000,0.000000}%
\pgfsetstrokecolor{currentstroke}%
\pgfsetstrokeopacity{0.000000}%
\pgfsetdash{}{0pt}%
\pgfpathmoveto{\pgfqpoint{0.790446in}{0.243862in}}%
\pgfpathlineto{\pgfqpoint{0.794397in}{0.237040in}}%
\pgfpathlineto{\pgfqpoint{0.793665in}{0.235412in}}%
\pgfpathlineto{\pgfqpoint{0.798949in}{0.233757in}}%
\pgfpathlineto{\pgfqpoint{0.797432in}{0.227487in}}%
\pgfpathlineto{\pgfqpoint{0.795131in}{0.227776in}}%
\pgfpathlineto{\pgfqpoint{0.789085in}{0.238709in}}%
\pgfpathlineto{\pgfqpoint{0.789945in}{0.233788in}}%
\pgfpathlineto{\pgfqpoint{0.788915in}{0.230955in}}%
\pgfpathlineto{\pgfqpoint{0.786354in}{0.229694in}}%
\pgfpathlineto{\pgfqpoint{0.784899in}{0.231052in}}%
\pgfpathlineto{\pgfqpoint{0.784086in}{0.237352in}}%
\pgfpathlineto{\pgfqpoint{0.784566in}{0.241438in}}%
\pgfpathlineto{\pgfqpoint{0.782860in}{0.243221in}}%
\pgfpathlineto{\pgfqpoint{0.787365in}{0.248382in}}%
\pgfpathlineto{\pgfqpoint{0.790321in}{0.249913in}}%
\pgfpathlineto{\pgfqpoint{0.791305in}{0.247901in}}%
\pgfpathlineto{\pgfqpoint{0.794389in}{0.249458in}}%
\pgfpathlineto{\pgfqpoint{0.796623in}{0.245229in}}%
\pgfpathlineto{\pgfqpoint{0.801456in}{0.239850in}}%
\pgfpathlineto{\pgfqpoint{0.800796in}{0.237413in}}%
\pgfpathlineto{\pgfqpoint{0.798113in}{0.234758in}}%
\pgfpathlineto{\pgfqpoint{0.795632in}{0.236009in}}%
\pgfpathclose%
\pgfusepath{fill}%
\end{pgfscope}%
\begin{pgfscope}%
\pgfpathrectangle{\pgfqpoint{0.100000in}{0.100000in}}{\pgfqpoint{2.857344in}{1.829167in}}%
\pgfusepath{clip}%
\pgfsetbuttcap%
\pgfsetmiterjoin%
\definecolor{currentfill}{rgb}{0.997155,0.911803,0.601077}%
\pgfsetfillcolor{currentfill}%
\pgfsetlinewidth{0.000000pt}%
\definecolor{currentstroke}{rgb}{0.000000,0.000000,0.000000}%
\pgfsetstrokecolor{currentstroke}%
\pgfsetstrokeopacity{0.000000}%
\pgfsetdash{}{0pt}%
\pgfpathmoveto{\pgfqpoint{0.594377in}{0.340596in}}%
\pgfpathlineto{\pgfqpoint{0.589482in}{0.339959in}}%
\pgfpathlineto{\pgfqpoint{0.582589in}{0.337310in}}%
\pgfpathlineto{\pgfqpoint{0.580982in}{0.338749in}}%
\pgfpathlineto{\pgfqpoint{0.588182in}{0.340376in}}%
\pgfpathlineto{\pgfqpoint{0.586252in}{0.341796in}}%
\pgfpathlineto{\pgfqpoint{0.581787in}{0.342836in}}%
\pgfpathlineto{\pgfqpoint{0.580596in}{0.346163in}}%
\pgfpathlineto{\pgfqpoint{0.582842in}{0.349029in}}%
\pgfpathlineto{\pgfqpoint{0.584734in}{0.355607in}}%
\pgfpathlineto{\pgfqpoint{0.591643in}{0.356832in}}%
\pgfpathlineto{\pgfqpoint{0.594989in}{0.353991in}}%
\pgfpathlineto{\pgfqpoint{0.598131in}{0.356741in}}%
\pgfpathlineto{\pgfqpoint{0.602073in}{0.356066in}}%
\pgfpathlineto{\pgfqpoint{0.604905in}{0.350986in}}%
\pgfpathlineto{\pgfqpoint{0.607304in}{0.355502in}}%
\pgfpathlineto{\pgfqpoint{0.614111in}{0.344940in}}%
\pgfpathlineto{\pgfqpoint{0.611652in}{0.344095in}}%
\pgfpathlineto{\pgfqpoint{0.610143in}{0.341806in}}%
\pgfpathlineto{\pgfqpoint{0.612994in}{0.339716in}}%
\pgfpathlineto{\pgfqpoint{0.608572in}{0.337758in}}%
\pgfpathlineto{\pgfqpoint{0.605366in}{0.338836in}}%
\pgfpathlineto{\pgfqpoint{0.599310in}{0.342411in}}%
\pgfpathclose%
\pgfusepath{fill}%
\end{pgfscope}%
\begin{pgfscope}%
\pgfpathrectangle{\pgfqpoint{0.100000in}{0.100000in}}{\pgfqpoint{2.857344in}{1.829167in}}%
\pgfusepath{clip}%
\pgfsetbuttcap%
\pgfsetmiterjoin%
\definecolor{currentfill}{rgb}{0.997155,0.911803,0.601077}%
\pgfsetfillcolor{currentfill}%
\pgfsetlinewidth{0.000000pt}%
\definecolor{currentstroke}{rgb}{0.000000,0.000000,0.000000}%
\pgfsetstrokecolor{currentstroke}%
\pgfsetstrokeopacity{0.000000}%
\pgfsetdash{}{0pt}%
\pgfpathmoveto{\pgfqpoint{0.786336in}{0.199025in}}%
\pgfpathlineto{\pgfqpoint{0.785366in}{0.208779in}}%
\pgfpathlineto{\pgfqpoint{0.787982in}{0.212361in}}%
\pgfpathlineto{\pgfqpoint{0.785185in}{0.212610in}}%
\pgfpathlineto{\pgfqpoint{0.786207in}{0.219576in}}%
\pgfpathlineto{\pgfqpoint{0.787412in}{0.223711in}}%
\pgfpathlineto{\pgfqpoint{0.786365in}{0.229267in}}%
\pgfpathlineto{\pgfqpoint{0.788927in}{0.230342in}}%
\pgfpathlineto{\pgfqpoint{0.789763in}{0.231779in}}%
\pgfpathlineto{\pgfqpoint{0.792286in}{0.231234in}}%
\pgfpathlineto{\pgfqpoint{0.794471in}{0.226783in}}%
\pgfpathlineto{\pgfqpoint{0.794877in}{0.222705in}}%
\pgfpathlineto{\pgfqpoint{0.792204in}{0.210491in}}%
\pgfpathclose%
\pgfusepath{fill}%
\end{pgfscope}%
\begin{pgfscope}%
\pgfpathrectangle{\pgfqpoint{0.100000in}{0.100000in}}{\pgfqpoint{2.857344in}{1.829167in}}%
\pgfusepath{clip}%
\pgfsetbuttcap%
\pgfsetmiterjoin%
\definecolor{currentfill}{rgb}{0.997155,0.911803,0.601077}%
\pgfsetfillcolor{currentfill}%
\pgfsetlinewidth{0.000000pt}%
\definecolor{currentstroke}{rgb}{0.000000,0.000000,0.000000}%
\pgfsetstrokecolor{currentstroke}%
\pgfsetstrokeopacity{0.000000}%
\pgfsetdash{}{0pt}%
\pgfpathmoveto{\pgfqpoint{0.785756in}{0.228801in}}%
\pgfpathlineto{\pgfqpoint{0.785622in}{0.224045in}}%
\pgfpathlineto{\pgfqpoint{0.783428in}{0.221886in}}%
\pgfpathlineto{\pgfqpoint{0.780893in}{0.222684in}}%
\pgfpathlineto{\pgfqpoint{0.784098in}{0.229592in}}%
\pgfpathclose%
\pgfusepath{fill}%
\end{pgfscope}%
\begin{pgfscope}%
\pgfpathrectangle{\pgfqpoint{0.100000in}{0.100000in}}{\pgfqpoint{2.857344in}{1.829167in}}%
\pgfusepath{clip}%
\pgfsetbuttcap%
\pgfsetmiterjoin%
\definecolor{currentfill}{rgb}{0.997155,0.911803,0.601077}%
\pgfsetfillcolor{currentfill}%
\pgfsetlinewidth{0.000000pt}%
\definecolor{currentstroke}{rgb}{0.000000,0.000000,0.000000}%
\pgfsetstrokecolor{currentstroke}%
\pgfsetstrokeopacity{0.000000}%
\pgfsetdash{}{0pt}%
\pgfpathmoveto{\pgfqpoint{0.812669in}{0.207686in}}%
\pgfpathlineto{\pgfqpoint{0.810799in}{0.199726in}}%
\pgfpathlineto{\pgfqpoint{0.806441in}{0.197210in}}%
\pgfpathlineto{\pgfqpoint{0.800723in}{0.199528in}}%
\pgfpathlineto{\pgfqpoint{0.802299in}{0.202621in}}%
\pgfpathlineto{\pgfqpoint{0.802267in}{0.204763in}}%
\pgfpathlineto{\pgfqpoint{0.804392in}{0.208402in}}%
\pgfpathlineto{\pgfqpoint{0.802898in}{0.207670in}}%
\pgfpathlineto{\pgfqpoint{0.803987in}{0.209447in}}%
\pgfpathlineto{\pgfqpoint{0.801918in}{0.213485in}}%
\pgfpathlineto{\pgfqpoint{0.804267in}{0.214229in}}%
\pgfpathlineto{\pgfqpoint{0.809703in}{0.209504in}}%
\pgfpathclose%
\pgfusepath{fill}%
\end{pgfscope}%
\begin{pgfscope}%
\pgfpathrectangle{\pgfqpoint{0.100000in}{0.100000in}}{\pgfqpoint{2.857344in}{1.829167in}}%
\pgfusepath{clip}%
\pgfsetbuttcap%
\pgfsetmiterjoin%
\definecolor{currentfill}{rgb}{0.997155,0.911803,0.601077}%
\pgfsetfillcolor{currentfill}%
\pgfsetlinewidth{0.000000pt}%
\definecolor{currentstroke}{rgb}{0.000000,0.000000,0.000000}%
\pgfsetstrokecolor{currentstroke}%
\pgfsetstrokeopacity{0.000000}%
\pgfsetdash{}{0pt}%
\pgfpathmoveto{\pgfqpoint{0.793592in}{0.193612in}}%
\pgfpathlineto{\pgfqpoint{0.791127in}{0.192653in}}%
\pgfpathlineto{\pgfqpoint{0.792535in}{0.201052in}}%
\pgfpathlineto{\pgfqpoint{0.795493in}{0.203012in}}%
\pgfpathlineto{\pgfqpoint{0.794772in}{0.209306in}}%
\pgfpathlineto{\pgfqpoint{0.795974in}{0.212096in}}%
\pgfpathlineto{\pgfqpoint{0.798352in}{0.213142in}}%
\pgfpathlineto{\pgfqpoint{0.800848in}{0.210404in}}%
\pgfpathlineto{\pgfqpoint{0.802991in}{0.206690in}}%
\pgfpathlineto{\pgfqpoint{0.796104in}{0.199413in}}%
\pgfpathclose%
\pgfusepath{fill}%
\end{pgfscope}%
\begin{pgfscope}%
\pgfpathrectangle{\pgfqpoint{0.100000in}{0.100000in}}{\pgfqpoint{2.857344in}{1.829167in}}%
\pgfusepath{clip}%
\pgfsetbuttcap%
\pgfsetmiterjoin%
\definecolor{currentfill}{rgb}{0.997155,0.911803,0.601077}%
\pgfsetfillcolor{currentfill}%
\pgfsetlinewidth{0.000000pt}%
\definecolor{currentstroke}{rgb}{0.000000,0.000000,0.000000}%
\pgfsetstrokecolor{currentstroke}%
\pgfsetstrokeopacity{0.000000}%
\pgfsetdash{}{0pt}%
\pgfpathmoveto{\pgfqpoint{0.813762in}{0.202366in}}%
\pgfpathlineto{\pgfqpoint{0.814970in}{0.196570in}}%
\pgfpathlineto{\pgfqpoint{0.809573in}{0.197233in}}%
\pgfpathclose%
\pgfusepath{fill}%
\end{pgfscope}%
\begin{pgfscope}%
\pgfpathrectangle{\pgfqpoint{0.100000in}{0.100000in}}{\pgfqpoint{2.857344in}{1.829167in}}%
\pgfusepath{clip}%
\pgfsetbuttcap%
\pgfsetmiterjoin%
\definecolor{currentfill}{rgb}{0.997155,0.911803,0.601077}%
\pgfsetfillcolor{currentfill}%
\pgfsetlinewidth{0.000000pt}%
\definecolor{currentstroke}{rgb}{0.000000,0.000000,0.000000}%
\pgfsetstrokecolor{currentstroke}%
\pgfsetstrokeopacity{0.000000}%
\pgfsetdash{}{0pt}%
\pgfpathmoveto{\pgfqpoint{0.816338in}{0.193510in}}%
\pgfpathlineto{\pgfqpoint{0.817222in}{0.189934in}}%
\pgfpathlineto{\pgfqpoint{0.818915in}{0.188790in}}%
\pgfpathlineto{\pgfqpoint{0.818608in}{0.185425in}}%
\pgfpathlineto{\pgfqpoint{0.816242in}{0.184293in}}%
\pgfpathlineto{\pgfqpoint{0.814322in}{0.189317in}}%
\pgfpathclose%
\pgfusepath{fill}%
\end{pgfscope}%
\begin{pgfscope}%
\pgfpathrectangle{\pgfqpoint{0.100000in}{0.100000in}}{\pgfqpoint{2.857344in}{1.829167in}}%
\pgfusepath{clip}%
\pgfsetbuttcap%
\pgfsetmiterjoin%
\definecolor{currentfill}{rgb}{0.997155,0.911803,0.601077}%
\pgfsetfillcolor{currentfill}%
\pgfsetlinewidth{0.000000pt}%
\definecolor{currentstroke}{rgb}{0.000000,0.000000,0.000000}%
\pgfsetstrokecolor{currentstroke}%
\pgfsetstrokeopacity{0.000000}%
\pgfsetdash{}{0pt}%
\pgfpathmoveto{\pgfqpoint{0.811834in}{0.194881in}}%
\pgfpathlineto{\pgfqpoint{0.810366in}{0.190728in}}%
\pgfpathlineto{\pgfqpoint{0.808086in}{0.190880in}}%
\pgfpathlineto{\pgfqpoint{0.806648in}{0.194305in}}%
\pgfpathlineto{\pgfqpoint{0.808990in}{0.195928in}}%
\pgfpathclose%
\pgfusepath{fill}%
\end{pgfscope}%
\begin{pgfscope}%
\pgfpathrectangle{\pgfqpoint{0.100000in}{0.100000in}}{\pgfqpoint{2.857344in}{1.829167in}}%
\pgfusepath{clip}%
\pgfsetbuttcap%
\pgfsetmiterjoin%
\definecolor{currentfill}{rgb}{0.997155,0.911803,0.601077}%
\pgfsetfillcolor{currentfill}%
\pgfsetlinewidth{0.000000pt}%
\definecolor{currentstroke}{rgb}{0.000000,0.000000,0.000000}%
\pgfsetstrokecolor{currentstroke}%
\pgfsetstrokeopacity{0.000000}%
\pgfsetdash{}{0pt}%
\pgfpathmoveto{\pgfqpoint{0.805683in}{0.173468in}}%
\pgfpathlineto{\pgfqpoint{0.807717in}{0.171243in}}%
\pgfpathlineto{\pgfqpoint{0.808165in}{0.166555in}}%
\pgfpathlineto{\pgfqpoint{0.809236in}{0.165192in}}%
\pgfpathlineto{\pgfqpoint{0.807487in}{0.160949in}}%
\pgfpathlineto{\pgfqpoint{0.805186in}{0.158561in}}%
\pgfpathlineto{\pgfqpoint{0.803962in}{0.154129in}}%
\pgfpathlineto{\pgfqpoint{0.801447in}{0.161035in}}%
\pgfpathlineto{\pgfqpoint{0.801030in}{0.167306in}}%
\pgfpathlineto{\pgfqpoint{0.799622in}{0.170428in}}%
\pgfpathlineto{\pgfqpoint{0.794776in}{0.172958in}}%
\pgfpathlineto{\pgfqpoint{0.799322in}{0.178236in}}%
\pgfpathlineto{\pgfqpoint{0.796291in}{0.180491in}}%
\pgfpathlineto{\pgfqpoint{0.798961in}{0.182739in}}%
\pgfpathlineto{\pgfqpoint{0.802506in}{0.191368in}}%
\pgfpathlineto{\pgfqpoint{0.798832in}{0.194411in}}%
\pgfpathlineto{\pgfqpoint{0.800247in}{0.197332in}}%
\pgfpathlineto{\pgfqpoint{0.804997in}{0.194741in}}%
\pgfpathlineto{\pgfqpoint{0.805478in}{0.190652in}}%
\pgfpathlineto{\pgfqpoint{0.803681in}{0.188323in}}%
\pgfpathlineto{\pgfqpoint{0.807242in}{0.185173in}}%
\pgfpathlineto{\pgfqpoint{0.808363in}{0.180266in}}%
\pgfpathlineto{\pgfqpoint{0.808211in}{0.173383in}}%
\pgfpathclose%
\pgfusepath{fill}%
\end{pgfscope}%
\begin{pgfscope}%
\pgfpathrectangle{\pgfqpoint{0.100000in}{0.100000in}}{\pgfqpoint{2.857344in}{1.829167in}}%
\pgfusepath{clip}%
\pgfsetbuttcap%
\pgfsetmiterjoin%
\definecolor{currentfill}{rgb}{0.997155,0.911803,0.601077}%
\pgfsetfillcolor{currentfill}%
\pgfsetlinewidth{0.000000pt}%
\definecolor{currentstroke}{rgb}{0.000000,0.000000,0.000000}%
\pgfsetstrokecolor{currentstroke}%
\pgfsetstrokeopacity{0.000000}%
\pgfsetdash{}{0pt}%
\pgfpathmoveto{\pgfqpoint{0.814379in}{0.190744in}}%
\pgfpathlineto{\pgfqpoint{0.813271in}{0.188357in}}%
\pgfpathlineto{\pgfqpoint{0.814311in}{0.184481in}}%
\pgfpathlineto{\pgfqpoint{0.811648in}{0.184155in}}%
\pgfpathlineto{\pgfqpoint{0.808856in}{0.185950in}}%
\pgfpathlineto{\pgfqpoint{0.809762in}{0.189903in}}%
\pgfpathclose%
\pgfusepath{fill}%
\end{pgfscope}%
\begin{pgfscope}%
\pgfpathrectangle{\pgfqpoint{0.100000in}{0.100000in}}{\pgfqpoint{2.857344in}{1.829167in}}%
\pgfusepath{clip}%
\pgfsetbuttcap%
\pgfsetmiterjoin%
\definecolor{currentfill}{rgb}{0.997155,0.911803,0.601077}%
\pgfsetfillcolor{currentfill}%
\pgfsetlinewidth{0.000000pt}%
\definecolor{currentstroke}{rgb}{0.000000,0.000000,0.000000}%
\pgfsetstrokecolor{currentstroke}%
\pgfsetstrokeopacity{0.000000}%
\pgfsetdash{}{0pt}%
\pgfpathmoveto{\pgfqpoint{0.801966in}{0.191015in}}%
\pgfpathlineto{\pgfqpoint{0.796087in}{0.188378in}}%
\pgfpathlineto{\pgfqpoint{0.794089in}{0.189327in}}%
\pgfpathlineto{\pgfqpoint{0.798098in}{0.192805in}}%
\pgfpathclose%
\pgfusepath{fill}%
\end{pgfscope}%
\begin{pgfscope}%
\pgfpathrectangle{\pgfqpoint{0.100000in}{0.100000in}}{\pgfqpoint{2.857344in}{1.829167in}}%
\pgfusepath{clip}%
\pgfsetbuttcap%
\pgfsetmiterjoin%
\definecolor{currentfill}{rgb}{0.997155,0.911803,0.601077}%
\pgfsetfillcolor{currentfill}%
\pgfsetlinewidth{0.000000pt}%
\definecolor{currentstroke}{rgb}{0.000000,0.000000,0.000000}%
\pgfsetstrokecolor{currentstroke}%
\pgfsetstrokeopacity{0.000000}%
\pgfsetdash{}{0pt}%
\pgfpathmoveto{\pgfqpoint{0.814443in}{0.165293in}}%
\pgfpathlineto{\pgfqpoint{0.813042in}{0.169263in}}%
\pgfpathlineto{\pgfqpoint{0.816006in}{0.170240in}}%
\pgfpathlineto{\pgfqpoint{0.816903in}{0.174521in}}%
\pgfpathlineto{\pgfqpoint{0.820116in}{0.174526in}}%
\pgfpathlineto{\pgfqpoint{0.820102in}{0.177507in}}%
\pgfpathlineto{\pgfqpoint{0.824567in}{0.176802in}}%
\pgfpathlineto{\pgfqpoint{0.825328in}{0.168720in}}%
\pgfpathlineto{\pgfqpoint{0.822738in}{0.163568in}}%
\pgfpathlineto{\pgfqpoint{0.818896in}{0.160350in}}%
\pgfpathclose%
\pgfusepath{fill}%
\end{pgfscope}%
\begin{pgfscope}%
\pgfpathrectangle{\pgfqpoint{0.100000in}{0.100000in}}{\pgfqpoint{2.857344in}{1.829167in}}%
\pgfusepath{clip}%
\pgfsetbuttcap%
\pgfsetmiterjoin%
\definecolor{currentfill}{rgb}{0.997155,0.911803,0.601077}%
\pgfsetfillcolor{currentfill}%
\pgfsetlinewidth{0.000000pt}%
\definecolor{currentstroke}{rgb}{0.000000,0.000000,0.000000}%
\pgfsetstrokecolor{currentstroke}%
\pgfsetstrokeopacity{0.000000}%
\pgfsetdash{}{0pt}%
\pgfpathmoveto{\pgfqpoint{0.812707in}{0.168347in}}%
\pgfpathlineto{\pgfqpoint{0.813973in}{0.164588in}}%
\pgfpathlineto{\pgfqpoint{0.810852in}{0.164131in}}%
\pgfpathclose%
\pgfusepath{fill}%
\end{pgfscope}%
\begin{pgfscope}%
\pgfpathrectangle{\pgfqpoint{0.100000in}{0.100000in}}{\pgfqpoint{2.857344in}{1.829167in}}%
\pgfusepath{clip}%
\pgfsetbuttcap%
\pgfsetmiterjoin%
\definecolor{currentfill}{rgb}{0.997155,0.911803,0.601077}%
\pgfsetfillcolor{currentfill}%
\pgfsetlinewidth{0.000000pt}%
\definecolor{currentstroke}{rgb}{0.000000,0.000000,0.000000}%
\pgfsetstrokecolor{currentstroke}%
\pgfsetstrokeopacity{0.000000}%
\pgfsetdash{}{0pt}%
\pgfpathmoveto{\pgfqpoint{0.815339in}{0.162818in}}%
\pgfpathlineto{\pgfqpoint{0.814904in}{0.158026in}}%
\pgfpathlineto{\pgfqpoint{0.811611in}{0.158423in}}%
\pgfpathlineto{\pgfqpoint{0.812676in}{0.160738in}}%
\pgfpathclose%
\pgfusepath{fill}%
\end{pgfscope}%
\begin{pgfscope}%
\definecolor{textcolor}{rgb}{0.000000,0.000000,0.000000}%
\pgfsetstrokecolor{textcolor}%
\pgfsetfillcolor{textcolor}%
\pgftext[x=1.014350in,y=1.910875in,left,base]{\color{textcolor}\setmainfont{Lato}\rmfamily\fontsize{9.000000}{10.800000}\selectfont Year ending 2020 Q4*}%
\end{pgfscope}%
\begin{pgfscope}%
\pgfpathrectangle{\pgfqpoint{3.525000in}{0.100000in}}{\pgfqpoint{2.857344in}{1.829167in}}%
\pgfusepath{clip}%
\pgfsetbuttcap%
\pgfsetmiterjoin%
\definecolor{currentfill}{rgb}{0.802153,0.920185,0.616378}%
\pgfsetfillcolor{currentfill}%
\pgfsetlinewidth{0.000000pt}%
\definecolor{currentstroke}{rgb}{0.000000,0.000000,0.000000}%
\pgfsetstrokecolor{currentstroke}%
\pgfsetstrokeopacity{0.000000}%
\pgfsetdash{}{0pt}%
\pgfpathmoveto{\pgfqpoint{4.478726in}{0.528908in}}%
\pgfpathlineto{\pgfqpoint{4.470514in}{0.529042in}}%
\pgfpathlineto{\pgfqpoint{4.464924in}{0.535229in}}%
\pgfpathlineto{\pgfqpoint{4.461123in}{0.551658in}}%
\pgfpathlineto{\pgfqpoint{4.469957in}{0.553716in}}%
\pgfpathlineto{\pgfqpoint{4.478930in}{0.551503in}}%
\pgfpathlineto{\pgfqpoint{4.488062in}{0.545030in}}%
\pgfpathlineto{\pgfqpoint{4.488058in}{0.536080in}}%
\pgfpathclose%
\pgfusepath{fill}%
\end{pgfscope}%
\begin{pgfscope}%
\pgfpathrectangle{\pgfqpoint{3.525000in}{0.100000in}}{\pgfqpoint{2.857344in}{1.829167in}}%
\pgfusepath{clip}%
\pgfsetbuttcap%
\pgfsetmiterjoin%
\definecolor{currentfill}{rgb}{0.802153,0.920185,0.616378}%
\pgfsetfillcolor{currentfill}%
\pgfsetlinewidth{0.000000pt}%
\definecolor{currentstroke}{rgb}{0.000000,0.000000,0.000000}%
\pgfsetstrokecolor{currentstroke}%
\pgfsetstrokeopacity{0.000000}%
\pgfsetdash{}{0pt}%
\pgfpathmoveto{\pgfqpoint{4.528712in}{0.431369in}}%
\pgfpathlineto{\pgfqpoint{4.519525in}{0.434867in}}%
\pgfpathlineto{\pgfqpoint{4.519483in}{0.441730in}}%
\pgfpathlineto{\pgfqpoint{4.508384in}{0.447533in}}%
\pgfpathlineto{\pgfqpoint{4.509736in}{0.462145in}}%
\pgfpathlineto{\pgfqpoint{4.512951in}{0.469040in}}%
\pgfpathlineto{\pgfqpoint{4.519737in}{0.462694in}}%
\pgfpathlineto{\pgfqpoint{4.530774in}{0.463754in}}%
\pgfpathlineto{\pgfqpoint{4.527987in}{0.443558in}}%
\pgfpathclose%
\pgfusepath{fill}%
\end{pgfscope}%
\begin{pgfscope}%
\pgfpathrectangle{\pgfqpoint{3.525000in}{0.100000in}}{\pgfqpoint{2.857344in}{1.829167in}}%
\pgfusepath{clip}%
\pgfsetbuttcap%
\pgfsetmiterjoin%
\definecolor{currentfill}{rgb}{0.802153,0.920185,0.616378}%
\pgfsetfillcolor{currentfill}%
\pgfsetlinewidth{0.000000pt}%
\definecolor{currentstroke}{rgb}{0.000000,0.000000,0.000000}%
\pgfsetstrokecolor{currentstroke}%
\pgfsetstrokeopacity{0.000000}%
\pgfsetdash{}{0pt}%
\pgfpathmoveto{\pgfqpoint{4.567276in}{0.386322in}}%
\pgfpathlineto{\pgfqpoint{4.555617in}{0.388307in}}%
\pgfpathlineto{\pgfqpoint{4.549169in}{0.398288in}}%
\pgfpathlineto{\pgfqpoint{4.548259in}{0.406495in}}%
\pgfpathclose%
\pgfusepath{fill}%
\end{pgfscope}%
\begin{pgfscope}%
\pgfpathrectangle{\pgfqpoint{3.525000in}{0.100000in}}{\pgfqpoint{2.857344in}{1.829167in}}%
\pgfusepath{clip}%
\pgfsetbuttcap%
\pgfsetmiterjoin%
\definecolor{currentfill}{rgb}{0.802153,0.920185,0.616378}%
\pgfsetfillcolor{currentfill}%
\pgfsetlinewidth{0.000000pt}%
\definecolor{currentstroke}{rgb}{0.000000,0.000000,0.000000}%
\pgfsetstrokecolor{currentstroke}%
\pgfsetstrokeopacity{0.000000}%
\pgfsetdash{}{0pt}%
\pgfpathmoveto{\pgfqpoint{4.543383in}{0.388062in}}%
\pgfpathlineto{\pgfqpoint{4.549496in}{0.383100in}}%
\pgfpathlineto{\pgfqpoint{4.550823in}{0.374568in}}%
\pgfpathlineto{\pgfqpoint{4.538518in}{0.375040in}}%
\pgfpathclose%
\pgfusepath{fill}%
\end{pgfscope}%
\begin{pgfscope}%
\pgfpathrectangle{\pgfqpoint{3.525000in}{0.100000in}}{\pgfqpoint{2.857344in}{1.829167in}}%
\pgfusepath{clip}%
\pgfsetbuttcap%
\pgfsetmiterjoin%
\definecolor{currentfill}{rgb}{0.802153,0.920185,0.616378}%
\pgfsetfillcolor{currentfill}%
\pgfsetlinewidth{0.000000pt}%
\definecolor{currentstroke}{rgb}{0.000000,0.000000,0.000000}%
\pgfsetstrokecolor{currentstroke}%
\pgfsetstrokeopacity{0.000000}%
\pgfsetdash{}{0pt}%
\pgfpathmoveto{\pgfqpoint{4.571456in}{0.340406in}}%
\pgfpathlineto{\pgfqpoint{4.559176in}{0.344820in}}%
\pgfpathlineto{\pgfqpoint{4.562695in}{0.355522in}}%
\pgfpathlineto{\pgfqpoint{4.557548in}{0.366426in}}%
\pgfpathlineto{\pgfqpoint{4.559164in}{0.373922in}}%
\pgfpathlineto{\pgfqpoint{4.569046in}{0.376780in}}%
\pgfpathlineto{\pgfqpoint{4.568856in}{0.364770in}}%
\pgfpathlineto{\pgfqpoint{4.580544in}{0.358108in}}%
\pgfpathlineto{\pgfqpoint{4.587229in}{0.342003in}}%
\pgfpathlineto{\pgfqpoint{4.579779in}{0.335800in}}%
\pgfpathclose%
\pgfusepath{fill}%
\end{pgfscope}%
\begin{pgfscope}%
\pgfpathrectangle{\pgfqpoint{3.525000in}{0.100000in}}{\pgfqpoint{2.857344in}{1.829167in}}%
\pgfusepath{clip}%
\pgfsetbuttcap%
\pgfsetmiterjoin%
\definecolor{currentfill}{rgb}{0.802153,0.920185,0.616378}%
\pgfsetfillcolor{currentfill}%
\pgfsetlinewidth{0.000000pt}%
\definecolor{currentstroke}{rgb}{0.000000,0.000000,0.000000}%
\pgfsetstrokecolor{currentstroke}%
\pgfsetstrokeopacity{0.000000}%
\pgfsetdash{}{0pt}%
\pgfpathmoveto{\pgfqpoint{4.527106in}{0.231508in}}%
\pgfpathlineto{\pgfqpoint{4.521256in}{0.243594in}}%
\pgfpathlineto{\pgfqpoint{4.523846in}{0.251722in}}%
\pgfpathlineto{\pgfqpoint{4.535321in}{0.262974in}}%
\pgfpathlineto{\pgfqpoint{4.536407in}{0.271599in}}%
\pgfpathlineto{\pgfqpoint{4.543917in}{0.290695in}}%
\pgfpathlineto{\pgfqpoint{4.563668in}{0.293958in}}%
\pgfpathlineto{\pgfqpoint{4.567448in}{0.305136in}}%
\pgfpathlineto{\pgfqpoint{4.576156in}{0.300945in}}%
\pgfpathlineto{\pgfqpoint{4.593851in}{0.269573in}}%
\pgfpathlineto{\pgfqpoint{4.594074in}{0.259331in}}%
\pgfpathlineto{\pgfqpoint{4.589073in}{0.250466in}}%
\pgfpathlineto{\pgfqpoint{4.595419in}{0.230112in}}%
\pgfpathlineto{\pgfqpoint{4.590984in}{0.226831in}}%
\pgfpathlineto{\pgfqpoint{4.575880in}{0.227785in}}%
\pgfpathlineto{\pgfqpoint{4.557717in}{0.233855in}}%
\pgfpathlineto{\pgfqpoint{4.542451in}{0.236362in}}%
\pgfpathlineto{\pgfqpoint{4.538374in}{0.232800in}}%
\pgfpathclose%
\pgfusepath{fill}%
\end{pgfscope}%
\begin{pgfscope}%
\pgfpathrectangle{\pgfqpoint{3.525000in}{0.100000in}}{\pgfqpoint{2.857344in}{1.829167in}}%
\pgfusepath{clip}%
\pgfsetbuttcap%
\pgfsetmiterjoin%
\definecolor{currentfill}{rgb}{0.304037,0.653749,0.691349}%
\pgfsetfillcolor{currentfill}%
\pgfsetlinewidth{0.000000pt}%
\definecolor{currentstroke}{rgb}{0.000000,0.000000,0.000000}%
\pgfsetstrokecolor{currentstroke}%
\pgfsetstrokeopacity{0.000000}%
\pgfsetdash{}{0pt}%
\pgfpathmoveto{\pgfqpoint{4.203933in}{1.821590in}}%
\pgfpathlineto{\pgfqpoint{4.189115in}{1.762876in}}%
\pgfpathlineto{\pgfqpoint{4.178579in}{1.720807in}}%
\pgfpathlineto{\pgfqpoint{4.166052in}{1.670200in}}%
\pgfpathlineto{\pgfqpoint{4.163592in}{1.657726in}}%
\pgfpathlineto{\pgfqpoint{4.165198in}{1.646285in}}%
\pgfpathlineto{\pgfqpoint{4.163069in}{1.635747in}}%
\pgfpathlineto{\pgfqpoint{4.075622in}{1.658566in}}%
\pgfpathlineto{\pgfqpoint{4.067722in}{1.655916in}}%
\pgfpathlineto{\pgfqpoint{4.060962in}{1.658216in}}%
\pgfpathlineto{\pgfqpoint{4.029663in}{1.658968in}}%
\pgfpathlineto{\pgfqpoint{4.019077in}{1.655983in}}%
\pgfpathlineto{\pgfqpoint{4.008565in}{1.656991in}}%
\pgfpathlineto{\pgfqpoint{4.004110in}{1.661663in}}%
\pgfpathlineto{\pgfqpoint{3.976088in}{1.660644in}}%
\pgfpathlineto{\pgfqpoint{3.971663in}{1.668278in}}%
\pgfpathlineto{\pgfqpoint{3.963262in}{1.672300in}}%
\pgfpathlineto{\pgfqpoint{3.951039in}{1.674751in}}%
\pgfpathlineto{\pgfqpoint{3.929941in}{1.671163in}}%
\pgfpathlineto{\pgfqpoint{3.922153in}{1.674678in}}%
\pgfpathlineto{\pgfqpoint{3.910145in}{1.684029in}}%
\pgfpathlineto{\pgfqpoint{3.913754in}{1.702356in}}%
\pgfpathlineto{\pgfqpoint{3.912489in}{1.711715in}}%
\pgfpathlineto{\pgfqpoint{3.902972in}{1.721650in}}%
\pgfpathlineto{\pgfqpoint{3.896876in}{1.721034in}}%
\pgfpathlineto{\pgfqpoint{3.892503in}{1.731084in}}%
\pgfpathlineto{\pgfqpoint{3.882144in}{1.735146in}}%
\pgfpathlineto{\pgfqpoint{3.874614in}{1.734629in}}%
\pgfpathlineto{\pgfqpoint{3.872179in}{1.744951in}}%
\pgfpathlineto{\pgfqpoint{3.879680in}{1.743852in}}%
\pgfpathlineto{\pgfqpoint{3.879084in}{1.758199in}}%
\pgfpathlineto{\pgfqpoint{3.875815in}{1.766717in}}%
\pgfpathlineto{\pgfqpoint{3.877256in}{1.777620in}}%
\pgfpathlineto{\pgfqpoint{3.881180in}{1.785821in}}%
\pgfpathlineto{\pgfqpoint{3.880959in}{1.801382in}}%
\pgfpathlineto{\pgfqpoint{3.878864in}{1.807028in}}%
\pgfpathlineto{\pgfqpoint{3.882482in}{1.825143in}}%
\pgfpathlineto{\pgfqpoint{3.881438in}{1.836817in}}%
\pgfpathlineto{\pgfqpoint{3.877820in}{1.842414in}}%
\pgfpathlineto{\pgfqpoint{3.878372in}{1.860716in}}%
\pgfpathlineto{\pgfqpoint{3.883634in}{1.874176in}}%
\pgfpathlineto{\pgfqpoint{3.889357in}{1.870882in}}%
\pgfpathlineto{\pgfqpoint{3.908307in}{1.851307in}}%
\pgfpathlineto{\pgfqpoint{3.931269in}{1.840594in}}%
\pgfpathlineto{\pgfqpoint{3.943027in}{1.839327in}}%
\pgfpathlineto{\pgfqpoint{3.954104in}{1.834766in}}%
\pgfpathlineto{\pgfqpoint{3.957248in}{1.819419in}}%
\pgfpathlineto{\pgfqpoint{3.947540in}{1.816512in}}%
\pgfpathlineto{\pgfqpoint{3.941108in}{1.804417in}}%
\pgfpathlineto{\pgfqpoint{3.948661in}{1.805089in}}%
\pgfpathlineto{\pgfqpoint{3.951706in}{1.810518in}}%
\pgfpathlineto{\pgfqpoint{3.962358in}{1.817307in}}%
\pgfpathlineto{\pgfqpoint{3.961808in}{1.807225in}}%
\pgfpathlineto{\pgfqpoint{3.954696in}{1.805600in}}%
\pgfpathlineto{\pgfqpoint{3.955877in}{1.792641in}}%
\pgfpathlineto{\pgfqpoint{3.947216in}{1.779660in}}%
\pgfpathlineto{\pgfqpoint{3.941673in}{1.785833in}}%
\pgfpathlineto{\pgfqpoint{3.925201in}{1.782477in}}%
\pgfpathlineto{\pgfqpoint{3.924334in}{1.774925in}}%
\pgfpathlineto{\pgfqpoint{3.930013in}{1.770325in}}%
\pgfpathlineto{\pgfqpoint{3.938676in}{1.769872in}}%
\pgfpathlineto{\pgfqpoint{3.950654in}{1.779703in}}%
\pgfpathlineto{\pgfqpoint{3.960135in}{1.780520in}}%
\pgfpathlineto{\pgfqpoint{3.960499in}{1.791402in}}%
\pgfpathlineto{\pgfqpoint{3.965385in}{1.807385in}}%
\pgfpathlineto{\pgfqpoint{3.976964in}{1.819272in}}%
\pgfpathlineto{\pgfqpoint{3.973103in}{1.828479in}}%
\pgfpathlineto{\pgfqpoint{3.975740in}{1.838391in}}%
\pgfpathlineto{\pgfqpoint{3.970603in}{1.848271in}}%
\pgfpathlineto{\pgfqpoint{3.977924in}{1.860904in}}%
\pgfpathlineto{\pgfqpoint{3.979021in}{1.868504in}}%
\pgfpathlineto{\pgfqpoint{3.972665in}{1.873500in}}%
\pgfpathlineto{\pgfqpoint{3.973723in}{1.886278in}}%
\pgfpathlineto{\pgfqpoint{4.049882in}{1.863139in}}%
\pgfpathlineto{\pgfqpoint{4.130756in}{1.840460in}}%
\pgfpathclose%
\pgfusepath{fill}%
\end{pgfscope}%
\begin{pgfscope}%
\pgfpathrectangle{\pgfqpoint{3.525000in}{0.100000in}}{\pgfqpoint{2.857344in}{1.829167in}}%
\pgfusepath{clip}%
\pgfsetbuttcap%
\pgfsetmiterjoin%
\definecolor{currentfill}{rgb}{0.304037,0.653749,0.691349}%
\pgfsetfillcolor{currentfill}%
\pgfsetlinewidth{0.000000pt}%
\definecolor{currentstroke}{rgb}{0.000000,0.000000,0.000000}%
\pgfsetstrokecolor{currentstroke}%
\pgfsetstrokeopacity{0.000000}%
\pgfsetdash{}{0pt}%
\pgfpathmoveto{\pgfqpoint{3.961195in}{1.842073in}}%
\pgfpathlineto{\pgfqpoint{3.964945in}{1.827616in}}%
\pgfpathlineto{\pgfqpoint{3.970825in}{1.822813in}}%
\pgfpathlineto{\pgfqpoint{3.966133in}{1.816700in}}%
\pgfpathlineto{\pgfqpoint{3.961560in}{1.825658in}}%
\pgfpathclose%
\pgfusepath{fill}%
\end{pgfscope}%
\begin{pgfscope}%
\pgfpathrectangle{\pgfqpoint{3.525000in}{0.100000in}}{\pgfqpoint{2.857344in}{1.829167in}}%
\pgfusepath{clip}%
\pgfsetbuttcap%
\pgfsetmiterjoin%
\definecolor{currentfill}{rgb}{0.484890,0.794002,0.645829}%
\pgfsetfillcolor{currentfill}%
\pgfsetlinewidth{0.000000pt}%
\definecolor{currentstroke}{rgb}{0.000000,0.000000,0.000000}%
\pgfsetstrokecolor{currentstroke}%
\pgfsetstrokeopacity{0.000000}%
\pgfsetdash{}{0pt}%
\pgfpathmoveto{\pgfqpoint{4.243206in}{1.812100in}}%
\pgfpathlineto{\pgfqpoint{4.324616in}{1.793831in}}%
\pgfpathlineto{\pgfqpoint{4.401286in}{1.778333in}}%
\pgfpathlineto{\pgfqpoint{4.460275in}{1.767525in}}%
\pgfpathlineto{\pgfqpoint{4.511702in}{1.758885in}}%
\pgfpathlineto{\pgfqpoint{4.563243in}{1.750954in}}%
\pgfpathlineto{\pgfqpoint{4.607136in}{1.744771in}}%
\pgfpathlineto{\pgfqpoint{4.651097in}{1.739101in}}%
\pgfpathlineto{\pgfqpoint{4.695122in}{1.733945in}}%
\pgfpathlineto{\pgfqpoint{4.736610in}{1.729564in}}%
\pgfpathlineto{\pgfqpoint{4.730904in}{1.666336in}}%
\pgfpathlineto{\pgfqpoint{4.722411in}{1.580891in}}%
\pgfpathlineto{\pgfqpoint{4.717957in}{1.536946in}}%
\pgfpathlineto{\pgfqpoint{4.711551in}{1.477721in}}%
\pgfpathlineto{\pgfqpoint{4.666200in}{1.482673in}}%
\pgfpathlineto{\pgfqpoint{4.614279in}{1.488567in}}%
\pgfpathlineto{\pgfqpoint{4.542186in}{1.498434in}}%
\pgfpathlineto{\pgfqpoint{4.509987in}{1.503027in}}%
\pgfpathlineto{\pgfqpoint{4.430657in}{1.515479in}}%
\pgfpathlineto{\pgfqpoint{4.403349in}{1.520475in}}%
\pgfpathlineto{\pgfqpoint{4.397657in}{1.488114in}}%
\pgfpathlineto{\pgfqpoint{4.394546in}{1.490423in}}%
\pgfpathlineto{\pgfqpoint{4.388698in}{1.505991in}}%
\pgfpathlineto{\pgfqpoint{4.381181in}{1.505032in}}%
\pgfpathlineto{\pgfqpoint{4.375800in}{1.497162in}}%
\pgfpathlineto{\pgfqpoint{4.364635in}{1.496109in}}%
\pgfpathlineto{\pgfqpoint{4.362392in}{1.499875in}}%
\pgfpathlineto{\pgfqpoint{4.351933in}{1.499419in}}%
\pgfpathlineto{\pgfqpoint{4.346675in}{1.503065in}}%
\pgfpathlineto{\pgfqpoint{4.339323in}{1.497424in}}%
\pgfpathlineto{\pgfqpoint{4.321466in}{1.502500in}}%
\pgfpathlineto{\pgfqpoint{4.313673in}{1.499774in}}%
\pgfpathlineto{\pgfqpoint{4.309480in}{1.511871in}}%
\pgfpathlineto{\pgfqpoint{4.309284in}{1.523380in}}%
\pgfpathlineto{\pgfqpoint{4.296975in}{1.531657in}}%
\pgfpathlineto{\pgfqpoint{4.298982in}{1.543586in}}%
\pgfpathlineto{\pgfqpoint{4.290217in}{1.563291in}}%
\pgfpathlineto{\pgfqpoint{4.290808in}{1.580042in}}%
\pgfpathlineto{\pgfqpoint{4.283275in}{1.588240in}}%
\pgfpathlineto{\pgfqpoint{4.276206in}{1.580987in}}%
\pgfpathlineto{\pgfqpoint{4.266586in}{1.577064in}}%
\pgfpathlineto{\pgfqpoint{4.259263in}{1.585150in}}%
\pgfpathlineto{\pgfqpoint{4.260453in}{1.598061in}}%
\pgfpathlineto{\pgfqpoint{4.269236in}{1.603148in}}%
\pgfpathlineto{\pgfqpoint{4.266901in}{1.611328in}}%
\pgfpathlineto{\pgfqpoint{4.282168in}{1.650923in}}%
\pgfpathlineto{\pgfqpoint{4.270130in}{1.651915in}}%
\pgfpathlineto{\pgfqpoint{4.268893in}{1.659021in}}%
\pgfpathlineto{\pgfqpoint{4.260124in}{1.665148in}}%
\pgfpathlineto{\pgfqpoint{4.260684in}{1.671997in}}%
\pgfpathlineto{\pgfqpoint{4.256059in}{1.677319in}}%
\pgfpathlineto{\pgfqpoint{4.248529in}{1.696639in}}%
\pgfpathlineto{\pgfqpoint{4.241648in}{1.700579in}}%
\pgfpathlineto{\pgfqpoint{4.236720in}{1.717253in}}%
\pgfpathlineto{\pgfqpoint{4.237883in}{1.728336in}}%
\pgfpathlineto{\pgfqpoint{4.228691in}{1.748749in}}%
\pgfpathclose%
\pgfusepath{fill}%
\end{pgfscope}%
\begin{pgfscope}%
\pgfpathrectangle{\pgfqpoint{3.525000in}{0.100000in}}{\pgfqpoint{2.857344in}{1.829167in}}%
\pgfusepath{clip}%
\pgfsetbuttcap%
\pgfsetmiterjoin%
\definecolor{currentfill}{rgb}{0.431834,0.773241,0.646597}%
\pgfsetfillcolor{currentfill}%
\pgfsetlinewidth{0.000000pt}%
\definecolor{currentstroke}{rgb}{0.000000,0.000000,0.000000}%
\pgfsetstrokecolor{currentstroke}%
\pgfsetstrokeopacity{0.000000}%
\pgfsetdash{}{0pt}%
\pgfpathmoveto{\pgfqpoint{6.208503in}{1.486275in}}%
\pgfpathlineto{\pgfqpoint{6.206874in}{1.493203in}}%
\pgfpathlineto{\pgfqpoint{6.197818in}{1.499273in}}%
\pgfpathlineto{\pgfqpoint{6.173887in}{1.577551in}}%
\pgfpathlineto{\pgfqpoint{6.160989in}{1.615326in}}%
\pgfpathlineto{\pgfqpoint{6.171358in}{1.626209in}}%
\pgfpathlineto{\pgfqpoint{6.181439in}{1.652843in}}%
\pgfpathlineto{\pgfqpoint{6.186466in}{1.661354in}}%
\pgfpathlineto{\pgfqpoint{6.182905in}{1.664929in}}%
\pgfpathlineto{\pgfqpoint{6.181712in}{1.688571in}}%
\pgfpathlineto{\pgfqpoint{6.186210in}{1.695818in}}%
\pgfpathlineto{\pgfqpoint{6.184238in}{1.712665in}}%
\pgfpathlineto{\pgfqpoint{6.202189in}{1.767935in}}%
\pgfpathlineto{\pgfqpoint{6.210277in}{1.768237in}}%
\pgfpathlineto{\pgfqpoint{6.213634in}{1.758364in}}%
\pgfpathlineto{\pgfqpoint{6.220822in}{1.755529in}}%
\pgfpathlineto{\pgfqpoint{6.234400in}{1.767142in}}%
\pgfpathlineto{\pgfqpoint{6.245047in}{1.774001in}}%
\pgfpathlineto{\pgfqpoint{6.268506in}{1.761923in}}%
\pgfpathlineto{\pgfqpoint{6.289699in}{1.694859in}}%
\pgfpathlineto{\pgfqpoint{6.293728in}{1.678360in}}%
\pgfpathlineto{\pgfqpoint{6.303011in}{1.676425in}}%
\pgfpathlineto{\pgfqpoint{6.315601in}{1.665219in}}%
\pgfpathlineto{\pgfqpoint{6.314888in}{1.658690in}}%
\pgfpathlineto{\pgfqpoint{6.323465in}{1.650951in}}%
\pgfpathlineto{\pgfqpoint{6.330432in}{1.655391in}}%
\pgfpathlineto{\pgfqpoint{6.345010in}{1.638389in}}%
\pgfpathlineto{\pgfqpoint{6.338495in}{1.624782in}}%
\pgfpathlineto{\pgfqpoint{6.329757in}{1.624512in}}%
\pgfpathlineto{\pgfqpoint{6.322747in}{1.612366in}}%
\pgfpathlineto{\pgfqpoint{6.314256in}{1.610643in}}%
\pgfpathlineto{\pgfqpoint{6.308022in}{1.604175in}}%
\pgfpathlineto{\pgfqpoint{6.289402in}{1.597239in}}%
\pgfpathlineto{\pgfqpoint{6.278133in}{1.586061in}}%
\pgfpathlineto{\pgfqpoint{6.272408in}{1.594084in}}%
\pgfpathlineto{\pgfqpoint{6.267206in}{1.588334in}}%
\pgfpathlineto{\pgfqpoint{6.268631in}{1.565087in}}%
\pgfpathlineto{\pgfqpoint{6.264512in}{1.555879in}}%
\pgfpathlineto{\pgfqpoint{6.255528in}{1.558394in}}%
\pgfpathlineto{\pgfqpoint{6.254078in}{1.548953in}}%
\pgfpathlineto{\pgfqpoint{6.250196in}{1.545071in}}%
\pgfpathlineto{\pgfqpoint{6.242426in}{1.546017in}}%
\pgfpathlineto{\pgfqpoint{6.242916in}{1.537191in}}%
\pgfpathlineto{\pgfqpoint{6.231140in}{1.539698in}}%
\pgfpathlineto{\pgfqpoint{6.224709in}{1.527468in}}%
\pgfpathlineto{\pgfqpoint{6.227137in}{1.521070in}}%
\pgfpathlineto{\pgfqpoint{6.223312in}{1.510427in}}%
\pgfpathlineto{\pgfqpoint{6.217289in}{1.502600in}}%
\pgfpathlineto{\pgfqpoint{6.215774in}{1.486259in}}%
\pgfpathclose%
\pgfusepath{fill}%
\end{pgfscope}%
\begin{pgfscope}%
\pgfpathrectangle{\pgfqpoint{3.525000in}{0.100000in}}{\pgfqpoint{2.857344in}{1.829167in}}%
\pgfusepath{clip}%
\pgfsetbuttcap%
\pgfsetmiterjoin%
\definecolor{currentfill}{rgb}{0.431834,0.773241,0.646597}%
\pgfsetfillcolor{currentfill}%
\pgfsetlinewidth{0.000000pt}%
\definecolor{currentstroke}{rgb}{0.000000,0.000000,0.000000}%
\pgfsetstrokecolor{currentstroke}%
\pgfsetstrokeopacity{0.000000}%
\pgfsetdash{}{0pt}%
\pgfpathmoveto{\pgfqpoint{6.292747in}{1.592465in}}%
\pgfpathlineto{\pgfqpoint{6.298063in}{1.598034in}}%
\pgfpathlineto{\pgfqpoint{6.303125in}{1.592800in}}%
\pgfpathlineto{\pgfqpoint{6.294033in}{1.585865in}}%
\pgfpathclose%
\pgfusepath{fill}%
\end{pgfscope}%
\begin{pgfscope}%
\pgfpathrectangle{\pgfqpoint{3.525000in}{0.100000in}}{\pgfqpoint{2.857344in}{1.829167in}}%
\pgfusepath{clip}%
\pgfsetbuttcap%
\pgfsetmiterjoin%
\definecolor{currentfill}{rgb}{0.892887,0.957093,0.597924}%
\pgfsetfillcolor{currentfill}%
\pgfsetlinewidth{0.000000pt}%
\definecolor{currentstroke}{rgb}{0.000000,0.000000,0.000000}%
\pgfsetstrokecolor{currentstroke}%
\pgfsetstrokeopacity{0.000000}%
\pgfsetdash{}{0pt}%
\pgfpathmoveto{\pgfqpoint{4.717957in}{1.536946in}}%
\pgfpathlineto{\pgfqpoint{4.722411in}{1.580891in}}%
\pgfpathlineto{\pgfqpoint{4.730904in}{1.666336in}}%
\pgfpathlineto{\pgfqpoint{4.736610in}{1.729564in}}%
\pgfpathlineto{\pgfqpoint{4.783337in}{1.725180in}}%
\pgfpathlineto{\pgfqpoint{4.843116in}{1.720421in}}%
\pgfpathlineto{\pgfqpoint{4.897756in}{1.716901in}}%
\pgfpathlineto{\pgfqpoint{4.947231in}{1.714398in}}%
\pgfpathlineto{\pgfqpoint{5.021066in}{1.711865in}}%
\pgfpathlineto{\pgfqpoint{5.026105in}{1.691056in}}%
\pgfpathlineto{\pgfqpoint{5.024301in}{1.681111in}}%
\pgfpathlineto{\pgfqpoint{5.023707in}{1.660670in}}%
\pgfpathlineto{\pgfqpoint{5.027135in}{1.645362in}}%
\pgfpathlineto{\pgfqpoint{5.034989in}{1.622777in}}%
\pgfpathlineto{\pgfqpoint{5.034971in}{1.594344in}}%
\pgfpathlineto{\pgfqpoint{5.036425in}{1.561312in}}%
\pgfpathlineto{\pgfqpoint{5.038420in}{1.552410in}}%
\pgfpathlineto{\pgfqpoint{5.044257in}{1.542637in}}%
\pgfpathlineto{\pgfqpoint{5.046182in}{1.527408in}}%
\pgfpathlineto{\pgfqpoint{5.045351in}{1.517241in}}%
\pgfpathlineto{\pgfqpoint{4.983454in}{1.518582in}}%
\pgfpathlineto{\pgfqpoint{4.938424in}{1.520865in}}%
\pgfpathlineto{\pgfqpoint{4.872414in}{1.524379in}}%
\pgfpathlineto{\pgfqpoint{4.807318in}{1.529003in}}%
\pgfpathlineto{\pgfqpoint{4.763961in}{1.532501in}}%
\pgfpathclose%
\pgfusepath{fill}%
\end{pgfscope}%
\begin{pgfscope}%
\pgfpathrectangle{\pgfqpoint{3.525000in}{0.100000in}}{\pgfqpoint{2.857344in}{1.829167in}}%
\pgfusepath{clip}%
\pgfsetbuttcap%
\pgfsetmiterjoin%
\definecolor{currentfill}{rgb}{0.684198,0.872203,0.640369}%
\pgfsetfillcolor{currentfill}%
\pgfsetlinewidth{0.000000pt}%
\definecolor{currentstroke}{rgb}{0.000000,0.000000,0.000000}%
\pgfsetstrokecolor{currentstroke}%
\pgfsetstrokeopacity{0.000000}%
\pgfsetdash{}{0pt}%
\pgfpathmoveto{\pgfqpoint{4.699247in}{1.353032in}}%
\pgfpathlineto{\pgfqpoint{4.706393in}{1.426697in}}%
\pgfpathlineto{\pgfqpoint{4.711551in}{1.477721in}}%
\pgfpathlineto{\pgfqpoint{4.717957in}{1.536946in}}%
\pgfpathlineto{\pgfqpoint{4.763961in}{1.532501in}}%
\pgfpathlineto{\pgfqpoint{4.807318in}{1.529003in}}%
\pgfpathlineto{\pgfqpoint{4.872414in}{1.524379in}}%
\pgfpathlineto{\pgfqpoint{4.938424in}{1.520865in}}%
\pgfpathlineto{\pgfqpoint{4.983454in}{1.518582in}}%
\pgfpathlineto{\pgfqpoint{5.045351in}{1.517241in}}%
\pgfpathlineto{\pgfqpoint{5.041160in}{1.505012in}}%
\pgfpathlineto{\pgfqpoint{5.032796in}{1.495412in}}%
\pgfpathlineto{\pgfqpoint{5.039201in}{1.484359in}}%
\pgfpathlineto{\pgfqpoint{5.046264in}{1.481997in}}%
\pgfpathlineto{\pgfqpoint{5.049616in}{1.475648in}}%
\pgfpathlineto{\pgfqpoint{5.048847in}{1.429290in}}%
\pgfpathlineto{\pgfqpoint{5.047559in}{1.364049in}}%
\pgfpathlineto{\pgfqpoint{5.042795in}{1.346914in}}%
\pgfpathlineto{\pgfqpoint{5.047056in}{1.337435in}}%
\pgfpathlineto{\pgfqpoint{5.038501in}{1.317067in}}%
\pgfpathlineto{\pgfqpoint{5.047514in}{1.300647in}}%
\pgfpathlineto{\pgfqpoint{5.039856in}{1.301904in}}%
\pgfpathlineto{\pgfqpoint{5.034632in}{1.312120in}}%
\pgfpathlineto{\pgfqpoint{5.015986in}{1.319122in}}%
\pgfpathlineto{\pgfqpoint{5.004228in}{1.325282in}}%
\pgfpathlineto{\pgfqpoint{4.984496in}{1.325794in}}%
\pgfpathlineto{\pgfqpoint{4.977647in}{1.320183in}}%
\pgfpathlineto{\pgfqpoint{4.955313in}{1.331232in}}%
\pgfpathlineto{\pgfqpoint{4.953600in}{1.334727in}}%
\pgfpathlineto{\pgfqpoint{4.875656in}{1.338416in}}%
\pgfpathlineto{\pgfqpoint{4.828301in}{1.341130in}}%
\pgfpathlineto{\pgfqpoint{4.789186in}{1.344193in}}%
\pgfpathlineto{\pgfqpoint{4.724554in}{1.350300in}}%
\pgfpathclose%
\pgfusepath{fill}%
\end{pgfscope}%
\begin{pgfscope}%
\pgfpathrectangle{\pgfqpoint{3.525000in}{0.100000in}}{\pgfqpoint{2.857344in}{1.829167in}}%
\pgfusepath{clip}%
\pgfsetbuttcap%
\pgfsetmiterjoin%
\definecolor{currentfill}{rgb}{0.747712,0.898039,0.627451}%
\pgfsetfillcolor{currentfill}%
\pgfsetlinewidth{0.000000pt}%
\definecolor{currentstroke}{rgb}{0.000000,0.000000,0.000000}%
\pgfsetstrokecolor{currentstroke}%
\pgfsetstrokeopacity{0.000000}%
\pgfsetdash{}{0pt}%
\pgfpathmoveto{\pgfqpoint{4.686989in}{1.228296in}}%
\pgfpathlineto{\pgfqpoint{4.645435in}{1.232091in}}%
\pgfpathlineto{\pgfqpoint{4.554859in}{1.243270in}}%
\pgfpathlineto{\pgfqpoint{4.505586in}{1.250342in}}%
\pgfpathlineto{\pgfqpoint{4.452739in}{1.257928in}}%
\pgfpathlineto{\pgfqpoint{4.408224in}{1.265052in}}%
\pgfpathlineto{\pgfqpoint{4.359362in}{1.273417in}}%
\pgfpathlineto{\pgfqpoint{4.370471in}{1.335044in}}%
\pgfpathlineto{\pgfqpoint{4.381725in}{1.398237in}}%
\pgfpathlineto{\pgfqpoint{4.397657in}{1.488114in}}%
\pgfpathlineto{\pgfqpoint{4.403349in}{1.520475in}}%
\pgfpathlineto{\pgfqpoint{4.430657in}{1.515479in}}%
\pgfpathlineto{\pgfqpoint{4.509987in}{1.503027in}}%
\pgfpathlineto{\pgfqpoint{4.542186in}{1.498434in}}%
\pgfpathlineto{\pgfqpoint{4.614279in}{1.488567in}}%
\pgfpathlineto{\pgfqpoint{4.666200in}{1.482673in}}%
\pgfpathlineto{\pgfqpoint{4.711551in}{1.477721in}}%
\pgfpathlineto{\pgfqpoint{4.706393in}{1.426697in}}%
\pgfpathlineto{\pgfqpoint{4.699247in}{1.353032in}}%
\pgfpathlineto{\pgfqpoint{4.693113in}{1.290427in}}%
\pgfpathclose%
\pgfusepath{fill}%
\end{pgfscope}%
\begin{pgfscope}%
\pgfpathrectangle{\pgfqpoint{3.525000in}{0.100000in}}{\pgfqpoint{2.857344in}{1.829167in}}%
\pgfusepath{clip}%
\pgfsetbuttcap%
\pgfsetmiterjoin%
\definecolor{currentfill}{rgb}{0.729566,0.890657,0.631142}%
\pgfsetfillcolor{currentfill}%
\pgfsetlinewidth{0.000000pt}%
\definecolor{currentstroke}{rgb}{0.000000,0.000000,0.000000}%
\pgfsetstrokecolor{currentstroke}%
\pgfsetstrokeopacity{0.000000}%
\pgfsetdash{}{0pt}%
\pgfpathmoveto{\pgfqpoint{5.446726in}{1.313305in}}%
\pgfpathlineto{\pgfqpoint{5.394068in}{1.309564in}}%
\pgfpathlineto{\pgfqpoint{5.315579in}{1.306219in}}%
\pgfpathlineto{\pgfqpoint{5.312596in}{1.314147in}}%
\pgfpathlineto{\pgfqpoint{5.295191in}{1.320074in}}%
\pgfpathlineto{\pgfqpoint{5.291352in}{1.331272in}}%
\pgfpathlineto{\pgfqpoint{5.289744in}{1.345124in}}%
\pgfpathlineto{\pgfqpoint{5.293667in}{1.352221in}}%
\pgfpathlineto{\pgfqpoint{5.287472in}{1.359032in}}%
\pgfpathlineto{\pgfqpoint{5.285978in}{1.367156in}}%
\pgfpathlineto{\pgfqpoint{5.283981in}{1.385126in}}%
\pgfpathlineto{\pgfqpoint{5.278044in}{1.394887in}}%
\pgfpathlineto{\pgfqpoint{5.267500in}{1.400366in}}%
\pgfpathlineto{\pgfqpoint{5.256027in}{1.409407in}}%
\pgfpathlineto{\pgfqpoint{5.250096in}{1.420104in}}%
\pgfpathlineto{\pgfqpoint{5.239454in}{1.424411in}}%
\pgfpathlineto{\pgfqpoint{5.233198in}{1.431422in}}%
\pgfpathlineto{\pgfqpoint{5.225619in}{1.432612in}}%
\pgfpathlineto{\pgfqpoint{5.212088in}{1.443021in}}%
\pgfpathlineto{\pgfqpoint{5.214286in}{1.455000in}}%
\pgfpathlineto{\pgfqpoint{5.213862in}{1.477780in}}%
\pgfpathlineto{\pgfqpoint{5.218035in}{1.484051in}}%
\pgfpathlineto{\pgfqpoint{5.214286in}{1.493519in}}%
\pgfpathlineto{\pgfqpoint{5.207682in}{1.495350in}}%
\pgfpathlineto{\pgfqpoint{5.208227in}{1.503666in}}%
\pgfpathlineto{\pgfqpoint{5.216422in}{1.516803in}}%
\pgfpathlineto{\pgfqpoint{5.232649in}{1.527191in}}%
\pgfpathlineto{\pgfqpoint{5.231638in}{1.564135in}}%
\pgfpathlineto{\pgfqpoint{5.239758in}{1.569684in}}%
\pgfpathlineto{\pgfqpoint{5.247442in}{1.565988in}}%
\pgfpathlineto{\pgfqpoint{5.263095in}{1.571400in}}%
\pgfpathlineto{\pgfqpoint{5.292545in}{1.585001in}}%
\pgfpathlineto{\pgfqpoint{5.296381in}{1.580789in}}%
\pgfpathlineto{\pgfqpoint{5.290803in}{1.561714in}}%
\pgfpathlineto{\pgfqpoint{5.299101in}{1.565900in}}%
\pgfpathlineto{\pgfqpoint{5.313314in}{1.561720in}}%
\pgfpathlineto{\pgfqpoint{5.322048in}{1.558231in}}%
\pgfpathlineto{\pgfqpoint{5.326945in}{1.548007in}}%
\pgfpathlineto{\pgfqpoint{5.371751in}{1.538358in}}%
\pgfpathlineto{\pgfqpoint{5.385144in}{1.531739in}}%
\pgfpathlineto{\pgfqpoint{5.398761in}{1.531816in}}%
\pgfpathlineto{\pgfqpoint{5.412753in}{1.529086in}}%
\pgfpathlineto{\pgfqpoint{5.421834in}{1.519758in}}%
\pgfpathlineto{\pgfqpoint{5.430511in}{1.515149in}}%
\pgfpathlineto{\pgfqpoint{5.432105in}{1.501855in}}%
\pgfpathlineto{\pgfqpoint{5.429541in}{1.493508in}}%
\pgfpathlineto{\pgfqpoint{5.439212in}{1.492891in}}%
\pgfpathlineto{\pgfqpoint{5.435954in}{1.483210in}}%
\pgfpathlineto{\pgfqpoint{5.439053in}{1.479771in}}%
\pgfpathlineto{\pgfqpoint{5.442135in}{1.470633in}}%
\pgfpathlineto{\pgfqpoint{5.432710in}{1.465795in}}%
\pgfpathlineto{\pgfqpoint{5.427238in}{1.452314in}}%
\pgfpathlineto{\pgfqpoint{5.425521in}{1.442782in}}%
\pgfpathlineto{\pgfqpoint{5.430771in}{1.441150in}}%
\pgfpathlineto{\pgfqpoint{5.437492in}{1.448298in}}%
\pgfpathlineto{\pgfqpoint{5.443207in}{1.460686in}}%
\pgfpathlineto{\pgfqpoint{5.450939in}{1.464988in}}%
\pgfpathlineto{\pgfqpoint{5.456749in}{1.459392in}}%
\pgfpathlineto{\pgfqpoint{5.451030in}{1.442453in}}%
\pgfpathlineto{\pgfqpoint{5.449236in}{1.429304in}}%
\pgfpathlineto{\pgfqpoint{5.450926in}{1.419852in}}%
\pgfpathlineto{\pgfqpoint{5.445614in}{1.414504in}}%
\pgfpathlineto{\pgfqpoint{5.442962in}{1.401437in}}%
\pgfpathlineto{\pgfqpoint{5.445129in}{1.387704in}}%
\pgfpathlineto{\pgfqpoint{5.438861in}{1.367394in}}%
\pgfpathlineto{\pgfqpoint{5.438993in}{1.357246in}}%
\pgfpathlineto{\pgfqpoint{5.443946in}{1.335254in}}%
\pgfpathlineto{\pgfqpoint{5.447156in}{1.331481in}}%
\pgfpathclose%
\pgfusepath{fill}%
\end{pgfscope}%
\begin{pgfscope}%
\pgfpathrectangle{\pgfqpoint{3.525000in}{0.100000in}}{\pgfqpoint{2.857344in}{1.829167in}}%
\pgfusepath{clip}%
\pgfsetbuttcap%
\pgfsetmiterjoin%
\definecolor{currentfill}{rgb}{0.729566,0.890657,0.631142}%
\pgfsetfillcolor{currentfill}%
\pgfsetlinewidth{0.000000pt}%
\definecolor{currentstroke}{rgb}{0.000000,0.000000,0.000000}%
\pgfsetstrokecolor{currentstroke}%
\pgfsetstrokeopacity{0.000000}%
\pgfsetdash{}{0pt}%
\pgfpathmoveto{\pgfqpoint{5.466469in}{1.491544in}}%
\pgfpathlineto{\pgfqpoint{5.467644in}{1.482772in}}%
\pgfpathlineto{\pgfqpoint{5.456876in}{1.459658in}}%
\pgfpathlineto{\pgfqpoint{5.452091in}{1.466354in}}%
\pgfpathclose%
\pgfusepath{fill}%
\end{pgfscope}%
\begin{pgfscope}%
\pgfpathrectangle{\pgfqpoint{3.525000in}{0.100000in}}{\pgfqpoint{2.857344in}{1.829167in}}%
\pgfusepath{clip}%
\pgfsetbuttcap%
\pgfsetmiterjoin%
\definecolor{currentfill}{rgb}{0.400000,0.760784,0.647059}%
\pgfsetfillcolor{currentfill}%
\pgfsetlinewidth{0.000000pt}%
\definecolor{currentstroke}{rgb}{0.000000,0.000000,0.000000}%
\pgfsetstrokecolor{currentstroke}%
\pgfsetstrokeopacity{0.000000}%
\pgfsetdash{}{0pt}%
\pgfpathmoveto{\pgfqpoint{4.163069in}{1.635747in}}%
\pgfpathlineto{\pgfqpoint{4.165198in}{1.646285in}}%
\pgfpathlineto{\pgfqpoint{4.163592in}{1.657726in}}%
\pgfpathlineto{\pgfqpoint{4.166052in}{1.670200in}}%
\pgfpathlineto{\pgfqpoint{4.178579in}{1.720807in}}%
\pgfpathlineto{\pgfqpoint{4.189115in}{1.762876in}}%
\pgfpathlineto{\pgfqpoint{4.203933in}{1.821590in}}%
\pgfpathlineto{\pgfqpoint{4.243206in}{1.812100in}}%
\pgfpathlineto{\pgfqpoint{4.228691in}{1.748749in}}%
\pgfpathlineto{\pgfqpoint{4.237883in}{1.728336in}}%
\pgfpathlineto{\pgfqpoint{4.236720in}{1.717253in}}%
\pgfpathlineto{\pgfqpoint{4.241648in}{1.700579in}}%
\pgfpathlineto{\pgfqpoint{4.248529in}{1.696639in}}%
\pgfpathlineto{\pgfqpoint{4.256059in}{1.677319in}}%
\pgfpathlineto{\pgfqpoint{4.260684in}{1.671997in}}%
\pgfpathlineto{\pgfqpoint{4.260124in}{1.665148in}}%
\pgfpathlineto{\pgfqpoint{4.268893in}{1.659021in}}%
\pgfpathlineto{\pgfqpoint{4.270130in}{1.651915in}}%
\pgfpathlineto{\pgfqpoint{4.282168in}{1.650923in}}%
\pgfpathlineto{\pgfqpoint{4.266901in}{1.611328in}}%
\pgfpathlineto{\pgfqpoint{4.269236in}{1.603148in}}%
\pgfpathlineto{\pgfqpoint{4.260453in}{1.598061in}}%
\pgfpathlineto{\pgfqpoint{4.259263in}{1.585150in}}%
\pgfpathlineto{\pgfqpoint{4.266586in}{1.577064in}}%
\pgfpathlineto{\pgfqpoint{4.276206in}{1.580987in}}%
\pgfpathlineto{\pgfqpoint{4.283275in}{1.588240in}}%
\pgfpathlineto{\pgfqpoint{4.290808in}{1.580042in}}%
\pgfpathlineto{\pgfqpoint{4.290217in}{1.563291in}}%
\pgfpathlineto{\pgfqpoint{4.298982in}{1.543586in}}%
\pgfpathlineto{\pgfqpoint{4.296975in}{1.531657in}}%
\pgfpathlineto{\pgfqpoint{4.309284in}{1.523380in}}%
\pgfpathlineto{\pgfqpoint{4.309480in}{1.511871in}}%
\pgfpathlineto{\pgfqpoint{4.313673in}{1.499774in}}%
\pgfpathlineto{\pgfqpoint{4.321466in}{1.502500in}}%
\pgfpathlineto{\pgfqpoint{4.339323in}{1.497424in}}%
\pgfpathlineto{\pgfqpoint{4.346675in}{1.503065in}}%
\pgfpathlineto{\pgfqpoint{4.351933in}{1.499419in}}%
\pgfpathlineto{\pgfqpoint{4.362392in}{1.499875in}}%
\pgfpathlineto{\pgfqpoint{4.364635in}{1.496109in}}%
\pgfpathlineto{\pgfqpoint{4.375800in}{1.497162in}}%
\pgfpathlineto{\pgfqpoint{4.381181in}{1.505032in}}%
\pgfpathlineto{\pgfqpoint{4.388698in}{1.505991in}}%
\pgfpathlineto{\pgfqpoint{4.394546in}{1.490423in}}%
\pgfpathlineto{\pgfqpoint{4.397657in}{1.488114in}}%
\pgfpathlineto{\pgfqpoint{4.381725in}{1.398237in}}%
\pgfpathlineto{\pgfqpoint{4.370471in}{1.335044in}}%
\pgfpathlineto{\pgfqpoint{4.281718in}{1.352178in}}%
\pgfpathlineto{\pgfqpoint{4.233777in}{1.361714in}}%
\pgfpathlineto{\pgfqpoint{4.188936in}{1.371575in}}%
\pgfpathlineto{\pgfqpoint{4.147022in}{1.381064in}}%
\pgfpathlineto{\pgfqpoint{4.098515in}{1.392762in}}%
\pgfpathlineto{\pgfqpoint{4.123535in}{1.495404in}}%
\pgfpathlineto{\pgfqpoint{4.124884in}{1.502905in}}%
\pgfpathlineto{\pgfqpoint{4.136011in}{1.522559in}}%
\pgfpathlineto{\pgfqpoint{4.124490in}{1.534373in}}%
\pgfpathlineto{\pgfqpoint{4.126882in}{1.545974in}}%
\pgfpathlineto{\pgfqpoint{4.131268in}{1.548877in}}%
\pgfpathlineto{\pgfqpoint{4.139100in}{1.560831in}}%
\pgfpathlineto{\pgfqpoint{4.150517in}{1.569101in}}%
\pgfpathlineto{\pgfqpoint{4.151141in}{1.575220in}}%
\pgfpathlineto{\pgfqpoint{4.158032in}{1.581334in}}%
\pgfpathlineto{\pgfqpoint{4.163753in}{1.592778in}}%
\pgfpathlineto{\pgfqpoint{4.175490in}{1.604886in}}%
\pgfpathlineto{\pgfqpoint{4.174657in}{1.616844in}}%
\pgfpathlineto{\pgfqpoint{4.166632in}{1.623486in}}%
\pgfpathclose%
\pgfusepath{fill}%
\end{pgfscope}%
\begin{pgfscope}%
\pgfpathrectangle{\pgfqpoint{3.525000in}{0.100000in}}{\pgfqpoint{2.857344in}{1.829167in}}%
\pgfusepath{clip}%
\pgfsetbuttcap%
\pgfsetmiterjoin%
\definecolor{currentfill}{rgb}{0.802153,0.920185,0.616378}%
\pgfsetfillcolor{currentfill}%
\pgfsetlinewidth{0.000000pt}%
\definecolor{currentstroke}{rgb}{0.000000,0.000000,0.000000}%
\pgfsetstrokecolor{currentstroke}%
\pgfsetstrokeopacity{0.000000}%
\pgfsetdash{}{0pt}%
\pgfpathmoveto{\pgfqpoint{6.105920in}{1.435444in}}%
\pgfpathlineto{\pgfqpoint{6.102232in}{1.447085in}}%
\pgfpathlineto{\pgfqpoint{6.095389in}{1.482398in}}%
\pgfpathlineto{\pgfqpoint{6.086541in}{1.493263in}}%
\pgfpathlineto{\pgfqpoint{6.078780in}{1.512801in}}%
\pgfpathlineto{\pgfqpoint{6.081328in}{1.527286in}}%
\pgfpathlineto{\pgfqpoint{6.079285in}{1.537962in}}%
\pgfpathlineto{\pgfqpoint{6.072654in}{1.548449in}}%
\pgfpathlineto{\pgfqpoint{6.072381in}{1.560000in}}%
\pgfpathlineto{\pgfqpoint{6.068531in}{1.572458in}}%
\pgfpathlineto{\pgfqpoint{6.102988in}{1.580938in}}%
\pgfpathlineto{\pgfqpoint{6.147744in}{1.593019in}}%
\pgfpathlineto{\pgfqpoint{6.149530in}{1.586091in}}%
\pgfpathlineto{\pgfqpoint{6.146681in}{1.575063in}}%
\pgfpathlineto{\pgfqpoint{6.153370in}{1.566239in}}%
\pgfpathlineto{\pgfqpoint{6.149817in}{1.555057in}}%
\pgfpathlineto{\pgfqpoint{6.135698in}{1.541001in}}%
\pgfpathlineto{\pgfqpoint{6.139579in}{1.530443in}}%
\pgfpathlineto{\pgfqpoint{6.136703in}{1.508126in}}%
\pgfpathlineto{\pgfqpoint{6.132676in}{1.495543in}}%
\pgfpathlineto{\pgfqpoint{6.137813in}{1.459760in}}%
\pgfpathlineto{\pgfqpoint{6.137121in}{1.447171in}}%
\pgfpathlineto{\pgfqpoint{6.142091in}{1.443130in}}%
\pgfpathclose%
\pgfusepath{fill}%
\end{pgfscope}%
\begin{pgfscope}%
\pgfpathrectangle{\pgfqpoint{3.525000in}{0.100000in}}{\pgfqpoint{2.857344in}{1.829167in}}%
\pgfusepath{clip}%
\pgfsetbuttcap%
\pgfsetmiterjoin%
\definecolor{currentfill}{rgb}{0.729566,0.890657,0.631142}%
\pgfsetfillcolor{currentfill}%
\pgfsetlinewidth{0.000000pt}%
\definecolor{currentstroke}{rgb}{0.000000,0.000000,0.000000}%
\pgfsetstrokecolor{currentstroke}%
\pgfsetstrokeopacity{0.000000}%
\pgfsetdash{}{0pt}%
\pgfpathmoveto{\pgfqpoint{5.047559in}{1.364049in}}%
\pgfpathlineto{\pgfqpoint{5.048847in}{1.429290in}}%
\pgfpathlineto{\pgfqpoint{5.049616in}{1.475648in}}%
\pgfpathlineto{\pgfqpoint{5.046264in}{1.481997in}}%
\pgfpathlineto{\pgfqpoint{5.039201in}{1.484359in}}%
\pgfpathlineto{\pgfqpoint{5.032796in}{1.495412in}}%
\pgfpathlineto{\pgfqpoint{5.041160in}{1.505012in}}%
\pgfpathlineto{\pgfqpoint{5.045351in}{1.517241in}}%
\pgfpathlineto{\pgfqpoint{5.046182in}{1.527408in}}%
\pgfpathlineto{\pgfqpoint{5.044257in}{1.542637in}}%
\pgfpathlineto{\pgfqpoint{5.038420in}{1.552410in}}%
\pgfpathlineto{\pgfqpoint{5.036425in}{1.561312in}}%
\pgfpathlineto{\pgfqpoint{5.034971in}{1.594344in}}%
\pgfpathlineto{\pgfqpoint{5.034989in}{1.622777in}}%
\pgfpathlineto{\pgfqpoint{5.027135in}{1.645362in}}%
\pgfpathlineto{\pgfqpoint{5.023707in}{1.660670in}}%
\pgfpathlineto{\pgfqpoint{5.024301in}{1.681111in}}%
\pgfpathlineto{\pgfqpoint{5.026105in}{1.691056in}}%
\pgfpathlineto{\pgfqpoint{5.021066in}{1.711865in}}%
\pgfpathlineto{\pgfqpoint{5.055372in}{1.711178in}}%
\pgfpathlineto{\pgfqpoint{5.107480in}{1.710731in}}%
\pgfpathlineto{\pgfqpoint{5.107765in}{1.734382in}}%
\pgfpathlineto{\pgfqpoint{5.121029in}{1.731778in}}%
\pgfpathlineto{\pgfqpoint{5.127386in}{1.702939in}}%
\pgfpathlineto{\pgfqpoint{5.132072in}{1.692570in}}%
\pgfpathlineto{\pgfqpoint{5.143727in}{1.692264in}}%
\pgfpathlineto{\pgfqpoint{5.146335in}{1.688745in}}%
\pgfpathlineto{\pgfqpoint{5.162592in}{1.687187in}}%
\pgfpathlineto{\pgfqpoint{5.165324in}{1.680037in}}%
\pgfpathlineto{\pgfqpoint{5.176527in}{1.681644in}}%
\pgfpathlineto{\pgfqpoint{5.185229in}{1.688331in}}%
\pgfpathlineto{\pgfqpoint{5.200235in}{1.688081in}}%
\pgfpathlineto{\pgfqpoint{5.209520in}{1.682703in}}%
\pgfpathlineto{\pgfqpoint{5.210587in}{1.677660in}}%
\pgfpathlineto{\pgfqpoint{5.219422in}{1.676604in}}%
\pgfpathlineto{\pgfqpoint{5.225188in}{1.662847in}}%
\pgfpathlineto{\pgfqpoint{5.228908in}{1.671306in}}%
\pgfpathlineto{\pgfqpoint{5.239057in}{1.671826in}}%
\pgfpathlineto{\pgfqpoint{5.241624in}{1.665238in}}%
\pgfpathlineto{\pgfqpoint{5.253043in}{1.662231in}}%
\pgfpathlineto{\pgfqpoint{5.259341in}{1.652742in}}%
\pgfpathlineto{\pgfqpoint{5.273307in}{1.655689in}}%
\pgfpathlineto{\pgfqpoint{5.288676in}{1.667333in}}%
\pgfpathlineto{\pgfqpoint{5.294279in}{1.657068in}}%
\pgfpathlineto{\pgfqpoint{5.319496in}{1.659878in}}%
\pgfpathlineto{\pgfqpoint{5.330276in}{1.652833in}}%
\pgfpathlineto{\pgfqpoint{5.336550in}{1.655348in}}%
\pgfpathlineto{\pgfqpoint{5.341581in}{1.651382in}}%
\pgfpathlineto{\pgfqpoint{5.326656in}{1.641972in}}%
\pgfpathlineto{\pgfqpoint{5.305352in}{1.633588in}}%
\pgfpathlineto{\pgfqpoint{5.284252in}{1.616829in}}%
\pgfpathlineto{\pgfqpoint{5.265959in}{1.594791in}}%
\pgfpathlineto{\pgfqpoint{5.252111in}{1.581761in}}%
\pgfpathlineto{\pgfqpoint{5.239981in}{1.572806in}}%
\pgfpathlineto{\pgfqpoint{5.231638in}{1.564135in}}%
\pgfpathlineto{\pgfqpoint{5.232649in}{1.527191in}}%
\pgfpathlineto{\pgfqpoint{5.216422in}{1.516803in}}%
\pgfpathlineto{\pgfqpoint{5.208227in}{1.503666in}}%
\pgfpathlineto{\pgfqpoint{5.207682in}{1.495350in}}%
\pgfpathlineto{\pgfqpoint{5.214286in}{1.493519in}}%
\pgfpathlineto{\pgfqpoint{5.218035in}{1.484051in}}%
\pgfpathlineto{\pgfqpoint{5.213862in}{1.477780in}}%
\pgfpathlineto{\pgfqpoint{5.214286in}{1.455000in}}%
\pgfpathlineto{\pgfqpoint{5.212088in}{1.443021in}}%
\pgfpathlineto{\pgfqpoint{5.225619in}{1.432612in}}%
\pgfpathlineto{\pgfqpoint{5.233198in}{1.431422in}}%
\pgfpathlineto{\pgfqpoint{5.239454in}{1.424411in}}%
\pgfpathlineto{\pgfqpoint{5.250096in}{1.420104in}}%
\pgfpathlineto{\pgfqpoint{5.256027in}{1.409407in}}%
\pgfpathlineto{\pgfqpoint{5.267500in}{1.400366in}}%
\pgfpathlineto{\pgfqpoint{5.278044in}{1.394887in}}%
\pgfpathlineto{\pgfqpoint{5.283981in}{1.385126in}}%
\pgfpathlineto{\pgfqpoint{5.285978in}{1.367156in}}%
\pgfpathlineto{\pgfqpoint{5.230022in}{1.365124in}}%
\pgfpathlineto{\pgfqpoint{5.182325in}{1.364126in}}%
\pgfpathlineto{\pgfqpoint{5.138869in}{1.363487in}}%
\pgfpathlineto{\pgfqpoint{5.092897in}{1.363558in}}%
\pgfpathclose%
\pgfusepath{fill}%
\end{pgfscope}%
\begin{pgfscope}%
\pgfpathrectangle{\pgfqpoint{3.525000in}{0.100000in}}{\pgfqpoint{2.857344in}{1.829167in}}%
\pgfusepath{clip}%
\pgfsetbuttcap%
\pgfsetmiterjoin%
\definecolor{currentfill}{rgb}{0.527336,0.810611,0.645213}%
\pgfsetfillcolor{currentfill}%
\pgfsetlinewidth{0.000000pt}%
\definecolor{currentstroke}{rgb}{0.000000,0.000000,0.000000}%
\pgfsetstrokecolor{currentstroke}%
\pgfsetstrokeopacity{0.000000}%
\pgfsetdash{}{0pt}%
\pgfpathmoveto{\pgfqpoint{3.777095in}{1.486015in}}%
\pgfpathlineto{\pgfqpoint{3.772663in}{1.494169in}}%
\pgfpathlineto{\pgfqpoint{3.775493in}{1.515099in}}%
\pgfpathlineto{\pgfqpoint{3.780954in}{1.526038in}}%
\pgfpathlineto{\pgfqpoint{3.778224in}{1.540835in}}%
\pgfpathlineto{\pgfqpoint{3.783921in}{1.547057in}}%
\pgfpathlineto{\pgfqpoint{3.794315in}{1.563830in}}%
\pgfpathlineto{\pgfqpoint{3.803128in}{1.573954in}}%
\pgfpathlineto{\pgfqpoint{3.816011in}{1.596006in}}%
\pgfpathlineto{\pgfqpoint{3.825891in}{1.620027in}}%
\pgfpathlineto{\pgfqpoint{3.836380in}{1.642556in}}%
\pgfpathlineto{\pgfqpoint{3.838500in}{1.651965in}}%
\pgfpathlineto{\pgfqpoint{3.852970in}{1.678883in}}%
\pgfpathlineto{\pgfqpoint{3.855756in}{1.690705in}}%
\pgfpathlineto{\pgfqpoint{3.861917in}{1.703119in}}%
\pgfpathlineto{\pgfqpoint{3.864119in}{1.718264in}}%
\pgfpathlineto{\pgfqpoint{3.875880in}{1.725648in}}%
\pgfpathlineto{\pgfqpoint{3.889826in}{1.729370in}}%
\pgfpathlineto{\pgfqpoint{3.896876in}{1.721034in}}%
\pgfpathlineto{\pgfqpoint{3.902972in}{1.721650in}}%
\pgfpathlineto{\pgfqpoint{3.912489in}{1.711715in}}%
\pgfpathlineto{\pgfqpoint{3.913754in}{1.702356in}}%
\pgfpathlineto{\pgfqpoint{3.910145in}{1.684029in}}%
\pgfpathlineto{\pgfqpoint{3.922153in}{1.674678in}}%
\pgfpathlineto{\pgfqpoint{3.929941in}{1.671163in}}%
\pgfpathlineto{\pgfqpoint{3.951039in}{1.674751in}}%
\pgfpathlineto{\pgfqpoint{3.963262in}{1.672300in}}%
\pgfpathlineto{\pgfqpoint{3.971663in}{1.668278in}}%
\pgfpathlineto{\pgfqpoint{3.976088in}{1.660644in}}%
\pgfpathlineto{\pgfqpoint{4.004110in}{1.661663in}}%
\pgfpathlineto{\pgfqpoint{4.008565in}{1.656991in}}%
\pgfpathlineto{\pgfqpoint{4.019077in}{1.655983in}}%
\pgfpathlineto{\pgfqpoint{4.029663in}{1.658968in}}%
\pgfpathlineto{\pgfqpoint{4.060962in}{1.658216in}}%
\pgfpathlineto{\pgfqpoint{4.067722in}{1.655916in}}%
\pgfpathlineto{\pgfqpoint{4.075622in}{1.658566in}}%
\pgfpathlineto{\pgfqpoint{4.163069in}{1.635747in}}%
\pgfpathlineto{\pgfqpoint{4.166632in}{1.623486in}}%
\pgfpathlineto{\pgfqpoint{4.174657in}{1.616844in}}%
\pgfpathlineto{\pgfqpoint{4.175490in}{1.604886in}}%
\pgfpathlineto{\pgfqpoint{4.163753in}{1.592778in}}%
\pgfpathlineto{\pgfqpoint{4.158032in}{1.581334in}}%
\pgfpathlineto{\pgfqpoint{4.151141in}{1.575220in}}%
\pgfpathlineto{\pgfqpoint{4.150517in}{1.569101in}}%
\pgfpathlineto{\pgfqpoint{4.139100in}{1.560831in}}%
\pgfpathlineto{\pgfqpoint{4.131268in}{1.548877in}}%
\pgfpathlineto{\pgfqpoint{4.126882in}{1.545974in}}%
\pgfpathlineto{\pgfqpoint{4.124490in}{1.534373in}}%
\pgfpathlineto{\pgfqpoint{4.136011in}{1.522559in}}%
\pgfpathlineto{\pgfqpoint{4.124884in}{1.502905in}}%
\pgfpathlineto{\pgfqpoint{4.123535in}{1.495404in}}%
\pgfpathlineto{\pgfqpoint{4.098515in}{1.392762in}}%
\pgfpathlineto{\pgfqpoint{4.045888in}{1.406247in}}%
\pgfpathlineto{\pgfqpoint{3.995118in}{1.419338in}}%
\pgfpathlineto{\pgfqpoint{3.964490in}{1.427861in}}%
\pgfpathlineto{\pgfqpoint{3.925137in}{1.439071in}}%
\pgfpathlineto{\pgfqpoint{3.862366in}{1.458829in}}%
\pgfpathlineto{\pgfqpoint{3.794129in}{1.480073in}}%
\pgfpathclose%
\pgfusepath{fill}%
\end{pgfscope}%
\begin{pgfscope}%
\pgfpathrectangle{\pgfqpoint{3.525000in}{0.100000in}}{\pgfqpoint{2.857344in}{1.829167in}}%
\pgfusepath{clip}%
\pgfsetbuttcap%
\pgfsetmiterjoin%
\definecolor{currentfill}{rgb}{0.765859,0.905421,0.623760}%
\pgfsetfillcolor{currentfill}%
\pgfsetlinewidth{0.000000pt}%
\definecolor{currentstroke}{rgb}{0.000000,0.000000,0.000000}%
\pgfsetstrokecolor{currentstroke}%
\pgfsetstrokeopacity{0.000000}%
\pgfsetdash{}{0pt}%
\pgfpathmoveto{\pgfqpoint{6.142091in}{1.443130in}}%
\pgfpathlineto{\pgfqpoint{6.137121in}{1.447171in}}%
\pgfpathlineto{\pgfqpoint{6.137813in}{1.459760in}}%
\pgfpathlineto{\pgfqpoint{6.132676in}{1.495543in}}%
\pgfpathlineto{\pgfqpoint{6.136703in}{1.508126in}}%
\pgfpathlineto{\pgfqpoint{6.139579in}{1.530443in}}%
\pgfpathlineto{\pgfqpoint{6.135698in}{1.541001in}}%
\pgfpathlineto{\pgfqpoint{6.149817in}{1.555057in}}%
\pgfpathlineto{\pgfqpoint{6.153370in}{1.566239in}}%
\pgfpathlineto{\pgfqpoint{6.146681in}{1.575063in}}%
\pgfpathlineto{\pgfqpoint{6.149530in}{1.586091in}}%
\pgfpathlineto{\pgfqpoint{6.147744in}{1.593019in}}%
\pgfpathlineto{\pgfqpoint{6.149278in}{1.607855in}}%
\pgfpathlineto{\pgfqpoint{6.152146in}{1.612435in}}%
\pgfpathlineto{\pgfqpoint{6.160989in}{1.615326in}}%
\pgfpathlineto{\pgfqpoint{6.173887in}{1.577551in}}%
\pgfpathlineto{\pgfqpoint{6.197818in}{1.499273in}}%
\pgfpathlineto{\pgfqpoint{6.206874in}{1.493203in}}%
\pgfpathlineto{\pgfqpoint{6.208503in}{1.486275in}}%
\pgfpathlineto{\pgfqpoint{6.213283in}{1.483476in}}%
\pgfpathlineto{\pgfqpoint{6.212919in}{1.470910in}}%
\pgfpathlineto{\pgfqpoint{6.207855in}{1.470708in}}%
\pgfpathlineto{\pgfqpoint{6.197595in}{1.462884in}}%
\pgfpathlineto{\pgfqpoint{6.194636in}{1.455033in}}%
\pgfpathlineto{\pgfqpoint{6.167174in}{1.448239in}}%
\pgfpathclose%
\pgfusepath{fill}%
\end{pgfscope}%
\begin{pgfscope}%
\pgfpathrectangle{\pgfqpoint{3.525000in}{0.100000in}}{\pgfqpoint{2.857344in}{1.829167in}}%
\pgfusepath{clip}%
\pgfsetbuttcap%
\pgfsetmiterjoin%
\definecolor{currentfill}{rgb}{0.838447,0.934948,0.608997}%
\pgfsetfillcolor{currentfill}%
\pgfsetlinewidth{0.000000pt}%
\definecolor{currentstroke}{rgb}{0.000000,0.000000,0.000000}%
\pgfsetstrokecolor{currentstroke}%
\pgfsetstrokeopacity{0.000000}%
\pgfsetdash{}{0pt}%
\pgfpathmoveto{\pgfqpoint{5.283393in}{1.170794in}}%
\pgfpathlineto{\pgfqpoint{5.268896in}{1.185151in}}%
\pgfpathlineto{\pgfqpoint{5.222555in}{1.182478in}}%
\pgfpathlineto{\pgfqpoint{5.150281in}{1.180096in}}%
\pgfpathlineto{\pgfqpoint{5.077572in}{1.181231in}}%
\pgfpathlineto{\pgfqpoint{5.072469in}{1.190130in}}%
\pgfpathlineto{\pgfqpoint{5.074552in}{1.198868in}}%
\pgfpathlineto{\pgfqpoint{5.073563in}{1.213838in}}%
\pgfpathlineto{\pgfqpoint{5.069720in}{1.228337in}}%
\pgfpathlineto{\pgfqpoint{5.070167in}{1.235994in}}%
\pgfpathlineto{\pgfqpoint{5.061999in}{1.244645in}}%
\pgfpathlineto{\pgfqpoint{5.063776in}{1.256683in}}%
\pgfpathlineto{\pgfqpoint{5.051247in}{1.280461in}}%
\pgfpathlineto{\pgfqpoint{5.047514in}{1.300647in}}%
\pgfpathlineto{\pgfqpoint{5.038501in}{1.317067in}}%
\pgfpathlineto{\pgfqpoint{5.047056in}{1.337435in}}%
\pgfpathlineto{\pgfqpoint{5.042795in}{1.346914in}}%
\pgfpathlineto{\pgfqpoint{5.047559in}{1.364049in}}%
\pgfpathlineto{\pgfqpoint{5.092897in}{1.363558in}}%
\pgfpathlineto{\pgfqpoint{5.138869in}{1.363487in}}%
\pgfpathlineto{\pgfqpoint{5.182325in}{1.364126in}}%
\pgfpathlineto{\pgfqpoint{5.230022in}{1.365124in}}%
\pgfpathlineto{\pgfqpoint{5.285978in}{1.367156in}}%
\pgfpathlineto{\pgfqpoint{5.287472in}{1.359032in}}%
\pgfpathlineto{\pgfqpoint{5.293667in}{1.352221in}}%
\pgfpathlineto{\pgfqpoint{5.289744in}{1.345124in}}%
\pgfpathlineto{\pgfqpoint{5.291352in}{1.331272in}}%
\pgfpathlineto{\pgfqpoint{5.295191in}{1.320074in}}%
\pgfpathlineto{\pgfqpoint{5.312596in}{1.314147in}}%
\pgfpathlineto{\pgfqpoint{5.315579in}{1.306219in}}%
\pgfpathlineto{\pgfqpoint{5.325130in}{1.297317in}}%
\pgfpathlineto{\pgfqpoint{5.329026in}{1.288109in}}%
\pgfpathlineto{\pgfqpoint{5.338704in}{1.281931in}}%
\pgfpathlineto{\pgfqpoint{5.340219in}{1.274495in}}%
\pgfpathlineto{\pgfqpoint{5.338334in}{1.263236in}}%
\pgfpathlineto{\pgfqpoint{5.333409in}{1.259862in}}%
\pgfpathlineto{\pgfqpoint{5.331923in}{1.249144in}}%
\pgfpathlineto{\pgfqpoint{5.317767in}{1.240633in}}%
\pgfpathlineto{\pgfqpoint{5.299315in}{1.235970in}}%
\pgfpathlineto{\pgfqpoint{5.297625in}{1.225248in}}%
\pgfpathlineto{\pgfqpoint{5.304743in}{1.217587in}}%
\pgfpathlineto{\pgfqpoint{5.305033in}{1.207956in}}%
\pgfpathlineto{\pgfqpoint{5.299290in}{1.200386in}}%
\pgfpathlineto{\pgfqpoint{5.296276in}{1.189138in}}%
\pgfpathlineto{\pgfqpoint{5.286293in}{1.185415in}}%
\pgfpathlineto{\pgfqpoint{5.286930in}{1.172881in}}%
\pgfpathclose%
\pgfusepath{fill}%
\end{pgfscope}%
\begin{pgfscope}%
\pgfpathrectangle{\pgfqpoint{3.525000in}{0.100000in}}{\pgfqpoint{2.857344in}{1.829167in}}%
\pgfusepath{clip}%
\pgfsetbuttcap%
\pgfsetmiterjoin%
\definecolor{currentfill}{rgb}{0.644060,0.856286,0.643522}%
\pgfsetfillcolor{currentfill}%
\pgfsetlinewidth{0.000000pt}%
\definecolor{currentstroke}{rgb}{0.000000,0.000000,0.000000}%
\pgfsetstrokecolor{currentstroke}%
\pgfsetstrokeopacity{0.000000}%
\pgfsetdash{}{0pt}%
\pgfpathmoveto{\pgfqpoint{6.182873in}{1.406908in}}%
\pgfpathlineto{\pgfqpoint{6.169142in}{1.404742in}}%
\pgfpathlineto{\pgfqpoint{6.106070in}{1.390410in}}%
\pgfpathlineto{\pgfqpoint{6.104968in}{1.392080in}}%
\pgfpathlineto{\pgfqpoint{6.105920in}{1.435444in}}%
\pgfpathlineto{\pgfqpoint{6.142091in}{1.443130in}}%
\pgfpathlineto{\pgfqpoint{6.167174in}{1.448239in}}%
\pgfpathlineto{\pgfqpoint{6.194636in}{1.455033in}}%
\pgfpathlineto{\pgfqpoint{6.197595in}{1.462884in}}%
\pgfpathlineto{\pgfqpoint{6.207855in}{1.470708in}}%
\pgfpathlineto{\pgfqpoint{6.212919in}{1.470910in}}%
\pgfpathlineto{\pgfqpoint{6.219577in}{1.459495in}}%
\pgfpathlineto{\pgfqpoint{6.213536in}{1.442831in}}%
\pgfpathlineto{\pgfqpoint{6.212649in}{1.433059in}}%
\pgfpathlineto{\pgfqpoint{6.224877in}{1.433977in}}%
\pgfpathlineto{\pgfqpoint{6.230422in}{1.429290in}}%
\pgfpathlineto{\pgfqpoint{6.242858in}{1.410130in}}%
\pgfpathlineto{\pgfqpoint{6.249039in}{1.407800in}}%
\pgfpathlineto{\pgfqpoint{6.259392in}{1.408664in}}%
\pgfpathlineto{\pgfqpoint{6.266597in}{1.415175in}}%
\pgfpathlineto{\pgfqpoint{6.271386in}{1.409365in}}%
\pgfpathlineto{\pgfqpoint{6.241269in}{1.393472in}}%
\pgfpathlineto{\pgfqpoint{6.240328in}{1.404876in}}%
\pgfpathlineto{\pgfqpoint{6.226647in}{1.387152in}}%
\pgfpathlineto{\pgfqpoint{6.221836in}{1.384098in}}%
\pgfpathlineto{\pgfqpoint{6.215127in}{1.394325in}}%
\pgfpathlineto{\pgfqpoint{6.213291in}{1.395727in}}%
\pgfpathlineto{\pgfqpoint{6.207045in}{1.399035in}}%
\pgfpathlineto{\pgfqpoint{6.201598in}{1.412454in}}%
\pgfpathclose%
\pgfusepath{fill}%
\end{pgfscope}%
\begin{pgfscope}%
\pgfpathrectangle{\pgfqpoint{3.525000in}{0.100000in}}{\pgfqpoint{2.857344in}{1.829167in}}%
\pgfusepath{clip}%
\pgfsetbuttcap%
\pgfsetmiterjoin%
\definecolor{currentfill}{rgb}{0.495502,0.798155,0.645675}%
\pgfsetfillcolor{currentfill}%
\pgfsetlinewidth{0.000000pt}%
\definecolor{currentstroke}{rgb}{0.000000,0.000000,0.000000}%
\pgfsetstrokecolor{currentstroke}%
\pgfsetstrokeopacity{0.000000}%
\pgfsetdash{}{0pt}%
\pgfpathmoveto{\pgfqpoint{4.776269in}{1.157246in}}%
\pgfpathlineto{\pgfqpoint{4.781297in}{1.219598in}}%
\pgfpathlineto{\pgfqpoint{4.752819in}{1.221909in}}%
\pgfpathlineto{\pgfqpoint{4.686989in}{1.228296in}}%
\pgfpathlineto{\pgfqpoint{4.693113in}{1.290427in}}%
\pgfpathlineto{\pgfqpoint{4.699247in}{1.353032in}}%
\pgfpathlineto{\pgfqpoint{4.724554in}{1.350300in}}%
\pgfpathlineto{\pgfqpoint{4.789186in}{1.344193in}}%
\pgfpathlineto{\pgfqpoint{4.828301in}{1.341130in}}%
\pgfpathlineto{\pgfqpoint{4.875656in}{1.338416in}}%
\pgfpathlineto{\pgfqpoint{4.953600in}{1.334727in}}%
\pgfpathlineto{\pgfqpoint{4.955313in}{1.331232in}}%
\pgfpathlineto{\pgfqpoint{4.977647in}{1.320183in}}%
\pgfpathlineto{\pgfqpoint{4.984496in}{1.325794in}}%
\pgfpathlineto{\pgfqpoint{5.004228in}{1.325282in}}%
\pgfpathlineto{\pgfqpoint{5.015986in}{1.319122in}}%
\pgfpathlineto{\pgfqpoint{5.034632in}{1.312120in}}%
\pgfpathlineto{\pgfqpoint{5.039856in}{1.301904in}}%
\pgfpathlineto{\pgfqpoint{5.047514in}{1.300647in}}%
\pgfpathlineto{\pgfqpoint{5.051247in}{1.280461in}}%
\pgfpathlineto{\pgfqpoint{5.063776in}{1.256683in}}%
\pgfpathlineto{\pgfqpoint{5.061999in}{1.244645in}}%
\pgfpathlineto{\pgfqpoint{5.070167in}{1.235994in}}%
\pgfpathlineto{\pgfqpoint{5.069720in}{1.228337in}}%
\pgfpathlineto{\pgfqpoint{5.073563in}{1.213838in}}%
\pgfpathlineto{\pgfqpoint{5.074552in}{1.198868in}}%
\pgfpathlineto{\pgfqpoint{5.072469in}{1.190130in}}%
\pgfpathlineto{\pgfqpoint{5.077572in}{1.181231in}}%
\pgfpathlineto{\pgfqpoint{5.084572in}{1.165048in}}%
\pgfpathlineto{\pgfqpoint{5.091272in}{1.158462in}}%
\pgfpathlineto{\pgfqpoint{5.099259in}{1.144189in}}%
\pgfpathlineto{\pgfqpoint{5.076623in}{1.143955in}}%
\pgfpathlineto{\pgfqpoint{5.027681in}{1.144712in}}%
\pgfpathlineto{\pgfqpoint{4.973607in}{1.146369in}}%
\pgfpathlineto{\pgfqpoint{4.919214in}{1.148456in}}%
\pgfpathlineto{\pgfqpoint{4.839241in}{1.152824in}}%
\pgfpathclose%
\pgfusepath{fill}%
\end{pgfscope}%
\begin{pgfscope}%
\pgfpathrectangle{\pgfqpoint{3.525000in}{0.100000in}}{\pgfqpoint{2.857344in}{1.829167in}}%
\pgfusepath{clip}%
\pgfsetbuttcap%
\pgfsetmiterjoin%
\definecolor{currentfill}{rgb}{0.665283,0.864591,0.643214}%
\pgfsetfillcolor{currentfill}%
\pgfsetlinewidth{0.000000pt}%
\definecolor{currentstroke}{rgb}{0.000000,0.000000,0.000000}%
\pgfsetstrokecolor{currentstroke}%
\pgfsetstrokeopacity{0.000000}%
\pgfsetdash{}{0pt}%
\pgfpathmoveto{\pgfqpoint{5.817558in}{1.345322in}}%
\pgfpathlineto{\pgfqpoint{5.842600in}{1.369200in}}%
\pgfpathlineto{\pgfqpoint{5.845713in}{1.377687in}}%
\pgfpathlineto{\pgfqpoint{5.853050in}{1.384953in}}%
\pgfpathlineto{\pgfqpoint{5.847544in}{1.395532in}}%
\pgfpathlineto{\pgfqpoint{5.840641in}{1.401720in}}%
\pgfpathlineto{\pgfqpoint{5.838646in}{1.412685in}}%
\pgfpathlineto{\pgfqpoint{5.864340in}{1.423944in}}%
\pgfpathlineto{\pgfqpoint{5.885614in}{1.427505in}}%
\pgfpathlineto{\pgfqpoint{5.897045in}{1.427743in}}%
\pgfpathlineto{\pgfqpoint{5.905753in}{1.423441in}}%
\pgfpathlineto{\pgfqpoint{5.914241in}{1.427283in}}%
\pgfpathlineto{\pgfqpoint{5.934945in}{1.431588in}}%
\pgfpathlineto{\pgfqpoint{5.942098in}{1.437145in}}%
\pgfpathlineto{\pgfqpoint{5.952705in}{1.449447in}}%
\pgfpathlineto{\pgfqpoint{5.962355in}{1.454884in}}%
\pgfpathlineto{\pgfqpoint{5.963035in}{1.460097in}}%
\pgfpathlineto{\pgfqpoint{5.957971in}{1.471992in}}%
\pgfpathlineto{\pgfqpoint{5.961632in}{1.478992in}}%
\pgfpathlineto{\pgfqpoint{5.956703in}{1.486519in}}%
\pgfpathlineto{\pgfqpoint{5.949154in}{1.487044in}}%
\pgfpathlineto{\pgfqpoint{5.968047in}{1.509780in}}%
\pgfpathlineto{\pgfqpoint{5.970291in}{1.518444in}}%
\pgfpathlineto{\pgfqpoint{5.985162in}{1.540542in}}%
\pgfpathlineto{\pgfqpoint{5.998966in}{1.552500in}}%
\pgfpathlineto{\pgfqpoint{6.008444in}{1.557494in}}%
\pgfpathlineto{\pgfqpoint{6.039461in}{1.564521in}}%
\pgfpathlineto{\pgfqpoint{6.068531in}{1.572458in}}%
\pgfpathlineto{\pgfqpoint{6.072381in}{1.560000in}}%
\pgfpathlineto{\pgfqpoint{6.072654in}{1.548449in}}%
\pgfpathlineto{\pgfqpoint{6.079285in}{1.537962in}}%
\pgfpathlineto{\pgfqpoint{6.081328in}{1.527286in}}%
\pgfpathlineto{\pgfqpoint{6.078780in}{1.512801in}}%
\pgfpathlineto{\pgfqpoint{6.086541in}{1.493263in}}%
\pgfpathlineto{\pgfqpoint{6.095389in}{1.482398in}}%
\pgfpathlineto{\pgfqpoint{6.102232in}{1.447085in}}%
\pgfpathlineto{\pgfqpoint{6.105920in}{1.435444in}}%
\pgfpathlineto{\pgfqpoint{6.104968in}{1.392080in}}%
\pgfpathlineto{\pgfqpoint{6.106070in}{1.390410in}}%
\pgfpathlineto{\pgfqpoint{6.114127in}{1.343772in}}%
\pgfpathlineto{\pgfqpoint{6.118645in}{1.339522in}}%
\pgfpathlineto{\pgfqpoint{6.108927in}{1.330080in}}%
\pgfpathlineto{\pgfqpoint{6.113722in}{1.324660in}}%
\pgfpathlineto{\pgfqpoint{6.109507in}{1.316463in}}%
\pgfpathlineto{\pgfqpoint{6.109543in}{1.312974in}}%
\pgfpathlineto{\pgfqpoint{6.104274in}{1.309825in}}%
\pgfpathlineto{\pgfqpoint{6.101708in}{1.302864in}}%
\pgfpathlineto{\pgfqpoint{6.102522in}{1.322005in}}%
\pgfpathlineto{\pgfqpoint{6.086190in}{1.326218in}}%
\pgfpathlineto{\pgfqpoint{6.060649in}{1.334930in}}%
\pgfpathlineto{\pgfqpoint{6.057037in}{1.339222in}}%
\pgfpathlineto{\pgfqpoint{6.046364in}{1.340250in}}%
\pgfpathlineto{\pgfqpoint{6.040225in}{1.346637in}}%
\pgfpathlineto{\pgfqpoint{6.036912in}{1.359358in}}%
\pgfpathlineto{\pgfqpoint{6.028209in}{1.360940in}}%
\pgfpathlineto{\pgfqpoint{6.022389in}{1.367613in}}%
\pgfpathlineto{\pgfqpoint{5.948292in}{1.352675in}}%
\pgfpathlineto{\pgfqpoint{5.912880in}{1.345351in}}%
\pgfpathlineto{\pgfqpoint{5.859113in}{1.335538in}}%
\pgfpathlineto{\pgfqpoint{5.820394in}{1.329006in}}%
\pgfpathclose%
\pgfusepath{fill}%
\end{pgfscope}%
\begin{pgfscope}%
\pgfpathrectangle{\pgfqpoint{3.525000in}{0.100000in}}{\pgfqpoint{2.857344in}{1.829167in}}%
\pgfusepath{clip}%
\pgfsetbuttcap%
\pgfsetmiterjoin%
\definecolor{currentfill}{rgb}{0.665283,0.864591,0.643214}%
\pgfsetfillcolor{currentfill}%
\pgfsetlinewidth{0.000000pt}%
\definecolor{currentstroke}{rgb}{0.000000,0.000000,0.000000}%
\pgfsetstrokecolor{currentstroke}%
\pgfsetstrokeopacity{0.000000}%
\pgfsetdash{}{0pt}%
\pgfpathmoveto{\pgfqpoint{6.115026in}{1.298977in}}%
\pgfpathlineto{\pgfqpoint{6.103567in}{1.295407in}}%
\pgfpathlineto{\pgfqpoint{6.101639in}{1.298689in}}%
\pgfpathlineto{\pgfqpoint{6.105317in}{1.309706in}}%
\pgfpathlineto{\pgfqpoint{6.112129in}{1.310790in}}%
\pgfpathlineto{\pgfqpoint{6.111519in}{1.314259in}}%
\pgfpathlineto{\pgfqpoint{6.117636in}{1.319465in}}%
\pgfpathlineto{\pgfqpoint{6.135319in}{1.323612in}}%
\pgfpathlineto{\pgfqpoint{6.143191in}{1.329891in}}%
\pgfpathlineto{\pgfqpoint{6.160875in}{1.335123in}}%
\pgfpathlineto{\pgfqpoint{6.168941in}{1.333208in}}%
\pgfpathlineto{\pgfqpoint{6.175726in}{1.341643in}}%
\pgfpathlineto{\pgfqpoint{6.185981in}{1.342758in}}%
\pgfpathlineto{\pgfqpoint{6.168486in}{1.326304in}}%
\pgfpathclose%
\pgfusepath{fill}%
\end{pgfscope}%
\begin{pgfscope}%
\pgfpathrectangle{\pgfqpoint{3.525000in}{0.100000in}}{\pgfqpoint{2.857344in}{1.829167in}}%
\pgfusepath{clip}%
\pgfsetbuttcap%
\pgfsetmiterjoin%
\definecolor{currentfill}{rgb}{0.612226,0.843829,0.643983}%
\pgfsetfillcolor{currentfill}%
\pgfsetlinewidth{0.000000pt}%
\definecolor{currentstroke}{rgb}{0.000000,0.000000,0.000000}%
\pgfsetstrokecolor{currentstroke}%
\pgfsetstrokeopacity{0.000000}%
\pgfsetdash{}{0pt}%
\pgfpathmoveto{\pgfqpoint{5.857573in}{1.190332in}}%
\pgfpathlineto{\pgfqpoint{5.808019in}{1.182138in}}%
\pgfpathlineto{\pgfqpoint{5.799030in}{1.238770in}}%
\pgfpathlineto{\pgfqpoint{5.785683in}{1.322248in}}%
\pgfpathlineto{\pgfqpoint{5.817558in}{1.345322in}}%
\pgfpathlineto{\pgfqpoint{5.820394in}{1.329006in}}%
\pgfpathlineto{\pgfqpoint{5.859113in}{1.335538in}}%
\pgfpathlineto{\pgfqpoint{5.912880in}{1.345351in}}%
\pgfpathlineto{\pgfqpoint{5.948292in}{1.352675in}}%
\pgfpathlineto{\pgfqpoint{6.022389in}{1.367613in}}%
\pgfpathlineto{\pgfqpoint{6.028209in}{1.360940in}}%
\pgfpathlineto{\pgfqpoint{6.036912in}{1.359358in}}%
\pgfpathlineto{\pgfqpoint{6.040225in}{1.346637in}}%
\pgfpathlineto{\pgfqpoint{6.046364in}{1.340250in}}%
\pgfpathlineto{\pgfqpoint{6.057037in}{1.339222in}}%
\pgfpathlineto{\pgfqpoint{6.060649in}{1.334930in}}%
\pgfpathlineto{\pgfqpoint{6.056982in}{1.331620in}}%
\pgfpathlineto{\pgfqpoint{6.053678in}{1.319905in}}%
\pgfpathlineto{\pgfqpoint{6.045936in}{1.306743in}}%
\pgfpathlineto{\pgfqpoint{6.051133in}{1.301019in}}%
\pgfpathlineto{\pgfqpoint{6.046880in}{1.294886in}}%
\pgfpathlineto{\pgfqpoint{6.049279in}{1.281436in}}%
\pgfpathlineto{\pgfqpoint{6.056938in}{1.274376in}}%
\pgfpathlineto{\pgfqpoint{6.075110in}{1.262906in}}%
\pgfpathlineto{\pgfqpoint{6.060476in}{1.246723in}}%
\pgfpathlineto{\pgfqpoint{6.060270in}{1.240580in}}%
\pgfpathlineto{\pgfqpoint{6.048356in}{1.232658in}}%
\pgfpathlineto{\pgfqpoint{6.035183in}{1.231171in}}%
\pgfpathlineto{\pgfqpoint{6.031924in}{1.224286in}}%
\pgfpathlineto{\pgfqpoint{5.995281in}{1.216357in}}%
\pgfpathlineto{\pgfqpoint{5.952524in}{1.207746in}}%
\pgfpathlineto{\pgfqpoint{5.923134in}{1.202483in}}%
\pgfpathclose%
\pgfusepath{fill}%
\end{pgfscope}%
\begin{pgfscope}%
\pgfpathrectangle{\pgfqpoint{3.525000in}{0.100000in}}{\pgfqpoint{2.857344in}{1.829167in}}%
\pgfusepath{clip}%
\pgfsetbuttcap%
\pgfsetmiterjoin%
\definecolor{currentfill}{rgb}{0.720492,0.886967,0.632987}%
\pgfsetfillcolor{currentfill}%
\pgfsetlinewidth{0.000000pt}%
\definecolor{currentstroke}{rgb}{0.000000,0.000000,0.000000}%
\pgfsetstrokecolor{currentstroke}%
\pgfsetstrokeopacity{0.000000}%
\pgfsetdash{}{0pt}%
\pgfpathmoveto{\pgfqpoint{6.106070in}{1.390410in}}%
\pgfpathlineto{\pgfqpoint{6.169142in}{1.404742in}}%
\pgfpathlineto{\pgfqpoint{6.182873in}{1.406908in}}%
\pgfpathlineto{\pgfqpoint{6.191961in}{1.371177in}}%
\pgfpathlineto{\pgfqpoint{6.190523in}{1.364778in}}%
\pgfpathlineto{\pgfqpoint{6.161329in}{1.353476in}}%
\pgfpathlineto{\pgfqpoint{6.143901in}{1.349526in}}%
\pgfpathlineto{\pgfqpoint{6.136495in}{1.340664in}}%
\pgfpathlineto{\pgfqpoint{6.113722in}{1.324660in}}%
\pgfpathlineto{\pgfqpoint{6.108927in}{1.330080in}}%
\pgfpathlineto{\pgfqpoint{6.118645in}{1.339522in}}%
\pgfpathlineto{\pgfqpoint{6.114127in}{1.343772in}}%
\pgfpathclose%
\pgfusepath{fill}%
\end{pgfscope}%
\begin{pgfscope}%
\pgfpathrectangle{\pgfqpoint{3.525000in}{0.100000in}}{\pgfqpoint{2.857344in}{1.829167in}}%
\pgfusepath{clip}%
\pgfsetbuttcap%
\pgfsetmiterjoin%
\definecolor{currentfill}{rgb}{0.874740,0.949712,0.601615}%
\pgfsetfillcolor{currentfill}%
\pgfsetlinewidth{0.000000pt}%
\definecolor{currentstroke}{rgb}{0.000000,0.000000,0.000000}%
\pgfsetstrokecolor{currentstroke}%
\pgfsetstrokeopacity{0.000000}%
\pgfsetdash{}{0pt}%
\pgfpathmoveto{\pgfqpoint{6.182873in}{1.406908in}}%
\pgfpathlineto{\pgfqpoint{6.201598in}{1.412454in}}%
\pgfpathlineto{\pgfqpoint{6.207045in}{1.399035in}}%
\pgfpathlineto{\pgfqpoint{6.213291in}{1.395727in}}%
\pgfpathlineto{\pgfqpoint{6.205583in}{1.390075in}}%
\pgfpathlineto{\pgfqpoint{6.206555in}{1.373474in}}%
\pgfpathlineto{\pgfqpoint{6.190523in}{1.364778in}}%
\pgfpathlineto{\pgfqpoint{6.191961in}{1.371177in}}%
\pgfpathclose%
\pgfusepath{fill}%
\end{pgfscope}%
\begin{pgfscope}%
\pgfpathrectangle{\pgfqpoint{3.525000in}{0.100000in}}{\pgfqpoint{2.857344in}{1.829167in}}%
\pgfusepath{clip}%
\pgfsetbuttcap%
\pgfsetmiterjoin%
\definecolor{currentfill}{rgb}{0.756786,0.901730,0.625606}%
\pgfsetfillcolor{currentfill}%
\pgfsetlinewidth{0.000000pt}%
\definecolor{currentstroke}{rgb}{0.000000,0.000000,0.000000}%
\pgfsetstrokecolor{currentstroke}%
\pgfsetstrokeopacity{0.000000}%
\pgfsetdash{}{0pt}%
\pgfpathmoveto{\pgfqpoint{6.046245in}{1.226807in}}%
\pgfpathlineto{\pgfqpoint{6.048356in}{1.232658in}}%
\pgfpathlineto{\pgfqpoint{6.060270in}{1.240580in}}%
\pgfpathlineto{\pgfqpoint{6.060476in}{1.246723in}}%
\pgfpathlineto{\pgfqpoint{6.075110in}{1.262906in}}%
\pgfpathlineto{\pgfqpoint{6.056938in}{1.274376in}}%
\pgfpathlineto{\pgfqpoint{6.049279in}{1.281436in}}%
\pgfpathlineto{\pgfqpoint{6.046880in}{1.294886in}}%
\pgfpathlineto{\pgfqpoint{6.051133in}{1.301019in}}%
\pgfpathlineto{\pgfqpoint{6.045936in}{1.306743in}}%
\pgfpathlineto{\pgfqpoint{6.053678in}{1.319905in}}%
\pgfpathlineto{\pgfqpoint{6.056982in}{1.331620in}}%
\pgfpathlineto{\pgfqpoint{6.060649in}{1.334930in}}%
\pgfpathlineto{\pgfqpoint{6.086190in}{1.326218in}}%
\pgfpathlineto{\pgfqpoint{6.102522in}{1.322005in}}%
\pgfpathlineto{\pgfqpoint{6.101708in}{1.302864in}}%
\pgfpathlineto{\pgfqpoint{6.091787in}{1.288350in}}%
\pgfpathlineto{\pgfqpoint{6.099962in}{1.286214in}}%
\pgfpathlineto{\pgfqpoint{6.108449in}{1.279985in}}%
\pgfpathlineto{\pgfqpoint{6.108950in}{1.262949in}}%
\pgfpathlineto{\pgfqpoint{6.106381in}{1.250888in}}%
\pgfpathlineto{\pgfqpoint{6.108091in}{1.240982in}}%
\pgfpathlineto{\pgfqpoint{6.093336in}{1.207554in}}%
\pgfpathlineto{\pgfqpoint{6.088034in}{1.191945in}}%
\pgfpathlineto{\pgfqpoint{6.080630in}{1.199532in}}%
\pgfpathlineto{\pgfqpoint{6.070812in}{1.198239in}}%
\pgfpathlineto{\pgfqpoint{6.046241in}{1.212435in}}%
\pgfpathlineto{\pgfqpoint{6.043727in}{1.220038in}}%
\pgfpathclose%
\pgfusepath{fill}%
\end{pgfscope}%
\begin{pgfscope}%
\pgfpathrectangle{\pgfqpoint{3.525000in}{0.100000in}}{\pgfqpoint{2.857344in}{1.829167in}}%
\pgfusepath{clip}%
\pgfsetbuttcap%
\pgfsetmiterjoin%
\definecolor{currentfill}{rgb}{0.675125,0.868512,0.642215}%
\pgfsetfillcolor{currentfill}%
\pgfsetlinewidth{0.000000pt}%
\definecolor{currentstroke}{rgb}{0.000000,0.000000,0.000000}%
\pgfsetstrokecolor{currentstroke}%
\pgfsetstrokeopacity{0.000000}%
\pgfsetdash{}{0pt}%
\pgfpathmoveto{\pgfqpoint{5.458197in}{1.020052in}}%
\pgfpathlineto{\pgfqpoint{5.457048in}{1.035560in}}%
\pgfpathlineto{\pgfqpoint{5.461513in}{1.043254in}}%
\pgfpathlineto{\pgfqpoint{5.458570in}{1.047023in}}%
\pgfpathlineto{\pgfqpoint{5.469301in}{1.059367in}}%
\pgfpathlineto{\pgfqpoint{5.468686in}{1.062342in}}%
\pgfpathlineto{\pgfqpoint{5.479113in}{1.083876in}}%
\pgfpathlineto{\pgfqpoint{5.477005in}{1.094261in}}%
\pgfpathlineto{\pgfqpoint{5.469445in}{1.105134in}}%
\pgfpathlineto{\pgfqpoint{5.474693in}{1.118362in}}%
\pgfpathlineto{\pgfqpoint{5.467890in}{1.205414in}}%
\pgfpathlineto{\pgfqpoint{5.462992in}{1.266484in}}%
\pgfpathlineto{\pgfqpoint{5.469759in}{1.261424in}}%
\pgfpathlineto{\pgfqpoint{5.477309in}{1.261561in}}%
\pgfpathlineto{\pgfqpoint{5.495140in}{1.271890in}}%
\pgfpathlineto{\pgfqpoint{5.549833in}{1.276993in}}%
\pgfpathlineto{\pgfqpoint{5.590316in}{1.281260in}}%
\pgfpathlineto{\pgfqpoint{5.590674in}{1.277298in}}%
\pgfpathlineto{\pgfqpoint{5.599915in}{1.193611in}}%
\pgfpathlineto{\pgfqpoint{5.607869in}{1.115743in}}%
\pgfpathlineto{\pgfqpoint{5.604440in}{1.112087in}}%
\pgfpathlineto{\pgfqpoint{5.609707in}{1.103013in}}%
\pgfpathlineto{\pgfqpoint{5.609680in}{1.096474in}}%
\pgfpathlineto{\pgfqpoint{5.602165in}{1.094831in}}%
\pgfpathlineto{\pgfqpoint{5.593755in}{1.088527in}}%
\pgfpathlineto{\pgfqpoint{5.588061in}{1.091009in}}%
\pgfpathlineto{\pgfqpoint{5.579539in}{1.086973in}}%
\pgfpathlineto{\pgfqpoint{5.582176in}{1.078868in}}%
\pgfpathlineto{\pgfqpoint{5.573416in}{1.070726in}}%
\pgfpathlineto{\pgfqpoint{5.570998in}{1.061307in}}%
\pgfpathlineto{\pgfqpoint{5.564975in}{1.059762in}}%
\pgfpathlineto{\pgfqpoint{5.560480in}{1.052628in}}%
\pgfpathlineto{\pgfqpoint{5.561068in}{1.045445in}}%
\pgfpathlineto{\pgfqpoint{5.555781in}{1.040393in}}%
\pgfpathlineto{\pgfqpoint{5.547824in}{1.041173in}}%
\pgfpathlineto{\pgfqpoint{5.541776in}{1.048924in}}%
\pgfpathlineto{\pgfqpoint{5.531527in}{1.041459in}}%
\pgfpathlineto{\pgfqpoint{5.532246in}{1.034937in}}%
\pgfpathlineto{\pgfqpoint{5.520848in}{1.031127in}}%
\pgfpathlineto{\pgfqpoint{5.517985in}{1.035923in}}%
\pgfpathlineto{\pgfqpoint{5.506888in}{1.030484in}}%
\pgfpathlineto{\pgfqpoint{5.502835in}{1.022698in}}%
\pgfpathlineto{\pgfqpoint{5.489471in}{1.030682in}}%
\pgfpathlineto{\pgfqpoint{5.468291in}{1.025393in}}%
\pgfpathclose%
\pgfusepath{fill}%
\end{pgfscope}%
\begin{pgfscope}%
\pgfpathrectangle{\pgfqpoint{3.525000in}{0.100000in}}{\pgfqpoint{2.857344in}{1.829167in}}%
\pgfusepath{clip}%
\pgfsetbuttcap%
\pgfsetmiterjoin%
\definecolor{currentfill}{rgb}{0.612226,0.843829,0.643983}%
\pgfsetfillcolor{currentfill}%
\pgfsetlinewidth{0.000000pt}%
\definecolor{currentstroke}{rgb}{0.000000,0.000000,0.000000}%
\pgfsetstrokecolor{currentstroke}%
\pgfsetstrokeopacity{0.000000}%
\pgfsetdash{}{0pt}%
\pgfpathmoveto{\pgfqpoint{3.964490in}{1.427861in}}%
\pgfpathlineto{\pgfqpoint{3.995118in}{1.419338in}}%
\pgfpathlineto{\pgfqpoint{4.045888in}{1.406247in}}%
\pgfpathlineto{\pgfqpoint{4.098515in}{1.392762in}}%
\pgfpathlineto{\pgfqpoint{4.147022in}{1.381064in}}%
\pgfpathlineto{\pgfqpoint{4.188936in}{1.371575in}}%
\pgfpathlineto{\pgfqpoint{4.233777in}{1.361714in}}%
\pgfpathlineto{\pgfqpoint{4.220822in}{1.300572in}}%
\pgfpathlineto{\pgfqpoint{4.209279in}{1.246274in}}%
\pgfpathlineto{\pgfqpoint{4.190345in}{1.158664in}}%
\pgfpathlineto{\pgfqpoint{4.176129in}{1.092538in}}%
\pgfpathlineto{\pgfqpoint{4.168449in}{1.055636in}}%
\pgfpathlineto{\pgfqpoint{4.158601in}{1.007741in}}%
\pgfpathlineto{\pgfqpoint{4.151788in}{0.998025in}}%
\pgfpathlineto{\pgfqpoint{4.146304in}{0.997701in}}%
\pgfpathlineto{\pgfqpoint{4.142393in}{1.006182in}}%
\pgfpathlineto{\pgfqpoint{4.133412in}{1.009262in}}%
\pgfpathlineto{\pgfqpoint{4.123734in}{1.008183in}}%
\pgfpathlineto{\pgfqpoint{4.120988in}{1.001245in}}%
\pgfpathlineto{\pgfqpoint{4.120701in}{0.977154in}}%
\pgfpathlineto{\pgfqpoint{4.117803in}{0.971654in}}%
\pgfpathlineto{\pgfqpoint{4.119459in}{0.952313in}}%
\pgfpathlineto{\pgfqpoint{4.113408in}{0.939428in}}%
\pgfpathlineto{\pgfqpoint{4.064332in}{1.014996in}}%
\pgfpathlineto{\pgfqpoint{4.016002in}{1.088427in}}%
\pgfpathlineto{\pgfqpoint{3.990843in}{1.126933in}}%
\pgfpathlineto{\pgfqpoint{3.969856in}{1.160169in}}%
\pgfpathlineto{\pgfqpoint{3.943412in}{1.201282in}}%
\pgfpathlineto{\pgfqpoint{3.913477in}{1.247399in}}%
\pgfpathlineto{\pgfqpoint{3.925784in}{1.291173in}}%
\pgfpathlineto{\pgfqpoint{3.950551in}{1.378961in}}%
\pgfpathclose%
\pgfusepath{fill}%
\end{pgfscope}%
\begin{pgfscope}%
\pgfpathrectangle{\pgfqpoint{3.525000in}{0.100000in}}{\pgfqpoint{2.857344in}{1.829167in}}%
\pgfusepath{clip}%
\pgfsetbuttcap%
\pgfsetmiterjoin%
\definecolor{currentfill}{rgb}{0.360015,0.716186,0.665513}%
\pgfsetfillcolor{currentfill}%
\pgfsetlinewidth{0.000000pt}%
\definecolor{currentstroke}{rgb}{0.000000,0.000000,0.000000}%
\pgfsetstrokecolor{currentstroke}%
\pgfsetstrokeopacity{0.000000}%
\pgfsetdash{}{0pt}%
\pgfpathmoveto{\pgfqpoint{4.233777in}{1.361714in}}%
\pgfpathlineto{\pgfqpoint{4.281718in}{1.352178in}}%
\pgfpathlineto{\pgfqpoint{4.370471in}{1.335044in}}%
\pgfpathlineto{\pgfqpoint{4.359362in}{1.273417in}}%
\pgfpathlineto{\pgfqpoint{4.408224in}{1.265052in}}%
\pgfpathlineto{\pgfqpoint{4.452739in}{1.257928in}}%
\pgfpathlineto{\pgfqpoint{4.445004in}{1.209206in}}%
\pgfpathlineto{\pgfqpoint{4.436808in}{1.156667in}}%
\pgfpathlineto{\pgfqpoint{4.425834in}{1.087682in}}%
\pgfpathlineto{\pgfqpoint{4.425550in}{1.081899in}}%
\pgfpathlineto{\pgfqpoint{4.414159in}{1.010405in}}%
\pgfpathlineto{\pgfqpoint{4.367269in}{1.017673in}}%
\pgfpathlineto{\pgfqpoint{4.343375in}{1.022471in}}%
\pgfpathlineto{\pgfqpoint{4.257010in}{1.037653in}}%
\pgfpathlineto{\pgfqpoint{4.224487in}{1.044067in}}%
\pgfpathlineto{\pgfqpoint{4.168449in}{1.055636in}}%
\pgfpathlineto{\pgfqpoint{4.176129in}{1.092538in}}%
\pgfpathlineto{\pgfqpoint{4.190345in}{1.158664in}}%
\pgfpathlineto{\pgfqpoint{4.209279in}{1.246274in}}%
\pgfpathlineto{\pgfqpoint{4.220822in}{1.300572in}}%
\pgfpathclose%
\pgfusepath{fill}%
\end{pgfscope}%
\begin{pgfscope}%
\pgfpathrectangle{\pgfqpoint{3.525000in}{0.100000in}}{\pgfqpoint{2.857344in}{1.829167in}}%
\pgfusepath{clip}%
\pgfsetbuttcap%
\pgfsetmiterjoin%
\definecolor{currentfill}{rgb}{0.384006,0.742945,0.654441}%
\pgfsetfillcolor{currentfill}%
\pgfsetlinewidth{0.000000pt}%
\definecolor{currentstroke}{rgb}{0.000000,0.000000,0.000000}%
\pgfsetstrokecolor{currentstroke}%
\pgfsetstrokeopacity{0.000000}%
\pgfsetdash{}{0pt}%
\pgfpathmoveto{\pgfqpoint{3.777095in}{1.486015in}}%
\pgfpathlineto{\pgfqpoint{3.794129in}{1.480073in}}%
\pgfpathlineto{\pgfqpoint{3.862366in}{1.458829in}}%
\pgfpathlineto{\pgfqpoint{3.925137in}{1.439071in}}%
\pgfpathlineto{\pgfqpoint{3.964490in}{1.427861in}}%
\pgfpathlineto{\pgfqpoint{3.950551in}{1.378961in}}%
\pgfpathlineto{\pgfqpoint{3.925784in}{1.291173in}}%
\pgfpathlineto{\pgfqpoint{3.913477in}{1.247399in}}%
\pgfpathlineto{\pgfqpoint{3.943412in}{1.201282in}}%
\pgfpathlineto{\pgfqpoint{3.969856in}{1.160169in}}%
\pgfpathlineto{\pgfqpoint{3.990843in}{1.126933in}}%
\pgfpathlineto{\pgfqpoint{4.016002in}{1.088427in}}%
\pgfpathlineto{\pgfqpoint{4.064332in}{1.014996in}}%
\pgfpathlineto{\pgfqpoint{4.113408in}{0.939428in}}%
\pgfpathlineto{\pgfqpoint{4.111437in}{0.931934in}}%
\pgfpathlineto{\pgfqpoint{4.117357in}{0.919999in}}%
\pgfpathlineto{\pgfqpoint{4.118527in}{0.903672in}}%
\pgfpathlineto{\pgfqpoint{4.127124in}{0.895753in}}%
\pgfpathlineto{\pgfqpoint{4.127464in}{0.889371in}}%
\pgfpathlineto{\pgfqpoint{4.112071in}{0.882129in}}%
\pgfpathlineto{\pgfqpoint{4.104738in}{0.874878in}}%
\pgfpathlineto{\pgfqpoint{4.102451in}{0.858865in}}%
\pgfpathlineto{\pgfqpoint{4.098739in}{0.850133in}}%
\pgfpathlineto{\pgfqpoint{4.090917in}{0.842772in}}%
\pgfpathlineto{\pgfqpoint{4.083138in}{0.823633in}}%
\pgfpathlineto{\pgfqpoint{4.094076in}{0.813662in}}%
\pgfpathlineto{\pgfqpoint{4.092664in}{0.805443in}}%
\pgfpathlineto{\pgfqpoint{4.083695in}{0.799724in}}%
\pgfpathlineto{\pgfqpoint{4.077512in}{0.800733in}}%
\pgfpathlineto{\pgfqpoint{4.004523in}{0.810862in}}%
\pgfpathlineto{\pgfqpoint{3.950614in}{0.818510in}}%
\pgfpathlineto{\pgfqpoint{3.952971in}{0.827214in}}%
\pgfpathlineto{\pgfqpoint{3.949462in}{0.841664in}}%
\pgfpathlineto{\pgfqpoint{3.949108in}{0.856240in}}%
\pgfpathlineto{\pgfqpoint{3.946815in}{0.864769in}}%
\pgfpathlineto{\pgfqpoint{3.939749in}{0.876964in}}%
\pgfpathlineto{\pgfqpoint{3.919427in}{0.905101in}}%
\pgfpathlineto{\pgfqpoint{3.912772in}{0.908549in}}%
\pgfpathlineto{\pgfqpoint{3.904196in}{0.908494in}}%
\pgfpathlineto{\pgfqpoint{3.906155in}{0.917370in}}%
\pgfpathlineto{\pgfqpoint{3.902094in}{0.928470in}}%
\pgfpathlineto{\pgfqpoint{3.882148in}{0.933987in}}%
\pgfpathlineto{\pgfqpoint{3.869990in}{0.944202in}}%
\pgfpathlineto{\pgfqpoint{3.868981in}{0.950453in}}%
\pgfpathlineto{\pgfqpoint{3.854981in}{0.965927in}}%
\pgfpathlineto{\pgfqpoint{3.841655in}{0.968889in}}%
\pgfpathlineto{\pgfqpoint{3.829308in}{0.976753in}}%
\pgfpathlineto{\pgfqpoint{3.813068in}{0.979474in}}%
\pgfpathlineto{\pgfqpoint{3.806130in}{0.989962in}}%
\pgfpathlineto{\pgfqpoint{3.812719in}{1.006565in}}%
\pgfpathlineto{\pgfqpoint{3.810704in}{1.010298in}}%
\pgfpathlineto{\pgfqpoint{3.816191in}{1.024138in}}%
\pgfpathlineto{\pgfqpoint{3.806430in}{1.031503in}}%
\pgfpathlineto{\pgfqpoint{3.809597in}{1.044856in}}%
\pgfpathlineto{\pgfqpoint{3.804383in}{1.048269in}}%
\pgfpathlineto{\pgfqpoint{3.799882in}{1.060890in}}%
\pgfpathlineto{\pgfqpoint{3.794449in}{1.064739in}}%
\pgfpathlineto{\pgfqpoint{3.794041in}{1.073855in}}%
\pgfpathlineto{\pgfqpoint{3.789785in}{1.080283in}}%
\pgfpathlineto{\pgfqpoint{3.783371in}{1.101933in}}%
\pgfpathlineto{\pgfqpoint{3.776354in}{1.112303in}}%
\pgfpathlineto{\pgfqpoint{3.777879in}{1.129906in}}%
\pgfpathlineto{\pgfqpoint{3.786131in}{1.131677in}}%
\pgfpathlineto{\pgfqpoint{3.791505in}{1.141225in}}%
\pgfpathlineto{\pgfqpoint{3.788258in}{1.151580in}}%
\pgfpathlineto{\pgfqpoint{3.779481in}{1.153309in}}%
\pgfpathlineto{\pgfqpoint{3.775102in}{1.158152in}}%
\pgfpathlineto{\pgfqpoint{3.768012in}{1.175968in}}%
\pgfpathlineto{\pgfqpoint{3.771326in}{1.182366in}}%
\pgfpathlineto{\pgfqpoint{3.768975in}{1.194280in}}%
\pgfpathlineto{\pgfqpoint{3.774180in}{1.209710in}}%
\pgfpathlineto{\pgfqpoint{3.780278in}{1.204027in}}%
\pgfpathlineto{\pgfqpoint{3.777511in}{1.197319in}}%
\pgfpathlineto{\pgfqpoint{3.788078in}{1.186685in}}%
\pgfpathlineto{\pgfqpoint{3.787403in}{1.202476in}}%
\pgfpathlineto{\pgfqpoint{3.782875in}{1.206711in}}%
\pgfpathlineto{\pgfqpoint{3.788051in}{1.220596in}}%
\pgfpathlineto{\pgfqpoint{3.807627in}{1.218385in}}%
\pgfpathlineto{\pgfqpoint{3.804992in}{1.223548in}}%
\pgfpathlineto{\pgfqpoint{3.792072in}{1.223040in}}%
\pgfpathlineto{\pgfqpoint{3.785926in}{1.230860in}}%
\pgfpathlineto{\pgfqpoint{3.778190in}{1.223916in}}%
\pgfpathlineto{\pgfqpoint{3.774084in}{1.212308in}}%
\pgfpathlineto{\pgfqpoint{3.763127in}{1.227928in}}%
\pgfpathlineto{\pgfqpoint{3.758907in}{1.230771in}}%
\pgfpathlineto{\pgfqpoint{3.760515in}{1.247771in}}%
\pgfpathlineto{\pgfqpoint{3.757175in}{1.257795in}}%
\pgfpathlineto{\pgfqpoint{3.751121in}{1.267213in}}%
\pgfpathlineto{\pgfqpoint{3.738723in}{1.296071in}}%
\pgfpathlineto{\pgfqpoint{3.742777in}{1.302452in}}%
\pgfpathlineto{\pgfqpoint{3.742730in}{1.322619in}}%
\pgfpathlineto{\pgfqpoint{3.749421in}{1.333868in}}%
\pgfpathlineto{\pgfqpoint{3.750955in}{1.351447in}}%
\pgfpathlineto{\pgfqpoint{3.744645in}{1.371588in}}%
\pgfpathlineto{\pgfqpoint{3.736275in}{1.384409in}}%
\pgfpathlineto{\pgfqpoint{3.737768in}{1.395978in}}%
\pgfpathlineto{\pgfqpoint{3.761274in}{1.423977in}}%
\pgfpathlineto{\pgfqpoint{3.762442in}{1.433528in}}%
\pgfpathlineto{\pgfqpoint{3.773033in}{1.451747in}}%
\pgfpathlineto{\pgfqpoint{3.774503in}{1.469017in}}%
\pgfpathlineto{\pgfqpoint{3.771097in}{1.473423in}}%
\pgfpathclose%
\pgfusepath{fill}%
\end{pgfscope}%
\begin{pgfscope}%
\pgfpathrectangle{\pgfqpoint{3.525000in}{0.100000in}}{\pgfqpoint{2.857344in}{1.829167in}}%
\pgfusepath{clip}%
\pgfsetbuttcap%
\pgfsetmiterjoin%
\definecolor{currentfill}{rgb}{0.729566,0.890657,0.631142}%
\pgfsetfillcolor{currentfill}%
\pgfsetlinewidth{0.000000pt}%
\definecolor{currentstroke}{rgb}{0.000000,0.000000,0.000000}%
\pgfsetstrokecolor{currentstroke}%
\pgfsetstrokeopacity{0.000000}%
\pgfsetdash{}{0pt}%
\pgfpathmoveto{\pgfqpoint{5.607869in}{1.115743in}}%
\pgfpathlineto{\pgfqpoint{5.599915in}{1.193611in}}%
\pgfpathlineto{\pgfqpoint{5.590674in}{1.277298in}}%
\pgfpathlineto{\pgfqpoint{5.651192in}{1.286311in}}%
\pgfpathlineto{\pgfqpoint{5.667254in}{1.282131in}}%
\pgfpathlineto{\pgfqpoint{5.674958in}{1.277593in}}%
\pgfpathlineto{\pgfqpoint{5.684611in}{1.278852in}}%
\pgfpathlineto{\pgfqpoint{5.697338in}{1.271339in}}%
\pgfpathlineto{\pgfqpoint{5.721045in}{1.282521in}}%
\pgfpathlineto{\pgfqpoint{5.734128in}{1.282894in}}%
\pgfpathlineto{\pgfqpoint{5.749405in}{1.299938in}}%
\pgfpathlineto{\pgfqpoint{5.764950in}{1.310306in}}%
\pgfpathlineto{\pgfqpoint{5.785683in}{1.322248in}}%
\pgfpathlineto{\pgfqpoint{5.799030in}{1.238770in}}%
\pgfpathlineto{\pgfqpoint{5.792829in}{1.233409in}}%
\pgfpathlineto{\pgfqpoint{5.796832in}{1.228484in}}%
\pgfpathlineto{\pgfqpoint{5.798427in}{1.217689in}}%
\pgfpathlineto{\pgfqpoint{5.795321in}{1.207550in}}%
\pgfpathlineto{\pgfqpoint{5.794910in}{1.192730in}}%
\pgfpathlineto{\pgfqpoint{5.791992in}{1.173441in}}%
\pgfpathlineto{\pgfqpoint{5.777231in}{1.156237in}}%
\pgfpathlineto{\pgfqpoint{5.771018in}{1.152585in}}%
\pgfpathlineto{\pgfqpoint{5.766185in}{1.155729in}}%
\pgfpathlineto{\pgfqpoint{5.754227in}{1.139332in}}%
\pgfpathlineto{\pgfqpoint{5.755347in}{1.123543in}}%
\pgfpathlineto{\pgfqpoint{5.742029in}{1.127340in}}%
\pgfpathlineto{\pgfqpoint{5.735722in}{1.111572in}}%
\pgfpathlineto{\pgfqpoint{5.738910in}{1.100398in}}%
\pgfpathlineto{\pgfqpoint{5.733920in}{1.098747in}}%
\pgfpathlineto{\pgfqpoint{5.733222in}{1.089919in}}%
\pgfpathlineto{\pgfqpoint{5.720973in}{1.086357in}}%
\pgfpathlineto{\pgfqpoint{5.714608in}{1.093486in}}%
\pgfpathlineto{\pgfqpoint{5.706417in}{1.096250in}}%
\pgfpathlineto{\pgfqpoint{5.704493in}{1.103478in}}%
\pgfpathlineto{\pgfqpoint{5.697088in}{1.102206in}}%
\pgfpathlineto{\pgfqpoint{5.692236in}{1.095558in}}%
\pgfpathlineto{\pgfqpoint{5.685296in}{1.093220in}}%
\pgfpathlineto{\pgfqpoint{5.673036in}{1.097925in}}%
\pgfpathlineto{\pgfqpoint{5.666256in}{1.092327in}}%
\pgfpathlineto{\pgfqpoint{5.656619in}{1.098950in}}%
\pgfpathlineto{\pgfqpoint{5.638131in}{1.100983in}}%
\pgfpathlineto{\pgfqpoint{5.632555in}{1.113014in}}%
\pgfpathlineto{\pgfqpoint{5.625490in}{1.118345in}}%
\pgfpathlineto{\pgfqpoint{5.618640in}{1.114933in}}%
\pgfpathclose%
\pgfusepath{fill}%
\end{pgfscope}%
\begin{pgfscope}%
\pgfpathrectangle{\pgfqpoint{3.525000in}{0.100000in}}{\pgfqpoint{2.857344in}{1.829167in}}%
\pgfusepath{clip}%
\pgfsetbuttcap%
\pgfsetmiterjoin%
\definecolor{currentfill}{rgb}{0.856594,0.942330,0.605306}%
\pgfsetfillcolor{currentfill}%
\pgfsetlinewidth{0.000000pt}%
\definecolor{currentstroke}{rgb}{0.000000,0.000000,0.000000}%
\pgfsetstrokecolor{currentstroke}%
\pgfsetstrokeopacity{0.000000}%
\pgfsetdash{}{0pt}%
\pgfpathmoveto{\pgfqpoint{5.407509in}{0.965148in}}%
\pgfpathlineto{\pgfqpoint{5.400418in}{0.970908in}}%
\pgfpathlineto{\pgfqpoint{5.394637in}{0.968168in}}%
\pgfpathlineto{\pgfqpoint{5.387245in}{0.981969in}}%
\pgfpathlineto{\pgfqpoint{5.391021in}{0.990644in}}%
\pgfpathlineto{\pgfqpoint{5.385579in}{1.000408in}}%
\pgfpathlineto{\pgfqpoint{5.385287in}{1.008095in}}%
\pgfpathlineto{\pgfqpoint{5.377933in}{1.010834in}}%
\pgfpathlineto{\pgfqpoint{5.374523in}{1.016626in}}%
\pgfpathlineto{\pgfqpoint{5.360145in}{1.023860in}}%
\pgfpathlineto{\pgfqpoint{5.347658in}{1.032772in}}%
\pgfpathlineto{\pgfqpoint{5.341856in}{1.039501in}}%
\pgfpathlineto{\pgfqpoint{5.341736in}{1.047707in}}%
\pgfpathlineto{\pgfqpoint{5.349500in}{1.063463in}}%
\pgfpathlineto{\pgfqpoint{5.351888in}{1.075514in}}%
\pgfpathlineto{\pgfqpoint{5.345562in}{1.082323in}}%
\pgfpathlineto{\pgfqpoint{5.337185in}{1.084889in}}%
\pgfpathlineto{\pgfqpoint{5.327022in}{1.079279in}}%
\pgfpathlineto{\pgfqpoint{5.322720in}{1.088911in}}%
\pgfpathlineto{\pgfqpoint{5.320549in}{1.101967in}}%
\pgfpathlineto{\pgfqpoint{5.305571in}{1.113606in}}%
\pgfpathlineto{\pgfqpoint{5.302578in}{1.118771in}}%
\pgfpathlineto{\pgfqpoint{5.288897in}{1.130464in}}%
\pgfpathlineto{\pgfqpoint{5.284609in}{1.138964in}}%
\pgfpathlineto{\pgfqpoint{5.280744in}{1.155821in}}%
\pgfpathlineto{\pgfqpoint{5.283393in}{1.170794in}}%
\pgfpathlineto{\pgfqpoint{5.286930in}{1.172881in}}%
\pgfpathlineto{\pgfqpoint{5.286293in}{1.185415in}}%
\pgfpathlineto{\pgfqpoint{5.296276in}{1.189138in}}%
\pgfpathlineto{\pgfqpoint{5.299290in}{1.200386in}}%
\pgfpathlineto{\pgfqpoint{5.305033in}{1.207956in}}%
\pgfpathlineto{\pgfqpoint{5.304743in}{1.217587in}}%
\pgfpathlineto{\pgfqpoint{5.297625in}{1.225248in}}%
\pgfpathlineto{\pgfqpoint{5.299315in}{1.235970in}}%
\pgfpathlineto{\pgfqpoint{5.317767in}{1.240633in}}%
\pgfpathlineto{\pgfqpoint{5.331923in}{1.249144in}}%
\pgfpathlineto{\pgfqpoint{5.333409in}{1.259862in}}%
\pgfpathlineto{\pgfqpoint{5.338334in}{1.263236in}}%
\pgfpathlineto{\pgfqpoint{5.340219in}{1.274495in}}%
\pgfpathlineto{\pgfqpoint{5.338704in}{1.281931in}}%
\pgfpathlineto{\pgfqpoint{5.329026in}{1.288109in}}%
\pgfpathlineto{\pgfqpoint{5.325130in}{1.297317in}}%
\pgfpathlineto{\pgfqpoint{5.315579in}{1.306219in}}%
\pgfpathlineto{\pgfqpoint{5.394068in}{1.309564in}}%
\pgfpathlineto{\pgfqpoint{5.446726in}{1.313305in}}%
\pgfpathlineto{\pgfqpoint{5.445763in}{1.302228in}}%
\pgfpathlineto{\pgfqpoint{5.454737in}{1.286949in}}%
\pgfpathlineto{\pgfqpoint{5.458505in}{1.273895in}}%
\pgfpathlineto{\pgfqpoint{5.462992in}{1.266484in}}%
\pgfpathlineto{\pgfqpoint{5.467890in}{1.205414in}}%
\pgfpathlineto{\pgfqpoint{5.474693in}{1.118362in}}%
\pgfpathlineto{\pgfqpoint{5.469445in}{1.105134in}}%
\pgfpathlineto{\pgfqpoint{5.477005in}{1.094261in}}%
\pgfpathlineto{\pgfqpoint{5.479113in}{1.083876in}}%
\pgfpathlineto{\pgfqpoint{5.468686in}{1.062342in}}%
\pgfpathlineto{\pgfqpoint{5.469301in}{1.059367in}}%
\pgfpathlineto{\pgfqpoint{5.458570in}{1.047023in}}%
\pgfpathlineto{\pgfqpoint{5.461513in}{1.043254in}}%
\pgfpathlineto{\pgfqpoint{5.457048in}{1.035560in}}%
\pgfpathlineto{\pgfqpoint{5.458197in}{1.020052in}}%
\pgfpathlineto{\pgfqpoint{5.452774in}{1.010536in}}%
\pgfpathlineto{\pgfqpoint{5.457184in}{0.999291in}}%
\pgfpathlineto{\pgfqpoint{5.438705in}{0.993181in}}%
\pgfpathlineto{\pgfqpoint{5.437002in}{0.986536in}}%
\pgfpathlineto{\pgfqpoint{5.442048in}{0.978113in}}%
\pgfpathlineto{\pgfqpoint{5.439726in}{0.972624in}}%
\pgfpathlineto{\pgfqpoint{5.419911in}{0.979405in}}%
\pgfpathlineto{\pgfqpoint{5.413373in}{0.980087in}}%
\pgfpathlineto{\pgfqpoint{5.405222in}{0.969778in}}%
\pgfpathclose%
\pgfusepath{fill}%
\end{pgfscope}%
\begin{pgfscope}%
\pgfpathrectangle{\pgfqpoint{3.525000in}{0.100000in}}{\pgfqpoint{2.857344in}{1.829167in}}%
\pgfusepath{clip}%
\pgfsetbuttcap%
\pgfsetmiterjoin%
\definecolor{currentfill}{rgb}{0.720492,0.886967,0.632987}%
\pgfsetfillcolor{currentfill}%
\pgfsetlinewidth{0.000000pt}%
\definecolor{currentstroke}{rgb}{0.000000,0.000000,0.000000}%
\pgfsetstrokecolor{currentstroke}%
\pgfsetstrokeopacity{0.000000}%
\pgfsetdash{}{0pt}%
\pgfpathmoveto{\pgfqpoint{5.984306in}{1.154607in}}%
\pgfpathlineto{\pgfqpoint{5.978852in}{1.162705in}}%
\pgfpathlineto{\pgfqpoint{5.981929in}{1.167236in}}%
\pgfpathlineto{\pgfqpoint{5.989466in}{1.162147in}}%
\pgfpathclose%
\pgfusepath{fill}%
\end{pgfscope}%
\begin{pgfscope}%
\pgfpathrectangle{\pgfqpoint{3.525000in}{0.100000in}}{\pgfqpoint{2.857344in}{1.829167in}}%
\pgfusepath{clip}%
\pgfsetbuttcap%
\pgfsetmiterjoin%
\definecolor{currentfill}{rgb}{0.720492,0.886967,0.632987}%
\pgfsetfillcolor{currentfill}%
\pgfsetlinewidth{0.000000pt}%
\definecolor{currentstroke}{rgb}{0.000000,0.000000,0.000000}%
\pgfsetstrokecolor{currentstroke}%
\pgfsetstrokeopacity{0.000000}%
\pgfsetdash{}{0pt}%
\pgfpathmoveto{\pgfqpoint{6.031924in}{1.224286in}}%
\pgfpathlineto{\pgfqpoint{6.035183in}{1.231171in}}%
\pgfpathlineto{\pgfqpoint{6.048356in}{1.232658in}}%
\pgfpathlineto{\pgfqpoint{6.046245in}{1.226807in}}%
\pgfpathlineto{\pgfqpoint{6.041899in}{1.219332in}}%
\pgfpathlineto{\pgfqpoint{6.044844in}{1.210426in}}%
\pgfpathlineto{\pgfqpoint{6.056467in}{1.199756in}}%
\pgfpathlineto{\pgfqpoint{6.059165in}{1.188519in}}%
\pgfpathlineto{\pgfqpoint{6.072556in}{1.174527in}}%
\pgfpathlineto{\pgfqpoint{6.077809in}{1.175125in}}%
\pgfpathlineto{\pgfqpoint{6.084353in}{1.154120in}}%
\pgfpathlineto{\pgfqpoint{6.083280in}{1.153911in}}%
\pgfpathlineto{\pgfqpoint{6.082087in}{1.153676in}}%
\pgfpathlineto{\pgfqpoint{6.052900in}{1.148089in}}%
\pgfpathlineto{\pgfqpoint{6.037285in}{1.203624in}}%
\pgfpathclose%
\pgfusepath{fill}%
\end{pgfscope}%
\begin{pgfscope}%
\pgfpathrectangle{\pgfqpoint{3.525000in}{0.100000in}}{\pgfqpoint{2.857344in}{1.829167in}}%
\pgfusepath{clip}%
\pgfsetbuttcap%
\pgfsetmiterjoin%
\definecolor{currentfill}{rgb}{0.990388,0.996155,0.734025}%
\pgfsetfillcolor{currentfill}%
\pgfsetlinewidth{0.000000pt}%
\definecolor{currentstroke}{rgb}{0.000000,0.000000,0.000000}%
\pgfsetstrokecolor{currentstroke}%
\pgfsetstrokeopacity{0.000000}%
\pgfsetdash{}{0pt}%
\pgfpathmoveto{\pgfqpoint{5.759313in}{1.036045in}}%
\pgfpathlineto{\pgfqpoint{5.750090in}{1.036384in}}%
\pgfpathlineto{\pgfqpoint{5.741592in}{1.042238in}}%
\pgfpathlineto{\pgfqpoint{5.730111in}{1.060030in}}%
\pgfpathlineto{\pgfqpoint{5.720338in}{1.069453in}}%
\pgfpathlineto{\pgfqpoint{5.722892in}{1.076731in}}%
\pgfpathlineto{\pgfqpoint{5.720973in}{1.086357in}}%
\pgfpathlineto{\pgfqpoint{5.733222in}{1.089919in}}%
\pgfpathlineto{\pgfqpoint{5.733920in}{1.098747in}}%
\pgfpathlineto{\pgfqpoint{5.738910in}{1.100398in}}%
\pgfpathlineto{\pgfqpoint{5.735722in}{1.111572in}}%
\pgfpathlineto{\pgfqpoint{5.742029in}{1.127340in}}%
\pgfpathlineto{\pgfqpoint{5.755347in}{1.123543in}}%
\pgfpathlineto{\pgfqpoint{5.754227in}{1.139332in}}%
\pgfpathlineto{\pgfqpoint{5.766185in}{1.155729in}}%
\pgfpathlineto{\pgfqpoint{5.771018in}{1.152585in}}%
\pgfpathlineto{\pgfqpoint{5.777231in}{1.156237in}}%
\pgfpathlineto{\pgfqpoint{5.791992in}{1.173441in}}%
\pgfpathlineto{\pgfqpoint{5.794910in}{1.192730in}}%
\pgfpathlineto{\pgfqpoint{5.795321in}{1.207550in}}%
\pgfpathlineto{\pgfqpoint{5.798427in}{1.217689in}}%
\pgfpathlineto{\pgfqpoint{5.796832in}{1.228484in}}%
\pgfpathlineto{\pgfqpoint{5.792829in}{1.233409in}}%
\pgfpathlineto{\pgfqpoint{5.799030in}{1.238770in}}%
\pgfpathlineto{\pgfqpoint{5.808019in}{1.182138in}}%
\pgfpathlineto{\pgfqpoint{5.857573in}{1.190332in}}%
\pgfpathlineto{\pgfqpoint{5.862709in}{1.158008in}}%
\pgfpathlineto{\pgfqpoint{5.880650in}{1.179344in}}%
\pgfpathlineto{\pgfqpoint{5.884877in}{1.177216in}}%
\pgfpathlineto{\pgfqpoint{5.891096in}{1.189579in}}%
\pgfpathlineto{\pgfqpoint{5.899710in}{1.185702in}}%
\pgfpathlineto{\pgfqpoint{5.907222in}{1.186439in}}%
\pgfpathlineto{\pgfqpoint{5.912178in}{1.195033in}}%
\pgfpathlineto{\pgfqpoint{5.919371in}{1.199814in}}%
\pgfpathlineto{\pgfqpoint{5.929372in}{1.195604in}}%
\pgfpathlineto{\pgfqpoint{5.935953in}{1.197058in}}%
\pgfpathlineto{\pgfqpoint{5.945380in}{1.180743in}}%
\pgfpathlineto{\pgfqpoint{5.942654in}{1.168391in}}%
\pgfpathlineto{\pgfqpoint{5.918031in}{1.182298in}}%
\pgfpathlineto{\pgfqpoint{5.913419in}{1.170881in}}%
\pgfpathlineto{\pgfqpoint{5.914941in}{1.165629in}}%
\pgfpathlineto{\pgfqpoint{5.904461in}{1.147872in}}%
\pgfpathlineto{\pgfqpoint{5.899515in}{1.144340in}}%
\pgfpathlineto{\pgfqpoint{5.897282in}{1.136510in}}%
\pgfpathlineto{\pgfqpoint{5.888850in}{1.137329in}}%
\pgfpathlineto{\pgfqpoint{5.882795in}{1.115957in}}%
\pgfpathlineto{\pgfqpoint{5.879405in}{1.111053in}}%
\pgfpathlineto{\pgfqpoint{5.870695in}{1.112688in}}%
\pgfpathlineto{\pgfqpoint{5.861786in}{1.119429in}}%
\pgfpathlineto{\pgfqpoint{5.861478in}{1.109084in}}%
\pgfpathlineto{\pgfqpoint{5.852853in}{1.091696in}}%
\pgfpathlineto{\pgfqpoint{5.850708in}{1.079318in}}%
\pgfpathlineto{\pgfqpoint{5.844091in}{1.071074in}}%
\pgfpathlineto{\pgfqpoint{5.839074in}{1.057909in}}%
\pgfpathlineto{\pgfqpoint{5.838828in}{1.045722in}}%
\pgfpathlineto{\pgfqpoint{5.831080in}{1.043446in}}%
\pgfpathlineto{\pgfqpoint{5.822291in}{1.036546in}}%
\pgfpathlineto{\pgfqpoint{5.813828in}{1.035233in}}%
\pgfpathlineto{\pgfqpoint{5.811878in}{1.029393in}}%
\pgfpathlineto{\pgfqpoint{5.798242in}{1.023413in}}%
\pgfpathlineto{\pgfqpoint{5.790622in}{1.028508in}}%
\pgfpathlineto{\pgfqpoint{5.782104in}{1.018833in}}%
\pgfpathlineto{\pgfqpoint{5.767463in}{1.021685in}}%
\pgfpathlineto{\pgfqpoint{5.758483in}{1.031840in}}%
\pgfpathclose%
\pgfusepath{fill}%
\end{pgfscope}%
\begin{pgfscope}%
\pgfpathrectangle{\pgfqpoint{3.525000in}{0.100000in}}{\pgfqpoint{2.857344in}{1.829167in}}%
\pgfusepath{clip}%
\pgfsetbuttcap%
\pgfsetmiterjoin%
\definecolor{currentfill}{rgb}{0.633449,0.852134,0.643676}%
\pgfsetfillcolor{currentfill}%
\pgfsetlinewidth{0.000000pt}%
\definecolor{currentstroke}{rgb}{0.000000,0.000000,0.000000}%
\pgfsetstrokecolor{currentstroke}%
\pgfsetstrokeopacity{0.000000}%
\pgfsetdash{}{0pt}%
\pgfpathmoveto{\pgfqpoint{6.082087in}{1.153676in}}%
\pgfpathlineto{\pgfqpoint{6.081713in}{1.142264in}}%
\pgfpathlineto{\pgfqpoint{6.077321in}{1.136654in}}%
\pgfpathlineto{\pgfqpoint{6.074518in}{1.124199in}}%
\pgfpathlineto{\pgfqpoint{6.061861in}{1.118458in}}%
\pgfpathlineto{\pgfqpoint{6.051242in}{1.116785in}}%
\pgfpathlineto{\pgfqpoint{6.054336in}{1.124976in}}%
\pgfpathlineto{\pgfqpoint{6.037986in}{1.131840in}}%
\pgfpathlineto{\pgfqpoint{6.024703in}{1.140433in}}%
\pgfpathlineto{\pgfqpoint{6.032926in}{1.153279in}}%
\pgfpathlineto{\pgfqpoint{6.026336in}{1.163264in}}%
\pgfpathlineto{\pgfqpoint{6.027112in}{1.171877in}}%
\pgfpathlineto{\pgfqpoint{6.022346in}{1.174241in}}%
\pgfpathlineto{\pgfqpoint{6.018479in}{1.188234in}}%
\pgfpathlineto{\pgfqpoint{6.027624in}{1.207202in}}%
\pgfpathlineto{\pgfqpoint{6.020753in}{1.210300in}}%
\pgfpathlineto{\pgfqpoint{6.018973in}{1.200942in}}%
\pgfpathlineto{\pgfqpoint{6.009165in}{1.198336in}}%
\pgfpathlineto{\pgfqpoint{6.010197in}{1.167547in}}%
\pgfpathlineto{\pgfqpoint{6.008403in}{1.157638in}}%
\pgfpathlineto{\pgfqpoint{6.013323in}{1.143515in}}%
\pgfpathlineto{\pgfqpoint{6.020890in}{1.136721in}}%
\pgfpathlineto{\pgfqpoint{6.017458in}{1.132454in}}%
\pgfpathlineto{\pgfqpoint{6.025183in}{1.126220in}}%
\pgfpathlineto{\pgfqpoint{6.027976in}{1.116101in}}%
\pgfpathlineto{\pgfqpoint{6.013817in}{1.124480in}}%
\pgfpathlineto{\pgfqpoint{6.004856in}{1.123392in}}%
\pgfpathlineto{\pgfqpoint{5.997898in}{1.132011in}}%
\pgfpathlineto{\pgfqpoint{5.990814in}{1.132851in}}%
\pgfpathlineto{\pgfqpoint{5.980746in}{1.128534in}}%
\pgfpathlineto{\pgfqpoint{5.976841in}{1.133914in}}%
\pgfpathlineto{\pgfqpoint{5.981973in}{1.145207in}}%
\pgfpathlineto{\pgfqpoint{5.984306in}{1.154607in}}%
\pgfpathlineto{\pgfqpoint{5.989466in}{1.162147in}}%
\pgfpathlineto{\pgfqpoint{5.981929in}{1.167236in}}%
\pgfpathlineto{\pgfqpoint{5.978852in}{1.162705in}}%
\pgfpathlineto{\pgfqpoint{5.971324in}{1.167304in}}%
\pgfpathlineto{\pgfqpoint{5.957993in}{1.170369in}}%
\pgfpathlineto{\pgfqpoint{5.959202in}{1.177108in}}%
\pgfpathlineto{\pgfqpoint{5.953160in}{1.181016in}}%
\pgfpathlineto{\pgfqpoint{5.945380in}{1.180743in}}%
\pgfpathlineto{\pgfqpoint{5.935953in}{1.197058in}}%
\pgfpathlineto{\pgfqpoint{5.929372in}{1.195604in}}%
\pgfpathlineto{\pgfqpoint{5.919371in}{1.199814in}}%
\pgfpathlineto{\pgfqpoint{5.912178in}{1.195033in}}%
\pgfpathlineto{\pgfqpoint{5.907222in}{1.186439in}}%
\pgfpathlineto{\pgfqpoint{5.899710in}{1.185702in}}%
\pgfpathlineto{\pgfqpoint{5.891096in}{1.189579in}}%
\pgfpathlineto{\pgfqpoint{5.884877in}{1.177216in}}%
\pgfpathlineto{\pgfqpoint{5.880650in}{1.179344in}}%
\pgfpathlineto{\pgfqpoint{5.862709in}{1.158008in}}%
\pgfpathlineto{\pgfqpoint{5.857573in}{1.190332in}}%
\pgfpathlineto{\pgfqpoint{5.923134in}{1.202483in}}%
\pgfpathlineto{\pgfqpoint{5.952524in}{1.207746in}}%
\pgfpathlineto{\pgfqpoint{5.995281in}{1.216357in}}%
\pgfpathlineto{\pgfqpoint{6.031924in}{1.224286in}}%
\pgfpathlineto{\pgfqpoint{6.037285in}{1.203624in}}%
\pgfpathlineto{\pgfqpoint{6.052900in}{1.148089in}}%
\pgfpathclose%
\pgfusepath{fill}%
\end{pgfscope}%
\begin{pgfscope}%
\pgfpathrectangle{\pgfqpoint{3.525000in}{0.100000in}}{\pgfqpoint{2.857344in}{1.829167in}}%
\pgfusepath{clip}%
\pgfsetbuttcap%
\pgfsetmiterjoin%
\definecolor{currentfill}{rgb}{0.368012,0.725106,0.661822}%
\pgfsetfillcolor{currentfill}%
\pgfsetlinewidth{0.000000pt}%
\definecolor{currentstroke}{rgb}{0.000000,0.000000,0.000000}%
\pgfsetstrokecolor{currentstroke}%
\pgfsetstrokeopacity{0.000000}%
\pgfsetdash{}{0pt}%
\pgfpathmoveto{\pgfqpoint{4.762379in}{0.969343in}}%
\pgfpathlineto{\pgfqpoint{4.714547in}{0.973908in}}%
\pgfpathlineto{\pgfqpoint{4.664925in}{0.978305in}}%
\pgfpathlineto{\pgfqpoint{4.607585in}{0.984286in}}%
\pgfpathlineto{\pgfqpoint{4.522393in}{0.994419in}}%
\pgfpathlineto{\pgfqpoint{4.492171in}{0.999075in}}%
\pgfpathlineto{\pgfqpoint{4.414159in}{1.010405in}}%
\pgfpathlineto{\pgfqpoint{4.425550in}{1.081899in}}%
\pgfpathlineto{\pgfqpoint{4.425834in}{1.087682in}}%
\pgfpathlineto{\pgfqpoint{4.436808in}{1.156667in}}%
\pgfpathlineto{\pgfqpoint{4.445004in}{1.209206in}}%
\pgfpathlineto{\pgfqpoint{4.452739in}{1.257928in}}%
\pgfpathlineto{\pgfqpoint{4.505586in}{1.250342in}}%
\pgfpathlineto{\pgfqpoint{4.554859in}{1.243270in}}%
\pgfpathlineto{\pgfqpoint{4.645435in}{1.232091in}}%
\pgfpathlineto{\pgfqpoint{4.686989in}{1.228296in}}%
\pgfpathlineto{\pgfqpoint{4.752819in}{1.221909in}}%
\pgfpathlineto{\pgfqpoint{4.781297in}{1.219598in}}%
\pgfpathlineto{\pgfqpoint{4.776269in}{1.157246in}}%
\pgfpathlineto{\pgfqpoint{4.771726in}{1.097211in}}%
\pgfpathlineto{\pgfqpoint{4.768071in}{1.048329in}}%
\pgfpathclose%
\pgfusepath{fill}%
\end{pgfscope}%
\begin{pgfscope}%
\pgfpathrectangle{\pgfqpoint{3.525000in}{0.100000in}}{\pgfqpoint{2.857344in}{1.829167in}}%
\pgfusepath{clip}%
\pgfsetbuttcap%
\pgfsetmiterjoin%
\definecolor{currentfill}{rgb}{0.675125,0.868512,0.642215}%
\pgfsetfillcolor{currentfill}%
\pgfsetlinewidth{0.000000pt}%
\definecolor{currentstroke}{rgb}{0.000000,0.000000,0.000000}%
\pgfsetstrokecolor{currentstroke}%
\pgfsetstrokeopacity{0.000000}%
\pgfsetdash{}{0pt}%
\pgfpathmoveto{\pgfqpoint{5.395155in}{0.933898in}}%
\pgfpathlineto{\pgfqpoint{5.396767in}{0.941126in}}%
\pgfpathlineto{\pgfqpoint{5.405125in}{0.939499in}}%
\pgfpathlineto{\pgfqpoint{5.407509in}{0.965148in}}%
\pgfpathlineto{\pgfqpoint{5.405222in}{0.969778in}}%
\pgfpathlineto{\pgfqpoint{5.413373in}{0.980087in}}%
\pgfpathlineto{\pgfqpoint{5.419911in}{0.979405in}}%
\pgfpathlineto{\pgfqpoint{5.439726in}{0.972624in}}%
\pgfpathlineto{\pgfqpoint{5.442048in}{0.978113in}}%
\pgfpathlineto{\pgfqpoint{5.437002in}{0.986536in}}%
\pgfpathlineto{\pgfqpoint{5.438705in}{0.993181in}}%
\pgfpathlineto{\pgfqpoint{5.457184in}{0.999291in}}%
\pgfpathlineto{\pgfqpoint{5.452774in}{1.010536in}}%
\pgfpathlineto{\pgfqpoint{5.458197in}{1.020052in}}%
\pgfpathlineto{\pgfqpoint{5.468291in}{1.025393in}}%
\pgfpathlineto{\pgfqpoint{5.489471in}{1.030682in}}%
\pgfpathlineto{\pgfqpoint{5.502835in}{1.022698in}}%
\pgfpathlineto{\pgfqpoint{5.506888in}{1.030484in}}%
\pgfpathlineto{\pgfqpoint{5.517985in}{1.035923in}}%
\pgfpathlineto{\pgfqpoint{5.520848in}{1.031127in}}%
\pgfpathlineto{\pgfqpoint{5.532246in}{1.034937in}}%
\pgfpathlineto{\pgfqpoint{5.531527in}{1.041459in}}%
\pgfpathlineto{\pgfqpoint{5.541776in}{1.048924in}}%
\pgfpathlineto{\pgfqpoint{5.547824in}{1.041173in}}%
\pgfpathlineto{\pgfqpoint{5.555781in}{1.040393in}}%
\pgfpathlineto{\pgfqpoint{5.561068in}{1.045445in}}%
\pgfpathlineto{\pgfqpoint{5.560480in}{1.052628in}}%
\pgfpathlineto{\pgfqpoint{5.564975in}{1.059762in}}%
\pgfpathlineto{\pgfqpoint{5.570998in}{1.061307in}}%
\pgfpathlineto{\pgfqpoint{5.573416in}{1.070726in}}%
\pgfpathlineto{\pgfqpoint{5.582176in}{1.078868in}}%
\pgfpathlineto{\pgfqpoint{5.579539in}{1.086973in}}%
\pgfpathlineto{\pgfqpoint{5.588061in}{1.091009in}}%
\pgfpathlineto{\pgfqpoint{5.593755in}{1.088527in}}%
\pgfpathlineto{\pgfqpoint{5.602165in}{1.094831in}}%
\pgfpathlineto{\pgfqpoint{5.609680in}{1.096474in}}%
\pgfpathlineto{\pgfqpoint{5.609707in}{1.103013in}}%
\pgfpathlineto{\pgfqpoint{5.604440in}{1.112087in}}%
\pgfpathlineto{\pgfqpoint{5.607869in}{1.115743in}}%
\pgfpathlineto{\pgfqpoint{5.618640in}{1.114933in}}%
\pgfpathlineto{\pgfqpoint{5.625490in}{1.118345in}}%
\pgfpathlineto{\pgfqpoint{5.632555in}{1.113014in}}%
\pgfpathlineto{\pgfqpoint{5.638131in}{1.100983in}}%
\pgfpathlineto{\pgfqpoint{5.656619in}{1.098950in}}%
\pgfpathlineto{\pgfqpoint{5.666256in}{1.092327in}}%
\pgfpathlineto{\pgfqpoint{5.673036in}{1.097925in}}%
\pgfpathlineto{\pgfqpoint{5.685296in}{1.093220in}}%
\pgfpathlineto{\pgfqpoint{5.692236in}{1.095558in}}%
\pgfpathlineto{\pgfqpoint{5.697088in}{1.102206in}}%
\pgfpathlineto{\pgfqpoint{5.704493in}{1.103478in}}%
\pgfpathlineto{\pgfqpoint{5.706417in}{1.096250in}}%
\pgfpathlineto{\pgfqpoint{5.714608in}{1.093486in}}%
\pgfpathlineto{\pgfqpoint{5.720973in}{1.086357in}}%
\pgfpathlineto{\pgfqpoint{5.722892in}{1.076731in}}%
\pgfpathlineto{\pgfqpoint{5.720338in}{1.069453in}}%
\pgfpathlineto{\pgfqpoint{5.730111in}{1.060030in}}%
\pgfpathlineto{\pgfqpoint{5.741592in}{1.042238in}}%
\pgfpathlineto{\pgfqpoint{5.750090in}{1.036384in}}%
\pgfpathlineto{\pgfqpoint{5.759313in}{1.036045in}}%
\pgfpathlineto{\pgfqpoint{5.742283in}{1.016505in}}%
\pgfpathlineto{\pgfqpoint{5.725521in}{1.004677in}}%
\pgfpathlineto{\pgfqpoint{5.719351in}{0.995285in}}%
\pgfpathlineto{\pgfqpoint{5.719456in}{0.990187in}}%
\pgfpathlineto{\pgfqpoint{5.710376in}{0.986281in}}%
\pgfpathlineto{\pgfqpoint{5.707775in}{0.978930in}}%
\pgfpathlineto{\pgfqpoint{5.688860in}{0.971525in}}%
\pgfpathlineto{\pgfqpoint{5.682156in}{0.966716in}}%
\pgfpathlineto{\pgfqpoint{5.681248in}{0.965690in}}%
\pgfpathlineto{\pgfqpoint{5.626842in}{0.960531in}}%
\pgfpathlineto{\pgfqpoint{5.593986in}{0.957768in}}%
\pgfpathlineto{\pgfqpoint{5.540071in}{0.954717in}}%
\pgfpathlineto{\pgfqpoint{5.472902in}{0.948072in}}%
\pgfpathlineto{\pgfqpoint{5.461805in}{0.949609in}}%
\pgfpathlineto{\pgfqpoint{5.464114in}{0.938281in}}%
\pgfpathclose%
\pgfusepath{fill}%
\end{pgfscope}%
\begin{pgfscope}%
\pgfpathrectangle{\pgfqpoint{3.525000in}{0.100000in}}{\pgfqpoint{2.857344in}{1.829167in}}%
\pgfusepath{clip}%
\pgfsetbuttcap%
\pgfsetmiterjoin%
\definecolor{currentfill}{rgb}{0.702345,0.879585,0.636678}%
\pgfsetfillcolor{currentfill}%
\pgfsetlinewidth{0.000000pt}%
\definecolor{currentstroke}{rgb}{0.000000,0.000000,0.000000}%
\pgfsetstrokecolor{currentstroke}%
\pgfsetstrokeopacity{0.000000}%
\pgfsetdash{}{0pt}%
\pgfpathmoveto{\pgfqpoint{4.762379in}{0.969343in}}%
\pgfpathlineto{\pgfqpoint{4.768071in}{1.048329in}}%
\pgfpathlineto{\pgfqpoint{4.771726in}{1.097211in}}%
\pgfpathlineto{\pgfqpoint{4.776269in}{1.157246in}}%
\pgfpathlineto{\pgfqpoint{4.839241in}{1.152824in}}%
\pgfpathlineto{\pgfqpoint{4.919214in}{1.148456in}}%
\pgfpathlineto{\pgfqpoint{4.973607in}{1.146369in}}%
\pgfpathlineto{\pgfqpoint{5.027681in}{1.144712in}}%
\pgfpathlineto{\pgfqpoint{5.076623in}{1.143955in}}%
\pgfpathlineto{\pgfqpoint{5.099259in}{1.144189in}}%
\pgfpathlineto{\pgfqpoint{5.109222in}{1.136058in}}%
\pgfpathlineto{\pgfqpoint{5.117027in}{1.137698in}}%
\pgfpathlineto{\pgfqpoint{5.117271in}{1.130606in}}%
\pgfpathlineto{\pgfqpoint{5.111475in}{1.118338in}}%
\pgfpathlineto{\pgfqpoint{5.112104in}{1.110587in}}%
\pgfpathlineto{\pgfqpoint{5.117653in}{1.105476in}}%
\pgfpathlineto{\pgfqpoint{5.122750in}{1.094827in}}%
\pgfpathlineto{\pgfqpoint{5.133091in}{1.088713in}}%
\pgfpathlineto{\pgfqpoint{5.132745in}{1.048571in}}%
\pgfpathlineto{\pgfqpoint{5.133049in}{0.956264in}}%
\pgfpathlineto{\pgfqpoint{5.063741in}{0.956583in}}%
\pgfpathlineto{\pgfqpoint{4.990758in}{0.957860in}}%
\pgfpathlineto{\pgfqpoint{4.937030in}{0.959756in}}%
\pgfpathlineto{\pgfqpoint{4.892183in}{0.961486in}}%
\pgfpathlineto{\pgfqpoint{4.816629in}{0.965941in}}%
\pgfpathclose%
\pgfusepath{fill}%
\end{pgfscope}%
\begin{pgfscope}%
\pgfpathrectangle{\pgfqpoint{3.525000in}{0.100000in}}{\pgfqpoint{2.857344in}{1.829167in}}%
\pgfusepath{clip}%
\pgfsetbuttcap%
\pgfsetmiterjoin%
\definecolor{currentfill}{rgb}{0.569781,0.827220,0.644598}%
\pgfsetfillcolor{currentfill}%
\pgfsetlinewidth{0.000000pt}%
\definecolor{currentstroke}{rgb}{0.000000,0.000000,0.000000}%
\pgfsetstrokecolor{currentstroke}%
\pgfsetstrokeopacity{0.000000}%
\pgfsetdash{}{0pt}%
\pgfpathmoveto{\pgfqpoint{5.782165in}{0.979801in}}%
\pgfpathlineto{\pgfqpoint{5.705402in}{0.969018in}}%
\pgfpathlineto{\pgfqpoint{5.682156in}{0.966716in}}%
\pgfpathlineto{\pgfqpoint{5.688860in}{0.971525in}}%
\pgfpathlineto{\pgfqpoint{5.707775in}{0.978930in}}%
\pgfpathlineto{\pgfqpoint{5.710376in}{0.986281in}}%
\pgfpathlineto{\pgfqpoint{5.719456in}{0.990187in}}%
\pgfpathlineto{\pgfqpoint{5.719351in}{0.995285in}}%
\pgfpathlineto{\pgfqpoint{5.725521in}{1.004677in}}%
\pgfpathlineto{\pgfqpoint{5.742283in}{1.016505in}}%
\pgfpathlineto{\pgfqpoint{5.759313in}{1.036045in}}%
\pgfpathlineto{\pgfqpoint{5.758483in}{1.031840in}}%
\pgfpathlineto{\pgfqpoint{5.767463in}{1.021685in}}%
\pgfpathlineto{\pgfqpoint{5.782104in}{1.018833in}}%
\pgfpathlineto{\pgfqpoint{5.790622in}{1.028508in}}%
\pgfpathlineto{\pgfqpoint{5.798242in}{1.023413in}}%
\pgfpathlineto{\pgfqpoint{5.811878in}{1.029393in}}%
\pgfpathlineto{\pgfqpoint{5.813828in}{1.035233in}}%
\pgfpathlineto{\pgfqpoint{5.822291in}{1.036546in}}%
\pgfpathlineto{\pgfqpoint{5.831080in}{1.043446in}}%
\pgfpathlineto{\pgfqpoint{5.838828in}{1.045722in}}%
\pgfpathlineto{\pgfqpoint{5.839074in}{1.057909in}}%
\pgfpathlineto{\pgfqpoint{5.844091in}{1.071074in}}%
\pgfpathlineto{\pgfqpoint{5.850708in}{1.079318in}}%
\pgfpathlineto{\pgfqpoint{5.852853in}{1.091696in}}%
\pgfpathlineto{\pgfqpoint{5.861478in}{1.109084in}}%
\pgfpathlineto{\pgfqpoint{5.861786in}{1.119429in}}%
\pgfpathlineto{\pgfqpoint{5.870695in}{1.112688in}}%
\pgfpathlineto{\pgfqpoint{5.879405in}{1.111053in}}%
\pgfpathlineto{\pgfqpoint{5.882795in}{1.115957in}}%
\pgfpathlineto{\pgfqpoint{5.888850in}{1.137329in}}%
\pgfpathlineto{\pgfqpoint{5.897282in}{1.136510in}}%
\pgfpathlineto{\pgfqpoint{5.899515in}{1.144340in}}%
\pgfpathlineto{\pgfqpoint{5.904461in}{1.147872in}}%
\pgfpathlineto{\pgfqpoint{5.914941in}{1.165629in}}%
\pgfpathlineto{\pgfqpoint{5.913419in}{1.170881in}}%
\pgfpathlineto{\pgfqpoint{5.918031in}{1.182298in}}%
\pgfpathlineto{\pgfqpoint{5.942654in}{1.168391in}}%
\pgfpathlineto{\pgfqpoint{5.945380in}{1.180743in}}%
\pgfpathlineto{\pgfqpoint{5.953160in}{1.181016in}}%
\pgfpathlineto{\pgfqpoint{5.959202in}{1.177108in}}%
\pgfpathlineto{\pgfqpoint{5.957993in}{1.170369in}}%
\pgfpathlineto{\pgfqpoint{5.971324in}{1.167304in}}%
\pgfpathlineto{\pgfqpoint{5.978852in}{1.162705in}}%
\pgfpathlineto{\pgfqpoint{5.984306in}{1.154607in}}%
\pgfpathlineto{\pgfqpoint{5.981973in}{1.145207in}}%
\pgfpathlineto{\pgfqpoint{5.977262in}{1.144436in}}%
\pgfpathlineto{\pgfqpoint{5.974529in}{1.130251in}}%
\pgfpathlineto{\pgfqpoint{5.980515in}{1.124705in}}%
\pgfpathlineto{\pgfqpoint{5.988940in}{1.129188in}}%
\pgfpathlineto{\pgfqpoint{5.996757in}{1.119721in}}%
\pgfpathlineto{\pgfqpoint{6.014223in}{1.118019in}}%
\pgfpathlineto{\pgfqpoint{6.017232in}{1.112574in}}%
\pgfpathlineto{\pgfqpoint{6.033396in}{1.107277in}}%
\pgfpathlineto{\pgfqpoint{6.031401in}{1.101027in}}%
\pgfpathlineto{\pgfqpoint{6.033504in}{1.083364in}}%
\pgfpathlineto{\pgfqpoint{6.040028in}{1.076694in}}%
\pgfpathlineto{\pgfqpoint{6.030401in}{1.073879in}}%
\pgfpathlineto{\pgfqpoint{6.031678in}{1.066338in}}%
\pgfpathlineto{\pgfqpoint{6.041947in}{1.059945in}}%
\pgfpathlineto{\pgfqpoint{6.042873in}{1.053638in}}%
\pgfpathlineto{\pgfqpoint{6.037075in}{1.048899in}}%
\pgfpathlineto{\pgfqpoint{6.025376in}{1.060127in}}%
\pgfpathlineto{\pgfqpoint{6.024219in}{1.051935in}}%
\pgfpathlineto{\pgfqpoint{6.034013in}{1.048039in}}%
\pgfpathlineto{\pgfqpoint{6.034993in}{1.044011in}}%
\pgfpathlineto{\pgfqpoint{6.048426in}{1.049335in}}%
\pgfpathlineto{\pgfqpoint{6.058720in}{1.050743in}}%
\pgfpathlineto{\pgfqpoint{6.069264in}{1.029468in}}%
\pgfpathlineto{\pgfqpoint{6.068088in}{1.029238in}}%
\pgfpathlineto{\pgfqpoint{6.063323in}{1.028253in}}%
\pgfpathlineto{\pgfqpoint{6.061919in}{1.027960in}}%
\pgfpathlineto{\pgfqpoint{6.060991in}{1.027778in}}%
\pgfpathlineto{\pgfqpoint{5.998209in}{1.014755in}}%
\pgfpathlineto{\pgfqpoint{5.942054in}{1.003262in}}%
\pgfpathlineto{\pgfqpoint{5.864415in}{0.989887in}}%
\pgfpathlineto{\pgfqpoint{5.798476in}{0.981178in}}%
\pgfpathclose%
\pgfusepath{fill}%
\end{pgfscope}%
\begin{pgfscope}%
\pgfpathrectangle{\pgfqpoint{3.525000in}{0.100000in}}{\pgfqpoint{2.857344in}{1.829167in}}%
\pgfusepath{clip}%
\pgfsetbuttcap%
\pgfsetmiterjoin%
\definecolor{currentfill}{rgb}{0.569781,0.827220,0.644598}%
\pgfsetfillcolor{currentfill}%
\pgfsetlinewidth{0.000000pt}%
\definecolor{currentstroke}{rgb}{0.000000,0.000000,0.000000}%
\pgfsetstrokecolor{currentstroke}%
\pgfsetstrokeopacity{0.000000}%
\pgfsetdash{}{0pt}%
\pgfpathmoveto{\pgfqpoint{6.061861in}{1.118458in}}%
\pgfpathlineto{\pgfqpoint{6.074518in}{1.124199in}}%
\pgfpathlineto{\pgfqpoint{6.064411in}{1.094632in}}%
\pgfpathlineto{\pgfqpoint{6.057726in}{1.078975in}}%
\pgfpathlineto{\pgfqpoint{6.052911in}{1.087836in}}%
\pgfpathlineto{\pgfqpoint{6.061474in}{1.109049in}}%
\pgfpathclose%
\pgfusepath{fill}%
\end{pgfscope}%
\begin{pgfscope}%
\pgfpathrectangle{\pgfqpoint{3.525000in}{0.100000in}}{\pgfqpoint{2.857344in}{1.829167in}}%
\pgfusepath{clip}%
\pgfsetbuttcap%
\pgfsetmiterjoin%
\definecolor{currentfill}{rgb}{0.729566,0.890657,0.631142}%
\pgfsetfillcolor{currentfill}%
\pgfsetlinewidth{0.000000pt}%
\definecolor{currentstroke}{rgb}{0.000000,0.000000,0.000000}%
\pgfsetstrokecolor{currentstroke}%
\pgfsetstrokeopacity{0.000000}%
\pgfsetdash{}{0pt}%
\pgfpathmoveto{\pgfqpoint{5.395155in}{0.933898in}}%
\pgfpathlineto{\pgfqpoint{5.392099in}{0.933455in}}%
\pgfpathlineto{\pgfqpoint{5.389216in}{0.933253in}}%
\pgfpathlineto{\pgfqpoint{5.390450in}{0.924404in}}%
\pgfpathlineto{\pgfqpoint{5.383014in}{0.915483in}}%
\pgfpathlineto{\pgfqpoint{5.381548in}{0.901525in}}%
\pgfpathlineto{\pgfqpoint{5.348305in}{0.899066in}}%
\pgfpathlineto{\pgfqpoint{5.351203in}{0.905620in}}%
\pgfpathlineto{\pgfqpoint{5.363189in}{0.917589in}}%
\pgfpathlineto{\pgfqpoint{5.363520in}{0.924528in}}%
\pgfpathlineto{\pgfqpoint{5.358214in}{0.931097in}}%
\pgfpathlineto{\pgfqpoint{5.308735in}{0.928480in}}%
\pgfpathlineto{\pgfqpoint{5.222229in}{0.925734in}}%
\pgfpathlineto{\pgfqpoint{5.171605in}{0.924811in}}%
\pgfpathlineto{\pgfqpoint{5.133339in}{0.924467in}}%
\pgfpathlineto{\pgfqpoint{5.133049in}{0.956264in}}%
\pgfpathlineto{\pgfqpoint{5.132745in}{1.048571in}}%
\pgfpathlineto{\pgfqpoint{5.133091in}{1.088713in}}%
\pgfpathlineto{\pgfqpoint{5.122750in}{1.094827in}}%
\pgfpathlineto{\pgfqpoint{5.117653in}{1.105476in}}%
\pgfpathlineto{\pgfqpoint{5.112104in}{1.110587in}}%
\pgfpathlineto{\pgfqpoint{5.111475in}{1.118338in}}%
\pgfpathlineto{\pgfqpoint{5.117271in}{1.130606in}}%
\pgfpathlineto{\pgfqpoint{5.117027in}{1.137698in}}%
\pgfpathlineto{\pgfqpoint{5.109222in}{1.136058in}}%
\pgfpathlineto{\pgfqpoint{5.099259in}{1.144189in}}%
\pgfpathlineto{\pgfqpoint{5.091272in}{1.158462in}}%
\pgfpathlineto{\pgfqpoint{5.084572in}{1.165048in}}%
\pgfpathlineto{\pgfqpoint{5.077572in}{1.181231in}}%
\pgfpathlineto{\pgfqpoint{5.150281in}{1.180096in}}%
\pgfpathlineto{\pgfqpoint{5.222555in}{1.182478in}}%
\pgfpathlineto{\pgfqpoint{5.268896in}{1.185151in}}%
\pgfpathlineto{\pgfqpoint{5.283393in}{1.170794in}}%
\pgfpathlineto{\pgfqpoint{5.280744in}{1.155821in}}%
\pgfpathlineto{\pgfqpoint{5.284609in}{1.138964in}}%
\pgfpathlineto{\pgfqpoint{5.288897in}{1.130464in}}%
\pgfpathlineto{\pgfqpoint{5.302578in}{1.118771in}}%
\pgfpathlineto{\pgfqpoint{5.305571in}{1.113606in}}%
\pgfpathlineto{\pgfqpoint{5.320549in}{1.101967in}}%
\pgfpathlineto{\pgfqpoint{5.322720in}{1.088911in}}%
\pgfpathlineto{\pgfqpoint{5.327022in}{1.079279in}}%
\pgfpathlineto{\pgfqpoint{5.337185in}{1.084889in}}%
\pgfpathlineto{\pgfqpoint{5.345562in}{1.082323in}}%
\pgfpathlineto{\pgfqpoint{5.351888in}{1.075514in}}%
\pgfpathlineto{\pgfqpoint{5.349500in}{1.063463in}}%
\pgfpathlineto{\pgfqpoint{5.341736in}{1.047707in}}%
\pgfpathlineto{\pgfqpoint{5.341856in}{1.039501in}}%
\pgfpathlineto{\pgfqpoint{5.347658in}{1.032772in}}%
\pgfpathlineto{\pgfqpoint{5.360145in}{1.023860in}}%
\pgfpathlineto{\pgfqpoint{5.374523in}{1.016626in}}%
\pgfpathlineto{\pgfqpoint{5.377933in}{1.010834in}}%
\pgfpathlineto{\pgfqpoint{5.385287in}{1.008095in}}%
\pgfpathlineto{\pgfqpoint{5.385579in}{1.000408in}}%
\pgfpathlineto{\pgfqpoint{5.391021in}{0.990644in}}%
\pgfpathlineto{\pgfqpoint{5.387245in}{0.981969in}}%
\pgfpathlineto{\pgfqpoint{5.394637in}{0.968168in}}%
\pgfpathlineto{\pgfqpoint{5.400418in}{0.970908in}}%
\pgfpathlineto{\pgfqpoint{5.407509in}{0.965148in}}%
\pgfpathlineto{\pgfqpoint{5.405125in}{0.939499in}}%
\pgfpathlineto{\pgfqpoint{5.396767in}{0.941126in}}%
\pgfpathclose%
\pgfusepath{fill}%
\end{pgfscope}%
\begin{pgfscope}%
\pgfpathrectangle{\pgfqpoint{3.525000in}{0.100000in}}{\pgfqpoint{2.857344in}{1.829167in}}%
\pgfusepath{clip}%
\pgfsetbuttcap%
\pgfsetmiterjoin%
\definecolor{currentfill}{rgb}{0.410611,0.764937,0.646905}%
\pgfsetfillcolor{currentfill}%
\pgfsetlinewidth{0.000000pt}%
\definecolor{currentstroke}{rgb}{0.000000,0.000000,0.000000}%
\pgfsetstrokecolor{currentstroke}%
\pgfsetstrokeopacity{0.000000}%
\pgfsetdash{}{0pt}%
\pgfpathmoveto{\pgfqpoint{4.113408in}{0.939428in}}%
\pgfpathlineto{\pgfqpoint{4.119459in}{0.952313in}}%
\pgfpathlineto{\pgfqpoint{4.117803in}{0.971654in}}%
\pgfpathlineto{\pgfqpoint{4.120701in}{0.977154in}}%
\pgfpathlineto{\pgfqpoint{4.120988in}{1.001245in}}%
\pgfpathlineto{\pgfqpoint{4.123734in}{1.008183in}}%
\pgfpathlineto{\pgfqpoint{4.133412in}{1.009262in}}%
\pgfpathlineto{\pgfqpoint{4.142393in}{1.006182in}}%
\pgfpathlineto{\pgfqpoint{4.146304in}{0.997701in}}%
\pgfpathlineto{\pgfqpoint{4.151788in}{0.998025in}}%
\pgfpathlineto{\pgfqpoint{4.158601in}{1.007741in}}%
\pgfpathlineto{\pgfqpoint{4.168449in}{1.055636in}}%
\pgfpathlineto{\pgfqpoint{4.224487in}{1.044067in}}%
\pgfpathlineto{\pgfqpoint{4.257010in}{1.037653in}}%
\pgfpathlineto{\pgfqpoint{4.343375in}{1.022471in}}%
\pgfpathlineto{\pgfqpoint{4.367269in}{1.017673in}}%
\pgfpathlineto{\pgfqpoint{4.414159in}{1.010405in}}%
\pgfpathlineto{\pgfqpoint{4.404545in}{0.948497in}}%
\pgfpathlineto{\pgfqpoint{4.394545in}{0.883922in}}%
\pgfpathlineto{\pgfqpoint{4.383029in}{0.811272in}}%
\pgfpathlineto{\pgfqpoint{4.370078in}{0.727879in}}%
\pgfpathlineto{\pgfqpoint{4.359616in}{0.659386in}}%
\pgfpathlineto{\pgfqpoint{4.284677in}{0.671286in}}%
\pgfpathlineto{\pgfqpoint{4.251750in}{0.676925in}}%
\pgfpathlineto{\pgfqpoint{4.237042in}{0.685731in}}%
\pgfpathlineto{\pgfqpoint{4.141140in}{0.743594in}}%
\pgfpathlineto{\pgfqpoint{4.069170in}{0.787505in}}%
\pgfpathlineto{\pgfqpoint{4.071567in}{0.795288in}}%
\pgfpathlineto{\pgfqpoint{4.077512in}{0.800733in}}%
\pgfpathlineto{\pgfqpoint{4.083695in}{0.799724in}}%
\pgfpathlineto{\pgfqpoint{4.092664in}{0.805443in}}%
\pgfpathlineto{\pgfqpoint{4.094076in}{0.813662in}}%
\pgfpathlineto{\pgfqpoint{4.083138in}{0.823633in}}%
\pgfpathlineto{\pgfqpoint{4.090917in}{0.842772in}}%
\pgfpathlineto{\pgfqpoint{4.098739in}{0.850133in}}%
\pgfpathlineto{\pgfqpoint{4.102451in}{0.858865in}}%
\pgfpathlineto{\pgfqpoint{4.104738in}{0.874878in}}%
\pgfpathlineto{\pgfqpoint{4.112071in}{0.882129in}}%
\pgfpathlineto{\pgfqpoint{4.127464in}{0.889371in}}%
\pgfpathlineto{\pgfqpoint{4.127124in}{0.895753in}}%
\pgfpathlineto{\pgfqpoint{4.118527in}{0.903672in}}%
\pgfpathlineto{\pgfqpoint{4.117357in}{0.919999in}}%
\pgfpathlineto{\pgfqpoint{4.111437in}{0.931934in}}%
\pgfpathclose%
\pgfusepath{fill}%
\end{pgfscope}%
\begin{pgfscope}%
\pgfpathrectangle{\pgfqpoint{3.525000in}{0.100000in}}{\pgfqpoint{2.857344in}{1.829167in}}%
\pgfusepath{clip}%
\pgfsetbuttcap%
\pgfsetmiterjoin%
\definecolor{currentfill}{rgb}{0.820300,0.927566,0.612687}%
\pgfsetfillcolor{currentfill}%
\pgfsetlinewidth{0.000000pt}%
\definecolor{currentstroke}{rgb}{0.000000,0.000000,0.000000}%
\pgfsetstrokecolor{currentstroke}%
\pgfsetstrokeopacity{0.000000}%
\pgfsetdash{}{0pt}%
\pgfpathmoveto{\pgfqpoint{5.141452in}{0.745221in}}%
\pgfpathlineto{\pgfqpoint{5.127560in}{0.749517in}}%
\pgfpathlineto{\pgfqpoint{5.109760in}{0.763149in}}%
\pgfpathlineto{\pgfqpoint{5.101853in}{0.766079in}}%
\pgfpathlineto{\pgfqpoint{5.096829in}{0.760192in}}%
\pgfpathlineto{\pgfqpoint{5.090484in}{0.759894in}}%
\pgfpathlineto{\pgfqpoint{5.082457in}{0.764893in}}%
\pgfpathlineto{\pgfqpoint{5.069861in}{0.758488in}}%
\pgfpathlineto{\pgfqpoint{5.065377in}{0.761642in}}%
\pgfpathlineto{\pgfqpoint{5.052982in}{0.754170in}}%
\pgfpathlineto{\pgfqpoint{5.039906in}{0.755549in}}%
\pgfpathlineto{\pgfqpoint{5.029821in}{0.760404in}}%
\pgfpathlineto{\pgfqpoint{5.022757in}{0.758577in}}%
\pgfpathlineto{\pgfqpoint{5.011451in}{0.765444in}}%
\pgfpathlineto{\pgfqpoint{5.005162in}{0.753333in}}%
\pgfpathlineto{\pgfqpoint{4.998730in}{0.763748in}}%
\pgfpathlineto{\pgfqpoint{4.986027in}{0.759707in}}%
\pgfpathlineto{\pgfqpoint{4.974915in}{0.769601in}}%
\pgfpathlineto{\pgfqpoint{4.965206in}{0.761626in}}%
\pgfpathlineto{\pgfqpoint{4.960343in}{0.768954in}}%
\pgfpathlineto{\pgfqpoint{4.953334in}{0.771334in}}%
\pgfpathlineto{\pgfqpoint{4.949066in}{0.778399in}}%
\pgfpathlineto{\pgfqpoint{4.939899in}{0.780410in}}%
\pgfpathlineto{\pgfqpoint{4.934609in}{0.775092in}}%
\pgfpathlineto{\pgfqpoint{4.925622in}{0.781984in}}%
\pgfpathlineto{\pgfqpoint{4.921433in}{0.780410in}}%
\pgfpathlineto{\pgfqpoint{4.906537in}{0.785994in}}%
\pgfpathlineto{\pgfqpoint{4.897208in}{0.786617in}}%
\pgfpathlineto{\pgfqpoint{4.893032in}{0.798477in}}%
\pgfpathlineto{\pgfqpoint{4.885554in}{0.797000in}}%
\pgfpathlineto{\pgfqpoint{4.876988in}{0.799926in}}%
\pgfpathlineto{\pgfqpoint{4.871347in}{0.798235in}}%
\pgfpathlineto{\pgfqpoint{4.859250in}{0.811555in}}%
\pgfpathlineto{\pgfqpoint{4.855879in}{0.810683in}}%
\pgfpathlineto{\pgfqpoint{4.858941in}{0.864698in}}%
\pgfpathlineto{\pgfqpoint{4.862276in}{0.931553in}}%
\pgfpathlineto{\pgfqpoint{4.807532in}{0.934616in}}%
\pgfpathlineto{\pgfqpoint{4.753511in}{0.938700in}}%
\pgfpathlineto{\pgfqpoint{4.711775in}{0.942324in}}%
\pgfpathlineto{\pgfqpoint{4.714547in}{0.973908in}}%
\pgfpathlineto{\pgfqpoint{4.762379in}{0.969343in}}%
\pgfpathlineto{\pgfqpoint{4.816629in}{0.965941in}}%
\pgfpathlineto{\pgfqpoint{4.892183in}{0.961486in}}%
\pgfpathlineto{\pgfqpoint{4.937030in}{0.959756in}}%
\pgfpathlineto{\pgfqpoint{4.990758in}{0.957860in}}%
\pgfpathlineto{\pgfqpoint{5.063741in}{0.956583in}}%
\pgfpathlineto{\pgfqpoint{5.133049in}{0.956264in}}%
\pgfpathlineto{\pgfqpoint{5.133339in}{0.924467in}}%
\pgfpathlineto{\pgfqpoint{5.137230in}{0.900512in}}%
\pgfpathlineto{\pgfqpoint{5.143273in}{0.856267in}}%
\pgfpathlineto{\pgfqpoint{5.142027in}{0.780708in}}%
\pgfpathclose%
\pgfusepath{fill}%
\end{pgfscope}%
\begin{pgfscope}%
\pgfpathrectangle{\pgfqpoint{3.525000in}{0.100000in}}{\pgfqpoint{2.857344in}{1.829167in}}%
\pgfusepath{clip}%
\pgfsetbuttcap%
\pgfsetmiterjoin%
\definecolor{currentfill}{rgb}{0.633449,0.852134,0.643676}%
\pgfsetfillcolor{currentfill}%
\pgfsetlinewidth{0.000000pt}%
\definecolor{currentstroke}{rgb}{0.000000,0.000000,0.000000}%
\pgfsetstrokecolor{currentstroke}%
\pgfsetstrokeopacity{0.000000}%
\pgfsetdash{}{0pt}%
\pgfpathmoveto{\pgfqpoint{5.661602in}{0.862564in}}%
\pgfpathlineto{\pgfqpoint{5.661643in}{0.876564in}}%
\pgfpathlineto{\pgfqpoint{5.673808in}{0.881940in}}%
\pgfpathlineto{\pgfqpoint{5.674319in}{0.890534in}}%
\pgfpathlineto{\pgfqpoint{5.685212in}{0.901033in}}%
\pgfpathlineto{\pgfqpoint{5.698828in}{0.903183in}}%
\pgfpathlineto{\pgfqpoint{5.710798in}{0.914753in}}%
\pgfpathlineto{\pgfqpoint{5.723299in}{0.919924in}}%
\pgfpathlineto{\pgfqpoint{5.732682in}{0.935449in}}%
\pgfpathlineto{\pgfqpoint{5.741227in}{0.934322in}}%
\pgfpathlineto{\pgfqpoint{5.759286in}{0.948467in}}%
\pgfpathlineto{\pgfqpoint{5.768823in}{0.948734in}}%
\pgfpathlineto{\pgfqpoint{5.772851in}{0.959512in}}%
\pgfpathlineto{\pgfqpoint{5.780430in}{0.967016in}}%
\pgfpathlineto{\pgfqpoint{5.782165in}{0.979801in}}%
\pgfpathlineto{\pgfqpoint{5.798476in}{0.981178in}}%
\pgfpathlineto{\pgfqpoint{5.864415in}{0.989887in}}%
\pgfpathlineto{\pgfqpoint{5.942054in}{1.003262in}}%
\pgfpathlineto{\pgfqpoint{5.998209in}{1.014755in}}%
\pgfpathlineto{\pgfqpoint{6.060991in}{1.027778in}}%
\pgfpathlineto{\pgfqpoint{6.078931in}{1.003148in}}%
\pgfpathlineto{\pgfqpoint{6.069213in}{1.004720in}}%
\pgfpathlineto{\pgfqpoint{6.056739in}{1.001681in}}%
\pgfpathlineto{\pgfqpoint{6.044448in}{0.989129in}}%
\pgfpathlineto{\pgfqpoint{6.035614in}{0.990031in}}%
\pgfpathlineto{\pgfqpoint{6.034496in}{0.982576in}}%
\pgfpathlineto{\pgfqpoint{6.050463in}{0.988472in}}%
\pgfpathlineto{\pgfqpoint{6.066531in}{0.990903in}}%
\pgfpathlineto{\pgfqpoint{6.072497in}{0.987668in}}%
\pgfpathlineto{\pgfqpoint{6.080526in}{0.991356in}}%
\pgfpathlineto{\pgfqpoint{6.084669in}{0.988772in}}%
\pgfpathlineto{\pgfqpoint{6.088336in}{0.976469in}}%
\pgfpathlineto{\pgfqpoint{6.080704in}{0.972659in}}%
\pgfpathlineto{\pgfqpoint{6.075467in}{0.957656in}}%
\pgfpathlineto{\pgfqpoint{6.069975in}{0.951815in}}%
\pgfpathlineto{\pgfqpoint{6.053141in}{0.953093in}}%
\pgfpathlineto{\pgfqpoint{6.054046in}{0.961902in}}%
\pgfpathlineto{\pgfqpoint{6.044852in}{0.958054in}}%
\pgfpathlineto{\pgfqpoint{6.042852in}{0.950707in}}%
\pgfpathlineto{\pgfqpoint{6.049154in}{0.943097in}}%
\pgfpathlineto{\pgfqpoint{6.051295in}{0.930180in}}%
\pgfpathlineto{\pgfqpoint{6.040983in}{0.922820in}}%
\pgfpathlineto{\pgfqpoint{6.052135in}{0.920231in}}%
\pgfpathlineto{\pgfqpoint{6.057182in}{0.925643in}}%
\pgfpathlineto{\pgfqpoint{6.068341in}{0.926302in}}%
\pgfpathlineto{\pgfqpoint{6.062606in}{0.914616in}}%
\pgfpathlineto{\pgfqpoint{6.055583in}{0.908985in}}%
\pgfpathlineto{\pgfqpoint{6.034338in}{0.903347in}}%
\pgfpathlineto{\pgfqpoint{6.012525in}{0.883564in}}%
\pgfpathlineto{\pgfqpoint{5.999075in}{0.864071in}}%
\pgfpathlineto{\pgfqpoint{5.998996in}{0.856155in}}%
\pgfpathlineto{\pgfqpoint{5.993623in}{0.845231in}}%
\pgfpathlineto{\pgfqpoint{5.966050in}{0.838044in}}%
\pgfpathlineto{\pgfqpoint{5.900476in}{0.885050in}}%
\pgfpathlineto{\pgfqpoint{5.842728in}{0.876565in}}%
\pgfpathlineto{\pgfqpoint{5.842244in}{0.884400in}}%
\pgfpathlineto{\pgfqpoint{5.833467in}{0.893223in}}%
\pgfpathlineto{\pgfqpoint{5.826822in}{0.895382in}}%
\pgfpathlineto{\pgfqpoint{5.764078in}{0.888863in}}%
\pgfpathlineto{\pgfqpoint{5.749657in}{0.883972in}}%
\pgfpathlineto{\pgfqpoint{5.723617in}{0.871018in}}%
\pgfpathlineto{\pgfqpoint{5.701114in}{0.867433in}}%
\pgfpathclose%
\pgfusepath{fill}%
\end{pgfscope}%
\begin{pgfscope}%
\pgfpathrectangle{\pgfqpoint{3.525000in}{0.100000in}}{\pgfqpoint{2.857344in}{1.829167in}}%
\pgfusepath{clip}%
\pgfsetbuttcap%
\pgfsetmiterjoin%
\definecolor{currentfill}{rgb}{0.747712,0.898039,0.627451}%
\pgfsetfillcolor{currentfill}%
\pgfsetlinewidth{0.000000pt}%
\definecolor{currentstroke}{rgb}{0.000000,0.000000,0.000000}%
\pgfsetstrokecolor{currentstroke}%
\pgfsetstrokeopacity{0.000000}%
\pgfsetdash{}{0pt}%
\pgfpathmoveto{\pgfqpoint{5.661602in}{0.862564in}}%
\pgfpathlineto{\pgfqpoint{5.595934in}{0.855352in}}%
\pgfpathlineto{\pgfqpoint{5.535860in}{0.849950in}}%
\pgfpathlineto{\pgfqpoint{5.474117in}{0.845957in}}%
\pgfpathlineto{\pgfqpoint{5.463519in}{0.844421in}}%
\pgfpathlineto{\pgfqpoint{5.421898in}{0.841219in}}%
\pgfpathlineto{\pgfqpoint{5.355235in}{0.837329in}}%
\pgfpathlineto{\pgfqpoint{5.367095in}{0.847174in}}%
\pgfpathlineto{\pgfqpoint{5.364389in}{0.857537in}}%
\pgfpathlineto{\pgfqpoint{5.367011in}{0.870099in}}%
\pgfpathlineto{\pgfqpoint{5.371016in}{0.876013in}}%
\pgfpathlineto{\pgfqpoint{5.370865in}{0.884241in}}%
\pgfpathlineto{\pgfqpoint{5.381530in}{0.889416in}}%
\pgfpathlineto{\pgfqpoint{5.381548in}{0.901525in}}%
\pgfpathlineto{\pgfqpoint{5.383014in}{0.915483in}}%
\pgfpathlineto{\pgfqpoint{5.390450in}{0.924404in}}%
\pgfpathlineto{\pgfqpoint{5.389216in}{0.933253in}}%
\pgfpathlineto{\pgfqpoint{5.392099in}{0.933455in}}%
\pgfpathlineto{\pgfqpoint{5.395155in}{0.933898in}}%
\pgfpathlineto{\pgfqpoint{5.464114in}{0.938281in}}%
\pgfpathlineto{\pgfqpoint{5.461805in}{0.949609in}}%
\pgfpathlineto{\pgfqpoint{5.472902in}{0.948072in}}%
\pgfpathlineto{\pgfqpoint{5.540071in}{0.954717in}}%
\pgfpathlineto{\pgfqpoint{5.593986in}{0.957768in}}%
\pgfpathlineto{\pgfqpoint{5.626842in}{0.960531in}}%
\pgfpathlineto{\pgfqpoint{5.681248in}{0.965690in}}%
\pgfpathlineto{\pgfqpoint{5.682156in}{0.966716in}}%
\pgfpathlineto{\pgfqpoint{5.705402in}{0.969018in}}%
\pgfpathlineto{\pgfqpoint{5.782165in}{0.979801in}}%
\pgfpathlineto{\pgfqpoint{5.780430in}{0.967016in}}%
\pgfpathlineto{\pgfqpoint{5.772851in}{0.959512in}}%
\pgfpathlineto{\pgfqpoint{5.768823in}{0.948734in}}%
\pgfpathlineto{\pgfqpoint{5.759286in}{0.948467in}}%
\pgfpathlineto{\pgfqpoint{5.741227in}{0.934322in}}%
\pgfpathlineto{\pgfqpoint{5.732682in}{0.935449in}}%
\pgfpathlineto{\pgfqpoint{5.723299in}{0.919924in}}%
\pgfpathlineto{\pgfqpoint{5.710798in}{0.914753in}}%
\pgfpathlineto{\pgfqpoint{5.698828in}{0.903183in}}%
\pgfpathlineto{\pgfqpoint{5.685212in}{0.901033in}}%
\pgfpathlineto{\pgfqpoint{5.674319in}{0.890534in}}%
\pgfpathlineto{\pgfqpoint{5.673808in}{0.881940in}}%
\pgfpathlineto{\pgfqpoint{5.661643in}{0.876564in}}%
\pgfpathclose%
\pgfusepath{fill}%
\end{pgfscope}%
\begin{pgfscope}%
\pgfpathrectangle{\pgfqpoint{3.525000in}{0.100000in}}{\pgfqpoint{2.857344in}{1.829167in}}%
\pgfusepath{clip}%
\pgfsetbuttcap%
\pgfsetmiterjoin%
\definecolor{currentfill}{rgb}{0.601615,0.839677,0.644137}%
\pgfsetfillcolor{currentfill}%
\pgfsetlinewidth{0.000000pt}%
\definecolor{currentstroke}{rgb}{0.000000,0.000000,0.000000}%
\pgfsetstrokecolor{currentstroke}%
\pgfsetstrokeopacity{0.000000}%
\pgfsetdash{}{0pt}%
\pgfpathmoveto{\pgfqpoint{4.496964in}{0.668149in}}%
\pgfpathlineto{\pgfqpoint{4.492281in}{0.675675in}}%
\pgfpathlineto{\pgfqpoint{4.494219in}{0.682157in}}%
\pgfpathlineto{\pgfqpoint{4.527139in}{0.678074in}}%
\pgfpathlineto{\pgfqpoint{4.588353in}{0.671121in}}%
\pgfpathlineto{\pgfqpoint{4.632614in}{0.666772in}}%
\pgfpathlineto{\pgfqpoint{4.683733in}{0.661614in}}%
\pgfpathlineto{\pgfqpoint{4.686514in}{0.693857in}}%
\pgfpathlineto{\pgfqpoint{4.695210in}{0.775867in}}%
\pgfpathlineto{\pgfqpoint{4.700850in}{0.833319in}}%
\pgfpathlineto{\pgfqpoint{4.705726in}{0.888200in}}%
\pgfpathlineto{\pgfqpoint{4.710272in}{0.942408in}}%
\pgfpathlineto{\pgfqpoint{4.711775in}{0.942324in}}%
\pgfpathlineto{\pgfqpoint{4.753511in}{0.938700in}}%
\pgfpathlineto{\pgfqpoint{4.807532in}{0.934616in}}%
\pgfpathlineto{\pgfqpoint{4.862276in}{0.931553in}}%
\pgfpathlineto{\pgfqpoint{4.858941in}{0.864698in}}%
\pgfpathlineto{\pgfqpoint{4.855879in}{0.810683in}}%
\pgfpathlineto{\pgfqpoint{4.859250in}{0.811555in}}%
\pgfpathlineto{\pgfqpoint{4.871347in}{0.798235in}}%
\pgfpathlineto{\pgfqpoint{4.876988in}{0.799926in}}%
\pgfpathlineto{\pgfqpoint{4.885554in}{0.797000in}}%
\pgfpathlineto{\pgfqpoint{4.893032in}{0.798477in}}%
\pgfpathlineto{\pgfqpoint{4.897208in}{0.786617in}}%
\pgfpathlineto{\pgfqpoint{4.906537in}{0.785994in}}%
\pgfpathlineto{\pgfqpoint{4.921433in}{0.780410in}}%
\pgfpathlineto{\pgfqpoint{4.925622in}{0.781984in}}%
\pgfpathlineto{\pgfqpoint{4.934609in}{0.775092in}}%
\pgfpathlineto{\pgfqpoint{4.939899in}{0.780410in}}%
\pgfpathlineto{\pgfqpoint{4.949066in}{0.778399in}}%
\pgfpathlineto{\pgfqpoint{4.953334in}{0.771334in}}%
\pgfpathlineto{\pgfqpoint{4.960343in}{0.768954in}}%
\pgfpathlineto{\pgfqpoint{4.965206in}{0.761626in}}%
\pgfpathlineto{\pgfqpoint{4.974915in}{0.769601in}}%
\pgfpathlineto{\pgfqpoint{4.986027in}{0.759707in}}%
\pgfpathlineto{\pgfqpoint{4.998730in}{0.763748in}}%
\pgfpathlineto{\pgfqpoint{5.005162in}{0.753333in}}%
\pgfpathlineto{\pgfqpoint{5.011451in}{0.765444in}}%
\pgfpathlineto{\pgfqpoint{5.022757in}{0.758577in}}%
\pgfpathlineto{\pgfqpoint{5.029821in}{0.760404in}}%
\pgfpathlineto{\pgfqpoint{5.039906in}{0.755549in}}%
\pgfpathlineto{\pgfqpoint{5.052982in}{0.754170in}}%
\pgfpathlineto{\pgfqpoint{5.065377in}{0.761642in}}%
\pgfpathlineto{\pgfqpoint{5.069861in}{0.758488in}}%
\pgfpathlineto{\pgfqpoint{5.082457in}{0.764893in}}%
\pgfpathlineto{\pgfqpoint{5.090484in}{0.759894in}}%
\pgfpathlineto{\pgfqpoint{5.096829in}{0.760192in}}%
\pgfpathlineto{\pgfqpoint{5.101853in}{0.766079in}}%
\pgfpathlineto{\pgfqpoint{5.109760in}{0.763149in}}%
\pgfpathlineto{\pgfqpoint{5.127560in}{0.749517in}}%
\pgfpathlineto{\pgfqpoint{5.141452in}{0.745221in}}%
\pgfpathlineto{\pgfqpoint{5.147023in}{0.739958in}}%
\pgfpathlineto{\pgfqpoint{5.153996in}{0.742824in}}%
\pgfpathlineto{\pgfqpoint{5.164562in}{0.740629in}}%
\pgfpathlineto{\pgfqpoint{5.164769in}{0.707128in}}%
\pgfpathlineto{\pgfqpoint{5.165651in}{0.642332in}}%
\pgfpathlineto{\pgfqpoint{5.172989in}{0.636106in}}%
\pgfpathlineto{\pgfqpoint{5.178872in}{0.624633in}}%
\pgfpathlineto{\pgfqpoint{5.176804in}{0.616957in}}%
\pgfpathlineto{\pgfqpoint{5.181423in}{0.613707in}}%
\pgfpathlineto{\pgfqpoint{5.185177in}{0.599552in}}%
\pgfpathlineto{\pgfqpoint{5.192366in}{0.591971in}}%
\pgfpathlineto{\pgfqpoint{5.193977in}{0.575877in}}%
\pgfpathlineto{\pgfqpoint{5.192727in}{0.569062in}}%
\pgfpathlineto{\pgfqpoint{5.182954in}{0.551028in}}%
\pgfpathlineto{\pgfqpoint{5.181831in}{0.538062in}}%
\pgfpathlineto{\pgfqpoint{5.185145in}{0.535356in}}%
\pgfpathlineto{\pgfqpoint{5.185434in}{0.523996in}}%
\pgfpathlineto{\pgfqpoint{5.182064in}{0.516857in}}%
\pgfpathlineto{\pgfqpoint{5.176824in}{0.514168in}}%
\pgfpathlineto{\pgfqpoint{5.171693in}{0.504838in}}%
\pgfpathlineto{\pgfqpoint{5.178232in}{0.495802in}}%
\pgfpathlineto{\pgfqpoint{5.165538in}{0.495624in}}%
\pgfpathlineto{\pgfqpoint{5.131594in}{0.480094in}}%
\pgfpathlineto{\pgfqpoint{5.138075in}{0.489388in}}%
\pgfpathlineto{\pgfqpoint{5.130227in}{0.494399in}}%
\pgfpathlineto{\pgfqpoint{5.128587in}{0.502919in}}%
\pgfpathlineto{\pgfqpoint{5.113271in}{0.488079in}}%
\pgfpathlineto{\pgfqpoint{5.120068in}{0.477936in}}%
\pgfpathlineto{\pgfqpoint{5.110371in}{0.465021in}}%
\pgfpathlineto{\pgfqpoint{5.105153in}{0.465291in}}%
\pgfpathlineto{\pgfqpoint{5.100244in}{0.451227in}}%
\pgfpathlineto{\pgfqpoint{5.084719in}{0.440163in}}%
\pgfpathlineto{\pgfqpoint{5.070214in}{0.436176in}}%
\pgfpathlineto{\pgfqpoint{5.044917in}{0.425803in}}%
\pgfpathlineto{\pgfqpoint{5.042295in}{0.431590in}}%
\pgfpathlineto{\pgfqpoint{5.026537in}{0.426264in}}%
\pgfpathlineto{\pgfqpoint{5.036229in}{0.417196in}}%
\pgfpathlineto{\pgfqpoint{5.020941in}{0.409318in}}%
\pgfpathlineto{\pgfqpoint{5.004451in}{0.397424in}}%
\pgfpathlineto{\pgfqpoint{5.000490in}{0.402953in}}%
\pgfpathlineto{\pgfqpoint{4.986995in}{0.394666in}}%
\pgfpathlineto{\pgfqpoint{5.000222in}{0.391519in}}%
\pgfpathlineto{\pgfqpoint{4.990249in}{0.378489in}}%
\pgfpathlineto{\pgfqpoint{4.985381in}{0.382367in}}%
\pgfpathlineto{\pgfqpoint{4.978843in}{0.376138in}}%
\pgfpathlineto{\pgfqpoint{4.986992in}{0.370699in}}%
\pgfpathlineto{\pgfqpoint{4.982174in}{0.362747in}}%
\pgfpathlineto{\pgfqpoint{4.976252in}{0.343923in}}%
\pgfpathlineto{\pgfqpoint{4.967711in}{0.325793in}}%
\pgfpathlineto{\pgfqpoint{4.971545in}{0.313909in}}%
\pgfpathlineto{\pgfqpoint{4.978063in}{0.285891in}}%
\pgfpathlineto{\pgfqpoint{4.978650in}{0.274552in}}%
\pgfpathlineto{\pgfqpoint{4.988720in}{0.259679in}}%
\pgfpathlineto{\pgfqpoint{4.980955in}{0.260548in}}%
\pgfpathlineto{\pgfqpoint{4.973412in}{0.253045in}}%
\pgfpathlineto{\pgfqpoint{4.964578in}{0.258942in}}%
\pgfpathlineto{\pgfqpoint{4.961407in}{0.264825in}}%
\pgfpathlineto{\pgfqpoint{4.948834in}{0.267561in}}%
\pgfpathlineto{\pgfqpoint{4.929629in}{0.267897in}}%
\pgfpathlineto{\pgfqpoint{4.915479in}{0.279038in}}%
\pgfpathlineto{\pgfqpoint{4.902631in}{0.280899in}}%
\pgfpathlineto{\pgfqpoint{4.894836in}{0.289752in}}%
\pgfpathlineto{\pgfqpoint{4.878511in}{0.293315in}}%
\pgfpathlineto{\pgfqpoint{4.869585in}{0.321783in}}%
\pgfpathlineto{\pgfqpoint{4.860458in}{0.333188in}}%
\pgfpathlineto{\pgfqpoint{4.862008in}{0.344024in}}%
\pgfpathlineto{\pgfqpoint{4.856349in}{0.351943in}}%
\pgfpathlineto{\pgfqpoint{4.859901in}{0.362767in}}%
\pgfpathlineto{\pgfqpoint{4.856971in}{0.370699in}}%
\pgfpathlineto{\pgfqpoint{4.847799in}{0.374293in}}%
\pgfpathlineto{\pgfqpoint{4.839224in}{0.383439in}}%
\pgfpathlineto{\pgfqpoint{4.836099in}{0.395685in}}%
\pgfpathlineto{\pgfqpoint{4.827984in}{0.406814in}}%
\pgfpathlineto{\pgfqpoint{4.817183in}{0.415473in}}%
\pgfpathlineto{\pgfqpoint{4.807454in}{0.440312in}}%
\pgfpathlineto{\pgfqpoint{4.799599in}{0.467478in}}%
\pgfpathlineto{\pgfqpoint{4.793153in}{0.478219in}}%
\pgfpathlineto{\pgfqpoint{4.781979in}{0.487262in}}%
\pgfpathlineto{\pgfqpoint{4.779186in}{0.493825in}}%
\pgfpathlineto{\pgfqpoint{4.768723in}{0.497901in}}%
\pgfpathlineto{\pgfqpoint{4.758397in}{0.515292in}}%
\pgfpathlineto{\pgfqpoint{4.748966in}{0.513957in}}%
\pgfpathlineto{\pgfqpoint{4.739390in}{0.518301in}}%
\pgfpathlineto{\pgfqpoint{4.725816in}{0.517461in}}%
\pgfpathlineto{\pgfqpoint{4.712025in}{0.524629in}}%
\pgfpathlineto{\pgfqpoint{4.708137in}{0.517812in}}%
\pgfpathlineto{\pgfqpoint{4.692021in}{0.517641in}}%
\pgfpathlineto{\pgfqpoint{4.683816in}{0.504718in}}%
\pgfpathlineto{\pgfqpoint{4.676681in}{0.488717in}}%
\pgfpathlineto{\pgfqpoint{4.661506in}{0.471540in}}%
\pgfpathlineto{\pgfqpoint{4.644314in}{0.479078in}}%
\pgfpathlineto{\pgfqpoint{4.641875in}{0.484060in}}%
\pgfpathlineto{\pgfqpoint{4.631427in}{0.487860in}}%
\pgfpathlineto{\pgfqpoint{4.628218in}{0.493057in}}%
\pgfpathlineto{\pgfqpoint{4.614348in}{0.498303in}}%
\pgfpathlineto{\pgfqpoint{4.606589in}{0.509028in}}%
\pgfpathlineto{\pgfqpoint{4.597528in}{0.514203in}}%
\pgfpathlineto{\pgfqpoint{4.589732in}{0.523233in}}%
\pgfpathlineto{\pgfqpoint{4.583659in}{0.538525in}}%
\pgfpathlineto{\pgfqpoint{4.584338in}{0.559422in}}%
\pgfpathlineto{\pgfqpoint{4.577213in}{0.569982in}}%
\pgfpathlineto{\pgfqpoint{4.576394in}{0.581419in}}%
\pgfpathlineto{\pgfqpoint{4.560586in}{0.598542in}}%
\pgfpathlineto{\pgfqpoint{4.551387in}{0.602198in}}%
\pgfpathlineto{\pgfqpoint{4.541627in}{0.618176in}}%
\pgfpathlineto{\pgfqpoint{4.533309in}{0.624525in}}%
\pgfpathlineto{\pgfqpoint{4.522741in}{0.639983in}}%
\pgfpathlineto{\pgfqpoint{4.511927in}{0.646673in}}%
\pgfpathlineto{\pgfqpoint{4.504846in}{0.663815in}}%
\pgfpathclose%
\pgfusepath{fill}%
\end{pgfscope}%
\begin{pgfscope}%
\pgfpathrectangle{\pgfqpoint{3.525000in}{0.100000in}}{\pgfqpoint{2.857344in}{1.829167in}}%
\pgfusepath{clip}%
\pgfsetbuttcap%
\pgfsetmiterjoin%
\definecolor{currentfill}{rgb}{0.256055,0.600231,0.713495}%
\pgfsetfillcolor{currentfill}%
\pgfsetlinewidth{0.000000pt}%
\definecolor{currentstroke}{rgb}{0.000000,0.000000,0.000000}%
\pgfsetstrokecolor{currentstroke}%
\pgfsetstrokeopacity{0.000000}%
\pgfsetdash{}{0pt}%
\pgfpathmoveto{\pgfqpoint{4.414159in}{1.010405in}}%
\pgfpathlineto{\pgfqpoint{4.492171in}{0.999075in}}%
\pgfpathlineto{\pgfqpoint{4.522393in}{0.994419in}}%
\pgfpathlineto{\pgfqpoint{4.607585in}{0.984286in}}%
\pgfpathlineto{\pgfqpoint{4.664925in}{0.978305in}}%
\pgfpathlineto{\pgfqpoint{4.714547in}{0.973908in}}%
\pgfpathlineto{\pgfqpoint{4.711775in}{0.942324in}}%
\pgfpathlineto{\pgfqpoint{4.710272in}{0.942408in}}%
\pgfpathlineto{\pgfqpoint{4.705726in}{0.888200in}}%
\pgfpathlineto{\pgfqpoint{4.700850in}{0.833319in}}%
\pgfpathlineto{\pgfqpoint{4.695210in}{0.775867in}}%
\pgfpathlineto{\pgfqpoint{4.686514in}{0.693857in}}%
\pgfpathlineto{\pgfqpoint{4.683733in}{0.661614in}}%
\pgfpathlineto{\pgfqpoint{4.632614in}{0.666772in}}%
\pgfpathlineto{\pgfqpoint{4.588353in}{0.671121in}}%
\pgfpathlineto{\pgfqpoint{4.527139in}{0.678074in}}%
\pgfpathlineto{\pgfqpoint{4.494219in}{0.682157in}}%
\pgfpathlineto{\pgfqpoint{4.492281in}{0.675675in}}%
\pgfpathlineto{\pgfqpoint{4.496964in}{0.668149in}}%
\pgfpathlineto{\pgfqpoint{4.457399in}{0.673287in}}%
\pgfpathlineto{\pgfqpoint{4.408585in}{0.680293in}}%
\pgfpathlineto{\pgfqpoint{4.404142in}{0.652684in}}%
\pgfpathlineto{\pgfqpoint{4.359616in}{0.659386in}}%
\pgfpathlineto{\pgfqpoint{4.370078in}{0.727879in}}%
\pgfpathlineto{\pgfqpoint{4.383029in}{0.811272in}}%
\pgfpathlineto{\pgfqpoint{4.394545in}{0.883922in}}%
\pgfpathlineto{\pgfqpoint{4.404545in}{0.948497in}}%
\pgfpathclose%
\pgfusepath{fill}%
\end{pgfscope}%
\begin{pgfscope}%
\pgfpathrectangle{\pgfqpoint{3.525000in}{0.100000in}}{\pgfqpoint{2.857344in}{1.829167in}}%
\pgfusepath{clip}%
\pgfsetbuttcap%
\pgfsetmiterjoin%
\definecolor{currentfill}{rgb}{0.747712,0.898039,0.627451}%
\pgfsetfillcolor{currentfill}%
\pgfsetlinewidth{0.000000pt}%
\definecolor{currentstroke}{rgb}{0.000000,0.000000,0.000000}%
\pgfsetstrokecolor{currentstroke}%
\pgfsetstrokeopacity{0.000000}%
\pgfsetdash{}{0pt}%
\pgfpathmoveto{\pgfqpoint{5.654545in}{0.609114in}}%
\pgfpathlineto{\pgfqpoint{5.580121in}{0.600848in}}%
\pgfpathlineto{\pgfqpoint{5.514518in}{0.595755in}}%
\pgfpathlineto{\pgfqpoint{5.513697in}{0.587716in}}%
\pgfpathlineto{\pgfqpoint{5.519723in}{0.580063in}}%
\pgfpathlineto{\pgfqpoint{5.527078in}{0.575568in}}%
\pgfpathlineto{\pgfqpoint{5.526967in}{0.563732in}}%
\pgfpathlineto{\pgfqpoint{5.525018in}{0.555823in}}%
\pgfpathlineto{\pgfqpoint{5.518527in}{0.550125in}}%
\pgfpathlineto{\pgfqpoint{5.509479in}{0.550728in}}%
\pgfpathlineto{\pgfqpoint{5.500924in}{0.557789in}}%
\pgfpathlineto{\pgfqpoint{5.499399in}{0.570355in}}%
\pgfpathlineto{\pgfqpoint{5.493028in}{0.577663in}}%
\pgfpathlineto{\pgfqpoint{5.488691in}{0.551507in}}%
\pgfpathlineto{\pgfqpoint{5.473956in}{0.553993in}}%
\pgfpathlineto{\pgfqpoint{5.468831in}{0.599731in}}%
\pgfpathlineto{\pgfqpoint{5.463282in}{0.647933in}}%
\pgfpathlineto{\pgfqpoint{5.465343in}{0.717478in}}%
\pgfpathlineto{\pgfqpoint{5.466622in}{0.765200in}}%
\pgfpathlineto{\pgfqpoint{5.469343in}{0.838013in}}%
\pgfpathlineto{\pgfqpoint{5.463519in}{0.844421in}}%
\pgfpathlineto{\pgfqpoint{5.474117in}{0.845957in}}%
\pgfpathlineto{\pgfqpoint{5.535860in}{0.849950in}}%
\pgfpathlineto{\pgfqpoint{5.595934in}{0.855352in}}%
\pgfpathlineto{\pgfqpoint{5.611738in}{0.799991in}}%
\pgfpathlineto{\pgfqpoint{5.622517in}{0.759356in}}%
\pgfpathlineto{\pgfqpoint{5.632179in}{0.725336in}}%
\pgfpathlineto{\pgfqpoint{5.637767in}{0.711638in}}%
\pgfpathlineto{\pgfqpoint{5.646565in}{0.698890in}}%
\pgfpathlineto{\pgfqpoint{5.645144in}{0.692403in}}%
\pgfpathlineto{\pgfqpoint{5.651429in}{0.689242in}}%
\pgfpathlineto{\pgfqpoint{5.644002in}{0.679409in}}%
\pgfpathlineto{\pgfqpoint{5.641493in}{0.661896in}}%
\pgfpathlineto{\pgfqpoint{5.648752in}{0.641414in}}%
\pgfpathlineto{\pgfqpoint{5.647744in}{0.620803in}}%
\pgfpathclose%
\pgfusepath{fill}%
\end{pgfscope}%
\begin{pgfscope}%
\pgfpathrectangle{\pgfqpoint{3.525000in}{0.100000in}}{\pgfqpoint{2.857344in}{1.829167in}}%
\pgfusepath{clip}%
\pgfsetbuttcap%
\pgfsetmiterjoin%
\definecolor{currentfill}{rgb}{0.747712,0.898039,0.627451}%
\pgfsetfillcolor{currentfill}%
\pgfsetlinewidth{0.000000pt}%
\definecolor{currentstroke}{rgb}{0.000000,0.000000,0.000000}%
\pgfsetstrokecolor{currentstroke}%
\pgfsetstrokeopacity{0.000000}%
\pgfsetdash{}{0pt}%
\pgfpathmoveto{\pgfqpoint{5.473956in}{0.553993in}}%
\pgfpathlineto{\pgfqpoint{5.458834in}{0.549666in}}%
\pgfpathlineto{\pgfqpoint{5.447562in}{0.552444in}}%
\pgfpathlineto{\pgfqpoint{5.426630in}{0.545786in}}%
\pgfpathlineto{\pgfqpoint{5.410842in}{0.537211in}}%
\pgfpathlineto{\pgfqpoint{5.404365in}{0.552706in}}%
\pgfpathlineto{\pgfqpoint{5.397687in}{0.559201in}}%
\pgfpathlineto{\pgfqpoint{5.394466in}{0.566972in}}%
\pgfpathlineto{\pgfqpoint{5.399188in}{0.588008in}}%
\pgfpathlineto{\pgfqpoint{5.354448in}{0.585253in}}%
\pgfpathlineto{\pgfqpoint{5.296435in}{0.582738in}}%
\pgfpathlineto{\pgfqpoint{5.299886in}{0.587974in}}%
\pgfpathlineto{\pgfqpoint{5.295620in}{0.597864in}}%
\pgfpathlineto{\pgfqpoint{5.299871in}{0.599879in}}%
\pgfpathlineto{\pgfqpoint{5.298928in}{0.609381in}}%
\pgfpathlineto{\pgfqpoint{5.306337in}{0.618595in}}%
\pgfpathlineto{\pgfqpoint{5.310391in}{0.636483in}}%
\pgfpathlineto{\pgfqpoint{5.323952in}{0.648268in}}%
\pgfpathlineto{\pgfqpoint{5.318926in}{0.659713in}}%
\pgfpathlineto{\pgfqpoint{5.328542in}{0.666301in}}%
\pgfpathlineto{\pgfqpoint{5.320247in}{0.678228in}}%
\pgfpathlineto{\pgfqpoint{5.314243in}{0.705532in}}%
\pgfpathlineto{\pgfqpoint{5.316520in}{0.710489in}}%
\pgfpathlineto{\pgfqpoint{5.320115in}{0.719999in}}%
\pgfpathlineto{\pgfqpoint{5.316769in}{0.729979in}}%
\pgfpathlineto{\pgfqpoint{5.318406in}{0.734520in}}%
\pgfpathlineto{\pgfqpoint{5.312519in}{0.751673in}}%
\pgfpathlineto{\pgfqpoint{5.329896in}{0.782473in}}%
\pgfpathlineto{\pgfqpoint{5.330704in}{0.790671in}}%
\pgfpathlineto{\pgfqpoint{5.339075in}{0.796632in}}%
\pgfpathlineto{\pgfqpoint{5.348006in}{0.816311in}}%
\pgfpathlineto{\pgfqpoint{5.347582in}{0.824310in}}%
\pgfpathlineto{\pgfqpoint{5.358706in}{0.832486in}}%
\pgfpathlineto{\pgfqpoint{5.355235in}{0.837329in}}%
\pgfpathlineto{\pgfqpoint{5.421898in}{0.841219in}}%
\pgfpathlineto{\pgfqpoint{5.463519in}{0.844421in}}%
\pgfpathlineto{\pgfqpoint{5.469343in}{0.838013in}}%
\pgfpathlineto{\pgfqpoint{5.466622in}{0.765200in}}%
\pgfpathlineto{\pgfqpoint{5.465343in}{0.717478in}}%
\pgfpathlineto{\pgfqpoint{5.463282in}{0.647933in}}%
\pgfpathlineto{\pgfqpoint{5.468831in}{0.599731in}}%
\pgfpathclose%
\pgfusepath{fill}%
\end{pgfscope}%
\begin{pgfscope}%
\pgfpathrectangle{\pgfqpoint{3.525000in}{0.100000in}}{\pgfqpoint{2.857344in}{1.829167in}}%
\pgfusepath{clip}%
\pgfsetbuttcap%
\pgfsetmiterjoin%
\definecolor{currentfill}{rgb}{0.729566,0.890657,0.631142}%
\pgfsetfillcolor{currentfill}%
\pgfsetlinewidth{0.000000pt}%
\definecolor{currentstroke}{rgb}{0.000000,0.000000,0.000000}%
\pgfsetstrokecolor{currentstroke}%
\pgfsetstrokeopacity{0.000000}%
\pgfsetdash{}{0pt}%
\pgfpathmoveto{\pgfqpoint{5.595934in}{0.855352in}}%
\pgfpathlineto{\pgfqpoint{5.661602in}{0.862564in}}%
\pgfpathlineto{\pgfqpoint{5.701114in}{0.867433in}}%
\pgfpathlineto{\pgfqpoint{5.723617in}{0.871018in}}%
\pgfpathlineto{\pgfqpoint{5.714295in}{0.856315in}}%
\pgfpathlineto{\pgfqpoint{5.714301in}{0.849383in}}%
\pgfpathlineto{\pgfqpoint{5.723897in}{0.845645in}}%
\pgfpathlineto{\pgfqpoint{5.730422in}{0.839610in}}%
\pgfpathlineto{\pgfqpoint{5.740284in}{0.838864in}}%
\pgfpathlineto{\pgfqpoint{5.749501in}{0.821874in}}%
\pgfpathlineto{\pgfqpoint{5.759499in}{0.809893in}}%
\pgfpathlineto{\pgfqpoint{5.777255in}{0.799823in}}%
\pgfpathlineto{\pgfqpoint{5.782302in}{0.791792in}}%
\pgfpathlineto{\pgfqpoint{5.798136in}{0.783106in}}%
\pgfpathlineto{\pgfqpoint{5.799327in}{0.776807in}}%
\pgfpathlineto{\pgfqpoint{5.812172in}{0.764281in}}%
\pgfpathlineto{\pgfqpoint{5.822012in}{0.760715in}}%
\pgfpathlineto{\pgfqpoint{5.828979in}{0.748888in}}%
\pgfpathlineto{\pgfqpoint{5.832063in}{0.735594in}}%
\pgfpathlineto{\pgfqpoint{5.842283in}{0.730456in}}%
\pgfpathlineto{\pgfqpoint{5.850511in}{0.716166in}}%
\pgfpathlineto{\pgfqpoint{5.853164in}{0.705657in}}%
\pgfpathlineto{\pgfqpoint{5.865221in}{0.701184in}}%
\pgfpathlineto{\pgfqpoint{5.855104in}{0.681813in}}%
\pgfpathlineto{\pgfqpoint{5.851297in}{0.670365in}}%
\pgfpathlineto{\pgfqpoint{5.853781in}{0.664999in}}%
\pgfpathlineto{\pgfqpoint{5.849443in}{0.656081in}}%
\pgfpathlineto{\pgfqpoint{5.847599in}{0.643751in}}%
\pgfpathlineto{\pgfqpoint{5.839886in}{0.641454in}}%
\pgfpathlineto{\pgfqpoint{5.843959in}{0.630165in}}%
\pgfpathlineto{\pgfqpoint{5.843699in}{0.615843in}}%
\pgfpathlineto{\pgfqpoint{5.829989in}{0.616380in}}%
\pgfpathlineto{\pgfqpoint{5.820438in}{0.618993in}}%
\pgfpathlineto{\pgfqpoint{5.816305in}{0.614136in}}%
\pgfpathlineto{\pgfqpoint{5.819389in}{0.602660in}}%
\pgfpathlineto{\pgfqpoint{5.818718in}{0.589319in}}%
\pgfpathlineto{\pgfqpoint{5.812692in}{0.588298in}}%
\pgfpathlineto{\pgfqpoint{5.807818in}{0.600758in}}%
\pgfpathlineto{\pgfqpoint{5.758184in}{0.597451in}}%
\pgfpathlineto{\pgfqpoint{5.695600in}{0.594036in}}%
\pgfpathlineto{\pgfqpoint{5.664037in}{0.591814in}}%
\pgfpathlineto{\pgfqpoint{5.654545in}{0.609114in}}%
\pgfpathlineto{\pgfqpoint{5.647744in}{0.620803in}}%
\pgfpathlineto{\pgfqpoint{5.648752in}{0.641414in}}%
\pgfpathlineto{\pgfqpoint{5.641493in}{0.661896in}}%
\pgfpathlineto{\pgfqpoint{5.644002in}{0.679409in}}%
\pgfpathlineto{\pgfqpoint{5.651429in}{0.689242in}}%
\pgfpathlineto{\pgfqpoint{5.645144in}{0.692403in}}%
\pgfpathlineto{\pgfqpoint{5.646565in}{0.698890in}}%
\pgfpathlineto{\pgfqpoint{5.637767in}{0.711638in}}%
\pgfpathlineto{\pgfqpoint{5.632179in}{0.725336in}}%
\pgfpathlineto{\pgfqpoint{5.622517in}{0.759356in}}%
\pgfpathlineto{\pgfqpoint{5.611738in}{0.799991in}}%
\pgfpathclose%
\pgfusepath{fill}%
\end{pgfscope}%
\begin{pgfscope}%
\pgfpathrectangle{\pgfqpoint{3.525000in}{0.100000in}}{\pgfqpoint{2.857344in}{1.829167in}}%
\pgfusepath{clip}%
\pgfsetbuttcap%
\pgfsetmiterjoin%
\definecolor{currentfill}{rgb}{0.580392,0.831373,0.644444}%
\pgfsetfillcolor{currentfill}%
\pgfsetlinewidth{0.000000pt}%
\definecolor{currentstroke}{rgb}{0.000000,0.000000,0.000000}%
\pgfsetstrokecolor{currentstroke}%
\pgfsetstrokeopacity{0.000000}%
\pgfsetdash{}{0pt}%
\pgfpathmoveto{\pgfqpoint{5.723617in}{0.871018in}}%
\pgfpathlineto{\pgfqpoint{5.749657in}{0.883972in}}%
\pgfpathlineto{\pgfqpoint{5.764078in}{0.888863in}}%
\pgfpathlineto{\pgfqpoint{5.826822in}{0.895382in}}%
\pgfpathlineto{\pgfqpoint{5.833467in}{0.893223in}}%
\pgfpathlineto{\pgfqpoint{5.842244in}{0.884400in}}%
\pgfpathlineto{\pgfqpoint{5.842728in}{0.876565in}}%
\pgfpathlineto{\pgfqpoint{5.900476in}{0.885050in}}%
\pgfpathlineto{\pgfqpoint{5.966050in}{0.838044in}}%
\pgfpathlineto{\pgfqpoint{5.953753in}{0.825226in}}%
\pgfpathlineto{\pgfqpoint{5.936879in}{0.795460in}}%
\pgfpathlineto{\pgfqpoint{5.941682in}{0.789062in}}%
\pgfpathlineto{\pgfqpoint{5.932680in}{0.776646in}}%
\pgfpathlineto{\pgfqpoint{5.923727in}{0.775240in}}%
\pgfpathlineto{\pgfqpoint{5.923693in}{0.767778in}}%
\pgfpathlineto{\pgfqpoint{5.917216in}{0.759971in}}%
\pgfpathlineto{\pgfqpoint{5.909156in}{0.758387in}}%
\pgfpathlineto{\pgfqpoint{5.910915in}{0.751450in}}%
\pgfpathlineto{\pgfqpoint{5.906417in}{0.746124in}}%
\pgfpathlineto{\pgfqpoint{5.895657in}{0.741524in}}%
\pgfpathlineto{\pgfqpoint{5.882542in}{0.731058in}}%
\pgfpathlineto{\pgfqpoint{5.885018in}{0.724287in}}%
\pgfpathlineto{\pgfqpoint{5.876764in}{0.720036in}}%
\pgfpathlineto{\pgfqpoint{5.869803in}{0.724506in}}%
\pgfpathlineto{\pgfqpoint{5.864712in}{0.705073in}}%
\pgfpathlineto{\pgfqpoint{5.853164in}{0.705657in}}%
\pgfpathlineto{\pgfqpoint{5.850511in}{0.716166in}}%
\pgfpathlineto{\pgfqpoint{5.842283in}{0.730456in}}%
\pgfpathlineto{\pgfqpoint{5.832063in}{0.735594in}}%
\pgfpathlineto{\pgfqpoint{5.828979in}{0.748888in}}%
\pgfpathlineto{\pgfqpoint{5.822012in}{0.760715in}}%
\pgfpathlineto{\pgfqpoint{5.812172in}{0.764281in}}%
\pgfpathlineto{\pgfqpoint{5.799327in}{0.776807in}}%
\pgfpathlineto{\pgfqpoint{5.798136in}{0.783106in}}%
\pgfpathlineto{\pgfqpoint{5.782302in}{0.791792in}}%
\pgfpathlineto{\pgfqpoint{5.777255in}{0.799823in}}%
\pgfpathlineto{\pgfqpoint{5.759499in}{0.809893in}}%
\pgfpathlineto{\pgfqpoint{5.749501in}{0.821874in}}%
\pgfpathlineto{\pgfqpoint{5.740284in}{0.838864in}}%
\pgfpathlineto{\pgfqpoint{5.730422in}{0.839610in}}%
\pgfpathlineto{\pgfqpoint{5.723897in}{0.845645in}}%
\pgfpathlineto{\pgfqpoint{5.714301in}{0.849383in}}%
\pgfpathlineto{\pgfqpoint{5.714295in}{0.856315in}}%
\pgfpathclose%
\pgfusepath{fill}%
\end{pgfscope}%
\begin{pgfscope}%
\pgfpathrectangle{\pgfqpoint{3.525000in}{0.100000in}}{\pgfqpoint{2.857344in}{1.829167in}}%
\pgfusepath{clip}%
\pgfsetbuttcap%
\pgfsetmiterjoin%
\definecolor{currentfill}{rgb}{0.802153,0.920185,0.616378}%
\pgfsetfillcolor{currentfill}%
\pgfsetlinewidth{0.000000pt}%
\definecolor{currentstroke}{rgb}{0.000000,0.000000,0.000000}%
\pgfsetstrokecolor{currentstroke}%
\pgfsetstrokeopacity{0.000000}%
\pgfsetdash{}{0pt}%
\pgfpathmoveto{\pgfqpoint{5.133339in}{0.924467in}}%
\pgfpathlineto{\pgfqpoint{5.171605in}{0.924811in}}%
\pgfpathlineto{\pgfqpoint{5.222229in}{0.925734in}}%
\pgfpathlineto{\pgfqpoint{5.308735in}{0.928480in}}%
\pgfpathlineto{\pgfqpoint{5.358214in}{0.931097in}}%
\pgfpathlineto{\pgfqpoint{5.363520in}{0.924528in}}%
\pgfpathlineto{\pgfqpoint{5.363189in}{0.917589in}}%
\pgfpathlineto{\pgfqpoint{5.351203in}{0.905620in}}%
\pgfpathlineto{\pgfqpoint{5.348305in}{0.899066in}}%
\pgfpathlineto{\pgfqpoint{5.381548in}{0.901525in}}%
\pgfpathlineto{\pgfqpoint{5.381530in}{0.889416in}}%
\pgfpathlineto{\pgfqpoint{5.370865in}{0.884241in}}%
\pgfpathlineto{\pgfqpoint{5.371016in}{0.876013in}}%
\pgfpathlineto{\pgfqpoint{5.367011in}{0.870099in}}%
\pgfpathlineto{\pgfqpoint{5.364389in}{0.857537in}}%
\pgfpathlineto{\pgfqpoint{5.367095in}{0.847174in}}%
\pgfpathlineto{\pgfqpoint{5.355235in}{0.837329in}}%
\pgfpathlineto{\pgfqpoint{5.358706in}{0.832486in}}%
\pgfpathlineto{\pgfqpoint{5.347582in}{0.824310in}}%
\pgfpathlineto{\pgfqpoint{5.348006in}{0.816311in}}%
\pgfpathlineto{\pgfqpoint{5.339075in}{0.796632in}}%
\pgfpathlineto{\pgfqpoint{5.330704in}{0.790671in}}%
\pgfpathlineto{\pgfqpoint{5.329896in}{0.782473in}}%
\pgfpathlineto{\pgfqpoint{5.312519in}{0.751673in}}%
\pgfpathlineto{\pgfqpoint{5.318406in}{0.734520in}}%
\pgfpathlineto{\pgfqpoint{5.316769in}{0.729979in}}%
\pgfpathlineto{\pgfqpoint{5.320115in}{0.719999in}}%
\pgfpathlineto{\pgfqpoint{5.316520in}{0.710489in}}%
\pgfpathlineto{\pgfqpoint{5.269003in}{0.708528in}}%
\pgfpathlineto{\pgfqpoint{5.207317in}{0.707512in}}%
\pgfpathlineto{\pgfqpoint{5.164769in}{0.707128in}}%
\pgfpathlineto{\pgfqpoint{5.164562in}{0.740629in}}%
\pgfpathlineto{\pgfqpoint{5.153996in}{0.742824in}}%
\pgfpathlineto{\pgfqpoint{5.147023in}{0.739958in}}%
\pgfpathlineto{\pgfqpoint{5.141452in}{0.745221in}}%
\pgfpathlineto{\pgfqpoint{5.142027in}{0.780708in}}%
\pgfpathlineto{\pgfqpoint{5.143273in}{0.856267in}}%
\pgfpathlineto{\pgfqpoint{5.137230in}{0.900512in}}%
\pgfpathclose%
\pgfusepath{fill}%
\end{pgfscope}%
\begin{pgfscope}%
\pgfpathrectangle{\pgfqpoint{3.525000in}{0.100000in}}{\pgfqpoint{2.857344in}{1.829167in}}%
\pgfusepath{clip}%
\pgfsetbuttcap%
\pgfsetmiterjoin%
\definecolor{currentfill}{rgb}{0.654671,0.860438,0.643368}%
\pgfsetfillcolor{currentfill}%
\pgfsetlinewidth{0.000000pt}%
\definecolor{currentstroke}{rgb}{0.000000,0.000000,0.000000}%
\pgfsetstrokecolor{currentstroke}%
\pgfsetstrokeopacity{0.000000}%
\pgfsetdash{}{0pt}%
\pgfpathmoveto{\pgfqpoint{5.164769in}{0.707128in}}%
\pgfpathlineto{\pgfqpoint{5.207317in}{0.707512in}}%
\pgfpathlineto{\pgfqpoint{5.269003in}{0.708528in}}%
\pgfpathlineto{\pgfqpoint{5.316520in}{0.710489in}}%
\pgfpathlineto{\pgfqpoint{5.314243in}{0.705532in}}%
\pgfpathlineto{\pgfqpoint{5.320247in}{0.678228in}}%
\pgfpathlineto{\pgfqpoint{5.328542in}{0.666301in}}%
\pgfpathlineto{\pgfqpoint{5.318926in}{0.659713in}}%
\pgfpathlineto{\pgfqpoint{5.323952in}{0.648268in}}%
\pgfpathlineto{\pgfqpoint{5.310391in}{0.636483in}}%
\pgfpathlineto{\pgfqpoint{5.306337in}{0.618595in}}%
\pgfpathlineto{\pgfqpoint{5.298928in}{0.609381in}}%
\pgfpathlineto{\pgfqpoint{5.299871in}{0.599879in}}%
\pgfpathlineto{\pgfqpoint{5.295620in}{0.597864in}}%
\pgfpathlineto{\pgfqpoint{5.299886in}{0.587974in}}%
\pgfpathlineto{\pgfqpoint{5.296435in}{0.582738in}}%
\pgfpathlineto{\pgfqpoint{5.354448in}{0.585253in}}%
\pgfpathlineto{\pgfqpoint{5.399188in}{0.588008in}}%
\pgfpathlineto{\pgfqpoint{5.394466in}{0.566972in}}%
\pgfpathlineto{\pgfqpoint{5.397687in}{0.559201in}}%
\pgfpathlineto{\pgfqpoint{5.404365in}{0.552706in}}%
\pgfpathlineto{\pgfqpoint{5.410842in}{0.537211in}}%
\pgfpathlineto{\pgfqpoint{5.390382in}{0.540781in}}%
\pgfpathlineto{\pgfqpoint{5.382835in}{0.546650in}}%
\pgfpathlineto{\pgfqpoint{5.373842in}{0.546929in}}%
\pgfpathlineto{\pgfqpoint{5.364388in}{0.534077in}}%
\pgfpathlineto{\pgfqpoint{5.366272in}{0.528225in}}%
\pgfpathlineto{\pgfqpoint{5.396475in}{0.524685in}}%
\pgfpathlineto{\pgfqpoint{5.404381in}{0.517958in}}%
\pgfpathlineto{\pgfqpoint{5.417248in}{0.514497in}}%
\pgfpathlineto{\pgfqpoint{5.411629in}{0.506533in}}%
\pgfpathlineto{\pgfqpoint{5.402161in}{0.499587in}}%
\pgfpathlineto{\pgfqpoint{5.427151in}{0.483968in}}%
\pgfpathlineto{\pgfqpoint{5.438791in}{0.481527in}}%
\pgfpathlineto{\pgfqpoint{5.443631in}{0.468760in}}%
\pgfpathlineto{\pgfqpoint{5.432672in}{0.466412in}}%
\pgfpathlineto{\pgfqpoint{5.412073in}{0.482469in}}%
\pgfpathlineto{\pgfqpoint{5.404019in}{0.484701in}}%
\pgfpathlineto{\pgfqpoint{5.400122in}{0.491045in}}%
\pgfpathlineto{\pgfqpoint{5.388176in}{0.488441in}}%
\pgfpathlineto{\pgfqpoint{5.384436in}{0.480267in}}%
\pgfpathlineto{\pgfqpoint{5.386805in}{0.471159in}}%
\pgfpathlineto{\pgfqpoint{5.378762in}{0.465784in}}%
\pgfpathlineto{\pgfqpoint{5.371423in}{0.479026in}}%
\pgfpathlineto{\pgfqpoint{5.358567in}{0.475077in}}%
\pgfpathlineto{\pgfqpoint{5.353750in}{0.467177in}}%
\pgfpathlineto{\pgfqpoint{5.344620in}{0.469424in}}%
\pgfpathlineto{\pgfqpoint{5.338783in}{0.479401in}}%
\pgfpathlineto{\pgfqpoint{5.323297in}{0.482696in}}%
\pgfpathlineto{\pgfqpoint{5.320373in}{0.487888in}}%
\pgfpathlineto{\pgfqpoint{5.304238in}{0.496932in}}%
\pgfpathlineto{\pgfqpoint{5.300233in}{0.504839in}}%
\pgfpathlineto{\pgfqpoint{5.286719in}{0.501607in}}%
\pgfpathlineto{\pgfqpoint{5.288435in}{0.508847in}}%
\pgfpathlineto{\pgfqpoint{5.271638in}{0.501420in}}%
\pgfpathlineto{\pgfqpoint{5.276150in}{0.493715in}}%
\pgfpathlineto{\pgfqpoint{5.263178in}{0.489160in}}%
\pgfpathlineto{\pgfqpoint{5.245997in}{0.491664in}}%
\pgfpathlineto{\pgfqpoint{5.211222in}{0.503575in}}%
\pgfpathlineto{\pgfqpoint{5.184390in}{0.501211in}}%
\pgfpathlineto{\pgfqpoint{5.182064in}{0.516857in}}%
\pgfpathlineto{\pgfqpoint{5.185434in}{0.523996in}}%
\pgfpathlineto{\pgfqpoint{5.185145in}{0.535356in}}%
\pgfpathlineto{\pgfqpoint{5.181831in}{0.538062in}}%
\pgfpathlineto{\pgfqpoint{5.182954in}{0.551028in}}%
\pgfpathlineto{\pgfqpoint{5.192727in}{0.569062in}}%
\pgfpathlineto{\pgfqpoint{5.193977in}{0.575877in}}%
\pgfpathlineto{\pgfqpoint{5.192366in}{0.591971in}}%
\pgfpathlineto{\pgfqpoint{5.185177in}{0.599552in}}%
\pgfpathlineto{\pgfqpoint{5.181423in}{0.613707in}}%
\pgfpathlineto{\pgfqpoint{5.176804in}{0.616957in}}%
\pgfpathlineto{\pgfqpoint{5.178872in}{0.624633in}}%
\pgfpathlineto{\pgfqpoint{5.172989in}{0.636106in}}%
\pgfpathlineto{\pgfqpoint{5.165651in}{0.642332in}}%
\pgfpathclose%
\pgfusepath{fill}%
\end{pgfscope}%
\begin{pgfscope}%
\pgfpathrectangle{\pgfqpoint{3.525000in}{0.100000in}}{\pgfqpoint{2.857344in}{1.829167in}}%
\pgfusepath{clip}%
\pgfsetbuttcap%
\pgfsetmiterjoin%
\definecolor{currentfill}{rgb}{0.654671,0.860438,0.643368}%
\pgfsetfillcolor{currentfill}%
\pgfsetlinewidth{0.000000pt}%
\definecolor{currentstroke}{rgb}{0.000000,0.000000,0.000000}%
\pgfsetstrokecolor{currentstroke}%
\pgfsetstrokeopacity{0.000000}%
\pgfsetdash{}{0pt}%
\pgfpathmoveto{\pgfqpoint{5.278563in}{0.493125in}}%
\pgfpathlineto{\pgfqpoint{5.284719in}{0.496786in}}%
\pgfpathlineto{\pgfqpoint{5.292190in}{0.492458in}}%
\pgfpathlineto{\pgfqpoint{5.288023in}{0.486502in}}%
\pgfpathclose%
\pgfusepath{fill}%
\end{pgfscope}%
\begin{pgfscope}%
\pgfpathrectangle{\pgfqpoint{3.525000in}{0.100000in}}{\pgfqpoint{2.857344in}{1.829167in}}%
\pgfusepath{clip}%
\pgfsetbuttcap%
\pgfsetmiterjoin%
\definecolor{currentfill}{rgb}{0.516724,0.806459,0.645367}%
\pgfsetfillcolor{currentfill}%
\pgfsetlinewidth{0.000000pt}%
\definecolor{currentstroke}{rgb}{0.000000,0.000000,0.000000}%
\pgfsetstrokecolor{currentstroke}%
\pgfsetstrokeopacity{0.000000}%
\pgfsetdash{}{0pt}%
\pgfpathmoveto{\pgfqpoint{5.526967in}{0.563732in}}%
\pgfpathlineto{\pgfqpoint{5.527078in}{0.575568in}}%
\pgfpathlineto{\pgfqpoint{5.519723in}{0.580063in}}%
\pgfpathlineto{\pgfqpoint{5.513697in}{0.587716in}}%
\pgfpathlineto{\pgfqpoint{5.514518in}{0.595755in}}%
\pgfpathlineto{\pgfqpoint{5.580121in}{0.600848in}}%
\pgfpathlineto{\pgfqpoint{5.654545in}{0.609114in}}%
\pgfpathlineto{\pgfqpoint{5.664037in}{0.591814in}}%
\pgfpathlineto{\pgfqpoint{5.695600in}{0.594036in}}%
\pgfpathlineto{\pgfqpoint{5.758184in}{0.597451in}}%
\pgfpathlineto{\pgfqpoint{5.807818in}{0.600758in}}%
\pgfpathlineto{\pgfqpoint{5.812692in}{0.588298in}}%
\pgfpathlineto{\pgfqpoint{5.818718in}{0.589319in}}%
\pgfpathlineto{\pgfqpoint{5.819389in}{0.602660in}}%
\pgfpathlineto{\pgfqpoint{5.816305in}{0.614136in}}%
\pgfpathlineto{\pgfqpoint{5.820438in}{0.618993in}}%
\pgfpathlineto{\pgfqpoint{5.829989in}{0.616380in}}%
\pgfpathlineto{\pgfqpoint{5.843699in}{0.615843in}}%
\pgfpathlineto{\pgfqpoint{5.845807in}{0.605591in}}%
\pgfpathlineto{\pgfqpoint{5.850025in}{0.599733in}}%
\pgfpathlineto{\pgfqpoint{5.853318in}{0.586916in}}%
\pgfpathlineto{\pgfqpoint{5.863523in}{0.567081in}}%
\pgfpathlineto{\pgfqpoint{5.863574in}{0.561716in}}%
\pgfpathlineto{\pgfqpoint{5.878655in}{0.538426in}}%
\pgfpathlineto{\pgfqpoint{5.880114in}{0.533615in}}%
\pgfpathlineto{\pgfqpoint{5.902740in}{0.500721in}}%
\pgfpathlineto{\pgfqpoint{5.899155in}{0.500189in}}%
\pgfpathlineto{\pgfqpoint{5.908693in}{0.476793in}}%
\pgfpathlineto{\pgfqpoint{5.934896in}{0.436119in}}%
\pgfpathlineto{\pgfqpoint{5.950109in}{0.406213in}}%
\pgfpathlineto{\pgfqpoint{5.955194in}{0.400977in}}%
\pgfpathlineto{\pgfqpoint{5.963907in}{0.382206in}}%
\pgfpathlineto{\pgfqpoint{5.966929in}{0.352148in}}%
\pgfpathlineto{\pgfqpoint{5.968132in}{0.329674in}}%
\pgfpathlineto{\pgfqpoint{5.966679in}{0.315283in}}%
\pgfpathlineto{\pgfqpoint{5.962008in}{0.304996in}}%
\pgfpathlineto{\pgfqpoint{5.964178in}{0.291507in}}%
\pgfpathlineto{\pgfqpoint{5.959148in}{0.280829in}}%
\pgfpathlineto{\pgfqpoint{5.944215in}{0.272073in}}%
\pgfpathlineto{\pgfqpoint{5.934502in}{0.272712in}}%
\pgfpathlineto{\pgfqpoint{5.928195in}{0.268135in}}%
\pgfpathlineto{\pgfqpoint{5.926388in}{0.280356in}}%
\pgfpathlineto{\pgfqpoint{5.915948in}{0.283582in}}%
\pgfpathlineto{\pgfqpoint{5.906588in}{0.300634in}}%
\pgfpathlineto{\pgfqpoint{5.905532in}{0.308399in}}%
\pgfpathlineto{\pgfqpoint{5.888782in}{0.313114in}}%
\pgfpathlineto{\pgfqpoint{5.877988in}{0.312093in}}%
\pgfpathlineto{\pgfqpoint{5.871881in}{0.323360in}}%
\pgfpathlineto{\pgfqpoint{5.864889in}{0.343612in}}%
\pgfpathlineto{\pgfqpoint{5.855185in}{0.347717in}}%
\pgfpathlineto{\pgfqpoint{5.849925in}{0.359322in}}%
\pgfpathlineto{\pgfqpoint{5.837060in}{0.366161in}}%
\pgfpathlineto{\pgfqpoint{5.829624in}{0.374717in}}%
\pgfpathlineto{\pgfqpoint{5.816796in}{0.397975in}}%
\pgfpathlineto{\pgfqpoint{5.815287in}{0.414346in}}%
\pgfpathlineto{\pgfqpoint{5.822278in}{0.424820in}}%
\pgfpathlineto{\pgfqpoint{5.821579in}{0.432098in}}%
\pgfpathlineto{\pgfqpoint{5.813501in}{0.432876in}}%
\pgfpathlineto{\pgfqpoint{5.806752in}{0.437938in}}%
\pgfpathlineto{\pgfqpoint{5.803056in}{0.431757in}}%
\pgfpathlineto{\pgfqpoint{5.809533in}{0.426720in}}%
\pgfpathlineto{\pgfqpoint{5.804819in}{0.417662in}}%
\pgfpathlineto{\pgfqpoint{5.797192in}{0.425180in}}%
\pgfpathlineto{\pgfqpoint{5.798076in}{0.446027in}}%
\pgfpathlineto{\pgfqpoint{5.801755in}{0.462940in}}%
\pgfpathlineto{\pgfqpoint{5.799864in}{0.491962in}}%
\pgfpathlineto{\pgfqpoint{5.792262in}{0.498874in}}%
\pgfpathlineto{\pgfqpoint{5.788439in}{0.507744in}}%
\pgfpathlineto{\pgfqpoint{5.775342in}{0.507547in}}%
\pgfpathlineto{\pgfqpoint{5.762397in}{0.522142in}}%
\pgfpathlineto{\pgfqpoint{5.753717in}{0.526517in}}%
\pgfpathlineto{\pgfqpoint{5.751143in}{0.535760in}}%
\pgfpathlineto{\pgfqpoint{5.742653in}{0.539006in}}%
\pgfpathlineto{\pgfqpoint{5.735617in}{0.549211in}}%
\pgfpathlineto{\pgfqpoint{5.717030in}{0.557541in}}%
\pgfpathlineto{\pgfqpoint{5.702580in}{0.557754in}}%
\pgfpathlineto{\pgfqpoint{5.696290in}{0.554555in}}%
\pgfpathlineto{\pgfqpoint{5.697853in}{0.544567in}}%
\pgfpathlineto{\pgfqpoint{5.691294in}{0.545040in}}%
\pgfpathlineto{\pgfqpoint{5.671116in}{0.531076in}}%
\pgfpathlineto{\pgfqpoint{5.646879in}{0.525541in}}%
\pgfpathlineto{\pgfqpoint{5.646481in}{0.532389in}}%
\pgfpathlineto{\pgfqpoint{5.641110in}{0.539082in}}%
\pgfpathlineto{\pgfqpoint{5.626698in}{0.548323in}}%
\pgfpathlineto{\pgfqpoint{5.606013in}{0.557796in}}%
\pgfpathlineto{\pgfqpoint{5.583544in}{0.562811in}}%
\pgfpathlineto{\pgfqpoint{5.579327in}{0.569631in}}%
\pgfpathlineto{\pgfqpoint{5.571225in}{0.563943in}}%
\pgfpathlineto{\pgfqpoint{5.561470in}{0.562689in}}%
\pgfpathlineto{\pgfqpoint{5.527513in}{0.553742in}}%
\pgfpathclose%
\pgfusepath{fill}%
\end{pgfscope}%
\begin{pgfscope}%
\pgfpathrectangle{\pgfqpoint{3.525000in}{0.100000in}}{\pgfqpoint{2.857344in}{1.829167in}}%
\pgfusepath{clip}%
\pgfsetbuttcap%
\pgfsetmiterjoin%
\definecolor{currentfill}{rgb}{0.516724,0.806459,0.645367}%
\pgfsetfillcolor{currentfill}%
\pgfsetlinewidth{0.000000pt}%
\definecolor{currentstroke}{rgb}{0.000000,0.000000,0.000000}%
\pgfsetstrokecolor{currentstroke}%
\pgfsetstrokeopacity{0.000000}%
\pgfsetdash{}{0pt}%
\pgfpathmoveto{\pgfqpoint{5.906036in}{0.501188in}}%
\pgfpathlineto{\pgfqpoint{5.914822in}{0.490968in}}%
\pgfpathlineto{\pgfqpoint{5.904674in}{0.490325in}}%
\pgfpathclose%
\pgfusepath{fill}%
\end{pgfscope}%
\begin{pgfscope}%
\pgfpathrectangle{\pgfqpoint{3.525000in}{0.100000in}}{\pgfqpoint{2.857344in}{1.829167in}}%
\pgfusepath{clip}%
\pgfsetbuttcap%
\pgfsetmiterjoin%
\definecolor{currentfill}{rgb}{0.738639,0.894348,0.629296}%
\pgfsetfillcolor{currentfill}%
\pgfsetlinewidth{0.000000pt}%
\definecolor{currentstroke}{rgb}{0.000000,0.000000,0.000000}%
\pgfsetstrokecolor{currentstroke}%
\pgfsetstrokeopacity{0.000000}%
\pgfsetdash{}{0pt}%
\pgfpathmoveto{\pgfqpoint{5.388842in}{1.668094in}}%
\pgfpathlineto{\pgfqpoint{5.384076in}{1.658816in}}%
\pgfpathlineto{\pgfqpoint{5.372704in}{1.653401in}}%
\pgfpathlineto{\pgfqpoint{5.367778in}{1.646113in}}%
\pgfpathlineto{\pgfqpoint{5.362009in}{1.651363in}}%
\pgfpathclose%
\pgfusepath{fill}%
\end{pgfscope}%
\begin{pgfscope}%
\pgfpathrectangle{\pgfqpoint{3.525000in}{0.100000in}}{\pgfqpoint{2.857344in}{1.829167in}}%
\pgfusepath{clip}%
\pgfsetbuttcap%
\pgfsetmiterjoin%
\definecolor{currentfill}{rgb}{0.738639,0.894348,0.629296}%
\pgfsetfillcolor{currentfill}%
\pgfsetlinewidth{0.000000pt}%
\definecolor{currentstroke}{rgb}{0.000000,0.000000,0.000000}%
\pgfsetstrokecolor{currentstroke}%
\pgfsetstrokeopacity{0.000000}%
\pgfsetdash{}{0pt}%
\pgfpathmoveto{\pgfqpoint{5.392709in}{1.612259in}}%
\pgfpathlineto{\pgfqpoint{5.404344in}{1.623114in}}%
\pgfpathlineto{\pgfqpoint{5.422291in}{1.625956in}}%
\pgfpathlineto{\pgfqpoint{5.417348in}{1.618407in}}%
\pgfpathlineto{\pgfqpoint{5.405044in}{1.607490in}}%
\pgfpathlineto{\pgfqpoint{5.397851in}{1.593477in}}%
\pgfpathlineto{\pgfqpoint{5.394805in}{1.601077in}}%
\pgfpathlineto{\pgfqpoint{5.389396in}{1.602169in}}%
\pgfpathlineto{\pgfqpoint{5.388852in}{1.609030in}}%
\pgfpathclose%
\pgfusepath{fill}%
\end{pgfscope}%
\begin{pgfscope}%
\pgfpathrectangle{\pgfqpoint{3.525000in}{0.100000in}}{\pgfqpoint{2.857344in}{1.829167in}}%
\pgfusepath{clip}%
\pgfsetbuttcap%
\pgfsetmiterjoin%
\definecolor{currentfill}{rgb}{0.738639,0.894348,0.629296}%
\pgfsetfillcolor{currentfill}%
\pgfsetlinewidth{0.000000pt}%
\definecolor{currentstroke}{rgb}{0.000000,0.000000,0.000000}%
\pgfsetstrokecolor{currentstroke}%
\pgfsetstrokeopacity{0.000000}%
\pgfsetdash{}{0pt}%
\pgfpathmoveto{\pgfqpoint{5.439053in}{1.479771in}}%
\pgfpathlineto{\pgfqpoint{5.435954in}{1.483210in}}%
\pgfpathlineto{\pgfqpoint{5.439212in}{1.492891in}}%
\pgfpathlineto{\pgfqpoint{5.429541in}{1.493508in}}%
\pgfpathlineto{\pgfqpoint{5.432105in}{1.501855in}}%
\pgfpathlineto{\pgfqpoint{5.430511in}{1.515149in}}%
\pgfpathlineto{\pgfqpoint{5.421834in}{1.519758in}}%
\pgfpathlineto{\pgfqpoint{5.412753in}{1.529086in}}%
\pgfpathlineto{\pgfqpoint{5.398761in}{1.531816in}}%
\pgfpathlineto{\pgfqpoint{5.385144in}{1.531739in}}%
\pgfpathlineto{\pgfqpoint{5.371751in}{1.538358in}}%
\pgfpathlineto{\pgfqpoint{5.326945in}{1.548007in}}%
\pgfpathlineto{\pgfqpoint{5.322048in}{1.558231in}}%
\pgfpathlineto{\pgfqpoint{5.313314in}{1.561720in}}%
\pgfpathlineto{\pgfqpoint{5.329812in}{1.569548in}}%
\pgfpathlineto{\pgfqpoint{5.339132in}{1.579314in}}%
\pgfpathlineto{\pgfqpoint{5.356491in}{1.581963in}}%
\pgfpathlineto{\pgfqpoint{5.367173in}{1.591910in}}%
\pgfpathlineto{\pgfqpoint{5.372777in}{1.592308in}}%
\pgfpathlineto{\pgfqpoint{5.377054in}{1.599402in}}%
\pgfpathlineto{\pgfqpoint{5.387480in}{1.607806in}}%
\pgfpathlineto{\pgfqpoint{5.388389in}{1.601866in}}%
\pgfpathlineto{\pgfqpoint{5.393087in}{1.600639in}}%
\pgfpathlineto{\pgfqpoint{5.396621in}{1.593555in}}%
\pgfpathlineto{\pgfqpoint{5.397258in}{1.581478in}}%
\pgfpathlineto{\pgfqpoint{5.407882in}{1.588695in}}%
\pgfpathlineto{\pgfqpoint{5.420261in}{1.590203in}}%
\pgfpathlineto{\pgfqpoint{5.430828in}{1.586422in}}%
\pgfpathlineto{\pgfqpoint{5.445158in}{1.566762in}}%
\pgfpathlineto{\pgfqpoint{5.460809in}{1.569904in}}%
\pgfpathlineto{\pgfqpoint{5.467169in}{1.564643in}}%
\pgfpathlineto{\pgfqpoint{5.477398in}{1.564173in}}%
\pgfpathlineto{\pgfqpoint{5.484193in}{1.573609in}}%
\pgfpathlineto{\pgfqpoint{5.497088in}{1.581950in}}%
\pgfpathlineto{\pgfqpoint{5.509482in}{1.584563in}}%
\pgfpathlineto{\pgfqpoint{5.524860in}{1.584835in}}%
\pgfpathlineto{\pgfqpoint{5.536098in}{1.591278in}}%
\pgfpathlineto{\pgfqpoint{5.545270in}{1.588310in}}%
\pgfpathlineto{\pgfqpoint{5.547223in}{1.574683in}}%
\pgfpathlineto{\pgfqpoint{5.566899in}{1.572549in}}%
\pgfpathlineto{\pgfqpoint{5.577586in}{1.578928in}}%
\pgfpathlineto{\pgfqpoint{5.584994in}{1.564557in}}%
\pgfpathlineto{\pgfqpoint{5.599130in}{1.547940in}}%
\pgfpathlineto{\pgfqpoint{5.579345in}{1.548036in}}%
\pgfpathlineto{\pgfqpoint{5.573094in}{1.545982in}}%
\pgfpathlineto{\pgfqpoint{5.564525in}{1.548652in}}%
\pgfpathlineto{\pgfqpoint{5.563948in}{1.537148in}}%
\pgfpathlineto{\pgfqpoint{5.548390in}{1.546138in}}%
\pgfpathlineto{\pgfqpoint{5.528380in}{1.548885in}}%
\pgfpathlineto{\pgfqpoint{5.522872in}{1.540130in}}%
\pgfpathlineto{\pgfqpoint{5.496692in}{1.535873in}}%
\pgfpathlineto{\pgfqpoint{5.493680in}{1.528456in}}%
\pgfpathlineto{\pgfqpoint{5.475462in}{1.526230in}}%
\pgfpathlineto{\pgfqpoint{5.469954in}{1.518691in}}%
\pgfpathlineto{\pgfqpoint{5.460318in}{1.516669in}}%
\pgfpathlineto{\pgfqpoint{5.452621in}{1.498791in}}%
\pgfpathlineto{\pgfqpoint{5.442876in}{1.481476in}}%
\pgfpathclose%
\pgfusepath{fill}%
\end{pgfscope}%
\begin{pgfscope}%
\pgfpathrectangle{\pgfqpoint{3.525000in}{0.100000in}}{\pgfqpoint{2.857344in}{1.829167in}}%
\pgfusepath{clip}%
\pgfsetbuttcap%
\pgfsetmiterjoin%
\definecolor{currentfill}{rgb}{0.738639,0.894348,0.629296}%
\pgfsetfillcolor{currentfill}%
\pgfsetlinewidth{0.000000pt}%
\definecolor{currentstroke}{rgb}{0.000000,0.000000,0.000000}%
\pgfsetstrokecolor{currentstroke}%
\pgfsetstrokeopacity{0.000000}%
\pgfsetdash{}{0pt}%
\pgfpathmoveto{\pgfqpoint{5.495140in}{1.271890in}}%
\pgfpathlineto{\pgfqpoint{5.504431in}{1.281669in}}%
\pgfpathlineto{\pgfqpoint{5.508664in}{1.295848in}}%
\pgfpathlineto{\pgfqpoint{5.513697in}{1.304066in}}%
\pgfpathlineto{\pgfqpoint{5.516784in}{1.315250in}}%
\pgfpathlineto{\pgfqpoint{5.517744in}{1.337550in}}%
\pgfpathlineto{\pgfqpoint{5.513096in}{1.358921in}}%
\pgfpathlineto{\pgfqpoint{5.497738in}{1.391647in}}%
\pgfpathlineto{\pgfqpoint{5.501897in}{1.401937in}}%
\pgfpathlineto{\pgfqpoint{5.496479in}{1.416159in}}%
\pgfpathlineto{\pgfqpoint{5.505773in}{1.435829in}}%
\pgfpathlineto{\pgfqpoint{5.504259in}{1.457761in}}%
\pgfpathlineto{\pgfqpoint{5.510733in}{1.460522in}}%
\pgfpathlineto{\pgfqpoint{5.511546in}{1.470957in}}%
\pgfpathlineto{\pgfqpoint{5.523037in}{1.477655in}}%
\pgfpathlineto{\pgfqpoint{5.529536in}{1.476574in}}%
\pgfpathlineto{\pgfqpoint{5.531363in}{1.465382in}}%
\pgfpathlineto{\pgfqpoint{5.536436in}{1.464943in}}%
\pgfpathlineto{\pgfqpoint{5.541064in}{1.481086in}}%
\pgfpathlineto{\pgfqpoint{5.539491in}{1.493642in}}%
\pgfpathlineto{\pgfqpoint{5.542500in}{1.500851in}}%
\pgfpathlineto{\pgfqpoint{5.558770in}{1.508299in}}%
\pgfpathlineto{\pgfqpoint{5.551356in}{1.510950in}}%
\pgfpathlineto{\pgfqpoint{5.548941in}{1.517348in}}%
\pgfpathlineto{\pgfqpoint{5.554282in}{1.528584in}}%
\pgfpathlineto{\pgfqpoint{5.564814in}{1.532464in}}%
\pgfpathlineto{\pgfqpoint{5.577044in}{1.525835in}}%
\pgfpathlineto{\pgfqpoint{5.588575in}{1.525752in}}%
\pgfpathlineto{\pgfqpoint{5.593930in}{1.518009in}}%
\pgfpathlineto{\pgfqpoint{5.602005in}{1.518547in}}%
\pgfpathlineto{\pgfqpoint{5.626934in}{1.507782in}}%
\pgfpathlineto{\pgfqpoint{5.631915in}{1.497361in}}%
\pgfpathlineto{\pgfqpoint{5.626475in}{1.493752in}}%
\pgfpathlineto{\pgfqpoint{5.628151in}{1.485943in}}%
\pgfpathlineto{\pgfqpoint{5.633525in}{1.482469in}}%
\pgfpathlineto{\pgfqpoint{5.636520in}{1.472869in}}%
\pgfpathlineto{\pgfqpoint{5.636100in}{1.449480in}}%
\pgfpathlineto{\pgfqpoint{5.629006in}{1.443892in}}%
\pgfpathlineto{\pgfqpoint{5.627413in}{1.431607in}}%
\pgfpathlineto{\pgfqpoint{5.614252in}{1.420253in}}%
\pgfpathlineto{\pgfqpoint{5.615033in}{1.406510in}}%
\pgfpathlineto{\pgfqpoint{5.626568in}{1.401695in}}%
\pgfpathlineto{\pgfqpoint{5.639581in}{1.418842in}}%
\pgfpathlineto{\pgfqpoint{5.640657in}{1.425062in}}%
\pgfpathlineto{\pgfqpoint{5.656886in}{1.435399in}}%
\pgfpathlineto{\pgfqpoint{5.667206in}{1.430611in}}%
\pgfpathlineto{\pgfqpoint{5.673683in}{1.419789in}}%
\pgfpathlineto{\pgfqpoint{5.684130in}{1.382218in}}%
\pgfpathlineto{\pgfqpoint{5.689667in}{1.370331in}}%
\pgfpathlineto{\pgfqpoint{5.687933in}{1.365417in}}%
\pgfpathlineto{\pgfqpoint{5.688102in}{1.348685in}}%
\pgfpathlineto{\pgfqpoint{5.678030in}{1.350291in}}%
\pgfpathlineto{\pgfqpoint{5.672342in}{1.337787in}}%
\pgfpathlineto{\pgfqpoint{5.671554in}{1.329287in}}%
\pgfpathlineto{\pgfqpoint{5.663946in}{1.323830in}}%
\pgfpathlineto{\pgfqpoint{5.662256in}{1.307253in}}%
\pgfpathlineto{\pgfqpoint{5.651192in}{1.286311in}}%
\pgfpathlineto{\pgfqpoint{5.590674in}{1.277298in}}%
\pgfpathlineto{\pgfqpoint{5.590316in}{1.281260in}}%
\pgfpathlineto{\pgfqpoint{5.549833in}{1.276993in}}%
\pgfpathclose%
\pgfusepath{fill}%
\end{pgfscope}%
\begin{pgfscope}%
\pgfpathrectangle{\pgfqpoint{3.525000in}{0.100000in}}{\pgfqpoint{2.857344in}{1.829167in}}%
\pgfusepath{clip}%
\pgfsetbuttcap%
\pgfsetmiterjoin%
\definecolor{currentfill}{rgb}{0.811226,0.923875,0.614533}%
\pgfsetfillcolor{currentfill}%
\pgfsetlinewidth{0.000000pt}%
\definecolor{currentstroke}{rgb}{0.000000,0.000000,0.000000}%
\pgfsetstrokecolor{currentstroke}%
\pgfsetstrokeopacity{0.000000}%
\pgfsetdash{}{0pt}%
\pgfpathmoveto{\pgfqpoint{3.966676in}{0.431164in}}%
\pgfpathlineto{\pgfqpoint{3.965468in}{0.430101in}}%
\pgfpathlineto{\pgfqpoint{3.959923in}{0.431518in}}%
\pgfpathlineto{\pgfqpoint{3.960337in}{0.434309in}}%
\pgfpathlineto{\pgfqpoint{3.963475in}{0.435587in}}%
\pgfpathlineto{\pgfqpoint{3.967950in}{0.442019in}}%
\pgfpathlineto{\pgfqpoint{3.968693in}{0.446367in}}%
\pgfpathlineto{\pgfqpoint{3.971209in}{0.447506in}}%
\pgfpathlineto{\pgfqpoint{3.975301in}{0.447511in}}%
\pgfpathlineto{\pgfqpoint{3.979580in}{0.461619in}}%
\pgfpathlineto{\pgfqpoint{3.981256in}{0.462780in}}%
\pgfpathlineto{\pgfqpoint{3.979753in}{0.465538in}}%
\pgfpathlineto{\pgfqpoint{3.976234in}{0.462748in}}%
\pgfpathlineto{\pgfqpoint{3.965896in}{0.466509in}}%
\pgfpathlineto{\pgfqpoint{3.959725in}{0.472846in}}%
\pgfpathlineto{\pgfqpoint{3.962416in}{0.475737in}}%
\pgfpathlineto{\pgfqpoint{3.961192in}{0.479815in}}%
\pgfpathlineto{\pgfqpoint{3.961946in}{0.484632in}}%
\pgfpathlineto{\pgfqpoint{3.960619in}{0.490264in}}%
\pgfpathlineto{\pgfqpoint{3.960407in}{0.495903in}}%
\pgfpathlineto{\pgfqpoint{3.966346in}{0.494959in}}%
\pgfpathlineto{\pgfqpoint{3.967522in}{0.496621in}}%
\pgfpathlineto{\pgfqpoint{3.970476in}{0.496460in}}%
\pgfpathlineto{\pgfqpoint{3.973878in}{0.490446in}}%
\pgfpathlineto{\pgfqpoint{3.977158in}{0.486044in}}%
\pgfpathlineto{\pgfqpoint{3.974531in}{0.483178in}}%
\pgfpathlineto{\pgfqpoint{3.979074in}{0.482079in}}%
\pgfpathlineto{\pgfqpoint{3.980109in}{0.483484in}}%
\pgfpathlineto{\pgfqpoint{3.977679in}{0.489187in}}%
\pgfpathlineto{\pgfqpoint{3.972564in}{0.493914in}}%
\pgfpathlineto{\pgfqpoint{3.973189in}{0.495895in}}%
\pgfpathlineto{\pgfqpoint{3.969107in}{0.500900in}}%
\pgfpathlineto{\pgfqpoint{3.973276in}{0.501916in}}%
\pgfpathlineto{\pgfqpoint{3.969664in}{0.512633in}}%
\pgfpathlineto{\pgfqpoint{3.972601in}{0.514119in}}%
\pgfpathlineto{\pgfqpoint{3.973317in}{0.516688in}}%
\pgfpathlineto{\pgfqpoint{3.971406in}{0.519741in}}%
\pgfpathlineto{\pgfqpoint{3.975349in}{0.520505in}}%
\pgfpathlineto{\pgfqpoint{3.976057in}{0.522813in}}%
\pgfpathlineto{\pgfqpoint{3.980760in}{0.519348in}}%
\pgfpathlineto{\pgfqpoint{3.981805in}{0.523105in}}%
\pgfpathlineto{\pgfqpoint{3.984277in}{0.524759in}}%
\pgfpathlineto{\pgfqpoint{3.996303in}{0.527058in}}%
\pgfpathlineto{\pgfqpoint{3.999994in}{0.526245in}}%
\pgfpathlineto{\pgfqpoint{4.003087in}{0.530808in}}%
\pgfpathlineto{\pgfqpoint{4.010425in}{0.533802in}}%
\pgfpathlineto{\pgfqpoint{4.016078in}{0.532667in}}%
\pgfpathlineto{\pgfqpoint{4.018657in}{0.528692in}}%
\pgfpathlineto{\pgfqpoint{4.018521in}{0.522378in}}%
\pgfpathlineto{\pgfqpoint{4.021414in}{0.520947in}}%
\pgfpathlineto{\pgfqpoint{4.027314in}{0.521147in}}%
\pgfpathlineto{\pgfqpoint{4.034872in}{0.522789in}}%
\pgfpathlineto{\pgfqpoint{4.034779in}{0.520770in}}%
\pgfpathlineto{\pgfqpoint{4.044332in}{0.515013in}}%
\pgfpathlineto{\pgfqpoint{4.051013in}{0.516556in}}%
\pgfpathlineto{\pgfqpoint{4.052980in}{0.518555in}}%
\pgfpathlineto{\pgfqpoint{4.055086in}{0.523672in}}%
\pgfpathlineto{\pgfqpoint{4.057698in}{0.526949in}}%
\pgfpathlineto{\pgfqpoint{4.057932in}{0.531859in}}%
\pgfpathlineto{\pgfqpoint{4.055540in}{0.533791in}}%
\pgfpathlineto{\pgfqpoint{4.059081in}{0.535536in}}%
\pgfpathlineto{\pgfqpoint{4.065032in}{0.532914in}}%
\pgfpathlineto{\pgfqpoint{4.066970in}{0.534533in}}%
\pgfpathlineto{\pgfqpoint{4.067312in}{0.541480in}}%
\pgfpathlineto{\pgfqpoint{4.063465in}{0.540028in}}%
\pgfpathlineto{\pgfqpoint{4.060798in}{0.542435in}}%
\pgfpathlineto{\pgfqpoint{4.056467in}{0.543936in}}%
\pgfpathlineto{\pgfqpoint{4.049593in}{0.543686in}}%
\pgfpathlineto{\pgfqpoint{4.043007in}{0.545968in}}%
\pgfpathlineto{\pgfqpoint{4.042677in}{0.551654in}}%
\pgfpathlineto{\pgfqpoint{4.036787in}{0.557694in}}%
\pgfpathlineto{\pgfqpoint{4.029317in}{0.560229in}}%
\pgfpathlineto{\pgfqpoint{4.022797in}{0.571922in}}%
\pgfpathlineto{\pgfqpoint{4.023332in}{0.576248in}}%
\pgfpathlineto{\pgfqpoint{4.026843in}{0.578221in}}%
\pgfpathlineto{\pgfqpoint{4.026326in}{0.582317in}}%
\pgfpathlineto{\pgfqpoint{4.027141in}{0.586392in}}%
\pgfpathlineto{\pgfqpoint{4.029919in}{0.583603in}}%
\pgfpathlineto{\pgfqpoint{4.033586in}{0.584766in}}%
\pgfpathlineto{\pgfqpoint{4.033403in}{0.588306in}}%
\pgfpathlineto{\pgfqpoint{4.028318in}{0.595412in}}%
\pgfpathlineto{\pgfqpoint{4.026829in}{0.602741in}}%
\pgfpathlineto{\pgfqpoint{4.031085in}{0.603300in}}%
\pgfpathlineto{\pgfqpoint{4.037456in}{0.602850in}}%
\pgfpathlineto{\pgfqpoint{4.039463in}{0.600399in}}%
\pgfpathlineto{\pgfqpoint{4.045174in}{0.601276in}}%
\pgfpathlineto{\pgfqpoint{4.050482in}{0.600124in}}%
\pgfpathlineto{\pgfqpoint{4.054071in}{0.598099in}}%
\pgfpathlineto{\pgfqpoint{4.056554in}{0.599025in}}%
\pgfpathlineto{\pgfqpoint{4.063150in}{0.597001in}}%
\pgfpathlineto{\pgfqpoint{4.063210in}{0.595210in}}%
\pgfpathlineto{\pgfqpoint{4.068232in}{0.595815in}}%
\pgfpathlineto{\pgfqpoint{4.074967in}{0.591444in}}%
\pgfpathlineto{\pgfqpoint{4.074706in}{0.588542in}}%
\pgfpathlineto{\pgfqpoint{4.071501in}{0.587202in}}%
\pgfpathlineto{\pgfqpoint{4.068095in}{0.584016in}}%
\pgfpathlineto{\pgfqpoint{4.067684in}{0.579641in}}%
\pgfpathlineto{\pgfqpoint{4.075020in}{0.574061in}}%
\pgfpathlineto{\pgfqpoint{4.073847in}{0.571387in}}%
\pgfpathlineto{\pgfqpoint{4.079067in}{0.569334in}}%
\pgfpathlineto{\pgfqpoint{4.080349in}{0.565980in}}%
\pgfpathlineto{\pgfqpoint{4.086974in}{0.568645in}}%
\pgfpathlineto{\pgfqpoint{4.089892in}{0.565152in}}%
\pgfpathlineto{\pgfqpoint{4.091238in}{0.567266in}}%
\pgfpathlineto{\pgfqpoint{4.087273in}{0.574147in}}%
\pgfpathlineto{\pgfqpoint{4.088682in}{0.576208in}}%
\pgfpathlineto{\pgfqpoint{4.089307in}{0.581643in}}%
\pgfpathlineto{\pgfqpoint{4.087727in}{0.584035in}}%
\pgfpathlineto{\pgfqpoint{4.088912in}{0.587249in}}%
\pgfpathlineto{\pgfqpoint{4.092794in}{0.586947in}}%
\pgfpathlineto{\pgfqpoint{4.091991in}{0.581479in}}%
\pgfpathlineto{\pgfqpoint{4.089512in}{0.579655in}}%
\pgfpathlineto{\pgfqpoint{4.089688in}{0.572299in}}%
\pgfpathlineto{\pgfqpoint{4.094135in}{0.571420in}}%
\pgfpathlineto{\pgfqpoint{4.094189in}{0.565951in}}%
\pgfpathlineto{\pgfqpoint{4.099045in}{0.562402in}}%
\pgfpathlineto{\pgfqpoint{4.101278in}{0.564640in}}%
\pgfpathlineto{\pgfqpoint{4.098838in}{0.571449in}}%
\pgfpathlineto{\pgfqpoint{4.096314in}{0.573172in}}%
\pgfpathlineto{\pgfqpoint{4.093832in}{0.572039in}}%
\pgfpathlineto{\pgfqpoint{4.091648in}{0.573485in}}%
\pgfpathlineto{\pgfqpoint{4.092029in}{0.579778in}}%
\pgfpathlineto{\pgfqpoint{4.095674in}{0.582351in}}%
\pgfpathlineto{\pgfqpoint{4.099261in}{0.582114in}}%
\pgfpathlineto{\pgfqpoint{4.097775in}{0.585675in}}%
\pgfpathlineto{\pgfqpoint{4.092398in}{0.588402in}}%
\pgfpathlineto{\pgfqpoint{4.085277in}{0.599268in}}%
\pgfpathlineto{\pgfqpoint{4.088831in}{0.604479in}}%
\pgfpathlineto{\pgfqpoint{4.090809in}{0.609387in}}%
\pgfpathlineto{\pgfqpoint{4.090965in}{0.619244in}}%
\pgfpathlineto{\pgfqpoint{4.090203in}{0.627559in}}%
\pgfpathlineto{\pgfqpoint{4.088030in}{0.633019in}}%
\pgfpathlineto{\pgfqpoint{4.088463in}{0.638478in}}%
\pgfpathlineto{\pgfqpoint{4.094021in}{0.643213in}}%
\pgfpathlineto{\pgfqpoint{4.099547in}{0.648839in}}%
\pgfpathlineto{\pgfqpoint{4.113402in}{0.637344in}}%
\pgfpathlineto{\pgfqpoint{4.122461in}{0.636496in}}%
\pgfpathlineto{\pgfqpoint{4.129937in}{0.639359in}}%
\pgfpathlineto{\pgfqpoint{4.136501in}{0.643128in}}%
\pgfpathlineto{\pgfqpoint{4.137896in}{0.644924in}}%
\pgfpathlineto{\pgfqpoint{4.154098in}{0.649020in}}%
\pgfpathlineto{\pgfqpoint{4.155505in}{0.646020in}}%
\pgfpathlineto{\pgfqpoint{4.161010in}{0.643565in}}%
\pgfpathlineto{\pgfqpoint{4.171517in}{0.644318in}}%
\pgfpathlineto{\pgfqpoint{4.173436in}{0.643647in}}%
\pgfpathlineto{\pgfqpoint{4.178159in}{0.645439in}}%
\pgfpathlineto{\pgfqpoint{4.180057in}{0.642341in}}%
\pgfpathlineto{\pgfqpoint{4.184592in}{0.639510in}}%
\pgfpathlineto{\pgfqpoint{4.186860in}{0.639830in}}%
\pgfpathlineto{\pgfqpoint{4.191251in}{0.636429in}}%
\pgfpathlineto{\pgfqpoint{4.198670in}{0.637054in}}%
\pgfpathlineto{\pgfqpoint{4.206081in}{0.639808in}}%
\pgfpathlineto{\pgfqpoint{4.212133in}{0.631003in}}%
\pgfpathlineto{\pgfqpoint{4.211663in}{0.629027in}}%
\pgfpathlineto{\pgfqpoint{4.204985in}{0.626756in}}%
\pgfpathlineto{\pgfqpoint{4.206770in}{0.623322in}}%
\pgfpathlineto{\pgfqpoint{4.212917in}{0.626936in}}%
\pgfpathlineto{\pgfqpoint{4.215427in}{0.626061in}}%
\pgfpathlineto{\pgfqpoint{4.216495in}{0.621738in}}%
\pgfpathlineto{\pgfqpoint{4.212825in}{0.619910in}}%
\pgfpathlineto{\pgfqpoint{4.215446in}{0.614445in}}%
\pgfpathlineto{\pgfqpoint{4.219252in}{0.615369in}}%
\pgfpathlineto{\pgfqpoint{4.224776in}{0.612548in}}%
\pgfpathlineto{\pgfqpoint{4.230058in}{0.604721in}}%
\pgfpathlineto{\pgfqpoint{4.225410in}{0.602548in}}%
\pgfpathlineto{\pgfqpoint{4.229516in}{0.596780in}}%
\pgfpathlineto{\pgfqpoint{4.226057in}{0.595432in}}%
\pgfpathlineto{\pgfqpoint{4.230361in}{0.589954in}}%
\pgfpathlineto{\pgfqpoint{4.235609in}{0.590538in}}%
\pgfpathlineto{\pgfqpoint{4.238373in}{0.588073in}}%
\pgfpathlineto{\pgfqpoint{4.245363in}{0.584439in}}%
\pgfpathlineto{\pgfqpoint{4.254380in}{0.571539in}}%
\pgfpathlineto{\pgfqpoint{4.254177in}{0.568454in}}%
\pgfpathlineto{\pgfqpoint{4.267700in}{0.558572in}}%
\pgfpathlineto{\pgfqpoint{4.267991in}{0.555076in}}%
\pgfpathlineto{\pgfqpoint{4.271596in}{0.549847in}}%
\pgfpathlineto{\pgfqpoint{4.282667in}{0.547232in}}%
\pgfpathlineto{\pgfqpoint{4.286773in}{0.545373in}}%
\pgfpathlineto{\pgfqpoint{4.290637in}{0.539608in}}%
\pgfpathlineto{\pgfqpoint{4.291134in}{0.534558in}}%
\pgfpathlineto{\pgfqpoint{4.294454in}{0.530441in}}%
\pgfpathlineto{\pgfqpoint{4.295451in}{0.525799in}}%
\pgfpathlineto{\pgfqpoint{4.298563in}{0.524104in}}%
\pgfpathlineto{\pgfqpoint{4.283149in}{0.496285in}}%
\pgfpathlineto{\pgfqpoint{4.251156in}{0.438531in}}%
\pgfpathlineto{\pgfqpoint{4.204216in}{0.353779in}}%
\pgfpathlineto{\pgfqpoint{4.185838in}{0.320590in}}%
\pgfpathlineto{\pgfqpoint{4.189753in}{0.316132in}}%
\pgfpathlineto{\pgfqpoint{4.191477in}{0.317555in}}%
\pgfpathlineto{\pgfqpoint{4.194973in}{0.312365in}}%
\pgfpathlineto{\pgfqpoint{4.199827in}{0.313871in}}%
\pgfpathlineto{\pgfqpoint{4.206270in}{0.310762in}}%
\pgfpathlineto{\pgfqpoint{4.202097in}{0.305904in}}%
\pgfpathlineto{\pgfqpoint{4.205319in}{0.299348in}}%
\pgfpathlineto{\pgfqpoint{4.204653in}{0.296156in}}%
\pgfpathlineto{\pgfqpoint{4.209793in}{0.279323in}}%
\pgfpathlineto{\pgfqpoint{4.207618in}{0.271668in}}%
\pgfpathlineto{\pgfqpoint{4.217438in}{0.273563in}}%
\pgfpathlineto{\pgfqpoint{4.219849in}{0.272321in}}%
\pgfpathlineto{\pgfqpoint{4.222467in}{0.274341in}}%
\pgfpathlineto{\pgfqpoint{4.224329in}{0.278157in}}%
\pgfpathlineto{\pgfqpoint{4.226995in}{0.280345in}}%
\pgfpathlineto{\pgfqpoint{4.238375in}{0.280068in}}%
\pgfpathlineto{\pgfqpoint{4.240947in}{0.272723in}}%
\pgfpathlineto{\pgfqpoint{4.238740in}{0.266275in}}%
\pgfpathlineto{\pgfqpoint{4.241248in}{0.264147in}}%
\pgfpathlineto{\pgfqpoint{4.242567in}{0.260508in}}%
\pgfpathlineto{\pgfqpoint{4.242279in}{0.253664in}}%
\pgfpathlineto{\pgfqpoint{4.245625in}{0.248757in}}%
\pgfpathlineto{\pgfqpoint{4.247460in}{0.241067in}}%
\pgfpathlineto{\pgfqpoint{4.246518in}{0.236913in}}%
\pgfpathlineto{\pgfqpoint{4.247654in}{0.232704in}}%
\pgfpathlineto{\pgfqpoint{4.249292in}{0.208325in}}%
\pgfpathlineto{\pgfqpoint{4.246945in}{0.206405in}}%
\pgfpathlineto{\pgfqpoint{4.250041in}{0.203752in}}%
\pgfpathlineto{\pgfqpoint{4.247608in}{0.200525in}}%
\pgfpathlineto{\pgfqpoint{4.249803in}{0.197893in}}%
\pgfpathlineto{\pgfqpoint{4.248389in}{0.193402in}}%
\pgfpathlineto{\pgfqpoint{4.251550in}{0.192184in}}%
\pgfpathlineto{\pgfqpoint{4.255616in}{0.185354in}}%
\pgfpathlineto{\pgfqpoint{4.258637in}{0.183295in}}%
\pgfpathlineto{\pgfqpoint{4.259427in}{0.180287in}}%
\pgfpathlineto{\pgfqpoint{4.261178in}{0.178983in}}%
\pgfpathlineto{\pgfqpoint{4.260813in}{0.176485in}}%
\pgfpathlineto{\pgfqpoint{4.264585in}{0.174675in}}%
\pgfpathlineto{\pgfqpoint{4.263654in}{0.169868in}}%
\pgfpathlineto{\pgfqpoint{4.260406in}{0.167350in}}%
\pgfpathlineto{\pgfqpoint{4.258931in}{0.162671in}}%
\pgfpathlineto{\pgfqpoint{4.258539in}{0.156407in}}%
\pgfpathlineto{\pgfqpoint{4.256735in}{0.155807in}}%
\pgfpathlineto{\pgfqpoint{4.250838in}{0.150315in}}%
\pgfpathlineto{\pgfqpoint{4.245597in}{0.148866in}}%
\pgfpathlineto{\pgfqpoint{4.243095in}{0.151080in}}%
\pgfpathlineto{\pgfqpoint{4.245512in}{0.156958in}}%
\pgfpathlineto{\pgfqpoint{4.244774in}{0.159844in}}%
\pgfpathlineto{\pgfqpoint{4.248237in}{0.161305in}}%
\pgfpathlineto{\pgfqpoint{4.251475in}{0.169810in}}%
\pgfpathlineto{\pgfqpoint{4.250108in}{0.177073in}}%
\pgfpathlineto{\pgfqpoint{4.247919in}{0.178788in}}%
\pgfpathlineto{\pgfqpoint{4.240599in}{0.178233in}}%
\pgfpathlineto{\pgfqpoint{4.242074in}{0.176335in}}%
\pgfpathlineto{\pgfqpoint{4.236776in}{0.170888in}}%
\pgfpathlineto{\pgfqpoint{4.235069in}{0.173917in}}%
\pgfpathlineto{\pgfqpoint{4.235513in}{0.177873in}}%
\pgfpathlineto{\pgfqpoint{4.238555in}{0.177800in}}%
\pgfpathlineto{\pgfqpoint{4.241262in}{0.180707in}}%
\pgfpathlineto{\pgfqpoint{4.242865in}{0.184951in}}%
\pgfpathlineto{\pgfqpoint{4.248902in}{0.183753in}}%
\pgfpathlineto{\pgfqpoint{4.244069in}{0.186020in}}%
\pgfpathlineto{\pgfqpoint{4.244494in}{0.189057in}}%
\pgfpathlineto{\pgfqpoint{4.242483in}{0.190428in}}%
\pgfpathlineto{\pgfqpoint{4.241659in}{0.194565in}}%
\pgfpathlineto{\pgfqpoint{4.243087in}{0.196720in}}%
\pgfpathlineto{\pgfqpoint{4.239682in}{0.203413in}}%
\pgfpathlineto{\pgfqpoint{4.239969in}{0.207671in}}%
\pgfpathlineto{\pgfqpoint{4.235979in}{0.211462in}}%
\pgfpathlineto{\pgfqpoint{4.233821in}{0.214530in}}%
\pgfpathlineto{\pgfqpoint{4.236802in}{0.217453in}}%
\pgfpathlineto{\pgfqpoint{4.237906in}{0.221822in}}%
\pgfpathlineto{\pgfqpoint{4.236920in}{0.224691in}}%
\pgfpathlineto{\pgfqpoint{4.238074in}{0.228113in}}%
\pgfpathlineto{\pgfqpoint{4.236604in}{0.239268in}}%
\pgfpathlineto{\pgfqpoint{4.234406in}{0.244329in}}%
\pgfpathlineto{\pgfqpoint{4.231893in}{0.246231in}}%
\pgfpathlineto{\pgfqpoint{4.232963in}{0.252814in}}%
\pgfpathlineto{\pgfqpoint{4.232208in}{0.257956in}}%
\pgfpathlineto{\pgfqpoint{4.233575in}{0.261254in}}%
\pgfpathlineto{\pgfqpoint{4.230747in}{0.261716in}}%
\pgfpathlineto{\pgfqpoint{4.229989in}{0.253705in}}%
\pgfpathlineto{\pgfqpoint{4.226810in}{0.244769in}}%
\pgfpathlineto{\pgfqpoint{4.224258in}{0.246804in}}%
\pgfpathlineto{\pgfqpoint{4.223917in}{0.249741in}}%
\pgfpathlineto{\pgfqpoint{4.220795in}{0.253753in}}%
\pgfpathlineto{\pgfqpoint{4.222381in}{0.256380in}}%
\pgfpathlineto{\pgfqpoint{4.221888in}{0.262293in}}%
\pgfpathlineto{\pgfqpoint{4.217847in}{0.260213in}}%
\pgfpathlineto{\pgfqpoint{4.218640in}{0.257006in}}%
\pgfpathlineto{\pgfqpoint{4.217879in}{0.252934in}}%
\pgfpathlineto{\pgfqpoint{4.213343in}{0.252942in}}%
\pgfpathlineto{\pgfqpoint{4.211310in}{0.255466in}}%
\pgfpathlineto{\pgfqpoint{4.208241in}{0.255947in}}%
\pgfpathlineto{\pgfqpoint{4.206176in}{0.258664in}}%
\pgfpathlineto{\pgfqpoint{4.202623in}{0.266126in}}%
\pgfpathlineto{\pgfqpoint{4.201472in}{0.271919in}}%
\pgfpathlineto{\pgfqpoint{4.202209in}{0.274382in}}%
\pgfpathlineto{\pgfqpoint{4.200489in}{0.279805in}}%
\pgfpathlineto{\pgfqpoint{4.197639in}{0.282944in}}%
\pgfpathlineto{\pgfqpoint{4.192711in}{0.290967in}}%
\pgfpathlineto{\pgfqpoint{4.189136in}{0.297935in}}%
\pgfpathlineto{\pgfqpoint{4.192784in}{0.298040in}}%
\pgfpathlineto{\pgfqpoint{4.194930in}{0.299503in}}%
\pgfpathlineto{\pgfqpoint{4.195352in}{0.304032in}}%
\pgfpathlineto{\pgfqpoint{4.185031in}{0.304579in}}%
\pgfpathlineto{\pgfqpoint{4.176254in}{0.314384in}}%
\pgfpathlineto{\pgfqpoint{4.178562in}{0.316963in}}%
\pgfpathlineto{\pgfqpoint{4.174317in}{0.317404in}}%
\pgfpathlineto{\pgfqpoint{4.165887in}{0.326520in}}%
\pgfpathlineto{\pgfqpoint{4.152531in}{0.331091in}}%
\pgfpathlineto{\pgfqpoint{4.150732in}{0.336448in}}%
\pgfpathlineto{\pgfqpoint{4.147961in}{0.339239in}}%
\pgfpathlineto{\pgfqpoint{4.147835in}{0.341859in}}%
\pgfpathlineto{\pgfqpoint{4.145750in}{0.343805in}}%
\pgfpathlineto{\pgfqpoint{4.148486in}{0.347057in}}%
\pgfpathlineto{\pgfqpoint{4.142247in}{0.347370in}}%
\pgfpathlineto{\pgfqpoint{4.140660in}{0.351578in}}%
\pgfpathlineto{\pgfqpoint{4.137821in}{0.353370in}}%
\pgfpathlineto{\pgfqpoint{4.143508in}{0.355533in}}%
\pgfpathlineto{\pgfqpoint{4.140550in}{0.359390in}}%
\pgfpathlineto{\pgfqpoint{4.135107in}{0.361009in}}%
\pgfpathlineto{\pgfqpoint{4.135954in}{0.368134in}}%
\pgfpathlineto{\pgfqpoint{4.134393in}{0.370656in}}%
\pgfpathlineto{\pgfqpoint{4.131583in}{0.372432in}}%
\pgfpathlineto{\pgfqpoint{4.129180in}{0.370651in}}%
\pgfpathlineto{\pgfqpoint{4.127760in}{0.372326in}}%
\pgfpathlineto{\pgfqpoint{4.123618in}{0.372806in}}%
\pgfpathlineto{\pgfqpoint{4.123952in}{0.379272in}}%
\pgfpathlineto{\pgfqpoint{4.117948in}{0.374905in}}%
\pgfpathlineto{\pgfqpoint{4.118075in}{0.365906in}}%
\pgfpathlineto{\pgfqpoint{4.111630in}{0.364180in}}%
\pgfpathlineto{\pgfqpoint{4.107422in}{0.360586in}}%
\pgfpathlineto{\pgfqpoint{4.104039in}{0.359829in}}%
\pgfpathlineto{\pgfqpoint{4.100378in}{0.363451in}}%
\pgfpathlineto{\pgfqpoint{4.097119in}{0.362971in}}%
\pgfpathlineto{\pgfqpoint{4.097723in}{0.366280in}}%
\pgfpathlineto{\pgfqpoint{4.093089in}{0.364498in}}%
\pgfpathlineto{\pgfqpoint{4.088796in}{0.364298in}}%
\pgfpathlineto{\pgfqpoint{4.083771in}{0.362492in}}%
\pgfpathlineto{\pgfqpoint{4.073573in}{0.363508in}}%
\pgfpathlineto{\pgfqpoint{4.070299in}{0.362840in}}%
\pgfpathlineto{\pgfqpoint{4.067937in}{0.365814in}}%
\pgfpathlineto{\pgfqpoint{4.063577in}{0.366112in}}%
\pgfpathlineto{\pgfqpoint{4.062670in}{0.369967in}}%
\pgfpathlineto{\pgfqpoint{4.065268in}{0.372067in}}%
\pgfpathlineto{\pgfqpoint{4.068318in}{0.371869in}}%
\pgfpathlineto{\pgfqpoint{4.070862in}{0.369006in}}%
\pgfpathlineto{\pgfqpoint{4.072538in}{0.373405in}}%
\pgfpathlineto{\pgfqpoint{4.070200in}{0.378278in}}%
\pgfpathlineto{\pgfqpoint{4.075492in}{0.382567in}}%
\pgfpathlineto{\pgfqpoint{4.080796in}{0.384148in}}%
\pgfpathlineto{\pgfqpoint{4.084489in}{0.386726in}}%
\pgfpathlineto{\pgfqpoint{4.086881in}{0.389504in}}%
\pgfpathlineto{\pgfqpoint{4.088215in}{0.394057in}}%
\pgfpathlineto{\pgfqpoint{4.092432in}{0.392818in}}%
\pgfpathlineto{\pgfqpoint{4.101576in}{0.393454in}}%
\pgfpathlineto{\pgfqpoint{4.103538in}{0.388462in}}%
\pgfpathlineto{\pgfqpoint{4.106745in}{0.388245in}}%
\pgfpathlineto{\pgfqpoint{4.112582in}{0.380602in}}%
\pgfpathlineto{\pgfqpoint{4.112593in}{0.383388in}}%
\pgfpathlineto{\pgfqpoint{4.110554in}{0.384294in}}%
\pgfpathlineto{\pgfqpoint{4.106822in}{0.393417in}}%
\pgfpathlineto{\pgfqpoint{4.108638in}{0.394657in}}%
\pgfpathlineto{\pgfqpoint{4.103654in}{0.399416in}}%
\pgfpathlineto{\pgfqpoint{4.099200in}{0.400343in}}%
\pgfpathlineto{\pgfqpoint{4.094956in}{0.398697in}}%
\pgfpathlineto{\pgfqpoint{4.087687in}{0.399979in}}%
\pgfpathlineto{\pgfqpoint{4.084240in}{0.397673in}}%
\pgfpathlineto{\pgfqpoint{4.076390in}{0.395892in}}%
\pgfpathlineto{\pgfqpoint{4.075672in}{0.393328in}}%
\pgfpathlineto{\pgfqpoint{4.069641in}{0.392629in}}%
\pgfpathlineto{\pgfqpoint{4.068093in}{0.389698in}}%
\pgfpathlineto{\pgfqpoint{4.064643in}{0.387551in}}%
\pgfpathlineto{\pgfqpoint{4.060476in}{0.387920in}}%
\pgfpathlineto{\pgfqpoint{4.058395in}{0.385844in}}%
\pgfpathlineto{\pgfqpoint{4.055770in}{0.385942in}}%
\pgfpathlineto{\pgfqpoint{4.053261in}{0.387810in}}%
\pgfpathlineto{\pgfqpoint{4.048875in}{0.387477in}}%
\pgfpathlineto{\pgfqpoint{4.047842in}{0.385754in}}%
\pgfpathlineto{\pgfqpoint{4.043235in}{0.387522in}}%
\pgfpathlineto{\pgfqpoint{4.039324in}{0.382960in}}%
\pgfpathlineto{\pgfqpoint{4.043074in}{0.378545in}}%
\pgfpathlineto{\pgfqpoint{4.044517in}{0.374914in}}%
\pgfpathlineto{\pgfqpoint{4.043187in}{0.371839in}}%
\pgfpathlineto{\pgfqpoint{4.037857in}{0.369717in}}%
\pgfpathlineto{\pgfqpoint{4.034770in}{0.371444in}}%
\pgfpathlineto{\pgfqpoint{4.030883in}{0.370509in}}%
\pgfpathlineto{\pgfqpoint{4.030415in}{0.367814in}}%
\pgfpathlineto{\pgfqpoint{4.025684in}{0.368768in}}%
\pgfpathlineto{\pgfqpoint{4.026732in}{0.366075in}}%
\pgfpathlineto{\pgfqpoint{4.019754in}{0.365507in}}%
\pgfpathlineto{\pgfqpoint{4.015237in}{0.368925in}}%
\pgfpathlineto{\pgfqpoint{4.012824in}{0.366751in}}%
\pgfpathlineto{\pgfqpoint{4.006404in}{0.369065in}}%
\pgfpathlineto{\pgfqpoint{4.004805in}{0.366722in}}%
\pgfpathlineto{\pgfqpoint{4.001622in}{0.365679in}}%
\pgfpathlineto{\pgfqpoint{3.999069in}{0.368427in}}%
\pgfpathlineto{\pgfqpoint{3.990338in}{0.367744in}}%
\pgfpathlineto{\pgfqpoint{3.990348in}{0.363690in}}%
\pgfpathlineto{\pgfqpoint{3.986840in}{0.362618in}}%
\pgfpathlineto{\pgfqpoint{3.984112in}{0.364701in}}%
\pgfpathlineto{\pgfqpoint{3.980505in}{0.363543in}}%
\pgfpathlineto{\pgfqpoint{3.975837in}{0.363268in}}%
\pgfpathlineto{\pgfqpoint{3.968883in}{0.366198in}}%
\pgfpathlineto{\pgfqpoint{3.964901in}{0.363176in}}%
\pgfpathlineto{\pgfqpoint{3.959205in}{0.367215in}}%
\pgfpathlineto{\pgfqpoint{3.957343in}{0.365685in}}%
\pgfpathlineto{\pgfqpoint{3.956036in}{0.361729in}}%
\pgfpathlineto{\pgfqpoint{3.952374in}{0.359410in}}%
\pgfpathlineto{\pgfqpoint{3.946404in}{0.362065in}}%
\pgfpathlineto{\pgfqpoint{3.936514in}{0.368623in}}%
\pgfpathlineto{\pgfqpoint{3.933631in}{0.369206in}}%
\pgfpathlineto{\pgfqpoint{3.932004in}{0.367839in}}%
\pgfpathlineto{\pgfqpoint{3.927654in}{0.368768in}}%
\pgfpathlineto{\pgfqpoint{3.925532in}{0.370809in}}%
\pgfpathlineto{\pgfqpoint{3.921187in}{0.371910in}}%
\pgfpathlineto{\pgfqpoint{3.915253in}{0.371973in}}%
\pgfpathlineto{\pgfqpoint{3.912897in}{0.374408in}}%
\pgfpathlineto{\pgfqpoint{3.917412in}{0.376928in}}%
\pgfpathlineto{\pgfqpoint{3.916358in}{0.379414in}}%
\pgfpathlineto{\pgfqpoint{3.913256in}{0.378856in}}%
\pgfpathlineto{\pgfqpoint{3.911706in}{0.376901in}}%
\pgfpathlineto{\pgfqpoint{3.904443in}{0.374097in}}%
\pgfpathlineto{\pgfqpoint{3.900771in}{0.375051in}}%
\pgfpathlineto{\pgfqpoint{3.899225in}{0.380743in}}%
\pgfpathlineto{\pgfqpoint{3.894644in}{0.378295in}}%
\pgfpathlineto{\pgfqpoint{3.893312in}{0.385509in}}%
\pgfpathlineto{\pgfqpoint{3.891141in}{0.383650in}}%
\pgfpathlineto{\pgfqpoint{3.891504in}{0.380874in}}%
\pgfpathlineto{\pgfqpoint{3.886512in}{0.381607in}}%
\pgfpathlineto{\pgfqpoint{3.891705in}{0.386444in}}%
\pgfpathlineto{\pgfqpoint{3.896968in}{0.383570in}}%
\pgfpathlineto{\pgfqpoint{3.898096in}{0.385013in}}%
\pgfpathlineto{\pgfqpoint{3.903632in}{0.383284in}}%
\pgfpathlineto{\pgfqpoint{3.903287in}{0.385253in}}%
\pgfpathlineto{\pgfqpoint{3.918391in}{0.385905in}}%
\pgfpathlineto{\pgfqpoint{3.925760in}{0.382833in}}%
\pgfpathlineto{\pgfqpoint{3.928612in}{0.379803in}}%
\pgfpathlineto{\pgfqpoint{3.927765in}{0.377157in}}%
\pgfpathlineto{\pgfqpoint{3.930629in}{0.376156in}}%
\pgfpathlineto{\pgfqpoint{3.937743in}{0.380331in}}%
\pgfpathlineto{\pgfqpoint{3.946946in}{0.380599in}}%
\pgfpathlineto{\pgfqpoint{3.955305in}{0.378082in}}%
\pgfpathlineto{\pgfqpoint{3.960040in}{0.378504in}}%
\pgfpathlineto{\pgfqpoint{3.961803in}{0.374600in}}%
\pgfpathlineto{\pgfqpoint{3.964451in}{0.378905in}}%
\pgfpathlineto{\pgfqpoint{3.966079in}{0.379837in}}%
\pgfpathlineto{\pgfqpoint{3.973613in}{0.381708in}}%
\pgfpathlineto{\pgfqpoint{3.976413in}{0.380564in}}%
\pgfpathlineto{\pgfqpoint{3.979441in}{0.381922in}}%
\pgfpathlineto{\pgfqpoint{3.983053in}{0.381212in}}%
\pgfpathlineto{\pgfqpoint{3.983906in}{0.382794in}}%
\pgfpathlineto{\pgfqpoint{3.991442in}{0.390052in}}%
\pgfpathlineto{\pgfqpoint{3.994634in}{0.391128in}}%
\pgfpathlineto{\pgfqpoint{3.996520in}{0.395046in}}%
\pgfpathlineto{\pgfqpoint{4.006530in}{0.398080in}}%
\pgfpathlineto{\pgfqpoint{4.007784in}{0.400394in}}%
\pgfpathlineto{\pgfqpoint{3.993596in}{0.403966in}}%
\pgfpathlineto{\pgfqpoint{3.994154in}{0.407808in}}%
\pgfpathlineto{\pgfqpoint{3.992696in}{0.412881in}}%
\pgfpathlineto{\pgfqpoint{3.989392in}{0.412460in}}%
\pgfpathlineto{\pgfqpoint{3.987144in}{0.405875in}}%
\pgfpathlineto{\pgfqpoint{3.984536in}{0.405333in}}%
\pgfpathlineto{\pgfqpoint{3.982838in}{0.407464in}}%
\pgfpathlineto{\pgfqpoint{3.984819in}{0.416788in}}%
\pgfpathlineto{\pgfqpoint{3.982846in}{0.420695in}}%
\pgfpathlineto{\pgfqpoint{3.979025in}{0.425301in}}%
\pgfpathlineto{\pgfqpoint{3.980862in}{0.428816in}}%
\pgfpathlineto{\pgfqpoint{3.972799in}{0.428942in}}%
\pgfpathlineto{\pgfqpoint{3.972982in}{0.430306in}}%
\pgfpathlineto{\pgfqpoint{3.968171in}{0.432216in}}%
\pgfpathclose%
\pgfusepath{fill}%
\end{pgfscope}%
\begin{pgfscope}%
\pgfpathrectangle{\pgfqpoint{3.525000in}{0.100000in}}{\pgfqpoint{2.857344in}{1.829167in}}%
\pgfusepath{clip}%
\pgfsetbuttcap%
\pgfsetmiterjoin%
\definecolor{currentfill}{rgb}{0.811226,0.923875,0.614533}%
\pgfsetfillcolor{currentfill}%
\pgfsetlinewidth{0.000000pt}%
\definecolor{currentstroke}{rgb}{0.000000,0.000000,0.000000}%
\pgfsetstrokecolor{currentstroke}%
\pgfsetstrokeopacity{0.000000}%
\pgfsetdash{}{0pt}%
\pgfpathmoveto{\pgfqpoint{3.952287in}{0.499074in}}%
\pgfpathlineto{\pgfqpoint{3.951290in}{0.497322in}}%
\pgfpathlineto{\pgfqpoint{3.953928in}{0.493591in}}%
\pgfpathlineto{\pgfqpoint{3.949758in}{0.490101in}}%
\pgfpathlineto{\pgfqpoint{3.949874in}{0.487099in}}%
\pgfpathlineto{\pgfqpoint{3.947989in}{0.486391in}}%
\pgfpathlineto{\pgfqpoint{3.941509in}{0.490867in}}%
\pgfpathlineto{\pgfqpoint{3.936890in}{0.500667in}}%
\pgfpathlineto{\pgfqpoint{3.936580in}{0.503498in}}%
\pgfpathlineto{\pgfqpoint{3.937768in}{0.506888in}}%
\pgfpathlineto{\pgfqpoint{3.942773in}{0.501703in}}%
\pgfpathlineto{\pgfqpoint{3.944304in}{0.502703in}}%
\pgfpathlineto{\pgfqpoint{3.948550in}{0.501752in}}%
\pgfpathclose%
\pgfusepath{fill}%
\end{pgfscope}%
\begin{pgfscope}%
\pgfpathrectangle{\pgfqpoint{3.525000in}{0.100000in}}{\pgfqpoint{2.857344in}{1.829167in}}%
\pgfusepath{clip}%
\pgfsetbuttcap%
\pgfsetmiterjoin%
\definecolor{currentfill}{rgb}{0.811226,0.923875,0.614533}%
\pgfsetfillcolor{currentfill}%
\pgfsetlinewidth{0.000000pt}%
\definecolor{currentstroke}{rgb}{0.000000,0.000000,0.000000}%
\pgfsetstrokecolor{currentstroke}%
\pgfsetstrokeopacity{0.000000}%
\pgfsetdash{}{0pt}%
\pgfpathmoveto{\pgfqpoint{3.875434in}{0.385546in}}%
\pgfpathlineto{\pgfqpoint{3.870880in}{0.384806in}}%
\pgfpathlineto{\pgfqpoint{3.867584in}{0.386330in}}%
\pgfpathlineto{\pgfqpoint{3.866143in}{0.388810in}}%
\pgfpathlineto{\pgfqpoint{3.867776in}{0.392368in}}%
\pgfpathlineto{\pgfqpoint{3.871412in}{0.391427in}}%
\pgfpathlineto{\pgfqpoint{3.877390in}{0.393538in}}%
\pgfpathlineto{\pgfqpoint{3.879484in}{0.390622in}}%
\pgfpathlineto{\pgfqpoint{3.881513in}{0.391160in}}%
\pgfpathlineto{\pgfqpoint{3.886427in}{0.389346in}}%
\pgfpathlineto{\pgfqpoint{3.888551in}{0.387041in}}%
\pgfpathlineto{\pgfqpoint{3.885960in}{0.381012in}}%
\pgfpathlineto{\pgfqpoint{3.883382in}{0.379734in}}%
\pgfpathlineto{\pgfqpoint{3.881066in}{0.380423in}}%
\pgfpathclose%
\pgfusepath{fill}%
\end{pgfscope}%
\begin{pgfscope}%
\pgfpathrectangle{\pgfqpoint{3.525000in}{0.100000in}}{\pgfqpoint{2.857344in}{1.829167in}}%
\pgfusepath{clip}%
\pgfsetbuttcap%
\pgfsetmiterjoin%
\definecolor{currentfill}{rgb}{0.811226,0.923875,0.614533}%
\pgfsetfillcolor{currentfill}%
\pgfsetlinewidth{0.000000pt}%
\definecolor{currentstroke}{rgb}{0.000000,0.000000,0.000000}%
\pgfsetstrokecolor{currentstroke}%
\pgfsetstrokeopacity{0.000000}%
\pgfsetdash{}{0pt}%
\pgfpathmoveto{\pgfqpoint{3.834711in}{0.391024in}}%
\pgfpathlineto{\pgfqpoint{3.828258in}{0.394102in}}%
\pgfpathlineto{\pgfqpoint{3.822286in}{0.395017in}}%
\pgfpathlineto{\pgfqpoint{3.820404in}{0.396328in}}%
\pgfpathlineto{\pgfqpoint{3.822547in}{0.398909in}}%
\pgfpathlineto{\pgfqpoint{3.826351in}{0.395685in}}%
\pgfpathlineto{\pgfqpoint{3.829541in}{0.396130in}}%
\pgfpathlineto{\pgfqpoint{3.834730in}{0.398362in}}%
\pgfpathlineto{\pgfqpoint{3.835149in}{0.401237in}}%
\pgfpathlineto{\pgfqpoint{3.841521in}{0.399814in}}%
\pgfpathlineto{\pgfqpoint{3.840029in}{0.395485in}}%
\pgfpathlineto{\pgfqpoint{3.844089in}{0.395169in}}%
\pgfpathlineto{\pgfqpoint{3.840683in}{0.390019in}}%
\pgfpathlineto{\pgfqpoint{3.837384in}{0.391688in}}%
\pgfpathclose%
\pgfusepath{fill}%
\end{pgfscope}%
\begin{pgfscope}%
\pgfpathrectangle{\pgfqpoint{3.525000in}{0.100000in}}{\pgfqpoint{2.857344in}{1.829167in}}%
\pgfusepath{clip}%
\pgfsetbuttcap%
\pgfsetmiterjoin%
\definecolor{currentfill}{rgb}{0.811226,0.923875,0.614533}%
\pgfsetfillcolor{currentfill}%
\pgfsetlinewidth{0.000000pt}%
\definecolor{currentstroke}{rgb}{0.000000,0.000000,0.000000}%
\pgfsetstrokecolor{currentstroke}%
\pgfsetstrokeopacity{0.000000}%
\pgfsetdash{}{0pt}%
\pgfpathmoveto{\pgfqpoint{3.823373in}{0.401666in}}%
\pgfpathlineto{\pgfqpoint{3.821001in}{0.400376in}}%
\pgfpathlineto{\pgfqpoint{3.814538in}{0.402358in}}%
\pgfpathlineto{\pgfqpoint{3.809709in}{0.401210in}}%
\pgfpathlineto{\pgfqpoint{3.807256in}{0.404472in}}%
\pgfpathlineto{\pgfqpoint{3.808326in}{0.405965in}}%
\pgfpathlineto{\pgfqpoint{3.811986in}{0.406215in}}%
\pgfpathlineto{\pgfqpoint{3.814785in}{0.405277in}}%
\pgfpathlineto{\pgfqpoint{3.817835in}{0.406761in}}%
\pgfpathlineto{\pgfqpoint{3.822512in}{0.404777in}}%
\pgfpathclose%
\pgfusepath{fill}%
\end{pgfscope}%
\begin{pgfscope}%
\pgfpathrectangle{\pgfqpoint{3.525000in}{0.100000in}}{\pgfqpoint{2.857344in}{1.829167in}}%
\pgfusepath{clip}%
\pgfsetbuttcap%
\pgfsetmiterjoin%
\definecolor{currentfill}{rgb}{0.811226,0.923875,0.614533}%
\pgfsetfillcolor{currentfill}%
\pgfsetlinewidth{0.000000pt}%
\definecolor{currentstroke}{rgb}{0.000000,0.000000,0.000000}%
\pgfsetstrokecolor{currentstroke}%
\pgfsetstrokeopacity{0.000000}%
\pgfsetdash{}{0pt}%
\pgfpathmoveto{\pgfqpoint{3.747189in}{0.449667in}}%
\pgfpathlineto{\pgfqpoint{3.747018in}{0.446469in}}%
\pgfpathlineto{\pgfqpoint{3.742300in}{0.444459in}}%
\pgfpathlineto{\pgfqpoint{3.737991in}{0.449490in}}%
\pgfpathlineto{\pgfqpoint{3.742823in}{0.451166in}}%
\pgfpathclose%
\pgfusepath{fill}%
\end{pgfscope}%
\begin{pgfscope}%
\pgfpathrectangle{\pgfqpoint{3.525000in}{0.100000in}}{\pgfqpoint{2.857344in}{1.829167in}}%
\pgfusepath{clip}%
\pgfsetbuttcap%
\pgfsetmiterjoin%
\definecolor{currentfill}{rgb}{0.811226,0.923875,0.614533}%
\pgfsetfillcolor{currentfill}%
\pgfsetlinewidth{0.000000pt}%
\definecolor{currentstroke}{rgb}{0.000000,0.000000,0.000000}%
\pgfsetstrokecolor{currentstroke}%
\pgfsetstrokeopacity{0.000000}%
\pgfsetdash{}{0pt}%
\pgfpathmoveto{\pgfqpoint{3.708276in}{0.467645in}}%
\pgfpathlineto{\pgfqpoint{3.709278in}{0.469036in}}%
\pgfpathlineto{\pgfqpoint{3.715065in}{0.468796in}}%
\pgfpathlineto{\pgfqpoint{3.716080in}{0.470864in}}%
\pgfpathlineto{\pgfqpoint{3.718620in}{0.469059in}}%
\pgfpathlineto{\pgfqpoint{3.717153in}{0.465586in}}%
\pgfpathlineto{\pgfqpoint{3.714997in}{0.465292in}}%
\pgfpathlineto{\pgfqpoint{3.709842in}{0.466382in}}%
\pgfpathclose%
\pgfusepath{fill}%
\end{pgfscope}%
\begin{pgfscope}%
\pgfpathrectangle{\pgfqpoint{3.525000in}{0.100000in}}{\pgfqpoint{2.857344in}{1.829167in}}%
\pgfusepath{clip}%
\pgfsetbuttcap%
\pgfsetmiterjoin%
\definecolor{currentfill}{rgb}{0.811226,0.923875,0.614533}%
\pgfsetfillcolor{currentfill}%
\pgfsetlinewidth{0.000000pt}%
\definecolor{currentstroke}{rgb}{0.000000,0.000000,0.000000}%
\pgfsetstrokecolor{currentstroke}%
\pgfsetstrokeopacity{0.000000}%
\pgfsetdash{}{0pt}%
\pgfpathmoveto{\pgfqpoint{3.700069in}{0.478409in}}%
\pgfpathlineto{\pgfqpoint{3.699718in}{0.481541in}}%
\pgfpathlineto{\pgfqpoint{3.702738in}{0.481309in}}%
\pgfpathlineto{\pgfqpoint{3.702527in}{0.485708in}}%
\pgfpathlineto{\pgfqpoint{3.705565in}{0.483334in}}%
\pgfpathlineto{\pgfqpoint{3.704514in}{0.479903in}}%
\pgfpathclose%
\pgfusepath{fill}%
\end{pgfscope}%
\begin{pgfscope}%
\pgfpathrectangle{\pgfqpoint{3.525000in}{0.100000in}}{\pgfqpoint{2.857344in}{1.829167in}}%
\pgfusepath{clip}%
\pgfsetbuttcap%
\pgfsetmiterjoin%
\definecolor{currentfill}{rgb}{0.811226,0.923875,0.614533}%
\pgfsetfillcolor{currentfill}%
\pgfsetlinewidth{0.000000pt}%
\definecolor{currentstroke}{rgb}{0.000000,0.000000,0.000000}%
\pgfsetstrokecolor{currentstroke}%
\pgfsetstrokeopacity{0.000000}%
\pgfsetdash{}{0pt}%
\pgfpathmoveto{\pgfqpoint{3.966109in}{0.590049in}}%
\pgfpathlineto{\pgfqpoint{3.961293in}{0.591089in}}%
\pgfpathlineto{\pgfqpoint{3.960258in}{0.594307in}}%
\pgfpathlineto{\pgfqpoint{3.961914in}{0.597524in}}%
\pgfpathlineto{\pgfqpoint{3.965621in}{0.599205in}}%
\pgfpathlineto{\pgfqpoint{3.969865in}{0.590946in}}%
\pgfpathlineto{\pgfqpoint{3.973715in}{0.590580in}}%
\pgfpathlineto{\pgfqpoint{3.976587in}{0.588031in}}%
\pgfpathlineto{\pgfqpoint{3.976531in}{0.584525in}}%
\pgfpathlineto{\pgfqpoint{3.974431in}{0.582844in}}%
\pgfpathlineto{\pgfqpoint{3.977846in}{0.572475in}}%
\pgfpathlineto{\pgfqpoint{3.981583in}{0.568669in}}%
\pgfpathlineto{\pgfqpoint{3.977762in}{0.567462in}}%
\pgfpathlineto{\pgfqpoint{3.974670in}{0.571731in}}%
\pgfpathlineto{\pgfqpoint{3.967949in}{0.570410in}}%
\pgfpathlineto{\pgfqpoint{3.970039in}{0.574635in}}%
\pgfpathlineto{\pgfqpoint{3.969066in}{0.576318in}}%
\pgfpathlineto{\pgfqpoint{3.969636in}{0.580722in}}%
\pgfpathlineto{\pgfqpoint{3.968304in}{0.588683in}}%
\pgfpathclose%
\pgfusepath{fill}%
\end{pgfscope}%
\begin{pgfscope}%
\pgfpathrectangle{\pgfqpoint{3.525000in}{0.100000in}}{\pgfqpoint{2.857344in}{1.829167in}}%
\pgfusepath{clip}%
\pgfsetbuttcap%
\pgfsetmiterjoin%
\definecolor{currentfill}{rgb}{0.811226,0.923875,0.614533}%
\pgfsetfillcolor{currentfill}%
\pgfsetlinewidth{0.000000pt}%
\definecolor{currentstroke}{rgb}{0.000000,0.000000,0.000000}%
\pgfsetstrokecolor{currentstroke}%
\pgfsetstrokeopacity{0.000000}%
\pgfsetdash{}{0pt}%
\pgfpathmoveto{\pgfqpoint{4.135314in}{0.352834in}}%
\pgfpathlineto{\pgfqpoint{4.128332in}{0.352913in}}%
\pgfpathlineto{\pgfqpoint{4.128870in}{0.356418in}}%
\pgfpathlineto{\pgfqpoint{4.131665in}{0.357671in}}%
\pgfpathclose%
\pgfusepath{fill}%
\end{pgfscope}%
\begin{pgfscope}%
\pgfpathrectangle{\pgfqpoint{3.525000in}{0.100000in}}{\pgfqpoint{2.857344in}{1.829167in}}%
\pgfusepath{clip}%
\pgfsetbuttcap%
\pgfsetmiterjoin%
\definecolor{currentfill}{rgb}{0.811226,0.923875,0.614533}%
\pgfsetfillcolor{currentfill}%
\pgfsetlinewidth{0.000000pt}%
\definecolor{currentstroke}{rgb}{0.000000,0.000000,0.000000}%
\pgfsetstrokecolor{currentstroke}%
\pgfsetstrokeopacity{0.000000}%
\pgfsetdash{}{0pt}%
\pgfpathmoveto{\pgfqpoint{4.125626in}{0.356008in}}%
\pgfpathlineto{\pgfqpoint{4.120624in}{0.354798in}}%
\pgfpathlineto{\pgfqpoint{4.115871in}{0.352726in}}%
\pgfpathlineto{\pgfqpoint{4.114404in}{0.355168in}}%
\pgfpathlineto{\pgfqpoint{4.122577in}{0.356821in}}%
\pgfpathclose%
\pgfusepath{fill}%
\end{pgfscope}%
\begin{pgfscope}%
\pgfpathrectangle{\pgfqpoint{3.525000in}{0.100000in}}{\pgfqpoint{2.857344in}{1.829167in}}%
\pgfusepath{clip}%
\pgfsetbuttcap%
\pgfsetmiterjoin%
\definecolor{currentfill}{rgb}{0.811226,0.923875,0.614533}%
\pgfsetfillcolor{currentfill}%
\pgfsetlinewidth{0.000000pt}%
\definecolor{currentstroke}{rgb}{0.000000,0.000000,0.000000}%
\pgfsetstrokecolor{currentstroke}%
\pgfsetstrokeopacity{0.000000}%
\pgfsetdash{}{0pt}%
\pgfpathmoveto{\pgfqpoint{4.048956in}{0.353527in}}%
\pgfpathlineto{\pgfqpoint{4.049147in}{0.351104in}}%
\pgfpathlineto{\pgfqpoint{4.046265in}{0.349597in}}%
\pgfpathlineto{\pgfqpoint{4.038222in}{0.353052in}}%
\pgfpathlineto{\pgfqpoint{4.036975in}{0.351662in}}%
\pgfpathlineto{\pgfqpoint{4.035029in}{0.358237in}}%
\pgfpathlineto{\pgfqpoint{4.039059in}{0.359152in}}%
\pgfpathlineto{\pgfqpoint{4.040921in}{0.356450in}}%
\pgfpathlineto{\pgfqpoint{4.044966in}{0.359832in}}%
\pgfpathclose%
\pgfusepath{fill}%
\end{pgfscope}%
\begin{pgfscope}%
\pgfpathrectangle{\pgfqpoint{3.525000in}{0.100000in}}{\pgfqpoint{2.857344in}{1.829167in}}%
\pgfusepath{clip}%
\pgfsetbuttcap%
\pgfsetmiterjoin%
\definecolor{currentfill}{rgb}{0.811226,0.923875,0.614533}%
\pgfsetfillcolor{currentfill}%
\pgfsetlinewidth{0.000000pt}%
\definecolor{currentstroke}{rgb}{0.000000,0.000000,0.000000}%
\pgfsetstrokecolor{currentstroke}%
\pgfsetstrokeopacity{0.000000}%
\pgfsetdash{}{0pt}%
\pgfpathmoveto{\pgfqpoint{4.230382in}{0.219176in}}%
\pgfpathlineto{\pgfqpoint{4.223637in}{0.216665in}}%
\pgfpathlineto{\pgfqpoint{4.220065in}{0.216515in}}%
\pgfpathlineto{\pgfqpoint{4.222973in}{0.221842in}}%
\pgfpathlineto{\pgfqpoint{4.225457in}{0.223828in}}%
\pgfpathlineto{\pgfqpoint{4.225377in}{0.229345in}}%
\pgfpathlineto{\pgfqpoint{4.228077in}{0.239303in}}%
\pgfpathlineto{\pgfqpoint{4.230564in}{0.241194in}}%
\pgfpathlineto{\pgfqpoint{4.236306in}{0.238456in}}%
\pgfpathlineto{\pgfqpoint{4.235467in}{0.236887in}}%
\pgfpathlineto{\pgfqpoint{4.235902in}{0.229349in}}%
\pgfpathlineto{\pgfqpoint{4.234041in}{0.227314in}}%
\pgfpathlineto{\pgfqpoint{4.233241in}{0.221960in}}%
\pgfpathclose%
\pgfusepath{fill}%
\end{pgfscope}%
\begin{pgfscope}%
\pgfpathrectangle{\pgfqpoint{3.525000in}{0.100000in}}{\pgfqpoint{2.857344in}{1.829167in}}%
\pgfusepath{clip}%
\pgfsetbuttcap%
\pgfsetmiterjoin%
\definecolor{currentfill}{rgb}{0.811226,0.923875,0.614533}%
\pgfsetfillcolor{currentfill}%
\pgfsetlinewidth{0.000000pt}%
\definecolor{currentstroke}{rgb}{0.000000,0.000000,0.000000}%
\pgfsetstrokecolor{currentstroke}%
\pgfsetstrokeopacity{0.000000}%
\pgfsetdash{}{0pt}%
\pgfpathmoveto{\pgfqpoint{4.215446in}{0.243862in}}%
\pgfpathlineto{\pgfqpoint{4.219397in}{0.237040in}}%
\pgfpathlineto{\pgfqpoint{4.218665in}{0.235412in}}%
\pgfpathlineto{\pgfqpoint{4.223949in}{0.233757in}}%
\pgfpathlineto{\pgfqpoint{4.222432in}{0.227487in}}%
\pgfpathlineto{\pgfqpoint{4.220131in}{0.227776in}}%
\pgfpathlineto{\pgfqpoint{4.214085in}{0.238709in}}%
\pgfpathlineto{\pgfqpoint{4.214945in}{0.233788in}}%
\pgfpathlineto{\pgfqpoint{4.213915in}{0.230955in}}%
\pgfpathlineto{\pgfqpoint{4.211354in}{0.229694in}}%
\pgfpathlineto{\pgfqpoint{4.209899in}{0.231052in}}%
\pgfpathlineto{\pgfqpoint{4.209086in}{0.237352in}}%
\pgfpathlineto{\pgfqpoint{4.209566in}{0.241438in}}%
\pgfpathlineto{\pgfqpoint{4.207860in}{0.243221in}}%
\pgfpathlineto{\pgfqpoint{4.212365in}{0.248382in}}%
\pgfpathlineto{\pgfqpoint{4.215321in}{0.249913in}}%
\pgfpathlineto{\pgfqpoint{4.216305in}{0.247901in}}%
\pgfpathlineto{\pgfqpoint{4.219389in}{0.249458in}}%
\pgfpathlineto{\pgfqpoint{4.221623in}{0.245229in}}%
\pgfpathlineto{\pgfqpoint{4.226456in}{0.239850in}}%
\pgfpathlineto{\pgfqpoint{4.225796in}{0.237413in}}%
\pgfpathlineto{\pgfqpoint{4.223113in}{0.234758in}}%
\pgfpathlineto{\pgfqpoint{4.220632in}{0.236009in}}%
\pgfpathclose%
\pgfusepath{fill}%
\end{pgfscope}%
\begin{pgfscope}%
\pgfpathrectangle{\pgfqpoint{3.525000in}{0.100000in}}{\pgfqpoint{2.857344in}{1.829167in}}%
\pgfusepath{clip}%
\pgfsetbuttcap%
\pgfsetmiterjoin%
\definecolor{currentfill}{rgb}{0.811226,0.923875,0.614533}%
\pgfsetfillcolor{currentfill}%
\pgfsetlinewidth{0.000000pt}%
\definecolor{currentstroke}{rgb}{0.000000,0.000000,0.000000}%
\pgfsetstrokecolor{currentstroke}%
\pgfsetstrokeopacity{0.000000}%
\pgfsetdash{}{0pt}%
\pgfpathmoveto{\pgfqpoint{4.019377in}{0.340596in}}%
\pgfpathlineto{\pgfqpoint{4.014482in}{0.339959in}}%
\pgfpathlineto{\pgfqpoint{4.007589in}{0.337310in}}%
\pgfpathlineto{\pgfqpoint{4.005982in}{0.338749in}}%
\pgfpathlineto{\pgfqpoint{4.013182in}{0.340376in}}%
\pgfpathlineto{\pgfqpoint{4.011252in}{0.341796in}}%
\pgfpathlineto{\pgfqpoint{4.006787in}{0.342836in}}%
\pgfpathlineto{\pgfqpoint{4.005596in}{0.346163in}}%
\pgfpathlineto{\pgfqpoint{4.007842in}{0.349029in}}%
\pgfpathlineto{\pgfqpoint{4.009734in}{0.355607in}}%
\pgfpathlineto{\pgfqpoint{4.016643in}{0.356832in}}%
\pgfpathlineto{\pgfqpoint{4.019989in}{0.353991in}}%
\pgfpathlineto{\pgfqpoint{4.023131in}{0.356741in}}%
\pgfpathlineto{\pgfqpoint{4.027073in}{0.356066in}}%
\pgfpathlineto{\pgfqpoint{4.029905in}{0.350986in}}%
\pgfpathlineto{\pgfqpoint{4.032304in}{0.355502in}}%
\pgfpathlineto{\pgfqpoint{4.039111in}{0.344940in}}%
\pgfpathlineto{\pgfqpoint{4.036652in}{0.344095in}}%
\pgfpathlineto{\pgfqpoint{4.035143in}{0.341806in}}%
\pgfpathlineto{\pgfqpoint{4.037994in}{0.339716in}}%
\pgfpathlineto{\pgfqpoint{4.033572in}{0.337758in}}%
\pgfpathlineto{\pgfqpoint{4.030366in}{0.338836in}}%
\pgfpathlineto{\pgfqpoint{4.024310in}{0.342411in}}%
\pgfpathclose%
\pgfusepath{fill}%
\end{pgfscope}%
\begin{pgfscope}%
\pgfpathrectangle{\pgfqpoint{3.525000in}{0.100000in}}{\pgfqpoint{2.857344in}{1.829167in}}%
\pgfusepath{clip}%
\pgfsetbuttcap%
\pgfsetmiterjoin%
\definecolor{currentfill}{rgb}{0.811226,0.923875,0.614533}%
\pgfsetfillcolor{currentfill}%
\pgfsetlinewidth{0.000000pt}%
\definecolor{currentstroke}{rgb}{0.000000,0.000000,0.000000}%
\pgfsetstrokecolor{currentstroke}%
\pgfsetstrokeopacity{0.000000}%
\pgfsetdash{}{0pt}%
\pgfpathmoveto{\pgfqpoint{4.211336in}{0.199025in}}%
\pgfpathlineto{\pgfqpoint{4.210366in}{0.208779in}}%
\pgfpathlineto{\pgfqpoint{4.212982in}{0.212361in}}%
\pgfpathlineto{\pgfqpoint{4.210185in}{0.212610in}}%
\pgfpathlineto{\pgfqpoint{4.211207in}{0.219576in}}%
\pgfpathlineto{\pgfqpoint{4.212412in}{0.223711in}}%
\pgfpathlineto{\pgfqpoint{4.211365in}{0.229267in}}%
\pgfpathlineto{\pgfqpoint{4.213927in}{0.230342in}}%
\pgfpathlineto{\pgfqpoint{4.214763in}{0.231779in}}%
\pgfpathlineto{\pgfqpoint{4.217286in}{0.231234in}}%
\pgfpathlineto{\pgfqpoint{4.219471in}{0.226783in}}%
\pgfpathlineto{\pgfqpoint{4.219877in}{0.222705in}}%
\pgfpathlineto{\pgfqpoint{4.217204in}{0.210491in}}%
\pgfpathclose%
\pgfusepath{fill}%
\end{pgfscope}%
\begin{pgfscope}%
\pgfpathrectangle{\pgfqpoint{3.525000in}{0.100000in}}{\pgfqpoint{2.857344in}{1.829167in}}%
\pgfusepath{clip}%
\pgfsetbuttcap%
\pgfsetmiterjoin%
\definecolor{currentfill}{rgb}{0.811226,0.923875,0.614533}%
\pgfsetfillcolor{currentfill}%
\pgfsetlinewidth{0.000000pt}%
\definecolor{currentstroke}{rgb}{0.000000,0.000000,0.000000}%
\pgfsetstrokecolor{currentstroke}%
\pgfsetstrokeopacity{0.000000}%
\pgfsetdash{}{0pt}%
\pgfpathmoveto{\pgfqpoint{4.210756in}{0.228801in}}%
\pgfpathlineto{\pgfqpoint{4.210622in}{0.224045in}}%
\pgfpathlineto{\pgfqpoint{4.208428in}{0.221886in}}%
\pgfpathlineto{\pgfqpoint{4.205893in}{0.222684in}}%
\pgfpathlineto{\pgfqpoint{4.209098in}{0.229592in}}%
\pgfpathclose%
\pgfusepath{fill}%
\end{pgfscope}%
\begin{pgfscope}%
\pgfpathrectangle{\pgfqpoint{3.525000in}{0.100000in}}{\pgfqpoint{2.857344in}{1.829167in}}%
\pgfusepath{clip}%
\pgfsetbuttcap%
\pgfsetmiterjoin%
\definecolor{currentfill}{rgb}{0.811226,0.923875,0.614533}%
\pgfsetfillcolor{currentfill}%
\pgfsetlinewidth{0.000000pt}%
\definecolor{currentstroke}{rgb}{0.000000,0.000000,0.000000}%
\pgfsetstrokecolor{currentstroke}%
\pgfsetstrokeopacity{0.000000}%
\pgfsetdash{}{0pt}%
\pgfpathmoveto{\pgfqpoint{4.237669in}{0.207686in}}%
\pgfpathlineto{\pgfqpoint{4.235799in}{0.199726in}}%
\pgfpathlineto{\pgfqpoint{4.231441in}{0.197210in}}%
\pgfpathlineto{\pgfqpoint{4.225723in}{0.199528in}}%
\pgfpathlineto{\pgfqpoint{4.227299in}{0.202621in}}%
\pgfpathlineto{\pgfqpoint{4.227267in}{0.204763in}}%
\pgfpathlineto{\pgfqpoint{4.229392in}{0.208402in}}%
\pgfpathlineto{\pgfqpoint{4.227898in}{0.207670in}}%
\pgfpathlineto{\pgfqpoint{4.228987in}{0.209447in}}%
\pgfpathlineto{\pgfqpoint{4.226918in}{0.213485in}}%
\pgfpathlineto{\pgfqpoint{4.229267in}{0.214229in}}%
\pgfpathlineto{\pgfqpoint{4.234703in}{0.209504in}}%
\pgfpathclose%
\pgfusepath{fill}%
\end{pgfscope}%
\begin{pgfscope}%
\pgfpathrectangle{\pgfqpoint{3.525000in}{0.100000in}}{\pgfqpoint{2.857344in}{1.829167in}}%
\pgfusepath{clip}%
\pgfsetbuttcap%
\pgfsetmiterjoin%
\definecolor{currentfill}{rgb}{0.811226,0.923875,0.614533}%
\pgfsetfillcolor{currentfill}%
\pgfsetlinewidth{0.000000pt}%
\definecolor{currentstroke}{rgb}{0.000000,0.000000,0.000000}%
\pgfsetstrokecolor{currentstroke}%
\pgfsetstrokeopacity{0.000000}%
\pgfsetdash{}{0pt}%
\pgfpathmoveto{\pgfqpoint{4.218592in}{0.193612in}}%
\pgfpathlineto{\pgfqpoint{4.216127in}{0.192653in}}%
\pgfpathlineto{\pgfqpoint{4.217535in}{0.201052in}}%
\pgfpathlineto{\pgfqpoint{4.220493in}{0.203012in}}%
\pgfpathlineto{\pgfqpoint{4.219772in}{0.209306in}}%
\pgfpathlineto{\pgfqpoint{4.220974in}{0.212096in}}%
\pgfpathlineto{\pgfqpoint{4.223352in}{0.213142in}}%
\pgfpathlineto{\pgfqpoint{4.225848in}{0.210404in}}%
\pgfpathlineto{\pgfqpoint{4.227991in}{0.206690in}}%
\pgfpathlineto{\pgfqpoint{4.221104in}{0.199413in}}%
\pgfpathclose%
\pgfusepath{fill}%
\end{pgfscope}%
\begin{pgfscope}%
\pgfpathrectangle{\pgfqpoint{3.525000in}{0.100000in}}{\pgfqpoint{2.857344in}{1.829167in}}%
\pgfusepath{clip}%
\pgfsetbuttcap%
\pgfsetmiterjoin%
\definecolor{currentfill}{rgb}{0.811226,0.923875,0.614533}%
\pgfsetfillcolor{currentfill}%
\pgfsetlinewidth{0.000000pt}%
\definecolor{currentstroke}{rgb}{0.000000,0.000000,0.000000}%
\pgfsetstrokecolor{currentstroke}%
\pgfsetstrokeopacity{0.000000}%
\pgfsetdash{}{0pt}%
\pgfpathmoveto{\pgfqpoint{4.238762in}{0.202366in}}%
\pgfpathlineto{\pgfqpoint{4.239970in}{0.196570in}}%
\pgfpathlineto{\pgfqpoint{4.234573in}{0.197233in}}%
\pgfpathclose%
\pgfusepath{fill}%
\end{pgfscope}%
\begin{pgfscope}%
\pgfpathrectangle{\pgfqpoint{3.525000in}{0.100000in}}{\pgfqpoint{2.857344in}{1.829167in}}%
\pgfusepath{clip}%
\pgfsetbuttcap%
\pgfsetmiterjoin%
\definecolor{currentfill}{rgb}{0.811226,0.923875,0.614533}%
\pgfsetfillcolor{currentfill}%
\pgfsetlinewidth{0.000000pt}%
\definecolor{currentstroke}{rgb}{0.000000,0.000000,0.000000}%
\pgfsetstrokecolor{currentstroke}%
\pgfsetstrokeopacity{0.000000}%
\pgfsetdash{}{0pt}%
\pgfpathmoveto{\pgfqpoint{4.241338in}{0.193510in}}%
\pgfpathlineto{\pgfqpoint{4.242222in}{0.189934in}}%
\pgfpathlineto{\pgfqpoint{4.243915in}{0.188790in}}%
\pgfpathlineto{\pgfqpoint{4.243608in}{0.185425in}}%
\pgfpathlineto{\pgfqpoint{4.241242in}{0.184293in}}%
\pgfpathlineto{\pgfqpoint{4.239322in}{0.189317in}}%
\pgfpathclose%
\pgfusepath{fill}%
\end{pgfscope}%
\begin{pgfscope}%
\pgfpathrectangle{\pgfqpoint{3.525000in}{0.100000in}}{\pgfqpoint{2.857344in}{1.829167in}}%
\pgfusepath{clip}%
\pgfsetbuttcap%
\pgfsetmiterjoin%
\definecolor{currentfill}{rgb}{0.811226,0.923875,0.614533}%
\pgfsetfillcolor{currentfill}%
\pgfsetlinewidth{0.000000pt}%
\definecolor{currentstroke}{rgb}{0.000000,0.000000,0.000000}%
\pgfsetstrokecolor{currentstroke}%
\pgfsetstrokeopacity{0.000000}%
\pgfsetdash{}{0pt}%
\pgfpathmoveto{\pgfqpoint{4.236834in}{0.194881in}}%
\pgfpathlineto{\pgfqpoint{4.235366in}{0.190728in}}%
\pgfpathlineto{\pgfqpoint{4.233086in}{0.190880in}}%
\pgfpathlineto{\pgfqpoint{4.231648in}{0.194305in}}%
\pgfpathlineto{\pgfqpoint{4.233990in}{0.195928in}}%
\pgfpathclose%
\pgfusepath{fill}%
\end{pgfscope}%
\begin{pgfscope}%
\pgfpathrectangle{\pgfqpoint{3.525000in}{0.100000in}}{\pgfqpoint{2.857344in}{1.829167in}}%
\pgfusepath{clip}%
\pgfsetbuttcap%
\pgfsetmiterjoin%
\definecolor{currentfill}{rgb}{0.811226,0.923875,0.614533}%
\pgfsetfillcolor{currentfill}%
\pgfsetlinewidth{0.000000pt}%
\definecolor{currentstroke}{rgb}{0.000000,0.000000,0.000000}%
\pgfsetstrokecolor{currentstroke}%
\pgfsetstrokeopacity{0.000000}%
\pgfsetdash{}{0pt}%
\pgfpathmoveto{\pgfqpoint{4.230683in}{0.173468in}}%
\pgfpathlineto{\pgfqpoint{4.232717in}{0.171243in}}%
\pgfpathlineto{\pgfqpoint{4.233165in}{0.166555in}}%
\pgfpathlineto{\pgfqpoint{4.234236in}{0.165192in}}%
\pgfpathlineto{\pgfqpoint{4.232487in}{0.160949in}}%
\pgfpathlineto{\pgfqpoint{4.230186in}{0.158561in}}%
\pgfpathlineto{\pgfqpoint{4.228962in}{0.154129in}}%
\pgfpathlineto{\pgfqpoint{4.226447in}{0.161035in}}%
\pgfpathlineto{\pgfqpoint{4.226030in}{0.167306in}}%
\pgfpathlineto{\pgfqpoint{4.224622in}{0.170428in}}%
\pgfpathlineto{\pgfqpoint{4.219776in}{0.172958in}}%
\pgfpathlineto{\pgfqpoint{4.224322in}{0.178236in}}%
\pgfpathlineto{\pgfqpoint{4.221291in}{0.180491in}}%
\pgfpathlineto{\pgfqpoint{4.223961in}{0.182739in}}%
\pgfpathlineto{\pgfqpoint{4.227506in}{0.191368in}}%
\pgfpathlineto{\pgfqpoint{4.223832in}{0.194411in}}%
\pgfpathlineto{\pgfqpoint{4.225247in}{0.197332in}}%
\pgfpathlineto{\pgfqpoint{4.229997in}{0.194741in}}%
\pgfpathlineto{\pgfqpoint{4.230478in}{0.190652in}}%
\pgfpathlineto{\pgfqpoint{4.228681in}{0.188323in}}%
\pgfpathlineto{\pgfqpoint{4.232242in}{0.185173in}}%
\pgfpathlineto{\pgfqpoint{4.233363in}{0.180266in}}%
\pgfpathlineto{\pgfqpoint{4.233211in}{0.173383in}}%
\pgfpathclose%
\pgfusepath{fill}%
\end{pgfscope}%
\begin{pgfscope}%
\pgfpathrectangle{\pgfqpoint{3.525000in}{0.100000in}}{\pgfqpoint{2.857344in}{1.829167in}}%
\pgfusepath{clip}%
\pgfsetbuttcap%
\pgfsetmiterjoin%
\definecolor{currentfill}{rgb}{0.811226,0.923875,0.614533}%
\pgfsetfillcolor{currentfill}%
\pgfsetlinewidth{0.000000pt}%
\definecolor{currentstroke}{rgb}{0.000000,0.000000,0.000000}%
\pgfsetstrokecolor{currentstroke}%
\pgfsetstrokeopacity{0.000000}%
\pgfsetdash{}{0pt}%
\pgfpathmoveto{\pgfqpoint{4.239379in}{0.190744in}}%
\pgfpathlineto{\pgfqpoint{4.238271in}{0.188357in}}%
\pgfpathlineto{\pgfqpoint{4.239311in}{0.184481in}}%
\pgfpathlineto{\pgfqpoint{4.236648in}{0.184155in}}%
\pgfpathlineto{\pgfqpoint{4.233856in}{0.185950in}}%
\pgfpathlineto{\pgfqpoint{4.234762in}{0.189903in}}%
\pgfpathclose%
\pgfusepath{fill}%
\end{pgfscope}%
\begin{pgfscope}%
\pgfpathrectangle{\pgfqpoint{3.525000in}{0.100000in}}{\pgfqpoint{2.857344in}{1.829167in}}%
\pgfusepath{clip}%
\pgfsetbuttcap%
\pgfsetmiterjoin%
\definecolor{currentfill}{rgb}{0.811226,0.923875,0.614533}%
\pgfsetfillcolor{currentfill}%
\pgfsetlinewidth{0.000000pt}%
\definecolor{currentstroke}{rgb}{0.000000,0.000000,0.000000}%
\pgfsetstrokecolor{currentstroke}%
\pgfsetstrokeopacity{0.000000}%
\pgfsetdash{}{0pt}%
\pgfpathmoveto{\pgfqpoint{4.226966in}{0.191015in}}%
\pgfpathlineto{\pgfqpoint{4.221087in}{0.188378in}}%
\pgfpathlineto{\pgfqpoint{4.219089in}{0.189327in}}%
\pgfpathlineto{\pgfqpoint{4.223098in}{0.192805in}}%
\pgfpathclose%
\pgfusepath{fill}%
\end{pgfscope}%
\begin{pgfscope}%
\pgfpathrectangle{\pgfqpoint{3.525000in}{0.100000in}}{\pgfqpoint{2.857344in}{1.829167in}}%
\pgfusepath{clip}%
\pgfsetbuttcap%
\pgfsetmiterjoin%
\definecolor{currentfill}{rgb}{0.811226,0.923875,0.614533}%
\pgfsetfillcolor{currentfill}%
\pgfsetlinewidth{0.000000pt}%
\definecolor{currentstroke}{rgb}{0.000000,0.000000,0.000000}%
\pgfsetstrokecolor{currentstroke}%
\pgfsetstrokeopacity{0.000000}%
\pgfsetdash{}{0pt}%
\pgfpathmoveto{\pgfqpoint{4.239443in}{0.165293in}}%
\pgfpathlineto{\pgfqpoint{4.238042in}{0.169263in}}%
\pgfpathlineto{\pgfqpoint{4.241006in}{0.170240in}}%
\pgfpathlineto{\pgfqpoint{4.241903in}{0.174521in}}%
\pgfpathlineto{\pgfqpoint{4.245116in}{0.174526in}}%
\pgfpathlineto{\pgfqpoint{4.245102in}{0.177507in}}%
\pgfpathlineto{\pgfqpoint{4.249567in}{0.176802in}}%
\pgfpathlineto{\pgfqpoint{4.250328in}{0.168720in}}%
\pgfpathlineto{\pgfqpoint{4.247738in}{0.163568in}}%
\pgfpathlineto{\pgfqpoint{4.243896in}{0.160350in}}%
\pgfpathclose%
\pgfusepath{fill}%
\end{pgfscope}%
\begin{pgfscope}%
\pgfpathrectangle{\pgfqpoint{3.525000in}{0.100000in}}{\pgfqpoint{2.857344in}{1.829167in}}%
\pgfusepath{clip}%
\pgfsetbuttcap%
\pgfsetmiterjoin%
\definecolor{currentfill}{rgb}{0.811226,0.923875,0.614533}%
\pgfsetfillcolor{currentfill}%
\pgfsetlinewidth{0.000000pt}%
\definecolor{currentstroke}{rgb}{0.000000,0.000000,0.000000}%
\pgfsetstrokecolor{currentstroke}%
\pgfsetstrokeopacity{0.000000}%
\pgfsetdash{}{0pt}%
\pgfpathmoveto{\pgfqpoint{4.237707in}{0.168347in}}%
\pgfpathlineto{\pgfqpoint{4.238973in}{0.164588in}}%
\pgfpathlineto{\pgfqpoint{4.235852in}{0.164131in}}%
\pgfpathclose%
\pgfusepath{fill}%
\end{pgfscope}%
\begin{pgfscope}%
\pgfpathrectangle{\pgfqpoint{3.525000in}{0.100000in}}{\pgfqpoint{2.857344in}{1.829167in}}%
\pgfusepath{clip}%
\pgfsetbuttcap%
\pgfsetmiterjoin%
\definecolor{currentfill}{rgb}{0.811226,0.923875,0.614533}%
\pgfsetfillcolor{currentfill}%
\pgfsetlinewidth{0.000000pt}%
\definecolor{currentstroke}{rgb}{0.000000,0.000000,0.000000}%
\pgfsetstrokecolor{currentstroke}%
\pgfsetstrokeopacity{0.000000}%
\pgfsetdash{}{0pt}%
\pgfpathmoveto{\pgfqpoint{4.240339in}{0.162818in}}%
\pgfpathlineto{\pgfqpoint{4.239904in}{0.158026in}}%
\pgfpathlineto{\pgfqpoint{4.236611in}{0.158423in}}%
\pgfpathlineto{\pgfqpoint{4.237676in}{0.160738in}}%
\pgfpathclose%
\pgfusepath{fill}%
\end{pgfscope}%
\begin{pgfscope}%
\definecolor{textcolor}{rgb}{0.000000,0.000000,0.000000}%
\pgfsetstrokecolor{textcolor}%
\pgfsetfillcolor{textcolor}%
\pgftext[x=4.439350in,y=1.910875in,left,base]{\color{textcolor}\setmainfont{Lato}\rmfamily\fontsize{9.000000}{10.800000}\selectfont Year ending 2019 Q4}%
\end{pgfscope}%
\begin{pgfscope}%
\pgfpathrectangle{\pgfqpoint{2.887899in}{0.169444in}}{\pgfqpoint{0.857203in}{1.280417in}}%
\pgfusepath{clip}%
\pgfsetbuttcap%
\pgfsetmiterjoin%
\definecolor{currentfill}{rgb}{0.619608,0.003922,0.258824}%
\pgfsetfillcolor{currentfill}%
\pgfsetlinewidth{0.000000pt}%
\definecolor{currentstroke}{rgb}{0.000000,0.000000,0.000000}%
\pgfsetstrokecolor{currentstroke}%
\pgfsetstrokeopacity{0.000000}%
\pgfsetdash{}{0pt}%
\pgfpathmoveto{\pgfqpoint{3.136488in}{0.169444in}}%
\pgfpathlineto{\pgfqpoint{3.102200in}{0.169444in}}%
\pgfpathlineto{\pgfqpoint{3.102200in}{0.189451in}}%
\pgfpathlineto{\pgfqpoint{3.136488in}{0.189451in}}%
\pgfpathclose%
\pgfusepath{fill}%
\end{pgfscope}%
\begin{pgfscope}%
\pgfpathrectangle{\pgfqpoint{2.887899in}{0.169444in}}{\pgfqpoint{0.857203in}{1.280417in}}%
\pgfusepath{clip}%
\pgfsetbuttcap%
\pgfsetmiterjoin%
\definecolor{currentfill}{rgb}{0.653441,0.041446,0.266820}%
\pgfsetfillcolor{currentfill}%
\pgfsetlinewidth{0.000000pt}%
\definecolor{currentstroke}{rgb}{0.000000,0.000000,0.000000}%
\pgfsetstrokecolor{currentstroke}%
\pgfsetstrokeopacity{0.000000}%
\pgfsetdash{}{0pt}%
\pgfpathmoveto{\pgfqpoint{3.136488in}{0.189451in}}%
\pgfpathlineto{\pgfqpoint{3.102200in}{0.189451in}}%
\pgfpathlineto{\pgfqpoint{3.102200in}{0.209457in}}%
\pgfpathlineto{\pgfqpoint{3.136488in}{0.209457in}}%
\pgfpathclose%
\pgfusepath{fill}%
\end{pgfscope}%
\begin{pgfscope}%
\pgfpathrectangle{\pgfqpoint{2.887899in}{0.169444in}}{\pgfqpoint{0.857203in}{1.280417in}}%
\pgfusepath{clip}%
\pgfsetbuttcap%
\pgfsetmiterjoin%
\definecolor{currentfill}{rgb}{0.687274,0.078970,0.274817}%
\pgfsetfillcolor{currentfill}%
\pgfsetlinewidth{0.000000pt}%
\definecolor{currentstroke}{rgb}{0.000000,0.000000,0.000000}%
\pgfsetstrokecolor{currentstroke}%
\pgfsetstrokeopacity{0.000000}%
\pgfsetdash{}{0pt}%
\pgfpathmoveto{\pgfqpoint{3.136488in}{0.209457in}}%
\pgfpathlineto{\pgfqpoint{3.102200in}{0.209457in}}%
\pgfpathlineto{\pgfqpoint{3.102200in}{0.229464in}}%
\pgfpathlineto{\pgfqpoint{3.136488in}{0.229464in}}%
\pgfpathclose%
\pgfusepath{fill}%
\end{pgfscope}%
\begin{pgfscope}%
\pgfpathrectangle{\pgfqpoint{2.887899in}{0.169444in}}{\pgfqpoint{0.857203in}{1.280417in}}%
\pgfusepath{clip}%
\pgfsetbuttcap%
\pgfsetmiterjoin%
\definecolor{currentfill}{rgb}{0.721107,0.116494,0.282814}%
\pgfsetfillcolor{currentfill}%
\pgfsetlinewidth{0.000000pt}%
\definecolor{currentstroke}{rgb}{0.000000,0.000000,0.000000}%
\pgfsetstrokecolor{currentstroke}%
\pgfsetstrokeopacity{0.000000}%
\pgfsetdash{}{0pt}%
\pgfpathmoveto{\pgfqpoint{3.136488in}{0.229464in}}%
\pgfpathlineto{\pgfqpoint{3.102200in}{0.229464in}}%
\pgfpathlineto{\pgfqpoint{3.102200in}{0.249470in}}%
\pgfpathlineto{\pgfqpoint{3.136488in}{0.249470in}}%
\pgfpathclose%
\pgfusepath{fill}%
\end{pgfscope}%
\begin{pgfscope}%
\pgfpathrectangle{\pgfqpoint{2.887899in}{0.169444in}}{\pgfqpoint{0.857203in}{1.280417in}}%
\pgfusepath{clip}%
\pgfsetbuttcap%
\pgfsetmiterjoin%
\definecolor{currentfill}{rgb}{0.754940,0.154018,0.290811}%
\pgfsetfillcolor{currentfill}%
\pgfsetlinewidth{0.000000pt}%
\definecolor{currentstroke}{rgb}{0.000000,0.000000,0.000000}%
\pgfsetstrokecolor{currentstroke}%
\pgfsetstrokeopacity{0.000000}%
\pgfsetdash{}{0pt}%
\pgfpathmoveto{\pgfqpoint{3.136488in}{0.249470in}}%
\pgfpathlineto{\pgfqpoint{3.102200in}{0.249470in}}%
\pgfpathlineto{\pgfqpoint{3.102200in}{0.269477in}}%
\pgfpathlineto{\pgfqpoint{3.136488in}{0.269477in}}%
\pgfpathclose%
\pgfusepath{fill}%
\end{pgfscope}%
\begin{pgfscope}%
\pgfpathrectangle{\pgfqpoint{2.887899in}{0.169444in}}{\pgfqpoint{0.857203in}{1.280417in}}%
\pgfusepath{clip}%
\pgfsetbuttcap%
\pgfsetmiterjoin%
\definecolor{currentfill}{rgb}{0.788774,0.191542,0.298808}%
\pgfsetfillcolor{currentfill}%
\pgfsetlinewidth{0.000000pt}%
\definecolor{currentstroke}{rgb}{0.000000,0.000000,0.000000}%
\pgfsetstrokecolor{currentstroke}%
\pgfsetstrokeopacity{0.000000}%
\pgfsetdash{}{0pt}%
\pgfpathmoveto{\pgfqpoint{3.136488in}{0.269477in}}%
\pgfpathlineto{\pgfqpoint{3.102200in}{0.269477in}}%
\pgfpathlineto{\pgfqpoint{3.102200in}{0.289484in}}%
\pgfpathlineto{\pgfqpoint{3.136488in}{0.289484in}}%
\pgfpathclose%
\pgfusepath{fill}%
\end{pgfscope}%
\begin{pgfscope}%
\pgfpathrectangle{\pgfqpoint{2.887899in}{0.169444in}}{\pgfqpoint{0.857203in}{1.280417in}}%
\pgfusepath{clip}%
\pgfsetbuttcap%
\pgfsetmiterjoin%
\definecolor{currentfill}{rgb}{0.822607,0.229066,0.306805}%
\pgfsetfillcolor{currentfill}%
\pgfsetlinewidth{0.000000pt}%
\definecolor{currentstroke}{rgb}{0.000000,0.000000,0.000000}%
\pgfsetstrokecolor{currentstroke}%
\pgfsetstrokeopacity{0.000000}%
\pgfsetdash{}{0pt}%
\pgfpathmoveto{\pgfqpoint{3.136488in}{0.289484in}}%
\pgfpathlineto{\pgfqpoint{3.102200in}{0.289484in}}%
\pgfpathlineto{\pgfqpoint{3.102200in}{0.309490in}}%
\pgfpathlineto{\pgfqpoint{3.136488in}{0.309490in}}%
\pgfpathclose%
\pgfusepath{fill}%
\end{pgfscope}%
\begin{pgfscope}%
\pgfpathrectangle{\pgfqpoint{2.887899in}{0.169444in}}{\pgfqpoint{0.857203in}{1.280417in}}%
\pgfusepath{clip}%
\pgfsetbuttcap%
\pgfsetmiterjoin%
\definecolor{currentfill}{rgb}{0.847213,0.261207,0.305190}%
\pgfsetfillcolor{currentfill}%
\pgfsetlinewidth{0.000000pt}%
\definecolor{currentstroke}{rgb}{0.000000,0.000000,0.000000}%
\pgfsetstrokecolor{currentstroke}%
\pgfsetstrokeopacity{0.000000}%
\pgfsetdash{}{0pt}%
\pgfpathmoveto{\pgfqpoint{3.136488in}{0.309490in}}%
\pgfpathlineto{\pgfqpoint{3.102200in}{0.309490in}}%
\pgfpathlineto{\pgfqpoint{3.102200in}{0.329497in}}%
\pgfpathlineto{\pgfqpoint{3.136488in}{0.329497in}}%
\pgfpathclose%
\pgfusepath{fill}%
\end{pgfscope}%
\begin{pgfscope}%
\pgfpathrectangle{\pgfqpoint{2.887899in}{0.169444in}}{\pgfqpoint{0.857203in}{1.280417in}}%
\pgfusepath{clip}%
\pgfsetbuttcap%
\pgfsetmiterjoin%
\definecolor{currentfill}{rgb}{0.866282,0.290119,0.297809}%
\pgfsetfillcolor{currentfill}%
\pgfsetlinewidth{0.000000pt}%
\definecolor{currentstroke}{rgb}{0.000000,0.000000,0.000000}%
\pgfsetstrokecolor{currentstroke}%
\pgfsetstrokeopacity{0.000000}%
\pgfsetdash{}{0pt}%
\pgfpathmoveto{\pgfqpoint{3.136488in}{0.329497in}}%
\pgfpathlineto{\pgfqpoint{3.102200in}{0.329497in}}%
\pgfpathlineto{\pgfqpoint{3.102200in}{0.349503in}}%
\pgfpathlineto{\pgfqpoint{3.136488in}{0.349503in}}%
\pgfpathclose%
\pgfusepath{fill}%
\end{pgfscope}%
\begin{pgfscope}%
\pgfpathrectangle{\pgfqpoint{2.887899in}{0.169444in}}{\pgfqpoint{0.857203in}{1.280417in}}%
\pgfusepath{clip}%
\pgfsetbuttcap%
\pgfsetmiterjoin%
\definecolor{currentfill}{rgb}{0.885352,0.319031,0.290427}%
\pgfsetfillcolor{currentfill}%
\pgfsetlinewidth{0.000000pt}%
\definecolor{currentstroke}{rgb}{0.000000,0.000000,0.000000}%
\pgfsetstrokecolor{currentstroke}%
\pgfsetstrokeopacity{0.000000}%
\pgfsetdash{}{0pt}%
\pgfpathmoveto{\pgfqpoint{3.136488in}{0.349503in}}%
\pgfpathlineto{\pgfqpoint{3.102200in}{0.349503in}}%
\pgfpathlineto{\pgfqpoint{3.102200in}{0.369510in}}%
\pgfpathlineto{\pgfqpoint{3.136488in}{0.369510in}}%
\pgfpathclose%
\pgfusepath{fill}%
\end{pgfscope}%
\begin{pgfscope}%
\pgfpathrectangle{\pgfqpoint{2.887899in}{0.169444in}}{\pgfqpoint{0.857203in}{1.280417in}}%
\pgfusepath{clip}%
\pgfsetbuttcap%
\pgfsetmiterjoin%
\definecolor{currentfill}{rgb}{0.904421,0.347943,0.283045}%
\pgfsetfillcolor{currentfill}%
\pgfsetlinewidth{0.000000pt}%
\definecolor{currentstroke}{rgb}{0.000000,0.000000,0.000000}%
\pgfsetstrokecolor{currentstroke}%
\pgfsetstrokeopacity{0.000000}%
\pgfsetdash{}{0pt}%
\pgfpathmoveto{\pgfqpoint{3.136488in}{0.369510in}}%
\pgfpathlineto{\pgfqpoint{3.102200in}{0.369510in}}%
\pgfpathlineto{\pgfqpoint{3.102200in}{0.389516in}}%
\pgfpathlineto{\pgfqpoint{3.136488in}{0.389516in}}%
\pgfpathclose%
\pgfusepath{fill}%
\end{pgfscope}%
\begin{pgfscope}%
\pgfpathrectangle{\pgfqpoint{2.887899in}{0.169444in}}{\pgfqpoint{0.857203in}{1.280417in}}%
\pgfusepath{clip}%
\pgfsetbuttcap%
\pgfsetmiterjoin%
\definecolor{currentfill}{rgb}{0.923491,0.376855,0.275663}%
\pgfsetfillcolor{currentfill}%
\pgfsetlinewidth{0.000000pt}%
\definecolor{currentstroke}{rgb}{0.000000,0.000000,0.000000}%
\pgfsetstrokecolor{currentstroke}%
\pgfsetstrokeopacity{0.000000}%
\pgfsetdash{}{0pt}%
\pgfpathmoveto{\pgfqpoint{3.136488in}{0.389516in}}%
\pgfpathlineto{\pgfqpoint{3.102200in}{0.389516in}}%
\pgfpathlineto{\pgfqpoint{3.102200in}{0.409523in}}%
\pgfpathlineto{\pgfqpoint{3.136488in}{0.409523in}}%
\pgfpathclose%
\pgfusepath{fill}%
\end{pgfscope}%
\begin{pgfscope}%
\pgfpathrectangle{\pgfqpoint{2.887899in}{0.169444in}}{\pgfqpoint{0.857203in}{1.280417in}}%
\pgfusepath{clip}%
\pgfsetbuttcap%
\pgfsetmiterjoin%
\definecolor{currentfill}{rgb}{0.942561,0.405767,0.268281}%
\pgfsetfillcolor{currentfill}%
\pgfsetlinewidth{0.000000pt}%
\definecolor{currentstroke}{rgb}{0.000000,0.000000,0.000000}%
\pgfsetstrokecolor{currentstroke}%
\pgfsetstrokeopacity{0.000000}%
\pgfsetdash{}{0pt}%
\pgfpathmoveto{\pgfqpoint{3.136488in}{0.409523in}}%
\pgfpathlineto{\pgfqpoint{3.102200in}{0.409523in}}%
\pgfpathlineto{\pgfqpoint{3.102200in}{0.429529in}}%
\pgfpathlineto{\pgfqpoint{3.136488in}{0.429529in}}%
\pgfpathclose%
\pgfusepath{fill}%
\end{pgfscope}%
\begin{pgfscope}%
\pgfpathrectangle{\pgfqpoint{2.887899in}{0.169444in}}{\pgfqpoint{0.857203in}{1.280417in}}%
\pgfusepath{clip}%
\pgfsetbuttcap%
\pgfsetmiterjoin%
\definecolor{currentfill}{rgb}{0.958247,0.437447,0.267359}%
\pgfsetfillcolor{currentfill}%
\pgfsetlinewidth{0.000000pt}%
\definecolor{currentstroke}{rgb}{0.000000,0.000000,0.000000}%
\pgfsetstrokecolor{currentstroke}%
\pgfsetstrokeopacity{0.000000}%
\pgfsetdash{}{0pt}%
\pgfpathmoveto{\pgfqpoint{3.136488in}{0.429529in}}%
\pgfpathlineto{\pgfqpoint{3.102200in}{0.429529in}}%
\pgfpathlineto{\pgfqpoint{3.102200in}{0.449536in}}%
\pgfpathlineto{\pgfqpoint{3.136488in}{0.449536in}}%
\pgfpathclose%
\pgfusepath{fill}%
\end{pgfscope}%
\begin{pgfscope}%
\pgfpathrectangle{\pgfqpoint{2.887899in}{0.169444in}}{\pgfqpoint{0.857203in}{1.280417in}}%
\pgfusepath{clip}%
\pgfsetbuttcap%
\pgfsetmiterjoin%
\definecolor{currentfill}{rgb}{0.963783,0.477432,0.285813}%
\pgfsetfillcolor{currentfill}%
\pgfsetlinewidth{0.000000pt}%
\definecolor{currentstroke}{rgb}{0.000000,0.000000,0.000000}%
\pgfsetstrokecolor{currentstroke}%
\pgfsetstrokeopacity{0.000000}%
\pgfsetdash{}{0pt}%
\pgfpathmoveto{\pgfqpoint{3.136488in}{0.449536in}}%
\pgfpathlineto{\pgfqpoint{3.102200in}{0.449536in}}%
\pgfpathlineto{\pgfqpoint{3.102200in}{0.469542in}}%
\pgfpathlineto{\pgfqpoint{3.136488in}{0.469542in}}%
\pgfpathclose%
\pgfusepath{fill}%
\end{pgfscope}%
\begin{pgfscope}%
\pgfpathrectangle{\pgfqpoint{2.887899in}{0.169444in}}{\pgfqpoint{0.857203in}{1.280417in}}%
\pgfusepath{clip}%
\pgfsetbuttcap%
\pgfsetmiterjoin%
\definecolor{currentfill}{rgb}{0.969319,0.517416,0.304268}%
\pgfsetfillcolor{currentfill}%
\pgfsetlinewidth{0.000000pt}%
\definecolor{currentstroke}{rgb}{0.000000,0.000000,0.000000}%
\pgfsetstrokecolor{currentstroke}%
\pgfsetstrokeopacity{0.000000}%
\pgfsetdash{}{0pt}%
\pgfpathmoveto{\pgfqpoint{3.136488in}{0.469542in}}%
\pgfpathlineto{\pgfqpoint{3.102200in}{0.469542in}}%
\pgfpathlineto{\pgfqpoint{3.102200in}{0.489549in}}%
\pgfpathlineto{\pgfqpoint{3.136488in}{0.489549in}}%
\pgfpathclose%
\pgfusepath{fill}%
\end{pgfscope}%
\begin{pgfscope}%
\pgfpathrectangle{\pgfqpoint{2.887899in}{0.169444in}}{\pgfqpoint{0.857203in}{1.280417in}}%
\pgfusepath{clip}%
\pgfsetbuttcap%
\pgfsetmiterjoin%
\definecolor{currentfill}{rgb}{0.974856,0.557401,0.322722}%
\pgfsetfillcolor{currentfill}%
\pgfsetlinewidth{0.000000pt}%
\definecolor{currentstroke}{rgb}{0.000000,0.000000,0.000000}%
\pgfsetstrokecolor{currentstroke}%
\pgfsetstrokeopacity{0.000000}%
\pgfsetdash{}{0pt}%
\pgfpathmoveto{\pgfqpoint{3.136488in}{0.489549in}}%
\pgfpathlineto{\pgfqpoint{3.102200in}{0.489549in}}%
\pgfpathlineto{\pgfqpoint{3.102200in}{0.509555in}}%
\pgfpathlineto{\pgfqpoint{3.136488in}{0.509555in}}%
\pgfpathclose%
\pgfusepath{fill}%
\end{pgfscope}%
\begin{pgfscope}%
\pgfpathrectangle{\pgfqpoint{2.887899in}{0.169444in}}{\pgfqpoint{0.857203in}{1.280417in}}%
\pgfusepath{clip}%
\pgfsetbuttcap%
\pgfsetmiterjoin%
\definecolor{currentfill}{rgb}{0.980392,0.597386,0.341176}%
\pgfsetfillcolor{currentfill}%
\pgfsetlinewidth{0.000000pt}%
\definecolor{currentstroke}{rgb}{0.000000,0.000000,0.000000}%
\pgfsetstrokecolor{currentstroke}%
\pgfsetstrokeopacity{0.000000}%
\pgfsetdash{}{0pt}%
\pgfpathmoveto{\pgfqpoint{3.136488in}{0.509555in}}%
\pgfpathlineto{\pgfqpoint{3.102200in}{0.509555in}}%
\pgfpathlineto{\pgfqpoint{3.102200in}{0.529562in}}%
\pgfpathlineto{\pgfqpoint{3.136488in}{0.529562in}}%
\pgfpathclose%
\pgfusepath{fill}%
\end{pgfscope}%
\begin{pgfscope}%
\pgfpathrectangle{\pgfqpoint{2.887899in}{0.169444in}}{\pgfqpoint{0.857203in}{1.280417in}}%
\pgfusepath{clip}%
\pgfsetbuttcap%
\pgfsetmiterjoin%
\definecolor{currentfill}{rgb}{0.985928,0.637370,0.359631}%
\pgfsetfillcolor{currentfill}%
\pgfsetlinewidth{0.000000pt}%
\definecolor{currentstroke}{rgb}{0.000000,0.000000,0.000000}%
\pgfsetstrokecolor{currentstroke}%
\pgfsetstrokeopacity{0.000000}%
\pgfsetdash{}{0pt}%
\pgfpathmoveto{\pgfqpoint{3.136488in}{0.529562in}}%
\pgfpathlineto{\pgfqpoint{3.102200in}{0.529562in}}%
\pgfpathlineto{\pgfqpoint{3.102200in}{0.549568in}}%
\pgfpathlineto{\pgfqpoint{3.136488in}{0.549568in}}%
\pgfpathclose%
\pgfusepath{fill}%
\end{pgfscope}%
\begin{pgfscope}%
\pgfpathrectangle{\pgfqpoint{2.887899in}{0.169444in}}{\pgfqpoint{0.857203in}{1.280417in}}%
\pgfusepath{clip}%
\pgfsetbuttcap%
\pgfsetmiterjoin%
\definecolor{currentfill}{rgb}{0.991465,0.677355,0.378085}%
\pgfsetfillcolor{currentfill}%
\pgfsetlinewidth{0.000000pt}%
\definecolor{currentstroke}{rgb}{0.000000,0.000000,0.000000}%
\pgfsetstrokecolor{currentstroke}%
\pgfsetstrokeopacity{0.000000}%
\pgfsetdash{}{0pt}%
\pgfpathmoveto{\pgfqpoint{3.136488in}{0.549568in}}%
\pgfpathlineto{\pgfqpoint{3.102200in}{0.549568in}}%
\pgfpathlineto{\pgfqpoint{3.102200in}{0.569575in}}%
\pgfpathlineto{\pgfqpoint{3.136488in}{0.569575in}}%
\pgfpathclose%
\pgfusepath{fill}%
\end{pgfscope}%
\begin{pgfscope}%
\pgfpathrectangle{\pgfqpoint{2.887899in}{0.169444in}}{\pgfqpoint{0.857203in}{1.280417in}}%
\pgfusepath{clip}%
\pgfsetbuttcap%
\pgfsetmiterjoin%
\definecolor{currentfill}{rgb}{0.992695,0.709266,0.402999}%
\pgfsetfillcolor{currentfill}%
\pgfsetlinewidth{0.000000pt}%
\definecolor{currentstroke}{rgb}{0.000000,0.000000,0.000000}%
\pgfsetstrokecolor{currentstroke}%
\pgfsetstrokeopacity{0.000000}%
\pgfsetdash{}{0pt}%
\pgfpathmoveto{\pgfqpoint{3.136488in}{0.569575in}}%
\pgfpathlineto{\pgfqpoint{3.102200in}{0.569575in}}%
\pgfpathlineto{\pgfqpoint{3.102200in}{0.589581in}}%
\pgfpathlineto{\pgfqpoint{3.136488in}{0.589581in}}%
\pgfpathclose%
\pgfusepath{fill}%
\end{pgfscope}%
\begin{pgfscope}%
\pgfpathrectangle{\pgfqpoint{2.887899in}{0.169444in}}{\pgfqpoint{0.857203in}{1.280417in}}%
\pgfusepath{clip}%
\pgfsetbuttcap%
\pgfsetmiterjoin%
\definecolor{currentfill}{rgb}{0.993310,0.740023,0.428835}%
\pgfsetfillcolor{currentfill}%
\pgfsetlinewidth{0.000000pt}%
\definecolor{currentstroke}{rgb}{0.000000,0.000000,0.000000}%
\pgfsetstrokecolor{currentstroke}%
\pgfsetstrokeopacity{0.000000}%
\pgfsetdash{}{0pt}%
\pgfpathmoveto{\pgfqpoint{3.136488in}{0.589581in}}%
\pgfpathlineto{\pgfqpoint{3.102200in}{0.589581in}}%
\pgfpathlineto{\pgfqpoint{3.102200in}{0.609588in}}%
\pgfpathlineto{\pgfqpoint{3.136488in}{0.609588in}}%
\pgfpathclose%
\pgfusepath{fill}%
\end{pgfscope}%
\begin{pgfscope}%
\pgfpathrectangle{\pgfqpoint{2.887899in}{0.169444in}}{\pgfqpoint{0.857203in}{1.280417in}}%
\pgfusepath{clip}%
\pgfsetbuttcap%
\pgfsetmiterjoin%
\definecolor{currentfill}{rgb}{0.993925,0.770780,0.454671}%
\pgfsetfillcolor{currentfill}%
\pgfsetlinewidth{0.000000pt}%
\definecolor{currentstroke}{rgb}{0.000000,0.000000,0.000000}%
\pgfsetstrokecolor{currentstroke}%
\pgfsetstrokeopacity{0.000000}%
\pgfsetdash{}{0pt}%
\pgfpathmoveto{\pgfqpoint{3.136488in}{0.609588in}}%
\pgfpathlineto{\pgfqpoint{3.102200in}{0.609588in}}%
\pgfpathlineto{\pgfqpoint{3.102200in}{0.629594in}}%
\pgfpathlineto{\pgfqpoint{3.136488in}{0.629594in}}%
\pgfpathclose%
\pgfusepath{fill}%
\end{pgfscope}%
\begin{pgfscope}%
\pgfpathrectangle{\pgfqpoint{2.887899in}{0.169444in}}{\pgfqpoint{0.857203in}{1.280417in}}%
\pgfusepath{clip}%
\pgfsetbuttcap%
\pgfsetmiterjoin%
\definecolor{currentfill}{rgb}{0.994541,0.801538,0.480507}%
\pgfsetfillcolor{currentfill}%
\pgfsetlinewidth{0.000000pt}%
\definecolor{currentstroke}{rgb}{0.000000,0.000000,0.000000}%
\pgfsetstrokecolor{currentstroke}%
\pgfsetstrokeopacity{0.000000}%
\pgfsetdash{}{0pt}%
\pgfpathmoveto{\pgfqpoint{3.136488in}{0.629594in}}%
\pgfpathlineto{\pgfqpoint{3.102200in}{0.629594in}}%
\pgfpathlineto{\pgfqpoint{3.102200in}{0.649601in}}%
\pgfpathlineto{\pgfqpoint{3.136488in}{0.649601in}}%
\pgfpathclose%
\pgfusepath{fill}%
\end{pgfscope}%
\begin{pgfscope}%
\pgfpathrectangle{\pgfqpoint{2.887899in}{0.169444in}}{\pgfqpoint{0.857203in}{1.280417in}}%
\pgfusepath{clip}%
\pgfsetbuttcap%
\pgfsetmiterjoin%
\definecolor{currentfill}{rgb}{0.995156,0.832295,0.506344}%
\pgfsetfillcolor{currentfill}%
\pgfsetlinewidth{0.000000pt}%
\definecolor{currentstroke}{rgb}{0.000000,0.000000,0.000000}%
\pgfsetstrokecolor{currentstroke}%
\pgfsetstrokeopacity{0.000000}%
\pgfsetdash{}{0pt}%
\pgfpathmoveto{\pgfqpoint{3.136488in}{0.649601in}}%
\pgfpathlineto{\pgfqpoint{3.102200in}{0.649601in}}%
\pgfpathlineto{\pgfqpoint{3.102200in}{0.669607in}}%
\pgfpathlineto{\pgfqpoint{3.136488in}{0.669607in}}%
\pgfpathclose%
\pgfusepath{fill}%
\end{pgfscope}%
\begin{pgfscope}%
\pgfpathrectangle{\pgfqpoint{2.887899in}{0.169444in}}{\pgfqpoint{0.857203in}{1.280417in}}%
\pgfusepath{clip}%
\pgfsetbuttcap%
\pgfsetmiterjoin%
\definecolor{currentfill}{rgb}{0.995771,0.863053,0.532180}%
\pgfsetfillcolor{currentfill}%
\pgfsetlinewidth{0.000000pt}%
\definecolor{currentstroke}{rgb}{0.000000,0.000000,0.000000}%
\pgfsetstrokecolor{currentstroke}%
\pgfsetstrokeopacity{0.000000}%
\pgfsetdash{}{0pt}%
\pgfpathmoveto{\pgfqpoint{3.136488in}{0.669607in}}%
\pgfpathlineto{\pgfqpoint{3.102200in}{0.669607in}}%
\pgfpathlineto{\pgfqpoint{3.102200in}{0.689614in}}%
\pgfpathlineto{\pgfqpoint{3.136488in}{0.689614in}}%
\pgfpathclose%
\pgfusepath{fill}%
\end{pgfscope}%
\begin{pgfscope}%
\pgfpathrectangle{\pgfqpoint{2.887899in}{0.169444in}}{\pgfqpoint{0.857203in}{1.280417in}}%
\pgfusepath{clip}%
\pgfsetbuttcap%
\pgfsetmiterjoin%
\definecolor{currentfill}{rgb}{0.996386,0.887966,0.561092}%
\pgfsetfillcolor{currentfill}%
\pgfsetlinewidth{0.000000pt}%
\definecolor{currentstroke}{rgb}{0.000000,0.000000,0.000000}%
\pgfsetstrokecolor{currentstroke}%
\pgfsetstrokeopacity{0.000000}%
\pgfsetdash{}{0pt}%
\pgfpathmoveto{\pgfqpoint{3.136488in}{0.689614in}}%
\pgfpathlineto{\pgfqpoint{3.102200in}{0.689614in}}%
\pgfpathlineto{\pgfqpoint{3.102200in}{0.709620in}}%
\pgfpathlineto{\pgfqpoint{3.136488in}{0.709620in}}%
\pgfpathclose%
\pgfusepath{fill}%
\end{pgfscope}%
\begin{pgfscope}%
\pgfpathrectangle{\pgfqpoint{2.887899in}{0.169444in}}{\pgfqpoint{0.857203in}{1.280417in}}%
\pgfusepath{clip}%
\pgfsetbuttcap%
\pgfsetmiterjoin%
\definecolor{currentfill}{rgb}{0.997001,0.907036,0.593080}%
\pgfsetfillcolor{currentfill}%
\pgfsetlinewidth{0.000000pt}%
\definecolor{currentstroke}{rgb}{0.000000,0.000000,0.000000}%
\pgfsetstrokecolor{currentstroke}%
\pgfsetstrokeopacity{0.000000}%
\pgfsetdash{}{0pt}%
\pgfpathmoveto{\pgfqpoint{3.136488in}{0.709620in}}%
\pgfpathlineto{\pgfqpoint{3.102200in}{0.709620in}}%
\pgfpathlineto{\pgfqpoint{3.102200in}{0.729627in}}%
\pgfpathlineto{\pgfqpoint{3.136488in}{0.729627in}}%
\pgfpathclose%
\pgfusepath{fill}%
\end{pgfscope}%
\begin{pgfscope}%
\pgfpathrectangle{\pgfqpoint{2.887899in}{0.169444in}}{\pgfqpoint{0.857203in}{1.280417in}}%
\pgfusepath{clip}%
\pgfsetbuttcap%
\pgfsetmiterjoin%
\definecolor{currentfill}{rgb}{0.997616,0.926105,0.625067}%
\pgfsetfillcolor{currentfill}%
\pgfsetlinewidth{0.000000pt}%
\definecolor{currentstroke}{rgb}{0.000000,0.000000,0.000000}%
\pgfsetstrokecolor{currentstroke}%
\pgfsetstrokeopacity{0.000000}%
\pgfsetdash{}{0pt}%
\pgfpathmoveto{\pgfqpoint{3.136488in}{0.729627in}}%
\pgfpathlineto{\pgfqpoint{3.102200in}{0.729627in}}%
\pgfpathlineto{\pgfqpoint{3.102200in}{0.749633in}}%
\pgfpathlineto{\pgfqpoint{3.136488in}{0.749633in}}%
\pgfpathclose%
\pgfusepath{fill}%
\end{pgfscope}%
\begin{pgfscope}%
\pgfpathrectangle{\pgfqpoint{2.887899in}{0.169444in}}{\pgfqpoint{0.857203in}{1.280417in}}%
\pgfusepath{clip}%
\pgfsetbuttcap%
\pgfsetmiterjoin%
\definecolor{currentfill}{rgb}{0.998231,0.945175,0.657055}%
\pgfsetfillcolor{currentfill}%
\pgfsetlinewidth{0.000000pt}%
\definecolor{currentstroke}{rgb}{0.000000,0.000000,0.000000}%
\pgfsetstrokecolor{currentstroke}%
\pgfsetstrokeopacity{0.000000}%
\pgfsetdash{}{0pt}%
\pgfpathmoveto{\pgfqpoint{3.136488in}{0.749633in}}%
\pgfpathlineto{\pgfqpoint{3.102200in}{0.749633in}}%
\pgfpathlineto{\pgfqpoint{3.102200in}{0.769640in}}%
\pgfpathlineto{\pgfqpoint{3.136488in}{0.769640in}}%
\pgfpathclose%
\pgfusepath{fill}%
\end{pgfscope}%
\begin{pgfscope}%
\pgfpathrectangle{\pgfqpoint{2.887899in}{0.169444in}}{\pgfqpoint{0.857203in}{1.280417in}}%
\pgfusepath{clip}%
\pgfsetbuttcap%
\pgfsetmiterjoin%
\definecolor{currentfill}{rgb}{0.998847,0.964245,0.689043}%
\pgfsetfillcolor{currentfill}%
\pgfsetlinewidth{0.000000pt}%
\definecolor{currentstroke}{rgb}{0.000000,0.000000,0.000000}%
\pgfsetstrokecolor{currentstroke}%
\pgfsetstrokeopacity{0.000000}%
\pgfsetdash{}{0pt}%
\pgfpathmoveto{\pgfqpoint{3.136488in}{0.769640in}}%
\pgfpathlineto{\pgfqpoint{3.102200in}{0.769640in}}%
\pgfpathlineto{\pgfqpoint{3.102200in}{0.789646in}}%
\pgfpathlineto{\pgfqpoint{3.136488in}{0.789646in}}%
\pgfpathclose%
\pgfusepath{fill}%
\end{pgfscope}%
\begin{pgfscope}%
\pgfpathrectangle{\pgfqpoint{2.887899in}{0.169444in}}{\pgfqpoint{0.857203in}{1.280417in}}%
\pgfusepath{clip}%
\pgfsetbuttcap%
\pgfsetmiterjoin%
\definecolor{currentfill}{rgb}{0.999462,0.983314,0.721030}%
\pgfsetfillcolor{currentfill}%
\pgfsetlinewidth{0.000000pt}%
\definecolor{currentstroke}{rgb}{0.000000,0.000000,0.000000}%
\pgfsetstrokecolor{currentstroke}%
\pgfsetstrokeopacity{0.000000}%
\pgfsetdash{}{0pt}%
\pgfpathmoveto{\pgfqpoint{3.136488in}{0.789646in}}%
\pgfpathlineto{\pgfqpoint{3.102200in}{0.789646in}}%
\pgfpathlineto{\pgfqpoint{3.102200in}{0.809653in}}%
\pgfpathlineto{\pgfqpoint{3.136488in}{0.809653in}}%
\pgfpathclose%
\pgfusepath{fill}%
\end{pgfscope}%
\begin{pgfscope}%
\pgfpathrectangle{\pgfqpoint{2.887899in}{0.169444in}}{\pgfqpoint{0.857203in}{1.280417in}}%
\pgfusepath{clip}%
\pgfsetbuttcap%
\pgfsetmiterjoin%
\definecolor{currentfill}{rgb}{0.998078,0.999231,0.746021}%
\pgfsetfillcolor{currentfill}%
\pgfsetlinewidth{0.000000pt}%
\definecolor{currentstroke}{rgb}{0.000000,0.000000,0.000000}%
\pgfsetstrokecolor{currentstroke}%
\pgfsetstrokeopacity{0.000000}%
\pgfsetdash{}{0pt}%
\pgfpathmoveto{\pgfqpoint{3.136488in}{0.809653in}}%
\pgfpathlineto{\pgfqpoint{3.102200in}{0.809653in}}%
\pgfpathlineto{\pgfqpoint{3.102200in}{0.829659in}}%
\pgfpathlineto{\pgfqpoint{3.136488in}{0.829659in}}%
\pgfpathclose%
\pgfusepath{fill}%
\end{pgfscope}%
\begin{pgfscope}%
\pgfpathrectangle{\pgfqpoint{2.887899in}{0.169444in}}{\pgfqpoint{0.857203in}{1.280417in}}%
\pgfusepath{clip}%
\pgfsetbuttcap%
\pgfsetmiterjoin%
\definecolor{currentfill}{rgb}{0.982699,0.993080,0.722030}%
\pgfsetfillcolor{currentfill}%
\pgfsetlinewidth{0.000000pt}%
\definecolor{currentstroke}{rgb}{0.000000,0.000000,0.000000}%
\pgfsetstrokecolor{currentstroke}%
\pgfsetstrokeopacity{0.000000}%
\pgfsetdash{}{0pt}%
\pgfpathmoveto{\pgfqpoint{3.136488in}{0.829659in}}%
\pgfpathlineto{\pgfqpoint{3.102200in}{0.829659in}}%
\pgfpathlineto{\pgfqpoint{3.102200in}{0.849666in}}%
\pgfpathlineto{\pgfqpoint{3.136488in}{0.849666in}}%
\pgfpathclose%
\pgfusepath{fill}%
\end{pgfscope}%
\begin{pgfscope}%
\pgfpathrectangle{\pgfqpoint{2.887899in}{0.169444in}}{\pgfqpoint{0.857203in}{1.280417in}}%
\pgfusepath{clip}%
\pgfsetbuttcap%
\pgfsetmiterjoin%
\definecolor{currentfill}{rgb}{0.967320,0.986928,0.698039}%
\pgfsetfillcolor{currentfill}%
\pgfsetlinewidth{0.000000pt}%
\definecolor{currentstroke}{rgb}{0.000000,0.000000,0.000000}%
\pgfsetstrokecolor{currentstroke}%
\pgfsetstrokeopacity{0.000000}%
\pgfsetdash{}{0pt}%
\pgfpathmoveto{\pgfqpoint{3.136488in}{0.849666in}}%
\pgfpathlineto{\pgfqpoint{3.102200in}{0.849666in}}%
\pgfpathlineto{\pgfqpoint{3.102200in}{0.869672in}}%
\pgfpathlineto{\pgfqpoint{3.136488in}{0.869672in}}%
\pgfpathclose%
\pgfusepath{fill}%
\end{pgfscope}%
\begin{pgfscope}%
\pgfpathrectangle{\pgfqpoint{2.887899in}{0.169444in}}{\pgfqpoint{0.857203in}{1.280417in}}%
\pgfusepath{clip}%
\pgfsetbuttcap%
\pgfsetmiterjoin%
\definecolor{currentfill}{rgb}{0.951942,0.980777,0.674048}%
\pgfsetfillcolor{currentfill}%
\pgfsetlinewidth{0.000000pt}%
\definecolor{currentstroke}{rgb}{0.000000,0.000000,0.000000}%
\pgfsetstrokecolor{currentstroke}%
\pgfsetstrokeopacity{0.000000}%
\pgfsetdash{}{0pt}%
\pgfpathmoveto{\pgfqpoint{3.136488in}{0.869672in}}%
\pgfpathlineto{\pgfqpoint{3.102200in}{0.869672in}}%
\pgfpathlineto{\pgfqpoint{3.102200in}{0.889679in}}%
\pgfpathlineto{\pgfqpoint{3.136488in}{0.889679in}}%
\pgfpathclose%
\pgfusepath{fill}%
\end{pgfscope}%
\begin{pgfscope}%
\pgfpathrectangle{\pgfqpoint{2.887899in}{0.169444in}}{\pgfqpoint{0.857203in}{1.280417in}}%
\pgfusepath{clip}%
\pgfsetbuttcap%
\pgfsetmiterjoin%
\definecolor{currentfill}{rgb}{0.936563,0.974625,0.650058}%
\pgfsetfillcolor{currentfill}%
\pgfsetlinewidth{0.000000pt}%
\definecolor{currentstroke}{rgb}{0.000000,0.000000,0.000000}%
\pgfsetstrokecolor{currentstroke}%
\pgfsetstrokeopacity{0.000000}%
\pgfsetdash{}{0pt}%
\pgfpathmoveto{\pgfqpoint{3.136488in}{0.889679in}}%
\pgfpathlineto{\pgfqpoint{3.102200in}{0.889679in}}%
\pgfpathlineto{\pgfqpoint{3.102200in}{0.909685in}}%
\pgfpathlineto{\pgfqpoint{3.136488in}{0.909685in}}%
\pgfpathclose%
\pgfusepath{fill}%
\end{pgfscope}%
\begin{pgfscope}%
\pgfpathrectangle{\pgfqpoint{2.887899in}{0.169444in}}{\pgfqpoint{0.857203in}{1.280417in}}%
\pgfusepath{clip}%
\pgfsetbuttcap%
\pgfsetmiterjoin%
\definecolor{currentfill}{rgb}{0.921184,0.968474,0.626067}%
\pgfsetfillcolor{currentfill}%
\pgfsetlinewidth{0.000000pt}%
\definecolor{currentstroke}{rgb}{0.000000,0.000000,0.000000}%
\pgfsetstrokecolor{currentstroke}%
\pgfsetstrokeopacity{0.000000}%
\pgfsetdash{}{0pt}%
\pgfpathmoveto{\pgfqpoint{3.136488in}{0.909685in}}%
\pgfpathlineto{\pgfqpoint{3.102200in}{0.909685in}}%
\pgfpathlineto{\pgfqpoint{3.102200in}{0.929692in}}%
\pgfpathlineto{\pgfqpoint{3.136488in}{0.929692in}}%
\pgfpathclose%
\pgfusepath{fill}%
\end{pgfscope}%
\begin{pgfscope}%
\pgfpathrectangle{\pgfqpoint{2.887899in}{0.169444in}}{\pgfqpoint{0.857203in}{1.280417in}}%
\pgfusepath{clip}%
\pgfsetbuttcap%
\pgfsetmiterjoin%
\definecolor{currentfill}{rgb}{0.905805,0.962322,0.602076}%
\pgfsetfillcolor{currentfill}%
\pgfsetlinewidth{0.000000pt}%
\definecolor{currentstroke}{rgb}{0.000000,0.000000,0.000000}%
\pgfsetstrokecolor{currentstroke}%
\pgfsetstrokeopacity{0.000000}%
\pgfsetdash{}{0pt}%
\pgfpathmoveto{\pgfqpoint{3.136488in}{0.929692in}}%
\pgfpathlineto{\pgfqpoint{3.102200in}{0.929692in}}%
\pgfpathlineto{\pgfqpoint{3.102200in}{0.949698in}}%
\pgfpathlineto{\pgfqpoint{3.136488in}{0.949698in}}%
\pgfpathclose%
\pgfusepath{fill}%
\end{pgfscope}%
\begin{pgfscope}%
\pgfpathrectangle{\pgfqpoint{2.887899in}{0.169444in}}{\pgfqpoint{0.857203in}{1.280417in}}%
\pgfusepath{clip}%
\pgfsetbuttcap%
\pgfsetmiterjoin%
\definecolor{currentfill}{rgb}{0.874740,0.949712,0.601615}%
\pgfsetfillcolor{currentfill}%
\pgfsetlinewidth{0.000000pt}%
\definecolor{currentstroke}{rgb}{0.000000,0.000000,0.000000}%
\pgfsetstrokecolor{currentstroke}%
\pgfsetstrokeopacity{0.000000}%
\pgfsetdash{}{0pt}%
\pgfpathmoveto{\pgfqpoint{3.136488in}{0.949698in}}%
\pgfpathlineto{\pgfqpoint{3.102200in}{0.949698in}}%
\pgfpathlineto{\pgfqpoint{3.102200in}{0.969705in}}%
\pgfpathlineto{\pgfqpoint{3.136488in}{0.969705in}}%
\pgfpathclose%
\pgfusepath{fill}%
\end{pgfscope}%
\begin{pgfscope}%
\pgfpathrectangle{\pgfqpoint{2.887899in}{0.169444in}}{\pgfqpoint{0.857203in}{1.280417in}}%
\pgfusepath{clip}%
\pgfsetbuttcap%
\pgfsetmiterjoin%
\definecolor{currentfill}{rgb}{0.838447,0.934948,0.608997}%
\pgfsetfillcolor{currentfill}%
\pgfsetlinewidth{0.000000pt}%
\definecolor{currentstroke}{rgb}{0.000000,0.000000,0.000000}%
\pgfsetstrokecolor{currentstroke}%
\pgfsetstrokeopacity{0.000000}%
\pgfsetdash{}{0pt}%
\pgfpathmoveto{\pgfqpoint{3.136488in}{0.969705in}}%
\pgfpathlineto{\pgfqpoint{3.102200in}{0.969705in}}%
\pgfpathlineto{\pgfqpoint{3.102200in}{0.989711in}}%
\pgfpathlineto{\pgfqpoint{3.136488in}{0.989711in}}%
\pgfpathclose%
\pgfusepath{fill}%
\end{pgfscope}%
\begin{pgfscope}%
\pgfpathrectangle{\pgfqpoint{2.887899in}{0.169444in}}{\pgfqpoint{0.857203in}{1.280417in}}%
\pgfusepath{clip}%
\pgfsetbuttcap%
\pgfsetmiterjoin%
\definecolor{currentfill}{rgb}{0.802153,0.920185,0.616378}%
\pgfsetfillcolor{currentfill}%
\pgfsetlinewidth{0.000000pt}%
\definecolor{currentstroke}{rgb}{0.000000,0.000000,0.000000}%
\pgfsetstrokecolor{currentstroke}%
\pgfsetstrokeopacity{0.000000}%
\pgfsetdash{}{0pt}%
\pgfpathmoveto{\pgfqpoint{3.136488in}{0.989711in}}%
\pgfpathlineto{\pgfqpoint{3.102200in}{0.989711in}}%
\pgfpathlineto{\pgfqpoint{3.102200in}{1.009718in}}%
\pgfpathlineto{\pgfqpoint{3.136488in}{1.009718in}}%
\pgfpathclose%
\pgfusepath{fill}%
\end{pgfscope}%
\begin{pgfscope}%
\pgfpathrectangle{\pgfqpoint{2.887899in}{0.169444in}}{\pgfqpoint{0.857203in}{1.280417in}}%
\pgfusepath{clip}%
\pgfsetbuttcap%
\pgfsetmiterjoin%
\definecolor{currentfill}{rgb}{0.765859,0.905421,0.623760}%
\pgfsetfillcolor{currentfill}%
\pgfsetlinewidth{0.000000pt}%
\definecolor{currentstroke}{rgb}{0.000000,0.000000,0.000000}%
\pgfsetstrokecolor{currentstroke}%
\pgfsetstrokeopacity{0.000000}%
\pgfsetdash{}{0pt}%
\pgfpathmoveto{\pgfqpoint{3.136488in}{1.009718in}}%
\pgfpathlineto{\pgfqpoint{3.102200in}{1.009718in}}%
\pgfpathlineto{\pgfqpoint{3.102200in}{1.029724in}}%
\pgfpathlineto{\pgfqpoint{3.136488in}{1.029724in}}%
\pgfpathclose%
\pgfusepath{fill}%
\end{pgfscope}%
\begin{pgfscope}%
\pgfpathrectangle{\pgfqpoint{2.887899in}{0.169444in}}{\pgfqpoint{0.857203in}{1.280417in}}%
\pgfusepath{clip}%
\pgfsetbuttcap%
\pgfsetmiterjoin%
\definecolor{currentfill}{rgb}{0.729566,0.890657,0.631142}%
\pgfsetfillcolor{currentfill}%
\pgfsetlinewidth{0.000000pt}%
\definecolor{currentstroke}{rgb}{0.000000,0.000000,0.000000}%
\pgfsetstrokecolor{currentstroke}%
\pgfsetstrokeopacity{0.000000}%
\pgfsetdash{}{0pt}%
\pgfpathmoveto{\pgfqpoint{3.136488in}{1.029724in}}%
\pgfpathlineto{\pgfqpoint{3.102200in}{1.029724in}}%
\pgfpathlineto{\pgfqpoint{3.102200in}{1.049731in}}%
\pgfpathlineto{\pgfqpoint{3.136488in}{1.049731in}}%
\pgfpathclose%
\pgfusepath{fill}%
\end{pgfscope}%
\begin{pgfscope}%
\pgfpathrectangle{\pgfqpoint{2.887899in}{0.169444in}}{\pgfqpoint{0.857203in}{1.280417in}}%
\pgfusepath{clip}%
\pgfsetbuttcap%
\pgfsetmiterjoin%
\definecolor{currentfill}{rgb}{0.693272,0.875894,0.638524}%
\pgfsetfillcolor{currentfill}%
\pgfsetlinewidth{0.000000pt}%
\definecolor{currentstroke}{rgb}{0.000000,0.000000,0.000000}%
\pgfsetstrokecolor{currentstroke}%
\pgfsetstrokeopacity{0.000000}%
\pgfsetdash{}{0pt}%
\pgfpathmoveto{\pgfqpoint{3.136488in}{1.049731in}}%
\pgfpathlineto{\pgfqpoint{3.102200in}{1.049731in}}%
\pgfpathlineto{\pgfqpoint{3.102200in}{1.069737in}}%
\pgfpathlineto{\pgfqpoint{3.136488in}{1.069737in}}%
\pgfpathclose%
\pgfusepath{fill}%
\end{pgfscope}%
\begin{pgfscope}%
\pgfpathrectangle{\pgfqpoint{2.887899in}{0.169444in}}{\pgfqpoint{0.857203in}{1.280417in}}%
\pgfusepath{clip}%
\pgfsetbuttcap%
\pgfsetmiterjoin%
\definecolor{currentfill}{rgb}{0.654671,0.860438,0.643368}%
\pgfsetfillcolor{currentfill}%
\pgfsetlinewidth{0.000000pt}%
\definecolor{currentstroke}{rgb}{0.000000,0.000000,0.000000}%
\pgfsetstrokecolor{currentstroke}%
\pgfsetstrokeopacity{0.000000}%
\pgfsetdash{}{0pt}%
\pgfpathmoveto{\pgfqpoint{3.136488in}{1.069737in}}%
\pgfpathlineto{\pgfqpoint{3.102200in}{1.069737in}}%
\pgfpathlineto{\pgfqpoint{3.102200in}{1.089744in}}%
\pgfpathlineto{\pgfqpoint{3.136488in}{1.089744in}}%
\pgfpathclose%
\pgfusepath{fill}%
\end{pgfscope}%
\begin{pgfscope}%
\pgfpathrectangle{\pgfqpoint{2.887899in}{0.169444in}}{\pgfqpoint{0.857203in}{1.280417in}}%
\pgfusepath{clip}%
\pgfsetbuttcap%
\pgfsetmiterjoin%
\definecolor{currentfill}{rgb}{0.612226,0.843829,0.643983}%
\pgfsetfillcolor{currentfill}%
\pgfsetlinewidth{0.000000pt}%
\definecolor{currentstroke}{rgb}{0.000000,0.000000,0.000000}%
\pgfsetstrokecolor{currentstroke}%
\pgfsetstrokeopacity{0.000000}%
\pgfsetdash{}{0pt}%
\pgfpathmoveto{\pgfqpoint{3.136488in}{1.089744in}}%
\pgfpathlineto{\pgfqpoint{3.102200in}{1.089744in}}%
\pgfpathlineto{\pgfqpoint{3.102200in}{1.109750in}}%
\pgfpathlineto{\pgfqpoint{3.136488in}{1.109750in}}%
\pgfpathclose%
\pgfusepath{fill}%
\end{pgfscope}%
\begin{pgfscope}%
\pgfpathrectangle{\pgfqpoint{2.887899in}{0.169444in}}{\pgfqpoint{0.857203in}{1.280417in}}%
\pgfusepath{clip}%
\pgfsetbuttcap%
\pgfsetmiterjoin%
\definecolor{currentfill}{rgb}{0.569781,0.827220,0.644598}%
\pgfsetfillcolor{currentfill}%
\pgfsetlinewidth{0.000000pt}%
\definecolor{currentstroke}{rgb}{0.000000,0.000000,0.000000}%
\pgfsetstrokecolor{currentstroke}%
\pgfsetstrokeopacity{0.000000}%
\pgfsetdash{}{0pt}%
\pgfpathmoveto{\pgfqpoint{3.136488in}{1.109750in}}%
\pgfpathlineto{\pgfqpoint{3.102200in}{1.109750in}}%
\pgfpathlineto{\pgfqpoint{3.102200in}{1.129757in}}%
\pgfpathlineto{\pgfqpoint{3.136488in}{1.129757in}}%
\pgfpathclose%
\pgfusepath{fill}%
\end{pgfscope}%
\begin{pgfscope}%
\pgfpathrectangle{\pgfqpoint{2.887899in}{0.169444in}}{\pgfqpoint{0.857203in}{1.280417in}}%
\pgfusepath{clip}%
\pgfsetbuttcap%
\pgfsetmiterjoin%
\definecolor{currentfill}{rgb}{0.527336,0.810611,0.645213}%
\pgfsetfillcolor{currentfill}%
\pgfsetlinewidth{0.000000pt}%
\definecolor{currentstroke}{rgb}{0.000000,0.000000,0.000000}%
\pgfsetstrokecolor{currentstroke}%
\pgfsetstrokeopacity{0.000000}%
\pgfsetdash{}{0pt}%
\pgfpathmoveto{\pgfqpoint{3.136488in}{1.129757in}}%
\pgfpathlineto{\pgfqpoint{3.102200in}{1.129757in}}%
\pgfpathlineto{\pgfqpoint{3.102200in}{1.149763in}}%
\pgfpathlineto{\pgfqpoint{3.136488in}{1.149763in}}%
\pgfpathclose%
\pgfusepath{fill}%
\end{pgfscope}%
\begin{pgfscope}%
\pgfpathrectangle{\pgfqpoint{2.887899in}{0.169444in}}{\pgfqpoint{0.857203in}{1.280417in}}%
\pgfusepath{clip}%
\pgfsetbuttcap%
\pgfsetmiterjoin%
\definecolor{currentfill}{rgb}{0.484890,0.794002,0.645829}%
\pgfsetfillcolor{currentfill}%
\pgfsetlinewidth{0.000000pt}%
\definecolor{currentstroke}{rgb}{0.000000,0.000000,0.000000}%
\pgfsetstrokecolor{currentstroke}%
\pgfsetstrokeopacity{0.000000}%
\pgfsetdash{}{0pt}%
\pgfpathmoveto{\pgfqpoint{3.136488in}{1.149763in}}%
\pgfpathlineto{\pgfqpoint{3.102200in}{1.149763in}}%
\pgfpathlineto{\pgfqpoint{3.102200in}{1.169770in}}%
\pgfpathlineto{\pgfqpoint{3.136488in}{1.169770in}}%
\pgfpathclose%
\pgfusepath{fill}%
\end{pgfscope}%
\begin{pgfscope}%
\pgfpathrectangle{\pgfqpoint{2.887899in}{0.169444in}}{\pgfqpoint{0.857203in}{1.280417in}}%
\pgfusepath{clip}%
\pgfsetbuttcap%
\pgfsetmiterjoin%
\definecolor{currentfill}{rgb}{0.442445,0.777393,0.646444}%
\pgfsetfillcolor{currentfill}%
\pgfsetlinewidth{0.000000pt}%
\definecolor{currentstroke}{rgb}{0.000000,0.000000,0.000000}%
\pgfsetstrokecolor{currentstroke}%
\pgfsetstrokeopacity{0.000000}%
\pgfsetdash{}{0pt}%
\pgfpathmoveto{\pgfqpoint{3.136488in}{1.169770in}}%
\pgfpathlineto{\pgfqpoint{3.102200in}{1.169770in}}%
\pgfpathlineto{\pgfqpoint{3.102200in}{1.189776in}}%
\pgfpathlineto{\pgfqpoint{3.136488in}{1.189776in}}%
\pgfpathclose%
\pgfusepath{fill}%
\end{pgfscope}%
\begin{pgfscope}%
\pgfpathrectangle{\pgfqpoint{2.887899in}{0.169444in}}{\pgfqpoint{0.857203in}{1.280417in}}%
\pgfusepath{clip}%
\pgfsetbuttcap%
\pgfsetmiterjoin%
\definecolor{currentfill}{rgb}{0.400000,0.760784,0.647059}%
\pgfsetfillcolor{currentfill}%
\pgfsetlinewidth{0.000000pt}%
\definecolor{currentstroke}{rgb}{0.000000,0.000000,0.000000}%
\pgfsetstrokecolor{currentstroke}%
\pgfsetstrokeopacity{0.000000}%
\pgfsetdash{}{0pt}%
\pgfpathmoveto{\pgfqpoint{3.136488in}{1.189776in}}%
\pgfpathlineto{\pgfqpoint{3.102200in}{1.189776in}}%
\pgfpathlineto{\pgfqpoint{3.102200in}{1.209783in}}%
\pgfpathlineto{\pgfqpoint{3.136488in}{1.209783in}}%
\pgfpathclose%
\pgfusepath{fill}%
\end{pgfscope}%
\begin{pgfscope}%
\pgfpathrectangle{\pgfqpoint{2.887899in}{0.169444in}}{\pgfqpoint{0.857203in}{1.280417in}}%
\pgfusepath{clip}%
\pgfsetbuttcap%
\pgfsetmiterjoin%
\definecolor{currentfill}{rgb}{0.368012,0.725106,0.661822}%
\pgfsetfillcolor{currentfill}%
\pgfsetlinewidth{0.000000pt}%
\definecolor{currentstroke}{rgb}{0.000000,0.000000,0.000000}%
\pgfsetstrokecolor{currentstroke}%
\pgfsetstrokeopacity{0.000000}%
\pgfsetdash{}{0pt}%
\pgfpathmoveto{\pgfqpoint{3.136488in}{1.209783in}}%
\pgfpathlineto{\pgfqpoint{3.102200in}{1.209783in}}%
\pgfpathlineto{\pgfqpoint{3.102200in}{1.229789in}}%
\pgfpathlineto{\pgfqpoint{3.136488in}{1.229789in}}%
\pgfpathclose%
\pgfusepath{fill}%
\end{pgfscope}%
\begin{pgfscope}%
\pgfpathrectangle{\pgfqpoint{2.887899in}{0.169444in}}{\pgfqpoint{0.857203in}{1.280417in}}%
\pgfusepath{clip}%
\pgfsetbuttcap%
\pgfsetmiterjoin%
\definecolor{currentfill}{rgb}{0.336025,0.689427,0.676586}%
\pgfsetfillcolor{currentfill}%
\pgfsetlinewidth{0.000000pt}%
\definecolor{currentstroke}{rgb}{0.000000,0.000000,0.000000}%
\pgfsetstrokecolor{currentstroke}%
\pgfsetstrokeopacity{0.000000}%
\pgfsetdash{}{0pt}%
\pgfpathmoveto{\pgfqpoint{3.136488in}{1.229789in}}%
\pgfpathlineto{\pgfqpoint{3.102200in}{1.229789in}}%
\pgfpathlineto{\pgfqpoint{3.102200in}{1.249796in}}%
\pgfpathlineto{\pgfqpoint{3.136488in}{1.249796in}}%
\pgfpathclose%
\pgfusepath{fill}%
\end{pgfscope}%
\begin{pgfscope}%
\pgfpathrectangle{\pgfqpoint{2.887899in}{0.169444in}}{\pgfqpoint{0.857203in}{1.280417in}}%
\pgfusepath{clip}%
\pgfsetbuttcap%
\pgfsetmiterjoin%
\definecolor{currentfill}{rgb}{0.304037,0.653749,0.691349}%
\pgfsetfillcolor{currentfill}%
\pgfsetlinewidth{0.000000pt}%
\definecolor{currentstroke}{rgb}{0.000000,0.000000,0.000000}%
\pgfsetstrokecolor{currentstroke}%
\pgfsetstrokeopacity{0.000000}%
\pgfsetdash{}{0pt}%
\pgfpathmoveto{\pgfqpoint{3.136488in}{1.249796in}}%
\pgfpathlineto{\pgfqpoint{3.102200in}{1.249796in}}%
\pgfpathlineto{\pgfqpoint{3.102200in}{1.269803in}}%
\pgfpathlineto{\pgfqpoint{3.136488in}{1.269803in}}%
\pgfpathclose%
\pgfusepath{fill}%
\end{pgfscope}%
\begin{pgfscope}%
\pgfpathrectangle{\pgfqpoint{2.887899in}{0.169444in}}{\pgfqpoint{0.857203in}{1.280417in}}%
\pgfusepath{clip}%
\pgfsetbuttcap%
\pgfsetmiterjoin%
\definecolor{currentfill}{rgb}{0.272049,0.618070,0.706113}%
\pgfsetfillcolor{currentfill}%
\pgfsetlinewidth{0.000000pt}%
\definecolor{currentstroke}{rgb}{0.000000,0.000000,0.000000}%
\pgfsetstrokecolor{currentstroke}%
\pgfsetstrokeopacity{0.000000}%
\pgfsetdash{}{0pt}%
\pgfpathmoveto{\pgfqpoint{3.136488in}{1.269803in}}%
\pgfpathlineto{\pgfqpoint{3.102200in}{1.269803in}}%
\pgfpathlineto{\pgfqpoint{3.102200in}{1.289809in}}%
\pgfpathlineto{\pgfqpoint{3.136488in}{1.289809in}}%
\pgfpathclose%
\pgfusepath{fill}%
\end{pgfscope}%
\begin{pgfscope}%
\pgfpathrectangle{\pgfqpoint{2.887899in}{0.169444in}}{\pgfqpoint{0.857203in}{1.280417in}}%
\pgfusepath{clip}%
\pgfsetbuttcap%
\pgfsetmiterjoin%
\definecolor{currentfill}{rgb}{0.240062,0.582391,0.720877}%
\pgfsetfillcolor{currentfill}%
\pgfsetlinewidth{0.000000pt}%
\definecolor{currentstroke}{rgb}{0.000000,0.000000,0.000000}%
\pgfsetstrokecolor{currentstroke}%
\pgfsetstrokeopacity{0.000000}%
\pgfsetdash{}{0pt}%
\pgfpathmoveto{\pgfqpoint{3.136488in}{1.289809in}}%
\pgfpathlineto{\pgfqpoint{3.102200in}{1.289809in}}%
\pgfpathlineto{\pgfqpoint{3.102200in}{1.309816in}}%
\pgfpathlineto{\pgfqpoint{3.136488in}{1.309816in}}%
\pgfpathclose%
\pgfusepath{fill}%
\end{pgfscope}%
\begin{pgfscope}%
\pgfpathrectangle{\pgfqpoint{2.887899in}{0.169444in}}{\pgfqpoint{0.857203in}{1.280417in}}%
\pgfusepath{clip}%
\pgfsetbuttcap%
\pgfsetmiterjoin%
\definecolor{currentfill}{rgb}{0.208074,0.546713,0.735640}%
\pgfsetfillcolor{currentfill}%
\pgfsetlinewidth{0.000000pt}%
\definecolor{currentstroke}{rgb}{0.000000,0.000000,0.000000}%
\pgfsetstrokecolor{currentstroke}%
\pgfsetstrokeopacity{0.000000}%
\pgfsetdash{}{0pt}%
\pgfpathmoveto{\pgfqpoint{3.136488in}{1.309816in}}%
\pgfpathlineto{\pgfqpoint{3.102200in}{1.309816in}}%
\pgfpathlineto{\pgfqpoint{3.102200in}{1.329822in}}%
\pgfpathlineto{\pgfqpoint{3.136488in}{1.329822in}}%
\pgfpathclose%
\pgfusepath{fill}%
\end{pgfscope}%
\begin{pgfscope}%
\pgfpathrectangle{\pgfqpoint{2.887899in}{0.169444in}}{\pgfqpoint{0.857203in}{1.280417in}}%
\pgfusepath{clip}%
\pgfsetbuttcap%
\pgfsetmiterjoin%
\definecolor{currentfill}{rgb}{0.212995,0.511419,0.730796}%
\pgfsetfillcolor{currentfill}%
\pgfsetlinewidth{0.000000pt}%
\definecolor{currentstroke}{rgb}{0.000000,0.000000,0.000000}%
\pgfsetstrokecolor{currentstroke}%
\pgfsetstrokeopacity{0.000000}%
\pgfsetdash{}{0pt}%
\pgfpathmoveto{\pgfqpoint{3.136488in}{1.329822in}}%
\pgfpathlineto{\pgfqpoint{3.102200in}{1.329822in}}%
\pgfpathlineto{\pgfqpoint{3.102200in}{1.349829in}}%
\pgfpathlineto{\pgfqpoint{3.136488in}{1.349829in}}%
\pgfpathclose%
\pgfusepath{fill}%
\end{pgfscope}%
\begin{pgfscope}%
\pgfpathrectangle{\pgfqpoint{2.887899in}{0.169444in}}{\pgfqpoint{0.857203in}{1.280417in}}%
\pgfusepath{clip}%
\pgfsetbuttcap%
\pgfsetmiterjoin%
\definecolor{currentfill}{rgb}{0.240062,0.476355,0.714187}%
\pgfsetfillcolor{currentfill}%
\pgfsetlinewidth{0.000000pt}%
\definecolor{currentstroke}{rgb}{0.000000,0.000000,0.000000}%
\pgfsetstrokecolor{currentstroke}%
\pgfsetstrokeopacity{0.000000}%
\pgfsetdash{}{0pt}%
\pgfpathmoveto{\pgfqpoint{3.136488in}{1.349829in}}%
\pgfpathlineto{\pgfqpoint{3.102200in}{1.349829in}}%
\pgfpathlineto{\pgfqpoint{3.102200in}{1.369835in}}%
\pgfpathlineto{\pgfqpoint{3.136488in}{1.369835in}}%
\pgfpathclose%
\pgfusepath{fill}%
\end{pgfscope}%
\begin{pgfscope}%
\pgfpathrectangle{\pgfqpoint{2.887899in}{0.169444in}}{\pgfqpoint{0.857203in}{1.280417in}}%
\pgfusepath{clip}%
\pgfsetbuttcap%
\pgfsetmiterjoin%
\definecolor{currentfill}{rgb}{0.267128,0.441292,0.697578}%
\pgfsetfillcolor{currentfill}%
\pgfsetlinewidth{0.000000pt}%
\definecolor{currentstroke}{rgb}{0.000000,0.000000,0.000000}%
\pgfsetstrokecolor{currentstroke}%
\pgfsetstrokeopacity{0.000000}%
\pgfsetdash{}{0pt}%
\pgfpathmoveto{\pgfqpoint{3.136488in}{1.369835in}}%
\pgfpathlineto{\pgfqpoint{3.102200in}{1.369835in}}%
\pgfpathlineto{\pgfqpoint{3.102200in}{1.389842in}}%
\pgfpathlineto{\pgfqpoint{3.136488in}{1.389842in}}%
\pgfpathclose%
\pgfusepath{fill}%
\end{pgfscope}%
\begin{pgfscope}%
\pgfpathrectangle{\pgfqpoint{2.887899in}{0.169444in}}{\pgfqpoint{0.857203in}{1.280417in}}%
\pgfusepath{clip}%
\pgfsetbuttcap%
\pgfsetmiterjoin%
\definecolor{currentfill}{rgb}{0.294195,0.406228,0.680969}%
\pgfsetfillcolor{currentfill}%
\pgfsetlinewidth{0.000000pt}%
\definecolor{currentstroke}{rgb}{0.000000,0.000000,0.000000}%
\pgfsetstrokecolor{currentstroke}%
\pgfsetstrokeopacity{0.000000}%
\pgfsetdash{}{0pt}%
\pgfpathmoveto{\pgfqpoint{3.136488in}{1.389842in}}%
\pgfpathlineto{\pgfqpoint{3.102200in}{1.389842in}}%
\pgfpathlineto{\pgfqpoint{3.102200in}{1.409848in}}%
\pgfpathlineto{\pgfqpoint{3.136488in}{1.409848in}}%
\pgfpathclose%
\pgfusepath{fill}%
\end{pgfscope}%
\begin{pgfscope}%
\pgfpathrectangle{\pgfqpoint{2.887899in}{0.169444in}}{\pgfqpoint{0.857203in}{1.280417in}}%
\pgfusepath{clip}%
\pgfsetbuttcap%
\pgfsetmiterjoin%
\definecolor{currentfill}{rgb}{0.321261,0.371165,0.664360}%
\pgfsetfillcolor{currentfill}%
\pgfsetlinewidth{0.000000pt}%
\definecolor{currentstroke}{rgb}{0.000000,0.000000,0.000000}%
\pgfsetstrokecolor{currentstroke}%
\pgfsetstrokeopacity{0.000000}%
\pgfsetdash{}{0pt}%
\pgfpathmoveto{\pgfqpoint{3.136488in}{1.409848in}}%
\pgfpathlineto{\pgfqpoint{3.102200in}{1.409848in}}%
\pgfpathlineto{\pgfqpoint{3.102200in}{1.429855in}}%
\pgfpathlineto{\pgfqpoint{3.136488in}{1.429855in}}%
\pgfpathclose%
\pgfusepath{fill}%
\end{pgfscope}%
\begin{pgfscope}%
\pgfpathrectangle{\pgfqpoint{2.887899in}{0.169444in}}{\pgfqpoint{0.857203in}{1.280417in}}%
\pgfusepath{clip}%
\pgfsetbuttcap%
\pgfsetmiterjoin%
\definecolor{currentfill}{rgb}{0.348328,0.336101,0.647751}%
\pgfsetfillcolor{currentfill}%
\pgfsetlinewidth{0.000000pt}%
\definecolor{currentstroke}{rgb}{0.000000,0.000000,0.000000}%
\pgfsetstrokecolor{currentstroke}%
\pgfsetstrokeopacity{0.000000}%
\pgfsetdash{}{0pt}%
\pgfpathmoveto{\pgfqpoint{3.136488in}{1.429855in}}%
\pgfpathlineto{\pgfqpoint{3.102200in}{1.429855in}}%
\pgfpathlineto{\pgfqpoint{3.102200in}{1.449861in}}%
\pgfpathlineto{\pgfqpoint{3.136488in}{1.449861in}}%
\pgfpathclose%
\pgfusepath{fill}%
\end{pgfscope}%
\begin{pgfscope}%
\pgfpathrectangle{\pgfqpoint{2.887899in}{0.169444in}}{\pgfqpoint{0.857203in}{1.280417in}}%
\pgfusepath{clip}%
\pgfsetbuttcap%
\pgfsetmiterjoin%
\definecolor{currentfill}{rgb}{0.368627,0.309804,0.635294}%
\pgfsetfillcolor{currentfill}%
\pgfsetlinewidth{0.000000pt}%
\definecolor{currentstroke}{rgb}{0.000000,0.000000,0.000000}%
\pgfsetstrokecolor{currentstroke}%
\pgfsetstrokeopacity{0.000000}%
\pgfsetdash{}{0pt}%
\pgfpathmoveto{\pgfqpoint{3.136488in}{1.449861in}}%
\pgfpathlineto{\pgfqpoint{3.102200in}{1.449861in}}%
\pgfpathlineto{\pgfqpoint{3.102200in}{1.469868in}}%
\pgfpathlineto{\pgfqpoint{3.136488in}{1.469868in}}%
\pgfpathclose%
\pgfusepath{fill}%
\end{pgfscope}%
\begin{pgfscope}%
\pgfpathrectangle{\pgfqpoint{2.887899in}{0.169444in}}{\pgfqpoint{0.857203in}{1.280417in}}%
\pgfusepath{clip}%
\pgfsetbuttcap%
\pgfsetmiterjoin%
\definecolor{currentfill}{rgb}{0.904421,0.347943,0.283045}%
\pgfsetfillcolor{currentfill}%
\pgfsetlinewidth{0.000000pt}%
\definecolor{currentstroke}{rgb}{0.000000,0.000000,0.000000}%
\pgfsetstrokecolor{currentstroke}%
\pgfsetstrokeopacity{0.000000}%
\pgfsetdash{}{0pt}%
\pgfpathmoveto{\pgfqpoint{3.042196in}{0.369510in}}%
\pgfpathlineto{\pgfqpoint{3.059340in}{0.369510in}}%
\pgfpathlineto{\pgfqpoint{3.059340in}{0.409523in}}%
\pgfpathlineto{\pgfqpoint{3.042196in}{0.409523in}}%
\pgfpathclose%
\pgfusepath{fill}%
\end{pgfscope}%
\begin{pgfscope}%
\pgfpathrectangle{\pgfqpoint{2.887899in}{0.169444in}}{\pgfqpoint{0.857203in}{1.280417in}}%
\pgfusepath{clip}%
\pgfsetbuttcap%
\pgfsetmiterjoin%
\definecolor{currentfill}{rgb}{0.974856,0.557401,0.322722}%
\pgfsetfillcolor{currentfill}%
\pgfsetlinewidth{0.000000pt}%
\definecolor{currentstroke}{rgb}{0.000000,0.000000,0.000000}%
\pgfsetstrokecolor{currentstroke}%
\pgfsetstrokeopacity{0.000000}%
\pgfsetdash{}{0pt}%
\pgfpathmoveto{\pgfqpoint{3.042196in}{0.489549in}}%
\pgfpathlineto{\pgfqpoint{3.059340in}{0.489549in}}%
\pgfpathlineto{\pgfqpoint{3.059340in}{0.529562in}}%
\pgfpathlineto{\pgfqpoint{3.042196in}{0.529562in}}%
\pgfpathclose%
\pgfusepath{fill}%
\end{pgfscope}%
\begin{pgfscope}%
\pgfpathrectangle{\pgfqpoint{2.887899in}{0.169444in}}{\pgfqpoint{0.857203in}{1.280417in}}%
\pgfusepath{clip}%
\pgfsetbuttcap%
\pgfsetmiterjoin%
\definecolor{currentfill}{rgb}{0.992695,0.709266,0.402999}%
\pgfsetfillcolor{currentfill}%
\pgfsetlinewidth{0.000000pt}%
\definecolor{currentstroke}{rgb}{0.000000,0.000000,0.000000}%
\pgfsetstrokecolor{currentstroke}%
\pgfsetstrokeopacity{0.000000}%
\pgfsetdash{}{0pt}%
\pgfpathmoveto{\pgfqpoint{3.042196in}{0.569575in}}%
\pgfpathlineto{\pgfqpoint{3.059340in}{0.569575in}}%
\pgfpathlineto{\pgfqpoint{3.059340in}{0.609588in}}%
\pgfpathlineto{\pgfqpoint{3.042196in}{0.609588in}}%
\pgfpathclose%
\pgfusepath{fill}%
\end{pgfscope}%
\begin{pgfscope}%
\pgfpathrectangle{\pgfqpoint{2.887899in}{0.169444in}}{\pgfqpoint{0.857203in}{1.280417in}}%
\pgfusepath{clip}%
\pgfsetbuttcap%
\pgfsetmiterjoin%
\definecolor{currentfill}{rgb}{0.993925,0.770780,0.454671}%
\pgfsetfillcolor{currentfill}%
\pgfsetlinewidth{0.000000pt}%
\definecolor{currentstroke}{rgb}{0.000000,0.000000,0.000000}%
\pgfsetstrokecolor{currentstroke}%
\pgfsetstrokeopacity{0.000000}%
\pgfsetdash{}{0pt}%
\pgfpathmoveto{\pgfqpoint{3.025052in}{0.609588in}}%
\pgfpathlineto{\pgfqpoint{3.059340in}{0.609588in}}%
\pgfpathlineto{\pgfqpoint{3.059340in}{0.649601in}}%
\pgfpathlineto{\pgfqpoint{3.025052in}{0.649601in}}%
\pgfpathclose%
\pgfusepath{fill}%
\end{pgfscope}%
\begin{pgfscope}%
\pgfpathrectangle{\pgfqpoint{2.887899in}{0.169444in}}{\pgfqpoint{0.857203in}{1.280417in}}%
\pgfusepath{clip}%
\pgfsetbuttcap%
\pgfsetmiterjoin%
\definecolor{currentfill}{rgb}{0.995156,0.832295,0.506344}%
\pgfsetfillcolor{currentfill}%
\pgfsetlinewidth{0.000000pt}%
\definecolor{currentstroke}{rgb}{0.000000,0.000000,0.000000}%
\pgfsetstrokecolor{currentstroke}%
\pgfsetstrokeopacity{0.000000}%
\pgfsetdash{}{0pt}%
\pgfpathmoveto{\pgfqpoint{3.025052in}{0.649601in}}%
\pgfpathlineto{\pgfqpoint{3.059340in}{0.649601in}}%
\pgfpathlineto{\pgfqpoint{3.059340in}{0.689614in}}%
\pgfpathlineto{\pgfqpoint{3.025052in}{0.689614in}}%
\pgfpathclose%
\pgfusepath{fill}%
\end{pgfscope}%
\begin{pgfscope}%
\pgfpathrectangle{\pgfqpoint{2.887899in}{0.169444in}}{\pgfqpoint{0.857203in}{1.280417in}}%
\pgfusepath{clip}%
\pgfsetbuttcap%
\pgfsetmiterjoin%
\definecolor{currentfill}{rgb}{0.996386,0.887966,0.561092}%
\pgfsetfillcolor{currentfill}%
\pgfsetlinewidth{0.000000pt}%
\definecolor{currentstroke}{rgb}{0.000000,0.000000,0.000000}%
\pgfsetstrokecolor{currentstroke}%
\pgfsetstrokeopacity{0.000000}%
\pgfsetdash{}{0pt}%
\pgfpathmoveto{\pgfqpoint{2.956476in}{0.689614in}}%
\pgfpathlineto{\pgfqpoint{3.059340in}{0.689614in}}%
\pgfpathlineto{\pgfqpoint{3.059340in}{0.729627in}}%
\pgfpathlineto{\pgfqpoint{2.956476in}{0.729627in}}%
\pgfpathclose%
\pgfusepath{fill}%
\end{pgfscope}%
\begin{pgfscope}%
\pgfpathrectangle{\pgfqpoint{2.887899in}{0.169444in}}{\pgfqpoint{0.857203in}{1.280417in}}%
\pgfusepath{clip}%
\pgfsetbuttcap%
\pgfsetmiterjoin%
\definecolor{currentfill}{rgb}{0.997616,0.926105,0.625067}%
\pgfsetfillcolor{currentfill}%
\pgfsetlinewidth{0.000000pt}%
\definecolor{currentstroke}{rgb}{0.000000,0.000000,0.000000}%
\pgfsetstrokecolor{currentstroke}%
\pgfsetstrokeopacity{0.000000}%
\pgfsetdash{}{0pt}%
\pgfpathmoveto{\pgfqpoint{2.922188in}{0.729627in}}%
\pgfpathlineto{\pgfqpoint{3.059340in}{0.729627in}}%
\pgfpathlineto{\pgfqpoint{3.059340in}{0.769640in}}%
\pgfpathlineto{\pgfqpoint{2.922188in}{0.769640in}}%
\pgfpathclose%
\pgfusepath{fill}%
\end{pgfscope}%
\begin{pgfscope}%
\pgfpathrectangle{\pgfqpoint{2.887899in}{0.169444in}}{\pgfqpoint{0.857203in}{1.280417in}}%
\pgfusepath{clip}%
\pgfsetbuttcap%
\pgfsetmiterjoin%
\definecolor{currentfill}{rgb}{0.998847,0.964245,0.689043}%
\pgfsetfillcolor{currentfill}%
\pgfsetlinewidth{0.000000pt}%
\definecolor{currentstroke}{rgb}{0.000000,0.000000,0.000000}%
\pgfsetstrokecolor{currentstroke}%
\pgfsetstrokeopacity{0.000000}%
\pgfsetdash{}{0pt}%
\pgfpathmoveto{\pgfqpoint{2.802179in}{0.769640in}}%
\pgfpathlineto{\pgfqpoint{3.059340in}{0.769640in}}%
\pgfpathlineto{\pgfqpoint{3.059340in}{0.809653in}}%
\pgfpathlineto{\pgfqpoint{2.802179in}{0.809653in}}%
\pgfpathclose%
\pgfusepath{fill}%
\end{pgfscope}%
\begin{pgfscope}%
\pgfpathrectangle{\pgfqpoint{2.887899in}{0.169444in}}{\pgfqpoint{0.857203in}{1.280417in}}%
\pgfusepath{clip}%
\pgfsetbuttcap%
\pgfsetmiterjoin%
\definecolor{currentfill}{rgb}{0.998078,0.999231,0.746021}%
\pgfsetfillcolor{currentfill}%
\pgfsetlinewidth{0.000000pt}%
\definecolor{currentstroke}{rgb}{0.000000,0.000000,0.000000}%
\pgfsetstrokecolor{currentstroke}%
\pgfsetstrokeopacity{0.000000}%
\pgfsetdash{}{0pt}%
\pgfpathmoveto{\pgfqpoint{2.887899in}{0.809653in}}%
\pgfpathlineto{\pgfqpoint{3.059340in}{0.809653in}}%
\pgfpathlineto{\pgfqpoint{3.059340in}{0.849666in}}%
\pgfpathlineto{\pgfqpoint{2.887899in}{0.849666in}}%
\pgfpathclose%
\pgfusepath{fill}%
\end{pgfscope}%
\begin{pgfscope}%
\pgfpathrectangle{\pgfqpoint{2.887899in}{0.169444in}}{\pgfqpoint{0.857203in}{1.280417in}}%
\pgfusepath{clip}%
\pgfsetbuttcap%
\pgfsetmiterjoin%
\definecolor{currentfill}{rgb}{0.967320,0.986928,0.698039}%
\pgfsetfillcolor{currentfill}%
\pgfsetlinewidth{0.000000pt}%
\definecolor{currentstroke}{rgb}{0.000000,0.000000,0.000000}%
\pgfsetstrokecolor{currentstroke}%
\pgfsetstrokeopacity{0.000000}%
\pgfsetdash{}{0pt}%
\pgfpathmoveto{\pgfqpoint{2.905044in}{0.849666in}}%
\pgfpathlineto{\pgfqpoint{3.059340in}{0.849666in}}%
\pgfpathlineto{\pgfqpoint{3.059340in}{0.889679in}}%
\pgfpathlineto{\pgfqpoint{2.905044in}{0.889679in}}%
\pgfpathclose%
\pgfusepath{fill}%
\end{pgfscope}%
\begin{pgfscope}%
\pgfpathrectangle{\pgfqpoint{2.887899in}{0.169444in}}{\pgfqpoint{0.857203in}{1.280417in}}%
\pgfusepath{clip}%
\pgfsetbuttcap%
\pgfsetmiterjoin%
\definecolor{currentfill}{rgb}{0.936563,0.974625,0.650058}%
\pgfsetfillcolor{currentfill}%
\pgfsetlinewidth{0.000000pt}%
\definecolor{currentstroke}{rgb}{0.000000,0.000000,0.000000}%
\pgfsetstrokecolor{currentstroke}%
\pgfsetstrokeopacity{0.000000}%
\pgfsetdash{}{0pt}%
\pgfpathmoveto{\pgfqpoint{3.042196in}{0.889679in}}%
\pgfpathlineto{\pgfqpoint{3.059340in}{0.889679in}}%
\pgfpathlineto{\pgfqpoint{3.059340in}{0.929692in}}%
\pgfpathlineto{\pgfqpoint{3.042196in}{0.929692in}}%
\pgfpathclose%
\pgfusepath{fill}%
\end{pgfscope}%
\begin{pgfscope}%
\pgfpathrectangle{\pgfqpoint{2.887899in}{0.169444in}}{\pgfqpoint{0.857203in}{1.280417in}}%
\pgfusepath{clip}%
\pgfsetbuttcap%
\pgfsetmiterjoin%
\definecolor{currentfill}{rgb}{0.905805,0.962322,0.602076}%
\pgfsetfillcolor{currentfill}%
\pgfsetlinewidth{0.000000pt}%
\definecolor{currentstroke}{rgb}{0.000000,0.000000,0.000000}%
\pgfsetstrokecolor{currentstroke}%
\pgfsetstrokeopacity{0.000000}%
\pgfsetdash{}{0pt}%
\pgfpathmoveto{\pgfqpoint{2.990764in}{0.929692in}}%
\pgfpathlineto{\pgfqpoint{3.059340in}{0.929692in}}%
\pgfpathlineto{\pgfqpoint{3.059340in}{0.969705in}}%
\pgfpathlineto{\pgfqpoint{2.990764in}{0.969705in}}%
\pgfpathclose%
\pgfusepath{fill}%
\end{pgfscope}%
\begin{pgfscope}%
\pgfpathrectangle{\pgfqpoint{2.887899in}{0.169444in}}{\pgfqpoint{0.857203in}{1.280417in}}%
\pgfusepath{clip}%
\pgfsetbuttcap%
\pgfsetmiterjoin%
\definecolor{currentfill}{rgb}{0.693272,0.875894,0.638524}%
\pgfsetfillcolor{currentfill}%
\pgfsetlinewidth{0.000000pt}%
\definecolor{currentstroke}{rgb}{0.000000,0.000000,0.000000}%
\pgfsetstrokecolor{currentstroke}%
\pgfsetstrokeopacity{0.000000}%
\pgfsetdash{}{0pt}%
\pgfpathmoveto{\pgfqpoint{3.042196in}{1.049731in}}%
\pgfpathlineto{\pgfqpoint{3.059340in}{1.049731in}}%
\pgfpathlineto{\pgfqpoint{3.059340in}{1.089744in}}%
\pgfpathlineto{\pgfqpoint{3.042196in}{1.089744in}}%
\pgfpathclose%
\pgfusepath{fill}%
\end{pgfscope}%
\begin{pgfscope}%
\pgfpathrectangle{\pgfqpoint{2.887899in}{0.169444in}}{\pgfqpoint{0.857203in}{1.280417in}}%
\pgfusepath{clip}%
\pgfsetbuttcap%
\pgfsetmiterjoin%
\definecolor{currentfill}{rgb}{0.619608,0.003922,0.258824}%
\pgfsetfillcolor{currentfill}%
\pgfsetlinewidth{0.000000pt}%
\definecolor{currentstroke}{rgb}{0.000000,0.000000,0.000000}%
\pgfsetstrokecolor{currentstroke}%
\pgfsetstrokeopacity{0.000000}%
\pgfsetdash{}{0pt}%
\pgfpathmoveto{\pgfqpoint{3.367933in}{0.169444in}}%
\pgfpathlineto{\pgfqpoint{3.333645in}{0.169444in}}%
\pgfpathlineto{\pgfqpoint{3.333645in}{0.189451in}}%
\pgfpathlineto{\pgfqpoint{3.367933in}{0.189451in}}%
\pgfpathclose%
\pgfusepath{fill}%
\end{pgfscope}%
\begin{pgfscope}%
\pgfpathrectangle{\pgfqpoint{2.887899in}{0.169444in}}{\pgfqpoint{0.857203in}{1.280417in}}%
\pgfusepath{clip}%
\pgfsetbuttcap%
\pgfsetmiterjoin%
\definecolor{currentfill}{rgb}{0.653441,0.041446,0.266820}%
\pgfsetfillcolor{currentfill}%
\pgfsetlinewidth{0.000000pt}%
\definecolor{currentstroke}{rgb}{0.000000,0.000000,0.000000}%
\pgfsetstrokecolor{currentstroke}%
\pgfsetstrokeopacity{0.000000}%
\pgfsetdash{}{0pt}%
\pgfpathmoveto{\pgfqpoint{3.367933in}{0.189451in}}%
\pgfpathlineto{\pgfqpoint{3.333645in}{0.189451in}}%
\pgfpathlineto{\pgfqpoint{3.333645in}{0.209457in}}%
\pgfpathlineto{\pgfqpoint{3.367933in}{0.209457in}}%
\pgfpathclose%
\pgfusepath{fill}%
\end{pgfscope}%
\begin{pgfscope}%
\pgfpathrectangle{\pgfqpoint{2.887899in}{0.169444in}}{\pgfqpoint{0.857203in}{1.280417in}}%
\pgfusepath{clip}%
\pgfsetbuttcap%
\pgfsetmiterjoin%
\definecolor{currentfill}{rgb}{0.687274,0.078970,0.274817}%
\pgfsetfillcolor{currentfill}%
\pgfsetlinewidth{0.000000pt}%
\definecolor{currentstroke}{rgb}{0.000000,0.000000,0.000000}%
\pgfsetstrokecolor{currentstroke}%
\pgfsetstrokeopacity{0.000000}%
\pgfsetdash{}{0pt}%
\pgfpathmoveto{\pgfqpoint{3.367933in}{0.209457in}}%
\pgfpathlineto{\pgfqpoint{3.333645in}{0.209457in}}%
\pgfpathlineto{\pgfqpoint{3.333645in}{0.229464in}}%
\pgfpathlineto{\pgfqpoint{3.367933in}{0.229464in}}%
\pgfpathclose%
\pgfusepath{fill}%
\end{pgfscope}%
\begin{pgfscope}%
\pgfpathrectangle{\pgfqpoint{2.887899in}{0.169444in}}{\pgfqpoint{0.857203in}{1.280417in}}%
\pgfusepath{clip}%
\pgfsetbuttcap%
\pgfsetmiterjoin%
\definecolor{currentfill}{rgb}{0.721107,0.116494,0.282814}%
\pgfsetfillcolor{currentfill}%
\pgfsetlinewidth{0.000000pt}%
\definecolor{currentstroke}{rgb}{0.000000,0.000000,0.000000}%
\pgfsetstrokecolor{currentstroke}%
\pgfsetstrokeopacity{0.000000}%
\pgfsetdash{}{0pt}%
\pgfpathmoveto{\pgfqpoint{3.367933in}{0.229464in}}%
\pgfpathlineto{\pgfqpoint{3.333645in}{0.229464in}}%
\pgfpathlineto{\pgfqpoint{3.333645in}{0.249470in}}%
\pgfpathlineto{\pgfqpoint{3.367933in}{0.249470in}}%
\pgfpathclose%
\pgfusepath{fill}%
\end{pgfscope}%
\begin{pgfscope}%
\pgfpathrectangle{\pgfqpoint{2.887899in}{0.169444in}}{\pgfqpoint{0.857203in}{1.280417in}}%
\pgfusepath{clip}%
\pgfsetbuttcap%
\pgfsetmiterjoin%
\definecolor{currentfill}{rgb}{0.754940,0.154018,0.290811}%
\pgfsetfillcolor{currentfill}%
\pgfsetlinewidth{0.000000pt}%
\definecolor{currentstroke}{rgb}{0.000000,0.000000,0.000000}%
\pgfsetstrokecolor{currentstroke}%
\pgfsetstrokeopacity{0.000000}%
\pgfsetdash{}{0pt}%
\pgfpathmoveto{\pgfqpoint{3.367933in}{0.249470in}}%
\pgfpathlineto{\pgfqpoint{3.333645in}{0.249470in}}%
\pgfpathlineto{\pgfqpoint{3.333645in}{0.269477in}}%
\pgfpathlineto{\pgfqpoint{3.367933in}{0.269477in}}%
\pgfpathclose%
\pgfusepath{fill}%
\end{pgfscope}%
\begin{pgfscope}%
\pgfpathrectangle{\pgfqpoint{2.887899in}{0.169444in}}{\pgfqpoint{0.857203in}{1.280417in}}%
\pgfusepath{clip}%
\pgfsetbuttcap%
\pgfsetmiterjoin%
\definecolor{currentfill}{rgb}{0.788774,0.191542,0.298808}%
\pgfsetfillcolor{currentfill}%
\pgfsetlinewidth{0.000000pt}%
\definecolor{currentstroke}{rgb}{0.000000,0.000000,0.000000}%
\pgfsetstrokecolor{currentstroke}%
\pgfsetstrokeopacity{0.000000}%
\pgfsetdash{}{0pt}%
\pgfpathmoveto{\pgfqpoint{3.367933in}{0.269477in}}%
\pgfpathlineto{\pgfqpoint{3.333645in}{0.269477in}}%
\pgfpathlineto{\pgfqpoint{3.333645in}{0.289484in}}%
\pgfpathlineto{\pgfqpoint{3.367933in}{0.289484in}}%
\pgfpathclose%
\pgfusepath{fill}%
\end{pgfscope}%
\begin{pgfscope}%
\pgfpathrectangle{\pgfqpoint{2.887899in}{0.169444in}}{\pgfqpoint{0.857203in}{1.280417in}}%
\pgfusepath{clip}%
\pgfsetbuttcap%
\pgfsetmiterjoin%
\definecolor{currentfill}{rgb}{0.822607,0.229066,0.306805}%
\pgfsetfillcolor{currentfill}%
\pgfsetlinewidth{0.000000pt}%
\definecolor{currentstroke}{rgb}{0.000000,0.000000,0.000000}%
\pgfsetstrokecolor{currentstroke}%
\pgfsetstrokeopacity{0.000000}%
\pgfsetdash{}{0pt}%
\pgfpathmoveto{\pgfqpoint{3.367933in}{0.289484in}}%
\pgfpathlineto{\pgfqpoint{3.333645in}{0.289484in}}%
\pgfpathlineto{\pgfqpoint{3.333645in}{0.309490in}}%
\pgfpathlineto{\pgfqpoint{3.367933in}{0.309490in}}%
\pgfpathclose%
\pgfusepath{fill}%
\end{pgfscope}%
\begin{pgfscope}%
\pgfpathrectangle{\pgfqpoint{2.887899in}{0.169444in}}{\pgfqpoint{0.857203in}{1.280417in}}%
\pgfusepath{clip}%
\pgfsetbuttcap%
\pgfsetmiterjoin%
\definecolor{currentfill}{rgb}{0.847213,0.261207,0.305190}%
\pgfsetfillcolor{currentfill}%
\pgfsetlinewidth{0.000000pt}%
\definecolor{currentstroke}{rgb}{0.000000,0.000000,0.000000}%
\pgfsetstrokecolor{currentstroke}%
\pgfsetstrokeopacity{0.000000}%
\pgfsetdash{}{0pt}%
\pgfpathmoveto{\pgfqpoint{3.367933in}{0.309490in}}%
\pgfpathlineto{\pgfqpoint{3.333645in}{0.309490in}}%
\pgfpathlineto{\pgfqpoint{3.333645in}{0.329497in}}%
\pgfpathlineto{\pgfqpoint{3.367933in}{0.329497in}}%
\pgfpathclose%
\pgfusepath{fill}%
\end{pgfscope}%
\begin{pgfscope}%
\pgfpathrectangle{\pgfqpoint{2.887899in}{0.169444in}}{\pgfqpoint{0.857203in}{1.280417in}}%
\pgfusepath{clip}%
\pgfsetbuttcap%
\pgfsetmiterjoin%
\definecolor{currentfill}{rgb}{0.866282,0.290119,0.297809}%
\pgfsetfillcolor{currentfill}%
\pgfsetlinewidth{0.000000pt}%
\definecolor{currentstroke}{rgb}{0.000000,0.000000,0.000000}%
\pgfsetstrokecolor{currentstroke}%
\pgfsetstrokeopacity{0.000000}%
\pgfsetdash{}{0pt}%
\pgfpathmoveto{\pgfqpoint{3.367933in}{0.329497in}}%
\pgfpathlineto{\pgfqpoint{3.333645in}{0.329497in}}%
\pgfpathlineto{\pgfqpoint{3.333645in}{0.349503in}}%
\pgfpathlineto{\pgfqpoint{3.367933in}{0.349503in}}%
\pgfpathclose%
\pgfusepath{fill}%
\end{pgfscope}%
\begin{pgfscope}%
\pgfpathrectangle{\pgfqpoint{2.887899in}{0.169444in}}{\pgfqpoint{0.857203in}{1.280417in}}%
\pgfusepath{clip}%
\pgfsetbuttcap%
\pgfsetmiterjoin%
\definecolor{currentfill}{rgb}{0.885352,0.319031,0.290427}%
\pgfsetfillcolor{currentfill}%
\pgfsetlinewidth{0.000000pt}%
\definecolor{currentstroke}{rgb}{0.000000,0.000000,0.000000}%
\pgfsetstrokecolor{currentstroke}%
\pgfsetstrokeopacity{0.000000}%
\pgfsetdash{}{0pt}%
\pgfpathmoveto{\pgfqpoint{3.367933in}{0.349503in}}%
\pgfpathlineto{\pgfqpoint{3.333645in}{0.349503in}}%
\pgfpathlineto{\pgfqpoint{3.333645in}{0.369510in}}%
\pgfpathlineto{\pgfqpoint{3.367933in}{0.369510in}}%
\pgfpathclose%
\pgfusepath{fill}%
\end{pgfscope}%
\begin{pgfscope}%
\pgfpathrectangle{\pgfqpoint{2.887899in}{0.169444in}}{\pgfqpoint{0.857203in}{1.280417in}}%
\pgfusepath{clip}%
\pgfsetbuttcap%
\pgfsetmiterjoin%
\definecolor{currentfill}{rgb}{0.904421,0.347943,0.283045}%
\pgfsetfillcolor{currentfill}%
\pgfsetlinewidth{0.000000pt}%
\definecolor{currentstroke}{rgb}{0.000000,0.000000,0.000000}%
\pgfsetstrokecolor{currentstroke}%
\pgfsetstrokeopacity{0.000000}%
\pgfsetdash{}{0pt}%
\pgfpathmoveto{\pgfqpoint{3.367933in}{0.369510in}}%
\pgfpathlineto{\pgfqpoint{3.333645in}{0.369510in}}%
\pgfpathlineto{\pgfqpoint{3.333645in}{0.389516in}}%
\pgfpathlineto{\pgfqpoint{3.367933in}{0.389516in}}%
\pgfpathclose%
\pgfusepath{fill}%
\end{pgfscope}%
\begin{pgfscope}%
\pgfpathrectangle{\pgfqpoint{2.887899in}{0.169444in}}{\pgfqpoint{0.857203in}{1.280417in}}%
\pgfusepath{clip}%
\pgfsetbuttcap%
\pgfsetmiterjoin%
\definecolor{currentfill}{rgb}{0.923491,0.376855,0.275663}%
\pgfsetfillcolor{currentfill}%
\pgfsetlinewidth{0.000000pt}%
\definecolor{currentstroke}{rgb}{0.000000,0.000000,0.000000}%
\pgfsetstrokecolor{currentstroke}%
\pgfsetstrokeopacity{0.000000}%
\pgfsetdash{}{0pt}%
\pgfpathmoveto{\pgfqpoint{3.367933in}{0.389516in}}%
\pgfpathlineto{\pgfqpoint{3.333645in}{0.389516in}}%
\pgfpathlineto{\pgfqpoint{3.333645in}{0.409523in}}%
\pgfpathlineto{\pgfqpoint{3.367933in}{0.409523in}}%
\pgfpathclose%
\pgfusepath{fill}%
\end{pgfscope}%
\begin{pgfscope}%
\pgfpathrectangle{\pgfqpoint{2.887899in}{0.169444in}}{\pgfqpoint{0.857203in}{1.280417in}}%
\pgfusepath{clip}%
\pgfsetbuttcap%
\pgfsetmiterjoin%
\definecolor{currentfill}{rgb}{0.942561,0.405767,0.268281}%
\pgfsetfillcolor{currentfill}%
\pgfsetlinewidth{0.000000pt}%
\definecolor{currentstroke}{rgb}{0.000000,0.000000,0.000000}%
\pgfsetstrokecolor{currentstroke}%
\pgfsetstrokeopacity{0.000000}%
\pgfsetdash{}{0pt}%
\pgfpathmoveto{\pgfqpoint{3.367933in}{0.409523in}}%
\pgfpathlineto{\pgfqpoint{3.333645in}{0.409523in}}%
\pgfpathlineto{\pgfqpoint{3.333645in}{0.429529in}}%
\pgfpathlineto{\pgfqpoint{3.367933in}{0.429529in}}%
\pgfpathclose%
\pgfusepath{fill}%
\end{pgfscope}%
\begin{pgfscope}%
\pgfpathrectangle{\pgfqpoint{2.887899in}{0.169444in}}{\pgfqpoint{0.857203in}{1.280417in}}%
\pgfusepath{clip}%
\pgfsetbuttcap%
\pgfsetmiterjoin%
\definecolor{currentfill}{rgb}{0.958247,0.437447,0.267359}%
\pgfsetfillcolor{currentfill}%
\pgfsetlinewidth{0.000000pt}%
\definecolor{currentstroke}{rgb}{0.000000,0.000000,0.000000}%
\pgfsetstrokecolor{currentstroke}%
\pgfsetstrokeopacity{0.000000}%
\pgfsetdash{}{0pt}%
\pgfpathmoveto{\pgfqpoint{3.367933in}{0.429529in}}%
\pgfpathlineto{\pgfqpoint{3.333645in}{0.429529in}}%
\pgfpathlineto{\pgfqpoint{3.333645in}{0.449536in}}%
\pgfpathlineto{\pgfqpoint{3.367933in}{0.449536in}}%
\pgfpathclose%
\pgfusepath{fill}%
\end{pgfscope}%
\begin{pgfscope}%
\pgfpathrectangle{\pgfqpoint{2.887899in}{0.169444in}}{\pgfqpoint{0.857203in}{1.280417in}}%
\pgfusepath{clip}%
\pgfsetbuttcap%
\pgfsetmiterjoin%
\definecolor{currentfill}{rgb}{0.963783,0.477432,0.285813}%
\pgfsetfillcolor{currentfill}%
\pgfsetlinewidth{0.000000pt}%
\definecolor{currentstroke}{rgb}{0.000000,0.000000,0.000000}%
\pgfsetstrokecolor{currentstroke}%
\pgfsetstrokeopacity{0.000000}%
\pgfsetdash{}{0pt}%
\pgfpathmoveto{\pgfqpoint{3.367933in}{0.449536in}}%
\pgfpathlineto{\pgfqpoint{3.333645in}{0.449536in}}%
\pgfpathlineto{\pgfqpoint{3.333645in}{0.469542in}}%
\pgfpathlineto{\pgfqpoint{3.367933in}{0.469542in}}%
\pgfpathclose%
\pgfusepath{fill}%
\end{pgfscope}%
\begin{pgfscope}%
\pgfpathrectangle{\pgfqpoint{2.887899in}{0.169444in}}{\pgfqpoint{0.857203in}{1.280417in}}%
\pgfusepath{clip}%
\pgfsetbuttcap%
\pgfsetmiterjoin%
\definecolor{currentfill}{rgb}{0.969319,0.517416,0.304268}%
\pgfsetfillcolor{currentfill}%
\pgfsetlinewidth{0.000000pt}%
\definecolor{currentstroke}{rgb}{0.000000,0.000000,0.000000}%
\pgfsetstrokecolor{currentstroke}%
\pgfsetstrokeopacity{0.000000}%
\pgfsetdash{}{0pt}%
\pgfpathmoveto{\pgfqpoint{3.367933in}{0.469542in}}%
\pgfpathlineto{\pgfqpoint{3.333645in}{0.469542in}}%
\pgfpathlineto{\pgfqpoint{3.333645in}{0.489549in}}%
\pgfpathlineto{\pgfqpoint{3.367933in}{0.489549in}}%
\pgfpathclose%
\pgfusepath{fill}%
\end{pgfscope}%
\begin{pgfscope}%
\pgfpathrectangle{\pgfqpoint{2.887899in}{0.169444in}}{\pgfqpoint{0.857203in}{1.280417in}}%
\pgfusepath{clip}%
\pgfsetbuttcap%
\pgfsetmiterjoin%
\definecolor{currentfill}{rgb}{0.974856,0.557401,0.322722}%
\pgfsetfillcolor{currentfill}%
\pgfsetlinewidth{0.000000pt}%
\definecolor{currentstroke}{rgb}{0.000000,0.000000,0.000000}%
\pgfsetstrokecolor{currentstroke}%
\pgfsetstrokeopacity{0.000000}%
\pgfsetdash{}{0pt}%
\pgfpathmoveto{\pgfqpoint{3.367933in}{0.489549in}}%
\pgfpathlineto{\pgfqpoint{3.333645in}{0.489549in}}%
\pgfpathlineto{\pgfqpoint{3.333645in}{0.509555in}}%
\pgfpathlineto{\pgfqpoint{3.367933in}{0.509555in}}%
\pgfpathclose%
\pgfusepath{fill}%
\end{pgfscope}%
\begin{pgfscope}%
\pgfpathrectangle{\pgfqpoint{2.887899in}{0.169444in}}{\pgfqpoint{0.857203in}{1.280417in}}%
\pgfusepath{clip}%
\pgfsetbuttcap%
\pgfsetmiterjoin%
\definecolor{currentfill}{rgb}{0.980392,0.597386,0.341176}%
\pgfsetfillcolor{currentfill}%
\pgfsetlinewidth{0.000000pt}%
\definecolor{currentstroke}{rgb}{0.000000,0.000000,0.000000}%
\pgfsetstrokecolor{currentstroke}%
\pgfsetstrokeopacity{0.000000}%
\pgfsetdash{}{0pt}%
\pgfpathmoveto{\pgfqpoint{3.367933in}{0.509555in}}%
\pgfpathlineto{\pgfqpoint{3.333645in}{0.509555in}}%
\pgfpathlineto{\pgfqpoint{3.333645in}{0.529562in}}%
\pgfpathlineto{\pgfqpoint{3.367933in}{0.529562in}}%
\pgfpathclose%
\pgfusepath{fill}%
\end{pgfscope}%
\begin{pgfscope}%
\pgfpathrectangle{\pgfqpoint{2.887899in}{0.169444in}}{\pgfqpoint{0.857203in}{1.280417in}}%
\pgfusepath{clip}%
\pgfsetbuttcap%
\pgfsetmiterjoin%
\definecolor{currentfill}{rgb}{0.985928,0.637370,0.359631}%
\pgfsetfillcolor{currentfill}%
\pgfsetlinewidth{0.000000pt}%
\definecolor{currentstroke}{rgb}{0.000000,0.000000,0.000000}%
\pgfsetstrokecolor{currentstroke}%
\pgfsetstrokeopacity{0.000000}%
\pgfsetdash{}{0pt}%
\pgfpathmoveto{\pgfqpoint{3.367933in}{0.529562in}}%
\pgfpathlineto{\pgfqpoint{3.333645in}{0.529562in}}%
\pgfpathlineto{\pgfqpoint{3.333645in}{0.549568in}}%
\pgfpathlineto{\pgfqpoint{3.367933in}{0.549568in}}%
\pgfpathclose%
\pgfusepath{fill}%
\end{pgfscope}%
\begin{pgfscope}%
\pgfpathrectangle{\pgfqpoint{2.887899in}{0.169444in}}{\pgfqpoint{0.857203in}{1.280417in}}%
\pgfusepath{clip}%
\pgfsetbuttcap%
\pgfsetmiterjoin%
\definecolor{currentfill}{rgb}{0.991465,0.677355,0.378085}%
\pgfsetfillcolor{currentfill}%
\pgfsetlinewidth{0.000000pt}%
\definecolor{currentstroke}{rgb}{0.000000,0.000000,0.000000}%
\pgfsetstrokecolor{currentstroke}%
\pgfsetstrokeopacity{0.000000}%
\pgfsetdash{}{0pt}%
\pgfpathmoveto{\pgfqpoint{3.367933in}{0.549568in}}%
\pgfpathlineto{\pgfqpoint{3.333645in}{0.549568in}}%
\pgfpathlineto{\pgfqpoint{3.333645in}{0.569575in}}%
\pgfpathlineto{\pgfqpoint{3.367933in}{0.569575in}}%
\pgfpathclose%
\pgfusepath{fill}%
\end{pgfscope}%
\begin{pgfscope}%
\pgfpathrectangle{\pgfqpoint{2.887899in}{0.169444in}}{\pgfqpoint{0.857203in}{1.280417in}}%
\pgfusepath{clip}%
\pgfsetbuttcap%
\pgfsetmiterjoin%
\definecolor{currentfill}{rgb}{0.992695,0.709266,0.402999}%
\pgfsetfillcolor{currentfill}%
\pgfsetlinewidth{0.000000pt}%
\definecolor{currentstroke}{rgb}{0.000000,0.000000,0.000000}%
\pgfsetstrokecolor{currentstroke}%
\pgfsetstrokeopacity{0.000000}%
\pgfsetdash{}{0pt}%
\pgfpathmoveto{\pgfqpoint{3.367933in}{0.569575in}}%
\pgfpathlineto{\pgfqpoint{3.333645in}{0.569575in}}%
\pgfpathlineto{\pgfqpoint{3.333645in}{0.589581in}}%
\pgfpathlineto{\pgfqpoint{3.367933in}{0.589581in}}%
\pgfpathclose%
\pgfusepath{fill}%
\end{pgfscope}%
\begin{pgfscope}%
\pgfpathrectangle{\pgfqpoint{2.887899in}{0.169444in}}{\pgfqpoint{0.857203in}{1.280417in}}%
\pgfusepath{clip}%
\pgfsetbuttcap%
\pgfsetmiterjoin%
\definecolor{currentfill}{rgb}{0.993310,0.740023,0.428835}%
\pgfsetfillcolor{currentfill}%
\pgfsetlinewidth{0.000000pt}%
\definecolor{currentstroke}{rgb}{0.000000,0.000000,0.000000}%
\pgfsetstrokecolor{currentstroke}%
\pgfsetstrokeopacity{0.000000}%
\pgfsetdash{}{0pt}%
\pgfpathmoveto{\pgfqpoint{3.367933in}{0.589581in}}%
\pgfpathlineto{\pgfqpoint{3.333645in}{0.589581in}}%
\pgfpathlineto{\pgfqpoint{3.333645in}{0.609588in}}%
\pgfpathlineto{\pgfqpoint{3.367933in}{0.609588in}}%
\pgfpathclose%
\pgfusepath{fill}%
\end{pgfscope}%
\begin{pgfscope}%
\pgfpathrectangle{\pgfqpoint{2.887899in}{0.169444in}}{\pgfqpoint{0.857203in}{1.280417in}}%
\pgfusepath{clip}%
\pgfsetbuttcap%
\pgfsetmiterjoin%
\definecolor{currentfill}{rgb}{0.993925,0.770780,0.454671}%
\pgfsetfillcolor{currentfill}%
\pgfsetlinewidth{0.000000pt}%
\definecolor{currentstroke}{rgb}{0.000000,0.000000,0.000000}%
\pgfsetstrokecolor{currentstroke}%
\pgfsetstrokeopacity{0.000000}%
\pgfsetdash{}{0pt}%
\pgfpathmoveto{\pgfqpoint{3.367933in}{0.609588in}}%
\pgfpathlineto{\pgfqpoint{3.333645in}{0.609588in}}%
\pgfpathlineto{\pgfqpoint{3.333645in}{0.629594in}}%
\pgfpathlineto{\pgfqpoint{3.367933in}{0.629594in}}%
\pgfpathclose%
\pgfusepath{fill}%
\end{pgfscope}%
\begin{pgfscope}%
\pgfpathrectangle{\pgfqpoint{2.887899in}{0.169444in}}{\pgfqpoint{0.857203in}{1.280417in}}%
\pgfusepath{clip}%
\pgfsetbuttcap%
\pgfsetmiterjoin%
\definecolor{currentfill}{rgb}{0.994541,0.801538,0.480507}%
\pgfsetfillcolor{currentfill}%
\pgfsetlinewidth{0.000000pt}%
\definecolor{currentstroke}{rgb}{0.000000,0.000000,0.000000}%
\pgfsetstrokecolor{currentstroke}%
\pgfsetstrokeopacity{0.000000}%
\pgfsetdash{}{0pt}%
\pgfpathmoveto{\pgfqpoint{3.367933in}{0.629594in}}%
\pgfpathlineto{\pgfqpoint{3.333645in}{0.629594in}}%
\pgfpathlineto{\pgfqpoint{3.333645in}{0.649601in}}%
\pgfpathlineto{\pgfqpoint{3.367933in}{0.649601in}}%
\pgfpathclose%
\pgfusepath{fill}%
\end{pgfscope}%
\begin{pgfscope}%
\pgfpathrectangle{\pgfqpoint{2.887899in}{0.169444in}}{\pgfqpoint{0.857203in}{1.280417in}}%
\pgfusepath{clip}%
\pgfsetbuttcap%
\pgfsetmiterjoin%
\definecolor{currentfill}{rgb}{0.995156,0.832295,0.506344}%
\pgfsetfillcolor{currentfill}%
\pgfsetlinewidth{0.000000pt}%
\definecolor{currentstroke}{rgb}{0.000000,0.000000,0.000000}%
\pgfsetstrokecolor{currentstroke}%
\pgfsetstrokeopacity{0.000000}%
\pgfsetdash{}{0pt}%
\pgfpathmoveto{\pgfqpoint{3.367933in}{0.649601in}}%
\pgfpathlineto{\pgfqpoint{3.333645in}{0.649601in}}%
\pgfpathlineto{\pgfqpoint{3.333645in}{0.669607in}}%
\pgfpathlineto{\pgfqpoint{3.367933in}{0.669607in}}%
\pgfpathclose%
\pgfusepath{fill}%
\end{pgfscope}%
\begin{pgfscope}%
\pgfpathrectangle{\pgfqpoint{2.887899in}{0.169444in}}{\pgfqpoint{0.857203in}{1.280417in}}%
\pgfusepath{clip}%
\pgfsetbuttcap%
\pgfsetmiterjoin%
\definecolor{currentfill}{rgb}{0.995771,0.863053,0.532180}%
\pgfsetfillcolor{currentfill}%
\pgfsetlinewidth{0.000000pt}%
\definecolor{currentstroke}{rgb}{0.000000,0.000000,0.000000}%
\pgfsetstrokecolor{currentstroke}%
\pgfsetstrokeopacity{0.000000}%
\pgfsetdash{}{0pt}%
\pgfpathmoveto{\pgfqpoint{3.367933in}{0.669607in}}%
\pgfpathlineto{\pgfqpoint{3.333645in}{0.669607in}}%
\pgfpathlineto{\pgfqpoint{3.333645in}{0.689614in}}%
\pgfpathlineto{\pgfqpoint{3.367933in}{0.689614in}}%
\pgfpathclose%
\pgfusepath{fill}%
\end{pgfscope}%
\begin{pgfscope}%
\pgfpathrectangle{\pgfqpoint{2.887899in}{0.169444in}}{\pgfqpoint{0.857203in}{1.280417in}}%
\pgfusepath{clip}%
\pgfsetbuttcap%
\pgfsetmiterjoin%
\definecolor{currentfill}{rgb}{0.996386,0.887966,0.561092}%
\pgfsetfillcolor{currentfill}%
\pgfsetlinewidth{0.000000pt}%
\definecolor{currentstroke}{rgb}{0.000000,0.000000,0.000000}%
\pgfsetstrokecolor{currentstroke}%
\pgfsetstrokeopacity{0.000000}%
\pgfsetdash{}{0pt}%
\pgfpathmoveto{\pgfqpoint{3.367933in}{0.689614in}}%
\pgfpathlineto{\pgfqpoint{3.333645in}{0.689614in}}%
\pgfpathlineto{\pgfqpoint{3.333645in}{0.709620in}}%
\pgfpathlineto{\pgfqpoint{3.367933in}{0.709620in}}%
\pgfpathclose%
\pgfusepath{fill}%
\end{pgfscope}%
\begin{pgfscope}%
\pgfpathrectangle{\pgfqpoint{2.887899in}{0.169444in}}{\pgfqpoint{0.857203in}{1.280417in}}%
\pgfusepath{clip}%
\pgfsetbuttcap%
\pgfsetmiterjoin%
\definecolor{currentfill}{rgb}{0.997001,0.907036,0.593080}%
\pgfsetfillcolor{currentfill}%
\pgfsetlinewidth{0.000000pt}%
\definecolor{currentstroke}{rgb}{0.000000,0.000000,0.000000}%
\pgfsetstrokecolor{currentstroke}%
\pgfsetstrokeopacity{0.000000}%
\pgfsetdash{}{0pt}%
\pgfpathmoveto{\pgfqpoint{3.367933in}{0.709620in}}%
\pgfpathlineto{\pgfqpoint{3.333645in}{0.709620in}}%
\pgfpathlineto{\pgfqpoint{3.333645in}{0.729627in}}%
\pgfpathlineto{\pgfqpoint{3.367933in}{0.729627in}}%
\pgfpathclose%
\pgfusepath{fill}%
\end{pgfscope}%
\begin{pgfscope}%
\pgfpathrectangle{\pgfqpoint{2.887899in}{0.169444in}}{\pgfqpoint{0.857203in}{1.280417in}}%
\pgfusepath{clip}%
\pgfsetbuttcap%
\pgfsetmiterjoin%
\definecolor{currentfill}{rgb}{0.997616,0.926105,0.625067}%
\pgfsetfillcolor{currentfill}%
\pgfsetlinewidth{0.000000pt}%
\definecolor{currentstroke}{rgb}{0.000000,0.000000,0.000000}%
\pgfsetstrokecolor{currentstroke}%
\pgfsetstrokeopacity{0.000000}%
\pgfsetdash{}{0pt}%
\pgfpathmoveto{\pgfqpoint{3.367933in}{0.729627in}}%
\pgfpathlineto{\pgfqpoint{3.333645in}{0.729627in}}%
\pgfpathlineto{\pgfqpoint{3.333645in}{0.749633in}}%
\pgfpathlineto{\pgfqpoint{3.367933in}{0.749633in}}%
\pgfpathclose%
\pgfusepath{fill}%
\end{pgfscope}%
\begin{pgfscope}%
\pgfpathrectangle{\pgfqpoint{2.887899in}{0.169444in}}{\pgfqpoint{0.857203in}{1.280417in}}%
\pgfusepath{clip}%
\pgfsetbuttcap%
\pgfsetmiterjoin%
\definecolor{currentfill}{rgb}{0.998231,0.945175,0.657055}%
\pgfsetfillcolor{currentfill}%
\pgfsetlinewidth{0.000000pt}%
\definecolor{currentstroke}{rgb}{0.000000,0.000000,0.000000}%
\pgfsetstrokecolor{currentstroke}%
\pgfsetstrokeopacity{0.000000}%
\pgfsetdash{}{0pt}%
\pgfpathmoveto{\pgfqpoint{3.367933in}{0.749633in}}%
\pgfpathlineto{\pgfqpoint{3.333645in}{0.749633in}}%
\pgfpathlineto{\pgfqpoint{3.333645in}{0.769640in}}%
\pgfpathlineto{\pgfqpoint{3.367933in}{0.769640in}}%
\pgfpathclose%
\pgfusepath{fill}%
\end{pgfscope}%
\begin{pgfscope}%
\pgfpathrectangle{\pgfqpoint{2.887899in}{0.169444in}}{\pgfqpoint{0.857203in}{1.280417in}}%
\pgfusepath{clip}%
\pgfsetbuttcap%
\pgfsetmiterjoin%
\definecolor{currentfill}{rgb}{0.998847,0.964245,0.689043}%
\pgfsetfillcolor{currentfill}%
\pgfsetlinewidth{0.000000pt}%
\definecolor{currentstroke}{rgb}{0.000000,0.000000,0.000000}%
\pgfsetstrokecolor{currentstroke}%
\pgfsetstrokeopacity{0.000000}%
\pgfsetdash{}{0pt}%
\pgfpathmoveto{\pgfqpoint{3.367933in}{0.769640in}}%
\pgfpathlineto{\pgfqpoint{3.333645in}{0.769640in}}%
\pgfpathlineto{\pgfqpoint{3.333645in}{0.789646in}}%
\pgfpathlineto{\pgfqpoint{3.367933in}{0.789646in}}%
\pgfpathclose%
\pgfusepath{fill}%
\end{pgfscope}%
\begin{pgfscope}%
\pgfpathrectangle{\pgfqpoint{2.887899in}{0.169444in}}{\pgfqpoint{0.857203in}{1.280417in}}%
\pgfusepath{clip}%
\pgfsetbuttcap%
\pgfsetmiterjoin%
\definecolor{currentfill}{rgb}{0.999462,0.983314,0.721030}%
\pgfsetfillcolor{currentfill}%
\pgfsetlinewidth{0.000000pt}%
\definecolor{currentstroke}{rgb}{0.000000,0.000000,0.000000}%
\pgfsetstrokecolor{currentstroke}%
\pgfsetstrokeopacity{0.000000}%
\pgfsetdash{}{0pt}%
\pgfpathmoveto{\pgfqpoint{3.367933in}{0.789646in}}%
\pgfpathlineto{\pgfqpoint{3.333645in}{0.789646in}}%
\pgfpathlineto{\pgfqpoint{3.333645in}{0.809653in}}%
\pgfpathlineto{\pgfqpoint{3.367933in}{0.809653in}}%
\pgfpathclose%
\pgfusepath{fill}%
\end{pgfscope}%
\begin{pgfscope}%
\pgfpathrectangle{\pgfqpoint{2.887899in}{0.169444in}}{\pgfqpoint{0.857203in}{1.280417in}}%
\pgfusepath{clip}%
\pgfsetbuttcap%
\pgfsetmiterjoin%
\definecolor{currentfill}{rgb}{0.998078,0.999231,0.746021}%
\pgfsetfillcolor{currentfill}%
\pgfsetlinewidth{0.000000pt}%
\definecolor{currentstroke}{rgb}{0.000000,0.000000,0.000000}%
\pgfsetstrokecolor{currentstroke}%
\pgfsetstrokeopacity{0.000000}%
\pgfsetdash{}{0pt}%
\pgfpathmoveto{\pgfqpoint{3.367933in}{0.809653in}}%
\pgfpathlineto{\pgfqpoint{3.333645in}{0.809653in}}%
\pgfpathlineto{\pgfqpoint{3.333645in}{0.829659in}}%
\pgfpathlineto{\pgfqpoint{3.367933in}{0.829659in}}%
\pgfpathclose%
\pgfusepath{fill}%
\end{pgfscope}%
\begin{pgfscope}%
\pgfpathrectangle{\pgfqpoint{2.887899in}{0.169444in}}{\pgfqpoint{0.857203in}{1.280417in}}%
\pgfusepath{clip}%
\pgfsetbuttcap%
\pgfsetmiterjoin%
\definecolor{currentfill}{rgb}{0.982699,0.993080,0.722030}%
\pgfsetfillcolor{currentfill}%
\pgfsetlinewidth{0.000000pt}%
\definecolor{currentstroke}{rgb}{0.000000,0.000000,0.000000}%
\pgfsetstrokecolor{currentstroke}%
\pgfsetstrokeopacity{0.000000}%
\pgfsetdash{}{0pt}%
\pgfpathmoveto{\pgfqpoint{3.367933in}{0.829659in}}%
\pgfpathlineto{\pgfqpoint{3.333645in}{0.829659in}}%
\pgfpathlineto{\pgfqpoint{3.333645in}{0.849666in}}%
\pgfpathlineto{\pgfqpoint{3.367933in}{0.849666in}}%
\pgfpathclose%
\pgfusepath{fill}%
\end{pgfscope}%
\begin{pgfscope}%
\pgfpathrectangle{\pgfqpoint{2.887899in}{0.169444in}}{\pgfqpoint{0.857203in}{1.280417in}}%
\pgfusepath{clip}%
\pgfsetbuttcap%
\pgfsetmiterjoin%
\definecolor{currentfill}{rgb}{0.967320,0.986928,0.698039}%
\pgfsetfillcolor{currentfill}%
\pgfsetlinewidth{0.000000pt}%
\definecolor{currentstroke}{rgb}{0.000000,0.000000,0.000000}%
\pgfsetstrokecolor{currentstroke}%
\pgfsetstrokeopacity{0.000000}%
\pgfsetdash{}{0pt}%
\pgfpathmoveto{\pgfqpoint{3.367933in}{0.849666in}}%
\pgfpathlineto{\pgfqpoint{3.333645in}{0.849666in}}%
\pgfpathlineto{\pgfqpoint{3.333645in}{0.869672in}}%
\pgfpathlineto{\pgfqpoint{3.367933in}{0.869672in}}%
\pgfpathclose%
\pgfusepath{fill}%
\end{pgfscope}%
\begin{pgfscope}%
\pgfpathrectangle{\pgfqpoint{2.887899in}{0.169444in}}{\pgfqpoint{0.857203in}{1.280417in}}%
\pgfusepath{clip}%
\pgfsetbuttcap%
\pgfsetmiterjoin%
\definecolor{currentfill}{rgb}{0.951942,0.980777,0.674048}%
\pgfsetfillcolor{currentfill}%
\pgfsetlinewidth{0.000000pt}%
\definecolor{currentstroke}{rgb}{0.000000,0.000000,0.000000}%
\pgfsetstrokecolor{currentstroke}%
\pgfsetstrokeopacity{0.000000}%
\pgfsetdash{}{0pt}%
\pgfpathmoveto{\pgfqpoint{3.367933in}{0.869672in}}%
\pgfpathlineto{\pgfqpoint{3.333645in}{0.869672in}}%
\pgfpathlineto{\pgfqpoint{3.333645in}{0.889679in}}%
\pgfpathlineto{\pgfqpoint{3.367933in}{0.889679in}}%
\pgfpathclose%
\pgfusepath{fill}%
\end{pgfscope}%
\begin{pgfscope}%
\pgfpathrectangle{\pgfqpoint{2.887899in}{0.169444in}}{\pgfqpoint{0.857203in}{1.280417in}}%
\pgfusepath{clip}%
\pgfsetbuttcap%
\pgfsetmiterjoin%
\definecolor{currentfill}{rgb}{0.936563,0.974625,0.650058}%
\pgfsetfillcolor{currentfill}%
\pgfsetlinewidth{0.000000pt}%
\definecolor{currentstroke}{rgb}{0.000000,0.000000,0.000000}%
\pgfsetstrokecolor{currentstroke}%
\pgfsetstrokeopacity{0.000000}%
\pgfsetdash{}{0pt}%
\pgfpathmoveto{\pgfqpoint{3.367933in}{0.889679in}}%
\pgfpathlineto{\pgfqpoint{3.333645in}{0.889679in}}%
\pgfpathlineto{\pgfqpoint{3.333645in}{0.909685in}}%
\pgfpathlineto{\pgfqpoint{3.367933in}{0.909685in}}%
\pgfpathclose%
\pgfusepath{fill}%
\end{pgfscope}%
\begin{pgfscope}%
\pgfpathrectangle{\pgfqpoint{2.887899in}{0.169444in}}{\pgfqpoint{0.857203in}{1.280417in}}%
\pgfusepath{clip}%
\pgfsetbuttcap%
\pgfsetmiterjoin%
\definecolor{currentfill}{rgb}{0.921184,0.968474,0.626067}%
\pgfsetfillcolor{currentfill}%
\pgfsetlinewidth{0.000000pt}%
\definecolor{currentstroke}{rgb}{0.000000,0.000000,0.000000}%
\pgfsetstrokecolor{currentstroke}%
\pgfsetstrokeopacity{0.000000}%
\pgfsetdash{}{0pt}%
\pgfpathmoveto{\pgfqpoint{3.367933in}{0.909685in}}%
\pgfpathlineto{\pgfqpoint{3.333645in}{0.909685in}}%
\pgfpathlineto{\pgfqpoint{3.333645in}{0.929692in}}%
\pgfpathlineto{\pgfqpoint{3.367933in}{0.929692in}}%
\pgfpathclose%
\pgfusepath{fill}%
\end{pgfscope}%
\begin{pgfscope}%
\pgfpathrectangle{\pgfqpoint{2.887899in}{0.169444in}}{\pgfqpoint{0.857203in}{1.280417in}}%
\pgfusepath{clip}%
\pgfsetbuttcap%
\pgfsetmiterjoin%
\definecolor{currentfill}{rgb}{0.905805,0.962322,0.602076}%
\pgfsetfillcolor{currentfill}%
\pgfsetlinewidth{0.000000pt}%
\definecolor{currentstroke}{rgb}{0.000000,0.000000,0.000000}%
\pgfsetstrokecolor{currentstroke}%
\pgfsetstrokeopacity{0.000000}%
\pgfsetdash{}{0pt}%
\pgfpathmoveto{\pgfqpoint{3.367933in}{0.929692in}}%
\pgfpathlineto{\pgfqpoint{3.333645in}{0.929692in}}%
\pgfpathlineto{\pgfqpoint{3.333645in}{0.949698in}}%
\pgfpathlineto{\pgfqpoint{3.367933in}{0.949698in}}%
\pgfpathclose%
\pgfusepath{fill}%
\end{pgfscope}%
\begin{pgfscope}%
\pgfpathrectangle{\pgfqpoint{2.887899in}{0.169444in}}{\pgfqpoint{0.857203in}{1.280417in}}%
\pgfusepath{clip}%
\pgfsetbuttcap%
\pgfsetmiterjoin%
\definecolor{currentfill}{rgb}{0.874740,0.949712,0.601615}%
\pgfsetfillcolor{currentfill}%
\pgfsetlinewidth{0.000000pt}%
\definecolor{currentstroke}{rgb}{0.000000,0.000000,0.000000}%
\pgfsetstrokecolor{currentstroke}%
\pgfsetstrokeopacity{0.000000}%
\pgfsetdash{}{0pt}%
\pgfpathmoveto{\pgfqpoint{3.367933in}{0.949698in}}%
\pgfpathlineto{\pgfqpoint{3.333645in}{0.949698in}}%
\pgfpathlineto{\pgfqpoint{3.333645in}{0.969705in}}%
\pgfpathlineto{\pgfqpoint{3.367933in}{0.969705in}}%
\pgfpathclose%
\pgfusepath{fill}%
\end{pgfscope}%
\begin{pgfscope}%
\pgfpathrectangle{\pgfqpoint{2.887899in}{0.169444in}}{\pgfqpoint{0.857203in}{1.280417in}}%
\pgfusepath{clip}%
\pgfsetbuttcap%
\pgfsetmiterjoin%
\definecolor{currentfill}{rgb}{0.838447,0.934948,0.608997}%
\pgfsetfillcolor{currentfill}%
\pgfsetlinewidth{0.000000pt}%
\definecolor{currentstroke}{rgb}{0.000000,0.000000,0.000000}%
\pgfsetstrokecolor{currentstroke}%
\pgfsetstrokeopacity{0.000000}%
\pgfsetdash{}{0pt}%
\pgfpathmoveto{\pgfqpoint{3.367933in}{0.969705in}}%
\pgfpathlineto{\pgfqpoint{3.333645in}{0.969705in}}%
\pgfpathlineto{\pgfqpoint{3.333645in}{0.989711in}}%
\pgfpathlineto{\pgfqpoint{3.367933in}{0.989711in}}%
\pgfpathclose%
\pgfusepath{fill}%
\end{pgfscope}%
\begin{pgfscope}%
\pgfpathrectangle{\pgfqpoint{2.887899in}{0.169444in}}{\pgfqpoint{0.857203in}{1.280417in}}%
\pgfusepath{clip}%
\pgfsetbuttcap%
\pgfsetmiterjoin%
\definecolor{currentfill}{rgb}{0.802153,0.920185,0.616378}%
\pgfsetfillcolor{currentfill}%
\pgfsetlinewidth{0.000000pt}%
\definecolor{currentstroke}{rgb}{0.000000,0.000000,0.000000}%
\pgfsetstrokecolor{currentstroke}%
\pgfsetstrokeopacity{0.000000}%
\pgfsetdash{}{0pt}%
\pgfpathmoveto{\pgfqpoint{3.367933in}{0.989711in}}%
\pgfpathlineto{\pgfqpoint{3.333645in}{0.989711in}}%
\pgfpathlineto{\pgfqpoint{3.333645in}{1.009718in}}%
\pgfpathlineto{\pgfqpoint{3.367933in}{1.009718in}}%
\pgfpathclose%
\pgfusepath{fill}%
\end{pgfscope}%
\begin{pgfscope}%
\pgfpathrectangle{\pgfqpoint{2.887899in}{0.169444in}}{\pgfqpoint{0.857203in}{1.280417in}}%
\pgfusepath{clip}%
\pgfsetbuttcap%
\pgfsetmiterjoin%
\definecolor{currentfill}{rgb}{0.765859,0.905421,0.623760}%
\pgfsetfillcolor{currentfill}%
\pgfsetlinewidth{0.000000pt}%
\definecolor{currentstroke}{rgb}{0.000000,0.000000,0.000000}%
\pgfsetstrokecolor{currentstroke}%
\pgfsetstrokeopacity{0.000000}%
\pgfsetdash{}{0pt}%
\pgfpathmoveto{\pgfqpoint{3.367933in}{1.009718in}}%
\pgfpathlineto{\pgfqpoint{3.333645in}{1.009718in}}%
\pgfpathlineto{\pgfqpoint{3.333645in}{1.029724in}}%
\pgfpathlineto{\pgfqpoint{3.367933in}{1.029724in}}%
\pgfpathclose%
\pgfusepath{fill}%
\end{pgfscope}%
\begin{pgfscope}%
\pgfpathrectangle{\pgfqpoint{2.887899in}{0.169444in}}{\pgfqpoint{0.857203in}{1.280417in}}%
\pgfusepath{clip}%
\pgfsetbuttcap%
\pgfsetmiterjoin%
\definecolor{currentfill}{rgb}{0.729566,0.890657,0.631142}%
\pgfsetfillcolor{currentfill}%
\pgfsetlinewidth{0.000000pt}%
\definecolor{currentstroke}{rgb}{0.000000,0.000000,0.000000}%
\pgfsetstrokecolor{currentstroke}%
\pgfsetstrokeopacity{0.000000}%
\pgfsetdash{}{0pt}%
\pgfpathmoveto{\pgfqpoint{3.367933in}{1.029724in}}%
\pgfpathlineto{\pgfqpoint{3.333645in}{1.029724in}}%
\pgfpathlineto{\pgfqpoint{3.333645in}{1.049731in}}%
\pgfpathlineto{\pgfqpoint{3.367933in}{1.049731in}}%
\pgfpathclose%
\pgfusepath{fill}%
\end{pgfscope}%
\begin{pgfscope}%
\pgfpathrectangle{\pgfqpoint{2.887899in}{0.169444in}}{\pgfqpoint{0.857203in}{1.280417in}}%
\pgfusepath{clip}%
\pgfsetbuttcap%
\pgfsetmiterjoin%
\definecolor{currentfill}{rgb}{0.693272,0.875894,0.638524}%
\pgfsetfillcolor{currentfill}%
\pgfsetlinewidth{0.000000pt}%
\definecolor{currentstroke}{rgb}{0.000000,0.000000,0.000000}%
\pgfsetstrokecolor{currentstroke}%
\pgfsetstrokeopacity{0.000000}%
\pgfsetdash{}{0pt}%
\pgfpathmoveto{\pgfqpoint{3.367933in}{1.049731in}}%
\pgfpathlineto{\pgfqpoint{3.333645in}{1.049731in}}%
\pgfpathlineto{\pgfqpoint{3.333645in}{1.069737in}}%
\pgfpathlineto{\pgfqpoint{3.367933in}{1.069737in}}%
\pgfpathclose%
\pgfusepath{fill}%
\end{pgfscope}%
\begin{pgfscope}%
\pgfpathrectangle{\pgfqpoint{2.887899in}{0.169444in}}{\pgfqpoint{0.857203in}{1.280417in}}%
\pgfusepath{clip}%
\pgfsetbuttcap%
\pgfsetmiterjoin%
\definecolor{currentfill}{rgb}{0.654671,0.860438,0.643368}%
\pgfsetfillcolor{currentfill}%
\pgfsetlinewidth{0.000000pt}%
\definecolor{currentstroke}{rgb}{0.000000,0.000000,0.000000}%
\pgfsetstrokecolor{currentstroke}%
\pgfsetstrokeopacity{0.000000}%
\pgfsetdash{}{0pt}%
\pgfpathmoveto{\pgfqpoint{3.367933in}{1.069737in}}%
\pgfpathlineto{\pgfqpoint{3.333645in}{1.069737in}}%
\pgfpathlineto{\pgfqpoint{3.333645in}{1.089744in}}%
\pgfpathlineto{\pgfqpoint{3.367933in}{1.089744in}}%
\pgfpathclose%
\pgfusepath{fill}%
\end{pgfscope}%
\begin{pgfscope}%
\pgfpathrectangle{\pgfqpoint{2.887899in}{0.169444in}}{\pgfqpoint{0.857203in}{1.280417in}}%
\pgfusepath{clip}%
\pgfsetbuttcap%
\pgfsetmiterjoin%
\definecolor{currentfill}{rgb}{0.612226,0.843829,0.643983}%
\pgfsetfillcolor{currentfill}%
\pgfsetlinewidth{0.000000pt}%
\definecolor{currentstroke}{rgb}{0.000000,0.000000,0.000000}%
\pgfsetstrokecolor{currentstroke}%
\pgfsetstrokeopacity{0.000000}%
\pgfsetdash{}{0pt}%
\pgfpathmoveto{\pgfqpoint{3.367933in}{1.089744in}}%
\pgfpathlineto{\pgfqpoint{3.333645in}{1.089744in}}%
\pgfpathlineto{\pgfqpoint{3.333645in}{1.109750in}}%
\pgfpathlineto{\pgfqpoint{3.367933in}{1.109750in}}%
\pgfpathclose%
\pgfusepath{fill}%
\end{pgfscope}%
\begin{pgfscope}%
\pgfpathrectangle{\pgfqpoint{2.887899in}{0.169444in}}{\pgfqpoint{0.857203in}{1.280417in}}%
\pgfusepath{clip}%
\pgfsetbuttcap%
\pgfsetmiterjoin%
\definecolor{currentfill}{rgb}{0.569781,0.827220,0.644598}%
\pgfsetfillcolor{currentfill}%
\pgfsetlinewidth{0.000000pt}%
\definecolor{currentstroke}{rgb}{0.000000,0.000000,0.000000}%
\pgfsetstrokecolor{currentstroke}%
\pgfsetstrokeopacity{0.000000}%
\pgfsetdash{}{0pt}%
\pgfpathmoveto{\pgfqpoint{3.367933in}{1.109750in}}%
\pgfpathlineto{\pgfqpoint{3.333645in}{1.109750in}}%
\pgfpathlineto{\pgfqpoint{3.333645in}{1.129757in}}%
\pgfpathlineto{\pgfqpoint{3.367933in}{1.129757in}}%
\pgfpathclose%
\pgfusepath{fill}%
\end{pgfscope}%
\begin{pgfscope}%
\pgfpathrectangle{\pgfqpoint{2.887899in}{0.169444in}}{\pgfqpoint{0.857203in}{1.280417in}}%
\pgfusepath{clip}%
\pgfsetbuttcap%
\pgfsetmiterjoin%
\definecolor{currentfill}{rgb}{0.527336,0.810611,0.645213}%
\pgfsetfillcolor{currentfill}%
\pgfsetlinewidth{0.000000pt}%
\definecolor{currentstroke}{rgb}{0.000000,0.000000,0.000000}%
\pgfsetstrokecolor{currentstroke}%
\pgfsetstrokeopacity{0.000000}%
\pgfsetdash{}{0pt}%
\pgfpathmoveto{\pgfqpoint{3.367933in}{1.129757in}}%
\pgfpathlineto{\pgfqpoint{3.333645in}{1.129757in}}%
\pgfpathlineto{\pgfqpoint{3.333645in}{1.149763in}}%
\pgfpathlineto{\pgfqpoint{3.367933in}{1.149763in}}%
\pgfpathclose%
\pgfusepath{fill}%
\end{pgfscope}%
\begin{pgfscope}%
\pgfpathrectangle{\pgfqpoint{2.887899in}{0.169444in}}{\pgfqpoint{0.857203in}{1.280417in}}%
\pgfusepath{clip}%
\pgfsetbuttcap%
\pgfsetmiterjoin%
\definecolor{currentfill}{rgb}{0.484890,0.794002,0.645829}%
\pgfsetfillcolor{currentfill}%
\pgfsetlinewidth{0.000000pt}%
\definecolor{currentstroke}{rgb}{0.000000,0.000000,0.000000}%
\pgfsetstrokecolor{currentstroke}%
\pgfsetstrokeopacity{0.000000}%
\pgfsetdash{}{0pt}%
\pgfpathmoveto{\pgfqpoint{3.367933in}{1.149763in}}%
\pgfpathlineto{\pgfqpoint{3.333645in}{1.149763in}}%
\pgfpathlineto{\pgfqpoint{3.333645in}{1.169770in}}%
\pgfpathlineto{\pgfqpoint{3.367933in}{1.169770in}}%
\pgfpathclose%
\pgfusepath{fill}%
\end{pgfscope}%
\begin{pgfscope}%
\pgfpathrectangle{\pgfqpoint{2.887899in}{0.169444in}}{\pgfqpoint{0.857203in}{1.280417in}}%
\pgfusepath{clip}%
\pgfsetbuttcap%
\pgfsetmiterjoin%
\definecolor{currentfill}{rgb}{0.442445,0.777393,0.646444}%
\pgfsetfillcolor{currentfill}%
\pgfsetlinewidth{0.000000pt}%
\definecolor{currentstroke}{rgb}{0.000000,0.000000,0.000000}%
\pgfsetstrokecolor{currentstroke}%
\pgfsetstrokeopacity{0.000000}%
\pgfsetdash{}{0pt}%
\pgfpathmoveto{\pgfqpoint{3.367933in}{1.169770in}}%
\pgfpathlineto{\pgfqpoint{3.333645in}{1.169770in}}%
\pgfpathlineto{\pgfqpoint{3.333645in}{1.189776in}}%
\pgfpathlineto{\pgfqpoint{3.367933in}{1.189776in}}%
\pgfpathclose%
\pgfusepath{fill}%
\end{pgfscope}%
\begin{pgfscope}%
\pgfpathrectangle{\pgfqpoint{2.887899in}{0.169444in}}{\pgfqpoint{0.857203in}{1.280417in}}%
\pgfusepath{clip}%
\pgfsetbuttcap%
\pgfsetmiterjoin%
\definecolor{currentfill}{rgb}{0.400000,0.760784,0.647059}%
\pgfsetfillcolor{currentfill}%
\pgfsetlinewidth{0.000000pt}%
\definecolor{currentstroke}{rgb}{0.000000,0.000000,0.000000}%
\pgfsetstrokecolor{currentstroke}%
\pgfsetstrokeopacity{0.000000}%
\pgfsetdash{}{0pt}%
\pgfpathmoveto{\pgfqpoint{3.367933in}{1.189776in}}%
\pgfpathlineto{\pgfqpoint{3.333645in}{1.189776in}}%
\pgfpathlineto{\pgfqpoint{3.333645in}{1.209783in}}%
\pgfpathlineto{\pgfqpoint{3.367933in}{1.209783in}}%
\pgfpathclose%
\pgfusepath{fill}%
\end{pgfscope}%
\begin{pgfscope}%
\pgfpathrectangle{\pgfqpoint{2.887899in}{0.169444in}}{\pgfqpoint{0.857203in}{1.280417in}}%
\pgfusepath{clip}%
\pgfsetbuttcap%
\pgfsetmiterjoin%
\definecolor{currentfill}{rgb}{0.368012,0.725106,0.661822}%
\pgfsetfillcolor{currentfill}%
\pgfsetlinewidth{0.000000pt}%
\definecolor{currentstroke}{rgb}{0.000000,0.000000,0.000000}%
\pgfsetstrokecolor{currentstroke}%
\pgfsetstrokeopacity{0.000000}%
\pgfsetdash{}{0pt}%
\pgfpathmoveto{\pgfqpoint{3.367933in}{1.209783in}}%
\pgfpathlineto{\pgfqpoint{3.333645in}{1.209783in}}%
\pgfpathlineto{\pgfqpoint{3.333645in}{1.229789in}}%
\pgfpathlineto{\pgfqpoint{3.367933in}{1.229789in}}%
\pgfpathclose%
\pgfusepath{fill}%
\end{pgfscope}%
\begin{pgfscope}%
\pgfpathrectangle{\pgfqpoint{2.887899in}{0.169444in}}{\pgfqpoint{0.857203in}{1.280417in}}%
\pgfusepath{clip}%
\pgfsetbuttcap%
\pgfsetmiterjoin%
\definecolor{currentfill}{rgb}{0.336025,0.689427,0.676586}%
\pgfsetfillcolor{currentfill}%
\pgfsetlinewidth{0.000000pt}%
\definecolor{currentstroke}{rgb}{0.000000,0.000000,0.000000}%
\pgfsetstrokecolor{currentstroke}%
\pgfsetstrokeopacity{0.000000}%
\pgfsetdash{}{0pt}%
\pgfpathmoveto{\pgfqpoint{3.367933in}{1.229789in}}%
\pgfpathlineto{\pgfqpoint{3.333645in}{1.229789in}}%
\pgfpathlineto{\pgfqpoint{3.333645in}{1.249796in}}%
\pgfpathlineto{\pgfqpoint{3.367933in}{1.249796in}}%
\pgfpathclose%
\pgfusepath{fill}%
\end{pgfscope}%
\begin{pgfscope}%
\pgfpathrectangle{\pgfqpoint{2.887899in}{0.169444in}}{\pgfqpoint{0.857203in}{1.280417in}}%
\pgfusepath{clip}%
\pgfsetbuttcap%
\pgfsetmiterjoin%
\definecolor{currentfill}{rgb}{0.304037,0.653749,0.691349}%
\pgfsetfillcolor{currentfill}%
\pgfsetlinewidth{0.000000pt}%
\definecolor{currentstroke}{rgb}{0.000000,0.000000,0.000000}%
\pgfsetstrokecolor{currentstroke}%
\pgfsetstrokeopacity{0.000000}%
\pgfsetdash{}{0pt}%
\pgfpathmoveto{\pgfqpoint{3.367933in}{1.249796in}}%
\pgfpathlineto{\pgfqpoint{3.333645in}{1.249796in}}%
\pgfpathlineto{\pgfqpoint{3.333645in}{1.269803in}}%
\pgfpathlineto{\pgfqpoint{3.367933in}{1.269803in}}%
\pgfpathclose%
\pgfusepath{fill}%
\end{pgfscope}%
\begin{pgfscope}%
\pgfpathrectangle{\pgfqpoint{2.887899in}{0.169444in}}{\pgfqpoint{0.857203in}{1.280417in}}%
\pgfusepath{clip}%
\pgfsetbuttcap%
\pgfsetmiterjoin%
\definecolor{currentfill}{rgb}{0.272049,0.618070,0.706113}%
\pgfsetfillcolor{currentfill}%
\pgfsetlinewidth{0.000000pt}%
\definecolor{currentstroke}{rgb}{0.000000,0.000000,0.000000}%
\pgfsetstrokecolor{currentstroke}%
\pgfsetstrokeopacity{0.000000}%
\pgfsetdash{}{0pt}%
\pgfpathmoveto{\pgfqpoint{3.367933in}{1.269803in}}%
\pgfpathlineto{\pgfqpoint{3.333645in}{1.269803in}}%
\pgfpathlineto{\pgfqpoint{3.333645in}{1.289809in}}%
\pgfpathlineto{\pgfqpoint{3.367933in}{1.289809in}}%
\pgfpathclose%
\pgfusepath{fill}%
\end{pgfscope}%
\begin{pgfscope}%
\pgfpathrectangle{\pgfqpoint{2.887899in}{0.169444in}}{\pgfqpoint{0.857203in}{1.280417in}}%
\pgfusepath{clip}%
\pgfsetbuttcap%
\pgfsetmiterjoin%
\definecolor{currentfill}{rgb}{0.240062,0.582391,0.720877}%
\pgfsetfillcolor{currentfill}%
\pgfsetlinewidth{0.000000pt}%
\definecolor{currentstroke}{rgb}{0.000000,0.000000,0.000000}%
\pgfsetstrokecolor{currentstroke}%
\pgfsetstrokeopacity{0.000000}%
\pgfsetdash{}{0pt}%
\pgfpathmoveto{\pgfqpoint{3.367933in}{1.289809in}}%
\pgfpathlineto{\pgfqpoint{3.333645in}{1.289809in}}%
\pgfpathlineto{\pgfqpoint{3.333645in}{1.309816in}}%
\pgfpathlineto{\pgfqpoint{3.367933in}{1.309816in}}%
\pgfpathclose%
\pgfusepath{fill}%
\end{pgfscope}%
\begin{pgfscope}%
\pgfpathrectangle{\pgfqpoint{2.887899in}{0.169444in}}{\pgfqpoint{0.857203in}{1.280417in}}%
\pgfusepath{clip}%
\pgfsetbuttcap%
\pgfsetmiterjoin%
\definecolor{currentfill}{rgb}{0.208074,0.546713,0.735640}%
\pgfsetfillcolor{currentfill}%
\pgfsetlinewidth{0.000000pt}%
\definecolor{currentstroke}{rgb}{0.000000,0.000000,0.000000}%
\pgfsetstrokecolor{currentstroke}%
\pgfsetstrokeopacity{0.000000}%
\pgfsetdash{}{0pt}%
\pgfpathmoveto{\pgfqpoint{3.367933in}{1.309816in}}%
\pgfpathlineto{\pgfqpoint{3.333645in}{1.309816in}}%
\pgfpathlineto{\pgfqpoint{3.333645in}{1.329822in}}%
\pgfpathlineto{\pgfqpoint{3.367933in}{1.329822in}}%
\pgfpathclose%
\pgfusepath{fill}%
\end{pgfscope}%
\begin{pgfscope}%
\pgfpathrectangle{\pgfqpoint{2.887899in}{0.169444in}}{\pgfqpoint{0.857203in}{1.280417in}}%
\pgfusepath{clip}%
\pgfsetbuttcap%
\pgfsetmiterjoin%
\definecolor{currentfill}{rgb}{0.212995,0.511419,0.730796}%
\pgfsetfillcolor{currentfill}%
\pgfsetlinewidth{0.000000pt}%
\definecolor{currentstroke}{rgb}{0.000000,0.000000,0.000000}%
\pgfsetstrokecolor{currentstroke}%
\pgfsetstrokeopacity{0.000000}%
\pgfsetdash{}{0pt}%
\pgfpathmoveto{\pgfqpoint{3.367933in}{1.329822in}}%
\pgfpathlineto{\pgfqpoint{3.333645in}{1.329822in}}%
\pgfpathlineto{\pgfqpoint{3.333645in}{1.349829in}}%
\pgfpathlineto{\pgfqpoint{3.367933in}{1.349829in}}%
\pgfpathclose%
\pgfusepath{fill}%
\end{pgfscope}%
\begin{pgfscope}%
\pgfpathrectangle{\pgfqpoint{2.887899in}{0.169444in}}{\pgfqpoint{0.857203in}{1.280417in}}%
\pgfusepath{clip}%
\pgfsetbuttcap%
\pgfsetmiterjoin%
\definecolor{currentfill}{rgb}{0.240062,0.476355,0.714187}%
\pgfsetfillcolor{currentfill}%
\pgfsetlinewidth{0.000000pt}%
\definecolor{currentstroke}{rgb}{0.000000,0.000000,0.000000}%
\pgfsetstrokecolor{currentstroke}%
\pgfsetstrokeopacity{0.000000}%
\pgfsetdash{}{0pt}%
\pgfpathmoveto{\pgfqpoint{3.367933in}{1.349829in}}%
\pgfpathlineto{\pgfqpoint{3.333645in}{1.349829in}}%
\pgfpathlineto{\pgfqpoint{3.333645in}{1.369835in}}%
\pgfpathlineto{\pgfqpoint{3.367933in}{1.369835in}}%
\pgfpathclose%
\pgfusepath{fill}%
\end{pgfscope}%
\begin{pgfscope}%
\pgfpathrectangle{\pgfqpoint{2.887899in}{0.169444in}}{\pgfqpoint{0.857203in}{1.280417in}}%
\pgfusepath{clip}%
\pgfsetbuttcap%
\pgfsetmiterjoin%
\definecolor{currentfill}{rgb}{0.267128,0.441292,0.697578}%
\pgfsetfillcolor{currentfill}%
\pgfsetlinewidth{0.000000pt}%
\definecolor{currentstroke}{rgb}{0.000000,0.000000,0.000000}%
\pgfsetstrokecolor{currentstroke}%
\pgfsetstrokeopacity{0.000000}%
\pgfsetdash{}{0pt}%
\pgfpathmoveto{\pgfqpoint{3.367933in}{1.369835in}}%
\pgfpathlineto{\pgfqpoint{3.333645in}{1.369835in}}%
\pgfpathlineto{\pgfqpoint{3.333645in}{1.389842in}}%
\pgfpathlineto{\pgfqpoint{3.367933in}{1.389842in}}%
\pgfpathclose%
\pgfusepath{fill}%
\end{pgfscope}%
\begin{pgfscope}%
\pgfpathrectangle{\pgfqpoint{2.887899in}{0.169444in}}{\pgfqpoint{0.857203in}{1.280417in}}%
\pgfusepath{clip}%
\pgfsetbuttcap%
\pgfsetmiterjoin%
\definecolor{currentfill}{rgb}{0.294195,0.406228,0.680969}%
\pgfsetfillcolor{currentfill}%
\pgfsetlinewidth{0.000000pt}%
\definecolor{currentstroke}{rgb}{0.000000,0.000000,0.000000}%
\pgfsetstrokecolor{currentstroke}%
\pgfsetstrokeopacity{0.000000}%
\pgfsetdash{}{0pt}%
\pgfpathmoveto{\pgfqpoint{3.367933in}{1.389842in}}%
\pgfpathlineto{\pgfqpoint{3.333645in}{1.389842in}}%
\pgfpathlineto{\pgfqpoint{3.333645in}{1.409848in}}%
\pgfpathlineto{\pgfqpoint{3.367933in}{1.409848in}}%
\pgfpathclose%
\pgfusepath{fill}%
\end{pgfscope}%
\begin{pgfscope}%
\pgfpathrectangle{\pgfqpoint{2.887899in}{0.169444in}}{\pgfqpoint{0.857203in}{1.280417in}}%
\pgfusepath{clip}%
\pgfsetbuttcap%
\pgfsetmiterjoin%
\definecolor{currentfill}{rgb}{0.321261,0.371165,0.664360}%
\pgfsetfillcolor{currentfill}%
\pgfsetlinewidth{0.000000pt}%
\definecolor{currentstroke}{rgb}{0.000000,0.000000,0.000000}%
\pgfsetstrokecolor{currentstroke}%
\pgfsetstrokeopacity{0.000000}%
\pgfsetdash{}{0pt}%
\pgfpathmoveto{\pgfqpoint{3.367933in}{1.409848in}}%
\pgfpathlineto{\pgfqpoint{3.333645in}{1.409848in}}%
\pgfpathlineto{\pgfqpoint{3.333645in}{1.429855in}}%
\pgfpathlineto{\pgfqpoint{3.367933in}{1.429855in}}%
\pgfpathclose%
\pgfusepath{fill}%
\end{pgfscope}%
\begin{pgfscope}%
\pgfpathrectangle{\pgfqpoint{2.887899in}{0.169444in}}{\pgfqpoint{0.857203in}{1.280417in}}%
\pgfusepath{clip}%
\pgfsetbuttcap%
\pgfsetmiterjoin%
\definecolor{currentfill}{rgb}{0.348328,0.336101,0.647751}%
\pgfsetfillcolor{currentfill}%
\pgfsetlinewidth{0.000000pt}%
\definecolor{currentstroke}{rgb}{0.000000,0.000000,0.000000}%
\pgfsetstrokecolor{currentstroke}%
\pgfsetstrokeopacity{0.000000}%
\pgfsetdash{}{0pt}%
\pgfpathmoveto{\pgfqpoint{3.367933in}{1.429855in}}%
\pgfpathlineto{\pgfqpoint{3.333645in}{1.429855in}}%
\pgfpathlineto{\pgfqpoint{3.333645in}{1.449861in}}%
\pgfpathlineto{\pgfqpoint{3.367933in}{1.449861in}}%
\pgfpathclose%
\pgfusepath{fill}%
\end{pgfscope}%
\begin{pgfscope}%
\pgfpathrectangle{\pgfqpoint{2.887899in}{0.169444in}}{\pgfqpoint{0.857203in}{1.280417in}}%
\pgfusepath{clip}%
\pgfsetbuttcap%
\pgfsetmiterjoin%
\definecolor{currentfill}{rgb}{0.368627,0.309804,0.635294}%
\pgfsetfillcolor{currentfill}%
\pgfsetlinewidth{0.000000pt}%
\definecolor{currentstroke}{rgb}{0.000000,0.000000,0.000000}%
\pgfsetstrokecolor{currentstroke}%
\pgfsetstrokeopacity{0.000000}%
\pgfsetdash{}{0pt}%
\pgfpathmoveto{\pgfqpoint{3.367933in}{1.449861in}}%
\pgfpathlineto{\pgfqpoint{3.333645in}{1.449861in}}%
\pgfpathlineto{\pgfqpoint{3.333645in}{1.469868in}}%
\pgfpathlineto{\pgfqpoint{3.367933in}{1.469868in}}%
\pgfpathclose%
\pgfusepath{fill}%
\end{pgfscope}%
\begin{pgfscope}%
\pgfpathrectangle{\pgfqpoint{2.887899in}{0.169444in}}{\pgfqpoint{0.857203in}{1.280417in}}%
\pgfusepath{clip}%
\pgfsetbuttcap%
\pgfsetmiterjoin%
\definecolor{currentfill}{rgb}{0.998078,0.999231,0.746021}%
\pgfsetfillcolor{currentfill}%
\pgfsetlinewidth{0.000000pt}%
\definecolor{currentstroke}{rgb}{0.000000,0.000000,0.000000}%
\pgfsetstrokecolor{currentstroke}%
\pgfsetstrokeopacity{0.000000}%
\pgfsetdash{}{0pt}%
\pgfpathmoveto{\pgfqpoint{3.410793in}{0.809653in}}%
\pgfpathlineto{\pgfqpoint{3.427937in}{0.809653in}}%
\pgfpathlineto{\pgfqpoint{3.427937in}{0.849666in}}%
\pgfpathlineto{\pgfqpoint{3.410793in}{0.849666in}}%
\pgfpathclose%
\pgfusepath{fill}%
\end{pgfscope}%
\begin{pgfscope}%
\pgfpathrectangle{\pgfqpoint{2.887899in}{0.169444in}}{\pgfqpoint{0.857203in}{1.280417in}}%
\pgfusepath{clip}%
\pgfsetbuttcap%
\pgfsetmiterjoin%
\definecolor{currentfill}{rgb}{0.905805,0.962322,0.602076}%
\pgfsetfillcolor{currentfill}%
\pgfsetlinewidth{0.000000pt}%
\definecolor{currentstroke}{rgb}{0.000000,0.000000,0.000000}%
\pgfsetstrokecolor{currentstroke}%
\pgfsetstrokeopacity{0.000000}%
\pgfsetdash{}{0pt}%
\pgfpathmoveto{\pgfqpoint{3.410793in}{0.929692in}}%
\pgfpathlineto{\pgfqpoint{3.462226in}{0.929692in}}%
\pgfpathlineto{\pgfqpoint{3.462226in}{0.969705in}}%
\pgfpathlineto{\pgfqpoint{3.410793in}{0.969705in}}%
\pgfpathclose%
\pgfusepath{fill}%
\end{pgfscope}%
\begin{pgfscope}%
\pgfpathrectangle{\pgfqpoint{2.887899in}{0.169444in}}{\pgfqpoint{0.857203in}{1.280417in}}%
\pgfusepath{clip}%
\pgfsetbuttcap%
\pgfsetmiterjoin%
\definecolor{currentfill}{rgb}{0.838447,0.934948,0.608997}%
\pgfsetfillcolor{currentfill}%
\pgfsetlinewidth{0.000000pt}%
\definecolor{currentstroke}{rgb}{0.000000,0.000000,0.000000}%
\pgfsetstrokecolor{currentstroke}%
\pgfsetstrokeopacity{0.000000}%
\pgfsetdash{}{0pt}%
\pgfpathmoveto{\pgfqpoint{3.410793in}{0.969705in}}%
\pgfpathlineto{\pgfqpoint{3.513658in}{0.969705in}}%
\pgfpathlineto{\pgfqpoint{3.513658in}{1.009718in}}%
\pgfpathlineto{\pgfqpoint{3.410793in}{1.009718in}}%
\pgfpathclose%
\pgfusepath{fill}%
\end{pgfscope}%
\begin{pgfscope}%
\pgfpathrectangle{\pgfqpoint{2.887899in}{0.169444in}}{\pgfqpoint{0.857203in}{1.280417in}}%
\pgfusepath{clip}%
\pgfsetbuttcap%
\pgfsetmiterjoin%
\definecolor{currentfill}{rgb}{0.765859,0.905421,0.623760}%
\pgfsetfillcolor{currentfill}%
\pgfsetlinewidth{0.000000pt}%
\definecolor{currentstroke}{rgb}{0.000000,0.000000,0.000000}%
\pgfsetstrokecolor{currentstroke}%
\pgfsetstrokeopacity{0.000000}%
\pgfsetdash{}{0pt}%
\pgfpathmoveto{\pgfqpoint{3.410793in}{1.009718in}}%
\pgfpathlineto{\pgfqpoint{3.736531in}{1.009718in}}%
\pgfpathlineto{\pgfqpoint{3.736531in}{1.049731in}}%
\pgfpathlineto{\pgfqpoint{3.410793in}{1.049731in}}%
\pgfpathclose%
\pgfusepath{fill}%
\end{pgfscope}%
\begin{pgfscope}%
\pgfpathrectangle{\pgfqpoint{2.887899in}{0.169444in}}{\pgfqpoint{0.857203in}{1.280417in}}%
\pgfusepath{clip}%
\pgfsetbuttcap%
\pgfsetmiterjoin%
\definecolor{currentfill}{rgb}{0.693272,0.875894,0.638524}%
\pgfsetfillcolor{currentfill}%
\pgfsetlinewidth{0.000000pt}%
\definecolor{currentstroke}{rgb}{0.000000,0.000000,0.000000}%
\pgfsetstrokecolor{currentstroke}%
\pgfsetstrokeopacity{0.000000}%
\pgfsetdash{}{0pt}%
\pgfpathmoveto{\pgfqpoint{3.410793in}{1.049731in}}%
\pgfpathlineto{\pgfqpoint{3.633666in}{1.049731in}}%
\pgfpathlineto{\pgfqpoint{3.633666in}{1.089744in}}%
\pgfpathlineto{\pgfqpoint{3.410793in}{1.089744in}}%
\pgfpathclose%
\pgfusepath{fill}%
\end{pgfscope}%
\begin{pgfscope}%
\pgfpathrectangle{\pgfqpoint{2.887899in}{0.169444in}}{\pgfqpoint{0.857203in}{1.280417in}}%
\pgfusepath{clip}%
\pgfsetbuttcap%
\pgfsetmiterjoin%
\definecolor{currentfill}{rgb}{0.612226,0.843829,0.643983}%
\pgfsetfillcolor{currentfill}%
\pgfsetlinewidth{0.000000pt}%
\definecolor{currentstroke}{rgb}{0.000000,0.000000,0.000000}%
\pgfsetstrokecolor{currentstroke}%
\pgfsetstrokeopacity{0.000000}%
\pgfsetdash{}{0pt}%
\pgfpathmoveto{\pgfqpoint{3.410793in}{1.089744in}}%
\pgfpathlineto{\pgfqpoint{3.496514in}{1.089744in}}%
\pgfpathlineto{\pgfqpoint{3.496514in}{1.129757in}}%
\pgfpathlineto{\pgfqpoint{3.410793in}{1.129757in}}%
\pgfpathclose%
\pgfusepath{fill}%
\end{pgfscope}%
\begin{pgfscope}%
\pgfpathrectangle{\pgfqpoint{2.887899in}{0.169444in}}{\pgfqpoint{0.857203in}{1.280417in}}%
\pgfusepath{clip}%
\pgfsetbuttcap%
\pgfsetmiterjoin%
\definecolor{currentfill}{rgb}{0.527336,0.810611,0.645213}%
\pgfsetfillcolor{currentfill}%
\pgfsetlinewidth{0.000000pt}%
\definecolor{currentstroke}{rgb}{0.000000,0.000000,0.000000}%
\pgfsetstrokecolor{currentstroke}%
\pgfsetstrokeopacity{0.000000}%
\pgfsetdash{}{0pt}%
\pgfpathmoveto{\pgfqpoint{3.410793in}{1.129757in}}%
\pgfpathlineto{\pgfqpoint{3.479370in}{1.129757in}}%
\pgfpathlineto{\pgfqpoint{3.479370in}{1.169770in}}%
\pgfpathlineto{\pgfqpoint{3.410793in}{1.169770in}}%
\pgfpathclose%
\pgfusepath{fill}%
\end{pgfscope}%
\begin{pgfscope}%
\pgfpathrectangle{\pgfqpoint{2.887899in}{0.169444in}}{\pgfqpoint{0.857203in}{1.280417in}}%
\pgfusepath{clip}%
\pgfsetbuttcap%
\pgfsetmiterjoin%
\definecolor{currentfill}{rgb}{0.442445,0.777393,0.646444}%
\pgfsetfillcolor{currentfill}%
\pgfsetlinewidth{0.000000pt}%
\definecolor{currentstroke}{rgb}{0.000000,0.000000,0.000000}%
\pgfsetstrokecolor{currentstroke}%
\pgfsetstrokeopacity{0.000000}%
\pgfsetdash{}{0pt}%
\pgfpathmoveto{\pgfqpoint{3.410793in}{1.169770in}}%
\pgfpathlineto{\pgfqpoint{3.513658in}{1.169770in}}%
\pgfpathlineto{\pgfqpoint{3.513658in}{1.209783in}}%
\pgfpathlineto{\pgfqpoint{3.410793in}{1.209783in}}%
\pgfpathclose%
\pgfusepath{fill}%
\end{pgfscope}%
\begin{pgfscope}%
\pgfpathrectangle{\pgfqpoint{2.887899in}{0.169444in}}{\pgfqpoint{0.857203in}{1.280417in}}%
\pgfusepath{clip}%
\pgfsetbuttcap%
\pgfsetmiterjoin%
\definecolor{currentfill}{rgb}{0.368012,0.725106,0.661822}%
\pgfsetfillcolor{currentfill}%
\pgfsetlinewidth{0.000000pt}%
\definecolor{currentstroke}{rgb}{0.000000,0.000000,0.000000}%
\pgfsetstrokecolor{currentstroke}%
\pgfsetstrokeopacity{0.000000}%
\pgfsetdash{}{0pt}%
\pgfpathmoveto{\pgfqpoint{3.410793in}{1.209783in}}%
\pgfpathlineto{\pgfqpoint{3.445081in}{1.209783in}}%
\pgfpathlineto{\pgfqpoint{3.445081in}{1.249796in}}%
\pgfpathlineto{\pgfqpoint{3.410793in}{1.249796in}}%
\pgfpathclose%
\pgfusepath{fill}%
\end{pgfscope}%
\begin{pgfscope}%
\pgfpathrectangle{\pgfqpoint{2.887899in}{0.169444in}}{\pgfqpoint{0.857203in}{1.280417in}}%
\pgfusepath{clip}%
\pgfsetbuttcap%
\pgfsetmiterjoin%
\definecolor{currentfill}{rgb}{0.304037,0.653749,0.691349}%
\pgfsetfillcolor{currentfill}%
\pgfsetlinewidth{0.000000pt}%
\definecolor{currentstroke}{rgb}{0.000000,0.000000,0.000000}%
\pgfsetstrokecolor{currentstroke}%
\pgfsetstrokeopacity{0.000000}%
\pgfsetdash{}{0pt}%
\pgfpathmoveto{\pgfqpoint{3.410793in}{1.249796in}}%
\pgfpathlineto{\pgfqpoint{3.445081in}{1.249796in}}%
\pgfpathlineto{\pgfqpoint{3.445081in}{1.289809in}}%
\pgfpathlineto{\pgfqpoint{3.410793in}{1.289809in}}%
\pgfpathclose%
\pgfusepath{fill}%
\end{pgfscope}%
\begin{pgfscope}%
\definecolor{textcolor}{rgb}{0.000000,0.000000,0.000000}%
\pgfsetstrokecolor{textcolor}%
\pgfsetfillcolor{textcolor}%
\pgftext[x=3.230781in,y=0.169444in,,]{\color{textcolor}\setmainfont{Lato}\rmfamily\fontsize{8.000000}{9.600000}\selectfont -12}%
\end{pgfscope}%
\begin{pgfscope}%
\definecolor{textcolor}{rgb}{0.000000,0.000000,0.000000}%
\pgfsetstrokecolor{textcolor}%
\pgfsetfillcolor{textcolor}%
\pgftext[x=3.230781in,y=0.425528in,,]{\color{textcolor}\setmainfont{Lato}\rmfamily\fontsize{8.000000}{9.600000}\selectfont -8}%
\end{pgfscope}%
\begin{pgfscope}%
\definecolor{textcolor}{rgb}{0.000000,0.000000,0.000000}%
\pgfsetstrokecolor{textcolor}%
\pgfsetfillcolor{textcolor}%
\pgftext[x=3.230781in,y=0.617590in,,]{\color{textcolor}\setmainfont{Lato}\rmfamily\fontsize{8.000000}{9.600000}\selectfont -5}%
\end{pgfscope}%
\begin{pgfscope}%
\definecolor{textcolor}{rgb}{0.000000,0.000000,0.000000}%
\pgfsetstrokecolor{textcolor}%
\pgfsetfillcolor{textcolor}%
\pgftext[x=3.230781in,y=0.937694in,,]{\color{textcolor}\setmainfont{Lato}\rmfamily\fontsize{8.000000}{9.600000}\selectfont 0}%
\end{pgfscope}%
\begin{pgfscope}%
\definecolor{textcolor}{rgb}{0.000000,0.000000,0.000000}%
\pgfsetstrokecolor{textcolor}%
\pgfsetfillcolor{textcolor}%
\pgftext[x=3.230781in,y=1.257799in,,]{\color{textcolor}\setmainfont{Lato}\rmfamily\fontsize{8.000000}{9.600000}\selectfont 5}%
\end{pgfscope}%
\begin{pgfscope}%
\definecolor{textcolor}{rgb}{0.000000,0.000000,0.000000}%
\pgfsetstrokecolor{textcolor}%
\pgfsetfillcolor{textcolor}%
\pgftext[x=3.230781in,y=1.449861in,,]{\color{textcolor}\setmainfont{Lato}\rmfamily\fontsize{8.000000}{9.600000}\selectfont 8}%
\end{pgfscope}%
\end{pgfpicture}%
\makeatother%
\endgroup%
 \\
\footnotesize{Source: Bureau of Economic Analysis}\\

\vspace{2mm}

\begin{minipage}{0.76\textwidth}

\small \input{text/gdp_state.txt} \\

\end{minipage}

\vspace{2mm}

\noindent \normalsize \textbf{Real GDP Growth by State}\\
\footnotesize{\textit{quarterly growth at seasonally adjusted annualized rate \hspace{20mm} total growth, \input{text/gdp_state_date.txt}}\\ 

\vspace{-4.5mm}
\hspace{-2mm} \noindent \rowcolors{1}{}{black!5} \setlength{\tabcolsep}{3.7pt} \color{black!90}
		{\renewcommand{\arraystretch}{1.44}
		 \begin{tabular}{p{30mm} R{7mm} R{7mm} R{7mm} R{7mm} R{7mm} p{0mm} R{9mm} R{9mm} R{10mm} }
 & 2020 Q3 & '20 Q2 & '20 Q1 & '19 Q4 & '19 Q3 & & 1-year* & 3-year & 10-year \\
\textbf{United States}  & 33.4 & -31.4 & -5.0 & 2.4 & 2.6 &  & -2.8 & 3.0 & 19.5 \\
\hspace{1mm} \textbf{Pacific}  & 32.2 & -30.9 & -4.1 & 5.8 & 2.3 &  & -1.9 & 6.0 & 34.1 \\
\hspace{3mm}  Washington  & 36.6 & -25.5 & -2.6 & 2.9 & 5.4 &  & 0.5 & 14.6 & 45.4 \\
\hspace{3mm}  California  & 31.2 & -31.5 & -4.3 & 6.6 & 1.6 &  & -2.1 & 5.0 & 34.1 \\
\hspace{3mm}  Oregon  & 35.1 & -31.9 & -4.0 & 4.9 & 3.4 &  & -1.9 & 6.2 & 31.9 \\
\hspace{3mm}  Hawaii  & 31.3 & -42.2 & -8.9 & 2.7 & 1.5 &  & -8.2 & -6.5 & 7.3 \\
\hspace{3mm}  Alaska  & 32.2 & -33.8 & -6.0 & -0.5 & 3.6 &  & -4.9 & -3.5 & -6.5 \\
\hspace{1mm} \textbf{Mountain}  & 33.3 & -29.0 & -3.3 & 3.4 & 5.5 &  & -1.4 & 7.7 & 26.1 \\
\hspace{3mm}  Utah  & 34.4 & -22.4 & -3.9 & 4.0 & 6.7 &  & 1.1 & 12.4 & 40.6 \\
\hspace{3mm}  Colorado  & 30.1 & -28.1 & -1.3 & 2.5 & 6.0 &  & -1.4 & 8.8 & 34.7 \\
\hspace{3mm}  Idaho  & 43.3 & -32.4 & -1.9 & 5.6 & 4.7 &  & 0.1 & 10.4 & 30.6 \\
\hspace{3mm}  Arizona  & 31.1 & -25.3 & -3.2 & 4.9 & 4.7 &  & -0.1 & 8.3 & 26.7 \\
\hspace{3mm}  Nevada  & 52.2 & -42.2 & -4.9 & 2.7 & 4.7 &  & -3.7 & 4.1 & 15.8 \\
\hspace{3mm}  Montana  & 30.8 & -30.8 & -4.8 & 3.5 & 6.0 &  & -2.8 & 2.3 & 15.8 \\
\hspace{3mm}  New Mexico  & 23.6 & -28.3 & -4.7 & 1.7 & 6.7 &  & -3.7 & 4.9 & 11.3 \\
\hspace{3mm}  Wyoming  & 19.4 & -32.5 & -10.5 & -0.2 & 4.2 &  & -7.9 & -4.9 & -9.0 \\
\hspace{1mm} \textbf{West South Central}  & 29.7 & -29.4 & -6.8 & 0.6 & 4.3 &  & -3.7 & 3.2 & 24.0 \\
\hspace{3mm}  Texas  & 29.7 & -29.0 & -6.2 & 0.7 & 4.8 &  & -3.4 & 4.3 & 32.3 \\
\hspace{3mm}  Oklahoma  & 24.2 & -31.1 & -7.3 & -2.6 & 1.6 &  & -6.3 & -1.4 & 16.5 \\
\hspace{3mm}  Arkansas  & 31.8 & -27.9 & -4.0 & 1.9 & 1.8 &  & -1.8 & 0.6 & 10.1 \\
\multicolumn{3}{l}{continued on next page . . .} & &  & & & & & \\
\end{tabular} } \\ \newpage

\hspace{-2mm} \noindent \rowcolors{1}{}{black!5} 
            \setlength{\tabcolsep}{3.8pt} \color{black!90}
            {\renewcommand{\arraystretch}{1.44}
             \begin{tabular}{p{30mm} R{7mm} R{7mm} R{7mm} R{7mm} 
             R{7mm} p{0mm} R{9mm} R{9mm} R{10mm} }
 & 2020 Q3 & '20 Q2 & '20 Q1 & '19 Q4 & '19 Q3 & & 1-year* & 3-year & 10-year \\
\multicolumn{3}{l}{continued from previous page . . .}  & &  & & & & & \\
\hspace{3mm}  Louisiana  & 33.1 & -31.4 & -11.9 & 1.4 & 4.3 &  & -5.0 & 0.2 & -8.1 \\
\hspace{1mm} \textbf{South Atlantic}  & 32.2 & -28.7 & -4.5 & 2.6 & 2.9 &  & -2.0 & 3.8 & 18.2 \\
\hspace{3mm}  Georgia  & 32.7 & -27.7 & -4.0 & 1.6 & 2.4 &  & -1.7 & 4.8 & 26.2 \\
\hspace{3mm}  Florida  & 33.4 & -30.1 & -4.3 & 3.4 & 3.0 &  & -2.0 & 5.5 & 23.7 \\
\hspace{3mm}  South Carolina  & 38.5 & -32.6 & -8.2 & 2.7 & 4.7 &  & -3.1 & 3.4 & 23.0 \\
\hspace{3mm}  North Carolina  & 35.7 & -30.5 & -3.5 & 2.7 & 2.9 &  & -1.7 & 3.7 & 16.8 \\
\hspace{3mm}  Maryland  & 29.2 & -27.7 & -3.6 & 2.8 & 1.8 &  & -1.9 & 0.8 & 12.9 \\
\hspace{3mm}  District of Columbia  & 19.2 & -20.4 & -1.2 & 2.7 & 1.7 &  & -0.9 & 2.9 & 12.0 \\
\hspace{3mm}  Virginia  & 29.5 & -27.0 & -5.0 & 2.7 & 4.5 &  & -2.0 & 3.0 & 10.0 \\
\hspace{3mm}  Delaware  & 27.6 & -21.9 & -11.4 & 1.7 & -0.4 &  & -2.7 & 2.5 & 4.5 \\
\hspace{3mm}  West Virginia  & 30.5 & -29.6 & -6.7 & -4.9 & 0.1 &  & -5.0 & -0.7 & 1.4 \\
\hspace{1mm} \textbf{West North Central}  & 35.0 & -30.6 & -4.5 & 1.8 & 2.6 &  & -2.3 & 1.3 & 14.3 \\
\hspace{3mm}  North Dakota  & 22.4 & -27.6 & -1.6 & -0.2 & -0.8 &  & -3.4 & 0.8 & 38.2 \\
\hspace{3mm}  South Dakota  & 32.1 & -28.8 & -2.4 & 2.4 & 4.5 &  & -1.5 & 1.6 & 18.4 \\
\hspace{3mm}  Nebraska  & 33.2 & -31.0 & -3.4 & 4.1 & 5.8 &  & -2.0 & 1.7 & 18.4 \\
\hspace{3mm}  Minnesota  & 36.3 & -31.3 & -6.8 & 2.4 & 2.7 &  & -2.8 & 1.5 & 17.4 \\
\hspace{3mm}  Kansas  & 34.3 & -30.3 & -3.5 & 2.4 & 0.9 &  & -1.9 & 1.9 & 17.0 \\
\hspace{3mm}  Iowa  & 36.4 & -28.2 & -2.0 & -1.3 & 2.4 &  & -1.4 & 1.6 & 14.5 \\
\hspace{3mm}  Missouri  & 36.7 & -32.1 & -5.1 & 1.8 & 2.7 &  & -2.7 & 0.5 & 4.2 \\
\hspace{1mm} \textbf{East North Central}  & 38.7 & -32.8 & -6.6 & 1.1 & 2.6 &  & -3.2 & 0.9 & 12.3 \\
\hspace{3mm}  Ohio  & 36.9 & -33.0 & -5.6 & 1.5 & 2.8 &  & -3.2 & 1.0 & 15.5 \\
\hspace{3mm}  Wisconsin  & 40.3 & -32.6 & -8.8 & 2.9 & 0.2 &  & -2.9 & 2.2 & 13.4 \\
\hspace{3mm}  Michigan  & 44.2 & -37.6 & -7.9 & 0.9 & 3.4 &  & -4.4 & -0.8 & 13.2 \\
\hspace{3mm}  Indiana  & 43.3 & -33.0 & -5.2 & 1.9 & 3.3 &  & -1.9 & 3.3 & 12.7 \\
\hspace{3mm}  Illinois  & 34.5 & -29.7 & -6.3 & -0.2 & 2.8 &  & -3.0 & 0.2 & 8.7 \\
\hspace{1mm} \textbf{East South Central}  & 41.4 & -35.6 & -4.0 & 1.2 & 2.7 &  & -3.0 & 1.3 & 11.3 \\
\hspace{3mm}  Tennessee  & 46.5 & -40.4 & -3.9 & -0.1 & 2.5 &  & -4.3 & 0.8 & 18.8 \\
\hspace{3mm}  Alabama  & 34.6 & -29.6 & -3.2 & 1.3 & 2.8 &  & -1.8 & 2.5 & 8.7 \\
\hspace{3mm}  Kentucky  & 41.2 & -34.5 & -5.0 & 2.3 & 2.7 &  & -2.6 & 1.4 & 8.4 \\
\hspace{3mm}  Mississippi  & 39.5 & -32.9 & -3.7 & 2.9 & 3.3 &  & -1.9 & 0.3 & 0.7 \\
\hspace{1mm} \textbf{New England}  & 34.2 & -32.3 & -4.8 & 1.0 & 1.6 &  & -3.3 & 1.1 & 10.0 \\
\hspace{3mm}  Massachusetts  & 33.1 & -31.6 & -4.3 & 0.4 & 2.3 &  & -3.3 & 3.3 & 19.3 \\
\hspace{3mm}  New Hampshire  & 40.9 & -36.9 & -2.2 & -0.5 & 0.3 &  & -3.6 & 1.1 & 11.3 \\
\hspace{3mm}  Maine  & 37.3 & -34.4 & -6.5 & 3.4 & 4.6 &  & -3.4 & 2.3 & 5.8 \\
\hspace{3mm}  Vermont  & 43.0 & -38.2 & -5.8 & 0.9 & 1.5 &  & -4.3 & -2.3 & 1.6 \\
\hspace{3mm}  Rhode Island  & 35.5 & -32.4 & -5.2 & 2.3 & -2.4 &  & -2.9 & -1.3 & 1.0 \\
\hspace{3mm}  Connecticut  & 32.6 & -31.1 & -6.0 & 1.9 & 0.6 &  & -3.3 & -2.4 & -2.5 \\
\hspace{1mm} \textbf{Middle Atlantic}  & 33.0 & -35.6 & -5.5 & 1.4 & 0.9 &  & -4.8 & 0.0 & 9.2 \\
\hspace{3mm}  Pennsylvania  & 35.5 & -34.0 & -5.8 & 1.4 & 2.5 &  & -3.9 & 1.2 & 13.1 \\
\hspace{3mm}  New York  & 30.3 & -36.3 & -6.2 & 1.3 & -0.5 &  & -5.8 & -0.5 & 9.3 \\
\hspace{3mm}  New Jersey  & 37.2 & -35.6 & -3.3 & 1.5 & 2.5 &  & -3.5 & -0.1 & 4.5 \\
 \hline
		\end{tabular}
		}	\\

\vspace{-3mm}	
\footnotesize{Source: Bureau of Economic Analysis}


\begin{minipage}{0.76\textwidth}

\section*{\color{darkgray}\LARGE \seriffont Financial Accounts}

\small A high-level overview of US financial activities can be provided by dividing the world economy into three sectors: the US private sector (see\cbox{green!70!black}), the US government (see\cbox{yellow!70!orange}), and the rest of the world (see\cbox{blue!90!black}), then examining the net lending and borrowing between the groups, which must sum to zero at an aggregate level. That is, if one sector is running a deficit, another sector must be running a surplus.\\

\vspace{2mm}

\noindent \normalsize \textbf{Sectoral Financial Balance}\\
\footnotesize{\textit{net lending (+) or borrowing (-), NIPA basis, by sector, as share of GDP}}\\
\noindent \hspace*{-3mm} \begin{tikzpicture}
	\begin{axis}[\bbar{y}{0}, \dateaxisticks ytick={-10, 0, 10},
		xticklabel={`\short{\year}}, yticklabel style={text width=1.5em}, clip=false, 
		legend style={at={(0.95, 1.13)}}]
	\rbars
	\sbar{green!70!black}{date}{PRIV}{data/sectbal.csv}
	\sbar{yellow!70!orange}{date}{GOV}{data/sectbal.csv}
	\sbar{blue!90!black}{date}{ROW}{data/sectbal.csv}
    \node[align=left] (source) at (axis cs:2017-06-15,10.5) {%
      	\tiny TCJA repatriation};
    \node (destination) at (axis cs:2017-11-01,3.1) {};
    \draw[-] (source)--(destination);
	\legend{Private, Government, Rest of World};
	\end{axis}
\end{tikzpicture}\\
\footnotesize{Source: Bureau of Economic Analysis}

\vspace{4mm}

\small \input{text/sectbal.txt}


\vspace{5mm}

New borrowing by sector\\


\subsection*{\color{black!70} \seriffont Wealth}

\small \textbf{Total US wealth} is the tangible assets of all non-corporate sectors of the US, plus the market value of domestic corporate equities, less US financial obligations to the rest of the world. \input{text/wealthgdp.txt}\\

\vspace{2mm}


\noindent \normalsize \textbf{Total US Wealth to GDP Ratio}\\
\footnotesize{\textit{total US wealth divided by GDP}}\\
\noindent \hspace*{-3mm} \begin{tikzpicture}
	\begin{axis}[\bbar{y}{0}, \dateaxisticks ytick={0, 1, 2, 3, 4, 5}, ymin=-0.2,
		xticklabel={`\short{\year}}, clip=false, 
		legend style={at={(0.95, 1.13)}}]
	\rbars
	\sbar{cyan!35!white}{date}{Other}{data/wealthgdp.csv}
	\sbar{green!80!blue}{date}{Real Estate}{data/wealthgdp.csv}
	\sbar{magenta!50!violet}{date}{Corporate Equities}{data/wealthgdp.csv}
	\legend{Other, Residential Real Estate, Corporate Equities};
	\end{axis}
\end{tikzpicture}\\
\footnotesize{Source: Federal Reserve}


\end{minipage}

\newpage
\section*{\color{darkgray}\LARGE \seriffont Households}

\begin{minipage}{0.76\textwidth}

\small This section covers the household sector of the economy loosely defined, and touches on demographics, personal income and outlays, residential fixed investment, household balance sheets, home ownership, housing prices, and housing construction and permitting.

\vspace{2mm}

[Table or chart on population]

\subsection*{\color{black!70} \seriffont Demographics and Household Formation}




\small The rate of household formation since 1989 can offer a high-level overview of some major demographic and economic developments. From 1989 to 1994, \\

This section should capture 1) population, 2) population growth, 3) aging, 4) increased education.\\

\vspace{2mm}


\noindent \normalsize \textbf{Household Formation by Type}\\
\footnotesize{\textit{one-year moving average of annual growth rates}}\\
\noindent \hspace*{-2mm} \begin{tikzpicture}
	\begin{axis}[\bbar{y}{0}, \dateaxisticks ytick={0, 1, 2},
		xticklabel={`\short{\year}}, clip=false, 
		legend style={at={(0.95, 1.13)}}]
	\rbars
	\sbar{magenta!90!blue}{date}{Renter}{data/hhform.csv}
	\sbar{yellow!60!orange}{date}{Owner}{data/hhform.csv}
	\stdline{black}{date}{pop}{data/hhform.csv}
	\legend{Rented, Owned, Population Growth};
	\end{axis}
\end{tikzpicture}\\
\footnotesize{Source: Census Bureau}
\vspace{6mm}


\subsection*{\color{black!70} \seriffont Income to Persons}
\small This section looks at income received by people, by type of income, adjusted for inflation using the PCE implicit price deflator. Income is divided into labor income (see\cbox{green!80!black}), which is measured as compensation of employees, capital income (see\cbox{orange!50!yellow}), measured as the sum of proprietor income, rental income, and dividend and interest income, and welfare income  (see\cbox{blue!80!white}), which is measured as transfers to persons less contributions to social insurance. 
\vspace{4mm}

\noindent \normalsize \textbf{Personal Income}\\
\footnotesize{\textit{percentage point contribution to real personal income growth}}\\
\noindent \hspace*{-2mm} \begin{tikzpicture}
	\begin{axis}[\bbar{y}{0}, \dateaxisticks ytick={-10, -5, 0, 5, 10},
		xticklabel={`\short{\year}}, clip=false, yticklabel style={text width=1.5em}, 
		legend style={at={(0.95, 1.13)}}]
	\rbars
	\sbar{green!80!black}{date}{A033RC}{data/pi.csv}
	\sbar{orange!50!yellow}{date}{CAPITAL}{data/pi.csv}
	\sbar{blue!80!white}{date}{TRANSFER}{data/pi.csv}
	\legend{Labor, Capital$^1$, Welfare$^2$};
	\stdnode{3.4cm}{1.1cm}{\scriptsize $^1$ Includes proprietor, rent, and asset income\\ \scriptsize $^2$ Current transfer receipts minus contributions\\\scriptsize \ \ \ \ to government social insurance}
	\end{axis}
\end{tikzpicture}\\
\footnotesize{Source: Bureau of Economic Analysis}

\end{minipage}

\newpage 

\begin{minipage}{0.76\textwidth}

\normalsize
[Gross Labor Income text and chart] \\

Capital Income \\

Welfare Income \\

[Breakout section on income of the aged] \\

[Income to persons detailed table]\\

\end{minipage}

\newpage 

\begin{minipage}{0.76\textwidth}
\subsection*{\color{black!70} \seriffont Household Expenditures}

\small This section covers household spending on goods (see\cbox{red}), services excluding housing and utilities (see\cbox{blue!75!white}), and shelter (see\cbox{green!85!blue}, calculated as housing services and utilities combined with residential fixed investment). \input{text/pce1.txt}
\vspace{3mm}

\noindent \normalsize \textbf{Consumer Spending and Residential Investment}\\
\footnotesize{\textit{percentage point contribution to GDP growth}}\\
\noindent \hspace*{-2mm} \begin{tikzpicture}
	\begin{axis}[\bbar{y}{0}, \dateaxisticks ytick={-5, 0, 5},
		xticklabel={`\short{\year}}, clip=false, 
		legend style={at={(0.95, 1.13)}}]
	\rbars
	\sbar{red}{date}{DGDSRY}{data/pce.csv}
	\sbar{blue!75!white}{date}{OTHSER}{data/pce.csv}
	\sbar{green!85!blue}{date}{HOUSING}{data/pce.csv}
	\legend{Goods, Services, Shelter$^1$};
	\stdnode{2.2cm}{0.3cm}{\footnotesize $^1$ Includes residential fixed investment}
	\end{axis}
\end{tikzpicture}\\
\footnotesize{Source: Bureau of Economic Analysis}\\

\vspace{2mm}

\small

\input{text/pce2.txt} \\

\end{minipage}

\noindent \normalsize \textbf{Consumer Spending and Residential Investment}\\
\footnotesize{\textit{percentage point contribution to real GDP growth \hspace{36mm} moving averages}\\ \vspace{4mm}
\noindent \rowcolors{1}{}{black!5} \setlength{\tabcolsep}{3.1pt} \color{black!90}
		{\renewcommand{\arraystretch}{1.55}
		 \begin{tabular}{p{2.0mm} p{38mm} R{6.7mm} R{6.7mm} R{6.7mm} R{6.7mm} R{6.7mm} 
		   R{7.8mm} R{6.7mm} R{6.7mm} }
			 & &  2019  Q3 & `19  Q2 & `19  Q1 & `18  Q4 & `18  Q3 &3-year&10-year&30-year\\  &  Total &1.98&3.00&1.32&0.56&2.21&1.83&1.60&1.73\\  \cbox{red}  &  Goods &1.17&1.74&0.32&0.33&0.75&0.86&0.77&0.76\\  &  \hspace{1mm}  Motor  Vehicles  and  Parts &0.07&0.37&-0.27&0.07&0.01&0.11&0.12&0.08\\  &  \hspace{1mm}  Furniture  and  HH  Equipment &0.10&0.14&0.03&-0.09&0.09&0.10&0.10&0.08\\  &  \hspace{1mm}  Recreational  Durable  Goods &0.31&0.32&0.23&0.04&0.12&0.19&0.17&0.21\\  &  \hspace{1mm}  Groceries &0.29&0.25&-0.08&0.07&0.13&0.15&0.10&0.08\\  &  \hspace{1mm}  Clothes  and  Shoes &-0.04&0.25&-0.07&0.00&0.15&0.05&0.05&0.08\\  \cbox{blue!75!white}  &  Services  (ex.  Shelter) &0.57&1.12&0.99&0.12&1.39&0.85&0.69&0.74\\  &  \hspace{1mm}  Health  Care  Services &0.13&0.38&0.72&-0.22&0.60&0.28&0.29&0.27\\  &  \hspace{1mm}  Transportation &0.09&0.17&0.01&-0.02&-0.02&0.07&0.07&0.06\\  &  \hspace{1mm}  Recreational &-0.02&0.17&-0.03&0.09&0.02&0.06&0.06&0.07\\  &  \hspace{1mm}  Food  and  Accommodations &0.17&0.22&-0.06&-0.12&0.35&0.13&0.12&0.09\\  &  \hspace{1mm}  Financial  and  Insurance &0.02&0.05&0.15&0.10&0.05&0.08&0.02&0.13\\  \cbox{green!85!blue}  &  Shelter   &0.42&0.03&-0.03&-0.06&-0.09&0.14&0.28&0.26\\  &  \hspace{1mm}  Housing  Services  and  Utilities   &0.24&0.14&0.01&0.12&0.07&0.13&0.15&0.23\\  &  \hspace{1mm}  Residential  Fixed  Investment &0.18&-0.11&-0.04&-0.18&-0.16&0.01&0.13&0.03\\ 
		\end{tabular}
		}	\\
		
\vspace{-6mm}
\footnotesize{Source: Bureau of Economic Analysis}

\newpage 

\begin{minipage}{0.76\textwidth}

\small Consumer spending is also reported on a monthly basis. Inflation- and population-adjusted consumer spending increased by 2.1 percent in July 2019 over July 2018. \\

\vspace{2mm}

\noindent \normalsize \textbf{Consumer Spending Growth}\\
\footnotesize{\textit{annual growth, per capita real personal consumption expenditures, percent}}\\
\noindent \hspace*{-2mm} \begin{tikzpicture}
	\begin{axis}[\bbar{y}{0}, \dateaxisticks ytick={-2, 0, 2, 4}, 
		xticklabel={`\short{\year}}, clip=false, height=4.0cm]
	\rbars
	\stdline{green!80!black}{date}{value2}{data/pcegrowth.csv}
	\end{axis}
\end{tikzpicture}\\
\footnotesize{Source: Bureau of Economic Analysis} \\

\vspace{2mm}

\normalsize

[Top quintile consumer spending share of gross pre-tax income and bottom 80 percent share] \\


\subsection*{\color{black!70} \seriffont Household Balance Sheets}

[Consumer Credit and Mortgages as share of DPI] \\

Housing prices \\

Housing permits/starts \\

Geographic location of housing permits \\

\end{minipage}

\newpage

\subsection*{\color{black!70} \seriffont Poverty}

\begin{minipage}{0.76\textwidth}
\small
Include data on number of people in poverty and the official poverty rate. Perhaps include a chart showing the official poverty rate over time. Perhaps also try to capture some concepts around methodology (SPM for example) and about relative poverty.\\

\end{minipage}

\noindent \normalsize \textbf{Share of local population in bottom third of housing-adjusted income, 2017}\\
\footnotesize{\textit{Share of commuting zone householders with after-housing-expense annual income below \$13,060}}

\vspace{-3mm}
\hspace{-15mm} \input{/home/brian/Documents/ACS/acs_map.pgf}

\vspace{-5mm}
\footnotesize{Source: American Community Survey}

\begin{minipage}{0.76\textwidth}

\end{minipage}

\newpage

\begin{minipage}{0.76\textwidth}

\section*{\color{darkgray}\LARGE \seriffont Businesses}
\small The factories, offices, and equipment that workers use to produce goods and services are all important to the economy. This section looks at the loosely defined business sector, with data covering business investment, retail sales, industrial production, corporate profits, and the financial activities of businesses.

\subsection*{\color{black!70} \seriffont Capital Investment}
\small Investments that make workers more productive, by definition, allow businesses to produce goods and services using less effort from people. Business gross investments are grouped broadly as structures (see\cbox{yellow!50!orange}), equipment (see\cbox{cyan!60!white}), and intellectual property products (see\cbox{violet}). 
\vspace{5mm}

\noindent \normalsize \textbf{Business Investment}\\
\footnotesize{\textit{percentage point contribution to GDP growth}}\\
\noindent \hspace*{-2mm} \begin{tikzpicture}
	\begin{axis}[\bbar{y}{0}, \dateaxisticks ytick={-2, 0, 2},
		xticklabel={`\short{\year}}, clip=false, 
		legend style={at={(0.95, 1.13)}}]
	\rbars
	\sbar{yellow!50!orange}{date}{A009RY}{data/businv.csv}
	\sbar{cyan!60!white}{date}{Y033RY}{data/businv.csv}
	\sbar{violet}{date}{Y001RY}{data/businv.csv}
	\legend{Structures, Equipment, Intellectual Property Products};
	\end{axis}
\end{tikzpicture}\\
\footnotesize{Source: Bureau of Economic Analysis}

\vspace{5mm}

\small

&   2023  Q2 & `23  Q1 & `22  Q4 & `22  Q3 & `22  Q2 &3-year&10-year&30-year\\ Total&0.98&0.76&0.24&0.62&0.68&0.55&0.55&0.60\\  \hspace{-2mm}\cbox{yellow!50!orange}Structures &0.46&0.77&0.17&-0.03&-0.01&-0.04&0.06&0.03\\  \hspace{-2mm}\cbox{cyan!60!white}Equipment &0.38&-0.21&-0.26&0.28&0.25&0.21&0.16&0.32\\  \hspace{4mm}  Information  processing &-0.11&-0.02&-0.38&0.09&-0.13&0.11&0.10&0.20\\  \hspace{6mm}  Computers  and  peripherals &0.02&-0.05&-0.23&0.09&-0.14&0.04&0.02&0.10\\  \hspace{4mm}  Industrial  equipment &-0.06&0.04&0.04&-0.07&-0.07&0.03&0.02&0.03\\  \hspace{4mm}  Transportation  equipment &0.54&-0.14&0.16&0.30&0.41&0.05&0.03&0.05\\  \hspace{-2mm}\cbox{violet}Intellectual  property  products &0.15&0.20&0.32&0.37&0.45&0.38&0.34&0.25\\  \hspace{4mm}  Software &0.13&0.16&0.31&0.28&0.21&0.25&0.20&0.15\\  \hspace{4mm}  Research  and  development &0.00&0.04&0.03&0.05&0.18&0.13&0.13&0.09\\  \\

\end{minipage}

\noindent \normalsize \textbf{Business Investment}\\
\footnotesize{\textit{percentage point contribution to real GDP growth \hspace{36mm} moving averages}\\ \vspace{4mm}
\noindent \rowcolors{1}{}{black!5} \setlength{\tabcolsep}{3.1pt} \color{black!90}
		{\renewcommand{\arraystretch}{1.55}
		 \begin{tabular}{p{42mm} R{6.7mm} R{6.7mm} R{6.7mm} R{6.7mm} R{6.7mm} 
		   R{7.8mm} R{6.7mm} R{6.7mm} }
			 &   2023  Q2 & `23  Q1 & `22  Q4 & `22  Q3 & `22  Q2 &3-year&10-year&30-year\\ Total&0.98&0.76&0.24&0.62&0.68&0.55&0.55&0.60\\  \hspace{-2mm}\cbox{yellow!50!orange}Structures &0.46&0.77&0.17&-0.03&-0.01&-0.04&0.06&0.03\\  \hspace{-2mm}\cbox{cyan!60!white}Equipment &0.38&-0.21&-0.26&0.28&0.25&0.21&0.16&0.32\\  \hspace{4mm}  Information  processing &-0.11&-0.02&-0.38&0.09&-0.13&0.11&0.10&0.20\\  \hspace{6mm}  Computers  and  peripherals &0.02&-0.05&-0.23&0.09&-0.14&0.04&0.02&0.10\\  \hspace{4mm}  Industrial  equipment &-0.06&0.04&0.04&-0.07&-0.07&0.03&0.02&0.03\\  \hspace{4mm}  Transportation  equipment &0.54&-0.14&0.16&0.30&0.41&0.05&0.03&0.05\\  \hspace{-2mm}\cbox{violet}Intellectual  property  products &0.15&0.20&0.32&0.37&0.45&0.38&0.34&0.25\\  \hspace{4mm}  Software &0.13&0.16&0.31&0.28&0.21&0.25&0.20&0.15\\  \hspace{4mm}  Research  and  development &0.00&0.04&0.03&0.05&0.18&0.13&0.13&0.09\\ 
		\end{tabular}
		}	\\
		
\vspace{-6mm}
\footnotesize{Source: Bureau of Economic Analysis}

\newpage

\begin{minipage}{0.76\textwidth}


\normalsize

\vspace{8mm}

Durable goods new orders\\

Corporate profits\\

Industrial production\\

Retail sales\\

Free cash flow \\

Balance sheets \\

Inventories \\

[Box on tech industry]\\

\end{minipage}

\newpage

\begin{minipage}{0.76\textwidth}

\section*{\color{darkgray}\LARGE \seriffont Government}

\normalsize

Overview\\

\vspace{2mm}

\subsection*{\color{black!70} \seriffont Government Consumption and Investment}
\small Government consumption and fixed investment provide one way to measure how much public sector activities are influencing the economy. The contribution to real GDP growth from the public sector can be broken down by federal defense (see\cbox{blue!60!black}), federal non-defense (see\cbox{green!85!black}), and state and local (see\cbox{purple!70!magenta}).  
\vspace{5mm}

\noindent \normalsize \textbf{Government Consumption and Investment}\\
\footnotesize{\textit{percentage point contribution to GDP growth}}\\
\noindent \hspace*{-2mm} \begin{tikzpicture}
	\begin{axis}[\bbar{y}{0}, \dateaxisticks ytick={-1, 0, 1},
		xticklabel={`\short{\year}}, clip=false, 
		legend style={at={(0.95, 1.13)}}]
	\rbars
	\sbar{blue!60!black}{date}{A824RY}{data/gov.csv}
	\sbar{green!85!black}{date}{A825RY}{data/gov.csv}
	\sbar{purple!70!magenta}{date}{A829RY}{data/gov.csv}
	\legend{Defense, Federal Non-Defense, State and Local};
	\end{axis}
\end{tikzpicture}\\
\footnotesize{Source: Bureau of Economic Analysis}
\vspace{10mm}

\normalsize

Federal\\

\hspace{5mm} Outlays on interest as share of GDP \\

State \\

Local \\

Balance sheets \\

\end{minipage}


\newpage
\begin{minipage}{0.76\textwidth}
\section*{\color{darkgray}\LARGE \seriffont External Sector}

\small

Overview text goes here. Perhaps mention the size of total trade, imports, exports, and the trade balance, as a share of GDP. Perhaps include the trade balance/ GDP chart here. Eventually, section should capture changes in trade by goods and by partner. Section also needs to capture cross border flows of investment, though much of FDI seems to be tax avoidance. Focus on the more meaningful series, like remittances. \\

\vspace{2mm}
\subsection*{\color{black!70} \seriffont Trade}
\small
The US runs a persistent trade deficit, which often shows up as a negative contributor to GDP growth. The trade balance is the exports of goods (see\cbox{green!70!white}) and the exports of services (see\cbox{green!50!black}) minus imports of goods (see\cbox{cyan!70!white}) and imports of services (see\cbox{blue!70!black}). 
\vspace{5mm}

\noindent \normalsize \textbf{International Trade}\\
\footnotesize{\textit{percentage point contribution to GDP growth}}\\
\noindent \hspace*{-2mm} \begin{tikzpicture}
	\begin{axis}[\bbar{y}{0}, \dateaxisticks ytick={-2, 0, 2, 5},
		xticklabel={`\short{\year}}, clip=false, 
		legend style={at={(0.95, 1.13)}}]
	\rbars
	\sbar{green!70!white}{date}{A253RY}{data/nx.csv}
	\sbar{green!50!black}{date}{A646RY}{data/nx.csv}
	\sbar{cyan!70!white}{date}{A255RY}{data/nx.csv}
	\sbar{blue!70!black}{date}{A656RY}{data/nx.csv}	
	\legend{Goods Exports, Services Exports, Goods Imports, Services Imports};
	\end{axis}
\end{tikzpicture}\\
\footnotesize{Source: Bureau of Economic Analysis}

\normalsize

\vspace{8mm}

Current account balance \\

Trade in Goods \\

Trade in Services \\

Trade balance \\

[One page table to capture lots of external sector items as contribution to GDP growth (where possible) or otherwise as a share of GDP]\\

Exchange rates \\

Direct and Portfolio Investment -- related here and to IIP below: the total value of domestic holdings of foreign assets is much smaller than the total value of foreign holdings of domestic assets, but, the return on foreign assets is so much higher than the return on domestic assets that the the US has positive net income from abroad.\\

International Investment Position \\

\end{minipage}

\newpage

\section*{\color{darkgray}\LARGE \seriffont Labor Markets}

\normalsize

Overview \\

Employment Rate \\

Unemployment Rate \\

Unemployment by reason \\

Unemployment by duration \\

Part-time and full-time and hours worked \\

Job growth \\

Wage growth: \\

\hspace{4mm} [AHE and UWE both in various forms] \\

\hspace{4mm} [Either FRB Atlanta Wage Tracker or replication]\\

Quits \\ 

Openings \\

Jobless claims \\

Flows \\

Reasons for non-participation \\

Union membership \\

State- and sub-state-level analysis 

\newpage

\section*{\color{darkgray}\LARGE \seriffont Capital Markets}

\normalsize

Overview \\

Equity markets \\

\hspace{4mm} [SP500] \\

\hspace{4mm} [VIX] \\

Interest rates \\

\hspace{4mm} [Fed funds rate] \\

\hspace{4mm} [Fed balance sheet or excess reserve or both] \\

\hspace{4mm} [10year and 2year] \\

\hspace{4mm} [AAA and high-yield] \\

Yield curve \\

Valuations \\

\hspace{4mm} [PE Ratio] \\

\newpage

\section*{\color{darkgray}\LARGE \seriffont Prices}

\normalsize

CPI: \\

\hspace{4mm} [CPI-U growth - core, all-items, CPI-U-RS]\\

\hspace{4mm} [CPI-U components contribution - horizontal range chart] \\ 

PPI \\

XMPI \\

PCE \\

Expectations \\

\newpage

\section*{\color{darkgray}\LARGE \seriffont International Comparisons}

\normalsize

Demographics \\

Economic Activity \\

Labor Markets \\

Poverty \\


\newpage

\begin{minipage}{0.76\textwidth}
\section*{\color{darkgray}\LARGE \seriffont References}

\small

List of tables and sources along with some notes... \\

One option for this section is to have some json data that captures what original data goes into each series and also what types of calculations are done on the original data.

\subsection*{\color{black!70} {\seriffont Acknowledgments}}

Gabriel Mathy, Iordan Koulov, Lara Merling, Kevin Cashman, Rebecca Watts, Dean Baker, Eileen Appelbaum, John Schmitt, Rainer K\"ohler, Gersenda Varisco, Venkat Josyula, Tom Augspurger, Mike Sieferling, Matt Bruenig, and Ernie Tedeschi.

\end{minipage}

\end{document}