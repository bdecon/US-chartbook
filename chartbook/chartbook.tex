% % % % % % % % % % % % % % 
%
%	U.S. Chartbook
%	Brian W. Dew (brianwdew@gmail.com)
%	Created: August 23, 2019
%	GitHub repo contains to do list (issues)
%
% % % % % % % % % % % % % %
\PassOptionsToPackage{table}{xcolor}
\documentclass{report}

%
% % % % % % Packages % % % % % % % % % 
%
	
	\usepackage[letterpaper, margin=1.2in]{geometry}
	\usepackage{microtype}
	\usepackage[default]{lato}
	\usepackage{pgfplots}
	\usepackage{xcolor}
	\usepackage{array}
	\usetikzlibrary{pgfplots.dateplot}

%
% % % % % Document Settings % % % % % % % 
%

	% Paragraph spacing
		\setlength{\parskip}{8pt}
		\setlength{\parindent}{0pt}
		
%
% % % % % Graph Settings % % % % % % % 
%
	
	% Color square
	\newcommand{\cbox}[1]{
		\begin{tikzpicture} \draw [#1, line width=6](0,0) -- (.2,0);  
		\end{tikzpicture}}
	
	% Column width and alignment
	\newcolumntype{R}[1]{>{\raggedleft\let\newline\\\arraybackslash\hspace{0pt}}m{#1}}	
	
	% Style for date plots
	\pgfplotsset{compat=newest, 
		scaled y ticks=false,
		axis line style={black!20}, 
		xtick style={black!20}, ytick style={draw=none},
		every tick label/.style={black!50, font=\scriptsize,
			/pgf/number format/assume math mode=true},
		width=12.9cm, height=4.6cm, 
		xticklabel style={align=left}, 
		yticklabel style={text width=0.85em, align=right},       
		axis x line*=bottom, x axis line style={black!50},
	    axis y line=left, y axis line style={opacity=0},
	    ymajorgrids, grid style={very thin, black!10},	        
	    every node near coord/.style={black!70},
	    legend style={legend columns=-1, draw=none, fill=none,
	    	/tikz/every even column/.append style={column sep=0.3cm}}}
	
	% stacked diverging bar
	\newcommand{\sbar}[4]{
		\addplot[ybar stacked, bar width=2.7pt, draw opacity=0, fill=#1] 
			table [x=#2, y=#3, col sep=comma]{#4};}
					
	% text node
	\newcommand{\stdnode}[3]{\node[below, align=left, shift=({#1,#2})]{#3};}	        
		        
	% Date (X) Axis Tick Marks, one tick per year, every even year labeled
	\newcommand{\dateaxisticks}{
		date coordinates in=x, axis line style={draw=none},
		xmax={2019-07-01},
		max space between ticks=40,	    
		xtick={{1990-01-01}, {1992-01-01}, {1994-01-01}, 
			{1996-01-01}, {1998-01-01}, {2000-01-01}, 
			{2002-01-01}, {2004-01-01}, {2006-01-01},
			{2008-01-01}, {2010-01-01}, {2012-01-01}, {2014-01-01},
		    {2016-01-01}, {2018-01-01}},
		minor xtick={{1989-01-01}, {1991-01-01}, {1993-01-01},
			{1995-01-01}, {1997-01-01}, {1999-01-01}, 
			{2001-01-01}, {2003-01-01}, {2005-01-01}, {2007-01-01},
		    {2009-01-01}, {2011-01-01}, {2013-01-01}, {2015-01-01},
		    {2017-01-01}, {2019-01-01}},
		enlarge y limits={0.04}, enlarge x limits={0.01},
		}
	
	% Solid bars at significant  x or y values
	\newcommand{\bbar}[2]{extra #1 ticks = {{#2}}, extra #1 tick labels = ,
		extra #1 tick style = {grid=major, grid style={thick, black!25}},}
		
	% Standard line
	\newcommand{\stdline}[4]{\addplot[very thick, no markers, color=#1] 
		table [x=#2, y=#3, col sep=comma] {#4};	}
		
	% Recession bars		
	\newcommand{\rbars}{
		\fill[color=black!10] (axis cs:{1990-07-01},\pgfkeysvalueof{/pgfplots/ymin}) rectangle 
			(axis cs:{1991-03-01}, \pgfkeysvalueof{/pgfplots/ymax});
		\fill[color=black!10] (axis cs:{2007-12-01},\pgfkeysvalueof{/pgfplots/ymin}) rectangle 
			(axis cs:{2009-07-01}, \pgfkeysvalueof{/pgfplots/ymax});
		\fill[color=black!10] (axis cs:{2001-03-01},\pgfkeysvalueof{/pgfplots/ymin}) rectangle 
			(axis cs:{2001-11-01}, \pgfkeysvalueof{/pgfplots/ymax});}
	
	\newfontfamily\seriffont{SourceSerifPro}	
			    		    
% % % % % % % %
%
%  Begin Document
%
% % % % % % % %		
\begin{document}

\chapter*{
		\textcolor{blue!70}{\rule[-1pt]{6pt}{20pt}}
		\textcolor{green!70!blue}{\rule[-1pt]{6pt}{32pt}} \ \color{darkgray}US Chartbook v0.0}
\vspace{4mm}

\begin{minipage}{0.76\textwidth}
\subsection*{\color{black!70} {\seriffont \textbf{Notes:}}}

Author: Brian Dew, brian.w.dew@gmail.com \\

Updated: \today \\

\subsection*{\color{black!70} {\seriffont \textbf{Proposed Contents:}}}

\begin{itemize}

\item Overall Economic Activity

\item Households

\item Businesses

\item Government

\item External Sector

\item Labor Markets

\item Capital Markets

\item Prices

\end{itemize}

\end{minipage}
\thispagestyle{empty}

\newpage

\section*{\color{darkgray} \LARGE Overall Economic Activity}
\input{text/gdp.txt}

\begin{minipage}{0.76\textwidth}
\subsection*{\color{black!70}Economic Growth}

\small GDP (see\cbox{red!95!black}) \input{text/gdp_gr.txt}
\vspace{5mm}

\noindent \normalsize \textbf{Real Gross Domestic Product Growth}\\
\footnotesize{\textit{quarterly growth at seasonally adjusted annual rate, percent}}\\
\noindent \hspace*{-2mm} \begin{tikzpicture}
	\begin{axis}[\bbar{y}{0}, \dateaxisticks ytick={-5, 0, 5}, 
		xticklabel={\year}, clip=false, height=4.0cm]
	\rbars
	\sbar{red!95!black}{date}{A191RL}{data/gdp.csv}
	\node[above, align=left] at (axis cs:2017-01-01,-8.5) {\scriptsize \input{data/gdp.txt}};
	\end{axis}
\end{tikzpicture}\\
\footnotesize{Source: Bureau of Economic Analysis} 

\subsection*{\color{black!70}Components of Growth}

\small The \textbf{expenditure approach} compiles GDP from the sum of spending on domestic goods and services. Major spending categories are consumer spending (see\cbox{yellow!80!orange}), private investment (gross spending on capital goods) and changes in private inventories (see\cbox{blue!70!black}), government spending and investment (see\cbox{cyan!50!white}), and net exports (see\cbox{green!60!black}) which is measured as foreign spending on US goods and services less US spending on goods and services produced by the rest of the world. 
\vspace{5mm}

\noindent \normalsize \textbf{Real GDP Growth by Expenditure Type}\\
\footnotesize{\textit{percentage point contribution to GDP growth}}\\
\noindent \hspace*{-2mm} \begin{tikzpicture}
	\begin{axis}[\bbar{y}{0}, \dateaxisticks ytick={-5, 0, 5},
		xticklabel={\year}, clip=false, legend style={at={(0.95, 1.13)}}]
	\rbars
	\sbar{yellow!80!orange}{date}{DPCERY}{data/comp.csv}
	\sbar{blue!70!black}{date}{A006RY}{data/comp.csv}
	\sbar{cyan!50!white}{date}{A822RY}{data/comp.csv}
	\sbar{green!60!black}{date}{A019RY}{data/comp.csv}
	\stdnode{2.2cm}{0.4cm}{\footnotesize $^*$ Includes change in private inventories}
	\legend{Consumer Spending, Investment$^*$, Government, Net Exports};
	\end{axis}
\end{tikzpicture}\\
\footnotesize{Source: Bureau of Economic Analysis}
\end{minipage}

\newpage
\begin{minipage}{0.76\textwidth}
\small The \textbf{production approach} calculates GDP as the sum of gross value added--output minus inputs--in each sector. This identifies contributions from: goods-producing sectors combined with trade, transportation, and utilities (see\cbox{purple!70!blue}), finance, insurance, and real estate (see\cbox{red!90!white}), other service-providing sectors (see\cbox{blue!90!white}), and government (see\cbox{orange!80!white}).
\vspace{5mm}

\noindent \normalsize \textbf{Real GDP Growth by Industry Group}\\
\footnotesize{\textit{percentage point contribution to GDP growth}}\\
\noindent \hspace*{-2mm} \begin{tikzpicture}
	\begin{axis}[\bbar{y}{0}, \dateaxisticks ytick={-5, 0, 5},
		xticklabel={\year}, clip=false, legend style={at={(0.95, 1.13)}}]
	\rbars
	\draw [dashed] (axis cs:{2004-10-01},\pgfkeysvalueof{/pgfplots/ymin}) -- (axis cs:{2004-10-01}, 			\pgfkeysvalueof{/pgfplots/ymax});
	\draw [dashed] (axis cs:{1998-01-01},\pgfkeysvalueof{/pgfplots/ymin}) -- (axis cs:{1998-01-01}, 			\pgfkeysvalueof{/pgfplots/ymax});
	\sbar{orange!80!white}{date}{Government}{data/gdpva.csv}
	\sbar{blue!90!white}{date}{Oth_Serv}{data/gdpva.csv}
	\sbar{red!90!white}{date}{FIRE}{data/gdpva.csv}
	\sbar{purple!70!blue}{date}{GoodsTTU}{data/gdpva.csv}
	\stdnode{0.65cm}{0.4cm}{\scriptsize historical data}
	\stdnode{4.1cm}{0.4cm}{\scriptsize annual data}
	\legend{Government, Other Services, FIRE, Goods and TTU};
	\end{axis}
\end{tikzpicture}\\
\footnotesize{Source: Bureau of Economic Analysis}
\vspace{6mm}

\small The \textbf{income approach} calculates GDP as the sum of market income to persons (in exchange for labor (see\cbox{magenta!90!blue}) or from returns on capital (see\cbox{yellow!60!orange})), indirect taxes such as sales taxes or tariffs (see\cbox{violet}), and depreciation (see\cbox{teal!60!white}). 
\vspace{5mm}

\noindent \normalsize \textbf{Real Gross Domestic Income Growth}\\
\footnotesize{\textit{percentage point contribution to GDI growth}}\\
\noindent \hspace*{-2mm} \begin{tikzpicture}
	\begin{axis}[\bbar{y}{0}, \dateaxisticks ytick={-5, 0, 5},
		xticklabel={\year}, clip=false, 
		legend style={at={(0.95, 1.13)}}]
	\rbars
	\sbar{magenta!90!blue}{date}{A4002C}{data/gdi.csv}
	\sbar{yellow!60!orange}{date}{W271RC}{data/gdi.csv}
	\sbar{teal!60!white}{date}{A262RC}{data/gdi.csv}
	\sbar{violet}{date}{indirect}{data/gdi.csv}	
	\legend{Labor, Profit, Depreciation, Indirect Taxes};
	\end{axis}
\end{tikzpicture}\\
\footnotesize{Source: Bureau of Economic Analysis}
\vspace{6mm}

\small Changes to GDP can be assigned to changes in \textbf{household inputs}: population (see\cbox{lime}), employment rates (see\cbox{green!30!teal!90!black}), average hours worked (see\cbox{blue}), and total economy productivity (see\cbox{cyan!50!white}). 
\vspace{5mm}

\noindent \normalsize \textbf{Real GDP Growth by Inputs}\\
\footnotesize{\textit{percentage point contribution to GDP growth}}\\
\noindent \hspace*{-2mm} \begin{tikzpicture}
	\begin{axis}[\bbar{y}{0}, \dateaxisticks ytick={-5, 0, 5},
		xticklabel={\year}, clip=false, 
		legend style={at={(0.95, 1.13)}}]
	\rbars
	\sbar{lime}{date}{pop_contr}{data/gdpjobs.csv}
	\sbar{green!30!teal!90!black}{date}{epop_contr}{data/gdpjobs.csv}
	\sbar{cyan!50!white}{date}{prod}{data/gdpjobs.csv}
	\sbar{blue}{date}{hours_contr}{data/gdpjobs.csv}	
	\legend{Population, Employment Rate, Productivity, Average Hours};
	\end{axis}
\end{tikzpicture}\\
\footnotesize{Source: Author's Calculations}
\end{minipage}

\newpage

\noindent \normalsize \textbf{Components of Economic Growth}\\
\footnotesize{\textit{percentage point contribution to real GDP/GDI growth \hspace{35mm} moving averages}\\ \vspace{4mm}
\noindent \rowcolors{1}{}{black!5} \setlength{\tabcolsep}{3.8pt} \color{black!90}
		{\renewcommand{\arraystretch}{1.55}
		 \begin{tabular}{p{2mm} p{36mm} R{7mm} R{7mm} R{7mm} R{7mm} R{7mm} p{0mm} R{7mm} R{7mm} R{7mm} }
& & 2020 Q3 & '20 Q2 & '20 Q1 & '19 Q4 & '19 Q3 & & 3-year & 10-year & 30-year \\
\cbox{red!95!black} & \textbf{Gross Domestic Product} & 33.1 & -31.4 & -5.0 & 2.4 & 2.6 & & 1.8 &  2.0 & 2.4 \\
\cbox{yellow!80!orange} & \hspace{2mm} Consumer Spending & 25.22 & -24.01 & -4.75 & 1.07 & 1.83 & & 1.06 &  1.45 & 1.74 \\
& \hspace{4mm} Durable Goods & 5.20 & 0.00 & -0.93 & 0.22 & 0.44 & & 0.66 &  0.51 & 0.46 \\
& \hspace{4mm} Non-durable Goods  & 4.29 & -2.05 & 0.97 & -0.10 & 0.43 & & 0.53 &  0.37 & 0.35 \\
& \hspace{4mm} Services  & 15.73 & -21.95 & -4.78 & 0.96 & 0.96 & & -0.13 &  0.57 & 0.93 \\
\cbox{blue!70!black} & \hspace{2mm} Gross Investment & 11.78 & -8.77 & -1.56 & -0.64 & 0.34 & & 0.55 &  0.75 & 0.61 \\
& \hspace{4mm} Non-residential  & 3.06 & -3.67 & -0.91 & -0.04 & 0.25 & & 0.32 &  0.54 & 0.52 \\
& \hspace{4mm} Residential  & 2.17 & -1.60 & 0.68 & 0.22 & 0.17 & & 0.09 &  0.16 & 0.05 \\
& \hspace{4mm} Change in inventories  & 6.55 & -3.50 & -1.34 & -0.82 & -0.09 & & 0.14 &  0.05 & 0.04 \\
\cbox{cyan!50!white} & \hspace{2mm} Government  & -0.76 & 0.77 & 0.22 & 0.42 & 0.37 & & 0.30 &  -0.01 & 0.22 \\
& \hspace{4mm} Federal  & -0.38 & 1.17 & 0.10 & 0.26 & 0.31 & & 0.24 &  -0.02 & 0.07 \\
& \hspace{4mm} State and Local  & -0.38 & -0.40 & 0.12 & 0.16 & 0.06 & & 0.05 &  0.01 & 0.15 \\
\cbox{green!60!black} & \hspace{2mm} Net Exports  & -3.18 & 0.62 & 1.13 & 1.52 & 0.04 & & -0.16 &  -0.16 & -0.16 \\
& \hspace{4mm} Exports  & 4.95 & -9.51 & -1.12 & 0.39 & 0.10 & & -0.27 &  0.21 & 0.43 \\
& \hspace{4mm} Imports  & -8.12 & 10.13 & 2.25 & 1.13 & -0.06 & & 0.11 &  -0.37 & -0.59 \\
& & & & & & & & & & \\
\cbox{purple!70!blue} & \hspace{2mm} Goods and TTU  & -- & -12.45 & -1.09 & 0.49 & 1.08 & & -0.37 &  0.40 & 0.77 \\
& \hspace{4mm} Manufacturing  & -- & -4.10 & -0.70 & 0.00 & 0.53 & & -0.14 &  0.04 & 0.29 \\
& \hspace{4mm} Construction  & -- & -1.12 & 0.02 & 0.00 & 0.00 & & -0.06 &  0.04 & -0.02 \\
& \hspace{4mm} Retail Trade  & -- & -1.75 & -0.39 & 0.14 & 0.21 & & -0.04 &  0.08 & 0.17 \\
\cbox{red!90!white} & \hspace{2mm} FIRE+  & -- & -0.53 & -1.27 & 1.12 & 0.28 & & 0.44 &  0.59 & 0.71 \\
\cbox{blue!90!white} & \hspace{2mm} Other Services  & -- & -16.47 & -2.36 & 0.47 & 1.14 & & -0.69 &  0.28 & 0.47 \\
& \hspace{4mm} Education \& Healthcare  & -- & -4.54 & -0.59 & 0.19 & 0.17 & & -0.23 &  0.06 & 0.15 \\
& \hspace{4mm} Professional \& Business & -- & -3.84 & -0.24 & 0.39 & 0.63 & & 0.18 &  0.34 & 0.30 \\
& \hspace{4mm} Information  & -- & -0.29 & -0.15 & 0.38 & 0.37 & & 0.28 &  0.27 & 0.24 \\
\cbox{orange!80!white} & \hspace{2mm} Government  & -- & -1.93 & -0.30 & 0.34 & 0.14 & & -0.04 &  -0.02 & 0.09 \\
& & & & & & & & & & \\
\cbox{lime!90!green} & \hspace{2mm} Population  & 0.76 & 0.31 & 0.40 & 0.57 & 0.57 & & 0.52 &  0.65 & 0.94 \\
\cbox{green!30!teal!90!black} & \hspace{2mm} Employment Rate  & 27.37 & -49.19 & -2.48 & 1.43 & 1.83 & & -1.23 &  0.07 & -0.13 \\
\cbox{blue} & \hspace{2mm} Average Hours & 0.47 & -4.68 & -3.19 & 0.38 & 0.71 & & -0.19 &  0.05 & -0.04 \\
\cbox{cyan!60!white} & \hspace{2mm} Productivity  & 4.47 & 22.17 & 0.32 & -0.01 & -0.54 & & 2.65 &  1.26 & 1.63 \\
& & & & & & & & & & \\& \textbf{Gross Domestic Income}  & 25.5 & -32.6 & -2.5 & 3.3 & 0.8 & & 0.9 &  1.9 & 2.4 \\
\cbox{magenta!90!blue} & \hspace{2mm} Labor  & 10.75 & -10.73 & 1.11 & 1.59 & 0.13 & & 1.11 &  1.16 & 1.25 \\
\cbox{yellow!60!orange} & \hspace{2mm} Profit  & 16.02 & -4.23 & -4.38 & 1.30 & 0.28 & & 0.80 &  0.77 & 0.75 \\
\cbox{teal!60!white} & \hspace{2mm} Depreciation  & 0.12 & 0.71 & 0.37 & 0.34 & 0.51 & & 0.41 &  0.36 & 0.41 \\
\cbox{violet} & \hspace{2mm} Indirect Taxes  & -1.40 & -18.33 & 0.37 & 0.07 & -0.09 & & -1.40 &  -0.35 & 0.00 \\

		\end{tabular}
		}	\\
\footnotesize{Source: Bureau of Economic Analysis and Author's Calculations}
		
\newpage

\noindent \normalsize \textbf{Real GDP Growth by State}\\
\footnotesize{\textit{percentage point change in real GDP}}\\
\vspace{-2mm}
\hspace{-8mm} %% Creator: Matplotlib, PGF backend
%%
%% To include the figure in your LaTeX document, write
%%   \input{<filename>.pgf}
%%
%% Make sure the required packages are loaded in your preamble
%%   \usepackage{pgf}
%%
%% Also ensure that all the required font packages are loaded; for instance,
%% the lmodern package is sometimes necessary when using math font.
%%   \usepackage{lmodern}
%%
%% Figures using additional raster images can only be included by \input if
%% they are in the same directory as the main LaTeX file. For loading figures
%% from other directories you can use the `import` package
%%   \usepackage{import}
%%
%% and then include the figures with
%%   \import{<path to file>}{<filename>.pgf}
%%
%% Matplotlib used the following preamble
%%   
%%   \usepackage{fontspec}
%%   \setmainfont{DejaVuSerif.ttf}[Path=\detokenize{/home/brian/miniconda3/lib/python3.8/site-packages/matplotlib/mpl-data/fonts/ttf/}]
%%   \setsansfont{DejaVuSans.ttf}[Path=\detokenize{/home/brian/miniconda3/lib/python3.8/site-packages/matplotlib/mpl-data/fonts/ttf/}]
%%   \setmonofont{DejaVuSansMono.ttf}[Path=\detokenize{/home/brian/miniconda3/lib/python3.8/site-packages/matplotlib/mpl-data/fonts/ttf/}]
%%   \makeatletter\@ifpackageloaded{underscore}{}{\usepackage[strings]{underscore}}\makeatother
%%
\begingroup%
\makeatletter%
\begin{pgfpicture}%
\pgfpathrectangle{\pgfpointorigin}{\pgfqpoint{6.714028in}{2.186394in}}%
\pgfusepath{use as bounding box, clip}%
\begin{pgfscope}%
\pgfsetbuttcap%
\pgfsetmiterjoin%
\definecolor{currentfill}{rgb}{1.000000,1.000000,1.000000}%
\pgfsetfillcolor{currentfill}%
\pgfsetlinewidth{0.000000pt}%
\definecolor{currentstroke}{rgb}{1.000000,1.000000,1.000000}%
\pgfsetstrokecolor{currentstroke}%
\pgfsetdash{}{0pt}%
\pgfpathmoveto{\pgfqpoint{0.000000in}{0.000000in}}%
\pgfpathlineto{\pgfqpoint{6.714028in}{0.000000in}}%
\pgfpathlineto{\pgfqpoint{6.714028in}{2.186394in}}%
\pgfpathlineto{\pgfqpoint{0.000000in}{2.186394in}}%
\pgfpathlineto{\pgfqpoint{0.000000in}{0.000000in}}%
\pgfpathclose%
\pgfusepath{fill}%
\end{pgfscope}%
\begin{pgfscope}%
\pgfpathrectangle{\pgfqpoint{0.100000in}{0.100000in}}{\pgfqpoint{2.989028in}{1.913466in}}%
\pgfusepath{clip}%
\pgfsetbuttcap%
\pgfsetmiterjoin%
\definecolor{currentfill}{rgb}{0.959631,0.983852,0.686044}%
\pgfsetfillcolor{currentfill}%
\pgfsetlinewidth{0.000000pt}%
\definecolor{currentstroke}{rgb}{0.000000,0.000000,0.000000}%
\pgfsetstrokecolor{currentstroke}%
\pgfsetstrokeopacity{0.000000}%
\pgfsetdash{}{0pt}%
\pgfpathmoveto{\pgfqpoint{1.097679in}{0.548675in}}%
\pgfpathlineto{\pgfqpoint{1.089089in}{0.548815in}}%
\pgfpathlineto{\pgfqpoint{1.083242in}{0.555287in}}%
\pgfpathlineto{\pgfqpoint{1.079266in}{0.572473in}}%
\pgfpathlineto{\pgfqpoint{1.088506in}{0.574627in}}%
\pgfpathlineto{\pgfqpoint{1.097893in}{0.572311in}}%
\pgfpathlineto{\pgfqpoint{1.107446in}{0.565540in}}%
\pgfpathlineto{\pgfqpoint{1.107441in}{0.556177in}}%
\pgfpathlineto{\pgfqpoint{1.097679in}{0.548675in}}%
\pgfpathclose%
\pgfusepath{fill}%
\end{pgfscope}%
\begin{pgfscope}%
\pgfpathrectangle{\pgfqpoint{0.100000in}{0.100000in}}{\pgfqpoint{2.989028in}{1.913466in}}%
\pgfusepath{clip}%
\pgfsetbuttcap%
\pgfsetmiterjoin%
\definecolor{currentfill}{rgb}{0.959631,0.983852,0.686044}%
\pgfsetfillcolor{currentfill}%
\pgfsetlinewidth{0.000000pt}%
\definecolor{currentstroke}{rgb}{0.000000,0.000000,0.000000}%
\pgfsetstrokecolor{currentstroke}%
\pgfsetstrokeopacity{0.000000}%
\pgfsetdash{}{0pt}%
\pgfpathmoveto{\pgfqpoint{1.149970in}{0.446640in}}%
\pgfpathlineto{\pgfqpoint{1.140359in}{0.450300in}}%
\pgfpathlineto{\pgfqpoint{1.140315in}{0.457479in}}%
\pgfpathlineto{\pgfqpoint{1.128705in}{0.463549in}}%
\pgfpathlineto{\pgfqpoint{1.130119in}{0.478835in}}%
\pgfpathlineto{\pgfqpoint{1.133481in}{0.486048in}}%
\pgfpathlineto{\pgfqpoint{1.140581in}{0.479409in}}%
\pgfpathlineto{\pgfqpoint{1.152127in}{0.480518in}}%
\pgfpathlineto{\pgfqpoint{1.149211in}{0.459391in}}%
\pgfpathlineto{\pgfqpoint{1.149970in}{0.446640in}}%
\pgfpathclose%
\pgfusepath{fill}%
\end{pgfscope}%
\begin{pgfscope}%
\pgfpathrectangle{\pgfqpoint{0.100000in}{0.100000in}}{\pgfqpoint{2.989028in}{1.913466in}}%
\pgfusepath{clip}%
\pgfsetbuttcap%
\pgfsetmiterjoin%
\definecolor{currentfill}{rgb}{0.959631,0.983852,0.686044}%
\pgfsetfillcolor{currentfill}%
\pgfsetlinewidth{0.000000pt}%
\definecolor{currentstroke}{rgb}{0.000000,0.000000,0.000000}%
\pgfsetstrokecolor{currentstroke}%
\pgfsetstrokeopacity{0.000000}%
\pgfsetdash{}{0pt}%
\pgfpathmoveto{\pgfqpoint{1.190310in}{0.399517in}}%
\pgfpathlineto{\pgfqpoint{1.178114in}{0.401594in}}%
\pgfpathlineto{\pgfqpoint{1.171369in}{0.412035in}}%
\pgfpathlineto{\pgfqpoint{1.170417in}{0.420620in}}%
\pgfpathlineto{\pgfqpoint{1.190310in}{0.399517in}}%
\pgfpathclose%
\pgfusepath{fill}%
\end{pgfscope}%
\begin{pgfscope}%
\pgfpathrectangle{\pgfqpoint{0.100000in}{0.100000in}}{\pgfqpoint{2.989028in}{1.913466in}}%
\pgfusepath{clip}%
\pgfsetbuttcap%
\pgfsetmiterjoin%
\definecolor{currentfill}{rgb}{0.959631,0.983852,0.686044}%
\pgfsetfillcolor{currentfill}%
\pgfsetlinewidth{0.000000pt}%
\definecolor{currentstroke}{rgb}{0.000000,0.000000,0.000000}%
\pgfsetstrokecolor{currentstroke}%
\pgfsetstrokeopacity{0.000000}%
\pgfsetdash{}{0pt}%
\pgfpathmoveto{\pgfqpoint{1.165317in}{0.401338in}}%
\pgfpathlineto{\pgfqpoint{1.171711in}{0.396147in}}%
\pgfpathlineto{\pgfqpoint{1.173099in}{0.387222in}}%
\pgfpathlineto{\pgfqpoint{1.160227in}{0.387715in}}%
\pgfpathlineto{\pgfqpoint{1.165317in}{0.401338in}}%
\pgfpathclose%
\pgfusepath{fill}%
\end{pgfscope}%
\begin{pgfscope}%
\pgfpathrectangle{\pgfqpoint{0.100000in}{0.100000in}}{\pgfqpoint{2.989028in}{1.913466in}}%
\pgfusepath{clip}%
\pgfsetbuttcap%
\pgfsetmiterjoin%
\definecolor{currentfill}{rgb}{0.959631,0.983852,0.686044}%
\pgfsetfillcolor{currentfill}%
\pgfsetlinewidth{0.000000pt}%
\definecolor{currentstroke}{rgb}{0.000000,0.000000,0.000000}%
\pgfsetstrokecolor{currentstroke}%
\pgfsetstrokeopacity{0.000000}%
\pgfsetdash{}{0pt}%
\pgfpathmoveto{\pgfqpoint{1.194683in}{0.351485in}}%
\pgfpathlineto{\pgfqpoint{1.181837in}{0.356103in}}%
\pgfpathlineto{\pgfqpoint{1.185518in}{0.367298in}}%
\pgfpathlineto{\pgfqpoint{1.180134in}{0.378705in}}%
\pgfpathlineto{\pgfqpoint{1.181825in}{0.386546in}}%
\pgfpathlineto{\pgfqpoint{1.192162in}{0.389535in}}%
\pgfpathlineto{\pgfqpoint{1.191963in}{0.376972in}}%
\pgfpathlineto{\pgfqpoint{1.204190in}{0.370003in}}%
\pgfpathlineto{\pgfqpoint{1.211183in}{0.353155in}}%
\pgfpathlineto{\pgfqpoint{1.203389in}{0.346667in}}%
\pgfpathlineto{\pgfqpoint{1.194683in}{0.351485in}}%
\pgfpathclose%
\pgfusepath{fill}%
\end{pgfscope}%
\begin{pgfscope}%
\pgfpathrectangle{\pgfqpoint{0.100000in}{0.100000in}}{\pgfqpoint{2.989028in}{1.913466in}}%
\pgfusepath{clip}%
\pgfsetbuttcap%
\pgfsetmiterjoin%
\definecolor{currentfill}{rgb}{0.959631,0.983852,0.686044}%
\pgfsetfillcolor{currentfill}%
\pgfsetlinewidth{0.000000pt}%
\definecolor{currentstroke}{rgb}{0.000000,0.000000,0.000000}%
\pgfsetstrokecolor{currentstroke}%
\pgfsetstrokeopacity{0.000000}%
\pgfsetdash{}{0pt}%
\pgfpathmoveto{\pgfqpoint{1.148289in}{0.237569in}}%
\pgfpathlineto{\pgfqpoint{1.142169in}{0.250211in}}%
\pgfpathlineto{\pgfqpoint{1.144879in}{0.258715in}}%
\pgfpathlineto{\pgfqpoint{1.156883in}{0.270485in}}%
\pgfpathlineto{\pgfqpoint{1.158019in}{0.279508in}}%
\pgfpathlineto{\pgfqpoint{1.165875in}{0.299484in}}%
\pgfpathlineto{\pgfqpoint{1.186536in}{0.302897in}}%
\pgfpathlineto{\pgfqpoint{1.190491in}{0.314590in}}%
\pgfpathlineto{\pgfqpoint{1.199600in}{0.310205in}}%
\pgfpathlineto{\pgfqpoint{1.218110in}{0.277388in}}%
\pgfpathlineto{\pgfqpoint{1.218344in}{0.266674in}}%
\pgfpathlineto{\pgfqpoint{1.213112in}{0.257400in}}%
\pgfpathlineto{\pgfqpoint{1.219750in}{0.236108in}}%
\pgfpathlineto{\pgfqpoint{1.215111in}{0.232676in}}%
\pgfpathlineto{\pgfqpoint{1.199311in}{0.233674in}}%
\pgfpathlineto{\pgfqpoint{1.180311in}{0.240024in}}%
\pgfpathlineto{\pgfqpoint{1.164342in}{0.242646in}}%
\pgfpathlineto{\pgfqpoint{1.160076in}{0.238921in}}%
\pgfpathlineto{\pgfqpoint{1.148289in}{0.237569in}}%
\pgfpathclose%
\pgfusepath{fill}%
\end{pgfscope}%
\begin{pgfscope}%
\pgfpathrectangle{\pgfqpoint{0.100000in}{0.100000in}}{\pgfqpoint{2.989028in}{1.913466in}}%
\pgfusepath{clip}%
\pgfsetbuttcap%
\pgfsetmiterjoin%
\definecolor{currentfill}{rgb}{0.711419,0.883276,0.634833}%
\pgfsetfillcolor{currentfill}%
\pgfsetlinewidth{0.000000pt}%
\definecolor{currentstroke}{rgb}{0.000000,0.000000,0.000000}%
\pgfsetstrokecolor{currentstroke}%
\pgfsetstrokeopacity{0.000000}%
\pgfsetdash{}{0pt}%
\pgfpathmoveto{\pgfqpoint{0.810222in}{1.900931in}}%
\pgfpathlineto{\pgfqpoint{0.794722in}{1.839512in}}%
\pgfpathlineto{\pgfqpoint{0.783700in}{1.795504in}}%
\pgfpathlineto{\pgfqpoint{0.770595in}{1.742564in}}%
\pgfpathlineto{\pgfqpoint{0.768022in}{1.729516in}}%
\pgfpathlineto{\pgfqpoint{0.769703in}{1.717547in}}%
\pgfpathlineto{\pgfqpoint{0.767475in}{1.706523in}}%
\pgfpathlineto{\pgfqpoint{0.675998in}{1.730394in}}%
\pgfpathlineto{\pgfqpoint{0.667734in}{1.727622in}}%
\pgfpathlineto{\pgfqpoint{0.660663in}{1.730028in}}%
\pgfpathlineto{\pgfqpoint{0.627921in}{1.730814in}}%
\pgfpathlineto{\pgfqpoint{0.616848in}{1.727692in}}%
\pgfpathlineto{\pgfqpoint{0.605851in}{1.728747in}}%
\pgfpathlineto{\pgfqpoint{0.601190in}{1.733634in}}%
\pgfpathlineto{\pgfqpoint{0.571877in}{1.732568in}}%
\pgfpathlineto{\pgfqpoint{0.567248in}{1.740554in}}%
\pgfpathlineto{\pgfqpoint{0.558460in}{1.744762in}}%
\pgfpathlineto{\pgfqpoint{0.545673in}{1.747325in}}%
\pgfpathlineto{\pgfqpoint{0.523603in}{1.743571in}}%
\pgfpathlineto{\pgfqpoint{0.515456in}{1.747249in}}%
\pgfpathlineto{\pgfqpoint{0.502895in}{1.757031in}}%
\pgfpathlineto{\pgfqpoint{0.506670in}{1.776202in}}%
\pgfpathlineto{\pgfqpoint{0.505347in}{1.785992in}}%
\pgfpathlineto{\pgfqpoint{0.495391in}{1.796386in}}%
\pgfpathlineto{\pgfqpoint{0.489015in}{1.795741in}}%
\pgfpathlineto{\pgfqpoint{0.484440in}{1.806255in}}%
\pgfpathlineto{\pgfqpoint{0.473604in}{1.810504in}}%
\pgfpathlineto{\pgfqpoint{0.465727in}{1.809963in}}%
\pgfpathlineto{\pgfqpoint{0.463179in}{1.820760in}}%
\pgfpathlineto{\pgfqpoint{0.471026in}{1.819611in}}%
\pgfpathlineto{\pgfqpoint{0.470402in}{1.834619in}}%
\pgfpathlineto{\pgfqpoint{0.466982in}{1.843530in}}%
\pgfpathlineto{\pgfqpoint{0.468490in}{1.854935in}}%
\pgfpathlineto{\pgfqpoint{0.472595in}{1.863514in}}%
\pgfpathlineto{\pgfqpoint{0.472364in}{1.879792in}}%
\pgfpathlineto{\pgfqpoint{0.470172in}{1.885699in}}%
\pgfpathlineto{\pgfqpoint{0.473957in}{1.904648in}}%
\pgfpathlineto{\pgfqpoint{0.472865in}{1.916860in}}%
\pgfpathlineto{\pgfqpoint{0.469081in}{1.922716in}}%
\pgfpathlineto{\pgfqpoint{0.469658in}{1.941861in}}%
\pgfpathlineto{\pgfqpoint{0.475162in}{1.955941in}}%
\pgfpathlineto{\pgfqpoint{0.481149in}{1.952495in}}%
\pgfpathlineto{\pgfqpoint{0.500972in}{1.932018in}}%
\pgfpathlineto{\pgfqpoint{0.524993in}{1.920812in}}%
\pgfpathlineto{\pgfqpoint{0.537292in}{1.919486in}}%
\pgfpathlineto{\pgfqpoint{0.548880in}{1.914715in}}%
\pgfpathlineto{\pgfqpoint{0.552168in}{1.898661in}}%
\pgfpathlineto{\pgfqpoint{0.542014in}{1.895620in}}%
\pgfpathlineto{\pgfqpoint{0.535285in}{1.882967in}}%
\pgfpathlineto{\pgfqpoint{0.543185in}{1.883670in}}%
\pgfpathlineto{\pgfqpoint{0.546372in}{1.889349in}}%
\pgfpathlineto{\pgfqpoint{0.557514in}{1.896451in}}%
\pgfpathlineto{\pgfqpoint{0.556939in}{1.885905in}}%
\pgfpathlineto{\pgfqpoint{0.549499in}{1.884204in}}%
\pgfpathlineto{\pgfqpoint{0.550734in}{1.870648in}}%
\pgfpathlineto{\pgfqpoint{0.541674in}{1.857069in}}%
\pgfpathlineto{\pgfqpoint{0.535876in}{1.863526in}}%
\pgfpathlineto{\pgfqpoint{0.518645in}{1.860015in}}%
\pgfpathlineto{\pgfqpoint{0.517738in}{1.852116in}}%
\pgfpathlineto{\pgfqpoint{0.523678in}{1.847303in}}%
\pgfpathlineto{\pgfqpoint{0.532741in}{1.846830in}}%
\pgfpathlineto{\pgfqpoint{0.545271in}{1.857114in}}%
\pgfpathlineto{\pgfqpoint{0.555189in}{1.857968in}}%
\pgfpathlineto{\pgfqpoint{0.555569in}{1.869352in}}%
\pgfpathlineto{\pgfqpoint{0.560680in}{1.886072in}}%
\pgfpathlineto{\pgfqpoint{0.572793in}{1.898506in}}%
\pgfpathlineto{\pgfqpoint{0.568754in}{1.908138in}}%
\pgfpathlineto{\pgfqpoint{0.571513in}{1.918507in}}%
\pgfpathlineto{\pgfqpoint{0.566139in}{1.928842in}}%
\pgfpathlineto{\pgfqpoint{0.573798in}{1.942057in}}%
\pgfpathlineto{\pgfqpoint{0.574945in}{1.950008in}}%
\pgfpathlineto{\pgfqpoint{0.568296in}{1.955234in}}%
\pgfpathlineto{\pgfqpoint{0.569403in}{1.968600in}}%
\pgfpathlineto{\pgfqpoint{0.649072in}{1.944395in}}%
\pgfpathlineto{\pgfqpoint{0.733673in}{1.920671in}}%
\pgfpathlineto{\pgfqpoint{0.810222in}{1.900931in}}%
\pgfpathclose%
\pgfusepath{fill}%
\end{pgfscope}%
\begin{pgfscope}%
\pgfpathrectangle{\pgfqpoint{0.100000in}{0.100000in}}{\pgfqpoint{2.989028in}{1.913466in}}%
\pgfusepath{clip}%
\pgfsetbuttcap%
\pgfsetmiterjoin%
\definecolor{currentfill}{rgb}{0.711419,0.883276,0.634833}%
\pgfsetfillcolor{currentfill}%
\pgfsetlinewidth{0.000000pt}%
\definecolor{currentstroke}{rgb}{0.000000,0.000000,0.000000}%
\pgfsetstrokecolor{currentstroke}%
\pgfsetstrokeopacity{0.000000}%
\pgfsetdash{}{0pt}%
\pgfpathmoveto{\pgfqpoint{0.556297in}{1.922359in}}%
\pgfpathlineto{\pgfqpoint{0.560220in}{1.907235in}}%
\pgfpathlineto{\pgfqpoint{0.566372in}{1.902211in}}%
\pgfpathlineto{\pgfqpoint{0.561463in}{1.895816in}}%
\pgfpathlineto{\pgfqpoint{0.556679in}{1.905186in}}%
\pgfpathlineto{\pgfqpoint{0.556297in}{1.922359in}}%
\pgfpathclose%
\pgfusepath{fill}%
\end{pgfscope}%
\begin{pgfscope}%
\pgfpathrectangle{\pgfqpoint{0.100000in}{0.100000in}}{\pgfqpoint{2.989028in}{1.913466in}}%
\pgfusepath{clip}%
\pgfsetbuttcap%
\pgfsetmiterjoin%
\definecolor{currentfill}{rgb}{0.793080,0.916494,0.618224}%
\pgfsetfillcolor{currentfill}%
\pgfsetlinewidth{0.000000pt}%
\definecolor{currentstroke}{rgb}{0.000000,0.000000,0.000000}%
\pgfsetstrokecolor{currentstroke}%
\pgfsetstrokeopacity{0.000000}%
\pgfsetdash{}{0pt}%
\pgfpathmoveto{\pgfqpoint{0.851305in}{1.891004in}}%
\pgfpathlineto{\pgfqpoint{0.936467in}{1.871893in}}%
\pgfpathlineto{\pgfqpoint{1.016670in}{1.855681in}}%
\pgfpathlineto{\pgfqpoint{1.078378in}{1.844375in}}%
\pgfpathlineto{\pgfqpoint{1.132175in}{1.835337in}}%
\pgfpathlineto{\pgfqpoint{1.186092in}{1.827040in}}%
\pgfpathlineto{\pgfqpoint{1.232007in}{1.820572in}}%
\pgfpathlineto{\pgfqpoint{1.277995in}{1.814641in}}%
\pgfpathlineto{\pgfqpoint{1.324048in}{1.809248in}}%
\pgfpathlineto{\pgfqpoint{1.367448in}{1.804664in}}%
\pgfpathlineto{\pgfqpoint{1.361479in}{1.738522in}}%
\pgfpathlineto{\pgfqpoint{1.352595in}{1.649140in}}%
\pgfpathlineto{\pgfqpoint{1.347936in}{1.603169in}}%
\pgfpathlineto{\pgfqpoint{1.341234in}{1.541215in}}%
\pgfpathlineto{\pgfqpoint{1.293794in}{1.546395in}}%
\pgfpathlineto{\pgfqpoint{1.239479in}{1.552561in}}%
\pgfpathlineto{\pgfqpoint{1.164064in}{1.562882in}}%
\pgfpathlineto{\pgfqpoint{1.130381in}{1.567687in}}%
\pgfpathlineto{\pgfqpoint{1.047396in}{1.580713in}}%
\pgfpathlineto{\pgfqpoint{1.018829in}{1.585939in}}%
\pgfpathlineto{\pgfqpoint{1.012875in}{1.552087in}}%
\pgfpathlineto{\pgfqpoint{1.009620in}{1.554502in}}%
\pgfpathlineto{\pgfqpoint{1.003502in}{1.570788in}}%
\pgfpathlineto{\pgfqpoint{0.995639in}{1.569785in}}%
\pgfpathlineto{\pgfqpoint{0.990011in}{1.561552in}}%
\pgfpathlineto{\pgfqpoint{0.978331in}{1.560451in}}%
\pgfpathlineto{\pgfqpoint{0.975984in}{1.564390in}}%
\pgfpathlineto{\pgfqpoint{0.965043in}{1.563913in}}%
\pgfpathlineto{\pgfqpoint{0.959543in}{1.567727in}}%
\pgfpathlineto{\pgfqpoint{0.951852in}{1.561826in}}%
\pgfpathlineto{\pgfqpoint{0.933172in}{1.567136in}}%
\pgfpathlineto{\pgfqpoint{0.925020in}{1.564285in}}%
\pgfpathlineto{\pgfqpoint{0.920634in}{1.576938in}}%
\pgfpathlineto{\pgfqpoint{0.920429in}{1.588978in}}%
\pgfpathlineto{\pgfqpoint{0.907552in}{1.597637in}}%
\pgfpathlineto{\pgfqpoint{0.909652in}{1.610116in}}%
\pgfpathlineto{\pgfqpoint{0.900483in}{1.630729in}}%
\pgfpathlineto{\pgfqpoint{0.901101in}{1.648251in}}%
\pgfpathlineto{\pgfqpoint{0.893220in}{1.656827in}}%
\pgfpathlineto{\pgfqpoint{0.885827in}{1.649240in}}%
\pgfpathlineto{\pgfqpoint{0.875763in}{1.645136in}}%
\pgfpathlineto{\pgfqpoint{0.868102in}{1.653594in}}%
\pgfpathlineto{\pgfqpoint{0.869348in}{1.667101in}}%
\pgfpathlineto{\pgfqpoint{0.878535in}{1.672422in}}%
\pgfpathlineto{\pgfqpoint{0.876093in}{1.680980in}}%
\pgfpathlineto{\pgfqpoint{0.892063in}{1.722399in}}%
\pgfpathlineto{\pgfqpoint{0.879470in}{1.723437in}}%
\pgfpathlineto{\pgfqpoint{0.878176in}{1.730870in}}%
\pgfpathlineto{\pgfqpoint{0.869003in}{1.737279in}}%
\pgfpathlineto{\pgfqpoint{0.869589in}{1.744445in}}%
\pgfpathlineto{\pgfqpoint{0.864750in}{1.750011in}}%
\pgfpathlineto{\pgfqpoint{0.856873in}{1.770222in}}%
\pgfpathlineto{\pgfqpoint{0.849675in}{1.774343in}}%
\pgfpathlineto{\pgfqpoint{0.844521in}{1.791786in}}%
\pgfpathlineto{\pgfqpoint{0.845737in}{1.803380in}}%
\pgfpathlineto{\pgfqpoint{0.836121in}{1.824733in}}%
\pgfpathlineto{\pgfqpoint{0.851305in}{1.891004in}}%
\pgfpathclose%
\pgfusepath{fill}%
\end{pgfscope}%
\begin{pgfscope}%
\pgfpathrectangle{\pgfqpoint{0.100000in}{0.100000in}}{\pgfqpoint{2.989028in}{1.913466in}}%
\pgfusepath{clip}%
\pgfsetbuttcap%
\pgfsetmiterjoin%
\definecolor{currentfill}{rgb}{0.963476,0.985390,0.692042}%
\pgfsetfillcolor{currentfill}%
\pgfsetlinewidth{0.000000pt}%
\definecolor{currentstroke}{rgb}{0.000000,0.000000,0.000000}%
\pgfsetstrokecolor{currentstroke}%
\pgfsetstrokeopacity{0.000000}%
\pgfsetdash{}{0pt}%
\pgfpathmoveto{\pgfqpoint{2.907175in}{1.550163in}}%
\pgfpathlineto{\pgfqpoint{2.905471in}{1.557411in}}%
\pgfpathlineto{\pgfqpoint{2.895998in}{1.563760in}}%
\pgfpathlineto{\pgfqpoint{2.870964in}{1.645645in}}%
\pgfpathlineto{\pgfqpoint{2.857471in}{1.685161in}}%
\pgfpathlineto{\pgfqpoint{2.868319in}{1.696546in}}%
\pgfpathlineto{\pgfqpoint{2.878864in}{1.724407in}}%
\pgfpathlineto{\pgfqpoint{2.884123in}{1.733311in}}%
\pgfpathlineto{\pgfqpoint{2.880397in}{1.737050in}}%
\pgfpathlineto{\pgfqpoint{2.879149in}{1.761782in}}%
\pgfpathlineto{\pgfqpoint{2.883854in}{1.769364in}}%
\pgfpathlineto{\pgfqpoint{2.881792in}{1.786986in}}%
\pgfpathlineto{\pgfqpoint{2.900570in}{1.844804in}}%
\pgfpathlineto{\pgfqpoint{2.909031in}{1.845120in}}%
\pgfpathlineto{\pgfqpoint{2.912543in}{1.834792in}}%
\pgfpathlineto{\pgfqpoint{2.920062in}{1.831826in}}%
\pgfpathlineto{\pgfqpoint{2.934266in}{1.843974in}}%
\pgfpathlineto{\pgfqpoint{2.945403in}{1.851149in}}%
\pgfpathlineto{\pgfqpoint{2.969943in}{1.838514in}}%
\pgfpathlineto{\pgfqpoint{2.992113in}{1.768360in}}%
\pgfpathlineto{\pgfqpoint{2.996328in}{1.751101in}}%
\pgfpathlineto{\pgfqpoint{3.006039in}{1.749076in}}%
\pgfpathlineto{\pgfqpoint{3.019209in}{1.737354in}}%
\pgfpathlineto{\pgfqpoint{3.018463in}{1.730524in}}%
\pgfpathlineto{\pgfqpoint{3.027435in}{1.722428in}}%
\pgfpathlineto{\pgfqpoint{3.034723in}{1.727073in}}%
\pgfpathlineto{\pgfqpoint{3.049974in}{1.709287in}}%
\pgfpathlineto{\pgfqpoint{3.043158in}{1.695053in}}%
\pgfpathlineto{\pgfqpoint{3.034018in}{1.694771in}}%
\pgfpathlineto{\pgfqpoint{3.026685in}{1.682065in}}%
\pgfpathlineto{\pgfqpoint{3.017802in}{1.680263in}}%
\pgfpathlineto{\pgfqpoint{3.011280in}{1.673497in}}%
\pgfpathlineto{\pgfqpoint{2.991803in}{1.666241in}}%
\pgfpathlineto{\pgfqpoint{2.980014in}{1.654548in}}%
\pgfpathlineto{\pgfqpoint{2.974025in}{1.662941in}}%
\pgfpathlineto{\pgfqpoint{2.968584in}{1.656926in}}%
\pgfpathlineto{\pgfqpoint{2.970074in}{1.632607in}}%
\pgfpathlineto{\pgfqpoint{2.965765in}{1.622974in}}%
\pgfpathlineto{\pgfqpoint{2.956367in}{1.625605in}}%
\pgfpathlineto{\pgfqpoint{2.954850in}{1.615730in}}%
\pgfpathlineto{\pgfqpoint{2.950789in}{1.611669in}}%
\pgfpathlineto{\pgfqpoint{2.942662in}{1.612659in}}%
\pgfpathlineto{\pgfqpoint{2.943174in}{1.603425in}}%
\pgfpathlineto{\pgfqpoint{2.930855in}{1.606048in}}%
\pgfpathlineto{\pgfqpoint{2.924128in}{1.593255in}}%
\pgfpathlineto{\pgfqpoint{2.926668in}{1.586561in}}%
\pgfpathlineto{\pgfqpoint{2.922666in}{1.575429in}}%
\pgfpathlineto{\pgfqpoint{2.916367in}{1.567240in}}%
\pgfpathlineto{\pgfqpoint{2.914781in}{1.550146in}}%
\pgfpathlineto{\pgfqpoint{2.907175in}{1.550163in}}%
\pgfpathclose%
\pgfusepath{fill}%
\end{pgfscope}%
\begin{pgfscope}%
\pgfpathrectangle{\pgfqpoint{0.100000in}{0.100000in}}{\pgfqpoint{2.989028in}{1.913466in}}%
\pgfusepath{clip}%
\pgfsetbuttcap%
\pgfsetmiterjoin%
\definecolor{currentfill}{rgb}{0.963476,0.985390,0.692042}%
\pgfsetfillcolor{currentfill}%
\pgfsetlinewidth{0.000000pt}%
\definecolor{currentstroke}{rgb}{0.000000,0.000000,0.000000}%
\pgfsetstrokecolor{currentstroke}%
\pgfsetstrokeopacity{0.000000}%
\pgfsetdash{}{0pt}%
\pgfpathmoveto{\pgfqpoint{2.995301in}{1.661247in}}%
\pgfpathlineto{\pgfqpoint{3.000863in}{1.667073in}}%
\pgfpathlineto{\pgfqpoint{3.006158in}{1.661597in}}%
\pgfpathlineto{\pgfqpoint{2.996647in}{1.654342in}}%
\pgfpathlineto{\pgfqpoint{2.995301in}{1.661247in}}%
\pgfpathclose%
\pgfusepath{fill}%
\end{pgfscope}%
\begin{pgfscope}%
\pgfpathrectangle{\pgfqpoint{0.100000in}{0.100000in}}{\pgfqpoint{2.989028in}{1.913466in}}%
\pgfusepath{clip}%
\pgfsetbuttcap%
\pgfsetmiterjoin%
\definecolor{currentfill}{rgb}{0.591003,0.835525,0.644291}%
\pgfsetfillcolor{currentfill}%
\pgfsetlinewidth{0.000000pt}%
\definecolor{currentstroke}{rgb}{0.000000,0.000000,0.000000}%
\pgfsetstrokecolor{currentstroke}%
\pgfsetstrokeopacity{0.000000}%
\pgfsetdash{}{0pt}%
\pgfpathmoveto{\pgfqpoint{1.347936in}{1.603169in}}%
\pgfpathlineto{\pgfqpoint{1.352595in}{1.649140in}}%
\pgfpathlineto{\pgfqpoint{1.361479in}{1.738522in}}%
\pgfpathlineto{\pgfqpoint{1.367448in}{1.804664in}}%
\pgfpathlineto{\pgfqpoint{1.416329in}{1.800078in}}%
\pgfpathlineto{\pgfqpoint{1.478862in}{1.795100in}}%
\pgfpathlineto{\pgfqpoint{1.536021in}{1.791418in}}%
\pgfpathlineto{\pgfqpoint{1.587777in}{1.788799in}}%
\pgfpathlineto{\pgfqpoint{1.665014in}{1.786150in}}%
\pgfpathlineto{\pgfqpoint{1.670285in}{1.764381in}}%
\pgfpathlineto{\pgfqpoint{1.668398in}{1.753978in}}%
\pgfpathlineto{\pgfqpoint{1.667777in}{1.732596in}}%
\pgfpathlineto{\pgfqpoint{1.671363in}{1.716582in}}%
\pgfpathlineto{\pgfqpoint{1.679579in}{1.692956in}}%
\pgfpathlineto{\pgfqpoint{1.679560in}{1.663212in}}%
\pgfpathlineto{\pgfqpoint{1.681080in}{1.628658in}}%
\pgfpathlineto{\pgfqpoint{1.683167in}{1.619346in}}%
\pgfpathlineto{\pgfqpoint{1.689274in}{1.609122in}}%
\pgfpathlineto{\pgfqpoint{1.691288in}{1.593192in}}%
\pgfpathlineto{\pgfqpoint{1.690419in}{1.582557in}}%
\pgfpathlineto{\pgfqpoint{1.625668in}{1.583959in}}%
\pgfpathlineto{\pgfqpoint{1.578564in}{1.586347in}}%
\pgfpathlineto{\pgfqpoint{1.509511in}{1.590023in}}%
\pgfpathlineto{\pgfqpoint{1.441415in}{1.594860in}}%
\pgfpathlineto{\pgfqpoint{1.396059in}{1.598519in}}%
\pgfpathlineto{\pgfqpoint{1.347936in}{1.603169in}}%
\pgfpathclose%
\pgfusepath{fill}%
\end{pgfscope}%
\begin{pgfscope}%
\pgfpathrectangle{\pgfqpoint{0.100000in}{0.100000in}}{\pgfqpoint{2.989028in}{1.913466in}}%
\pgfusepath{clip}%
\pgfsetbuttcap%
\pgfsetmiterjoin%
\definecolor{currentfill}{rgb}{0.940408,0.976163,0.656055}%
\pgfsetfillcolor{currentfill}%
\pgfsetlinewidth{0.000000pt}%
\definecolor{currentstroke}{rgb}{0.000000,0.000000,0.000000}%
\pgfsetstrokecolor{currentstroke}%
\pgfsetstrokeopacity{0.000000}%
\pgfsetdash{}{0pt}%
\pgfpathmoveto{\pgfqpoint{1.328363in}{1.410780in}}%
\pgfpathlineto{\pgfqpoint{1.335839in}{1.487840in}}%
\pgfpathlineto{\pgfqpoint{1.341234in}{1.541215in}}%
\pgfpathlineto{\pgfqpoint{1.347936in}{1.603169in}}%
\pgfpathlineto{\pgfqpoint{1.396059in}{1.598519in}}%
\pgfpathlineto{\pgfqpoint{1.441415in}{1.594860in}}%
\pgfpathlineto{\pgfqpoint{1.509511in}{1.590023in}}%
\pgfpathlineto{\pgfqpoint{1.578564in}{1.586347in}}%
\pgfpathlineto{\pgfqpoint{1.625668in}{1.583959in}}%
\pgfpathlineto{\pgfqpoint{1.690419in}{1.582557in}}%
\pgfpathlineto{\pgfqpoint{1.686034in}{1.569764in}}%
\pgfpathlineto{\pgfqpoint{1.677284in}{1.559721in}}%
\pgfpathlineto{\pgfqpoint{1.683985in}{1.548158in}}%
\pgfpathlineto{\pgfqpoint{1.691374in}{1.545688in}}%
\pgfpathlineto{\pgfqpoint{1.694880in}{1.539046in}}%
\pgfpathlineto{\pgfqpoint{1.694075in}{1.490552in}}%
\pgfpathlineto{\pgfqpoint{1.692728in}{1.422304in}}%
\pgfpathlineto{\pgfqpoint{1.687744in}{1.404379in}}%
\pgfpathlineto{\pgfqpoint{1.692201in}{1.394464in}}%
\pgfpathlineto{\pgfqpoint{1.683252in}{1.373157in}}%
\pgfpathlineto{\pgfqpoint{1.692680in}{1.355981in}}%
\pgfpathlineto{\pgfqpoint{1.684670in}{1.357295in}}%
\pgfpathlineto{\pgfqpoint{1.679205in}{1.367982in}}%
\pgfpathlineto{\pgfqpoint{1.659700in}{1.375307in}}%
\pgfpathlineto{\pgfqpoint{1.647400in}{1.381750in}}%
\pgfpathlineto{\pgfqpoint{1.626759in}{1.382286in}}%
\pgfpathlineto{\pgfqpoint{1.619594in}{1.376416in}}%
\pgfpathlineto{\pgfqpoint{1.596231in}{1.387975in}}%
\pgfpathlineto{\pgfqpoint{1.594439in}{1.391630in}}%
\pgfpathlineto{\pgfqpoint{1.512902in}{1.395489in}}%
\pgfpathlineto{\pgfqpoint{1.463365in}{1.398329in}}%
\pgfpathlineto{\pgfqpoint{1.422447in}{1.401533in}}%
\pgfpathlineto{\pgfqpoint{1.354836in}{1.407921in}}%
\pgfpathlineto{\pgfqpoint{1.328363in}{1.410780in}}%
\pgfpathclose%
\pgfusepath{fill}%
\end{pgfscope}%
\begin{pgfscope}%
\pgfpathrectangle{\pgfqpoint{0.100000in}{0.100000in}}{\pgfqpoint{2.989028in}{1.913466in}}%
\pgfusepath{clip}%
\pgfsetbuttcap%
\pgfsetmiterjoin%
\definecolor{currentfill}{rgb}{0.368627,0.309804,0.635294}%
\pgfsetfillcolor{currentfill}%
\pgfsetlinewidth{0.000000pt}%
\definecolor{currentstroke}{rgb}{0.000000,0.000000,0.000000}%
\pgfsetstrokecolor{currentstroke}%
\pgfsetstrokeopacity{0.000000}%
\pgfsetdash{}{0pt}%
\pgfpathmoveto{\pgfqpoint{1.315540in}{1.280295in}}%
\pgfpathlineto{\pgfqpoint{1.272072in}{1.284265in}}%
\pgfpathlineto{\pgfqpoint{1.177321in}{1.295959in}}%
\pgfpathlineto{\pgfqpoint{1.125778in}{1.303357in}}%
\pgfpathlineto{\pgfqpoint{1.070495in}{1.311293in}}%
\pgfpathlineto{\pgfqpoint{1.023928in}{1.318744in}}%
\pgfpathlineto{\pgfqpoint{0.972815in}{1.327496in}}%
\pgfpathlineto{\pgfqpoint{0.984436in}{1.391963in}}%
\pgfpathlineto{\pgfqpoint{0.996208in}{1.458068in}}%
\pgfpathlineto{\pgfqpoint{1.012875in}{1.552087in}}%
\pgfpathlineto{\pgfqpoint{1.018829in}{1.585939in}}%
\pgfpathlineto{\pgfqpoint{1.047396in}{1.580713in}}%
\pgfpathlineto{\pgfqpoint{1.130381in}{1.567687in}}%
\pgfpathlineto{\pgfqpoint{1.164064in}{1.562882in}}%
\pgfpathlineto{\pgfqpoint{1.239479in}{1.552561in}}%
\pgfpathlineto{\pgfqpoint{1.293794in}{1.546395in}}%
\pgfpathlineto{\pgfqpoint{1.341234in}{1.541215in}}%
\pgfpathlineto{\pgfqpoint{1.335839in}{1.487840in}}%
\pgfpathlineto{\pgfqpoint{1.328363in}{1.410780in}}%
\pgfpathlineto{\pgfqpoint{1.321947in}{1.345289in}}%
\pgfpathlineto{\pgfqpoint{1.315540in}{1.280295in}}%
\pgfpathclose%
\pgfusepath{fill}%
\end{pgfscope}%
\begin{pgfscope}%
\pgfpathrectangle{\pgfqpoint{0.100000in}{0.100000in}}{\pgfqpoint{2.989028in}{1.913466in}}%
\pgfusepath{clip}%
\pgfsetbuttcap%
\pgfsetmiterjoin%
\definecolor{currentfill}{rgb}{0.999616,0.988082,0.729027}%
\pgfsetfillcolor{currentfill}%
\pgfsetlinewidth{0.000000pt}%
\definecolor{currentstroke}{rgb}{0.000000,0.000000,0.000000}%
\pgfsetstrokecolor{currentstroke}%
\pgfsetstrokeopacity{0.000000}%
\pgfsetdash{}{0pt}%
\pgfpathmoveto{\pgfqpoint{2.110291in}{1.369221in}}%
\pgfpathlineto{\pgfqpoint{2.055206in}{1.365308in}}%
\pgfpathlineto{\pgfqpoint{1.973100in}{1.361809in}}%
\pgfpathlineto{\pgfqpoint{1.969980in}{1.370103in}}%
\pgfpathlineto{\pgfqpoint{1.951772in}{1.376303in}}%
\pgfpathlineto{\pgfqpoint{1.947756in}{1.388016in}}%
\pgfpathlineto{\pgfqpoint{1.946075in}{1.402507in}}%
\pgfpathlineto{\pgfqpoint{1.950178in}{1.409931in}}%
\pgfpathlineto{\pgfqpoint{1.943697in}{1.417056in}}%
\pgfpathlineto{\pgfqpoint{1.942134in}{1.425554in}}%
\pgfpathlineto{\pgfqpoint{1.940045in}{1.444352in}}%
\pgfpathlineto{\pgfqpoint{1.933835in}{1.454564in}}%
\pgfpathlineto{\pgfqpoint{1.922805in}{1.460295in}}%
\pgfpathlineto{\pgfqpoint{1.910803in}{1.469752in}}%
\pgfpathlineto{\pgfqpoint{1.904599in}{1.480942in}}%
\pgfpathlineto{\pgfqpoint{1.893466in}{1.485448in}}%
\pgfpathlineto{\pgfqpoint{1.886922in}{1.492782in}}%
\pgfpathlineto{\pgfqpoint{1.878994in}{1.494027in}}%
\pgfpathlineto{\pgfqpoint{1.864840in}{1.504916in}}%
\pgfpathlineto{\pgfqpoint{1.867139in}{1.517447in}}%
\pgfpathlineto{\pgfqpoint{1.866696in}{1.541277in}}%
\pgfpathlineto{\pgfqpoint{1.871061in}{1.547836in}}%
\pgfpathlineto{\pgfqpoint{1.867139in}{1.557741in}}%
\pgfpathlineto{\pgfqpoint{1.860230in}{1.559657in}}%
\pgfpathlineto{\pgfqpoint{1.860801in}{1.568355in}}%
\pgfpathlineto{\pgfqpoint{1.869373in}{1.582098in}}%
\pgfpathlineto{\pgfqpoint{1.886348in}{1.592965in}}%
\pgfpathlineto{\pgfqpoint{1.885290in}{1.631611in}}%
\pgfpathlineto{\pgfqpoint{1.893784in}{1.637416in}}%
\pgfpathlineto{\pgfqpoint{1.901822in}{1.633549in}}%
\pgfpathlineto{\pgfqpoint{1.918197in}{1.639211in}}%
\pgfpathlineto{\pgfqpoint{1.949004in}{1.653439in}}%
\pgfpathlineto{\pgfqpoint{1.953017in}{1.649033in}}%
\pgfpathlineto{\pgfqpoint{1.947182in}{1.629079in}}%
\pgfpathlineto{\pgfqpoint{1.955863in}{1.633458in}}%
\pgfpathlineto{\pgfqpoint{1.970731in}{1.629085in}}%
\pgfpathlineto{\pgfqpoint{1.979867in}{1.625436in}}%
\pgfpathlineto{\pgfqpoint{1.984989in}{1.614740in}}%
\pgfpathlineto{\pgfqpoint{2.031860in}{1.604646in}}%
\pgfpathlineto{\pgfqpoint{2.045871in}{1.597722in}}%
\pgfpathlineto{\pgfqpoint{2.060116in}{1.597803in}}%
\pgfpathlineto{\pgfqpoint{2.074752in}{1.594947in}}%
\pgfpathlineto{\pgfqpoint{2.084251in}{1.585189in}}%
\pgfpathlineto{\pgfqpoint{2.093328in}{1.580368in}}%
\pgfpathlineto{\pgfqpoint{2.094996in}{1.566462in}}%
\pgfpathlineto{\pgfqpoint{2.092314in}{1.557730in}}%
\pgfpathlineto{\pgfqpoint{2.102430in}{1.557084in}}%
\pgfpathlineto{\pgfqpoint{2.099022in}{1.546957in}}%
\pgfpathlineto{\pgfqpoint{2.102264in}{1.543360in}}%
\pgfpathlineto{\pgfqpoint{2.105488in}{1.533800in}}%
\pgfpathlineto{\pgfqpoint{2.095629in}{1.528739in}}%
\pgfpathlineto{\pgfqpoint{2.089905in}{1.514637in}}%
\pgfpathlineto{\pgfqpoint{2.088109in}{1.504665in}}%
\pgfpathlineto{\pgfqpoint{2.093600in}{1.502958in}}%
\pgfpathlineto{\pgfqpoint{2.100632in}{1.510436in}}%
\pgfpathlineto{\pgfqpoint{2.106610in}{1.523394in}}%
\pgfpathlineto{\pgfqpoint{2.114698in}{1.527895in}}%
\pgfpathlineto{\pgfqpoint{2.120776in}{1.522041in}}%
\pgfpathlineto{\pgfqpoint{2.114793in}{1.504322in}}%
\pgfpathlineto{\pgfqpoint{2.112916in}{1.490567in}}%
\pgfpathlineto{\pgfqpoint{2.114684in}{1.480679in}}%
\pgfpathlineto{\pgfqpoint{2.109128in}{1.475085in}}%
\pgfpathlineto{\pgfqpoint{2.106353in}{1.461416in}}%
\pgfpathlineto{\pgfqpoint{2.108620in}{1.447050in}}%
\pgfpathlineto{\pgfqpoint{2.102064in}{1.425804in}}%
\pgfpathlineto{\pgfqpoint{2.102202in}{1.415188in}}%
\pgfpathlineto{\pgfqpoint{2.107383in}{1.392182in}}%
\pgfpathlineto{\pgfqpoint{2.110741in}{1.388236in}}%
\pgfpathlineto{\pgfqpoint{2.110291in}{1.369221in}}%
\pgfpathclose%
\pgfusepath{fill}%
\end{pgfscope}%
\begin{pgfscope}%
\pgfpathrectangle{\pgfqpoint{0.100000in}{0.100000in}}{\pgfqpoint{2.989028in}{1.913466in}}%
\pgfusepath{clip}%
\pgfsetbuttcap%
\pgfsetmiterjoin%
\definecolor{currentfill}{rgb}{0.999616,0.988082,0.729027}%
\pgfsetfillcolor{currentfill}%
\pgfsetlinewidth{0.000000pt}%
\definecolor{currentstroke}{rgb}{0.000000,0.000000,0.000000}%
\pgfsetstrokecolor{currentstroke}%
\pgfsetstrokeopacity{0.000000}%
\pgfsetdash{}{0pt}%
\pgfpathmoveto{\pgfqpoint{2.130944in}{1.555675in}}%
\pgfpathlineto{\pgfqpoint{2.132173in}{1.546498in}}%
\pgfpathlineto{\pgfqpoint{2.120909in}{1.522320in}}%
\pgfpathlineto{\pgfqpoint{2.115904in}{1.529323in}}%
\pgfpathlineto{\pgfqpoint{2.130944in}{1.555675in}}%
\pgfpathclose%
\pgfusepath{fill}%
\end{pgfscope}%
\begin{pgfscope}%
\pgfpathrectangle{\pgfqpoint{0.100000in}{0.100000in}}{\pgfqpoint{2.989028in}{1.913466in}}%
\pgfusepath{clip}%
\pgfsetbuttcap%
\pgfsetmiterjoin%
\definecolor{currentfill}{rgb}{0.944252,0.977701,0.662053}%
\pgfsetfillcolor{currentfill}%
\pgfsetlinewidth{0.000000pt}%
\definecolor{currentstroke}{rgb}{0.000000,0.000000,0.000000}%
\pgfsetstrokecolor{currentstroke}%
\pgfsetstrokeopacity{0.000000}%
\pgfsetdash{}{0pt}%
\pgfpathmoveto{\pgfqpoint{0.767475in}{1.706523in}}%
\pgfpathlineto{\pgfqpoint{0.769703in}{1.717547in}}%
\pgfpathlineto{\pgfqpoint{0.768022in}{1.729516in}}%
\pgfpathlineto{\pgfqpoint{0.770595in}{1.742564in}}%
\pgfpathlineto{\pgfqpoint{0.783700in}{1.795504in}}%
\pgfpathlineto{\pgfqpoint{0.794722in}{1.839512in}}%
\pgfpathlineto{\pgfqpoint{0.810222in}{1.900931in}}%
\pgfpathlineto{\pgfqpoint{0.851305in}{1.891004in}}%
\pgfpathlineto{\pgfqpoint{0.836121in}{1.824733in}}%
\pgfpathlineto{\pgfqpoint{0.845737in}{1.803380in}}%
\pgfpathlineto{\pgfqpoint{0.844521in}{1.791786in}}%
\pgfpathlineto{\pgfqpoint{0.849675in}{1.774343in}}%
\pgfpathlineto{\pgfqpoint{0.856873in}{1.770222in}}%
\pgfpathlineto{\pgfqpoint{0.864750in}{1.750011in}}%
\pgfpathlineto{\pgfqpoint{0.869589in}{1.744445in}}%
\pgfpathlineto{\pgfqpoint{0.869003in}{1.737279in}}%
\pgfpathlineto{\pgfqpoint{0.878176in}{1.730870in}}%
\pgfpathlineto{\pgfqpoint{0.879470in}{1.723437in}}%
\pgfpathlineto{\pgfqpoint{0.892063in}{1.722399in}}%
\pgfpathlineto{\pgfqpoint{0.876093in}{1.680980in}}%
\pgfpathlineto{\pgfqpoint{0.878535in}{1.672422in}}%
\pgfpathlineto{\pgfqpoint{0.869348in}{1.667101in}}%
\pgfpathlineto{\pgfqpoint{0.868102in}{1.653594in}}%
\pgfpathlineto{\pgfqpoint{0.875763in}{1.645136in}}%
\pgfpathlineto{\pgfqpoint{0.885827in}{1.649240in}}%
\pgfpathlineto{\pgfqpoint{0.893220in}{1.656827in}}%
\pgfpathlineto{\pgfqpoint{0.901101in}{1.648251in}}%
\pgfpathlineto{\pgfqpoint{0.900483in}{1.630729in}}%
\pgfpathlineto{\pgfqpoint{0.909652in}{1.610116in}}%
\pgfpathlineto{\pgfqpoint{0.907552in}{1.597637in}}%
\pgfpathlineto{\pgfqpoint{0.920429in}{1.588978in}}%
\pgfpathlineto{\pgfqpoint{0.920634in}{1.576938in}}%
\pgfpathlineto{\pgfqpoint{0.925020in}{1.564285in}}%
\pgfpathlineto{\pgfqpoint{0.933172in}{1.567136in}}%
\pgfpathlineto{\pgfqpoint{0.951852in}{1.561826in}}%
\pgfpathlineto{\pgfqpoint{0.959543in}{1.567727in}}%
\pgfpathlineto{\pgfqpoint{0.965043in}{1.563913in}}%
\pgfpathlineto{\pgfqpoint{0.975984in}{1.564390in}}%
\pgfpathlineto{\pgfqpoint{0.978331in}{1.560451in}}%
\pgfpathlineto{\pgfqpoint{0.990011in}{1.561552in}}%
\pgfpathlineto{\pgfqpoint{0.995639in}{1.569785in}}%
\pgfpathlineto{\pgfqpoint{1.003502in}{1.570788in}}%
\pgfpathlineto{\pgfqpoint{1.009620in}{1.554502in}}%
\pgfpathlineto{\pgfqpoint{1.012875in}{1.552087in}}%
\pgfpathlineto{\pgfqpoint{0.996208in}{1.458068in}}%
\pgfpathlineto{\pgfqpoint{0.984436in}{1.391963in}}%
\pgfpathlineto{\pgfqpoint{0.891592in}{1.409886in}}%
\pgfpathlineto{\pgfqpoint{0.841442in}{1.419862in}}%
\pgfpathlineto{\pgfqpoint{0.794534in}{1.430177in}}%
\pgfpathlineto{\pgfqpoint{0.750689in}{1.440103in}}%
\pgfpathlineto{\pgfqpoint{0.699946in}{1.452341in}}%
\pgfpathlineto{\pgfqpoint{0.726119in}{1.559712in}}%
\pgfpathlineto{\pgfqpoint{0.727531in}{1.567559in}}%
\pgfpathlineto{\pgfqpoint{0.739170in}{1.588120in}}%
\pgfpathlineto{\pgfqpoint{0.727118in}{1.600478in}}%
\pgfpathlineto{\pgfqpoint{0.729620in}{1.612613in}}%
\pgfpathlineto{\pgfqpoint{0.734208in}{1.615650in}}%
\pgfpathlineto{\pgfqpoint{0.742402in}{1.628155in}}%
\pgfpathlineto{\pgfqpoint{0.754345in}{1.636806in}}%
\pgfpathlineto{\pgfqpoint{0.754997in}{1.643207in}}%
\pgfpathlineto{\pgfqpoint{0.762206in}{1.649603in}}%
\pgfpathlineto{\pgfqpoint{0.768191in}{1.661575in}}%
\pgfpathlineto{\pgfqpoint{0.780469in}{1.674240in}}%
\pgfpathlineto{\pgfqpoint{0.779597in}{1.686749in}}%
\pgfpathlineto{\pgfqpoint{0.771203in}{1.693698in}}%
\pgfpathlineto{\pgfqpoint{0.767475in}{1.706523in}}%
\pgfpathclose%
\pgfusepath{fill}%
\end{pgfscope}%
\begin{pgfscope}%
\pgfpathrectangle{\pgfqpoint{0.100000in}{0.100000in}}{\pgfqpoint{2.989028in}{1.913466in}}%
\pgfusepath{clip}%
\pgfsetbuttcap%
\pgfsetmiterjoin%
\definecolor{currentfill}{rgb}{0.995463,0.847674,0.519262}%
\pgfsetfillcolor{currentfill}%
\pgfsetlinewidth{0.000000pt}%
\definecolor{currentstroke}{rgb}{0.000000,0.000000,0.000000}%
\pgfsetstrokecolor{currentstroke}%
\pgfsetstrokeopacity{0.000000}%
\pgfsetdash{}{0pt}%
\pgfpathmoveto{\pgfqpoint{2.799865in}{1.496989in}}%
\pgfpathlineto{\pgfqpoint{2.796007in}{1.509167in}}%
\pgfpathlineto{\pgfqpoint{2.788848in}{1.546108in}}%
\pgfpathlineto{\pgfqpoint{2.779593in}{1.557473in}}%
\pgfpathlineto{\pgfqpoint{2.771474in}{1.577911in}}%
\pgfpathlineto{\pgfqpoint{2.774140in}{1.593064in}}%
\pgfpathlineto{\pgfqpoint{2.772002in}{1.604233in}}%
\pgfpathlineto{\pgfqpoint{2.765065in}{1.615202in}}%
\pgfpathlineto{\pgfqpoint{2.764780in}{1.627285in}}%
\pgfpathlineto{\pgfqpoint{2.760752in}{1.640318in}}%
\pgfpathlineto{\pgfqpoint{2.796797in}{1.649189in}}%
\pgfpathlineto{\pgfqpoint{2.843616in}{1.661826in}}%
\pgfpathlineto{\pgfqpoint{2.845485in}{1.654580in}}%
\pgfpathlineto{\pgfqpoint{2.842504in}{1.643043in}}%
\pgfpathlineto{\pgfqpoint{2.849502in}{1.633812in}}%
\pgfpathlineto{\pgfqpoint{2.845785in}{1.622115in}}%
\pgfpathlineto{\pgfqpoint{2.831015in}{1.607411in}}%
\pgfpathlineto{\pgfqpoint{2.835075in}{1.596366in}}%
\pgfpathlineto{\pgfqpoint{2.832066in}{1.573021in}}%
\pgfpathlineto{\pgfqpoint{2.827854in}{1.559858in}}%
\pgfpathlineto{\pgfqpoint{2.833228in}{1.522426in}}%
\pgfpathlineto{\pgfqpoint{2.832504in}{1.509256in}}%
\pgfpathlineto{\pgfqpoint{2.837703in}{1.505030in}}%
\pgfpathlineto{\pgfqpoint{2.799865in}{1.496989in}}%
\pgfpathclose%
\pgfusepath{fill}%
\end{pgfscope}%
\begin{pgfscope}%
\pgfpathrectangle{\pgfqpoint{0.100000in}{0.100000in}}{\pgfqpoint{2.989028in}{1.913466in}}%
\pgfusepath{clip}%
\pgfsetbuttcap%
\pgfsetmiterjoin%
\definecolor{currentfill}{rgb}{0.959631,0.983852,0.686044}%
\pgfsetfillcolor{currentfill}%
\pgfsetlinewidth{0.000000pt}%
\definecolor{currentstroke}{rgb}{0.000000,0.000000,0.000000}%
\pgfsetstrokecolor{currentstroke}%
\pgfsetstrokeopacity{0.000000}%
\pgfsetdash{}{0pt}%
\pgfpathmoveto{\pgfqpoint{1.692728in}{1.422304in}}%
\pgfpathlineto{\pgfqpoint{1.694075in}{1.490552in}}%
\pgfpathlineto{\pgfqpoint{1.694880in}{1.539046in}}%
\pgfpathlineto{\pgfqpoint{1.691374in}{1.545688in}}%
\pgfpathlineto{\pgfqpoint{1.683985in}{1.548158in}}%
\pgfpathlineto{\pgfqpoint{1.677284in}{1.559721in}}%
\pgfpathlineto{\pgfqpoint{1.686034in}{1.569764in}}%
\pgfpathlineto{\pgfqpoint{1.690419in}{1.582557in}}%
\pgfpathlineto{\pgfqpoint{1.691288in}{1.593192in}}%
\pgfpathlineto{\pgfqpoint{1.689274in}{1.609122in}}%
\pgfpathlineto{\pgfqpoint{1.683167in}{1.619346in}}%
\pgfpathlineto{\pgfqpoint{1.681080in}{1.628658in}}%
\pgfpathlineto{\pgfqpoint{1.679560in}{1.663212in}}%
\pgfpathlineto{\pgfqpoint{1.679579in}{1.692956in}}%
\pgfpathlineto{\pgfqpoint{1.671363in}{1.716582in}}%
\pgfpathlineto{\pgfqpoint{1.667777in}{1.732596in}}%
\pgfpathlineto{\pgfqpoint{1.668398in}{1.753978in}}%
\pgfpathlineto{\pgfqpoint{1.670285in}{1.764381in}}%
\pgfpathlineto{\pgfqpoint{1.665014in}{1.786150in}}%
\pgfpathlineto{\pgfqpoint{1.700900in}{1.785431in}}%
\pgfpathlineto{\pgfqpoint{1.755411in}{1.784963in}}%
\pgfpathlineto{\pgfqpoint{1.755709in}{1.809705in}}%
\pgfpathlineto{\pgfqpoint{1.769584in}{1.806981in}}%
\pgfpathlineto{\pgfqpoint{1.776233in}{1.776812in}}%
\pgfpathlineto{\pgfqpoint{1.781136in}{1.765965in}}%
\pgfpathlineto{\pgfqpoint{1.793328in}{1.765645in}}%
\pgfpathlineto{\pgfqpoint{1.796056in}{1.761964in}}%
\pgfpathlineto{\pgfqpoint{1.813062in}{1.760334in}}%
\pgfpathlineto{\pgfqpoint{1.815920in}{1.752855in}}%
\pgfpathlineto{\pgfqpoint{1.827640in}{1.754536in}}%
\pgfpathlineto{\pgfqpoint{1.836742in}{1.761531in}}%
\pgfpathlineto{\pgfqpoint{1.852440in}{1.761270in}}%
\pgfpathlineto{\pgfqpoint{1.862153in}{1.755644in}}%
\pgfpathlineto{\pgfqpoint{1.863270in}{1.750368in}}%
\pgfpathlineto{\pgfqpoint{1.872512in}{1.749264in}}%
\pgfpathlineto{\pgfqpoint{1.878543in}{1.734873in}}%
\pgfpathlineto{\pgfqpoint{1.882434in}{1.743721in}}%
\pgfpathlineto{\pgfqpoint{1.893051in}{1.744265in}}%
\pgfpathlineto{\pgfqpoint{1.895736in}{1.737374in}}%
\pgfpathlineto{\pgfqpoint{1.907682in}{1.734228in}}%
\pgfpathlineto{\pgfqpoint{1.914270in}{1.724302in}}%
\pgfpathlineto{\pgfqpoint{1.928880in}{1.727384in}}%
\pgfpathlineto{\pgfqpoint{1.944957in}{1.739565in}}%
\pgfpathlineto{\pgfqpoint{1.950818in}{1.728827in}}%
\pgfpathlineto{\pgfqpoint{1.977198in}{1.731766in}}%
\pgfpathlineto{\pgfqpoint{1.988474in}{1.724397in}}%
\pgfpathlineto{\pgfqpoint{1.995037in}{1.727028in}}%
\pgfpathlineto{\pgfqpoint{2.000300in}{1.722879in}}%
\pgfpathlineto{\pgfqpoint{1.984687in}{1.713035in}}%
\pgfpathlineto{\pgfqpoint{1.962401in}{1.704265in}}%
\pgfpathlineto{\pgfqpoint{1.940329in}{1.686734in}}%
\pgfpathlineto{\pgfqpoint{1.921193in}{1.663680in}}%
\pgfpathlineto{\pgfqpoint{1.906707in}{1.650050in}}%
\pgfpathlineto{\pgfqpoint{1.894018in}{1.640682in}}%
\pgfpathlineto{\pgfqpoint{1.885290in}{1.631611in}}%
\pgfpathlineto{\pgfqpoint{1.886348in}{1.592965in}}%
\pgfpathlineto{\pgfqpoint{1.869373in}{1.582098in}}%
\pgfpathlineto{\pgfqpoint{1.860801in}{1.568355in}}%
\pgfpathlineto{\pgfqpoint{1.860230in}{1.559657in}}%
\pgfpathlineto{\pgfqpoint{1.867139in}{1.557741in}}%
\pgfpathlineto{\pgfqpoint{1.871061in}{1.547836in}}%
\pgfpathlineto{\pgfqpoint{1.866696in}{1.541277in}}%
\pgfpathlineto{\pgfqpoint{1.867139in}{1.517447in}}%
\pgfpathlineto{\pgfqpoint{1.864840in}{1.504916in}}%
\pgfpathlineto{\pgfqpoint{1.878994in}{1.494027in}}%
\pgfpathlineto{\pgfqpoint{1.886922in}{1.492782in}}%
\pgfpathlineto{\pgfqpoint{1.893466in}{1.485448in}}%
\pgfpathlineto{\pgfqpoint{1.904599in}{1.480942in}}%
\pgfpathlineto{\pgfqpoint{1.910803in}{1.469752in}}%
\pgfpathlineto{\pgfqpoint{1.922805in}{1.460295in}}%
\pgfpathlineto{\pgfqpoint{1.933835in}{1.454564in}}%
\pgfpathlineto{\pgfqpoint{1.940045in}{1.444352in}}%
\pgfpathlineto{\pgfqpoint{1.942134in}{1.425554in}}%
\pgfpathlineto{\pgfqpoint{1.883600in}{1.423428in}}%
\pgfpathlineto{\pgfqpoint{1.833705in}{1.422385in}}%
\pgfpathlineto{\pgfqpoint{1.788246in}{1.421717in}}%
\pgfpathlineto{\pgfqpoint{1.740155in}{1.421790in}}%
\pgfpathlineto{\pgfqpoint{1.692728in}{1.422304in}}%
\pgfpathclose%
\pgfusepath{fill}%
\end{pgfscope}%
\begin{pgfscope}%
\pgfpathrectangle{\pgfqpoint{0.100000in}{0.100000in}}{\pgfqpoint{2.989028in}{1.913466in}}%
\pgfusepath{clip}%
\pgfsetbuttcap%
\pgfsetmiterjoin%
\definecolor{currentfill}{rgb}{0.883814,0.953403,0.599769}%
\pgfsetfillcolor{currentfill}%
\pgfsetlinewidth{0.000000pt}%
\definecolor{currentstroke}{rgb}{0.000000,0.000000,0.000000}%
\pgfsetstrokecolor{currentstroke}%
\pgfsetstrokeopacity{0.000000}%
\pgfsetdash{}{0pt}%
\pgfpathmoveto{\pgfqpoint{0.363713in}{1.549891in}}%
\pgfpathlineto{\pgfqpoint{0.359077in}{1.558421in}}%
\pgfpathlineto{\pgfqpoint{0.362037in}{1.580316in}}%
\pgfpathlineto{\pgfqpoint{0.367750in}{1.591758in}}%
\pgfpathlineto{\pgfqpoint{0.364894in}{1.607237in}}%
\pgfpathlineto{\pgfqpoint{0.370854in}{1.613747in}}%
\pgfpathlineto{\pgfqpoint{0.381727in}{1.631292in}}%
\pgfpathlineto{\pgfqpoint{0.390946in}{1.641883in}}%
\pgfpathlineto{\pgfqpoint{0.404422in}{1.664951in}}%
\pgfpathlineto{\pgfqpoint{0.414758in}{1.690079in}}%
\pgfpathlineto{\pgfqpoint{0.425730in}{1.713646in}}%
\pgfpathlineto{\pgfqpoint{0.427948in}{1.723489in}}%
\pgfpathlineto{\pgfqpoint{0.443085in}{1.751647in}}%
\pgfpathlineto{\pgfqpoint{0.445999in}{1.764014in}}%
\pgfpathlineto{\pgfqpoint{0.452444in}{1.777001in}}%
\pgfpathlineto{\pgfqpoint{0.454747in}{1.792843in}}%
\pgfpathlineto{\pgfqpoint{0.467051in}{1.800568in}}%
\pgfpathlineto{\pgfqpoint{0.481639in}{1.804461in}}%
\pgfpathlineto{\pgfqpoint{0.489015in}{1.795741in}}%
\pgfpathlineto{\pgfqpoint{0.495391in}{1.796386in}}%
\pgfpathlineto{\pgfqpoint{0.505347in}{1.785992in}}%
\pgfpathlineto{\pgfqpoint{0.506670in}{1.776202in}}%
\pgfpathlineto{\pgfqpoint{0.502895in}{1.757031in}}%
\pgfpathlineto{\pgfqpoint{0.515456in}{1.747249in}}%
\pgfpathlineto{\pgfqpoint{0.523603in}{1.743571in}}%
\pgfpathlineto{\pgfqpoint{0.545673in}{1.747325in}}%
\pgfpathlineto{\pgfqpoint{0.558460in}{1.744762in}}%
\pgfpathlineto{\pgfqpoint{0.567248in}{1.740554in}}%
\pgfpathlineto{\pgfqpoint{0.571877in}{1.732568in}}%
\pgfpathlineto{\pgfqpoint{0.601190in}{1.733634in}}%
\pgfpathlineto{\pgfqpoint{0.605851in}{1.728747in}}%
\pgfpathlineto{\pgfqpoint{0.616848in}{1.727692in}}%
\pgfpathlineto{\pgfqpoint{0.627921in}{1.730814in}}%
\pgfpathlineto{\pgfqpoint{0.660663in}{1.730028in}}%
\pgfpathlineto{\pgfqpoint{0.667734in}{1.727622in}}%
\pgfpathlineto{\pgfqpoint{0.675998in}{1.730394in}}%
\pgfpathlineto{\pgfqpoint{0.767475in}{1.706523in}}%
\pgfpathlineto{\pgfqpoint{0.771203in}{1.693698in}}%
\pgfpathlineto{\pgfqpoint{0.779597in}{1.686749in}}%
\pgfpathlineto{\pgfqpoint{0.780469in}{1.674240in}}%
\pgfpathlineto{\pgfqpoint{0.768191in}{1.661575in}}%
\pgfpathlineto{\pgfqpoint{0.762206in}{1.649603in}}%
\pgfpathlineto{\pgfqpoint{0.754997in}{1.643207in}}%
\pgfpathlineto{\pgfqpoint{0.754345in}{1.636806in}}%
\pgfpathlineto{\pgfqpoint{0.742402in}{1.628155in}}%
\pgfpathlineto{\pgfqpoint{0.734208in}{1.615650in}}%
\pgfpathlineto{\pgfqpoint{0.729620in}{1.612613in}}%
\pgfpathlineto{\pgfqpoint{0.727118in}{1.600478in}}%
\pgfpathlineto{\pgfqpoint{0.739170in}{1.588120in}}%
\pgfpathlineto{\pgfqpoint{0.727531in}{1.567559in}}%
\pgfpathlineto{\pgfqpoint{0.726119in}{1.559712in}}%
\pgfpathlineto{\pgfqpoint{0.699946in}{1.452341in}}%
\pgfpathlineto{\pgfqpoint{0.644893in}{1.466447in}}%
\pgfpathlineto{\pgfqpoint{0.591784in}{1.480141in}}%
\pgfpathlineto{\pgfqpoint{0.559744in}{1.489057in}}%
\pgfpathlineto{\pgfqpoint{0.518578in}{1.500784in}}%
\pgfpathlineto{\pgfqpoint{0.452913in}{1.521452in}}%
\pgfpathlineto{\pgfqpoint{0.381532in}{1.543676in}}%
\pgfpathlineto{\pgfqpoint{0.363713in}{1.549891in}}%
\pgfpathclose%
\pgfusepath{fill}%
\end{pgfscope}%
\begin{pgfscope}%
\pgfpathrectangle{\pgfqpoint{0.100000in}{0.100000in}}{\pgfqpoint{2.989028in}{1.913466in}}%
\pgfusepath{clip}%
\pgfsetbuttcap%
\pgfsetmiterjoin%
\definecolor{currentfill}{rgb}{0.955786,0.982314,0.680046}%
\pgfsetfillcolor{currentfill}%
\pgfsetlinewidth{0.000000pt}%
\definecolor{currentstroke}{rgb}{0.000000,0.000000,0.000000}%
\pgfsetstrokecolor{currentstroke}%
\pgfsetstrokeopacity{0.000000}%
\pgfsetdash{}{0pt}%
\pgfpathmoveto{\pgfqpoint{2.837703in}{1.505030in}}%
\pgfpathlineto{\pgfqpoint{2.832504in}{1.509256in}}%
\pgfpathlineto{\pgfqpoint{2.833228in}{1.522426in}}%
\pgfpathlineto{\pgfqpoint{2.827854in}{1.559858in}}%
\pgfpathlineto{\pgfqpoint{2.832066in}{1.573021in}}%
\pgfpathlineto{\pgfqpoint{2.835075in}{1.596366in}}%
\pgfpathlineto{\pgfqpoint{2.831015in}{1.607411in}}%
\pgfpathlineto{\pgfqpoint{2.845785in}{1.622115in}}%
\pgfpathlineto{\pgfqpoint{2.849502in}{1.633812in}}%
\pgfpathlineto{\pgfqpoint{2.842504in}{1.643043in}}%
\pgfpathlineto{\pgfqpoint{2.845485in}{1.654580in}}%
\pgfpathlineto{\pgfqpoint{2.843616in}{1.661826in}}%
\pgfpathlineto{\pgfqpoint{2.845221in}{1.677346in}}%
\pgfpathlineto{\pgfqpoint{2.848221in}{1.682137in}}%
\pgfpathlineto{\pgfqpoint{2.857471in}{1.685161in}}%
\pgfpathlineto{\pgfqpoint{2.870964in}{1.645645in}}%
\pgfpathlineto{\pgfqpoint{2.895998in}{1.563760in}}%
\pgfpathlineto{\pgfqpoint{2.905471in}{1.557411in}}%
\pgfpathlineto{\pgfqpoint{2.907175in}{1.550163in}}%
\pgfpathlineto{\pgfqpoint{2.912175in}{1.547235in}}%
\pgfpathlineto{\pgfqpoint{2.911795in}{1.534090in}}%
\pgfpathlineto{\pgfqpoint{2.906497in}{1.533879in}}%
\pgfpathlineto{\pgfqpoint{2.895764in}{1.525694in}}%
\pgfpathlineto{\pgfqpoint{2.892669in}{1.517482in}}%
\pgfpathlineto{\pgfqpoint{2.863942in}{1.510374in}}%
\pgfpathlineto{\pgfqpoint{2.837703in}{1.505030in}}%
\pgfpathclose%
\pgfusepath{fill}%
\end{pgfscope}%
\begin{pgfscope}%
\pgfpathrectangle{\pgfqpoint{0.100000in}{0.100000in}}{\pgfqpoint{2.989028in}{1.913466in}}%
\pgfusepath{clip}%
\pgfsetbuttcap%
\pgfsetmiterjoin%
\definecolor{currentfill}{rgb}{0.936563,0.974625,0.650058}%
\pgfsetfillcolor{currentfill}%
\pgfsetlinewidth{0.000000pt}%
\definecolor{currentstroke}{rgb}{0.000000,0.000000,0.000000}%
\pgfsetstrokecolor{currentstroke}%
\pgfsetstrokeopacity{0.000000}%
\pgfsetdash{}{0pt}%
\pgfpathmoveto{\pgfqpoint{1.939431in}{1.220143in}}%
\pgfpathlineto{\pgfqpoint{1.924266in}{1.235161in}}%
\pgfpathlineto{\pgfqpoint{1.875789in}{1.232365in}}%
\pgfpathlineto{\pgfqpoint{1.800184in}{1.229874in}}%
\pgfpathlineto{\pgfqpoint{1.724124in}{1.231060in}}%
\pgfpathlineto{\pgfqpoint{1.718785in}{1.240370in}}%
\pgfpathlineto{\pgfqpoint{1.720965in}{1.249511in}}%
\pgfpathlineto{\pgfqpoint{1.719930in}{1.265170in}}%
\pgfpathlineto{\pgfqpoint{1.715910in}{1.280338in}}%
\pgfpathlineto{\pgfqpoint{1.716378in}{1.288348in}}%
\pgfpathlineto{\pgfqpoint{1.707833in}{1.297398in}}%
\pgfpathlineto{\pgfqpoint{1.709692in}{1.309990in}}%
\pgfpathlineto{\pgfqpoint{1.696585in}{1.334864in}}%
\pgfpathlineto{\pgfqpoint{1.692680in}{1.355981in}}%
\pgfpathlineto{\pgfqpoint{1.683252in}{1.373157in}}%
\pgfpathlineto{\pgfqpoint{1.692201in}{1.394464in}}%
\pgfpathlineto{\pgfqpoint{1.687744in}{1.404379in}}%
\pgfpathlineto{\pgfqpoint{1.692728in}{1.422304in}}%
\pgfpathlineto{\pgfqpoint{1.740155in}{1.421790in}}%
\pgfpathlineto{\pgfqpoint{1.788246in}{1.421717in}}%
\pgfpathlineto{\pgfqpoint{1.833705in}{1.422385in}}%
\pgfpathlineto{\pgfqpoint{1.883600in}{1.423428in}}%
\pgfpathlineto{\pgfqpoint{1.942134in}{1.425554in}}%
\pgfpathlineto{\pgfqpoint{1.943697in}{1.417056in}}%
\pgfpathlineto{\pgfqpoint{1.950178in}{1.409931in}}%
\pgfpathlineto{\pgfqpoint{1.946075in}{1.402507in}}%
\pgfpathlineto{\pgfqpoint{1.947756in}{1.388016in}}%
\pgfpathlineto{\pgfqpoint{1.951772in}{1.376303in}}%
\pgfpathlineto{\pgfqpoint{1.969980in}{1.370103in}}%
\pgfpathlineto{\pgfqpoint{1.973100in}{1.361809in}}%
\pgfpathlineto{\pgfqpoint{1.983091in}{1.352497in}}%
\pgfpathlineto{\pgfqpoint{1.987167in}{1.342864in}}%
\pgfpathlineto{\pgfqpoint{1.997291in}{1.336402in}}%
\pgfpathlineto{\pgfqpoint{1.998875in}{1.328623in}}%
\pgfpathlineto{\pgfqpoint{1.996904in}{1.316845in}}%
\pgfpathlineto{\pgfqpoint{1.991751in}{1.313315in}}%
\pgfpathlineto{\pgfqpoint{1.990197in}{1.302104in}}%
\pgfpathlineto{\pgfqpoint{1.975389in}{1.293200in}}%
\pgfpathlineto{\pgfqpoint{1.956087in}{1.288322in}}%
\pgfpathlineto{\pgfqpoint{1.954318in}{1.277106in}}%
\pgfpathlineto{\pgfqpoint{1.961764in}{1.269093in}}%
\pgfpathlineto{\pgfqpoint{1.962068in}{1.259018in}}%
\pgfpathlineto{\pgfqpoint{1.956060in}{1.251098in}}%
\pgfpathlineto{\pgfqpoint{1.952907in}{1.239332in}}%
\pgfpathlineto{\pgfqpoint{1.942465in}{1.235438in}}%
\pgfpathlineto{\pgfqpoint{1.943130in}{1.222326in}}%
\pgfpathlineto{\pgfqpoint{1.939431in}{1.220143in}}%
\pgfpathclose%
\pgfusepath{fill}%
\end{pgfscope}%
\begin{pgfscope}%
\pgfpathrectangle{\pgfqpoint{0.100000in}{0.100000in}}{\pgfqpoint{2.989028in}{1.913466in}}%
\pgfusepath{clip}%
\pgfsetbuttcap%
\pgfsetmiterjoin%
\definecolor{currentfill}{rgb}{0.711419,0.883276,0.634833}%
\pgfsetfillcolor{currentfill}%
\pgfsetlinewidth{0.000000pt}%
\definecolor{currentstroke}{rgb}{0.000000,0.000000,0.000000}%
\pgfsetstrokecolor{currentstroke}%
\pgfsetstrokeopacity{0.000000}%
\pgfsetdash{}{0pt}%
\pgfpathmoveto{\pgfqpoint{2.880364in}{1.467139in}}%
\pgfpathlineto{\pgfqpoint{2.866000in}{1.464872in}}%
\pgfpathlineto{\pgfqpoint{2.800022in}{1.449880in}}%
\pgfpathlineto{\pgfqpoint{2.798869in}{1.451627in}}%
\pgfpathlineto{\pgfqpoint{2.799865in}{1.496989in}}%
\pgfpathlineto{\pgfqpoint{2.837703in}{1.505030in}}%
\pgfpathlineto{\pgfqpoint{2.863942in}{1.510374in}}%
\pgfpathlineto{\pgfqpoint{2.892669in}{1.517482in}}%
\pgfpathlineto{\pgfqpoint{2.895764in}{1.525694in}}%
\pgfpathlineto{\pgfqpoint{2.906497in}{1.533879in}}%
\pgfpathlineto{\pgfqpoint{2.911795in}{1.534090in}}%
\pgfpathlineto{\pgfqpoint{2.918759in}{1.522149in}}%
\pgfpathlineto{\pgfqpoint{2.912440in}{1.504717in}}%
\pgfpathlineto{\pgfqpoint{2.911512in}{1.494495in}}%
\pgfpathlineto{\pgfqpoint{2.924304in}{1.495455in}}%
\pgfpathlineto{\pgfqpoint{2.930105in}{1.490552in}}%
\pgfpathlineto{\pgfqpoint{2.943114in}{1.470509in}}%
\pgfpathlineto{\pgfqpoint{2.949579in}{1.468072in}}%
\pgfpathlineto{\pgfqpoint{2.960410in}{1.468976in}}%
\pgfpathlineto{\pgfqpoint{2.967946in}{1.475787in}}%
\pgfpathlineto{\pgfqpoint{2.972956in}{1.469708in}}%
\pgfpathlineto{\pgfqpoint{2.941451in}{1.453083in}}%
\pgfpathlineto{\pgfqpoint{2.940467in}{1.465013in}}%
\pgfpathlineto{\pgfqpoint{2.926155in}{1.446471in}}%
\pgfpathlineto{\pgfqpoint{2.921123in}{1.443277in}}%
\pgfpathlineto{\pgfqpoint{2.914105in}{1.453976in}}%
\pgfpathlineto{\pgfqpoint{2.912183in}{1.455442in}}%
\pgfpathlineto{\pgfqpoint{2.905650in}{1.458903in}}%
\pgfpathlineto{\pgfqpoint{2.899952in}{1.472940in}}%
\pgfpathlineto{\pgfqpoint{2.880364in}{1.467139in}}%
\pgfpathclose%
\pgfusepath{fill}%
\end{pgfscope}%
\begin{pgfscope}%
\pgfpathrectangle{\pgfqpoint{0.100000in}{0.100000in}}{\pgfqpoint{2.989028in}{1.913466in}}%
\pgfusepath{clip}%
\pgfsetbuttcap%
\pgfsetmiterjoin%
\definecolor{currentfill}{rgb}{0.256055,0.600231,0.713495}%
\pgfsetfillcolor{currentfill}%
\pgfsetlinewidth{0.000000pt}%
\definecolor{currentstroke}{rgb}{0.000000,0.000000,0.000000}%
\pgfsetstrokecolor{currentstroke}%
\pgfsetstrokeopacity{0.000000}%
\pgfsetdash{}{0pt}%
\pgfpathmoveto{\pgfqpoint{1.408935in}{1.205970in}}%
\pgfpathlineto{\pgfqpoint{1.414195in}{1.271196in}}%
\pgfpathlineto{\pgfqpoint{1.384404in}{1.273614in}}%
\pgfpathlineto{\pgfqpoint{1.315540in}{1.280295in}}%
\pgfpathlineto{\pgfqpoint{1.321947in}{1.345289in}}%
\pgfpathlineto{\pgfqpoint{1.328363in}{1.410780in}}%
\pgfpathlineto{\pgfqpoint{1.354836in}{1.407921in}}%
\pgfpathlineto{\pgfqpoint{1.422447in}{1.401533in}}%
\pgfpathlineto{\pgfqpoint{1.463365in}{1.398329in}}%
\pgfpathlineto{\pgfqpoint{1.512902in}{1.395489in}}%
\pgfpathlineto{\pgfqpoint{1.594439in}{1.391630in}}%
\pgfpathlineto{\pgfqpoint{1.596231in}{1.387975in}}%
\pgfpathlineto{\pgfqpoint{1.619594in}{1.376416in}}%
\pgfpathlineto{\pgfqpoint{1.626759in}{1.382286in}}%
\pgfpathlineto{\pgfqpoint{1.647400in}{1.381750in}}%
\pgfpathlineto{\pgfqpoint{1.659700in}{1.375307in}}%
\pgfpathlineto{\pgfqpoint{1.679205in}{1.367982in}}%
\pgfpathlineto{\pgfqpoint{1.684670in}{1.357295in}}%
\pgfpathlineto{\pgfqpoint{1.692680in}{1.355981in}}%
\pgfpathlineto{\pgfqpoint{1.696585in}{1.334864in}}%
\pgfpathlineto{\pgfqpoint{1.709692in}{1.309990in}}%
\pgfpathlineto{\pgfqpoint{1.707833in}{1.297398in}}%
\pgfpathlineto{\pgfqpoint{1.716378in}{1.288348in}}%
\pgfpathlineto{\pgfqpoint{1.715910in}{1.280338in}}%
\pgfpathlineto{\pgfqpoint{1.719930in}{1.265170in}}%
\pgfpathlineto{\pgfqpoint{1.720965in}{1.249511in}}%
\pgfpathlineto{\pgfqpoint{1.718785in}{1.240370in}}%
\pgfpathlineto{\pgfqpoint{1.724124in}{1.231060in}}%
\pgfpathlineto{\pgfqpoint{1.731447in}{1.214132in}}%
\pgfpathlineto{\pgfqpoint{1.738455in}{1.207242in}}%
\pgfpathlineto{\pgfqpoint{1.746810in}{1.192312in}}%
\pgfpathlineto{\pgfqpoint{1.723131in}{1.192066in}}%
\pgfpathlineto{\pgfqpoint{1.671933in}{1.192859in}}%
\pgfpathlineto{\pgfqpoint{1.615367in}{1.194592in}}%
\pgfpathlineto{\pgfqpoint{1.558468in}{1.196775in}}%
\pgfpathlineto{\pgfqpoint{1.474809in}{1.201345in}}%
\pgfpathlineto{\pgfqpoint{1.408935in}{1.205970in}}%
\pgfpathclose%
\pgfusepath{fill}%
\end{pgfscope}%
\begin{pgfscope}%
\pgfpathrectangle{\pgfqpoint{0.100000in}{0.100000in}}{\pgfqpoint{2.989028in}{1.913466in}}%
\pgfusepath{clip}%
\pgfsetbuttcap%
\pgfsetmiterjoin%
\definecolor{currentfill}{rgb}{0.975010,0.990004,0.710035}%
\pgfsetfillcolor{currentfill}%
\pgfsetlinewidth{0.000000pt}%
\definecolor{currentstroke}{rgb}{0.000000,0.000000,0.000000}%
\pgfsetstrokecolor{currentstroke}%
\pgfsetstrokeopacity{0.000000}%
\pgfsetdash{}{0pt}%
\pgfpathmoveto{\pgfqpoint{2.498213in}{1.402714in}}%
\pgfpathlineto{\pgfqpoint{2.524409in}{1.427692in}}%
\pgfpathlineto{\pgfqpoint{2.527665in}{1.436570in}}%
\pgfpathlineto{\pgfqpoint{2.535341in}{1.444171in}}%
\pgfpathlineto{\pgfqpoint{2.529581in}{1.455238in}}%
\pgfpathlineto{\pgfqpoint{2.522360in}{1.461711in}}%
\pgfpathlineto{\pgfqpoint{2.520273in}{1.473182in}}%
\pgfpathlineto{\pgfqpoint{2.547151in}{1.484959in}}%
\pgfpathlineto{\pgfqpoint{2.569405in}{1.488685in}}%
\pgfpathlineto{\pgfqpoint{2.581363in}{1.488934in}}%
\pgfpathlineto{\pgfqpoint{2.590473in}{1.484433in}}%
\pgfpathlineto{\pgfqpoint{2.599352in}{1.488453in}}%
\pgfpathlineto{\pgfqpoint{2.621010in}{1.492956in}}%
\pgfpathlineto{\pgfqpoint{2.628493in}{1.498769in}}%
\pgfpathlineto{\pgfqpoint{2.639589in}{1.511638in}}%
\pgfpathlineto{\pgfqpoint{2.649683in}{1.517325in}}%
\pgfpathlineto{\pgfqpoint{2.650395in}{1.522779in}}%
\pgfpathlineto{\pgfqpoint{2.645097in}{1.535222in}}%
\pgfpathlineto{\pgfqpoint{2.648927in}{1.542544in}}%
\pgfpathlineto{\pgfqpoint{2.643770in}{1.550418in}}%
\pgfpathlineto{\pgfqpoint{2.635873in}{1.550968in}}%
\pgfpathlineto{\pgfqpoint{2.655638in}{1.574751in}}%
\pgfpathlineto{\pgfqpoint{2.657985in}{1.583815in}}%
\pgfpathlineto{\pgfqpoint{2.673541in}{1.606931in}}%
\pgfpathlineto{\pgfqpoint{2.687981in}{1.619440in}}%
\pgfpathlineto{\pgfqpoint{2.697897in}{1.624664in}}%
\pgfpathlineto{\pgfqpoint{2.730342in}{1.632016in}}%
\pgfpathlineto{\pgfqpoint{2.760752in}{1.640318in}}%
\pgfpathlineto{\pgfqpoint{2.764780in}{1.627285in}}%
\pgfpathlineto{\pgfqpoint{2.765065in}{1.615202in}}%
\pgfpathlineto{\pgfqpoint{2.772002in}{1.604233in}}%
\pgfpathlineto{\pgfqpoint{2.774140in}{1.593064in}}%
\pgfpathlineto{\pgfqpoint{2.771474in}{1.577911in}}%
\pgfpathlineto{\pgfqpoint{2.779593in}{1.557473in}}%
\pgfpathlineto{\pgfqpoint{2.788848in}{1.546108in}}%
\pgfpathlineto{\pgfqpoint{2.796007in}{1.509167in}}%
\pgfpathlineto{\pgfqpoint{2.799865in}{1.496989in}}%
\pgfpathlineto{\pgfqpoint{2.798869in}{1.451627in}}%
\pgfpathlineto{\pgfqpoint{2.800022in}{1.449880in}}%
\pgfpathlineto{\pgfqpoint{2.808450in}{1.401092in}}%
\pgfpathlineto{\pgfqpoint{2.813176in}{1.396647in}}%
\pgfpathlineto{\pgfqpoint{2.803010in}{1.386769in}}%
\pgfpathlineto{\pgfqpoint{2.808026in}{1.381100in}}%
\pgfpathlineto{\pgfqpoint{2.803617in}{1.372526in}}%
\pgfpathlineto{\pgfqpoint{2.803654in}{1.368875in}}%
\pgfpathlineto{\pgfqpoint{2.798142in}{1.365581in}}%
\pgfpathlineto{\pgfqpoint{2.795458in}{1.358300in}}%
\pgfpathlineto{\pgfqpoint{2.796310in}{1.378322in}}%
\pgfpathlineto{\pgfqpoint{2.779225in}{1.382730in}}%
\pgfpathlineto{\pgfqpoint{2.752507in}{1.391844in}}%
\pgfpathlineto{\pgfqpoint{2.748728in}{1.396333in}}%
\pgfpathlineto{\pgfqpoint{2.737564in}{1.397408in}}%
\pgfpathlineto{\pgfqpoint{2.731142in}{1.404090in}}%
\pgfpathlineto{\pgfqpoint{2.727676in}{1.417397in}}%
\pgfpathlineto{\pgfqpoint{2.718572in}{1.419052in}}%
\pgfpathlineto{\pgfqpoint{2.712484in}{1.426033in}}%
\pgfpathlineto{\pgfqpoint{2.634972in}{1.410406in}}%
\pgfpathlineto{\pgfqpoint{2.597928in}{1.402745in}}%
\pgfpathlineto{\pgfqpoint{2.541683in}{1.392479in}}%
\pgfpathlineto{\pgfqpoint{2.501180in}{1.385646in}}%
\pgfpathlineto{\pgfqpoint{2.498213in}{1.402714in}}%
\pgfpathclose%
\pgfusepath{fill}%
\end{pgfscope}%
\begin{pgfscope}%
\pgfpathrectangle{\pgfqpoint{0.100000in}{0.100000in}}{\pgfqpoint{2.989028in}{1.913466in}}%
\pgfusepath{clip}%
\pgfsetbuttcap%
\pgfsetmiterjoin%
\definecolor{currentfill}{rgb}{0.975010,0.990004,0.710035}%
\pgfsetfillcolor{currentfill}%
\pgfsetlinewidth{0.000000pt}%
\definecolor{currentstroke}{rgb}{0.000000,0.000000,0.000000}%
\pgfsetstrokecolor{currentstroke}%
\pgfsetstrokeopacity{0.000000}%
\pgfsetdash{}{0pt}%
\pgfpathmoveto{\pgfqpoint{2.809390in}{1.354233in}}%
\pgfpathlineto{\pgfqpoint{2.797403in}{1.350498in}}%
\pgfpathlineto{\pgfqpoint{2.795386in}{1.353932in}}%
\pgfpathlineto{\pgfqpoint{2.799234in}{1.365456in}}%
\pgfpathlineto{\pgfqpoint{2.806360in}{1.366591in}}%
\pgfpathlineto{\pgfqpoint{2.805722in}{1.370219in}}%
\pgfpathlineto{\pgfqpoint{2.812120in}{1.375665in}}%
\pgfpathlineto{\pgfqpoint{2.830618in}{1.380004in}}%
\pgfpathlineto{\pgfqpoint{2.838853in}{1.386571in}}%
\pgfpathlineto{\pgfqpoint{2.857353in}{1.392045in}}%
\pgfpathlineto{\pgfqpoint{2.865790in}{1.390042in}}%
\pgfpathlineto{\pgfqpoint{2.872888in}{1.398866in}}%
\pgfpathlineto{\pgfqpoint{2.883615in}{1.400032in}}%
\pgfpathlineto{\pgfqpoint{2.865314in}{1.382820in}}%
\pgfpathlineto{\pgfqpoint{2.809390in}{1.354233in}}%
\pgfpathclose%
\pgfusepath{fill}%
\end{pgfscope}%
\begin{pgfscope}%
\pgfpathrectangle{\pgfqpoint{0.100000in}{0.100000in}}{\pgfqpoint{2.989028in}{1.913466in}}%
\pgfusepath{clip}%
\pgfsetbuttcap%
\pgfsetmiterjoin%
\definecolor{currentfill}{rgb}{0.793080,0.916494,0.618224}%
\pgfsetfillcolor{currentfill}%
\pgfsetlinewidth{0.000000pt}%
\definecolor{currentstroke}{rgb}{0.000000,0.000000,0.000000}%
\pgfsetstrokecolor{currentstroke}%
\pgfsetstrokeopacity{0.000000}%
\pgfsetdash{}{0pt}%
\pgfpathmoveto{\pgfqpoint{2.540072in}{1.240581in}}%
\pgfpathlineto{\pgfqpoint{2.488234in}{1.232010in}}%
\pgfpathlineto{\pgfqpoint{2.478831in}{1.291252in}}%
\pgfpathlineto{\pgfqpoint{2.464869in}{1.378577in}}%
\pgfpathlineto{\pgfqpoint{2.498213in}{1.402714in}}%
\pgfpathlineto{\pgfqpoint{2.501180in}{1.385646in}}%
\pgfpathlineto{\pgfqpoint{2.541683in}{1.392479in}}%
\pgfpathlineto{\pgfqpoint{2.597928in}{1.402745in}}%
\pgfpathlineto{\pgfqpoint{2.634972in}{1.410406in}}%
\pgfpathlineto{\pgfqpoint{2.712484in}{1.426033in}}%
\pgfpathlineto{\pgfqpoint{2.718572in}{1.419052in}}%
\pgfpathlineto{\pgfqpoint{2.727676in}{1.417397in}}%
\pgfpathlineto{\pgfqpoint{2.731142in}{1.404090in}}%
\pgfpathlineto{\pgfqpoint{2.737564in}{1.397408in}}%
\pgfpathlineto{\pgfqpoint{2.748728in}{1.396333in}}%
\pgfpathlineto{\pgfqpoint{2.752507in}{1.391844in}}%
\pgfpathlineto{\pgfqpoint{2.748671in}{1.388380in}}%
\pgfpathlineto{\pgfqpoint{2.745215in}{1.376126in}}%
\pgfpathlineto{\pgfqpoint{2.737116in}{1.362357in}}%
\pgfpathlineto{\pgfqpoint{2.742552in}{1.356369in}}%
\pgfpathlineto{\pgfqpoint{2.738104in}{1.349953in}}%
\pgfpathlineto{\pgfqpoint{2.740613in}{1.335884in}}%
\pgfpathlineto{\pgfqpoint{2.748625in}{1.328499in}}%
\pgfpathlineto{\pgfqpoint{2.767634in}{1.316500in}}%
\pgfpathlineto{\pgfqpoint{2.752326in}{1.299570in}}%
\pgfpathlineto{\pgfqpoint{2.752111in}{1.293145in}}%
\pgfpathlineto{\pgfqpoint{2.739648in}{1.284858in}}%
\pgfpathlineto{\pgfqpoint{2.725867in}{1.283302in}}%
\pgfpathlineto{\pgfqpoint{2.722458in}{1.276100in}}%
\pgfpathlineto{\pgfqpoint{2.684127in}{1.267806in}}%
\pgfpathlineto{\pgfqpoint{2.639399in}{1.258797in}}%
\pgfpathlineto{\pgfqpoint{2.608655in}{1.253292in}}%
\pgfpathlineto{\pgfqpoint{2.540072in}{1.240581in}}%
\pgfpathclose%
\pgfusepath{fill}%
\end{pgfscope}%
\begin{pgfscope}%
\pgfpathrectangle{\pgfqpoint{0.100000in}{0.100000in}}{\pgfqpoint{2.989028in}{1.913466in}}%
\pgfusepath{clip}%
\pgfsetbuttcap%
\pgfsetmiterjoin%
\definecolor{currentfill}{rgb}{0.951942,0.980777,0.674048}%
\pgfsetfillcolor{currentfill}%
\pgfsetlinewidth{0.000000pt}%
\definecolor{currentstroke}{rgb}{0.000000,0.000000,0.000000}%
\pgfsetstrokecolor{currentstroke}%
\pgfsetstrokeopacity{0.000000}%
\pgfsetdash{}{0pt}%
\pgfpathmoveto{\pgfqpoint{2.800022in}{1.449880in}}%
\pgfpathlineto{\pgfqpoint{2.866000in}{1.464872in}}%
\pgfpathlineto{\pgfqpoint{2.880364in}{1.467139in}}%
\pgfpathlineto{\pgfqpoint{2.889871in}{1.429760in}}%
\pgfpathlineto{\pgfqpoint{2.888366in}{1.423067in}}%
\pgfpathlineto{\pgfqpoint{2.857827in}{1.411244in}}%
\pgfpathlineto{\pgfqpoint{2.839596in}{1.407111in}}%
\pgfpathlineto{\pgfqpoint{2.831849in}{1.397842in}}%
\pgfpathlineto{\pgfqpoint{2.808026in}{1.381100in}}%
\pgfpathlineto{\pgfqpoint{2.803010in}{1.386769in}}%
\pgfpathlineto{\pgfqpoint{2.813176in}{1.396647in}}%
\pgfpathlineto{\pgfqpoint{2.808450in}{1.401092in}}%
\pgfpathlineto{\pgfqpoint{2.800022in}{1.449880in}}%
\pgfpathclose%
\pgfusepath{fill}%
\end{pgfscope}%
\begin{pgfscope}%
\pgfpathrectangle{\pgfqpoint{0.100000in}{0.100000in}}{\pgfqpoint{2.989028in}{1.913466in}}%
\pgfusepath{clip}%
\pgfsetbuttcap%
\pgfsetmiterjoin%
\definecolor{currentfill}{rgb}{0.959631,0.983852,0.686044}%
\pgfsetfillcolor{currentfill}%
\pgfsetlinewidth{0.000000pt}%
\definecolor{currentstroke}{rgb}{0.000000,0.000000,0.000000}%
\pgfsetstrokecolor{currentstroke}%
\pgfsetstrokeopacity{0.000000}%
\pgfsetdash{}{0pt}%
\pgfpathmoveto{\pgfqpoint{2.880364in}{1.467139in}}%
\pgfpathlineto{\pgfqpoint{2.899952in}{1.472940in}}%
\pgfpathlineto{\pgfqpoint{2.905650in}{1.458903in}}%
\pgfpathlineto{\pgfqpoint{2.912183in}{1.455442in}}%
\pgfpathlineto{\pgfqpoint{2.904121in}{1.449529in}}%
\pgfpathlineto{\pgfqpoint{2.905138in}{1.432164in}}%
\pgfpathlineto{\pgfqpoint{2.888366in}{1.423067in}}%
\pgfpathlineto{\pgfqpoint{2.889871in}{1.429760in}}%
\pgfpathlineto{\pgfqpoint{2.880364in}{1.467139in}}%
\pgfpathclose%
\pgfusepath{fill}%
\end{pgfscope}%
\begin{pgfscope}%
\pgfpathrectangle{\pgfqpoint{0.100000in}{0.100000in}}{\pgfqpoint{2.989028in}{1.913466in}}%
\pgfusepath{clip}%
\pgfsetbuttcap%
\pgfsetmiterjoin%
\definecolor{currentfill}{rgb}{0.793080,0.916494,0.618224}%
\pgfsetfillcolor{currentfill}%
\pgfsetlinewidth{0.000000pt}%
\definecolor{currentstroke}{rgb}{0.000000,0.000000,0.000000}%
\pgfsetstrokecolor{currentstroke}%
\pgfsetstrokeopacity{0.000000}%
\pgfsetdash{}{0pt}%
\pgfpathmoveto{\pgfqpoint{2.737440in}{1.278737in}}%
\pgfpathlineto{\pgfqpoint{2.739648in}{1.284858in}}%
\pgfpathlineto{\pgfqpoint{2.752111in}{1.293145in}}%
\pgfpathlineto{\pgfqpoint{2.752326in}{1.299570in}}%
\pgfpathlineto{\pgfqpoint{2.767634in}{1.316500in}}%
\pgfpathlineto{\pgfqpoint{2.748625in}{1.328499in}}%
\pgfpathlineto{\pgfqpoint{2.740613in}{1.335884in}}%
\pgfpathlineto{\pgfqpoint{2.738104in}{1.349953in}}%
\pgfpathlineto{\pgfqpoint{2.742552in}{1.356369in}}%
\pgfpathlineto{\pgfqpoint{2.737116in}{1.362357in}}%
\pgfpathlineto{\pgfqpoint{2.745215in}{1.376126in}}%
\pgfpathlineto{\pgfqpoint{2.748671in}{1.388380in}}%
\pgfpathlineto{\pgfqpoint{2.752507in}{1.391844in}}%
\pgfpathlineto{\pgfqpoint{2.779225in}{1.382730in}}%
\pgfpathlineto{\pgfqpoint{2.796310in}{1.378322in}}%
\pgfpathlineto{\pgfqpoint{2.795458in}{1.358300in}}%
\pgfpathlineto{\pgfqpoint{2.785080in}{1.343116in}}%
\pgfpathlineto{\pgfqpoint{2.793632in}{1.340882in}}%
\pgfpathlineto{\pgfqpoint{2.802511in}{1.334366in}}%
\pgfpathlineto{\pgfqpoint{2.803034in}{1.316545in}}%
\pgfpathlineto{\pgfqpoint{2.800347in}{1.303928in}}%
\pgfpathlineto{\pgfqpoint{2.802135in}{1.293566in}}%
\pgfpathlineto{\pgfqpoint{2.786701in}{1.258597in}}%
\pgfpathlineto{\pgfqpoint{2.781154in}{1.242268in}}%
\pgfpathlineto{\pgfqpoint{2.773409in}{1.250205in}}%
\pgfpathlineto{\pgfqpoint{2.763139in}{1.248853in}}%
\pgfpathlineto{\pgfqpoint{2.737435in}{1.263703in}}%
\pgfpathlineto{\pgfqpoint{2.734805in}{1.271656in}}%
\pgfpathlineto{\pgfqpoint{2.737440in}{1.278737in}}%
\pgfpathclose%
\pgfusepath{fill}%
\end{pgfscope}%
\begin{pgfscope}%
\pgfpathrectangle{\pgfqpoint{0.100000in}{0.100000in}}{\pgfqpoint{2.989028in}{1.913466in}}%
\pgfusepath{clip}%
\pgfsetbuttcap%
\pgfsetmiterjoin%
\definecolor{currentfill}{rgb}{0.913495,0.965398,0.614072}%
\pgfsetfillcolor{currentfill}%
\pgfsetlinewidth{0.000000pt}%
\definecolor{currentstroke}{rgb}{0.000000,0.000000,0.000000}%
\pgfsetstrokecolor{currentstroke}%
\pgfsetstrokeopacity{0.000000}%
\pgfsetdash{}{0pt}%
\pgfpathmoveto{\pgfqpoint{2.122290in}{1.062454in}}%
\pgfpathlineto{\pgfqpoint{2.121088in}{1.078676in}}%
\pgfpathlineto{\pgfqpoint{2.125759in}{1.086724in}}%
\pgfpathlineto{\pgfqpoint{2.122681in}{1.090668in}}%
\pgfpathlineto{\pgfqpoint{2.133907in}{1.103580in}}%
\pgfpathlineto{\pgfqpoint{2.133263in}{1.106692in}}%
\pgfpathlineto{\pgfqpoint{2.144170in}{1.129219in}}%
\pgfpathlineto{\pgfqpoint{2.141966in}{1.140083in}}%
\pgfpathlineto{\pgfqpoint{2.134057in}{1.151457in}}%
\pgfpathlineto{\pgfqpoint{2.139547in}{1.165295in}}%
\pgfpathlineto{\pgfqpoint{2.132431in}{1.256358in}}%
\pgfpathlineto{\pgfqpoint{2.127306in}{1.320243in}}%
\pgfpathlineto{\pgfqpoint{2.134385in}{1.314949in}}%
\pgfpathlineto{\pgfqpoint{2.142283in}{1.315093in}}%
\pgfpathlineto{\pgfqpoint{2.160936in}{1.325898in}}%
\pgfpathlineto{\pgfqpoint{2.218150in}{1.331236in}}%
\pgfpathlineto{\pgfqpoint{2.260498in}{1.335699in}}%
\pgfpathlineto{\pgfqpoint{2.260872in}{1.331556in}}%
\pgfpathlineto{\pgfqpoint{2.270539in}{1.244011in}}%
\pgfpathlineto{\pgfqpoint{2.278860in}{1.162554in}}%
\pgfpathlineto{\pgfqpoint{2.275273in}{1.158730in}}%
\pgfpathlineto{\pgfqpoint{2.280783in}{1.149238in}}%
\pgfpathlineto{\pgfqpoint{2.280755in}{1.142397in}}%
\pgfpathlineto{\pgfqpoint{2.272893in}{1.140679in}}%
\pgfpathlineto{\pgfqpoint{2.264096in}{1.134084in}}%
\pgfpathlineto{\pgfqpoint{2.258139in}{1.136680in}}%
\pgfpathlineto{\pgfqpoint{2.249225in}{1.132459in}}%
\pgfpathlineto{\pgfqpoint{2.251983in}{1.123981in}}%
\pgfpathlineto{\pgfqpoint{2.242820in}{1.115463in}}%
\pgfpathlineto{\pgfqpoint{2.240290in}{1.105610in}}%
\pgfpathlineto{\pgfqpoint{2.233989in}{1.103993in}}%
\pgfpathlineto{\pgfqpoint{2.229287in}{1.096531in}}%
\pgfpathlineto{\pgfqpoint{2.229903in}{1.089016in}}%
\pgfpathlineto{\pgfqpoint{2.224371in}{1.083732in}}%
\pgfpathlineto{\pgfqpoint{2.216048in}{1.084548in}}%
\pgfpathlineto{\pgfqpoint{2.209721in}{1.092656in}}%
\pgfpathlineto{\pgfqpoint{2.199000in}{1.084848in}}%
\pgfpathlineto{\pgfqpoint{2.199753in}{1.078025in}}%
\pgfpathlineto{\pgfqpoint{2.187829in}{1.074039in}}%
\pgfpathlineto{\pgfqpoint{2.184834in}{1.079056in}}%
\pgfpathlineto{\pgfqpoint{2.173226in}{1.073367in}}%
\pgfpathlineto{\pgfqpoint{2.168985in}{1.065222in}}%
\pgfpathlineto{\pgfqpoint{2.155006in}{1.073574in}}%
\pgfpathlineto{\pgfqpoint{2.132850in}{1.068041in}}%
\pgfpathlineto{\pgfqpoint{2.122290in}{1.062454in}}%
\pgfpathclose%
\pgfusepath{fill}%
\end{pgfscope}%
\begin{pgfscope}%
\pgfpathrectangle{\pgfqpoint{0.100000in}{0.100000in}}{\pgfqpoint{2.989028in}{1.913466in}}%
\pgfusepath{clip}%
\pgfsetbuttcap%
\pgfsetmiterjoin%
\definecolor{currentfill}{rgb}{0.865667,0.946021,0.603460}%
\pgfsetfillcolor{currentfill}%
\pgfsetlinewidth{0.000000pt}%
\definecolor{currentstroke}{rgb}{0.000000,0.000000,0.000000}%
\pgfsetstrokecolor{currentstroke}%
\pgfsetstrokeopacity{0.000000}%
\pgfsetdash{}{0pt}%
\pgfpathmoveto{\pgfqpoint{0.559744in}{1.489057in}}%
\pgfpathlineto{\pgfqpoint{0.591784in}{1.480141in}}%
\pgfpathlineto{\pgfqpoint{0.644893in}{1.466447in}}%
\pgfpathlineto{\pgfqpoint{0.699946in}{1.452341in}}%
\pgfpathlineto{\pgfqpoint{0.750689in}{1.440103in}}%
\pgfpathlineto{\pgfqpoint{0.794534in}{1.430177in}}%
\pgfpathlineto{\pgfqpoint{0.841442in}{1.419862in}}%
\pgfpathlineto{\pgfqpoint{0.827889in}{1.355901in}}%
\pgfpathlineto{\pgfqpoint{0.815815in}{1.299101in}}%
\pgfpathlineto{\pgfqpoint{0.796008in}{1.207454in}}%
\pgfpathlineto{\pgfqpoint{0.781137in}{1.138280in}}%
\pgfpathlineto{\pgfqpoint{0.773103in}{1.099678in}}%
\pgfpathlineto{\pgfqpoint{0.762801in}{1.049576in}}%
\pgfpathlineto{\pgfqpoint{0.755674in}{1.039412in}}%
\pgfpathlineto{\pgfqpoint{0.749938in}{1.039072in}}%
\pgfpathlineto{\pgfqpoint{0.745846in}{1.047945in}}%
\pgfpathlineto{\pgfqpoint{0.736451in}{1.051167in}}%
\pgfpathlineto{\pgfqpoint{0.726328in}{1.050037in}}%
\pgfpathlineto{\pgfqpoint{0.723454in}{1.042780in}}%
\pgfpathlineto{\pgfqpoint{0.723155in}{1.017578in}}%
\pgfpathlineto{\pgfqpoint{0.720123in}{1.011825in}}%
\pgfpathlineto{\pgfqpoint{0.721856in}{0.991592in}}%
\pgfpathlineto{\pgfqpoint{0.715525in}{0.978113in}}%
\pgfpathlineto{\pgfqpoint{0.664188in}{1.057164in}}%
\pgfpathlineto{\pgfqpoint{0.613630in}{1.133980in}}%
\pgfpathlineto{\pgfqpoint{0.587312in}{1.174260in}}%
\pgfpathlineto{\pgfqpoint{0.565358in}{1.209028in}}%
\pgfpathlineto{\pgfqpoint{0.537695in}{1.252036in}}%
\pgfpathlineto{\pgfqpoint{0.506381in}{1.300279in}}%
\pgfpathlineto{\pgfqpoint{0.519255in}{1.346069in}}%
\pgfpathlineto{\pgfqpoint{0.545163in}{1.437904in}}%
\pgfpathlineto{\pgfqpoint{0.559744in}{1.489057in}}%
\pgfpathclose%
\pgfusepath{fill}%
\end{pgfscope}%
\begin{pgfscope}%
\pgfpathrectangle{\pgfqpoint{0.100000in}{0.100000in}}{\pgfqpoint{2.989028in}{1.913466in}}%
\pgfusepath{clip}%
\pgfsetbuttcap%
\pgfsetmiterjoin%
\definecolor{currentfill}{rgb}{0.774933,0.909112,0.621915}%
\pgfsetfillcolor{currentfill}%
\pgfsetlinewidth{0.000000pt}%
\definecolor{currentstroke}{rgb}{0.000000,0.000000,0.000000}%
\pgfsetstrokecolor{currentstroke}%
\pgfsetstrokeopacity{0.000000}%
\pgfsetdash{}{0pt}%
\pgfpathmoveto{\pgfqpoint{0.841442in}{1.419862in}}%
\pgfpathlineto{\pgfqpoint{0.891592in}{1.409886in}}%
\pgfpathlineto{\pgfqpoint{0.984436in}{1.391963in}}%
\pgfpathlineto{\pgfqpoint{0.972815in}{1.327496in}}%
\pgfpathlineto{\pgfqpoint{1.023928in}{1.318744in}}%
\pgfpathlineto{\pgfqpoint{1.070495in}{1.311293in}}%
\pgfpathlineto{\pgfqpoint{1.062404in}{1.260326in}}%
\pgfpathlineto{\pgfqpoint{1.053829in}{1.205365in}}%
\pgfpathlineto{\pgfqpoint{1.042350in}{1.133200in}}%
\pgfpathlineto{\pgfqpoint{1.042053in}{1.127151in}}%
\pgfpathlineto{\pgfqpoint{1.030137in}{1.052362in}}%
\pgfpathlineto{\pgfqpoint{0.981086in}{1.059964in}}%
\pgfpathlineto{\pgfqpoint{0.956091in}{1.064984in}}%
\pgfpathlineto{\pgfqpoint{0.865746in}{1.080865in}}%
\pgfpathlineto{\pgfqpoint{0.831724in}{1.087575in}}%
\pgfpathlineto{\pgfqpoint{0.773103in}{1.099678in}}%
\pgfpathlineto{\pgfqpoint{0.781137in}{1.138280in}}%
\pgfpathlineto{\pgfqpoint{0.796008in}{1.207454in}}%
\pgfpathlineto{\pgfqpoint{0.815815in}{1.299101in}}%
\pgfpathlineto{\pgfqpoint{0.827889in}{1.355901in}}%
\pgfpathlineto{\pgfqpoint{0.841442in}{1.419862in}}%
\pgfpathclose%
\pgfusepath{fill}%
\end{pgfscope}%
\begin{pgfscope}%
\pgfpathrectangle{\pgfqpoint{0.100000in}{0.100000in}}{\pgfqpoint{2.989028in}{1.913466in}}%
\pgfusepath{clip}%
\pgfsetbuttcap%
\pgfsetmiterjoin%
\definecolor{currentfill}{rgb}{0.729566,0.890657,0.631142}%
\pgfsetfillcolor{currentfill}%
\pgfsetlinewidth{0.000000pt}%
\definecolor{currentstroke}{rgb}{0.000000,0.000000,0.000000}%
\pgfsetstrokecolor{currentstroke}%
\pgfsetstrokeopacity{0.000000}%
\pgfsetdash{}{0pt}%
\pgfpathmoveto{\pgfqpoint{0.363713in}{1.549891in}}%
\pgfpathlineto{\pgfqpoint{0.381532in}{1.543676in}}%
\pgfpathlineto{\pgfqpoint{0.452913in}{1.521452in}}%
\pgfpathlineto{\pgfqpoint{0.518578in}{1.500784in}}%
\pgfpathlineto{\pgfqpoint{0.559744in}{1.489057in}}%
\pgfpathlineto{\pgfqpoint{0.545163in}{1.437904in}}%
\pgfpathlineto{\pgfqpoint{0.519255in}{1.346069in}}%
\pgfpathlineto{\pgfqpoint{0.506381in}{1.300279in}}%
\pgfpathlineto{\pgfqpoint{0.537695in}{1.252036in}}%
\pgfpathlineto{\pgfqpoint{0.565358in}{1.209028in}}%
\pgfpathlineto{\pgfqpoint{0.587312in}{1.174260in}}%
\pgfpathlineto{\pgfqpoint{0.613630in}{1.133980in}}%
\pgfpathlineto{\pgfqpoint{0.664188in}{1.057164in}}%
\pgfpathlineto{\pgfqpoint{0.715525in}{0.978113in}}%
\pgfpathlineto{\pgfqpoint{0.713464in}{0.970274in}}%
\pgfpathlineto{\pgfqpoint{0.719656in}{0.957789in}}%
\pgfpathlineto{\pgfqpoint{0.720881in}{0.940711in}}%
\pgfpathlineto{\pgfqpoint{0.729874in}{0.932426in}}%
\pgfpathlineto{\pgfqpoint{0.730230in}{0.925750in}}%
\pgfpathlineto{\pgfqpoint{0.714127in}{0.918174in}}%
\pgfpathlineto{\pgfqpoint{0.706455in}{0.910589in}}%
\pgfpathlineto{\pgfqpoint{0.704063in}{0.893839in}}%
\pgfpathlineto{\pgfqpoint{0.700180in}{0.884704in}}%
\pgfpathlineto{\pgfqpoint{0.691998in}{0.877004in}}%
\pgfpathlineto{\pgfqpoint{0.683861in}{0.856982in}}%
\pgfpathlineto{\pgfqpoint{0.695303in}{0.846552in}}%
\pgfpathlineto{\pgfqpoint{0.693826in}{0.837954in}}%
\pgfpathlineto{\pgfqpoint{0.684444in}{0.831972in}}%
\pgfpathlineto{\pgfqpoint{0.677975in}{0.833027in}}%
\pgfpathlineto{\pgfqpoint{0.601622in}{0.843623in}}%
\pgfpathlineto{\pgfqpoint{0.545229in}{0.851623in}}%
\pgfpathlineto{\pgfqpoint{0.547695in}{0.860728in}}%
\pgfpathlineto{\pgfqpoint{0.544024in}{0.875844in}}%
\pgfpathlineto{\pgfqpoint{0.543653in}{0.891092in}}%
\pgfpathlineto{\pgfqpoint{0.541254in}{0.900014in}}%
\pgfpathlineto{\pgfqpoint{0.533864in}{0.912772in}}%
\pgfpathlineto{\pgfqpoint{0.512605in}{0.942205in}}%
\pgfpathlineto{\pgfqpoint{0.505643in}{0.945811in}}%
\pgfpathlineto{\pgfqpoint{0.496672in}{0.945755in}}%
\pgfpathlineto{\pgfqpoint{0.498721in}{0.955040in}}%
\pgfpathlineto{\pgfqpoint{0.494473in}{0.966651in}}%
\pgfpathlineto{\pgfqpoint{0.473608in}{0.972422in}}%
\pgfpathlineto{\pgfqpoint{0.460889in}{0.983108in}}%
\pgfpathlineto{\pgfqpoint{0.459834in}{0.989647in}}%
\pgfpathlineto{\pgfqpoint{0.445189in}{1.005835in}}%
\pgfpathlineto{\pgfqpoint{0.431249in}{1.008932in}}%
\pgfpathlineto{\pgfqpoint{0.418332in}{1.017159in}}%
\pgfpathlineto{\pgfqpoint{0.401344in}{1.020005in}}%
\pgfpathlineto{\pgfqpoint{0.394086in}{1.030977in}}%
\pgfpathlineto{\pgfqpoint{0.400979in}{1.048345in}}%
\pgfpathlineto{\pgfqpoint{0.398871in}{1.052250in}}%
\pgfpathlineto{\pgfqpoint{0.404611in}{1.066728in}}%
\pgfpathlineto{\pgfqpoint{0.394400in}{1.074432in}}%
\pgfpathlineto{\pgfqpoint{0.397713in}{1.088401in}}%
\pgfpathlineto{\pgfqpoint{0.392259in}{1.091971in}}%
\pgfpathlineto{\pgfqpoint{0.387550in}{1.105174in}}%
\pgfpathlineto{\pgfqpoint{0.381867in}{1.109200in}}%
\pgfpathlineto{\pgfqpoint{0.381440in}{1.118736in}}%
\pgfpathlineto{\pgfqpoint{0.376988in}{1.125461in}}%
\pgfpathlineto{\pgfqpoint{0.370279in}{1.148109in}}%
\pgfpathlineto{\pgfqpoint{0.362938in}{1.158956in}}%
\pgfpathlineto{\pgfqpoint{0.364533in}{1.177370in}}%
\pgfpathlineto{\pgfqpoint{0.373166in}{1.179223in}}%
\pgfpathlineto{\pgfqpoint{0.378787in}{1.189211in}}%
\pgfpathlineto{\pgfqpoint{0.375390in}{1.200043in}}%
\pgfpathlineto{\pgfqpoint{0.366209in}{1.201851in}}%
\pgfpathlineto{\pgfqpoint{0.361629in}{1.206918in}}%
\pgfpathlineto{\pgfqpoint{0.354211in}{1.225555in}}%
\pgfpathlineto{\pgfqpoint{0.357678in}{1.232248in}}%
\pgfpathlineto{\pgfqpoint{0.355219in}{1.244711in}}%
\pgfpathlineto{\pgfqpoint{0.360664in}{1.260852in}}%
\pgfpathlineto{\pgfqpoint{0.367043in}{1.254907in}}%
\pgfpathlineto{\pgfqpoint{0.364148in}{1.247890in}}%
\pgfpathlineto{\pgfqpoint{0.375202in}{1.236766in}}%
\pgfpathlineto{\pgfqpoint{0.374497in}{1.253285in}}%
\pgfpathlineto{\pgfqpoint{0.369760in}{1.257715in}}%
\pgfpathlineto{\pgfqpoint{0.375174in}{1.272240in}}%
\pgfpathlineto{\pgfqpoint{0.395652in}{1.269927in}}%
\pgfpathlineto{\pgfqpoint{0.392896in}{1.275328in}}%
\pgfpathlineto{\pgfqpoint{0.379381in}{1.274797in}}%
\pgfpathlineto{\pgfqpoint{0.372951in}{1.282977in}}%
\pgfpathlineto{\pgfqpoint{0.364858in}{1.275713in}}%
\pgfpathlineto{\pgfqpoint{0.360564in}{1.263570in}}%
\pgfpathlineto{\pgfqpoint{0.349102in}{1.279909in}}%
\pgfpathlineto{\pgfqpoint{0.344687in}{1.282884in}}%
\pgfpathlineto{\pgfqpoint{0.346369in}{1.300667in}}%
\pgfpathlineto{\pgfqpoint{0.342875in}{1.311153in}}%
\pgfpathlineto{\pgfqpoint{0.336542in}{1.321006in}}%
\pgfpathlineto{\pgfqpoint{0.323573in}{1.351193in}}%
\pgfpathlineto{\pgfqpoint{0.327813in}{1.357869in}}%
\pgfpathlineto{\pgfqpoint{0.327764in}{1.378965in}}%
\pgfpathlineto{\pgfqpoint{0.334763in}{1.390732in}}%
\pgfpathlineto{\pgfqpoint{0.336368in}{1.409121in}}%
\pgfpathlineto{\pgfqpoint{0.329768in}{1.430191in}}%
\pgfpathlineto{\pgfqpoint{0.321012in}{1.443602in}}%
\pgfpathlineto{\pgfqpoint{0.322574in}{1.455704in}}%
\pgfpathlineto{\pgfqpoint{0.347163in}{1.484994in}}%
\pgfpathlineto{\pgfqpoint{0.348384in}{1.494985in}}%
\pgfpathlineto{\pgfqpoint{0.359464in}{1.514044in}}%
\pgfpathlineto{\pgfqpoint{0.361001in}{1.532109in}}%
\pgfpathlineto{\pgfqpoint{0.357439in}{1.536719in}}%
\pgfpathlineto{\pgfqpoint{0.363713in}{1.549891in}}%
\pgfpathclose%
\pgfusepath{fill}%
\end{pgfscope}%
\begin{pgfscope}%
\pgfpathrectangle{\pgfqpoint{0.100000in}{0.100000in}}{\pgfqpoint{2.989028in}{1.913466in}}%
\pgfusepath{clip}%
\pgfsetbuttcap%
\pgfsetmiterjoin%
\definecolor{currentfill}{rgb}{0.865667,0.946021,0.603460}%
\pgfsetfillcolor{currentfill}%
\pgfsetlinewidth{0.000000pt}%
\definecolor{currentstroke}{rgb}{0.000000,0.000000,0.000000}%
\pgfsetstrokecolor{currentstroke}%
\pgfsetstrokeopacity{0.000000}%
\pgfsetdash{}{0pt}%
\pgfpathmoveto{\pgfqpoint{2.278860in}{1.162554in}}%
\pgfpathlineto{\pgfqpoint{2.270539in}{1.244011in}}%
\pgfpathlineto{\pgfqpoint{2.260872in}{1.331556in}}%
\pgfpathlineto{\pgfqpoint{2.324180in}{1.340984in}}%
\pgfpathlineto{\pgfqpoint{2.340982in}{1.336611in}}%
\pgfpathlineto{\pgfqpoint{2.349041in}{1.331864in}}%
\pgfpathlineto{\pgfqpoint{2.359139in}{1.333180in}}%
\pgfpathlineto{\pgfqpoint{2.372453in}{1.325321in}}%
\pgfpathlineto{\pgfqpoint{2.397252in}{1.337018in}}%
\pgfpathlineto{\pgfqpoint{2.410938in}{1.337409in}}%
\pgfpathlineto{\pgfqpoint{2.426920in}{1.355239in}}%
\pgfpathlineto{\pgfqpoint{2.443180in}{1.366084in}}%
\pgfpathlineto{\pgfqpoint{2.464869in}{1.378577in}}%
\pgfpathlineto{\pgfqpoint{2.478831in}{1.291252in}}%
\pgfpathlineto{\pgfqpoint{2.472344in}{1.285643in}}%
\pgfpathlineto{\pgfqpoint{2.476532in}{1.280492in}}%
\pgfpathlineto{\pgfqpoint{2.478201in}{1.269199in}}%
\pgfpathlineto{\pgfqpoint{2.474951in}{1.258593in}}%
\pgfpathlineto{\pgfqpoint{2.474521in}{1.243089in}}%
\pgfpathlineto{\pgfqpoint{2.471469in}{1.222911in}}%
\pgfpathlineto{\pgfqpoint{2.456028in}{1.204915in}}%
\pgfpathlineto{\pgfqpoint{2.449528in}{1.201094in}}%
\pgfpathlineto{\pgfqpoint{2.444473in}{1.204383in}}%
\pgfpathlineto{\pgfqpoint{2.431963in}{1.187230in}}%
\pgfpathlineto{\pgfqpoint{2.433135in}{1.170714in}}%
\pgfpathlineto{\pgfqpoint{2.419203in}{1.174686in}}%
\pgfpathlineto{\pgfqpoint{2.412606in}{1.158191in}}%
\pgfpathlineto{\pgfqpoint{2.415941in}{1.146502in}}%
\pgfpathlineto{\pgfqpoint{2.410720in}{1.144776in}}%
\pgfpathlineto{\pgfqpoint{2.409990in}{1.135540in}}%
\pgfpathlineto{\pgfqpoint{2.397176in}{1.131815in}}%
\pgfpathlineto{\pgfqpoint{2.390518in}{1.139272in}}%
\pgfpathlineto{\pgfqpoint{2.381950in}{1.142163in}}%
\pgfpathlineto{\pgfqpoint{2.379937in}{1.149724in}}%
\pgfpathlineto{\pgfqpoint{2.372191in}{1.148394in}}%
\pgfpathlineto{\pgfqpoint{2.367116in}{1.141440in}}%
\pgfpathlineto{\pgfqpoint{2.359855in}{1.138994in}}%
\pgfpathlineto{\pgfqpoint{2.347031in}{1.143916in}}%
\pgfpathlineto{\pgfqpoint{2.339938in}{1.138060in}}%
\pgfpathlineto{\pgfqpoint{2.329857in}{1.144988in}}%
\pgfpathlineto{\pgfqpoint{2.310517in}{1.147114in}}%
\pgfpathlineto{\pgfqpoint{2.304684in}{1.159700in}}%
\pgfpathlineto{\pgfqpoint{2.297293in}{1.165276in}}%
\pgfpathlineto{\pgfqpoint{2.290128in}{1.161707in}}%
\pgfpathlineto{\pgfqpoint{2.278860in}{1.162554in}}%
\pgfpathclose%
\pgfusepath{fill}%
\end{pgfscope}%
\begin{pgfscope}%
\pgfpathrectangle{\pgfqpoint{0.100000in}{0.100000in}}{\pgfqpoint{2.989028in}{1.913466in}}%
\pgfusepath{clip}%
\pgfsetbuttcap%
\pgfsetmiterjoin%
\definecolor{currentfill}{rgb}{0.928874,0.971549,0.638062}%
\pgfsetfillcolor{currentfill}%
\pgfsetlinewidth{0.000000pt}%
\definecolor{currentstroke}{rgb}{0.000000,0.000000,0.000000}%
\pgfsetstrokecolor{currentstroke}%
\pgfsetstrokeopacity{0.000000}%
\pgfsetdash{}{0pt}%
\pgfpathmoveto{\pgfqpoint{2.069266in}{1.005020in}}%
\pgfpathlineto{\pgfqpoint{2.061849in}{1.011045in}}%
\pgfpathlineto{\pgfqpoint{2.055801in}{1.008178in}}%
\pgfpathlineto{\pgfqpoint{2.048069in}{1.022616in}}%
\pgfpathlineto{\pgfqpoint{2.052019in}{1.031690in}}%
\pgfpathlineto{\pgfqpoint{2.046326in}{1.041904in}}%
\pgfpathlineto{\pgfqpoint{2.046020in}{1.049946in}}%
\pgfpathlineto{\pgfqpoint{2.038328in}{1.052811in}}%
\pgfpathlineto{\pgfqpoint{2.034760in}{1.058870in}}%
\pgfpathlineto{\pgfqpoint{2.019719in}{1.066437in}}%
\pgfpathlineto{\pgfqpoint{2.006657in}{1.075760in}}%
\pgfpathlineto{\pgfqpoint{2.000587in}{1.082799in}}%
\pgfpathlineto{\pgfqpoint{2.000462in}{1.091383in}}%
\pgfpathlineto{\pgfqpoint{2.008584in}{1.107865in}}%
\pgfpathlineto{\pgfqpoint{2.011082in}{1.120472in}}%
\pgfpathlineto{\pgfqpoint{2.004465in}{1.127595in}}%
\pgfpathlineto{\pgfqpoint{1.995702in}{1.130279in}}%
\pgfpathlineto{\pgfqpoint{1.985070in}{1.124410in}}%
\pgfpathlineto{\pgfqpoint{1.980570in}{1.134486in}}%
\pgfpathlineto{\pgfqpoint{1.978299in}{1.148144in}}%
\pgfpathlineto{\pgfqpoint{1.962630in}{1.160319in}}%
\pgfpathlineto{\pgfqpoint{1.959499in}{1.165722in}}%
\pgfpathlineto{\pgfqpoint{1.945188in}{1.177954in}}%
\pgfpathlineto{\pgfqpoint{1.940703in}{1.186846in}}%
\pgfpathlineto{\pgfqpoint{1.936659in}{1.204480in}}%
\pgfpathlineto{\pgfqpoint{1.939431in}{1.220143in}}%
\pgfpathlineto{\pgfqpoint{1.943130in}{1.222326in}}%
\pgfpathlineto{\pgfqpoint{1.942465in}{1.235438in}}%
\pgfpathlineto{\pgfqpoint{1.952907in}{1.239332in}}%
\pgfpathlineto{\pgfqpoint{1.956060in}{1.251098in}}%
\pgfpathlineto{\pgfqpoint{1.962068in}{1.259018in}}%
\pgfpathlineto{\pgfqpoint{1.961764in}{1.269093in}}%
\pgfpathlineto{\pgfqpoint{1.954318in}{1.277106in}}%
\pgfpathlineto{\pgfqpoint{1.956087in}{1.288322in}}%
\pgfpathlineto{\pgfqpoint{1.975389in}{1.293200in}}%
\pgfpathlineto{\pgfqpoint{1.990197in}{1.302104in}}%
\pgfpathlineto{\pgfqpoint{1.991751in}{1.313315in}}%
\pgfpathlineto{\pgfqpoint{1.996904in}{1.316845in}}%
\pgfpathlineto{\pgfqpoint{1.998875in}{1.328623in}}%
\pgfpathlineto{\pgfqpoint{1.997291in}{1.336402in}}%
\pgfpathlineto{\pgfqpoint{1.987167in}{1.342864in}}%
\pgfpathlineto{\pgfqpoint{1.983091in}{1.352497in}}%
\pgfpathlineto{\pgfqpoint{1.973100in}{1.361809in}}%
\pgfpathlineto{\pgfqpoint{2.055206in}{1.365308in}}%
\pgfpathlineto{\pgfqpoint{2.110291in}{1.369221in}}%
\pgfpathlineto{\pgfqpoint{2.109283in}{1.357634in}}%
\pgfpathlineto{\pgfqpoint{2.118671in}{1.341651in}}%
\pgfpathlineto{\pgfqpoint{2.122612in}{1.327996in}}%
\pgfpathlineto{\pgfqpoint{2.127306in}{1.320243in}}%
\pgfpathlineto{\pgfqpoint{2.132431in}{1.256358in}}%
\pgfpathlineto{\pgfqpoint{2.139547in}{1.165295in}}%
\pgfpathlineto{\pgfqpoint{2.134057in}{1.151457in}}%
\pgfpathlineto{\pgfqpoint{2.141966in}{1.140083in}}%
\pgfpathlineto{\pgfqpoint{2.144170in}{1.129219in}}%
\pgfpathlineto{\pgfqpoint{2.133263in}{1.106692in}}%
\pgfpathlineto{\pgfqpoint{2.133907in}{1.103580in}}%
\pgfpathlineto{\pgfqpoint{2.122681in}{1.090668in}}%
\pgfpathlineto{\pgfqpoint{2.125759in}{1.086724in}}%
\pgfpathlineto{\pgfqpoint{2.121088in}{1.078676in}}%
\pgfpathlineto{\pgfqpoint{2.122290in}{1.062454in}}%
\pgfpathlineto{\pgfqpoint{2.116617in}{1.052499in}}%
\pgfpathlineto{\pgfqpoint{2.121231in}{1.040736in}}%
\pgfpathlineto{\pgfqpoint{2.101900in}{1.034344in}}%
\pgfpathlineto{\pgfqpoint{2.100119in}{1.027393in}}%
\pgfpathlineto{\pgfqpoint{2.105397in}{1.018582in}}%
\pgfpathlineto{\pgfqpoint{2.102969in}{1.012840in}}%
\pgfpathlineto{\pgfqpoint{2.082240in}{1.019933in}}%
\pgfpathlineto{\pgfqpoint{2.075401in}{1.020647in}}%
\pgfpathlineto{\pgfqpoint{2.066874in}{1.009863in}}%
\pgfpathlineto{\pgfqpoint{2.069266in}{1.005020in}}%
\pgfpathclose%
\pgfusepath{fill}%
\end{pgfscope}%
\begin{pgfscope}%
\pgfpathrectangle{\pgfqpoint{0.100000in}{0.100000in}}{\pgfqpoint{2.989028in}{1.913466in}}%
\pgfusepath{clip}%
\pgfsetbuttcap%
\pgfsetmiterjoin%
\definecolor{currentfill}{rgb}{0.905805,0.962322,0.602076}%
\pgfsetfillcolor{currentfill}%
\pgfsetlinewidth{0.000000pt}%
\definecolor{currentstroke}{rgb}{0.000000,0.000000,0.000000}%
\pgfsetstrokecolor{currentstroke}%
\pgfsetstrokeopacity{0.000000}%
\pgfsetdash{}{0pt}%
\pgfpathmoveto{\pgfqpoint{2.672645in}{1.203210in}}%
\pgfpathlineto{\pgfqpoint{2.666941in}{1.211681in}}%
\pgfpathlineto{\pgfqpoint{2.670160in}{1.216421in}}%
\pgfpathlineto{\pgfqpoint{2.678043in}{1.211098in}}%
\pgfpathlineto{\pgfqpoint{2.672645in}{1.203210in}}%
\pgfpathclose%
\pgfusepath{fill}%
\end{pgfscope}%
\begin{pgfscope}%
\pgfpathrectangle{\pgfqpoint{0.100000in}{0.100000in}}{\pgfqpoint{2.989028in}{1.913466in}}%
\pgfusepath{clip}%
\pgfsetbuttcap%
\pgfsetmiterjoin%
\definecolor{currentfill}{rgb}{0.998231,0.945175,0.657055}%
\pgfsetfillcolor{currentfill}%
\pgfsetlinewidth{0.000000pt}%
\definecolor{currentstroke}{rgb}{0.000000,0.000000,0.000000}%
\pgfsetstrokecolor{currentstroke}%
\pgfsetstrokeopacity{0.000000}%
\pgfsetdash{}{0pt}%
\pgfpathmoveto{\pgfqpoint{2.722458in}{1.276100in}}%
\pgfpathlineto{\pgfqpoint{2.725867in}{1.283302in}}%
\pgfpathlineto{\pgfqpoint{2.739648in}{1.284858in}}%
\pgfpathlineto{\pgfqpoint{2.737440in}{1.278737in}}%
\pgfpathlineto{\pgfqpoint{2.732893in}{1.270918in}}%
\pgfpathlineto{\pgfqpoint{2.735974in}{1.261602in}}%
\pgfpathlineto{\pgfqpoint{2.748133in}{1.250439in}}%
\pgfpathlineto{\pgfqpoint{2.750955in}{1.238685in}}%
\pgfpathlineto{\pgfqpoint{2.764963in}{1.224047in}}%
\pgfpathlineto{\pgfqpoint{2.770458in}{1.224674in}}%
\pgfpathlineto{\pgfqpoint{2.777304in}{1.202700in}}%
\pgfpathlineto{\pgfqpoint{2.776181in}{1.202481in}}%
\pgfpathlineto{\pgfqpoint{2.774933in}{1.202236in}}%
\pgfpathlineto{\pgfqpoint{2.744401in}{1.196391in}}%
\pgfpathlineto{\pgfqpoint{2.728066in}{1.254485in}}%
\pgfpathlineto{\pgfqpoint{2.722458in}{1.276100in}}%
\pgfpathclose%
\pgfusepath{fill}%
\end{pgfscope}%
\begin{pgfscope}%
\pgfpathrectangle{\pgfqpoint{0.100000in}{0.100000in}}{\pgfqpoint{2.989028in}{1.913466in}}%
\pgfusepath{clip}%
\pgfsetbuttcap%
\pgfsetmiterjoin%
\definecolor{currentfill}{rgb}{0.601615,0.839677,0.644137}%
\pgfsetfillcolor{currentfill}%
\pgfsetlinewidth{0.000000pt}%
\definecolor{currentstroke}{rgb}{0.000000,0.000000,0.000000}%
\pgfsetstrokecolor{currentstroke}%
\pgfsetstrokeopacity{0.000000}%
\pgfsetdash{}{0pt}%
\pgfpathmoveto{\pgfqpoint{2.437284in}{1.079184in}}%
\pgfpathlineto{\pgfqpoint{2.427636in}{1.079539in}}%
\pgfpathlineto{\pgfqpoint{2.418746in}{1.085662in}}%
\pgfpathlineto{\pgfqpoint{2.406736in}{1.104274in}}%
\pgfpathlineto{\pgfqpoint{2.396513in}{1.114131in}}%
\pgfpathlineto{\pgfqpoint{2.399185in}{1.121744in}}%
\pgfpathlineto{\pgfqpoint{2.397176in}{1.131815in}}%
\pgfpathlineto{\pgfqpoint{2.409990in}{1.135540in}}%
\pgfpathlineto{\pgfqpoint{2.410720in}{1.144776in}}%
\pgfpathlineto{\pgfqpoint{2.415941in}{1.146502in}}%
\pgfpathlineto{\pgfqpoint{2.412606in}{1.158191in}}%
\pgfpathlineto{\pgfqpoint{2.419203in}{1.174686in}}%
\pgfpathlineto{\pgfqpoint{2.433135in}{1.170714in}}%
\pgfpathlineto{\pgfqpoint{2.431963in}{1.187230in}}%
\pgfpathlineto{\pgfqpoint{2.444473in}{1.204383in}}%
\pgfpathlineto{\pgfqpoint{2.449528in}{1.201094in}}%
\pgfpathlineto{\pgfqpoint{2.456028in}{1.204915in}}%
\pgfpathlineto{\pgfqpoint{2.471469in}{1.222911in}}%
\pgfpathlineto{\pgfqpoint{2.474521in}{1.243089in}}%
\pgfpathlineto{\pgfqpoint{2.474951in}{1.258593in}}%
\pgfpathlineto{\pgfqpoint{2.478201in}{1.269199in}}%
\pgfpathlineto{\pgfqpoint{2.476532in}{1.280492in}}%
\pgfpathlineto{\pgfqpoint{2.472344in}{1.285643in}}%
\pgfpathlineto{\pgfqpoint{2.478831in}{1.291252in}}%
\pgfpathlineto{\pgfqpoint{2.488234in}{1.232010in}}%
\pgfpathlineto{\pgfqpoint{2.540072in}{1.240581in}}%
\pgfpathlineto{\pgfqpoint{2.545445in}{1.206768in}}%
\pgfpathlineto{\pgfqpoint{2.564213in}{1.229086in}}%
\pgfpathlineto{\pgfqpoint{2.568635in}{1.226861in}}%
\pgfpathlineto{\pgfqpoint{2.575140in}{1.239793in}}%
\pgfpathlineto{\pgfqpoint{2.584152in}{1.235738in}}%
\pgfpathlineto{\pgfqpoint{2.592010in}{1.236509in}}%
\pgfpathlineto{\pgfqpoint{2.597194in}{1.245499in}}%
\pgfpathlineto{\pgfqpoint{2.604719in}{1.250500in}}%
\pgfpathlineto{\pgfqpoint{2.615180in}{1.246096in}}%
\pgfpathlineto{\pgfqpoint{2.622064in}{1.247618in}}%
\pgfpathlineto{\pgfqpoint{2.631926in}{1.230550in}}%
\pgfpathlineto{\pgfqpoint{2.629074in}{1.217629in}}%
\pgfpathlineto{\pgfqpoint{2.603317in}{1.232177in}}%
\pgfpathlineto{\pgfqpoint{2.598492in}{1.220234in}}%
\pgfpathlineto{\pgfqpoint{2.600084in}{1.214739in}}%
\pgfpathlineto{\pgfqpoint{2.589121in}{1.196164in}}%
\pgfpathlineto{\pgfqpoint{2.583947in}{1.192470in}}%
\pgfpathlineto{\pgfqpoint{2.581611in}{1.184279in}}%
\pgfpathlineto{\pgfqpoint{2.572791in}{1.185136in}}%
\pgfpathlineto{\pgfqpoint{2.566456in}{1.162779in}}%
\pgfpathlineto{\pgfqpoint{2.562911in}{1.157648in}}%
\pgfpathlineto{\pgfqpoint{2.553799in}{1.159359in}}%
\pgfpathlineto{\pgfqpoint{2.544480in}{1.166411in}}%
\pgfpathlineto{\pgfqpoint{2.544157in}{1.155588in}}%
\pgfpathlineto{\pgfqpoint{2.535135in}{1.137399in}}%
\pgfpathlineto{\pgfqpoint{2.532891in}{1.124451in}}%
\pgfpathlineto{\pgfqpoint{2.525969in}{1.115827in}}%
\pgfpathlineto{\pgfqpoint{2.520721in}{1.102055in}}%
\pgfpathlineto{\pgfqpoint{2.520463in}{1.089306in}}%
\pgfpathlineto{\pgfqpoint{2.512358in}{1.086925in}}%
\pgfpathlineto{\pgfqpoint{2.503164in}{1.079708in}}%
\pgfpathlineto{\pgfqpoint{2.494311in}{1.078334in}}%
\pgfpathlineto{\pgfqpoint{2.492271in}{1.072225in}}%
\pgfpathlineto{\pgfqpoint{2.478007in}{1.065970in}}%
\pgfpathlineto{\pgfqpoint{2.470035in}{1.071300in}}%
\pgfpathlineto{\pgfqpoint{2.461126in}{1.061179in}}%
\pgfpathlineto{\pgfqpoint{2.445810in}{1.064162in}}%
\pgfpathlineto{\pgfqpoint{2.436416in}{1.074785in}}%
\pgfpathlineto{\pgfqpoint{2.437284in}{1.079184in}}%
\pgfpathclose%
\pgfusepath{fill}%
\end{pgfscope}%
\begin{pgfscope}%
\pgfpathrectangle{\pgfqpoint{0.100000in}{0.100000in}}{\pgfqpoint{2.989028in}{1.913466in}}%
\pgfusepath{clip}%
\pgfsetbuttcap%
\pgfsetmiterjoin%
\definecolor{currentfill}{rgb}{0.892887,0.957093,0.597924}%
\pgfsetfillcolor{currentfill}%
\pgfsetlinewidth{0.000000pt}%
\definecolor{currentstroke}{rgb}{0.000000,0.000000,0.000000}%
\pgfsetstrokecolor{currentstroke}%
\pgfsetstrokeopacity{0.000000}%
\pgfsetdash{}{0pt}%
\pgfpathmoveto{\pgfqpoint{2.774933in}{1.202236in}}%
\pgfpathlineto{\pgfqpoint{2.774542in}{1.190298in}}%
\pgfpathlineto{\pgfqpoint{2.769947in}{1.184429in}}%
\pgfpathlineto{\pgfqpoint{2.767015in}{1.171401in}}%
\pgfpathlineto{\pgfqpoint{2.753775in}{1.165395in}}%
\pgfpathlineto{\pgfqpoint{2.742667in}{1.163645in}}%
\pgfpathlineto{\pgfqpoint{2.745904in}{1.172214in}}%
\pgfpathlineto{\pgfqpoint{2.728800in}{1.179393in}}%
\pgfpathlineto{\pgfqpoint{2.714904in}{1.188383in}}%
\pgfpathlineto{\pgfqpoint{2.723507in}{1.201821in}}%
\pgfpathlineto{\pgfqpoint{2.716613in}{1.212266in}}%
\pgfpathlineto{\pgfqpoint{2.717424in}{1.221276in}}%
\pgfpathlineto{\pgfqpoint{2.712439in}{1.223749in}}%
\pgfpathlineto{\pgfqpoint{2.708394in}{1.238386in}}%
\pgfpathlineto{\pgfqpoint{2.717960in}{1.258229in}}%
\pgfpathlineto{\pgfqpoint{2.710773in}{1.261469in}}%
\pgfpathlineto{\pgfqpoint{2.708910in}{1.251680in}}%
\pgfpathlineto{\pgfqpoint{2.698651in}{1.248954in}}%
\pgfpathlineto{\pgfqpoint{2.699730in}{1.216746in}}%
\pgfpathlineto{\pgfqpoint{2.697854in}{1.206381in}}%
\pgfpathlineto{\pgfqpoint{2.703000in}{1.191607in}}%
\pgfpathlineto{\pgfqpoint{2.710916in}{1.184499in}}%
\pgfpathlineto{\pgfqpoint{2.707326in}{1.180036in}}%
\pgfpathlineto{\pgfqpoint{2.715407in}{1.173514in}}%
\pgfpathlineto{\pgfqpoint{2.718328in}{1.162929in}}%
\pgfpathlineto{\pgfqpoint{2.703517in}{1.171694in}}%
\pgfpathlineto{\pgfqpoint{2.694143in}{1.170556in}}%
\pgfpathlineto{\pgfqpoint{2.686865in}{1.179573in}}%
\pgfpathlineto{\pgfqpoint{2.679454in}{1.180451in}}%
\pgfpathlineto{\pgfqpoint{2.668921in}{1.175935in}}%
\pgfpathlineto{\pgfqpoint{2.664837in}{1.181563in}}%
\pgfpathlineto{\pgfqpoint{2.670205in}{1.193376in}}%
\pgfpathlineto{\pgfqpoint{2.672645in}{1.203210in}}%
\pgfpathlineto{\pgfqpoint{2.678043in}{1.211098in}}%
\pgfpathlineto{\pgfqpoint{2.670160in}{1.216421in}}%
\pgfpathlineto{\pgfqpoint{2.666941in}{1.211681in}}%
\pgfpathlineto{\pgfqpoint{2.659066in}{1.216492in}}%
\pgfpathlineto{\pgfqpoint{2.645120in}{1.219698in}}%
\pgfpathlineto{\pgfqpoint{2.646385in}{1.226747in}}%
\pgfpathlineto{\pgfqpoint{2.640064in}{1.230836in}}%
\pgfpathlineto{\pgfqpoint{2.631926in}{1.230550in}}%
\pgfpathlineto{\pgfqpoint{2.622064in}{1.247618in}}%
\pgfpathlineto{\pgfqpoint{2.615180in}{1.246096in}}%
\pgfpathlineto{\pgfqpoint{2.604719in}{1.250500in}}%
\pgfpathlineto{\pgfqpoint{2.597194in}{1.245499in}}%
\pgfpathlineto{\pgfqpoint{2.592010in}{1.236509in}}%
\pgfpathlineto{\pgfqpoint{2.584152in}{1.235738in}}%
\pgfpathlineto{\pgfqpoint{2.575140in}{1.239793in}}%
\pgfpathlineto{\pgfqpoint{2.568635in}{1.226861in}}%
\pgfpathlineto{\pgfqpoint{2.564213in}{1.229086in}}%
\pgfpathlineto{\pgfqpoint{2.545445in}{1.206768in}}%
\pgfpathlineto{\pgfqpoint{2.540072in}{1.240581in}}%
\pgfpathlineto{\pgfqpoint{2.608655in}{1.253292in}}%
\pgfpathlineto{\pgfqpoint{2.639399in}{1.258797in}}%
\pgfpathlineto{\pgfqpoint{2.684127in}{1.267806in}}%
\pgfpathlineto{\pgfqpoint{2.722458in}{1.276100in}}%
\pgfpathlineto{\pgfqpoint{2.728066in}{1.254485in}}%
\pgfpathlineto{\pgfqpoint{2.744401in}{1.196391in}}%
\pgfpathlineto{\pgfqpoint{2.774933in}{1.202236in}}%
\pgfpathclose%
\pgfusepath{fill}%
\end{pgfscope}%
\begin{pgfscope}%
\pgfpathrectangle{\pgfqpoint{0.100000in}{0.100000in}}{\pgfqpoint{2.989028in}{1.913466in}}%
\pgfusepath{clip}%
\pgfsetbuttcap%
\pgfsetmiterjoin%
\definecolor{currentfill}{rgb}{0.756786,0.901730,0.625606}%
\pgfsetfillcolor{currentfill}%
\pgfsetlinewidth{0.000000pt}%
\definecolor{currentstroke}{rgb}{0.000000,0.000000,0.000000}%
\pgfsetstrokecolor{currentstroke}%
\pgfsetstrokeopacity{0.000000}%
\pgfsetdash{}{0pt}%
\pgfpathmoveto{\pgfqpoint{1.394404in}{1.009408in}}%
\pgfpathlineto{\pgfqpoint{1.344368in}{1.014183in}}%
\pgfpathlineto{\pgfqpoint{1.292460in}{1.018783in}}%
\pgfpathlineto{\pgfqpoint{1.232477in}{1.025040in}}%
\pgfpathlineto{\pgfqpoint{1.143359in}{1.035639in}}%
\pgfpathlineto{\pgfqpoint{1.111744in}{1.040510in}}%
\pgfpathlineto{\pgfqpoint{1.030137in}{1.052362in}}%
\pgfpathlineto{\pgfqpoint{1.042053in}{1.127151in}}%
\pgfpathlineto{\pgfqpoint{1.042350in}{1.133200in}}%
\pgfpathlineto{\pgfqpoint{1.053829in}{1.205365in}}%
\pgfpathlineto{\pgfqpoint{1.062404in}{1.260326in}}%
\pgfpathlineto{\pgfqpoint{1.070495in}{1.311293in}}%
\pgfpathlineto{\pgfqpoint{1.125778in}{1.303357in}}%
\pgfpathlineto{\pgfqpoint{1.177321in}{1.295959in}}%
\pgfpathlineto{\pgfqpoint{1.272072in}{1.284265in}}%
\pgfpathlineto{\pgfqpoint{1.315540in}{1.280295in}}%
\pgfpathlineto{\pgfqpoint{1.384404in}{1.273614in}}%
\pgfpathlineto{\pgfqpoint{1.414195in}{1.271196in}}%
\pgfpathlineto{\pgfqpoint{1.408935in}{1.205970in}}%
\pgfpathlineto{\pgfqpoint{1.404183in}{1.143169in}}%
\pgfpathlineto{\pgfqpoint{1.400359in}{1.092033in}}%
\pgfpathlineto{\pgfqpoint{1.394404in}{1.009408in}}%
\pgfpathclose%
\pgfusepath{fill}%
\end{pgfscope}%
\begin{pgfscope}%
\pgfpathrectangle{\pgfqpoint{0.100000in}{0.100000in}}{\pgfqpoint{2.989028in}{1.913466in}}%
\pgfusepath{clip}%
\pgfsetbuttcap%
\pgfsetmiterjoin%
\definecolor{currentfill}{rgb}{0.747712,0.898039,0.627451}%
\pgfsetfillcolor{currentfill}%
\pgfsetlinewidth{0.000000pt}%
\definecolor{currentstroke}{rgb}{0.000000,0.000000,0.000000}%
\pgfsetstrokecolor{currentstroke}%
\pgfsetstrokeopacity{0.000000}%
\pgfsetdash{}{0pt}%
\pgfpathmoveto{\pgfqpoint{2.056343in}{0.972329in}}%
\pgfpathlineto{\pgfqpoint{2.058030in}{0.979890in}}%
\pgfpathlineto{\pgfqpoint{2.066773in}{0.978188in}}%
\pgfpathlineto{\pgfqpoint{2.069266in}{1.005020in}}%
\pgfpathlineto{\pgfqpoint{2.066874in}{1.009863in}}%
\pgfpathlineto{\pgfqpoint{2.075401in}{1.020647in}}%
\pgfpathlineto{\pgfqpoint{2.082240in}{1.019933in}}%
\pgfpathlineto{\pgfqpoint{2.102969in}{1.012840in}}%
\pgfpathlineto{\pgfqpoint{2.105397in}{1.018582in}}%
\pgfpathlineto{\pgfqpoint{2.100119in}{1.027393in}}%
\pgfpathlineto{\pgfqpoint{2.101900in}{1.034344in}}%
\pgfpathlineto{\pgfqpoint{2.121231in}{1.040736in}}%
\pgfpathlineto{\pgfqpoint{2.116617in}{1.052499in}}%
\pgfpathlineto{\pgfqpoint{2.122290in}{1.062454in}}%
\pgfpathlineto{\pgfqpoint{2.132850in}{1.068041in}}%
\pgfpathlineto{\pgfqpoint{2.155006in}{1.073574in}}%
\pgfpathlineto{\pgfqpoint{2.168985in}{1.065222in}}%
\pgfpathlineto{\pgfqpoint{2.173226in}{1.073367in}}%
\pgfpathlineto{\pgfqpoint{2.184834in}{1.079056in}}%
\pgfpathlineto{\pgfqpoint{2.187829in}{1.074039in}}%
\pgfpathlineto{\pgfqpoint{2.199753in}{1.078025in}}%
\pgfpathlineto{\pgfqpoint{2.199000in}{1.084848in}}%
\pgfpathlineto{\pgfqpoint{2.209721in}{1.092656in}}%
\pgfpathlineto{\pgfqpoint{2.216048in}{1.084548in}}%
\pgfpathlineto{\pgfqpoint{2.224371in}{1.083732in}}%
\pgfpathlineto{\pgfqpoint{2.229903in}{1.089016in}}%
\pgfpathlineto{\pgfqpoint{2.229287in}{1.096531in}}%
\pgfpathlineto{\pgfqpoint{2.233989in}{1.103993in}}%
\pgfpathlineto{\pgfqpoint{2.240290in}{1.105610in}}%
\pgfpathlineto{\pgfqpoint{2.242820in}{1.115463in}}%
\pgfpathlineto{\pgfqpoint{2.251983in}{1.123981in}}%
\pgfpathlineto{\pgfqpoint{2.249225in}{1.132459in}}%
\pgfpathlineto{\pgfqpoint{2.258139in}{1.136680in}}%
\pgfpathlineto{\pgfqpoint{2.264096in}{1.134084in}}%
\pgfpathlineto{\pgfqpoint{2.272893in}{1.140679in}}%
\pgfpathlineto{\pgfqpoint{2.280755in}{1.142397in}}%
\pgfpathlineto{\pgfqpoint{2.280783in}{1.149238in}}%
\pgfpathlineto{\pgfqpoint{2.275273in}{1.158730in}}%
\pgfpathlineto{\pgfqpoint{2.278860in}{1.162554in}}%
\pgfpathlineto{\pgfqpoint{2.290128in}{1.161707in}}%
\pgfpathlineto{\pgfqpoint{2.297293in}{1.165276in}}%
\pgfpathlineto{\pgfqpoint{2.304684in}{1.159700in}}%
\pgfpathlineto{\pgfqpoint{2.310517in}{1.147114in}}%
\pgfpathlineto{\pgfqpoint{2.329857in}{1.144988in}}%
\pgfpathlineto{\pgfqpoint{2.339938in}{1.138060in}}%
\pgfpathlineto{\pgfqpoint{2.347031in}{1.143916in}}%
\pgfpathlineto{\pgfqpoint{2.359855in}{1.138994in}}%
\pgfpathlineto{\pgfqpoint{2.367116in}{1.141440in}}%
\pgfpathlineto{\pgfqpoint{2.372191in}{1.148394in}}%
\pgfpathlineto{\pgfqpoint{2.379937in}{1.149724in}}%
\pgfpathlineto{\pgfqpoint{2.381950in}{1.142163in}}%
\pgfpathlineto{\pgfqpoint{2.390518in}{1.139272in}}%
\pgfpathlineto{\pgfqpoint{2.397176in}{1.131815in}}%
\pgfpathlineto{\pgfqpoint{2.399185in}{1.121744in}}%
\pgfpathlineto{\pgfqpoint{2.396513in}{1.114131in}}%
\pgfpathlineto{\pgfqpoint{2.406736in}{1.104274in}}%
\pgfpathlineto{\pgfqpoint{2.418746in}{1.085662in}}%
\pgfpathlineto{\pgfqpoint{2.427636in}{1.079539in}}%
\pgfpathlineto{\pgfqpoint{2.437284in}{1.079184in}}%
\pgfpathlineto{\pgfqpoint{2.419469in}{1.058743in}}%
\pgfpathlineto{\pgfqpoint{2.401934in}{1.046371in}}%
\pgfpathlineto{\pgfqpoint{2.395481in}{1.036545in}}%
\pgfpathlineto{\pgfqpoint{2.395590in}{1.031213in}}%
\pgfpathlineto{\pgfqpoint{2.386092in}{1.027126in}}%
\pgfpathlineto{\pgfqpoint{2.383370in}{1.019436in}}%
\pgfpathlineto{\pgfqpoint{2.363584in}{1.011690in}}%
\pgfpathlineto{\pgfqpoint{2.356571in}{1.006660in}}%
\pgfpathlineto{\pgfqpoint{2.355622in}{1.005586in}}%
\pgfpathlineto{\pgfqpoint{2.298708in}{1.000190in}}%
\pgfpathlineto{\pgfqpoint{2.264337in}{0.997299in}}%
\pgfpathlineto{\pgfqpoint{2.207938in}{0.994107in}}%
\pgfpathlineto{\pgfqpoint{2.137674in}{0.987156in}}%
\pgfpathlineto{\pgfqpoint{2.126065in}{0.988764in}}%
\pgfpathlineto{\pgfqpoint{2.128480in}{0.976914in}}%
\pgfpathlineto{\pgfqpoint{2.056343in}{0.972329in}}%
\pgfpathclose%
\pgfusepath{fill}%
\end{pgfscope}%
\begin{pgfscope}%
\pgfpathrectangle{\pgfqpoint{0.100000in}{0.100000in}}{\pgfqpoint{2.989028in}{1.913466in}}%
\pgfusepath{clip}%
\pgfsetbuttcap%
\pgfsetmiterjoin%
\definecolor{currentfill}{rgb}{0.307728,0.388697,0.672664}%
\pgfsetfillcolor{currentfill}%
\pgfsetlinewidth{0.000000pt}%
\definecolor{currentstroke}{rgb}{0.000000,0.000000,0.000000}%
\pgfsetstrokecolor{currentstroke}%
\pgfsetstrokeopacity{0.000000}%
\pgfsetdash{}{0pt}%
\pgfpathmoveto{\pgfqpoint{1.394404in}{1.009408in}}%
\pgfpathlineto{\pgfqpoint{1.400359in}{1.092033in}}%
\pgfpathlineto{\pgfqpoint{1.404183in}{1.143169in}}%
\pgfpathlineto{\pgfqpoint{1.408935in}{1.205970in}}%
\pgfpathlineto{\pgfqpoint{1.474809in}{1.201345in}}%
\pgfpathlineto{\pgfqpoint{1.558468in}{1.196775in}}%
\pgfpathlineto{\pgfqpoint{1.615367in}{1.194592in}}%
\pgfpathlineto{\pgfqpoint{1.671933in}{1.192859in}}%
\pgfpathlineto{\pgfqpoint{1.723131in}{1.192066in}}%
\pgfpathlineto{\pgfqpoint{1.746810in}{1.192312in}}%
\pgfpathlineto{\pgfqpoint{1.757233in}{1.183806in}}%
\pgfpathlineto{\pgfqpoint{1.765397in}{1.185521in}}%
\pgfpathlineto{\pgfqpoint{1.765653in}{1.178102in}}%
\pgfpathlineto{\pgfqpoint{1.759590in}{1.165269in}}%
\pgfpathlineto{\pgfqpoint{1.760247in}{1.157161in}}%
\pgfpathlineto{\pgfqpoint{1.766052in}{1.151814in}}%
\pgfpathlineto{\pgfqpoint{1.771384in}{1.140674in}}%
\pgfpathlineto{\pgfqpoint{1.782201in}{1.134279in}}%
\pgfpathlineto{\pgfqpoint{1.781840in}{1.092287in}}%
\pgfpathlineto{\pgfqpoint{1.782158in}{0.995725in}}%
\pgfpathlineto{\pgfqpoint{1.709656in}{0.996060in}}%
\pgfpathlineto{\pgfqpoint{1.633309in}{0.997396in}}%
\pgfpathlineto{\pgfqpoint{1.577105in}{0.999379in}}%
\pgfpathlineto{\pgfqpoint{1.530191in}{1.001189in}}%
\pgfpathlineto{\pgfqpoint{1.451155in}{1.005849in}}%
\pgfpathlineto{\pgfqpoint{1.394404in}{1.009408in}}%
\pgfpathclose%
\pgfusepath{fill}%
\end{pgfscope}%
\begin{pgfscope}%
\pgfpathrectangle{\pgfqpoint{0.100000in}{0.100000in}}{\pgfqpoint{2.989028in}{1.913466in}}%
\pgfusepath{clip}%
\pgfsetbuttcap%
\pgfsetmiterjoin%
\definecolor{currentfill}{rgb}{0.917339,0.966936,0.620069}%
\pgfsetfillcolor{currentfill}%
\pgfsetlinewidth{0.000000pt}%
\definecolor{currentstroke}{rgb}{0.000000,0.000000,0.000000}%
\pgfsetstrokecolor{currentstroke}%
\pgfsetstrokeopacity{0.000000}%
\pgfsetdash{}{0pt}%
\pgfpathmoveto{\pgfqpoint{2.461189in}{1.020348in}}%
\pgfpathlineto{\pgfqpoint{2.380888in}{1.009068in}}%
\pgfpathlineto{\pgfqpoint{2.356571in}{1.006660in}}%
\pgfpathlineto{\pgfqpoint{2.363584in}{1.011690in}}%
\pgfpathlineto{\pgfqpoint{2.383370in}{1.019436in}}%
\pgfpathlineto{\pgfqpoint{2.386092in}{1.027126in}}%
\pgfpathlineto{\pgfqpoint{2.395590in}{1.031213in}}%
\pgfpathlineto{\pgfqpoint{2.395481in}{1.036545in}}%
\pgfpathlineto{\pgfqpoint{2.401934in}{1.046371in}}%
\pgfpathlineto{\pgfqpoint{2.419469in}{1.058743in}}%
\pgfpathlineto{\pgfqpoint{2.437284in}{1.079184in}}%
\pgfpathlineto{\pgfqpoint{2.436416in}{1.074785in}}%
\pgfpathlineto{\pgfqpoint{2.445810in}{1.064162in}}%
\pgfpathlineto{\pgfqpoint{2.461126in}{1.061179in}}%
\pgfpathlineto{\pgfqpoint{2.470035in}{1.071300in}}%
\pgfpathlineto{\pgfqpoint{2.478007in}{1.065970in}}%
\pgfpathlineto{\pgfqpoint{2.492271in}{1.072225in}}%
\pgfpathlineto{\pgfqpoint{2.494311in}{1.078334in}}%
\pgfpathlineto{\pgfqpoint{2.503164in}{1.079708in}}%
\pgfpathlineto{\pgfqpoint{2.512358in}{1.086925in}}%
\pgfpathlineto{\pgfqpoint{2.520463in}{1.089306in}}%
\pgfpathlineto{\pgfqpoint{2.520721in}{1.102055in}}%
\pgfpathlineto{\pgfqpoint{2.525969in}{1.115827in}}%
\pgfpathlineto{\pgfqpoint{2.532891in}{1.124451in}}%
\pgfpathlineto{\pgfqpoint{2.535135in}{1.137399in}}%
\pgfpathlineto{\pgfqpoint{2.544157in}{1.155588in}}%
\pgfpathlineto{\pgfqpoint{2.544480in}{1.166411in}}%
\pgfpathlineto{\pgfqpoint{2.553799in}{1.159359in}}%
\pgfpathlineto{\pgfqpoint{2.562911in}{1.157648in}}%
\pgfpathlineto{\pgfqpoint{2.566456in}{1.162779in}}%
\pgfpathlineto{\pgfqpoint{2.572791in}{1.185136in}}%
\pgfpathlineto{\pgfqpoint{2.581611in}{1.184279in}}%
\pgfpathlineto{\pgfqpoint{2.583947in}{1.192470in}}%
\pgfpathlineto{\pgfqpoint{2.589121in}{1.196164in}}%
\pgfpathlineto{\pgfqpoint{2.600084in}{1.214739in}}%
\pgfpathlineto{\pgfqpoint{2.598492in}{1.220234in}}%
\pgfpathlineto{\pgfqpoint{2.603317in}{1.232177in}}%
\pgfpathlineto{\pgfqpoint{2.629074in}{1.217629in}}%
\pgfpathlineto{\pgfqpoint{2.631926in}{1.230550in}}%
\pgfpathlineto{\pgfqpoint{2.640064in}{1.230836in}}%
\pgfpathlineto{\pgfqpoint{2.646385in}{1.226747in}}%
\pgfpathlineto{\pgfqpoint{2.645120in}{1.219698in}}%
\pgfpathlineto{\pgfqpoint{2.659066in}{1.216492in}}%
\pgfpathlineto{\pgfqpoint{2.666941in}{1.211681in}}%
\pgfpathlineto{\pgfqpoint{2.672645in}{1.203210in}}%
\pgfpathlineto{\pgfqpoint{2.670205in}{1.193376in}}%
\pgfpathlineto{\pgfqpoint{2.665277in}{1.192570in}}%
\pgfpathlineto{\pgfqpoint{2.662419in}{1.177731in}}%
\pgfpathlineto{\pgfqpoint{2.668680in}{1.171929in}}%
\pgfpathlineto{\pgfqpoint{2.677493in}{1.176619in}}%
\pgfpathlineto{\pgfqpoint{2.685671in}{1.166716in}}%
\pgfpathlineto{\pgfqpoint{2.703942in}{1.164936in}}%
\pgfpathlineto{\pgfqpoint{2.707089in}{1.159239in}}%
\pgfpathlineto{\pgfqpoint{2.723998in}{1.153699in}}%
\pgfpathlineto{\pgfqpoint{2.721912in}{1.147161in}}%
\pgfpathlineto{\pgfqpoint{2.724111in}{1.128683in}}%
\pgfpathlineto{\pgfqpoint{2.730936in}{1.121706in}}%
\pgfpathlineto{\pgfqpoint{2.720866in}{1.118761in}}%
\pgfpathlineto{\pgfqpoint{2.722201in}{1.110873in}}%
\pgfpathlineto{\pgfqpoint{2.732944in}{1.104185in}}%
\pgfpathlineto{\pgfqpoint{2.733912in}{1.097588in}}%
\pgfpathlineto{\pgfqpoint{2.727847in}{1.092630in}}%
\pgfpathlineto{\pgfqpoint{2.715609in}{1.104375in}}%
\pgfpathlineto{\pgfqpoint{2.714399in}{1.095806in}}%
\pgfpathlineto{\pgfqpoint{2.724644in}{1.091731in}}%
\pgfpathlineto{\pgfqpoint{2.725669in}{1.087517in}}%
\pgfpathlineto{\pgfqpoint{2.739721in}{1.093086in}}%
\pgfpathlineto{\pgfqpoint{2.750489in}{1.094559in}}%
\pgfpathlineto{\pgfqpoint{2.761520in}{1.072304in}}%
\pgfpathlineto{\pgfqpoint{2.760289in}{1.072063in}}%
\pgfpathlineto{\pgfqpoint{2.755305in}{1.071033in}}%
\pgfpathlineto{\pgfqpoint{2.753836in}{1.070726in}}%
\pgfpathlineto{\pgfqpoint{2.752865in}{1.070536in}}%
\pgfpathlineto{\pgfqpoint{2.687189in}{1.056913in}}%
\pgfpathlineto{\pgfqpoint{2.628447in}{1.044890in}}%
\pgfpathlineto{\pgfqpoint{2.547229in}{1.030898in}}%
\pgfpathlineto{\pgfqpoint{2.478251in}{1.021788in}}%
\pgfpathlineto{\pgfqpoint{2.461189in}{1.020348in}}%
\pgfpathclose%
\pgfusepath{fill}%
\end{pgfscope}%
\begin{pgfscope}%
\pgfpathrectangle{\pgfqpoint{0.100000in}{0.100000in}}{\pgfqpoint{2.989028in}{1.913466in}}%
\pgfusepath{clip}%
\pgfsetbuttcap%
\pgfsetmiterjoin%
\definecolor{currentfill}{rgb}{0.917339,0.966936,0.620069}%
\pgfsetfillcolor{currentfill}%
\pgfsetlinewidth{0.000000pt}%
\definecolor{currentstroke}{rgb}{0.000000,0.000000,0.000000}%
\pgfsetstrokecolor{currentstroke}%
\pgfsetstrokeopacity{0.000000}%
\pgfsetdash{}{0pt}%
\pgfpathmoveto{\pgfqpoint{2.753775in}{1.165395in}}%
\pgfpathlineto{\pgfqpoint{2.767015in}{1.171401in}}%
\pgfpathlineto{\pgfqpoint{2.756442in}{1.140471in}}%
\pgfpathlineto{\pgfqpoint{2.749450in}{1.124092in}}%
\pgfpathlineto{\pgfqpoint{2.744413in}{1.133361in}}%
\pgfpathlineto{\pgfqpoint{2.753370in}{1.155552in}}%
\pgfpathlineto{\pgfqpoint{2.753775in}{1.165395in}}%
\pgfpathclose%
\pgfusepath{fill}%
\end{pgfscope}%
\begin{pgfscope}%
\pgfpathrectangle{\pgfqpoint{0.100000in}{0.100000in}}{\pgfqpoint{2.989028in}{1.913466in}}%
\pgfusepath{clip}%
\pgfsetbuttcap%
\pgfsetmiterjoin%
\definecolor{currentfill}{rgb}{0.999769,0.992849,0.737024}%
\pgfsetfillcolor{currentfill}%
\pgfsetlinewidth{0.000000pt}%
\definecolor{currentstroke}{rgb}{0.000000,0.000000,0.000000}%
\pgfsetstrokecolor{currentstroke}%
\pgfsetstrokeopacity{0.000000}%
\pgfsetdash{}{0pt}%
\pgfpathmoveto{\pgfqpoint{2.056343in}{0.972329in}}%
\pgfpathlineto{\pgfqpoint{2.053146in}{0.971866in}}%
\pgfpathlineto{\pgfqpoint{2.050131in}{0.971655in}}%
\pgfpathlineto{\pgfqpoint{2.051421in}{0.962397in}}%
\pgfpathlineto{\pgfqpoint{2.043642in}{0.953066in}}%
\pgfpathlineto{\pgfqpoint{2.042109in}{0.938464in}}%
\pgfpathlineto{\pgfqpoint{2.007334in}{0.935892in}}%
\pgfpathlineto{\pgfqpoint{2.010365in}{0.942748in}}%
\pgfpathlineto{\pgfqpoint{2.022904in}{0.955269in}}%
\pgfpathlineto{\pgfqpoint{2.023251in}{0.962527in}}%
\pgfpathlineto{\pgfqpoint{2.017700in}{0.969399in}}%
\pgfpathlineto{\pgfqpoint{1.965941in}{0.966662in}}%
\pgfpathlineto{\pgfqpoint{1.875448in}{0.963789in}}%
\pgfpathlineto{\pgfqpoint{1.822491in}{0.962824in}}%
\pgfpathlineto{\pgfqpoint{1.782462in}{0.962463in}}%
\pgfpathlineto{\pgfqpoint{1.782158in}{0.995725in}}%
\pgfpathlineto{\pgfqpoint{1.781840in}{1.092287in}}%
\pgfpathlineto{\pgfqpoint{1.782201in}{1.134279in}}%
\pgfpathlineto{\pgfqpoint{1.771384in}{1.140674in}}%
\pgfpathlineto{\pgfqpoint{1.766052in}{1.151814in}}%
\pgfpathlineto{\pgfqpoint{1.760247in}{1.157161in}}%
\pgfpathlineto{\pgfqpoint{1.759590in}{1.165269in}}%
\pgfpathlineto{\pgfqpoint{1.765653in}{1.178102in}}%
\pgfpathlineto{\pgfqpoint{1.765397in}{1.185521in}}%
\pgfpathlineto{\pgfqpoint{1.757233in}{1.183806in}}%
\pgfpathlineto{\pgfqpoint{1.746810in}{1.192312in}}%
\pgfpathlineto{\pgfqpoint{1.738455in}{1.207242in}}%
\pgfpathlineto{\pgfqpoint{1.731447in}{1.214132in}}%
\pgfpathlineto{\pgfqpoint{1.724124in}{1.231060in}}%
\pgfpathlineto{\pgfqpoint{1.800184in}{1.229874in}}%
\pgfpathlineto{\pgfqpoint{1.875789in}{1.232365in}}%
\pgfpathlineto{\pgfqpoint{1.924266in}{1.235161in}}%
\pgfpathlineto{\pgfqpoint{1.939431in}{1.220143in}}%
\pgfpathlineto{\pgfqpoint{1.936659in}{1.204480in}}%
\pgfpathlineto{\pgfqpoint{1.940703in}{1.186846in}}%
\pgfpathlineto{\pgfqpoint{1.945188in}{1.177954in}}%
\pgfpathlineto{\pgfqpoint{1.959499in}{1.165722in}}%
\pgfpathlineto{\pgfqpoint{1.962630in}{1.160319in}}%
\pgfpathlineto{\pgfqpoint{1.978299in}{1.148144in}}%
\pgfpathlineto{\pgfqpoint{1.980570in}{1.134486in}}%
\pgfpathlineto{\pgfqpoint{1.985070in}{1.124410in}}%
\pgfpathlineto{\pgfqpoint{1.995702in}{1.130279in}}%
\pgfpathlineto{\pgfqpoint{2.004465in}{1.127595in}}%
\pgfpathlineto{\pgfqpoint{2.011082in}{1.120472in}}%
\pgfpathlineto{\pgfqpoint{2.008584in}{1.107865in}}%
\pgfpathlineto{\pgfqpoint{2.000462in}{1.091383in}}%
\pgfpathlineto{\pgfqpoint{2.000587in}{1.082799in}}%
\pgfpathlineto{\pgfqpoint{2.006657in}{1.075760in}}%
\pgfpathlineto{\pgfqpoint{2.019719in}{1.066437in}}%
\pgfpathlineto{\pgfqpoint{2.034760in}{1.058870in}}%
\pgfpathlineto{\pgfqpoint{2.038328in}{1.052811in}}%
\pgfpathlineto{\pgfqpoint{2.046020in}{1.049946in}}%
\pgfpathlineto{\pgfqpoint{2.046326in}{1.041904in}}%
\pgfpathlineto{\pgfqpoint{2.052019in}{1.031690in}}%
\pgfpathlineto{\pgfqpoint{2.048069in}{1.022616in}}%
\pgfpathlineto{\pgfqpoint{2.055801in}{1.008178in}}%
\pgfpathlineto{\pgfqpoint{2.061849in}{1.011045in}}%
\pgfpathlineto{\pgfqpoint{2.069266in}{1.005020in}}%
\pgfpathlineto{\pgfqpoint{2.066773in}{0.978188in}}%
\pgfpathlineto{\pgfqpoint{2.058030in}{0.979890in}}%
\pgfpathlineto{\pgfqpoint{2.056343in}{0.972329in}}%
\pgfpathclose%
\pgfusepath{fill}%
\end{pgfscope}%
\begin{pgfscope}%
\pgfpathrectangle{\pgfqpoint{0.100000in}{0.100000in}}{\pgfqpoint{2.989028in}{1.913466in}}%
\pgfusepath{clip}%
\pgfsetbuttcap%
\pgfsetmiterjoin%
\definecolor{currentfill}{rgb}{0.909650,0.963860,0.608074}%
\pgfsetfillcolor{currentfill}%
\pgfsetlinewidth{0.000000pt}%
\definecolor{currentstroke}{rgb}{0.000000,0.000000,0.000000}%
\pgfsetstrokecolor{currentstroke}%
\pgfsetstrokeopacity{0.000000}%
\pgfsetdash{}{0pt}%
\pgfpathmoveto{\pgfqpoint{0.715525in}{0.978113in}}%
\pgfpathlineto{\pgfqpoint{0.721856in}{0.991592in}}%
\pgfpathlineto{\pgfqpoint{0.720123in}{1.011825in}}%
\pgfpathlineto{\pgfqpoint{0.723155in}{1.017578in}}%
\pgfpathlineto{\pgfqpoint{0.723454in}{1.042780in}}%
\pgfpathlineto{\pgfqpoint{0.726328in}{1.050037in}}%
\pgfpathlineto{\pgfqpoint{0.736451in}{1.051167in}}%
\pgfpathlineto{\pgfqpoint{0.745846in}{1.047945in}}%
\pgfpathlineto{\pgfqpoint{0.749938in}{1.039072in}}%
\pgfpathlineto{\pgfqpoint{0.755674in}{1.039412in}}%
\pgfpathlineto{\pgfqpoint{0.762801in}{1.049576in}}%
\pgfpathlineto{\pgfqpoint{0.773103in}{1.099678in}}%
\pgfpathlineto{\pgfqpoint{0.831724in}{1.087575in}}%
\pgfpathlineto{\pgfqpoint{0.865746in}{1.080865in}}%
\pgfpathlineto{\pgfqpoint{0.956091in}{1.064984in}}%
\pgfpathlineto{\pgfqpoint{0.981086in}{1.059964in}}%
\pgfpathlineto{\pgfqpoint{1.030137in}{1.052362in}}%
\pgfpathlineto{\pgfqpoint{1.020079in}{0.987601in}}%
\pgfpathlineto{\pgfqpoint{1.009619in}{0.920049in}}%
\pgfpathlineto{\pgfqpoint{0.997573in}{0.844052in}}%
\pgfpathlineto{\pgfqpoint{0.984025in}{0.756815in}}%
\pgfpathlineto{\pgfqpoint{0.973080in}{0.685166in}}%
\pgfpathlineto{\pgfqpoint{0.894687in}{0.697614in}}%
\pgfpathlineto{\pgfqpoint{0.860244in}{0.703513in}}%
\pgfpathlineto{\pgfqpoint{0.844857in}{0.712725in}}%
\pgfpathlineto{\pgfqpoint{0.744535in}{0.773254in}}%
\pgfpathlineto{\pgfqpoint{0.669248in}{0.819189in}}%
\pgfpathlineto{\pgfqpoint{0.671756in}{0.827331in}}%
\pgfpathlineto{\pgfqpoint{0.677975in}{0.833027in}}%
\pgfpathlineto{\pgfqpoint{0.684444in}{0.831972in}}%
\pgfpathlineto{\pgfqpoint{0.693826in}{0.837954in}}%
\pgfpathlineto{\pgfqpoint{0.695303in}{0.846552in}}%
\pgfpathlineto{\pgfqpoint{0.683861in}{0.856982in}}%
\pgfpathlineto{\pgfqpoint{0.691998in}{0.877004in}}%
\pgfpathlineto{\pgfqpoint{0.700180in}{0.884704in}}%
\pgfpathlineto{\pgfqpoint{0.704063in}{0.893839in}}%
\pgfpathlineto{\pgfqpoint{0.706455in}{0.910589in}}%
\pgfpathlineto{\pgfqpoint{0.714127in}{0.918174in}}%
\pgfpathlineto{\pgfqpoint{0.730230in}{0.925750in}}%
\pgfpathlineto{\pgfqpoint{0.729874in}{0.932426in}}%
\pgfpathlineto{\pgfqpoint{0.720881in}{0.940711in}}%
\pgfpathlineto{\pgfqpoint{0.719656in}{0.957789in}}%
\pgfpathlineto{\pgfqpoint{0.713464in}{0.970274in}}%
\pgfpathlineto{\pgfqpoint{0.715525in}{0.978113in}}%
\pgfpathclose%
\pgfusepath{fill}%
\end{pgfscope}%
\begin{pgfscope}%
\pgfpathrectangle{\pgfqpoint{0.100000in}{0.100000in}}{\pgfqpoint{2.989028in}{1.913466in}}%
\pgfusepath{clip}%
\pgfsetbuttcap%
\pgfsetmiterjoin%
\definecolor{currentfill}{rgb}{0.442445,0.777393,0.646444}%
\pgfsetfillcolor{currentfill}%
\pgfsetlinewidth{0.000000pt}%
\definecolor{currentstroke}{rgb}{0.000000,0.000000,0.000000}%
\pgfsetstrokecolor{currentstroke}%
\pgfsetstrokeopacity{0.000000}%
\pgfsetdash{}{0pt}%
\pgfpathmoveto{\pgfqpoint{1.790948in}{0.774957in}}%
\pgfpathlineto{\pgfqpoint{1.776416in}{0.779451in}}%
\pgfpathlineto{\pgfqpoint{1.757795in}{0.793711in}}%
\pgfpathlineto{\pgfqpoint{1.749524in}{0.796776in}}%
\pgfpathlineto{\pgfqpoint{1.744268in}{0.790617in}}%
\pgfpathlineto{\pgfqpoint{1.737631in}{0.790306in}}%
\pgfpathlineto{\pgfqpoint{1.729234in}{0.795535in}}%
\pgfpathlineto{\pgfqpoint{1.716057in}{0.788835in}}%
\pgfpathlineto{\pgfqpoint{1.711367in}{0.792135in}}%
\pgfpathlineto{\pgfqpoint{1.698401in}{0.784318in}}%
\pgfpathlineto{\pgfqpoint{1.684722in}{0.785761in}}%
\pgfpathlineto{\pgfqpoint{1.674172in}{0.790840in}}%
\pgfpathlineto{\pgfqpoint{1.666783in}{0.788928in}}%
\pgfpathlineto{\pgfqpoint{1.654956in}{0.796112in}}%
\pgfpathlineto{\pgfqpoint{1.648377in}{0.783442in}}%
\pgfpathlineto{\pgfqpoint{1.641649in}{0.794338in}}%
\pgfpathlineto{\pgfqpoint{1.628360in}{0.790111in}}%
\pgfpathlineto{\pgfqpoint{1.616736in}{0.800460in}}%
\pgfpathlineto{\pgfqpoint{1.606579in}{0.792118in}}%
\pgfpathlineto{\pgfqpoint{1.601493in}{0.799784in}}%
\pgfpathlineto{\pgfqpoint{1.594161in}{0.802273in}}%
\pgfpathlineto{\pgfqpoint{1.589696in}{0.809664in}}%
\pgfpathlineto{\pgfqpoint{1.580106in}{0.811767in}}%
\pgfpathlineto{\pgfqpoint{1.574573in}{0.806204in}}%
\pgfpathlineto{\pgfqpoint{1.565171in}{0.813414in}}%
\pgfpathlineto{\pgfqpoint{1.560789in}{0.811767in}}%
\pgfpathlineto{\pgfqpoint{1.545207in}{0.817609in}}%
\pgfpathlineto{\pgfqpoint{1.535448in}{0.818261in}}%
\pgfpathlineto{\pgfqpoint{1.531080in}{0.830667in}}%
\pgfpathlineto{\pgfqpoint{1.523256in}{0.829122in}}%
\pgfpathlineto{\pgfqpoint{1.514296in}{0.832183in}}%
\pgfpathlineto{\pgfqpoint{1.508395in}{0.830414in}}%
\pgfpathlineto{\pgfqpoint{1.495741in}{0.844347in}}%
\pgfpathlineto{\pgfqpoint{1.492215in}{0.843436in}}%
\pgfpathlineto{\pgfqpoint{1.495417in}{0.899940in}}%
\pgfpathlineto{\pgfqpoint{1.498906in}{0.969876in}}%
\pgfpathlineto{\pgfqpoint{1.441639in}{0.973080in}}%
\pgfpathlineto{\pgfqpoint{1.385128in}{0.977352in}}%
\pgfpathlineto{\pgfqpoint{1.341469in}{0.981143in}}%
\pgfpathlineto{\pgfqpoint{1.344368in}{1.014183in}}%
\pgfpathlineto{\pgfqpoint{1.394404in}{1.009408in}}%
\pgfpathlineto{\pgfqpoint{1.451155in}{1.005849in}}%
\pgfpathlineto{\pgfqpoint{1.530191in}{1.001189in}}%
\pgfpathlineto{\pgfqpoint{1.577105in}{0.999379in}}%
\pgfpathlineto{\pgfqpoint{1.633309in}{0.997396in}}%
\pgfpathlineto{\pgfqpoint{1.709656in}{0.996060in}}%
\pgfpathlineto{\pgfqpoint{1.782158in}{0.995725in}}%
\pgfpathlineto{\pgfqpoint{1.782462in}{0.962463in}}%
\pgfpathlineto{\pgfqpoint{1.786531in}{0.937404in}}%
\pgfpathlineto{\pgfqpoint{1.792853in}{0.891120in}}%
\pgfpathlineto{\pgfqpoint{1.791549in}{0.812079in}}%
\pgfpathlineto{\pgfqpoint{1.790948in}{0.774957in}}%
\pgfpathclose%
\pgfusepath{fill}%
\end{pgfscope}%
\begin{pgfscope}%
\pgfpathrectangle{\pgfqpoint{0.100000in}{0.100000in}}{\pgfqpoint{2.989028in}{1.913466in}}%
\pgfusepath{clip}%
\pgfsetbuttcap%
\pgfsetmiterjoin%
\definecolor{currentfill}{rgb}{0.932718,0.973087,0.644060}%
\pgfsetfillcolor{currentfill}%
\pgfsetlinewidth{0.000000pt}%
\definecolor{currentstroke}{rgb}{0.000000,0.000000,0.000000}%
\pgfsetstrokecolor{currentstroke}%
\pgfsetstrokeopacity{0.000000}%
\pgfsetdash{}{0pt}%
\pgfpathmoveto{\pgfqpoint{2.335069in}{0.897708in}}%
\pgfpathlineto{\pgfqpoint{2.335112in}{0.912353in}}%
\pgfpathlineto{\pgfqpoint{2.347839in}{0.917977in}}%
\pgfpathlineto{\pgfqpoint{2.348372in}{0.926967in}}%
\pgfpathlineto{\pgfqpoint{2.359767in}{0.937949in}}%
\pgfpathlineto{\pgfqpoint{2.374011in}{0.940198in}}%
\pgfpathlineto{\pgfqpoint{2.386533in}{0.952301in}}%
\pgfpathlineto{\pgfqpoint{2.399610in}{0.957711in}}%
\pgfpathlineto{\pgfqpoint{2.409425in}{0.973951in}}%
\pgfpathlineto{\pgfqpoint{2.418364in}{0.972772in}}%
\pgfpathlineto{\pgfqpoint{2.437256in}{0.987570in}}%
\pgfpathlineto{\pgfqpoint{2.447232in}{0.987849in}}%
\pgfpathlineto{\pgfqpoint{2.451445in}{0.999123in}}%
\pgfpathlineto{\pgfqpoint{2.459375in}{1.006973in}}%
\pgfpathlineto{\pgfqpoint{2.461189in}{1.020348in}}%
\pgfpathlineto{\pgfqpoint{2.478251in}{1.021788in}}%
\pgfpathlineto{\pgfqpoint{2.547229in}{1.030898in}}%
\pgfpathlineto{\pgfqpoint{2.628447in}{1.044890in}}%
\pgfpathlineto{\pgfqpoint{2.687189in}{1.056913in}}%
\pgfpathlineto{\pgfqpoint{2.752865in}{1.070536in}}%
\pgfpathlineto{\pgfqpoint{2.771631in}{1.044770in}}%
\pgfpathlineto{\pgfqpoint{2.761466in}{1.046415in}}%
\pgfpathlineto{\pgfqpoint{2.748417in}{1.043236in}}%
\pgfpathlineto{\pgfqpoint{2.735559in}{1.030105in}}%
\pgfpathlineto{\pgfqpoint{2.726318in}{1.031050in}}%
\pgfpathlineto{\pgfqpoint{2.725149in}{1.023251in}}%
\pgfpathlineto{\pgfqpoint{2.741852in}{1.029418in}}%
\pgfpathlineto{\pgfqpoint{2.758660in}{1.031961in}}%
\pgfpathlineto{\pgfqpoint{2.764901in}{1.028577in}}%
\pgfpathlineto{\pgfqpoint{2.773300in}{1.032435in}}%
\pgfpathlineto{\pgfqpoint{2.777634in}{1.029732in}}%
\pgfpathlineto{\pgfqpoint{2.781470in}{1.016862in}}%
\pgfpathlineto{\pgfqpoint{2.773487in}{1.012876in}}%
\pgfpathlineto{\pgfqpoint{2.768008in}{0.997182in}}%
\pgfpathlineto{\pgfqpoint{2.762263in}{0.991072in}}%
\pgfpathlineto{\pgfqpoint{2.744653in}{0.992409in}}%
\pgfpathlineto{\pgfqpoint{2.745600in}{1.001624in}}%
\pgfpathlineto{\pgfqpoint{2.735982in}{0.997599in}}%
\pgfpathlineto{\pgfqpoint{2.733890in}{0.989913in}}%
\pgfpathlineto{\pgfqpoint{2.740483in}{0.981952in}}%
\pgfpathlineto{\pgfqpoint{2.742722in}{0.968440in}}%
\pgfpathlineto{\pgfqpoint{2.731935in}{0.960741in}}%
\pgfpathlineto{\pgfqpoint{2.743601in}{0.958032in}}%
\pgfpathlineto{\pgfqpoint{2.748881in}{0.963694in}}%
\pgfpathlineto{\pgfqpoint{2.760553in}{0.964383in}}%
\pgfpathlineto{\pgfqpoint{2.754555in}{0.952158in}}%
\pgfpathlineto{\pgfqpoint{2.747208in}{0.946268in}}%
\pgfpathlineto{\pgfqpoint{2.724984in}{0.940370in}}%
\pgfpathlineto{\pgfqpoint{2.702166in}{0.919675in}}%
\pgfpathlineto{\pgfqpoint{2.688096in}{0.899284in}}%
\pgfpathlineto{\pgfqpoint{2.688013in}{0.891003in}}%
\pgfpathlineto{\pgfqpoint{2.682393in}{0.879575in}}%
\pgfpathlineto{\pgfqpoint{2.653548in}{0.872057in}}%
\pgfpathlineto{\pgfqpoint{2.584952in}{0.921230in}}%
\pgfpathlineto{\pgfqpoint{2.524543in}{0.912354in}}%
\pgfpathlineto{\pgfqpoint{2.524037in}{0.920550in}}%
\pgfpathlineto{\pgfqpoint{2.514855in}{0.929779in}}%
\pgfpathlineto{\pgfqpoint{2.507904in}{0.932038in}}%
\pgfpathlineto{\pgfqpoint{2.442269in}{0.925218in}}%
\pgfpathlineto{\pgfqpoint{2.427183in}{0.920102in}}%
\pgfpathlineto{\pgfqpoint{2.399943in}{0.906551in}}%
\pgfpathlineto{\pgfqpoint{2.376402in}{0.902801in}}%
\pgfpathlineto{\pgfqpoint{2.335069in}{0.897708in}}%
\pgfpathclose%
\pgfusepath{fill}%
\end{pgfscope}%
\begin{pgfscope}%
\pgfpathrectangle{\pgfqpoint{0.100000in}{0.100000in}}{\pgfqpoint{2.989028in}{1.913466in}}%
\pgfusepath{clip}%
\pgfsetbuttcap%
\pgfsetmiterjoin%
\definecolor{currentfill}{rgb}{0.955786,0.982314,0.680046}%
\pgfsetfillcolor{currentfill}%
\pgfsetlinewidth{0.000000pt}%
\definecolor{currentstroke}{rgb}{0.000000,0.000000,0.000000}%
\pgfsetstrokecolor{currentstroke}%
\pgfsetstrokeopacity{0.000000}%
\pgfsetdash{}{0pt}%
\pgfpathmoveto{\pgfqpoint{2.335069in}{0.897708in}}%
\pgfpathlineto{\pgfqpoint{2.266375in}{0.890163in}}%
\pgfpathlineto{\pgfqpoint{2.203533in}{0.884512in}}%
\pgfpathlineto{\pgfqpoint{2.138945in}{0.880335in}}%
\pgfpathlineto{\pgfqpoint{2.127858in}{0.878728in}}%
\pgfpathlineto{\pgfqpoint{2.084319in}{0.875379in}}%
\pgfpathlineto{\pgfqpoint{2.014584in}{0.871310in}}%
\pgfpathlineto{\pgfqpoint{2.026990in}{0.881608in}}%
\pgfpathlineto{\pgfqpoint{2.024159in}{0.892449in}}%
\pgfpathlineto{\pgfqpoint{2.026902in}{0.905590in}}%
\pgfpathlineto{\pgfqpoint{2.031092in}{0.911777in}}%
\pgfpathlineto{\pgfqpoint{2.030934in}{0.920384in}}%
\pgfpathlineto{\pgfqpoint{2.042090in}{0.925797in}}%
\pgfpathlineto{\pgfqpoint{2.042109in}{0.938464in}}%
\pgfpathlineto{\pgfqpoint{2.043642in}{0.953066in}}%
\pgfpathlineto{\pgfqpoint{2.051421in}{0.962397in}}%
\pgfpathlineto{\pgfqpoint{2.050131in}{0.971655in}}%
\pgfpathlineto{\pgfqpoint{2.053146in}{0.971866in}}%
\pgfpathlineto{\pgfqpoint{2.056343in}{0.972329in}}%
\pgfpathlineto{\pgfqpoint{2.128480in}{0.976914in}}%
\pgfpathlineto{\pgfqpoint{2.126065in}{0.988764in}}%
\pgfpathlineto{\pgfqpoint{2.137674in}{0.987156in}}%
\pgfpathlineto{\pgfqpoint{2.207938in}{0.994107in}}%
\pgfpathlineto{\pgfqpoint{2.264337in}{0.997299in}}%
\pgfpathlineto{\pgfqpoint{2.298708in}{1.000190in}}%
\pgfpathlineto{\pgfqpoint{2.355622in}{1.005586in}}%
\pgfpathlineto{\pgfqpoint{2.356571in}{1.006660in}}%
\pgfpathlineto{\pgfqpoint{2.380888in}{1.009068in}}%
\pgfpathlineto{\pgfqpoint{2.461189in}{1.020348in}}%
\pgfpathlineto{\pgfqpoint{2.459375in}{1.006973in}}%
\pgfpathlineto{\pgfqpoint{2.451445in}{0.999123in}}%
\pgfpathlineto{\pgfqpoint{2.447232in}{0.987849in}}%
\pgfpathlineto{\pgfqpoint{2.437256in}{0.987570in}}%
\pgfpathlineto{\pgfqpoint{2.418364in}{0.972772in}}%
\pgfpathlineto{\pgfqpoint{2.409425in}{0.973951in}}%
\pgfpathlineto{\pgfqpoint{2.399610in}{0.957711in}}%
\pgfpathlineto{\pgfqpoint{2.386533in}{0.952301in}}%
\pgfpathlineto{\pgfqpoint{2.374011in}{0.940198in}}%
\pgfpathlineto{\pgfqpoint{2.359767in}{0.937949in}}%
\pgfpathlineto{\pgfqpoint{2.348372in}{0.926967in}}%
\pgfpathlineto{\pgfqpoint{2.347839in}{0.917977in}}%
\pgfpathlineto{\pgfqpoint{2.335112in}{0.912353in}}%
\pgfpathlineto{\pgfqpoint{2.335069in}{0.897708in}}%
\pgfpathclose%
\pgfusepath{fill}%
\end{pgfscope}%
\begin{pgfscope}%
\pgfpathrectangle{\pgfqpoint{0.100000in}{0.100000in}}{\pgfqpoint{2.989028in}{1.913466in}}%
\pgfusepath{clip}%
\pgfsetbuttcap%
\pgfsetmiterjoin%
\definecolor{currentfill}{rgb}{0.384006,0.742945,0.654441}%
\pgfsetfillcolor{currentfill}%
\pgfsetlinewidth{0.000000pt}%
\definecolor{currentstroke}{rgb}{0.000000,0.000000,0.000000}%
\pgfsetstrokecolor{currentstroke}%
\pgfsetstrokeopacity{0.000000}%
\pgfsetdash{}{0pt}%
\pgfpathmoveto{\pgfqpoint{1.116758in}{0.694333in}}%
\pgfpathlineto{\pgfqpoint{1.111859in}{0.702205in}}%
\pgfpathlineto{\pgfqpoint{1.113886in}{0.708987in}}%
\pgfpathlineto{\pgfqpoint{1.148324in}{0.704715in}}%
\pgfpathlineto{\pgfqpoint{1.212359in}{0.697442in}}%
\pgfpathlineto{\pgfqpoint{1.258660in}{0.692892in}}%
\pgfpathlineto{\pgfqpoint{1.312135in}{0.687496in}}%
\pgfpathlineto{\pgfqpoint{1.315044in}{0.721225in}}%
\pgfpathlineto{\pgfqpoint{1.324140in}{0.807015in}}%
\pgfpathlineto{\pgfqpoint{1.330041in}{0.867115in}}%
\pgfpathlineto{\pgfqpoint{1.335142in}{0.924525in}}%
\pgfpathlineto{\pgfqpoint{1.339896in}{0.981231in}}%
\pgfpathlineto{\pgfqpoint{1.341469in}{0.981143in}}%
\pgfpathlineto{\pgfqpoint{1.385128in}{0.977352in}}%
\pgfpathlineto{\pgfqpoint{1.441639in}{0.973080in}}%
\pgfpathlineto{\pgfqpoint{1.498906in}{0.969876in}}%
\pgfpathlineto{\pgfqpoint{1.495417in}{0.899940in}}%
\pgfpathlineto{\pgfqpoint{1.492215in}{0.843436in}}%
\pgfpathlineto{\pgfqpoint{1.495741in}{0.844347in}}%
\pgfpathlineto{\pgfqpoint{1.508395in}{0.830414in}}%
\pgfpathlineto{\pgfqpoint{1.514296in}{0.832183in}}%
\pgfpathlineto{\pgfqpoint{1.523256in}{0.829122in}}%
\pgfpathlineto{\pgfqpoint{1.531080in}{0.830667in}}%
\pgfpathlineto{\pgfqpoint{1.535448in}{0.818261in}}%
\pgfpathlineto{\pgfqpoint{1.545207in}{0.817609in}}%
\pgfpathlineto{\pgfqpoint{1.560789in}{0.811767in}}%
\pgfpathlineto{\pgfqpoint{1.565171in}{0.813414in}}%
\pgfpathlineto{\pgfqpoint{1.574573in}{0.806204in}}%
\pgfpathlineto{\pgfqpoint{1.580106in}{0.811767in}}%
\pgfpathlineto{\pgfqpoint{1.589696in}{0.809664in}}%
\pgfpathlineto{\pgfqpoint{1.594161in}{0.802273in}}%
\pgfpathlineto{\pgfqpoint{1.601493in}{0.799784in}}%
\pgfpathlineto{\pgfqpoint{1.606579in}{0.792118in}}%
\pgfpathlineto{\pgfqpoint{1.616736in}{0.800460in}}%
\pgfpathlineto{\pgfqpoint{1.628360in}{0.790111in}}%
\pgfpathlineto{\pgfqpoint{1.641649in}{0.794338in}}%
\pgfpathlineto{\pgfqpoint{1.648377in}{0.783442in}}%
\pgfpathlineto{\pgfqpoint{1.654956in}{0.796112in}}%
\pgfpathlineto{\pgfqpoint{1.666783in}{0.788928in}}%
\pgfpathlineto{\pgfqpoint{1.674172in}{0.790840in}}%
\pgfpathlineto{\pgfqpoint{1.684722in}{0.785761in}}%
\pgfpathlineto{\pgfqpoint{1.698401in}{0.784318in}}%
\pgfpathlineto{\pgfqpoint{1.711367in}{0.792135in}}%
\pgfpathlineto{\pgfqpoint{1.716057in}{0.788835in}}%
\pgfpathlineto{\pgfqpoint{1.729234in}{0.795535in}}%
\pgfpathlineto{\pgfqpoint{1.737631in}{0.790306in}}%
\pgfpathlineto{\pgfqpoint{1.744268in}{0.790617in}}%
\pgfpathlineto{\pgfqpoint{1.749524in}{0.796776in}}%
\pgfpathlineto{\pgfqpoint{1.757795in}{0.793711in}}%
\pgfpathlineto{\pgfqpoint{1.776416in}{0.779451in}}%
\pgfpathlineto{\pgfqpoint{1.790948in}{0.774957in}}%
\pgfpathlineto{\pgfqpoint{1.796775in}{0.769451in}}%
\pgfpathlineto{\pgfqpoint{1.804070in}{0.772449in}}%
\pgfpathlineto{\pgfqpoint{1.815123in}{0.770153in}}%
\pgfpathlineto{\pgfqpoint{1.815340in}{0.735108in}}%
\pgfpathlineto{\pgfqpoint{1.816262in}{0.667326in}}%
\pgfpathlineto{\pgfqpoint{1.823939in}{0.660813in}}%
\pgfpathlineto{\pgfqpoint{1.830092in}{0.648811in}}%
\pgfpathlineto{\pgfqpoint{1.827929in}{0.640781in}}%
\pgfpathlineto{\pgfqpoint{1.832761in}{0.637381in}}%
\pgfpathlineto{\pgfqpoint{1.836688in}{0.622574in}}%
\pgfpathlineto{\pgfqpoint{1.844209in}{0.614644in}}%
\pgfpathlineto{\pgfqpoint{1.845894in}{0.597808in}}%
\pgfpathlineto{\pgfqpoint{1.844586in}{0.590680in}}%
\pgfpathlineto{\pgfqpoint{1.834363in}{0.571815in}}%
\pgfpathlineto{\pgfqpoint{1.833187in}{0.558251in}}%
\pgfpathlineto{\pgfqpoint{1.836655in}{0.555420in}}%
\pgfpathlineto{\pgfqpoint{1.836956in}{0.543537in}}%
\pgfpathlineto{\pgfqpoint{1.833432in}{0.536068in}}%
\pgfpathlineto{\pgfqpoint{1.827950in}{0.533255in}}%
\pgfpathlineto{\pgfqpoint{1.822582in}{0.523496in}}%
\pgfpathlineto{\pgfqpoint{1.829424in}{0.514043in}}%
\pgfpathlineto{\pgfqpoint{1.816144in}{0.513857in}}%
\pgfpathlineto{\pgfqpoint{1.780636in}{0.497611in}}%
\pgfpathlineto{\pgfqpoint{1.787415in}{0.507333in}}%
\pgfpathlineto{\pgfqpoint{1.779206in}{0.512575in}}%
\pgfpathlineto{\pgfqpoint{1.777490in}{0.521488in}}%
\pgfpathlineto{\pgfqpoint{1.761468in}{0.505964in}}%
\pgfpathlineto{\pgfqpoint{1.768579in}{0.495354in}}%
\pgfpathlineto{\pgfqpoint{1.758435in}{0.481844in}}%
\pgfpathlineto{\pgfqpoint{1.752976in}{0.482126in}}%
\pgfpathlineto{\pgfqpoint{1.747841in}{0.467413in}}%
\pgfpathlineto{\pgfqpoint{1.731600in}{0.455840in}}%
\pgfpathlineto{\pgfqpoint{1.716427in}{0.451669in}}%
\pgfpathlineto{\pgfqpoint{1.689964in}{0.440819in}}%
\pgfpathlineto{\pgfqpoint{1.687222in}{0.446871in}}%
\pgfpathlineto{\pgfqpoint{1.670737in}{0.441300in}}%
\pgfpathlineto{\pgfqpoint{1.680876in}{0.431814in}}%
\pgfpathlineto{\pgfqpoint{1.664883in}{0.423573in}}%
\pgfpathlineto{\pgfqpoint{1.647633in}{0.411131in}}%
\pgfpathlineto{\pgfqpoint{1.643490in}{0.416915in}}%
\pgfpathlineto{\pgfqpoint{1.629373in}{0.408246in}}%
\pgfpathlineto{\pgfqpoint{1.643210in}{0.404954in}}%
\pgfpathlineto{\pgfqpoint{1.632777in}{0.391324in}}%
\pgfpathlineto{\pgfqpoint{1.627685in}{0.395380in}}%
\pgfpathlineto{\pgfqpoint{1.620845in}{0.388865in}}%
\pgfpathlineto{\pgfqpoint{1.629369in}{0.383174in}}%
\pgfpathlineto{\pgfqpoint{1.624329in}{0.374856in}}%
\pgfpathlineto{\pgfqpoint{1.618135in}{0.355164in}}%
\pgfpathlineto{\pgfqpoint{1.609200in}{0.336199in}}%
\pgfpathlineto{\pgfqpoint{1.613211in}{0.323767in}}%
\pgfpathlineto{\pgfqpoint{1.620029in}{0.294458in}}%
\pgfpathlineto{\pgfqpoint{1.620643in}{0.282596in}}%
\pgfpathlineto{\pgfqpoint{1.631178in}{0.267038in}}%
\pgfpathlineto{\pgfqpoint{1.623054in}{0.267947in}}%
\pgfpathlineto{\pgfqpoint{1.615164in}{0.260098in}}%
\pgfpathlineto{\pgfqpoint{1.605923in}{0.266267in}}%
\pgfpathlineto{\pgfqpoint{1.602605in}{0.272421in}}%
\pgfpathlineto{\pgfqpoint{1.589453in}{0.275284in}}%
\pgfpathlineto{\pgfqpoint{1.569363in}{0.275635in}}%
\pgfpathlineto{\pgfqpoint{1.554561in}{0.287289in}}%
\pgfpathlineto{\pgfqpoint{1.541120in}{0.289236in}}%
\pgfpathlineto{\pgfqpoint{1.532966in}{0.298497in}}%
\pgfpathlineto{\pgfqpoint{1.515889in}{0.302225in}}%
\pgfpathlineto{\pgfqpoint{1.506552in}{0.332004in}}%
\pgfpathlineto{\pgfqpoint{1.497004in}{0.343934in}}%
\pgfpathlineto{\pgfqpoint{1.498626in}{0.355270in}}%
\pgfpathlineto{\pgfqpoint{1.492706in}{0.363554in}}%
\pgfpathlineto{\pgfqpoint{1.496421in}{0.374877in}}%
\pgfpathlineto{\pgfqpoint{1.493356in}{0.383175in}}%
\pgfpathlineto{\pgfqpoint{1.483762in}{0.386934in}}%
\pgfpathlineto{\pgfqpoint{1.474791in}{0.396502in}}%
\pgfpathlineto{\pgfqpoint{1.471522in}{0.409312in}}%
\pgfpathlineto{\pgfqpoint{1.463033in}{0.420954in}}%
\pgfpathlineto{\pgfqpoint{1.451735in}{0.430012in}}%
\pgfpathlineto{\pgfqpoint{1.441558in}{0.455996in}}%
\pgfpathlineto{\pgfqpoint{1.433340in}{0.484413in}}%
\pgfpathlineto{\pgfqpoint{1.426597in}{0.495650in}}%
\pgfpathlineto{\pgfqpoint{1.414908in}{0.505109in}}%
\pgfpathlineto{\pgfqpoint{1.411987in}{0.511975in}}%
\pgfpathlineto{\pgfqpoint{1.401042in}{0.516238in}}%
\pgfpathlineto{\pgfqpoint{1.390239in}{0.534431in}}%
\pgfpathlineto{\pgfqpoint{1.380374in}{0.533035in}}%
\pgfpathlineto{\pgfqpoint{1.370356in}{0.537579in}}%
\pgfpathlineto{\pgfqpoint{1.356156in}{0.536700in}}%
\pgfpathlineto{\pgfqpoint{1.341731in}{0.544199in}}%
\pgfpathlineto{\pgfqpoint{1.337663in}{0.537067in}}%
\pgfpathlineto{\pgfqpoint{1.320804in}{0.536889in}}%
\pgfpathlineto{\pgfqpoint{1.312222in}{0.523370in}}%
\pgfpathlineto{\pgfqpoint{1.304757in}{0.506631in}}%
\pgfpathlineto{\pgfqpoint{1.288883in}{0.488663in}}%
\pgfpathlineto{\pgfqpoint{1.270899in}{0.496548in}}%
\pgfpathlineto{\pgfqpoint{1.268347in}{0.501760in}}%
\pgfpathlineto{\pgfqpoint{1.257418in}{0.505734in}}%
\pgfpathlineto{\pgfqpoint{1.254061in}{0.511171in}}%
\pgfpathlineto{\pgfqpoint{1.239552in}{0.516660in}}%
\pgfpathlineto{\pgfqpoint{1.231436in}{0.527879in}}%
\pgfpathlineto{\pgfqpoint{1.221957in}{0.533292in}}%
\pgfpathlineto{\pgfqpoint{1.213802in}{0.542738in}}%
\pgfpathlineto{\pgfqpoint{1.207448in}{0.558735in}}%
\pgfpathlineto{\pgfqpoint{1.208159in}{0.580595in}}%
\pgfpathlineto{\pgfqpoint{1.200706in}{0.591641in}}%
\pgfpathlineto{\pgfqpoint{1.199848in}{0.603605in}}%
\pgfpathlineto{\pgfqpoint{1.183312in}{0.621518in}}%
\pgfpathlineto{\pgfqpoint{1.173690in}{0.625342in}}%
\pgfpathlineto{\pgfqpoint{1.163479in}{0.642057in}}%
\pgfpathlineto{\pgfqpoint{1.154778in}{0.648698in}}%
\pgfpathlineto{\pgfqpoint{1.143723in}{0.664869in}}%
\pgfpathlineto{\pgfqpoint{1.132411in}{0.671867in}}%
\pgfpathlineto{\pgfqpoint{1.125003in}{0.689800in}}%
\pgfpathlineto{\pgfqpoint{1.116758in}{0.694333in}}%
\pgfpathclose%
\pgfusepath{fill}%
\end{pgfscope}%
\begin{pgfscope}%
\pgfpathrectangle{\pgfqpoint{0.100000in}{0.100000in}}{\pgfqpoint{2.989028in}{1.913466in}}%
\pgfusepath{clip}%
\pgfsetbuttcap%
\pgfsetmiterjoin%
\definecolor{currentfill}{rgb}{0.580392,0.831373,0.644444}%
\pgfsetfillcolor{currentfill}%
\pgfsetlinewidth{0.000000pt}%
\definecolor{currentstroke}{rgb}{0.000000,0.000000,0.000000}%
\pgfsetstrokecolor{currentstroke}%
\pgfsetstrokeopacity{0.000000}%
\pgfsetdash{}{0pt}%
\pgfpathmoveto{\pgfqpoint{1.030137in}{1.052362in}}%
\pgfpathlineto{\pgfqpoint{1.111744in}{1.040510in}}%
\pgfpathlineto{\pgfqpoint{1.143359in}{1.035639in}}%
\pgfpathlineto{\pgfqpoint{1.232477in}{1.025040in}}%
\pgfpathlineto{\pgfqpoint{1.292460in}{1.018783in}}%
\pgfpathlineto{\pgfqpoint{1.344368in}{1.014183in}}%
\pgfpathlineto{\pgfqpoint{1.341469in}{0.981143in}}%
\pgfpathlineto{\pgfqpoint{1.339896in}{0.981231in}}%
\pgfpathlineto{\pgfqpoint{1.335142in}{0.924525in}}%
\pgfpathlineto{\pgfqpoint{1.330041in}{0.867115in}}%
\pgfpathlineto{\pgfqpoint{1.324140in}{0.807015in}}%
\pgfpathlineto{\pgfqpoint{1.315044in}{0.721225in}}%
\pgfpathlineto{\pgfqpoint{1.312135in}{0.687496in}}%
\pgfpathlineto{\pgfqpoint{1.258660in}{0.692892in}}%
\pgfpathlineto{\pgfqpoint{1.212359in}{0.697442in}}%
\pgfpathlineto{\pgfqpoint{1.148324in}{0.704715in}}%
\pgfpathlineto{\pgfqpoint{1.113886in}{0.708987in}}%
\pgfpathlineto{\pgfqpoint{1.111859in}{0.702205in}}%
\pgfpathlineto{\pgfqpoint{1.116758in}{0.694333in}}%
\pgfpathlineto{\pgfqpoint{1.075370in}{0.699708in}}%
\pgfpathlineto{\pgfqpoint{1.024306in}{0.707037in}}%
\pgfpathlineto{\pgfqpoint{1.019659in}{0.678155in}}%
\pgfpathlineto{\pgfqpoint{0.973080in}{0.685166in}}%
\pgfpathlineto{\pgfqpoint{0.984025in}{0.756815in}}%
\pgfpathlineto{\pgfqpoint{0.997573in}{0.844052in}}%
\pgfpathlineto{\pgfqpoint{1.009619in}{0.920049in}}%
\pgfpathlineto{\pgfqpoint{1.020079in}{0.987601in}}%
\pgfpathlineto{\pgfqpoint{1.030137in}{1.052362in}}%
\pgfpathclose%
\pgfusepath{fill}%
\end{pgfscope}%
\begin{pgfscope}%
\pgfpathrectangle{\pgfqpoint{0.100000in}{0.100000in}}{\pgfqpoint{2.989028in}{1.913466in}}%
\pgfusepath{clip}%
\pgfsetbuttcap%
\pgfsetmiterjoin%
\definecolor{currentfill}{rgb}{0.982699,0.993080,0.722030}%
\pgfsetfillcolor{currentfill}%
\pgfsetlinewidth{0.000000pt}%
\definecolor{currentstroke}{rgb}{0.000000,0.000000,0.000000}%
\pgfsetstrokecolor{currentstroke}%
\pgfsetstrokeopacity{0.000000}%
\pgfsetdash{}{0pt}%
\pgfpathmoveto{\pgfqpoint{2.327687in}{0.632577in}}%
\pgfpathlineto{\pgfqpoint{2.249834in}{0.623930in}}%
\pgfpathlineto{\pgfqpoint{2.181207in}{0.618602in}}%
\pgfpathlineto{\pgfqpoint{2.180349in}{0.610192in}}%
\pgfpathlineto{\pgfqpoint{2.186652in}{0.602188in}}%
\pgfpathlineto{\pgfqpoint{2.194346in}{0.597485in}}%
\pgfpathlineto{\pgfqpoint{2.194230in}{0.585103in}}%
\pgfpathlineto{\pgfqpoint{2.192191in}{0.576831in}}%
\pgfpathlineto{\pgfqpoint{2.185400in}{0.570869in}}%
\pgfpathlineto{\pgfqpoint{2.175936in}{0.571500in}}%
\pgfpathlineto{\pgfqpoint{2.166987in}{0.578887in}}%
\pgfpathlineto{\pgfqpoint{2.165392in}{0.592032in}}%
\pgfpathlineto{\pgfqpoint{2.158727in}{0.599677in}}%
\pgfpathlineto{\pgfqpoint{2.154190in}{0.572315in}}%
\pgfpathlineto{\pgfqpoint{2.138775in}{0.574915in}}%
\pgfpathlineto{\pgfqpoint{2.133415in}{0.622761in}}%
\pgfpathlineto{\pgfqpoint{2.127610in}{0.673185in}}%
\pgfpathlineto{\pgfqpoint{2.129766in}{0.745935in}}%
\pgfpathlineto{\pgfqpoint{2.131104in}{0.795857in}}%
\pgfpathlineto{\pgfqpoint{2.133950in}{0.872025in}}%
\pgfpathlineto{\pgfqpoint{2.127858in}{0.878728in}}%
\pgfpathlineto{\pgfqpoint{2.138945in}{0.880335in}}%
\pgfpathlineto{\pgfqpoint{2.203533in}{0.884512in}}%
\pgfpathlineto{\pgfqpoint{2.266375in}{0.890163in}}%
\pgfpathlineto{\pgfqpoint{2.282908in}{0.832251in}}%
\pgfpathlineto{\pgfqpoint{2.294184in}{0.789743in}}%
\pgfpathlineto{\pgfqpoint{2.304291in}{0.754156in}}%
\pgfpathlineto{\pgfqpoint{2.310136in}{0.739826in}}%
\pgfpathlineto{\pgfqpoint{2.319340in}{0.726491in}}%
\pgfpathlineto{\pgfqpoint{2.317853in}{0.719705in}}%
\pgfpathlineto{\pgfqpoint{2.324428in}{0.716398in}}%
\pgfpathlineto{\pgfqpoint{2.316659in}{0.706111in}}%
\pgfpathlineto{\pgfqpoint{2.314034in}{0.687792in}}%
\pgfpathlineto{\pgfqpoint{2.321628in}{0.666365in}}%
\pgfpathlineto{\pgfqpoint{2.320573in}{0.644805in}}%
\pgfpathlineto{\pgfqpoint{2.327687in}{0.632577in}}%
\pgfpathclose%
\pgfusepath{fill}%
\end{pgfscope}%
\begin{pgfscope}%
\pgfpathrectangle{\pgfqpoint{0.100000in}{0.100000in}}{\pgfqpoint{2.989028in}{1.913466in}}%
\pgfusepath{clip}%
\pgfsetbuttcap%
\pgfsetmiterjoin%
\definecolor{currentfill}{rgb}{0.995617,0.855363,0.525721}%
\pgfsetfillcolor{currentfill}%
\pgfsetlinewidth{0.000000pt}%
\definecolor{currentstroke}{rgb}{0.000000,0.000000,0.000000}%
\pgfsetstrokecolor{currentstroke}%
\pgfsetstrokeopacity{0.000000}%
\pgfsetdash{}{0pt}%
\pgfpathmoveto{\pgfqpoint{2.138775in}{0.574915in}}%
\pgfpathlineto{\pgfqpoint{2.122957in}{0.570389in}}%
\pgfpathlineto{\pgfqpoint{2.111166in}{0.573295in}}%
\pgfpathlineto{\pgfqpoint{2.089268in}{0.566330in}}%
\pgfpathlineto{\pgfqpoint{2.072753in}{0.557360in}}%
\pgfpathlineto{\pgfqpoint{2.065977in}{0.573569in}}%
\pgfpathlineto{\pgfqpoint{2.058992in}{0.580363in}}%
\pgfpathlineto{\pgfqpoint{2.055622in}{0.588493in}}%
\pgfpathlineto{\pgfqpoint{2.060562in}{0.610498in}}%
\pgfpathlineto{\pgfqpoint{2.013761in}{0.607617in}}%
\pgfpathlineto{\pgfqpoint{1.953073in}{0.604986in}}%
\pgfpathlineto{\pgfqpoint{1.956684in}{0.610463in}}%
\pgfpathlineto{\pgfqpoint{1.952221in}{0.620808in}}%
\pgfpathlineto{\pgfqpoint{1.956668in}{0.622916in}}%
\pgfpathlineto{\pgfqpoint{1.955681in}{0.632856in}}%
\pgfpathlineto{\pgfqpoint{1.963432in}{0.642495in}}%
\pgfpathlineto{\pgfqpoint{1.967673in}{0.661207in}}%
\pgfpathlineto{\pgfqpoint{1.981859in}{0.673535in}}%
\pgfpathlineto{\pgfqpoint{1.976601in}{0.685508in}}%
\pgfpathlineto{\pgfqpoint{1.986660in}{0.692400in}}%
\pgfpathlineto{\pgfqpoint{1.977983in}{0.704877in}}%
\pgfpathlineto{\pgfqpoint{1.971703in}{0.733439in}}%
\pgfpathlineto{\pgfqpoint{1.974085in}{0.738625in}}%
\pgfpathlineto{\pgfqpoint{1.977845in}{0.748573in}}%
\pgfpathlineto{\pgfqpoint{1.974345in}{0.759012in}}%
\pgfpathlineto{\pgfqpoint{1.976057in}{0.763762in}}%
\pgfpathlineto{\pgfqpoint{1.969899in}{0.781706in}}%
\pgfpathlineto{\pgfqpoint{1.988077in}{0.813926in}}%
\pgfpathlineto{\pgfqpoint{1.988922in}{0.822501in}}%
\pgfpathlineto{\pgfqpoint{1.997678in}{0.828737in}}%
\pgfpathlineto{\pgfqpoint{2.007021in}{0.849323in}}%
\pgfpathlineto{\pgfqpoint{2.006577in}{0.857690in}}%
\pgfpathlineto{\pgfqpoint{2.018214in}{0.866243in}}%
\pgfpathlineto{\pgfqpoint{2.014584in}{0.871310in}}%
\pgfpathlineto{\pgfqpoint{2.084319in}{0.875379in}}%
\pgfpathlineto{\pgfqpoint{2.127858in}{0.878728in}}%
\pgfpathlineto{\pgfqpoint{2.133950in}{0.872025in}}%
\pgfpathlineto{\pgfqpoint{2.131104in}{0.795857in}}%
\pgfpathlineto{\pgfqpoint{2.129766in}{0.745935in}}%
\pgfpathlineto{\pgfqpoint{2.127610in}{0.673185in}}%
\pgfpathlineto{\pgfqpoint{2.133415in}{0.622761in}}%
\pgfpathlineto{\pgfqpoint{2.138775in}{0.574915in}}%
\pgfpathclose%
\pgfusepath{fill}%
\end{pgfscope}%
\begin{pgfscope}%
\pgfpathrectangle{\pgfqpoint{0.100000in}{0.100000in}}{\pgfqpoint{2.989028in}{1.913466in}}%
\pgfusepath{clip}%
\pgfsetbuttcap%
\pgfsetmiterjoin%
\definecolor{currentfill}{rgb}{0.982699,0.993080,0.722030}%
\pgfsetfillcolor{currentfill}%
\pgfsetlinewidth{0.000000pt}%
\definecolor{currentstroke}{rgb}{0.000000,0.000000,0.000000}%
\pgfsetstrokecolor{currentstroke}%
\pgfsetstrokeopacity{0.000000}%
\pgfsetdash{}{0pt}%
\pgfpathmoveto{\pgfqpoint{2.266375in}{0.890163in}}%
\pgfpathlineto{\pgfqpoint{2.335069in}{0.897708in}}%
\pgfpathlineto{\pgfqpoint{2.376402in}{0.902801in}}%
\pgfpathlineto{\pgfqpoint{2.399943in}{0.906551in}}%
\pgfpathlineto{\pgfqpoint{2.390192in}{0.891171in}}%
\pgfpathlineto{\pgfqpoint{2.390198in}{0.883919in}}%
\pgfpathlineto{\pgfqpoint{2.400235in}{0.880009in}}%
\pgfpathlineto{\pgfqpoint{2.407061in}{0.873695in}}%
\pgfpathlineto{\pgfqpoint{2.417377in}{0.872915in}}%
\pgfpathlineto{\pgfqpoint{2.427019in}{0.855143in}}%
\pgfpathlineto{\pgfqpoint{2.437478in}{0.842609in}}%
\pgfpathlineto{\pgfqpoint{2.456053in}{0.832075in}}%
\pgfpathlineto{\pgfqpoint{2.461332in}{0.823674in}}%
\pgfpathlineto{\pgfqpoint{2.477896in}{0.814587in}}%
\pgfpathlineto{\pgfqpoint{2.479142in}{0.807998in}}%
\pgfpathlineto{\pgfqpoint{2.492579in}{0.794896in}}%
\pgfpathlineto{\pgfqpoint{2.502872in}{0.791165in}}%
\pgfpathlineto{\pgfqpoint{2.510161in}{0.778793in}}%
\pgfpathlineto{\pgfqpoint{2.513387in}{0.764886in}}%
\pgfpathlineto{\pgfqpoint{2.524078in}{0.759511in}}%
\pgfpathlineto{\pgfqpoint{2.532685in}{0.744563in}}%
\pgfpathlineto{\pgfqpoint{2.535460in}{0.733570in}}%
\pgfpathlineto{\pgfqpoint{2.548073in}{0.728890in}}%
\pgfpathlineto{\pgfqpoint{2.537490in}{0.708626in}}%
\pgfpathlineto{\pgfqpoint{2.533507in}{0.696651in}}%
\pgfpathlineto{\pgfqpoint{2.536105in}{0.691037in}}%
\pgfpathlineto{\pgfqpoint{2.531568in}{0.681708in}}%
\pgfpathlineto{\pgfqpoint{2.529639in}{0.668810in}}%
\pgfpathlineto{\pgfqpoint{2.521570in}{0.666408in}}%
\pgfpathlineto{\pgfqpoint{2.525831in}{0.654599in}}%
\pgfpathlineto{\pgfqpoint{2.525559in}{0.639617in}}%
\pgfpathlineto{\pgfqpoint{2.511217in}{0.640178in}}%
\pgfpathlineto{\pgfqpoint{2.501225in}{0.642912in}}%
\pgfpathlineto{\pgfqpoint{2.496902in}{0.637830in}}%
\pgfpathlineto{\pgfqpoint{2.500128in}{0.625826in}}%
\pgfpathlineto{\pgfqpoint{2.499427in}{0.611870in}}%
\pgfpathlineto{\pgfqpoint{2.493123in}{0.610802in}}%
\pgfpathlineto{\pgfqpoint{2.488024in}{0.623836in}}%
\pgfpathlineto{\pgfqpoint{2.436103in}{0.620376in}}%
\pgfpathlineto{\pgfqpoint{2.370635in}{0.616804in}}%
\pgfpathlineto{\pgfqpoint{2.337617in}{0.614480in}}%
\pgfpathlineto{\pgfqpoint{2.327687in}{0.632577in}}%
\pgfpathlineto{\pgfqpoint{2.320573in}{0.644805in}}%
\pgfpathlineto{\pgfqpoint{2.321628in}{0.666365in}}%
\pgfpathlineto{\pgfqpoint{2.314034in}{0.687792in}}%
\pgfpathlineto{\pgfqpoint{2.316659in}{0.706111in}}%
\pgfpathlineto{\pgfqpoint{2.324428in}{0.716398in}}%
\pgfpathlineto{\pgfqpoint{2.317853in}{0.719705in}}%
\pgfpathlineto{\pgfqpoint{2.319340in}{0.726491in}}%
\pgfpathlineto{\pgfqpoint{2.310136in}{0.739826in}}%
\pgfpathlineto{\pgfqpoint{2.304291in}{0.754156in}}%
\pgfpathlineto{\pgfqpoint{2.294184in}{0.789743in}}%
\pgfpathlineto{\pgfqpoint{2.282908in}{0.832251in}}%
\pgfpathlineto{\pgfqpoint{2.266375in}{0.890163in}}%
\pgfpathclose%
\pgfusepath{fill}%
\end{pgfscope}%
\begin{pgfscope}%
\pgfpathrectangle{\pgfqpoint{0.100000in}{0.100000in}}{\pgfqpoint{2.989028in}{1.913466in}}%
\pgfusepath{clip}%
\pgfsetbuttcap%
\pgfsetmiterjoin%
\definecolor{currentfill}{rgb}{0.865667,0.946021,0.603460}%
\pgfsetfillcolor{currentfill}%
\pgfsetlinewidth{0.000000pt}%
\definecolor{currentstroke}{rgb}{0.000000,0.000000,0.000000}%
\pgfsetstrokecolor{currentstroke}%
\pgfsetstrokeopacity{0.000000}%
\pgfsetdash{}{0pt}%
\pgfpathmoveto{\pgfqpoint{2.399943in}{0.906551in}}%
\pgfpathlineto{\pgfqpoint{2.427183in}{0.920102in}}%
\pgfpathlineto{\pgfqpoint{2.442269in}{0.925218in}}%
\pgfpathlineto{\pgfqpoint{2.507904in}{0.932038in}}%
\pgfpathlineto{\pgfqpoint{2.514855in}{0.929779in}}%
\pgfpathlineto{\pgfqpoint{2.524037in}{0.920550in}}%
\pgfpathlineto{\pgfqpoint{2.524543in}{0.912354in}}%
\pgfpathlineto{\pgfqpoint{2.584952in}{0.921230in}}%
\pgfpathlineto{\pgfqpoint{2.653548in}{0.872057in}}%
\pgfpathlineto{\pgfqpoint{2.640684in}{0.858649in}}%
\pgfpathlineto{\pgfqpoint{2.623033in}{0.827511in}}%
\pgfpathlineto{\pgfqpoint{2.628057in}{0.820819in}}%
\pgfpathlineto{\pgfqpoint{2.618641in}{0.807830in}}%
\pgfpathlineto{\pgfqpoint{2.609275in}{0.806359in}}%
\pgfpathlineto{\pgfqpoint{2.609240in}{0.798554in}}%
\pgfpathlineto{\pgfqpoint{2.602464in}{0.790386in}}%
\pgfpathlineto{\pgfqpoint{2.594032in}{0.788730in}}%
\pgfpathlineto{\pgfqpoint{2.595872in}{0.781473in}}%
\pgfpathlineto{\pgfqpoint{2.591167in}{0.775902in}}%
\pgfpathlineto{\pgfqpoint{2.579911in}{0.771089in}}%
\pgfpathlineto{\pgfqpoint{2.566192in}{0.760141in}}%
\pgfpathlineto{\pgfqpoint{2.568782in}{0.753058in}}%
\pgfpathlineto{\pgfqpoint{2.560148in}{0.748611in}}%
\pgfpathlineto{\pgfqpoint{2.552866in}{0.753287in}}%
\pgfpathlineto{\pgfqpoint{2.547540in}{0.732958in}}%
\pgfpathlineto{\pgfqpoint{2.535460in}{0.733570in}}%
\pgfpathlineto{\pgfqpoint{2.532685in}{0.744563in}}%
\pgfpathlineto{\pgfqpoint{2.524078in}{0.759511in}}%
\pgfpathlineto{\pgfqpoint{2.513387in}{0.764886in}}%
\pgfpathlineto{\pgfqpoint{2.510161in}{0.778793in}}%
\pgfpathlineto{\pgfqpoint{2.502872in}{0.791165in}}%
\pgfpathlineto{\pgfqpoint{2.492579in}{0.794896in}}%
\pgfpathlineto{\pgfqpoint{2.479142in}{0.807998in}}%
\pgfpathlineto{\pgfqpoint{2.477896in}{0.814587in}}%
\pgfpathlineto{\pgfqpoint{2.461332in}{0.823674in}}%
\pgfpathlineto{\pgfqpoint{2.456053in}{0.832075in}}%
\pgfpathlineto{\pgfqpoint{2.437478in}{0.842609in}}%
\pgfpathlineto{\pgfqpoint{2.427019in}{0.855143in}}%
\pgfpathlineto{\pgfqpoint{2.417377in}{0.872915in}}%
\pgfpathlineto{\pgfqpoint{2.407061in}{0.873695in}}%
\pgfpathlineto{\pgfqpoint{2.400235in}{0.880009in}}%
\pgfpathlineto{\pgfqpoint{2.390198in}{0.883919in}}%
\pgfpathlineto{\pgfqpoint{2.390192in}{0.891171in}}%
\pgfpathlineto{\pgfqpoint{2.399943in}{0.906551in}}%
\pgfpathclose%
\pgfusepath{fill}%
\end{pgfscope}%
\begin{pgfscope}%
\pgfpathrectangle{\pgfqpoint{0.100000in}{0.100000in}}{\pgfqpoint{2.989028in}{1.913466in}}%
\pgfusepath{clip}%
\pgfsetbuttcap%
\pgfsetmiterjoin%
\definecolor{currentfill}{rgb}{0.999000,0.969012,0.697040}%
\pgfsetfillcolor{currentfill}%
\pgfsetlinewidth{0.000000pt}%
\definecolor{currentstroke}{rgb}{0.000000,0.000000,0.000000}%
\pgfsetstrokecolor{currentstroke}%
\pgfsetstrokeopacity{0.000000}%
\pgfsetdash{}{0pt}%
\pgfpathmoveto{\pgfqpoint{1.782462in}{0.962463in}}%
\pgfpathlineto{\pgfqpoint{1.822491in}{0.962824in}}%
\pgfpathlineto{\pgfqpoint{1.875448in}{0.963789in}}%
\pgfpathlineto{\pgfqpoint{1.965941in}{0.966662in}}%
\pgfpathlineto{\pgfqpoint{2.017700in}{0.969399in}}%
\pgfpathlineto{\pgfqpoint{2.023251in}{0.962527in}}%
\pgfpathlineto{\pgfqpoint{2.022904in}{0.955269in}}%
\pgfpathlineto{\pgfqpoint{2.010365in}{0.942748in}}%
\pgfpathlineto{\pgfqpoint{2.007334in}{0.935892in}}%
\pgfpathlineto{\pgfqpoint{2.042109in}{0.938464in}}%
\pgfpathlineto{\pgfqpoint{2.042090in}{0.925797in}}%
\pgfpathlineto{\pgfqpoint{2.030934in}{0.920384in}}%
\pgfpathlineto{\pgfqpoint{2.031092in}{0.911777in}}%
\pgfpathlineto{\pgfqpoint{2.026902in}{0.905590in}}%
\pgfpathlineto{\pgfqpoint{2.024159in}{0.892449in}}%
\pgfpathlineto{\pgfqpoint{2.026990in}{0.881608in}}%
\pgfpathlineto{\pgfqpoint{2.014584in}{0.871310in}}%
\pgfpathlineto{\pgfqpoint{2.018214in}{0.866243in}}%
\pgfpathlineto{\pgfqpoint{2.006577in}{0.857690in}}%
\pgfpathlineto{\pgfqpoint{2.007021in}{0.849323in}}%
\pgfpathlineto{\pgfqpoint{1.997678in}{0.828737in}}%
\pgfpathlineto{\pgfqpoint{1.988922in}{0.822501in}}%
\pgfpathlineto{\pgfqpoint{1.988077in}{0.813926in}}%
\pgfpathlineto{\pgfqpoint{1.969899in}{0.781706in}}%
\pgfpathlineto{\pgfqpoint{1.976057in}{0.763762in}}%
\pgfpathlineto{\pgfqpoint{1.974345in}{0.759012in}}%
\pgfpathlineto{\pgfqpoint{1.977845in}{0.748573in}}%
\pgfpathlineto{\pgfqpoint{1.974085in}{0.738625in}}%
\pgfpathlineto{\pgfqpoint{1.924378in}{0.736573in}}%
\pgfpathlineto{\pgfqpoint{1.859848in}{0.735510in}}%
\pgfpathlineto{\pgfqpoint{1.815340in}{0.735108in}}%
\pgfpathlineto{\pgfqpoint{1.815123in}{0.770153in}}%
\pgfpathlineto{\pgfqpoint{1.804070in}{0.772449in}}%
\pgfpathlineto{\pgfqpoint{1.796775in}{0.769451in}}%
\pgfpathlineto{\pgfqpoint{1.790948in}{0.774957in}}%
\pgfpathlineto{\pgfqpoint{1.791549in}{0.812079in}}%
\pgfpathlineto{\pgfqpoint{1.792853in}{0.891120in}}%
\pgfpathlineto{\pgfqpoint{1.786531in}{0.937404in}}%
\pgfpathlineto{\pgfqpoint{1.782462in}{0.962463in}}%
\pgfpathclose%
\pgfusepath{fill}%
\end{pgfscope}%
\begin{pgfscope}%
\pgfpathrectangle{\pgfqpoint{0.100000in}{0.100000in}}{\pgfqpoint{2.989028in}{1.913466in}}%
\pgfusepath{clip}%
\pgfsetbuttcap%
\pgfsetmiterjoin%
\definecolor{currentfill}{rgb}{0.665283,0.864591,0.643214}%
\pgfsetfillcolor{currentfill}%
\pgfsetlinewidth{0.000000pt}%
\definecolor{currentstroke}{rgb}{0.000000,0.000000,0.000000}%
\pgfsetstrokecolor{currentstroke}%
\pgfsetstrokeopacity{0.000000}%
\pgfsetdash{}{0pt}%
\pgfpathmoveto{\pgfqpoint{1.815340in}{0.735108in}}%
\pgfpathlineto{\pgfqpoint{1.859848in}{0.735510in}}%
\pgfpathlineto{\pgfqpoint{1.924378in}{0.736573in}}%
\pgfpathlineto{\pgfqpoint{1.974085in}{0.738625in}}%
\pgfpathlineto{\pgfqpoint{1.971703in}{0.733439in}}%
\pgfpathlineto{\pgfqpoint{1.977983in}{0.704877in}}%
\pgfpathlineto{\pgfqpoint{1.986660in}{0.692400in}}%
\pgfpathlineto{\pgfqpoint{1.976601in}{0.685508in}}%
\pgfpathlineto{\pgfqpoint{1.981859in}{0.673535in}}%
\pgfpathlineto{\pgfqpoint{1.967673in}{0.661207in}}%
\pgfpathlineto{\pgfqpoint{1.963432in}{0.642495in}}%
\pgfpathlineto{\pgfqpoint{1.955681in}{0.632856in}}%
\pgfpathlineto{\pgfqpoint{1.956668in}{0.622916in}}%
\pgfpathlineto{\pgfqpoint{1.952221in}{0.620808in}}%
\pgfpathlineto{\pgfqpoint{1.956684in}{0.610463in}}%
\pgfpathlineto{\pgfqpoint{1.953073in}{0.604986in}}%
\pgfpathlineto{\pgfqpoint{2.013761in}{0.607617in}}%
\pgfpathlineto{\pgfqpoint{2.060562in}{0.610498in}}%
\pgfpathlineto{\pgfqpoint{2.055622in}{0.588493in}}%
\pgfpathlineto{\pgfqpoint{2.058992in}{0.580363in}}%
\pgfpathlineto{\pgfqpoint{2.065977in}{0.573569in}}%
\pgfpathlineto{\pgfqpoint{2.072753in}{0.557360in}}%
\pgfpathlineto{\pgfqpoint{2.051350in}{0.561094in}}%
\pgfpathlineto{\pgfqpoint{2.043455in}{0.567234in}}%
\pgfpathlineto{\pgfqpoint{2.034048in}{0.567526in}}%
\pgfpathlineto{\pgfqpoint{2.024158in}{0.554082in}}%
\pgfpathlineto{\pgfqpoint{2.026129in}{0.547960in}}%
\pgfpathlineto{\pgfqpoint{2.057724in}{0.544257in}}%
\pgfpathlineto{\pgfqpoint{2.065995in}{0.537220in}}%
\pgfpathlineto{\pgfqpoint{2.079454in}{0.533600in}}%
\pgfpathlineto{\pgfqpoint{2.073576in}{0.525268in}}%
\pgfpathlineto{\pgfqpoint{2.063672in}{0.518003in}}%
\pgfpathlineto{\pgfqpoint{2.089813in}{0.501664in}}%
\pgfpathlineto{\pgfqpoint{2.101990in}{0.499111in}}%
\pgfpathlineto{\pgfqpoint{2.107054in}{0.485755in}}%
\pgfpathlineto{\pgfqpoint{2.095589in}{0.483298in}}%
\pgfpathlineto{\pgfqpoint{2.074041in}{0.500096in}}%
\pgfpathlineto{\pgfqpoint{2.065616in}{0.502430in}}%
\pgfpathlineto{\pgfqpoint{2.061539in}{0.509066in}}%
\pgfpathlineto{\pgfqpoint{2.049043in}{0.506343in}}%
\pgfpathlineto{\pgfqpoint{2.045130in}{0.497792in}}%
\pgfpathlineto{\pgfqpoint{2.047608in}{0.488264in}}%
\pgfpathlineto{\pgfqpoint{2.039195in}{0.482642in}}%
\pgfpathlineto{\pgfqpoint{2.031517in}{0.496493in}}%
\pgfpathlineto{\pgfqpoint{2.018069in}{0.492363in}}%
\pgfpathlineto{\pgfqpoint{2.013030in}{0.484098in}}%
\pgfpathlineto{\pgfqpoint{2.003479in}{0.486449in}}%
\pgfpathlineto{\pgfqpoint{1.997373in}{0.496886in}}%
\pgfpathlineto{\pgfqpoint{1.981173in}{0.500333in}}%
\pgfpathlineto{\pgfqpoint{1.978115in}{0.505764in}}%
\pgfpathlineto{\pgfqpoint{1.961236in}{0.515225in}}%
\pgfpathlineto{\pgfqpoint{1.957047in}{0.523496in}}%
\pgfpathlineto{\pgfqpoint{1.942910in}{0.520116in}}%
\pgfpathlineto{\pgfqpoint{1.944705in}{0.527689in}}%
\pgfpathlineto{\pgfqpoint{1.927134in}{0.519920in}}%
\pgfpathlineto{\pgfqpoint{1.931854in}{0.511860in}}%
\pgfpathlineto{\pgfqpoint{1.918284in}{0.507095in}}%
\pgfpathlineto{\pgfqpoint{1.900311in}{0.509714in}}%
\pgfpathlineto{\pgfqpoint{1.863934in}{0.522174in}}%
\pgfpathlineto{\pgfqpoint{1.835864in}{0.519701in}}%
\pgfpathlineto{\pgfqpoint{1.833432in}{0.536068in}}%
\pgfpathlineto{\pgfqpoint{1.836956in}{0.543537in}}%
\pgfpathlineto{\pgfqpoint{1.836655in}{0.555420in}}%
\pgfpathlineto{\pgfqpoint{1.833187in}{0.558251in}}%
\pgfpathlineto{\pgfqpoint{1.834363in}{0.571815in}}%
\pgfpathlineto{\pgfqpoint{1.844586in}{0.590680in}}%
\pgfpathlineto{\pgfqpoint{1.845894in}{0.597808in}}%
\pgfpathlineto{\pgfqpoint{1.844209in}{0.614644in}}%
\pgfpathlineto{\pgfqpoint{1.836688in}{0.622574in}}%
\pgfpathlineto{\pgfqpoint{1.832761in}{0.637381in}}%
\pgfpathlineto{\pgfqpoint{1.827929in}{0.640781in}}%
\pgfpathlineto{\pgfqpoint{1.830092in}{0.648811in}}%
\pgfpathlineto{\pgfqpoint{1.823939in}{0.660813in}}%
\pgfpathlineto{\pgfqpoint{1.816262in}{0.667326in}}%
\pgfpathlineto{\pgfqpoint{1.815340in}{0.735108in}}%
\pgfpathclose%
\pgfusepath{fill}%
\end{pgfscope}%
\begin{pgfscope}%
\pgfpathrectangle{\pgfqpoint{0.100000in}{0.100000in}}{\pgfqpoint{2.989028in}{1.913466in}}%
\pgfusepath{clip}%
\pgfsetbuttcap%
\pgfsetmiterjoin%
\definecolor{currentfill}{rgb}{0.665283,0.864591,0.643214}%
\pgfsetfillcolor{currentfill}%
\pgfsetlinewidth{0.000000pt}%
\definecolor{currentstroke}{rgb}{0.000000,0.000000,0.000000}%
\pgfsetstrokecolor{currentstroke}%
\pgfsetstrokeopacity{0.000000}%
\pgfsetdash{}{0pt}%
\pgfpathmoveto{\pgfqpoint{1.934378in}{0.511243in}}%
\pgfpathlineto{\pgfqpoint{1.940817in}{0.515073in}}%
\pgfpathlineto{\pgfqpoint{1.948633in}{0.510545in}}%
\pgfpathlineto{\pgfqpoint{1.944274in}{0.504314in}}%
\pgfpathlineto{\pgfqpoint{1.934378in}{0.511243in}}%
\pgfpathclose%
\pgfusepath{fill}%
\end{pgfscope}%
\begin{pgfscope}%
\pgfpathrectangle{\pgfqpoint{0.100000in}{0.100000in}}{\pgfqpoint{2.989028in}{1.913466in}}%
\pgfusepath{clip}%
\pgfsetbuttcap%
\pgfsetmiterjoin%
\definecolor{currentfill}{rgb}{0.802153,0.920185,0.616378}%
\pgfsetfillcolor{currentfill}%
\pgfsetlinewidth{0.000000pt}%
\definecolor{currentstroke}{rgb}{0.000000,0.000000,0.000000}%
\pgfsetstrokecolor{currentstroke}%
\pgfsetstrokeopacity{0.000000}%
\pgfsetdash{}{0pt}%
\pgfpathmoveto{\pgfqpoint{2.194230in}{0.585103in}}%
\pgfpathlineto{\pgfqpoint{2.194346in}{0.597485in}}%
\pgfpathlineto{\pgfqpoint{2.186652in}{0.602188in}}%
\pgfpathlineto{\pgfqpoint{2.180349in}{0.610192in}}%
\pgfpathlineto{\pgfqpoint{2.181207in}{0.618602in}}%
\pgfpathlineto{\pgfqpoint{2.249834in}{0.623930in}}%
\pgfpathlineto{\pgfqpoint{2.327687in}{0.632577in}}%
\pgfpathlineto{\pgfqpoint{2.337617in}{0.614480in}}%
\pgfpathlineto{\pgfqpoint{2.370635in}{0.616804in}}%
\pgfpathlineto{\pgfqpoint{2.436103in}{0.620376in}}%
\pgfpathlineto{\pgfqpoint{2.488024in}{0.623836in}}%
\pgfpathlineto{\pgfqpoint{2.493123in}{0.610802in}}%
\pgfpathlineto{\pgfqpoint{2.499427in}{0.611870in}}%
\pgfpathlineto{\pgfqpoint{2.500128in}{0.625826in}}%
\pgfpathlineto{\pgfqpoint{2.496902in}{0.637830in}}%
\pgfpathlineto{\pgfqpoint{2.501225in}{0.642912in}}%
\pgfpathlineto{\pgfqpoint{2.511217in}{0.640178in}}%
\pgfpathlineto{\pgfqpoint{2.525559in}{0.639617in}}%
\pgfpathlineto{\pgfqpoint{2.527764in}{0.628892in}}%
\pgfpathlineto{\pgfqpoint{2.532176in}{0.622764in}}%
\pgfpathlineto{\pgfqpoint{2.535621in}{0.609356in}}%
\pgfpathlineto{\pgfqpoint{2.546296in}{0.588607in}}%
\pgfpathlineto{\pgfqpoint{2.546349in}{0.582995in}}%
\pgfpathlineto{\pgfqpoint{2.562126in}{0.558631in}}%
\pgfpathlineto{\pgfqpoint{2.563652in}{0.553598in}}%
\pgfpathlineto{\pgfqpoint{2.587320in}{0.519188in}}%
\pgfpathlineto{\pgfqpoint{2.583570in}{0.518632in}}%
\pgfpathlineto{\pgfqpoint{2.593548in}{0.494158in}}%
\pgfpathlineto{\pgfqpoint{2.620959in}{0.451609in}}%
\pgfpathlineto{\pgfqpoint{2.636873in}{0.420325in}}%
\pgfpathlineto{\pgfqpoint{2.642192in}{0.414848in}}%
\pgfpathlineto{\pgfqpoint{2.651306in}{0.395211in}}%
\pgfpathlineto{\pgfqpoint{2.654468in}{0.363769in}}%
\pgfpathlineto{\pgfqpoint{2.655727in}{0.340258in}}%
\pgfpathlineto{\pgfqpoint{2.654207in}{0.325205in}}%
\pgfpathlineto{\pgfqpoint{2.649320in}{0.314444in}}%
\pgfpathlineto{\pgfqpoint{2.651590in}{0.300332in}}%
\pgfpathlineto{\pgfqpoint{2.646329in}{0.289163in}}%
\pgfpathlineto{\pgfqpoint{2.630707in}{0.280003in}}%
\pgfpathlineto{\pgfqpoint{2.620547in}{0.280672in}}%
\pgfpathlineto{\pgfqpoint{2.613949in}{0.275884in}}%
\pgfpathlineto{\pgfqpoint{2.612059in}{0.288668in}}%
\pgfpathlineto{\pgfqpoint{2.601137in}{0.292042in}}%
\pgfpathlineto{\pgfqpoint{2.591346in}{0.309880in}}%
\pgfpathlineto{\pgfqpoint{2.590241in}{0.318003in}}%
\pgfpathlineto{\pgfqpoint{2.572719in}{0.322936in}}%
\pgfpathlineto{\pgfqpoint{2.561428in}{0.321868in}}%
\pgfpathlineto{\pgfqpoint{2.555040in}{0.333654in}}%
\pgfpathlineto{\pgfqpoint{2.547725in}{0.354839in}}%
\pgfpathlineto{\pgfqpoint{2.537575in}{0.359133in}}%
\pgfpathlineto{\pgfqpoint{2.532072in}{0.371273in}}%
\pgfpathlineto{\pgfqpoint{2.518614in}{0.378428in}}%
\pgfpathlineto{\pgfqpoint{2.510835in}{0.387377in}}%
\pgfpathlineto{\pgfqpoint{2.497416in}{0.411708in}}%
\pgfpathlineto{\pgfqpoint{2.495838in}{0.428833in}}%
\pgfpathlineto{\pgfqpoint{2.503150in}{0.439789in}}%
\pgfpathlineto{\pgfqpoint{2.502419in}{0.447403in}}%
\pgfpathlineto{\pgfqpoint{2.493969in}{0.448217in}}%
\pgfpathlineto{\pgfqpoint{2.486909in}{0.453513in}}%
\pgfpathlineto{\pgfqpoint{2.483042in}{0.447046in}}%
\pgfpathlineto{\pgfqpoint{2.489818in}{0.441777in}}%
\pgfpathlineto{\pgfqpoint{2.484887in}{0.432302in}}%
\pgfpathlineto{\pgfqpoint{2.476908in}{0.440166in}}%
\pgfpathlineto{\pgfqpoint{2.477834in}{0.461974in}}%
\pgfpathlineto{\pgfqpoint{2.481682in}{0.479667in}}%
\pgfpathlineto{\pgfqpoint{2.479704in}{0.510026in}}%
\pgfpathlineto{\pgfqpoint{2.471751in}{0.517257in}}%
\pgfpathlineto{\pgfqpoint{2.467752in}{0.526536in}}%
\pgfpathlineto{\pgfqpoint{2.454052in}{0.526329in}}%
\pgfpathlineto{\pgfqpoint{2.440510in}{0.541597in}}%
\pgfpathlineto{\pgfqpoint{2.431430in}{0.546173in}}%
\pgfpathlineto{\pgfqpoint{2.428737in}{0.555842in}}%
\pgfpathlineto{\pgfqpoint{2.419856in}{0.559238in}}%
\pgfpathlineto{\pgfqpoint{2.412496in}{0.569913in}}%
\pgfpathlineto{\pgfqpoint{2.393052in}{0.578628in}}%
\pgfpathlineto{\pgfqpoint{2.377937in}{0.578850in}}%
\pgfpathlineto{\pgfqpoint{2.371356in}{0.575503in}}%
\pgfpathlineto{\pgfqpoint{2.372991in}{0.565056in}}%
\pgfpathlineto{\pgfqpoint{2.366130in}{0.565551in}}%
\pgfpathlineto{\pgfqpoint{2.345022in}{0.550943in}}%
\pgfpathlineto{\pgfqpoint{2.319668in}{0.545153in}}%
\pgfpathlineto{\pgfqpoint{2.319252in}{0.552317in}}%
\pgfpathlineto{\pgfqpoint{2.313633in}{0.559318in}}%
\pgfpathlineto{\pgfqpoint{2.298558in}{0.568984in}}%
\pgfpathlineto{\pgfqpoint{2.276918in}{0.578894in}}%
\pgfpathlineto{\pgfqpoint{2.253414in}{0.584141in}}%
\pgfpathlineto{\pgfqpoint{2.249002in}{0.591275in}}%
\pgfpathlineto{\pgfqpoint{2.240528in}{0.585325in}}%
\pgfpathlineto{\pgfqpoint{2.230323in}{0.584012in}}%
\pgfpathlineto{\pgfqpoint{2.194801in}{0.574653in}}%
\pgfpathlineto{\pgfqpoint{2.194230in}{0.585103in}}%
\pgfpathclose%
\pgfusepath{fill}%
\end{pgfscope}%
\begin{pgfscope}%
\pgfpathrectangle{\pgfqpoint{0.100000in}{0.100000in}}{\pgfqpoint{2.989028in}{1.913466in}}%
\pgfusepath{clip}%
\pgfsetbuttcap%
\pgfsetmiterjoin%
\definecolor{currentfill}{rgb}{0.802153,0.920185,0.616378}%
\pgfsetfillcolor{currentfill}%
\pgfsetlinewidth{0.000000pt}%
\definecolor{currentstroke}{rgb}{0.000000,0.000000,0.000000}%
\pgfsetstrokecolor{currentstroke}%
\pgfsetstrokeopacity{0.000000}%
\pgfsetdash{}{0pt}%
\pgfpathmoveto{\pgfqpoint{2.590769in}{0.519678in}}%
\pgfpathlineto{\pgfqpoint{2.599959in}{0.508987in}}%
\pgfpathlineto{\pgfqpoint{2.589344in}{0.508314in}}%
\pgfpathlineto{\pgfqpoint{2.590769in}{0.519678in}}%
\pgfpathclose%
\pgfusepath{fill}%
\end{pgfscope}%
\begin{pgfscope}%
\pgfpathrectangle{\pgfqpoint{0.100000in}{0.100000in}}{\pgfqpoint{2.989028in}{1.913466in}}%
\pgfusepath{clip}%
\pgfsetbuttcap%
\pgfsetmiterjoin%
\definecolor{currentfill}{rgb}{0.820300,0.927566,0.612687}%
\pgfsetfillcolor{currentfill}%
\pgfsetlinewidth{0.000000pt}%
\definecolor{currentstroke}{rgb}{0.000000,0.000000,0.000000}%
\pgfsetstrokecolor{currentstroke}%
\pgfsetstrokeopacity{0.000000}%
\pgfsetdash{}{0pt}%
\pgfpathmoveto{\pgfqpoint{2.049740in}{1.740361in}}%
\pgfpathlineto{\pgfqpoint{2.044753in}{1.730656in}}%
\pgfpathlineto{\pgfqpoint{2.032857in}{1.724992in}}%
\pgfpathlineto{\pgfqpoint{2.027705in}{1.717368in}}%
\pgfpathlineto{\pgfqpoint{2.021669in}{1.722859in}}%
\pgfpathlineto{\pgfqpoint{2.049740in}{1.740361in}}%
\pgfpathclose%
\pgfusepath{fill}%
\end{pgfscope}%
\begin{pgfscope}%
\pgfpathrectangle{\pgfqpoint{0.100000in}{0.100000in}}{\pgfqpoint{2.989028in}{1.913466in}}%
\pgfusepath{clip}%
\pgfsetbuttcap%
\pgfsetmiterjoin%
\definecolor{currentfill}{rgb}{0.820300,0.927566,0.612687}%
\pgfsetfillcolor{currentfill}%
\pgfsetlinewidth{0.000000pt}%
\definecolor{currentstroke}{rgb}{0.000000,0.000000,0.000000}%
\pgfsetstrokecolor{currentstroke}%
\pgfsetstrokeopacity{0.000000}%
\pgfsetdash{}{0pt}%
\pgfpathmoveto{\pgfqpoint{2.053784in}{1.681953in}}%
\pgfpathlineto{\pgfqpoint{2.065956in}{1.693308in}}%
\pgfpathlineto{\pgfqpoint{2.084730in}{1.696281in}}%
\pgfpathlineto{\pgfqpoint{2.079559in}{1.688384in}}%
\pgfpathlineto{\pgfqpoint{2.066688in}{1.676964in}}%
\pgfpathlineto{\pgfqpoint{2.059164in}{1.662306in}}%
\pgfpathlineto{\pgfqpoint{2.055977in}{1.670256in}}%
\pgfpathlineto{\pgfqpoint{2.050318in}{1.671398in}}%
\pgfpathlineto{\pgfqpoint{2.049750in}{1.678576in}}%
\pgfpathlineto{\pgfqpoint{2.053784in}{1.681953in}}%
\pgfpathclose%
\pgfusepath{fill}%
\end{pgfscope}%
\begin{pgfscope}%
\pgfpathrectangle{\pgfqpoint{0.100000in}{0.100000in}}{\pgfqpoint{2.989028in}{1.913466in}}%
\pgfusepath{clip}%
\pgfsetbuttcap%
\pgfsetmiterjoin%
\definecolor{currentfill}{rgb}{0.820300,0.927566,0.612687}%
\pgfsetfillcolor{currentfill}%
\pgfsetlinewidth{0.000000pt}%
\definecolor{currentstroke}{rgb}{0.000000,0.000000,0.000000}%
\pgfsetstrokecolor{currentstroke}%
\pgfsetstrokeopacity{0.000000}%
\pgfsetdash{}{0pt}%
\pgfpathmoveto{\pgfqpoint{2.102264in}{1.543360in}}%
\pgfpathlineto{\pgfqpoint{2.099022in}{1.546957in}}%
\pgfpathlineto{\pgfqpoint{2.102430in}{1.557084in}}%
\pgfpathlineto{\pgfqpoint{2.092314in}{1.557730in}}%
\pgfpathlineto{\pgfqpoint{2.094996in}{1.566462in}}%
\pgfpathlineto{\pgfqpoint{2.093328in}{1.580368in}}%
\pgfpathlineto{\pgfqpoint{2.084251in}{1.585189in}}%
\pgfpathlineto{\pgfqpoint{2.074752in}{1.594947in}}%
\pgfpathlineto{\pgfqpoint{2.060116in}{1.597803in}}%
\pgfpathlineto{\pgfqpoint{2.045871in}{1.597722in}}%
\pgfpathlineto{\pgfqpoint{2.031860in}{1.604646in}}%
\pgfpathlineto{\pgfqpoint{1.984989in}{1.614740in}}%
\pgfpathlineto{\pgfqpoint{1.979867in}{1.625436in}}%
\pgfpathlineto{\pgfqpoint{1.970731in}{1.629085in}}%
\pgfpathlineto{\pgfqpoint{1.987989in}{1.637274in}}%
\pgfpathlineto{\pgfqpoint{1.997738in}{1.647490in}}%
\pgfpathlineto{\pgfqpoint{2.015897in}{1.650260in}}%
\pgfpathlineto{\pgfqpoint{2.027071in}{1.660666in}}%
\pgfpathlineto{\pgfqpoint{2.032934in}{1.661083in}}%
\pgfpathlineto{\pgfqpoint{2.037408in}{1.668504in}}%
\pgfpathlineto{\pgfqpoint{2.048315in}{1.677295in}}%
\pgfpathlineto{\pgfqpoint{2.049265in}{1.671081in}}%
\pgfpathlineto{\pgfqpoint{2.054179in}{1.669798in}}%
\pgfpathlineto{\pgfqpoint{2.057877in}{1.662388in}}%
\pgfpathlineto{\pgfqpoint{2.058543in}{1.649753in}}%
\pgfpathlineto{\pgfqpoint{2.069656in}{1.657304in}}%
\pgfpathlineto{\pgfqpoint{2.082606in}{1.658880in}}%
\pgfpathlineto{\pgfqpoint{2.093660in}{1.654925in}}%
\pgfpathlineto{\pgfqpoint{2.108651in}{1.634360in}}%
\pgfpathlineto{\pgfqpoint{2.125023in}{1.637647in}}%
\pgfpathlineto{\pgfqpoint{2.131676in}{1.632143in}}%
\pgfpathlineto{\pgfqpoint{2.142376in}{1.631651in}}%
\pgfpathlineto{\pgfqpoint{2.149485in}{1.641522in}}%
\pgfpathlineto{\pgfqpoint{2.162974in}{1.650248in}}%
\pgfpathlineto{\pgfqpoint{2.175939in}{1.652980in}}%
\pgfpathlineto{\pgfqpoint{2.192026in}{1.653265in}}%
\pgfpathlineto{\pgfqpoint{2.203782in}{1.660005in}}%
\pgfpathlineto{\pgfqpoint{2.213376in}{1.656901in}}%
\pgfpathlineto{\pgfqpoint{2.215419in}{1.642645in}}%
\pgfpathlineto{\pgfqpoint{2.236002in}{1.640413in}}%
\pgfpathlineto{\pgfqpoint{2.247182in}{1.647086in}}%
\pgfpathlineto{\pgfqpoint{2.254932in}{1.632053in}}%
\pgfpathlineto{\pgfqpoint{2.269719in}{1.614670in}}%
\pgfpathlineto{\pgfqpoint{2.249022in}{1.614771in}}%
\pgfpathlineto{\pgfqpoint{2.242482in}{1.612622in}}%
\pgfpathlineto{\pgfqpoint{2.233519in}{1.615414in}}%
\pgfpathlineto{\pgfqpoint{2.232916in}{1.603380in}}%
\pgfpathlineto{\pgfqpoint{2.216640in}{1.612785in}}%
\pgfpathlineto{\pgfqpoint{2.195708in}{1.615659in}}%
\pgfpathlineto{\pgfqpoint{2.189946in}{1.606500in}}%
\pgfpathlineto{\pgfqpoint{2.162559in}{1.602046in}}%
\pgfpathlineto{\pgfqpoint{2.159409in}{1.594288in}}%
\pgfpathlineto{\pgfqpoint{2.140351in}{1.591960in}}%
\pgfpathlineto{\pgfqpoint{2.134590in}{1.584073in}}%
\pgfpathlineto{\pgfqpoint{2.124509in}{1.581958in}}%
\pgfpathlineto{\pgfqpoint{2.116458in}{1.563256in}}%
\pgfpathlineto{\pgfqpoint{2.106264in}{1.545142in}}%
\pgfpathlineto{\pgfqpoint{2.102264in}{1.543360in}}%
\pgfpathclose%
\pgfusepath{fill}%
\end{pgfscope}%
\begin{pgfscope}%
\pgfpathrectangle{\pgfqpoint{0.100000in}{0.100000in}}{\pgfqpoint{2.989028in}{1.913466in}}%
\pgfusepath{clip}%
\pgfsetbuttcap%
\pgfsetmiterjoin%
\definecolor{currentfill}{rgb}{0.820300,0.927566,0.612687}%
\pgfsetfillcolor{currentfill}%
\pgfsetlinewidth{0.000000pt}%
\definecolor{currentstroke}{rgb}{0.000000,0.000000,0.000000}%
\pgfsetstrokecolor{currentstroke}%
\pgfsetstrokeopacity{0.000000}%
\pgfsetdash{}{0pt}%
\pgfpathmoveto{\pgfqpoint{2.160936in}{1.325898in}}%
\pgfpathlineto{\pgfqpoint{2.170655in}{1.336128in}}%
\pgfpathlineto{\pgfqpoint{2.175083in}{1.350960in}}%
\pgfpathlineto{\pgfqpoint{2.180349in}{1.359557in}}%
\pgfpathlineto{\pgfqpoint{2.183578in}{1.371256in}}%
\pgfpathlineto{\pgfqpoint{2.184581in}{1.394584in}}%
\pgfpathlineto{\pgfqpoint{2.179720in}{1.416940in}}%
\pgfpathlineto{\pgfqpoint{2.163654in}{1.451174in}}%
\pgfpathlineto{\pgfqpoint{2.168005in}{1.461938in}}%
\pgfpathlineto{\pgfqpoint{2.162337in}{1.476816in}}%
\pgfpathlineto{\pgfqpoint{2.172059in}{1.497392in}}%
\pgfpathlineto{\pgfqpoint{2.170475in}{1.520335in}}%
\pgfpathlineto{\pgfqpoint{2.177247in}{1.523223in}}%
\pgfpathlineto{\pgfqpoint{2.178098in}{1.534139in}}%
\pgfpathlineto{\pgfqpoint{2.190119in}{1.541146in}}%
\pgfpathlineto{\pgfqpoint{2.196918in}{1.540015in}}%
\pgfpathlineto{\pgfqpoint{2.198829in}{1.528307in}}%
\pgfpathlineto{\pgfqpoint{2.204135in}{1.527848in}}%
\pgfpathlineto{\pgfqpoint{2.208977in}{1.544735in}}%
\pgfpathlineto{\pgfqpoint{2.207331in}{1.557869in}}%
\pgfpathlineto{\pgfqpoint{2.210479in}{1.565411in}}%
\pgfpathlineto{\pgfqpoint{2.227498in}{1.573202in}}%
\pgfpathlineto{\pgfqpoint{2.219743in}{1.575975in}}%
\pgfpathlineto{\pgfqpoint{2.217216in}{1.582668in}}%
\pgfpathlineto{\pgfqpoint{2.222803in}{1.594421in}}%
\pgfpathlineto{\pgfqpoint{2.233821in}{1.598481in}}%
\pgfpathlineto{\pgfqpoint{2.246615in}{1.591546in}}%
\pgfpathlineto{\pgfqpoint{2.258677in}{1.591460in}}%
\pgfpathlineto{\pgfqpoint{2.264279in}{1.583359in}}%
\pgfpathlineto{\pgfqpoint{2.272727in}{1.583922in}}%
\pgfpathlineto{\pgfqpoint{2.298804in}{1.572661in}}%
\pgfpathlineto{\pgfqpoint{2.304014in}{1.561760in}}%
\pgfpathlineto{\pgfqpoint{2.298324in}{1.557985in}}%
\pgfpathlineto{\pgfqpoint{2.300077in}{1.549815in}}%
\pgfpathlineto{\pgfqpoint{2.305699in}{1.546182in}}%
\pgfpathlineto{\pgfqpoint{2.308831in}{1.536139in}}%
\pgfpathlineto{\pgfqpoint{2.308393in}{1.511672in}}%
\pgfpathlineto{\pgfqpoint{2.300971in}{1.505827in}}%
\pgfpathlineto{\pgfqpoint{2.299305in}{1.492975in}}%
\pgfpathlineto{\pgfqpoint{2.285538in}{1.481098in}}%
\pgfpathlineto{\pgfqpoint{2.286355in}{1.466722in}}%
\pgfpathlineto{\pgfqpoint{2.298421in}{1.461685in}}%
\pgfpathlineto{\pgfqpoint{2.312034in}{1.479623in}}%
\pgfpathlineto{\pgfqpoint{2.313159in}{1.486129in}}%
\pgfpathlineto{\pgfqpoint{2.330136in}{1.496942in}}%
\pgfpathlineto{\pgfqpoint{2.340932in}{1.491933in}}%
\pgfpathlineto{\pgfqpoint{2.347707in}{1.480613in}}%
\pgfpathlineto{\pgfqpoint{2.358636in}{1.441310in}}%
\pgfpathlineto{\pgfqpoint{2.364428in}{1.428876in}}%
\pgfpathlineto{\pgfqpoint{2.362615in}{1.423735in}}%
\pgfpathlineto{\pgfqpoint{2.362791in}{1.406232in}}%
\pgfpathlineto{\pgfqpoint{2.352255in}{1.407912in}}%
\pgfpathlineto{\pgfqpoint{2.346305in}{1.394832in}}%
\pgfpathlineto{\pgfqpoint{2.345480in}{1.385940in}}%
\pgfpathlineto{\pgfqpoint{2.337522in}{1.380232in}}%
\pgfpathlineto{\pgfqpoint{2.335753in}{1.362891in}}%
\pgfpathlineto{\pgfqpoint{2.324180in}{1.340984in}}%
\pgfpathlineto{\pgfqpoint{2.260872in}{1.331556in}}%
\pgfpathlineto{\pgfqpoint{2.260498in}{1.335699in}}%
\pgfpathlineto{\pgfqpoint{2.218150in}{1.331236in}}%
\pgfpathlineto{\pgfqpoint{2.160936in}{1.325898in}}%
\pgfpathclose%
\pgfusepath{fill}%
\end{pgfscope}%
\begin{pgfscope}%
\pgfpathrectangle{\pgfqpoint{0.100000in}{0.100000in}}{\pgfqpoint{2.989028in}{1.913466in}}%
\pgfusepath{clip}%
\pgfsetbuttcap%
\pgfsetmiterjoin%
\definecolor{currentfill}{rgb}{0.400000,0.760784,0.647059}%
\pgfsetfillcolor{currentfill}%
\pgfsetlinewidth{0.000000pt}%
\definecolor{currentstroke}{rgb}{0.000000,0.000000,0.000000}%
\pgfsetstrokecolor{currentstroke}%
\pgfsetstrokeopacity{0.000000}%
\pgfsetdash{}{0pt}%
\pgfpathmoveto{\pgfqpoint{0.562031in}{0.446426in}}%
\pgfpathlineto{\pgfqpoint{0.560767in}{0.445314in}}%
\pgfpathlineto{\pgfqpoint{0.554967in}{0.446797in}}%
\pgfpathlineto{\pgfqpoint{0.555400in}{0.449716in}}%
\pgfpathlineto{\pgfqpoint{0.558683in}{0.451053in}}%
\pgfpathlineto{\pgfqpoint{0.563364in}{0.457782in}}%
\pgfpathlineto{\pgfqpoint{0.564142in}{0.462330in}}%
\pgfpathlineto{\pgfqpoint{0.566773in}{0.463522in}}%
\pgfpathlineto{\pgfqpoint{0.571054in}{0.463526in}}%
\pgfpathlineto{\pgfqpoint{0.575530in}{0.478285in}}%
\pgfpathlineto{\pgfqpoint{0.577283in}{0.479499in}}%
\pgfpathlineto{\pgfqpoint{0.575711in}{0.482384in}}%
\pgfpathlineto{\pgfqpoint{0.572030in}{0.479466in}}%
\pgfpathlineto{\pgfqpoint{0.561215in}{0.483400in}}%
\pgfpathlineto{\pgfqpoint{0.554760in}{0.490029in}}%
\pgfpathlineto{\pgfqpoint{0.557575in}{0.493053in}}%
\pgfpathlineto{\pgfqpoint{0.556295in}{0.497319in}}%
\pgfpathlineto{\pgfqpoint{0.557083in}{0.502359in}}%
\pgfpathlineto{\pgfqpoint{0.555695in}{0.508250in}}%
\pgfpathlineto{\pgfqpoint{0.555473in}{0.514148in}}%
\pgfpathlineto{\pgfqpoint{0.561686in}{0.513161in}}%
\pgfpathlineto{\pgfqpoint{0.562916in}{0.514899in}}%
\pgfpathlineto{\pgfqpoint{0.566006in}{0.514731in}}%
\pgfpathlineto{\pgfqpoint{0.569565in}{0.508441in}}%
\pgfpathlineto{\pgfqpoint{0.572996in}{0.503835in}}%
\pgfpathlineto{\pgfqpoint{0.570248in}{0.500837in}}%
\pgfpathlineto{\pgfqpoint{0.575000in}{0.499687in}}%
\pgfpathlineto{\pgfqpoint{0.576083in}{0.501158in}}%
\pgfpathlineto{\pgfqpoint{0.573542in}{0.507123in}}%
\pgfpathlineto{\pgfqpoint{0.568190in}{0.512068in}}%
\pgfpathlineto{\pgfqpoint{0.568845in}{0.514140in}}%
\pgfpathlineto{\pgfqpoint{0.564574in}{0.519376in}}%
\pgfpathlineto{\pgfqpoint{0.568935in}{0.520439in}}%
\pgfpathlineto{\pgfqpoint{0.565157in}{0.531650in}}%
\pgfpathlineto{\pgfqpoint{0.568229in}{0.533204in}}%
\pgfpathlineto{\pgfqpoint{0.568978in}{0.535892in}}%
\pgfpathlineto{\pgfqpoint{0.566980in}{0.539085in}}%
\pgfpathlineto{\pgfqpoint{0.571104in}{0.539884in}}%
\pgfpathlineto{\pgfqpoint{0.571845in}{0.542299in}}%
\pgfpathlineto{\pgfqpoint{0.576765in}{0.538674in}}%
\pgfpathlineto{\pgfqpoint{0.577857in}{0.542605in}}%
\pgfpathlineto{\pgfqpoint{0.580443in}{0.544334in}}%
\pgfpathlineto{\pgfqpoint{0.593023in}{0.546739in}}%
\pgfpathlineto{\pgfqpoint{0.596885in}{0.545889in}}%
\pgfpathlineto{\pgfqpoint{0.600120in}{0.550663in}}%
\pgfpathlineto{\pgfqpoint{0.607797in}{0.553794in}}%
\pgfpathlineto{\pgfqpoint{0.613710in}{0.552606in}}%
\pgfpathlineto{\pgfqpoint{0.616407in}{0.548449in}}%
\pgfpathlineto{\pgfqpoint{0.616266in}{0.541844in}}%
\pgfpathlineto{\pgfqpoint{0.619292in}{0.540347in}}%
\pgfpathlineto{\pgfqpoint{0.625464in}{0.540556in}}%
\pgfpathlineto{\pgfqpoint{0.633370in}{0.542274in}}%
\pgfpathlineto{\pgfqpoint{0.633272in}{0.540162in}}%
\pgfpathlineto{\pgfqpoint{0.643266in}{0.534139in}}%
\pgfpathlineto{\pgfqpoint{0.650255in}{0.535754in}}%
\pgfpathlineto{\pgfqpoint{0.652313in}{0.537844in}}%
\pgfpathlineto{\pgfqpoint{0.654516in}{0.543197in}}%
\pgfpathlineto{\pgfqpoint{0.657248in}{0.546625in}}%
\pgfpathlineto{\pgfqpoint{0.657492in}{0.551762in}}%
\pgfpathlineto{\pgfqpoint{0.654990in}{0.553782in}}%
\pgfpathlineto{\pgfqpoint{0.658695in}{0.555608in}}%
\pgfpathlineto{\pgfqpoint{0.664920in}{0.552865in}}%
\pgfpathlineto{\pgfqpoint{0.666948in}{0.554559in}}%
\pgfpathlineto{\pgfqpoint{0.667305in}{0.561826in}}%
\pgfpathlineto{\pgfqpoint{0.663281in}{0.560307in}}%
\pgfpathlineto{\pgfqpoint{0.660490in}{0.562825in}}%
\pgfpathlineto{\pgfqpoint{0.655960in}{0.564395in}}%
\pgfpathlineto{\pgfqpoint{0.648770in}{0.564133in}}%
\pgfpathlineto{\pgfqpoint{0.641880in}{0.566521in}}%
\pgfpathlineto{\pgfqpoint{0.641535in}{0.572469in}}%
\pgfpathlineto{\pgfqpoint{0.635373in}{0.578787in}}%
\pgfpathlineto{\pgfqpoint{0.627559in}{0.581439in}}%
\pgfpathlineto{\pgfqpoint{0.620739in}{0.593671in}}%
\pgfpathlineto{\pgfqpoint{0.621298in}{0.598197in}}%
\pgfpathlineto{\pgfqpoint{0.624971in}{0.600261in}}%
\pgfpathlineto{\pgfqpoint{0.624430in}{0.604545in}}%
\pgfpathlineto{\pgfqpoint{0.625282in}{0.608808in}}%
\pgfpathlineto{\pgfqpoint{0.628188in}{0.605890in}}%
\pgfpathlineto{\pgfqpoint{0.632025in}{0.607107in}}%
\pgfpathlineto{\pgfqpoint{0.631833in}{0.610810in}}%
\pgfpathlineto{\pgfqpoint{0.626514in}{0.618243in}}%
\pgfpathlineto{\pgfqpoint{0.624956in}{0.625911in}}%
\pgfpathlineto{\pgfqpoint{0.629409in}{0.626496in}}%
\pgfpathlineto{\pgfqpoint{0.636074in}{0.626025in}}%
\pgfpathlineto{\pgfqpoint{0.638172in}{0.623460in}}%
\pgfpathlineto{\pgfqpoint{0.644147in}{0.624378in}}%
\pgfpathlineto{\pgfqpoint{0.649700in}{0.623173in}}%
\pgfpathlineto{\pgfqpoint{0.653454in}{0.621055in}}%
\pgfpathlineto{\pgfqpoint{0.656052in}{0.622023in}}%
\pgfpathlineto{\pgfqpoint{0.662951in}{0.619906in}}%
\pgfpathlineto{\pgfqpoint{0.663014in}{0.618032in}}%
\pgfpathlineto{\pgfqpoint{0.668268in}{0.618665in}}%
\pgfpathlineto{\pgfqpoint{0.675313in}{0.614092in}}%
\pgfpathlineto{\pgfqpoint{0.675039in}{0.611057in}}%
\pgfpathlineto{\pgfqpoint{0.671687in}{0.609656in}}%
\pgfpathlineto{\pgfqpoint{0.668124in}{0.606323in}}%
\pgfpathlineto{\pgfqpoint{0.667694in}{0.601746in}}%
\pgfpathlineto{\pgfqpoint{0.675369in}{0.595909in}}%
\pgfpathlineto{\pgfqpoint{0.674141in}{0.593112in}}%
\pgfpathlineto{\pgfqpoint{0.679602in}{0.590964in}}%
\pgfpathlineto{\pgfqpoint{0.680943in}{0.587455in}}%
\pgfpathlineto{\pgfqpoint{0.687873in}{0.590243in}}%
\pgfpathlineto{\pgfqpoint{0.690925in}{0.586589in}}%
\pgfpathlineto{\pgfqpoint{0.692334in}{0.588801in}}%
\pgfpathlineto{\pgfqpoint{0.688186in}{0.595999in}}%
\pgfpathlineto{\pgfqpoint{0.689660in}{0.598155in}}%
\pgfpathlineto{\pgfqpoint{0.690313in}{0.603841in}}%
\pgfpathlineto{\pgfqpoint{0.688661in}{0.606342in}}%
\pgfpathlineto{\pgfqpoint{0.689901in}{0.609704in}}%
\pgfpathlineto{\pgfqpoint{0.693961in}{0.609388in}}%
\pgfpathlineto{\pgfqpoint{0.693121in}{0.603668in}}%
\pgfpathlineto{\pgfqpoint{0.690529in}{0.601760in}}%
\pgfpathlineto{\pgfqpoint{0.690712in}{0.594065in}}%
\pgfpathlineto{\pgfqpoint{0.695364in}{0.593146in}}%
\pgfpathlineto{\pgfqpoint{0.695421in}{0.587424in}}%
\pgfpathlineto{\pgfqpoint{0.700501in}{0.583713in}}%
\pgfpathlineto{\pgfqpoint{0.702836in}{0.586053in}}%
\pgfpathlineto{\pgfqpoint{0.700284in}{0.593176in}}%
\pgfpathlineto{\pgfqpoint{0.697644in}{0.594979in}}%
\pgfpathlineto{\pgfqpoint{0.695047in}{0.593793in}}%
\pgfpathlineto{\pgfqpoint{0.692763in}{0.595306in}}%
\pgfpathlineto{\pgfqpoint{0.693161in}{0.601889in}}%
\pgfpathlineto{\pgfqpoint{0.696974in}{0.604581in}}%
\pgfpathlineto{\pgfqpoint{0.700726in}{0.604333in}}%
\pgfpathlineto{\pgfqpoint{0.699172in}{0.608058in}}%
\pgfpathlineto{\pgfqpoint{0.693547in}{0.610911in}}%
\pgfpathlineto{\pgfqpoint{0.686098in}{0.622277in}}%
\pgfpathlineto{\pgfqpoint{0.689816in}{0.627728in}}%
\pgfpathlineto{\pgfqpoint{0.691885in}{0.632863in}}%
\pgfpathlineto{\pgfqpoint{0.692048in}{0.643174in}}%
\pgfpathlineto{\pgfqpoint{0.691251in}{0.651872in}}%
\pgfpathlineto{\pgfqpoint{0.688978in}{0.657584in}}%
\pgfpathlineto{\pgfqpoint{0.689431in}{0.663294in}}%
\pgfpathlineto{\pgfqpoint{0.695244in}{0.668247in}}%
\pgfpathlineto{\pgfqpoint{0.701026in}{0.674133in}}%
\pgfpathlineto{\pgfqpoint{0.715519in}{0.662108in}}%
\pgfpathlineto{\pgfqpoint{0.724995in}{0.661221in}}%
\pgfpathlineto{\pgfqpoint{0.732817in}{0.664216in}}%
\pgfpathlineto{\pgfqpoint{0.739683in}{0.668159in}}%
\pgfpathlineto{\pgfqpoint{0.741141in}{0.670038in}}%
\pgfpathlineto{\pgfqpoint{0.758090in}{0.674322in}}%
\pgfpathlineto{\pgfqpoint{0.759562in}{0.671184in}}%
\pgfpathlineto{\pgfqpoint{0.765321in}{0.668615in}}%
\pgfpathlineto{\pgfqpoint{0.776313in}{0.669403in}}%
\pgfpathlineto{\pgfqpoint{0.778320in}{0.668702in}}%
\pgfpathlineto{\pgfqpoint{0.783260in}{0.670576in}}%
\pgfpathlineto{\pgfqpoint{0.785246in}{0.667335in}}%
\pgfpathlineto{\pgfqpoint{0.789990in}{0.664374in}}%
\pgfpathlineto{\pgfqpoint{0.792363in}{0.664709in}}%
\pgfpathlineto{\pgfqpoint{0.796956in}{0.661151in}}%
\pgfpathlineto{\pgfqpoint{0.804717in}{0.661805in}}%
\pgfpathlineto{\pgfqpoint{0.812469in}{0.664685in}}%
\pgfpathlineto{\pgfqpoint{0.818800in}{0.655475in}}%
\pgfpathlineto{\pgfqpoint{0.818309in}{0.653408in}}%
\pgfpathlineto{\pgfqpoint{0.811323in}{0.651032in}}%
\pgfpathlineto{\pgfqpoint{0.813190in}{0.647440in}}%
\pgfpathlineto{\pgfqpoint{0.819621in}{0.651221in}}%
\pgfpathlineto{\pgfqpoint{0.822246in}{0.650305in}}%
\pgfpathlineto{\pgfqpoint{0.823363in}{0.645783in}}%
\pgfpathlineto{\pgfqpoint{0.819524in}{0.643870in}}%
\pgfpathlineto{\pgfqpoint{0.822266in}{0.638154in}}%
\pgfpathlineto{\pgfqpoint{0.826248in}{0.639120in}}%
\pgfpathlineto{\pgfqpoint{0.832026in}{0.636170in}}%
\pgfpathlineto{\pgfqpoint{0.837551in}{0.627982in}}%
\pgfpathlineto{\pgfqpoint{0.832689in}{0.625708in}}%
\pgfpathlineto{\pgfqpoint{0.836985in}{0.619675in}}%
\pgfpathlineto{\pgfqpoint{0.833366in}{0.618264in}}%
\pgfpathlineto{\pgfqpoint{0.837868in}{0.612534in}}%
\pgfpathlineto{\pgfqpoint{0.843358in}{0.613145in}}%
\pgfpathlineto{\pgfqpoint{0.846249in}{0.610567in}}%
\pgfpathlineto{\pgfqpoint{0.853562in}{0.606765in}}%
\pgfpathlineto{\pgfqpoint{0.862995in}{0.593271in}}%
\pgfpathlineto{\pgfqpoint{0.862782in}{0.590043in}}%
\pgfpathlineto{\pgfqpoint{0.876928in}{0.579706in}}%
\pgfpathlineto{\pgfqpoint{0.877233in}{0.576049in}}%
\pgfpathlineto{\pgfqpoint{0.881003in}{0.570578in}}%
\pgfpathlineto{\pgfqpoint{0.892585in}{0.567843in}}%
\pgfpathlineto{\pgfqpoint{0.896881in}{0.565899in}}%
\pgfpathlineto{\pgfqpoint{0.900922in}{0.559868in}}%
\pgfpathlineto{\pgfqpoint{0.901442in}{0.554585in}}%
\pgfpathlineto{\pgfqpoint{0.904915in}{0.550279in}}%
\pgfpathlineto{\pgfqpoint{0.905958in}{0.545423in}}%
\pgfpathlineto{\pgfqpoint{0.909213in}{0.543649in}}%
\pgfpathlineto{\pgfqpoint{0.893089in}{0.514548in}}%
\pgfpathlineto{\pgfqpoint{0.859622in}{0.454133in}}%
\pgfpathlineto{\pgfqpoint{0.810518in}{0.365475in}}%
\pgfpathlineto{\pgfqpoint{0.791294in}{0.330756in}}%
\pgfpathlineto{\pgfqpoint{0.795388in}{0.326093in}}%
\pgfpathlineto{\pgfqpoint{0.797192in}{0.327581in}}%
\pgfpathlineto{\pgfqpoint{0.800850in}{0.322152in}}%
\pgfpathlineto{\pgfqpoint{0.805927in}{0.323727in}}%
\pgfpathlineto{\pgfqpoint{0.812667in}{0.320475in}}%
\pgfpathlineto{\pgfqpoint{0.808301in}{0.315393in}}%
\pgfpathlineto{\pgfqpoint{0.811672in}{0.308535in}}%
\pgfpathlineto{\pgfqpoint{0.810976in}{0.305196in}}%
\pgfpathlineto{\pgfqpoint{0.816353in}{0.287587in}}%
\pgfpathlineto{\pgfqpoint{0.814077in}{0.279580in}}%
\pgfpathlineto{\pgfqpoint{0.824350in}{0.281562in}}%
\pgfpathlineto{\pgfqpoint{0.826872in}{0.280263in}}%
\pgfpathlineto{\pgfqpoint{0.829610in}{0.282376in}}%
\pgfpathlineto{\pgfqpoint{0.831558in}{0.286368in}}%
\pgfpathlineto{\pgfqpoint{0.834347in}{0.288657in}}%
\pgfpathlineto{\pgfqpoint{0.846252in}{0.288367in}}%
\pgfpathlineto{\pgfqpoint{0.848943in}{0.280683in}}%
\pgfpathlineto{\pgfqpoint{0.846634in}{0.273938in}}%
\pgfpathlineto{\pgfqpoint{0.849257in}{0.271712in}}%
\pgfpathlineto{\pgfqpoint{0.850636in}{0.267905in}}%
\pgfpathlineto{\pgfqpoint{0.850336in}{0.260746in}}%
\pgfpathlineto{\pgfqpoint{0.853836in}{0.255613in}}%
\pgfpathlineto{\pgfqpoint{0.855755in}{0.247568in}}%
\pgfpathlineto{\pgfqpoint{0.854770in}{0.243223in}}%
\pgfpathlineto{\pgfqpoint{0.855958in}{0.238819in}}%
\pgfpathlineto{\pgfqpoint{0.857671in}{0.213318in}}%
\pgfpathlineto{\pgfqpoint{0.855216in}{0.211309in}}%
\pgfpathlineto{\pgfqpoint{0.858455in}{0.208533in}}%
\pgfpathlineto{\pgfqpoint{0.855910in}{0.205157in}}%
\pgfpathlineto{\pgfqpoint{0.858206in}{0.202405in}}%
\pgfpathlineto{\pgfqpoint{0.856727in}{0.197707in}}%
\pgfpathlineto{\pgfqpoint{0.860034in}{0.196433in}}%
\pgfpathlineto{\pgfqpoint{0.864287in}{0.189287in}}%
\pgfpathlineto{\pgfqpoint{0.867447in}{0.187134in}}%
\pgfpathlineto{\pgfqpoint{0.868274in}{0.183987in}}%
\pgfpathlineto{\pgfqpoint{0.870105in}{0.182623in}}%
\pgfpathlineto{\pgfqpoint{0.869723in}{0.180010in}}%
\pgfpathlineto{\pgfqpoint{0.873670in}{0.178116in}}%
\pgfpathlineto{\pgfqpoint{0.872696in}{0.173088in}}%
\pgfpathlineto{\pgfqpoint{0.869298in}{0.170454in}}%
\pgfpathlineto{\pgfqpoint{0.867755in}{0.165560in}}%
\pgfpathlineto{\pgfqpoint{0.867345in}{0.159006in}}%
\pgfpathlineto{\pgfqpoint{0.865458in}{0.158379in}}%
\pgfpathlineto{\pgfqpoint{0.859289in}{0.152634in}}%
\pgfpathlineto{\pgfqpoint{0.853806in}{0.151118in}}%
\pgfpathlineto{\pgfqpoint{0.851190in}{0.153435in}}%
\pgfpathlineto{\pgfqpoint{0.853718in}{0.159583in}}%
\pgfpathlineto{\pgfqpoint{0.852946in}{0.162601in}}%
\pgfpathlineto{\pgfqpoint{0.856568in}{0.164131in}}%
\pgfpathlineto{\pgfqpoint{0.859956in}{0.173027in}}%
\pgfpathlineto{\pgfqpoint{0.858525in}{0.180625in}}%
\pgfpathlineto{\pgfqpoint{0.856235in}{0.182419in}}%
\pgfpathlineto{\pgfqpoint{0.848578in}{0.181838in}}%
\pgfpathlineto{\pgfqpoint{0.850121in}{0.179853in}}%
\pgfpathlineto{\pgfqpoint{0.844579in}{0.174155in}}%
\pgfpathlineto{\pgfqpoint{0.842793in}{0.177323in}}%
\pgfpathlineto{\pgfqpoint{0.843258in}{0.181462in}}%
\pgfpathlineto{\pgfqpoint{0.846439in}{0.181386in}}%
\pgfpathlineto{\pgfqpoint{0.849272in}{0.184426in}}%
\pgfpathlineto{\pgfqpoint{0.850948in}{0.188866in}}%
\pgfpathlineto{\pgfqpoint{0.857264in}{0.187613in}}%
\pgfpathlineto{\pgfqpoint{0.852208in}{0.189984in}}%
\pgfpathlineto{\pgfqpoint{0.852653in}{0.193161in}}%
\pgfpathlineto{\pgfqpoint{0.850549in}{0.194596in}}%
\pgfpathlineto{\pgfqpoint{0.849687in}{0.198923in}}%
\pgfpathlineto{\pgfqpoint{0.851181in}{0.201177in}}%
\pgfpathlineto{\pgfqpoint{0.847619in}{0.208179in}}%
\pgfpathlineto{\pgfqpoint{0.847919in}{0.212633in}}%
\pgfpathlineto{\pgfqpoint{0.843746in}{0.216599in}}%
\pgfpathlineto{\pgfqpoint{0.841488in}{0.219808in}}%
\pgfpathlineto{\pgfqpoint{0.844606in}{0.222866in}}%
\pgfpathlineto{\pgfqpoint{0.845761in}{0.227437in}}%
\pgfpathlineto{\pgfqpoint{0.844730in}{0.230438in}}%
\pgfpathlineto{\pgfqpoint{0.845936in}{0.234017in}}%
\pgfpathlineto{\pgfqpoint{0.844399in}{0.245686in}}%
\pgfpathlineto{\pgfqpoint{0.842100in}{0.250980in}}%
\pgfpathlineto{\pgfqpoint{0.839471in}{0.252970in}}%
\pgfpathlineto{\pgfqpoint{0.840590in}{0.259857in}}%
\pgfpathlineto{\pgfqpoint{0.839801in}{0.265235in}}%
\pgfpathlineto{\pgfqpoint{0.841230in}{0.268685in}}%
\pgfpathlineto{\pgfqpoint{0.838272in}{0.269169in}}%
\pgfpathlineto{\pgfqpoint{0.837479in}{0.260789in}}%
\pgfpathlineto{\pgfqpoint{0.834154in}{0.251441in}}%
\pgfpathlineto{\pgfqpoint{0.831484in}{0.253570in}}%
\pgfpathlineto{\pgfqpoint{0.831127in}{0.256642in}}%
\pgfpathlineto{\pgfqpoint{0.827861in}{0.260839in}}%
\pgfpathlineto{\pgfqpoint{0.829521in}{0.263587in}}%
\pgfpathlineto{\pgfqpoint{0.829005in}{0.269772in}}%
\pgfpathlineto{\pgfqpoint{0.824778in}{0.267597in}}%
\pgfpathlineto{\pgfqpoint{0.825607in}{0.264242in}}%
\pgfpathlineto{\pgfqpoint{0.824811in}{0.259983in}}%
\pgfpathlineto{\pgfqpoint{0.820066in}{0.259990in}}%
\pgfpathlineto{\pgfqpoint{0.817939in}{0.262630in}}%
\pgfpathlineto{\pgfqpoint{0.814729in}{0.263134in}}%
\pgfpathlineto{\pgfqpoint{0.812569in}{0.265977in}}%
\pgfpathlineto{\pgfqpoint{0.808852in}{0.273782in}}%
\pgfpathlineto{\pgfqpoint{0.807648in}{0.279842in}}%
\pgfpathlineto{\pgfqpoint{0.808418in}{0.282418in}}%
\pgfpathlineto{\pgfqpoint{0.806620in}{0.288091in}}%
\pgfpathlineto{\pgfqpoint{0.803639in}{0.291375in}}%
\pgfpathlineto{\pgfqpoint{0.798483in}{0.299768in}}%
\pgfpathlineto{\pgfqpoint{0.794744in}{0.307057in}}%
\pgfpathlineto{\pgfqpoint{0.798559in}{0.307167in}}%
\pgfpathlineto{\pgfqpoint{0.800804in}{0.308698in}}%
\pgfpathlineto{\pgfqpoint{0.801246in}{0.313435in}}%
\pgfpathlineto{\pgfqpoint{0.790450in}{0.314007in}}%
\pgfpathlineto{\pgfqpoint{0.781267in}{0.324264in}}%
\pgfpathlineto{\pgfqpoint{0.783682in}{0.326962in}}%
\pgfpathlineto{\pgfqpoint{0.779241in}{0.327423in}}%
\pgfpathlineto{\pgfqpoint{0.770423in}{0.336959in}}%
\pgfpathlineto{\pgfqpoint{0.756451in}{0.341741in}}%
\pgfpathlineto{\pgfqpoint{0.754570in}{0.347345in}}%
\pgfpathlineto{\pgfqpoint{0.751671in}{0.350264in}}%
\pgfpathlineto{\pgfqpoint{0.751539in}{0.353005in}}%
\pgfpathlineto{\pgfqpoint{0.749358in}{0.355041in}}%
\pgfpathlineto{\pgfqpoint{0.752220in}{0.358442in}}%
\pgfpathlineto{\pgfqpoint{0.745693in}{0.358770in}}%
\pgfpathlineto{\pgfqpoint{0.744034in}{0.363173in}}%
\pgfpathlineto{\pgfqpoint{0.741063in}{0.365047in}}%
\pgfpathlineto{\pgfqpoint{0.747012in}{0.367309in}}%
\pgfpathlineto{\pgfqpoint{0.743918in}{0.371345in}}%
\pgfpathlineto{\pgfqpoint{0.738224in}{0.373038in}}%
\pgfpathlineto{\pgfqpoint{0.739110in}{0.380491in}}%
\pgfpathlineto{\pgfqpoint{0.737478in}{0.383130in}}%
\pgfpathlineto{\pgfqpoint{0.734538in}{0.384988in}}%
\pgfpathlineto{\pgfqpoint{0.732024in}{0.383124in}}%
\pgfpathlineto{\pgfqpoint{0.730538in}{0.384877in}}%
\pgfpathlineto{\pgfqpoint{0.726206in}{0.385379in}}%
\pgfpathlineto{\pgfqpoint{0.726555in}{0.392143in}}%
\pgfpathlineto{\pgfqpoint{0.720275in}{0.387574in}}%
\pgfpathlineto{\pgfqpoint{0.720407in}{0.378160in}}%
\pgfpathlineto{\pgfqpoint{0.713666in}{0.376355in}}%
\pgfpathlineto{\pgfqpoint{0.709263in}{0.372596in}}%
\pgfpathlineto{\pgfqpoint{0.705725in}{0.371804in}}%
\pgfpathlineto{\pgfqpoint{0.701895in}{0.375593in}}%
\pgfpathlineto{\pgfqpoint{0.698485in}{0.375090in}}%
\pgfpathlineto{\pgfqpoint{0.699118in}{0.378551in}}%
\pgfpathlineto{\pgfqpoint{0.694270in}{0.376688in}}%
\pgfpathlineto{\pgfqpoint{0.689779in}{0.376478in}}%
\pgfpathlineto{\pgfqpoint{0.684523in}{0.374589in}}%
\pgfpathlineto{\pgfqpoint{0.673855in}{0.375652in}}%
\pgfpathlineto{\pgfqpoint{0.670430in}{0.374953in}}%
\pgfpathlineto{\pgfqpoint{0.667959in}{0.378065in}}%
\pgfpathlineto{\pgfqpoint{0.663397in}{0.378376in}}%
\pgfpathlineto{\pgfqpoint{0.662449in}{0.382409in}}%
\pgfpathlineto{\pgfqpoint{0.665167in}{0.384606in}}%
\pgfpathlineto{\pgfqpoint{0.668357in}{0.384399in}}%
\pgfpathlineto{\pgfqpoint{0.671019in}{0.381403in}}%
\pgfpathlineto{\pgfqpoint{0.672772in}{0.386005in}}%
\pgfpathlineto{\pgfqpoint{0.670326in}{0.391102in}}%
\pgfpathlineto{\pgfqpoint{0.675863in}{0.395589in}}%
\pgfpathlineto{\pgfqpoint{0.681411in}{0.397243in}}%
\pgfpathlineto{\pgfqpoint{0.685273in}{0.399940in}}%
\pgfpathlineto{\pgfqpoint{0.687775in}{0.402846in}}%
\pgfpathlineto{\pgfqpoint{0.689172in}{0.407609in}}%
\pgfpathlineto{\pgfqpoint{0.693583in}{0.406312in}}%
\pgfpathlineto{\pgfqpoint{0.703148in}{0.406979in}}%
\pgfpathlineto{\pgfqpoint{0.705201in}{0.401756in}}%
\pgfpathlineto{\pgfqpoint{0.708556in}{0.401529in}}%
\pgfpathlineto{\pgfqpoint{0.714662in}{0.393534in}}%
\pgfpathlineto{\pgfqpoint{0.714673in}{0.396448in}}%
\pgfpathlineto{\pgfqpoint{0.712540in}{0.397396in}}%
\pgfpathlineto{\pgfqpoint{0.708636in}{0.406940in}}%
\pgfpathlineto{\pgfqpoint{0.710536in}{0.408236in}}%
\pgfpathlineto{\pgfqpoint{0.705322in}{0.413215in}}%
\pgfpathlineto{\pgfqpoint{0.700663in}{0.414184in}}%
\pgfpathlineto{\pgfqpoint{0.696223in}{0.412463in}}%
\pgfpathlineto{\pgfqpoint{0.688619in}{0.413804in}}%
\pgfpathlineto{\pgfqpoint{0.685013in}{0.411392in}}%
\pgfpathlineto{\pgfqpoint{0.676801in}{0.409529in}}%
\pgfpathlineto{\pgfqpoint{0.676050in}{0.406847in}}%
\pgfpathlineto{\pgfqpoint{0.669741in}{0.406115in}}%
\pgfpathlineto{\pgfqpoint{0.668122in}{0.403049in}}%
\pgfpathlineto{\pgfqpoint{0.664513in}{0.400803in}}%
\pgfpathlineto{\pgfqpoint{0.660154in}{0.401189in}}%
\pgfpathlineto{\pgfqpoint{0.657977in}{0.399018in}}%
\pgfpathlineto{\pgfqpoint{0.655231in}{0.399119in}}%
\pgfpathlineto{\pgfqpoint{0.652607in}{0.401074in}}%
\pgfpathlineto{\pgfqpoint{0.648019in}{0.400725in}}%
\pgfpathlineto{\pgfqpoint{0.646938in}{0.398924in}}%
\pgfpathlineto{\pgfqpoint{0.642119in}{0.400773in}}%
\pgfpathlineto{\pgfqpoint{0.638027in}{0.396000in}}%
\pgfpathlineto{\pgfqpoint{0.641950in}{0.391382in}}%
\pgfpathlineto{\pgfqpoint{0.643460in}{0.387584in}}%
\pgfpathlineto{\pgfqpoint{0.642068in}{0.384367in}}%
\pgfpathlineto{\pgfqpoint{0.636493in}{0.382148in}}%
\pgfpathlineto{\pgfqpoint{0.633263in}{0.383954in}}%
\pgfpathlineto{\pgfqpoint{0.629198in}{0.382976in}}%
\pgfpathlineto{\pgfqpoint{0.628708in}{0.380157in}}%
\pgfpathlineto{\pgfqpoint{0.623759in}{0.381154in}}%
\pgfpathlineto{\pgfqpoint{0.624855in}{0.378337in}}%
\pgfpathlineto{\pgfqpoint{0.617555in}{0.377743in}}%
\pgfpathlineto{\pgfqpoint{0.612830in}{0.381319in}}%
\pgfpathlineto{\pgfqpoint{0.610306in}{0.379045in}}%
\pgfpathlineto{\pgfqpoint{0.603590in}{0.381466in}}%
\pgfpathlineto{\pgfqpoint{0.601918in}{0.379014in}}%
\pgfpathlineto{\pgfqpoint{0.598588in}{0.377923in}}%
\pgfpathlineto{\pgfqpoint{0.595917in}{0.380797in}}%
\pgfpathlineto{\pgfqpoint{0.586784in}{0.380083in}}%
\pgfpathlineto{\pgfqpoint{0.586794in}{0.375843in}}%
\pgfpathlineto{\pgfqpoint{0.583125in}{0.374721in}}%
\pgfpathlineto{\pgfqpoint{0.580270in}{0.376900in}}%
\pgfpathlineto{\pgfqpoint{0.576497in}{0.375689in}}%
\pgfpathlineto{\pgfqpoint{0.571614in}{0.375401in}}%
\pgfpathlineto{\pgfqpoint{0.564340in}{0.378466in}}%
\pgfpathlineto{\pgfqpoint{0.560174in}{0.375305in}}%
\pgfpathlineto{\pgfqpoint{0.554216in}{0.379530in}}%
\pgfpathlineto{\pgfqpoint{0.552268in}{0.377929in}}%
\pgfpathlineto{\pgfqpoint{0.550900in}{0.373791in}}%
\pgfpathlineto{\pgfqpoint{0.547071in}{0.371365in}}%
\pgfpathlineto{\pgfqpoint{0.540825in}{0.374142in}}%
\pgfpathlineto{\pgfqpoint{0.530479in}{0.381003in}}%
\pgfpathlineto{\pgfqpoint{0.527463in}{0.381613in}}%
\pgfpathlineto{\pgfqpoint{0.525762in}{0.380183in}}%
\pgfpathlineto{\pgfqpoint{0.521211in}{0.381155in}}%
\pgfpathlineto{\pgfqpoint{0.518991in}{0.383290in}}%
\pgfpathlineto{\pgfqpoint{0.514445in}{0.384441in}}%
\pgfpathlineto{\pgfqpoint{0.508238in}{0.384507in}}%
\pgfpathlineto{\pgfqpoint{0.505773in}{0.387054in}}%
\pgfpathlineto{\pgfqpoint{0.510497in}{0.389690in}}%
\pgfpathlineto{\pgfqpoint{0.509395in}{0.392291in}}%
\pgfpathlineto{\pgfqpoint{0.506149in}{0.391707in}}%
\pgfpathlineto{\pgfqpoint{0.504528in}{0.389662in}}%
\pgfpathlineto{\pgfqpoint{0.496930in}{0.386730in}}%
\pgfpathlineto{\pgfqpoint{0.493088in}{0.387727in}}%
\pgfpathlineto{\pgfqpoint{0.491471in}{0.393681in}}%
\pgfpathlineto{\pgfqpoint{0.486679in}{0.391121in}}%
\pgfpathlineto{\pgfqpoint{0.485286in}{0.398667in}}%
\pgfpathlineto{\pgfqpoint{0.483015in}{0.396722in}}%
\pgfpathlineto{\pgfqpoint{0.483395in}{0.393819in}}%
\pgfpathlineto{\pgfqpoint{0.478173in}{0.394586in}}%
\pgfpathlineto{\pgfqpoint{0.483605in}{0.399646in}}%
\pgfpathlineto{\pgfqpoint{0.489110in}{0.396639in}}%
\pgfpathlineto{\pgfqpoint{0.490291in}{0.398148in}}%
\pgfpathlineto{\pgfqpoint{0.496082in}{0.396339in}}%
\pgfpathlineto{\pgfqpoint{0.495720in}{0.398399in}}%
\pgfpathlineto{\pgfqpoint{0.511521in}{0.399081in}}%
\pgfpathlineto{\pgfqpoint{0.519230in}{0.395868in}}%
\pgfpathlineto{\pgfqpoint{0.522212in}{0.392698in}}%
\pgfpathlineto{\pgfqpoint{0.521327in}{0.389931in}}%
\pgfpathlineto{\pgfqpoint{0.524323in}{0.388883in}}%
\pgfpathlineto{\pgfqpoint{0.531765in}{0.393251in}}%
\pgfpathlineto{\pgfqpoint{0.541392in}{0.393531in}}%
\pgfpathlineto{\pgfqpoint{0.550136in}{0.390898in}}%
\pgfpathlineto{\pgfqpoint{0.555089in}{0.391340in}}%
\pgfpathlineto{\pgfqpoint{0.556933in}{0.387255in}}%
\pgfpathlineto{\pgfqpoint{0.559703in}{0.391758in}}%
\pgfpathlineto{\pgfqpoint{0.561407in}{0.392733in}}%
\pgfpathlineto{\pgfqpoint{0.569288in}{0.394691in}}%
\pgfpathlineto{\pgfqpoint{0.572217in}{0.393494in}}%
\pgfpathlineto{\pgfqpoint{0.575385in}{0.394915in}}%
\pgfpathlineto{\pgfqpoint{0.579163in}{0.394172in}}%
\pgfpathlineto{\pgfqpoint{0.580055in}{0.395827in}}%
\pgfpathlineto{\pgfqpoint{0.587939in}{0.403420in}}%
\pgfpathlineto{\pgfqpoint{0.591278in}{0.404545in}}%
\pgfpathlineto{\pgfqpoint{0.593251in}{0.408644in}}%
\pgfpathlineto{\pgfqpoint{0.603722in}{0.411817in}}%
\pgfpathlineto{\pgfqpoint{0.605033in}{0.414238in}}%
\pgfpathlineto{\pgfqpoint{0.590192in}{0.417974in}}%
\pgfpathlineto{\pgfqpoint{0.590776in}{0.421993in}}%
\pgfpathlineto{\pgfqpoint{0.589250in}{0.427301in}}%
\pgfpathlineto{\pgfqpoint{0.585794in}{0.426860in}}%
\pgfpathlineto{\pgfqpoint{0.583443in}{0.419972in}}%
\pgfpathlineto{\pgfqpoint{0.580714in}{0.419404in}}%
\pgfpathlineto{\pgfqpoint{0.578938in}{0.421634in}}%
\pgfpathlineto{\pgfqpoint{0.581010in}{0.431388in}}%
\pgfpathlineto{\pgfqpoint{0.578947in}{0.435475in}}%
\pgfpathlineto{\pgfqpoint{0.574950in}{0.440293in}}%
\pgfpathlineto{\pgfqpoint{0.576871in}{0.443969in}}%
\pgfpathlineto{\pgfqpoint{0.568436in}{0.444102in}}%
\pgfpathlineto{\pgfqpoint{0.568628in}{0.445528in}}%
\pgfpathlineto{\pgfqpoint{0.563595in}{0.447526in}}%
\pgfpathlineto{\pgfqpoint{0.562031in}{0.446426in}}%
\pgfpathclose%
\pgfusepath{fill}%
\end{pgfscope}%
\begin{pgfscope}%
\pgfpathrectangle{\pgfqpoint{0.100000in}{0.100000in}}{\pgfqpoint{2.989028in}{1.913466in}}%
\pgfusepath{clip}%
\pgfsetbuttcap%
\pgfsetmiterjoin%
\definecolor{currentfill}{rgb}{0.400000,0.760784,0.647059}%
\pgfsetfillcolor{currentfill}%
\pgfsetlinewidth{0.000000pt}%
\definecolor{currentstroke}{rgb}{0.000000,0.000000,0.000000}%
\pgfsetstrokecolor{currentstroke}%
\pgfsetstrokeopacity{0.000000}%
\pgfsetdash{}{0pt}%
\pgfpathmoveto{\pgfqpoint{0.546979in}{0.517466in}}%
\pgfpathlineto{\pgfqpoint{0.545936in}{0.515633in}}%
\pgfpathlineto{\pgfqpoint{0.548696in}{0.511730in}}%
\pgfpathlineto{\pgfqpoint{0.544334in}{0.508079in}}%
\pgfpathlineto{\pgfqpoint{0.544455in}{0.504939in}}%
\pgfpathlineto{\pgfqpoint{0.542483in}{0.504198in}}%
\pgfpathlineto{\pgfqpoint{0.535704in}{0.508881in}}%
\pgfpathlineto{\pgfqpoint{0.530872in}{0.519132in}}%
\pgfpathlineto{\pgfqpoint{0.530549in}{0.522094in}}%
\pgfpathlineto{\pgfqpoint{0.531791in}{0.525640in}}%
\pgfpathlineto{\pgfqpoint{0.537027in}{0.520216in}}%
\pgfpathlineto{\pgfqpoint{0.538628in}{0.521262in}}%
\pgfpathlineto{\pgfqpoint{0.543070in}{0.520267in}}%
\pgfpathlineto{\pgfqpoint{0.546979in}{0.517466in}}%
\pgfpathclose%
\pgfusepath{fill}%
\end{pgfscope}%
\begin{pgfscope}%
\pgfpathrectangle{\pgfqpoint{0.100000in}{0.100000in}}{\pgfqpoint{2.989028in}{1.913466in}}%
\pgfusepath{clip}%
\pgfsetbuttcap%
\pgfsetmiterjoin%
\definecolor{currentfill}{rgb}{0.400000,0.760784,0.647059}%
\pgfsetfillcolor{currentfill}%
\pgfsetlinewidth{0.000000pt}%
\definecolor{currentstroke}{rgb}{0.000000,0.000000,0.000000}%
\pgfsetstrokecolor{currentstroke}%
\pgfsetstrokeopacity{0.000000}%
\pgfsetdash{}{0pt}%
\pgfpathmoveto{\pgfqpoint{0.466584in}{0.398705in}}%
\pgfpathlineto{\pgfqpoint{0.461820in}{0.397932in}}%
\pgfpathlineto{\pgfqpoint{0.458372in}{0.399525in}}%
\pgfpathlineto{\pgfqpoint{0.456865in}{0.402120in}}%
\pgfpathlineto{\pgfqpoint{0.458573in}{0.405842in}}%
\pgfpathlineto{\pgfqpoint{0.462377in}{0.404858in}}%
\pgfpathlineto{\pgfqpoint{0.468630in}{0.407066in}}%
\pgfpathlineto{\pgfqpoint{0.470820in}{0.404016in}}%
\pgfpathlineto{\pgfqpoint{0.472944in}{0.404579in}}%
\pgfpathlineto{\pgfqpoint{0.478084in}{0.402681in}}%
\pgfpathlineto{\pgfqpoint{0.480305in}{0.400269in}}%
\pgfpathlineto{\pgfqpoint{0.477595in}{0.393963in}}%
\pgfpathlineto{\pgfqpoint{0.474899in}{0.392626in}}%
\pgfpathlineto{\pgfqpoint{0.472476in}{0.393346in}}%
\pgfpathlineto{\pgfqpoint{0.466584in}{0.398705in}}%
\pgfpathclose%
\pgfusepath{fill}%
\end{pgfscope}%
\begin{pgfscope}%
\pgfpathrectangle{\pgfqpoint{0.100000in}{0.100000in}}{\pgfqpoint{2.989028in}{1.913466in}}%
\pgfusepath{clip}%
\pgfsetbuttcap%
\pgfsetmiterjoin%
\definecolor{currentfill}{rgb}{0.400000,0.760784,0.647059}%
\pgfsetfillcolor{currentfill}%
\pgfsetlinewidth{0.000000pt}%
\definecolor{currentstroke}{rgb}{0.000000,0.000000,0.000000}%
\pgfsetstrokecolor{currentstroke}%
\pgfsetstrokeopacity{0.000000}%
\pgfsetdash{}{0pt}%
\pgfpathmoveto{\pgfqpoint{0.423985in}{0.404436in}}%
\pgfpathlineto{\pgfqpoint{0.417234in}{0.407656in}}%
\pgfpathlineto{\pgfqpoint{0.410987in}{0.408614in}}%
\pgfpathlineto{\pgfqpoint{0.409018in}{0.409985in}}%
\pgfpathlineto{\pgfqpoint{0.411260in}{0.412685in}}%
\pgfpathlineto{\pgfqpoint{0.415239in}{0.409312in}}%
\pgfpathlineto{\pgfqpoint{0.418576in}{0.409777in}}%
\pgfpathlineto{\pgfqpoint{0.424004in}{0.412113in}}%
\pgfpathlineto{\pgfqpoint{0.424442in}{0.415120in}}%
\pgfpathlineto{\pgfqpoint{0.431109in}{0.413632in}}%
\pgfpathlineto{\pgfqpoint{0.429548in}{0.409103in}}%
\pgfpathlineto{\pgfqpoint{0.433794in}{0.408772in}}%
\pgfpathlineto{\pgfqpoint{0.430231in}{0.403385in}}%
\pgfpathlineto{\pgfqpoint{0.426781in}{0.405131in}}%
\pgfpathlineto{\pgfqpoint{0.423985in}{0.404436in}}%
\pgfpathclose%
\pgfusepath{fill}%
\end{pgfscope}%
\begin{pgfscope}%
\pgfpathrectangle{\pgfqpoint{0.100000in}{0.100000in}}{\pgfqpoint{2.989028in}{1.913466in}}%
\pgfusepath{clip}%
\pgfsetbuttcap%
\pgfsetmiterjoin%
\definecolor{currentfill}{rgb}{0.400000,0.760784,0.647059}%
\pgfsetfillcolor{currentfill}%
\pgfsetlinewidth{0.000000pt}%
\definecolor{currentstroke}{rgb}{0.000000,0.000000,0.000000}%
\pgfsetstrokecolor{currentstroke}%
\pgfsetstrokeopacity{0.000000}%
\pgfsetdash{}{0pt}%
\pgfpathmoveto{\pgfqpoint{0.412124in}{0.415568in}}%
\pgfpathlineto{\pgfqpoint{0.409643in}{0.414219in}}%
\pgfpathlineto{\pgfqpoint{0.402882in}{0.416293in}}%
\pgfpathlineto{\pgfqpoint{0.397830in}{0.415092in}}%
\pgfpathlineto{\pgfqpoint{0.395264in}{0.418504in}}%
\pgfpathlineto{\pgfqpoint{0.396383in}{0.420066in}}%
\pgfpathlineto{\pgfqpoint{0.400212in}{0.420327in}}%
\pgfpathlineto{\pgfqpoint{0.403140in}{0.419346in}}%
\pgfpathlineto{\pgfqpoint{0.406331in}{0.420898in}}%
\pgfpathlineto{\pgfqpoint{0.411223in}{0.418823in}}%
\pgfpathlineto{\pgfqpoint{0.412124in}{0.415568in}}%
\pgfpathclose%
\pgfusepath{fill}%
\end{pgfscope}%
\begin{pgfscope}%
\pgfpathrectangle{\pgfqpoint{0.100000in}{0.100000in}}{\pgfqpoint{2.989028in}{1.913466in}}%
\pgfusepath{clip}%
\pgfsetbuttcap%
\pgfsetmiterjoin%
\definecolor{currentfill}{rgb}{0.400000,0.760784,0.647059}%
\pgfsetfillcolor{currentfill}%
\pgfsetlinewidth{0.000000pt}%
\definecolor{currentstroke}{rgb}{0.000000,0.000000,0.000000}%
\pgfsetstrokecolor{currentstroke}%
\pgfsetstrokeopacity{0.000000}%
\pgfsetdash{}{0pt}%
\pgfpathmoveto{\pgfqpoint{0.332429in}{0.465782in}}%
\pgfpathlineto{\pgfqpoint{0.332250in}{0.462436in}}%
\pgfpathlineto{\pgfqpoint{0.327314in}{0.460334in}}%
\pgfpathlineto{\pgfqpoint{0.322807in}{0.465597in}}%
\pgfpathlineto{\pgfqpoint{0.327862in}{0.467350in}}%
\pgfpathlineto{\pgfqpoint{0.332429in}{0.465782in}}%
\pgfpathclose%
\pgfusepath{fill}%
\end{pgfscope}%
\begin{pgfscope}%
\pgfpathrectangle{\pgfqpoint{0.100000in}{0.100000in}}{\pgfqpoint{2.989028in}{1.913466in}}%
\pgfusepath{clip}%
\pgfsetbuttcap%
\pgfsetmiterjoin%
\definecolor{currentfill}{rgb}{0.400000,0.760784,0.647059}%
\pgfsetfillcolor{currentfill}%
\pgfsetlinewidth{0.000000pt}%
\definecolor{currentstroke}{rgb}{0.000000,0.000000,0.000000}%
\pgfsetstrokecolor{currentstroke}%
\pgfsetstrokeopacity{0.000000}%
\pgfsetdash{}{0pt}%
\pgfpathmoveto{\pgfqpoint{0.291722in}{0.484589in}}%
\pgfpathlineto{\pgfqpoint{0.292771in}{0.486043in}}%
\pgfpathlineto{\pgfqpoint{0.298825in}{0.485792in}}%
\pgfpathlineto{\pgfqpoint{0.299887in}{0.487956in}}%
\pgfpathlineto{\pgfqpoint{0.302543in}{0.486068in}}%
\pgfpathlineto{\pgfqpoint{0.301009in}{0.482435in}}%
\pgfpathlineto{\pgfqpoint{0.298753in}{0.482127in}}%
\pgfpathlineto{\pgfqpoint{0.293361in}{0.483267in}}%
\pgfpathlineto{\pgfqpoint{0.291722in}{0.484589in}}%
\pgfpathclose%
\pgfusepath{fill}%
\end{pgfscope}%
\begin{pgfscope}%
\pgfpathrectangle{\pgfqpoint{0.100000in}{0.100000in}}{\pgfqpoint{2.989028in}{1.913466in}}%
\pgfusepath{clip}%
\pgfsetbuttcap%
\pgfsetmiterjoin%
\definecolor{currentfill}{rgb}{0.400000,0.760784,0.647059}%
\pgfsetfillcolor{currentfill}%
\pgfsetlinewidth{0.000000pt}%
\definecolor{currentstroke}{rgb}{0.000000,0.000000,0.000000}%
\pgfsetstrokecolor{currentstroke}%
\pgfsetstrokeopacity{0.000000}%
\pgfsetdash{}{0pt}%
\pgfpathmoveto{\pgfqpoint{0.283137in}{0.495848in}}%
\pgfpathlineto{\pgfqpoint{0.282770in}{0.499125in}}%
\pgfpathlineto{\pgfqpoint{0.285929in}{0.498882in}}%
\pgfpathlineto{\pgfqpoint{0.285709in}{0.503484in}}%
\pgfpathlineto{\pgfqpoint{0.288886in}{0.501000in}}%
\pgfpathlineto{\pgfqpoint{0.287787in}{0.497411in}}%
\pgfpathlineto{\pgfqpoint{0.283137in}{0.495848in}}%
\pgfpathclose%
\pgfusepath{fill}%
\end{pgfscope}%
\begin{pgfscope}%
\pgfpathrectangle{\pgfqpoint{0.100000in}{0.100000in}}{\pgfqpoint{2.989028in}{1.913466in}}%
\pgfusepath{clip}%
\pgfsetbuttcap%
\pgfsetmiterjoin%
\definecolor{currentfill}{rgb}{0.400000,0.760784,0.647059}%
\pgfsetfillcolor{currentfill}%
\pgfsetlinewidth{0.000000pt}%
\definecolor{currentstroke}{rgb}{0.000000,0.000000,0.000000}%
\pgfsetstrokecolor{currentstroke}%
\pgfsetstrokeopacity{0.000000}%
\pgfsetdash{}{0pt}%
\pgfpathmoveto{\pgfqpoint{0.561438in}{0.612633in}}%
\pgfpathlineto{\pgfqpoint{0.556400in}{0.613721in}}%
\pgfpathlineto{\pgfqpoint{0.555318in}{0.617088in}}%
\pgfpathlineto{\pgfqpoint{0.557049in}{0.620453in}}%
\pgfpathlineto{\pgfqpoint{0.560927in}{0.622211in}}%
\pgfpathlineto{\pgfqpoint{0.565367in}{0.613572in}}%
\pgfpathlineto{\pgfqpoint{0.569395in}{0.613189in}}%
\pgfpathlineto{\pgfqpoint{0.572399in}{0.610523in}}%
\pgfpathlineto{\pgfqpoint{0.572340in}{0.606855in}}%
\pgfpathlineto{\pgfqpoint{0.570143in}{0.605097in}}%
\pgfpathlineto{\pgfqpoint{0.573716in}{0.594249in}}%
\pgfpathlineto{\pgfqpoint{0.577625in}{0.590269in}}%
\pgfpathlineto{\pgfqpoint{0.573628in}{0.589006in}}%
\pgfpathlineto{\pgfqpoint{0.570393in}{0.593472in}}%
\pgfpathlineto{\pgfqpoint{0.563363in}{0.592089in}}%
\pgfpathlineto{\pgfqpoint{0.565549in}{0.596509in}}%
\pgfpathlineto{\pgfqpoint{0.564531in}{0.598269in}}%
\pgfpathlineto{\pgfqpoint{0.565128in}{0.602876in}}%
\pgfpathlineto{\pgfqpoint{0.563735in}{0.611204in}}%
\pgfpathlineto{\pgfqpoint{0.561438in}{0.612633in}}%
\pgfpathclose%
\pgfusepath{fill}%
\end{pgfscope}%
\begin{pgfscope}%
\pgfpathrectangle{\pgfqpoint{0.100000in}{0.100000in}}{\pgfqpoint{2.989028in}{1.913466in}}%
\pgfusepath{clip}%
\pgfsetbuttcap%
\pgfsetmiterjoin%
\definecolor{currentfill}{rgb}{0.400000,0.760784,0.647059}%
\pgfsetfillcolor{currentfill}%
\pgfsetlinewidth{0.000000pt}%
\definecolor{currentstroke}{rgb}{0.000000,0.000000,0.000000}%
\pgfsetstrokecolor{currentstroke}%
\pgfsetstrokeopacity{0.000000}%
\pgfsetdash{}{0pt}%
\pgfpathmoveto{\pgfqpoint{0.738441in}{0.364487in}}%
\pgfpathlineto{\pgfqpoint{0.731138in}{0.364569in}}%
\pgfpathlineto{\pgfqpoint{0.731700in}{0.368235in}}%
\pgfpathlineto{\pgfqpoint{0.734623in}{0.369546in}}%
\pgfpathlineto{\pgfqpoint{0.738441in}{0.364487in}}%
\pgfpathclose%
\pgfusepath{fill}%
\end{pgfscope}%
\begin{pgfscope}%
\pgfpathrectangle{\pgfqpoint{0.100000in}{0.100000in}}{\pgfqpoint{2.989028in}{1.913466in}}%
\pgfusepath{clip}%
\pgfsetbuttcap%
\pgfsetmiterjoin%
\definecolor{currentfill}{rgb}{0.400000,0.760784,0.647059}%
\pgfsetfillcolor{currentfill}%
\pgfsetlinewidth{0.000000pt}%
\definecolor{currentstroke}{rgb}{0.000000,0.000000,0.000000}%
\pgfsetstrokecolor{currentstroke}%
\pgfsetstrokeopacity{0.000000}%
\pgfsetdash{}{0pt}%
\pgfpathmoveto{\pgfqpoint{0.728306in}{0.367807in}}%
\pgfpathlineto{\pgfqpoint{0.723074in}{0.366541in}}%
\pgfpathlineto{\pgfqpoint{0.718102in}{0.364373in}}%
\pgfpathlineto{\pgfqpoint{0.716567in}{0.366927in}}%
\pgfpathlineto{\pgfqpoint{0.725117in}{0.368657in}}%
\pgfpathlineto{\pgfqpoint{0.728306in}{0.367807in}}%
\pgfpathclose%
\pgfusepath{fill}%
\end{pgfscope}%
\begin{pgfscope}%
\pgfpathrectangle{\pgfqpoint{0.100000in}{0.100000in}}{\pgfqpoint{2.989028in}{1.913466in}}%
\pgfusepath{clip}%
\pgfsetbuttcap%
\pgfsetmiterjoin%
\definecolor{currentfill}{rgb}{0.400000,0.760784,0.647059}%
\pgfsetfillcolor{currentfill}%
\pgfsetlinewidth{0.000000pt}%
\definecolor{currentstroke}{rgb}{0.000000,0.000000,0.000000}%
\pgfsetstrokecolor{currentstroke}%
\pgfsetstrokeopacity{0.000000}%
\pgfsetdash{}{0pt}%
\pgfpathmoveto{\pgfqpoint{0.648103in}{0.365212in}}%
\pgfpathlineto{\pgfqpoint{0.648303in}{0.362676in}}%
\pgfpathlineto{\pgfqpoint{0.645288in}{0.361100in}}%
\pgfpathlineto{\pgfqpoint{0.636875in}{0.364714in}}%
\pgfpathlineto{\pgfqpoint{0.635570in}{0.363261in}}%
\pgfpathlineto{\pgfqpoint{0.633535in}{0.370138in}}%
\pgfpathlineto{\pgfqpoint{0.637750in}{0.371095in}}%
\pgfpathlineto{\pgfqpoint{0.639698in}{0.368269in}}%
\pgfpathlineto{\pgfqpoint{0.643929in}{0.371807in}}%
\pgfpathlineto{\pgfqpoint{0.648103in}{0.365212in}}%
\pgfpathclose%
\pgfusepath{fill}%
\end{pgfscope}%
\begin{pgfscope}%
\pgfpathrectangle{\pgfqpoint{0.100000in}{0.100000in}}{\pgfqpoint{2.989028in}{1.913466in}}%
\pgfusepath{clip}%
\pgfsetbuttcap%
\pgfsetmiterjoin%
\definecolor{currentfill}{rgb}{0.400000,0.760784,0.647059}%
\pgfsetfillcolor{currentfill}%
\pgfsetlinewidth{0.000000pt}%
\definecolor{currentstroke}{rgb}{0.000000,0.000000,0.000000}%
\pgfsetstrokecolor{currentstroke}%
\pgfsetstrokeopacity{0.000000}%
\pgfsetdash{}{0pt}%
\pgfpathmoveto{\pgfqpoint{0.837890in}{0.224669in}}%
\pgfpathlineto{\pgfqpoint{0.830835in}{0.222042in}}%
\pgfpathlineto{\pgfqpoint{0.827098in}{0.221885in}}%
\pgfpathlineto{\pgfqpoint{0.830139in}{0.227457in}}%
\pgfpathlineto{\pgfqpoint{0.832738in}{0.229535in}}%
\pgfpathlineto{\pgfqpoint{0.832654in}{0.235306in}}%
\pgfpathlineto{\pgfqpoint{0.835479in}{0.245723in}}%
\pgfpathlineto{\pgfqpoint{0.838081in}{0.247702in}}%
\pgfpathlineto{\pgfqpoint{0.844087in}{0.244837in}}%
\pgfpathlineto{\pgfqpoint{0.843210in}{0.243196in}}%
\pgfpathlineto{\pgfqpoint{0.843665in}{0.235310in}}%
\pgfpathlineto{\pgfqpoint{0.841718in}{0.233182in}}%
\pgfpathlineto{\pgfqpoint{0.840881in}{0.227581in}}%
\pgfpathlineto{\pgfqpoint{0.837890in}{0.224669in}}%
\pgfpathclose%
\pgfusepath{fill}%
\end{pgfscope}%
\begin{pgfscope}%
\pgfpathrectangle{\pgfqpoint{0.100000in}{0.100000in}}{\pgfqpoint{2.989028in}{1.913466in}}%
\pgfusepath{clip}%
\pgfsetbuttcap%
\pgfsetmiterjoin%
\definecolor{currentfill}{rgb}{0.400000,0.760784,0.647059}%
\pgfsetfillcolor{currentfill}%
\pgfsetlinewidth{0.000000pt}%
\definecolor{currentstroke}{rgb}{0.000000,0.000000,0.000000}%
\pgfsetstrokecolor{currentstroke}%
\pgfsetstrokeopacity{0.000000}%
\pgfsetdash{}{0pt}%
\pgfpathmoveto{\pgfqpoint{0.822266in}{0.250492in}}%
\pgfpathlineto{\pgfqpoint{0.826399in}{0.243355in}}%
\pgfpathlineto{\pgfqpoint{0.825633in}{0.241653in}}%
\pgfpathlineto{\pgfqpoint{0.831161in}{0.239922in}}%
\pgfpathlineto{\pgfqpoint{0.829574in}{0.233362in}}%
\pgfpathlineto{\pgfqpoint{0.827167in}{0.233665in}}%
\pgfpathlineto{\pgfqpoint{0.820842in}{0.245101in}}%
\pgfpathlineto{\pgfqpoint{0.821742in}{0.239954in}}%
\pgfpathlineto{\pgfqpoint{0.820664in}{0.236990in}}%
\pgfpathlineto{\pgfqpoint{0.817986in}{0.235671in}}%
\pgfpathlineto{\pgfqpoint{0.816463in}{0.237092in}}%
\pgfpathlineto{\pgfqpoint{0.815613in}{0.243682in}}%
\pgfpathlineto{\pgfqpoint{0.816115in}{0.247956in}}%
\pgfpathlineto{\pgfqpoint{0.814331in}{0.249822in}}%
\pgfpathlineto{\pgfqpoint{0.819043in}{0.255220in}}%
\pgfpathlineto{\pgfqpoint{0.822135in}{0.256822in}}%
\pgfpathlineto{\pgfqpoint{0.823165in}{0.254717in}}%
\pgfpathlineto{\pgfqpoint{0.826391in}{0.256346in}}%
\pgfpathlineto{\pgfqpoint{0.828728in}{0.251922in}}%
\pgfpathlineto{\pgfqpoint{0.833783in}{0.246295in}}%
\pgfpathlineto{\pgfqpoint{0.833093in}{0.243746in}}%
\pgfpathlineto{\pgfqpoint{0.830286in}{0.240969in}}%
\pgfpathlineto{\pgfqpoint{0.827691in}{0.242277in}}%
\pgfpathlineto{\pgfqpoint{0.822266in}{0.250492in}}%
\pgfpathclose%
\pgfusepath{fill}%
\end{pgfscope}%
\begin{pgfscope}%
\pgfpathrectangle{\pgfqpoint{0.100000in}{0.100000in}}{\pgfqpoint{2.989028in}{1.913466in}}%
\pgfusepath{clip}%
\pgfsetbuttcap%
\pgfsetmiterjoin%
\definecolor{currentfill}{rgb}{0.400000,0.760784,0.647059}%
\pgfsetfillcolor{currentfill}%
\pgfsetlinewidth{0.000000pt}%
\definecolor{currentstroke}{rgb}{0.000000,0.000000,0.000000}%
\pgfsetstrokecolor{currentstroke}%
\pgfsetstrokeopacity{0.000000}%
\pgfsetdash{}{0pt}%
\pgfpathmoveto{\pgfqpoint{0.617161in}{0.351684in}}%
\pgfpathlineto{\pgfqpoint{0.612041in}{0.351018in}}%
\pgfpathlineto{\pgfqpoint{0.604830in}{0.348246in}}%
\pgfpathlineto{\pgfqpoint{0.603148in}{0.349752in}}%
\pgfpathlineto{\pgfqpoint{0.610680in}{0.351454in}}%
\pgfpathlineto{\pgfqpoint{0.608661in}{0.352939in}}%
\pgfpathlineto{\pgfqpoint{0.603991in}{0.354027in}}%
\pgfpathlineto{\pgfqpoint{0.602744in}{0.357507in}}%
\pgfpathlineto{\pgfqpoint{0.605094in}{0.360506in}}%
\pgfpathlineto{\pgfqpoint{0.607073in}{0.367387in}}%
\pgfpathlineto{\pgfqpoint{0.614301in}{0.368669in}}%
\pgfpathlineto{\pgfqpoint{0.617801in}{0.365696in}}%
\pgfpathlineto{\pgfqpoint{0.621088in}{0.368573in}}%
\pgfpathlineto{\pgfqpoint{0.625211in}{0.367867in}}%
\pgfpathlineto{\pgfqpoint{0.628174in}{0.362553in}}%
\pgfpathlineto{\pgfqpoint{0.630683in}{0.367277in}}%
\pgfpathlineto{\pgfqpoint{0.637804in}{0.356228in}}%
\pgfpathlineto{\pgfqpoint{0.635232in}{0.355345in}}%
\pgfpathlineto{\pgfqpoint{0.633653in}{0.352950in}}%
\pgfpathlineto{\pgfqpoint{0.636636in}{0.350764in}}%
\pgfpathlineto{\pgfqpoint{0.632011in}{0.348715in}}%
\pgfpathlineto{\pgfqpoint{0.628656in}{0.349843in}}%
\pgfpathlineto{\pgfqpoint{0.622321in}{0.353583in}}%
\pgfpathlineto{\pgfqpoint{0.617161in}{0.351684in}}%
\pgfpathclose%
\pgfusepath{fill}%
\end{pgfscope}%
\begin{pgfscope}%
\pgfpathrectangle{\pgfqpoint{0.100000in}{0.100000in}}{\pgfqpoint{2.989028in}{1.913466in}}%
\pgfusepath{clip}%
\pgfsetbuttcap%
\pgfsetmiterjoin%
\definecolor{currentfill}{rgb}{0.400000,0.760784,0.647059}%
\pgfsetfillcolor{currentfill}%
\pgfsetlinewidth{0.000000pt}%
\definecolor{currentstroke}{rgb}{0.000000,0.000000,0.000000}%
\pgfsetstrokecolor{currentstroke}%
\pgfsetstrokeopacity{0.000000}%
\pgfsetdash{}{0pt}%
\pgfpathmoveto{\pgfqpoint{0.817967in}{0.203589in}}%
\pgfpathlineto{\pgfqpoint{0.816952in}{0.213793in}}%
\pgfpathlineto{\pgfqpoint{0.819688in}{0.217539in}}%
\pgfpathlineto{\pgfqpoint{0.816762in}{0.217800in}}%
\pgfpathlineto{\pgfqpoint{0.817832in}{0.225086in}}%
\pgfpathlineto{\pgfqpoint{0.819093in}{0.229412in}}%
\pgfpathlineto{\pgfqpoint{0.817997in}{0.235224in}}%
\pgfpathlineto{\pgfqpoint{0.820677in}{0.236349in}}%
\pgfpathlineto{\pgfqpoint{0.821551in}{0.237852in}}%
\pgfpathlineto{\pgfqpoint{0.824191in}{0.237282in}}%
\pgfpathlineto{\pgfqpoint{0.826476in}{0.232626in}}%
\pgfpathlineto{\pgfqpoint{0.826901in}{0.228360in}}%
\pgfpathlineto{\pgfqpoint{0.824105in}{0.215583in}}%
\pgfpathlineto{\pgfqpoint{0.817967in}{0.203589in}}%
\pgfpathclose%
\pgfusepath{fill}%
\end{pgfscope}%
\begin{pgfscope}%
\pgfpathrectangle{\pgfqpoint{0.100000in}{0.100000in}}{\pgfqpoint{2.989028in}{1.913466in}}%
\pgfusepath{clip}%
\pgfsetbuttcap%
\pgfsetmiterjoin%
\definecolor{currentfill}{rgb}{0.400000,0.760784,0.647059}%
\pgfsetfillcolor{currentfill}%
\pgfsetlinewidth{0.000000pt}%
\definecolor{currentstroke}{rgb}{0.000000,0.000000,0.000000}%
\pgfsetstrokecolor{currentstroke}%
\pgfsetstrokeopacity{0.000000}%
\pgfsetdash{}{0pt}%
\pgfpathmoveto{\pgfqpoint{0.817360in}{0.234737in}}%
\pgfpathlineto{\pgfqpoint{0.817219in}{0.229762in}}%
\pgfpathlineto{\pgfqpoint{0.814925in}{0.227503in}}%
\pgfpathlineto{\pgfqpoint{0.812273in}{0.228338in}}%
\pgfpathlineto{\pgfqpoint{0.815625in}{0.235565in}}%
\pgfpathlineto{\pgfqpoint{0.817360in}{0.234737in}}%
\pgfpathclose%
\pgfusepath{fill}%
\end{pgfscope}%
\begin{pgfscope}%
\pgfpathrectangle{\pgfqpoint{0.100000in}{0.100000in}}{\pgfqpoint{2.989028in}{1.913466in}}%
\pgfusepath{clip}%
\pgfsetbuttcap%
\pgfsetmiterjoin%
\definecolor{currentfill}{rgb}{0.400000,0.760784,0.647059}%
\pgfsetfillcolor{currentfill}%
\pgfsetlinewidth{0.000000pt}%
\definecolor{currentstroke}{rgb}{0.000000,0.000000,0.000000}%
\pgfsetstrokecolor{currentstroke}%
\pgfsetstrokeopacity{0.000000}%
\pgfsetdash{}{0pt}%
\pgfpathmoveto{\pgfqpoint{0.845513in}{0.212648in}}%
\pgfpathlineto{\pgfqpoint{0.843557in}{0.204322in}}%
\pgfpathlineto{\pgfqpoint{0.838998in}{0.201690in}}%
\pgfpathlineto{\pgfqpoint{0.833017in}{0.204115in}}%
\pgfpathlineto{\pgfqpoint{0.834665in}{0.207350in}}%
\pgfpathlineto{\pgfqpoint{0.834632in}{0.209591in}}%
\pgfpathlineto{\pgfqpoint{0.836855in}{0.213398in}}%
\pgfpathlineto{\pgfqpoint{0.835292in}{0.212632in}}%
\pgfpathlineto{\pgfqpoint{0.836432in}{0.214491in}}%
\pgfpathlineto{\pgfqpoint{0.834267in}{0.218715in}}%
\pgfpathlineto{\pgfqpoint{0.836723in}{0.219493in}}%
\pgfpathlineto{\pgfqpoint{0.842410in}{0.214551in}}%
\pgfpathlineto{\pgfqpoint{0.845513in}{0.212648in}}%
\pgfpathclose%
\pgfusepath{fill}%
\end{pgfscope}%
\begin{pgfscope}%
\pgfpathrectangle{\pgfqpoint{0.100000in}{0.100000in}}{\pgfqpoint{2.989028in}{1.913466in}}%
\pgfusepath{clip}%
\pgfsetbuttcap%
\pgfsetmiterjoin%
\definecolor{currentfill}{rgb}{0.400000,0.760784,0.647059}%
\pgfsetfillcolor{currentfill}%
\pgfsetlinewidth{0.000000pt}%
\definecolor{currentstroke}{rgb}{0.000000,0.000000,0.000000}%
\pgfsetstrokecolor{currentstroke}%
\pgfsetstrokeopacity{0.000000}%
\pgfsetdash{}{0pt}%
\pgfpathmoveto{\pgfqpoint{0.825557in}{0.197926in}}%
\pgfpathlineto{\pgfqpoint{0.822979in}{0.196923in}}%
\pgfpathlineto{\pgfqpoint{0.824451in}{0.205709in}}%
\pgfpathlineto{\pgfqpoint{0.827546in}{0.207760in}}%
\pgfpathlineto{\pgfqpoint{0.826791in}{0.214343in}}%
\pgfpathlineto{\pgfqpoint{0.828049in}{0.217262in}}%
\pgfpathlineto{\pgfqpoint{0.830536in}{0.218356in}}%
\pgfpathlineto{\pgfqpoint{0.833148in}{0.215493in}}%
\pgfpathlineto{\pgfqpoint{0.835389in}{0.211607in}}%
\pgfpathlineto{\pgfqpoint{0.828184in}{0.203995in}}%
\pgfpathlineto{\pgfqpoint{0.825557in}{0.197926in}}%
\pgfpathclose%
\pgfusepath{fill}%
\end{pgfscope}%
\begin{pgfscope}%
\pgfpathrectangle{\pgfqpoint{0.100000in}{0.100000in}}{\pgfqpoint{2.989028in}{1.913466in}}%
\pgfusepath{clip}%
\pgfsetbuttcap%
\pgfsetmiterjoin%
\definecolor{currentfill}{rgb}{0.400000,0.760784,0.647059}%
\pgfsetfillcolor{currentfill}%
\pgfsetlinewidth{0.000000pt}%
\definecolor{currentstroke}{rgb}{0.000000,0.000000,0.000000}%
\pgfsetstrokecolor{currentstroke}%
\pgfsetstrokeopacity{0.000000}%
\pgfsetdash{}{0pt}%
\pgfpathmoveto{\pgfqpoint{0.846657in}{0.207083in}}%
\pgfpathlineto{\pgfqpoint{0.847921in}{0.201020in}}%
\pgfpathlineto{\pgfqpoint{0.842275in}{0.201714in}}%
\pgfpathlineto{\pgfqpoint{0.846657in}{0.207083in}}%
\pgfpathclose%
\pgfusepath{fill}%
\end{pgfscope}%
\begin{pgfscope}%
\pgfpathrectangle{\pgfqpoint{0.100000in}{0.100000in}}{\pgfqpoint{2.989028in}{1.913466in}}%
\pgfusepath{clip}%
\pgfsetbuttcap%
\pgfsetmiterjoin%
\definecolor{currentfill}{rgb}{0.400000,0.760784,0.647059}%
\pgfsetfillcolor{currentfill}%
\pgfsetlinewidth{0.000000pt}%
\definecolor{currentstroke}{rgb}{0.000000,0.000000,0.000000}%
\pgfsetstrokecolor{currentstroke}%
\pgfsetstrokeopacity{0.000000}%
\pgfsetdash{}{0pt}%
\pgfpathmoveto{\pgfqpoint{0.849351in}{0.197819in}}%
\pgfpathlineto{\pgfqpoint{0.850276in}{0.194079in}}%
\pgfpathlineto{\pgfqpoint{0.852047in}{0.192882in}}%
\pgfpathlineto{\pgfqpoint{0.851725in}{0.189362in}}%
\pgfpathlineto{\pgfqpoint{0.849251in}{0.188177in}}%
\pgfpathlineto{\pgfqpoint{0.847242in}{0.193433in}}%
\pgfpathlineto{\pgfqpoint{0.849351in}{0.197819in}}%
\pgfpathclose%
\pgfusepath{fill}%
\end{pgfscope}%
\begin{pgfscope}%
\pgfpathrectangle{\pgfqpoint{0.100000in}{0.100000in}}{\pgfqpoint{2.989028in}{1.913466in}}%
\pgfusepath{clip}%
\pgfsetbuttcap%
\pgfsetmiterjoin%
\definecolor{currentfill}{rgb}{0.400000,0.760784,0.647059}%
\pgfsetfillcolor{currentfill}%
\pgfsetlinewidth{0.000000pt}%
\definecolor{currentstroke}{rgb}{0.000000,0.000000,0.000000}%
\pgfsetstrokecolor{currentstroke}%
\pgfsetstrokeopacity{0.000000}%
\pgfsetdash{}{0pt}%
\pgfpathmoveto{\pgfqpoint{0.844640in}{0.199254in}}%
\pgfpathlineto{\pgfqpoint{0.843104in}{0.194909in}}%
\pgfpathlineto{\pgfqpoint{0.840719in}{0.195069in}}%
\pgfpathlineto{\pgfqpoint{0.839215in}{0.198652in}}%
\pgfpathlineto{\pgfqpoint{0.841665in}{0.200349in}}%
\pgfpathlineto{\pgfqpoint{0.844640in}{0.199254in}}%
\pgfpathclose%
\pgfusepath{fill}%
\end{pgfscope}%
\begin{pgfscope}%
\pgfpathrectangle{\pgfqpoint{0.100000in}{0.100000in}}{\pgfqpoint{2.989028in}{1.913466in}}%
\pgfusepath{clip}%
\pgfsetbuttcap%
\pgfsetmiterjoin%
\definecolor{currentfill}{rgb}{0.400000,0.760784,0.647059}%
\pgfsetfillcolor{currentfill}%
\pgfsetlinewidth{0.000000pt}%
\definecolor{currentstroke}{rgb}{0.000000,0.000000,0.000000}%
\pgfsetstrokecolor{currentstroke}%
\pgfsetstrokeopacity{0.000000}%
\pgfsetdash{}{0pt}%
\pgfpathmoveto{\pgfqpoint{0.838205in}{0.176854in}}%
\pgfpathlineto{\pgfqpoint{0.840333in}{0.174527in}}%
\pgfpathlineto{\pgfqpoint{0.840802in}{0.169622in}}%
\pgfpathlineto{\pgfqpoint{0.841922in}{0.168197in}}%
\pgfpathlineto{\pgfqpoint{0.840092in}{0.163758in}}%
\pgfpathlineto{\pgfqpoint{0.837686in}{0.161260in}}%
\pgfpathlineto{\pgfqpoint{0.836405in}{0.156623in}}%
\pgfpathlineto{\pgfqpoint{0.833774in}{0.163848in}}%
\pgfpathlineto{\pgfqpoint{0.833338in}{0.170408in}}%
\pgfpathlineto{\pgfqpoint{0.831865in}{0.173673in}}%
\pgfpathlineto{\pgfqpoint{0.826795in}{0.176321in}}%
\pgfpathlineto{\pgfqpoint{0.831551in}{0.181841in}}%
\pgfpathlineto{\pgfqpoint{0.828380in}{0.184201in}}%
\pgfpathlineto{\pgfqpoint{0.831174in}{0.186552in}}%
\pgfpathlineto{\pgfqpoint{0.834882in}{0.195579in}}%
\pgfpathlineto{\pgfqpoint{0.831038in}{0.198762in}}%
\pgfpathlineto{\pgfqpoint{0.832519in}{0.201818in}}%
\pgfpathlineto{\pgfqpoint{0.837487in}{0.199107in}}%
\pgfpathlineto{\pgfqpoint{0.837990in}{0.194829in}}%
\pgfpathlineto{\pgfqpoint{0.836111in}{0.192394in}}%
\pgfpathlineto{\pgfqpoint{0.839836in}{0.189098in}}%
\pgfpathlineto{\pgfqpoint{0.841009in}{0.183965in}}%
\pgfpathlineto{\pgfqpoint{0.840850in}{0.176765in}}%
\pgfpathlineto{\pgfqpoint{0.838205in}{0.176854in}}%
\pgfpathclose%
\pgfusepath{fill}%
\end{pgfscope}%
\begin{pgfscope}%
\pgfpathrectangle{\pgfqpoint{0.100000in}{0.100000in}}{\pgfqpoint{2.989028in}{1.913466in}}%
\pgfusepath{clip}%
\pgfsetbuttcap%
\pgfsetmiterjoin%
\definecolor{currentfill}{rgb}{0.400000,0.760784,0.647059}%
\pgfsetfillcolor{currentfill}%
\pgfsetlinewidth{0.000000pt}%
\definecolor{currentstroke}{rgb}{0.000000,0.000000,0.000000}%
\pgfsetstrokecolor{currentstroke}%
\pgfsetstrokeopacity{0.000000}%
\pgfsetdash{}{0pt}%
\pgfpathmoveto{\pgfqpoint{0.847302in}{0.194926in}}%
\pgfpathlineto{\pgfqpoint{0.846143in}{0.192429in}}%
\pgfpathlineto{\pgfqpoint{0.847231in}{0.188374in}}%
\pgfpathlineto{\pgfqpoint{0.844445in}{0.188033in}}%
\pgfpathlineto{\pgfqpoint{0.841524in}{0.189911in}}%
\pgfpathlineto{\pgfqpoint{0.842472in}{0.194046in}}%
\pgfpathlineto{\pgfqpoint{0.847302in}{0.194926in}}%
\pgfpathclose%
\pgfusepath{fill}%
\end{pgfscope}%
\begin{pgfscope}%
\pgfpathrectangle{\pgfqpoint{0.100000in}{0.100000in}}{\pgfqpoint{2.989028in}{1.913466in}}%
\pgfusepath{clip}%
\pgfsetbuttcap%
\pgfsetmiterjoin%
\definecolor{currentfill}{rgb}{0.400000,0.760784,0.647059}%
\pgfsetfillcolor{currentfill}%
\pgfsetlinewidth{0.000000pt}%
\definecolor{currentstroke}{rgb}{0.000000,0.000000,0.000000}%
\pgfsetstrokecolor{currentstroke}%
\pgfsetstrokeopacity{0.000000}%
\pgfsetdash{}{0pt}%
\pgfpathmoveto{\pgfqpoint{0.834316in}{0.195209in}}%
\pgfpathlineto{\pgfqpoint{0.828167in}{0.192451in}}%
\pgfpathlineto{\pgfqpoint{0.826076in}{0.193444in}}%
\pgfpathlineto{\pgfqpoint{0.830270in}{0.197082in}}%
\pgfpathlineto{\pgfqpoint{0.834316in}{0.195209in}}%
\pgfpathclose%
\pgfusepath{fill}%
\end{pgfscope}%
\begin{pgfscope}%
\pgfpathrectangle{\pgfqpoint{0.100000in}{0.100000in}}{\pgfqpoint{2.989028in}{1.913466in}}%
\pgfusepath{clip}%
\pgfsetbuttcap%
\pgfsetmiterjoin%
\definecolor{currentfill}{rgb}{0.400000,0.760784,0.647059}%
\pgfsetfillcolor{currentfill}%
\pgfsetlinewidth{0.000000pt}%
\definecolor{currentstroke}{rgb}{0.000000,0.000000,0.000000}%
\pgfsetstrokecolor{currentstroke}%
\pgfsetstrokeopacity{0.000000}%
\pgfsetdash{}{0pt}%
\pgfpathmoveto{\pgfqpoint{0.847369in}{0.168302in}}%
\pgfpathlineto{\pgfqpoint{0.845903in}{0.172455in}}%
\pgfpathlineto{\pgfqpoint{0.849004in}{0.173477in}}%
\pgfpathlineto{\pgfqpoint{0.849943in}{0.177955in}}%
\pgfpathlineto{\pgfqpoint{0.853303in}{0.177960in}}%
\pgfpathlineto{\pgfqpoint{0.853289in}{0.181079in}}%
\pgfpathlineto{\pgfqpoint{0.857959in}{0.180341in}}%
\pgfpathlineto{\pgfqpoint{0.858755in}{0.171887in}}%
\pgfpathlineto{\pgfqpoint{0.856046in}{0.166498in}}%
\pgfpathlineto{\pgfqpoint{0.852027in}{0.163131in}}%
\pgfpathlineto{\pgfqpoint{0.847369in}{0.168302in}}%
\pgfpathclose%
\pgfusepath{fill}%
\end{pgfscope}%
\begin{pgfscope}%
\pgfpathrectangle{\pgfqpoint{0.100000in}{0.100000in}}{\pgfqpoint{2.989028in}{1.913466in}}%
\pgfusepath{clip}%
\pgfsetbuttcap%
\pgfsetmiterjoin%
\definecolor{currentfill}{rgb}{0.400000,0.760784,0.647059}%
\pgfsetfillcolor{currentfill}%
\pgfsetlinewidth{0.000000pt}%
\definecolor{currentstroke}{rgb}{0.000000,0.000000,0.000000}%
\pgfsetstrokecolor{currentstroke}%
\pgfsetstrokeopacity{0.000000}%
\pgfsetdash{}{0pt}%
\pgfpathmoveto{\pgfqpoint{0.845553in}{0.171496in}}%
\pgfpathlineto{\pgfqpoint{0.846878in}{0.167564in}}%
\pgfpathlineto{\pgfqpoint{0.843612in}{0.167086in}}%
\pgfpathlineto{\pgfqpoint{0.845553in}{0.171496in}}%
\pgfpathclose%
\pgfusepath{fill}%
\end{pgfscope}%
\begin{pgfscope}%
\pgfpathrectangle{\pgfqpoint{0.100000in}{0.100000in}}{\pgfqpoint{2.989028in}{1.913466in}}%
\pgfusepath{clip}%
\pgfsetbuttcap%
\pgfsetmiterjoin%
\definecolor{currentfill}{rgb}{0.400000,0.760784,0.647059}%
\pgfsetfillcolor{currentfill}%
\pgfsetlinewidth{0.000000pt}%
\definecolor{currentstroke}{rgb}{0.000000,0.000000,0.000000}%
\pgfsetstrokecolor{currentstroke}%
\pgfsetstrokeopacity{0.000000}%
\pgfsetdash{}{0pt}%
\pgfpathmoveto{\pgfqpoint{0.848306in}{0.165713in}}%
\pgfpathlineto{\pgfqpoint{0.847851in}{0.160700in}}%
\pgfpathlineto{\pgfqpoint{0.844406in}{0.161116in}}%
\pgfpathlineto{\pgfqpoint{0.845520in}{0.163537in}}%
\pgfpathlineto{\pgfqpoint{0.848306in}{0.165713in}}%
\pgfpathclose%
\pgfusepath{fill}%
\end{pgfscope}%
\begin{pgfscope}%
\pgfsetbuttcap%
\pgfsetmiterjoin%
\definecolor{currentfill}{rgb}{1.000000,1.000000,1.000000}%
\pgfsetfillcolor{currentfill}%
\pgfsetlinewidth{1.003750pt}%
\definecolor{currentstroke}{rgb}{0.827451,0.827451,0.827451}%
\pgfsetstrokecolor{currentstroke}%
\pgfsetdash{}{0pt}%
\pgfpathmoveto{\pgfqpoint{1.015239in}{1.931644in}}%
\pgfpathlineto{\pgfqpoint{2.626989in}{1.931644in}}%
\pgfpathlineto{\pgfqpoint{2.626989in}{2.127644in}}%
\pgfpathlineto{\pgfqpoint{1.015239in}{2.127644in}}%
\pgfpathlineto{\pgfqpoint{1.015239in}{1.931644in}}%
\pgfpathclose%
\pgfusepath{stroke,fill}%
\end{pgfscope}%
\begin{pgfscope}%
\definecolor{textcolor}{rgb}{0.000000,0.000000,0.000000}%
\pgfsetstrokecolor{textcolor}%
\pgfsetfillcolor{textcolor}%
\pgftext[x=1.056489in,y=1.994331in,left,base]{\color{textcolor}\setmainfont{Lato}\rmfamily\fontsize{9.000000}{10.800000}\selectfont Quarterly Growth, 2023 Q2}%
\end{pgfscope}%
\begin{pgfscope}%
\pgfpathrectangle{\pgfqpoint{3.625000in}{0.100000in}}{\pgfqpoint{2.989028in}{1.913466in}}%
\pgfusepath{clip}%
\pgfsetbuttcap%
\pgfsetmiterjoin%
\definecolor{currentfill}{rgb}{0.793080,0.916494,0.618224}%
\pgfsetfillcolor{currentfill}%
\pgfsetlinewidth{0.000000pt}%
\definecolor{currentstroke}{rgb}{0.000000,0.000000,0.000000}%
\pgfsetstrokecolor{currentstroke}%
\pgfsetstrokeopacity{0.000000}%
\pgfsetdash{}{0pt}%
\pgfpathmoveto{\pgfqpoint{4.622679in}{0.548675in}}%
\pgfpathlineto{\pgfqpoint{4.614089in}{0.548815in}}%
\pgfpathlineto{\pgfqpoint{4.608242in}{0.555287in}}%
\pgfpathlineto{\pgfqpoint{4.604266in}{0.572473in}}%
\pgfpathlineto{\pgfqpoint{4.613506in}{0.574627in}}%
\pgfpathlineto{\pgfqpoint{4.622893in}{0.572311in}}%
\pgfpathlineto{\pgfqpoint{4.632446in}{0.565540in}}%
\pgfpathlineto{\pgfqpoint{4.632441in}{0.556177in}}%
\pgfpathlineto{\pgfqpoint{4.622679in}{0.548675in}}%
\pgfpathclose%
\pgfusepath{fill}%
\end{pgfscope}%
\begin{pgfscope}%
\pgfpathrectangle{\pgfqpoint{3.625000in}{0.100000in}}{\pgfqpoint{2.989028in}{1.913466in}}%
\pgfusepath{clip}%
\pgfsetbuttcap%
\pgfsetmiterjoin%
\definecolor{currentfill}{rgb}{0.793080,0.916494,0.618224}%
\pgfsetfillcolor{currentfill}%
\pgfsetlinewidth{0.000000pt}%
\definecolor{currentstroke}{rgb}{0.000000,0.000000,0.000000}%
\pgfsetstrokecolor{currentstroke}%
\pgfsetstrokeopacity{0.000000}%
\pgfsetdash{}{0pt}%
\pgfpathmoveto{\pgfqpoint{4.674970in}{0.446640in}}%
\pgfpathlineto{\pgfqpoint{4.665359in}{0.450300in}}%
\pgfpathlineto{\pgfqpoint{4.665315in}{0.457479in}}%
\pgfpathlineto{\pgfqpoint{4.653705in}{0.463549in}}%
\pgfpathlineto{\pgfqpoint{4.655119in}{0.478835in}}%
\pgfpathlineto{\pgfqpoint{4.658481in}{0.486048in}}%
\pgfpathlineto{\pgfqpoint{4.665581in}{0.479409in}}%
\pgfpathlineto{\pgfqpoint{4.677127in}{0.480518in}}%
\pgfpathlineto{\pgfqpoint{4.674211in}{0.459391in}}%
\pgfpathlineto{\pgfqpoint{4.674970in}{0.446640in}}%
\pgfpathclose%
\pgfusepath{fill}%
\end{pgfscope}%
\begin{pgfscope}%
\pgfpathrectangle{\pgfqpoint{3.625000in}{0.100000in}}{\pgfqpoint{2.989028in}{1.913466in}}%
\pgfusepath{clip}%
\pgfsetbuttcap%
\pgfsetmiterjoin%
\definecolor{currentfill}{rgb}{0.793080,0.916494,0.618224}%
\pgfsetfillcolor{currentfill}%
\pgfsetlinewidth{0.000000pt}%
\definecolor{currentstroke}{rgb}{0.000000,0.000000,0.000000}%
\pgfsetstrokecolor{currentstroke}%
\pgfsetstrokeopacity{0.000000}%
\pgfsetdash{}{0pt}%
\pgfpathmoveto{\pgfqpoint{4.715310in}{0.399517in}}%
\pgfpathlineto{\pgfqpoint{4.703114in}{0.401594in}}%
\pgfpathlineto{\pgfqpoint{4.696369in}{0.412035in}}%
\pgfpathlineto{\pgfqpoint{4.695417in}{0.420620in}}%
\pgfpathlineto{\pgfqpoint{4.715310in}{0.399517in}}%
\pgfpathclose%
\pgfusepath{fill}%
\end{pgfscope}%
\begin{pgfscope}%
\pgfpathrectangle{\pgfqpoint{3.625000in}{0.100000in}}{\pgfqpoint{2.989028in}{1.913466in}}%
\pgfusepath{clip}%
\pgfsetbuttcap%
\pgfsetmiterjoin%
\definecolor{currentfill}{rgb}{0.793080,0.916494,0.618224}%
\pgfsetfillcolor{currentfill}%
\pgfsetlinewidth{0.000000pt}%
\definecolor{currentstroke}{rgb}{0.000000,0.000000,0.000000}%
\pgfsetstrokecolor{currentstroke}%
\pgfsetstrokeopacity{0.000000}%
\pgfsetdash{}{0pt}%
\pgfpathmoveto{\pgfqpoint{4.690317in}{0.401338in}}%
\pgfpathlineto{\pgfqpoint{4.696711in}{0.396147in}}%
\pgfpathlineto{\pgfqpoint{4.698099in}{0.387222in}}%
\pgfpathlineto{\pgfqpoint{4.685227in}{0.387715in}}%
\pgfpathlineto{\pgfqpoint{4.690317in}{0.401338in}}%
\pgfpathclose%
\pgfusepath{fill}%
\end{pgfscope}%
\begin{pgfscope}%
\pgfpathrectangle{\pgfqpoint{3.625000in}{0.100000in}}{\pgfqpoint{2.989028in}{1.913466in}}%
\pgfusepath{clip}%
\pgfsetbuttcap%
\pgfsetmiterjoin%
\definecolor{currentfill}{rgb}{0.793080,0.916494,0.618224}%
\pgfsetfillcolor{currentfill}%
\pgfsetlinewidth{0.000000pt}%
\definecolor{currentstroke}{rgb}{0.000000,0.000000,0.000000}%
\pgfsetstrokecolor{currentstroke}%
\pgfsetstrokeopacity{0.000000}%
\pgfsetdash{}{0pt}%
\pgfpathmoveto{\pgfqpoint{4.719683in}{0.351485in}}%
\pgfpathlineto{\pgfqpoint{4.706837in}{0.356103in}}%
\pgfpathlineto{\pgfqpoint{4.710518in}{0.367298in}}%
\pgfpathlineto{\pgfqpoint{4.705134in}{0.378705in}}%
\pgfpathlineto{\pgfqpoint{4.706825in}{0.386546in}}%
\pgfpathlineto{\pgfqpoint{4.717162in}{0.389535in}}%
\pgfpathlineto{\pgfqpoint{4.716963in}{0.376972in}}%
\pgfpathlineto{\pgfqpoint{4.729190in}{0.370003in}}%
\pgfpathlineto{\pgfqpoint{4.736183in}{0.353155in}}%
\pgfpathlineto{\pgfqpoint{4.728389in}{0.346667in}}%
\pgfpathlineto{\pgfqpoint{4.719683in}{0.351485in}}%
\pgfpathclose%
\pgfusepath{fill}%
\end{pgfscope}%
\begin{pgfscope}%
\pgfpathrectangle{\pgfqpoint{3.625000in}{0.100000in}}{\pgfqpoint{2.989028in}{1.913466in}}%
\pgfusepath{clip}%
\pgfsetbuttcap%
\pgfsetmiterjoin%
\definecolor{currentfill}{rgb}{0.793080,0.916494,0.618224}%
\pgfsetfillcolor{currentfill}%
\pgfsetlinewidth{0.000000pt}%
\definecolor{currentstroke}{rgb}{0.000000,0.000000,0.000000}%
\pgfsetstrokecolor{currentstroke}%
\pgfsetstrokeopacity{0.000000}%
\pgfsetdash{}{0pt}%
\pgfpathmoveto{\pgfqpoint{4.673289in}{0.237569in}}%
\pgfpathlineto{\pgfqpoint{4.667169in}{0.250211in}}%
\pgfpathlineto{\pgfqpoint{4.669879in}{0.258715in}}%
\pgfpathlineto{\pgfqpoint{4.681883in}{0.270485in}}%
\pgfpathlineto{\pgfqpoint{4.683019in}{0.279508in}}%
\pgfpathlineto{\pgfqpoint{4.690875in}{0.299484in}}%
\pgfpathlineto{\pgfqpoint{4.711536in}{0.302897in}}%
\pgfpathlineto{\pgfqpoint{4.715491in}{0.314590in}}%
\pgfpathlineto{\pgfqpoint{4.724600in}{0.310205in}}%
\pgfpathlineto{\pgfqpoint{4.743110in}{0.277388in}}%
\pgfpathlineto{\pgfqpoint{4.743344in}{0.266674in}}%
\pgfpathlineto{\pgfqpoint{4.738112in}{0.257400in}}%
\pgfpathlineto{\pgfqpoint{4.744750in}{0.236108in}}%
\pgfpathlineto{\pgfqpoint{4.740111in}{0.232676in}}%
\pgfpathlineto{\pgfqpoint{4.724311in}{0.233674in}}%
\pgfpathlineto{\pgfqpoint{4.705311in}{0.240024in}}%
\pgfpathlineto{\pgfqpoint{4.689342in}{0.242646in}}%
\pgfpathlineto{\pgfqpoint{4.685076in}{0.238921in}}%
\pgfpathlineto{\pgfqpoint{4.673289in}{0.237569in}}%
\pgfpathclose%
\pgfusepath{fill}%
\end{pgfscope}%
\begin{pgfscope}%
\pgfpathrectangle{\pgfqpoint{3.625000in}{0.100000in}}{\pgfqpoint{2.989028in}{1.913466in}}%
\pgfusepath{clip}%
\pgfsetbuttcap%
\pgfsetmiterjoin%
\definecolor{currentfill}{rgb}{0.344022,0.698347,0.672895}%
\pgfsetfillcolor{currentfill}%
\pgfsetlinewidth{0.000000pt}%
\definecolor{currentstroke}{rgb}{0.000000,0.000000,0.000000}%
\pgfsetstrokecolor{currentstroke}%
\pgfsetstrokeopacity{0.000000}%
\pgfsetdash{}{0pt}%
\pgfpathmoveto{\pgfqpoint{4.335222in}{1.900931in}}%
\pgfpathlineto{\pgfqpoint{4.319722in}{1.839512in}}%
\pgfpathlineto{\pgfqpoint{4.308700in}{1.795504in}}%
\pgfpathlineto{\pgfqpoint{4.295595in}{1.742564in}}%
\pgfpathlineto{\pgfqpoint{4.293022in}{1.729516in}}%
\pgfpathlineto{\pgfqpoint{4.294703in}{1.717547in}}%
\pgfpathlineto{\pgfqpoint{4.292475in}{1.706523in}}%
\pgfpathlineto{\pgfqpoint{4.200998in}{1.730394in}}%
\pgfpathlineto{\pgfqpoint{4.192734in}{1.727622in}}%
\pgfpathlineto{\pgfqpoint{4.185663in}{1.730028in}}%
\pgfpathlineto{\pgfqpoint{4.152921in}{1.730814in}}%
\pgfpathlineto{\pgfqpoint{4.141848in}{1.727692in}}%
\pgfpathlineto{\pgfqpoint{4.130851in}{1.728747in}}%
\pgfpathlineto{\pgfqpoint{4.126190in}{1.733634in}}%
\pgfpathlineto{\pgfqpoint{4.096877in}{1.732568in}}%
\pgfpathlineto{\pgfqpoint{4.092248in}{1.740554in}}%
\pgfpathlineto{\pgfqpoint{4.083460in}{1.744762in}}%
\pgfpathlineto{\pgfqpoint{4.070673in}{1.747325in}}%
\pgfpathlineto{\pgfqpoint{4.048603in}{1.743571in}}%
\pgfpathlineto{\pgfqpoint{4.040456in}{1.747249in}}%
\pgfpathlineto{\pgfqpoint{4.027895in}{1.757031in}}%
\pgfpathlineto{\pgfqpoint{4.031670in}{1.776202in}}%
\pgfpathlineto{\pgfqpoint{4.030347in}{1.785992in}}%
\pgfpathlineto{\pgfqpoint{4.020391in}{1.796386in}}%
\pgfpathlineto{\pgfqpoint{4.014015in}{1.795741in}}%
\pgfpathlineto{\pgfqpoint{4.009440in}{1.806255in}}%
\pgfpathlineto{\pgfqpoint{3.998604in}{1.810504in}}%
\pgfpathlineto{\pgfqpoint{3.990727in}{1.809963in}}%
\pgfpathlineto{\pgfqpoint{3.988179in}{1.820760in}}%
\pgfpathlineto{\pgfqpoint{3.996026in}{1.819611in}}%
\pgfpathlineto{\pgfqpoint{3.995402in}{1.834619in}}%
\pgfpathlineto{\pgfqpoint{3.991982in}{1.843530in}}%
\pgfpathlineto{\pgfqpoint{3.993490in}{1.854935in}}%
\pgfpathlineto{\pgfqpoint{3.997595in}{1.863514in}}%
\pgfpathlineto{\pgfqpoint{3.997364in}{1.879792in}}%
\pgfpathlineto{\pgfqpoint{3.995172in}{1.885699in}}%
\pgfpathlineto{\pgfqpoint{3.998957in}{1.904648in}}%
\pgfpathlineto{\pgfqpoint{3.997865in}{1.916860in}}%
\pgfpathlineto{\pgfqpoint{3.994081in}{1.922716in}}%
\pgfpathlineto{\pgfqpoint{3.994658in}{1.941861in}}%
\pgfpathlineto{\pgfqpoint{4.000162in}{1.955941in}}%
\pgfpathlineto{\pgfqpoint{4.006149in}{1.952495in}}%
\pgfpathlineto{\pgfqpoint{4.025972in}{1.932018in}}%
\pgfpathlineto{\pgfqpoint{4.049993in}{1.920812in}}%
\pgfpathlineto{\pgfqpoint{4.062292in}{1.919486in}}%
\pgfpathlineto{\pgfqpoint{4.073880in}{1.914715in}}%
\pgfpathlineto{\pgfqpoint{4.077168in}{1.898661in}}%
\pgfpathlineto{\pgfqpoint{4.067014in}{1.895620in}}%
\pgfpathlineto{\pgfqpoint{4.060285in}{1.882967in}}%
\pgfpathlineto{\pgfqpoint{4.068185in}{1.883670in}}%
\pgfpathlineto{\pgfqpoint{4.071372in}{1.889349in}}%
\pgfpathlineto{\pgfqpoint{4.082514in}{1.896451in}}%
\pgfpathlineto{\pgfqpoint{4.081939in}{1.885905in}}%
\pgfpathlineto{\pgfqpoint{4.074499in}{1.884204in}}%
\pgfpathlineto{\pgfqpoint{4.075734in}{1.870648in}}%
\pgfpathlineto{\pgfqpoint{4.066674in}{1.857069in}}%
\pgfpathlineto{\pgfqpoint{4.060876in}{1.863526in}}%
\pgfpathlineto{\pgfqpoint{4.043645in}{1.860015in}}%
\pgfpathlineto{\pgfqpoint{4.042738in}{1.852116in}}%
\pgfpathlineto{\pgfqpoint{4.048678in}{1.847303in}}%
\pgfpathlineto{\pgfqpoint{4.057741in}{1.846830in}}%
\pgfpathlineto{\pgfqpoint{4.070271in}{1.857114in}}%
\pgfpathlineto{\pgfqpoint{4.080189in}{1.857968in}}%
\pgfpathlineto{\pgfqpoint{4.080569in}{1.869352in}}%
\pgfpathlineto{\pgfqpoint{4.085680in}{1.886072in}}%
\pgfpathlineto{\pgfqpoint{4.097793in}{1.898506in}}%
\pgfpathlineto{\pgfqpoint{4.093754in}{1.908138in}}%
\pgfpathlineto{\pgfqpoint{4.096513in}{1.918507in}}%
\pgfpathlineto{\pgfqpoint{4.091139in}{1.928842in}}%
\pgfpathlineto{\pgfqpoint{4.098798in}{1.942057in}}%
\pgfpathlineto{\pgfqpoint{4.099945in}{1.950008in}}%
\pgfpathlineto{\pgfqpoint{4.093296in}{1.955234in}}%
\pgfpathlineto{\pgfqpoint{4.094403in}{1.968600in}}%
\pgfpathlineto{\pgfqpoint{4.174072in}{1.944395in}}%
\pgfpathlineto{\pgfqpoint{4.258673in}{1.920671in}}%
\pgfpathlineto{\pgfqpoint{4.335222in}{1.900931in}}%
\pgfpathclose%
\pgfusepath{fill}%
\end{pgfscope}%
\begin{pgfscope}%
\pgfpathrectangle{\pgfqpoint{3.625000in}{0.100000in}}{\pgfqpoint{2.989028in}{1.913466in}}%
\pgfusepath{clip}%
\pgfsetbuttcap%
\pgfsetmiterjoin%
\definecolor{currentfill}{rgb}{0.344022,0.698347,0.672895}%
\pgfsetfillcolor{currentfill}%
\pgfsetlinewidth{0.000000pt}%
\definecolor{currentstroke}{rgb}{0.000000,0.000000,0.000000}%
\pgfsetstrokecolor{currentstroke}%
\pgfsetstrokeopacity{0.000000}%
\pgfsetdash{}{0pt}%
\pgfpathmoveto{\pgfqpoint{4.081297in}{1.922359in}}%
\pgfpathlineto{\pgfqpoint{4.085220in}{1.907235in}}%
\pgfpathlineto{\pgfqpoint{4.091372in}{1.902211in}}%
\pgfpathlineto{\pgfqpoint{4.086463in}{1.895816in}}%
\pgfpathlineto{\pgfqpoint{4.081679in}{1.905186in}}%
\pgfpathlineto{\pgfqpoint{4.081297in}{1.922359in}}%
\pgfpathclose%
\pgfusepath{fill}%
\end{pgfscope}%
\begin{pgfscope}%
\pgfpathrectangle{\pgfqpoint{3.625000in}{0.100000in}}{\pgfqpoint{2.989028in}{1.913466in}}%
\pgfusepath{clip}%
\pgfsetbuttcap%
\pgfsetmiterjoin%
\definecolor{currentfill}{rgb}{0.847520,0.938639,0.607151}%
\pgfsetfillcolor{currentfill}%
\pgfsetlinewidth{0.000000pt}%
\definecolor{currentstroke}{rgb}{0.000000,0.000000,0.000000}%
\pgfsetstrokecolor{currentstroke}%
\pgfsetstrokeopacity{0.000000}%
\pgfsetdash{}{0pt}%
\pgfpathmoveto{\pgfqpoint{4.376305in}{1.891004in}}%
\pgfpathlineto{\pgfqpoint{4.461467in}{1.871893in}}%
\pgfpathlineto{\pgfqpoint{4.541670in}{1.855681in}}%
\pgfpathlineto{\pgfqpoint{4.603378in}{1.844375in}}%
\pgfpathlineto{\pgfqpoint{4.657175in}{1.835337in}}%
\pgfpathlineto{\pgfqpoint{4.711092in}{1.827040in}}%
\pgfpathlineto{\pgfqpoint{4.757007in}{1.820572in}}%
\pgfpathlineto{\pgfqpoint{4.802995in}{1.814641in}}%
\pgfpathlineto{\pgfqpoint{4.849048in}{1.809248in}}%
\pgfpathlineto{\pgfqpoint{4.892448in}{1.804664in}}%
\pgfpathlineto{\pgfqpoint{4.886479in}{1.738522in}}%
\pgfpathlineto{\pgfqpoint{4.877595in}{1.649140in}}%
\pgfpathlineto{\pgfqpoint{4.872936in}{1.603169in}}%
\pgfpathlineto{\pgfqpoint{4.866234in}{1.541215in}}%
\pgfpathlineto{\pgfqpoint{4.818794in}{1.546395in}}%
\pgfpathlineto{\pgfqpoint{4.764479in}{1.552561in}}%
\pgfpathlineto{\pgfqpoint{4.689064in}{1.562882in}}%
\pgfpathlineto{\pgfqpoint{4.655381in}{1.567687in}}%
\pgfpathlineto{\pgfqpoint{4.572396in}{1.580713in}}%
\pgfpathlineto{\pgfqpoint{4.543829in}{1.585939in}}%
\pgfpathlineto{\pgfqpoint{4.537875in}{1.552087in}}%
\pgfpathlineto{\pgfqpoint{4.534620in}{1.554502in}}%
\pgfpathlineto{\pgfqpoint{4.528502in}{1.570788in}}%
\pgfpathlineto{\pgfqpoint{4.520639in}{1.569785in}}%
\pgfpathlineto{\pgfqpoint{4.515011in}{1.561552in}}%
\pgfpathlineto{\pgfqpoint{4.503331in}{1.560451in}}%
\pgfpathlineto{\pgfqpoint{4.500984in}{1.564390in}}%
\pgfpathlineto{\pgfqpoint{4.490043in}{1.563913in}}%
\pgfpathlineto{\pgfqpoint{4.484543in}{1.567727in}}%
\pgfpathlineto{\pgfqpoint{4.476852in}{1.561826in}}%
\pgfpathlineto{\pgfqpoint{4.458172in}{1.567136in}}%
\pgfpathlineto{\pgfqpoint{4.450020in}{1.564285in}}%
\pgfpathlineto{\pgfqpoint{4.445634in}{1.576938in}}%
\pgfpathlineto{\pgfqpoint{4.445429in}{1.588978in}}%
\pgfpathlineto{\pgfqpoint{4.432552in}{1.597637in}}%
\pgfpathlineto{\pgfqpoint{4.434652in}{1.610116in}}%
\pgfpathlineto{\pgfqpoint{4.425483in}{1.630729in}}%
\pgfpathlineto{\pgfqpoint{4.426101in}{1.648251in}}%
\pgfpathlineto{\pgfqpoint{4.418220in}{1.656827in}}%
\pgfpathlineto{\pgfqpoint{4.410827in}{1.649240in}}%
\pgfpathlineto{\pgfqpoint{4.400763in}{1.645136in}}%
\pgfpathlineto{\pgfqpoint{4.393102in}{1.653594in}}%
\pgfpathlineto{\pgfqpoint{4.394348in}{1.667101in}}%
\pgfpathlineto{\pgfqpoint{4.403535in}{1.672422in}}%
\pgfpathlineto{\pgfqpoint{4.401093in}{1.680980in}}%
\pgfpathlineto{\pgfqpoint{4.417063in}{1.722399in}}%
\pgfpathlineto{\pgfqpoint{4.404470in}{1.723437in}}%
\pgfpathlineto{\pgfqpoint{4.403176in}{1.730870in}}%
\pgfpathlineto{\pgfqpoint{4.394003in}{1.737279in}}%
\pgfpathlineto{\pgfqpoint{4.394589in}{1.744445in}}%
\pgfpathlineto{\pgfqpoint{4.389750in}{1.750011in}}%
\pgfpathlineto{\pgfqpoint{4.381873in}{1.770222in}}%
\pgfpathlineto{\pgfqpoint{4.374675in}{1.774343in}}%
\pgfpathlineto{\pgfqpoint{4.369521in}{1.791786in}}%
\pgfpathlineto{\pgfqpoint{4.370737in}{1.803380in}}%
\pgfpathlineto{\pgfqpoint{4.361121in}{1.824733in}}%
\pgfpathlineto{\pgfqpoint{4.376305in}{1.891004in}}%
\pgfpathclose%
\pgfusepath{fill}%
\end{pgfscope}%
\begin{pgfscope}%
\pgfpathrectangle{\pgfqpoint{3.625000in}{0.100000in}}{\pgfqpoint{2.989028in}{1.913466in}}%
\pgfusepath{clip}%
\pgfsetbuttcap%
\pgfsetmiterjoin%
\definecolor{currentfill}{rgb}{0.932718,0.973087,0.644060}%
\pgfsetfillcolor{currentfill}%
\pgfsetlinewidth{0.000000pt}%
\definecolor{currentstroke}{rgb}{0.000000,0.000000,0.000000}%
\pgfsetstrokecolor{currentstroke}%
\pgfsetstrokeopacity{0.000000}%
\pgfsetdash{}{0pt}%
\pgfpathmoveto{\pgfqpoint{6.432175in}{1.550163in}}%
\pgfpathlineto{\pgfqpoint{6.430471in}{1.557411in}}%
\pgfpathlineto{\pgfqpoint{6.420998in}{1.563760in}}%
\pgfpathlineto{\pgfqpoint{6.395964in}{1.645645in}}%
\pgfpathlineto{\pgfqpoint{6.382471in}{1.685161in}}%
\pgfpathlineto{\pgfqpoint{6.393319in}{1.696546in}}%
\pgfpathlineto{\pgfqpoint{6.403864in}{1.724407in}}%
\pgfpathlineto{\pgfqpoint{6.409123in}{1.733311in}}%
\pgfpathlineto{\pgfqpoint{6.405397in}{1.737050in}}%
\pgfpathlineto{\pgfqpoint{6.404149in}{1.761782in}}%
\pgfpathlineto{\pgfqpoint{6.408854in}{1.769364in}}%
\pgfpathlineto{\pgfqpoint{6.406792in}{1.786986in}}%
\pgfpathlineto{\pgfqpoint{6.425570in}{1.844804in}}%
\pgfpathlineto{\pgfqpoint{6.434031in}{1.845120in}}%
\pgfpathlineto{\pgfqpoint{6.437543in}{1.834792in}}%
\pgfpathlineto{\pgfqpoint{6.445062in}{1.831826in}}%
\pgfpathlineto{\pgfqpoint{6.459266in}{1.843974in}}%
\pgfpathlineto{\pgfqpoint{6.470403in}{1.851149in}}%
\pgfpathlineto{\pgfqpoint{6.494943in}{1.838514in}}%
\pgfpathlineto{\pgfqpoint{6.517113in}{1.768360in}}%
\pgfpathlineto{\pgfqpoint{6.521328in}{1.751101in}}%
\pgfpathlineto{\pgfqpoint{6.531039in}{1.749076in}}%
\pgfpathlineto{\pgfqpoint{6.544209in}{1.737354in}}%
\pgfpathlineto{\pgfqpoint{6.543463in}{1.730524in}}%
\pgfpathlineto{\pgfqpoint{6.552435in}{1.722428in}}%
\pgfpathlineto{\pgfqpoint{6.559723in}{1.727073in}}%
\pgfpathlineto{\pgfqpoint{6.574974in}{1.709287in}}%
\pgfpathlineto{\pgfqpoint{6.568158in}{1.695053in}}%
\pgfpathlineto{\pgfqpoint{6.559018in}{1.694771in}}%
\pgfpathlineto{\pgfqpoint{6.551685in}{1.682065in}}%
\pgfpathlineto{\pgfqpoint{6.542802in}{1.680263in}}%
\pgfpathlineto{\pgfqpoint{6.536280in}{1.673497in}}%
\pgfpathlineto{\pgfqpoint{6.516803in}{1.666241in}}%
\pgfpathlineto{\pgfqpoint{6.505014in}{1.654548in}}%
\pgfpathlineto{\pgfqpoint{6.499025in}{1.662941in}}%
\pgfpathlineto{\pgfqpoint{6.493584in}{1.656926in}}%
\pgfpathlineto{\pgfqpoint{6.495074in}{1.632607in}}%
\pgfpathlineto{\pgfqpoint{6.490765in}{1.622974in}}%
\pgfpathlineto{\pgfqpoint{6.481367in}{1.625605in}}%
\pgfpathlineto{\pgfqpoint{6.479850in}{1.615730in}}%
\pgfpathlineto{\pgfqpoint{6.475789in}{1.611669in}}%
\pgfpathlineto{\pgfqpoint{6.467662in}{1.612659in}}%
\pgfpathlineto{\pgfqpoint{6.468174in}{1.603425in}}%
\pgfpathlineto{\pgfqpoint{6.455855in}{1.606048in}}%
\pgfpathlineto{\pgfqpoint{6.449128in}{1.593255in}}%
\pgfpathlineto{\pgfqpoint{6.451668in}{1.586561in}}%
\pgfpathlineto{\pgfqpoint{6.447666in}{1.575429in}}%
\pgfpathlineto{\pgfqpoint{6.441367in}{1.567240in}}%
\pgfpathlineto{\pgfqpoint{6.439781in}{1.550146in}}%
\pgfpathlineto{\pgfqpoint{6.432175in}{1.550163in}}%
\pgfpathclose%
\pgfusepath{fill}%
\end{pgfscope}%
\begin{pgfscope}%
\pgfpathrectangle{\pgfqpoint{3.625000in}{0.100000in}}{\pgfqpoint{2.989028in}{1.913466in}}%
\pgfusepath{clip}%
\pgfsetbuttcap%
\pgfsetmiterjoin%
\definecolor{currentfill}{rgb}{0.932718,0.973087,0.644060}%
\pgfsetfillcolor{currentfill}%
\pgfsetlinewidth{0.000000pt}%
\definecolor{currentstroke}{rgb}{0.000000,0.000000,0.000000}%
\pgfsetstrokecolor{currentstroke}%
\pgfsetstrokeopacity{0.000000}%
\pgfsetdash{}{0pt}%
\pgfpathmoveto{\pgfqpoint{6.520301in}{1.661247in}}%
\pgfpathlineto{\pgfqpoint{6.525863in}{1.667073in}}%
\pgfpathlineto{\pgfqpoint{6.531158in}{1.661597in}}%
\pgfpathlineto{\pgfqpoint{6.521647in}{1.654342in}}%
\pgfpathlineto{\pgfqpoint{6.520301in}{1.661247in}}%
\pgfpathclose%
\pgfusepath{fill}%
\end{pgfscope}%
\begin{pgfscope}%
\pgfpathrectangle{\pgfqpoint{3.625000in}{0.100000in}}{\pgfqpoint{2.989028in}{1.913466in}}%
\pgfusepath{clip}%
\pgfsetbuttcap%
\pgfsetmiterjoin%
\definecolor{currentfill}{rgb}{0.246828,0.467589,0.710035}%
\pgfsetfillcolor{currentfill}%
\pgfsetlinewidth{0.000000pt}%
\definecolor{currentstroke}{rgb}{0.000000,0.000000,0.000000}%
\pgfsetstrokecolor{currentstroke}%
\pgfsetstrokeopacity{0.000000}%
\pgfsetdash{}{0pt}%
\pgfpathmoveto{\pgfqpoint{4.872936in}{1.603169in}}%
\pgfpathlineto{\pgfqpoint{4.877595in}{1.649140in}}%
\pgfpathlineto{\pgfqpoint{4.886479in}{1.738522in}}%
\pgfpathlineto{\pgfqpoint{4.892448in}{1.804664in}}%
\pgfpathlineto{\pgfqpoint{4.941329in}{1.800078in}}%
\pgfpathlineto{\pgfqpoint{5.003862in}{1.795100in}}%
\pgfpathlineto{\pgfqpoint{5.061021in}{1.791418in}}%
\pgfpathlineto{\pgfqpoint{5.112777in}{1.788799in}}%
\pgfpathlineto{\pgfqpoint{5.190014in}{1.786150in}}%
\pgfpathlineto{\pgfqpoint{5.195285in}{1.764381in}}%
\pgfpathlineto{\pgfqpoint{5.193398in}{1.753978in}}%
\pgfpathlineto{\pgfqpoint{5.192777in}{1.732596in}}%
\pgfpathlineto{\pgfqpoint{5.196363in}{1.716582in}}%
\pgfpathlineto{\pgfqpoint{5.204579in}{1.692956in}}%
\pgfpathlineto{\pgfqpoint{5.204560in}{1.663212in}}%
\pgfpathlineto{\pgfqpoint{5.206080in}{1.628658in}}%
\pgfpathlineto{\pgfqpoint{5.208167in}{1.619346in}}%
\pgfpathlineto{\pgfqpoint{5.214274in}{1.609122in}}%
\pgfpathlineto{\pgfqpoint{5.216288in}{1.593192in}}%
\pgfpathlineto{\pgfqpoint{5.215419in}{1.582557in}}%
\pgfpathlineto{\pgfqpoint{5.150668in}{1.583959in}}%
\pgfpathlineto{\pgfqpoint{5.103564in}{1.586347in}}%
\pgfpathlineto{\pgfqpoint{5.034511in}{1.590023in}}%
\pgfpathlineto{\pgfqpoint{4.966415in}{1.594860in}}%
\pgfpathlineto{\pgfqpoint{4.921059in}{1.598519in}}%
\pgfpathlineto{\pgfqpoint{4.872936in}{1.603169in}}%
\pgfpathclose%
\pgfusepath{fill}%
\end{pgfscope}%
\begin{pgfscope}%
\pgfpathrectangle{\pgfqpoint{3.625000in}{0.100000in}}{\pgfqpoint{2.989028in}{1.913466in}}%
\pgfusepath{clip}%
\pgfsetbuttcap%
\pgfsetmiterjoin%
\definecolor{currentfill}{rgb}{0.905805,0.962322,0.602076}%
\pgfsetfillcolor{currentfill}%
\pgfsetlinewidth{0.000000pt}%
\definecolor{currentstroke}{rgb}{0.000000,0.000000,0.000000}%
\pgfsetstrokecolor{currentstroke}%
\pgfsetstrokeopacity{0.000000}%
\pgfsetdash{}{0pt}%
\pgfpathmoveto{\pgfqpoint{4.853363in}{1.410780in}}%
\pgfpathlineto{\pgfqpoint{4.860839in}{1.487840in}}%
\pgfpathlineto{\pgfqpoint{4.866234in}{1.541215in}}%
\pgfpathlineto{\pgfqpoint{4.872936in}{1.603169in}}%
\pgfpathlineto{\pgfqpoint{4.921059in}{1.598519in}}%
\pgfpathlineto{\pgfqpoint{4.966415in}{1.594860in}}%
\pgfpathlineto{\pgfqpoint{5.034511in}{1.590023in}}%
\pgfpathlineto{\pgfqpoint{5.103564in}{1.586347in}}%
\pgfpathlineto{\pgfqpoint{5.150668in}{1.583959in}}%
\pgfpathlineto{\pgfqpoint{5.215419in}{1.582557in}}%
\pgfpathlineto{\pgfqpoint{5.211034in}{1.569764in}}%
\pgfpathlineto{\pgfqpoint{5.202284in}{1.559721in}}%
\pgfpathlineto{\pgfqpoint{5.208985in}{1.548158in}}%
\pgfpathlineto{\pgfqpoint{5.216374in}{1.545688in}}%
\pgfpathlineto{\pgfqpoint{5.219880in}{1.539046in}}%
\pgfpathlineto{\pgfqpoint{5.219075in}{1.490552in}}%
\pgfpathlineto{\pgfqpoint{5.217728in}{1.422304in}}%
\pgfpathlineto{\pgfqpoint{5.212744in}{1.404379in}}%
\pgfpathlineto{\pgfqpoint{5.217201in}{1.394464in}}%
\pgfpathlineto{\pgfqpoint{5.208252in}{1.373157in}}%
\pgfpathlineto{\pgfqpoint{5.217680in}{1.355981in}}%
\pgfpathlineto{\pgfqpoint{5.209670in}{1.357295in}}%
\pgfpathlineto{\pgfqpoint{5.204205in}{1.367982in}}%
\pgfpathlineto{\pgfqpoint{5.184700in}{1.375307in}}%
\pgfpathlineto{\pgfqpoint{5.172400in}{1.381750in}}%
\pgfpathlineto{\pgfqpoint{5.151759in}{1.382286in}}%
\pgfpathlineto{\pgfqpoint{5.144594in}{1.376416in}}%
\pgfpathlineto{\pgfqpoint{5.121231in}{1.387975in}}%
\pgfpathlineto{\pgfqpoint{5.119439in}{1.391630in}}%
\pgfpathlineto{\pgfqpoint{5.037902in}{1.395489in}}%
\pgfpathlineto{\pgfqpoint{4.988365in}{1.398329in}}%
\pgfpathlineto{\pgfqpoint{4.947447in}{1.401533in}}%
\pgfpathlineto{\pgfqpoint{4.879836in}{1.407921in}}%
\pgfpathlineto{\pgfqpoint{4.853363in}{1.410780in}}%
\pgfpathclose%
\pgfusepath{fill}%
\end{pgfscope}%
\begin{pgfscope}%
\pgfpathrectangle{\pgfqpoint{3.625000in}{0.100000in}}{\pgfqpoint{2.989028in}{1.913466in}}%
\pgfusepath{clip}%
\pgfsetbuttcap%
\pgfsetmiterjoin%
\definecolor{currentfill}{rgb}{0.280661,0.423760,0.689273}%
\pgfsetfillcolor{currentfill}%
\pgfsetlinewidth{0.000000pt}%
\definecolor{currentstroke}{rgb}{0.000000,0.000000,0.000000}%
\pgfsetstrokecolor{currentstroke}%
\pgfsetstrokeopacity{0.000000}%
\pgfsetdash{}{0pt}%
\pgfpathmoveto{\pgfqpoint{4.840540in}{1.280295in}}%
\pgfpathlineto{\pgfqpoint{4.797072in}{1.284265in}}%
\pgfpathlineto{\pgfqpoint{4.702321in}{1.295959in}}%
\pgfpathlineto{\pgfqpoint{4.650778in}{1.303357in}}%
\pgfpathlineto{\pgfqpoint{4.595495in}{1.311293in}}%
\pgfpathlineto{\pgfqpoint{4.548928in}{1.318744in}}%
\pgfpathlineto{\pgfqpoint{4.497815in}{1.327496in}}%
\pgfpathlineto{\pgfqpoint{4.509436in}{1.391963in}}%
\pgfpathlineto{\pgfqpoint{4.521208in}{1.458068in}}%
\pgfpathlineto{\pgfqpoint{4.537875in}{1.552087in}}%
\pgfpathlineto{\pgfqpoint{4.543829in}{1.585939in}}%
\pgfpathlineto{\pgfqpoint{4.572396in}{1.580713in}}%
\pgfpathlineto{\pgfqpoint{4.655381in}{1.567687in}}%
\pgfpathlineto{\pgfqpoint{4.689064in}{1.562882in}}%
\pgfpathlineto{\pgfqpoint{4.764479in}{1.552561in}}%
\pgfpathlineto{\pgfqpoint{4.818794in}{1.546395in}}%
\pgfpathlineto{\pgfqpoint{4.866234in}{1.541215in}}%
\pgfpathlineto{\pgfqpoint{4.860839in}{1.487840in}}%
\pgfpathlineto{\pgfqpoint{4.853363in}{1.410780in}}%
\pgfpathlineto{\pgfqpoint{4.846947in}{1.345289in}}%
\pgfpathlineto{\pgfqpoint{4.840540in}{1.280295in}}%
\pgfpathclose%
\pgfusepath{fill}%
\end{pgfscope}%
\begin{pgfscope}%
\pgfpathrectangle{\pgfqpoint{3.625000in}{0.100000in}}{\pgfqpoint{2.989028in}{1.913466in}}%
\pgfusepath{clip}%
\pgfsetbuttcap%
\pgfsetmiterjoin%
\definecolor{currentfill}{rgb}{0.999616,0.988082,0.729027}%
\pgfsetfillcolor{currentfill}%
\pgfsetlinewidth{0.000000pt}%
\definecolor{currentstroke}{rgb}{0.000000,0.000000,0.000000}%
\pgfsetstrokecolor{currentstroke}%
\pgfsetstrokeopacity{0.000000}%
\pgfsetdash{}{0pt}%
\pgfpathmoveto{\pgfqpoint{5.635291in}{1.369221in}}%
\pgfpathlineto{\pgfqpoint{5.580206in}{1.365308in}}%
\pgfpathlineto{\pgfqpoint{5.498100in}{1.361809in}}%
\pgfpathlineto{\pgfqpoint{5.494980in}{1.370103in}}%
\pgfpathlineto{\pgfqpoint{5.476772in}{1.376303in}}%
\pgfpathlineto{\pgfqpoint{5.472756in}{1.388016in}}%
\pgfpathlineto{\pgfqpoint{5.471075in}{1.402507in}}%
\pgfpathlineto{\pgfqpoint{5.475178in}{1.409931in}}%
\pgfpathlineto{\pgfqpoint{5.468697in}{1.417056in}}%
\pgfpathlineto{\pgfqpoint{5.467134in}{1.425554in}}%
\pgfpathlineto{\pgfqpoint{5.465045in}{1.444352in}}%
\pgfpathlineto{\pgfqpoint{5.458835in}{1.454564in}}%
\pgfpathlineto{\pgfqpoint{5.447805in}{1.460295in}}%
\pgfpathlineto{\pgfqpoint{5.435803in}{1.469752in}}%
\pgfpathlineto{\pgfqpoint{5.429599in}{1.480942in}}%
\pgfpathlineto{\pgfqpoint{5.418466in}{1.485448in}}%
\pgfpathlineto{\pgfqpoint{5.411922in}{1.492782in}}%
\pgfpathlineto{\pgfqpoint{5.403994in}{1.494027in}}%
\pgfpathlineto{\pgfqpoint{5.389840in}{1.504916in}}%
\pgfpathlineto{\pgfqpoint{5.392139in}{1.517447in}}%
\pgfpathlineto{\pgfqpoint{5.391696in}{1.541277in}}%
\pgfpathlineto{\pgfqpoint{5.396061in}{1.547836in}}%
\pgfpathlineto{\pgfqpoint{5.392139in}{1.557741in}}%
\pgfpathlineto{\pgfqpoint{5.385230in}{1.559657in}}%
\pgfpathlineto{\pgfqpoint{5.385801in}{1.568355in}}%
\pgfpathlineto{\pgfqpoint{5.394373in}{1.582098in}}%
\pgfpathlineto{\pgfqpoint{5.411348in}{1.592965in}}%
\pgfpathlineto{\pgfqpoint{5.410290in}{1.631611in}}%
\pgfpathlineto{\pgfqpoint{5.418784in}{1.637416in}}%
\pgfpathlineto{\pgfqpoint{5.426822in}{1.633549in}}%
\pgfpathlineto{\pgfqpoint{5.443197in}{1.639211in}}%
\pgfpathlineto{\pgfqpoint{5.474004in}{1.653439in}}%
\pgfpathlineto{\pgfqpoint{5.478017in}{1.649033in}}%
\pgfpathlineto{\pgfqpoint{5.472182in}{1.629079in}}%
\pgfpathlineto{\pgfqpoint{5.480863in}{1.633458in}}%
\pgfpathlineto{\pgfqpoint{5.495731in}{1.629085in}}%
\pgfpathlineto{\pgfqpoint{5.504867in}{1.625436in}}%
\pgfpathlineto{\pgfqpoint{5.509989in}{1.614740in}}%
\pgfpathlineto{\pgfqpoint{5.556860in}{1.604646in}}%
\pgfpathlineto{\pgfqpoint{5.570871in}{1.597722in}}%
\pgfpathlineto{\pgfqpoint{5.585116in}{1.597803in}}%
\pgfpathlineto{\pgfqpoint{5.599752in}{1.594947in}}%
\pgfpathlineto{\pgfqpoint{5.609251in}{1.585189in}}%
\pgfpathlineto{\pgfqpoint{5.618328in}{1.580368in}}%
\pgfpathlineto{\pgfqpoint{5.619996in}{1.566462in}}%
\pgfpathlineto{\pgfqpoint{5.617314in}{1.557730in}}%
\pgfpathlineto{\pgfqpoint{5.627430in}{1.557084in}}%
\pgfpathlineto{\pgfqpoint{5.624022in}{1.546957in}}%
\pgfpathlineto{\pgfqpoint{5.627264in}{1.543360in}}%
\pgfpathlineto{\pgfqpoint{5.630488in}{1.533800in}}%
\pgfpathlineto{\pgfqpoint{5.620629in}{1.528739in}}%
\pgfpathlineto{\pgfqpoint{5.614905in}{1.514637in}}%
\pgfpathlineto{\pgfqpoint{5.613109in}{1.504665in}}%
\pgfpathlineto{\pgfqpoint{5.618600in}{1.502958in}}%
\pgfpathlineto{\pgfqpoint{5.625632in}{1.510436in}}%
\pgfpathlineto{\pgfqpoint{5.631610in}{1.523394in}}%
\pgfpathlineto{\pgfqpoint{5.639698in}{1.527895in}}%
\pgfpathlineto{\pgfqpoint{5.645776in}{1.522041in}}%
\pgfpathlineto{\pgfqpoint{5.639793in}{1.504322in}}%
\pgfpathlineto{\pgfqpoint{5.637916in}{1.490567in}}%
\pgfpathlineto{\pgfqpoint{5.639684in}{1.480679in}}%
\pgfpathlineto{\pgfqpoint{5.634128in}{1.475085in}}%
\pgfpathlineto{\pgfqpoint{5.631353in}{1.461416in}}%
\pgfpathlineto{\pgfqpoint{5.633620in}{1.447050in}}%
\pgfpathlineto{\pgfqpoint{5.627064in}{1.425804in}}%
\pgfpathlineto{\pgfqpoint{5.627202in}{1.415188in}}%
\pgfpathlineto{\pgfqpoint{5.632383in}{1.392182in}}%
\pgfpathlineto{\pgfqpoint{5.635741in}{1.388236in}}%
\pgfpathlineto{\pgfqpoint{5.635291in}{1.369221in}}%
\pgfpathclose%
\pgfusepath{fill}%
\end{pgfscope}%
\begin{pgfscope}%
\pgfpathrectangle{\pgfqpoint{3.625000in}{0.100000in}}{\pgfqpoint{2.989028in}{1.913466in}}%
\pgfusepath{clip}%
\pgfsetbuttcap%
\pgfsetmiterjoin%
\definecolor{currentfill}{rgb}{0.999616,0.988082,0.729027}%
\pgfsetfillcolor{currentfill}%
\pgfsetlinewidth{0.000000pt}%
\definecolor{currentstroke}{rgb}{0.000000,0.000000,0.000000}%
\pgfsetstrokecolor{currentstroke}%
\pgfsetstrokeopacity{0.000000}%
\pgfsetdash{}{0pt}%
\pgfpathmoveto{\pgfqpoint{5.655944in}{1.555675in}}%
\pgfpathlineto{\pgfqpoint{5.657173in}{1.546498in}}%
\pgfpathlineto{\pgfqpoint{5.645909in}{1.522320in}}%
\pgfpathlineto{\pgfqpoint{5.640904in}{1.529323in}}%
\pgfpathlineto{\pgfqpoint{5.655944in}{1.555675in}}%
\pgfpathclose%
\pgfusepath{fill}%
\end{pgfscope}%
\begin{pgfscope}%
\pgfpathrectangle{\pgfqpoint{3.625000in}{0.100000in}}{\pgfqpoint{2.989028in}{1.913466in}}%
\pgfusepath{clip}%
\pgfsetbuttcap%
\pgfsetmiterjoin%
\definecolor{currentfill}{rgb}{0.702345,0.879585,0.636678}%
\pgfsetfillcolor{currentfill}%
\pgfsetlinewidth{0.000000pt}%
\definecolor{currentstroke}{rgb}{0.000000,0.000000,0.000000}%
\pgfsetstrokecolor{currentstroke}%
\pgfsetstrokeopacity{0.000000}%
\pgfsetdash{}{0pt}%
\pgfpathmoveto{\pgfqpoint{4.292475in}{1.706523in}}%
\pgfpathlineto{\pgfqpoint{4.294703in}{1.717547in}}%
\pgfpathlineto{\pgfqpoint{4.293022in}{1.729516in}}%
\pgfpathlineto{\pgfqpoint{4.295595in}{1.742564in}}%
\pgfpathlineto{\pgfqpoint{4.308700in}{1.795504in}}%
\pgfpathlineto{\pgfqpoint{4.319722in}{1.839512in}}%
\pgfpathlineto{\pgfqpoint{4.335222in}{1.900931in}}%
\pgfpathlineto{\pgfqpoint{4.376305in}{1.891004in}}%
\pgfpathlineto{\pgfqpoint{4.361121in}{1.824733in}}%
\pgfpathlineto{\pgfqpoint{4.370737in}{1.803380in}}%
\pgfpathlineto{\pgfqpoint{4.369521in}{1.791786in}}%
\pgfpathlineto{\pgfqpoint{4.374675in}{1.774343in}}%
\pgfpathlineto{\pgfqpoint{4.381873in}{1.770222in}}%
\pgfpathlineto{\pgfqpoint{4.389750in}{1.750011in}}%
\pgfpathlineto{\pgfqpoint{4.394589in}{1.744445in}}%
\pgfpathlineto{\pgfqpoint{4.394003in}{1.737279in}}%
\pgfpathlineto{\pgfqpoint{4.403176in}{1.730870in}}%
\pgfpathlineto{\pgfqpoint{4.404470in}{1.723437in}}%
\pgfpathlineto{\pgfqpoint{4.417063in}{1.722399in}}%
\pgfpathlineto{\pgfqpoint{4.401093in}{1.680980in}}%
\pgfpathlineto{\pgfqpoint{4.403535in}{1.672422in}}%
\pgfpathlineto{\pgfqpoint{4.394348in}{1.667101in}}%
\pgfpathlineto{\pgfqpoint{4.393102in}{1.653594in}}%
\pgfpathlineto{\pgfqpoint{4.400763in}{1.645136in}}%
\pgfpathlineto{\pgfqpoint{4.410827in}{1.649240in}}%
\pgfpathlineto{\pgfqpoint{4.418220in}{1.656827in}}%
\pgfpathlineto{\pgfqpoint{4.426101in}{1.648251in}}%
\pgfpathlineto{\pgfqpoint{4.425483in}{1.630729in}}%
\pgfpathlineto{\pgfqpoint{4.434652in}{1.610116in}}%
\pgfpathlineto{\pgfqpoint{4.432552in}{1.597637in}}%
\pgfpathlineto{\pgfqpoint{4.445429in}{1.588978in}}%
\pgfpathlineto{\pgfqpoint{4.445634in}{1.576938in}}%
\pgfpathlineto{\pgfqpoint{4.450020in}{1.564285in}}%
\pgfpathlineto{\pgfqpoint{4.458172in}{1.567136in}}%
\pgfpathlineto{\pgfqpoint{4.476852in}{1.561826in}}%
\pgfpathlineto{\pgfqpoint{4.484543in}{1.567727in}}%
\pgfpathlineto{\pgfqpoint{4.490043in}{1.563913in}}%
\pgfpathlineto{\pgfqpoint{4.500984in}{1.564390in}}%
\pgfpathlineto{\pgfqpoint{4.503331in}{1.560451in}}%
\pgfpathlineto{\pgfqpoint{4.515011in}{1.561552in}}%
\pgfpathlineto{\pgfqpoint{4.520639in}{1.569785in}}%
\pgfpathlineto{\pgfqpoint{4.528502in}{1.570788in}}%
\pgfpathlineto{\pgfqpoint{4.534620in}{1.554502in}}%
\pgfpathlineto{\pgfqpoint{4.537875in}{1.552087in}}%
\pgfpathlineto{\pgfqpoint{4.521208in}{1.458068in}}%
\pgfpathlineto{\pgfqpoint{4.509436in}{1.391963in}}%
\pgfpathlineto{\pgfqpoint{4.416592in}{1.409886in}}%
\pgfpathlineto{\pgfqpoint{4.366442in}{1.419862in}}%
\pgfpathlineto{\pgfqpoint{4.319534in}{1.430177in}}%
\pgfpathlineto{\pgfqpoint{4.275689in}{1.440103in}}%
\pgfpathlineto{\pgfqpoint{4.224946in}{1.452341in}}%
\pgfpathlineto{\pgfqpoint{4.251119in}{1.559712in}}%
\pgfpathlineto{\pgfqpoint{4.252531in}{1.567559in}}%
\pgfpathlineto{\pgfqpoint{4.264170in}{1.588120in}}%
\pgfpathlineto{\pgfqpoint{4.252118in}{1.600478in}}%
\pgfpathlineto{\pgfqpoint{4.254620in}{1.612613in}}%
\pgfpathlineto{\pgfqpoint{4.259208in}{1.615650in}}%
\pgfpathlineto{\pgfqpoint{4.267402in}{1.628155in}}%
\pgfpathlineto{\pgfqpoint{4.279345in}{1.636806in}}%
\pgfpathlineto{\pgfqpoint{4.279997in}{1.643207in}}%
\pgfpathlineto{\pgfqpoint{4.287206in}{1.649603in}}%
\pgfpathlineto{\pgfqpoint{4.293191in}{1.661575in}}%
\pgfpathlineto{\pgfqpoint{4.305469in}{1.674240in}}%
\pgfpathlineto{\pgfqpoint{4.304597in}{1.686749in}}%
\pgfpathlineto{\pgfqpoint{4.296203in}{1.693698in}}%
\pgfpathlineto{\pgfqpoint{4.292475in}{1.706523in}}%
\pgfpathclose%
\pgfusepath{fill}%
\end{pgfscope}%
\begin{pgfscope}%
\pgfpathrectangle{\pgfqpoint{3.625000in}{0.100000in}}{\pgfqpoint{2.989028in}{1.913466in}}%
\pgfusepath{clip}%
\pgfsetbuttcap%
\pgfsetmiterjoin%
\definecolor{currentfill}{rgb}{0.975010,0.990004,0.710035}%
\pgfsetfillcolor{currentfill}%
\pgfsetlinewidth{0.000000pt}%
\definecolor{currentstroke}{rgb}{0.000000,0.000000,0.000000}%
\pgfsetstrokecolor{currentstroke}%
\pgfsetstrokeopacity{0.000000}%
\pgfsetdash{}{0pt}%
\pgfpathmoveto{\pgfqpoint{6.324865in}{1.496989in}}%
\pgfpathlineto{\pgfqpoint{6.321007in}{1.509167in}}%
\pgfpathlineto{\pgfqpoint{6.313848in}{1.546108in}}%
\pgfpathlineto{\pgfqpoint{6.304593in}{1.557473in}}%
\pgfpathlineto{\pgfqpoint{6.296474in}{1.577911in}}%
\pgfpathlineto{\pgfqpoint{6.299140in}{1.593064in}}%
\pgfpathlineto{\pgfqpoint{6.297002in}{1.604233in}}%
\pgfpathlineto{\pgfqpoint{6.290065in}{1.615202in}}%
\pgfpathlineto{\pgfqpoint{6.289780in}{1.627285in}}%
\pgfpathlineto{\pgfqpoint{6.285752in}{1.640318in}}%
\pgfpathlineto{\pgfqpoint{6.321797in}{1.649189in}}%
\pgfpathlineto{\pgfqpoint{6.368616in}{1.661826in}}%
\pgfpathlineto{\pgfqpoint{6.370485in}{1.654580in}}%
\pgfpathlineto{\pgfqpoint{6.367504in}{1.643043in}}%
\pgfpathlineto{\pgfqpoint{6.374502in}{1.633812in}}%
\pgfpathlineto{\pgfqpoint{6.370785in}{1.622115in}}%
\pgfpathlineto{\pgfqpoint{6.356015in}{1.607411in}}%
\pgfpathlineto{\pgfqpoint{6.360075in}{1.596366in}}%
\pgfpathlineto{\pgfqpoint{6.357066in}{1.573021in}}%
\pgfpathlineto{\pgfqpoint{6.352854in}{1.559858in}}%
\pgfpathlineto{\pgfqpoint{6.358228in}{1.522426in}}%
\pgfpathlineto{\pgfqpoint{6.357504in}{1.509256in}}%
\pgfpathlineto{\pgfqpoint{6.362703in}{1.505030in}}%
\pgfpathlineto{\pgfqpoint{6.324865in}{1.496989in}}%
\pgfpathclose%
\pgfusepath{fill}%
\end{pgfscope}%
\begin{pgfscope}%
\pgfpathrectangle{\pgfqpoint{3.625000in}{0.100000in}}{\pgfqpoint{2.989028in}{1.913466in}}%
\pgfusepath{clip}%
\pgfsetbuttcap%
\pgfsetmiterjoin%
\definecolor{currentfill}{rgb}{0.948097,0.979239,0.668051}%
\pgfsetfillcolor{currentfill}%
\pgfsetlinewidth{0.000000pt}%
\definecolor{currentstroke}{rgb}{0.000000,0.000000,0.000000}%
\pgfsetstrokecolor{currentstroke}%
\pgfsetstrokeopacity{0.000000}%
\pgfsetdash{}{0pt}%
\pgfpathmoveto{\pgfqpoint{5.217728in}{1.422304in}}%
\pgfpathlineto{\pgfqpoint{5.219075in}{1.490552in}}%
\pgfpathlineto{\pgfqpoint{5.219880in}{1.539046in}}%
\pgfpathlineto{\pgfqpoint{5.216374in}{1.545688in}}%
\pgfpathlineto{\pgfqpoint{5.208985in}{1.548158in}}%
\pgfpathlineto{\pgfqpoint{5.202284in}{1.559721in}}%
\pgfpathlineto{\pgfqpoint{5.211034in}{1.569764in}}%
\pgfpathlineto{\pgfqpoint{5.215419in}{1.582557in}}%
\pgfpathlineto{\pgfqpoint{5.216288in}{1.593192in}}%
\pgfpathlineto{\pgfqpoint{5.214274in}{1.609122in}}%
\pgfpathlineto{\pgfqpoint{5.208167in}{1.619346in}}%
\pgfpathlineto{\pgfqpoint{5.206080in}{1.628658in}}%
\pgfpathlineto{\pgfqpoint{5.204560in}{1.663212in}}%
\pgfpathlineto{\pgfqpoint{5.204579in}{1.692956in}}%
\pgfpathlineto{\pgfqpoint{5.196363in}{1.716582in}}%
\pgfpathlineto{\pgfqpoint{5.192777in}{1.732596in}}%
\pgfpathlineto{\pgfqpoint{5.193398in}{1.753978in}}%
\pgfpathlineto{\pgfqpoint{5.195285in}{1.764381in}}%
\pgfpathlineto{\pgfqpoint{5.190014in}{1.786150in}}%
\pgfpathlineto{\pgfqpoint{5.225900in}{1.785431in}}%
\pgfpathlineto{\pgfqpoint{5.280411in}{1.784963in}}%
\pgfpathlineto{\pgfqpoint{5.280709in}{1.809705in}}%
\pgfpathlineto{\pgfqpoint{5.294584in}{1.806981in}}%
\pgfpathlineto{\pgfqpoint{5.301233in}{1.776812in}}%
\pgfpathlineto{\pgfqpoint{5.306136in}{1.765965in}}%
\pgfpathlineto{\pgfqpoint{5.318328in}{1.765645in}}%
\pgfpathlineto{\pgfqpoint{5.321056in}{1.761964in}}%
\pgfpathlineto{\pgfqpoint{5.338062in}{1.760334in}}%
\pgfpathlineto{\pgfqpoint{5.340920in}{1.752855in}}%
\pgfpathlineto{\pgfqpoint{5.352640in}{1.754536in}}%
\pgfpathlineto{\pgfqpoint{5.361742in}{1.761531in}}%
\pgfpathlineto{\pgfqpoint{5.377440in}{1.761270in}}%
\pgfpathlineto{\pgfqpoint{5.387153in}{1.755644in}}%
\pgfpathlineto{\pgfqpoint{5.388270in}{1.750368in}}%
\pgfpathlineto{\pgfqpoint{5.397512in}{1.749264in}}%
\pgfpathlineto{\pgfqpoint{5.403543in}{1.734873in}}%
\pgfpathlineto{\pgfqpoint{5.407434in}{1.743721in}}%
\pgfpathlineto{\pgfqpoint{5.418051in}{1.744265in}}%
\pgfpathlineto{\pgfqpoint{5.420736in}{1.737374in}}%
\pgfpathlineto{\pgfqpoint{5.432682in}{1.734228in}}%
\pgfpathlineto{\pgfqpoint{5.439270in}{1.724302in}}%
\pgfpathlineto{\pgfqpoint{5.453880in}{1.727384in}}%
\pgfpathlineto{\pgfqpoint{5.469957in}{1.739565in}}%
\pgfpathlineto{\pgfqpoint{5.475818in}{1.728827in}}%
\pgfpathlineto{\pgfqpoint{5.502198in}{1.731766in}}%
\pgfpathlineto{\pgfqpoint{5.513474in}{1.724397in}}%
\pgfpathlineto{\pgfqpoint{5.520037in}{1.727028in}}%
\pgfpathlineto{\pgfqpoint{5.525300in}{1.722879in}}%
\pgfpathlineto{\pgfqpoint{5.509687in}{1.713035in}}%
\pgfpathlineto{\pgfqpoint{5.487401in}{1.704265in}}%
\pgfpathlineto{\pgfqpoint{5.465329in}{1.686734in}}%
\pgfpathlineto{\pgfqpoint{5.446193in}{1.663680in}}%
\pgfpathlineto{\pgfqpoint{5.431707in}{1.650050in}}%
\pgfpathlineto{\pgfqpoint{5.419018in}{1.640682in}}%
\pgfpathlineto{\pgfqpoint{5.410290in}{1.631611in}}%
\pgfpathlineto{\pgfqpoint{5.411348in}{1.592965in}}%
\pgfpathlineto{\pgfqpoint{5.394373in}{1.582098in}}%
\pgfpathlineto{\pgfqpoint{5.385801in}{1.568355in}}%
\pgfpathlineto{\pgfqpoint{5.385230in}{1.559657in}}%
\pgfpathlineto{\pgfqpoint{5.392139in}{1.557741in}}%
\pgfpathlineto{\pgfqpoint{5.396061in}{1.547836in}}%
\pgfpathlineto{\pgfqpoint{5.391696in}{1.541277in}}%
\pgfpathlineto{\pgfqpoint{5.392139in}{1.517447in}}%
\pgfpathlineto{\pgfqpoint{5.389840in}{1.504916in}}%
\pgfpathlineto{\pgfqpoint{5.403994in}{1.494027in}}%
\pgfpathlineto{\pgfqpoint{5.411922in}{1.492782in}}%
\pgfpathlineto{\pgfqpoint{5.418466in}{1.485448in}}%
\pgfpathlineto{\pgfqpoint{5.429599in}{1.480942in}}%
\pgfpathlineto{\pgfqpoint{5.435803in}{1.469752in}}%
\pgfpathlineto{\pgfqpoint{5.447805in}{1.460295in}}%
\pgfpathlineto{\pgfqpoint{5.458835in}{1.454564in}}%
\pgfpathlineto{\pgfqpoint{5.465045in}{1.444352in}}%
\pgfpathlineto{\pgfqpoint{5.467134in}{1.425554in}}%
\pgfpathlineto{\pgfqpoint{5.408600in}{1.423428in}}%
\pgfpathlineto{\pgfqpoint{5.358705in}{1.422385in}}%
\pgfpathlineto{\pgfqpoint{5.313246in}{1.421717in}}%
\pgfpathlineto{\pgfqpoint{5.265155in}{1.421790in}}%
\pgfpathlineto{\pgfqpoint{5.217728in}{1.422304in}}%
\pgfpathclose%
\pgfusepath{fill}%
\end{pgfscope}%
\begin{pgfscope}%
\pgfpathrectangle{\pgfqpoint{3.625000in}{0.100000in}}{\pgfqpoint{2.989028in}{1.913466in}}%
\pgfusepath{clip}%
\pgfsetbuttcap%
\pgfsetmiterjoin%
\definecolor{currentfill}{rgb}{0.838447,0.934948,0.608997}%
\pgfsetfillcolor{currentfill}%
\pgfsetlinewidth{0.000000pt}%
\definecolor{currentstroke}{rgb}{0.000000,0.000000,0.000000}%
\pgfsetstrokecolor{currentstroke}%
\pgfsetstrokeopacity{0.000000}%
\pgfsetdash{}{0pt}%
\pgfpathmoveto{\pgfqpoint{3.888713in}{1.549891in}}%
\pgfpathlineto{\pgfqpoint{3.884077in}{1.558421in}}%
\pgfpathlineto{\pgfqpoint{3.887037in}{1.580316in}}%
\pgfpathlineto{\pgfqpoint{3.892750in}{1.591758in}}%
\pgfpathlineto{\pgfqpoint{3.889894in}{1.607237in}}%
\pgfpathlineto{\pgfqpoint{3.895854in}{1.613747in}}%
\pgfpathlineto{\pgfqpoint{3.906727in}{1.631292in}}%
\pgfpathlineto{\pgfqpoint{3.915946in}{1.641883in}}%
\pgfpathlineto{\pgfqpoint{3.929422in}{1.664951in}}%
\pgfpathlineto{\pgfqpoint{3.939758in}{1.690079in}}%
\pgfpathlineto{\pgfqpoint{3.950730in}{1.713646in}}%
\pgfpathlineto{\pgfqpoint{3.952948in}{1.723489in}}%
\pgfpathlineto{\pgfqpoint{3.968085in}{1.751647in}}%
\pgfpathlineto{\pgfqpoint{3.970999in}{1.764014in}}%
\pgfpathlineto{\pgfqpoint{3.977444in}{1.777001in}}%
\pgfpathlineto{\pgfqpoint{3.979747in}{1.792843in}}%
\pgfpathlineto{\pgfqpoint{3.992051in}{1.800568in}}%
\pgfpathlineto{\pgfqpoint{4.006639in}{1.804461in}}%
\pgfpathlineto{\pgfqpoint{4.014015in}{1.795741in}}%
\pgfpathlineto{\pgfqpoint{4.020391in}{1.796386in}}%
\pgfpathlineto{\pgfqpoint{4.030347in}{1.785992in}}%
\pgfpathlineto{\pgfqpoint{4.031670in}{1.776202in}}%
\pgfpathlineto{\pgfqpoint{4.027895in}{1.757031in}}%
\pgfpathlineto{\pgfqpoint{4.040456in}{1.747249in}}%
\pgfpathlineto{\pgfqpoint{4.048603in}{1.743571in}}%
\pgfpathlineto{\pgfqpoint{4.070673in}{1.747325in}}%
\pgfpathlineto{\pgfqpoint{4.083460in}{1.744762in}}%
\pgfpathlineto{\pgfqpoint{4.092248in}{1.740554in}}%
\pgfpathlineto{\pgfqpoint{4.096877in}{1.732568in}}%
\pgfpathlineto{\pgfqpoint{4.126190in}{1.733634in}}%
\pgfpathlineto{\pgfqpoint{4.130851in}{1.728747in}}%
\pgfpathlineto{\pgfqpoint{4.141848in}{1.727692in}}%
\pgfpathlineto{\pgfqpoint{4.152921in}{1.730814in}}%
\pgfpathlineto{\pgfqpoint{4.185663in}{1.730028in}}%
\pgfpathlineto{\pgfqpoint{4.192734in}{1.727622in}}%
\pgfpathlineto{\pgfqpoint{4.200998in}{1.730394in}}%
\pgfpathlineto{\pgfqpoint{4.292475in}{1.706523in}}%
\pgfpathlineto{\pgfqpoint{4.296203in}{1.693698in}}%
\pgfpathlineto{\pgfqpoint{4.304597in}{1.686749in}}%
\pgfpathlineto{\pgfqpoint{4.305469in}{1.674240in}}%
\pgfpathlineto{\pgfqpoint{4.293191in}{1.661575in}}%
\pgfpathlineto{\pgfqpoint{4.287206in}{1.649603in}}%
\pgfpathlineto{\pgfqpoint{4.279997in}{1.643207in}}%
\pgfpathlineto{\pgfqpoint{4.279345in}{1.636806in}}%
\pgfpathlineto{\pgfqpoint{4.267402in}{1.628155in}}%
\pgfpathlineto{\pgfqpoint{4.259208in}{1.615650in}}%
\pgfpathlineto{\pgfqpoint{4.254620in}{1.612613in}}%
\pgfpathlineto{\pgfqpoint{4.252118in}{1.600478in}}%
\pgfpathlineto{\pgfqpoint{4.264170in}{1.588120in}}%
\pgfpathlineto{\pgfqpoint{4.252531in}{1.567559in}}%
\pgfpathlineto{\pgfqpoint{4.251119in}{1.559712in}}%
\pgfpathlineto{\pgfqpoint{4.224946in}{1.452341in}}%
\pgfpathlineto{\pgfqpoint{4.169893in}{1.466447in}}%
\pgfpathlineto{\pgfqpoint{4.116784in}{1.480141in}}%
\pgfpathlineto{\pgfqpoint{4.084744in}{1.489057in}}%
\pgfpathlineto{\pgfqpoint{4.043578in}{1.500784in}}%
\pgfpathlineto{\pgfqpoint{3.977913in}{1.521452in}}%
\pgfpathlineto{\pgfqpoint{3.906532in}{1.543676in}}%
\pgfpathlineto{\pgfqpoint{3.888713in}{1.549891in}}%
\pgfpathclose%
\pgfusepath{fill}%
\end{pgfscope}%
\begin{pgfscope}%
\pgfpathrectangle{\pgfqpoint{3.625000in}{0.100000in}}{\pgfqpoint{2.989028in}{1.913466in}}%
\pgfusepath{clip}%
\pgfsetbuttcap%
\pgfsetmiterjoin%
\definecolor{currentfill}{rgb}{0.982699,0.993080,0.722030}%
\pgfsetfillcolor{currentfill}%
\pgfsetlinewidth{0.000000pt}%
\definecolor{currentstroke}{rgb}{0.000000,0.000000,0.000000}%
\pgfsetstrokecolor{currentstroke}%
\pgfsetstrokeopacity{0.000000}%
\pgfsetdash{}{0pt}%
\pgfpathmoveto{\pgfqpoint{6.362703in}{1.505030in}}%
\pgfpathlineto{\pgfqpoint{6.357504in}{1.509256in}}%
\pgfpathlineto{\pgfqpoint{6.358228in}{1.522426in}}%
\pgfpathlineto{\pgfqpoint{6.352854in}{1.559858in}}%
\pgfpathlineto{\pgfqpoint{6.357066in}{1.573021in}}%
\pgfpathlineto{\pgfqpoint{6.360075in}{1.596366in}}%
\pgfpathlineto{\pgfqpoint{6.356015in}{1.607411in}}%
\pgfpathlineto{\pgfqpoint{6.370785in}{1.622115in}}%
\pgfpathlineto{\pgfqpoint{6.374502in}{1.633812in}}%
\pgfpathlineto{\pgfqpoint{6.367504in}{1.643043in}}%
\pgfpathlineto{\pgfqpoint{6.370485in}{1.654580in}}%
\pgfpathlineto{\pgfqpoint{6.368616in}{1.661826in}}%
\pgfpathlineto{\pgfqpoint{6.370221in}{1.677346in}}%
\pgfpathlineto{\pgfqpoint{6.373221in}{1.682137in}}%
\pgfpathlineto{\pgfqpoint{6.382471in}{1.685161in}}%
\pgfpathlineto{\pgfqpoint{6.395964in}{1.645645in}}%
\pgfpathlineto{\pgfqpoint{6.420998in}{1.563760in}}%
\pgfpathlineto{\pgfqpoint{6.430471in}{1.557411in}}%
\pgfpathlineto{\pgfqpoint{6.432175in}{1.550163in}}%
\pgfpathlineto{\pgfqpoint{6.437175in}{1.547235in}}%
\pgfpathlineto{\pgfqpoint{6.436795in}{1.534090in}}%
\pgfpathlineto{\pgfqpoint{6.431497in}{1.533879in}}%
\pgfpathlineto{\pgfqpoint{6.420764in}{1.525694in}}%
\pgfpathlineto{\pgfqpoint{6.417669in}{1.517482in}}%
\pgfpathlineto{\pgfqpoint{6.388942in}{1.510374in}}%
\pgfpathlineto{\pgfqpoint{6.362703in}{1.505030in}}%
\pgfpathclose%
\pgfusepath{fill}%
\end{pgfscope}%
\begin{pgfscope}%
\pgfpathrectangle{\pgfqpoint{3.625000in}{0.100000in}}{\pgfqpoint{2.989028in}{1.913466in}}%
\pgfusepath{clip}%
\pgfsetbuttcap%
\pgfsetmiterjoin%
\definecolor{currentfill}{rgb}{0.932718,0.973087,0.644060}%
\pgfsetfillcolor{currentfill}%
\pgfsetlinewidth{0.000000pt}%
\definecolor{currentstroke}{rgb}{0.000000,0.000000,0.000000}%
\pgfsetstrokecolor{currentstroke}%
\pgfsetstrokeopacity{0.000000}%
\pgfsetdash{}{0pt}%
\pgfpathmoveto{\pgfqpoint{5.464431in}{1.220143in}}%
\pgfpathlineto{\pgfqpoint{5.449266in}{1.235161in}}%
\pgfpathlineto{\pgfqpoint{5.400789in}{1.232365in}}%
\pgfpathlineto{\pgfqpoint{5.325184in}{1.229874in}}%
\pgfpathlineto{\pgfqpoint{5.249124in}{1.231060in}}%
\pgfpathlineto{\pgfqpoint{5.243785in}{1.240370in}}%
\pgfpathlineto{\pgfqpoint{5.245965in}{1.249511in}}%
\pgfpathlineto{\pgfqpoint{5.244930in}{1.265170in}}%
\pgfpathlineto{\pgfqpoint{5.240910in}{1.280338in}}%
\pgfpathlineto{\pgfqpoint{5.241378in}{1.288348in}}%
\pgfpathlineto{\pgfqpoint{5.232833in}{1.297398in}}%
\pgfpathlineto{\pgfqpoint{5.234692in}{1.309990in}}%
\pgfpathlineto{\pgfqpoint{5.221585in}{1.334864in}}%
\pgfpathlineto{\pgfqpoint{5.217680in}{1.355981in}}%
\pgfpathlineto{\pgfqpoint{5.208252in}{1.373157in}}%
\pgfpathlineto{\pgfqpoint{5.217201in}{1.394464in}}%
\pgfpathlineto{\pgfqpoint{5.212744in}{1.404379in}}%
\pgfpathlineto{\pgfqpoint{5.217728in}{1.422304in}}%
\pgfpathlineto{\pgfqpoint{5.265155in}{1.421790in}}%
\pgfpathlineto{\pgfqpoint{5.313246in}{1.421717in}}%
\pgfpathlineto{\pgfqpoint{5.358705in}{1.422385in}}%
\pgfpathlineto{\pgfqpoint{5.408600in}{1.423428in}}%
\pgfpathlineto{\pgfqpoint{5.467134in}{1.425554in}}%
\pgfpathlineto{\pgfqpoint{5.468697in}{1.417056in}}%
\pgfpathlineto{\pgfqpoint{5.475178in}{1.409931in}}%
\pgfpathlineto{\pgfqpoint{5.471075in}{1.402507in}}%
\pgfpathlineto{\pgfqpoint{5.472756in}{1.388016in}}%
\pgfpathlineto{\pgfqpoint{5.476772in}{1.376303in}}%
\pgfpathlineto{\pgfqpoint{5.494980in}{1.370103in}}%
\pgfpathlineto{\pgfqpoint{5.498100in}{1.361809in}}%
\pgfpathlineto{\pgfqpoint{5.508091in}{1.352497in}}%
\pgfpathlineto{\pgfqpoint{5.512167in}{1.342864in}}%
\pgfpathlineto{\pgfqpoint{5.522291in}{1.336402in}}%
\pgfpathlineto{\pgfqpoint{5.523875in}{1.328623in}}%
\pgfpathlineto{\pgfqpoint{5.521904in}{1.316845in}}%
\pgfpathlineto{\pgfqpoint{5.516751in}{1.313315in}}%
\pgfpathlineto{\pgfqpoint{5.515197in}{1.302104in}}%
\pgfpathlineto{\pgfqpoint{5.500389in}{1.293200in}}%
\pgfpathlineto{\pgfqpoint{5.481087in}{1.288322in}}%
\pgfpathlineto{\pgfqpoint{5.479318in}{1.277106in}}%
\pgfpathlineto{\pgfqpoint{5.486764in}{1.269093in}}%
\pgfpathlineto{\pgfqpoint{5.487068in}{1.259018in}}%
\pgfpathlineto{\pgfqpoint{5.481060in}{1.251098in}}%
\pgfpathlineto{\pgfqpoint{5.477907in}{1.239332in}}%
\pgfpathlineto{\pgfqpoint{5.467465in}{1.235438in}}%
\pgfpathlineto{\pgfqpoint{5.468130in}{1.222326in}}%
\pgfpathlineto{\pgfqpoint{5.464431in}{1.220143in}}%
\pgfpathclose%
\pgfusepath{fill}%
\end{pgfscope}%
\begin{pgfscope}%
\pgfpathrectangle{\pgfqpoint{3.625000in}{0.100000in}}{\pgfqpoint{2.989028in}{1.913466in}}%
\pgfusepath{clip}%
\pgfsetbuttcap%
\pgfsetmiterjoin%
\definecolor{currentfill}{rgb}{0.921184,0.968474,0.626067}%
\pgfsetfillcolor{currentfill}%
\pgfsetlinewidth{0.000000pt}%
\definecolor{currentstroke}{rgb}{0.000000,0.000000,0.000000}%
\pgfsetstrokecolor{currentstroke}%
\pgfsetstrokeopacity{0.000000}%
\pgfsetdash{}{0pt}%
\pgfpathmoveto{\pgfqpoint{6.405364in}{1.467139in}}%
\pgfpathlineto{\pgfqpoint{6.391000in}{1.464872in}}%
\pgfpathlineto{\pgfqpoint{6.325022in}{1.449880in}}%
\pgfpathlineto{\pgfqpoint{6.323869in}{1.451627in}}%
\pgfpathlineto{\pgfqpoint{6.324865in}{1.496989in}}%
\pgfpathlineto{\pgfqpoint{6.362703in}{1.505030in}}%
\pgfpathlineto{\pgfqpoint{6.388942in}{1.510374in}}%
\pgfpathlineto{\pgfqpoint{6.417669in}{1.517482in}}%
\pgfpathlineto{\pgfqpoint{6.420764in}{1.525694in}}%
\pgfpathlineto{\pgfqpoint{6.431497in}{1.533879in}}%
\pgfpathlineto{\pgfqpoint{6.436795in}{1.534090in}}%
\pgfpathlineto{\pgfqpoint{6.443759in}{1.522149in}}%
\pgfpathlineto{\pgfqpoint{6.437440in}{1.504717in}}%
\pgfpathlineto{\pgfqpoint{6.436512in}{1.494495in}}%
\pgfpathlineto{\pgfqpoint{6.449304in}{1.495455in}}%
\pgfpathlineto{\pgfqpoint{6.455105in}{1.490552in}}%
\pgfpathlineto{\pgfqpoint{6.468114in}{1.470509in}}%
\pgfpathlineto{\pgfqpoint{6.474579in}{1.468072in}}%
\pgfpathlineto{\pgfqpoint{6.485410in}{1.468976in}}%
\pgfpathlineto{\pgfqpoint{6.492946in}{1.475787in}}%
\pgfpathlineto{\pgfqpoint{6.497956in}{1.469708in}}%
\pgfpathlineto{\pgfqpoint{6.466451in}{1.453083in}}%
\pgfpathlineto{\pgfqpoint{6.465467in}{1.465013in}}%
\pgfpathlineto{\pgfqpoint{6.451155in}{1.446471in}}%
\pgfpathlineto{\pgfqpoint{6.446123in}{1.443277in}}%
\pgfpathlineto{\pgfqpoint{6.439105in}{1.453976in}}%
\pgfpathlineto{\pgfqpoint{6.437183in}{1.455442in}}%
\pgfpathlineto{\pgfqpoint{6.430650in}{1.458903in}}%
\pgfpathlineto{\pgfqpoint{6.424952in}{1.472940in}}%
\pgfpathlineto{\pgfqpoint{6.405364in}{1.467139in}}%
\pgfpathclose%
\pgfusepath{fill}%
\end{pgfscope}%
\begin{pgfscope}%
\pgfpathrectangle{\pgfqpoint{3.625000in}{0.100000in}}{\pgfqpoint{2.989028in}{1.913466in}}%
\pgfusepath{clip}%
\pgfsetbuttcap%
\pgfsetmiterjoin%
\definecolor{currentfill}{rgb}{0.328028,0.680507,0.680277}%
\pgfsetfillcolor{currentfill}%
\pgfsetlinewidth{0.000000pt}%
\definecolor{currentstroke}{rgb}{0.000000,0.000000,0.000000}%
\pgfsetstrokecolor{currentstroke}%
\pgfsetstrokeopacity{0.000000}%
\pgfsetdash{}{0pt}%
\pgfpathmoveto{\pgfqpoint{4.933935in}{1.205970in}}%
\pgfpathlineto{\pgfqpoint{4.939195in}{1.271196in}}%
\pgfpathlineto{\pgfqpoint{4.909404in}{1.273614in}}%
\pgfpathlineto{\pgfqpoint{4.840540in}{1.280295in}}%
\pgfpathlineto{\pgfqpoint{4.846947in}{1.345289in}}%
\pgfpathlineto{\pgfqpoint{4.853363in}{1.410780in}}%
\pgfpathlineto{\pgfqpoint{4.879836in}{1.407921in}}%
\pgfpathlineto{\pgfqpoint{4.947447in}{1.401533in}}%
\pgfpathlineto{\pgfqpoint{4.988365in}{1.398329in}}%
\pgfpathlineto{\pgfqpoint{5.037902in}{1.395489in}}%
\pgfpathlineto{\pgfqpoint{5.119439in}{1.391630in}}%
\pgfpathlineto{\pgfqpoint{5.121231in}{1.387975in}}%
\pgfpathlineto{\pgfqpoint{5.144594in}{1.376416in}}%
\pgfpathlineto{\pgfqpoint{5.151759in}{1.382286in}}%
\pgfpathlineto{\pgfqpoint{5.172400in}{1.381750in}}%
\pgfpathlineto{\pgfqpoint{5.184700in}{1.375307in}}%
\pgfpathlineto{\pgfqpoint{5.204205in}{1.367982in}}%
\pgfpathlineto{\pgfqpoint{5.209670in}{1.357295in}}%
\pgfpathlineto{\pgfqpoint{5.217680in}{1.355981in}}%
\pgfpathlineto{\pgfqpoint{5.221585in}{1.334864in}}%
\pgfpathlineto{\pgfqpoint{5.234692in}{1.309990in}}%
\pgfpathlineto{\pgfqpoint{5.232833in}{1.297398in}}%
\pgfpathlineto{\pgfqpoint{5.241378in}{1.288348in}}%
\pgfpathlineto{\pgfqpoint{5.240910in}{1.280338in}}%
\pgfpathlineto{\pgfqpoint{5.244930in}{1.265170in}}%
\pgfpathlineto{\pgfqpoint{5.245965in}{1.249511in}}%
\pgfpathlineto{\pgfqpoint{5.243785in}{1.240370in}}%
\pgfpathlineto{\pgfqpoint{5.249124in}{1.231060in}}%
\pgfpathlineto{\pgfqpoint{5.256447in}{1.214132in}}%
\pgfpathlineto{\pgfqpoint{5.263455in}{1.207242in}}%
\pgfpathlineto{\pgfqpoint{5.271810in}{1.192312in}}%
\pgfpathlineto{\pgfqpoint{5.248131in}{1.192066in}}%
\pgfpathlineto{\pgfqpoint{5.196933in}{1.192859in}}%
\pgfpathlineto{\pgfqpoint{5.140367in}{1.194592in}}%
\pgfpathlineto{\pgfqpoint{5.083468in}{1.196775in}}%
\pgfpathlineto{\pgfqpoint{4.999809in}{1.201345in}}%
\pgfpathlineto{\pgfqpoint{4.933935in}{1.205970in}}%
\pgfpathclose%
\pgfusepath{fill}%
\end{pgfscope}%
\begin{pgfscope}%
\pgfpathrectangle{\pgfqpoint{3.625000in}{0.100000in}}{\pgfqpoint{2.989028in}{1.913466in}}%
\pgfusepath{clip}%
\pgfsetbuttcap%
\pgfsetmiterjoin%
\definecolor{currentfill}{rgb}{0.971165,0.988466,0.704037}%
\pgfsetfillcolor{currentfill}%
\pgfsetlinewidth{0.000000pt}%
\definecolor{currentstroke}{rgb}{0.000000,0.000000,0.000000}%
\pgfsetstrokecolor{currentstroke}%
\pgfsetstrokeopacity{0.000000}%
\pgfsetdash{}{0pt}%
\pgfpathmoveto{\pgfqpoint{6.023213in}{1.402714in}}%
\pgfpathlineto{\pgfqpoint{6.049409in}{1.427692in}}%
\pgfpathlineto{\pgfqpoint{6.052665in}{1.436570in}}%
\pgfpathlineto{\pgfqpoint{6.060341in}{1.444171in}}%
\pgfpathlineto{\pgfqpoint{6.054581in}{1.455238in}}%
\pgfpathlineto{\pgfqpoint{6.047360in}{1.461711in}}%
\pgfpathlineto{\pgfqpoint{6.045273in}{1.473182in}}%
\pgfpathlineto{\pgfqpoint{6.072151in}{1.484959in}}%
\pgfpathlineto{\pgfqpoint{6.094405in}{1.488685in}}%
\pgfpathlineto{\pgfqpoint{6.106363in}{1.488934in}}%
\pgfpathlineto{\pgfqpoint{6.115473in}{1.484433in}}%
\pgfpathlineto{\pgfqpoint{6.124352in}{1.488453in}}%
\pgfpathlineto{\pgfqpoint{6.146010in}{1.492956in}}%
\pgfpathlineto{\pgfqpoint{6.153493in}{1.498769in}}%
\pgfpathlineto{\pgfqpoint{6.164589in}{1.511638in}}%
\pgfpathlineto{\pgfqpoint{6.174683in}{1.517325in}}%
\pgfpathlineto{\pgfqpoint{6.175395in}{1.522779in}}%
\pgfpathlineto{\pgfqpoint{6.170097in}{1.535222in}}%
\pgfpathlineto{\pgfqpoint{6.173927in}{1.542544in}}%
\pgfpathlineto{\pgfqpoint{6.168770in}{1.550418in}}%
\pgfpathlineto{\pgfqpoint{6.160873in}{1.550968in}}%
\pgfpathlineto{\pgfqpoint{6.180638in}{1.574751in}}%
\pgfpathlineto{\pgfqpoint{6.182985in}{1.583815in}}%
\pgfpathlineto{\pgfqpoint{6.198541in}{1.606931in}}%
\pgfpathlineto{\pgfqpoint{6.212981in}{1.619440in}}%
\pgfpathlineto{\pgfqpoint{6.222897in}{1.624664in}}%
\pgfpathlineto{\pgfqpoint{6.255342in}{1.632016in}}%
\pgfpathlineto{\pgfqpoint{6.285752in}{1.640318in}}%
\pgfpathlineto{\pgfqpoint{6.289780in}{1.627285in}}%
\pgfpathlineto{\pgfqpoint{6.290065in}{1.615202in}}%
\pgfpathlineto{\pgfqpoint{6.297002in}{1.604233in}}%
\pgfpathlineto{\pgfqpoint{6.299140in}{1.593064in}}%
\pgfpathlineto{\pgfqpoint{6.296474in}{1.577911in}}%
\pgfpathlineto{\pgfqpoint{6.304593in}{1.557473in}}%
\pgfpathlineto{\pgfqpoint{6.313848in}{1.546108in}}%
\pgfpathlineto{\pgfqpoint{6.321007in}{1.509167in}}%
\pgfpathlineto{\pgfqpoint{6.324865in}{1.496989in}}%
\pgfpathlineto{\pgfqpoint{6.323869in}{1.451627in}}%
\pgfpathlineto{\pgfqpoint{6.325022in}{1.449880in}}%
\pgfpathlineto{\pgfqpoint{6.333450in}{1.401092in}}%
\pgfpathlineto{\pgfqpoint{6.338176in}{1.396647in}}%
\pgfpathlineto{\pgfqpoint{6.328010in}{1.386769in}}%
\pgfpathlineto{\pgfqpoint{6.333026in}{1.381100in}}%
\pgfpathlineto{\pgfqpoint{6.328617in}{1.372526in}}%
\pgfpathlineto{\pgfqpoint{6.328654in}{1.368875in}}%
\pgfpathlineto{\pgfqpoint{6.323142in}{1.365581in}}%
\pgfpathlineto{\pgfqpoint{6.320458in}{1.358300in}}%
\pgfpathlineto{\pgfqpoint{6.321310in}{1.378322in}}%
\pgfpathlineto{\pgfqpoint{6.304225in}{1.382730in}}%
\pgfpathlineto{\pgfqpoint{6.277507in}{1.391844in}}%
\pgfpathlineto{\pgfqpoint{6.273728in}{1.396333in}}%
\pgfpathlineto{\pgfqpoint{6.262564in}{1.397408in}}%
\pgfpathlineto{\pgfqpoint{6.256142in}{1.404090in}}%
\pgfpathlineto{\pgfqpoint{6.252676in}{1.417397in}}%
\pgfpathlineto{\pgfqpoint{6.243572in}{1.419052in}}%
\pgfpathlineto{\pgfqpoint{6.237484in}{1.426033in}}%
\pgfpathlineto{\pgfqpoint{6.159972in}{1.410406in}}%
\pgfpathlineto{\pgfqpoint{6.122928in}{1.402745in}}%
\pgfpathlineto{\pgfqpoint{6.066683in}{1.392479in}}%
\pgfpathlineto{\pgfqpoint{6.026180in}{1.385646in}}%
\pgfpathlineto{\pgfqpoint{6.023213in}{1.402714in}}%
\pgfpathclose%
\pgfusepath{fill}%
\end{pgfscope}%
\begin{pgfscope}%
\pgfpathrectangle{\pgfqpoint{3.625000in}{0.100000in}}{\pgfqpoint{2.989028in}{1.913466in}}%
\pgfusepath{clip}%
\pgfsetbuttcap%
\pgfsetmiterjoin%
\definecolor{currentfill}{rgb}{0.971165,0.988466,0.704037}%
\pgfsetfillcolor{currentfill}%
\pgfsetlinewidth{0.000000pt}%
\definecolor{currentstroke}{rgb}{0.000000,0.000000,0.000000}%
\pgfsetstrokecolor{currentstroke}%
\pgfsetstrokeopacity{0.000000}%
\pgfsetdash{}{0pt}%
\pgfpathmoveto{\pgfqpoint{6.334390in}{1.354233in}}%
\pgfpathlineto{\pgfqpoint{6.322403in}{1.350498in}}%
\pgfpathlineto{\pgfqpoint{6.320386in}{1.353932in}}%
\pgfpathlineto{\pgfqpoint{6.324234in}{1.365456in}}%
\pgfpathlineto{\pgfqpoint{6.331360in}{1.366591in}}%
\pgfpathlineto{\pgfqpoint{6.330722in}{1.370219in}}%
\pgfpathlineto{\pgfqpoint{6.337120in}{1.375665in}}%
\pgfpathlineto{\pgfqpoint{6.355618in}{1.380004in}}%
\pgfpathlineto{\pgfqpoint{6.363853in}{1.386571in}}%
\pgfpathlineto{\pgfqpoint{6.382353in}{1.392045in}}%
\pgfpathlineto{\pgfqpoint{6.390790in}{1.390042in}}%
\pgfpathlineto{\pgfqpoint{6.397888in}{1.398866in}}%
\pgfpathlineto{\pgfqpoint{6.408615in}{1.400032in}}%
\pgfpathlineto{\pgfqpoint{6.390314in}{1.382820in}}%
\pgfpathlineto{\pgfqpoint{6.334390in}{1.354233in}}%
\pgfpathclose%
\pgfusepath{fill}%
\end{pgfscope}%
\begin{pgfscope}%
\pgfpathrectangle{\pgfqpoint{3.625000in}{0.100000in}}{\pgfqpoint{2.989028in}{1.913466in}}%
\pgfusepath{clip}%
\pgfsetbuttcap%
\pgfsetmiterjoin%
\definecolor{currentfill}{rgb}{0.874740,0.949712,0.601615}%
\pgfsetfillcolor{currentfill}%
\pgfsetlinewidth{0.000000pt}%
\definecolor{currentstroke}{rgb}{0.000000,0.000000,0.000000}%
\pgfsetstrokecolor{currentstroke}%
\pgfsetstrokeopacity{0.000000}%
\pgfsetdash{}{0pt}%
\pgfpathmoveto{\pgfqpoint{6.065072in}{1.240581in}}%
\pgfpathlineto{\pgfqpoint{6.013234in}{1.232010in}}%
\pgfpathlineto{\pgfqpoint{6.003831in}{1.291252in}}%
\pgfpathlineto{\pgfqpoint{5.989869in}{1.378577in}}%
\pgfpathlineto{\pgfqpoint{6.023213in}{1.402714in}}%
\pgfpathlineto{\pgfqpoint{6.026180in}{1.385646in}}%
\pgfpathlineto{\pgfqpoint{6.066683in}{1.392479in}}%
\pgfpathlineto{\pgfqpoint{6.122928in}{1.402745in}}%
\pgfpathlineto{\pgfqpoint{6.159972in}{1.410406in}}%
\pgfpathlineto{\pgfqpoint{6.237484in}{1.426033in}}%
\pgfpathlineto{\pgfqpoint{6.243572in}{1.419052in}}%
\pgfpathlineto{\pgfqpoint{6.252676in}{1.417397in}}%
\pgfpathlineto{\pgfqpoint{6.256142in}{1.404090in}}%
\pgfpathlineto{\pgfqpoint{6.262564in}{1.397408in}}%
\pgfpathlineto{\pgfqpoint{6.273728in}{1.396333in}}%
\pgfpathlineto{\pgfqpoint{6.277507in}{1.391844in}}%
\pgfpathlineto{\pgfqpoint{6.273671in}{1.388380in}}%
\pgfpathlineto{\pgfqpoint{6.270215in}{1.376126in}}%
\pgfpathlineto{\pgfqpoint{6.262116in}{1.362357in}}%
\pgfpathlineto{\pgfqpoint{6.267552in}{1.356369in}}%
\pgfpathlineto{\pgfqpoint{6.263104in}{1.349953in}}%
\pgfpathlineto{\pgfqpoint{6.265613in}{1.335884in}}%
\pgfpathlineto{\pgfqpoint{6.273625in}{1.328499in}}%
\pgfpathlineto{\pgfqpoint{6.292634in}{1.316500in}}%
\pgfpathlineto{\pgfqpoint{6.277326in}{1.299570in}}%
\pgfpathlineto{\pgfqpoint{6.277111in}{1.293145in}}%
\pgfpathlineto{\pgfqpoint{6.264648in}{1.284858in}}%
\pgfpathlineto{\pgfqpoint{6.250867in}{1.283302in}}%
\pgfpathlineto{\pgfqpoint{6.247458in}{1.276100in}}%
\pgfpathlineto{\pgfqpoint{6.209127in}{1.267806in}}%
\pgfpathlineto{\pgfqpoint{6.164399in}{1.258797in}}%
\pgfpathlineto{\pgfqpoint{6.133655in}{1.253292in}}%
\pgfpathlineto{\pgfqpoint{6.065072in}{1.240581in}}%
\pgfpathclose%
\pgfusepath{fill}%
\end{pgfscope}%
\begin{pgfscope}%
\pgfpathrectangle{\pgfqpoint{3.625000in}{0.100000in}}{\pgfqpoint{2.989028in}{1.913466in}}%
\pgfusepath{clip}%
\pgfsetbuttcap%
\pgfsetmiterjoin%
\definecolor{currentfill}{rgb}{0.820300,0.927566,0.612687}%
\pgfsetfillcolor{currentfill}%
\pgfsetlinewidth{0.000000pt}%
\definecolor{currentstroke}{rgb}{0.000000,0.000000,0.000000}%
\pgfsetstrokecolor{currentstroke}%
\pgfsetstrokeopacity{0.000000}%
\pgfsetdash{}{0pt}%
\pgfpathmoveto{\pgfqpoint{6.325022in}{1.449880in}}%
\pgfpathlineto{\pgfqpoint{6.391000in}{1.464872in}}%
\pgfpathlineto{\pgfqpoint{6.405364in}{1.467139in}}%
\pgfpathlineto{\pgfqpoint{6.414871in}{1.429760in}}%
\pgfpathlineto{\pgfqpoint{6.413366in}{1.423067in}}%
\pgfpathlineto{\pgfqpoint{6.382827in}{1.411244in}}%
\pgfpathlineto{\pgfqpoint{6.364596in}{1.407111in}}%
\pgfpathlineto{\pgfqpoint{6.356849in}{1.397842in}}%
\pgfpathlineto{\pgfqpoint{6.333026in}{1.381100in}}%
\pgfpathlineto{\pgfqpoint{6.328010in}{1.386769in}}%
\pgfpathlineto{\pgfqpoint{6.338176in}{1.396647in}}%
\pgfpathlineto{\pgfqpoint{6.333450in}{1.401092in}}%
\pgfpathlineto{\pgfqpoint{6.325022in}{1.449880in}}%
\pgfpathclose%
\pgfusepath{fill}%
\end{pgfscope}%
\begin{pgfscope}%
\pgfpathrectangle{\pgfqpoint{3.625000in}{0.100000in}}{\pgfqpoint{2.989028in}{1.913466in}}%
\pgfusepath{clip}%
\pgfsetbuttcap%
\pgfsetmiterjoin%
\definecolor{currentfill}{rgb}{0.982699,0.993080,0.722030}%
\pgfsetfillcolor{currentfill}%
\pgfsetlinewidth{0.000000pt}%
\definecolor{currentstroke}{rgb}{0.000000,0.000000,0.000000}%
\pgfsetstrokecolor{currentstroke}%
\pgfsetstrokeopacity{0.000000}%
\pgfsetdash{}{0pt}%
\pgfpathmoveto{\pgfqpoint{6.405364in}{1.467139in}}%
\pgfpathlineto{\pgfqpoint{6.424952in}{1.472940in}}%
\pgfpathlineto{\pgfqpoint{6.430650in}{1.458903in}}%
\pgfpathlineto{\pgfqpoint{6.437183in}{1.455442in}}%
\pgfpathlineto{\pgfqpoint{6.429121in}{1.449529in}}%
\pgfpathlineto{\pgfqpoint{6.430138in}{1.432164in}}%
\pgfpathlineto{\pgfqpoint{6.413366in}{1.423067in}}%
\pgfpathlineto{\pgfqpoint{6.414871in}{1.429760in}}%
\pgfpathlineto{\pgfqpoint{6.405364in}{1.467139in}}%
\pgfpathclose%
\pgfusepath{fill}%
\end{pgfscope}%
\begin{pgfscope}%
\pgfpathrectangle{\pgfqpoint{3.625000in}{0.100000in}}{\pgfqpoint{2.989028in}{1.913466in}}%
\pgfusepath{clip}%
\pgfsetbuttcap%
\pgfsetmiterjoin%
\definecolor{currentfill}{rgb}{0.940408,0.976163,0.656055}%
\pgfsetfillcolor{currentfill}%
\pgfsetlinewidth{0.000000pt}%
\definecolor{currentstroke}{rgb}{0.000000,0.000000,0.000000}%
\pgfsetstrokecolor{currentstroke}%
\pgfsetstrokeopacity{0.000000}%
\pgfsetdash{}{0pt}%
\pgfpathmoveto{\pgfqpoint{6.262440in}{1.278737in}}%
\pgfpathlineto{\pgfqpoint{6.264648in}{1.284858in}}%
\pgfpathlineto{\pgfqpoint{6.277111in}{1.293145in}}%
\pgfpathlineto{\pgfqpoint{6.277326in}{1.299570in}}%
\pgfpathlineto{\pgfqpoint{6.292634in}{1.316500in}}%
\pgfpathlineto{\pgfqpoint{6.273625in}{1.328499in}}%
\pgfpathlineto{\pgfqpoint{6.265613in}{1.335884in}}%
\pgfpathlineto{\pgfqpoint{6.263104in}{1.349953in}}%
\pgfpathlineto{\pgfqpoint{6.267552in}{1.356369in}}%
\pgfpathlineto{\pgfqpoint{6.262116in}{1.362357in}}%
\pgfpathlineto{\pgfqpoint{6.270215in}{1.376126in}}%
\pgfpathlineto{\pgfqpoint{6.273671in}{1.388380in}}%
\pgfpathlineto{\pgfqpoint{6.277507in}{1.391844in}}%
\pgfpathlineto{\pgfqpoint{6.304225in}{1.382730in}}%
\pgfpathlineto{\pgfqpoint{6.321310in}{1.378322in}}%
\pgfpathlineto{\pgfqpoint{6.320458in}{1.358300in}}%
\pgfpathlineto{\pgfqpoint{6.310080in}{1.343116in}}%
\pgfpathlineto{\pgfqpoint{6.318632in}{1.340882in}}%
\pgfpathlineto{\pgfqpoint{6.327511in}{1.334366in}}%
\pgfpathlineto{\pgfqpoint{6.328034in}{1.316545in}}%
\pgfpathlineto{\pgfqpoint{6.325347in}{1.303928in}}%
\pgfpathlineto{\pgfqpoint{6.327135in}{1.293566in}}%
\pgfpathlineto{\pgfqpoint{6.311701in}{1.258597in}}%
\pgfpathlineto{\pgfqpoint{6.306154in}{1.242268in}}%
\pgfpathlineto{\pgfqpoint{6.298409in}{1.250205in}}%
\pgfpathlineto{\pgfqpoint{6.288139in}{1.248853in}}%
\pgfpathlineto{\pgfqpoint{6.262435in}{1.263703in}}%
\pgfpathlineto{\pgfqpoint{6.259805in}{1.271656in}}%
\pgfpathlineto{\pgfqpoint{6.262440in}{1.278737in}}%
\pgfpathclose%
\pgfusepath{fill}%
\end{pgfscope}%
\begin{pgfscope}%
\pgfpathrectangle{\pgfqpoint{3.625000in}{0.100000in}}{\pgfqpoint{2.989028in}{1.913466in}}%
\pgfusepath{clip}%
\pgfsetbuttcap%
\pgfsetmiterjoin%
\definecolor{currentfill}{rgb}{0.951942,0.980777,0.674048}%
\pgfsetfillcolor{currentfill}%
\pgfsetlinewidth{0.000000pt}%
\definecolor{currentstroke}{rgb}{0.000000,0.000000,0.000000}%
\pgfsetstrokecolor{currentstroke}%
\pgfsetstrokeopacity{0.000000}%
\pgfsetdash{}{0pt}%
\pgfpathmoveto{\pgfqpoint{5.647290in}{1.062454in}}%
\pgfpathlineto{\pgfqpoint{5.646088in}{1.078676in}}%
\pgfpathlineto{\pgfqpoint{5.650759in}{1.086724in}}%
\pgfpathlineto{\pgfqpoint{5.647681in}{1.090668in}}%
\pgfpathlineto{\pgfqpoint{5.658907in}{1.103580in}}%
\pgfpathlineto{\pgfqpoint{5.658263in}{1.106692in}}%
\pgfpathlineto{\pgfqpoint{5.669170in}{1.129219in}}%
\pgfpathlineto{\pgfqpoint{5.666966in}{1.140083in}}%
\pgfpathlineto{\pgfqpoint{5.659057in}{1.151457in}}%
\pgfpathlineto{\pgfqpoint{5.664547in}{1.165295in}}%
\pgfpathlineto{\pgfqpoint{5.657431in}{1.256358in}}%
\pgfpathlineto{\pgfqpoint{5.652306in}{1.320243in}}%
\pgfpathlineto{\pgfqpoint{5.659385in}{1.314949in}}%
\pgfpathlineto{\pgfqpoint{5.667283in}{1.315093in}}%
\pgfpathlineto{\pgfqpoint{5.685936in}{1.325898in}}%
\pgfpathlineto{\pgfqpoint{5.743150in}{1.331236in}}%
\pgfpathlineto{\pgfqpoint{5.785498in}{1.335699in}}%
\pgfpathlineto{\pgfqpoint{5.785872in}{1.331556in}}%
\pgfpathlineto{\pgfqpoint{5.795539in}{1.244011in}}%
\pgfpathlineto{\pgfqpoint{5.803860in}{1.162554in}}%
\pgfpathlineto{\pgfqpoint{5.800273in}{1.158730in}}%
\pgfpathlineto{\pgfqpoint{5.805783in}{1.149238in}}%
\pgfpathlineto{\pgfqpoint{5.805755in}{1.142397in}}%
\pgfpathlineto{\pgfqpoint{5.797893in}{1.140679in}}%
\pgfpathlineto{\pgfqpoint{5.789096in}{1.134084in}}%
\pgfpathlineto{\pgfqpoint{5.783139in}{1.136680in}}%
\pgfpathlineto{\pgfqpoint{5.774225in}{1.132459in}}%
\pgfpathlineto{\pgfqpoint{5.776983in}{1.123981in}}%
\pgfpathlineto{\pgfqpoint{5.767820in}{1.115463in}}%
\pgfpathlineto{\pgfqpoint{5.765290in}{1.105610in}}%
\pgfpathlineto{\pgfqpoint{5.758989in}{1.103993in}}%
\pgfpathlineto{\pgfqpoint{5.754287in}{1.096531in}}%
\pgfpathlineto{\pgfqpoint{5.754903in}{1.089016in}}%
\pgfpathlineto{\pgfqpoint{5.749371in}{1.083732in}}%
\pgfpathlineto{\pgfqpoint{5.741048in}{1.084548in}}%
\pgfpathlineto{\pgfqpoint{5.734721in}{1.092656in}}%
\pgfpathlineto{\pgfqpoint{5.724000in}{1.084848in}}%
\pgfpathlineto{\pgfqpoint{5.724753in}{1.078025in}}%
\pgfpathlineto{\pgfqpoint{5.712829in}{1.074039in}}%
\pgfpathlineto{\pgfqpoint{5.709834in}{1.079056in}}%
\pgfpathlineto{\pgfqpoint{5.698226in}{1.073367in}}%
\pgfpathlineto{\pgfqpoint{5.693985in}{1.065222in}}%
\pgfpathlineto{\pgfqpoint{5.680006in}{1.073574in}}%
\pgfpathlineto{\pgfqpoint{5.657850in}{1.068041in}}%
\pgfpathlineto{\pgfqpoint{5.647290in}{1.062454in}}%
\pgfpathclose%
\pgfusepath{fill}%
\end{pgfscope}%
\begin{pgfscope}%
\pgfpathrectangle{\pgfqpoint{3.625000in}{0.100000in}}{\pgfqpoint{2.989028in}{1.913466in}}%
\pgfusepath{clip}%
\pgfsetbuttcap%
\pgfsetmiterjoin%
\definecolor{currentfill}{rgb}{0.905805,0.962322,0.602076}%
\pgfsetfillcolor{currentfill}%
\pgfsetlinewidth{0.000000pt}%
\definecolor{currentstroke}{rgb}{0.000000,0.000000,0.000000}%
\pgfsetstrokecolor{currentstroke}%
\pgfsetstrokeopacity{0.000000}%
\pgfsetdash{}{0pt}%
\pgfpathmoveto{\pgfqpoint{4.084744in}{1.489057in}}%
\pgfpathlineto{\pgfqpoint{4.116784in}{1.480141in}}%
\pgfpathlineto{\pgfqpoint{4.169893in}{1.466447in}}%
\pgfpathlineto{\pgfqpoint{4.224946in}{1.452341in}}%
\pgfpathlineto{\pgfqpoint{4.275689in}{1.440103in}}%
\pgfpathlineto{\pgfqpoint{4.319534in}{1.430177in}}%
\pgfpathlineto{\pgfqpoint{4.366442in}{1.419862in}}%
\pgfpathlineto{\pgfqpoint{4.352889in}{1.355901in}}%
\pgfpathlineto{\pgfqpoint{4.340815in}{1.299101in}}%
\pgfpathlineto{\pgfqpoint{4.321008in}{1.207454in}}%
\pgfpathlineto{\pgfqpoint{4.306137in}{1.138280in}}%
\pgfpathlineto{\pgfqpoint{4.298103in}{1.099678in}}%
\pgfpathlineto{\pgfqpoint{4.287801in}{1.049576in}}%
\pgfpathlineto{\pgfqpoint{4.280674in}{1.039412in}}%
\pgfpathlineto{\pgfqpoint{4.274938in}{1.039072in}}%
\pgfpathlineto{\pgfqpoint{4.270846in}{1.047945in}}%
\pgfpathlineto{\pgfqpoint{4.261451in}{1.051167in}}%
\pgfpathlineto{\pgfqpoint{4.251328in}{1.050037in}}%
\pgfpathlineto{\pgfqpoint{4.248454in}{1.042780in}}%
\pgfpathlineto{\pgfqpoint{4.248155in}{1.017578in}}%
\pgfpathlineto{\pgfqpoint{4.245123in}{1.011825in}}%
\pgfpathlineto{\pgfqpoint{4.246856in}{0.991592in}}%
\pgfpathlineto{\pgfqpoint{4.240525in}{0.978113in}}%
\pgfpathlineto{\pgfqpoint{4.189188in}{1.057164in}}%
\pgfpathlineto{\pgfqpoint{4.138630in}{1.133980in}}%
\pgfpathlineto{\pgfqpoint{4.112312in}{1.174260in}}%
\pgfpathlineto{\pgfqpoint{4.090358in}{1.209028in}}%
\pgfpathlineto{\pgfqpoint{4.062695in}{1.252036in}}%
\pgfpathlineto{\pgfqpoint{4.031381in}{1.300279in}}%
\pgfpathlineto{\pgfqpoint{4.044255in}{1.346069in}}%
\pgfpathlineto{\pgfqpoint{4.070163in}{1.437904in}}%
\pgfpathlineto{\pgfqpoint{4.084744in}{1.489057in}}%
\pgfpathclose%
\pgfusepath{fill}%
\end{pgfscope}%
\begin{pgfscope}%
\pgfpathrectangle{\pgfqpoint{3.625000in}{0.100000in}}{\pgfqpoint{2.989028in}{1.913466in}}%
\pgfusepath{clip}%
\pgfsetbuttcap%
\pgfsetmiterjoin%
\definecolor{currentfill}{rgb}{0.847520,0.938639,0.607151}%
\pgfsetfillcolor{currentfill}%
\pgfsetlinewidth{0.000000pt}%
\definecolor{currentstroke}{rgb}{0.000000,0.000000,0.000000}%
\pgfsetstrokecolor{currentstroke}%
\pgfsetstrokeopacity{0.000000}%
\pgfsetdash{}{0pt}%
\pgfpathmoveto{\pgfqpoint{4.366442in}{1.419862in}}%
\pgfpathlineto{\pgfqpoint{4.416592in}{1.409886in}}%
\pgfpathlineto{\pgfqpoint{4.509436in}{1.391963in}}%
\pgfpathlineto{\pgfqpoint{4.497815in}{1.327496in}}%
\pgfpathlineto{\pgfqpoint{4.548928in}{1.318744in}}%
\pgfpathlineto{\pgfqpoint{4.595495in}{1.311293in}}%
\pgfpathlineto{\pgfqpoint{4.587404in}{1.260326in}}%
\pgfpathlineto{\pgfqpoint{4.578829in}{1.205365in}}%
\pgfpathlineto{\pgfqpoint{4.567350in}{1.133200in}}%
\pgfpathlineto{\pgfqpoint{4.567053in}{1.127151in}}%
\pgfpathlineto{\pgfqpoint{4.555137in}{1.052362in}}%
\pgfpathlineto{\pgfqpoint{4.506086in}{1.059964in}}%
\pgfpathlineto{\pgfqpoint{4.481091in}{1.064984in}}%
\pgfpathlineto{\pgfqpoint{4.390746in}{1.080865in}}%
\pgfpathlineto{\pgfqpoint{4.356724in}{1.087575in}}%
\pgfpathlineto{\pgfqpoint{4.298103in}{1.099678in}}%
\pgfpathlineto{\pgfqpoint{4.306137in}{1.138280in}}%
\pgfpathlineto{\pgfqpoint{4.321008in}{1.207454in}}%
\pgfpathlineto{\pgfqpoint{4.340815in}{1.299101in}}%
\pgfpathlineto{\pgfqpoint{4.352889in}{1.355901in}}%
\pgfpathlineto{\pgfqpoint{4.366442in}{1.419862in}}%
\pgfpathclose%
\pgfusepath{fill}%
\end{pgfscope}%
\begin{pgfscope}%
\pgfpathrectangle{\pgfqpoint{3.625000in}{0.100000in}}{\pgfqpoint{2.989028in}{1.913466in}}%
\pgfusepath{clip}%
\pgfsetbuttcap%
\pgfsetmiterjoin%
\definecolor{currentfill}{rgb}{0.892887,0.957093,0.597924}%
\pgfsetfillcolor{currentfill}%
\pgfsetlinewidth{0.000000pt}%
\definecolor{currentstroke}{rgb}{0.000000,0.000000,0.000000}%
\pgfsetstrokecolor{currentstroke}%
\pgfsetstrokeopacity{0.000000}%
\pgfsetdash{}{0pt}%
\pgfpathmoveto{\pgfqpoint{3.888713in}{1.549891in}}%
\pgfpathlineto{\pgfqpoint{3.906532in}{1.543676in}}%
\pgfpathlineto{\pgfqpoint{3.977913in}{1.521452in}}%
\pgfpathlineto{\pgfqpoint{4.043578in}{1.500784in}}%
\pgfpathlineto{\pgfqpoint{4.084744in}{1.489057in}}%
\pgfpathlineto{\pgfqpoint{4.070163in}{1.437904in}}%
\pgfpathlineto{\pgfqpoint{4.044255in}{1.346069in}}%
\pgfpathlineto{\pgfqpoint{4.031381in}{1.300279in}}%
\pgfpathlineto{\pgfqpoint{4.062695in}{1.252036in}}%
\pgfpathlineto{\pgfqpoint{4.090358in}{1.209028in}}%
\pgfpathlineto{\pgfqpoint{4.112312in}{1.174260in}}%
\pgfpathlineto{\pgfqpoint{4.138630in}{1.133980in}}%
\pgfpathlineto{\pgfqpoint{4.189188in}{1.057164in}}%
\pgfpathlineto{\pgfqpoint{4.240525in}{0.978113in}}%
\pgfpathlineto{\pgfqpoint{4.238464in}{0.970274in}}%
\pgfpathlineto{\pgfqpoint{4.244656in}{0.957789in}}%
\pgfpathlineto{\pgfqpoint{4.245881in}{0.940711in}}%
\pgfpathlineto{\pgfqpoint{4.254874in}{0.932426in}}%
\pgfpathlineto{\pgfqpoint{4.255230in}{0.925750in}}%
\pgfpathlineto{\pgfqpoint{4.239127in}{0.918174in}}%
\pgfpathlineto{\pgfqpoint{4.231455in}{0.910589in}}%
\pgfpathlineto{\pgfqpoint{4.229063in}{0.893839in}}%
\pgfpathlineto{\pgfqpoint{4.225180in}{0.884704in}}%
\pgfpathlineto{\pgfqpoint{4.216998in}{0.877004in}}%
\pgfpathlineto{\pgfqpoint{4.208861in}{0.856982in}}%
\pgfpathlineto{\pgfqpoint{4.220303in}{0.846552in}}%
\pgfpathlineto{\pgfqpoint{4.218826in}{0.837954in}}%
\pgfpathlineto{\pgfqpoint{4.209444in}{0.831972in}}%
\pgfpathlineto{\pgfqpoint{4.202975in}{0.833027in}}%
\pgfpathlineto{\pgfqpoint{4.126622in}{0.843623in}}%
\pgfpathlineto{\pgfqpoint{4.070229in}{0.851623in}}%
\pgfpathlineto{\pgfqpoint{4.072695in}{0.860728in}}%
\pgfpathlineto{\pgfqpoint{4.069024in}{0.875844in}}%
\pgfpathlineto{\pgfqpoint{4.068653in}{0.891092in}}%
\pgfpathlineto{\pgfqpoint{4.066254in}{0.900014in}}%
\pgfpathlineto{\pgfqpoint{4.058864in}{0.912772in}}%
\pgfpathlineto{\pgfqpoint{4.037605in}{0.942205in}}%
\pgfpathlineto{\pgfqpoint{4.030643in}{0.945811in}}%
\pgfpathlineto{\pgfqpoint{4.021672in}{0.945755in}}%
\pgfpathlineto{\pgfqpoint{4.023721in}{0.955040in}}%
\pgfpathlineto{\pgfqpoint{4.019473in}{0.966651in}}%
\pgfpathlineto{\pgfqpoint{3.998608in}{0.972422in}}%
\pgfpathlineto{\pgfqpoint{3.985889in}{0.983108in}}%
\pgfpathlineto{\pgfqpoint{3.984834in}{0.989647in}}%
\pgfpathlineto{\pgfqpoint{3.970189in}{1.005835in}}%
\pgfpathlineto{\pgfqpoint{3.956249in}{1.008932in}}%
\pgfpathlineto{\pgfqpoint{3.943332in}{1.017159in}}%
\pgfpathlineto{\pgfqpoint{3.926344in}{1.020005in}}%
\pgfpathlineto{\pgfqpoint{3.919086in}{1.030977in}}%
\pgfpathlineto{\pgfqpoint{3.925979in}{1.048345in}}%
\pgfpathlineto{\pgfqpoint{3.923871in}{1.052250in}}%
\pgfpathlineto{\pgfqpoint{3.929611in}{1.066728in}}%
\pgfpathlineto{\pgfqpoint{3.919400in}{1.074432in}}%
\pgfpathlineto{\pgfqpoint{3.922713in}{1.088401in}}%
\pgfpathlineto{\pgfqpoint{3.917259in}{1.091971in}}%
\pgfpathlineto{\pgfqpoint{3.912550in}{1.105174in}}%
\pgfpathlineto{\pgfqpoint{3.906867in}{1.109200in}}%
\pgfpathlineto{\pgfqpoint{3.906440in}{1.118736in}}%
\pgfpathlineto{\pgfqpoint{3.901988in}{1.125461in}}%
\pgfpathlineto{\pgfqpoint{3.895279in}{1.148109in}}%
\pgfpathlineto{\pgfqpoint{3.887938in}{1.158956in}}%
\pgfpathlineto{\pgfqpoint{3.889533in}{1.177370in}}%
\pgfpathlineto{\pgfqpoint{3.898166in}{1.179223in}}%
\pgfpathlineto{\pgfqpoint{3.903787in}{1.189211in}}%
\pgfpathlineto{\pgfqpoint{3.900390in}{1.200043in}}%
\pgfpathlineto{\pgfqpoint{3.891209in}{1.201851in}}%
\pgfpathlineto{\pgfqpoint{3.886629in}{1.206918in}}%
\pgfpathlineto{\pgfqpoint{3.879211in}{1.225555in}}%
\pgfpathlineto{\pgfqpoint{3.882678in}{1.232248in}}%
\pgfpathlineto{\pgfqpoint{3.880219in}{1.244711in}}%
\pgfpathlineto{\pgfqpoint{3.885664in}{1.260852in}}%
\pgfpathlineto{\pgfqpoint{3.892043in}{1.254907in}}%
\pgfpathlineto{\pgfqpoint{3.889148in}{1.247890in}}%
\pgfpathlineto{\pgfqpoint{3.900202in}{1.236766in}}%
\pgfpathlineto{\pgfqpoint{3.899497in}{1.253285in}}%
\pgfpathlineto{\pgfqpoint{3.894760in}{1.257715in}}%
\pgfpathlineto{\pgfqpoint{3.900174in}{1.272240in}}%
\pgfpathlineto{\pgfqpoint{3.920652in}{1.269927in}}%
\pgfpathlineto{\pgfqpoint{3.917896in}{1.275328in}}%
\pgfpathlineto{\pgfqpoint{3.904381in}{1.274797in}}%
\pgfpathlineto{\pgfqpoint{3.897951in}{1.282977in}}%
\pgfpathlineto{\pgfqpoint{3.889858in}{1.275713in}}%
\pgfpathlineto{\pgfqpoint{3.885564in}{1.263570in}}%
\pgfpathlineto{\pgfqpoint{3.874102in}{1.279909in}}%
\pgfpathlineto{\pgfqpoint{3.869687in}{1.282884in}}%
\pgfpathlineto{\pgfqpoint{3.871369in}{1.300667in}}%
\pgfpathlineto{\pgfqpoint{3.867875in}{1.311153in}}%
\pgfpathlineto{\pgfqpoint{3.861542in}{1.321006in}}%
\pgfpathlineto{\pgfqpoint{3.848573in}{1.351193in}}%
\pgfpathlineto{\pgfqpoint{3.852813in}{1.357869in}}%
\pgfpathlineto{\pgfqpoint{3.852764in}{1.378965in}}%
\pgfpathlineto{\pgfqpoint{3.859763in}{1.390732in}}%
\pgfpathlineto{\pgfqpoint{3.861368in}{1.409121in}}%
\pgfpathlineto{\pgfqpoint{3.854768in}{1.430191in}}%
\pgfpathlineto{\pgfqpoint{3.846012in}{1.443602in}}%
\pgfpathlineto{\pgfqpoint{3.847574in}{1.455704in}}%
\pgfpathlineto{\pgfqpoint{3.872163in}{1.484994in}}%
\pgfpathlineto{\pgfqpoint{3.873384in}{1.494985in}}%
\pgfpathlineto{\pgfqpoint{3.884464in}{1.514044in}}%
\pgfpathlineto{\pgfqpoint{3.886001in}{1.532109in}}%
\pgfpathlineto{\pgfqpoint{3.882439in}{1.536719in}}%
\pgfpathlineto{\pgfqpoint{3.888713in}{1.549891in}}%
\pgfpathclose%
\pgfusepath{fill}%
\end{pgfscope}%
\begin{pgfscope}%
\pgfpathrectangle{\pgfqpoint{3.625000in}{0.100000in}}{\pgfqpoint{2.989028in}{1.913466in}}%
\pgfusepath{clip}%
\pgfsetbuttcap%
\pgfsetmiterjoin%
\definecolor{currentfill}{rgb}{0.955786,0.982314,0.680046}%
\pgfsetfillcolor{currentfill}%
\pgfsetlinewidth{0.000000pt}%
\definecolor{currentstroke}{rgb}{0.000000,0.000000,0.000000}%
\pgfsetstrokecolor{currentstroke}%
\pgfsetstrokeopacity{0.000000}%
\pgfsetdash{}{0pt}%
\pgfpathmoveto{\pgfqpoint{5.803860in}{1.162554in}}%
\pgfpathlineto{\pgfqpoint{5.795539in}{1.244011in}}%
\pgfpathlineto{\pgfqpoint{5.785872in}{1.331556in}}%
\pgfpathlineto{\pgfqpoint{5.849180in}{1.340984in}}%
\pgfpathlineto{\pgfqpoint{5.865982in}{1.336611in}}%
\pgfpathlineto{\pgfqpoint{5.874041in}{1.331864in}}%
\pgfpathlineto{\pgfqpoint{5.884139in}{1.333180in}}%
\pgfpathlineto{\pgfqpoint{5.897453in}{1.325321in}}%
\pgfpathlineto{\pgfqpoint{5.922252in}{1.337018in}}%
\pgfpathlineto{\pgfqpoint{5.935938in}{1.337409in}}%
\pgfpathlineto{\pgfqpoint{5.951920in}{1.355239in}}%
\pgfpathlineto{\pgfqpoint{5.968180in}{1.366084in}}%
\pgfpathlineto{\pgfqpoint{5.989869in}{1.378577in}}%
\pgfpathlineto{\pgfqpoint{6.003831in}{1.291252in}}%
\pgfpathlineto{\pgfqpoint{5.997344in}{1.285643in}}%
\pgfpathlineto{\pgfqpoint{6.001532in}{1.280492in}}%
\pgfpathlineto{\pgfqpoint{6.003201in}{1.269199in}}%
\pgfpathlineto{\pgfqpoint{5.999951in}{1.258593in}}%
\pgfpathlineto{\pgfqpoint{5.999521in}{1.243089in}}%
\pgfpathlineto{\pgfqpoint{5.996469in}{1.222911in}}%
\pgfpathlineto{\pgfqpoint{5.981028in}{1.204915in}}%
\pgfpathlineto{\pgfqpoint{5.974528in}{1.201094in}}%
\pgfpathlineto{\pgfqpoint{5.969473in}{1.204383in}}%
\pgfpathlineto{\pgfqpoint{5.956963in}{1.187230in}}%
\pgfpathlineto{\pgfqpoint{5.958135in}{1.170714in}}%
\pgfpathlineto{\pgfqpoint{5.944203in}{1.174686in}}%
\pgfpathlineto{\pgfqpoint{5.937606in}{1.158191in}}%
\pgfpathlineto{\pgfqpoint{5.940941in}{1.146502in}}%
\pgfpathlineto{\pgfqpoint{5.935720in}{1.144776in}}%
\pgfpathlineto{\pgfqpoint{5.934990in}{1.135540in}}%
\pgfpathlineto{\pgfqpoint{5.922176in}{1.131815in}}%
\pgfpathlineto{\pgfqpoint{5.915518in}{1.139272in}}%
\pgfpathlineto{\pgfqpoint{5.906950in}{1.142163in}}%
\pgfpathlineto{\pgfqpoint{5.904937in}{1.149724in}}%
\pgfpathlineto{\pgfqpoint{5.897191in}{1.148394in}}%
\pgfpathlineto{\pgfqpoint{5.892116in}{1.141440in}}%
\pgfpathlineto{\pgfqpoint{5.884855in}{1.138994in}}%
\pgfpathlineto{\pgfqpoint{5.872031in}{1.143916in}}%
\pgfpathlineto{\pgfqpoint{5.864938in}{1.138060in}}%
\pgfpathlineto{\pgfqpoint{5.854857in}{1.144988in}}%
\pgfpathlineto{\pgfqpoint{5.835517in}{1.147114in}}%
\pgfpathlineto{\pgfqpoint{5.829684in}{1.159700in}}%
\pgfpathlineto{\pgfqpoint{5.822293in}{1.165276in}}%
\pgfpathlineto{\pgfqpoint{5.815128in}{1.161707in}}%
\pgfpathlineto{\pgfqpoint{5.803860in}{1.162554in}}%
\pgfpathclose%
\pgfusepath{fill}%
\end{pgfscope}%
\begin{pgfscope}%
\pgfpathrectangle{\pgfqpoint{3.625000in}{0.100000in}}{\pgfqpoint{2.989028in}{1.913466in}}%
\pgfusepath{clip}%
\pgfsetbuttcap%
\pgfsetmiterjoin%
\definecolor{currentfill}{rgb}{0.944252,0.977701,0.662053}%
\pgfsetfillcolor{currentfill}%
\pgfsetlinewidth{0.000000pt}%
\definecolor{currentstroke}{rgb}{0.000000,0.000000,0.000000}%
\pgfsetstrokecolor{currentstroke}%
\pgfsetstrokeopacity{0.000000}%
\pgfsetdash{}{0pt}%
\pgfpathmoveto{\pgfqpoint{5.594266in}{1.005020in}}%
\pgfpathlineto{\pgfqpoint{5.586849in}{1.011045in}}%
\pgfpathlineto{\pgfqpoint{5.580801in}{1.008178in}}%
\pgfpathlineto{\pgfqpoint{5.573069in}{1.022616in}}%
\pgfpathlineto{\pgfqpoint{5.577019in}{1.031690in}}%
\pgfpathlineto{\pgfqpoint{5.571326in}{1.041904in}}%
\pgfpathlineto{\pgfqpoint{5.571020in}{1.049946in}}%
\pgfpathlineto{\pgfqpoint{5.563328in}{1.052811in}}%
\pgfpathlineto{\pgfqpoint{5.559760in}{1.058870in}}%
\pgfpathlineto{\pgfqpoint{5.544719in}{1.066437in}}%
\pgfpathlineto{\pgfqpoint{5.531657in}{1.075760in}}%
\pgfpathlineto{\pgfqpoint{5.525587in}{1.082799in}}%
\pgfpathlineto{\pgfqpoint{5.525462in}{1.091383in}}%
\pgfpathlineto{\pgfqpoint{5.533584in}{1.107865in}}%
\pgfpathlineto{\pgfqpoint{5.536082in}{1.120472in}}%
\pgfpathlineto{\pgfqpoint{5.529465in}{1.127595in}}%
\pgfpathlineto{\pgfqpoint{5.520702in}{1.130279in}}%
\pgfpathlineto{\pgfqpoint{5.510070in}{1.124410in}}%
\pgfpathlineto{\pgfqpoint{5.505570in}{1.134486in}}%
\pgfpathlineto{\pgfqpoint{5.503299in}{1.148144in}}%
\pgfpathlineto{\pgfqpoint{5.487630in}{1.160319in}}%
\pgfpathlineto{\pgfqpoint{5.484499in}{1.165722in}}%
\pgfpathlineto{\pgfqpoint{5.470188in}{1.177954in}}%
\pgfpathlineto{\pgfqpoint{5.465703in}{1.186846in}}%
\pgfpathlineto{\pgfqpoint{5.461659in}{1.204480in}}%
\pgfpathlineto{\pgfqpoint{5.464431in}{1.220143in}}%
\pgfpathlineto{\pgfqpoint{5.468130in}{1.222326in}}%
\pgfpathlineto{\pgfqpoint{5.467465in}{1.235438in}}%
\pgfpathlineto{\pgfqpoint{5.477907in}{1.239332in}}%
\pgfpathlineto{\pgfqpoint{5.481060in}{1.251098in}}%
\pgfpathlineto{\pgfqpoint{5.487068in}{1.259018in}}%
\pgfpathlineto{\pgfqpoint{5.486764in}{1.269093in}}%
\pgfpathlineto{\pgfqpoint{5.479318in}{1.277106in}}%
\pgfpathlineto{\pgfqpoint{5.481087in}{1.288322in}}%
\pgfpathlineto{\pgfqpoint{5.500389in}{1.293200in}}%
\pgfpathlineto{\pgfqpoint{5.515197in}{1.302104in}}%
\pgfpathlineto{\pgfqpoint{5.516751in}{1.313315in}}%
\pgfpathlineto{\pgfqpoint{5.521904in}{1.316845in}}%
\pgfpathlineto{\pgfqpoint{5.523875in}{1.328623in}}%
\pgfpathlineto{\pgfqpoint{5.522291in}{1.336402in}}%
\pgfpathlineto{\pgfqpoint{5.512167in}{1.342864in}}%
\pgfpathlineto{\pgfqpoint{5.508091in}{1.352497in}}%
\pgfpathlineto{\pgfqpoint{5.498100in}{1.361809in}}%
\pgfpathlineto{\pgfqpoint{5.580206in}{1.365308in}}%
\pgfpathlineto{\pgfqpoint{5.635291in}{1.369221in}}%
\pgfpathlineto{\pgfqpoint{5.634283in}{1.357634in}}%
\pgfpathlineto{\pgfqpoint{5.643671in}{1.341651in}}%
\pgfpathlineto{\pgfqpoint{5.647612in}{1.327996in}}%
\pgfpathlineto{\pgfqpoint{5.652306in}{1.320243in}}%
\pgfpathlineto{\pgfqpoint{5.657431in}{1.256358in}}%
\pgfpathlineto{\pgfqpoint{5.664547in}{1.165295in}}%
\pgfpathlineto{\pgfqpoint{5.659057in}{1.151457in}}%
\pgfpathlineto{\pgfqpoint{5.666966in}{1.140083in}}%
\pgfpathlineto{\pgfqpoint{5.669170in}{1.129219in}}%
\pgfpathlineto{\pgfqpoint{5.658263in}{1.106692in}}%
\pgfpathlineto{\pgfqpoint{5.658907in}{1.103580in}}%
\pgfpathlineto{\pgfqpoint{5.647681in}{1.090668in}}%
\pgfpathlineto{\pgfqpoint{5.650759in}{1.086724in}}%
\pgfpathlineto{\pgfqpoint{5.646088in}{1.078676in}}%
\pgfpathlineto{\pgfqpoint{5.647290in}{1.062454in}}%
\pgfpathlineto{\pgfqpoint{5.641617in}{1.052499in}}%
\pgfpathlineto{\pgfqpoint{5.646231in}{1.040736in}}%
\pgfpathlineto{\pgfqpoint{5.626900in}{1.034344in}}%
\pgfpathlineto{\pgfqpoint{5.625119in}{1.027393in}}%
\pgfpathlineto{\pgfqpoint{5.630397in}{1.018582in}}%
\pgfpathlineto{\pgfqpoint{5.627969in}{1.012840in}}%
\pgfpathlineto{\pgfqpoint{5.607240in}{1.019933in}}%
\pgfpathlineto{\pgfqpoint{5.600401in}{1.020647in}}%
\pgfpathlineto{\pgfqpoint{5.591874in}{1.009863in}}%
\pgfpathlineto{\pgfqpoint{5.594266in}{1.005020in}}%
\pgfpathclose%
\pgfusepath{fill}%
\end{pgfscope}%
\begin{pgfscope}%
\pgfpathrectangle{\pgfqpoint{3.625000in}{0.100000in}}{\pgfqpoint{2.989028in}{1.913466in}}%
\pgfusepath{clip}%
\pgfsetbuttcap%
\pgfsetmiterjoin%
\definecolor{currentfill}{rgb}{0.951942,0.980777,0.674048}%
\pgfsetfillcolor{currentfill}%
\pgfsetlinewidth{0.000000pt}%
\definecolor{currentstroke}{rgb}{0.000000,0.000000,0.000000}%
\pgfsetstrokecolor{currentstroke}%
\pgfsetstrokeopacity{0.000000}%
\pgfsetdash{}{0pt}%
\pgfpathmoveto{\pgfqpoint{6.197645in}{1.203210in}}%
\pgfpathlineto{\pgfqpoint{6.191941in}{1.211681in}}%
\pgfpathlineto{\pgfqpoint{6.195160in}{1.216421in}}%
\pgfpathlineto{\pgfqpoint{6.203043in}{1.211098in}}%
\pgfpathlineto{\pgfqpoint{6.197645in}{1.203210in}}%
\pgfpathclose%
\pgfusepath{fill}%
\end{pgfscope}%
\begin{pgfscope}%
\pgfpathrectangle{\pgfqpoint{3.625000in}{0.100000in}}{\pgfqpoint{2.989028in}{1.913466in}}%
\pgfusepath{clip}%
\pgfsetbuttcap%
\pgfsetmiterjoin%
\definecolor{currentfill}{rgb}{0.995617,0.855363,0.525721}%
\pgfsetfillcolor{currentfill}%
\pgfsetlinewidth{0.000000pt}%
\definecolor{currentstroke}{rgb}{0.000000,0.000000,0.000000}%
\pgfsetstrokecolor{currentstroke}%
\pgfsetstrokeopacity{0.000000}%
\pgfsetdash{}{0pt}%
\pgfpathmoveto{\pgfqpoint{6.247458in}{1.276100in}}%
\pgfpathlineto{\pgfqpoint{6.250867in}{1.283302in}}%
\pgfpathlineto{\pgfqpoint{6.264648in}{1.284858in}}%
\pgfpathlineto{\pgfqpoint{6.262440in}{1.278737in}}%
\pgfpathlineto{\pgfqpoint{6.257893in}{1.270918in}}%
\pgfpathlineto{\pgfqpoint{6.260974in}{1.261602in}}%
\pgfpathlineto{\pgfqpoint{6.273133in}{1.250439in}}%
\pgfpathlineto{\pgfqpoint{6.275955in}{1.238685in}}%
\pgfpathlineto{\pgfqpoint{6.289963in}{1.224047in}}%
\pgfpathlineto{\pgfqpoint{6.295458in}{1.224674in}}%
\pgfpathlineto{\pgfqpoint{6.302304in}{1.202700in}}%
\pgfpathlineto{\pgfqpoint{6.301181in}{1.202481in}}%
\pgfpathlineto{\pgfqpoint{6.299933in}{1.202236in}}%
\pgfpathlineto{\pgfqpoint{6.269401in}{1.196391in}}%
\pgfpathlineto{\pgfqpoint{6.253066in}{1.254485in}}%
\pgfpathlineto{\pgfqpoint{6.247458in}{1.276100in}}%
\pgfpathclose%
\pgfusepath{fill}%
\end{pgfscope}%
\begin{pgfscope}%
\pgfpathrectangle{\pgfqpoint{3.625000in}{0.100000in}}{\pgfqpoint{2.989028in}{1.913466in}}%
\pgfusepath{clip}%
\pgfsetbuttcap%
\pgfsetmiterjoin%
\definecolor{currentfill}{rgb}{0.280046,0.626990,0.702422}%
\pgfsetfillcolor{currentfill}%
\pgfsetlinewidth{0.000000pt}%
\definecolor{currentstroke}{rgb}{0.000000,0.000000,0.000000}%
\pgfsetstrokecolor{currentstroke}%
\pgfsetstrokeopacity{0.000000}%
\pgfsetdash{}{0pt}%
\pgfpathmoveto{\pgfqpoint{5.962284in}{1.079184in}}%
\pgfpathlineto{\pgfqpoint{5.952636in}{1.079539in}}%
\pgfpathlineto{\pgfqpoint{5.943746in}{1.085662in}}%
\pgfpathlineto{\pgfqpoint{5.931736in}{1.104274in}}%
\pgfpathlineto{\pgfqpoint{5.921513in}{1.114131in}}%
\pgfpathlineto{\pgfqpoint{5.924185in}{1.121744in}}%
\pgfpathlineto{\pgfqpoint{5.922176in}{1.131815in}}%
\pgfpathlineto{\pgfqpoint{5.934990in}{1.135540in}}%
\pgfpathlineto{\pgfqpoint{5.935720in}{1.144776in}}%
\pgfpathlineto{\pgfqpoint{5.940941in}{1.146502in}}%
\pgfpathlineto{\pgfqpoint{5.937606in}{1.158191in}}%
\pgfpathlineto{\pgfqpoint{5.944203in}{1.174686in}}%
\pgfpathlineto{\pgfqpoint{5.958135in}{1.170714in}}%
\pgfpathlineto{\pgfqpoint{5.956963in}{1.187230in}}%
\pgfpathlineto{\pgfqpoint{5.969473in}{1.204383in}}%
\pgfpathlineto{\pgfqpoint{5.974528in}{1.201094in}}%
\pgfpathlineto{\pgfqpoint{5.981028in}{1.204915in}}%
\pgfpathlineto{\pgfqpoint{5.996469in}{1.222911in}}%
\pgfpathlineto{\pgfqpoint{5.999521in}{1.243089in}}%
\pgfpathlineto{\pgfqpoint{5.999951in}{1.258593in}}%
\pgfpathlineto{\pgfqpoint{6.003201in}{1.269199in}}%
\pgfpathlineto{\pgfqpoint{6.001532in}{1.280492in}}%
\pgfpathlineto{\pgfqpoint{5.997344in}{1.285643in}}%
\pgfpathlineto{\pgfqpoint{6.003831in}{1.291252in}}%
\pgfpathlineto{\pgfqpoint{6.013234in}{1.232010in}}%
\pgfpathlineto{\pgfqpoint{6.065072in}{1.240581in}}%
\pgfpathlineto{\pgfqpoint{6.070445in}{1.206768in}}%
\pgfpathlineto{\pgfqpoint{6.089213in}{1.229086in}}%
\pgfpathlineto{\pgfqpoint{6.093635in}{1.226861in}}%
\pgfpathlineto{\pgfqpoint{6.100140in}{1.239793in}}%
\pgfpathlineto{\pgfqpoint{6.109152in}{1.235738in}}%
\pgfpathlineto{\pgfqpoint{6.117010in}{1.236509in}}%
\pgfpathlineto{\pgfqpoint{6.122194in}{1.245499in}}%
\pgfpathlineto{\pgfqpoint{6.129719in}{1.250500in}}%
\pgfpathlineto{\pgfqpoint{6.140180in}{1.246096in}}%
\pgfpathlineto{\pgfqpoint{6.147064in}{1.247618in}}%
\pgfpathlineto{\pgfqpoint{6.156926in}{1.230550in}}%
\pgfpathlineto{\pgfqpoint{6.154074in}{1.217629in}}%
\pgfpathlineto{\pgfqpoint{6.128317in}{1.232177in}}%
\pgfpathlineto{\pgfqpoint{6.123492in}{1.220234in}}%
\pgfpathlineto{\pgfqpoint{6.125084in}{1.214739in}}%
\pgfpathlineto{\pgfqpoint{6.114121in}{1.196164in}}%
\pgfpathlineto{\pgfqpoint{6.108947in}{1.192470in}}%
\pgfpathlineto{\pgfqpoint{6.106611in}{1.184279in}}%
\pgfpathlineto{\pgfqpoint{6.097791in}{1.185136in}}%
\pgfpathlineto{\pgfqpoint{6.091456in}{1.162779in}}%
\pgfpathlineto{\pgfqpoint{6.087911in}{1.157648in}}%
\pgfpathlineto{\pgfqpoint{6.078799in}{1.159359in}}%
\pgfpathlineto{\pgfqpoint{6.069480in}{1.166411in}}%
\pgfpathlineto{\pgfqpoint{6.069157in}{1.155588in}}%
\pgfpathlineto{\pgfqpoint{6.060135in}{1.137399in}}%
\pgfpathlineto{\pgfqpoint{6.057891in}{1.124451in}}%
\pgfpathlineto{\pgfqpoint{6.050969in}{1.115827in}}%
\pgfpathlineto{\pgfqpoint{6.045721in}{1.102055in}}%
\pgfpathlineto{\pgfqpoint{6.045463in}{1.089306in}}%
\pgfpathlineto{\pgfqpoint{6.037358in}{1.086925in}}%
\pgfpathlineto{\pgfqpoint{6.028164in}{1.079708in}}%
\pgfpathlineto{\pgfqpoint{6.019311in}{1.078334in}}%
\pgfpathlineto{\pgfqpoint{6.017271in}{1.072225in}}%
\pgfpathlineto{\pgfqpoint{6.003007in}{1.065970in}}%
\pgfpathlineto{\pgfqpoint{5.995035in}{1.071300in}}%
\pgfpathlineto{\pgfqpoint{5.986126in}{1.061179in}}%
\pgfpathlineto{\pgfqpoint{5.970810in}{1.064162in}}%
\pgfpathlineto{\pgfqpoint{5.961416in}{1.074785in}}%
\pgfpathlineto{\pgfqpoint{5.962284in}{1.079184in}}%
\pgfpathclose%
\pgfusepath{fill}%
\end{pgfscope}%
\begin{pgfscope}%
\pgfpathrectangle{\pgfqpoint{3.625000in}{0.100000in}}{\pgfqpoint{2.989028in}{1.913466in}}%
\pgfusepath{clip}%
\pgfsetbuttcap%
\pgfsetmiterjoin%
\definecolor{currentfill}{rgb}{0.838447,0.934948,0.608997}%
\pgfsetfillcolor{currentfill}%
\pgfsetlinewidth{0.000000pt}%
\definecolor{currentstroke}{rgb}{0.000000,0.000000,0.000000}%
\pgfsetstrokecolor{currentstroke}%
\pgfsetstrokeopacity{0.000000}%
\pgfsetdash{}{0pt}%
\pgfpathmoveto{\pgfqpoint{6.299933in}{1.202236in}}%
\pgfpathlineto{\pgfqpoint{6.299542in}{1.190298in}}%
\pgfpathlineto{\pgfqpoint{6.294947in}{1.184429in}}%
\pgfpathlineto{\pgfqpoint{6.292015in}{1.171401in}}%
\pgfpathlineto{\pgfqpoint{6.278775in}{1.165395in}}%
\pgfpathlineto{\pgfqpoint{6.267667in}{1.163645in}}%
\pgfpathlineto{\pgfqpoint{6.270904in}{1.172214in}}%
\pgfpathlineto{\pgfqpoint{6.253800in}{1.179393in}}%
\pgfpathlineto{\pgfqpoint{6.239904in}{1.188383in}}%
\pgfpathlineto{\pgfqpoint{6.248507in}{1.201821in}}%
\pgfpathlineto{\pgfqpoint{6.241613in}{1.212266in}}%
\pgfpathlineto{\pgfqpoint{6.242424in}{1.221276in}}%
\pgfpathlineto{\pgfqpoint{6.237439in}{1.223749in}}%
\pgfpathlineto{\pgfqpoint{6.233394in}{1.238386in}}%
\pgfpathlineto{\pgfqpoint{6.242960in}{1.258229in}}%
\pgfpathlineto{\pgfqpoint{6.235773in}{1.261469in}}%
\pgfpathlineto{\pgfqpoint{6.233910in}{1.251680in}}%
\pgfpathlineto{\pgfqpoint{6.223651in}{1.248954in}}%
\pgfpathlineto{\pgfqpoint{6.224730in}{1.216746in}}%
\pgfpathlineto{\pgfqpoint{6.222854in}{1.206381in}}%
\pgfpathlineto{\pgfqpoint{6.228000in}{1.191607in}}%
\pgfpathlineto{\pgfqpoint{6.235916in}{1.184499in}}%
\pgfpathlineto{\pgfqpoint{6.232326in}{1.180036in}}%
\pgfpathlineto{\pgfqpoint{6.240407in}{1.173514in}}%
\pgfpathlineto{\pgfqpoint{6.243328in}{1.162929in}}%
\pgfpathlineto{\pgfqpoint{6.228517in}{1.171694in}}%
\pgfpathlineto{\pgfqpoint{6.219143in}{1.170556in}}%
\pgfpathlineto{\pgfqpoint{6.211865in}{1.179573in}}%
\pgfpathlineto{\pgfqpoint{6.204454in}{1.180451in}}%
\pgfpathlineto{\pgfqpoint{6.193921in}{1.175935in}}%
\pgfpathlineto{\pgfqpoint{6.189837in}{1.181563in}}%
\pgfpathlineto{\pgfqpoint{6.195205in}{1.193376in}}%
\pgfpathlineto{\pgfqpoint{6.197645in}{1.203210in}}%
\pgfpathlineto{\pgfqpoint{6.203043in}{1.211098in}}%
\pgfpathlineto{\pgfqpoint{6.195160in}{1.216421in}}%
\pgfpathlineto{\pgfqpoint{6.191941in}{1.211681in}}%
\pgfpathlineto{\pgfqpoint{6.184066in}{1.216492in}}%
\pgfpathlineto{\pgfqpoint{6.170120in}{1.219698in}}%
\pgfpathlineto{\pgfqpoint{6.171385in}{1.226747in}}%
\pgfpathlineto{\pgfqpoint{6.165064in}{1.230836in}}%
\pgfpathlineto{\pgfqpoint{6.156926in}{1.230550in}}%
\pgfpathlineto{\pgfqpoint{6.147064in}{1.247618in}}%
\pgfpathlineto{\pgfqpoint{6.140180in}{1.246096in}}%
\pgfpathlineto{\pgfqpoint{6.129719in}{1.250500in}}%
\pgfpathlineto{\pgfqpoint{6.122194in}{1.245499in}}%
\pgfpathlineto{\pgfqpoint{6.117010in}{1.236509in}}%
\pgfpathlineto{\pgfqpoint{6.109152in}{1.235738in}}%
\pgfpathlineto{\pgfqpoint{6.100140in}{1.239793in}}%
\pgfpathlineto{\pgfqpoint{6.093635in}{1.226861in}}%
\pgfpathlineto{\pgfqpoint{6.089213in}{1.229086in}}%
\pgfpathlineto{\pgfqpoint{6.070445in}{1.206768in}}%
\pgfpathlineto{\pgfqpoint{6.065072in}{1.240581in}}%
\pgfpathlineto{\pgfqpoint{6.133655in}{1.253292in}}%
\pgfpathlineto{\pgfqpoint{6.164399in}{1.258797in}}%
\pgfpathlineto{\pgfqpoint{6.209127in}{1.267806in}}%
\pgfpathlineto{\pgfqpoint{6.247458in}{1.276100in}}%
\pgfpathlineto{\pgfqpoint{6.253066in}{1.254485in}}%
\pgfpathlineto{\pgfqpoint{6.269401in}{1.196391in}}%
\pgfpathlineto{\pgfqpoint{6.299933in}{1.202236in}}%
\pgfpathclose%
\pgfusepath{fill}%
\end{pgfscope}%
\begin{pgfscope}%
\pgfpathrectangle{\pgfqpoint{3.625000in}{0.100000in}}{\pgfqpoint{2.989028in}{1.913466in}}%
\pgfusepath{clip}%
\pgfsetbuttcap%
\pgfsetmiterjoin%
\definecolor{currentfill}{rgb}{0.684198,0.872203,0.640369}%
\pgfsetfillcolor{currentfill}%
\pgfsetlinewidth{0.000000pt}%
\definecolor{currentstroke}{rgb}{0.000000,0.000000,0.000000}%
\pgfsetstrokecolor{currentstroke}%
\pgfsetstrokeopacity{0.000000}%
\pgfsetdash{}{0pt}%
\pgfpathmoveto{\pgfqpoint{4.919404in}{1.009408in}}%
\pgfpathlineto{\pgfqpoint{4.869368in}{1.014183in}}%
\pgfpathlineto{\pgfqpoint{4.817460in}{1.018783in}}%
\pgfpathlineto{\pgfqpoint{4.757477in}{1.025040in}}%
\pgfpathlineto{\pgfqpoint{4.668359in}{1.035639in}}%
\pgfpathlineto{\pgfqpoint{4.636744in}{1.040510in}}%
\pgfpathlineto{\pgfqpoint{4.555137in}{1.052362in}}%
\pgfpathlineto{\pgfqpoint{4.567053in}{1.127151in}}%
\pgfpathlineto{\pgfqpoint{4.567350in}{1.133200in}}%
\pgfpathlineto{\pgfqpoint{4.578829in}{1.205365in}}%
\pgfpathlineto{\pgfqpoint{4.587404in}{1.260326in}}%
\pgfpathlineto{\pgfqpoint{4.595495in}{1.311293in}}%
\pgfpathlineto{\pgfqpoint{4.650778in}{1.303357in}}%
\pgfpathlineto{\pgfqpoint{4.702321in}{1.295959in}}%
\pgfpathlineto{\pgfqpoint{4.797072in}{1.284265in}}%
\pgfpathlineto{\pgfqpoint{4.840540in}{1.280295in}}%
\pgfpathlineto{\pgfqpoint{4.909404in}{1.273614in}}%
\pgfpathlineto{\pgfqpoint{4.939195in}{1.271196in}}%
\pgfpathlineto{\pgfqpoint{4.933935in}{1.205970in}}%
\pgfpathlineto{\pgfqpoint{4.929183in}{1.143169in}}%
\pgfpathlineto{\pgfqpoint{4.925359in}{1.092033in}}%
\pgfpathlineto{\pgfqpoint{4.919404in}{1.009408in}}%
\pgfpathclose%
\pgfusepath{fill}%
\end{pgfscope}%
\begin{pgfscope}%
\pgfpathrectangle{\pgfqpoint{3.625000in}{0.100000in}}{\pgfqpoint{2.989028in}{1.913466in}}%
\pgfusepath{clip}%
\pgfsetbuttcap%
\pgfsetmiterjoin%
\definecolor{currentfill}{rgb}{0.665283,0.864591,0.643214}%
\pgfsetfillcolor{currentfill}%
\pgfsetlinewidth{0.000000pt}%
\definecolor{currentstroke}{rgb}{0.000000,0.000000,0.000000}%
\pgfsetstrokecolor{currentstroke}%
\pgfsetstrokeopacity{0.000000}%
\pgfsetdash{}{0pt}%
\pgfpathmoveto{\pgfqpoint{5.581343in}{0.972329in}}%
\pgfpathlineto{\pgfqpoint{5.583030in}{0.979890in}}%
\pgfpathlineto{\pgfqpoint{5.591773in}{0.978188in}}%
\pgfpathlineto{\pgfqpoint{5.594266in}{1.005020in}}%
\pgfpathlineto{\pgfqpoint{5.591874in}{1.009863in}}%
\pgfpathlineto{\pgfqpoint{5.600401in}{1.020647in}}%
\pgfpathlineto{\pgfqpoint{5.607240in}{1.019933in}}%
\pgfpathlineto{\pgfqpoint{5.627969in}{1.012840in}}%
\pgfpathlineto{\pgfqpoint{5.630397in}{1.018582in}}%
\pgfpathlineto{\pgfqpoint{5.625119in}{1.027393in}}%
\pgfpathlineto{\pgfqpoint{5.626900in}{1.034344in}}%
\pgfpathlineto{\pgfqpoint{5.646231in}{1.040736in}}%
\pgfpathlineto{\pgfqpoint{5.641617in}{1.052499in}}%
\pgfpathlineto{\pgfqpoint{5.647290in}{1.062454in}}%
\pgfpathlineto{\pgfqpoint{5.657850in}{1.068041in}}%
\pgfpathlineto{\pgfqpoint{5.680006in}{1.073574in}}%
\pgfpathlineto{\pgfqpoint{5.693985in}{1.065222in}}%
\pgfpathlineto{\pgfqpoint{5.698226in}{1.073367in}}%
\pgfpathlineto{\pgfqpoint{5.709834in}{1.079056in}}%
\pgfpathlineto{\pgfqpoint{5.712829in}{1.074039in}}%
\pgfpathlineto{\pgfqpoint{5.724753in}{1.078025in}}%
\pgfpathlineto{\pgfqpoint{5.724000in}{1.084848in}}%
\pgfpathlineto{\pgfqpoint{5.734721in}{1.092656in}}%
\pgfpathlineto{\pgfqpoint{5.741048in}{1.084548in}}%
\pgfpathlineto{\pgfqpoint{5.749371in}{1.083732in}}%
\pgfpathlineto{\pgfqpoint{5.754903in}{1.089016in}}%
\pgfpathlineto{\pgfqpoint{5.754287in}{1.096531in}}%
\pgfpathlineto{\pgfqpoint{5.758989in}{1.103993in}}%
\pgfpathlineto{\pgfqpoint{5.765290in}{1.105610in}}%
\pgfpathlineto{\pgfqpoint{5.767820in}{1.115463in}}%
\pgfpathlineto{\pgfqpoint{5.776983in}{1.123981in}}%
\pgfpathlineto{\pgfqpoint{5.774225in}{1.132459in}}%
\pgfpathlineto{\pgfqpoint{5.783139in}{1.136680in}}%
\pgfpathlineto{\pgfqpoint{5.789096in}{1.134084in}}%
\pgfpathlineto{\pgfqpoint{5.797893in}{1.140679in}}%
\pgfpathlineto{\pgfqpoint{5.805755in}{1.142397in}}%
\pgfpathlineto{\pgfqpoint{5.805783in}{1.149238in}}%
\pgfpathlineto{\pgfqpoint{5.800273in}{1.158730in}}%
\pgfpathlineto{\pgfqpoint{5.803860in}{1.162554in}}%
\pgfpathlineto{\pgfqpoint{5.815128in}{1.161707in}}%
\pgfpathlineto{\pgfqpoint{5.822293in}{1.165276in}}%
\pgfpathlineto{\pgfqpoint{5.829684in}{1.159700in}}%
\pgfpathlineto{\pgfqpoint{5.835517in}{1.147114in}}%
\pgfpathlineto{\pgfqpoint{5.854857in}{1.144988in}}%
\pgfpathlineto{\pgfqpoint{5.864938in}{1.138060in}}%
\pgfpathlineto{\pgfqpoint{5.872031in}{1.143916in}}%
\pgfpathlineto{\pgfqpoint{5.884855in}{1.138994in}}%
\pgfpathlineto{\pgfqpoint{5.892116in}{1.141440in}}%
\pgfpathlineto{\pgfqpoint{5.897191in}{1.148394in}}%
\pgfpathlineto{\pgfqpoint{5.904937in}{1.149724in}}%
\pgfpathlineto{\pgfqpoint{5.906950in}{1.142163in}}%
\pgfpathlineto{\pgfqpoint{5.915518in}{1.139272in}}%
\pgfpathlineto{\pgfqpoint{5.922176in}{1.131815in}}%
\pgfpathlineto{\pgfqpoint{5.924185in}{1.121744in}}%
\pgfpathlineto{\pgfqpoint{5.921513in}{1.114131in}}%
\pgfpathlineto{\pgfqpoint{5.931736in}{1.104274in}}%
\pgfpathlineto{\pgfqpoint{5.943746in}{1.085662in}}%
\pgfpathlineto{\pgfqpoint{5.952636in}{1.079539in}}%
\pgfpathlineto{\pgfqpoint{5.962284in}{1.079184in}}%
\pgfpathlineto{\pgfqpoint{5.944469in}{1.058743in}}%
\pgfpathlineto{\pgfqpoint{5.926934in}{1.046371in}}%
\pgfpathlineto{\pgfqpoint{5.920481in}{1.036545in}}%
\pgfpathlineto{\pgfqpoint{5.920590in}{1.031213in}}%
\pgfpathlineto{\pgfqpoint{5.911092in}{1.027126in}}%
\pgfpathlineto{\pgfqpoint{5.908370in}{1.019436in}}%
\pgfpathlineto{\pgfqpoint{5.888584in}{1.011690in}}%
\pgfpathlineto{\pgfqpoint{5.881571in}{1.006660in}}%
\pgfpathlineto{\pgfqpoint{5.880622in}{1.005586in}}%
\pgfpathlineto{\pgfqpoint{5.823708in}{1.000190in}}%
\pgfpathlineto{\pgfqpoint{5.789337in}{0.997299in}}%
\pgfpathlineto{\pgfqpoint{5.732938in}{0.994107in}}%
\pgfpathlineto{\pgfqpoint{5.662674in}{0.987156in}}%
\pgfpathlineto{\pgfqpoint{5.651065in}{0.988764in}}%
\pgfpathlineto{\pgfqpoint{5.653480in}{0.976914in}}%
\pgfpathlineto{\pgfqpoint{5.581343in}{0.972329in}}%
\pgfpathclose%
\pgfusepath{fill}%
\end{pgfscope}%
\begin{pgfscope}%
\pgfpathrectangle{\pgfqpoint{3.625000in}{0.100000in}}{\pgfqpoint{2.989028in}{1.913466in}}%
\pgfusepath{clip}%
\pgfsetbuttcap%
\pgfsetmiterjoin%
\definecolor{currentfill}{rgb}{0.384006,0.742945,0.654441}%
\pgfsetfillcolor{currentfill}%
\pgfsetlinewidth{0.000000pt}%
\definecolor{currentstroke}{rgb}{0.000000,0.000000,0.000000}%
\pgfsetstrokecolor{currentstroke}%
\pgfsetstrokeopacity{0.000000}%
\pgfsetdash{}{0pt}%
\pgfpathmoveto{\pgfqpoint{4.919404in}{1.009408in}}%
\pgfpathlineto{\pgfqpoint{4.925359in}{1.092033in}}%
\pgfpathlineto{\pgfqpoint{4.929183in}{1.143169in}}%
\pgfpathlineto{\pgfqpoint{4.933935in}{1.205970in}}%
\pgfpathlineto{\pgfqpoint{4.999809in}{1.201345in}}%
\pgfpathlineto{\pgfqpoint{5.083468in}{1.196775in}}%
\pgfpathlineto{\pgfqpoint{5.140367in}{1.194592in}}%
\pgfpathlineto{\pgfqpoint{5.196933in}{1.192859in}}%
\pgfpathlineto{\pgfqpoint{5.248131in}{1.192066in}}%
\pgfpathlineto{\pgfqpoint{5.271810in}{1.192312in}}%
\pgfpathlineto{\pgfqpoint{5.282233in}{1.183806in}}%
\pgfpathlineto{\pgfqpoint{5.290397in}{1.185521in}}%
\pgfpathlineto{\pgfqpoint{5.290653in}{1.178102in}}%
\pgfpathlineto{\pgfqpoint{5.284590in}{1.165269in}}%
\pgfpathlineto{\pgfqpoint{5.285247in}{1.157161in}}%
\pgfpathlineto{\pgfqpoint{5.291052in}{1.151814in}}%
\pgfpathlineto{\pgfqpoint{5.296384in}{1.140674in}}%
\pgfpathlineto{\pgfqpoint{5.307201in}{1.134279in}}%
\pgfpathlineto{\pgfqpoint{5.306840in}{1.092287in}}%
\pgfpathlineto{\pgfqpoint{5.307158in}{0.995725in}}%
\pgfpathlineto{\pgfqpoint{5.234656in}{0.996060in}}%
\pgfpathlineto{\pgfqpoint{5.158309in}{0.997396in}}%
\pgfpathlineto{\pgfqpoint{5.102105in}{0.999379in}}%
\pgfpathlineto{\pgfqpoint{5.055191in}{1.001189in}}%
\pgfpathlineto{\pgfqpoint{4.976155in}{1.005849in}}%
\pgfpathlineto{\pgfqpoint{4.919404in}{1.009408in}}%
\pgfpathclose%
\pgfusepath{fill}%
\end{pgfscope}%
\begin{pgfscope}%
\pgfpathrectangle{\pgfqpoint{3.625000in}{0.100000in}}{\pgfqpoint{2.989028in}{1.913466in}}%
\pgfusepath{clip}%
\pgfsetbuttcap%
\pgfsetmiterjoin%
\definecolor{currentfill}{rgb}{0.774933,0.909112,0.621915}%
\pgfsetfillcolor{currentfill}%
\pgfsetlinewidth{0.000000pt}%
\definecolor{currentstroke}{rgb}{0.000000,0.000000,0.000000}%
\pgfsetstrokecolor{currentstroke}%
\pgfsetstrokeopacity{0.000000}%
\pgfsetdash{}{0pt}%
\pgfpathmoveto{\pgfqpoint{5.986189in}{1.020348in}}%
\pgfpathlineto{\pgfqpoint{5.905888in}{1.009068in}}%
\pgfpathlineto{\pgfqpoint{5.881571in}{1.006660in}}%
\pgfpathlineto{\pgfqpoint{5.888584in}{1.011690in}}%
\pgfpathlineto{\pgfqpoint{5.908370in}{1.019436in}}%
\pgfpathlineto{\pgfqpoint{5.911092in}{1.027126in}}%
\pgfpathlineto{\pgfqpoint{5.920590in}{1.031213in}}%
\pgfpathlineto{\pgfqpoint{5.920481in}{1.036545in}}%
\pgfpathlineto{\pgfqpoint{5.926934in}{1.046371in}}%
\pgfpathlineto{\pgfqpoint{5.944469in}{1.058743in}}%
\pgfpathlineto{\pgfqpoint{5.962284in}{1.079184in}}%
\pgfpathlineto{\pgfqpoint{5.961416in}{1.074785in}}%
\pgfpathlineto{\pgfqpoint{5.970810in}{1.064162in}}%
\pgfpathlineto{\pgfqpoint{5.986126in}{1.061179in}}%
\pgfpathlineto{\pgfqpoint{5.995035in}{1.071300in}}%
\pgfpathlineto{\pgfqpoint{6.003007in}{1.065970in}}%
\pgfpathlineto{\pgfqpoint{6.017271in}{1.072225in}}%
\pgfpathlineto{\pgfqpoint{6.019311in}{1.078334in}}%
\pgfpathlineto{\pgfqpoint{6.028164in}{1.079708in}}%
\pgfpathlineto{\pgfqpoint{6.037358in}{1.086925in}}%
\pgfpathlineto{\pgfqpoint{6.045463in}{1.089306in}}%
\pgfpathlineto{\pgfqpoint{6.045721in}{1.102055in}}%
\pgfpathlineto{\pgfqpoint{6.050969in}{1.115827in}}%
\pgfpathlineto{\pgfqpoint{6.057891in}{1.124451in}}%
\pgfpathlineto{\pgfqpoint{6.060135in}{1.137399in}}%
\pgfpathlineto{\pgfqpoint{6.069157in}{1.155588in}}%
\pgfpathlineto{\pgfqpoint{6.069480in}{1.166411in}}%
\pgfpathlineto{\pgfqpoint{6.078799in}{1.159359in}}%
\pgfpathlineto{\pgfqpoint{6.087911in}{1.157648in}}%
\pgfpathlineto{\pgfqpoint{6.091456in}{1.162779in}}%
\pgfpathlineto{\pgfqpoint{6.097791in}{1.185136in}}%
\pgfpathlineto{\pgfqpoint{6.106611in}{1.184279in}}%
\pgfpathlineto{\pgfqpoint{6.108947in}{1.192470in}}%
\pgfpathlineto{\pgfqpoint{6.114121in}{1.196164in}}%
\pgfpathlineto{\pgfqpoint{6.125084in}{1.214739in}}%
\pgfpathlineto{\pgfqpoint{6.123492in}{1.220234in}}%
\pgfpathlineto{\pgfqpoint{6.128317in}{1.232177in}}%
\pgfpathlineto{\pgfqpoint{6.154074in}{1.217629in}}%
\pgfpathlineto{\pgfqpoint{6.156926in}{1.230550in}}%
\pgfpathlineto{\pgfqpoint{6.165064in}{1.230836in}}%
\pgfpathlineto{\pgfqpoint{6.171385in}{1.226747in}}%
\pgfpathlineto{\pgfqpoint{6.170120in}{1.219698in}}%
\pgfpathlineto{\pgfqpoint{6.184066in}{1.216492in}}%
\pgfpathlineto{\pgfqpoint{6.191941in}{1.211681in}}%
\pgfpathlineto{\pgfqpoint{6.197645in}{1.203210in}}%
\pgfpathlineto{\pgfqpoint{6.195205in}{1.193376in}}%
\pgfpathlineto{\pgfqpoint{6.190277in}{1.192570in}}%
\pgfpathlineto{\pgfqpoint{6.187419in}{1.177731in}}%
\pgfpathlineto{\pgfqpoint{6.193680in}{1.171929in}}%
\pgfpathlineto{\pgfqpoint{6.202493in}{1.176619in}}%
\pgfpathlineto{\pgfqpoint{6.210671in}{1.166716in}}%
\pgfpathlineto{\pgfqpoint{6.228942in}{1.164936in}}%
\pgfpathlineto{\pgfqpoint{6.232089in}{1.159239in}}%
\pgfpathlineto{\pgfqpoint{6.248998in}{1.153699in}}%
\pgfpathlineto{\pgfqpoint{6.246912in}{1.147161in}}%
\pgfpathlineto{\pgfqpoint{6.249111in}{1.128683in}}%
\pgfpathlineto{\pgfqpoint{6.255936in}{1.121706in}}%
\pgfpathlineto{\pgfqpoint{6.245866in}{1.118761in}}%
\pgfpathlineto{\pgfqpoint{6.247201in}{1.110873in}}%
\pgfpathlineto{\pgfqpoint{6.257944in}{1.104185in}}%
\pgfpathlineto{\pgfqpoint{6.258912in}{1.097588in}}%
\pgfpathlineto{\pgfqpoint{6.252847in}{1.092630in}}%
\pgfpathlineto{\pgfqpoint{6.240609in}{1.104375in}}%
\pgfpathlineto{\pgfqpoint{6.239399in}{1.095806in}}%
\pgfpathlineto{\pgfqpoint{6.249644in}{1.091731in}}%
\pgfpathlineto{\pgfqpoint{6.250669in}{1.087517in}}%
\pgfpathlineto{\pgfqpoint{6.264721in}{1.093086in}}%
\pgfpathlineto{\pgfqpoint{6.275489in}{1.094559in}}%
\pgfpathlineto{\pgfqpoint{6.286520in}{1.072304in}}%
\pgfpathlineto{\pgfqpoint{6.285289in}{1.072063in}}%
\pgfpathlineto{\pgfqpoint{6.280305in}{1.071033in}}%
\pgfpathlineto{\pgfqpoint{6.278836in}{1.070726in}}%
\pgfpathlineto{\pgfqpoint{6.277865in}{1.070536in}}%
\pgfpathlineto{\pgfqpoint{6.212189in}{1.056913in}}%
\pgfpathlineto{\pgfqpoint{6.153447in}{1.044890in}}%
\pgfpathlineto{\pgfqpoint{6.072229in}{1.030898in}}%
\pgfpathlineto{\pgfqpoint{6.003251in}{1.021788in}}%
\pgfpathlineto{\pgfqpoint{5.986189in}{1.020348in}}%
\pgfpathclose%
\pgfusepath{fill}%
\end{pgfscope}%
\begin{pgfscope}%
\pgfpathrectangle{\pgfqpoint{3.625000in}{0.100000in}}{\pgfqpoint{2.989028in}{1.913466in}}%
\pgfusepath{clip}%
\pgfsetbuttcap%
\pgfsetmiterjoin%
\definecolor{currentfill}{rgb}{0.774933,0.909112,0.621915}%
\pgfsetfillcolor{currentfill}%
\pgfsetlinewidth{0.000000pt}%
\definecolor{currentstroke}{rgb}{0.000000,0.000000,0.000000}%
\pgfsetstrokecolor{currentstroke}%
\pgfsetstrokeopacity{0.000000}%
\pgfsetdash{}{0pt}%
\pgfpathmoveto{\pgfqpoint{6.278775in}{1.165395in}}%
\pgfpathlineto{\pgfqpoint{6.292015in}{1.171401in}}%
\pgfpathlineto{\pgfqpoint{6.281442in}{1.140471in}}%
\pgfpathlineto{\pgfqpoint{6.274450in}{1.124092in}}%
\pgfpathlineto{\pgfqpoint{6.269413in}{1.133361in}}%
\pgfpathlineto{\pgfqpoint{6.278370in}{1.155552in}}%
\pgfpathlineto{\pgfqpoint{6.278775in}{1.165395in}}%
\pgfpathclose%
\pgfusepath{fill}%
\end{pgfscope}%
\begin{pgfscope}%
\pgfpathrectangle{\pgfqpoint{3.625000in}{0.100000in}}{\pgfqpoint{2.989028in}{1.913466in}}%
\pgfusepath{clip}%
\pgfsetbuttcap%
\pgfsetmiterjoin%
\definecolor{currentfill}{rgb}{0.874740,0.949712,0.601615}%
\pgfsetfillcolor{currentfill}%
\pgfsetlinewidth{0.000000pt}%
\definecolor{currentstroke}{rgb}{0.000000,0.000000,0.000000}%
\pgfsetstrokecolor{currentstroke}%
\pgfsetstrokeopacity{0.000000}%
\pgfsetdash{}{0pt}%
\pgfpathmoveto{\pgfqpoint{5.581343in}{0.972329in}}%
\pgfpathlineto{\pgfqpoint{5.578146in}{0.971866in}}%
\pgfpathlineto{\pgfqpoint{5.575131in}{0.971655in}}%
\pgfpathlineto{\pgfqpoint{5.576421in}{0.962397in}}%
\pgfpathlineto{\pgfqpoint{5.568642in}{0.953066in}}%
\pgfpathlineto{\pgfqpoint{5.567109in}{0.938464in}}%
\pgfpathlineto{\pgfqpoint{5.532334in}{0.935892in}}%
\pgfpathlineto{\pgfqpoint{5.535365in}{0.942748in}}%
\pgfpathlineto{\pgfqpoint{5.547904in}{0.955269in}}%
\pgfpathlineto{\pgfqpoint{5.548251in}{0.962527in}}%
\pgfpathlineto{\pgfqpoint{5.542700in}{0.969399in}}%
\pgfpathlineto{\pgfqpoint{5.490941in}{0.966662in}}%
\pgfpathlineto{\pgfqpoint{5.400448in}{0.963789in}}%
\pgfpathlineto{\pgfqpoint{5.347491in}{0.962824in}}%
\pgfpathlineto{\pgfqpoint{5.307462in}{0.962463in}}%
\pgfpathlineto{\pgfqpoint{5.307158in}{0.995725in}}%
\pgfpathlineto{\pgfqpoint{5.306840in}{1.092287in}}%
\pgfpathlineto{\pgfqpoint{5.307201in}{1.134279in}}%
\pgfpathlineto{\pgfqpoint{5.296384in}{1.140674in}}%
\pgfpathlineto{\pgfqpoint{5.291052in}{1.151814in}}%
\pgfpathlineto{\pgfqpoint{5.285247in}{1.157161in}}%
\pgfpathlineto{\pgfqpoint{5.284590in}{1.165269in}}%
\pgfpathlineto{\pgfqpoint{5.290653in}{1.178102in}}%
\pgfpathlineto{\pgfqpoint{5.290397in}{1.185521in}}%
\pgfpathlineto{\pgfqpoint{5.282233in}{1.183806in}}%
\pgfpathlineto{\pgfqpoint{5.271810in}{1.192312in}}%
\pgfpathlineto{\pgfqpoint{5.263455in}{1.207242in}}%
\pgfpathlineto{\pgfqpoint{5.256447in}{1.214132in}}%
\pgfpathlineto{\pgfqpoint{5.249124in}{1.231060in}}%
\pgfpathlineto{\pgfqpoint{5.325184in}{1.229874in}}%
\pgfpathlineto{\pgfqpoint{5.400789in}{1.232365in}}%
\pgfpathlineto{\pgfqpoint{5.449266in}{1.235161in}}%
\pgfpathlineto{\pgfqpoint{5.464431in}{1.220143in}}%
\pgfpathlineto{\pgfqpoint{5.461659in}{1.204480in}}%
\pgfpathlineto{\pgfqpoint{5.465703in}{1.186846in}}%
\pgfpathlineto{\pgfqpoint{5.470188in}{1.177954in}}%
\pgfpathlineto{\pgfqpoint{5.484499in}{1.165722in}}%
\pgfpathlineto{\pgfqpoint{5.487630in}{1.160319in}}%
\pgfpathlineto{\pgfqpoint{5.503299in}{1.148144in}}%
\pgfpathlineto{\pgfqpoint{5.505570in}{1.134486in}}%
\pgfpathlineto{\pgfqpoint{5.510070in}{1.124410in}}%
\pgfpathlineto{\pgfqpoint{5.520702in}{1.130279in}}%
\pgfpathlineto{\pgfqpoint{5.529465in}{1.127595in}}%
\pgfpathlineto{\pgfqpoint{5.536082in}{1.120472in}}%
\pgfpathlineto{\pgfqpoint{5.533584in}{1.107865in}}%
\pgfpathlineto{\pgfqpoint{5.525462in}{1.091383in}}%
\pgfpathlineto{\pgfqpoint{5.525587in}{1.082799in}}%
\pgfpathlineto{\pgfqpoint{5.531657in}{1.075760in}}%
\pgfpathlineto{\pgfqpoint{5.544719in}{1.066437in}}%
\pgfpathlineto{\pgfqpoint{5.559760in}{1.058870in}}%
\pgfpathlineto{\pgfqpoint{5.563328in}{1.052811in}}%
\pgfpathlineto{\pgfqpoint{5.571020in}{1.049946in}}%
\pgfpathlineto{\pgfqpoint{5.571326in}{1.041904in}}%
\pgfpathlineto{\pgfqpoint{5.577019in}{1.031690in}}%
\pgfpathlineto{\pgfqpoint{5.573069in}{1.022616in}}%
\pgfpathlineto{\pgfqpoint{5.580801in}{1.008178in}}%
\pgfpathlineto{\pgfqpoint{5.586849in}{1.011045in}}%
\pgfpathlineto{\pgfqpoint{5.594266in}{1.005020in}}%
\pgfpathlineto{\pgfqpoint{5.591773in}{0.978188in}}%
\pgfpathlineto{\pgfqpoint{5.583030in}{0.979890in}}%
\pgfpathlineto{\pgfqpoint{5.581343in}{0.972329in}}%
\pgfpathclose%
\pgfusepath{fill}%
\end{pgfscope}%
\begin{pgfscope}%
\pgfpathrectangle{\pgfqpoint{3.625000in}{0.100000in}}{\pgfqpoint{2.989028in}{1.913466in}}%
\pgfusepath{clip}%
\pgfsetbuttcap%
\pgfsetmiterjoin%
\definecolor{currentfill}{rgb}{0.784006,0.912803,0.620069}%
\pgfsetfillcolor{currentfill}%
\pgfsetlinewidth{0.000000pt}%
\definecolor{currentstroke}{rgb}{0.000000,0.000000,0.000000}%
\pgfsetstrokecolor{currentstroke}%
\pgfsetstrokeopacity{0.000000}%
\pgfsetdash{}{0pt}%
\pgfpathmoveto{\pgfqpoint{4.240525in}{0.978113in}}%
\pgfpathlineto{\pgfqpoint{4.246856in}{0.991592in}}%
\pgfpathlineto{\pgfqpoint{4.245123in}{1.011825in}}%
\pgfpathlineto{\pgfqpoint{4.248155in}{1.017578in}}%
\pgfpathlineto{\pgfqpoint{4.248454in}{1.042780in}}%
\pgfpathlineto{\pgfqpoint{4.251328in}{1.050037in}}%
\pgfpathlineto{\pgfqpoint{4.261451in}{1.051167in}}%
\pgfpathlineto{\pgfqpoint{4.270846in}{1.047945in}}%
\pgfpathlineto{\pgfqpoint{4.274938in}{1.039072in}}%
\pgfpathlineto{\pgfqpoint{4.280674in}{1.039412in}}%
\pgfpathlineto{\pgfqpoint{4.287801in}{1.049576in}}%
\pgfpathlineto{\pgfqpoint{4.298103in}{1.099678in}}%
\pgfpathlineto{\pgfqpoint{4.356724in}{1.087575in}}%
\pgfpathlineto{\pgfqpoint{4.390746in}{1.080865in}}%
\pgfpathlineto{\pgfqpoint{4.481091in}{1.064984in}}%
\pgfpathlineto{\pgfqpoint{4.506086in}{1.059964in}}%
\pgfpathlineto{\pgfqpoint{4.555137in}{1.052362in}}%
\pgfpathlineto{\pgfqpoint{4.545079in}{0.987601in}}%
\pgfpathlineto{\pgfqpoint{4.534619in}{0.920049in}}%
\pgfpathlineto{\pgfqpoint{4.522573in}{0.844052in}}%
\pgfpathlineto{\pgfqpoint{4.509025in}{0.756815in}}%
\pgfpathlineto{\pgfqpoint{4.498080in}{0.685166in}}%
\pgfpathlineto{\pgfqpoint{4.419687in}{0.697614in}}%
\pgfpathlineto{\pgfqpoint{4.385244in}{0.703513in}}%
\pgfpathlineto{\pgfqpoint{4.369857in}{0.712725in}}%
\pgfpathlineto{\pgfqpoint{4.269535in}{0.773254in}}%
\pgfpathlineto{\pgfqpoint{4.194248in}{0.819189in}}%
\pgfpathlineto{\pgfqpoint{4.196756in}{0.827331in}}%
\pgfpathlineto{\pgfqpoint{4.202975in}{0.833027in}}%
\pgfpathlineto{\pgfqpoint{4.209444in}{0.831972in}}%
\pgfpathlineto{\pgfqpoint{4.218826in}{0.837954in}}%
\pgfpathlineto{\pgfqpoint{4.220303in}{0.846552in}}%
\pgfpathlineto{\pgfqpoint{4.208861in}{0.856982in}}%
\pgfpathlineto{\pgfqpoint{4.216998in}{0.877004in}}%
\pgfpathlineto{\pgfqpoint{4.225180in}{0.884704in}}%
\pgfpathlineto{\pgfqpoint{4.229063in}{0.893839in}}%
\pgfpathlineto{\pgfqpoint{4.231455in}{0.910589in}}%
\pgfpathlineto{\pgfqpoint{4.239127in}{0.918174in}}%
\pgfpathlineto{\pgfqpoint{4.255230in}{0.925750in}}%
\pgfpathlineto{\pgfqpoint{4.254874in}{0.932426in}}%
\pgfpathlineto{\pgfqpoint{4.245881in}{0.940711in}}%
\pgfpathlineto{\pgfqpoint{4.244656in}{0.957789in}}%
\pgfpathlineto{\pgfqpoint{4.238464in}{0.970274in}}%
\pgfpathlineto{\pgfqpoint{4.240525in}{0.978113in}}%
\pgfpathclose%
\pgfusepath{fill}%
\end{pgfscope}%
\begin{pgfscope}%
\pgfpathrectangle{\pgfqpoint{3.625000in}{0.100000in}}{\pgfqpoint{2.989028in}{1.913466in}}%
\pgfusepath{clip}%
\pgfsetbuttcap%
\pgfsetmiterjoin%
\definecolor{currentfill}{rgb}{0.256055,0.600231,0.713495}%
\pgfsetfillcolor{currentfill}%
\pgfsetlinewidth{0.000000pt}%
\definecolor{currentstroke}{rgb}{0.000000,0.000000,0.000000}%
\pgfsetstrokecolor{currentstroke}%
\pgfsetstrokeopacity{0.000000}%
\pgfsetdash{}{0pt}%
\pgfpathmoveto{\pgfqpoint{5.315948in}{0.774957in}}%
\pgfpathlineto{\pgfqpoint{5.301416in}{0.779451in}}%
\pgfpathlineto{\pgfqpoint{5.282795in}{0.793711in}}%
\pgfpathlineto{\pgfqpoint{5.274524in}{0.796776in}}%
\pgfpathlineto{\pgfqpoint{5.269268in}{0.790617in}}%
\pgfpathlineto{\pgfqpoint{5.262631in}{0.790306in}}%
\pgfpathlineto{\pgfqpoint{5.254234in}{0.795535in}}%
\pgfpathlineto{\pgfqpoint{5.241057in}{0.788835in}}%
\pgfpathlineto{\pgfqpoint{5.236367in}{0.792135in}}%
\pgfpathlineto{\pgfqpoint{5.223401in}{0.784318in}}%
\pgfpathlineto{\pgfqpoint{5.209722in}{0.785761in}}%
\pgfpathlineto{\pgfqpoint{5.199172in}{0.790840in}}%
\pgfpathlineto{\pgfqpoint{5.191783in}{0.788928in}}%
\pgfpathlineto{\pgfqpoint{5.179956in}{0.796112in}}%
\pgfpathlineto{\pgfqpoint{5.173377in}{0.783442in}}%
\pgfpathlineto{\pgfqpoint{5.166649in}{0.794338in}}%
\pgfpathlineto{\pgfqpoint{5.153360in}{0.790111in}}%
\pgfpathlineto{\pgfqpoint{5.141736in}{0.800460in}}%
\pgfpathlineto{\pgfqpoint{5.131579in}{0.792118in}}%
\pgfpathlineto{\pgfqpoint{5.126493in}{0.799784in}}%
\pgfpathlineto{\pgfqpoint{5.119161in}{0.802273in}}%
\pgfpathlineto{\pgfqpoint{5.114696in}{0.809664in}}%
\pgfpathlineto{\pgfqpoint{5.105106in}{0.811767in}}%
\pgfpathlineto{\pgfqpoint{5.099573in}{0.806204in}}%
\pgfpathlineto{\pgfqpoint{5.090171in}{0.813414in}}%
\pgfpathlineto{\pgfqpoint{5.085789in}{0.811767in}}%
\pgfpathlineto{\pgfqpoint{5.070207in}{0.817609in}}%
\pgfpathlineto{\pgfqpoint{5.060448in}{0.818261in}}%
\pgfpathlineto{\pgfqpoint{5.056080in}{0.830667in}}%
\pgfpathlineto{\pgfqpoint{5.048256in}{0.829122in}}%
\pgfpathlineto{\pgfqpoint{5.039296in}{0.832183in}}%
\pgfpathlineto{\pgfqpoint{5.033395in}{0.830414in}}%
\pgfpathlineto{\pgfqpoint{5.020741in}{0.844347in}}%
\pgfpathlineto{\pgfqpoint{5.017215in}{0.843436in}}%
\pgfpathlineto{\pgfqpoint{5.020417in}{0.899940in}}%
\pgfpathlineto{\pgfqpoint{5.023906in}{0.969876in}}%
\pgfpathlineto{\pgfqpoint{4.966639in}{0.973080in}}%
\pgfpathlineto{\pgfqpoint{4.910128in}{0.977352in}}%
\pgfpathlineto{\pgfqpoint{4.866469in}{0.981143in}}%
\pgfpathlineto{\pgfqpoint{4.869368in}{1.014183in}}%
\pgfpathlineto{\pgfqpoint{4.919404in}{1.009408in}}%
\pgfpathlineto{\pgfqpoint{4.976155in}{1.005849in}}%
\pgfpathlineto{\pgfqpoint{5.055191in}{1.001189in}}%
\pgfpathlineto{\pgfqpoint{5.102105in}{0.999379in}}%
\pgfpathlineto{\pgfqpoint{5.158309in}{0.997396in}}%
\pgfpathlineto{\pgfqpoint{5.234656in}{0.996060in}}%
\pgfpathlineto{\pgfqpoint{5.307158in}{0.995725in}}%
\pgfpathlineto{\pgfqpoint{5.307462in}{0.962463in}}%
\pgfpathlineto{\pgfqpoint{5.311531in}{0.937404in}}%
\pgfpathlineto{\pgfqpoint{5.317853in}{0.891120in}}%
\pgfpathlineto{\pgfqpoint{5.316549in}{0.812079in}}%
\pgfpathlineto{\pgfqpoint{5.315948in}{0.774957in}}%
\pgfpathclose%
\pgfusepath{fill}%
\end{pgfscope}%
\begin{pgfscope}%
\pgfpathrectangle{\pgfqpoint{3.625000in}{0.100000in}}{\pgfqpoint{2.989028in}{1.913466in}}%
\pgfusepath{clip}%
\pgfsetbuttcap%
\pgfsetmiterjoin%
\definecolor{currentfill}{rgb}{0.711419,0.883276,0.634833}%
\pgfsetfillcolor{currentfill}%
\pgfsetlinewidth{0.000000pt}%
\definecolor{currentstroke}{rgb}{0.000000,0.000000,0.000000}%
\pgfsetstrokecolor{currentstroke}%
\pgfsetstrokeopacity{0.000000}%
\pgfsetdash{}{0pt}%
\pgfpathmoveto{\pgfqpoint{5.860069in}{0.897708in}}%
\pgfpathlineto{\pgfqpoint{5.860112in}{0.912353in}}%
\pgfpathlineto{\pgfqpoint{5.872839in}{0.917977in}}%
\pgfpathlineto{\pgfqpoint{5.873372in}{0.926967in}}%
\pgfpathlineto{\pgfqpoint{5.884767in}{0.937949in}}%
\pgfpathlineto{\pgfqpoint{5.899011in}{0.940198in}}%
\pgfpathlineto{\pgfqpoint{5.911533in}{0.952301in}}%
\pgfpathlineto{\pgfqpoint{5.924610in}{0.957711in}}%
\pgfpathlineto{\pgfqpoint{5.934425in}{0.973951in}}%
\pgfpathlineto{\pgfqpoint{5.943364in}{0.972772in}}%
\pgfpathlineto{\pgfqpoint{5.962256in}{0.987570in}}%
\pgfpathlineto{\pgfqpoint{5.972232in}{0.987849in}}%
\pgfpathlineto{\pgfqpoint{5.976445in}{0.999123in}}%
\pgfpathlineto{\pgfqpoint{5.984375in}{1.006973in}}%
\pgfpathlineto{\pgfqpoint{5.986189in}{1.020348in}}%
\pgfpathlineto{\pgfqpoint{6.003251in}{1.021788in}}%
\pgfpathlineto{\pgfqpoint{6.072229in}{1.030898in}}%
\pgfpathlineto{\pgfqpoint{6.153447in}{1.044890in}}%
\pgfpathlineto{\pgfqpoint{6.212189in}{1.056913in}}%
\pgfpathlineto{\pgfqpoint{6.277865in}{1.070536in}}%
\pgfpathlineto{\pgfqpoint{6.296631in}{1.044770in}}%
\pgfpathlineto{\pgfqpoint{6.286466in}{1.046415in}}%
\pgfpathlineto{\pgfqpoint{6.273417in}{1.043236in}}%
\pgfpathlineto{\pgfqpoint{6.260559in}{1.030105in}}%
\pgfpathlineto{\pgfqpoint{6.251318in}{1.031050in}}%
\pgfpathlineto{\pgfqpoint{6.250149in}{1.023251in}}%
\pgfpathlineto{\pgfqpoint{6.266852in}{1.029418in}}%
\pgfpathlineto{\pgfqpoint{6.283660in}{1.031961in}}%
\pgfpathlineto{\pgfqpoint{6.289901in}{1.028577in}}%
\pgfpathlineto{\pgfqpoint{6.298300in}{1.032435in}}%
\pgfpathlineto{\pgfqpoint{6.302634in}{1.029732in}}%
\pgfpathlineto{\pgfqpoint{6.306470in}{1.016862in}}%
\pgfpathlineto{\pgfqpoint{6.298487in}{1.012876in}}%
\pgfpathlineto{\pgfqpoint{6.293008in}{0.997182in}}%
\pgfpathlineto{\pgfqpoint{6.287263in}{0.991072in}}%
\pgfpathlineto{\pgfqpoint{6.269653in}{0.992409in}}%
\pgfpathlineto{\pgfqpoint{6.270600in}{1.001624in}}%
\pgfpathlineto{\pgfqpoint{6.260982in}{0.997599in}}%
\pgfpathlineto{\pgfqpoint{6.258890in}{0.989913in}}%
\pgfpathlineto{\pgfqpoint{6.265483in}{0.981952in}}%
\pgfpathlineto{\pgfqpoint{6.267722in}{0.968440in}}%
\pgfpathlineto{\pgfqpoint{6.256935in}{0.960741in}}%
\pgfpathlineto{\pgfqpoint{6.268601in}{0.958032in}}%
\pgfpathlineto{\pgfqpoint{6.273881in}{0.963694in}}%
\pgfpathlineto{\pgfqpoint{6.285553in}{0.964383in}}%
\pgfpathlineto{\pgfqpoint{6.279555in}{0.952158in}}%
\pgfpathlineto{\pgfqpoint{6.272208in}{0.946268in}}%
\pgfpathlineto{\pgfqpoint{6.249984in}{0.940370in}}%
\pgfpathlineto{\pgfqpoint{6.227166in}{0.919675in}}%
\pgfpathlineto{\pgfqpoint{6.213096in}{0.899284in}}%
\pgfpathlineto{\pgfqpoint{6.213013in}{0.891003in}}%
\pgfpathlineto{\pgfqpoint{6.207393in}{0.879575in}}%
\pgfpathlineto{\pgfqpoint{6.178548in}{0.872057in}}%
\pgfpathlineto{\pgfqpoint{6.109952in}{0.921230in}}%
\pgfpathlineto{\pgfqpoint{6.049543in}{0.912354in}}%
\pgfpathlineto{\pgfqpoint{6.049037in}{0.920550in}}%
\pgfpathlineto{\pgfqpoint{6.039855in}{0.929779in}}%
\pgfpathlineto{\pgfqpoint{6.032904in}{0.932038in}}%
\pgfpathlineto{\pgfqpoint{5.967269in}{0.925218in}}%
\pgfpathlineto{\pgfqpoint{5.952183in}{0.920102in}}%
\pgfpathlineto{\pgfqpoint{5.924943in}{0.906551in}}%
\pgfpathlineto{\pgfqpoint{5.901402in}{0.902801in}}%
\pgfpathlineto{\pgfqpoint{5.860069in}{0.897708in}}%
\pgfpathclose%
\pgfusepath{fill}%
\end{pgfscope}%
\begin{pgfscope}%
\pgfpathrectangle{\pgfqpoint{3.625000in}{0.100000in}}{\pgfqpoint{2.989028in}{1.913466in}}%
\pgfusepath{clip}%
\pgfsetbuttcap%
\pgfsetmiterjoin%
\definecolor{currentfill}{rgb}{0.644060,0.856286,0.643522}%
\pgfsetfillcolor{currentfill}%
\pgfsetlinewidth{0.000000pt}%
\definecolor{currentstroke}{rgb}{0.000000,0.000000,0.000000}%
\pgfsetstrokecolor{currentstroke}%
\pgfsetstrokeopacity{0.000000}%
\pgfsetdash{}{0pt}%
\pgfpathmoveto{\pgfqpoint{5.860069in}{0.897708in}}%
\pgfpathlineto{\pgfqpoint{5.791375in}{0.890163in}}%
\pgfpathlineto{\pgfqpoint{5.728533in}{0.884512in}}%
\pgfpathlineto{\pgfqpoint{5.663945in}{0.880335in}}%
\pgfpathlineto{\pgfqpoint{5.652858in}{0.878728in}}%
\pgfpathlineto{\pgfqpoint{5.609319in}{0.875379in}}%
\pgfpathlineto{\pgfqpoint{5.539584in}{0.871310in}}%
\pgfpathlineto{\pgfqpoint{5.551990in}{0.881608in}}%
\pgfpathlineto{\pgfqpoint{5.549159in}{0.892449in}}%
\pgfpathlineto{\pgfqpoint{5.551902in}{0.905590in}}%
\pgfpathlineto{\pgfqpoint{5.556092in}{0.911777in}}%
\pgfpathlineto{\pgfqpoint{5.555934in}{0.920384in}}%
\pgfpathlineto{\pgfqpoint{5.567090in}{0.925797in}}%
\pgfpathlineto{\pgfqpoint{5.567109in}{0.938464in}}%
\pgfpathlineto{\pgfqpoint{5.568642in}{0.953066in}}%
\pgfpathlineto{\pgfqpoint{5.576421in}{0.962397in}}%
\pgfpathlineto{\pgfqpoint{5.575131in}{0.971655in}}%
\pgfpathlineto{\pgfqpoint{5.578146in}{0.971866in}}%
\pgfpathlineto{\pgfqpoint{5.581343in}{0.972329in}}%
\pgfpathlineto{\pgfqpoint{5.653480in}{0.976914in}}%
\pgfpathlineto{\pgfqpoint{5.651065in}{0.988764in}}%
\pgfpathlineto{\pgfqpoint{5.662674in}{0.987156in}}%
\pgfpathlineto{\pgfqpoint{5.732938in}{0.994107in}}%
\pgfpathlineto{\pgfqpoint{5.789337in}{0.997299in}}%
\pgfpathlineto{\pgfqpoint{5.823708in}{1.000190in}}%
\pgfpathlineto{\pgfqpoint{5.880622in}{1.005586in}}%
\pgfpathlineto{\pgfqpoint{5.881571in}{1.006660in}}%
\pgfpathlineto{\pgfqpoint{5.905888in}{1.009068in}}%
\pgfpathlineto{\pgfqpoint{5.986189in}{1.020348in}}%
\pgfpathlineto{\pgfqpoint{5.984375in}{1.006973in}}%
\pgfpathlineto{\pgfqpoint{5.976445in}{0.999123in}}%
\pgfpathlineto{\pgfqpoint{5.972232in}{0.987849in}}%
\pgfpathlineto{\pgfqpoint{5.962256in}{0.987570in}}%
\pgfpathlineto{\pgfqpoint{5.943364in}{0.972772in}}%
\pgfpathlineto{\pgfqpoint{5.934425in}{0.973951in}}%
\pgfpathlineto{\pgfqpoint{5.924610in}{0.957711in}}%
\pgfpathlineto{\pgfqpoint{5.911533in}{0.952301in}}%
\pgfpathlineto{\pgfqpoint{5.899011in}{0.940198in}}%
\pgfpathlineto{\pgfqpoint{5.884767in}{0.937949in}}%
\pgfpathlineto{\pgfqpoint{5.873372in}{0.926967in}}%
\pgfpathlineto{\pgfqpoint{5.872839in}{0.917977in}}%
\pgfpathlineto{\pgfqpoint{5.860112in}{0.912353in}}%
\pgfpathlineto{\pgfqpoint{5.860069in}{0.897708in}}%
\pgfpathclose%
\pgfusepath{fill}%
\end{pgfscope}%
\begin{pgfscope}%
\pgfpathrectangle{\pgfqpoint{3.625000in}{0.100000in}}{\pgfqpoint{2.989028in}{1.913466in}}%
\pgfusepath{clip}%
\pgfsetbuttcap%
\pgfsetmiterjoin%
\definecolor{currentfill}{rgb}{0.224068,0.564552,0.728258}%
\pgfsetfillcolor{currentfill}%
\pgfsetlinewidth{0.000000pt}%
\definecolor{currentstroke}{rgb}{0.000000,0.000000,0.000000}%
\pgfsetstrokecolor{currentstroke}%
\pgfsetstrokeopacity{0.000000}%
\pgfsetdash{}{0pt}%
\pgfpathmoveto{\pgfqpoint{4.641758in}{0.694333in}}%
\pgfpathlineto{\pgfqpoint{4.636859in}{0.702205in}}%
\pgfpathlineto{\pgfqpoint{4.638886in}{0.708987in}}%
\pgfpathlineto{\pgfqpoint{4.673324in}{0.704715in}}%
\pgfpathlineto{\pgfqpoint{4.737359in}{0.697442in}}%
\pgfpathlineto{\pgfqpoint{4.783660in}{0.692892in}}%
\pgfpathlineto{\pgfqpoint{4.837135in}{0.687496in}}%
\pgfpathlineto{\pgfqpoint{4.840044in}{0.721225in}}%
\pgfpathlineto{\pgfqpoint{4.849140in}{0.807015in}}%
\pgfpathlineto{\pgfqpoint{4.855041in}{0.867115in}}%
\pgfpathlineto{\pgfqpoint{4.860142in}{0.924525in}}%
\pgfpathlineto{\pgfqpoint{4.864896in}{0.981231in}}%
\pgfpathlineto{\pgfqpoint{4.866469in}{0.981143in}}%
\pgfpathlineto{\pgfqpoint{4.910128in}{0.977352in}}%
\pgfpathlineto{\pgfqpoint{4.966639in}{0.973080in}}%
\pgfpathlineto{\pgfqpoint{5.023906in}{0.969876in}}%
\pgfpathlineto{\pgfqpoint{5.020417in}{0.899940in}}%
\pgfpathlineto{\pgfqpoint{5.017215in}{0.843436in}}%
\pgfpathlineto{\pgfqpoint{5.020741in}{0.844347in}}%
\pgfpathlineto{\pgfqpoint{5.033395in}{0.830414in}}%
\pgfpathlineto{\pgfqpoint{5.039296in}{0.832183in}}%
\pgfpathlineto{\pgfqpoint{5.048256in}{0.829122in}}%
\pgfpathlineto{\pgfqpoint{5.056080in}{0.830667in}}%
\pgfpathlineto{\pgfqpoint{5.060448in}{0.818261in}}%
\pgfpathlineto{\pgfqpoint{5.070207in}{0.817609in}}%
\pgfpathlineto{\pgfqpoint{5.085789in}{0.811767in}}%
\pgfpathlineto{\pgfqpoint{5.090171in}{0.813414in}}%
\pgfpathlineto{\pgfqpoint{5.099573in}{0.806204in}}%
\pgfpathlineto{\pgfqpoint{5.105106in}{0.811767in}}%
\pgfpathlineto{\pgfqpoint{5.114696in}{0.809664in}}%
\pgfpathlineto{\pgfqpoint{5.119161in}{0.802273in}}%
\pgfpathlineto{\pgfqpoint{5.126493in}{0.799784in}}%
\pgfpathlineto{\pgfqpoint{5.131579in}{0.792118in}}%
\pgfpathlineto{\pgfqpoint{5.141736in}{0.800460in}}%
\pgfpathlineto{\pgfqpoint{5.153360in}{0.790111in}}%
\pgfpathlineto{\pgfqpoint{5.166649in}{0.794338in}}%
\pgfpathlineto{\pgfqpoint{5.173377in}{0.783442in}}%
\pgfpathlineto{\pgfqpoint{5.179956in}{0.796112in}}%
\pgfpathlineto{\pgfqpoint{5.191783in}{0.788928in}}%
\pgfpathlineto{\pgfqpoint{5.199172in}{0.790840in}}%
\pgfpathlineto{\pgfqpoint{5.209722in}{0.785761in}}%
\pgfpathlineto{\pgfqpoint{5.223401in}{0.784318in}}%
\pgfpathlineto{\pgfqpoint{5.236367in}{0.792135in}}%
\pgfpathlineto{\pgfqpoint{5.241057in}{0.788835in}}%
\pgfpathlineto{\pgfqpoint{5.254234in}{0.795535in}}%
\pgfpathlineto{\pgfqpoint{5.262631in}{0.790306in}}%
\pgfpathlineto{\pgfqpoint{5.269268in}{0.790617in}}%
\pgfpathlineto{\pgfqpoint{5.274524in}{0.796776in}}%
\pgfpathlineto{\pgfqpoint{5.282795in}{0.793711in}}%
\pgfpathlineto{\pgfqpoint{5.301416in}{0.779451in}}%
\pgfpathlineto{\pgfqpoint{5.315948in}{0.774957in}}%
\pgfpathlineto{\pgfqpoint{5.321775in}{0.769451in}}%
\pgfpathlineto{\pgfqpoint{5.329070in}{0.772449in}}%
\pgfpathlineto{\pgfqpoint{5.340123in}{0.770153in}}%
\pgfpathlineto{\pgfqpoint{5.340340in}{0.735108in}}%
\pgfpathlineto{\pgfqpoint{5.341262in}{0.667326in}}%
\pgfpathlineto{\pgfqpoint{5.348939in}{0.660813in}}%
\pgfpathlineto{\pgfqpoint{5.355092in}{0.648811in}}%
\pgfpathlineto{\pgfqpoint{5.352929in}{0.640781in}}%
\pgfpathlineto{\pgfqpoint{5.357761in}{0.637381in}}%
\pgfpathlineto{\pgfqpoint{5.361688in}{0.622574in}}%
\pgfpathlineto{\pgfqpoint{5.369209in}{0.614644in}}%
\pgfpathlineto{\pgfqpoint{5.370894in}{0.597808in}}%
\pgfpathlineto{\pgfqpoint{5.369586in}{0.590680in}}%
\pgfpathlineto{\pgfqpoint{5.359363in}{0.571815in}}%
\pgfpathlineto{\pgfqpoint{5.358187in}{0.558251in}}%
\pgfpathlineto{\pgfqpoint{5.361655in}{0.555420in}}%
\pgfpathlineto{\pgfqpoint{5.361956in}{0.543537in}}%
\pgfpathlineto{\pgfqpoint{5.358432in}{0.536068in}}%
\pgfpathlineto{\pgfqpoint{5.352950in}{0.533255in}}%
\pgfpathlineto{\pgfqpoint{5.347582in}{0.523496in}}%
\pgfpathlineto{\pgfqpoint{5.354424in}{0.514043in}}%
\pgfpathlineto{\pgfqpoint{5.341144in}{0.513857in}}%
\pgfpathlineto{\pgfqpoint{5.305636in}{0.497611in}}%
\pgfpathlineto{\pgfqpoint{5.312415in}{0.507333in}}%
\pgfpathlineto{\pgfqpoint{5.304206in}{0.512575in}}%
\pgfpathlineto{\pgfqpoint{5.302490in}{0.521488in}}%
\pgfpathlineto{\pgfqpoint{5.286468in}{0.505964in}}%
\pgfpathlineto{\pgfqpoint{5.293579in}{0.495354in}}%
\pgfpathlineto{\pgfqpoint{5.283435in}{0.481844in}}%
\pgfpathlineto{\pgfqpoint{5.277976in}{0.482126in}}%
\pgfpathlineto{\pgfqpoint{5.272841in}{0.467413in}}%
\pgfpathlineto{\pgfqpoint{5.256600in}{0.455840in}}%
\pgfpathlineto{\pgfqpoint{5.241427in}{0.451669in}}%
\pgfpathlineto{\pgfqpoint{5.214964in}{0.440819in}}%
\pgfpathlineto{\pgfqpoint{5.212222in}{0.446871in}}%
\pgfpathlineto{\pgfqpoint{5.195737in}{0.441300in}}%
\pgfpathlineto{\pgfqpoint{5.205876in}{0.431814in}}%
\pgfpathlineto{\pgfqpoint{5.189883in}{0.423573in}}%
\pgfpathlineto{\pgfqpoint{5.172633in}{0.411131in}}%
\pgfpathlineto{\pgfqpoint{5.168490in}{0.416915in}}%
\pgfpathlineto{\pgfqpoint{5.154373in}{0.408246in}}%
\pgfpathlineto{\pgfqpoint{5.168210in}{0.404954in}}%
\pgfpathlineto{\pgfqpoint{5.157777in}{0.391324in}}%
\pgfpathlineto{\pgfqpoint{5.152685in}{0.395380in}}%
\pgfpathlineto{\pgfqpoint{5.145845in}{0.388865in}}%
\pgfpathlineto{\pgfqpoint{5.154369in}{0.383174in}}%
\pgfpathlineto{\pgfqpoint{5.149329in}{0.374856in}}%
\pgfpathlineto{\pgfqpoint{5.143135in}{0.355164in}}%
\pgfpathlineto{\pgfqpoint{5.134200in}{0.336199in}}%
\pgfpathlineto{\pgfqpoint{5.138211in}{0.323767in}}%
\pgfpathlineto{\pgfqpoint{5.145029in}{0.294458in}}%
\pgfpathlineto{\pgfqpoint{5.145643in}{0.282596in}}%
\pgfpathlineto{\pgfqpoint{5.156178in}{0.267038in}}%
\pgfpathlineto{\pgfqpoint{5.148054in}{0.267947in}}%
\pgfpathlineto{\pgfqpoint{5.140164in}{0.260098in}}%
\pgfpathlineto{\pgfqpoint{5.130923in}{0.266267in}}%
\pgfpathlineto{\pgfqpoint{5.127605in}{0.272421in}}%
\pgfpathlineto{\pgfqpoint{5.114453in}{0.275284in}}%
\pgfpathlineto{\pgfqpoint{5.094363in}{0.275635in}}%
\pgfpathlineto{\pgfqpoint{5.079561in}{0.287289in}}%
\pgfpathlineto{\pgfqpoint{5.066120in}{0.289236in}}%
\pgfpathlineto{\pgfqpoint{5.057966in}{0.298497in}}%
\pgfpathlineto{\pgfqpoint{5.040889in}{0.302225in}}%
\pgfpathlineto{\pgfqpoint{5.031552in}{0.332004in}}%
\pgfpathlineto{\pgfqpoint{5.022004in}{0.343934in}}%
\pgfpathlineto{\pgfqpoint{5.023626in}{0.355270in}}%
\pgfpathlineto{\pgfqpoint{5.017706in}{0.363554in}}%
\pgfpathlineto{\pgfqpoint{5.021421in}{0.374877in}}%
\pgfpathlineto{\pgfqpoint{5.018356in}{0.383175in}}%
\pgfpathlineto{\pgfqpoint{5.008762in}{0.386934in}}%
\pgfpathlineto{\pgfqpoint{4.999791in}{0.396502in}}%
\pgfpathlineto{\pgfqpoint{4.996522in}{0.409312in}}%
\pgfpathlineto{\pgfqpoint{4.988033in}{0.420954in}}%
\pgfpathlineto{\pgfqpoint{4.976735in}{0.430012in}}%
\pgfpathlineto{\pgfqpoint{4.966558in}{0.455996in}}%
\pgfpathlineto{\pgfqpoint{4.958340in}{0.484413in}}%
\pgfpathlineto{\pgfqpoint{4.951597in}{0.495650in}}%
\pgfpathlineto{\pgfqpoint{4.939908in}{0.505109in}}%
\pgfpathlineto{\pgfqpoint{4.936987in}{0.511975in}}%
\pgfpathlineto{\pgfqpoint{4.926042in}{0.516238in}}%
\pgfpathlineto{\pgfqpoint{4.915239in}{0.534431in}}%
\pgfpathlineto{\pgfqpoint{4.905374in}{0.533035in}}%
\pgfpathlineto{\pgfqpoint{4.895356in}{0.537579in}}%
\pgfpathlineto{\pgfqpoint{4.881156in}{0.536700in}}%
\pgfpathlineto{\pgfqpoint{4.866731in}{0.544199in}}%
\pgfpathlineto{\pgfqpoint{4.862663in}{0.537067in}}%
\pgfpathlineto{\pgfqpoint{4.845804in}{0.536889in}}%
\pgfpathlineto{\pgfqpoint{4.837222in}{0.523370in}}%
\pgfpathlineto{\pgfqpoint{4.829757in}{0.506631in}}%
\pgfpathlineto{\pgfqpoint{4.813883in}{0.488663in}}%
\pgfpathlineto{\pgfqpoint{4.795899in}{0.496548in}}%
\pgfpathlineto{\pgfqpoint{4.793347in}{0.501760in}}%
\pgfpathlineto{\pgfqpoint{4.782418in}{0.505734in}}%
\pgfpathlineto{\pgfqpoint{4.779061in}{0.511171in}}%
\pgfpathlineto{\pgfqpoint{4.764552in}{0.516660in}}%
\pgfpathlineto{\pgfqpoint{4.756436in}{0.527879in}}%
\pgfpathlineto{\pgfqpoint{4.746957in}{0.533292in}}%
\pgfpathlineto{\pgfqpoint{4.738802in}{0.542738in}}%
\pgfpathlineto{\pgfqpoint{4.732448in}{0.558735in}}%
\pgfpathlineto{\pgfqpoint{4.733159in}{0.580595in}}%
\pgfpathlineto{\pgfqpoint{4.725706in}{0.591641in}}%
\pgfpathlineto{\pgfqpoint{4.724848in}{0.603605in}}%
\pgfpathlineto{\pgfqpoint{4.708312in}{0.621518in}}%
\pgfpathlineto{\pgfqpoint{4.698690in}{0.625342in}}%
\pgfpathlineto{\pgfqpoint{4.688479in}{0.642057in}}%
\pgfpathlineto{\pgfqpoint{4.679778in}{0.648698in}}%
\pgfpathlineto{\pgfqpoint{4.668723in}{0.664869in}}%
\pgfpathlineto{\pgfqpoint{4.657411in}{0.671867in}}%
\pgfpathlineto{\pgfqpoint{4.650003in}{0.689800in}}%
\pgfpathlineto{\pgfqpoint{4.641758in}{0.694333in}}%
\pgfpathclose%
\pgfusepath{fill}%
\end{pgfscope}%
\begin{pgfscope}%
\pgfpathrectangle{\pgfqpoint{3.625000in}{0.100000in}}{\pgfqpoint{2.989028in}{1.913466in}}%
\pgfusepath{clip}%
\pgfsetbuttcap%
\pgfsetmiterjoin%
\definecolor{currentfill}{rgb}{0.400000,0.760784,0.647059}%
\pgfsetfillcolor{currentfill}%
\pgfsetlinewidth{0.000000pt}%
\definecolor{currentstroke}{rgb}{0.000000,0.000000,0.000000}%
\pgfsetstrokecolor{currentstroke}%
\pgfsetstrokeopacity{0.000000}%
\pgfsetdash{}{0pt}%
\pgfpathmoveto{\pgfqpoint{4.555137in}{1.052362in}}%
\pgfpathlineto{\pgfqpoint{4.636744in}{1.040510in}}%
\pgfpathlineto{\pgfqpoint{4.668359in}{1.035639in}}%
\pgfpathlineto{\pgfqpoint{4.757477in}{1.025040in}}%
\pgfpathlineto{\pgfqpoint{4.817460in}{1.018783in}}%
\pgfpathlineto{\pgfqpoint{4.869368in}{1.014183in}}%
\pgfpathlineto{\pgfqpoint{4.866469in}{0.981143in}}%
\pgfpathlineto{\pgfqpoint{4.864896in}{0.981231in}}%
\pgfpathlineto{\pgfqpoint{4.860142in}{0.924525in}}%
\pgfpathlineto{\pgfqpoint{4.855041in}{0.867115in}}%
\pgfpathlineto{\pgfqpoint{4.849140in}{0.807015in}}%
\pgfpathlineto{\pgfqpoint{4.840044in}{0.721225in}}%
\pgfpathlineto{\pgfqpoint{4.837135in}{0.687496in}}%
\pgfpathlineto{\pgfqpoint{4.783660in}{0.692892in}}%
\pgfpathlineto{\pgfqpoint{4.737359in}{0.697442in}}%
\pgfpathlineto{\pgfqpoint{4.673324in}{0.704715in}}%
\pgfpathlineto{\pgfqpoint{4.638886in}{0.708987in}}%
\pgfpathlineto{\pgfqpoint{4.636859in}{0.702205in}}%
\pgfpathlineto{\pgfqpoint{4.641758in}{0.694333in}}%
\pgfpathlineto{\pgfqpoint{4.600370in}{0.699708in}}%
\pgfpathlineto{\pgfqpoint{4.549306in}{0.707037in}}%
\pgfpathlineto{\pgfqpoint{4.544659in}{0.678155in}}%
\pgfpathlineto{\pgfqpoint{4.498080in}{0.685166in}}%
\pgfpathlineto{\pgfqpoint{4.509025in}{0.756815in}}%
\pgfpathlineto{\pgfqpoint{4.522573in}{0.844052in}}%
\pgfpathlineto{\pgfqpoint{4.534619in}{0.920049in}}%
\pgfpathlineto{\pgfqpoint{4.545079in}{0.987601in}}%
\pgfpathlineto{\pgfqpoint{4.555137in}{1.052362in}}%
\pgfpathclose%
\pgfusepath{fill}%
\end{pgfscope}%
\begin{pgfscope}%
\pgfpathrectangle{\pgfqpoint{3.625000in}{0.100000in}}{\pgfqpoint{2.989028in}{1.913466in}}%
\pgfusepath{clip}%
\pgfsetbuttcap%
\pgfsetmiterjoin%
\definecolor{currentfill}{rgb}{0.756786,0.901730,0.625606}%
\pgfsetfillcolor{currentfill}%
\pgfsetlinewidth{0.000000pt}%
\definecolor{currentstroke}{rgb}{0.000000,0.000000,0.000000}%
\pgfsetstrokecolor{currentstroke}%
\pgfsetstrokeopacity{0.000000}%
\pgfsetdash{}{0pt}%
\pgfpathmoveto{\pgfqpoint{5.852687in}{0.632577in}}%
\pgfpathlineto{\pgfqpoint{5.774834in}{0.623930in}}%
\pgfpathlineto{\pgfqpoint{5.706207in}{0.618602in}}%
\pgfpathlineto{\pgfqpoint{5.705349in}{0.610192in}}%
\pgfpathlineto{\pgfqpoint{5.711652in}{0.602188in}}%
\pgfpathlineto{\pgfqpoint{5.719346in}{0.597485in}}%
\pgfpathlineto{\pgfqpoint{5.719230in}{0.585103in}}%
\pgfpathlineto{\pgfqpoint{5.717191in}{0.576831in}}%
\pgfpathlineto{\pgfqpoint{5.710400in}{0.570869in}}%
\pgfpathlineto{\pgfqpoint{5.700936in}{0.571500in}}%
\pgfpathlineto{\pgfqpoint{5.691987in}{0.578887in}}%
\pgfpathlineto{\pgfqpoint{5.690392in}{0.592032in}}%
\pgfpathlineto{\pgfqpoint{5.683727in}{0.599677in}}%
\pgfpathlineto{\pgfqpoint{5.679190in}{0.572315in}}%
\pgfpathlineto{\pgfqpoint{5.663775in}{0.574915in}}%
\pgfpathlineto{\pgfqpoint{5.658415in}{0.622761in}}%
\pgfpathlineto{\pgfqpoint{5.652610in}{0.673185in}}%
\pgfpathlineto{\pgfqpoint{5.654766in}{0.745935in}}%
\pgfpathlineto{\pgfqpoint{5.656104in}{0.795857in}}%
\pgfpathlineto{\pgfqpoint{5.658950in}{0.872025in}}%
\pgfpathlineto{\pgfqpoint{5.652858in}{0.878728in}}%
\pgfpathlineto{\pgfqpoint{5.663945in}{0.880335in}}%
\pgfpathlineto{\pgfqpoint{5.728533in}{0.884512in}}%
\pgfpathlineto{\pgfqpoint{5.791375in}{0.890163in}}%
\pgfpathlineto{\pgfqpoint{5.807908in}{0.832251in}}%
\pgfpathlineto{\pgfqpoint{5.819184in}{0.789743in}}%
\pgfpathlineto{\pgfqpoint{5.829291in}{0.754156in}}%
\pgfpathlineto{\pgfqpoint{5.835136in}{0.739826in}}%
\pgfpathlineto{\pgfqpoint{5.844340in}{0.726491in}}%
\pgfpathlineto{\pgfqpoint{5.842853in}{0.719705in}}%
\pgfpathlineto{\pgfqpoint{5.849428in}{0.716398in}}%
\pgfpathlineto{\pgfqpoint{5.841659in}{0.706111in}}%
\pgfpathlineto{\pgfqpoint{5.839034in}{0.687792in}}%
\pgfpathlineto{\pgfqpoint{5.846628in}{0.666365in}}%
\pgfpathlineto{\pgfqpoint{5.845573in}{0.644805in}}%
\pgfpathlineto{\pgfqpoint{5.852687in}{0.632577in}}%
\pgfpathclose%
\pgfusepath{fill}%
\end{pgfscope}%
\begin{pgfscope}%
\pgfpathrectangle{\pgfqpoint{3.625000in}{0.100000in}}{\pgfqpoint{2.989028in}{1.913466in}}%
\pgfusepath{clip}%
\pgfsetbuttcap%
\pgfsetmiterjoin%
\definecolor{currentfill}{rgb}{0.944252,0.977701,0.662053}%
\pgfsetfillcolor{currentfill}%
\pgfsetlinewidth{0.000000pt}%
\definecolor{currentstroke}{rgb}{0.000000,0.000000,0.000000}%
\pgfsetstrokecolor{currentstroke}%
\pgfsetstrokeopacity{0.000000}%
\pgfsetdash{}{0pt}%
\pgfpathmoveto{\pgfqpoint{5.663775in}{0.574915in}}%
\pgfpathlineto{\pgfqpoint{5.647957in}{0.570389in}}%
\pgfpathlineto{\pgfqpoint{5.636166in}{0.573295in}}%
\pgfpathlineto{\pgfqpoint{5.614268in}{0.566330in}}%
\pgfpathlineto{\pgfqpoint{5.597753in}{0.557360in}}%
\pgfpathlineto{\pgfqpoint{5.590977in}{0.573569in}}%
\pgfpathlineto{\pgfqpoint{5.583992in}{0.580363in}}%
\pgfpathlineto{\pgfqpoint{5.580622in}{0.588493in}}%
\pgfpathlineto{\pgfqpoint{5.585562in}{0.610498in}}%
\pgfpathlineto{\pgfqpoint{5.538761in}{0.607617in}}%
\pgfpathlineto{\pgfqpoint{5.478073in}{0.604986in}}%
\pgfpathlineto{\pgfqpoint{5.481684in}{0.610463in}}%
\pgfpathlineto{\pgfqpoint{5.477221in}{0.620808in}}%
\pgfpathlineto{\pgfqpoint{5.481668in}{0.622916in}}%
\pgfpathlineto{\pgfqpoint{5.480681in}{0.632856in}}%
\pgfpathlineto{\pgfqpoint{5.488432in}{0.642495in}}%
\pgfpathlineto{\pgfqpoint{5.492673in}{0.661207in}}%
\pgfpathlineto{\pgfqpoint{5.506859in}{0.673535in}}%
\pgfpathlineto{\pgfqpoint{5.501601in}{0.685508in}}%
\pgfpathlineto{\pgfqpoint{5.511660in}{0.692400in}}%
\pgfpathlineto{\pgfqpoint{5.502983in}{0.704877in}}%
\pgfpathlineto{\pgfqpoint{5.496703in}{0.733439in}}%
\pgfpathlineto{\pgfqpoint{5.499085in}{0.738625in}}%
\pgfpathlineto{\pgfqpoint{5.502845in}{0.748573in}}%
\pgfpathlineto{\pgfqpoint{5.499345in}{0.759012in}}%
\pgfpathlineto{\pgfqpoint{5.501057in}{0.763762in}}%
\pgfpathlineto{\pgfqpoint{5.494899in}{0.781706in}}%
\pgfpathlineto{\pgfqpoint{5.513077in}{0.813926in}}%
\pgfpathlineto{\pgfqpoint{5.513922in}{0.822501in}}%
\pgfpathlineto{\pgfqpoint{5.522678in}{0.828737in}}%
\pgfpathlineto{\pgfqpoint{5.532021in}{0.849323in}}%
\pgfpathlineto{\pgfqpoint{5.531577in}{0.857690in}}%
\pgfpathlineto{\pgfqpoint{5.543214in}{0.866243in}}%
\pgfpathlineto{\pgfqpoint{5.539584in}{0.871310in}}%
\pgfpathlineto{\pgfqpoint{5.609319in}{0.875379in}}%
\pgfpathlineto{\pgfqpoint{5.652858in}{0.878728in}}%
\pgfpathlineto{\pgfqpoint{5.658950in}{0.872025in}}%
\pgfpathlineto{\pgfqpoint{5.656104in}{0.795857in}}%
\pgfpathlineto{\pgfqpoint{5.654766in}{0.745935in}}%
\pgfpathlineto{\pgfqpoint{5.652610in}{0.673185in}}%
\pgfpathlineto{\pgfqpoint{5.658415in}{0.622761in}}%
\pgfpathlineto{\pgfqpoint{5.663775in}{0.574915in}}%
\pgfpathclose%
\pgfusepath{fill}%
\end{pgfscope}%
\begin{pgfscope}%
\pgfpathrectangle{\pgfqpoint{3.625000in}{0.100000in}}{\pgfqpoint{2.989028in}{1.913466in}}%
\pgfusepath{clip}%
\pgfsetbuttcap%
\pgfsetmiterjoin%
\definecolor{currentfill}{rgb}{0.990388,0.996155,0.734025}%
\pgfsetfillcolor{currentfill}%
\pgfsetlinewidth{0.000000pt}%
\definecolor{currentstroke}{rgb}{0.000000,0.000000,0.000000}%
\pgfsetstrokecolor{currentstroke}%
\pgfsetstrokeopacity{0.000000}%
\pgfsetdash{}{0pt}%
\pgfpathmoveto{\pgfqpoint{5.791375in}{0.890163in}}%
\pgfpathlineto{\pgfqpoint{5.860069in}{0.897708in}}%
\pgfpathlineto{\pgfqpoint{5.901402in}{0.902801in}}%
\pgfpathlineto{\pgfqpoint{5.924943in}{0.906551in}}%
\pgfpathlineto{\pgfqpoint{5.915192in}{0.891171in}}%
\pgfpathlineto{\pgfqpoint{5.915198in}{0.883919in}}%
\pgfpathlineto{\pgfqpoint{5.925235in}{0.880009in}}%
\pgfpathlineto{\pgfqpoint{5.932061in}{0.873695in}}%
\pgfpathlineto{\pgfqpoint{5.942377in}{0.872915in}}%
\pgfpathlineto{\pgfqpoint{5.952019in}{0.855143in}}%
\pgfpathlineto{\pgfqpoint{5.962478in}{0.842609in}}%
\pgfpathlineto{\pgfqpoint{5.981053in}{0.832075in}}%
\pgfpathlineto{\pgfqpoint{5.986332in}{0.823674in}}%
\pgfpathlineto{\pgfqpoint{6.002896in}{0.814587in}}%
\pgfpathlineto{\pgfqpoint{6.004142in}{0.807998in}}%
\pgfpathlineto{\pgfqpoint{6.017579in}{0.794896in}}%
\pgfpathlineto{\pgfqpoint{6.027872in}{0.791165in}}%
\pgfpathlineto{\pgfqpoint{6.035161in}{0.778793in}}%
\pgfpathlineto{\pgfqpoint{6.038387in}{0.764886in}}%
\pgfpathlineto{\pgfqpoint{6.049078in}{0.759511in}}%
\pgfpathlineto{\pgfqpoint{6.057685in}{0.744563in}}%
\pgfpathlineto{\pgfqpoint{6.060460in}{0.733570in}}%
\pgfpathlineto{\pgfqpoint{6.073073in}{0.728890in}}%
\pgfpathlineto{\pgfqpoint{6.062490in}{0.708626in}}%
\pgfpathlineto{\pgfqpoint{6.058507in}{0.696651in}}%
\pgfpathlineto{\pgfqpoint{6.061105in}{0.691037in}}%
\pgfpathlineto{\pgfqpoint{6.056568in}{0.681708in}}%
\pgfpathlineto{\pgfqpoint{6.054639in}{0.668810in}}%
\pgfpathlineto{\pgfqpoint{6.046570in}{0.666408in}}%
\pgfpathlineto{\pgfqpoint{6.050831in}{0.654599in}}%
\pgfpathlineto{\pgfqpoint{6.050559in}{0.639617in}}%
\pgfpathlineto{\pgfqpoint{6.036217in}{0.640178in}}%
\pgfpathlineto{\pgfqpoint{6.026225in}{0.642912in}}%
\pgfpathlineto{\pgfqpoint{6.021902in}{0.637830in}}%
\pgfpathlineto{\pgfqpoint{6.025128in}{0.625826in}}%
\pgfpathlineto{\pgfqpoint{6.024427in}{0.611870in}}%
\pgfpathlineto{\pgfqpoint{6.018123in}{0.610802in}}%
\pgfpathlineto{\pgfqpoint{6.013024in}{0.623836in}}%
\pgfpathlineto{\pgfqpoint{5.961103in}{0.620376in}}%
\pgfpathlineto{\pgfqpoint{5.895635in}{0.616804in}}%
\pgfpathlineto{\pgfqpoint{5.862617in}{0.614480in}}%
\pgfpathlineto{\pgfqpoint{5.852687in}{0.632577in}}%
\pgfpathlineto{\pgfqpoint{5.845573in}{0.644805in}}%
\pgfpathlineto{\pgfqpoint{5.846628in}{0.666365in}}%
\pgfpathlineto{\pgfqpoint{5.839034in}{0.687792in}}%
\pgfpathlineto{\pgfqpoint{5.841659in}{0.706111in}}%
\pgfpathlineto{\pgfqpoint{5.849428in}{0.716398in}}%
\pgfpathlineto{\pgfqpoint{5.842853in}{0.719705in}}%
\pgfpathlineto{\pgfqpoint{5.844340in}{0.726491in}}%
\pgfpathlineto{\pgfqpoint{5.835136in}{0.739826in}}%
\pgfpathlineto{\pgfqpoint{5.829291in}{0.754156in}}%
\pgfpathlineto{\pgfqpoint{5.819184in}{0.789743in}}%
\pgfpathlineto{\pgfqpoint{5.807908in}{0.832251in}}%
\pgfpathlineto{\pgfqpoint{5.791375in}{0.890163in}}%
\pgfpathclose%
\pgfusepath{fill}%
\end{pgfscope}%
\begin{pgfscope}%
\pgfpathrectangle{\pgfqpoint{3.625000in}{0.100000in}}{\pgfqpoint{2.989028in}{1.913466in}}%
\pgfusepath{clip}%
\pgfsetbuttcap%
\pgfsetmiterjoin%
\definecolor{currentfill}{rgb}{0.622837,0.847982,0.643829}%
\pgfsetfillcolor{currentfill}%
\pgfsetlinewidth{0.000000pt}%
\definecolor{currentstroke}{rgb}{0.000000,0.000000,0.000000}%
\pgfsetstrokecolor{currentstroke}%
\pgfsetstrokeopacity{0.000000}%
\pgfsetdash{}{0pt}%
\pgfpathmoveto{\pgfqpoint{5.924943in}{0.906551in}}%
\pgfpathlineto{\pgfqpoint{5.952183in}{0.920102in}}%
\pgfpathlineto{\pgfqpoint{5.967269in}{0.925218in}}%
\pgfpathlineto{\pgfqpoint{6.032904in}{0.932038in}}%
\pgfpathlineto{\pgfqpoint{6.039855in}{0.929779in}}%
\pgfpathlineto{\pgfqpoint{6.049037in}{0.920550in}}%
\pgfpathlineto{\pgfqpoint{6.049543in}{0.912354in}}%
\pgfpathlineto{\pgfqpoint{6.109952in}{0.921230in}}%
\pgfpathlineto{\pgfqpoint{6.178548in}{0.872057in}}%
\pgfpathlineto{\pgfqpoint{6.165684in}{0.858649in}}%
\pgfpathlineto{\pgfqpoint{6.148033in}{0.827511in}}%
\pgfpathlineto{\pgfqpoint{6.153057in}{0.820819in}}%
\pgfpathlineto{\pgfqpoint{6.143641in}{0.807830in}}%
\pgfpathlineto{\pgfqpoint{6.134275in}{0.806359in}}%
\pgfpathlineto{\pgfqpoint{6.134240in}{0.798554in}}%
\pgfpathlineto{\pgfqpoint{6.127464in}{0.790386in}}%
\pgfpathlineto{\pgfqpoint{6.119032in}{0.788730in}}%
\pgfpathlineto{\pgfqpoint{6.120872in}{0.781473in}}%
\pgfpathlineto{\pgfqpoint{6.116167in}{0.775902in}}%
\pgfpathlineto{\pgfqpoint{6.104911in}{0.771089in}}%
\pgfpathlineto{\pgfqpoint{6.091192in}{0.760141in}}%
\pgfpathlineto{\pgfqpoint{6.093782in}{0.753058in}}%
\pgfpathlineto{\pgfqpoint{6.085148in}{0.748611in}}%
\pgfpathlineto{\pgfqpoint{6.077866in}{0.753287in}}%
\pgfpathlineto{\pgfqpoint{6.072540in}{0.732958in}}%
\pgfpathlineto{\pgfqpoint{6.060460in}{0.733570in}}%
\pgfpathlineto{\pgfqpoint{6.057685in}{0.744563in}}%
\pgfpathlineto{\pgfqpoint{6.049078in}{0.759511in}}%
\pgfpathlineto{\pgfqpoint{6.038387in}{0.764886in}}%
\pgfpathlineto{\pgfqpoint{6.035161in}{0.778793in}}%
\pgfpathlineto{\pgfqpoint{6.027872in}{0.791165in}}%
\pgfpathlineto{\pgfqpoint{6.017579in}{0.794896in}}%
\pgfpathlineto{\pgfqpoint{6.004142in}{0.807998in}}%
\pgfpathlineto{\pgfqpoint{6.002896in}{0.814587in}}%
\pgfpathlineto{\pgfqpoint{5.986332in}{0.823674in}}%
\pgfpathlineto{\pgfqpoint{5.981053in}{0.832075in}}%
\pgfpathlineto{\pgfqpoint{5.962478in}{0.842609in}}%
\pgfpathlineto{\pgfqpoint{5.952019in}{0.855143in}}%
\pgfpathlineto{\pgfqpoint{5.942377in}{0.872915in}}%
\pgfpathlineto{\pgfqpoint{5.932061in}{0.873695in}}%
\pgfpathlineto{\pgfqpoint{5.925235in}{0.880009in}}%
\pgfpathlineto{\pgfqpoint{5.915198in}{0.883919in}}%
\pgfpathlineto{\pgfqpoint{5.915192in}{0.891171in}}%
\pgfpathlineto{\pgfqpoint{5.924943in}{0.906551in}}%
\pgfpathclose%
\pgfusepath{fill}%
\end{pgfscope}%
\begin{pgfscope}%
\pgfpathrectangle{\pgfqpoint{3.625000in}{0.100000in}}{\pgfqpoint{2.989028in}{1.913466in}}%
\pgfusepath{clip}%
\pgfsetbuttcap%
\pgfsetmiterjoin%
\definecolor{currentfill}{rgb}{0.537947,0.814764,0.645060}%
\pgfsetfillcolor{currentfill}%
\pgfsetlinewidth{0.000000pt}%
\definecolor{currentstroke}{rgb}{0.000000,0.000000,0.000000}%
\pgfsetstrokecolor{currentstroke}%
\pgfsetstrokeopacity{0.000000}%
\pgfsetdash{}{0pt}%
\pgfpathmoveto{\pgfqpoint{5.307462in}{0.962463in}}%
\pgfpathlineto{\pgfqpoint{5.347491in}{0.962824in}}%
\pgfpathlineto{\pgfqpoint{5.400448in}{0.963789in}}%
\pgfpathlineto{\pgfqpoint{5.490941in}{0.966662in}}%
\pgfpathlineto{\pgfqpoint{5.542700in}{0.969399in}}%
\pgfpathlineto{\pgfqpoint{5.548251in}{0.962527in}}%
\pgfpathlineto{\pgfqpoint{5.547904in}{0.955269in}}%
\pgfpathlineto{\pgfqpoint{5.535365in}{0.942748in}}%
\pgfpathlineto{\pgfqpoint{5.532334in}{0.935892in}}%
\pgfpathlineto{\pgfqpoint{5.567109in}{0.938464in}}%
\pgfpathlineto{\pgfqpoint{5.567090in}{0.925797in}}%
\pgfpathlineto{\pgfqpoint{5.555934in}{0.920384in}}%
\pgfpathlineto{\pgfqpoint{5.556092in}{0.911777in}}%
\pgfpathlineto{\pgfqpoint{5.551902in}{0.905590in}}%
\pgfpathlineto{\pgfqpoint{5.549159in}{0.892449in}}%
\pgfpathlineto{\pgfqpoint{5.551990in}{0.881608in}}%
\pgfpathlineto{\pgfqpoint{5.539584in}{0.871310in}}%
\pgfpathlineto{\pgfqpoint{5.543214in}{0.866243in}}%
\pgfpathlineto{\pgfqpoint{5.531577in}{0.857690in}}%
\pgfpathlineto{\pgfqpoint{5.532021in}{0.849323in}}%
\pgfpathlineto{\pgfqpoint{5.522678in}{0.828737in}}%
\pgfpathlineto{\pgfqpoint{5.513922in}{0.822501in}}%
\pgfpathlineto{\pgfqpoint{5.513077in}{0.813926in}}%
\pgfpathlineto{\pgfqpoint{5.494899in}{0.781706in}}%
\pgfpathlineto{\pgfqpoint{5.501057in}{0.763762in}}%
\pgfpathlineto{\pgfqpoint{5.499345in}{0.759012in}}%
\pgfpathlineto{\pgfqpoint{5.502845in}{0.748573in}}%
\pgfpathlineto{\pgfqpoint{5.499085in}{0.738625in}}%
\pgfpathlineto{\pgfqpoint{5.449378in}{0.736573in}}%
\pgfpathlineto{\pgfqpoint{5.384848in}{0.735510in}}%
\pgfpathlineto{\pgfqpoint{5.340340in}{0.735108in}}%
\pgfpathlineto{\pgfqpoint{5.340123in}{0.770153in}}%
\pgfpathlineto{\pgfqpoint{5.329070in}{0.772449in}}%
\pgfpathlineto{\pgfqpoint{5.321775in}{0.769451in}}%
\pgfpathlineto{\pgfqpoint{5.315948in}{0.774957in}}%
\pgfpathlineto{\pgfqpoint{5.316549in}{0.812079in}}%
\pgfpathlineto{\pgfqpoint{5.317853in}{0.891120in}}%
\pgfpathlineto{\pgfqpoint{5.311531in}{0.937404in}}%
\pgfpathlineto{\pgfqpoint{5.307462in}{0.962463in}}%
\pgfpathclose%
\pgfusepath{fill}%
\end{pgfscope}%
\begin{pgfscope}%
\pgfpathrectangle{\pgfqpoint{3.625000in}{0.100000in}}{\pgfqpoint{2.989028in}{1.913466in}}%
\pgfusepath{clip}%
\pgfsetbuttcap%
\pgfsetmiterjoin%
\definecolor{currentfill}{rgb}{0.738639,0.894348,0.629296}%
\pgfsetfillcolor{currentfill}%
\pgfsetlinewidth{0.000000pt}%
\definecolor{currentstroke}{rgb}{0.000000,0.000000,0.000000}%
\pgfsetstrokecolor{currentstroke}%
\pgfsetstrokeopacity{0.000000}%
\pgfsetdash{}{0pt}%
\pgfpathmoveto{\pgfqpoint{5.340340in}{0.735108in}}%
\pgfpathlineto{\pgfqpoint{5.384848in}{0.735510in}}%
\pgfpathlineto{\pgfqpoint{5.449378in}{0.736573in}}%
\pgfpathlineto{\pgfqpoint{5.499085in}{0.738625in}}%
\pgfpathlineto{\pgfqpoint{5.496703in}{0.733439in}}%
\pgfpathlineto{\pgfqpoint{5.502983in}{0.704877in}}%
\pgfpathlineto{\pgfqpoint{5.511660in}{0.692400in}}%
\pgfpathlineto{\pgfqpoint{5.501601in}{0.685508in}}%
\pgfpathlineto{\pgfqpoint{5.506859in}{0.673535in}}%
\pgfpathlineto{\pgfqpoint{5.492673in}{0.661207in}}%
\pgfpathlineto{\pgfqpoint{5.488432in}{0.642495in}}%
\pgfpathlineto{\pgfqpoint{5.480681in}{0.632856in}}%
\pgfpathlineto{\pgfqpoint{5.481668in}{0.622916in}}%
\pgfpathlineto{\pgfqpoint{5.477221in}{0.620808in}}%
\pgfpathlineto{\pgfqpoint{5.481684in}{0.610463in}}%
\pgfpathlineto{\pgfqpoint{5.478073in}{0.604986in}}%
\pgfpathlineto{\pgfqpoint{5.538761in}{0.607617in}}%
\pgfpathlineto{\pgfqpoint{5.585562in}{0.610498in}}%
\pgfpathlineto{\pgfqpoint{5.580622in}{0.588493in}}%
\pgfpathlineto{\pgfqpoint{5.583992in}{0.580363in}}%
\pgfpathlineto{\pgfqpoint{5.590977in}{0.573569in}}%
\pgfpathlineto{\pgfqpoint{5.597753in}{0.557360in}}%
\pgfpathlineto{\pgfqpoint{5.576350in}{0.561094in}}%
\pgfpathlineto{\pgfqpoint{5.568455in}{0.567234in}}%
\pgfpathlineto{\pgfqpoint{5.559048in}{0.567526in}}%
\pgfpathlineto{\pgfqpoint{5.549158in}{0.554082in}}%
\pgfpathlineto{\pgfqpoint{5.551129in}{0.547960in}}%
\pgfpathlineto{\pgfqpoint{5.582724in}{0.544257in}}%
\pgfpathlineto{\pgfqpoint{5.590995in}{0.537220in}}%
\pgfpathlineto{\pgfqpoint{5.604454in}{0.533600in}}%
\pgfpathlineto{\pgfqpoint{5.598576in}{0.525268in}}%
\pgfpathlineto{\pgfqpoint{5.588672in}{0.518003in}}%
\pgfpathlineto{\pgfqpoint{5.614813in}{0.501664in}}%
\pgfpathlineto{\pgfqpoint{5.626990in}{0.499111in}}%
\pgfpathlineto{\pgfqpoint{5.632054in}{0.485755in}}%
\pgfpathlineto{\pgfqpoint{5.620589in}{0.483298in}}%
\pgfpathlineto{\pgfqpoint{5.599041in}{0.500096in}}%
\pgfpathlineto{\pgfqpoint{5.590616in}{0.502430in}}%
\pgfpathlineto{\pgfqpoint{5.586539in}{0.509066in}}%
\pgfpathlineto{\pgfqpoint{5.574043in}{0.506343in}}%
\pgfpathlineto{\pgfqpoint{5.570130in}{0.497792in}}%
\pgfpathlineto{\pgfqpoint{5.572608in}{0.488264in}}%
\pgfpathlineto{\pgfqpoint{5.564195in}{0.482642in}}%
\pgfpathlineto{\pgfqpoint{5.556517in}{0.496493in}}%
\pgfpathlineto{\pgfqpoint{5.543069in}{0.492363in}}%
\pgfpathlineto{\pgfqpoint{5.538030in}{0.484098in}}%
\pgfpathlineto{\pgfqpoint{5.528479in}{0.486449in}}%
\pgfpathlineto{\pgfqpoint{5.522373in}{0.496886in}}%
\pgfpathlineto{\pgfqpoint{5.506173in}{0.500333in}}%
\pgfpathlineto{\pgfqpoint{5.503115in}{0.505764in}}%
\pgfpathlineto{\pgfqpoint{5.486236in}{0.515225in}}%
\pgfpathlineto{\pgfqpoint{5.482047in}{0.523496in}}%
\pgfpathlineto{\pgfqpoint{5.467910in}{0.520116in}}%
\pgfpathlineto{\pgfqpoint{5.469705in}{0.527689in}}%
\pgfpathlineto{\pgfqpoint{5.452134in}{0.519920in}}%
\pgfpathlineto{\pgfqpoint{5.456854in}{0.511860in}}%
\pgfpathlineto{\pgfqpoint{5.443284in}{0.507095in}}%
\pgfpathlineto{\pgfqpoint{5.425311in}{0.509714in}}%
\pgfpathlineto{\pgfqpoint{5.388934in}{0.522174in}}%
\pgfpathlineto{\pgfqpoint{5.360864in}{0.519701in}}%
\pgfpathlineto{\pgfqpoint{5.358432in}{0.536068in}}%
\pgfpathlineto{\pgfqpoint{5.361956in}{0.543537in}}%
\pgfpathlineto{\pgfqpoint{5.361655in}{0.555420in}}%
\pgfpathlineto{\pgfqpoint{5.358187in}{0.558251in}}%
\pgfpathlineto{\pgfqpoint{5.359363in}{0.571815in}}%
\pgfpathlineto{\pgfqpoint{5.369586in}{0.590680in}}%
\pgfpathlineto{\pgfqpoint{5.370894in}{0.597808in}}%
\pgfpathlineto{\pgfqpoint{5.369209in}{0.614644in}}%
\pgfpathlineto{\pgfqpoint{5.361688in}{0.622574in}}%
\pgfpathlineto{\pgfqpoint{5.357761in}{0.637381in}}%
\pgfpathlineto{\pgfqpoint{5.352929in}{0.640781in}}%
\pgfpathlineto{\pgfqpoint{5.355092in}{0.648811in}}%
\pgfpathlineto{\pgfqpoint{5.348939in}{0.660813in}}%
\pgfpathlineto{\pgfqpoint{5.341262in}{0.667326in}}%
\pgfpathlineto{\pgfqpoint{5.340340in}{0.735108in}}%
\pgfpathclose%
\pgfusepath{fill}%
\end{pgfscope}%
\begin{pgfscope}%
\pgfpathrectangle{\pgfqpoint{3.625000in}{0.100000in}}{\pgfqpoint{2.989028in}{1.913466in}}%
\pgfusepath{clip}%
\pgfsetbuttcap%
\pgfsetmiterjoin%
\definecolor{currentfill}{rgb}{0.738639,0.894348,0.629296}%
\pgfsetfillcolor{currentfill}%
\pgfsetlinewidth{0.000000pt}%
\definecolor{currentstroke}{rgb}{0.000000,0.000000,0.000000}%
\pgfsetstrokecolor{currentstroke}%
\pgfsetstrokeopacity{0.000000}%
\pgfsetdash{}{0pt}%
\pgfpathmoveto{\pgfqpoint{5.459378in}{0.511243in}}%
\pgfpathlineto{\pgfqpoint{5.465817in}{0.515073in}}%
\pgfpathlineto{\pgfqpoint{5.473633in}{0.510545in}}%
\pgfpathlineto{\pgfqpoint{5.469274in}{0.504314in}}%
\pgfpathlineto{\pgfqpoint{5.459378in}{0.511243in}}%
\pgfpathclose%
\pgfusepath{fill}%
\end{pgfscope}%
\begin{pgfscope}%
\pgfpathrectangle{\pgfqpoint{3.625000in}{0.100000in}}{\pgfqpoint{2.989028in}{1.913466in}}%
\pgfusepath{clip}%
\pgfsetbuttcap%
\pgfsetmiterjoin%
\definecolor{currentfill}{rgb}{0.368012,0.725106,0.661822}%
\pgfsetfillcolor{currentfill}%
\pgfsetlinewidth{0.000000pt}%
\definecolor{currentstroke}{rgb}{0.000000,0.000000,0.000000}%
\pgfsetstrokecolor{currentstroke}%
\pgfsetstrokeopacity{0.000000}%
\pgfsetdash{}{0pt}%
\pgfpathmoveto{\pgfqpoint{5.719230in}{0.585103in}}%
\pgfpathlineto{\pgfqpoint{5.719346in}{0.597485in}}%
\pgfpathlineto{\pgfqpoint{5.711652in}{0.602188in}}%
\pgfpathlineto{\pgfqpoint{5.705349in}{0.610192in}}%
\pgfpathlineto{\pgfqpoint{5.706207in}{0.618602in}}%
\pgfpathlineto{\pgfqpoint{5.774834in}{0.623930in}}%
\pgfpathlineto{\pgfqpoint{5.852687in}{0.632577in}}%
\pgfpathlineto{\pgfqpoint{5.862617in}{0.614480in}}%
\pgfpathlineto{\pgfqpoint{5.895635in}{0.616804in}}%
\pgfpathlineto{\pgfqpoint{5.961103in}{0.620376in}}%
\pgfpathlineto{\pgfqpoint{6.013024in}{0.623836in}}%
\pgfpathlineto{\pgfqpoint{6.018123in}{0.610802in}}%
\pgfpathlineto{\pgfqpoint{6.024427in}{0.611870in}}%
\pgfpathlineto{\pgfqpoint{6.025128in}{0.625826in}}%
\pgfpathlineto{\pgfqpoint{6.021902in}{0.637830in}}%
\pgfpathlineto{\pgfqpoint{6.026225in}{0.642912in}}%
\pgfpathlineto{\pgfqpoint{6.036217in}{0.640178in}}%
\pgfpathlineto{\pgfqpoint{6.050559in}{0.639617in}}%
\pgfpathlineto{\pgfqpoint{6.052764in}{0.628892in}}%
\pgfpathlineto{\pgfqpoint{6.057176in}{0.622764in}}%
\pgfpathlineto{\pgfqpoint{6.060621in}{0.609356in}}%
\pgfpathlineto{\pgfqpoint{6.071296in}{0.588607in}}%
\pgfpathlineto{\pgfqpoint{6.071349in}{0.582995in}}%
\pgfpathlineto{\pgfqpoint{6.087126in}{0.558631in}}%
\pgfpathlineto{\pgfqpoint{6.088652in}{0.553598in}}%
\pgfpathlineto{\pgfqpoint{6.112320in}{0.519188in}}%
\pgfpathlineto{\pgfqpoint{6.108570in}{0.518632in}}%
\pgfpathlineto{\pgfqpoint{6.118548in}{0.494158in}}%
\pgfpathlineto{\pgfqpoint{6.145959in}{0.451609in}}%
\pgfpathlineto{\pgfqpoint{6.161873in}{0.420325in}}%
\pgfpathlineto{\pgfqpoint{6.167192in}{0.414848in}}%
\pgfpathlineto{\pgfqpoint{6.176306in}{0.395211in}}%
\pgfpathlineto{\pgfqpoint{6.179468in}{0.363769in}}%
\pgfpathlineto{\pgfqpoint{6.180727in}{0.340258in}}%
\pgfpathlineto{\pgfqpoint{6.179207in}{0.325205in}}%
\pgfpathlineto{\pgfqpoint{6.174320in}{0.314444in}}%
\pgfpathlineto{\pgfqpoint{6.176590in}{0.300332in}}%
\pgfpathlineto{\pgfqpoint{6.171329in}{0.289163in}}%
\pgfpathlineto{\pgfqpoint{6.155707in}{0.280003in}}%
\pgfpathlineto{\pgfqpoint{6.145547in}{0.280672in}}%
\pgfpathlineto{\pgfqpoint{6.138949in}{0.275884in}}%
\pgfpathlineto{\pgfqpoint{6.137059in}{0.288668in}}%
\pgfpathlineto{\pgfqpoint{6.126137in}{0.292042in}}%
\pgfpathlineto{\pgfqpoint{6.116346in}{0.309880in}}%
\pgfpathlineto{\pgfqpoint{6.115241in}{0.318003in}}%
\pgfpathlineto{\pgfqpoint{6.097719in}{0.322936in}}%
\pgfpathlineto{\pgfqpoint{6.086428in}{0.321868in}}%
\pgfpathlineto{\pgfqpoint{6.080040in}{0.333654in}}%
\pgfpathlineto{\pgfqpoint{6.072725in}{0.354839in}}%
\pgfpathlineto{\pgfqpoint{6.062575in}{0.359133in}}%
\pgfpathlineto{\pgfqpoint{6.057072in}{0.371273in}}%
\pgfpathlineto{\pgfqpoint{6.043614in}{0.378428in}}%
\pgfpathlineto{\pgfqpoint{6.035835in}{0.387377in}}%
\pgfpathlineto{\pgfqpoint{6.022416in}{0.411708in}}%
\pgfpathlineto{\pgfqpoint{6.020838in}{0.428833in}}%
\pgfpathlineto{\pgfqpoint{6.028150in}{0.439789in}}%
\pgfpathlineto{\pgfqpoint{6.027419in}{0.447403in}}%
\pgfpathlineto{\pgfqpoint{6.018969in}{0.448217in}}%
\pgfpathlineto{\pgfqpoint{6.011909in}{0.453513in}}%
\pgfpathlineto{\pgfqpoint{6.008042in}{0.447046in}}%
\pgfpathlineto{\pgfqpoint{6.014818in}{0.441777in}}%
\pgfpathlineto{\pgfqpoint{6.009887in}{0.432302in}}%
\pgfpathlineto{\pgfqpoint{6.001908in}{0.440166in}}%
\pgfpathlineto{\pgfqpoint{6.002834in}{0.461974in}}%
\pgfpathlineto{\pgfqpoint{6.006682in}{0.479667in}}%
\pgfpathlineto{\pgfqpoint{6.004704in}{0.510026in}}%
\pgfpathlineto{\pgfqpoint{5.996751in}{0.517257in}}%
\pgfpathlineto{\pgfqpoint{5.992752in}{0.526536in}}%
\pgfpathlineto{\pgfqpoint{5.979052in}{0.526329in}}%
\pgfpathlineto{\pgfqpoint{5.965510in}{0.541597in}}%
\pgfpathlineto{\pgfqpoint{5.956430in}{0.546173in}}%
\pgfpathlineto{\pgfqpoint{5.953737in}{0.555842in}}%
\pgfpathlineto{\pgfqpoint{5.944856in}{0.559238in}}%
\pgfpathlineto{\pgfqpoint{5.937496in}{0.569913in}}%
\pgfpathlineto{\pgfqpoint{5.918052in}{0.578628in}}%
\pgfpathlineto{\pgfqpoint{5.902937in}{0.578850in}}%
\pgfpathlineto{\pgfqpoint{5.896356in}{0.575503in}}%
\pgfpathlineto{\pgfqpoint{5.897991in}{0.565056in}}%
\pgfpathlineto{\pgfqpoint{5.891130in}{0.565551in}}%
\pgfpathlineto{\pgfqpoint{5.870022in}{0.550943in}}%
\pgfpathlineto{\pgfqpoint{5.844668in}{0.545153in}}%
\pgfpathlineto{\pgfqpoint{5.844252in}{0.552317in}}%
\pgfpathlineto{\pgfqpoint{5.838633in}{0.559318in}}%
\pgfpathlineto{\pgfqpoint{5.823558in}{0.568984in}}%
\pgfpathlineto{\pgfqpoint{5.801918in}{0.578894in}}%
\pgfpathlineto{\pgfqpoint{5.778414in}{0.584141in}}%
\pgfpathlineto{\pgfqpoint{5.774002in}{0.591275in}}%
\pgfpathlineto{\pgfqpoint{5.765528in}{0.585325in}}%
\pgfpathlineto{\pgfqpoint{5.755323in}{0.584012in}}%
\pgfpathlineto{\pgfqpoint{5.719801in}{0.574653in}}%
\pgfpathlineto{\pgfqpoint{5.719230in}{0.585103in}}%
\pgfpathclose%
\pgfusepath{fill}%
\end{pgfscope}%
\begin{pgfscope}%
\pgfpathrectangle{\pgfqpoint{3.625000in}{0.100000in}}{\pgfqpoint{2.989028in}{1.913466in}}%
\pgfusepath{clip}%
\pgfsetbuttcap%
\pgfsetmiterjoin%
\definecolor{currentfill}{rgb}{0.368012,0.725106,0.661822}%
\pgfsetfillcolor{currentfill}%
\pgfsetlinewidth{0.000000pt}%
\definecolor{currentstroke}{rgb}{0.000000,0.000000,0.000000}%
\pgfsetstrokecolor{currentstroke}%
\pgfsetstrokeopacity{0.000000}%
\pgfsetdash{}{0pt}%
\pgfpathmoveto{\pgfqpoint{6.115769in}{0.519678in}}%
\pgfpathlineto{\pgfqpoint{6.124959in}{0.508987in}}%
\pgfpathlineto{\pgfqpoint{6.114344in}{0.508314in}}%
\pgfpathlineto{\pgfqpoint{6.115769in}{0.519678in}}%
\pgfpathclose%
\pgfusepath{fill}%
\end{pgfscope}%
\begin{pgfscope}%
\pgfpathrectangle{\pgfqpoint{3.625000in}{0.100000in}}{\pgfqpoint{2.989028in}{1.913466in}}%
\pgfusepath{clip}%
\pgfsetbuttcap%
\pgfsetmiterjoin%
\definecolor{currentfill}{rgb}{0.932718,0.973087,0.644060}%
\pgfsetfillcolor{currentfill}%
\pgfsetlinewidth{0.000000pt}%
\definecolor{currentstroke}{rgb}{0.000000,0.000000,0.000000}%
\pgfsetstrokecolor{currentstroke}%
\pgfsetstrokeopacity{0.000000}%
\pgfsetdash{}{0pt}%
\pgfpathmoveto{\pgfqpoint{5.574740in}{1.740361in}}%
\pgfpathlineto{\pgfqpoint{5.569753in}{1.730656in}}%
\pgfpathlineto{\pgfqpoint{5.557857in}{1.724992in}}%
\pgfpathlineto{\pgfqpoint{5.552705in}{1.717368in}}%
\pgfpathlineto{\pgfqpoint{5.546669in}{1.722859in}}%
\pgfpathlineto{\pgfqpoint{5.574740in}{1.740361in}}%
\pgfpathclose%
\pgfusepath{fill}%
\end{pgfscope}%
\begin{pgfscope}%
\pgfpathrectangle{\pgfqpoint{3.625000in}{0.100000in}}{\pgfqpoint{2.989028in}{1.913466in}}%
\pgfusepath{clip}%
\pgfsetbuttcap%
\pgfsetmiterjoin%
\definecolor{currentfill}{rgb}{0.932718,0.973087,0.644060}%
\pgfsetfillcolor{currentfill}%
\pgfsetlinewidth{0.000000pt}%
\definecolor{currentstroke}{rgb}{0.000000,0.000000,0.000000}%
\pgfsetstrokecolor{currentstroke}%
\pgfsetstrokeopacity{0.000000}%
\pgfsetdash{}{0pt}%
\pgfpathmoveto{\pgfqpoint{5.578784in}{1.681953in}}%
\pgfpathlineto{\pgfqpoint{5.590956in}{1.693308in}}%
\pgfpathlineto{\pgfqpoint{5.609730in}{1.696281in}}%
\pgfpathlineto{\pgfqpoint{5.604559in}{1.688384in}}%
\pgfpathlineto{\pgfqpoint{5.591688in}{1.676964in}}%
\pgfpathlineto{\pgfqpoint{5.584164in}{1.662306in}}%
\pgfpathlineto{\pgfqpoint{5.580977in}{1.670256in}}%
\pgfpathlineto{\pgfqpoint{5.575318in}{1.671398in}}%
\pgfpathlineto{\pgfqpoint{5.574750in}{1.678576in}}%
\pgfpathlineto{\pgfqpoint{5.578784in}{1.681953in}}%
\pgfpathclose%
\pgfusepath{fill}%
\end{pgfscope}%
\begin{pgfscope}%
\pgfpathrectangle{\pgfqpoint{3.625000in}{0.100000in}}{\pgfqpoint{2.989028in}{1.913466in}}%
\pgfusepath{clip}%
\pgfsetbuttcap%
\pgfsetmiterjoin%
\definecolor{currentfill}{rgb}{0.932718,0.973087,0.644060}%
\pgfsetfillcolor{currentfill}%
\pgfsetlinewidth{0.000000pt}%
\definecolor{currentstroke}{rgb}{0.000000,0.000000,0.000000}%
\pgfsetstrokecolor{currentstroke}%
\pgfsetstrokeopacity{0.000000}%
\pgfsetdash{}{0pt}%
\pgfpathmoveto{\pgfqpoint{5.627264in}{1.543360in}}%
\pgfpathlineto{\pgfqpoint{5.624022in}{1.546957in}}%
\pgfpathlineto{\pgfqpoint{5.627430in}{1.557084in}}%
\pgfpathlineto{\pgfqpoint{5.617314in}{1.557730in}}%
\pgfpathlineto{\pgfqpoint{5.619996in}{1.566462in}}%
\pgfpathlineto{\pgfqpoint{5.618328in}{1.580368in}}%
\pgfpathlineto{\pgfqpoint{5.609251in}{1.585189in}}%
\pgfpathlineto{\pgfqpoint{5.599752in}{1.594947in}}%
\pgfpathlineto{\pgfqpoint{5.585116in}{1.597803in}}%
\pgfpathlineto{\pgfqpoint{5.570871in}{1.597722in}}%
\pgfpathlineto{\pgfqpoint{5.556860in}{1.604646in}}%
\pgfpathlineto{\pgfqpoint{5.509989in}{1.614740in}}%
\pgfpathlineto{\pgfqpoint{5.504867in}{1.625436in}}%
\pgfpathlineto{\pgfqpoint{5.495731in}{1.629085in}}%
\pgfpathlineto{\pgfqpoint{5.512989in}{1.637274in}}%
\pgfpathlineto{\pgfqpoint{5.522738in}{1.647490in}}%
\pgfpathlineto{\pgfqpoint{5.540897in}{1.650260in}}%
\pgfpathlineto{\pgfqpoint{5.552071in}{1.660666in}}%
\pgfpathlineto{\pgfqpoint{5.557934in}{1.661083in}}%
\pgfpathlineto{\pgfqpoint{5.562408in}{1.668504in}}%
\pgfpathlineto{\pgfqpoint{5.573315in}{1.677295in}}%
\pgfpathlineto{\pgfqpoint{5.574265in}{1.671081in}}%
\pgfpathlineto{\pgfqpoint{5.579179in}{1.669798in}}%
\pgfpathlineto{\pgfqpoint{5.582877in}{1.662388in}}%
\pgfpathlineto{\pgfqpoint{5.583543in}{1.649753in}}%
\pgfpathlineto{\pgfqpoint{5.594656in}{1.657304in}}%
\pgfpathlineto{\pgfqpoint{5.607606in}{1.658880in}}%
\pgfpathlineto{\pgfqpoint{5.618660in}{1.654925in}}%
\pgfpathlineto{\pgfqpoint{5.633651in}{1.634360in}}%
\pgfpathlineto{\pgfqpoint{5.650023in}{1.637647in}}%
\pgfpathlineto{\pgfqpoint{5.656676in}{1.632143in}}%
\pgfpathlineto{\pgfqpoint{5.667376in}{1.631651in}}%
\pgfpathlineto{\pgfqpoint{5.674485in}{1.641522in}}%
\pgfpathlineto{\pgfqpoint{5.687974in}{1.650248in}}%
\pgfpathlineto{\pgfqpoint{5.700939in}{1.652980in}}%
\pgfpathlineto{\pgfqpoint{5.717026in}{1.653265in}}%
\pgfpathlineto{\pgfqpoint{5.728782in}{1.660005in}}%
\pgfpathlineto{\pgfqpoint{5.738376in}{1.656901in}}%
\pgfpathlineto{\pgfqpoint{5.740419in}{1.642645in}}%
\pgfpathlineto{\pgfqpoint{5.761002in}{1.640413in}}%
\pgfpathlineto{\pgfqpoint{5.772182in}{1.647086in}}%
\pgfpathlineto{\pgfqpoint{5.779932in}{1.632053in}}%
\pgfpathlineto{\pgfqpoint{5.794719in}{1.614670in}}%
\pgfpathlineto{\pgfqpoint{5.774022in}{1.614771in}}%
\pgfpathlineto{\pgfqpoint{5.767482in}{1.612622in}}%
\pgfpathlineto{\pgfqpoint{5.758519in}{1.615414in}}%
\pgfpathlineto{\pgfqpoint{5.757916in}{1.603380in}}%
\pgfpathlineto{\pgfqpoint{5.741640in}{1.612785in}}%
\pgfpathlineto{\pgfqpoint{5.720708in}{1.615659in}}%
\pgfpathlineto{\pgfqpoint{5.714946in}{1.606500in}}%
\pgfpathlineto{\pgfqpoint{5.687559in}{1.602046in}}%
\pgfpathlineto{\pgfqpoint{5.684409in}{1.594288in}}%
\pgfpathlineto{\pgfqpoint{5.665351in}{1.591960in}}%
\pgfpathlineto{\pgfqpoint{5.659590in}{1.584073in}}%
\pgfpathlineto{\pgfqpoint{5.649509in}{1.581958in}}%
\pgfpathlineto{\pgfqpoint{5.641458in}{1.563256in}}%
\pgfpathlineto{\pgfqpoint{5.631264in}{1.545142in}}%
\pgfpathlineto{\pgfqpoint{5.627264in}{1.543360in}}%
\pgfpathclose%
\pgfusepath{fill}%
\end{pgfscope}%
\begin{pgfscope}%
\pgfpathrectangle{\pgfqpoint{3.625000in}{0.100000in}}{\pgfqpoint{2.989028in}{1.913466in}}%
\pgfusepath{clip}%
\pgfsetbuttcap%
\pgfsetmiterjoin%
\definecolor{currentfill}{rgb}{0.932718,0.973087,0.644060}%
\pgfsetfillcolor{currentfill}%
\pgfsetlinewidth{0.000000pt}%
\definecolor{currentstroke}{rgb}{0.000000,0.000000,0.000000}%
\pgfsetstrokecolor{currentstroke}%
\pgfsetstrokeopacity{0.000000}%
\pgfsetdash{}{0pt}%
\pgfpathmoveto{\pgfqpoint{5.685936in}{1.325898in}}%
\pgfpathlineto{\pgfqpoint{5.695655in}{1.336128in}}%
\pgfpathlineto{\pgfqpoint{5.700083in}{1.350960in}}%
\pgfpathlineto{\pgfqpoint{5.705349in}{1.359557in}}%
\pgfpathlineto{\pgfqpoint{5.708578in}{1.371256in}}%
\pgfpathlineto{\pgfqpoint{5.709581in}{1.394584in}}%
\pgfpathlineto{\pgfqpoint{5.704720in}{1.416940in}}%
\pgfpathlineto{\pgfqpoint{5.688654in}{1.451174in}}%
\pgfpathlineto{\pgfqpoint{5.693005in}{1.461938in}}%
\pgfpathlineto{\pgfqpoint{5.687337in}{1.476816in}}%
\pgfpathlineto{\pgfqpoint{5.697059in}{1.497392in}}%
\pgfpathlineto{\pgfqpoint{5.695475in}{1.520335in}}%
\pgfpathlineto{\pgfqpoint{5.702247in}{1.523223in}}%
\pgfpathlineto{\pgfqpoint{5.703098in}{1.534139in}}%
\pgfpathlineto{\pgfqpoint{5.715119in}{1.541146in}}%
\pgfpathlineto{\pgfqpoint{5.721918in}{1.540015in}}%
\pgfpathlineto{\pgfqpoint{5.723829in}{1.528307in}}%
\pgfpathlineto{\pgfqpoint{5.729135in}{1.527848in}}%
\pgfpathlineto{\pgfqpoint{5.733977in}{1.544735in}}%
\pgfpathlineto{\pgfqpoint{5.732331in}{1.557869in}}%
\pgfpathlineto{\pgfqpoint{5.735479in}{1.565411in}}%
\pgfpathlineto{\pgfqpoint{5.752498in}{1.573202in}}%
\pgfpathlineto{\pgfqpoint{5.744743in}{1.575975in}}%
\pgfpathlineto{\pgfqpoint{5.742216in}{1.582668in}}%
\pgfpathlineto{\pgfqpoint{5.747803in}{1.594421in}}%
\pgfpathlineto{\pgfqpoint{5.758821in}{1.598481in}}%
\pgfpathlineto{\pgfqpoint{5.771615in}{1.591546in}}%
\pgfpathlineto{\pgfqpoint{5.783677in}{1.591460in}}%
\pgfpathlineto{\pgfqpoint{5.789279in}{1.583359in}}%
\pgfpathlineto{\pgfqpoint{5.797727in}{1.583922in}}%
\pgfpathlineto{\pgfqpoint{5.823804in}{1.572661in}}%
\pgfpathlineto{\pgfqpoint{5.829014in}{1.561760in}}%
\pgfpathlineto{\pgfqpoint{5.823324in}{1.557985in}}%
\pgfpathlineto{\pgfqpoint{5.825077in}{1.549815in}}%
\pgfpathlineto{\pgfqpoint{5.830699in}{1.546182in}}%
\pgfpathlineto{\pgfqpoint{5.833831in}{1.536139in}}%
\pgfpathlineto{\pgfqpoint{5.833393in}{1.511672in}}%
\pgfpathlineto{\pgfqpoint{5.825971in}{1.505827in}}%
\pgfpathlineto{\pgfqpoint{5.824305in}{1.492975in}}%
\pgfpathlineto{\pgfqpoint{5.810538in}{1.481098in}}%
\pgfpathlineto{\pgfqpoint{5.811355in}{1.466722in}}%
\pgfpathlineto{\pgfqpoint{5.823421in}{1.461685in}}%
\pgfpathlineto{\pgfqpoint{5.837034in}{1.479623in}}%
\pgfpathlineto{\pgfqpoint{5.838159in}{1.486129in}}%
\pgfpathlineto{\pgfqpoint{5.855136in}{1.496942in}}%
\pgfpathlineto{\pgfqpoint{5.865932in}{1.491933in}}%
\pgfpathlineto{\pgfqpoint{5.872707in}{1.480613in}}%
\pgfpathlineto{\pgfqpoint{5.883636in}{1.441310in}}%
\pgfpathlineto{\pgfqpoint{5.889428in}{1.428876in}}%
\pgfpathlineto{\pgfqpoint{5.887615in}{1.423735in}}%
\pgfpathlineto{\pgfqpoint{5.887791in}{1.406232in}}%
\pgfpathlineto{\pgfqpoint{5.877255in}{1.407912in}}%
\pgfpathlineto{\pgfqpoint{5.871305in}{1.394832in}}%
\pgfpathlineto{\pgfqpoint{5.870480in}{1.385940in}}%
\pgfpathlineto{\pgfqpoint{5.862522in}{1.380232in}}%
\pgfpathlineto{\pgfqpoint{5.860753in}{1.362891in}}%
\pgfpathlineto{\pgfqpoint{5.849180in}{1.340984in}}%
\pgfpathlineto{\pgfqpoint{5.785872in}{1.331556in}}%
\pgfpathlineto{\pgfqpoint{5.785498in}{1.335699in}}%
\pgfpathlineto{\pgfqpoint{5.743150in}{1.331236in}}%
\pgfpathlineto{\pgfqpoint{5.685936in}{1.325898in}}%
\pgfpathclose%
\pgfusepath{fill}%
\end{pgfscope}%
\begin{pgfscope}%
\pgfpathrectangle{\pgfqpoint{3.625000in}{0.100000in}}{\pgfqpoint{2.989028in}{1.913466in}}%
\pgfusepath{clip}%
\pgfsetbuttcap%
\pgfsetmiterjoin%
\definecolor{currentfill}{rgb}{0.312034,0.662668,0.687659}%
\pgfsetfillcolor{currentfill}%
\pgfsetlinewidth{0.000000pt}%
\definecolor{currentstroke}{rgb}{0.000000,0.000000,0.000000}%
\pgfsetstrokecolor{currentstroke}%
\pgfsetstrokeopacity{0.000000}%
\pgfsetdash{}{0pt}%
\pgfpathmoveto{\pgfqpoint{4.087031in}{0.446426in}}%
\pgfpathlineto{\pgfqpoint{4.085767in}{0.445314in}}%
\pgfpathlineto{\pgfqpoint{4.079967in}{0.446797in}}%
\pgfpathlineto{\pgfqpoint{4.080400in}{0.449716in}}%
\pgfpathlineto{\pgfqpoint{4.083683in}{0.451053in}}%
\pgfpathlineto{\pgfqpoint{4.088364in}{0.457782in}}%
\pgfpathlineto{\pgfqpoint{4.089142in}{0.462330in}}%
\pgfpathlineto{\pgfqpoint{4.091773in}{0.463522in}}%
\pgfpathlineto{\pgfqpoint{4.096054in}{0.463526in}}%
\pgfpathlineto{\pgfqpoint{4.100530in}{0.478285in}}%
\pgfpathlineto{\pgfqpoint{4.102283in}{0.479499in}}%
\pgfpathlineto{\pgfqpoint{4.100711in}{0.482384in}}%
\pgfpathlineto{\pgfqpoint{4.097030in}{0.479466in}}%
\pgfpathlineto{\pgfqpoint{4.086215in}{0.483400in}}%
\pgfpathlineto{\pgfqpoint{4.079760in}{0.490029in}}%
\pgfpathlineto{\pgfqpoint{4.082575in}{0.493053in}}%
\pgfpathlineto{\pgfqpoint{4.081295in}{0.497319in}}%
\pgfpathlineto{\pgfqpoint{4.082083in}{0.502359in}}%
\pgfpathlineto{\pgfqpoint{4.080695in}{0.508250in}}%
\pgfpathlineto{\pgfqpoint{4.080473in}{0.514148in}}%
\pgfpathlineto{\pgfqpoint{4.086686in}{0.513161in}}%
\pgfpathlineto{\pgfqpoint{4.087916in}{0.514899in}}%
\pgfpathlineto{\pgfqpoint{4.091006in}{0.514731in}}%
\pgfpathlineto{\pgfqpoint{4.094565in}{0.508441in}}%
\pgfpathlineto{\pgfqpoint{4.097996in}{0.503835in}}%
\pgfpathlineto{\pgfqpoint{4.095248in}{0.500837in}}%
\pgfpathlineto{\pgfqpoint{4.100000in}{0.499687in}}%
\pgfpathlineto{\pgfqpoint{4.101083in}{0.501158in}}%
\pgfpathlineto{\pgfqpoint{4.098542in}{0.507123in}}%
\pgfpathlineto{\pgfqpoint{4.093190in}{0.512068in}}%
\pgfpathlineto{\pgfqpoint{4.093845in}{0.514140in}}%
\pgfpathlineto{\pgfqpoint{4.089574in}{0.519376in}}%
\pgfpathlineto{\pgfqpoint{4.093935in}{0.520439in}}%
\pgfpathlineto{\pgfqpoint{4.090157in}{0.531650in}}%
\pgfpathlineto{\pgfqpoint{4.093229in}{0.533204in}}%
\pgfpathlineto{\pgfqpoint{4.093978in}{0.535892in}}%
\pgfpathlineto{\pgfqpoint{4.091980in}{0.539085in}}%
\pgfpathlineto{\pgfqpoint{4.096104in}{0.539884in}}%
\pgfpathlineto{\pgfqpoint{4.096845in}{0.542299in}}%
\pgfpathlineto{\pgfqpoint{4.101765in}{0.538674in}}%
\pgfpathlineto{\pgfqpoint{4.102857in}{0.542605in}}%
\pgfpathlineto{\pgfqpoint{4.105443in}{0.544334in}}%
\pgfpathlineto{\pgfqpoint{4.118023in}{0.546739in}}%
\pgfpathlineto{\pgfqpoint{4.121885in}{0.545889in}}%
\pgfpathlineto{\pgfqpoint{4.125120in}{0.550663in}}%
\pgfpathlineto{\pgfqpoint{4.132797in}{0.553794in}}%
\pgfpathlineto{\pgfqpoint{4.138710in}{0.552606in}}%
\pgfpathlineto{\pgfqpoint{4.141407in}{0.548449in}}%
\pgfpathlineto{\pgfqpoint{4.141266in}{0.541844in}}%
\pgfpathlineto{\pgfqpoint{4.144292in}{0.540347in}}%
\pgfpathlineto{\pgfqpoint{4.150464in}{0.540556in}}%
\pgfpathlineto{\pgfqpoint{4.158370in}{0.542274in}}%
\pgfpathlineto{\pgfqpoint{4.158272in}{0.540162in}}%
\pgfpathlineto{\pgfqpoint{4.168266in}{0.534139in}}%
\pgfpathlineto{\pgfqpoint{4.175255in}{0.535754in}}%
\pgfpathlineto{\pgfqpoint{4.177313in}{0.537844in}}%
\pgfpathlineto{\pgfqpoint{4.179516in}{0.543197in}}%
\pgfpathlineto{\pgfqpoint{4.182248in}{0.546625in}}%
\pgfpathlineto{\pgfqpoint{4.182492in}{0.551762in}}%
\pgfpathlineto{\pgfqpoint{4.179990in}{0.553782in}}%
\pgfpathlineto{\pgfqpoint{4.183695in}{0.555608in}}%
\pgfpathlineto{\pgfqpoint{4.189920in}{0.552865in}}%
\pgfpathlineto{\pgfqpoint{4.191948in}{0.554559in}}%
\pgfpathlineto{\pgfqpoint{4.192305in}{0.561826in}}%
\pgfpathlineto{\pgfqpoint{4.188281in}{0.560307in}}%
\pgfpathlineto{\pgfqpoint{4.185490in}{0.562825in}}%
\pgfpathlineto{\pgfqpoint{4.180960in}{0.564395in}}%
\pgfpathlineto{\pgfqpoint{4.173770in}{0.564133in}}%
\pgfpathlineto{\pgfqpoint{4.166880in}{0.566521in}}%
\pgfpathlineto{\pgfqpoint{4.166535in}{0.572469in}}%
\pgfpathlineto{\pgfqpoint{4.160373in}{0.578787in}}%
\pgfpathlineto{\pgfqpoint{4.152559in}{0.581439in}}%
\pgfpathlineto{\pgfqpoint{4.145739in}{0.593671in}}%
\pgfpathlineto{\pgfqpoint{4.146298in}{0.598197in}}%
\pgfpathlineto{\pgfqpoint{4.149971in}{0.600261in}}%
\pgfpathlineto{\pgfqpoint{4.149430in}{0.604545in}}%
\pgfpathlineto{\pgfqpoint{4.150282in}{0.608808in}}%
\pgfpathlineto{\pgfqpoint{4.153188in}{0.605890in}}%
\pgfpathlineto{\pgfqpoint{4.157025in}{0.607107in}}%
\pgfpathlineto{\pgfqpoint{4.156833in}{0.610810in}}%
\pgfpathlineto{\pgfqpoint{4.151514in}{0.618243in}}%
\pgfpathlineto{\pgfqpoint{4.149956in}{0.625911in}}%
\pgfpathlineto{\pgfqpoint{4.154409in}{0.626496in}}%
\pgfpathlineto{\pgfqpoint{4.161074in}{0.626025in}}%
\pgfpathlineto{\pgfqpoint{4.163172in}{0.623460in}}%
\pgfpathlineto{\pgfqpoint{4.169147in}{0.624378in}}%
\pgfpathlineto{\pgfqpoint{4.174700in}{0.623173in}}%
\pgfpathlineto{\pgfqpoint{4.178454in}{0.621055in}}%
\pgfpathlineto{\pgfqpoint{4.181052in}{0.622023in}}%
\pgfpathlineto{\pgfqpoint{4.187951in}{0.619906in}}%
\pgfpathlineto{\pgfqpoint{4.188014in}{0.618032in}}%
\pgfpathlineto{\pgfqpoint{4.193268in}{0.618665in}}%
\pgfpathlineto{\pgfqpoint{4.200313in}{0.614092in}}%
\pgfpathlineto{\pgfqpoint{4.200039in}{0.611057in}}%
\pgfpathlineto{\pgfqpoint{4.196687in}{0.609656in}}%
\pgfpathlineto{\pgfqpoint{4.193124in}{0.606323in}}%
\pgfpathlineto{\pgfqpoint{4.192694in}{0.601746in}}%
\pgfpathlineto{\pgfqpoint{4.200369in}{0.595909in}}%
\pgfpathlineto{\pgfqpoint{4.199141in}{0.593112in}}%
\pgfpathlineto{\pgfqpoint{4.204602in}{0.590964in}}%
\pgfpathlineto{\pgfqpoint{4.205943in}{0.587455in}}%
\pgfpathlineto{\pgfqpoint{4.212873in}{0.590243in}}%
\pgfpathlineto{\pgfqpoint{4.215925in}{0.586589in}}%
\pgfpathlineto{\pgfqpoint{4.217334in}{0.588801in}}%
\pgfpathlineto{\pgfqpoint{4.213186in}{0.595999in}}%
\pgfpathlineto{\pgfqpoint{4.214660in}{0.598155in}}%
\pgfpathlineto{\pgfqpoint{4.215313in}{0.603841in}}%
\pgfpathlineto{\pgfqpoint{4.213661in}{0.606342in}}%
\pgfpathlineto{\pgfqpoint{4.214901in}{0.609704in}}%
\pgfpathlineto{\pgfqpoint{4.218961in}{0.609388in}}%
\pgfpathlineto{\pgfqpoint{4.218121in}{0.603668in}}%
\pgfpathlineto{\pgfqpoint{4.215529in}{0.601760in}}%
\pgfpathlineto{\pgfqpoint{4.215712in}{0.594065in}}%
\pgfpathlineto{\pgfqpoint{4.220364in}{0.593146in}}%
\pgfpathlineto{\pgfqpoint{4.220421in}{0.587424in}}%
\pgfpathlineto{\pgfqpoint{4.225501in}{0.583713in}}%
\pgfpathlineto{\pgfqpoint{4.227836in}{0.586053in}}%
\pgfpathlineto{\pgfqpoint{4.225284in}{0.593176in}}%
\pgfpathlineto{\pgfqpoint{4.222644in}{0.594979in}}%
\pgfpathlineto{\pgfqpoint{4.220047in}{0.593793in}}%
\pgfpathlineto{\pgfqpoint{4.217763in}{0.595306in}}%
\pgfpathlineto{\pgfqpoint{4.218161in}{0.601889in}}%
\pgfpathlineto{\pgfqpoint{4.221974in}{0.604581in}}%
\pgfpathlineto{\pgfqpoint{4.225726in}{0.604333in}}%
\pgfpathlineto{\pgfqpoint{4.224172in}{0.608058in}}%
\pgfpathlineto{\pgfqpoint{4.218547in}{0.610911in}}%
\pgfpathlineto{\pgfqpoint{4.211098in}{0.622277in}}%
\pgfpathlineto{\pgfqpoint{4.214816in}{0.627728in}}%
\pgfpathlineto{\pgfqpoint{4.216885in}{0.632863in}}%
\pgfpathlineto{\pgfqpoint{4.217048in}{0.643174in}}%
\pgfpathlineto{\pgfqpoint{4.216251in}{0.651872in}}%
\pgfpathlineto{\pgfqpoint{4.213978in}{0.657584in}}%
\pgfpathlineto{\pgfqpoint{4.214431in}{0.663294in}}%
\pgfpathlineto{\pgfqpoint{4.220244in}{0.668247in}}%
\pgfpathlineto{\pgfqpoint{4.226026in}{0.674133in}}%
\pgfpathlineto{\pgfqpoint{4.240519in}{0.662108in}}%
\pgfpathlineto{\pgfqpoint{4.249995in}{0.661221in}}%
\pgfpathlineto{\pgfqpoint{4.257817in}{0.664216in}}%
\pgfpathlineto{\pgfqpoint{4.264683in}{0.668159in}}%
\pgfpathlineto{\pgfqpoint{4.266141in}{0.670038in}}%
\pgfpathlineto{\pgfqpoint{4.283090in}{0.674322in}}%
\pgfpathlineto{\pgfqpoint{4.284562in}{0.671184in}}%
\pgfpathlineto{\pgfqpoint{4.290321in}{0.668615in}}%
\pgfpathlineto{\pgfqpoint{4.301313in}{0.669403in}}%
\pgfpathlineto{\pgfqpoint{4.303320in}{0.668702in}}%
\pgfpathlineto{\pgfqpoint{4.308260in}{0.670576in}}%
\pgfpathlineto{\pgfqpoint{4.310246in}{0.667335in}}%
\pgfpathlineto{\pgfqpoint{4.314990in}{0.664374in}}%
\pgfpathlineto{\pgfqpoint{4.317363in}{0.664709in}}%
\pgfpathlineto{\pgfqpoint{4.321956in}{0.661151in}}%
\pgfpathlineto{\pgfqpoint{4.329717in}{0.661805in}}%
\pgfpathlineto{\pgfqpoint{4.337469in}{0.664685in}}%
\pgfpathlineto{\pgfqpoint{4.343800in}{0.655475in}}%
\pgfpathlineto{\pgfqpoint{4.343309in}{0.653408in}}%
\pgfpathlineto{\pgfqpoint{4.336323in}{0.651032in}}%
\pgfpathlineto{\pgfqpoint{4.338190in}{0.647440in}}%
\pgfpathlineto{\pgfqpoint{4.344621in}{0.651221in}}%
\pgfpathlineto{\pgfqpoint{4.347246in}{0.650305in}}%
\pgfpathlineto{\pgfqpoint{4.348363in}{0.645783in}}%
\pgfpathlineto{\pgfqpoint{4.344524in}{0.643870in}}%
\pgfpathlineto{\pgfqpoint{4.347266in}{0.638154in}}%
\pgfpathlineto{\pgfqpoint{4.351248in}{0.639120in}}%
\pgfpathlineto{\pgfqpoint{4.357026in}{0.636170in}}%
\pgfpathlineto{\pgfqpoint{4.362551in}{0.627982in}}%
\pgfpathlineto{\pgfqpoint{4.357689in}{0.625708in}}%
\pgfpathlineto{\pgfqpoint{4.361985in}{0.619675in}}%
\pgfpathlineto{\pgfqpoint{4.358366in}{0.618264in}}%
\pgfpathlineto{\pgfqpoint{4.362868in}{0.612534in}}%
\pgfpathlineto{\pgfqpoint{4.368358in}{0.613145in}}%
\pgfpathlineto{\pgfqpoint{4.371249in}{0.610567in}}%
\pgfpathlineto{\pgfqpoint{4.378562in}{0.606765in}}%
\pgfpathlineto{\pgfqpoint{4.387995in}{0.593271in}}%
\pgfpathlineto{\pgfqpoint{4.387782in}{0.590043in}}%
\pgfpathlineto{\pgfqpoint{4.401928in}{0.579706in}}%
\pgfpathlineto{\pgfqpoint{4.402233in}{0.576049in}}%
\pgfpathlineto{\pgfqpoint{4.406003in}{0.570578in}}%
\pgfpathlineto{\pgfqpoint{4.417585in}{0.567843in}}%
\pgfpathlineto{\pgfqpoint{4.421881in}{0.565899in}}%
\pgfpathlineto{\pgfqpoint{4.425922in}{0.559868in}}%
\pgfpathlineto{\pgfqpoint{4.426442in}{0.554585in}}%
\pgfpathlineto{\pgfqpoint{4.429915in}{0.550279in}}%
\pgfpathlineto{\pgfqpoint{4.430958in}{0.545423in}}%
\pgfpathlineto{\pgfqpoint{4.434213in}{0.543649in}}%
\pgfpathlineto{\pgfqpoint{4.418089in}{0.514548in}}%
\pgfpathlineto{\pgfqpoint{4.384622in}{0.454133in}}%
\pgfpathlineto{\pgfqpoint{4.335518in}{0.365475in}}%
\pgfpathlineto{\pgfqpoint{4.316294in}{0.330756in}}%
\pgfpathlineto{\pgfqpoint{4.320388in}{0.326093in}}%
\pgfpathlineto{\pgfqpoint{4.322192in}{0.327581in}}%
\pgfpathlineto{\pgfqpoint{4.325850in}{0.322152in}}%
\pgfpathlineto{\pgfqpoint{4.330927in}{0.323727in}}%
\pgfpathlineto{\pgfqpoint{4.337667in}{0.320475in}}%
\pgfpathlineto{\pgfqpoint{4.333301in}{0.315393in}}%
\pgfpathlineto{\pgfqpoint{4.336672in}{0.308535in}}%
\pgfpathlineto{\pgfqpoint{4.335976in}{0.305196in}}%
\pgfpathlineto{\pgfqpoint{4.341353in}{0.287587in}}%
\pgfpathlineto{\pgfqpoint{4.339077in}{0.279580in}}%
\pgfpathlineto{\pgfqpoint{4.349350in}{0.281562in}}%
\pgfpathlineto{\pgfqpoint{4.351872in}{0.280263in}}%
\pgfpathlineto{\pgfqpoint{4.354610in}{0.282376in}}%
\pgfpathlineto{\pgfqpoint{4.356558in}{0.286368in}}%
\pgfpathlineto{\pgfqpoint{4.359347in}{0.288657in}}%
\pgfpathlineto{\pgfqpoint{4.371252in}{0.288367in}}%
\pgfpathlineto{\pgfqpoint{4.373943in}{0.280683in}}%
\pgfpathlineto{\pgfqpoint{4.371634in}{0.273938in}}%
\pgfpathlineto{\pgfqpoint{4.374257in}{0.271712in}}%
\pgfpathlineto{\pgfqpoint{4.375636in}{0.267905in}}%
\pgfpathlineto{\pgfqpoint{4.375336in}{0.260746in}}%
\pgfpathlineto{\pgfqpoint{4.378836in}{0.255613in}}%
\pgfpathlineto{\pgfqpoint{4.380755in}{0.247568in}}%
\pgfpathlineto{\pgfqpoint{4.379770in}{0.243223in}}%
\pgfpathlineto{\pgfqpoint{4.380958in}{0.238819in}}%
\pgfpathlineto{\pgfqpoint{4.382671in}{0.213318in}}%
\pgfpathlineto{\pgfqpoint{4.380216in}{0.211309in}}%
\pgfpathlineto{\pgfqpoint{4.383455in}{0.208533in}}%
\pgfpathlineto{\pgfqpoint{4.380910in}{0.205157in}}%
\pgfpathlineto{\pgfqpoint{4.383206in}{0.202405in}}%
\pgfpathlineto{\pgfqpoint{4.381727in}{0.197707in}}%
\pgfpathlineto{\pgfqpoint{4.385034in}{0.196433in}}%
\pgfpathlineto{\pgfqpoint{4.389287in}{0.189287in}}%
\pgfpathlineto{\pgfqpoint{4.392447in}{0.187134in}}%
\pgfpathlineto{\pgfqpoint{4.393274in}{0.183987in}}%
\pgfpathlineto{\pgfqpoint{4.395105in}{0.182623in}}%
\pgfpathlineto{\pgfqpoint{4.394723in}{0.180010in}}%
\pgfpathlineto{\pgfqpoint{4.398670in}{0.178116in}}%
\pgfpathlineto{\pgfqpoint{4.397696in}{0.173088in}}%
\pgfpathlineto{\pgfqpoint{4.394298in}{0.170454in}}%
\pgfpathlineto{\pgfqpoint{4.392755in}{0.165560in}}%
\pgfpathlineto{\pgfqpoint{4.392345in}{0.159006in}}%
\pgfpathlineto{\pgfqpoint{4.390458in}{0.158379in}}%
\pgfpathlineto{\pgfqpoint{4.384289in}{0.152634in}}%
\pgfpathlineto{\pgfqpoint{4.378806in}{0.151118in}}%
\pgfpathlineto{\pgfqpoint{4.376190in}{0.153435in}}%
\pgfpathlineto{\pgfqpoint{4.378718in}{0.159583in}}%
\pgfpathlineto{\pgfqpoint{4.377946in}{0.162601in}}%
\pgfpathlineto{\pgfqpoint{4.381568in}{0.164131in}}%
\pgfpathlineto{\pgfqpoint{4.384956in}{0.173027in}}%
\pgfpathlineto{\pgfqpoint{4.383525in}{0.180625in}}%
\pgfpathlineto{\pgfqpoint{4.381235in}{0.182419in}}%
\pgfpathlineto{\pgfqpoint{4.373578in}{0.181838in}}%
\pgfpathlineto{\pgfqpoint{4.375121in}{0.179853in}}%
\pgfpathlineto{\pgfqpoint{4.369579in}{0.174155in}}%
\pgfpathlineto{\pgfqpoint{4.367793in}{0.177323in}}%
\pgfpathlineto{\pgfqpoint{4.368258in}{0.181462in}}%
\pgfpathlineto{\pgfqpoint{4.371439in}{0.181386in}}%
\pgfpathlineto{\pgfqpoint{4.374272in}{0.184426in}}%
\pgfpathlineto{\pgfqpoint{4.375948in}{0.188866in}}%
\pgfpathlineto{\pgfqpoint{4.382264in}{0.187613in}}%
\pgfpathlineto{\pgfqpoint{4.377208in}{0.189984in}}%
\pgfpathlineto{\pgfqpoint{4.377653in}{0.193161in}}%
\pgfpathlineto{\pgfqpoint{4.375549in}{0.194596in}}%
\pgfpathlineto{\pgfqpoint{4.374687in}{0.198923in}}%
\pgfpathlineto{\pgfqpoint{4.376181in}{0.201177in}}%
\pgfpathlineto{\pgfqpoint{4.372619in}{0.208179in}}%
\pgfpathlineto{\pgfqpoint{4.372919in}{0.212633in}}%
\pgfpathlineto{\pgfqpoint{4.368746in}{0.216599in}}%
\pgfpathlineto{\pgfqpoint{4.366488in}{0.219808in}}%
\pgfpathlineto{\pgfqpoint{4.369606in}{0.222866in}}%
\pgfpathlineto{\pgfqpoint{4.370761in}{0.227437in}}%
\pgfpathlineto{\pgfqpoint{4.369730in}{0.230438in}}%
\pgfpathlineto{\pgfqpoint{4.370936in}{0.234017in}}%
\pgfpathlineto{\pgfqpoint{4.369399in}{0.245686in}}%
\pgfpathlineto{\pgfqpoint{4.367100in}{0.250980in}}%
\pgfpathlineto{\pgfqpoint{4.364471in}{0.252970in}}%
\pgfpathlineto{\pgfqpoint{4.365590in}{0.259857in}}%
\pgfpathlineto{\pgfqpoint{4.364801in}{0.265235in}}%
\pgfpathlineto{\pgfqpoint{4.366230in}{0.268685in}}%
\pgfpathlineto{\pgfqpoint{4.363272in}{0.269169in}}%
\pgfpathlineto{\pgfqpoint{4.362479in}{0.260789in}}%
\pgfpathlineto{\pgfqpoint{4.359154in}{0.251441in}}%
\pgfpathlineto{\pgfqpoint{4.356484in}{0.253570in}}%
\pgfpathlineto{\pgfqpoint{4.356127in}{0.256642in}}%
\pgfpathlineto{\pgfqpoint{4.352861in}{0.260839in}}%
\pgfpathlineto{\pgfqpoint{4.354521in}{0.263587in}}%
\pgfpathlineto{\pgfqpoint{4.354005in}{0.269772in}}%
\pgfpathlineto{\pgfqpoint{4.349778in}{0.267597in}}%
\pgfpathlineto{\pgfqpoint{4.350607in}{0.264242in}}%
\pgfpathlineto{\pgfqpoint{4.349811in}{0.259983in}}%
\pgfpathlineto{\pgfqpoint{4.345066in}{0.259990in}}%
\pgfpathlineto{\pgfqpoint{4.342939in}{0.262630in}}%
\pgfpathlineto{\pgfqpoint{4.339729in}{0.263134in}}%
\pgfpathlineto{\pgfqpoint{4.337569in}{0.265977in}}%
\pgfpathlineto{\pgfqpoint{4.333852in}{0.273782in}}%
\pgfpathlineto{\pgfqpoint{4.332648in}{0.279842in}}%
\pgfpathlineto{\pgfqpoint{4.333418in}{0.282418in}}%
\pgfpathlineto{\pgfqpoint{4.331620in}{0.288091in}}%
\pgfpathlineto{\pgfqpoint{4.328639in}{0.291375in}}%
\pgfpathlineto{\pgfqpoint{4.323483in}{0.299768in}}%
\pgfpathlineto{\pgfqpoint{4.319744in}{0.307057in}}%
\pgfpathlineto{\pgfqpoint{4.323559in}{0.307167in}}%
\pgfpathlineto{\pgfqpoint{4.325804in}{0.308698in}}%
\pgfpathlineto{\pgfqpoint{4.326246in}{0.313435in}}%
\pgfpathlineto{\pgfqpoint{4.315450in}{0.314007in}}%
\pgfpathlineto{\pgfqpoint{4.306267in}{0.324264in}}%
\pgfpathlineto{\pgfqpoint{4.308682in}{0.326962in}}%
\pgfpathlineto{\pgfqpoint{4.304241in}{0.327423in}}%
\pgfpathlineto{\pgfqpoint{4.295423in}{0.336959in}}%
\pgfpathlineto{\pgfqpoint{4.281451in}{0.341741in}}%
\pgfpathlineto{\pgfqpoint{4.279570in}{0.347345in}}%
\pgfpathlineto{\pgfqpoint{4.276671in}{0.350264in}}%
\pgfpathlineto{\pgfqpoint{4.276539in}{0.353005in}}%
\pgfpathlineto{\pgfqpoint{4.274358in}{0.355041in}}%
\pgfpathlineto{\pgfqpoint{4.277220in}{0.358442in}}%
\pgfpathlineto{\pgfqpoint{4.270693in}{0.358770in}}%
\pgfpathlineto{\pgfqpoint{4.269034in}{0.363173in}}%
\pgfpathlineto{\pgfqpoint{4.266063in}{0.365047in}}%
\pgfpathlineto{\pgfqpoint{4.272012in}{0.367309in}}%
\pgfpathlineto{\pgfqpoint{4.268918in}{0.371345in}}%
\pgfpathlineto{\pgfqpoint{4.263224in}{0.373038in}}%
\pgfpathlineto{\pgfqpoint{4.264110in}{0.380491in}}%
\pgfpathlineto{\pgfqpoint{4.262478in}{0.383130in}}%
\pgfpathlineto{\pgfqpoint{4.259538in}{0.384988in}}%
\pgfpathlineto{\pgfqpoint{4.257024in}{0.383124in}}%
\pgfpathlineto{\pgfqpoint{4.255538in}{0.384877in}}%
\pgfpathlineto{\pgfqpoint{4.251206in}{0.385379in}}%
\pgfpathlineto{\pgfqpoint{4.251555in}{0.392143in}}%
\pgfpathlineto{\pgfqpoint{4.245275in}{0.387574in}}%
\pgfpathlineto{\pgfqpoint{4.245407in}{0.378160in}}%
\pgfpathlineto{\pgfqpoint{4.238666in}{0.376355in}}%
\pgfpathlineto{\pgfqpoint{4.234263in}{0.372596in}}%
\pgfpathlineto{\pgfqpoint{4.230725in}{0.371804in}}%
\pgfpathlineto{\pgfqpoint{4.226895in}{0.375593in}}%
\pgfpathlineto{\pgfqpoint{4.223485in}{0.375090in}}%
\pgfpathlineto{\pgfqpoint{4.224118in}{0.378551in}}%
\pgfpathlineto{\pgfqpoint{4.219270in}{0.376688in}}%
\pgfpathlineto{\pgfqpoint{4.214779in}{0.376478in}}%
\pgfpathlineto{\pgfqpoint{4.209523in}{0.374589in}}%
\pgfpathlineto{\pgfqpoint{4.198855in}{0.375652in}}%
\pgfpathlineto{\pgfqpoint{4.195430in}{0.374953in}}%
\pgfpathlineto{\pgfqpoint{4.192959in}{0.378065in}}%
\pgfpathlineto{\pgfqpoint{4.188397in}{0.378376in}}%
\pgfpathlineto{\pgfqpoint{4.187449in}{0.382409in}}%
\pgfpathlineto{\pgfqpoint{4.190167in}{0.384606in}}%
\pgfpathlineto{\pgfqpoint{4.193357in}{0.384399in}}%
\pgfpathlineto{\pgfqpoint{4.196019in}{0.381403in}}%
\pgfpathlineto{\pgfqpoint{4.197772in}{0.386005in}}%
\pgfpathlineto{\pgfqpoint{4.195326in}{0.391102in}}%
\pgfpathlineto{\pgfqpoint{4.200863in}{0.395589in}}%
\pgfpathlineto{\pgfqpoint{4.206411in}{0.397243in}}%
\pgfpathlineto{\pgfqpoint{4.210273in}{0.399940in}}%
\pgfpathlineto{\pgfqpoint{4.212775in}{0.402846in}}%
\pgfpathlineto{\pgfqpoint{4.214172in}{0.407609in}}%
\pgfpathlineto{\pgfqpoint{4.218583in}{0.406312in}}%
\pgfpathlineto{\pgfqpoint{4.228148in}{0.406979in}}%
\pgfpathlineto{\pgfqpoint{4.230201in}{0.401756in}}%
\pgfpathlineto{\pgfqpoint{4.233556in}{0.401529in}}%
\pgfpathlineto{\pgfqpoint{4.239662in}{0.393534in}}%
\pgfpathlineto{\pgfqpoint{4.239673in}{0.396448in}}%
\pgfpathlineto{\pgfqpoint{4.237540in}{0.397396in}}%
\pgfpathlineto{\pgfqpoint{4.233636in}{0.406940in}}%
\pgfpathlineto{\pgfqpoint{4.235536in}{0.408236in}}%
\pgfpathlineto{\pgfqpoint{4.230322in}{0.413215in}}%
\pgfpathlineto{\pgfqpoint{4.225663in}{0.414184in}}%
\pgfpathlineto{\pgfqpoint{4.221223in}{0.412463in}}%
\pgfpathlineto{\pgfqpoint{4.213619in}{0.413804in}}%
\pgfpathlineto{\pgfqpoint{4.210013in}{0.411392in}}%
\pgfpathlineto{\pgfqpoint{4.201801in}{0.409529in}}%
\pgfpathlineto{\pgfqpoint{4.201050in}{0.406847in}}%
\pgfpathlineto{\pgfqpoint{4.194741in}{0.406115in}}%
\pgfpathlineto{\pgfqpoint{4.193122in}{0.403049in}}%
\pgfpathlineto{\pgfqpoint{4.189513in}{0.400803in}}%
\pgfpathlineto{\pgfqpoint{4.185154in}{0.401189in}}%
\pgfpathlineto{\pgfqpoint{4.182977in}{0.399018in}}%
\pgfpathlineto{\pgfqpoint{4.180231in}{0.399119in}}%
\pgfpathlineto{\pgfqpoint{4.177607in}{0.401074in}}%
\pgfpathlineto{\pgfqpoint{4.173019in}{0.400725in}}%
\pgfpathlineto{\pgfqpoint{4.171938in}{0.398924in}}%
\pgfpathlineto{\pgfqpoint{4.167119in}{0.400773in}}%
\pgfpathlineto{\pgfqpoint{4.163027in}{0.396000in}}%
\pgfpathlineto{\pgfqpoint{4.166950in}{0.391382in}}%
\pgfpathlineto{\pgfqpoint{4.168460in}{0.387584in}}%
\pgfpathlineto{\pgfqpoint{4.167068in}{0.384367in}}%
\pgfpathlineto{\pgfqpoint{4.161493in}{0.382148in}}%
\pgfpathlineto{\pgfqpoint{4.158263in}{0.383954in}}%
\pgfpathlineto{\pgfqpoint{4.154198in}{0.382976in}}%
\pgfpathlineto{\pgfqpoint{4.153708in}{0.380157in}}%
\pgfpathlineto{\pgfqpoint{4.148759in}{0.381154in}}%
\pgfpathlineto{\pgfqpoint{4.149855in}{0.378337in}}%
\pgfpathlineto{\pgfqpoint{4.142555in}{0.377743in}}%
\pgfpathlineto{\pgfqpoint{4.137830in}{0.381319in}}%
\pgfpathlineto{\pgfqpoint{4.135306in}{0.379045in}}%
\pgfpathlineto{\pgfqpoint{4.128590in}{0.381466in}}%
\pgfpathlineto{\pgfqpoint{4.126918in}{0.379014in}}%
\pgfpathlineto{\pgfqpoint{4.123588in}{0.377923in}}%
\pgfpathlineto{\pgfqpoint{4.120917in}{0.380797in}}%
\pgfpathlineto{\pgfqpoint{4.111784in}{0.380083in}}%
\pgfpathlineto{\pgfqpoint{4.111794in}{0.375843in}}%
\pgfpathlineto{\pgfqpoint{4.108125in}{0.374721in}}%
\pgfpathlineto{\pgfqpoint{4.105270in}{0.376900in}}%
\pgfpathlineto{\pgfqpoint{4.101497in}{0.375689in}}%
\pgfpathlineto{\pgfqpoint{4.096614in}{0.375401in}}%
\pgfpathlineto{\pgfqpoint{4.089340in}{0.378466in}}%
\pgfpathlineto{\pgfqpoint{4.085174in}{0.375305in}}%
\pgfpathlineto{\pgfqpoint{4.079216in}{0.379530in}}%
\pgfpathlineto{\pgfqpoint{4.077268in}{0.377929in}}%
\pgfpathlineto{\pgfqpoint{4.075900in}{0.373791in}}%
\pgfpathlineto{\pgfqpoint{4.072071in}{0.371365in}}%
\pgfpathlineto{\pgfqpoint{4.065825in}{0.374142in}}%
\pgfpathlineto{\pgfqpoint{4.055479in}{0.381003in}}%
\pgfpathlineto{\pgfqpoint{4.052463in}{0.381613in}}%
\pgfpathlineto{\pgfqpoint{4.050762in}{0.380183in}}%
\pgfpathlineto{\pgfqpoint{4.046211in}{0.381155in}}%
\pgfpathlineto{\pgfqpoint{4.043991in}{0.383290in}}%
\pgfpathlineto{\pgfqpoint{4.039445in}{0.384441in}}%
\pgfpathlineto{\pgfqpoint{4.033238in}{0.384507in}}%
\pgfpathlineto{\pgfqpoint{4.030773in}{0.387054in}}%
\pgfpathlineto{\pgfqpoint{4.035497in}{0.389690in}}%
\pgfpathlineto{\pgfqpoint{4.034395in}{0.392291in}}%
\pgfpathlineto{\pgfqpoint{4.031149in}{0.391707in}}%
\pgfpathlineto{\pgfqpoint{4.029528in}{0.389662in}}%
\pgfpathlineto{\pgfqpoint{4.021930in}{0.386730in}}%
\pgfpathlineto{\pgfqpoint{4.018088in}{0.387727in}}%
\pgfpathlineto{\pgfqpoint{4.016471in}{0.393681in}}%
\pgfpathlineto{\pgfqpoint{4.011679in}{0.391121in}}%
\pgfpathlineto{\pgfqpoint{4.010286in}{0.398667in}}%
\pgfpathlineto{\pgfqpoint{4.008015in}{0.396722in}}%
\pgfpathlineto{\pgfqpoint{4.008395in}{0.393819in}}%
\pgfpathlineto{\pgfqpoint{4.003173in}{0.394586in}}%
\pgfpathlineto{\pgfqpoint{4.008605in}{0.399646in}}%
\pgfpathlineto{\pgfqpoint{4.014110in}{0.396639in}}%
\pgfpathlineto{\pgfqpoint{4.015291in}{0.398148in}}%
\pgfpathlineto{\pgfqpoint{4.021082in}{0.396339in}}%
\pgfpathlineto{\pgfqpoint{4.020720in}{0.398399in}}%
\pgfpathlineto{\pgfqpoint{4.036521in}{0.399081in}}%
\pgfpathlineto{\pgfqpoint{4.044230in}{0.395868in}}%
\pgfpathlineto{\pgfqpoint{4.047212in}{0.392698in}}%
\pgfpathlineto{\pgfqpoint{4.046327in}{0.389931in}}%
\pgfpathlineto{\pgfqpoint{4.049323in}{0.388883in}}%
\pgfpathlineto{\pgfqpoint{4.056765in}{0.393251in}}%
\pgfpathlineto{\pgfqpoint{4.066392in}{0.393531in}}%
\pgfpathlineto{\pgfqpoint{4.075136in}{0.390898in}}%
\pgfpathlineto{\pgfqpoint{4.080089in}{0.391340in}}%
\pgfpathlineto{\pgfqpoint{4.081933in}{0.387255in}}%
\pgfpathlineto{\pgfqpoint{4.084703in}{0.391758in}}%
\pgfpathlineto{\pgfqpoint{4.086407in}{0.392733in}}%
\pgfpathlineto{\pgfqpoint{4.094288in}{0.394691in}}%
\pgfpathlineto{\pgfqpoint{4.097217in}{0.393494in}}%
\pgfpathlineto{\pgfqpoint{4.100385in}{0.394915in}}%
\pgfpathlineto{\pgfqpoint{4.104163in}{0.394172in}}%
\pgfpathlineto{\pgfqpoint{4.105055in}{0.395827in}}%
\pgfpathlineto{\pgfqpoint{4.112939in}{0.403420in}}%
\pgfpathlineto{\pgfqpoint{4.116278in}{0.404545in}}%
\pgfpathlineto{\pgfqpoint{4.118251in}{0.408644in}}%
\pgfpathlineto{\pgfqpoint{4.128722in}{0.411817in}}%
\pgfpathlineto{\pgfqpoint{4.130033in}{0.414238in}}%
\pgfpathlineto{\pgfqpoint{4.115192in}{0.417974in}}%
\pgfpathlineto{\pgfqpoint{4.115776in}{0.421993in}}%
\pgfpathlineto{\pgfqpoint{4.114250in}{0.427301in}}%
\pgfpathlineto{\pgfqpoint{4.110794in}{0.426860in}}%
\pgfpathlineto{\pgfqpoint{4.108443in}{0.419972in}}%
\pgfpathlineto{\pgfqpoint{4.105714in}{0.419404in}}%
\pgfpathlineto{\pgfqpoint{4.103938in}{0.421634in}}%
\pgfpathlineto{\pgfqpoint{4.106010in}{0.431388in}}%
\pgfpathlineto{\pgfqpoint{4.103947in}{0.435475in}}%
\pgfpathlineto{\pgfqpoint{4.099950in}{0.440293in}}%
\pgfpathlineto{\pgfqpoint{4.101871in}{0.443969in}}%
\pgfpathlineto{\pgfqpoint{4.093436in}{0.444102in}}%
\pgfpathlineto{\pgfqpoint{4.093628in}{0.445528in}}%
\pgfpathlineto{\pgfqpoint{4.088595in}{0.447526in}}%
\pgfpathlineto{\pgfqpoint{4.087031in}{0.446426in}}%
\pgfpathclose%
\pgfusepath{fill}%
\end{pgfscope}%
\begin{pgfscope}%
\pgfpathrectangle{\pgfqpoint{3.625000in}{0.100000in}}{\pgfqpoint{2.989028in}{1.913466in}}%
\pgfusepath{clip}%
\pgfsetbuttcap%
\pgfsetmiterjoin%
\definecolor{currentfill}{rgb}{0.312034,0.662668,0.687659}%
\pgfsetfillcolor{currentfill}%
\pgfsetlinewidth{0.000000pt}%
\definecolor{currentstroke}{rgb}{0.000000,0.000000,0.000000}%
\pgfsetstrokecolor{currentstroke}%
\pgfsetstrokeopacity{0.000000}%
\pgfsetdash{}{0pt}%
\pgfpathmoveto{\pgfqpoint{4.071979in}{0.517466in}}%
\pgfpathlineto{\pgfqpoint{4.070936in}{0.515633in}}%
\pgfpathlineto{\pgfqpoint{4.073696in}{0.511730in}}%
\pgfpathlineto{\pgfqpoint{4.069334in}{0.508079in}}%
\pgfpathlineto{\pgfqpoint{4.069455in}{0.504939in}}%
\pgfpathlineto{\pgfqpoint{4.067483in}{0.504198in}}%
\pgfpathlineto{\pgfqpoint{4.060704in}{0.508881in}}%
\pgfpathlineto{\pgfqpoint{4.055872in}{0.519132in}}%
\pgfpathlineto{\pgfqpoint{4.055549in}{0.522094in}}%
\pgfpathlineto{\pgfqpoint{4.056791in}{0.525640in}}%
\pgfpathlineto{\pgfqpoint{4.062027in}{0.520216in}}%
\pgfpathlineto{\pgfqpoint{4.063628in}{0.521262in}}%
\pgfpathlineto{\pgfqpoint{4.068070in}{0.520267in}}%
\pgfpathlineto{\pgfqpoint{4.071979in}{0.517466in}}%
\pgfpathclose%
\pgfusepath{fill}%
\end{pgfscope}%
\begin{pgfscope}%
\pgfpathrectangle{\pgfqpoint{3.625000in}{0.100000in}}{\pgfqpoint{2.989028in}{1.913466in}}%
\pgfusepath{clip}%
\pgfsetbuttcap%
\pgfsetmiterjoin%
\definecolor{currentfill}{rgb}{0.312034,0.662668,0.687659}%
\pgfsetfillcolor{currentfill}%
\pgfsetlinewidth{0.000000pt}%
\definecolor{currentstroke}{rgb}{0.000000,0.000000,0.000000}%
\pgfsetstrokecolor{currentstroke}%
\pgfsetstrokeopacity{0.000000}%
\pgfsetdash{}{0pt}%
\pgfpathmoveto{\pgfqpoint{3.991584in}{0.398705in}}%
\pgfpathlineto{\pgfqpoint{3.986820in}{0.397932in}}%
\pgfpathlineto{\pgfqpoint{3.983372in}{0.399525in}}%
\pgfpathlineto{\pgfqpoint{3.981865in}{0.402120in}}%
\pgfpathlineto{\pgfqpoint{3.983573in}{0.405842in}}%
\pgfpathlineto{\pgfqpoint{3.987377in}{0.404858in}}%
\pgfpathlineto{\pgfqpoint{3.993630in}{0.407066in}}%
\pgfpathlineto{\pgfqpoint{3.995820in}{0.404016in}}%
\pgfpathlineto{\pgfqpoint{3.997944in}{0.404579in}}%
\pgfpathlineto{\pgfqpoint{4.003084in}{0.402681in}}%
\pgfpathlineto{\pgfqpoint{4.005305in}{0.400269in}}%
\pgfpathlineto{\pgfqpoint{4.002595in}{0.393963in}}%
\pgfpathlineto{\pgfqpoint{3.999899in}{0.392626in}}%
\pgfpathlineto{\pgfqpoint{3.997476in}{0.393346in}}%
\pgfpathlineto{\pgfqpoint{3.991584in}{0.398705in}}%
\pgfpathclose%
\pgfusepath{fill}%
\end{pgfscope}%
\begin{pgfscope}%
\pgfpathrectangle{\pgfqpoint{3.625000in}{0.100000in}}{\pgfqpoint{2.989028in}{1.913466in}}%
\pgfusepath{clip}%
\pgfsetbuttcap%
\pgfsetmiterjoin%
\definecolor{currentfill}{rgb}{0.312034,0.662668,0.687659}%
\pgfsetfillcolor{currentfill}%
\pgfsetlinewidth{0.000000pt}%
\definecolor{currentstroke}{rgb}{0.000000,0.000000,0.000000}%
\pgfsetstrokecolor{currentstroke}%
\pgfsetstrokeopacity{0.000000}%
\pgfsetdash{}{0pt}%
\pgfpathmoveto{\pgfqpoint{3.948985in}{0.404436in}}%
\pgfpathlineto{\pgfqpoint{3.942234in}{0.407656in}}%
\pgfpathlineto{\pgfqpoint{3.935987in}{0.408614in}}%
\pgfpathlineto{\pgfqpoint{3.934018in}{0.409985in}}%
\pgfpathlineto{\pgfqpoint{3.936260in}{0.412685in}}%
\pgfpathlineto{\pgfqpoint{3.940239in}{0.409312in}}%
\pgfpathlineto{\pgfqpoint{3.943576in}{0.409777in}}%
\pgfpathlineto{\pgfqpoint{3.949004in}{0.412113in}}%
\pgfpathlineto{\pgfqpoint{3.949442in}{0.415120in}}%
\pgfpathlineto{\pgfqpoint{3.956109in}{0.413632in}}%
\pgfpathlineto{\pgfqpoint{3.954548in}{0.409103in}}%
\pgfpathlineto{\pgfqpoint{3.958794in}{0.408772in}}%
\pgfpathlineto{\pgfqpoint{3.955231in}{0.403385in}}%
\pgfpathlineto{\pgfqpoint{3.951781in}{0.405131in}}%
\pgfpathlineto{\pgfqpoint{3.948985in}{0.404436in}}%
\pgfpathclose%
\pgfusepath{fill}%
\end{pgfscope}%
\begin{pgfscope}%
\pgfpathrectangle{\pgfqpoint{3.625000in}{0.100000in}}{\pgfqpoint{2.989028in}{1.913466in}}%
\pgfusepath{clip}%
\pgfsetbuttcap%
\pgfsetmiterjoin%
\definecolor{currentfill}{rgb}{0.312034,0.662668,0.687659}%
\pgfsetfillcolor{currentfill}%
\pgfsetlinewidth{0.000000pt}%
\definecolor{currentstroke}{rgb}{0.000000,0.000000,0.000000}%
\pgfsetstrokecolor{currentstroke}%
\pgfsetstrokeopacity{0.000000}%
\pgfsetdash{}{0pt}%
\pgfpathmoveto{\pgfqpoint{3.937124in}{0.415568in}}%
\pgfpathlineto{\pgfqpoint{3.934643in}{0.414219in}}%
\pgfpathlineto{\pgfqpoint{3.927882in}{0.416293in}}%
\pgfpathlineto{\pgfqpoint{3.922830in}{0.415092in}}%
\pgfpathlineto{\pgfqpoint{3.920264in}{0.418504in}}%
\pgfpathlineto{\pgfqpoint{3.921383in}{0.420066in}}%
\pgfpathlineto{\pgfqpoint{3.925212in}{0.420327in}}%
\pgfpathlineto{\pgfqpoint{3.928140in}{0.419346in}}%
\pgfpathlineto{\pgfqpoint{3.931331in}{0.420898in}}%
\pgfpathlineto{\pgfqpoint{3.936223in}{0.418823in}}%
\pgfpathlineto{\pgfqpoint{3.937124in}{0.415568in}}%
\pgfpathclose%
\pgfusepath{fill}%
\end{pgfscope}%
\begin{pgfscope}%
\pgfpathrectangle{\pgfqpoint{3.625000in}{0.100000in}}{\pgfqpoint{2.989028in}{1.913466in}}%
\pgfusepath{clip}%
\pgfsetbuttcap%
\pgfsetmiterjoin%
\definecolor{currentfill}{rgb}{0.312034,0.662668,0.687659}%
\pgfsetfillcolor{currentfill}%
\pgfsetlinewidth{0.000000pt}%
\definecolor{currentstroke}{rgb}{0.000000,0.000000,0.000000}%
\pgfsetstrokecolor{currentstroke}%
\pgfsetstrokeopacity{0.000000}%
\pgfsetdash{}{0pt}%
\pgfpathmoveto{\pgfqpoint{3.857429in}{0.465782in}}%
\pgfpathlineto{\pgfqpoint{3.857250in}{0.462436in}}%
\pgfpathlineto{\pgfqpoint{3.852314in}{0.460334in}}%
\pgfpathlineto{\pgfqpoint{3.847807in}{0.465597in}}%
\pgfpathlineto{\pgfqpoint{3.852862in}{0.467350in}}%
\pgfpathlineto{\pgfqpoint{3.857429in}{0.465782in}}%
\pgfpathclose%
\pgfusepath{fill}%
\end{pgfscope}%
\begin{pgfscope}%
\pgfpathrectangle{\pgfqpoint{3.625000in}{0.100000in}}{\pgfqpoint{2.989028in}{1.913466in}}%
\pgfusepath{clip}%
\pgfsetbuttcap%
\pgfsetmiterjoin%
\definecolor{currentfill}{rgb}{0.312034,0.662668,0.687659}%
\pgfsetfillcolor{currentfill}%
\pgfsetlinewidth{0.000000pt}%
\definecolor{currentstroke}{rgb}{0.000000,0.000000,0.000000}%
\pgfsetstrokecolor{currentstroke}%
\pgfsetstrokeopacity{0.000000}%
\pgfsetdash{}{0pt}%
\pgfpathmoveto{\pgfqpoint{3.816722in}{0.484589in}}%
\pgfpathlineto{\pgfqpoint{3.817771in}{0.486043in}}%
\pgfpathlineto{\pgfqpoint{3.823825in}{0.485792in}}%
\pgfpathlineto{\pgfqpoint{3.824887in}{0.487956in}}%
\pgfpathlineto{\pgfqpoint{3.827543in}{0.486068in}}%
\pgfpathlineto{\pgfqpoint{3.826009in}{0.482435in}}%
\pgfpathlineto{\pgfqpoint{3.823753in}{0.482127in}}%
\pgfpathlineto{\pgfqpoint{3.818361in}{0.483267in}}%
\pgfpathlineto{\pgfqpoint{3.816722in}{0.484589in}}%
\pgfpathclose%
\pgfusepath{fill}%
\end{pgfscope}%
\begin{pgfscope}%
\pgfpathrectangle{\pgfqpoint{3.625000in}{0.100000in}}{\pgfqpoint{2.989028in}{1.913466in}}%
\pgfusepath{clip}%
\pgfsetbuttcap%
\pgfsetmiterjoin%
\definecolor{currentfill}{rgb}{0.312034,0.662668,0.687659}%
\pgfsetfillcolor{currentfill}%
\pgfsetlinewidth{0.000000pt}%
\definecolor{currentstroke}{rgb}{0.000000,0.000000,0.000000}%
\pgfsetstrokecolor{currentstroke}%
\pgfsetstrokeopacity{0.000000}%
\pgfsetdash{}{0pt}%
\pgfpathmoveto{\pgfqpoint{3.808137in}{0.495848in}}%
\pgfpathlineto{\pgfqpoint{3.807770in}{0.499125in}}%
\pgfpathlineto{\pgfqpoint{3.810929in}{0.498882in}}%
\pgfpathlineto{\pgfqpoint{3.810709in}{0.503484in}}%
\pgfpathlineto{\pgfqpoint{3.813886in}{0.501000in}}%
\pgfpathlineto{\pgfqpoint{3.812787in}{0.497411in}}%
\pgfpathlineto{\pgfqpoint{3.808137in}{0.495848in}}%
\pgfpathclose%
\pgfusepath{fill}%
\end{pgfscope}%
\begin{pgfscope}%
\pgfpathrectangle{\pgfqpoint{3.625000in}{0.100000in}}{\pgfqpoint{2.989028in}{1.913466in}}%
\pgfusepath{clip}%
\pgfsetbuttcap%
\pgfsetmiterjoin%
\definecolor{currentfill}{rgb}{0.312034,0.662668,0.687659}%
\pgfsetfillcolor{currentfill}%
\pgfsetlinewidth{0.000000pt}%
\definecolor{currentstroke}{rgb}{0.000000,0.000000,0.000000}%
\pgfsetstrokecolor{currentstroke}%
\pgfsetstrokeopacity{0.000000}%
\pgfsetdash{}{0pt}%
\pgfpathmoveto{\pgfqpoint{4.086438in}{0.612633in}}%
\pgfpathlineto{\pgfqpoint{4.081400in}{0.613721in}}%
\pgfpathlineto{\pgfqpoint{4.080318in}{0.617088in}}%
\pgfpathlineto{\pgfqpoint{4.082049in}{0.620453in}}%
\pgfpathlineto{\pgfqpoint{4.085927in}{0.622211in}}%
\pgfpathlineto{\pgfqpoint{4.090367in}{0.613572in}}%
\pgfpathlineto{\pgfqpoint{4.094395in}{0.613189in}}%
\pgfpathlineto{\pgfqpoint{4.097399in}{0.610523in}}%
\pgfpathlineto{\pgfqpoint{4.097340in}{0.606855in}}%
\pgfpathlineto{\pgfqpoint{4.095143in}{0.605097in}}%
\pgfpathlineto{\pgfqpoint{4.098716in}{0.594249in}}%
\pgfpathlineto{\pgfqpoint{4.102625in}{0.590269in}}%
\pgfpathlineto{\pgfqpoint{4.098628in}{0.589006in}}%
\pgfpathlineto{\pgfqpoint{4.095393in}{0.593472in}}%
\pgfpathlineto{\pgfqpoint{4.088363in}{0.592089in}}%
\pgfpathlineto{\pgfqpoint{4.090549in}{0.596509in}}%
\pgfpathlineto{\pgfqpoint{4.089531in}{0.598269in}}%
\pgfpathlineto{\pgfqpoint{4.090128in}{0.602876in}}%
\pgfpathlineto{\pgfqpoint{4.088735in}{0.611204in}}%
\pgfpathlineto{\pgfqpoint{4.086438in}{0.612633in}}%
\pgfpathclose%
\pgfusepath{fill}%
\end{pgfscope}%
\begin{pgfscope}%
\pgfpathrectangle{\pgfqpoint{3.625000in}{0.100000in}}{\pgfqpoint{2.989028in}{1.913466in}}%
\pgfusepath{clip}%
\pgfsetbuttcap%
\pgfsetmiterjoin%
\definecolor{currentfill}{rgb}{0.312034,0.662668,0.687659}%
\pgfsetfillcolor{currentfill}%
\pgfsetlinewidth{0.000000pt}%
\definecolor{currentstroke}{rgb}{0.000000,0.000000,0.000000}%
\pgfsetstrokecolor{currentstroke}%
\pgfsetstrokeopacity{0.000000}%
\pgfsetdash{}{0pt}%
\pgfpathmoveto{\pgfqpoint{4.263441in}{0.364487in}}%
\pgfpathlineto{\pgfqpoint{4.256138in}{0.364569in}}%
\pgfpathlineto{\pgfqpoint{4.256700in}{0.368235in}}%
\pgfpathlineto{\pgfqpoint{4.259623in}{0.369546in}}%
\pgfpathlineto{\pgfqpoint{4.263441in}{0.364487in}}%
\pgfpathclose%
\pgfusepath{fill}%
\end{pgfscope}%
\begin{pgfscope}%
\pgfpathrectangle{\pgfqpoint{3.625000in}{0.100000in}}{\pgfqpoint{2.989028in}{1.913466in}}%
\pgfusepath{clip}%
\pgfsetbuttcap%
\pgfsetmiterjoin%
\definecolor{currentfill}{rgb}{0.312034,0.662668,0.687659}%
\pgfsetfillcolor{currentfill}%
\pgfsetlinewidth{0.000000pt}%
\definecolor{currentstroke}{rgb}{0.000000,0.000000,0.000000}%
\pgfsetstrokecolor{currentstroke}%
\pgfsetstrokeopacity{0.000000}%
\pgfsetdash{}{0pt}%
\pgfpathmoveto{\pgfqpoint{4.253306in}{0.367807in}}%
\pgfpathlineto{\pgfqpoint{4.248074in}{0.366541in}}%
\pgfpathlineto{\pgfqpoint{4.243102in}{0.364373in}}%
\pgfpathlineto{\pgfqpoint{4.241567in}{0.366927in}}%
\pgfpathlineto{\pgfqpoint{4.250117in}{0.368657in}}%
\pgfpathlineto{\pgfqpoint{4.253306in}{0.367807in}}%
\pgfpathclose%
\pgfusepath{fill}%
\end{pgfscope}%
\begin{pgfscope}%
\pgfpathrectangle{\pgfqpoint{3.625000in}{0.100000in}}{\pgfqpoint{2.989028in}{1.913466in}}%
\pgfusepath{clip}%
\pgfsetbuttcap%
\pgfsetmiterjoin%
\definecolor{currentfill}{rgb}{0.312034,0.662668,0.687659}%
\pgfsetfillcolor{currentfill}%
\pgfsetlinewidth{0.000000pt}%
\definecolor{currentstroke}{rgb}{0.000000,0.000000,0.000000}%
\pgfsetstrokecolor{currentstroke}%
\pgfsetstrokeopacity{0.000000}%
\pgfsetdash{}{0pt}%
\pgfpathmoveto{\pgfqpoint{4.173103in}{0.365212in}}%
\pgfpathlineto{\pgfqpoint{4.173303in}{0.362676in}}%
\pgfpathlineto{\pgfqpoint{4.170288in}{0.361100in}}%
\pgfpathlineto{\pgfqpoint{4.161875in}{0.364714in}}%
\pgfpathlineto{\pgfqpoint{4.160570in}{0.363261in}}%
\pgfpathlineto{\pgfqpoint{4.158535in}{0.370138in}}%
\pgfpathlineto{\pgfqpoint{4.162750in}{0.371095in}}%
\pgfpathlineto{\pgfqpoint{4.164698in}{0.368269in}}%
\pgfpathlineto{\pgfqpoint{4.168929in}{0.371807in}}%
\pgfpathlineto{\pgfqpoint{4.173103in}{0.365212in}}%
\pgfpathclose%
\pgfusepath{fill}%
\end{pgfscope}%
\begin{pgfscope}%
\pgfpathrectangle{\pgfqpoint{3.625000in}{0.100000in}}{\pgfqpoint{2.989028in}{1.913466in}}%
\pgfusepath{clip}%
\pgfsetbuttcap%
\pgfsetmiterjoin%
\definecolor{currentfill}{rgb}{0.312034,0.662668,0.687659}%
\pgfsetfillcolor{currentfill}%
\pgfsetlinewidth{0.000000pt}%
\definecolor{currentstroke}{rgb}{0.000000,0.000000,0.000000}%
\pgfsetstrokecolor{currentstroke}%
\pgfsetstrokeopacity{0.000000}%
\pgfsetdash{}{0pt}%
\pgfpathmoveto{\pgfqpoint{4.362890in}{0.224669in}}%
\pgfpathlineto{\pgfqpoint{4.355835in}{0.222042in}}%
\pgfpathlineto{\pgfqpoint{4.352098in}{0.221885in}}%
\pgfpathlineto{\pgfqpoint{4.355139in}{0.227457in}}%
\pgfpathlineto{\pgfqpoint{4.357738in}{0.229535in}}%
\pgfpathlineto{\pgfqpoint{4.357654in}{0.235306in}}%
\pgfpathlineto{\pgfqpoint{4.360479in}{0.245723in}}%
\pgfpathlineto{\pgfqpoint{4.363081in}{0.247702in}}%
\pgfpathlineto{\pgfqpoint{4.369087in}{0.244837in}}%
\pgfpathlineto{\pgfqpoint{4.368210in}{0.243196in}}%
\pgfpathlineto{\pgfqpoint{4.368665in}{0.235310in}}%
\pgfpathlineto{\pgfqpoint{4.366718in}{0.233182in}}%
\pgfpathlineto{\pgfqpoint{4.365881in}{0.227581in}}%
\pgfpathlineto{\pgfqpoint{4.362890in}{0.224669in}}%
\pgfpathclose%
\pgfusepath{fill}%
\end{pgfscope}%
\begin{pgfscope}%
\pgfpathrectangle{\pgfqpoint{3.625000in}{0.100000in}}{\pgfqpoint{2.989028in}{1.913466in}}%
\pgfusepath{clip}%
\pgfsetbuttcap%
\pgfsetmiterjoin%
\definecolor{currentfill}{rgb}{0.312034,0.662668,0.687659}%
\pgfsetfillcolor{currentfill}%
\pgfsetlinewidth{0.000000pt}%
\definecolor{currentstroke}{rgb}{0.000000,0.000000,0.000000}%
\pgfsetstrokecolor{currentstroke}%
\pgfsetstrokeopacity{0.000000}%
\pgfsetdash{}{0pt}%
\pgfpathmoveto{\pgfqpoint{4.347266in}{0.250492in}}%
\pgfpathlineto{\pgfqpoint{4.351399in}{0.243355in}}%
\pgfpathlineto{\pgfqpoint{4.350633in}{0.241653in}}%
\pgfpathlineto{\pgfqpoint{4.356161in}{0.239922in}}%
\pgfpathlineto{\pgfqpoint{4.354574in}{0.233362in}}%
\pgfpathlineto{\pgfqpoint{4.352167in}{0.233665in}}%
\pgfpathlineto{\pgfqpoint{4.345842in}{0.245101in}}%
\pgfpathlineto{\pgfqpoint{4.346742in}{0.239954in}}%
\pgfpathlineto{\pgfqpoint{4.345664in}{0.236990in}}%
\pgfpathlineto{\pgfqpoint{4.342986in}{0.235671in}}%
\pgfpathlineto{\pgfqpoint{4.341463in}{0.237092in}}%
\pgfpathlineto{\pgfqpoint{4.340613in}{0.243682in}}%
\pgfpathlineto{\pgfqpoint{4.341115in}{0.247956in}}%
\pgfpathlineto{\pgfqpoint{4.339331in}{0.249822in}}%
\pgfpathlineto{\pgfqpoint{4.344043in}{0.255220in}}%
\pgfpathlineto{\pgfqpoint{4.347135in}{0.256822in}}%
\pgfpathlineto{\pgfqpoint{4.348165in}{0.254717in}}%
\pgfpathlineto{\pgfqpoint{4.351391in}{0.256346in}}%
\pgfpathlineto{\pgfqpoint{4.353728in}{0.251922in}}%
\pgfpathlineto{\pgfqpoint{4.358783in}{0.246295in}}%
\pgfpathlineto{\pgfqpoint{4.358093in}{0.243746in}}%
\pgfpathlineto{\pgfqpoint{4.355286in}{0.240969in}}%
\pgfpathlineto{\pgfqpoint{4.352691in}{0.242277in}}%
\pgfpathlineto{\pgfqpoint{4.347266in}{0.250492in}}%
\pgfpathclose%
\pgfusepath{fill}%
\end{pgfscope}%
\begin{pgfscope}%
\pgfpathrectangle{\pgfqpoint{3.625000in}{0.100000in}}{\pgfqpoint{2.989028in}{1.913466in}}%
\pgfusepath{clip}%
\pgfsetbuttcap%
\pgfsetmiterjoin%
\definecolor{currentfill}{rgb}{0.312034,0.662668,0.687659}%
\pgfsetfillcolor{currentfill}%
\pgfsetlinewidth{0.000000pt}%
\definecolor{currentstroke}{rgb}{0.000000,0.000000,0.000000}%
\pgfsetstrokecolor{currentstroke}%
\pgfsetstrokeopacity{0.000000}%
\pgfsetdash{}{0pt}%
\pgfpathmoveto{\pgfqpoint{4.142161in}{0.351684in}}%
\pgfpathlineto{\pgfqpoint{4.137041in}{0.351018in}}%
\pgfpathlineto{\pgfqpoint{4.129830in}{0.348246in}}%
\pgfpathlineto{\pgfqpoint{4.128148in}{0.349752in}}%
\pgfpathlineto{\pgfqpoint{4.135680in}{0.351454in}}%
\pgfpathlineto{\pgfqpoint{4.133661in}{0.352939in}}%
\pgfpathlineto{\pgfqpoint{4.128991in}{0.354027in}}%
\pgfpathlineto{\pgfqpoint{4.127744in}{0.357507in}}%
\pgfpathlineto{\pgfqpoint{4.130094in}{0.360506in}}%
\pgfpathlineto{\pgfqpoint{4.132073in}{0.367387in}}%
\pgfpathlineto{\pgfqpoint{4.139301in}{0.368669in}}%
\pgfpathlineto{\pgfqpoint{4.142801in}{0.365696in}}%
\pgfpathlineto{\pgfqpoint{4.146088in}{0.368573in}}%
\pgfpathlineto{\pgfqpoint{4.150211in}{0.367867in}}%
\pgfpathlineto{\pgfqpoint{4.153174in}{0.362553in}}%
\pgfpathlineto{\pgfqpoint{4.155683in}{0.367277in}}%
\pgfpathlineto{\pgfqpoint{4.162804in}{0.356228in}}%
\pgfpathlineto{\pgfqpoint{4.160232in}{0.355345in}}%
\pgfpathlineto{\pgfqpoint{4.158653in}{0.352950in}}%
\pgfpathlineto{\pgfqpoint{4.161636in}{0.350764in}}%
\pgfpathlineto{\pgfqpoint{4.157011in}{0.348715in}}%
\pgfpathlineto{\pgfqpoint{4.153656in}{0.349843in}}%
\pgfpathlineto{\pgfqpoint{4.147321in}{0.353583in}}%
\pgfpathlineto{\pgfqpoint{4.142161in}{0.351684in}}%
\pgfpathclose%
\pgfusepath{fill}%
\end{pgfscope}%
\begin{pgfscope}%
\pgfpathrectangle{\pgfqpoint{3.625000in}{0.100000in}}{\pgfqpoint{2.989028in}{1.913466in}}%
\pgfusepath{clip}%
\pgfsetbuttcap%
\pgfsetmiterjoin%
\definecolor{currentfill}{rgb}{0.312034,0.662668,0.687659}%
\pgfsetfillcolor{currentfill}%
\pgfsetlinewidth{0.000000pt}%
\definecolor{currentstroke}{rgb}{0.000000,0.000000,0.000000}%
\pgfsetstrokecolor{currentstroke}%
\pgfsetstrokeopacity{0.000000}%
\pgfsetdash{}{0pt}%
\pgfpathmoveto{\pgfqpoint{4.342967in}{0.203589in}}%
\pgfpathlineto{\pgfqpoint{4.341952in}{0.213793in}}%
\pgfpathlineto{\pgfqpoint{4.344688in}{0.217539in}}%
\pgfpathlineto{\pgfqpoint{4.341762in}{0.217800in}}%
\pgfpathlineto{\pgfqpoint{4.342832in}{0.225086in}}%
\pgfpathlineto{\pgfqpoint{4.344093in}{0.229412in}}%
\pgfpathlineto{\pgfqpoint{4.342997in}{0.235224in}}%
\pgfpathlineto{\pgfqpoint{4.345677in}{0.236349in}}%
\pgfpathlineto{\pgfqpoint{4.346551in}{0.237852in}}%
\pgfpathlineto{\pgfqpoint{4.349191in}{0.237282in}}%
\pgfpathlineto{\pgfqpoint{4.351476in}{0.232626in}}%
\pgfpathlineto{\pgfqpoint{4.351901in}{0.228360in}}%
\pgfpathlineto{\pgfqpoint{4.349105in}{0.215583in}}%
\pgfpathlineto{\pgfqpoint{4.342967in}{0.203589in}}%
\pgfpathclose%
\pgfusepath{fill}%
\end{pgfscope}%
\begin{pgfscope}%
\pgfpathrectangle{\pgfqpoint{3.625000in}{0.100000in}}{\pgfqpoint{2.989028in}{1.913466in}}%
\pgfusepath{clip}%
\pgfsetbuttcap%
\pgfsetmiterjoin%
\definecolor{currentfill}{rgb}{0.312034,0.662668,0.687659}%
\pgfsetfillcolor{currentfill}%
\pgfsetlinewidth{0.000000pt}%
\definecolor{currentstroke}{rgb}{0.000000,0.000000,0.000000}%
\pgfsetstrokecolor{currentstroke}%
\pgfsetstrokeopacity{0.000000}%
\pgfsetdash{}{0pt}%
\pgfpathmoveto{\pgfqpoint{4.342360in}{0.234737in}}%
\pgfpathlineto{\pgfqpoint{4.342219in}{0.229762in}}%
\pgfpathlineto{\pgfqpoint{4.339925in}{0.227503in}}%
\pgfpathlineto{\pgfqpoint{4.337273in}{0.228338in}}%
\pgfpathlineto{\pgfqpoint{4.340625in}{0.235565in}}%
\pgfpathlineto{\pgfqpoint{4.342360in}{0.234737in}}%
\pgfpathclose%
\pgfusepath{fill}%
\end{pgfscope}%
\begin{pgfscope}%
\pgfpathrectangle{\pgfqpoint{3.625000in}{0.100000in}}{\pgfqpoint{2.989028in}{1.913466in}}%
\pgfusepath{clip}%
\pgfsetbuttcap%
\pgfsetmiterjoin%
\definecolor{currentfill}{rgb}{0.312034,0.662668,0.687659}%
\pgfsetfillcolor{currentfill}%
\pgfsetlinewidth{0.000000pt}%
\definecolor{currentstroke}{rgb}{0.000000,0.000000,0.000000}%
\pgfsetstrokecolor{currentstroke}%
\pgfsetstrokeopacity{0.000000}%
\pgfsetdash{}{0pt}%
\pgfpathmoveto{\pgfqpoint{4.370513in}{0.212648in}}%
\pgfpathlineto{\pgfqpoint{4.368557in}{0.204322in}}%
\pgfpathlineto{\pgfqpoint{4.363998in}{0.201690in}}%
\pgfpathlineto{\pgfqpoint{4.358017in}{0.204115in}}%
\pgfpathlineto{\pgfqpoint{4.359665in}{0.207350in}}%
\pgfpathlineto{\pgfqpoint{4.359632in}{0.209591in}}%
\pgfpathlineto{\pgfqpoint{4.361855in}{0.213398in}}%
\pgfpathlineto{\pgfqpoint{4.360292in}{0.212632in}}%
\pgfpathlineto{\pgfqpoint{4.361432in}{0.214491in}}%
\pgfpathlineto{\pgfqpoint{4.359267in}{0.218715in}}%
\pgfpathlineto{\pgfqpoint{4.361723in}{0.219493in}}%
\pgfpathlineto{\pgfqpoint{4.367410in}{0.214551in}}%
\pgfpathlineto{\pgfqpoint{4.370513in}{0.212648in}}%
\pgfpathclose%
\pgfusepath{fill}%
\end{pgfscope}%
\begin{pgfscope}%
\pgfpathrectangle{\pgfqpoint{3.625000in}{0.100000in}}{\pgfqpoint{2.989028in}{1.913466in}}%
\pgfusepath{clip}%
\pgfsetbuttcap%
\pgfsetmiterjoin%
\definecolor{currentfill}{rgb}{0.312034,0.662668,0.687659}%
\pgfsetfillcolor{currentfill}%
\pgfsetlinewidth{0.000000pt}%
\definecolor{currentstroke}{rgb}{0.000000,0.000000,0.000000}%
\pgfsetstrokecolor{currentstroke}%
\pgfsetstrokeopacity{0.000000}%
\pgfsetdash{}{0pt}%
\pgfpathmoveto{\pgfqpoint{4.350557in}{0.197926in}}%
\pgfpathlineto{\pgfqpoint{4.347979in}{0.196923in}}%
\pgfpathlineto{\pgfqpoint{4.349451in}{0.205709in}}%
\pgfpathlineto{\pgfqpoint{4.352546in}{0.207760in}}%
\pgfpathlineto{\pgfqpoint{4.351791in}{0.214343in}}%
\pgfpathlineto{\pgfqpoint{4.353049in}{0.217262in}}%
\pgfpathlineto{\pgfqpoint{4.355536in}{0.218356in}}%
\pgfpathlineto{\pgfqpoint{4.358148in}{0.215493in}}%
\pgfpathlineto{\pgfqpoint{4.360389in}{0.211607in}}%
\pgfpathlineto{\pgfqpoint{4.353184in}{0.203995in}}%
\pgfpathlineto{\pgfqpoint{4.350557in}{0.197926in}}%
\pgfpathclose%
\pgfusepath{fill}%
\end{pgfscope}%
\begin{pgfscope}%
\pgfpathrectangle{\pgfqpoint{3.625000in}{0.100000in}}{\pgfqpoint{2.989028in}{1.913466in}}%
\pgfusepath{clip}%
\pgfsetbuttcap%
\pgfsetmiterjoin%
\definecolor{currentfill}{rgb}{0.312034,0.662668,0.687659}%
\pgfsetfillcolor{currentfill}%
\pgfsetlinewidth{0.000000pt}%
\definecolor{currentstroke}{rgb}{0.000000,0.000000,0.000000}%
\pgfsetstrokecolor{currentstroke}%
\pgfsetstrokeopacity{0.000000}%
\pgfsetdash{}{0pt}%
\pgfpathmoveto{\pgfqpoint{4.371657in}{0.207083in}}%
\pgfpathlineto{\pgfqpoint{4.372921in}{0.201020in}}%
\pgfpathlineto{\pgfqpoint{4.367275in}{0.201714in}}%
\pgfpathlineto{\pgfqpoint{4.371657in}{0.207083in}}%
\pgfpathclose%
\pgfusepath{fill}%
\end{pgfscope}%
\begin{pgfscope}%
\pgfpathrectangle{\pgfqpoint{3.625000in}{0.100000in}}{\pgfqpoint{2.989028in}{1.913466in}}%
\pgfusepath{clip}%
\pgfsetbuttcap%
\pgfsetmiterjoin%
\definecolor{currentfill}{rgb}{0.312034,0.662668,0.687659}%
\pgfsetfillcolor{currentfill}%
\pgfsetlinewidth{0.000000pt}%
\definecolor{currentstroke}{rgb}{0.000000,0.000000,0.000000}%
\pgfsetstrokecolor{currentstroke}%
\pgfsetstrokeopacity{0.000000}%
\pgfsetdash{}{0pt}%
\pgfpathmoveto{\pgfqpoint{4.374351in}{0.197819in}}%
\pgfpathlineto{\pgfqpoint{4.375276in}{0.194079in}}%
\pgfpathlineto{\pgfqpoint{4.377047in}{0.192882in}}%
\pgfpathlineto{\pgfqpoint{4.376725in}{0.189362in}}%
\pgfpathlineto{\pgfqpoint{4.374251in}{0.188177in}}%
\pgfpathlineto{\pgfqpoint{4.372242in}{0.193433in}}%
\pgfpathlineto{\pgfqpoint{4.374351in}{0.197819in}}%
\pgfpathclose%
\pgfusepath{fill}%
\end{pgfscope}%
\begin{pgfscope}%
\pgfpathrectangle{\pgfqpoint{3.625000in}{0.100000in}}{\pgfqpoint{2.989028in}{1.913466in}}%
\pgfusepath{clip}%
\pgfsetbuttcap%
\pgfsetmiterjoin%
\definecolor{currentfill}{rgb}{0.312034,0.662668,0.687659}%
\pgfsetfillcolor{currentfill}%
\pgfsetlinewidth{0.000000pt}%
\definecolor{currentstroke}{rgb}{0.000000,0.000000,0.000000}%
\pgfsetstrokecolor{currentstroke}%
\pgfsetstrokeopacity{0.000000}%
\pgfsetdash{}{0pt}%
\pgfpathmoveto{\pgfqpoint{4.369640in}{0.199254in}}%
\pgfpathlineto{\pgfqpoint{4.368104in}{0.194909in}}%
\pgfpathlineto{\pgfqpoint{4.365719in}{0.195069in}}%
\pgfpathlineto{\pgfqpoint{4.364215in}{0.198652in}}%
\pgfpathlineto{\pgfqpoint{4.366665in}{0.200349in}}%
\pgfpathlineto{\pgfqpoint{4.369640in}{0.199254in}}%
\pgfpathclose%
\pgfusepath{fill}%
\end{pgfscope}%
\begin{pgfscope}%
\pgfpathrectangle{\pgfqpoint{3.625000in}{0.100000in}}{\pgfqpoint{2.989028in}{1.913466in}}%
\pgfusepath{clip}%
\pgfsetbuttcap%
\pgfsetmiterjoin%
\definecolor{currentfill}{rgb}{0.312034,0.662668,0.687659}%
\pgfsetfillcolor{currentfill}%
\pgfsetlinewidth{0.000000pt}%
\definecolor{currentstroke}{rgb}{0.000000,0.000000,0.000000}%
\pgfsetstrokecolor{currentstroke}%
\pgfsetstrokeopacity{0.000000}%
\pgfsetdash{}{0pt}%
\pgfpathmoveto{\pgfqpoint{4.363205in}{0.176854in}}%
\pgfpathlineto{\pgfqpoint{4.365333in}{0.174527in}}%
\pgfpathlineto{\pgfqpoint{4.365802in}{0.169622in}}%
\pgfpathlineto{\pgfqpoint{4.366922in}{0.168197in}}%
\pgfpathlineto{\pgfqpoint{4.365092in}{0.163758in}}%
\pgfpathlineto{\pgfqpoint{4.362686in}{0.161260in}}%
\pgfpathlineto{\pgfqpoint{4.361405in}{0.156623in}}%
\pgfpathlineto{\pgfqpoint{4.358774in}{0.163848in}}%
\pgfpathlineto{\pgfqpoint{4.358338in}{0.170408in}}%
\pgfpathlineto{\pgfqpoint{4.356865in}{0.173673in}}%
\pgfpathlineto{\pgfqpoint{4.351795in}{0.176321in}}%
\pgfpathlineto{\pgfqpoint{4.356551in}{0.181841in}}%
\pgfpathlineto{\pgfqpoint{4.353380in}{0.184201in}}%
\pgfpathlineto{\pgfqpoint{4.356174in}{0.186552in}}%
\pgfpathlineto{\pgfqpoint{4.359882in}{0.195579in}}%
\pgfpathlineto{\pgfqpoint{4.356038in}{0.198762in}}%
\pgfpathlineto{\pgfqpoint{4.357519in}{0.201818in}}%
\pgfpathlineto{\pgfqpoint{4.362487in}{0.199107in}}%
\pgfpathlineto{\pgfqpoint{4.362990in}{0.194829in}}%
\pgfpathlineto{\pgfqpoint{4.361111in}{0.192394in}}%
\pgfpathlineto{\pgfqpoint{4.364836in}{0.189098in}}%
\pgfpathlineto{\pgfqpoint{4.366009in}{0.183965in}}%
\pgfpathlineto{\pgfqpoint{4.365850in}{0.176765in}}%
\pgfpathlineto{\pgfqpoint{4.363205in}{0.176854in}}%
\pgfpathclose%
\pgfusepath{fill}%
\end{pgfscope}%
\begin{pgfscope}%
\pgfpathrectangle{\pgfqpoint{3.625000in}{0.100000in}}{\pgfqpoint{2.989028in}{1.913466in}}%
\pgfusepath{clip}%
\pgfsetbuttcap%
\pgfsetmiterjoin%
\definecolor{currentfill}{rgb}{0.312034,0.662668,0.687659}%
\pgfsetfillcolor{currentfill}%
\pgfsetlinewidth{0.000000pt}%
\definecolor{currentstroke}{rgb}{0.000000,0.000000,0.000000}%
\pgfsetstrokecolor{currentstroke}%
\pgfsetstrokeopacity{0.000000}%
\pgfsetdash{}{0pt}%
\pgfpathmoveto{\pgfqpoint{4.372302in}{0.194926in}}%
\pgfpathlineto{\pgfqpoint{4.371143in}{0.192429in}}%
\pgfpathlineto{\pgfqpoint{4.372231in}{0.188374in}}%
\pgfpathlineto{\pgfqpoint{4.369445in}{0.188033in}}%
\pgfpathlineto{\pgfqpoint{4.366524in}{0.189911in}}%
\pgfpathlineto{\pgfqpoint{4.367472in}{0.194046in}}%
\pgfpathlineto{\pgfqpoint{4.372302in}{0.194926in}}%
\pgfpathclose%
\pgfusepath{fill}%
\end{pgfscope}%
\begin{pgfscope}%
\pgfpathrectangle{\pgfqpoint{3.625000in}{0.100000in}}{\pgfqpoint{2.989028in}{1.913466in}}%
\pgfusepath{clip}%
\pgfsetbuttcap%
\pgfsetmiterjoin%
\definecolor{currentfill}{rgb}{0.312034,0.662668,0.687659}%
\pgfsetfillcolor{currentfill}%
\pgfsetlinewidth{0.000000pt}%
\definecolor{currentstroke}{rgb}{0.000000,0.000000,0.000000}%
\pgfsetstrokecolor{currentstroke}%
\pgfsetstrokeopacity{0.000000}%
\pgfsetdash{}{0pt}%
\pgfpathmoveto{\pgfqpoint{4.359316in}{0.195209in}}%
\pgfpathlineto{\pgfqpoint{4.353167in}{0.192451in}}%
\pgfpathlineto{\pgfqpoint{4.351076in}{0.193444in}}%
\pgfpathlineto{\pgfqpoint{4.355270in}{0.197082in}}%
\pgfpathlineto{\pgfqpoint{4.359316in}{0.195209in}}%
\pgfpathclose%
\pgfusepath{fill}%
\end{pgfscope}%
\begin{pgfscope}%
\pgfpathrectangle{\pgfqpoint{3.625000in}{0.100000in}}{\pgfqpoint{2.989028in}{1.913466in}}%
\pgfusepath{clip}%
\pgfsetbuttcap%
\pgfsetmiterjoin%
\definecolor{currentfill}{rgb}{0.312034,0.662668,0.687659}%
\pgfsetfillcolor{currentfill}%
\pgfsetlinewidth{0.000000pt}%
\definecolor{currentstroke}{rgb}{0.000000,0.000000,0.000000}%
\pgfsetstrokecolor{currentstroke}%
\pgfsetstrokeopacity{0.000000}%
\pgfsetdash{}{0pt}%
\pgfpathmoveto{\pgfqpoint{4.372369in}{0.168302in}}%
\pgfpathlineto{\pgfqpoint{4.370903in}{0.172455in}}%
\pgfpathlineto{\pgfqpoint{4.374004in}{0.173477in}}%
\pgfpathlineto{\pgfqpoint{4.374943in}{0.177955in}}%
\pgfpathlineto{\pgfqpoint{4.378303in}{0.177960in}}%
\pgfpathlineto{\pgfqpoint{4.378289in}{0.181079in}}%
\pgfpathlineto{\pgfqpoint{4.382959in}{0.180341in}}%
\pgfpathlineto{\pgfqpoint{4.383755in}{0.171887in}}%
\pgfpathlineto{\pgfqpoint{4.381046in}{0.166498in}}%
\pgfpathlineto{\pgfqpoint{4.377027in}{0.163131in}}%
\pgfpathlineto{\pgfqpoint{4.372369in}{0.168302in}}%
\pgfpathclose%
\pgfusepath{fill}%
\end{pgfscope}%
\begin{pgfscope}%
\pgfpathrectangle{\pgfqpoint{3.625000in}{0.100000in}}{\pgfqpoint{2.989028in}{1.913466in}}%
\pgfusepath{clip}%
\pgfsetbuttcap%
\pgfsetmiterjoin%
\definecolor{currentfill}{rgb}{0.312034,0.662668,0.687659}%
\pgfsetfillcolor{currentfill}%
\pgfsetlinewidth{0.000000pt}%
\definecolor{currentstroke}{rgb}{0.000000,0.000000,0.000000}%
\pgfsetstrokecolor{currentstroke}%
\pgfsetstrokeopacity{0.000000}%
\pgfsetdash{}{0pt}%
\pgfpathmoveto{\pgfqpoint{4.370553in}{0.171496in}}%
\pgfpathlineto{\pgfqpoint{4.371878in}{0.167564in}}%
\pgfpathlineto{\pgfqpoint{4.368612in}{0.167086in}}%
\pgfpathlineto{\pgfqpoint{4.370553in}{0.171496in}}%
\pgfpathclose%
\pgfusepath{fill}%
\end{pgfscope}%
\begin{pgfscope}%
\pgfpathrectangle{\pgfqpoint{3.625000in}{0.100000in}}{\pgfqpoint{2.989028in}{1.913466in}}%
\pgfusepath{clip}%
\pgfsetbuttcap%
\pgfsetmiterjoin%
\definecolor{currentfill}{rgb}{0.312034,0.662668,0.687659}%
\pgfsetfillcolor{currentfill}%
\pgfsetlinewidth{0.000000pt}%
\definecolor{currentstroke}{rgb}{0.000000,0.000000,0.000000}%
\pgfsetstrokecolor{currentstroke}%
\pgfsetstrokeopacity{0.000000}%
\pgfsetdash{}{0pt}%
\pgfpathmoveto{\pgfqpoint{4.373306in}{0.165713in}}%
\pgfpathlineto{\pgfqpoint{4.372851in}{0.160700in}}%
\pgfpathlineto{\pgfqpoint{4.369406in}{0.161116in}}%
\pgfpathlineto{\pgfqpoint{4.370520in}{0.163537in}}%
\pgfpathlineto{\pgfqpoint{4.373306in}{0.165713in}}%
\pgfpathclose%
\pgfusepath{fill}%
\end{pgfscope}%
\begin{pgfscope}%
\pgfsetbuttcap%
\pgfsetmiterjoin%
\definecolor{currentfill}{rgb}{1.000000,1.000000,1.000000}%
\pgfsetfillcolor{currentfill}%
\pgfsetlinewidth{1.003750pt}%
\definecolor{currentstroke}{rgb}{0.827451,0.827451,0.827451}%
\pgfsetstrokecolor{currentstroke}%
\pgfsetdash{}{0pt}%
\pgfpathmoveto{\pgfqpoint{4.540239in}{1.931644in}}%
\pgfpathlineto{\pgfqpoint{6.189614in}{1.931644in}}%
\pgfpathlineto{\pgfqpoint{6.189614in}{2.127644in}}%
\pgfpathlineto{\pgfqpoint{4.540239in}{2.127644in}}%
\pgfpathlineto{\pgfqpoint{4.540239in}{1.931644in}}%
\pgfpathclose%
\pgfusepath{stroke,fill}%
\end{pgfscope}%
\begin{pgfscope}%
\definecolor{textcolor}{rgb}{0.000000,0.000000,0.000000}%
\pgfsetstrokecolor{textcolor}%
\pgfsetfillcolor{textcolor}%
\pgftext[x=4.581489in,y=1.994331in,left,base]{\color{textcolor}\setmainfont{Lato}\rmfamily\fontsize{9.000000}{10.800000}\selectfont One-Year Growth, 2023 Q2*}%
\end{pgfscope}%
\begin{pgfscope}%
\pgfpathrectangle{\pgfqpoint{3.019583in}{0.169444in}}{\pgfqpoint{0.896708in}{1.339426in}}%
\pgfusepath{clip}%
\pgfsetbuttcap%
\pgfsetmiterjoin%
\definecolor{currentfill}{rgb}{0.619608,0.003922,0.258824}%
\pgfsetfillcolor{currentfill}%
\pgfsetlinewidth{0.000000pt}%
\definecolor{currentstroke}{rgb}{0.000000,0.000000,0.000000}%
\pgfsetstrokecolor{currentstroke}%
\pgfsetstrokeopacity{0.000000}%
\pgfsetdash{}{0pt}%
\pgfpathmoveto{\pgfqpoint{3.279629in}{0.169444in}}%
\pgfpathlineto{\pgfqpoint{3.243760in}{0.169444in}}%
\pgfpathlineto{\pgfqpoint{3.243760in}{0.190373in}}%
\pgfpathlineto{\pgfqpoint{3.279629in}{0.190373in}}%
\pgfpathlineto{\pgfqpoint{3.279629in}{0.169444in}}%
\pgfpathclose%
\pgfusepath{fill}%
\end{pgfscope}%
\begin{pgfscope}%
\pgfpathrectangle{\pgfqpoint{3.019583in}{0.169444in}}{\pgfqpoint{0.896708in}{1.339426in}}%
\pgfusepath{clip}%
\pgfsetbuttcap%
\pgfsetmiterjoin%
\definecolor{currentfill}{rgb}{0.653441,0.041446,0.266820}%
\pgfsetfillcolor{currentfill}%
\pgfsetlinewidth{0.000000pt}%
\definecolor{currentstroke}{rgb}{0.000000,0.000000,0.000000}%
\pgfsetstrokecolor{currentstroke}%
\pgfsetstrokeopacity{0.000000}%
\pgfsetdash{}{0pt}%
\pgfpathmoveto{\pgfqpoint{3.279629in}{0.190373in}}%
\pgfpathlineto{\pgfqpoint{3.243760in}{0.190373in}}%
\pgfpathlineto{\pgfqpoint{3.243760in}{0.211302in}}%
\pgfpathlineto{\pgfqpoint{3.279629in}{0.211302in}}%
\pgfpathlineto{\pgfqpoint{3.279629in}{0.190373in}}%
\pgfpathclose%
\pgfusepath{fill}%
\end{pgfscope}%
\begin{pgfscope}%
\pgfpathrectangle{\pgfqpoint{3.019583in}{0.169444in}}{\pgfqpoint{0.896708in}{1.339426in}}%
\pgfusepath{clip}%
\pgfsetbuttcap%
\pgfsetmiterjoin%
\definecolor{currentfill}{rgb}{0.687274,0.078970,0.274817}%
\pgfsetfillcolor{currentfill}%
\pgfsetlinewidth{0.000000pt}%
\definecolor{currentstroke}{rgb}{0.000000,0.000000,0.000000}%
\pgfsetstrokecolor{currentstroke}%
\pgfsetstrokeopacity{0.000000}%
\pgfsetdash{}{0pt}%
\pgfpathmoveto{\pgfqpoint{3.279629in}{0.211302in}}%
\pgfpathlineto{\pgfqpoint{3.243760in}{0.211302in}}%
\pgfpathlineto{\pgfqpoint{3.243760in}{0.232230in}}%
\pgfpathlineto{\pgfqpoint{3.279629in}{0.232230in}}%
\pgfpathlineto{\pgfqpoint{3.279629in}{0.211302in}}%
\pgfpathclose%
\pgfusepath{fill}%
\end{pgfscope}%
\begin{pgfscope}%
\pgfpathrectangle{\pgfqpoint{3.019583in}{0.169444in}}{\pgfqpoint{0.896708in}{1.339426in}}%
\pgfusepath{clip}%
\pgfsetbuttcap%
\pgfsetmiterjoin%
\definecolor{currentfill}{rgb}{0.721107,0.116494,0.282814}%
\pgfsetfillcolor{currentfill}%
\pgfsetlinewidth{0.000000pt}%
\definecolor{currentstroke}{rgb}{0.000000,0.000000,0.000000}%
\pgfsetstrokecolor{currentstroke}%
\pgfsetstrokeopacity{0.000000}%
\pgfsetdash{}{0pt}%
\pgfpathmoveto{\pgfqpoint{3.279629in}{0.232230in}}%
\pgfpathlineto{\pgfqpoint{3.243760in}{0.232230in}}%
\pgfpathlineto{\pgfqpoint{3.243760in}{0.253159in}}%
\pgfpathlineto{\pgfqpoint{3.279629in}{0.253159in}}%
\pgfpathlineto{\pgfqpoint{3.279629in}{0.232230in}}%
\pgfpathclose%
\pgfusepath{fill}%
\end{pgfscope}%
\begin{pgfscope}%
\pgfpathrectangle{\pgfqpoint{3.019583in}{0.169444in}}{\pgfqpoint{0.896708in}{1.339426in}}%
\pgfusepath{clip}%
\pgfsetbuttcap%
\pgfsetmiterjoin%
\definecolor{currentfill}{rgb}{0.754940,0.154018,0.290811}%
\pgfsetfillcolor{currentfill}%
\pgfsetlinewidth{0.000000pt}%
\definecolor{currentstroke}{rgb}{0.000000,0.000000,0.000000}%
\pgfsetstrokecolor{currentstroke}%
\pgfsetstrokeopacity{0.000000}%
\pgfsetdash{}{0pt}%
\pgfpathmoveto{\pgfqpoint{3.279629in}{0.253159in}}%
\pgfpathlineto{\pgfqpoint{3.243760in}{0.253159in}}%
\pgfpathlineto{\pgfqpoint{3.243760in}{0.274087in}}%
\pgfpathlineto{\pgfqpoint{3.279629in}{0.274087in}}%
\pgfpathlineto{\pgfqpoint{3.279629in}{0.253159in}}%
\pgfpathclose%
\pgfusepath{fill}%
\end{pgfscope}%
\begin{pgfscope}%
\pgfpathrectangle{\pgfqpoint{3.019583in}{0.169444in}}{\pgfqpoint{0.896708in}{1.339426in}}%
\pgfusepath{clip}%
\pgfsetbuttcap%
\pgfsetmiterjoin%
\definecolor{currentfill}{rgb}{0.788774,0.191542,0.298808}%
\pgfsetfillcolor{currentfill}%
\pgfsetlinewidth{0.000000pt}%
\definecolor{currentstroke}{rgb}{0.000000,0.000000,0.000000}%
\pgfsetstrokecolor{currentstroke}%
\pgfsetstrokeopacity{0.000000}%
\pgfsetdash{}{0pt}%
\pgfpathmoveto{\pgfqpoint{3.279629in}{0.274087in}}%
\pgfpathlineto{\pgfqpoint{3.243760in}{0.274087in}}%
\pgfpathlineto{\pgfqpoint{3.243760in}{0.295016in}}%
\pgfpathlineto{\pgfqpoint{3.279629in}{0.295016in}}%
\pgfpathlineto{\pgfqpoint{3.279629in}{0.274087in}}%
\pgfpathclose%
\pgfusepath{fill}%
\end{pgfscope}%
\begin{pgfscope}%
\pgfpathrectangle{\pgfqpoint{3.019583in}{0.169444in}}{\pgfqpoint{0.896708in}{1.339426in}}%
\pgfusepath{clip}%
\pgfsetbuttcap%
\pgfsetmiterjoin%
\definecolor{currentfill}{rgb}{0.822607,0.229066,0.306805}%
\pgfsetfillcolor{currentfill}%
\pgfsetlinewidth{0.000000pt}%
\definecolor{currentstroke}{rgb}{0.000000,0.000000,0.000000}%
\pgfsetstrokecolor{currentstroke}%
\pgfsetstrokeopacity{0.000000}%
\pgfsetdash{}{0pt}%
\pgfpathmoveto{\pgfqpoint{3.279629in}{0.295016in}}%
\pgfpathlineto{\pgfqpoint{3.243760in}{0.295016in}}%
\pgfpathlineto{\pgfqpoint{3.243760in}{0.315944in}}%
\pgfpathlineto{\pgfqpoint{3.279629in}{0.315944in}}%
\pgfpathlineto{\pgfqpoint{3.279629in}{0.295016in}}%
\pgfpathclose%
\pgfusepath{fill}%
\end{pgfscope}%
\begin{pgfscope}%
\pgfpathrectangle{\pgfqpoint{3.019583in}{0.169444in}}{\pgfqpoint{0.896708in}{1.339426in}}%
\pgfusepath{clip}%
\pgfsetbuttcap%
\pgfsetmiterjoin%
\definecolor{currentfill}{rgb}{0.847213,0.261207,0.305190}%
\pgfsetfillcolor{currentfill}%
\pgfsetlinewidth{0.000000pt}%
\definecolor{currentstroke}{rgb}{0.000000,0.000000,0.000000}%
\pgfsetstrokecolor{currentstroke}%
\pgfsetstrokeopacity{0.000000}%
\pgfsetdash{}{0pt}%
\pgfpathmoveto{\pgfqpoint{3.279629in}{0.315944in}}%
\pgfpathlineto{\pgfqpoint{3.243760in}{0.315944in}}%
\pgfpathlineto{\pgfqpoint{3.243760in}{0.336873in}}%
\pgfpathlineto{\pgfqpoint{3.279629in}{0.336873in}}%
\pgfpathlineto{\pgfqpoint{3.279629in}{0.315944in}}%
\pgfpathclose%
\pgfusepath{fill}%
\end{pgfscope}%
\begin{pgfscope}%
\pgfpathrectangle{\pgfqpoint{3.019583in}{0.169444in}}{\pgfqpoint{0.896708in}{1.339426in}}%
\pgfusepath{clip}%
\pgfsetbuttcap%
\pgfsetmiterjoin%
\definecolor{currentfill}{rgb}{0.866282,0.290119,0.297809}%
\pgfsetfillcolor{currentfill}%
\pgfsetlinewidth{0.000000pt}%
\definecolor{currentstroke}{rgb}{0.000000,0.000000,0.000000}%
\pgfsetstrokecolor{currentstroke}%
\pgfsetstrokeopacity{0.000000}%
\pgfsetdash{}{0pt}%
\pgfpathmoveto{\pgfqpoint{3.279629in}{0.336873in}}%
\pgfpathlineto{\pgfqpoint{3.243760in}{0.336873in}}%
\pgfpathlineto{\pgfqpoint{3.243760in}{0.357801in}}%
\pgfpathlineto{\pgfqpoint{3.279629in}{0.357801in}}%
\pgfpathlineto{\pgfqpoint{3.279629in}{0.336873in}}%
\pgfpathclose%
\pgfusepath{fill}%
\end{pgfscope}%
\begin{pgfscope}%
\pgfpathrectangle{\pgfqpoint{3.019583in}{0.169444in}}{\pgfqpoint{0.896708in}{1.339426in}}%
\pgfusepath{clip}%
\pgfsetbuttcap%
\pgfsetmiterjoin%
\definecolor{currentfill}{rgb}{0.885352,0.319031,0.290427}%
\pgfsetfillcolor{currentfill}%
\pgfsetlinewidth{0.000000pt}%
\definecolor{currentstroke}{rgb}{0.000000,0.000000,0.000000}%
\pgfsetstrokecolor{currentstroke}%
\pgfsetstrokeopacity{0.000000}%
\pgfsetdash{}{0pt}%
\pgfpathmoveto{\pgfqpoint{3.279629in}{0.357801in}}%
\pgfpathlineto{\pgfqpoint{3.243760in}{0.357801in}}%
\pgfpathlineto{\pgfqpoint{3.243760in}{0.378730in}}%
\pgfpathlineto{\pgfqpoint{3.279629in}{0.378730in}}%
\pgfpathlineto{\pgfqpoint{3.279629in}{0.357801in}}%
\pgfpathclose%
\pgfusepath{fill}%
\end{pgfscope}%
\begin{pgfscope}%
\pgfpathrectangle{\pgfqpoint{3.019583in}{0.169444in}}{\pgfqpoint{0.896708in}{1.339426in}}%
\pgfusepath{clip}%
\pgfsetbuttcap%
\pgfsetmiterjoin%
\definecolor{currentfill}{rgb}{0.904421,0.347943,0.283045}%
\pgfsetfillcolor{currentfill}%
\pgfsetlinewidth{0.000000pt}%
\definecolor{currentstroke}{rgb}{0.000000,0.000000,0.000000}%
\pgfsetstrokecolor{currentstroke}%
\pgfsetstrokeopacity{0.000000}%
\pgfsetdash{}{0pt}%
\pgfpathmoveto{\pgfqpoint{3.279629in}{0.378730in}}%
\pgfpathlineto{\pgfqpoint{3.243760in}{0.378730in}}%
\pgfpathlineto{\pgfqpoint{3.243760in}{0.399658in}}%
\pgfpathlineto{\pgfqpoint{3.279629in}{0.399658in}}%
\pgfpathlineto{\pgfqpoint{3.279629in}{0.378730in}}%
\pgfpathclose%
\pgfusepath{fill}%
\end{pgfscope}%
\begin{pgfscope}%
\pgfpathrectangle{\pgfqpoint{3.019583in}{0.169444in}}{\pgfqpoint{0.896708in}{1.339426in}}%
\pgfusepath{clip}%
\pgfsetbuttcap%
\pgfsetmiterjoin%
\definecolor{currentfill}{rgb}{0.923491,0.376855,0.275663}%
\pgfsetfillcolor{currentfill}%
\pgfsetlinewidth{0.000000pt}%
\definecolor{currentstroke}{rgb}{0.000000,0.000000,0.000000}%
\pgfsetstrokecolor{currentstroke}%
\pgfsetstrokeopacity{0.000000}%
\pgfsetdash{}{0pt}%
\pgfpathmoveto{\pgfqpoint{3.279629in}{0.399658in}}%
\pgfpathlineto{\pgfqpoint{3.243760in}{0.399658in}}%
\pgfpathlineto{\pgfqpoint{3.243760in}{0.420587in}}%
\pgfpathlineto{\pgfqpoint{3.279629in}{0.420587in}}%
\pgfpathlineto{\pgfqpoint{3.279629in}{0.399658in}}%
\pgfpathclose%
\pgfusepath{fill}%
\end{pgfscope}%
\begin{pgfscope}%
\pgfpathrectangle{\pgfqpoint{3.019583in}{0.169444in}}{\pgfqpoint{0.896708in}{1.339426in}}%
\pgfusepath{clip}%
\pgfsetbuttcap%
\pgfsetmiterjoin%
\definecolor{currentfill}{rgb}{0.942561,0.405767,0.268281}%
\pgfsetfillcolor{currentfill}%
\pgfsetlinewidth{0.000000pt}%
\definecolor{currentstroke}{rgb}{0.000000,0.000000,0.000000}%
\pgfsetstrokecolor{currentstroke}%
\pgfsetstrokeopacity{0.000000}%
\pgfsetdash{}{0pt}%
\pgfpathmoveto{\pgfqpoint{3.279629in}{0.420587in}}%
\pgfpathlineto{\pgfqpoint{3.243760in}{0.420587in}}%
\pgfpathlineto{\pgfqpoint{3.243760in}{0.441515in}}%
\pgfpathlineto{\pgfqpoint{3.279629in}{0.441515in}}%
\pgfpathlineto{\pgfqpoint{3.279629in}{0.420587in}}%
\pgfpathclose%
\pgfusepath{fill}%
\end{pgfscope}%
\begin{pgfscope}%
\pgfpathrectangle{\pgfqpoint{3.019583in}{0.169444in}}{\pgfqpoint{0.896708in}{1.339426in}}%
\pgfusepath{clip}%
\pgfsetbuttcap%
\pgfsetmiterjoin%
\definecolor{currentfill}{rgb}{0.958247,0.437447,0.267359}%
\pgfsetfillcolor{currentfill}%
\pgfsetlinewidth{0.000000pt}%
\definecolor{currentstroke}{rgb}{0.000000,0.000000,0.000000}%
\pgfsetstrokecolor{currentstroke}%
\pgfsetstrokeopacity{0.000000}%
\pgfsetdash{}{0pt}%
\pgfpathmoveto{\pgfqpoint{3.279629in}{0.441515in}}%
\pgfpathlineto{\pgfqpoint{3.243760in}{0.441515in}}%
\pgfpathlineto{\pgfqpoint{3.243760in}{0.462444in}}%
\pgfpathlineto{\pgfqpoint{3.279629in}{0.462444in}}%
\pgfpathlineto{\pgfqpoint{3.279629in}{0.441515in}}%
\pgfpathclose%
\pgfusepath{fill}%
\end{pgfscope}%
\begin{pgfscope}%
\pgfpathrectangle{\pgfqpoint{3.019583in}{0.169444in}}{\pgfqpoint{0.896708in}{1.339426in}}%
\pgfusepath{clip}%
\pgfsetbuttcap%
\pgfsetmiterjoin%
\definecolor{currentfill}{rgb}{0.963783,0.477432,0.285813}%
\pgfsetfillcolor{currentfill}%
\pgfsetlinewidth{0.000000pt}%
\definecolor{currentstroke}{rgb}{0.000000,0.000000,0.000000}%
\pgfsetstrokecolor{currentstroke}%
\pgfsetstrokeopacity{0.000000}%
\pgfsetdash{}{0pt}%
\pgfpathmoveto{\pgfqpoint{3.279629in}{0.462444in}}%
\pgfpathlineto{\pgfqpoint{3.243760in}{0.462444in}}%
\pgfpathlineto{\pgfqpoint{3.243760in}{0.483372in}}%
\pgfpathlineto{\pgfqpoint{3.279629in}{0.483372in}}%
\pgfpathlineto{\pgfqpoint{3.279629in}{0.462444in}}%
\pgfpathclose%
\pgfusepath{fill}%
\end{pgfscope}%
\begin{pgfscope}%
\pgfpathrectangle{\pgfqpoint{3.019583in}{0.169444in}}{\pgfqpoint{0.896708in}{1.339426in}}%
\pgfusepath{clip}%
\pgfsetbuttcap%
\pgfsetmiterjoin%
\definecolor{currentfill}{rgb}{0.969319,0.517416,0.304268}%
\pgfsetfillcolor{currentfill}%
\pgfsetlinewidth{0.000000pt}%
\definecolor{currentstroke}{rgb}{0.000000,0.000000,0.000000}%
\pgfsetstrokecolor{currentstroke}%
\pgfsetstrokeopacity{0.000000}%
\pgfsetdash{}{0pt}%
\pgfpathmoveto{\pgfqpoint{3.279629in}{0.483372in}}%
\pgfpathlineto{\pgfqpoint{3.243760in}{0.483372in}}%
\pgfpathlineto{\pgfqpoint{3.243760in}{0.504301in}}%
\pgfpathlineto{\pgfqpoint{3.279629in}{0.504301in}}%
\pgfpathlineto{\pgfqpoint{3.279629in}{0.483372in}}%
\pgfpathclose%
\pgfusepath{fill}%
\end{pgfscope}%
\begin{pgfscope}%
\pgfpathrectangle{\pgfqpoint{3.019583in}{0.169444in}}{\pgfqpoint{0.896708in}{1.339426in}}%
\pgfusepath{clip}%
\pgfsetbuttcap%
\pgfsetmiterjoin%
\definecolor{currentfill}{rgb}{0.974856,0.557401,0.322722}%
\pgfsetfillcolor{currentfill}%
\pgfsetlinewidth{0.000000pt}%
\definecolor{currentstroke}{rgb}{0.000000,0.000000,0.000000}%
\pgfsetstrokecolor{currentstroke}%
\pgfsetstrokeopacity{0.000000}%
\pgfsetdash{}{0pt}%
\pgfpathmoveto{\pgfqpoint{3.279629in}{0.504301in}}%
\pgfpathlineto{\pgfqpoint{3.243760in}{0.504301in}}%
\pgfpathlineto{\pgfqpoint{3.243760in}{0.525230in}}%
\pgfpathlineto{\pgfqpoint{3.279629in}{0.525230in}}%
\pgfpathlineto{\pgfqpoint{3.279629in}{0.504301in}}%
\pgfpathclose%
\pgfusepath{fill}%
\end{pgfscope}%
\begin{pgfscope}%
\pgfpathrectangle{\pgfqpoint{3.019583in}{0.169444in}}{\pgfqpoint{0.896708in}{1.339426in}}%
\pgfusepath{clip}%
\pgfsetbuttcap%
\pgfsetmiterjoin%
\definecolor{currentfill}{rgb}{0.980392,0.597386,0.341176}%
\pgfsetfillcolor{currentfill}%
\pgfsetlinewidth{0.000000pt}%
\definecolor{currentstroke}{rgb}{0.000000,0.000000,0.000000}%
\pgfsetstrokecolor{currentstroke}%
\pgfsetstrokeopacity{0.000000}%
\pgfsetdash{}{0pt}%
\pgfpathmoveto{\pgfqpoint{3.279629in}{0.525230in}}%
\pgfpathlineto{\pgfqpoint{3.243760in}{0.525230in}}%
\pgfpathlineto{\pgfqpoint{3.243760in}{0.546158in}}%
\pgfpathlineto{\pgfqpoint{3.279629in}{0.546158in}}%
\pgfpathlineto{\pgfqpoint{3.279629in}{0.525230in}}%
\pgfpathclose%
\pgfusepath{fill}%
\end{pgfscope}%
\begin{pgfscope}%
\pgfpathrectangle{\pgfqpoint{3.019583in}{0.169444in}}{\pgfqpoint{0.896708in}{1.339426in}}%
\pgfusepath{clip}%
\pgfsetbuttcap%
\pgfsetmiterjoin%
\definecolor{currentfill}{rgb}{0.985928,0.637370,0.359631}%
\pgfsetfillcolor{currentfill}%
\pgfsetlinewidth{0.000000pt}%
\definecolor{currentstroke}{rgb}{0.000000,0.000000,0.000000}%
\pgfsetstrokecolor{currentstroke}%
\pgfsetstrokeopacity{0.000000}%
\pgfsetdash{}{0pt}%
\pgfpathmoveto{\pgfqpoint{3.279629in}{0.546158in}}%
\pgfpathlineto{\pgfqpoint{3.243760in}{0.546158in}}%
\pgfpathlineto{\pgfqpoint{3.243760in}{0.567087in}}%
\pgfpathlineto{\pgfqpoint{3.279629in}{0.567087in}}%
\pgfpathlineto{\pgfqpoint{3.279629in}{0.546158in}}%
\pgfpathclose%
\pgfusepath{fill}%
\end{pgfscope}%
\begin{pgfscope}%
\pgfpathrectangle{\pgfqpoint{3.019583in}{0.169444in}}{\pgfqpoint{0.896708in}{1.339426in}}%
\pgfusepath{clip}%
\pgfsetbuttcap%
\pgfsetmiterjoin%
\definecolor{currentfill}{rgb}{0.991465,0.677355,0.378085}%
\pgfsetfillcolor{currentfill}%
\pgfsetlinewidth{0.000000pt}%
\definecolor{currentstroke}{rgb}{0.000000,0.000000,0.000000}%
\pgfsetstrokecolor{currentstroke}%
\pgfsetstrokeopacity{0.000000}%
\pgfsetdash{}{0pt}%
\pgfpathmoveto{\pgfqpoint{3.279629in}{0.567087in}}%
\pgfpathlineto{\pgfqpoint{3.243760in}{0.567087in}}%
\pgfpathlineto{\pgfqpoint{3.243760in}{0.588015in}}%
\pgfpathlineto{\pgfqpoint{3.279629in}{0.588015in}}%
\pgfpathlineto{\pgfqpoint{3.279629in}{0.567087in}}%
\pgfpathclose%
\pgfusepath{fill}%
\end{pgfscope}%
\begin{pgfscope}%
\pgfpathrectangle{\pgfqpoint{3.019583in}{0.169444in}}{\pgfqpoint{0.896708in}{1.339426in}}%
\pgfusepath{clip}%
\pgfsetbuttcap%
\pgfsetmiterjoin%
\definecolor{currentfill}{rgb}{0.992695,0.709266,0.402999}%
\pgfsetfillcolor{currentfill}%
\pgfsetlinewidth{0.000000pt}%
\definecolor{currentstroke}{rgb}{0.000000,0.000000,0.000000}%
\pgfsetstrokecolor{currentstroke}%
\pgfsetstrokeopacity{0.000000}%
\pgfsetdash{}{0pt}%
\pgfpathmoveto{\pgfqpoint{3.279629in}{0.588015in}}%
\pgfpathlineto{\pgfqpoint{3.243760in}{0.588015in}}%
\pgfpathlineto{\pgfqpoint{3.243760in}{0.608944in}}%
\pgfpathlineto{\pgfqpoint{3.279629in}{0.608944in}}%
\pgfpathlineto{\pgfqpoint{3.279629in}{0.588015in}}%
\pgfpathclose%
\pgfusepath{fill}%
\end{pgfscope}%
\begin{pgfscope}%
\pgfpathrectangle{\pgfqpoint{3.019583in}{0.169444in}}{\pgfqpoint{0.896708in}{1.339426in}}%
\pgfusepath{clip}%
\pgfsetbuttcap%
\pgfsetmiterjoin%
\definecolor{currentfill}{rgb}{0.993310,0.740023,0.428835}%
\pgfsetfillcolor{currentfill}%
\pgfsetlinewidth{0.000000pt}%
\definecolor{currentstroke}{rgb}{0.000000,0.000000,0.000000}%
\pgfsetstrokecolor{currentstroke}%
\pgfsetstrokeopacity{0.000000}%
\pgfsetdash{}{0pt}%
\pgfpathmoveto{\pgfqpoint{3.279629in}{0.608944in}}%
\pgfpathlineto{\pgfqpoint{3.243760in}{0.608944in}}%
\pgfpathlineto{\pgfqpoint{3.243760in}{0.629872in}}%
\pgfpathlineto{\pgfqpoint{3.279629in}{0.629872in}}%
\pgfpathlineto{\pgfqpoint{3.279629in}{0.608944in}}%
\pgfpathclose%
\pgfusepath{fill}%
\end{pgfscope}%
\begin{pgfscope}%
\pgfpathrectangle{\pgfqpoint{3.019583in}{0.169444in}}{\pgfqpoint{0.896708in}{1.339426in}}%
\pgfusepath{clip}%
\pgfsetbuttcap%
\pgfsetmiterjoin%
\definecolor{currentfill}{rgb}{0.993925,0.770780,0.454671}%
\pgfsetfillcolor{currentfill}%
\pgfsetlinewidth{0.000000pt}%
\definecolor{currentstroke}{rgb}{0.000000,0.000000,0.000000}%
\pgfsetstrokecolor{currentstroke}%
\pgfsetstrokeopacity{0.000000}%
\pgfsetdash{}{0pt}%
\pgfpathmoveto{\pgfqpoint{3.279629in}{0.629872in}}%
\pgfpathlineto{\pgfqpoint{3.243760in}{0.629872in}}%
\pgfpathlineto{\pgfqpoint{3.243760in}{0.650801in}}%
\pgfpathlineto{\pgfqpoint{3.279629in}{0.650801in}}%
\pgfpathlineto{\pgfqpoint{3.279629in}{0.629872in}}%
\pgfpathclose%
\pgfusepath{fill}%
\end{pgfscope}%
\begin{pgfscope}%
\pgfpathrectangle{\pgfqpoint{3.019583in}{0.169444in}}{\pgfqpoint{0.896708in}{1.339426in}}%
\pgfusepath{clip}%
\pgfsetbuttcap%
\pgfsetmiterjoin%
\definecolor{currentfill}{rgb}{0.994541,0.801538,0.480507}%
\pgfsetfillcolor{currentfill}%
\pgfsetlinewidth{0.000000pt}%
\definecolor{currentstroke}{rgb}{0.000000,0.000000,0.000000}%
\pgfsetstrokecolor{currentstroke}%
\pgfsetstrokeopacity{0.000000}%
\pgfsetdash{}{0pt}%
\pgfpathmoveto{\pgfqpoint{3.279629in}{0.650801in}}%
\pgfpathlineto{\pgfqpoint{3.243760in}{0.650801in}}%
\pgfpathlineto{\pgfqpoint{3.243760in}{0.671729in}}%
\pgfpathlineto{\pgfqpoint{3.279629in}{0.671729in}}%
\pgfpathlineto{\pgfqpoint{3.279629in}{0.650801in}}%
\pgfpathclose%
\pgfusepath{fill}%
\end{pgfscope}%
\begin{pgfscope}%
\pgfpathrectangle{\pgfqpoint{3.019583in}{0.169444in}}{\pgfqpoint{0.896708in}{1.339426in}}%
\pgfusepath{clip}%
\pgfsetbuttcap%
\pgfsetmiterjoin%
\definecolor{currentfill}{rgb}{0.995156,0.832295,0.506344}%
\pgfsetfillcolor{currentfill}%
\pgfsetlinewidth{0.000000pt}%
\definecolor{currentstroke}{rgb}{0.000000,0.000000,0.000000}%
\pgfsetstrokecolor{currentstroke}%
\pgfsetstrokeopacity{0.000000}%
\pgfsetdash{}{0pt}%
\pgfpathmoveto{\pgfqpoint{3.279629in}{0.671729in}}%
\pgfpathlineto{\pgfqpoint{3.243760in}{0.671729in}}%
\pgfpathlineto{\pgfqpoint{3.243760in}{0.692658in}}%
\pgfpathlineto{\pgfqpoint{3.279629in}{0.692658in}}%
\pgfpathlineto{\pgfqpoint{3.279629in}{0.671729in}}%
\pgfpathclose%
\pgfusepath{fill}%
\end{pgfscope}%
\begin{pgfscope}%
\pgfpathrectangle{\pgfqpoint{3.019583in}{0.169444in}}{\pgfqpoint{0.896708in}{1.339426in}}%
\pgfusepath{clip}%
\pgfsetbuttcap%
\pgfsetmiterjoin%
\definecolor{currentfill}{rgb}{0.995771,0.863053,0.532180}%
\pgfsetfillcolor{currentfill}%
\pgfsetlinewidth{0.000000pt}%
\definecolor{currentstroke}{rgb}{0.000000,0.000000,0.000000}%
\pgfsetstrokecolor{currentstroke}%
\pgfsetstrokeopacity{0.000000}%
\pgfsetdash{}{0pt}%
\pgfpathmoveto{\pgfqpoint{3.279629in}{0.692658in}}%
\pgfpathlineto{\pgfqpoint{3.243760in}{0.692658in}}%
\pgfpathlineto{\pgfqpoint{3.243760in}{0.713586in}}%
\pgfpathlineto{\pgfqpoint{3.279629in}{0.713586in}}%
\pgfpathlineto{\pgfqpoint{3.279629in}{0.692658in}}%
\pgfpathclose%
\pgfusepath{fill}%
\end{pgfscope}%
\begin{pgfscope}%
\pgfpathrectangle{\pgfqpoint{3.019583in}{0.169444in}}{\pgfqpoint{0.896708in}{1.339426in}}%
\pgfusepath{clip}%
\pgfsetbuttcap%
\pgfsetmiterjoin%
\definecolor{currentfill}{rgb}{0.996386,0.887966,0.561092}%
\pgfsetfillcolor{currentfill}%
\pgfsetlinewidth{0.000000pt}%
\definecolor{currentstroke}{rgb}{0.000000,0.000000,0.000000}%
\pgfsetstrokecolor{currentstroke}%
\pgfsetstrokeopacity{0.000000}%
\pgfsetdash{}{0pt}%
\pgfpathmoveto{\pgfqpoint{3.279629in}{0.713586in}}%
\pgfpathlineto{\pgfqpoint{3.243760in}{0.713586in}}%
\pgfpathlineto{\pgfqpoint{3.243760in}{0.734515in}}%
\pgfpathlineto{\pgfqpoint{3.279629in}{0.734515in}}%
\pgfpathlineto{\pgfqpoint{3.279629in}{0.713586in}}%
\pgfpathclose%
\pgfusepath{fill}%
\end{pgfscope}%
\begin{pgfscope}%
\pgfpathrectangle{\pgfqpoint{3.019583in}{0.169444in}}{\pgfqpoint{0.896708in}{1.339426in}}%
\pgfusepath{clip}%
\pgfsetbuttcap%
\pgfsetmiterjoin%
\definecolor{currentfill}{rgb}{0.997001,0.907036,0.593080}%
\pgfsetfillcolor{currentfill}%
\pgfsetlinewidth{0.000000pt}%
\definecolor{currentstroke}{rgb}{0.000000,0.000000,0.000000}%
\pgfsetstrokecolor{currentstroke}%
\pgfsetstrokeopacity{0.000000}%
\pgfsetdash{}{0pt}%
\pgfpathmoveto{\pgfqpoint{3.279629in}{0.734515in}}%
\pgfpathlineto{\pgfqpoint{3.243760in}{0.734515in}}%
\pgfpathlineto{\pgfqpoint{3.243760in}{0.755443in}}%
\pgfpathlineto{\pgfqpoint{3.279629in}{0.755443in}}%
\pgfpathlineto{\pgfqpoint{3.279629in}{0.734515in}}%
\pgfpathclose%
\pgfusepath{fill}%
\end{pgfscope}%
\begin{pgfscope}%
\pgfpathrectangle{\pgfqpoint{3.019583in}{0.169444in}}{\pgfqpoint{0.896708in}{1.339426in}}%
\pgfusepath{clip}%
\pgfsetbuttcap%
\pgfsetmiterjoin%
\definecolor{currentfill}{rgb}{0.997616,0.926105,0.625067}%
\pgfsetfillcolor{currentfill}%
\pgfsetlinewidth{0.000000pt}%
\definecolor{currentstroke}{rgb}{0.000000,0.000000,0.000000}%
\pgfsetstrokecolor{currentstroke}%
\pgfsetstrokeopacity{0.000000}%
\pgfsetdash{}{0pt}%
\pgfpathmoveto{\pgfqpoint{3.279629in}{0.755443in}}%
\pgfpathlineto{\pgfqpoint{3.243760in}{0.755443in}}%
\pgfpathlineto{\pgfqpoint{3.243760in}{0.776372in}}%
\pgfpathlineto{\pgfqpoint{3.279629in}{0.776372in}}%
\pgfpathlineto{\pgfqpoint{3.279629in}{0.755443in}}%
\pgfpathclose%
\pgfusepath{fill}%
\end{pgfscope}%
\begin{pgfscope}%
\pgfpathrectangle{\pgfqpoint{3.019583in}{0.169444in}}{\pgfqpoint{0.896708in}{1.339426in}}%
\pgfusepath{clip}%
\pgfsetbuttcap%
\pgfsetmiterjoin%
\definecolor{currentfill}{rgb}{0.998231,0.945175,0.657055}%
\pgfsetfillcolor{currentfill}%
\pgfsetlinewidth{0.000000pt}%
\definecolor{currentstroke}{rgb}{0.000000,0.000000,0.000000}%
\pgfsetstrokecolor{currentstroke}%
\pgfsetstrokeopacity{0.000000}%
\pgfsetdash{}{0pt}%
\pgfpathmoveto{\pgfqpoint{3.279629in}{0.776372in}}%
\pgfpathlineto{\pgfqpoint{3.243760in}{0.776372in}}%
\pgfpathlineto{\pgfqpoint{3.243760in}{0.797300in}}%
\pgfpathlineto{\pgfqpoint{3.279629in}{0.797300in}}%
\pgfpathlineto{\pgfqpoint{3.279629in}{0.776372in}}%
\pgfpathclose%
\pgfusepath{fill}%
\end{pgfscope}%
\begin{pgfscope}%
\pgfpathrectangle{\pgfqpoint{3.019583in}{0.169444in}}{\pgfqpoint{0.896708in}{1.339426in}}%
\pgfusepath{clip}%
\pgfsetbuttcap%
\pgfsetmiterjoin%
\definecolor{currentfill}{rgb}{0.998847,0.964245,0.689043}%
\pgfsetfillcolor{currentfill}%
\pgfsetlinewidth{0.000000pt}%
\definecolor{currentstroke}{rgb}{0.000000,0.000000,0.000000}%
\pgfsetstrokecolor{currentstroke}%
\pgfsetstrokeopacity{0.000000}%
\pgfsetdash{}{0pt}%
\pgfpathmoveto{\pgfqpoint{3.279629in}{0.797300in}}%
\pgfpathlineto{\pgfqpoint{3.243760in}{0.797300in}}%
\pgfpathlineto{\pgfqpoint{3.243760in}{0.818229in}}%
\pgfpathlineto{\pgfqpoint{3.279629in}{0.818229in}}%
\pgfpathlineto{\pgfqpoint{3.279629in}{0.797300in}}%
\pgfpathclose%
\pgfusepath{fill}%
\end{pgfscope}%
\begin{pgfscope}%
\pgfpathrectangle{\pgfqpoint{3.019583in}{0.169444in}}{\pgfqpoint{0.896708in}{1.339426in}}%
\pgfusepath{clip}%
\pgfsetbuttcap%
\pgfsetmiterjoin%
\definecolor{currentfill}{rgb}{0.999462,0.983314,0.721030}%
\pgfsetfillcolor{currentfill}%
\pgfsetlinewidth{0.000000pt}%
\definecolor{currentstroke}{rgb}{0.000000,0.000000,0.000000}%
\pgfsetstrokecolor{currentstroke}%
\pgfsetstrokeopacity{0.000000}%
\pgfsetdash{}{0pt}%
\pgfpathmoveto{\pgfqpoint{3.279629in}{0.818229in}}%
\pgfpathlineto{\pgfqpoint{3.243760in}{0.818229in}}%
\pgfpathlineto{\pgfqpoint{3.243760in}{0.839157in}}%
\pgfpathlineto{\pgfqpoint{3.279629in}{0.839157in}}%
\pgfpathlineto{\pgfqpoint{3.279629in}{0.818229in}}%
\pgfpathclose%
\pgfusepath{fill}%
\end{pgfscope}%
\begin{pgfscope}%
\pgfpathrectangle{\pgfqpoint{3.019583in}{0.169444in}}{\pgfqpoint{0.896708in}{1.339426in}}%
\pgfusepath{clip}%
\pgfsetbuttcap%
\pgfsetmiterjoin%
\definecolor{currentfill}{rgb}{0.998078,0.999231,0.746021}%
\pgfsetfillcolor{currentfill}%
\pgfsetlinewidth{0.000000pt}%
\definecolor{currentstroke}{rgb}{0.000000,0.000000,0.000000}%
\pgfsetstrokecolor{currentstroke}%
\pgfsetstrokeopacity{0.000000}%
\pgfsetdash{}{0pt}%
\pgfpathmoveto{\pgfqpoint{3.279629in}{0.839157in}}%
\pgfpathlineto{\pgfqpoint{3.243760in}{0.839157in}}%
\pgfpathlineto{\pgfqpoint{3.243760in}{0.860086in}}%
\pgfpathlineto{\pgfqpoint{3.279629in}{0.860086in}}%
\pgfpathlineto{\pgfqpoint{3.279629in}{0.839157in}}%
\pgfpathclose%
\pgfusepath{fill}%
\end{pgfscope}%
\begin{pgfscope}%
\pgfpathrectangle{\pgfqpoint{3.019583in}{0.169444in}}{\pgfqpoint{0.896708in}{1.339426in}}%
\pgfusepath{clip}%
\pgfsetbuttcap%
\pgfsetmiterjoin%
\definecolor{currentfill}{rgb}{0.982699,0.993080,0.722030}%
\pgfsetfillcolor{currentfill}%
\pgfsetlinewidth{0.000000pt}%
\definecolor{currentstroke}{rgb}{0.000000,0.000000,0.000000}%
\pgfsetstrokecolor{currentstroke}%
\pgfsetstrokeopacity{0.000000}%
\pgfsetdash{}{0pt}%
\pgfpathmoveto{\pgfqpoint{3.279629in}{0.860086in}}%
\pgfpathlineto{\pgfqpoint{3.243760in}{0.860086in}}%
\pgfpathlineto{\pgfqpoint{3.243760in}{0.881015in}}%
\pgfpathlineto{\pgfqpoint{3.279629in}{0.881015in}}%
\pgfpathlineto{\pgfqpoint{3.279629in}{0.860086in}}%
\pgfpathclose%
\pgfusepath{fill}%
\end{pgfscope}%
\begin{pgfscope}%
\pgfpathrectangle{\pgfqpoint{3.019583in}{0.169444in}}{\pgfqpoint{0.896708in}{1.339426in}}%
\pgfusepath{clip}%
\pgfsetbuttcap%
\pgfsetmiterjoin%
\definecolor{currentfill}{rgb}{0.967320,0.986928,0.698039}%
\pgfsetfillcolor{currentfill}%
\pgfsetlinewidth{0.000000pt}%
\definecolor{currentstroke}{rgb}{0.000000,0.000000,0.000000}%
\pgfsetstrokecolor{currentstroke}%
\pgfsetstrokeopacity{0.000000}%
\pgfsetdash{}{0pt}%
\pgfpathmoveto{\pgfqpoint{3.279629in}{0.881015in}}%
\pgfpathlineto{\pgfqpoint{3.243760in}{0.881015in}}%
\pgfpathlineto{\pgfqpoint{3.243760in}{0.901943in}}%
\pgfpathlineto{\pgfqpoint{3.279629in}{0.901943in}}%
\pgfpathlineto{\pgfqpoint{3.279629in}{0.881015in}}%
\pgfpathclose%
\pgfusepath{fill}%
\end{pgfscope}%
\begin{pgfscope}%
\pgfpathrectangle{\pgfqpoint{3.019583in}{0.169444in}}{\pgfqpoint{0.896708in}{1.339426in}}%
\pgfusepath{clip}%
\pgfsetbuttcap%
\pgfsetmiterjoin%
\definecolor{currentfill}{rgb}{0.951942,0.980777,0.674048}%
\pgfsetfillcolor{currentfill}%
\pgfsetlinewidth{0.000000pt}%
\definecolor{currentstroke}{rgb}{0.000000,0.000000,0.000000}%
\pgfsetstrokecolor{currentstroke}%
\pgfsetstrokeopacity{0.000000}%
\pgfsetdash{}{0pt}%
\pgfpathmoveto{\pgfqpoint{3.279629in}{0.901943in}}%
\pgfpathlineto{\pgfqpoint{3.243760in}{0.901943in}}%
\pgfpathlineto{\pgfqpoint{3.243760in}{0.922872in}}%
\pgfpathlineto{\pgfqpoint{3.279629in}{0.922872in}}%
\pgfpathlineto{\pgfqpoint{3.279629in}{0.901943in}}%
\pgfpathclose%
\pgfusepath{fill}%
\end{pgfscope}%
\begin{pgfscope}%
\pgfpathrectangle{\pgfqpoint{3.019583in}{0.169444in}}{\pgfqpoint{0.896708in}{1.339426in}}%
\pgfusepath{clip}%
\pgfsetbuttcap%
\pgfsetmiterjoin%
\definecolor{currentfill}{rgb}{0.936563,0.974625,0.650058}%
\pgfsetfillcolor{currentfill}%
\pgfsetlinewidth{0.000000pt}%
\definecolor{currentstroke}{rgb}{0.000000,0.000000,0.000000}%
\pgfsetstrokecolor{currentstroke}%
\pgfsetstrokeopacity{0.000000}%
\pgfsetdash{}{0pt}%
\pgfpathmoveto{\pgfqpoint{3.279629in}{0.922872in}}%
\pgfpathlineto{\pgfqpoint{3.243760in}{0.922872in}}%
\pgfpathlineto{\pgfqpoint{3.243760in}{0.943800in}}%
\pgfpathlineto{\pgfqpoint{3.279629in}{0.943800in}}%
\pgfpathlineto{\pgfqpoint{3.279629in}{0.922872in}}%
\pgfpathclose%
\pgfusepath{fill}%
\end{pgfscope}%
\begin{pgfscope}%
\pgfpathrectangle{\pgfqpoint{3.019583in}{0.169444in}}{\pgfqpoint{0.896708in}{1.339426in}}%
\pgfusepath{clip}%
\pgfsetbuttcap%
\pgfsetmiterjoin%
\definecolor{currentfill}{rgb}{0.921184,0.968474,0.626067}%
\pgfsetfillcolor{currentfill}%
\pgfsetlinewidth{0.000000pt}%
\definecolor{currentstroke}{rgb}{0.000000,0.000000,0.000000}%
\pgfsetstrokecolor{currentstroke}%
\pgfsetstrokeopacity{0.000000}%
\pgfsetdash{}{0pt}%
\pgfpathmoveto{\pgfqpoint{3.279629in}{0.943800in}}%
\pgfpathlineto{\pgfqpoint{3.243760in}{0.943800in}}%
\pgfpathlineto{\pgfqpoint{3.243760in}{0.964729in}}%
\pgfpathlineto{\pgfqpoint{3.279629in}{0.964729in}}%
\pgfpathlineto{\pgfqpoint{3.279629in}{0.943800in}}%
\pgfpathclose%
\pgfusepath{fill}%
\end{pgfscope}%
\begin{pgfscope}%
\pgfpathrectangle{\pgfqpoint{3.019583in}{0.169444in}}{\pgfqpoint{0.896708in}{1.339426in}}%
\pgfusepath{clip}%
\pgfsetbuttcap%
\pgfsetmiterjoin%
\definecolor{currentfill}{rgb}{0.905805,0.962322,0.602076}%
\pgfsetfillcolor{currentfill}%
\pgfsetlinewidth{0.000000pt}%
\definecolor{currentstroke}{rgb}{0.000000,0.000000,0.000000}%
\pgfsetstrokecolor{currentstroke}%
\pgfsetstrokeopacity{0.000000}%
\pgfsetdash{}{0pt}%
\pgfpathmoveto{\pgfqpoint{3.279629in}{0.964729in}}%
\pgfpathlineto{\pgfqpoint{3.243760in}{0.964729in}}%
\pgfpathlineto{\pgfqpoint{3.243760in}{0.985657in}}%
\pgfpathlineto{\pgfqpoint{3.279629in}{0.985657in}}%
\pgfpathlineto{\pgfqpoint{3.279629in}{0.964729in}}%
\pgfpathclose%
\pgfusepath{fill}%
\end{pgfscope}%
\begin{pgfscope}%
\pgfpathrectangle{\pgfqpoint{3.019583in}{0.169444in}}{\pgfqpoint{0.896708in}{1.339426in}}%
\pgfusepath{clip}%
\pgfsetbuttcap%
\pgfsetmiterjoin%
\definecolor{currentfill}{rgb}{0.874740,0.949712,0.601615}%
\pgfsetfillcolor{currentfill}%
\pgfsetlinewidth{0.000000pt}%
\definecolor{currentstroke}{rgb}{0.000000,0.000000,0.000000}%
\pgfsetstrokecolor{currentstroke}%
\pgfsetstrokeopacity{0.000000}%
\pgfsetdash{}{0pt}%
\pgfpathmoveto{\pgfqpoint{3.279629in}{0.985657in}}%
\pgfpathlineto{\pgfqpoint{3.243760in}{0.985657in}}%
\pgfpathlineto{\pgfqpoint{3.243760in}{1.006586in}}%
\pgfpathlineto{\pgfqpoint{3.279629in}{1.006586in}}%
\pgfpathlineto{\pgfqpoint{3.279629in}{0.985657in}}%
\pgfpathclose%
\pgfusepath{fill}%
\end{pgfscope}%
\begin{pgfscope}%
\pgfpathrectangle{\pgfqpoint{3.019583in}{0.169444in}}{\pgfqpoint{0.896708in}{1.339426in}}%
\pgfusepath{clip}%
\pgfsetbuttcap%
\pgfsetmiterjoin%
\definecolor{currentfill}{rgb}{0.838447,0.934948,0.608997}%
\pgfsetfillcolor{currentfill}%
\pgfsetlinewidth{0.000000pt}%
\definecolor{currentstroke}{rgb}{0.000000,0.000000,0.000000}%
\pgfsetstrokecolor{currentstroke}%
\pgfsetstrokeopacity{0.000000}%
\pgfsetdash{}{0pt}%
\pgfpathmoveto{\pgfqpoint{3.279629in}{1.006586in}}%
\pgfpathlineto{\pgfqpoint{3.243760in}{1.006586in}}%
\pgfpathlineto{\pgfqpoint{3.243760in}{1.027514in}}%
\pgfpathlineto{\pgfqpoint{3.279629in}{1.027514in}}%
\pgfpathlineto{\pgfqpoint{3.279629in}{1.006586in}}%
\pgfpathclose%
\pgfusepath{fill}%
\end{pgfscope}%
\begin{pgfscope}%
\pgfpathrectangle{\pgfqpoint{3.019583in}{0.169444in}}{\pgfqpoint{0.896708in}{1.339426in}}%
\pgfusepath{clip}%
\pgfsetbuttcap%
\pgfsetmiterjoin%
\definecolor{currentfill}{rgb}{0.802153,0.920185,0.616378}%
\pgfsetfillcolor{currentfill}%
\pgfsetlinewidth{0.000000pt}%
\definecolor{currentstroke}{rgb}{0.000000,0.000000,0.000000}%
\pgfsetstrokecolor{currentstroke}%
\pgfsetstrokeopacity{0.000000}%
\pgfsetdash{}{0pt}%
\pgfpathmoveto{\pgfqpoint{3.279629in}{1.027514in}}%
\pgfpathlineto{\pgfqpoint{3.243760in}{1.027514in}}%
\pgfpathlineto{\pgfqpoint{3.243760in}{1.048443in}}%
\pgfpathlineto{\pgfqpoint{3.279629in}{1.048443in}}%
\pgfpathlineto{\pgfqpoint{3.279629in}{1.027514in}}%
\pgfpathclose%
\pgfusepath{fill}%
\end{pgfscope}%
\begin{pgfscope}%
\pgfpathrectangle{\pgfqpoint{3.019583in}{0.169444in}}{\pgfqpoint{0.896708in}{1.339426in}}%
\pgfusepath{clip}%
\pgfsetbuttcap%
\pgfsetmiterjoin%
\definecolor{currentfill}{rgb}{0.765859,0.905421,0.623760}%
\pgfsetfillcolor{currentfill}%
\pgfsetlinewidth{0.000000pt}%
\definecolor{currentstroke}{rgb}{0.000000,0.000000,0.000000}%
\pgfsetstrokecolor{currentstroke}%
\pgfsetstrokeopacity{0.000000}%
\pgfsetdash{}{0pt}%
\pgfpathmoveto{\pgfqpoint{3.279629in}{1.048443in}}%
\pgfpathlineto{\pgfqpoint{3.243760in}{1.048443in}}%
\pgfpathlineto{\pgfqpoint{3.243760in}{1.069371in}}%
\pgfpathlineto{\pgfqpoint{3.279629in}{1.069371in}}%
\pgfpathlineto{\pgfqpoint{3.279629in}{1.048443in}}%
\pgfpathclose%
\pgfusepath{fill}%
\end{pgfscope}%
\begin{pgfscope}%
\pgfpathrectangle{\pgfqpoint{3.019583in}{0.169444in}}{\pgfqpoint{0.896708in}{1.339426in}}%
\pgfusepath{clip}%
\pgfsetbuttcap%
\pgfsetmiterjoin%
\definecolor{currentfill}{rgb}{0.729566,0.890657,0.631142}%
\pgfsetfillcolor{currentfill}%
\pgfsetlinewidth{0.000000pt}%
\definecolor{currentstroke}{rgb}{0.000000,0.000000,0.000000}%
\pgfsetstrokecolor{currentstroke}%
\pgfsetstrokeopacity{0.000000}%
\pgfsetdash{}{0pt}%
\pgfpathmoveto{\pgfqpoint{3.279629in}{1.069371in}}%
\pgfpathlineto{\pgfqpoint{3.243760in}{1.069371in}}%
\pgfpathlineto{\pgfqpoint{3.243760in}{1.090300in}}%
\pgfpathlineto{\pgfqpoint{3.279629in}{1.090300in}}%
\pgfpathlineto{\pgfqpoint{3.279629in}{1.069371in}}%
\pgfpathclose%
\pgfusepath{fill}%
\end{pgfscope}%
\begin{pgfscope}%
\pgfpathrectangle{\pgfqpoint{3.019583in}{0.169444in}}{\pgfqpoint{0.896708in}{1.339426in}}%
\pgfusepath{clip}%
\pgfsetbuttcap%
\pgfsetmiterjoin%
\definecolor{currentfill}{rgb}{0.693272,0.875894,0.638524}%
\pgfsetfillcolor{currentfill}%
\pgfsetlinewidth{0.000000pt}%
\definecolor{currentstroke}{rgb}{0.000000,0.000000,0.000000}%
\pgfsetstrokecolor{currentstroke}%
\pgfsetstrokeopacity{0.000000}%
\pgfsetdash{}{0pt}%
\pgfpathmoveto{\pgfqpoint{3.279629in}{1.090300in}}%
\pgfpathlineto{\pgfqpoint{3.243760in}{1.090300in}}%
\pgfpathlineto{\pgfqpoint{3.243760in}{1.111228in}}%
\pgfpathlineto{\pgfqpoint{3.279629in}{1.111228in}}%
\pgfpathlineto{\pgfqpoint{3.279629in}{1.090300in}}%
\pgfpathclose%
\pgfusepath{fill}%
\end{pgfscope}%
\begin{pgfscope}%
\pgfpathrectangle{\pgfqpoint{3.019583in}{0.169444in}}{\pgfqpoint{0.896708in}{1.339426in}}%
\pgfusepath{clip}%
\pgfsetbuttcap%
\pgfsetmiterjoin%
\definecolor{currentfill}{rgb}{0.654671,0.860438,0.643368}%
\pgfsetfillcolor{currentfill}%
\pgfsetlinewidth{0.000000pt}%
\definecolor{currentstroke}{rgb}{0.000000,0.000000,0.000000}%
\pgfsetstrokecolor{currentstroke}%
\pgfsetstrokeopacity{0.000000}%
\pgfsetdash{}{0pt}%
\pgfpathmoveto{\pgfqpoint{3.279629in}{1.111228in}}%
\pgfpathlineto{\pgfqpoint{3.243760in}{1.111228in}}%
\pgfpathlineto{\pgfqpoint{3.243760in}{1.132157in}}%
\pgfpathlineto{\pgfqpoint{3.279629in}{1.132157in}}%
\pgfpathlineto{\pgfqpoint{3.279629in}{1.111228in}}%
\pgfpathclose%
\pgfusepath{fill}%
\end{pgfscope}%
\begin{pgfscope}%
\pgfpathrectangle{\pgfqpoint{3.019583in}{0.169444in}}{\pgfqpoint{0.896708in}{1.339426in}}%
\pgfusepath{clip}%
\pgfsetbuttcap%
\pgfsetmiterjoin%
\definecolor{currentfill}{rgb}{0.612226,0.843829,0.643983}%
\pgfsetfillcolor{currentfill}%
\pgfsetlinewidth{0.000000pt}%
\definecolor{currentstroke}{rgb}{0.000000,0.000000,0.000000}%
\pgfsetstrokecolor{currentstroke}%
\pgfsetstrokeopacity{0.000000}%
\pgfsetdash{}{0pt}%
\pgfpathmoveto{\pgfqpoint{3.279629in}{1.132157in}}%
\pgfpathlineto{\pgfqpoint{3.243760in}{1.132157in}}%
\pgfpathlineto{\pgfqpoint{3.243760in}{1.153085in}}%
\pgfpathlineto{\pgfqpoint{3.279629in}{1.153085in}}%
\pgfpathlineto{\pgfqpoint{3.279629in}{1.132157in}}%
\pgfpathclose%
\pgfusepath{fill}%
\end{pgfscope}%
\begin{pgfscope}%
\pgfpathrectangle{\pgfqpoint{3.019583in}{0.169444in}}{\pgfqpoint{0.896708in}{1.339426in}}%
\pgfusepath{clip}%
\pgfsetbuttcap%
\pgfsetmiterjoin%
\definecolor{currentfill}{rgb}{0.569781,0.827220,0.644598}%
\pgfsetfillcolor{currentfill}%
\pgfsetlinewidth{0.000000pt}%
\definecolor{currentstroke}{rgb}{0.000000,0.000000,0.000000}%
\pgfsetstrokecolor{currentstroke}%
\pgfsetstrokeopacity{0.000000}%
\pgfsetdash{}{0pt}%
\pgfpathmoveto{\pgfqpoint{3.279629in}{1.153085in}}%
\pgfpathlineto{\pgfqpoint{3.243760in}{1.153085in}}%
\pgfpathlineto{\pgfqpoint{3.243760in}{1.174014in}}%
\pgfpathlineto{\pgfqpoint{3.279629in}{1.174014in}}%
\pgfpathlineto{\pgfqpoint{3.279629in}{1.153085in}}%
\pgfpathclose%
\pgfusepath{fill}%
\end{pgfscope}%
\begin{pgfscope}%
\pgfpathrectangle{\pgfqpoint{3.019583in}{0.169444in}}{\pgfqpoint{0.896708in}{1.339426in}}%
\pgfusepath{clip}%
\pgfsetbuttcap%
\pgfsetmiterjoin%
\definecolor{currentfill}{rgb}{0.527336,0.810611,0.645213}%
\pgfsetfillcolor{currentfill}%
\pgfsetlinewidth{0.000000pt}%
\definecolor{currentstroke}{rgb}{0.000000,0.000000,0.000000}%
\pgfsetstrokecolor{currentstroke}%
\pgfsetstrokeopacity{0.000000}%
\pgfsetdash{}{0pt}%
\pgfpathmoveto{\pgfqpoint{3.279629in}{1.174014in}}%
\pgfpathlineto{\pgfqpoint{3.243760in}{1.174014in}}%
\pgfpathlineto{\pgfqpoint{3.243760in}{1.194943in}}%
\pgfpathlineto{\pgfqpoint{3.279629in}{1.194943in}}%
\pgfpathlineto{\pgfqpoint{3.279629in}{1.174014in}}%
\pgfpathclose%
\pgfusepath{fill}%
\end{pgfscope}%
\begin{pgfscope}%
\pgfpathrectangle{\pgfqpoint{3.019583in}{0.169444in}}{\pgfqpoint{0.896708in}{1.339426in}}%
\pgfusepath{clip}%
\pgfsetbuttcap%
\pgfsetmiterjoin%
\definecolor{currentfill}{rgb}{0.484890,0.794002,0.645829}%
\pgfsetfillcolor{currentfill}%
\pgfsetlinewidth{0.000000pt}%
\definecolor{currentstroke}{rgb}{0.000000,0.000000,0.000000}%
\pgfsetstrokecolor{currentstroke}%
\pgfsetstrokeopacity{0.000000}%
\pgfsetdash{}{0pt}%
\pgfpathmoveto{\pgfqpoint{3.279629in}{1.194943in}}%
\pgfpathlineto{\pgfqpoint{3.243760in}{1.194943in}}%
\pgfpathlineto{\pgfqpoint{3.243760in}{1.215871in}}%
\pgfpathlineto{\pgfqpoint{3.279629in}{1.215871in}}%
\pgfpathlineto{\pgfqpoint{3.279629in}{1.194943in}}%
\pgfpathclose%
\pgfusepath{fill}%
\end{pgfscope}%
\begin{pgfscope}%
\pgfpathrectangle{\pgfqpoint{3.019583in}{0.169444in}}{\pgfqpoint{0.896708in}{1.339426in}}%
\pgfusepath{clip}%
\pgfsetbuttcap%
\pgfsetmiterjoin%
\definecolor{currentfill}{rgb}{0.442445,0.777393,0.646444}%
\pgfsetfillcolor{currentfill}%
\pgfsetlinewidth{0.000000pt}%
\definecolor{currentstroke}{rgb}{0.000000,0.000000,0.000000}%
\pgfsetstrokecolor{currentstroke}%
\pgfsetstrokeopacity{0.000000}%
\pgfsetdash{}{0pt}%
\pgfpathmoveto{\pgfqpoint{3.279629in}{1.215871in}}%
\pgfpathlineto{\pgfqpoint{3.243760in}{1.215871in}}%
\pgfpathlineto{\pgfqpoint{3.243760in}{1.236800in}}%
\pgfpathlineto{\pgfqpoint{3.279629in}{1.236800in}}%
\pgfpathlineto{\pgfqpoint{3.279629in}{1.215871in}}%
\pgfpathclose%
\pgfusepath{fill}%
\end{pgfscope}%
\begin{pgfscope}%
\pgfpathrectangle{\pgfqpoint{3.019583in}{0.169444in}}{\pgfqpoint{0.896708in}{1.339426in}}%
\pgfusepath{clip}%
\pgfsetbuttcap%
\pgfsetmiterjoin%
\definecolor{currentfill}{rgb}{0.400000,0.760784,0.647059}%
\pgfsetfillcolor{currentfill}%
\pgfsetlinewidth{0.000000pt}%
\definecolor{currentstroke}{rgb}{0.000000,0.000000,0.000000}%
\pgfsetstrokecolor{currentstroke}%
\pgfsetstrokeopacity{0.000000}%
\pgfsetdash{}{0pt}%
\pgfpathmoveto{\pgfqpoint{3.279629in}{1.236800in}}%
\pgfpathlineto{\pgfqpoint{3.243760in}{1.236800in}}%
\pgfpathlineto{\pgfqpoint{3.243760in}{1.257728in}}%
\pgfpathlineto{\pgfqpoint{3.279629in}{1.257728in}}%
\pgfpathlineto{\pgfqpoint{3.279629in}{1.236800in}}%
\pgfpathclose%
\pgfusepath{fill}%
\end{pgfscope}%
\begin{pgfscope}%
\pgfpathrectangle{\pgfqpoint{3.019583in}{0.169444in}}{\pgfqpoint{0.896708in}{1.339426in}}%
\pgfusepath{clip}%
\pgfsetbuttcap%
\pgfsetmiterjoin%
\definecolor{currentfill}{rgb}{0.368012,0.725106,0.661822}%
\pgfsetfillcolor{currentfill}%
\pgfsetlinewidth{0.000000pt}%
\definecolor{currentstroke}{rgb}{0.000000,0.000000,0.000000}%
\pgfsetstrokecolor{currentstroke}%
\pgfsetstrokeopacity{0.000000}%
\pgfsetdash{}{0pt}%
\pgfpathmoveto{\pgfqpoint{3.279629in}{1.257728in}}%
\pgfpathlineto{\pgfqpoint{3.243760in}{1.257728in}}%
\pgfpathlineto{\pgfqpoint{3.243760in}{1.278657in}}%
\pgfpathlineto{\pgfqpoint{3.279629in}{1.278657in}}%
\pgfpathlineto{\pgfqpoint{3.279629in}{1.257728in}}%
\pgfpathclose%
\pgfusepath{fill}%
\end{pgfscope}%
\begin{pgfscope}%
\pgfpathrectangle{\pgfqpoint{3.019583in}{0.169444in}}{\pgfqpoint{0.896708in}{1.339426in}}%
\pgfusepath{clip}%
\pgfsetbuttcap%
\pgfsetmiterjoin%
\definecolor{currentfill}{rgb}{0.336025,0.689427,0.676586}%
\pgfsetfillcolor{currentfill}%
\pgfsetlinewidth{0.000000pt}%
\definecolor{currentstroke}{rgb}{0.000000,0.000000,0.000000}%
\pgfsetstrokecolor{currentstroke}%
\pgfsetstrokeopacity{0.000000}%
\pgfsetdash{}{0pt}%
\pgfpathmoveto{\pgfqpoint{3.279629in}{1.278657in}}%
\pgfpathlineto{\pgfqpoint{3.243760in}{1.278657in}}%
\pgfpathlineto{\pgfqpoint{3.243760in}{1.299585in}}%
\pgfpathlineto{\pgfqpoint{3.279629in}{1.299585in}}%
\pgfpathlineto{\pgfqpoint{3.279629in}{1.278657in}}%
\pgfpathclose%
\pgfusepath{fill}%
\end{pgfscope}%
\begin{pgfscope}%
\pgfpathrectangle{\pgfqpoint{3.019583in}{0.169444in}}{\pgfqpoint{0.896708in}{1.339426in}}%
\pgfusepath{clip}%
\pgfsetbuttcap%
\pgfsetmiterjoin%
\definecolor{currentfill}{rgb}{0.304037,0.653749,0.691349}%
\pgfsetfillcolor{currentfill}%
\pgfsetlinewidth{0.000000pt}%
\definecolor{currentstroke}{rgb}{0.000000,0.000000,0.000000}%
\pgfsetstrokecolor{currentstroke}%
\pgfsetstrokeopacity{0.000000}%
\pgfsetdash{}{0pt}%
\pgfpathmoveto{\pgfqpoint{3.279629in}{1.299585in}}%
\pgfpathlineto{\pgfqpoint{3.243760in}{1.299585in}}%
\pgfpathlineto{\pgfqpoint{3.243760in}{1.320514in}}%
\pgfpathlineto{\pgfqpoint{3.279629in}{1.320514in}}%
\pgfpathlineto{\pgfqpoint{3.279629in}{1.299585in}}%
\pgfpathclose%
\pgfusepath{fill}%
\end{pgfscope}%
\begin{pgfscope}%
\pgfpathrectangle{\pgfqpoint{3.019583in}{0.169444in}}{\pgfqpoint{0.896708in}{1.339426in}}%
\pgfusepath{clip}%
\pgfsetbuttcap%
\pgfsetmiterjoin%
\definecolor{currentfill}{rgb}{0.272049,0.618070,0.706113}%
\pgfsetfillcolor{currentfill}%
\pgfsetlinewidth{0.000000pt}%
\definecolor{currentstroke}{rgb}{0.000000,0.000000,0.000000}%
\pgfsetstrokecolor{currentstroke}%
\pgfsetstrokeopacity{0.000000}%
\pgfsetdash{}{0pt}%
\pgfpathmoveto{\pgfqpoint{3.279629in}{1.320514in}}%
\pgfpathlineto{\pgfqpoint{3.243760in}{1.320514in}}%
\pgfpathlineto{\pgfqpoint{3.243760in}{1.341442in}}%
\pgfpathlineto{\pgfqpoint{3.279629in}{1.341442in}}%
\pgfpathlineto{\pgfqpoint{3.279629in}{1.320514in}}%
\pgfpathclose%
\pgfusepath{fill}%
\end{pgfscope}%
\begin{pgfscope}%
\pgfpathrectangle{\pgfqpoint{3.019583in}{0.169444in}}{\pgfqpoint{0.896708in}{1.339426in}}%
\pgfusepath{clip}%
\pgfsetbuttcap%
\pgfsetmiterjoin%
\definecolor{currentfill}{rgb}{0.240062,0.582391,0.720877}%
\pgfsetfillcolor{currentfill}%
\pgfsetlinewidth{0.000000pt}%
\definecolor{currentstroke}{rgb}{0.000000,0.000000,0.000000}%
\pgfsetstrokecolor{currentstroke}%
\pgfsetstrokeopacity{0.000000}%
\pgfsetdash{}{0pt}%
\pgfpathmoveto{\pgfqpoint{3.279629in}{1.341442in}}%
\pgfpathlineto{\pgfqpoint{3.243760in}{1.341442in}}%
\pgfpathlineto{\pgfqpoint{3.243760in}{1.362371in}}%
\pgfpathlineto{\pgfqpoint{3.279629in}{1.362371in}}%
\pgfpathlineto{\pgfqpoint{3.279629in}{1.341442in}}%
\pgfpathclose%
\pgfusepath{fill}%
\end{pgfscope}%
\begin{pgfscope}%
\pgfpathrectangle{\pgfqpoint{3.019583in}{0.169444in}}{\pgfqpoint{0.896708in}{1.339426in}}%
\pgfusepath{clip}%
\pgfsetbuttcap%
\pgfsetmiterjoin%
\definecolor{currentfill}{rgb}{0.208074,0.546713,0.735640}%
\pgfsetfillcolor{currentfill}%
\pgfsetlinewidth{0.000000pt}%
\definecolor{currentstroke}{rgb}{0.000000,0.000000,0.000000}%
\pgfsetstrokecolor{currentstroke}%
\pgfsetstrokeopacity{0.000000}%
\pgfsetdash{}{0pt}%
\pgfpathmoveto{\pgfqpoint{3.279629in}{1.362371in}}%
\pgfpathlineto{\pgfqpoint{3.243760in}{1.362371in}}%
\pgfpathlineto{\pgfqpoint{3.243760in}{1.383299in}}%
\pgfpathlineto{\pgfqpoint{3.279629in}{1.383299in}}%
\pgfpathlineto{\pgfqpoint{3.279629in}{1.362371in}}%
\pgfpathclose%
\pgfusepath{fill}%
\end{pgfscope}%
\begin{pgfscope}%
\pgfpathrectangle{\pgfqpoint{3.019583in}{0.169444in}}{\pgfqpoint{0.896708in}{1.339426in}}%
\pgfusepath{clip}%
\pgfsetbuttcap%
\pgfsetmiterjoin%
\definecolor{currentfill}{rgb}{0.212995,0.511419,0.730796}%
\pgfsetfillcolor{currentfill}%
\pgfsetlinewidth{0.000000pt}%
\definecolor{currentstroke}{rgb}{0.000000,0.000000,0.000000}%
\pgfsetstrokecolor{currentstroke}%
\pgfsetstrokeopacity{0.000000}%
\pgfsetdash{}{0pt}%
\pgfpathmoveto{\pgfqpoint{3.279629in}{1.383299in}}%
\pgfpathlineto{\pgfqpoint{3.243760in}{1.383299in}}%
\pgfpathlineto{\pgfqpoint{3.243760in}{1.404228in}}%
\pgfpathlineto{\pgfqpoint{3.279629in}{1.404228in}}%
\pgfpathlineto{\pgfqpoint{3.279629in}{1.383299in}}%
\pgfpathclose%
\pgfusepath{fill}%
\end{pgfscope}%
\begin{pgfscope}%
\pgfpathrectangle{\pgfqpoint{3.019583in}{0.169444in}}{\pgfqpoint{0.896708in}{1.339426in}}%
\pgfusepath{clip}%
\pgfsetbuttcap%
\pgfsetmiterjoin%
\definecolor{currentfill}{rgb}{0.240062,0.476355,0.714187}%
\pgfsetfillcolor{currentfill}%
\pgfsetlinewidth{0.000000pt}%
\definecolor{currentstroke}{rgb}{0.000000,0.000000,0.000000}%
\pgfsetstrokecolor{currentstroke}%
\pgfsetstrokeopacity{0.000000}%
\pgfsetdash{}{0pt}%
\pgfpathmoveto{\pgfqpoint{3.279629in}{1.404228in}}%
\pgfpathlineto{\pgfqpoint{3.243760in}{1.404228in}}%
\pgfpathlineto{\pgfqpoint{3.243760in}{1.425156in}}%
\pgfpathlineto{\pgfqpoint{3.279629in}{1.425156in}}%
\pgfpathlineto{\pgfqpoint{3.279629in}{1.404228in}}%
\pgfpathclose%
\pgfusepath{fill}%
\end{pgfscope}%
\begin{pgfscope}%
\pgfpathrectangle{\pgfqpoint{3.019583in}{0.169444in}}{\pgfqpoint{0.896708in}{1.339426in}}%
\pgfusepath{clip}%
\pgfsetbuttcap%
\pgfsetmiterjoin%
\definecolor{currentfill}{rgb}{0.267128,0.441292,0.697578}%
\pgfsetfillcolor{currentfill}%
\pgfsetlinewidth{0.000000pt}%
\definecolor{currentstroke}{rgb}{0.000000,0.000000,0.000000}%
\pgfsetstrokecolor{currentstroke}%
\pgfsetstrokeopacity{0.000000}%
\pgfsetdash{}{0pt}%
\pgfpathmoveto{\pgfqpoint{3.279629in}{1.425156in}}%
\pgfpathlineto{\pgfqpoint{3.243760in}{1.425156in}}%
\pgfpathlineto{\pgfqpoint{3.243760in}{1.446085in}}%
\pgfpathlineto{\pgfqpoint{3.279629in}{1.446085in}}%
\pgfpathlineto{\pgfqpoint{3.279629in}{1.425156in}}%
\pgfpathclose%
\pgfusepath{fill}%
\end{pgfscope}%
\begin{pgfscope}%
\pgfpathrectangle{\pgfqpoint{3.019583in}{0.169444in}}{\pgfqpoint{0.896708in}{1.339426in}}%
\pgfusepath{clip}%
\pgfsetbuttcap%
\pgfsetmiterjoin%
\definecolor{currentfill}{rgb}{0.294195,0.406228,0.680969}%
\pgfsetfillcolor{currentfill}%
\pgfsetlinewidth{0.000000pt}%
\definecolor{currentstroke}{rgb}{0.000000,0.000000,0.000000}%
\pgfsetstrokecolor{currentstroke}%
\pgfsetstrokeopacity{0.000000}%
\pgfsetdash{}{0pt}%
\pgfpathmoveto{\pgfqpoint{3.279629in}{1.446085in}}%
\pgfpathlineto{\pgfqpoint{3.243760in}{1.446085in}}%
\pgfpathlineto{\pgfqpoint{3.243760in}{1.467013in}}%
\pgfpathlineto{\pgfqpoint{3.279629in}{1.467013in}}%
\pgfpathlineto{\pgfqpoint{3.279629in}{1.446085in}}%
\pgfpathclose%
\pgfusepath{fill}%
\end{pgfscope}%
\begin{pgfscope}%
\pgfpathrectangle{\pgfqpoint{3.019583in}{0.169444in}}{\pgfqpoint{0.896708in}{1.339426in}}%
\pgfusepath{clip}%
\pgfsetbuttcap%
\pgfsetmiterjoin%
\definecolor{currentfill}{rgb}{0.321261,0.371165,0.664360}%
\pgfsetfillcolor{currentfill}%
\pgfsetlinewidth{0.000000pt}%
\definecolor{currentstroke}{rgb}{0.000000,0.000000,0.000000}%
\pgfsetstrokecolor{currentstroke}%
\pgfsetstrokeopacity{0.000000}%
\pgfsetdash{}{0pt}%
\pgfpathmoveto{\pgfqpoint{3.279629in}{1.467013in}}%
\pgfpathlineto{\pgfqpoint{3.243760in}{1.467013in}}%
\pgfpathlineto{\pgfqpoint{3.243760in}{1.487942in}}%
\pgfpathlineto{\pgfqpoint{3.279629in}{1.487942in}}%
\pgfpathlineto{\pgfqpoint{3.279629in}{1.467013in}}%
\pgfpathclose%
\pgfusepath{fill}%
\end{pgfscope}%
\begin{pgfscope}%
\pgfpathrectangle{\pgfqpoint{3.019583in}{0.169444in}}{\pgfqpoint{0.896708in}{1.339426in}}%
\pgfusepath{clip}%
\pgfsetbuttcap%
\pgfsetmiterjoin%
\definecolor{currentfill}{rgb}{0.348328,0.336101,0.647751}%
\pgfsetfillcolor{currentfill}%
\pgfsetlinewidth{0.000000pt}%
\definecolor{currentstroke}{rgb}{0.000000,0.000000,0.000000}%
\pgfsetstrokecolor{currentstroke}%
\pgfsetstrokeopacity{0.000000}%
\pgfsetdash{}{0pt}%
\pgfpathmoveto{\pgfqpoint{3.279629in}{1.487942in}}%
\pgfpathlineto{\pgfqpoint{3.243760in}{1.487942in}}%
\pgfpathlineto{\pgfqpoint{3.243760in}{1.508871in}}%
\pgfpathlineto{\pgfqpoint{3.279629in}{1.508871in}}%
\pgfpathlineto{\pgfqpoint{3.279629in}{1.487942in}}%
\pgfpathclose%
\pgfusepath{fill}%
\end{pgfscope}%
\begin{pgfscope}%
\pgfpathrectangle{\pgfqpoint{3.019583in}{0.169444in}}{\pgfqpoint{0.896708in}{1.339426in}}%
\pgfusepath{clip}%
\pgfsetbuttcap%
\pgfsetmiterjoin%
\definecolor{currentfill}{rgb}{0.368627,0.309804,0.635294}%
\pgfsetfillcolor{currentfill}%
\pgfsetlinewidth{0.000000pt}%
\definecolor{currentstroke}{rgb}{0.000000,0.000000,0.000000}%
\pgfsetstrokecolor{currentstroke}%
\pgfsetstrokeopacity{0.000000}%
\pgfsetdash{}{0pt}%
\pgfpathmoveto{\pgfqpoint{3.279629in}{1.508871in}}%
\pgfpathlineto{\pgfqpoint{3.243760in}{1.508871in}}%
\pgfpathlineto{\pgfqpoint{3.243760in}{1.529799in}}%
\pgfpathlineto{\pgfqpoint{3.279629in}{1.529799in}}%
\pgfpathlineto{\pgfqpoint{3.279629in}{1.508871in}}%
\pgfpathclose%
\pgfusepath{fill}%
\end{pgfscope}%
\begin{pgfscope}%
\pgfpathrectangle{\pgfqpoint{3.019583in}{0.169444in}}{\pgfqpoint{0.896708in}{1.339426in}}%
\pgfusepath{clip}%
\pgfsetbuttcap%
\pgfsetmiterjoin%
\definecolor{currentfill}{rgb}{0.995156,0.832295,0.506344}%
\pgfsetfillcolor{currentfill}%
\pgfsetlinewidth{0.000000pt}%
\definecolor{currentstroke}{rgb}{0.000000,0.000000,0.000000}%
\pgfsetstrokecolor{currentstroke}%
\pgfsetstrokeopacity{0.000000}%
\pgfsetdash{}{0pt}%
\pgfpathmoveto{\pgfqpoint{3.163057in}{0.671729in}}%
\pgfpathlineto{\pgfqpoint{3.198925in}{0.671729in}}%
\pgfpathlineto{\pgfqpoint{3.198925in}{0.713586in}}%
\pgfpathlineto{\pgfqpoint{3.163057in}{0.713586in}}%
\pgfpathlineto{\pgfqpoint{3.163057in}{0.671729in}}%
\pgfpathclose%
\pgfusepath{fill}%
\end{pgfscope}%
\begin{pgfscope}%
\pgfpathrectangle{\pgfqpoint{3.019583in}{0.169444in}}{\pgfqpoint{0.896708in}{1.339426in}}%
\pgfusepath{clip}%
\pgfsetbuttcap%
\pgfsetmiterjoin%
\definecolor{currentfill}{rgb}{0.997616,0.926105,0.625067}%
\pgfsetfillcolor{currentfill}%
\pgfsetlinewidth{0.000000pt}%
\definecolor{currentstroke}{rgb}{0.000000,0.000000,0.000000}%
\pgfsetstrokecolor{currentstroke}%
\pgfsetstrokeopacity{0.000000}%
\pgfsetdash{}{0pt}%
\pgfpathmoveto{\pgfqpoint{3.180991in}{0.755443in}}%
\pgfpathlineto{\pgfqpoint{3.198925in}{0.755443in}}%
\pgfpathlineto{\pgfqpoint{3.198925in}{0.797300in}}%
\pgfpathlineto{\pgfqpoint{3.180991in}{0.797300in}}%
\pgfpathlineto{\pgfqpoint{3.180991in}{0.755443in}}%
\pgfpathclose%
\pgfusepath{fill}%
\end{pgfscope}%
\begin{pgfscope}%
\pgfpathrectangle{\pgfqpoint{3.019583in}{0.169444in}}{\pgfqpoint{0.896708in}{1.339426in}}%
\pgfusepath{clip}%
\pgfsetbuttcap%
\pgfsetmiterjoin%
\definecolor{currentfill}{rgb}{0.998847,0.964245,0.689043}%
\pgfsetfillcolor{currentfill}%
\pgfsetlinewidth{0.000000pt}%
\definecolor{currentstroke}{rgb}{0.000000,0.000000,0.000000}%
\pgfsetstrokecolor{currentstroke}%
\pgfsetstrokeopacity{0.000000}%
\pgfsetdash{}{0pt}%
\pgfpathmoveto{\pgfqpoint{3.145122in}{0.797300in}}%
\pgfpathlineto{\pgfqpoint{3.198925in}{0.797300in}}%
\pgfpathlineto{\pgfqpoint{3.198925in}{0.839157in}}%
\pgfpathlineto{\pgfqpoint{3.145122in}{0.839157in}}%
\pgfpathlineto{\pgfqpoint{3.145122in}{0.797300in}}%
\pgfpathclose%
\pgfusepath{fill}%
\end{pgfscope}%
\begin{pgfscope}%
\pgfpathrectangle{\pgfqpoint{3.019583in}{0.169444in}}{\pgfqpoint{0.896708in}{1.339426in}}%
\pgfusepath{clip}%
\pgfsetbuttcap%
\pgfsetmiterjoin%
\definecolor{currentfill}{rgb}{0.998078,0.999231,0.746021}%
\pgfsetfillcolor{currentfill}%
\pgfsetlinewidth{0.000000pt}%
\definecolor{currentstroke}{rgb}{0.000000,0.000000,0.000000}%
\pgfsetstrokecolor{currentstroke}%
\pgfsetstrokeopacity{0.000000}%
\pgfsetdash{}{0pt}%
\pgfpathmoveto{\pgfqpoint{3.145122in}{0.839157in}}%
\pgfpathlineto{\pgfqpoint{3.198925in}{0.839157in}}%
\pgfpathlineto{\pgfqpoint{3.198925in}{0.881015in}}%
\pgfpathlineto{\pgfqpoint{3.145122in}{0.881015in}}%
\pgfpathlineto{\pgfqpoint{3.145122in}{0.839157in}}%
\pgfpathclose%
\pgfusepath{fill}%
\end{pgfscope}%
\begin{pgfscope}%
\pgfpathrectangle{\pgfqpoint{3.019583in}{0.169444in}}{\pgfqpoint{0.896708in}{1.339426in}}%
\pgfusepath{clip}%
\pgfsetbuttcap%
\pgfsetmiterjoin%
\definecolor{currentfill}{rgb}{0.967320,0.986928,0.698039}%
\pgfsetfillcolor{currentfill}%
\pgfsetlinewidth{0.000000pt}%
\definecolor{currentstroke}{rgb}{0.000000,0.000000,0.000000}%
\pgfsetstrokecolor{currentstroke}%
\pgfsetstrokeopacity{0.000000}%
\pgfsetdash{}{0pt}%
\pgfpathmoveto{\pgfqpoint{3.037517in}{0.881015in}}%
\pgfpathlineto{\pgfqpoint{3.198925in}{0.881015in}}%
\pgfpathlineto{\pgfqpoint{3.198925in}{0.922872in}}%
\pgfpathlineto{\pgfqpoint{3.037517in}{0.922872in}}%
\pgfpathlineto{\pgfqpoint{3.037517in}{0.881015in}}%
\pgfpathclose%
\pgfusepath{fill}%
\end{pgfscope}%
\begin{pgfscope}%
\pgfpathrectangle{\pgfqpoint{3.019583in}{0.169444in}}{\pgfqpoint{0.896708in}{1.339426in}}%
\pgfusepath{clip}%
\pgfsetbuttcap%
\pgfsetmiterjoin%
\definecolor{currentfill}{rgb}{0.936563,0.974625,0.650058}%
\pgfsetfillcolor{currentfill}%
\pgfsetlinewidth{0.000000pt}%
\definecolor{currentstroke}{rgb}{0.000000,0.000000,0.000000}%
\pgfsetstrokecolor{currentstroke}%
\pgfsetstrokeopacity{0.000000}%
\pgfsetdash{}{0pt}%
\pgfpathmoveto{\pgfqpoint{3.091320in}{0.922872in}}%
\pgfpathlineto{\pgfqpoint{3.198925in}{0.922872in}}%
\pgfpathlineto{\pgfqpoint{3.198925in}{0.964729in}}%
\pgfpathlineto{\pgfqpoint{3.091320in}{0.964729in}}%
\pgfpathlineto{\pgfqpoint{3.091320in}{0.922872in}}%
\pgfpathclose%
\pgfusepath{fill}%
\end{pgfscope}%
\begin{pgfscope}%
\pgfpathrectangle{\pgfqpoint{3.019583in}{0.169444in}}{\pgfqpoint{0.896708in}{1.339426in}}%
\pgfusepath{clip}%
\pgfsetbuttcap%
\pgfsetmiterjoin%
\definecolor{currentfill}{rgb}{0.905805,0.962322,0.602076}%
\pgfsetfillcolor{currentfill}%
\pgfsetlinewidth{0.000000pt}%
\definecolor{currentstroke}{rgb}{0.000000,0.000000,0.000000}%
\pgfsetstrokecolor{currentstroke}%
\pgfsetstrokeopacity{0.000000}%
\pgfsetdash{}{0pt}%
\pgfpathmoveto{\pgfqpoint{3.091320in}{0.964729in}}%
\pgfpathlineto{\pgfqpoint{3.198925in}{0.964729in}}%
\pgfpathlineto{\pgfqpoint{3.198925in}{1.006586in}}%
\pgfpathlineto{\pgfqpoint{3.091320in}{1.006586in}}%
\pgfpathlineto{\pgfqpoint{3.091320in}{0.964729in}}%
\pgfpathclose%
\pgfusepath{fill}%
\end{pgfscope}%
\begin{pgfscope}%
\pgfpathrectangle{\pgfqpoint{3.019583in}{0.169444in}}{\pgfqpoint{0.896708in}{1.339426in}}%
\pgfusepath{clip}%
\pgfsetbuttcap%
\pgfsetmiterjoin%
\definecolor{currentfill}{rgb}{0.838447,0.934948,0.608997}%
\pgfsetfillcolor{currentfill}%
\pgfsetlinewidth{0.000000pt}%
\definecolor{currentstroke}{rgb}{0.000000,0.000000,0.000000}%
\pgfsetstrokecolor{currentstroke}%
\pgfsetstrokeopacity{0.000000}%
\pgfsetdash{}{0pt}%
\pgfpathmoveto{\pgfqpoint{3.091320in}{1.006586in}}%
\pgfpathlineto{\pgfqpoint{3.198925in}{1.006586in}}%
\pgfpathlineto{\pgfqpoint{3.198925in}{1.048443in}}%
\pgfpathlineto{\pgfqpoint{3.091320in}{1.048443in}}%
\pgfpathlineto{\pgfqpoint{3.091320in}{1.006586in}}%
\pgfpathclose%
\pgfusepath{fill}%
\end{pgfscope}%
\begin{pgfscope}%
\pgfpathrectangle{\pgfqpoint{3.019583in}{0.169444in}}{\pgfqpoint{0.896708in}{1.339426in}}%
\pgfusepath{clip}%
\pgfsetbuttcap%
\pgfsetmiterjoin%
\definecolor{currentfill}{rgb}{0.765859,0.905421,0.623760}%
\pgfsetfillcolor{currentfill}%
\pgfsetlinewidth{0.000000pt}%
\definecolor{currentstroke}{rgb}{0.000000,0.000000,0.000000}%
\pgfsetstrokecolor{currentstroke}%
\pgfsetstrokeopacity{0.000000}%
\pgfsetdash{}{0pt}%
\pgfpathmoveto{\pgfqpoint{3.109254in}{1.048443in}}%
\pgfpathlineto{\pgfqpoint{3.198925in}{1.048443in}}%
\pgfpathlineto{\pgfqpoint{3.198925in}{1.090300in}}%
\pgfpathlineto{\pgfqpoint{3.109254in}{1.090300in}}%
\pgfpathlineto{\pgfqpoint{3.109254in}{1.048443in}}%
\pgfpathclose%
\pgfusepath{fill}%
\end{pgfscope}%
\begin{pgfscope}%
\pgfpathrectangle{\pgfqpoint{3.019583in}{0.169444in}}{\pgfqpoint{0.896708in}{1.339426in}}%
\pgfusepath{clip}%
\pgfsetbuttcap%
\pgfsetmiterjoin%
\definecolor{currentfill}{rgb}{0.693272,0.875894,0.638524}%
\pgfsetfillcolor{currentfill}%
\pgfsetlinewidth{0.000000pt}%
\definecolor{currentstroke}{rgb}{0.000000,0.000000,0.000000}%
\pgfsetstrokecolor{currentstroke}%
\pgfsetstrokeopacity{0.000000}%
\pgfsetdash{}{0pt}%
\pgfpathmoveto{\pgfqpoint{3.180991in}{1.090300in}}%
\pgfpathlineto{\pgfqpoint{3.198925in}{1.090300in}}%
\pgfpathlineto{\pgfqpoint{3.198925in}{1.132157in}}%
\pgfpathlineto{\pgfqpoint{3.180991in}{1.132157in}}%
\pgfpathlineto{\pgfqpoint{3.180991in}{1.090300in}}%
\pgfpathclose%
\pgfusepath{fill}%
\end{pgfscope}%
\begin{pgfscope}%
\pgfpathrectangle{\pgfqpoint{3.019583in}{0.169444in}}{\pgfqpoint{0.896708in}{1.339426in}}%
\pgfusepath{clip}%
\pgfsetbuttcap%
\pgfsetmiterjoin%
\definecolor{currentfill}{rgb}{0.612226,0.843829,0.643983}%
\pgfsetfillcolor{currentfill}%
\pgfsetlinewidth{0.000000pt}%
\definecolor{currentstroke}{rgb}{0.000000,0.000000,0.000000}%
\pgfsetstrokecolor{currentstroke}%
\pgfsetstrokeopacity{0.000000}%
\pgfsetdash{}{0pt}%
\pgfpathmoveto{\pgfqpoint{3.145122in}{1.132157in}}%
\pgfpathlineto{\pgfqpoint{3.198925in}{1.132157in}}%
\pgfpathlineto{\pgfqpoint{3.198925in}{1.174014in}}%
\pgfpathlineto{\pgfqpoint{3.145122in}{1.174014in}}%
\pgfpathlineto{\pgfqpoint{3.145122in}{1.132157in}}%
\pgfpathclose%
\pgfusepath{fill}%
\end{pgfscope}%
\begin{pgfscope}%
\pgfpathrectangle{\pgfqpoint{3.019583in}{0.169444in}}{\pgfqpoint{0.896708in}{1.339426in}}%
\pgfusepath{clip}%
\pgfsetbuttcap%
\pgfsetmiterjoin%
\definecolor{currentfill}{rgb}{0.442445,0.777393,0.646444}%
\pgfsetfillcolor{currentfill}%
\pgfsetlinewidth{0.000000pt}%
\definecolor{currentstroke}{rgb}{0.000000,0.000000,0.000000}%
\pgfsetstrokecolor{currentstroke}%
\pgfsetstrokeopacity{0.000000}%
\pgfsetdash{}{0pt}%
\pgfpathmoveto{\pgfqpoint{3.145122in}{1.215871in}}%
\pgfpathlineto{\pgfqpoint{3.198925in}{1.215871in}}%
\pgfpathlineto{\pgfqpoint{3.198925in}{1.257728in}}%
\pgfpathlineto{\pgfqpoint{3.145122in}{1.257728in}}%
\pgfpathlineto{\pgfqpoint{3.145122in}{1.215871in}}%
\pgfpathclose%
\pgfusepath{fill}%
\end{pgfscope}%
\begin{pgfscope}%
\pgfpathrectangle{\pgfqpoint{3.019583in}{0.169444in}}{\pgfqpoint{0.896708in}{1.339426in}}%
\pgfusepath{clip}%
\pgfsetbuttcap%
\pgfsetmiterjoin%
\definecolor{currentfill}{rgb}{0.304037,0.653749,0.691349}%
\pgfsetfillcolor{currentfill}%
\pgfsetlinewidth{0.000000pt}%
\definecolor{currentstroke}{rgb}{0.000000,0.000000,0.000000}%
\pgfsetstrokecolor{currentstroke}%
\pgfsetstrokeopacity{0.000000}%
\pgfsetdash{}{0pt}%
\pgfpathmoveto{\pgfqpoint{3.180991in}{1.299585in}}%
\pgfpathlineto{\pgfqpoint{3.198925in}{1.299585in}}%
\pgfpathlineto{\pgfqpoint{3.198925in}{1.341442in}}%
\pgfpathlineto{\pgfqpoint{3.180991in}{1.341442in}}%
\pgfpathlineto{\pgfqpoint{3.180991in}{1.299585in}}%
\pgfpathclose%
\pgfusepath{fill}%
\end{pgfscope}%
\begin{pgfscope}%
\pgfpathrectangle{\pgfqpoint{3.019583in}{0.169444in}}{\pgfqpoint{0.896708in}{1.339426in}}%
\pgfusepath{clip}%
\pgfsetbuttcap%
\pgfsetmiterjoin%
\definecolor{currentfill}{rgb}{0.267128,0.441292,0.697578}%
\pgfsetfillcolor{currentfill}%
\pgfsetlinewidth{0.000000pt}%
\definecolor{currentstroke}{rgb}{0.000000,0.000000,0.000000}%
\pgfsetstrokecolor{currentstroke}%
\pgfsetstrokeopacity{0.000000}%
\pgfsetdash{}{0pt}%
\pgfpathmoveto{\pgfqpoint{3.180991in}{1.425156in}}%
\pgfpathlineto{\pgfqpoint{3.198925in}{1.425156in}}%
\pgfpathlineto{\pgfqpoint{3.198925in}{1.467013in}}%
\pgfpathlineto{\pgfqpoint{3.180991in}{1.467013in}}%
\pgfpathlineto{\pgfqpoint{3.180991in}{1.425156in}}%
\pgfpathclose%
\pgfusepath{fill}%
\end{pgfscope}%
\begin{pgfscope}%
\pgfpathrectangle{\pgfqpoint{3.019583in}{0.169444in}}{\pgfqpoint{0.896708in}{1.339426in}}%
\pgfusepath{clip}%
\pgfsetbuttcap%
\pgfsetmiterjoin%
\definecolor{currentfill}{rgb}{0.321261,0.371165,0.664360}%
\pgfsetfillcolor{currentfill}%
\pgfsetlinewidth{0.000000pt}%
\definecolor{currentstroke}{rgb}{0.000000,0.000000,0.000000}%
\pgfsetstrokecolor{currentstroke}%
\pgfsetstrokeopacity{0.000000}%
\pgfsetdash{}{0pt}%
\pgfpathmoveto{\pgfqpoint{3.180991in}{1.467013in}}%
\pgfpathlineto{\pgfqpoint{3.198925in}{1.467013in}}%
\pgfpathlineto{\pgfqpoint{3.198925in}{1.508871in}}%
\pgfpathlineto{\pgfqpoint{3.180991in}{1.508871in}}%
\pgfpathlineto{\pgfqpoint{3.180991in}{1.467013in}}%
\pgfpathclose%
\pgfusepath{fill}%
\end{pgfscope}%
\begin{pgfscope}%
\pgfpathrectangle{\pgfqpoint{3.019583in}{0.169444in}}{\pgfqpoint{0.896708in}{1.339426in}}%
\pgfusepath{clip}%
\pgfsetbuttcap%
\pgfsetmiterjoin%
\definecolor{currentfill}{rgb}{0.619608,0.003922,0.258824}%
\pgfsetfillcolor{currentfill}%
\pgfsetlinewidth{0.000000pt}%
\definecolor{currentstroke}{rgb}{0.000000,0.000000,0.000000}%
\pgfsetstrokecolor{currentstroke}%
\pgfsetstrokeopacity{0.000000}%
\pgfsetdash{}{0pt}%
\pgfpathmoveto{\pgfqpoint{3.521740in}{0.169444in}}%
\pgfpathlineto{\pgfqpoint{3.485872in}{0.169444in}}%
\pgfpathlineto{\pgfqpoint{3.485872in}{0.190373in}}%
\pgfpathlineto{\pgfqpoint{3.521740in}{0.190373in}}%
\pgfpathlineto{\pgfqpoint{3.521740in}{0.169444in}}%
\pgfpathclose%
\pgfusepath{fill}%
\end{pgfscope}%
\begin{pgfscope}%
\pgfpathrectangle{\pgfqpoint{3.019583in}{0.169444in}}{\pgfqpoint{0.896708in}{1.339426in}}%
\pgfusepath{clip}%
\pgfsetbuttcap%
\pgfsetmiterjoin%
\definecolor{currentfill}{rgb}{0.653441,0.041446,0.266820}%
\pgfsetfillcolor{currentfill}%
\pgfsetlinewidth{0.000000pt}%
\definecolor{currentstroke}{rgb}{0.000000,0.000000,0.000000}%
\pgfsetstrokecolor{currentstroke}%
\pgfsetstrokeopacity{0.000000}%
\pgfsetdash{}{0pt}%
\pgfpathmoveto{\pgfqpoint{3.521740in}{0.190373in}}%
\pgfpathlineto{\pgfqpoint{3.485872in}{0.190373in}}%
\pgfpathlineto{\pgfqpoint{3.485872in}{0.211302in}}%
\pgfpathlineto{\pgfqpoint{3.521740in}{0.211302in}}%
\pgfpathlineto{\pgfqpoint{3.521740in}{0.190373in}}%
\pgfpathclose%
\pgfusepath{fill}%
\end{pgfscope}%
\begin{pgfscope}%
\pgfpathrectangle{\pgfqpoint{3.019583in}{0.169444in}}{\pgfqpoint{0.896708in}{1.339426in}}%
\pgfusepath{clip}%
\pgfsetbuttcap%
\pgfsetmiterjoin%
\definecolor{currentfill}{rgb}{0.687274,0.078970,0.274817}%
\pgfsetfillcolor{currentfill}%
\pgfsetlinewidth{0.000000pt}%
\definecolor{currentstroke}{rgb}{0.000000,0.000000,0.000000}%
\pgfsetstrokecolor{currentstroke}%
\pgfsetstrokeopacity{0.000000}%
\pgfsetdash{}{0pt}%
\pgfpathmoveto{\pgfqpoint{3.521740in}{0.211302in}}%
\pgfpathlineto{\pgfqpoint{3.485872in}{0.211302in}}%
\pgfpathlineto{\pgfqpoint{3.485872in}{0.232230in}}%
\pgfpathlineto{\pgfqpoint{3.521740in}{0.232230in}}%
\pgfpathlineto{\pgfqpoint{3.521740in}{0.211302in}}%
\pgfpathclose%
\pgfusepath{fill}%
\end{pgfscope}%
\begin{pgfscope}%
\pgfpathrectangle{\pgfqpoint{3.019583in}{0.169444in}}{\pgfqpoint{0.896708in}{1.339426in}}%
\pgfusepath{clip}%
\pgfsetbuttcap%
\pgfsetmiterjoin%
\definecolor{currentfill}{rgb}{0.721107,0.116494,0.282814}%
\pgfsetfillcolor{currentfill}%
\pgfsetlinewidth{0.000000pt}%
\definecolor{currentstroke}{rgb}{0.000000,0.000000,0.000000}%
\pgfsetstrokecolor{currentstroke}%
\pgfsetstrokeopacity{0.000000}%
\pgfsetdash{}{0pt}%
\pgfpathmoveto{\pgfqpoint{3.521740in}{0.232230in}}%
\pgfpathlineto{\pgfqpoint{3.485872in}{0.232230in}}%
\pgfpathlineto{\pgfqpoint{3.485872in}{0.253159in}}%
\pgfpathlineto{\pgfqpoint{3.521740in}{0.253159in}}%
\pgfpathlineto{\pgfqpoint{3.521740in}{0.232230in}}%
\pgfpathclose%
\pgfusepath{fill}%
\end{pgfscope}%
\begin{pgfscope}%
\pgfpathrectangle{\pgfqpoint{3.019583in}{0.169444in}}{\pgfqpoint{0.896708in}{1.339426in}}%
\pgfusepath{clip}%
\pgfsetbuttcap%
\pgfsetmiterjoin%
\definecolor{currentfill}{rgb}{0.754940,0.154018,0.290811}%
\pgfsetfillcolor{currentfill}%
\pgfsetlinewidth{0.000000pt}%
\definecolor{currentstroke}{rgb}{0.000000,0.000000,0.000000}%
\pgfsetstrokecolor{currentstroke}%
\pgfsetstrokeopacity{0.000000}%
\pgfsetdash{}{0pt}%
\pgfpathmoveto{\pgfqpoint{3.521740in}{0.253159in}}%
\pgfpathlineto{\pgfqpoint{3.485872in}{0.253159in}}%
\pgfpathlineto{\pgfqpoint{3.485872in}{0.274087in}}%
\pgfpathlineto{\pgfqpoint{3.521740in}{0.274087in}}%
\pgfpathlineto{\pgfqpoint{3.521740in}{0.253159in}}%
\pgfpathclose%
\pgfusepath{fill}%
\end{pgfscope}%
\begin{pgfscope}%
\pgfpathrectangle{\pgfqpoint{3.019583in}{0.169444in}}{\pgfqpoint{0.896708in}{1.339426in}}%
\pgfusepath{clip}%
\pgfsetbuttcap%
\pgfsetmiterjoin%
\definecolor{currentfill}{rgb}{0.788774,0.191542,0.298808}%
\pgfsetfillcolor{currentfill}%
\pgfsetlinewidth{0.000000pt}%
\definecolor{currentstroke}{rgb}{0.000000,0.000000,0.000000}%
\pgfsetstrokecolor{currentstroke}%
\pgfsetstrokeopacity{0.000000}%
\pgfsetdash{}{0pt}%
\pgfpathmoveto{\pgfqpoint{3.521740in}{0.274087in}}%
\pgfpathlineto{\pgfqpoint{3.485872in}{0.274087in}}%
\pgfpathlineto{\pgfqpoint{3.485872in}{0.295016in}}%
\pgfpathlineto{\pgfqpoint{3.521740in}{0.295016in}}%
\pgfpathlineto{\pgfqpoint{3.521740in}{0.274087in}}%
\pgfpathclose%
\pgfusepath{fill}%
\end{pgfscope}%
\begin{pgfscope}%
\pgfpathrectangle{\pgfqpoint{3.019583in}{0.169444in}}{\pgfqpoint{0.896708in}{1.339426in}}%
\pgfusepath{clip}%
\pgfsetbuttcap%
\pgfsetmiterjoin%
\definecolor{currentfill}{rgb}{0.822607,0.229066,0.306805}%
\pgfsetfillcolor{currentfill}%
\pgfsetlinewidth{0.000000pt}%
\definecolor{currentstroke}{rgb}{0.000000,0.000000,0.000000}%
\pgfsetstrokecolor{currentstroke}%
\pgfsetstrokeopacity{0.000000}%
\pgfsetdash{}{0pt}%
\pgfpathmoveto{\pgfqpoint{3.521740in}{0.295016in}}%
\pgfpathlineto{\pgfqpoint{3.485872in}{0.295016in}}%
\pgfpathlineto{\pgfqpoint{3.485872in}{0.315944in}}%
\pgfpathlineto{\pgfqpoint{3.521740in}{0.315944in}}%
\pgfpathlineto{\pgfqpoint{3.521740in}{0.295016in}}%
\pgfpathclose%
\pgfusepath{fill}%
\end{pgfscope}%
\begin{pgfscope}%
\pgfpathrectangle{\pgfqpoint{3.019583in}{0.169444in}}{\pgfqpoint{0.896708in}{1.339426in}}%
\pgfusepath{clip}%
\pgfsetbuttcap%
\pgfsetmiterjoin%
\definecolor{currentfill}{rgb}{0.847213,0.261207,0.305190}%
\pgfsetfillcolor{currentfill}%
\pgfsetlinewidth{0.000000pt}%
\definecolor{currentstroke}{rgb}{0.000000,0.000000,0.000000}%
\pgfsetstrokecolor{currentstroke}%
\pgfsetstrokeopacity{0.000000}%
\pgfsetdash{}{0pt}%
\pgfpathmoveto{\pgfqpoint{3.521740in}{0.315944in}}%
\pgfpathlineto{\pgfqpoint{3.485872in}{0.315944in}}%
\pgfpathlineto{\pgfqpoint{3.485872in}{0.336873in}}%
\pgfpathlineto{\pgfqpoint{3.521740in}{0.336873in}}%
\pgfpathlineto{\pgfqpoint{3.521740in}{0.315944in}}%
\pgfpathclose%
\pgfusepath{fill}%
\end{pgfscope}%
\begin{pgfscope}%
\pgfpathrectangle{\pgfqpoint{3.019583in}{0.169444in}}{\pgfqpoint{0.896708in}{1.339426in}}%
\pgfusepath{clip}%
\pgfsetbuttcap%
\pgfsetmiterjoin%
\definecolor{currentfill}{rgb}{0.866282,0.290119,0.297809}%
\pgfsetfillcolor{currentfill}%
\pgfsetlinewidth{0.000000pt}%
\definecolor{currentstroke}{rgb}{0.000000,0.000000,0.000000}%
\pgfsetstrokecolor{currentstroke}%
\pgfsetstrokeopacity{0.000000}%
\pgfsetdash{}{0pt}%
\pgfpathmoveto{\pgfqpoint{3.521740in}{0.336873in}}%
\pgfpathlineto{\pgfqpoint{3.485872in}{0.336873in}}%
\pgfpathlineto{\pgfqpoint{3.485872in}{0.357801in}}%
\pgfpathlineto{\pgfqpoint{3.521740in}{0.357801in}}%
\pgfpathlineto{\pgfqpoint{3.521740in}{0.336873in}}%
\pgfpathclose%
\pgfusepath{fill}%
\end{pgfscope}%
\begin{pgfscope}%
\pgfpathrectangle{\pgfqpoint{3.019583in}{0.169444in}}{\pgfqpoint{0.896708in}{1.339426in}}%
\pgfusepath{clip}%
\pgfsetbuttcap%
\pgfsetmiterjoin%
\definecolor{currentfill}{rgb}{0.885352,0.319031,0.290427}%
\pgfsetfillcolor{currentfill}%
\pgfsetlinewidth{0.000000pt}%
\definecolor{currentstroke}{rgb}{0.000000,0.000000,0.000000}%
\pgfsetstrokecolor{currentstroke}%
\pgfsetstrokeopacity{0.000000}%
\pgfsetdash{}{0pt}%
\pgfpathmoveto{\pgfqpoint{3.521740in}{0.357801in}}%
\pgfpathlineto{\pgfqpoint{3.485872in}{0.357801in}}%
\pgfpathlineto{\pgfqpoint{3.485872in}{0.378730in}}%
\pgfpathlineto{\pgfqpoint{3.521740in}{0.378730in}}%
\pgfpathlineto{\pgfqpoint{3.521740in}{0.357801in}}%
\pgfpathclose%
\pgfusepath{fill}%
\end{pgfscope}%
\begin{pgfscope}%
\pgfpathrectangle{\pgfqpoint{3.019583in}{0.169444in}}{\pgfqpoint{0.896708in}{1.339426in}}%
\pgfusepath{clip}%
\pgfsetbuttcap%
\pgfsetmiterjoin%
\definecolor{currentfill}{rgb}{0.904421,0.347943,0.283045}%
\pgfsetfillcolor{currentfill}%
\pgfsetlinewidth{0.000000pt}%
\definecolor{currentstroke}{rgb}{0.000000,0.000000,0.000000}%
\pgfsetstrokecolor{currentstroke}%
\pgfsetstrokeopacity{0.000000}%
\pgfsetdash{}{0pt}%
\pgfpathmoveto{\pgfqpoint{3.521740in}{0.378730in}}%
\pgfpathlineto{\pgfqpoint{3.485872in}{0.378730in}}%
\pgfpathlineto{\pgfqpoint{3.485872in}{0.399658in}}%
\pgfpathlineto{\pgfqpoint{3.521740in}{0.399658in}}%
\pgfpathlineto{\pgfqpoint{3.521740in}{0.378730in}}%
\pgfpathclose%
\pgfusepath{fill}%
\end{pgfscope}%
\begin{pgfscope}%
\pgfpathrectangle{\pgfqpoint{3.019583in}{0.169444in}}{\pgfqpoint{0.896708in}{1.339426in}}%
\pgfusepath{clip}%
\pgfsetbuttcap%
\pgfsetmiterjoin%
\definecolor{currentfill}{rgb}{0.923491,0.376855,0.275663}%
\pgfsetfillcolor{currentfill}%
\pgfsetlinewidth{0.000000pt}%
\definecolor{currentstroke}{rgb}{0.000000,0.000000,0.000000}%
\pgfsetstrokecolor{currentstroke}%
\pgfsetstrokeopacity{0.000000}%
\pgfsetdash{}{0pt}%
\pgfpathmoveto{\pgfqpoint{3.521740in}{0.399658in}}%
\pgfpathlineto{\pgfqpoint{3.485872in}{0.399658in}}%
\pgfpathlineto{\pgfqpoint{3.485872in}{0.420587in}}%
\pgfpathlineto{\pgfqpoint{3.521740in}{0.420587in}}%
\pgfpathlineto{\pgfqpoint{3.521740in}{0.399658in}}%
\pgfpathclose%
\pgfusepath{fill}%
\end{pgfscope}%
\begin{pgfscope}%
\pgfpathrectangle{\pgfqpoint{3.019583in}{0.169444in}}{\pgfqpoint{0.896708in}{1.339426in}}%
\pgfusepath{clip}%
\pgfsetbuttcap%
\pgfsetmiterjoin%
\definecolor{currentfill}{rgb}{0.942561,0.405767,0.268281}%
\pgfsetfillcolor{currentfill}%
\pgfsetlinewidth{0.000000pt}%
\definecolor{currentstroke}{rgb}{0.000000,0.000000,0.000000}%
\pgfsetstrokecolor{currentstroke}%
\pgfsetstrokeopacity{0.000000}%
\pgfsetdash{}{0pt}%
\pgfpathmoveto{\pgfqpoint{3.521740in}{0.420587in}}%
\pgfpathlineto{\pgfqpoint{3.485872in}{0.420587in}}%
\pgfpathlineto{\pgfqpoint{3.485872in}{0.441515in}}%
\pgfpathlineto{\pgfqpoint{3.521740in}{0.441515in}}%
\pgfpathlineto{\pgfqpoint{3.521740in}{0.420587in}}%
\pgfpathclose%
\pgfusepath{fill}%
\end{pgfscope}%
\begin{pgfscope}%
\pgfpathrectangle{\pgfqpoint{3.019583in}{0.169444in}}{\pgfqpoint{0.896708in}{1.339426in}}%
\pgfusepath{clip}%
\pgfsetbuttcap%
\pgfsetmiterjoin%
\definecolor{currentfill}{rgb}{0.958247,0.437447,0.267359}%
\pgfsetfillcolor{currentfill}%
\pgfsetlinewidth{0.000000pt}%
\definecolor{currentstroke}{rgb}{0.000000,0.000000,0.000000}%
\pgfsetstrokecolor{currentstroke}%
\pgfsetstrokeopacity{0.000000}%
\pgfsetdash{}{0pt}%
\pgfpathmoveto{\pgfqpoint{3.521740in}{0.441515in}}%
\pgfpathlineto{\pgfqpoint{3.485872in}{0.441515in}}%
\pgfpathlineto{\pgfqpoint{3.485872in}{0.462444in}}%
\pgfpathlineto{\pgfqpoint{3.521740in}{0.462444in}}%
\pgfpathlineto{\pgfqpoint{3.521740in}{0.441515in}}%
\pgfpathclose%
\pgfusepath{fill}%
\end{pgfscope}%
\begin{pgfscope}%
\pgfpathrectangle{\pgfqpoint{3.019583in}{0.169444in}}{\pgfqpoint{0.896708in}{1.339426in}}%
\pgfusepath{clip}%
\pgfsetbuttcap%
\pgfsetmiterjoin%
\definecolor{currentfill}{rgb}{0.963783,0.477432,0.285813}%
\pgfsetfillcolor{currentfill}%
\pgfsetlinewidth{0.000000pt}%
\definecolor{currentstroke}{rgb}{0.000000,0.000000,0.000000}%
\pgfsetstrokecolor{currentstroke}%
\pgfsetstrokeopacity{0.000000}%
\pgfsetdash{}{0pt}%
\pgfpathmoveto{\pgfqpoint{3.521740in}{0.462444in}}%
\pgfpathlineto{\pgfqpoint{3.485872in}{0.462444in}}%
\pgfpathlineto{\pgfqpoint{3.485872in}{0.483372in}}%
\pgfpathlineto{\pgfqpoint{3.521740in}{0.483372in}}%
\pgfpathlineto{\pgfqpoint{3.521740in}{0.462444in}}%
\pgfpathclose%
\pgfusepath{fill}%
\end{pgfscope}%
\begin{pgfscope}%
\pgfpathrectangle{\pgfqpoint{3.019583in}{0.169444in}}{\pgfqpoint{0.896708in}{1.339426in}}%
\pgfusepath{clip}%
\pgfsetbuttcap%
\pgfsetmiterjoin%
\definecolor{currentfill}{rgb}{0.969319,0.517416,0.304268}%
\pgfsetfillcolor{currentfill}%
\pgfsetlinewidth{0.000000pt}%
\definecolor{currentstroke}{rgb}{0.000000,0.000000,0.000000}%
\pgfsetstrokecolor{currentstroke}%
\pgfsetstrokeopacity{0.000000}%
\pgfsetdash{}{0pt}%
\pgfpathmoveto{\pgfqpoint{3.521740in}{0.483372in}}%
\pgfpathlineto{\pgfqpoint{3.485872in}{0.483372in}}%
\pgfpathlineto{\pgfqpoint{3.485872in}{0.504301in}}%
\pgfpathlineto{\pgfqpoint{3.521740in}{0.504301in}}%
\pgfpathlineto{\pgfqpoint{3.521740in}{0.483372in}}%
\pgfpathclose%
\pgfusepath{fill}%
\end{pgfscope}%
\begin{pgfscope}%
\pgfpathrectangle{\pgfqpoint{3.019583in}{0.169444in}}{\pgfqpoint{0.896708in}{1.339426in}}%
\pgfusepath{clip}%
\pgfsetbuttcap%
\pgfsetmiterjoin%
\definecolor{currentfill}{rgb}{0.974856,0.557401,0.322722}%
\pgfsetfillcolor{currentfill}%
\pgfsetlinewidth{0.000000pt}%
\definecolor{currentstroke}{rgb}{0.000000,0.000000,0.000000}%
\pgfsetstrokecolor{currentstroke}%
\pgfsetstrokeopacity{0.000000}%
\pgfsetdash{}{0pt}%
\pgfpathmoveto{\pgfqpoint{3.521740in}{0.504301in}}%
\pgfpathlineto{\pgfqpoint{3.485872in}{0.504301in}}%
\pgfpathlineto{\pgfqpoint{3.485872in}{0.525230in}}%
\pgfpathlineto{\pgfqpoint{3.521740in}{0.525230in}}%
\pgfpathlineto{\pgfqpoint{3.521740in}{0.504301in}}%
\pgfpathclose%
\pgfusepath{fill}%
\end{pgfscope}%
\begin{pgfscope}%
\pgfpathrectangle{\pgfqpoint{3.019583in}{0.169444in}}{\pgfqpoint{0.896708in}{1.339426in}}%
\pgfusepath{clip}%
\pgfsetbuttcap%
\pgfsetmiterjoin%
\definecolor{currentfill}{rgb}{0.980392,0.597386,0.341176}%
\pgfsetfillcolor{currentfill}%
\pgfsetlinewidth{0.000000pt}%
\definecolor{currentstroke}{rgb}{0.000000,0.000000,0.000000}%
\pgfsetstrokecolor{currentstroke}%
\pgfsetstrokeopacity{0.000000}%
\pgfsetdash{}{0pt}%
\pgfpathmoveto{\pgfqpoint{3.521740in}{0.525230in}}%
\pgfpathlineto{\pgfqpoint{3.485872in}{0.525230in}}%
\pgfpathlineto{\pgfqpoint{3.485872in}{0.546158in}}%
\pgfpathlineto{\pgfqpoint{3.521740in}{0.546158in}}%
\pgfpathlineto{\pgfqpoint{3.521740in}{0.525230in}}%
\pgfpathclose%
\pgfusepath{fill}%
\end{pgfscope}%
\begin{pgfscope}%
\pgfpathrectangle{\pgfqpoint{3.019583in}{0.169444in}}{\pgfqpoint{0.896708in}{1.339426in}}%
\pgfusepath{clip}%
\pgfsetbuttcap%
\pgfsetmiterjoin%
\definecolor{currentfill}{rgb}{0.985928,0.637370,0.359631}%
\pgfsetfillcolor{currentfill}%
\pgfsetlinewidth{0.000000pt}%
\definecolor{currentstroke}{rgb}{0.000000,0.000000,0.000000}%
\pgfsetstrokecolor{currentstroke}%
\pgfsetstrokeopacity{0.000000}%
\pgfsetdash{}{0pt}%
\pgfpathmoveto{\pgfqpoint{3.521740in}{0.546158in}}%
\pgfpathlineto{\pgfqpoint{3.485872in}{0.546158in}}%
\pgfpathlineto{\pgfqpoint{3.485872in}{0.567087in}}%
\pgfpathlineto{\pgfqpoint{3.521740in}{0.567087in}}%
\pgfpathlineto{\pgfqpoint{3.521740in}{0.546158in}}%
\pgfpathclose%
\pgfusepath{fill}%
\end{pgfscope}%
\begin{pgfscope}%
\pgfpathrectangle{\pgfqpoint{3.019583in}{0.169444in}}{\pgfqpoint{0.896708in}{1.339426in}}%
\pgfusepath{clip}%
\pgfsetbuttcap%
\pgfsetmiterjoin%
\definecolor{currentfill}{rgb}{0.991465,0.677355,0.378085}%
\pgfsetfillcolor{currentfill}%
\pgfsetlinewidth{0.000000pt}%
\definecolor{currentstroke}{rgb}{0.000000,0.000000,0.000000}%
\pgfsetstrokecolor{currentstroke}%
\pgfsetstrokeopacity{0.000000}%
\pgfsetdash{}{0pt}%
\pgfpathmoveto{\pgfqpoint{3.521740in}{0.567087in}}%
\pgfpathlineto{\pgfqpoint{3.485872in}{0.567087in}}%
\pgfpathlineto{\pgfqpoint{3.485872in}{0.588015in}}%
\pgfpathlineto{\pgfqpoint{3.521740in}{0.588015in}}%
\pgfpathlineto{\pgfqpoint{3.521740in}{0.567087in}}%
\pgfpathclose%
\pgfusepath{fill}%
\end{pgfscope}%
\begin{pgfscope}%
\pgfpathrectangle{\pgfqpoint{3.019583in}{0.169444in}}{\pgfqpoint{0.896708in}{1.339426in}}%
\pgfusepath{clip}%
\pgfsetbuttcap%
\pgfsetmiterjoin%
\definecolor{currentfill}{rgb}{0.992695,0.709266,0.402999}%
\pgfsetfillcolor{currentfill}%
\pgfsetlinewidth{0.000000pt}%
\definecolor{currentstroke}{rgb}{0.000000,0.000000,0.000000}%
\pgfsetstrokecolor{currentstroke}%
\pgfsetstrokeopacity{0.000000}%
\pgfsetdash{}{0pt}%
\pgfpathmoveto{\pgfqpoint{3.521740in}{0.588015in}}%
\pgfpathlineto{\pgfqpoint{3.485872in}{0.588015in}}%
\pgfpathlineto{\pgfqpoint{3.485872in}{0.608944in}}%
\pgfpathlineto{\pgfqpoint{3.521740in}{0.608944in}}%
\pgfpathlineto{\pgfqpoint{3.521740in}{0.588015in}}%
\pgfpathclose%
\pgfusepath{fill}%
\end{pgfscope}%
\begin{pgfscope}%
\pgfpathrectangle{\pgfqpoint{3.019583in}{0.169444in}}{\pgfqpoint{0.896708in}{1.339426in}}%
\pgfusepath{clip}%
\pgfsetbuttcap%
\pgfsetmiterjoin%
\definecolor{currentfill}{rgb}{0.993310,0.740023,0.428835}%
\pgfsetfillcolor{currentfill}%
\pgfsetlinewidth{0.000000pt}%
\definecolor{currentstroke}{rgb}{0.000000,0.000000,0.000000}%
\pgfsetstrokecolor{currentstroke}%
\pgfsetstrokeopacity{0.000000}%
\pgfsetdash{}{0pt}%
\pgfpathmoveto{\pgfqpoint{3.521740in}{0.608944in}}%
\pgfpathlineto{\pgfqpoint{3.485872in}{0.608944in}}%
\pgfpathlineto{\pgfqpoint{3.485872in}{0.629872in}}%
\pgfpathlineto{\pgfqpoint{3.521740in}{0.629872in}}%
\pgfpathlineto{\pgfqpoint{3.521740in}{0.608944in}}%
\pgfpathclose%
\pgfusepath{fill}%
\end{pgfscope}%
\begin{pgfscope}%
\pgfpathrectangle{\pgfqpoint{3.019583in}{0.169444in}}{\pgfqpoint{0.896708in}{1.339426in}}%
\pgfusepath{clip}%
\pgfsetbuttcap%
\pgfsetmiterjoin%
\definecolor{currentfill}{rgb}{0.993925,0.770780,0.454671}%
\pgfsetfillcolor{currentfill}%
\pgfsetlinewidth{0.000000pt}%
\definecolor{currentstroke}{rgb}{0.000000,0.000000,0.000000}%
\pgfsetstrokecolor{currentstroke}%
\pgfsetstrokeopacity{0.000000}%
\pgfsetdash{}{0pt}%
\pgfpathmoveto{\pgfqpoint{3.521740in}{0.629872in}}%
\pgfpathlineto{\pgfqpoint{3.485872in}{0.629872in}}%
\pgfpathlineto{\pgfqpoint{3.485872in}{0.650801in}}%
\pgfpathlineto{\pgfqpoint{3.521740in}{0.650801in}}%
\pgfpathlineto{\pgfqpoint{3.521740in}{0.629872in}}%
\pgfpathclose%
\pgfusepath{fill}%
\end{pgfscope}%
\begin{pgfscope}%
\pgfpathrectangle{\pgfqpoint{3.019583in}{0.169444in}}{\pgfqpoint{0.896708in}{1.339426in}}%
\pgfusepath{clip}%
\pgfsetbuttcap%
\pgfsetmiterjoin%
\definecolor{currentfill}{rgb}{0.994541,0.801538,0.480507}%
\pgfsetfillcolor{currentfill}%
\pgfsetlinewidth{0.000000pt}%
\definecolor{currentstroke}{rgb}{0.000000,0.000000,0.000000}%
\pgfsetstrokecolor{currentstroke}%
\pgfsetstrokeopacity{0.000000}%
\pgfsetdash{}{0pt}%
\pgfpathmoveto{\pgfqpoint{3.521740in}{0.650801in}}%
\pgfpathlineto{\pgfqpoint{3.485872in}{0.650801in}}%
\pgfpathlineto{\pgfqpoint{3.485872in}{0.671729in}}%
\pgfpathlineto{\pgfqpoint{3.521740in}{0.671729in}}%
\pgfpathlineto{\pgfqpoint{3.521740in}{0.650801in}}%
\pgfpathclose%
\pgfusepath{fill}%
\end{pgfscope}%
\begin{pgfscope}%
\pgfpathrectangle{\pgfqpoint{3.019583in}{0.169444in}}{\pgfqpoint{0.896708in}{1.339426in}}%
\pgfusepath{clip}%
\pgfsetbuttcap%
\pgfsetmiterjoin%
\definecolor{currentfill}{rgb}{0.995156,0.832295,0.506344}%
\pgfsetfillcolor{currentfill}%
\pgfsetlinewidth{0.000000pt}%
\definecolor{currentstroke}{rgb}{0.000000,0.000000,0.000000}%
\pgfsetstrokecolor{currentstroke}%
\pgfsetstrokeopacity{0.000000}%
\pgfsetdash{}{0pt}%
\pgfpathmoveto{\pgfqpoint{3.521740in}{0.671729in}}%
\pgfpathlineto{\pgfqpoint{3.485872in}{0.671729in}}%
\pgfpathlineto{\pgfqpoint{3.485872in}{0.692658in}}%
\pgfpathlineto{\pgfqpoint{3.521740in}{0.692658in}}%
\pgfpathlineto{\pgfqpoint{3.521740in}{0.671729in}}%
\pgfpathclose%
\pgfusepath{fill}%
\end{pgfscope}%
\begin{pgfscope}%
\pgfpathrectangle{\pgfqpoint{3.019583in}{0.169444in}}{\pgfqpoint{0.896708in}{1.339426in}}%
\pgfusepath{clip}%
\pgfsetbuttcap%
\pgfsetmiterjoin%
\definecolor{currentfill}{rgb}{0.995771,0.863053,0.532180}%
\pgfsetfillcolor{currentfill}%
\pgfsetlinewidth{0.000000pt}%
\definecolor{currentstroke}{rgb}{0.000000,0.000000,0.000000}%
\pgfsetstrokecolor{currentstroke}%
\pgfsetstrokeopacity{0.000000}%
\pgfsetdash{}{0pt}%
\pgfpathmoveto{\pgfqpoint{3.521740in}{0.692658in}}%
\pgfpathlineto{\pgfqpoint{3.485872in}{0.692658in}}%
\pgfpathlineto{\pgfqpoint{3.485872in}{0.713586in}}%
\pgfpathlineto{\pgfqpoint{3.521740in}{0.713586in}}%
\pgfpathlineto{\pgfqpoint{3.521740in}{0.692658in}}%
\pgfpathclose%
\pgfusepath{fill}%
\end{pgfscope}%
\begin{pgfscope}%
\pgfpathrectangle{\pgfqpoint{3.019583in}{0.169444in}}{\pgfqpoint{0.896708in}{1.339426in}}%
\pgfusepath{clip}%
\pgfsetbuttcap%
\pgfsetmiterjoin%
\definecolor{currentfill}{rgb}{0.996386,0.887966,0.561092}%
\pgfsetfillcolor{currentfill}%
\pgfsetlinewidth{0.000000pt}%
\definecolor{currentstroke}{rgb}{0.000000,0.000000,0.000000}%
\pgfsetstrokecolor{currentstroke}%
\pgfsetstrokeopacity{0.000000}%
\pgfsetdash{}{0pt}%
\pgfpathmoveto{\pgfqpoint{3.521740in}{0.713586in}}%
\pgfpathlineto{\pgfqpoint{3.485872in}{0.713586in}}%
\pgfpathlineto{\pgfqpoint{3.485872in}{0.734515in}}%
\pgfpathlineto{\pgfqpoint{3.521740in}{0.734515in}}%
\pgfpathlineto{\pgfqpoint{3.521740in}{0.713586in}}%
\pgfpathclose%
\pgfusepath{fill}%
\end{pgfscope}%
\begin{pgfscope}%
\pgfpathrectangle{\pgfqpoint{3.019583in}{0.169444in}}{\pgfqpoint{0.896708in}{1.339426in}}%
\pgfusepath{clip}%
\pgfsetbuttcap%
\pgfsetmiterjoin%
\definecolor{currentfill}{rgb}{0.997001,0.907036,0.593080}%
\pgfsetfillcolor{currentfill}%
\pgfsetlinewidth{0.000000pt}%
\definecolor{currentstroke}{rgb}{0.000000,0.000000,0.000000}%
\pgfsetstrokecolor{currentstroke}%
\pgfsetstrokeopacity{0.000000}%
\pgfsetdash{}{0pt}%
\pgfpathmoveto{\pgfqpoint{3.521740in}{0.734515in}}%
\pgfpathlineto{\pgfqpoint{3.485872in}{0.734515in}}%
\pgfpathlineto{\pgfqpoint{3.485872in}{0.755443in}}%
\pgfpathlineto{\pgfqpoint{3.521740in}{0.755443in}}%
\pgfpathlineto{\pgfqpoint{3.521740in}{0.734515in}}%
\pgfpathclose%
\pgfusepath{fill}%
\end{pgfscope}%
\begin{pgfscope}%
\pgfpathrectangle{\pgfqpoint{3.019583in}{0.169444in}}{\pgfqpoint{0.896708in}{1.339426in}}%
\pgfusepath{clip}%
\pgfsetbuttcap%
\pgfsetmiterjoin%
\definecolor{currentfill}{rgb}{0.997616,0.926105,0.625067}%
\pgfsetfillcolor{currentfill}%
\pgfsetlinewidth{0.000000pt}%
\definecolor{currentstroke}{rgb}{0.000000,0.000000,0.000000}%
\pgfsetstrokecolor{currentstroke}%
\pgfsetstrokeopacity{0.000000}%
\pgfsetdash{}{0pt}%
\pgfpathmoveto{\pgfqpoint{3.521740in}{0.755443in}}%
\pgfpathlineto{\pgfqpoint{3.485872in}{0.755443in}}%
\pgfpathlineto{\pgfqpoint{3.485872in}{0.776372in}}%
\pgfpathlineto{\pgfqpoint{3.521740in}{0.776372in}}%
\pgfpathlineto{\pgfqpoint{3.521740in}{0.755443in}}%
\pgfpathclose%
\pgfusepath{fill}%
\end{pgfscope}%
\begin{pgfscope}%
\pgfpathrectangle{\pgfqpoint{3.019583in}{0.169444in}}{\pgfqpoint{0.896708in}{1.339426in}}%
\pgfusepath{clip}%
\pgfsetbuttcap%
\pgfsetmiterjoin%
\definecolor{currentfill}{rgb}{0.998231,0.945175,0.657055}%
\pgfsetfillcolor{currentfill}%
\pgfsetlinewidth{0.000000pt}%
\definecolor{currentstroke}{rgb}{0.000000,0.000000,0.000000}%
\pgfsetstrokecolor{currentstroke}%
\pgfsetstrokeopacity{0.000000}%
\pgfsetdash{}{0pt}%
\pgfpathmoveto{\pgfqpoint{3.521740in}{0.776372in}}%
\pgfpathlineto{\pgfqpoint{3.485872in}{0.776372in}}%
\pgfpathlineto{\pgfqpoint{3.485872in}{0.797300in}}%
\pgfpathlineto{\pgfqpoint{3.521740in}{0.797300in}}%
\pgfpathlineto{\pgfqpoint{3.521740in}{0.776372in}}%
\pgfpathclose%
\pgfusepath{fill}%
\end{pgfscope}%
\begin{pgfscope}%
\pgfpathrectangle{\pgfqpoint{3.019583in}{0.169444in}}{\pgfqpoint{0.896708in}{1.339426in}}%
\pgfusepath{clip}%
\pgfsetbuttcap%
\pgfsetmiterjoin%
\definecolor{currentfill}{rgb}{0.998847,0.964245,0.689043}%
\pgfsetfillcolor{currentfill}%
\pgfsetlinewidth{0.000000pt}%
\definecolor{currentstroke}{rgb}{0.000000,0.000000,0.000000}%
\pgfsetstrokecolor{currentstroke}%
\pgfsetstrokeopacity{0.000000}%
\pgfsetdash{}{0pt}%
\pgfpathmoveto{\pgfqpoint{3.521740in}{0.797300in}}%
\pgfpathlineto{\pgfqpoint{3.485872in}{0.797300in}}%
\pgfpathlineto{\pgfqpoint{3.485872in}{0.818229in}}%
\pgfpathlineto{\pgfqpoint{3.521740in}{0.818229in}}%
\pgfpathlineto{\pgfqpoint{3.521740in}{0.797300in}}%
\pgfpathclose%
\pgfusepath{fill}%
\end{pgfscope}%
\begin{pgfscope}%
\pgfpathrectangle{\pgfqpoint{3.019583in}{0.169444in}}{\pgfqpoint{0.896708in}{1.339426in}}%
\pgfusepath{clip}%
\pgfsetbuttcap%
\pgfsetmiterjoin%
\definecolor{currentfill}{rgb}{0.999462,0.983314,0.721030}%
\pgfsetfillcolor{currentfill}%
\pgfsetlinewidth{0.000000pt}%
\definecolor{currentstroke}{rgb}{0.000000,0.000000,0.000000}%
\pgfsetstrokecolor{currentstroke}%
\pgfsetstrokeopacity{0.000000}%
\pgfsetdash{}{0pt}%
\pgfpathmoveto{\pgfqpoint{3.521740in}{0.818229in}}%
\pgfpathlineto{\pgfqpoint{3.485872in}{0.818229in}}%
\pgfpathlineto{\pgfqpoint{3.485872in}{0.839157in}}%
\pgfpathlineto{\pgfqpoint{3.521740in}{0.839157in}}%
\pgfpathlineto{\pgfqpoint{3.521740in}{0.818229in}}%
\pgfpathclose%
\pgfusepath{fill}%
\end{pgfscope}%
\begin{pgfscope}%
\pgfpathrectangle{\pgfqpoint{3.019583in}{0.169444in}}{\pgfqpoint{0.896708in}{1.339426in}}%
\pgfusepath{clip}%
\pgfsetbuttcap%
\pgfsetmiterjoin%
\definecolor{currentfill}{rgb}{0.998078,0.999231,0.746021}%
\pgfsetfillcolor{currentfill}%
\pgfsetlinewidth{0.000000pt}%
\definecolor{currentstroke}{rgb}{0.000000,0.000000,0.000000}%
\pgfsetstrokecolor{currentstroke}%
\pgfsetstrokeopacity{0.000000}%
\pgfsetdash{}{0pt}%
\pgfpathmoveto{\pgfqpoint{3.521740in}{0.839157in}}%
\pgfpathlineto{\pgfqpoint{3.485872in}{0.839157in}}%
\pgfpathlineto{\pgfqpoint{3.485872in}{0.860086in}}%
\pgfpathlineto{\pgfqpoint{3.521740in}{0.860086in}}%
\pgfpathlineto{\pgfqpoint{3.521740in}{0.839157in}}%
\pgfpathclose%
\pgfusepath{fill}%
\end{pgfscope}%
\begin{pgfscope}%
\pgfpathrectangle{\pgfqpoint{3.019583in}{0.169444in}}{\pgfqpoint{0.896708in}{1.339426in}}%
\pgfusepath{clip}%
\pgfsetbuttcap%
\pgfsetmiterjoin%
\definecolor{currentfill}{rgb}{0.982699,0.993080,0.722030}%
\pgfsetfillcolor{currentfill}%
\pgfsetlinewidth{0.000000pt}%
\definecolor{currentstroke}{rgb}{0.000000,0.000000,0.000000}%
\pgfsetstrokecolor{currentstroke}%
\pgfsetstrokeopacity{0.000000}%
\pgfsetdash{}{0pt}%
\pgfpathmoveto{\pgfqpoint{3.521740in}{0.860086in}}%
\pgfpathlineto{\pgfqpoint{3.485872in}{0.860086in}}%
\pgfpathlineto{\pgfqpoint{3.485872in}{0.881015in}}%
\pgfpathlineto{\pgfqpoint{3.521740in}{0.881015in}}%
\pgfpathlineto{\pgfqpoint{3.521740in}{0.860086in}}%
\pgfpathclose%
\pgfusepath{fill}%
\end{pgfscope}%
\begin{pgfscope}%
\pgfpathrectangle{\pgfqpoint{3.019583in}{0.169444in}}{\pgfqpoint{0.896708in}{1.339426in}}%
\pgfusepath{clip}%
\pgfsetbuttcap%
\pgfsetmiterjoin%
\definecolor{currentfill}{rgb}{0.967320,0.986928,0.698039}%
\pgfsetfillcolor{currentfill}%
\pgfsetlinewidth{0.000000pt}%
\definecolor{currentstroke}{rgb}{0.000000,0.000000,0.000000}%
\pgfsetstrokecolor{currentstroke}%
\pgfsetstrokeopacity{0.000000}%
\pgfsetdash{}{0pt}%
\pgfpathmoveto{\pgfqpoint{3.521740in}{0.881015in}}%
\pgfpathlineto{\pgfqpoint{3.485872in}{0.881015in}}%
\pgfpathlineto{\pgfqpoint{3.485872in}{0.901943in}}%
\pgfpathlineto{\pgfqpoint{3.521740in}{0.901943in}}%
\pgfpathlineto{\pgfqpoint{3.521740in}{0.881015in}}%
\pgfpathclose%
\pgfusepath{fill}%
\end{pgfscope}%
\begin{pgfscope}%
\pgfpathrectangle{\pgfqpoint{3.019583in}{0.169444in}}{\pgfqpoint{0.896708in}{1.339426in}}%
\pgfusepath{clip}%
\pgfsetbuttcap%
\pgfsetmiterjoin%
\definecolor{currentfill}{rgb}{0.951942,0.980777,0.674048}%
\pgfsetfillcolor{currentfill}%
\pgfsetlinewidth{0.000000pt}%
\definecolor{currentstroke}{rgb}{0.000000,0.000000,0.000000}%
\pgfsetstrokecolor{currentstroke}%
\pgfsetstrokeopacity{0.000000}%
\pgfsetdash{}{0pt}%
\pgfpathmoveto{\pgfqpoint{3.521740in}{0.901943in}}%
\pgfpathlineto{\pgfqpoint{3.485872in}{0.901943in}}%
\pgfpathlineto{\pgfqpoint{3.485872in}{0.922872in}}%
\pgfpathlineto{\pgfqpoint{3.521740in}{0.922872in}}%
\pgfpathlineto{\pgfqpoint{3.521740in}{0.901943in}}%
\pgfpathclose%
\pgfusepath{fill}%
\end{pgfscope}%
\begin{pgfscope}%
\pgfpathrectangle{\pgfqpoint{3.019583in}{0.169444in}}{\pgfqpoint{0.896708in}{1.339426in}}%
\pgfusepath{clip}%
\pgfsetbuttcap%
\pgfsetmiterjoin%
\definecolor{currentfill}{rgb}{0.936563,0.974625,0.650058}%
\pgfsetfillcolor{currentfill}%
\pgfsetlinewidth{0.000000pt}%
\definecolor{currentstroke}{rgb}{0.000000,0.000000,0.000000}%
\pgfsetstrokecolor{currentstroke}%
\pgfsetstrokeopacity{0.000000}%
\pgfsetdash{}{0pt}%
\pgfpathmoveto{\pgfqpoint{3.521740in}{0.922872in}}%
\pgfpathlineto{\pgfqpoint{3.485872in}{0.922872in}}%
\pgfpathlineto{\pgfqpoint{3.485872in}{0.943800in}}%
\pgfpathlineto{\pgfqpoint{3.521740in}{0.943800in}}%
\pgfpathlineto{\pgfqpoint{3.521740in}{0.922872in}}%
\pgfpathclose%
\pgfusepath{fill}%
\end{pgfscope}%
\begin{pgfscope}%
\pgfpathrectangle{\pgfqpoint{3.019583in}{0.169444in}}{\pgfqpoint{0.896708in}{1.339426in}}%
\pgfusepath{clip}%
\pgfsetbuttcap%
\pgfsetmiterjoin%
\definecolor{currentfill}{rgb}{0.921184,0.968474,0.626067}%
\pgfsetfillcolor{currentfill}%
\pgfsetlinewidth{0.000000pt}%
\definecolor{currentstroke}{rgb}{0.000000,0.000000,0.000000}%
\pgfsetstrokecolor{currentstroke}%
\pgfsetstrokeopacity{0.000000}%
\pgfsetdash{}{0pt}%
\pgfpathmoveto{\pgfqpoint{3.521740in}{0.943800in}}%
\pgfpathlineto{\pgfqpoint{3.485872in}{0.943800in}}%
\pgfpathlineto{\pgfqpoint{3.485872in}{0.964729in}}%
\pgfpathlineto{\pgfqpoint{3.521740in}{0.964729in}}%
\pgfpathlineto{\pgfqpoint{3.521740in}{0.943800in}}%
\pgfpathclose%
\pgfusepath{fill}%
\end{pgfscope}%
\begin{pgfscope}%
\pgfpathrectangle{\pgfqpoint{3.019583in}{0.169444in}}{\pgfqpoint{0.896708in}{1.339426in}}%
\pgfusepath{clip}%
\pgfsetbuttcap%
\pgfsetmiterjoin%
\definecolor{currentfill}{rgb}{0.905805,0.962322,0.602076}%
\pgfsetfillcolor{currentfill}%
\pgfsetlinewidth{0.000000pt}%
\definecolor{currentstroke}{rgb}{0.000000,0.000000,0.000000}%
\pgfsetstrokecolor{currentstroke}%
\pgfsetstrokeopacity{0.000000}%
\pgfsetdash{}{0pt}%
\pgfpathmoveto{\pgfqpoint{3.521740in}{0.964729in}}%
\pgfpathlineto{\pgfqpoint{3.485872in}{0.964729in}}%
\pgfpathlineto{\pgfqpoint{3.485872in}{0.985657in}}%
\pgfpathlineto{\pgfqpoint{3.521740in}{0.985657in}}%
\pgfpathlineto{\pgfqpoint{3.521740in}{0.964729in}}%
\pgfpathclose%
\pgfusepath{fill}%
\end{pgfscope}%
\begin{pgfscope}%
\pgfpathrectangle{\pgfqpoint{3.019583in}{0.169444in}}{\pgfqpoint{0.896708in}{1.339426in}}%
\pgfusepath{clip}%
\pgfsetbuttcap%
\pgfsetmiterjoin%
\definecolor{currentfill}{rgb}{0.874740,0.949712,0.601615}%
\pgfsetfillcolor{currentfill}%
\pgfsetlinewidth{0.000000pt}%
\definecolor{currentstroke}{rgb}{0.000000,0.000000,0.000000}%
\pgfsetstrokecolor{currentstroke}%
\pgfsetstrokeopacity{0.000000}%
\pgfsetdash{}{0pt}%
\pgfpathmoveto{\pgfqpoint{3.521740in}{0.985657in}}%
\pgfpathlineto{\pgfqpoint{3.485872in}{0.985657in}}%
\pgfpathlineto{\pgfqpoint{3.485872in}{1.006586in}}%
\pgfpathlineto{\pgfqpoint{3.521740in}{1.006586in}}%
\pgfpathlineto{\pgfqpoint{3.521740in}{0.985657in}}%
\pgfpathclose%
\pgfusepath{fill}%
\end{pgfscope}%
\begin{pgfscope}%
\pgfpathrectangle{\pgfqpoint{3.019583in}{0.169444in}}{\pgfqpoint{0.896708in}{1.339426in}}%
\pgfusepath{clip}%
\pgfsetbuttcap%
\pgfsetmiterjoin%
\definecolor{currentfill}{rgb}{0.838447,0.934948,0.608997}%
\pgfsetfillcolor{currentfill}%
\pgfsetlinewidth{0.000000pt}%
\definecolor{currentstroke}{rgb}{0.000000,0.000000,0.000000}%
\pgfsetstrokecolor{currentstroke}%
\pgfsetstrokeopacity{0.000000}%
\pgfsetdash{}{0pt}%
\pgfpathmoveto{\pgfqpoint{3.521740in}{1.006586in}}%
\pgfpathlineto{\pgfqpoint{3.485872in}{1.006586in}}%
\pgfpathlineto{\pgfqpoint{3.485872in}{1.027514in}}%
\pgfpathlineto{\pgfqpoint{3.521740in}{1.027514in}}%
\pgfpathlineto{\pgfqpoint{3.521740in}{1.006586in}}%
\pgfpathclose%
\pgfusepath{fill}%
\end{pgfscope}%
\begin{pgfscope}%
\pgfpathrectangle{\pgfqpoint{3.019583in}{0.169444in}}{\pgfqpoint{0.896708in}{1.339426in}}%
\pgfusepath{clip}%
\pgfsetbuttcap%
\pgfsetmiterjoin%
\definecolor{currentfill}{rgb}{0.802153,0.920185,0.616378}%
\pgfsetfillcolor{currentfill}%
\pgfsetlinewidth{0.000000pt}%
\definecolor{currentstroke}{rgb}{0.000000,0.000000,0.000000}%
\pgfsetstrokecolor{currentstroke}%
\pgfsetstrokeopacity{0.000000}%
\pgfsetdash{}{0pt}%
\pgfpathmoveto{\pgfqpoint{3.521740in}{1.027514in}}%
\pgfpathlineto{\pgfqpoint{3.485872in}{1.027514in}}%
\pgfpathlineto{\pgfqpoint{3.485872in}{1.048443in}}%
\pgfpathlineto{\pgfqpoint{3.521740in}{1.048443in}}%
\pgfpathlineto{\pgfqpoint{3.521740in}{1.027514in}}%
\pgfpathclose%
\pgfusepath{fill}%
\end{pgfscope}%
\begin{pgfscope}%
\pgfpathrectangle{\pgfqpoint{3.019583in}{0.169444in}}{\pgfqpoint{0.896708in}{1.339426in}}%
\pgfusepath{clip}%
\pgfsetbuttcap%
\pgfsetmiterjoin%
\definecolor{currentfill}{rgb}{0.765859,0.905421,0.623760}%
\pgfsetfillcolor{currentfill}%
\pgfsetlinewidth{0.000000pt}%
\definecolor{currentstroke}{rgb}{0.000000,0.000000,0.000000}%
\pgfsetstrokecolor{currentstroke}%
\pgfsetstrokeopacity{0.000000}%
\pgfsetdash{}{0pt}%
\pgfpathmoveto{\pgfqpoint{3.521740in}{1.048443in}}%
\pgfpathlineto{\pgfqpoint{3.485872in}{1.048443in}}%
\pgfpathlineto{\pgfqpoint{3.485872in}{1.069371in}}%
\pgfpathlineto{\pgfqpoint{3.521740in}{1.069371in}}%
\pgfpathlineto{\pgfqpoint{3.521740in}{1.048443in}}%
\pgfpathclose%
\pgfusepath{fill}%
\end{pgfscope}%
\begin{pgfscope}%
\pgfpathrectangle{\pgfqpoint{3.019583in}{0.169444in}}{\pgfqpoint{0.896708in}{1.339426in}}%
\pgfusepath{clip}%
\pgfsetbuttcap%
\pgfsetmiterjoin%
\definecolor{currentfill}{rgb}{0.729566,0.890657,0.631142}%
\pgfsetfillcolor{currentfill}%
\pgfsetlinewidth{0.000000pt}%
\definecolor{currentstroke}{rgb}{0.000000,0.000000,0.000000}%
\pgfsetstrokecolor{currentstroke}%
\pgfsetstrokeopacity{0.000000}%
\pgfsetdash{}{0pt}%
\pgfpathmoveto{\pgfqpoint{3.521740in}{1.069371in}}%
\pgfpathlineto{\pgfqpoint{3.485872in}{1.069371in}}%
\pgfpathlineto{\pgfqpoint{3.485872in}{1.090300in}}%
\pgfpathlineto{\pgfqpoint{3.521740in}{1.090300in}}%
\pgfpathlineto{\pgfqpoint{3.521740in}{1.069371in}}%
\pgfpathclose%
\pgfusepath{fill}%
\end{pgfscope}%
\begin{pgfscope}%
\pgfpathrectangle{\pgfqpoint{3.019583in}{0.169444in}}{\pgfqpoint{0.896708in}{1.339426in}}%
\pgfusepath{clip}%
\pgfsetbuttcap%
\pgfsetmiterjoin%
\definecolor{currentfill}{rgb}{0.693272,0.875894,0.638524}%
\pgfsetfillcolor{currentfill}%
\pgfsetlinewidth{0.000000pt}%
\definecolor{currentstroke}{rgb}{0.000000,0.000000,0.000000}%
\pgfsetstrokecolor{currentstroke}%
\pgfsetstrokeopacity{0.000000}%
\pgfsetdash{}{0pt}%
\pgfpathmoveto{\pgfqpoint{3.521740in}{1.090300in}}%
\pgfpathlineto{\pgfqpoint{3.485872in}{1.090300in}}%
\pgfpathlineto{\pgfqpoint{3.485872in}{1.111228in}}%
\pgfpathlineto{\pgfqpoint{3.521740in}{1.111228in}}%
\pgfpathlineto{\pgfqpoint{3.521740in}{1.090300in}}%
\pgfpathclose%
\pgfusepath{fill}%
\end{pgfscope}%
\begin{pgfscope}%
\pgfpathrectangle{\pgfqpoint{3.019583in}{0.169444in}}{\pgfqpoint{0.896708in}{1.339426in}}%
\pgfusepath{clip}%
\pgfsetbuttcap%
\pgfsetmiterjoin%
\definecolor{currentfill}{rgb}{0.654671,0.860438,0.643368}%
\pgfsetfillcolor{currentfill}%
\pgfsetlinewidth{0.000000pt}%
\definecolor{currentstroke}{rgb}{0.000000,0.000000,0.000000}%
\pgfsetstrokecolor{currentstroke}%
\pgfsetstrokeopacity{0.000000}%
\pgfsetdash{}{0pt}%
\pgfpathmoveto{\pgfqpoint{3.521740in}{1.111228in}}%
\pgfpathlineto{\pgfqpoint{3.485872in}{1.111228in}}%
\pgfpathlineto{\pgfqpoint{3.485872in}{1.132157in}}%
\pgfpathlineto{\pgfqpoint{3.521740in}{1.132157in}}%
\pgfpathlineto{\pgfqpoint{3.521740in}{1.111228in}}%
\pgfpathclose%
\pgfusepath{fill}%
\end{pgfscope}%
\begin{pgfscope}%
\pgfpathrectangle{\pgfqpoint{3.019583in}{0.169444in}}{\pgfqpoint{0.896708in}{1.339426in}}%
\pgfusepath{clip}%
\pgfsetbuttcap%
\pgfsetmiterjoin%
\definecolor{currentfill}{rgb}{0.612226,0.843829,0.643983}%
\pgfsetfillcolor{currentfill}%
\pgfsetlinewidth{0.000000pt}%
\definecolor{currentstroke}{rgb}{0.000000,0.000000,0.000000}%
\pgfsetstrokecolor{currentstroke}%
\pgfsetstrokeopacity{0.000000}%
\pgfsetdash{}{0pt}%
\pgfpathmoveto{\pgfqpoint{3.521740in}{1.132157in}}%
\pgfpathlineto{\pgfqpoint{3.485872in}{1.132157in}}%
\pgfpathlineto{\pgfqpoint{3.485872in}{1.153085in}}%
\pgfpathlineto{\pgfqpoint{3.521740in}{1.153085in}}%
\pgfpathlineto{\pgfqpoint{3.521740in}{1.132157in}}%
\pgfpathclose%
\pgfusepath{fill}%
\end{pgfscope}%
\begin{pgfscope}%
\pgfpathrectangle{\pgfqpoint{3.019583in}{0.169444in}}{\pgfqpoint{0.896708in}{1.339426in}}%
\pgfusepath{clip}%
\pgfsetbuttcap%
\pgfsetmiterjoin%
\definecolor{currentfill}{rgb}{0.569781,0.827220,0.644598}%
\pgfsetfillcolor{currentfill}%
\pgfsetlinewidth{0.000000pt}%
\definecolor{currentstroke}{rgb}{0.000000,0.000000,0.000000}%
\pgfsetstrokecolor{currentstroke}%
\pgfsetstrokeopacity{0.000000}%
\pgfsetdash{}{0pt}%
\pgfpathmoveto{\pgfqpoint{3.521740in}{1.153085in}}%
\pgfpathlineto{\pgfqpoint{3.485872in}{1.153085in}}%
\pgfpathlineto{\pgfqpoint{3.485872in}{1.174014in}}%
\pgfpathlineto{\pgfqpoint{3.521740in}{1.174014in}}%
\pgfpathlineto{\pgfqpoint{3.521740in}{1.153085in}}%
\pgfpathclose%
\pgfusepath{fill}%
\end{pgfscope}%
\begin{pgfscope}%
\pgfpathrectangle{\pgfqpoint{3.019583in}{0.169444in}}{\pgfqpoint{0.896708in}{1.339426in}}%
\pgfusepath{clip}%
\pgfsetbuttcap%
\pgfsetmiterjoin%
\definecolor{currentfill}{rgb}{0.527336,0.810611,0.645213}%
\pgfsetfillcolor{currentfill}%
\pgfsetlinewidth{0.000000pt}%
\definecolor{currentstroke}{rgb}{0.000000,0.000000,0.000000}%
\pgfsetstrokecolor{currentstroke}%
\pgfsetstrokeopacity{0.000000}%
\pgfsetdash{}{0pt}%
\pgfpathmoveto{\pgfqpoint{3.521740in}{1.174014in}}%
\pgfpathlineto{\pgfqpoint{3.485872in}{1.174014in}}%
\pgfpathlineto{\pgfqpoint{3.485872in}{1.194943in}}%
\pgfpathlineto{\pgfqpoint{3.521740in}{1.194943in}}%
\pgfpathlineto{\pgfqpoint{3.521740in}{1.174014in}}%
\pgfpathclose%
\pgfusepath{fill}%
\end{pgfscope}%
\begin{pgfscope}%
\pgfpathrectangle{\pgfqpoint{3.019583in}{0.169444in}}{\pgfqpoint{0.896708in}{1.339426in}}%
\pgfusepath{clip}%
\pgfsetbuttcap%
\pgfsetmiterjoin%
\definecolor{currentfill}{rgb}{0.484890,0.794002,0.645829}%
\pgfsetfillcolor{currentfill}%
\pgfsetlinewidth{0.000000pt}%
\definecolor{currentstroke}{rgb}{0.000000,0.000000,0.000000}%
\pgfsetstrokecolor{currentstroke}%
\pgfsetstrokeopacity{0.000000}%
\pgfsetdash{}{0pt}%
\pgfpathmoveto{\pgfqpoint{3.521740in}{1.194943in}}%
\pgfpathlineto{\pgfqpoint{3.485872in}{1.194943in}}%
\pgfpathlineto{\pgfqpoint{3.485872in}{1.215871in}}%
\pgfpathlineto{\pgfqpoint{3.521740in}{1.215871in}}%
\pgfpathlineto{\pgfqpoint{3.521740in}{1.194943in}}%
\pgfpathclose%
\pgfusepath{fill}%
\end{pgfscope}%
\begin{pgfscope}%
\pgfpathrectangle{\pgfqpoint{3.019583in}{0.169444in}}{\pgfqpoint{0.896708in}{1.339426in}}%
\pgfusepath{clip}%
\pgfsetbuttcap%
\pgfsetmiterjoin%
\definecolor{currentfill}{rgb}{0.442445,0.777393,0.646444}%
\pgfsetfillcolor{currentfill}%
\pgfsetlinewidth{0.000000pt}%
\definecolor{currentstroke}{rgb}{0.000000,0.000000,0.000000}%
\pgfsetstrokecolor{currentstroke}%
\pgfsetstrokeopacity{0.000000}%
\pgfsetdash{}{0pt}%
\pgfpathmoveto{\pgfqpoint{3.521740in}{1.215871in}}%
\pgfpathlineto{\pgfqpoint{3.485872in}{1.215871in}}%
\pgfpathlineto{\pgfqpoint{3.485872in}{1.236800in}}%
\pgfpathlineto{\pgfqpoint{3.521740in}{1.236800in}}%
\pgfpathlineto{\pgfqpoint{3.521740in}{1.215871in}}%
\pgfpathclose%
\pgfusepath{fill}%
\end{pgfscope}%
\begin{pgfscope}%
\pgfpathrectangle{\pgfqpoint{3.019583in}{0.169444in}}{\pgfqpoint{0.896708in}{1.339426in}}%
\pgfusepath{clip}%
\pgfsetbuttcap%
\pgfsetmiterjoin%
\definecolor{currentfill}{rgb}{0.400000,0.760784,0.647059}%
\pgfsetfillcolor{currentfill}%
\pgfsetlinewidth{0.000000pt}%
\definecolor{currentstroke}{rgb}{0.000000,0.000000,0.000000}%
\pgfsetstrokecolor{currentstroke}%
\pgfsetstrokeopacity{0.000000}%
\pgfsetdash{}{0pt}%
\pgfpathmoveto{\pgfqpoint{3.521740in}{1.236800in}}%
\pgfpathlineto{\pgfqpoint{3.485872in}{1.236800in}}%
\pgfpathlineto{\pgfqpoint{3.485872in}{1.257728in}}%
\pgfpathlineto{\pgfqpoint{3.521740in}{1.257728in}}%
\pgfpathlineto{\pgfqpoint{3.521740in}{1.236800in}}%
\pgfpathclose%
\pgfusepath{fill}%
\end{pgfscope}%
\begin{pgfscope}%
\pgfpathrectangle{\pgfqpoint{3.019583in}{0.169444in}}{\pgfqpoint{0.896708in}{1.339426in}}%
\pgfusepath{clip}%
\pgfsetbuttcap%
\pgfsetmiterjoin%
\definecolor{currentfill}{rgb}{0.368012,0.725106,0.661822}%
\pgfsetfillcolor{currentfill}%
\pgfsetlinewidth{0.000000pt}%
\definecolor{currentstroke}{rgb}{0.000000,0.000000,0.000000}%
\pgfsetstrokecolor{currentstroke}%
\pgfsetstrokeopacity{0.000000}%
\pgfsetdash{}{0pt}%
\pgfpathmoveto{\pgfqpoint{3.521740in}{1.257728in}}%
\pgfpathlineto{\pgfqpoint{3.485872in}{1.257728in}}%
\pgfpathlineto{\pgfqpoint{3.485872in}{1.278657in}}%
\pgfpathlineto{\pgfqpoint{3.521740in}{1.278657in}}%
\pgfpathlineto{\pgfqpoint{3.521740in}{1.257728in}}%
\pgfpathclose%
\pgfusepath{fill}%
\end{pgfscope}%
\begin{pgfscope}%
\pgfpathrectangle{\pgfqpoint{3.019583in}{0.169444in}}{\pgfqpoint{0.896708in}{1.339426in}}%
\pgfusepath{clip}%
\pgfsetbuttcap%
\pgfsetmiterjoin%
\definecolor{currentfill}{rgb}{0.336025,0.689427,0.676586}%
\pgfsetfillcolor{currentfill}%
\pgfsetlinewidth{0.000000pt}%
\definecolor{currentstroke}{rgb}{0.000000,0.000000,0.000000}%
\pgfsetstrokecolor{currentstroke}%
\pgfsetstrokeopacity{0.000000}%
\pgfsetdash{}{0pt}%
\pgfpathmoveto{\pgfqpoint{3.521740in}{1.278657in}}%
\pgfpathlineto{\pgfqpoint{3.485872in}{1.278657in}}%
\pgfpathlineto{\pgfqpoint{3.485872in}{1.299585in}}%
\pgfpathlineto{\pgfqpoint{3.521740in}{1.299585in}}%
\pgfpathlineto{\pgfqpoint{3.521740in}{1.278657in}}%
\pgfpathclose%
\pgfusepath{fill}%
\end{pgfscope}%
\begin{pgfscope}%
\pgfpathrectangle{\pgfqpoint{3.019583in}{0.169444in}}{\pgfqpoint{0.896708in}{1.339426in}}%
\pgfusepath{clip}%
\pgfsetbuttcap%
\pgfsetmiterjoin%
\definecolor{currentfill}{rgb}{0.304037,0.653749,0.691349}%
\pgfsetfillcolor{currentfill}%
\pgfsetlinewidth{0.000000pt}%
\definecolor{currentstroke}{rgb}{0.000000,0.000000,0.000000}%
\pgfsetstrokecolor{currentstroke}%
\pgfsetstrokeopacity{0.000000}%
\pgfsetdash{}{0pt}%
\pgfpathmoveto{\pgfqpoint{3.521740in}{1.299585in}}%
\pgfpathlineto{\pgfqpoint{3.485872in}{1.299585in}}%
\pgfpathlineto{\pgfqpoint{3.485872in}{1.320514in}}%
\pgfpathlineto{\pgfqpoint{3.521740in}{1.320514in}}%
\pgfpathlineto{\pgfqpoint{3.521740in}{1.299585in}}%
\pgfpathclose%
\pgfusepath{fill}%
\end{pgfscope}%
\begin{pgfscope}%
\pgfpathrectangle{\pgfqpoint{3.019583in}{0.169444in}}{\pgfqpoint{0.896708in}{1.339426in}}%
\pgfusepath{clip}%
\pgfsetbuttcap%
\pgfsetmiterjoin%
\definecolor{currentfill}{rgb}{0.272049,0.618070,0.706113}%
\pgfsetfillcolor{currentfill}%
\pgfsetlinewidth{0.000000pt}%
\definecolor{currentstroke}{rgb}{0.000000,0.000000,0.000000}%
\pgfsetstrokecolor{currentstroke}%
\pgfsetstrokeopacity{0.000000}%
\pgfsetdash{}{0pt}%
\pgfpathmoveto{\pgfqpoint{3.521740in}{1.320514in}}%
\pgfpathlineto{\pgfqpoint{3.485872in}{1.320514in}}%
\pgfpathlineto{\pgfqpoint{3.485872in}{1.341442in}}%
\pgfpathlineto{\pgfqpoint{3.521740in}{1.341442in}}%
\pgfpathlineto{\pgfqpoint{3.521740in}{1.320514in}}%
\pgfpathclose%
\pgfusepath{fill}%
\end{pgfscope}%
\begin{pgfscope}%
\pgfpathrectangle{\pgfqpoint{3.019583in}{0.169444in}}{\pgfqpoint{0.896708in}{1.339426in}}%
\pgfusepath{clip}%
\pgfsetbuttcap%
\pgfsetmiterjoin%
\definecolor{currentfill}{rgb}{0.240062,0.582391,0.720877}%
\pgfsetfillcolor{currentfill}%
\pgfsetlinewidth{0.000000pt}%
\definecolor{currentstroke}{rgb}{0.000000,0.000000,0.000000}%
\pgfsetstrokecolor{currentstroke}%
\pgfsetstrokeopacity{0.000000}%
\pgfsetdash{}{0pt}%
\pgfpathmoveto{\pgfqpoint{3.521740in}{1.341442in}}%
\pgfpathlineto{\pgfqpoint{3.485872in}{1.341442in}}%
\pgfpathlineto{\pgfqpoint{3.485872in}{1.362371in}}%
\pgfpathlineto{\pgfqpoint{3.521740in}{1.362371in}}%
\pgfpathlineto{\pgfqpoint{3.521740in}{1.341442in}}%
\pgfpathclose%
\pgfusepath{fill}%
\end{pgfscope}%
\begin{pgfscope}%
\pgfpathrectangle{\pgfqpoint{3.019583in}{0.169444in}}{\pgfqpoint{0.896708in}{1.339426in}}%
\pgfusepath{clip}%
\pgfsetbuttcap%
\pgfsetmiterjoin%
\definecolor{currentfill}{rgb}{0.208074,0.546713,0.735640}%
\pgfsetfillcolor{currentfill}%
\pgfsetlinewidth{0.000000pt}%
\definecolor{currentstroke}{rgb}{0.000000,0.000000,0.000000}%
\pgfsetstrokecolor{currentstroke}%
\pgfsetstrokeopacity{0.000000}%
\pgfsetdash{}{0pt}%
\pgfpathmoveto{\pgfqpoint{3.521740in}{1.362371in}}%
\pgfpathlineto{\pgfqpoint{3.485872in}{1.362371in}}%
\pgfpathlineto{\pgfqpoint{3.485872in}{1.383299in}}%
\pgfpathlineto{\pgfqpoint{3.521740in}{1.383299in}}%
\pgfpathlineto{\pgfqpoint{3.521740in}{1.362371in}}%
\pgfpathclose%
\pgfusepath{fill}%
\end{pgfscope}%
\begin{pgfscope}%
\pgfpathrectangle{\pgfqpoint{3.019583in}{0.169444in}}{\pgfqpoint{0.896708in}{1.339426in}}%
\pgfusepath{clip}%
\pgfsetbuttcap%
\pgfsetmiterjoin%
\definecolor{currentfill}{rgb}{0.212995,0.511419,0.730796}%
\pgfsetfillcolor{currentfill}%
\pgfsetlinewidth{0.000000pt}%
\definecolor{currentstroke}{rgb}{0.000000,0.000000,0.000000}%
\pgfsetstrokecolor{currentstroke}%
\pgfsetstrokeopacity{0.000000}%
\pgfsetdash{}{0pt}%
\pgfpathmoveto{\pgfqpoint{3.521740in}{1.383299in}}%
\pgfpathlineto{\pgfqpoint{3.485872in}{1.383299in}}%
\pgfpathlineto{\pgfqpoint{3.485872in}{1.404228in}}%
\pgfpathlineto{\pgfqpoint{3.521740in}{1.404228in}}%
\pgfpathlineto{\pgfqpoint{3.521740in}{1.383299in}}%
\pgfpathclose%
\pgfusepath{fill}%
\end{pgfscope}%
\begin{pgfscope}%
\pgfpathrectangle{\pgfqpoint{3.019583in}{0.169444in}}{\pgfqpoint{0.896708in}{1.339426in}}%
\pgfusepath{clip}%
\pgfsetbuttcap%
\pgfsetmiterjoin%
\definecolor{currentfill}{rgb}{0.240062,0.476355,0.714187}%
\pgfsetfillcolor{currentfill}%
\pgfsetlinewidth{0.000000pt}%
\definecolor{currentstroke}{rgb}{0.000000,0.000000,0.000000}%
\pgfsetstrokecolor{currentstroke}%
\pgfsetstrokeopacity{0.000000}%
\pgfsetdash{}{0pt}%
\pgfpathmoveto{\pgfqpoint{3.521740in}{1.404228in}}%
\pgfpathlineto{\pgfqpoint{3.485872in}{1.404228in}}%
\pgfpathlineto{\pgfqpoint{3.485872in}{1.425156in}}%
\pgfpathlineto{\pgfqpoint{3.521740in}{1.425156in}}%
\pgfpathlineto{\pgfqpoint{3.521740in}{1.404228in}}%
\pgfpathclose%
\pgfusepath{fill}%
\end{pgfscope}%
\begin{pgfscope}%
\pgfpathrectangle{\pgfqpoint{3.019583in}{0.169444in}}{\pgfqpoint{0.896708in}{1.339426in}}%
\pgfusepath{clip}%
\pgfsetbuttcap%
\pgfsetmiterjoin%
\definecolor{currentfill}{rgb}{0.267128,0.441292,0.697578}%
\pgfsetfillcolor{currentfill}%
\pgfsetlinewidth{0.000000pt}%
\definecolor{currentstroke}{rgb}{0.000000,0.000000,0.000000}%
\pgfsetstrokecolor{currentstroke}%
\pgfsetstrokeopacity{0.000000}%
\pgfsetdash{}{0pt}%
\pgfpathmoveto{\pgfqpoint{3.521740in}{1.425156in}}%
\pgfpathlineto{\pgfqpoint{3.485872in}{1.425156in}}%
\pgfpathlineto{\pgfqpoint{3.485872in}{1.446085in}}%
\pgfpathlineto{\pgfqpoint{3.521740in}{1.446085in}}%
\pgfpathlineto{\pgfqpoint{3.521740in}{1.425156in}}%
\pgfpathclose%
\pgfusepath{fill}%
\end{pgfscope}%
\begin{pgfscope}%
\pgfpathrectangle{\pgfqpoint{3.019583in}{0.169444in}}{\pgfqpoint{0.896708in}{1.339426in}}%
\pgfusepath{clip}%
\pgfsetbuttcap%
\pgfsetmiterjoin%
\definecolor{currentfill}{rgb}{0.294195,0.406228,0.680969}%
\pgfsetfillcolor{currentfill}%
\pgfsetlinewidth{0.000000pt}%
\definecolor{currentstroke}{rgb}{0.000000,0.000000,0.000000}%
\pgfsetstrokecolor{currentstroke}%
\pgfsetstrokeopacity{0.000000}%
\pgfsetdash{}{0pt}%
\pgfpathmoveto{\pgfqpoint{3.521740in}{1.446085in}}%
\pgfpathlineto{\pgfqpoint{3.485872in}{1.446085in}}%
\pgfpathlineto{\pgfqpoint{3.485872in}{1.467013in}}%
\pgfpathlineto{\pgfqpoint{3.521740in}{1.467013in}}%
\pgfpathlineto{\pgfqpoint{3.521740in}{1.446085in}}%
\pgfpathclose%
\pgfusepath{fill}%
\end{pgfscope}%
\begin{pgfscope}%
\pgfpathrectangle{\pgfqpoint{3.019583in}{0.169444in}}{\pgfqpoint{0.896708in}{1.339426in}}%
\pgfusepath{clip}%
\pgfsetbuttcap%
\pgfsetmiterjoin%
\definecolor{currentfill}{rgb}{0.321261,0.371165,0.664360}%
\pgfsetfillcolor{currentfill}%
\pgfsetlinewidth{0.000000pt}%
\definecolor{currentstroke}{rgb}{0.000000,0.000000,0.000000}%
\pgfsetstrokecolor{currentstroke}%
\pgfsetstrokeopacity{0.000000}%
\pgfsetdash{}{0pt}%
\pgfpathmoveto{\pgfqpoint{3.521740in}{1.467013in}}%
\pgfpathlineto{\pgfqpoint{3.485872in}{1.467013in}}%
\pgfpathlineto{\pgfqpoint{3.485872in}{1.487942in}}%
\pgfpathlineto{\pgfqpoint{3.521740in}{1.487942in}}%
\pgfpathlineto{\pgfqpoint{3.521740in}{1.467013in}}%
\pgfpathclose%
\pgfusepath{fill}%
\end{pgfscope}%
\begin{pgfscope}%
\pgfpathrectangle{\pgfqpoint{3.019583in}{0.169444in}}{\pgfqpoint{0.896708in}{1.339426in}}%
\pgfusepath{clip}%
\pgfsetbuttcap%
\pgfsetmiterjoin%
\definecolor{currentfill}{rgb}{0.348328,0.336101,0.647751}%
\pgfsetfillcolor{currentfill}%
\pgfsetlinewidth{0.000000pt}%
\definecolor{currentstroke}{rgb}{0.000000,0.000000,0.000000}%
\pgfsetstrokecolor{currentstroke}%
\pgfsetstrokeopacity{0.000000}%
\pgfsetdash{}{0pt}%
\pgfpathmoveto{\pgfqpoint{3.521740in}{1.487942in}}%
\pgfpathlineto{\pgfqpoint{3.485872in}{1.487942in}}%
\pgfpathlineto{\pgfqpoint{3.485872in}{1.508871in}}%
\pgfpathlineto{\pgfqpoint{3.521740in}{1.508871in}}%
\pgfpathlineto{\pgfqpoint{3.521740in}{1.487942in}}%
\pgfpathclose%
\pgfusepath{fill}%
\end{pgfscope}%
\begin{pgfscope}%
\pgfpathrectangle{\pgfqpoint{3.019583in}{0.169444in}}{\pgfqpoint{0.896708in}{1.339426in}}%
\pgfusepath{clip}%
\pgfsetbuttcap%
\pgfsetmiterjoin%
\definecolor{currentfill}{rgb}{0.368627,0.309804,0.635294}%
\pgfsetfillcolor{currentfill}%
\pgfsetlinewidth{0.000000pt}%
\definecolor{currentstroke}{rgb}{0.000000,0.000000,0.000000}%
\pgfsetstrokecolor{currentstroke}%
\pgfsetstrokeopacity{0.000000}%
\pgfsetdash{}{0pt}%
\pgfpathmoveto{\pgfqpoint{3.521740in}{1.508871in}}%
\pgfpathlineto{\pgfqpoint{3.485872in}{1.508871in}}%
\pgfpathlineto{\pgfqpoint{3.485872in}{1.529799in}}%
\pgfpathlineto{\pgfqpoint{3.521740in}{1.529799in}}%
\pgfpathlineto{\pgfqpoint{3.521740in}{1.508871in}}%
\pgfpathclose%
\pgfusepath{fill}%
\end{pgfscope}%
\begin{pgfscope}%
\pgfpathrectangle{\pgfqpoint{3.019583in}{0.169444in}}{\pgfqpoint{0.896708in}{1.339426in}}%
\pgfusepath{clip}%
\pgfsetbuttcap%
\pgfsetmiterjoin%
\definecolor{currentfill}{rgb}{0.995156,0.832295,0.506344}%
\pgfsetfillcolor{currentfill}%
\pgfsetlinewidth{0.000000pt}%
\definecolor{currentstroke}{rgb}{0.000000,0.000000,0.000000}%
\pgfsetstrokecolor{currentstroke}%
\pgfsetstrokeopacity{0.000000}%
\pgfsetdash{}{0pt}%
\pgfpathmoveto{\pgfqpoint{3.566575in}{0.671729in}}%
\pgfpathlineto{\pgfqpoint{3.584510in}{0.671729in}}%
\pgfpathlineto{\pgfqpoint{3.584510in}{0.713586in}}%
\pgfpathlineto{\pgfqpoint{3.566575in}{0.713586in}}%
\pgfpathlineto{\pgfqpoint{3.566575in}{0.671729in}}%
\pgfpathclose%
\pgfusepath{fill}%
\end{pgfscope}%
\begin{pgfscope}%
\pgfpathrectangle{\pgfqpoint{3.019583in}{0.169444in}}{\pgfqpoint{0.896708in}{1.339426in}}%
\pgfusepath{clip}%
\pgfsetbuttcap%
\pgfsetmiterjoin%
\definecolor{currentfill}{rgb}{0.998847,0.964245,0.689043}%
\pgfsetfillcolor{currentfill}%
\pgfsetlinewidth{0.000000pt}%
\definecolor{currentstroke}{rgb}{0.000000,0.000000,0.000000}%
\pgfsetstrokecolor{currentstroke}%
\pgfsetstrokeopacity{0.000000}%
\pgfsetdash{}{0pt}%
\pgfpathmoveto{\pgfqpoint{3.566575in}{0.797300in}}%
\pgfpathlineto{\pgfqpoint{3.584510in}{0.797300in}}%
\pgfpathlineto{\pgfqpoint{3.584510in}{0.839157in}}%
\pgfpathlineto{\pgfqpoint{3.566575in}{0.839157in}}%
\pgfpathlineto{\pgfqpoint{3.566575in}{0.797300in}}%
\pgfpathclose%
\pgfusepath{fill}%
\end{pgfscope}%
\begin{pgfscope}%
\pgfpathrectangle{\pgfqpoint{3.019583in}{0.169444in}}{\pgfqpoint{0.896708in}{1.339426in}}%
\pgfusepath{clip}%
\pgfsetbuttcap%
\pgfsetmiterjoin%
\definecolor{currentfill}{rgb}{0.998078,0.999231,0.746021}%
\pgfsetfillcolor{currentfill}%
\pgfsetlinewidth{0.000000pt}%
\definecolor{currentstroke}{rgb}{0.000000,0.000000,0.000000}%
\pgfsetstrokecolor{currentstroke}%
\pgfsetstrokeopacity{0.000000}%
\pgfsetdash{}{0pt}%
\pgfpathmoveto{\pgfqpoint{3.566575in}{0.839157in}}%
\pgfpathlineto{\pgfqpoint{3.656246in}{0.839157in}}%
\pgfpathlineto{\pgfqpoint{3.656246in}{0.881015in}}%
\pgfpathlineto{\pgfqpoint{3.566575in}{0.881015in}}%
\pgfpathlineto{\pgfqpoint{3.566575in}{0.839157in}}%
\pgfpathclose%
\pgfusepath{fill}%
\end{pgfscope}%
\begin{pgfscope}%
\pgfpathrectangle{\pgfqpoint{3.019583in}{0.169444in}}{\pgfqpoint{0.896708in}{1.339426in}}%
\pgfusepath{clip}%
\pgfsetbuttcap%
\pgfsetmiterjoin%
\definecolor{currentfill}{rgb}{0.967320,0.986928,0.698039}%
\pgfsetfillcolor{currentfill}%
\pgfsetlinewidth{0.000000pt}%
\definecolor{currentstroke}{rgb}{0.000000,0.000000,0.000000}%
\pgfsetstrokecolor{currentstroke}%
\pgfsetstrokeopacity{0.000000}%
\pgfsetdash{}{0pt}%
\pgfpathmoveto{\pgfqpoint{3.566575in}{0.881015in}}%
\pgfpathlineto{\pgfqpoint{3.692115in}{0.881015in}}%
\pgfpathlineto{\pgfqpoint{3.692115in}{0.922872in}}%
\pgfpathlineto{\pgfqpoint{3.566575in}{0.922872in}}%
\pgfpathlineto{\pgfqpoint{3.566575in}{0.881015in}}%
\pgfpathclose%
\pgfusepath{fill}%
\end{pgfscope}%
\begin{pgfscope}%
\pgfpathrectangle{\pgfqpoint{3.019583in}{0.169444in}}{\pgfqpoint{0.896708in}{1.339426in}}%
\pgfusepath{clip}%
\pgfsetbuttcap%
\pgfsetmiterjoin%
\definecolor{currentfill}{rgb}{0.936563,0.974625,0.650058}%
\pgfsetfillcolor{currentfill}%
\pgfsetlinewidth{0.000000pt}%
\definecolor{currentstroke}{rgb}{0.000000,0.000000,0.000000}%
\pgfsetstrokecolor{currentstroke}%
\pgfsetstrokeopacity{0.000000}%
\pgfsetdash{}{0pt}%
\pgfpathmoveto{\pgfqpoint{3.566575in}{0.922872in}}%
\pgfpathlineto{\pgfqpoint{3.638312in}{0.922872in}}%
\pgfpathlineto{\pgfqpoint{3.638312in}{0.964729in}}%
\pgfpathlineto{\pgfqpoint{3.566575in}{0.964729in}}%
\pgfpathlineto{\pgfqpoint{3.566575in}{0.922872in}}%
\pgfpathclose%
\pgfusepath{fill}%
\end{pgfscope}%
\begin{pgfscope}%
\pgfpathrectangle{\pgfqpoint{3.019583in}{0.169444in}}{\pgfqpoint{0.896708in}{1.339426in}}%
\pgfusepath{clip}%
\pgfsetbuttcap%
\pgfsetmiterjoin%
\definecolor{currentfill}{rgb}{0.905805,0.962322,0.602076}%
\pgfsetfillcolor{currentfill}%
\pgfsetlinewidth{0.000000pt}%
\definecolor{currentstroke}{rgb}{0.000000,0.000000,0.000000}%
\pgfsetstrokecolor{currentstroke}%
\pgfsetstrokeopacity{0.000000}%
\pgfsetdash{}{0pt}%
\pgfpathmoveto{\pgfqpoint{3.566575in}{0.964729in}}%
\pgfpathlineto{\pgfqpoint{3.692115in}{0.964729in}}%
\pgfpathlineto{\pgfqpoint{3.692115in}{1.006586in}}%
\pgfpathlineto{\pgfqpoint{3.566575in}{1.006586in}}%
\pgfpathlineto{\pgfqpoint{3.566575in}{0.964729in}}%
\pgfpathclose%
\pgfusepath{fill}%
\end{pgfscope}%
\begin{pgfscope}%
\pgfpathrectangle{\pgfqpoint{3.019583in}{0.169444in}}{\pgfqpoint{0.896708in}{1.339426in}}%
\pgfusepath{clip}%
\pgfsetbuttcap%
\pgfsetmiterjoin%
\definecolor{currentfill}{rgb}{0.838447,0.934948,0.608997}%
\pgfsetfillcolor{currentfill}%
\pgfsetlinewidth{0.000000pt}%
\definecolor{currentstroke}{rgb}{0.000000,0.000000,0.000000}%
\pgfsetstrokecolor{currentstroke}%
\pgfsetstrokeopacity{0.000000}%
\pgfsetdash{}{0pt}%
\pgfpathmoveto{\pgfqpoint{3.566575in}{1.006586in}}%
\pgfpathlineto{\pgfqpoint{3.674180in}{1.006586in}}%
\pgfpathlineto{\pgfqpoint{3.674180in}{1.048443in}}%
\pgfpathlineto{\pgfqpoint{3.566575in}{1.048443in}}%
\pgfpathlineto{\pgfqpoint{3.566575in}{1.006586in}}%
\pgfpathclose%
\pgfusepath{fill}%
\end{pgfscope}%
\begin{pgfscope}%
\pgfpathrectangle{\pgfqpoint{3.019583in}{0.169444in}}{\pgfqpoint{0.896708in}{1.339426in}}%
\pgfusepath{clip}%
\pgfsetbuttcap%
\pgfsetmiterjoin%
\definecolor{currentfill}{rgb}{0.765859,0.905421,0.623760}%
\pgfsetfillcolor{currentfill}%
\pgfsetlinewidth{0.000000pt}%
\definecolor{currentstroke}{rgb}{0.000000,0.000000,0.000000}%
\pgfsetstrokecolor{currentstroke}%
\pgfsetstrokeopacity{0.000000}%
\pgfsetdash{}{0pt}%
\pgfpathmoveto{\pgfqpoint{3.566575in}{1.048443in}}%
\pgfpathlineto{\pgfqpoint{3.638312in}{1.048443in}}%
\pgfpathlineto{\pgfqpoint{3.638312in}{1.090300in}}%
\pgfpathlineto{\pgfqpoint{3.566575in}{1.090300in}}%
\pgfpathlineto{\pgfqpoint{3.566575in}{1.048443in}}%
\pgfpathclose%
\pgfusepath{fill}%
\end{pgfscope}%
\begin{pgfscope}%
\pgfpathrectangle{\pgfqpoint{3.019583in}{0.169444in}}{\pgfqpoint{0.896708in}{1.339426in}}%
\pgfusepath{clip}%
\pgfsetbuttcap%
\pgfsetmiterjoin%
\definecolor{currentfill}{rgb}{0.693272,0.875894,0.638524}%
\pgfsetfillcolor{currentfill}%
\pgfsetlinewidth{0.000000pt}%
\definecolor{currentstroke}{rgb}{0.000000,0.000000,0.000000}%
\pgfsetstrokecolor{currentstroke}%
\pgfsetstrokeopacity{0.000000}%
\pgfsetdash{}{0pt}%
\pgfpathmoveto{\pgfqpoint{3.566575in}{1.090300in}}%
\pgfpathlineto{\pgfqpoint{3.638312in}{1.090300in}}%
\pgfpathlineto{\pgfqpoint{3.638312in}{1.132157in}}%
\pgfpathlineto{\pgfqpoint{3.566575in}{1.132157in}}%
\pgfpathlineto{\pgfqpoint{3.566575in}{1.090300in}}%
\pgfpathclose%
\pgfusepath{fill}%
\end{pgfscope}%
\begin{pgfscope}%
\pgfpathrectangle{\pgfqpoint{3.019583in}{0.169444in}}{\pgfqpoint{0.896708in}{1.339426in}}%
\pgfusepath{clip}%
\pgfsetbuttcap%
\pgfsetmiterjoin%
\definecolor{currentfill}{rgb}{0.612226,0.843829,0.643983}%
\pgfsetfillcolor{currentfill}%
\pgfsetlinewidth{0.000000pt}%
\definecolor{currentstroke}{rgb}{0.000000,0.000000,0.000000}%
\pgfsetstrokecolor{currentstroke}%
\pgfsetstrokeopacity{0.000000}%
\pgfsetdash{}{0pt}%
\pgfpathmoveto{\pgfqpoint{3.566575in}{1.132157in}}%
\pgfpathlineto{\pgfqpoint{3.584510in}{1.132157in}}%
\pgfpathlineto{\pgfqpoint{3.584510in}{1.174014in}}%
\pgfpathlineto{\pgfqpoint{3.566575in}{1.174014in}}%
\pgfpathlineto{\pgfqpoint{3.566575in}{1.132157in}}%
\pgfpathclose%
\pgfusepath{fill}%
\end{pgfscope}%
\begin{pgfscope}%
\pgfpathrectangle{\pgfqpoint{3.019583in}{0.169444in}}{\pgfqpoint{0.896708in}{1.339426in}}%
\pgfusepath{clip}%
\pgfsetbuttcap%
\pgfsetmiterjoin%
\definecolor{currentfill}{rgb}{0.442445,0.777393,0.646444}%
\pgfsetfillcolor{currentfill}%
\pgfsetlinewidth{0.000000pt}%
\definecolor{currentstroke}{rgb}{0.000000,0.000000,0.000000}%
\pgfsetstrokecolor{currentstroke}%
\pgfsetstrokeopacity{0.000000}%
\pgfsetdash{}{0pt}%
\pgfpathmoveto{\pgfqpoint{3.566575in}{1.215871in}}%
\pgfpathlineto{\pgfqpoint{3.602444in}{1.215871in}}%
\pgfpathlineto{\pgfqpoint{3.602444in}{1.257728in}}%
\pgfpathlineto{\pgfqpoint{3.566575in}{1.257728in}}%
\pgfpathlineto{\pgfqpoint{3.566575in}{1.215871in}}%
\pgfpathclose%
\pgfusepath{fill}%
\end{pgfscope}%
\begin{pgfscope}%
\pgfpathrectangle{\pgfqpoint{3.019583in}{0.169444in}}{\pgfqpoint{0.896708in}{1.339426in}}%
\pgfusepath{clip}%
\pgfsetbuttcap%
\pgfsetmiterjoin%
\definecolor{currentfill}{rgb}{0.368012,0.725106,0.661822}%
\pgfsetfillcolor{currentfill}%
\pgfsetlinewidth{0.000000pt}%
\definecolor{currentstroke}{rgb}{0.000000,0.000000,0.000000}%
\pgfsetstrokecolor{currentstroke}%
\pgfsetstrokeopacity{0.000000}%
\pgfsetdash{}{0pt}%
\pgfpathmoveto{\pgfqpoint{3.566575in}{1.257728in}}%
\pgfpathlineto{\pgfqpoint{3.638312in}{1.257728in}}%
\pgfpathlineto{\pgfqpoint{3.638312in}{1.299585in}}%
\pgfpathlineto{\pgfqpoint{3.566575in}{1.299585in}}%
\pgfpathlineto{\pgfqpoint{3.566575in}{1.257728in}}%
\pgfpathclose%
\pgfusepath{fill}%
\end{pgfscope}%
\begin{pgfscope}%
\pgfpathrectangle{\pgfqpoint{3.019583in}{0.169444in}}{\pgfqpoint{0.896708in}{1.339426in}}%
\pgfusepath{clip}%
\pgfsetbuttcap%
\pgfsetmiterjoin%
\definecolor{currentfill}{rgb}{0.304037,0.653749,0.691349}%
\pgfsetfillcolor{currentfill}%
\pgfsetlinewidth{0.000000pt}%
\definecolor{currentstroke}{rgb}{0.000000,0.000000,0.000000}%
\pgfsetstrokecolor{currentstroke}%
\pgfsetstrokeopacity{0.000000}%
\pgfsetdash{}{0pt}%
\pgfpathmoveto{\pgfqpoint{3.566575in}{1.299585in}}%
\pgfpathlineto{\pgfqpoint{3.602444in}{1.299585in}}%
\pgfpathlineto{\pgfqpoint{3.602444in}{1.341442in}}%
\pgfpathlineto{\pgfqpoint{3.566575in}{1.341442in}}%
\pgfpathlineto{\pgfqpoint{3.566575in}{1.299585in}}%
\pgfpathclose%
\pgfusepath{fill}%
\end{pgfscope}%
\begin{pgfscope}%
\pgfpathrectangle{\pgfqpoint{3.019583in}{0.169444in}}{\pgfqpoint{0.896708in}{1.339426in}}%
\pgfusepath{clip}%
\pgfsetbuttcap%
\pgfsetmiterjoin%
\definecolor{currentfill}{rgb}{0.240062,0.582391,0.720877}%
\pgfsetfillcolor{currentfill}%
\pgfsetlinewidth{0.000000pt}%
\definecolor{currentstroke}{rgb}{0.000000,0.000000,0.000000}%
\pgfsetstrokecolor{currentstroke}%
\pgfsetstrokeopacity{0.000000}%
\pgfsetdash{}{0pt}%
\pgfpathmoveto{\pgfqpoint{3.566575in}{1.341442in}}%
\pgfpathlineto{\pgfqpoint{3.584510in}{1.341442in}}%
\pgfpathlineto{\pgfqpoint{3.584510in}{1.383299in}}%
\pgfpathlineto{\pgfqpoint{3.566575in}{1.383299in}}%
\pgfpathlineto{\pgfqpoint{3.566575in}{1.341442in}}%
\pgfpathclose%
\pgfusepath{fill}%
\end{pgfscope}%
\begin{pgfscope}%
\pgfpathrectangle{\pgfqpoint{3.019583in}{0.169444in}}{\pgfqpoint{0.896708in}{1.339426in}}%
\pgfusepath{clip}%
\pgfsetbuttcap%
\pgfsetmiterjoin%
\definecolor{currentfill}{rgb}{0.212995,0.511419,0.730796}%
\pgfsetfillcolor{currentfill}%
\pgfsetlinewidth{0.000000pt}%
\definecolor{currentstroke}{rgb}{0.000000,0.000000,0.000000}%
\pgfsetstrokecolor{currentstroke}%
\pgfsetstrokeopacity{0.000000}%
\pgfsetdash{}{0pt}%
\pgfpathmoveto{\pgfqpoint{3.566575in}{1.383299in}}%
\pgfpathlineto{\pgfqpoint{3.584510in}{1.383299in}}%
\pgfpathlineto{\pgfqpoint{3.584510in}{1.425156in}}%
\pgfpathlineto{\pgfqpoint{3.566575in}{1.425156in}}%
\pgfpathlineto{\pgfqpoint{3.566575in}{1.383299in}}%
\pgfpathclose%
\pgfusepath{fill}%
\end{pgfscope}%
\begin{pgfscope}%
\pgfpathrectangle{\pgfqpoint{3.019583in}{0.169444in}}{\pgfqpoint{0.896708in}{1.339426in}}%
\pgfusepath{clip}%
\pgfsetbuttcap%
\pgfsetmiterjoin%
\definecolor{currentfill}{rgb}{0.267128,0.441292,0.697578}%
\pgfsetfillcolor{currentfill}%
\pgfsetlinewidth{0.000000pt}%
\definecolor{currentstroke}{rgb}{0.000000,0.000000,0.000000}%
\pgfsetstrokecolor{currentstroke}%
\pgfsetstrokeopacity{0.000000}%
\pgfsetdash{}{0pt}%
\pgfpathmoveto{\pgfqpoint{3.566575in}{1.425156in}}%
\pgfpathlineto{\pgfqpoint{3.584510in}{1.425156in}}%
\pgfpathlineto{\pgfqpoint{3.584510in}{1.467013in}}%
\pgfpathlineto{\pgfqpoint{3.566575in}{1.467013in}}%
\pgfpathlineto{\pgfqpoint{3.566575in}{1.425156in}}%
\pgfpathclose%
\pgfusepath{fill}%
\end{pgfscope}%
\begin{pgfscope}%
\definecolor{textcolor}{rgb}{0.000000,0.000000,0.000000}%
\pgfsetstrokecolor{textcolor}%
\pgfsetfillcolor{textcolor}%
\pgftext[x=3.378267in,y=0.169444in,,]{\color{textcolor}\setmainfont{Lato}\rmfamily\fontsize{8.000000}{9.600000}\selectfont -8}%
\end{pgfscope}%
\begin{pgfscope}%
\definecolor{textcolor}{rgb}{0.000000,0.000000,0.000000}%
\pgfsetstrokecolor{textcolor}%
\pgfsetfillcolor{textcolor}%
\pgftext[x=3.378267in,y=0.504301in,,]{\color{textcolor}\setmainfont{Lato}\rmfamily\fontsize{8.000000}{9.600000}\selectfont -4}%
\end{pgfscope}%
\begin{pgfscope}%
\definecolor{textcolor}{rgb}{0.000000,0.000000,0.000000}%
\pgfsetstrokecolor{textcolor}%
\pgfsetfillcolor{textcolor}%
\pgftext[x=3.378267in,y=0.839157in,,]{\color{textcolor}\setmainfont{Lato}\rmfamily\fontsize{8.000000}{9.600000}\selectfont 0}%
\end{pgfscope}%
\begin{pgfscope}%
\definecolor{textcolor}{rgb}{0.000000,0.000000,0.000000}%
\pgfsetstrokecolor{textcolor}%
\pgfsetfillcolor{textcolor}%
\pgftext[x=3.378267in,y=1.174014in,,]{\color{textcolor}\setmainfont{Lato}\rmfamily\fontsize{8.000000}{9.600000}\selectfont 4}%
\end{pgfscope}%
\begin{pgfscope}%
\definecolor{textcolor}{rgb}{0.000000,0.000000,0.000000}%
\pgfsetstrokecolor{textcolor}%
\pgfsetfillcolor{textcolor}%
\pgftext[x=3.378267in,y=1.508871in,,]{\color{textcolor}\setmainfont{Lato}\rmfamily\fontsize{8.000000}{9.600000}\selectfont 8+}%
\end{pgfscope}%
\end{pgfpicture}%
\makeatother%
\endgroup%
 \\
\footnotesize{Source: Bureau of Economic Analysis}\\

\vspace{2mm}

\begin{minipage}{0.76\textwidth}

\small \input{text/gdp_state.txt} \\

\end{minipage}

\vspace{2mm}

\noindent \normalsize \textbf{Real GDP Growth by State}\\
\footnotesize{\textit{quarterly growth at seasonally adjusted annualized rate \hspace{20mm} total growth, \input{text/gdp_state_date.txt}}\\ 

\vspace{-4.5mm}
\hspace{-2mm} \noindent \rowcolors{1}{}{black!5} \setlength{\tabcolsep}{3.8pt} \color{black!90}
		{\renewcommand{\arraystretch}{1.44}
		 \begin{tabular}{p{30mm} R{7mm} R{7mm} R{7mm} R{7mm} R{7mm} p{0mm} R{8mm} R{9mm} R{10mm} }
 & 2021 Q3 & '21 Q2 & '21 Q1 & '20 Q4 & '20 Q3 & & 1-year* & 3-year & 10-year \\
\textbf{United States}  & 2.3 & 6.7 & 6.3 & 4.5 & 33.8 &  & 4.9 & 4.8 & 22.7 \\
\hspace{1mm} \textbf{Pacific}  & 2.9 & 7.9 & 10.7 & 1.7 & 33.2 &  & 5.8 & 7.8 & 37.2 \\
\hspace{3mm}  Washington  & 2.7 & 8.2 & 8.9 & 3.5 & 30.8 &  & 5.8 & 10.4 & 47.7 \\
\hspace{3mm}  California  & 2.9 & 8.1 & 11.7 & 1.2 & 33.6 &  & 5.9 & 8.2 & 38.1 \\
\hspace{3mm}  Oregon  & 3.5 & 6.0 & 7.4 & 2.5 & 35.9 &  & 4.8 & 5.0 & 29.9 \\
\hspace{3mm}  Hawaii  & 6.0 & 8.9 & 5.2 & 4.4 & 34.5 &  & 6.1 & -5.8 & 7.0 \\
\hspace{3mm}  Alaska  & -0.6 & 1.8 & -4.5 & 4.0 & 27.6 &  & 0.2 & -6.0 & -8.2 \\
\hspace{1mm} \textbf{West South Central}  & 2.6 & 5.8 & 3.4 & 5.3 & 36.0 &  & 4.2 & 4.5 & 28.4 \\
\hspace{3mm}  Texas  & 3.5 & 6.4 & 3.9 & 5.6 & 35.6 &  & 4.8 & 6.2 & 36.2 \\
\hspace{3mm}  Oklahoma  & 1.0 & 3.5 & -0.4 & 5.0 & 38.8 &  & 2.2 & -0.1 & 19.8 \\
\hspace{3mm}  Arkansas  & 0.7 & 4.2 & 7.2 & 4.8 & 38.1 &  & 4.2 & 4.5 & 12.5 \\
\hspace{3mm}  Louisiana  & -2.7 & 4.0 & 0.7 & 3.1 & 35.7 &  & 1.2 & -4.1 & -3.1 \\
\hspace{1mm} \textbf{Mountain}  & 2.0 & 6.4 & 4.5 & 3.6 & 32.8 &  & 4.1 & 6.2 & 26.3 \\
\hspace{3mm}  Utah  & 2.7 & 6.3 & 4.0 & 2.4 & 30.8 &  & 3.8 & 10.0 & 40.0 \\
\hspace{3mm}  Colorado  & 2.3 & 7.2 & 7.3 & 5.3 & 29.0 &  & 5.5 & 7.2 & 35.3 \\
\hspace{3mm}  Idaho  & -1.0 & 4.0 & 7.4 & 1.1 & 38.6 &  & 2.8 & 7.9 & 31.8 \\
\hspace{3mm}  Arizona  & 3.2 & 5.6 & 0.8 & 3.5 & 32.5 &  & 3.2 & 6.5 & 26.2 \\
\hspace{3mm}  Nevada  & 2.6 & 9.7 & 6.2 & 4.0 & 45.3 &  & 5.6 & 2.2 & 16.8 \\
\hspace{3mm}  Montana  & -0.6 & 5.5 & 11.2 & 2.0 & 30.8 &  & 4.4 & 3.7 & 16.5 \\
\hspace{3mm}  New Mexico  & -0.5 & 5.3 & 0.8 & 3.3 & 30.8 &  & 2.2 & 4.9 & 9.3 \\
\multicolumn{3}{l}{continued on next page . . .} & &  & & & & & \\
\end{tabular} } \\ \newpage

\hspace{-2mm} \noindent \rowcolors{1}{}{black!5} 
            \setlength{\tabcolsep}{3.8pt} \color{black!90}
            {\renewcommand{\arraystretch}{1.44}
             \begin{tabular}{p{30mm} R{7mm} R{7mm} R{7mm} R{7mm} 
             R{7mm} p{0mm} R{9mm} R{9mm} R{10mm} }
 & 2021 Q3 & '21 Q2 & '21 Q1 & '20 Q4 & '20 Q3 & & 1-year* & 3-year & 10-year \\
\multicolumn{3}{l}{continued from previous page . . .}  & &  & & & & & \\
\hspace{3mm}  Wyoming  & -1.5 & 2.3 & 3.7 & 1.2 & 28.9 &  & 1.4 & -3.2 & -7.0 \\
\hspace{1mm} \textbf{South Atlantic}  & 2.9 & 6.2 & 6.1 & 4.4 & 31.3 &  & 4.9 & 4.8 & 20.9 \\
\hspace{3mm}  Florida  & 3.7 & 6.7 & 7.5 & 3.4 & 32.6 &  & 5.3 & 6.5 & 29.4 \\
\hspace{3mm}  Georgia  & 3.3 & 6.0 & 6.9 & 4.1 & 30.4 &  & 5.1 & 5.6 & 29.3 \\
\hspace{3mm}  South Carolina  & 1.6 & 6.1 & 5.1 & 4.4 & 39.1 &  & 4.3 & 5.9 & 24.6 \\
\hspace{3mm}  North Carolina  & 2.4 & 6.2 & 9.5 & 5.5 & 34.8 &  & 5.9 & 5.7 & 19.4 \\
\hspace{3mm}  District of Columbia  & 3.9 & 7.2 & -1.6 & 5.5 & 18.6 &  & 3.7 & 2.5 & 13.7 \\
\hspace{3mm}  Maryland  & 1.8 & 6.1 & 8.9 & 4.2 & 27.8 &  & 5.2 & 1.5 & 12.2 \\
\hspace{3mm}  Virginia  & 2.8 & 5.8 & 1.4 & 5.2 & 29.2 &  & 3.8 & 3.6 & 11.6 \\
\hspace{3mm}  West Virginia  & -0.6 & 7.3 & 1.1 & 4.1 & 33.3 &  & 3.0 & -1.0 & 2.9 \\
\hspace{3mm}  Delaware  & 4.7 & 2.8 & -3.6 & 8.1 & 27.1 &  & 2.9 & 4.7 & 2.5 \\
\hspace{1mm} \textbf{West North Central}  & 0.8 & 6.2 & 5.8 & 6.7 & 35.4 &  & 4.8 & 4.3 & 17.5 \\
\hspace{3mm}  North Dakota  & -3.3 & 5.5 & 10.1 & 4.6 & 26.9 &  & 4.1 & 2.2 & 39.9 \\
\hspace{3mm}  Nebraska  & 0.8 & 2.9 & 6.4 & 11.8 & 37.1 &  & 5.4 & 7.5 & 22.4 \\
\hspace{3mm}  Iowa  & 0.5 & 7.7 & 8.6 & 9.3 & 38.1 &  & 6.5 & 5.9 & 21.1 \\
\hspace{3mm}  Kansas  & -0.3 & 6.1 & 3.5 & 6.2 & 39.2 &  & 3.8 & 4.0 & 17.4 \\
\hspace{3mm}  Minnesota  & 1.8 & 6.5 & 5.3 & 5.6 & 33.9 &  & 4.8 & 2.3 & 16.9 \\
\hspace{3mm}  South Dakota  & -0.8 & 4.2 & 6.3 & 5.9 & 33.1 &  & 3.9 & 4.9 & 14.4 \\
\hspace{3mm}  Missouri  & 1.6 & 6.7 & 4.7 & 5.3 & 34.7 &  & 4.5 & 4.7 & 11.3 \\
\hspace{1mm} \textbf{Middle Atlantic}  & 2.5 & 7.2 & 6.2 & 6.1 & 31.3 &  & 5.5 & 3.7 & 16.2 \\
\hspace{3mm}  New York  & 2.2 & 8.1 & 7.1 & 6.8 & 28.9 &  & 6.0 & 4.2 & 19.2 \\
\hspace{3mm}  Pennsylvania  & 2.2 & 5.9 & 3.0 & 5.2 & 34.8 &  & 4.1 & 2.2 & 13.0 \\
\hspace{3mm}  New Jersey  & 3.7 & 6.4 & 7.9 & 5.1 & 33.3 &  & 5.8 & 4.2 & 12.6 \\
\hspace{1mm} \textbf{East South Central}  & 1.8 & 5.3 & 7.3 & 5.6 & 43.8 &  & 5.0 & 4.3 & 15.4 \\
\hspace{3mm}  Tennessee  & 3.1 & 5.6 & 13.4 & 7.6 & 51.2 &  & 7.4 & 6.2 & 24.5 \\
\hspace{3mm}  Kentucky  & 1.1 & 6.5 & 5.5 & 4.7 & 40.8 &  & 4.4 & 4.3 & 12.4 \\
\hspace{3mm}  Alabama  & 1.3 & 4.1 & 2.4 & 3.5 & 36.0 &  & 2.8 & 1.8 & 9.6 \\
\hspace{3mm}  Mississippi  & 0.2 & 4.5 & 2.0 & 5.0 & 43.1 &  & 2.9 & 2.6 & 6.0 \\
\hspace{1mm} \textbf{New England}  & 2.6 & 7.1 & 3.8 & 3.5 & 35.2 &  & 4.2 & 3.2 & 13.5 \\
\hspace{3mm}  Massachusetts  & 3.7 & 8.0 & 6.1 & 1.4 & 35.2 &  & 4.8 & 5.1 & 20.5 \\
\hspace{3mm}  New Hampshire  & -3.3 & 5.6 & 2.5 & 5.3 & 40.0 &  & 2.5 & 3.3 & 15.6 \\
\hspace{3mm}  Maine  & 1.7 & 5.5 & 1.3 & 3.8 & 37.2 &  & 3.1 & 5.6 & 14.4 \\
\hspace{3mm}  Rhode Island  & 2.2 & 7.5 & -2.9 & 6.0 & 31.6 &  & 3.1 & 1.7 & 3.3 \\
\hspace{3mm}  Vermont  & 0.4 & 6.5 & 0.9 & 3.6 & 43.0 &  & 2.8 & 0.7 & 3.3 \\
\hspace{3mm}  Connecticut  & 2.7 & 5.9 & 1.8 & 6.8 & 33.0 &  & 4.3 & -0.8 & 3.3 \\
\hspace{1mm} \textbf{East North Central}  & 0.8 & 6.4 & 5.2 & 5.1 & 36.9 &  & 4.4 & 2.1 & 13.3 \\
\hspace{3mm}  Indiana  & 0.2 & 6.1 & 9.4 & 5.0 & 43.1 &  & 5.2 & 4.5 & 17.9 \\
\hspace{3mm}  Ohio  & 0.9 & 5.2 & 3.5 & 5.5 & 35.7 &  & 3.8 & 3.4 & 14.9 \\
\hspace{3mm}  Michigan  & -0.3 & 8.3 & 2.8 & 5.0 & 39.8 &  & 3.9 & 0.2 & 14.2 \\
\hspace{3mm}  Wisconsin  & -0.2 & 5.7 & 1.2 & 3.6 & 38.8 &  & 2.6 & 1.4 & 11.5 \\
\hspace{3mm}  Illinois  & 1.9 & 6.8 & 7.7 & 5.6 & 32.9 &  & 5.5 & 1.5 & 10.3 \\
 \hline
		\end{tabular}
		}	\\

\vspace{-3mm}	
\footnotesize{Source: Bureau of Economic Analysis}


\begin{minipage}{0.76\textwidth}

\section*{\color{darkgray}\LARGE Financial Accounts}

\small A high-level overview of US financial activities can be provided by dividing the world economy into three sectors: the US private sector (see\cbox{green!70!black}), the US government (see\cbox{yellow!70!orange}), and the rest of the world (see\cbox{blue!90!black}), then examining the net lending and borrowing between the groups, which must sum to zero at an aggregate level. That is, if one sector is running a deficit, another sector must be running a surplus.\\

\vspace{2mm}

\noindent \normalsize \textbf{Sectoral Financial Balance}\\
\footnotesize{\textit{net lending (+) or borrowing (-), NIPA basis, by sector, as share of GDP}}\\
\noindent \hspace*{-3mm} \begin{tikzpicture}
	\begin{axis}[\bbar{y}{0}, \dateaxisticks ytick={-10, 0, 10},
		xticklabel={\year}, yticklabel style={text width=1.5em}, clip=false, 
		legend style={at={(0.95, 1.13)}}]
	\rbars
	\sbar{green!70!black}{date}{PRIV}{data/sectbal.csv}
	\sbar{yellow!70!orange}{date}{GOV}{data/sectbal.csv}
	\sbar{blue!90!black}{date}{ROW}{data/sectbal.csv}
    \node[align=left] (source) at (axis cs:2017-06-15,10.5) {%
      	\tiny TCJA repatriation};
    \node (destination) at (axis cs:2017-11-01,3.1) {};
    \draw[-] (source)--(destination);
	\legend{Private, Government, Rest of World};
	\end{axis}
\end{tikzpicture}\\
\footnotesize{Source: Bureau of Economic Analysis}

\vspace{4mm}

\small \input{text/sectbal.txt}. 

\vspace{2mm}

\subsection*{\color{black!70}Wealth}

\small \textbf{Total US wealth} is the tangible assets of all non-corporate sectors of the US, plus the market value of domestic corporate equities, less US financial obligations to the rest of the world. \input{text/wealthgdp.txt}\\

\vspace{2mm}


\noindent \normalsize \textbf{Total US Wealth to GDP Ratio}\\
\footnotesize{\textit{total US wealth divided by GDP}}\\
\noindent \hspace*{-3mm} \begin{tikzpicture}
	\begin{axis}[\bbar{y}{0}, \dateaxisticks ytick={0, 1, 2, 3, 4, 5}, ymin=-0.2,
		xticklabel={\year}, yticklabel style={text width=1.5em}, clip=false, 
		legend style={at={(0.95, 1.13)}}]
	\rbars
	\sbar{cyan!35!white}{date}{OTHER}{data/wealthgdp.csv}
	\sbar{green!80!blue}{date}{HNOREMV}{data/wealthgdp.csv}
	\sbar{magenta!50!violet}{date}{CORP}{data/wealthgdp.csv}
	\legend{Other, Residential Real Estate, Corporate Equities};
	\end{axis}
\end{tikzpicture}\\
\footnotesize{Source: Federal Reserve}


\end{minipage}

\newpage
\section*{\color{darkgray}\LARGE Households}

\begin{minipage}{0.76\textwidth}

\small Households are housing units occupied by people. This section covers the household sector of the economy loosely defined, and touches on demographics, personal income and outlays, residential fixed investment, household balance sheets, home ownership, housing prices, and housing construction and permitting.\\

\vspace{2mm}


\noindent \normalsize \textbf{Household Formation by Type}\\
\footnotesize{\textit{one-year moving average of annual growth rates}}\\
\noindent \hspace*{-2mm} \begin{tikzpicture}
	\begin{axis}[\bbar{y}{0}, \dateaxisticks ytick={0, 1, 2},
		xticklabel={\year}, clip=false, 
		legend style={at={(0.95, 1.13)}}]
	\rbars
	\sbar{magenta!90!blue}{date}{Renter}{data/hhform.csv}
	\sbar{yellow!60!orange}{date}{Owner}{data/hhform.csv}
	\stdline{black}{date}{pop}{data/hhform.csv}
	\legend{Rented, Owned, Population Growth};
	\end{axis}
\end{tikzpicture}\\
\footnotesize{Source: Census Bureau}
\vspace{6mm}

\normalsize

Overview\\

Demographics (population, ages, marriage, children, etc.)\\

Personal Income \\

PCE \\

Residential fixed investment \\

Household balance sheets \\

Housing prices \\

Housing permits \\


\end{minipage}

\newpage

\section*{\color{darkgray}\LARGE Poverty}

\noindent \normalsize \textbf{Share of local population in bottom third of housing-adjusted income, 2017}\\
\footnotesize{\textit{Share of commuting zone householders with after-housing-expense annual income below \$13,060}}

\vspace{-3mm}
\hspace{-15mm} \input{/home/brian/Documents/ACS/acs_map.pgf}

\vspace{-5mm}
\footnotesize{Source: American Community Survey}

\newpage


\end{document}