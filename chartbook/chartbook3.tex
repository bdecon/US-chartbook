% % % % % % % % % % % % % % 
%
%	U.S. Chartbook
%	Brian W. Dew (brianwdew@gmail.com)
%	Updated: December 15, 2019
%	GitHub repo contains to do list (issues)
%   https://github.com/bdecon/US-chartbook
%
% % % % % % % % % % % % % %
\PassOptionsToPackage{table}{xcolor}
\documentclass{report}

%
% % % % % % Packages % % % % % % % % % 
%
	
	\usepackage[letterpaper, margin=1.18in]{geometry}
	\usepackage{microtype}
	\usepackage[default]{lato}
	\usepackage{pgfplots, pgfplotstable}
	\usepackage[eulergreek]{sansmath}
	\usepackage{xcolor}
	\usepackage{array}
	\usepackage{fontawesome5}
	\usepackage{titlesec}
	\usepackage{imakeidx}
	\usepackage{fancyhdr}
	\usepackage[colorlinks, linkcolor=blue, filecolor=blue, 
		citecolor=blue, urlcolor=blue, linktoc=all, 
		pdfencoding=auto]{hyperref}
	\usetikzlibrary{pgfplots.dateplot, pgfplots.fillbetween, patterns,
	                pgfplots.groupplots, shapes.geometric}

%
% % % % % Document Settings % % % % % % % 
%

	% Paragraph spacing
	\usepackage{parskip}
	\setlength\parindent{0pt}
	\setlength{\parskip}{8pt}
	\makeatletter
		\newcommand{\@minipagerestore}{\setlength{\parskip}{8pt}}
	\makeatother
	
	% Section and Subsection Headings
	\titleformat{\section}
  		{\color{darkgray} \LARGE \seriffont \bfseries}
  		{\thesection}{1em}{}
	\titleformat{\subsection}
  		{\color{black!70} \seriffont \bfseries \large}
  		{\thesection}{1em}{}
	\titleformat{\subsubsection}
  		{\color{black!70} \seriffont \bfseries \normalsize}
  		{\thesection}{1em}{}		
%
% % % % % Graph Settings % % % % % % % 
%
	
	% Header and footer
	
	\pagestyle{fancy}
	\fancyhf{}
	\renewcommand{\headrulewidth}{0pt}
	\fancyhead[R]{\rightmark}
	\rfoot{\hyperlink{toc}{\faList}}
	\cfoot{\thepage}	
	
	
	% Index
	\indexsetup{level=\section*,noclearpage}
	\makeindex
	
	% Color square
	\newcommand{\cbox}[1]{
		\begin{tikzpicture} \draw [#1, line width=6](0,0) -- (.2,0);  
		\end{tikzpicture}}
	\newcommand{\colorline}[2]{
		\begin{tikzpicture} \draw [#1, line width=1.8](0,0.2) -- +(0.6,0) node[right, black!80] {#2}; 
		\end{tikzpicture}}
		
	% Table link
	\newcommand{\tbllink}[1]{\href{https://raw.githubusercontent.com/bdecon/US-chartbook/master/chartbook/data/#1}{\faTable}}
	
	% Last two digits of year
	\makeatletter
	\newcommand*\short[1]{\expandafter\@gobbletwo\number\numexpr#1\relax}
	\makeatother	
	
	% Column width and alignment
	\newcolumntype{R}[1]{>{\raggedleft\let\newline\\\arraybackslash\hspace{0pt}}m{#1}}	
	\newcolumntype{C}[1]{>{\centering\let\newline\\\arraybackslash\hspace{0pt}}m{#1}}
	
	% Style for date plots
	\pgfplotsset{compat=newest, 
		scaled y ticks=false,
		axis line style={black!20}, 
		xtick style={black!20}, ytick style={draw=none},
		every tick label/.style={black!50, font=\scriptsize,
			/pgf/number format/assume math mode=true},
		width=12.8cm, height=4.8cm, 
		xticklabel style={align=left}, 
		yticklabel style={text width=0.9em, align=right},       
		axis x line*=bottom, x axis line style={black!50},
	    axis y line=left, y axis line style={opacity=0},
	    ymajorgrids, grid style={very thin, black!10},	        
	    every node near coord/.style={/pgf/number format/fixed,
	    	font=\scriptsize, style={black!70}},
	    legend style={legend columns=-1, draw=none, fill=none,
	    	/tikz/every even column/.append style={column sep=0.3cm}}}
	    	
	
	% stacked diverging bar
	\newcommand{\sbar}[4]{
		\addplot[ybar stacked, bar width=2.4pt, draw opacity=0, fill=#1] 
			table [x=#2, y=#3, col sep=comma]{#4};}

	% stacked diverging bar area legend
	\newcommand{\sbaral}[4]{
		\addplot[ybar stacked, bar width=2.4pt, draw opacity=0, fill=#1, area legend] 
			table [x=#2, y=#3, col sep=comma]{#4};}
			
	% thin stacked diverging bar
	\newcommand{\tsbar}[4]{
		\addplot[ybar stacked, bar width=2.2pt, draw opacity=0, fill=#1] 
			table [x=#2, y=#3, col sep=comma]{#4};}
			
	% custom width stacked diverging bar
	\newcommand{\ctsbar}[5]{
		\addplot[ybar stacked, bar width=#5, draw opacity=0, fill=#1] 
			table [x=#2, y=#3, col sep=comma]{#4};}
			
	% area plot segment
	\newcommand{\abar}[4]{
		\addplot[stack plots=y, area style, draw=none, fill=#1] 
			table [x=#2, y=#3, col sep=comma]{#4}\closedcycle;}
					
	% text node
	\newcommand{\stdnode}[3]{\node[below, align=left, shift=({#1,#2})]{#3};}	
	
	% text node located by data	 
	\newcommand{\absnode}[3]{\node[below right, align=left] at (axis cs: #1,#2) {#3};}   
	
	% multiline text node located by data	 
	\newcommand{\absnodeml}[4]{\node[below right, align=left, text width=#4cm] 
		at (axis cs: #1,#2) {#3};}       
		        
	% Date (X) Axis Tick Marks, one tick per year, every even year labeled
	\newcommand{\dateaxisticks}{
		date coordinates in=x, axis line style={draw=none},
		xmax={2023-03-31},
		max space between ticks=40,	    
		xtick={{1990-01-01}, {1992-01-01}, {1994-01-01}, 
			{1996-01-01}, {1998-01-01}, {2000-01-01}, 
			{2002-01-01}, {2004-01-01}, {2006-01-01},
			{2008-01-01}, {2010-01-01}, {2012-01-01}, {2014-01-01},
		    {2016-01-01}, {2018-01-01}, {2020-01-01}, {2022-01-01}, 
		    {2024-01-01}, {2026-01-01}},
		minor xtick={{1989-01-01}, {1991-01-01}, {1993-01-01},
			{1995-01-01}, {1997-01-01}, {1999-01-01}, 
			{2001-01-01}, {2003-01-01}, {2005-01-01}, {2007-01-01},
		    {2009-01-01}, {2011-01-01}, {2013-01-01}, {2015-01-01},
		    {2017-01-01}, {2019-01-01}, {2021-01-01}, {2023-01-01}, 
		    {2025-01-01}, {2027-01-01}},
		enlarge y limits={0.06}, enlarge x limits={0.01},
		}
		
	% Date (X) Axis Tick Marks, one tick per year, every even year labeled
	\newcommand{\shdateaxisticks}{
		date coordinates in=x, axis line style={draw=none},
		xmax={2023-03-31},
		max space between ticks=40,	    
		xtick={{1990-01-01}, {1995-01-01}, {2000-01-01}, 
			{2005-01-01}, {2010-01-01}, {2015-01-01}, {2020-01-01}},
		minor xtick={},
		enlarge y limits={0.06}, enlarge x limits={0.01},
		}
		
	% Date (X) Axis Tick Marks, one tick per year, every even year labeled
	\newcommand{\ltdateaxisticks}{
		date coordinates in=x, axis line style={draw=none},
		xmax={2023-03-31},
		max space between ticks=40,	    
		xtick={{2013-01-01}, {2014-01-01}, {2015-01-01}, {2016-01-01}, {2017-01-01}, {2018-01-01}, 
		    {2019-01-01}, {2020-01-01}, {2021-01-01}, {2022-01-01}, {2023-01-01}},
		enlarge y limits={0.06}, enlarge x limits={0.01},
		}
		
	% Date (X) Axis Tick Marks, one tick per year, every even year labeled
	\newcommand{\lfdateaxisticks}{
		date coordinates in=x, axis line style={draw=none},
		xmin={2018-01-01}, xmax={2023-03-31},   
		xtick={{2018-01-01}, {2019-01-01}, {2020-01-01}, {2021-01-01}, {2022-01-01}, {2023-01-01}},
		enlarge y limits={value=0.12, upper}, enlarge x limits={0.02}, ymin=0,
		yticklabel style={text width=1.0em},
		height=3.8cm, width=6.4cm,
		}
		
	% Date (X) Axis Tick Marks, one tick per year, every even year labeled
	\newcommand{\tydateaxisticks}{
		date coordinates in=x, axis line style={draw=none},
		xmax={2023-03-31}, max space between ticks=40,	    
		xtick={{2011-01-01}, {2012-01-01}, {2013-01-01}, {2014-01-01}, {2015-01-01}, {2016-01-01}, 
			{2017-01-01}, {2018-01-01}, {2019-01-01}, {2020-01-01}, {2021-01-01}, {2022-01-01}, {2023-01-01}},
		enlarge y limits={0.06}, enlarge x limits={0.01},
		}
		
	% Date (X) Axis Tick Marks, very short term monthly ticks
	\newcommand{\shticks}{
		date coordinates in=x, axis line style={draw=none},
		xmax={2023-03-31},
		}
		
	% Settings for y label text in horizontal bar charts
	\newcommand{\barylab}[2]{yticklabel style={text width=#1, align=right, 
		style={black!70}, text height=#2},}
	
	% Solid bars at significant  x or y values
	\newcommand{\bbar}[2]{extra #1 ticks = {{#2}}, extra #1 tick labels = ,
		extra #1 tick style = {grid=major, grid style={thick, black!25}},}
		
	% Dashed line at significant  x or y values
	\newcommand{\dbar}[2]{extra #1 ticks = {{#2}}, extra #1 tick labels = ,
		extra #1 tick style = {grid=major, grid style={dashed, thick, black!50}},}
		
	% Standard line
	\newcommand{\stdline}[4]{\addplot[very thick, no markers, color=#1] 
		table [x=#2, y=#3, col sep=comma] {#4};	}
		
	% Thin line
	\newcommand{\thinline}[4]{\addplot[no markers, color=#1] 
		table [x=#2, y=#3, col sep=comma] {#4};	}
		
	% Dashed line
	\newcommand{\dashline}[4]{\addplot[very thick, dashed, no markers, color=#1] 
		table [x=#2, y=#3, col sep=comma] {#4};	}
		
	% Thicker line
	\newcommand{\thickline}[4]{\addplot[ultra thick, no markers, color=#1] 
		table [x=#2, y=#3, col sep=comma] {#4};	}
		
	% Style for bar plots legend symbol		
	\pgfplotsset{/pgfplots/area legend/.style={/pgfplots/legend image code/.code={
		\fill[##1] (0cm, -0.1cm) rectangle (0.6cm, 0.1cm);}},}		
		
	% Additional bar plot settings
	\newcommand{\barplotnogrid}{xbar=0pt, axis line style={draw=none},
	    yticklabel style={align=left, anchor=east},
      		xmajorticks=false, ymajorgrids=false,   
	    ytick=data, tickwidth=0pt, area legend, reverse legend,
	    nodes near coords align={horizontal},}  
		
	% Recession bars		
	\newcommand{\rbars}{
		\fill[color=black!10] (axis cs:{1990-07-01},\pgfkeysvalueof{/pgfplots/ymin}) rectangle 
			(axis cs:{1991-03-01}, \pgfkeysvalueof{/pgfplots/ymax});
		\fill[color=black!10] (axis cs:{2007-12-01},\pgfkeysvalueof{/pgfplots/ymin}) rectangle 
			(axis cs:{2009-07-01}, \pgfkeysvalueof{/pgfplots/ymax});
		\fill[color=black!10] (axis cs:{2001-03-01},\pgfkeysvalueof{/pgfplots/ymin}) rectangle 
			(axis cs:{2001-11-01}, \pgfkeysvalueof{/pgfplots/ymax});
		\fill[color=black!10] (axis cs:{2020-02-01},\pgfkeysvalueof{/pgfplots/ymin}) rectangle 
			(axis cs:{2020-05-01}, \pgfkeysvalueof{/pgfplots/ymax});}
			
	\newcommand{\rebars}{
		\fill[color=black!10] (axis cs:{2007-12-01},\pgfkeysvalueof{/pgfplots/ymin}) rectangle 
			(axis cs:{2009-07-01}, \pgfkeysvalueof{/pgfplots/ymax});
		\fill[color=black!10] (axis cs:{2001-03-01},\pgfkeysvalueof{/pgfplots/ymin}) rectangle 
			(axis cs:{2001-11-01}, \pgfkeysvalueof{/pgfplots/ymax});
		\fill[color=black!10] (axis cs:{2020-02-01},\pgfkeysvalueof{/pgfplots/ymin}) rectangle 
			(axis cs:{2020-05-01}, \pgfkeysvalueof{/pgfplots/ymax});}
			
	\newcommand{\recbars}{
		\fill[color=black!10] (axis cs:{2007-12-01},\pgfkeysvalueof{/pgfplots/ymin}) rectangle 
			(axis cs:{2009-07-01}, \pgfkeysvalueof{/pgfplots/ymax});
		\fill[color=black!10] (axis cs:{2020-02-01},\pgfkeysvalueof{/pgfplots/ymin}) rectangle 
			(axis cs:{2020-05-01}, \pgfkeysvalueof{/pgfplots/ymax});}
			
	\newcommand{\rbar}{
		\fill[color=black!10] (axis cs:{2020-02-01},\pgfkeysvalueof{/pgfplots/ymin}) rectangle 
			(axis cs:{2020-05-01}, \pgfkeysvalueof{/pgfplots/ymax});}
	
	\newfontfamily\seriffont{RobotoSlab}	
	
	\pgfplotstableread[header=true, col sep=comma]{data/cpi_comp.csv}\cpi
	\pgfplotstableread[header=true, col sep=semicolon]{data/ip_comp.csv}\ip
	\pgfplotstableread[header=true, col sep=comma]{data/rs_comp.csv}\rs
	\pgfplotstableread[header=true, col sep=comma]{data/ahe_ind.csv}\ahe
	\pgfplotstableread[header=true, col sep=comma]{data/poor.csv}\poor
	\pgfplotstableread[header=true, col sep=comma]{data/poor2.csv}\pvrt
	\pgfplotstableread[header=true, col sep=comma]{data/spmtbl21.csv}\spm
	\pgfplotstableread[header=true, col sep=semicolon]{data/occs.csv}\occ
	\pgfplotstableread[header=true, col sep=comma]{data/empgroups.csv}\emp
	\pgfplotstableread[header=true, col sep=comma]{data/empgroups2.csv}\empt
	\pgfplotstableread[header=true, col sep=comma]{data/unempgroups.csv}\unemp
	\pgfplotstableread[header=true, col sep=comma]{data/unempgroups2.csv}\unempt
	\pgfplotstableread[header=true, col sep=comma]{data/unempgroups3.csv}\unemptt
	\pgfplotstableread[header=true, col sep=semicolon]{data/cps_educ.csv}\edsh
	\pgfplotstableread[header=true, col sep=comma]{data/cps_educ_tot.csv}\edtot
	\pgfplotstableread[header=true, col sep=comma]{data/cps_age.csv}\agesh
	\pgfplotstableread[header=true, col sep=semicolon]{data/union_ind.csv}\unmem
	\pgfplotstableread[header=true, col sep=semicolon]{data/quits_ind.csv}\quits
	\pgfplotstableread[header=true, col sep=semicolon]{data/state_pa_epop.csv}\paepop
	\pgfplotstableread[header=true, col sep=semicolon]{data/state_pa_epop2.csv}\paepopt
	\pgfplotstableread[header=true, col sep=semicolon]{data/state_pa_epop3.csv}\paepoptt
	\pgfplotstableread[header=true, col sep=semicolon]{data/openings_ind.csv}\opens
	\pgfplotstableread[header=true, col sep=comma]{data/nilf_comp.csv}\nilf
	\pgfplotstableread[header=true, col sep=comma]{data/pinc.csv}\pinc
	\pgfplotstableread[header=true, col sep=comma]{data/unemp_grp.csv}\ungrp
	\pgfplotstableread[header=true, col sep=comma]{data/unemp_grpsh.csv}\ungrpsh
	\pgfplotstableread[header=true, col sep=comma]{data/ce_age.csv}\ceage
	\pgfplotstableread[header=true, col sep=comma]{data/ce_inc.csv}\ceinc
	\pgfplotstableread[header=true, col sep=comma]{data/cpi_monthly.csv}\cpimo
	\pgfplotstableread[header=true, col sep=comma]{data/ppi_monthly.csv}\ppimo
	\pgfplotstableread[header=true, col sep=comma]{data/ccdebtbar.csv}\ccbar
	\pgfplotstableread[header=true, col sep=comma]{data/gdp_rec.csv}\gdprec
	\pgfplotstableread[header=true, col sep=comma]{data/educ_wage_bar.csv}\web
	\pgfplotstableread[header=true, col sep=comma]{data/unemp_reason_mon.csv}\unrsn
	\pgfplotstableread[header=true, col sep=comma]{data/inf_exp_ch.csv}\rych
	\pgfplotstableread[header=true, col sep=comma]{data/inf_exp_ch2.csv}\rytch
	\pgfplotstableread[header=true, col sep=comma]{data/jobs_tercile.csv}\jobt
	\pgfplotstableread[header=true, col sep=comma]{data/jobs_ss.csv}\jobss
	\pgfplotstableread[header=true, col sep=comma]{data/emp_lt.csv}\emplt
	\pgfplotstableread[header=true, col sep=comma]{data/icsa_mon.csv}\icsamon
	\pgfplotstableread[header=true, col sep=comma]{data/ccsa_mon.csv}\ccsamon
	\pgfplotstableread[header=true, col sep=comma]{data/uwe_cps_sh.csv}\uwecps
	
	% Required for bar plots with individual bar colors for categories
	\pgfplotsset{discard if not/.style 2 args={
        x filter/.code={
            \edef\tempa{\thisrow{#1}}
            \edef\tempb{#2}
            \ifx\tempa\tempb
            \else
                \def\pgfmathresult{inf}
            \fi}}}	
	
% % % % % % % %
%
%  Begin Document
%
% % % % % % % %		
\begin{document}
\chapter*{
		\textcolor{blue!70}{\rule[-1pt]{6pt}{20pt}}
		\textcolor{green!70!blue}{\rule[-1pt]{6pt}{32pt}} \ \color{darkgray} US Chartbook}
\vspace*{-16mm}

\footnotesize \hspace{11mm} v0.0; Last updated: \today \normalsize 

\vspace{5mm}

\small \textit{Open source notes on the United States economy}

\vspace{4mm}

\thispagestyle{empty}

\begin{minipage}{0.36\textwidth}
\subsection*{ {\color{red} \faExclamationTriangle} \seriffont Warning}

\small {\color{red} \textbf{Early stage draft!}} \\
This early draft contains many errors! 
\vfill

\end{minipage} \hspace{18mm}
\begin{minipage}{0.36\textwidth}
\subsection*{{\color{gray} \faUser} Contact}

\textbf{Brian Dew} \  \\
\small {\color{gray} \faEnvelope} \ brian.w.dew@gmail.com \ \\
{\color{gray} \faTwitter} \ @bd\_econ \ \\
{\color{gray} \faGithub} \ \ \href{https://github.com/bdecon/US-chartbook}{bdecon/US-chartbook}
\end{minipage}
\vspace{5mm}

\begin{minipage}{0.76\textwidth}
\subsection*{\textcolor{blue!70}{\rule[-0.5pt]{3pt}{7.5pt}}
		\textcolor{green!70!blue}{\rule[-0.5pt]{3pt}{12pt}} \ About the Chartbook}
\vspace{2mm}

\small \textit{I like a place with a lot of items on a menu. Because you know they do them all beautifully.} \textbf{Will Ferrell}

\vspace{2mm}

This chartbook offers a big menu of US economic and social indicators. To keep the data fresh and cover a wide-variety of topics, shortcuts are made on the back end. Most of the text is generated by simple scripts. Likewise, the charts are standardized with each other in ways that reduce how well they represent a topic. 

As a result of these shortcuts, it is unlikely that you will be completely satisfied with the content of the chartbook. To sweeten the deal, I've added links to the data and \href{https://github.com/bdecon/US-chartbook}{source code}. Hopefully the end result can inspire and facilitate further exploration of topics of interest.

%Please be aware that this chartbook is an early-stage draft. Content is being added, removed, and improved. In the meantime, the current draft contains many errors and is not particularly comprehensible without lots of patience. I'm correcting the errors as I find them and gradually editing the text for clarity. 
%
\textbf{Version 0.1 release planned for Spring 2023}
\end{minipage}
\newpage
\section*{\hyperlink{toc}{\faList} \ {Contents}}
\markright{\seriffont Contents}
\hypertarget{toc}{}
\vspace{3mm}
\small

\begin{minipage}{0.4\textwidth}
\begin{description}
\item {\hyperlink{oea}{Overall Economic Activity}}
\begin{description}
\item {\hyperlink{oety}{Types of Activity}}
\item {\hyperlink{oegr}{Economic Growth}}
\item {\hyperlink{oegc}{Components of Growth}}
\end{description}
\item {\hyperlink{ofa}{Overall Financial Activity}}
\begin{description}
\item {\hyperlink{ofl}{Liabilities}}
\item {\hyperlink{ofsb}{Sectoral Balances}}
\item {\hyperlink{ofw}{Wealth}}
\item {\hyperlink{ofi}{Investment}}
\end{description}
\item {\hyperlink{hh}{Households}}
\begin{description}
\item {\hyperlink{hhdem}{Demographics}}
\item {\hyperlink{hhinc}{Income}}
\item {\hyperlink{hhss}{Spending and Saving}}
\item {\hyperlink{hhbs}{Balance Sheets}}
\item {\hyperlink{hhh}{Housing}}
\item {\hyperlink{hhpov}{Poverty}}
\end{description}
\item {\hyperlink{bus}{Businesses}}
\begin{description}
\item {\hyperlink{busin}{Investment}}
\item {\hyperlink{buspr}{Corporate Profits}}
\item {Balance Sheets}
\item {\hyperlink{busip}{Industrial Production}}
\item {\hyperlink{busrs}{Retail Sales}}
\end{description}
\item {\hyperlink{gov}{Government}}
\begin{description}
\item Spending and Investment
\item Revenue
\item {\hyperlink{govbs}{Balance Sheets}}
\end{description}
\item {\hyperlink{ext}{External Sector}}
\begin{description}
\item {\hyperlink{exbop}{Balance of Payments}}
\item {\hyperlink{extt}{Trade}}
\item {\hyperlink{exiip}{International Investment Position}}
\item {\hyperlink{excf}{Capital Flows}}
\item {\hyperlink{extfx}{Exchange Rates}}
\end{description}
\end{description}
\end{minipage} \hspace{10mm} 
\begin{minipage}{0.4\textwidth}
\begin{description}
\item {\hyperlink{lab}{Labor Markets}}
\begin{description}
\item {\hyperlink{labe}{Employment}}
\item {\hyperlink{labu}{Unemployment}}
\item {\hyperlink{labp}{Participation}}
\item {\hyperlink{labf}{Labor Force Flows}}
\item {\hyperlink{labh}{Hours}}
\item {\hyperlink{labns}{Nonstandard Work Arrangements}}
\item {\hyperlink{labw}{Wages}}
\item {\hyperlink{labprod}{Productivity}}
\item {\hyperlink{labun}{Union Membership}}
\end{description}
\item {\hyperlink{cap}{Capital Markets}}
\begin{description}
\item {\hyperlink{capeq}{Equity Markets}}
\item {\hyperlink{capint}{Interest Rates}}
\item {\hyperlink{capmm}{Money and Monetary Policy}}
\end{description}
\item {\hyperlink{pr}{Prices}}
\begin{description}
\item {\hyperlink{prin}{Consumer Price Index}}
\item {\hyperlink{prie}{Inflation Expectations}}
\item {\hyperlink{prpce}{PCE Price Index}}
\item {\hyperlink{prp}{Producer Prices}}
\item {\hyperlink{prex}{Import and Export Prices}}
\item {\hyperlink{prco}{Commodities}}
\end{description}
\item {\hyperlink{index}{Index}}
\end{description}
\vspace{4.6cm}
\end{minipage}
\newpage
\begin{minipage}{0.76\textwidth}
\subsubsection*{Jobless Claims}
\small Each week, the Department of Labor \href{https://www.dol.gov/ui/data.pdf}{present} the unemployment insurance (UI) claims reported by state unemployment offices. An initial claim for UI is filed by an unemployed person, after a separation from an employer, to determine eligibility for benefits.
\end{minipage}
\vspace{1mm}

\begin{minipage}{0.42\textwidth}
%\index{unemployment!initial claims}
\normalsize \textbf{New Jobless Claims}\\
\footnotesize{\textit{initial claims per week, thousands, seasonally adjusted}}

\hspace*{-2mm} \begin{tikzpicture}
	\begin{axis}[\shticks
	\bbar{y}{0}, height=5.2cm, width=6.65cm, yticklabel style={text width=1.6em}, 
		ymin=0, enlarge y limits={lower, 0.0}, enlarge x limits={0.01}, xtick=data,
		minor xtick={{2022-01-01}, {2023-01-01}, {2024-01-01}},
		clip=false, xticklabel style={align=center}, minor tick length=7pt, 
		xticklabels from table={\icsamon}{label}, ]
	\stdline{blue!0}{date}{icsa}{\icsamon}
	%\stdline{black!12}{date}{V12M}{data/icsa.csv}
	\stdline{orange!80!yellow}{date}{icsa}{data/icsa.csv}
	\input{text/icsa_node.txt}
	\end{axis}
\end{tikzpicture}\\
\footnotesize{Source: Department of Labor} \hfill \tbllink{icsa.csv} \hspace{1mm}
\end{minipage} \hspace{4mm} \begin{minipage}{0.3\textwidth}
\small \input{text/icsa.txt}

Initial claims are considered a leading indicator of labor market conditions. An increase in jobless claims suggests a deterioration in economic conditions. 
\end{minipage}
\vspace{1mm}

\begin{minipage}{0.76\textwidth}
\small The Labor Department additionally report continued claims for UI, also referred to as insured unemployment. Insured unemployment is the number of people receiving UI benefits during a given week. 
\end{minipage}
\vspace{0.5mm}

\begin{minipage}{0.42\textwidth}
%\index{unemployment!initial claims}
\normalsize \textbf{Insured Unemployed}\\
\footnotesize{\textit{continuing claims, thousands, seasonally adjusted}}

\hspace*{-2mm} \begin{tikzpicture}
	\begin{axis}[\shticks
	\bbar{y}{0}, height=5.4cm, width=6.5cm, yticklabel style={text width=2.2em}, 
		ymin=0, enlarge y limits={lower, 0.0}, enlarge x limits={0.01}, xtick=data,
		minor xtick={{2022-01-01}, {2023-01-01}, {2024-01-01}},
		clip=false, xticklabel style={align=center}, minor tick length=7pt, 
		xticklabels from table={\ccsamon}{label}, ]
	\stdline{blue!0}{date}{ccsa}{\ccsamon}
	%\stdline{black!12}{date}{V12M}{data/icsa.csv}
	\stdline{green!75!black}{date}{ccsa}{data/ccsa.csv}
	\input{text/ccsa_node.txt}
	\end{axis}
\end{tikzpicture}\\
\footnotesize{Source: Department of Labor} \hfill \tbllink{ccsa.csv} \hspace{1mm}
\end{minipage} \hspace{4mm} \begin{minipage}{0.3\textwidth}
\small \input{text/ccsa.txt}

\input{text/ccsa_alt.txt}
\end{minipage}
\vspace{5mm}

\begin{minipage}{0.76\textwidth}
\normalsize \textbf{Jobless Claims}\\
\footnotesize{\textit{thousands per week \hspace{52mm} period averages}}\\
\noindent \hspace*{-2mm} \rowcolors{1}{}{black!5} \setlength{\tabcolsep}{4.1pt} \color{black!90}
		{\renewcommand{\arraystretch}{1.6}
		 \begin{tabular}{p{32mm} R{10mm} R{10mm} R{10mm} R{10mm} R{8mm} R{8mm} R{8mm} }
			 \input data/jobless_claims.tex \hline
		\end{tabular}}\vspace{-2mm}
		
\footnotesize{Source: Department of Labor}
\end{minipage}
\newpage
\begin{minipage}{0.76\textwidth}
\subsubsection*{Jobless Claims}
\small Each week, the Department of Labor \href{https://www.dol.gov/ui/data.pdf}{present} the unemployment insurance (UI) claims reported by state unemployment offices. Initial claims are filed by an unemployed person, after a separation from an employer, to determine eligibility for benefits. Initial claims are considered a leading indicator of labor market conditions.

\input{text/icnsa.txt}

\input{text/ccnsa.txt}
\end{minipage}
\vspace{1mm}

\begin{minipage}{0.345\textwidth}
\normalsize \textbf{New UI Claims}\\
\footnotesize{\textit{initial claims per week, in millions,}}\\
\footnotesize{\textit{not seasonally adjusted}}\\
\hspace*{-2mm} \begin{tikzpicture}
	\begin{axis}[\bbar{y}{0}, \ltdateaxisticks ytick={0, 1, 2, 3, 4, 5, 6}, 
		enlarge y limits={0.05}, ymin=0.25, xmin={2018-01-01},
		xticklabel={`\short{\year}}, height=7.2cm, width=6.5cm]
	\rbars
	\thinline{blue!50!purple!80!black}{date}{pua_ic}{data/fed_uic.csv}
	\stdline{cyan!80!blue}{date}{VALUE}{data/icnsa.csv}
	\stdnode{3.4cm}{0.35cm}{\footnotesize \color{blue!50!purple!80!black}PUA}
	\stdnode{2.45cm}{0.7cm}{\footnotesize \color{cyan!80!blue}State}
	\end{axis}
\end{tikzpicture}\\
\footnotesize{Source: Department of Labor} \hfill \tbllink{icnsa.csv}
\end{minipage} \hspace{9mm}
\begin{minipage}{0.35\textwidth}
\normalsize \textbf{Continued UI Claims}\\
\footnotesize{\textit{insured unemployed, in millions,}}\\
\footnotesize{\textit{not seasonally adjusted}}\\
\hspace*{-2mm} \begin{tikzpicture}
	\begin{axis}[\bbar{y}{0}, \ltdateaxisticks ytick={0, 5, 10, 15, 20}, 
		enlarge y limits={0.05}, ymin=1, xmin={2018-01-01},
		yticklabel style={text width=1.0em},
		xticklabel={`\short{\year}}, height=7.2cm, width=6.5cm]
	\rbars
	\thinline{green!50!blue}{date}{fed_cc}{data/fed_uic.csv}
	\stdline{green!90!blue}{date}{VALUE}{data/ccnsa.csv}
	\stdnode{4.3cm}{4.0cm}{\footnotesize \color{green!50!blue}PUA+\\ \footnotesize \color{green!50!blue}PEUC}
	\stdnode{2.4cm}{0.85cm}{\footnotesize \color{green!90!blue}State}
	\end{axis}
\end{tikzpicture}\\
\footnotesize{Source: Department of Labor} \hfill \tbllink{ccnsa.csv}
\end{minipage}
\vspace{1mm}

\begin{minipage}{0.76\textwidth}
\small In response to the COVID-19 pandemic, traditional state-run unemployment insurance was temporarily boosted by federal programs that expanded eligibility for benefits and increased the amount of benefit payments. These programs were ended on September 6, 2021. 

%\input{text/fed_uic.txt}

\end{minipage}
\newpage
\begin{minipage}{0.76\textwidth}
\small The \textbf{Treasury yield curve} \href{https://www.treasury.gov/resource-center/data-chart-center/interest-rates/Pages/TextView.aspx?data=yield}{shows} the interest rates on different maturities of US Treasury bonds and bills, at a given point in time. The yield curve summarizes the term structure of interest rates, how much it costs to borrow for different periods of time, and has traditionally been considered an indicator of how markets view short-term economic conditions relative to longer-term conditions. 

The yield curve is normally upward sloping as investors expect to be compensated for lending for a longer period of time. The shape of the yield curve changes over time and is affected by several factors, including the term premium, the monetary policy of the Federal Reserve, and expectations about future inflation. The curve can become steeper, for example, if interest rates or inflation is expected to be higher in the future. 
\end{minipage}
\vspace{1mm}

\begin{minipage}{0.43\textwidth}
\index{treasuries!yield curve}
\normalsize \textbf{Treasury Yield Curve}\\
\footnotesize{\textit{constant maturity yield, percent}}\\
\hspace*{-2mm} \begin{tikzpicture}[trim axis right]
	\begin{axis}[height=7.2cm, width=7.7cm,	xmajorgrids, ymax=8.1, ymin=0.35,
		axis line style={draw=none}, nodes near coords style={style={black!80, fill=white,
		 yshift=1.5mm, inner sep=0.5}, /pgf/number format/.cd, fixed zerofill, 
		 precision=2, assume math mode}, legend style={at={(0.5, 0.92)}, 
		 legend columns=1},
		xtick=data, xticklabels={1M, 3M, 6M, 1Y, 2Y, 5Y, 10Y, 20Y, 30Y}, 
		legend cell align={left},
		reverse legend, xticklabel style={black!70, font=\small\bfseries}, 
		enlarge y limits={0.08}, enlarge x limits={0.02}, \bbar{y}{0}]
	\addplot[thick, mark=*, color=blue!65!black!30] 
		table [x=number, y=fiveyear, col sep=comma] {data/yc.csv};
	\addplot[thick, mark=*, color=blue!65!black!60] 
		table [x=number, y=oneyear, col sep=comma] {data/yc.csv};			
	\addplot[ultra thick, mark=*, nodes near coords, 
    		color=blue!60!black] table [x=number, y=value, col sep=comma] {data/yc.csv};
    \input{text/yc_date.txt}
    \legend{Five Years Ago, One Year Ago, Most Recent}	
	\end{axis}
\end{tikzpicture}\\
\footnotesize{Source: Federal Reserve} 
\end{minipage} \hspace{4mm}
\begin{minipage}{0.29\textwidth}
\small The yield curve can also become \textit{inverted} when yields on shorter-term debt are higher than yields on longer-term debt. An inverted yield curve can be a sign of worsening economic conditions. For example, short term rates may exceed longer-term rates if the Federal Reserve is expected to lower interest rates in the future, or if inflation is expected to fall due to weakened economic conditions. 

\input{text/yc_inversion.txt}
\end{minipage}
\vspace{3mm}

\begin{minipage}{0.76\textwidth}
\index{treasuries!yield spread}
\small Another measure of the term structure of interest rates is the \textit{spread} between treasuries with different maturities. \textbf{Treasury yield spreads} can be used to track changes in the term structure over time.

\input{text/spread_basic.txt}
\vspace{2mm}

\normalsize \textbf{Treasury Yield Spreads}\\
\footnotesize{\textit{percentage points}}\\
\hspace*{-3mm} \begin{tikzpicture}
    \begin{groupplot}[group style={group size=2 by 1, horizontal sep=52pt,}]
    \nextgroupplot[\bbar{y}{0}, \ltdateaxisticks ytick={-1, 0, 1, 2, 3}, ymin=-1.45, ymax=2.3,
    	yticklabel style={text width=1.2em}, 
     xticklabel={`\short{\year}},  height=5.6cm, width=5.9cm, clip=false]
    \rbar
	\thinline{blue!70!cyan!80!white}{date}{Ten-3M}{data/spread.csv}
	\node[text width=3.8cm, anchor=west] at (axis description cs: 0, 0.95) 
		{\small \color{blue!70!cyan!80!white}\textbf{10-Year - 3-Month}};
	\input{text/spread_node.txt}
    \nextgroupplot[\bbar{y}{0}, \ltdateaxisticks ytick={-1, 0, 1, 2, 3}, ymin=-1.45, ymax=2.3,
    	yticklabel style={text width=1.2em},
     xticklabel={`\short{\year}}, height=5.6cm, width=5.9cm, clip=false]
	\rbar
	\thinline{red!60!violet!90!white}{date}{Ten-2Y}{data/spread.csv}
	\node[text width=3.8cm, anchor=west] at (axis description cs: 0, 0.95) 
		{\small \color{red!60!violet!90!white}\textbf{10-Year - 2-Year}};
	\input{text/spread_node2.txt}
	\end{groupplot}
	\end{tikzpicture}\\
\footnotesize{Source: Federal Reserve} \hfill \tbllink{spread.csv} 
\end{minipage}
\newpage 
\hypertarget{pr}{} \markright{\seriffont Prices} \index{prices!consumer price index}
\begin{minipage}{0.76\textwidth} 
\section*{Prices}
\vspace*{-2mm}

\small The price of goods and services determine how much can be purchased by a fixed income. Researchers are interested in the prices of specific goods, as well as changes in overall purchasing power, more generally.

To understand the overall change in prices paid or charged by a group, such as consumers or manufacturers, researchers create a representative ``basket'' of the goods and services relevant to the group, and track the changes in the basket, and the price of the basket, over time. The end result of these methods is a price index. Researchers can then use the price index to calculate the rate of inflation.

Inflation is typically calculated as the 12-month percent change in the price index. This annual inflation rate measures how prices in a given month compare to prices during the same month, one year prior. 
\vspace{1mm}

\normalsize \textbf{Price Growth, Various Measures}\\
\footnotesize{\textit{one-year growth, percent}}
\vspace*{-4mm}

\hspace*{-2mm} \rowcolors{1}{}{black!5} \setlength{\tabcolsep}{3.1pt} \color{black!90}
	{\renewcommand{\arraystretch}{1.5}
		\begin{tabular}{p{30mm} R{7.6mm} R{7.6mm} R{7.6mm} R{7.6mm} R{7.6mm} R{7.6mm} 
		   R{9.2mm} R{8.5mm}}
			 \input data/prices_12m.tex \hline
		\end{tabular}}\vspace{-1mm}
		
\footnotesize{Source: BLS, BEA, Federal Reserve Bank of Dallas}
\vspace{2mm}

\small In effect, the 12-month percent change in prices is smoothed, relative to the one-month change, by including information on price changes that happened over the past year. While the chartbook uses less-volatile 12-month inflation rates in most cases, the \textbf{one-month rate} can be more useful for examining short-term trends, for example by eliminating the base effects from changes in prices a year ago. 

\input{text/cpi_monthly.txt} The Cleveland Fed \href{https://www.clevelandfed.org/indicators-and-data/inflation-nowcasting}{nowcasts} current inflation by combining recent inflation data with current oil and gasoline prices. \input{text/cpinow.txt}
\vspace{1mm}

\normalsize \textbf{CPI One-Month Change}\\
\footnotesize{\textit{percent change from previous month}}
\vspace*{-4mm}

\hspace*{-3mm} \begin{tikzpicture} 
	\begin{axis}[\bbar{y}{0}, date coordinates in=x, axis line style={draw=none},
		enlarge y limits={0.24}, enlarge x limits={0.02}, 
		width=13.0cm, height=4.7cm,
		yticklabel style={text width=1.0em}, clip=false,
		nodes near coords style={/pgf/number format/.cd, fixed zerofill,
			precision=1, assume math mode}, xtick=data,
		xticklabels from table={\cpimo}{label}]
	\addplot[thick, draw opacity=0] table [x=date, y=FILL, col sep=comma]{\cpimo};
	\addplot[ybar, fill=blue, draw opacity=0, nodes near coords] table [x=date, y=ALL_S, col sep=comma]{\cpimo};
	\draw [black!40!white, dashed] (rel axis cs:0.35,0) -- 
		(rel axis cs:0.35, 1.06);
	\node[below right, align=left] at (rel axis cs:0.35, 1.08) {\color{black!40!white}\footnotesize {latest 12 months}};
	\draw [black!40!white, dashed] (rel axis cs:0.96,0) -- 
		(rel axis cs:0.96, 1.06);
	\node[below right, align=left] at (rel axis cs:0.96, 1.08) {\color{black!40!white}\footnotesize {now-}};
	\node[below right, align=left] at (rel axis cs:0.96, 1.02) {\color{black!40!white}\footnotesize {cast}};
	\input{text/cpinow_node.txt}
	\end{axis}
\end{tikzpicture}\\
\footnotesize{Source: Bureau of Labor Statistics, Federal Reserve Bank of Cleveland} \hfill \tbllink{cpi_monthly.csv}
\end{minipage}
\newpage
\vspace{-9mm}

\begin{minipage}{0.76\textwidth} \index{usual weekly earnings} \index{wages} \index{median wage}
\subsection*{Wages} \hypertarget{labw}{}
\small Wages are an important indicator and are closely monitored by economists. Wages are the majority of income in the economy and the main cost for businesses. Wage growth is particularly closely monitored as it affects quality of life and can affect inflation rates. 

This subsection covers several wage measures. First, the distribution of usual weekly earnings provides the median, or typical, full-time wage, and insight into wage growth for low-wage workers and other groups. Next, the section discusses average hourly earnings, including by industry, and employee and benefit costs. Finally, we measure the wages of the same individuals, over time. 

\subsubsection*{Usual Weekly Earnings}
\small The Bureau of Labor Statistics (BLS) \href{https://www.bls.gov/webapps/legacy/cpswktab5.htm}{report} the \textbf{usual wages of full-time workers} at various points in the income distribution, including by decile and by quartile. The most commonly used of these measures is the median usual weekly earnings, which represents the middle wage; half of wages are above and half are below.

\input{text/uwe_median.txt}
\vspace{1mm}

\normalsize \textbf{Median Usual Weekly Earnings}\\
\footnotesize{\textit{one-year growth, percent, full-time, wage and salary earners, age 16+}}
\vspace{2.8cm}

\hspace*{4mm} \begin{tikzpicture}[overlay]
	\begin{axis}[\bbar{y}{0}, \dateaxisticks ytick={-5, 0, 5, 10, 15}, 
		enlarge y limits={0.05}, xticklabel={`\short{\year}}, ymax=10.2,
		clip=false, width=12.7cm, height=4.8cm, xmin=1989-07-01, 
		legend style={at={(0.44, 1.05)}, legend columns=1}, 
		legend cell align={left}]
	\rbars
	\thickline{cyan!60!white}{date}{p50uwe}{data/uwe_bls_gr.csv}
	\thinline{violet!80!blue}{date}{p50_3M_gr}{data/uwe_cps.csv}
    \legend{BLS (quarterly), CPS (3-month moving average)}	
	\end{axis}
\end{tikzpicture}

\footnotesize{Source: Bureau of Labor Statistics, Author's Calculations} \hfill \tbllink{uwe_bls_gr.csv} \ \ \tbllink{uwe_cps.csv}
\vspace{2mm}

\small The primary source for BLS quarterly estimates of usual weekly earnings is the \href{https://www.census.gov/data/datasets/time-series/demo/cps/cps-basic.html}{Current Population Survey} (CPS). Using the CPS, more-volatile monthly estimates can be calculated before the next BLS quarterly estimate is available. 
\end{minipage}
\vspace{1mm}

\begin{minipage}{0.35\textwidth}
\small \input{text/uwe_median_cps.txt}
\end{minipage} \hspace{6mm} \begin{minipage}{0.35\textwidth}
\normalsize \textbf{Median Usual Weekly Earnings}\\
\footnotesize{\textit{one-year growth, percent}}
\vspace*{-2mm}

\hspace*{-3mm} \begin{tikzpicture} 
	\begin{axis}[\bbar{y}{0}, date coordinates in=x, axis line style={draw=none},
		enlarge y limits={0.16}, enlarge x limits={0.01}, 
		width=6.6cm, height=5.0cm, 
		yticklabel style={text width=1.0em}, 
		minor xtick={{2023-01-01}, {2024-01-01}},
		xticklabel style={align=center}, minor tick length=7pt, 
		nodes near coords style={black, /pgf/number format/.cd, fixed zerofill,
			precision=1, assume math mode}, xtick=data,
		xticklabels from table={\uwecps}{label}]
	\addplot[thick, draw opacity=0] table [x=date, y=FILL, col sep=comma]{\uwecps};
	\addplot[ybar, bar width=50.0pt, fill=cyan!30!white, draw=cyan!90!blue] table [x=date, y=p50uwe, col sep=comma]{data/uwe_bls_sh.csv};
	\addplot[ybar, bar width=9.6pt, fill=violet!80!blue, draw opacity=0, nodes near coords] table [x=date, y=p50_gr, col sep=comma]{data/uwe_cps_shift.csv};
	\end{axis}
\end{tikzpicture}\\
\footnotesize{Source: BLS, Author} \hfill \tbllink{uwe_cps.csv}

\end{minipage}
\newpage
\vspace{-8mm}

\begin{minipage}{0.76\textwidth} \index{usual weekly earnings} \index{wages} \index{first decile wage}
\small The income distribution also tells us the earnings of low-wage workers, represented here by the first decile. Only ten percent of workers earn less than the first decile wage. \input{text/uwe_p10_basic.txt}
\vspace{1mm}

\normalsize \textbf{First Decile Usual Weekly Earnings}\\
\footnotesize{\textit{one-year growth, percent, full-time, wage and salary earners, age 16+}}
\vspace{2.6cm}

\hspace*{4mm} \begin{tikzpicture}[overlay]
	\begin{axis}[\bbar{y}{0}, \dateaxisticks ytick={-5, 0, 5, 10, 15}, 
		enlarge y limits={0.05}, xticklabel={`\short{\year}}, ymax=9.8,
		clip=false, width=12.7cm, height=4.6cm, xmin=1990-01-01, reverse legend,
		legend style={at={(0.72, 1.02)}, legend columns=-1}, 
		legend cell align={left}]
	\rbars
	\thinline{lime!65!green!90!black}{date}{p10_3M_gr}{data/uwe_cps.csv}
	\thickline{blue!65!black}{date}{p10uwe}{data/uwe_bls_gr.csv}
    \legend{CPS (3-month moving average), BLS (quarterly)}	
	\end{axis}
\end{tikzpicture}

\footnotesize{Source: Bureau of Labor Statistics, Author's Calculations} \hfill \tbllink{uwe_bls_gr.csv} \ \ \tbllink{uwe_cps.csv}
\vspace{2mm}

\small The following tables present the BLS published estimates for usual weekly earnings of full-time wage and salary earnings. The first table presents the earnings in levels, and the second table shows the one-year percent change.
\vspace{2mm}

\normalsize \textbf{Usual Weekly Earnings}\\
\footnotesize{\textit{full-time, wage and salary earners, age 16+, nominal USD}}\\
\rowcolors{1}{}{black!5} \setlength{\tabcolsep}{3.1pt} \color{black!90}
		{\renewcommand{\arraystretch}{1.54}
		 \begin{tabular}{p{19mm} R{8.8mm} R{8.8mm} R{8.8mm} R{8.8mm} R{8.8mm} R{8.8mm} 
		   R{8.8mm} R{8.8mm} R{8.8mm} R{8.8mm}}
			 \input data/wage_dist_bls2.tex \hline
		\end{tabular}}
\vspace{-3mm}
		
\footnotesize{Source: Bureau of Labor Statistics}
\vspace{3mm}

\normalsize \textbf{Weekly Earnings Growth}\\
\footnotesize{\textit{full-time, wage and salary earners, age 16+, one-year growth, percent}}\\
\rowcolors{1}{}{black!5} \setlength{\tabcolsep}{3.1pt} \color{black!90}
	{\renewcommand{\arraystretch}{1.54}
		\begin{tabular}{p{19mm} R{8.8mm} R{8.8mm} R{8.8mm} R{8.8mm} R{8.8mm} R{8.8mm} 
		   R{8.8mm} R{8.8mm} R{8.8mm} R{8.8mm}}
			 \input data/wage_dist_bls.tex \hline
		\end{tabular}}\vspace{-2mm}
				
\footnotesize{Source: Bureau of Labor Statistics}
\end{minipage}
\newpage
\vspace{-10mm}

\begin{minipage}{0.76\textwidth} \index{prices!consumer price index}
\small \input{text/cpi_monthly_rel.txt}
\end{minipage}
\vspace{1mm}

\normalsize \textbf{Selected CPI Categories, Monthly Rate}\\
\footnotesize{\textit{one-month growth, seasonally adjusted, percent}\\
\hspace*{-3mm} \rowcolors{1}{}{black!5} \setlength{\tabcolsep}{2.6pt} \color{black!90}
		{\renewcommand{\arraystretch}{1.4}
\begin{tabular}{p{38mm} R{8.2mm} R{8.2mm} R{8.2mm} R{8.2mm} R{8.2mm} R{8.2mm} 
		R{8.2mm} R{8.2mm}} % 
			 \input data/cpi_comp_mo.tex \hline
		\end{tabular}}}
\vspace{-2mm}		
		
\footnotesize{Source: Bureau of Labor Statistics; *not seasonally adjusted}
\newpage
\vspace*{-8mm}
\hypertarget{extt}{}
\begin{minipage}{0.76\textwidth}
\subsection*{International Trade}
\index{trade!overview}
\small Each month, the Census Bureau \href{https://www.census.gov/foreign-trade/Press-Release/current\_press\_release/index.html}{report} \textbf{goods and services trade} between the US and the rest of the world. US purchases of foreign goods and services are classified as imports and foreign purchases of US goods and services are exports. The trade of goods includes consumer goods, industrial equipment, and agricultural products. Services trade includes travel and tourism, business services, and charges for the use of intellectual property, among other services. 
\end{minipage}
\vspace*{-1mm}

\begin{minipage}{0.325\textwidth}
\normalsize \textbf{US Imports and Exports}\\
\footnotesize{\textit{billions of US dollars, seasonally adjusted}}\\
\hspace*{-3mm} \begin{tikzpicture}
	\begin{axis}[\ltdateaxisticks 
		enlarge y limits={0.05}, clip=false, max space between ticks=30,
		yticklabel style={text width=1.3em},
		xticklabel={`\short{\year}}, height=5.8cm, width=5.4cm]
	\rbar
	\stdline{green!80!blue}{date}{EXP}{data/tradelt.csv}
	\stdline{blue!80!violet}{date}{IMP}{data/tradelt.csv}
	\input{text/tradelt_nodes.txt}
	\absnode{{2017-01-01}}{190}{\small \color{green!80!blue}Exports}
	\absnode{{2015-02-01}}{265}{\small \color{blue!80!violet}Imports}
	\end{axis}
\end{tikzpicture}

\normalsize \textbf{Trade Balance}\footnotesize\\
\hspace*{-5mm} \begin{tikzpicture}
	\begin{axis}[\ltdateaxisticks 
		enlarge y limits={0.06}, clip=false, 
		yticklabel style={text width=2.0em},
		xticklabel={`\short{\year}}, height=4.5cm, width=5.4cm]
	\rbar
	\stdline{red}{date}{BAL}{data/tradelt.csv}
	\input{text/ballt.txt}
	\end{axis}
\end{tikzpicture}\\
\footnotesize{Source: Census Bureau} \hfill \tbllink{tradelt.csv}
\end{minipage} \hspace{5mm}
\begin{minipage}{0.39\textwidth}
\small \input{text/tradeltlevels.txt}
\end{minipage}
\vspace{3mm}

\begin{minipage}{0.76\textwidth}
\normalsize \textbf{International Trade}\\
\footnotesize{\textit{millions of US dollars, seasonally adjusted}\\
\hspace*{-3mm} \rowcolors{1}{}{black!5} \setlength{\tabcolsep}{2.6pt} \color{black!90}
		{\renewcommand{\arraystretch}{1.6}
\begin{tabular}{p{22mm} R{14mm} R{14mm} R{14mm} R{14mm} R{14mm} R{14mm}} % 
			 \input data/trade_mo_summary.tex \hline
		\end{tabular}}}
\vspace{-1mm}		
		
\footnotesize{Source: Census Bureau} \hfill \tbllink{pinc08.csv}
\end{minipage}
\newpage
\begin{minipage}{0.76\textwidth}
\subsection*{Union Membership}
\index{unions}
\hypertarget{labun}{}
\small Membership in \textbf{unions and employee associations} has diminished in the United States over the past fifty years. Unionized jobs typically offer higher wages and better benefits and union membership tends to increase wages and benefits even in nonunion jobs. Many researchers argue that lower union membership increases income inequality. 

\input{text/union.txt}
\vspace{1mm}

\normalsize \textbf{Union Membership and Coverage}\\
\footnotesize{\textit{union or employee association share of jobs, percent, one-year moving average}}\\
\hspace*{-2mm} \begin{tikzpicture}
	\begin{axis}[\bbar{y}{0}, \dateaxisticks ytick={0, 5, 10, 15, 20}, ymin=1,
		height=5.2cm, width=12.2cm, enlarge y limits={0.09},
		yticklabel style={text width=1.0em}, 
		xticklabel={`\short{\year}}, clip=false]
	\rbars
	\ctsbar{violet}{date}{Membership}{data/union.csv}{1.8pt}
	\node[text width=3.0cm, anchor=west] at 
		({1995-01-01}, 7){\color{white}\textbf{Union/Employee \\ Association Members}};
	\ctsbar{magenta!70!purple}{date}{Diff}{data/union.csv}{1.8pt}	
	\node[text width=2.2cm, anchor=west] at 
		({2010-02-01}, 15.4){\color{magenta!70!purple}\textbf{Covered Non-Members}};
	\input{text/union_nodes.txt}
	\end{axis}
\end{tikzpicture}\\
\footnotesize{Source: Author's Calculations from Current Population Survey} \hfill \tbllink{union.csv}
\end{minipage}
\vspace{3mm}

\begin{minipage}{0.43\textwidth}
\normalsize \textbf{Union Membership Rate by Industry}\\
\footnotesize{\textit{union or employee association member, percent}}\\
\hspace*{-3mm} \begin{tikzpicture}
	\begin{axis}[clip=false, stack negative=separate, ymajorgrids=false,
            yticklabels from table={\unmem}{name},
			y axis line style={opacity=0}, 
		    x axis line style={opacity=0}, 
			yticklabel style={black, font=\footnotesize, 
			align=left, anchor=east},  
			nodes near coords align={above},    
			nodes near coords style={/pgf/number format/.cd,fixed zerofill, 
				precision=1, assume math mode},  
            ytick=data, tickwidth=0pt, xmin=0,
            bar width=2.2ex, 
            extra x ticks={0}, extra x tick style={
                grid style={black}, xticklabel=\empty,},
            enlarge y limits={0.08}, enlarge x limits={0.02},
            width=5.75cm, height=8.2cm, legend style={at={(1.0, 1.07)}},
            \barylab{2.15cm}{1.5ex}]
	\addplot[xbar stacked, bar shift=0pt, fill=none, draw=none] 
		table [y expr=-\coordindex, x index=1] {\unmem};
	\addplot[xbar stacked, bar shift=0pt, fill=violet!60!blue!90, draw=none] 
		table [y expr=-\coordindex, x index=2] {\unmem};	
	\addplot[only marks, mark=square*, mark options={fill=white, 
		draw=magenta, scale=1.4}] 
		table [y expr=-\coordindex, x index=4] {\unmem};	        
	\addplot[only marks, mark=diamond*, mark options={fill=orange!75!yellow!90!white, 
		draw=magenta, scale=2.4}] 
		table [y expr=-\coordindex, x index=3] {\unmem};
	\addplot [nodes near coords={\pgfmathprintnumber{\pgfkeysvalueof{/data point/x}}}, only marks, mark=none] table[y=y, x index=3] {\unmem};
	\legend{, 30-year range,\input{text/union_yrdt.txt},\input{text/union_ltdt.txt}}%
	\end{axis}
\end{tikzpicture}\\
\footnotesize{Source: Author's Calculations from CPS} \hspace{12mm} \tbllink{union_ind.csv}
\end{minipage}\hspace{4mm}
\begin{minipage}{0.3\textwidth}
\small Union membership rates vary substantially by industry. \input{text/union_ind.txt}
\end{minipage}
\newpage
\vspace*{-6mm}
\begin{minipage}{0.76\textwidth} \index{employment!employment rate}
\subsubsection*{Employment Rates} 
\vspace{-0.5mm}
\small The \textbf{employment rate}, or the employment-population ratio, is the share of a group that is employed. Employment rates can provide useful insight into macroeconomic conditions. A high employment rate means available labor are being utilized in the productive process. All else equal, higher employment results in both increased supply, as the result of more labor being used for production, and increased demand, as the result of higher levels of income.

Economists are interested in both the overall employment rate and in the employment rates for individual groups of people. The overall employment rate provides insight into the overall utilization of labor of a society and is affected by demographic and macroeconomic factors. Employment rates for individual groups can tell us about macroeconomic conditions and even tell us about differences in local economic conditions. 

\input{text/epop_text2.txt} 
\vspace{1mm}

\normalsize \textbf{Employment Rate, Age 16 and Older}\\
\footnotesize{\textit{employed share of age 16 and older population, percent, seasonally adjusted}}\\
\hspace*{-2mm} \begin{tikzpicture}
	\begin{axis}[\dateaxisticks ytick={50, 55, 60, 65, 70}, 
		yticklabel style={text width=1.0em}, 
		width=12.2cm, enlarge y limits={0.1},
		xticklabel={`\short{\year}}, clip=false, height=4.8cm]
	\rbars
	\stdline{green!60!black}{date}{EPOP}{data/epop.csv}
	\input{text/epop_node2.txt}
	\end{axis}
\end{tikzpicture}\\
\footnotesize{Source: Bureau of Labor Statistics} \hfill \tbllink{epop.csv}
\vspace{3mm} \index{employment!prime-age}

\small Importantly, a larger share of the US population is of retirement age, reducing the overall US employment rate. To examine macroeconomic conditions separate from demographic developments, BLS \href{LNS12300060}{report} the employment rate for a more-narrow age group, specifically, those age 25 to 54. This group has the highest employment rate and are sometimes considered the ``prime'' age for labor market purposes. 

The \textbf{age 25 to 54 employment rate} is an important measure of labor market utilization. In a tight labor market, the age group is employed at a very high rate. \input{text/epop_text.txt} 
\vspace{1mm}

\normalsize \textbf{Employment Rate, Age 25 to 54}\\
\footnotesize{\textit{employed share of age 25 to 54 population, percent, seasonally adjusted}}\\
\hspace*{-2mm} \begin{tikzpicture}
	\begin{axis}[\dateaxisticks ytick={70, 75, 80}, 
		yticklabel style={text width=1.0em}, 
		width=12.2cm, enlarge y limits={0.1},
		xticklabel={`\short{\year}}, clip=false, height=4.8cm]
	\rbars
	\stdline{blue!90!cyan}{date}{PA_EPOP}{data/epop.csv}
	\input{text/epop_node.txt}
	\end{axis}
\end{tikzpicture}\\
\footnotesize{Source: Bureau of Labor Statistics} \hfill \tbllink{epop.csv}
\end{minipage}
\newpage
\vspace*{-6mm}
\index{employment!with disability}
\begin{minipage}{0.76\textwidth} 
\small Next, BLS also \href{https://data.bls.gov/timeseries/LNU02374597}{report} the \textbf{employment rate for people with disabilities}. People with disabilities may be limited in their ability to participate in labor markets and can also face discrimination during hiring. Labor market prospects for the group are also affected by economic conditions. A tight labor market pushes businesses to accommodate disabilities and to discriminate less in hiring.

In June 2008, the Current Population Survey (CPS), started to ask respondents age 16 and older whether they have difficulty with any of the following: hearing, seeing (even while wearing glasses), walking or climbing stairs, concentrating, remembering, making decisions, dressing or bathing, or running errands alone. In the latest data, covering March 2023, around XX.X million people or XX.X percent of those age 16 and older report at least one such disability. The rate of disability is relatively stable over time, and averages XX.X percent since 2008. 


\end{minipage}
\vspace{1mm}

\begin{minipage}{0.43\textwidth} 
\normalsize \textbf{Employment Rate, with Disability}\\
\footnotesize{\textit{employed share of age group, persons with disabilities,}}\\
\footnotesize{\textit{percent, not seasonally adjusted}}\\
\hspace*{-2mm} \begin{tikzpicture}
	\begin{axis}[\bbar{y}{0}, \dateaxisticks ymin=3, 
		yticklabel style={text width=1.0em}, xmin={2007-10-01},
		width=7.0cm, enlarge y limits={0.08},
		xticklabel={`\short{\year}}, clip=false, height=6.4cm]
	\recbars
	\stdline{red}{date}{bls}{data/dis_emp_rate.csv}
	\stdline{cyan}{date}{cps}{data/dis_emp_rate.csv}
	\absnode{{2012-07-01}}{16}{\footnotesize \color{red}\textbf{Age 16 and older}}
	\absnode{{2010-07-01}}{36}{\footnotesize \color{cyan}\textbf{Age 25 to 54}}
	\input{text/dis_emp_nodes.txt}
	\end{axis}
\end{tikzpicture}\\
\footnotesize{Source: Bureau of Labor Statistics; Author} \hfill \tbllink{dis_emp_rate.csv}


\end{minipage}
\newpage
\begin{minipage}{0.76\textwidth} \index{employment!employment rate}
\small The monthly jobs report describes employment at a given point in time, by asking about activities during a specific week of the previous month. To instead examine activities over a period of time, annual data on weeks worked per year and hours worked per week can be combined to identify the \textit{fully-employed}, or \textit{full-time, full-year workers}, who usually work 35 hours per week or more for 50 weeks per year or more. The Census Bureau \href{https://www.census.gov/data/tables/time-series/demo/income-poverty/cps-pinc/pinc-01.html#par_textimage_14}{report} \input{text/asec_ftfy_tot.txt}

Employment rates vary dramatically by location. \input{text/acs_ftfy_text.txt}The top ten and bottom ten commuter zones by fully-employed rate are listed below.
\end{minipage}
\vspace{1mm}

\begin{minipage}{0.55\textwidth}
\normalsize \textbf{Commuter Zone Fully-Employed Rate}\\
\footnotesize{\textit{full-time, full-year worker share of population, \input{text/acs_ftfy_year.txt}}}\\
\vspace*{-6mm}

\hspace{-11mm} %% Creator: Matplotlib, PGF backend
%%
%% To include the figure in your LaTeX document, write
%%   \input{<filename>.pgf}
%%
%% Make sure the required packages are loaded in your preamble
%%   \usepackage{pgf}
%%
%% and, on pdftex
%%   \usepackage[utf8]{inputenc}\DeclareUnicodeCharacter{2212}{-}
%%
%% or, on luatex and xetex
%%   \usepackage{unicode-math}
%%
%% Figures using additional raster images can only be included by \input if
%% they are in the same directory as the main LaTeX file. For loading figures
%% from other directories you can use the `import` package
%%   \usepackage{import}
%%
%% and then include the figures with
%%   \import{<path to file>}{<filename>.pgf}
%%
%% Matplotlib used the following preamble
%%   \usepackage{fontspec}
%%   \setmainfont{DejaVuSerif.ttf}[Path=/home/brian/miniconda3/lib/python3.8/site-packages/matplotlib/mpl-data/fonts/ttf/]
%%   \setsansfont{DejaVuSans.ttf}[Path=/home/brian/miniconda3/lib/python3.8/site-packages/matplotlib/mpl-data/fonts/ttf/]
%%   \setmonofont{DejaVuSansMono.ttf}[Path=/home/brian/miniconda3/lib/python3.8/site-packages/matplotlib/mpl-data/fonts/ttf/]
%%
\begingroup%
\makeatletter%
\begin{pgfpicture}%
\pgfpathrectangle{\pgfpointorigin}{\pgfqpoint{3.645016in}{2.389500in}}%
\pgfusepath{use as bounding box, clip}%
\begin{pgfscope}%
\pgfsetbuttcap%
\pgfsetmiterjoin%
\pgfsetlinewidth{0.000000pt}%
\definecolor{currentstroke}{rgb}{0.000000,0.000000,0.000000}%
\pgfsetstrokecolor{currentstroke}%
\pgfsetstrokeopacity{0.000000}%
\pgfsetdash{}{0pt}%
\pgfpathmoveto{\pgfqpoint{-0.000000in}{0.000000in}}%
\pgfpathlineto{\pgfqpoint{3.645016in}{0.000000in}}%
\pgfpathlineto{\pgfqpoint{3.645016in}{2.389500in}}%
\pgfpathlineto{\pgfqpoint{-0.000000in}{2.389500in}}%
\pgfpathclose%
\pgfusepath{}%
\end{pgfscope}%
\begin{pgfscope}%
\pgfpathrectangle{\pgfqpoint{0.100000in}{0.100000in}}{\pgfqpoint{3.420221in}{2.189500in}}%
\pgfusepath{clip}%
\pgfsetbuttcap%
\pgfsetmiterjoin%
\pgfsetlinewidth{0.000000pt}%
\definecolor{currentstroke}{rgb}{0.000000,0.000000,0.000000}%
\pgfsetstrokecolor{currentstroke}%
\pgfsetstrokeopacity{0.000000}%
\pgfsetdash{}{0pt}%
\pgfpathmoveto{\pgfqpoint{0.100000in}{0.100000in}}%
\pgfpathlineto{\pgfqpoint{3.520221in}{0.100000in}}%
\pgfpathlineto{\pgfqpoint{3.520221in}{2.289500in}}%
\pgfpathlineto{\pgfqpoint{0.100000in}{2.289500in}}%
\pgfpathclose%
\pgfusepath{}%
\end{pgfscope}%
\begin{pgfscope}%
\pgfpathrectangle{\pgfqpoint{0.100000in}{0.100000in}}{\pgfqpoint{3.420221in}{2.189500in}}%
\pgfusepath{clip}%
\pgfsetbuttcap%
\pgfsetmiterjoin%
\definecolor{currentfill}{rgb}{0.000000,0.215686,0.892157}%
\pgfsetfillcolor{currentfill}%
\pgfsetlinewidth{0.000000pt}%
\definecolor{currentstroke}{rgb}{0.000000,0.000000,0.000000}%
\pgfsetstrokecolor{currentstroke}%
\pgfsetstrokeopacity{0.000000}%
\pgfsetdash{}{0pt}%
\pgfpathmoveto{\pgfqpoint{2.019009in}{1.436722in}}%
\pgfpathlineto{\pgfqpoint{1.993314in}{1.436796in}}%
\pgfpathlineto{\pgfqpoint{1.993414in}{1.462723in}}%
\pgfpathlineto{\pgfqpoint{1.974130in}{1.462854in}}%
\pgfpathlineto{\pgfqpoint{1.970816in}{1.470092in}}%
\pgfpathlineto{\pgfqpoint{1.970982in}{1.489681in}}%
\pgfpathlineto{\pgfqpoint{2.016364in}{1.489528in}}%
\pgfpathlineto{\pgfqpoint{2.016458in}{1.470009in}}%
\pgfpathlineto{\pgfqpoint{2.018894in}{1.470019in}}%
\pgfpathclose%
\pgfusepath{fill}%
\end{pgfscope}%
\begin{pgfscope}%
\pgfpathrectangle{\pgfqpoint{0.100000in}{0.100000in}}{\pgfqpoint{3.420221in}{2.189500in}}%
\pgfusepath{clip}%
\pgfsetbuttcap%
\pgfsetmiterjoin%
\definecolor{currentfill}{rgb}{0.000000,0.650980,0.674510}%
\pgfsetfillcolor{currentfill}%
\pgfsetlinewidth{0.000000pt}%
\definecolor{currentstroke}{rgb}{0.000000,0.000000,0.000000}%
\pgfsetstrokecolor{currentstroke}%
\pgfsetstrokeopacity{0.000000}%
\pgfsetdash{}{0pt}%
\pgfpathmoveto{\pgfqpoint{2.587935in}{1.145769in}}%
\pgfpathlineto{\pgfqpoint{2.593943in}{1.144379in}}%
\pgfpathlineto{\pgfqpoint{2.609417in}{1.158920in}}%
\pgfpathlineto{\pgfqpoint{2.604795in}{1.162160in}}%
\pgfpathlineto{\pgfqpoint{2.604501in}{1.167336in}}%
\pgfpathlineto{\pgfqpoint{2.614187in}{1.177617in}}%
\pgfpathlineto{\pgfqpoint{2.622914in}{1.187490in}}%
\pgfpathlineto{\pgfqpoint{2.626317in}{1.185821in}}%
\pgfpathlineto{\pgfqpoint{2.640541in}{1.174149in}}%
\pgfpathlineto{\pgfqpoint{2.639155in}{1.167640in}}%
\pgfpathlineto{\pgfqpoint{2.641750in}{1.156121in}}%
\pgfpathlineto{\pgfqpoint{2.639531in}{1.148355in}}%
\pgfpathlineto{\pgfqpoint{2.643863in}{1.138251in}}%
\pgfpathlineto{\pgfqpoint{2.649158in}{1.132929in}}%
\pgfpathlineto{\pgfqpoint{2.647967in}{1.127351in}}%
\pgfpathlineto{\pgfqpoint{2.643262in}{1.125226in}}%
\pgfpathlineto{\pgfqpoint{2.645727in}{1.109300in}}%
\pgfpathlineto{\pgfqpoint{2.641617in}{1.100004in}}%
\pgfpathlineto{\pgfqpoint{2.631801in}{1.104082in}}%
\pgfpathlineto{\pgfqpoint{2.622724in}{1.113145in}}%
\pgfpathlineto{\pgfqpoint{2.625016in}{1.115245in}}%
\pgfpathlineto{\pgfqpoint{2.619560in}{1.124392in}}%
\pgfpathlineto{\pgfqpoint{2.615694in}{1.129893in}}%
\pgfpathlineto{\pgfqpoint{2.586134in}{1.128384in}}%
\pgfpathclose%
\pgfusepath{fill}%
\end{pgfscope}%
\begin{pgfscope}%
\pgfpathrectangle{\pgfqpoint{0.100000in}{0.100000in}}{\pgfqpoint{3.420221in}{2.189500in}}%
\pgfusepath{clip}%
\pgfsetbuttcap%
\pgfsetmiterjoin%
\definecolor{currentfill}{rgb}{0.000000,0.494118,0.752941}%
\pgfsetfillcolor{currentfill}%
\pgfsetlinewidth{0.000000pt}%
\definecolor{currentstroke}{rgb}{0.000000,0.000000,0.000000}%
\pgfsetstrokecolor{currentstroke}%
\pgfsetstrokeopacity{0.000000}%
\pgfsetdash{}{0pt}%
\pgfpathmoveto{\pgfqpoint{1.659767in}{0.933914in}}%
\pgfpathlineto{\pgfqpoint{1.657499in}{0.898006in}}%
\pgfpathlineto{\pgfqpoint{1.624911in}{0.899929in}}%
\pgfpathlineto{\pgfqpoint{1.627283in}{0.935857in}}%
\pgfpathclose%
\pgfusepath{fill}%
\end{pgfscope}%
\begin{pgfscope}%
\pgfpathrectangle{\pgfqpoint{0.100000in}{0.100000in}}{\pgfqpoint{3.420221in}{2.189500in}}%
\pgfusepath{clip}%
\pgfsetbuttcap%
\pgfsetmiterjoin%
\definecolor{currentfill}{rgb}{0.000000,0.607843,0.696078}%
\pgfsetfillcolor{currentfill}%
\pgfsetlinewidth{0.000000pt}%
\definecolor{currentstroke}{rgb}{0.000000,0.000000,0.000000}%
\pgfsetstrokecolor{currentstroke}%
\pgfsetstrokeopacity{0.000000}%
\pgfsetdash{}{0pt}%
\pgfpathmoveto{\pgfqpoint{1.588148in}{1.236253in}}%
\pgfpathlineto{\pgfqpoint{1.588122in}{1.235779in}}%
\pgfpathlineto{\pgfqpoint{1.585215in}{1.196628in}}%
\pgfpathlineto{\pgfqpoint{1.584701in}{1.189583in}}%
\pgfpathlineto{\pgfqpoint{1.523533in}{1.194617in}}%
\pgfpathlineto{\pgfqpoint{1.465511in}{1.200150in}}%
\pgfpathlineto{\pgfqpoint{1.466119in}{1.206877in}}%
\pgfpathlineto{\pgfqpoint{1.432830in}{1.222847in}}%
\pgfpathlineto{\pgfqpoint{1.411079in}{1.223758in}}%
\pgfpathlineto{\pgfqpoint{1.409235in}{1.226541in}}%
\pgfpathlineto{\pgfqpoint{1.412107in}{1.252043in}}%
\pgfpathlineto{\pgfqpoint{1.418475in}{1.251362in}}%
\pgfpathlineto{\pgfqpoint{1.420544in}{1.270830in}}%
\pgfpathlineto{\pgfqpoint{1.472309in}{1.265613in}}%
\pgfpathlineto{\pgfqpoint{1.504310in}{1.262061in}}%
\pgfpathlineto{\pgfqpoint{1.502721in}{1.243292in}}%
\pgfpathlineto{\pgfqpoint{1.547228in}{1.239452in}}%
\pgfpathclose%
\pgfusepath{fill}%
\end{pgfscope}%
\begin{pgfscope}%
\pgfpathrectangle{\pgfqpoint{0.100000in}{0.100000in}}{\pgfqpoint{3.420221in}{2.189500in}}%
\pgfusepath{clip}%
\pgfsetbuttcap%
\pgfsetmiterjoin%
\definecolor{currentfill}{rgb}{0.000000,0.207843,0.896078}%
\pgfsetfillcolor{currentfill}%
\pgfsetlinewidth{0.000000pt}%
\definecolor{currentstroke}{rgb}{0.000000,0.000000,0.000000}%
\pgfsetstrokecolor{currentstroke}%
\pgfsetstrokeopacity{0.000000}%
\pgfsetdash{}{0pt}%
\pgfpathmoveto{\pgfqpoint{1.998833in}{1.273645in}}%
\pgfpathlineto{\pgfqpoint{1.992674in}{1.273579in}}%
\pgfpathlineto{\pgfqpoint{1.991648in}{1.284651in}}%
\pgfpathlineto{\pgfqpoint{1.991707in}{1.306381in}}%
\pgfpathlineto{\pgfqpoint{1.969091in}{1.306486in}}%
\pgfpathlineto{\pgfqpoint{1.969567in}{1.324003in}}%
\pgfpathlineto{\pgfqpoint{1.982509in}{1.323938in}}%
\pgfpathlineto{\pgfqpoint{1.995465in}{1.323874in}}%
\pgfpathlineto{\pgfqpoint{1.998937in}{1.321697in}}%
\pgfpathlineto{\pgfqpoint{1.996165in}{1.314943in}}%
\pgfpathlineto{\pgfqpoint{2.002832in}{1.308293in}}%
\pgfpathlineto{\pgfqpoint{2.011023in}{1.295104in}}%
\pgfpathlineto{\pgfqpoint{2.007892in}{1.290126in}}%
\pgfpathlineto{\pgfqpoint{2.007400in}{1.274311in}}%
\pgfpathclose%
\pgfusepath{fill}%
\end{pgfscope}%
\begin{pgfscope}%
\pgfpathrectangle{\pgfqpoint{0.100000in}{0.100000in}}{\pgfqpoint{3.420221in}{2.189500in}}%
\pgfusepath{clip}%
\pgfsetbuttcap%
\pgfsetmiterjoin%
\definecolor{currentfill}{rgb}{0.000000,0.384314,0.807843}%
\pgfsetfillcolor{currentfill}%
\pgfsetlinewidth{0.000000pt}%
\definecolor{currentstroke}{rgb}{0.000000,0.000000,0.000000}%
\pgfsetstrokecolor{currentstroke}%
\pgfsetstrokeopacity{0.000000}%
\pgfsetdash{}{0pt}%
\pgfpathmoveto{\pgfqpoint{2.621985in}{0.926685in}}%
\pgfpathlineto{\pgfqpoint{2.601137in}{0.924223in}}%
\pgfpathlineto{\pgfqpoint{2.599881in}{0.928810in}}%
\pgfpathlineto{\pgfqpoint{2.591844in}{0.927347in}}%
\pgfpathlineto{\pgfqpoint{2.578746in}{0.926936in}}%
\pgfpathlineto{\pgfqpoint{2.570155in}{0.946901in}}%
\pgfpathlineto{\pgfqpoint{2.574347in}{0.948455in}}%
\pgfpathlineto{\pgfqpoint{2.582016in}{0.957368in}}%
\pgfpathlineto{\pgfqpoint{2.584794in}{0.966559in}}%
\pgfpathlineto{\pgfqpoint{2.588368in}{0.969935in}}%
\pgfpathlineto{\pgfqpoint{2.587030in}{0.974669in}}%
\pgfpathlineto{\pgfqpoint{2.615751in}{0.977635in}}%
\pgfpathlineto{\pgfqpoint{2.615254in}{0.987618in}}%
\pgfpathlineto{\pgfqpoint{2.608008in}{0.986849in}}%
\pgfpathlineto{\pgfqpoint{2.609477in}{0.994521in}}%
\pgfpathlineto{\pgfqpoint{2.613775in}{0.994976in}}%
\pgfpathlineto{\pgfqpoint{2.617575in}{1.008104in}}%
\pgfpathlineto{\pgfqpoint{2.639305in}{1.010604in}}%
\pgfpathlineto{\pgfqpoint{2.640350in}{1.000805in}}%
\pgfpathlineto{\pgfqpoint{2.642938in}{0.998657in}}%
\pgfpathlineto{\pgfqpoint{2.639341in}{0.990993in}}%
\pgfpathlineto{\pgfqpoint{2.641045in}{0.977578in}}%
\pgfpathlineto{\pgfqpoint{2.644765in}{0.942370in}}%
\pgfpathlineto{\pgfqpoint{2.628419in}{0.940866in}}%
\pgfpathlineto{\pgfqpoint{2.626042in}{0.930682in}}%
\pgfpathclose%
\pgfusepath{fill}%
\end{pgfscope}%
\begin{pgfscope}%
\pgfpathrectangle{\pgfqpoint{0.100000in}{0.100000in}}{\pgfqpoint{3.420221in}{2.189500in}}%
\pgfusepath{clip}%
\pgfsetbuttcap%
\pgfsetmiterjoin%
\definecolor{currentfill}{rgb}{0.000000,0.345098,0.827451}%
\pgfsetfillcolor{currentfill}%
\pgfsetlinewidth{0.000000pt}%
\definecolor{currentstroke}{rgb}{0.000000,0.000000,0.000000}%
\pgfsetstrokecolor{currentstroke}%
\pgfsetstrokeopacity{0.000000}%
\pgfsetdash{}{0pt}%
\pgfpathmoveto{\pgfqpoint{1.632361in}{1.775143in}}%
\pgfpathlineto{\pgfqpoint{1.629845in}{1.742526in}}%
\pgfpathlineto{\pgfqpoint{1.579054in}{1.746777in}}%
\pgfpathlineto{\pgfqpoint{1.580186in}{1.759793in}}%
\pgfpathlineto{\pgfqpoint{1.585836in}{1.814741in}}%
\pgfpathlineto{\pgfqpoint{1.583050in}{1.814998in}}%
\pgfpathlineto{\pgfqpoint{1.585270in}{1.840235in}}%
\pgfpathlineto{\pgfqpoint{1.588963in}{1.839973in}}%
\pgfpathlineto{\pgfqpoint{1.591078in}{1.866225in}}%
\pgfpathlineto{\pgfqpoint{1.634094in}{1.862737in}}%
\pgfpathlineto{\pgfqpoint{1.636456in}{1.862539in}}%
\pgfpathlineto{\pgfqpoint{1.634646in}{1.836364in}}%
\pgfpathlineto{\pgfqpoint{1.637187in}{1.836166in}}%
\pgfpathlineto{\pgfqpoint{1.635812in}{1.818838in}}%
\pgfpathclose%
\pgfusepath{fill}%
\end{pgfscope}%
\begin{pgfscope}%
\pgfpathrectangle{\pgfqpoint{0.100000in}{0.100000in}}{\pgfqpoint{3.420221in}{2.189500in}}%
\pgfusepath{clip}%
\pgfsetbuttcap%
\pgfsetmiterjoin%
\definecolor{currentfill}{rgb}{0.000000,0.478431,0.760784}%
\pgfsetfillcolor{currentfill}%
\pgfsetlinewidth{0.000000pt}%
\definecolor{currentstroke}{rgb}{0.000000,0.000000,0.000000}%
\pgfsetstrokecolor{currentstroke}%
\pgfsetstrokeopacity{0.000000}%
\pgfsetdash{}{0pt}%
\pgfpathmoveto{\pgfqpoint{3.131509in}{1.316108in}}%
\pgfpathlineto{\pgfqpoint{3.137566in}{1.321122in}}%
\pgfpathlineto{\pgfqpoint{3.159060in}{1.328324in}}%
\pgfpathlineto{\pgfqpoint{3.154404in}{1.314416in}}%
\pgfpathlineto{\pgfqpoint{3.150770in}{1.314874in}}%
\pgfpathlineto{\pgfqpoint{3.146794in}{1.307777in}}%
\pgfpathlineto{\pgfqpoint{3.144528in}{1.292850in}}%
\pgfpathlineto{\pgfqpoint{3.146048in}{1.290273in}}%
\pgfpathlineto{\pgfqpoint{3.141671in}{1.271731in}}%
\pgfpathlineto{\pgfqpoint{3.139934in}{1.260651in}}%
\pgfpathlineto{\pgfqpoint{3.135470in}{1.253447in}}%
\pgfpathlineto{\pgfqpoint{3.131221in}{1.252495in}}%
\pgfpathlineto{\pgfqpoint{3.125803in}{1.262105in}}%
\pgfpathlineto{\pgfqpoint{3.125960in}{1.285546in}}%
\pgfpathlineto{\pgfqpoint{3.129268in}{1.295290in}}%
\pgfpathlineto{\pgfqpoint{3.129498in}{1.303845in}}%
\pgfpathlineto{\pgfqpoint{3.134795in}{1.305417in}}%
\pgfpathlineto{\pgfqpoint{3.135969in}{1.312043in}}%
\pgfpathclose%
\pgfusepath{fill}%
\end{pgfscope}%
\begin{pgfscope}%
\pgfpathrectangle{\pgfqpoint{0.100000in}{0.100000in}}{\pgfqpoint{3.420221in}{2.189500in}}%
\pgfusepath{clip}%
\pgfsetbuttcap%
\pgfsetmiterjoin%
\definecolor{currentfill}{rgb}{0.000000,0.305882,0.847059}%
\pgfsetfillcolor{currentfill}%
\pgfsetlinewidth{0.000000pt}%
\definecolor{currentstroke}{rgb}{0.000000,0.000000,0.000000}%
\pgfsetstrokecolor{currentstroke}%
\pgfsetstrokeopacity{0.000000}%
\pgfsetdash{}{0pt}%
\pgfpathmoveto{\pgfqpoint{2.363892in}{1.707475in}}%
\pgfpathlineto{\pgfqpoint{2.344565in}{1.706507in}}%
\pgfpathlineto{\pgfqpoint{2.344150in}{1.713105in}}%
\pgfpathlineto{\pgfqpoint{2.311095in}{1.711121in}}%
\pgfpathlineto{\pgfqpoint{2.308979in}{1.743978in}}%
\pgfpathlineto{\pgfqpoint{2.298264in}{1.743355in}}%
\pgfpathlineto{\pgfqpoint{2.296723in}{1.769546in}}%
\pgfpathlineto{\pgfqpoint{2.323211in}{1.771085in}}%
\pgfpathlineto{\pgfqpoint{2.323491in}{1.764568in}}%
\pgfpathlineto{\pgfqpoint{2.336602in}{1.765467in}}%
\pgfpathlineto{\pgfqpoint{2.337393in}{1.752399in}}%
\pgfpathlineto{\pgfqpoint{2.339916in}{1.745954in}}%
\pgfpathlineto{\pgfqpoint{2.348267in}{1.746512in}}%
\pgfpathlineto{\pgfqpoint{2.349391in}{1.726840in}}%
\pgfpathlineto{\pgfqpoint{2.362220in}{1.727568in}}%
\pgfpathclose%
\pgfusepath{fill}%
\end{pgfscope}%
\begin{pgfscope}%
\pgfpathrectangle{\pgfqpoint{0.100000in}{0.100000in}}{\pgfqpoint{3.420221in}{2.189500in}}%
\pgfusepath{clip}%
\pgfsetbuttcap%
\pgfsetmiterjoin%
\definecolor{currentfill}{rgb}{0.000000,0.482353,0.758824}%
\pgfsetfillcolor{currentfill}%
\pgfsetlinewidth{0.000000pt}%
\definecolor{currentstroke}{rgb}{0.000000,0.000000,0.000000}%
\pgfsetstrokecolor{currentstroke}%
\pgfsetstrokeopacity{0.000000}%
\pgfsetdash{}{0pt}%
\pgfpathmoveto{\pgfqpoint{3.212962in}{1.649734in}}%
\pgfpathlineto{\pgfqpoint{3.189295in}{1.644720in}}%
\pgfpathlineto{\pgfqpoint{3.188777in}{1.644598in}}%
\pgfpathlineto{\pgfqpoint{3.187524in}{1.647119in}}%
\pgfpathlineto{\pgfqpoint{3.188954in}{1.698406in}}%
\pgfpathlineto{\pgfqpoint{3.186610in}{1.704915in}}%
\pgfpathlineto{\pgfqpoint{3.179384in}{1.740036in}}%
\pgfpathlineto{\pgfqpoint{3.200162in}{1.744073in}}%
\pgfpathlineto{\pgfqpoint{3.203559in}{1.741416in}}%
\pgfpathlineto{\pgfqpoint{3.203515in}{1.730389in}}%
\pgfpathlineto{\pgfqpoint{3.196586in}{1.729131in}}%
\pgfpathlineto{\pgfqpoint{3.199395in}{1.715204in}}%
\pgfpathlineto{\pgfqpoint{3.204082in}{1.716108in}}%
\pgfpathlineto{\pgfqpoint{3.206993in}{1.702326in}}%
\pgfpathlineto{\pgfqpoint{3.202662in}{1.698418in}}%
\pgfpathlineto{\pgfqpoint{3.207638in}{1.694932in}}%
\pgfpathlineto{\pgfqpoint{3.209099in}{1.675049in}}%
\pgfpathclose%
\pgfusepath{fill}%
\end{pgfscope}%
\begin{pgfscope}%
\pgfpathrectangle{\pgfqpoint{0.100000in}{0.100000in}}{\pgfqpoint{3.420221in}{2.189500in}}%
\pgfusepath{clip}%
\pgfsetbuttcap%
\pgfsetmiterjoin%
\definecolor{currentfill}{rgb}{0.000000,0.741176,0.629412}%
\pgfsetfillcolor{currentfill}%
\pgfsetlinewidth{0.000000pt}%
\definecolor{currentstroke}{rgb}{0.000000,0.000000,0.000000}%
\pgfsetstrokecolor{currentstroke}%
\pgfsetstrokeopacity{0.000000}%
\pgfsetdash{}{0pt}%
\pgfpathmoveto{\pgfqpoint{2.749742in}{1.248131in}}%
\pgfpathlineto{\pgfqpoint{2.744881in}{1.241341in}}%
\pgfpathlineto{\pgfqpoint{2.738831in}{1.245353in}}%
\pgfpathlineto{\pgfqpoint{2.740080in}{1.249814in}}%
\pgfpathlineto{\pgfqpoint{2.728388in}{1.259951in}}%
\pgfpathlineto{\pgfqpoint{2.729310in}{1.268475in}}%
\pgfpathlineto{\pgfqpoint{2.717713in}{1.266128in}}%
\pgfpathlineto{\pgfqpoint{2.711546in}{1.260678in}}%
\pgfpathlineto{\pgfqpoint{2.713971in}{1.255931in}}%
\pgfpathlineto{\pgfqpoint{2.707210in}{1.247120in}}%
\pgfpathlineto{\pgfqpoint{2.697141in}{1.246059in}}%
\pgfpathlineto{\pgfqpoint{2.692313in}{1.249102in}}%
\pgfpathlineto{\pgfqpoint{2.682851in}{1.245988in}}%
\pgfpathlineto{\pgfqpoint{2.669625in}{1.255899in}}%
\pgfpathlineto{\pgfqpoint{2.676259in}{1.263201in}}%
\pgfpathlineto{\pgfqpoint{2.676031in}{1.268899in}}%
\pgfpathlineto{\pgfqpoint{2.680403in}{1.273026in}}%
\pgfpathlineto{\pgfqpoint{2.684626in}{1.267955in}}%
\pgfpathlineto{\pgfqpoint{2.691790in}{1.270218in}}%
\pgfpathlineto{\pgfqpoint{2.690654in}{1.275872in}}%
\pgfpathlineto{\pgfqpoint{2.692976in}{1.282205in}}%
\pgfpathlineto{\pgfqpoint{2.698592in}{1.290419in}}%
\pgfpathlineto{\pgfqpoint{2.699633in}{1.300420in}}%
\pgfpathlineto{\pgfqpoint{2.707242in}{1.303776in}}%
\pgfpathlineto{\pgfqpoint{2.711161in}{1.293000in}}%
\pgfpathlineto{\pgfqpoint{2.717753in}{1.293143in}}%
\pgfpathlineto{\pgfqpoint{2.720720in}{1.304949in}}%
\pgfpathlineto{\pgfqpoint{2.720435in}{1.312555in}}%
\pgfpathlineto{\pgfqpoint{2.724789in}{1.312804in}}%
\pgfpathlineto{\pgfqpoint{2.725045in}{1.307894in}}%
\pgfpathlineto{\pgfqpoint{2.730849in}{1.308181in}}%
\pgfpathlineto{\pgfqpoint{2.732417in}{1.301591in}}%
\pgfpathlineto{\pgfqpoint{2.739317in}{1.302083in}}%
\pgfpathlineto{\pgfqpoint{2.739844in}{1.295281in}}%
\pgfpathlineto{\pgfqpoint{2.748051in}{1.296928in}}%
\pgfpathlineto{\pgfqpoint{2.758795in}{1.289595in}}%
\pgfpathlineto{\pgfqpoint{2.760325in}{1.282231in}}%
\pgfpathlineto{\pgfqpoint{2.752165in}{1.277163in}}%
\pgfpathlineto{\pgfqpoint{2.748674in}{1.267808in}}%
\pgfpathlineto{\pgfqpoint{2.748849in}{1.263177in}}%
\pgfpathlineto{\pgfqpoint{2.754691in}{1.255281in}}%
\pgfpathclose%
\pgfusepath{fill}%
\end{pgfscope}%
\begin{pgfscope}%
\pgfpathrectangle{\pgfqpoint{0.100000in}{0.100000in}}{\pgfqpoint{3.420221in}{2.189500in}}%
\pgfusepath{clip}%
\pgfsetbuttcap%
\pgfsetmiterjoin%
\definecolor{currentfill}{rgb}{0.000000,0.560784,0.719608}%
\pgfsetfillcolor{currentfill}%
\pgfsetlinewidth{0.000000pt}%
\definecolor{currentstroke}{rgb}{0.000000,0.000000,0.000000}%
\pgfsetstrokecolor{currentstroke}%
\pgfsetstrokeopacity{0.000000}%
\pgfsetdash{}{0pt}%
\pgfpathmoveto{\pgfqpoint{1.344853in}{1.016966in}}%
\pgfpathlineto{\pgfqpoint{1.312520in}{1.020777in}}%
\pgfpathlineto{\pgfqpoint{1.318889in}{1.074107in}}%
\pgfpathlineto{\pgfqpoint{1.310426in}{1.080623in}}%
\pgfpathlineto{\pgfqpoint{1.311663in}{1.090483in}}%
\pgfpathlineto{\pgfqpoint{1.321229in}{1.092420in}}%
\pgfpathlineto{\pgfqpoint{1.282794in}{1.097145in}}%
\pgfpathlineto{\pgfqpoint{1.284940in}{1.113328in}}%
\pgfpathlineto{\pgfqpoint{1.240643in}{1.119582in}}%
\pgfpathlineto{\pgfqpoint{1.247068in}{1.162977in}}%
\pgfpathlineto{\pgfqpoint{1.258466in}{1.169768in}}%
\pgfpathlineto{\pgfqpoint{1.260802in}{1.175852in}}%
\pgfpathlineto{\pgfqpoint{1.316812in}{1.167972in}}%
\pgfpathlineto{\pgfqpoint{1.391467in}{1.159123in}}%
\pgfpathlineto{\pgfqpoint{1.391399in}{1.153256in}}%
\pgfpathlineto{\pgfqpoint{1.387357in}{1.138627in}}%
\pgfpathlineto{\pgfqpoint{1.379836in}{1.135903in}}%
\pgfpathlineto{\pgfqpoint{1.380097in}{1.116128in}}%
\pgfpathlineto{\pgfqpoint{1.378071in}{1.105333in}}%
\pgfpathlineto{\pgfqpoint{1.375200in}{1.104952in}}%
\pgfpathlineto{\pgfqpoint{1.372199in}{1.096151in}}%
\pgfpathlineto{\pgfqpoint{1.361641in}{1.085349in}}%
\pgfpathlineto{\pgfqpoint{1.352954in}{1.086847in}}%
\pgfpathclose%
\pgfusepath{fill}%
\end{pgfscope}%
\begin{pgfscope}%
\pgfpathrectangle{\pgfqpoint{0.100000in}{0.100000in}}{\pgfqpoint{3.420221in}{2.189500in}}%
\pgfusepath{clip}%
\pgfsetbuttcap%
\pgfsetmiterjoin%
\definecolor{currentfill}{rgb}{0.000000,0.486275,0.756863}%
\pgfsetfillcolor{currentfill}%
\pgfsetlinewidth{0.000000pt}%
\definecolor{currentstroke}{rgb}{0.000000,0.000000,0.000000}%
\pgfsetstrokecolor{currentstroke}%
\pgfsetstrokeopacity{0.000000}%
\pgfsetdash{}{0pt}%
\pgfpathmoveto{\pgfqpoint{2.287048in}{1.156221in}}%
\pgfpathlineto{\pgfqpoint{2.267269in}{1.155412in}}%
\pgfpathlineto{\pgfqpoint{2.267382in}{1.152042in}}%
\pgfpathlineto{\pgfqpoint{2.256375in}{1.151396in}}%
\pgfpathlineto{\pgfqpoint{2.253783in}{1.158975in}}%
\pgfpathlineto{\pgfqpoint{2.253236in}{1.176157in}}%
\pgfpathlineto{\pgfqpoint{2.240282in}{1.175633in}}%
\pgfpathlineto{\pgfqpoint{2.230366in}{1.174021in}}%
\pgfpathlineto{\pgfqpoint{2.230123in}{1.182205in}}%
\pgfpathlineto{\pgfqpoint{2.233145in}{1.185535in}}%
\pgfpathlineto{\pgfqpoint{2.231903in}{1.220287in}}%
\pgfpathlineto{\pgfqpoint{2.250464in}{1.221119in}}%
\pgfpathlineto{\pgfqpoint{2.262772in}{1.206656in}}%
\pgfpathlineto{\pgfqpoint{2.272492in}{1.210049in}}%
\pgfpathlineto{\pgfqpoint{2.277406in}{1.214212in}}%
\pgfpathlineto{\pgfqpoint{2.284647in}{1.214102in}}%
\pgfpathlineto{\pgfqpoint{2.289220in}{1.212423in}}%
\pgfpathlineto{\pgfqpoint{2.297320in}{1.205526in}}%
\pgfpathlineto{\pgfqpoint{2.300424in}{1.199656in}}%
\pgfpathlineto{\pgfqpoint{2.306859in}{1.201604in}}%
\pgfpathlineto{\pgfqpoint{2.313178in}{1.197594in}}%
\pgfpathlineto{\pgfqpoint{2.323023in}{1.188139in}}%
\pgfpathlineto{\pgfqpoint{2.327215in}{1.186638in}}%
\pgfpathlineto{\pgfqpoint{2.329751in}{1.179956in}}%
\pgfpathlineto{\pgfqpoint{2.327323in}{1.177357in}}%
\pgfpathlineto{\pgfqpoint{2.322061in}{1.179420in}}%
\pgfpathlineto{\pgfqpoint{2.290138in}{1.177584in}}%
\pgfpathlineto{\pgfqpoint{2.291343in}{1.156233in}}%
\pgfpathclose%
\pgfusepath{fill}%
\end{pgfscope}%
\begin{pgfscope}%
\pgfpathrectangle{\pgfqpoint{0.100000in}{0.100000in}}{\pgfqpoint{3.420221in}{2.189500in}}%
\pgfusepath{clip}%
\pgfsetbuttcap%
\pgfsetmiterjoin%
\definecolor{currentfill}{rgb}{0.000000,0.384314,0.807843}%
\pgfsetfillcolor{currentfill}%
\pgfsetlinewidth{0.000000pt}%
\definecolor{currentstroke}{rgb}{0.000000,0.000000,0.000000}%
\pgfsetstrokecolor{currentstroke}%
\pgfsetstrokeopacity{0.000000}%
\pgfsetdash{}{0pt}%
\pgfpathmoveto{\pgfqpoint{2.613593in}{0.792591in}}%
\pgfpathlineto{\pgfqpoint{2.613479in}{0.782585in}}%
\pgfpathlineto{\pgfqpoint{2.603506in}{0.780204in}}%
\pgfpathlineto{\pgfqpoint{2.599935in}{0.770949in}}%
\pgfpathlineto{\pgfqpoint{2.591429in}{0.770084in}}%
\pgfpathlineto{\pgfqpoint{2.590829in}{0.776617in}}%
\pgfpathlineto{\pgfqpoint{2.584821in}{0.776009in}}%
\pgfpathlineto{\pgfqpoint{2.583624in}{0.781965in}}%
\pgfpathlineto{\pgfqpoint{2.577028in}{0.781586in}}%
\pgfpathlineto{\pgfqpoint{2.575447in}{0.802931in}}%
\pgfpathlineto{\pgfqpoint{2.572867in}{0.808389in}}%
\pgfpathlineto{\pgfqpoint{2.580736in}{0.815466in}}%
\pgfpathlineto{\pgfqpoint{2.586357in}{0.816124in}}%
\pgfpathlineto{\pgfqpoint{2.585596in}{0.822624in}}%
\pgfpathlineto{\pgfqpoint{2.591982in}{0.824406in}}%
\pgfpathlineto{\pgfqpoint{2.591144in}{0.832011in}}%
\pgfpathlineto{\pgfqpoint{2.597438in}{0.835018in}}%
\pgfpathlineto{\pgfqpoint{2.616847in}{0.837234in}}%
\pgfpathlineto{\pgfqpoint{2.626172in}{0.839440in}}%
\pgfpathlineto{\pgfqpoint{2.628888in}{0.831927in}}%
\pgfpathlineto{\pgfqpoint{2.636568in}{0.822680in}}%
\pgfpathlineto{\pgfqpoint{2.633197in}{0.819458in}}%
\pgfpathlineto{\pgfqpoint{2.615868in}{0.817258in}}%
\pgfpathlineto{\pgfqpoint{2.616565in}{0.812949in}}%
\pgfpathlineto{\pgfqpoint{2.609995in}{0.812175in}}%
\pgfpathlineto{\pgfqpoint{2.611415in}{0.799037in}}%
\pgfpathclose%
\pgfusepath{fill}%
\end{pgfscope}%
\begin{pgfscope}%
\pgfpathrectangle{\pgfqpoint{0.100000in}{0.100000in}}{\pgfqpoint{3.420221in}{2.189500in}}%
\pgfusepath{clip}%
\pgfsetbuttcap%
\pgfsetmiterjoin%
\definecolor{currentfill}{rgb}{0.000000,0.415686,0.792157}%
\pgfsetfillcolor{currentfill}%
\pgfsetlinewidth{0.000000pt}%
\definecolor{currentstroke}{rgb}{0.000000,0.000000,0.000000}%
\pgfsetstrokecolor{currentstroke}%
\pgfsetstrokeopacity{0.000000}%
\pgfsetdash{}{0pt}%
\pgfpathmoveto{\pgfqpoint{0.929478in}{1.763712in}}%
\pgfpathlineto{\pgfqpoint{0.928311in}{1.758929in}}%
\pgfpathlineto{\pgfqpoint{0.922541in}{1.750344in}}%
\pgfpathlineto{\pgfqpoint{0.923535in}{1.746496in}}%
\pgfpathlineto{\pgfqpoint{0.918068in}{1.740546in}}%
\pgfpathlineto{\pgfqpoint{0.906510in}{1.690032in}}%
\pgfpathlineto{\pgfqpoint{0.909131in}{1.689261in}}%
\pgfpathlineto{\pgfqpoint{0.894039in}{1.622190in}}%
\pgfpathlineto{\pgfqpoint{0.821804in}{1.638957in}}%
\pgfpathlineto{\pgfqpoint{0.786661in}{1.647747in}}%
\pgfpathlineto{\pgfqpoint{0.786248in}{1.647919in}}%
\pgfpathlineto{\pgfqpoint{0.816517in}{1.770421in}}%
\pgfpathlineto{\pgfqpoint{0.820584in}{1.783477in}}%
\pgfpathlineto{\pgfqpoint{0.823895in}{1.784008in}}%
\pgfpathlineto{\pgfqpoint{0.827665in}{1.776326in}}%
\pgfpathlineto{\pgfqpoint{0.835327in}{1.775551in}}%
\pgfpathlineto{\pgfqpoint{0.838434in}{1.788296in}}%
\pgfpathlineto{\pgfqpoint{0.845798in}{1.786437in}}%
\pgfpathlineto{\pgfqpoint{0.849419in}{1.792194in}}%
\pgfpathlineto{\pgfqpoint{0.853589in}{1.791130in}}%
\pgfpathlineto{\pgfqpoint{0.855131in}{1.797421in}}%
\pgfpathlineto{\pgfqpoint{0.860031in}{1.796218in}}%
\pgfpathlineto{\pgfqpoint{0.863583in}{1.808624in}}%
\pgfpathlineto{\pgfqpoint{0.868236in}{1.816925in}}%
\pgfpathlineto{\pgfqpoint{0.876888in}{1.819108in}}%
\pgfpathlineto{\pgfqpoint{0.873865in}{1.806326in}}%
\pgfpathlineto{\pgfqpoint{0.870732in}{1.807061in}}%
\pgfpathlineto{\pgfqpoint{0.867721in}{1.794406in}}%
\pgfpathlineto{\pgfqpoint{0.872917in}{1.792904in}}%
\pgfpathlineto{\pgfqpoint{0.874935in}{1.799287in}}%
\pgfpathlineto{\pgfqpoint{0.905496in}{1.792053in}}%
\pgfpathlineto{\pgfqpoint{0.908912in}{1.791859in}}%
\pgfpathlineto{\pgfqpoint{0.919091in}{1.796990in}}%
\pgfpathlineto{\pgfqpoint{0.924441in}{1.792459in}}%
\pgfpathlineto{\pgfqpoint{0.923511in}{1.785445in}}%
\pgfpathlineto{\pgfqpoint{0.930196in}{1.780527in}}%
\pgfpathlineto{\pgfqpoint{0.927447in}{1.773268in}}%
\pgfpathclose%
\pgfusepath{fill}%
\end{pgfscope}%
\begin{pgfscope}%
\pgfpathrectangle{\pgfqpoint{0.100000in}{0.100000in}}{\pgfqpoint{3.420221in}{2.189500in}}%
\pgfusepath{clip}%
\pgfsetbuttcap%
\pgfsetmiterjoin%
\definecolor{currentfill}{rgb}{0.000000,0.631373,0.684314}%
\pgfsetfillcolor{currentfill}%
\pgfsetlinewidth{0.000000pt}%
\definecolor{currentstroke}{rgb}{0.000000,0.000000,0.000000}%
\pgfsetstrokecolor{currentstroke}%
\pgfsetstrokeopacity{0.000000}%
\pgfsetdash{}{0pt}%
\pgfpathmoveto{\pgfqpoint{2.159436in}{0.568114in}}%
\pgfpathlineto{\pgfqpoint{2.151249in}{0.570590in}}%
\pgfpathlineto{\pgfqpoint{2.134095in}{0.578201in}}%
\pgfpathlineto{\pgfqpoint{2.120910in}{0.581963in}}%
\pgfpathlineto{\pgfqpoint{2.110938in}{0.580582in}}%
\pgfpathlineto{\pgfqpoint{2.100871in}{0.581136in}}%
\pgfpathlineto{\pgfqpoint{2.083390in}{0.577960in}}%
\pgfpathlineto{\pgfqpoint{2.078696in}{0.574988in}}%
\pgfpathlineto{\pgfqpoint{2.072820in}{0.584301in}}%
\pgfpathlineto{\pgfqpoint{2.079746in}{0.591980in}}%
\pgfpathlineto{\pgfqpoint{2.081616in}{0.597541in}}%
\pgfpathlineto{\pgfqpoint{2.086794in}{0.602613in}}%
\pgfpathlineto{\pgfqpoint{2.086854in}{0.620494in}}%
\pgfpathlineto{\pgfqpoint{2.082811in}{0.623705in}}%
\pgfpathlineto{\pgfqpoint{2.087142in}{0.631936in}}%
\pgfpathlineto{\pgfqpoint{2.084698in}{0.638772in}}%
\pgfpathlineto{\pgfqpoint{2.087882in}{0.647107in}}%
\pgfpathlineto{\pgfqpoint{2.091203in}{0.649972in}}%
\pgfpathlineto{\pgfqpoint{2.094929in}{0.664959in}}%
\pgfpathlineto{\pgfqpoint{2.097749in}{0.669154in}}%
\pgfpathlineto{\pgfqpoint{2.095179in}{0.673629in}}%
\pgfpathlineto{\pgfqpoint{2.095722in}{0.688451in}}%
\pgfpathlineto{\pgfqpoint{2.096726in}{0.695335in}}%
\pgfpathlineto{\pgfqpoint{2.102724in}{0.695459in}}%
\pgfpathlineto{\pgfqpoint{2.105970in}{0.702125in}}%
\pgfpathlineto{\pgfqpoint{2.115931in}{0.702337in}}%
\pgfpathlineto{\pgfqpoint{2.129080in}{0.702592in}}%
\pgfpathlineto{\pgfqpoint{2.132390in}{0.701330in}}%
\pgfpathlineto{\pgfqpoint{2.139005in}{0.699267in}}%
\pgfpathlineto{\pgfqpoint{2.142356in}{0.694424in}}%
\pgfpathlineto{\pgfqpoint{2.143401in}{0.667041in}}%
\pgfpathlineto{\pgfqpoint{2.158051in}{0.667872in}}%
\pgfpathlineto{\pgfqpoint{2.158706in}{0.644078in}}%
\pgfpathlineto{\pgfqpoint{2.156522in}{0.636574in}}%
\pgfpathlineto{\pgfqpoint{2.158596in}{0.602883in}}%
\pgfpathclose%
\pgfusepath{fill}%
\end{pgfscope}%
\begin{pgfscope}%
\pgfpathrectangle{\pgfqpoint{0.100000in}{0.100000in}}{\pgfqpoint{3.420221in}{2.189500in}}%
\pgfusepath{clip}%
\pgfsetbuttcap%
\pgfsetmiterjoin%
\definecolor{currentfill}{rgb}{0.000000,0.741176,0.629412}%
\pgfsetfillcolor{currentfill}%
\pgfsetlinewidth{0.000000pt}%
\definecolor{currentstroke}{rgb}{0.000000,0.000000,0.000000}%
\pgfsetstrokecolor{currentstroke}%
\pgfsetstrokeopacity{0.000000}%
\pgfsetdash{}{0pt}%
\pgfpathmoveto{\pgfqpoint{0.756481in}{1.177901in}}%
\pgfpathlineto{\pgfqpoint{0.750963in}{1.177996in}}%
\pgfpathlineto{\pgfqpoint{0.731142in}{1.182626in}}%
\pgfpathlineto{\pgfqpoint{0.661344in}{1.199703in}}%
\pgfpathlineto{\pgfqpoint{0.638838in}{1.205551in}}%
\pgfpathlineto{\pgfqpoint{0.616577in}{1.210635in}}%
\pgfpathlineto{\pgfqpoint{0.620089in}{1.217906in}}%
\pgfpathlineto{\pgfqpoint{0.618817in}{1.233638in}}%
\pgfpathlineto{\pgfqpoint{0.620731in}{1.239086in}}%
\pgfpathlineto{\pgfqpoint{0.619220in}{1.249559in}}%
\pgfpathlineto{\pgfqpoint{0.621647in}{1.252486in}}%
\pgfpathlineto{\pgfqpoint{0.615663in}{1.264966in}}%
\pgfpathlineto{\pgfqpoint{0.616499in}{1.269019in}}%
\pgfpathlineto{\pgfqpoint{0.613414in}{1.279921in}}%
\pgfpathlineto{\pgfqpoint{0.614053in}{1.292322in}}%
\pgfpathlineto{\pgfqpoint{0.616746in}{1.295743in}}%
\pgfpathlineto{\pgfqpoint{0.615093in}{1.307905in}}%
\pgfpathlineto{\pgfqpoint{0.608126in}{1.317437in}}%
\pgfpathlineto{\pgfqpoint{0.604621in}{1.318655in}}%
\pgfpathlineto{\pgfqpoint{0.604679in}{1.333098in}}%
\pgfpathlineto{\pgfqpoint{0.601118in}{1.334568in}}%
\pgfpathlineto{\pgfqpoint{0.604403in}{1.341943in}}%
\pgfpathlineto{\pgfqpoint{0.599539in}{1.346128in}}%
\pgfpathlineto{\pgfqpoint{0.597437in}{1.352277in}}%
\pgfpathlineto{\pgfqpoint{0.592090in}{1.356469in}}%
\pgfpathlineto{\pgfqpoint{0.591576in}{1.364281in}}%
\pgfpathlineto{\pgfqpoint{0.589162in}{1.368771in}}%
\pgfpathlineto{\pgfqpoint{0.580995in}{1.371409in}}%
\pgfpathlineto{\pgfqpoint{0.585849in}{1.374410in}}%
\pgfpathlineto{\pgfqpoint{0.586345in}{1.383239in}}%
\pgfpathlineto{\pgfqpoint{0.582854in}{1.387029in}}%
\pgfpathlineto{\pgfqpoint{0.583420in}{1.397463in}}%
\pgfpathlineto{\pgfqpoint{0.571803in}{1.406463in}}%
\pgfpathlineto{\pgfqpoint{0.570130in}{1.416943in}}%
\pgfpathlineto{\pgfqpoint{0.571639in}{1.419667in}}%
\pgfpathlineto{\pgfqpoint{0.577123in}{1.424141in}}%
\pgfpathlineto{\pgfqpoint{0.580549in}{1.430631in}}%
\pgfpathlineto{\pgfqpoint{0.578326in}{1.439321in}}%
\pgfpathlineto{\pgfqpoint{0.582602in}{1.446075in}}%
\pgfpathlineto{\pgfqpoint{0.593367in}{1.429827in}}%
\pgfpathlineto{\pgfqpoint{0.600695in}{1.418501in}}%
\pgfpathlineto{\pgfqpoint{0.632115in}{1.369859in}}%
\pgfpathlineto{\pgfqpoint{0.658152in}{1.329689in}}%
\pgfpathlineto{\pgfqpoint{0.687663in}{1.284051in}}%
\pgfpathlineto{\pgfqpoint{0.744934in}{1.195624in}}%
\pgfpathclose%
\pgfusepath{fill}%
\end{pgfscope}%
\begin{pgfscope}%
\pgfpathrectangle{\pgfqpoint{0.100000in}{0.100000in}}{\pgfqpoint{3.420221in}{2.189500in}}%
\pgfusepath{clip}%
\pgfsetbuttcap%
\pgfsetmiterjoin%
\definecolor{currentfill}{rgb}{0.000000,0.556863,0.721569}%
\pgfsetfillcolor{currentfill}%
\pgfsetlinewidth{0.000000pt}%
\definecolor{currentstroke}{rgb}{0.000000,0.000000,0.000000}%
\pgfsetstrokecolor{currentstroke}%
\pgfsetstrokeopacity{0.000000}%
\pgfsetdash{}{0pt}%
\pgfpathmoveto{\pgfqpoint{2.763856in}{0.759120in}}%
\pgfpathlineto{\pgfqpoint{2.749007in}{0.757925in}}%
\pgfpathlineto{\pgfqpoint{2.745533in}{0.762949in}}%
\pgfpathlineto{\pgfqpoint{2.740229in}{0.763533in}}%
\pgfpathlineto{\pgfqpoint{2.739686in}{0.773053in}}%
\pgfpathlineto{\pgfqpoint{2.741977in}{0.774235in}}%
\pgfpathlineto{\pgfqpoint{2.738914in}{0.784466in}}%
\pgfpathlineto{\pgfqpoint{2.758147in}{0.787000in}}%
\pgfpathlineto{\pgfqpoint{2.770382in}{0.783477in}}%
\pgfpathlineto{\pgfqpoint{2.771381in}{0.774463in}}%
\pgfpathlineto{\pgfqpoint{2.768039in}{0.770276in}}%
\pgfpathlineto{\pgfqpoint{2.766010in}{0.759858in}}%
\pgfpathclose%
\pgfusepath{fill}%
\end{pgfscope}%
\begin{pgfscope}%
\pgfpathrectangle{\pgfqpoint{0.100000in}{0.100000in}}{\pgfqpoint{3.420221in}{2.189500in}}%
\pgfusepath{clip}%
\pgfsetbuttcap%
\pgfsetmiterjoin%
\definecolor{currentfill}{rgb}{0.000000,0.788235,0.605882}%
\pgfsetfillcolor{currentfill}%
\pgfsetlinewidth{0.000000pt}%
\definecolor{currentstroke}{rgb}{0.000000,0.000000,0.000000}%
\pgfsetstrokecolor{currentstroke}%
\pgfsetstrokeopacity{0.000000}%
\pgfsetdash{}{0pt}%
\pgfpathmoveto{\pgfqpoint{0.446110in}{1.806553in}}%
\pgfpathlineto{\pgfqpoint{0.458498in}{1.796156in}}%
\pgfpathlineto{\pgfqpoint{0.463269in}{1.796894in}}%
\pgfpathlineto{\pgfqpoint{0.471077in}{1.793199in}}%
\pgfpathlineto{\pgfqpoint{0.475081in}{1.794998in}}%
\pgfpathlineto{\pgfqpoint{0.482457in}{1.792328in}}%
\pgfpathlineto{\pgfqpoint{0.494344in}{1.790708in}}%
\pgfpathlineto{\pgfqpoint{0.510253in}{1.798280in}}%
\pgfpathlineto{\pgfqpoint{0.515527in}{1.796686in}}%
\pgfpathlineto{\pgfqpoint{0.519917in}{1.799771in}}%
\pgfpathlineto{\pgfqpoint{0.526142in}{1.797850in}}%
\pgfpathlineto{\pgfqpoint{0.503786in}{1.727081in}}%
\pgfpathlineto{\pgfqpoint{0.438361in}{1.747283in}}%
\pgfpathlineto{\pgfqpoint{0.422148in}{1.752150in}}%
\pgfpathlineto{\pgfqpoint{0.423690in}{1.763145in}}%
\pgfpathlineto{\pgfqpoint{0.429381in}{1.768800in}}%
\pgfpathlineto{\pgfqpoint{0.426886in}{1.778784in}}%
\pgfpathlineto{\pgfqpoint{0.419892in}{1.781620in}}%
\pgfpathlineto{\pgfqpoint{0.424079in}{1.790266in}}%
\pgfpathlineto{\pgfqpoint{0.430341in}{1.789335in}}%
\pgfpathlineto{\pgfqpoint{0.436517in}{1.797749in}}%
\pgfpathclose%
\pgfusepath{fill}%
\end{pgfscope}%
\begin{pgfscope}%
\pgfpathrectangle{\pgfqpoint{0.100000in}{0.100000in}}{\pgfqpoint{3.420221in}{2.189500in}}%
\pgfusepath{clip}%
\pgfsetbuttcap%
\pgfsetmiterjoin%
\definecolor{currentfill}{rgb}{0.000000,0.576471,0.711765}%
\pgfsetfillcolor{currentfill}%
\pgfsetlinewidth{0.000000pt}%
\definecolor{currentstroke}{rgb}{0.000000,0.000000,0.000000}%
\pgfsetstrokecolor{currentstroke}%
\pgfsetstrokeopacity{0.000000}%
\pgfsetdash{}{0pt}%
\pgfpathmoveto{\pgfqpoint{2.597438in}{0.835018in}}%
\pgfpathlineto{\pgfqpoint{2.591144in}{0.832011in}}%
\pgfpathlineto{\pgfqpoint{2.591982in}{0.824406in}}%
\pgfpathlineto{\pgfqpoint{2.585596in}{0.822624in}}%
\pgfpathlineto{\pgfqpoint{2.586357in}{0.816124in}}%
\pgfpathlineto{\pgfqpoint{2.580736in}{0.815466in}}%
\pgfpathlineto{\pgfqpoint{2.579159in}{0.835118in}}%
\pgfpathlineto{\pgfqpoint{2.547789in}{0.832030in}}%
\pgfpathlineto{\pgfqpoint{2.542333in}{0.836023in}}%
\pgfpathlineto{\pgfqpoint{2.538561in}{0.843190in}}%
\pgfpathlineto{\pgfqpoint{2.536876in}{0.862823in}}%
\pgfpathlineto{\pgfqpoint{2.539246in}{0.868080in}}%
\pgfpathlineto{\pgfqpoint{2.545156in}{0.872734in}}%
\pgfpathlineto{\pgfqpoint{2.543155in}{0.879760in}}%
\pgfpathlineto{\pgfqpoint{2.553815in}{0.892336in}}%
\pgfpathlineto{\pgfqpoint{2.551589in}{0.895462in}}%
\pgfpathlineto{\pgfqpoint{2.555629in}{0.902748in}}%
\pgfpathlineto{\pgfqpoint{2.561555in}{0.902991in}}%
\pgfpathlineto{\pgfqpoint{2.565704in}{0.896738in}}%
\pgfpathlineto{\pgfqpoint{2.578210in}{0.897922in}}%
\pgfpathlineto{\pgfqpoint{2.580690in}{0.891599in}}%
\pgfpathlineto{\pgfqpoint{2.588324in}{0.892178in}}%
\pgfpathlineto{\pgfqpoint{2.590699in}{0.862960in}}%
\pgfpathlineto{\pgfqpoint{2.594488in}{0.863405in}}%
\pgfpathclose%
\pgfusepath{fill}%
\end{pgfscope}%
\begin{pgfscope}%
\pgfpathrectangle{\pgfqpoint{0.100000in}{0.100000in}}{\pgfqpoint{3.420221in}{2.189500in}}%
\pgfusepath{clip}%
\pgfsetbuttcap%
\pgfsetmiterjoin%
\definecolor{currentfill}{rgb}{0.000000,0.466667,0.766667}%
\pgfsetfillcolor{currentfill}%
\pgfsetlinewidth{0.000000pt}%
\definecolor{currentstroke}{rgb}{0.000000,0.000000,0.000000}%
\pgfsetstrokecolor{currentstroke}%
\pgfsetstrokeopacity{0.000000}%
\pgfsetdash{}{0pt}%
\pgfpathmoveto{\pgfqpoint{2.337710in}{1.364365in}}%
\pgfpathlineto{\pgfqpoint{2.336350in}{1.381796in}}%
\pgfpathlineto{\pgfqpoint{2.329581in}{1.384603in}}%
\pgfpathlineto{\pgfqpoint{2.327998in}{1.404709in}}%
\pgfpathlineto{\pgfqpoint{2.335685in}{1.405385in}}%
\pgfpathlineto{\pgfqpoint{2.343810in}{1.411041in}}%
\pgfpathlineto{\pgfqpoint{2.346504in}{1.417855in}}%
\pgfpathlineto{\pgfqpoint{2.344650in}{1.444212in}}%
\pgfpathlineto{\pgfqpoint{2.364070in}{1.445705in}}%
\pgfpathlineto{\pgfqpoint{2.382882in}{1.447526in}}%
\pgfpathlineto{\pgfqpoint{2.383796in}{1.438635in}}%
\pgfpathlineto{\pgfqpoint{2.386590in}{1.410534in}}%
\pgfpathlineto{\pgfqpoint{2.373856in}{1.409540in}}%
\pgfpathlineto{\pgfqpoint{2.375651in}{1.384447in}}%
\pgfpathlineto{\pgfqpoint{2.369073in}{1.383972in}}%
\pgfpathlineto{\pgfqpoint{2.360514in}{1.366349in}}%
\pgfpathclose%
\pgfusepath{fill}%
\end{pgfscope}%
\begin{pgfscope}%
\pgfpathrectangle{\pgfqpoint{0.100000in}{0.100000in}}{\pgfqpoint{3.420221in}{2.189500in}}%
\pgfusepath{clip}%
\pgfsetbuttcap%
\pgfsetmiterjoin%
\definecolor{currentfill}{rgb}{0.000000,0.576471,0.711765}%
\pgfsetfillcolor{currentfill}%
\pgfsetlinewidth{0.000000pt}%
\definecolor{currentstroke}{rgb}{0.000000,0.000000,0.000000}%
\pgfsetstrokecolor{currentstroke}%
\pgfsetstrokeopacity{0.000000}%
\pgfsetdash{}{0pt}%
\pgfpathmoveto{\pgfqpoint{2.206618in}{0.697375in}}%
\pgfpathlineto{\pgfqpoint{2.216971in}{0.691822in}}%
\pgfpathlineto{\pgfqpoint{2.214422in}{0.687687in}}%
\pgfpathlineto{\pgfqpoint{2.214358in}{0.680846in}}%
\pgfpathlineto{\pgfqpoint{2.209245in}{0.675376in}}%
\pgfpathlineto{\pgfqpoint{2.208955in}{0.665805in}}%
\pgfpathlineto{\pgfqpoint{2.181526in}{0.664915in}}%
\pgfpathlineto{\pgfqpoint{2.178535in}{0.671466in}}%
\pgfpathlineto{\pgfqpoint{2.171349in}{0.676213in}}%
\pgfpathlineto{\pgfqpoint{2.165133in}{0.668959in}}%
\pgfpathlineto{\pgfqpoint{2.158051in}{0.667872in}}%
\pgfpathlineto{\pgfqpoint{2.143401in}{0.667041in}}%
\pgfpathlineto{\pgfqpoint{2.142356in}{0.694424in}}%
\pgfpathlineto{\pgfqpoint{2.139005in}{0.699267in}}%
\pgfpathlineto{\pgfqpoint{2.132390in}{0.701330in}}%
\pgfpathlineto{\pgfqpoint{2.136769in}{0.703061in}}%
\pgfpathlineto{\pgfqpoint{2.149003in}{0.714645in}}%
\pgfpathlineto{\pgfqpoint{2.143551in}{0.720512in}}%
\pgfpathlineto{\pgfqpoint{2.136757in}{0.722363in}}%
\pgfpathlineto{\pgfqpoint{2.132292in}{0.728706in}}%
\pgfpathlineto{\pgfqpoint{2.154998in}{0.729305in}}%
\pgfpathlineto{\pgfqpoint{2.154873in}{0.735918in}}%
\pgfpathlineto{\pgfqpoint{2.171335in}{0.736334in}}%
\pgfpathlineto{\pgfqpoint{2.172359in}{0.723291in}}%
\pgfpathlineto{\pgfqpoint{2.177624in}{0.716120in}}%
\pgfpathlineto{\pgfqpoint{2.185084in}{0.711748in}}%
\pgfpathlineto{\pgfqpoint{2.185955in}{0.707221in}}%
\pgfpathlineto{\pgfqpoint{2.195679in}{0.700885in}}%
\pgfpathclose%
\pgfusepath{fill}%
\end{pgfscope}%
\begin{pgfscope}%
\pgfpathrectangle{\pgfqpoint{0.100000in}{0.100000in}}{\pgfqpoint{3.420221in}{2.189500in}}%
\pgfusepath{clip}%
\pgfsetbuttcap%
\pgfsetmiterjoin%
\definecolor{currentfill}{rgb}{0.000000,0.572549,0.713725}%
\pgfsetfillcolor{currentfill}%
\pgfsetlinewidth{0.000000pt}%
\definecolor{currentstroke}{rgb}{0.000000,0.000000,0.000000}%
\pgfsetstrokecolor{currentstroke}%
\pgfsetstrokeopacity{0.000000}%
\pgfsetdash{}{0pt}%
\pgfpathmoveto{\pgfqpoint{2.410202in}{1.207405in}}%
\pgfpathlineto{\pgfqpoint{2.407485in}{1.208449in}}%
\pgfpathlineto{\pgfqpoint{2.393456in}{1.207469in}}%
\pgfpathlineto{\pgfqpoint{2.391798in}{1.233394in}}%
\pgfpathlineto{\pgfqpoint{2.372315in}{1.232121in}}%
\pgfpathlineto{\pgfqpoint{2.370731in}{1.258264in}}%
\pgfpathlineto{\pgfqpoint{2.396731in}{1.259629in}}%
\pgfpathlineto{\pgfqpoint{2.395290in}{1.263834in}}%
\pgfpathlineto{\pgfqpoint{2.395077in}{1.278139in}}%
\pgfpathlineto{\pgfqpoint{2.415463in}{1.279902in}}%
\pgfpathlineto{\pgfqpoint{2.416854in}{1.258989in}}%
\pgfpathlineto{\pgfqpoint{2.414665in}{1.253298in}}%
\pgfpathlineto{\pgfqpoint{2.414162in}{1.244298in}}%
\pgfpathlineto{\pgfqpoint{2.416324in}{1.239224in}}%
\pgfpathlineto{\pgfqpoint{2.414061in}{1.235413in}}%
\pgfpathlineto{\pgfqpoint{2.417481in}{1.225676in}}%
\pgfpathclose%
\pgfusepath{fill}%
\end{pgfscope}%
\begin{pgfscope}%
\pgfpathrectangle{\pgfqpoint{0.100000in}{0.100000in}}{\pgfqpoint{3.420221in}{2.189500in}}%
\pgfusepath{clip}%
\pgfsetbuttcap%
\pgfsetmiterjoin%
\definecolor{currentfill}{rgb}{0.000000,0.690196,0.654902}%
\pgfsetfillcolor{currentfill}%
\pgfsetlinewidth{0.000000pt}%
\definecolor{currentstroke}{rgb}{0.000000,0.000000,0.000000}%
\pgfsetstrokecolor{currentstroke}%
\pgfsetstrokeopacity{0.000000}%
\pgfsetdash{}{0pt}%
\pgfpathmoveto{\pgfqpoint{2.772938in}{1.307263in}}%
\pgfpathlineto{\pgfqpoint{2.762603in}{1.303981in}}%
\pgfpathlineto{\pgfqpoint{2.756828in}{1.299412in}}%
\pgfpathlineto{\pgfqpoint{2.758795in}{1.289595in}}%
\pgfpathlineto{\pgfqpoint{2.748051in}{1.296928in}}%
\pgfpathlineto{\pgfqpoint{2.739844in}{1.295281in}}%
\pgfpathlineto{\pgfqpoint{2.739317in}{1.302083in}}%
\pgfpathlineto{\pgfqpoint{2.732417in}{1.301591in}}%
\pgfpathlineto{\pgfqpoint{2.730849in}{1.308181in}}%
\pgfpathlineto{\pgfqpoint{2.725045in}{1.307894in}}%
\pgfpathlineto{\pgfqpoint{2.724789in}{1.312804in}}%
\pgfpathlineto{\pgfqpoint{2.731746in}{1.313774in}}%
\pgfpathlineto{\pgfqpoint{2.730929in}{1.328077in}}%
\pgfpathlineto{\pgfqpoint{2.737485in}{1.328426in}}%
\pgfpathlineto{\pgfqpoint{2.736564in}{1.341751in}}%
\pgfpathlineto{\pgfqpoint{2.738321in}{1.348607in}}%
\pgfpathlineto{\pgfqpoint{2.735692in}{1.355144in}}%
\pgfpathlineto{\pgfqpoint{2.735295in}{1.361776in}}%
\pgfpathlineto{\pgfqpoint{2.741822in}{1.362216in}}%
\pgfpathlineto{\pgfqpoint{2.741341in}{1.368968in}}%
\pgfpathlineto{\pgfqpoint{2.747799in}{1.369459in}}%
\pgfpathlineto{\pgfqpoint{2.748261in}{1.362666in}}%
\pgfpathlineto{\pgfqpoint{2.760459in}{1.363673in}}%
\pgfpathlineto{\pgfqpoint{2.761394in}{1.353807in}}%
\pgfpathlineto{\pgfqpoint{2.769478in}{1.351324in}}%
\pgfpathlineto{\pgfqpoint{2.768483in}{1.344419in}}%
\pgfpathlineto{\pgfqpoint{2.769799in}{1.332126in}}%
\pgfpathlineto{\pgfqpoint{2.763451in}{1.319695in}}%
\pgfpathclose%
\pgfusepath{fill}%
\end{pgfscope}%
\begin{pgfscope}%
\pgfpathrectangle{\pgfqpoint{0.100000in}{0.100000in}}{\pgfqpoint{3.420221in}{2.189500in}}%
\pgfusepath{clip}%
\pgfsetbuttcap%
\pgfsetmiterjoin%
\definecolor{currentfill}{rgb}{0.000000,0.270588,0.864706}%
\pgfsetfillcolor{currentfill}%
\pgfsetlinewidth{0.000000pt}%
\definecolor{currentstroke}{rgb}{0.000000,0.000000,0.000000}%
\pgfsetstrokecolor{currentstroke}%
\pgfsetstrokeopacity{0.000000}%
\pgfsetdash{}{0pt}%
\pgfpathmoveto{\pgfqpoint{1.799725in}{1.321408in}}%
\pgfpathlineto{\pgfqpoint{1.833002in}{1.320180in}}%
\pgfpathlineto{\pgfqpoint{1.832327in}{1.300676in}}%
\pgfpathlineto{\pgfqpoint{1.832074in}{1.294144in}}%
\pgfpathlineto{\pgfqpoint{1.799608in}{1.295325in}}%
\pgfpathlineto{\pgfqpoint{1.799364in}{1.288818in}}%
\pgfpathlineto{\pgfqpoint{1.766978in}{1.290183in}}%
\pgfpathlineto{\pgfqpoint{1.768636in}{1.322719in}}%
\pgfpathclose%
\pgfusepath{fill}%
\end{pgfscope}%
\begin{pgfscope}%
\pgfpathrectangle{\pgfqpoint{0.100000in}{0.100000in}}{\pgfqpoint{3.420221in}{2.189500in}}%
\pgfusepath{clip}%
\pgfsetbuttcap%
\pgfsetmiterjoin%
\definecolor{currentfill}{rgb}{0.000000,0.211765,0.894118}%
\pgfsetfillcolor{currentfill}%
\pgfsetlinewidth{0.000000pt}%
\definecolor{currentstroke}{rgb}{0.000000,0.000000,0.000000}%
\pgfsetstrokecolor{currentstroke}%
\pgfsetstrokeopacity{0.000000}%
\pgfsetdash{}{0pt}%
\pgfpathmoveto{\pgfqpoint{3.051661in}{1.204962in}}%
\pgfpathlineto{\pgfqpoint{3.047937in}{1.205255in}}%
\pgfpathlineto{\pgfqpoint{3.047743in}{1.216061in}}%
\pgfpathlineto{\pgfqpoint{3.040973in}{1.213872in}}%
\pgfpathlineto{\pgfqpoint{3.035557in}{1.216103in}}%
\pgfpathlineto{\pgfqpoint{3.028095in}{1.220784in}}%
\pgfpathlineto{\pgfqpoint{3.020586in}{1.220429in}}%
\pgfpathlineto{\pgfqpoint{3.017791in}{1.231733in}}%
\pgfpathlineto{\pgfqpoint{3.023229in}{1.236485in}}%
\pgfpathlineto{\pgfqpoint{2.996268in}{1.239293in}}%
\pgfpathlineto{\pgfqpoint{2.995055in}{1.244552in}}%
\pgfpathlineto{\pgfqpoint{2.999890in}{1.252063in}}%
\pgfpathlineto{\pgfqpoint{3.000170in}{1.267227in}}%
\pgfpathlineto{\pgfqpoint{2.994238in}{1.273383in}}%
\pgfpathlineto{\pgfqpoint{2.997658in}{1.285990in}}%
\pgfpathlineto{\pgfqpoint{3.005162in}{1.282897in}}%
\pgfpathlineto{\pgfqpoint{3.009839in}{1.277278in}}%
\pgfpathlineto{\pgfqpoint{3.015583in}{1.275977in}}%
\pgfpathlineto{\pgfqpoint{3.018032in}{1.297713in}}%
\pgfpathlineto{\pgfqpoint{3.033718in}{1.292115in}}%
\pgfpathlineto{\pgfqpoint{3.035720in}{1.287292in}}%
\pgfpathlineto{\pgfqpoint{3.040722in}{1.285463in}}%
\pgfpathlineto{\pgfqpoint{3.048193in}{1.299274in}}%
\pgfpathlineto{\pgfqpoint{3.054420in}{1.301559in}}%
\pgfpathlineto{\pgfqpoint{3.059156in}{1.293342in}}%
\pgfpathlineto{\pgfqpoint{3.065002in}{1.290428in}}%
\pgfpathlineto{\pgfqpoint{3.072485in}{1.292188in}}%
\pgfpathlineto{\pgfqpoint{3.076423in}{1.287943in}}%
\pgfpathlineto{\pgfqpoint{3.083942in}{1.279885in}}%
\pgfpathlineto{\pgfqpoint{3.085338in}{1.268870in}}%
\pgfpathlineto{\pgfqpoint{3.072468in}{1.264155in}}%
\pgfpathlineto{\pgfqpoint{3.076256in}{1.249249in}}%
\pgfpathlineto{\pgfqpoint{3.070656in}{1.250066in}}%
\pgfpathlineto{\pgfqpoint{3.061888in}{1.238029in}}%
\pgfpathlineto{\pgfqpoint{3.072877in}{1.235309in}}%
\pgfpathlineto{\pgfqpoint{3.076257in}{1.227944in}}%
\pgfpathclose%
\pgfusepath{fill}%
\end{pgfscope}%
\begin{pgfscope}%
\pgfpathrectangle{\pgfqpoint{0.100000in}{0.100000in}}{\pgfqpoint{3.420221in}{2.189500in}}%
\pgfusepath{clip}%
\pgfsetbuttcap%
\pgfsetmiterjoin%
\definecolor{currentfill}{rgb}{0.000000,0.435294,0.782353}%
\pgfsetfillcolor{currentfill}%
\pgfsetlinewidth{0.000000pt}%
\definecolor{currentstroke}{rgb}{0.000000,0.000000,0.000000}%
\pgfsetstrokecolor{currentstroke}%
\pgfsetstrokeopacity{0.000000}%
\pgfsetdash{}{0pt}%
\pgfpathmoveto{\pgfqpoint{1.898873in}{0.691284in}}%
\pgfpathlineto{\pgfqpoint{1.883685in}{0.682871in}}%
\pgfpathlineto{\pgfqpoint{1.867874in}{0.674158in}}%
\pgfpathlineto{\pgfqpoint{1.854940in}{0.696668in}}%
\pgfpathlineto{\pgfqpoint{1.850664in}{0.694092in}}%
\pgfpathlineto{\pgfqpoint{1.845959in}{0.699962in}}%
\pgfpathlineto{\pgfqpoint{1.829327in}{0.729970in}}%
\pgfpathlineto{\pgfqpoint{1.824320in}{0.727207in}}%
\pgfpathlineto{\pgfqpoint{1.809836in}{0.753750in}}%
\pgfpathlineto{\pgfqpoint{1.818986in}{0.758709in}}%
\pgfpathlineto{\pgfqpoint{1.834203in}{0.767279in}}%
\pgfpathlineto{\pgfqpoint{1.838979in}{0.762779in}}%
\pgfpathlineto{\pgfqpoint{1.844029in}{0.765094in}}%
\pgfpathlineto{\pgfqpoint{1.869055in}{0.770787in}}%
\pgfpathlineto{\pgfqpoint{1.877992in}{0.754486in}}%
\pgfpathlineto{\pgfqpoint{1.880869in}{0.756078in}}%
\pgfpathlineto{\pgfqpoint{1.891820in}{0.736299in}}%
\pgfpathlineto{\pgfqpoint{1.877980in}{0.728570in}}%
\pgfpathclose%
\pgfusepath{fill}%
\end{pgfscope}%
\begin{pgfscope}%
\pgfpathrectangle{\pgfqpoint{0.100000in}{0.100000in}}{\pgfqpoint{3.420221in}{2.189500in}}%
\pgfusepath{clip}%
\pgfsetbuttcap%
\pgfsetmiterjoin%
\definecolor{currentfill}{rgb}{0.000000,0.427451,0.786275}%
\pgfsetfillcolor{currentfill}%
\pgfsetlinewidth{0.000000pt}%
\definecolor{currentstroke}{rgb}{0.000000,0.000000,0.000000}%
\pgfsetstrokecolor{currentstroke}%
\pgfsetstrokeopacity{0.000000}%
\pgfsetdash{}{0pt}%
\pgfpathmoveto{\pgfqpoint{2.828466in}{1.347165in}}%
\pgfpathlineto{\pgfqpoint{2.823445in}{1.347414in}}%
\pgfpathlineto{\pgfqpoint{2.815761in}{1.358611in}}%
\pgfpathlineto{\pgfqpoint{2.816952in}{1.360416in}}%
\pgfpathlineto{\pgfqpoint{2.830577in}{1.374407in}}%
\pgfpathlineto{\pgfqpoint{2.836472in}{1.374185in}}%
\pgfpathlineto{\pgfqpoint{2.838583in}{1.379703in}}%
\pgfpathlineto{\pgfqpoint{2.836118in}{1.383833in}}%
\pgfpathlineto{\pgfqpoint{2.841325in}{1.390874in}}%
\pgfpathlineto{\pgfqpoint{2.838600in}{1.396246in}}%
\pgfpathlineto{\pgfqpoint{2.892296in}{1.405213in}}%
\pgfpathlineto{\pgfqpoint{2.898397in}{1.366262in}}%
\pgfpathlineto{\pgfqpoint{2.885984in}{1.369951in}}%
\pgfpathlineto{\pgfqpoint{2.879492in}{1.365745in}}%
\pgfpathlineto{\pgfqpoint{2.873162in}{1.370169in}}%
\pgfpathlineto{\pgfqpoint{2.867050in}{1.364980in}}%
\pgfpathlineto{\pgfqpoint{2.858934in}{1.363211in}}%
\pgfpathlineto{\pgfqpoint{2.857024in}{1.353066in}}%
\pgfpathlineto{\pgfqpoint{2.852948in}{1.351724in}}%
\pgfpathlineto{\pgfqpoint{2.836280in}{1.353875in}}%
\pgfpathclose%
\pgfusepath{fill}%
\end{pgfscope}%
\begin{pgfscope}%
\pgfpathrectangle{\pgfqpoint{0.100000in}{0.100000in}}{\pgfqpoint{3.420221in}{2.189500in}}%
\pgfusepath{clip}%
\pgfsetbuttcap%
\pgfsetmiterjoin%
\definecolor{currentfill}{rgb}{0.000000,0.423529,0.788235}%
\pgfsetfillcolor{currentfill}%
\pgfsetlinewidth{0.000000pt}%
\definecolor{currentstroke}{rgb}{0.000000,0.000000,0.000000}%
\pgfsetstrokecolor{currentstroke}%
\pgfsetstrokeopacity{0.000000}%
\pgfsetdash{}{0pt}%
\pgfpathmoveto{\pgfqpoint{3.031305in}{1.540824in}}%
\pgfpathlineto{\pgfqpoint{3.033892in}{1.534319in}}%
\pgfpathlineto{\pgfqpoint{3.041575in}{1.526894in}}%
\pgfpathlineto{\pgfqpoint{3.044281in}{1.518095in}}%
\pgfpathlineto{\pgfqpoint{3.049192in}{1.519913in}}%
\pgfpathlineto{\pgfqpoint{3.051509in}{1.515893in}}%
\pgfpathlineto{\pgfqpoint{3.035408in}{1.503588in}}%
\pgfpathlineto{\pgfqpoint{3.023622in}{1.497459in}}%
\pgfpathlineto{\pgfqpoint{3.016341in}{1.501376in}}%
\pgfpathlineto{\pgfqpoint{2.998384in}{1.499501in}}%
\pgfpathlineto{\pgfqpoint{2.995770in}{1.510110in}}%
\pgfpathlineto{\pgfqpoint{3.005121in}{1.526964in}}%
\pgfpathlineto{\pgfqpoint{3.013627in}{1.532040in}}%
\pgfpathlineto{\pgfqpoint{3.013747in}{1.536730in}}%
\pgfpathlineto{\pgfqpoint{3.022826in}{1.540629in}}%
\pgfpathclose%
\pgfusepath{fill}%
\end{pgfscope}%
\begin{pgfscope}%
\pgfpathrectangle{\pgfqpoint{0.100000in}{0.100000in}}{\pgfqpoint{3.420221in}{2.189500in}}%
\pgfusepath{clip}%
\pgfsetbuttcap%
\pgfsetmiterjoin%
\definecolor{currentfill}{rgb}{0.000000,0.749020,0.625490}%
\pgfsetfillcolor{currentfill}%
\pgfsetlinewidth{0.000000pt}%
\definecolor{currentstroke}{rgb}{0.000000,0.000000,0.000000}%
\pgfsetstrokecolor{currentstroke}%
\pgfsetstrokeopacity{0.000000}%
\pgfsetdash{}{0pt}%
\pgfpathmoveto{\pgfqpoint{2.421694in}{0.741515in}}%
\pgfpathlineto{\pgfqpoint{2.420068in}{0.756234in}}%
\pgfpathlineto{\pgfqpoint{2.421713in}{0.808110in}}%
\pgfpathlineto{\pgfqpoint{2.422371in}{0.834679in}}%
\pgfpathlineto{\pgfqpoint{2.422499in}{0.839381in}}%
\pgfpathlineto{\pgfqpoint{2.433094in}{0.840521in}}%
\pgfpathlineto{\pgfqpoint{2.436340in}{0.844264in}}%
\pgfpathlineto{\pgfqpoint{2.441674in}{0.844700in}}%
\pgfpathlineto{\pgfqpoint{2.446608in}{0.850037in}}%
\pgfpathlineto{\pgfqpoint{2.453219in}{0.853992in}}%
\pgfpathlineto{\pgfqpoint{2.454360in}{0.843778in}}%
\pgfpathlineto{\pgfqpoint{2.461989in}{0.844422in}}%
\pgfpathlineto{\pgfqpoint{2.461210in}{0.839185in}}%
\pgfpathlineto{\pgfqpoint{2.456787in}{0.834914in}}%
\pgfpathlineto{\pgfqpoint{2.456424in}{0.824943in}}%
\pgfpathlineto{\pgfqpoint{2.459757in}{0.813063in}}%
\pgfpathlineto{\pgfqpoint{2.458532in}{0.806861in}}%
\pgfpathlineto{\pgfqpoint{2.464106in}{0.804005in}}%
\pgfpathlineto{\pgfqpoint{2.477049in}{0.805141in}}%
\pgfpathlineto{\pgfqpoint{2.478144in}{0.792008in}}%
\pgfpathlineto{\pgfqpoint{2.481717in}{0.789097in}}%
\pgfpathlineto{\pgfqpoint{2.478525in}{0.788069in}}%
\pgfpathlineto{\pgfqpoint{2.479375in}{0.778902in}}%
\pgfpathlineto{\pgfqpoint{2.472878in}{0.778335in}}%
\pgfpathlineto{\pgfqpoint{2.473734in}{0.768799in}}%
\pgfpathlineto{\pgfqpoint{2.470948in}{0.767503in}}%
\pgfpathlineto{\pgfqpoint{2.445064in}{0.765379in}}%
\pgfpathlineto{\pgfqpoint{2.439186in}{0.751924in}}%
\pgfpathlineto{\pgfqpoint{2.444370in}{0.747958in}}%
\pgfpathlineto{\pgfqpoint{2.445788in}{0.743409in}}%
\pgfpathclose%
\pgfusepath{fill}%
\end{pgfscope}%
\begin{pgfscope}%
\pgfpathrectangle{\pgfqpoint{0.100000in}{0.100000in}}{\pgfqpoint{3.420221in}{2.189500in}}%
\pgfusepath{clip}%
\pgfsetbuttcap%
\pgfsetmiterjoin%
\definecolor{currentfill}{rgb}{0.000000,0.552941,0.723529}%
\pgfsetfillcolor{currentfill}%
\pgfsetlinewidth{0.000000pt}%
\definecolor{currentstroke}{rgb}{0.000000,0.000000,0.000000}%
\pgfsetstrokecolor{currentstroke}%
\pgfsetstrokeopacity{0.000000}%
\pgfsetdash{}{0pt}%
\pgfpathmoveto{\pgfqpoint{2.243689in}{0.914018in}}%
\pgfpathlineto{\pgfqpoint{2.239012in}{0.908751in}}%
\pgfpathlineto{\pgfqpoint{2.243296in}{0.901753in}}%
\pgfpathlineto{\pgfqpoint{2.232275in}{0.901572in}}%
\pgfpathlineto{\pgfqpoint{2.224170in}{0.906301in}}%
\pgfpathlineto{\pgfqpoint{2.222834in}{0.914295in}}%
\pgfpathlineto{\pgfqpoint{2.218894in}{0.914810in}}%
\pgfpathlineto{\pgfqpoint{2.217089in}{0.920892in}}%
\pgfpathlineto{\pgfqpoint{2.206437in}{0.921187in}}%
\pgfpathlineto{\pgfqpoint{2.206170in}{0.939887in}}%
\pgfpathlineto{\pgfqpoint{2.210145in}{0.944527in}}%
\pgfpathlineto{\pgfqpoint{2.209897in}{0.950723in}}%
\pgfpathlineto{\pgfqpoint{2.205922in}{0.954125in}}%
\pgfpathlineto{\pgfqpoint{2.205687in}{0.972729in}}%
\pgfpathlineto{\pgfqpoint{2.198900in}{0.972629in}}%
\pgfpathlineto{\pgfqpoint{2.198697in}{0.980799in}}%
\pgfpathlineto{\pgfqpoint{2.204243in}{0.983415in}}%
\pgfpathlineto{\pgfqpoint{2.212072in}{0.980880in}}%
\pgfpathlineto{\pgfqpoint{2.211949in}{0.985864in}}%
\pgfpathlineto{\pgfqpoint{2.219258in}{0.985996in}}%
\pgfpathlineto{\pgfqpoint{2.225591in}{0.983424in}}%
\pgfpathlineto{\pgfqpoint{2.230139in}{0.978586in}}%
\pgfpathlineto{\pgfqpoint{2.238891in}{0.980257in}}%
\pgfpathlineto{\pgfqpoint{2.239024in}{0.973451in}}%
\pgfpathlineto{\pgfqpoint{2.242404in}{0.970242in}}%
\pgfpathlineto{\pgfqpoint{2.242666in}{0.960303in}}%
\pgfpathlineto{\pgfqpoint{2.245946in}{0.960385in}}%
\pgfpathlineto{\pgfqpoint{2.246468in}{0.940607in}}%
\pgfpathlineto{\pgfqpoint{2.249834in}{0.940650in}}%
\pgfpathlineto{\pgfqpoint{2.250146in}{0.930639in}}%
\pgfpathlineto{\pgfqpoint{2.246812in}{0.930520in}}%
\pgfpathlineto{\pgfqpoint{2.245706in}{0.917272in}}%
\pgfpathclose%
\pgfusepath{fill}%
\end{pgfscope}%
\begin{pgfscope}%
\pgfpathrectangle{\pgfqpoint{0.100000in}{0.100000in}}{\pgfqpoint{3.420221in}{2.189500in}}%
\pgfusepath{clip}%
\pgfsetbuttcap%
\pgfsetmiterjoin%
\definecolor{currentfill}{rgb}{0.000000,0.427451,0.786275}%
\pgfsetfillcolor{currentfill}%
\pgfsetlinewidth{0.000000pt}%
\definecolor{currentstroke}{rgb}{0.000000,0.000000,0.000000}%
\pgfsetstrokecolor{currentstroke}%
\pgfsetstrokeopacity{0.000000}%
\pgfsetdash{}{0pt}%
\pgfpathmoveto{\pgfqpoint{1.588148in}{1.236253in}}%
\pgfpathlineto{\pgfqpoint{1.547228in}{1.239452in}}%
\pgfpathlineto{\pgfqpoint{1.502721in}{1.243292in}}%
\pgfpathlineto{\pgfqpoint{1.504310in}{1.262061in}}%
\pgfpathlineto{\pgfqpoint{1.523738in}{1.260933in}}%
\pgfpathlineto{\pgfqpoint{1.525354in}{1.273196in}}%
\pgfpathlineto{\pgfqpoint{1.528358in}{1.305826in}}%
\pgfpathlineto{\pgfqpoint{1.531803in}{1.338478in}}%
\pgfpathlineto{\pgfqpoint{1.595419in}{1.333771in}}%
\pgfpathlineto{\pgfqpoint{1.595388in}{1.333331in}}%
\pgfpathlineto{\pgfqpoint{1.590598in}{1.268282in}}%
\pgfpathclose%
\pgfusepath{fill}%
\end{pgfscope}%
\begin{pgfscope}%
\pgfpathrectangle{\pgfqpoint{0.100000in}{0.100000in}}{\pgfqpoint{3.420221in}{2.189500in}}%
\pgfusepath{clip}%
\pgfsetbuttcap%
\pgfsetmiterjoin%
\definecolor{currentfill}{rgb}{0.000000,0.623529,0.688235}%
\pgfsetfillcolor{currentfill}%
\pgfsetlinewidth{0.000000pt}%
\definecolor{currentstroke}{rgb}{0.000000,0.000000,0.000000}%
\pgfsetstrokecolor{currentstroke}%
\pgfsetstrokeopacity{0.000000}%
\pgfsetdash{}{0pt}%
\pgfpathmoveto{\pgfqpoint{2.276314in}{0.946045in}}%
\pgfpathlineto{\pgfqpoint{2.276030in}{0.937429in}}%
\pgfpathlineto{\pgfqpoint{2.265311in}{0.933660in}}%
\pgfpathlineto{\pgfqpoint{2.265059in}{0.927121in}}%
\pgfpathlineto{\pgfqpoint{2.253756in}{0.914534in}}%
\pgfpathlineto{\pgfqpoint{2.253810in}{0.914439in}}%
\pgfpathlineto{\pgfqpoint{2.243689in}{0.914018in}}%
\pgfpathlineto{\pgfqpoint{2.245706in}{0.917272in}}%
\pgfpathlineto{\pgfqpoint{2.246812in}{0.930520in}}%
\pgfpathlineto{\pgfqpoint{2.250146in}{0.930639in}}%
\pgfpathlineto{\pgfqpoint{2.249834in}{0.940650in}}%
\pgfpathlineto{\pgfqpoint{2.246468in}{0.940607in}}%
\pgfpathlineto{\pgfqpoint{2.245946in}{0.960385in}}%
\pgfpathlineto{\pgfqpoint{2.242666in}{0.960303in}}%
\pgfpathlineto{\pgfqpoint{2.242404in}{0.970242in}}%
\pgfpathlineto{\pgfqpoint{2.239024in}{0.973451in}}%
\pgfpathlineto{\pgfqpoint{2.238891in}{0.980257in}}%
\pgfpathlineto{\pgfqpoint{2.245324in}{0.980415in}}%
\pgfpathlineto{\pgfqpoint{2.244240in}{1.013593in}}%
\pgfpathlineto{\pgfqpoint{2.270403in}{1.015231in}}%
\pgfpathlineto{\pgfqpoint{2.290209in}{1.015457in}}%
\pgfpathlineto{\pgfqpoint{2.299184in}{1.015785in}}%
\pgfpathlineto{\pgfqpoint{2.303383in}{1.012066in}}%
\pgfpathlineto{\pgfqpoint{2.299916in}{1.001964in}}%
\pgfpathlineto{\pgfqpoint{2.304103in}{0.999635in}}%
\pgfpathlineto{\pgfqpoint{2.302901in}{0.991928in}}%
\pgfpathlineto{\pgfqpoint{2.298417in}{0.991792in}}%
\pgfpathlineto{\pgfqpoint{2.296281in}{0.984739in}}%
\pgfpathlineto{\pgfqpoint{2.291435in}{0.980767in}}%
\pgfpathlineto{\pgfqpoint{2.294553in}{0.975738in}}%
\pgfpathlineto{\pgfqpoint{2.284235in}{0.969669in}}%
\pgfpathlineto{\pgfqpoint{2.281796in}{0.957018in}}%
\pgfpathlineto{\pgfqpoint{2.274877in}{0.952811in}}%
\pgfpathclose%
\pgfusepath{fill}%
\end{pgfscope}%
\begin{pgfscope}%
\pgfpathrectangle{\pgfqpoint{0.100000in}{0.100000in}}{\pgfqpoint{3.420221in}{2.189500in}}%
\pgfusepath{clip}%
\pgfsetbuttcap%
\pgfsetmiterjoin%
\definecolor{currentfill}{rgb}{0.000000,0.305882,0.847059}%
\pgfsetfillcolor{currentfill}%
\pgfsetlinewidth{0.000000pt}%
\definecolor{currentstroke}{rgb}{0.000000,0.000000,0.000000}%
\pgfsetstrokecolor{currentstroke}%
\pgfsetstrokeopacity{0.000000}%
\pgfsetdash{}{0pt}%
\pgfpathmoveto{\pgfqpoint{2.583686in}{1.289084in}}%
\pgfpathlineto{\pgfqpoint{2.573957in}{1.282710in}}%
\pgfpathlineto{\pgfqpoint{2.572066in}{1.299273in}}%
\pgfpathlineto{\pgfqpoint{2.558061in}{1.297713in}}%
\pgfpathlineto{\pgfqpoint{2.556119in}{1.318801in}}%
\pgfpathlineto{\pgfqpoint{2.549316in}{1.313353in}}%
\pgfpathlineto{\pgfqpoint{2.542365in}{1.312473in}}%
\pgfpathlineto{\pgfqpoint{2.540808in}{1.328822in}}%
\pgfpathlineto{\pgfqpoint{2.543185in}{1.336838in}}%
\pgfpathlineto{\pgfqpoint{2.562288in}{1.338860in}}%
\pgfpathlineto{\pgfqpoint{2.561678in}{1.344236in}}%
\pgfpathlineto{\pgfqpoint{2.589440in}{1.347028in}}%
\pgfpathlineto{\pgfqpoint{2.592652in}{1.315991in}}%
\pgfpathlineto{\pgfqpoint{2.588644in}{1.311542in}}%
\pgfpathlineto{\pgfqpoint{2.592613in}{1.307196in}}%
\pgfpathlineto{\pgfqpoint{2.590871in}{1.301278in}}%
\pgfpathlineto{\pgfqpoint{2.596571in}{1.299433in}}%
\pgfpathlineto{\pgfqpoint{2.591336in}{1.292416in}}%
\pgfpathclose%
\pgfusepath{fill}%
\end{pgfscope}%
\begin{pgfscope}%
\pgfpathrectangle{\pgfqpoint{0.100000in}{0.100000in}}{\pgfqpoint{3.420221in}{2.189500in}}%
\pgfusepath{clip}%
\pgfsetbuttcap%
\pgfsetmiterjoin%
\definecolor{currentfill}{rgb}{0.000000,0.443137,0.778431}%
\pgfsetfillcolor{currentfill}%
\pgfsetlinewidth{0.000000pt}%
\definecolor{currentstroke}{rgb}{0.000000,0.000000,0.000000}%
\pgfsetstrokecolor{currentstroke}%
\pgfsetstrokeopacity{0.000000}%
\pgfsetdash{}{0pt}%
\pgfpathmoveto{\pgfqpoint{1.624911in}{0.899929in}}%
\pgfpathlineto{\pgfqpoint{1.657499in}{0.898006in}}%
\pgfpathlineto{\pgfqpoint{1.686831in}{0.896323in}}%
\pgfpathlineto{\pgfqpoint{1.708710in}{0.894375in}}%
\pgfpathlineto{\pgfqpoint{1.721171in}{0.889409in}}%
\pgfpathlineto{\pgfqpoint{1.722266in}{0.886745in}}%
\pgfpathlineto{\pgfqpoint{1.721156in}{0.861597in}}%
\pgfpathlineto{\pgfqpoint{1.719594in}{0.828347in}}%
\pgfpathlineto{\pgfqpoint{1.677128in}{0.830819in}}%
\pgfpathlineto{\pgfqpoint{1.653480in}{0.832431in}}%
\pgfpathlineto{\pgfqpoint{1.655578in}{0.865118in}}%
\pgfpathlineto{\pgfqpoint{1.622877in}{0.867151in}}%
\pgfpathclose%
\pgfusepath{fill}%
\end{pgfscope}%
\begin{pgfscope}%
\pgfpathrectangle{\pgfqpoint{0.100000in}{0.100000in}}{\pgfqpoint{3.420221in}{2.189500in}}%
\pgfusepath{clip}%
\pgfsetbuttcap%
\pgfsetmiterjoin%
\definecolor{currentfill}{rgb}{0.000000,0.843137,0.578431}%
\pgfsetfillcolor{currentfill}%
\pgfsetlinewidth{0.000000pt}%
\definecolor{currentstroke}{rgb}{0.000000,0.000000,0.000000}%
\pgfsetstrokecolor{currentstroke}%
\pgfsetstrokeopacity{0.000000}%
\pgfsetdash{}{0pt}%
\pgfpathmoveto{\pgfqpoint{2.220108in}{0.677662in}}%
\pgfpathlineto{\pgfqpoint{2.214358in}{0.680846in}}%
\pgfpathlineto{\pgfqpoint{2.214422in}{0.687687in}}%
\pgfpathlineto{\pgfqpoint{2.216971in}{0.691822in}}%
\pgfpathlineto{\pgfqpoint{2.206618in}{0.697375in}}%
\pgfpathlineto{\pgfqpoint{2.204170in}{0.712402in}}%
\pgfpathlineto{\pgfqpoint{2.209278in}{0.717683in}}%
\pgfpathlineto{\pgfqpoint{2.214481in}{0.733261in}}%
\pgfpathlineto{\pgfqpoint{2.217924in}{0.732702in}}%
\pgfpathlineto{\pgfqpoint{2.221631in}{0.738133in}}%
\pgfpathlineto{\pgfqpoint{2.221561in}{0.744501in}}%
\pgfpathlineto{\pgfqpoint{2.226362in}{0.758830in}}%
\pgfpathlineto{\pgfqpoint{2.225943in}{0.768896in}}%
\pgfpathlineto{\pgfqpoint{2.236916in}{0.769262in}}%
\pgfpathlineto{\pgfqpoint{2.248690in}{0.773291in}}%
\pgfpathlineto{\pgfqpoint{2.247037in}{0.764533in}}%
\pgfpathlineto{\pgfqpoint{2.255463in}{0.762339in}}%
\pgfpathlineto{\pgfqpoint{2.253136in}{0.755921in}}%
\pgfpathlineto{\pgfqpoint{2.246138in}{0.751433in}}%
\pgfpathlineto{\pgfqpoint{2.245366in}{0.747366in}}%
\pgfpathlineto{\pgfqpoint{2.236414in}{0.742080in}}%
\pgfpathlineto{\pgfqpoint{2.240146in}{0.729147in}}%
\pgfpathlineto{\pgfqpoint{2.249451in}{0.725023in}}%
\pgfpathlineto{\pgfqpoint{2.276217in}{0.726297in}}%
\pgfpathlineto{\pgfqpoint{2.282904in}{0.726637in}}%
\pgfpathlineto{\pgfqpoint{2.283848in}{0.706845in}}%
\pgfpathlineto{\pgfqpoint{2.259051in}{0.705578in}}%
\pgfpathlineto{\pgfqpoint{2.253112in}{0.703237in}}%
\pgfpathlineto{\pgfqpoint{2.244798in}{0.706728in}}%
\pgfpathlineto{\pgfqpoint{2.230085in}{0.702267in}}%
\pgfpathlineto{\pgfqpoint{2.229053in}{0.695193in}}%
\pgfpathlineto{\pgfqpoint{2.224382in}{0.695696in}}%
\pgfpathlineto{\pgfqpoint{2.220555in}{0.687844in}}%
\pgfpathlineto{\pgfqpoint{2.224632in}{0.682934in}}%
\pgfpathclose%
\pgfusepath{fill}%
\end{pgfscope}%
\begin{pgfscope}%
\pgfpathrectangle{\pgfqpoint{0.100000in}{0.100000in}}{\pgfqpoint{3.420221in}{2.189500in}}%
\pgfusepath{clip}%
\pgfsetbuttcap%
\pgfsetmiterjoin%
\definecolor{currentfill}{rgb}{0.000000,0.411765,0.794118}%
\pgfsetfillcolor{currentfill}%
\pgfsetlinewidth{0.000000pt}%
\definecolor{currentstroke}{rgb}{0.000000,0.000000,0.000000}%
\pgfsetstrokecolor{currentstroke}%
\pgfsetstrokeopacity{0.000000}%
\pgfsetdash{}{0pt}%
\pgfpathmoveto{\pgfqpoint{2.250300in}{1.297612in}}%
\pgfpathlineto{\pgfqpoint{2.223557in}{1.296655in}}%
\pgfpathlineto{\pgfqpoint{2.223684in}{1.290036in}}%
\pgfpathlineto{\pgfqpoint{2.219345in}{1.289927in}}%
\pgfpathlineto{\pgfqpoint{2.210698in}{1.289780in}}%
\pgfpathlineto{\pgfqpoint{2.208485in}{1.302902in}}%
\pgfpathlineto{\pgfqpoint{2.192229in}{1.303820in}}%
\pgfpathlineto{\pgfqpoint{2.191541in}{1.327674in}}%
\pgfpathlineto{\pgfqpoint{2.183941in}{1.327471in}}%
\pgfpathlineto{\pgfqpoint{2.183555in}{1.349175in}}%
\pgfpathlineto{\pgfqpoint{2.177103in}{1.349044in}}%
\pgfpathlineto{\pgfqpoint{2.176440in}{1.372100in}}%
\pgfpathlineto{\pgfqpoint{2.202382in}{1.372399in}}%
\pgfpathlineto{\pgfqpoint{2.202058in}{1.368766in}}%
\pgfpathlineto{\pgfqpoint{2.236009in}{1.369620in}}%
\pgfpathlineto{\pgfqpoint{2.236918in}{1.343567in}}%
\pgfpathlineto{\pgfqpoint{2.256755in}{1.344066in}}%
\pgfpathlineto{\pgfqpoint{2.252989in}{1.333305in}}%
\pgfpathlineto{\pgfqpoint{2.257819in}{1.321603in}}%
\pgfpathlineto{\pgfqpoint{2.255948in}{1.310757in}}%
\pgfpathlineto{\pgfqpoint{2.237254in}{1.310167in}}%
\pgfpathlineto{\pgfqpoint{2.245852in}{1.303808in}}%
\pgfpathclose%
\pgfusepath{fill}%
\end{pgfscope}%
\begin{pgfscope}%
\pgfpathrectangle{\pgfqpoint{0.100000in}{0.100000in}}{\pgfqpoint{3.420221in}{2.189500in}}%
\pgfusepath{clip}%
\pgfsetbuttcap%
\pgfsetmiterjoin%
\definecolor{currentfill}{rgb}{0.000000,0.564706,0.717647}%
\pgfsetfillcolor{currentfill}%
\pgfsetlinewidth{0.000000pt}%
\definecolor{currentstroke}{rgb}{0.000000,0.000000,0.000000}%
\pgfsetstrokecolor{currentstroke}%
\pgfsetstrokeopacity{0.000000}%
\pgfsetdash{}{0pt}%
\pgfpathmoveto{\pgfqpoint{0.960452in}{2.149271in}}%
\pgfpathlineto{\pgfqpoint{0.951744in}{2.112346in}}%
\pgfpathlineto{\pgfqpoint{0.915701in}{2.120890in}}%
\pgfpathlineto{\pgfqpoint{0.921948in}{2.146507in}}%
\pgfpathlineto{\pgfqpoint{0.909910in}{2.149399in}}%
\pgfpathlineto{\pgfqpoint{0.912703in}{2.160672in}}%
\pgfpathclose%
\pgfusepath{fill}%
\end{pgfscope}%
\begin{pgfscope}%
\pgfpathrectangle{\pgfqpoint{0.100000in}{0.100000in}}{\pgfqpoint{3.420221in}{2.189500in}}%
\pgfusepath{clip}%
\pgfsetbuttcap%
\pgfsetmiterjoin%
\definecolor{currentfill}{rgb}{0.000000,0.407843,0.796078}%
\pgfsetfillcolor{currentfill}%
\pgfsetlinewidth{0.000000pt}%
\definecolor{currentstroke}{rgb}{0.000000,0.000000,0.000000}%
\pgfsetstrokecolor{currentstroke}%
\pgfsetstrokeopacity{0.000000}%
\pgfsetdash{}{0pt}%
\pgfpathmoveto{\pgfqpoint{2.658630in}{1.544709in}}%
\pgfpathlineto{\pgfqpoint{2.651890in}{1.548800in}}%
\pgfpathlineto{\pgfqpoint{2.625557in}{1.544565in}}%
\pgfpathlineto{\pgfqpoint{2.605908in}{1.541387in}}%
\pgfpathlineto{\pgfqpoint{2.602224in}{1.567583in}}%
\pgfpathlineto{\pgfqpoint{2.598123in}{1.593702in}}%
\pgfpathlineto{\pgfqpoint{2.586879in}{1.592322in}}%
\pgfpathlineto{\pgfqpoint{2.583661in}{1.617791in}}%
\pgfpathlineto{\pgfqpoint{2.583571in}{1.618567in}}%
\pgfpathlineto{\pgfqpoint{2.607408in}{1.621761in}}%
\pgfpathlineto{\pgfqpoint{2.606439in}{1.628258in}}%
\pgfpathlineto{\pgfqpoint{2.632017in}{1.631677in}}%
\pgfpathlineto{\pgfqpoint{2.637809in}{1.633333in}}%
\pgfpathlineto{\pgfqpoint{2.636829in}{1.639811in}}%
\pgfpathlineto{\pgfqpoint{2.650369in}{1.641957in}}%
\pgfpathlineto{\pgfqpoint{2.646202in}{1.667792in}}%
\pgfpathlineto{\pgfqpoint{2.673579in}{1.672594in}}%
\pgfpathlineto{\pgfqpoint{2.675869in}{1.663101in}}%
\pgfpathlineto{\pgfqpoint{2.679960in}{1.653085in}}%
\pgfpathlineto{\pgfqpoint{2.683361in}{1.637709in}}%
\pgfpathlineto{\pgfqpoint{2.686125in}{1.629796in}}%
\pgfpathlineto{\pgfqpoint{2.690875in}{1.622943in}}%
\pgfpathlineto{\pgfqpoint{2.689390in}{1.607919in}}%
\pgfpathlineto{\pgfqpoint{2.690667in}{1.604555in}}%
\pgfpathlineto{\pgfqpoint{2.689204in}{1.592650in}}%
\pgfpathlineto{\pgfqpoint{2.678941in}{1.597543in}}%
\pgfpathlineto{\pgfqpoint{2.673997in}{1.594181in}}%
\pgfpathlineto{\pgfqpoint{2.670616in}{1.582055in}}%
\pgfpathlineto{\pgfqpoint{2.671777in}{1.578440in}}%
\pgfpathlineto{\pgfqpoint{2.669407in}{1.570493in}}%
\pgfpathlineto{\pgfqpoint{2.662446in}{1.567091in}}%
\pgfpathlineto{\pgfqpoint{2.659702in}{1.560758in}}%
\pgfpathlineto{\pgfqpoint{2.660854in}{1.549503in}}%
\pgfpathclose%
\pgfusepath{fill}%
\end{pgfscope}%
\begin{pgfscope}%
\pgfpathrectangle{\pgfqpoint{0.100000in}{0.100000in}}{\pgfqpoint{3.420221in}{2.189500in}}%
\pgfusepath{clip}%
\pgfsetbuttcap%
\pgfsetmiterjoin%
\definecolor{currentfill}{rgb}{0.000000,0.513725,0.743137}%
\pgfsetfillcolor{currentfill}%
\pgfsetlinewidth{0.000000pt}%
\definecolor{currentstroke}{rgb}{0.000000,0.000000,0.000000}%
\pgfsetstrokecolor{currentstroke}%
\pgfsetstrokeopacity{0.000000}%
\pgfsetdash{}{0pt}%
\pgfpathmoveto{\pgfqpoint{0.728634in}{2.210416in}}%
\pgfpathlineto{\pgfqpoint{0.730468in}{2.185708in}}%
\pgfpathlineto{\pgfqpoint{0.726097in}{2.183989in}}%
\pgfpathlineto{\pgfqpoint{0.718101in}{2.185011in}}%
\pgfpathlineto{\pgfqpoint{0.695982in}{2.191609in}}%
\pgfpathlineto{\pgfqpoint{0.642178in}{2.208304in}}%
\pgfpathlineto{\pgfqpoint{0.642758in}{2.218536in}}%
\pgfpathlineto{\pgfqpoint{0.637788in}{2.217427in}}%
\pgfpathlineto{\pgfqpoint{0.633535in}{2.230249in}}%
\pgfpathlineto{\pgfqpoint{0.637581in}{2.238238in}}%
\pgfpathlineto{\pgfqpoint{0.685496in}{2.223026in}}%
\pgfpathclose%
\pgfusepath{fill}%
\end{pgfscope}%
\begin{pgfscope}%
\pgfpathrectangle{\pgfqpoint{0.100000in}{0.100000in}}{\pgfqpoint{3.420221in}{2.189500in}}%
\pgfusepath{clip}%
\pgfsetbuttcap%
\pgfsetmiterjoin%
\definecolor{currentfill}{rgb}{0.000000,1.000000,0.500000}%
\pgfsetfillcolor{currentfill}%
\pgfsetlinewidth{0.000000pt}%
\definecolor{currentstroke}{rgb}{0.000000,0.000000,0.000000}%
\pgfsetstrokecolor{currentstroke}%
\pgfsetstrokeopacity{0.000000}%
\pgfsetdash{}{0pt}%
\pgfpathmoveto{\pgfqpoint{1.164577in}{1.189894in}}%
\pgfpathlineto{\pgfqpoint{1.153026in}{1.116042in}}%
\pgfpathlineto{\pgfqpoint{1.238228in}{1.103295in}}%
\pgfpathlineto{\pgfqpoint{1.257348in}{1.100532in}}%
\pgfpathlineto{\pgfqpoint{1.250210in}{1.048975in}}%
\pgfpathlineto{\pgfqpoint{1.179959in}{1.059125in}}%
\pgfpathlineto{\pgfqpoint{1.176061in}{1.033301in}}%
\pgfpathlineto{\pgfqpoint{1.140985in}{1.038709in}}%
\pgfpathlineto{\pgfqpoint{1.136566in}{1.010472in}}%
\pgfpathlineto{\pgfqpoint{1.127245in}{0.950968in}}%
\pgfpathlineto{\pgfqpoint{1.108659in}{0.953842in}}%
\pgfpathlineto{\pgfqpoint{1.099418in}{0.948488in}}%
\pgfpathlineto{\pgfqpoint{1.095330in}{0.943942in}}%
\pgfpathlineto{\pgfqpoint{1.090801in}{0.944112in}}%
\pgfpathlineto{\pgfqpoint{1.079115in}{0.935600in}}%
\pgfpathlineto{\pgfqpoint{1.072052in}{0.944383in}}%
\pgfpathlineto{\pgfqpoint{1.065882in}{0.945522in}}%
\pgfpathlineto{\pgfqpoint{1.071111in}{0.976916in}}%
\pgfpathlineto{\pgfqpoint{1.025018in}{0.984790in}}%
\pgfpathlineto{\pgfqpoint{1.028431in}{1.004289in}}%
\pgfpathlineto{\pgfqpoint{1.063895in}{1.206860in}}%
\pgfpathlineto{\pgfqpoint{1.108090in}{1.198913in}}%
\pgfpathclose%
\pgfusepath{fill}%
\end{pgfscope}%
\begin{pgfscope}%
\pgfpathrectangle{\pgfqpoint{0.100000in}{0.100000in}}{\pgfqpoint{3.420221in}{2.189500in}}%
\pgfusepath{clip}%
\pgfsetbuttcap%
\pgfsetmiterjoin%
\definecolor{currentfill}{rgb}{0.000000,0.360784,0.819608}%
\pgfsetfillcolor{currentfill}%
\pgfsetlinewidth{0.000000pt}%
\definecolor{currentstroke}{rgb}{0.000000,0.000000,0.000000}%
\pgfsetstrokecolor{currentstroke}%
\pgfsetstrokeopacity{0.000000}%
\pgfsetdash{}{0pt}%
\pgfpathmoveto{\pgfqpoint{2.156396in}{1.815109in}}%
\pgfpathlineto{\pgfqpoint{2.182449in}{1.815976in}}%
\pgfpathlineto{\pgfqpoint{2.215209in}{1.817157in}}%
\pgfpathlineto{\pgfqpoint{2.215765in}{1.804057in}}%
\pgfpathlineto{\pgfqpoint{2.228768in}{1.804696in}}%
\pgfpathlineto{\pgfqpoint{2.230839in}{1.759197in}}%
\pgfpathlineto{\pgfqpoint{2.217798in}{1.758721in}}%
\pgfpathlineto{\pgfqpoint{2.218087in}{1.752145in}}%
\pgfpathlineto{\pgfqpoint{2.185452in}{1.750784in}}%
\pgfpathlineto{\pgfqpoint{2.185649in}{1.744345in}}%
\pgfpathlineto{\pgfqpoint{2.153048in}{1.743437in}}%
\pgfpathlineto{\pgfqpoint{2.152118in}{1.775862in}}%
\pgfpathlineto{\pgfqpoint{2.158637in}{1.776085in}}%
\pgfpathlineto{\pgfqpoint{2.157674in}{1.802019in}}%
\pgfpathclose%
\pgfusepath{fill}%
\end{pgfscope}%
\begin{pgfscope}%
\pgfpathrectangle{\pgfqpoint{0.100000in}{0.100000in}}{\pgfqpoint{3.420221in}{2.189500in}}%
\pgfusepath{clip}%
\pgfsetbuttcap%
\pgfsetmiterjoin%
\definecolor{currentfill}{rgb}{0.000000,0.486275,0.756863}%
\pgfsetfillcolor{currentfill}%
\pgfsetlinewidth{0.000000pt}%
\definecolor{currentstroke}{rgb}{0.000000,0.000000,0.000000}%
\pgfsetstrokecolor{currentstroke}%
\pgfsetstrokeopacity{0.000000}%
\pgfsetdash{}{0pt}%
\pgfpathmoveto{\pgfqpoint{1.640138in}{0.766963in}}%
\pgfpathlineto{\pgfqpoint{1.607058in}{0.769259in}}%
\pgfpathlineto{\pgfqpoint{1.609897in}{0.802098in}}%
\pgfpathlineto{\pgfqpoint{1.612221in}{0.835139in}}%
\pgfpathlineto{\pgfqpoint{1.620750in}{0.835066in}}%
\pgfpathlineto{\pgfqpoint{1.622877in}{0.867151in}}%
\pgfpathlineto{\pgfqpoint{1.655578in}{0.865118in}}%
\pgfpathlineto{\pgfqpoint{1.653480in}{0.832431in}}%
\pgfpathlineto{\pgfqpoint{1.644861in}{0.833002in}}%
\pgfpathclose%
\pgfusepath{fill}%
\end{pgfscope}%
\begin{pgfscope}%
\pgfpathrectangle{\pgfqpoint{0.100000in}{0.100000in}}{\pgfqpoint{3.420221in}{2.189500in}}%
\pgfusepath{clip}%
\pgfsetbuttcap%
\pgfsetmiterjoin%
\definecolor{currentfill}{rgb}{0.000000,0.596078,0.701961}%
\pgfsetfillcolor{currentfill}%
\pgfsetlinewidth{0.000000pt}%
\definecolor{currentstroke}{rgb}{0.000000,0.000000,0.000000}%
\pgfsetstrokecolor{currentstroke}%
\pgfsetstrokeopacity{0.000000}%
\pgfsetdash{}{0pt}%
\pgfpathmoveto{\pgfqpoint{2.350959in}{0.856816in}}%
\pgfpathlineto{\pgfqpoint{2.359592in}{0.857326in}}%
\pgfpathlineto{\pgfqpoint{2.361084in}{0.830708in}}%
\pgfpathlineto{\pgfqpoint{2.335151in}{0.829013in}}%
\pgfpathlineto{\pgfqpoint{2.335357in}{0.825618in}}%
\pgfpathlineto{\pgfqpoint{2.320479in}{0.824381in}}%
\pgfpathlineto{\pgfqpoint{2.296441in}{0.832947in}}%
\pgfpathlineto{\pgfqpoint{2.290210in}{0.839101in}}%
\pgfpathlineto{\pgfqpoint{2.296464in}{0.842378in}}%
\pgfpathlineto{\pgfqpoint{2.295734in}{0.854983in}}%
\pgfpathlineto{\pgfqpoint{2.298857in}{0.857910in}}%
\pgfpathlineto{\pgfqpoint{2.305477in}{0.857795in}}%
\pgfpathlineto{\pgfqpoint{2.332692in}{0.850341in}}%
\pgfpathlineto{\pgfqpoint{2.335393in}{0.855854in}}%
\pgfpathclose%
\pgfusepath{fill}%
\end{pgfscope}%
\begin{pgfscope}%
\pgfpathrectangle{\pgfqpoint{0.100000in}{0.100000in}}{\pgfqpoint{3.420221in}{2.189500in}}%
\pgfusepath{clip}%
\pgfsetbuttcap%
\pgfsetmiterjoin%
\definecolor{currentfill}{rgb}{0.000000,0.400000,0.800000}%
\pgfsetfillcolor{currentfill}%
\pgfsetlinewidth{0.000000pt}%
\definecolor{currentstroke}{rgb}{0.000000,0.000000,0.000000}%
\pgfsetstrokecolor{currentstroke}%
\pgfsetstrokeopacity{0.000000}%
\pgfsetdash{}{0pt}%
\pgfpathmoveto{\pgfqpoint{2.230936in}{1.435391in}}%
\pgfpathlineto{\pgfqpoint{2.207141in}{1.434494in}}%
\pgfpathlineto{\pgfqpoint{2.206902in}{1.441020in}}%
\pgfpathlineto{\pgfqpoint{2.187412in}{1.440395in}}%
\pgfpathlineto{\pgfqpoint{2.174444in}{1.439993in}}%
\pgfpathlineto{\pgfqpoint{2.173554in}{1.466014in}}%
\pgfpathlineto{\pgfqpoint{2.180099in}{1.466291in}}%
\pgfpathlineto{\pgfqpoint{2.179060in}{1.492544in}}%
\pgfpathlineto{\pgfqpoint{2.205050in}{1.493383in}}%
\pgfpathlineto{\pgfqpoint{2.204802in}{1.499922in}}%
\pgfpathlineto{\pgfqpoint{2.230851in}{1.500944in}}%
\pgfpathlineto{\pgfqpoint{2.231453in}{1.487855in}}%
\pgfpathlineto{\pgfqpoint{2.231957in}{1.474796in}}%
\pgfpathlineto{\pgfqpoint{2.238462in}{1.475046in}}%
\pgfpathlineto{\pgfqpoint{2.238823in}{1.464283in}}%
\pgfpathlineto{\pgfqpoint{2.224375in}{1.460720in}}%
\pgfpathlineto{\pgfqpoint{2.221160in}{1.447515in}}%
\pgfpathlineto{\pgfqpoint{2.228388in}{1.441507in}}%
\pgfpathclose%
\pgfusepath{fill}%
\end{pgfscope}%
\begin{pgfscope}%
\pgfpathrectangle{\pgfqpoint{0.100000in}{0.100000in}}{\pgfqpoint{3.420221in}{2.189500in}}%
\pgfusepath{clip}%
\pgfsetbuttcap%
\pgfsetmiterjoin%
\definecolor{currentfill}{rgb}{0.000000,0.360784,0.819608}%
\pgfsetfillcolor{currentfill}%
\pgfsetlinewidth{0.000000pt}%
\definecolor{currentstroke}{rgb}{0.000000,0.000000,0.000000}%
\pgfsetstrokecolor{currentstroke}%
\pgfsetstrokeopacity{0.000000}%
\pgfsetdash{}{0pt}%
\pgfpathmoveto{\pgfqpoint{1.994758in}{1.783602in}}%
\pgfpathlineto{\pgfqpoint{1.995139in}{1.756374in}}%
\pgfpathlineto{\pgfqpoint{1.943001in}{1.756699in}}%
\pgfpathlineto{\pgfqpoint{1.936420in}{1.756837in}}%
\pgfpathlineto{\pgfqpoint{1.935975in}{1.769873in}}%
\pgfpathlineto{\pgfqpoint{1.936310in}{1.796229in}}%
\pgfpathlineto{\pgfqpoint{1.935770in}{1.802787in}}%
\pgfpathlineto{\pgfqpoint{1.935295in}{1.822483in}}%
\pgfpathlineto{\pgfqpoint{1.935618in}{1.848817in}}%
\pgfpathlineto{\pgfqpoint{1.941260in}{1.848732in}}%
\pgfpathlineto{\pgfqpoint{1.941226in}{1.855330in}}%
\pgfpathlineto{\pgfqpoint{1.993634in}{1.854921in}}%
\pgfpathlineto{\pgfqpoint{1.994510in}{1.822020in}}%
\pgfpathclose%
\pgfusepath{fill}%
\end{pgfscope}%
\begin{pgfscope}%
\pgfpathrectangle{\pgfqpoint{0.100000in}{0.100000in}}{\pgfqpoint{3.420221in}{2.189500in}}%
\pgfusepath{clip}%
\pgfsetbuttcap%
\pgfsetmiterjoin%
\definecolor{currentfill}{rgb}{0.000000,0.635294,0.682353}%
\pgfsetfillcolor{currentfill}%
\pgfsetlinewidth{0.000000pt}%
\definecolor{currentstroke}{rgb}{0.000000,0.000000,0.000000}%
\pgfsetstrokecolor{currentstroke}%
\pgfsetstrokeopacity{0.000000}%
\pgfsetdash{}{0pt}%
\pgfpathmoveto{\pgfqpoint{2.675925in}{1.322230in}}%
\pgfpathlineto{\pgfqpoint{2.663225in}{1.317872in}}%
\pgfpathlineto{\pgfqpoint{2.648150in}{1.316177in}}%
\pgfpathlineto{\pgfqpoint{2.646458in}{1.333031in}}%
\pgfpathlineto{\pgfqpoint{2.638259in}{1.332657in}}%
\pgfpathlineto{\pgfqpoint{2.637130in}{1.356179in}}%
\pgfpathlineto{\pgfqpoint{2.654891in}{1.356947in}}%
\pgfpathlineto{\pgfqpoint{2.654303in}{1.369454in}}%
\pgfpathlineto{\pgfqpoint{2.677427in}{1.370789in}}%
\pgfpathlineto{\pgfqpoint{2.678338in}{1.357370in}}%
\pgfpathlineto{\pgfqpoint{2.673588in}{1.346263in}}%
\pgfpathlineto{\pgfqpoint{2.674328in}{1.336710in}}%
\pgfpathlineto{\pgfqpoint{2.676651in}{1.335763in}}%
\pgfpathclose%
\pgfusepath{fill}%
\end{pgfscope}%
\begin{pgfscope}%
\pgfpathrectangle{\pgfqpoint{0.100000in}{0.100000in}}{\pgfqpoint{3.420221in}{2.189500in}}%
\pgfusepath{clip}%
\pgfsetbuttcap%
\pgfsetmiterjoin%
\definecolor{currentfill}{rgb}{0.000000,0.803922,0.598039}%
\pgfsetfillcolor{currentfill}%
\pgfsetlinewidth{0.000000pt}%
\definecolor{currentstroke}{rgb}{0.000000,0.000000,0.000000}%
\pgfsetstrokecolor{currentstroke}%
\pgfsetstrokeopacity{0.000000}%
\pgfsetdash{}{0pt}%
\pgfpathmoveto{\pgfqpoint{3.413931in}{1.989428in}}%
\pgfpathlineto{\pgfqpoint{3.436249in}{1.986049in}}%
\pgfpathlineto{\pgfqpoint{3.433678in}{1.980572in}}%
\pgfpathlineto{\pgfqpoint{3.440469in}{1.971095in}}%
\pgfpathlineto{\pgfqpoint{3.439590in}{1.965417in}}%
\pgfpathlineto{\pgfqpoint{3.449910in}{1.956779in}}%
\pgfpathlineto{\pgfqpoint{3.451911in}{1.962593in}}%
\pgfpathlineto{\pgfqpoint{3.458273in}{1.962291in}}%
\pgfpathlineto{\pgfqpoint{3.472062in}{1.947210in}}%
\pgfpathlineto{\pgfqpoint{3.475154in}{1.940088in}}%
\pgfpathlineto{\pgfqpoint{3.471145in}{1.933261in}}%
\pgfpathlineto{\pgfqpoint{3.468452in}{1.924634in}}%
\pgfpathlineto{\pgfqpoint{3.457986in}{1.923686in}}%
\pgfpathlineto{\pgfqpoint{3.458523in}{1.917383in}}%
\pgfpathlineto{\pgfqpoint{3.450587in}{1.917168in}}%
\pgfpathlineto{\pgfqpoint{3.448940in}{1.909137in}}%
\pgfpathlineto{\pgfqpoint{3.440474in}{1.907647in}}%
\pgfpathlineto{\pgfqpoint{3.437874in}{1.896555in}}%
\pgfpathlineto{\pgfqpoint{3.434518in}{1.895708in}}%
\pgfpathlineto{\pgfqpoint{3.427359in}{1.907488in}}%
\pgfpathlineto{\pgfqpoint{3.415119in}{1.931322in}}%
\pgfpathlineto{\pgfqpoint{3.421034in}{1.934432in}}%
\pgfpathlineto{\pgfqpoint{3.410721in}{1.954879in}}%
\pgfpathlineto{\pgfqpoint{3.416375in}{1.957568in}}%
\pgfpathlineto{\pgfqpoint{3.402633in}{1.982740in}}%
\pgfpathclose%
\pgfusepath{fill}%
\end{pgfscope}%
\begin{pgfscope}%
\pgfpathrectangle{\pgfqpoint{0.100000in}{0.100000in}}{\pgfqpoint{3.420221in}{2.189500in}}%
\pgfusepath{clip}%
\pgfsetbuttcap%
\pgfsetmiterjoin%
\definecolor{currentfill}{rgb}{0.000000,0.486275,0.756863}%
\pgfsetfillcolor{currentfill}%
\pgfsetlinewidth{0.000000pt}%
\definecolor{currentstroke}{rgb}{0.000000,0.000000,0.000000}%
\pgfsetstrokecolor{currentstroke}%
\pgfsetstrokeopacity{0.000000}%
\pgfsetdash{}{0pt}%
\pgfpathmoveto{\pgfqpoint{2.691355in}{1.508921in}}%
\pgfpathlineto{\pgfqpoint{2.677075in}{1.503983in}}%
\pgfpathlineto{\pgfqpoint{2.655638in}{1.501130in}}%
\pgfpathlineto{\pgfqpoint{2.650963in}{1.503759in}}%
\pgfpathlineto{\pgfqpoint{2.653779in}{1.478903in}}%
\pgfpathlineto{\pgfqpoint{2.628008in}{1.475749in}}%
\pgfpathlineto{\pgfqpoint{2.608577in}{1.473259in}}%
\pgfpathlineto{\pgfqpoint{2.606245in}{1.492759in}}%
\pgfpathlineto{\pgfqpoint{2.599908in}{1.491989in}}%
\pgfpathlineto{\pgfqpoint{2.596987in}{1.498174in}}%
\pgfpathlineto{\pgfqpoint{2.594271in}{1.512331in}}%
\pgfpathlineto{\pgfqpoint{2.629480in}{1.517925in}}%
\pgfpathlineto{\pgfqpoint{2.625557in}{1.544565in}}%
\pgfpathlineto{\pgfqpoint{2.651890in}{1.548800in}}%
\pgfpathlineto{\pgfqpoint{2.658630in}{1.544709in}}%
\pgfpathlineto{\pgfqpoint{2.652281in}{1.536921in}}%
\pgfpathlineto{\pgfqpoint{2.646630in}{1.526904in}}%
\pgfpathlineto{\pgfqpoint{2.645987in}{1.517982in}}%
\pgfpathlineto{\pgfqpoint{2.652765in}{1.518315in}}%
\pgfpathlineto{\pgfqpoint{2.669126in}{1.513576in}}%
\pgfpathlineto{\pgfqpoint{2.675441in}{1.508708in}}%
\pgfpathlineto{\pgfqpoint{2.688679in}{1.512079in}}%
\pgfpathclose%
\pgfusepath{fill}%
\end{pgfscope}%
\begin{pgfscope}%
\pgfpathrectangle{\pgfqpoint{0.100000in}{0.100000in}}{\pgfqpoint{3.420221in}{2.189500in}}%
\pgfusepath{clip}%
\pgfsetbuttcap%
\pgfsetmiterjoin%
\definecolor{currentfill}{rgb}{0.000000,0.450980,0.774510}%
\pgfsetfillcolor{currentfill}%
\pgfsetlinewidth{0.000000pt}%
\definecolor{currentstroke}{rgb}{0.000000,0.000000,0.000000}%
\pgfsetstrokecolor{currentstroke}%
\pgfsetstrokeopacity{0.000000}%
\pgfsetdash{}{0pt}%
\pgfpathmoveto{\pgfqpoint{2.261925in}{1.806421in}}%
\pgfpathlineto{\pgfqpoint{2.263712in}{1.774146in}}%
\pgfpathlineto{\pgfqpoint{2.264481in}{1.761226in}}%
\pgfpathlineto{\pgfqpoint{2.230839in}{1.759197in}}%
\pgfpathlineto{\pgfqpoint{2.228768in}{1.804696in}}%
\pgfpathlineto{\pgfqpoint{2.215765in}{1.804057in}}%
\pgfpathlineto{\pgfqpoint{2.215209in}{1.817157in}}%
\pgfpathlineto{\pgfqpoint{2.182449in}{1.815976in}}%
\pgfpathlineto{\pgfqpoint{2.180748in}{1.861235in}}%
\pgfpathlineto{\pgfqpoint{2.190000in}{1.864314in}}%
\pgfpathlineto{\pgfqpoint{2.198154in}{1.870279in}}%
\pgfpathlineto{\pgfqpoint{2.203360in}{1.869777in}}%
\pgfpathlineto{\pgfqpoint{2.210029in}{1.876423in}}%
\pgfpathlineto{\pgfqpoint{2.215862in}{1.878275in}}%
\pgfpathlineto{\pgfqpoint{2.221718in}{1.873272in}}%
\pgfpathlineto{\pgfqpoint{2.215249in}{1.862430in}}%
\pgfpathlineto{\pgfqpoint{2.214662in}{1.849496in}}%
\pgfpathlineto{\pgfqpoint{2.225394in}{1.856156in}}%
\pgfpathlineto{\pgfqpoint{2.233206in}{1.850454in}}%
\pgfpathlineto{\pgfqpoint{2.234348in}{1.824567in}}%
\pgfpathlineto{\pgfqpoint{2.240934in}{1.824864in}}%
\pgfpathlineto{\pgfqpoint{2.241246in}{1.818346in}}%
\pgfpathlineto{\pgfqpoint{2.247679in}{1.818735in}}%
\pgfpathlineto{\pgfqpoint{2.248409in}{1.805657in}}%
\pgfpathclose%
\pgfusepath{fill}%
\end{pgfscope}%
\begin{pgfscope}%
\pgfpathrectangle{\pgfqpoint{0.100000in}{0.100000in}}{\pgfqpoint{3.420221in}{2.189500in}}%
\pgfusepath{clip}%
\pgfsetbuttcap%
\pgfsetmiterjoin%
\definecolor{currentfill}{rgb}{0.000000,0.474510,0.762745}%
\pgfsetfillcolor{currentfill}%
\pgfsetlinewidth{0.000000pt}%
\definecolor{currentstroke}{rgb}{0.000000,0.000000,0.000000}%
\pgfsetstrokecolor{currentstroke}%
\pgfsetstrokeopacity{0.000000}%
\pgfsetdash{}{0pt}%
\pgfpathmoveto{\pgfqpoint{2.502400in}{1.390197in}}%
\pgfpathlineto{\pgfqpoint{2.502677in}{1.387583in}}%
\pgfpathlineto{\pgfqpoint{2.476844in}{1.384992in}}%
\pgfpathlineto{\pgfqpoint{2.476602in}{1.387657in}}%
\pgfpathlineto{\pgfqpoint{2.454005in}{1.385668in}}%
\pgfpathlineto{\pgfqpoint{2.455779in}{1.366133in}}%
\pgfpathlineto{\pgfqpoint{2.436942in}{1.364482in}}%
\pgfpathlineto{\pgfqpoint{2.436624in}{1.377617in}}%
\pgfpathlineto{\pgfqpoint{2.429374in}{1.378558in}}%
\pgfpathlineto{\pgfqpoint{2.430901in}{1.358769in}}%
\pgfpathlineto{\pgfqpoint{2.407745in}{1.356666in}}%
\pgfpathlineto{\pgfqpoint{2.405465in}{1.382473in}}%
\pgfpathlineto{\pgfqpoint{2.404337in}{1.401975in}}%
\pgfpathlineto{\pgfqpoint{2.427548in}{1.404238in}}%
\pgfpathlineto{\pgfqpoint{2.426052in}{1.422567in}}%
\pgfpathlineto{\pgfqpoint{2.450247in}{1.424587in}}%
\pgfpathlineto{\pgfqpoint{2.449521in}{1.432129in}}%
\pgfpathlineto{\pgfqpoint{2.455922in}{1.432713in}}%
\pgfpathlineto{\pgfqpoint{2.458575in}{1.438561in}}%
\pgfpathlineto{\pgfqpoint{2.478282in}{1.440236in}}%
\pgfpathlineto{\pgfqpoint{2.479485in}{1.427161in}}%
\pgfpathlineto{\pgfqpoint{2.482990in}{1.424222in}}%
\pgfpathlineto{\pgfqpoint{2.491564in}{1.425048in}}%
\pgfpathlineto{\pgfqpoint{2.493288in}{1.405583in}}%
\pgfpathlineto{\pgfqpoint{2.501245in}{1.401981in}}%
\pgfpathclose%
\pgfusepath{fill}%
\end{pgfscope}%
\begin{pgfscope}%
\pgfpathrectangle{\pgfqpoint{0.100000in}{0.100000in}}{\pgfqpoint{3.420221in}{2.189500in}}%
\pgfusepath{clip}%
\pgfsetbuttcap%
\pgfsetmiterjoin%
\definecolor{currentfill}{rgb}{0.000000,0.654902,0.672549}%
\pgfsetfillcolor{currentfill}%
\pgfsetlinewidth{0.000000pt}%
\definecolor{currentstroke}{rgb}{0.000000,0.000000,0.000000}%
\pgfsetstrokecolor{currentstroke}%
\pgfsetstrokeopacity{0.000000}%
\pgfsetdash{}{0pt}%
\pgfpathmoveto{\pgfqpoint{2.092651in}{0.687691in}}%
\pgfpathlineto{\pgfqpoint{2.072562in}{0.686048in}}%
\pgfpathlineto{\pgfqpoint{2.058468in}{0.681450in}}%
\pgfpathlineto{\pgfqpoint{2.045075in}{0.691189in}}%
\pgfpathlineto{\pgfqpoint{2.038397in}{0.697422in}}%
\pgfpathlineto{\pgfqpoint{2.034131in}{0.704820in}}%
\pgfpathlineto{\pgfqpoint{2.019601in}{0.708285in}}%
\pgfpathlineto{\pgfqpoint{2.010751in}{0.713514in}}%
\pgfpathlineto{\pgfqpoint{2.006948in}{0.717750in}}%
\pgfpathlineto{\pgfqpoint{2.003744in}{0.732771in}}%
\pgfpathlineto{\pgfqpoint{2.006144in}{0.737572in}}%
\pgfpathlineto{\pgfqpoint{2.037236in}{0.737580in}}%
\pgfpathlineto{\pgfqpoint{2.033391in}{0.747350in}}%
\pgfpathlineto{\pgfqpoint{2.065129in}{0.748051in}}%
\pgfpathlineto{\pgfqpoint{2.073998in}{0.737919in}}%
\pgfpathlineto{\pgfqpoint{2.077687in}{0.732624in}}%
\pgfpathlineto{\pgfqpoint{2.078940in}{0.725883in}}%
\pgfpathlineto{\pgfqpoint{2.076778in}{0.719433in}}%
\pgfpathlineto{\pgfqpoint{2.082698in}{0.709660in}}%
\pgfpathlineto{\pgfqpoint{2.087633in}{0.704363in}}%
\pgfpathlineto{\pgfqpoint{2.086863in}{0.697786in}}%
\pgfpathclose%
\pgfusepath{fill}%
\end{pgfscope}%
\begin{pgfscope}%
\pgfpathrectangle{\pgfqpoint{0.100000in}{0.100000in}}{\pgfqpoint{3.420221in}{2.189500in}}%
\pgfusepath{clip}%
\pgfsetbuttcap%
\pgfsetmiterjoin%
\definecolor{currentfill}{rgb}{0.000000,0.529412,0.735294}%
\pgfsetfillcolor{currentfill}%
\pgfsetlinewidth{0.000000pt}%
\definecolor{currentstroke}{rgb}{0.000000,0.000000,0.000000}%
\pgfsetstrokecolor{currentstroke}%
\pgfsetstrokeopacity{0.000000}%
\pgfsetdash{}{0pt}%
\pgfpathmoveto{\pgfqpoint{2.988099in}{0.917970in}}%
\pgfpathlineto{\pgfqpoint{2.976134in}{0.925835in}}%
\pgfpathlineto{\pgfqpoint{2.957927in}{0.929208in}}%
\pgfpathlineto{\pgfqpoint{2.940913in}{0.940144in}}%
\pgfpathlineto{\pgfqpoint{2.932992in}{0.938520in}}%
\pgfpathlineto{\pgfqpoint{2.938091in}{0.956508in}}%
\pgfpathlineto{\pgfqpoint{2.936207in}{0.959726in}}%
\pgfpathlineto{\pgfqpoint{2.940531in}{0.964321in}}%
\pgfpathlineto{\pgfqpoint{2.935293in}{0.973196in}}%
\pgfpathlineto{\pgfqpoint{2.939759in}{0.977192in}}%
\pgfpathlineto{\pgfqpoint{2.921338in}{0.987279in}}%
\pgfpathlineto{\pgfqpoint{2.923392in}{0.992722in}}%
\pgfpathlineto{\pgfqpoint{2.918317in}{0.999328in}}%
\pgfpathlineto{\pgfqpoint{2.910879in}{1.001149in}}%
\pgfpathlineto{\pgfqpoint{2.924164in}{1.012946in}}%
\pgfpathlineto{\pgfqpoint{2.938926in}{1.014824in}}%
\pgfpathlineto{\pgfqpoint{2.951835in}{1.002790in}}%
\pgfpathlineto{\pgfqpoint{2.958924in}{1.028641in}}%
\pgfpathlineto{\pgfqpoint{2.986217in}{1.008864in}}%
\pgfpathlineto{\pgfqpoint{3.024580in}{0.981508in}}%
\pgfpathlineto{\pgfqpoint{3.015274in}{0.976282in}}%
\pgfpathlineto{\pgfqpoint{3.009459in}{0.970603in}}%
\pgfpathlineto{\pgfqpoint{3.002560in}{0.960611in}}%
\pgfpathlineto{\pgfqpoint{2.996081in}{0.948299in}}%
\pgfpathlineto{\pgfqpoint{2.993567in}{0.939616in}}%
\pgfpathlineto{\pgfqpoint{2.993738in}{0.924151in}}%
\pgfpathclose%
\pgfusepath{fill}%
\end{pgfscope}%
\begin{pgfscope}%
\pgfpathrectangle{\pgfqpoint{0.100000in}{0.100000in}}{\pgfqpoint{3.420221in}{2.189500in}}%
\pgfusepath{clip}%
\pgfsetbuttcap%
\pgfsetmiterjoin%
\definecolor{currentfill}{rgb}{0.000000,0.658824,0.670588}%
\pgfsetfillcolor{currentfill}%
\pgfsetlinewidth{0.000000pt}%
\definecolor{currentstroke}{rgb}{0.000000,0.000000,0.000000}%
\pgfsetstrokecolor{currentstroke}%
\pgfsetstrokeopacity{0.000000}%
\pgfsetdash{}{0pt}%
\pgfpathmoveto{\pgfqpoint{2.480510in}{1.822354in}}%
\pgfpathlineto{\pgfqpoint{2.475157in}{1.876631in}}%
\pgfpathlineto{\pgfqpoint{2.493936in}{1.877519in}}%
\pgfpathlineto{\pgfqpoint{2.504114in}{1.883721in}}%
\pgfpathlineto{\pgfqpoint{2.520586in}{1.887665in}}%
\pgfpathlineto{\pgfqpoint{2.518137in}{1.880696in}}%
\pgfpathlineto{\pgfqpoint{2.519587in}{1.865805in}}%
\pgfpathlineto{\pgfqpoint{2.530604in}{1.863787in}}%
\pgfpathlineto{\pgfqpoint{2.538074in}{1.867947in}}%
\pgfpathlineto{\pgfqpoint{2.542604in}{1.863287in}}%
\pgfpathlineto{\pgfqpoint{2.551210in}{1.870536in}}%
\pgfpathlineto{\pgfqpoint{2.567146in}{1.872659in}}%
\pgfpathlineto{\pgfqpoint{2.566112in}{1.866020in}}%
\pgfpathlineto{\pgfqpoint{2.569049in}{1.859026in}}%
\pgfpathlineto{\pgfqpoint{2.567432in}{1.849059in}}%
\pgfpathlineto{\pgfqpoint{2.574934in}{1.845267in}}%
\pgfpathlineto{\pgfqpoint{2.577728in}{1.838049in}}%
\pgfpathlineto{\pgfqpoint{2.587692in}{1.836505in}}%
\pgfpathlineto{\pgfqpoint{2.598465in}{1.844863in}}%
\pgfpathlineto{\pgfqpoint{2.605081in}{1.837865in}}%
\pgfpathlineto{\pgfqpoint{2.603461in}{1.832887in}}%
\pgfpathlineto{\pgfqpoint{2.596221in}{1.833445in}}%
\pgfpathlineto{\pgfqpoint{2.588806in}{1.831814in}}%
\pgfpathlineto{\pgfqpoint{2.583393in}{1.834208in}}%
\pgfpathlineto{\pgfqpoint{2.570871in}{1.832784in}}%
\pgfpathlineto{\pgfqpoint{2.558420in}{1.828029in}}%
\pgfpathlineto{\pgfqpoint{2.550034in}{1.829859in}}%
\pgfpathlineto{\pgfqpoint{2.547590in}{1.834489in}}%
\pgfpathlineto{\pgfqpoint{2.541224in}{1.833701in}}%
\pgfpathlineto{\pgfqpoint{2.539300in}{1.819041in}}%
\pgfpathlineto{\pgfqpoint{2.530619in}{1.824147in}}%
\pgfpathlineto{\pgfqpoint{2.525493in}{1.829659in}}%
\pgfpathlineto{\pgfqpoint{2.517264in}{1.832261in}}%
\pgfpathlineto{\pgfqpoint{2.503511in}{1.834494in}}%
\pgfpathlineto{\pgfqpoint{2.497477in}{1.833446in}}%
\pgfpathlineto{\pgfqpoint{2.490678in}{1.823160in}}%
\pgfpathclose%
\pgfusepath{fill}%
\end{pgfscope}%
\begin{pgfscope}%
\pgfpathrectangle{\pgfqpoint{0.100000in}{0.100000in}}{\pgfqpoint{3.420221in}{2.189500in}}%
\pgfusepath{clip}%
\pgfsetbuttcap%
\pgfsetmiterjoin%
\definecolor{currentfill}{rgb}{0.000000,0.509804,0.745098}%
\pgfsetfillcolor{currentfill}%
\pgfsetlinewidth{0.000000pt}%
\definecolor{currentstroke}{rgb}{0.000000,0.000000,0.000000}%
\pgfsetstrokecolor{currentstroke}%
\pgfsetstrokeopacity{0.000000}%
\pgfsetdash{}{0pt}%
\pgfpathmoveto{\pgfqpoint{2.357760in}{1.016259in}}%
\pgfpathlineto{\pgfqpoint{2.340008in}{1.015594in}}%
\pgfpathlineto{\pgfqpoint{2.338651in}{1.028377in}}%
\pgfpathlineto{\pgfqpoint{2.331022in}{1.032193in}}%
\pgfpathlineto{\pgfqpoint{2.319455in}{1.032189in}}%
\pgfpathlineto{\pgfqpoint{2.312832in}{1.024146in}}%
\pgfpathlineto{\pgfqpoint{2.310395in}{1.028799in}}%
\pgfpathlineto{\pgfqpoint{2.309528in}{1.039039in}}%
\pgfpathlineto{\pgfqpoint{2.317298in}{1.040847in}}%
\pgfpathlineto{\pgfqpoint{2.320460in}{1.044996in}}%
\pgfpathlineto{\pgfqpoint{2.320093in}{1.051122in}}%
\pgfpathlineto{\pgfqpoint{2.322882in}{1.056675in}}%
\pgfpathlineto{\pgfqpoint{2.324451in}{1.065745in}}%
\pgfpathlineto{\pgfqpoint{2.329459in}{1.069399in}}%
\pgfpathlineto{\pgfqpoint{2.323065in}{1.076369in}}%
\pgfpathlineto{\pgfqpoint{2.332481in}{1.079018in}}%
\pgfpathlineto{\pgfqpoint{2.331467in}{1.085272in}}%
\pgfpathlineto{\pgfqpoint{2.331108in}{1.097375in}}%
\pgfpathlineto{\pgfqpoint{2.334359in}{1.097534in}}%
\pgfpathlineto{\pgfqpoint{2.338433in}{1.097884in}}%
\pgfpathlineto{\pgfqpoint{2.342291in}{1.091378in}}%
\pgfpathlineto{\pgfqpoint{2.338842in}{1.079921in}}%
\pgfpathlineto{\pgfqpoint{2.335796in}{1.076210in}}%
\pgfpathlineto{\pgfqpoint{2.355665in}{1.076848in}}%
\pgfpathlineto{\pgfqpoint{2.355985in}{1.063440in}}%
\pgfpathlineto{\pgfqpoint{2.345956in}{1.045562in}}%
\pgfpathlineto{\pgfqpoint{2.363299in}{1.038853in}}%
\pgfpathlineto{\pgfqpoint{2.364008in}{1.019243in}}%
\pgfpathclose%
\pgfusepath{fill}%
\end{pgfscope}%
\begin{pgfscope}%
\pgfpathrectangle{\pgfqpoint{0.100000in}{0.100000in}}{\pgfqpoint{3.420221in}{2.189500in}}%
\pgfusepath{clip}%
\pgfsetbuttcap%
\pgfsetmiterjoin%
\definecolor{currentfill}{rgb}{0.000000,0.290196,0.854902}%
\pgfsetfillcolor{currentfill}%
\pgfsetlinewidth{0.000000pt}%
\definecolor{currentstroke}{rgb}{0.000000,0.000000,0.000000}%
\pgfsetstrokecolor{currentstroke}%
\pgfsetstrokeopacity{0.000000}%
\pgfsetdash{}{0pt}%
\pgfpathmoveto{\pgfqpoint{2.821177in}{1.078551in}}%
\pgfpathlineto{\pgfqpoint{2.828518in}{1.093957in}}%
\pgfpathlineto{\pgfqpoint{2.824078in}{1.095407in}}%
\pgfpathlineto{\pgfqpoint{2.817566in}{1.093835in}}%
\pgfpathlineto{\pgfqpoint{2.807211in}{1.099719in}}%
\pgfpathlineto{\pgfqpoint{2.800436in}{1.105317in}}%
\pgfpathlineto{\pgfqpoint{2.806628in}{1.118391in}}%
\pgfpathlineto{\pgfqpoint{2.814436in}{1.119164in}}%
\pgfpathlineto{\pgfqpoint{2.823048in}{1.117017in}}%
\pgfpathlineto{\pgfqpoint{2.828801in}{1.112156in}}%
\pgfpathlineto{\pgfqpoint{2.836592in}{1.116409in}}%
\pgfpathlineto{\pgfqpoint{2.842253in}{1.116623in}}%
\pgfpathlineto{\pgfqpoint{2.854882in}{1.120667in}}%
\pgfpathlineto{\pgfqpoint{2.866287in}{1.122114in}}%
\pgfpathlineto{\pgfqpoint{2.867714in}{1.107272in}}%
\pgfpathlineto{\pgfqpoint{2.865965in}{1.093485in}}%
\pgfpathlineto{\pgfqpoint{2.870008in}{1.081374in}}%
\pgfpathlineto{\pgfqpoint{2.856683in}{1.080134in}}%
\pgfpathlineto{\pgfqpoint{2.856081in}{1.082279in}}%
\pgfpathclose%
\pgfusepath{fill}%
\end{pgfscope}%
\begin{pgfscope}%
\pgfpathrectangle{\pgfqpoint{0.100000in}{0.100000in}}{\pgfqpoint{3.420221in}{2.189500in}}%
\pgfusepath{clip}%
\pgfsetbuttcap%
\pgfsetmiterjoin%
\definecolor{currentfill}{rgb}{0.000000,0.337255,0.831373}%
\pgfsetfillcolor{currentfill}%
\pgfsetlinewidth{0.000000pt}%
\definecolor{currentstroke}{rgb}{0.000000,0.000000,0.000000}%
\pgfsetstrokecolor{currentstroke}%
\pgfsetstrokeopacity{0.000000}%
\pgfsetdash{}{0pt}%
\pgfpathmoveto{\pgfqpoint{2.660228in}{1.429749in}}%
\pgfpathlineto{\pgfqpoint{2.654491in}{1.428927in}}%
\pgfpathlineto{\pgfqpoint{2.633828in}{1.428964in}}%
\pgfpathlineto{\pgfqpoint{2.628008in}{1.475749in}}%
\pgfpathlineto{\pgfqpoint{2.653779in}{1.478903in}}%
\pgfpathlineto{\pgfqpoint{2.650963in}{1.503759in}}%
\pgfpathlineto{\pgfqpoint{2.655638in}{1.501130in}}%
\pgfpathlineto{\pgfqpoint{2.677075in}{1.503983in}}%
\pgfpathlineto{\pgfqpoint{2.683233in}{1.502705in}}%
\pgfpathlineto{\pgfqpoint{2.688576in}{1.470632in}}%
\pgfpathlineto{\pgfqpoint{2.672731in}{1.468279in}}%
\pgfpathlineto{\pgfqpoint{2.675643in}{1.446694in}}%
\pgfpathlineto{\pgfqpoint{2.658416in}{1.443238in}}%
\pgfpathclose%
\pgfusepath{fill}%
\end{pgfscope}%
\begin{pgfscope}%
\pgfpathrectangle{\pgfqpoint{0.100000in}{0.100000in}}{\pgfqpoint{3.420221in}{2.189500in}}%
\pgfusepath{clip}%
\pgfsetbuttcap%
\pgfsetmiterjoin%
\definecolor{currentfill}{rgb}{0.000000,0.505882,0.747059}%
\pgfsetfillcolor{currentfill}%
\pgfsetlinewidth{0.000000pt}%
\definecolor{currentstroke}{rgb}{0.000000,0.000000,0.000000}%
\pgfsetstrokecolor{currentstroke}%
\pgfsetstrokeopacity{0.000000}%
\pgfsetdash{}{0pt}%
\pgfpathmoveto{\pgfqpoint{3.262715in}{1.661511in}}%
\pgfpathlineto{\pgfqpoint{3.212962in}{1.649734in}}%
\pgfpathlineto{\pgfqpoint{3.209099in}{1.675049in}}%
\pgfpathlineto{\pgfqpoint{3.207638in}{1.694932in}}%
\pgfpathlineto{\pgfqpoint{3.202662in}{1.698418in}}%
\pgfpathlineto{\pgfqpoint{3.206993in}{1.702326in}}%
\pgfpathlineto{\pgfqpoint{3.241932in}{1.709740in}}%
\pgfpathlineto{\pgfqpoint{3.245880in}{1.706203in}}%
\pgfpathlineto{\pgfqpoint{3.244928in}{1.699331in}}%
\pgfpathlineto{\pgfqpoint{3.247251in}{1.681816in}}%
\pgfpathlineto{\pgfqpoint{3.254217in}{1.676237in}}%
\pgfpathlineto{\pgfqpoint{3.252943in}{1.670906in}}%
\pgfpathlineto{\pgfqpoint{3.260308in}{1.671078in}}%
\pgfpathclose%
\pgfusepath{fill}%
\end{pgfscope}%
\begin{pgfscope}%
\pgfpathrectangle{\pgfqpoint{0.100000in}{0.100000in}}{\pgfqpoint{3.420221in}{2.189500in}}%
\pgfusepath{clip}%
\pgfsetbuttcap%
\pgfsetmiterjoin%
\definecolor{currentfill}{rgb}{0.000000,0.305882,0.847059}%
\pgfsetfillcolor{currentfill}%
\pgfsetlinewidth{0.000000pt}%
\definecolor{currentstroke}{rgb}{0.000000,0.000000,0.000000}%
\pgfsetstrokecolor{currentstroke}%
\pgfsetstrokeopacity{0.000000}%
\pgfsetdash{}{0pt}%
\pgfpathmoveto{\pgfqpoint{1.842398in}{1.195912in}}%
\pgfpathlineto{\pgfqpoint{1.842977in}{1.215438in}}%
\pgfpathlineto{\pgfqpoint{1.830006in}{1.215843in}}%
\pgfpathlineto{\pgfqpoint{1.830928in}{1.248509in}}%
\pgfpathlineto{\pgfqpoint{1.863264in}{1.247528in}}%
\pgfpathlineto{\pgfqpoint{1.889116in}{1.246916in}}%
\pgfpathlineto{\pgfqpoint{1.888993in}{1.240408in}}%
\pgfpathlineto{\pgfqpoint{1.895457in}{1.240273in}}%
\pgfpathlineto{\pgfqpoint{1.894125in}{1.227274in}}%
\pgfpathlineto{\pgfqpoint{1.893548in}{1.207568in}}%
\pgfpathlineto{\pgfqpoint{1.875150in}{1.208129in}}%
\pgfpathlineto{\pgfqpoint{1.874879in}{1.195020in}}%
\pgfpathclose%
\pgfusepath{fill}%
\end{pgfscope}%
\begin{pgfscope}%
\pgfpathrectangle{\pgfqpoint{0.100000in}{0.100000in}}{\pgfqpoint{3.420221in}{2.189500in}}%
\pgfusepath{clip}%
\pgfsetbuttcap%
\pgfsetmiterjoin%
\definecolor{currentfill}{rgb}{0.000000,0.286275,0.856863}%
\pgfsetfillcolor{currentfill}%
\pgfsetlinewidth{0.000000pt}%
\definecolor{currentstroke}{rgb}{0.000000,0.000000,0.000000}%
\pgfsetstrokecolor{currentstroke}%
\pgfsetstrokeopacity{0.000000}%
\pgfsetdash{}{0pt}%
\pgfpathmoveto{\pgfqpoint{1.634626in}{1.869267in}}%
\pgfpathlineto{\pgfqpoint{1.634094in}{1.862737in}}%
\pgfpathlineto{\pgfqpoint{1.591078in}{1.866225in}}%
\pgfpathlineto{\pgfqpoint{1.555861in}{1.869418in}}%
\pgfpathlineto{\pgfqpoint{1.558348in}{1.895779in}}%
\pgfpathlineto{\pgfqpoint{1.555395in}{1.896052in}}%
\pgfpathlineto{\pgfqpoint{1.557929in}{1.922392in}}%
\pgfpathlineto{\pgfqpoint{1.586932in}{1.919663in}}%
\pgfpathlineto{\pgfqpoint{1.589228in}{1.945752in}}%
\pgfpathlineto{\pgfqpoint{1.612469in}{1.943696in}}%
\pgfpathlineto{\pgfqpoint{1.617248in}{1.951376in}}%
\pgfpathlineto{\pgfqpoint{1.626060in}{1.948968in}}%
\pgfpathlineto{\pgfqpoint{1.632542in}{1.950224in}}%
\pgfpathlineto{\pgfqpoint{1.631368in}{1.938249in}}%
\pgfpathlineto{\pgfqpoint{1.634124in}{1.934364in}}%
\pgfpathlineto{\pgfqpoint{1.632633in}{1.915750in}}%
\pgfpathlineto{\pgfqpoint{1.635765in}{1.915512in}}%
\pgfpathlineto{\pgfqpoint{1.633690in}{1.889320in}}%
\pgfpathlineto{\pgfqpoint{1.636156in}{1.889136in}}%
\pgfpathclose%
\pgfusepath{fill}%
\end{pgfscope}%
\begin{pgfscope}%
\pgfpathrectangle{\pgfqpoint{0.100000in}{0.100000in}}{\pgfqpoint{3.420221in}{2.189500in}}%
\pgfusepath{clip}%
\pgfsetbuttcap%
\pgfsetmiterjoin%
\definecolor{currentfill}{rgb}{0.000000,0.419608,0.790196}%
\pgfsetfillcolor{currentfill}%
\pgfsetlinewidth{0.000000pt}%
\definecolor{currentstroke}{rgb}{0.000000,0.000000,0.000000}%
\pgfsetstrokecolor{currentstroke}%
\pgfsetstrokeopacity{0.000000}%
\pgfsetdash{}{0pt}%
\pgfpathmoveto{\pgfqpoint{0.890837in}{0.370837in}}%
\pgfpathlineto{\pgfqpoint{0.892141in}{0.373800in}}%
\pgfpathlineto{\pgfqpoint{0.893759in}{0.374629in}}%
\pgfpathlineto{\pgfqpoint{0.896642in}{0.373002in}}%
\pgfpathlineto{\pgfqpoint{0.896197in}{0.371663in}}%
\pgfpathlineto{\pgfqpoint{0.898995in}{0.369143in}}%
\pgfpathlineto{\pgfqpoint{0.892691in}{0.369459in}}%
\pgfpathclose%
\pgfusepath{fill}%
\end{pgfscope}%
\begin{pgfscope}%
\pgfpathrectangle{\pgfqpoint{0.100000in}{0.100000in}}{\pgfqpoint{3.420221in}{2.189500in}}%
\pgfusepath{clip}%
\pgfsetbuttcap%
\pgfsetmiterjoin%
\definecolor{currentfill}{rgb}{0.000000,0.419608,0.790196}%
\pgfsetfillcolor{currentfill}%
\pgfsetlinewidth{0.000000pt}%
\definecolor{currentstroke}{rgb}{0.000000,0.000000,0.000000}%
\pgfsetstrokecolor{currentstroke}%
\pgfsetstrokeopacity{0.000000}%
\pgfsetdash{}{0pt}%
\pgfpathmoveto{\pgfqpoint{0.869384in}{0.368814in}}%
\pgfpathlineto{\pgfqpoint{0.872068in}{0.370690in}}%
\pgfpathlineto{\pgfqpoint{0.874603in}{0.370829in}}%
\pgfpathlineto{\pgfqpoint{0.884224in}{0.373672in}}%
\pgfpathlineto{\pgfqpoint{0.885192in}{0.375816in}}%
\pgfpathlineto{\pgfqpoint{0.887296in}{0.374524in}}%
\pgfpathlineto{\pgfqpoint{0.887278in}{0.372356in}}%
\pgfpathlineto{\pgfqpoint{0.881731in}{0.371060in}}%
\pgfpathlineto{\pgfqpoint{0.877680in}{0.368284in}}%
\pgfpathlineto{\pgfqpoint{0.875649in}{0.368433in}}%
\pgfpathlineto{\pgfqpoint{0.874791in}{0.366930in}}%
\pgfpathlineto{\pgfqpoint{0.870552in}{0.367825in}}%
\pgfpathclose%
\pgfusepath{fill}%
\end{pgfscope}%
\begin{pgfscope}%
\pgfpathrectangle{\pgfqpoint{0.100000in}{0.100000in}}{\pgfqpoint{3.420221in}{2.189500in}}%
\pgfusepath{clip}%
\pgfsetbuttcap%
\pgfsetmiterjoin%
\definecolor{currentfill}{rgb}{0.000000,0.419608,0.790196}%
\pgfsetfillcolor{currentfill}%
\pgfsetlinewidth{0.000000pt}%
\definecolor{currentstroke}{rgb}{0.000000,0.000000,0.000000}%
\pgfsetstrokecolor{currentstroke}%
\pgfsetstrokeopacity{0.000000}%
\pgfsetdash{}{0pt}%
\pgfpathmoveto{\pgfqpoint{0.875988in}{0.375806in}}%
\pgfpathlineto{\pgfqpoint{0.879162in}{0.382211in}}%
\pgfpathlineto{\pgfqpoint{0.879883in}{0.380088in}}%
\pgfpathclose%
\pgfusepath{fill}%
\end{pgfscope}%
\begin{pgfscope}%
\pgfpathrectangle{\pgfqpoint{0.100000in}{0.100000in}}{\pgfqpoint{3.420221in}{2.189500in}}%
\pgfusepath{clip}%
\pgfsetbuttcap%
\pgfsetmiterjoin%
\definecolor{currentfill}{rgb}{0.000000,0.419608,0.790196}%
\pgfsetfillcolor{currentfill}%
\pgfsetlinewidth{0.000000pt}%
\definecolor{currentstroke}{rgb}{0.000000,0.000000,0.000000}%
\pgfsetstrokecolor{currentstroke}%
\pgfsetstrokeopacity{0.000000}%
\pgfsetdash{}{0pt}%
\pgfpathmoveto{\pgfqpoint{0.933697in}{0.433889in}}%
\pgfpathlineto{\pgfqpoint{0.937806in}{0.431288in}}%
\pgfpathlineto{\pgfqpoint{0.940384in}{0.435494in}}%
\pgfpathlineto{\pgfqpoint{0.950757in}{0.429003in}}%
\pgfpathlineto{\pgfqpoint{0.949467in}{0.426889in}}%
\pgfpathlineto{\pgfqpoint{0.955336in}{0.423256in}}%
\pgfpathlineto{\pgfqpoint{0.971950in}{0.413652in}}%
\pgfpathlineto{\pgfqpoint{0.970014in}{0.408060in}}%
\pgfpathlineto{\pgfqpoint{0.966548in}{0.406661in}}%
\pgfpathlineto{\pgfqpoint{0.967330in}{0.404302in}}%
\pgfpathlineto{\pgfqpoint{0.964631in}{0.400756in}}%
\pgfpathlineto{\pgfqpoint{0.967310in}{0.398737in}}%
\pgfpathlineto{\pgfqpoint{0.970382in}{0.400610in}}%
\pgfpathlineto{\pgfqpoint{0.971017in}{0.399375in}}%
\pgfpathlineto{\pgfqpoint{0.974566in}{0.397281in}}%
\pgfpathlineto{\pgfqpoint{0.973321in}{0.395240in}}%
\pgfpathlineto{\pgfqpoint{0.975480in}{0.393949in}}%
\pgfpathlineto{\pgfqpoint{0.974277in}{0.391940in}}%
\pgfpathlineto{\pgfqpoint{0.976179in}{0.390826in}}%
\pgfpathlineto{\pgfqpoint{0.971298in}{0.382530in}}%
\pgfpathlineto{\pgfqpoint{0.972506in}{0.380590in}}%
\pgfpathlineto{\pgfqpoint{0.969453in}{0.375163in}}%
\pgfpathlineto{\pgfqpoint{0.978816in}{0.369945in}}%
\pgfpathlineto{\pgfqpoint{0.968270in}{0.350919in}}%
\pgfpathlineto{\pgfqpoint{0.957419in}{0.331347in}}%
\pgfpathlineto{\pgfqpoint{0.950261in}{0.318434in}}%
\pgfpathlineto{\pgfqpoint{0.946522in}{0.324473in}}%
\pgfpathlineto{\pgfqpoint{0.950087in}{0.324706in}}%
\pgfpathlineto{\pgfqpoint{0.951022in}{0.327404in}}%
\pgfpathlineto{\pgfqpoint{0.948907in}{0.328785in}}%
\pgfpathlineto{\pgfqpoint{0.948464in}{0.326879in}}%
\pgfpathlineto{\pgfqpoint{0.943917in}{0.327661in}}%
\pgfpathlineto{\pgfqpoint{0.942569in}{0.330290in}}%
\pgfpathlineto{\pgfqpoint{0.933485in}{0.338668in}}%
\pgfpathlineto{\pgfqpoint{0.929429in}{0.339618in}}%
\pgfpathlineto{\pgfqpoint{0.923923in}{0.342100in}}%
\pgfpathlineto{\pgfqpoint{0.918554in}{0.343740in}}%
\pgfpathlineto{\pgfqpoint{0.917400in}{0.345292in}}%
\pgfpathlineto{\pgfqpoint{0.917309in}{0.348451in}}%
\pgfpathlineto{\pgfqpoint{0.916124in}{0.350941in}}%
\pgfpathlineto{\pgfqpoint{0.913115in}{0.353209in}}%
\pgfpathlineto{\pgfqpoint{0.913398in}{0.355926in}}%
\pgfpathlineto{\pgfqpoint{0.912216in}{0.357828in}}%
\pgfpathlineto{\pgfqpoint{0.915027in}{0.360922in}}%
\pgfpathlineto{\pgfqpoint{0.915497in}{0.362614in}}%
\pgfpathlineto{\pgfqpoint{0.913851in}{0.363889in}}%
\pgfpathlineto{\pgfqpoint{0.911450in}{0.362240in}}%
\pgfpathlineto{\pgfqpoint{0.908890in}{0.363474in}}%
\pgfpathlineto{\pgfqpoint{0.907694in}{0.362608in}}%
\pgfpathlineto{\pgfqpoint{0.905037in}{0.367400in}}%
\pgfpathlineto{\pgfqpoint{0.903033in}{0.368378in}}%
\pgfpathlineto{\pgfqpoint{0.905009in}{0.370274in}}%
\pgfpathlineto{\pgfqpoint{0.905524in}{0.374774in}}%
\pgfpathlineto{\pgfqpoint{0.905279in}{0.377843in}}%
\pgfpathlineto{\pgfqpoint{0.901294in}{0.378766in}}%
\pgfpathlineto{\pgfqpoint{0.903186in}{0.380685in}}%
\pgfpathlineto{\pgfqpoint{0.898976in}{0.382586in}}%
\pgfpathlineto{\pgfqpoint{0.899543in}{0.386699in}}%
\pgfpathlineto{\pgfqpoint{0.897535in}{0.387794in}}%
\pgfpathlineto{\pgfqpoint{0.897036in}{0.389620in}}%
\pgfpathlineto{\pgfqpoint{0.894577in}{0.389281in}}%
\pgfpathlineto{\pgfqpoint{0.893679in}{0.387213in}}%
\pgfpathlineto{\pgfqpoint{0.890482in}{0.391113in}}%
\pgfpathlineto{\pgfqpoint{0.891940in}{0.394111in}}%
\pgfpathlineto{\pgfqpoint{0.887311in}{0.393006in}}%
\pgfpathlineto{\pgfqpoint{0.881842in}{0.392720in}}%
\pgfpathlineto{\pgfqpoint{0.883578in}{0.395509in}}%
\pgfpathlineto{\pgfqpoint{0.885265in}{0.394458in}}%
\pgfpathlineto{\pgfqpoint{0.885471in}{0.398082in}}%
\pgfpathlineto{\pgfqpoint{0.883820in}{0.398141in}}%
\pgfpathlineto{\pgfqpoint{0.879215in}{0.395145in}}%
\pgfpathlineto{\pgfqpoint{0.875835in}{0.391954in}}%
\pgfpathlineto{\pgfqpoint{0.876929in}{0.391433in}}%
\pgfpathlineto{\pgfqpoint{0.879116in}{0.393137in}}%
\pgfpathlineto{\pgfqpoint{0.880933in}{0.391490in}}%
\pgfpathlineto{\pgfqpoint{0.878561in}{0.389577in}}%
\pgfpathlineto{\pgfqpoint{0.876575in}{0.389180in}}%
\pgfpathlineto{\pgfqpoint{0.873517in}{0.390366in}}%
\pgfpathlineto{\pgfqpoint{0.874963in}{0.386469in}}%
\pgfpathlineto{\pgfqpoint{0.877099in}{0.385125in}}%
\pgfpathlineto{\pgfqpoint{0.871322in}{0.382901in}}%
\pgfpathlineto{\pgfqpoint{0.870472in}{0.381776in}}%
\pgfpathlineto{\pgfqpoint{0.873866in}{0.377716in}}%
\pgfpathlineto{\pgfqpoint{0.872794in}{0.373096in}}%
\pgfpathlineto{\pgfqpoint{0.870892in}{0.373091in}}%
\pgfpathlineto{\pgfqpoint{0.869245in}{0.375498in}}%
\pgfpathlineto{\pgfqpoint{0.870080in}{0.378299in}}%
\pgfpathlineto{\pgfqpoint{0.868649in}{0.379607in}}%
\pgfpathlineto{\pgfqpoint{0.866127in}{0.376759in}}%
\pgfpathlineto{\pgfqpoint{0.864088in}{0.378009in}}%
\pgfpathlineto{\pgfqpoint{0.866480in}{0.381088in}}%
\pgfpathlineto{\pgfqpoint{0.871019in}{0.387827in}}%
\pgfpathlineto{\pgfqpoint{0.869845in}{0.388628in}}%
\pgfpathlineto{\pgfqpoint{0.874767in}{0.395699in}}%
\pgfpathlineto{\pgfqpoint{0.873744in}{0.396362in}}%
\pgfpathlineto{\pgfqpoint{0.875473in}{0.398948in}}%
\pgfpathlineto{\pgfqpoint{0.878600in}{0.397005in}}%
\pgfpathlineto{\pgfqpoint{0.883377in}{0.404310in}}%
\pgfpathlineto{\pgfqpoint{0.887640in}{0.410227in}}%
\pgfpathlineto{\pgfqpoint{0.901579in}{0.401011in}}%
\pgfpathlineto{\pgfqpoint{0.902337in}{0.402174in}}%
\pgfpathlineto{\pgfqpoint{0.905234in}{0.400269in}}%
\pgfpathlineto{\pgfqpoint{0.915345in}{0.416446in}}%
\pgfpathlineto{\pgfqpoint{0.916427in}{0.418626in}}%
\pgfpathlineto{\pgfqpoint{0.922510in}{0.414648in}}%
\pgfpathlineto{\pgfqpoint{0.931589in}{0.429229in}}%
\pgfpathclose%
\pgfusepath{fill}%
\end{pgfscope}%
\begin{pgfscope}%
\pgfpathrectangle{\pgfqpoint{0.100000in}{0.100000in}}{\pgfqpoint{3.420221in}{2.189500in}}%
\pgfusepath{clip}%
\pgfsetbuttcap%
\pgfsetmiterjoin%
\definecolor{currentfill}{rgb}{0.000000,0.854902,0.572549}%
\pgfsetfillcolor{currentfill}%
\pgfsetlinewidth{0.000000pt}%
\definecolor{currentstroke}{rgb}{0.000000,0.000000,0.000000}%
\pgfsetstrokecolor{currentstroke}%
\pgfsetstrokeopacity{0.000000}%
\pgfsetdash{}{0pt}%
\pgfpathmoveto{\pgfqpoint{2.776519in}{1.219287in}}%
\pgfpathlineto{\pgfqpoint{2.780353in}{1.222630in}}%
\pgfpathlineto{\pgfqpoint{2.782260in}{1.231360in}}%
\pgfpathlineto{\pgfqpoint{2.787752in}{1.235265in}}%
\pgfpathlineto{\pgfqpoint{2.783521in}{1.238452in}}%
\pgfpathlineto{\pgfqpoint{2.797754in}{1.243552in}}%
\pgfpathlineto{\pgfqpoint{2.803271in}{1.241265in}}%
\pgfpathlineto{\pgfqpoint{2.812107in}{1.229451in}}%
\pgfpathlineto{\pgfqpoint{2.818035in}{1.225335in}}%
\pgfpathlineto{\pgfqpoint{2.816028in}{1.224152in}}%
\pgfpathlineto{\pgfqpoint{2.811229in}{1.209761in}}%
\pgfpathlineto{\pgfqpoint{2.801838in}{1.199960in}}%
\pgfpathlineto{\pgfqpoint{2.794605in}{1.198401in}}%
\pgfpathlineto{\pgfqpoint{2.787570in}{1.203491in}}%
\pgfpathlineto{\pgfqpoint{2.778873in}{1.207287in}}%
\pgfpathlineto{\pgfqpoint{2.776238in}{1.213801in}}%
\pgfpathclose%
\pgfusepath{fill}%
\end{pgfscope}%
\begin{pgfscope}%
\pgfpathrectangle{\pgfqpoint{0.100000in}{0.100000in}}{\pgfqpoint{3.420221in}{2.189500in}}%
\pgfusepath{clip}%
\pgfsetbuttcap%
\pgfsetmiterjoin%
\definecolor{currentfill}{rgb}{0.000000,0.294118,0.852941}%
\pgfsetfillcolor{currentfill}%
\pgfsetlinewidth{0.000000pt}%
\definecolor{currentstroke}{rgb}{0.000000,0.000000,0.000000}%
\pgfsetstrokecolor{currentstroke}%
\pgfsetstrokeopacity{0.000000}%
\pgfsetdash{}{0pt}%
\pgfpathmoveto{\pgfqpoint{2.867475in}{0.531354in}}%
\pgfpathlineto{\pgfqpoint{2.865300in}{0.547175in}}%
\pgfpathlineto{\pgfqpoint{2.854622in}{0.549533in}}%
\pgfpathlineto{\pgfqpoint{2.849939in}{0.556223in}}%
\pgfpathlineto{\pgfqpoint{2.854800in}{0.566437in}}%
\pgfpathlineto{\pgfqpoint{2.843629in}{0.577826in}}%
\pgfpathlineto{\pgfqpoint{2.886626in}{0.584033in}}%
\pgfpathlineto{\pgfqpoint{2.886970in}{0.590796in}}%
\pgfpathlineto{\pgfqpoint{2.884207in}{0.608000in}}%
\pgfpathlineto{\pgfqpoint{2.886992in}{0.602738in}}%
\pgfpathlineto{\pgfqpoint{2.892525in}{0.601567in}}%
\pgfpathlineto{\pgfqpoint{2.895556in}{0.595802in}}%
\pgfpathlineto{\pgfqpoint{2.906249in}{0.589014in}}%
\pgfpathlineto{\pgfqpoint{2.910342in}{0.577922in}}%
\pgfpathlineto{\pgfqpoint{2.916566in}{0.578690in}}%
\pgfpathlineto{\pgfqpoint{2.922950in}{0.576569in}}%
\pgfpathlineto{\pgfqpoint{2.925488in}{0.579457in}}%
\pgfpathlineto{\pgfqpoint{2.934858in}{0.564607in}}%
\pgfpathlineto{\pgfqpoint{2.942819in}{0.557881in}}%
\pgfpathlineto{\pgfqpoint{2.942658in}{0.551963in}}%
\pgfpathlineto{\pgfqpoint{2.946201in}{0.545912in}}%
\pgfpathlineto{\pgfqpoint{2.953850in}{0.492665in}}%
\pgfpathlineto{\pgfqpoint{2.935857in}{0.489944in}}%
\pgfpathlineto{\pgfqpoint{2.934427in}{0.495471in}}%
\pgfpathlineto{\pgfqpoint{2.928816in}{0.503272in}}%
\pgfpathlineto{\pgfqpoint{2.922481in}{0.504860in}}%
\pgfpathlineto{\pgfqpoint{2.921269in}{0.509192in}}%
\pgfpathlineto{\pgfqpoint{2.915296in}{0.515661in}}%
\pgfpathlineto{\pgfqpoint{2.904675in}{0.523772in}}%
\pgfpathlineto{\pgfqpoint{2.901156in}{0.532230in}}%
\pgfpathlineto{\pgfqpoint{2.894535in}{0.531236in}}%
\pgfpathlineto{\pgfqpoint{2.893550in}{0.537863in}}%
\pgfpathlineto{\pgfqpoint{2.873642in}{0.534802in}}%
\pgfpathclose%
\pgfusepath{fill}%
\end{pgfscope}%
\begin{pgfscope}%
\pgfpathrectangle{\pgfqpoint{0.100000in}{0.100000in}}{\pgfqpoint{3.420221in}{2.189500in}}%
\pgfusepath{clip}%
\pgfsetbuttcap%
\pgfsetmiterjoin%
\definecolor{currentfill}{rgb}{0.000000,0.349020,0.825490}%
\pgfsetfillcolor{currentfill}%
\pgfsetlinewidth{0.000000pt}%
\definecolor{currentstroke}{rgb}{0.000000,0.000000,0.000000}%
\pgfsetstrokecolor{currentstroke}%
\pgfsetstrokeopacity{0.000000}%
\pgfsetdash{}{0pt}%
\pgfpathmoveto{\pgfqpoint{1.639164in}{1.137076in}}%
\pgfpathlineto{\pgfqpoint{1.646404in}{1.136614in}}%
\pgfpathlineto{\pgfqpoint{1.643444in}{1.099348in}}%
\pgfpathlineto{\pgfqpoint{1.603165in}{1.102134in}}%
\pgfpathlineto{\pgfqpoint{1.600767in}{1.068907in}}%
\pgfpathlineto{\pgfqpoint{1.568137in}{1.071376in}}%
\pgfpathlineto{\pgfqpoint{1.570798in}{1.104653in}}%
\pgfpathlineto{\pgfqpoint{1.520342in}{1.108878in}}%
\pgfpathlineto{\pgfqpoint{1.523651in}{1.146199in}}%
\pgfpathlineto{\pgfqpoint{1.541755in}{1.144259in}}%
\pgfpathlineto{\pgfqpoint{1.580884in}{1.140923in}}%
\pgfpathlineto{\pgfqpoint{1.583218in}{1.170522in}}%
\pgfpathlineto{\pgfqpoint{1.612016in}{1.168311in}}%
\pgfpathlineto{\pgfqpoint{1.609973in}{1.138918in}}%
\pgfpathclose%
\pgfusepath{fill}%
\end{pgfscope}%
\begin{pgfscope}%
\pgfpathrectangle{\pgfqpoint{0.100000in}{0.100000in}}{\pgfqpoint{3.420221in}{2.189500in}}%
\pgfusepath{clip}%
\pgfsetbuttcap%
\pgfsetmiterjoin%
\definecolor{currentfill}{rgb}{0.000000,0.521569,0.739216}%
\pgfsetfillcolor{currentfill}%
\pgfsetlinewidth{0.000000pt}%
\definecolor{currentstroke}{rgb}{0.000000,0.000000,0.000000}%
\pgfsetstrokecolor{currentstroke}%
\pgfsetstrokeopacity{0.000000}%
\pgfsetdash{}{0pt}%
\pgfpathmoveto{\pgfqpoint{2.509846in}{1.173025in}}%
\pgfpathlineto{\pgfqpoint{2.515047in}{1.162139in}}%
\pgfpathlineto{\pgfqpoint{2.507335in}{1.161916in}}%
\pgfpathlineto{\pgfqpoint{2.502423in}{1.159672in}}%
\pgfpathlineto{\pgfqpoint{2.499852in}{1.147904in}}%
\pgfpathlineto{\pgfqpoint{2.489296in}{1.151218in}}%
\pgfpathlineto{\pgfqpoint{2.476867in}{1.150453in}}%
\pgfpathlineto{\pgfqpoint{2.465269in}{1.147902in}}%
\pgfpathlineto{\pgfqpoint{2.459552in}{1.156183in}}%
\pgfpathlineto{\pgfqpoint{2.455730in}{1.163251in}}%
\pgfpathlineto{\pgfqpoint{2.460094in}{1.172603in}}%
\pgfpathlineto{\pgfqpoint{2.456718in}{1.176229in}}%
\pgfpathlineto{\pgfqpoint{2.455493in}{1.185294in}}%
\pgfpathlineto{\pgfqpoint{2.447988in}{1.188551in}}%
\pgfpathlineto{\pgfqpoint{2.456043in}{1.205634in}}%
\pgfpathlineto{\pgfqpoint{2.456709in}{1.211714in}}%
\pgfpathlineto{\pgfqpoint{2.471182in}{1.206165in}}%
\pgfpathlineto{\pgfqpoint{2.471851in}{1.212375in}}%
\pgfpathlineto{\pgfqpoint{2.485977in}{1.221108in}}%
\pgfpathlineto{\pgfqpoint{2.489382in}{1.215287in}}%
\pgfpathlineto{\pgfqpoint{2.495231in}{1.215889in}}%
\pgfpathlineto{\pgfqpoint{2.495202in}{1.210912in}}%
\pgfpathlineto{\pgfqpoint{2.494206in}{1.207082in}}%
\pgfpathlineto{\pgfqpoint{2.498444in}{1.197106in}}%
\pgfpathlineto{\pgfqpoint{2.496901in}{1.188062in}}%
\pgfpathlineto{\pgfqpoint{2.500840in}{1.177777in}}%
\pgfpathlineto{\pgfqpoint{2.508477in}{1.176013in}}%
\pgfpathclose%
\pgfusepath{fill}%
\end{pgfscope}%
\begin{pgfscope}%
\pgfpathrectangle{\pgfqpoint{0.100000in}{0.100000in}}{\pgfqpoint{3.420221in}{2.189500in}}%
\pgfusepath{clip}%
\pgfsetbuttcap%
\pgfsetmiterjoin%
\definecolor{currentfill}{rgb}{0.000000,0.439216,0.780392}%
\pgfsetfillcolor{currentfill}%
\pgfsetlinewidth{0.000000pt}%
\definecolor{currentstroke}{rgb}{0.000000,0.000000,0.000000}%
\pgfsetstrokecolor{currentstroke}%
\pgfsetstrokeopacity{0.000000}%
\pgfsetdash{}{0pt}%
\pgfpathmoveto{\pgfqpoint{2.108054in}{1.639769in}}%
\pgfpathlineto{\pgfqpoint{2.127337in}{1.640225in}}%
\pgfpathlineto{\pgfqpoint{2.140428in}{1.639493in}}%
\pgfpathlineto{\pgfqpoint{2.141137in}{1.614429in}}%
\pgfpathlineto{\pgfqpoint{2.109782in}{1.613581in}}%
\pgfpathlineto{\pgfqpoint{2.075764in}{1.612957in}}%
\pgfpathlineto{\pgfqpoint{2.075381in}{1.639155in}}%
\pgfpathclose%
\pgfusepath{fill}%
\end{pgfscope}%
\begin{pgfscope}%
\pgfpathrectangle{\pgfqpoint{0.100000in}{0.100000in}}{\pgfqpoint{3.420221in}{2.189500in}}%
\pgfusepath{clip}%
\pgfsetbuttcap%
\pgfsetmiterjoin%
\definecolor{currentfill}{rgb}{0.000000,0.015686,0.992157}%
\pgfsetfillcolor{currentfill}%
\pgfsetlinewidth{0.000000pt}%
\definecolor{currentstroke}{rgb}{0.000000,0.000000,0.000000}%
\pgfsetstrokecolor{currentstroke}%
\pgfsetstrokeopacity{0.000000}%
\pgfsetdash{}{0pt}%
\pgfpathmoveto{\pgfqpoint{1.891017in}{1.558192in}}%
\pgfpathlineto{\pgfqpoint{1.883118in}{1.561255in}}%
\pgfpathlineto{\pgfqpoint{1.883782in}{1.589055in}}%
\pgfpathlineto{\pgfqpoint{1.870695in}{1.589338in}}%
\pgfpathlineto{\pgfqpoint{1.871228in}{1.614140in}}%
\pgfpathlineto{\pgfqpoint{1.860027in}{1.614455in}}%
\pgfpathlineto{\pgfqpoint{1.860708in}{1.640699in}}%
\pgfpathlineto{\pgfqpoint{1.899719in}{1.639679in}}%
\pgfpathlineto{\pgfqpoint{1.923322in}{1.639349in}}%
\pgfpathlineto{\pgfqpoint{1.922914in}{1.613106in}}%
\pgfpathlineto{\pgfqpoint{1.914964in}{1.613242in}}%
\pgfpathlineto{\pgfqpoint{1.915178in}{1.608195in}}%
\pgfpathlineto{\pgfqpoint{1.918897in}{1.605279in}}%
\pgfpathlineto{\pgfqpoint{1.918445in}{1.598121in}}%
\pgfpathlineto{\pgfqpoint{1.921328in}{1.592176in}}%
\pgfpathlineto{\pgfqpoint{1.922357in}{1.581778in}}%
\pgfpathlineto{\pgfqpoint{1.903076in}{1.582126in}}%
\pgfpathlineto{\pgfqpoint{1.902479in}{1.553657in}}%
\pgfpathclose%
\pgfusepath{fill}%
\end{pgfscope}%
\begin{pgfscope}%
\pgfpathrectangle{\pgfqpoint{0.100000in}{0.100000in}}{\pgfqpoint{3.420221in}{2.189500in}}%
\pgfusepath{clip}%
\pgfsetbuttcap%
\pgfsetmiterjoin%
\definecolor{currentfill}{rgb}{0.000000,0.329412,0.835294}%
\pgfsetfillcolor{currentfill}%
\pgfsetlinewidth{0.000000pt}%
\definecolor{currentstroke}{rgb}{0.000000,0.000000,0.000000}%
\pgfsetstrokecolor{currentstroke}%
\pgfsetstrokeopacity{0.000000}%
\pgfsetdash{}{0pt}%
\pgfpathmoveto{\pgfqpoint{1.918109in}{1.350651in}}%
\pgfpathlineto{\pgfqpoint{1.930990in}{1.350438in}}%
\pgfpathlineto{\pgfqpoint{1.930558in}{1.317893in}}%
\pgfpathlineto{\pgfqpoint{1.865335in}{1.319239in}}%
\pgfpathlineto{\pgfqpoint{1.866155in}{1.351828in}}%
\pgfpathclose%
\pgfusepath{fill}%
\end{pgfscope}%
\begin{pgfscope}%
\pgfpathrectangle{\pgfqpoint{0.100000in}{0.100000in}}{\pgfqpoint{3.420221in}{2.189500in}}%
\pgfusepath{clip}%
\pgfsetbuttcap%
\pgfsetmiterjoin%
\definecolor{currentfill}{rgb}{0.000000,0.490196,0.754902}%
\pgfsetfillcolor{currentfill}%
\pgfsetlinewidth{0.000000pt}%
\definecolor{currentstroke}{rgb}{0.000000,0.000000,0.000000}%
\pgfsetstrokecolor{currentstroke}%
\pgfsetstrokeopacity{0.000000}%
\pgfsetdash{}{0pt}%
\pgfpathmoveto{\pgfqpoint{2.276776in}{1.370938in}}%
\pgfpathlineto{\pgfqpoint{2.262354in}{1.370567in}}%
\pgfpathlineto{\pgfqpoint{2.262093in}{1.377123in}}%
\pgfpathlineto{\pgfqpoint{2.235823in}{1.376468in}}%
\pgfpathlineto{\pgfqpoint{2.234943in}{1.403048in}}%
\pgfpathlineto{\pgfqpoint{2.241474in}{1.403083in}}%
\pgfpathlineto{\pgfqpoint{2.240214in}{1.435516in}}%
\pgfpathlineto{\pgfqpoint{2.259756in}{1.436105in}}%
\pgfpathlineto{\pgfqpoint{2.259526in}{1.442641in}}%
\pgfpathlineto{\pgfqpoint{2.285074in}{1.443849in}}%
\pgfpathlineto{\pgfqpoint{2.285737in}{1.430772in}}%
\pgfpathlineto{\pgfqpoint{2.286981in}{1.404646in}}%
\pgfpathlineto{\pgfqpoint{2.293551in}{1.404925in}}%
\pgfpathlineto{\pgfqpoint{2.294081in}{1.396589in}}%
\pgfpathlineto{\pgfqpoint{2.292313in}{1.391397in}}%
\pgfpathlineto{\pgfqpoint{2.285232in}{1.385733in}}%
\pgfpathlineto{\pgfqpoint{2.281084in}{1.374849in}}%
\pgfpathclose%
\pgfusepath{fill}%
\end{pgfscope}%
\begin{pgfscope}%
\pgfpathrectangle{\pgfqpoint{0.100000in}{0.100000in}}{\pgfqpoint{3.420221in}{2.189500in}}%
\pgfusepath{clip}%
\pgfsetbuttcap%
\pgfsetmiterjoin%
\definecolor{currentfill}{rgb}{0.000000,0.396078,0.801961}%
\pgfsetfillcolor{currentfill}%
\pgfsetlinewidth{0.000000pt}%
\definecolor{currentstroke}{rgb}{0.000000,0.000000,0.000000}%
\pgfsetstrokecolor{currentstroke}%
\pgfsetstrokeopacity{0.000000}%
\pgfsetdash{}{0pt}%
\pgfpathmoveto{\pgfqpoint{2.592758in}{1.198374in}}%
\pgfpathlineto{\pgfqpoint{2.593547in}{1.188427in}}%
\pgfpathlineto{\pgfqpoint{2.585616in}{1.191554in}}%
\pgfpathlineto{\pgfqpoint{2.581575in}{1.187757in}}%
\pgfpathlineto{\pgfqpoint{2.575075in}{1.190934in}}%
\pgfpathlineto{\pgfqpoint{2.568187in}{1.189724in}}%
\pgfpathlineto{\pgfqpoint{2.565682in}{1.195350in}}%
\pgfpathlineto{\pgfqpoint{2.559193in}{1.195010in}}%
\pgfpathlineto{\pgfqpoint{2.558961in}{1.205950in}}%
\pgfpathlineto{\pgfqpoint{2.556238in}{1.205886in}}%
\pgfpathlineto{\pgfqpoint{2.549465in}{1.213876in}}%
\pgfpathlineto{\pgfqpoint{2.554927in}{1.216131in}}%
\pgfpathlineto{\pgfqpoint{2.562718in}{1.228735in}}%
\pgfpathlineto{\pgfqpoint{2.568351in}{1.225486in}}%
\pgfpathlineto{\pgfqpoint{2.581739in}{1.229328in}}%
\pgfpathlineto{\pgfqpoint{2.583224in}{1.223915in}}%
\pgfpathlineto{\pgfqpoint{2.590455in}{1.224248in}}%
\pgfpathlineto{\pgfqpoint{2.592522in}{1.221667in}}%
\pgfpathclose%
\pgfusepath{fill}%
\end{pgfscope}%
\begin{pgfscope}%
\pgfpathrectangle{\pgfqpoint{0.100000in}{0.100000in}}{\pgfqpoint{3.420221in}{2.189500in}}%
\pgfusepath{clip}%
\pgfsetbuttcap%
\pgfsetmiterjoin%
\definecolor{currentfill}{rgb}{0.000000,0.501961,0.749020}%
\pgfsetfillcolor{currentfill}%
\pgfsetlinewidth{0.000000pt}%
\definecolor{currentstroke}{rgb}{0.000000,0.000000,0.000000}%
\pgfsetstrokecolor{currentstroke}%
\pgfsetstrokeopacity{0.000000}%
\pgfsetdash{}{0pt}%
\pgfpathmoveto{\pgfqpoint{2.723978in}{0.963056in}}%
\pgfpathlineto{\pgfqpoint{2.727017in}{0.966485in}}%
\pgfpathlineto{\pgfqpoint{2.741060in}{0.965503in}}%
\pgfpathlineto{\pgfqpoint{2.740048in}{0.961268in}}%
\pgfpathlineto{\pgfqpoint{2.751535in}{0.950789in}}%
\pgfpathlineto{\pgfqpoint{2.762024in}{0.948882in}}%
\pgfpathlineto{\pgfqpoint{2.756679in}{0.944479in}}%
\pgfpathlineto{\pgfqpoint{2.751051in}{0.932978in}}%
\pgfpathlineto{\pgfqpoint{2.750216in}{0.926076in}}%
\pgfpathlineto{\pgfqpoint{2.743710in}{0.925776in}}%
\pgfpathlineto{\pgfqpoint{2.733043in}{0.929172in}}%
\pgfpathlineto{\pgfqpoint{2.725880in}{0.923443in}}%
\pgfpathlineto{\pgfqpoint{2.720064in}{0.930845in}}%
\pgfpathlineto{\pgfqpoint{2.707030in}{0.937684in}}%
\pgfpathlineto{\pgfqpoint{2.701750in}{0.935581in}}%
\pgfpathlineto{\pgfqpoint{2.694308in}{0.942676in}}%
\pgfpathlineto{\pgfqpoint{2.696353in}{0.952258in}}%
\pgfpathlineto{\pgfqpoint{2.700168in}{0.948015in}}%
\pgfpathlineto{\pgfqpoint{2.712974in}{0.947085in}}%
\pgfpathlineto{\pgfqpoint{2.715193in}{0.942368in}}%
\pgfpathlineto{\pgfqpoint{2.723685in}{0.949844in}}%
\pgfpathlineto{\pgfqpoint{2.724901in}{0.955002in}}%
\pgfpathlineto{\pgfqpoint{2.721313in}{0.960704in}}%
\pgfpathclose%
\pgfusepath{fill}%
\end{pgfscope}%
\begin{pgfscope}%
\pgfpathrectangle{\pgfqpoint{0.100000in}{0.100000in}}{\pgfqpoint{3.420221in}{2.189500in}}%
\pgfusepath{clip}%
\pgfsetbuttcap%
\pgfsetmiterjoin%
\definecolor{currentfill}{rgb}{0.000000,0.266667,0.866667}%
\pgfsetfillcolor{currentfill}%
\pgfsetlinewidth{0.000000pt}%
\definecolor{currentstroke}{rgb}{0.000000,0.000000,0.000000}%
\pgfsetstrokecolor{currentstroke}%
\pgfsetstrokeopacity{0.000000}%
\pgfsetdash{}{0pt}%
\pgfpathmoveto{\pgfqpoint{1.766442in}{1.511909in}}%
\pgfpathlineto{\pgfqpoint{1.741283in}{1.513010in}}%
\pgfpathlineto{\pgfqpoint{1.714561in}{1.514444in}}%
\pgfpathlineto{\pgfqpoint{1.715183in}{1.540531in}}%
\pgfpathlineto{\pgfqpoint{1.716664in}{1.566578in}}%
\pgfpathlineto{\pgfqpoint{1.716764in}{1.582932in}}%
\pgfpathlineto{\pgfqpoint{1.768544in}{1.580207in}}%
\pgfpathclose%
\pgfusepath{fill}%
\end{pgfscope}%
\begin{pgfscope}%
\pgfpathrectangle{\pgfqpoint{0.100000in}{0.100000in}}{\pgfqpoint{3.420221in}{2.189500in}}%
\pgfusepath{clip}%
\pgfsetbuttcap%
\pgfsetmiterjoin%
\definecolor{currentfill}{rgb}{0.000000,0.105882,0.947059}%
\pgfsetfillcolor{currentfill}%
\pgfsetlinewidth{0.000000pt}%
\definecolor{currentstroke}{rgb}{0.000000,0.000000,0.000000}%
\pgfsetstrokecolor{currentstroke}%
\pgfsetstrokeopacity{0.000000}%
\pgfsetdash{}{0pt}%
\pgfpathmoveto{\pgfqpoint{1.938486in}{1.490187in}}%
\pgfpathlineto{\pgfqpoint{1.942581in}{1.484796in}}%
\pgfpathlineto{\pgfqpoint{1.939287in}{1.476419in}}%
\pgfpathlineto{\pgfqpoint{1.921199in}{1.476739in}}%
\pgfpathlineto{\pgfqpoint{1.921296in}{1.481092in}}%
\pgfpathlineto{\pgfqpoint{1.895456in}{1.481643in}}%
\pgfpathlineto{\pgfqpoint{1.889090in}{1.481804in}}%
\pgfpathlineto{\pgfqpoint{1.889699in}{1.507861in}}%
\pgfpathlineto{\pgfqpoint{1.900574in}{1.507631in}}%
\pgfpathlineto{\pgfqpoint{1.900916in}{1.520671in}}%
\pgfpathlineto{\pgfqpoint{1.906258in}{1.521595in}}%
\pgfpathlineto{\pgfqpoint{1.926810in}{1.521132in}}%
\pgfpathlineto{\pgfqpoint{1.927364in}{1.512541in}}%
\pgfpathlineto{\pgfqpoint{1.935640in}{1.499098in}}%
\pgfpathlineto{\pgfqpoint{1.939074in}{1.498184in}}%
\pgfpathclose%
\pgfusepath{fill}%
\end{pgfscope}%
\begin{pgfscope}%
\pgfpathrectangle{\pgfqpoint{0.100000in}{0.100000in}}{\pgfqpoint{3.420221in}{2.189500in}}%
\pgfusepath{clip}%
\pgfsetbuttcap%
\pgfsetmiterjoin%
\definecolor{currentfill}{rgb}{0.000000,0.160784,0.919608}%
\pgfsetfillcolor{currentfill}%
\pgfsetlinewidth{0.000000pt}%
\definecolor{currentstroke}{rgb}{0.000000,0.000000,0.000000}%
\pgfsetstrokecolor{currentstroke}%
\pgfsetstrokeopacity{0.000000}%
\pgfsetdash{}{0pt}%
\pgfpathmoveto{\pgfqpoint{2.146061in}{1.692855in}}%
\pgfpathlineto{\pgfqpoint{2.135150in}{1.695118in}}%
\pgfpathlineto{\pgfqpoint{2.129435in}{1.698288in}}%
\pgfpathlineto{\pgfqpoint{2.120340in}{1.698846in}}%
\pgfpathlineto{\pgfqpoint{2.120524in}{1.692360in}}%
\pgfpathlineto{\pgfqpoint{2.107323in}{1.689965in}}%
\pgfpathlineto{\pgfqpoint{2.107400in}{1.686648in}}%
\pgfpathlineto{\pgfqpoint{2.094372in}{1.686390in}}%
\pgfpathlineto{\pgfqpoint{2.094293in}{1.691816in}}%
\pgfpathlineto{\pgfqpoint{2.081265in}{1.691557in}}%
\pgfpathlineto{\pgfqpoint{2.060534in}{1.691242in}}%
\pgfpathlineto{\pgfqpoint{2.061565in}{1.704354in}}%
\pgfpathlineto{\pgfqpoint{2.055057in}{1.704294in}}%
\pgfpathlineto{\pgfqpoint{2.054739in}{1.723973in}}%
\pgfpathlineto{\pgfqpoint{2.041756in}{1.723906in}}%
\pgfpathlineto{\pgfqpoint{2.041308in}{1.746848in}}%
\pgfpathlineto{\pgfqpoint{2.048992in}{1.750305in}}%
\pgfpathlineto{\pgfqpoint{2.052546in}{1.757374in}}%
\pgfpathlineto{\pgfqpoint{2.068172in}{1.747944in}}%
\pgfpathlineto{\pgfqpoint{2.077326in}{1.747872in}}%
\pgfpathlineto{\pgfqpoint{2.081054in}{1.744447in}}%
\pgfpathlineto{\pgfqpoint{2.080363in}{1.781293in}}%
\pgfpathlineto{\pgfqpoint{2.115787in}{1.781674in}}%
\pgfpathlineto{\pgfqpoint{2.113810in}{1.769967in}}%
\pgfpathlineto{\pgfqpoint{2.119818in}{1.769466in}}%
\pgfpathlineto{\pgfqpoint{2.126546in}{1.760332in}}%
\pgfpathlineto{\pgfqpoint{2.126468in}{1.756930in}}%
\pgfpathlineto{\pgfqpoint{2.120938in}{1.748680in}}%
\pgfpathlineto{\pgfqpoint{2.121127in}{1.742548in}}%
\pgfpathlineto{\pgfqpoint{2.153048in}{1.743437in}}%
\pgfpathlineto{\pgfqpoint{2.154340in}{1.736830in}}%
\pgfpathlineto{\pgfqpoint{2.155763in}{1.693044in}}%
\pgfpathclose%
\pgfusepath{fill}%
\end{pgfscope}%
\begin{pgfscope}%
\pgfpathrectangle{\pgfqpoint{0.100000in}{0.100000in}}{\pgfqpoint{3.420221in}{2.189500in}}%
\pgfusepath{clip}%
\pgfsetbuttcap%
\pgfsetmiterjoin%
\definecolor{currentfill}{rgb}{0.000000,0.423529,0.788235}%
\pgfsetfillcolor{currentfill}%
\pgfsetlinewidth{0.000000pt}%
\definecolor{currentstroke}{rgb}{0.000000,0.000000,0.000000}%
\pgfsetstrokecolor{currentstroke}%
\pgfsetstrokeopacity{0.000000}%
\pgfsetdash{}{0pt}%
\pgfpathmoveto{\pgfqpoint{2.303644in}{0.681420in}}%
\pgfpathlineto{\pgfqpoint{2.310780in}{0.660653in}}%
\pgfpathlineto{\pgfqpoint{2.313459in}{0.623097in}}%
\pgfpathlineto{\pgfqpoint{2.309615in}{0.628155in}}%
\pgfpathlineto{\pgfqpoint{2.298489in}{0.625745in}}%
\pgfpathlineto{\pgfqpoint{2.294751in}{0.621161in}}%
\pgfpathlineto{\pgfqpoint{2.287592in}{0.617403in}}%
\pgfpathlineto{\pgfqpoint{2.269861in}{0.612543in}}%
\pgfpathlineto{\pgfqpoint{2.266861in}{0.608853in}}%
\pgfpathlineto{\pgfqpoint{2.264061in}{0.609692in}}%
\pgfpathlineto{\pgfqpoint{2.249902in}{0.605006in}}%
\pgfpathlineto{\pgfqpoint{2.233966in}{0.610219in}}%
\pgfpathlineto{\pgfqpoint{2.232841in}{0.619809in}}%
\pgfpathlineto{\pgfqpoint{2.227083in}{0.620533in}}%
\pgfpathlineto{\pgfqpoint{2.223078in}{0.625539in}}%
\pgfpathlineto{\pgfqpoint{2.221461in}{0.635545in}}%
\pgfpathlineto{\pgfqpoint{2.215029in}{0.640276in}}%
\pgfpathlineto{\pgfqpoint{2.213484in}{0.644465in}}%
\pgfpathlineto{\pgfqpoint{2.214650in}{0.654922in}}%
\pgfpathlineto{\pgfqpoint{2.208955in}{0.665805in}}%
\pgfpathlineto{\pgfqpoint{2.209245in}{0.675376in}}%
\pgfpathlineto{\pgfqpoint{2.214358in}{0.680846in}}%
\pgfpathlineto{\pgfqpoint{2.220108in}{0.677662in}}%
\pgfpathlineto{\pgfqpoint{2.272640in}{0.679838in}}%
\pgfpathclose%
\pgfusepath{fill}%
\end{pgfscope}%
\begin{pgfscope}%
\pgfpathrectangle{\pgfqpoint{0.100000in}{0.100000in}}{\pgfqpoint{3.420221in}{2.189500in}}%
\pgfusepath{clip}%
\pgfsetbuttcap%
\pgfsetmiterjoin%
\definecolor{currentfill}{rgb}{0.000000,0.435294,0.782353}%
\pgfsetfillcolor{currentfill}%
\pgfsetlinewidth{0.000000pt}%
\definecolor{currentstroke}{rgb}{0.000000,0.000000,0.000000}%
\pgfsetstrokecolor{currentstroke}%
\pgfsetstrokeopacity{0.000000}%
\pgfsetdash{}{0pt}%
\pgfpathmoveto{\pgfqpoint{1.801759in}{1.128650in}}%
\pgfpathlineto{\pgfqpoint{1.762663in}{1.130354in}}%
\pgfpathlineto{\pgfqpoint{1.763905in}{1.159164in}}%
\pgfpathlineto{\pgfqpoint{1.731658in}{1.160443in}}%
\pgfpathlineto{\pgfqpoint{1.732013in}{1.186971in}}%
\pgfpathlineto{\pgfqpoint{1.764329in}{1.185302in}}%
\pgfpathlineto{\pgfqpoint{1.764647in}{1.192197in}}%
\pgfpathlineto{\pgfqpoint{1.796636in}{1.190847in}}%
\pgfpathlineto{\pgfqpoint{1.796820in}{1.183949in}}%
\pgfpathlineto{\pgfqpoint{1.795819in}{1.157836in}}%
\pgfpathlineto{\pgfqpoint{1.802669in}{1.157580in}}%
\pgfpathclose%
\pgfusepath{fill}%
\end{pgfscope}%
\begin{pgfscope}%
\pgfpathrectangle{\pgfqpoint{0.100000in}{0.100000in}}{\pgfqpoint{3.420221in}{2.189500in}}%
\pgfusepath{clip}%
\pgfsetbuttcap%
\pgfsetmiterjoin%
\definecolor{currentfill}{rgb}{0.000000,0.188235,0.905882}%
\pgfsetfillcolor{currentfill}%
\pgfsetlinewidth{0.000000pt}%
\definecolor{currentstroke}{rgb}{0.000000,0.000000,0.000000}%
\pgfsetstrokecolor{currentstroke}%
\pgfsetstrokeopacity{0.000000}%
\pgfsetdash{}{0pt}%
\pgfpathmoveto{\pgfqpoint{2.374277in}{1.629687in}}%
\pgfpathlineto{\pgfqpoint{2.361278in}{1.628752in}}%
\pgfpathlineto{\pgfqpoint{2.360764in}{1.635303in}}%
\pgfpathlineto{\pgfqpoint{2.327879in}{1.633172in}}%
\pgfpathlineto{\pgfqpoint{2.314880in}{1.633082in}}%
\pgfpathlineto{\pgfqpoint{2.314292in}{1.642342in}}%
\pgfpathlineto{\pgfqpoint{2.317951in}{1.650877in}}%
\pgfpathlineto{\pgfqpoint{2.317428in}{1.658857in}}%
\pgfpathlineto{\pgfqpoint{2.294226in}{1.657377in}}%
\pgfpathlineto{\pgfqpoint{2.293049in}{1.677160in}}%
\pgfpathlineto{\pgfqpoint{2.313127in}{1.678231in}}%
\pgfpathlineto{\pgfqpoint{2.311095in}{1.711121in}}%
\pgfpathlineto{\pgfqpoint{2.344150in}{1.713105in}}%
\pgfpathlineto{\pgfqpoint{2.344565in}{1.706507in}}%
\pgfpathlineto{\pgfqpoint{2.363892in}{1.707475in}}%
\pgfpathlineto{\pgfqpoint{2.366799in}{1.707844in}}%
\pgfpathlineto{\pgfqpoint{2.368634in}{1.681962in}}%
\pgfpathlineto{\pgfqpoint{2.376664in}{1.682492in}}%
\pgfpathlineto{\pgfqpoint{2.378754in}{1.656309in}}%
\pgfpathlineto{\pgfqpoint{2.372255in}{1.655800in}}%
\pgfpathclose%
\pgfusepath{fill}%
\end{pgfscope}%
\begin{pgfscope}%
\pgfpathrectangle{\pgfqpoint{0.100000in}{0.100000in}}{\pgfqpoint{3.420221in}{2.189500in}}%
\pgfusepath{clip}%
\pgfsetbuttcap%
\pgfsetmiterjoin%
\definecolor{currentfill}{rgb}{0.000000,0.168627,0.915686}%
\pgfsetfillcolor{currentfill}%
\pgfsetlinewidth{0.000000pt}%
\definecolor{currentstroke}{rgb}{0.000000,0.000000,0.000000}%
\pgfsetstrokecolor{currentstroke}%
\pgfsetstrokeopacity{0.000000}%
\pgfsetdash{}{0pt}%
\pgfpathmoveto{\pgfqpoint{2.988099in}{0.917970in}}%
\pgfpathlineto{\pgfqpoint{2.985243in}{0.909947in}}%
\pgfpathlineto{\pgfqpoint{2.975787in}{0.910212in}}%
\pgfpathlineto{\pgfqpoint{2.969919in}{0.903052in}}%
\pgfpathlineto{\pgfqpoint{2.973103in}{0.900203in}}%
\pgfpathlineto{\pgfqpoint{2.966254in}{0.894690in}}%
\pgfpathlineto{\pgfqpoint{2.965022in}{0.891187in}}%
\pgfpathlineto{\pgfqpoint{2.956930in}{0.884573in}}%
\pgfpathlineto{\pgfqpoint{2.956562in}{0.880384in}}%
\pgfpathlineto{\pgfqpoint{2.949786in}{0.873445in}}%
\pgfpathlineto{\pgfqpoint{2.942784in}{0.870908in}}%
\pgfpathlineto{\pgfqpoint{2.930988in}{0.860367in}}%
\pgfpathlineto{\pgfqpoint{2.921904in}{0.859461in}}%
\pgfpathlineto{\pgfqpoint{2.916233in}{0.864727in}}%
\pgfpathlineto{\pgfqpoint{2.911697in}{0.865179in}}%
\pgfpathlineto{\pgfqpoint{2.906620in}{0.871372in}}%
\pgfpathlineto{\pgfqpoint{2.897367in}{0.872307in}}%
\pgfpathlineto{\pgfqpoint{2.891197in}{0.881189in}}%
\pgfpathlineto{\pgfqpoint{2.885005in}{0.885412in}}%
\pgfpathlineto{\pgfqpoint{2.877600in}{0.893793in}}%
\pgfpathlineto{\pgfqpoint{2.893603in}{0.907760in}}%
\pgfpathlineto{\pgfqpoint{2.910088in}{0.922398in}}%
\pgfpathlineto{\pgfqpoint{2.915712in}{0.917453in}}%
\pgfpathlineto{\pgfqpoint{2.925969in}{0.922296in}}%
\pgfpathlineto{\pgfqpoint{2.926173in}{0.933340in}}%
\pgfpathlineto{\pgfqpoint{2.932992in}{0.938520in}}%
\pgfpathlineto{\pgfqpoint{2.940913in}{0.940144in}}%
\pgfpathlineto{\pgfqpoint{2.957927in}{0.929208in}}%
\pgfpathlineto{\pgfqpoint{2.976134in}{0.925835in}}%
\pgfpathclose%
\pgfusepath{fill}%
\end{pgfscope}%
\begin{pgfscope}%
\pgfpathrectangle{\pgfqpoint{0.100000in}{0.100000in}}{\pgfqpoint{3.420221in}{2.189500in}}%
\pgfusepath{clip}%
\pgfsetbuttcap%
\pgfsetmiterjoin%
\definecolor{currentfill}{rgb}{0.000000,0.243137,0.878431}%
\pgfsetfillcolor{currentfill}%
\pgfsetlinewidth{0.000000pt}%
\definecolor{currentstroke}{rgb}{0.000000,0.000000,0.000000}%
\pgfsetstrokecolor{currentstroke}%
\pgfsetstrokeopacity{0.000000}%
\pgfsetdash{}{0pt}%
\pgfpathmoveto{\pgfqpoint{3.029806in}{1.115037in}}%
\pgfpathlineto{\pgfqpoint{3.026711in}{1.101640in}}%
\pgfpathlineto{\pgfqpoint{3.027731in}{1.096697in}}%
\pgfpathlineto{\pgfqpoint{3.020139in}{1.092579in}}%
\pgfpathlineto{\pgfqpoint{3.013544in}{1.086877in}}%
\pgfpathlineto{\pgfqpoint{3.008560in}{1.086495in}}%
\pgfpathlineto{\pgfqpoint{3.004846in}{1.089773in}}%
\pgfpathlineto{\pgfqpoint{3.001215in}{1.083802in}}%
\pgfpathlineto{\pgfqpoint{2.990246in}{1.083365in}}%
\pgfpathlineto{\pgfqpoint{2.973010in}{1.074654in}}%
\pgfpathlineto{\pgfqpoint{2.964354in}{1.078920in}}%
\pgfpathlineto{\pgfqpoint{2.953638in}{1.092737in}}%
\pgfpathlineto{\pgfqpoint{2.953469in}{1.096061in}}%
\pgfpathlineto{\pgfqpoint{2.941115in}{1.093711in}}%
\pgfpathlineto{\pgfqpoint{2.937751in}{1.118126in}}%
\pgfpathlineto{\pgfqpoint{2.956026in}{1.121353in}}%
\pgfpathlineto{\pgfqpoint{2.953579in}{1.125751in}}%
\pgfpathlineto{\pgfqpoint{2.949686in}{1.150689in}}%
\pgfpathlineto{\pgfqpoint{2.955859in}{1.151639in}}%
\pgfpathlineto{\pgfqpoint{2.952898in}{1.173935in}}%
\pgfpathlineto{\pgfqpoint{2.993264in}{1.181265in}}%
\pgfpathlineto{\pgfqpoint{2.991240in}{1.174168in}}%
\pgfpathlineto{\pgfqpoint{2.995070in}{1.154170in}}%
\pgfpathlineto{\pgfqpoint{3.001001in}{1.154259in}}%
\pgfpathlineto{\pgfqpoint{3.005946in}{1.162530in}}%
\pgfpathlineto{\pgfqpoint{3.016565in}{1.163005in}}%
\pgfpathlineto{\pgfqpoint{3.024677in}{1.161202in}}%
\pgfpathlineto{\pgfqpoint{3.015195in}{1.130027in}}%
\pgfpathlineto{\pgfqpoint{3.020917in}{1.124162in}}%
\pgfpathlineto{\pgfqpoint{3.025904in}{1.115649in}}%
\pgfpathclose%
\pgfusepath{fill}%
\end{pgfscope}%
\begin{pgfscope}%
\pgfpathrectangle{\pgfqpoint{0.100000in}{0.100000in}}{\pgfqpoint{3.420221in}{2.189500in}}%
\pgfusepath{clip}%
\pgfsetbuttcap%
\pgfsetmiterjoin%
\definecolor{currentfill}{rgb}{0.000000,0.239216,0.880392}%
\pgfsetfillcolor{currentfill}%
\pgfsetlinewidth{0.000000pt}%
\definecolor{currentstroke}{rgb}{0.000000,0.000000,0.000000}%
\pgfsetstrokecolor{currentstroke}%
\pgfsetstrokeopacity{0.000000}%
\pgfsetdash{}{0pt}%
\pgfpathmoveto{\pgfqpoint{1.788305in}{1.354451in}}%
\pgfpathlineto{\pgfqpoint{1.762349in}{1.355562in}}%
\pgfpathlineto{\pgfqpoint{1.764596in}{1.404862in}}%
\pgfpathlineto{\pgfqpoint{1.751120in}{1.406270in}}%
\pgfpathlineto{\pgfqpoint{1.751998in}{1.434416in}}%
\pgfpathlineto{\pgfqpoint{1.764328in}{1.433867in}}%
\pgfpathlineto{\pgfqpoint{1.765780in}{1.459855in}}%
\pgfpathlineto{\pgfqpoint{1.791568in}{1.458730in}}%
\pgfpathlineto{\pgfqpoint{1.790258in}{1.432707in}}%
\pgfpathlineto{\pgfqpoint{1.791757in}{1.432638in}}%
\pgfpathlineto{\pgfqpoint{1.790722in}{1.406614in}}%
\pgfpathclose%
\pgfusepath{fill}%
\end{pgfscope}%
\begin{pgfscope}%
\pgfpathrectangle{\pgfqpoint{0.100000in}{0.100000in}}{\pgfqpoint{3.420221in}{2.189500in}}%
\pgfusepath{clip}%
\pgfsetbuttcap%
\pgfsetmiterjoin%
\definecolor{currentfill}{rgb}{0.000000,0.615686,0.692157}%
\pgfsetfillcolor{currentfill}%
\pgfsetlinewidth{0.000000pt}%
\definecolor{currentstroke}{rgb}{0.000000,0.000000,0.000000}%
\pgfsetstrokecolor{currentstroke}%
\pgfsetstrokeopacity{0.000000}%
\pgfsetdash{}{0pt}%
\pgfpathmoveto{\pgfqpoint{0.797880in}{2.108980in}}%
\pgfpathlineto{\pgfqpoint{0.791910in}{2.110578in}}%
\pgfpathlineto{\pgfqpoint{0.785130in}{2.106594in}}%
\pgfpathlineto{\pgfqpoint{0.783377in}{2.100236in}}%
\pgfpathlineto{\pgfqpoint{0.777689in}{2.096102in}}%
\pgfpathlineto{\pgfqpoint{0.773250in}{2.088207in}}%
\pgfpathlineto{\pgfqpoint{0.770125in}{2.089076in}}%
\pgfpathlineto{\pgfqpoint{0.762120in}{2.084615in}}%
\pgfpathlineto{\pgfqpoint{0.760341in}{2.078322in}}%
\pgfpathlineto{\pgfqpoint{0.743371in}{2.082741in}}%
\pgfpathlineto{\pgfqpoint{0.739753in}{2.078024in}}%
\pgfpathlineto{\pgfqpoint{0.734598in}{2.075981in}}%
\pgfpathlineto{\pgfqpoint{0.732427in}{2.068726in}}%
\pgfpathlineto{\pgfqpoint{0.728971in}{2.073016in}}%
\pgfpathlineto{\pgfqpoint{0.714388in}{2.077072in}}%
\pgfpathlineto{\pgfqpoint{0.709387in}{2.083201in}}%
\pgfpathlineto{\pgfqpoint{0.702501in}{2.087001in}}%
\pgfpathlineto{\pgfqpoint{0.700209in}{2.091621in}}%
\pgfpathlineto{\pgfqpoint{0.691651in}{2.097276in}}%
\pgfpathlineto{\pgfqpoint{0.689924in}{2.104506in}}%
\pgfpathlineto{\pgfqpoint{0.685861in}{2.111912in}}%
\pgfpathlineto{\pgfqpoint{0.690480in}{2.119171in}}%
\pgfpathlineto{\pgfqpoint{0.689070in}{2.134810in}}%
\pgfpathlineto{\pgfqpoint{0.694594in}{2.144668in}}%
\pgfpathlineto{\pgfqpoint{0.704909in}{2.147368in}}%
\pgfpathlineto{\pgfqpoint{0.708077in}{2.152301in}}%
\pgfpathlineto{\pgfqpoint{0.703739in}{2.164769in}}%
\pgfpathlineto{\pgfqpoint{0.706856in}{2.173979in}}%
\pgfpathlineto{\pgfqpoint{0.718709in}{2.177531in}}%
\pgfpathlineto{\pgfqpoint{0.725860in}{2.174090in}}%
\pgfpathlineto{\pgfqpoint{0.725746in}{2.167661in}}%
\pgfpathlineto{\pgfqpoint{0.730387in}{2.155398in}}%
\pgfpathlineto{\pgfqpoint{0.736080in}{2.144858in}}%
\pgfpathlineto{\pgfqpoint{0.734545in}{2.142147in}}%
\pgfpathlineto{\pgfqpoint{0.744680in}{2.128230in}}%
\pgfpathlineto{\pgfqpoint{0.745410in}{2.123458in}}%
\pgfpathlineto{\pgfqpoint{0.754359in}{2.120835in}}%
\pgfpathlineto{\pgfqpoint{0.755449in}{2.128147in}}%
\pgfpathlineto{\pgfqpoint{0.763502in}{2.129377in}}%
\pgfpathlineto{\pgfqpoint{0.765691in}{2.120904in}}%
\pgfpathlineto{\pgfqpoint{0.777734in}{2.123614in}}%
\pgfpathlineto{\pgfqpoint{0.782060in}{2.120169in}}%
\pgfpathlineto{\pgfqpoint{0.794397in}{2.124782in}}%
\pgfpathlineto{\pgfqpoint{0.797843in}{2.122472in}}%
\pgfpathclose%
\pgfusepath{fill}%
\end{pgfscope}%
\begin{pgfscope}%
\pgfpathrectangle{\pgfqpoint{0.100000in}{0.100000in}}{\pgfqpoint{3.420221in}{2.189500in}}%
\pgfusepath{clip}%
\pgfsetbuttcap%
\pgfsetmiterjoin%
\definecolor{currentfill}{rgb}{0.000000,0.545098,0.727451}%
\pgfsetfillcolor{currentfill}%
\pgfsetlinewidth{0.000000pt}%
\definecolor{currentstroke}{rgb}{0.000000,0.000000,0.000000}%
\pgfsetstrokecolor{currentstroke}%
\pgfsetstrokeopacity{0.000000}%
\pgfsetdash{}{0pt}%
\pgfpathmoveto{\pgfqpoint{2.360626in}{0.622577in}}%
\pgfpathlineto{\pgfqpoint{2.354474in}{0.625298in}}%
\pgfpathlineto{\pgfqpoint{2.349045in}{0.642328in}}%
\pgfpathlineto{\pgfqpoint{2.343283in}{0.646667in}}%
\pgfpathlineto{\pgfqpoint{2.337584in}{0.657686in}}%
\pgfpathlineto{\pgfqpoint{2.338129in}{0.666820in}}%
\pgfpathlineto{\pgfqpoint{2.341743in}{0.673282in}}%
\pgfpathlineto{\pgfqpoint{2.343715in}{0.683758in}}%
\pgfpathlineto{\pgfqpoint{2.348509in}{0.684053in}}%
\pgfpathlineto{\pgfqpoint{2.358335in}{0.685504in}}%
\pgfpathlineto{\pgfqpoint{2.368794in}{0.685172in}}%
\pgfpathlineto{\pgfqpoint{2.369214in}{0.678290in}}%
\pgfpathlineto{\pgfqpoint{2.402010in}{0.680459in}}%
\pgfpathlineto{\pgfqpoint{2.398862in}{0.719911in}}%
\pgfpathlineto{\pgfqpoint{2.424080in}{0.721840in}}%
\pgfpathlineto{\pgfqpoint{2.428014in}{0.688958in}}%
\pgfpathlineto{\pgfqpoint{2.433344in}{0.642630in}}%
\pgfpathlineto{\pgfqpoint{2.428696in}{0.637522in}}%
\pgfpathlineto{\pgfqpoint{2.419444in}{0.640905in}}%
\pgfpathlineto{\pgfqpoint{2.413978in}{0.638382in}}%
\pgfpathlineto{\pgfqpoint{2.399016in}{0.641068in}}%
\pgfpathlineto{\pgfqpoint{2.389727in}{0.638883in}}%
\pgfpathlineto{\pgfqpoint{2.372775in}{0.632327in}}%
\pgfpathclose%
\pgfusepath{fill}%
\end{pgfscope}%
\begin{pgfscope}%
\pgfpathrectangle{\pgfqpoint{0.100000in}{0.100000in}}{\pgfqpoint{3.420221in}{2.189500in}}%
\pgfusepath{clip}%
\pgfsetbuttcap%
\pgfsetmiterjoin%
\definecolor{currentfill}{rgb}{0.000000,0.478431,0.760784}%
\pgfsetfillcolor{currentfill}%
\pgfsetlinewidth{0.000000pt}%
\definecolor{currentstroke}{rgb}{0.000000,0.000000,0.000000}%
\pgfsetstrokecolor{currentstroke}%
\pgfsetstrokeopacity{0.000000}%
\pgfsetdash{}{0pt}%
\pgfpathmoveto{\pgfqpoint{3.387999in}{1.645418in}}%
\pgfpathlineto{\pgfqpoint{3.395089in}{1.638805in}}%
\pgfpathlineto{\pgfqpoint{3.393014in}{1.635123in}}%
\pgfpathlineto{\pgfqpoint{3.384837in}{1.633825in}}%
\pgfpathlineto{\pgfqpoint{3.389794in}{1.641502in}}%
\pgfpathclose%
\pgfusepath{fill}%
\end{pgfscope}%
\begin{pgfscope}%
\pgfpathrectangle{\pgfqpoint{0.100000in}{0.100000in}}{\pgfqpoint{3.420221in}{2.189500in}}%
\pgfusepath{clip}%
\pgfsetbuttcap%
\pgfsetmiterjoin%
\definecolor{currentfill}{rgb}{0.000000,0.396078,0.801961}%
\pgfsetfillcolor{currentfill}%
\pgfsetlinewidth{0.000000pt}%
\definecolor{currentstroke}{rgb}{0.000000,0.000000,0.000000}%
\pgfsetstrokecolor{currentstroke}%
\pgfsetstrokeopacity{0.000000}%
\pgfsetdash{}{0pt}%
\pgfpathmoveto{\pgfqpoint{2.872118in}{1.172995in}}%
\pgfpathlineto{\pgfqpoint{2.859407in}{1.187895in}}%
\pgfpathlineto{\pgfqpoint{2.853723in}{1.182986in}}%
\pgfpathlineto{\pgfqpoint{2.842953in}{1.192554in}}%
\pgfpathlineto{\pgfqpoint{2.839730in}{1.198703in}}%
\pgfpathlineto{\pgfqpoint{2.843841in}{1.201805in}}%
\pgfpathlineto{\pgfqpoint{2.832935in}{1.209806in}}%
\pgfpathlineto{\pgfqpoint{2.841829in}{1.216619in}}%
\pgfpathlineto{\pgfqpoint{2.840274in}{1.222647in}}%
\pgfpathlineto{\pgfqpoint{2.848802in}{1.246585in}}%
\pgfpathlineto{\pgfqpoint{2.861206in}{1.245276in}}%
\pgfpathlineto{\pgfqpoint{2.870218in}{1.247340in}}%
\pgfpathlineto{\pgfqpoint{2.880254in}{1.241743in}}%
\pgfpathlineto{\pgfqpoint{2.892242in}{1.251746in}}%
\pgfpathlineto{\pgfqpoint{2.896904in}{1.260063in}}%
\pgfpathlineto{\pgfqpoint{2.905710in}{1.258672in}}%
\pgfpathlineto{\pgfqpoint{2.906425in}{1.251087in}}%
\pgfpathlineto{\pgfqpoint{2.918882in}{1.243428in}}%
\pgfpathlineto{\pgfqpoint{2.912880in}{1.236707in}}%
\pgfpathlineto{\pgfqpoint{2.910126in}{1.238283in}}%
\pgfpathlineto{\pgfqpoint{2.903330in}{1.231342in}}%
\pgfpathlineto{\pgfqpoint{2.901329in}{1.224115in}}%
\pgfpathlineto{\pgfqpoint{2.903233in}{1.216675in}}%
\pgfpathlineto{\pgfqpoint{2.915058in}{1.211890in}}%
\pgfpathlineto{\pgfqpoint{2.919493in}{1.206218in}}%
\pgfpathlineto{\pgfqpoint{2.919091in}{1.191996in}}%
\pgfpathlineto{\pgfqpoint{2.912778in}{1.185767in}}%
\pgfpathlineto{\pgfqpoint{2.883706in}{1.187293in}}%
\pgfpathlineto{\pgfqpoint{2.879194in}{1.184393in}}%
\pgfpathclose%
\pgfusepath{fill}%
\end{pgfscope}%
\begin{pgfscope}%
\pgfpathrectangle{\pgfqpoint{0.100000in}{0.100000in}}{\pgfqpoint{3.420221in}{2.189500in}}%
\pgfusepath{clip}%
\pgfsetbuttcap%
\pgfsetmiterjoin%
\definecolor{currentfill}{rgb}{0.000000,0.478431,0.760784}%
\pgfsetfillcolor{currentfill}%
\pgfsetlinewidth{0.000000pt}%
\definecolor{currentstroke}{rgb}{0.000000,0.000000,0.000000}%
\pgfsetstrokecolor{currentstroke}%
\pgfsetstrokeopacity{0.000000}%
\pgfsetdash{}{0pt}%
\pgfpathmoveto{\pgfqpoint{1.206560in}{1.460491in}}%
\pgfpathlineto{\pgfqpoint{1.200574in}{1.463720in}}%
\pgfpathlineto{\pgfqpoint{1.201620in}{1.471299in}}%
\pgfpathlineto{\pgfqpoint{1.200246in}{1.477036in}}%
\pgfpathlineto{\pgfqpoint{1.189533in}{1.478692in}}%
\pgfpathlineto{\pgfqpoint{1.188745in}{1.473718in}}%
\pgfpathlineto{\pgfqpoint{1.183113in}{1.470719in}}%
\pgfpathlineto{\pgfqpoint{1.181626in}{1.477648in}}%
\pgfpathlineto{\pgfqpoint{1.161891in}{1.476028in}}%
\pgfpathlineto{\pgfqpoint{1.155070in}{1.480949in}}%
\pgfpathlineto{\pgfqpoint{1.157331in}{1.494524in}}%
\pgfpathlineto{\pgfqpoint{1.129408in}{1.499057in}}%
\pgfpathlineto{\pgfqpoint{1.099025in}{1.504628in}}%
\pgfpathlineto{\pgfqpoint{1.102376in}{1.523354in}}%
\pgfpathlineto{\pgfqpoint{1.092376in}{1.518376in}}%
\pgfpathlineto{\pgfqpoint{1.089498in}{1.525118in}}%
\pgfpathlineto{\pgfqpoint{1.089637in}{1.531640in}}%
\pgfpathlineto{\pgfqpoint{1.082482in}{1.535352in}}%
\pgfpathlineto{\pgfqpoint{1.078948in}{1.540654in}}%
\pgfpathlineto{\pgfqpoint{1.084160in}{1.547947in}}%
\pgfpathlineto{\pgfqpoint{1.088100in}{1.556587in}}%
\pgfpathlineto{\pgfqpoint{1.086753in}{1.569839in}}%
\pgfpathlineto{\pgfqpoint{1.084931in}{1.572303in}}%
\pgfpathlineto{\pgfqpoint{1.088111in}{1.578889in}}%
\pgfpathlineto{\pgfqpoint{1.086956in}{1.583137in}}%
\pgfpathlineto{\pgfqpoint{1.112247in}{1.578714in}}%
\pgfpathlineto{\pgfqpoint{1.125812in}{1.653943in}}%
\pgfpathlineto{\pgfqpoint{1.129687in}{1.675810in}}%
\pgfpathlineto{\pgfqpoint{1.141886in}{1.672415in}}%
\pgfpathlineto{\pgfqpoint{1.141051in}{1.667652in}}%
\pgfpathlineto{\pgfqpoint{1.153739in}{1.665442in}}%
\pgfpathlineto{\pgfqpoint{1.154477in}{1.669615in}}%
\pgfpathlineto{\pgfqpoint{1.167103in}{1.667410in}}%
\pgfpathlineto{\pgfqpoint{1.168185in}{1.673837in}}%
\pgfpathlineto{\pgfqpoint{1.183631in}{1.671217in}}%
\pgfpathlineto{\pgfqpoint{1.185018in}{1.677614in}}%
\pgfpathlineto{\pgfqpoint{1.200971in}{1.674853in}}%
\pgfpathlineto{\pgfqpoint{1.203048in}{1.667065in}}%
\pgfpathlineto{\pgfqpoint{1.202330in}{1.656427in}}%
\pgfpathlineto{\pgfqpoint{1.204086in}{1.648719in}}%
\pgfpathlineto{\pgfqpoint{1.209149in}{1.639075in}}%
\pgfpathlineto{\pgfqpoint{1.214602in}{1.632752in}}%
\pgfpathlineto{\pgfqpoint{1.220782in}{1.619446in}}%
\pgfpathlineto{\pgfqpoint{1.228774in}{1.611866in}}%
\pgfpathlineto{\pgfqpoint{1.225948in}{1.592553in}}%
\pgfpathlineto{\pgfqpoint{1.225389in}{1.579745in}}%
\pgfpathlineto{\pgfqpoint{1.282658in}{1.571037in}}%
\pgfpathlineto{\pgfqpoint{1.308933in}{1.567300in}}%
\pgfpathlineto{\pgfqpoint{1.308183in}{1.554240in}}%
\pgfpathlineto{\pgfqpoint{1.303427in}{1.522299in}}%
\pgfpathlineto{\pgfqpoint{1.280147in}{1.525742in}}%
\pgfpathlineto{\pgfqpoint{1.278075in}{1.506260in}}%
\pgfpathlineto{\pgfqpoint{1.273790in}{1.476848in}}%
\pgfpathlineto{\pgfqpoint{1.239756in}{1.481650in}}%
\pgfpathlineto{\pgfqpoint{1.210463in}{1.486210in}}%
\pgfpathclose%
\pgfusepath{fill}%
\end{pgfscope}%
\begin{pgfscope}%
\pgfpathrectangle{\pgfqpoint{0.100000in}{0.100000in}}{\pgfqpoint{3.420221in}{2.189500in}}%
\pgfusepath{clip}%
\pgfsetbuttcap%
\pgfsetmiterjoin%
\definecolor{currentfill}{rgb}{0.000000,0.352941,0.823529}%
\pgfsetfillcolor{currentfill}%
\pgfsetlinewidth{0.000000pt}%
\definecolor{currentstroke}{rgb}{0.000000,0.000000,0.000000}%
\pgfsetstrokecolor{currentstroke}%
\pgfsetstrokeopacity{0.000000}%
\pgfsetdash{}{0pt}%
\pgfpathmoveto{\pgfqpoint{2.394235in}{1.631207in}}%
\pgfpathlineto{\pgfqpoint{2.374277in}{1.629687in}}%
\pgfpathlineto{\pgfqpoint{2.372255in}{1.655800in}}%
\pgfpathlineto{\pgfqpoint{2.378754in}{1.656309in}}%
\pgfpathlineto{\pgfqpoint{2.376664in}{1.682492in}}%
\pgfpathlineto{\pgfqpoint{2.385006in}{1.683097in}}%
\pgfpathlineto{\pgfqpoint{2.384514in}{1.689643in}}%
\pgfpathlineto{\pgfqpoint{2.403073in}{1.691155in}}%
\pgfpathlineto{\pgfqpoint{2.405379in}{1.680830in}}%
\pgfpathlineto{\pgfqpoint{2.398806in}{1.673958in}}%
\pgfpathlineto{\pgfqpoint{2.395345in}{1.656836in}}%
\pgfpathlineto{\pgfqpoint{2.398302in}{1.642455in}}%
\pgfpathclose%
\pgfusepath{fill}%
\end{pgfscope}%
\begin{pgfscope}%
\pgfpathrectangle{\pgfqpoint{0.100000in}{0.100000in}}{\pgfqpoint{3.420221in}{2.189500in}}%
\pgfusepath{clip}%
\pgfsetbuttcap%
\pgfsetmiterjoin%
\definecolor{currentfill}{rgb}{0.000000,0.470588,0.764706}%
\pgfsetfillcolor{currentfill}%
\pgfsetlinewidth{0.000000pt}%
\definecolor{currentstroke}{rgb}{0.000000,0.000000,0.000000}%
\pgfsetstrokecolor{currentstroke}%
\pgfsetstrokeopacity{0.000000}%
\pgfsetdash{}{0pt}%
\pgfpathmoveto{\pgfqpoint{1.499877in}{0.908964in}}%
\pgfpathlineto{\pgfqpoint{1.503330in}{0.945506in}}%
\pgfpathlineto{\pgfqpoint{1.535397in}{0.942773in}}%
\pgfpathlineto{\pgfqpoint{1.538122in}{0.975267in}}%
\pgfpathlineto{\pgfqpoint{1.570538in}{0.972766in}}%
\pgfpathlineto{\pgfqpoint{1.568050in}{0.940159in}}%
\pgfpathlineto{\pgfqpoint{1.562332in}{0.940610in}}%
\pgfpathlineto{\pgfqpoint{1.559762in}{0.903967in}}%
\pgfpathclose%
\pgfusepath{fill}%
\end{pgfscope}%
\begin{pgfscope}%
\pgfpathrectangle{\pgfqpoint{0.100000in}{0.100000in}}{\pgfqpoint{3.420221in}{2.189500in}}%
\pgfusepath{clip}%
\pgfsetbuttcap%
\pgfsetmiterjoin%
\definecolor{currentfill}{rgb}{0.000000,0.717647,0.641176}%
\pgfsetfillcolor{currentfill}%
\pgfsetlinewidth{0.000000pt}%
\definecolor{currentstroke}{rgb}{0.000000,0.000000,0.000000}%
\pgfsetstrokecolor{currentstroke}%
\pgfsetstrokeopacity{0.000000}%
\pgfsetdash{}{0pt}%
\pgfpathmoveto{\pgfqpoint{0.533937in}{2.015037in}}%
\pgfpathlineto{\pgfqpoint{0.515853in}{2.020394in}}%
\pgfpathlineto{\pgfqpoint{0.503586in}{2.025224in}}%
\pgfpathlineto{\pgfqpoint{0.512202in}{2.045295in}}%
\pgfpathlineto{\pgfqpoint{0.512090in}{2.056490in}}%
\pgfpathlineto{\pgfqpoint{0.517728in}{2.050340in}}%
\pgfpathlineto{\pgfqpoint{0.524279in}{2.052566in}}%
\pgfpathlineto{\pgfqpoint{0.531814in}{2.052206in}}%
\pgfpathlineto{\pgfqpoint{0.528745in}{2.058379in}}%
\pgfpathlineto{\pgfqpoint{0.519050in}{2.056382in}}%
\pgfpathlineto{\pgfqpoint{0.511822in}{2.066039in}}%
\pgfpathlineto{\pgfqpoint{0.515605in}{2.075983in}}%
\pgfpathlineto{\pgfqpoint{0.524713in}{2.076759in}}%
\pgfpathlineto{\pgfqpoint{0.523803in}{2.091584in}}%
\pgfpathlineto{\pgfqpoint{0.519965in}{2.095690in}}%
\pgfpathlineto{\pgfqpoint{0.521997in}{2.108395in}}%
\pgfpathlineto{\pgfqpoint{0.525900in}{2.106040in}}%
\pgfpathlineto{\pgfqpoint{0.535444in}{2.109024in}}%
\pgfpathlineto{\pgfqpoint{0.524015in}{2.117121in}}%
\pgfpathlineto{\pgfqpoint{0.526160in}{2.137543in}}%
\pgfpathlineto{\pgfqpoint{0.524120in}{2.147631in}}%
\pgfpathlineto{\pgfqpoint{0.526319in}{2.157140in}}%
\pgfpathlineto{\pgfqpoint{0.540920in}{2.151029in}}%
\pgfpathlineto{\pgfqpoint{0.567299in}{2.142345in}}%
\pgfpathlineto{\pgfqpoint{0.561151in}{2.123661in}}%
\pgfpathlineto{\pgfqpoint{0.557886in}{2.110776in}}%
\pgfpathlineto{\pgfqpoint{0.572015in}{2.106393in}}%
\pgfpathlineto{\pgfqpoint{0.572111in}{2.099317in}}%
\pgfpathlineto{\pgfqpoint{0.567327in}{2.084740in}}%
\pgfpathlineto{\pgfqpoint{0.556943in}{2.087997in}}%
\pgfpathlineto{\pgfqpoint{0.548087in}{2.058433in}}%
\pgfpathlineto{\pgfqpoint{0.603598in}{2.040987in}}%
\pgfpathlineto{\pgfqpoint{0.596012in}{2.017138in}}%
\pgfpathlineto{\pgfqpoint{0.588299in}{2.011855in}}%
\pgfpathlineto{\pgfqpoint{0.582909in}{2.016004in}}%
\pgfpathlineto{\pgfqpoint{0.567345in}{2.013820in}}%
\pgfpathlineto{\pgfqpoint{0.563769in}{2.029590in}}%
\pgfpathlineto{\pgfqpoint{0.553107in}{2.041328in}}%
\pgfpathlineto{\pgfqpoint{0.542302in}{2.041419in}}%
\pgfpathclose%
\pgfusepath{fill}%
\end{pgfscope}%
\begin{pgfscope}%
\pgfpathrectangle{\pgfqpoint{0.100000in}{0.100000in}}{\pgfqpoint{3.420221in}{2.189500in}}%
\pgfusepath{clip}%
\pgfsetbuttcap%
\pgfsetmiterjoin%
\definecolor{currentfill}{rgb}{0.000000,0.658824,0.670588}%
\pgfsetfillcolor{currentfill}%
\pgfsetlinewidth{0.000000pt}%
\definecolor{currentstroke}{rgb}{0.000000,0.000000,0.000000}%
\pgfsetstrokecolor{currentstroke}%
\pgfsetstrokeopacity{0.000000}%
\pgfsetdash{}{0pt}%
\pgfpathmoveto{\pgfqpoint{0.910368in}{2.000022in}}%
\pgfpathlineto{\pgfqpoint{0.904991in}{1.977468in}}%
\pgfpathlineto{\pgfqpoint{0.890276in}{1.981136in}}%
\pgfpathlineto{\pgfqpoint{0.883858in}{1.975951in}}%
\pgfpathlineto{\pgfqpoint{0.867594in}{1.979867in}}%
\pgfpathlineto{\pgfqpoint{0.865475in}{1.971333in}}%
\pgfpathlineto{\pgfqpoint{0.857332in}{1.972934in}}%
\pgfpathlineto{\pgfqpoint{0.856857in}{1.982769in}}%
\pgfpathlineto{\pgfqpoint{0.851556in}{1.989763in}}%
\pgfpathlineto{\pgfqpoint{0.849461in}{1.996056in}}%
\pgfpathlineto{\pgfqpoint{0.841715in}{1.996554in}}%
\pgfpathlineto{\pgfqpoint{0.837956in}{1.999890in}}%
\pgfpathlineto{\pgfqpoint{0.828218in}{1.996264in}}%
\pgfpathlineto{\pgfqpoint{0.810807in}{1.995474in}}%
\pgfpathlineto{\pgfqpoint{0.810327in}{2.009397in}}%
\pgfpathlineto{\pgfqpoint{0.812406in}{2.009276in}}%
\pgfpathlineto{\pgfqpoint{0.823294in}{2.010548in}}%
\pgfpathlineto{\pgfqpoint{0.828322in}{2.018891in}}%
\pgfpathlineto{\pgfqpoint{0.834795in}{2.044232in}}%
\pgfpathlineto{\pgfqpoint{0.880612in}{2.032558in}}%
\pgfpathlineto{\pgfqpoint{0.878226in}{2.022853in}}%
\pgfpathlineto{\pgfqpoint{0.890041in}{2.012951in}}%
\pgfpathlineto{\pgfqpoint{0.905695in}{2.009090in}}%
\pgfpathlineto{\pgfqpoint{0.911915in}{2.006493in}}%
\pgfpathclose%
\pgfusepath{fill}%
\end{pgfscope}%
\begin{pgfscope}%
\pgfpathrectangle{\pgfqpoint{0.100000in}{0.100000in}}{\pgfqpoint{3.420221in}{2.189500in}}%
\pgfusepath{clip}%
\pgfsetbuttcap%
\pgfsetmiterjoin%
\definecolor{currentfill}{rgb}{0.000000,0.615686,0.692157}%
\pgfsetfillcolor{currentfill}%
\pgfsetlinewidth{0.000000pt}%
\definecolor{currentstroke}{rgb}{0.000000,0.000000,0.000000}%
\pgfsetstrokecolor{currentstroke}%
\pgfsetstrokeopacity{0.000000}%
\pgfsetdash{}{0pt}%
\pgfpathmoveto{\pgfqpoint{0.837200in}{0.870088in}}%
\pgfpathlineto{\pgfqpoint{0.844233in}{0.904518in}}%
\pgfpathlineto{\pgfqpoint{0.891374in}{0.895100in}}%
\pgfpathlineto{\pgfqpoint{0.954398in}{0.883239in}}%
\pgfpathlineto{\pgfqpoint{1.024476in}{0.871303in}}%
\pgfpathlineto{\pgfqpoint{1.023355in}{0.864881in}}%
\pgfpathlineto{\pgfqpoint{1.040987in}{0.861807in}}%
\pgfpathlineto{\pgfqpoint{1.111527in}{0.850369in}}%
\pgfpathlineto{\pgfqpoint{1.098657in}{0.768794in}}%
\pgfpathlineto{\pgfqpoint{1.038080in}{0.778608in}}%
\pgfpathlineto{\pgfqpoint{0.969728in}{0.790340in}}%
\pgfpathlineto{\pgfqpoint{0.952462in}{0.800662in}}%
\pgfpathlineto{\pgfqpoint{0.858331in}{0.857080in}}%
\pgfpathclose%
\pgfusepath{fill}%
\end{pgfscope}%
\begin{pgfscope}%
\pgfpathrectangle{\pgfqpoint{0.100000in}{0.100000in}}{\pgfqpoint{3.420221in}{2.189500in}}%
\pgfusepath{clip}%
\pgfsetbuttcap%
\pgfsetmiterjoin%
\definecolor{currentfill}{rgb}{0.000000,0.239216,0.880392}%
\pgfsetfillcolor{currentfill}%
\pgfsetlinewidth{0.000000pt}%
\definecolor{currentstroke}{rgb}{0.000000,0.000000,0.000000}%
\pgfsetstrokecolor{currentstroke}%
\pgfsetstrokeopacity{0.000000}%
\pgfsetdash{}{0pt}%
\pgfpathmoveto{\pgfqpoint{2.994042in}{1.378885in}}%
\pgfpathlineto{\pgfqpoint{2.992810in}{1.374739in}}%
\pgfpathlineto{\pgfqpoint{2.986124in}{1.365721in}}%
\pgfpathlineto{\pgfqpoint{2.982442in}{1.368041in}}%
\pgfpathlineto{\pgfqpoint{2.968001in}{1.364489in}}%
\pgfpathlineto{\pgfqpoint{2.961721in}{1.369893in}}%
\pgfpathlineto{\pgfqpoint{2.959744in}{1.364453in}}%
\pgfpathlineto{\pgfqpoint{2.954170in}{1.365655in}}%
\pgfpathlineto{\pgfqpoint{2.948792in}{1.369383in}}%
\pgfpathlineto{\pgfqpoint{2.931978in}{1.372464in}}%
\pgfpathlineto{\pgfqpoint{2.926844in}{1.374496in}}%
\pgfpathlineto{\pgfqpoint{2.932977in}{1.382690in}}%
\pgfpathlineto{\pgfqpoint{2.935228in}{1.390228in}}%
\pgfpathlineto{\pgfqpoint{2.941259in}{1.399741in}}%
\pgfpathlineto{\pgfqpoint{2.952251in}{1.400351in}}%
\pgfpathlineto{\pgfqpoint{2.959852in}{1.397886in}}%
\pgfpathclose%
\pgfusepath{fill}%
\end{pgfscope}%
\begin{pgfscope}%
\pgfpathrectangle{\pgfqpoint{0.100000in}{0.100000in}}{\pgfqpoint{3.420221in}{2.189500in}}%
\pgfusepath{clip}%
\pgfsetbuttcap%
\pgfsetmiterjoin%
\definecolor{currentfill}{rgb}{0.000000,0.431373,0.784314}%
\pgfsetfillcolor{currentfill}%
\pgfsetlinewidth{0.000000pt}%
\definecolor{currentstroke}{rgb}{0.000000,0.000000,0.000000}%
\pgfsetstrokecolor{currentstroke}%
\pgfsetstrokeopacity{0.000000}%
\pgfsetdash{}{0pt}%
\pgfpathmoveto{\pgfqpoint{1.867874in}{0.674158in}}%
\pgfpathlineto{\pgfqpoint{1.855447in}{0.667142in}}%
\pgfpathlineto{\pgfqpoint{1.851520in}{0.656853in}}%
\pgfpathlineto{\pgfqpoint{1.831660in}{0.666323in}}%
\pgfpathlineto{\pgfqpoint{1.818539in}{0.669780in}}%
\pgfpathlineto{\pgfqpoint{1.813444in}{0.679336in}}%
\pgfpathlineto{\pgfqpoint{1.779231in}{0.680125in}}%
\pgfpathlineto{\pgfqpoint{1.773900in}{0.684815in}}%
\pgfpathlineto{\pgfqpoint{1.771858in}{0.695623in}}%
\pgfpathlineto{\pgfqpoint{1.791205in}{0.708951in}}%
\pgfpathlineto{\pgfqpoint{1.824320in}{0.727207in}}%
\pgfpathlineto{\pgfqpoint{1.829327in}{0.729970in}}%
\pgfpathlineto{\pgfqpoint{1.845959in}{0.699962in}}%
\pgfpathlineto{\pgfqpoint{1.850664in}{0.694092in}}%
\pgfpathlineto{\pgfqpoint{1.854940in}{0.696668in}}%
\pgfpathclose%
\pgfusepath{fill}%
\end{pgfscope}%
\begin{pgfscope}%
\pgfpathrectangle{\pgfqpoint{0.100000in}{0.100000in}}{\pgfqpoint{3.420221in}{2.189500in}}%
\pgfusepath{clip}%
\pgfsetbuttcap%
\pgfsetmiterjoin%
\definecolor{currentfill}{rgb}{0.000000,0.745098,0.627451}%
\pgfsetfillcolor{currentfill}%
\pgfsetlinewidth{0.000000pt}%
\definecolor{currentstroke}{rgb}{0.000000,0.000000,0.000000}%
\pgfsetstrokecolor{currentstroke}%
\pgfsetstrokeopacity{0.000000}%
\pgfsetdash{}{0pt}%
\pgfpathmoveto{\pgfqpoint{1.716417in}{0.343192in}}%
\pgfpathlineto{\pgfqpoint{1.727608in}{0.350616in}}%
\pgfpathlineto{\pgfqpoint{1.731867in}{0.358914in}}%
\pgfpathlineto{\pgfqpoint{1.768173in}{0.357274in}}%
\pgfpathlineto{\pgfqpoint{1.798033in}{0.356020in}}%
\pgfpathlineto{\pgfqpoint{1.797630in}{0.343377in}}%
\pgfpathlineto{\pgfqpoint{1.806051in}{0.341740in}}%
\pgfpathlineto{\pgfqpoint{1.834991in}{0.341043in}}%
\pgfpathlineto{\pgfqpoint{1.836947in}{0.333338in}}%
\pgfpathlineto{\pgfqpoint{1.834688in}{0.330531in}}%
\pgfpathlineto{\pgfqpoint{1.841137in}{0.322376in}}%
\pgfpathlineto{\pgfqpoint{1.840863in}{0.315722in}}%
\pgfpathlineto{\pgfqpoint{1.844104in}{0.309688in}}%
\pgfpathlineto{\pgfqpoint{1.844474in}{0.302443in}}%
\pgfpathlineto{\pgfqpoint{1.853598in}{0.298334in}}%
\pgfpathlineto{\pgfqpoint{1.853967in}{0.292307in}}%
\pgfpathlineto{\pgfqpoint{1.838526in}{0.288429in}}%
\pgfpathlineto{\pgfqpoint{1.834362in}{0.282801in}}%
\pgfpathlineto{\pgfqpoint{1.828553in}{0.286528in}}%
\pgfpathlineto{\pgfqpoint{1.817310in}{0.297509in}}%
\pgfpathlineto{\pgfqpoint{1.803055in}{0.301031in}}%
\pgfpathlineto{\pgfqpoint{1.790830in}{0.299849in}}%
\pgfpathlineto{\pgfqpoint{1.778086in}{0.302434in}}%
\pgfpathlineto{\pgfqpoint{1.763909in}{0.314672in}}%
\pgfpathlineto{\pgfqpoint{1.750075in}{0.315932in}}%
\pgfpathlineto{\pgfqpoint{1.740562in}{0.326609in}}%
\pgfpathlineto{\pgfqpoint{1.738106in}{0.325930in}}%
\pgfpathlineto{\pgfqpoint{1.719924in}{0.331873in}}%
\pgfpathlineto{\pgfqpoint{1.721417in}{0.335697in}}%
\pgfpathclose%
\pgfusepath{fill}%
\end{pgfscope}%
\begin{pgfscope}%
\pgfpathrectangle{\pgfqpoint{0.100000in}{0.100000in}}{\pgfqpoint{3.420221in}{2.189500in}}%
\pgfusepath{clip}%
\pgfsetbuttcap%
\pgfsetmiterjoin%
\definecolor{currentfill}{rgb}{0.000000,0.435294,0.782353}%
\pgfsetfillcolor{currentfill}%
\pgfsetlinewidth{0.000000pt}%
\definecolor{currentstroke}{rgb}{0.000000,0.000000,0.000000}%
\pgfsetstrokecolor{currentstroke}%
\pgfsetstrokeopacity{0.000000}%
\pgfsetdash{}{0pt}%
\pgfpathmoveto{\pgfqpoint{2.370731in}{1.258264in}}%
\pgfpathlineto{\pgfqpoint{2.369769in}{1.274725in}}%
\pgfpathlineto{\pgfqpoint{2.343993in}{1.272843in}}%
\pgfpathlineto{\pgfqpoint{2.344404in}{1.266271in}}%
\pgfpathlineto{\pgfqpoint{2.337622in}{1.266263in}}%
\pgfpathlineto{\pgfqpoint{2.335587in}{1.301823in}}%
\pgfpathlineto{\pgfqpoint{2.341983in}{1.302263in}}%
\pgfpathlineto{\pgfqpoint{2.380743in}{1.304684in}}%
\pgfpathlineto{\pgfqpoint{2.380979in}{1.301424in}}%
\pgfpathlineto{\pgfqpoint{2.422921in}{1.305214in}}%
\pgfpathlineto{\pgfqpoint{2.429834in}{1.304131in}}%
\pgfpathlineto{\pgfqpoint{2.433697in}{1.296918in}}%
\pgfpathlineto{\pgfqpoint{2.433957in}{1.291594in}}%
\pgfpathlineto{\pgfqpoint{2.437529in}{1.285590in}}%
\pgfpathlineto{\pgfqpoint{2.437359in}{1.281826in}}%
\pgfpathlineto{\pgfqpoint{2.415463in}{1.279902in}}%
\pgfpathlineto{\pgfqpoint{2.395077in}{1.278139in}}%
\pgfpathlineto{\pgfqpoint{2.395290in}{1.263834in}}%
\pgfpathlineto{\pgfqpoint{2.396731in}{1.259629in}}%
\pgfpathclose%
\pgfusepath{fill}%
\end{pgfscope}%
\begin{pgfscope}%
\pgfpathrectangle{\pgfqpoint{0.100000in}{0.100000in}}{\pgfqpoint{3.420221in}{2.189500in}}%
\pgfusepath{clip}%
\pgfsetbuttcap%
\pgfsetmiterjoin%
\definecolor{currentfill}{rgb}{0.000000,0.466667,0.766667}%
\pgfsetfillcolor{currentfill}%
\pgfsetlinewidth{0.000000pt}%
\definecolor{currentstroke}{rgb}{0.000000,0.000000,0.000000}%
\pgfsetstrokecolor{currentstroke}%
\pgfsetstrokeopacity{0.000000}%
\pgfsetdash{}{0pt}%
\pgfpathmoveto{\pgfqpoint{2.883974in}{1.101519in}}%
\pgfpathlineto{\pgfqpoint{2.872372in}{1.107473in}}%
\pgfpathlineto{\pgfqpoint{2.867714in}{1.107272in}}%
\pgfpathlineto{\pgfqpoint{2.866287in}{1.122114in}}%
\pgfpathlineto{\pgfqpoint{2.854882in}{1.120667in}}%
\pgfpathlineto{\pgfqpoint{2.852533in}{1.140854in}}%
\pgfpathlineto{\pgfqpoint{2.845785in}{1.145513in}}%
\pgfpathlineto{\pgfqpoint{2.844764in}{1.148659in}}%
\pgfpathlineto{\pgfqpoint{2.847846in}{1.157982in}}%
\pgfpathlineto{\pgfqpoint{2.865120in}{1.160414in}}%
\pgfpathlineto{\pgfqpoint{2.884132in}{1.162430in}}%
\pgfpathlineto{\pgfqpoint{2.900079in}{1.164945in}}%
\pgfpathlineto{\pgfqpoint{2.905672in}{1.125420in}}%
\pgfpathlineto{\pgfqpoint{2.895251in}{1.124938in}}%
\pgfpathlineto{\pgfqpoint{2.886154in}{1.118609in}}%
\pgfpathlineto{\pgfqpoint{2.886411in}{1.109411in}}%
\pgfpathlineto{\pgfqpoint{2.881990in}{1.107689in}}%
\pgfpathclose%
\pgfusepath{fill}%
\end{pgfscope}%
\begin{pgfscope}%
\pgfpathrectangle{\pgfqpoint{0.100000in}{0.100000in}}{\pgfqpoint{3.420221in}{2.189500in}}%
\pgfusepath{clip}%
\pgfsetbuttcap%
\pgfsetmiterjoin%
\definecolor{currentfill}{rgb}{0.000000,0.450980,0.774510}%
\pgfsetfillcolor{currentfill}%
\pgfsetlinewidth{0.000000pt}%
\definecolor{currentstroke}{rgb}{0.000000,0.000000,0.000000}%
\pgfsetstrokecolor{currentstroke}%
\pgfsetstrokeopacity{0.000000}%
\pgfsetdash{}{0pt}%
\pgfpathmoveto{\pgfqpoint{3.188696in}{1.446475in}}%
\pgfpathlineto{\pgfqpoint{3.183866in}{1.446964in}}%
\pgfpathlineto{\pgfqpoint{3.178732in}{1.466911in}}%
\pgfpathlineto{\pgfqpoint{3.161529in}{1.491002in}}%
\pgfpathlineto{\pgfqpoint{3.159358in}{1.493926in}}%
\pgfpathlineto{\pgfqpoint{3.170324in}{1.504143in}}%
\pgfpathlineto{\pgfqpoint{3.176568in}{1.518622in}}%
\pgfpathlineto{\pgfqpoint{3.181324in}{1.520025in}}%
\pgfpathlineto{\pgfqpoint{3.189743in}{1.518516in}}%
\pgfpathlineto{\pgfqpoint{3.192821in}{1.510661in}}%
\pgfpathlineto{\pgfqpoint{3.192899in}{1.497234in}}%
\pgfpathlineto{\pgfqpoint{3.195217in}{1.470945in}}%
\pgfpathlineto{\pgfqpoint{3.193832in}{1.463136in}}%
\pgfpathclose%
\pgfusepath{fill}%
\end{pgfscope}%
\begin{pgfscope}%
\pgfpathrectangle{\pgfqpoint{0.100000in}{0.100000in}}{\pgfqpoint{3.420221in}{2.189500in}}%
\pgfusepath{clip}%
\pgfsetbuttcap%
\pgfsetmiterjoin%
\definecolor{currentfill}{rgb}{0.000000,0.156863,0.921569}%
\pgfsetfillcolor{currentfill}%
\pgfsetlinewidth{0.000000pt}%
\definecolor{currentstroke}{rgb}{0.000000,0.000000,0.000000}%
\pgfsetstrokecolor{currentstroke}%
\pgfsetstrokeopacity{0.000000}%
\pgfsetdash{}{0pt}%
\pgfpathmoveto{\pgfqpoint{1.843697in}{1.739436in}}%
\pgfpathlineto{\pgfqpoint{1.842625in}{1.700167in}}%
\pgfpathlineto{\pgfqpoint{1.803654in}{1.701784in}}%
\pgfpathlineto{\pgfqpoint{1.804479in}{1.721565in}}%
\pgfpathlineto{\pgfqpoint{1.758248in}{1.723697in}}%
\pgfpathlineto{\pgfqpoint{1.759521in}{1.749882in}}%
\pgfpathlineto{\pgfqpoint{1.752338in}{1.750273in}}%
\pgfpathlineto{\pgfqpoint{1.754598in}{1.802660in}}%
\pgfpathlineto{\pgfqpoint{1.791943in}{1.800793in}}%
\pgfpathlineto{\pgfqpoint{1.845789in}{1.798416in}}%
\pgfpathclose%
\pgfusepath{fill}%
\end{pgfscope}%
\begin{pgfscope}%
\pgfpathrectangle{\pgfqpoint{0.100000in}{0.100000in}}{\pgfqpoint{3.420221in}{2.189500in}}%
\pgfusepath{clip}%
\pgfsetbuttcap%
\pgfsetmiterjoin%
\definecolor{currentfill}{rgb}{0.000000,0.490196,0.754902}%
\pgfsetfillcolor{currentfill}%
\pgfsetlinewidth{0.000000pt}%
\definecolor{currentstroke}{rgb}{0.000000,0.000000,0.000000}%
\pgfsetstrokecolor{currentstroke}%
\pgfsetstrokeopacity{0.000000}%
\pgfsetdash{}{0pt}%
\pgfpathmoveto{\pgfqpoint{2.065129in}{0.748051in}}%
\pgfpathlineto{\pgfqpoint{2.033391in}{0.747350in}}%
\pgfpathlineto{\pgfqpoint{2.037236in}{0.737580in}}%
\pgfpathlineto{\pgfqpoint{2.006144in}{0.737572in}}%
\pgfpathlineto{\pgfqpoint{2.003069in}{0.737619in}}%
\pgfpathlineto{\pgfqpoint{2.002898in}{0.789759in}}%
\pgfpathlineto{\pgfqpoint{1.992363in}{0.792246in}}%
\pgfpathlineto{\pgfqpoint{1.992487in}{0.817292in}}%
\pgfpathlineto{\pgfqpoint{2.019817in}{0.817465in}}%
\pgfpathlineto{\pgfqpoint{2.020719in}{0.815560in}}%
\pgfpathlineto{\pgfqpoint{2.062668in}{0.815983in}}%
\pgfpathlineto{\pgfqpoint{2.063388in}{0.749102in}}%
\pgfpathclose%
\pgfusepath{fill}%
\end{pgfscope}%
\begin{pgfscope}%
\pgfpathrectangle{\pgfqpoint{0.100000in}{0.100000in}}{\pgfqpoint{3.420221in}{2.189500in}}%
\pgfusepath{clip}%
\pgfsetbuttcap%
\pgfsetmiterjoin%
\definecolor{currentfill}{rgb}{0.000000,0.768627,0.615686}%
\pgfsetfillcolor{currentfill}%
\pgfsetlinewidth{0.000000pt}%
\definecolor{currentstroke}{rgb}{0.000000,0.000000,0.000000}%
\pgfsetstrokecolor{currentstroke}%
\pgfsetstrokeopacity{0.000000}%
\pgfsetdash{}{0pt}%
\pgfpathmoveto{\pgfqpoint{0.930585in}{0.613220in}}%
\pgfpathlineto{\pgfqpoint{0.925986in}{0.608337in}}%
\pgfpathlineto{\pgfqpoint{0.927967in}{0.606750in}}%
\pgfpathlineto{\pgfqpoint{0.926346in}{0.604707in}}%
\pgfpathlineto{\pgfqpoint{0.928440in}{0.603101in}}%
\pgfpathlineto{\pgfqpoint{0.926839in}{0.601138in}}%
\pgfpathlineto{\pgfqpoint{0.928891in}{0.599536in}}%
\pgfpathlineto{\pgfqpoint{0.926740in}{0.597895in}}%
\pgfpathlineto{\pgfqpoint{0.922036in}{0.591838in}}%
\pgfpathlineto{\pgfqpoint{0.926093in}{0.588734in}}%
\pgfpathlineto{\pgfqpoint{0.924531in}{0.586622in}}%
\pgfpathlineto{\pgfqpoint{0.925977in}{0.585574in}}%
\pgfpathlineto{\pgfqpoint{0.919217in}{0.576999in}}%
\pgfpathlineto{\pgfqpoint{0.917029in}{0.575298in}}%
\pgfpathlineto{\pgfqpoint{0.911046in}{0.580147in}}%
\pgfpathlineto{\pgfqpoint{0.907862in}{0.576122in}}%
\pgfpathlineto{\pgfqpoint{0.902178in}{0.580644in}}%
\pgfpathlineto{\pgfqpoint{0.897021in}{0.574977in}}%
\pgfpathlineto{\pgfqpoint{0.893274in}{0.577907in}}%
\pgfpathlineto{\pgfqpoint{0.894880in}{0.579958in}}%
\pgfpathlineto{\pgfqpoint{0.892907in}{0.581574in}}%
\pgfpathlineto{\pgfqpoint{0.894450in}{0.583528in}}%
\pgfpathlineto{\pgfqpoint{0.890522in}{0.586706in}}%
\pgfpathlineto{\pgfqpoint{0.887331in}{0.582733in}}%
\pgfpathlineto{\pgfqpoint{0.883365in}{0.585915in}}%
\pgfpathlineto{\pgfqpoint{0.886593in}{0.589956in}}%
\pgfpathlineto{\pgfqpoint{0.878686in}{0.596417in}}%
\pgfpathlineto{\pgfqpoint{0.872287in}{0.588423in}}%
\pgfpathlineto{\pgfqpoint{0.862578in}{0.596628in}}%
\pgfpathlineto{\pgfqpoint{0.859512in}{0.592908in}}%
\pgfpathlineto{\pgfqpoint{0.853810in}{0.597799in}}%
\pgfpathlineto{\pgfqpoint{0.845545in}{0.588170in}}%
\pgfpathlineto{\pgfqpoint{0.843653in}{0.589802in}}%
\pgfpathlineto{\pgfqpoint{0.841972in}{0.587921in}}%
\pgfpathlineto{\pgfqpoint{0.830629in}{0.597951in}}%
\pgfpathlineto{\pgfqpoint{0.814302in}{0.612722in}}%
\pgfpathlineto{\pgfqpoint{0.806570in}{0.619928in}}%
\pgfpathlineto{\pgfqpoint{0.809992in}{0.623567in}}%
\pgfpathlineto{\pgfqpoint{0.808145in}{0.625326in}}%
\pgfpathlineto{\pgfqpoint{0.811646in}{0.629015in}}%
\pgfpathlineto{\pgfqpoint{0.809247in}{0.631278in}}%
\pgfpathlineto{\pgfqpoint{0.816261in}{0.638589in}}%
\pgfpathlineto{\pgfqpoint{0.815661in}{0.639219in}}%
\pgfpathlineto{\pgfqpoint{0.822648in}{0.646562in}}%
\pgfpathlineto{\pgfqpoint{0.824092in}{0.649459in}}%
\pgfpathlineto{\pgfqpoint{0.827148in}{0.646977in}}%
\pgfpathlineto{\pgfqpoint{0.828773in}{0.643411in}}%
\pgfpathlineto{\pgfqpoint{0.826791in}{0.641572in}}%
\pgfpathlineto{\pgfqpoint{0.824146in}{0.640354in}}%
\pgfpathlineto{\pgfqpoint{0.822551in}{0.638186in}}%
\pgfpathlineto{\pgfqpoint{0.819912in}{0.638241in}}%
\pgfpathlineto{\pgfqpoint{0.819017in}{0.633304in}}%
\pgfpathlineto{\pgfqpoint{0.821437in}{0.631544in}}%
\pgfpathlineto{\pgfqpoint{0.825339in}{0.627577in}}%
\pgfpathlineto{\pgfqpoint{0.828625in}{0.625852in}}%
\pgfpathlineto{\pgfqpoint{0.828279in}{0.623979in}}%
\pgfpathlineto{\pgfqpoint{0.830441in}{0.621280in}}%
\pgfpathlineto{\pgfqpoint{0.833465in}{0.620502in}}%
\pgfpathlineto{\pgfqpoint{0.834740in}{0.618478in}}%
\pgfpathlineto{\pgfqpoint{0.833494in}{0.615881in}}%
\pgfpathlineto{\pgfqpoint{0.838978in}{0.618048in}}%
\pgfpathlineto{\pgfqpoint{0.842174in}{0.619978in}}%
\pgfpathlineto{\pgfqpoint{0.843623in}{0.616995in}}%
\pgfpathlineto{\pgfqpoint{0.847042in}{0.615886in}}%
\pgfpathlineto{\pgfqpoint{0.847144in}{0.618949in}}%
\pgfpathlineto{\pgfqpoint{0.844834in}{0.623152in}}%
\pgfpathlineto{\pgfqpoint{0.841386in}{0.625057in}}%
\pgfpathlineto{\pgfqpoint{0.844134in}{0.627280in}}%
\pgfpathlineto{\pgfqpoint{0.844245in}{0.631322in}}%
\pgfpathlineto{\pgfqpoint{0.845190in}{0.634473in}}%
\pgfpathlineto{\pgfqpoint{0.843337in}{0.637003in}}%
\pgfpathlineto{\pgfqpoint{0.845024in}{0.641416in}}%
\pgfpathlineto{\pgfqpoint{0.847312in}{0.641783in}}%
\pgfpathlineto{\pgfqpoint{0.848913in}{0.640846in}}%
\pgfpathlineto{\pgfqpoint{0.848125in}{0.634785in}}%
\pgfpathlineto{\pgfqpoint{0.845526in}{0.631434in}}%
\pgfpathlineto{\pgfqpoint{0.845812in}{0.624792in}}%
\pgfpathlineto{\pgfqpoint{0.849959in}{0.622886in}}%
\pgfpathlineto{\pgfqpoint{0.850770in}{0.621021in}}%
\pgfpathlineto{\pgfqpoint{0.850566in}{0.617247in}}%
\pgfpathlineto{\pgfqpoint{0.855995in}{0.612162in}}%
\pgfpathlineto{\pgfqpoint{0.858506in}{0.614185in}}%
\pgfpathlineto{\pgfqpoint{0.859557in}{0.618244in}}%
\pgfpathlineto{\pgfqpoint{0.856721in}{0.619951in}}%
\pgfpathlineto{\pgfqpoint{0.855822in}{0.623481in}}%
\pgfpathlineto{\pgfqpoint{0.852290in}{0.624479in}}%
\pgfpathlineto{\pgfqpoint{0.850242in}{0.623931in}}%
\pgfpathlineto{\pgfqpoint{0.848420in}{0.626250in}}%
\pgfpathlineto{\pgfqpoint{0.848234in}{0.632423in}}%
\pgfpathlineto{\pgfqpoint{0.850402in}{0.634067in}}%
\pgfpathlineto{\pgfqpoint{0.855604in}{0.636839in}}%
\pgfpathlineto{\pgfqpoint{0.854941in}{0.639377in}}%
\pgfpathlineto{\pgfqpoint{0.849104in}{0.642358in}}%
\pgfpathlineto{\pgfqpoint{0.850371in}{0.643981in}}%
\pgfpathlineto{\pgfqpoint{0.846390in}{0.647144in}}%
\pgfpathlineto{\pgfqpoint{0.843888in}{0.649972in}}%
\pgfpathlineto{\pgfqpoint{0.840854in}{0.655645in}}%
\pgfpathlineto{\pgfqpoint{0.844981in}{0.661039in}}%
\pgfpathlineto{\pgfqpoint{0.847026in}{0.667382in}}%
\pgfpathlineto{\pgfqpoint{0.847319in}{0.670143in}}%
\pgfpathlineto{\pgfqpoint{0.846244in}{0.675257in}}%
\pgfpathlineto{\pgfqpoint{0.846333in}{0.677947in}}%
\pgfpathlineto{\pgfqpoint{0.845701in}{0.687817in}}%
\pgfpathlineto{\pgfqpoint{0.852720in}{0.680997in}}%
\pgfpathlineto{\pgfqpoint{0.856807in}{0.685255in}}%
\pgfpathlineto{\pgfqpoint{0.871218in}{0.671691in}}%
\pgfpathlineto{\pgfqpoint{0.872610in}{0.673199in}}%
\pgfpathlineto{\pgfqpoint{0.878506in}{0.667801in}}%
\pgfpathlineto{\pgfqpoint{0.877126in}{0.666282in}}%
\pgfpathlineto{\pgfqpoint{0.881114in}{0.662682in}}%
\pgfpathlineto{\pgfqpoint{0.882484in}{0.664207in}}%
\pgfpathlineto{\pgfqpoint{0.885972in}{0.661087in}}%
\pgfpathlineto{\pgfqpoint{0.884608in}{0.659555in}}%
\pgfpathlineto{\pgfqpoint{0.892735in}{0.652391in}}%
\pgfpathlineto{\pgfqpoint{0.894073in}{0.653926in}}%
\pgfpathlineto{\pgfqpoint{0.897659in}{0.650815in}}%
\pgfpathlineto{\pgfqpoint{0.896331in}{0.649277in}}%
\pgfpathlineto{\pgfqpoint{0.909903in}{0.637743in}}%
\pgfpathlineto{\pgfqpoint{0.918248in}{0.630876in}}%
\pgfpathlineto{\pgfqpoint{0.916545in}{0.628787in}}%
\pgfpathlineto{\pgfqpoint{0.924458in}{0.622307in}}%
\pgfpathlineto{\pgfqpoint{0.922894in}{0.620349in}}%
\pgfpathlineto{\pgfqpoint{0.931090in}{0.613884in}}%
\pgfpathclose%
\pgfusepath{fill}%
\end{pgfscope}%
\begin{pgfscope}%
\pgfpathrectangle{\pgfqpoint{0.100000in}{0.100000in}}{\pgfqpoint{3.420221in}{2.189500in}}%
\pgfusepath{clip}%
\pgfsetbuttcap%
\pgfsetmiterjoin%
\definecolor{currentfill}{rgb}{0.000000,0.458824,0.770588}%
\pgfsetfillcolor{currentfill}%
\pgfsetlinewidth{0.000000pt}%
\definecolor{currentstroke}{rgb}{0.000000,0.000000,0.000000}%
\pgfsetstrokecolor{currentstroke}%
\pgfsetstrokeopacity{0.000000}%
\pgfsetdash{}{0pt}%
\pgfpathmoveto{\pgfqpoint{1.712700in}{1.671249in}}%
\pgfpathlineto{\pgfqpoint{1.711329in}{1.647107in}}%
\pgfpathlineto{\pgfqpoint{1.712227in}{1.637261in}}%
\pgfpathlineto{\pgfqpoint{1.704927in}{1.637977in}}%
\pgfpathlineto{\pgfqpoint{1.699733in}{1.640707in}}%
\pgfpathlineto{\pgfqpoint{1.688012in}{1.639753in}}%
\pgfpathlineto{\pgfqpoint{1.681293in}{1.647090in}}%
\pgfpathlineto{\pgfqpoint{1.673547in}{1.649089in}}%
\pgfpathlineto{\pgfqpoint{1.676099in}{1.673547in}}%
\pgfpathclose%
\pgfusepath{fill}%
\end{pgfscope}%
\begin{pgfscope}%
\pgfpathrectangle{\pgfqpoint{0.100000in}{0.100000in}}{\pgfqpoint{3.420221in}{2.189500in}}%
\pgfusepath{clip}%
\pgfsetbuttcap%
\pgfsetmiterjoin%
\definecolor{currentfill}{rgb}{0.000000,0.270588,0.864706}%
\pgfsetfillcolor{currentfill}%
\pgfsetlinewidth{0.000000pt}%
\definecolor{currentstroke}{rgb}{0.000000,0.000000,0.000000}%
\pgfsetstrokecolor{currentstroke}%
\pgfsetstrokeopacity{0.000000}%
\pgfsetdash{}{0pt}%
\pgfpathmoveto{\pgfqpoint{2.268056in}{1.490509in}}%
\pgfpathlineto{\pgfqpoint{2.261417in}{1.486033in}}%
\pgfpathlineto{\pgfqpoint{2.257301in}{1.488279in}}%
\pgfpathlineto{\pgfqpoint{2.231453in}{1.487855in}}%
\pgfpathlineto{\pgfqpoint{2.230851in}{1.500944in}}%
\pgfpathlineto{\pgfqpoint{2.229698in}{1.527114in}}%
\pgfpathlineto{\pgfqpoint{2.242589in}{1.527674in}}%
\pgfpathlineto{\pgfqpoint{2.242304in}{1.534233in}}%
\pgfpathlineto{\pgfqpoint{2.252895in}{1.534694in}}%
\pgfpathlineto{\pgfqpoint{2.256278in}{1.530629in}}%
\pgfpathlineto{\pgfqpoint{2.257946in}{1.523356in}}%
\pgfpathlineto{\pgfqpoint{2.262291in}{1.521054in}}%
\pgfpathlineto{\pgfqpoint{2.271208in}{1.515609in}}%
\pgfpathlineto{\pgfqpoint{2.272886in}{1.507312in}}%
\pgfpathlineto{\pgfqpoint{2.271333in}{1.492492in}}%
\pgfpathclose%
\pgfusepath{fill}%
\end{pgfscope}%
\begin{pgfscope}%
\pgfpathrectangle{\pgfqpoint{0.100000in}{0.100000in}}{\pgfqpoint{3.420221in}{2.189500in}}%
\pgfusepath{clip}%
\pgfsetbuttcap%
\pgfsetmiterjoin%
\definecolor{currentfill}{rgb}{0.000000,0.278431,0.860784}%
\pgfsetfillcolor{currentfill}%
\pgfsetlinewidth{0.000000pt}%
\definecolor{currentstroke}{rgb}{0.000000,0.000000,0.000000}%
\pgfsetstrokecolor{currentstroke}%
\pgfsetstrokeopacity{0.000000}%
\pgfsetdash{}{0pt}%
\pgfpathmoveto{\pgfqpoint{1.762349in}{1.355562in}}%
\pgfpathlineto{\pgfqpoint{1.788305in}{1.354451in}}%
\pgfpathlineto{\pgfqpoint{1.801027in}{1.353950in}}%
\pgfpathlineto{\pgfqpoint{1.799725in}{1.321408in}}%
\pgfpathlineto{\pgfqpoint{1.768636in}{1.322719in}}%
\pgfpathlineto{\pgfqpoint{1.734953in}{1.324296in}}%
\pgfpathlineto{\pgfqpoint{1.736723in}{1.356796in}}%
\pgfpathclose%
\pgfusepath{fill}%
\end{pgfscope}%
\begin{pgfscope}%
\pgfpathrectangle{\pgfqpoint{0.100000in}{0.100000in}}{\pgfqpoint{3.420221in}{2.189500in}}%
\pgfusepath{clip}%
\pgfsetbuttcap%
\pgfsetmiterjoin%
\definecolor{currentfill}{rgb}{0.000000,0.443137,0.778431}%
\pgfsetfillcolor{currentfill}%
\pgfsetlinewidth{0.000000pt}%
\definecolor{currentstroke}{rgb}{0.000000,0.000000,0.000000}%
\pgfsetstrokecolor{currentstroke}%
\pgfsetstrokeopacity{0.000000}%
\pgfsetdash{}{0pt}%
\pgfpathmoveto{\pgfqpoint{2.255948in}{1.310757in}}%
\pgfpathlineto{\pgfqpoint{2.257819in}{1.321603in}}%
\pgfpathlineto{\pgfqpoint{2.252989in}{1.333305in}}%
\pgfpathlineto{\pgfqpoint{2.256755in}{1.344066in}}%
\pgfpathlineto{\pgfqpoint{2.236918in}{1.343567in}}%
\pgfpathlineto{\pgfqpoint{2.236009in}{1.369620in}}%
\pgfpathlineto{\pgfqpoint{2.235823in}{1.376468in}}%
\pgfpathlineto{\pgfqpoint{2.262093in}{1.377123in}}%
\pgfpathlineto{\pgfqpoint{2.262354in}{1.370567in}}%
\pgfpathlineto{\pgfqpoint{2.276776in}{1.370938in}}%
\pgfpathlineto{\pgfqpoint{2.270478in}{1.365500in}}%
\pgfpathlineto{\pgfqpoint{2.271841in}{1.361627in}}%
\pgfpathlineto{\pgfqpoint{2.277542in}{1.361124in}}%
\pgfpathlineto{\pgfqpoint{2.288796in}{1.365916in}}%
\pgfpathlineto{\pgfqpoint{2.290184in}{1.341751in}}%
\pgfpathlineto{\pgfqpoint{2.295169in}{1.322281in}}%
\pgfpathlineto{\pgfqpoint{2.282045in}{1.321421in}}%
\pgfpathlineto{\pgfqpoint{2.283406in}{1.302120in}}%
\pgfpathlineto{\pgfqpoint{2.274160in}{1.295043in}}%
\pgfpathlineto{\pgfqpoint{2.264066in}{1.294605in}}%
\pgfpathlineto{\pgfqpoint{2.257955in}{1.296255in}}%
\pgfpathclose%
\pgfusepath{fill}%
\end{pgfscope}%
\begin{pgfscope}%
\pgfpathrectangle{\pgfqpoint{0.100000in}{0.100000in}}{\pgfqpoint{3.420221in}{2.189500in}}%
\pgfusepath{clip}%
\pgfsetbuttcap%
\pgfsetmiterjoin%
\definecolor{currentfill}{rgb}{0.000000,0.600000,0.700000}%
\pgfsetfillcolor{currentfill}%
\pgfsetlinewidth{0.000000pt}%
\definecolor{currentstroke}{rgb}{0.000000,0.000000,0.000000}%
\pgfsetstrokecolor{currentstroke}%
\pgfsetstrokeopacity{0.000000}%
\pgfsetdash{}{0pt}%
\pgfpathmoveto{\pgfqpoint{3.016741in}{1.731302in}}%
\pgfpathlineto{\pgfqpoint{3.013034in}{1.740130in}}%
\pgfpathlineto{\pgfqpoint{3.017204in}{1.747876in}}%
\pgfpathlineto{\pgfqpoint{3.014689in}{1.752317in}}%
\pgfpathlineto{\pgfqpoint{3.006269in}{1.754354in}}%
\pgfpathlineto{\pgfqpoint{3.002631in}{1.762901in}}%
\pgfpathlineto{\pgfqpoint{3.010912in}{1.771605in}}%
\pgfpathlineto{\pgfqpoint{3.009483in}{1.775927in}}%
\pgfpathlineto{\pgfqpoint{3.021227in}{1.785128in}}%
\pgfpathlineto{\pgfqpoint{3.025063in}{1.790884in}}%
\pgfpathlineto{\pgfqpoint{3.026611in}{1.797651in}}%
\pgfpathlineto{\pgfqpoint{3.037223in}{1.814507in}}%
\pgfpathlineto{\pgfqpoint{3.053217in}{1.834234in}}%
\pgfpathlineto{\pgfqpoint{3.060522in}{1.841135in}}%
\pgfpathlineto{\pgfqpoint{3.067567in}{1.845224in}}%
\pgfpathlineto{\pgfqpoint{3.073128in}{1.844861in}}%
\pgfpathlineto{\pgfqpoint{3.078217in}{1.842756in}}%
\pgfpathlineto{\pgfqpoint{3.097403in}{1.786761in}}%
\pgfpathlineto{\pgfqpoint{3.098104in}{1.781414in}}%
\pgfpathlineto{\pgfqpoint{3.081784in}{1.775456in}}%
\pgfpathlineto{\pgfqpoint{3.071108in}{1.771545in}}%
\pgfpathlineto{\pgfqpoint{3.064684in}{1.773704in}}%
\pgfpathlineto{\pgfqpoint{3.075711in}{1.739052in}}%
\pgfpathlineto{\pgfqpoint{3.056336in}{1.719750in}}%
\pgfpathlineto{\pgfqpoint{3.043659in}{1.720900in}}%
\pgfpathlineto{\pgfqpoint{3.039232in}{1.736744in}}%
\pgfpathlineto{\pgfqpoint{3.025632in}{1.735237in}}%
\pgfpathclose%
\pgfusepath{fill}%
\end{pgfscope}%
\begin{pgfscope}%
\pgfpathrectangle{\pgfqpoint{0.100000in}{0.100000in}}{\pgfqpoint{3.420221in}{2.189500in}}%
\pgfusepath{clip}%
\pgfsetbuttcap%
\pgfsetmiterjoin%
\definecolor{currentfill}{rgb}{0.000000,0.454902,0.772549}%
\pgfsetfillcolor{currentfill}%
\pgfsetlinewidth{0.000000pt}%
\definecolor{currentstroke}{rgb}{0.000000,0.000000,0.000000}%
\pgfsetstrokecolor{currentstroke}%
\pgfsetstrokeopacity{0.000000}%
\pgfsetdash{}{0pt}%
\pgfpathmoveto{\pgfqpoint{2.155359in}{1.191691in}}%
\pgfpathlineto{\pgfqpoint{2.147470in}{1.186987in}}%
\pgfpathlineto{\pgfqpoint{2.142816in}{1.193833in}}%
\pgfpathlineto{\pgfqpoint{2.116066in}{1.193971in}}%
\pgfpathlineto{\pgfqpoint{2.116239in}{1.205956in}}%
\pgfpathlineto{\pgfqpoint{2.115043in}{1.215226in}}%
\pgfpathlineto{\pgfqpoint{2.115532in}{1.253198in}}%
\pgfpathlineto{\pgfqpoint{2.128427in}{1.252806in}}%
\pgfpathlineto{\pgfqpoint{2.128460in}{1.246274in}}%
\pgfpathlineto{\pgfqpoint{2.141469in}{1.234157in}}%
\pgfpathlineto{\pgfqpoint{2.154487in}{1.234104in}}%
\pgfpathlineto{\pgfqpoint{2.154378in}{1.227806in}}%
\pgfpathlineto{\pgfqpoint{2.160602in}{1.227633in}}%
\pgfpathlineto{\pgfqpoint{2.166987in}{1.224499in}}%
\pgfpathlineto{\pgfqpoint{2.168232in}{1.203856in}}%
\pgfpathlineto{\pgfqpoint{2.155187in}{1.203831in}}%
\pgfpathclose%
\pgfusepath{fill}%
\end{pgfscope}%
\begin{pgfscope}%
\pgfpathrectangle{\pgfqpoint{0.100000in}{0.100000in}}{\pgfqpoint{3.420221in}{2.189500in}}%
\pgfusepath{clip}%
\pgfsetbuttcap%
\pgfsetmiterjoin%
\definecolor{currentfill}{rgb}{0.000000,0.521569,0.739216}%
\pgfsetfillcolor{currentfill}%
\pgfsetlinewidth{0.000000pt}%
\definecolor{currentstroke}{rgb}{0.000000,0.000000,0.000000}%
\pgfsetstrokecolor{currentstroke}%
\pgfsetstrokeopacity{0.000000}%
\pgfsetdash{}{0pt}%
\pgfpathmoveto{\pgfqpoint{2.312832in}{1.024146in}}%
\pgfpathlineto{\pgfqpoint{2.304901in}{1.024524in}}%
\pgfpathlineto{\pgfqpoint{2.306483in}{1.018299in}}%
\pgfpathlineto{\pgfqpoint{2.299184in}{1.015785in}}%
\pgfpathlineto{\pgfqpoint{2.290209in}{1.015457in}}%
\pgfpathlineto{\pgfqpoint{2.287935in}{1.057260in}}%
\pgfpathlineto{\pgfqpoint{2.285733in}{1.064174in}}%
\pgfpathlineto{\pgfqpoint{2.291337in}{1.071587in}}%
\pgfpathlineto{\pgfqpoint{2.299205in}{1.078555in}}%
\pgfpathlineto{\pgfqpoint{2.300022in}{1.087035in}}%
\pgfpathlineto{\pgfqpoint{2.290087in}{1.095036in}}%
\pgfpathlineto{\pgfqpoint{2.293907in}{1.105211in}}%
\pgfpathlineto{\pgfqpoint{2.305297in}{1.105723in}}%
\pgfpathlineto{\pgfqpoint{2.305474in}{1.087726in}}%
\pgfpathlineto{\pgfqpoint{2.318776in}{1.088249in}}%
\pgfpathlineto{\pgfqpoint{2.328349in}{1.090102in}}%
\pgfpathlineto{\pgfqpoint{2.331467in}{1.085272in}}%
\pgfpathlineto{\pgfqpoint{2.332481in}{1.079018in}}%
\pgfpathlineto{\pgfqpoint{2.323065in}{1.076369in}}%
\pgfpathlineto{\pgfqpoint{2.329459in}{1.069399in}}%
\pgfpathlineto{\pgfqpoint{2.324451in}{1.065745in}}%
\pgfpathlineto{\pgfqpoint{2.322882in}{1.056675in}}%
\pgfpathlineto{\pgfqpoint{2.320093in}{1.051122in}}%
\pgfpathlineto{\pgfqpoint{2.320460in}{1.044996in}}%
\pgfpathlineto{\pgfqpoint{2.317298in}{1.040847in}}%
\pgfpathlineto{\pgfqpoint{2.309528in}{1.039039in}}%
\pgfpathlineto{\pgfqpoint{2.310395in}{1.028799in}}%
\pgfpathclose%
\pgfusepath{fill}%
\end{pgfscope}%
\begin{pgfscope}%
\pgfpathrectangle{\pgfqpoint{0.100000in}{0.100000in}}{\pgfqpoint{3.420221in}{2.189500in}}%
\pgfusepath{clip}%
\pgfsetbuttcap%
\pgfsetmiterjoin%
\definecolor{currentfill}{rgb}{0.000000,0.643137,0.678431}%
\pgfsetfillcolor{currentfill}%
\pgfsetlinewidth{0.000000pt}%
\definecolor{currentstroke}{rgb}{0.000000,0.000000,0.000000}%
\pgfsetstrokecolor{currentstroke}%
\pgfsetstrokeopacity{0.000000}%
\pgfsetdash{}{0pt}%
\pgfpathmoveto{\pgfqpoint{2.843479in}{0.822023in}}%
\pgfpathlineto{\pgfqpoint{2.840395in}{0.828065in}}%
\pgfpathlineto{\pgfqpoint{2.813198in}{0.830343in}}%
\pgfpathlineto{\pgfqpoint{2.805829in}{0.827285in}}%
\pgfpathlineto{\pgfqpoint{2.801465in}{0.832261in}}%
\pgfpathlineto{\pgfqpoint{2.802524in}{0.838111in}}%
\pgfpathlineto{\pgfqpoint{2.795105in}{0.841350in}}%
\pgfpathlineto{\pgfqpoint{2.785056in}{0.841197in}}%
\pgfpathlineto{\pgfqpoint{2.787720in}{0.846628in}}%
\pgfpathlineto{\pgfqpoint{2.794958in}{0.852801in}}%
\pgfpathlineto{\pgfqpoint{2.792572in}{0.855434in}}%
\pgfpathlineto{\pgfqpoint{2.795897in}{0.861685in}}%
\pgfpathlineto{\pgfqpoint{2.802257in}{0.868513in}}%
\pgfpathlineto{\pgfqpoint{2.809963in}{0.866837in}}%
\pgfpathlineto{\pgfqpoint{2.813246in}{0.868503in}}%
\pgfpathlineto{\pgfqpoint{2.816374in}{0.876340in}}%
\pgfpathlineto{\pgfqpoint{2.829978in}{0.881128in}}%
\pgfpathlineto{\pgfqpoint{2.836100in}{0.878627in}}%
\pgfpathlineto{\pgfqpoint{2.848715in}{0.890873in}}%
\pgfpathlineto{\pgfqpoint{2.852261in}{0.888580in}}%
\pgfpathlineto{\pgfqpoint{2.852405in}{0.883023in}}%
\pgfpathlineto{\pgfqpoint{2.858215in}{0.876688in}}%
\pgfpathlineto{\pgfqpoint{2.859707in}{0.866329in}}%
\pgfpathlineto{\pgfqpoint{2.863299in}{0.858754in}}%
\pgfpathlineto{\pgfqpoint{2.854883in}{0.850905in}}%
\pgfpathlineto{\pgfqpoint{2.856504in}{0.842230in}}%
\pgfpathlineto{\pgfqpoint{2.862519in}{0.836609in}}%
\pgfpathlineto{\pgfqpoint{2.864350in}{0.831907in}}%
\pgfpathclose%
\pgfusepath{fill}%
\end{pgfscope}%
\begin{pgfscope}%
\pgfpathrectangle{\pgfqpoint{0.100000in}{0.100000in}}{\pgfqpoint{3.420221in}{2.189500in}}%
\pgfusepath{clip}%
\pgfsetbuttcap%
\pgfsetmiterjoin%
\definecolor{currentfill}{rgb}{0.000000,0.572549,0.713725}%
\pgfsetfillcolor{currentfill}%
\pgfsetlinewidth{0.000000pt}%
\definecolor{currentstroke}{rgb}{0.000000,0.000000,0.000000}%
\pgfsetstrokecolor{currentstroke}%
\pgfsetstrokeopacity{0.000000}%
\pgfsetdash{}{0pt}%
\pgfpathmoveto{\pgfqpoint{1.368559in}{0.783876in}}%
\pgfpathlineto{\pgfqpoint{1.365815in}{0.755663in}}%
\pgfpathlineto{\pgfqpoint{1.360223in}{0.701157in}}%
\pgfpathlineto{\pgfqpoint{1.357623in}{0.683419in}}%
\pgfpathlineto{\pgfqpoint{1.353269in}{0.681268in}}%
\pgfpathlineto{\pgfqpoint{1.345185in}{0.691345in}}%
\pgfpathlineto{\pgfqpoint{1.339539in}{0.696268in}}%
\pgfpathlineto{\pgfqpoint{1.328396in}{0.701553in}}%
\pgfpathlineto{\pgfqpoint{1.328326in}{0.703925in}}%
\pgfpathlineto{\pgfqpoint{1.319095in}{0.712592in}}%
\pgfpathlineto{\pgfqpoint{1.316919in}{0.719941in}}%
\pgfpathlineto{\pgfqpoint{1.306758in}{0.727586in}}%
\pgfpathlineto{\pgfqpoint{1.296895in}{0.743838in}}%
\pgfpathlineto{\pgfqpoint{1.288928in}{0.747566in}}%
\pgfpathlineto{\pgfqpoint{1.281594in}{0.753620in}}%
\pgfpathlineto{\pgfqpoint{1.273047in}{0.774718in}}%
\pgfpathlineto{\pgfqpoint{1.264144in}{0.779789in}}%
\pgfpathlineto{\pgfqpoint{1.215257in}{0.786241in}}%
\pgfpathlineto{\pgfqpoint{1.223452in}{0.847680in}}%
\pgfpathlineto{\pgfqpoint{1.225235in}{0.860730in}}%
\pgfpathlineto{\pgfqpoint{1.251075in}{0.857198in}}%
\pgfpathlineto{\pgfqpoint{1.251495in}{0.860451in}}%
\pgfpathlineto{\pgfqpoint{1.288030in}{0.873223in}}%
\pgfpathlineto{\pgfqpoint{1.284875in}{0.866418in}}%
\pgfpathlineto{\pgfqpoint{1.275850in}{0.794839in}}%
\pgfpathlineto{\pgfqpoint{1.336058in}{0.787536in}}%
\pgfpathclose%
\pgfusepath{fill}%
\end{pgfscope}%
\begin{pgfscope}%
\pgfpathrectangle{\pgfqpoint{0.100000in}{0.100000in}}{\pgfqpoint{3.420221in}{2.189500in}}%
\pgfusepath{clip}%
\pgfsetbuttcap%
\pgfsetmiterjoin%
\definecolor{currentfill}{rgb}{0.000000,0.490196,0.754902}%
\pgfsetfillcolor{currentfill}%
\pgfsetlinewidth{0.000000pt}%
\definecolor{currentstroke}{rgb}{0.000000,0.000000,0.000000}%
\pgfsetstrokecolor{currentstroke}%
\pgfsetstrokeopacity{0.000000}%
\pgfsetdash{}{0pt}%
\pgfpathmoveto{\pgfqpoint{2.626172in}{0.839440in}}%
\pgfpathlineto{\pgfqpoint{2.622072in}{0.848394in}}%
\pgfpathlineto{\pgfqpoint{2.642360in}{0.850476in}}%
\pgfpathlineto{\pgfqpoint{2.639977in}{0.877124in}}%
\pgfpathlineto{\pgfqpoint{2.661915in}{0.879426in}}%
\pgfpathlineto{\pgfqpoint{2.661467in}{0.863462in}}%
\pgfpathlineto{\pgfqpoint{2.664364in}{0.854200in}}%
\pgfpathlineto{\pgfqpoint{2.672679in}{0.847767in}}%
\pgfpathlineto{\pgfqpoint{2.679603in}{0.845589in}}%
\pgfpathlineto{\pgfqpoint{2.671287in}{0.830538in}}%
\pgfpathlineto{\pgfqpoint{2.659341in}{0.827091in}}%
\pgfpathlineto{\pgfqpoint{2.658419in}{0.823956in}}%
\pgfpathlineto{\pgfqpoint{2.660563in}{0.804305in}}%
\pgfpathlineto{\pgfqpoint{2.643832in}{0.802266in}}%
\pgfpathlineto{\pgfqpoint{2.635869in}{0.794039in}}%
\pgfpathlineto{\pgfqpoint{2.636779in}{0.788748in}}%
\pgfpathlineto{\pgfqpoint{2.628587in}{0.787752in}}%
\pgfpathlineto{\pgfqpoint{2.623267in}{0.793746in}}%
\pgfpathlineto{\pgfqpoint{2.613593in}{0.792591in}}%
\pgfpathlineto{\pgfqpoint{2.611415in}{0.799037in}}%
\pgfpathlineto{\pgfqpoint{2.609995in}{0.812175in}}%
\pgfpathlineto{\pgfqpoint{2.616565in}{0.812949in}}%
\pgfpathlineto{\pgfqpoint{2.615868in}{0.817258in}}%
\pgfpathlineto{\pgfqpoint{2.633197in}{0.819458in}}%
\pgfpathlineto{\pgfqpoint{2.636568in}{0.822680in}}%
\pgfpathlineto{\pgfqpoint{2.628888in}{0.831927in}}%
\pgfpathclose%
\pgfusepath{fill}%
\end{pgfscope}%
\begin{pgfscope}%
\pgfpathrectangle{\pgfqpoint{0.100000in}{0.100000in}}{\pgfqpoint{3.420221in}{2.189500in}}%
\pgfusepath{clip}%
\pgfsetbuttcap%
\pgfsetmiterjoin%
\definecolor{currentfill}{rgb}{0.000000,0.364706,0.817647}%
\pgfsetfillcolor{currentfill}%
\pgfsetlinewidth{0.000000pt}%
\definecolor{currentstroke}{rgb}{0.000000,0.000000,0.000000}%
\pgfsetstrokecolor{currentstroke}%
\pgfsetstrokeopacity{0.000000}%
\pgfsetdash{}{0pt}%
\pgfpathmoveto{\pgfqpoint{2.547789in}{0.832030in}}%
\pgfpathlineto{\pgfqpoint{2.545857in}{0.828293in}}%
\pgfpathlineto{\pgfqpoint{2.526856in}{0.826396in}}%
\pgfpathlineto{\pgfqpoint{2.527210in}{0.823101in}}%
\pgfpathlineto{\pgfqpoint{2.514329in}{0.822121in}}%
\pgfpathlineto{\pgfqpoint{2.507970in}{0.821491in}}%
\pgfpathlineto{\pgfqpoint{2.506727in}{0.834528in}}%
\pgfpathlineto{\pgfqpoint{2.515752in}{0.835271in}}%
\pgfpathlineto{\pgfqpoint{2.514268in}{0.851300in}}%
\pgfpathlineto{\pgfqpoint{2.508133in}{0.854419in}}%
\pgfpathlineto{\pgfqpoint{2.504151in}{0.859163in}}%
\pgfpathlineto{\pgfqpoint{2.503539in}{0.865207in}}%
\pgfpathlineto{\pgfqpoint{2.493471in}{0.869972in}}%
\pgfpathlineto{\pgfqpoint{2.486144in}{0.878371in}}%
\pgfpathlineto{\pgfqpoint{2.486773in}{0.883890in}}%
\pgfpathlineto{\pgfqpoint{2.491591in}{0.891922in}}%
\pgfpathlineto{\pgfqpoint{2.496912in}{0.893445in}}%
\pgfpathlineto{\pgfqpoint{2.499838in}{0.898039in}}%
\pgfpathlineto{\pgfqpoint{2.500291in}{0.904684in}}%
\pgfpathlineto{\pgfqpoint{2.503456in}{0.911430in}}%
\pgfpathlineto{\pgfqpoint{2.495766in}{0.914405in}}%
\pgfpathlineto{\pgfqpoint{2.493708in}{0.918031in}}%
\pgfpathlineto{\pgfqpoint{2.491461in}{0.943662in}}%
\pgfpathlineto{\pgfqpoint{2.491354in}{0.944747in}}%
\pgfpathlineto{\pgfqpoint{2.524078in}{0.947004in}}%
\pgfpathlineto{\pgfqpoint{2.530547in}{0.947463in}}%
\pgfpathlineto{\pgfqpoint{2.542768in}{0.933242in}}%
\pgfpathlineto{\pgfqpoint{2.541354in}{0.929132in}}%
\pgfpathlineto{\pgfqpoint{2.544427in}{0.924686in}}%
\pgfpathlineto{\pgfqpoint{2.550048in}{0.925622in}}%
\pgfpathlineto{\pgfqpoint{2.554196in}{0.919259in}}%
\pgfpathlineto{\pgfqpoint{2.559427in}{0.915451in}}%
\pgfpathlineto{\pgfqpoint{2.560929in}{0.910799in}}%
\pgfpathlineto{\pgfqpoint{2.555629in}{0.902748in}}%
\pgfpathlineto{\pgfqpoint{2.551589in}{0.895462in}}%
\pgfpathlineto{\pgfqpoint{2.553815in}{0.892336in}}%
\pgfpathlineto{\pgfqpoint{2.543155in}{0.879760in}}%
\pgfpathlineto{\pgfqpoint{2.545156in}{0.872734in}}%
\pgfpathlineto{\pgfqpoint{2.539246in}{0.868080in}}%
\pgfpathlineto{\pgfqpoint{2.536876in}{0.862823in}}%
\pgfpathlineto{\pgfqpoint{2.538561in}{0.843190in}}%
\pgfpathlineto{\pgfqpoint{2.542333in}{0.836023in}}%
\pgfpathclose%
\pgfusepath{fill}%
\end{pgfscope}%
\begin{pgfscope}%
\pgfpathrectangle{\pgfqpoint{0.100000in}{0.100000in}}{\pgfqpoint{3.420221in}{2.189500in}}%
\pgfusepath{clip}%
\pgfsetbuttcap%
\pgfsetmiterjoin%
\definecolor{currentfill}{rgb}{0.000000,0.203922,0.898039}%
\pgfsetfillcolor{currentfill}%
\pgfsetlinewidth{0.000000pt}%
\definecolor{currentstroke}{rgb}{0.000000,0.000000,0.000000}%
\pgfsetstrokecolor{currentstroke}%
\pgfsetstrokeopacity{0.000000}%
\pgfsetdash{}{0pt}%
\pgfpathmoveto{\pgfqpoint{1.931934in}{2.028569in}}%
\pgfpathlineto{\pgfqpoint{1.931624in}{2.007059in}}%
\pgfpathlineto{\pgfqpoint{1.932284in}{1.993898in}}%
\pgfpathlineto{\pgfqpoint{1.893283in}{1.994630in}}%
\pgfpathlineto{\pgfqpoint{1.894774in}{1.988481in}}%
\pgfpathlineto{\pgfqpoint{1.893272in}{1.982670in}}%
\pgfpathlineto{\pgfqpoint{1.895156in}{1.974588in}}%
\pgfpathlineto{\pgfqpoint{1.893757in}{1.968006in}}%
\pgfpathlineto{\pgfqpoint{1.855200in}{1.969173in}}%
\pgfpathlineto{\pgfqpoint{1.835528in}{1.969852in}}%
\pgfpathlineto{\pgfqpoint{1.836026in}{1.983102in}}%
\pgfpathlineto{\pgfqpoint{1.835198in}{1.996375in}}%
\pgfpathlineto{\pgfqpoint{1.802403in}{1.997722in}}%
\pgfpathlineto{\pgfqpoint{1.801504in}{2.010992in}}%
\pgfpathlineto{\pgfqpoint{1.802451in}{2.032421in}}%
\pgfpathlineto{\pgfqpoint{1.849912in}{2.030614in}}%
\pgfpathlineto{\pgfqpoint{1.890819in}{2.029445in}}%
\pgfpathclose%
\pgfusepath{fill}%
\end{pgfscope}%
\begin{pgfscope}%
\pgfpathrectangle{\pgfqpoint{0.100000in}{0.100000in}}{\pgfqpoint{3.420221in}{2.189500in}}%
\pgfusepath{clip}%
\pgfsetbuttcap%
\pgfsetmiterjoin%
\definecolor{currentfill}{rgb}{0.000000,0.545098,0.727451}%
\pgfsetfillcolor{currentfill}%
\pgfsetlinewidth{0.000000pt}%
\definecolor{currentstroke}{rgb}{0.000000,0.000000,0.000000}%
\pgfsetstrokecolor{currentstroke}%
\pgfsetstrokeopacity{0.000000}%
\pgfsetdash{}{0pt}%
\pgfpathmoveto{\pgfqpoint{1.716417in}{0.343192in}}%
\pgfpathlineto{\pgfqpoint{1.714257in}{0.355034in}}%
\pgfpathlineto{\pgfqpoint{1.710603in}{0.364286in}}%
\pgfpathlineto{\pgfqpoint{1.703590in}{0.371541in}}%
\pgfpathlineto{\pgfqpoint{1.698999in}{0.381929in}}%
\pgfpathlineto{\pgfqpoint{1.701081in}{0.390291in}}%
\pgfpathlineto{\pgfqpoint{1.696379in}{0.402928in}}%
\pgfpathlineto{\pgfqpoint{1.698585in}{0.414071in}}%
\pgfpathlineto{\pgfqpoint{1.695539in}{0.415055in}}%
\pgfpathlineto{\pgfqpoint{1.694885in}{0.423430in}}%
\pgfpathlineto{\pgfqpoint{1.682967in}{0.428655in}}%
\pgfpathlineto{\pgfqpoint{1.677219in}{0.437198in}}%
\pgfpathlineto{\pgfqpoint{1.672990in}{0.439443in}}%
\pgfpathlineto{\pgfqpoint{1.672314in}{0.447033in}}%
\pgfpathlineto{\pgfqpoint{1.666169in}{0.454629in}}%
\pgfpathlineto{\pgfqpoint{1.660570in}{0.466379in}}%
\pgfpathlineto{\pgfqpoint{1.652157in}{0.470888in}}%
\pgfpathlineto{\pgfqpoint{1.658765in}{0.470587in}}%
\pgfpathlineto{\pgfqpoint{1.707060in}{0.468578in}}%
\pgfpathlineto{\pgfqpoint{1.706770in}{0.455293in}}%
\pgfpathlineto{\pgfqpoint{1.746239in}{0.455579in}}%
\pgfpathlineto{\pgfqpoint{1.744331in}{0.401905in}}%
\pgfpathlineto{\pgfqpoint{1.761013in}{0.401617in}}%
\pgfpathlineto{\pgfqpoint{1.764696in}{0.391978in}}%
\pgfpathlineto{\pgfqpoint{1.765964in}{0.378131in}}%
\pgfpathlineto{\pgfqpoint{1.769279in}{0.378049in}}%
\pgfpathlineto{\pgfqpoint{1.768173in}{0.357274in}}%
\pgfpathlineto{\pgfqpoint{1.731867in}{0.358914in}}%
\pgfpathlineto{\pgfqpoint{1.727608in}{0.350616in}}%
\pgfpathclose%
\pgfusepath{fill}%
\end{pgfscope}%
\begin{pgfscope}%
\pgfpathrectangle{\pgfqpoint{0.100000in}{0.100000in}}{\pgfqpoint{3.420221in}{2.189500in}}%
\pgfusepath{clip}%
\pgfsetbuttcap%
\pgfsetmiterjoin%
\definecolor{currentfill}{rgb}{0.000000,0.305882,0.847059}%
\pgfsetfillcolor{currentfill}%
\pgfsetlinewidth{0.000000pt}%
\definecolor{currentstroke}{rgb}{0.000000,0.000000,0.000000}%
\pgfsetstrokecolor{currentstroke}%
\pgfsetstrokeopacity{0.000000}%
\pgfsetdash{}{0pt}%
\pgfpathmoveto{\pgfqpoint{2.833520in}{0.733453in}}%
\pgfpathlineto{\pgfqpoint{2.827837in}{0.738300in}}%
\pgfpathlineto{\pgfqpoint{2.827447in}{0.747545in}}%
\pgfpathlineto{\pgfqpoint{2.833319in}{0.753762in}}%
\pgfpathlineto{\pgfqpoint{2.839273in}{0.756963in}}%
\pgfpathlineto{\pgfqpoint{2.835475in}{0.761421in}}%
\pgfpathlineto{\pgfqpoint{2.843287in}{0.760451in}}%
\pgfpathlineto{\pgfqpoint{2.850887in}{0.762924in}}%
\pgfpathlineto{\pgfqpoint{2.856140in}{0.760891in}}%
\pgfpathlineto{\pgfqpoint{2.861733in}{0.772666in}}%
\pgfpathlineto{\pgfqpoint{2.857466in}{0.777143in}}%
\pgfpathlineto{\pgfqpoint{2.863488in}{0.781076in}}%
\pgfpathlineto{\pgfqpoint{2.866872in}{0.790921in}}%
\pgfpathlineto{\pgfqpoint{2.871330in}{0.787092in}}%
\pgfpathlineto{\pgfqpoint{2.882045in}{0.789449in}}%
\pgfpathlineto{\pgfqpoint{2.887327in}{0.783872in}}%
\pgfpathlineto{\pgfqpoint{2.888744in}{0.780908in}}%
\pgfpathlineto{\pgfqpoint{2.884827in}{0.769859in}}%
\pgfpathlineto{\pgfqpoint{2.886687in}{0.760281in}}%
\pgfpathlineto{\pgfqpoint{2.878941in}{0.748101in}}%
\pgfpathlineto{\pgfqpoint{2.877879in}{0.739852in}}%
\pgfpathlineto{\pgfqpoint{2.868426in}{0.745230in}}%
\pgfpathlineto{\pgfqpoint{2.854035in}{0.747218in}}%
\pgfpathlineto{\pgfqpoint{2.845693in}{0.738038in}}%
\pgfpathlineto{\pgfqpoint{2.837381in}{0.738967in}}%
\pgfpathclose%
\pgfusepath{fill}%
\end{pgfscope}%
\begin{pgfscope}%
\pgfpathrectangle{\pgfqpoint{0.100000in}{0.100000in}}{\pgfqpoint{3.420221in}{2.189500in}}%
\pgfusepath{clip}%
\pgfsetbuttcap%
\pgfsetmiterjoin%
\definecolor{currentfill}{rgb}{0.000000,0.266667,0.866667}%
\pgfsetfillcolor{currentfill}%
\pgfsetlinewidth{0.000000pt}%
\definecolor{currentstroke}{rgb}{0.000000,0.000000,0.000000}%
\pgfsetstrokecolor{currentstroke}%
\pgfsetstrokeopacity{0.000000}%
\pgfsetdash{}{0pt}%
\pgfpathmoveto{\pgfqpoint{1.495057in}{1.492423in}}%
\pgfpathlineto{\pgfqpoint{1.502857in}{1.570607in}}%
\pgfpathlineto{\pgfqpoint{1.505742in}{1.599669in}}%
\pgfpathlineto{\pgfqpoint{1.563276in}{1.594266in}}%
\pgfpathlineto{\pgfqpoint{1.574694in}{1.593214in}}%
\pgfpathlineto{\pgfqpoint{1.574342in}{1.577063in}}%
\pgfpathlineto{\pgfqpoint{1.572128in}{1.551237in}}%
\pgfpathlineto{\pgfqpoint{1.573451in}{1.551129in}}%
\pgfpathlineto{\pgfqpoint{1.571626in}{1.525102in}}%
\pgfpathlineto{\pgfqpoint{1.574663in}{1.518290in}}%
\pgfpathlineto{\pgfqpoint{1.573431in}{1.504659in}}%
\pgfpathlineto{\pgfqpoint{1.575003in}{1.496376in}}%
\pgfpathlineto{\pgfqpoint{1.573370in}{1.475782in}}%
\pgfpathlineto{\pgfqpoint{1.532356in}{1.479298in}}%
\pgfpathlineto{\pgfqpoint{1.531986in}{1.476085in}}%
\pgfpathlineto{\pgfqpoint{1.493796in}{1.479665in}}%
\pgfpathclose%
\pgfusepath{fill}%
\end{pgfscope}%
\begin{pgfscope}%
\pgfpathrectangle{\pgfqpoint{0.100000in}{0.100000in}}{\pgfqpoint{3.420221in}{2.189500in}}%
\pgfusepath{clip}%
\pgfsetbuttcap%
\pgfsetmiterjoin%
\definecolor{currentfill}{rgb}{0.000000,0.807843,0.596078}%
\pgfsetfillcolor{currentfill}%
\pgfsetlinewidth{0.000000pt}%
\definecolor{currentstroke}{rgb}{0.000000,0.000000,0.000000}%
\pgfsetstrokecolor{currentstroke}%
\pgfsetstrokeopacity{0.000000}%
\pgfsetdash{}{0pt}%
\pgfpathmoveto{\pgfqpoint{2.273744in}{0.802309in}}%
\pgfpathlineto{\pgfqpoint{2.273549in}{0.799049in}}%
\pgfpathlineto{\pgfqpoint{2.264985in}{0.798606in}}%
\pgfpathlineto{\pgfqpoint{2.265391in}{0.788582in}}%
\pgfpathlineto{\pgfqpoint{2.259549in}{0.788233in}}%
\pgfpathlineto{\pgfqpoint{2.258416in}{0.798343in}}%
\pgfpathlineto{\pgfqpoint{2.253368in}{0.798171in}}%
\pgfpathlineto{\pgfqpoint{2.247981in}{0.798710in}}%
\pgfpathlineto{\pgfqpoint{2.246235in}{0.802730in}}%
\pgfpathlineto{\pgfqpoint{2.251781in}{0.809512in}}%
\pgfpathlineto{\pgfqpoint{2.245071in}{0.811827in}}%
\pgfpathlineto{\pgfqpoint{2.245299in}{0.816386in}}%
\pgfpathlineto{\pgfqpoint{2.250633in}{0.821619in}}%
\pgfpathlineto{\pgfqpoint{2.249326in}{0.828169in}}%
\pgfpathlineto{\pgfqpoint{2.244232in}{0.830051in}}%
\pgfpathlineto{\pgfqpoint{2.248526in}{0.840348in}}%
\pgfpathlineto{\pgfqpoint{2.248223in}{0.846557in}}%
\pgfpathlineto{\pgfqpoint{2.244845in}{0.852296in}}%
\pgfpathlineto{\pgfqpoint{2.244507in}{0.858382in}}%
\pgfpathlineto{\pgfqpoint{2.238807in}{0.872271in}}%
\pgfpathlineto{\pgfqpoint{2.244363in}{0.874547in}}%
\pgfpathlineto{\pgfqpoint{2.239025in}{0.879265in}}%
\pgfpathlineto{\pgfqpoint{2.243563in}{0.888479in}}%
\pgfpathlineto{\pgfqpoint{2.252172in}{0.888041in}}%
\pgfpathlineto{\pgfqpoint{2.247433in}{0.895180in}}%
\pgfpathlineto{\pgfqpoint{2.250986in}{0.899765in}}%
\pgfpathlineto{\pgfqpoint{2.246019in}{0.903296in}}%
\pgfpathlineto{\pgfqpoint{2.257596in}{0.907455in}}%
\pgfpathlineto{\pgfqpoint{2.259131in}{0.911482in}}%
\pgfpathlineto{\pgfqpoint{2.253810in}{0.914439in}}%
\pgfpathlineto{\pgfqpoint{2.253756in}{0.914534in}}%
\pgfpathlineto{\pgfqpoint{2.272198in}{0.915290in}}%
\pgfpathlineto{\pgfqpoint{2.272613in}{0.905427in}}%
\pgfpathlineto{\pgfqpoint{2.285679in}{0.905923in}}%
\pgfpathlineto{\pgfqpoint{2.286275in}{0.892826in}}%
\pgfpathlineto{\pgfqpoint{2.287992in}{0.856600in}}%
\pgfpathlineto{\pgfqpoint{2.281562in}{0.856258in}}%
\pgfpathlineto{\pgfqpoint{2.281792in}{0.851849in}}%
\pgfpathlineto{\pgfqpoint{2.271674in}{0.851325in}}%
\pgfpathlineto{\pgfqpoint{2.275960in}{0.838447in}}%
\pgfpathlineto{\pgfqpoint{2.276676in}{0.825358in}}%
\pgfpathlineto{\pgfqpoint{2.270461in}{0.818516in}}%
\pgfpathlineto{\pgfqpoint{2.273685in}{0.816434in}}%
\pgfpathclose%
\pgfusepath{fill}%
\end{pgfscope}%
\begin{pgfscope}%
\pgfpathrectangle{\pgfqpoint{0.100000in}{0.100000in}}{\pgfqpoint{3.420221in}{2.189500in}}%
\pgfusepath{clip}%
\pgfsetbuttcap%
\pgfsetmiterjoin%
\definecolor{currentfill}{rgb}{0.000000,0.756863,0.621569}%
\pgfsetfillcolor{currentfill}%
\pgfsetlinewidth{0.000000pt}%
\definecolor{currentstroke}{rgb}{0.000000,0.000000,0.000000}%
\pgfsetstrokecolor{currentstroke}%
\pgfsetstrokeopacity{0.000000}%
\pgfsetdash{}{0pt}%
\pgfpathmoveto{\pgfqpoint{0.460516in}{1.919845in}}%
\pgfpathlineto{\pgfqpoint{0.472459in}{1.946104in}}%
\pgfpathlineto{\pgfqpoint{0.474328in}{1.954622in}}%
\pgfpathlineto{\pgfqpoint{0.488671in}{1.980454in}}%
\pgfpathlineto{\pgfqpoint{0.493266in}{1.994614in}}%
\pgfpathlineto{\pgfqpoint{0.502258in}{2.016135in}}%
\pgfpathlineto{\pgfqpoint{0.503586in}{2.025224in}}%
\pgfpathlineto{\pgfqpoint{0.515853in}{2.020394in}}%
\pgfpathlineto{\pgfqpoint{0.533937in}{2.015037in}}%
\pgfpathlineto{\pgfqpoint{0.532297in}{2.009940in}}%
\pgfpathlineto{\pgfqpoint{0.527488in}{2.005744in}}%
\pgfpathlineto{\pgfqpoint{0.526542in}{1.999227in}}%
\pgfpathlineto{\pgfqpoint{0.520695in}{1.991907in}}%
\pgfpathlineto{\pgfqpoint{0.515725in}{1.976277in}}%
\pgfpathlineto{\pgfqpoint{0.499511in}{1.981542in}}%
\pgfpathlineto{\pgfqpoint{0.496213in}{1.971544in}}%
\pgfpathlineto{\pgfqpoint{0.499270in}{1.970525in}}%
\pgfpathlineto{\pgfqpoint{0.491343in}{1.946410in}}%
\pgfpathlineto{\pgfqpoint{0.497148in}{1.943101in}}%
\pgfpathlineto{\pgfqpoint{0.490703in}{1.922376in}}%
\pgfpathlineto{\pgfqpoint{0.484414in}{1.924441in}}%
\pgfpathlineto{\pgfqpoint{0.482985in}{1.919170in}}%
\pgfpathlineto{\pgfqpoint{0.478103in}{1.914672in}}%
\pgfpathclose%
\pgfusepath{fill}%
\end{pgfscope}%
\begin{pgfscope}%
\pgfpathrectangle{\pgfqpoint{0.100000in}{0.100000in}}{\pgfqpoint{3.420221in}{2.189500in}}%
\pgfusepath{clip}%
\pgfsetbuttcap%
\pgfsetmiterjoin%
\definecolor{currentfill}{rgb}{0.000000,0.768627,0.615686}%
\pgfsetfillcolor{currentfill}%
\pgfsetlinewidth{0.000000pt}%
\definecolor{currentstroke}{rgb}{0.000000,0.000000,0.000000}%
\pgfsetstrokecolor{currentstroke}%
\pgfsetstrokeopacity{0.000000}%
\pgfsetdash{}{0pt}%
\pgfpathmoveto{\pgfqpoint{0.361031in}{0.547278in}}%
\pgfpathlineto{\pgfqpoint{0.360318in}{0.552008in}}%
\pgfpathlineto{\pgfqpoint{0.361407in}{0.552199in}}%
\pgfpathclose%
\pgfusepath{fill}%
\end{pgfscope}%
\begin{pgfscope}%
\pgfpathrectangle{\pgfqpoint{0.100000in}{0.100000in}}{\pgfqpoint{3.420221in}{2.189500in}}%
\pgfusepath{clip}%
\pgfsetbuttcap%
\pgfsetmiterjoin%
\definecolor{currentfill}{rgb}{0.000000,0.768627,0.615686}%
\pgfsetfillcolor{currentfill}%
\pgfsetlinewidth{0.000000pt}%
\definecolor{currentstroke}{rgb}{0.000000,0.000000,0.000000}%
\pgfsetstrokecolor{currentstroke}%
\pgfsetstrokeopacity{0.000000}%
\pgfsetdash{}{0pt}%
\pgfpathmoveto{\pgfqpoint{0.393149in}{0.515249in}}%
\pgfpathlineto{\pgfqpoint{0.392731in}{0.519144in}}%
\pgfpathlineto{\pgfqpoint{0.394700in}{0.517671in}}%
\pgfpathlineto{\pgfqpoint{0.395892in}{0.518168in}}%
\pgfpathlineto{\pgfqpoint{0.395197in}{0.523154in}}%
\pgfpathlineto{\pgfqpoint{0.396471in}{0.523567in}}%
\pgfpathlineto{\pgfqpoint{0.398706in}{0.520373in}}%
\pgfpathlineto{\pgfqpoint{0.398247in}{0.519272in}}%
\pgfpathlineto{\pgfqpoint{0.399259in}{0.516349in}}%
\pgfpathlineto{\pgfqpoint{0.394263in}{0.516261in}}%
\pgfpathclose%
\pgfusepath{fill}%
\end{pgfscope}%
\begin{pgfscope}%
\pgfpathrectangle{\pgfqpoint{0.100000in}{0.100000in}}{\pgfqpoint{3.420221in}{2.189500in}}%
\pgfusepath{clip}%
\pgfsetbuttcap%
\pgfsetmiterjoin%
\definecolor{currentfill}{rgb}{0.000000,0.768627,0.615686}%
\pgfsetfillcolor{currentfill}%
\pgfsetlinewidth{0.000000pt}%
\definecolor{currentstroke}{rgb}{0.000000,0.000000,0.000000}%
\pgfsetstrokecolor{currentstroke}%
\pgfsetstrokeopacity{0.000000}%
\pgfsetdash{}{0pt}%
\pgfpathmoveto{\pgfqpoint{0.404167in}{0.501003in}}%
\pgfpathlineto{\pgfqpoint{0.403681in}{0.504234in}}%
\pgfpathlineto{\pgfqpoint{0.408795in}{0.504250in}}%
\pgfpathlineto{\pgfqpoint{0.410886in}{0.506391in}}%
\pgfpathlineto{\pgfqpoint{0.412196in}{0.506002in}}%
\pgfpathlineto{\pgfqpoint{0.414053in}{0.503810in}}%
\pgfpathlineto{\pgfqpoint{0.412055in}{0.503405in}}%
\pgfpathlineto{\pgfqpoint{0.410961in}{0.501966in}}%
\pgfpathlineto{\pgfqpoint{0.412779in}{0.499252in}}%
\pgfpathlineto{\pgfqpoint{0.413236in}{0.497277in}}%
\pgfpathlineto{\pgfqpoint{0.411187in}{0.497157in}}%
\pgfpathlineto{\pgfqpoint{0.409977in}{0.498648in}}%
\pgfpathlineto{\pgfqpoint{0.406848in}{0.500857in}}%
\pgfpathclose%
\pgfusepath{fill}%
\end{pgfscope}%
\begin{pgfscope}%
\pgfpathrectangle{\pgfqpoint{0.100000in}{0.100000in}}{\pgfqpoint{3.420221in}{2.189500in}}%
\pgfusepath{clip}%
\pgfsetbuttcap%
\pgfsetmiterjoin%
\definecolor{currentfill}{rgb}{0.000000,0.768627,0.615686}%
\pgfsetfillcolor{currentfill}%
\pgfsetlinewidth{0.000000pt}%
\definecolor{currentstroke}{rgb}{0.000000,0.000000,0.000000}%
\pgfsetstrokecolor{currentstroke}%
\pgfsetstrokeopacity{0.000000}%
\pgfsetdash{}{0pt}%
\pgfpathmoveto{\pgfqpoint{0.400992in}{0.507991in}}%
\pgfpathlineto{\pgfqpoint{0.400985in}{0.510287in}}%
\pgfpathlineto{\pgfqpoint{0.396607in}{0.512457in}}%
\pgfpathlineto{\pgfqpoint{0.398460in}{0.513824in}}%
\pgfpathlineto{\pgfqpoint{0.401054in}{0.511088in}}%
\pgfpathlineto{\pgfqpoint{0.404683in}{0.509685in}}%
\pgfpathlineto{\pgfqpoint{0.407091in}{0.511227in}}%
\pgfpathlineto{\pgfqpoint{0.407625in}{0.508750in}}%
\pgfpathlineto{\pgfqpoint{0.405059in}{0.508632in}}%
\pgfpathlineto{\pgfqpoint{0.403101in}{0.506980in}}%
\pgfpathclose%
\pgfusepath{fill}%
\end{pgfscope}%
\begin{pgfscope}%
\pgfpathrectangle{\pgfqpoint{0.100000in}{0.100000in}}{\pgfqpoint{3.420221in}{2.189500in}}%
\pgfusepath{clip}%
\pgfsetbuttcap%
\pgfsetmiterjoin%
\definecolor{currentfill}{rgb}{0.000000,0.768627,0.615686}%
\pgfsetfillcolor{currentfill}%
\pgfsetlinewidth{0.000000pt}%
\definecolor{currentstroke}{rgb}{0.000000,0.000000,0.000000}%
\pgfsetstrokecolor{currentstroke}%
\pgfsetstrokeopacity{0.000000}%
\pgfsetdash{}{0pt}%
\pgfpathmoveto{\pgfqpoint{0.374821in}{0.550868in}}%
\pgfpathlineto{\pgfqpoint{0.374334in}{0.553379in}}%
\pgfpathlineto{\pgfqpoint{0.377376in}{0.553869in}}%
\pgfpathlineto{\pgfqpoint{0.378148in}{0.552731in}}%
\pgfpathlineto{\pgfqpoint{0.377426in}{0.550537in}}%
\pgfpathclose%
\pgfusepath{fill}%
\end{pgfscope}%
\begin{pgfscope}%
\pgfpathrectangle{\pgfqpoint{0.100000in}{0.100000in}}{\pgfqpoint{3.420221in}{2.189500in}}%
\pgfusepath{clip}%
\pgfsetbuttcap%
\pgfsetmiterjoin%
\definecolor{currentfill}{rgb}{0.000000,0.768627,0.615686}%
\pgfsetfillcolor{currentfill}%
\pgfsetlinewidth{0.000000pt}%
\definecolor{currentstroke}{rgb}{0.000000,0.000000,0.000000}%
\pgfsetstrokecolor{currentstroke}%
\pgfsetstrokeopacity{0.000000}%
\pgfsetdash{}{0pt}%
\pgfpathmoveto{\pgfqpoint{0.415038in}{0.495208in}}%
\pgfpathlineto{\pgfqpoint{0.414377in}{0.497557in}}%
\pgfpathlineto{\pgfqpoint{0.418678in}{0.495518in}}%
\pgfpathlineto{\pgfqpoint{0.417421in}{0.494277in}}%
\pgfpathclose%
\pgfusepath{fill}%
\end{pgfscope}%
\begin{pgfscope}%
\pgfpathrectangle{\pgfqpoint{0.100000in}{0.100000in}}{\pgfqpoint{3.420221in}{2.189500in}}%
\pgfusepath{clip}%
\pgfsetbuttcap%
\pgfsetmiterjoin%
\definecolor{currentfill}{rgb}{0.000000,0.768627,0.615686}%
\pgfsetfillcolor{currentfill}%
\pgfsetlinewidth{0.000000pt}%
\definecolor{currentstroke}{rgb}{0.000000,0.000000,0.000000}%
\pgfsetstrokecolor{currentstroke}%
\pgfsetstrokeopacity{0.000000}%
\pgfsetdash{}{0pt}%
\pgfpathmoveto{\pgfqpoint{0.364389in}{0.565983in}}%
\pgfpathlineto{\pgfqpoint{0.364584in}{0.567783in}}%
\pgfpathlineto{\pgfqpoint{0.366769in}{0.567042in}}%
\pgfpathlineto{\pgfqpoint{0.366235in}{0.565358in}}%
\pgfpathclose%
\pgfusepath{fill}%
\end{pgfscope}%
\begin{pgfscope}%
\pgfpathrectangle{\pgfqpoint{0.100000in}{0.100000in}}{\pgfqpoint{3.420221in}{2.189500in}}%
\pgfusepath{clip}%
\pgfsetbuttcap%
\pgfsetmiterjoin%
\definecolor{currentfill}{rgb}{0.000000,0.768627,0.615686}%
\pgfsetfillcolor{currentfill}%
\pgfsetlinewidth{0.000000pt}%
\definecolor{currentstroke}{rgb}{0.000000,0.000000,0.000000}%
\pgfsetstrokecolor{currentstroke}%
\pgfsetstrokeopacity{0.000000}%
\pgfsetdash{}{0pt}%
\pgfpathmoveto{\pgfqpoint{0.418993in}{0.497427in}}%
\pgfpathlineto{\pgfqpoint{0.418827in}{0.499877in}}%
\pgfpathlineto{\pgfqpoint{0.420893in}{0.501042in}}%
\pgfpathlineto{\pgfqpoint{0.421674in}{0.498768in}}%
\pgfpathclose%
\pgfusepath{fill}%
\end{pgfscope}%
\begin{pgfscope}%
\pgfpathrectangle{\pgfqpoint{0.100000in}{0.100000in}}{\pgfqpoint{3.420221in}{2.189500in}}%
\pgfusepath{clip}%
\pgfsetbuttcap%
\pgfsetmiterjoin%
\definecolor{currentfill}{rgb}{0.000000,0.768627,0.615686}%
\pgfsetfillcolor{currentfill}%
\pgfsetlinewidth{0.000000pt}%
\definecolor{currentstroke}{rgb}{0.000000,0.000000,0.000000}%
\pgfsetstrokecolor{currentstroke}%
\pgfsetstrokeopacity{0.000000}%
\pgfsetdash{}{0pt}%
\pgfpathmoveto{\pgfqpoint{0.352060in}{0.580786in}}%
\pgfpathlineto{\pgfqpoint{0.352128in}{0.583037in}}%
\pgfpathlineto{\pgfqpoint{0.355288in}{0.582539in}}%
\pgfpathlineto{\pgfqpoint{0.357385in}{0.581275in}}%
\pgfpathlineto{\pgfqpoint{0.359883in}{0.581936in}}%
\pgfpathlineto{\pgfqpoint{0.361218in}{0.580321in}}%
\pgfpathlineto{\pgfqpoint{0.357536in}{0.580035in}}%
\pgfpathlineto{\pgfqpoint{0.356628in}{0.578521in}}%
\pgfpathlineto{\pgfqpoint{0.354683in}{0.581418in}}%
\pgfpathclose%
\pgfusepath{fill}%
\end{pgfscope}%
\begin{pgfscope}%
\pgfpathrectangle{\pgfqpoint{0.100000in}{0.100000in}}{\pgfqpoint{3.420221in}{2.189500in}}%
\pgfusepath{clip}%
\pgfsetbuttcap%
\pgfsetmiterjoin%
\definecolor{currentfill}{rgb}{0.000000,0.768627,0.615686}%
\pgfsetfillcolor{currentfill}%
\pgfsetlinewidth{0.000000pt}%
\definecolor{currentstroke}{rgb}{0.000000,0.000000,0.000000}%
\pgfsetstrokecolor{currentstroke}%
\pgfsetstrokeopacity{0.000000}%
\pgfsetdash{}{0pt}%
\pgfpathmoveto{\pgfqpoint{0.461014in}{0.458739in}}%
\pgfpathlineto{\pgfqpoint{0.460464in}{0.460368in}}%
\pgfpathlineto{\pgfqpoint{0.464410in}{0.460254in}}%
\pgfpathlineto{\pgfqpoint{0.465597in}{0.458173in}}%
\pgfpathlineto{\pgfqpoint{0.464746in}{0.457287in}}%
\pgfpathclose%
\pgfusepath{fill}%
\end{pgfscope}%
\begin{pgfscope}%
\pgfpathrectangle{\pgfqpoint{0.100000in}{0.100000in}}{\pgfqpoint{3.420221in}{2.189500in}}%
\pgfusepath{clip}%
\pgfsetbuttcap%
\pgfsetmiterjoin%
\definecolor{currentfill}{rgb}{0.000000,0.768627,0.615686}%
\pgfsetfillcolor{currentfill}%
\pgfsetlinewidth{0.000000pt}%
\definecolor{currentstroke}{rgb}{0.000000,0.000000,0.000000}%
\pgfsetstrokecolor{currentstroke}%
\pgfsetstrokeopacity{0.000000}%
\pgfsetdash{}{0pt}%
\pgfpathmoveto{\pgfqpoint{0.331550in}{0.636674in}}%
\pgfpathlineto{\pgfqpoint{0.331097in}{0.639550in}}%
\pgfpathlineto{\pgfqpoint{0.331864in}{0.640654in}}%
\pgfpathlineto{\pgfqpoint{0.333967in}{0.640090in}}%
\pgfpathlineto{\pgfqpoint{0.333645in}{0.638636in}}%
\pgfpathclose%
\pgfusepath{fill}%
\end{pgfscope}%
\begin{pgfscope}%
\pgfpathrectangle{\pgfqpoint{0.100000in}{0.100000in}}{\pgfqpoint{3.420221in}{2.189500in}}%
\pgfusepath{clip}%
\pgfsetbuttcap%
\pgfsetmiterjoin%
\definecolor{currentfill}{rgb}{0.000000,0.768627,0.615686}%
\pgfsetfillcolor{currentfill}%
\pgfsetlinewidth{0.000000pt}%
\definecolor{currentstroke}{rgb}{0.000000,0.000000,0.000000}%
\pgfsetstrokecolor{currentstroke}%
\pgfsetstrokeopacity{0.000000}%
\pgfsetdash{}{0pt}%
\pgfpathmoveto{\pgfqpoint{0.485585in}{0.442921in}}%
\pgfpathlineto{\pgfqpoint{0.486410in}{0.444829in}}%
\pgfpathlineto{\pgfqpoint{0.490009in}{0.444005in}}%
\pgfpathlineto{\pgfqpoint{0.489107in}{0.442096in}}%
\pgfpathclose%
\pgfusepath{fill}%
\end{pgfscope}%
\begin{pgfscope}%
\pgfpathrectangle{\pgfqpoint{0.100000in}{0.100000in}}{\pgfqpoint{3.420221in}{2.189500in}}%
\pgfusepath{clip}%
\pgfsetbuttcap%
\pgfsetmiterjoin%
\definecolor{currentfill}{rgb}{0.000000,0.768627,0.615686}%
\pgfsetfillcolor{currentfill}%
\pgfsetlinewidth{0.000000pt}%
\definecolor{currentstroke}{rgb}{0.000000,0.000000,0.000000}%
\pgfsetstrokecolor{currentstroke}%
\pgfsetstrokeopacity{0.000000}%
\pgfsetdash{}{0pt}%
\pgfpathmoveto{\pgfqpoint{0.336447in}{0.650347in}}%
\pgfpathlineto{\pgfqpoint{0.336235in}{0.652522in}}%
\pgfpathlineto{\pgfqpoint{0.332822in}{0.655944in}}%
\pgfpathlineto{\pgfqpoint{0.335399in}{0.656565in}}%
\pgfpathlineto{\pgfqpoint{0.333705in}{0.658594in}}%
\pgfpathlineto{\pgfqpoint{0.334719in}{0.659757in}}%
\pgfpathlineto{\pgfqpoint{0.336980in}{0.660135in}}%
\pgfpathlineto{\pgfqpoint{0.338526in}{0.656567in}}%
\pgfpathlineto{\pgfqpoint{0.340511in}{0.653366in}}%
\pgfpathlineto{\pgfqpoint{0.340745in}{0.650563in}}%
\pgfpathlineto{\pgfqpoint{0.339048in}{0.649614in}}%
\pgfpathclose%
\pgfusepath{fill}%
\end{pgfscope}%
\begin{pgfscope}%
\pgfpathrectangle{\pgfqpoint{0.100000in}{0.100000in}}{\pgfqpoint{3.420221in}{2.189500in}}%
\pgfusepath{clip}%
\pgfsetbuttcap%
\pgfsetmiterjoin%
\definecolor{currentfill}{rgb}{0.000000,0.768627,0.615686}%
\pgfsetfillcolor{currentfill}%
\pgfsetlinewidth{0.000000pt}%
\definecolor{currentstroke}{rgb}{0.000000,0.000000,0.000000}%
\pgfsetstrokecolor{currentstroke}%
\pgfsetstrokeopacity{0.000000}%
\pgfsetdash{}{0pt}%
\pgfpathmoveto{\pgfqpoint{0.532731in}{0.419363in}}%
\pgfpathlineto{\pgfqpoint{0.532199in}{0.422063in}}%
\pgfpathlineto{\pgfqpoint{0.533709in}{0.422474in}}%
\pgfpathlineto{\pgfqpoint{0.537257in}{0.421345in}}%
\pgfpathlineto{\pgfqpoint{0.538964in}{0.419914in}}%
\pgfpathlineto{\pgfqpoint{0.540929in}{0.419565in}}%
\pgfpathlineto{\pgfqpoint{0.542533in}{0.417956in}}%
\pgfpathlineto{\pgfqpoint{0.544713in}{0.419663in}}%
\pgfpathlineto{\pgfqpoint{0.548209in}{0.421080in}}%
\pgfpathlineto{\pgfqpoint{0.551105in}{0.418231in}}%
\pgfpathlineto{\pgfqpoint{0.549434in}{0.424381in}}%
\pgfpathlineto{\pgfqpoint{0.551683in}{0.425419in}}%
\pgfpathlineto{\pgfqpoint{0.554179in}{0.424884in}}%
\pgfpathlineto{\pgfqpoint{0.558208in}{0.422696in}}%
\pgfpathlineto{\pgfqpoint{0.557281in}{0.420963in}}%
\pgfpathlineto{\pgfqpoint{0.559284in}{0.418675in}}%
\pgfpathlineto{\pgfqpoint{0.561363in}{0.419241in}}%
\pgfpathlineto{\pgfqpoint{0.560656in}{0.416389in}}%
\pgfpathlineto{\pgfqpoint{0.557246in}{0.416078in}}%
\pgfpathlineto{\pgfqpoint{0.555860in}{0.412465in}}%
\pgfpathlineto{\pgfqpoint{0.550059in}{0.414544in}}%
\pgfpathlineto{\pgfqpoint{0.547932in}{0.414751in}}%
\pgfpathlineto{\pgfqpoint{0.546537in}{0.412914in}}%
\pgfpathlineto{\pgfqpoint{0.544892in}{0.415629in}}%
\pgfpathlineto{\pgfqpoint{0.543283in}{0.416426in}}%
\pgfpathlineto{\pgfqpoint{0.538587in}{0.417247in}}%
\pgfpathlineto{\pgfqpoint{0.536474in}{0.418402in}}%
\pgfpathclose%
\pgfusepath{fill}%
\end{pgfscope}%
\begin{pgfscope}%
\pgfpathrectangle{\pgfqpoint{0.100000in}{0.100000in}}{\pgfqpoint{3.420221in}{2.189500in}}%
\pgfusepath{clip}%
\pgfsetbuttcap%
\pgfsetmiterjoin%
\definecolor{currentfill}{rgb}{0.000000,0.768627,0.615686}%
\pgfsetfillcolor{currentfill}%
\pgfsetlinewidth{0.000000pt}%
\definecolor{currentstroke}{rgb}{0.000000,0.000000,0.000000}%
\pgfsetstrokecolor{currentstroke}%
\pgfsetstrokeopacity{0.000000}%
\pgfsetdash{}{0pt}%
\pgfpathmoveto{\pgfqpoint{0.501378in}{0.434568in}}%
\pgfpathlineto{\pgfqpoint{0.500949in}{0.437148in}}%
\pgfpathlineto{\pgfqpoint{0.503043in}{0.437407in}}%
\pgfpathlineto{\pgfqpoint{0.503096in}{0.435312in}}%
\pgfpathclose%
\pgfusepath{fill}%
\end{pgfscope}%
\begin{pgfscope}%
\pgfpathrectangle{\pgfqpoint{0.100000in}{0.100000in}}{\pgfqpoint{3.420221in}{2.189500in}}%
\pgfusepath{clip}%
\pgfsetbuttcap%
\pgfsetmiterjoin%
\definecolor{currentfill}{rgb}{0.000000,0.768627,0.615686}%
\pgfsetfillcolor{currentfill}%
\pgfsetlinewidth{0.000000pt}%
\definecolor{currentstroke}{rgb}{0.000000,0.000000,0.000000}%
\pgfsetstrokecolor{currentstroke}%
\pgfsetstrokeopacity{0.000000}%
\pgfsetdash{}{0pt}%
\pgfpathmoveto{\pgfqpoint{0.509724in}{0.427876in}}%
\pgfpathlineto{\pgfqpoint{0.510240in}{0.428795in}}%
\pgfpathlineto{\pgfqpoint{0.515557in}{0.428946in}}%
\pgfpathlineto{\pgfqpoint{0.518709in}{0.431005in}}%
\pgfpathlineto{\pgfqpoint{0.522752in}{0.430757in}}%
\pgfpathlineto{\pgfqpoint{0.525265in}{0.427568in}}%
\pgfpathlineto{\pgfqpoint{0.527563in}{0.430580in}}%
\pgfpathlineto{\pgfqpoint{0.530062in}{0.431412in}}%
\pgfpathlineto{\pgfqpoint{0.532836in}{0.430823in}}%
\pgfpathlineto{\pgfqpoint{0.534944in}{0.429349in}}%
\pgfpathlineto{\pgfqpoint{0.535675in}{0.426929in}}%
\pgfpathlineto{\pgfqpoint{0.535247in}{0.425372in}}%
\pgfpathlineto{\pgfqpoint{0.532979in}{0.424115in}}%
\pgfpathlineto{\pgfqpoint{0.525369in}{0.426335in}}%
\pgfpathlineto{\pgfqpoint{0.520417in}{0.424910in}}%
\pgfpathlineto{\pgfqpoint{0.518529in}{0.425770in}}%
\pgfpathclose%
\pgfusepath{fill}%
\end{pgfscope}%
\begin{pgfscope}%
\pgfpathrectangle{\pgfqpoint{0.100000in}{0.100000in}}{\pgfqpoint{3.420221in}{2.189500in}}%
\pgfusepath{clip}%
\pgfsetbuttcap%
\pgfsetmiterjoin%
\definecolor{currentfill}{rgb}{0.000000,0.768627,0.615686}%
\pgfsetfillcolor{currentfill}%
\pgfsetlinewidth{0.000000pt}%
\definecolor{currentstroke}{rgb}{0.000000,0.000000,0.000000}%
\pgfsetstrokecolor{currentstroke}%
\pgfsetstrokeopacity{0.000000}%
\pgfsetdash{}{0pt}%
\pgfpathmoveto{\pgfqpoint{0.448005in}{0.465447in}}%
\pgfpathlineto{\pgfqpoint{0.446054in}{0.467532in}}%
\pgfpathlineto{\pgfqpoint{0.444896in}{0.469978in}}%
\pgfpathlineto{\pgfqpoint{0.443031in}{0.471818in}}%
\pgfpathlineto{\pgfqpoint{0.443477in}{0.474357in}}%
\pgfpathlineto{\pgfqpoint{0.447411in}{0.470547in}}%
\pgfpathlineto{\pgfqpoint{0.449728in}{0.465791in}}%
\pgfpathclose%
\pgfusepath{fill}%
\end{pgfscope}%
\begin{pgfscope}%
\pgfpathrectangle{\pgfqpoint{0.100000in}{0.100000in}}{\pgfqpoint{3.420221in}{2.189500in}}%
\pgfusepath{clip}%
\pgfsetbuttcap%
\pgfsetmiterjoin%
\definecolor{currentfill}{rgb}{0.000000,0.768627,0.615686}%
\pgfsetfillcolor{currentfill}%
\pgfsetlinewidth{0.000000pt}%
\definecolor{currentstroke}{rgb}{0.000000,0.000000,0.000000}%
\pgfsetstrokecolor{currentstroke}%
\pgfsetstrokeopacity{0.000000}%
\pgfsetdash{}{0pt}%
\pgfpathmoveto{\pgfqpoint{0.437467in}{0.478228in}}%
\pgfpathlineto{\pgfqpoint{0.436171in}{0.480548in}}%
\pgfpathlineto{\pgfqpoint{0.433557in}{0.481381in}}%
\pgfpathlineto{\pgfqpoint{0.432733in}{0.485102in}}%
\pgfpathlineto{\pgfqpoint{0.435631in}{0.483460in}}%
\pgfpathlineto{\pgfqpoint{0.436831in}{0.481294in}}%
\pgfpathlineto{\pgfqpoint{0.438838in}{0.481755in}}%
\pgfpathlineto{\pgfqpoint{0.442722in}{0.481376in}}%
\pgfpathlineto{\pgfqpoint{0.444820in}{0.482644in}}%
\pgfpathlineto{\pgfqpoint{0.446798in}{0.481820in}}%
\pgfpathlineto{\pgfqpoint{0.447832in}{0.479519in}}%
\pgfpathlineto{\pgfqpoint{0.447470in}{0.478107in}}%
\pgfpathlineto{\pgfqpoint{0.445303in}{0.476983in}}%
\pgfpathlineto{\pgfqpoint{0.443446in}{0.478577in}}%
\pgfpathlineto{\pgfqpoint{0.442460in}{0.475851in}}%
\pgfpathlineto{\pgfqpoint{0.441081in}{0.476344in}}%
\pgfpathlineto{\pgfqpoint{0.438971in}{0.479035in}}%
\pgfpathclose%
\pgfusepath{fill}%
\end{pgfscope}%
\begin{pgfscope}%
\pgfpathrectangle{\pgfqpoint{0.100000in}{0.100000in}}{\pgfqpoint{3.420221in}{2.189500in}}%
\pgfusepath{clip}%
\pgfsetbuttcap%
\pgfsetmiterjoin%
\definecolor{currentfill}{rgb}{0.000000,0.560784,0.719608}%
\pgfsetfillcolor{currentfill}%
\pgfsetlinewidth{0.000000pt}%
\definecolor{currentstroke}{rgb}{0.000000,0.000000,0.000000}%
\pgfsetstrokecolor{currentstroke}%
\pgfsetstrokeopacity{0.000000}%
\pgfsetdash{}{0pt}%
\pgfpathmoveto{\pgfqpoint{2.120392in}{0.859963in}}%
\pgfpathlineto{\pgfqpoint{2.113822in}{0.858892in}}%
\pgfpathlineto{\pgfqpoint{2.097317in}{0.859103in}}%
\pgfpathlineto{\pgfqpoint{2.097337in}{0.861276in}}%
\pgfpathlineto{\pgfqpoint{2.082279in}{0.861488in}}%
\pgfpathlineto{\pgfqpoint{2.075179in}{0.870823in}}%
\pgfpathlineto{\pgfqpoint{2.066863in}{0.875444in}}%
\pgfpathlineto{\pgfqpoint{2.067236in}{0.881542in}}%
\pgfpathlineto{\pgfqpoint{2.065933in}{0.886473in}}%
\pgfpathlineto{\pgfqpoint{2.061239in}{0.891651in}}%
\pgfpathlineto{\pgfqpoint{2.058990in}{0.902728in}}%
\pgfpathlineto{\pgfqpoint{2.058030in}{0.914570in}}%
\pgfpathlineto{\pgfqpoint{2.049138in}{0.914570in}}%
\pgfpathlineto{\pgfqpoint{2.048359in}{0.926817in}}%
\pgfpathlineto{\pgfqpoint{2.068176in}{0.926610in}}%
\pgfpathlineto{\pgfqpoint{2.068199in}{0.950323in}}%
\pgfpathlineto{\pgfqpoint{2.071881in}{0.953906in}}%
\pgfpathlineto{\pgfqpoint{2.081721in}{0.956487in}}%
\pgfpathlineto{\pgfqpoint{2.101228in}{0.956655in}}%
\pgfpathlineto{\pgfqpoint{2.107760in}{0.959013in}}%
\pgfpathlineto{\pgfqpoint{2.127413in}{0.959348in}}%
\pgfpathlineto{\pgfqpoint{2.126981in}{0.949190in}}%
\pgfpathlineto{\pgfqpoint{2.133439in}{0.949180in}}%
\pgfpathlineto{\pgfqpoint{2.137688in}{0.945920in}}%
\pgfpathlineto{\pgfqpoint{2.138755in}{0.939421in}}%
\pgfpathlineto{\pgfqpoint{2.145518in}{0.937290in}}%
\pgfpathlineto{\pgfqpoint{2.146643in}{0.932899in}}%
\pgfpathlineto{\pgfqpoint{2.146551in}{0.912997in}}%
\pgfpathlineto{\pgfqpoint{2.133608in}{0.913152in}}%
\pgfpathlineto{\pgfqpoint{2.133286in}{0.890644in}}%
\pgfpathlineto{\pgfqpoint{2.130230in}{0.886062in}}%
\pgfpathlineto{\pgfqpoint{2.120642in}{0.884382in}}%
\pgfpathclose%
\pgfusepath{fill}%
\end{pgfscope}%
\begin{pgfscope}%
\pgfpathrectangle{\pgfqpoint{0.100000in}{0.100000in}}{\pgfqpoint{3.420221in}{2.189500in}}%
\pgfusepath{clip}%
\pgfsetbuttcap%
\pgfsetmiterjoin%
\definecolor{currentfill}{rgb}{0.000000,0.592157,0.703922}%
\pgfsetfillcolor{currentfill}%
\pgfsetlinewidth{0.000000pt}%
\definecolor{currentstroke}{rgb}{0.000000,0.000000,0.000000}%
\pgfsetstrokecolor{currentstroke}%
\pgfsetstrokeopacity{0.000000}%
\pgfsetdash{}{0pt}%
\pgfpathmoveto{\pgfqpoint{0.664497in}{1.899801in}}%
\pgfpathlineto{\pgfqpoint{0.627712in}{1.910636in}}%
\pgfpathlineto{\pgfqpoint{0.593489in}{1.921058in}}%
\pgfpathlineto{\pgfqpoint{0.596123in}{1.925186in}}%
\pgfpathlineto{\pgfqpoint{0.595144in}{1.934937in}}%
\pgfpathlineto{\pgfqpoint{0.602386in}{1.937372in}}%
\pgfpathlineto{\pgfqpoint{0.602060in}{1.946847in}}%
\pgfpathlineto{\pgfqpoint{0.606772in}{1.950475in}}%
\pgfpathlineto{\pgfqpoint{0.608445in}{1.960387in}}%
\pgfpathlineto{\pgfqpoint{0.601240in}{1.973430in}}%
\pgfpathlineto{\pgfqpoint{0.603289in}{1.982947in}}%
\pgfpathlineto{\pgfqpoint{0.607038in}{1.985286in}}%
\pgfpathlineto{\pgfqpoint{0.613712in}{1.983318in}}%
\pgfpathlineto{\pgfqpoint{0.624607in}{1.982592in}}%
\pgfpathlineto{\pgfqpoint{0.622004in}{1.987791in}}%
\pgfpathlineto{\pgfqpoint{0.627515in}{2.006723in}}%
\pgfpathlineto{\pgfqpoint{0.631964in}{2.005400in}}%
\pgfpathlineto{\pgfqpoint{0.686950in}{1.988930in}}%
\pgfpathlineto{\pgfqpoint{0.715303in}{1.980922in}}%
\pgfpathlineto{\pgfqpoint{0.711034in}{1.966073in}}%
\pgfpathlineto{\pgfqpoint{0.703905in}{1.966239in}}%
\pgfpathlineto{\pgfqpoint{0.691429in}{1.962991in}}%
\pgfpathlineto{\pgfqpoint{0.681090in}{1.963852in}}%
\pgfpathlineto{\pgfqpoint{0.675859in}{1.961915in}}%
\pgfpathlineto{\pgfqpoint{0.678585in}{1.956413in}}%
\pgfpathlineto{\pgfqpoint{0.678818in}{1.948599in}}%
\pgfpathlineto{\pgfqpoint{0.671538in}{1.948228in}}%
\pgfpathlineto{\pgfqpoint{0.664144in}{1.933697in}}%
\pgfpathlineto{\pgfqpoint{0.666648in}{1.925979in}}%
\pgfpathlineto{\pgfqpoint{0.661170in}{1.910916in}}%
\pgfpathlineto{\pgfqpoint{0.661149in}{1.904173in}}%
\pgfpathclose%
\pgfusepath{fill}%
\end{pgfscope}%
\begin{pgfscope}%
\pgfpathrectangle{\pgfqpoint{0.100000in}{0.100000in}}{\pgfqpoint{3.420221in}{2.189500in}}%
\pgfusepath{clip}%
\pgfsetbuttcap%
\pgfsetmiterjoin%
\definecolor{currentfill}{rgb}{0.000000,0.415686,0.792157}%
\pgfsetfillcolor{currentfill}%
\pgfsetlinewidth{0.000000pt}%
\definecolor{currentstroke}{rgb}{0.000000,0.000000,0.000000}%
\pgfsetstrokecolor{currentstroke}%
\pgfsetstrokeopacity{0.000000}%
\pgfsetdash{}{0pt}%
\pgfpathmoveto{\pgfqpoint{1.662404in}{0.999165in}}%
\pgfpathlineto{\pgfqpoint{1.628880in}{1.001237in}}%
\pgfpathlineto{\pgfqpoint{1.596032in}{1.003610in}}%
\pgfpathlineto{\pgfqpoint{1.598430in}{1.036334in}}%
\pgfpathlineto{\pgfqpoint{1.600767in}{1.068907in}}%
\pgfpathlineto{\pgfqpoint{1.666321in}{1.064596in}}%
\pgfpathlineto{\pgfqpoint{1.699001in}{1.062637in}}%
\pgfpathlineto{\pgfqpoint{1.698280in}{1.049547in}}%
\pgfpathlineto{\pgfqpoint{1.697200in}{1.029928in}}%
\pgfpathlineto{\pgfqpoint{1.664323in}{1.031851in}}%
\pgfpathclose%
\pgfusepath{fill}%
\end{pgfscope}%
\begin{pgfscope}%
\pgfpathrectangle{\pgfqpoint{0.100000in}{0.100000in}}{\pgfqpoint{3.420221in}{2.189500in}}%
\pgfusepath{clip}%
\pgfsetbuttcap%
\pgfsetmiterjoin%
\definecolor{currentfill}{rgb}{0.000000,0.564706,0.717647}%
\pgfsetfillcolor{currentfill}%
\pgfsetlinewidth{0.000000pt}%
\definecolor{currentstroke}{rgb}{0.000000,0.000000,0.000000}%
\pgfsetstrokecolor{currentstroke}%
\pgfsetstrokeopacity{0.000000}%
\pgfsetdash{}{0pt}%
\pgfpathmoveto{\pgfqpoint{1.791205in}{0.708951in}}%
\pgfpathlineto{\pgfqpoint{1.771858in}{0.695623in}}%
\pgfpathlineto{\pgfqpoint{1.773900in}{0.684815in}}%
\pgfpathlineto{\pgfqpoint{1.779231in}{0.680125in}}%
\pgfpathlineto{\pgfqpoint{1.778527in}{0.671968in}}%
\pgfpathlineto{\pgfqpoint{1.744882in}{0.673335in}}%
\pgfpathlineto{\pgfqpoint{1.736668in}{0.675187in}}%
\pgfpathlineto{\pgfqpoint{1.738547in}{0.714445in}}%
\pgfpathlineto{\pgfqpoint{1.729253in}{0.716524in}}%
\pgfpathlineto{\pgfqpoint{1.723953in}{0.711250in}}%
\pgfpathlineto{\pgfqpoint{1.716028in}{0.716581in}}%
\pgfpathlineto{\pgfqpoint{1.704148in}{0.716603in}}%
\pgfpathlineto{\pgfqpoint{1.698297in}{0.724109in}}%
\pgfpathlineto{\pgfqpoint{1.700850in}{0.763241in}}%
\pgfpathlineto{\pgfqpoint{1.751262in}{0.760536in}}%
\pgfpathlineto{\pgfqpoint{1.766463in}{0.731376in}}%
\pgfpathlineto{\pgfqpoint{1.777808in}{0.732674in}}%
\pgfpathclose%
\pgfusepath{fill}%
\end{pgfscope}%
\begin{pgfscope}%
\pgfpathrectangle{\pgfqpoint{0.100000in}{0.100000in}}{\pgfqpoint{3.420221in}{2.189500in}}%
\pgfusepath{clip}%
\pgfsetbuttcap%
\pgfsetmiterjoin%
\definecolor{currentfill}{rgb}{0.000000,0.745098,0.627451}%
\pgfsetfillcolor{currentfill}%
\pgfsetlinewidth{0.000000pt}%
\definecolor{currentstroke}{rgb}{0.000000,0.000000,0.000000}%
\pgfsetstrokecolor{currentstroke}%
\pgfsetstrokeopacity{0.000000}%
\pgfsetdash{}{0pt}%
\pgfpathmoveto{\pgfqpoint{2.764031in}{0.621193in}}%
\pgfpathlineto{\pgfqpoint{2.754440in}{0.625302in}}%
\pgfpathlineto{\pgfqpoint{2.747346in}{0.635903in}}%
\pgfpathlineto{\pgfqpoint{2.737698in}{0.641472in}}%
\pgfpathlineto{\pgfqpoint{2.727115in}{0.644929in}}%
\pgfpathlineto{\pgfqpoint{2.722442in}{0.648313in}}%
\pgfpathlineto{\pgfqpoint{2.725934in}{0.660710in}}%
\pgfpathlineto{\pgfqpoint{2.733268in}{0.667269in}}%
\pgfpathlineto{\pgfqpoint{2.731840in}{0.670565in}}%
\pgfpathlineto{\pgfqpoint{2.736761in}{0.676480in}}%
\pgfpathlineto{\pgfqpoint{2.735109in}{0.681941in}}%
\pgfpathlineto{\pgfqpoint{2.742990in}{0.688544in}}%
\pgfpathlineto{\pgfqpoint{2.741849in}{0.693692in}}%
\pgfpathlineto{\pgfqpoint{2.808819in}{0.697974in}}%
\pgfpathlineto{\pgfqpoint{2.816932in}{0.698517in}}%
\pgfpathlineto{\pgfqpoint{2.821673in}{0.664921in}}%
\pgfpathlineto{\pgfqpoint{2.816146in}{0.651817in}}%
\pgfpathlineto{\pgfqpoint{2.815772in}{0.643097in}}%
\pgfpathlineto{\pgfqpoint{2.812968in}{0.640082in}}%
\pgfpathlineto{\pgfqpoint{2.806718in}{0.641452in}}%
\pgfpathlineto{\pgfqpoint{2.802590in}{0.646729in}}%
\pgfpathlineto{\pgfqpoint{2.797672in}{0.644415in}}%
\pgfpathlineto{\pgfqpoint{2.796564in}{0.637847in}}%
\pgfpathlineto{\pgfqpoint{2.768815in}{0.633824in}}%
\pgfpathlineto{\pgfqpoint{2.766774in}{0.621574in}}%
\pgfpathclose%
\pgfusepath{fill}%
\end{pgfscope}%
\begin{pgfscope}%
\pgfpathrectangle{\pgfqpoint{0.100000in}{0.100000in}}{\pgfqpoint{3.420221in}{2.189500in}}%
\pgfusepath{clip}%
\pgfsetbuttcap%
\pgfsetmiterjoin%
\definecolor{currentfill}{rgb}{0.000000,0.568627,0.715686}%
\pgfsetfillcolor{currentfill}%
\pgfsetlinewidth{0.000000pt}%
\definecolor{currentstroke}{rgb}{0.000000,0.000000,0.000000}%
\pgfsetstrokecolor{currentstroke}%
\pgfsetstrokeopacity{0.000000}%
\pgfsetdash{}{0pt}%
\pgfpathmoveto{\pgfqpoint{2.415601in}{1.121818in}}%
\pgfpathlineto{\pgfqpoint{2.421473in}{1.106757in}}%
\pgfpathlineto{\pgfqpoint{2.420478in}{1.103406in}}%
\pgfpathlineto{\pgfqpoint{2.394245in}{1.101756in}}%
\pgfpathlineto{\pgfqpoint{2.393118in}{1.120413in}}%
\pgfpathclose%
\pgfusepath{fill}%
\end{pgfscope}%
\begin{pgfscope}%
\pgfpathrectangle{\pgfqpoint{0.100000in}{0.100000in}}{\pgfqpoint{3.420221in}{2.189500in}}%
\pgfusepath{clip}%
\pgfsetbuttcap%
\pgfsetmiterjoin%
\definecolor{currentfill}{rgb}{0.000000,0.470588,0.764706}%
\pgfsetfillcolor{currentfill}%
\pgfsetlinewidth{0.000000pt}%
\definecolor{currentstroke}{rgb}{0.000000,0.000000,0.000000}%
\pgfsetstrokecolor{currentstroke}%
\pgfsetstrokeopacity{0.000000}%
\pgfsetdash{}{0pt}%
\pgfpathmoveto{\pgfqpoint{2.103417in}{1.833445in}}%
\pgfpathlineto{\pgfqpoint{2.103890in}{1.813724in}}%
\pgfpathlineto{\pgfqpoint{2.084131in}{1.813030in}}%
\pgfpathlineto{\pgfqpoint{2.084080in}{1.819988in}}%
\pgfpathlineto{\pgfqpoint{2.064218in}{1.819460in}}%
\pgfpathlineto{\pgfqpoint{2.064332in}{1.812925in}}%
\pgfpathlineto{\pgfqpoint{2.034822in}{1.812668in}}%
\pgfpathlineto{\pgfqpoint{2.036006in}{1.823123in}}%
\pgfpathlineto{\pgfqpoint{2.032022in}{1.825885in}}%
\pgfpathlineto{\pgfqpoint{2.026391in}{1.823622in}}%
\pgfpathlineto{\pgfqpoint{2.013620in}{1.830480in}}%
\pgfpathlineto{\pgfqpoint{2.013137in}{1.861460in}}%
\pgfpathlineto{\pgfqpoint{2.019715in}{1.861437in}}%
\pgfpathlineto{\pgfqpoint{2.019014in}{1.907352in}}%
\pgfpathlineto{\pgfqpoint{2.032024in}{1.907545in}}%
\pgfpathlineto{\pgfqpoint{2.031703in}{1.940437in}}%
\pgfpathlineto{\pgfqpoint{2.064392in}{1.940807in}}%
\pgfpathlineto{\pgfqpoint{2.064362in}{1.944793in}}%
\pgfpathlineto{\pgfqpoint{2.099691in}{1.944868in}}%
\pgfpathlineto{\pgfqpoint{2.100981in}{1.931873in}}%
\pgfpathlineto{\pgfqpoint{2.101084in}{1.907111in}}%
\pgfpathlineto{\pgfqpoint{2.102378in}{1.879342in}}%
\pgfpathlineto{\pgfqpoint{2.102592in}{1.849385in}}%
\pgfpathclose%
\pgfusepath{fill}%
\end{pgfscope}%
\begin{pgfscope}%
\pgfpathrectangle{\pgfqpoint{0.100000in}{0.100000in}}{\pgfqpoint{3.420221in}{2.189500in}}%
\pgfusepath{clip}%
\pgfsetbuttcap%
\pgfsetmiterjoin%
\definecolor{currentfill}{rgb}{0.000000,0.450980,0.774510}%
\pgfsetfillcolor{currentfill}%
\pgfsetlinewidth{0.000000pt}%
\definecolor{currentstroke}{rgb}{0.000000,0.000000,0.000000}%
\pgfsetstrokecolor{currentstroke}%
\pgfsetstrokeopacity{0.000000}%
\pgfsetdash{}{0pt}%
\pgfpathmoveto{\pgfqpoint{1.520434in}{1.749117in}}%
\pgfpathlineto{\pgfqpoint{1.468508in}{1.754814in}}%
\pgfpathlineto{\pgfqpoint{1.471439in}{1.781154in}}%
\pgfpathlineto{\pgfqpoint{1.474128in}{1.780854in}}%
\pgfpathlineto{\pgfqpoint{1.477993in}{1.813430in}}%
\pgfpathlineto{\pgfqpoint{1.480231in}{1.813181in}}%
\pgfpathlineto{\pgfqpoint{1.483103in}{1.839413in}}%
\pgfpathlineto{\pgfqpoint{1.485889in}{1.839116in}}%
\pgfpathlineto{\pgfqpoint{1.505134in}{1.837027in}}%
\pgfpathlineto{\pgfqpoint{1.504467in}{1.830892in}}%
\pgfpathlineto{\pgfqpoint{1.511034in}{1.830234in}}%
\pgfpathlineto{\pgfqpoint{1.510345in}{1.823880in}}%
\pgfpathlineto{\pgfqpoint{1.523703in}{1.822411in}}%
\pgfpathlineto{\pgfqpoint{1.528185in}{1.820168in}}%
\pgfpathlineto{\pgfqpoint{1.583050in}{1.814998in}}%
\pgfpathlineto{\pgfqpoint{1.585836in}{1.814741in}}%
\pgfpathlineto{\pgfqpoint{1.580186in}{1.759793in}}%
\pgfpathlineto{\pgfqpoint{1.522987in}{1.765179in}}%
\pgfpathclose%
\pgfusepath{fill}%
\end{pgfscope}%
\begin{pgfscope}%
\pgfpathrectangle{\pgfqpoint{0.100000in}{0.100000in}}{\pgfqpoint{3.420221in}{2.189500in}}%
\pgfusepath{clip}%
\pgfsetbuttcap%
\pgfsetmiterjoin%
\definecolor{currentfill}{rgb}{0.000000,0.619608,0.690196}%
\pgfsetfillcolor{currentfill}%
\pgfsetlinewidth{0.000000pt}%
\definecolor{currentstroke}{rgb}{0.000000,0.000000,0.000000}%
\pgfsetstrokecolor{currentstroke}%
\pgfsetstrokeopacity{0.000000}%
\pgfsetdash{}{0pt}%
\pgfpathmoveto{\pgfqpoint{2.487428in}{1.660113in}}%
\pgfpathlineto{\pgfqpoint{2.466224in}{1.658370in}}%
\pgfpathlineto{\pgfqpoint{2.463445in}{1.669438in}}%
\pgfpathlineto{\pgfqpoint{2.460193in}{1.674724in}}%
\pgfpathlineto{\pgfqpoint{2.464561in}{1.681608in}}%
\pgfpathlineto{\pgfqpoint{2.471229in}{1.698527in}}%
\pgfpathlineto{\pgfqpoint{2.471779in}{1.711703in}}%
\pgfpathlineto{\pgfqpoint{2.494079in}{1.713495in}}%
\pgfpathlineto{\pgfqpoint{2.496525in}{1.687374in}}%
\pgfpathlineto{\pgfqpoint{2.484632in}{1.686465in}}%
\pgfpathclose%
\pgfusepath{fill}%
\end{pgfscope}%
\begin{pgfscope}%
\pgfpathrectangle{\pgfqpoint{0.100000in}{0.100000in}}{\pgfqpoint{3.420221in}{2.189500in}}%
\pgfusepath{clip}%
\pgfsetbuttcap%
\pgfsetmiterjoin%
\definecolor{currentfill}{rgb}{0.000000,0.686275,0.656863}%
\pgfsetfillcolor{currentfill}%
\pgfsetlinewidth{0.000000pt}%
\definecolor{currentstroke}{rgb}{0.000000,0.000000,0.000000}%
\pgfsetstrokecolor{currentstroke}%
\pgfsetstrokeopacity{0.000000}%
\pgfsetdash{}{0pt}%
\pgfpathmoveto{\pgfqpoint{1.137086in}{1.381237in}}%
\pgfpathlineto{\pgfqpoint{1.135847in}{1.373114in}}%
\pgfpathlineto{\pgfqpoint{1.131822in}{1.368310in}}%
\pgfpathlineto{\pgfqpoint{1.132517in}{1.364345in}}%
\pgfpathlineto{\pgfqpoint{1.125647in}{1.354636in}}%
\pgfpathlineto{\pgfqpoint{1.121754in}{1.338399in}}%
\pgfpathlineto{\pgfqpoint{1.124740in}{1.330810in}}%
\pgfpathlineto{\pgfqpoint{1.122918in}{1.328024in}}%
\pgfpathlineto{\pgfqpoint{1.126850in}{1.313229in}}%
\pgfpathlineto{\pgfqpoint{1.125155in}{1.310182in}}%
\pgfpathlineto{\pgfqpoint{1.087507in}{1.316602in}}%
\pgfpathlineto{\pgfqpoint{1.051193in}{1.323059in}}%
\pgfpathlineto{\pgfqpoint{1.051332in}{1.323810in}}%
\pgfpathlineto{\pgfqpoint{1.058683in}{1.362229in}}%
\pgfpathlineto{\pgfqpoint{1.026979in}{1.368132in}}%
\pgfpathlineto{\pgfqpoint{1.017964in}{1.370775in}}%
\pgfpathlineto{\pgfqpoint{1.021947in}{1.392243in}}%
\pgfpathlineto{\pgfqpoint{1.027870in}{1.394529in}}%
\pgfpathlineto{\pgfqpoint{1.037871in}{1.392760in}}%
\pgfpathlineto{\pgfqpoint{1.040857in}{1.398092in}}%
\pgfpathlineto{\pgfqpoint{1.043765in}{1.415423in}}%
\pgfpathlineto{\pgfqpoint{1.052560in}{1.420586in}}%
\pgfpathlineto{\pgfqpoint{1.049714in}{1.423341in}}%
\pgfpathlineto{\pgfqpoint{1.094137in}{1.415334in}}%
\pgfpathlineto{\pgfqpoint{1.113304in}{1.411483in}}%
\pgfpathlineto{\pgfqpoint{1.149245in}{1.405348in}}%
\pgfpathlineto{\pgfqpoint{1.142202in}{1.401252in}}%
\pgfpathlineto{\pgfqpoint{1.141849in}{1.395819in}}%
\pgfpathlineto{\pgfqpoint{1.137594in}{1.389126in}}%
\pgfpathclose%
\pgfusepath{fill}%
\end{pgfscope}%
\begin{pgfscope}%
\pgfpathrectangle{\pgfqpoint{0.100000in}{0.100000in}}{\pgfqpoint{3.420221in}{2.189500in}}%
\pgfusepath{clip}%
\pgfsetbuttcap%
\pgfsetmiterjoin%
\definecolor{currentfill}{rgb}{0.000000,0.984314,0.507843}%
\pgfsetfillcolor{currentfill}%
\pgfsetlinewidth{0.000000pt}%
\definecolor{currentstroke}{rgb}{0.000000,0.000000,0.000000}%
\pgfsetstrokecolor{currentstroke}%
\pgfsetstrokeopacity{0.000000}%
\pgfsetdash{}{0pt}%
\pgfpathmoveto{\pgfqpoint{0.438361in}{1.747283in}}%
\pgfpathlineto{\pgfqpoint{0.433956in}{1.741230in}}%
\pgfpathlineto{\pgfqpoint{0.424521in}{1.738272in}}%
\pgfpathlineto{\pgfqpoint{0.424146in}{1.729287in}}%
\pgfpathlineto{\pgfqpoint{0.419134in}{1.723110in}}%
\pgfpathlineto{\pgfqpoint{0.420630in}{1.716310in}}%
\pgfpathlineto{\pgfqpoint{0.419641in}{1.708558in}}%
\pgfpathlineto{\pgfqpoint{0.416363in}{1.705641in}}%
\pgfpathlineto{\pgfqpoint{0.410510in}{1.707451in}}%
\pgfpathlineto{\pgfqpoint{0.412441in}{1.713400in}}%
\pgfpathlineto{\pgfqpoint{0.396730in}{1.718631in}}%
\pgfpathlineto{\pgfqpoint{0.398337in}{1.737662in}}%
\pgfpathlineto{\pgfqpoint{0.394109in}{1.744642in}}%
\pgfpathlineto{\pgfqpoint{0.400527in}{1.754573in}}%
\pgfpathlineto{\pgfqpoint{0.401538in}{1.759752in}}%
\pgfpathlineto{\pgfqpoint{0.396601in}{1.768967in}}%
\pgfpathlineto{\pgfqpoint{0.398019in}{1.774903in}}%
\pgfpathlineto{\pgfqpoint{0.397642in}{1.786238in}}%
\pgfpathlineto{\pgfqpoint{0.402222in}{1.797991in}}%
\pgfpathlineto{\pgfqpoint{0.405901in}{1.803339in}}%
\pgfpathlineto{\pgfqpoint{0.406618in}{1.809915in}}%
\pgfpathlineto{\pgfqpoint{0.403342in}{1.816679in}}%
\pgfpathlineto{\pgfqpoint{0.403788in}{1.825163in}}%
\pgfpathlineto{\pgfqpoint{0.410259in}{1.831955in}}%
\pgfpathlineto{\pgfqpoint{0.421872in}{1.828075in}}%
\pgfpathlineto{\pgfqpoint{0.425378in}{1.819401in}}%
\pgfpathlineto{\pgfqpoint{0.421315in}{1.805793in}}%
\pgfpathlineto{\pgfqpoint{0.429100in}{1.804386in}}%
\pgfpathlineto{\pgfqpoint{0.437881in}{1.808534in}}%
\pgfpathlineto{\pgfqpoint{0.446110in}{1.806553in}}%
\pgfpathlineto{\pgfqpoint{0.436517in}{1.797749in}}%
\pgfpathlineto{\pgfqpoint{0.430341in}{1.789335in}}%
\pgfpathlineto{\pgfqpoint{0.424079in}{1.790266in}}%
\pgfpathlineto{\pgfqpoint{0.419892in}{1.781620in}}%
\pgfpathlineto{\pgfqpoint{0.426886in}{1.778784in}}%
\pgfpathlineto{\pgfqpoint{0.429381in}{1.768800in}}%
\pgfpathlineto{\pgfqpoint{0.423690in}{1.763145in}}%
\pgfpathlineto{\pgfqpoint{0.422148in}{1.752150in}}%
\pgfpathclose%
\pgfusepath{fill}%
\end{pgfscope}%
\begin{pgfscope}%
\pgfpathrectangle{\pgfqpoint{0.100000in}{0.100000in}}{\pgfqpoint{3.420221in}{2.189500in}}%
\pgfusepath{clip}%
\pgfsetbuttcap%
\pgfsetmiterjoin%
\definecolor{currentfill}{rgb}{0.000000,0.549020,0.725490}%
\pgfsetfillcolor{currentfill}%
\pgfsetlinewidth{0.000000pt}%
\definecolor{currentstroke}{rgb}{0.000000,0.000000,0.000000}%
\pgfsetstrokecolor{currentstroke}%
\pgfsetstrokeopacity{0.000000}%
\pgfsetdash{}{0pt}%
\pgfpathmoveto{\pgfqpoint{2.583486in}{1.405784in}}%
\pgfpathlineto{\pgfqpoint{2.585651in}{1.383015in}}%
\pgfpathlineto{\pgfqpoint{2.563370in}{1.380525in}}%
\pgfpathlineto{\pgfqpoint{2.563991in}{1.364311in}}%
\pgfpathlineto{\pgfqpoint{2.559393in}{1.363745in}}%
\pgfpathlineto{\pgfqpoint{2.542491in}{1.361890in}}%
\pgfpathlineto{\pgfqpoint{2.542430in}{1.373883in}}%
\pgfpathlineto{\pgfqpoint{2.526098in}{1.372042in}}%
\pgfpathlineto{\pgfqpoint{2.524058in}{1.392572in}}%
\pgfpathlineto{\pgfqpoint{2.520639in}{1.424829in}}%
\pgfpathlineto{\pgfqpoint{2.533426in}{1.426220in}}%
\pgfpathlineto{\pgfqpoint{2.544153in}{1.427388in}}%
\pgfpathlineto{\pgfqpoint{2.544939in}{1.420922in}}%
\pgfpathlineto{\pgfqpoint{2.558849in}{1.422438in}}%
\pgfpathlineto{\pgfqpoint{2.559952in}{1.402931in}}%
\pgfpathclose%
\pgfusepath{fill}%
\end{pgfscope}%
\begin{pgfscope}%
\pgfpathrectangle{\pgfqpoint{0.100000in}{0.100000in}}{\pgfqpoint{3.420221in}{2.189500in}}%
\pgfusepath{clip}%
\pgfsetbuttcap%
\pgfsetmiterjoin%
\definecolor{currentfill}{rgb}{0.000000,0.584314,0.707843}%
\pgfsetfillcolor{currentfill}%
\pgfsetlinewidth{0.000000pt}%
\definecolor{currentstroke}{rgb}{0.000000,0.000000,0.000000}%
\pgfsetstrokecolor{currentstroke}%
\pgfsetstrokeopacity{0.000000}%
\pgfsetdash{}{0pt}%
\pgfpathmoveto{\pgfqpoint{2.668623in}{1.192429in}}%
\pgfpathlineto{\pgfqpoint{2.660150in}{1.183974in}}%
\pgfpathlineto{\pgfqpoint{2.656427in}{1.183734in}}%
\pgfpathlineto{\pgfqpoint{2.648455in}{1.188870in}}%
\pgfpathlineto{\pgfqpoint{2.641954in}{1.181008in}}%
\pgfpathlineto{\pgfqpoint{2.640541in}{1.174149in}}%
\pgfpathlineto{\pgfqpoint{2.626317in}{1.185821in}}%
\pgfpathlineto{\pgfqpoint{2.628273in}{1.197919in}}%
\pgfpathlineto{\pgfqpoint{2.633654in}{1.202565in}}%
\pgfpathlineto{\pgfqpoint{2.621176in}{1.218506in}}%
\pgfpathlineto{\pgfqpoint{2.634813in}{1.231263in}}%
\pgfpathlineto{\pgfqpoint{2.651165in}{1.226891in}}%
\pgfpathlineto{\pgfqpoint{2.660160in}{1.221842in}}%
\pgfpathlineto{\pgfqpoint{2.668925in}{1.220227in}}%
\pgfpathlineto{\pgfqpoint{2.663477in}{1.215339in}}%
\pgfpathlineto{\pgfqpoint{2.659711in}{1.205992in}}%
\pgfpathlineto{\pgfqpoint{2.661595in}{1.201267in}}%
\pgfpathclose%
\pgfusepath{fill}%
\end{pgfscope}%
\begin{pgfscope}%
\pgfpathrectangle{\pgfqpoint{0.100000in}{0.100000in}}{\pgfqpoint{3.420221in}{2.189500in}}%
\pgfusepath{clip}%
\pgfsetbuttcap%
\pgfsetmiterjoin%
\definecolor{currentfill}{rgb}{0.000000,0.764706,0.617647}%
\pgfsetfillcolor{currentfill}%
\pgfsetlinewidth{0.000000pt}%
\definecolor{currentstroke}{rgb}{0.000000,0.000000,0.000000}%
\pgfsetstrokecolor{currentstroke}%
\pgfsetstrokeopacity{0.000000}%
\pgfsetdash{}{0pt}%
\pgfpathmoveto{\pgfqpoint{2.073998in}{0.737919in}}%
\pgfpathlineto{\pgfqpoint{2.102077in}{0.738404in}}%
\pgfpathlineto{\pgfqpoint{2.102209in}{0.728531in}}%
\pgfpathlineto{\pgfqpoint{2.108811in}{0.728632in}}%
\pgfpathlineto{\pgfqpoint{2.109152in}{0.715330in}}%
\pgfpathlineto{\pgfqpoint{2.115640in}{0.715470in}}%
\pgfpathlineto{\pgfqpoint{2.115931in}{0.702337in}}%
\pgfpathlineto{\pgfqpoint{2.105970in}{0.702125in}}%
\pgfpathlineto{\pgfqpoint{2.102724in}{0.695459in}}%
\pgfpathlineto{\pgfqpoint{2.096726in}{0.695335in}}%
\pgfpathlineto{\pgfqpoint{2.095722in}{0.688451in}}%
\pgfpathlineto{\pgfqpoint{2.092651in}{0.687691in}}%
\pgfpathlineto{\pgfqpoint{2.086863in}{0.697786in}}%
\pgfpathlineto{\pgfqpoint{2.087633in}{0.704363in}}%
\pgfpathlineto{\pgfqpoint{2.082698in}{0.709660in}}%
\pgfpathlineto{\pgfqpoint{2.076778in}{0.719433in}}%
\pgfpathlineto{\pgfqpoint{2.078940in}{0.725883in}}%
\pgfpathlineto{\pgfqpoint{2.077687in}{0.732624in}}%
\pgfpathclose%
\pgfusepath{fill}%
\end{pgfscope}%
\begin{pgfscope}%
\pgfpathrectangle{\pgfqpoint{0.100000in}{0.100000in}}{\pgfqpoint{3.420221in}{2.189500in}}%
\pgfusepath{clip}%
\pgfsetbuttcap%
\pgfsetmiterjoin%
\definecolor{currentfill}{rgb}{0.000000,0.298039,0.850980}%
\pgfsetfillcolor{currentfill}%
\pgfsetlinewidth{0.000000pt}%
\definecolor{currentstroke}{rgb}{0.000000,0.000000,0.000000}%
\pgfsetstrokecolor{currentstroke}%
\pgfsetstrokeopacity{0.000000}%
\pgfsetdash{}{0pt}%
\pgfpathmoveto{\pgfqpoint{1.647383in}{1.414389in}}%
\pgfpathlineto{\pgfqpoint{1.601856in}{1.417708in}}%
\pgfpathlineto{\pgfqpoint{1.602161in}{1.421596in}}%
\pgfpathlineto{\pgfqpoint{1.568209in}{1.424322in}}%
\pgfpathlineto{\pgfqpoint{1.569677in}{1.443236in}}%
\pgfpathlineto{\pgfqpoint{1.572404in}{1.446458in}}%
\pgfpathlineto{\pgfqpoint{1.573482in}{1.459420in}}%
\pgfpathlineto{\pgfqpoint{1.604664in}{1.456886in}}%
\pgfpathlineto{\pgfqpoint{1.605689in}{1.469862in}}%
\pgfpathlineto{\pgfqpoint{1.609647in}{1.469522in}}%
\pgfpathlineto{\pgfqpoint{1.642038in}{1.467184in}}%
\pgfpathlineto{\pgfqpoint{1.649754in}{1.466609in}}%
\pgfpathlineto{\pgfqpoint{1.647918in}{1.440620in}}%
\pgfpathlineto{\pgfqpoint{1.649067in}{1.440539in}}%
\pgfpathclose%
\pgfusepath{fill}%
\end{pgfscope}%
\begin{pgfscope}%
\pgfpathrectangle{\pgfqpoint{0.100000in}{0.100000in}}{\pgfqpoint{3.420221in}{2.189500in}}%
\pgfusepath{clip}%
\pgfsetbuttcap%
\pgfsetmiterjoin%
\definecolor{currentfill}{rgb}{0.000000,0.584314,0.707843}%
\pgfsetfillcolor{currentfill}%
\pgfsetlinewidth{0.000000pt}%
\definecolor{currentstroke}{rgb}{0.000000,0.000000,0.000000}%
\pgfsetstrokecolor{currentstroke}%
\pgfsetstrokeopacity{0.000000}%
\pgfsetdash{}{0pt}%
\pgfpathmoveto{\pgfqpoint{2.479149in}{1.119212in}}%
\pgfpathlineto{\pgfqpoint{2.476867in}{1.150453in}}%
\pgfpathlineto{\pgfqpoint{2.489296in}{1.151218in}}%
\pgfpathlineto{\pgfqpoint{2.499852in}{1.147904in}}%
\pgfpathlineto{\pgfqpoint{2.502423in}{1.159672in}}%
\pgfpathlineto{\pgfqpoint{2.507335in}{1.161916in}}%
\pgfpathlineto{\pgfqpoint{2.515047in}{1.162139in}}%
\pgfpathlineto{\pgfqpoint{2.509846in}{1.173025in}}%
\pgfpathlineto{\pgfqpoint{2.528212in}{1.175891in}}%
\pgfpathlineto{\pgfqpoint{2.535530in}{1.167564in}}%
\pgfpathlineto{\pgfqpoint{2.535228in}{1.155311in}}%
\pgfpathlineto{\pgfqpoint{2.530590in}{1.145864in}}%
\pgfpathlineto{\pgfqpoint{2.542992in}{1.132576in}}%
\pgfpathlineto{\pgfqpoint{2.544205in}{1.124211in}}%
\pgfpathlineto{\pgfqpoint{2.522784in}{1.123629in}}%
\pgfpathlineto{\pgfqpoint{2.507298in}{1.122451in}}%
\pgfpathclose%
\pgfusepath{fill}%
\end{pgfscope}%
\begin{pgfscope}%
\pgfpathrectangle{\pgfqpoint{0.100000in}{0.100000in}}{\pgfqpoint{3.420221in}{2.189500in}}%
\pgfusepath{clip}%
\pgfsetbuttcap%
\pgfsetmiterjoin%
\definecolor{currentfill}{rgb}{0.000000,0.392157,0.803922}%
\pgfsetfillcolor{currentfill}%
\pgfsetlinewidth{0.000000pt}%
\definecolor{currentstroke}{rgb}{0.000000,0.000000,0.000000}%
\pgfsetstrokecolor{currentstroke}%
\pgfsetstrokeopacity{0.000000}%
\pgfsetdash{}{0pt}%
\pgfpathmoveto{\pgfqpoint{0.929478in}{1.763712in}}%
\pgfpathlineto{\pgfqpoint{0.935542in}{1.766578in}}%
\pgfpathlineto{\pgfqpoint{0.937633in}{1.761103in}}%
\pgfpathlineto{\pgfqpoint{0.944645in}{1.759550in}}%
\pgfpathlineto{\pgfqpoint{0.952908in}{1.755206in}}%
\pgfpathlineto{\pgfqpoint{0.965173in}{1.751301in}}%
\pgfpathlineto{\pgfqpoint{0.966046in}{1.745912in}}%
\pgfpathlineto{\pgfqpoint{0.978666in}{1.735241in}}%
\pgfpathlineto{\pgfqpoint{0.985859in}{1.723095in}}%
\pgfpathlineto{\pgfqpoint{0.989920in}{1.723862in}}%
\pgfpathlineto{\pgfqpoint{0.992760in}{1.712732in}}%
\pgfpathlineto{\pgfqpoint{0.991445in}{1.706479in}}%
\pgfpathlineto{\pgfqpoint{1.006071in}{1.703435in}}%
\pgfpathlineto{\pgfqpoint{1.004826in}{1.697357in}}%
\pgfpathlineto{\pgfqpoint{1.023833in}{1.693513in}}%
\pgfpathlineto{\pgfqpoint{1.021244in}{1.680720in}}%
\pgfpathlineto{\pgfqpoint{1.008569in}{1.683288in}}%
\pgfpathlineto{\pgfqpoint{1.003703in}{1.657554in}}%
\pgfpathlineto{\pgfqpoint{1.005535in}{1.650532in}}%
\pgfpathlineto{\pgfqpoint{1.001651in}{1.647471in}}%
\pgfpathlineto{\pgfqpoint{0.995206in}{1.653570in}}%
\pgfpathlineto{\pgfqpoint{0.989590in}{1.653309in}}%
\pgfpathlineto{\pgfqpoint{0.992319in}{1.666542in}}%
\pgfpathlineto{\pgfqpoint{0.995507in}{1.665888in}}%
\pgfpathlineto{\pgfqpoint{1.000768in}{1.691681in}}%
\pgfpathlineto{\pgfqpoint{0.984634in}{1.695042in}}%
\pgfpathlineto{\pgfqpoint{0.979296in}{1.669284in}}%
\pgfpathlineto{\pgfqpoint{0.975251in}{1.663573in}}%
\pgfpathlineto{\pgfqpoint{0.966186in}{1.665540in}}%
\pgfpathlineto{\pgfqpoint{0.962678in}{1.648654in}}%
\pgfpathlineto{\pgfqpoint{0.955035in}{1.648683in}}%
\pgfpathlineto{\pgfqpoint{0.953399in}{1.641560in}}%
\pgfpathlineto{\pgfqpoint{0.941688in}{1.644080in}}%
\pgfpathlineto{\pgfqpoint{0.935196in}{1.613006in}}%
\pgfpathlineto{\pgfqpoint{0.905413in}{1.620130in}}%
\pgfpathlineto{\pgfqpoint{0.894039in}{1.622190in}}%
\pgfpathlineto{\pgfqpoint{0.909131in}{1.689261in}}%
\pgfpathlineto{\pgfqpoint{0.906510in}{1.690032in}}%
\pgfpathlineto{\pgfqpoint{0.918068in}{1.740546in}}%
\pgfpathlineto{\pgfqpoint{0.923535in}{1.746496in}}%
\pgfpathlineto{\pgfqpoint{0.922541in}{1.750344in}}%
\pgfpathlineto{\pgfqpoint{0.928311in}{1.758929in}}%
\pgfpathclose%
\pgfusepath{fill}%
\end{pgfscope}%
\begin{pgfscope}%
\pgfpathrectangle{\pgfqpoint{0.100000in}{0.100000in}}{\pgfqpoint{3.420221in}{2.189500in}}%
\pgfusepath{clip}%
\pgfsetbuttcap%
\pgfsetmiterjoin%
\definecolor{currentfill}{rgb}{0.000000,0.772549,0.613725}%
\pgfsetfillcolor{currentfill}%
\pgfsetlinewidth{0.000000pt}%
\definecolor{currentstroke}{rgb}{0.000000,0.000000,0.000000}%
\pgfsetstrokecolor{currentstroke}%
\pgfsetstrokeopacity{0.000000}%
\pgfsetdash{}{0pt}%
\pgfpathmoveto{\pgfqpoint{0.474232in}{1.671474in}}%
\pgfpathlineto{\pgfqpoint{0.468887in}{1.659928in}}%
\pgfpathlineto{\pgfqpoint{0.464920in}{1.658890in}}%
\pgfpathlineto{\pgfqpoint{0.455755in}{1.645066in}}%
\pgfpathlineto{\pgfqpoint{0.449506in}{1.638594in}}%
\pgfpathlineto{\pgfqpoint{0.450219in}{1.631697in}}%
\pgfpathlineto{\pgfqpoint{0.440206in}{1.628939in}}%
\pgfpathlineto{\pgfqpoint{0.435814in}{1.625621in}}%
\pgfpathlineto{\pgfqpoint{0.429934in}{1.625355in}}%
\pgfpathlineto{\pgfqpoint{0.424640in}{1.621297in}}%
\pgfpathlineto{\pgfqpoint{0.423337in}{1.616417in}}%
\pgfpathlineto{\pgfqpoint{0.426794in}{1.613035in}}%
\pgfpathlineto{\pgfqpoint{0.424225in}{1.605703in}}%
\pgfpathlineto{\pgfqpoint{0.423474in}{1.592812in}}%
\pgfpathlineto{\pgfqpoint{0.390074in}{1.603451in}}%
\pgfpathlineto{\pgfqpoint{0.390656in}{1.605224in}}%
\pgfpathlineto{\pgfqpoint{0.364478in}{1.613856in}}%
\pgfpathlineto{\pgfqpoint{0.362210in}{1.622163in}}%
\pgfpathlineto{\pgfqpoint{0.358680in}{1.625793in}}%
\pgfpathlineto{\pgfqpoint{0.352094in}{1.638023in}}%
\pgfpathlineto{\pgfqpoint{0.354199in}{1.641844in}}%
\pgfpathlineto{\pgfqpoint{0.353927in}{1.651774in}}%
\pgfpathlineto{\pgfqpoint{0.363933in}{1.664463in}}%
\pgfpathlineto{\pgfqpoint{0.379578in}{1.680791in}}%
\pgfpathlineto{\pgfqpoint{0.383821in}{1.688429in}}%
\pgfpathlineto{\pgfqpoint{0.382320in}{1.692173in}}%
\pgfpathlineto{\pgfqpoint{0.390649in}{1.705717in}}%
\pgfpathlineto{\pgfqpoint{0.396730in}{1.718631in}}%
\pgfpathlineto{\pgfqpoint{0.412441in}{1.713400in}}%
\pgfpathlineto{\pgfqpoint{0.410510in}{1.707451in}}%
\pgfpathlineto{\pgfqpoint{0.416363in}{1.705641in}}%
\pgfpathlineto{\pgfqpoint{0.424944in}{1.702887in}}%
\pgfpathlineto{\pgfqpoint{0.425646in}{1.685217in}}%
\pgfpathlineto{\pgfqpoint{0.431177in}{1.680416in}}%
\pgfpathlineto{\pgfqpoint{0.431662in}{1.676724in}}%
\pgfpathlineto{\pgfqpoint{0.438930in}{1.674194in}}%
\pgfpathlineto{\pgfqpoint{0.441182in}{1.667869in}}%
\pgfpathlineto{\pgfqpoint{0.449282in}{1.667261in}}%
\pgfpathlineto{\pgfqpoint{0.446966in}{1.674323in}}%
\pgfpathlineto{\pgfqpoint{0.450450in}{1.678603in}}%
\pgfpathlineto{\pgfqpoint{0.457833in}{1.678178in}}%
\pgfpathlineto{\pgfqpoint{0.473506in}{1.685552in}}%
\pgfpathlineto{\pgfqpoint{0.477517in}{1.683039in}}%
\pgfpathclose%
\pgfusepath{fill}%
\end{pgfscope}%
\begin{pgfscope}%
\pgfpathrectangle{\pgfqpoint{0.100000in}{0.100000in}}{\pgfqpoint{3.420221in}{2.189500in}}%
\pgfusepath{clip}%
\pgfsetbuttcap%
\pgfsetmiterjoin%
\definecolor{currentfill}{rgb}{0.000000,0.517647,0.741176}%
\pgfsetfillcolor{currentfill}%
\pgfsetlinewidth{0.000000pt}%
\definecolor{currentstroke}{rgb}{0.000000,0.000000,0.000000}%
\pgfsetstrokecolor{currentstroke}%
\pgfsetstrokeopacity{0.000000}%
\pgfsetdash{}{0pt}%
\pgfpathmoveto{\pgfqpoint{2.134964in}{1.300751in}}%
\pgfpathlineto{\pgfqpoint{2.130115in}{1.302277in}}%
\pgfpathlineto{\pgfqpoint{2.126904in}{1.293377in}}%
\pgfpathlineto{\pgfqpoint{2.122094in}{1.293487in}}%
\pgfpathlineto{\pgfqpoint{2.119768in}{1.299919in}}%
\pgfpathlineto{\pgfqpoint{2.113610in}{1.301438in}}%
\pgfpathlineto{\pgfqpoint{2.110870in}{1.307415in}}%
\pgfpathlineto{\pgfqpoint{2.105149in}{1.308234in}}%
\pgfpathlineto{\pgfqpoint{2.101749in}{1.312344in}}%
\pgfpathlineto{\pgfqpoint{2.102722in}{1.317540in}}%
\pgfpathlineto{\pgfqpoint{2.100965in}{1.327249in}}%
\pgfpathlineto{\pgfqpoint{2.096356in}{1.328557in}}%
\pgfpathlineto{\pgfqpoint{2.096159in}{1.348314in}}%
\pgfpathlineto{\pgfqpoint{2.096079in}{1.353229in}}%
\pgfpathlineto{\pgfqpoint{2.125076in}{1.354142in}}%
\pgfpathlineto{\pgfqpoint{2.125576in}{1.354158in}}%
\pgfpathlineto{\pgfqpoint{2.125573in}{1.328878in}}%
\pgfpathlineto{\pgfqpoint{2.135292in}{1.328973in}}%
\pgfpathclose%
\pgfusepath{fill}%
\end{pgfscope}%
\begin{pgfscope}%
\pgfpathrectangle{\pgfqpoint{0.100000in}{0.100000in}}{\pgfqpoint{3.420221in}{2.189500in}}%
\pgfusepath{clip}%
\pgfsetbuttcap%
\pgfsetmiterjoin%
\definecolor{currentfill}{rgb}{0.000000,0.474510,0.762745}%
\pgfsetfillcolor{currentfill}%
\pgfsetlinewidth{0.000000pt}%
\definecolor{currentstroke}{rgb}{0.000000,0.000000,0.000000}%
\pgfsetstrokecolor{currentstroke}%
\pgfsetstrokeopacity{0.000000}%
\pgfsetdash{}{0pt}%
\pgfpathmoveto{\pgfqpoint{1.765510in}{0.949553in}}%
\pgfpathlineto{\pgfqpoint{1.765243in}{0.943022in}}%
\pgfpathlineto{\pgfqpoint{1.775410in}{0.942610in}}%
\pgfpathlineto{\pgfqpoint{1.774886in}{0.929552in}}%
\pgfpathlineto{\pgfqpoint{1.778125in}{0.929439in}}%
\pgfpathlineto{\pgfqpoint{1.777557in}{0.916175in}}%
\pgfpathlineto{\pgfqpoint{1.772516in}{0.914576in}}%
\pgfpathlineto{\pgfqpoint{1.761097in}{0.916745in}}%
\pgfpathlineto{\pgfqpoint{1.756457in}{0.921234in}}%
\pgfpathlineto{\pgfqpoint{1.741743in}{0.922019in}}%
\pgfpathlineto{\pgfqpoint{1.740838in}{0.931281in}}%
\pgfpathlineto{\pgfqpoint{1.740724in}{0.943097in}}%
\pgfpathlineto{\pgfqpoint{1.744657in}{0.950005in}}%
\pgfpathclose%
\pgfusepath{fill}%
\end{pgfscope}%
\begin{pgfscope}%
\pgfpathrectangle{\pgfqpoint{0.100000in}{0.100000in}}{\pgfqpoint{3.420221in}{2.189500in}}%
\pgfusepath{clip}%
\pgfsetbuttcap%
\pgfsetmiterjoin%
\definecolor{currentfill}{rgb}{0.000000,0.286275,0.856863}%
\pgfsetfillcolor{currentfill}%
\pgfsetlinewidth{0.000000pt}%
\definecolor{currentstroke}{rgb}{0.000000,0.000000,0.000000}%
\pgfsetstrokecolor{currentstroke}%
\pgfsetstrokeopacity{0.000000}%
\pgfsetdash{}{0pt}%
\pgfpathmoveto{\pgfqpoint{1.944719in}{1.612797in}}%
\pgfpathlineto{\pgfqpoint{1.922914in}{1.613106in}}%
\pgfpathlineto{\pgfqpoint{1.923322in}{1.639349in}}%
\pgfpathlineto{\pgfqpoint{1.924224in}{1.691542in}}%
\pgfpathlineto{\pgfqpoint{1.924343in}{1.698139in}}%
\pgfpathlineto{\pgfqpoint{1.969756in}{1.697528in}}%
\pgfpathlineto{\pgfqpoint{1.970197in}{1.690924in}}%
\pgfpathlineto{\pgfqpoint{1.970103in}{1.664851in}}%
\pgfpathlineto{\pgfqpoint{1.944742in}{1.665145in}}%
\pgfpathlineto{\pgfqpoint{1.945068in}{1.639015in}}%
\pgfpathclose%
\pgfusepath{fill}%
\end{pgfscope}%
\begin{pgfscope}%
\pgfpathrectangle{\pgfqpoint{0.100000in}{0.100000in}}{\pgfqpoint{3.420221in}{2.189500in}}%
\pgfusepath{clip}%
\pgfsetbuttcap%
\pgfsetmiterjoin%
\definecolor{currentfill}{rgb}{0.000000,0.270588,0.864706}%
\pgfsetfillcolor{currentfill}%
\pgfsetlinewidth{0.000000pt}%
\definecolor{currentstroke}{rgb}{0.000000,0.000000,0.000000}%
\pgfsetstrokecolor{currentstroke}%
\pgfsetstrokeopacity{0.000000}%
\pgfsetdash{}{0pt}%
\pgfpathmoveto{\pgfqpoint{1.390297in}{1.569654in}}%
\pgfpathlineto{\pgfqpoint{1.387052in}{1.543851in}}%
\pgfpathlineto{\pgfqpoint{1.381110in}{1.492321in}}%
\pgfpathlineto{\pgfqpoint{1.366907in}{1.494088in}}%
\pgfpathlineto{\pgfqpoint{1.363314in}{1.464617in}}%
\pgfpathlineto{\pgfqpoint{1.333239in}{1.468741in}}%
\pgfpathlineto{\pgfqpoint{1.273790in}{1.476848in}}%
\pgfpathlineto{\pgfqpoint{1.278075in}{1.506260in}}%
\pgfpathlineto{\pgfqpoint{1.280147in}{1.525742in}}%
\pgfpathlineto{\pgfqpoint{1.303427in}{1.522299in}}%
\pgfpathlineto{\pgfqpoint{1.308183in}{1.554240in}}%
\pgfpathlineto{\pgfqpoint{1.308933in}{1.567300in}}%
\pgfpathlineto{\pgfqpoint{1.310701in}{1.580131in}}%
\pgfpathlineto{\pgfqpoint{1.359960in}{1.573255in}}%
\pgfpathclose%
\pgfusepath{fill}%
\end{pgfscope}%
\begin{pgfscope}%
\pgfpathrectangle{\pgfqpoint{0.100000in}{0.100000in}}{\pgfqpoint{3.420221in}{2.189500in}}%
\pgfusepath{clip}%
\pgfsetbuttcap%
\pgfsetmiterjoin%
\definecolor{currentfill}{rgb}{0.000000,0.168627,0.915686}%
\pgfsetfillcolor{currentfill}%
\pgfsetlinewidth{0.000000pt}%
\definecolor{currentstroke}{rgb}{0.000000,0.000000,0.000000}%
\pgfsetstrokecolor{currentstroke}%
\pgfsetstrokeopacity{0.000000}%
\pgfsetdash{}{0pt}%
\pgfpathmoveto{\pgfqpoint{1.866155in}{1.351828in}}%
\pgfpathlineto{\pgfqpoint{1.866876in}{1.377916in}}%
\pgfpathlineto{\pgfqpoint{1.868247in}{1.430099in}}%
\pgfpathlineto{\pgfqpoint{1.894224in}{1.429423in}}%
\pgfpathlineto{\pgfqpoint{1.919370in}{1.428885in}}%
\pgfpathlineto{\pgfqpoint{1.919043in}{1.409319in}}%
\pgfpathlineto{\pgfqpoint{1.918109in}{1.350651in}}%
\pgfpathclose%
\pgfusepath{fill}%
\end{pgfscope}%
\begin{pgfscope}%
\pgfpathrectangle{\pgfqpoint{0.100000in}{0.100000in}}{\pgfqpoint{3.420221in}{2.189500in}}%
\pgfusepath{clip}%
\pgfsetbuttcap%
\pgfsetmiterjoin%
\definecolor{currentfill}{rgb}{0.000000,0.207843,0.896078}%
\pgfsetfillcolor{currentfill}%
\pgfsetlinewidth{0.000000pt}%
\definecolor{currentstroke}{rgb}{0.000000,0.000000,0.000000}%
\pgfsetstrokecolor{currentstroke}%
\pgfsetstrokeopacity{0.000000}%
\pgfsetdash{}{0pt}%
\pgfpathmoveto{\pgfqpoint{1.863028in}{1.896506in}}%
\pgfpathlineto{\pgfqpoint{1.849927in}{1.896953in}}%
\pgfpathlineto{\pgfqpoint{1.848833in}{1.903474in}}%
\pgfpathlineto{\pgfqpoint{1.849785in}{1.929677in}}%
\pgfpathlineto{\pgfqpoint{1.854997in}{1.929515in}}%
\pgfpathlineto{\pgfqpoint{1.855859in}{1.955811in}}%
\pgfpathlineto{\pgfqpoint{1.855200in}{1.969173in}}%
\pgfpathlineto{\pgfqpoint{1.893757in}{1.968006in}}%
\pgfpathlineto{\pgfqpoint{1.893444in}{1.966466in}}%
\pgfpathlineto{\pgfqpoint{1.926137in}{1.965878in}}%
\pgfpathlineto{\pgfqpoint{1.926685in}{1.941078in}}%
\pgfpathlineto{\pgfqpoint{1.933258in}{1.940959in}}%
\pgfpathlineto{\pgfqpoint{1.933211in}{1.934379in}}%
\pgfpathlineto{\pgfqpoint{1.959446in}{1.934083in}}%
\pgfpathlineto{\pgfqpoint{1.959519in}{1.940720in}}%
\pgfpathlineto{\pgfqpoint{1.966019in}{1.940691in}}%
\pgfpathlineto{\pgfqpoint{1.966050in}{1.947309in}}%
\pgfpathlineto{\pgfqpoint{1.972522in}{1.947123in}}%
\pgfpathlineto{\pgfqpoint{1.972565in}{1.927375in}}%
\pgfpathlineto{\pgfqpoint{1.973742in}{1.914239in}}%
\pgfpathlineto{\pgfqpoint{1.947469in}{1.914387in}}%
\pgfpathlineto{\pgfqpoint{1.907479in}{1.915012in}}%
\pgfpathlineto{\pgfqpoint{1.907894in}{1.895201in}}%
\pgfpathclose%
\pgfusepath{fill}%
\end{pgfscope}%
\begin{pgfscope}%
\pgfpathrectangle{\pgfqpoint{0.100000in}{0.100000in}}{\pgfqpoint{3.420221in}{2.189500in}}%
\pgfusepath{clip}%
\pgfsetbuttcap%
\pgfsetmiterjoin%
\definecolor{currentfill}{rgb}{0.000000,0.239216,0.880392}%
\pgfsetfillcolor{currentfill}%
\pgfsetlinewidth{0.000000pt}%
\definecolor{currentstroke}{rgb}{0.000000,0.000000,0.000000}%
\pgfsetstrokecolor{currentstroke}%
\pgfsetstrokeopacity{0.000000}%
\pgfsetdash{}{0pt}%
\pgfpathmoveto{\pgfqpoint{1.996243in}{1.664701in}}%
\pgfpathlineto{\pgfqpoint{1.970103in}{1.664851in}}%
\pgfpathlineto{\pgfqpoint{1.970197in}{1.690924in}}%
\pgfpathlineto{\pgfqpoint{1.982776in}{1.690811in}}%
\pgfpathlineto{\pgfqpoint{1.982835in}{1.702622in}}%
\pgfpathlineto{\pgfqpoint{1.976292in}{1.706766in}}%
\pgfpathlineto{\pgfqpoint{1.976316in}{1.717134in}}%
\pgfpathlineto{\pgfqpoint{2.015011in}{1.717178in}}%
\pgfpathlineto{\pgfqpoint{2.028909in}{1.717250in}}%
\pgfpathlineto{\pgfqpoint{2.028962in}{1.704071in}}%
\pgfpathlineto{\pgfqpoint{2.021917in}{1.704006in}}%
\pgfpathlineto{\pgfqpoint{2.022294in}{1.684387in}}%
\pgfpathlineto{\pgfqpoint{2.013338in}{1.684387in}}%
\pgfpathlineto{\pgfqpoint{2.009313in}{1.687507in}}%
\pgfpathlineto{\pgfqpoint{2.009313in}{1.671263in}}%
\pgfpathlineto{\pgfqpoint{1.996323in}{1.671225in}}%
\pgfpathclose%
\pgfusepath{fill}%
\end{pgfscope}%
\begin{pgfscope}%
\pgfpathrectangle{\pgfqpoint{0.100000in}{0.100000in}}{\pgfqpoint{3.420221in}{2.189500in}}%
\pgfusepath{clip}%
\pgfsetbuttcap%
\pgfsetmiterjoin%
\definecolor{currentfill}{rgb}{0.000000,0.541176,0.729412}%
\pgfsetfillcolor{currentfill}%
\pgfsetlinewidth{0.000000pt}%
\definecolor{currentstroke}{rgb}{0.000000,0.000000,0.000000}%
\pgfsetstrokecolor{currentstroke}%
\pgfsetstrokeopacity{0.000000}%
\pgfsetdash{}{0pt}%
\pgfpathmoveto{\pgfqpoint{2.202883in}{1.970731in}}%
\pgfpathlineto{\pgfqpoint{2.212013in}{1.975395in}}%
\pgfpathlineto{\pgfqpoint{2.217394in}{1.963980in}}%
\pgfpathlineto{\pgfqpoint{2.236485in}{1.964911in}}%
\pgfpathlineto{\pgfqpoint{2.248475in}{1.967143in}}%
\pgfpathlineto{\pgfqpoint{2.254277in}{1.965379in}}%
\pgfpathlineto{\pgfqpoint{2.256023in}{1.961179in}}%
\pgfpathlineto{\pgfqpoint{2.262311in}{1.958279in}}%
\pgfpathlineto{\pgfqpoint{2.267450in}{1.961364in}}%
\pgfpathlineto{\pgfqpoint{2.275145in}{1.960963in}}%
\pgfpathlineto{\pgfqpoint{2.274356in}{1.956603in}}%
\pgfpathlineto{\pgfqpoint{2.258021in}{1.946424in}}%
\pgfpathlineto{\pgfqpoint{2.247204in}{1.941526in}}%
\pgfpathlineto{\pgfqpoint{2.229747in}{1.935106in}}%
\pgfpathlineto{\pgfqpoint{2.220003in}{1.928720in}}%
\pgfpathlineto{\pgfqpoint{2.205721in}{1.915932in}}%
\pgfpathlineto{\pgfqpoint{2.204439in}{1.947977in}}%
\pgfpathclose%
\pgfusepath{fill}%
\end{pgfscope}%
\begin{pgfscope}%
\pgfpathrectangle{\pgfqpoint{0.100000in}{0.100000in}}{\pgfqpoint{3.420221in}{2.189500in}}%
\pgfusepath{clip}%
\pgfsetbuttcap%
\pgfsetmiterjoin%
\definecolor{currentfill}{rgb}{0.000000,0.603922,0.698039}%
\pgfsetfillcolor{currentfill}%
\pgfsetlinewidth{0.000000pt}%
\definecolor{currentstroke}{rgb}{0.000000,0.000000,0.000000}%
\pgfsetstrokecolor{currentstroke}%
\pgfsetstrokeopacity{0.000000}%
\pgfsetdash{}{0pt}%
\pgfpathmoveto{\pgfqpoint{2.699633in}{1.300420in}}%
\pgfpathlineto{\pgfqpoint{2.693351in}{1.291655in}}%
\pgfpathlineto{\pgfqpoint{2.686944in}{1.290644in}}%
\pgfpathlineto{\pgfqpoint{2.682927in}{1.320166in}}%
\pgfpathlineto{\pgfqpoint{2.675925in}{1.322230in}}%
\pgfpathlineto{\pgfqpoint{2.676651in}{1.335763in}}%
\pgfpathlineto{\pgfqpoint{2.674328in}{1.336710in}}%
\pgfpathlineto{\pgfqpoint{2.673588in}{1.346263in}}%
\pgfpathlineto{\pgfqpoint{2.678338in}{1.357370in}}%
\pgfpathlineto{\pgfqpoint{2.694600in}{1.358789in}}%
\pgfpathlineto{\pgfqpoint{2.708967in}{1.357806in}}%
\pgfpathlineto{\pgfqpoint{2.709514in}{1.350316in}}%
\pgfpathlineto{\pgfqpoint{2.722756in}{1.352228in}}%
\pgfpathlineto{\pgfqpoint{2.722664in}{1.354204in}}%
\pgfpathlineto{\pgfqpoint{2.735692in}{1.355144in}}%
\pgfpathlineto{\pgfqpoint{2.738321in}{1.348607in}}%
\pgfpathlineto{\pgfqpoint{2.736564in}{1.341751in}}%
\pgfpathlineto{\pgfqpoint{2.737485in}{1.328426in}}%
\pgfpathlineto{\pgfqpoint{2.730929in}{1.328077in}}%
\pgfpathlineto{\pgfqpoint{2.731746in}{1.313774in}}%
\pgfpathlineto{\pgfqpoint{2.724789in}{1.312804in}}%
\pgfpathlineto{\pgfqpoint{2.720435in}{1.312555in}}%
\pgfpathlineto{\pgfqpoint{2.720720in}{1.304949in}}%
\pgfpathlineto{\pgfqpoint{2.717753in}{1.293143in}}%
\pgfpathlineto{\pgfqpoint{2.711161in}{1.293000in}}%
\pgfpathlineto{\pgfqpoint{2.707242in}{1.303776in}}%
\pgfpathclose%
\pgfusepath{fill}%
\end{pgfscope}%
\begin{pgfscope}%
\pgfpathrectangle{\pgfqpoint{0.100000in}{0.100000in}}{\pgfqpoint{3.420221in}{2.189500in}}%
\pgfusepath{clip}%
\pgfsetbuttcap%
\pgfsetmiterjoin%
\definecolor{currentfill}{rgb}{0.000000,0.301961,0.849020}%
\pgfsetfillcolor{currentfill}%
\pgfsetlinewidth{0.000000pt}%
\definecolor{currentstroke}{rgb}{0.000000,0.000000,0.000000}%
\pgfsetstrokecolor{currentstroke}%
\pgfsetstrokeopacity{0.000000}%
\pgfsetdash{}{0pt}%
\pgfpathmoveto{\pgfqpoint{2.282670in}{1.545611in}}%
\pgfpathlineto{\pgfqpoint{2.284345in}{1.522444in}}%
\pgfpathlineto{\pgfqpoint{2.262291in}{1.521054in}}%
\pgfpathlineto{\pgfqpoint{2.257946in}{1.523356in}}%
\pgfpathlineto{\pgfqpoint{2.256278in}{1.530629in}}%
\pgfpathlineto{\pgfqpoint{2.252895in}{1.534694in}}%
\pgfpathlineto{\pgfqpoint{2.242304in}{1.534233in}}%
\pgfpathlineto{\pgfqpoint{2.242589in}{1.527674in}}%
\pgfpathlineto{\pgfqpoint{2.229698in}{1.527114in}}%
\pgfpathlineto{\pgfqpoint{2.190939in}{1.525590in}}%
\pgfpathlineto{\pgfqpoint{2.189812in}{1.538531in}}%
\pgfpathlineto{\pgfqpoint{2.188271in}{1.584492in}}%
\pgfpathlineto{\pgfqpoint{2.186791in}{1.615926in}}%
\pgfpathlineto{\pgfqpoint{2.208205in}{1.616773in}}%
\pgfpathlineto{\pgfqpoint{2.209111in}{1.610975in}}%
\pgfpathlineto{\pgfqpoint{2.209544in}{1.605439in}}%
\pgfpathlineto{\pgfqpoint{2.217706in}{1.598739in}}%
\pgfpathlineto{\pgfqpoint{2.211674in}{1.589406in}}%
\pgfpathlineto{\pgfqpoint{2.214152in}{1.572223in}}%
\pgfpathlineto{\pgfqpoint{2.216574in}{1.570704in}}%
\pgfpathlineto{\pgfqpoint{2.219620in}{1.559951in}}%
\pgfpathlineto{\pgfqpoint{2.225981in}{1.556144in}}%
\pgfpathlineto{\pgfqpoint{2.238336in}{1.553473in}}%
\pgfpathlineto{\pgfqpoint{2.243185in}{1.543750in}}%
\pgfpathclose%
\pgfusepath{fill}%
\end{pgfscope}%
\begin{pgfscope}%
\pgfpathrectangle{\pgfqpoint{0.100000in}{0.100000in}}{\pgfqpoint{3.420221in}{2.189500in}}%
\pgfusepath{clip}%
\pgfsetbuttcap%
\pgfsetmiterjoin%
\definecolor{currentfill}{rgb}{0.000000,0.360784,0.819608}%
\pgfsetfillcolor{currentfill}%
\pgfsetlinewidth{0.000000pt}%
\definecolor{currentstroke}{rgb}{0.000000,0.000000,0.000000}%
\pgfsetstrokecolor{currentstroke}%
\pgfsetstrokeopacity{0.000000}%
\pgfsetdash{}{0pt}%
\pgfpathmoveto{\pgfqpoint{2.948821in}{1.212787in}}%
\pgfpathlineto{\pgfqpoint{2.942969in}{1.211233in}}%
\pgfpathlineto{\pgfqpoint{2.939060in}{1.215848in}}%
\pgfpathlineto{\pgfqpoint{2.933306in}{1.215168in}}%
\pgfpathlineto{\pgfqpoint{2.925981in}{1.209025in}}%
\pgfpathlineto{\pgfqpoint{2.919493in}{1.206218in}}%
\pgfpathlineto{\pgfqpoint{2.915058in}{1.211890in}}%
\pgfpathlineto{\pgfqpoint{2.903233in}{1.216675in}}%
\pgfpathlineto{\pgfqpoint{2.901329in}{1.224115in}}%
\pgfpathlineto{\pgfqpoint{2.903330in}{1.231342in}}%
\pgfpathlineto{\pgfqpoint{2.910126in}{1.238283in}}%
\pgfpathlineto{\pgfqpoint{2.912880in}{1.236707in}}%
\pgfpathlineto{\pgfqpoint{2.918882in}{1.243428in}}%
\pgfpathlineto{\pgfqpoint{2.920458in}{1.248748in}}%
\pgfpathlineto{\pgfqpoint{2.926215in}{1.254415in}}%
\pgfpathlineto{\pgfqpoint{2.929214in}{1.265132in}}%
\pgfpathlineto{\pgfqpoint{2.936332in}{1.267861in}}%
\pgfpathlineto{\pgfqpoint{2.942362in}{1.261178in}}%
\pgfpathlineto{\pgfqpoint{2.946711in}{1.260814in}}%
\pgfpathlineto{\pgfqpoint{2.953851in}{1.250420in}}%
\pgfpathlineto{\pgfqpoint{2.958031in}{1.251815in}}%
\pgfpathlineto{\pgfqpoint{2.967963in}{1.244216in}}%
\pgfpathlineto{\pgfqpoint{2.973603in}{1.242864in}}%
\pgfpathlineto{\pgfqpoint{2.970496in}{1.230929in}}%
\pgfpathlineto{\pgfqpoint{2.962697in}{1.226216in}}%
\pgfpathlineto{\pgfqpoint{2.960392in}{1.211860in}}%
\pgfpathclose%
\pgfusepath{fill}%
\end{pgfscope}%
\begin{pgfscope}%
\pgfpathrectangle{\pgfqpoint{0.100000in}{0.100000in}}{\pgfqpoint{3.420221in}{2.189500in}}%
\pgfusepath{clip}%
\pgfsetbuttcap%
\pgfsetmiterjoin%
\definecolor{currentfill}{rgb}{0.000000,0.764706,0.617647}%
\pgfsetfillcolor{currentfill}%
\pgfsetlinewidth{0.000000pt}%
\definecolor{currentstroke}{rgb}{0.000000,0.000000,0.000000}%
\pgfsetstrokecolor{currentstroke}%
\pgfsetstrokeopacity{0.000000}%
\pgfsetdash{}{0pt}%
\pgfpathmoveto{\pgfqpoint{0.589314in}{1.827339in}}%
\pgfpathlineto{\pgfqpoint{0.639058in}{1.812355in}}%
\pgfpathlineto{\pgfqpoint{0.664264in}{1.805137in}}%
\pgfpathlineto{\pgfqpoint{0.655476in}{1.773949in}}%
\pgfpathlineto{\pgfqpoint{0.653595in}{1.774476in}}%
\pgfpathlineto{\pgfqpoint{0.644185in}{1.743373in}}%
\pgfpathlineto{\pgfqpoint{0.674970in}{1.735022in}}%
\pgfpathlineto{\pgfqpoint{0.660242in}{1.680338in}}%
\pgfpathlineto{\pgfqpoint{0.625927in}{1.689905in}}%
\pgfpathlineto{\pgfqpoint{0.578745in}{1.703409in}}%
\pgfpathlineto{\pgfqpoint{0.594355in}{1.757492in}}%
\pgfpathlineto{\pgfqpoint{0.569749in}{1.764981in}}%
\pgfpathlineto{\pgfqpoint{0.582916in}{1.808864in}}%
\pgfpathclose%
\pgfusepath{fill}%
\end{pgfscope}%
\begin{pgfscope}%
\pgfpathrectangle{\pgfqpoint{0.100000in}{0.100000in}}{\pgfqpoint{3.420221in}{2.189500in}}%
\pgfusepath{clip}%
\pgfsetbuttcap%
\pgfsetmiterjoin%
\definecolor{currentfill}{rgb}{0.000000,0.321569,0.839216}%
\pgfsetfillcolor{currentfill}%
\pgfsetlinewidth{0.000000pt}%
\definecolor{currentstroke}{rgb}{0.000000,0.000000,0.000000}%
\pgfsetstrokecolor{currentstroke}%
\pgfsetstrokeopacity{0.000000}%
\pgfsetdash{}{0pt}%
\pgfpathmoveto{\pgfqpoint{2.032974in}{1.032132in}}%
\pgfpathlineto{\pgfqpoint{2.028586in}{1.062314in}}%
\pgfpathlineto{\pgfqpoint{2.001425in}{1.062254in}}%
\pgfpathlineto{\pgfqpoint{2.002136in}{1.100420in}}%
\pgfpathlineto{\pgfqpoint{2.025079in}{1.100257in}}%
\pgfpathlineto{\pgfqpoint{2.025034in}{1.107657in}}%
\pgfpathlineto{\pgfqpoint{2.058049in}{1.106494in}}%
\pgfpathlineto{\pgfqpoint{2.057743in}{1.087809in}}%
\pgfpathlineto{\pgfqpoint{2.070421in}{1.087952in}}%
\pgfpathlineto{\pgfqpoint{2.070461in}{1.073606in}}%
\pgfpathlineto{\pgfqpoint{2.083243in}{1.073452in}}%
\pgfpathlineto{\pgfqpoint{2.085631in}{1.065848in}}%
\pgfpathlineto{\pgfqpoint{2.089902in}{1.060281in}}%
\pgfpathlineto{\pgfqpoint{2.096342in}{1.056981in}}%
\pgfpathlineto{\pgfqpoint{2.096178in}{1.048480in}}%
\pgfpathlineto{\pgfqpoint{2.091855in}{1.041962in}}%
\pgfpathlineto{\pgfqpoint{2.092270in}{1.033108in}}%
\pgfpathlineto{\pgfqpoint{2.068429in}{1.032488in}}%
\pgfpathlineto{\pgfqpoint{2.055403in}{1.031647in}}%
\pgfpathclose%
\pgfusepath{fill}%
\end{pgfscope}%
\begin{pgfscope}%
\pgfpathrectangle{\pgfqpoint{0.100000in}{0.100000in}}{\pgfqpoint{3.420221in}{2.189500in}}%
\pgfusepath{clip}%
\pgfsetbuttcap%
\pgfsetmiterjoin%
\definecolor{currentfill}{rgb}{0.000000,0.368627,0.815686}%
\pgfsetfillcolor{currentfill}%
\pgfsetlinewidth{0.000000pt}%
\definecolor{currentstroke}{rgb}{0.000000,0.000000,0.000000}%
\pgfsetstrokecolor{currentstroke}%
\pgfsetstrokeopacity{0.000000}%
\pgfsetdash{}{0pt}%
\pgfpathmoveto{\pgfqpoint{2.926173in}{0.933340in}}%
\pgfpathlineto{\pgfqpoint{2.925969in}{0.922296in}}%
\pgfpathlineto{\pgfqpoint{2.915712in}{0.917453in}}%
\pgfpathlineto{\pgfqpoint{2.910088in}{0.922398in}}%
\pgfpathlineto{\pgfqpoint{2.893603in}{0.907760in}}%
\pgfpathlineto{\pgfqpoint{2.881371in}{0.916643in}}%
\pgfpathlineto{\pgfqpoint{2.866576in}{0.921396in}}%
\pgfpathlineto{\pgfqpoint{2.859091in}{0.925505in}}%
\pgfpathlineto{\pgfqpoint{2.854258in}{0.925569in}}%
\pgfpathlineto{\pgfqpoint{2.863931in}{0.939400in}}%
\pgfpathlineto{\pgfqpoint{2.848095in}{0.943859in}}%
\pgfpathlineto{\pgfqpoint{2.837647in}{0.952515in}}%
\pgfpathlineto{\pgfqpoint{2.833375in}{0.947100in}}%
\pgfpathlineto{\pgfqpoint{2.821373in}{0.949289in}}%
\pgfpathlineto{\pgfqpoint{2.816623in}{0.956882in}}%
\pgfpathlineto{\pgfqpoint{2.809775in}{0.954853in}}%
\pgfpathlineto{\pgfqpoint{2.814593in}{0.969607in}}%
\pgfpathlineto{\pgfqpoint{2.812007in}{0.974776in}}%
\pgfpathlineto{\pgfqpoint{2.813065in}{0.981065in}}%
\pgfpathlineto{\pgfqpoint{2.819299in}{0.986906in}}%
\pgfpathlineto{\pgfqpoint{2.825974in}{1.000665in}}%
\pgfpathlineto{\pgfqpoint{2.832152in}{0.997272in}}%
\pgfpathlineto{\pgfqpoint{2.839927in}{1.000439in}}%
\pgfpathlineto{\pgfqpoint{2.839061in}{1.005538in}}%
\pgfpathlineto{\pgfqpoint{2.872740in}{1.008515in}}%
\pgfpathlineto{\pgfqpoint{2.873627in}{1.002160in}}%
\pgfpathlineto{\pgfqpoint{2.881263in}{0.997223in}}%
\pgfpathlineto{\pgfqpoint{2.877414in}{0.981799in}}%
\pgfpathlineto{\pgfqpoint{2.887990in}{0.974832in}}%
\pgfpathlineto{\pgfqpoint{2.892801in}{0.977782in}}%
\pgfpathlineto{\pgfqpoint{2.895643in}{0.973761in}}%
\pgfpathlineto{\pgfqpoint{2.895848in}{0.959538in}}%
\pgfpathlineto{\pgfqpoint{2.898575in}{0.956373in}}%
\pgfpathlineto{\pgfqpoint{2.897625in}{0.950640in}}%
\pgfpathlineto{\pgfqpoint{2.904896in}{0.944603in}}%
\pgfpathlineto{\pgfqpoint{2.910722in}{0.935339in}}%
\pgfpathlineto{\pgfqpoint{2.918286in}{0.931024in}}%
\pgfpathclose%
\pgfusepath{fill}%
\end{pgfscope}%
\begin{pgfscope}%
\pgfpathrectangle{\pgfqpoint{0.100000in}{0.100000in}}{\pgfqpoint{3.420221in}{2.189500in}}%
\pgfusepath{clip}%
\pgfsetbuttcap%
\pgfsetmiterjoin%
\definecolor{currentfill}{rgb}{0.000000,0.607843,0.696078}%
\pgfsetfillcolor{currentfill}%
\pgfsetlinewidth{0.000000pt}%
\definecolor{currentstroke}{rgb}{0.000000,0.000000,0.000000}%
\pgfsetstrokecolor{currentstroke}%
\pgfsetstrokeopacity{0.000000}%
\pgfsetdash{}{0pt}%
\pgfpathmoveto{\pgfqpoint{2.731574in}{1.022907in}}%
\pgfpathlineto{\pgfqpoint{2.731380in}{1.018465in}}%
\pgfpathlineto{\pgfqpoint{2.725427in}{1.012933in}}%
\pgfpathlineto{\pgfqpoint{2.720465in}{1.005440in}}%
\pgfpathlineto{\pgfqpoint{2.719427in}{0.999752in}}%
\pgfpathlineto{\pgfqpoint{2.706921in}{1.000730in}}%
\pgfpathlineto{\pgfqpoint{2.704040in}{1.005711in}}%
\pgfpathlineto{\pgfqpoint{2.700065in}{1.005393in}}%
\pgfpathlineto{\pgfqpoint{2.699411in}{1.009368in}}%
\pgfpathlineto{\pgfqpoint{2.706657in}{1.019763in}}%
\pgfpathlineto{\pgfqpoint{2.697265in}{1.030022in}}%
\pgfpathlineto{\pgfqpoint{2.691717in}{1.029547in}}%
\pgfpathlineto{\pgfqpoint{2.693020in}{1.036722in}}%
\pgfpathlineto{\pgfqpoint{2.694535in}{1.039478in}}%
\pgfpathlineto{\pgfqpoint{2.707520in}{1.042670in}}%
\pgfpathlineto{\pgfqpoint{2.714299in}{1.045662in}}%
\pgfpathlineto{\pgfqpoint{2.723028in}{1.033653in}}%
\pgfpathlineto{\pgfqpoint{2.727917in}{1.029811in}}%
\pgfpathclose%
\pgfusepath{fill}%
\end{pgfscope}%
\begin{pgfscope}%
\pgfpathrectangle{\pgfqpoint{0.100000in}{0.100000in}}{\pgfqpoint{3.420221in}{2.189500in}}%
\pgfusepath{clip}%
\pgfsetbuttcap%
\pgfsetmiterjoin%
\definecolor{currentfill}{rgb}{0.000000,0.596078,0.701961}%
\pgfsetfillcolor{currentfill}%
\pgfsetlinewidth{0.000000pt}%
\definecolor{currentstroke}{rgb}{0.000000,0.000000,0.000000}%
\pgfsetstrokecolor{currentstroke}%
\pgfsetstrokeopacity{0.000000}%
\pgfsetdash{}{0pt}%
\pgfpathmoveto{\pgfqpoint{2.174022in}{1.072405in}}%
\pgfpathlineto{\pgfqpoint{2.171980in}{1.062716in}}%
\pgfpathlineto{\pgfqpoint{2.172031in}{1.057186in}}%
\pgfpathlineto{\pgfqpoint{2.165522in}{1.057123in}}%
\pgfpathlineto{\pgfqpoint{2.165578in}{1.050591in}}%
\pgfpathlineto{\pgfqpoint{2.159074in}{1.050540in}}%
\pgfpathlineto{\pgfqpoint{2.159059in}{1.057061in}}%
\pgfpathlineto{\pgfqpoint{2.133205in}{1.056938in}}%
\pgfpathlineto{\pgfqpoint{2.129921in}{1.060163in}}%
\pgfpathlineto{\pgfqpoint{2.129611in}{1.088422in}}%
\pgfpathlineto{\pgfqpoint{2.131453in}{1.089002in}}%
\pgfpathlineto{\pgfqpoint{2.136386in}{1.089101in}}%
\pgfpathlineto{\pgfqpoint{2.136270in}{1.112219in}}%
\pgfpathlineto{\pgfqpoint{2.175450in}{1.112439in}}%
\pgfpathlineto{\pgfqpoint{2.175674in}{1.090287in}}%
\pgfpathlineto{\pgfqpoint{2.173867in}{1.090211in}}%
\pgfpathclose%
\pgfusepath{fill}%
\end{pgfscope}%
\begin{pgfscope}%
\pgfpathrectangle{\pgfqpoint{0.100000in}{0.100000in}}{\pgfqpoint{3.420221in}{2.189500in}}%
\pgfusepath{clip}%
\pgfsetbuttcap%
\pgfsetmiterjoin%
\definecolor{currentfill}{rgb}{0.000000,0.556863,0.721569}%
\pgfsetfillcolor{currentfill}%
\pgfsetlinewidth{0.000000pt}%
\definecolor{currentstroke}{rgb}{0.000000,0.000000,0.000000}%
\pgfsetstrokecolor{currentstroke}%
\pgfsetstrokeopacity{0.000000}%
\pgfsetdash{}{0pt}%
\pgfpathmoveto{\pgfqpoint{3.035557in}{1.216103in}}%
\pgfpathlineto{\pgfqpoint{3.038259in}{1.202805in}}%
\pgfpathlineto{\pgfqpoint{3.034068in}{1.189160in}}%
\pgfpathlineto{\pgfqpoint{3.026229in}{1.187604in}}%
\pgfpathlineto{\pgfqpoint{2.993264in}{1.181265in}}%
\pgfpathlineto{\pgfqpoint{2.952898in}{1.173935in}}%
\pgfpathlineto{\pgfqpoint{2.948111in}{1.173085in}}%
\pgfpathlineto{\pgfqpoint{2.948821in}{1.212787in}}%
\pgfpathlineto{\pgfqpoint{2.960392in}{1.211860in}}%
\pgfpathlineto{\pgfqpoint{2.962697in}{1.226216in}}%
\pgfpathlineto{\pgfqpoint{2.970496in}{1.230929in}}%
\pgfpathlineto{\pgfqpoint{2.973603in}{1.242864in}}%
\pgfpathlineto{\pgfqpoint{2.981969in}{1.239955in}}%
\pgfpathlineto{\pgfqpoint{2.990564in}{1.264098in}}%
\pgfpathlineto{\pgfqpoint{2.990351in}{1.268242in}}%
\pgfpathlineto{\pgfqpoint{2.994238in}{1.273383in}}%
\pgfpathlineto{\pgfqpoint{3.000170in}{1.267227in}}%
\pgfpathlineto{\pgfqpoint{2.999890in}{1.252063in}}%
\pgfpathlineto{\pgfqpoint{2.995055in}{1.244552in}}%
\pgfpathlineto{\pgfqpoint{2.996268in}{1.239293in}}%
\pgfpathlineto{\pgfqpoint{3.023229in}{1.236485in}}%
\pgfpathlineto{\pgfqpoint{3.017791in}{1.231733in}}%
\pgfpathlineto{\pgfqpoint{3.020586in}{1.220429in}}%
\pgfpathlineto{\pgfqpoint{3.028095in}{1.220784in}}%
\pgfpathclose%
\pgfusepath{fill}%
\end{pgfscope}%
\begin{pgfscope}%
\pgfpathrectangle{\pgfqpoint{0.100000in}{0.100000in}}{\pgfqpoint{3.420221in}{2.189500in}}%
\pgfusepath{clip}%
\pgfsetbuttcap%
\pgfsetmiterjoin%
\definecolor{currentfill}{rgb}{0.000000,0.611765,0.694118}%
\pgfsetfillcolor{currentfill}%
\pgfsetlinewidth{0.000000pt}%
\definecolor{currentstroke}{rgb}{0.000000,0.000000,0.000000}%
\pgfsetstrokecolor{currentstroke}%
\pgfsetstrokeopacity{0.000000}%
\pgfsetdash{}{0pt}%
\pgfpathmoveto{\pgfqpoint{0.985859in}{1.723095in}}%
\pgfpathlineto{\pgfqpoint{0.978666in}{1.735241in}}%
\pgfpathlineto{\pgfqpoint{0.966046in}{1.745912in}}%
\pgfpathlineto{\pgfqpoint{0.965173in}{1.751301in}}%
\pgfpathlineto{\pgfqpoint{0.952908in}{1.755206in}}%
\pgfpathlineto{\pgfqpoint{0.944645in}{1.759550in}}%
\pgfpathlineto{\pgfqpoint{0.937633in}{1.761103in}}%
\pgfpathlineto{\pgfqpoint{0.935542in}{1.766578in}}%
\pgfpathlineto{\pgfqpoint{0.929478in}{1.763712in}}%
\pgfpathlineto{\pgfqpoint{0.927447in}{1.773268in}}%
\pgfpathlineto{\pgfqpoint{0.930196in}{1.780527in}}%
\pgfpathlineto{\pgfqpoint{0.923511in}{1.785445in}}%
\pgfpathlineto{\pgfqpoint{0.924441in}{1.792459in}}%
\pgfpathlineto{\pgfqpoint{0.919091in}{1.796990in}}%
\pgfpathlineto{\pgfqpoint{0.923675in}{1.801792in}}%
\pgfpathlineto{\pgfqpoint{0.925085in}{1.807234in}}%
\pgfpathlineto{\pgfqpoint{0.922296in}{1.811132in}}%
\pgfpathlineto{\pgfqpoint{0.924401in}{1.816821in}}%
\pgfpathlineto{\pgfqpoint{0.931012in}{1.818300in}}%
\pgfpathlineto{\pgfqpoint{0.935266in}{1.825471in}}%
\pgfpathlineto{\pgfqpoint{0.943209in}{1.820987in}}%
\pgfpathlineto{\pgfqpoint{0.957265in}{1.830065in}}%
\pgfpathlineto{\pgfqpoint{0.961805in}{1.850104in}}%
\pgfpathlineto{\pgfqpoint{0.964504in}{1.853012in}}%
\pgfpathlineto{\pgfqpoint{0.970766in}{1.859237in}}%
\pgfpathlineto{\pgfqpoint{0.963651in}{1.874783in}}%
\pgfpathlineto{\pgfqpoint{0.970527in}{1.872842in}}%
\pgfpathlineto{\pgfqpoint{0.977017in}{1.878098in}}%
\pgfpathlineto{\pgfqpoint{0.979613in}{1.878334in}}%
\pgfpathlineto{\pgfqpoint{0.986109in}{1.871103in}}%
\pgfpathlineto{\pgfqpoint{0.999184in}{1.873285in}}%
\pgfpathlineto{\pgfqpoint{1.003780in}{1.877721in}}%
\pgfpathlineto{\pgfqpoint{1.011530in}{1.881323in}}%
\pgfpathlineto{\pgfqpoint{1.016847in}{1.872988in}}%
\pgfpathlineto{\pgfqpoint{1.014146in}{1.867281in}}%
\pgfpathlineto{\pgfqpoint{1.017377in}{1.856546in}}%
\pgfpathlineto{\pgfqpoint{1.016206in}{1.851908in}}%
\pgfpathlineto{\pgfqpoint{1.022384in}{1.834159in}}%
\pgfpathlineto{\pgfqpoint{1.026921in}{1.829095in}}%
\pgfpathlineto{\pgfqpoint{1.023830in}{1.814795in}}%
\pgfpathlineto{\pgfqpoint{1.028302in}{1.810400in}}%
\pgfpathlineto{\pgfqpoint{1.036558in}{1.807322in}}%
\pgfpathlineto{\pgfqpoint{1.040240in}{1.800998in}}%
\pgfpathlineto{\pgfqpoint{1.042216in}{1.785000in}}%
\pgfpathlineto{\pgfqpoint{1.041288in}{1.779426in}}%
\pgfpathlineto{\pgfqpoint{1.048288in}{1.770922in}}%
\pgfpathlineto{\pgfqpoint{1.050199in}{1.771971in}}%
\pgfpathlineto{\pgfqpoint{1.048049in}{1.761168in}}%
\pgfpathlineto{\pgfqpoint{1.016239in}{1.767806in}}%
\pgfpathlineto{\pgfqpoint{1.013627in}{1.754982in}}%
\pgfpathlineto{\pgfqpoint{1.010918in}{1.749688in}}%
\pgfpathlineto{\pgfqpoint{1.007249in}{1.733966in}}%
\pgfpathlineto{\pgfqpoint{0.999763in}{1.729917in}}%
\pgfpathlineto{\pgfqpoint{0.988761in}{1.726669in}}%
\pgfpathclose%
\pgfusepath{fill}%
\end{pgfscope}%
\begin{pgfscope}%
\pgfpathrectangle{\pgfqpoint{0.100000in}{0.100000in}}{\pgfqpoint{3.420221in}{2.189500in}}%
\pgfusepath{clip}%
\pgfsetbuttcap%
\pgfsetmiterjoin%
\definecolor{currentfill}{rgb}{0.000000,0.713725,0.643137}%
\pgfsetfillcolor{currentfill}%
\pgfsetlinewidth{0.000000pt}%
\definecolor{currentstroke}{rgb}{0.000000,0.000000,0.000000}%
\pgfsetstrokecolor{currentstroke}%
\pgfsetstrokeopacity{0.000000}%
\pgfsetdash{}{0pt}%
\pgfpathmoveto{\pgfqpoint{1.126531in}{1.283514in}}%
\pgfpathlineto{\pgfqpoint{1.121083in}{1.281150in}}%
\pgfpathlineto{\pgfqpoint{1.115011in}{1.273492in}}%
\pgfpathlineto{\pgfqpoint{1.112576in}{1.274715in}}%
\pgfpathlineto{\pgfqpoint{1.103888in}{1.268117in}}%
\pgfpathlineto{\pgfqpoint{1.095395in}{1.268094in}}%
\pgfpathlineto{\pgfqpoint{1.087072in}{1.252209in}}%
\pgfpathlineto{\pgfqpoint{1.080385in}{1.249515in}}%
\pgfpathlineto{\pgfqpoint{1.076953in}{1.245506in}}%
\pgfpathlineto{\pgfqpoint{1.052426in}{1.249751in}}%
\pgfpathlineto{\pgfqpoint{0.976668in}{1.263573in}}%
\pgfpathlineto{\pgfqpoint{0.957985in}{1.267733in}}%
\pgfpathlineto{\pgfqpoint{0.961440in}{1.287111in}}%
\pgfpathlineto{\pgfqpoint{0.967852in}{1.285754in}}%
\pgfpathlineto{\pgfqpoint{0.968978in}{1.291459in}}%
\pgfpathlineto{\pgfqpoint{0.975495in}{1.290822in}}%
\pgfpathlineto{\pgfqpoint{0.978565in}{1.309833in}}%
\pgfpathlineto{\pgfqpoint{0.984234in}{1.313214in}}%
\pgfpathlineto{\pgfqpoint{0.989491in}{1.321881in}}%
\pgfpathlineto{\pgfqpoint{0.986009in}{1.329158in}}%
\pgfpathlineto{\pgfqpoint{0.982076in}{1.332373in}}%
\pgfpathlineto{\pgfqpoint{0.982552in}{1.341497in}}%
\pgfpathlineto{\pgfqpoint{0.987984in}{1.348746in}}%
\pgfpathlineto{\pgfqpoint{0.995833in}{1.346850in}}%
\pgfpathlineto{\pgfqpoint{1.001796in}{1.349628in}}%
\pgfpathlineto{\pgfqpoint{1.003381in}{1.359811in}}%
\pgfpathlineto{\pgfqpoint{1.009974in}{1.365535in}}%
\pgfpathlineto{\pgfqpoint{1.013814in}{1.364877in}}%
\pgfpathlineto{\pgfqpoint{1.017964in}{1.370775in}}%
\pgfpathlineto{\pgfqpoint{1.026979in}{1.368132in}}%
\pgfpathlineto{\pgfqpoint{1.058683in}{1.362229in}}%
\pgfpathlineto{\pgfqpoint{1.051332in}{1.323810in}}%
\pgfpathlineto{\pgfqpoint{1.025360in}{1.328572in}}%
\pgfpathlineto{\pgfqpoint{1.018848in}{1.323304in}}%
\pgfpathlineto{\pgfqpoint{1.015420in}{1.303090in}}%
\pgfpathlineto{\pgfqpoint{1.082408in}{1.290979in}}%
\pgfpathclose%
\pgfusepath{fill}%
\end{pgfscope}%
\begin{pgfscope}%
\pgfpathrectangle{\pgfqpoint{0.100000in}{0.100000in}}{\pgfqpoint{3.420221in}{2.189500in}}%
\pgfusepath{clip}%
\pgfsetbuttcap%
\pgfsetmiterjoin%
\definecolor{currentfill}{rgb}{0.000000,0.458824,0.770588}%
\pgfsetfillcolor{currentfill}%
\pgfsetlinewidth{0.000000pt}%
\definecolor{currentstroke}{rgb}{0.000000,0.000000,0.000000}%
\pgfsetstrokecolor{currentstroke}%
\pgfsetstrokeopacity{0.000000}%
\pgfsetdash{}{0pt}%
\pgfpathmoveto{\pgfqpoint{1.997708in}{1.125075in}}%
\pgfpathlineto{\pgfqpoint{1.977702in}{1.125108in}}%
\pgfpathlineto{\pgfqpoint{1.976242in}{1.120881in}}%
\pgfpathlineto{\pgfqpoint{1.976133in}{1.094997in}}%
\pgfpathlineto{\pgfqpoint{1.953200in}{1.095151in}}%
\pgfpathlineto{\pgfqpoint{1.953098in}{1.082096in}}%
\pgfpathlineto{\pgfqpoint{1.941696in}{1.082230in}}%
\pgfpathlineto{\pgfqpoint{1.942192in}{1.125355in}}%
\pgfpathlineto{\pgfqpoint{1.944380in}{1.125349in}}%
\pgfpathlineto{\pgfqpoint{1.945166in}{1.180450in}}%
\pgfpathlineto{\pgfqpoint{1.996892in}{1.180023in}}%
\pgfpathlineto{\pgfqpoint{1.997645in}{1.153883in}}%
\pgfpathclose%
\pgfusepath{fill}%
\end{pgfscope}%
\begin{pgfscope}%
\pgfpathrectangle{\pgfqpoint{0.100000in}{0.100000in}}{\pgfqpoint{3.420221in}{2.189500in}}%
\pgfusepath{clip}%
\pgfsetbuttcap%
\pgfsetmiterjoin%
\definecolor{currentfill}{rgb}{0.000000,0.443137,0.778431}%
\pgfsetfillcolor{currentfill}%
\pgfsetlinewidth{0.000000pt}%
\definecolor{currentstroke}{rgb}{0.000000,0.000000,0.000000}%
\pgfsetstrokecolor{currentstroke}%
\pgfsetstrokeopacity{0.000000}%
\pgfsetdash{}{0pt}%
\pgfpathmoveto{\pgfqpoint{1.343039in}{1.957270in}}%
\pgfpathlineto{\pgfqpoint{1.337886in}{1.942708in}}%
\pgfpathlineto{\pgfqpoint{1.337684in}{1.932883in}}%
\pgfpathlineto{\pgfqpoint{1.339267in}{1.924153in}}%
\pgfpathlineto{\pgfqpoint{1.335317in}{1.914996in}}%
\pgfpathlineto{\pgfqpoint{1.337478in}{1.912255in}}%
\pgfpathlineto{\pgfqpoint{1.339766in}{1.904622in}}%
\pgfpathlineto{\pgfqpoint{1.298514in}{1.910176in}}%
\pgfpathlineto{\pgfqpoint{1.259726in}{1.916326in}}%
\pgfpathlineto{\pgfqpoint{1.259031in}{1.912046in}}%
\pgfpathlineto{\pgfqpoint{1.240781in}{1.915007in}}%
\pgfpathlineto{\pgfqpoint{1.240262in}{1.928370in}}%
\pgfpathlineto{\pgfqpoint{1.242759in}{1.930195in}}%
\pgfpathlineto{\pgfqpoint{1.246223in}{1.951816in}}%
\pgfpathlineto{\pgfqpoint{1.240895in}{1.956022in}}%
\pgfpathlineto{\pgfqpoint{1.234410in}{1.957089in}}%
\pgfpathlineto{\pgfqpoint{1.228479in}{1.961394in}}%
\pgfpathlineto{\pgfqpoint{1.230863in}{1.973266in}}%
\pgfpathlineto{\pgfqpoint{1.246036in}{1.979217in}}%
\pgfpathlineto{\pgfqpoint{1.249223in}{1.992063in}}%
\pgfpathlineto{\pgfqpoint{1.263993in}{1.991396in}}%
\pgfpathlineto{\pgfqpoint{1.267822in}{1.989603in}}%
\pgfpathlineto{\pgfqpoint{1.277851in}{1.993331in}}%
\pgfpathlineto{\pgfqpoint{1.284984in}{1.991326in}}%
\pgfpathlineto{\pgfqpoint{1.292940in}{1.991712in}}%
\pgfpathlineto{\pgfqpoint{1.296549in}{1.985918in}}%
\pgfpathlineto{\pgfqpoint{1.301793in}{1.978174in}}%
\pgfpathlineto{\pgfqpoint{1.311200in}{1.974108in}}%
\pgfpathlineto{\pgfqpoint{1.317972in}{1.973322in}}%
\pgfpathlineto{\pgfqpoint{1.321989in}{1.969961in}}%
\pgfpathlineto{\pgfqpoint{1.336774in}{1.969487in}}%
\pgfpathlineto{\pgfqpoint{1.343297in}{1.966187in}}%
\pgfpathclose%
\pgfusepath{fill}%
\end{pgfscope}%
\begin{pgfscope}%
\pgfpathrectangle{\pgfqpoint{0.100000in}{0.100000in}}{\pgfqpoint{3.420221in}{2.189500in}}%
\pgfusepath{clip}%
\pgfsetbuttcap%
\pgfsetmiterjoin%
\definecolor{currentfill}{rgb}{0.000000,0.341176,0.829412}%
\pgfsetfillcolor{currentfill}%
\pgfsetlinewidth{0.000000pt}%
\definecolor{currentstroke}{rgb}{0.000000,0.000000,0.000000}%
\pgfsetstrokecolor{currentstroke}%
\pgfsetstrokeopacity{0.000000}%
\pgfsetdash{}{0pt}%
\pgfpathmoveto{\pgfqpoint{1.903269in}{1.023911in}}%
\pgfpathlineto{\pgfqpoint{1.903647in}{1.046608in}}%
\pgfpathlineto{\pgfqpoint{1.903919in}{1.062991in}}%
\pgfpathlineto{\pgfqpoint{1.891950in}{1.063142in}}%
\pgfpathlineto{\pgfqpoint{1.892089in}{1.069712in}}%
\pgfpathlineto{\pgfqpoint{1.885669in}{1.069846in}}%
\pgfpathlineto{\pgfqpoint{1.885817in}{1.076349in}}%
\pgfpathlineto{\pgfqpoint{1.879345in}{1.076498in}}%
\pgfpathlineto{\pgfqpoint{1.879630in}{1.089519in}}%
\pgfpathlineto{\pgfqpoint{1.887590in}{1.092701in}}%
\pgfpathlineto{\pgfqpoint{1.886511in}{1.096393in}}%
\pgfpathlineto{\pgfqpoint{1.853928in}{1.096638in}}%
\pgfpathlineto{\pgfqpoint{1.854713in}{1.127012in}}%
\pgfpathlineto{\pgfqpoint{1.873540in}{1.126566in}}%
\pgfpathlineto{\pgfqpoint{1.942192in}{1.125355in}}%
\pgfpathlineto{\pgfqpoint{1.941696in}{1.082230in}}%
\pgfpathlineto{\pgfqpoint{1.953098in}{1.082096in}}%
\pgfpathlineto{\pgfqpoint{1.953200in}{1.095151in}}%
\pgfpathlineto{\pgfqpoint{1.976133in}{1.094997in}}%
\pgfpathlineto{\pgfqpoint{1.976242in}{1.120881in}}%
\pgfpathlineto{\pgfqpoint{1.977702in}{1.125108in}}%
\pgfpathlineto{\pgfqpoint{1.997708in}{1.125075in}}%
\pgfpathlineto{\pgfqpoint{2.002152in}{1.120857in}}%
\pgfpathlineto{\pgfqpoint{2.002136in}{1.100420in}}%
\pgfpathlineto{\pgfqpoint{2.001425in}{1.062254in}}%
\pgfpathlineto{\pgfqpoint{1.994936in}{1.062246in}}%
\pgfpathlineto{\pgfqpoint{1.994943in}{1.055704in}}%
\pgfpathlineto{\pgfqpoint{1.989501in}{1.055735in}}%
\pgfpathlineto{\pgfqpoint{1.985185in}{1.047448in}}%
\pgfpathlineto{\pgfqpoint{1.985943in}{1.036128in}}%
\pgfpathlineto{\pgfqpoint{1.973048in}{1.035501in}}%
\pgfpathlineto{\pgfqpoint{1.966270in}{1.032556in}}%
\pgfpathlineto{\pgfqpoint{1.962557in}{1.039509in}}%
\pgfpathlineto{\pgfqpoint{1.955518in}{1.039538in}}%
\pgfpathlineto{\pgfqpoint{1.929595in}{1.039864in}}%
\pgfpathlineto{\pgfqpoint{1.929375in}{1.023511in}}%
\pgfpathclose%
\pgfusepath{fill}%
\end{pgfscope}%
\begin{pgfscope}%
\pgfpathrectangle{\pgfqpoint{0.100000in}{0.100000in}}{\pgfqpoint{3.420221in}{2.189500in}}%
\pgfusepath{clip}%
\pgfsetbuttcap%
\pgfsetmiterjoin%
\definecolor{currentfill}{rgb}{0.000000,0.427451,0.786275}%
\pgfsetfillcolor{currentfill}%
\pgfsetlinewidth{0.000000pt}%
\definecolor{currentstroke}{rgb}{0.000000,0.000000,0.000000}%
\pgfsetstrokecolor{currentstroke}%
\pgfsetstrokeopacity{0.000000}%
\pgfsetdash{}{0pt}%
\pgfpathmoveto{\pgfqpoint{1.750793in}{1.672835in}}%
\pgfpathlineto{\pgfqpoint{1.752167in}{1.663762in}}%
\pgfpathlineto{\pgfqpoint{1.745114in}{1.666933in}}%
\pgfpathlineto{\pgfqpoint{1.740485in}{1.666348in}}%
\pgfpathlineto{\pgfqpoint{1.735918in}{1.671942in}}%
\pgfpathlineto{\pgfqpoint{1.712814in}{1.673209in}}%
\pgfpathlineto{\pgfqpoint{1.712700in}{1.671249in}}%
\pgfpathlineto{\pgfqpoint{1.676099in}{1.673547in}}%
\pgfpathlineto{\pgfqpoint{1.669646in}{1.673951in}}%
\pgfpathlineto{\pgfqpoint{1.671160in}{1.687049in}}%
\pgfpathlineto{\pgfqpoint{1.672963in}{1.713088in}}%
\pgfpathlineto{\pgfqpoint{1.674179in}{1.717315in}}%
\pgfpathlineto{\pgfqpoint{1.680165in}{1.715175in}}%
\pgfpathlineto{\pgfqpoint{1.690745in}{1.718949in}}%
\pgfpathlineto{\pgfqpoint{1.693792in}{1.716869in}}%
\pgfpathlineto{\pgfqpoint{1.699606in}{1.720933in}}%
\pgfpathlineto{\pgfqpoint{1.706651in}{1.716658in}}%
\pgfpathlineto{\pgfqpoint{1.714372in}{1.723836in}}%
\pgfpathlineto{\pgfqpoint{1.713940in}{1.726154in}}%
\pgfpathlineto{\pgfqpoint{1.752841in}{1.723989in}}%
\pgfpathlineto{\pgfqpoint{1.751380in}{1.697708in}}%
\pgfpathclose%
\pgfusepath{fill}%
\end{pgfscope}%
\begin{pgfscope}%
\pgfpathrectangle{\pgfqpoint{0.100000in}{0.100000in}}{\pgfqpoint{3.420221in}{2.189500in}}%
\pgfusepath{clip}%
\pgfsetbuttcap%
\pgfsetmiterjoin%
\definecolor{currentfill}{rgb}{0.000000,0.631373,0.684314}%
\pgfsetfillcolor{currentfill}%
\pgfsetlinewidth{0.000000pt}%
\definecolor{currentstroke}{rgb}{0.000000,0.000000,0.000000}%
\pgfsetstrokecolor{currentstroke}%
\pgfsetstrokeopacity{0.000000}%
\pgfsetdash{}{0pt}%
\pgfpathmoveto{\pgfqpoint{2.283848in}{0.706845in}}%
\pgfpathlineto{\pgfqpoint{2.282904in}{0.726637in}}%
\pgfpathlineto{\pgfqpoint{2.276217in}{0.726297in}}%
\pgfpathlineto{\pgfqpoint{2.275940in}{0.732891in}}%
\pgfpathlineto{\pgfqpoint{2.305255in}{0.734445in}}%
\pgfpathlineto{\pgfqpoint{2.314895in}{0.738933in}}%
\pgfpathlineto{\pgfqpoint{2.324590in}{0.740155in}}%
\pgfpathlineto{\pgfqpoint{2.325494in}{0.722112in}}%
\pgfpathlineto{\pgfqpoint{2.327148in}{0.712214in}}%
\pgfpathlineto{\pgfqpoint{2.321878in}{0.712241in}}%
\pgfpathlineto{\pgfqpoint{2.317666in}{0.708697in}}%
\pgfpathclose%
\pgfusepath{fill}%
\end{pgfscope}%
\begin{pgfscope}%
\pgfpathrectangle{\pgfqpoint{0.100000in}{0.100000in}}{\pgfqpoint{3.420221in}{2.189500in}}%
\pgfusepath{clip}%
\pgfsetbuttcap%
\pgfsetmiterjoin%
\definecolor{currentfill}{rgb}{0.000000,0.184314,0.907843}%
\pgfsetfillcolor{currentfill}%
\pgfsetlinewidth{0.000000pt}%
\definecolor{currentstroke}{rgb}{0.000000,0.000000,0.000000}%
\pgfsetstrokecolor{currentstroke}%
\pgfsetstrokeopacity{0.000000}%
\pgfsetdash{}{0pt}%
\pgfpathmoveto{\pgfqpoint{1.563625in}{0.696062in}}%
\pgfpathlineto{\pgfqpoint{1.522991in}{0.699845in}}%
\pgfpathlineto{\pgfqpoint{1.521440in}{0.709449in}}%
\pgfpathlineto{\pgfqpoint{1.507081in}{0.719306in}}%
\pgfpathlineto{\pgfqpoint{1.501013in}{0.717314in}}%
\pgfpathlineto{\pgfqpoint{1.503330in}{0.744262in}}%
\pgfpathlineto{\pgfqpoint{1.501299in}{0.744442in}}%
\pgfpathlineto{\pgfqpoint{1.504099in}{0.777089in}}%
\pgfpathlineto{\pgfqpoint{1.487214in}{0.778661in}}%
\pgfpathlineto{\pgfqpoint{1.490102in}{0.811357in}}%
\pgfpathlineto{\pgfqpoint{1.544712in}{0.806841in}}%
\pgfpathlineto{\pgfqpoint{1.577299in}{0.804493in}}%
\pgfpathlineto{\pgfqpoint{1.574471in}{0.771596in}}%
\pgfpathlineto{\pgfqpoint{1.569304in}{0.771922in}}%
\pgfpathclose%
\pgfusepath{fill}%
\end{pgfscope}%
\begin{pgfscope}%
\pgfpathrectangle{\pgfqpoint{0.100000in}{0.100000in}}{\pgfqpoint{3.420221in}{2.189500in}}%
\pgfusepath{clip}%
\pgfsetbuttcap%
\pgfsetmiterjoin%
\definecolor{currentfill}{rgb}{0.000000,0.286275,0.856863}%
\pgfsetfillcolor{currentfill}%
\pgfsetlinewidth{0.000000pt}%
\definecolor{currentstroke}{rgb}{0.000000,0.000000,0.000000}%
\pgfsetstrokecolor{currentstroke}%
\pgfsetstrokeopacity{0.000000}%
\pgfsetdash{}{0pt}%
\pgfpathmoveto{\pgfqpoint{2.311593in}{1.258160in}}%
\pgfpathlineto{\pgfqpoint{2.283868in}{1.257006in}}%
\pgfpathlineto{\pgfqpoint{2.282188in}{1.262303in}}%
\pgfpathlineto{\pgfqpoint{2.287079in}{1.271354in}}%
\pgfpathlineto{\pgfqpoint{2.278602in}{1.276244in}}%
\pgfpathlineto{\pgfqpoint{2.266972in}{1.279195in}}%
\pgfpathlineto{\pgfqpoint{2.261888in}{1.271989in}}%
\pgfpathlineto{\pgfqpoint{2.255131in}{1.275053in}}%
\pgfpathlineto{\pgfqpoint{2.251377in}{1.284903in}}%
\pgfpathlineto{\pgfqpoint{2.253049in}{1.287560in}}%
\pgfpathlineto{\pgfqpoint{2.250300in}{1.297612in}}%
\pgfpathlineto{\pgfqpoint{2.245852in}{1.303808in}}%
\pgfpathlineto{\pgfqpoint{2.237254in}{1.310167in}}%
\pgfpathlineto{\pgfqpoint{2.255948in}{1.310757in}}%
\pgfpathlineto{\pgfqpoint{2.257955in}{1.296255in}}%
\pgfpathlineto{\pgfqpoint{2.264066in}{1.294605in}}%
\pgfpathlineto{\pgfqpoint{2.274160in}{1.295043in}}%
\pgfpathlineto{\pgfqpoint{2.283406in}{1.302120in}}%
\pgfpathlineto{\pgfqpoint{2.282045in}{1.321421in}}%
\pgfpathlineto{\pgfqpoint{2.295169in}{1.322281in}}%
\pgfpathlineto{\pgfqpoint{2.317794in}{1.323777in}}%
\pgfpathlineto{\pgfqpoint{2.318753in}{1.310639in}}%
\pgfpathlineto{\pgfqpoint{2.341370in}{1.312070in}}%
\pgfpathlineto{\pgfqpoint{2.341983in}{1.302263in}}%
\pgfpathlineto{\pgfqpoint{2.335587in}{1.301823in}}%
\pgfpathlineto{\pgfqpoint{2.337622in}{1.266263in}}%
\pgfpathlineto{\pgfqpoint{2.317603in}{1.265127in}}%
\pgfpathlineto{\pgfqpoint{2.318114in}{1.258602in}}%
\pgfpathclose%
\pgfusepath{fill}%
\end{pgfscope}%
\begin{pgfscope}%
\pgfpathrectangle{\pgfqpoint{0.100000in}{0.100000in}}{\pgfqpoint{3.420221in}{2.189500in}}%
\pgfusepath{clip}%
\pgfsetbuttcap%
\pgfsetmiterjoin%
\definecolor{currentfill}{rgb}{0.000000,0.737255,0.631373}%
\pgfsetfillcolor{currentfill}%
\pgfsetlinewidth{0.000000pt}%
\definecolor{currentstroke}{rgb}{0.000000,0.000000,0.000000}%
\pgfsetstrokecolor{currentstroke}%
\pgfsetstrokeopacity{0.000000}%
\pgfsetdash{}{0pt}%
\pgfpathmoveto{\pgfqpoint{2.225649in}{0.829433in}}%
\pgfpathlineto{\pgfqpoint{2.187216in}{0.828233in}}%
\pgfpathlineto{\pgfqpoint{2.184673in}{0.829947in}}%
\pgfpathlineto{\pgfqpoint{2.182528in}{0.839749in}}%
\pgfpathlineto{\pgfqpoint{2.183006in}{0.842900in}}%
\pgfpathlineto{\pgfqpoint{2.192005in}{0.850284in}}%
\pgfpathlineto{\pgfqpoint{2.191227in}{0.857753in}}%
\pgfpathlineto{\pgfqpoint{2.224716in}{0.858265in}}%
\pgfpathclose%
\pgfusepath{fill}%
\end{pgfscope}%
\begin{pgfscope}%
\pgfpathrectangle{\pgfqpoint{0.100000in}{0.100000in}}{\pgfqpoint{3.420221in}{2.189500in}}%
\pgfusepath{clip}%
\pgfsetbuttcap%
\pgfsetmiterjoin%
\definecolor{currentfill}{rgb}{0.000000,0.266667,0.866667}%
\pgfsetfillcolor{currentfill}%
\pgfsetlinewidth{0.000000pt}%
\definecolor{currentstroke}{rgb}{0.000000,0.000000,0.000000}%
\pgfsetstrokecolor{currentstroke}%
\pgfsetstrokeopacity{0.000000}%
\pgfsetdash{}{0pt}%
\pgfpathmoveto{\pgfqpoint{1.642038in}{1.467184in}}%
\pgfpathlineto{\pgfqpoint{1.609647in}{1.469522in}}%
\pgfpathlineto{\pgfqpoint{1.611599in}{1.495513in}}%
\pgfpathlineto{\pgfqpoint{1.607001in}{1.495855in}}%
\pgfpathlineto{\pgfqpoint{1.608628in}{1.515823in}}%
\pgfpathlineto{\pgfqpoint{1.610602in}{1.522257in}}%
\pgfpathlineto{\pgfqpoint{1.612730in}{1.548174in}}%
\pgfpathlineto{\pgfqpoint{1.611155in}{1.548307in}}%
\pgfpathlineto{\pgfqpoint{1.613138in}{1.573904in}}%
\pgfpathlineto{\pgfqpoint{1.613528in}{1.589980in}}%
\pgfpathlineto{\pgfqpoint{1.660333in}{1.586435in}}%
\pgfpathlineto{\pgfqpoint{1.716764in}{1.582932in}}%
\pgfpathlineto{\pgfqpoint{1.716664in}{1.566578in}}%
\pgfpathlineto{\pgfqpoint{1.715183in}{1.540531in}}%
\pgfpathlineto{\pgfqpoint{1.714561in}{1.514444in}}%
\pgfpathlineto{\pgfqpoint{1.676865in}{1.516905in}}%
\pgfpathlineto{\pgfqpoint{1.675394in}{1.490861in}}%
\pgfpathlineto{\pgfqpoint{1.643919in}{1.493196in}}%
\pgfpathclose%
\pgfusepath{fill}%
\end{pgfscope}%
\begin{pgfscope}%
\pgfpathrectangle{\pgfqpoint{0.100000in}{0.100000in}}{\pgfqpoint{3.420221in}{2.189500in}}%
\pgfusepath{clip}%
\pgfsetbuttcap%
\pgfsetmiterjoin%
\definecolor{currentfill}{rgb}{0.000000,0.541176,0.729412}%
\pgfsetfillcolor{currentfill}%
\pgfsetlinewidth{0.000000pt}%
\definecolor{currentstroke}{rgb}{0.000000,0.000000,0.000000}%
\pgfsetstrokecolor{currentstroke}%
\pgfsetstrokeopacity{0.000000}%
\pgfsetdash{}{0pt}%
\pgfpathmoveto{\pgfqpoint{2.594271in}{1.512331in}}%
\pgfpathlineto{\pgfqpoint{2.571733in}{1.509011in}}%
\pgfpathlineto{\pgfqpoint{2.570134in}{1.513654in}}%
\pgfpathlineto{\pgfqpoint{2.567450in}{1.536947in}}%
\pgfpathlineto{\pgfqpoint{2.573929in}{1.537514in}}%
\pgfpathlineto{\pgfqpoint{2.570463in}{1.563673in}}%
\pgfpathlineto{\pgfqpoint{2.602224in}{1.567583in}}%
\pgfpathlineto{\pgfqpoint{2.605908in}{1.541387in}}%
\pgfpathlineto{\pgfqpoint{2.625557in}{1.544565in}}%
\pgfpathlineto{\pgfqpoint{2.629480in}{1.517925in}}%
\pgfpathclose%
\pgfusepath{fill}%
\end{pgfscope}%
\begin{pgfscope}%
\pgfpathrectangle{\pgfqpoint{0.100000in}{0.100000in}}{\pgfqpoint{3.420221in}{2.189500in}}%
\pgfusepath{clip}%
\pgfsetbuttcap%
\pgfsetmiterjoin%
\definecolor{currentfill}{rgb}{0.000000,0.333333,0.833333}%
\pgfsetfillcolor{currentfill}%
\pgfsetlinewidth{0.000000pt}%
\definecolor{currentstroke}{rgb}{0.000000,0.000000,0.000000}%
\pgfsetstrokecolor{currentstroke}%
\pgfsetstrokeopacity{0.000000}%
\pgfsetdash{}{0pt}%
\pgfpathmoveto{\pgfqpoint{3.103958in}{1.501884in}}%
\pgfpathlineto{\pgfqpoint{3.080185in}{1.514408in}}%
\pgfpathlineto{\pgfqpoint{3.086700in}{1.520262in}}%
\pgfpathlineto{\pgfqpoint{3.070632in}{1.530438in}}%
\pgfpathlineto{\pgfqpoint{3.083822in}{1.540551in}}%
\pgfpathlineto{\pgfqpoint{3.080593in}{1.545086in}}%
\pgfpathlineto{\pgfqpoint{3.084104in}{1.550087in}}%
\pgfpathlineto{\pgfqpoint{3.088106in}{1.549393in}}%
\pgfpathlineto{\pgfqpoint{3.097469in}{1.541665in}}%
\pgfpathlineto{\pgfqpoint{3.093860in}{1.537960in}}%
\pgfpathlineto{\pgfqpoint{3.101171in}{1.529477in}}%
\pgfpathlineto{\pgfqpoint{3.110703in}{1.535350in}}%
\pgfpathlineto{\pgfqpoint{3.118270in}{1.545029in}}%
\pgfpathlineto{\pgfqpoint{3.123503in}{1.538927in}}%
\pgfpathlineto{\pgfqpoint{3.121813in}{1.532210in}}%
\pgfpathlineto{\pgfqpoint{3.118536in}{1.530457in}}%
\pgfpathlineto{\pgfqpoint{3.119879in}{1.517769in}}%
\pgfpathlineto{\pgfqpoint{3.109921in}{1.506258in}}%
\pgfpathclose%
\pgfusepath{fill}%
\end{pgfscope}%
\begin{pgfscope}%
\pgfpathrectangle{\pgfqpoint{0.100000in}{0.100000in}}{\pgfqpoint{3.420221in}{2.189500in}}%
\pgfusepath{clip}%
\pgfsetbuttcap%
\pgfsetmiterjoin%
\definecolor{currentfill}{rgb}{0.000000,0.396078,0.801961}%
\pgfsetfillcolor{currentfill}%
\pgfsetlinewidth{0.000000pt}%
\definecolor{currentstroke}{rgb}{0.000000,0.000000,0.000000}%
\pgfsetstrokecolor{currentstroke}%
\pgfsetstrokeopacity{0.000000}%
\pgfsetdash{}{0pt}%
\pgfpathmoveto{\pgfqpoint{1.736668in}{0.675187in}}%
\pgfpathlineto{\pgfqpoint{1.744882in}{0.673335in}}%
\pgfpathlineto{\pgfqpoint{1.743511in}{0.641354in}}%
\pgfpathlineto{\pgfqpoint{1.721367in}{0.642463in}}%
\pgfpathlineto{\pgfqpoint{1.709602in}{0.643018in}}%
\pgfpathlineto{\pgfqpoint{1.711232in}{0.676357in}}%
\pgfpathclose%
\pgfusepath{fill}%
\end{pgfscope}%
\begin{pgfscope}%
\pgfpathrectangle{\pgfqpoint{0.100000in}{0.100000in}}{\pgfqpoint{3.420221in}{2.189500in}}%
\pgfusepath{clip}%
\pgfsetbuttcap%
\pgfsetmiterjoin%
\definecolor{currentfill}{rgb}{0.000000,0.600000,0.700000}%
\pgfsetfillcolor{currentfill}%
\pgfsetlinewidth{0.000000pt}%
\definecolor{currentstroke}{rgb}{0.000000,0.000000,0.000000}%
\pgfsetstrokecolor{currentstroke}%
\pgfsetstrokeopacity{0.000000}%
\pgfsetdash{}{0pt}%
\pgfpathmoveto{\pgfqpoint{2.125076in}{1.354142in}}%
\pgfpathlineto{\pgfqpoint{2.096079in}{1.353229in}}%
\pgfpathlineto{\pgfqpoint{2.094629in}{1.394208in}}%
\pgfpathlineto{\pgfqpoint{2.132169in}{1.395705in}}%
\pgfpathlineto{\pgfqpoint{2.134992in}{1.382551in}}%
\pgfpathlineto{\pgfqpoint{2.134358in}{1.377321in}}%
\pgfpathlineto{\pgfqpoint{2.124569in}{1.377019in}}%
\pgfpathclose%
\pgfusepath{fill}%
\end{pgfscope}%
\begin{pgfscope}%
\pgfpathrectangle{\pgfqpoint{0.100000in}{0.100000in}}{\pgfqpoint{3.420221in}{2.189500in}}%
\pgfusepath{clip}%
\pgfsetbuttcap%
\pgfsetmiterjoin%
\definecolor{currentfill}{rgb}{0.000000,0.321569,0.839216}%
\pgfsetfillcolor{currentfill}%
\pgfsetlinewidth{0.000000pt}%
\definecolor{currentstroke}{rgb}{0.000000,0.000000,0.000000}%
\pgfsetstrokecolor{currentstroke}%
\pgfsetstrokeopacity{0.000000}%
\pgfsetdash{}{0pt}%
\pgfpathmoveto{\pgfqpoint{2.958031in}{1.251815in}}%
\pgfpathlineto{\pgfqpoint{2.953851in}{1.250420in}}%
\pgfpathlineto{\pgfqpoint{2.946711in}{1.260814in}}%
\pgfpathlineto{\pgfqpoint{2.942362in}{1.261178in}}%
\pgfpathlineto{\pgfqpoint{2.936332in}{1.267861in}}%
\pgfpathlineto{\pgfqpoint{2.929214in}{1.265132in}}%
\pgfpathlineto{\pgfqpoint{2.926215in}{1.254415in}}%
\pgfpathlineto{\pgfqpoint{2.920458in}{1.248748in}}%
\pgfpathlineto{\pgfqpoint{2.918882in}{1.243428in}}%
\pgfpathlineto{\pgfqpoint{2.906425in}{1.251087in}}%
\pgfpathlineto{\pgfqpoint{2.905710in}{1.258672in}}%
\pgfpathlineto{\pgfqpoint{2.896904in}{1.260063in}}%
\pgfpathlineto{\pgfqpoint{2.892242in}{1.251746in}}%
\pgfpathlineto{\pgfqpoint{2.880254in}{1.241743in}}%
\pgfpathlineto{\pgfqpoint{2.870218in}{1.247340in}}%
\pgfpathlineto{\pgfqpoint{2.873323in}{1.256808in}}%
\pgfpathlineto{\pgfqpoint{2.883849in}{1.272609in}}%
\pgfpathlineto{\pgfqpoint{2.885302in}{1.278077in}}%
\pgfpathlineto{\pgfqpoint{2.886103in}{1.287105in}}%
\pgfpathlineto{\pgfqpoint{2.892673in}{1.294867in}}%
\pgfpathlineto{\pgfqpoint{2.891005in}{1.297370in}}%
\pgfpathlineto{\pgfqpoint{2.896398in}{1.307889in}}%
\pgfpathlineto{\pgfqpoint{2.896731in}{1.320111in}}%
\pgfpathlineto{\pgfqpoint{2.903722in}{1.318222in}}%
\pgfpathlineto{\pgfqpoint{2.908393in}{1.311904in}}%
\pgfpathlineto{\pgfqpoint{2.918478in}{1.310204in}}%
\pgfpathlineto{\pgfqpoint{2.922709in}{1.315961in}}%
\pgfpathlineto{\pgfqpoint{2.953750in}{1.300847in}}%
\pgfpathlineto{\pgfqpoint{2.953656in}{1.292142in}}%
\pgfpathlineto{\pgfqpoint{2.950652in}{1.288179in}}%
\pgfpathlineto{\pgfqpoint{2.962584in}{1.271232in}}%
\pgfpathlineto{\pgfqpoint{2.963734in}{1.263484in}}%
\pgfpathlineto{\pgfqpoint{2.959010in}{1.258807in}}%
\pgfpathclose%
\pgfusepath{fill}%
\end{pgfscope}%
\begin{pgfscope}%
\pgfpathrectangle{\pgfqpoint{0.100000in}{0.100000in}}{\pgfqpoint{3.420221in}{2.189500in}}%
\pgfusepath{clip}%
\pgfsetbuttcap%
\pgfsetmiterjoin%
\definecolor{currentfill}{rgb}{0.000000,0.396078,0.801961}%
\pgfsetfillcolor{currentfill}%
\pgfsetlinewidth{0.000000pt}%
\definecolor{currentstroke}{rgb}{0.000000,0.000000,0.000000}%
\pgfsetstrokecolor{currentstroke}%
\pgfsetstrokeopacity{0.000000}%
\pgfsetdash{}{0pt}%
\pgfpathmoveto{\pgfqpoint{1.721367in}{0.642463in}}%
\pgfpathlineto{\pgfqpoint{1.720734in}{0.626327in}}%
\pgfpathlineto{\pgfqpoint{1.691200in}{0.628101in}}%
\pgfpathlineto{\pgfqpoint{1.667531in}{0.629362in}}%
\pgfpathlineto{\pgfqpoint{1.669335in}{0.661125in}}%
\pgfpathlineto{\pgfqpoint{1.671009in}{0.689648in}}%
\pgfpathlineto{\pgfqpoint{1.704116in}{0.687823in}}%
\pgfpathlineto{\pgfqpoint{1.703529in}{0.676752in}}%
\pgfpathlineto{\pgfqpoint{1.711232in}{0.676357in}}%
\pgfpathlineto{\pgfqpoint{1.709602in}{0.643018in}}%
\pgfpathclose%
\pgfusepath{fill}%
\end{pgfscope}%
\begin{pgfscope}%
\pgfpathrectangle{\pgfqpoint{0.100000in}{0.100000in}}{\pgfqpoint{3.420221in}{2.189500in}}%
\pgfusepath{clip}%
\pgfsetbuttcap%
\pgfsetmiterjoin%
\definecolor{currentfill}{rgb}{0.000000,0.552941,0.723529}%
\pgfsetfillcolor{currentfill}%
\pgfsetlinewidth{0.000000pt}%
\definecolor{currentstroke}{rgb}{0.000000,0.000000,0.000000}%
\pgfsetstrokecolor{currentstroke}%
\pgfsetstrokeopacity{0.000000}%
\pgfsetdash{}{0pt}%
\pgfpathmoveto{\pgfqpoint{3.024580in}{0.981508in}}%
\pgfpathlineto{\pgfqpoint{2.986217in}{1.008864in}}%
\pgfpathlineto{\pgfqpoint{2.991145in}{1.020679in}}%
\pgfpathlineto{\pgfqpoint{2.997384in}{1.026746in}}%
\pgfpathlineto{\pgfqpoint{2.997200in}{1.040699in}}%
\pgfpathlineto{\pgfqpoint{2.989523in}{1.050329in}}%
\pgfpathlineto{\pgfqpoint{2.995435in}{1.052555in}}%
\pgfpathlineto{\pgfqpoint{3.013882in}{1.056385in}}%
\pgfpathlineto{\pgfqpoint{3.021614in}{1.049679in}}%
\pgfpathlineto{\pgfqpoint{3.032635in}{1.037417in}}%
\pgfpathlineto{\pgfqpoint{3.038870in}{1.050770in}}%
\pgfpathlineto{\pgfqpoint{3.065131in}{1.055667in}}%
\pgfpathlineto{\pgfqpoint{3.074132in}{1.040535in}}%
\pgfpathlineto{\pgfqpoint{3.079049in}{1.036839in}}%
\pgfpathlineto{\pgfqpoint{3.067736in}{1.021610in}}%
\pgfpathlineto{\pgfqpoint{3.063674in}{1.011637in}}%
\pgfpathlineto{\pgfqpoint{3.060186in}{0.990247in}}%
\pgfpathlineto{\pgfqpoint{3.047086in}{0.990456in}}%
\pgfpathlineto{\pgfqpoint{3.035211in}{0.987465in}}%
\pgfpathclose%
\pgfusepath{fill}%
\end{pgfscope}%
\begin{pgfscope}%
\pgfpathrectangle{\pgfqpoint{0.100000in}{0.100000in}}{\pgfqpoint{3.420221in}{2.189500in}}%
\pgfusepath{clip}%
\pgfsetbuttcap%
\pgfsetmiterjoin%
\definecolor{currentfill}{rgb}{0.000000,0.309804,0.845098}%
\pgfsetfillcolor{currentfill}%
\pgfsetlinewidth{0.000000pt}%
\definecolor{currentstroke}{rgb}{0.000000,0.000000,0.000000}%
\pgfsetstrokecolor{currentstroke}%
\pgfsetstrokeopacity{0.000000}%
\pgfsetdash{}{0pt}%
\pgfpathmoveto{\pgfqpoint{0.600695in}{1.418501in}}%
\pgfpathlineto{\pgfqpoint{0.593367in}{1.429827in}}%
\pgfpathlineto{\pgfqpoint{0.596010in}{1.444205in}}%
\pgfpathlineto{\pgfqpoint{0.592913in}{1.445380in}}%
\pgfpathlineto{\pgfqpoint{0.596496in}{1.461593in}}%
\pgfpathlineto{\pgfqpoint{0.603395in}{1.462495in}}%
\pgfpathlineto{\pgfqpoint{0.605179in}{1.469222in}}%
\pgfpathlineto{\pgfqpoint{0.580523in}{1.476145in}}%
\pgfpathlineto{\pgfqpoint{0.580600in}{1.478387in}}%
\pgfpathlineto{\pgfqpoint{0.567019in}{1.482099in}}%
\pgfpathlineto{\pgfqpoint{0.573802in}{1.506061in}}%
\pgfpathlineto{\pgfqpoint{0.579546in}{1.526027in}}%
\pgfpathlineto{\pgfqpoint{0.595667in}{1.582010in}}%
\pgfpathlineto{\pgfqpoint{0.609379in}{1.631416in}}%
\pgfpathlineto{\pgfqpoint{0.625927in}{1.689905in}}%
\pgfpathlineto{\pgfqpoint{0.660242in}{1.680338in}}%
\pgfpathlineto{\pgfqpoint{0.662174in}{1.679793in}}%
\pgfpathlineto{\pgfqpoint{0.650723in}{1.637863in}}%
\pgfpathlineto{\pgfqpoint{0.641370in}{1.605121in}}%
\pgfpathlineto{\pgfqpoint{0.642468in}{1.604815in}}%
\pgfpathlineto{\pgfqpoint{0.633521in}{1.572913in}}%
\pgfpathlineto{\pgfqpoint{0.632112in}{1.573260in}}%
\pgfpathlineto{\pgfqpoint{0.621773in}{1.535754in}}%
\pgfpathlineto{\pgfqpoint{0.662413in}{1.524753in}}%
\pgfpathlineto{\pgfqpoint{0.721574in}{1.509614in}}%
\pgfpathlineto{\pgfqpoint{0.724234in}{1.502842in}}%
\pgfpathlineto{\pgfqpoint{0.719519in}{1.490610in}}%
\pgfpathlineto{\pgfqpoint{0.719103in}{1.481002in}}%
\pgfpathlineto{\pgfqpoint{0.715753in}{1.475140in}}%
\pgfpathlineto{\pgfqpoint{0.705191in}{1.472132in}}%
\pgfpathlineto{\pgfqpoint{0.697228in}{1.465697in}}%
\pgfpathlineto{\pgfqpoint{0.697480in}{1.456624in}}%
\pgfpathlineto{\pgfqpoint{0.692853in}{1.453131in}}%
\pgfpathlineto{\pgfqpoint{0.691711in}{1.447049in}}%
\pgfpathlineto{\pgfqpoint{0.686294in}{1.446910in}}%
\pgfpathlineto{\pgfqpoint{0.626420in}{1.462678in}}%
\pgfpathlineto{\pgfqpoint{0.619091in}{1.454541in}}%
\pgfpathlineto{\pgfqpoint{0.617307in}{1.447881in}}%
\pgfpathlineto{\pgfqpoint{0.623655in}{1.446234in}}%
\pgfpathlineto{\pgfqpoint{0.614865in}{1.414658in}}%
\pgfpathclose%
\pgfusepath{fill}%
\end{pgfscope}%
\begin{pgfscope}%
\pgfpathrectangle{\pgfqpoint{0.100000in}{0.100000in}}{\pgfqpoint{3.420221in}{2.189500in}}%
\pgfusepath{clip}%
\pgfsetbuttcap%
\pgfsetmiterjoin%
\definecolor{currentfill}{rgb}{0.000000,0.156863,0.921569}%
\pgfsetfillcolor{currentfill}%
\pgfsetlinewidth{0.000000pt}%
\definecolor{currentstroke}{rgb}{0.000000,0.000000,0.000000}%
\pgfsetstrokecolor{currentstroke}%
\pgfsetstrokeopacity{0.000000}%
\pgfsetdash{}{0pt}%
\pgfpathmoveto{\pgfqpoint{1.795993in}{1.649473in}}%
\pgfpathlineto{\pgfqpoint{1.789530in}{1.649742in}}%
\pgfpathlineto{\pgfqpoint{1.790424in}{1.669387in}}%
\pgfpathlineto{\pgfqpoint{1.802581in}{1.668886in}}%
\pgfpathlineto{\pgfqpoint{1.803654in}{1.701784in}}%
\pgfpathlineto{\pgfqpoint{1.842625in}{1.700167in}}%
\pgfpathlineto{\pgfqpoint{1.849126in}{1.699917in}}%
\pgfpathlineto{\pgfqpoint{1.849082in}{1.693376in}}%
\pgfpathlineto{\pgfqpoint{1.848235in}{1.667121in}}%
\pgfpathlineto{\pgfqpoint{1.822463in}{1.668113in}}%
\pgfpathlineto{\pgfqpoint{1.821765in}{1.648660in}}%
\pgfpathclose%
\pgfusepath{fill}%
\end{pgfscope}%
\begin{pgfscope}%
\pgfpathrectangle{\pgfqpoint{0.100000in}{0.100000in}}{\pgfqpoint{3.420221in}{2.189500in}}%
\pgfusepath{clip}%
\pgfsetbuttcap%
\pgfsetmiterjoin%
\definecolor{currentfill}{rgb}{0.000000,0.478431,0.760784}%
\pgfsetfillcolor{currentfill}%
\pgfsetlinewidth{0.000000pt}%
\definecolor{currentstroke}{rgb}{0.000000,0.000000,0.000000}%
\pgfsetstrokecolor{currentstroke}%
\pgfsetstrokeopacity{0.000000}%
\pgfsetdash{}{0pt}%
\pgfpathmoveto{\pgfqpoint{3.097597in}{1.283948in}}%
\pgfpathlineto{\pgfqpoint{3.090969in}{1.283373in}}%
\pgfpathlineto{\pgfqpoint{3.085810in}{1.290511in}}%
\pgfpathlineto{\pgfqpoint{3.080491in}{1.292581in}}%
\pgfpathlineto{\pgfqpoint{3.076423in}{1.287943in}}%
\pgfpathlineto{\pgfqpoint{3.072485in}{1.292188in}}%
\pgfpathlineto{\pgfqpoint{3.065002in}{1.290428in}}%
\pgfpathlineto{\pgfqpoint{3.059156in}{1.293342in}}%
\pgfpathlineto{\pgfqpoint{3.054420in}{1.301559in}}%
\pgfpathlineto{\pgfqpoint{3.046999in}{1.308770in}}%
\pgfpathlineto{\pgfqpoint{3.052475in}{1.317800in}}%
\pgfpathlineto{\pgfqpoint{3.053997in}{1.325414in}}%
\pgfpathlineto{\pgfqpoint{3.062625in}{1.319062in}}%
\pgfpathlineto{\pgfqpoint{3.078224in}{1.320434in}}%
\pgfpathlineto{\pgfqpoint{3.089441in}{1.310395in}}%
\pgfpathlineto{\pgfqpoint{3.103819in}{1.305847in}}%
\pgfpathlineto{\pgfqpoint{3.101623in}{1.297138in}}%
\pgfpathlineto{\pgfqpoint{3.103061in}{1.290860in}}%
\pgfpathlineto{\pgfqpoint{3.100874in}{1.283825in}}%
\pgfpathclose%
\pgfusepath{fill}%
\end{pgfscope}%
\begin{pgfscope}%
\pgfpathrectangle{\pgfqpoint{0.100000in}{0.100000in}}{\pgfqpoint{3.420221in}{2.189500in}}%
\pgfusepath{clip}%
\pgfsetbuttcap%
\pgfsetmiterjoin%
\definecolor{currentfill}{rgb}{0.000000,0.305882,0.847059}%
\pgfsetfillcolor{currentfill}%
\pgfsetlinewidth{0.000000pt}%
\definecolor{currentstroke}{rgb}{0.000000,0.000000,0.000000}%
\pgfsetstrokecolor{currentstroke}%
\pgfsetstrokeopacity{0.000000}%
\pgfsetdash{}{0pt}%
\pgfpathmoveto{\pgfqpoint{1.842398in}{1.195912in}}%
\pgfpathlineto{\pgfqpoint{1.842202in}{1.182595in}}%
\pgfpathlineto{\pgfqpoint{1.796820in}{1.183949in}}%
\pgfpathlineto{\pgfqpoint{1.796636in}{1.190847in}}%
\pgfpathlineto{\pgfqpoint{1.798329in}{1.256189in}}%
\pgfpathlineto{\pgfqpoint{1.798902in}{1.269198in}}%
\pgfpathlineto{\pgfqpoint{1.831301in}{1.268083in}}%
\pgfpathlineto{\pgfqpoint{1.830928in}{1.248509in}}%
\pgfpathlineto{\pgfqpoint{1.830006in}{1.215843in}}%
\pgfpathlineto{\pgfqpoint{1.842977in}{1.215438in}}%
\pgfpathclose%
\pgfusepath{fill}%
\end{pgfscope}%
\begin{pgfscope}%
\pgfpathrectangle{\pgfqpoint{0.100000in}{0.100000in}}{\pgfqpoint{3.420221in}{2.189500in}}%
\pgfusepath{clip}%
\pgfsetbuttcap%
\pgfsetmiterjoin%
\definecolor{currentfill}{rgb}{0.000000,0.564706,0.717647}%
\pgfsetfillcolor{currentfill}%
\pgfsetlinewidth{0.000000pt}%
\definecolor{currentstroke}{rgb}{0.000000,0.000000,0.000000}%
\pgfsetstrokecolor{currentstroke}%
\pgfsetstrokeopacity{0.000000}%
\pgfsetdash{}{0pt}%
\pgfpathmoveto{\pgfqpoint{2.712617in}{0.722552in}}%
\pgfpathlineto{\pgfqpoint{2.688645in}{0.719806in}}%
\pgfpathlineto{\pgfqpoint{2.677939in}{0.718577in}}%
\pgfpathlineto{\pgfqpoint{2.676760in}{0.731907in}}%
\pgfpathlineto{\pgfqpoint{2.670020in}{0.731361in}}%
\pgfpathlineto{\pgfqpoint{2.668779in}{0.744530in}}%
\pgfpathlineto{\pgfqpoint{2.678232in}{0.745160in}}%
\pgfpathlineto{\pgfqpoint{2.682990in}{0.747945in}}%
\pgfpathlineto{\pgfqpoint{2.679480in}{0.754482in}}%
\pgfpathlineto{\pgfqpoint{2.679127in}{0.760020in}}%
\pgfpathlineto{\pgfqpoint{2.673171in}{0.763869in}}%
\pgfpathlineto{\pgfqpoint{2.668073in}{0.770155in}}%
\pgfpathlineto{\pgfqpoint{2.667007in}{0.781293in}}%
\pgfpathlineto{\pgfqpoint{2.710175in}{0.785759in}}%
\pgfpathlineto{\pgfqpoint{2.710622in}{0.781011in}}%
\pgfpathlineto{\pgfqpoint{2.718799in}{0.778604in}}%
\pgfpathlineto{\pgfqpoint{2.720848in}{0.765117in}}%
\pgfpathlineto{\pgfqpoint{2.725869in}{0.765672in}}%
\pgfpathlineto{\pgfqpoint{2.730768in}{0.762189in}}%
\pgfpathlineto{\pgfqpoint{2.732707in}{0.744358in}}%
\pgfpathlineto{\pgfqpoint{2.710553in}{0.741927in}}%
\pgfpathclose%
\pgfusepath{fill}%
\end{pgfscope}%
\begin{pgfscope}%
\pgfpathrectangle{\pgfqpoint{0.100000in}{0.100000in}}{\pgfqpoint{3.420221in}{2.189500in}}%
\pgfusepath{clip}%
\pgfsetbuttcap%
\pgfsetmiterjoin%
\definecolor{currentfill}{rgb}{0.000000,0.670588,0.664706}%
\pgfsetfillcolor{currentfill}%
\pgfsetlinewidth{0.000000pt}%
\definecolor{currentstroke}{rgb}{0.000000,0.000000,0.000000}%
\pgfsetstrokecolor{currentstroke}%
\pgfsetstrokeopacity{0.000000}%
\pgfsetdash{}{0pt}%
\pgfpathmoveto{\pgfqpoint{2.499503in}{1.794246in}}%
\pgfpathlineto{\pgfqpoint{2.495676in}{1.795842in}}%
\pgfpathlineto{\pgfqpoint{2.497735in}{1.808330in}}%
\pgfpathlineto{\pgfqpoint{2.501556in}{1.806171in}}%
\pgfpathlineto{\pgfqpoint{2.502894in}{1.797416in}}%
\pgfpathclose%
\pgfusepath{fill}%
\end{pgfscope}%
\begin{pgfscope}%
\pgfpathrectangle{\pgfqpoint{0.100000in}{0.100000in}}{\pgfqpoint{3.420221in}{2.189500in}}%
\pgfusepath{clip}%
\pgfsetbuttcap%
\pgfsetmiterjoin%
\definecolor{currentfill}{rgb}{0.000000,0.670588,0.664706}%
\pgfsetfillcolor{currentfill}%
\pgfsetlinewidth{0.000000pt}%
\definecolor{currentstroke}{rgb}{0.000000,0.000000,0.000000}%
\pgfsetstrokecolor{currentstroke}%
\pgfsetstrokeopacity{0.000000}%
\pgfsetdash{}{0pt}%
\pgfpathmoveto{\pgfqpoint{2.570147in}{1.806265in}}%
\pgfpathlineto{\pgfqpoint{2.567696in}{1.805968in}}%
\pgfpathlineto{\pgfqpoint{2.571775in}{1.774012in}}%
\pgfpathlineto{\pgfqpoint{2.546043in}{1.771263in}}%
\pgfpathlineto{\pgfqpoint{2.546753in}{1.764724in}}%
\pgfpathlineto{\pgfqpoint{2.540293in}{1.764119in}}%
\pgfpathlineto{\pgfqpoint{2.520879in}{1.762090in}}%
\pgfpathlineto{\pgfqpoint{2.520221in}{1.768684in}}%
\pgfpathlineto{\pgfqpoint{2.511485in}{1.767799in}}%
\pgfpathlineto{\pgfqpoint{2.515246in}{1.776884in}}%
\pgfpathlineto{\pgfqpoint{2.523329in}{1.781713in}}%
\pgfpathlineto{\pgfqpoint{2.527978in}{1.787082in}}%
\pgfpathlineto{\pgfqpoint{2.524637in}{1.790364in}}%
\pgfpathlineto{\pgfqpoint{2.522803in}{1.797575in}}%
\pgfpathlineto{\pgfqpoint{2.530766in}{1.808802in}}%
\pgfpathlineto{\pgfqpoint{2.537826in}{1.811700in}}%
\pgfpathlineto{\pgfqpoint{2.538969in}{1.815067in}}%
\pgfpathlineto{\pgfqpoint{2.556640in}{1.806603in}}%
\pgfpathlineto{\pgfqpoint{2.563256in}{1.808370in}}%
\pgfpathclose%
\pgfusepath{fill}%
\end{pgfscope}%
\begin{pgfscope}%
\pgfpathrectangle{\pgfqpoint{0.100000in}{0.100000in}}{\pgfqpoint{3.420221in}{2.189500in}}%
\pgfusepath{clip}%
\pgfsetbuttcap%
\pgfsetmiterjoin%
\definecolor{currentfill}{rgb}{0.000000,0.454902,0.772549}%
\pgfsetfillcolor{currentfill}%
\pgfsetlinewidth{0.000000pt}%
\definecolor{currentstroke}{rgb}{0.000000,0.000000,0.000000}%
\pgfsetstrokecolor{currentstroke}%
\pgfsetstrokeopacity{0.000000}%
\pgfsetdash{}{0pt}%
\pgfpathmoveto{\pgfqpoint{2.733239in}{0.693158in}}%
\pgfpathlineto{\pgfqpoint{2.741849in}{0.693692in}}%
\pgfpathlineto{\pgfqpoint{2.742990in}{0.688544in}}%
\pgfpathlineto{\pgfqpoint{2.735109in}{0.681941in}}%
\pgfpathlineto{\pgfqpoint{2.736761in}{0.676480in}}%
\pgfpathlineto{\pgfqpoint{2.731840in}{0.670565in}}%
\pgfpathlineto{\pgfqpoint{2.733268in}{0.667269in}}%
\pgfpathlineto{\pgfqpoint{2.725934in}{0.660710in}}%
\pgfpathlineto{\pgfqpoint{2.722442in}{0.648313in}}%
\pgfpathlineto{\pgfqpoint{2.711238in}{0.645564in}}%
\pgfpathlineto{\pgfqpoint{2.704068in}{0.646322in}}%
\pgfpathlineto{\pgfqpoint{2.698659in}{0.640531in}}%
\pgfpathlineto{\pgfqpoint{2.700892in}{0.635976in}}%
\pgfpathlineto{\pgfqpoint{2.687095in}{0.638220in}}%
\pgfpathlineto{\pgfqpoint{2.656858in}{0.635025in}}%
\pgfpathlineto{\pgfqpoint{2.648269in}{0.636349in}}%
\pgfpathlineto{\pgfqpoint{2.647191in}{0.646691in}}%
\pgfpathlineto{\pgfqpoint{2.630428in}{0.646459in}}%
\pgfpathlineto{\pgfqpoint{2.627862in}{0.674027in}}%
\pgfpathlineto{\pgfqpoint{2.621248in}{0.673377in}}%
\pgfpathlineto{\pgfqpoint{2.622170in}{0.663536in}}%
\pgfpathlineto{\pgfqpoint{2.598374in}{0.661370in}}%
\pgfpathlineto{\pgfqpoint{2.589728in}{0.656602in}}%
\pgfpathlineto{\pgfqpoint{2.597600in}{0.665053in}}%
\pgfpathlineto{\pgfqpoint{2.592887in}{0.675713in}}%
\pgfpathlineto{\pgfqpoint{2.596130in}{0.680767in}}%
\pgfpathlineto{\pgfqpoint{2.584454in}{0.680056in}}%
\pgfpathlineto{\pgfqpoint{2.582467in}{0.701816in}}%
\pgfpathlineto{\pgfqpoint{2.649003in}{0.709343in}}%
\pgfpathlineto{\pgfqpoint{2.653797in}{0.701334in}}%
\pgfpathlineto{\pgfqpoint{2.654690in}{0.696287in}}%
\pgfpathlineto{\pgfqpoint{2.660309in}{0.688587in}}%
\pgfpathlineto{\pgfqpoint{2.691790in}{0.690555in}}%
\pgfpathclose%
\pgfusepath{fill}%
\end{pgfscope}%
\begin{pgfscope}%
\pgfpathrectangle{\pgfqpoint{0.100000in}{0.100000in}}{\pgfqpoint{3.420221in}{2.189500in}}%
\pgfusepath{clip}%
\pgfsetbuttcap%
\pgfsetmiterjoin%
\definecolor{currentfill}{rgb}{0.000000,0.286275,0.856863}%
\pgfsetfillcolor{currentfill}%
\pgfsetlinewidth{0.000000pt}%
\definecolor{currentstroke}{rgb}{0.000000,0.000000,0.000000}%
\pgfsetstrokecolor{currentstroke}%
\pgfsetstrokeopacity{0.000000}%
\pgfsetdash{}{0pt}%
\pgfpathmoveto{\pgfqpoint{1.634094in}{1.862737in}}%
\pgfpathlineto{\pgfqpoint{1.634626in}{1.869267in}}%
\pgfpathlineto{\pgfqpoint{1.654170in}{1.867802in}}%
\pgfpathlineto{\pgfqpoint{1.653677in}{1.861242in}}%
\pgfpathlineto{\pgfqpoint{1.675386in}{1.859635in}}%
\pgfpathlineto{\pgfqpoint{1.674029in}{1.840117in}}%
\pgfpathlineto{\pgfqpoint{1.687055in}{1.839227in}}%
\pgfpathlineto{\pgfqpoint{1.687807in}{1.832577in}}%
\pgfpathlineto{\pgfqpoint{1.672808in}{1.822068in}}%
\pgfpathlineto{\pgfqpoint{1.670030in}{1.818285in}}%
\pgfpathlineto{\pgfqpoint{1.658917in}{1.813889in}}%
\pgfpathlineto{\pgfqpoint{1.645789in}{1.816340in}}%
\pgfpathlineto{\pgfqpoint{1.641297in}{1.819607in}}%
\pgfpathlineto{\pgfqpoint{1.635812in}{1.818838in}}%
\pgfpathlineto{\pgfqpoint{1.637187in}{1.836166in}}%
\pgfpathlineto{\pgfqpoint{1.634646in}{1.836364in}}%
\pgfpathlineto{\pgfqpoint{1.636456in}{1.862539in}}%
\pgfpathclose%
\pgfusepath{fill}%
\end{pgfscope}%
\begin{pgfscope}%
\pgfpathrectangle{\pgfqpoint{0.100000in}{0.100000in}}{\pgfqpoint{3.420221in}{2.189500in}}%
\pgfusepath{clip}%
\pgfsetbuttcap%
\pgfsetmiterjoin%
\definecolor{currentfill}{rgb}{0.000000,0.188235,0.905882}%
\pgfsetfillcolor{currentfill}%
\pgfsetlinewidth{0.000000pt}%
\definecolor{currentstroke}{rgb}{0.000000,0.000000,0.000000}%
\pgfsetstrokecolor{currentstroke}%
\pgfsetstrokeopacity{0.000000}%
\pgfsetdash{}{0pt}%
\pgfpathmoveto{\pgfqpoint{1.588122in}{1.235779in}}%
\pgfpathlineto{\pgfqpoint{1.588148in}{1.236253in}}%
\pgfpathlineto{\pgfqpoint{1.590598in}{1.268282in}}%
\pgfpathlineto{\pgfqpoint{1.623298in}{1.266030in}}%
\pgfpathlineto{\pgfqpoint{1.669769in}{1.262911in}}%
\pgfpathlineto{\pgfqpoint{1.667922in}{1.230313in}}%
\pgfpathlineto{\pgfqpoint{1.616100in}{1.233729in}}%
\pgfpathclose%
\pgfusepath{fill}%
\end{pgfscope}%
\begin{pgfscope}%
\pgfpathrectangle{\pgfqpoint{0.100000in}{0.100000in}}{\pgfqpoint{3.420221in}{2.189500in}}%
\pgfusepath{clip}%
\pgfsetbuttcap%
\pgfsetmiterjoin%
\definecolor{currentfill}{rgb}{0.000000,0.415686,0.792157}%
\pgfsetfillcolor{currentfill}%
\pgfsetlinewidth{0.000000pt}%
\definecolor{currentstroke}{rgb}{0.000000,0.000000,0.000000}%
\pgfsetstrokecolor{currentstroke}%
\pgfsetstrokeopacity{0.000000}%
\pgfsetdash{}{0pt}%
\pgfpathmoveto{\pgfqpoint{1.980877in}{1.594089in}}%
\pgfpathlineto{\pgfqpoint{1.980763in}{1.568157in}}%
\pgfpathlineto{\pgfqpoint{1.954833in}{1.568286in}}%
\pgfpathlineto{\pgfqpoint{1.917514in}{1.568759in}}%
\pgfpathlineto{\pgfqpoint{1.919739in}{1.572387in}}%
\pgfpathlineto{\pgfqpoint{1.919199in}{1.579302in}}%
\pgfpathlineto{\pgfqpoint{1.922357in}{1.581778in}}%
\pgfpathlineto{\pgfqpoint{1.921328in}{1.592176in}}%
\pgfpathlineto{\pgfqpoint{1.918445in}{1.598121in}}%
\pgfpathlineto{\pgfqpoint{1.918897in}{1.605279in}}%
\pgfpathlineto{\pgfqpoint{1.915178in}{1.608195in}}%
\pgfpathlineto{\pgfqpoint{1.914964in}{1.613242in}}%
\pgfpathlineto{\pgfqpoint{1.922914in}{1.613106in}}%
\pgfpathlineto{\pgfqpoint{1.944719in}{1.612797in}}%
\pgfpathlineto{\pgfqpoint{1.955173in}{1.612662in}}%
\pgfpathlineto{\pgfqpoint{1.954946in}{1.594456in}}%
\pgfpathclose%
\pgfusepath{fill}%
\end{pgfscope}%
\begin{pgfscope}%
\pgfpathrectangle{\pgfqpoint{0.100000in}{0.100000in}}{\pgfqpoint{3.420221in}{2.189500in}}%
\pgfusepath{clip}%
\pgfsetbuttcap%
\pgfsetmiterjoin%
\definecolor{currentfill}{rgb}{0.000000,0.403922,0.798039}%
\pgfsetfillcolor{currentfill}%
\pgfsetlinewidth{0.000000pt}%
\definecolor{currentstroke}{rgb}{0.000000,0.000000,0.000000}%
\pgfsetstrokecolor{currentstroke}%
\pgfsetstrokeopacity{0.000000}%
\pgfsetdash{}{0pt}%
\pgfpathmoveto{\pgfqpoint{3.247365in}{1.742708in}}%
\pgfpathlineto{\pgfqpoint{3.249530in}{1.728998in}}%
\pgfpathlineto{\pgfqpoint{3.252544in}{1.729704in}}%
\pgfpathlineto{\pgfqpoint{3.257139in}{1.718909in}}%
\pgfpathlineto{\pgfqpoint{3.261023in}{1.713932in}}%
\pgfpathlineto{\pgfqpoint{3.241932in}{1.709740in}}%
\pgfpathlineto{\pgfqpoint{3.206993in}{1.702326in}}%
\pgfpathlineto{\pgfqpoint{3.204082in}{1.716108in}}%
\pgfpathlineto{\pgfqpoint{3.199395in}{1.715204in}}%
\pgfpathlineto{\pgfqpoint{3.196586in}{1.729131in}}%
\pgfpathlineto{\pgfqpoint{3.203515in}{1.730389in}}%
\pgfpathlineto{\pgfqpoint{3.203559in}{1.741416in}}%
\pgfpathlineto{\pgfqpoint{3.211256in}{1.740781in}}%
\pgfpathlineto{\pgfqpoint{3.224432in}{1.744940in}}%
\pgfpathlineto{\pgfqpoint{3.228870in}{1.740212in}}%
\pgfpathlineto{\pgfqpoint{3.240088in}{1.744957in}}%
\pgfpathclose%
\pgfusepath{fill}%
\end{pgfscope}%
\begin{pgfscope}%
\pgfpathrectangle{\pgfqpoint{0.100000in}{0.100000in}}{\pgfqpoint{3.420221in}{2.189500in}}%
\pgfusepath{clip}%
\pgfsetbuttcap%
\pgfsetmiterjoin%
\definecolor{currentfill}{rgb}{0.000000,0.098039,0.950980}%
\pgfsetfillcolor{currentfill}%
\pgfsetlinewidth{0.000000pt}%
\definecolor{currentstroke}{rgb}{0.000000,0.000000,0.000000}%
\pgfsetstrokecolor{currentstroke}%
\pgfsetstrokeopacity{0.000000}%
\pgfsetdash{}{0pt}%
\pgfpathmoveto{\pgfqpoint{1.291743in}{0.500480in}}%
\pgfpathlineto{\pgfqpoint{1.285073in}{0.512288in}}%
\pgfpathlineto{\pgfqpoint{1.276957in}{0.517003in}}%
\pgfpathlineto{\pgfqpoint{1.278898in}{0.522806in}}%
\pgfpathlineto{\pgfqpoint{1.279302in}{0.533971in}}%
\pgfpathlineto{\pgfqpoint{1.282944in}{0.540962in}}%
\pgfpathlineto{\pgfqpoint{1.291103in}{0.534962in}}%
\pgfpathlineto{\pgfqpoint{1.302457in}{0.536912in}}%
\pgfpathlineto{\pgfqpoint{1.305524in}{0.535338in}}%
\pgfpathlineto{\pgfqpoint{1.304322in}{0.520853in}}%
\pgfpathlineto{\pgfqpoint{1.300140in}{0.514807in}}%
\pgfpathlineto{\pgfqpoint{1.303663in}{0.503422in}}%
\pgfpathlineto{\pgfqpoint{1.302828in}{0.496010in}}%
\pgfpathclose%
\pgfusepath{fill}%
\end{pgfscope}%
\begin{pgfscope}%
\pgfpathrectangle{\pgfqpoint{0.100000in}{0.100000in}}{\pgfqpoint{3.420221in}{2.189500in}}%
\pgfusepath{clip}%
\pgfsetbuttcap%
\pgfsetmiterjoin%
\definecolor{currentfill}{rgb}{0.000000,0.356863,0.821569}%
\pgfsetfillcolor{currentfill}%
\pgfsetlinewidth{0.000000pt}%
\definecolor{currentstroke}{rgb}{0.000000,0.000000,0.000000}%
\pgfsetstrokecolor{currentstroke}%
\pgfsetstrokeopacity{0.000000}%
\pgfsetdash{}{0pt}%
\pgfpathmoveto{\pgfqpoint{0.550262in}{1.034405in}}%
\pgfpathlineto{\pgfqpoint{0.542317in}{1.038738in}}%
\pgfpathlineto{\pgfqpoint{0.542487in}{1.048350in}}%
\pgfpathlineto{\pgfqpoint{0.549688in}{1.043318in}}%
\pgfpathclose%
\pgfusepath{fill}%
\end{pgfscope}%
\begin{pgfscope}%
\pgfpathrectangle{\pgfqpoint{0.100000in}{0.100000in}}{\pgfqpoint{3.420221in}{2.189500in}}%
\pgfusepath{clip}%
\pgfsetbuttcap%
\pgfsetmiterjoin%
\definecolor{currentfill}{rgb}{0.000000,0.356863,0.821569}%
\pgfsetfillcolor{currentfill}%
\pgfsetlinewidth{0.000000pt}%
\definecolor{currentstroke}{rgb}{0.000000,0.000000,0.000000}%
\pgfsetstrokecolor{currentstroke}%
\pgfsetstrokeopacity{0.000000}%
\pgfsetdash{}{0pt}%
\pgfpathmoveto{\pgfqpoint{0.596428in}{1.029491in}}%
\pgfpathlineto{\pgfqpoint{0.587828in}{1.043636in}}%
\pgfpathlineto{\pgfqpoint{0.580342in}{1.050545in}}%
\pgfpathlineto{\pgfqpoint{0.572858in}{1.062050in}}%
\pgfpathlineto{\pgfqpoint{0.567965in}{1.065988in}}%
\pgfpathlineto{\pgfqpoint{0.559942in}{1.063774in}}%
\pgfpathlineto{\pgfqpoint{0.553633in}{1.068000in}}%
\pgfpathlineto{\pgfqpoint{0.556704in}{1.074798in}}%
\pgfpathlineto{\pgfqpoint{0.554409in}{1.087001in}}%
\pgfpathlineto{\pgfqpoint{0.551535in}{1.091622in}}%
\pgfpathlineto{\pgfqpoint{0.539122in}{1.094536in}}%
\pgfpathlineto{\pgfqpoint{0.534420in}{1.093934in}}%
\pgfpathlineto{\pgfqpoint{0.513061in}{1.110390in}}%
\pgfpathlineto{\pgfqpoint{0.511586in}{1.120288in}}%
\pgfpathlineto{\pgfqpoint{0.501924in}{1.131217in}}%
\pgfpathlineto{\pgfqpoint{0.505599in}{1.136334in}}%
\pgfpathlineto{\pgfqpoint{0.514520in}{1.168905in}}%
\pgfpathlineto{\pgfqpoint{0.523917in}{1.164641in}}%
\pgfpathlineto{\pgfqpoint{0.524599in}{1.159354in}}%
\pgfpathlineto{\pgfqpoint{0.572078in}{1.146879in}}%
\pgfpathlineto{\pgfqpoint{0.620855in}{1.134593in}}%
\pgfpathlineto{\pgfqpoint{0.638838in}{1.205551in}}%
\pgfpathlineto{\pgfqpoint{0.661344in}{1.199703in}}%
\pgfpathlineto{\pgfqpoint{0.731142in}{1.182626in}}%
\pgfpathlineto{\pgfqpoint{0.750963in}{1.177996in}}%
\pgfpathlineto{\pgfqpoint{0.756481in}{1.177901in}}%
\pgfpathlineto{\pgfqpoint{0.803701in}{1.105168in}}%
\pgfpathlineto{\pgfqpoint{0.801658in}{1.096929in}}%
\pgfpathlineto{\pgfqpoint{0.804756in}{1.086851in}}%
\pgfpathlineto{\pgfqpoint{0.808809in}{1.081764in}}%
\pgfpathlineto{\pgfqpoint{0.809020in}{1.072788in}}%
\pgfpathlineto{\pgfqpoint{0.812720in}{1.060824in}}%
\pgfpathlineto{\pgfqpoint{0.822469in}{1.047375in}}%
\pgfpathlineto{\pgfqpoint{0.828314in}{1.042813in}}%
\pgfpathlineto{\pgfqpoint{0.853701in}{1.035118in}}%
\pgfpathlineto{\pgfqpoint{0.858871in}{1.039923in}}%
\pgfpathlineto{\pgfqpoint{0.871604in}{1.038223in}}%
\pgfpathlineto{\pgfqpoint{0.866788in}{1.014753in}}%
\pgfpathlineto{\pgfqpoint{0.857403in}{0.968918in}}%
\pgfpathlineto{\pgfqpoint{0.818995in}{0.976922in}}%
\pgfpathlineto{\pgfqpoint{0.820352in}{0.983335in}}%
\pgfpathlineto{\pgfqpoint{0.801314in}{0.987316in}}%
\pgfpathlineto{\pgfqpoint{0.794383in}{0.955399in}}%
\pgfpathlineto{\pgfqpoint{0.770439in}{0.960916in}}%
\pgfpathlineto{\pgfqpoint{0.768616in}{0.966832in}}%
\pgfpathlineto{\pgfqpoint{0.772920in}{0.978857in}}%
\pgfpathlineto{\pgfqpoint{0.772267in}{0.989060in}}%
\pgfpathlineto{\pgfqpoint{0.751952in}{0.995434in}}%
\pgfpathlineto{\pgfqpoint{0.689353in}{1.009685in}}%
\pgfpathlineto{\pgfqpoint{0.631624in}{1.023719in}}%
\pgfpathlineto{\pgfqpoint{0.618875in}{1.027275in}}%
\pgfpathlineto{\pgfqpoint{0.612074in}{1.033541in}}%
\pgfpathlineto{\pgfqpoint{0.603881in}{1.036716in}}%
\pgfpathlineto{\pgfqpoint{0.598753in}{1.034048in}}%
\pgfpathclose%
\pgfusepath{fill}%
\end{pgfscope}%
\begin{pgfscope}%
\pgfpathrectangle{\pgfqpoint{0.100000in}{0.100000in}}{\pgfqpoint{3.420221in}{2.189500in}}%
\pgfusepath{clip}%
\pgfsetbuttcap%
\pgfsetmiterjoin%
\definecolor{currentfill}{rgb}{0.000000,0.560784,0.719608}%
\pgfsetfillcolor{currentfill}%
\pgfsetlinewidth{0.000000pt}%
\definecolor{currentstroke}{rgb}{0.000000,0.000000,0.000000}%
\pgfsetstrokecolor{currentstroke}%
\pgfsetstrokeopacity{0.000000}%
\pgfsetdash{}{0pt}%
\pgfpathmoveto{\pgfqpoint{2.535789in}{0.737656in}}%
\pgfpathlineto{\pgfqpoint{2.548753in}{0.738972in}}%
\pgfpathlineto{\pgfqpoint{2.547816in}{0.748742in}}%
\pgfpathlineto{\pgfqpoint{2.551071in}{0.749076in}}%
\pgfpathlineto{\pgfqpoint{2.548906in}{0.772308in}}%
\pgfpathlineto{\pgfqpoint{2.550966in}{0.779031in}}%
\pgfpathlineto{\pgfqpoint{2.557495in}{0.779725in}}%
\pgfpathlineto{\pgfqpoint{2.558209in}{0.773227in}}%
\pgfpathlineto{\pgfqpoint{2.577747in}{0.775324in}}%
\pgfpathlineto{\pgfqpoint{2.577028in}{0.781586in}}%
\pgfpathlineto{\pgfqpoint{2.583624in}{0.781965in}}%
\pgfpathlineto{\pgfqpoint{2.584821in}{0.776009in}}%
\pgfpathlineto{\pgfqpoint{2.590829in}{0.776617in}}%
\pgfpathlineto{\pgfqpoint{2.591429in}{0.770084in}}%
\pgfpathlineto{\pgfqpoint{2.599935in}{0.770949in}}%
\pgfpathlineto{\pgfqpoint{2.600366in}{0.763774in}}%
\pgfpathlineto{\pgfqpoint{2.596171in}{0.750656in}}%
\pgfpathlineto{\pgfqpoint{2.570723in}{0.748090in}}%
\pgfpathlineto{\pgfqpoint{2.568235in}{0.741184in}}%
\pgfpathlineto{\pgfqpoint{2.572640in}{0.700905in}}%
\pgfpathlineto{\pgfqpoint{2.540291in}{0.697885in}}%
\pgfpathlineto{\pgfqpoint{2.539481in}{0.698260in}}%
\pgfpathclose%
\pgfusepath{fill}%
\end{pgfscope}%
\begin{pgfscope}%
\pgfpathrectangle{\pgfqpoint{0.100000in}{0.100000in}}{\pgfqpoint{3.420221in}{2.189500in}}%
\pgfusepath{clip}%
\pgfsetbuttcap%
\pgfsetmiterjoin%
\definecolor{currentfill}{rgb}{0.000000,0.313725,0.843137}%
\pgfsetfillcolor{currentfill}%
\pgfsetlinewidth{0.000000pt}%
\definecolor{currentstroke}{rgb}{0.000000,0.000000,0.000000}%
\pgfsetstrokecolor{currentstroke}%
\pgfsetstrokeopacity{0.000000}%
\pgfsetdash{}{0pt}%
\pgfpathmoveto{\pgfqpoint{2.299184in}{1.015785in}}%
\pgfpathlineto{\pgfqpoint{2.306483in}{1.018299in}}%
\pgfpathlineto{\pgfqpoint{2.304901in}{1.024524in}}%
\pgfpathlineto{\pgfqpoint{2.312832in}{1.024146in}}%
\pgfpathlineto{\pgfqpoint{2.319455in}{1.032189in}}%
\pgfpathlineto{\pgfqpoint{2.331022in}{1.032193in}}%
\pgfpathlineto{\pgfqpoint{2.338651in}{1.028377in}}%
\pgfpathlineto{\pgfqpoint{2.340008in}{1.015594in}}%
\pgfpathlineto{\pgfqpoint{2.357760in}{1.016259in}}%
\pgfpathlineto{\pgfqpoint{2.358813in}{0.986031in}}%
\pgfpathlineto{\pgfqpoint{2.349336in}{0.985386in}}%
\pgfpathlineto{\pgfqpoint{2.350032in}{0.974184in}}%
\pgfpathlineto{\pgfqpoint{2.353322in}{0.974388in}}%
\pgfpathlineto{\pgfqpoint{2.354609in}{0.954784in}}%
\pgfpathlineto{\pgfqpoint{2.358247in}{0.948440in}}%
\pgfpathlineto{\pgfqpoint{2.343730in}{0.951946in}}%
\pgfpathlineto{\pgfqpoint{2.299118in}{0.949428in}}%
\pgfpathlineto{\pgfqpoint{2.292761in}{0.945724in}}%
\pgfpathlineto{\pgfqpoint{2.293243in}{0.939346in}}%
\pgfpathlineto{\pgfqpoint{2.287201in}{0.938981in}}%
\pgfpathlineto{\pgfqpoint{2.276314in}{0.946045in}}%
\pgfpathlineto{\pgfqpoint{2.274877in}{0.952811in}}%
\pgfpathlineto{\pgfqpoint{2.281796in}{0.957018in}}%
\pgfpathlineto{\pgfqpoint{2.284235in}{0.969669in}}%
\pgfpathlineto{\pgfqpoint{2.294553in}{0.975738in}}%
\pgfpathlineto{\pgfqpoint{2.291435in}{0.980767in}}%
\pgfpathlineto{\pgfqpoint{2.296281in}{0.984739in}}%
\pgfpathlineto{\pgfqpoint{2.298417in}{0.991792in}}%
\pgfpathlineto{\pgfqpoint{2.302901in}{0.991928in}}%
\pgfpathlineto{\pgfqpoint{2.304103in}{0.999635in}}%
\pgfpathlineto{\pgfqpoint{2.299916in}{1.001964in}}%
\pgfpathlineto{\pgfqpoint{2.303383in}{1.012066in}}%
\pgfpathclose%
\pgfusepath{fill}%
\end{pgfscope}%
\begin{pgfscope}%
\pgfpathrectangle{\pgfqpoint{0.100000in}{0.100000in}}{\pgfqpoint{3.420221in}{2.189500in}}%
\pgfusepath{clip}%
\pgfsetbuttcap%
\pgfsetmiterjoin%
\definecolor{currentfill}{rgb}{0.000000,0.874510,0.562745}%
\pgfsetfillcolor{currentfill}%
\pgfsetlinewidth{0.000000pt}%
\definecolor{currentstroke}{rgb}{0.000000,0.000000,0.000000}%
\pgfsetstrokecolor{currentstroke}%
\pgfsetstrokeopacity{0.000000}%
\pgfsetdash{}{0pt}%
\pgfpathmoveto{\pgfqpoint{2.684730in}{1.187635in}}%
\pgfpathlineto{\pgfqpoint{2.674932in}{1.192333in}}%
\pgfpathlineto{\pgfqpoint{2.668623in}{1.192429in}}%
\pgfpathlineto{\pgfqpoint{2.661595in}{1.201267in}}%
\pgfpathlineto{\pgfqpoint{2.659711in}{1.205992in}}%
\pgfpathlineto{\pgfqpoint{2.663477in}{1.215339in}}%
\pgfpathlineto{\pgfqpoint{2.668925in}{1.220227in}}%
\pgfpathlineto{\pgfqpoint{2.673255in}{1.229010in}}%
\pgfpathlineto{\pgfqpoint{2.678833in}{1.229736in}}%
\pgfpathlineto{\pgfqpoint{2.683114in}{1.239099in}}%
\pgfpathlineto{\pgfqpoint{2.682851in}{1.245988in}}%
\pgfpathlineto{\pgfqpoint{2.692313in}{1.249102in}}%
\pgfpathlineto{\pgfqpoint{2.697141in}{1.246059in}}%
\pgfpathlineto{\pgfqpoint{2.707210in}{1.247120in}}%
\pgfpathlineto{\pgfqpoint{2.711636in}{1.239856in}}%
\pgfpathlineto{\pgfqpoint{2.704471in}{1.237923in}}%
\pgfpathlineto{\pgfqpoint{2.695042in}{1.227310in}}%
\pgfpathlineto{\pgfqpoint{2.696544in}{1.221009in}}%
\pgfpathlineto{\pgfqpoint{2.706346in}{1.218031in}}%
\pgfpathlineto{\pgfqpoint{2.716523in}{1.210261in}}%
\pgfpathlineto{\pgfqpoint{2.708027in}{1.210595in}}%
\pgfpathlineto{\pgfqpoint{2.707104in}{1.201613in}}%
\pgfpathlineto{\pgfqpoint{2.698563in}{1.199999in}}%
\pgfpathlineto{\pgfqpoint{2.691181in}{1.194518in}}%
\pgfpathlineto{\pgfqpoint{2.688434in}{1.196590in}}%
\pgfpathlineto{\pgfqpoint{2.682891in}{1.192361in}}%
\pgfpathclose%
\pgfusepath{fill}%
\end{pgfscope}%
\begin{pgfscope}%
\pgfpathrectangle{\pgfqpoint{0.100000in}{0.100000in}}{\pgfqpoint{3.420221in}{2.189500in}}%
\pgfusepath{clip}%
\pgfsetbuttcap%
\pgfsetmiterjoin%
\definecolor{currentfill}{rgb}{0.000000,0.662745,0.668627}%
\pgfsetfillcolor{currentfill}%
\pgfsetlinewidth{0.000000pt}%
\definecolor{currentstroke}{rgb}{0.000000,0.000000,0.000000}%
\pgfsetstrokecolor{currentstroke}%
\pgfsetstrokeopacity{0.000000}%
\pgfsetdash{}{0pt}%
\pgfpathmoveto{\pgfqpoint{2.969786in}{0.388022in}}%
\pgfpathlineto{\pgfqpoint{2.975639in}{0.353660in}}%
\pgfpathlineto{\pgfqpoint{2.943450in}{0.348492in}}%
\pgfpathlineto{\pgfqpoint{2.937784in}{0.352335in}}%
\pgfpathlineto{\pgfqpoint{2.921244in}{0.352347in}}%
\pgfpathlineto{\pgfqpoint{2.913902in}{0.355692in}}%
\pgfpathlineto{\pgfqpoint{2.909095in}{0.365388in}}%
\pgfpathlineto{\pgfqpoint{2.904444in}{0.381117in}}%
\pgfpathlineto{\pgfqpoint{2.901124in}{0.386824in}}%
\pgfpathlineto{\pgfqpoint{2.892604in}{0.393524in}}%
\pgfpathlineto{\pgfqpoint{2.884029in}{0.393589in}}%
\pgfpathlineto{\pgfqpoint{2.879454in}{0.404637in}}%
\pgfpathlineto{\pgfqpoint{2.882883in}{0.406599in}}%
\pgfpathlineto{\pgfqpoint{2.883802in}{0.414752in}}%
\pgfpathlineto{\pgfqpoint{2.917346in}{0.419570in}}%
\pgfpathlineto{\pgfqpoint{2.920396in}{0.400172in}}%
\pgfpathlineto{\pgfqpoint{2.940159in}{0.403438in}}%
\pgfpathlineto{\pgfqpoint{2.943425in}{0.383453in}}%
\pgfpathclose%
\pgfusepath{fill}%
\end{pgfscope}%
\begin{pgfscope}%
\pgfpathrectangle{\pgfqpoint{0.100000in}{0.100000in}}{\pgfqpoint{3.420221in}{2.189500in}}%
\pgfusepath{clip}%
\pgfsetbuttcap%
\pgfsetmiterjoin%
\definecolor{currentfill}{rgb}{0.000000,0.239216,0.880392}%
\pgfsetfillcolor{currentfill}%
\pgfsetlinewidth{0.000000pt}%
\definecolor{currentstroke}{rgb}{0.000000,0.000000,0.000000}%
\pgfsetstrokecolor{currentstroke}%
\pgfsetstrokeopacity{0.000000}%
\pgfsetdash{}{0pt}%
\pgfpathmoveto{\pgfqpoint{0.487010in}{1.245847in}}%
\pgfpathlineto{\pgfqpoint{0.449032in}{1.257001in}}%
\pgfpathlineto{\pgfqpoint{0.420817in}{1.265677in}}%
\pgfpathlineto{\pgfqpoint{0.415996in}{1.274234in}}%
\pgfpathlineto{\pgfqpoint{0.415917in}{1.283588in}}%
\pgfpathlineto{\pgfqpoint{0.412925in}{1.285488in}}%
\pgfpathlineto{\pgfqpoint{0.408109in}{1.300521in}}%
\pgfpathlineto{\pgfqpoint{0.403327in}{1.305276in}}%
\pgfpathlineto{\pgfqpoint{0.400268in}{1.312347in}}%
\pgfpathlineto{\pgfqpoint{0.401670in}{1.333109in}}%
\pgfpathlineto{\pgfqpoint{0.408425in}{1.332949in}}%
\pgfpathlineto{\pgfqpoint{0.412529in}{1.336697in}}%
\pgfpathlineto{\pgfqpoint{0.417187in}{1.345870in}}%
\pgfpathlineto{\pgfqpoint{0.415633in}{1.356407in}}%
\pgfpathlineto{\pgfqpoint{0.402305in}{1.362178in}}%
\pgfpathlineto{\pgfqpoint{0.397137in}{1.369199in}}%
\pgfpathlineto{\pgfqpoint{0.394930in}{1.376292in}}%
\pgfpathlineto{\pgfqpoint{0.395314in}{1.382459in}}%
\pgfpathlineto{\pgfqpoint{0.405350in}{1.381594in}}%
\pgfpathlineto{\pgfqpoint{0.405786in}{1.393067in}}%
\pgfpathlineto{\pgfqpoint{0.407957in}{1.397700in}}%
\pgfpathlineto{\pgfqpoint{0.415193in}{1.399143in}}%
\pgfpathlineto{\pgfqpoint{0.423567in}{1.394752in}}%
\pgfpathlineto{\pgfqpoint{0.427638in}{1.395869in}}%
\pgfpathlineto{\pgfqpoint{0.449934in}{1.388939in}}%
\pgfpathlineto{\pgfqpoint{0.451367in}{1.381255in}}%
\pgfpathlineto{\pgfqpoint{0.446735in}{1.374396in}}%
\pgfpathlineto{\pgfqpoint{0.446758in}{1.364592in}}%
\pgfpathlineto{\pgfqpoint{0.456423in}{1.361584in}}%
\pgfpathlineto{\pgfqpoint{0.456099in}{1.354724in}}%
\pgfpathlineto{\pgfqpoint{0.451650in}{1.344932in}}%
\pgfpathlineto{\pgfqpoint{0.455080in}{1.336918in}}%
\pgfpathlineto{\pgfqpoint{0.465872in}{1.326215in}}%
\pgfpathlineto{\pgfqpoint{0.478787in}{1.302760in}}%
\pgfpathlineto{\pgfqpoint{0.475871in}{1.291131in}}%
\pgfpathlineto{\pgfqpoint{0.471713in}{1.291234in}}%
\pgfpathlineto{\pgfqpoint{0.468216in}{1.280235in}}%
\pgfpathlineto{\pgfqpoint{0.473419in}{1.268086in}}%
\pgfpathlineto{\pgfqpoint{0.477913in}{1.262471in}}%
\pgfpathlineto{\pgfqpoint{0.482060in}{1.260928in}}%
\pgfpathclose%
\pgfusepath{fill}%
\end{pgfscope}%
\begin{pgfscope}%
\pgfpathrectangle{\pgfqpoint{0.100000in}{0.100000in}}{\pgfqpoint{3.420221in}{2.189500in}}%
\pgfusepath{clip}%
\pgfsetbuttcap%
\pgfsetmiterjoin%
\definecolor{currentfill}{rgb}{0.000000,0.454902,0.772549}%
\pgfsetfillcolor{currentfill}%
\pgfsetlinewidth{0.000000pt}%
\definecolor{currentstroke}{rgb}{0.000000,0.000000,0.000000}%
\pgfsetstrokecolor{currentstroke}%
\pgfsetstrokeopacity{0.000000}%
\pgfsetdash{}{0pt}%
\pgfpathmoveto{\pgfqpoint{1.030854in}{0.205095in}}%
\pgfpathlineto{\pgfqpoint{1.029340in}{0.205932in}}%
\pgfpathlineto{\pgfqpoint{1.027942in}{0.208695in}}%
\pgfpathlineto{\pgfqpoint{1.022440in}{0.208302in}}%
\pgfpathlineto{\pgfqpoint{1.020838in}{0.206249in}}%
\pgfpathlineto{\pgfqpoint{1.019070in}{0.205564in}}%
\pgfpathlineto{\pgfqpoint{1.017424in}{0.207836in}}%
\pgfpathlineto{\pgfqpoint{1.013767in}{0.207633in}}%
\pgfpathlineto{\pgfqpoint{1.015636in}{0.210326in}}%
\pgfpathlineto{\pgfqpoint{1.017925in}{0.210430in}}%
\pgfpathlineto{\pgfqpoint{1.017192in}{0.212625in}}%
\pgfpathlineto{\pgfqpoint{1.017205in}{0.215483in}}%
\pgfpathlineto{\pgfqpoint{1.018035in}{0.217465in}}%
\pgfpathlineto{\pgfqpoint{1.016683in}{0.219475in}}%
\pgfpathlineto{\pgfqpoint{1.017515in}{0.220764in}}%
\pgfpathlineto{\pgfqpoint{1.020252in}{0.221635in}}%
\pgfpathlineto{\pgfqpoint{1.020823in}{0.224894in}}%
\pgfpathlineto{\pgfqpoint{1.018000in}{0.224255in}}%
\pgfpathlineto{\pgfqpoint{1.018137in}{0.228805in}}%
\pgfpathlineto{\pgfqpoint{1.016655in}{0.231673in}}%
\pgfpathlineto{\pgfqpoint{1.017305in}{0.234954in}}%
\pgfpathlineto{\pgfqpoint{1.016905in}{0.236634in}}%
\pgfpathlineto{\pgfqpoint{1.015161in}{0.237195in}}%
\pgfpathlineto{\pgfqpoint{1.011655in}{0.239906in}}%
\pgfpathlineto{\pgfqpoint{1.011168in}{0.242213in}}%
\pgfpathlineto{\pgfqpoint{1.012151in}{0.243324in}}%
\pgfpathlineto{\pgfqpoint{1.011072in}{0.245058in}}%
\pgfpathlineto{\pgfqpoint{1.011824in}{0.247074in}}%
\pgfpathlineto{\pgfqpoint{1.011543in}{0.252429in}}%
\pgfpathlineto{\pgfqpoint{1.012491in}{0.252934in}}%
\pgfpathlineto{\pgfqpoint{1.013650in}{0.256637in}}%
\pgfpathlineto{\pgfqpoint{1.011675in}{0.256425in}}%
\pgfpathlineto{\pgfqpoint{1.011893in}{0.259135in}}%
\pgfpathlineto{\pgfqpoint{1.013770in}{0.267295in}}%
\pgfpathlineto{\pgfqpoint{1.014497in}{0.272612in}}%
\pgfpathlineto{\pgfqpoint{1.012330in}{0.271944in}}%
\pgfpathlineto{\pgfqpoint{1.012623in}{0.269584in}}%
\pgfpathlineto{\pgfqpoint{1.011225in}{0.269237in}}%
\pgfpathlineto{\pgfqpoint{1.011284in}{0.266706in}}%
\pgfpathlineto{\pgfqpoint{1.009509in}{0.260973in}}%
\pgfpathlineto{\pgfqpoint{1.009122in}{0.255073in}}%
\pgfpathlineto{\pgfqpoint{1.007799in}{0.250840in}}%
\pgfpathlineto{\pgfqpoint{1.006792in}{0.245540in}}%
\pgfpathlineto{\pgfqpoint{1.004456in}{0.243117in}}%
\pgfpathlineto{\pgfqpoint{1.002364in}{0.246675in}}%
\pgfpathlineto{\pgfqpoint{1.002660in}{0.249465in}}%
\pgfpathlineto{\pgfqpoint{1.000784in}{0.250644in}}%
\pgfpathlineto{\pgfqpoint{0.996215in}{0.252063in}}%
\pgfpathlineto{\pgfqpoint{0.998825in}{0.255242in}}%
\pgfpathlineto{\pgfqpoint{0.999389in}{0.260231in}}%
\pgfpathlineto{\pgfqpoint{0.999268in}{0.262917in}}%
\pgfpathlineto{\pgfqpoint{0.996619in}{0.262972in}}%
\pgfpathlineto{\pgfqpoint{0.995505in}{0.265850in}}%
\pgfpathlineto{\pgfqpoint{0.994332in}{0.266616in}}%
\pgfpathlineto{\pgfqpoint{0.994829in}{0.270091in}}%
\pgfpathlineto{\pgfqpoint{0.991854in}{0.271110in}}%
\pgfpathlineto{\pgfqpoint{0.989999in}{0.273458in}}%
\pgfpathlineto{\pgfqpoint{0.988959in}{0.271866in}}%
\pgfpathlineto{\pgfqpoint{0.991512in}{0.269199in}}%
\pgfpathlineto{\pgfqpoint{0.991659in}{0.265639in}}%
\pgfpathlineto{\pgfqpoint{0.994323in}{0.262028in}}%
\pgfpathlineto{\pgfqpoint{0.991912in}{0.261215in}}%
\pgfpathlineto{\pgfqpoint{0.993056in}{0.259618in}}%
\pgfpathlineto{\pgfqpoint{0.994980in}{0.260068in}}%
\pgfpathlineto{\pgfqpoint{0.994653in}{0.252837in}}%
\pgfpathlineto{\pgfqpoint{0.990644in}{0.252625in}}%
\pgfpathlineto{\pgfqpoint{0.991345in}{0.254973in}}%
\pgfpathlineto{\pgfqpoint{0.988496in}{0.254261in}}%
\pgfpathlineto{\pgfqpoint{0.984640in}{0.252289in}}%
\pgfpathlineto{\pgfqpoint{0.984647in}{0.255868in}}%
\pgfpathlineto{\pgfqpoint{0.983547in}{0.258582in}}%
\pgfpathlineto{\pgfqpoint{0.981113in}{0.259295in}}%
\pgfpathlineto{\pgfqpoint{0.977670in}{0.267258in}}%
\pgfpathlineto{\pgfqpoint{0.978053in}{0.268922in}}%
\pgfpathlineto{\pgfqpoint{0.975934in}{0.273167in}}%
\pgfpathlineto{\pgfqpoint{0.976198in}{0.278332in}}%
\pgfpathlineto{\pgfqpoint{0.974322in}{0.283838in}}%
\pgfpathlineto{\pgfqpoint{0.971050in}{0.288052in}}%
\pgfpathlineto{\pgfqpoint{0.969594in}{0.291194in}}%
\pgfpathlineto{\pgfqpoint{0.965210in}{0.297141in}}%
\pgfpathlineto{\pgfqpoint{0.960936in}{0.305239in}}%
\pgfpathlineto{\pgfqpoint{0.963703in}{0.304798in}}%
\pgfpathlineto{\pgfqpoint{0.967622in}{0.306686in}}%
\pgfpathlineto{\pgfqpoint{0.968036in}{0.311879in}}%
\pgfpathlineto{\pgfqpoint{0.969003in}{0.313788in}}%
\pgfpathlineto{\pgfqpoint{0.964023in}{0.312011in}}%
\pgfpathlineto{\pgfqpoint{0.962366in}{0.314782in}}%
\pgfpathlineto{\pgfqpoint{0.958892in}{0.313923in}}%
\pgfpathlineto{\pgfqpoint{0.957939in}{0.311982in}}%
\pgfpathlineto{\pgfqpoint{0.953101in}{0.315145in}}%
\pgfpathlineto{\pgfqpoint{0.950261in}{0.318434in}}%
\pgfpathlineto{\pgfqpoint{0.957419in}{0.331347in}}%
\pgfpathlineto{\pgfqpoint{0.961735in}{0.326340in}}%
\pgfpathlineto{\pgfqpoint{0.964478in}{0.326904in}}%
\pgfpathlineto{\pgfqpoint{0.967827in}{0.321893in}}%
\pgfpathlineto{\pgfqpoint{0.973633in}{0.323635in}}%
\pgfpathlineto{\pgfqpoint{0.979518in}{0.320931in}}%
\pgfpathlineto{\pgfqpoint{0.980657in}{0.319336in}}%
\pgfpathlineto{\pgfqpoint{0.976127in}{0.314498in}}%
\pgfpathlineto{\pgfqpoint{0.976789in}{0.311274in}}%
\pgfpathlineto{\pgfqpoint{0.979814in}{0.307029in}}%
\pgfpathlineto{\pgfqpoint{0.979157in}{0.302974in}}%
\pgfpathlineto{\pgfqpoint{0.985183in}{0.283862in}}%
\pgfpathlineto{\pgfqpoint{0.983797in}{0.278005in}}%
\pgfpathlineto{\pgfqpoint{0.982101in}{0.274862in}}%
\pgfpathlineto{\pgfqpoint{0.983133in}{0.274427in}}%
\pgfpathlineto{\pgfqpoint{0.986626in}{0.275552in}}%
\pgfpathlineto{\pgfqpoint{0.992589in}{0.276282in}}%
\pgfpathlineto{\pgfqpoint{0.997181in}{0.275275in}}%
\pgfpathlineto{\pgfqpoint{1.000072in}{0.277511in}}%
\pgfpathlineto{\pgfqpoint{1.001574in}{0.280752in}}%
\pgfpathlineto{\pgfqpoint{1.004316in}{0.280941in}}%
\pgfpathlineto{\pgfqpoint{1.008178in}{0.284509in}}%
\pgfpathlineto{\pgfqpoint{1.011155in}{0.283690in}}%
\pgfpathlineto{\pgfqpoint{1.017747in}{0.284160in}}%
\pgfpathlineto{\pgfqpoint{1.019028in}{0.283365in}}%
\pgfpathlineto{\pgfqpoint{1.021198in}{0.277394in}}%
\pgfpathlineto{\pgfqpoint{1.021055in}{0.274683in}}%
\pgfpathlineto{\pgfqpoint{1.019054in}{0.272678in}}%
\pgfpathlineto{\pgfqpoint{1.019016in}{0.268273in}}%
\pgfpathlineto{\pgfqpoint{1.021865in}{0.265818in}}%
\pgfpathlineto{\pgfqpoint{1.021899in}{0.262781in}}%
\pgfpathlineto{\pgfqpoint{1.023271in}{0.260582in}}%
\pgfpathlineto{\pgfqpoint{1.022755in}{0.257879in}}%
\pgfpathlineto{\pgfqpoint{1.023096in}{0.253613in}}%
\pgfpathlineto{\pgfqpoint{1.026848in}{0.247912in}}%
\pgfpathlineto{\pgfqpoint{1.027264in}{0.242421in}}%
\pgfpathlineto{\pgfqpoint{1.028881in}{0.239210in}}%
\pgfpathlineto{\pgfqpoint{1.027851in}{0.237206in}}%
\pgfpathlineto{\pgfqpoint{1.028861in}{0.228647in}}%
\pgfpathlineto{\pgfqpoint{1.028585in}{0.225653in}}%
\pgfpathlineto{\pgfqpoint{1.029591in}{0.220042in}}%
\pgfpathlineto{\pgfqpoint{1.029999in}{0.210064in}}%
\pgfpathclose%
\pgfusepath{fill}%
\end{pgfscope}%
\begin{pgfscope}%
\pgfpathrectangle{\pgfqpoint{0.100000in}{0.100000in}}{\pgfqpoint{3.420221in}{2.189500in}}%
\pgfusepath{clip}%
\pgfsetbuttcap%
\pgfsetmiterjoin%
\definecolor{currentfill}{rgb}{0.000000,0.454902,0.772549}%
\pgfsetfillcolor{currentfill}%
\pgfsetlinewidth{0.000000pt}%
\definecolor{currentstroke}{rgb}{0.000000,0.000000,0.000000}%
\pgfsetstrokecolor{currentstroke}%
\pgfsetstrokeopacity{0.000000}%
\pgfsetdash{}{0pt}%
\pgfpathmoveto{\pgfqpoint{0.997384in}{0.210405in}}%
\pgfpathlineto{\pgfqpoint{0.999944in}{0.215624in}}%
\pgfpathlineto{\pgfqpoint{1.001857in}{0.217118in}}%
\pgfpathlineto{\pgfqpoint{1.003510in}{0.219471in}}%
\pgfpathlineto{\pgfqpoint{1.003922in}{0.227269in}}%
\pgfpathlineto{\pgfqpoint{1.005781in}{0.232160in}}%
\pgfpathlineto{\pgfqpoint{1.006564in}{0.236475in}}%
\pgfpathlineto{\pgfqpoint{1.008028in}{0.238134in}}%
\pgfpathlineto{\pgfqpoint{1.007393in}{0.240316in}}%
\pgfpathlineto{\pgfqpoint{1.007908in}{0.245297in}}%
\pgfpathlineto{\pgfqpoint{1.009798in}{0.243173in}}%
\pgfpathlineto{\pgfqpoint{1.009374in}{0.239152in}}%
\pgfpathlineto{\pgfqpoint{1.011103in}{0.238326in}}%
\pgfpathlineto{\pgfqpoint{1.015224in}{0.234989in}}%
\pgfpathlineto{\pgfqpoint{1.015613in}{0.229730in}}%
\pgfpathlineto{\pgfqpoint{1.015264in}{0.223357in}}%
\pgfpathlineto{\pgfqpoint{1.013718in}{0.224422in}}%
\pgfpathlineto{\pgfqpoint{1.014036in}{0.232393in}}%
\pgfpathlineto{\pgfqpoint{1.012657in}{0.232062in}}%
\pgfpathlineto{\pgfqpoint{1.010798in}{0.228635in}}%
\pgfpathlineto{\pgfqpoint{1.012566in}{0.228010in}}%
\pgfpathlineto{\pgfqpoint{1.012489in}{0.225866in}}%
\pgfpathlineto{\pgfqpoint{1.011322in}{0.222635in}}%
\pgfpathlineto{\pgfqpoint{1.012738in}{0.221782in}}%
\pgfpathlineto{\pgfqpoint{1.012235in}{0.217498in}}%
\pgfpathlineto{\pgfqpoint{1.009268in}{0.218468in}}%
\pgfpathlineto{\pgfqpoint{1.011320in}{0.214697in}}%
\pgfpathlineto{\pgfqpoint{1.009256in}{0.213523in}}%
\pgfpathlineto{\pgfqpoint{1.008346in}{0.214786in}}%
\pgfpathlineto{\pgfqpoint{1.006564in}{0.213074in}}%
\pgfpathlineto{\pgfqpoint{1.003817in}{0.212691in}}%
\pgfpathlineto{\pgfqpoint{1.000782in}{0.210612in}}%
\pgfpathclose%
\pgfusepath{fill}%
\end{pgfscope}%
\begin{pgfscope}%
\pgfpathrectangle{\pgfqpoint{0.100000in}{0.100000in}}{\pgfqpoint{3.420221in}{2.189500in}}%
\pgfusepath{clip}%
\pgfsetbuttcap%
\pgfsetmiterjoin%
\definecolor{currentfill}{rgb}{0.000000,0.454902,0.772549}%
\pgfsetfillcolor{currentfill}%
\pgfsetlinewidth{0.000000pt}%
\definecolor{currentstroke}{rgb}{0.000000,0.000000,0.000000}%
\pgfsetstrokecolor{currentstroke}%
\pgfsetstrokeopacity{0.000000}%
\pgfsetdash{}{0pt}%
\pgfpathmoveto{\pgfqpoint{0.983286in}{0.241682in}}%
\pgfpathlineto{\pgfqpoint{0.983304in}{0.244418in}}%
\pgfpathlineto{\pgfqpoint{0.984506in}{0.248378in}}%
\pgfpathlineto{\pgfqpoint{0.988520in}{0.248288in}}%
\pgfpathlineto{\pgfqpoint{0.990068in}{0.247517in}}%
\pgfpathlineto{\pgfqpoint{0.991266in}{0.249247in}}%
\pgfpathlineto{\pgfqpoint{0.992630in}{0.247618in}}%
\pgfpathlineto{\pgfqpoint{0.996173in}{0.248868in}}%
\pgfpathlineto{\pgfqpoint{0.998549in}{0.245354in}}%
\pgfpathlineto{\pgfqpoint{0.999096in}{0.241884in}}%
\pgfpathlineto{\pgfqpoint{1.003279in}{0.239373in}}%
\pgfpathlineto{\pgfqpoint{1.004635in}{0.237530in}}%
\pgfpathlineto{\pgfqpoint{1.003949in}{0.235017in}}%
\pgfpathlineto{\pgfqpoint{1.002371in}{0.233931in}}%
\pgfpathlineto{\pgfqpoint{1.000807in}{0.231344in}}%
\pgfpathlineto{\pgfqpoint{0.997680in}{0.232561in}}%
\pgfpathlineto{\pgfqpoint{0.995282in}{0.237264in}}%
\pgfpathlineto{\pgfqpoint{0.993433in}{0.239620in}}%
\pgfpathlineto{\pgfqpoint{0.992007in}{0.242719in}}%
\pgfpathlineto{\pgfqpoint{0.988455in}{0.239295in}}%
\pgfpathclose%
\pgfusepath{fill}%
\end{pgfscope}%
\begin{pgfscope}%
\pgfpathrectangle{\pgfqpoint{0.100000in}{0.100000in}}{\pgfqpoint{3.420221in}{2.189500in}}%
\pgfusepath{clip}%
\pgfsetbuttcap%
\pgfsetmiterjoin%
\definecolor{currentfill}{rgb}{0.000000,0.239216,0.880392}%
\pgfsetfillcolor{currentfill}%
\pgfsetlinewidth{0.000000pt}%
\definecolor{currentstroke}{rgb}{0.000000,0.000000,0.000000}%
\pgfsetstrokecolor{currentstroke}%
\pgfsetstrokeopacity{0.000000}%
\pgfsetdash{}{0pt}%
\pgfpathmoveto{\pgfqpoint{3.174987in}{1.585419in}}%
\pgfpathlineto{\pgfqpoint{3.174778in}{1.597972in}}%
\pgfpathlineto{\pgfqpoint{3.196152in}{1.606066in}}%
\pgfpathlineto{\pgfqpoint{3.199075in}{1.588896in}}%
\pgfpathlineto{\pgfqpoint{3.204266in}{1.583772in}}%
\pgfpathlineto{\pgfqpoint{3.192775in}{1.572376in}}%
\pgfpathlineto{\pgfqpoint{3.198653in}{1.564938in}}%
\pgfpathlineto{\pgfqpoint{3.201557in}{1.559064in}}%
\pgfpathlineto{\pgfqpoint{3.208601in}{1.562487in}}%
\pgfpathlineto{\pgfqpoint{3.223725in}{1.564883in}}%
\pgfpathlineto{\pgfqpoint{3.232264in}{1.571453in}}%
\pgfpathlineto{\pgfqpoint{3.247715in}{1.575434in}}%
\pgfpathlineto{\pgfqpoint{3.255107in}{1.578408in}}%
\pgfpathlineto{\pgfqpoint{3.265453in}{1.589558in}}%
\pgfpathlineto{\pgfqpoint{3.267387in}{1.593720in}}%
\pgfpathlineto{\pgfqpoint{3.274217in}{1.587940in}}%
\pgfpathlineto{\pgfqpoint{3.279613in}{1.590209in}}%
\pgfpathlineto{\pgfqpoint{3.283640in}{1.586587in}}%
\pgfpathlineto{\pgfqpoint{3.290281in}{1.594402in}}%
\pgfpathlineto{\pgfqpoint{3.295283in}{1.594147in}}%
\pgfpathlineto{\pgfqpoint{3.264755in}{1.569245in}}%
\pgfpathlineto{\pgfqpoint{3.245843in}{1.556405in}}%
\pgfpathlineto{\pgfqpoint{3.238616in}{1.550436in}}%
\pgfpathlineto{\pgfqpoint{3.226973in}{1.543649in}}%
\pgfpathlineto{\pgfqpoint{3.222263in}{1.543364in}}%
\pgfpathlineto{\pgfqpoint{3.211002in}{1.536877in}}%
\pgfpathlineto{\pgfqpoint{3.199174in}{1.534627in}}%
\pgfpathlineto{\pgfqpoint{3.191550in}{1.529752in}}%
\pgfpathlineto{\pgfqpoint{3.183421in}{1.531809in}}%
\pgfpathlineto{\pgfqpoint{3.181126in}{1.526965in}}%
\pgfpathlineto{\pgfqpoint{3.173989in}{1.521185in}}%
\pgfpathlineto{\pgfqpoint{3.175317in}{1.533250in}}%
\pgfpathlineto{\pgfqpoint{3.182432in}{1.535324in}}%
\pgfpathlineto{\pgfqpoint{3.185556in}{1.561494in}}%
\pgfpathlineto{\pgfqpoint{3.183350in}{1.573034in}}%
\pgfpathlineto{\pgfqpoint{3.176888in}{1.579951in}}%
\pgfpathclose%
\pgfusepath{fill}%
\end{pgfscope}%
\begin{pgfscope}%
\pgfpathrectangle{\pgfqpoint{0.100000in}{0.100000in}}{\pgfqpoint{3.420221in}{2.189500in}}%
\pgfusepath{clip}%
\pgfsetbuttcap%
\pgfsetmiterjoin%
\definecolor{currentfill}{rgb}{0.000000,0.588235,0.705882}%
\pgfsetfillcolor{currentfill}%
\pgfsetlinewidth{0.000000pt}%
\definecolor{currentstroke}{rgb}{0.000000,0.000000,0.000000}%
\pgfsetstrokecolor{currentstroke}%
\pgfsetstrokeopacity{0.000000}%
\pgfsetdash{}{0pt}%
\pgfpathmoveto{\pgfqpoint{1.102559in}{1.867035in}}%
\pgfpathlineto{\pgfqpoint{1.092166in}{1.875884in}}%
\pgfpathlineto{\pgfqpoint{1.093013in}{1.895851in}}%
\pgfpathlineto{\pgfqpoint{1.088809in}{1.902795in}}%
\pgfpathlineto{\pgfqpoint{1.090767in}{1.911577in}}%
\pgfpathlineto{\pgfqpoint{1.093346in}{1.913918in}}%
\pgfpathlineto{\pgfqpoint{1.103333in}{1.915851in}}%
\pgfpathlineto{\pgfqpoint{1.106652in}{1.920240in}}%
\pgfpathlineto{\pgfqpoint{1.109426in}{1.931652in}}%
\pgfpathlineto{\pgfqpoint{1.104825in}{1.939769in}}%
\pgfpathlineto{\pgfqpoint{1.098470in}{1.940916in}}%
\pgfpathlineto{\pgfqpoint{1.100425in}{1.951001in}}%
\pgfpathlineto{\pgfqpoint{1.087202in}{1.953404in}}%
\pgfpathlineto{\pgfqpoint{1.092223in}{1.979056in}}%
\pgfpathlineto{\pgfqpoint{1.079022in}{1.981861in}}%
\pgfpathlineto{\pgfqpoint{1.083608in}{2.004787in}}%
\pgfpathlineto{\pgfqpoint{1.080946in}{2.013694in}}%
\pgfpathlineto{\pgfqpoint{1.081767in}{2.023146in}}%
\pgfpathlineto{\pgfqpoint{1.086452in}{2.024455in}}%
\pgfpathlineto{\pgfqpoint{1.089384in}{2.034350in}}%
\pgfpathlineto{\pgfqpoint{1.094304in}{2.038665in}}%
\pgfpathlineto{\pgfqpoint{1.096036in}{2.028452in}}%
\pgfpathlineto{\pgfqpoint{1.093969in}{2.018698in}}%
\pgfpathlineto{\pgfqpoint{1.097469in}{2.011264in}}%
\pgfpathlineto{\pgfqpoint{1.109896in}{2.010783in}}%
\pgfpathlineto{\pgfqpoint{1.117293in}{2.006994in}}%
\pgfpathlineto{\pgfqpoint{1.117341in}{2.003240in}}%
\pgfpathlineto{\pgfqpoint{1.123097in}{1.998390in}}%
\pgfpathlineto{\pgfqpoint{1.134856in}{1.996920in}}%
\pgfpathlineto{\pgfqpoint{1.130410in}{1.972843in}}%
\pgfpathlineto{\pgfqpoint{1.135051in}{1.968034in}}%
\pgfpathlineto{\pgfqpoint{1.142445in}{1.965665in}}%
\pgfpathlineto{\pgfqpoint{1.139417in}{1.949675in}}%
\pgfpathlineto{\pgfqpoint{1.146113in}{1.948385in}}%
\pgfpathlineto{\pgfqpoint{1.145784in}{1.943085in}}%
\pgfpathlineto{\pgfqpoint{1.152129in}{1.939198in}}%
\pgfpathlineto{\pgfqpoint{1.154959in}{1.926267in}}%
\pgfpathlineto{\pgfqpoint{1.158654in}{1.919655in}}%
\pgfpathlineto{\pgfqpoint{1.161477in}{1.908319in}}%
\pgfpathlineto{\pgfqpoint{1.165834in}{1.909249in}}%
\pgfpathlineto{\pgfqpoint{1.169833in}{1.904480in}}%
\pgfpathlineto{\pgfqpoint{1.165617in}{1.899000in}}%
\pgfpathlineto{\pgfqpoint{1.166816in}{1.889269in}}%
\pgfpathlineto{\pgfqpoint{1.152147in}{1.891545in}}%
\pgfpathlineto{\pgfqpoint{1.151109in}{1.886731in}}%
\pgfpathlineto{\pgfqpoint{1.146405in}{1.882816in}}%
\pgfpathlineto{\pgfqpoint{1.142509in}{1.875146in}}%
\pgfpathlineto{\pgfqpoint{1.135614in}{1.872312in}}%
\pgfpathlineto{\pgfqpoint{1.128740in}{1.866106in}}%
\pgfpathlineto{\pgfqpoint{1.120262in}{1.869223in}}%
\pgfpathlineto{\pgfqpoint{1.117614in}{1.872511in}}%
\pgfpathlineto{\pgfqpoint{1.110257in}{1.873573in}}%
\pgfpathclose%
\pgfusepath{fill}%
\end{pgfscope}%
\begin{pgfscope}%
\pgfpathrectangle{\pgfqpoint{0.100000in}{0.100000in}}{\pgfqpoint{3.420221in}{2.189500in}}%
\pgfusepath{clip}%
\pgfsetbuttcap%
\pgfsetmiterjoin%
\definecolor{currentfill}{rgb}{0.000000,0.243137,0.878431}%
\pgfsetfillcolor{currentfill}%
\pgfsetlinewidth{0.000000pt}%
\definecolor{currentstroke}{rgb}{0.000000,0.000000,0.000000}%
\pgfsetstrokecolor{currentstroke}%
\pgfsetstrokeopacity{0.000000}%
\pgfsetdash{}{0pt}%
\pgfpathmoveto{\pgfqpoint{1.866876in}{1.377916in}}%
\pgfpathlineto{\pgfqpoint{1.840861in}{1.378665in}}%
\pgfpathlineto{\pgfqpoint{1.841579in}{1.404755in}}%
\pgfpathlineto{\pgfqpoint{1.842536in}{1.440340in}}%
\pgfpathlineto{\pgfqpoint{1.846196in}{1.444822in}}%
\pgfpathlineto{\pgfqpoint{1.855846in}{1.451895in}}%
\pgfpathlineto{\pgfqpoint{1.868928in}{1.456187in}}%
\pgfpathlineto{\pgfqpoint{1.868247in}{1.430099in}}%
\pgfpathclose%
\pgfusepath{fill}%
\end{pgfscope}%
\begin{pgfscope}%
\pgfpathrectangle{\pgfqpoint{0.100000in}{0.100000in}}{\pgfqpoint{3.420221in}{2.189500in}}%
\pgfusepath{clip}%
\pgfsetbuttcap%
\pgfsetmiterjoin%
\definecolor{currentfill}{rgb}{0.000000,0.517647,0.741176}%
\pgfsetfillcolor{currentfill}%
\pgfsetlinewidth{0.000000pt}%
\definecolor{currentstroke}{rgb}{0.000000,0.000000,0.000000}%
\pgfsetstrokecolor{currentstroke}%
\pgfsetstrokeopacity{0.000000}%
\pgfsetdash{}{0pt}%
\pgfpathmoveto{\pgfqpoint{0.703950in}{1.902176in}}%
\pgfpathlineto{\pgfqpoint{0.701514in}{1.889466in}}%
\pgfpathlineto{\pgfqpoint{0.690922in}{1.852082in}}%
\pgfpathlineto{\pgfqpoint{0.678305in}{1.855586in}}%
\pgfpathlineto{\pgfqpoint{0.679982in}{1.861592in}}%
\pgfpathlineto{\pgfqpoint{0.674628in}{1.866933in}}%
\pgfpathlineto{\pgfqpoint{0.655798in}{1.872215in}}%
\pgfpathlineto{\pgfqpoint{0.662554in}{1.895639in}}%
\pgfpathlineto{\pgfqpoint{0.664497in}{1.899801in}}%
\pgfpathlineto{\pgfqpoint{0.661149in}{1.904173in}}%
\pgfpathlineto{\pgfqpoint{0.661170in}{1.910916in}}%
\pgfpathlineto{\pgfqpoint{0.666648in}{1.925979in}}%
\pgfpathlineto{\pgfqpoint{0.664144in}{1.933697in}}%
\pgfpathlineto{\pgfqpoint{0.671538in}{1.948228in}}%
\pgfpathlineto{\pgfqpoint{0.678818in}{1.948599in}}%
\pgfpathlineto{\pgfqpoint{0.678585in}{1.956413in}}%
\pgfpathlineto{\pgfqpoint{0.675859in}{1.961915in}}%
\pgfpathlineto{\pgfqpoint{0.681090in}{1.963852in}}%
\pgfpathlineto{\pgfqpoint{0.691429in}{1.962991in}}%
\pgfpathlineto{\pgfqpoint{0.703905in}{1.966239in}}%
\pgfpathlineto{\pgfqpoint{0.692166in}{1.926063in}}%
\pgfpathlineto{\pgfqpoint{0.698503in}{1.924161in}}%
\pgfpathlineto{\pgfqpoint{0.696694in}{1.917914in}}%
\pgfpathlineto{\pgfqpoint{0.702939in}{1.916002in}}%
\pgfpathlineto{\pgfqpoint{0.697846in}{1.903896in}}%
\pgfpathclose%
\pgfusepath{fill}%
\end{pgfscope}%
\begin{pgfscope}%
\pgfpathrectangle{\pgfqpoint{0.100000in}{0.100000in}}{\pgfqpoint{3.420221in}{2.189500in}}%
\pgfusepath{clip}%
\pgfsetbuttcap%
\pgfsetmiterjoin%
\definecolor{currentfill}{rgb}{0.000000,0.266667,0.866667}%
\pgfsetfillcolor{currentfill}%
\pgfsetlinewidth{0.000000pt}%
\definecolor{currentstroke}{rgb}{0.000000,0.000000,0.000000}%
\pgfsetstrokecolor{currentstroke}%
\pgfsetstrokeopacity{0.000000}%
\pgfsetdash{}{0pt}%
\pgfpathmoveto{\pgfqpoint{1.704193in}{1.358534in}}%
\pgfpathlineto{\pgfqpoint{1.639449in}{1.362727in}}%
\pgfpathlineto{\pgfqpoint{1.641298in}{1.388721in}}%
\pgfpathlineto{\pgfqpoint{1.599819in}{1.391683in}}%
\pgfpathlineto{\pgfqpoint{1.601856in}{1.417708in}}%
\pgfpathlineto{\pgfqpoint{1.647383in}{1.414389in}}%
\pgfpathlineto{\pgfqpoint{1.674048in}{1.412800in}}%
\pgfpathlineto{\pgfqpoint{1.672408in}{1.386568in}}%
\pgfpathlineto{\pgfqpoint{1.705444in}{1.384614in}}%
\pgfpathclose%
\pgfusepath{fill}%
\end{pgfscope}%
\begin{pgfscope}%
\pgfpathrectangle{\pgfqpoint{0.100000in}{0.100000in}}{\pgfqpoint{3.420221in}{2.189500in}}%
\pgfusepath{clip}%
\pgfsetbuttcap%
\pgfsetmiterjoin%
\definecolor{currentfill}{rgb}{0.000000,0.560784,0.719608}%
\pgfsetfillcolor{currentfill}%
\pgfsetlinewidth{0.000000pt}%
\definecolor{currentstroke}{rgb}{0.000000,0.000000,0.000000}%
\pgfsetstrokecolor{currentstroke}%
\pgfsetstrokeopacity{0.000000}%
\pgfsetdash{}{0pt}%
\pgfpathmoveto{\pgfqpoint{1.164577in}{1.189894in}}%
\pgfpathlineto{\pgfqpoint{1.108090in}{1.198913in}}%
\pgfpathlineto{\pgfqpoint{1.063895in}{1.206860in}}%
\pgfpathlineto{\pgfqpoint{1.027530in}{1.213272in}}%
\pgfpathlineto{\pgfqpoint{1.036119in}{1.214492in}}%
\pgfpathlineto{\pgfqpoint{1.040007in}{1.218670in}}%
\pgfpathlineto{\pgfqpoint{1.047066in}{1.217159in}}%
\pgfpathlineto{\pgfqpoint{1.057265in}{1.221112in}}%
\pgfpathlineto{\pgfqpoint{1.060752in}{1.229263in}}%
\pgfpathlineto{\pgfqpoint{1.068804in}{1.231965in}}%
\pgfpathlineto{\pgfqpoint{1.076953in}{1.245506in}}%
\pgfpathlineto{\pgfqpoint{1.080385in}{1.249515in}}%
\pgfpathlineto{\pgfqpoint{1.087072in}{1.252209in}}%
\pgfpathlineto{\pgfqpoint{1.095395in}{1.268094in}}%
\pgfpathlineto{\pgfqpoint{1.103888in}{1.268117in}}%
\pgfpathlineto{\pgfqpoint{1.112576in}{1.274715in}}%
\pgfpathlineto{\pgfqpoint{1.115011in}{1.273492in}}%
\pgfpathlineto{\pgfqpoint{1.121083in}{1.281150in}}%
\pgfpathlineto{\pgfqpoint{1.126531in}{1.283514in}}%
\pgfpathlineto{\pgfqpoint{1.129369in}{1.286933in}}%
\pgfpathlineto{\pgfqpoint{1.126995in}{1.293955in}}%
\pgfpathlineto{\pgfqpoint{1.125155in}{1.310182in}}%
\pgfpathlineto{\pgfqpoint{1.181037in}{1.301162in}}%
\pgfpathlineto{\pgfqpoint{1.178440in}{1.284544in}}%
\pgfpathlineto{\pgfqpoint{1.178225in}{1.276176in}}%
\pgfpathlineto{\pgfqpoint{1.174971in}{1.255182in}}%
\pgfpathlineto{\pgfqpoint{1.188265in}{1.254218in}}%
\pgfpathlineto{\pgfqpoint{1.221021in}{1.249232in}}%
\pgfpathlineto{\pgfqpoint{1.222934in}{1.243386in}}%
\pgfpathlineto{\pgfqpoint{1.236242in}{1.244152in}}%
\pgfpathlineto{\pgfqpoint{1.242917in}{1.237047in}}%
\pgfpathlineto{\pgfqpoint{1.241096in}{1.232984in}}%
\pgfpathlineto{\pgfqpoint{1.231564in}{1.224248in}}%
\pgfpathlineto{\pgfqpoint{1.229489in}{1.215460in}}%
\pgfpathlineto{\pgfqpoint{1.224291in}{1.209918in}}%
\pgfpathlineto{\pgfqpoint{1.218619in}{1.208725in}}%
\pgfpathlineto{\pgfqpoint{1.211705in}{1.199766in}}%
\pgfpathlineto{\pgfqpoint{1.207266in}{1.187644in}}%
\pgfpathlineto{\pgfqpoint{1.204003in}{1.183932in}}%
\pgfpathclose%
\pgfusepath{fill}%
\end{pgfscope}%
\begin{pgfscope}%
\pgfpathrectangle{\pgfqpoint{0.100000in}{0.100000in}}{\pgfqpoint{3.420221in}{2.189500in}}%
\pgfusepath{clip}%
\pgfsetbuttcap%
\pgfsetmiterjoin%
\definecolor{currentfill}{rgb}{0.000000,0.317647,0.841176}%
\pgfsetfillcolor{currentfill}%
\pgfsetlinewidth{0.000000pt}%
\definecolor{currentstroke}{rgb}{0.000000,0.000000,0.000000}%
\pgfsetstrokecolor{currentstroke}%
\pgfsetstrokeopacity{0.000000}%
\pgfsetdash{}{0pt}%
\pgfpathmoveto{\pgfqpoint{2.535887in}{0.999923in}}%
\pgfpathlineto{\pgfqpoint{2.534274in}{0.990908in}}%
\pgfpathlineto{\pgfqpoint{2.536104in}{0.987774in}}%
\pgfpathlineto{\pgfqpoint{2.535519in}{0.972835in}}%
\pgfpathlineto{\pgfqpoint{2.539121in}{0.966130in}}%
\pgfpathlineto{\pgfqpoint{2.532588in}{0.960995in}}%
\pgfpathlineto{\pgfqpoint{2.525691in}{0.962109in}}%
\pgfpathlineto{\pgfqpoint{2.524078in}{0.947004in}}%
\pgfpathlineto{\pgfqpoint{2.491354in}{0.944747in}}%
\pgfpathlineto{\pgfqpoint{2.491461in}{0.943662in}}%
\pgfpathlineto{\pgfqpoint{2.465452in}{0.941858in}}%
\pgfpathlineto{\pgfqpoint{2.463838in}{0.961521in}}%
\pgfpathlineto{\pgfqpoint{2.469308in}{0.972666in}}%
\pgfpathlineto{\pgfqpoint{2.468740in}{0.979489in}}%
\pgfpathlineto{\pgfqpoint{2.476434in}{0.977234in}}%
\pgfpathlineto{\pgfqpoint{2.481643in}{0.981788in}}%
\pgfpathlineto{\pgfqpoint{2.480721in}{0.995506in}}%
\pgfpathlineto{\pgfqpoint{2.503714in}{0.996982in}}%
\pgfpathlineto{\pgfqpoint{2.502375in}{1.017363in}}%
\pgfpathlineto{\pgfqpoint{2.507418in}{1.017006in}}%
\pgfpathlineto{\pgfqpoint{2.515714in}{1.023357in}}%
\pgfpathlineto{\pgfqpoint{2.515642in}{1.026211in}}%
\pgfpathlineto{\pgfqpoint{2.529196in}{1.017712in}}%
\pgfpathlineto{\pgfqpoint{2.531256in}{1.010309in}}%
\pgfpathlineto{\pgfqpoint{2.534490in}{1.010056in}}%
\pgfpathclose%
\pgfusepath{fill}%
\end{pgfscope}%
\begin{pgfscope}%
\pgfpathrectangle{\pgfqpoint{0.100000in}{0.100000in}}{\pgfqpoint{3.420221in}{2.189500in}}%
\pgfusepath{clip}%
\pgfsetbuttcap%
\pgfsetmiterjoin%
\definecolor{currentfill}{rgb}{0.000000,0.345098,0.827451}%
\pgfsetfillcolor{currentfill}%
\pgfsetlinewidth{0.000000pt}%
\definecolor{currentstroke}{rgb}{0.000000,0.000000,0.000000}%
\pgfsetstrokecolor{currentstroke}%
\pgfsetstrokeopacity{0.000000}%
\pgfsetdash{}{0pt}%
\pgfpathmoveto{\pgfqpoint{1.194936in}{1.386988in}}%
\pgfpathlineto{\pgfqpoint{1.206560in}{1.460491in}}%
\pgfpathlineto{\pgfqpoint{1.210463in}{1.486210in}}%
\pgfpathlineto{\pgfqpoint{1.239756in}{1.481650in}}%
\pgfpathlineto{\pgfqpoint{1.273790in}{1.476848in}}%
\pgfpathlineto{\pgfqpoint{1.333239in}{1.468741in}}%
\pgfpathlineto{\pgfqpoint{1.332812in}{1.462916in}}%
\pgfpathlineto{\pgfqpoint{1.343167in}{1.455843in}}%
\pgfpathlineto{\pgfqpoint{1.343665in}{1.451458in}}%
\pgfpathlineto{\pgfqpoint{1.337871in}{1.436041in}}%
\pgfpathlineto{\pgfqpoint{1.339434in}{1.425804in}}%
\pgfpathlineto{\pgfqpoint{1.339565in}{1.417954in}}%
\pgfpathlineto{\pgfqpoint{1.336047in}{1.392762in}}%
\pgfpathlineto{\pgfqpoint{1.335903in}{1.386939in}}%
\pgfpathlineto{\pgfqpoint{1.312761in}{1.389525in}}%
\pgfpathlineto{\pgfqpoint{1.314150in}{1.402388in}}%
\pgfpathlineto{\pgfqpoint{1.298257in}{1.404472in}}%
\pgfpathlineto{\pgfqpoint{1.296377in}{1.391435in}}%
\pgfpathlineto{\pgfqpoint{1.289977in}{1.392521in}}%
\pgfpathlineto{\pgfqpoint{1.289058in}{1.385889in}}%
\pgfpathlineto{\pgfqpoint{1.260151in}{1.389695in}}%
\pgfpathlineto{\pgfqpoint{1.258760in}{1.380015in}}%
\pgfpathlineto{\pgfqpoint{1.223608in}{1.385113in}}%
\pgfpathlineto{\pgfqpoint{1.223129in}{1.381891in}}%
\pgfpathclose%
\pgfusepath{fill}%
\end{pgfscope}%
\begin{pgfscope}%
\pgfpathrectangle{\pgfqpoint{0.100000in}{0.100000in}}{\pgfqpoint{3.420221in}{2.189500in}}%
\pgfusepath{clip}%
\pgfsetbuttcap%
\pgfsetmiterjoin%
\definecolor{currentfill}{rgb}{0.000000,0.698039,0.650980}%
\pgfsetfillcolor{currentfill}%
\pgfsetlinewidth{0.000000pt}%
\definecolor{currentstroke}{rgb}{0.000000,0.000000,0.000000}%
\pgfsetstrokecolor{currentstroke}%
\pgfsetstrokeopacity{0.000000}%
\pgfsetdash{}{0pt}%
\pgfpathmoveto{\pgfqpoint{1.625833in}{0.540144in}}%
\pgfpathlineto{\pgfqpoint{1.662723in}{0.538108in}}%
\pgfpathlineto{\pgfqpoint{1.658765in}{0.470587in}}%
\pgfpathlineto{\pgfqpoint{1.652157in}{0.470888in}}%
\pgfpathlineto{\pgfqpoint{1.647127in}{0.477473in}}%
\pgfpathlineto{\pgfqpoint{1.644885in}{0.489121in}}%
\pgfpathlineto{\pgfqpoint{1.640425in}{0.498606in}}%
\pgfpathlineto{\pgfqpoint{1.641308in}{0.501260in}}%
\pgfpathlineto{\pgfqpoint{1.635049in}{0.507317in}}%
\pgfpathlineto{\pgfqpoint{1.632638in}{0.519995in}}%
\pgfpathlineto{\pgfqpoint{1.626230in}{0.529388in}}%
\pgfpathclose%
\pgfusepath{fill}%
\end{pgfscope}%
\begin{pgfscope}%
\pgfpathrectangle{\pgfqpoint{0.100000in}{0.100000in}}{\pgfqpoint{3.420221in}{2.189500in}}%
\pgfusepath{clip}%
\pgfsetbuttcap%
\pgfsetmiterjoin%
\definecolor{currentfill}{rgb}{0.000000,0.513725,0.743137}%
\pgfsetfillcolor{currentfill}%
\pgfsetlinewidth{0.000000pt}%
\definecolor{currentstroke}{rgb}{0.000000,0.000000,0.000000}%
\pgfsetstrokecolor{currentstroke}%
\pgfsetstrokeopacity{0.000000}%
\pgfsetdash{}{0pt}%
\pgfpathmoveto{\pgfqpoint{2.006144in}{0.737572in}}%
\pgfpathlineto{\pgfqpoint{2.003744in}{0.732771in}}%
\pgfpathlineto{\pgfqpoint{2.006948in}{0.717750in}}%
\pgfpathlineto{\pgfqpoint{2.010751in}{0.713514in}}%
\pgfpathlineto{\pgfqpoint{2.001901in}{0.705887in}}%
\pgfpathlineto{\pgfqpoint{1.995278in}{0.710187in}}%
\pgfpathlineto{\pgfqpoint{1.992469in}{0.716729in}}%
\pgfpathlineto{\pgfqpoint{1.984517in}{0.718513in}}%
\pgfpathlineto{\pgfqpoint{1.960145in}{0.714779in}}%
\pgfpathlineto{\pgfqpoint{1.954447in}{0.711960in}}%
\pgfpathlineto{\pgfqpoint{1.956358in}{0.719557in}}%
\pgfpathlineto{\pgfqpoint{1.946527in}{0.726236in}}%
\pgfpathlineto{\pgfqpoint{1.945859in}{0.731063in}}%
\pgfpathlineto{\pgfqpoint{1.940990in}{0.732629in}}%
\pgfpathlineto{\pgfqpoint{1.933919in}{0.752509in}}%
\pgfpathlineto{\pgfqpoint{1.923306in}{0.768704in}}%
\pgfpathlineto{\pgfqpoint{1.913946in}{0.774695in}}%
\pgfpathlineto{\pgfqpoint{1.933568in}{0.776610in}}%
\pgfpathlineto{\pgfqpoint{1.933940in}{0.812869in}}%
\pgfpathlineto{\pgfqpoint{1.942951in}{0.812673in}}%
\pgfpathlineto{\pgfqpoint{1.948644in}{0.807489in}}%
\pgfpathlineto{\pgfqpoint{1.951758in}{0.807978in}}%
\pgfpathlineto{\pgfqpoint{1.961832in}{0.803699in}}%
\pgfpathlineto{\pgfqpoint{1.960052in}{0.821807in}}%
\pgfpathlineto{\pgfqpoint{1.982581in}{0.821847in}}%
\pgfpathlineto{\pgfqpoint{1.991165in}{0.821771in}}%
\pgfpathlineto{\pgfqpoint{1.992487in}{0.817292in}}%
\pgfpathlineto{\pgfqpoint{1.992363in}{0.792246in}}%
\pgfpathlineto{\pgfqpoint{2.002898in}{0.789759in}}%
\pgfpathlineto{\pgfqpoint{2.003069in}{0.737619in}}%
\pgfpathclose%
\pgfusepath{fill}%
\end{pgfscope}%
\begin{pgfscope}%
\pgfpathrectangle{\pgfqpoint{0.100000in}{0.100000in}}{\pgfqpoint{3.420221in}{2.189500in}}%
\pgfusepath{clip}%
\pgfsetbuttcap%
\pgfsetmiterjoin%
\definecolor{currentfill}{rgb}{0.000000,0.478431,0.760784}%
\pgfsetfillcolor{currentfill}%
\pgfsetlinewidth{0.000000pt}%
\definecolor{currentstroke}{rgb}{0.000000,0.000000,0.000000}%
\pgfsetstrokecolor{currentstroke}%
\pgfsetstrokeopacity{0.000000}%
\pgfsetdash{}{0pt}%
\pgfpathmoveto{\pgfqpoint{0.449968in}{1.388899in}}%
\pgfpathlineto{\pgfqpoint{0.446457in}{1.394723in}}%
\pgfpathlineto{\pgfqpoint{0.452447in}{1.414640in}}%
\pgfpathlineto{\pgfqpoint{0.452303in}{1.419122in}}%
\pgfpathlineto{\pgfqpoint{0.456967in}{1.434698in}}%
\pgfpathlineto{\pgfqpoint{0.452454in}{1.436188in}}%
\pgfpathlineto{\pgfqpoint{0.452500in}{1.441186in}}%
\pgfpathlineto{\pgfqpoint{0.457465in}{1.442763in}}%
\pgfpathlineto{\pgfqpoint{0.461047in}{1.450614in}}%
\pgfpathlineto{\pgfqpoint{0.455353in}{1.452384in}}%
\pgfpathlineto{\pgfqpoint{0.459955in}{1.467688in}}%
\pgfpathlineto{\pgfqpoint{0.446212in}{1.472384in}}%
\pgfpathlineto{\pgfqpoint{0.441325in}{1.470373in}}%
\pgfpathlineto{\pgfqpoint{0.436751in}{1.473688in}}%
\pgfpathlineto{\pgfqpoint{0.436242in}{1.484977in}}%
\pgfpathlineto{\pgfqpoint{0.434160in}{1.488812in}}%
\pgfpathlineto{\pgfqpoint{0.433643in}{1.500257in}}%
\pgfpathlineto{\pgfqpoint{0.428217in}{1.503848in}}%
\pgfpathlineto{\pgfqpoint{0.432606in}{1.507192in}}%
\pgfpathlineto{\pgfqpoint{0.460710in}{1.498576in}}%
\pgfpathlineto{\pgfqpoint{0.465159in}{1.492099in}}%
\pgfpathlineto{\pgfqpoint{0.464362in}{1.488016in}}%
\pgfpathlineto{\pgfqpoint{0.469625in}{1.481111in}}%
\pgfpathlineto{\pgfqpoint{0.477422in}{1.479975in}}%
\pgfpathlineto{\pgfqpoint{0.481190in}{1.492511in}}%
\pgfpathlineto{\pgfqpoint{0.485875in}{1.497103in}}%
\pgfpathlineto{\pgfqpoint{0.492841in}{1.498925in}}%
\pgfpathlineto{\pgfqpoint{0.498294in}{1.511210in}}%
\pgfpathlineto{\pgfqpoint{0.509905in}{1.519445in}}%
\pgfpathlineto{\pgfqpoint{0.516018in}{1.518528in}}%
\pgfpathlineto{\pgfqpoint{0.533038in}{1.519030in}}%
\pgfpathlineto{\pgfqpoint{0.539243in}{1.522304in}}%
\pgfpathlineto{\pgfqpoint{0.543468in}{1.520815in}}%
\pgfpathlineto{\pgfqpoint{0.545879in}{1.514102in}}%
\pgfpathlineto{\pgfqpoint{0.573802in}{1.506061in}}%
\pgfpathlineto{\pgfqpoint{0.567019in}{1.482099in}}%
\pgfpathlineto{\pgfqpoint{0.580600in}{1.478387in}}%
\pgfpathlineto{\pgfqpoint{0.580523in}{1.476145in}}%
\pgfpathlineto{\pgfqpoint{0.605179in}{1.469222in}}%
\pgfpathlineto{\pgfqpoint{0.603395in}{1.462495in}}%
\pgfpathlineto{\pgfqpoint{0.596496in}{1.461593in}}%
\pgfpathlineto{\pgfqpoint{0.592913in}{1.445380in}}%
\pgfpathlineto{\pgfqpoint{0.596010in}{1.444205in}}%
\pgfpathlineto{\pgfqpoint{0.593367in}{1.429827in}}%
\pgfpathlineto{\pgfqpoint{0.582602in}{1.446075in}}%
\pgfpathlineto{\pgfqpoint{0.578326in}{1.439321in}}%
\pgfpathlineto{\pgfqpoint{0.580549in}{1.430631in}}%
\pgfpathlineto{\pgfqpoint{0.577123in}{1.424141in}}%
\pgfpathlineto{\pgfqpoint{0.571639in}{1.419667in}}%
\pgfpathlineto{\pgfqpoint{0.567128in}{1.427943in}}%
\pgfpathlineto{\pgfqpoint{0.558333in}{1.425664in}}%
\pgfpathlineto{\pgfqpoint{0.552257in}{1.433384in}}%
\pgfpathlineto{\pgfqpoint{0.542200in}{1.431593in}}%
\pgfpathlineto{\pgfqpoint{0.526617in}{1.418655in}}%
\pgfpathlineto{\pgfqpoint{0.514360in}{1.406799in}}%
\pgfpathlineto{\pgfqpoint{0.503949in}{1.400364in}}%
\pgfpathlineto{\pgfqpoint{0.493631in}{1.422507in}}%
\pgfpathlineto{\pgfqpoint{0.486813in}{1.398059in}}%
\pgfpathlineto{\pgfqpoint{0.483174in}{1.400853in}}%
\pgfpathlineto{\pgfqpoint{0.469645in}{1.399676in}}%
\pgfpathclose%
\pgfusepath{fill}%
\end{pgfscope}%
\begin{pgfscope}%
\pgfpathrectangle{\pgfqpoint{0.100000in}{0.100000in}}{\pgfqpoint{3.420221in}{2.189500in}}%
\pgfusepath{clip}%
\pgfsetbuttcap%
\pgfsetmiterjoin%
\definecolor{currentfill}{rgb}{0.000000,0.254902,0.872549}%
\pgfsetfillcolor{currentfill}%
\pgfsetlinewidth{0.000000pt}%
\definecolor{currentstroke}{rgb}{0.000000,0.000000,0.000000}%
\pgfsetstrokecolor{currentstroke}%
\pgfsetstrokeopacity{0.000000}%
\pgfsetdash{}{0pt}%
\pgfpathmoveto{\pgfqpoint{1.793212in}{1.826664in}}%
\pgfpathlineto{\pgfqpoint{1.791430in}{1.826755in}}%
\pgfpathlineto{\pgfqpoint{1.792641in}{1.852968in}}%
\pgfpathlineto{\pgfqpoint{1.771221in}{1.854064in}}%
\pgfpathlineto{\pgfqpoint{1.772681in}{1.880434in}}%
\pgfpathlineto{\pgfqpoint{1.770893in}{1.880507in}}%
\pgfpathlineto{\pgfqpoint{1.772202in}{1.906702in}}%
\pgfpathlineto{\pgfqpoint{1.783199in}{1.906173in}}%
\pgfpathlineto{\pgfqpoint{1.822533in}{1.904442in}}%
\pgfpathlineto{\pgfqpoint{1.823915in}{1.897869in}}%
\pgfpathlineto{\pgfqpoint{1.849927in}{1.896953in}}%
\pgfpathlineto{\pgfqpoint{1.863028in}{1.896506in}}%
\pgfpathlineto{\pgfqpoint{1.863696in}{1.876696in}}%
\pgfpathlineto{\pgfqpoint{1.862904in}{1.850337in}}%
\pgfpathlineto{\pgfqpoint{1.844653in}{1.850966in}}%
\pgfpathlineto{\pgfqpoint{1.843740in}{1.824676in}}%
\pgfpathclose%
\pgfusepath{fill}%
\end{pgfscope}%
\begin{pgfscope}%
\pgfpathrectangle{\pgfqpoint{0.100000in}{0.100000in}}{\pgfqpoint{3.420221in}{2.189500in}}%
\pgfusepath{clip}%
\pgfsetbuttcap%
\pgfsetmiterjoin%
\definecolor{currentfill}{rgb}{0.000000,0.415686,0.792157}%
\pgfsetfillcolor{currentfill}%
\pgfsetlinewidth{0.000000pt}%
\definecolor{currentstroke}{rgb}{0.000000,0.000000,0.000000}%
\pgfsetstrokecolor{currentstroke}%
\pgfsetstrokeopacity{0.000000}%
\pgfsetdash{}{0pt}%
\pgfpathmoveto{\pgfqpoint{3.247365in}{1.742708in}}%
\pgfpathlineto{\pgfqpoint{3.240088in}{1.744957in}}%
\pgfpathlineto{\pgfqpoint{3.228870in}{1.740212in}}%
\pgfpathlineto{\pgfqpoint{3.224432in}{1.744940in}}%
\pgfpathlineto{\pgfqpoint{3.211256in}{1.740781in}}%
\pgfpathlineto{\pgfqpoint{3.203559in}{1.741416in}}%
\pgfpathlineto{\pgfqpoint{3.200162in}{1.744073in}}%
\pgfpathlineto{\pgfqpoint{3.204275in}{1.749569in}}%
\pgfpathlineto{\pgfqpoint{3.204565in}{1.758178in}}%
\pgfpathlineto{\pgfqpoint{3.200707in}{1.759250in}}%
\pgfpathlineto{\pgfqpoint{3.202044in}{1.774135in}}%
\pgfpathlineto{\pgfqpoint{3.195146in}{1.775046in}}%
\pgfpathlineto{\pgfqpoint{3.195514in}{1.781903in}}%
\pgfpathlineto{\pgfqpoint{3.184410in}{1.784119in}}%
\pgfpathlineto{\pgfqpoint{3.188825in}{1.787818in}}%
\pgfpathlineto{\pgfqpoint{3.193582in}{1.798886in}}%
\pgfpathlineto{\pgfqpoint{3.183630in}{1.799703in}}%
\pgfpathlineto{\pgfqpoint{3.181039in}{1.806162in}}%
\pgfpathlineto{\pgfqpoint{3.180106in}{1.815085in}}%
\pgfpathlineto{\pgfqpoint{3.182576in}{1.828873in}}%
\pgfpathlineto{\pgfqpoint{3.191403in}{1.826934in}}%
\pgfpathlineto{\pgfqpoint{3.192917in}{1.833605in}}%
\pgfpathlineto{\pgfqpoint{3.199438in}{1.831090in}}%
\pgfpathlineto{\pgfqpoint{3.200881in}{1.837697in}}%
\pgfpathlineto{\pgfqpoint{3.213344in}{1.834548in}}%
\pgfpathlineto{\pgfqpoint{3.209914in}{1.816543in}}%
\pgfpathlineto{\pgfqpoint{3.213614in}{1.815862in}}%
\pgfpathlineto{\pgfqpoint{3.225967in}{1.819545in}}%
\pgfpathlineto{\pgfqpoint{3.224309in}{1.825639in}}%
\pgfpathlineto{\pgfqpoint{3.227465in}{1.831313in}}%
\pgfpathlineto{\pgfqpoint{3.233333in}{1.832866in}}%
\pgfpathlineto{\pgfqpoint{3.237577in}{1.839415in}}%
\pgfpathlineto{\pgfqpoint{3.242536in}{1.834934in}}%
\pgfpathlineto{\pgfqpoint{3.248766in}{1.834429in}}%
\pgfpathlineto{\pgfqpoint{3.250318in}{1.830941in}}%
\pgfpathlineto{\pgfqpoint{3.257321in}{1.832372in}}%
\pgfpathlineto{\pgfqpoint{3.259472in}{1.830046in}}%
\pgfpathlineto{\pgfqpoint{3.264796in}{1.820324in}}%
\pgfpathlineto{\pgfqpoint{3.268017in}{1.809242in}}%
\pgfpathlineto{\pgfqpoint{3.257833in}{1.804798in}}%
\pgfpathlineto{\pgfqpoint{3.263819in}{1.791144in}}%
\pgfpathlineto{\pgfqpoint{3.257404in}{1.788597in}}%
\pgfpathlineto{\pgfqpoint{3.258505in}{1.783840in}}%
\pgfpathlineto{\pgfqpoint{3.251189in}{1.776475in}}%
\pgfpathlineto{\pgfqpoint{3.248237in}{1.780405in}}%
\pgfpathlineto{\pgfqpoint{3.242994in}{1.765491in}}%
\pgfpathlineto{\pgfqpoint{3.240982in}{1.756093in}}%
\pgfpathlineto{\pgfqpoint{3.247802in}{1.747403in}}%
\pgfpathclose%
\pgfusepath{fill}%
\end{pgfscope}%
\begin{pgfscope}%
\pgfpathrectangle{\pgfqpoint{0.100000in}{0.100000in}}{\pgfqpoint{3.420221in}{2.189500in}}%
\pgfusepath{clip}%
\pgfsetbuttcap%
\pgfsetmiterjoin%
\definecolor{currentfill}{rgb}{0.000000,0.494118,0.752941}%
\pgfsetfillcolor{currentfill}%
\pgfsetlinewidth{0.000000pt}%
\definecolor{currentstroke}{rgb}{0.000000,0.000000,0.000000}%
\pgfsetstrokecolor{currentstroke}%
\pgfsetstrokeopacity{0.000000}%
\pgfsetdash{}{0pt}%
\pgfpathmoveto{\pgfqpoint{2.889929in}{1.505930in}}%
\pgfpathlineto{\pgfqpoint{2.886670in}{1.502243in}}%
\pgfpathlineto{\pgfqpoint{2.875359in}{1.497930in}}%
\pgfpathlineto{\pgfqpoint{2.868011in}{1.497128in}}%
\pgfpathlineto{\pgfqpoint{2.861658in}{1.510354in}}%
\pgfpathlineto{\pgfqpoint{2.844711in}{1.507420in}}%
\pgfpathlineto{\pgfqpoint{2.840792in}{1.530951in}}%
\pgfpathlineto{\pgfqpoint{2.833947in}{1.529084in}}%
\pgfpathlineto{\pgfqpoint{2.811882in}{1.526972in}}%
\pgfpathlineto{\pgfqpoint{2.806184in}{1.562333in}}%
\pgfpathlineto{\pgfqpoint{2.823250in}{1.574156in}}%
\pgfpathlineto{\pgfqpoint{2.838617in}{1.587018in}}%
\pgfpathlineto{\pgfqpoint{2.848221in}{1.593674in}}%
\pgfpathlineto{\pgfqpoint{2.859561in}{1.604526in}}%
\pgfpathlineto{\pgfqpoint{2.868799in}{1.614991in}}%
\pgfpathlineto{\pgfqpoint{2.874496in}{1.618908in}}%
\pgfpathlineto{\pgfqpoint{2.879054in}{1.617283in}}%
\pgfpathlineto{\pgfqpoint{2.886080in}{1.577485in}}%
\pgfpathlineto{\pgfqpoint{2.893910in}{1.578792in}}%
\pgfpathlineto{\pgfqpoint{2.896155in}{1.567214in}}%
\pgfpathlineto{\pgfqpoint{2.894482in}{1.566120in}}%
\pgfpathlineto{\pgfqpoint{2.896798in}{1.550794in}}%
\pgfpathlineto{\pgfqpoint{2.897925in}{1.537886in}}%
\pgfpathlineto{\pgfqpoint{2.891827in}{1.534635in}}%
\pgfpathlineto{\pgfqpoint{2.892820in}{1.528533in}}%
\pgfpathlineto{\pgfqpoint{2.886621in}{1.526756in}}%
\pgfpathclose%
\pgfusepath{fill}%
\end{pgfscope}%
\begin{pgfscope}%
\pgfpathrectangle{\pgfqpoint{0.100000in}{0.100000in}}{\pgfqpoint{3.420221in}{2.189500in}}%
\pgfusepath{clip}%
\pgfsetbuttcap%
\pgfsetmiterjoin%
\definecolor{currentfill}{rgb}{0.000000,0.125490,0.937255}%
\pgfsetfillcolor{currentfill}%
\pgfsetlinewidth{0.000000pt}%
\definecolor{currentstroke}{rgb}{0.000000,0.000000,0.000000}%
\pgfsetstrokecolor{currentstroke}%
\pgfsetstrokeopacity{0.000000}%
\pgfsetdash{}{0pt}%
\pgfpathmoveto{\pgfqpoint{1.894224in}{1.429423in}}%
\pgfpathlineto{\pgfqpoint{1.868247in}{1.430099in}}%
\pgfpathlineto{\pgfqpoint{1.868928in}{1.456187in}}%
\pgfpathlineto{\pgfqpoint{1.855846in}{1.451895in}}%
\pgfpathlineto{\pgfqpoint{1.855980in}{1.456548in}}%
\pgfpathlineto{\pgfqpoint{1.850060in}{1.456712in}}%
\pgfpathlineto{\pgfqpoint{1.849771in}{1.450287in}}%
\pgfpathlineto{\pgfqpoint{1.830165in}{1.449697in}}%
\pgfpathlineto{\pgfqpoint{1.827355in}{1.447614in}}%
\pgfpathlineto{\pgfqpoint{1.817573in}{1.447983in}}%
\pgfpathlineto{\pgfqpoint{1.817016in}{1.457714in}}%
\pgfpathlineto{\pgfqpoint{1.817723in}{1.483763in}}%
\pgfpathlineto{\pgfqpoint{1.818204in}{1.496781in}}%
\pgfpathlineto{\pgfqpoint{1.844000in}{1.496017in}}%
\pgfpathlineto{\pgfqpoint{1.843731in}{1.482953in}}%
\pgfpathlineto{\pgfqpoint{1.889090in}{1.481804in}}%
\pgfpathlineto{\pgfqpoint{1.895456in}{1.481643in}}%
\pgfpathclose%
\pgfusepath{fill}%
\end{pgfscope}%
\begin{pgfscope}%
\pgfpathrectangle{\pgfqpoint{0.100000in}{0.100000in}}{\pgfqpoint{3.420221in}{2.189500in}}%
\pgfusepath{clip}%
\pgfsetbuttcap%
\pgfsetmiterjoin%
\definecolor{currentfill}{rgb}{0.000000,0.368627,0.815686}%
\pgfsetfillcolor{currentfill}%
\pgfsetlinewidth{0.000000pt}%
\definecolor{currentstroke}{rgb}{0.000000,0.000000,0.000000}%
\pgfsetstrokecolor{currentstroke}%
\pgfsetstrokeopacity{0.000000}%
\pgfsetdash{}{0pt}%
\pgfpathmoveto{\pgfqpoint{2.010002in}{1.638606in}}%
\pgfpathlineto{\pgfqpoint{2.009676in}{1.658152in}}%
\pgfpathlineto{\pgfqpoint{1.996698in}{1.658161in}}%
\pgfpathlineto{\pgfqpoint{1.996243in}{1.664701in}}%
\pgfpathlineto{\pgfqpoint{1.996323in}{1.671225in}}%
\pgfpathlineto{\pgfqpoint{2.009313in}{1.671263in}}%
\pgfpathlineto{\pgfqpoint{2.009313in}{1.687507in}}%
\pgfpathlineto{\pgfqpoint{2.013338in}{1.684387in}}%
\pgfpathlineto{\pgfqpoint{2.022294in}{1.684387in}}%
\pgfpathlineto{\pgfqpoint{2.059616in}{1.684719in}}%
\pgfpathlineto{\pgfqpoint{2.060534in}{1.691242in}}%
\pgfpathlineto{\pgfqpoint{2.081265in}{1.691557in}}%
\pgfpathlineto{\pgfqpoint{2.081641in}{1.665413in}}%
\pgfpathlineto{\pgfqpoint{2.068551in}{1.665207in}}%
\pgfpathlineto{\pgfqpoint{2.068917in}{1.639049in}}%
\pgfpathclose%
\pgfusepath{fill}%
\end{pgfscope}%
\begin{pgfscope}%
\pgfpathrectangle{\pgfqpoint{0.100000in}{0.100000in}}{\pgfqpoint{3.420221in}{2.189500in}}%
\pgfusepath{clip}%
\pgfsetbuttcap%
\pgfsetmiterjoin%
\definecolor{currentfill}{rgb}{0.000000,0.333333,0.833333}%
\pgfsetfillcolor{currentfill}%
\pgfsetlinewidth{0.000000pt}%
\definecolor{currentstroke}{rgb}{0.000000,0.000000,0.000000}%
\pgfsetstrokecolor{currentstroke}%
\pgfsetstrokeopacity{0.000000}%
\pgfsetdash{}{0pt}%
\pgfpathmoveto{\pgfqpoint{2.136517in}{1.395922in}}%
\pgfpathlineto{\pgfqpoint{2.135845in}{1.419001in}}%
\pgfpathlineto{\pgfqpoint{2.109881in}{1.418359in}}%
\pgfpathlineto{\pgfqpoint{2.109378in}{1.438016in}}%
\pgfpathlineto{\pgfqpoint{2.096481in}{1.437752in}}%
\pgfpathlineto{\pgfqpoint{2.095955in}{1.463778in}}%
\pgfpathlineto{\pgfqpoint{2.153835in}{1.465307in}}%
\pgfpathlineto{\pgfqpoint{2.173554in}{1.466014in}}%
\pgfpathlineto{\pgfqpoint{2.174444in}{1.439993in}}%
\pgfpathlineto{\pgfqpoint{2.187412in}{1.440395in}}%
\pgfpathlineto{\pgfqpoint{2.188147in}{1.399231in}}%
\pgfpathlineto{\pgfqpoint{2.176000in}{1.398222in}}%
\pgfpathclose%
\pgfusepath{fill}%
\end{pgfscope}%
\begin{pgfscope}%
\pgfpathrectangle{\pgfqpoint{0.100000in}{0.100000in}}{\pgfqpoint{3.420221in}{2.189500in}}%
\pgfusepath{clip}%
\pgfsetbuttcap%
\pgfsetmiterjoin%
\definecolor{currentfill}{rgb}{0.000000,0.576471,0.711765}%
\pgfsetfillcolor{currentfill}%
\pgfsetlinewidth{0.000000pt}%
\definecolor{currentstroke}{rgb}{0.000000,0.000000,0.000000}%
\pgfsetstrokecolor{currentstroke}%
\pgfsetstrokeopacity{0.000000}%
\pgfsetdash{}{0pt}%
\pgfpathmoveto{\pgfqpoint{1.086956in}{1.583137in}}%
\pgfpathlineto{\pgfqpoint{1.088111in}{1.578889in}}%
\pgfpathlineto{\pgfqpoint{1.084931in}{1.572303in}}%
\pgfpathlineto{\pgfqpoint{1.086753in}{1.569839in}}%
\pgfpathlineto{\pgfqpoint{1.088100in}{1.556587in}}%
\pgfpathlineto{\pgfqpoint{1.084160in}{1.547947in}}%
\pgfpathlineto{\pgfqpoint{1.078948in}{1.540654in}}%
\pgfpathlineto{\pgfqpoint{1.068615in}{1.542100in}}%
\pgfpathlineto{\pgfqpoint{1.064315in}{1.539244in}}%
\pgfpathlineto{\pgfqpoint{1.057759in}{1.544217in}}%
\pgfpathlineto{\pgfqpoint{1.049135in}{1.539609in}}%
\pgfpathlineto{\pgfqpoint{1.037555in}{1.541841in}}%
\pgfpathlineto{\pgfqpoint{1.019705in}{1.525441in}}%
\pgfpathlineto{\pgfqpoint{1.001627in}{1.523152in}}%
\pgfpathlineto{\pgfqpoint{0.932755in}{1.537283in}}%
\pgfpathlineto{\pgfqpoint{0.948279in}{1.610168in}}%
\pgfpathlineto{\pgfqpoint{0.957238in}{1.607833in}}%
\pgfpathlineto{\pgfqpoint{1.005175in}{1.598755in}}%
\pgfpathlineto{\pgfqpoint{1.010036in}{1.622975in}}%
\pgfpathlineto{\pgfqpoint{1.028927in}{1.619326in}}%
\pgfpathlineto{\pgfqpoint{1.030189in}{1.625741in}}%
\pgfpathlineto{\pgfqpoint{1.038681in}{1.624077in}}%
\pgfpathlineto{\pgfqpoint{1.039911in}{1.630495in}}%
\pgfpathlineto{\pgfqpoint{1.046079in}{1.629285in}}%
\pgfpathlineto{\pgfqpoint{1.049680in}{1.620534in}}%
\pgfpathlineto{\pgfqpoint{1.057055in}{1.610550in}}%
\pgfpathlineto{\pgfqpoint{1.064801in}{1.609097in}}%
\pgfpathlineto{\pgfqpoint{1.069149in}{1.606068in}}%
\pgfpathlineto{\pgfqpoint{1.072544in}{1.617623in}}%
\pgfpathlineto{\pgfqpoint{1.087622in}{1.614775in}}%
\pgfpathlineto{\pgfqpoint{1.089287in}{1.609600in}}%
\pgfpathlineto{\pgfqpoint{1.087583in}{1.603916in}}%
\pgfpathlineto{\pgfqpoint{1.083291in}{1.599763in}}%
\pgfpathlineto{\pgfqpoint{1.083457in}{1.590954in}}%
\pgfpathlineto{\pgfqpoint{1.087197in}{1.588806in}}%
\pgfpathclose%
\pgfusepath{fill}%
\end{pgfscope}%
\begin{pgfscope}%
\pgfpathrectangle{\pgfqpoint{0.100000in}{0.100000in}}{\pgfqpoint{3.420221in}{2.189500in}}%
\pgfusepath{clip}%
\pgfsetbuttcap%
\pgfsetmiterjoin%
\definecolor{currentfill}{rgb}{0.000000,0.647059,0.676471}%
\pgfsetfillcolor{currentfill}%
\pgfsetlinewidth{0.000000pt}%
\definecolor{currentstroke}{rgb}{0.000000,0.000000,0.000000}%
\pgfsetstrokecolor{currentstroke}%
\pgfsetstrokeopacity{0.000000}%
\pgfsetdash{}{0pt}%
\pgfpathmoveto{\pgfqpoint{2.669625in}{1.255899in}}%
\pgfpathlineto{\pgfqpoint{2.663005in}{1.262200in}}%
\pgfpathlineto{\pgfqpoint{2.656168in}{1.262476in}}%
\pgfpathlineto{\pgfqpoint{2.646926in}{1.273074in}}%
\pgfpathlineto{\pgfqpoint{2.649610in}{1.276432in}}%
\pgfpathlineto{\pgfqpoint{2.646568in}{1.282870in}}%
\pgfpathlineto{\pgfqpoint{2.639780in}{1.283243in}}%
\pgfpathlineto{\pgfqpoint{2.632795in}{1.281175in}}%
\pgfpathlineto{\pgfqpoint{2.628882in}{1.298824in}}%
\pgfpathlineto{\pgfqpoint{2.637696in}{1.296257in}}%
\pgfpathlineto{\pgfqpoint{2.644114in}{1.298006in}}%
\pgfpathlineto{\pgfqpoint{2.650464in}{1.296921in}}%
\pgfpathlineto{\pgfqpoint{2.657473in}{1.289695in}}%
\pgfpathlineto{\pgfqpoint{2.663718in}{1.288412in}}%
\pgfpathlineto{\pgfqpoint{2.666174in}{1.292986in}}%
\pgfpathlineto{\pgfqpoint{2.671440in}{1.295060in}}%
\pgfpathlineto{\pgfqpoint{2.686944in}{1.290644in}}%
\pgfpathlineto{\pgfqpoint{2.693351in}{1.291655in}}%
\pgfpathlineto{\pgfqpoint{2.699633in}{1.300420in}}%
\pgfpathlineto{\pgfqpoint{2.698592in}{1.290419in}}%
\pgfpathlineto{\pgfqpoint{2.692976in}{1.282205in}}%
\pgfpathlineto{\pgfqpoint{2.690654in}{1.275872in}}%
\pgfpathlineto{\pgfqpoint{2.691790in}{1.270218in}}%
\pgfpathlineto{\pgfqpoint{2.684626in}{1.267955in}}%
\pgfpathlineto{\pgfqpoint{2.680403in}{1.273026in}}%
\pgfpathlineto{\pgfqpoint{2.676031in}{1.268899in}}%
\pgfpathlineto{\pgfqpoint{2.676259in}{1.263201in}}%
\pgfpathclose%
\pgfusepath{fill}%
\end{pgfscope}%
\begin{pgfscope}%
\pgfpathrectangle{\pgfqpoint{0.100000in}{0.100000in}}{\pgfqpoint{3.420221in}{2.189500in}}%
\pgfusepath{clip}%
\pgfsetbuttcap%
\pgfsetmiterjoin%
\definecolor{currentfill}{rgb}{0.000000,0.439216,0.780392}%
\pgfsetfillcolor{currentfill}%
\pgfsetlinewidth{0.000000pt}%
\definecolor{currentstroke}{rgb}{0.000000,0.000000,0.000000}%
\pgfsetstrokecolor{currentstroke}%
\pgfsetstrokeopacity{0.000000}%
\pgfsetdash{}{0pt}%
\pgfpathmoveto{\pgfqpoint{2.696323in}{0.830472in}}%
\pgfpathlineto{\pgfqpoint{2.692114in}{0.828749in}}%
\pgfpathlineto{\pgfqpoint{2.688279in}{0.820302in}}%
\pgfpathlineto{\pgfqpoint{2.680569in}{0.817230in}}%
\pgfpathlineto{\pgfqpoint{2.674578in}{0.821981in}}%
\pgfpathlineto{\pgfqpoint{2.671287in}{0.830538in}}%
\pgfpathlineto{\pgfqpoint{2.679603in}{0.845589in}}%
\pgfpathlineto{\pgfqpoint{2.685397in}{0.841915in}}%
\pgfpathlineto{\pgfqpoint{2.687756in}{0.850637in}}%
\pgfpathlineto{\pgfqpoint{2.688212in}{0.860632in}}%
\pgfpathlineto{\pgfqpoint{2.693043in}{0.862407in}}%
\pgfpathlineto{\pgfqpoint{2.690899in}{0.881475in}}%
\pgfpathlineto{\pgfqpoint{2.701749in}{0.882747in}}%
\pgfpathlineto{\pgfqpoint{2.704826in}{0.881484in}}%
\pgfpathlineto{\pgfqpoint{2.705662in}{0.877911in}}%
\pgfpathlineto{\pgfqpoint{2.722210in}{0.883017in}}%
\pgfpathlineto{\pgfqpoint{2.729356in}{0.884961in}}%
\pgfpathlineto{\pgfqpoint{2.736311in}{0.866162in}}%
\pgfpathlineto{\pgfqpoint{2.733474in}{0.863678in}}%
\pgfpathlineto{\pgfqpoint{2.747861in}{0.841676in}}%
\pgfpathlineto{\pgfqpoint{2.754960in}{0.830347in}}%
\pgfpathlineto{\pgfqpoint{2.752509in}{0.832267in}}%
\pgfpathlineto{\pgfqpoint{2.745283in}{0.822917in}}%
\pgfpathlineto{\pgfqpoint{2.741795in}{0.815449in}}%
\pgfpathlineto{\pgfqpoint{2.747718in}{0.810831in}}%
\pgfpathlineto{\pgfqpoint{2.745504in}{0.804988in}}%
\pgfpathlineto{\pgfqpoint{2.728198in}{0.803727in}}%
\pgfpathlineto{\pgfqpoint{2.726722in}{0.816477in}}%
\pgfpathlineto{\pgfqpoint{2.711388in}{0.814745in}}%
\pgfpathlineto{\pgfqpoint{2.710760in}{0.825929in}}%
\pgfpathlineto{\pgfqpoint{2.707714in}{0.829380in}}%
\pgfpathclose%
\pgfusepath{fill}%
\end{pgfscope}%
\begin{pgfscope}%
\pgfpathrectangle{\pgfqpoint{0.100000in}{0.100000in}}{\pgfqpoint{3.420221in}{2.189500in}}%
\pgfusepath{clip}%
\pgfsetbuttcap%
\pgfsetmiterjoin%
\definecolor{currentfill}{rgb}{0.000000,0.156863,0.921569}%
\pgfsetfillcolor{currentfill}%
\pgfsetlinewidth{0.000000pt}%
\definecolor{currentstroke}{rgb}{0.000000,0.000000,0.000000}%
\pgfsetstrokecolor{currentstroke}%
\pgfsetstrokeopacity{0.000000}%
\pgfsetdash{}{0pt}%
\pgfpathmoveto{\pgfqpoint{1.883810in}{1.738175in}}%
\pgfpathlineto{\pgfqpoint{1.843697in}{1.739436in}}%
\pgfpathlineto{\pgfqpoint{1.845789in}{1.798416in}}%
\pgfpathlineto{\pgfqpoint{1.885162in}{1.797233in}}%
\pgfpathlineto{\pgfqpoint{1.920061in}{1.796513in}}%
\pgfpathlineto{\pgfqpoint{1.918871in}{1.787838in}}%
\pgfpathlineto{\pgfqpoint{1.915135in}{1.782379in}}%
\pgfpathlineto{\pgfqpoint{1.905171in}{1.775078in}}%
\pgfpathlineto{\pgfqpoint{1.904100in}{1.771966in}}%
\pgfpathlineto{\pgfqpoint{1.913179in}{1.757047in}}%
\pgfpathlineto{\pgfqpoint{1.921532in}{1.754272in}}%
\pgfpathlineto{\pgfqpoint{1.924188in}{1.750527in}}%
\pgfpathlineto{\pgfqpoint{1.896507in}{1.751080in}}%
\pgfpathlineto{\pgfqpoint{1.895683in}{1.748835in}}%
\pgfpathlineto{\pgfqpoint{1.884078in}{1.749170in}}%
\pgfpathclose%
\pgfusepath{fill}%
\end{pgfscope}%
\begin{pgfscope}%
\pgfpathrectangle{\pgfqpoint{0.100000in}{0.100000in}}{\pgfqpoint{3.420221in}{2.189500in}}%
\pgfusepath{clip}%
\pgfsetbuttcap%
\pgfsetmiterjoin%
\definecolor{currentfill}{rgb}{0.000000,0.560784,0.719608}%
\pgfsetfillcolor{currentfill}%
\pgfsetlinewidth{0.000000pt}%
\definecolor{currentstroke}{rgb}{0.000000,0.000000,0.000000}%
\pgfsetstrokecolor{currentstroke}%
\pgfsetstrokeopacity{0.000000}%
\pgfsetdash{}{0pt}%
\pgfpathmoveto{\pgfqpoint{2.822137in}{1.463239in}}%
\pgfpathlineto{\pgfqpoint{2.814435in}{1.457734in}}%
\pgfpathlineto{\pgfqpoint{2.803330in}{1.457263in}}%
\pgfpathlineto{\pgfqpoint{2.799665in}{1.460032in}}%
\pgfpathlineto{\pgfqpoint{2.798753in}{1.466221in}}%
\pgfpathlineto{\pgfqpoint{2.789195in}{1.464808in}}%
\pgfpathlineto{\pgfqpoint{2.786247in}{1.484093in}}%
\pgfpathlineto{\pgfqpoint{2.791115in}{1.484818in}}%
\pgfpathlineto{\pgfqpoint{2.785011in}{1.522861in}}%
\pgfpathlineto{\pgfqpoint{2.811882in}{1.526972in}}%
\pgfpathlineto{\pgfqpoint{2.833947in}{1.529084in}}%
\pgfpathlineto{\pgfqpoint{2.840792in}{1.530951in}}%
\pgfpathlineto{\pgfqpoint{2.844711in}{1.507420in}}%
\pgfpathlineto{\pgfqpoint{2.837585in}{1.493217in}}%
\pgfpathlineto{\pgfqpoint{2.839733in}{1.482531in}}%
\pgfpathlineto{\pgfqpoint{2.819608in}{1.478966in}}%
\pgfpathclose%
\pgfusepath{fill}%
\end{pgfscope}%
\begin{pgfscope}%
\pgfpathrectangle{\pgfqpoint{0.100000in}{0.100000in}}{\pgfqpoint{3.420221in}{2.189500in}}%
\pgfusepath{clip}%
\pgfsetbuttcap%
\pgfsetmiterjoin%
\definecolor{currentfill}{rgb}{0.000000,0.384314,0.807843}%
\pgfsetfillcolor{currentfill}%
\pgfsetlinewidth{0.000000pt}%
\definecolor{currentstroke}{rgb}{0.000000,0.000000,0.000000}%
\pgfsetstrokecolor{currentstroke}%
\pgfsetstrokeopacity{0.000000}%
\pgfsetdash{}{0pt}%
\pgfpathmoveto{\pgfqpoint{1.563276in}{1.594266in}}%
\pgfpathlineto{\pgfqpoint{1.505742in}{1.599669in}}%
\pgfpathlineto{\pgfqpoint{1.509406in}{1.637237in}}%
\pgfpathlineto{\pgfqpoint{1.514487in}{1.687903in}}%
\pgfpathlineto{\pgfqpoint{1.520434in}{1.749117in}}%
\pgfpathlineto{\pgfqpoint{1.522987in}{1.765179in}}%
\pgfpathlineto{\pgfqpoint{1.580186in}{1.759793in}}%
\pgfpathlineto{\pgfqpoint{1.579054in}{1.746777in}}%
\pgfpathlineto{\pgfqpoint{1.629845in}{1.742526in}}%
\pgfpathlineto{\pgfqpoint{1.626772in}{1.702970in}}%
\pgfpathlineto{\pgfqpoint{1.623407in}{1.664221in}}%
\pgfpathlineto{\pgfqpoint{1.621113in}{1.642783in}}%
\pgfpathlineto{\pgfqpoint{1.614564in}{1.642760in}}%
\pgfpathlineto{\pgfqpoint{1.612382in}{1.641929in}}%
\pgfpathlineto{\pgfqpoint{1.578137in}{1.644808in}}%
\pgfpathlineto{\pgfqpoint{1.573103in}{1.644246in}}%
\pgfpathlineto{\pgfqpoint{1.567264in}{1.639556in}}%
\pgfpathclose%
\pgfusepath{fill}%
\end{pgfscope}%
\begin{pgfscope}%
\pgfpathrectangle{\pgfqpoint{0.100000in}{0.100000in}}{\pgfqpoint{3.420221in}{2.189500in}}%
\pgfusepath{clip}%
\pgfsetbuttcap%
\pgfsetmiterjoin%
\definecolor{currentfill}{rgb}{0.000000,0.803922,0.598039}%
\pgfsetfillcolor{currentfill}%
\pgfsetlinewidth{0.000000pt}%
\definecolor{currentstroke}{rgb}{0.000000,0.000000,0.000000}%
\pgfsetstrokecolor{currentstroke}%
\pgfsetstrokeopacity{0.000000}%
\pgfsetdash{}{0pt}%
\pgfpathmoveto{\pgfqpoint{2.723344in}{1.142396in}}%
\pgfpathlineto{\pgfqpoint{2.715776in}{1.135184in}}%
\pgfpathlineto{\pgfqpoint{2.712529in}{1.128154in}}%
\pgfpathlineto{\pgfqpoint{2.707627in}{1.125277in}}%
\pgfpathlineto{\pgfqpoint{2.703105in}{1.126591in}}%
\pgfpathlineto{\pgfqpoint{2.693843in}{1.121300in}}%
\pgfpathlineto{\pgfqpoint{2.684932in}{1.118580in}}%
\pgfpathlineto{\pgfqpoint{2.672485in}{1.123858in}}%
\pgfpathlineto{\pgfqpoint{2.668871in}{1.122893in}}%
\pgfpathlineto{\pgfqpoint{2.663498in}{1.134462in}}%
\pgfpathlineto{\pgfqpoint{2.664871in}{1.138828in}}%
\pgfpathlineto{\pgfqpoint{2.671216in}{1.144736in}}%
\pgfpathlineto{\pgfqpoint{2.673572in}{1.151623in}}%
\pgfpathlineto{\pgfqpoint{2.688809in}{1.163964in}}%
\pgfpathlineto{\pgfqpoint{2.691271in}{1.157522in}}%
\pgfpathlineto{\pgfqpoint{2.691235in}{1.148781in}}%
\pgfpathlineto{\pgfqpoint{2.694271in}{1.143988in}}%
\pgfpathlineto{\pgfqpoint{2.685754in}{1.140807in}}%
\pgfpathlineto{\pgfqpoint{2.685277in}{1.137791in}}%
\pgfpathclose%
\pgfusepath{fill}%
\end{pgfscope}%
\begin{pgfscope}%
\pgfpathrectangle{\pgfqpoint{0.100000in}{0.100000in}}{\pgfqpoint{3.420221in}{2.189500in}}%
\pgfusepath{clip}%
\pgfsetbuttcap%
\pgfsetmiterjoin%
\definecolor{currentfill}{rgb}{0.000000,0.607843,0.696078}%
\pgfsetfillcolor{currentfill}%
\pgfsetlinewidth{0.000000pt}%
\definecolor{currentstroke}{rgb}{0.000000,0.000000,0.000000}%
\pgfsetstrokecolor{currentstroke}%
\pgfsetstrokeopacity{0.000000}%
\pgfsetdash{}{0pt}%
\pgfpathmoveto{\pgfqpoint{2.700065in}{1.005393in}}%
\pgfpathlineto{\pgfqpoint{2.690669in}{1.001428in}}%
\pgfpathlineto{\pgfqpoint{2.688489in}{0.996322in}}%
\pgfpathlineto{\pgfqpoint{2.683256in}{0.997095in}}%
\pgfpathlineto{\pgfqpoint{2.678182in}{0.988987in}}%
\pgfpathlineto{\pgfqpoint{2.670692in}{0.988606in}}%
\pgfpathlineto{\pgfqpoint{2.673264in}{1.000477in}}%
\pgfpathlineto{\pgfqpoint{2.667025in}{1.011688in}}%
\pgfpathlineto{\pgfqpoint{2.657615in}{1.012741in}}%
\pgfpathlineto{\pgfqpoint{2.657454in}{1.030655in}}%
\pgfpathlineto{\pgfqpoint{2.661119in}{1.034380in}}%
\pgfpathlineto{\pgfqpoint{2.666516in}{1.033241in}}%
\pgfpathlineto{\pgfqpoint{2.672754in}{1.037526in}}%
\pgfpathlineto{\pgfqpoint{2.677716in}{1.032437in}}%
\pgfpathlineto{\pgfqpoint{2.685161in}{1.036301in}}%
\pgfpathlineto{\pgfqpoint{2.693020in}{1.036722in}}%
\pgfpathlineto{\pgfqpoint{2.691717in}{1.029547in}}%
\pgfpathlineto{\pgfqpoint{2.697265in}{1.030022in}}%
\pgfpathlineto{\pgfqpoint{2.706657in}{1.019763in}}%
\pgfpathlineto{\pgfqpoint{2.699411in}{1.009368in}}%
\pgfpathclose%
\pgfusepath{fill}%
\end{pgfscope}%
\begin{pgfscope}%
\pgfpathrectangle{\pgfqpoint{0.100000in}{0.100000in}}{\pgfqpoint{3.420221in}{2.189500in}}%
\pgfusepath{clip}%
\pgfsetbuttcap%
\pgfsetmiterjoin%
\definecolor{currentfill}{rgb}{0.000000,0.360784,0.819608}%
\pgfsetfillcolor{currentfill}%
\pgfsetlinewidth{0.000000pt}%
\definecolor{currentstroke}{rgb}{0.000000,0.000000,0.000000}%
\pgfsetstrokecolor{currentstroke}%
\pgfsetstrokeopacity{0.000000}%
\pgfsetdash{}{0pt}%
\pgfpathmoveto{\pgfqpoint{2.727854in}{1.475723in}}%
\pgfpathlineto{\pgfqpoint{2.724745in}{1.480691in}}%
\pgfpathlineto{\pgfqpoint{2.724060in}{1.486161in}}%
\pgfpathlineto{\pgfqpoint{2.728837in}{1.491597in}}%
\pgfpathlineto{\pgfqpoint{2.734310in}{1.492383in}}%
\pgfpathlineto{\pgfqpoint{2.733610in}{1.498004in}}%
\pgfpathlineto{\pgfqpoint{2.738885in}{1.498746in}}%
\pgfpathlineto{\pgfqpoint{2.738125in}{1.504353in}}%
\pgfpathlineto{\pgfqpoint{2.732864in}{1.503652in}}%
\pgfpathlineto{\pgfqpoint{2.731399in}{1.515170in}}%
\pgfpathlineto{\pgfqpoint{2.744317in}{1.515715in}}%
\pgfpathlineto{\pgfqpoint{2.751246in}{1.522205in}}%
\pgfpathlineto{\pgfqpoint{2.758727in}{1.531893in}}%
\pgfpathlineto{\pgfqpoint{2.780552in}{1.548854in}}%
\pgfpathlineto{\pgfqpoint{2.786264in}{1.550843in}}%
\pgfpathlineto{\pgfqpoint{2.806184in}{1.562333in}}%
\pgfpathlineto{\pgfqpoint{2.811882in}{1.526972in}}%
\pgfpathlineto{\pgfqpoint{2.785011in}{1.522861in}}%
\pgfpathlineto{\pgfqpoint{2.791115in}{1.484818in}}%
\pgfpathlineto{\pgfqpoint{2.786247in}{1.484093in}}%
\pgfpathlineto{\pgfqpoint{2.767812in}{1.481306in}}%
\pgfpathlineto{\pgfqpoint{2.768512in}{1.475211in}}%
\pgfpathlineto{\pgfqpoint{2.755670in}{1.473877in}}%
\pgfpathlineto{\pgfqpoint{2.754890in}{1.479339in}}%
\pgfpathclose%
\pgfusepath{fill}%
\end{pgfscope}%
\begin{pgfscope}%
\pgfpathrectangle{\pgfqpoint{0.100000in}{0.100000in}}{\pgfqpoint{3.420221in}{2.189500in}}%
\pgfusepath{clip}%
\pgfsetbuttcap%
\pgfsetmiterjoin%
\definecolor{currentfill}{rgb}{0.000000,0.588235,0.705882}%
\pgfsetfillcolor{currentfill}%
\pgfsetlinewidth{0.000000pt}%
\definecolor{currentstroke}{rgb}{0.000000,0.000000,0.000000}%
\pgfsetstrokecolor{currentstroke}%
\pgfsetstrokeopacity{0.000000}%
\pgfsetdash{}{0pt}%
\pgfpathmoveto{\pgfqpoint{0.703950in}{1.902176in}}%
\pgfpathlineto{\pgfqpoint{0.697846in}{1.903896in}}%
\pgfpathlineto{\pgfqpoint{0.702939in}{1.916002in}}%
\pgfpathlineto{\pgfqpoint{0.696694in}{1.917914in}}%
\pgfpathlineto{\pgfqpoint{0.698503in}{1.924161in}}%
\pgfpathlineto{\pgfqpoint{0.692166in}{1.926063in}}%
\pgfpathlineto{\pgfqpoint{0.703905in}{1.966239in}}%
\pgfpathlineto{\pgfqpoint{0.711034in}{1.966073in}}%
\pgfpathlineto{\pgfqpoint{0.715303in}{1.980922in}}%
\pgfpathlineto{\pgfqpoint{0.726916in}{2.023792in}}%
\pgfpathlineto{\pgfqpoint{0.739642in}{2.021524in}}%
\pgfpathlineto{\pgfqpoint{0.746829in}{2.025999in}}%
\pgfpathlineto{\pgfqpoint{0.748980in}{2.021627in}}%
\pgfpathlineto{\pgfqpoint{0.754324in}{2.024708in}}%
\pgfpathlineto{\pgfqpoint{0.785388in}{2.016141in}}%
\pgfpathlineto{\pgfqpoint{0.812406in}{2.009276in}}%
\pgfpathlineto{\pgfqpoint{0.810327in}{2.009397in}}%
\pgfpathlineto{\pgfqpoint{0.810807in}{1.995474in}}%
\pgfpathlineto{\pgfqpoint{0.828218in}{1.996264in}}%
\pgfpathlineto{\pgfqpoint{0.824750in}{1.985151in}}%
\pgfpathlineto{\pgfqpoint{0.831107in}{1.983555in}}%
\pgfpathlineto{\pgfqpoint{0.831748in}{1.972963in}}%
\pgfpathlineto{\pgfqpoint{0.834963in}{1.972213in}}%
\pgfpathlineto{\pgfqpoint{0.829088in}{1.947246in}}%
\pgfpathlineto{\pgfqpoint{0.810096in}{1.952255in}}%
\pgfpathlineto{\pgfqpoint{0.807697in}{1.941951in}}%
\pgfpathlineto{\pgfqpoint{0.803206in}{1.939723in}}%
\pgfpathlineto{\pgfqpoint{0.799657in}{1.930610in}}%
\pgfpathlineto{\pgfqpoint{0.797093in}{1.931286in}}%
\pgfpathlineto{\pgfqpoint{0.792965in}{1.915392in}}%
\pgfpathlineto{\pgfqpoint{0.787995in}{1.913290in}}%
\pgfpathlineto{\pgfqpoint{0.776233in}{1.916515in}}%
\pgfpathlineto{\pgfqpoint{0.774825in}{1.911133in}}%
\pgfpathlineto{\pgfqpoint{0.760837in}{1.914063in}}%
\pgfpathlineto{\pgfqpoint{0.760106in}{1.902690in}}%
\pgfpathlineto{\pgfqpoint{0.765706in}{1.901210in}}%
\pgfpathlineto{\pgfqpoint{0.763248in}{1.886195in}}%
\pgfpathclose%
\pgfusepath{fill}%
\end{pgfscope}%
\begin{pgfscope}%
\pgfpathrectangle{\pgfqpoint{0.100000in}{0.100000in}}{\pgfqpoint{3.420221in}{2.189500in}}%
\pgfusepath{clip}%
\pgfsetbuttcap%
\pgfsetmiterjoin%
\definecolor{currentfill}{rgb}{0.000000,0.635294,0.682353}%
\pgfsetfillcolor{currentfill}%
\pgfsetlinewidth{0.000000pt}%
\definecolor{currentstroke}{rgb}{0.000000,0.000000,0.000000}%
\pgfsetstrokecolor{currentstroke}%
\pgfsetstrokeopacity{0.000000}%
\pgfsetdash{}{0pt}%
\pgfpathmoveto{\pgfqpoint{1.523533in}{1.194617in}}%
\pgfpathlineto{\pgfqpoint{1.518648in}{1.146655in}}%
\pgfpathlineto{\pgfqpoint{1.463680in}{1.151536in}}%
\pgfpathlineto{\pgfqpoint{1.439814in}{1.153754in}}%
\pgfpathlineto{\pgfqpoint{1.395369in}{1.158691in}}%
\pgfpathlineto{\pgfqpoint{1.397910in}{1.180875in}}%
\pgfpathlineto{\pgfqpoint{1.399733in}{1.188783in}}%
\pgfpathlineto{\pgfqpoint{1.398337in}{1.200654in}}%
\pgfpathlineto{\pgfqpoint{1.392609in}{1.208657in}}%
\pgfpathlineto{\pgfqpoint{1.380281in}{1.207129in}}%
\pgfpathlineto{\pgfqpoint{1.386428in}{1.224315in}}%
\pgfpathlineto{\pgfqpoint{1.384205in}{1.227909in}}%
\pgfpathlineto{\pgfqpoint{1.387657in}{1.227100in}}%
\pgfpathlineto{\pgfqpoint{1.398552in}{1.231310in}}%
\pgfpathlineto{\pgfqpoint{1.403125in}{1.235025in}}%
\pgfpathlineto{\pgfqpoint{1.409235in}{1.226541in}}%
\pgfpathlineto{\pgfqpoint{1.411079in}{1.223758in}}%
\pgfpathlineto{\pgfqpoint{1.432830in}{1.222847in}}%
\pgfpathlineto{\pgfqpoint{1.466119in}{1.206877in}}%
\pgfpathlineto{\pgfqpoint{1.465511in}{1.200150in}}%
\pgfpathclose%
\pgfusepath{fill}%
\end{pgfscope}%
\begin{pgfscope}%
\pgfpathrectangle{\pgfqpoint{0.100000in}{0.100000in}}{\pgfqpoint{3.420221in}{2.189500in}}%
\pgfusepath{clip}%
\pgfsetbuttcap%
\pgfsetmiterjoin%
\definecolor{currentfill}{rgb}{0.000000,0.227451,0.886275}%
\pgfsetfillcolor{currentfill}%
\pgfsetlinewidth{0.000000pt}%
\definecolor{currentstroke}{rgb}{0.000000,0.000000,0.000000}%
\pgfsetstrokecolor{currentstroke}%
\pgfsetstrokeopacity{0.000000}%
\pgfsetdash{}{0pt}%
\pgfpathmoveto{\pgfqpoint{2.230123in}{1.182205in}}%
\pgfpathlineto{\pgfqpoint{2.220795in}{1.182022in}}%
\pgfpathlineto{\pgfqpoint{2.220669in}{1.188552in}}%
\pgfpathlineto{\pgfqpoint{2.207617in}{1.188172in}}%
\pgfpathlineto{\pgfqpoint{2.206799in}{1.215394in}}%
\pgfpathlineto{\pgfqpoint{2.199738in}{1.215222in}}%
\pgfpathlineto{\pgfqpoint{2.199375in}{1.225317in}}%
\pgfpathlineto{\pgfqpoint{2.198482in}{1.256419in}}%
\pgfpathlineto{\pgfqpoint{2.203803in}{1.254555in}}%
\pgfpathlineto{\pgfqpoint{2.211418in}{1.257362in}}%
\pgfpathlineto{\pgfqpoint{2.211182in}{1.267514in}}%
\pgfpathlineto{\pgfqpoint{2.219979in}{1.267758in}}%
\pgfpathlineto{\pgfqpoint{2.219345in}{1.289927in}}%
\pgfpathlineto{\pgfqpoint{2.223684in}{1.290036in}}%
\pgfpathlineto{\pgfqpoint{2.223557in}{1.296655in}}%
\pgfpathlineto{\pgfqpoint{2.250300in}{1.297612in}}%
\pgfpathlineto{\pgfqpoint{2.253049in}{1.287560in}}%
\pgfpathlineto{\pgfqpoint{2.251377in}{1.284903in}}%
\pgfpathlineto{\pgfqpoint{2.255131in}{1.275053in}}%
\pgfpathlineto{\pgfqpoint{2.261888in}{1.271989in}}%
\pgfpathlineto{\pgfqpoint{2.266972in}{1.279195in}}%
\pgfpathlineto{\pgfqpoint{2.278602in}{1.276244in}}%
\pgfpathlineto{\pgfqpoint{2.287079in}{1.271354in}}%
\pgfpathlineto{\pgfqpoint{2.282188in}{1.262303in}}%
\pgfpathlineto{\pgfqpoint{2.283868in}{1.257006in}}%
\pgfpathlineto{\pgfqpoint{2.311593in}{1.258160in}}%
\pgfpathlineto{\pgfqpoint{2.313716in}{1.225575in}}%
\pgfpathlineto{\pgfqpoint{2.294122in}{1.224753in}}%
\pgfpathlineto{\pgfqpoint{2.294503in}{1.218208in}}%
\pgfpathlineto{\pgfqpoint{2.284647in}{1.214102in}}%
\pgfpathlineto{\pgfqpoint{2.277406in}{1.214212in}}%
\pgfpathlineto{\pgfqpoint{2.272492in}{1.210049in}}%
\pgfpathlineto{\pgfqpoint{2.262772in}{1.206656in}}%
\pgfpathlineto{\pgfqpoint{2.250464in}{1.221119in}}%
\pgfpathlineto{\pgfqpoint{2.231903in}{1.220287in}}%
\pgfpathlineto{\pgfqpoint{2.233145in}{1.185535in}}%
\pgfpathclose%
\pgfusepath{fill}%
\end{pgfscope}%
\begin{pgfscope}%
\pgfpathrectangle{\pgfqpoint{0.100000in}{0.100000in}}{\pgfqpoint{3.420221in}{2.189500in}}%
\pgfusepath{clip}%
\pgfsetbuttcap%
\pgfsetmiterjoin%
\definecolor{currentfill}{rgb}{0.000000,0.188235,0.905882}%
\pgfsetfillcolor{currentfill}%
\pgfsetlinewidth{0.000000pt}%
\definecolor{currentstroke}{rgb}{0.000000,0.000000,0.000000}%
\pgfsetstrokecolor{currentstroke}%
\pgfsetstrokeopacity{0.000000}%
\pgfsetdash{}{0pt}%
\pgfpathmoveto{\pgfqpoint{2.024890in}{1.202876in}}%
\pgfpathlineto{\pgfqpoint{1.997524in}{1.202886in}}%
\pgfpathlineto{\pgfqpoint{1.998280in}{1.219448in}}%
\pgfpathlineto{\pgfqpoint{1.998265in}{1.252126in}}%
\pgfpathlineto{\pgfqpoint{1.998810in}{1.255392in}}%
\pgfpathlineto{\pgfqpoint{1.998833in}{1.273645in}}%
\pgfpathlineto{\pgfqpoint{2.007400in}{1.274311in}}%
\pgfpathlineto{\pgfqpoint{2.007892in}{1.290126in}}%
\pgfpathlineto{\pgfqpoint{2.011023in}{1.295104in}}%
\pgfpathlineto{\pgfqpoint{2.002832in}{1.308293in}}%
\pgfpathlineto{\pgfqpoint{1.996165in}{1.314943in}}%
\pgfpathlineto{\pgfqpoint{2.025149in}{1.314901in}}%
\pgfpathlineto{\pgfqpoint{2.025198in}{1.309186in}}%
\pgfpathlineto{\pgfqpoint{2.047769in}{1.309244in}}%
\pgfpathlineto{\pgfqpoint{2.047844in}{1.314677in}}%
\pgfpathlineto{\pgfqpoint{2.073810in}{1.314766in}}%
\pgfpathlineto{\pgfqpoint{2.074177in}{1.290983in}}%
\pgfpathlineto{\pgfqpoint{2.080017in}{1.293987in}}%
\pgfpathlineto{\pgfqpoint{2.089153in}{1.292032in}}%
\pgfpathlineto{\pgfqpoint{2.089714in}{1.271428in}}%
\pgfpathlineto{\pgfqpoint{2.089325in}{1.242445in}}%
\pgfpathlineto{\pgfqpoint{2.056918in}{1.242853in}}%
\pgfpathlineto{\pgfqpoint{2.056548in}{1.216411in}}%
\pgfpathlineto{\pgfqpoint{2.057682in}{1.203082in}}%
\pgfpathlineto{\pgfqpoint{2.046951in}{1.204051in}}%
\pgfpathlineto{\pgfqpoint{2.024878in}{1.204610in}}%
\pgfpathclose%
\pgfusepath{fill}%
\end{pgfscope}%
\begin{pgfscope}%
\pgfpathrectangle{\pgfqpoint{0.100000in}{0.100000in}}{\pgfqpoint{3.420221in}{2.189500in}}%
\pgfusepath{clip}%
\pgfsetbuttcap%
\pgfsetmiterjoin%
\definecolor{currentfill}{rgb}{0.000000,0.443137,0.778431}%
\pgfsetfillcolor{currentfill}%
\pgfsetlinewidth{0.000000pt}%
\definecolor{currentstroke}{rgb}{0.000000,0.000000,0.000000}%
\pgfsetstrokecolor{currentstroke}%
\pgfsetstrokeopacity{0.000000}%
\pgfsetdash{}{0pt}%
\pgfpathmoveto{\pgfqpoint{1.448618in}{2.061832in}}%
\pgfpathlineto{\pgfqpoint{1.500921in}{2.055722in}}%
\pgfpathlineto{\pgfqpoint{1.501131in}{2.048913in}}%
\pgfpathlineto{\pgfqpoint{1.498935in}{2.029236in}}%
\pgfpathlineto{\pgfqpoint{1.501467in}{2.022301in}}%
\pgfpathlineto{\pgfqpoint{1.459939in}{2.027077in}}%
\pgfpathlineto{\pgfqpoint{1.449108in}{2.028377in}}%
\pgfpathlineto{\pgfqpoint{1.451504in}{2.048053in}}%
\pgfpathlineto{\pgfqpoint{1.445012in}{2.048848in}}%
\pgfpathclose%
\pgfusepath{fill}%
\end{pgfscope}%
\begin{pgfscope}%
\pgfpathrectangle{\pgfqpoint{0.100000in}{0.100000in}}{\pgfqpoint{3.420221in}{2.189500in}}%
\pgfusepath{clip}%
\pgfsetbuttcap%
\pgfsetmiterjoin%
\definecolor{currentfill}{rgb}{0.000000,0.270588,0.864706}%
\pgfsetfillcolor{currentfill}%
\pgfsetlinewidth{0.000000pt}%
\definecolor{currentstroke}{rgb}{0.000000,0.000000,0.000000}%
\pgfsetstrokecolor{currentstroke}%
\pgfsetstrokeopacity{0.000000}%
\pgfsetdash{}{0pt}%
\pgfpathmoveto{\pgfqpoint{2.907801in}{0.641775in}}%
\pgfpathlineto{\pgfqpoint{2.901069in}{0.637292in}}%
\pgfpathlineto{\pgfqpoint{2.888080in}{0.635097in}}%
\pgfpathlineto{\pgfqpoint{2.886443in}{0.645363in}}%
\pgfpathlineto{\pgfqpoint{2.881929in}{0.650846in}}%
\pgfpathlineto{\pgfqpoint{2.866882in}{0.648387in}}%
\pgfpathlineto{\pgfqpoint{2.859543in}{0.640495in}}%
\pgfpathlineto{\pgfqpoint{2.852731in}{0.637305in}}%
\pgfpathlineto{\pgfqpoint{2.848148in}{0.669171in}}%
\pgfpathlineto{\pgfqpoint{2.821673in}{0.664921in}}%
\pgfpathlineto{\pgfqpoint{2.816932in}{0.698517in}}%
\pgfpathlineto{\pgfqpoint{2.832894in}{0.699556in}}%
\pgfpathlineto{\pgfqpoint{2.834781in}{0.716309in}}%
\pgfpathlineto{\pgfqpoint{2.817254in}{0.714675in}}%
\pgfpathlineto{\pgfqpoint{2.815263in}{0.731093in}}%
\pgfpathlineto{\pgfqpoint{2.833520in}{0.733453in}}%
\pgfpathlineto{\pgfqpoint{2.837381in}{0.738967in}}%
\pgfpathlineto{\pgfqpoint{2.845693in}{0.738038in}}%
\pgfpathlineto{\pgfqpoint{2.854035in}{0.747218in}}%
\pgfpathlineto{\pgfqpoint{2.868426in}{0.745230in}}%
\pgfpathlineto{\pgfqpoint{2.877879in}{0.739852in}}%
\pgfpathlineto{\pgfqpoint{2.881248in}{0.732566in}}%
\pgfpathlineto{\pgfqpoint{2.879555in}{0.720089in}}%
\pgfpathlineto{\pgfqpoint{2.882233in}{0.716655in}}%
\pgfpathlineto{\pgfqpoint{2.882425in}{0.708329in}}%
\pgfpathlineto{\pgfqpoint{2.887844in}{0.694437in}}%
\pgfpathlineto{\pgfqpoint{2.888995in}{0.687462in}}%
\pgfpathlineto{\pgfqpoint{2.898498in}{0.662391in}}%
\pgfpathclose%
\pgfusepath{fill}%
\end{pgfscope}%
\begin{pgfscope}%
\pgfpathrectangle{\pgfqpoint{0.100000in}{0.100000in}}{\pgfqpoint{3.420221in}{2.189500in}}%
\pgfusepath{clip}%
\pgfsetbuttcap%
\pgfsetmiterjoin%
\definecolor{currentfill}{rgb}{0.000000,0.552941,0.723529}%
\pgfsetfillcolor{currentfill}%
\pgfsetlinewidth{0.000000pt}%
\definecolor{currentstroke}{rgb}{0.000000,0.000000,0.000000}%
\pgfsetstrokecolor{currentstroke}%
\pgfsetstrokeopacity{0.000000}%
\pgfsetdash{}{0pt}%
\pgfpathmoveto{\pgfqpoint{1.512957in}{1.052211in}}%
\pgfpathlineto{\pgfqpoint{1.512174in}{1.043481in}}%
\pgfpathlineto{\pgfqpoint{1.507670in}{0.993501in}}%
\pgfpathlineto{\pgfqpoint{1.492943in}{0.994844in}}%
\pgfpathlineto{\pgfqpoint{1.492265in}{0.988159in}}%
\pgfpathlineto{\pgfqpoint{1.485799in}{0.988783in}}%
\pgfpathlineto{\pgfqpoint{1.485145in}{0.982282in}}%
\pgfpathlineto{\pgfqpoint{1.478688in}{0.982915in}}%
\pgfpathlineto{\pgfqpoint{1.478048in}{0.976397in}}%
\pgfpathlineto{\pgfqpoint{1.465158in}{0.977690in}}%
\pgfpathlineto{\pgfqpoint{1.464474in}{0.971123in}}%
\pgfpathlineto{\pgfqpoint{1.449555in}{0.972610in}}%
\pgfpathlineto{\pgfqpoint{1.438476in}{0.973730in}}%
\pgfpathlineto{\pgfqpoint{1.442768in}{1.013837in}}%
\pgfpathlineto{\pgfqpoint{1.459832in}{1.019690in}}%
\pgfpathlineto{\pgfqpoint{1.473279in}{1.018293in}}%
\pgfpathlineto{\pgfqpoint{1.474561in}{1.031831in}}%
\pgfpathlineto{\pgfqpoint{1.469241in}{1.032356in}}%
\pgfpathlineto{\pgfqpoint{1.456691in}{1.062199in}}%
\pgfpathlineto{\pgfqpoint{1.453849in}{1.060957in}}%
\pgfpathlineto{\pgfqpoint{1.432893in}{1.063080in}}%
\pgfpathlineto{\pgfqpoint{1.434656in}{1.071635in}}%
\pgfpathlineto{\pgfqpoint{1.435028in}{1.082670in}}%
\pgfpathlineto{\pgfqpoint{1.432115in}{1.096148in}}%
\pgfpathlineto{\pgfqpoint{1.457840in}{1.093473in}}%
\pgfpathlineto{\pgfqpoint{1.457514in}{1.090225in}}%
\pgfpathlineto{\pgfqpoint{1.470466in}{1.088914in}}%
\pgfpathlineto{\pgfqpoint{1.469805in}{1.082330in}}%
\pgfpathlineto{\pgfqpoint{1.495722in}{1.079920in}}%
\pgfpathlineto{\pgfqpoint{1.492727in}{1.054098in}}%
\pgfpathclose%
\pgfusepath{fill}%
\end{pgfscope}%
\begin{pgfscope}%
\pgfpathrectangle{\pgfqpoint{0.100000in}{0.100000in}}{\pgfqpoint{3.420221in}{2.189500in}}%
\pgfusepath{clip}%
\pgfsetbuttcap%
\pgfsetmiterjoin%
\definecolor{currentfill}{rgb}{0.000000,0.560784,0.719608}%
\pgfsetfillcolor{currentfill}%
\pgfsetlinewidth{0.000000pt}%
\definecolor{currentstroke}{rgb}{0.000000,0.000000,0.000000}%
\pgfsetstrokecolor{currentstroke}%
\pgfsetstrokeopacity{0.000000}%
\pgfsetdash{}{0pt}%
\pgfpathmoveto{\pgfqpoint{2.972715in}{1.558699in}}%
\pgfpathlineto{\pgfqpoint{2.965632in}{1.592246in}}%
\pgfpathlineto{\pgfqpoint{2.957971in}{1.590724in}}%
\pgfpathlineto{\pgfqpoint{2.952749in}{1.625817in}}%
\pgfpathlineto{\pgfqpoint{2.954622in}{1.634518in}}%
\pgfpathlineto{\pgfqpoint{2.982664in}{1.639752in}}%
\pgfpathlineto{\pgfqpoint{2.983479in}{1.634037in}}%
\pgfpathlineto{\pgfqpoint{2.997396in}{1.633934in}}%
\pgfpathlineto{\pgfqpoint{2.995851in}{1.639685in}}%
\pgfpathlineto{\pgfqpoint{3.013288in}{1.643881in}}%
\pgfpathlineto{\pgfqpoint{3.007819in}{1.648395in}}%
\pgfpathlineto{\pgfqpoint{3.029503in}{1.652890in}}%
\pgfpathlineto{\pgfqpoint{3.033485in}{1.637159in}}%
\pgfpathlineto{\pgfqpoint{3.035367in}{1.629051in}}%
\pgfpathlineto{\pgfqpoint{3.026059in}{1.628774in}}%
\pgfpathlineto{\pgfqpoint{3.019928in}{1.624702in}}%
\pgfpathlineto{\pgfqpoint{3.023112in}{1.603787in}}%
\pgfpathlineto{\pgfqpoint{3.002935in}{1.599794in}}%
\pgfpathlineto{\pgfqpoint{3.011820in}{1.570591in}}%
\pgfpathlineto{\pgfqpoint{3.007775in}{1.566319in}}%
\pgfpathclose%
\pgfusepath{fill}%
\end{pgfscope}%
\begin{pgfscope}%
\pgfpathrectangle{\pgfqpoint{0.100000in}{0.100000in}}{\pgfqpoint{3.420221in}{2.189500in}}%
\pgfusepath{clip}%
\pgfsetbuttcap%
\pgfsetmiterjoin%
\definecolor{currentfill}{rgb}{0.000000,0.611765,0.694118}%
\pgfsetfillcolor{currentfill}%
\pgfsetlinewidth{0.000000pt}%
\definecolor{currentstroke}{rgb}{0.000000,0.000000,0.000000}%
\pgfsetstrokecolor{currentstroke}%
\pgfsetstrokeopacity{0.000000}%
\pgfsetdash{}{0pt}%
\pgfpathmoveto{\pgfqpoint{0.763945in}{1.828419in}}%
\pgfpathlineto{\pgfqpoint{0.768366in}{1.828401in}}%
\pgfpathlineto{\pgfqpoint{0.777411in}{1.837284in}}%
\pgfpathlineto{\pgfqpoint{0.781055in}{1.838529in}}%
\pgfpathlineto{\pgfqpoint{0.800885in}{1.833540in}}%
\pgfpathlineto{\pgfqpoint{0.805313in}{1.828013in}}%
\pgfpathlineto{\pgfqpoint{0.803709in}{1.821722in}}%
\pgfpathlineto{\pgfqpoint{0.817604in}{1.818231in}}%
\pgfpathlineto{\pgfqpoint{0.820510in}{1.831143in}}%
\pgfpathlineto{\pgfqpoint{0.826324in}{1.836783in}}%
\pgfpathlineto{\pgfqpoint{0.835179in}{1.848515in}}%
\pgfpathlineto{\pgfqpoint{0.844033in}{1.853501in}}%
\pgfpathlineto{\pgfqpoint{0.858260in}{1.849885in}}%
\pgfpathlineto{\pgfqpoint{0.855169in}{1.837022in}}%
\pgfpathlineto{\pgfqpoint{0.859309in}{1.832288in}}%
\pgfpathlineto{\pgfqpoint{0.857179in}{1.823185in}}%
\pgfpathlineto{\pgfqpoint{0.862604in}{1.818252in}}%
\pgfpathlineto{\pgfqpoint{0.868236in}{1.816925in}}%
\pgfpathlineto{\pgfqpoint{0.863583in}{1.808624in}}%
\pgfpathlineto{\pgfqpoint{0.860031in}{1.796218in}}%
\pgfpathlineto{\pgfqpoint{0.855131in}{1.797421in}}%
\pgfpathlineto{\pgfqpoint{0.853589in}{1.791130in}}%
\pgfpathlineto{\pgfqpoint{0.849419in}{1.792194in}}%
\pgfpathlineto{\pgfqpoint{0.845798in}{1.786437in}}%
\pgfpathlineto{\pgfqpoint{0.838434in}{1.788296in}}%
\pgfpathlineto{\pgfqpoint{0.835327in}{1.775551in}}%
\pgfpathlineto{\pgfqpoint{0.827665in}{1.776326in}}%
\pgfpathlineto{\pgfqpoint{0.823895in}{1.784008in}}%
\pgfpathlineto{\pgfqpoint{0.820584in}{1.783477in}}%
\pgfpathlineto{\pgfqpoint{0.816517in}{1.770421in}}%
\pgfpathlineto{\pgfqpoint{0.786248in}{1.647919in}}%
\pgfpathlineto{\pgfqpoint{0.722921in}{1.663756in}}%
\pgfpathlineto{\pgfqpoint{0.727264in}{1.684258in}}%
\pgfpathlineto{\pgfqpoint{0.739271in}{1.730658in}}%
\pgfpathlineto{\pgfqpoint{0.738615in}{1.730832in}}%
\pgfpathlineto{\pgfqpoint{0.754882in}{1.793703in}}%
\pgfpathlineto{\pgfqpoint{0.760080in}{1.812628in}}%
\pgfpathclose%
\pgfusepath{fill}%
\end{pgfscope}%
\begin{pgfscope}%
\pgfpathrectangle{\pgfqpoint{0.100000in}{0.100000in}}{\pgfqpoint{3.420221in}{2.189500in}}%
\pgfusepath{clip}%
\pgfsetbuttcap%
\pgfsetmiterjoin%
\definecolor{currentfill}{rgb}{0.000000,0.600000,0.700000}%
\pgfsetfillcolor{currentfill}%
\pgfsetlinewidth{0.000000pt}%
\definecolor{currentstroke}{rgb}{0.000000,0.000000,0.000000}%
\pgfsetstrokecolor{currentstroke}%
\pgfsetstrokeopacity{0.000000}%
\pgfsetdash{}{0pt}%
\pgfpathmoveto{\pgfqpoint{2.718799in}{0.778604in}}%
\pgfpathlineto{\pgfqpoint{2.710622in}{0.781011in}}%
\pgfpathlineto{\pgfqpoint{2.710175in}{0.785759in}}%
\pgfpathlineto{\pgfqpoint{2.704197in}{0.799013in}}%
\pgfpathlineto{\pgfqpoint{2.700950in}{0.802977in}}%
\pgfpathlineto{\pgfqpoint{2.704572in}{0.810850in}}%
\pgfpathlineto{\pgfqpoint{2.711388in}{0.814745in}}%
\pgfpathlineto{\pgfqpoint{2.726722in}{0.816477in}}%
\pgfpathlineto{\pgfqpoint{2.728198in}{0.803727in}}%
\pgfpathlineto{\pgfqpoint{2.745504in}{0.804988in}}%
\pgfpathlineto{\pgfqpoint{2.748458in}{0.803306in}}%
\pgfpathlineto{\pgfqpoint{2.750090in}{0.795690in}}%
\pgfpathlineto{\pgfqpoint{2.754151in}{0.793358in}}%
\pgfpathlineto{\pgfqpoint{2.758147in}{0.787000in}}%
\pgfpathlineto{\pgfqpoint{2.738914in}{0.784466in}}%
\pgfpathlineto{\pgfqpoint{2.730447in}{0.783909in}}%
\pgfpathlineto{\pgfqpoint{2.730894in}{0.780163in}}%
\pgfpathclose%
\pgfusepath{fill}%
\end{pgfscope}%
\begin{pgfscope}%
\pgfpathrectangle{\pgfqpoint{0.100000in}{0.100000in}}{\pgfqpoint{3.420221in}{2.189500in}}%
\pgfusepath{clip}%
\pgfsetbuttcap%
\pgfsetmiterjoin%
\definecolor{currentfill}{rgb}{0.000000,0.262745,0.868627}%
\pgfsetfillcolor{currentfill}%
\pgfsetlinewidth{0.000000pt}%
\definecolor{currentstroke}{rgb}{0.000000,0.000000,0.000000}%
\pgfsetstrokecolor{currentstroke}%
\pgfsetstrokeopacity{0.000000}%
\pgfsetdash{}{0pt}%
\pgfpathmoveto{\pgfqpoint{1.705097in}{1.358483in}}%
\pgfpathlineto{\pgfqpoint{1.704193in}{1.358534in}}%
\pgfpathlineto{\pgfqpoint{1.705444in}{1.384614in}}%
\pgfpathlineto{\pgfqpoint{1.672408in}{1.386568in}}%
\pgfpathlineto{\pgfqpoint{1.674048in}{1.412800in}}%
\pgfpathlineto{\pgfqpoint{1.705498in}{1.410906in}}%
\pgfpathlineto{\pgfqpoint{1.706936in}{1.436820in}}%
\pgfpathlineto{\pgfqpoint{1.751998in}{1.434416in}}%
\pgfpathlineto{\pgfqpoint{1.751120in}{1.406270in}}%
\pgfpathlineto{\pgfqpoint{1.764596in}{1.404862in}}%
\pgfpathlineto{\pgfqpoint{1.762349in}{1.355562in}}%
\pgfpathlineto{\pgfqpoint{1.736723in}{1.356796in}}%
\pgfpathclose%
\pgfusepath{fill}%
\end{pgfscope}%
\begin{pgfscope}%
\pgfpathrectangle{\pgfqpoint{0.100000in}{0.100000in}}{\pgfqpoint{3.420221in}{2.189500in}}%
\pgfusepath{clip}%
\pgfsetbuttcap%
\pgfsetmiterjoin%
\definecolor{currentfill}{rgb}{0.000000,0.258824,0.870588}%
\pgfsetfillcolor{currentfill}%
\pgfsetlinewidth{0.000000pt}%
\definecolor{currentstroke}{rgb}{0.000000,0.000000,0.000000}%
\pgfsetstrokecolor{currentstroke}%
\pgfsetstrokeopacity{0.000000}%
\pgfsetdash{}{0pt}%
\pgfpathmoveto{\pgfqpoint{1.463565in}{1.641775in}}%
\pgfpathlineto{\pgfqpoint{1.509406in}{1.637237in}}%
\pgfpathlineto{\pgfqpoint{1.505742in}{1.599669in}}%
\pgfpathlineto{\pgfqpoint{1.502857in}{1.570607in}}%
\pgfpathlineto{\pgfqpoint{1.456705in}{1.575235in}}%
\pgfpathlineto{\pgfqpoint{1.457920in}{1.589872in}}%
\pgfpathclose%
\pgfusepath{fill}%
\end{pgfscope}%
\begin{pgfscope}%
\pgfpathrectangle{\pgfqpoint{0.100000in}{0.100000in}}{\pgfqpoint{3.420221in}{2.189500in}}%
\pgfusepath{clip}%
\pgfsetbuttcap%
\pgfsetmiterjoin%
\definecolor{currentfill}{rgb}{0.000000,0.152941,0.923529}%
\pgfsetfillcolor{currentfill}%
\pgfsetlinewidth{0.000000pt}%
\definecolor{currentstroke}{rgb}{0.000000,0.000000,0.000000}%
\pgfsetstrokecolor{currentstroke}%
\pgfsetstrokeopacity{0.000000}%
\pgfsetdash{}{0pt}%
\pgfpathmoveto{\pgfqpoint{0.411870in}{1.452811in}}%
\pgfpathlineto{\pgfqpoint{0.415238in}{1.454474in}}%
\pgfpathlineto{\pgfqpoint{0.415789in}{1.463804in}}%
\pgfpathlineto{\pgfqpoint{0.414088in}{1.475195in}}%
\pgfpathlineto{\pgfqpoint{0.408261in}{1.489232in}}%
\pgfpathlineto{\pgfqpoint{0.410869in}{1.493824in}}%
\pgfpathlineto{\pgfqpoint{0.420837in}{1.493666in}}%
\pgfpathlineto{\pgfqpoint{0.428368in}{1.500095in}}%
\pgfpathlineto{\pgfqpoint{0.428217in}{1.503848in}}%
\pgfpathlineto{\pgfqpoint{0.433643in}{1.500257in}}%
\pgfpathlineto{\pgfqpoint{0.434160in}{1.488812in}}%
\pgfpathlineto{\pgfqpoint{0.436242in}{1.484977in}}%
\pgfpathlineto{\pgfqpoint{0.436751in}{1.473688in}}%
\pgfpathlineto{\pgfqpoint{0.441325in}{1.470373in}}%
\pgfpathlineto{\pgfqpoint{0.446212in}{1.472384in}}%
\pgfpathlineto{\pgfqpoint{0.459955in}{1.467688in}}%
\pgfpathlineto{\pgfqpoint{0.455353in}{1.452384in}}%
\pgfpathlineto{\pgfqpoint{0.461047in}{1.450614in}}%
\pgfpathlineto{\pgfqpoint{0.457465in}{1.442763in}}%
\pgfpathlineto{\pgfqpoint{0.452500in}{1.441186in}}%
\pgfpathlineto{\pgfqpoint{0.452454in}{1.436188in}}%
\pgfpathlineto{\pgfqpoint{0.456967in}{1.434698in}}%
\pgfpathlineto{\pgfqpoint{0.452303in}{1.419122in}}%
\pgfpathlineto{\pgfqpoint{0.452447in}{1.414640in}}%
\pgfpathlineto{\pgfqpoint{0.446457in}{1.394723in}}%
\pgfpathlineto{\pgfqpoint{0.449968in}{1.388899in}}%
\pgfpathlineto{\pgfqpoint{0.449934in}{1.388939in}}%
\pgfpathlineto{\pgfqpoint{0.427638in}{1.395869in}}%
\pgfpathlineto{\pgfqpoint{0.423567in}{1.394752in}}%
\pgfpathlineto{\pgfqpoint{0.415193in}{1.399143in}}%
\pgfpathlineto{\pgfqpoint{0.407957in}{1.397700in}}%
\pgfpathlineto{\pgfqpoint{0.405786in}{1.393067in}}%
\pgfpathlineto{\pgfqpoint{0.405350in}{1.381594in}}%
\pgfpathlineto{\pgfqpoint{0.395314in}{1.382459in}}%
\pgfpathlineto{\pgfqpoint{0.394930in}{1.376292in}}%
\pgfpathlineto{\pgfqpoint{0.390791in}{1.383454in}}%
\pgfpathlineto{\pgfqpoint{0.390756in}{1.388111in}}%
\pgfpathlineto{\pgfqpoint{0.394202in}{1.395651in}}%
\pgfpathlineto{\pgfqpoint{0.394196in}{1.405816in}}%
\pgfpathlineto{\pgfqpoint{0.391388in}{1.410201in}}%
\pgfpathlineto{\pgfqpoint{0.395481in}{1.418240in}}%
\pgfpathlineto{\pgfqpoint{0.397321in}{1.428444in}}%
\pgfpathlineto{\pgfqpoint{0.404131in}{1.428681in}}%
\pgfpathlineto{\pgfqpoint{0.401860in}{1.412169in}}%
\pgfpathlineto{\pgfqpoint{0.406915in}{1.409111in}}%
\pgfpathlineto{\pgfqpoint{0.411463in}{1.402064in}}%
\pgfpathlineto{\pgfqpoint{0.414617in}{1.403366in}}%
\pgfpathlineto{\pgfqpoint{0.414765in}{1.412914in}}%
\pgfpathlineto{\pgfqpoint{0.408040in}{1.426409in}}%
\pgfpathlineto{\pgfqpoint{0.411343in}{1.433230in}}%
\pgfpathlineto{\pgfqpoint{0.407831in}{1.439150in}}%
\pgfpathlineto{\pgfqpoint{0.417607in}{1.443463in}}%
\pgfpathlineto{\pgfqpoint{0.416212in}{1.448664in}}%
\pgfpathclose%
\pgfusepath{fill}%
\end{pgfscope}%
\begin{pgfscope}%
\pgfpathrectangle{\pgfqpoint{0.100000in}{0.100000in}}{\pgfqpoint{3.420221in}{2.189500in}}%
\pgfusepath{clip}%
\pgfsetbuttcap%
\pgfsetmiterjoin%
\definecolor{currentfill}{rgb}{0.000000,0.152941,0.923529}%
\pgfsetfillcolor{currentfill}%
\pgfsetlinewidth{0.000000pt}%
\definecolor{currentstroke}{rgb}{0.000000,0.000000,0.000000}%
\pgfsetstrokecolor{currentstroke}%
\pgfsetstrokeopacity{0.000000}%
\pgfsetdash{}{0pt}%
\pgfpathmoveto{\pgfqpoint{0.406132in}{1.451449in}}%
\pgfpathlineto{\pgfqpoint{0.402652in}{1.436241in}}%
\pgfpathlineto{\pgfqpoint{0.397831in}{1.431745in}}%
\pgfpathlineto{\pgfqpoint{0.392372in}{1.439512in}}%
\pgfpathlineto{\pgfqpoint{0.389239in}{1.439717in}}%
\pgfpathlineto{\pgfqpoint{0.383538in}{1.451131in}}%
\pgfpathlineto{\pgfqpoint{0.377772in}{1.453999in}}%
\pgfpathlineto{\pgfqpoint{0.381165in}{1.462489in}}%
\pgfpathlineto{\pgfqpoint{0.381487in}{1.473896in}}%
\pgfpathlineto{\pgfqpoint{0.387741in}{1.473561in}}%
\pgfpathlineto{\pgfqpoint{0.394222in}{1.462892in}}%
\pgfpathlineto{\pgfqpoint{0.403286in}{1.458482in}}%
\pgfpathclose%
\pgfusepath{fill}%
\end{pgfscope}%
\begin{pgfscope}%
\pgfpathrectangle{\pgfqpoint{0.100000in}{0.100000in}}{\pgfqpoint{3.420221in}{2.189500in}}%
\pgfusepath{clip}%
\pgfsetbuttcap%
\pgfsetmiterjoin%
\definecolor{currentfill}{rgb}{0.000000,0.400000,0.800000}%
\pgfsetfillcolor{currentfill}%
\pgfsetlinewidth{0.000000pt}%
\definecolor{currentstroke}{rgb}{0.000000,0.000000,0.000000}%
\pgfsetstrokecolor{currentstroke}%
\pgfsetstrokeopacity{0.000000}%
\pgfsetdash{}{0pt}%
\pgfpathmoveto{\pgfqpoint{2.290116in}{1.476751in}}%
\pgfpathlineto{\pgfqpoint{2.272014in}{1.475825in}}%
\pgfpathlineto{\pgfqpoint{2.268056in}{1.490509in}}%
\pgfpathlineto{\pgfqpoint{2.271333in}{1.492492in}}%
\pgfpathlineto{\pgfqpoint{2.272886in}{1.507312in}}%
\pgfpathlineto{\pgfqpoint{2.271208in}{1.515609in}}%
\pgfpathlineto{\pgfqpoint{2.262291in}{1.521054in}}%
\pgfpathlineto{\pgfqpoint{2.284345in}{1.522444in}}%
\pgfpathlineto{\pgfqpoint{2.282670in}{1.545611in}}%
\pgfpathlineto{\pgfqpoint{2.313651in}{1.547024in}}%
\pgfpathlineto{\pgfqpoint{2.350052in}{1.548913in}}%
\pgfpathlineto{\pgfqpoint{2.351914in}{1.523465in}}%
\pgfpathlineto{\pgfqpoint{2.358413in}{1.523924in}}%
\pgfpathlineto{\pgfqpoint{2.358136in}{1.517350in}}%
\pgfpathlineto{\pgfqpoint{2.360395in}{1.484794in}}%
\pgfpathlineto{\pgfqpoint{2.328824in}{1.482466in}}%
\pgfpathlineto{\pgfqpoint{2.329072in}{1.479199in}}%
\pgfpathclose%
\pgfusepath{fill}%
\end{pgfscope}%
\begin{pgfscope}%
\pgfpathrectangle{\pgfqpoint{0.100000in}{0.100000in}}{\pgfqpoint{3.420221in}{2.189500in}}%
\pgfusepath{clip}%
\pgfsetbuttcap%
\pgfsetmiterjoin%
\definecolor{currentfill}{rgb}{0.000000,0.419608,0.790196}%
\pgfsetfillcolor{currentfill}%
\pgfsetlinewidth{0.000000pt}%
\definecolor{currentstroke}{rgb}{0.000000,0.000000,0.000000}%
\pgfsetstrokecolor{currentstroke}%
\pgfsetstrokeopacity{0.000000}%
\pgfsetdash{}{0pt}%
\pgfpathmoveto{\pgfqpoint{2.453887in}{1.287376in}}%
\pgfpathlineto{\pgfqpoint{2.460331in}{1.287890in}}%
\pgfpathlineto{\pgfqpoint{2.460123in}{1.283074in}}%
\pgfpathlineto{\pgfqpoint{2.454557in}{1.280800in}}%
\pgfpathlineto{\pgfqpoint{2.453011in}{1.271112in}}%
\pgfpathlineto{\pgfqpoint{2.456130in}{1.260278in}}%
\pgfpathlineto{\pgfqpoint{2.443307in}{1.258338in}}%
\pgfpathlineto{\pgfqpoint{2.436766in}{1.254047in}}%
\pgfpathlineto{\pgfqpoint{2.432080in}{1.260021in}}%
\pgfpathlineto{\pgfqpoint{2.416854in}{1.258989in}}%
\pgfpathlineto{\pgfqpoint{2.415463in}{1.279902in}}%
\pgfpathlineto{\pgfqpoint{2.437359in}{1.281826in}}%
\pgfpathlineto{\pgfqpoint{2.437529in}{1.285590in}}%
\pgfpathclose%
\pgfusepath{fill}%
\end{pgfscope}%
\begin{pgfscope}%
\pgfpathrectangle{\pgfqpoint{0.100000in}{0.100000in}}{\pgfqpoint{3.420221in}{2.189500in}}%
\pgfusepath{clip}%
\pgfsetbuttcap%
\pgfsetmiterjoin%
\definecolor{currentfill}{rgb}{0.000000,0.458824,0.770588}%
\pgfsetfillcolor{currentfill}%
\pgfsetlinewidth{0.000000pt}%
\definecolor{currentstroke}{rgb}{0.000000,0.000000,0.000000}%
\pgfsetstrokecolor{currentstroke}%
\pgfsetstrokeopacity{0.000000}%
\pgfsetdash{}{0pt}%
\pgfpathmoveto{\pgfqpoint{1.960545in}{0.944696in}}%
\pgfpathlineto{\pgfqpoint{1.960513in}{0.938154in}}%
\pgfpathlineto{\pgfqpoint{1.953843in}{0.938189in}}%
\pgfpathlineto{\pgfqpoint{1.953619in}{0.911892in}}%
\pgfpathlineto{\pgfqpoint{1.940418in}{0.912029in}}%
\pgfpathlineto{\pgfqpoint{1.914535in}{0.912409in}}%
\pgfpathlineto{\pgfqpoint{1.914858in}{0.932044in}}%
\pgfpathlineto{\pgfqpoint{1.908260in}{0.932231in}}%
\pgfpathlineto{\pgfqpoint{1.908363in}{0.938676in}}%
\pgfpathlineto{\pgfqpoint{1.888957in}{0.939078in}}%
\pgfpathlineto{\pgfqpoint{1.889089in}{0.945628in}}%
\pgfpathlineto{\pgfqpoint{1.882587in}{0.945758in}}%
\pgfpathlineto{\pgfqpoint{1.882992in}{0.965391in}}%
\pgfpathlineto{\pgfqpoint{1.885751in}{0.970896in}}%
\pgfpathlineto{\pgfqpoint{1.890325in}{0.972096in}}%
\pgfpathlineto{\pgfqpoint{1.895921in}{0.965304in}}%
\pgfpathlineto{\pgfqpoint{1.900660in}{0.968627in}}%
\pgfpathlineto{\pgfqpoint{1.910292in}{0.969092in}}%
\pgfpathlineto{\pgfqpoint{1.910566in}{0.984511in}}%
\pgfpathlineto{\pgfqpoint{1.913581in}{0.984453in}}%
\pgfpathlineto{\pgfqpoint{1.913786in}{0.997539in}}%
\pgfpathlineto{\pgfqpoint{1.941973in}{0.997147in}}%
\pgfpathlineto{\pgfqpoint{1.941718in}{0.986762in}}%
\pgfpathlineto{\pgfqpoint{1.935172in}{0.979200in}}%
\pgfpathlineto{\pgfqpoint{1.934634in}{0.951465in}}%
\pgfpathlineto{\pgfqpoint{1.947612in}{0.951316in}}%
\pgfpathlineto{\pgfqpoint{1.947544in}{0.944783in}}%
\pgfpathclose%
\pgfusepath{fill}%
\end{pgfscope}%
\begin{pgfscope}%
\pgfpathrectangle{\pgfqpoint{0.100000in}{0.100000in}}{\pgfqpoint{3.420221in}{2.189500in}}%
\pgfusepath{clip}%
\pgfsetbuttcap%
\pgfsetmiterjoin%
\definecolor{currentfill}{rgb}{0.000000,0.403922,0.798039}%
\pgfsetfillcolor{currentfill}%
\pgfsetlinewidth{0.000000pt}%
\definecolor{currentstroke}{rgb}{0.000000,0.000000,0.000000}%
\pgfsetstrokecolor{currentstroke}%
\pgfsetstrokeopacity{0.000000}%
\pgfsetdash{}{0pt}%
\pgfpathmoveto{\pgfqpoint{2.993264in}{1.181265in}}%
\pgfpathlineto{\pgfqpoint{3.026229in}{1.187604in}}%
\pgfpathlineto{\pgfqpoint{3.027780in}{1.175502in}}%
\pgfpathlineto{\pgfqpoint{3.024677in}{1.161202in}}%
\pgfpathlineto{\pgfqpoint{3.016565in}{1.163005in}}%
\pgfpathlineto{\pgfqpoint{3.005946in}{1.162530in}}%
\pgfpathlineto{\pgfqpoint{3.001001in}{1.154259in}}%
\pgfpathlineto{\pgfqpoint{2.995070in}{1.154170in}}%
\pgfpathlineto{\pgfqpoint{2.991240in}{1.174168in}}%
\pgfpathclose%
\pgfusepath{fill}%
\end{pgfscope}%
\begin{pgfscope}%
\pgfpathrectangle{\pgfqpoint{0.100000in}{0.100000in}}{\pgfqpoint{3.420221in}{2.189500in}}%
\pgfusepath{clip}%
\pgfsetbuttcap%
\pgfsetmiterjoin%
\definecolor{currentfill}{rgb}{0.000000,0.701961,0.649020}%
\pgfsetfillcolor{currentfill}%
\pgfsetlinewidth{0.000000pt}%
\definecolor{currentstroke}{rgb}{0.000000,0.000000,0.000000}%
\pgfsetstrokecolor{currentstroke}%
\pgfsetstrokeopacity{0.000000}%
\pgfsetdash{}{0pt}%
\pgfpathmoveto{\pgfqpoint{1.127245in}{0.950968in}}%
\pgfpathlineto{\pgfqpoint{1.120630in}{0.908635in}}%
\pgfpathlineto{\pgfqpoint{1.111527in}{0.850369in}}%
\pgfpathlineto{\pgfqpoint{1.040987in}{0.861807in}}%
\pgfpathlineto{\pgfqpoint{1.023355in}{0.864881in}}%
\pgfpathlineto{\pgfqpoint{1.024476in}{0.871303in}}%
\pgfpathlineto{\pgfqpoint{1.033293in}{0.921832in}}%
\pgfpathlineto{\pgfqpoint{1.036145in}{0.935622in}}%
\pgfpathlineto{\pgfqpoint{1.044265in}{0.938176in}}%
\pgfpathlineto{\pgfqpoint{1.054048in}{0.939013in}}%
\pgfpathlineto{\pgfqpoint{1.064508in}{0.937276in}}%
\pgfpathlineto{\pgfqpoint{1.065882in}{0.945522in}}%
\pgfpathlineto{\pgfqpoint{1.072052in}{0.944383in}}%
\pgfpathlineto{\pgfqpoint{1.079115in}{0.935600in}}%
\pgfpathlineto{\pgfqpoint{1.090801in}{0.944112in}}%
\pgfpathlineto{\pgfqpoint{1.095330in}{0.943942in}}%
\pgfpathlineto{\pgfqpoint{1.099418in}{0.948488in}}%
\pgfpathlineto{\pgfqpoint{1.108659in}{0.953842in}}%
\pgfpathclose%
\pgfusepath{fill}%
\end{pgfscope}%
\begin{pgfscope}%
\pgfpathrectangle{\pgfqpoint{0.100000in}{0.100000in}}{\pgfqpoint{3.420221in}{2.189500in}}%
\pgfusepath{clip}%
\pgfsetbuttcap%
\pgfsetmiterjoin%
\definecolor{currentfill}{rgb}{0.000000,0.529412,0.735294}%
\pgfsetfillcolor{currentfill}%
\pgfsetlinewidth{0.000000pt}%
\definecolor{currentstroke}{rgb}{0.000000,0.000000,0.000000}%
\pgfsetstrokecolor{currentstroke}%
\pgfsetstrokeopacity{0.000000}%
\pgfsetdash{}{0pt}%
\pgfpathmoveto{\pgfqpoint{1.954447in}{0.711960in}}%
\pgfpathlineto{\pgfqpoint{1.958759in}{0.696206in}}%
\pgfpathlineto{\pgfqpoint{1.955232in}{0.694323in}}%
\pgfpathlineto{\pgfqpoint{1.951930in}{0.684231in}}%
\pgfpathlineto{\pgfqpoint{1.952608in}{0.681005in}}%
\pgfpathlineto{\pgfqpoint{1.938901in}{0.680954in}}%
\pgfpathlineto{\pgfqpoint{1.921621in}{0.672218in}}%
\pgfpathlineto{\pgfqpoint{1.916335in}{0.695976in}}%
\pgfpathlineto{\pgfqpoint{1.916975in}{0.701293in}}%
\pgfpathlineto{\pgfqpoint{1.898873in}{0.691284in}}%
\pgfpathlineto{\pgfqpoint{1.877980in}{0.728570in}}%
\pgfpathlineto{\pgfqpoint{1.891820in}{0.736299in}}%
\pgfpathlineto{\pgfqpoint{1.880869in}{0.756078in}}%
\pgfpathlineto{\pgfqpoint{1.913946in}{0.774695in}}%
\pgfpathlineto{\pgfqpoint{1.923306in}{0.768704in}}%
\pgfpathlineto{\pgfqpoint{1.933919in}{0.752509in}}%
\pgfpathlineto{\pgfqpoint{1.940990in}{0.732629in}}%
\pgfpathlineto{\pgfqpoint{1.945859in}{0.731063in}}%
\pgfpathlineto{\pgfqpoint{1.946527in}{0.726236in}}%
\pgfpathlineto{\pgfqpoint{1.956358in}{0.719557in}}%
\pgfpathclose%
\pgfusepath{fill}%
\end{pgfscope}%
\begin{pgfscope}%
\pgfpathrectangle{\pgfqpoint{0.100000in}{0.100000in}}{\pgfqpoint{3.420221in}{2.189500in}}%
\pgfusepath{clip}%
\pgfsetbuttcap%
\pgfsetmiterjoin%
\definecolor{currentfill}{rgb}{0.000000,0.235294,0.882353}%
\pgfsetfillcolor{currentfill}%
\pgfsetlinewidth{0.000000pt}%
\definecolor{currentstroke}{rgb}{0.000000,0.000000,0.000000}%
\pgfsetstrokecolor{currentstroke}%
\pgfsetstrokeopacity{0.000000}%
\pgfsetdash{}{0pt}%
\pgfpathmoveto{\pgfqpoint{1.855200in}{1.969173in}}%
\pgfpathlineto{\pgfqpoint{1.855859in}{1.955811in}}%
\pgfpathlineto{\pgfqpoint{1.854997in}{1.929515in}}%
\pgfpathlineto{\pgfqpoint{1.849785in}{1.929677in}}%
\pgfpathlineto{\pgfqpoint{1.823502in}{1.930624in}}%
\pgfpathlineto{\pgfqpoint{1.822718in}{1.943858in}}%
\pgfpathlineto{\pgfqpoint{1.783464in}{1.945588in}}%
\pgfpathlineto{\pgfqpoint{1.757312in}{1.946954in}}%
\pgfpathlineto{\pgfqpoint{1.756178in}{1.960223in}}%
\pgfpathlineto{\pgfqpoint{1.757669in}{1.986742in}}%
\pgfpathlineto{\pgfqpoint{1.775492in}{1.985799in}}%
\pgfpathlineto{\pgfqpoint{1.801799in}{1.984536in}}%
\pgfpathlineto{\pgfqpoint{1.802403in}{1.997722in}}%
\pgfpathlineto{\pgfqpoint{1.835198in}{1.996375in}}%
\pgfpathlineto{\pgfqpoint{1.836026in}{1.983102in}}%
\pgfpathlineto{\pgfqpoint{1.835528in}{1.969852in}}%
\pgfpathclose%
\pgfusepath{fill}%
\end{pgfscope}%
\begin{pgfscope}%
\pgfpathrectangle{\pgfqpoint{0.100000in}{0.100000in}}{\pgfqpoint{3.420221in}{2.189500in}}%
\pgfusepath{clip}%
\pgfsetbuttcap%
\pgfsetmiterjoin%
\definecolor{currentfill}{rgb}{0.000000,0.517647,0.741176}%
\pgfsetfillcolor{currentfill}%
\pgfsetlinewidth{0.000000pt}%
\definecolor{currentstroke}{rgb}{0.000000,0.000000,0.000000}%
\pgfsetstrokecolor{currentstroke}%
\pgfsetstrokeopacity{0.000000}%
\pgfsetdash{}{0pt}%
\pgfpathmoveto{\pgfqpoint{2.330118in}{0.564192in}}%
\pgfpathlineto{\pgfqpoint{2.329792in}{0.555106in}}%
\pgfpathlineto{\pgfqpoint{2.325227in}{0.553352in}}%
\pgfpathlineto{\pgfqpoint{2.329393in}{0.546739in}}%
\pgfpathlineto{\pgfqpoint{2.327876in}{0.543119in}}%
\pgfpathlineto{\pgfqpoint{2.319269in}{0.536778in}}%
\pgfpathlineto{\pgfqpoint{2.315866in}{0.541355in}}%
\pgfpathlineto{\pgfqpoint{2.313580in}{0.549435in}}%
\pgfpathlineto{\pgfqpoint{2.309682in}{0.552025in}}%
\pgfpathlineto{\pgfqpoint{2.306895in}{0.545866in}}%
\pgfpathlineto{\pgfqpoint{2.300275in}{0.551203in}}%
\pgfpathlineto{\pgfqpoint{2.291344in}{0.543987in}}%
\pgfpathlineto{\pgfqpoint{2.291028in}{0.538674in}}%
\pgfpathlineto{\pgfqpoint{2.285455in}{0.537901in}}%
\pgfpathlineto{\pgfqpoint{2.277905in}{0.533604in}}%
\pgfpathlineto{\pgfqpoint{2.270035in}{0.541602in}}%
\pgfpathlineto{\pgfqpoint{2.261111in}{0.541699in}}%
\pgfpathlineto{\pgfqpoint{2.248909in}{0.545834in}}%
\pgfpathlineto{\pgfqpoint{2.245094in}{0.549245in}}%
\pgfpathlineto{\pgfqpoint{2.251148in}{0.555543in}}%
\pgfpathlineto{\pgfqpoint{2.248721in}{0.563059in}}%
\pgfpathlineto{\pgfqpoint{2.230662in}{0.569417in}}%
\pgfpathlineto{\pgfqpoint{2.224682in}{0.573672in}}%
\pgfpathlineto{\pgfqpoint{2.224703in}{0.581929in}}%
\pgfpathlineto{\pgfqpoint{2.214390in}{0.582094in}}%
\pgfpathlineto{\pgfqpoint{2.207653in}{0.580053in}}%
\pgfpathlineto{\pgfqpoint{2.218461in}{0.590176in}}%
\pgfpathlineto{\pgfqpoint{2.223865in}{0.597905in}}%
\pgfpathlineto{\pgfqpoint{2.233865in}{0.599098in}}%
\pgfpathlineto{\pgfqpoint{2.240688in}{0.590137in}}%
\pgfpathlineto{\pgfqpoint{2.239996in}{0.586258in}}%
\pgfpathlineto{\pgfqpoint{2.253669in}{0.582638in}}%
\pgfpathlineto{\pgfqpoint{2.259629in}{0.584935in}}%
\pgfpathlineto{\pgfqpoint{2.257979in}{0.590256in}}%
\pgfpathlineto{\pgfqpoint{2.253086in}{0.591581in}}%
\pgfpathlineto{\pgfqpoint{2.247683in}{0.603007in}}%
\pgfpathlineto{\pgfqpoint{2.249902in}{0.605006in}}%
\pgfpathlineto{\pgfqpoint{2.264061in}{0.609692in}}%
\pgfpathlineto{\pgfqpoint{2.266861in}{0.608853in}}%
\pgfpathlineto{\pgfqpoint{2.267410in}{0.602164in}}%
\pgfpathlineto{\pgfqpoint{2.272532in}{0.596877in}}%
\pgfpathlineto{\pgfqpoint{2.295254in}{0.596944in}}%
\pgfpathlineto{\pgfqpoint{2.299821in}{0.591150in}}%
\pgfpathlineto{\pgfqpoint{2.306732in}{0.587609in}}%
\pgfpathlineto{\pgfqpoint{2.309238in}{0.582525in}}%
\pgfpathlineto{\pgfqpoint{2.318241in}{0.582887in}}%
\pgfpathlineto{\pgfqpoint{2.321496in}{0.580669in}}%
\pgfpathlineto{\pgfqpoint{2.319836in}{0.573219in}}%
\pgfpathlineto{\pgfqpoint{2.323388in}{0.566863in}}%
\pgfpathclose%
\pgfusepath{fill}%
\end{pgfscope}%
\begin{pgfscope}%
\pgfpathrectangle{\pgfqpoint{0.100000in}{0.100000in}}{\pgfqpoint{3.420221in}{2.189500in}}%
\pgfusepath{clip}%
\pgfsetbuttcap%
\pgfsetmiterjoin%
\definecolor{currentfill}{rgb}{0.000000,0.525490,0.737255}%
\pgfsetfillcolor{currentfill}%
\pgfsetlinewidth{0.000000pt}%
\definecolor{currentstroke}{rgb}{0.000000,0.000000,0.000000}%
\pgfsetstrokecolor{currentstroke}%
\pgfsetstrokeopacity{0.000000}%
\pgfsetdash{}{0pt}%
\pgfpathmoveto{\pgfqpoint{1.694769in}{0.985795in}}%
\pgfpathlineto{\pgfqpoint{1.697200in}{1.029928in}}%
\pgfpathlineto{\pgfqpoint{1.698280in}{1.049547in}}%
\pgfpathlineto{\pgfqpoint{1.704007in}{1.055319in}}%
\pgfpathlineto{\pgfqpoint{1.710015in}{1.052516in}}%
\pgfpathlineto{\pgfqpoint{1.715395in}{1.047040in}}%
\pgfpathlineto{\pgfqpoint{1.725051in}{1.047014in}}%
\pgfpathlineto{\pgfqpoint{1.729066in}{1.054891in}}%
\pgfpathlineto{\pgfqpoint{1.736360in}{1.057495in}}%
\pgfpathlineto{\pgfqpoint{1.735938in}{1.042436in}}%
\pgfpathlineto{\pgfqpoint{1.781161in}{1.040456in}}%
\pgfpathlineto{\pgfqpoint{1.780925in}{1.020880in}}%
\pgfpathlineto{\pgfqpoint{1.779694in}{0.986799in}}%
\pgfpathlineto{\pgfqpoint{1.771901in}{0.988551in}}%
\pgfpathlineto{\pgfqpoint{1.708469in}{0.991555in}}%
\pgfpathlineto{\pgfqpoint{1.708119in}{0.985056in}}%
\pgfpathclose%
\pgfusepath{fill}%
\end{pgfscope}%
\begin{pgfscope}%
\pgfpathrectangle{\pgfqpoint{0.100000in}{0.100000in}}{\pgfqpoint{3.420221in}{2.189500in}}%
\pgfusepath{clip}%
\pgfsetbuttcap%
\pgfsetmiterjoin%
\definecolor{currentfill}{rgb}{0.000000,0.584314,0.707843}%
\pgfsetfillcolor{currentfill}%
\pgfsetlinewidth{0.000000pt}%
\definecolor{currentstroke}{rgb}{0.000000,0.000000,0.000000}%
\pgfsetstrokecolor{currentstroke}%
\pgfsetstrokeopacity{0.000000}%
\pgfsetdash{}{0pt}%
\pgfpathmoveto{\pgfqpoint{2.857024in}{1.353066in}}%
\pgfpathlineto{\pgfqpoint{2.868718in}{1.344562in}}%
\pgfpathlineto{\pgfqpoint{2.866416in}{1.338229in}}%
\pgfpathlineto{\pgfqpoint{2.867299in}{1.326897in}}%
\pgfpathlineto{\pgfqpoint{2.860941in}{1.325692in}}%
\pgfpathlineto{\pgfqpoint{2.859039in}{1.321669in}}%
\pgfpathlineto{\pgfqpoint{2.849782in}{1.322776in}}%
\pgfpathlineto{\pgfqpoint{2.846207in}{1.330088in}}%
\pgfpathlineto{\pgfqpoint{2.838371in}{1.333586in}}%
\pgfpathlineto{\pgfqpoint{2.830150in}{1.340814in}}%
\pgfpathlineto{\pgfqpoint{2.828466in}{1.347165in}}%
\pgfpathlineto{\pgfqpoint{2.836280in}{1.353875in}}%
\pgfpathlineto{\pgfqpoint{2.852948in}{1.351724in}}%
\pgfpathclose%
\pgfusepath{fill}%
\end{pgfscope}%
\begin{pgfscope}%
\pgfpathrectangle{\pgfqpoint{0.100000in}{0.100000in}}{\pgfqpoint{3.420221in}{2.189500in}}%
\pgfusepath{clip}%
\pgfsetbuttcap%
\pgfsetmiterjoin%
\definecolor{currentfill}{rgb}{0.000000,0.352941,0.823529}%
\pgfsetfillcolor{currentfill}%
\pgfsetlinewidth{0.000000pt}%
\definecolor{currentstroke}{rgb}{0.000000,0.000000,0.000000}%
\pgfsetstrokecolor{currentstroke}%
\pgfsetstrokeopacity{0.000000}%
\pgfsetdash{}{0pt}%
\pgfpathmoveto{\pgfqpoint{2.673255in}{1.229010in}}%
\pgfpathlineto{\pgfqpoint{2.668925in}{1.220227in}}%
\pgfpathlineto{\pgfqpoint{2.660160in}{1.221842in}}%
\pgfpathlineto{\pgfqpoint{2.651165in}{1.226891in}}%
\pgfpathlineto{\pgfqpoint{2.634813in}{1.231263in}}%
\pgfpathlineto{\pgfqpoint{2.621176in}{1.218506in}}%
\pgfpathlineto{\pgfqpoint{2.618781in}{1.215147in}}%
\pgfpathlineto{\pgfqpoint{2.602425in}{1.230209in}}%
\pgfpathlineto{\pgfqpoint{2.595668in}{1.229524in}}%
\pgfpathlineto{\pgfqpoint{2.590455in}{1.224248in}}%
\pgfpathlineto{\pgfqpoint{2.583224in}{1.223915in}}%
\pgfpathlineto{\pgfqpoint{2.581739in}{1.229328in}}%
\pgfpathlineto{\pgfqpoint{2.587662in}{1.237772in}}%
\pgfpathlineto{\pgfqpoint{2.589395in}{1.244715in}}%
\pgfpathlineto{\pgfqpoint{2.589168in}{1.253000in}}%
\pgfpathlineto{\pgfqpoint{2.593748in}{1.261034in}}%
\pgfpathlineto{\pgfqpoint{2.587934in}{1.268910in}}%
\pgfpathlineto{\pgfqpoint{2.586075in}{1.274301in}}%
\pgfpathlineto{\pgfqpoint{2.582127in}{1.276486in}}%
\pgfpathlineto{\pgfqpoint{2.589723in}{1.282241in}}%
\pgfpathlineto{\pgfqpoint{2.597872in}{1.287524in}}%
\pgfpathlineto{\pgfqpoint{2.599389in}{1.280039in}}%
\pgfpathlineto{\pgfqpoint{2.611910in}{1.270491in}}%
\pgfpathlineto{\pgfqpoint{2.621640in}{1.278242in}}%
\pgfpathlineto{\pgfqpoint{2.632795in}{1.281175in}}%
\pgfpathlineto{\pgfqpoint{2.639780in}{1.283243in}}%
\pgfpathlineto{\pgfqpoint{2.646568in}{1.282870in}}%
\pgfpathlineto{\pgfqpoint{2.649610in}{1.276432in}}%
\pgfpathlineto{\pgfqpoint{2.646926in}{1.273074in}}%
\pgfpathlineto{\pgfqpoint{2.656168in}{1.262476in}}%
\pgfpathlineto{\pgfqpoint{2.644253in}{1.247307in}}%
\pgfpathlineto{\pgfqpoint{2.652964in}{1.234692in}}%
\pgfpathlineto{\pgfqpoint{2.665224in}{1.233614in}}%
\pgfpathclose%
\pgfusepath{fill}%
\end{pgfscope}%
\begin{pgfscope}%
\pgfpathrectangle{\pgfqpoint{0.100000in}{0.100000in}}{\pgfqpoint{3.420221in}{2.189500in}}%
\pgfusepath{clip}%
\pgfsetbuttcap%
\pgfsetmiterjoin%
\definecolor{currentfill}{rgb}{0.000000,0.407843,0.796078}%
\pgfsetfillcolor{currentfill}%
\pgfsetlinewidth{0.000000pt}%
\definecolor{currentstroke}{rgb}{0.000000,0.000000,0.000000}%
\pgfsetstrokecolor{currentstroke}%
\pgfsetstrokeopacity{0.000000}%
\pgfsetdash{}{0pt}%
\pgfpathmoveto{\pgfqpoint{2.524058in}{1.392572in}}%
\pgfpathlineto{\pgfqpoint{2.502400in}{1.390197in}}%
\pgfpathlineto{\pgfqpoint{2.501245in}{1.401981in}}%
\pgfpathlineto{\pgfqpoint{2.493288in}{1.405583in}}%
\pgfpathlineto{\pgfqpoint{2.491564in}{1.425048in}}%
\pgfpathlineto{\pgfqpoint{2.482990in}{1.424222in}}%
\pgfpathlineto{\pgfqpoint{2.479485in}{1.427161in}}%
\pgfpathlineto{\pgfqpoint{2.478282in}{1.440236in}}%
\pgfpathlineto{\pgfqpoint{2.458575in}{1.438561in}}%
\pgfpathlineto{\pgfqpoint{2.456869in}{1.457965in}}%
\pgfpathlineto{\pgfqpoint{2.482907in}{1.460294in}}%
\pgfpathlineto{\pgfqpoint{2.484664in}{1.440750in}}%
\pgfpathlineto{\pgfqpoint{2.501550in}{1.442353in}}%
\pgfpathlineto{\pgfqpoint{2.500963in}{1.448821in}}%
\pgfpathlineto{\pgfqpoint{2.513434in}{1.450271in}}%
\pgfpathlineto{\pgfqpoint{2.516407in}{1.424352in}}%
\pgfpathlineto{\pgfqpoint{2.520639in}{1.424829in}}%
\pgfpathclose%
\pgfusepath{fill}%
\end{pgfscope}%
\begin{pgfscope}%
\pgfpathrectangle{\pgfqpoint{0.100000in}{0.100000in}}{\pgfqpoint{3.420221in}{2.189500in}}%
\pgfusepath{clip}%
\pgfsetbuttcap%
\pgfsetmiterjoin%
\definecolor{currentfill}{rgb}{0.000000,0.498039,0.750980}%
\pgfsetfillcolor{currentfill}%
\pgfsetlinewidth{0.000000pt}%
\definecolor{currentstroke}{rgb}{0.000000,0.000000,0.000000}%
\pgfsetstrokecolor{currentstroke}%
\pgfsetstrokeopacity{0.000000}%
\pgfsetdash{}{0pt}%
\pgfpathmoveto{\pgfqpoint{2.238891in}{0.980257in}}%
\pgfpathlineto{\pgfqpoint{2.230139in}{0.978586in}}%
\pgfpathlineto{\pgfqpoint{2.225591in}{0.983424in}}%
\pgfpathlineto{\pgfqpoint{2.219258in}{0.985996in}}%
\pgfpathlineto{\pgfqpoint{2.211949in}{0.985864in}}%
\pgfpathlineto{\pgfqpoint{2.212072in}{0.980880in}}%
\pgfpathlineto{\pgfqpoint{2.204243in}{0.983415in}}%
\pgfpathlineto{\pgfqpoint{2.198697in}{0.980799in}}%
\pgfpathlineto{\pgfqpoint{2.194232in}{0.980117in}}%
\pgfpathlineto{\pgfqpoint{2.185959in}{0.983174in}}%
\pgfpathlineto{\pgfqpoint{2.179115in}{0.982850in}}%
\pgfpathlineto{\pgfqpoint{2.178916in}{1.005102in}}%
\pgfpathlineto{\pgfqpoint{2.194806in}{1.005296in}}%
\pgfpathlineto{\pgfqpoint{2.197981in}{1.012716in}}%
\pgfpathlineto{\pgfqpoint{2.197889in}{1.018611in}}%
\pgfpathlineto{\pgfqpoint{2.223815in}{1.019004in}}%
\pgfpathlineto{\pgfqpoint{2.223522in}{1.027529in}}%
\pgfpathlineto{\pgfqpoint{2.226624in}{1.037089in}}%
\pgfpathlineto{\pgfqpoint{2.233111in}{1.046699in}}%
\pgfpathlineto{\pgfqpoint{2.243212in}{1.046759in}}%
\pgfpathlineto{\pgfqpoint{2.244240in}{1.013593in}}%
\pgfpathlineto{\pgfqpoint{2.245324in}{0.980415in}}%
\pgfpathclose%
\pgfusepath{fill}%
\end{pgfscope}%
\begin{pgfscope}%
\pgfpathrectangle{\pgfqpoint{0.100000in}{0.100000in}}{\pgfqpoint{3.420221in}{2.189500in}}%
\pgfusepath{clip}%
\pgfsetbuttcap%
\pgfsetmiterjoin%
\definecolor{currentfill}{rgb}{0.000000,0.392157,0.803922}%
\pgfsetfillcolor{currentfill}%
\pgfsetlinewidth{0.000000pt}%
\definecolor{currentstroke}{rgb}{0.000000,0.000000,0.000000}%
\pgfsetstrokecolor{currentstroke}%
\pgfsetstrokeopacity{0.000000}%
\pgfsetdash{}{0pt}%
\pgfpathmoveto{\pgfqpoint{1.753701in}{0.859974in}}%
\pgfpathlineto{\pgfqpoint{1.755182in}{0.892795in}}%
\pgfpathlineto{\pgfqpoint{1.756457in}{0.921234in}}%
\pgfpathlineto{\pgfqpoint{1.761097in}{0.916745in}}%
\pgfpathlineto{\pgfqpoint{1.772516in}{0.914576in}}%
\pgfpathlineto{\pgfqpoint{1.777557in}{0.916175in}}%
\pgfpathlineto{\pgfqpoint{1.784983in}{0.908773in}}%
\pgfpathlineto{\pgfqpoint{1.790476in}{0.911378in}}%
\pgfpathlineto{\pgfqpoint{1.792759in}{0.915583in}}%
\pgfpathlineto{\pgfqpoint{1.804759in}{0.911929in}}%
\pgfpathlineto{\pgfqpoint{1.812092in}{0.902672in}}%
\pgfpathlineto{\pgfqpoint{1.818444in}{0.902148in}}%
\pgfpathlineto{\pgfqpoint{1.816212in}{0.894668in}}%
\pgfpathlineto{\pgfqpoint{1.815104in}{0.862873in}}%
\pgfpathlineto{\pgfqpoint{1.787337in}{0.863827in}}%
\pgfpathlineto{\pgfqpoint{1.787184in}{0.858514in}}%
\pgfpathclose%
\pgfusepath{fill}%
\end{pgfscope}%
\begin{pgfscope}%
\pgfpathrectangle{\pgfqpoint{0.100000in}{0.100000in}}{\pgfqpoint{3.420221in}{2.189500in}}%
\pgfusepath{clip}%
\pgfsetbuttcap%
\pgfsetmiterjoin%
\definecolor{currentfill}{rgb}{0.000000,0.545098,0.727451}%
\pgfsetfillcolor{currentfill}%
\pgfsetlinewidth{0.000000pt}%
\definecolor{currentstroke}{rgb}{0.000000,0.000000,0.000000}%
\pgfsetstrokecolor{currentstroke}%
\pgfsetstrokeopacity{0.000000}%
\pgfsetdash{}{0pt}%
\pgfpathmoveto{\pgfqpoint{2.116239in}{1.205956in}}%
\pgfpathlineto{\pgfqpoint{2.090345in}{1.206259in}}%
\pgfpathlineto{\pgfqpoint{2.090063in}{1.194302in}}%
\pgfpathlineto{\pgfqpoint{2.086718in}{1.194349in}}%
\pgfpathlineto{\pgfqpoint{2.083284in}{1.187891in}}%
\pgfpathlineto{\pgfqpoint{2.072437in}{1.188130in}}%
\pgfpathlineto{\pgfqpoint{2.072567in}{1.192484in}}%
\pgfpathlineto{\pgfqpoint{2.057351in}{1.192924in}}%
\pgfpathlineto{\pgfqpoint{2.057682in}{1.203082in}}%
\pgfpathlineto{\pgfqpoint{2.056548in}{1.216411in}}%
\pgfpathlineto{\pgfqpoint{2.056918in}{1.242853in}}%
\pgfpathlineto{\pgfqpoint{2.089325in}{1.242445in}}%
\pgfpathlineto{\pgfqpoint{2.089714in}{1.271428in}}%
\pgfpathlineto{\pgfqpoint{2.115718in}{1.270824in}}%
\pgfpathlineto{\pgfqpoint{2.115532in}{1.253198in}}%
\pgfpathlineto{\pgfqpoint{2.115043in}{1.215226in}}%
\pgfpathclose%
\pgfusepath{fill}%
\end{pgfscope}%
\begin{pgfscope}%
\pgfpathrectangle{\pgfqpoint{0.100000in}{0.100000in}}{\pgfqpoint{3.420221in}{2.189500in}}%
\pgfusepath{clip}%
\pgfsetbuttcap%
\pgfsetmiterjoin%
\definecolor{currentfill}{rgb}{0.000000,0.596078,0.701961}%
\pgfsetfillcolor{currentfill}%
\pgfsetlinewidth{0.000000pt}%
\definecolor{currentstroke}{rgb}{0.000000,0.000000,0.000000}%
\pgfsetstrokecolor{currentstroke}%
\pgfsetstrokeopacity{0.000000}%
\pgfsetdash{}{0pt}%
\pgfpathmoveto{\pgfqpoint{0.478865in}{1.105297in}}%
\pgfpathlineto{\pgfqpoint{0.470153in}{1.109415in}}%
\pgfpathlineto{\pgfqpoint{0.473007in}{1.115698in}}%
\pgfpathlineto{\pgfqpoint{0.482787in}{1.108970in}}%
\pgfpathclose%
\pgfusepath{fill}%
\end{pgfscope}%
\begin{pgfscope}%
\pgfpathrectangle{\pgfqpoint{0.100000in}{0.100000in}}{\pgfqpoint{3.420221in}{2.189500in}}%
\pgfusepath{clip}%
\pgfsetbuttcap%
\pgfsetmiterjoin%
\definecolor{currentfill}{rgb}{0.000000,0.596078,0.701961}%
\pgfsetfillcolor{currentfill}%
\pgfsetlinewidth{0.000000pt}%
\definecolor{currentstroke}{rgb}{0.000000,0.000000,0.000000}%
\pgfsetstrokecolor{currentstroke}%
\pgfsetstrokeopacity{0.000000}%
\pgfsetdash{}{0pt}%
\pgfpathmoveto{\pgfqpoint{0.454303in}{1.107209in}}%
\pgfpathlineto{\pgfqpoint{0.449343in}{1.115892in}}%
\pgfpathlineto{\pgfqpoint{0.454907in}{1.117308in}}%
\pgfpathlineto{\pgfqpoint{0.463826in}{1.111559in}}%
\pgfpathlineto{\pgfqpoint{0.462110in}{1.108681in}}%
\pgfpathclose%
\pgfusepath{fill}%
\end{pgfscope}%
\begin{pgfscope}%
\pgfpathrectangle{\pgfqpoint{0.100000in}{0.100000in}}{\pgfqpoint{3.420221in}{2.189500in}}%
\pgfusepath{clip}%
\pgfsetbuttcap%
\pgfsetmiterjoin%
\definecolor{currentfill}{rgb}{0.000000,0.596078,0.701961}%
\pgfsetfillcolor{currentfill}%
\pgfsetlinewidth{0.000000pt}%
\definecolor{currentstroke}{rgb}{0.000000,0.000000,0.000000}%
\pgfsetstrokecolor{currentstroke}%
\pgfsetstrokeopacity{0.000000}%
\pgfsetdash{}{0pt}%
\pgfpathmoveto{\pgfqpoint{0.514520in}{1.168905in}}%
\pgfpathlineto{\pgfqpoint{0.505599in}{1.136334in}}%
\pgfpathlineto{\pgfqpoint{0.501924in}{1.131217in}}%
\pgfpathlineto{\pgfqpoint{0.494685in}{1.136825in}}%
\pgfpathlineto{\pgfqpoint{0.487082in}{1.137061in}}%
\pgfpathlineto{\pgfqpoint{0.478759in}{1.140276in}}%
\pgfpathlineto{\pgfqpoint{0.472577in}{1.145892in}}%
\pgfpathlineto{\pgfqpoint{0.459455in}{1.150831in}}%
\pgfpathlineto{\pgfqpoint{0.445038in}{1.152519in}}%
\pgfpathlineto{\pgfqpoint{0.443288in}{1.159455in}}%
\pgfpathlineto{\pgfqpoint{0.436318in}{1.165686in}}%
\pgfpathlineto{\pgfqpoint{0.441631in}{1.174575in}}%
\pgfpathlineto{\pgfqpoint{0.440561in}{1.178434in}}%
\pgfpathlineto{\pgfqpoint{0.444193in}{1.184369in}}%
\pgfpathlineto{\pgfqpoint{0.441686in}{1.189532in}}%
\pgfpathlineto{\pgfqpoint{0.445637in}{1.196247in}}%
\pgfpathlineto{\pgfqpoint{0.448025in}{1.206858in}}%
\pgfpathlineto{\pgfqpoint{0.441455in}{1.209797in}}%
\pgfpathlineto{\pgfqpoint{0.435741in}{1.219053in}}%
\pgfpathlineto{\pgfqpoint{0.440238in}{1.227102in}}%
\pgfpathlineto{\pgfqpoint{0.439293in}{1.232998in}}%
\pgfpathlineto{\pgfqpoint{0.433761in}{1.235465in}}%
\pgfpathlineto{\pgfqpoint{0.428161in}{1.250329in}}%
\pgfpathlineto{\pgfqpoint{0.421743in}{1.255139in}}%
\pgfpathlineto{\pgfqpoint{0.420817in}{1.265677in}}%
\pgfpathlineto{\pgfqpoint{0.449032in}{1.257001in}}%
\pgfpathlineto{\pgfqpoint{0.487010in}{1.245847in}}%
\pgfpathlineto{\pgfqpoint{0.488169in}{1.245512in}}%
\pgfpathlineto{\pgfqpoint{0.484591in}{1.232887in}}%
\pgfpathlineto{\pgfqpoint{0.490933in}{1.231100in}}%
\pgfpathlineto{\pgfqpoint{0.489090in}{1.224759in}}%
\pgfpathlineto{\pgfqpoint{0.493159in}{1.219096in}}%
\pgfpathlineto{\pgfqpoint{0.499463in}{1.215016in}}%
\pgfpathlineto{\pgfqpoint{0.497671in}{1.208663in}}%
\pgfpathlineto{\pgfqpoint{0.501840in}{1.207467in}}%
\pgfpathlineto{\pgfqpoint{0.500067in}{1.201159in}}%
\pgfpathlineto{\pgfqpoint{0.508477in}{1.198730in}}%
\pgfpathlineto{\pgfqpoint{0.506687in}{1.192390in}}%
\pgfpathlineto{\pgfqpoint{0.513498in}{1.190902in}}%
\pgfpathlineto{\pgfqpoint{0.511214in}{1.184343in}}%
\pgfpathlineto{\pgfqpoint{0.516225in}{1.182117in}}%
\pgfpathclose%
\pgfusepath{fill}%
\end{pgfscope}%
\begin{pgfscope}%
\pgfpathrectangle{\pgfqpoint{0.100000in}{0.100000in}}{\pgfqpoint{3.420221in}{2.189500in}}%
\pgfusepath{clip}%
\pgfsetbuttcap%
\pgfsetmiterjoin%
\definecolor{currentfill}{rgb}{0.000000,0.211765,0.894118}%
\pgfsetfillcolor{currentfill}%
\pgfsetlinewidth{0.000000pt}%
\definecolor{currentstroke}{rgb}{0.000000,0.000000,0.000000}%
\pgfsetstrokecolor{currentstroke}%
\pgfsetstrokeopacity{0.000000}%
\pgfsetdash{}{0pt}%
\pgfpathmoveto{\pgfqpoint{3.200162in}{1.744073in}}%
\pgfpathlineto{\pgfqpoint{3.179384in}{1.740036in}}%
\pgfpathlineto{\pgfqpoint{3.175442in}{1.757573in}}%
\pgfpathlineto{\pgfqpoint{3.164792in}{1.765623in}}%
\pgfpathlineto{\pgfqpoint{3.164020in}{1.774485in}}%
\pgfpathlineto{\pgfqpoint{3.156756in}{1.791032in}}%
\pgfpathlineto{\pgfqpoint{3.156649in}{1.802296in}}%
\pgfpathlineto{\pgfqpoint{3.159337in}{1.809401in}}%
\pgfpathlineto{\pgfqpoint{3.156415in}{1.816282in}}%
\pgfpathlineto{\pgfqpoint{3.155540in}{1.825986in}}%
\pgfpathlineto{\pgfqpoint{3.149091in}{1.833562in}}%
\pgfpathlineto{\pgfqpoint{3.148662in}{1.847850in}}%
\pgfpathlineto{\pgfqpoint{3.145755in}{1.849771in}}%
\pgfpathlineto{\pgfqpoint{3.146427in}{1.856239in}}%
\pgfpathlineto{\pgfqpoint{3.144567in}{1.862878in}}%
\pgfpathlineto{\pgfqpoint{3.181543in}{1.872294in}}%
\pgfpathlineto{\pgfqpoint{3.185390in}{1.872783in}}%
\pgfpathlineto{\pgfqpoint{3.189732in}{1.860303in}}%
\pgfpathlineto{\pgfqpoint{3.188331in}{1.855917in}}%
\pgfpathlineto{\pgfqpoint{3.197752in}{1.853740in}}%
\pgfpathlineto{\pgfqpoint{3.196024in}{1.846393in}}%
\pgfpathlineto{\pgfqpoint{3.202734in}{1.844896in}}%
\pgfpathlineto{\pgfqpoint{3.200881in}{1.837697in}}%
\pgfpathlineto{\pgfqpoint{3.199438in}{1.831090in}}%
\pgfpathlineto{\pgfqpoint{3.192917in}{1.833605in}}%
\pgfpathlineto{\pgfqpoint{3.191403in}{1.826934in}}%
\pgfpathlineto{\pgfqpoint{3.182576in}{1.828873in}}%
\pgfpathlineto{\pgfqpoint{3.180106in}{1.815085in}}%
\pgfpathlineto{\pgfqpoint{3.181039in}{1.806162in}}%
\pgfpathlineto{\pgfqpoint{3.183630in}{1.799703in}}%
\pgfpathlineto{\pgfqpoint{3.193582in}{1.798886in}}%
\pgfpathlineto{\pgfqpoint{3.188825in}{1.787818in}}%
\pgfpathlineto{\pgfqpoint{3.184410in}{1.784119in}}%
\pgfpathlineto{\pgfqpoint{3.195514in}{1.781903in}}%
\pgfpathlineto{\pgfqpoint{3.195146in}{1.775046in}}%
\pgfpathlineto{\pgfqpoint{3.202044in}{1.774135in}}%
\pgfpathlineto{\pgfqpoint{3.200707in}{1.759250in}}%
\pgfpathlineto{\pgfqpoint{3.204565in}{1.758178in}}%
\pgfpathlineto{\pgfqpoint{3.204275in}{1.749569in}}%
\pgfpathclose%
\pgfusepath{fill}%
\end{pgfscope}%
\begin{pgfscope}%
\pgfpathrectangle{\pgfqpoint{0.100000in}{0.100000in}}{\pgfqpoint{3.420221in}{2.189500in}}%
\pgfusepath{clip}%
\pgfsetbuttcap%
\pgfsetmiterjoin%
\definecolor{currentfill}{rgb}{0.000000,0.607843,0.696078}%
\pgfsetfillcolor{currentfill}%
\pgfsetlinewidth{0.000000pt}%
\definecolor{currentstroke}{rgb}{0.000000,0.000000,0.000000}%
\pgfsetstrokecolor{currentstroke}%
\pgfsetstrokeopacity{0.000000}%
\pgfsetdash{}{0pt}%
\pgfpathmoveto{\pgfqpoint{2.225943in}{0.768896in}}%
\pgfpathlineto{\pgfqpoint{2.223339in}{0.775609in}}%
\pgfpathlineto{\pgfqpoint{2.226956in}{0.777554in}}%
\pgfpathlineto{\pgfqpoint{2.225587in}{0.790893in}}%
\pgfpathlineto{\pgfqpoint{2.227198in}{0.794188in}}%
\pgfpathlineto{\pgfqpoint{2.247770in}{0.795038in}}%
\pgfpathlineto{\pgfqpoint{2.253368in}{0.798171in}}%
\pgfpathlineto{\pgfqpoint{2.258416in}{0.798343in}}%
\pgfpathlineto{\pgfqpoint{2.259549in}{0.788233in}}%
\pgfpathlineto{\pgfqpoint{2.265391in}{0.788582in}}%
\pgfpathlineto{\pgfqpoint{2.264985in}{0.798606in}}%
\pgfpathlineto{\pgfqpoint{2.273549in}{0.799049in}}%
\pgfpathlineto{\pgfqpoint{2.273744in}{0.802309in}}%
\pgfpathlineto{\pgfqpoint{2.284699in}{0.794436in}}%
\pgfpathlineto{\pgfqpoint{2.281604in}{0.792301in}}%
\pgfpathlineto{\pgfqpoint{2.279961in}{0.784207in}}%
\pgfpathlineto{\pgfqpoint{2.275685in}{0.781902in}}%
\pgfpathlineto{\pgfqpoint{2.274598in}{0.772623in}}%
\pgfpathlineto{\pgfqpoint{2.276385in}{0.754011in}}%
\pgfpathlineto{\pgfqpoint{2.275940in}{0.732891in}}%
\pgfpathlineto{\pgfqpoint{2.276217in}{0.726297in}}%
\pgfpathlineto{\pgfqpoint{2.249451in}{0.725023in}}%
\pgfpathlineto{\pgfqpoint{2.240146in}{0.729147in}}%
\pgfpathlineto{\pgfqpoint{2.236414in}{0.742080in}}%
\pgfpathlineto{\pgfqpoint{2.245366in}{0.747366in}}%
\pgfpathlineto{\pgfqpoint{2.246138in}{0.751433in}}%
\pgfpathlineto{\pgfqpoint{2.253136in}{0.755921in}}%
\pgfpathlineto{\pgfqpoint{2.255463in}{0.762339in}}%
\pgfpathlineto{\pgfqpoint{2.247037in}{0.764533in}}%
\pgfpathlineto{\pgfqpoint{2.248690in}{0.773291in}}%
\pgfpathlineto{\pgfqpoint{2.236916in}{0.769262in}}%
\pgfpathclose%
\pgfusepath{fill}%
\end{pgfscope}%
\begin{pgfscope}%
\pgfpathrectangle{\pgfqpoint{0.100000in}{0.100000in}}{\pgfqpoint{3.420221in}{2.189500in}}%
\pgfusepath{clip}%
\pgfsetbuttcap%
\pgfsetmiterjoin%
\definecolor{currentfill}{rgb}{0.000000,0.725490,0.637255}%
\pgfsetfillcolor{currentfill}%
\pgfsetlinewidth{0.000000pt}%
\definecolor{currentstroke}{rgb}{0.000000,0.000000,0.000000}%
\pgfsetstrokecolor{currentstroke}%
\pgfsetstrokeopacity{0.000000}%
\pgfsetdash{}{0pt}%
\pgfpathmoveto{\pgfqpoint{0.488169in}{1.245512in}}%
\pgfpathlineto{\pgfqpoint{0.487010in}{1.245847in}}%
\pgfpathlineto{\pgfqpoint{0.482060in}{1.260928in}}%
\pgfpathlineto{\pgfqpoint{0.477913in}{1.262471in}}%
\pgfpathlineto{\pgfqpoint{0.473419in}{1.268086in}}%
\pgfpathlineto{\pgfqpoint{0.468216in}{1.280235in}}%
\pgfpathlineto{\pgfqpoint{0.471713in}{1.291234in}}%
\pgfpathlineto{\pgfqpoint{0.475871in}{1.291131in}}%
\pgfpathlineto{\pgfqpoint{0.478787in}{1.302760in}}%
\pgfpathlineto{\pgfqpoint{0.465872in}{1.326215in}}%
\pgfpathlineto{\pgfqpoint{0.498761in}{1.344453in}}%
\pgfpathlineto{\pgfqpoint{0.510656in}{1.345426in}}%
\pgfpathlineto{\pgfqpoint{0.524944in}{1.343740in}}%
\pgfpathlineto{\pgfqpoint{0.546350in}{1.355942in}}%
\pgfpathlineto{\pgfqpoint{0.552707in}{1.354226in}}%
\pgfpathlineto{\pgfqpoint{0.558087in}{1.358710in}}%
\pgfpathlineto{\pgfqpoint{0.559438in}{1.363441in}}%
\pgfpathlineto{\pgfqpoint{0.579476in}{1.374837in}}%
\pgfpathlineto{\pgfqpoint{0.580995in}{1.371409in}}%
\pgfpathlineto{\pgfqpoint{0.589162in}{1.368771in}}%
\pgfpathlineto{\pgfqpoint{0.591576in}{1.364281in}}%
\pgfpathlineto{\pgfqpoint{0.592090in}{1.356469in}}%
\pgfpathlineto{\pgfqpoint{0.597437in}{1.352277in}}%
\pgfpathlineto{\pgfqpoint{0.599539in}{1.346128in}}%
\pgfpathlineto{\pgfqpoint{0.604403in}{1.341943in}}%
\pgfpathlineto{\pgfqpoint{0.601118in}{1.334568in}}%
\pgfpathlineto{\pgfqpoint{0.604679in}{1.333098in}}%
\pgfpathlineto{\pgfqpoint{0.604621in}{1.318655in}}%
\pgfpathlineto{\pgfqpoint{0.608126in}{1.317437in}}%
\pgfpathlineto{\pgfqpoint{0.615093in}{1.307905in}}%
\pgfpathlineto{\pgfqpoint{0.616746in}{1.295743in}}%
\pgfpathlineto{\pgfqpoint{0.614053in}{1.292322in}}%
\pgfpathlineto{\pgfqpoint{0.613414in}{1.279921in}}%
\pgfpathlineto{\pgfqpoint{0.616499in}{1.269019in}}%
\pgfpathlineto{\pgfqpoint{0.615663in}{1.264966in}}%
\pgfpathlineto{\pgfqpoint{0.621647in}{1.252486in}}%
\pgfpathlineto{\pgfqpoint{0.619220in}{1.249559in}}%
\pgfpathlineto{\pgfqpoint{0.620731in}{1.239086in}}%
\pgfpathlineto{\pgfqpoint{0.618817in}{1.233638in}}%
\pgfpathlineto{\pgfqpoint{0.620089in}{1.217906in}}%
\pgfpathlineto{\pgfqpoint{0.616577in}{1.210635in}}%
\pgfpathlineto{\pgfqpoint{0.580156in}{1.220219in}}%
\pgfpathlineto{\pgfqpoint{0.526622in}{1.234749in}}%
\pgfpathclose%
\pgfusepath{fill}%
\end{pgfscope}%
\begin{pgfscope}%
\pgfpathrectangle{\pgfqpoint{0.100000in}{0.100000in}}{\pgfqpoint{3.420221in}{2.189500in}}%
\pgfusepath{clip}%
\pgfsetbuttcap%
\pgfsetmiterjoin%
\definecolor{currentfill}{rgb}{0.000000,0.431373,0.784314}%
\pgfsetfillcolor{currentfill}%
\pgfsetlinewidth{0.000000pt}%
\definecolor{currentstroke}{rgb}{0.000000,0.000000,0.000000}%
\pgfsetstrokecolor{currentstroke}%
\pgfsetstrokeopacity{0.000000}%
\pgfsetdash{}{0pt}%
\pgfpathmoveto{\pgfqpoint{0.960452in}{2.149271in}}%
\pgfpathlineto{\pgfqpoint{1.001353in}{2.139692in}}%
\pgfpathlineto{\pgfqpoint{1.024736in}{2.134554in}}%
\pgfpathlineto{\pgfqpoint{1.023944in}{2.122412in}}%
\pgfpathlineto{\pgfqpoint{1.025060in}{2.117243in}}%
\pgfpathlineto{\pgfqpoint{1.021348in}{2.110711in}}%
\pgfpathlineto{\pgfqpoint{1.011280in}{2.110902in}}%
\pgfpathlineto{\pgfqpoint{1.011792in}{2.104174in}}%
\pgfpathlineto{\pgfqpoint{1.006159in}{2.078392in}}%
\pgfpathlineto{\pgfqpoint{0.997658in}{2.080271in}}%
\pgfpathlineto{\pgfqpoint{0.994661in}{2.064735in}}%
\pgfpathlineto{\pgfqpoint{0.987549in}{2.066343in}}%
\pgfpathlineto{\pgfqpoint{0.984851in}{2.060321in}}%
\pgfpathlineto{\pgfqpoint{0.979492in}{2.058425in}}%
\pgfpathlineto{\pgfqpoint{0.967344in}{2.062481in}}%
\pgfpathlineto{\pgfqpoint{0.967967in}{2.067970in}}%
\pgfpathlineto{\pgfqpoint{0.964515in}{2.076365in}}%
\pgfpathlineto{\pgfqpoint{0.964080in}{2.089642in}}%
\pgfpathlineto{\pgfqpoint{0.947803in}{2.086498in}}%
\pgfpathlineto{\pgfqpoint{0.946711in}{2.091138in}}%
\pgfpathlineto{\pgfqpoint{0.951744in}{2.112346in}}%
\pgfpathclose%
\pgfusepath{fill}%
\end{pgfscope}%
\begin{pgfscope}%
\pgfpathrectangle{\pgfqpoint{0.100000in}{0.100000in}}{\pgfqpoint{3.420221in}{2.189500in}}%
\pgfusepath{clip}%
\pgfsetbuttcap%
\pgfsetmiterjoin%
\definecolor{currentfill}{rgb}{0.000000,0.521569,0.739216}%
\pgfsetfillcolor{currentfill}%
\pgfsetlinewidth{0.000000pt}%
\definecolor{currentstroke}{rgb}{0.000000,0.000000,0.000000}%
\pgfsetstrokecolor{currentstroke}%
\pgfsetstrokeopacity{0.000000}%
\pgfsetdash{}{0pt}%
\pgfpathmoveto{\pgfqpoint{2.389677in}{0.951522in}}%
\pgfpathlineto{\pgfqpoint{2.389499in}{0.970179in}}%
\pgfpathlineto{\pgfqpoint{2.386851in}{0.977676in}}%
\pgfpathlineto{\pgfqpoint{2.382516in}{0.977368in}}%
\pgfpathlineto{\pgfqpoint{2.381832in}{0.987606in}}%
\pgfpathlineto{\pgfqpoint{2.384082in}{0.987761in}}%
\pgfpathlineto{\pgfqpoint{2.383083in}{1.006687in}}%
\pgfpathlineto{\pgfqpoint{2.386877in}{1.006945in}}%
\pgfpathlineto{\pgfqpoint{2.391109in}{1.015740in}}%
\pgfpathlineto{\pgfqpoint{2.395215in}{1.017654in}}%
\pgfpathlineto{\pgfqpoint{2.407991in}{1.018513in}}%
\pgfpathlineto{\pgfqpoint{2.407898in}{1.021344in}}%
\pgfpathlineto{\pgfqpoint{2.422972in}{1.019911in}}%
\pgfpathlineto{\pgfqpoint{2.428545in}{1.020886in}}%
\pgfpathlineto{\pgfqpoint{2.431738in}{1.013857in}}%
\pgfpathlineto{\pgfqpoint{2.433212in}{0.992148in}}%
\pgfpathlineto{\pgfqpoint{2.419827in}{0.991287in}}%
\pgfpathlineto{\pgfqpoint{2.426932in}{0.983099in}}%
\pgfpathlineto{\pgfqpoint{2.425769in}{0.950686in}}%
\pgfpathlineto{\pgfqpoint{2.401920in}{0.949093in}}%
\pgfpathlineto{\pgfqpoint{2.401703in}{0.952386in}}%
\pgfpathclose%
\pgfusepath{fill}%
\end{pgfscope}%
\begin{pgfscope}%
\pgfpathrectangle{\pgfqpoint{0.100000in}{0.100000in}}{\pgfqpoint{3.420221in}{2.189500in}}%
\pgfusepath{clip}%
\pgfsetbuttcap%
\pgfsetmiterjoin%
\definecolor{currentfill}{rgb}{0.000000,0.823529,0.588235}%
\pgfsetfillcolor{currentfill}%
\pgfsetlinewidth{0.000000pt}%
\definecolor{currentstroke}{rgb}{0.000000,0.000000,0.000000}%
\pgfsetstrokecolor{currentstroke}%
\pgfsetstrokeopacity{0.000000}%
\pgfsetdash{}{0pt}%
\pgfpathmoveto{\pgfqpoint{0.619068in}{2.158988in}}%
\pgfpathlineto{\pgfqpoint{0.619623in}{2.154193in}}%
\pgfpathlineto{\pgfqpoint{0.611584in}{2.149681in}}%
\pgfpathlineto{\pgfqpoint{0.607311in}{2.142824in}}%
\pgfpathlineto{\pgfqpoint{0.596321in}{2.140085in}}%
\pgfpathlineto{\pgfqpoint{0.569389in}{2.148789in}}%
\pgfpathlineto{\pgfqpoint{0.567299in}{2.142345in}}%
\pgfpathlineto{\pgfqpoint{0.540920in}{2.151029in}}%
\pgfpathlineto{\pgfqpoint{0.526319in}{2.157140in}}%
\pgfpathlineto{\pgfqpoint{0.527885in}{2.173097in}}%
\pgfpathlineto{\pgfqpoint{0.526520in}{2.180077in}}%
\pgfpathlineto{\pgfqpoint{0.521583in}{2.187472in}}%
\pgfpathlineto{\pgfqpoint{0.521450in}{2.193486in}}%
\pgfpathlineto{\pgfqpoint{0.525216in}{2.209532in}}%
\pgfpathlineto{\pgfqpoint{0.533382in}{2.223092in}}%
\pgfpathlineto{\pgfqpoint{0.542632in}{2.212026in}}%
\pgfpathlineto{\pgfqpoint{0.549247in}{2.207701in}}%
\pgfpathlineto{\pgfqpoint{0.557713in}{2.197769in}}%
\pgfpathlineto{\pgfqpoint{0.566056in}{2.193686in}}%
\pgfpathlineto{\pgfqpoint{0.572920in}{2.192622in}}%
\pgfpathlineto{\pgfqpoint{0.577851in}{2.188053in}}%
\pgfpathlineto{\pgfqpoint{0.593554in}{2.181645in}}%
\pgfpathlineto{\pgfqpoint{0.599716in}{2.184252in}}%
\pgfpathlineto{\pgfqpoint{0.603041in}{2.176021in}}%
\pgfpathlineto{\pgfqpoint{0.618100in}{2.175929in}}%
\pgfpathlineto{\pgfqpoint{0.620207in}{2.171937in}}%
\pgfpathlineto{\pgfqpoint{0.618209in}{2.165009in}}%
\pgfpathclose%
\pgfusepath{fill}%
\end{pgfscope}%
\begin{pgfscope}%
\pgfpathrectangle{\pgfqpoint{0.100000in}{0.100000in}}{\pgfqpoint{3.420221in}{2.189500in}}%
\pgfusepath{clip}%
\pgfsetbuttcap%
\pgfsetmiterjoin%
\definecolor{currentfill}{rgb}{0.000000,0.470588,0.764706}%
\pgfsetfillcolor{currentfill}%
\pgfsetlinewidth{0.000000pt}%
\definecolor{currentstroke}{rgb}{0.000000,0.000000,0.000000}%
\pgfsetstrokecolor{currentstroke}%
\pgfsetstrokeopacity{0.000000}%
\pgfsetdash{}{0pt}%
\pgfpathmoveto{\pgfqpoint{2.589083in}{1.350403in}}%
\pgfpathlineto{\pgfqpoint{2.589440in}{1.347028in}}%
\pgfpathlineto{\pgfqpoint{2.561678in}{1.344236in}}%
\pgfpathlineto{\pgfqpoint{2.559393in}{1.363745in}}%
\pgfpathlineto{\pgfqpoint{2.563991in}{1.364311in}}%
\pgfpathlineto{\pgfqpoint{2.563370in}{1.380525in}}%
\pgfpathlineto{\pgfqpoint{2.585651in}{1.383015in}}%
\pgfpathclose%
\pgfusepath{fill}%
\end{pgfscope}%
\begin{pgfscope}%
\pgfpathrectangle{\pgfqpoint{0.100000in}{0.100000in}}{\pgfqpoint{3.420221in}{2.189500in}}%
\pgfusepath{clip}%
\pgfsetbuttcap%
\pgfsetmiterjoin%
\definecolor{currentfill}{rgb}{0.000000,0.321569,0.839216}%
\pgfsetfillcolor{currentfill}%
\pgfsetlinewidth{0.000000pt}%
\definecolor{currentstroke}{rgb}{0.000000,0.000000,0.000000}%
\pgfsetstrokecolor{currentstroke}%
\pgfsetstrokeopacity{0.000000}%
\pgfsetdash{}{0pt}%
\pgfpathmoveto{\pgfqpoint{2.153048in}{1.743437in}}%
\pgfpathlineto{\pgfqpoint{2.121127in}{1.742548in}}%
\pgfpathlineto{\pgfqpoint{2.120938in}{1.748680in}}%
\pgfpathlineto{\pgfqpoint{2.126468in}{1.756930in}}%
\pgfpathlineto{\pgfqpoint{2.126546in}{1.760332in}}%
\pgfpathlineto{\pgfqpoint{2.119818in}{1.769466in}}%
\pgfpathlineto{\pgfqpoint{2.113810in}{1.769967in}}%
\pgfpathlineto{\pgfqpoint{2.115787in}{1.781674in}}%
\pgfpathlineto{\pgfqpoint{2.118635in}{1.784268in}}%
\pgfpathlineto{\pgfqpoint{2.122208in}{1.793985in}}%
\pgfpathlineto{\pgfqpoint{2.125967in}{1.797209in}}%
\pgfpathlineto{\pgfqpoint{2.140984in}{1.803957in}}%
\pgfpathlineto{\pgfqpoint{2.143808in}{1.808422in}}%
\pgfpathlineto{\pgfqpoint{2.143638in}{1.814683in}}%
\pgfpathlineto{\pgfqpoint{2.156396in}{1.815109in}}%
\pgfpathlineto{\pgfqpoint{2.157674in}{1.802019in}}%
\pgfpathlineto{\pgfqpoint{2.158637in}{1.776085in}}%
\pgfpathlineto{\pgfqpoint{2.152118in}{1.775862in}}%
\pgfpathclose%
\pgfusepath{fill}%
\end{pgfscope}%
\begin{pgfscope}%
\pgfpathrectangle{\pgfqpoint{0.100000in}{0.100000in}}{\pgfqpoint{3.420221in}{2.189500in}}%
\pgfusepath{clip}%
\pgfsetbuttcap%
\pgfsetmiterjoin%
\definecolor{currentfill}{rgb}{0.000000,0.600000,0.700000}%
\pgfsetfillcolor{currentfill}%
\pgfsetlinewidth{0.000000pt}%
\definecolor{currentstroke}{rgb}{0.000000,0.000000,0.000000}%
\pgfsetstrokecolor{currentstroke}%
\pgfsetstrokeopacity{0.000000}%
\pgfsetdash{}{0pt}%
\pgfpathmoveto{\pgfqpoint{2.320479in}{0.824381in}}%
\pgfpathlineto{\pgfqpoint{2.316751in}{0.815147in}}%
\pgfpathlineto{\pgfqpoint{2.306701in}{0.807375in}}%
\pgfpathlineto{\pgfqpoint{2.291991in}{0.800715in}}%
\pgfpathlineto{\pgfqpoint{2.284699in}{0.794436in}}%
\pgfpathlineto{\pgfqpoint{2.273744in}{0.802309in}}%
\pgfpathlineto{\pgfqpoint{2.273685in}{0.816434in}}%
\pgfpathlineto{\pgfqpoint{2.270461in}{0.818516in}}%
\pgfpathlineto{\pgfqpoint{2.276676in}{0.825358in}}%
\pgfpathlineto{\pgfqpoint{2.275960in}{0.838447in}}%
\pgfpathlineto{\pgfqpoint{2.271674in}{0.851325in}}%
\pgfpathlineto{\pgfqpoint{2.281792in}{0.851849in}}%
\pgfpathlineto{\pgfqpoint{2.281562in}{0.856258in}}%
\pgfpathlineto{\pgfqpoint{2.287992in}{0.856600in}}%
\pgfpathlineto{\pgfqpoint{2.288188in}{0.851922in}}%
\pgfpathlineto{\pgfqpoint{2.295734in}{0.854983in}}%
\pgfpathlineto{\pgfqpoint{2.296464in}{0.842378in}}%
\pgfpathlineto{\pgfqpoint{2.290210in}{0.839101in}}%
\pgfpathlineto{\pgfqpoint{2.296441in}{0.832947in}}%
\pgfpathclose%
\pgfusepath{fill}%
\end{pgfscope}%
\begin{pgfscope}%
\pgfpathrectangle{\pgfqpoint{0.100000in}{0.100000in}}{\pgfqpoint{3.420221in}{2.189500in}}%
\pgfusepath{clip}%
\pgfsetbuttcap%
\pgfsetmiterjoin%
\definecolor{currentfill}{rgb}{0.000000,0.407843,0.796078}%
\pgfsetfillcolor{currentfill}%
\pgfsetlinewidth{0.000000pt}%
\definecolor{currentstroke}{rgb}{0.000000,0.000000,0.000000}%
\pgfsetstrokecolor{currentstroke}%
\pgfsetstrokeopacity{0.000000}%
\pgfsetdash{}{0pt}%
\pgfpathmoveto{\pgfqpoint{2.064332in}{1.812925in}}%
\pgfpathlineto{\pgfqpoint{2.064505in}{1.799823in}}%
\pgfpathlineto{\pgfqpoint{2.066679in}{1.799816in}}%
\pgfpathlineto{\pgfqpoint{2.067103in}{1.787715in}}%
\pgfpathlineto{\pgfqpoint{2.036134in}{1.787383in}}%
\pgfpathlineto{\pgfqpoint{2.040430in}{1.783885in}}%
\pgfpathlineto{\pgfqpoint{1.994758in}{1.783602in}}%
\pgfpathlineto{\pgfqpoint{1.994510in}{1.822020in}}%
\pgfpathlineto{\pgfqpoint{1.993634in}{1.854921in}}%
\pgfpathlineto{\pgfqpoint{1.993682in}{1.861514in}}%
\pgfpathlineto{\pgfqpoint{2.013137in}{1.861460in}}%
\pgfpathlineto{\pgfqpoint{2.013620in}{1.830480in}}%
\pgfpathlineto{\pgfqpoint{2.026391in}{1.823622in}}%
\pgfpathlineto{\pgfqpoint{2.032022in}{1.825885in}}%
\pgfpathlineto{\pgfqpoint{2.036006in}{1.823123in}}%
\pgfpathlineto{\pgfqpoint{2.034822in}{1.812668in}}%
\pgfpathclose%
\pgfusepath{fill}%
\end{pgfscope}%
\begin{pgfscope}%
\pgfpathrectangle{\pgfqpoint{0.100000in}{0.100000in}}{\pgfqpoint{3.420221in}{2.189500in}}%
\pgfusepath{clip}%
\pgfsetbuttcap%
\pgfsetmiterjoin%
\definecolor{currentfill}{rgb}{0.000000,0.505882,0.747059}%
\pgfsetfillcolor{currentfill}%
\pgfsetlinewidth{0.000000pt}%
\definecolor{currentstroke}{rgb}{0.000000,0.000000,0.000000}%
\pgfsetstrokecolor{currentstroke}%
\pgfsetstrokeopacity{0.000000}%
\pgfsetdash{}{0pt}%
\pgfpathmoveto{\pgfqpoint{2.120392in}{0.859963in}}%
\pgfpathlineto{\pgfqpoint{2.120339in}{0.854495in}}%
\pgfpathlineto{\pgfqpoint{2.129177in}{0.854458in}}%
\pgfpathlineto{\pgfqpoint{2.129133in}{0.827351in}}%
\pgfpathlineto{\pgfqpoint{2.077586in}{0.826587in}}%
\pgfpathlineto{\pgfqpoint{2.074034in}{0.831961in}}%
\pgfpathlineto{\pgfqpoint{2.074614in}{0.836039in}}%
\pgfpathlineto{\pgfqpoint{2.080404in}{0.837602in}}%
\pgfpathlineto{\pgfqpoint{2.086042in}{0.850263in}}%
\pgfpathlineto{\pgfqpoint{2.081817in}{0.857201in}}%
\pgfpathlineto{\pgfqpoint{2.082279in}{0.861488in}}%
\pgfpathlineto{\pgfqpoint{2.097337in}{0.861276in}}%
\pgfpathlineto{\pgfqpoint{2.097317in}{0.859103in}}%
\pgfpathlineto{\pgfqpoint{2.113822in}{0.858892in}}%
\pgfpathclose%
\pgfusepath{fill}%
\end{pgfscope}%
\begin{pgfscope}%
\pgfpathrectangle{\pgfqpoint{0.100000in}{0.100000in}}{\pgfqpoint{3.420221in}{2.189500in}}%
\pgfusepath{clip}%
\pgfsetbuttcap%
\pgfsetmiterjoin%
\definecolor{currentfill}{rgb}{0.000000,0.074510,0.962745}%
\pgfsetfillcolor{currentfill}%
\pgfsetlinewidth{0.000000pt}%
\definecolor{currentstroke}{rgb}{0.000000,0.000000,0.000000}%
\pgfsetstrokecolor{currentstroke}%
\pgfsetstrokeopacity{0.000000}%
\pgfsetdash{}{0pt}%
\pgfpathmoveto{\pgfqpoint{2.331404in}{1.574268in}}%
\pgfpathlineto{\pgfqpoint{2.311883in}{1.572845in}}%
\pgfpathlineto{\pgfqpoint{2.286050in}{1.572239in}}%
\pgfpathlineto{\pgfqpoint{2.286223in}{1.568970in}}%
\pgfpathlineto{\pgfqpoint{2.253857in}{1.567165in}}%
\pgfpathlineto{\pgfqpoint{2.252236in}{1.596264in}}%
\pgfpathlineto{\pgfqpoint{2.245877in}{1.596739in}}%
\pgfpathlineto{\pgfqpoint{2.239381in}{1.593441in}}%
\pgfpathlineto{\pgfqpoint{2.238385in}{1.612287in}}%
\pgfpathlineto{\pgfqpoint{2.237751in}{1.622035in}}%
\pgfpathlineto{\pgfqpoint{2.257368in}{1.623082in}}%
\pgfpathlineto{\pgfqpoint{2.256952in}{1.629612in}}%
\pgfpathlineto{\pgfqpoint{2.285551in}{1.631176in}}%
\pgfpathlineto{\pgfqpoint{2.295635in}{1.631881in}}%
\pgfpathlineto{\pgfqpoint{2.294226in}{1.657377in}}%
\pgfpathlineto{\pgfqpoint{2.317428in}{1.658857in}}%
\pgfpathlineto{\pgfqpoint{2.317951in}{1.650877in}}%
\pgfpathlineto{\pgfqpoint{2.314292in}{1.642342in}}%
\pgfpathlineto{\pgfqpoint{2.314880in}{1.633082in}}%
\pgfpathlineto{\pgfqpoint{2.327879in}{1.633172in}}%
\pgfpathclose%
\pgfusepath{fill}%
\end{pgfscope}%
\begin{pgfscope}%
\pgfpathrectangle{\pgfqpoint{0.100000in}{0.100000in}}{\pgfqpoint{3.420221in}{2.189500in}}%
\pgfusepath{clip}%
\pgfsetbuttcap%
\pgfsetmiterjoin%
\definecolor{currentfill}{rgb}{0.000000,0.407843,0.796078}%
\pgfsetfillcolor{currentfill}%
\pgfsetlinewidth{0.000000pt}%
\definecolor{currentstroke}{rgb}{0.000000,0.000000,0.000000}%
\pgfsetstrokecolor{currentstroke}%
\pgfsetstrokeopacity{0.000000}%
\pgfsetdash{}{0pt}%
\pgfpathmoveto{\pgfqpoint{1.217375in}{1.789489in}}%
\pgfpathlineto{\pgfqpoint{1.253980in}{1.783901in}}%
\pgfpathlineto{\pgfqpoint{1.254982in}{1.783265in}}%
\pgfpathlineto{\pgfqpoint{1.298884in}{1.776601in}}%
\pgfpathlineto{\pgfqpoint{1.316700in}{1.774159in}}%
\pgfpathlineto{\pgfqpoint{1.317807in}{1.764287in}}%
\pgfpathlineto{\pgfqpoint{1.322655in}{1.752598in}}%
\pgfpathlineto{\pgfqpoint{1.329176in}{1.749577in}}%
\pgfpathlineto{\pgfqpoint{1.338406in}{1.741463in}}%
\pgfpathlineto{\pgfqpoint{1.343916in}{1.731960in}}%
\pgfpathlineto{\pgfqpoint{1.348518in}{1.727558in}}%
\pgfpathlineto{\pgfqpoint{1.350357in}{1.718012in}}%
\pgfpathlineto{\pgfqpoint{1.348673in}{1.706219in}}%
\pgfpathlineto{\pgfqpoint{1.273721in}{1.717100in}}%
\pgfpathlineto{\pgfqpoint{1.272716in}{1.710478in}}%
\pgfpathlineto{\pgfqpoint{1.259817in}{1.712449in}}%
\pgfpathlineto{\pgfqpoint{1.258837in}{1.705861in}}%
\pgfpathlineto{\pgfqpoint{1.252192in}{1.706867in}}%
\pgfpathlineto{\pgfqpoint{1.251377in}{1.700442in}}%
\pgfpathlineto{\pgfqpoint{1.241733in}{1.701918in}}%
\pgfpathlineto{\pgfqpoint{1.240747in}{1.695356in}}%
\pgfpathlineto{\pgfqpoint{1.228882in}{1.697003in}}%
\pgfpathlineto{\pgfqpoint{1.223344in}{1.706806in}}%
\pgfpathlineto{\pgfqpoint{1.218369in}{1.709960in}}%
\pgfpathlineto{\pgfqpoint{1.212589in}{1.707959in}}%
\pgfpathlineto{\pgfqpoint{1.210289in}{1.704167in}}%
\pgfpathlineto{\pgfqpoint{1.203328in}{1.700206in}}%
\pgfpathlineto{\pgfqpoint{1.199941in}{1.711997in}}%
\pgfpathlineto{\pgfqpoint{1.194304in}{1.712969in}}%
\pgfpathlineto{\pgfqpoint{1.191665in}{1.717904in}}%
\pgfpathlineto{\pgfqpoint{1.193201in}{1.727149in}}%
\pgfpathlineto{\pgfqpoint{1.190621in}{1.731916in}}%
\pgfpathlineto{\pgfqpoint{1.189365in}{1.740527in}}%
\pgfpathlineto{\pgfqpoint{1.184011in}{1.751383in}}%
\pgfpathlineto{\pgfqpoint{1.186106in}{1.757553in}}%
\pgfpathlineto{\pgfqpoint{1.181769in}{1.763451in}}%
\pgfpathlineto{\pgfqpoint{1.166310in}{1.766100in}}%
\pgfpathlineto{\pgfqpoint{1.167390in}{1.772292in}}%
\pgfpathlineto{\pgfqpoint{1.146960in}{1.775893in}}%
\pgfpathlineto{\pgfqpoint{1.151401in}{1.800731in}}%
\pgfpathlineto{\pgfqpoint{1.169639in}{1.796823in}}%
\pgfpathclose%
\pgfusepath{fill}%
\end{pgfscope}%
\begin{pgfscope}%
\pgfpathrectangle{\pgfqpoint{0.100000in}{0.100000in}}{\pgfqpoint{3.420221in}{2.189500in}}%
\pgfusepath{clip}%
\pgfsetbuttcap%
\pgfsetmiterjoin%
\definecolor{currentfill}{rgb}{0.000000,0.415686,0.792157}%
\pgfsetfillcolor{currentfill}%
\pgfsetlinewidth{0.000000pt}%
\definecolor{currentstroke}{rgb}{0.000000,0.000000,0.000000}%
\pgfsetstrokecolor{currentstroke}%
\pgfsetstrokeopacity{0.000000}%
\pgfsetdash{}{0pt}%
\pgfpathmoveto{\pgfqpoint{2.424054in}{0.896398in}}%
\pgfpathlineto{\pgfqpoint{2.419301in}{0.896087in}}%
\pgfpathlineto{\pgfqpoint{2.416792in}{0.891401in}}%
\pgfpathlineto{\pgfqpoint{2.408029in}{0.888072in}}%
\pgfpathlineto{\pgfqpoint{2.395019in}{0.889429in}}%
\pgfpathlineto{\pgfqpoint{2.394314in}{0.899243in}}%
\pgfpathlineto{\pgfqpoint{2.381141in}{0.898448in}}%
\pgfpathlineto{\pgfqpoint{2.381406in}{0.894088in}}%
\pgfpathlineto{\pgfqpoint{2.365312in}{0.890611in}}%
\pgfpathlineto{\pgfqpoint{2.345624in}{0.889359in}}%
\pgfpathlineto{\pgfqpoint{2.344952in}{0.900296in}}%
\pgfpathlineto{\pgfqpoint{2.343465in}{0.922403in}}%
\pgfpathlineto{\pgfqpoint{2.359878in}{0.923352in}}%
\pgfpathlineto{\pgfqpoint{2.358787in}{0.939731in}}%
\pgfpathlineto{\pgfqpoint{2.378437in}{0.940997in}}%
\pgfpathlineto{\pgfqpoint{2.386974in}{0.942656in}}%
\pgfpathlineto{\pgfqpoint{2.386476in}{0.950242in}}%
\pgfpathlineto{\pgfqpoint{2.389677in}{0.951522in}}%
\pgfpathlineto{\pgfqpoint{2.401703in}{0.952386in}}%
\pgfpathlineto{\pgfqpoint{2.401920in}{0.949093in}}%
\pgfpathlineto{\pgfqpoint{2.425769in}{0.950686in}}%
\pgfpathlineto{\pgfqpoint{2.425499in}{0.939950in}}%
\pgfpathclose%
\pgfusepath{fill}%
\end{pgfscope}%
\begin{pgfscope}%
\pgfpathrectangle{\pgfqpoint{0.100000in}{0.100000in}}{\pgfqpoint{3.420221in}{2.189500in}}%
\pgfusepath{clip}%
\pgfsetbuttcap%
\pgfsetmiterjoin%
\definecolor{currentfill}{rgb}{0.000000,0.803922,0.598039}%
\pgfsetfillcolor{currentfill}%
\pgfsetlinewidth{0.000000pt}%
\definecolor{currentstroke}{rgb}{0.000000,0.000000,0.000000}%
\pgfsetstrokecolor{currentstroke}%
\pgfsetstrokeopacity{0.000000}%
\pgfsetdash{}{0pt}%
\pgfpathmoveto{\pgfqpoint{3.283579in}{2.021263in}}%
\pgfpathlineto{\pgfqpoint{3.282448in}{2.030639in}}%
\pgfpathlineto{\pgfqpoint{3.304707in}{2.097043in}}%
\pgfpathlineto{\pgfqpoint{3.314997in}{2.096508in}}%
\pgfpathlineto{\pgfqpoint{3.317561in}{2.084836in}}%
\pgfpathlineto{\pgfqpoint{3.326634in}{2.081287in}}%
\pgfpathlineto{\pgfqpoint{3.340140in}{2.093875in}}%
\pgfpathlineto{\pgfqpoint{3.349718in}{2.096714in}}%
\pgfpathlineto{\pgfqpoint{3.349106in}{2.101752in}}%
\pgfpathlineto{\pgfqpoint{3.355866in}{2.103852in}}%
\pgfpathlineto{\pgfqpoint{3.372793in}{2.096779in}}%
\pgfpathlineto{\pgfqpoint{3.378361in}{2.091133in}}%
\pgfpathlineto{\pgfqpoint{3.383833in}{2.089718in}}%
\pgfpathlineto{\pgfqpoint{3.394330in}{2.056064in}}%
\pgfpathlineto{\pgfqpoint{3.409349in}{2.008720in}}%
\pgfpathlineto{\pgfqpoint{3.409543in}{2.004068in}}%
\pgfpathlineto{\pgfqpoint{3.413931in}{1.989428in}}%
\pgfpathlineto{\pgfqpoint{3.402633in}{1.982740in}}%
\pgfpathlineto{\pgfqpoint{3.385814in}{1.972367in}}%
\pgfpathlineto{\pgfqpoint{3.384510in}{1.972619in}}%
\pgfpathlineto{\pgfqpoint{3.376860in}{1.998945in}}%
\pgfpathlineto{\pgfqpoint{3.366904in}{2.030425in}}%
\pgfpathlineto{\pgfqpoint{3.347484in}{2.025715in}}%
\pgfpathlineto{\pgfqpoint{3.343558in}{2.038485in}}%
\pgfpathlineto{\pgfqpoint{3.298672in}{2.025538in}}%
\pgfpathclose%
\pgfusepath{fill}%
\end{pgfscope}%
\begin{pgfscope}%
\pgfpathrectangle{\pgfqpoint{0.100000in}{0.100000in}}{\pgfqpoint{3.420221in}{2.189500in}}%
\pgfusepath{clip}%
\pgfsetbuttcap%
\pgfsetmiterjoin%
\definecolor{currentfill}{rgb}{0.000000,0.286275,0.856863}%
\pgfsetfillcolor{currentfill}%
\pgfsetlinewidth{0.000000pt}%
\definecolor{currentstroke}{rgb}{0.000000,0.000000,0.000000}%
\pgfsetstrokecolor{currentstroke}%
\pgfsetstrokeopacity{0.000000}%
\pgfsetdash{}{0pt}%
\pgfpathmoveto{\pgfqpoint{2.577821in}{1.456363in}}%
\pgfpathlineto{\pgfqpoint{2.571733in}{1.509011in}}%
\pgfpathlineto{\pgfqpoint{2.594271in}{1.512331in}}%
\pgfpathlineto{\pgfqpoint{2.596987in}{1.498174in}}%
\pgfpathlineto{\pgfqpoint{2.599908in}{1.491989in}}%
\pgfpathlineto{\pgfqpoint{2.606245in}{1.492759in}}%
\pgfpathlineto{\pgfqpoint{2.608577in}{1.473259in}}%
\pgfpathlineto{\pgfqpoint{2.602241in}{1.472478in}}%
\pgfpathlineto{\pgfqpoint{2.603746in}{1.459466in}}%
\pgfpathclose%
\pgfusepath{fill}%
\end{pgfscope}%
\begin{pgfscope}%
\pgfpathrectangle{\pgfqpoint{0.100000in}{0.100000in}}{\pgfqpoint{3.420221in}{2.189500in}}%
\pgfusepath{clip}%
\pgfsetbuttcap%
\pgfsetmiterjoin%
\definecolor{currentfill}{rgb}{0.000000,0.643137,0.678431}%
\pgfsetfillcolor{currentfill}%
\pgfsetlinewidth{0.000000pt}%
\definecolor{currentstroke}{rgb}{0.000000,0.000000,0.000000}%
\pgfsetstrokecolor{currentstroke}%
\pgfsetstrokeopacity{0.000000}%
\pgfsetdash{}{0pt}%
\pgfpathmoveto{\pgfqpoint{1.240643in}{1.119582in}}%
\pgfpathlineto{\pgfqpoint{1.238228in}{1.103295in}}%
\pgfpathlineto{\pgfqpoint{1.153026in}{1.116042in}}%
\pgfpathlineto{\pgfqpoint{1.164577in}{1.189894in}}%
\pgfpathlineto{\pgfqpoint{1.204003in}{1.183932in}}%
\pgfpathlineto{\pgfqpoint{1.207266in}{1.187644in}}%
\pgfpathlineto{\pgfqpoint{1.211705in}{1.199766in}}%
\pgfpathlineto{\pgfqpoint{1.218619in}{1.208725in}}%
\pgfpathlineto{\pgfqpoint{1.224291in}{1.209918in}}%
\pgfpathlineto{\pgfqpoint{1.229489in}{1.215460in}}%
\pgfpathlineto{\pgfqpoint{1.231564in}{1.224248in}}%
\pgfpathlineto{\pgfqpoint{1.241096in}{1.232984in}}%
\pgfpathlineto{\pgfqpoint{1.242917in}{1.237047in}}%
\pgfpathlineto{\pgfqpoint{1.251464in}{1.245668in}}%
\pgfpathlineto{\pgfqpoint{1.261999in}{1.248700in}}%
\pgfpathlineto{\pgfqpoint{1.263905in}{1.246838in}}%
\pgfpathlineto{\pgfqpoint{1.265048in}{1.233381in}}%
\pgfpathlineto{\pgfqpoint{1.261509in}{1.207736in}}%
\pgfpathlineto{\pgfqpoint{1.282357in}{1.204916in}}%
\pgfpathlineto{\pgfqpoint{1.282049in}{1.202648in}}%
\pgfpathlineto{\pgfqpoint{1.308676in}{1.199455in}}%
\pgfpathlineto{\pgfqpoint{1.307064in}{1.186966in}}%
\pgfpathlineto{\pgfqpoint{1.316812in}{1.167972in}}%
\pgfpathlineto{\pgfqpoint{1.260802in}{1.175852in}}%
\pgfpathlineto{\pgfqpoint{1.258466in}{1.169768in}}%
\pgfpathlineto{\pgfqpoint{1.247068in}{1.162977in}}%
\pgfpathclose%
\pgfusepath{fill}%
\end{pgfscope}%
\begin{pgfscope}%
\pgfpathrectangle{\pgfqpoint{0.100000in}{0.100000in}}{\pgfqpoint{3.420221in}{2.189500in}}%
\pgfusepath{clip}%
\pgfsetbuttcap%
\pgfsetmiterjoin%
\definecolor{currentfill}{rgb}{0.000000,0.792157,0.603922}%
\pgfsetfillcolor{currentfill}%
\pgfsetlinewidth{0.000000pt}%
\definecolor{currentstroke}{rgb}{0.000000,0.000000,0.000000}%
\pgfsetstrokecolor{currentstroke}%
\pgfsetstrokeopacity{0.000000}%
\pgfsetdash{}{0pt}%
\pgfpathmoveto{\pgfqpoint{2.299087in}{1.948058in}}%
\pgfpathlineto{\pgfqpoint{2.293781in}{1.951012in}}%
\pgfpathlineto{\pgfqpoint{2.296621in}{1.955699in}}%
\pgfpathlineto{\pgfqpoint{2.316899in}{1.967531in}}%
\pgfpathlineto{\pgfqpoint{2.324124in}{1.974451in}}%
\pgfpathlineto{\pgfqpoint{2.332840in}{1.978021in}}%
\pgfpathlineto{\pgfqpoint{2.327724in}{1.967764in}}%
\pgfpathlineto{\pgfqpoint{2.322133in}{1.963122in}}%
\pgfpathlineto{\pgfqpoint{2.309276in}{1.956698in}}%
\pgfpathlineto{\pgfqpoint{2.311329in}{1.953802in}}%
\pgfpathclose%
\pgfusepath{fill}%
\end{pgfscope}%
\begin{pgfscope}%
\pgfpathrectangle{\pgfqpoint{0.100000in}{0.100000in}}{\pgfqpoint{3.420221in}{2.189500in}}%
\pgfusepath{clip}%
\pgfsetbuttcap%
\pgfsetmiterjoin%
\definecolor{currentfill}{rgb}{0.000000,0.792157,0.603922}%
\pgfsetfillcolor{currentfill}%
\pgfsetlinewidth{0.000000pt}%
\definecolor{currentstroke}{rgb}{0.000000,0.000000,0.000000}%
\pgfsetstrokecolor{currentstroke}%
\pgfsetstrokeopacity{0.000000}%
\pgfsetdash{}{0pt}%
\pgfpathmoveto{\pgfqpoint{2.361209in}{1.883366in}}%
\pgfpathlineto{\pgfqpoint{2.362030in}{1.872303in}}%
\pgfpathlineto{\pgfqpoint{2.365282in}{1.866024in}}%
\pgfpathlineto{\pgfqpoint{2.358860in}{1.865458in}}%
\pgfpathlineto{\pgfqpoint{2.360277in}{1.845958in}}%
\pgfpathlineto{\pgfqpoint{2.314865in}{1.842736in}}%
\pgfpathlineto{\pgfqpoint{2.313651in}{1.862384in}}%
\pgfpathlineto{\pgfqpoint{2.320114in}{1.862758in}}%
\pgfpathlineto{\pgfqpoint{2.316175in}{1.868973in}}%
\pgfpathlineto{\pgfqpoint{2.314771in}{1.889216in}}%
\pgfpathlineto{\pgfqpoint{2.316698in}{1.894508in}}%
\pgfpathlineto{\pgfqpoint{2.328011in}{1.904699in}}%
\pgfpathlineto{\pgfqpoint{2.331891in}{1.906197in}}%
\pgfpathlineto{\pgfqpoint{2.339277in}{1.916640in}}%
\pgfpathlineto{\pgfqpoint{2.349171in}{1.923324in}}%
\pgfpathlineto{\pgfqpoint{2.362768in}{1.926500in}}%
\pgfpathlineto{\pgfqpoint{2.370257in}{1.926477in}}%
\pgfpathlineto{\pgfqpoint{2.375366in}{1.921384in}}%
\pgfpathlineto{\pgfqpoint{2.365325in}{1.920206in}}%
\pgfpathlineto{\pgfqpoint{2.363978in}{1.915595in}}%
\pgfpathlineto{\pgfqpoint{2.358282in}{1.912254in}}%
\pgfpathlineto{\pgfqpoint{2.350112in}{1.904360in}}%
\pgfpathlineto{\pgfqpoint{2.350087in}{1.900101in}}%
\pgfpathlineto{\pgfqpoint{2.340352in}{1.886102in}}%
\pgfpathlineto{\pgfqpoint{2.339087in}{1.876134in}}%
\pgfpathlineto{\pgfqpoint{2.341879in}{1.873111in}}%
\pgfpathlineto{\pgfqpoint{2.350766in}{1.883887in}}%
\pgfpathlineto{\pgfqpoint{2.355128in}{1.886902in}}%
\pgfpathclose%
\pgfusepath{fill}%
\end{pgfscope}%
\begin{pgfscope}%
\pgfpathrectangle{\pgfqpoint{0.100000in}{0.100000in}}{\pgfqpoint{3.420221in}{2.189500in}}%
\pgfusepath{clip}%
\pgfsetbuttcap%
\pgfsetmiterjoin%
\definecolor{currentfill}{rgb}{0.000000,0.325490,0.837255}%
\pgfsetfillcolor{currentfill}%
\pgfsetlinewidth{0.000000pt}%
\definecolor{currentstroke}{rgb}{0.000000,0.000000,0.000000}%
\pgfsetstrokecolor{currentstroke}%
\pgfsetstrokeopacity{0.000000}%
\pgfsetdash{}{0pt}%
\pgfpathmoveto{\pgfqpoint{1.409235in}{1.226541in}}%
\pgfpathlineto{\pgfqpoint{1.403125in}{1.235025in}}%
\pgfpathlineto{\pgfqpoint{1.398552in}{1.231310in}}%
\pgfpathlineto{\pgfqpoint{1.387657in}{1.227100in}}%
\pgfpathlineto{\pgfqpoint{1.384205in}{1.227909in}}%
\pgfpathlineto{\pgfqpoint{1.378322in}{1.233966in}}%
\pgfpathlineto{\pgfqpoint{1.374472in}{1.247898in}}%
\pgfpathlineto{\pgfqpoint{1.357626in}{1.272609in}}%
\pgfpathlineto{\pgfqpoint{1.365991in}{1.280285in}}%
\pgfpathlineto{\pgfqpoint{1.366454in}{1.284676in}}%
\pgfpathlineto{\pgfqpoint{1.362244in}{1.290688in}}%
\pgfpathlineto{\pgfqpoint{1.399475in}{1.286581in}}%
\pgfpathlineto{\pgfqpoint{1.403074in}{1.318781in}}%
\pgfpathlineto{\pgfqpoint{1.476962in}{1.310767in}}%
\pgfpathlineto{\pgfqpoint{1.474804in}{1.291448in}}%
\pgfpathlineto{\pgfqpoint{1.472309in}{1.265613in}}%
\pgfpathlineto{\pgfqpoint{1.420544in}{1.270830in}}%
\pgfpathlineto{\pgfqpoint{1.418475in}{1.251362in}}%
\pgfpathlineto{\pgfqpoint{1.412107in}{1.252043in}}%
\pgfpathclose%
\pgfusepath{fill}%
\end{pgfscope}%
\begin{pgfscope}%
\pgfpathrectangle{\pgfqpoint{0.100000in}{0.100000in}}{\pgfqpoint{3.420221in}{2.189500in}}%
\pgfusepath{clip}%
\pgfsetbuttcap%
\pgfsetmiterjoin%
\definecolor{currentfill}{rgb}{0.000000,0.474510,0.762745}%
\pgfsetfillcolor{currentfill}%
\pgfsetlinewidth{0.000000pt}%
\definecolor{currentstroke}{rgb}{0.000000,0.000000,0.000000}%
\pgfsetstrokecolor{currentstroke}%
\pgfsetstrokeopacity{0.000000}%
\pgfsetdash{}{0pt}%
\pgfpathmoveto{\pgfqpoint{2.827828in}{1.427788in}}%
\pgfpathlineto{\pgfqpoint{2.807256in}{1.424511in}}%
\pgfpathlineto{\pgfqpoint{2.805182in}{1.444167in}}%
\pgfpathlineto{\pgfqpoint{2.800878in}{1.443628in}}%
\pgfpathlineto{\pgfqpoint{2.800516in}{1.453554in}}%
\pgfpathlineto{\pgfqpoint{2.803330in}{1.457263in}}%
\pgfpathlineto{\pgfqpoint{2.814435in}{1.457734in}}%
\pgfpathlineto{\pgfqpoint{2.822137in}{1.463239in}}%
\pgfpathclose%
\pgfusepath{fill}%
\end{pgfscope}%
\begin{pgfscope}%
\pgfpathrectangle{\pgfqpoint{0.100000in}{0.100000in}}{\pgfqpoint{3.420221in}{2.189500in}}%
\pgfusepath{clip}%
\pgfsetbuttcap%
\pgfsetmiterjoin%
\definecolor{currentfill}{rgb}{0.000000,0.670588,0.664706}%
\pgfsetfillcolor{currentfill}%
\pgfsetlinewidth{0.000000pt}%
\definecolor{currentstroke}{rgb}{0.000000,0.000000,0.000000}%
\pgfsetstrokecolor{currentstroke}%
\pgfsetstrokeopacity{0.000000}%
\pgfsetdash{}{0pt}%
\pgfpathmoveto{\pgfqpoint{2.444652in}{1.055763in}}%
\pgfpathlineto{\pgfqpoint{2.444061in}{1.042556in}}%
\pgfpathlineto{\pgfqpoint{2.450300in}{1.037559in}}%
\pgfpathlineto{\pgfqpoint{2.446738in}{1.029172in}}%
\pgfpathlineto{\pgfqpoint{2.431260in}{1.026207in}}%
\pgfpathlineto{\pgfqpoint{2.428545in}{1.020886in}}%
\pgfpathlineto{\pgfqpoint{2.422972in}{1.019911in}}%
\pgfpathlineto{\pgfqpoint{2.407898in}{1.021344in}}%
\pgfpathlineto{\pgfqpoint{2.404266in}{1.025849in}}%
\pgfpathlineto{\pgfqpoint{2.398573in}{1.027673in}}%
\pgfpathlineto{\pgfqpoint{2.391575in}{1.032908in}}%
\pgfpathlineto{\pgfqpoint{2.390924in}{1.047996in}}%
\pgfpathlineto{\pgfqpoint{2.393003in}{1.050812in}}%
\pgfpathlineto{\pgfqpoint{2.416733in}{1.051939in}}%
\pgfpathlineto{\pgfqpoint{2.424272in}{1.055009in}}%
\pgfpathlineto{\pgfqpoint{2.425926in}{1.052204in}}%
\pgfpathlineto{\pgfqpoint{2.432918in}{1.054802in}}%
\pgfpathclose%
\pgfusepath{fill}%
\end{pgfscope}%
\begin{pgfscope}%
\pgfpathrectangle{\pgfqpoint{0.100000in}{0.100000in}}{\pgfqpoint{3.420221in}{2.189500in}}%
\pgfusepath{clip}%
\pgfsetbuttcap%
\pgfsetmiterjoin%
\definecolor{currentfill}{rgb}{0.000000,0.458824,0.770588}%
\pgfsetfillcolor{currentfill}%
\pgfsetlinewidth{0.000000pt}%
\definecolor{currentstroke}{rgb}{0.000000,0.000000,0.000000}%
\pgfsetstrokecolor{currentstroke}%
\pgfsetstrokeopacity{0.000000}%
\pgfsetdash{}{0pt}%
\pgfpathmoveto{\pgfqpoint{1.444666in}{0.776020in}}%
\pgfpathlineto{\pgfqpoint{1.425448in}{0.777887in}}%
\pgfpathlineto{\pgfqpoint{1.373045in}{0.783387in}}%
\pgfpathlineto{\pgfqpoint{1.377051in}{0.822409in}}%
\pgfpathlineto{\pgfqpoint{1.345295in}{0.825822in}}%
\pgfpathlineto{\pgfqpoint{1.349178in}{0.858999in}}%
\pgfpathlineto{\pgfqpoint{1.351464in}{0.858740in}}%
\pgfpathlineto{\pgfqpoint{1.352870in}{0.871484in}}%
\pgfpathlineto{\pgfqpoint{1.378690in}{0.869084in}}%
\pgfpathlineto{\pgfqpoint{1.381648in}{0.888384in}}%
\pgfpathlineto{\pgfqpoint{1.387246in}{0.940041in}}%
\pgfpathlineto{\pgfqpoint{1.393860in}{0.939282in}}%
\pgfpathlineto{\pgfqpoint{1.393168in}{0.932722in}}%
\pgfpathlineto{\pgfqpoint{1.432180in}{0.928290in}}%
\pgfpathlineto{\pgfqpoint{1.432871in}{0.934865in}}%
\pgfpathlineto{\pgfqpoint{1.445874in}{0.933525in}}%
\pgfpathlineto{\pgfqpoint{1.452332in}{0.932808in}}%
\pgfpathlineto{\pgfqpoint{1.450397in}{0.913165in}}%
\pgfpathlineto{\pgfqpoint{1.458181in}{0.912414in}}%
\pgfpathlineto{\pgfqpoint{1.456913in}{0.900203in}}%
\pgfpathlineto{\pgfqpoint{1.469923in}{0.899045in}}%
\pgfpathlineto{\pgfqpoint{1.469277in}{0.892536in}}%
\pgfpathlineto{\pgfqpoint{1.456135in}{0.893742in}}%
\pgfpathlineto{\pgfqpoint{1.454842in}{0.880726in}}%
\pgfpathlineto{\pgfqpoint{1.452020in}{0.880967in}}%
\pgfpathlineto{\pgfqpoint{1.448906in}{0.848715in}}%
\pgfpathlineto{\pgfqpoint{1.445870in}{0.849008in}}%
\pgfpathlineto{\pgfqpoint{1.442678in}{0.815793in}}%
\pgfpathlineto{\pgfqpoint{1.448438in}{0.815245in}}%
\pgfpathclose%
\pgfusepath{fill}%
\end{pgfscope}%
\begin{pgfscope}%
\pgfpathrectangle{\pgfqpoint{0.100000in}{0.100000in}}{\pgfqpoint{3.420221in}{2.189500in}}%
\pgfusepath{clip}%
\pgfsetbuttcap%
\pgfsetmiterjoin%
\definecolor{currentfill}{rgb}{0.000000,0.490196,0.754902}%
\pgfsetfillcolor{currentfill}%
\pgfsetlinewidth{0.000000pt}%
\definecolor{currentstroke}{rgb}{0.000000,0.000000,0.000000}%
\pgfsetstrokecolor{currentstroke}%
\pgfsetstrokeopacity{0.000000}%
\pgfsetdash{}{0pt}%
\pgfpathmoveto{\pgfqpoint{1.343500in}{2.076042in}}%
\pgfpathlineto{\pgfqpoint{1.339524in}{2.056832in}}%
\pgfpathlineto{\pgfqpoint{1.336558in}{2.057276in}}%
\pgfpathlineto{\pgfqpoint{1.334643in}{2.044198in}}%
\pgfpathlineto{\pgfqpoint{1.331671in}{2.033412in}}%
\pgfpathlineto{\pgfqpoint{1.328665in}{2.035591in}}%
\pgfpathlineto{\pgfqpoint{1.325017in}{2.018824in}}%
\pgfpathlineto{\pgfqpoint{1.322296in}{2.000591in}}%
\pgfpathlineto{\pgfqpoint{1.313718in}{2.002992in}}%
\pgfpathlineto{\pgfqpoint{1.312034in}{1.997856in}}%
\pgfpathlineto{\pgfqpoint{1.298867in}{2.000012in}}%
\pgfpathlineto{\pgfqpoint{1.296549in}{1.985918in}}%
\pgfpathlineto{\pgfqpoint{1.292940in}{1.991712in}}%
\pgfpathlineto{\pgfqpoint{1.284984in}{1.991326in}}%
\pgfpathlineto{\pgfqpoint{1.277851in}{1.993331in}}%
\pgfpathlineto{\pgfqpoint{1.267822in}{1.989603in}}%
\pgfpathlineto{\pgfqpoint{1.263993in}{1.991396in}}%
\pgfpathlineto{\pgfqpoint{1.268989in}{2.021035in}}%
\pgfpathlineto{\pgfqpoint{1.259321in}{2.022415in}}%
\pgfpathlineto{\pgfqpoint{1.260414in}{2.029067in}}%
\pgfpathlineto{\pgfqpoint{1.254757in}{2.030011in}}%
\pgfpathlineto{\pgfqpoint{1.255528in}{2.036375in}}%
\pgfpathlineto{\pgfqpoint{1.216708in}{2.042984in}}%
\pgfpathlineto{\pgfqpoint{1.215569in}{2.036488in}}%
\pgfpathlineto{\pgfqpoint{1.205528in}{2.038247in}}%
\pgfpathlineto{\pgfqpoint{1.204389in}{2.031780in}}%
\pgfpathlineto{\pgfqpoint{1.175357in}{2.036904in}}%
\pgfpathlineto{\pgfqpoint{1.176553in}{2.043426in}}%
\pgfpathlineto{\pgfqpoint{1.183231in}{2.042209in}}%
\pgfpathlineto{\pgfqpoint{1.193962in}{2.100364in}}%
\pgfpathlineto{\pgfqpoint{1.245773in}{2.091432in}}%
\pgfpathlineto{\pgfqpoint{1.305962in}{2.081671in}}%
\pgfpathclose%
\pgfusepath{fill}%
\end{pgfscope}%
\begin{pgfscope}%
\pgfpathrectangle{\pgfqpoint{0.100000in}{0.100000in}}{\pgfqpoint{3.420221in}{2.189500in}}%
\pgfusepath{clip}%
\pgfsetbuttcap%
\pgfsetmiterjoin%
\definecolor{currentfill}{rgb}{0.000000,0.556863,0.721569}%
\pgfsetfillcolor{currentfill}%
\pgfsetlinewidth{0.000000pt}%
\definecolor{currentstroke}{rgb}{0.000000,0.000000,0.000000}%
\pgfsetstrokecolor{currentstroke}%
\pgfsetstrokeopacity{0.000000}%
\pgfsetdash{}{0pt}%
\pgfpathmoveto{\pgfqpoint{3.120643in}{1.113708in}}%
\pgfpathlineto{\pgfqpoint{3.120844in}{1.106154in}}%
\pgfpathlineto{\pgfqpoint{3.108335in}{1.102137in}}%
\pgfpathlineto{\pgfqpoint{3.100224in}{1.107012in}}%
\pgfpathlineto{\pgfqpoint{3.084622in}{1.112919in}}%
\pgfpathlineto{\pgfqpoint{3.084017in}{1.120469in}}%
\pgfpathlineto{\pgfqpoint{3.086444in}{1.123595in}}%
\pgfpathlineto{\pgfqpoint{3.082447in}{1.129067in}}%
\pgfpathlineto{\pgfqpoint{3.080520in}{1.136733in}}%
\pgfpathlineto{\pgfqpoint{3.069212in}{1.140495in}}%
\pgfpathlineto{\pgfqpoint{3.068450in}{1.146933in}}%
\pgfpathlineto{\pgfqpoint{3.063421in}{1.153559in}}%
\pgfpathlineto{\pgfqpoint{3.067515in}{1.158318in}}%
\pgfpathlineto{\pgfqpoint{3.075057in}{1.156021in}}%
\pgfpathlineto{\pgfqpoint{3.076193in}{1.151685in}}%
\pgfpathlineto{\pgfqpoint{3.083033in}{1.151299in}}%
\pgfpathlineto{\pgfqpoint{3.089264in}{1.147479in}}%
\pgfpathlineto{\pgfqpoint{3.093602in}{1.151382in}}%
\pgfpathlineto{\pgfqpoint{3.097203in}{1.145499in}}%
\pgfpathlineto{\pgfqpoint{3.100727in}{1.151736in}}%
\pgfpathlineto{\pgfqpoint{3.104318in}{1.151334in}}%
\pgfpathlineto{\pgfqpoint{3.106548in}{1.157625in}}%
\pgfpathlineto{\pgfqpoint{3.116177in}{1.159433in}}%
\pgfpathlineto{\pgfqpoint{3.123242in}{1.163898in}}%
\pgfpathlineto{\pgfqpoint{3.127067in}{1.161796in}}%
\pgfpathlineto{\pgfqpoint{3.134142in}{1.166601in}}%
\pgfpathlineto{\pgfqpoint{3.143289in}{1.168957in}}%
\pgfpathlineto{\pgfqpoint{3.150613in}{1.163290in}}%
\pgfpathlineto{\pgfqpoint{3.153514in}{1.169721in}}%
\pgfpathlineto{\pgfqpoint{3.159134in}{1.169523in}}%
\pgfpathlineto{\pgfqpoint{3.163596in}{1.164990in}}%
\pgfpathlineto{\pgfqpoint{3.168692in}{1.151869in}}%
\pgfpathlineto{\pgfqpoint{3.167311in}{1.142415in}}%
\pgfpathlineto{\pgfqpoint{3.160531in}{1.140532in}}%
\pgfpathlineto{\pgfqpoint{3.155574in}{1.131292in}}%
\pgfpathlineto{\pgfqpoint{3.154897in}{1.126626in}}%
\pgfpathlineto{\pgfqpoint{3.148218in}{1.118755in}}%
\pgfpathlineto{\pgfqpoint{3.140009in}{1.118541in}}%
\pgfpathlineto{\pgfqpoint{3.130508in}{1.121191in}}%
\pgfpathlineto{\pgfqpoint{3.128372in}{1.117952in}}%
\pgfpathlineto{\pgfqpoint{3.119262in}{1.118335in}}%
\pgfpathclose%
\pgfusepath{fill}%
\end{pgfscope}%
\begin{pgfscope}%
\pgfpathrectangle{\pgfqpoint{0.100000in}{0.100000in}}{\pgfqpoint{3.420221in}{2.189500in}}%
\pgfusepath{clip}%
\pgfsetbuttcap%
\pgfsetmiterjoin%
\definecolor{currentfill}{rgb}{0.000000,0.556863,0.721569}%
\pgfsetfillcolor{currentfill}%
\pgfsetlinewidth{0.000000pt}%
\definecolor{currentstroke}{rgb}{0.000000,0.000000,0.000000}%
\pgfsetstrokecolor{currentstroke}%
\pgfsetstrokeopacity{0.000000}%
\pgfsetdash{}{0pt}%
\pgfpathmoveto{\pgfqpoint{3.156706in}{1.190091in}}%
\pgfpathlineto{\pgfqpoint{3.169205in}{1.172941in}}%
\pgfpathlineto{\pgfqpoint{3.163041in}{1.173825in}}%
\pgfpathlineto{\pgfqpoint{3.155923in}{1.189701in}}%
\pgfpathclose%
\pgfusepath{fill}%
\end{pgfscope}%
\begin{pgfscope}%
\pgfpathrectangle{\pgfqpoint{0.100000in}{0.100000in}}{\pgfqpoint{3.420221in}{2.189500in}}%
\pgfusepath{clip}%
\pgfsetbuttcap%
\pgfsetmiterjoin%
\definecolor{currentfill}{rgb}{0.000000,0.517647,0.741176}%
\pgfsetfillcolor{currentfill}%
\pgfsetlinewidth{0.000000pt}%
\definecolor{currentstroke}{rgb}{0.000000,0.000000,0.000000}%
\pgfsetstrokecolor{currentstroke}%
\pgfsetstrokeopacity{0.000000}%
\pgfsetdash{}{0pt}%
\pgfpathmoveto{\pgfqpoint{1.828448in}{0.573892in}}%
\pgfpathlineto{\pgfqpoint{1.830739in}{0.572366in}}%
\pgfpathlineto{\pgfqpoint{1.849494in}{0.583546in}}%
\pgfpathlineto{\pgfqpoint{1.860710in}{0.571385in}}%
\pgfpathlineto{\pgfqpoint{1.856610in}{0.567877in}}%
\pgfpathlineto{\pgfqpoint{1.853866in}{0.553047in}}%
\pgfpathlineto{\pgfqpoint{1.828587in}{0.532838in}}%
\pgfpathlineto{\pgfqpoint{1.821187in}{0.541571in}}%
\pgfpathlineto{\pgfqpoint{1.812860in}{0.551796in}}%
\pgfpathclose%
\pgfusepath{fill}%
\end{pgfscope}%
\begin{pgfscope}%
\pgfpathrectangle{\pgfqpoint{0.100000in}{0.100000in}}{\pgfqpoint{3.420221in}{2.189500in}}%
\pgfusepath{clip}%
\pgfsetbuttcap%
\pgfsetmiterjoin%
\definecolor{currentfill}{rgb}{0.000000,0.427451,0.786275}%
\pgfsetfillcolor{currentfill}%
\pgfsetlinewidth{0.000000pt}%
\definecolor{currentstroke}{rgb}{0.000000,0.000000,0.000000}%
\pgfsetstrokecolor{currentstroke}%
\pgfsetstrokeopacity{0.000000}%
\pgfsetdash{}{0pt}%
\pgfpathmoveto{\pgfqpoint{1.795333in}{1.616714in}}%
\pgfpathlineto{\pgfqpoint{1.860027in}{1.614455in}}%
\pgfpathlineto{\pgfqpoint{1.871228in}{1.614140in}}%
\pgfpathlineto{\pgfqpoint{1.870695in}{1.589338in}}%
\pgfpathlineto{\pgfqpoint{1.833414in}{1.590379in}}%
\pgfpathlineto{\pgfqpoint{1.831514in}{1.583314in}}%
\pgfpathlineto{\pgfqpoint{1.832222in}{1.575813in}}%
\pgfpathlineto{\pgfqpoint{1.830527in}{1.568438in}}%
\pgfpathlineto{\pgfqpoint{1.827563in}{1.565694in}}%
\pgfpathlineto{\pgfqpoint{1.820061in}{1.569379in}}%
\pgfpathlineto{\pgfqpoint{1.811580in}{1.574577in}}%
\pgfpathlineto{\pgfqpoint{1.810014in}{1.578470in}}%
\pgfpathlineto{\pgfqpoint{1.808568in}{1.581734in}}%
\pgfpathlineto{\pgfqpoint{1.799721in}{1.584739in}}%
\pgfpathlineto{\pgfqpoint{1.795390in}{1.590131in}}%
\pgfpathlineto{\pgfqpoint{1.791030in}{1.590967in}}%
\pgfpathlineto{\pgfqpoint{1.788647in}{1.598062in}}%
\pgfpathlineto{\pgfqpoint{1.779779in}{1.603945in}}%
\pgfpathlineto{\pgfqpoint{1.774955in}{1.612694in}}%
\pgfpathlineto{\pgfqpoint{1.770244in}{1.613093in}}%
\pgfpathlineto{\pgfqpoint{1.767928in}{1.617953in}}%
\pgfpathclose%
\pgfusepath{fill}%
\end{pgfscope}%
\begin{pgfscope}%
\pgfpathrectangle{\pgfqpoint{0.100000in}{0.100000in}}{\pgfqpoint{3.420221in}{2.189500in}}%
\pgfusepath{clip}%
\pgfsetbuttcap%
\pgfsetmiterjoin%
\definecolor{currentfill}{rgb}{0.000000,0.243137,0.878431}%
\pgfsetfillcolor{currentfill}%
\pgfsetlinewidth{0.000000pt}%
\definecolor{currentstroke}{rgb}{0.000000,0.000000,0.000000}%
\pgfsetstrokecolor{currentstroke}%
\pgfsetstrokeopacity{0.000000}%
\pgfsetdash{}{0pt}%
\pgfpathmoveto{\pgfqpoint{2.553189in}{1.274098in}}%
\pgfpathlineto{\pgfqpoint{2.540179in}{1.272652in}}%
\pgfpathlineto{\pgfqpoint{2.535037in}{1.269916in}}%
\pgfpathlineto{\pgfqpoint{2.533680in}{1.281751in}}%
\pgfpathlineto{\pgfqpoint{2.529249in}{1.283551in}}%
\pgfpathlineto{\pgfqpoint{2.510973in}{1.281729in}}%
\pgfpathlineto{\pgfqpoint{2.509044in}{1.298646in}}%
\pgfpathlineto{\pgfqpoint{2.506445in}{1.302745in}}%
\pgfpathlineto{\pgfqpoint{2.520208in}{1.304192in}}%
\pgfpathlineto{\pgfqpoint{2.517744in}{1.326075in}}%
\pgfpathlineto{\pgfqpoint{2.540808in}{1.328822in}}%
\pgfpathlineto{\pgfqpoint{2.542365in}{1.312473in}}%
\pgfpathlineto{\pgfqpoint{2.549316in}{1.313353in}}%
\pgfpathlineto{\pgfqpoint{2.556119in}{1.318801in}}%
\pgfpathlineto{\pgfqpoint{2.558061in}{1.297713in}}%
\pgfpathlineto{\pgfqpoint{2.549140in}{1.290097in}}%
\pgfpathlineto{\pgfqpoint{2.544925in}{1.289637in}}%
\pgfpathlineto{\pgfqpoint{2.545588in}{1.283144in}}%
\pgfpathlineto{\pgfqpoint{2.552572in}{1.279491in}}%
\pgfpathclose%
\pgfusepath{fill}%
\end{pgfscope}%
\begin{pgfscope}%
\pgfpathrectangle{\pgfqpoint{0.100000in}{0.100000in}}{\pgfqpoint{3.420221in}{2.189500in}}%
\pgfusepath{clip}%
\pgfsetbuttcap%
\pgfsetmiterjoin%
\definecolor{currentfill}{rgb}{0.000000,0.396078,0.801961}%
\pgfsetfillcolor{currentfill}%
\pgfsetlinewidth{0.000000pt}%
\definecolor{currentstroke}{rgb}{0.000000,0.000000,0.000000}%
\pgfsetstrokecolor{currentstroke}%
\pgfsetstrokeopacity{0.000000}%
\pgfsetdash{}{0pt}%
\pgfpathmoveto{\pgfqpoint{1.942192in}{1.125355in}}%
\pgfpathlineto{\pgfqpoint{1.873540in}{1.126566in}}%
\pgfpathlineto{\pgfqpoint{1.874054in}{1.162275in}}%
\pgfpathlineto{\pgfqpoint{1.911373in}{1.161584in}}%
\pgfpathlineto{\pgfqpoint{1.912311in}{1.207284in}}%
\pgfpathlineto{\pgfqpoint{1.922021in}{1.207088in}}%
\pgfpathlineto{\pgfqpoint{1.922167in}{1.213597in}}%
\pgfpathlineto{\pgfqpoint{1.945616in}{1.213179in}}%
\pgfpathlineto{\pgfqpoint{1.945500in}{1.203383in}}%
\pgfpathlineto{\pgfqpoint{1.945166in}{1.180450in}}%
\pgfpathlineto{\pgfqpoint{1.944380in}{1.125349in}}%
\pgfpathclose%
\pgfusepath{fill}%
\end{pgfscope}%
\begin{pgfscope}%
\pgfpathrectangle{\pgfqpoint{0.100000in}{0.100000in}}{\pgfqpoint{3.420221in}{2.189500in}}%
\pgfusepath{clip}%
\pgfsetbuttcap%
\pgfsetmiterjoin%
\definecolor{currentfill}{rgb}{0.000000,0.890196,0.554902}%
\pgfsetfillcolor{currentfill}%
\pgfsetlinewidth{0.000000pt}%
\definecolor{currentstroke}{rgb}{0.000000,0.000000,0.000000}%
\pgfsetstrokecolor{currentstroke}%
\pgfsetstrokeopacity{0.000000}%
\pgfsetdash{}{0pt}%
\pgfpathmoveto{\pgfqpoint{2.723344in}{1.142396in}}%
\pgfpathlineto{\pgfqpoint{2.685277in}{1.137791in}}%
\pgfpathlineto{\pgfqpoint{2.685754in}{1.140807in}}%
\pgfpathlineto{\pgfqpoint{2.694271in}{1.143988in}}%
\pgfpathlineto{\pgfqpoint{2.712885in}{1.152292in}}%
\pgfpathlineto{\pgfqpoint{2.715424in}{1.160774in}}%
\pgfpathlineto{\pgfqpoint{2.726702in}{1.165186in}}%
\pgfpathlineto{\pgfqpoint{2.726405in}{1.171115in}}%
\pgfpathlineto{\pgfqpoint{2.733294in}{1.177717in}}%
\pgfpathlineto{\pgfqpoint{2.733722in}{1.183614in}}%
\pgfpathlineto{\pgfqpoint{2.742229in}{1.190524in}}%
\pgfpathlineto{\pgfqpoint{2.756011in}{1.199988in}}%
\pgfpathlineto{\pgfqpoint{2.761543in}{1.194371in}}%
\pgfpathlineto{\pgfqpoint{2.769107in}{1.183255in}}%
\pgfpathlineto{\pgfqpoint{2.758610in}{1.175757in}}%
\pgfpathlineto{\pgfqpoint{2.761038in}{1.171140in}}%
\pgfpathlineto{\pgfqpoint{2.754925in}{1.168084in}}%
\pgfpathlineto{\pgfqpoint{2.741367in}{1.166373in}}%
\pgfpathlineto{\pgfqpoint{2.735060in}{1.162053in}}%
\pgfpathlineto{\pgfqpoint{2.731610in}{1.154061in}}%
\pgfpathlineto{\pgfqpoint{2.724773in}{1.148797in}}%
\pgfpathclose%
\pgfusepath{fill}%
\end{pgfscope}%
\begin{pgfscope}%
\pgfpathrectangle{\pgfqpoint{0.100000in}{0.100000in}}{\pgfqpoint{3.420221in}{2.189500in}}%
\pgfusepath{clip}%
\pgfsetbuttcap%
\pgfsetmiterjoin%
\definecolor{currentfill}{rgb}{0.000000,0.415686,0.792157}%
\pgfsetfillcolor{currentfill}%
\pgfsetlinewidth{0.000000pt}%
\definecolor{currentstroke}{rgb}{0.000000,0.000000,0.000000}%
\pgfsetstrokecolor{currentstroke}%
\pgfsetstrokeopacity{0.000000}%
\pgfsetdash{}{0pt}%
\pgfpathmoveto{\pgfqpoint{1.600767in}{1.068907in}}%
\pgfpathlineto{\pgfqpoint{1.598430in}{1.036334in}}%
\pgfpathlineto{\pgfqpoint{1.565623in}{1.039378in}}%
\pgfpathlineto{\pgfqpoint{1.512174in}{1.043481in}}%
\pgfpathlineto{\pgfqpoint{1.512957in}{1.052211in}}%
\pgfpathlineto{\pgfqpoint{1.517971in}{1.109091in}}%
\pgfpathlineto{\pgfqpoint{1.520342in}{1.108878in}}%
\pgfpathlineto{\pgfqpoint{1.570798in}{1.104653in}}%
\pgfpathlineto{\pgfqpoint{1.568137in}{1.071376in}}%
\pgfpathclose%
\pgfusepath{fill}%
\end{pgfscope}%
\begin{pgfscope}%
\pgfpathrectangle{\pgfqpoint{0.100000in}{0.100000in}}{\pgfqpoint{3.420221in}{2.189500in}}%
\pgfusepath{clip}%
\pgfsetbuttcap%
\pgfsetmiterjoin%
\definecolor{currentfill}{rgb}{0.000000,0.282353,0.858824}%
\pgfsetfillcolor{currentfill}%
\pgfsetlinewidth{0.000000pt}%
\definecolor{currentstroke}{rgb}{0.000000,0.000000,0.000000}%
\pgfsetstrokecolor{currentstroke}%
\pgfsetstrokeopacity{0.000000}%
\pgfsetdash{}{0pt}%
\pgfpathmoveto{\pgfqpoint{1.812796in}{0.462255in}}%
\pgfpathlineto{\pgfqpoint{1.806850in}{0.457671in}}%
\pgfpathlineto{\pgfqpoint{1.808119in}{0.453257in}}%
\pgfpathlineto{\pgfqpoint{1.746239in}{0.455579in}}%
\pgfpathlineto{\pgfqpoint{1.748286in}{0.500487in}}%
\pgfpathlineto{\pgfqpoint{1.749425in}{0.534165in}}%
\pgfpathlineto{\pgfqpoint{1.709016in}{0.536062in}}%
\pgfpathlineto{\pgfqpoint{1.711129in}{0.576738in}}%
\pgfpathlineto{\pgfqpoint{1.698521in}{0.577338in}}%
\pgfpathlineto{\pgfqpoint{1.699618in}{0.598582in}}%
\pgfpathlineto{\pgfqpoint{1.727528in}{0.596275in}}%
\pgfpathlineto{\pgfqpoint{1.744205in}{0.586924in}}%
\pgfpathlineto{\pgfqpoint{1.745207in}{0.613960in}}%
\pgfpathlineto{\pgfqpoint{1.766971in}{0.613112in}}%
\pgfpathlineto{\pgfqpoint{1.777787in}{0.597413in}}%
\pgfpathlineto{\pgfqpoint{1.785709in}{0.604742in}}%
\pgfpathlineto{\pgfqpoint{1.802749in}{0.589771in}}%
\pgfpathlineto{\pgfqpoint{1.804564in}{0.582416in}}%
\pgfpathlineto{\pgfqpoint{1.813002in}{0.590178in}}%
\pgfpathlineto{\pgfqpoint{1.823930in}{0.575456in}}%
\pgfpathlineto{\pgfqpoint{1.828448in}{0.573892in}}%
\pgfpathlineto{\pgfqpoint{1.812860in}{0.551796in}}%
\pgfpathlineto{\pgfqpoint{1.821187in}{0.541571in}}%
\pgfpathlineto{\pgfqpoint{1.789621in}{0.516780in}}%
\pgfpathlineto{\pgfqpoint{1.801502in}{0.501749in}}%
\pgfpathlineto{\pgfqpoint{1.795792in}{0.499865in}}%
\pgfpathlineto{\pgfqpoint{1.813420in}{0.462742in}}%
\pgfpathclose%
\pgfusepath{fill}%
\end{pgfscope}%
\begin{pgfscope}%
\pgfpathrectangle{\pgfqpoint{0.100000in}{0.100000in}}{\pgfqpoint{3.420221in}{2.189500in}}%
\pgfusepath{clip}%
\pgfsetbuttcap%
\pgfsetmiterjoin%
\definecolor{currentfill}{rgb}{0.000000,0.525490,0.737255}%
\pgfsetfillcolor{currentfill}%
\pgfsetlinewidth{0.000000pt}%
\definecolor{currentstroke}{rgb}{0.000000,0.000000,0.000000}%
\pgfsetstrokecolor{currentstroke}%
\pgfsetstrokeopacity{0.000000}%
\pgfsetdash{}{0pt}%
\pgfpathmoveto{\pgfqpoint{1.740838in}{0.931281in}}%
\pgfpathlineto{\pgfqpoint{1.730730in}{0.940797in}}%
\pgfpathlineto{\pgfqpoint{1.728701in}{0.934562in}}%
\pgfpathlineto{\pgfqpoint{1.722254in}{0.938006in}}%
\pgfpathlineto{\pgfqpoint{1.716862in}{0.935293in}}%
\pgfpathlineto{\pgfqpoint{1.710993in}{0.935868in}}%
\pgfpathlineto{\pgfqpoint{1.697408in}{0.951473in}}%
\pgfpathlineto{\pgfqpoint{1.693006in}{0.950545in}}%
\pgfpathlineto{\pgfqpoint{1.693596in}{0.964501in}}%
\pgfpathlineto{\pgfqpoint{1.694769in}{0.985795in}}%
\pgfpathlineto{\pgfqpoint{1.708119in}{0.985056in}}%
\pgfpathlineto{\pgfqpoint{1.708469in}{0.991555in}}%
\pgfpathlineto{\pgfqpoint{1.771901in}{0.988551in}}%
\pgfpathlineto{\pgfqpoint{1.779694in}{0.986799in}}%
\pgfpathlineto{\pgfqpoint{1.779045in}{0.968604in}}%
\pgfpathlineto{\pgfqpoint{1.766330in}{0.969160in}}%
\pgfpathlineto{\pgfqpoint{1.765510in}{0.949553in}}%
\pgfpathlineto{\pgfqpoint{1.744657in}{0.950005in}}%
\pgfpathlineto{\pgfqpoint{1.740724in}{0.943097in}}%
\pgfpathclose%
\pgfusepath{fill}%
\end{pgfscope}%
\begin{pgfscope}%
\pgfpathrectangle{\pgfqpoint{0.100000in}{0.100000in}}{\pgfqpoint{3.420221in}{2.189500in}}%
\pgfusepath{clip}%
\pgfsetbuttcap%
\pgfsetmiterjoin%
\definecolor{currentfill}{rgb}{0.000000,0.360784,0.819608}%
\pgfsetfillcolor{currentfill}%
\pgfsetlinewidth{0.000000pt}%
\definecolor{currentstroke}{rgb}{0.000000,0.000000,0.000000}%
\pgfsetstrokecolor{currentstroke}%
\pgfsetstrokeopacity{0.000000}%
\pgfsetdash{}{0pt}%
\pgfpathmoveto{\pgfqpoint{1.923073in}{1.256037in}}%
\pgfpathlineto{\pgfqpoint{1.922761in}{1.239736in}}%
\pgfpathlineto{\pgfqpoint{1.895457in}{1.240273in}}%
\pgfpathlineto{\pgfqpoint{1.888993in}{1.240408in}}%
\pgfpathlineto{\pgfqpoint{1.889116in}{1.246916in}}%
\pgfpathlineto{\pgfqpoint{1.863264in}{1.247528in}}%
\pgfpathlineto{\pgfqpoint{1.864761in}{1.299711in}}%
\pgfpathlineto{\pgfqpoint{1.865335in}{1.319239in}}%
\pgfpathlineto{\pgfqpoint{1.930558in}{1.317893in}}%
\pgfpathlineto{\pgfqpoint{1.942281in}{1.317735in}}%
\pgfpathlineto{\pgfqpoint{1.942005in}{1.291546in}}%
\pgfpathlineto{\pgfqpoint{1.933785in}{1.289556in}}%
\pgfpathlineto{\pgfqpoint{1.929888in}{1.291060in}}%
\pgfpathlineto{\pgfqpoint{1.921450in}{1.288539in}}%
\pgfpathlineto{\pgfqpoint{1.921193in}{1.278845in}}%
\pgfpathlineto{\pgfqpoint{1.914770in}{1.278975in}}%
\pgfpathlineto{\pgfqpoint{1.914474in}{1.262689in}}%
\pgfpathlineto{\pgfqpoint{1.920956in}{1.262554in}}%
\pgfpathclose%
\pgfusepath{fill}%
\end{pgfscope}%
\begin{pgfscope}%
\pgfpathrectangle{\pgfqpoint{0.100000in}{0.100000in}}{\pgfqpoint{3.420221in}{2.189500in}}%
\pgfusepath{clip}%
\pgfsetbuttcap%
\pgfsetmiterjoin%
\definecolor{currentfill}{rgb}{0.000000,0.392157,0.803922}%
\pgfsetfillcolor{currentfill}%
\pgfsetlinewidth{0.000000pt}%
\definecolor{currentstroke}{rgb}{0.000000,0.000000,0.000000}%
\pgfsetstrokecolor{currentstroke}%
\pgfsetstrokeopacity{0.000000}%
\pgfsetdash{}{0pt}%
\pgfpathmoveto{\pgfqpoint{2.835831in}{1.050089in}}%
\pgfpathlineto{\pgfqpoint{2.798997in}{1.046347in}}%
\pgfpathlineto{\pgfqpoint{2.798272in}{1.051567in}}%
\pgfpathlineto{\pgfqpoint{2.785092in}{1.061179in}}%
\pgfpathlineto{\pgfqpoint{2.779092in}{1.059106in}}%
\pgfpathlineto{\pgfqpoint{2.777423in}{1.062602in}}%
\pgfpathlineto{\pgfqpoint{2.779477in}{1.068751in}}%
\pgfpathlineto{\pgfqpoint{2.783269in}{1.069921in}}%
\pgfpathlineto{\pgfqpoint{2.795123in}{1.071370in}}%
\pgfpathlineto{\pgfqpoint{2.802846in}{1.073825in}}%
\pgfpathlineto{\pgfqpoint{2.807396in}{1.079812in}}%
\pgfpathlineto{\pgfqpoint{2.814323in}{1.076986in}}%
\pgfpathlineto{\pgfqpoint{2.821177in}{1.078551in}}%
\pgfpathlineto{\pgfqpoint{2.856081in}{1.082279in}}%
\pgfpathlineto{\pgfqpoint{2.856683in}{1.080134in}}%
\pgfpathlineto{\pgfqpoint{2.858078in}{1.069488in}}%
\pgfpathlineto{\pgfqpoint{2.855967in}{1.061094in}}%
\pgfpathlineto{\pgfqpoint{2.856814in}{1.052049in}}%
\pgfpathclose%
\pgfusepath{fill}%
\end{pgfscope}%
\begin{pgfscope}%
\pgfpathrectangle{\pgfqpoint{0.100000in}{0.100000in}}{\pgfqpoint{3.420221in}{2.189500in}}%
\pgfusepath{clip}%
\pgfsetbuttcap%
\pgfsetmiterjoin%
\definecolor{currentfill}{rgb}{0.000000,0.474510,0.762745}%
\pgfsetfillcolor{currentfill}%
\pgfsetlinewidth{0.000000pt}%
\definecolor{currentstroke}{rgb}{0.000000,0.000000,0.000000}%
\pgfsetstrokecolor{currentstroke}%
\pgfsetstrokeopacity{0.000000}%
\pgfsetdash{}{0pt}%
\pgfpathmoveto{\pgfqpoint{2.775451in}{0.949570in}}%
\pgfpathlineto{\pgfqpoint{2.769484in}{0.951418in}}%
\pgfpathlineto{\pgfqpoint{2.762024in}{0.948882in}}%
\pgfpathlineto{\pgfqpoint{2.751535in}{0.950789in}}%
\pgfpathlineto{\pgfqpoint{2.740048in}{0.961268in}}%
\pgfpathlineto{\pgfqpoint{2.741060in}{0.965503in}}%
\pgfpathlineto{\pgfqpoint{2.727017in}{0.966485in}}%
\pgfpathlineto{\pgfqpoint{2.723978in}{0.963056in}}%
\pgfpathlineto{\pgfqpoint{2.720633in}{0.970266in}}%
\pgfpathlineto{\pgfqpoint{2.720306in}{0.978149in}}%
\pgfpathlineto{\pgfqpoint{2.715081in}{0.981409in}}%
\pgfpathlineto{\pgfqpoint{2.719033in}{0.996619in}}%
\pgfpathlineto{\pgfqpoint{2.720569in}{0.997703in}}%
\pgfpathlineto{\pgfqpoint{2.727782in}{0.992925in}}%
\pgfpathlineto{\pgfqpoint{2.731499in}{0.993122in}}%
\pgfpathlineto{\pgfqpoint{2.738862in}{0.986082in}}%
\pgfpathlineto{\pgfqpoint{2.743158in}{0.984399in}}%
\pgfpathlineto{\pgfqpoint{2.749263in}{0.986234in}}%
\pgfpathlineto{\pgfqpoint{2.761447in}{0.970060in}}%
\pgfpathlineto{\pgfqpoint{2.764026in}{0.962785in}}%
\pgfpathlineto{\pgfqpoint{2.772852in}{0.954747in}}%
\pgfpathclose%
\pgfusepath{fill}%
\end{pgfscope}%
\begin{pgfscope}%
\pgfpathrectangle{\pgfqpoint{0.100000in}{0.100000in}}{\pgfqpoint{3.420221in}{2.189500in}}%
\pgfusepath{clip}%
\pgfsetbuttcap%
\pgfsetmiterjoin%
\definecolor{currentfill}{rgb}{0.000000,0.580392,0.709804}%
\pgfsetfillcolor{currentfill}%
\pgfsetlinewidth{0.000000pt}%
\definecolor{currentstroke}{rgb}{0.000000,0.000000,0.000000}%
\pgfsetstrokecolor{currentstroke}%
\pgfsetstrokeopacity{0.000000}%
\pgfsetdash{}{0pt}%
\pgfpathmoveto{\pgfqpoint{0.915423in}{1.457160in}}%
\pgfpathlineto{\pgfqpoint{0.909727in}{1.430544in}}%
\pgfpathlineto{\pgfqpoint{0.896116in}{1.367156in}}%
\pgfpathlineto{\pgfqpoint{0.841775in}{1.379011in}}%
\pgfpathlineto{\pgfqpoint{0.798389in}{1.426183in}}%
\pgfpathlineto{\pgfqpoint{0.759175in}{1.435495in}}%
\pgfpathlineto{\pgfqpoint{0.717990in}{1.445800in}}%
\pgfpathlineto{\pgfqpoint{0.691711in}{1.447049in}}%
\pgfpathlineto{\pgfqpoint{0.692853in}{1.453131in}}%
\pgfpathlineto{\pgfqpoint{0.697480in}{1.456624in}}%
\pgfpathlineto{\pgfqpoint{0.697228in}{1.465697in}}%
\pgfpathlineto{\pgfqpoint{0.705191in}{1.472132in}}%
\pgfpathlineto{\pgfqpoint{0.715753in}{1.475140in}}%
\pgfpathlineto{\pgfqpoint{0.719103in}{1.481002in}}%
\pgfpathlineto{\pgfqpoint{0.719519in}{1.490610in}}%
\pgfpathlineto{\pgfqpoint{0.724234in}{1.502842in}}%
\pgfpathlineto{\pgfqpoint{0.721574in}{1.509614in}}%
\pgfpathlineto{\pgfqpoint{0.749641in}{1.552115in}}%
\pgfpathlineto{\pgfqpoint{0.762241in}{1.549030in}}%
\pgfpathlineto{\pgfqpoint{0.768679in}{1.575019in}}%
\pgfpathlineto{\pgfqpoint{0.786661in}{1.647747in}}%
\pgfpathlineto{\pgfqpoint{0.821804in}{1.638957in}}%
\pgfpathlineto{\pgfqpoint{0.894039in}{1.622190in}}%
\pgfpathlineto{\pgfqpoint{0.905413in}{1.620130in}}%
\pgfpathlineto{\pgfqpoint{0.935196in}{1.613006in}}%
\pgfpathlineto{\pgfqpoint{0.948279in}{1.610168in}}%
\pgfpathlineto{\pgfqpoint{0.932755in}{1.537283in}}%
\pgfpathclose%
\pgfusepath{fill}%
\end{pgfscope}%
\begin{pgfscope}%
\pgfpathrectangle{\pgfqpoint{0.100000in}{0.100000in}}{\pgfqpoint{3.420221in}{2.189500in}}%
\pgfusepath{clip}%
\pgfsetbuttcap%
\pgfsetmiterjoin%
\definecolor{currentfill}{rgb}{0.000000,0.427451,0.786275}%
\pgfsetfillcolor{currentfill}%
\pgfsetlinewidth{0.000000pt}%
\definecolor{currentstroke}{rgb}{0.000000,0.000000,0.000000}%
\pgfsetstrokecolor{currentstroke}%
\pgfsetstrokeopacity{0.000000}%
\pgfsetdash{}{0pt}%
\pgfpathmoveto{\pgfqpoint{1.767928in}{1.617953in}}%
\pgfpathlineto{\pgfqpoint{1.770244in}{1.613093in}}%
\pgfpathlineto{\pgfqpoint{1.774955in}{1.612694in}}%
\pgfpathlineto{\pgfqpoint{1.779779in}{1.603945in}}%
\pgfpathlineto{\pgfqpoint{1.788647in}{1.598062in}}%
\pgfpathlineto{\pgfqpoint{1.791030in}{1.590967in}}%
\pgfpathlineto{\pgfqpoint{1.795390in}{1.590131in}}%
\pgfpathlineto{\pgfqpoint{1.799721in}{1.584739in}}%
\pgfpathlineto{\pgfqpoint{1.808568in}{1.581734in}}%
\pgfpathlineto{\pgfqpoint{1.810014in}{1.578470in}}%
\pgfpathlineto{\pgfqpoint{1.768544in}{1.580207in}}%
\pgfpathlineto{\pgfqpoint{1.716764in}{1.582932in}}%
\pgfpathlineto{\pgfqpoint{1.716481in}{1.592728in}}%
\pgfpathlineto{\pgfqpoint{1.717951in}{1.618928in}}%
\pgfpathlineto{\pgfqpoint{1.718078in}{1.636688in}}%
\pgfpathlineto{\pgfqpoint{1.727787in}{1.634509in}}%
\pgfpathlineto{\pgfqpoint{1.739185in}{1.633624in}}%
\pgfpathlineto{\pgfqpoint{1.751135in}{1.637119in}}%
\pgfpathlineto{\pgfqpoint{1.750202in}{1.618833in}}%
\pgfpathclose%
\pgfusepath{fill}%
\end{pgfscope}%
\begin{pgfscope}%
\pgfpathrectangle{\pgfqpoint{0.100000in}{0.100000in}}{\pgfqpoint{3.420221in}{2.189500in}}%
\pgfusepath{clip}%
\pgfsetbuttcap%
\pgfsetmiterjoin%
\definecolor{currentfill}{rgb}{0.000000,0.541176,0.729412}%
\pgfsetfillcolor{currentfill}%
\pgfsetlinewidth{0.000000pt}%
\definecolor{currentstroke}{rgb}{0.000000,0.000000,0.000000}%
\pgfsetstrokecolor{currentstroke}%
\pgfsetstrokeopacity{0.000000}%
\pgfsetdash{}{0pt}%
\pgfpathmoveto{\pgfqpoint{2.670020in}{0.731361in}}%
\pgfpathlineto{\pgfqpoint{2.652328in}{0.729137in}}%
\pgfpathlineto{\pgfqpoint{2.653553in}{0.715320in}}%
\pgfpathlineto{\pgfqpoint{2.646694in}{0.714784in}}%
\pgfpathlineto{\pgfqpoint{2.641342in}{0.721040in}}%
\pgfpathlineto{\pgfqpoint{2.639738in}{0.729706in}}%
\pgfpathlineto{\pgfqpoint{2.642010in}{0.749780in}}%
\pgfpathlineto{\pgfqpoint{2.640352in}{0.755383in}}%
\pgfpathlineto{\pgfqpoint{2.635374in}{0.760639in}}%
\pgfpathlineto{\pgfqpoint{2.616720in}{0.759523in}}%
\pgfpathlineto{\pgfqpoint{2.617388in}{0.752971in}}%
\pgfpathlineto{\pgfqpoint{2.596171in}{0.750656in}}%
\pgfpathlineto{\pgfqpoint{2.600366in}{0.763774in}}%
\pgfpathlineto{\pgfqpoint{2.599935in}{0.770949in}}%
\pgfpathlineto{\pgfqpoint{2.603506in}{0.780204in}}%
\pgfpathlineto{\pgfqpoint{2.613479in}{0.782585in}}%
\pgfpathlineto{\pgfqpoint{2.613593in}{0.792591in}}%
\pgfpathlineto{\pgfqpoint{2.623267in}{0.793746in}}%
\pgfpathlineto{\pgfqpoint{2.628587in}{0.787752in}}%
\pgfpathlineto{\pgfqpoint{2.636779in}{0.788748in}}%
\pgfpathlineto{\pgfqpoint{2.638390in}{0.782289in}}%
\pgfpathlineto{\pgfqpoint{2.646757in}{0.779342in}}%
\pgfpathlineto{\pgfqpoint{2.667007in}{0.781293in}}%
\pgfpathlineto{\pgfqpoint{2.668073in}{0.770155in}}%
\pgfpathlineto{\pgfqpoint{2.673171in}{0.763869in}}%
\pgfpathlineto{\pgfqpoint{2.679127in}{0.760020in}}%
\pgfpathlineto{\pgfqpoint{2.679480in}{0.754482in}}%
\pgfpathlineto{\pgfqpoint{2.682990in}{0.747945in}}%
\pgfpathlineto{\pgfqpoint{2.678232in}{0.745160in}}%
\pgfpathlineto{\pgfqpoint{2.668779in}{0.744530in}}%
\pgfpathclose%
\pgfusepath{fill}%
\end{pgfscope}%
\begin{pgfscope}%
\pgfpathrectangle{\pgfqpoint{0.100000in}{0.100000in}}{\pgfqpoint{3.420221in}{2.189500in}}%
\pgfusepath{clip}%
\pgfsetbuttcap%
\pgfsetmiterjoin%
\definecolor{currentfill}{rgb}{0.000000,0.615686,0.692157}%
\pgfsetfillcolor{currentfill}%
\pgfsetlinewidth{0.000000pt}%
\definecolor{currentstroke}{rgb}{0.000000,0.000000,0.000000}%
\pgfsetstrokecolor{currentstroke}%
\pgfsetstrokeopacity{0.000000}%
\pgfsetdash{}{0pt}%
\pgfpathmoveto{\pgfqpoint{1.036983in}{1.894726in}}%
\pgfpathlineto{\pgfqpoint{1.034820in}{1.890409in}}%
\pgfpathlineto{\pgfqpoint{1.026000in}{1.889654in}}%
\pgfpathlineto{\pgfqpoint{1.019033in}{1.884505in}}%
\pgfpathlineto{\pgfqpoint{1.011530in}{1.881323in}}%
\pgfpathlineto{\pgfqpoint{1.003780in}{1.877721in}}%
\pgfpathlineto{\pgfqpoint{0.999184in}{1.873285in}}%
\pgfpathlineto{\pgfqpoint{0.986109in}{1.871103in}}%
\pgfpathlineto{\pgfqpoint{0.979613in}{1.878334in}}%
\pgfpathlineto{\pgfqpoint{0.977017in}{1.878098in}}%
\pgfpathlineto{\pgfqpoint{0.982168in}{1.885566in}}%
\pgfpathlineto{\pgfqpoint{0.980431in}{1.894039in}}%
\pgfpathlineto{\pgfqpoint{0.984507in}{1.898684in}}%
\pgfpathlineto{\pgfqpoint{0.991478in}{1.900635in}}%
\pgfpathlineto{\pgfqpoint{0.988643in}{1.913763in}}%
\pgfpathlineto{\pgfqpoint{0.992907in}{1.921757in}}%
\pgfpathlineto{\pgfqpoint{0.993643in}{1.929733in}}%
\pgfpathlineto{\pgfqpoint{0.999885in}{1.939013in}}%
\pgfpathlineto{\pgfqpoint{1.006951in}{1.955930in}}%
\pgfpathlineto{\pgfqpoint{1.005194in}{1.957799in}}%
\pgfpathlineto{\pgfqpoint{0.991947in}{1.958416in}}%
\pgfpathlineto{\pgfqpoint{0.992726in}{1.963526in}}%
\pgfpathlineto{\pgfqpoint{0.979536in}{1.975757in}}%
\pgfpathlineto{\pgfqpoint{0.979308in}{1.983961in}}%
\pgfpathlineto{\pgfqpoint{0.974880in}{1.989787in}}%
\pgfpathlineto{\pgfqpoint{0.966567in}{2.011955in}}%
\pgfpathlineto{\pgfqpoint{0.956801in}{2.017491in}}%
\pgfpathlineto{\pgfqpoint{0.956571in}{2.021476in}}%
\pgfpathlineto{\pgfqpoint{0.948286in}{2.030531in}}%
\pgfpathlineto{\pgfqpoint{0.954903in}{2.031949in}}%
\pgfpathlineto{\pgfqpoint{0.963906in}{2.028834in}}%
\pgfpathlineto{\pgfqpoint{0.972309in}{2.028527in}}%
\pgfpathlineto{\pgfqpoint{0.985938in}{2.018362in}}%
\pgfpathlineto{\pgfqpoint{0.985087in}{2.012119in}}%
\pgfpathlineto{\pgfqpoint{0.990720in}{2.007460in}}%
\pgfpathlineto{\pgfqpoint{0.998622in}{2.006162in}}%
\pgfpathlineto{\pgfqpoint{1.010779in}{1.998519in}}%
\pgfpathlineto{\pgfqpoint{1.017863in}{1.990055in}}%
\pgfpathlineto{\pgfqpoint{1.021519in}{1.990629in}}%
\pgfpathlineto{\pgfqpoint{1.034294in}{1.987915in}}%
\pgfpathlineto{\pgfqpoint{1.040598in}{1.989940in}}%
\pgfpathlineto{\pgfqpoint{1.039745in}{1.998406in}}%
\pgfpathlineto{\pgfqpoint{1.041422in}{2.006055in}}%
\pgfpathlineto{\pgfqpoint{1.038710in}{2.013123in}}%
\pgfpathlineto{\pgfqpoint{1.041716in}{2.021843in}}%
\pgfpathlineto{\pgfqpoint{1.064900in}{2.017251in}}%
\pgfpathlineto{\pgfqpoint{1.058483in}{1.986012in}}%
\pgfpathlineto{\pgfqpoint{1.066718in}{1.984360in}}%
\pgfpathlineto{\pgfqpoint{1.061456in}{1.958613in}}%
\pgfpathlineto{\pgfqpoint{1.055714in}{1.959774in}}%
\pgfpathlineto{\pgfqpoint{1.051162in}{1.953993in}}%
\pgfpathlineto{\pgfqpoint{1.041183in}{1.953830in}}%
\pgfpathlineto{\pgfqpoint{1.040319in}{1.949562in}}%
\pgfpathlineto{\pgfqpoint{1.033914in}{1.950893in}}%
\pgfpathlineto{\pgfqpoint{1.027437in}{1.940734in}}%
\pgfpathlineto{\pgfqpoint{1.026949in}{1.933890in}}%
\pgfpathlineto{\pgfqpoint{1.029878in}{1.929035in}}%
\pgfpathlineto{\pgfqpoint{1.030107in}{1.921690in}}%
\pgfpathlineto{\pgfqpoint{1.025971in}{1.913483in}}%
\pgfpathlineto{\pgfqpoint{1.025117in}{1.907610in}}%
\pgfpathlineto{\pgfqpoint{1.029681in}{1.903705in}}%
\pgfpathlineto{\pgfqpoint{1.030613in}{1.898083in}}%
\pgfpathclose%
\pgfusepath{fill}%
\end{pgfscope}%
\begin{pgfscope}%
\pgfpathrectangle{\pgfqpoint{0.100000in}{0.100000in}}{\pgfqpoint{3.420221in}{2.189500in}}%
\pgfusepath{clip}%
\pgfsetbuttcap%
\pgfsetmiterjoin%
\definecolor{currentfill}{rgb}{0.000000,0.403922,0.798039}%
\pgfsetfillcolor{currentfill}%
\pgfsetlinewidth{0.000000pt}%
\definecolor{currentstroke}{rgb}{0.000000,0.000000,0.000000}%
\pgfsetstrokecolor{currentstroke}%
\pgfsetstrokeopacity{0.000000}%
\pgfsetdash{}{0pt}%
\pgfpathmoveto{\pgfqpoint{2.084131in}{1.813030in}}%
\pgfpathlineto{\pgfqpoint{2.103890in}{1.813724in}}%
\pgfpathlineto{\pgfqpoint{2.103417in}{1.833445in}}%
\pgfpathlineto{\pgfqpoint{2.143098in}{1.834306in}}%
\pgfpathlineto{\pgfqpoint{2.143638in}{1.814683in}}%
\pgfpathlineto{\pgfqpoint{2.143808in}{1.808422in}}%
\pgfpathlineto{\pgfqpoint{2.140984in}{1.803957in}}%
\pgfpathlineto{\pgfqpoint{2.125967in}{1.797209in}}%
\pgfpathlineto{\pgfqpoint{2.122208in}{1.793985in}}%
\pgfpathlineto{\pgfqpoint{2.118635in}{1.784268in}}%
\pgfpathlineto{\pgfqpoint{2.115787in}{1.781674in}}%
\pgfpathlineto{\pgfqpoint{2.080363in}{1.781293in}}%
\pgfpathlineto{\pgfqpoint{2.079833in}{1.799955in}}%
\pgfpathlineto{\pgfqpoint{2.084321in}{1.800001in}}%
\pgfpathclose%
\pgfusepath{fill}%
\end{pgfscope}%
\begin{pgfscope}%
\pgfpathrectangle{\pgfqpoint{0.100000in}{0.100000in}}{\pgfqpoint{3.420221in}{2.189500in}}%
\pgfusepath{clip}%
\pgfsetbuttcap%
\pgfsetmiterjoin%
\definecolor{currentfill}{rgb}{0.000000,0.435294,0.782353}%
\pgfsetfillcolor{currentfill}%
\pgfsetlinewidth{0.000000pt}%
\definecolor{currentstroke}{rgb}{0.000000,0.000000,0.000000}%
\pgfsetstrokecolor{currentstroke}%
\pgfsetstrokeopacity{0.000000}%
\pgfsetdash{}{0pt}%
\pgfpathmoveto{\pgfqpoint{1.762663in}{1.130354in}}%
\pgfpathlineto{\pgfqpoint{1.735430in}{1.131613in}}%
\pgfpathlineto{\pgfqpoint{1.730353in}{1.131874in}}%
\pgfpathlineto{\pgfqpoint{1.731658in}{1.160443in}}%
\pgfpathlineto{\pgfqpoint{1.763905in}{1.159164in}}%
\pgfpathclose%
\pgfusepath{fill}%
\end{pgfscope}%
\begin{pgfscope}%
\pgfpathrectangle{\pgfqpoint{0.100000in}{0.100000in}}{\pgfqpoint{3.420221in}{2.189500in}}%
\pgfusepath{clip}%
\pgfsetbuttcap%
\pgfsetmiterjoin%
\definecolor{currentfill}{rgb}{0.000000,0.529412,0.735294}%
\pgfsetfillcolor{currentfill}%
\pgfsetlinewidth{0.000000pt}%
\definecolor{currentstroke}{rgb}{0.000000,0.000000,0.000000}%
\pgfsetstrokecolor{currentstroke}%
\pgfsetstrokeopacity{0.000000}%
\pgfsetdash{}{0pt}%
\pgfpathmoveto{\pgfqpoint{2.459552in}{1.156183in}}%
\pgfpathlineto{\pgfqpoint{2.448686in}{1.151389in}}%
\pgfpathlineto{\pgfqpoint{2.438930in}{1.153919in}}%
\pgfpathlineto{\pgfqpoint{2.437904in}{1.144214in}}%
\pgfpathlineto{\pgfqpoint{2.434954in}{1.140453in}}%
\pgfpathlineto{\pgfqpoint{2.427538in}{1.138774in}}%
\pgfpathlineto{\pgfqpoint{2.412131in}{1.130599in}}%
\pgfpathlineto{\pgfqpoint{2.406441in}{1.146645in}}%
\pgfpathlineto{\pgfqpoint{2.408730in}{1.152021in}}%
\pgfpathlineto{\pgfqpoint{2.406206in}{1.160670in}}%
\pgfpathlineto{\pgfqpoint{2.397160in}{1.169901in}}%
\pgfpathlineto{\pgfqpoint{2.401453in}{1.173805in}}%
\pgfpathlineto{\pgfqpoint{2.412858in}{1.176050in}}%
\pgfpathlineto{\pgfqpoint{2.407560in}{1.190110in}}%
\pgfpathlineto{\pgfqpoint{2.414510in}{1.200841in}}%
\pgfpathlineto{\pgfqpoint{2.421617in}{1.202024in}}%
\pgfpathlineto{\pgfqpoint{2.419270in}{1.208092in}}%
\pgfpathlineto{\pgfqpoint{2.425709in}{1.207524in}}%
\pgfpathlineto{\pgfqpoint{2.434375in}{1.210233in}}%
\pgfpathlineto{\pgfqpoint{2.438734in}{1.205268in}}%
\pgfpathlineto{\pgfqpoint{2.441905in}{1.212484in}}%
\pgfpathlineto{\pgfqpoint{2.448956in}{1.214546in}}%
\pgfpathlineto{\pgfqpoint{2.456709in}{1.211714in}}%
\pgfpathlineto{\pgfqpoint{2.456043in}{1.205634in}}%
\pgfpathlineto{\pgfqpoint{2.447988in}{1.188551in}}%
\pgfpathlineto{\pgfqpoint{2.455493in}{1.185294in}}%
\pgfpathlineto{\pgfqpoint{2.456718in}{1.176229in}}%
\pgfpathlineto{\pgfqpoint{2.460094in}{1.172603in}}%
\pgfpathlineto{\pgfqpoint{2.455730in}{1.163251in}}%
\pgfpathclose%
\pgfusepath{fill}%
\end{pgfscope}%
\begin{pgfscope}%
\pgfpathrectangle{\pgfqpoint{0.100000in}{0.100000in}}{\pgfqpoint{3.420221in}{2.189500in}}%
\pgfusepath{clip}%
\pgfsetbuttcap%
\pgfsetmiterjoin%
\definecolor{currentfill}{rgb}{0.000000,0.411765,0.794118}%
\pgfsetfillcolor{currentfill}%
\pgfsetlinewidth{0.000000pt}%
\definecolor{currentstroke}{rgb}{0.000000,0.000000,0.000000}%
\pgfsetstrokecolor{currentstroke}%
\pgfsetstrokeopacity{0.000000}%
\pgfsetdash{}{0pt}%
\pgfpathmoveto{\pgfqpoint{1.557929in}{1.922392in}}%
\pgfpathlineto{\pgfqpoint{1.555395in}{1.896052in}}%
\pgfpathlineto{\pgfqpoint{1.558348in}{1.895779in}}%
\pgfpathlineto{\pgfqpoint{1.555861in}{1.869418in}}%
\pgfpathlineto{\pgfqpoint{1.545954in}{1.870352in}}%
\pgfpathlineto{\pgfqpoint{1.545306in}{1.863670in}}%
\pgfpathlineto{\pgfqpoint{1.532672in}{1.864929in}}%
\pgfpathlineto{\pgfqpoint{1.533419in}{1.872490in}}%
\pgfpathlineto{\pgfqpoint{1.517484in}{1.874108in}}%
\pgfpathlineto{\pgfqpoint{1.514524in}{1.877717in}}%
\pgfpathlineto{\pgfqpoint{1.504764in}{1.878745in}}%
\pgfpathlineto{\pgfqpoint{1.506335in}{1.891959in}}%
\pgfpathlineto{\pgfqpoint{1.484552in}{1.894368in}}%
\pgfpathlineto{\pgfqpoint{1.482877in}{1.898940in}}%
\pgfpathlineto{\pgfqpoint{1.476332in}{1.899677in}}%
\pgfpathlineto{\pgfqpoint{1.476835in}{1.904036in}}%
\pgfpathlineto{\pgfqpoint{1.470261in}{1.904780in}}%
\pgfpathlineto{\pgfqpoint{1.472018in}{1.920141in}}%
\pgfpathlineto{\pgfqpoint{1.467857in}{1.920627in}}%
\pgfpathlineto{\pgfqpoint{1.470851in}{1.946786in}}%
\pgfpathlineto{\pgfqpoint{1.473041in}{1.946549in}}%
\pgfpathlineto{\pgfqpoint{1.475316in}{1.966203in}}%
\pgfpathlineto{\pgfqpoint{1.481817in}{1.965489in}}%
\pgfpathlineto{\pgfqpoint{1.481061in}{1.958927in}}%
\pgfpathlineto{\pgfqpoint{1.494083in}{1.957455in}}%
\pgfpathlineto{\pgfqpoint{1.493333in}{1.950894in}}%
\pgfpathlineto{\pgfqpoint{1.512880in}{1.948756in}}%
\pgfpathlineto{\pgfqpoint{1.512176in}{1.942201in}}%
\pgfpathlineto{\pgfqpoint{1.516297in}{1.941788in}}%
\pgfpathlineto{\pgfqpoint{1.514894in}{1.928641in}}%
\pgfpathlineto{\pgfqpoint{1.534373in}{1.926592in}}%
\pgfpathlineto{\pgfqpoint{1.538591in}{1.924341in}}%
\pgfpathclose%
\pgfusepath{fill}%
\end{pgfscope}%
\begin{pgfscope}%
\pgfpathrectangle{\pgfqpoint{0.100000in}{0.100000in}}{\pgfqpoint{3.420221in}{2.189500in}}%
\pgfusepath{clip}%
\pgfsetbuttcap%
\pgfsetmiterjoin%
\definecolor{currentfill}{rgb}{0.000000,0.517647,0.741176}%
\pgfsetfillcolor{currentfill}%
\pgfsetlinewidth{0.000000pt}%
\definecolor{currentstroke}{rgb}{0.000000,0.000000,0.000000}%
\pgfsetstrokecolor{currentstroke}%
\pgfsetstrokeopacity{0.000000}%
\pgfsetdash{}{0pt}%
\pgfpathmoveto{\pgfqpoint{2.407745in}{1.356666in}}%
\pgfpathlineto{\pgfqpoint{2.407207in}{1.342053in}}%
\pgfpathlineto{\pgfqpoint{2.401832in}{1.339131in}}%
\pgfpathlineto{\pgfqpoint{2.378335in}{1.337307in}}%
\pgfpathlineto{\pgfqpoint{2.380743in}{1.304684in}}%
\pgfpathlineto{\pgfqpoint{2.341983in}{1.302263in}}%
\pgfpathlineto{\pgfqpoint{2.341370in}{1.312070in}}%
\pgfpathlineto{\pgfqpoint{2.347993in}{1.312284in}}%
\pgfpathlineto{\pgfqpoint{2.346460in}{1.335309in}}%
\pgfpathlineto{\pgfqpoint{2.339929in}{1.334967in}}%
\pgfpathlineto{\pgfqpoint{2.339069in}{1.345836in}}%
\pgfpathlineto{\pgfqpoint{2.334661in}{1.346489in}}%
\pgfpathlineto{\pgfqpoint{2.334151in}{1.354252in}}%
\pgfpathlineto{\pgfqpoint{2.338419in}{1.354591in}}%
\pgfpathlineto{\pgfqpoint{2.337710in}{1.364365in}}%
\pgfpathlineto{\pgfqpoint{2.360514in}{1.366349in}}%
\pgfpathlineto{\pgfqpoint{2.369073in}{1.383972in}}%
\pgfpathlineto{\pgfqpoint{2.375651in}{1.384447in}}%
\pgfpathlineto{\pgfqpoint{2.373856in}{1.409540in}}%
\pgfpathlineto{\pgfqpoint{2.386590in}{1.410534in}}%
\pgfpathlineto{\pgfqpoint{2.383796in}{1.438635in}}%
\pgfpathlineto{\pgfqpoint{2.390289in}{1.439339in}}%
\pgfpathlineto{\pgfqpoint{2.393956in}{1.401317in}}%
\pgfpathlineto{\pgfqpoint{2.404337in}{1.401975in}}%
\pgfpathlineto{\pgfqpoint{2.405465in}{1.382473in}}%
\pgfpathclose%
\pgfusepath{fill}%
\end{pgfscope}%
\begin{pgfscope}%
\pgfpathrectangle{\pgfqpoint{0.100000in}{0.100000in}}{\pgfqpoint{3.420221in}{2.189500in}}%
\pgfusepath{clip}%
\pgfsetbuttcap%
\pgfsetmiterjoin%
\definecolor{currentfill}{rgb}{0.000000,0.341176,0.829412}%
\pgfsetfillcolor{currentfill}%
\pgfsetlinewidth{0.000000pt}%
\definecolor{currentstroke}{rgb}{0.000000,0.000000,0.000000}%
\pgfsetstrokecolor{currentstroke}%
\pgfsetstrokeopacity{0.000000}%
\pgfsetdash{}{0pt}%
\pgfpathmoveto{\pgfqpoint{1.557929in}{1.922392in}}%
\pgfpathlineto{\pgfqpoint{1.538591in}{1.924341in}}%
\pgfpathlineto{\pgfqpoint{1.534373in}{1.926592in}}%
\pgfpathlineto{\pgfqpoint{1.514894in}{1.928641in}}%
\pgfpathlineto{\pgfqpoint{1.516297in}{1.941788in}}%
\pgfpathlineto{\pgfqpoint{1.512176in}{1.942201in}}%
\pgfpathlineto{\pgfqpoint{1.512880in}{1.948756in}}%
\pgfpathlineto{\pgfqpoint{1.493333in}{1.950894in}}%
\pgfpathlineto{\pgfqpoint{1.494083in}{1.957455in}}%
\pgfpathlineto{\pgfqpoint{1.481061in}{1.958927in}}%
\pgfpathlineto{\pgfqpoint{1.481817in}{1.965489in}}%
\pgfpathlineto{\pgfqpoint{1.484483in}{1.971097in}}%
\pgfpathlineto{\pgfqpoint{1.486170in}{1.985985in}}%
\pgfpathlineto{\pgfqpoint{1.493269in}{1.984614in}}%
\pgfpathlineto{\pgfqpoint{1.505517in}{1.987939in}}%
\pgfpathlineto{\pgfqpoint{1.512446in}{1.985950in}}%
\pgfpathlineto{\pgfqpoint{1.523523in}{1.986154in}}%
\pgfpathlineto{\pgfqpoint{1.528466in}{1.980121in}}%
\pgfpathlineto{\pgfqpoint{1.539973in}{1.978104in}}%
\pgfpathlineto{\pgfqpoint{1.543671in}{1.974567in}}%
\pgfpathlineto{\pgfqpoint{1.552230in}{1.971795in}}%
\pgfpathlineto{\pgfqpoint{1.562829in}{1.982271in}}%
\pgfpathlineto{\pgfqpoint{1.566570in}{1.981365in}}%
\pgfpathlineto{\pgfqpoint{1.567945in}{1.975228in}}%
\pgfpathlineto{\pgfqpoint{1.571998in}{1.972636in}}%
\pgfpathlineto{\pgfqpoint{1.582856in}{1.975541in}}%
\pgfpathlineto{\pgfqpoint{1.587192in}{1.981029in}}%
\pgfpathlineto{\pgfqpoint{1.604114in}{1.979668in}}%
\pgfpathlineto{\pgfqpoint{1.607608in}{1.997345in}}%
\pgfpathlineto{\pgfqpoint{1.604724in}{1.997573in}}%
\pgfpathlineto{\pgfqpoint{1.605842in}{2.010721in}}%
\pgfpathlineto{\pgfqpoint{1.638517in}{2.007966in}}%
\pgfpathlineto{\pgfqpoint{1.651609in}{2.007035in}}%
\pgfpathlineto{\pgfqpoint{1.650585in}{1.993775in}}%
\pgfpathlineto{\pgfqpoint{1.653161in}{1.993596in}}%
\pgfpathlineto{\pgfqpoint{1.651134in}{1.967171in}}%
\pgfpathlineto{\pgfqpoint{1.653702in}{1.966967in}}%
\pgfpathlineto{\pgfqpoint{1.652699in}{1.953699in}}%
\pgfpathlineto{\pgfqpoint{1.626616in}{1.955744in}}%
\pgfpathlineto{\pgfqpoint{1.626060in}{1.948968in}}%
\pgfpathlineto{\pgfqpoint{1.617248in}{1.951376in}}%
\pgfpathlineto{\pgfqpoint{1.612469in}{1.943696in}}%
\pgfpathlineto{\pgfqpoint{1.589228in}{1.945752in}}%
\pgfpathlineto{\pgfqpoint{1.586932in}{1.919663in}}%
\pgfpathclose%
\pgfusepath{fill}%
\end{pgfscope}%
\begin{pgfscope}%
\pgfpathrectangle{\pgfqpoint{0.100000in}{0.100000in}}{\pgfqpoint{3.420221in}{2.189500in}}%
\pgfusepath{clip}%
\pgfsetbuttcap%
\pgfsetmiterjoin%
\definecolor{currentfill}{rgb}{0.000000,0.478431,0.760784}%
\pgfsetfillcolor{currentfill}%
\pgfsetlinewidth{0.000000pt}%
\definecolor{currentstroke}{rgb}{0.000000,0.000000,0.000000}%
\pgfsetstrokecolor{currentstroke}%
\pgfsetstrokeopacity{0.000000}%
\pgfsetdash{}{0pt}%
\pgfpathmoveto{\pgfqpoint{3.355944in}{1.643616in}}%
\pgfpathlineto{\pgfqpoint{3.367082in}{1.636476in}}%
\pgfpathlineto{\pgfqpoint{3.353143in}{1.632089in}}%
\pgfpathlineto{\pgfqpoint{3.351798in}{1.638864in}}%
\pgfpathclose%
\pgfusepath{fill}%
\end{pgfscope}%
\begin{pgfscope}%
\pgfpathrectangle{\pgfqpoint{0.100000in}{0.100000in}}{\pgfqpoint{3.420221in}{2.189500in}}%
\pgfusepath{clip}%
\pgfsetbuttcap%
\pgfsetmiterjoin%
\definecolor{currentfill}{rgb}{0.000000,0.282353,0.858824}%
\pgfsetfillcolor{currentfill}%
\pgfsetlinewidth{0.000000pt}%
\definecolor{currentstroke}{rgb}{0.000000,0.000000,0.000000}%
\pgfsetstrokecolor{currentstroke}%
\pgfsetstrokeopacity{0.000000}%
\pgfsetdash{}{0pt}%
\pgfpathmoveto{\pgfqpoint{2.994238in}{1.273383in}}%
\pgfpathlineto{\pgfqpoint{2.990351in}{1.268242in}}%
\pgfpathlineto{\pgfqpoint{2.990564in}{1.264098in}}%
\pgfpathlineto{\pgfqpoint{2.981969in}{1.239955in}}%
\pgfpathlineto{\pgfqpoint{2.973603in}{1.242864in}}%
\pgfpathlineto{\pgfqpoint{2.967963in}{1.244216in}}%
\pgfpathlineto{\pgfqpoint{2.958031in}{1.251815in}}%
\pgfpathlineto{\pgfqpoint{2.959010in}{1.258807in}}%
\pgfpathlineto{\pgfqpoint{2.963734in}{1.263484in}}%
\pgfpathlineto{\pgfqpoint{2.962584in}{1.271232in}}%
\pgfpathlineto{\pgfqpoint{2.950652in}{1.288179in}}%
\pgfpathlineto{\pgfqpoint{2.953656in}{1.292142in}}%
\pgfpathlineto{\pgfqpoint{2.953750in}{1.300847in}}%
\pgfpathlineto{\pgfqpoint{2.958482in}{1.305816in}}%
\pgfpathlineto{\pgfqpoint{2.965897in}{1.318284in}}%
\pgfpathlineto{\pgfqpoint{2.967394in}{1.326906in}}%
\pgfpathlineto{\pgfqpoint{2.971807in}{1.336194in}}%
\pgfpathlineto{\pgfqpoint{2.979254in}{1.330335in}}%
\pgfpathlineto{\pgfqpoint{2.982931in}{1.330309in}}%
\pgfpathlineto{\pgfqpoint{2.994009in}{1.345576in}}%
\pgfpathlineto{\pgfqpoint{2.998994in}{1.338041in}}%
\pgfpathlineto{\pgfqpoint{3.009477in}{1.327012in}}%
\pgfpathlineto{\pgfqpoint{3.015325in}{1.327855in}}%
\pgfpathlineto{\pgfqpoint{3.001405in}{1.303080in}}%
\pgfpathlineto{\pgfqpoint{3.008087in}{1.302911in}}%
\pgfpathlineto{\pgfqpoint{3.018032in}{1.297713in}}%
\pgfpathlineto{\pgfqpoint{3.015583in}{1.275977in}}%
\pgfpathlineto{\pgfqpoint{3.009839in}{1.277278in}}%
\pgfpathlineto{\pgfqpoint{3.005162in}{1.282897in}}%
\pgfpathlineto{\pgfqpoint{2.997658in}{1.285990in}}%
\pgfpathclose%
\pgfusepath{fill}%
\end{pgfscope}%
\begin{pgfscope}%
\pgfpathrectangle{\pgfqpoint{0.100000in}{0.100000in}}{\pgfqpoint{3.420221in}{2.189500in}}%
\pgfusepath{clip}%
\pgfsetbuttcap%
\pgfsetmiterjoin%
\definecolor{currentfill}{rgb}{0.000000,0.294118,0.852941}%
\pgfsetfillcolor{currentfill}%
\pgfsetlinewidth{0.000000pt}%
\definecolor{currentstroke}{rgb}{0.000000,0.000000,0.000000}%
\pgfsetstrokecolor{currentstroke}%
\pgfsetstrokeopacity{0.000000}%
\pgfsetdash{}{0pt}%
\pgfpathmoveto{\pgfqpoint{1.628880in}{1.001237in}}%
\pgfpathlineto{\pgfqpoint{1.626424in}{0.968739in}}%
\pgfpathlineto{\pgfqpoint{1.602969in}{0.970304in}}%
\pgfpathlineto{\pgfqpoint{1.570538in}{0.972766in}}%
\pgfpathlineto{\pgfqpoint{1.538122in}{0.975267in}}%
\pgfpathlineto{\pgfqpoint{1.506292in}{0.978032in}}%
\pgfpathlineto{\pgfqpoint{1.507670in}{0.993501in}}%
\pgfpathlineto{\pgfqpoint{1.512174in}{1.043481in}}%
\pgfpathlineto{\pgfqpoint{1.565623in}{1.039378in}}%
\pgfpathlineto{\pgfqpoint{1.598430in}{1.036334in}}%
\pgfpathlineto{\pgfqpoint{1.596032in}{1.003610in}}%
\pgfpathclose%
\pgfusepath{fill}%
\end{pgfscope}%
\begin{pgfscope}%
\pgfpathrectangle{\pgfqpoint{0.100000in}{0.100000in}}{\pgfqpoint{3.420221in}{2.189500in}}%
\pgfusepath{clip}%
\pgfsetbuttcap%
\pgfsetmiterjoin%
\definecolor{currentfill}{rgb}{0.000000,0.278431,0.860784}%
\pgfsetfillcolor{currentfill}%
\pgfsetlinewidth{0.000000pt}%
\definecolor{currentstroke}{rgb}{0.000000,0.000000,0.000000}%
\pgfsetstrokecolor{currentstroke}%
\pgfsetstrokeopacity{0.000000}%
\pgfsetdash{}{0pt}%
\pgfpathmoveto{\pgfqpoint{2.168009in}{0.906817in}}%
\pgfpathlineto{\pgfqpoint{2.158058in}{0.912898in}}%
\pgfpathlineto{\pgfqpoint{2.146551in}{0.912997in}}%
\pgfpathlineto{\pgfqpoint{2.146643in}{0.932899in}}%
\pgfpathlineto{\pgfqpoint{2.145518in}{0.937290in}}%
\pgfpathlineto{\pgfqpoint{2.138755in}{0.939421in}}%
\pgfpathlineto{\pgfqpoint{2.137688in}{0.945920in}}%
\pgfpathlineto{\pgfqpoint{2.133439in}{0.949180in}}%
\pgfpathlineto{\pgfqpoint{2.126981in}{0.949190in}}%
\pgfpathlineto{\pgfqpoint{2.127413in}{0.959348in}}%
\pgfpathlineto{\pgfqpoint{2.107760in}{0.959013in}}%
\pgfpathlineto{\pgfqpoint{2.106511in}{0.966683in}}%
\pgfpathlineto{\pgfqpoint{2.115323in}{0.974397in}}%
\pgfpathlineto{\pgfqpoint{2.129349in}{0.989036in}}%
\pgfpathlineto{\pgfqpoint{2.133750in}{0.989583in}}%
\pgfpathlineto{\pgfqpoint{2.133504in}{1.011369in}}%
\pgfpathlineto{\pgfqpoint{2.156196in}{1.011393in}}%
\pgfpathlineto{\pgfqpoint{2.156264in}{1.004853in}}%
\pgfpathlineto{\pgfqpoint{2.178916in}{1.005102in}}%
\pgfpathlineto{\pgfqpoint{2.179115in}{0.982850in}}%
\pgfpathlineto{\pgfqpoint{2.185959in}{0.983174in}}%
\pgfpathlineto{\pgfqpoint{2.194232in}{0.980117in}}%
\pgfpathlineto{\pgfqpoint{2.198697in}{0.980799in}}%
\pgfpathlineto{\pgfqpoint{2.198900in}{0.972629in}}%
\pgfpathlineto{\pgfqpoint{2.205687in}{0.972729in}}%
\pgfpathlineto{\pgfqpoint{2.205922in}{0.954125in}}%
\pgfpathlineto{\pgfqpoint{2.209897in}{0.950723in}}%
\pgfpathlineto{\pgfqpoint{2.210145in}{0.944527in}}%
\pgfpathlineto{\pgfqpoint{2.206170in}{0.939887in}}%
\pgfpathlineto{\pgfqpoint{2.175034in}{0.939603in}}%
\pgfpathlineto{\pgfqpoint{2.175317in}{0.919876in}}%
\pgfpathlineto{\pgfqpoint{2.174430in}{0.907210in}}%
\pgfpathclose%
\pgfusepath{fill}%
\end{pgfscope}%
\begin{pgfscope}%
\pgfpathrectangle{\pgfqpoint{0.100000in}{0.100000in}}{\pgfqpoint{3.420221in}{2.189500in}}%
\pgfusepath{clip}%
\pgfsetbuttcap%
\pgfsetmiterjoin%
\definecolor{currentfill}{rgb}{0.000000,0.443137,0.778431}%
\pgfsetfillcolor{currentfill}%
\pgfsetlinewidth{0.000000pt}%
\definecolor{currentstroke}{rgb}{0.000000,0.000000,0.000000}%
\pgfsetstrokecolor{currentstroke}%
\pgfsetstrokeopacity{0.000000}%
\pgfsetdash{}{0pt}%
\pgfpathmoveto{\pgfqpoint{1.477993in}{1.813430in}}%
\pgfpathlineto{\pgfqpoint{1.474128in}{1.780854in}}%
\pgfpathlineto{\pgfqpoint{1.471439in}{1.781154in}}%
\pgfpathlineto{\pgfqpoint{1.468508in}{1.754814in}}%
\pgfpathlineto{\pgfqpoint{1.422219in}{1.760219in}}%
\pgfpathlineto{\pgfqpoint{1.416260in}{1.760382in}}%
\pgfpathlineto{\pgfqpoint{1.403638in}{1.761954in}}%
\pgfpathlineto{\pgfqpoint{1.405365in}{1.775860in}}%
\pgfpathlineto{\pgfqpoint{1.378792in}{1.779238in}}%
\pgfpathlineto{\pgfqpoint{1.382484in}{1.791831in}}%
\pgfpathlineto{\pgfqpoint{1.385596in}{1.815858in}}%
\pgfpathlineto{\pgfqpoint{1.376126in}{1.817941in}}%
\pgfpathlineto{\pgfqpoint{1.376601in}{1.832033in}}%
\pgfpathlineto{\pgfqpoint{1.365121in}{1.840167in}}%
\pgfpathlineto{\pgfqpoint{1.352196in}{1.841927in}}%
\pgfpathlineto{\pgfqpoint{1.353089in}{1.848403in}}%
\pgfpathlineto{\pgfqpoint{1.348591in}{1.849014in}}%
\pgfpathlineto{\pgfqpoint{1.352320in}{1.858733in}}%
\pgfpathlineto{\pgfqpoint{1.346101in}{1.869544in}}%
\pgfpathlineto{\pgfqpoint{1.342376in}{1.882222in}}%
\pgfpathlineto{\pgfqpoint{1.338590in}{1.884933in}}%
\pgfpathlineto{\pgfqpoint{1.340743in}{1.898702in}}%
\pgfpathlineto{\pgfqpoint{1.339766in}{1.904622in}}%
\pgfpathlineto{\pgfqpoint{1.337478in}{1.912255in}}%
\pgfpathlineto{\pgfqpoint{1.397628in}{1.904675in}}%
\pgfpathlineto{\pgfqpoint{1.397477in}{1.903600in}}%
\pgfpathlineto{\pgfqpoint{1.430010in}{1.899550in}}%
\pgfpathlineto{\pgfqpoint{1.428108in}{1.898667in}}%
\pgfpathlineto{\pgfqpoint{1.424925in}{1.872572in}}%
\pgfpathlineto{\pgfqpoint{1.423124in}{1.872785in}}%
\pgfpathlineto{\pgfqpoint{1.420025in}{1.846812in}}%
\pgfpathlineto{\pgfqpoint{1.418041in}{1.847044in}}%
\pgfpathlineto{\pgfqpoint{1.414789in}{1.820947in}}%
\pgfpathclose%
\pgfusepath{fill}%
\end{pgfscope}%
\begin{pgfscope}%
\pgfpathrectangle{\pgfqpoint{0.100000in}{0.100000in}}{\pgfqpoint{3.420221in}{2.189500in}}%
\pgfusepath{clip}%
\pgfsetbuttcap%
\pgfsetmiterjoin%
\definecolor{currentfill}{rgb}{0.000000,0.670588,0.664706}%
\pgfsetfillcolor{currentfill}%
\pgfsetlinewidth{0.000000pt}%
\definecolor{currentstroke}{rgb}{0.000000,0.000000,0.000000}%
\pgfsetstrokecolor{currentstroke}%
\pgfsetstrokeopacity{0.000000}%
\pgfsetdash{}{0pt}%
\pgfpathmoveto{\pgfqpoint{2.828555in}{0.767904in}}%
\pgfpathlineto{\pgfqpoint{2.820739in}{0.773264in}}%
\pgfpathlineto{\pgfqpoint{2.804649in}{0.780507in}}%
\pgfpathlineto{\pgfqpoint{2.781599in}{0.776663in}}%
\pgfpathlineto{\pgfqpoint{2.778813in}{0.784810in}}%
\pgfpathlineto{\pgfqpoint{2.770382in}{0.783477in}}%
\pgfpathlineto{\pgfqpoint{2.758147in}{0.787000in}}%
\pgfpathlineto{\pgfqpoint{2.754151in}{0.793358in}}%
\pgfpathlineto{\pgfqpoint{2.750090in}{0.795690in}}%
\pgfpathlineto{\pgfqpoint{2.748458in}{0.803306in}}%
\pgfpathlineto{\pgfqpoint{2.745504in}{0.804988in}}%
\pgfpathlineto{\pgfqpoint{2.747718in}{0.810831in}}%
\pgfpathlineto{\pgfqpoint{2.741795in}{0.815449in}}%
\pgfpathlineto{\pgfqpoint{2.745283in}{0.822917in}}%
\pgfpathlineto{\pgfqpoint{2.752509in}{0.832267in}}%
\pgfpathlineto{\pgfqpoint{2.754960in}{0.830347in}}%
\pgfpathlineto{\pgfqpoint{2.767042in}{0.810958in}}%
\pgfpathlineto{\pgfqpoint{2.774969in}{0.815604in}}%
\pgfpathlineto{\pgfqpoint{2.782427in}{0.825358in}}%
\pgfpathlineto{\pgfqpoint{2.781297in}{0.827579in}}%
\pgfpathlineto{\pgfqpoint{2.785056in}{0.841197in}}%
\pgfpathlineto{\pgfqpoint{2.795105in}{0.841350in}}%
\pgfpathlineto{\pgfqpoint{2.802524in}{0.838111in}}%
\pgfpathlineto{\pgfqpoint{2.801465in}{0.832261in}}%
\pgfpathlineto{\pgfqpoint{2.805829in}{0.827285in}}%
\pgfpathlineto{\pgfqpoint{2.813198in}{0.830343in}}%
\pgfpathlineto{\pgfqpoint{2.818056in}{0.819380in}}%
\pgfpathlineto{\pgfqpoint{2.817994in}{0.800107in}}%
\pgfpathlineto{\pgfqpoint{2.825742in}{0.798969in}}%
\pgfpathlineto{\pgfqpoint{2.828880in}{0.795948in}}%
\pgfpathlineto{\pgfqpoint{2.825306in}{0.790493in}}%
\pgfpathclose%
\pgfusepath{fill}%
\end{pgfscope}%
\begin{pgfscope}%
\pgfpathrectangle{\pgfqpoint{0.100000in}{0.100000in}}{\pgfqpoint{3.420221in}{2.189500in}}%
\pgfusepath{clip}%
\pgfsetbuttcap%
\pgfsetmiterjoin%
\definecolor{currentfill}{rgb}{0.000000,0.270588,0.864706}%
\pgfsetfillcolor{currentfill}%
\pgfsetlinewidth{0.000000pt}%
\definecolor{currentstroke}{rgb}{0.000000,0.000000,0.000000}%
\pgfsetstrokecolor{currentstroke}%
\pgfsetstrokeopacity{0.000000}%
\pgfsetdash{}{0pt}%
\pgfpathmoveto{\pgfqpoint{3.104158in}{1.232425in}}%
\pgfpathlineto{\pgfqpoint{3.101354in}{1.230845in}}%
\pgfpathlineto{\pgfqpoint{3.083997in}{1.206525in}}%
\pgfpathlineto{\pgfqpoint{3.084398in}{1.198942in}}%
\pgfpathlineto{\pgfqpoint{3.058776in}{1.194345in}}%
\pgfpathlineto{\pgfqpoint{3.051661in}{1.204962in}}%
\pgfpathlineto{\pgfqpoint{3.076257in}{1.227944in}}%
\pgfpathlineto{\pgfqpoint{3.072877in}{1.235309in}}%
\pgfpathlineto{\pgfqpoint{3.061888in}{1.238029in}}%
\pgfpathlineto{\pgfqpoint{3.070656in}{1.250066in}}%
\pgfpathlineto{\pgfqpoint{3.076256in}{1.249249in}}%
\pgfpathlineto{\pgfqpoint{3.072468in}{1.264155in}}%
\pgfpathlineto{\pgfqpoint{3.085338in}{1.268870in}}%
\pgfpathlineto{\pgfqpoint{3.083942in}{1.279885in}}%
\pgfpathlineto{\pgfqpoint{3.076423in}{1.287943in}}%
\pgfpathlineto{\pgfqpoint{3.080491in}{1.292581in}}%
\pgfpathlineto{\pgfqpoint{3.085810in}{1.290511in}}%
\pgfpathlineto{\pgfqpoint{3.090969in}{1.283373in}}%
\pgfpathlineto{\pgfqpoint{3.097597in}{1.283948in}}%
\pgfpathlineto{\pgfqpoint{3.102686in}{1.276519in}}%
\pgfpathlineto{\pgfqpoint{3.108118in}{1.275965in}}%
\pgfpathlineto{\pgfqpoint{3.111039in}{1.268233in}}%
\pgfpathlineto{\pgfqpoint{3.108386in}{1.263758in}}%
\pgfpathlineto{\pgfqpoint{3.103460in}{1.266880in}}%
\pgfpathlineto{\pgfqpoint{3.105581in}{1.253008in}}%
\pgfpathlineto{\pgfqpoint{3.114175in}{1.246467in}}%
\pgfpathlineto{\pgfqpoint{3.113069in}{1.239724in}}%
\pgfpathlineto{\pgfqpoint{3.109067in}{1.238497in}}%
\pgfpathclose%
\pgfusepath{fill}%
\end{pgfscope}%
\begin{pgfscope}%
\pgfpathrectangle{\pgfqpoint{0.100000in}{0.100000in}}{\pgfqpoint{3.420221in}{2.189500in}}%
\pgfusepath{clip}%
\pgfsetbuttcap%
\pgfsetmiterjoin%
\definecolor{currentfill}{rgb}{0.000000,0.537255,0.731373}%
\pgfsetfillcolor{currentfill}%
\pgfsetlinewidth{0.000000pt}%
\definecolor{currentstroke}{rgb}{0.000000,0.000000,0.000000}%
\pgfsetstrokecolor{currentstroke}%
\pgfsetstrokeopacity{0.000000}%
\pgfsetdash{}{0pt}%
\pgfpathmoveto{\pgfqpoint{2.700892in}{0.635976in}}%
\pgfpathlineto{\pgfqpoint{2.700986in}{0.631136in}}%
\pgfpathlineto{\pgfqpoint{2.693119in}{0.632449in}}%
\pgfpathlineto{\pgfqpoint{2.684876in}{0.627980in}}%
\pgfpathlineto{\pgfqpoint{2.668495in}{0.615680in}}%
\pgfpathlineto{\pgfqpoint{2.658384in}{0.612210in}}%
\pgfpathlineto{\pgfqpoint{2.652793in}{0.611995in}}%
\pgfpathlineto{\pgfqpoint{2.643237in}{0.608045in}}%
\pgfpathlineto{\pgfqpoint{2.639738in}{0.609432in}}%
\pgfpathlineto{\pgfqpoint{2.639273in}{0.617408in}}%
\pgfpathlineto{\pgfqpoint{2.630193in}{0.627182in}}%
\pgfpathlineto{\pgfqpoint{2.625569in}{0.627762in}}%
\pgfpathlineto{\pgfqpoint{2.618834in}{0.633459in}}%
\pgfpathlineto{\pgfqpoint{2.604870in}{0.640911in}}%
\pgfpathlineto{\pgfqpoint{2.590388in}{0.647524in}}%
\pgfpathlineto{\pgfqpoint{2.589728in}{0.656602in}}%
\pgfpathlineto{\pgfqpoint{2.598374in}{0.661370in}}%
\pgfpathlineto{\pgfqpoint{2.622170in}{0.663536in}}%
\pgfpathlineto{\pgfqpoint{2.621248in}{0.673377in}}%
\pgfpathlineto{\pgfqpoint{2.627862in}{0.674027in}}%
\pgfpathlineto{\pgfqpoint{2.630428in}{0.646459in}}%
\pgfpathlineto{\pgfqpoint{2.647191in}{0.646691in}}%
\pgfpathlineto{\pgfqpoint{2.648269in}{0.636349in}}%
\pgfpathlineto{\pgfqpoint{2.656858in}{0.635025in}}%
\pgfpathlineto{\pgfqpoint{2.687095in}{0.638220in}}%
\pgfpathclose%
\pgfusepath{fill}%
\end{pgfscope}%
\begin{pgfscope}%
\pgfpathrectangle{\pgfqpoint{0.100000in}{0.100000in}}{\pgfqpoint{3.420221in}{2.189500in}}%
\pgfusepath{clip}%
\pgfsetbuttcap%
\pgfsetmiterjoin%
\definecolor{currentfill}{rgb}{0.000000,0.423529,0.788235}%
\pgfsetfillcolor{currentfill}%
\pgfsetlinewidth{0.000000pt}%
\definecolor{currentstroke}{rgb}{0.000000,0.000000,0.000000}%
\pgfsetstrokecolor{currentstroke}%
\pgfsetstrokeopacity{0.000000}%
\pgfsetdash{}{0pt}%
\pgfpathmoveto{\pgfqpoint{2.919091in}{1.191996in}}%
\pgfpathlineto{\pgfqpoint{2.918648in}{1.168023in}}%
\pgfpathlineto{\pgfqpoint{2.900079in}{1.164945in}}%
\pgfpathlineto{\pgfqpoint{2.884132in}{1.162430in}}%
\pgfpathlineto{\pgfqpoint{2.865120in}{1.160414in}}%
\pgfpathlineto{\pgfqpoint{2.863805in}{1.164921in}}%
\pgfpathlineto{\pgfqpoint{2.872435in}{1.170339in}}%
\pgfpathlineto{\pgfqpoint{2.872118in}{1.172995in}}%
\pgfpathlineto{\pgfqpoint{2.879194in}{1.184393in}}%
\pgfpathlineto{\pgfqpoint{2.883706in}{1.187293in}}%
\pgfpathlineto{\pgfqpoint{2.912778in}{1.185767in}}%
\pgfpathclose%
\pgfusepath{fill}%
\end{pgfscope}%
\begin{pgfscope}%
\pgfpathrectangle{\pgfqpoint{0.100000in}{0.100000in}}{\pgfqpoint{3.420221in}{2.189500in}}%
\pgfusepath{clip}%
\pgfsetbuttcap%
\pgfsetmiterjoin%
\definecolor{currentfill}{rgb}{0.000000,0.568627,0.715686}%
\pgfsetfillcolor{currentfill}%
\pgfsetlinewidth{0.000000pt}%
\definecolor{currentstroke}{rgb}{0.000000,0.000000,0.000000}%
\pgfsetstrokecolor{currentstroke}%
\pgfsetstrokeopacity{0.000000}%
\pgfsetdash{}{0pt}%
\pgfpathmoveto{\pgfqpoint{1.449555in}{0.972610in}}%
\pgfpathlineto{\pgfqpoint{1.445874in}{0.933525in}}%
\pgfpathlineto{\pgfqpoint{1.432871in}{0.934865in}}%
\pgfpathlineto{\pgfqpoint{1.432180in}{0.928290in}}%
\pgfpathlineto{\pgfqpoint{1.393168in}{0.932722in}}%
\pgfpathlineto{\pgfqpoint{1.393860in}{0.939282in}}%
\pgfpathlineto{\pgfqpoint{1.387246in}{0.940041in}}%
\pgfpathlineto{\pgfqpoint{1.389439in}{0.959366in}}%
\pgfpathlineto{\pgfqpoint{1.363463in}{0.962245in}}%
\pgfpathlineto{\pgfqpoint{1.365700in}{0.981682in}}%
\pgfpathlineto{\pgfqpoint{1.367087in}{0.981553in}}%
\pgfpathlineto{\pgfqpoint{1.370729in}{1.014003in}}%
\pgfpathlineto{\pgfqpoint{1.372180in}{1.027007in}}%
\pgfpathlineto{\pgfqpoint{1.443279in}{1.019332in}}%
\pgfpathlineto{\pgfqpoint{1.442768in}{1.013837in}}%
\pgfpathlineto{\pgfqpoint{1.438476in}{0.973730in}}%
\pgfpathclose%
\pgfusepath{fill}%
\end{pgfscope}%
\begin{pgfscope}%
\pgfpathrectangle{\pgfqpoint{0.100000in}{0.100000in}}{\pgfqpoint{3.420221in}{2.189500in}}%
\pgfusepath{clip}%
\pgfsetbuttcap%
\pgfsetmiterjoin%
\definecolor{currentfill}{rgb}{0.000000,0.172549,0.913725}%
\pgfsetfillcolor{currentfill}%
\pgfsetlinewidth{0.000000pt}%
\definecolor{currentstroke}{rgb}{0.000000,0.000000,0.000000}%
\pgfsetstrokecolor{currentstroke}%
\pgfsetstrokeopacity{0.000000}%
\pgfsetdash{}{0pt}%
\pgfpathmoveto{\pgfqpoint{3.351842in}{1.645176in}}%
\pgfpathlineto{\pgfqpoint{3.350323in}{1.645474in}}%
\pgfpathlineto{\pgfqpoint{3.352227in}{1.650073in}}%
\pgfpathlineto{\pgfqpoint{3.348102in}{1.659840in}}%
\pgfpathlineto{\pgfqpoint{3.340283in}{1.650836in}}%
\pgfpathlineto{\pgfqpoint{3.335318in}{1.659744in}}%
\pgfpathlineto{\pgfqpoint{3.322521in}{1.672082in}}%
\pgfpathlineto{\pgfqpoint{3.318313in}{1.681224in}}%
\pgfpathlineto{\pgfqpoint{3.305174in}{1.669132in}}%
\pgfpathlineto{\pgfqpoint{3.303598in}{1.671313in}}%
\pgfpathlineto{\pgfqpoint{3.281284in}{1.664602in}}%
\pgfpathlineto{\pgfqpoint{3.262715in}{1.661511in}}%
\pgfpathlineto{\pgfqpoint{3.260308in}{1.671078in}}%
\pgfpathlineto{\pgfqpoint{3.252943in}{1.670906in}}%
\pgfpathlineto{\pgfqpoint{3.254217in}{1.676237in}}%
\pgfpathlineto{\pgfqpoint{3.247251in}{1.681816in}}%
\pgfpathlineto{\pgfqpoint{3.244928in}{1.699331in}}%
\pgfpathlineto{\pgfqpoint{3.245880in}{1.706203in}}%
\pgfpathlineto{\pgfqpoint{3.241932in}{1.709740in}}%
\pgfpathlineto{\pgfqpoint{3.261023in}{1.713932in}}%
\pgfpathlineto{\pgfqpoint{3.295169in}{1.721735in}}%
\pgfpathlineto{\pgfqpoint{3.301361in}{1.733063in}}%
\pgfpathlineto{\pgfqpoint{3.310843in}{1.740538in}}%
\pgfpathlineto{\pgfqpoint{3.317116in}{1.741270in}}%
\pgfpathlineto{\pgfqpoint{3.324024in}{1.728606in}}%
\pgfpathlineto{\pgfqpoint{3.333429in}{1.727256in}}%
\pgfpathlineto{\pgfqpoint{3.329829in}{1.722627in}}%
\pgfpathlineto{\pgfqpoint{3.321265in}{1.716857in}}%
\pgfpathlineto{\pgfqpoint{3.323034in}{1.714610in}}%
\pgfpathlineto{\pgfqpoint{3.317003in}{1.705092in}}%
\pgfpathlineto{\pgfqpoint{3.318961in}{1.698333in}}%
\pgfpathlineto{\pgfqpoint{3.324751in}{1.699319in}}%
\pgfpathlineto{\pgfqpoint{3.335773in}{1.694482in}}%
\pgfpathlineto{\pgfqpoint{3.342039in}{1.687269in}}%
\pgfpathlineto{\pgfqpoint{3.340945in}{1.678938in}}%
\pgfpathlineto{\pgfqpoint{3.350913in}{1.676492in}}%
\pgfpathlineto{\pgfqpoint{3.352829in}{1.668857in}}%
\pgfpathlineto{\pgfqpoint{3.360610in}{1.665557in}}%
\pgfpathlineto{\pgfqpoint{3.367913in}{1.666716in}}%
\pgfpathlineto{\pgfqpoint{3.385937in}{1.676533in}}%
\pgfpathlineto{\pgfqpoint{3.388496in}{1.667877in}}%
\pgfpathlineto{\pgfqpoint{3.362658in}{1.655212in}}%
\pgfpathlineto{\pgfqpoint{3.361419in}{1.651147in}}%
\pgfpathclose%
\pgfusepath{fill}%
\end{pgfscope}%
\begin{pgfscope}%
\pgfpathrectangle{\pgfqpoint{0.100000in}{0.100000in}}{\pgfqpoint{3.420221in}{2.189500in}}%
\pgfusepath{clip}%
\pgfsetbuttcap%
\pgfsetmiterjoin%
\definecolor{currentfill}{rgb}{0.000000,0.768627,0.615686}%
\pgfsetfillcolor{currentfill}%
\pgfsetlinewidth{0.000000pt}%
\definecolor{currentstroke}{rgb}{0.000000,0.000000,0.000000}%
\pgfsetstrokecolor{currentstroke}%
\pgfsetstrokeopacity{0.000000}%
\pgfsetdash{}{0pt}%
\pgfpathmoveto{\pgfqpoint{0.704136in}{0.621564in}}%
\pgfpathlineto{\pgfqpoint{0.705494in}{0.624888in}}%
\pgfpathlineto{\pgfqpoint{0.705557in}{0.628240in}}%
\pgfpathlineto{\pgfqpoint{0.704395in}{0.633133in}}%
\pgfpathlineto{\pgfqpoint{0.705115in}{0.636119in}}%
\pgfpathlineto{\pgfqpoint{0.703526in}{0.642127in}}%
\pgfpathlineto{\pgfqpoint{0.701690in}{0.644631in}}%
\pgfpathlineto{\pgfqpoint{0.696200in}{0.645730in}}%
\pgfpathlineto{\pgfqpoint{0.695534in}{0.647945in}}%
\pgfpathlineto{\pgfqpoint{0.696061in}{0.651965in}}%
\pgfpathlineto{\pgfqpoint{0.698150in}{0.654131in}}%
\pgfpathlineto{\pgfqpoint{0.700859in}{0.654982in}}%
\pgfpathlineto{\pgfqpoint{0.703228in}{0.656816in}}%
\pgfpathlineto{\pgfqpoint{0.704024in}{0.655891in}}%
\pgfpathlineto{\pgfqpoint{0.703033in}{0.653316in}}%
\pgfpathlineto{\pgfqpoint{0.705631in}{0.646295in}}%
\pgfpathlineto{\pgfqpoint{0.707342in}{0.645453in}}%
\pgfpathlineto{\pgfqpoint{0.712900in}{0.644020in}}%
\pgfpathlineto{\pgfqpoint{0.714162in}{0.641324in}}%
\pgfpathlineto{\pgfqpoint{0.714062in}{0.638401in}}%
\pgfpathlineto{\pgfqpoint{0.712138in}{0.636227in}}%
\pgfpathlineto{\pgfqpoint{0.713509in}{0.632029in}}%
\pgfpathlineto{\pgfqpoint{0.713701in}{0.628979in}}%
\pgfpathlineto{\pgfqpoint{0.715690in}{0.625008in}}%
\pgfpathlineto{\pgfqpoint{0.718235in}{0.623319in}}%
\pgfpathlineto{\pgfqpoint{0.719687in}{0.620401in}}%
\pgfpathlineto{\pgfqpoint{0.718778in}{0.619076in}}%
\pgfpathlineto{\pgfqpoint{0.715715in}{0.618211in}}%
\pgfpathlineto{\pgfqpoint{0.714209in}{0.620716in}}%
\pgfpathlineto{\pgfqpoint{0.711879in}{0.622771in}}%
\pgfpathlineto{\pgfqpoint{0.708881in}{0.623299in}}%
\pgfpathclose%
\pgfusepath{fill}%
\end{pgfscope}%
\begin{pgfscope}%
\pgfpathrectangle{\pgfqpoint{0.100000in}{0.100000in}}{\pgfqpoint{3.420221in}{2.189500in}}%
\pgfusepath{clip}%
\pgfsetbuttcap%
\pgfsetmiterjoin%
\definecolor{currentfill}{rgb}{0.000000,0.768627,0.615686}%
\pgfsetfillcolor{currentfill}%
\pgfsetlinewidth{0.000000pt}%
\definecolor{currentstroke}{rgb}{0.000000,0.000000,0.000000}%
\pgfsetstrokecolor{currentstroke}%
\pgfsetstrokeopacity{0.000000}%
\pgfsetdash{}{0pt}%
\pgfpathmoveto{\pgfqpoint{0.824092in}{0.649459in}}%
\pgfpathlineto{\pgfqpoint{0.822648in}{0.646562in}}%
\pgfpathlineto{\pgfqpoint{0.815661in}{0.639219in}}%
\pgfpathlineto{\pgfqpoint{0.816261in}{0.638589in}}%
\pgfpathlineto{\pgfqpoint{0.809247in}{0.631278in}}%
\pgfpathlineto{\pgfqpoint{0.811646in}{0.629015in}}%
\pgfpathlineto{\pgfqpoint{0.808145in}{0.625326in}}%
\pgfpathlineto{\pgfqpoint{0.809992in}{0.623567in}}%
\pgfpathlineto{\pgfqpoint{0.806570in}{0.619928in}}%
\pgfpathlineto{\pgfqpoint{0.814302in}{0.612722in}}%
\pgfpathlineto{\pgfqpoint{0.830629in}{0.597951in}}%
\pgfpathlineto{\pgfqpoint{0.841972in}{0.587921in}}%
\pgfpathlineto{\pgfqpoint{0.843653in}{0.589802in}}%
\pgfpathlineto{\pgfqpoint{0.845545in}{0.588170in}}%
\pgfpathlineto{\pgfqpoint{0.847443in}{0.586556in}}%
\pgfpathlineto{\pgfqpoint{0.842523in}{0.580778in}}%
\pgfpathlineto{\pgfqpoint{0.840759in}{0.582332in}}%
\pgfpathlineto{\pgfqpoint{0.834379in}{0.574888in}}%
\pgfpathlineto{\pgfqpoint{0.832511in}{0.576474in}}%
\pgfpathlineto{\pgfqpoint{0.829283in}{0.572676in}}%
\pgfpathlineto{\pgfqpoint{0.827352in}{0.574281in}}%
\pgfpathlineto{\pgfqpoint{0.814431in}{0.558988in}}%
\pgfpathlineto{\pgfqpoint{0.816586in}{0.557173in}}%
\pgfpathlineto{\pgfqpoint{0.811532in}{0.551759in}}%
\pgfpathlineto{\pgfqpoint{0.809861in}{0.553230in}}%
\pgfpathlineto{\pgfqpoint{0.808189in}{0.551273in}}%
\pgfpathlineto{\pgfqpoint{0.806504in}{0.552690in}}%
\pgfpathlineto{\pgfqpoint{0.804917in}{0.550755in}}%
\pgfpathlineto{\pgfqpoint{0.801137in}{0.554034in}}%
\pgfpathlineto{\pgfqpoint{0.799686in}{0.552410in}}%
\pgfpathlineto{\pgfqpoint{0.797801in}{0.554049in}}%
\pgfpathlineto{\pgfqpoint{0.793006in}{0.548213in}}%
\pgfpathlineto{\pgfqpoint{0.791293in}{0.549666in}}%
\pgfpathlineto{\pgfqpoint{0.786440in}{0.544258in}}%
\pgfpathlineto{\pgfqpoint{0.775547in}{0.554024in}}%
\pgfpathlineto{\pgfqpoint{0.777020in}{0.555870in}}%
\pgfpathlineto{\pgfqpoint{0.775372in}{0.557388in}}%
\pgfpathlineto{\pgfqpoint{0.777022in}{0.559333in}}%
\pgfpathlineto{\pgfqpoint{0.775234in}{0.560965in}}%
\pgfpathlineto{\pgfqpoint{0.776855in}{0.562805in}}%
\pgfpathlineto{\pgfqpoint{0.774936in}{0.564678in}}%
\pgfpathlineto{\pgfqpoint{0.781232in}{0.566580in}}%
\pgfpathlineto{\pgfqpoint{0.781376in}{0.568981in}}%
\pgfpathlineto{\pgfqpoint{0.779459in}{0.570577in}}%
\pgfpathlineto{\pgfqpoint{0.781058in}{0.572225in}}%
\pgfpathlineto{\pgfqpoint{0.783931in}{0.570521in}}%
\pgfpathlineto{\pgfqpoint{0.783388in}{0.566386in}}%
\pgfpathlineto{\pgfqpoint{0.785354in}{0.561464in}}%
\pgfpathlineto{\pgfqpoint{0.791201in}{0.557661in}}%
\pgfpathlineto{\pgfqpoint{0.793745in}{0.557472in}}%
\pgfpathlineto{\pgfqpoint{0.797878in}{0.558365in}}%
\pgfpathlineto{\pgfqpoint{0.801254in}{0.559640in}}%
\pgfpathlineto{\pgfqpoint{0.802651in}{0.560935in}}%
\pgfpathlineto{\pgfqpoint{0.805570in}{0.567631in}}%
\pgfpathlineto{\pgfqpoint{0.808346in}{0.571393in}}%
\pgfpathlineto{\pgfqpoint{0.809103in}{0.577354in}}%
\pgfpathlineto{\pgfqpoint{0.807028in}{0.579622in}}%
\pgfpathlineto{\pgfqpoint{0.809627in}{0.581125in}}%
\pgfpathlineto{\pgfqpoint{0.812359in}{0.577660in}}%
\pgfpathlineto{\pgfqpoint{0.817150in}{0.577989in}}%
\pgfpathlineto{\pgfqpoint{0.819323in}{0.580162in}}%
\pgfpathlineto{\pgfqpoint{0.820696in}{0.582807in}}%
\pgfpathlineto{\pgfqpoint{0.819412in}{0.587907in}}%
\pgfpathlineto{\pgfqpoint{0.815397in}{0.586524in}}%
\pgfpathlineto{\pgfqpoint{0.812843in}{0.586712in}}%
\pgfpathlineto{\pgfqpoint{0.811828in}{0.589262in}}%
\pgfpathlineto{\pgfqpoint{0.808353in}{0.589502in}}%
\pgfpathlineto{\pgfqpoint{0.804847in}{0.590460in}}%
\pgfpathlineto{\pgfqpoint{0.802009in}{0.590366in}}%
\pgfpathlineto{\pgfqpoint{0.793439in}{0.588334in}}%
\pgfpathlineto{\pgfqpoint{0.795567in}{0.592129in}}%
\pgfpathlineto{\pgfqpoint{0.796084in}{0.597675in}}%
\pgfpathlineto{\pgfqpoint{0.793299in}{0.598031in}}%
\pgfpathlineto{\pgfqpoint{0.793702in}{0.595363in}}%
\pgfpathlineto{\pgfqpoint{0.791615in}{0.594588in}}%
\pgfpathlineto{\pgfqpoint{0.791660in}{0.597407in}}%
\pgfpathlineto{\pgfqpoint{0.790616in}{0.600358in}}%
\pgfpathlineto{\pgfqpoint{0.784711in}{0.606083in}}%
\pgfpathlineto{\pgfqpoint{0.778018in}{0.608186in}}%
\pgfpathlineto{\pgfqpoint{0.776266in}{0.609114in}}%
\pgfpathlineto{\pgfqpoint{0.768969in}{0.621199in}}%
\pgfpathlineto{\pgfqpoint{0.767633in}{0.623966in}}%
\pgfpathlineto{\pgfqpoint{0.767595in}{0.626360in}}%
\pgfpathlineto{\pgfqpoint{0.770072in}{0.629723in}}%
\pgfpathlineto{\pgfqpoint{0.772295in}{0.631540in}}%
\pgfpathlineto{\pgfqpoint{0.772679in}{0.639175in}}%
\pgfpathlineto{\pgfqpoint{0.776421in}{0.637478in}}%
\pgfpathlineto{\pgfqpoint{0.781087in}{0.639537in}}%
\pgfpathlineto{\pgfqpoint{0.776546in}{0.647356in}}%
\pgfpathlineto{\pgfqpoint{0.774626in}{0.649844in}}%
\pgfpathlineto{\pgfqpoint{0.773531in}{0.655849in}}%
\pgfpathlineto{\pgfqpoint{0.772359in}{0.659728in}}%
\pgfpathlineto{\pgfqpoint{0.774724in}{0.662089in}}%
\pgfpathlineto{\pgfqpoint{0.785128in}{0.659125in}}%
\pgfpathlineto{\pgfqpoint{0.785785in}{0.657233in}}%
\pgfpathlineto{\pgfqpoint{0.789599in}{0.656580in}}%
\pgfpathlineto{\pgfqpoint{0.789461in}{0.658599in}}%
\pgfpathlineto{\pgfqpoint{0.799616in}{0.657177in}}%
\pgfpathlineto{\pgfqpoint{0.802105in}{0.651829in}}%
\pgfpathlineto{\pgfqpoint{0.803871in}{0.650065in}}%
\pgfpathlineto{\pgfqpoint{0.805926in}{0.650320in}}%
\pgfpathlineto{\pgfqpoint{0.804660in}{0.653018in}}%
\pgfpathlineto{\pgfqpoint{0.807864in}{0.655527in}}%
\pgfpathlineto{\pgfqpoint{0.816048in}{0.652989in}}%
\pgfpathlineto{\pgfqpoint{0.821240in}{0.651050in}}%
\pgfpathclose%
\pgfusepath{fill}%
\end{pgfscope}%
\begin{pgfscope}%
\pgfpathrectangle{\pgfqpoint{0.100000in}{0.100000in}}{\pgfqpoint{3.420221in}{2.189500in}}%
\pgfusepath{clip}%
\pgfsetbuttcap%
\pgfsetmiterjoin%
\definecolor{currentfill}{rgb}{0.000000,0.458824,0.770588}%
\pgfsetfillcolor{currentfill}%
\pgfsetlinewidth{0.000000pt}%
\definecolor{currentstroke}{rgb}{0.000000,0.000000,0.000000}%
\pgfsetstrokecolor{currentstroke}%
\pgfsetstrokeopacity{0.000000}%
\pgfsetdash{}{0pt}%
\pgfpathmoveto{\pgfqpoint{1.716764in}{1.582932in}}%
\pgfpathlineto{\pgfqpoint{1.660333in}{1.586435in}}%
\pgfpathlineto{\pgfqpoint{1.662326in}{1.615766in}}%
\pgfpathlineto{\pgfqpoint{1.664443in}{1.645920in}}%
\pgfpathlineto{\pgfqpoint{1.673547in}{1.649089in}}%
\pgfpathlineto{\pgfqpoint{1.681293in}{1.647090in}}%
\pgfpathlineto{\pgfqpoint{1.688012in}{1.639753in}}%
\pgfpathlineto{\pgfqpoint{1.699733in}{1.640707in}}%
\pgfpathlineto{\pgfqpoint{1.704927in}{1.637977in}}%
\pgfpathlineto{\pgfqpoint{1.712227in}{1.637261in}}%
\pgfpathlineto{\pgfqpoint{1.718078in}{1.636688in}}%
\pgfpathlineto{\pgfqpoint{1.717951in}{1.618928in}}%
\pgfpathlineto{\pgfqpoint{1.716481in}{1.592728in}}%
\pgfpathclose%
\pgfusepath{fill}%
\end{pgfscope}%
\begin{pgfscope}%
\pgfpathrectangle{\pgfqpoint{0.100000in}{0.100000in}}{\pgfqpoint{3.420221in}{2.189500in}}%
\pgfusepath{clip}%
\pgfsetbuttcap%
\pgfsetmiterjoin%
\definecolor{currentfill}{rgb}{0.000000,0.627451,0.686275}%
\pgfsetfillcolor{currentfill}%
\pgfsetlinewidth{0.000000pt}%
\definecolor{currentstroke}{rgb}{0.000000,0.000000,0.000000}%
\pgfsetstrokecolor{currentstroke}%
\pgfsetstrokeopacity{0.000000}%
\pgfsetdash{}{0pt}%
\pgfpathmoveto{\pgfqpoint{0.854302in}{1.869474in}}%
\pgfpathlineto{\pgfqpoint{0.849982in}{1.866155in}}%
\pgfpathlineto{\pgfqpoint{0.849145in}{1.860003in}}%
\pgfpathlineto{\pgfqpoint{0.844033in}{1.853501in}}%
\pgfpathlineto{\pgfqpoint{0.835179in}{1.848515in}}%
\pgfpathlineto{\pgfqpoint{0.826324in}{1.836783in}}%
\pgfpathlineto{\pgfqpoint{0.820510in}{1.831143in}}%
\pgfpathlineto{\pgfqpoint{0.817604in}{1.818231in}}%
\pgfpathlineto{\pgfqpoint{0.803709in}{1.821722in}}%
\pgfpathlineto{\pgfqpoint{0.805313in}{1.828013in}}%
\pgfpathlineto{\pgfqpoint{0.800885in}{1.833540in}}%
\pgfpathlineto{\pgfqpoint{0.781055in}{1.838529in}}%
\pgfpathlineto{\pgfqpoint{0.777411in}{1.837284in}}%
\pgfpathlineto{\pgfqpoint{0.768366in}{1.828401in}}%
\pgfpathlineto{\pgfqpoint{0.763945in}{1.828419in}}%
\pgfpathlineto{\pgfqpoint{0.750110in}{1.831978in}}%
\pgfpathlineto{\pgfqpoint{0.756738in}{1.839434in}}%
\pgfpathlineto{\pgfqpoint{0.757735in}{1.845008in}}%
\pgfpathlineto{\pgfqpoint{0.766402in}{1.853531in}}%
\pgfpathlineto{\pgfqpoint{0.765320in}{1.857286in}}%
\pgfpathlineto{\pgfqpoint{0.757638in}{1.861950in}}%
\pgfpathlineto{\pgfqpoint{0.757712in}{1.864804in}}%
\pgfpathlineto{\pgfqpoint{0.770612in}{1.865160in}}%
\pgfpathlineto{\pgfqpoint{0.770503in}{1.871844in}}%
\pgfpathlineto{\pgfqpoint{0.776125in}{1.873424in}}%
\pgfpathlineto{\pgfqpoint{0.777344in}{1.877877in}}%
\pgfpathlineto{\pgfqpoint{0.771217in}{1.883742in}}%
\pgfpathlineto{\pgfqpoint{0.763248in}{1.886195in}}%
\pgfpathlineto{\pgfqpoint{0.765706in}{1.901210in}}%
\pgfpathlineto{\pgfqpoint{0.760106in}{1.902690in}}%
\pgfpathlineto{\pgfqpoint{0.760837in}{1.914063in}}%
\pgfpathlineto{\pgfqpoint{0.774825in}{1.911133in}}%
\pgfpathlineto{\pgfqpoint{0.776233in}{1.916515in}}%
\pgfpathlineto{\pgfqpoint{0.787995in}{1.913290in}}%
\pgfpathlineto{\pgfqpoint{0.792965in}{1.915392in}}%
\pgfpathlineto{\pgfqpoint{0.797093in}{1.931286in}}%
\pgfpathlineto{\pgfqpoint{0.799657in}{1.930610in}}%
\pgfpathlineto{\pgfqpoint{0.803206in}{1.939723in}}%
\pgfpathlineto{\pgfqpoint{0.807697in}{1.941951in}}%
\pgfpathlineto{\pgfqpoint{0.819187in}{1.939030in}}%
\pgfpathlineto{\pgfqpoint{0.813924in}{1.927026in}}%
\pgfpathlineto{\pgfqpoint{0.815662in}{1.920952in}}%
\pgfpathlineto{\pgfqpoint{0.813742in}{1.913459in}}%
\pgfpathlineto{\pgfqpoint{0.814718in}{1.899400in}}%
\pgfpathlineto{\pgfqpoint{0.819693in}{1.892738in}}%
\pgfpathlineto{\pgfqpoint{0.820112in}{1.884668in}}%
\pgfpathlineto{\pgfqpoint{0.830933in}{1.881991in}}%
\pgfpathlineto{\pgfqpoint{0.829409in}{1.875766in}}%
\pgfpathclose%
\pgfusepath{fill}%
\end{pgfscope}%
\begin{pgfscope}%
\pgfpathrectangle{\pgfqpoint{0.100000in}{0.100000in}}{\pgfqpoint{3.420221in}{2.189500in}}%
\pgfusepath{clip}%
\pgfsetbuttcap%
\pgfsetmiterjoin%
\definecolor{currentfill}{rgb}{0.000000,0.278431,0.860784}%
\pgfsetfillcolor{currentfill}%
\pgfsetlinewidth{0.000000pt}%
\definecolor{currentstroke}{rgb}{0.000000,0.000000,0.000000}%
\pgfsetstrokecolor{currentstroke}%
\pgfsetstrokeopacity{0.000000}%
\pgfsetdash{}{0pt}%
\pgfpathmoveto{\pgfqpoint{1.703121in}{1.293354in}}%
\pgfpathlineto{\pgfqpoint{1.700925in}{1.260846in}}%
\pgfpathlineto{\pgfqpoint{1.669769in}{1.262911in}}%
\pgfpathlineto{\pgfqpoint{1.623298in}{1.266030in}}%
\pgfpathlineto{\pgfqpoint{1.625992in}{1.298488in}}%
\pgfpathlineto{\pgfqpoint{1.631005in}{1.298160in}}%
\pgfpathlineto{\pgfqpoint{1.633440in}{1.330519in}}%
\pgfpathlineto{\pgfqpoint{1.703097in}{1.325976in}}%
\pgfpathlineto{\pgfqpoint{1.704176in}{1.325944in}}%
\pgfpathlineto{\pgfqpoint{1.702166in}{1.293388in}}%
\pgfpathclose%
\pgfusepath{fill}%
\end{pgfscope}%
\begin{pgfscope}%
\pgfpathrectangle{\pgfqpoint{0.100000in}{0.100000in}}{\pgfqpoint{3.420221in}{2.189500in}}%
\pgfusepath{clip}%
\pgfsetbuttcap%
\pgfsetmiterjoin%
\definecolor{currentfill}{rgb}{0.000000,0.415686,0.792157}%
\pgfsetfillcolor{currentfill}%
\pgfsetlinewidth{0.000000pt}%
\definecolor{currentstroke}{rgb}{0.000000,0.000000,0.000000}%
\pgfsetstrokecolor{currentstroke}%
\pgfsetstrokeopacity{0.000000}%
\pgfsetdash{}{0pt}%
\pgfpathmoveto{\pgfqpoint{2.341370in}{1.312070in}}%
\pgfpathlineto{\pgfqpoint{2.318753in}{1.310639in}}%
\pgfpathlineto{\pgfqpoint{2.317794in}{1.323777in}}%
\pgfpathlineto{\pgfqpoint{2.295169in}{1.322281in}}%
\pgfpathlineto{\pgfqpoint{2.290184in}{1.341751in}}%
\pgfpathlineto{\pgfqpoint{2.288796in}{1.365916in}}%
\pgfpathlineto{\pgfqpoint{2.290038in}{1.368516in}}%
\pgfpathlineto{\pgfqpoint{2.297925in}{1.367631in}}%
\pgfpathlineto{\pgfqpoint{2.306264in}{1.370307in}}%
\pgfpathlineto{\pgfqpoint{2.311224in}{1.368257in}}%
\pgfpathlineto{\pgfqpoint{2.310271in}{1.383031in}}%
\pgfpathlineto{\pgfqpoint{2.329581in}{1.384603in}}%
\pgfpathlineto{\pgfqpoint{2.336350in}{1.381796in}}%
\pgfpathlineto{\pgfqpoint{2.337710in}{1.364365in}}%
\pgfpathlineto{\pgfqpoint{2.338419in}{1.354591in}}%
\pgfpathlineto{\pgfqpoint{2.334151in}{1.354252in}}%
\pgfpathlineto{\pgfqpoint{2.334661in}{1.346489in}}%
\pgfpathlineto{\pgfqpoint{2.339069in}{1.345836in}}%
\pgfpathlineto{\pgfqpoint{2.339929in}{1.334967in}}%
\pgfpathlineto{\pgfqpoint{2.346460in}{1.335309in}}%
\pgfpathlineto{\pgfqpoint{2.347993in}{1.312284in}}%
\pgfpathclose%
\pgfusepath{fill}%
\end{pgfscope}%
\begin{pgfscope}%
\pgfpathrectangle{\pgfqpoint{0.100000in}{0.100000in}}{\pgfqpoint{3.420221in}{2.189500in}}%
\pgfusepath{clip}%
\pgfsetbuttcap%
\pgfsetmiterjoin%
\definecolor{currentfill}{rgb}{0.000000,0.364706,0.817647}%
\pgfsetfillcolor{currentfill}%
\pgfsetlinewidth{0.000000pt}%
\definecolor{currentstroke}{rgb}{0.000000,0.000000,0.000000}%
\pgfsetstrokecolor{currentstroke}%
\pgfsetstrokeopacity{0.000000}%
\pgfsetdash{}{0pt}%
\pgfpathmoveto{\pgfqpoint{3.174987in}{1.585419in}}%
\pgfpathlineto{\pgfqpoint{3.164200in}{1.568896in}}%
\pgfpathlineto{\pgfqpoint{3.135304in}{1.578757in}}%
\pgfpathlineto{\pgfqpoint{3.132035in}{1.583034in}}%
\pgfpathlineto{\pgfqpoint{3.126285in}{1.582420in}}%
\pgfpathlineto{\pgfqpoint{3.116755in}{1.586204in}}%
\pgfpathlineto{\pgfqpoint{3.111826in}{1.593853in}}%
\pgfpathlineto{\pgfqpoint{3.109194in}{1.603181in}}%
\pgfpathlineto{\pgfqpoint{3.102490in}{1.609314in}}%
\pgfpathlineto{\pgfqpoint{3.134996in}{1.641143in}}%
\pgfpathlineto{\pgfqpoint{3.143734in}{1.638935in}}%
\pgfpathlineto{\pgfqpoint{3.156654in}{1.640578in}}%
\pgfpathlineto{\pgfqpoint{3.157145in}{1.646380in}}%
\pgfpathlineto{\pgfqpoint{3.164573in}{1.644933in}}%
\pgfpathlineto{\pgfqpoint{3.166263in}{1.641419in}}%
\pgfpathlineto{\pgfqpoint{3.177952in}{1.638592in}}%
\pgfpathlineto{\pgfqpoint{3.188401in}{1.638975in}}%
\pgfpathlineto{\pgfqpoint{3.188777in}{1.644598in}}%
\pgfpathlineto{\pgfqpoint{3.189295in}{1.644720in}}%
\pgfpathlineto{\pgfqpoint{3.196152in}{1.606066in}}%
\pgfpathlineto{\pgfqpoint{3.174778in}{1.597972in}}%
\pgfpathclose%
\pgfusepath{fill}%
\end{pgfscope}%
\begin{pgfscope}%
\pgfpathrectangle{\pgfqpoint{0.100000in}{0.100000in}}{\pgfqpoint{3.420221in}{2.189500in}}%
\pgfusepath{clip}%
\pgfsetbuttcap%
\pgfsetmiterjoin%
\definecolor{currentfill}{rgb}{0.000000,0.666667,0.666667}%
\pgfsetfillcolor{currentfill}%
\pgfsetlinewidth{0.000000pt}%
\definecolor{currentstroke}{rgb}{0.000000,0.000000,0.000000}%
\pgfsetstrokecolor{currentstroke}%
\pgfsetstrokeopacity{0.000000}%
\pgfsetdash{}{0pt}%
\pgfpathmoveto{\pgfqpoint{1.373045in}{0.783387in}}%
\pgfpathlineto{\pgfqpoint{1.368559in}{0.783876in}}%
\pgfpathlineto{\pgfqpoint{1.336058in}{0.787536in}}%
\pgfpathlineto{\pgfqpoint{1.275850in}{0.794839in}}%
\pgfpathlineto{\pgfqpoint{1.284875in}{0.866418in}}%
\pgfpathlineto{\pgfqpoint{1.288030in}{0.873223in}}%
\pgfpathlineto{\pgfqpoint{1.290863in}{0.898520in}}%
\pgfpathlineto{\pgfqpoint{1.290004in}{0.905403in}}%
\pgfpathlineto{\pgfqpoint{1.291645in}{0.918872in}}%
\pgfpathlineto{\pgfqpoint{1.311512in}{0.915642in}}%
\pgfpathlineto{\pgfqpoint{1.313113in}{0.928753in}}%
\pgfpathlineto{\pgfqpoint{1.321059in}{0.927763in}}%
\pgfpathlineto{\pgfqpoint{1.325022in}{0.960197in}}%
\pgfpathlineto{\pgfqpoint{1.362711in}{0.955774in}}%
\pgfpathlineto{\pgfqpoint{1.363463in}{0.962245in}}%
\pgfpathlineto{\pgfqpoint{1.389439in}{0.959366in}}%
\pgfpathlineto{\pgfqpoint{1.387246in}{0.940041in}}%
\pgfpathlineto{\pgfqpoint{1.381648in}{0.888384in}}%
\pgfpathlineto{\pgfqpoint{1.378690in}{0.869084in}}%
\pgfpathlineto{\pgfqpoint{1.352870in}{0.871484in}}%
\pgfpathlineto{\pgfqpoint{1.351464in}{0.858740in}}%
\pgfpathlineto{\pgfqpoint{1.349178in}{0.858999in}}%
\pgfpathlineto{\pgfqpoint{1.345295in}{0.825822in}}%
\pgfpathlineto{\pgfqpoint{1.377051in}{0.822409in}}%
\pgfpathclose%
\pgfusepath{fill}%
\end{pgfscope}%
\begin{pgfscope}%
\pgfpathrectangle{\pgfqpoint{0.100000in}{0.100000in}}{\pgfqpoint{3.420221in}{2.189500in}}%
\pgfusepath{clip}%
\pgfsetbuttcap%
\pgfsetmiterjoin%
\definecolor{currentfill}{rgb}{0.000000,0.564706,0.717647}%
\pgfsetfillcolor{currentfill}%
\pgfsetlinewidth{0.000000pt}%
\definecolor{currentstroke}{rgb}{0.000000,0.000000,0.000000}%
\pgfsetstrokecolor{currentstroke}%
\pgfsetstrokeopacity{0.000000}%
\pgfsetdash{}{0pt}%
\pgfpathmoveto{\pgfqpoint{2.710175in}{0.785759in}}%
\pgfpathlineto{\pgfqpoint{2.667007in}{0.781293in}}%
\pgfpathlineto{\pgfqpoint{2.646757in}{0.779342in}}%
\pgfpathlineto{\pgfqpoint{2.638390in}{0.782289in}}%
\pgfpathlineto{\pgfqpoint{2.636779in}{0.788748in}}%
\pgfpathlineto{\pgfqpoint{2.635869in}{0.794039in}}%
\pgfpathlineto{\pgfqpoint{2.643832in}{0.802266in}}%
\pgfpathlineto{\pgfqpoint{2.660563in}{0.804305in}}%
\pgfpathlineto{\pgfqpoint{2.658419in}{0.823956in}}%
\pgfpathlineto{\pgfqpoint{2.659341in}{0.827091in}}%
\pgfpathlineto{\pgfqpoint{2.671287in}{0.830538in}}%
\pgfpathlineto{\pgfqpoint{2.674578in}{0.821981in}}%
\pgfpathlineto{\pgfqpoint{2.680569in}{0.817230in}}%
\pgfpathlineto{\pgfqpoint{2.688279in}{0.820302in}}%
\pgfpathlineto{\pgfqpoint{2.692114in}{0.828749in}}%
\pgfpathlineto{\pgfqpoint{2.696323in}{0.830472in}}%
\pgfpathlineto{\pgfqpoint{2.707714in}{0.829380in}}%
\pgfpathlineto{\pgfqpoint{2.710760in}{0.825929in}}%
\pgfpathlineto{\pgfqpoint{2.711388in}{0.814745in}}%
\pgfpathlineto{\pgfqpoint{2.704572in}{0.810850in}}%
\pgfpathlineto{\pgfqpoint{2.700950in}{0.802977in}}%
\pgfpathlineto{\pgfqpoint{2.704197in}{0.799013in}}%
\pgfpathclose%
\pgfusepath{fill}%
\end{pgfscope}%
\begin{pgfscope}%
\pgfpathrectangle{\pgfqpoint{0.100000in}{0.100000in}}{\pgfqpoint{3.420221in}{2.189500in}}%
\pgfusepath{clip}%
\pgfsetbuttcap%
\pgfsetmiterjoin%
\definecolor{currentfill}{rgb}{0.000000,0.854902,0.572549}%
\pgfsetfillcolor{currentfill}%
\pgfsetlinewidth{0.000000pt}%
\definecolor{currentstroke}{rgb}{0.000000,0.000000,0.000000}%
\pgfsetstrokecolor{currentstroke}%
\pgfsetstrokeopacity{0.000000}%
\pgfsetdash{}{0pt}%
\pgfpathmoveto{\pgfqpoint{2.617490in}{1.701080in}}%
\pgfpathlineto{\pgfqpoint{2.600403in}{1.698834in}}%
\pgfpathlineto{\pgfqpoint{2.592142in}{1.763691in}}%
\pgfpathlineto{\pgfqpoint{2.591004in}{1.776583in}}%
\pgfpathlineto{\pgfqpoint{2.571775in}{1.774012in}}%
\pgfpathlineto{\pgfqpoint{2.567696in}{1.805968in}}%
\pgfpathlineto{\pgfqpoint{2.570147in}{1.806265in}}%
\pgfpathlineto{\pgfqpoint{2.576180in}{1.797762in}}%
\pgfpathlineto{\pgfqpoint{2.586328in}{1.797694in}}%
\pgfpathlineto{\pgfqpoint{2.594189in}{1.793136in}}%
\pgfpathlineto{\pgfqpoint{2.611718in}{1.789082in}}%
\pgfpathlineto{\pgfqpoint{2.615376in}{1.782395in}}%
\pgfpathlineto{\pgfqpoint{2.621361in}{1.775773in}}%
\pgfpathlineto{\pgfqpoint{2.625237in}{1.767127in}}%
\pgfpathlineto{\pgfqpoint{2.615867in}{1.768601in}}%
\pgfpathlineto{\pgfqpoint{2.617232in}{1.759421in}}%
\pgfpathlineto{\pgfqpoint{2.624041in}{1.756514in}}%
\pgfpathlineto{\pgfqpoint{2.627897in}{1.743959in}}%
\pgfpathlineto{\pgfqpoint{2.626596in}{1.736029in}}%
\pgfpathlineto{\pgfqpoint{2.628210in}{1.715657in}}%
\pgfpathlineto{\pgfqpoint{2.623879in}{1.710122in}}%
\pgfpathlineto{\pgfqpoint{2.619236in}{1.709353in}}%
\pgfpathclose%
\pgfusepath{fill}%
\end{pgfscope}%
\begin{pgfscope}%
\pgfpathrectangle{\pgfqpoint{0.100000in}{0.100000in}}{\pgfqpoint{3.420221in}{2.189500in}}%
\pgfusepath{clip}%
\pgfsetbuttcap%
\pgfsetmiterjoin%
\definecolor{currentfill}{rgb}{0.000000,0.290196,0.854902}%
\pgfsetfillcolor{currentfill}%
\pgfsetlinewidth{0.000000pt}%
\definecolor{currentstroke}{rgb}{0.000000,0.000000,0.000000}%
\pgfsetstrokecolor{currentstroke}%
\pgfsetstrokeopacity{0.000000}%
\pgfsetdash{}{0pt}%
\pgfpathmoveto{\pgfqpoint{2.633828in}{1.428964in}}%
\pgfpathlineto{\pgfqpoint{2.627438in}{1.427917in}}%
\pgfpathlineto{\pgfqpoint{2.627905in}{1.424023in}}%
\pgfpathlineto{\pgfqpoint{2.608369in}{1.421563in}}%
\pgfpathlineto{\pgfqpoint{2.609292in}{1.413955in}}%
\pgfpathlineto{\pgfqpoint{2.604063in}{1.411475in}}%
\pgfpathlineto{\pgfqpoint{2.583116in}{1.408965in}}%
\pgfpathlineto{\pgfqpoint{2.577821in}{1.456363in}}%
\pgfpathlineto{\pgfqpoint{2.603746in}{1.459466in}}%
\pgfpathlineto{\pgfqpoint{2.602241in}{1.472478in}}%
\pgfpathlineto{\pgfqpoint{2.608577in}{1.473259in}}%
\pgfpathlineto{\pgfqpoint{2.628008in}{1.475749in}}%
\pgfpathclose%
\pgfusepath{fill}%
\end{pgfscope}%
\begin{pgfscope}%
\pgfpathrectangle{\pgfqpoint{0.100000in}{0.100000in}}{\pgfqpoint{3.420221in}{2.189500in}}%
\pgfusepath{clip}%
\pgfsetbuttcap%
\pgfsetmiterjoin%
\definecolor{currentfill}{rgb}{0.000000,0.580392,0.709804}%
\pgfsetfillcolor{currentfill}%
\pgfsetlinewidth{0.000000pt}%
\definecolor{currentstroke}{rgb}{0.000000,0.000000,0.000000}%
\pgfsetstrokecolor{currentstroke}%
\pgfsetstrokeopacity{0.000000}%
\pgfsetdash{}{0pt}%
\pgfpathmoveto{\pgfqpoint{2.972715in}{1.558699in}}%
\pgfpathlineto{\pgfqpoint{2.973705in}{1.554026in}}%
\pgfpathlineto{\pgfqpoint{2.952182in}{1.549615in}}%
\pgfpathlineto{\pgfqpoint{2.948792in}{1.549023in}}%
\pgfpathlineto{\pgfqpoint{2.938379in}{1.557955in}}%
\pgfpathlineto{\pgfqpoint{2.922787in}{1.555960in}}%
\pgfpathlineto{\pgfqpoint{2.896798in}{1.550794in}}%
\pgfpathlineto{\pgfqpoint{2.894482in}{1.566120in}}%
\pgfpathlineto{\pgfqpoint{2.896155in}{1.567214in}}%
\pgfpathlineto{\pgfqpoint{2.893910in}{1.578792in}}%
\pgfpathlineto{\pgfqpoint{2.886080in}{1.577485in}}%
\pgfpathlineto{\pgfqpoint{2.879054in}{1.617283in}}%
\pgfpathlineto{\pgfqpoint{2.882861in}{1.617368in}}%
\pgfpathlineto{\pgfqpoint{2.889402in}{1.611633in}}%
\pgfpathlineto{\pgfqpoint{2.895489in}{1.613932in}}%
\pgfpathlineto{\pgfqpoint{2.910243in}{1.622669in}}%
\pgfpathlineto{\pgfqpoint{2.911749in}{1.621762in}}%
\pgfpathlineto{\pgfqpoint{2.945693in}{1.628021in}}%
\pgfpathlineto{\pgfqpoint{2.952749in}{1.625817in}}%
\pgfpathlineto{\pgfqpoint{2.957971in}{1.590724in}}%
\pgfpathlineto{\pgfqpoint{2.965632in}{1.592246in}}%
\pgfpathclose%
\pgfusepath{fill}%
\end{pgfscope}%
\begin{pgfscope}%
\pgfpathrectangle{\pgfqpoint{0.100000in}{0.100000in}}{\pgfqpoint{3.420221in}{2.189500in}}%
\pgfusepath{clip}%
\pgfsetbuttcap%
\pgfsetmiterjoin%
\definecolor{currentfill}{rgb}{0.000000,0.407843,0.796078}%
\pgfsetfillcolor{currentfill}%
\pgfsetlinewidth{0.000000pt}%
\definecolor{currentstroke}{rgb}{0.000000,0.000000,0.000000}%
\pgfsetstrokecolor{currentstroke}%
\pgfsetstrokeopacity{0.000000}%
\pgfsetdash{}{0pt}%
\pgfpathmoveto{\pgfqpoint{2.548030in}{1.587505in}}%
\pgfpathlineto{\pgfqpoint{2.561011in}{1.588939in}}%
\pgfpathlineto{\pgfqpoint{2.558113in}{1.614925in}}%
\pgfpathlineto{\pgfqpoint{2.583661in}{1.617791in}}%
\pgfpathlineto{\pgfqpoint{2.586879in}{1.592322in}}%
\pgfpathlineto{\pgfqpoint{2.598123in}{1.593702in}}%
\pgfpathlineto{\pgfqpoint{2.602224in}{1.567583in}}%
\pgfpathlineto{\pgfqpoint{2.570463in}{1.563673in}}%
\pgfpathlineto{\pgfqpoint{2.551025in}{1.561460in}}%
\pgfpathclose%
\pgfusepath{fill}%
\end{pgfscope}%
\begin{pgfscope}%
\pgfpathrectangle{\pgfqpoint{0.100000in}{0.100000in}}{\pgfqpoint{3.420221in}{2.189500in}}%
\pgfusepath{clip}%
\pgfsetbuttcap%
\pgfsetmiterjoin%
\definecolor{currentfill}{rgb}{0.000000,0.321569,0.839216}%
\pgfsetfillcolor{currentfill}%
\pgfsetlinewidth{0.000000pt}%
\definecolor{currentstroke}{rgb}{0.000000,0.000000,0.000000}%
\pgfsetstrokecolor{currentstroke}%
\pgfsetstrokeopacity{0.000000}%
\pgfsetdash{}{0pt}%
\pgfpathmoveto{\pgfqpoint{1.893757in}{1.968006in}}%
\pgfpathlineto{\pgfqpoint{1.895156in}{1.974588in}}%
\pgfpathlineto{\pgfqpoint{1.893272in}{1.982670in}}%
\pgfpathlineto{\pgfqpoint{1.894774in}{1.988481in}}%
\pgfpathlineto{\pgfqpoint{1.893283in}{1.994630in}}%
\pgfpathlineto{\pgfqpoint{1.932284in}{1.993898in}}%
\pgfpathlineto{\pgfqpoint{1.971800in}{1.993054in}}%
\pgfpathlineto{\pgfqpoint{1.972039in}{1.953696in}}%
\pgfpathlineto{\pgfqpoint{1.972522in}{1.947123in}}%
\pgfpathlineto{\pgfqpoint{1.966050in}{1.947309in}}%
\pgfpathlineto{\pgfqpoint{1.966019in}{1.940691in}}%
\pgfpathlineto{\pgfqpoint{1.959519in}{1.940720in}}%
\pgfpathlineto{\pgfqpoint{1.959446in}{1.934083in}}%
\pgfpathlineto{\pgfqpoint{1.933211in}{1.934379in}}%
\pgfpathlineto{\pgfqpoint{1.933258in}{1.940959in}}%
\pgfpathlineto{\pgfqpoint{1.926685in}{1.941078in}}%
\pgfpathlineto{\pgfqpoint{1.926137in}{1.965878in}}%
\pgfpathlineto{\pgfqpoint{1.893444in}{1.966466in}}%
\pgfpathclose%
\pgfusepath{fill}%
\end{pgfscope}%
\begin{pgfscope}%
\pgfpathrectangle{\pgfqpoint{0.100000in}{0.100000in}}{\pgfqpoint{3.420221in}{2.189500in}}%
\pgfusepath{clip}%
\pgfsetbuttcap%
\pgfsetmiterjoin%
\definecolor{currentfill}{rgb}{0.000000,0.215686,0.892157}%
\pgfsetfillcolor{currentfill}%
\pgfsetlinewidth{0.000000pt}%
\definecolor{currentstroke}{rgb}{0.000000,0.000000,0.000000}%
\pgfsetstrokecolor{currentstroke}%
\pgfsetstrokeopacity{0.000000}%
\pgfsetdash{}{0pt}%
\pgfpathmoveto{\pgfqpoint{3.020353in}{1.428862in}}%
\pgfpathlineto{\pgfqpoint{3.006635in}{1.426164in}}%
\pgfpathlineto{\pgfqpoint{3.002720in}{1.442543in}}%
\pgfpathlineto{\pgfqpoint{2.987668in}{1.464488in}}%
\pgfpathlineto{\pgfqpoint{2.985139in}{1.463500in}}%
\pgfpathlineto{\pgfqpoint{2.980367in}{1.470586in}}%
\pgfpathlineto{\pgfqpoint{2.975079in}{1.469307in}}%
\pgfpathlineto{\pgfqpoint{2.972163in}{1.472905in}}%
\pgfpathlineto{\pgfqpoint{2.975479in}{1.479645in}}%
\pgfpathlineto{\pgfqpoint{2.973347in}{1.483154in}}%
\pgfpathlineto{\pgfqpoint{2.981135in}{1.499447in}}%
\pgfpathlineto{\pgfqpoint{2.995770in}{1.510110in}}%
\pgfpathlineto{\pgfqpoint{2.998384in}{1.499501in}}%
\pgfpathlineto{\pgfqpoint{3.016341in}{1.501376in}}%
\pgfpathlineto{\pgfqpoint{3.023622in}{1.497459in}}%
\pgfpathlineto{\pgfqpoint{3.035408in}{1.503588in}}%
\pgfpathlineto{\pgfqpoint{3.046213in}{1.497933in}}%
\pgfpathlineto{\pgfqpoint{3.039465in}{1.490395in}}%
\pgfpathlineto{\pgfqpoint{3.049723in}{1.471198in}}%
\pgfpathlineto{\pgfqpoint{3.042913in}{1.465774in}}%
\pgfpathlineto{\pgfqpoint{3.046396in}{1.460355in}}%
\pgfpathlineto{\pgfqpoint{3.053852in}{1.460951in}}%
\pgfpathlineto{\pgfqpoint{3.058931in}{1.454019in}}%
\pgfpathlineto{\pgfqpoint{3.064079in}{1.452421in}}%
\pgfpathlineto{\pgfqpoint{3.075488in}{1.440206in}}%
\pgfpathclose%
\pgfusepath{fill}%
\end{pgfscope}%
\begin{pgfscope}%
\pgfpathrectangle{\pgfqpoint{0.100000in}{0.100000in}}{\pgfqpoint{3.420221in}{2.189500in}}%
\pgfusepath{clip}%
\pgfsetbuttcap%
\pgfsetmiterjoin%
\definecolor{currentfill}{rgb}{0.000000,0.392157,0.803922}%
\pgfsetfillcolor{currentfill}%
\pgfsetlinewidth{0.000000pt}%
\definecolor{currentstroke}{rgb}{0.000000,0.000000,0.000000}%
\pgfsetstrokecolor{currentstroke}%
\pgfsetstrokeopacity{0.000000}%
\pgfsetdash{}{0pt}%
\pgfpathmoveto{\pgfqpoint{1.573370in}{1.475782in}}%
\pgfpathlineto{\pgfqpoint{1.574624in}{1.472371in}}%
\pgfpathlineto{\pgfqpoint{1.573482in}{1.459420in}}%
\pgfpathlineto{\pgfqpoint{1.572404in}{1.446458in}}%
\pgfpathlineto{\pgfqpoint{1.569677in}{1.443236in}}%
\pgfpathlineto{\pgfqpoint{1.517776in}{1.447837in}}%
\pgfpathlineto{\pgfqpoint{1.458619in}{1.453758in}}%
\pgfpathlineto{\pgfqpoint{1.421950in}{1.457544in}}%
\pgfpathlineto{\pgfqpoint{1.427460in}{1.506606in}}%
\pgfpathlineto{\pgfqpoint{1.462234in}{1.502502in}}%
\pgfpathlineto{\pgfqpoint{1.461553in}{1.495939in}}%
\pgfpathlineto{\pgfqpoint{1.495057in}{1.492423in}}%
\pgfpathlineto{\pgfqpoint{1.493796in}{1.479665in}}%
\pgfpathlineto{\pgfqpoint{1.531986in}{1.476085in}}%
\pgfpathlineto{\pgfqpoint{1.532356in}{1.479298in}}%
\pgfpathclose%
\pgfusepath{fill}%
\end{pgfscope}%
\begin{pgfscope}%
\pgfpathrectangle{\pgfqpoint{0.100000in}{0.100000in}}{\pgfqpoint{3.420221in}{2.189500in}}%
\pgfusepath{clip}%
\pgfsetbuttcap%
\pgfsetmiterjoin%
\definecolor{currentfill}{rgb}{0.000000,0.639216,0.680392}%
\pgfsetfillcolor{currentfill}%
\pgfsetlinewidth{0.000000pt}%
\definecolor{currentstroke}{rgb}{0.000000,0.000000,0.000000}%
\pgfsetstrokecolor{currentstroke}%
\pgfsetstrokeopacity{0.000000}%
\pgfsetdash{}{0pt}%
\pgfpathmoveto{\pgfqpoint{1.352738in}{1.319215in}}%
\pgfpathlineto{\pgfqpoint{1.352268in}{1.316407in}}%
\pgfpathlineto{\pgfqpoint{1.361115in}{1.309640in}}%
\pgfpathlineto{\pgfqpoint{1.366438in}{1.303660in}}%
\pgfpathlineto{\pgfqpoint{1.366927in}{1.298680in}}%
\pgfpathlineto{\pgfqpoint{1.362244in}{1.290688in}}%
\pgfpathlineto{\pgfqpoint{1.366454in}{1.284676in}}%
\pgfpathlineto{\pgfqpoint{1.365991in}{1.280285in}}%
\pgfpathlineto{\pgfqpoint{1.357626in}{1.272609in}}%
\pgfpathlineto{\pgfqpoint{1.353623in}{1.271379in}}%
\pgfpathlineto{\pgfqpoint{1.343652in}{1.272532in}}%
\pgfpathlineto{\pgfqpoint{1.338815in}{1.279185in}}%
\pgfpathlineto{\pgfqpoint{1.336567in}{1.296292in}}%
\pgfpathlineto{\pgfqpoint{1.337562in}{1.303598in}}%
\pgfpathlineto{\pgfqpoint{1.343757in}{1.308151in}}%
\pgfpathlineto{\pgfqpoint{1.335547in}{1.310367in}}%
\pgfpathlineto{\pgfqpoint{1.328624in}{1.317869in}}%
\pgfpathlineto{\pgfqpoint{1.330421in}{1.322164in}}%
\pgfpathclose%
\pgfusepath{fill}%
\end{pgfscope}%
\begin{pgfscope}%
\pgfpathrectangle{\pgfqpoint{0.100000in}{0.100000in}}{\pgfqpoint{3.420221in}{2.189500in}}%
\pgfusepath{clip}%
\pgfsetbuttcap%
\pgfsetmiterjoin%
\definecolor{currentfill}{rgb}{0.000000,0.458824,0.770588}%
\pgfsetfillcolor{currentfill}%
\pgfsetlinewidth{0.000000pt}%
\definecolor{currentstroke}{rgb}{0.000000,0.000000,0.000000}%
\pgfsetstrokecolor{currentstroke}%
\pgfsetstrokeopacity{0.000000}%
\pgfsetdash{}{0pt}%
\pgfpathmoveto{\pgfqpoint{2.390924in}{1.047996in}}%
\pgfpathlineto{\pgfqpoint{2.384846in}{1.047700in}}%
\pgfpathlineto{\pgfqpoint{2.384289in}{1.068094in}}%
\pgfpathlineto{\pgfqpoint{2.386255in}{1.072336in}}%
\pgfpathlineto{\pgfqpoint{2.393715in}{1.072719in}}%
\pgfpathlineto{\pgfqpoint{2.392607in}{1.101666in}}%
\pgfpathlineto{\pgfqpoint{2.394245in}{1.101756in}}%
\pgfpathlineto{\pgfqpoint{2.420478in}{1.103406in}}%
\pgfpathlineto{\pgfqpoint{2.420583in}{1.098954in}}%
\pgfpathlineto{\pgfqpoint{2.427307in}{1.090235in}}%
\pgfpathlineto{\pgfqpoint{2.428181in}{1.084896in}}%
\pgfpathlineto{\pgfqpoint{2.441402in}{1.085362in}}%
\pgfpathlineto{\pgfqpoint{2.450386in}{1.083246in}}%
\pgfpathlineto{\pgfqpoint{2.453832in}{1.087382in}}%
\pgfpathlineto{\pgfqpoint{2.453911in}{1.093847in}}%
\pgfpathlineto{\pgfqpoint{2.467660in}{1.094038in}}%
\pgfpathlineto{\pgfqpoint{2.475574in}{1.093646in}}%
\pgfpathlineto{\pgfqpoint{2.474318in}{1.086247in}}%
\pgfpathlineto{\pgfqpoint{2.476009in}{1.082158in}}%
\pgfpathlineto{\pgfqpoint{2.474945in}{1.067350in}}%
\pgfpathlineto{\pgfqpoint{2.454737in}{1.068227in}}%
\pgfpathlineto{\pgfqpoint{2.444841in}{1.060810in}}%
\pgfpathlineto{\pgfqpoint{2.444652in}{1.055763in}}%
\pgfpathlineto{\pgfqpoint{2.432918in}{1.054802in}}%
\pgfpathlineto{\pgfqpoint{2.425926in}{1.052204in}}%
\pgfpathlineto{\pgfqpoint{2.424272in}{1.055009in}}%
\pgfpathlineto{\pgfqpoint{2.416733in}{1.051939in}}%
\pgfpathlineto{\pgfqpoint{2.393003in}{1.050812in}}%
\pgfpathclose%
\pgfusepath{fill}%
\end{pgfscope}%
\begin{pgfscope}%
\pgfpathrectangle{\pgfqpoint{0.100000in}{0.100000in}}{\pgfqpoint{3.420221in}{2.189500in}}%
\pgfusepath{clip}%
\pgfsetbuttcap%
\pgfsetmiterjoin%
\definecolor{currentfill}{rgb}{0.000000,0.439216,0.780392}%
\pgfsetfillcolor{currentfill}%
\pgfsetlinewidth{0.000000pt}%
\definecolor{currentstroke}{rgb}{0.000000,0.000000,0.000000}%
\pgfsetstrokecolor{currentstroke}%
\pgfsetstrokeopacity{0.000000}%
\pgfsetdash{}{0pt}%
\pgfpathmoveto{\pgfqpoint{0.978816in}{0.369945in}}%
\pgfpathlineto{\pgfqpoint{0.969453in}{0.375163in}}%
\pgfpathlineto{\pgfqpoint{0.972506in}{0.380590in}}%
\pgfpathlineto{\pgfqpoint{0.971298in}{0.382530in}}%
\pgfpathlineto{\pgfqpoint{0.976179in}{0.390826in}}%
\pgfpathlineto{\pgfqpoint{0.974277in}{0.391940in}}%
\pgfpathlineto{\pgfqpoint{0.975480in}{0.393949in}}%
\pgfpathlineto{\pgfqpoint{0.973321in}{0.395240in}}%
\pgfpathlineto{\pgfqpoint{0.974566in}{0.397281in}}%
\pgfpathlineto{\pgfqpoint{0.971017in}{0.399375in}}%
\pgfpathlineto{\pgfqpoint{0.970382in}{0.400610in}}%
\pgfpathlineto{\pgfqpoint{0.967310in}{0.398737in}}%
\pgfpathlineto{\pgfqpoint{0.964631in}{0.400756in}}%
\pgfpathlineto{\pgfqpoint{0.967330in}{0.404302in}}%
\pgfpathlineto{\pgfqpoint{0.966548in}{0.406661in}}%
\pgfpathlineto{\pgfqpoint{0.970014in}{0.408060in}}%
\pgfpathlineto{\pgfqpoint{0.971950in}{0.413652in}}%
\pgfpathlineto{\pgfqpoint{0.955336in}{0.423256in}}%
\pgfpathlineto{\pgfqpoint{0.949467in}{0.426889in}}%
\pgfpathlineto{\pgfqpoint{0.950757in}{0.429003in}}%
\pgfpathlineto{\pgfqpoint{0.940384in}{0.435494in}}%
\pgfpathlineto{\pgfqpoint{0.937806in}{0.431288in}}%
\pgfpathlineto{\pgfqpoint{0.933697in}{0.433889in}}%
\pgfpathlineto{\pgfqpoint{0.940211in}{0.444350in}}%
\pgfpathlineto{\pgfqpoint{0.927328in}{0.452402in}}%
\pgfpathlineto{\pgfqpoint{0.923645in}{0.454885in}}%
\pgfpathlineto{\pgfqpoint{0.921474in}{0.451616in}}%
\pgfpathlineto{\pgfqpoint{0.905818in}{0.462274in}}%
\pgfpathlineto{\pgfqpoint{0.896369in}{0.463780in}}%
\pgfpathlineto{\pgfqpoint{0.872819in}{0.467673in}}%
\pgfpathlineto{\pgfqpoint{0.871589in}{0.466101in}}%
\pgfpathlineto{\pgfqpoint{0.860358in}{0.474731in}}%
\pgfpathlineto{\pgfqpoint{0.853197in}{0.465288in}}%
\pgfpathlineto{\pgfqpoint{0.851055in}{0.463937in}}%
\pgfpathlineto{\pgfqpoint{0.847674in}{0.466581in}}%
\pgfpathlineto{\pgfqpoint{0.846189in}{0.464685in}}%
\pgfpathlineto{\pgfqpoint{0.842415in}{0.467677in}}%
\pgfpathlineto{\pgfqpoint{0.840208in}{0.464836in}}%
\pgfpathlineto{\pgfqpoint{0.838061in}{0.466246in}}%
\pgfpathlineto{\pgfqpoint{0.827581in}{0.474284in}}%
\pgfpathlineto{\pgfqpoint{0.814712in}{0.484379in}}%
\pgfpathlineto{\pgfqpoint{0.806220in}{0.491468in}}%
\pgfpathlineto{\pgfqpoint{0.807964in}{0.493605in}}%
\pgfpathlineto{\pgfqpoint{0.793488in}{0.505755in}}%
\pgfpathlineto{\pgfqpoint{0.791930in}{0.503912in}}%
\pgfpathlineto{\pgfqpoint{0.784286in}{0.510390in}}%
\pgfpathlineto{\pgfqpoint{0.782292in}{0.508068in}}%
\pgfpathlineto{\pgfqpoint{0.775602in}{0.513824in}}%
\pgfpathlineto{\pgfqpoint{0.766043in}{0.522351in}}%
\pgfpathlineto{\pgfqpoint{0.770766in}{0.527717in}}%
\pgfpathlineto{\pgfqpoint{0.769426in}{0.528919in}}%
\pgfpathlineto{\pgfqpoint{0.775908in}{0.536162in}}%
\pgfpathlineto{\pgfqpoint{0.776368in}{0.535802in}}%
\pgfpathlineto{\pgfqpoint{0.782922in}{0.543042in}}%
\pgfpathlineto{\pgfqpoint{0.785309in}{0.544738in}}%
\pgfpathlineto{\pgfqpoint{0.786440in}{0.544258in}}%
\pgfpathlineto{\pgfqpoint{0.791293in}{0.549666in}}%
\pgfpathlineto{\pgfqpoint{0.793006in}{0.548213in}}%
\pgfpathlineto{\pgfqpoint{0.797801in}{0.554049in}}%
\pgfpathlineto{\pgfqpoint{0.799686in}{0.552410in}}%
\pgfpathlineto{\pgfqpoint{0.801137in}{0.554034in}}%
\pgfpathlineto{\pgfqpoint{0.804917in}{0.550755in}}%
\pgfpathlineto{\pgfqpoint{0.806504in}{0.552690in}}%
\pgfpathlineto{\pgfqpoint{0.808189in}{0.551273in}}%
\pgfpathlineto{\pgfqpoint{0.809861in}{0.553230in}}%
\pgfpathlineto{\pgfqpoint{0.811532in}{0.551759in}}%
\pgfpathlineto{\pgfqpoint{0.816586in}{0.557173in}}%
\pgfpathlineto{\pgfqpoint{0.814431in}{0.558988in}}%
\pgfpathlineto{\pgfqpoint{0.827352in}{0.574281in}}%
\pgfpathlineto{\pgfqpoint{0.829283in}{0.572676in}}%
\pgfpathlineto{\pgfqpoint{0.832511in}{0.576474in}}%
\pgfpathlineto{\pgfqpoint{0.834379in}{0.574888in}}%
\pgfpathlineto{\pgfqpoint{0.840759in}{0.582332in}}%
\pgfpathlineto{\pgfqpoint{0.842523in}{0.580778in}}%
\pgfpathlineto{\pgfqpoint{0.847443in}{0.586556in}}%
\pgfpathlineto{\pgfqpoint{0.845545in}{0.588170in}}%
\pgfpathlineto{\pgfqpoint{0.853810in}{0.597799in}}%
\pgfpathlineto{\pgfqpoint{0.859512in}{0.592908in}}%
\pgfpathlineto{\pgfqpoint{0.862578in}{0.596628in}}%
\pgfpathlineto{\pgfqpoint{0.872287in}{0.588423in}}%
\pgfpathlineto{\pgfqpoint{0.878686in}{0.596417in}}%
\pgfpathlineto{\pgfqpoint{0.886593in}{0.589956in}}%
\pgfpathlineto{\pgfqpoint{0.883365in}{0.585915in}}%
\pgfpathlineto{\pgfqpoint{0.887331in}{0.582733in}}%
\pgfpathlineto{\pgfqpoint{0.890522in}{0.586706in}}%
\pgfpathlineto{\pgfqpoint{0.894450in}{0.583528in}}%
\pgfpathlineto{\pgfqpoint{0.892907in}{0.581574in}}%
\pgfpathlineto{\pgfqpoint{0.894880in}{0.579958in}}%
\pgfpathlineto{\pgfqpoint{0.893274in}{0.577907in}}%
\pgfpathlineto{\pgfqpoint{0.897021in}{0.574977in}}%
\pgfpathlineto{\pgfqpoint{0.902178in}{0.580644in}}%
\pgfpathlineto{\pgfqpoint{0.907862in}{0.576122in}}%
\pgfpathlineto{\pgfqpoint{0.911046in}{0.580147in}}%
\pgfpathlineto{\pgfqpoint{0.917029in}{0.575298in}}%
\pgfpathlineto{\pgfqpoint{0.919217in}{0.576999in}}%
\pgfpathlineto{\pgfqpoint{0.925977in}{0.585574in}}%
\pgfpathlineto{\pgfqpoint{0.924531in}{0.586622in}}%
\pgfpathlineto{\pgfqpoint{0.926093in}{0.588734in}}%
\pgfpathlineto{\pgfqpoint{0.922036in}{0.591838in}}%
\pgfpathlineto{\pgfqpoint{0.926740in}{0.597895in}}%
\pgfpathlineto{\pgfqpoint{0.928891in}{0.599536in}}%
\pgfpathlineto{\pgfqpoint{0.926839in}{0.601138in}}%
\pgfpathlineto{\pgfqpoint{0.928440in}{0.603101in}}%
\pgfpathlineto{\pgfqpoint{0.926346in}{0.604707in}}%
\pgfpathlineto{\pgfqpoint{0.927967in}{0.606750in}}%
\pgfpathlineto{\pgfqpoint{0.925986in}{0.608337in}}%
\pgfpathlineto{\pgfqpoint{0.930585in}{0.613220in}}%
\pgfpathlineto{\pgfqpoint{0.951054in}{0.597366in}}%
\pgfpathlineto{\pgfqpoint{0.978272in}{0.577564in}}%
\pgfpathlineto{\pgfqpoint{1.002494in}{0.560994in}}%
\pgfpathlineto{\pgfqpoint{1.006548in}{0.558967in}}%
\pgfpathlineto{\pgfqpoint{1.015804in}{0.553076in}}%
\pgfpathlineto{\pgfqpoint{1.023374in}{0.564348in}}%
\pgfpathlineto{\pgfqpoint{1.031807in}{0.559415in}}%
\pgfpathlineto{\pgfqpoint{1.057682in}{0.543496in}}%
\pgfpathlineto{\pgfqpoint{1.070657in}{0.535615in}}%
\pgfpathlineto{\pgfqpoint{1.061406in}{0.518928in}}%
\pgfpathlineto{\pgfqpoint{1.048815in}{0.496215in}}%
\pgfpathlineto{\pgfqpoint{1.036384in}{0.473792in}}%
\pgfpathlineto{\pgfqpoint{1.025030in}{0.453310in}}%
\pgfpathlineto{\pgfqpoint{1.012869in}{0.431373in}}%
\pgfpathlineto{\pgfqpoint{1.000831in}{0.409657in}}%
\pgfpathlineto{\pgfqpoint{0.988904in}{0.388141in}}%
\pgfpathclose%
\pgfusepath{fill}%
\end{pgfscope}%
\begin{pgfscope}%
\pgfpathrectangle{\pgfqpoint{0.100000in}{0.100000in}}{\pgfqpoint{3.420221in}{2.189500in}}%
\pgfusepath{clip}%
\pgfsetbuttcap%
\pgfsetmiterjoin%
\definecolor{currentfill}{rgb}{0.000000,0.454902,0.772549}%
\pgfsetfillcolor{currentfill}%
\pgfsetlinewidth{0.000000pt}%
\definecolor{currentstroke}{rgb}{0.000000,0.000000,0.000000}%
\pgfsetstrokecolor{currentstroke}%
\pgfsetstrokeopacity{0.000000}%
\pgfsetdash{}{0pt}%
\pgfpathmoveto{\pgfqpoint{2.175450in}{1.112439in}}%
\pgfpathlineto{\pgfqpoint{2.136270in}{1.112219in}}%
\pgfpathlineto{\pgfqpoint{2.101613in}{1.112322in}}%
\pgfpathlineto{\pgfqpoint{2.101648in}{1.125459in}}%
\pgfpathlineto{\pgfqpoint{2.085321in}{1.125434in}}%
\pgfpathlineto{\pgfqpoint{2.085293in}{1.146851in}}%
\pgfpathlineto{\pgfqpoint{2.084056in}{1.146860in}}%
\pgfpathlineto{\pgfqpoint{2.084502in}{1.181336in}}%
\pgfpathlineto{\pgfqpoint{2.083284in}{1.187891in}}%
\pgfpathlineto{\pgfqpoint{2.086718in}{1.194349in}}%
\pgfpathlineto{\pgfqpoint{2.090063in}{1.194302in}}%
\pgfpathlineto{\pgfqpoint{2.090345in}{1.206259in}}%
\pgfpathlineto{\pgfqpoint{2.116239in}{1.205956in}}%
\pgfpathlineto{\pgfqpoint{2.116066in}{1.193971in}}%
\pgfpathlineto{\pgfqpoint{2.142816in}{1.193833in}}%
\pgfpathlineto{\pgfqpoint{2.147470in}{1.186987in}}%
\pgfpathlineto{\pgfqpoint{2.155359in}{1.191691in}}%
\pgfpathlineto{\pgfqpoint{2.155406in}{1.180805in}}%
\pgfpathlineto{\pgfqpoint{2.165320in}{1.176185in}}%
\pgfpathlineto{\pgfqpoint{2.165384in}{1.172896in}}%
\pgfpathlineto{\pgfqpoint{2.166506in}{1.132070in}}%
\pgfpathlineto{\pgfqpoint{2.176237in}{1.132118in}}%
\pgfpathclose%
\pgfusepath{fill}%
\end{pgfscope}%
\begin{pgfscope}%
\pgfpathrectangle{\pgfqpoint{0.100000in}{0.100000in}}{\pgfqpoint{3.420221in}{2.189500in}}%
\pgfusepath{clip}%
\pgfsetbuttcap%
\pgfsetmiterjoin%
\definecolor{currentfill}{rgb}{0.000000,0.619608,0.690196}%
\pgfsetfillcolor{currentfill}%
\pgfsetlinewidth{0.000000pt}%
\definecolor{currentstroke}{rgb}{0.000000,0.000000,0.000000}%
\pgfsetstrokecolor{currentstroke}%
\pgfsetstrokeopacity{0.000000}%
\pgfsetdash{}{0pt}%
\pgfpathmoveto{\pgfqpoint{2.067236in}{0.881542in}}%
\pgfpathlineto{\pgfqpoint{2.062081in}{0.883822in}}%
\pgfpathlineto{\pgfqpoint{2.049722in}{0.885210in}}%
\pgfpathlineto{\pgfqpoint{2.040516in}{0.895557in}}%
\pgfpathlineto{\pgfqpoint{2.034723in}{0.895551in}}%
\pgfpathlineto{\pgfqpoint{2.034317in}{0.872749in}}%
\pgfpathlineto{\pgfqpoint{2.018001in}{0.877582in}}%
\pgfpathlineto{\pgfqpoint{2.010254in}{0.880804in}}%
\pgfpathlineto{\pgfqpoint{2.005015in}{0.886097in}}%
\pgfpathlineto{\pgfqpoint{1.988426in}{0.897026in}}%
\pgfpathlineto{\pgfqpoint{1.983689in}{0.890342in}}%
\pgfpathlineto{\pgfqpoint{1.976639in}{0.889902in}}%
\pgfpathlineto{\pgfqpoint{1.967638in}{0.891613in}}%
\pgfpathlineto{\pgfqpoint{1.965546in}{0.895090in}}%
\pgfpathlineto{\pgfqpoint{1.954519in}{0.890543in}}%
\pgfpathlineto{\pgfqpoint{1.949613in}{0.890679in}}%
\pgfpathlineto{\pgfqpoint{1.943592in}{0.896311in}}%
\pgfpathlineto{\pgfqpoint{1.943715in}{0.909646in}}%
\pgfpathlineto{\pgfqpoint{1.940418in}{0.912029in}}%
\pgfpathlineto{\pgfqpoint{1.953619in}{0.911892in}}%
\pgfpathlineto{\pgfqpoint{1.953843in}{0.938189in}}%
\pgfpathlineto{\pgfqpoint{1.960513in}{0.938154in}}%
\pgfpathlineto{\pgfqpoint{1.960545in}{0.944696in}}%
\pgfpathlineto{\pgfqpoint{1.970297in}{0.944655in}}%
\pgfpathlineto{\pgfqpoint{1.970321in}{0.951191in}}%
\pgfpathlineto{\pgfqpoint{2.005955in}{0.951083in}}%
\pgfpathlineto{\pgfqpoint{2.005973in}{0.938009in}}%
\pgfpathlineto{\pgfqpoint{2.035479in}{0.938149in}}%
\pgfpathlineto{\pgfqpoint{2.035787in}{0.954795in}}%
\pgfpathlineto{\pgfqpoint{2.044257in}{0.952482in}}%
\pgfpathlineto{\pgfqpoint{2.062760in}{0.952281in}}%
\pgfpathlineto{\pgfqpoint{2.068199in}{0.950323in}}%
\pgfpathlineto{\pgfqpoint{2.068176in}{0.926610in}}%
\pgfpathlineto{\pgfqpoint{2.048359in}{0.926817in}}%
\pgfpathlineto{\pgfqpoint{2.049138in}{0.914570in}}%
\pgfpathlineto{\pgfqpoint{2.058030in}{0.914570in}}%
\pgfpathlineto{\pgfqpoint{2.058990in}{0.902728in}}%
\pgfpathlineto{\pgfqpoint{2.061239in}{0.891651in}}%
\pgfpathlineto{\pgfqpoint{2.065933in}{0.886473in}}%
\pgfpathclose%
\pgfusepath{fill}%
\end{pgfscope}%
\begin{pgfscope}%
\pgfpathrectangle{\pgfqpoint{0.100000in}{0.100000in}}{\pgfqpoint{3.420221in}{2.189500in}}%
\pgfusepath{clip}%
\pgfsetbuttcap%
\pgfsetmiterjoin%
\definecolor{currentfill}{rgb}{0.000000,0.152941,0.923529}%
\pgfsetfillcolor{currentfill}%
\pgfsetlinewidth{0.000000pt}%
\definecolor{currentstroke}{rgb}{0.000000,0.000000,0.000000}%
\pgfsetstrokecolor{currentstroke}%
\pgfsetstrokeopacity{0.000000}%
\pgfsetdash{}{0pt}%
\pgfpathmoveto{\pgfqpoint{2.031989in}{1.436739in}}%
\pgfpathlineto{\pgfqpoint{2.019009in}{1.436722in}}%
\pgfpathlineto{\pgfqpoint{2.018894in}{1.470019in}}%
\pgfpathlineto{\pgfqpoint{2.016458in}{1.470009in}}%
\pgfpathlineto{\pgfqpoint{2.016364in}{1.489528in}}%
\pgfpathlineto{\pgfqpoint{2.048799in}{1.489785in}}%
\pgfpathlineto{\pgfqpoint{2.048535in}{1.515790in}}%
\pgfpathlineto{\pgfqpoint{2.087519in}{1.516249in}}%
\pgfpathlineto{\pgfqpoint{2.100376in}{1.516456in}}%
\pgfpathlineto{\pgfqpoint{2.100868in}{1.490491in}}%
\pgfpathlineto{\pgfqpoint{2.094386in}{1.490398in}}%
\pgfpathlineto{\pgfqpoint{2.094709in}{1.470727in}}%
\pgfpathlineto{\pgfqpoint{2.095955in}{1.463778in}}%
\pgfpathlineto{\pgfqpoint{2.096481in}{1.437752in}}%
\pgfpathlineto{\pgfqpoint{2.083525in}{1.437583in}}%
\pgfpathlineto{\pgfqpoint{2.064270in}{1.436940in}}%
\pgfpathclose%
\pgfusepath{fill}%
\end{pgfscope}%
\begin{pgfscope}%
\pgfpathrectangle{\pgfqpoint{0.100000in}{0.100000in}}{\pgfqpoint{3.420221in}{2.189500in}}%
\pgfusepath{clip}%
\pgfsetbuttcap%
\pgfsetmiterjoin%
\definecolor{currentfill}{rgb}{0.000000,0.415686,0.792157}%
\pgfsetfillcolor{currentfill}%
\pgfsetlinewidth{0.000000pt}%
\definecolor{currentstroke}{rgb}{0.000000,0.000000,0.000000}%
\pgfsetstrokecolor{currentstroke}%
\pgfsetstrokeopacity{0.000000}%
\pgfsetdash{}{0pt}%
\pgfpathmoveto{\pgfqpoint{1.694769in}{0.985795in}}%
\pgfpathlineto{\pgfqpoint{1.693596in}{0.964501in}}%
\pgfpathlineto{\pgfqpoint{1.667994in}{0.966050in}}%
\pgfpathlineto{\pgfqpoint{1.660304in}{0.966529in}}%
\pgfpathlineto{\pgfqpoint{1.662404in}{0.999165in}}%
\pgfpathlineto{\pgfqpoint{1.664323in}{1.031851in}}%
\pgfpathlineto{\pgfqpoint{1.697200in}{1.029928in}}%
\pgfpathclose%
\pgfusepath{fill}%
\end{pgfscope}%
\begin{pgfscope}%
\pgfpathrectangle{\pgfqpoint{0.100000in}{0.100000in}}{\pgfqpoint{3.420221in}{2.189500in}}%
\pgfusepath{clip}%
\pgfsetbuttcap%
\pgfsetmiterjoin%
\definecolor{currentfill}{rgb}{0.000000,0.352941,0.823529}%
\pgfsetfillcolor{currentfill}%
\pgfsetlinewidth{0.000000pt}%
\definecolor{currentstroke}{rgb}{0.000000,0.000000,0.000000}%
\pgfsetstrokecolor{currentstroke}%
\pgfsetstrokeopacity{0.000000}%
\pgfsetdash{}{0pt}%
\pgfpathmoveto{\pgfqpoint{1.799835in}{1.020152in}}%
\pgfpathlineto{\pgfqpoint{1.780925in}{1.020880in}}%
\pgfpathlineto{\pgfqpoint{1.781161in}{1.040456in}}%
\pgfpathlineto{\pgfqpoint{1.735938in}{1.042436in}}%
\pgfpathlineto{\pgfqpoint{1.736360in}{1.057495in}}%
\pgfpathlineto{\pgfqpoint{1.736839in}{1.068841in}}%
\pgfpathlineto{\pgfqpoint{1.762618in}{1.067453in}}%
\pgfpathlineto{\pgfqpoint{1.763799in}{1.100240in}}%
\pgfpathlineto{\pgfqpoint{1.753609in}{1.114508in}}%
\pgfpathlineto{\pgfqpoint{1.746816in}{1.114997in}}%
\pgfpathlineto{\pgfqpoint{1.739867in}{1.122447in}}%
\pgfpathlineto{\pgfqpoint{1.735430in}{1.131613in}}%
\pgfpathlineto{\pgfqpoint{1.762663in}{1.130354in}}%
\pgfpathlineto{\pgfqpoint{1.801759in}{1.128650in}}%
\pgfpathlineto{\pgfqpoint{1.854713in}{1.127012in}}%
\pgfpathlineto{\pgfqpoint{1.853928in}{1.096638in}}%
\pgfpathlineto{\pgfqpoint{1.853111in}{1.064475in}}%
\pgfpathlineto{\pgfqpoint{1.807728in}{1.065923in}}%
\pgfpathlineto{\pgfqpoint{1.806768in}{1.032943in}}%
\pgfpathlineto{\pgfqpoint{1.800303in}{1.033179in}}%
\pgfpathclose%
\pgfusepath{fill}%
\end{pgfscope}%
\begin{pgfscope}%
\pgfpathrectangle{\pgfqpoint{0.100000in}{0.100000in}}{\pgfqpoint{3.420221in}{2.189500in}}%
\pgfusepath{clip}%
\pgfsetbuttcap%
\pgfsetmiterjoin%
\definecolor{currentfill}{rgb}{0.000000,0.666667,0.666667}%
\pgfsetfillcolor{currentfill}%
\pgfsetlinewidth{0.000000pt}%
\definecolor{currentstroke}{rgb}{0.000000,0.000000,0.000000}%
\pgfsetstrokecolor{currentstroke}%
\pgfsetstrokeopacity{0.000000}%
\pgfsetdash{}{0pt}%
\pgfpathmoveto{\pgfqpoint{1.017964in}{1.370775in}}%
\pgfpathlineto{\pgfqpoint{1.013814in}{1.364877in}}%
\pgfpathlineto{\pgfqpoint{1.009974in}{1.365535in}}%
\pgfpathlineto{\pgfqpoint{1.003381in}{1.359811in}}%
\pgfpathlineto{\pgfqpoint{1.001796in}{1.349628in}}%
\pgfpathlineto{\pgfqpoint{0.995833in}{1.346850in}}%
\pgfpathlineto{\pgfqpoint{0.987984in}{1.348746in}}%
\pgfpathlineto{\pgfqpoint{0.982552in}{1.341497in}}%
\pgfpathlineto{\pgfqpoint{0.953379in}{1.347413in}}%
\pgfpathlineto{\pgfqpoint{0.894470in}{1.359506in}}%
\pgfpathlineto{\pgfqpoint{0.896116in}{1.367156in}}%
\pgfpathlineto{\pgfqpoint{0.909727in}{1.430544in}}%
\pgfpathlineto{\pgfqpoint{0.915423in}{1.457160in}}%
\pgfpathlineto{\pgfqpoint{0.985709in}{1.442499in}}%
\pgfpathlineto{\pgfqpoint{1.011444in}{1.437538in}}%
\pgfpathlineto{\pgfqpoint{1.022099in}{1.443683in}}%
\pgfpathlineto{\pgfqpoint{1.026340in}{1.455124in}}%
\pgfpathlineto{\pgfqpoint{1.024958in}{1.462399in}}%
\pgfpathlineto{\pgfqpoint{1.026582in}{1.476974in}}%
\pgfpathlineto{\pgfqpoint{1.036407in}{1.475513in}}%
\pgfpathlineto{\pgfqpoint{1.038853in}{1.471193in}}%
\pgfpathlineto{\pgfqpoint{1.052700in}{1.477124in}}%
\pgfpathlineto{\pgfqpoint{1.063893in}{1.476679in}}%
\pgfpathlineto{\pgfqpoint{1.059590in}{1.470024in}}%
\pgfpathlineto{\pgfqpoint{1.066969in}{1.456748in}}%
\pgfpathlineto{\pgfqpoint{1.075510in}{1.453491in}}%
\pgfpathlineto{\pgfqpoint{1.077995in}{1.445072in}}%
\pgfpathlineto{\pgfqpoint{1.075183in}{1.437137in}}%
\pgfpathlineto{\pgfqpoint{1.083058in}{1.426908in}}%
\pgfpathlineto{\pgfqpoint{1.082517in}{1.424000in}}%
\pgfpathlineto{\pgfqpoint{1.093336in}{1.422040in}}%
\pgfpathlineto{\pgfqpoint{1.094137in}{1.415334in}}%
\pgfpathlineto{\pgfqpoint{1.049714in}{1.423341in}}%
\pgfpathlineto{\pgfqpoint{1.052560in}{1.420586in}}%
\pgfpathlineto{\pgfqpoint{1.043765in}{1.415423in}}%
\pgfpathlineto{\pgfqpoint{1.040857in}{1.398092in}}%
\pgfpathlineto{\pgfqpoint{1.037871in}{1.392760in}}%
\pgfpathlineto{\pgfqpoint{1.027870in}{1.394529in}}%
\pgfpathlineto{\pgfqpoint{1.021947in}{1.392243in}}%
\pgfpathclose%
\pgfusepath{fill}%
\end{pgfscope}%
\begin{pgfscope}%
\pgfpathrectangle{\pgfqpoint{0.100000in}{0.100000in}}{\pgfqpoint{3.420221in}{2.189500in}}%
\pgfusepath{clip}%
\pgfsetbuttcap%
\pgfsetmiterjoin%
\definecolor{currentfill}{rgb}{0.000000,0.403922,0.798039}%
\pgfsetfillcolor{currentfill}%
\pgfsetlinewidth{0.000000pt}%
\definecolor{currentstroke}{rgb}{0.000000,0.000000,0.000000}%
\pgfsetstrokecolor{currentstroke}%
\pgfsetstrokeopacity{0.000000}%
\pgfsetdash{}{0pt}%
\pgfpathmoveto{\pgfqpoint{2.688576in}{1.470632in}}%
\pgfpathlineto{\pgfqpoint{2.683233in}{1.502705in}}%
\pgfpathlineto{\pgfqpoint{2.677075in}{1.503983in}}%
\pgfpathlineto{\pgfqpoint{2.691355in}{1.508921in}}%
\pgfpathlineto{\pgfqpoint{2.697515in}{1.503455in}}%
\pgfpathlineto{\pgfqpoint{2.704554in}{1.501973in}}%
\pgfpathlineto{\pgfqpoint{2.716203in}{1.507567in}}%
\pgfpathlineto{\pgfqpoint{2.728842in}{1.515518in}}%
\pgfpathlineto{\pgfqpoint{2.731399in}{1.515170in}}%
\pgfpathlineto{\pgfqpoint{2.732864in}{1.503652in}}%
\pgfpathlineto{\pgfqpoint{2.738125in}{1.504353in}}%
\pgfpathlineto{\pgfqpoint{2.738885in}{1.498746in}}%
\pgfpathlineto{\pgfqpoint{2.733610in}{1.498004in}}%
\pgfpathlineto{\pgfqpoint{2.734310in}{1.492383in}}%
\pgfpathlineto{\pgfqpoint{2.728837in}{1.491597in}}%
\pgfpathlineto{\pgfqpoint{2.724060in}{1.486161in}}%
\pgfpathlineto{\pgfqpoint{2.724745in}{1.480691in}}%
\pgfpathlineto{\pgfqpoint{2.709871in}{1.478759in}}%
\pgfpathlineto{\pgfqpoint{2.710846in}{1.473413in}}%
\pgfpathclose%
\pgfusepath{fill}%
\end{pgfscope}%
\begin{pgfscope}%
\pgfpathrectangle{\pgfqpoint{0.100000in}{0.100000in}}{\pgfqpoint{3.420221in}{2.189500in}}%
\pgfusepath{clip}%
\pgfsetbuttcap%
\pgfsetmiterjoin%
\definecolor{currentfill}{rgb}{0.000000,0.443137,0.778431}%
\pgfsetfillcolor{currentfill}%
\pgfsetlinewidth{0.000000pt}%
\definecolor{currentstroke}{rgb}{0.000000,0.000000,0.000000}%
\pgfsetstrokecolor{currentstroke}%
\pgfsetstrokeopacity{0.000000}%
\pgfsetdash{}{0pt}%
\pgfpathmoveto{\pgfqpoint{1.532672in}{1.864929in}}%
\pgfpathlineto{\pgfqpoint{1.528185in}{1.820168in}}%
\pgfpathlineto{\pgfqpoint{1.523703in}{1.822411in}}%
\pgfpathlineto{\pgfqpoint{1.510345in}{1.823880in}}%
\pgfpathlineto{\pgfqpoint{1.511034in}{1.830234in}}%
\pgfpathlineto{\pgfqpoint{1.504467in}{1.830892in}}%
\pgfpathlineto{\pgfqpoint{1.505134in}{1.837027in}}%
\pgfpathlineto{\pgfqpoint{1.485889in}{1.839116in}}%
\pgfpathlineto{\pgfqpoint{1.487686in}{1.865172in}}%
\pgfpathlineto{\pgfqpoint{1.496625in}{1.864207in}}%
\pgfpathlineto{\pgfqpoint{1.498043in}{1.877284in}}%
\pgfpathlineto{\pgfqpoint{1.504764in}{1.878745in}}%
\pgfpathlineto{\pgfqpoint{1.514524in}{1.877717in}}%
\pgfpathlineto{\pgfqpoint{1.517484in}{1.874108in}}%
\pgfpathlineto{\pgfqpoint{1.533419in}{1.872490in}}%
\pgfpathclose%
\pgfusepath{fill}%
\end{pgfscope}%
\begin{pgfscope}%
\pgfpathrectangle{\pgfqpoint{0.100000in}{0.100000in}}{\pgfqpoint{3.420221in}{2.189500in}}%
\pgfusepath{clip}%
\pgfsetbuttcap%
\pgfsetmiterjoin%
\definecolor{currentfill}{rgb}{0.000000,0.541176,0.729412}%
\pgfsetfillcolor{currentfill}%
\pgfsetlinewidth{0.000000pt}%
\definecolor{currentstroke}{rgb}{0.000000,0.000000,0.000000}%
\pgfsetstrokecolor{currentstroke}%
\pgfsetstrokeopacity{0.000000}%
\pgfsetdash{}{0pt}%
\pgfpathmoveto{\pgfqpoint{3.131509in}{1.316108in}}%
\pgfpathlineto{\pgfqpoint{3.124754in}{1.312375in}}%
\pgfpathlineto{\pgfqpoint{3.122026in}{1.316279in}}%
\pgfpathlineto{\pgfqpoint{3.124586in}{1.321814in}}%
\pgfpathlineto{\pgfqpoint{3.122994in}{1.328863in}}%
\pgfpathlineto{\pgfqpoint{3.115981in}{1.327393in}}%
\pgfpathlineto{\pgfqpoint{3.116341in}{1.335878in}}%
\pgfpathlineto{\pgfqpoint{3.115322in}{1.340266in}}%
\pgfpathlineto{\pgfqpoint{3.118418in}{1.354310in}}%
\pgfpathlineto{\pgfqpoint{3.124163in}{1.361653in}}%
\pgfpathlineto{\pgfqpoint{3.109232in}{1.414832in}}%
\pgfpathlineto{\pgfqpoint{3.115476in}{1.415806in}}%
\pgfpathlineto{\pgfqpoint{3.122136in}{1.422988in}}%
\pgfpathlineto{\pgfqpoint{3.127172in}{1.419991in}}%
\pgfpathlineto{\pgfqpoint{3.131580in}{1.412544in}}%
\pgfpathlineto{\pgfqpoint{3.133224in}{1.402446in}}%
\pgfpathlineto{\pgfqpoint{3.137391in}{1.399849in}}%
\pgfpathlineto{\pgfqpoint{3.141127in}{1.392496in}}%
\pgfpathlineto{\pgfqpoint{3.149696in}{1.385881in}}%
\pgfpathlineto{\pgfqpoint{3.156163in}{1.384882in}}%
\pgfpathlineto{\pgfqpoint{3.163292in}{1.361840in}}%
\pgfpathlineto{\pgfqpoint{3.160750in}{1.334009in}}%
\pgfpathlineto{\pgfqpoint{3.159060in}{1.328324in}}%
\pgfpathlineto{\pgfqpoint{3.137566in}{1.321122in}}%
\pgfpathclose%
\pgfusepath{fill}%
\end{pgfscope}%
\begin{pgfscope}%
\pgfpathrectangle{\pgfqpoint{0.100000in}{0.100000in}}{\pgfqpoint{3.420221in}{2.189500in}}%
\pgfusepath{clip}%
\pgfsetbuttcap%
\pgfsetmiterjoin%
\definecolor{currentfill}{rgb}{0.000000,0.372549,0.813725}%
\pgfsetfillcolor{currentfill}%
\pgfsetlinewidth{0.000000pt}%
\definecolor{currentstroke}{rgb}{0.000000,0.000000,0.000000}%
\pgfsetstrokecolor{currentstroke}%
\pgfsetstrokeopacity{0.000000}%
\pgfsetdash{}{0pt}%
\pgfpathmoveto{\pgfqpoint{2.009974in}{1.515561in}}%
\pgfpathlineto{\pgfqpoint{1.964818in}{1.515815in}}%
\pgfpathlineto{\pgfqpoint{1.965036in}{1.535579in}}%
\pgfpathlineto{\pgfqpoint{1.961113in}{1.542111in}}%
\pgfpathlineto{\pgfqpoint{1.954558in}{1.542144in}}%
\pgfpathlineto{\pgfqpoint{1.954833in}{1.568286in}}%
\pgfpathlineto{\pgfqpoint{1.980763in}{1.568157in}}%
\pgfpathlineto{\pgfqpoint{2.006833in}{1.568114in}}%
\pgfpathlineto{\pgfqpoint{2.006829in}{1.535419in}}%
\pgfpathlineto{\pgfqpoint{2.010033in}{1.535401in}}%
\pgfpathclose%
\pgfusepath{fill}%
\end{pgfscope}%
\begin{pgfscope}%
\pgfpathrectangle{\pgfqpoint{0.100000in}{0.100000in}}{\pgfqpoint{3.420221in}{2.189500in}}%
\pgfusepath{clip}%
\pgfsetbuttcap%
\pgfsetmiterjoin%
\definecolor{currentfill}{rgb}{0.000000,0.662745,0.668627}%
\pgfsetfillcolor{currentfill}%
\pgfsetlinewidth{0.000000pt}%
\definecolor{currentstroke}{rgb}{0.000000,0.000000,0.000000}%
\pgfsetstrokecolor{currentstroke}%
\pgfsetstrokeopacity{0.000000}%
\pgfsetdash{}{0pt}%
\pgfpathmoveto{\pgfqpoint{1.137086in}{1.381237in}}%
\pgfpathlineto{\pgfqpoint{1.189457in}{1.372792in}}%
\pgfpathlineto{\pgfqpoint{1.193046in}{1.374938in}}%
\pgfpathlineto{\pgfqpoint{1.191570in}{1.365236in}}%
\pgfpathlineto{\pgfqpoint{1.181037in}{1.301162in}}%
\pgfpathlineto{\pgfqpoint{1.125155in}{1.310182in}}%
\pgfpathlineto{\pgfqpoint{1.126850in}{1.313229in}}%
\pgfpathlineto{\pgfqpoint{1.122918in}{1.328024in}}%
\pgfpathlineto{\pgfqpoint{1.124740in}{1.330810in}}%
\pgfpathlineto{\pgfqpoint{1.121754in}{1.338399in}}%
\pgfpathlineto{\pgfqpoint{1.125647in}{1.354636in}}%
\pgfpathlineto{\pgfqpoint{1.132517in}{1.364345in}}%
\pgfpathlineto{\pgfqpoint{1.131822in}{1.368310in}}%
\pgfpathlineto{\pgfqpoint{1.135847in}{1.373114in}}%
\pgfpathclose%
\pgfusepath{fill}%
\end{pgfscope}%
\begin{pgfscope}%
\pgfpathrectangle{\pgfqpoint{0.100000in}{0.100000in}}{\pgfqpoint{3.420221in}{2.189500in}}%
\pgfusepath{clip}%
\pgfsetbuttcap%
\pgfsetmiterjoin%
\definecolor{currentfill}{rgb}{0.000000,0.615686,0.692157}%
\pgfsetfillcolor{currentfill}%
\pgfsetlinewidth{0.000000pt}%
\definecolor{currentstroke}{rgb}{0.000000,0.000000,0.000000}%
\pgfsetstrokecolor{currentstroke}%
\pgfsetstrokeopacity{0.000000}%
\pgfsetdash{}{0pt}%
\pgfpathmoveto{\pgfqpoint{2.641045in}{0.977578in}}%
\pgfpathlineto{\pgfqpoint{2.639341in}{0.990993in}}%
\pgfpathlineto{\pgfqpoint{2.642938in}{0.998657in}}%
\pgfpathlineto{\pgfqpoint{2.640350in}{1.000805in}}%
\pgfpathlineto{\pgfqpoint{2.639305in}{1.010604in}}%
\pgfpathlineto{\pgfqpoint{2.657615in}{1.012741in}}%
\pgfpathlineto{\pgfqpoint{2.667025in}{1.011688in}}%
\pgfpathlineto{\pgfqpoint{2.673264in}{1.000477in}}%
\pgfpathlineto{\pgfqpoint{2.670692in}{0.988606in}}%
\pgfpathlineto{\pgfqpoint{2.669197in}{0.984930in}}%
\pgfpathlineto{\pgfqpoint{2.660754in}{0.982495in}}%
\pgfpathlineto{\pgfqpoint{2.658448in}{0.979561in}}%
\pgfpathclose%
\pgfusepath{fill}%
\end{pgfscope}%
\begin{pgfscope}%
\pgfpathrectangle{\pgfqpoint{0.100000in}{0.100000in}}{\pgfqpoint{3.420221in}{2.189500in}}%
\pgfusepath{clip}%
\pgfsetbuttcap%
\pgfsetmiterjoin%
\definecolor{currentfill}{rgb}{0.000000,0.843137,0.578431}%
\pgfsetfillcolor{currentfill}%
\pgfsetlinewidth{0.000000pt}%
\definecolor{currentstroke}{rgb}{0.000000,0.000000,0.000000}%
\pgfsetstrokecolor{currentstroke}%
\pgfsetstrokeopacity{0.000000}%
\pgfsetdash{}{0pt}%
\pgfpathmoveto{\pgfqpoint{0.410259in}{1.831955in}}%
\pgfpathlineto{\pgfqpoint{0.422913in}{1.852856in}}%
\pgfpathlineto{\pgfqpoint{0.437584in}{1.870890in}}%
\pgfpathlineto{\pgfqpoint{0.448546in}{1.891138in}}%
\pgfpathlineto{\pgfqpoint{0.459487in}{1.887476in}}%
\pgfpathlineto{\pgfqpoint{0.467490in}{1.891357in}}%
\pgfpathlineto{\pgfqpoint{0.477629in}{1.886070in}}%
\pgfpathlineto{\pgfqpoint{0.480168in}{1.878077in}}%
\pgfpathlineto{\pgfqpoint{0.488394in}{1.871520in}}%
\pgfpathlineto{\pgfqpoint{0.499234in}{1.867999in}}%
\pgfpathlineto{\pgfqpoint{0.495287in}{1.855554in}}%
\pgfpathlineto{\pgfqpoint{0.495312in}{1.850341in}}%
\pgfpathlineto{\pgfqpoint{0.514402in}{1.844642in}}%
\pgfpathlineto{\pgfqpoint{0.512029in}{1.836928in}}%
\pgfpathlineto{\pgfqpoint{0.543810in}{1.827290in}}%
\pgfpathlineto{\pgfqpoint{0.547842in}{1.819890in}}%
\pgfpathlineto{\pgfqpoint{0.547870in}{1.812283in}}%
\pgfpathlineto{\pgfqpoint{0.540923in}{1.805575in}}%
\pgfpathlineto{\pgfqpoint{0.537902in}{1.800390in}}%
\pgfpathlineto{\pgfqpoint{0.527696in}{1.802948in}}%
\pgfpathlineto{\pgfqpoint{0.526142in}{1.797850in}}%
\pgfpathlineto{\pgfqpoint{0.519917in}{1.799771in}}%
\pgfpathlineto{\pgfqpoint{0.515527in}{1.796686in}}%
\pgfpathlineto{\pgfqpoint{0.510253in}{1.798280in}}%
\pgfpathlineto{\pgfqpoint{0.494344in}{1.790708in}}%
\pgfpathlineto{\pgfqpoint{0.482457in}{1.792328in}}%
\pgfpathlineto{\pgfqpoint{0.475081in}{1.794998in}}%
\pgfpathlineto{\pgfqpoint{0.471077in}{1.793199in}}%
\pgfpathlineto{\pgfqpoint{0.463269in}{1.796894in}}%
\pgfpathlineto{\pgfqpoint{0.458498in}{1.796156in}}%
\pgfpathlineto{\pgfqpoint{0.446110in}{1.806553in}}%
\pgfpathlineto{\pgfqpoint{0.437881in}{1.808534in}}%
\pgfpathlineto{\pgfqpoint{0.429100in}{1.804386in}}%
\pgfpathlineto{\pgfqpoint{0.421315in}{1.805793in}}%
\pgfpathlineto{\pgfqpoint{0.425378in}{1.819401in}}%
\pgfpathlineto{\pgfqpoint{0.421872in}{1.828075in}}%
\pgfpathclose%
\pgfusepath{fill}%
\end{pgfscope}%
\begin{pgfscope}%
\pgfpathrectangle{\pgfqpoint{0.100000in}{0.100000in}}{\pgfqpoint{3.420221in}{2.189500in}}%
\pgfusepath{clip}%
\pgfsetbuttcap%
\pgfsetmiterjoin%
\definecolor{currentfill}{rgb}{0.000000,0.568627,0.715686}%
\pgfsetfillcolor{currentfill}%
\pgfsetlinewidth{0.000000pt}%
\definecolor{currentstroke}{rgb}{0.000000,0.000000,0.000000}%
\pgfsetstrokecolor{currentstroke}%
\pgfsetstrokeopacity{0.000000}%
\pgfsetdash{}{0pt}%
\pgfpathmoveto{\pgfqpoint{2.949616in}{0.580603in}}%
\pgfpathlineto{\pgfqpoint{2.934062in}{0.578133in}}%
\pgfpathlineto{\pgfqpoint{2.934858in}{0.564607in}}%
\pgfpathlineto{\pgfqpoint{2.925488in}{0.579457in}}%
\pgfpathlineto{\pgfqpoint{2.922950in}{0.576569in}}%
\pgfpathlineto{\pgfqpoint{2.916566in}{0.578690in}}%
\pgfpathlineto{\pgfqpoint{2.910342in}{0.577922in}}%
\pgfpathlineto{\pgfqpoint{2.906249in}{0.589014in}}%
\pgfpathlineto{\pgfqpoint{2.895556in}{0.595802in}}%
\pgfpathlineto{\pgfqpoint{2.892525in}{0.601567in}}%
\pgfpathlineto{\pgfqpoint{2.886992in}{0.602738in}}%
\pgfpathlineto{\pgfqpoint{2.884207in}{0.608000in}}%
\pgfpathlineto{\pgfqpoint{2.876644in}{0.614106in}}%
\pgfpathlineto{\pgfqpoint{2.873084in}{0.622520in}}%
\pgfpathlineto{\pgfqpoint{2.868347in}{0.624403in}}%
\pgfpathlineto{\pgfqpoint{2.854959in}{0.618679in}}%
\pgfpathlineto{\pgfqpoint{2.852731in}{0.637305in}}%
\pgfpathlineto{\pgfqpoint{2.859543in}{0.640495in}}%
\pgfpathlineto{\pgfqpoint{2.866882in}{0.648387in}}%
\pgfpathlineto{\pgfqpoint{2.881929in}{0.650846in}}%
\pgfpathlineto{\pgfqpoint{2.886443in}{0.645363in}}%
\pgfpathlineto{\pgfqpoint{2.888080in}{0.635097in}}%
\pgfpathlineto{\pgfqpoint{2.901069in}{0.637292in}}%
\pgfpathlineto{\pgfqpoint{2.907801in}{0.641775in}}%
\pgfpathlineto{\pgfqpoint{2.922851in}{0.616222in}}%
\pgfpathlineto{\pgfqpoint{2.930496in}{0.604306in}}%
\pgfpathclose%
\pgfusepath{fill}%
\end{pgfscope}%
\begin{pgfscope}%
\pgfpathrectangle{\pgfqpoint{0.100000in}{0.100000in}}{\pgfqpoint{3.420221in}{2.189500in}}%
\pgfusepath{clip}%
\pgfsetbuttcap%
\pgfsetmiterjoin%
\definecolor{currentfill}{rgb}{0.000000,0.352941,0.823529}%
\pgfsetfillcolor{currentfill}%
\pgfsetlinewidth{0.000000pt}%
\definecolor{currentstroke}{rgb}{0.000000,0.000000,0.000000}%
\pgfsetstrokecolor{currentstroke}%
\pgfsetstrokeopacity{0.000000}%
\pgfsetdash{}{0pt}%
\pgfpathmoveto{\pgfqpoint{0.871604in}{1.038223in}}%
\pgfpathlineto{\pgfqpoint{0.858871in}{1.039923in}}%
\pgfpathlineto{\pgfqpoint{0.853701in}{1.035118in}}%
\pgfpathlineto{\pgfqpoint{0.828314in}{1.042813in}}%
\pgfpathlineto{\pgfqpoint{0.822469in}{1.047375in}}%
\pgfpathlineto{\pgfqpoint{0.812720in}{1.060824in}}%
\pgfpathlineto{\pgfqpoint{0.809020in}{1.072788in}}%
\pgfpathlineto{\pgfqpoint{0.808809in}{1.081764in}}%
\pgfpathlineto{\pgfqpoint{0.804756in}{1.086851in}}%
\pgfpathlineto{\pgfqpoint{0.801658in}{1.096929in}}%
\pgfpathlineto{\pgfqpoint{0.803701in}{1.105168in}}%
\pgfpathlineto{\pgfqpoint{0.756481in}{1.177901in}}%
\pgfpathlineto{\pgfqpoint{0.744934in}{1.195624in}}%
\pgfpathlineto{\pgfqpoint{0.687663in}{1.284051in}}%
\pgfpathlineto{\pgfqpoint{0.658152in}{1.329689in}}%
\pgfpathlineto{\pgfqpoint{0.632115in}{1.369859in}}%
\pgfpathlineto{\pgfqpoint{0.636451in}{1.368523in}}%
\pgfpathlineto{\pgfqpoint{0.684999in}{1.400852in}}%
\pgfpathlineto{\pgfqpoint{0.664727in}{1.440531in}}%
\pgfpathlineto{\pgfqpoint{0.666255in}{1.446394in}}%
\pgfpathlineto{\pgfqpoint{0.686294in}{1.446910in}}%
\pgfpathlineto{\pgfqpoint{0.691711in}{1.447049in}}%
\pgfpathlineto{\pgfqpoint{0.717990in}{1.445800in}}%
\pgfpathlineto{\pgfqpoint{0.759175in}{1.435495in}}%
\pgfpathlineto{\pgfqpoint{0.798389in}{1.426183in}}%
\pgfpathlineto{\pgfqpoint{0.841775in}{1.379011in}}%
\pgfpathlineto{\pgfqpoint{0.896116in}{1.367156in}}%
\pgfpathlineto{\pgfqpoint{0.894470in}{1.359506in}}%
\pgfpathlineto{\pgfqpoint{0.882667in}{1.303530in}}%
\pgfpathlineto{\pgfqpoint{0.869922in}{1.244248in}}%
\pgfpathlineto{\pgfqpoint{0.937424in}{1.230332in}}%
\pgfpathlineto{\pgfqpoint{0.958514in}{1.226188in}}%
\pgfpathlineto{\pgfqpoint{0.948035in}{1.203324in}}%
\pgfpathlineto{\pgfqpoint{0.944684in}{1.191769in}}%
\pgfpathlineto{\pgfqpoint{0.944986in}{1.183466in}}%
\pgfpathlineto{\pgfqpoint{0.940664in}{1.179335in}}%
\pgfpathlineto{\pgfqpoint{0.932897in}{1.176449in}}%
\pgfpathlineto{\pgfqpoint{0.916713in}{1.173988in}}%
\pgfpathlineto{\pgfqpoint{0.907012in}{1.171373in}}%
\pgfpathlineto{\pgfqpoint{0.904734in}{1.167363in}}%
\pgfpathlineto{\pgfqpoint{0.899610in}{1.169085in}}%
\pgfpathlineto{\pgfqpoint{0.896293in}{1.165197in}}%
\pgfpathlineto{\pgfqpoint{0.897283in}{1.158903in}}%
\pgfpathlineto{\pgfqpoint{0.893922in}{1.147506in}}%
\pgfpathlineto{\pgfqpoint{0.889779in}{1.127288in}}%
\pgfpathclose%
\pgfusepath{fill}%
\end{pgfscope}%
\begin{pgfscope}%
\pgfpathrectangle{\pgfqpoint{0.100000in}{0.100000in}}{\pgfqpoint{3.420221in}{2.189500in}}%
\pgfusepath{clip}%
\pgfsetbuttcap%
\pgfsetmiterjoin%
\definecolor{currentfill}{rgb}{0.000000,0.443137,0.778431}%
\pgfsetfillcolor{currentfill}%
\pgfsetlinewidth{0.000000pt}%
\definecolor{currentstroke}{rgb}{0.000000,0.000000,0.000000}%
\pgfsetstrokecolor{currentstroke}%
\pgfsetstrokeopacity{0.000000}%
\pgfsetdash{}{0pt}%
\pgfpathmoveto{\pgfqpoint{2.807256in}{1.424511in}}%
\pgfpathlineto{\pgfqpoint{2.787724in}{1.422292in}}%
\pgfpathlineto{\pgfqpoint{2.780881in}{1.424599in}}%
\pgfpathlineto{\pgfqpoint{2.762144in}{1.422418in}}%
\pgfpathlineto{\pgfqpoint{2.762739in}{1.417091in}}%
\pgfpathlineto{\pgfqpoint{2.733296in}{1.414033in}}%
\pgfpathlineto{\pgfqpoint{2.732055in}{1.419319in}}%
\pgfpathlineto{\pgfqpoint{2.729508in}{1.444244in}}%
\pgfpathlineto{\pgfqpoint{2.727229in}{1.443546in}}%
\pgfpathlineto{\pgfqpoint{2.726153in}{1.450916in}}%
\pgfpathlineto{\pgfqpoint{2.731463in}{1.451719in}}%
\pgfpathlineto{\pgfqpoint{2.727854in}{1.475723in}}%
\pgfpathlineto{\pgfqpoint{2.754890in}{1.479339in}}%
\pgfpathlineto{\pgfqpoint{2.755670in}{1.473877in}}%
\pgfpathlineto{\pgfqpoint{2.768512in}{1.475211in}}%
\pgfpathlineto{\pgfqpoint{2.767812in}{1.481306in}}%
\pgfpathlineto{\pgfqpoint{2.786247in}{1.484093in}}%
\pgfpathlineto{\pgfqpoint{2.789195in}{1.464808in}}%
\pgfpathlineto{\pgfqpoint{2.798753in}{1.466221in}}%
\pgfpathlineto{\pgfqpoint{2.799665in}{1.460032in}}%
\pgfpathlineto{\pgfqpoint{2.803330in}{1.457263in}}%
\pgfpathlineto{\pgfqpoint{2.800516in}{1.453554in}}%
\pgfpathlineto{\pgfqpoint{2.800878in}{1.443628in}}%
\pgfpathlineto{\pgfqpoint{2.805182in}{1.444167in}}%
\pgfpathclose%
\pgfusepath{fill}%
\end{pgfscope}%
\begin{pgfscope}%
\pgfpathrectangle{\pgfqpoint{0.100000in}{0.100000in}}{\pgfqpoint{3.420221in}{2.189500in}}%
\pgfusepath{clip}%
\pgfsetbuttcap%
\pgfsetmiterjoin%
\definecolor{currentfill}{rgb}{0.000000,0.525490,0.737255}%
\pgfsetfillcolor{currentfill}%
\pgfsetlinewidth{0.000000pt}%
\definecolor{currentstroke}{rgb}{0.000000,0.000000,0.000000}%
\pgfsetstrokecolor{currentstroke}%
\pgfsetstrokeopacity{0.000000}%
\pgfsetdash{}{0pt}%
\pgfpathmoveto{\pgfqpoint{2.845785in}{1.145513in}}%
\pgfpathlineto{\pgfqpoint{2.852533in}{1.140854in}}%
\pgfpathlineto{\pgfqpoint{2.854882in}{1.120667in}}%
\pgfpathlineto{\pgfqpoint{2.842253in}{1.116623in}}%
\pgfpathlineto{\pgfqpoint{2.836592in}{1.116409in}}%
\pgfpathlineto{\pgfqpoint{2.828801in}{1.112156in}}%
\pgfpathlineto{\pgfqpoint{2.823048in}{1.117017in}}%
\pgfpathlineto{\pgfqpoint{2.814436in}{1.119164in}}%
\pgfpathlineto{\pgfqpoint{2.818512in}{1.126411in}}%
\pgfpathlineto{\pgfqpoint{2.806325in}{1.135560in}}%
\pgfpathlineto{\pgfqpoint{2.799777in}{1.138089in}}%
\pgfpathlineto{\pgfqpoint{2.801618in}{1.143842in}}%
\pgfpathlineto{\pgfqpoint{2.801362in}{1.152916in}}%
\pgfpathlineto{\pgfqpoint{2.820818in}{1.154808in}}%
\pgfpathlineto{\pgfqpoint{2.821021in}{1.152797in}}%
\pgfpathlineto{\pgfqpoint{2.830662in}{1.141269in}}%
\pgfpathlineto{\pgfqpoint{2.837275in}{1.146680in}}%
\pgfpathclose%
\pgfusepath{fill}%
\end{pgfscope}%
\begin{pgfscope}%
\pgfpathrectangle{\pgfqpoint{0.100000in}{0.100000in}}{\pgfqpoint{3.420221in}{2.189500in}}%
\pgfusepath{clip}%
\pgfsetbuttcap%
\pgfsetmiterjoin%
\definecolor{currentfill}{rgb}{0.000000,0.266667,0.866667}%
\pgfsetfillcolor{currentfill}%
\pgfsetlinewidth{0.000000pt}%
\definecolor{currentstroke}{rgb}{0.000000,0.000000,0.000000}%
\pgfsetstrokecolor{currentstroke}%
\pgfsetstrokeopacity{0.000000}%
\pgfsetdash{}{0pt}%
\pgfpathmoveto{\pgfqpoint{1.574663in}{1.518290in}}%
\pgfpathlineto{\pgfqpoint{1.608628in}{1.515823in}}%
\pgfpathlineto{\pgfqpoint{1.607001in}{1.495855in}}%
\pgfpathlineto{\pgfqpoint{1.611599in}{1.495513in}}%
\pgfpathlineto{\pgfqpoint{1.609647in}{1.469522in}}%
\pgfpathlineto{\pgfqpoint{1.605689in}{1.469862in}}%
\pgfpathlineto{\pgfqpoint{1.604664in}{1.456886in}}%
\pgfpathlineto{\pgfqpoint{1.573482in}{1.459420in}}%
\pgfpathlineto{\pgfqpoint{1.574624in}{1.472371in}}%
\pgfpathlineto{\pgfqpoint{1.573370in}{1.475782in}}%
\pgfpathlineto{\pgfqpoint{1.575003in}{1.496376in}}%
\pgfpathlineto{\pgfqpoint{1.573431in}{1.504659in}}%
\pgfpathclose%
\pgfusepath{fill}%
\end{pgfscope}%
\begin{pgfscope}%
\pgfpathrectangle{\pgfqpoint{0.100000in}{0.100000in}}{\pgfqpoint{3.420221in}{2.189500in}}%
\pgfusepath{clip}%
\pgfsetbuttcap%
\pgfsetmiterjoin%
\definecolor{currentfill}{rgb}{0.000000,0.615686,0.692157}%
\pgfsetfillcolor{currentfill}%
\pgfsetlinewidth{0.000000pt}%
\definecolor{currentstroke}{rgb}{0.000000,0.000000,0.000000}%
\pgfsetstrokecolor{currentstroke}%
\pgfsetstrokeopacity{0.000000}%
\pgfsetdash{}{0pt}%
\pgfpathmoveto{\pgfqpoint{2.586134in}{1.128384in}}%
\pgfpathlineto{\pgfqpoint{2.585532in}{1.120825in}}%
\pgfpathlineto{\pgfqpoint{2.577397in}{1.117996in}}%
\pgfpathlineto{\pgfqpoint{2.574770in}{1.110337in}}%
\pgfpathlineto{\pgfqpoint{2.569396in}{1.114925in}}%
\pgfpathlineto{\pgfqpoint{2.561330in}{1.117929in}}%
\pgfpathlineto{\pgfqpoint{2.554920in}{1.115547in}}%
\pgfpathlineto{\pgfqpoint{2.555100in}{1.124163in}}%
\pgfpathlineto{\pgfqpoint{2.544205in}{1.124211in}}%
\pgfpathlineto{\pgfqpoint{2.542992in}{1.132576in}}%
\pgfpathlineto{\pgfqpoint{2.530590in}{1.145864in}}%
\pgfpathlineto{\pgfqpoint{2.535228in}{1.155311in}}%
\pgfpathlineto{\pgfqpoint{2.535530in}{1.167564in}}%
\pgfpathlineto{\pgfqpoint{2.528212in}{1.175891in}}%
\pgfpathlineto{\pgfqpoint{2.532023in}{1.177429in}}%
\pgfpathlineto{\pgfqpoint{2.533899in}{1.185051in}}%
\pgfpathlineto{\pgfqpoint{2.557212in}{1.185310in}}%
\pgfpathlineto{\pgfqpoint{2.555754in}{1.176086in}}%
\pgfpathlineto{\pgfqpoint{2.557312in}{1.167238in}}%
\pgfpathlineto{\pgfqpoint{2.560946in}{1.163313in}}%
\pgfpathlineto{\pgfqpoint{2.567375in}{1.162795in}}%
\pgfpathlineto{\pgfqpoint{2.569575in}{1.155323in}}%
\pgfpathlineto{\pgfqpoint{2.573177in}{1.150506in}}%
\pgfpathlineto{\pgfqpoint{2.586385in}{1.150927in}}%
\pgfpathlineto{\pgfqpoint{2.587935in}{1.145769in}}%
\pgfpathclose%
\pgfusepath{fill}%
\end{pgfscope}%
\begin{pgfscope}%
\pgfpathrectangle{\pgfqpoint{0.100000in}{0.100000in}}{\pgfqpoint{3.420221in}{2.189500in}}%
\pgfusepath{clip}%
\pgfsetbuttcap%
\pgfsetmiterjoin%
\definecolor{currentfill}{rgb}{0.000000,0.560784,0.719608}%
\pgfsetfillcolor{currentfill}%
\pgfsetlinewidth{0.000000pt}%
\definecolor{currentstroke}{rgb}{0.000000,0.000000,0.000000}%
\pgfsetstrokecolor{currentstroke}%
\pgfsetstrokeopacity{0.000000}%
\pgfsetdash{}{0pt}%
\pgfpathmoveto{\pgfqpoint{3.044436in}{1.116845in}}%
\pgfpathlineto{\pgfqpoint{3.029806in}{1.115037in}}%
\pgfpathlineto{\pgfqpoint{3.025904in}{1.115649in}}%
\pgfpathlineto{\pgfqpoint{3.020917in}{1.124162in}}%
\pgfpathlineto{\pgfqpoint{3.015195in}{1.130027in}}%
\pgfpathlineto{\pgfqpoint{3.024677in}{1.161202in}}%
\pgfpathlineto{\pgfqpoint{3.034061in}{1.158889in}}%
\pgfpathlineto{\pgfqpoint{3.044870in}{1.161366in}}%
\pgfpathlineto{\pgfqpoint{3.054402in}{1.158355in}}%
\pgfpathlineto{\pgfqpoint{3.055560in}{1.154512in}}%
\pgfpathlineto{\pgfqpoint{3.063421in}{1.153559in}}%
\pgfpathlineto{\pgfqpoint{3.068450in}{1.146933in}}%
\pgfpathlineto{\pgfqpoint{3.069212in}{1.140495in}}%
\pgfpathlineto{\pgfqpoint{3.066652in}{1.141039in}}%
\pgfpathclose%
\pgfusepath{fill}%
\end{pgfscope}%
\begin{pgfscope}%
\pgfpathrectangle{\pgfqpoint{0.100000in}{0.100000in}}{\pgfqpoint{3.420221in}{2.189500in}}%
\pgfusepath{clip}%
\pgfsetbuttcap%
\pgfsetmiterjoin%
\definecolor{currentfill}{rgb}{0.000000,0.392157,0.803922}%
\pgfsetfillcolor{currentfill}%
\pgfsetlinewidth{0.000000pt}%
\definecolor{currentstroke}{rgb}{0.000000,0.000000,0.000000}%
\pgfsetstrokecolor{currentstroke}%
\pgfsetstrokeopacity{0.000000}%
\pgfsetdash{}{0pt}%
\pgfpathmoveto{\pgfqpoint{1.001651in}{1.647471in}}%
\pgfpathlineto{\pgfqpoint{1.004430in}{1.644154in}}%
\pgfpathlineto{\pgfqpoint{1.013776in}{1.642167in}}%
\pgfpathlineto{\pgfqpoint{1.010036in}{1.622975in}}%
\pgfpathlineto{\pgfqpoint{1.005175in}{1.598755in}}%
\pgfpathlineto{\pgfqpoint{0.957238in}{1.607833in}}%
\pgfpathlineto{\pgfqpoint{0.948279in}{1.610168in}}%
\pgfpathlineto{\pgfqpoint{0.935196in}{1.613006in}}%
\pgfpathlineto{\pgfqpoint{0.941688in}{1.644080in}}%
\pgfpathlineto{\pgfqpoint{0.953399in}{1.641560in}}%
\pgfpathlineto{\pgfqpoint{0.955035in}{1.648683in}}%
\pgfpathlineto{\pgfqpoint{0.962678in}{1.648654in}}%
\pgfpathlineto{\pgfqpoint{0.966186in}{1.665540in}}%
\pgfpathlineto{\pgfqpoint{0.975251in}{1.663573in}}%
\pgfpathlineto{\pgfqpoint{0.979296in}{1.669284in}}%
\pgfpathlineto{\pgfqpoint{0.984634in}{1.695042in}}%
\pgfpathlineto{\pgfqpoint{1.000768in}{1.691681in}}%
\pgfpathlineto{\pgfqpoint{0.995507in}{1.665888in}}%
\pgfpathlineto{\pgfqpoint{0.992319in}{1.666542in}}%
\pgfpathlineto{\pgfqpoint{0.989590in}{1.653309in}}%
\pgfpathlineto{\pgfqpoint{0.995206in}{1.653570in}}%
\pgfpathclose%
\pgfusepath{fill}%
\end{pgfscope}%
\begin{pgfscope}%
\pgfpathrectangle{\pgfqpoint{0.100000in}{0.100000in}}{\pgfqpoint{3.420221in}{2.189500in}}%
\pgfusepath{clip}%
\pgfsetbuttcap%
\pgfsetmiterjoin%
\definecolor{currentfill}{rgb}{0.000000,0.415686,0.792157}%
\pgfsetfillcolor{currentfill}%
\pgfsetlinewidth{0.000000pt}%
\definecolor{currentstroke}{rgb}{0.000000,0.000000,0.000000}%
\pgfsetstrokecolor{currentstroke}%
\pgfsetstrokeopacity{0.000000}%
\pgfsetdash{}{0pt}%
\pgfpathmoveto{\pgfqpoint{2.587030in}{0.974669in}}%
\pgfpathlineto{\pgfqpoint{2.579166in}{1.003760in}}%
\pgfpathlineto{\pgfqpoint{2.563217in}{1.002398in}}%
\pgfpathlineto{\pgfqpoint{2.561499in}{1.019741in}}%
\pgfpathlineto{\pgfqpoint{2.558005in}{1.024559in}}%
\pgfpathlineto{\pgfqpoint{2.559539in}{1.029238in}}%
\pgfpathlineto{\pgfqpoint{2.558469in}{1.042357in}}%
\pgfpathlineto{\pgfqpoint{2.562173in}{1.041928in}}%
\pgfpathlineto{\pgfqpoint{2.577812in}{1.045009in}}%
\pgfpathlineto{\pgfqpoint{2.585662in}{1.048432in}}%
\pgfpathlineto{\pgfqpoint{2.587506in}{1.042954in}}%
\pgfpathlineto{\pgfqpoint{2.599396in}{1.033810in}}%
\pgfpathlineto{\pgfqpoint{2.602084in}{1.041763in}}%
\pgfpathlineto{\pgfqpoint{2.611697in}{1.039306in}}%
\pgfpathlineto{\pgfqpoint{2.617237in}{1.030317in}}%
\pgfpathlineto{\pgfqpoint{2.613415in}{1.018205in}}%
\pgfpathlineto{\pgfqpoint{2.617575in}{1.008104in}}%
\pgfpathlineto{\pgfqpoint{2.613775in}{0.994976in}}%
\pgfpathlineto{\pgfqpoint{2.609477in}{0.994521in}}%
\pgfpathlineto{\pgfqpoint{2.608008in}{0.986849in}}%
\pgfpathlineto{\pgfqpoint{2.615254in}{0.987618in}}%
\pgfpathlineto{\pgfqpoint{2.615751in}{0.977635in}}%
\pgfpathclose%
\pgfusepath{fill}%
\end{pgfscope}%
\begin{pgfscope}%
\pgfpathrectangle{\pgfqpoint{0.100000in}{0.100000in}}{\pgfqpoint{3.420221in}{2.189500in}}%
\pgfusepath{clip}%
\pgfsetbuttcap%
\pgfsetmiterjoin%
\definecolor{currentfill}{rgb}{0.000000,0.419608,0.790196}%
\pgfsetfillcolor{currentfill}%
\pgfsetlinewidth{0.000000pt}%
\definecolor{currentstroke}{rgb}{0.000000,0.000000,0.000000}%
\pgfsetstrokecolor{currentstroke}%
\pgfsetstrokeopacity{0.000000}%
\pgfsetdash{}{0pt}%
\pgfpathmoveto{\pgfqpoint{2.533426in}{1.426220in}}%
\pgfpathlineto{\pgfqpoint{2.520639in}{1.424829in}}%
\pgfpathlineto{\pgfqpoint{2.516407in}{1.424352in}}%
\pgfpathlineto{\pgfqpoint{2.513434in}{1.450271in}}%
\pgfpathlineto{\pgfqpoint{2.500963in}{1.448821in}}%
\pgfpathlineto{\pgfqpoint{2.501550in}{1.442353in}}%
\pgfpathlineto{\pgfqpoint{2.484664in}{1.440750in}}%
\pgfpathlineto{\pgfqpoint{2.482907in}{1.460294in}}%
\pgfpathlineto{\pgfqpoint{2.506093in}{1.462644in}}%
\pgfpathlineto{\pgfqpoint{2.503884in}{1.482216in}}%
\pgfpathlineto{\pgfqpoint{2.526546in}{1.484582in}}%
\pgfpathlineto{\pgfqpoint{2.527732in}{1.473996in}}%
\pgfpathlineto{\pgfqpoint{2.526683in}{1.465115in}}%
\pgfpathlineto{\pgfqpoint{2.528225in}{1.451969in}}%
\pgfpathlineto{\pgfqpoint{2.530437in}{1.452229in}}%
\pgfpathclose%
\pgfusepath{fill}%
\end{pgfscope}%
\begin{pgfscope}%
\pgfpathrectangle{\pgfqpoint{0.100000in}{0.100000in}}{\pgfqpoint{3.420221in}{2.189500in}}%
\pgfusepath{clip}%
\pgfsetbuttcap%
\pgfsetmiterjoin%
\definecolor{currentfill}{rgb}{0.000000,0.549020,0.725490}%
\pgfsetfillcolor{currentfill}%
\pgfsetlinewidth{0.000000pt}%
\definecolor{currentstroke}{rgb}{0.000000,0.000000,0.000000}%
\pgfsetstrokecolor{currentstroke}%
\pgfsetstrokeopacity{0.000000}%
\pgfsetdash{}{0pt}%
\pgfpathmoveto{\pgfqpoint{0.411870in}{1.452811in}}%
\pgfpathlineto{\pgfqpoint{0.406132in}{1.451449in}}%
\pgfpathlineto{\pgfqpoint{0.403286in}{1.458482in}}%
\pgfpathlineto{\pgfqpoint{0.394222in}{1.462892in}}%
\pgfpathlineto{\pgfqpoint{0.387741in}{1.473561in}}%
\pgfpathlineto{\pgfqpoint{0.381487in}{1.473896in}}%
\pgfpathlineto{\pgfqpoint{0.377951in}{1.487250in}}%
\pgfpathlineto{\pgfqpoint{0.372571in}{1.493520in}}%
\pgfpathlineto{\pgfqpoint{0.368009in}{1.502709in}}%
\pgfpathlineto{\pgfqpoint{0.365570in}{1.512154in}}%
\pgfpathlineto{\pgfqpoint{0.358204in}{1.524407in}}%
\pgfpathlineto{\pgfqpoint{0.355494in}{1.531161in}}%
\pgfpathlineto{\pgfqpoint{0.360371in}{1.539710in}}%
\pgfpathlineto{\pgfqpoint{0.359503in}{1.551733in}}%
\pgfpathlineto{\pgfqpoint{0.360097in}{1.563791in}}%
\pgfpathlineto{\pgfqpoint{0.362938in}{1.570644in}}%
\pgfpathlineto{\pgfqpoint{0.368140in}{1.577472in}}%
\pgfpathlineto{\pgfqpoint{0.369719in}{1.587511in}}%
\pgfpathlineto{\pgfqpoint{0.369873in}{1.598697in}}%
\pgfpathlineto{\pgfqpoint{0.364478in}{1.613856in}}%
\pgfpathlineto{\pgfqpoint{0.390656in}{1.605224in}}%
\pgfpathlineto{\pgfqpoint{0.390074in}{1.603451in}}%
\pgfpathlineto{\pgfqpoint{0.423474in}{1.592812in}}%
\pgfpathlineto{\pgfqpoint{0.419194in}{1.580023in}}%
\pgfpathlineto{\pgfqpoint{0.419641in}{1.572866in}}%
\pgfpathlineto{\pgfqpoint{0.417130in}{1.563539in}}%
\pgfpathlineto{\pgfqpoint{0.425385in}{1.560976in}}%
\pgfpathlineto{\pgfqpoint{0.422628in}{1.551639in}}%
\pgfpathlineto{\pgfqpoint{0.417527in}{1.540172in}}%
\pgfpathlineto{\pgfqpoint{0.422884in}{1.533660in}}%
\pgfpathlineto{\pgfqpoint{0.429225in}{1.530605in}}%
\pgfpathlineto{\pgfqpoint{0.427040in}{1.519065in}}%
\pgfpathlineto{\pgfqpoint{0.430694in}{1.514620in}}%
\pgfpathlineto{\pgfqpoint{0.432606in}{1.507192in}}%
\pgfpathlineto{\pgfqpoint{0.428217in}{1.503848in}}%
\pgfpathlineto{\pgfqpoint{0.428368in}{1.500095in}}%
\pgfpathlineto{\pgfqpoint{0.420837in}{1.493666in}}%
\pgfpathlineto{\pgfqpoint{0.410869in}{1.493824in}}%
\pgfpathlineto{\pgfqpoint{0.408261in}{1.489232in}}%
\pgfpathlineto{\pgfqpoint{0.414088in}{1.475195in}}%
\pgfpathlineto{\pgfqpoint{0.415789in}{1.463804in}}%
\pgfpathlineto{\pgfqpoint{0.415238in}{1.454474in}}%
\pgfpathclose%
\pgfusepath{fill}%
\end{pgfscope}%
\begin{pgfscope}%
\pgfpathrectangle{\pgfqpoint{0.100000in}{0.100000in}}{\pgfqpoint{3.420221in}{2.189500in}}%
\pgfusepath{clip}%
\pgfsetbuttcap%
\pgfsetmiterjoin%
\definecolor{currentfill}{rgb}{0.000000,0.592157,0.703922}%
\pgfsetfillcolor{currentfill}%
\pgfsetlinewidth{0.000000pt}%
\definecolor{currentstroke}{rgb}{0.000000,0.000000,0.000000}%
\pgfsetstrokecolor{currentstroke}%
\pgfsetstrokeopacity{0.000000}%
\pgfsetdash{}{0pt}%
\pgfpathmoveto{\pgfqpoint{2.782260in}{1.231360in}}%
\pgfpathlineto{\pgfqpoint{2.769279in}{1.230057in}}%
\pgfpathlineto{\pgfqpoint{2.762895in}{1.235158in}}%
\pgfpathlineto{\pgfqpoint{2.761558in}{1.243339in}}%
\pgfpathlineto{\pgfqpoint{2.756185in}{1.251398in}}%
\pgfpathlineto{\pgfqpoint{2.749742in}{1.248131in}}%
\pgfpathlineto{\pgfqpoint{2.754691in}{1.255281in}}%
\pgfpathlineto{\pgfqpoint{2.748849in}{1.263177in}}%
\pgfpathlineto{\pgfqpoint{2.748674in}{1.267808in}}%
\pgfpathlineto{\pgfqpoint{2.752165in}{1.277163in}}%
\pgfpathlineto{\pgfqpoint{2.760325in}{1.282231in}}%
\pgfpathlineto{\pgfqpoint{2.758795in}{1.289595in}}%
\pgfpathlineto{\pgfqpoint{2.756828in}{1.299412in}}%
\pgfpathlineto{\pgfqpoint{2.762603in}{1.303981in}}%
\pgfpathlineto{\pgfqpoint{2.772938in}{1.307263in}}%
\pgfpathlineto{\pgfqpoint{2.782440in}{1.299093in}}%
\pgfpathlineto{\pgfqpoint{2.788194in}{1.304363in}}%
\pgfpathlineto{\pgfqpoint{2.792181in}{1.300199in}}%
\pgfpathlineto{\pgfqpoint{2.808147in}{1.300931in}}%
\pgfpathlineto{\pgfqpoint{2.815948in}{1.312761in}}%
\pgfpathlineto{\pgfqpoint{2.823059in}{1.307156in}}%
\pgfpathlineto{\pgfqpoint{2.826687in}{1.302013in}}%
\pgfpathlineto{\pgfqpoint{2.825830in}{1.295111in}}%
\pgfpathlineto{\pgfqpoint{2.808953in}{1.281024in}}%
\pgfpathlineto{\pgfqpoint{2.804192in}{1.272571in}}%
\pgfpathlineto{\pgfqpoint{2.803576in}{1.257907in}}%
\pgfpathlineto{\pgfqpoint{2.795978in}{1.258228in}}%
\pgfpathlineto{\pgfqpoint{2.792892in}{1.253169in}}%
\pgfpathlineto{\pgfqpoint{2.797754in}{1.243552in}}%
\pgfpathlineto{\pgfqpoint{2.783521in}{1.238452in}}%
\pgfpathlineto{\pgfqpoint{2.787752in}{1.235265in}}%
\pgfpathclose%
\pgfusepath{fill}%
\end{pgfscope}%
\begin{pgfscope}%
\pgfpathrectangle{\pgfqpoint{0.100000in}{0.100000in}}{\pgfqpoint{3.420221in}{2.189500in}}%
\pgfusepath{clip}%
\pgfsetbuttcap%
\pgfsetmiterjoin%
\definecolor{currentfill}{rgb}{0.000000,0.270588,0.864706}%
\pgfsetfillcolor{currentfill}%
\pgfsetlinewidth{0.000000pt}%
\definecolor{currentstroke}{rgb}{0.000000,0.000000,0.000000}%
\pgfsetstrokecolor{currentstroke}%
\pgfsetstrokeopacity{0.000000}%
\pgfsetdash{}{0pt}%
\pgfpathmoveto{\pgfqpoint{1.463565in}{1.641775in}}%
\pgfpathlineto{\pgfqpoint{1.457920in}{1.589872in}}%
\pgfpathlineto{\pgfqpoint{1.456705in}{1.575235in}}%
\pgfpathlineto{\pgfqpoint{1.435043in}{1.577414in}}%
\pgfpathlineto{\pgfqpoint{1.433707in}{1.564391in}}%
\pgfpathlineto{\pgfqpoint{1.431023in}{1.564674in}}%
\pgfpathlineto{\pgfqpoint{1.427457in}{1.554435in}}%
\pgfpathlineto{\pgfqpoint{1.414442in}{1.557026in}}%
\pgfpathlineto{\pgfqpoint{1.415036in}{1.562470in}}%
\pgfpathlineto{\pgfqpoint{1.418726in}{1.566054in}}%
\pgfpathlineto{\pgfqpoint{1.390297in}{1.569654in}}%
\pgfpathlineto{\pgfqpoint{1.359960in}{1.573255in}}%
\pgfpathlineto{\pgfqpoint{1.310701in}{1.580131in}}%
\pgfpathlineto{\pgfqpoint{1.309652in}{1.580276in}}%
\pgfpathlineto{\pgfqpoint{1.313155in}{1.606101in}}%
\pgfpathlineto{\pgfqpoint{1.315452in}{1.605775in}}%
\pgfpathlineto{\pgfqpoint{1.318972in}{1.631563in}}%
\pgfpathlineto{\pgfqpoint{1.321728in}{1.657315in}}%
\pgfpathlineto{\pgfqpoint{1.321060in}{1.659586in}}%
\pgfpathlineto{\pgfqpoint{1.343956in}{1.656380in}}%
\pgfpathlineto{\pgfqpoint{1.371748in}{1.652066in}}%
\pgfpathlineto{\pgfqpoint{1.403019in}{1.648421in}}%
\pgfpathclose%
\pgfusepath{fill}%
\end{pgfscope}%
\begin{pgfscope}%
\pgfpathrectangle{\pgfqpoint{0.100000in}{0.100000in}}{\pgfqpoint{3.420221in}{2.189500in}}%
\pgfusepath{clip}%
\pgfsetbuttcap%
\pgfsetmiterjoin%
\definecolor{currentfill}{rgb}{0.000000,0.243137,0.878431}%
\pgfsetfillcolor{currentfill}%
\pgfsetlinewidth{0.000000pt}%
\definecolor{currentstroke}{rgb}{0.000000,0.000000,0.000000}%
\pgfsetstrokecolor{currentstroke}%
\pgfsetstrokeopacity{0.000000}%
\pgfsetdash{}{0pt}%
\pgfpathmoveto{\pgfqpoint{1.984415in}{1.349932in}}%
\pgfpathlineto{\pgfqpoint{1.982601in}{1.349938in}}%
\pgfpathlineto{\pgfqpoint{1.930990in}{1.350438in}}%
\pgfpathlineto{\pgfqpoint{1.918109in}{1.350651in}}%
\pgfpathlineto{\pgfqpoint{1.919043in}{1.409319in}}%
\pgfpathlineto{\pgfqpoint{1.954775in}{1.408853in}}%
\pgfpathlineto{\pgfqpoint{1.951644in}{1.405323in}}%
\pgfpathlineto{\pgfqpoint{1.956887in}{1.399735in}}%
\pgfpathlineto{\pgfqpoint{1.958536in}{1.393937in}}%
\pgfpathlineto{\pgfqpoint{1.966804in}{1.373529in}}%
\pgfpathlineto{\pgfqpoint{1.974067in}{1.368632in}}%
\pgfpathlineto{\pgfqpoint{1.974679in}{1.363885in}}%
\pgfpathlineto{\pgfqpoint{1.978942in}{1.359621in}}%
\pgfpathlineto{\pgfqpoint{1.977930in}{1.354375in}}%
\pgfpathclose%
\pgfusepath{fill}%
\end{pgfscope}%
\begin{pgfscope}%
\pgfpathrectangle{\pgfqpoint{0.100000in}{0.100000in}}{\pgfqpoint{3.420221in}{2.189500in}}%
\pgfusepath{clip}%
\pgfsetbuttcap%
\pgfsetmiterjoin%
\definecolor{currentfill}{rgb}{0.000000,0.301961,0.849020}%
\pgfsetfillcolor{currentfill}%
\pgfsetlinewidth{0.000000pt}%
\definecolor{currentstroke}{rgb}{0.000000,0.000000,0.000000}%
\pgfsetstrokecolor{currentstroke}%
\pgfsetstrokeopacity{0.000000}%
\pgfsetdash{}{0pt}%
\pgfpathmoveto{\pgfqpoint{1.544712in}{0.806841in}}%
\pgfpathlineto{\pgfqpoint{1.490102in}{0.811357in}}%
\pgfpathlineto{\pgfqpoint{1.487214in}{0.778661in}}%
\pgfpathlineto{\pgfqpoint{1.486630in}{0.772157in}}%
\pgfpathlineto{\pgfqpoint{1.444666in}{0.776020in}}%
\pgfpathlineto{\pgfqpoint{1.448438in}{0.815245in}}%
\pgfpathlineto{\pgfqpoint{1.442678in}{0.815793in}}%
\pgfpathlineto{\pgfqpoint{1.445870in}{0.849008in}}%
\pgfpathlineto{\pgfqpoint{1.448906in}{0.848715in}}%
\pgfpathlineto{\pgfqpoint{1.452020in}{0.880967in}}%
\pgfpathlineto{\pgfqpoint{1.454842in}{0.880726in}}%
\pgfpathlineto{\pgfqpoint{1.456135in}{0.893742in}}%
\pgfpathlineto{\pgfqpoint{1.469277in}{0.892536in}}%
\pgfpathlineto{\pgfqpoint{1.497857in}{0.889950in}}%
\pgfpathlineto{\pgfqpoint{1.496390in}{0.876315in}}%
\pgfpathlineto{\pgfqpoint{1.557824in}{0.871277in}}%
\pgfpathlineto{\pgfqpoint{1.555288in}{0.839003in}}%
\pgfpathlineto{\pgfqpoint{1.546947in}{0.839603in}}%
\pgfpathclose%
\pgfusepath{fill}%
\end{pgfscope}%
\begin{pgfscope}%
\pgfpathrectangle{\pgfqpoint{0.100000in}{0.100000in}}{\pgfqpoint{3.420221in}{2.189500in}}%
\pgfusepath{clip}%
\pgfsetbuttcap%
\pgfsetmiterjoin%
\definecolor{currentfill}{rgb}{0.000000,0.192157,0.903922}%
\pgfsetfillcolor{currentfill}%
\pgfsetlinewidth{0.000000pt}%
\definecolor{currentstroke}{rgb}{0.000000,0.000000,0.000000}%
\pgfsetstrokecolor{currentstroke}%
\pgfsetstrokeopacity{0.000000}%
\pgfsetdash{}{0pt}%
\pgfpathmoveto{\pgfqpoint{2.420189in}{1.495153in}}%
\pgfpathlineto{\pgfqpoint{2.422561in}{1.464491in}}%
\pgfpathlineto{\pgfqpoint{2.395244in}{1.461869in}}%
\pgfpathlineto{\pgfqpoint{2.395863in}{1.455400in}}%
\pgfpathlineto{\pgfqpoint{2.382806in}{1.454090in}}%
\pgfpathlineto{\pgfqpoint{2.382882in}{1.447526in}}%
\pgfpathlineto{\pgfqpoint{2.364070in}{1.445705in}}%
\pgfpathlineto{\pgfqpoint{2.360395in}{1.484794in}}%
\pgfpathlineto{\pgfqpoint{2.358136in}{1.517350in}}%
\pgfpathlineto{\pgfqpoint{2.358413in}{1.523924in}}%
\pgfpathlineto{\pgfqpoint{2.351914in}{1.523465in}}%
\pgfpathlineto{\pgfqpoint{2.350052in}{1.548913in}}%
\pgfpathlineto{\pgfqpoint{2.399978in}{1.552493in}}%
\pgfpathlineto{\pgfqpoint{2.399430in}{1.537581in}}%
\pgfpathlineto{\pgfqpoint{2.401818in}{1.531275in}}%
\pgfpathlineto{\pgfqpoint{2.409136in}{1.521903in}}%
\pgfpathlineto{\pgfqpoint{2.414588in}{1.504998in}}%
\pgfpathclose%
\pgfusepath{fill}%
\end{pgfscope}%
\begin{pgfscope}%
\pgfpathrectangle{\pgfqpoint{0.100000in}{0.100000in}}{\pgfqpoint{3.420221in}{2.189500in}}%
\pgfusepath{clip}%
\pgfsetbuttcap%
\pgfsetmiterjoin%
\definecolor{currentfill}{rgb}{0.000000,0.270588,0.864706}%
\pgfsetfillcolor{currentfill}%
\pgfsetlinewidth{0.000000pt}%
\definecolor{currentstroke}{rgb}{0.000000,0.000000,0.000000}%
\pgfsetstrokecolor{currentstroke}%
\pgfsetstrokeopacity{0.000000}%
\pgfsetdash{}{0pt}%
\pgfpathmoveto{\pgfqpoint{1.865335in}{1.319239in}}%
\pgfpathlineto{\pgfqpoint{1.864761in}{1.299711in}}%
\pgfpathlineto{\pgfqpoint{1.832327in}{1.300676in}}%
\pgfpathlineto{\pgfqpoint{1.833002in}{1.320180in}}%
\pgfpathlineto{\pgfqpoint{1.799725in}{1.321408in}}%
\pgfpathlineto{\pgfqpoint{1.801027in}{1.353950in}}%
\pgfpathlineto{\pgfqpoint{1.814244in}{1.353473in}}%
\pgfpathlineto{\pgfqpoint{1.866155in}{1.351828in}}%
\pgfpathclose%
\pgfusepath{fill}%
\end{pgfscope}%
\begin{pgfscope}%
\pgfpathrectangle{\pgfqpoint{0.100000in}{0.100000in}}{\pgfqpoint{3.420221in}{2.189500in}}%
\pgfusepath{clip}%
\pgfsetbuttcap%
\pgfsetmiterjoin%
\definecolor{currentfill}{rgb}{0.000000,0.411765,0.794118}%
\pgfsetfillcolor{currentfill}%
\pgfsetlinewidth{0.000000pt}%
\definecolor{currentstroke}{rgb}{0.000000,0.000000,0.000000}%
\pgfsetstrokecolor{currentstroke}%
\pgfsetstrokeopacity{0.000000}%
\pgfsetdash{}{0pt}%
\pgfpathmoveto{\pgfqpoint{1.608628in}{1.515823in}}%
\pgfpathlineto{\pgfqpoint{1.574663in}{1.518290in}}%
\pgfpathlineto{\pgfqpoint{1.571626in}{1.525102in}}%
\pgfpathlineto{\pgfqpoint{1.573451in}{1.551129in}}%
\pgfpathlineto{\pgfqpoint{1.572128in}{1.551237in}}%
\pgfpathlineto{\pgfqpoint{1.574342in}{1.577063in}}%
\pgfpathlineto{\pgfqpoint{1.574694in}{1.593214in}}%
\pgfpathlineto{\pgfqpoint{1.563276in}{1.594266in}}%
\pgfpathlineto{\pgfqpoint{1.567264in}{1.639556in}}%
\pgfpathlineto{\pgfqpoint{1.573103in}{1.644246in}}%
\pgfpathlineto{\pgfqpoint{1.578137in}{1.644808in}}%
\pgfpathlineto{\pgfqpoint{1.612382in}{1.641929in}}%
\pgfpathlineto{\pgfqpoint{1.614564in}{1.642760in}}%
\pgfpathlineto{\pgfqpoint{1.613248in}{1.625968in}}%
\pgfpathlineto{\pgfqpoint{1.614273in}{1.619347in}}%
\pgfpathlineto{\pgfqpoint{1.662326in}{1.615766in}}%
\pgfpathlineto{\pgfqpoint{1.660333in}{1.586435in}}%
\pgfpathlineto{\pgfqpoint{1.613528in}{1.589980in}}%
\pgfpathlineto{\pgfqpoint{1.613138in}{1.573904in}}%
\pgfpathlineto{\pgfqpoint{1.611155in}{1.548307in}}%
\pgfpathlineto{\pgfqpoint{1.612730in}{1.548174in}}%
\pgfpathlineto{\pgfqpoint{1.610602in}{1.522257in}}%
\pgfpathclose%
\pgfusepath{fill}%
\end{pgfscope}%
\begin{pgfscope}%
\pgfpathrectangle{\pgfqpoint{0.100000in}{0.100000in}}{\pgfqpoint{3.420221in}{2.189500in}}%
\pgfusepath{clip}%
\pgfsetbuttcap%
\pgfsetmiterjoin%
\definecolor{currentfill}{rgb}{0.000000,0.541176,0.729412}%
\pgfsetfillcolor{currentfill}%
\pgfsetlinewidth{0.000000pt}%
\definecolor{currentstroke}{rgb}{0.000000,0.000000,0.000000}%
\pgfsetstrokecolor{currentstroke}%
\pgfsetstrokeopacity{0.000000}%
\pgfsetdash{}{0pt}%
\pgfpathmoveto{\pgfqpoint{2.030697in}{2.005365in}}%
\pgfpathlineto{\pgfqpoint{2.039063in}{2.005231in}}%
\pgfpathlineto{\pgfqpoint{2.040055in}{2.001825in}}%
\pgfpathlineto{\pgfqpoint{2.060676in}{1.999878in}}%
\pgfpathlineto{\pgfqpoint{2.062789in}{1.991625in}}%
\pgfpathlineto{\pgfqpoint{2.079206in}{1.994126in}}%
\pgfpathlineto{\pgfqpoint{2.079240in}{1.997344in}}%
\pgfpathlineto{\pgfqpoint{2.092146in}{2.001724in}}%
\pgfpathlineto{\pgfqpoint{2.098144in}{2.000716in}}%
\pgfpathlineto{\pgfqpoint{2.098632in}{1.958199in}}%
\pgfpathlineto{\pgfqpoint{2.099691in}{1.944868in}}%
\pgfpathlineto{\pgfqpoint{2.064362in}{1.944793in}}%
\pgfpathlineto{\pgfqpoint{2.064392in}{1.940807in}}%
\pgfpathlineto{\pgfqpoint{2.031703in}{1.940437in}}%
\pgfpathlineto{\pgfqpoint{2.030936in}{1.980015in}}%
\pgfpathclose%
\pgfusepath{fill}%
\end{pgfscope}%
\begin{pgfscope}%
\pgfpathrectangle{\pgfqpoint{0.100000in}{0.100000in}}{\pgfqpoint{3.420221in}{2.189500in}}%
\pgfusepath{clip}%
\pgfsetbuttcap%
\pgfsetmiterjoin%
\definecolor{currentfill}{rgb}{0.000000,0.388235,0.805882}%
\pgfsetfillcolor{currentfill}%
\pgfsetlinewidth{0.000000pt}%
\definecolor{currentstroke}{rgb}{0.000000,0.000000,0.000000}%
\pgfsetstrokecolor{currentstroke}%
\pgfsetstrokeopacity{0.000000}%
\pgfsetdash{}{0pt}%
\pgfpathmoveto{\pgfqpoint{2.426052in}{1.422567in}}%
\pgfpathlineto{\pgfqpoint{2.427548in}{1.404238in}}%
\pgfpathlineto{\pgfqpoint{2.404337in}{1.401975in}}%
\pgfpathlineto{\pgfqpoint{2.393956in}{1.401317in}}%
\pgfpathlineto{\pgfqpoint{2.390289in}{1.439339in}}%
\pgfpathlineto{\pgfqpoint{2.383796in}{1.438635in}}%
\pgfpathlineto{\pgfqpoint{2.382882in}{1.447526in}}%
\pgfpathlineto{\pgfqpoint{2.382806in}{1.454090in}}%
\pgfpathlineto{\pgfqpoint{2.395863in}{1.455400in}}%
\pgfpathlineto{\pgfqpoint{2.395244in}{1.461869in}}%
\pgfpathlineto{\pgfqpoint{2.422561in}{1.464491in}}%
\pgfpathclose%
\pgfusepath{fill}%
\end{pgfscope}%
\begin{pgfscope}%
\pgfpathrectangle{\pgfqpoint{0.100000in}{0.100000in}}{\pgfqpoint{3.420221in}{2.189500in}}%
\pgfusepath{clip}%
\pgfsetbuttcap%
\pgfsetmiterjoin%
\definecolor{currentfill}{rgb}{0.000000,0.345098,0.827451}%
\pgfsetfillcolor{currentfill}%
\pgfsetlinewidth{0.000000pt}%
\definecolor{currentstroke}{rgb}{0.000000,0.000000,0.000000}%
\pgfsetstrokecolor{currentstroke}%
\pgfsetstrokeopacity{0.000000}%
\pgfsetdash{}{0pt}%
\pgfpathmoveto{\pgfqpoint{2.330118in}{0.564192in}}%
\pgfpathlineto{\pgfqpoint{2.323388in}{0.566863in}}%
\pgfpathlineto{\pgfqpoint{2.319836in}{0.573219in}}%
\pgfpathlineto{\pgfqpoint{2.321496in}{0.580669in}}%
\pgfpathlineto{\pgfqpoint{2.318241in}{0.582887in}}%
\pgfpathlineto{\pgfqpoint{2.309238in}{0.582525in}}%
\pgfpathlineto{\pgfqpoint{2.306732in}{0.587609in}}%
\pgfpathlineto{\pgfqpoint{2.299821in}{0.591150in}}%
\pgfpathlineto{\pgfqpoint{2.295254in}{0.596944in}}%
\pgfpathlineto{\pgfqpoint{2.272532in}{0.596877in}}%
\pgfpathlineto{\pgfqpoint{2.267410in}{0.602164in}}%
\pgfpathlineto{\pgfqpoint{2.266861in}{0.608853in}}%
\pgfpathlineto{\pgfqpoint{2.269861in}{0.612543in}}%
\pgfpathlineto{\pgfqpoint{2.287592in}{0.617403in}}%
\pgfpathlineto{\pgfqpoint{2.294751in}{0.621161in}}%
\pgfpathlineto{\pgfqpoint{2.298489in}{0.625745in}}%
\pgfpathlineto{\pgfqpoint{2.309615in}{0.628155in}}%
\pgfpathlineto{\pgfqpoint{2.313459in}{0.623097in}}%
\pgfpathlineto{\pgfqpoint{2.310780in}{0.660653in}}%
\pgfpathlineto{\pgfqpoint{2.303644in}{0.681420in}}%
\pgfpathlineto{\pgfqpoint{2.343715in}{0.683758in}}%
\pgfpathlineto{\pgfqpoint{2.341743in}{0.673282in}}%
\pgfpathlineto{\pgfqpoint{2.338129in}{0.666820in}}%
\pgfpathlineto{\pgfqpoint{2.337584in}{0.657686in}}%
\pgfpathlineto{\pgfqpoint{2.343283in}{0.646667in}}%
\pgfpathlineto{\pgfqpoint{2.349045in}{0.642328in}}%
\pgfpathlineto{\pgfqpoint{2.354474in}{0.625298in}}%
\pgfpathlineto{\pgfqpoint{2.360626in}{0.622577in}}%
\pgfpathlineto{\pgfqpoint{2.354557in}{0.620247in}}%
\pgfpathlineto{\pgfqpoint{2.348459in}{0.610420in}}%
\pgfpathlineto{\pgfqpoint{2.342439in}{0.610995in}}%
\pgfpathlineto{\pgfqpoint{2.340927in}{0.603817in}}%
\pgfpathlineto{\pgfqpoint{2.347091in}{0.603819in}}%
\pgfpathlineto{\pgfqpoint{2.349703in}{0.598854in}}%
\pgfpathlineto{\pgfqpoint{2.357167in}{0.599513in}}%
\pgfpathlineto{\pgfqpoint{2.357756in}{0.608158in}}%
\pgfpathlineto{\pgfqpoint{2.363267in}{0.612002in}}%
\pgfpathlineto{\pgfqpoint{2.375526in}{0.603811in}}%
\pgfpathlineto{\pgfqpoint{2.368973in}{0.589597in}}%
\pgfpathlineto{\pgfqpoint{2.354269in}{0.589315in}}%
\pgfpathlineto{\pgfqpoint{2.356665in}{0.582434in}}%
\pgfpathlineto{\pgfqpoint{2.353528in}{0.573445in}}%
\pgfpathlineto{\pgfqpoint{2.360893in}{0.570312in}}%
\pgfpathlineto{\pgfqpoint{2.367713in}{0.563687in}}%
\pgfpathlineto{\pgfqpoint{2.385972in}{0.560731in}}%
\pgfpathlineto{\pgfqpoint{2.394287in}{0.549699in}}%
\pgfpathlineto{\pgfqpoint{2.399924in}{0.548720in}}%
\pgfpathlineto{\pgfqpoint{2.396597in}{0.540060in}}%
\pgfpathlineto{\pgfqpoint{2.390462in}{0.536194in}}%
\pgfpathlineto{\pgfqpoint{2.386002in}{0.538200in}}%
\pgfpathlineto{\pgfqpoint{2.377847in}{0.537067in}}%
\pgfpathlineto{\pgfqpoint{2.366899in}{0.549824in}}%
\pgfpathlineto{\pgfqpoint{2.351631in}{0.555091in}}%
\pgfpathlineto{\pgfqpoint{2.343256in}{0.555537in}}%
\pgfpathlineto{\pgfqpoint{2.344725in}{0.560259in}}%
\pgfpathlineto{\pgfqpoint{2.343284in}{0.567277in}}%
\pgfpathclose%
\pgfusepath{fill}%
\end{pgfscope}%
\begin{pgfscope}%
\pgfpathrectangle{\pgfqpoint{0.100000in}{0.100000in}}{\pgfqpoint{3.420221in}{2.189500in}}%
\pgfusepath{clip}%
\pgfsetbuttcap%
\pgfsetmiterjoin%
\definecolor{currentfill}{rgb}{0.000000,0.301961,0.849020}%
\pgfsetfillcolor{currentfill}%
\pgfsetlinewidth{0.000000pt}%
\definecolor{currentstroke}{rgb}{0.000000,0.000000,0.000000}%
\pgfsetstrokecolor{currentstroke}%
\pgfsetstrokeopacity{0.000000}%
\pgfsetdash{}{0pt}%
\pgfpathmoveto{\pgfqpoint{2.190939in}{1.525590in}}%
\pgfpathlineto{\pgfqpoint{2.152004in}{1.524370in}}%
\pgfpathlineto{\pgfqpoint{2.125969in}{1.523603in}}%
\pgfpathlineto{\pgfqpoint{2.126163in}{1.517114in}}%
\pgfpathlineto{\pgfqpoint{2.113161in}{1.516746in}}%
\pgfpathlineto{\pgfqpoint{2.112659in}{1.536363in}}%
\pgfpathlineto{\pgfqpoint{2.111313in}{1.536342in}}%
\pgfpathlineto{\pgfqpoint{2.110721in}{1.569135in}}%
\pgfpathlineto{\pgfqpoint{2.162575in}{1.570476in}}%
\pgfpathlineto{\pgfqpoint{2.162184in}{1.583671in}}%
\pgfpathlineto{\pgfqpoint{2.188271in}{1.584492in}}%
\pgfpathlineto{\pgfqpoint{2.189812in}{1.538531in}}%
\pgfpathclose%
\pgfusepath{fill}%
\end{pgfscope}%
\begin{pgfscope}%
\pgfpathrectangle{\pgfqpoint{0.100000in}{0.100000in}}{\pgfqpoint{3.420221in}{2.189500in}}%
\pgfusepath{clip}%
\pgfsetbuttcap%
\pgfsetmiterjoin%
\definecolor{currentfill}{rgb}{0.000000,0.149020,0.925490}%
\pgfsetfillcolor{currentfill}%
\pgfsetlinewidth{0.000000pt}%
\definecolor{currentstroke}{rgb}{0.000000,0.000000,0.000000}%
\pgfsetstrokecolor{currentstroke}%
\pgfsetstrokeopacity{0.000000}%
\pgfsetdash{}{0pt}%
\pgfpathmoveto{\pgfqpoint{2.722210in}{0.883017in}}%
\pgfpathlineto{\pgfqpoint{2.705662in}{0.877911in}}%
\pgfpathlineto{\pgfqpoint{2.704826in}{0.881484in}}%
\pgfpathlineto{\pgfqpoint{2.704743in}{0.886634in}}%
\pgfpathlineto{\pgfqpoint{2.700344in}{0.895885in}}%
\pgfpathlineto{\pgfqpoint{2.696002in}{0.900954in}}%
\pgfpathlineto{\pgfqpoint{2.688011in}{0.888652in}}%
\pgfpathlineto{\pgfqpoint{2.676698in}{0.891144in}}%
\pgfpathlineto{\pgfqpoint{2.667908in}{0.890104in}}%
\pgfpathlineto{\pgfqpoint{2.665953in}{0.882593in}}%
\pgfpathlineto{\pgfqpoint{2.661897in}{0.882196in}}%
\pgfpathlineto{\pgfqpoint{2.655481in}{0.889099in}}%
\pgfpathlineto{\pgfqpoint{2.651572in}{0.900217in}}%
\pgfpathlineto{\pgfqpoint{2.639090in}{0.898974in}}%
\pgfpathlineto{\pgfqpoint{2.633704in}{0.902920in}}%
\pgfpathlineto{\pgfqpoint{2.632240in}{0.918452in}}%
\pgfpathlineto{\pgfqpoint{2.623515in}{0.919850in}}%
\pgfpathlineto{\pgfqpoint{2.621985in}{0.926685in}}%
\pgfpathlineto{\pgfqpoint{2.626042in}{0.930682in}}%
\pgfpathlineto{\pgfqpoint{2.628419in}{0.940866in}}%
\pgfpathlineto{\pgfqpoint{2.644765in}{0.942370in}}%
\pgfpathlineto{\pgfqpoint{2.641045in}{0.977578in}}%
\pgfpathlineto{\pgfqpoint{2.658448in}{0.979561in}}%
\pgfpathlineto{\pgfqpoint{2.660754in}{0.982495in}}%
\pgfpathlineto{\pgfqpoint{2.669197in}{0.984930in}}%
\pgfpathlineto{\pgfqpoint{2.669449in}{0.980294in}}%
\pgfpathlineto{\pgfqpoint{2.676351in}{0.974994in}}%
\pgfpathlineto{\pgfqpoint{2.683704in}{0.972759in}}%
\pgfpathlineto{\pgfqpoint{2.688203in}{0.962761in}}%
\pgfpathlineto{\pgfqpoint{2.685111in}{0.956009in}}%
\pgfpathlineto{\pgfqpoint{2.693513in}{0.949691in}}%
\pgfpathlineto{\pgfqpoint{2.696353in}{0.952258in}}%
\pgfpathlineto{\pgfqpoint{2.694308in}{0.942676in}}%
\pgfpathlineto{\pgfqpoint{2.701750in}{0.935581in}}%
\pgfpathlineto{\pgfqpoint{2.707030in}{0.937684in}}%
\pgfpathlineto{\pgfqpoint{2.720064in}{0.930845in}}%
\pgfpathlineto{\pgfqpoint{2.725880in}{0.923443in}}%
\pgfpathlineto{\pgfqpoint{2.735793in}{0.908412in}}%
\pgfpathlineto{\pgfqpoint{2.720451in}{0.902727in}}%
\pgfpathclose%
\pgfusepath{fill}%
\end{pgfscope}%
\begin{pgfscope}%
\pgfpathrectangle{\pgfqpoint{0.100000in}{0.100000in}}{\pgfqpoint{3.420221in}{2.189500in}}%
\pgfusepath{clip}%
\pgfsetbuttcap%
\pgfsetmiterjoin%
\definecolor{currentfill}{rgb}{0.000000,0.600000,0.700000}%
\pgfsetfillcolor{currentfill}%
\pgfsetlinewidth{0.000000pt}%
\definecolor{currentstroke}{rgb}{0.000000,0.000000,0.000000}%
\pgfsetstrokecolor{currentstroke}%
\pgfsetstrokeopacity{0.000000}%
\pgfsetdash{}{0pt}%
\pgfpathmoveto{\pgfqpoint{2.737398in}{1.025703in}}%
\pgfpathlineto{\pgfqpoint{2.731574in}{1.022907in}}%
\pgfpathlineto{\pgfqpoint{2.727917in}{1.029811in}}%
\pgfpathlineto{\pgfqpoint{2.723028in}{1.033653in}}%
\pgfpathlineto{\pgfqpoint{2.714299in}{1.045662in}}%
\pgfpathlineto{\pgfqpoint{2.707520in}{1.042670in}}%
\pgfpathlineto{\pgfqpoint{2.694535in}{1.039478in}}%
\pgfpathlineto{\pgfqpoint{2.693020in}{1.036722in}}%
\pgfpathlineto{\pgfqpoint{2.685161in}{1.036301in}}%
\pgfpathlineto{\pgfqpoint{2.677716in}{1.032437in}}%
\pgfpathlineto{\pgfqpoint{2.672754in}{1.037526in}}%
\pgfpathlineto{\pgfqpoint{2.672407in}{1.045621in}}%
\pgfpathlineto{\pgfqpoint{2.675333in}{1.050814in}}%
\pgfpathlineto{\pgfqpoint{2.685910in}{1.059556in}}%
\pgfpathlineto{\pgfqpoint{2.692435in}{1.060901in}}%
\pgfpathlineto{\pgfqpoint{2.702464in}{1.061702in}}%
\pgfpathlineto{\pgfqpoint{2.715961in}{1.073488in}}%
\pgfpathlineto{\pgfqpoint{2.720573in}{1.072221in}}%
\pgfpathlineto{\pgfqpoint{2.722510in}{1.061157in}}%
\pgfpathlineto{\pgfqpoint{2.739610in}{1.046866in}}%
\pgfpathlineto{\pgfqpoint{2.740082in}{1.041314in}}%
\pgfpathlineto{\pgfqpoint{2.734645in}{1.027887in}}%
\pgfpathclose%
\pgfusepath{fill}%
\end{pgfscope}%
\begin{pgfscope}%
\pgfpathrectangle{\pgfqpoint{0.100000in}{0.100000in}}{\pgfqpoint{3.420221in}{2.189500in}}%
\pgfusepath{clip}%
\pgfsetbuttcap%
\pgfsetmiterjoin%
\definecolor{currentfill}{rgb}{0.000000,0.215686,0.892157}%
\pgfsetfillcolor{currentfill}%
\pgfsetlinewidth{0.000000pt}%
\definecolor{currentstroke}{rgb}{0.000000,0.000000,0.000000}%
\pgfsetstrokecolor{currentstroke}%
\pgfsetstrokeopacity{0.000000}%
\pgfsetdash{}{0pt}%
\pgfpathmoveto{\pgfqpoint{2.311883in}{1.572845in}}%
\pgfpathlineto{\pgfqpoint{2.313651in}{1.547024in}}%
\pgfpathlineto{\pgfqpoint{2.282670in}{1.545611in}}%
\pgfpathlineto{\pgfqpoint{2.243185in}{1.543750in}}%
\pgfpathlineto{\pgfqpoint{2.238336in}{1.553473in}}%
\pgfpathlineto{\pgfqpoint{2.225981in}{1.556144in}}%
\pgfpathlineto{\pgfqpoint{2.219620in}{1.559951in}}%
\pgfpathlineto{\pgfqpoint{2.216574in}{1.570704in}}%
\pgfpathlineto{\pgfqpoint{2.214152in}{1.572223in}}%
\pgfpathlineto{\pgfqpoint{2.211674in}{1.589406in}}%
\pgfpathlineto{\pgfqpoint{2.217706in}{1.598739in}}%
\pgfpathlineto{\pgfqpoint{2.209544in}{1.605439in}}%
\pgfpathlineto{\pgfqpoint{2.209111in}{1.610975in}}%
\pgfpathlineto{\pgfqpoint{2.238385in}{1.612287in}}%
\pgfpathlineto{\pgfqpoint{2.239381in}{1.593441in}}%
\pgfpathlineto{\pgfqpoint{2.245877in}{1.596739in}}%
\pgfpathlineto{\pgfqpoint{2.252236in}{1.596264in}}%
\pgfpathlineto{\pgfqpoint{2.253857in}{1.567165in}}%
\pgfpathlineto{\pgfqpoint{2.286223in}{1.568970in}}%
\pgfpathlineto{\pgfqpoint{2.286050in}{1.572239in}}%
\pgfpathclose%
\pgfusepath{fill}%
\end{pgfscope}%
\begin{pgfscope}%
\pgfpathrectangle{\pgfqpoint{0.100000in}{0.100000in}}{\pgfqpoint{3.420221in}{2.189500in}}%
\pgfusepath{clip}%
\pgfsetbuttcap%
\pgfsetmiterjoin%
\definecolor{currentfill}{rgb}{0.000000,0.694118,0.652941}%
\pgfsetfillcolor{currentfill}%
\pgfsetlinewidth{0.000000pt}%
\definecolor{currentstroke}{rgb}{0.000000,0.000000,0.000000}%
\pgfsetstrokecolor{currentstroke}%
\pgfsetstrokeopacity{0.000000}%
\pgfsetdash{}{0pt}%
\pgfpathmoveto{\pgfqpoint{2.174022in}{1.072405in}}%
\pgfpathlineto{\pgfqpoint{2.173867in}{1.090211in}}%
\pgfpathlineto{\pgfqpoint{2.175674in}{1.090287in}}%
\pgfpathlineto{\pgfqpoint{2.175450in}{1.112439in}}%
\pgfpathlineto{\pgfqpoint{2.176237in}{1.132118in}}%
\pgfpathlineto{\pgfqpoint{2.166506in}{1.132070in}}%
\pgfpathlineto{\pgfqpoint{2.165384in}{1.172896in}}%
\pgfpathlineto{\pgfqpoint{2.194751in}{1.173438in}}%
\pgfpathlineto{\pgfqpoint{2.195204in}{1.160350in}}%
\pgfpathlineto{\pgfqpoint{2.229707in}{1.160995in}}%
\pgfpathlineto{\pgfqpoint{2.229984in}{1.153339in}}%
\pgfpathlineto{\pgfqpoint{2.233331in}{1.144995in}}%
\pgfpathlineto{\pgfqpoint{2.240024in}{1.141086in}}%
\pgfpathlineto{\pgfqpoint{2.240130in}{1.137500in}}%
\pgfpathlineto{\pgfqpoint{2.228249in}{1.136281in}}%
\pgfpathlineto{\pgfqpoint{2.228601in}{1.116748in}}%
\pgfpathlineto{\pgfqpoint{2.235137in}{1.116873in}}%
\pgfpathlineto{\pgfqpoint{2.235543in}{1.092426in}}%
\pgfpathlineto{\pgfqpoint{2.218643in}{1.091687in}}%
\pgfpathlineto{\pgfqpoint{2.216060in}{1.091625in}}%
\pgfpathlineto{\pgfqpoint{2.216266in}{1.079461in}}%
\pgfpathlineto{\pgfqpoint{2.209534in}{1.079347in}}%
\pgfpathlineto{\pgfqpoint{2.209770in}{1.072792in}}%
\pgfpathclose%
\pgfusepath{fill}%
\end{pgfscope}%
\begin{pgfscope}%
\pgfpathrectangle{\pgfqpoint{0.100000in}{0.100000in}}{\pgfqpoint{3.420221in}{2.189500in}}%
\pgfusepath{clip}%
\pgfsetbuttcap%
\pgfsetmiterjoin%
\definecolor{currentfill}{rgb}{0.000000,0.509804,0.745098}%
\pgfsetfillcolor{currentfill}%
\pgfsetlinewidth{0.000000pt}%
\definecolor{currentstroke}{rgb}{0.000000,0.000000,0.000000}%
\pgfsetstrokecolor{currentstroke}%
\pgfsetstrokeopacity{0.000000}%
\pgfsetdash{}{0pt}%
\pgfpathmoveto{\pgfqpoint{1.144753in}{1.761592in}}%
\pgfpathlineto{\pgfqpoint{1.146960in}{1.775893in}}%
\pgfpathlineto{\pgfqpoint{1.167390in}{1.772292in}}%
\pgfpathlineto{\pgfqpoint{1.166310in}{1.766100in}}%
\pgfpathlineto{\pgfqpoint{1.181769in}{1.763451in}}%
\pgfpathlineto{\pgfqpoint{1.186106in}{1.757553in}}%
\pgfpathlineto{\pgfqpoint{1.184011in}{1.751383in}}%
\pgfpathlineto{\pgfqpoint{1.189365in}{1.740527in}}%
\pgfpathlineto{\pgfqpoint{1.190621in}{1.731916in}}%
\pgfpathlineto{\pgfqpoint{1.193201in}{1.727149in}}%
\pgfpathlineto{\pgfqpoint{1.191665in}{1.717904in}}%
\pgfpathlineto{\pgfqpoint{1.185018in}{1.677614in}}%
\pgfpathlineto{\pgfqpoint{1.183631in}{1.671217in}}%
\pgfpathlineto{\pgfqpoint{1.168185in}{1.673837in}}%
\pgfpathlineto{\pgfqpoint{1.167103in}{1.667410in}}%
\pgfpathlineto{\pgfqpoint{1.154477in}{1.669615in}}%
\pgfpathlineto{\pgfqpoint{1.153739in}{1.665442in}}%
\pgfpathlineto{\pgfqpoint{1.141051in}{1.667652in}}%
\pgfpathlineto{\pgfqpoint{1.141886in}{1.672415in}}%
\pgfpathlineto{\pgfqpoint{1.129687in}{1.675810in}}%
\pgfpathlineto{\pgfqpoint{1.132087in}{1.689533in}}%
\pgfpathlineto{\pgfqpoint{1.124361in}{1.693859in}}%
\pgfpathlineto{\pgfqpoint{1.123055in}{1.700360in}}%
\pgfpathlineto{\pgfqpoint{1.114707in}{1.701900in}}%
\pgfpathlineto{\pgfqpoint{1.118825in}{1.724160in}}%
\pgfpathlineto{\pgfqpoint{1.129473in}{1.723048in}}%
\pgfpathlineto{\pgfqpoint{1.138371in}{1.725234in}}%
\pgfpathclose%
\pgfusepath{fill}%
\end{pgfscope}%
\begin{pgfscope}%
\pgfpathrectangle{\pgfqpoint{0.100000in}{0.100000in}}{\pgfqpoint{3.420221in}{2.189500in}}%
\pgfusepath{clip}%
\pgfsetbuttcap%
\pgfsetmiterjoin%
\definecolor{currentfill}{rgb}{0.000000,0.333333,0.833333}%
\pgfsetfillcolor{currentfill}%
\pgfsetlinewidth{0.000000pt}%
\definecolor{currentstroke}{rgb}{0.000000,0.000000,0.000000}%
\pgfsetstrokecolor{currentstroke}%
\pgfsetstrokeopacity{0.000000}%
\pgfsetdash{}{0pt}%
\pgfpathmoveto{\pgfqpoint{1.781217in}{0.792157in}}%
\pgfpathlineto{\pgfqpoint{1.780628in}{0.776032in}}%
\pgfpathlineto{\pgfqpoint{1.751262in}{0.760536in}}%
\pgfpathlineto{\pgfqpoint{1.700850in}{0.763241in}}%
\pgfpathlineto{\pgfqpoint{1.672894in}{0.764801in}}%
\pgfpathlineto{\pgfqpoint{1.677128in}{0.830819in}}%
\pgfpathlineto{\pgfqpoint{1.719594in}{0.828347in}}%
\pgfpathlineto{\pgfqpoint{1.776144in}{0.825640in}}%
\pgfpathlineto{\pgfqpoint{1.774802in}{0.792566in}}%
\pgfpathclose%
\pgfusepath{fill}%
\end{pgfscope}%
\begin{pgfscope}%
\pgfpathrectangle{\pgfqpoint{0.100000in}{0.100000in}}{\pgfqpoint{3.420221in}{2.189500in}}%
\pgfusepath{clip}%
\pgfsetbuttcap%
\pgfsetmiterjoin%
\definecolor{currentfill}{rgb}{0.000000,0.713725,0.643137}%
\pgfsetfillcolor{currentfill}%
\pgfsetlinewidth{0.000000pt}%
\definecolor{currentstroke}{rgb}{0.000000,0.000000,0.000000}%
\pgfsetstrokecolor{currentstroke}%
\pgfsetstrokeopacity{0.000000}%
\pgfsetdash{}{0pt}%
\pgfpathmoveto{\pgfqpoint{1.051332in}{1.323810in}}%
\pgfpathlineto{\pgfqpoint{1.051193in}{1.323059in}}%
\pgfpathlineto{\pgfqpoint{1.087507in}{1.316602in}}%
\pgfpathlineto{\pgfqpoint{1.125155in}{1.310182in}}%
\pgfpathlineto{\pgfqpoint{1.126995in}{1.293955in}}%
\pgfpathlineto{\pgfqpoint{1.129369in}{1.286933in}}%
\pgfpathlineto{\pgfqpoint{1.126531in}{1.283514in}}%
\pgfpathlineto{\pgfqpoint{1.082408in}{1.290979in}}%
\pgfpathlineto{\pgfqpoint{1.015420in}{1.303090in}}%
\pgfpathlineto{\pgfqpoint{1.018848in}{1.323304in}}%
\pgfpathlineto{\pgfqpoint{1.025360in}{1.328572in}}%
\pgfpathclose%
\pgfusepath{fill}%
\end{pgfscope}%
\begin{pgfscope}%
\pgfpathrectangle{\pgfqpoint{0.100000in}{0.100000in}}{\pgfqpoint{3.420221in}{2.189500in}}%
\pgfusepath{clip}%
\pgfsetbuttcap%
\pgfsetmiterjoin%
\definecolor{currentfill}{rgb}{0.000000,0.505882,0.747059}%
\pgfsetfillcolor{currentfill}%
\pgfsetlinewidth{0.000000pt}%
\definecolor{currentstroke}{rgb}{0.000000,0.000000,0.000000}%
\pgfsetstrokecolor{currentstroke}%
\pgfsetstrokeopacity{0.000000}%
\pgfsetdash{}{0pt}%
\pgfpathmoveto{\pgfqpoint{2.018001in}{0.877582in}}%
\pgfpathlineto{\pgfqpoint{2.034317in}{0.872749in}}%
\pgfpathlineto{\pgfqpoint{2.034723in}{0.895551in}}%
\pgfpathlineto{\pgfqpoint{2.040516in}{0.895557in}}%
\pgfpathlineto{\pgfqpoint{2.049722in}{0.885210in}}%
\pgfpathlineto{\pgfqpoint{2.062081in}{0.883822in}}%
\pgfpathlineto{\pgfqpoint{2.067236in}{0.881542in}}%
\pgfpathlineto{\pgfqpoint{2.066863in}{0.875444in}}%
\pgfpathlineto{\pgfqpoint{2.075179in}{0.870823in}}%
\pgfpathlineto{\pgfqpoint{2.082279in}{0.861488in}}%
\pgfpathlineto{\pgfqpoint{2.081817in}{0.857201in}}%
\pgfpathlineto{\pgfqpoint{2.086042in}{0.850263in}}%
\pgfpathlineto{\pgfqpoint{2.080404in}{0.837602in}}%
\pgfpathlineto{\pgfqpoint{2.074614in}{0.836039in}}%
\pgfpathlineto{\pgfqpoint{2.074034in}{0.831961in}}%
\pgfpathlineto{\pgfqpoint{2.077586in}{0.826587in}}%
\pgfpathlineto{\pgfqpoint{2.062553in}{0.826398in}}%
\pgfpathlineto{\pgfqpoint{2.062668in}{0.815983in}}%
\pgfpathlineto{\pgfqpoint{2.020719in}{0.815560in}}%
\pgfpathlineto{\pgfqpoint{2.019817in}{0.817465in}}%
\pgfpathlineto{\pgfqpoint{1.992487in}{0.817292in}}%
\pgfpathlineto{\pgfqpoint{1.991165in}{0.821771in}}%
\pgfpathlineto{\pgfqpoint{1.982581in}{0.821847in}}%
\pgfpathlineto{\pgfqpoint{1.982713in}{0.853040in}}%
\pgfpathlineto{\pgfqpoint{1.996243in}{0.854575in}}%
\pgfpathlineto{\pgfqpoint{2.009891in}{0.854342in}}%
\pgfpathlineto{\pgfqpoint{2.018028in}{0.849427in}}%
\pgfpathclose%
\pgfusepath{fill}%
\end{pgfscope}%
\begin{pgfscope}%
\pgfpathrectangle{\pgfqpoint{0.100000in}{0.100000in}}{\pgfqpoint{3.420221in}{2.189500in}}%
\pgfusepath{clip}%
\pgfsetbuttcap%
\pgfsetmiterjoin%
\definecolor{currentfill}{rgb}{0.000000,0.890196,0.554902}%
\pgfsetfillcolor{currentfill}%
\pgfsetlinewidth{0.000000pt}%
\definecolor{currentstroke}{rgb}{0.000000,0.000000,0.000000}%
\pgfsetstrokecolor{currentstroke}%
\pgfsetstrokeopacity{0.000000}%
\pgfsetdash{}{0pt}%
\pgfpathmoveto{\pgfqpoint{2.849939in}{0.556223in}}%
\pgfpathlineto{\pgfqpoint{2.839382in}{0.556836in}}%
\pgfpathlineto{\pgfqpoint{2.824824in}{0.554806in}}%
\pgfpathlineto{\pgfqpoint{2.822972in}{0.569091in}}%
\pgfpathlineto{\pgfqpoint{2.813898in}{0.576352in}}%
\pgfpathlineto{\pgfqpoint{2.816463in}{0.579447in}}%
\pgfpathlineto{\pgfqpoint{2.823282in}{0.578757in}}%
\pgfpathlineto{\pgfqpoint{2.827989in}{0.582136in}}%
\pgfpathlineto{\pgfqpoint{2.826200in}{0.594925in}}%
\pgfpathlineto{\pgfqpoint{2.834914in}{0.596202in}}%
\pgfpathlineto{\pgfqpoint{2.831860in}{0.616441in}}%
\pgfpathlineto{\pgfqpoint{2.844620in}{0.618162in}}%
\pgfpathlineto{\pgfqpoint{2.845245in}{0.614077in}}%
\pgfpathlineto{\pgfqpoint{2.854959in}{0.618679in}}%
\pgfpathlineto{\pgfqpoint{2.868347in}{0.624403in}}%
\pgfpathlineto{\pgfqpoint{2.873084in}{0.622520in}}%
\pgfpathlineto{\pgfqpoint{2.876644in}{0.614106in}}%
\pgfpathlineto{\pgfqpoint{2.884207in}{0.608000in}}%
\pgfpathlineto{\pgfqpoint{2.886970in}{0.590796in}}%
\pgfpathlineto{\pgfqpoint{2.886626in}{0.584033in}}%
\pgfpathlineto{\pgfqpoint{2.843629in}{0.577826in}}%
\pgfpathlineto{\pgfqpoint{2.854800in}{0.566437in}}%
\pgfpathclose%
\pgfusepath{fill}%
\end{pgfscope}%
\begin{pgfscope}%
\pgfpathrectangle{\pgfqpoint{0.100000in}{0.100000in}}{\pgfqpoint{3.420221in}{2.189500in}}%
\pgfusepath{clip}%
\pgfsetbuttcap%
\pgfsetmiterjoin%
\definecolor{currentfill}{rgb}{0.000000,0.647059,0.676471}%
\pgfsetfillcolor{currentfill}%
\pgfsetlinewidth{0.000000pt}%
\definecolor{currentstroke}{rgb}{0.000000,0.000000,0.000000}%
\pgfsetstrokecolor{currentstroke}%
\pgfsetstrokeopacity{0.000000}%
\pgfsetdash{}{0pt}%
\pgfpathmoveto{\pgfqpoint{1.903269in}{1.023911in}}%
\pgfpathlineto{\pgfqpoint{1.929375in}{1.023511in}}%
\pgfpathlineto{\pgfqpoint{1.929595in}{1.039864in}}%
\pgfpathlineto{\pgfqpoint{1.955518in}{1.039538in}}%
\pgfpathlineto{\pgfqpoint{1.955433in}{1.029731in}}%
\pgfpathlineto{\pgfqpoint{1.958670in}{1.029717in}}%
\pgfpathlineto{\pgfqpoint{1.958575in}{1.016653in}}%
\pgfpathlineto{\pgfqpoint{1.951897in}{1.016704in}}%
\pgfpathlineto{\pgfqpoint{1.951772in}{1.010118in}}%
\pgfpathlineto{\pgfqpoint{1.948516in}{1.003596in}}%
\pgfpathlineto{\pgfqpoint{1.942047in}{1.003672in}}%
\pgfpathlineto{\pgfqpoint{1.941973in}{0.997147in}}%
\pgfpathlineto{\pgfqpoint{1.913786in}{0.997539in}}%
\pgfpathlineto{\pgfqpoint{1.910517in}{1.007993in}}%
\pgfpathlineto{\pgfqpoint{1.902764in}{1.006049in}}%
\pgfpathclose%
\pgfusepath{fill}%
\end{pgfscope}%
\begin{pgfscope}%
\pgfpathrectangle{\pgfqpoint{0.100000in}{0.100000in}}{\pgfqpoint{3.420221in}{2.189500in}}%
\pgfusepath{clip}%
\pgfsetbuttcap%
\pgfsetmiterjoin%
\definecolor{currentfill}{rgb}{0.000000,0.650980,0.674510}%
\pgfsetfillcolor{currentfill}%
\pgfsetlinewidth{0.000000pt}%
\definecolor{currentstroke}{rgb}{0.000000,0.000000,0.000000}%
\pgfsetstrokecolor{currentstroke}%
\pgfsetstrokeopacity{0.000000}%
\pgfsetdash{}{0pt}%
\pgfpathmoveto{\pgfqpoint{2.800436in}{1.105317in}}%
\pgfpathlineto{\pgfqpoint{2.794469in}{1.107334in}}%
\pgfpathlineto{\pgfqpoint{2.790484in}{1.100040in}}%
\pgfpathlineto{\pgfqpoint{2.787002in}{1.101858in}}%
\pgfpathlineto{\pgfqpoint{2.782511in}{1.113410in}}%
\pgfpathlineto{\pgfqpoint{2.785358in}{1.115371in}}%
\pgfpathlineto{\pgfqpoint{2.788542in}{1.126307in}}%
\pgfpathlineto{\pgfqpoint{2.782179in}{1.134821in}}%
\pgfpathlineto{\pgfqpoint{2.785919in}{1.142179in}}%
\pgfpathlineto{\pgfqpoint{2.786655in}{1.149234in}}%
\pgfpathlineto{\pgfqpoint{2.792191in}{1.153548in}}%
\pgfpathlineto{\pgfqpoint{2.802926in}{1.154951in}}%
\pgfpathlineto{\pgfqpoint{2.801362in}{1.152916in}}%
\pgfpathlineto{\pgfqpoint{2.801618in}{1.143842in}}%
\pgfpathlineto{\pgfqpoint{2.799777in}{1.138089in}}%
\pgfpathlineto{\pgfqpoint{2.806325in}{1.135560in}}%
\pgfpathlineto{\pgfqpoint{2.818512in}{1.126411in}}%
\pgfpathlineto{\pgfqpoint{2.814436in}{1.119164in}}%
\pgfpathlineto{\pgfqpoint{2.806628in}{1.118391in}}%
\pgfpathclose%
\pgfusepath{fill}%
\end{pgfscope}%
\begin{pgfscope}%
\pgfpathrectangle{\pgfqpoint{0.100000in}{0.100000in}}{\pgfqpoint{3.420221in}{2.189500in}}%
\pgfusepath{clip}%
\pgfsetbuttcap%
\pgfsetmiterjoin%
\definecolor{currentfill}{rgb}{0.000000,0.458824,0.770588}%
\pgfsetfillcolor{currentfill}%
\pgfsetlinewidth{0.000000pt}%
\definecolor{currentstroke}{rgb}{0.000000,0.000000,0.000000}%
\pgfsetstrokecolor{currentstroke}%
\pgfsetstrokeopacity{0.000000}%
\pgfsetdash{}{0pt}%
\pgfpathmoveto{\pgfqpoint{1.723158in}{1.751959in}}%
\pgfpathlineto{\pgfqpoint{1.752338in}{1.750273in}}%
\pgfpathlineto{\pgfqpoint{1.759521in}{1.749882in}}%
\pgfpathlineto{\pgfqpoint{1.758248in}{1.723697in}}%
\pgfpathlineto{\pgfqpoint{1.752841in}{1.723989in}}%
\pgfpathlineto{\pgfqpoint{1.713940in}{1.726154in}}%
\pgfpathlineto{\pgfqpoint{1.713369in}{1.733308in}}%
\pgfpathlineto{\pgfqpoint{1.720592in}{1.734573in}}%
\pgfpathlineto{\pgfqpoint{1.719130in}{1.741235in}}%
\pgfpathclose%
\pgfusepath{fill}%
\end{pgfscope}%
\begin{pgfscope}%
\pgfpathrectangle{\pgfqpoint{0.100000in}{0.100000in}}{\pgfqpoint{3.420221in}{2.189500in}}%
\pgfusepath{clip}%
\pgfsetbuttcap%
\pgfsetmiterjoin%
\definecolor{currentfill}{rgb}{0.000000,0.478431,0.760784}%
\pgfsetfillcolor{currentfill}%
\pgfsetlinewidth{0.000000pt}%
\definecolor{currentstroke}{rgb}{0.000000,0.000000,0.000000}%
\pgfsetstrokecolor{currentstroke}%
\pgfsetstrokeopacity{0.000000}%
\pgfsetdash{}{0pt}%
\pgfpathmoveto{\pgfqpoint{1.425448in}{0.777887in}}%
\pgfpathlineto{\pgfqpoint{1.444666in}{0.776020in}}%
\pgfpathlineto{\pgfqpoint{1.486630in}{0.772157in}}%
\pgfpathlineto{\pgfqpoint{1.487214in}{0.778661in}}%
\pgfpathlineto{\pgfqpoint{1.504099in}{0.777089in}}%
\pgfpathlineto{\pgfqpoint{1.501299in}{0.744442in}}%
\pgfpathlineto{\pgfqpoint{1.503330in}{0.744262in}}%
\pgfpathlineto{\pgfqpoint{1.501013in}{0.717314in}}%
\pgfpathlineto{\pgfqpoint{1.497408in}{0.715681in}}%
\pgfpathlineto{\pgfqpoint{1.490676in}{0.722476in}}%
\pgfpathlineto{\pgfqpoint{1.485843in}{0.724518in}}%
\pgfpathlineto{\pgfqpoint{1.444647in}{0.682354in}}%
\pgfpathlineto{\pgfqpoint{1.413730in}{0.711118in}}%
\pgfpathclose%
\pgfusepath{fill}%
\end{pgfscope}%
\begin{pgfscope}%
\pgfpathrectangle{\pgfqpoint{0.100000in}{0.100000in}}{\pgfqpoint{3.420221in}{2.189500in}}%
\pgfusepath{clip}%
\pgfsetbuttcap%
\pgfsetmiterjoin%
\definecolor{currentfill}{rgb}{0.000000,0.156863,0.921569}%
\pgfsetfillcolor{currentfill}%
\pgfsetlinewidth{0.000000pt}%
\definecolor{currentstroke}{rgb}{0.000000,0.000000,0.000000}%
\pgfsetstrokecolor{currentstroke}%
\pgfsetstrokeopacity{0.000000}%
\pgfsetdash{}{0pt}%
\pgfpathmoveto{\pgfqpoint{3.105841in}{1.164131in}}%
\pgfpathlineto{\pgfqpoint{3.100694in}{1.172877in}}%
\pgfpathlineto{\pgfqpoint{3.101069in}{1.183376in}}%
\pgfpathlineto{\pgfqpoint{3.094922in}{1.187227in}}%
\pgfpathlineto{\pgfqpoint{3.085581in}{1.188260in}}%
\pgfpathlineto{\pgfqpoint{3.084398in}{1.198942in}}%
\pgfpathlineto{\pgfqpoint{3.083997in}{1.206525in}}%
\pgfpathlineto{\pgfqpoint{3.101354in}{1.230845in}}%
\pgfpathlineto{\pgfqpoint{3.104158in}{1.232425in}}%
\pgfpathlineto{\pgfqpoint{3.110255in}{1.231123in}}%
\pgfpathlineto{\pgfqpoint{3.115141in}{1.237775in}}%
\pgfpathlineto{\pgfqpoint{3.127172in}{1.235656in}}%
\pgfpathlineto{\pgfqpoint{3.132774in}{1.237912in}}%
\pgfpathlineto{\pgfqpoint{3.138022in}{1.226695in}}%
\pgfpathlineto{\pgfqpoint{3.145229in}{1.214638in}}%
\pgfpathlineto{\pgfqpoint{3.156706in}{1.190091in}}%
\pgfpathlineto{\pgfqpoint{3.155923in}{1.189701in}}%
\pgfpathlineto{\pgfqpoint{3.148744in}{1.203223in}}%
\pgfpathlineto{\pgfqpoint{3.145781in}{1.198429in}}%
\pgfpathlineto{\pgfqpoint{3.158184in}{1.178753in}}%
\pgfpathlineto{\pgfqpoint{3.151158in}{1.182981in}}%
\pgfpathlineto{\pgfqpoint{3.144363in}{1.183145in}}%
\pgfpathlineto{\pgfqpoint{3.132793in}{1.174692in}}%
\pgfpathlineto{\pgfqpoint{3.122995in}{1.171358in}}%
\pgfpathlineto{\pgfqpoint{3.117197in}{1.165007in}}%
\pgfpathclose%
\pgfusepath{fill}%
\end{pgfscope}%
\begin{pgfscope}%
\pgfpathrectangle{\pgfqpoint{0.100000in}{0.100000in}}{\pgfqpoint{3.420221in}{2.189500in}}%
\pgfusepath{clip}%
\pgfsetbuttcap%
\pgfsetmiterjoin%
\definecolor{currentfill}{rgb}{0.000000,0.572549,0.713725}%
\pgfsetfillcolor{currentfill}%
\pgfsetlinewidth{0.000000pt}%
\definecolor{currentstroke}{rgb}{0.000000,0.000000,0.000000}%
\pgfsetstrokecolor{currentstroke}%
\pgfsetstrokeopacity{0.000000}%
\pgfsetdash{}{0pt}%
\pgfpathmoveto{\pgfqpoint{1.903647in}{1.046608in}}%
\pgfpathlineto{\pgfqpoint{1.872104in}{1.047284in}}%
\pgfpathlineto{\pgfqpoint{1.859232in}{1.050839in}}%
\pgfpathlineto{\pgfqpoint{1.859580in}{1.064290in}}%
\pgfpathlineto{\pgfqpoint{1.853111in}{1.064475in}}%
\pgfpathlineto{\pgfqpoint{1.853928in}{1.096638in}}%
\pgfpathlineto{\pgfqpoint{1.886511in}{1.096393in}}%
\pgfpathlineto{\pgfqpoint{1.887590in}{1.092701in}}%
\pgfpathlineto{\pgfqpoint{1.879630in}{1.089519in}}%
\pgfpathlineto{\pgfqpoint{1.879345in}{1.076498in}}%
\pgfpathlineto{\pgfqpoint{1.885817in}{1.076349in}}%
\pgfpathlineto{\pgfqpoint{1.885669in}{1.069846in}}%
\pgfpathlineto{\pgfqpoint{1.892089in}{1.069712in}}%
\pgfpathlineto{\pgfqpoint{1.891950in}{1.063142in}}%
\pgfpathlineto{\pgfqpoint{1.903919in}{1.062991in}}%
\pgfpathclose%
\pgfusepath{fill}%
\end{pgfscope}%
\begin{pgfscope}%
\pgfpathrectangle{\pgfqpoint{0.100000in}{0.100000in}}{\pgfqpoint{3.420221in}{2.189500in}}%
\pgfusepath{clip}%
\pgfsetbuttcap%
\pgfsetmiterjoin%
\definecolor{currentfill}{rgb}{0.000000,0.137255,0.931373}%
\pgfsetfillcolor{currentfill}%
\pgfsetlinewidth{0.000000pt}%
\definecolor{currentstroke}{rgb}{0.000000,0.000000,0.000000}%
\pgfsetstrokecolor{currentstroke}%
\pgfsetstrokeopacity{0.000000}%
\pgfsetdash{}{0pt}%
\pgfpathmoveto{\pgfqpoint{1.765780in}{1.459855in}}%
\pgfpathlineto{\pgfqpoint{1.764328in}{1.433867in}}%
\pgfpathlineto{\pgfqpoint{1.751998in}{1.434416in}}%
\pgfpathlineto{\pgfqpoint{1.706936in}{1.436820in}}%
\pgfpathlineto{\pgfqpoint{1.708415in}{1.488792in}}%
\pgfpathlineto{\pgfqpoint{1.675394in}{1.490861in}}%
\pgfpathlineto{\pgfqpoint{1.676865in}{1.516905in}}%
\pgfpathlineto{\pgfqpoint{1.714561in}{1.514444in}}%
\pgfpathlineto{\pgfqpoint{1.741283in}{1.513010in}}%
\pgfpathlineto{\pgfqpoint{1.739953in}{1.487110in}}%
\pgfpathlineto{\pgfqpoint{1.766492in}{1.485831in}}%
\pgfpathclose%
\pgfusepath{fill}%
\end{pgfscope}%
\begin{pgfscope}%
\pgfpathrectangle{\pgfqpoint{0.100000in}{0.100000in}}{\pgfqpoint{3.420221in}{2.189500in}}%
\pgfusepath{clip}%
\pgfsetbuttcap%
\pgfsetmiterjoin%
\definecolor{currentfill}{rgb}{0.000000,0.564706,0.717647}%
\pgfsetfillcolor{currentfill}%
\pgfsetlinewidth{0.000000pt}%
\definecolor{currentstroke}{rgb}{0.000000,0.000000,0.000000}%
\pgfsetstrokecolor{currentstroke}%
\pgfsetstrokeopacity{0.000000}%
\pgfsetdash{}{0pt}%
\pgfpathmoveto{\pgfqpoint{2.290209in}{1.015457in}}%
\pgfpathlineto{\pgfqpoint{2.270403in}{1.015231in}}%
\pgfpathlineto{\pgfqpoint{2.244240in}{1.013593in}}%
\pgfpathlineto{\pgfqpoint{2.243212in}{1.046759in}}%
\pgfpathlineto{\pgfqpoint{2.233111in}{1.046699in}}%
\pgfpathlineto{\pgfqpoint{2.223473in}{1.046349in}}%
\pgfpathlineto{\pgfqpoint{2.222896in}{1.071879in}}%
\pgfpathlineto{\pgfqpoint{2.228322in}{1.073038in}}%
\pgfpathlineto{\pgfqpoint{2.226905in}{1.082822in}}%
\pgfpathlineto{\pgfqpoint{2.218643in}{1.091687in}}%
\pgfpathlineto{\pgfqpoint{2.235543in}{1.092426in}}%
\pgfpathlineto{\pgfqpoint{2.290087in}{1.095036in}}%
\pgfpathlineto{\pgfqpoint{2.300022in}{1.087035in}}%
\pgfpathlineto{\pgfqpoint{2.299205in}{1.078555in}}%
\pgfpathlineto{\pgfqpoint{2.291337in}{1.071587in}}%
\pgfpathlineto{\pgfqpoint{2.285733in}{1.064174in}}%
\pgfpathlineto{\pgfqpoint{2.287935in}{1.057260in}}%
\pgfpathclose%
\pgfusepath{fill}%
\end{pgfscope}%
\begin{pgfscope}%
\pgfpathrectangle{\pgfqpoint{0.100000in}{0.100000in}}{\pgfqpoint{3.420221in}{2.189500in}}%
\pgfusepath{clip}%
\pgfsetbuttcap%
\pgfsetmiterjoin%
\definecolor{currentfill}{rgb}{0.000000,0.623529,0.688235}%
\pgfsetfillcolor{currentfill}%
\pgfsetlinewidth{0.000000pt}%
\definecolor{currentstroke}{rgb}{0.000000,0.000000,0.000000}%
\pgfsetstrokecolor{currentstroke}%
\pgfsetstrokeopacity{0.000000}%
\pgfsetdash{}{0pt}%
\pgfpathmoveto{\pgfqpoint{1.316812in}{1.167972in}}%
\pgfpathlineto{\pgfqpoint{1.307064in}{1.186966in}}%
\pgfpathlineto{\pgfqpoint{1.308676in}{1.199455in}}%
\pgfpathlineto{\pgfqpoint{1.282049in}{1.202648in}}%
\pgfpathlineto{\pgfqpoint{1.282357in}{1.204916in}}%
\pgfpathlineto{\pgfqpoint{1.283908in}{1.223805in}}%
\pgfpathlineto{\pgfqpoint{1.286945in}{1.243309in}}%
\pgfpathlineto{\pgfqpoint{1.295173in}{1.243509in}}%
\pgfpathlineto{\pgfqpoint{1.299806in}{1.278387in}}%
\pgfpathlineto{\pgfqpoint{1.343652in}{1.272532in}}%
\pgfpathlineto{\pgfqpoint{1.353623in}{1.271379in}}%
\pgfpathlineto{\pgfqpoint{1.357626in}{1.272609in}}%
\pgfpathlineto{\pgfqpoint{1.374472in}{1.247898in}}%
\pgfpathlineto{\pgfqpoint{1.378322in}{1.233966in}}%
\pgfpathlineto{\pgfqpoint{1.384205in}{1.227909in}}%
\pgfpathlineto{\pgfqpoint{1.386428in}{1.224315in}}%
\pgfpathlineto{\pgfqpoint{1.380281in}{1.207129in}}%
\pgfpathlineto{\pgfqpoint{1.392609in}{1.208657in}}%
\pgfpathlineto{\pgfqpoint{1.398337in}{1.200654in}}%
\pgfpathlineto{\pgfqpoint{1.399733in}{1.188783in}}%
\pgfpathlineto{\pgfqpoint{1.397910in}{1.180875in}}%
\pgfpathlineto{\pgfqpoint{1.395369in}{1.158691in}}%
\pgfpathlineto{\pgfqpoint{1.391467in}{1.159123in}}%
\pgfpathclose%
\pgfusepath{fill}%
\end{pgfscope}%
\begin{pgfscope}%
\pgfpathrectangle{\pgfqpoint{0.100000in}{0.100000in}}{\pgfqpoint{3.420221in}{2.189500in}}%
\pgfusepath{clip}%
\pgfsetbuttcap%
\pgfsetmiterjoin%
\definecolor{currentfill}{rgb}{0.000000,0.368627,0.815686}%
\pgfsetfillcolor{currentfill}%
\pgfsetlinewidth{0.000000pt}%
\definecolor{currentstroke}{rgb}{0.000000,0.000000,0.000000}%
\pgfsetstrokecolor{currentstroke}%
\pgfsetstrokeopacity{0.000000}%
\pgfsetdash{}{0pt}%
\pgfpathmoveto{\pgfqpoint{2.094629in}{1.394208in}}%
\pgfpathlineto{\pgfqpoint{2.084243in}{1.394033in}}%
\pgfpathlineto{\pgfqpoint{2.083525in}{1.437583in}}%
\pgfpathlineto{\pgfqpoint{2.096481in}{1.437752in}}%
\pgfpathlineto{\pgfqpoint{2.109378in}{1.438016in}}%
\pgfpathlineto{\pgfqpoint{2.109881in}{1.418359in}}%
\pgfpathlineto{\pgfqpoint{2.135845in}{1.419001in}}%
\pgfpathlineto{\pgfqpoint{2.136517in}{1.395922in}}%
\pgfpathlineto{\pgfqpoint{2.132169in}{1.395705in}}%
\pgfpathclose%
\pgfusepath{fill}%
\end{pgfscope}%
\begin{pgfscope}%
\pgfpathrectangle{\pgfqpoint{0.100000in}{0.100000in}}{\pgfqpoint{3.420221in}{2.189500in}}%
\pgfusepath{clip}%
\pgfsetbuttcap%
\pgfsetmiterjoin%
\definecolor{currentfill}{rgb}{0.000000,0.039216,0.980392}%
\pgfsetfillcolor{currentfill}%
\pgfsetlinewidth{0.000000pt}%
\definecolor{currentstroke}{rgb}{0.000000,0.000000,0.000000}%
\pgfsetstrokecolor{currentstroke}%
\pgfsetstrokeopacity{0.000000}%
\pgfsetdash{}{0pt}%
\pgfpathmoveto{\pgfqpoint{1.849494in}{0.583546in}}%
\pgfpathlineto{\pgfqpoint{1.830739in}{0.572366in}}%
\pgfpathlineto{\pgfqpoint{1.828448in}{0.573892in}}%
\pgfpathlineto{\pgfqpoint{1.823930in}{0.575456in}}%
\pgfpathlineto{\pgfqpoint{1.813002in}{0.590178in}}%
\pgfpathlineto{\pgfqpoint{1.804564in}{0.582416in}}%
\pgfpathlineto{\pgfqpoint{1.802749in}{0.589771in}}%
\pgfpathlineto{\pgfqpoint{1.785709in}{0.604742in}}%
\pgfpathlineto{\pgfqpoint{1.777787in}{0.597413in}}%
\pgfpathlineto{\pgfqpoint{1.766971in}{0.613112in}}%
\pgfpathlineto{\pgfqpoint{1.767804in}{0.640448in}}%
\pgfpathlineto{\pgfqpoint{1.779559in}{0.640164in}}%
\pgfpathlineto{\pgfqpoint{1.798006in}{0.633746in}}%
\pgfpathlineto{\pgfqpoint{1.798798in}{0.639328in}}%
\pgfpathlineto{\pgfqpoint{1.809514in}{0.660601in}}%
\pgfpathlineto{\pgfqpoint{1.818539in}{0.669780in}}%
\pgfpathlineto{\pgfqpoint{1.831660in}{0.666323in}}%
\pgfpathlineto{\pgfqpoint{1.851520in}{0.656853in}}%
\pgfpathlineto{\pgfqpoint{1.855447in}{0.667142in}}%
\pgfpathlineto{\pgfqpoint{1.867874in}{0.674158in}}%
\pgfpathlineto{\pgfqpoint{1.883685in}{0.682871in}}%
\pgfpathlineto{\pgfqpoint{1.886405in}{0.679461in}}%
\pgfpathlineto{\pgfqpoint{1.888100in}{0.669322in}}%
\pgfpathlineto{\pgfqpoint{1.895025in}{0.661414in}}%
\pgfpathlineto{\pgfqpoint{1.896622in}{0.654234in}}%
\pgfpathlineto{\pgfqpoint{1.874072in}{0.641534in}}%
\pgfpathlineto{\pgfqpoint{1.877588in}{0.638837in}}%
\pgfpathlineto{\pgfqpoint{1.880110in}{0.632064in}}%
\pgfpathlineto{\pgfqpoint{1.888049in}{0.623285in}}%
\pgfpathlineto{\pgfqpoint{1.894784in}{0.621395in}}%
\pgfpathlineto{\pgfqpoint{1.888676in}{0.617453in}}%
\pgfpathlineto{\pgfqpoint{1.884510in}{0.611267in}}%
\pgfpathlineto{\pgfqpoint{1.881080in}{0.607961in}}%
\pgfpathlineto{\pgfqpoint{1.869261in}{0.603314in}}%
\pgfpathclose%
\pgfusepath{fill}%
\end{pgfscope}%
\begin{pgfscope}%
\pgfpathrectangle{\pgfqpoint{0.100000in}{0.100000in}}{\pgfqpoint{3.420221in}{2.189500in}}%
\pgfusepath{clip}%
\pgfsetbuttcap%
\pgfsetmiterjoin%
\definecolor{currentfill}{rgb}{0.000000,0.623529,0.688235}%
\pgfsetfillcolor{currentfill}%
\pgfsetlinewidth{0.000000pt}%
\definecolor{currentstroke}{rgb}{0.000000,0.000000,0.000000}%
\pgfsetstrokecolor{currentstroke}%
\pgfsetstrokeopacity{0.000000}%
\pgfsetdash{}{0pt}%
\pgfpathmoveto{\pgfqpoint{2.174251in}{0.746294in}}%
\pgfpathlineto{\pgfqpoint{2.173423in}{0.775966in}}%
\pgfpathlineto{\pgfqpoint{2.166636in}{0.782338in}}%
\pgfpathlineto{\pgfqpoint{2.166098in}{0.802102in}}%
\pgfpathlineto{\pgfqpoint{2.159545in}{0.801966in}}%
\pgfpathlineto{\pgfqpoint{2.154579in}{0.808246in}}%
\pgfpathlineto{\pgfqpoint{2.146242in}{0.808318in}}%
\pgfpathlineto{\pgfqpoint{2.145790in}{0.827512in}}%
\pgfpathlineto{\pgfqpoint{2.187216in}{0.828233in}}%
\pgfpathlineto{\pgfqpoint{2.225649in}{0.829433in}}%
\pgfpathlineto{\pgfqpoint{2.227203in}{0.829507in}}%
\pgfpathlineto{\pgfqpoint{2.223821in}{0.820682in}}%
\pgfpathlineto{\pgfqpoint{2.220920in}{0.819497in}}%
\pgfpathlineto{\pgfqpoint{2.215228in}{0.806156in}}%
\pgfpathlineto{\pgfqpoint{2.218284in}{0.797196in}}%
\pgfpathlineto{\pgfqpoint{2.227852in}{0.797545in}}%
\pgfpathlineto{\pgfqpoint{2.227198in}{0.794188in}}%
\pgfpathlineto{\pgfqpoint{2.225587in}{0.790893in}}%
\pgfpathlineto{\pgfqpoint{2.226956in}{0.777554in}}%
\pgfpathlineto{\pgfqpoint{2.223339in}{0.775609in}}%
\pgfpathlineto{\pgfqpoint{2.225943in}{0.768896in}}%
\pgfpathlineto{\pgfqpoint{2.226362in}{0.758830in}}%
\pgfpathlineto{\pgfqpoint{2.221561in}{0.744501in}}%
\pgfpathlineto{\pgfqpoint{2.219956in}{0.751027in}}%
\pgfpathlineto{\pgfqpoint{2.208482in}{0.743229in}}%
\pgfpathlineto{\pgfqpoint{2.203626in}{0.749572in}}%
\pgfpathlineto{\pgfqpoint{2.200391in}{0.747140in}}%
\pgfpathclose%
\pgfusepath{fill}%
\end{pgfscope}%
\begin{pgfscope}%
\pgfpathrectangle{\pgfqpoint{0.100000in}{0.100000in}}{\pgfqpoint{3.420221in}{2.189500in}}%
\pgfusepath{clip}%
\pgfsetbuttcap%
\pgfsetmiterjoin%
\definecolor{currentfill}{rgb}{0.000000,0.721569,0.639216}%
\pgfsetfillcolor{currentfill}%
\pgfsetlinewidth{0.000000pt}%
\definecolor{currentstroke}{rgb}{0.000000,0.000000,0.000000}%
\pgfsetstrokecolor{currentstroke}%
\pgfsetstrokeopacity{0.000000}%
\pgfsetdash{}{0pt}%
\pgfpathmoveto{\pgfqpoint{2.480510in}{1.822354in}}%
\pgfpathlineto{\pgfqpoint{2.460000in}{1.818731in}}%
\pgfpathlineto{\pgfqpoint{2.457172in}{1.815447in}}%
\pgfpathlineto{\pgfqpoint{2.456524in}{1.807045in}}%
\pgfpathlineto{\pgfqpoint{2.447578in}{1.802453in}}%
\pgfpathlineto{\pgfqpoint{2.443104in}{1.794917in}}%
\pgfpathlineto{\pgfqpoint{2.437992in}{1.796409in}}%
\pgfpathlineto{\pgfqpoint{2.443920in}{1.804728in}}%
\pgfpathlineto{\pgfqpoint{2.446686in}{1.811454in}}%
\pgfpathlineto{\pgfqpoint{2.434284in}{1.805951in}}%
\pgfpathlineto{\pgfqpoint{2.431284in}{1.798778in}}%
\pgfpathlineto{\pgfqpoint{2.425026in}{1.794443in}}%
\pgfpathlineto{\pgfqpoint{2.419757in}{1.798989in}}%
\pgfpathlineto{\pgfqpoint{2.414211in}{1.792771in}}%
\pgfpathlineto{\pgfqpoint{2.410143in}{1.784162in}}%
\pgfpathlineto{\pgfqpoint{2.406801in}{1.783876in}}%
\pgfpathlineto{\pgfqpoint{2.404677in}{1.810052in}}%
\pgfpathlineto{\pgfqpoint{2.401932in}{1.816384in}}%
\pgfpathlineto{\pgfqpoint{2.388897in}{1.815352in}}%
\pgfpathlineto{\pgfqpoint{2.387357in}{1.834992in}}%
\pgfpathlineto{\pgfqpoint{2.361254in}{1.832926in}}%
\pgfpathlineto{\pgfqpoint{2.360277in}{1.845958in}}%
\pgfpathlineto{\pgfqpoint{2.358860in}{1.865458in}}%
\pgfpathlineto{\pgfqpoint{2.365282in}{1.866024in}}%
\pgfpathlineto{\pgfqpoint{2.362030in}{1.872303in}}%
\pgfpathlineto{\pgfqpoint{2.361209in}{1.883366in}}%
\pgfpathlineto{\pgfqpoint{2.368592in}{1.883992in}}%
\pgfpathlineto{\pgfqpoint{2.383111in}{1.877597in}}%
\pgfpathlineto{\pgfqpoint{2.385803in}{1.872340in}}%
\pgfpathlineto{\pgfqpoint{2.400272in}{1.855042in}}%
\pgfpathlineto{\pgfqpoint{2.411730in}{1.855873in}}%
\pgfpathlineto{\pgfqpoint{2.416510in}{1.859078in}}%
\pgfpathlineto{\pgfqpoint{2.424564in}{1.852522in}}%
\pgfpathlineto{\pgfqpoint{2.435516in}{1.862744in}}%
\pgfpathlineto{\pgfqpoint{2.439606in}{1.856500in}}%
\pgfpathlineto{\pgfqpoint{2.445086in}{1.863418in}}%
\pgfpathlineto{\pgfqpoint{2.462193in}{1.872558in}}%
\pgfpathlineto{\pgfqpoint{2.475157in}{1.876631in}}%
\pgfpathclose%
\pgfusepath{fill}%
\end{pgfscope}%
\begin{pgfscope}%
\pgfpathrectangle{\pgfqpoint{0.100000in}{0.100000in}}{\pgfqpoint{3.420221in}{2.189500in}}%
\pgfusepath{clip}%
\pgfsetbuttcap%
\pgfsetmiterjoin%
\definecolor{currentfill}{rgb}{0.000000,0.376471,0.811765}%
\pgfsetfillcolor{currentfill}%
\pgfsetlinewidth{0.000000pt}%
\definecolor{currentstroke}{rgb}{0.000000,0.000000,0.000000}%
\pgfsetstrokecolor{currentstroke}%
\pgfsetstrokeopacity{0.000000}%
\pgfsetdash{}{0pt}%
\pgfpathmoveto{\pgfqpoint{3.118270in}{1.545029in}}%
\pgfpathlineto{\pgfqpoint{3.110703in}{1.535350in}}%
\pgfpathlineto{\pgfqpoint{3.101171in}{1.529477in}}%
\pgfpathlineto{\pgfqpoint{3.093860in}{1.537960in}}%
\pgfpathlineto{\pgfqpoint{3.097469in}{1.541665in}}%
\pgfpathlineto{\pgfqpoint{3.088106in}{1.549393in}}%
\pgfpathlineto{\pgfqpoint{3.084104in}{1.550087in}}%
\pgfpathlineto{\pgfqpoint{3.080593in}{1.545086in}}%
\pgfpathlineto{\pgfqpoint{3.083822in}{1.540551in}}%
\pgfpathlineto{\pgfqpoint{3.070632in}{1.530438in}}%
\pgfpathlineto{\pgfqpoint{3.058402in}{1.530673in}}%
\pgfpathlineto{\pgfqpoint{3.055161in}{1.524928in}}%
\pgfpathlineto{\pgfqpoint{3.055162in}{1.518667in}}%
\pgfpathlineto{\pgfqpoint{3.051509in}{1.515893in}}%
\pgfpathlineto{\pgfqpoint{3.049192in}{1.519913in}}%
\pgfpathlineto{\pgfqpoint{3.044281in}{1.518095in}}%
\pgfpathlineto{\pgfqpoint{3.041575in}{1.526894in}}%
\pgfpathlineto{\pgfqpoint{3.033892in}{1.534319in}}%
\pgfpathlineto{\pgfqpoint{3.031305in}{1.540824in}}%
\pgfpathlineto{\pgfqpoint{3.033936in}{1.541521in}}%
\pgfpathlineto{\pgfqpoint{3.041840in}{1.554670in}}%
\pgfpathlineto{\pgfqpoint{3.047204in}{1.555823in}}%
\pgfpathlineto{\pgfqpoint{3.048598in}{1.571517in}}%
\pgfpathlineto{\pgfqpoint{3.048128in}{1.581912in}}%
\pgfpathlineto{\pgfqpoint{3.052654in}{1.583281in}}%
\pgfpathlineto{\pgfqpoint{3.088600in}{1.590182in}}%
\pgfpathlineto{\pgfqpoint{3.081732in}{1.616141in}}%
\pgfpathlineto{\pgfqpoint{3.088473in}{1.617631in}}%
\pgfpathlineto{\pgfqpoint{3.093833in}{1.614161in}}%
\pgfpathlineto{\pgfqpoint{3.095767in}{1.609352in}}%
\pgfpathlineto{\pgfqpoint{3.102490in}{1.609314in}}%
\pgfpathlineto{\pgfqpoint{3.109194in}{1.603181in}}%
\pgfpathlineto{\pgfqpoint{3.111826in}{1.593853in}}%
\pgfpathlineto{\pgfqpoint{3.116755in}{1.586204in}}%
\pgfpathlineto{\pgfqpoint{3.126285in}{1.582420in}}%
\pgfpathlineto{\pgfqpoint{3.132035in}{1.583034in}}%
\pgfpathlineto{\pgfqpoint{3.135304in}{1.578757in}}%
\pgfpathlineto{\pgfqpoint{3.129045in}{1.571941in}}%
\pgfpathlineto{\pgfqpoint{3.127933in}{1.563516in}}%
\pgfpathclose%
\pgfusepath{fill}%
\end{pgfscope}%
\begin{pgfscope}%
\pgfpathrectangle{\pgfqpoint{0.100000in}{0.100000in}}{\pgfqpoint{3.420221in}{2.189500in}}%
\pgfusepath{clip}%
\pgfsetbuttcap%
\pgfsetmiterjoin%
\definecolor{currentfill}{rgb}{0.000000,0.278431,0.860784}%
\pgfsetfillcolor{currentfill}%
\pgfsetlinewidth{0.000000pt}%
\definecolor{currentstroke}{rgb}{0.000000,0.000000,0.000000}%
\pgfsetstrokecolor{currentstroke}%
\pgfsetstrokeopacity{0.000000}%
\pgfsetdash{}{0pt}%
\pgfpathmoveto{\pgfqpoint{1.766978in}{1.290183in}}%
\pgfpathlineto{\pgfqpoint{1.799364in}{1.288818in}}%
\pgfpathlineto{\pgfqpoint{1.798902in}{1.269198in}}%
\pgfpathlineto{\pgfqpoint{1.798329in}{1.256189in}}%
\pgfpathlineto{\pgfqpoint{1.734136in}{1.259007in}}%
\pgfpathlineto{\pgfqpoint{1.700925in}{1.260846in}}%
\pgfpathlineto{\pgfqpoint{1.703121in}{1.293354in}}%
\pgfpathclose%
\pgfusepath{fill}%
\end{pgfscope}%
\begin{pgfscope}%
\pgfpathrectangle{\pgfqpoint{0.100000in}{0.100000in}}{\pgfqpoint{3.420221in}{2.189500in}}%
\pgfusepath{clip}%
\pgfsetbuttcap%
\pgfsetmiterjoin%
\definecolor{currentfill}{rgb}{0.000000,0.768627,0.615686}%
\pgfsetfillcolor{currentfill}%
\pgfsetlinewidth{0.000000pt}%
\definecolor{currentstroke}{rgb}{0.000000,0.000000,0.000000}%
\pgfsetstrokecolor{currentstroke}%
\pgfsetstrokeopacity{0.000000}%
\pgfsetdash{}{0pt}%
\pgfpathmoveto{\pgfqpoint{0.567235in}{0.414344in}}%
\pgfpathlineto{\pgfqpoint{0.566100in}{0.418200in}}%
\pgfpathlineto{\pgfqpoint{0.568341in}{0.418949in}}%
\pgfpathlineto{\pgfqpoint{0.570367in}{0.417955in}}%
\pgfpathlineto{\pgfqpoint{0.570897in}{0.414189in}}%
\pgfpathlineto{\pgfqpoint{0.569410in}{0.413614in}}%
\pgfpathclose%
\pgfusepath{fill}%
\end{pgfscope}%
\begin{pgfscope}%
\pgfpathrectangle{\pgfqpoint{0.100000in}{0.100000in}}{\pgfqpoint{3.420221in}{2.189500in}}%
\pgfusepath{clip}%
\pgfsetbuttcap%
\pgfsetmiterjoin%
\definecolor{currentfill}{rgb}{0.000000,0.768627,0.615686}%
\pgfsetfillcolor{currentfill}%
\pgfsetlinewidth{0.000000pt}%
\definecolor{currentstroke}{rgb}{0.000000,0.000000,0.000000}%
\pgfsetstrokecolor{currentstroke}%
\pgfsetstrokeopacity{0.000000}%
\pgfsetdash{}{0pt}%
\pgfpathmoveto{\pgfqpoint{0.572037in}{0.412270in}}%
\pgfpathlineto{\pgfqpoint{0.573332in}{0.414873in}}%
\pgfpathlineto{\pgfqpoint{0.574794in}{0.414377in}}%
\pgfpathlineto{\pgfqpoint{0.575785in}{0.412473in}}%
\pgfpathclose%
\pgfusepath{fill}%
\end{pgfscope}%
\begin{pgfscope}%
\pgfpathrectangle{\pgfqpoint{0.100000in}{0.100000in}}{\pgfqpoint{3.420221in}{2.189500in}}%
\pgfusepath{clip}%
\pgfsetbuttcap%
\pgfsetmiterjoin%
\definecolor{currentfill}{rgb}{0.000000,0.768627,0.615686}%
\pgfsetfillcolor{currentfill}%
\pgfsetlinewidth{0.000000pt}%
\definecolor{currentstroke}{rgb}{0.000000,0.000000,0.000000}%
\pgfsetstrokecolor{currentstroke}%
\pgfsetstrokeopacity{0.000000}%
\pgfsetdash{}{0pt}%
\pgfpathmoveto{\pgfqpoint{0.648832in}{0.381631in}}%
\pgfpathlineto{\pgfqpoint{0.647191in}{0.379484in}}%
\pgfpathlineto{\pgfqpoint{0.646233in}{0.382875in}}%
\pgfpathlineto{\pgfqpoint{0.643601in}{0.382223in}}%
\pgfpathlineto{\pgfqpoint{0.646677in}{0.386649in}}%
\pgfpathlineto{\pgfqpoint{0.650087in}{0.386791in}}%
\pgfpathlineto{\pgfqpoint{0.649202in}{0.385056in}}%
\pgfpathlineto{\pgfqpoint{0.650453in}{0.383460in}}%
\pgfpathlineto{\pgfqpoint{0.653276in}{0.382432in}}%
\pgfpathlineto{\pgfqpoint{0.651322in}{0.379765in}}%
\pgfpathlineto{\pgfqpoint{0.650083in}{0.382038in}}%
\pgfpathclose%
\pgfusepath{fill}%
\end{pgfscope}%
\begin{pgfscope}%
\pgfpathrectangle{\pgfqpoint{0.100000in}{0.100000in}}{\pgfqpoint{3.420221in}{2.189500in}}%
\pgfusepath{clip}%
\pgfsetbuttcap%
\pgfsetmiterjoin%
\definecolor{currentfill}{rgb}{0.000000,0.768627,0.615686}%
\pgfsetfillcolor{currentfill}%
\pgfsetlinewidth{0.000000pt}%
\definecolor{currentstroke}{rgb}{0.000000,0.000000,0.000000}%
\pgfsetstrokecolor{currentstroke}%
\pgfsetstrokeopacity{0.000000}%
\pgfsetdash{}{0pt}%
\pgfpathmoveto{\pgfqpoint{0.621437in}{0.391796in}}%
\pgfpathlineto{\pgfqpoint{0.621654in}{0.394679in}}%
\pgfpathlineto{\pgfqpoint{0.624282in}{0.394875in}}%
\pgfpathlineto{\pgfqpoint{0.623936in}{0.392516in}}%
\pgfpathclose%
\pgfusepath{fill}%
\end{pgfscope}%
\begin{pgfscope}%
\pgfpathrectangle{\pgfqpoint{0.100000in}{0.100000in}}{\pgfqpoint{3.420221in}{2.189500in}}%
\pgfusepath{clip}%
\pgfsetbuttcap%
\pgfsetmiterjoin%
\definecolor{currentfill}{rgb}{0.000000,0.768627,0.615686}%
\pgfsetfillcolor{currentfill}%
\pgfsetlinewidth{0.000000pt}%
\definecolor{currentstroke}{rgb}{0.000000,0.000000,0.000000}%
\pgfsetstrokecolor{currentstroke}%
\pgfsetstrokeopacity{0.000000}%
\pgfsetdash{}{0pt}%
\pgfpathmoveto{\pgfqpoint{0.646605in}{0.371489in}}%
\pgfpathlineto{\pgfqpoint{0.653434in}{0.375383in}}%
\pgfpathlineto{\pgfqpoint{0.655159in}{0.373241in}}%
\pgfpathlineto{\pgfqpoint{0.652942in}{0.372329in}}%
\pgfpathclose%
\pgfusepath{fill}%
\end{pgfscope}%
\begin{pgfscope}%
\pgfpathrectangle{\pgfqpoint{0.100000in}{0.100000in}}{\pgfqpoint{3.420221in}{2.189500in}}%
\pgfusepath{clip}%
\pgfsetbuttcap%
\pgfsetmiterjoin%
\definecolor{currentfill}{rgb}{0.000000,0.768627,0.615686}%
\pgfsetfillcolor{currentfill}%
\pgfsetlinewidth{0.000000pt}%
\definecolor{currentstroke}{rgb}{0.000000,0.000000,0.000000}%
\pgfsetstrokecolor{currentstroke}%
\pgfsetstrokeopacity{0.000000}%
\pgfsetdash{}{0pt}%
\pgfpathmoveto{\pgfqpoint{0.667048in}{0.380157in}}%
\pgfpathlineto{\pgfqpoint{0.665856in}{0.381796in}}%
\pgfpathlineto{\pgfqpoint{0.669706in}{0.383695in}}%
\pgfpathlineto{\pgfqpoint{0.668669in}{0.386287in}}%
\pgfpathlineto{\pgfqpoint{0.666603in}{0.386281in}}%
\pgfpathlineto{\pgfqpoint{0.665332in}{0.387333in}}%
\pgfpathlineto{\pgfqpoint{0.660210in}{0.387903in}}%
\pgfpathlineto{\pgfqpoint{0.655942in}{0.387987in}}%
\pgfpathlineto{\pgfqpoint{0.654192in}{0.389390in}}%
\pgfpathlineto{\pgfqpoint{0.652468in}{0.388329in}}%
\pgfpathlineto{\pgfqpoint{0.650102in}{0.388402in}}%
\pgfpathlineto{\pgfqpoint{0.650414in}{0.390675in}}%
\pgfpathlineto{\pgfqpoint{0.645284in}{0.390796in}}%
\pgfpathlineto{\pgfqpoint{0.642568in}{0.391555in}}%
\pgfpathlineto{\pgfqpoint{0.640257in}{0.393808in}}%
\pgfpathlineto{\pgfqpoint{0.642259in}{0.395564in}}%
\pgfpathlineto{\pgfqpoint{0.646344in}{0.397702in}}%
\pgfpathlineto{\pgfqpoint{0.644319in}{0.399725in}}%
\pgfpathlineto{\pgfqpoint{0.642189in}{0.399837in}}%
\pgfpathlineto{\pgfqpoint{0.637846in}{0.396386in}}%
\pgfpathlineto{\pgfqpoint{0.634978in}{0.395834in}}%
\pgfpathlineto{\pgfqpoint{0.632451in}{0.396363in}}%
\pgfpathlineto{\pgfqpoint{0.630979in}{0.393740in}}%
\pgfpathlineto{\pgfqpoint{0.628875in}{0.393805in}}%
\pgfpathlineto{\pgfqpoint{0.624113in}{0.396878in}}%
\pgfpathlineto{\pgfqpoint{0.623922in}{0.398688in}}%
\pgfpathlineto{\pgfqpoint{0.625843in}{0.398867in}}%
\pgfpathlineto{\pgfqpoint{0.626346in}{0.401040in}}%
\pgfpathlineto{\pgfqpoint{0.625991in}{0.404081in}}%
\pgfpathlineto{\pgfqpoint{0.622175in}{0.399609in}}%
\pgfpathlineto{\pgfqpoint{0.621962in}{0.397951in}}%
\pgfpathlineto{\pgfqpoint{0.618953in}{0.398322in}}%
\pgfpathlineto{\pgfqpoint{0.616951in}{0.399992in}}%
\pgfpathlineto{\pgfqpoint{0.618218in}{0.403233in}}%
\pgfpathlineto{\pgfqpoint{0.617047in}{0.406197in}}%
\pgfpathlineto{\pgfqpoint{0.615838in}{0.405852in}}%
\pgfpathlineto{\pgfqpoint{0.615630in}{0.401462in}}%
\pgfpathlineto{\pgfqpoint{0.608873in}{0.401798in}}%
\pgfpathlineto{\pgfqpoint{0.610370in}{0.399903in}}%
\pgfpathlineto{\pgfqpoint{0.608192in}{0.399335in}}%
\pgfpathlineto{\pgfqpoint{0.607414in}{0.401253in}}%
\pgfpathlineto{\pgfqpoint{0.605429in}{0.400377in}}%
\pgfpathlineto{\pgfqpoint{0.602321in}{0.402391in}}%
\pgfpathlineto{\pgfqpoint{0.598458in}{0.406022in}}%
\pgfpathlineto{\pgfqpoint{0.593970in}{0.407227in}}%
\pgfpathlineto{\pgfqpoint{0.590062in}{0.406326in}}%
\pgfpathlineto{\pgfqpoint{0.587527in}{0.407726in}}%
\pgfpathlineto{\pgfqpoint{0.586240in}{0.412076in}}%
\pgfpathlineto{\pgfqpoint{0.587361in}{0.414434in}}%
\pgfpathlineto{\pgfqpoint{0.592561in}{0.413892in}}%
\pgfpathlineto{\pgfqpoint{0.598896in}{0.416277in}}%
\pgfpathlineto{\pgfqpoint{0.600191in}{0.414078in}}%
\pgfpathlineto{\pgfqpoint{0.602400in}{0.412968in}}%
\pgfpathlineto{\pgfqpoint{0.603882in}{0.413362in}}%
\pgfpathlineto{\pgfqpoint{0.608876in}{0.411552in}}%
\pgfpathlineto{\pgfqpoint{0.611831in}{0.408903in}}%
\pgfpathlineto{\pgfqpoint{0.610795in}{0.405079in}}%
\pgfpathlineto{\pgfqpoint{0.612614in}{0.404509in}}%
\pgfpathlineto{\pgfqpoint{0.615265in}{0.407979in}}%
\pgfpathlineto{\pgfqpoint{0.619368in}{0.407081in}}%
\pgfpathlineto{\pgfqpoint{0.620582in}{0.404720in}}%
\pgfpathlineto{\pgfqpoint{0.622850in}{0.404887in}}%
\pgfpathlineto{\pgfqpoint{0.638044in}{0.407782in}}%
\pgfpathlineto{\pgfqpoint{0.643750in}{0.407287in}}%
\pgfpathlineto{\pgfqpoint{0.646365in}{0.407460in}}%
\pgfpathlineto{\pgfqpoint{0.652489in}{0.404682in}}%
\pgfpathlineto{\pgfqpoint{0.656136in}{0.401157in}}%
\pgfpathlineto{\pgfqpoint{0.654948in}{0.399692in}}%
\pgfpathlineto{\pgfqpoint{0.654909in}{0.395438in}}%
\pgfpathlineto{\pgfqpoint{0.657049in}{0.396008in}}%
\pgfpathlineto{\pgfqpoint{0.660314in}{0.394138in}}%
\pgfpathlineto{\pgfqpoint{0.659467in}{0.392027in}}%
\pgfpathlineto{\pgfqpoint{0.660845in}{0.390409in}}%
\pgfpathlineto{\pgfqpoint{0.662624in}{0.391604in}}%
\pgfpathlineto{\pgfqpoint{0.661912in}{0.394268in}}%
\pgfpathlineto{\pgfqpoint{0.660413in}{0.395770in}}%
\pgfpathlineto{\pgfqpoint{0.661710in}{0.397055in}}%
\pgfpathlineto{\pgfqpoint{0.665429in}{0.398639in}}%
\pgfpathlineto{\pgfqpoint{0.667862in}{0.400268in}}%
\pgfpathlineto{\pgfqpoint{0.673554in}{0.401256in}}%
\pgfpathlineto{\pgfqpoint{0.676403in}{0.400715in}}%
\pgfpathlineto{\pgfqpoint{0.679331in}{0.401077in}}%
\pgfpathlineto{\pgfqpoint{0.682200in}{0.400621in}}%
\pgfpathlineto{\pgfqpoint{0.689463in}{0.398163in}}%
\pgfpathlineto{\pgfqpoint{0.689150in}{0.399328in}}%
\pgfpathlineto{\pgfqpoint{0.693123in}{0.398904in}}%
\pgfpathlineto{\pgfqpoint{0.694924in}{0.397371in}}%
\pgfpathlineto{\pgfqpoint{0.692536in}{0.397682in}}%
\pgfpathlineto{\pgfqpoint{0.691064in}{0.395949in}}%
\pgfpathlineto{\pgfqpoint{0.688332in}{0.398361in}}%
\pgfpathlineto{\pgfqpoint{0.686747in}{0.396507in}}%
\pgfpathlineto{\pgfqpoint{0.683188in}{0.399559in}}%
\pgfpathlineto{\pgfqpoint{0.681667in}{0.397810in}}%
\pgfpathlineto{\pgfqpoint{0.678194in}{0.400721in}}%
\pgfpathlineto{\pgfqpoint{0.675171in}{0.397330in}}%
\pgfpathlineto{\pgfqpoint{0.676211in}{0.396397in}}%
\pgfpathlineto{\pgfqpoint{0.670323in}{0.389489in}}%
\pgfpathlineto{\pgfqpoint{0.667965in}{0.388203in}}%
\pgfpathlineto{\pgfqpoint{0.671407in}{0.385231in}}%
\pgfpathclose%
\pgfusepath{fill}%
\end{pgfscope}%
\begin{pgfscope}%
\pgfpathrectangle{\pgfqpoint{0.100000in}{0.100000in}}{\pgfqpoint{3.420221in}{2.189500in}}%
\pgfusepath{clip}%
\pgfsetbuttcap%
\pgfsetmiterjoin%
\definecolor{currentfill}{rgb}{0.000000,0.258824,0.870588}%
\pgfsetfillcolor{currentfill}%
\pgfsetlinewidth{0.000000pt}%
\definecolor{currentstroke}{rgb}{0.000000,0.000000,0.000000}%
\pgfsetstrokecolor{currentstroke}%
\pgfsetstrokeopacity{0.000000}%
\pgfsetdash{}{0pt}%
\pgfpathmoveto{\pgfqpoint{1.793212in}{1.826664in}}%
\pgfpathlineto{\pgfqpoint{1.791943in}{1.800793in}}%
\pgfpathlineto{\pgfqpoint{1.754598in}{1.802660in}}%
\pgfpathlineto{\pgfqpoint{1.713684in}{1.805164in}}%
\pgfpathlineto{\pgfqpoint{1.708436in}{1.814446in}}%
\pgfpathlineto{\pgfqpoint{1.708393in}{1.822143in}}%
\pgfpathlineto{\pgfqpoint{1.711993in}{1.827807in}}%
\pgfpathlineto{\pgfqpoint{1.712889in}{1.836978in}}%
\pgfpathlineto{\pgfqpoint{1.711159in}{1.841959in}}%
\pgfpathlineto{\pgfqpoint{1.714019in}{1.849799in}}%
\pgfpathlineto{\pgfqpoint{1.713074in}{1.854420in}}%
\pgfpathlineto{\pgfqpoint{1.708404in}{1.857771in}}%
\pgfpathlineto{\pgfqpoint{1.738489in}{1.855910in}}%
\pgfpathlineto{\pgfqpoint{1.771221in}{1.854064in}}%
\pgfpathlineto{\pgfqpoint{1.792641in}{1.852968in}}%
\pgfpathlineto{\pgfqpoint{1.791430in}{1.826755in}}%
\pgfpathclose%
\pgfusepath{fill}%
\end{pgfscope}%
\begin{pgfscope}%
\pgfpathrectangle{\pgfqpoint{0.100000in}{0.100000in}}{\pgfqpoint{3.420221in}{2.189500in}}%
\pgfusepath{clip}%
\pgfsetbuttcap%
\pgfsetmiterjoin%
\definecolor{currentfill}{rgb}{0.000000,0.219608,0.890196}%
\pgfsetfillcolor{currentfill}%
\pgfsetlinewidth{0.000000pt}%
\definecolor{currentstroke}{rgb}{0.000000,0.000000,0.000000}%
\pgfsetstrokecolor{currentstroke}%
\pgfsetstrokeopacity{0.000000}%
\pgfsetdash{}{0pt}%
\pgfpathmoveto{\pgfqpoint{1.702774in}{1.133481in}}%
\pgfpathlineto{\pgfqpoint{1.700635in}{1.095898in}}%
\pgfpathlineto{\pgfqpoint{1.643444in}{1.099348in}}%
\pgfpathlineto{\pgfqpoint{1.646404in}{1.136614in}}%
\pgfpathlineto{\pgfqpoint{1.639164in}{1.137076in}}%
\pgfpathlineto{\pgfqpoint{1.641044in}{1.166240in}}%
\pgfpathlineto{\pgfqpoint{1.665720in}{1.164627in}}%
\pgfpathlineto{\pgfqpoint{1.666112in}{1.171164in}}%
\pgfpathlineto{\pgfqpoint{1.698514in}{1.169238in}}%
\pgfpathlineto{\pgfqpoint{1.699076in}{1.162603in}}%
\pgfpathlineto{\pgfqpoint{1.697580in}{1.133815in}}%
\pgfpathclose%
\pgfusepath{fill}%
\end{pgfscope}%
\begin{pgfscope}%
\pgfpathrectangle{\pgfqpoint{0.100000in}{0.100000in}}{\pgfqpoint{3.420221in}{2.189500in}}%
\pgfusepath{clip}%
\pgfsetbuttcap%
\pgfsetmiterjoin%
\definecolor{currentfill}{rgb}{0.000000,0.360784,0.819608}%
\pgfsetfillcolor{currentfill}%
\pgfsetlinewidth{0.000000pt}%
\definecolor{currentstroke}{rgb}{0.000000,0.000000,0.000000}%
\pgfsetstrokecolor{currentstroke}%
\pgfsetstrokeopacity{0.000000}%
\pgfsetdash{}{0pt}%
\pgfpathmoveto{\pgfqpoint{1.980877in}{1.594089in}}%
\pgfpathlineto{\pgfqpoint{1.954946in}{1.594456in}}%
\pgfpathlineto{\pgfqpoint{1.955173in}{1.612662in}}%
\pgfpathlineto{\pgfqpoint{1.944719in}{1.612797in}}%
\pgfpathlineto{\pgfqpoint{1.945068in}{1.639015in}}%
\pgfpathlineto{\pgfqpoint{1.944742in}{1.665145in}}%
\pgfpathlineto{\pgfqpoint{1.970103in}{1.664851in}}%
\pgfpathlineto{\pgfqpoint{1.996243in}{1.664701in}}%
\pgfpathlineto{\pgfqpoint{1.996698in}{1.658161in}}%
\pgfpathlineto{\pgfqpoint{2.009676in}{1.658152in}}%
\pgfpathlineto{\pgfqpoint{2.010002in}{1.638606in}}%
\pgfpathlineto{\pgfqpoint{2.010044in}{1.612494in}}%
\pgfpathlineto{\pgfqpoint{2.006748in}{1.612479in}}%
\pgfpathlineto{\pgfqpoint{1.980970in}{1.612505in}}%
\pgfpathclose%
\pgfusepath{fill}%
\end{pgfscope}%
\begin{pgfscope}%
\pgfpathrectangle{\pgfqpoint{0.100000in}{0.100000in}}{\pgfqpoint{3.420221in}{2.189500in}}%
\pgfusepath{clip}%
\pgfsetbuttcap%
\pgfsetmiterjoin%
\definecolor{currentfill}{rgb}{0.000000,0.227451,0.886275}%
\pgfsetfillcolor{currentfill}%
\pgfsetlinewidth{0.000000pt}%
\definecolor{currentstroke}{rgb}{0.000000,0.000000,0.000000}%
\pgfsetstrokecolor{currentstroke}%
\pgfsetstrokeopacity{0.000000}%
\pgfsetdash{}{0pt}%
\pgfpathmoveto{\pgfqpoint{2.032210in}{1.392764in}}%
\pgfpathlineto{\pgfqpoint{2.023056in}{1.392746in}}%
\pgfpathlineto{\pgfqpoint{1.958536in}{1.393937in}}%
\pgfpathlineto{\pgfqpoint{1.956887in}{1.399735in}}%
\pgfpathlineto{\pgfqpoint{1.951644in}{1.405323in}}%
\pgfpathlineto{\pgfqpoint{1.954775in}{1.408853in}}%
\pgfpathlineto{\pgfqpoint{1.956026in}{1.417680in}}%
\pgfpathlineto{\pgfqpoint{1.980301in}{1.417509in}}%
\pgfpathlineto{\pgfqpoint{1.980478in}{1.436807in}}%
\pgfpathlineto{\pgfqpoint{1.993314in}{1.436796in}}%
\pgfpathlineto{\pgfqpoint{2.019009in}{1.436722in}}%
\pgfpathlineto{\pgfqpoint{2.031989in}{1.436739in}}%
\pgfpathclose%
\pgfusepath{fill}%
\end{pgfscope}%
\begin{pgfscope}%
\pgfpathrectangle{\pgfqpoint{0.100000in}{0.100000in}}{\pgfqpoint{3.420221in}{2.189500in}}%
\pgfusepath{clip}%
\pgfsetbuttcap%
\pgfsetmiterjoin%
\definecolor{currentfill}{rgb}{0.000000,0.419608,0.790196}%
\pgfsetfillcolor{currentfill}%
\pgfsetlinewidth{0.000000pt}%
\definecolor{currentstroke}{rgb}{0.000000,0.000000,0.000000}%
\pgfsetstrokecolor{currentstroke}%
\pgfsetstrokeopacity{0.000000}%
\pgfsetdash{}{0pt}%
\pgfpathmoveto{\pgfqpoint{1.845429in}{0.340792in}}%
\pgfpathlineto{\pgfqpoint{1.843204in}{0.340846in}}%
\pgfpathlineto{\pgfqpoint{1.838959in}{0.358180in}}%
\pgfpathlineto{\pgfqpoint{1.839494in}{0.383570in}}%
\pgfpathlineto{\pgfqpoint{1.842133in}{0.400037in}}%
\pgfpathlineto{\pgfqpoint{1.859732in}{0.433026in}}%
\pgfpathlineto{\pgfqpoint{1.864256in}{0.436567in}}%
\pgfpathlineto{\pgfqpoint{1.869209in}{0.448032in}}%
\pgfpathlineto{\pgfqpoint{1.876122in}{0.457277in}}%
\pgfpathlineto{\pgfqpoint{1.879080in}{0.455758in}}%
\pgfpathlineto{\pgfqpoint{1.877454in}{0.451812in}}%
\pgfpathlineto{\pgfqpoint{1.865569in}{0.438273in}}%
\pgfpathlineto{\pgfqpoint{1.855642in}{0.422931in}}%
\pgfpathlineto{\pgfqpoint{1.849188in}{0.410359in}}%
\pgfpathlineto{\pgfqpoint{1.844176in}{0.398048in}}%
\pgfpathlineto{\pgfqpoint{1.841226in}{0.386721in}}%
\pgfpathlineto{\pgfqpoint{1.840004in}{0.369149in}}%
\pgfpathlineto{\pgfqpoint{1.842132in}{0.353155in}}%
\pgfpathclose%
\pgfusepath{fill}%
\end{pgfscope}%
\begin{pgfscope}%
\pgfpathrectangle{\pgfqpoint{0.100000in}{0.100000in}}{\pgfqpoint{3.420221in}{2.189500in}}%
\pgfusepath{clip}%
\pgfsetbuttcap%
\pgfsetmiterjoin%
\definecolor{currentfill}{rgb}{0.000000,0.419608,0.790196}%
\pgfsetfillcolor{currentfill}%
\pgfsetlinewidth{0.000000pt}%
\definecolor{currentstroke}{rgb}{0.000000,0.000000,0.000000}%
\pgfsetstrokecolor{currentstroke}%
\pgfsetstrokeopacity{0.000000}%
\pgfsetdash{}{0pt}%
\pgfpathmoveto{\pgfqpoint{1.834991in}{0.341043in}}%
\pgfpathlineto{\pgfqpoint{1.806051in}{0.341740in}}%
\pgfpathlineto{\pgfqpoint{1.797630in}{0.343377in}}%
\pgfpathlineto{\pgfqpoint{1.798033in}{0.356020in}}%
\pgfpathlineto{\pgfqpoint{1.768173in}{0.357274in}}%
\pgfpathlineto{\pgfqpoint{1.769279in}{0.378049in}}%
\pgfpathlineto{\pgfqpoint{1.765964in}{0.378131in}}%
\pgfpathlineto{\pgfqpoint{1.764696in}{0.391978in}}%
\pgfpathlineto{\pgfqpoint{1.761013in}{0.401617in}}%
\pgfpathlineto{\pgfqpoint{1.744331in}{0.401905in}}%
\pgfpathlineto{\pgfqpoint{1.746239in}{0.455579in}}%
\pgfpathlineto{\pgfqpoint{1.808119in}{0.453257in}}%
\pgfpathlineto{\pgfqpoint{1.806850in}{0.457671in}}%
\pgfpathlineto{\pgfqpoint{1.812796in}{0.462255in}}%
\pgfpathlineto{\pgfqpoint{1.829500in}{0.458168in}}%
\pgfpathlineto{\pgfqpoint{1.842914in}{0.477536in}}%
\pgfpathlineto{\pgfqpoint{1.857623in}{0.489736in}}%
\pgfpathlineto{\pgfqpoint{1.865089in}{0.489288in}}%
\pgfpathlineto{\pgfqpoint{1.869224in}{0.485387in}}%
\pgfpathlineto{\pgfqpoint{1.876238in}{0.485555in}}%
\pgfpathlineto{\pgfqpoint{1.882920in}{0.477565in}}%
\pgfpathlineto{\pgfqpoint{1.881419in}{0.468130in}}%
\pgfpathlineto{\pgfqpoint{1.880608in}{0.463842in}}%
\pgfpathlineto{\pgfqpoint{1.872223in}{0.456656in}}%
\pgfpathlineto{\pgfqpoint{1.867828in}{0.456585in}}%
\pgfpathlineto{\pgfqpoint{1.864206in}{0.449173in}}%
\pgfpathlineto{\pgfqpoint{1.854446in}{0.434174in}}%
\pgfpathlineto{\pgfqpoint{1.848855in}{0.438716in}}%
\pgfpathlineto{\pgfqpoint{1.842274in}{0.435863in}}%
\pgfpathlineto{\pgfqpoint{1.842533in}{0.427643in}}%
\pgfpathlineto{\pgfqpoint{1.849757in}{0.424627in}}%
\pgfpathlineto{\pgfqpoint{1.844979in}{0.415055in}}%
\pgfpathlineto{\pgfqpoint{1.838954in}{0.396985in}}%
\pgfpathlineto{\pgfqpoint{1.835764in}{0.380724in}}%
\pgfpathlineto{\pgfqpoint{1.827985in}{0.370920in}}%
\pgfpathlineto{\pgfqpoint{1.827506in}{0.359427in}}%
\pgfpathlineto{\pgfqpoint{1.833093in}{0.353268in}}%
\pgfpathclose%
\pgfusepath{fill}%
\end{pgfscope}%
\begin{pgfscope}%
\pgfpathrectangle{\pgfqpoint{0.100000in}{0.100000in}}{\pgfqpoint{3.420221in}{2.189500in}}%
\pgfusepath{clip}%
\pgfsetbuttcap%
\pgfsetmiterjoin%
\definecolor{currentfill}{rgb}{0.000000,0.207843,0.896078}%
\pgfsetfillcolor{currentfill}%
\pgfsetlinewidth{0.000000pt}%
\definecolor{currentstroke}{rgb}{0.000000,0.000000,0.000000}%
\pgfsetstrokecolor{currentstroke}%
\pgfsetstrokeopacity{0.000000}%
\pgfsetdash{}{0pt}%
\pgfpathmoveto{\pgfqpoint{1.923322in}{1.639349in}}%
\pgfpathlineto{\pgfqpoint{1.899719in}{1.639679in}}%
\pgfpathlineto{\pgfqpoint{1.900226in}{1.665818in}}%
\pgfpathlineto{\pgfqpoint{1.848459in}{1.667117in}}%
\pgfpathlineto{\pgfqpoint{1.848235in}{1.667121in}}%
\pgfpathlineto{\pgfqpoint{1.849082in}{1.693376in}}%
\pgfpathlineto{\pgfqpoint{1.887960in}{1.692218in}}%
\pgfpathlineto{\pgfqpoint{1.924224in}{1.691542in}}%
\pgfpathclose%
\pgfusepath{fill}%
\end{pgfscope}%
\begin{pgfscope}%
\pgfpathrectangle{\pgfqpoint{0.100000in}{0.100000in}}{\pgfqpoint{3.420221in}{2.189500in}}%
\pgfusepath{clip}%
\pgfsetbuttcap%
\pgfsetmiterjoin%
\definecolor{currentfill}{rgb}{0.000000,0.776471,0.611765}%
\pgfsetfillcolor{currentfill}%
\pgfsetlinewidth{0.000000pt}%
\definecolor{currentstroke}{rgb}{0.000000,0.000000,0.000000}%
\pgfsetstrokecolor{currentstroke}%
\pgfsetstrokeopacity{0.000000}%
\pgfsetdash{}{0pt}%
\pgfpathmoveto{\pgfqpoint{2.197889in}{1.018611in}}%
\pgfpathlineto{\pgfqpoint{2.197981in}{1.012716in}}%
\pgfpathlineto{\pgfqpoint{2.194806in}{1.005296in}}%
\pgfpathlineto{\pgfqpoint{2.178916in}{1.005102in}}%
\pgfpathlineto{\pgfqpoint{2.156264in}{1.004853in}}%
\pgfpathlineto{\pgfqpoint{2.156196in}{1.011393in}}%
\pgfpathlineto{\pgfqpoint{2.133504in}{1.011369in}}%
\pgfpathlineto{\pgfqpoint{2.135623in}{1.017880in}}%
\pgfpathlineto{\pgfqpoint{2.135532in}{1.030896in}}%
\pgfpathlineto{\pgfqpoint{2.146414in}{1.030866in}}%
\pgfpathlineto{\pgfqpoint{2.146357in}{1.036304in}}%
\pgfpathlineto{\pgfqpoint{2.159328in}{1.036490in}}%
\pgfpathlineto{\pgfqpoint{2.159074in}{1.050540in}}%
\pgfpathlineto{\pgfqpoint{2.165578in}{1.050591in}}%
\pgfpathlineto{\pgfqpoint{2.165522in}{1.057123in}}%
\pgfpathlineto{\pgfqpoint{2.172031in}{1.057186in}}%
\pgfpathlineto{\pgfqpoint{2.171980in}{1.062716in}}%
\pgfpathlineto{\pgfqpoint{2.180602in}{1.055599in}}%
\pgfpathlineto{\pgfqpoint{2.178890in}{1.050162in}}%
\pgfpathlineto{\pgfqpoint{2.187155in}{1.048123in}}%
\pgfpathlineto{\pgfqpoint{2.193644in}{1.041681in}}%
\pgfpathlineto{\pgfqpoint{2.192475in}{1.039127in}}%
\pgfpathlineto{\pgfqpoint{2.197670in}{1.031942in}}%
\pgfpathclose%
\pgfusepath{fill}%
\end{pgfscope}%
\begin{pgfscope}%
\pgfpathrectangle{\pgfqpoint{0.100000in}{0.100000in}}{\pgfqpoint{3.420221in}{2.189500in}}%
\pgfusepath{clip}%
\pgfsetbuttcap%
\pgfsetmiterjoin%
\definecolor{currentfill}{rgb}{0.000000,0.419608,0.790196}%
\pgfsetfillcolor{currentfill}%
\pgfsetlinewidth{0.000000pt}%
\definecolor{currentstroke}{rgb}{0.000000,0.000000,0.000000}%
\pgfsetstrokecolor{currentstroke}%
\pgfsetstrokeopacity{0.000000}%
\pgfsetdash{}{0pt}%
\pgfpathmoveto{\pgfqpoint{2.809775in}{0.954853in}}%
\pgfpathlineto{\pgfqpoint{2.800993in}{0.951131in}}%
\pgfpathlineto{\pgfqpoint{2.804628in}{0.944072in}}%
\pgfpathlineto{\pgfqpoint{2.801484in}{0.935623in}}%
\pgfpathlineto{\pgfqpoint{2.807851in}{0.929674in}}%
\pgfpathlineto{\pgfqpoint{2.805960in}{0.926095in}}%
\pgfpathlineto{\pgfqpoint{2.799666in}{0.932480in}}%
\pgfpathlineto{\pgfqpoint{2.793412in}{0.940284in}}%
\pgfpathlineto{\pgfqpoint{2.775451in}{0.949570in}}%
\pgfpathlineto{\pgfqpoint{2.772852in}{0.954747in}}%
\pgfpathlineto{\pgfqpoint{2.764026in}{0.962785in}}%
\pgfpathlineto{\pgfqpoint{2.761447in}{0.970060in}}%
\pgfpathlineto{\pgfqpoint{2.749263in}{0.986234in}}%
\pgfpathlineto{\pgfqpoint{2.743158in}{0.984399in}}%
\pgfpathlineto{\pgfqpoint{2.738862in}{0.986082in}}%
\pgfpathlineto{\pgfqpoint{2.731499in}{0.993122in}}%
\pgfpathlineto{\pgfqpoint{2.727782in}{0.992925in}}%
\pgfpathlineto{\pgfqpoint{2.720569in}{0.997703in}}%
\pgfpathlineto{\pgfqpoint{2.719427in}{0.999752in}}%
\pgfpathlineto{\pgfqpoint{2.720465in}{1.005440in}}%
\pgfpathlineto{\pgfqpoint{2.725427in}{1.012933in}}%
\pgfpathlineto{\pgfqpoint{2.731380in}{1.018465in}}%
\pgfpathlineto{\pgfqpoint{2.731574in}{1.022907in}}%
\pgfpathlineto{\pgfqpoint{2.737398in}{1.025703in}}%
\pgfpathlineto{\pgfqpoint{2.756928in}{1.035146in}}%
\pgfpathlineto{\pgfqpoint{2.775613in}{1.043290in}}%
\pgfpathlineto{\pgfqpoint{2.783907in}{1.044854in}}%
\pgfpathlineto{\pgfqpoint{2.786919in}{1.019091in}}%
\pgfpathlineto{\pgfqpoint{2.802514in}{1.007006in}}%
\pgfpathlineto{\pgfqpoint{2.808840in}{1.003455in}}%
\pgfpathlineto{\pgfqpoint{2.825974in}{1.000665in}}%
\pgfpathlineto{\pgfqpoint{2.819299in}{0.986906in}}%
\pgfpathlineto{\pgfqpoint{2.813065in}{0.981065in}}%
\pgfpathlineto{\pgfqpoint{2.812007in}{0.974776in}}%
\pgfpathlineto{\pgfqpoint{2.814593in}{0.969607in}}%
\pgfpathclose%
\pgfusepath{fill}%
\end{pgfscope}%
\begin{pgfscope}%
\pgfpathrectangle{\pgfqpoint{0.100000in}{0.100000in}}{\pgfqpoint{3.420221in}{2.189500in}}%
\pgfusepath{clip}%
\pgfsetbuttcap%
\pgfsetmiterjoin%
\definecolor{currentfill}{rgb}{0.000000,0.321569,0.839216}%
\pgfsetfillcolor{currentfill}%
\pgfsetlinewidth{0.000000pt}%
\definecolor{currentstroke}{rgb}{0.000000,0.000000,0.000000}%
\pgfsetstrokecolor{currentstroke}%
\pgfsetstrokeopacity{0.000000}%
\pgfsetdash{}{0pt}%
\pgfpathmoveto{\pgfqpoint{2.549465in}{1.213876in}}%
\pgfpathlineto{\pgfqpoint{2.544423in}{1.216866in}}%
\pgfpathlineto{\pgfqpoint{2.541238in}{1.223259in}}%
\pgfpathlineto{\pgfqpoint{2.535607in}{1.227157in}}%
\pgfpathlineto{\pgfqpoint{2.522445in}{1.226093in}}%
\pgfpathlineto{\pgfqpoint{2.516902in}{1.228778in}}%
\pgfpathlineto{\pgfqpoint{2.515758in}{1.234671in}}%
\pgfpathlineto{\pgfqpoint{2.510924in}{1.238837in}}%
\pgfpathlineto{\pgfqpoint{2.508081in}{1.231929in}}%
\pgfpathlineto{\pgfqpoint{2.504621in}{1.232676in}}%
\pgfpathlineto{\pgfqpoint{2.504122in}{1.239158in}}%
\pgfpathlineto{\pgfqpoint{2.497597in}{1.238601in}}%
\pgfpathlineto{\pgfqpoint{2.497278in}{1.242995in}}%
\pgfpathlineto{\pgfqpoint{2.490894in}{1.242161in}}%
\pgfpathlineto{\pgfqpoint{2.489880in}{1.251976in}}%
\pgfpathlineto{\pgfqpoint{2.511665in}{1.253968in}}%
\pgfpathlineto{\pgfqpoint{2.509170in}{1.279248in}}%
\pgfpathlineto{\pgfqpoint{2.510973in}{1.281729in}}%
\pgfpathlineto{\pgfqpoint{2.529249in}{1.283551in}}%
\pgfpathlineto{\pgfqpoint{2.533680in}{1.281751in}}%
\pgfpathlineto{\pgfqpoint{2.535037in}{1.269916in}}%
\pgfpathlineto{\pgfqpoint{2.540179in}{1.272652in}}%
\pgfpathlineto{\pgfqpoint{2.553189in}{1.274098in}}%
\pgfpathlineto{\pgfqpoint{2.561605in}{1.273482in}}%
\pgfpathlineto{\pgfqpoint{2.561806in}{1.268799in}}%
\pgfpathlineto{\pgfqpoint{2.571009in}{1.267726in}}%
\pgfpathlineto{\pgfqpoint{2.576744in}{1.275058in}}%
\pgfpathlineto{\pgfqpoint{2.582127in}{1.276486in}}%
\pgfpathlineto{\pgfqpoint{2.586075in}{1.274301in}}%
\pgfpathlineto{\pgfqpoint{2.587934in}{1.268910in}}%
\pgfpathlineto{\pgfqpoint{2.593748in}{1.261034in}}%
\pgfpathlineto{\pgfqpoint{2.589168in}{1.253000in}}%
\pgfpathlineto{\pgfqpoint{2.589395in}{1.244715in}}%
\pgfpathlineto{\pgfqpoint{2.587662in}{1.237772in}}%
\pgfpathlineto{\pgfqpoint{2.581739in}{1.229328in}}%
\pgfpathlineto{\pgfqpoint{2.568351in}{1.225486in}}%
\pgfpathlineto{\pgfqpoint{2.562718in}{1.228735in}}%
\pgfpathlineto{\pgfqpoint{2.554927in}{1.216131in}}%
\pgfpathclose%
\pgfusepath{fill}%
\end{pgfscope}%
\begin{pgfscope}%
\pgfpathrectangle{\pgfqpoint{0.100000in}{0.100000in}}{\pgfqpoint{3.420221in}{2.189500in}}%
\pgfusepath{clip}%
\pgfsetbuttcap%
\pgfsetmiterjoin%
\definecolor{currentfill}{rgb}{0.000000,0.258824,0.870588}%
\pgfsetfillcolor{currentfill}%
\pgfsetlinewidth{0.000000pt}%
\definecolor{currentstroke}{rgb}{0.000000,0.000000,0.000000}%
\pgfsetstrokecolor{currentstroke}%
\pgfsetstrokeopacity{0.000000}%
\pgfsetdash{}{0pt}%
\pgfpathmoveto{\pgfqpoint{1.818204in}{1.496781in}}%
\pgfpathlineto{\pgfqpoint{1.817723in}{1.483763in}}%
\pgfpathlineto{\pgfqpoint{1.791810in}{1.484703in}}%
\pgfpathlineto{\pgfqpoint{1.792767in}{1.510793in}}%
\pgfpathlineto{\pgfqpoint{1.766442in}{1.511909in}}%
\pgfpathlineto{\pgfqpoint{1.768544in}{1.580207in}}%
\pgfpathlineto{\pgfqpoint{1.810014in}{1.578470in}}%
\pgfpathlineto{\pgfqpoint{1.811580in}{1.574577in}}%
\pgfpathlineto{\pgfqpoint{1.820061in}{1.569379in}}%
\pgfpathlineto{\pgfqpoint{1.819375in}{1.535927in}}%
\pgfpathclose%
\pgfusepath{fill}%
\end{pgfscope}%
\begin{pgfscope}%
\pgfpathrectangle{\pgfqpoint{0.100000in}{0.100000in}}{\pgfqpoint{3.420221in}{2.189500in}}%
\pgfusepath{clip}%
\pgfsetbuttcap%
\pgfsetmiterjoin%
\definecolor{currentfill}{rgb}{0.000000,0.341176,0.829412}%
\pgfsetfillcolor{currentfill}%
\pgfsetlinewidth{0.000000pt}%
\definecolor{currentstroke}{rgb}{0.000000,0.000000,0.000000}%
\pgfsetstrokecolor{currentstroke}%
\pgfsetstrokeopacity{0.000000}%
\pgfsetdash{}{0pt}%
\pgfpathmoveto{\pgfqpoint{2.583116in}{1.408965in}}%
\pgfpathlineto{\pgfqpoint{2.583486in}{1.405784in}}%
\pgfpathlineto{\pgfqpoint{2.559952in}{1.402931in}}%
\pgfpathlineto{\pgfqpoint{2.558849in}{1.422438in}}%
\pgfpathlineto{\pgfqpoint{2.544939in}{1.420922in}}%
\pgfpathlineto{\pgfqpoint{2.544153in}{1.427388in}}%
\pgfpathlineto{\pgfqpoint{2.533426in}{1.426220in}}%
\pgfpathlineto{\pgfqpoint{2.530437in}{1.452229in}}%
\pgfpathlineto{\pgfqpoint{2.528225in}{1.451969in}}%
\pgfpathlineto{\pgfqpoint{2.526683in}{1.465115in}}%
\pgfpathlineto{\pgfqpoint{2.527732in}{1.473996in}}%
\pgfpathlineto{\pgfqpoint{2.534384in}{1.472530in}}%
\pgfpathlineto{\pgfqpoint{2.553736in}{1.474448in}}%
\pgfpathlineto{\pgfqpoint{2.549482in}{1.511355in}}%
\pgfpathlineto{\pgfqpoint{2.570134in}{1.513654in}}%
\pgfpathlineto{\pgfqpoint{2.571733in}{1.509011in}}%
\pgfpathlineto{\pgfqpoint{2.577821in}{1.456363in}}%
\pgfpathclose%
\pgfusepath{fill}%
\end{pgfscope}%
\begin{pgfscope}%
\pgfpathrectangle{\pgfqpoint{0.100000in}{0.100000in}}{\pgfqpoint{3.420221in}{2.189500in}}%
\pgfusepath{clip}%
\pgfsetbuttcap%
\pgfsetmiterjoin%
\definecolor{currentfill}{rgb}{0.000000,0.980392,0.509804}%
\pgfsetfillcolor{currentfill}%
\pgfsetlinewidth{0.000000pt}%
\definecolor{currentstroke}{rgb}{0.000000,0.000000,0.000000}%
\pgfsetstrokecolor{currentstroke}%
\pgfsetstrokeopacity{0.000000}%
\pgfsetdash{}{0pt}%
\pgfpathmoveto{\pgfqpoint{2.694271in}{1.143988in}}%
\pgfpathlineto{\pgfqpoint{2.691235in}{1.148781in}}%
\pgfpathlineto{\pgfqpoint{2.691271in}{1.157522in}}%
\pgfpathlineto{\pgfqpoint{2.688809in}{1.163964in}}%
\pgfpathlineto{\pgfqpoint{2.684127in}{1.180690in}}%
\pgfpathlineto{\pgfqpoint{2.684730in}{1.187635in}}%
\pgfpathlineto{\pgfqpoint{2.682891in}{1.192361in}}%
\pgfpathlineto{\pgfqpoint{2.688434in}{1.196590in}}%
\pgfpathlineto{\pgfqpoint{2.691181in}{1.194518in}}%
\pgfpathlineto{\pgfqpoint{2.698563in}{1.199999in}}%
\pgfpathlineto{\pgfqpoint{2.707104in}{1.201613in}}%
\pgfpathlineto{\pgfqpoint{2.708027in}{1.210595in}}%
\pgfpathlineto{\pgfqpoint{2.716523in}{1.210261in}}%
\pgfpathlineto{\pgfqpoint{2.727914in}{1.206079in}}%
\pgfpathlineto{\pgfqpoint{2.729982in}{1.197519in}}%
\pgfpathlineto{\pgfqpoint{2.735619in}{1.191972in}}%
\pgfpathlineto{\pgfqpoint{2.742229in}{1.190524in}}%
\pgfpathlineto{\pgfqpoint{2.733722in}{1.183614in}}%
\pgfpathlineto{\pgfqpoint{2.733294in}{1.177717in}}%
\pgfpathlineto{\pgfqpoint{2.726405in}{1.171115in}}%
\pgfpathlineto{\pgfqpoint{2.726702in}{1.165186in}}%
\pgfpathlineto{\pgfqpoint{2.715424in}{1.160774in}}%
\pgfpathlineto{\pgfqpoint{2.712885in}{1.152292in}}%
\pgfpathclose%
\pgfusepath{fill}%
\end{pgfscope}%
\begin{pgfscope}%
\pgfpathrectangle{\pgfqpoint{0.100000in}{0.100000in}}{\pgfqpoint{3.420221in}{2.189500in}}%
\pgfusepath{clip}%
\pgfsetbuttcap%
\pgfsetmiterjoin%
\definecolor{currentfill}{rgb}{0.000000,0.427451,0.786275}%
\pgfsetfillcolor{currentfill}%
\pgfsetlinewidth{0.000000pt}%
\definecolor{currentstroke}{rgb}{0.000000,0.000000,0.000000}%
\pgfsetstrokecolor{currentstroke}%
\pgfsetstrokeopacity{0.000000}%
\pgfsetdash{}{0pt}%
\pgfpathmoveto{\pgfqpoint{2.839061in}{1.005538in}}%
\pgfpathlineto{\pgfqpoint{2.839927in}{1.000439in}}%
\pgfpathlineto{\pgfqpoint{2.832152in}{0.997272in}}%
\pgfpathlineto{\pgfqpoint{2.825974in}{1.000665in}}%
\pgfpathlineto{\pgfqpoint{2.808840in}{1.003455in}}%
\pgfpathlineto{\pgfqpoint{2.802514in}{1.007006in}}%
\pgfpathlineto{\pgfqpoint{2.786919in}{1.019091in}}%
\pgfpathlineto{\pgfqpoint{2.783907in}{1.044854in}}%
\pgfpathlineto{\pgfqpoint{2.775613in}{1.043290in}}%
\pgfpathlineto{\pgfqpoint{2.775085in}{1.050472in}}%
\pgfpathlineto{\pgfqpoint{2.779092in}{1.059106in}}%
\pgfpathlineto{\pgfqpoint{2.785092in}{1.061179in}}%
\pgfpathlineto{\pgfqpoint{2.798272in}{1.051567in}}%
\pgfpathlineto{\pgfqpoint{2.798997in}{1.046347in}}%
\pgfpathlineto{\pgfqpoint{2.835831in}{1.050089in}}%
\pgfpathlineto{\pgfqpoint{2.834554in}{1.040897in}}%
\pgfpathlineto{\pgfqpoint{2.829974in}{1.039333in}}%
\pgfpathlineto{\pgfqpoint{2.834117in}{1.025686in}}%
\pgfpathlineto{\pgfqpoint{2.835983in}{1.012914in}}%
\pgfpathclose%
\pgfusepath{fill}%
\end{pgfscope}%
\begin{pgfscope}%
\pgfpathrectangle{\pgfqpoint{0.100000in}{0.100000in}}{\pgfqpoint{3.420221in}{2.189500in}}%
\pgfusepath{clip}%
\pgfsetbuttcap%
\pgfsetmiterjoin%
\definecolor{currentfill}{rgb}{0.000000,0.564706,0.717647}%
\pgfsetfillcolor{currentfill}%
\pgfsetlinewidth{0.000000pt}%
\definecolor{currentstroke}{rgb}{0.000000,0.000000,0.000000}%
\pgfsetstrokecolor{currentstroke}%
\pgfsetstrokeopacity{0.000000}%
\pgfsetdash{}{0pt}%
\pgfpathmoveto{\pgfqpoint{0.988449in}{0.179361in}}%
\pgfpathlineto{\pgfqpoint{0.988809in}{0.182163in}}%
\pgfpathlineto{\pgfqpoint{0.990735in}{0.180260in}}%
\pgfpathclose%
\pgfusepath{fill}%
\end{pgfscope}%
\begin{pgfscope}%
\pgfpathrectangle{\pgfqpoint{0.100000in}{0.100000in}}{\pgfqpoint{3.420221in}{2.189500in}}%
\pgfusepath{clip}%
\pgfsetbuttcap%
\pgfsetmiterjoin%
\definecolor{currentfill}{rgb}{0.000000,0.564706,0.717647}%
\pgfsetfillcolor{currentfill}%
\pgfsetlinewidth{0.000000pt}%
\definecolor{currentstroke}{rgb}{0.000000,0.000000,0.000000}%
\pgfsetstrokecolor{currentstroke}%
\pgfsetstrokeopacity{0.000000}%
\pgfsetdash{}{0pt}%
\pgfpathmoveto{\pgfqpoint{0.987154in}{0.190246in}}%
\pgfpathlineto{\pgfqpoint{0.986386in}{0.192208in}}%
\pgfpathlineto{\pgfqpoint{0.985906in}{0.201327in}}%
\pgfpathlineto{\pgfqpoint{0.986523in}{0.202358in}}%
\pgfpathlineto{\pgfqpoint{0.986182in}{0.206262in}}%
\pgfpathlineto{\pgfqpoint{0.987023in}{0.207474in}}%
\pgfpathlineto{\pgfqpoint{0.985454in}{0.209041in}}%
\pgfpathlineto{\pgfqpoint{0.985288in}{0.211215in}}%
\pgfpathlineto{\pgfqpoint{0.988499in}{0.212410in}}%
\pgfpathlineto{\pgfqpoint{0.989829in}{0.214735in}}%
\pgfpathlineto{\pgfqpoint{0.988131in}{0.215702in}}%
\pgfpathlineto{\pgfqpoint{0.987928in}{0.217500in}}%
\pgfpathlineto{\pgfqpoint{0.989564in}{0.219851in}}%
\pgfpathlineto{\pgfqpoint{0.987596in}{0.224720in}}%
\pgfpathlineto{\pgfqpoint{0.992519in}{0.228638in}}%
\pgfpathlineto{\pgfqpoint{0.994587in}{0.224786in}}%
\pgfpathlineto{\pgfqpoint{0.998854in}{0.221787in}}%
\pgfpathlineto{\pgfqpoint{0.995857in}{0.220658in}}%
\pgfpathlineto{\pgfqpoint{0.996829in}{0.218165in}}%
\pgfpathlineto{\pgfqpoint{0.996576in}{0.215406in}}%
\pgfpathlineto{\pgfqpoint{0.995321in}{0.211698in}}%
\pgfpathlineto{\pgfqpoint{0.994865in}{0.208239in}}%
\pgfpathlineto{\pgfqpoint{0.993373in}{0.205082in}}%
\pgfpathlineto{\pgfqpoint{0.993828in}{0.203905in}}%
\pgfpathlineto{\pgfqpoint{0.992712in}{0.200658in}}%
\pgfpathlineto{\pgfqpoint{0.990403in}{0.199800in}}%
\pgfpathlineto{\pgfqpoint{0.991038in}{0.198215in}}%
\pgfpathlineto{\pgfqpoint{0.988724in}{0.195552in}}%
\pgfpathclose%
\pgfusepath{fill}%
\end{pgfscope}%
\begin{pgfscope}%
\pgfpathrectangle{\pgfqpoint{0.100000in}{0.100000in}}{\pgfqpoint{3.420221in}{2.189500in}}%
\pgfusepath{clip}%
\pgfsetbuttcap%
\pgfsetmiterjoin%
\definecolor{currentfill}{rgb}{0.000000,0.564706,0.717647}%
\pgfsetfillcolor{currentfill}%
\pgfsetlinewidth{0.000000pt}%
\definecolor{currentstroke}{rgb}{0.000000,0.000000,0.000000}%
\pgfsetstrokecolor{currentstroke}%
\pgfsetstrokeopacity{0.000000}%
\pgfsetdash{}{0pt}%
\pgfpathmoveto{\pgfqpoint{0.980743in}{0.217551in}}%
\pgfpathlineto{\pgfqpoint{0.983619in}{0.220143in}}%
\pgfpathlineto{\pgfqpoint{0.983720in}{0.223581in}}%
\pgfpathlineto{\pgfqpoint{0.984643in}{0.225295in}}%
\pgfpathlineto{\pgfqpoint{0.986518in}{0.224390in}}%
\pgfpathlineto{\pgfqpoint{0.985925in}{0.219839in}}%
\pgfpathlineto{\pgfqpoint{0.985157in}{0.218000in}}%
\pgfpathlineto{\pgfqpoint{0.983318in}{0.216995in}}%
\pgfpathclose%
\pgfusepath{fill}%
\end{pgfscope}%
\begin{pgfscope}%
\pgfpathrectangle{\pgfqpoint{0.100000in}{0.100000in}}{\pgfqpoint{3.420221in}{2.189500in}}%
\pgfusepath{clip}%
\pgfsetbuttcap%
\pgfsetmiterjoin%
\definecolor{currentfill}{rgb}{0.000000,0.564706,0.717647}%
\pgfsetfillcolor{currentfill}%
\pgfsetlinewidth{0.000000pt}%
\definecolor{currentstroke}{rgb}{0.000000,0.000000,0.000000}%
\pgfsetstrokecolor{currentstroke}%
\pgfsetstrokeopacity{0.000000}%
\pgfsetdash{}{0pt}%
\pgfpathmoveto{\pgfqpoint{0.983286in}{0.241682in}}%
\pgfpathlineto{\pgfqpoint{0.988455in}{0.239295in}}%
\pgfpathlineto{\pgfqpoint{0.992007in}{0.242719in}}%
\pgfpathlineto{\pgfqpoint{0.993433in}{0.239620in}}%
\pgfpathlineto{\pgfqpoint{0.995282in}{0.237264in}}%
\pgfpathlineto{\pgfqpoint{0.997680in}{0.232561in}}%
\pgfpathlineto{\pgfqpoint{1.000807in}{0.231344in}}%
\pgfpathlineto{\pgfqpoint{1.001566in}{0.230360in}}%
\pgfpathlineto{\pgfqpoint{0.999662in}{0.223160in}}%
\pgfpathlineto{\pgfqpoint{0.996986in}{0.223617in}}%
\pgfpathlineto{\pgfqpoint{0.991063in}{0.234361in}}%
\pgfpathlineto{\pgfqpoint{0.990572in}{0.232349in}}%
\pgfpathlineto{\pgfqpoint{0.991461in}{0.230450in}}%
\pgfpathlineto{\pgfqpoint{0.990056in}{0.227704in}}%
\pgfpathlineto{\pgfqpoint{0.987176in}{0.225786in}}%
\pgfpathlineto{\pgfqpoint{0.985628in}{0.227141in}}%
\pgfpathlineto{\pgfqpoint{0.985186in}{0.230566in}}%
\pgfpathlineto{\pgfqpoint{0.985978in}{0.234274in}}%
\pgfpathclose%
\pgfusepath{fill}%
\end{pgfscope}%
\begin{pgfscope}%
\pgfpathrectangle{\pgfqpoint{0.100000in}{0.100000in}}{\pgfqpoint{3.420221in}{2.189500in}}%
\pgfusepath{clip}%
\pgfsetbuttcap%
\pgfsetmiterjoin%
\definecolor{currentfill}{rgb}{0.000000,0.564706,0.717647}%
\pgfsetfillcolor{currentfill}%
\pgfsetlinewidth{0.000000pt}%
\definecolor{currentstroke}{rgb}{0.000000,0.000000,0.000000}%
\pgfsetstrokecolor{currentstroke}%
\pgfsetstrokeopacity{0.000000}%
\pgfsetdash{}{0pt}%
\pgfpathmoveto{\pgfqpoint{1.012077in}{0.180668in}}%
\pgfpathlineto{\pgfqpoint{1.010709in}{0.184700in}}%
\pgfpathlineto{\pgfqpoint{1.013759in}{0.186503in}}%
\pgfpathlineto{\pgfqpoint{1.016046in}{0.185425in}}%
\pgfpathlineto{\pgfqpoint{1.016948in}{0.183394in}}%
\pgfpathlineto{\pgfqpoint{1.014797in}{0.180419in}}%
\pgfpathclose%
\pgfusepath{fill}%
\end{pgfscope}%
\begin{pgfscope}%
\pgfpathrectangle{\pgfqpoint{0.100000in}{0.100000in}}{\pgfqpoint{3.420221in}{2.189500in}}%
\pgfusepath{clip}%
\pgfsetbuttcap%
\pgfsetmiterjoin%
\definecolor{currentfill}{rgb}{0.000000,0.564706,0.717647}%
\pgfsetfillcolor{currentfill}%
\pgfsetlinewidth{0.000000pt}%
\definecolor{currentstroke}{rgb}{0.000000,0.000000,0.000000}%
\pgfsetstrokecolor{currentstroke}%
\pgfsetstrokeopacity{0.000000}%
\pgfsetdash{}{0pt}%
\pgfpathmoveto{\pgfqpoint{1.003344in}{0.194515in}}%
\pgfpathlineto{\pgfqpoint{1.000188in}{0.193920in}}%
\pgfpathlineto{\pgfqpoint{0.999971in}{0.189846in}}%
\pgfpathlineto{\pgfqpoint{0.997793in}{0.189147in}}%
\pgfpathlineto{\pgfqpoint{0.998081in}{0.186427in}}%
\pgfpathlineto{\pgfqpoint{0.995943in}{0.184020in}}%
\pgfpathlineto{\pgfqpoint{0.995126in}{0.187196in}}%
\pgfpathlineto{\pgfqpoint{0.993322in}{0.185905in}}%
\pgfpathlineto{\pgfqpoint{0.992289in}{0.186714in}}%
\pgfpathlineto{\pgfqpoint{0.993805in}{0.189925in}}%
\pgfpathlineto{\pgfqpoint{0.996209in}{0.191827in}}%
\pgfpathlineto{\pgfqpoint{0.997870in}{0.191763in}}%
\pgfpathlineto{\pgfqpoint{0.999079in}{0.194654in}}%
\pgfpathlineto{\pgfqpoint{0.997662in}{0.195166in}}%
\pgfpathlineto{\pgfqpoint{0.996665in}{0.197157in}}%
\pgfpathlineto{\pgfqpoint{0.997312in}{0.199506in}}%
\pgfpathlineto{\pgfqpoint{0.997656in}{0.204003in}}%
\pgfpathlineto{\pgfqpoint{1.000503in}{0.205331in}}%
\pgfpathlineto{\pgfqpoint{1.004247in}{0.199099in}}%
\pgfpathlineto{\pgfqpoint{1.005112in}{0.196581in}}%
\pgfpathclose%
\pgfusepath{fill}%
\end{pgfscope}%
\begin{pgfscope}%
\pgfpathrectangle{\pgfqpoint{0.100000in}{0.100000in}}{\pgfqpoint{3.420221in}{2.189500in}}%
\pgfusepath{clip}%
\pgfsetbuttcap%
\pgfsetmiterjoin%
\definecolor{currentfill}{rgb}{0.000000,0.564706,0.717647}%
\pgfsetfillcolor{currentfill}%
\pgfsetlinewidth{0.000000pt}%
\definecolor{currentstroke}{rgb}{0.000000,0.000000,0.000000}%
\pgfsetstrokecolor{currentstroke}%
\pgfsetstrokeopacity{0.000000}%
\pgfsetdash{}{0pt}%
\pgfpathmoveto{\pgfqpoint{1.013767in}{0.207633in}}%
\pgfpathlineto{\pgfqpoint{1.017424in}{0.207836in}}%
\pgfpathlineto{\pgfqpoint{1.019070in}{0.205564in}}%
\pgfpathlineto{\pgfqpoint{1.020838in}{0.206249in}}%
\pgfpathlineto{\pgfqpoint{1.022440in}{0.208302in}}%
\pgfpathlineto{\pgfqpoint{1.027942in}{0.208695in}}%
\pgfpathlineto{\pgfqpoint{1.029340in}{0.205932in}}%
\pgfpathlineto{\pgfqpoint{1.030854in}{0.205095in}}%
\pgfpathlineto{\pgfqpoint{1.030948in}{0.202134in}}%
\pgfpathlineto{\pgfqpoint{1.028103in}{0.198857in}}%
\pgfpathlineto{\pgfqpoint{1.031347in}{0.196301in}}%
\pgfpathlineto{\pgfqpoint{1.029052in}{0.191763in}}%
\pgfpathlineto{\pgfqpoint{1.031156in}{0.189386in}}%
\pgfpathlineto{\pgfqpoint{1.030061in}{0.183414in}}%
\pgfpathlineto{\pgfqpoint{1.032037in}{0.182774in}}%
\pgfpathlineto{\pgfqpoint{1.034547in}{0.179945in}}%
\pgfpathlineto{\pgfqpoint{1.038404in}{0.174086in}}%
\pgfpathlineto{\pgfqpoint{1.038283in}{0.171601in}}%
\pgfpathlineto{\pgfqpoint{1.036326in}{0.169278in}}%
\pgfpathlineto{\pgfqpoint{1.033749in}{0.170216in}}%
\pgfpathlineto{\pgfqpoint{1.030590in}{0.169668in}}%
\pgfpathlineto{\pgfqpoint{1.030811in}{0.167866in}}%
\pgfpathlineto{\pgfqpoint{1.028725in}{0.166915in}}%
\pgfpathlineto{\pgfqpoint{1.027282in}{0.170030in}}%
\pgfpathlineto{\pgfqpoint{1.024595in}{0.169689in}}%
\pgfpathlineto{\pgfqpoint{1.022469in}{0.168543in}}%
\pgfpathlineto{\pgfqpoint{1.022191in}{0.167275in}}%
\pgfpathlineto{\pgfqpoint{1.019647in}{0.166703in}}%
\pgfpathlineto{\pgfqpoint{1.020191in}{0.164491in}}%
\pgfpathlineto{\pgfqpoint{1.018756in}{0.162209in}}%
\pgfpathlineto{\pgfqpoint{1.016941in}{0.161937in}}%
\pgfpathlineto{\pgfqpoint{1.015118in}{0.165753in}}%
\pgfpathlineto{\pgfqpoint{1.018612in}{0.166105in}}%
\pgfpathlineto{\pgfqpoint{1.019698in}{0.167804in}}%
\pgfpathlineto{\pgfqpoint{1.022091in}{0.169111in}}%
\pgfpathlineto{\pgfqpoint{1.023193in}{0.172695in}}%
\pgfpathlineto{\pgfqpoint{1.025055in}{0.173827in}}%
\pgfpathlineto{\pgfqpoint{1.023081in}{0.175795in}}%
\pgfpathlineto{\pgfqpoint{1.022390in}{0.173239in}}%
\pgfpathlineto{\pgfqpoint{1.019193in}{0.172857in}}%
\pgfpathlineto{\pgfqpoint{1.018078in}{0.169837in}}%
\pgfpathlineto{\pgfqpoint{1.015512in}{0.170676in}}%
\pgfpathlineto{\pgfqpoint{1.015253in}{0.175442in}}%
\pgfpathlineto{\pgfqpoint{1.013046in}{0.175381in}}%
\pgfpathlineto{\pgfqpoint{1.014116in}{0.178858in}}%
\pgfpathlineto{\pgfqpoint{1.019345in}{0.180893in}}%
\pgfpathlineto{\pgfqpoint{1.018872in}{0.178260in}}%
\pgfpathlineto{\pgfqpoint{1.020007in}{0.176382in}}%
\pgfpathlineto{\pgfqpoint{1.021234in}{0.183386in}}%
\pgfpathlineto{\pgfqpoint{1.022380in}{0.185395in}}%
\pgfpathlineto{\pgfqpoint{1.025662in}{0.188196in}}%
\pgfpathlineto{\pgfqpoint{1.021601in}{0.190179in}}%
\pgfpathlineto{\pgfqpoint{1.021730in}{0.191963in}}%
\pgfpathlineto{\pgfqpoint{1.020229in}{0.194444in}}%
\pgfpathlineto{\pgfqpoint{1.019515in}{0.198619in}}%
\pgfpathlineto{\pgfqpoint{1.021102in}{0.198906in}}%
\pgfpathlineto{\pgfqpoint{1.021098in}{0.201174in}}%
\pgfpathlineto{\pgfqpoint{1.018950in}{0.201800in}}%
\pgfpathlineto{\pgfqpoint{1.017365in}{0.203754in}}%
\pgfpathlineto{\pgfqpoint{1.017063in}{0.205808in}}%
\pgfpathlineto{\pgfqpoint{1.015209in}{0.205387in}}%
\pgfpathclose%
\pgfusepath{fill}%
\end{pgfscope}%
\begin{pgfscope}%
\pgfpathrectangle{\pgfqpoint{0.100000in}{0.100000in}}{\pgfqpoint{3.420221in}{2.189500in}}%
\pgfusepath{clip}%
\pgfsetbuttcap%
\pgfsetmiterjoin%
\definecolor{currentfill}{rgb}{0.000000,0.564706,0.717647}%
\pgfsetfillcolor{currentfill}%
\pgfsetlinewidth{0.000000pt}%
\definecolor{currentstroke}{rgb}{0.000000,0.000000,0.000000}%
\pgfsetstrokecolor{currentstroke}%
\pgfsetstrokeopacity{0.000000}%
\pgfsetdash{}{0pt}%
\pgfpathmoveto{\pgfqpoint{1.003771in}{0.190666in}}%
\pgfpathlineto{\pgfqpoint{1.006357in}{0.197924in}}%
\pgfpathlineto{\pgfqpoint{1.005949in}{0.202390in}}%
\pgfpathlineto{\pgfqpoint{1.006513in}{0.203145in}}%
\pgfpathlineto{\pgfqpoint{1.005608in}{0.207393in}}%
\pgfpathlineto{\pgfqpoint{1.007628in}{0.208042in}}%
\pgfpathlineto{\pgfqpoint{1.013871in}{0.202600in}}%
\pgfpathlineto{\pgfqpoint{1.017454in}{0.200122in}}%
\pgfpathlineto{\pgfqpoint{1.017943in}{0.197590in}}%
\pgfpathlineto{\pgfqpoint{1.016912in}{0.195558in}}%
\pgfpathlineto{\pgfqpoint{1.018860in}{0.193523in}}%
\pgfpathlineto{\pgfqpoint{1.020362in}{0.187364in}}%
\pgfpathlineto{\pgfqpoint{1.016109in}{0.186741in}}%
\pgfpathlineto{\pgfqpoint{1.014044in}{0.187870in}}%
\pgfpathlineto{\pgfqpoint{1.016086in}{0.191544in}}%
\pgfpathlineto{\pgfqpoint{1.013860in}{0.191246in}}%
\pgfpathlineto{\pgfqpoint{1.012377in}{0.189307in}}%
\pgfpathlineto{\pgfqpoint{1.009557in}{0.188264in}}%
\pgfpathlineto{\pgfqpoint{1.007259in}{0.190553in}}%
\pgfpathlineto{\pgfqpoint{1.006060in}{0.189736in}}%
\pgfpathclose%
\pgfusepath{fill}%
\end{pgfscope}%
\begin{pgfscope}%
\pgfpathrectangle{\pgfqpoint{0.100000in}{0.100000in}}{\pgfqpoint{3.420221in}{2.189500in}}%
\pgfusepath{clip}%
\pgfsetbuttcap%
\pgfsetmiterjoin%
\definecolor{currentfill}{rgb}{0.000000,0.588235,0.705882}%
\pgfsetfillcolor{currentfill}%
\pgfsetlinewidth{0.000000pt}%
\definecolor{currentstroke}{rgb}{0.000000,0.000000,0.000000}%
\pgfsetstrokecolor{currentstroke}%
\pgfsetstrokeopacity{0.000000}%
\pgfsetdash{}{0pt}%
\pgfpathmoveto{\pgfqpoint{1.102559in}{1.867035in}}%
\pgfpathlineto{\pgfqpoint{1.085771in}{1.870248in}}%
\pgfpathlineto{\pgfqpoint{1.083024in}{1.865892in}}%
\pgfpathlineto{\pgfqpoint{1.075098in}{1.862908in}}%
\pgfpathlineto{\pgfqpoint{1.073154in}{1.872670in}}%
\pgfpathlineto{\pgfqpoint{1.060925in}{1.880603in}}%
\pgfpathlineto{\pgfqpoint{1.057529in}{1.886121in}}%
\pgfpathlineto{\pgfqpoint{1.050918in}{1.886535in}}%
\pgfpathlineto{\pgfqpoint{1.040548in}{1.880815in}}%
\pgfpathlineto{\pgfqpoint{1.036983in}{1.894726in}}%
\pgfpathlineto{\pgfqpoint{1.030613in}{1.898083in}}%
\pgfpathlineto{\pgfqpoint{1.029681in}{1.903705in}}%
\pgfpathlineto{\pgfqpoint{1.025117in}{1.907610in}}%
\pgfpathlineto{\pgfqpoint{1.025971in}{1.913483in}}%
\pgfpathlineto{\pgfqpoint{1.030107in}{1.921690in}}%
\pgfpathlineto{\pgfqpoint{1.029878in}{1.929035in}}%
\pgfpathlineto{\pgfqpoint{1.026949in}{1.933890in}}%
\pgfpathlineto{\pgfqpoint{1.027437in}{1.940734in}}%
\pgfpathlineto{\pgfqpoint{1.033914in}{1.950893in}}%
\pgfpathlineto{\pgfqpoint{1.040319in}{1.949562in}}%
\pgfpathlineto{\pgfqpoint{1.041183in}{1.953830in}}%
\pgfpathlineto{\pgfqpoint{1.051162in}{1.953993in}}%
\pgfpathlineto{\pgfqpoint{1.055714in}{1.959774in}}%
\pgfpathlineto{\pgfqpoint{1.061456in}{1.958613in}}%
\pgfpathlineto{\pgfqpoint{1.066718in}{1.984360in}}%
\pgfpathlineto{\pgfqpoint{1.058483in}{1.986012in}}%
\pgfpathlineto{\pgfqpoint{1.064900in}{2.017251in}}%
\pgfpathlineto{\pgfqpoint{1.080946in}{2.013694in}}%
\pgfpathlineto{\pgfqpoint{1.083608in}{2.004787in}}%
\pgfpathlineto{\pgfqpoint{1.079022in}{1.981861in}}%
\pgfpathlineto{\pgfqpoint{1.092223in}{1.979056in}}%
\pgfpathlineto{\pgfqpoint{1.087202in}{1.953404in}}%
\pgfpathlineto{\pgfqpoint{1.100425in}{1.951001in}}%
\pgfpathlineto{\pgfqpoint{1.098470in}{1.940916in}}%
\pgfpathlineto{\pgfqpoint{1.104825in}{1.939769in}}%
\pgfpathlineto{\pgfqpoint{1.109426in}{1.931652in}}%
\pgfpathlineto{\pgfqpoint{1.106652in}{1.920240in}}%
\pgfpathlineto{\pgfqpoint{1.103333in}{1.915851in}}%
\pgfpathlineto{\pgfqpoint{1.093346in}{1.913918in}}%
\pgfpathlineto{\pgfqpoint{1.090767in}{1.911577in}}%
\pgfpathlineto{\pgfqpoint{1.088809in}{1.902795in}}%
\pgfpathlineto{\pgfqpoint{1.093013in}{1.895851in}}%
\pgfpathlineto{\pgfqpoint{1.092166in}{1.875884in}}%
\pgfpathclose%
\pgfusepath{fill}%
\end{pgfscope}%
\begin{pgfscope}%
\pgfpathrectangle{\pgfqpoint{0.100000in}{0.100000in}}{\pgfqpoint{3.420221in}{2.189500in}}%
\pgfusepath{clip}%
\pgfsetbuttcap%
\pgfsetmiterjoin%
\definecolor{currentfill}{rgb}{0.000000,0.129412,0.935294}%
\pgfsetfillcolor{currentfill}%
\pgfsetlinewidth{0.000000pt}%
\definecolor{currentstroke}{rgb}{0.000000,0.000000,0.000000}%
\pgfsetstrokecolor{currentstroke}%
\pgfsetstrokeopacity{0.000000}%
\pgfsetdash{}{0pt}%
\pgfpathmoveto{\pgfqpoint{1.841579in}{1.404755in}}%
\pgfpathlineto{\pgfqpoint{1.790722in}{1.406614in}}%
\pgfpathlineto{\pgfqpoint{1.791757in}{1.432638in}}%
\pgfpathlineto{\pgfqpoint{1.790258in}{1.432707in}}%
\pgfpathlineto{\pgfqpoint{1.791568in}{1.458730in}}%
\pgfpathlineto{\pgfqpoint{1.817016in}{1.457714in}}%
\pgfpathlineto{\pgfqpoint{1.817573in}{1.447983in}}%
\pgfpathlineto{\pgfqpoint{1.827355in}{1.447614in}}%
\pgfpathlineto{\pgfqpoint{1.830165in}{1.449697in}}%
\pgfpathlineto{\pgfqpoint{1.849771in}{1.450287in}}%
\pgfpathlineto{\pgfqpoint{1.850060in}{1.456712in}}%
\pgfpathlineto{\pgfqpoint{1.855980in}{1.456548in}}%
\pgfpathlineto{\pgfqpoint{1.855846in}{1.451895in}}%
\pgfpathlineto{\pgfqpoint{1.846196in}{1.444822in}}%
\pgfpathlineto{\pgfqpoint{1.842536in}{1.440340in}}%
\pgfpathclose%
\pgfusepath{fill}%
\end{pgfscope}%
\begin{pgfscope}%
\pgfpathrectangle{\pgfqpoint{0.100000in}{0.100000in}}{\pgfqpoint{3.420221in}{2.189500in}}%
\pgfusepath{clip}%
\pgfsetbuttcap%
\pgfsetmiterjoin%
\definecolor{currentfill}{rgb}{0.000000,0.517647,0.741176}%
\pgfsetfillcolor{currentfill}%
\pgfsetlinewidth{0.000000pt}%
\definecolor{currentstroke}{rgb}{0.000000,0.000000,0.000000}%
\pgfsetstrokecolor{currentstroke}%
\pgfsetstrokeopacity{0.000000}%
\pgfsetdash{}{0pt}%
\pgfpathmoveto{\pgfqpoint{0.763248in}{1.886195in}}%
\pgfpathlineto{\pgfqpoint{0.771217in}{1.883742in}}%
\pgfpathlineto{\pgfqpoint{0.777344in}{1.877877in}}%
\pgfpathlineto{\pgfqpoint{0.776125in}{1.873424in}}%
\pgfpathlineto{\pgfqpoint{0.770503in}{1.871844in}}%
\pgfpathlineto{\pgfqpoint{0.770612in}{1.865160in}}%
\pgfpathlineto{\pgfqpoint{0.757712in}{1.864804in}}%
\pgfpathlineto{\pgfqpoint{0.757638in}{1.861950in}}%
\pgfpathlineto{\pgfqpoint{0.765320in}{1.857286in}}%
\pgfpathlineto{\pgfqpoint{0.766402in}{1.853531in}}%
\pgfpathlineto{\pgfqpoint{0.757735in}{1.845008in}}%
\pgfpathlineto{\pgfqpoint{0.756738in}{1.839434in}}%
\pgfpathlineto{\pgfqpoint{0.750110in}{1.831978in}}%
\pgfpathlineto{\pgfqpoint{0.763945in}{1.828419in}}%
\pgfpathlineto{\pgfqpoint{0.760080in}{1.812628in}}%
\pgfpathlineto{\pgfqpoint{0.729459in}{1.821334in}}%
\pgfpathlineto{\pgfqpoint{0.727730in}{1.814970in}}%
\pgfpathlineto{\pgfqpoint{0.683805in}{1.826899in}}%
\pgfpathlineto{\pgfqpoint{0.690922in}{1.852082in}}%
\pgfpathlineto{\pgfqpoint{0.701514in}{1.889466in}}%
\pgfpathlineto{\pgfqpoint{0.703950in}{1.902176in}}%
\pgfpathclose%
\pgfusepath{fill}%
\end{pgfscope}%
\begin{pgfscope}%
\pgfpathrectangle{\pgfqpoint{0.100000in}{0.100000in}}{\pgfqpoint{3.420221in}{2.189500in}}%
\pgfusepath{clip}%
\pgfsetbuttcap%
\pgfsetmiterjoin%
\definecolor{currentfill}{rgb}{0.000000,0.662745,0.668627}%
\pgfsetfillcolor{currentfill}%
\pgfsetlinewidth{0.000000pt}%
\definecolor{currentstroke}{rgb}{0.000000,0.000000,0.000000}%
\pgfsetstrokecolor{currentstroke}%
\pgfsetstrokeopacity{0.000000}%
\pgfsetdash{}{0pt}%
\pgfpathmoveto{\pgfqpoint{1.206560in}{1.460491in}}%
\pgfpathlineto{\pgfqpoint{1.194936in}{1.386988in}}%
\pgfpathlineto{\pgfqpoint{1.193046in}{1.374938in}}%
\pgfpathlineto{\pgfqpoint{1.189457in}{1.372792in}}%
\pgfpathlineto{\pgfqpoint{1.137086in}{1.381237in}}%
\pgfpathlineto{\pgfqpoint{1.137594in}{1.389126in}}%
\pgfpathlineto{\pgfqpoint{1.141849in}{1.395819in}}%
\pgfpathlineto{\pgfqpoint{1.142202in}{1.401252in}}%
\pgfpathlineto{\pgfqpoint{1.149245in}{1.405348in}}%
\pgfpathlineto{\pgfqpoint{1.113304in}{1.411483in}}%
\pgfpathlineto{\pgfqpoint{1.094137in}{1.415334in}}%
\pgfpathlineto{\pgfqpoint{1.093336in}{1.422040in}}%
\pgfpathlineto{\pgfqpoint{1.102917in}{1.479593in}}%
\pgfpathlineto{\pgfqpoint{1.112235in}{1.483241in}}%
\pgfpathlineto{\pgfqpoint{1.117404in}{1.481801in}}%
\pgfpathlineto{\pgfqpoint{1.133812in}{1.482595in}}%
\pgfpathlineto{\pgfqpoint{1.140808in}{1.484998in}}%
\pgfpathlineto{\pgfqpoint{1.145712in}{1.482426in}}%
\pgfpathlineto{\pgfqpoint{1.155070in}{1.480949in}}%
\pgfpathlineto{\pgfqpoint{1.161891in}{1.476028in}}%
\pgfpathlineto{\pgfqpoint{1.181626in}{1.477648in}}%
\pgfpathlineto{\pgfqpoint{1.183113in}{1.470719in}}%
\pgfpathlineto{\pgfqpoint{1.188745in}{1.473718in}}%
\pgfpathlineto{\pgfqpoint{1.189533in}{1.478692in}}%
\pgfpathlineto{\pgfqpoint{1.200246in}{1.477036in}}%
\pgfpathlineto{\pgfqpoint{1.201620in}{1.471299in}}%
\pgfpathlineto{\pgfqpoint{1.200574in}{1.463720in}}%
\pgfpathclose%
\pgfusepath{fill}%
\end{pgfscope}%
\begin{pgfscope}%
\pgfpathrectangle{\pgfqpoint{0.100000in}{0.100000in}}{\pgfqpoint{3.420221in}{2.189500in}}%
\pgfusepath{clip}%
\pgfsetbuttcap%
\pgfsetmiterjoin%
\definecolor{currentfill}{rgb}{0.000000,0.505882,0.747059}%
\pgfsetfillcolor{currentfill}%
\pgfsetlinewidth{0.000000pt}%
\definecolor{currentstroke}{rgb}{0.000000,0.000000,0.000000}%
\pgfsetstrokecolor{currentstroke}%
\pgfsetstrokeopacity{0.000000}%
\pgfsetdash{}{0pt}%
\pgfpathmoveto{\pgfqpoint{2.002136in}{1.100420in}}%
\pgfpathlineto{\pgfqpoint{2.002152in}{1.120857in}}%
\pgfpathlineto{\pgfqpoint{1.997708in}{1.125075in}}%
\pgfpathlineto{\pgfqpoint{1.997645in}{1.153883in}}%
\pgfpathlineto{\pgfqpoint{1.996892in}{1.180023in}}%
\pgfpathlineto{\pgfqpoint{1.997524in}{1.202886in}}%
\pgfpathlineto{\pgfqpoint{2.024890in}{1.202876in}}%
\pgfpathlineto{\pgfqpoint{2.024892in}{1.152471in}}%
\pgfpathlineto{\pgfqpoint{2.056798in}{1.151566in}}%
\pgfpathlineto{\pgfqpoint{2.058615in}{1.147148in}}%
\pgfpathlineto{\pgfqpoint{2.058049in}{1.106494in}}%
\pgfpathlineto{\pgfqpoint{2.025034in}{1.107657in}}%
\pgfpathlineto{\pgfqpoint{2.025079in}{1.100257in}}%
\pgfpathclose%
\pgfusepath{fill}%
\end{pgfscope}%
\begin{pgfscope}%
\pgfpathrectangle{\pgfqpoint{0.100000in}{0.100000in}}{\pgfqpoint{3.420221in}{2.189500in}}%
\pgfusepath{clip}%
\pgfsetbuttcap%
\pgfsetmiterjoin%
\definecolor{currentfill}{rgb}{0.000000,0.137255,0.931373}%
\pgfsetfillcolor{currentfill}%
\pgfsetlinewidth{0.000000pt}%
\definecolor{currentstroke}{rgb}{0.000000,0.000000,0.000000}%
\pgfsetstrokecolor{currentstroke}%
\pgfsetstrokeopacity{0.000000}%
\pgfsetdash{}{0pt}%
\pgfpathmoveto{\pgfqpoint{2.925444in}{0.247787in}}%
\pgfpathlineto{\pgfqpoint{2.926355in}{0.250561in}}%
\pgfpathlineto{\pgfqpoint{2.936966in}{0.255565in}}%
\pgfpathlineto{\pgfqpoint{2.941062in}{0.261680in}}%
\pgfpathlineto{\pgfqpoint{2.947836in}{0.266216in}}%
\pgfpathlineto{\pgfqpoint{2.957625in}{0.264995in}}%
\pgfpathlineto{\pgfqpoint{2.962054in}{0.262197in}}%
\pgfpathlineto{\pgfqpoint{2.953651in}{0.257839in}}%
\pgfpathlineto{\pgfqpoint{2.946490in}{0.259847in}}%
\pgfpathlineto{\pgfqpoint{2.944958in}{0.256368in}}%
\pgfpathlineto{\pgfqpoint{2.934015in}{0.250116in}}%
\pgfpathclose%
\pgfusepath{fill}%
\end{pgfscope}%
\begin{pgfscope}%
\pgfpathrectangle{\pgfqpoint{0.100000in}{0.100000in}}{\pgfqpoint{3.420221in}{2.189500in}}%
\pgfusepath{clip}%
\pgfsetbuttcap%
\pgfsetmiterjoin%
\definecolor{currentfill}{rgb}{0.000000,0.137255,0.931373}%
\pgfsetfillcolor{currentfill}%
\pgfsetlinewidth{0.000000pt}%
\definecolor{currentstroke}{rgb}{0.000000,0.000000,0.000000}%
\pgfsetstrokecolor{currentstroke}%
\pgfsetstrokeopacity{0.000000}%
\pgfsetdash{}{0pt}%
\pgfpathmoveto{\pgfqpoint{2.943450in}{0.348492in}}%
\pgfpathlineto{\pgfqpoint{2.975639in}{0.353660in}}%
\pgfpathlineto{\pgfqpoint{2.969786in}{0.388022in}}%
\pgfpathlineto{\pgfqpoint{2.968807in}{0.393642in}}%
\pgfpathlineto{\pgfqpoint{3.013221in}{0.400828in}}%
\pgfpathlineto{\pgfqpoint{3.023709in}{0.401480in}}%
\pgfpathlineto{\pgfqpoint{3.024268in}{0.383398in}}%
\pgfpathlineto{\pgfqpoint{3.026867in}{0.362681in}}%
\pgfpathlineto{\pgfqpoint{3.026122in}{0.356571in}}%
\pgfpathlineto{\pgfqpoint{3.019470in}{0.353554in}}%
\pgfpathlineto{\pgfqpoint{3.016185in}{0.334062in}}%
\pgfpathlineto{\pgfqpoint{3.017951in}{0.323830in}}%
\pgfpathlineto{\pgfqpoint{3.008300in}{0.311868in}}%
\pgfpathlineto{\pgfqpoint{3.003288in}{0.313852in}}%
\pgfpathlineto{\pgfqpoint{3.000985in}{0.308591in}}%
\pgfpathlineto{\pgfqpoint{2.995253in}{0.304927in}}%
\pgfpathlineto{\pgfqpoint{2.984798in}{0.306140in}}%
\pgfpathlineto{\pgfqpoint{2.981733in}{0.302698in}}%
\pgfpathlineto{\pgfqpoint{2.969097in}{0.298909in}}%
\pgfpathlineto{\pgfqpoint{2.962014in}{0.306050in}}%
\pgfpathlineto{\pgfqpoint{2.962191in}{0.314235in}}%
\pgfpathlineto{\pgfqpoint{2.959454in}{0.324533in}}%
\pgfpathlineto{\pgfqpoint{2.956021in}{0.327374in}}%
\pgfpathlineto{\pgfqpoint{2.952146in}{0.336458in}}%
\pgfpathlineto{\pgfqpoint{2.945035in}{0.341965in}}%
\pgfpathclose%
\pgfusepath{fill}%
\end{pgfscope}%
\begin{pgfscope}%
\pgfpathrectangle{\pgfqpoint{0.100000in}{0.100000in}}{\pgfqpoint{3.420221in}{2.189500in}}%
\pgfusepath{clip}%
\pgfsetbuttcap%
\pgfsetmiterjoin%
\definecolor{currentfill}{rgb}{0.000000,0.525490,0.737255}%
\pgfsetfillcolor{currentfill}%
\pgfsetlinewidth{0.000000pt}%
\definecolor{currentstroke}{rgb}{0.000000,0.000000,0.000000}%
\pgfsetstrokecolor{currentstroke}%
\pgfsetstrokeopacity{0.000000}%
\pgfsetdash{}{0pt}%
\pgfpathmoveto{\pgfqpoint{2.420068in}{0.756234in}}%
\pgfpathlineto{\pgfqpoint{2.409571in}{0.752466in}}%
\pgfpathlineto{\pgfqpoint{2.392453in}{0.749228in}}%
\pgfpathlineto{\pgfqpoint{2.390228in}{0.779149in}}%
\pgfpathlineto{\pgfqpoint{2.364194in}{0.777408in}}%
\pgfpathlineto{\pgfqpoint{2.361084in}{0.830708in}}%
\pgfpathlineto{\pgfqpoint{2.393064in}{0.832373in}}%
\pgfpathlineto{\pgfqpoint{2.422371in}{0.834679in}}%
\pgfpathlineto{\pgfqpoint{2.421713in}{0.808110in}}%
\pgfpathclose%
\pgfusepath{fill}%
\end{pgfscope}%
\begin{pgfscope}%
\pgfpathrectangle{\pgfqpoint{0.100000in}{0.100000in}}{\pgfqpoint{3.420221in}{2.189500in}}%
\pgfusepath{clip}%
\pgfsetbuttcap%
\pgfsetmiterjoin%
\definecolor{currentfill}{rgb}{0.000000,0.309804,0.845098}%
\pgfsetfillcolor{currentfill}%
\pgfsetlinewidth{0.000000pt}%
\definecolor{currentstroke}{rgb}{0.000000,0.000000,0.000000}%
\pgfsetstrokecolor{currentstroke}%
\pgfsetstrokeopacity{0.000000}%
\pgfsetdash{}{0pt}%
\pgfpathmoveto{\pgfqpoint{1.185179in}{0.644722in}}%
\pgfpathlineto{\pgfqpoint{1.189621in}{0.652344in}}%
\pgfpathlineto{\pgfqpoint{1.202957in}{0.649662in}}%
\pgfpathlineto{\pgfqpoint{1.198197in}{0.645446in}}%
\pgfpathlineto{\pgfqpoint{1.191979in}{0.647634in}}%
\pgfpathclose%
\pgfusepath{fill}%
\end{pgfscope}%
\begin{pgfscope}%
\pgfpathrectangle{\pgfqpoint{0.100000in}{0.100000in}}{\pgfqpoint{3.420221in}{2.189500in}}%
\pgfusepath{clip}%
\pgfsetbuttcap%
\pgfsetmiterjoin%
\definecolor{currentfill}{rgb}{0.000000,0.309804,0.845098}%
\pgfsetfillcolor{currentfill}%
\pgfsetlinewidth{0.000000pt}%
\definecolor{currentstroke}{rgb}{0.000000,0.000000,0.000000}%
\pgfsetstrokecolor{currentstroke}%
\pgfsetstrokeopacity{0.000000}%
\pgfsetdash{}{0pt}%
\pgfpathmoveto{\pgfqpoint{1.231517in}{0.614132in}}%
\pgfpathlineto{\pgfqpoint{1.223455in}{0.625527in}}%
\pgfpathlineto{\pgfqpoint{1.219950in}{0.635416in}}%
\pgfpathlineto{\pgfqpoint{1.220795in}{0.639839in}}%
\pgfpathlineto{\pgfqpoint{1.230590in}{0.643949in}}%
\pgfpathlineto{\pgfqpoint{1.242064in}{0.641651in}}%
\pgfpathlineto{\pgfqpoint{1.245897in}{0.636862in}}%
\pgfpathlineto{\pgfqpoint{1.252517in}{0.633317in}}%
\pgfpathlineto{\pgfqpoint{1.254770in}{0.626608in}}%
\pgfpathlineto{\pgfqpoint{1.252079in}{0.621438in}}%
\pgfpathlineto{\pgfqpoint{1.246645in}{0.619611in}}%
\pgfpathlineto{\pgfqpoint{1.242469in}{0.614214in}}%
\pgfpathclose%
\pgfusepath{fill}%
\end{pgfscope}%
\begin{pgfscope}%
\pgfpathrectangle{\pgfqpoint{0.100000in}{0.100000in}}{\pgfqpoint{3.420221in}{2.189500in}}%
\pgfusepath{clip}%
\pgfsetbuttcap%
\pgfsetmiterjoin%
\definecolor{currentfill}{rgb}{0.000000,0.254902,0.872549}%
\pgfsetfillcolor{currentfill}%
\pgfsetlinewidth{0.000000pt}%
\definecolor{currentstroke}{rgb}{0.000000,0.000000,0.000000}%
\pgfsetstrokecolor{currentstroke}%
\pgfsetstrokeopacity{0.000000}%
\pgfsetdash{}{0pt}%
\pgfpathmoveto{\pgfqpoint{1.849785in}{1.929677in}}%
\pgfpathlineto{\pgfqpoint{1.848833in}{1.903474in}}%
\pgfpathlineto{\pgfqpoint{1.849927in}{1.896953in}}%
\pgfpathlineto{\pgfqpoint{1.823915in}{1.897869in}}%
\pgfpathlineto{\pgfqpoint{1.822533in}{1.904442in}}%
\pgfpathlineto{\pgfqpoint{1.823502in}{1.930624in}}%
\pgfpathclose%
\pgfusepath{fill}%
\end{pgfscope}%
\begin{pgfscope}%
\pgfpathrectangle{\pgfqpoint{0.100000in}{0.100000in}}{\pgfqpoint{3.420221in}{2.189500in}}%
\pgfusepath{clip}%
\pgfsetbuttcap%
\pgfsetmiterjoin%
\definecolor{currentfill}{rgb}{0.000000,0.309804,0.845098}%
\pgfsetfillcolor{currentfill}%
\pgfsetlinewidth{0.000000pt}%
\definecolor{currentstroke}{rgb}{0.000000,0.000000,0.000000}%
\pgfsetstrokecolor{currentstroke}%
\pgfsetstrokeopacity{0.000000}%
\pgfsetdash{}{0pt}%
\pgfpathmoveto{\pgfqpoint{1.403019in}{1.648421in}}%
\pgfpathlineto{\pgfqpoint{1.371748in}{1.652066in}}%
\pgfpathlineto{\pgfqpoint{1.343956in}{1.656380in}}%
\pgfpathlineto{\pgfqpoint{1.348673in}{1.706219in}}%
\pgfpathlineto{\pgfqpoint{1.350357in}{1.718012in}}%
\pgfpathlineto{\pgfqpoint{1.348518in}{1.727558in}}%
\pgfpathlineto{\pgfqpoint{1.343916in}{1.731960in}}%
\pgfpathlineto{\pgfqpoint{1.338406in}{1.741463in}}%
\pgfpathlineto{\pgfqpoint{1.329176in}{1.749577in}}%
\pgfpathlineto{\pgfqpoint{1.322655in}{1.752598in}}%
\pgfpathlineto{\pgfqpoint{1.317807in}{1.764287in}}%
\pgfpathlineto{\pgfqpoint{1.316700in}{1.774159in}}%
\pgfpathlineto{\pgfqpoint{1.298884in}{1.776601in}}%
\pgfpathlineto{\pgfqpoint{1.306452in}{1.790243in}}%
\pgfpathlineto{\pgfqpoint{1.309242in}{1.791914in}}%
\pgfpathlineto{\pgfqpoint{1.277639in}{1.796419in}}%
\pgfpathlineto{\pgfqpoint{1.280987in}{1.812314in}}%
\pgfpathlineto{\pgfqpoint{1.283280in}{1.814330in}}%
\pgfpathlineto{\pgfqpoint{1.300425in}{1.811495in}}%
\pgfpathlineto{\pgfqpoint{1.313956in}{1.813916in}}%
\pgfpathlineto{\pgfqpoint{1.316859in}{1.833719in}}%
\pgfpathlineto{\pgfqpoint{1.319235in}{1.842198in}}%
\pgfpathlineto{\pgfqpoint{1.327852in}{1.840933in}}%
\pgfpathlineto{\pgfqpoint{1.333113in}{1.846795in}}%
\pgfpathlineto{\pgfqpoint{1.339532in}{1.845889in}}%
\pgfpathlineto{\pgfqpoint{1.340149in}{1.850219in}}%
\pgfpathlineto{\pgfqpoint{1.348591in}{1.849014in}}%
\pgfpathlineto{\pgfqpoint{1.353089in}{1.848403in}}%
\pgfpathlineto{\pgfqpoint{1.352196in}{1.841927in}}%
\pgfpathlineto{\pgfqpoint{1.365121in}{1.840167in}}%
\pgfpathlineto{\pgfqpoint{1.376601in}{1.832033in}}%
\pgfpathlineto{\pgfqpoint{1.376126in}{1.817941in}}%
\pgfpathlineto{\pgfqpoint{1.385596in}{1.815858in}}%
\pgfpathlineto{\pgfqpoint{1.382484in}{1.791831in}}%
\pgfpathlineto{\pgfqpoint{1.378792in}{1.779238in}}%
\pgfpathlineto{\pgfqpoint{1.405365in}{1.775860in}}%
\pgfpathlineto{\pgfqpoint{1.403638in}{1.761954in}}%
\pgfpathlineto{\pgfqpoint{1.416260in}{1.760382in}}%
\pgfpathlineto{\pgfqpoint{1.414273in}{1.737835in}}%
\pgfpathlineto{\pgfqpoint{1.409200in}{1.698771in}}%
\pgfpathclose%
\pgfusepath{fill}%
\end{pgfscope}%
\begin{pgfscope}%
\pgfpathrectangle{\pgfqpoint{0.100000in}{0.100000in}}{\pgfqpoint{3.420221in}{2.189500in}}%
\pgfusepath{clip}%
\pgfsetbuttcap%
\pgfsetmiterjoin%
\definecolor{currentfill}{rgb}{0.000000,0.396078,0.801961}%
\pgfsetfillcolor{currentfill}%
\pgfsetlinewidth{0.000000pt}%
\definecolor{currentstroke}{rgb}{0.000000,0.000000,0.000000}%
\pgfsetstrokecolor{currentstroke}%
\pgfsetstrokeopacity{0.000000}%
\pgfsetdash{}{0pt}%
\pgfpathmoveto{\pgfqpoint{2.075381in}{1.639155in}}%
\pgfpathlineto{\pgfqpoint{2.068917in}{1.639049in}}%
\pgfpathlineto{\pgfqpoint{2.068551in}{1.665207in}}%
\pgfpathlineto{\pgfqpoint{2.081641in}{1.665413in}}%
\pgfpathlineto{\pgfqpoint{2.081265in}{1.691557in}}%
\pgfpathlineto{\pgfqpoint{2.094293in}{1.691816in}}%
\pgfpathlineto{\pgfqpoint{2.094372in}{1.686390in}}%
\pgfpathlineto{\pgfqpoint{2.107400in}{1.686648in}}%
\pgfpathlineto{\pgfqpoint{2.108054in}{1.639769in}}%
\pgfpathclose%
\pgfusepath{fill}%
\end{pgfscope}%
\begin{pgfscope}%
\pgfpathrectangle{\pgfqpoint{0.100000in}{0.100000in}}{\pgfqpoint{3.420221in}{2.189500in}}%
\pgfusepath{clip}%
\pgfsetbuttcap%
\pgfsetmiterjoin%
\definecolor{currentfill}{rgb}{0.000000,0.188235,0.905882}%
\pgfsetfillcolor{currentfill}%
\pgfsetlinewidth{0.000000pt}%
\definecolor{currentstroke}{rgb}{0.000000,0.000000,0.000000}%
\pgfsetstrokecolor{currentstroke}%
\pgfsetstrokeopacity{0.000000}%
\pgfsetdash{}{0pt}%
\pgfpathmoveto{\pgfqpoint{1.700925in}{1.260846in}}%
\pgfpathlineto{\pgfqpoint{1.734136in}{1.259007in}}%
\pgfpathlineto{\pgfqpoint{1.732547in}{1.226504in}}%
\pgfpathlineto{\pgfqpoint{1.694779in}{1.228547in}}%
\pgfpathlineto{\pgfqpoint{1.667922in}{1.230313in}}%
\pgfpathlineto{\pgfqpoint{1.669769in}{1.262911in}}%
\pgfpathclose%
\pgfusepath{fill}%
\end{pgfscope}%
\begin{pgfscope}%
\pgfpathrectangle{\pgfqpoint{0.100000in}{0.100000in}}{\pgfqpoint{3.420221in}{2.189500in}}%
\pgfusepath{clip}%
\pgfsetbuttcap%
\pgfsetmiterjoin%
\definecolor{currentfill}{rgb}{0.000000,0.368627,0.815686}%
\pgfsetfillcolor{currentfill}%
\pgfsetlinewidth{0.000000pt}%
\definecolor{currentstroke}{rgb}{0.000000,0.000000,0.000000}%
\pgfsetstrokecolor{currentstroke}%
\pgfsetstrokeopacity{0.000000}%
\pgfsetdash{}{0pt}%
\pgfpathmoveto{\pgfqpoint{1.635812in}{1.818838in}}%
\pgfpathlineto{\pgfqpoint{1.641297in}{1.819607in}}%
\pgfpathlineto{\pgfqpoint{1.645789in}{1.816340in}}%
\pgfpathlineto{\pgfqpoint{1.658917in}{1.813889in}}%
\pgfpathlineto{\pgfqpoint{1.670030in}{1.818285in}}%
\pgfpathlineto{\pgfqpoint{1.672808in}{1.822068in}}%
\pgfpathlineto{\pgfqpoint{1.687807in}{1.832577in}}%
\pgfpathlineto{\pgfqpoint{1.695956in}{1.841534in}}%
\pgfpathlineto{\pgfqpoint{1.699558in}{1.839401in}}%
\pgfpathlineto{\pgfqpoint{1.711159in}{1.841959in}}%
\pgfpathlineto{\pgfqpoint{1.712889in}{1.836978in}}%
\pgfpathlineto{\pgfqpoint{1.711993in}{1.827807in}}%
\pgfpathlineto{\pgfqpoint{1.708393in}{1.822143in}}%
\pgfpathlineto{\pgfqpoint{1.708436in}{1.814446in}}%
\pgfpathlineto{\pgfqpoint{1.713684in}{1.805164in}}%
\pgfpathlineto{\pgfqpoint{1.719230in}{1.799199in}}%
\pgfpathlineto{\pgfqpoint{1.722595in}{1.786818in}}%
\pgfpathlineto{\pgfqpoint{1.718238in}{1.783426in}}%
\pgfpathlineto{\pgfqpoint{1.713200in}{1.773159in}}%
\pgfpathlineto{\pgfqpoint{1.719850in}{1.769218in}}%
\pgfpathlineto{\pgfqpoint{1.686296in}{1.771266in}}%
\pgfpathlineto{\pgfqpoint{1.632361in}{1.775143in}}%
\pgfpathclose%
\pgfusepath{fill}%
\end{pgfscope}%
\begin{pgfscope}%
\pgfpathrectangle{\pgfqpoint{0.100000in}{0.100000in}}{\pgfqpoint{3.420221in}{2.189500in}}%
\pgfusepath{clip}%
\pgfsetbuttcap%
\pgfsetmiterjoin%
\definecolor{currentfill}{rgb}{0.000000,0.101961,0.949020}%
\pgfsetfillcolor{currentfill}%
\pgfsetlinewidth{0.000000pt}%
\definecolor{currentstroke}{rgb}{0.000000,0.000000,0.000000}%
\pgfsetstrokecolor{currentstroke}%
\pgfsetstrokeopacity{0.000000}%
\pgfsetdash{}{0pt}%
\pgfpathmoveto{\pgfqpoint{1.282357in}{1.204916in}}%
\pgfpathlineto{\pgfqpoint{1.261509in}{1.207736in}}%
\pgfpathlineto{\pgfqpoint{1.265048in}{1.233381in}}%
\pgfpathlineto{\pgfqpoint{1.263905in}{1.246838in}}%
\pgfpathlineto{\pgfqpoint{1.261999in}{1.248700in}}%
\pgfpathlineto{\pgfqpoint{1.264044in}{1.257078in}}%
\pgfpathlineto{\pgfqpoint{1.264081in}{1.268519in}}%
\pgfpathlineto{\pgfqpoint{1.261626in}{1.274238in}}%
\pgfpathlineto{\pgfqpoint{1.269489in}{1.273146in}}%
\pgfpathlineto{\pgfqpoint{1.278839in}{1.341129in}}%
\pgfpathlineto{\pgfqpoint{1.285356in}{1.343136in}}%
\pgfpathlineto{\pgfqpoint{1.286871in}{1.336633in}}%
\pgfpathlineto{\pgfqpoint{1.290404in}{1.332043in}}%
\pgfpathlineto{\pgfqpoint{1.302857in}{1.330371in}}%
\pgfpathlineto{\pgfqpoint{1.310732in}{1.319787in}}%
\pgfpathlineto{\pgfqpoint{1.316870in}{1.318005in}}%
\pgfpathlineto{\pgfqpoint{1.320384in}{1.322298in}}%
\pgfpathlineto{\pgfqpoint{1.325819in}{1.321737in}}%
\pgfpathlineto{\pgfqpoint{1.328624in}{1.317869in}}%
\pgfpathlineto{\pgfqpoint{1.335547in}{1.310367in}}%
\pgfpathlineto{\pgfqpoint{1.343757in}{1.308151in}}%
\pgfpathlineto{\pgfqpoint{1.337562in}{1.303598in}}%
\pgfpathlineto{\pgfqpoint{1.336567in}{1.296292in}}%
\pgfpathlineto{\pgfqpoint{1.338815in}{1.279185in}}%
\pgfpathlineto{\pgfqpoint{1.343652in}{1.272532in}}%
\pgfpathlineto{\pgfqpoint{1.299806in}{1.278387in}}%
\pgfpathlineto{\pgfqpoint{1.295173in}{1.243509in}}%
\pgfpathlineto{\pgfqpoint{1.286945in}{1.243309in}}%
\pgfpathlineto{\pgfqpoint{1.283908in}{1.223805in}}%
\pgfpathclose%
\pgfusepath{fill}%
\end{pgfscope}%
\begin{pgfscope}%
\pgfpathrectangle{\pgfqpoint{0.100000in}{0.100000in}}{\pgfqpoint{3.420221in}{2.189500in}}%
\pgfusepath{clip}%
\pgfsetbuttcap%
\pgfsetmiterjoin%
\definecolor{currentfill}{rgb}{0.000000,0.458824,0.770588}%
\pgfsetfillcolor{currentfill}%
\pgfsetlinewidth{0.000000pt}%
\definecolor{currentstroke}{rgb}{0.000000,0.000000,0.000000}%
\pgfsetstrokecolor{currentstroke}%
\pgfsetstrokeopacity{0.000000}%
\pgfsetdash{}{0pt}%
\pgfpathmoveto{\pgfqpoint{1.632361in}{1.775143in}}%
\pgfpathlineto{\pgfqpoint{1.686296in}{1.771266in}}%
\pgfpathlineto{\pgfqpoint{1.719850in}{1.769218in}}%
\pgfpathlineto{\pgfqpoint{1.723020in}{1.762111in}}%
\pgfpathlineto{\pgfqpoint{1.720482in}{1.757483in}}%
\pgfpathlineto{\pgfqpoint{1.723158in}{1.751959in}}%
\pgfpathlineto{\pgfqpoint{1.719130in}{1.741235in}}%
\pgfpathlineto{\pgfqpoint{1.720592in}{1.734573in}}%
\pgfpathlineto{\pgfqpoint{1.713369in}{1.733308in}}%
\pgfpathlineto{\pgfqpoint{1.713940in}{1.726154in}}%
\pgfpathlineto{\pgfqpoint{1.714372in}{1.723836in}}%
\pgfpathlineto{\pgfqpoint{1.706651in}{1.716658in}}%
\pgfpathlineto{\pgfqpoint{1.699606in}{1.720933in}}%
\pgfpathlineto{\pgfqpoint{1.693792in}{1.716869in}}%
\pgfpathlineto{\pgfqpoint{1.690745in}{1.718949in}}%
\pgfpathlineto{\pgfqpoint{1.680165in}{1.715175in}}%
\pgfpathlineto{\pgfqpoint{1.674179in}{1.717315in}}%
\pgfpathlineto{\pgfqpoint{1.668661in}{1.713429in}}%
\pgfpathlineto{\pgfqpoint{1.661991in}{1.713775in}}%
\pgfpathlineto{\pgfqpoint{1.652199in}{1.705328in}}%
\pgfpathlineto{\pgfqpoint{1.644782in}{1.707049in}}%
\pgfpathlineto{\pgfqpoint{1.638598in}{1.703926in}}%
\pgfpathlineto{\pgfqpoint{1.626772in}{1.702970in}}%
\pgfpathlineto{\pgfqpoint{1.629845in}{1.742526in}}%
\pgfpathclose%
\pgfusepath{fill}%
\end{pgfscope}%
\begin{pgfscope}%
\pgfpathrectangle{\pgfqpoint{0.100000in}{0.100000in}}{\pgfqpoint{3.420221in}{2.189500in}}%
\pgfusepath{clip}%
\pgfsetbuttcap%
\pgfsetmiterjoin%
\definecolor{currentfill}{rgb}{0.000000,0.372549,0.813725}%
\pgfsetfillcolor{currentfill}%
\pgfsetlinewidth{0.000000pt}%
\definecolor{currentstroke}{rgb}{0.000000,0.000000,0.000000}%
\pgfsetstrokecolor{currentstroke}%
\pgfsetstrokeopacity{0.000000}%
\pgfsetdash{}{0pt}%
\pgfpathmoveto{\pgfqpoint{2.692435in}{1.060901in}}%
\pgfpathlineto{\pgfqpoint{2.685910in}{1.059556in}}%
\pgfpathlineto{\pgfqpoint{2.675333in}{1.050814in}}%
\pgfpathlineto{\pgfqpoint{2.664650in}{1.058409in}}%
\pgfpathlineto{\pgfqpoint{2.649679in}{1.063255in}}%
\pgfpathlineto{\pgfqpoint{2.636043in}{1.058954in}}%
\pgfpathlineto{\pgfqpoint{2.633023in}{1.064552in}}%
\pgfpathlineto{\pgfqpoint{2.626571in}{1.066921in}}%
\pgfpathlineto{\pgfqpoint{2.622280in}{1.071714in}}%
\pgfpathlineto{\pgfqpoint{2.627863in}{1.078816in}}%
\pgfpathlineto{\pgfqpoint{2.624710in}{1.084986in}}%
\pgfpathlineto{\pgfqpoint{2.612102in}{1.095756in}}%
\pgfpathlineto{\pgfqpoint{2.612819in}{1.105194in}}%
\pgfpathlineto{\pgfqpoint{2.622724in}{1.113145in}}%
\pgfpathlineto{\pgfqpoint{2.631801in}{1.104082in}}%
\pgfpathlineto{\pgfqpoint{2.641617in}{1.100004in}}%
\pgfpathlineto{\pgfqpoint{2.645727in}{1.109300in}}%
\pgfpathlineto{\pgfqpoint{2.643262in}{1.125226in}}%
\pgfpathlineto{\pgfqpoint{2.647967in}{1.127351in}}%
\pgfpathlineto{\pgfqpoint{2.649158in}{1.132929in}}%
\pgfpathlineto{\pgfqpoint{2.663498in}{1.134462in}}%
\pgfpathlineto{\pgfqpoint{2.668871in}{1.122893in}}%
\pgfpathlineto{\pgfqpoint{2.672485in}{1.123858in}}%
\pgfpathlineto{\pgfqpoint{2.684932in}{1.118580in}}%
\pgfpathlineto{\pgfqpoint{2.682654in}{1.106375in}}%
\pgfpathlineto{\pgfqpoint{2.689178in}{1.090955in}}%
\pgfpathlineto{\pgfqpoint{2.681543in}{1.083589in}}%
\pgfpathlineto{\pgfqpoint{2.687386in}{1.073776in}}%
\pgfpathlineto{\pgfqpoint{2.691310in}{1.069948in}}%
\pgfpathclose%
\pgfusepath{fill}%
\end{pgfscope}%
\begin{pgfscope}%
\pgfpathrectangle{\pgfqpoint{0.100000in}{0.100000in}}{\pgfqpoint{3.420221in}{2.189500in}}%
\pgfusepath{clip}%
\pgfsetbuttcap%
\pgfsetmiterjoin%
\definecolor{currentfill}{rgb}{0.000000,0.278431,0.860784}%
\pgfsetfillcolor{currentfill}%
\pgfsetlinewidth{0.000000pt}%
\definecolor{currentstroke}{rgb}{0.000000,0.000000,0.000000}%
\pgfsetstrokecolor{currentstroke}%
\pgfsetstrokeopacity{0.000000}%
\pgfsetdash{}{0pt}%
\pgfpathmoveto{\pgfqpoint{1.705097in}{1.358483in}}%
\pgfpathlineto{\pgfqpoint{1.703097in}{1.325976in}}%
\pgfpathlineto{\pgfqpoint{1.633440in}{1.330519in}}%
\pgfpathlineto{\pgfqpoint{1.632008in}{1.330624in}}%
\pgfpathlineto{\pgfqpoint{1.634466in}{1.363052in}}%
\pgfpathlineto{\pgfqpoint{1.639449in}{1.362727in}}%
\pgfpathlineto{\pgfqpoint{1.704193in}{1.358534in}}%
\pgfpathclose%
\pgfusepath{fill}%
\end{pgfscope}%
\begin{pgfscope}%
\pgfpathrectangle{\pgfqpoint{0.100000in}{0.100000in}}{\pgfqpoint{3.420221in}{2.189500in}}%
\pgfusepath{clip}%
\pgfsetbuttcap%
\pgfsetmiterjoin%
\definecolor{currentfill}{rgb}{0.000000,0.498039,0.750980}%
\pgfsetfillcolor{currentfill}%
\pgfsetlinewidth{0.000000pt}%
\definecolor{currentstroke}{rgb}{0.000000,0.000000,0.000000}%
\pgfsetstrokecolor{currentstroke}%
\pgfsetstrokeopacity{0.000000}%
\pgfsetdash{}{0pt}%
\pgfpathmoveto{\pgfqpoint{2.927511in}{1.037437in}}%
\pgfpathlineto{\pgfqpoint{2.942880in}{1.039893in}}%
\pgfpathlineto{\pgfqpoint{2.958924in}{1.028641in}}%
\pgfpathlineto{\pgfqpoint{2.951835in}{1.002790in}}%
\pgfpathlineto{\pgfqpoint{2.938926in}{1.014824in}}%
\pgfpathlineto{\pgfqpoint{2.924164in}{1.012946in}}%
\pgfpathlineto{\pgfqpoint{2.910879in}{1.001149in}}%
\pgfpathlineto{\pgfqpoint{2.906453in}{1.010891in}}%
\pgfpathlineto{\pgfqpoint{2.900556in}{1.018396in}}%
\pgfpathlineto{\pgfqpoint{2.888787in}{1.031926in}}%
\pgfpathclose%
\pgfusepath{fill}%
\end{pgfscope}%
\begin{pgfscope}%
\pgfpathrectangle{\pgfqpoint{0.100000in}{0.100000in}}{\pgfqpoint{3.420221in}{2.189500in}}%
\pgfusepath{clip}%
\pgfsetbuttcap%
\pgfsetmiterjoin%
\definecolor{currentfill}{rgb}{0.000000,0.690196,0.654902}%
\pgfsetfillcolor{currentfill}%
\pgfsetlinewidth{0.000000pt}%
\definecolor{currentstroke}{rgb}{0.000000,0.000000,0.000000}%
\pgfsetstrokecolor{currentstroke}%
\pgfsetstrokeopacity{0.000000}%
\pgfsetdash{}{0pt}%
\pgfpathmoveto{\pgfqpoint{1.662723in}{0.538108in}}%
\pgfpathlineto{\pgfqpoint{1.625833in}{0.540144in}}%
\pgfpathlineto{\pgfqpoint{1.619078in}{0.547353in}}%
\pgfpathlineto{\pgfqpoint{1.617841in}{0.553186in}}%
\pgfpathlineto{\pgfqpoint{1.604497in}{0.563210in}}%
\pgfpathlineto{\pgfqpoint{1.599974in}{0.571143in}}%
\pgfpathlineto{\pgfqpoint{1.595818in}{0.571956in}}%
\pgfpathlineto{\pgfqpoint{1.590322in}{0.579076in}}%
\pgfpathlineto{\pgfqpoint{1.585757in}{0.581062in}}%
\pgfpathlineto{\pgfqpoint{1.579795in}{0.593621in}}%
\pgfpathlineto{\pgfqpoint{1.572226in}{0.598052in}}%
\pgfpathlineto{\pgfqpoint{1.567589in}{0.596616in}}%
\pgfpathlineto{\pgfqpoint{1.544190in}{0.600257in}}%
\pgfpathlineto{\pgfqpoint{1.538273in}{0.599684in}}%
\pgfpathlineto{\pgfqpoint{1.521140in}{0.608133in}}%
\pgfpathlineto{\pgfqpoint{1.506106in}{0.622590in}}%
\pgfpathlineto{\pgfqpoint{1.507553in}{0.640011in}}%
\pgfpathlineto{\pgfqpoint{1.522156in}{0.638874in}}%
\pgfpathlineto{\pgfqpoint{1.524085in}{0.662629in}}%
\pgfpathlineto{\pgfqpoint{1.537369in}{0.661475in}}%
\pgfpathlineto{\pgfqpoint{1.537757in}{0.665884in}}%
\pgfpathlineto{\pgfqpoint{1.563804in}{0.663750in}}%
\pgfpathlineto{\pgfqpoint{1.568005in}{0.662236in}}%
\pgfpathlineto{\pgfqpoint{1.568703in}{0.655417in}}%
\pgfpathlineto{\pgfqpoint{1.562231in}{0.646219in}}%
\pgfpathlineto{\pgfqpoint{1.567162in}{0.641054in}}%
\pgfpathlineto{\pgfqpoint{1.560320in}{0.636205in}}%
\pgfpathlineto{\pgfqpoint{1.612387in}{0.632536in}}%
\pgfpathlineto{\pgfqpoint{1.629383in}{0.631481in}}%
\pgfpathlineto{\pgfqpoint{1.626255in}{0.581190in}}%
\pgfpathlineto{\pgfqpoint{1.664970in}{0.578836in}}%
\pgfpathclose%
\pgfusepath{fill}%
\end{pgfscope}%
\begin{pgfscope}%
\pgfpathrectangle{\pgfqpoint{0.100000in}{0.100000in}}{\pgfqpoint{3.420221in}{2.189500in}}%
\pgfusepath{clip}%
\pgfsetbuttcap%
\pgfsetmiterjoin%
\definecolor{currentfill}{rgb}{0.000000,0.490196,0.754902}%
\pgfsetfillcolor{currentfill}%
\pgfsetlinewidth{0.000000pt}%
\definecolor{currentstroke}{rgb}{0.000000,0.000000,0.000000}%
\pgfsetstrokecolor{currentstroke}%
\pgfsetstrokeopacity{0.000000}%
\pgfsetdash{}{0pt}%
\pgfpathmoveto{\pgfqpoint{2.897071in}{1.406077in}}%
\pgfpathlineto{\pgfqpoint{2.897321in}{1.413647in}}%
\pgfpathlineto{\pgfqpoint{2.894424in}{1.419233in}}%
\pgfpathlineto{\pgfqpoint{2.900073in}{1.434495in}}%
\pgfpathlineto{\pgfqpoint{2.907018in}{1.445573in}}%
\pgfpathlineto{\pgfqpoint{2.913987in}{1.468265in}}%
\pgfpathlineto{\pgfqpoint{2.917269in}{1.485969in}}%
\pgfpathlineto{\pgfqpoint{2.942478in}{1.490597in}}%
\pgfpathlineto{\pgfqpoint{2.950343in}{1.488226in}}%
\pgfpathlineto{\pgfqpoint{2.954186in}{1.484678in}}%
\pgfpathlineto{\pgfqpoint{2.953333in}{1.479934in}}%
\pgfpathlineto{\pgfqpoint{2.958140in}{1.475005in}}%
\pgfpathlineto{\pgfqpoint{2.953538in}{1.460106in}}%
\pgfpathlineto{\pgfqpoint{2.955736in}{1.454929in}}%
\pgfpathlineto{\pgfqpoint{2.962266in}{1.451636in}}%
\pgfpathlineto{\pgfqpoint{2.959299in}{1.445059in}}%
\pgfpathlineto{\pgfqpoint{2.959851in}{1.438076in}}%
\pgfpathlineto{\pgfqpoint{2.957679in}{1.424882in}}%
\pgfpathlineto{\pgfqpoint{2.954465in}{1.416446in}}%
\pgfpathclose%
\pgfusepath{fill}%
\end{pgfscope}%
\begin{pgfscope}%
\pgfpathrectangle{\pgfqpoint{0.100000in}{0.100000in}}{\pgfqpoint{3.420221in}{2.189500in}}%
\pgfusepath{clip}%
\pgfsetbuttcap%
\pgfsetmiterjoin%
\definecolor{currentfill}{rgb}{0.000000,0.313725,0.843137}%
\pgfsetfillcolor{currentfill}%
\pgfsetlinewidth{0.000000pt}%
\definecolor{currentstroke}{rgb}{0.000000,0.000000,0.000000}%
\pgfsetstrokecolor{currentstroke}%
\pgfsetstrokeopacity{0.000000}%
\pgfsetdash{}{0pt}%
\pgfpathmoveto{\pgfqpoint{2.219345in}{1.289927in}}%
\pgfpathlineto{\pgfqpoint{2.219979in}{1.267758in}}%
\pgfpathlineto{\pgfqpoint{2.211182in}{1.267514in}}%
\pgfpathlineto{\pgfqpoint{2.211418in}{1.257362in}}%
\pgfpathlineto{\pgfqpoint{2.203803in}{1.254555in}}%
\pgfpathlineto{\pgfqpoint{2.198482in}{1.256419in}}%
\pgfpathlineto{\pgfqpoint{2.197868in}{1.283028in}}%
\pgfpathlineto{\pgfqpoint{2.170166in}{1.282459in}}%
\pgfpathlineto{\pgfqpoint{2.170080in}{1.295630in}}%
\pgfpathlineto{\pgfqpoint{2.157893in}{1.295761in}}%
\pgfpathlineto{\pgfqpoint{2.157906in}{1.303342in}}%
\pgfpathlineto{\pgfqpoint{2.192229in}{1.303820in}}%
\pgfpathlineto{\pgfqpoint{2.208485in}{1.302902in}}%
\pgfpathlineto{\pgfqpoint{2.210698in}{1.289780in}}%
\pgfpathclose%
\pgfusepath{fill}%
\end{pgfscope}%
\begin{pgfscope}%
\pgfpathrectangle{\pgfqpoint{0.100000in}{0.100000in}}{\pgfqpoint{3.420221in}{2.189500in}}%
\pgfusepath{clip}%
\pgfsetbuttcap%
\pgfsetmiterjoin%
\definecolor{currentfill}{rgb}{0.000000,0.392157,0.803922}%
\pgfsetfillcolor{currentfill}%
\pgfsetlinewidth{0.000000pt}%
\definecolor{currentstroke}{rgb}{0.000000,0.000000,0.000000}%
\pgfsetstrokecolor{currentstroke}%
\pgfsetstrokeopacity{0.000000}%
\pgfsetdash{}{0pt}%
\pgfpathmoveto{\pgfqpoint{2.073810in}{1.314766in}}%
\pgfpathlineto{\pgfqpoint{2.047844in}{1.314677in}}%
\pgfpathlineto{\pgfqpoint{2.047769in}{1.309244in}}%
\pgfpathlineto{\pgfqpoint{2.025198in}{1.309186in}}%
\pgfpathlineto{\pgfqpoint{2.025149in}{1.314901in}}%
\pgfpathlineto{\pgfqpoint{1.996165in}{1.314943in}}%
\pgfpathlineto{\pgfqpoint{1.998937in}{1.321697in}}%
\pgfpathlineto{\pgfqpoint{1.995465in}{1.323874in}}%
\pgfpathlineto{\pgfqpoint{1.982509in}{1.323938in}}%
\pgfpathlineto{\pgfqpoint{1.982601in}{1.349938in}}%
\pgfpathlineto{\pgfqpoint{1.984415in}{1.349932in}}%
\pgfpathlineto{\pgfqpoint{1.990503in}{1.342726in}}%
\pgfpathlineto{\pgfqpoint{1.997221in}{1.339556in}}%
\pgfpathlineto{\pgfqpoint{2.002489in}{1.343536in}}%
\pgfpathlineto{\pgfqpoint{1.999646in}{1.359667in}}%
\pgfpathlineto{\pgfqpoint{2.024686in}{1.359404in}}%
\pgfpathlineto{\pgfqpoint{2.024746in}{1.352890in}}%
\pgfpathlineto{\pgfqpoint{2.046924in}{1.352708in}}%
\pgfpathlineto{\pgfqpoint{2.046994in}{1.360312in}}%
\pgfpathlineto{\pgfqpoint{2.072911in}{1.360341in}}%
\pgfpathlineto{\pgfqpoint{2.073222in}{1.347352in}}%
\pgfpathclose%
\pgfusepath{fill}%
\end{pgfscope}%
\begin{pgfscope}%
\pgfpathrectangle{\pgfqpoint{0.100000in}{0.100000in}}{\pgfqpoint{3.420221in}{2.189500in}}%
\pgfusepath{clip}%
\pgfsetbuttcap%
\pgfsetmiterjoin%
\definecolor{currentfill}{rgb}{0.000000,0.545098,0.727451}%
\pgfsetfillcolor{currentfill}%
\pgfsetlinewidth{0.000000pt}%
\definecolor{currentstroke}{rgb}{0.000000,0.000000,0.000000}%
\pgfsetstrokecolor{currentstroke}%
\pgfsetstrokeopacity{0.000000}%
\pgfsetdash{}{0pt}%
\pgfpathmoveto{\pgfqpoint{2.338433in}{1.097884in}}%
\pgfpathlineto{\pgfqpoint{2.334359in}{1.097534in}}%
\pgfpathlineto{\pgfqpoint{2.331108in}{1.097375in}}%
\pgfpathlineto{\pgfqpoint{2.331467in}{1.085272in}}%
\pgfpathlineto{\pgfqpoint{2.328349in}{1.090102in}}%
\pgfpathlineto{\pgfqpoint{2.318776in}{1.088249in}}%
\pgfpathlineto{\pgfqpoint{2.305474in}{1.087726in}}%
\pgfpathlineto{\pgfqpoint{2.305297in}{1.105723in}}%
\pgfpathlineto{\pgfqpoint{2.293907in}{1.105211in}}%
\pgfpathlineto{\pgfqpoint{2.292963in}{1.112090in}}%
\pgfpathlineto{\pgfqpoint{2.286141in}{1.126680in}}%
\pgfpathlineto{\pgfqpoint{2.294358in}{1.139414in}}%
\pgfpathlineto{\pgfqpoint{2.287615in}{1.139098in}}%
\pgfpathlineto{\pgfqpoint{2.287048in}{1.156221in}}%
\pgfpathlineto{\pgfqpoint{2.291343in}{1.156233in}}%
\pgfpathlineto{\pgfqpoint{2.290138in}{1.177584in}}%
\pgfpathlineto{\pgfqpoint{2.322061in}{1.179420in}}%
\pgfpathlineto{\pgfqpoint{2.327323in}{1.177357in}}%
\pgfpathlineto{\pgfqpoint{2.334084in}{1.164694in}}%
\pgfpathlineto{\pgfqpoint{2.328783in}{1.156504in}}%
\pgfpathlineto{\pgfqpoint{2.337775in}{1.139697in}}%
\pgfpathlineto{\pgfqpoint{2.343643in}{1.135256in}}%
\pgfpathlineto{\pgfqpoint{2.348986in}{1.137396in}}%
\pgfpathlineto{\pgfqpoint{2.353219in}{1.135116in}}%
\pgfpathlineto{\pgfqpoint{2.355333in}{1.133690in}}%
\pgfpathlineto{\pgfqpoint{2.353555in}{1.125007in}}%
\pgfpathlineto{\pgfqpoint{2.354720in}{1.120421in}}%
\pgfpathlineto{\pgfqpoint{2.350500in}{1.114945in}}%
\pgfpathlineto{\pgfqpoint{2.353162in}{1.111382in}}%
\pgfpathlineto{\pgfqpoint{2.349528in}{1.103863in}}%
\pgfpathlineto{\pgfqpoint{2.341047in}{1.107577in}}%
\pgfpathclose%
\pgfusepath{fill}%
\end{pgfscope}%
\begin{pgfscope}%
\pgfpathrectangle{\pgfqpoint{0.100000in}{0.100000in}}{\pgfqpoint{3.420221in}{2.189500in}}%
\pgfusepath{clip}%
\pgfsetbuttcap%
\pgfsetmiterjoin%
\definecolor{currentfill}{rgb}{0.000000,0.643137,0.678431}%
\pgfsetfillcolor{currentfill}%
\pgfsetlinewidth{0.000000pt}%
\definecolor{currentstroke}{rgb}{0.000000,0.000000,0.000000}%
\pgfsetstrokecolor{currentstroke}%
\pgfsetstrokeopacity{0.000000}%
\pgfsetdash{}{0pt}%
\pgfpathmoveto{\pgfqpoint{2.420068in}{0.756234in}}%
\pgfpathlineto{\pgfqpoint{2.421694in}{0.741515in}}%
\pgfpathlineto{\pgfqpoint{2.424080in}{0.721840in}}%
\pgfpathlineto{\pgfqpoint{2.398862in}{0.719911in}}%
\pgfpathlineto{\pgfqpoint{2.362994in}{0.717612in}}%
\pgfpathlineto{\pgfqpoint{2.361174in}{0.744960in}}%
\pgfpathlineto{\pgfqpoint{2.345111in}{0.742764in}}%
\pgfpathlineto{\pgfqpoint{2.343993in}{0.760321in}}%
\pgfpathlineto{\pgfqpoint{2.339014in}{0.762555in}}%
\pgfpathlineto{\pgfqpoint{2.338229in}{0.775716in}}%
\pgfpathlineto{\pgfqpoint{2.364194in}{0.777408in}}%
\pgfpathlineto{\pgfqpoint{2.390228in}{0.779149in}}%
\pgfpathlineto{\pgfqpoint{2.392453in}{0.749228in}}%
\pgfpathlineto{\pgfqpoint{2.409571in}{0.752466in}}%
\pgfpathclose%
\pgfusepath{fill}%
\end{pgfscope}%
\begin{pgfscope}%
\pgfpathrectangle{\pgfqpoint{0.100000in}{0.100000in}}{\pgfqpoint{3.420221in}{2.189500in}}%
\pgfusepath{clip}%
\pgfsetbuttcap%
\pgfsetmiterjoin%
\definecolor{currentfill}{rgb}{0.000000,0.443137,0.778431}%
\pgfsetfillcolor{currentfill}%
\pgfsetlinewidth{0.000000pt}%
\definecolor{currentstroke}{rgb}{0.000000,0.000000,0.000000}%
\pgfsetstrokecolor{currentstroke}%
\pgfsetstrokeopacity{0.000000}%
\pgfsetdash{}{0pt}%
\pgfpathmoveto{\pgfqpoint{1.481817in}{1.965489in}}%
\pgfpathlineto{\pgfqpoint{1.475316in}{1.966203in}}%
\pgfpathlineto{\pgfqpoint{1.473041in}{1.946549in}}%
\pgfpathlineto{\pgfqpoint{1.470851in}{1.946786in}}%
\pgfpathlineto{\pgfqpoint{1.467857in}{1.920627in}}%
\pgfpathlineto{\pgfqpoint{1.446050in}{1.923144in}}%
\pgfpathlineto{\pgfqpoint{1.445261in}{1.916541in}}%
\pgfpathlineto{\pgfqpoint{1.438748in}{1.917310in}}%
\pgfpathlineto{\pgfqpoint{1.439553in}{1.923934in}}%
\pgfpathlineto{\pgfqpoint{1.433071in}{1.924714in}}%
\pgfpathlineto{\pgfqpoint{1.428809in}{1.925236in}}%
\pgfpathlineto{\pgfqpoint{1.430409in}{1.938297in}}%
\pgfpathlineto{\pgfqpoint{1.423915in}{1.939080in}}%
\pgfpathlineto{\pgfqpoint{1.427464in}{1.951967in}}%
\pgfpathlineto{\pgfqpoint{1.430617in}{1.977495in}}%
\pgfpathlineto{\pgfqpoint{1.425711in}{1.978117in}}%
\pgfpathlineto{\pgfqpoint{1.423514in}{1.985198in}}%
\pgfpathlineto{\pgfqpoint{1.425762in}{1.993121in}}%
\pgfpathlineto{\pgfqpoint{1.433700in}{1.989199in}}%
\pgfpathlineto{\pgfqpoint{1.452974in}{1.985616in}}%
\pgfpathlineto{\pgfqpoint{1.454876in}{2.001343in}}%
\pgfpathlineto{\pgfqpoint{1.456842in}{2.001118in}}%
\pgfpathlineto{\pgfqpoint{1.459939in}{2.027077in}}%
\pgfpathlineto{\pgfqpoint{1.501467in}{2.022301in}}%
\pgfpathlineto{\pgfqpoint{1.512254in}{2.021118in}}%
\pgfpathlineto{\pgfqpoint{1.511536in}{2.014579in}}%
\pgfpathlineto{\pgfqpoint{1.518050in}{2.013865in}}%
\pgfpathlineto{\pgfqpoint{1.517336in}{2.007286in}}%
\pgfpathlineto{\pgfqpoint{1.546498in}{2.004294in}}%
\pgfpathlineto{\pgfqpoint{1.543671in}{1.974567in}}%
\pgfpathlineto{\pgfqpoint{1.539973in}{1.978104in}}%
\pgfpathlineto{\pgfqpoint{1.528466in}{1.980121in}}%
\pgfpathlineto{\pgfqpoint{1.523523in}{1.986154in}}%
\pgfpathlineto{\pgfqpoint{1.512446in}{1.985950in}}%
\pgfpathlineto{\pgfqpoint{1.505517in}{1.987939in}}%
\pgfpathlineto{\pgfqpoint{1.493269in}{1.984614in}}%
\pgfpathlineto{\pgfqpoint{1.486170in}{1.985985in}}%
\pgfpathlineto{\pgfqpoint{1.484483in}{1.971097in}}%
\pgfpathclose%
\pgfusepath{fill}%
\end{pgfscope}%
\begin{pgfscope}%
\pgfpathrectangle{\pgfqpoint{0.100000in}{0.100000in}}{\pgfqpoint{3.420221in}{2.189500in}}%
\pgfusepath{clip}%
\pgfsetbuttcap%
\pgfsetmiterjoin%
\definecolor{currentfill}{rgb}{0.000000,0.698039,0.650980}%
\pgfsetfillcolor{currentfill}%
\pgfsetlinewidth{0.000000pt}%
\definecolor{currentstroke}{rgb}{0.000000,0.000000,0.000000}%
\pgfsetstrokecolor{currentstroke}%
\pgfsetstrokeopacity{0.000000}%
\pgfsetdash{}{0pt}%
\pgfpathmoveto{\pgfqpoint{1.707060in}{0.468578in}}%
\pgfpathlineto{\pgfqpoint{1.658765in}{0.470587in}}%
\pgfpathlineto{\pgfqpoint{1.662723in}{0.538108in}}%
\pgfpathlineto{\pgfqpoint{1.709016in}{0.536062in}}%
\pgfpathlineto{\pgfqpoint{1.707675in}{0.501795in}}%
\pgfpathlineto{\pgfqpoint{1.708563in}{0.501769in}}%
\pgfpathclose%
\pgfusepath{fill}%
\end{pgfscope}%
\begin{pgfscope}%
\pgfpathrectangle{\pgfqpoint{0.100000in}{0.100000in}}{\pgfqpoint{3.420221in}{2.189500in}}%
\pgfusepath{clip}%
\pgfsetbuttcap%
\pgfsetmiterjoin%
\definecolor{currentfill}{rgb}{0.000000,0.478431,0.760784}%
\pgfsetfillcolor{currentfill}%
\pgfsetlinewidth{0.000000pt}%
\definecolor{currentstroke}{rgb}{0.000000,0.000000,0.000000}%
\pgfsetstrokecolor{currentstroke}%
\pgfsetstrokeopacity{0.000000}%
\pgfsetdash{}{0pt}%
\pgfpathmoveto{\pgfqpoint{1.373045in}{0.783387in}}%
\pgfpathlineto{\pgfqpoint{1.425448in}{0.777887in}}%
\pgfpathlineto{\pgfqpoint{1.413730in}{0.711118in}}%
\pgfpathlineto{\pgfqpoint{1.357623in}{0.683419in}}%
\pgfpathlineto{\pgfqpoint{1.360223in}{0.701157in}}%
\pgfpathlineto{\pgfqpoint{1.365815in}{0.755663in}}%
\pgfpathlineto{\pgfqpoint{1.368559in}{0.783876in}}%
\pgfpathclose%
\pgfusepath{fill}%
\end{pgfscope}%
\begin{pgfscope}%
\pgfpathrectangle{\pgfqpoint{0.100000in}{0.100000in}}{\pgfqpoint{3.420221in}{2.189500in}}%
\pgfusepath{clip}%
\pgfsetbuttcap%
\pgfsetmiterjoin%
\definecolor{currentfill}{rgb}{0.000000,0.552941,0.723529}%
\pgfsetfillcolor{currentfill}%
\pgfsetlinewidth{0.000000pt}%
\definecolor{currentstroke}{rgb}{0.000000,0.000000,0.000000}%
\pgfsetstrokecolor{currentstroke}%
\pgfsetstrokeopacity{0.000000}%
\pgfsetdash{}{0pt}%
\pgfpathmoveto{\pgfqpoint{1.518648in}{1.146655in}}%
\pgfpathlineto{\pgfqpoint{1.523651in}{1.146199in}}%
\pgfpathlineto{\pgfqpoint{1.520342in}{1.108878in}}%
\pgfpathlineto{\pgfqpoint{1.517971in}{1.109091in}}%
\pgfpathlineto{\pgfqpoint{1.512957in}{1.052211in}}%
\pgfpathlineto{\pgfqpoint{1.492727in}{1.054098in}}%
\pgfpathlineto{\pgfqpoint{1.495722in}{1.079920in}}%
\pgfpathlineto{\pgfqpoint{1.469805in}{1.082330in}}%
\pgfpathlineto{\pgfqpoint{1.470466in}{1.088914in}}%
\pgfpathlineto{\pgfqpoint{1.457514in}{1.090225in}}%
\pgfpathlineto{\pgfqpoint{1.457840in}{1.093473in}}%
\pgfpathlineto{\pgfqpoint{1.463680in}{1.151536in}}%
\pgfpathclose%
\pgfusepath{fill}%
\end{pgfscope}%
\begin{pgfscope}%
\pgfpathrectangle{\pgfqpoint{0.100000in}{0.100000in}}{\pgfqpoint{3.420221in}{2.189500in}}%
\pgfusepath{clip}%
\pgfsetbuttcap%
\pgfsetmiterjoin%
\definecolor{currentfill}{rgb}{0.000000,0.768627,0.615686}%
\pgfsetfillcolor{currentfill}%
\pgfsetlinewidth{0.000000pt}%
\definecolor{currentstroke}{rgb}{0.000000,0.000000,0.000000}%
\pgfsetstrokecolor{currentstroke}%
\pgfsetstrokeopacity{0.000000}%
\pgfsetdash{}{0pt}%
\pgfpathmoveto{\pgfqpoint{0.626398in}{0.598666in}}%
\pgfpathlineto{\pgfqpoint{0.623471in}{0.601924in}}%
\pgfpathlineto{\pgfqpoint{0.623660in}{0.607835in}}%
\pgfpathlineto{\pgfqpoint{0.622792in}{0.610576in}}%
\pgfpathlineto{\pgfqpoint{0.626042in}{0.611348in}}%
\pgfpathlineto{\pgfqpoint{0.624634in}{0.607526in}}%
\pgfpathlineto{\pgfqpoint{0.624938in}{0.604516in}}%
\pgfpathlineto{\pgfqpoint{0.627066in}{0.600476in}}%
\pgfpathclose%
\pgfusepath{fill}%
\end{pgfscope}%
\begin{pgfscope}%
\pgfpathrectangle{\pgfqpoint{0.100000in}{0.100000in}}{\pgfqpoint{3.420221in}{2.189500in}}%
\pgfusepath{clip}%
\pgfsetbuttcap%
\pgfsetmiterjoin%
\definecolor{currentfill}{rgb}{0.000000,0.768627,0.615686}%
\pgfsetfillcolor{currentfill}%
\pgfsetlinewidth{0.000000pt}%
\definecolor{currentstroke}{rgb}{0.000000,0.000000,0.000000}%
\pgfsetstrokecolor{currentstroke}%
\pgfsetstrokeopacity{0.000000}%
\pgfsetdash{}{0pt}%
\pgfpathmoveto{\pgfqpoint{0.673769in}{0.528170in}}%
\pgfpathlineto{\pgfqpoint{0.673457in}{0.529875in}}%
\pgfpathlineto{\pgfqpoint{0.671000in}{0.531833in}}%
\pgfpathlineto{\pgfqpoint{0.670049in}{0.534709in}}%
\pgfpathlineto{\pgfqpoint{0.669893in}{0.537381in}}%
\pgfpathlineto{\pgfqpoint{0.668017in}{0.540834in}}%
\pgfpathlineto{\pgfqpoint{0.668352in}{0.545186in}}%
\pgfpathlineto{\pgfqpoint{0.669509in}{0.547494in}}%
\pgfpathlineto{\pgfqpoint{0.671084in}{0.547123in}}%
\pgfpathlineto{\pgfqpoint{0.674624in}{0.542077in}}%
\pgfpathlineto{\pgfqpoint{0.679045in}{0.542629in}}%
\pgfpathlineto{\pgfqpoint{0.681866in}{0.541903in}}%
\pgfpathlineto{\pgfqpoint{0.682326in}{0.540443in}}%
\pgfpathlineto{\pgfqpoint{0.684760in}{0.539321in}}%
\pgfpathlineto{\pgfqpoint{0.684583in}{0.537068in}}%
\pgfpathlineto{\pgfqpoint{0.687066in}{0.534794in}}%
\pgfpathlineto{\pgfqpoint{0.686968in}{0.531828in}}%
\pgfpathlineto{\pgfqpoint{0.684960in}{0.529517in}}%
\pgfpathlineto{\pgfqpoint{0.682970in}{0.528324in}}%
\pgfpathlineto{\pgfqpoint{0.682268in}{0.523587in}}%
\pgfpathlineto{\pgfqpoint{0.680014in}{0.524284in}}%
\pgfpathlineto{\pgfqpoint{0.677172in}{0.526734in}}%
\pgfpathclose%
\pgfusepath{fill}%
\end{pgfscope}%
\begin{pgfscope}%
\pgfpathrectangle{\pgfqpoint{0.100000in}{0.100000in}}{\pgfqpoint{3.420221in}{2.189500in}}%
\pgfusepath{clip}%
\pgfsetbuttcap%
\pgfsetmiterjoin%
\definecolor{currentfill}{rgb}{0.000000,0.768627,0.615686}%
\pgfsetfillcolor{currentfill}%
\pgfsetlinewidth{0.000000pt}%
\definecolor{currentstroke}{rgb}{0.000000,0.000000,0.000000}%
\pgfsetstrokecolor{currentstroke}%
\pgfsetstrokeopacity{0.000000}%
\pgfsetdash{}{0pt}%
\pgfpathmoveto{\pgfqpoint{0.853197in}{0.465288in}}%
\pgfpathlineto{\pgfqpoint{0.838124in}{0.445034in}}%
\pgfpathlineto{\pgfqpoint{0.837771in}{0.447122in}}%
\pgfpathlineto{\pgfqpoint{0.833858in}{0.450233in}}%
\pgfpathlineto{\pgfqpoint{0.830174in}{0.445346in}}%
\pgfpathlineto{\pgfqpoint{0.829091in}{0.442571in}}%
\pgfpathlineto{\pgfqpoint{0.824868in}{0.436985in}}%
\pgfpathlineto{\pgfqpoint{0.814246in}{0.444980in}}%
\pgfpathlineto{\pgfqpoint{0.797891in}{0.457757in}}%
\pgfpathlineto{\pgfqpoint{0.785188in}{0.467882in}}%
\pgfpathlineto{\pgfqpoint{0.786829in}{0.469921in}}%
\pgfpathlineto{\pgfqpoint{0.783177in}{0.472910in}}%
\pgfpathlineto{\pgfqpoint{0.781520in}{0.470904in}}%
\pgfpathlineto{\pgfqpoint{0.779586in}{0.472327in}}%
\pgfpathlineto{\pgfqpoint{0.778072in}{0.470436in}}%
\pgfpathlineto{\pgfqpoint{0.776181in}{0.472015in}}%
\pgfpathlineto{\pgfqpoint{0.777662in}{0.473881in}}%
\pgfpathlineto{\pgfqpoint{0.767347in}{0.482344in}}%
\pgfpathlineto{\pgfqpoint{0.765790in}{0.480531in}}%
\pgfpathlineto{\pgfqpoint{0.764377in}{0.481657in}}%
\pgfpathlineto{\pgfqpoint{0.762737in}{0.479737in}}%
\pgfpathlineto{\pgfqpoint{0.760979in}{0.481161in}}%
\pgfpathlineto{\pgfqpoint{0.758926in}{0.479765in}}%
\pgfpathlineto{\pgfqpoint{0.755809in}{0.476145in}}%
\pgfpathlineto{\pgfqpoint{0.753782in}{0.477906in}}%
\pgfpathlineto{\pgfqpoint{0.752217in}{0.476095in}}%
\pgfpathlineto{\pgfqpoint{0.750410in}{0.477654in}}%
\pgfpathlineto{\pgfqpoint{0.747320in}{0.474107in}}%
\pgfpathlineto{\pgfqpoint{0.745516in}{0.475618in}}%
\pgfpathlineto{\pgfqpoint{0.743293in}{0.474389in}}%
\pgfpathlineto{\pgfqpoint{0.740281in}{0.471011in}}%
\pgfpathlineto{\pgfqpoint{0.738553in}{0.472489in}}%
\pgfpathlineto{\pgfqpoint{0.735431in}{0.468920in}}%
\pgfpathlineto{\pgfqpoint{0.733007in}{0.470950in}}%
\pgfpathlineto{\pgfqpoint{0.729876in}{0.467393in}}%
\pgfpathlineto{\pgfqpoint{0.728070in}{0.468974in}}%
\pgfpathlineto{\pgfqpoint{0.724914in}{0.465385in}}%
\pgfpathlineto{\pgfqpoint{0.722723in}{0.464078in}}%
\pgfpathlineto{\pgfqpoint{0.720992in}{0.465660in}}%
\pgfpathlineto{\pgfqpoint{0.718816in}{0.464400in}}%
\pgfpathlineto{\pgfqpoint{0.715663in}{0.460844in}}%
\pgfpathlineto{\pgfqpoint{0.713896in}{0.462362in}}%
\pgfpathlineto{\pgfqpoint{0.712466in}{0.460737in}}%
\pgfpathlineto{\pgfqpoint{0.710745in}{0.462233in}}%
\pgfpathlineto{\pgfqpoint{0.707299in}{0.458956in}}%
\pgfpathlineto{\pgfqpoint{0.707125in}{0.458976in}}%
\pgfpathlineto{\pgfqpoint{0.704030in}{0.461752in}}%
\pgfpathlineto{\pgfqpoint{0.702788in}{0.459696in}}%
\pgfpathlineto{\pgfqpoint{0.701773in}{0.458460in}}%
\pgfpathlineto{\pgfqpoint{0.694997in}{0.460016in}}%
\pgfpathlineto{\pgfqpoint{0.695717in}{0.463248in}}%
\pgfpathlineto{\pgfqpoint{0.698836in}{0.464424in}}%
\pgfpathlineto{\pgfqpoint{0.702241in}{0.468496in}}%
\pgfpathlineto{\pgfqpoint{0.703489in}{0.470802in}}%
\pgfpathlineto{\pgfqpoint{0.703511in}{0.474039in}}%
\pgfpathlineto{\pgfqpoint{0.704718in}{0.476754in}}%
\pgfpathlineto{\pgfqpoint{0.705988in}{0.475957in}}%
\pgfpathlineto{\pgfqpoint{0.707649in}{0.478069in}}%
\pgfpathlineto{\pgfqpoint{0.711938in}{0.478182in}}%
\pgfpathlineto{\pgfqpoint{0.713213in}{0.480485in}}%
\pgfpathlineto{\pgfqpoint{0.715073in}{0.485819in}}%
\pgfpathlineto{\pgfqpoint{0.715607in}{0.488587in}}%
\pgfpathlineto{\pgfqpoint{0.716935in}{0.490754in}}%
\pgfpathlineto{\pgfqpoint{0.717083in}{0.493745in}}%
\pgfpathlineto{\pgfqpoint{0.719919in}{0.495380in}}%
\pgfpathlineto{\pgfqpoint{0.719828in}{0.499079in}}%
\pgfpathlineto{\pgfqpoint{0.717998in}{0.500077in}}%
\pgfpathlineto{\pgfqpoint{0.714722in}{0.496668in}}%
\pgfpathlineto{\pgfqpoint{0.712149in}{0.496446in}}%
\pgfpathlineto{\pgfqpoint{0.711057in}{0.497978in}}%
\pgfpathlineto{\pgfqpoint{0.705663in}{0.498580in}}%
\pgfpathlineto{\pgfqpoint{0.700341in}{0.501244in}}%
\pgfpathlineto{\pgfqpoint{0.697953in}{0.503075in}}%
\pgfpathlineto{\pgfqpoint{0.695255in}{0.506491in}}%
\pgfpathlineto{\pgfqpoint{0.694589in}{0.508417in}}%
\pgfpathlineto{\pgfqpoint{0.695709in}{0.510792in}}%
\pgfpathlineto{\pgfqpoint{0.697360in}{0.511562in}}%
\pgfpathlineto{\pgfqpoint{0.696883in}{0.514348in}}%
\pgfpathlineto{\pgfqpoint{0.697143in}{0.517467in}}%
\pgfpathlineto{\pgfqpoint{0.698664in}{0.517940in}}%
\pgfpathlineto{\pgfqpoint{0.697338in}{0.522036in}}%
\pgfpathlineto{\pgfqpoint{0.698618in}{0.522851in}}%
\pgfpathlineto{\pgfqpoint{0.696112in}{0.524870in}}%
\pgfpathlineto{\pgfqpoint{0.695735in}{0.527707in}}%
\pgfpathlineto{\pgfqpoint{0.696437in}{0.530093in}}%
\pgfpathlineto{\pgfqpoint{0.698224in}{0.529692in}}%
\pgfpathlineto{\pgfqpoint{0.698848in}{0.531407in}}%
\pgfpathlineto{\pgfqpoint{0.696245in}{0.532149in}}%
\pgfpathlineto{\pgfqpoint{0.695323in}{0.535035in}}%
\pgfpathlineto{\pgfqpoint{0.699004in}{0.534156in}}%
\pgfpathlineto{\pgfqpoint{0.703491in}{0.536035in}}%
\pgfpathlineto{\pgfqpoint{0.706507in}{0.536095in}}%
\pgfpathlineto{\pgfqpoint{0.705030in}{0.539614in}}%
\pgfpathlineto{\pgfqpoint{0.705785in}{0.541366in}}%
\pgfpathlineto{\pgfqpoint{0.708490in}{0.542301in}}%
\pgfpathlineto{\pgfqpoint{0.708739in}{0.546394in}}%
\pgfpathlineto{\pgfqpoint{0.706387in}{0.544895in}}%
\pgfpathlineto{\pgfqpoint{0.705179in}{0.548496in}}%
\pgfpathlineto{\pgfqpoint{0.708212in}{0.551443in}}%
\pgfpathlineto{\pgfqpoint{0.706289in}{0.554953in}}%
\pgfpathlineto{\pgfqpoint{0.707691in}{0.556569in}}%
\pgfpathlineto{\pgfqpoint{0.709627in}{0.556281in}}%
\pgfpathlineto{\pgfqpoint{0.710671in}{0.557981in}}%
\pgfpathlineto{\pgfqpoint{0.710055in}{0.559847in}}%
\pgfpathlineto{\pgfqpoint{0.706902in}{0.560154in}}%
\pgfpathlineto{\pgfqpoint{0.708362in}{0.562827in}}%
\pgfpathlineto{\pgfqpoint{0.712656in}{0.563505in}}%
\pgfpathlineto{\pgfqpoint{0.713648in}{0.566517in}}%
\pgfpathlineto{\pgfqpoint{0.716795in}{0.563675in}}%
\pgfpathlineto{\pgfqpoint{0.719345in}{0.564357in}}%
\pgfpathlineto{\pgfqpoint{0.720555in}{0.567095in}}%
\pgfpathlineto{\pgfqpoint{0.722760in}{0.568398in}}%
\pgfpathlineto{\pgfqpoint{0.732981in}{0.570917in}}%
\pgfpathlineto{\pgfqpoint{0.737697in}{0.571326in}}%
\pgfpathlineto{\pgfqpoint{0.739278in}{0.569534in}}%
\pgfpathlineto{\pgfqpoint{0.741938in}{0.570607in}}%
\pgfpathlineto{\pgfqpoint{0.742678in}{0.573177in}}%
\pgfpathlineto{\pgfqpoint{0.748020in}{0.577501in}}%
\pgfpathlineto{\pgfqpoint{0.753134in}{0.579429in}}%
\pgfpathlineto{\pgfqpoint{0.758618in}{0.579129in}}%
\pgfpathlineto{\pgfqpoint{0.761925in}{0.576298in}}%
\pgfpathlineto{\pgfqpoint{0.762872in}{0.571995in}}%
\pgfpathlineto{\pgfqpoint{0.763248in}{0.567706in}}%
\pgfpathlineto{\pgfqpoint{0.764505in}{0.565189in}}%
\pgfpathlineto{\pgfqpoint{0.767136in}{0.563919in}}%
\pgfpathlineto{\pgfqpoint{0.771711in}{0.565133in}}%
\pgfpathlineto{\pgfqpoint{0.774936in}{0.564678in}}%
\pgfpathlineto{\pgfqpoint{0.776855in}{0.562805in}}%
\pgfpathlineto{\pgfqpoint{0.775234in}{0.560965in}}%
\pgfpathlineto{\pgfqpoint{0.777022in}{0.559333in}}%
\pgfpathlineto{\pgfqpoint{0.775372in}{0.557388in}}%
\pgfpathlineto{\pgfqpoint{0.777020in}{0.555870in}}%
\pgfpathlineto{\pgfqpoint{0.775547in}{0.554024in}}%
\pgfpathlineto{\pgfqpoint{0.786440in}{0.544258in}}%
\pgfpathlineto{\pgfqpoint{0.785309in}{0.544738in}}%
\pgfpathlineto{\pgfqpoint{0.782922in}{0.543042in}}%
\pgfpathlineto{\pgfqpoint{0.776368in}{0.535802in}}%
\pgfpathlineto{\pgfqpoint{0.775908in}{0.536162in}}%
\pgfpathlineto{\pgfqpoint{0.769426in}{0.528919in}}%
\pgfpathlineto{\pgfqpoint{0.770766in}{0.527717in}}%
\pgfpathlineto{\pgfqpoint{0.766043in}{0.522351in}}%
\pgfpathlineto{\pgfqpoint{0.775602in}{0.513824in}}%
\pgfpathlineto{\pgfqpoint{0.782292in}{0.508068in}}%
\pgfpathlineto{\pgfqpoint{0.784286in}{0.510390in}}%
\pgfpathlineto{\pgfqpoint{0.791930in}{0.503912in}}%
\pgfpathlineto{\pgfqpoint{0.793488in}{0.505755in}}%
\pgfpathlineto{\pgfqpoint{0.807964in}{0.493605in}}%
\pgfpathlineto{\pgfqpoint{0.806220in}{0.491468in}}%
\pgfpathlineto{\pgfqpoint{0.814712in}{0.484379in}}%
\pgfpathlineto{\pgfqpoint{0.827581in}{0.474284in}}%
\pgfpathlineto{\pgfqpoint{0.838061in}{0.466246in}}%
\pgfpathlineto{\pgfqpoint{0.840208in}{0.464836in}}%
\pgfpathlineto{\pgfqpoint{0.842415in}{0.467677in}}%
\pgfpathlineto{\pgfqpoint{0.846189in}{0.464685in}}%
\pgfpathlineto{\pgfqpoint{0.847674in}{0.466581in}}%
\pgfpathlineto{\pgfqpoint{0.851055in}{0.463937in}}%
\pgfpathclose%
\pgfusepath{fill}%
\end{pgfscope}%
\begin{pgfscope}%
\pgfpathrectangle{\pgfqpoint{0.100000in}{0.100000in}}{\pgfqpoint{3.420221in}{2.189500in}}%
\pgfusepath{clip}%
\pgfsetbuttcap%
\pgfsetmiterjoin%
\definecolor{currentfill}{rgb}{0.000000,0.470588,0.764706}%
\pgfsetfillcolor{currentfill}%
\pgfsetlinewidth{0.000000pt}%
\definecolor{currentstroke}{rgb}{0.000000,0.000000,0.000000}%
\pgfsetstrokecolor{currentstroke}%
\pgfsetstrokeopacity{0.000000}%
\pgfsetdash{}{0pt}%
\pgfpathmoveto{\pgfqpoint{0.964504in}{1.853012in}}%
\pgfpathlineto{\pgfqpoint{0.961805in}{1.850104in}}%
\pgfpathlineto{\pgfqpoint{0.957265in}{1.830065in}}%
\pgfpathlineto{\pgfqpoint{0.943209in}{1.820987in}}%
\pgfpathlineto{\pgfqpoint{0.935266in}{1.825471in}}%
\pgfpathlineto{\pgfqpoint{0.931012in}{1.818300in}}%
\pgfpathlineto{\pgfqpoint{0.924401in}{1.816821in}}%
\pgfpathlineto{\pgfqpoint{0.922296in}{1.811132in}}%
\pgfpathlineto{\pgfqpoint{0.925085in}{1.807234in}}%
\pgfpathlineto{\pgfqpoint{0.923675in}{1.801792in}}%
\pgfpathlineto{\pgfqpoint{0.919091in}{1.796990in}}%
\pgfpathlineto{\pgfqpoint{0.908912in}{1.791859in}}%
\pgfpathlineto{\pgfqpoint{0.905496in}{1.792053in}}%
\pgfpathlineto{\pgfqpoint{0.874935in}{1.799287in}}%
\pgfpathlineto{\pgfqpoint{0.872917in}{1.792904in}}%
\pgfpathlineto{\pgfqpoint{0.867721in}{1.794406in}}%
\pgfpathlineto{\pgfqpoint{0.870732in}{1.807061in}}%
\pgfpathlineto{\pgfqpoint{0.873865in}{1.806326in}}%
\pgfpathlineto{\pgfqpoint{0.876888in}{1.819108in}}%
\pgfpathlineto{\pgfqpoint{0.868236in}{1.816925in}}%
\pgfpathlineto{\pgfqpoint{0.862604in}{1.818252in}}%
\pgfpathlineto{\pgfqpoint{0.857179in}{1.823185in}}%
\pgfpathlineto{\pgfqpoint{0.859309in}{1.832288in}}%
\pgfpathlineto{\pgfqpoint{0.855169in}{1.837022in}}%
\pgfpathlineto{\pgfqpoint{0.858260in}{1.849885in}}%
\pgfpathlineto{\pgfqpoint{0.844033in}{1.853501in}}%
\pgfpathlineto{\pgfqpoint{0.849145in}{1.860003in}}%
\pgfpathlineto{\pgfqpoint{0.849982in}{1.866155in}}%
\pgfpathlineto{\pgfqpoint{0.854302in}{1.869474in}}%
\pgfpathlineto{\pgfqpoint{0.858234in}{1.873406in}}%
\pgfpathlineto{\pgfqpoint{0.862672in}{1.882079in}}%
\pgfpathlineto{\pgfqpoint{0.880422in}{1.877811in}}%
\pgfpathlineto{\pgfqpoint{0.880987in}{1.866921in}}%
\pgfpathlineto{\pgfqpoint{0.887868in}{1.863610in}}%
\pgfpathlineto{\pgfqpoint{0.898770in}{1.867828in}}%
\pgfpathclose%
\pgfusepath{fill}%
\end{pgfscope}%
\begin{pgfscope}%
\pgfpathrectangle{\pgfqpoint{0.100000in}{0.100000in}}{\pgfqpoint{3.420221in}{2.189500in}}%
\pgfusepath{clip}%
\pgfsetbuttcap%
\pgfsetmiterjoin%
\definecolor{currentfill}{rgb}{0.000000,0.600000,0.700000}%
\pgfsetfillcolor{currentfill}%
\pgfsetlinewidth{0.000000pt}%
\definecolor{currentstroke}{rgb}{0.000000,0.000000,0.000000}%
\pgfsetstrokecolor{currentstroke}%
\pgfsetstrokeopacity{0.000000}%
\pgfsetdash{}{0pt}%
\pgfpathmoveto{\pgfqpoint{2.353219in}{1.135116in}}%
\pgfpathlineto{\pgfqpoint{2.348986in}{1.137396in}}%
\pgfpathlineto{\pgfqpoint{2.343643in}{1.135256in}}%
\pgfpathlineto{\pgfqpoint{2.337775in}{1.139697in}}%
\pgfpathlineto{\pgfqpoint{2.328783in}{1.156504in}}%
\pgfpathlineto{\pgfqpoint{2.334084in}{1.164694in}}%
\pgfpathlineto{\pgfqpoint{2.327323in}{1.177357in}}%
\pgfpathlineto{\pgfqpoint{2.329751in}{1.179956in}}%
\pgfpathlineto{\pgfqpoint{2.327215in}{1.186638in}}%
\pgfpathlineto{\pgfqpoint{2.323023in}{1.188139in}}%
\pgfpathlineto{\pgfqpoint{2.313178in}{1.197594in}}%
\pgfpathlineto{\pgfqpoint{2.306859in}{1.201604in}}%
\pgfpathlineto{\pgfqpoint{2.300424in}{1.199656in}}%
\pgfpathlineto{\pgfqpoint{2.297320in}{1.205526in}}%
\pgfpathlineto{\pgfqpoint{2.289220in}{1.212423in}}%
\pgfpathlineto{\pgfqpoint{2.284647in}{1.214102in}}%
\pgfpathlineto{\pgfqpoint{2.294503in}{1.218208in}}%
\pgfpathlineto{\pgfqpoint{2.294122in}{1.224753in}}%
\pgfpathlineto{\pgfqpoint{2.313716in}{1.225575in}}%
\pgfpathlineto{\pgfqpoint{2.346403in}{1.227100in}}%
\pgfpathlineto{\pgfqpoint{2.346690in}{1.220515in}}%
\pgfpathlineto{\pgfqpoint{2.372849in}{1.222302in}}%
\pgfpathlineto{\pgfqpoint{2.373872in}{1.205956in}}%
\pgfpathlineto{\pgfqpoint{2.375361in}{1.182955in}}%
\pgfpathlineto{\pgfqpoint{2.376595in}{1.163344in}}%
\pgfpathlineto{\pgfqpoint{2.364834in}{1.162429in}}%
\pgfpathlineto{\pgfqpoint{2.364225in}{1.154194in}}%
\pgfpathlineto{\pgfqpoint{2.358229in}{1.152641in}}%
\pgfpathlineto{\pgfqpoint{2.349972in}{1.139300in}}%
\pgfpathclose%
\pgfusepath{fill}%
\end{pgfscope}%
\begin{pgfscope}%
\pgfpathrectangle{\pgfqpoint{0.100000in}{0.100000in}}{\pgfqpoint{3.420221in}{2.189500in}}%
\pgfusepath{clip}%
\pgfsetbuttcap%
\pgfsetmiterjoin%
\definecolor{currentfill}{rgb}{0.000000,0.290196,0.854902}%
\pgfsetfillcolor{currentfill}%
\pgfsetlinewidth{0.000000pt}%
\definecolor{currentstroke}{rgb}{0.000000,0.000000,0.000000}%
\pgfsetstrokecolor{currentstroke}%
\pgfsetstrokeopacity{0.000000}%
\pgfsetdash{}{0pt}%
\pgfpathmoveto{\pgfqpoint{3.081378in}{1.441459in}}%
\pgfpathlineto{\pgfqpoint{3.075868in}{1.440278in}}%
\pgfpathlineto{\pgfqpoint{3.075488in}{1.440206in}}%
\pgfpathlineto{\pgfqpoint{3.064079in}{1.452421in}}%
\pgfpathlineto{\pgfqpoint{3.058931in}{1.454019in}}%
\pgfpathlineto{\pgfqpoint{3.053852in}{1.460951in}}%
\pgfpathlineto{\pgfqpoint{3.046396in}{1.460355in}}%
\pgfpathlineto{\pgfqpoint{3.042913in}{1.465774in}}%
\pgfpathlineto{\pgfqpoint{3.049723in}{1.471198in}}%
\pgfpathlineto{\pgfqpoint{3.039465in}{1.490395in}}%
\pgfpathlineto{\pgfqpoint{3.046213in}{1.497933in}}%
\pgfpathlineto{\pgfqpoint{3.035408in}{1.503588in}}%
\pgfpathlineto{\pgfqpoint{3.051509in}{1.515893in}}%
\pgfpathlineto{\pgfqpoint{3.055162in}{1.518667in}}%
\pgfpathlineto{\pgfqpoint{3.055161in}{1.524928in}}%
\pgfpathlineto{\pgfqpoint{3.058402in}{1.530673in}}%
\pgfpathlineto{\pgfqpoint{3.070632in}{1.530438in}}%
\pgfpathlineto{\pgfqpoint{3.086700in}{1.520262in}}%
\pgfpathlineto{\pgfqpoint{3.080185in}{1.514408in}}%
\pgfpathlineto{\pgfqpoint{3.103958in}{1.501884in}}%
\pgfpathlineto{\pgfqpoint{3.097884in}{1.484823in}}%
\pgfpathlineto{\pgfqpoint{3.086169in}{1.472092in}}%
\pgfpathlineto{\pgfqpoint{3.087727in}{1.467166in}}%
\pgfpathlineto{\pgfqpoint{3.085737in}{1.460138in}}%
\pgfpathlineto{\pgfqpoint{3.086567in}{1.452593in}}%
\pgfpathlineto{\pgfqpoint{3.084659in}{1.445088in}}%
\pgfpathclose%
\pgfusepath{fill}%
\end{pgfscope}%
\begin{pgfscope}%
\pgfpathrectangle{\pgfqpoint{0.100000in}{0.100000in}}{\pgfqpoint{3.420221in}{2.189500in}}%
\pgfusepath{clip}%
\pgfsetbuttcap%
\pgfsetmiterjoin%
\definecolor{currentfill}{rgb}{0.000000,0.525490,0.737255}%
\pgfsetfillcolor{currentfill}%
\pgfsetlinewidth{0.000000pt}%
\definecolor{currentstroke}{rgb}{0.000000,0.000000,0.000000}%
\pgfsetstrokecolor{currentstroke}%
\pgfsetstrokeopacity{0.000000}%
\pgfsetdash{}{0pt}%
\pgfpathmoveto{\pgfqpoint{2.741849in}{0.693692in}}%
\pgfpathlineto{\pgfqpoint{2.733239in}{0.693158in}}%
\pgfpathlineto{\pgfqpoint{2.730176in}{0.721700in}}%
\pgfpathlineto{\pgfqpoint{2.712929in}{0.719879in}}%
\pgfpathlineto{\pgfqpoint{2.712617in}{0.722552in}}%
\pgfpathlineto{\pgfqpoint{2.710553in}{0.741927in}}%
\pgfpathlineto{\pgfqpoint{2.732707in}{0.744358in}}%
\pgfpathlineto{\pgfqpoint{2.730768in}{0.762189in}}%
\pgfpathlineto{\pgfqpoint{2.725869in}{0.765672in}}%
\pgfpathlineto{\pgfqpoint{2.720848in}{0.765117in}}%
\pgfpathlineto{\pgfqpoint{2.718799in}{0.778604in}}%
\pgfpathlineto{\pgfqpoint{2.730894in}{0.780163in}}%
\pgfpathlineto{\pgfqpoint{2.730447in}{0.783909in}}%
\pgfpathlineto{\pgfqpoint{2.738914in}{0.784466in}}%
\pgfpathlineto{\pgfqpoint{2.741977in}{0.774235in}}%
\pgfpathlineto{\pgfqpoint{2.739686in}{0.773053in}}%
\pgfpathlineto{\pgfqpoint{2.740229in}{0.763533in}}%
\pgfpathlineto{\pgfqpoint{2.745533in}{0.762949in}}%
\pgfpathlineto{\pgfqpoint{2.749007in}{0.757925in}}%
\pgfpathlineto{\pgfqpoint{2.763856in}{0.759120in}}%
\pgfpathlineto{\pgfqpoint{2.765122in}{0.753714in}}%
\pgfpathlineto{\pgfqpoint{2.772985in}{0.745298in}}%
\pgfpathlineto{\pgfqpoint{2.773048in}{0.738346in}}%
\pgfpathlineto{\pgfqpoint{2.777889in}{0.739034in}}%
\pgfpathlineto{\pgfqpoint{2.781026in}{0.715406in}}%
\pgfpathlineto{\pgfqpoint{2.789415in}{0.713911in}}%
\pgfpathlineto{\pgfqpoint{2.795571in}{0.706923in}}%
\pgfpathlineto{\pgfqpoint{2.806281in}{0.706921in}}%
\pgfpathlineto{\pgfqpoint{2.808819in}{0.697974in}}%
\pgfpathclose%
\pgfusepath{fill}%
\end{pgfscope}%
\begin{pgfscope}%
\pgfpathrectangle{\pgfqpoint{0.100000in}{0.100000in}}{\pgfqpoint{3.420221in}{2.189500in}}%
\pgfusepath{clip}%
\pgfsetbuttcap%
\pgfsetmiterjoin%
\definecolor{currentfill}{rgb}{0.000000,0.709804,0.645098}%
\pgfsetfillcolor{currentfill}%
\pgfsetlinewidth{0.000000pt}%
\definecolor{currentstroke}{rgb}{0.000000,0.000000,0.000000}%
\pgfsetstrokecolor{currentstroke}%
\pgfsetstrokeopacity{0.000000}%
\pgfsetdash{}{0pt}%
\pgfpathmoveto{\pgfqpoint{2.295734in}{0.854983in}}%
\pgfpathlineto{\pgfqpoint{2.288188in}{0.851922in}}%
\pgfpathlineto{\pgfqpoint{2.287992in}{0.856600in}}%
\pgfpathlineto{\pgfqpoint{2.286275in}{0.892826in}}%
\pgfpathlineto{\pgfqpoint{2.303078in}{0.893537in}}%
\pgfpathlineto{\pgfqpoint{2.295669in}{0.888124in}}%
\pgfpathlineto{\pgfqpoint{2.306289in}{0.887103in}}%
\pgfpathlineto{\pgfqpoint{2.305955in}{0.893690in}}%
\pgfpathlineto{\pgfqpoint{2.319013in}{0.895500in}}%
\pgfpathlineto{\pgfqpoint{2.318714in}{0.901066in}}%
\pgfpathlineto{\pgfqpoint{2.328628in}{0.899322in}}%
\pgfpathlineto{\pgfqpoint{2.344952in}{0.900296in}}%
\pgfpathlineto{\pgfqpoint{2.345624in}{0.889359in}}%
\pgfpathlineto{\pgfqpoint{2.346858in}{0.869680in}}%
\pgfpathlineto{\pgfqpoint{2.350267in}{0.867886in}}%
\pgfpathlineto{\pgfqpoint{2.350959in}{0.856816in}}%
\pgfpathlineto{\pgfqpoint{2.335393in}{0.855854in}}%
\pgfpathlineto{\pgfqpoint{2.332692in}{0.850341in}}%
\pgfpathlineto{\pgfqpoint{2.305477in}{0.857795in}}%
\pgfpathlineto{\pgfqpoint{2.298857in}{0.857910in}}%
\pgfpathclose%
\pgfusepath{fill}%
\end{pgfscope}%
\begin{pgfscope}%
\pgfpathrectangle{\pgfqpoint{0.100000in}{0.100000in}}{\pgfqpoint{3.420221in}{2.189500in}}%
\pgfusepath{clip}%
\pgfsetbuttcap%
\pgfsetmiterjoin%
\definecolor{currentfill}{rgb}{0.000000,0.211765,0.894118}%
\pgfsetfillcolor{currentfill}%
\pgfsetlinewidth{0.000000pt}%
\definecolor{currentstroke}{rgb}{0.000000,0.000000,0.000000}%
\pgfsetstrokecolor{currentstroke}%
\pgfsetstrokeopacity{0.000000}%
\pgfsetdash{}{0pt}%
\pgfpathmoveto{\pgfqpoint{1.818986in}{0.758709in}}%
\pgfpathlineto{\pgfqpoint{1.814180in}{0.769894in}}%
\pgfpathlineto{\pgfqpoint{1.807050in}{0.791120in}}%
\pgfpathlineto{\pgfqpoint{1.781217in}{0.792157in}}%
\pgfpathlineto{\pgfqpoint{1.774802in}{0.792566in}}%
\pgfpathlineto{\pgfqpoint{1.776144in}{0.825640in}}%
\pgfpathlineto{\pgfqpoint{1.785512in}{0.825124in}}%
\pgfpathlineto{\pgfqpoint{1.785712in}{0.829309in}}%
\pgfpathlineto{\pgfqpoint{1.817571in}{0.827696in}}%
\pgfpathlineto{\pgfqpoint{1.818835in}{0.860248in}}%
\pgfpathlineto{\pgfqpoint{1.815104in}{0.862873in}}%
\pgfpathlineto{\pgfqpoint{1.816212in}{0.894668in}}%
\pgfpathlineto{\pgfqpoint{1.823111in}{0.891395in}}%
\pgfpathlineto{\pgfqpoint{1.835570in}{0.901713in}}%
\pgfpathlineto{\pgfqpoint{1.842257in}{0.894605in}}%
\pgfpathlineto{\pgfqpoint{1.847076in}{0.895655in}}%
\pgfpathlineto{\pgfqpoint{1.845903in}{0.859439in}}%
\pgfpathlineto{\pgfqpoint{1.852435in}{0.859193in}}%
\pgfpathlineto{\pgfqpoint{1.850583in}{0.825894in}}%
\pgfpathlineto{\pgfqpoint{1.873699in}{0.824910in}}%
\pgfpathlineto{\pgfqpoint{1.872583in}{0.792069in}}%
\pgfpathlineto{\pgfqpoint{1.869506in}{0.792199in}}%
\pgfpathlineto{\pgfqpoint{1.869055in}{0.770787in}}%
\pgfpathlineto{\pgfqpoint{1.844029in}{0.765094in}}%
\pgfpathlineto{\pgfqpoint{1.838979in}{0.762779in}}%
\pgfpathlineto{\pgfqpoint{1.834203in}{0.767279in}}%
\pgfpathclose%
\pgfusepath{fill}%
\end{pgfscope}%
\begin{pgfscope}%
\pgfpathrectangle{\pgfqpoint{0.100000in}{0.100000in}}{\pgfqpoint{3.420221in}{2.189500in}}%
\pgfusepath{clip}%
\pgfsetbuttcap%
\pgfsetmiterjoin%
\definecolor{currentfill}{rgb}{0.000000,0.309804,0.845098}%
\pgfsetfillcolor{currentfill}%
\pgfsetlinewidth{0.000000pt}%
\definecolor{currentstroke}{rgb}{0.000000,0.000000,0.000000}%
\pgfsetstrokecolor{currentstroke}%
\pgfsetstrokeopacity{0.000000}%
\pgfsetdash{}{0pt}%
\pgfpathmoveto{\pgfqpoint{2.563217in}{1.002398in}}%
\pgfpathlineto{\pgfqpoint{2.535887in}{0.999923in}}%
\pgfpathlineto{\pgfqpoint{2.534490in}{1.010056in}}%
\pgfpathlineto{\pgfqpoint{2.531256in}{1.010309in}}%
\pgfpathlineto{\pgfqpoint{2.529196in}{1.017712in}}%
\pgfpathlineto{\pgfqpoint{2.515642in}{1.026211in}}%
\pgfpathlineto{\pgfqpoint{2.511575in}{1.031138in}}%
\pgfpathlineto{\pgfqpoint{2.510993in}{1.049975in}}%
\pgfpathlineto{\pgfqpoint{2.518497in}{1.050860in}}%
\pgfpathlineto{\pgfqpoint{2.521360in}{1.047953in}}%
\pgfpathlineto{\pgfqpoint{2.534883in}{1.052457in}}%
\pgfpathlineto{\pgfqpoint{2.546871in}{1.050242in}}%
\pgfpathlineto{\pgfqpoint{2.550889in}{1.051856in}}%
\pgfpathlineto{\pgfqpoint{2.553746in}{1.045623in}}%
\pgfpathlineto{\pgfqpoint{2.558469in}{1.042357in}}%
\pgfpathlineto{\pgfqpoint{2.559539in}{1.029238in}}%
\pgfpathlineto{\pgfqpoint{2.558005in}{1.024559in}}%
\pgfpathlineto{\pgfqpoint{2.561499in}{1.019741in}}%
\pgfpathclose%
\pgfusepath{fill}%
\end{pgfscope}%
\begin{pgfscope}%
\pgfpathrectangle{\pgfqpoint{0.100000in}{0.100000in}}{\pgfqpoint{3.420221in}{2.189500in}}%
\pgfusepath{clip}%
\pgfsetbuttcap%
\pgfsetmiterjoin%
\definecolor{currentfill}{rgb}{0.000000,0.596078,0.701961}%
\pgfsetfillcolor{currentfill}%
\pgfsetlinewidth{0.000000pt}%
\definecolor{currentstroke}{rgb}{0.000000,0.000000,0.000000}%
\pgfsetstrokecolor{currentstroke}%
\pgfsetstrokeopacity{0.000000}%
\pgfsetdash{}{0pt}%
\pgfpathmoveto{\pgfqpoint{2.422371in}{0.834679in}}%
\pgfpathlineto{\pgfqpoint{2.393064in}{0.832373in}}%
\pgfpathlineto{\pgfqpoint{2.391200in}{0.859454in}}%
\pgfpathlineto{\pgfqpoint{2.400237in}{0.860115in}}%
\pgfpathlineto{\pgfqpoint{2.398952in}{0.876570in}}%
\pgfpathlineto{\pgfqpoint{2.407531in}{0.879522in}}%
\pgfpathlineto{\pgfqpoint{2.410713in}{0.884733in}}%
\pgfpathlineto{\pgfqpoint{2.408029in}{0.888072in}}%
\pgfpathlineto{\pgfqpoint{2.416792in}{0.891401in}}%
\pgfpathlineto{\pgfqpoint{2.419301in}{0.896087in}}%
\pgfpathlineto{\pgfqpoint{2.424054in}{0.896398in}}%
\pgfpathlineto{\pgfqpoint{2.423624in}{0.880447in}}%
\pgfpathlineto{\pgfqpoint{2.450820in}{0.881836in}}%
\pgfpathlineto{\pgfqpoint{2.453219in}{0.853992in}}%
\pgfpathlineto{\pgfqpoint{2.446608in}{0.850037in}}%
\pgfpathlineto{\pgfqpoint{2.441674in}{0.844700in}}%
\pgfpathlineto{\pgfqpoint{2.436340in}{0.844264in}}%
\pgfpathlineto{\pgfqpoint{2.433094in}{0.840521in}}%
\pgfpathlineto{\pgfqpoint{2.422499in}{0.839381in}}%
\pgfpathclose%
\pgfusepath{fill}%
\end{pgfscope}%
\begin{pgfscope}%
\pgfpathrectangle{\pgfqpoint{0.100000in}{0.100000in}}{\pgfqpoint{3.420221in}{2.189500in}}%
\pgfusepath{clip}%
\pgfsetbuttcap%
\pgfsetmiterjoin%
\definecolor{currentfill}{rgb}{0.000000,0.607843,0.696078}%
\pgfsetfillcolor{currentfill}%
\pgfsetlinewidth{0.000000pt}%
\definecolor{currentstroke}{rgb}{0.000000,0.000000,0.000000}%
\pgfsetstrokecolor{currentstroke}%
\pgfsetstrokeopacity{0.000000}%
\pgfsetdash{}{0pt}%
\pgfpathmoveto{\pgfqpoint{3.052654in}{1.583281in}}%
\pgfpathlineto{\pgfqpoint{3.045578in}{1.608356in}}%
\pgfpathlineto{\pgfqpoint{3.023112in}{1.603787in}}%
\pgfpathlineto{\pgfqpoint{3.019928in}{1.624702in}}%
\pgfpathlineto{\pgfqpoint{3.026059in}{1.628774in}}%
\pgfpathlineto{\pgfqpoint{3.035367in}{1.629051in}}%
\pgfpathlineto{\pgfqpoint{3.033485in}{1.637159in}}%
\pgfpathlineto{\pgfqpoint{3.054463in}{1.642220in}}%
\pgfpathlineto{\pgfqpoint{3.058008in}{1.631007in}}%
\pgfpathlineto{\pgfqpoint{3.069339in}{1.632598in}}%
\pgfpathlineto{\pgfqpoint{3.070245in}{1.628722in}}%
\pgfpathlineto{\pgfqpoint{3.082132in}{1.631262in}}%
\pgfpathlineto{\pgfqpoint{3.084411in}{1.619978in}}%
\pgfpathlineto{\pgfqpoint{3.088473in}{1.617631in}}%
\pgfpathlineto{\pgfqpoint{3.081732in}{1.616141in}}%
\pgfpathlineto{\pgfqpoint{3.088600in}{1.590182in}}%
\pgfpathclose%
\pgfusepath{fill}%
\end{pgfscope}%
\begin{pgfscope}%
\pgfpathrectangle{\pgfqpoint{0.100000in}{0.100000in}}{\pgfqpoint{3.420221in}{2.189500in}}%
\pgfusepath{clip}%
\pgfsetbuttcap%
\pgfsetmiterjoin%
\definecolor{currentfill}{rgb}{0.000000,0.360784,0.819608}%
\pgfsetfillcolor{currentfill}%
\pgfsetlinewidth{0.000000pt}%
\definecolor{currentstroke}{rgb}{0.000000,0.000000,0.000000}%
\pgfsetstrokecolor{currentstroke}%
\pgfsetstrokeopacity{0.000000}%
\pgfsetdash{}{0pt}%
\pgfpathmoveto{\pgfqpoint{1.920651in}{0.851717in}}%
\pgfpathlineto{\pgfqpoint{1.915101in}{0.855254in}}%
\pgfpathlineto{\pgfqpoint{1.886420in}{0.856348in}}%
\pgfpathlineto{\pgfqpoint{1.879987in}{0.857296in}}%
\pgfpathlineto{\pgfqpoint{1.880797in}{0.897379in}}%
\pgfpathlineto{\pgfqpoint{1.884513in}{0.891192in}}%
\pgfpathlineto{\pgfqpoint{1.895114in}{0.888244in}}%
\pgfpathlineto{\pgfqpoint{1.900911in}{0.900986in}}%
\pgfpathlineto{\pgfqpoint{1.914535in}{0.912409in}}%
\pgfpathlineto{\pgfqpoint{1.940418in}{0.912029in}}%
\pgfpathlineto{\pgfqpoint{1.943715in}{0.909646in}}%
\pgfpathlineto{\pgfqpoint{1.943592in}{0.896311in}}%
\pgfpathlineto{\pgfqpoint{1.949613in}{0.890679in}}%
\pgfpathlineto{\pgfqpoint{1.954519in}{0.890543in}}%
\pgfpathlineto{\pgfqpoint{1.948662in}{0.887163in}}%
\pgfpathlineto{\pgfqpoint{1.948151in}{0.855746in}}%
\pgfpathclose%
\pgfusepath{fill}%
\end{pgfscope}%
\begin{pgfscope}%
\pgfpathrectangle{\pgfqpoint{0.100000in}{0.100000in}}{\pgfqpoint{3.420221in}{2.189500in}}%
\pgfusepath{clip}%
\pgfsetbuttcap%
\pgfsetmiterjoin%
\definecolor{currentfill}{rgb}{0.000000,0.513725,0.743137}%
\pgfsetfillcolor{currentfill}%
\pgfsetlinewidth{0.000000pt}%
\definecolor{currentstroke}{rgb}{0.000000,0.000000,0.000000}%
\pgfsetstrokecolor{currentstroke}%
\pgfsetstrokeopacity{0.000000}%
\pgfsetdash{}{0pt}%
\pgfpathmoveto{\pgfqpoint{2.433344in}{0.642630in}}%
\pgfpathlineto{\pgfqpoint{2.428014in}{0.688958in}}%
\pgfpathlineto{\pgfqpoint{2.424080in}{0.721840in}}%
\pgfpathlineto{\pgfqpoint{2.421694in}{0.741515in}}%
\pgfpathlineto{\pgfqpoint{2.445788in}{0.743409in}}%
\pgfpathlineto{\pgfqpoint{2.444370in}{0.747958in}}%
\pgfpathlineto{\pgfqpoint{2.439186in}{0.751924in}}%
\pgfpathlineto{\pgfqpoint{2.445064in}{0.765379in}}%
\pgfpathlineto{\pgfqpoint{2.470948in}{0.767503in}}%
\pgfpathlineto{\pgfqpoint{2.473734in}{0.768799in}}%
\pgfpathlineto{\pgfqpoint{2.472878in}{0.778335in}}%
\pgfpathlineto{\pgfqpoint{2.479375in}{0.778902in}}%
\pgfpathlineto{\pgfqpoint{2.478525in}{0.788069in}}%
\pgfpathlineto{\pgfqpoint{2.481717in}{0.789097in}}%
\pgfpathlineto{\pgfqpoint{2.491110in}{0.784824in}}%
\pgfpathlineto{\pgfqpoint{2.501822in}{0.774336in}}%
\pgfpathlineto{\pgfqpoint{2.519094in}{0.775887in}}%
\pgfpathlineto{\pgfqpoint{2.520557in}{0.759546in}}%
\pgfpathlineto{\pgfqpoint{2.482601in}{0.756178in}}%
\pgfpathlineto{\pgfqpoint{2.482446in}{0.746207in}}%
\pgfpathlineto{\pgfqpoint{2.479202in}{0.745886in}}%
\pgfpathlineto{\pgfqpoint{2.480437in}{0.726286in}}%
\pgfpathlineto{\pgfqpoint{2.474548in}{0.724742in}}%
\pgfpathlineto{\pgfqpoint{2.470973in}{0.720707in}}%
\pgfpathlineto{\pgfqpoint{2.472149in}{0.715370in}}%
\pgfpathlineto{\pgfqpoint{2.478837in}{0.711538in}}%
\pgfpathlineto{\pgfqpoint{2.481449in}{0.693010in}}%
\pgfpathlineto{\pgfqpoint{2.479928in}{0.682899in}}%
\pgfpathlineto{\pgfqpoint{2.487498in}{0.674027in}}%
\pgfpathlineto{\pgfqpoint{2.495889in}{0.669731in}}%
\pgfpathlineto{\pgfqpoint{2.497042in}{0.665279in}}%
\pgfpathlineto{\pgfqpoint{2.494249in}{0.658342in}}%
\pgfpathlineto{\pgfqpoint{2.499982in}{0.651786in}}%
\pgfpathlineto{\pgfqpoint{2.495486in}{0.648141in}}%
\pgfpathlineto{\pgfqpoint{2.491118in}{0.639334in}}%
\pgfpathlineto{\pgfqpoint{2.482152in}{0.636253in}}%
\pgfpathlineto{\pgfqpoint{2.475668in}{0.638155in}}%
\pgfpathlineto{\pgfqpoint{2.464855in}{0.647395in}}%
\pgfpathlineto{\pgfqpoint{2.462595in}{0.652401in}}%
\pgfpathlineto{\pgfqpoint{2.464359in}{0.656726in}}%
\pgfpathlineto{\pgfqpoint{2.462068in}{0.665186in}}%
\pgfpathlineto{\pgfqpoint{2.458335in}{0.667500in}}%
\pgfpathlineto{\pgfqpoint{2.453777in}{0.664102in}}%
\pgfpathlineto{\pgfqpoint{2.451827in}{0.653736in}}%
\pgfpathlineto{\pgfqpoint{2.452155in}{0.643540in}}%
\pgfpathlineto{\pgfqpoint{2.450695in}{0.639614in}}%
\pgfpathlineto{\pgfqpoint{2.436342in}{0.644396in}}%
\pgfpathclose%
\pgfusepath{fill}%
\end{pgfscope}%
\begin{pgfscope}%
\pgfpathrectangle{\pgfqpoint{0.100000in}{0.100000in}}{\pgfqpoint{3.420221in}{2.189500in}}%
\pgfusepath{clip}%
\pgfsetbuttcap%
\pgfsetmiterjoin%
\definecolor{currentfill}{rgb}{0.000000,0.392157,0.803922}%
\pgfsetfillcolor{currentfill}%
\pgfsetlinewidth{0.000000pt}%
\definecolor{currentstroke}{rgb}{0.000000,0.000000,0.000000}%
\pgfsetstrokecolor{currentstroke}%
\pgfsetstrokeopacity{0.000000}%
\pgfsetdash{}{0pt}%
\pgfpathmoveto{\pgfqpoint{1.370729in}{1.014003in}}%
\pgfpathlineto{\pgfqpoint{1.367087in}{0.981553in}}%
\pgfpathlineto{\pgfqpoint{1.365700in}{0.981682in}}%
\pgfpathlineto{\pgfqpoint{1.363463in}{0.962245in}}%
\pgfpathlineto{\pgfqpoint{1.362711in}{0.955774in}}%
\pgfpathlineto{\pgfqpoint{1.325022in}{0.960197in}}%
\pgfpathlineto{\pgfqpoint{1.294687in}{0.963920in}}%
\pgfpathlineto{\pgfqpoint{1.296314in}{0.977358in}}%
\pgfpathlineto{\pgfqpoint{1.275604in}{0.986945in}}%
\pgfpathlineto{\pgfqpoint{1.269062in}{0.991454in}}%
\pgfpathlineto{\pgfqpoint{1.217413in}{0.998383in}}%
\pgfpathlineto{\pgfqpoint{1.177571in}{1.004184in}}%
\pgfpathlineto{\pgfqpoint{1.136566in}{1.010472in}}%
\pgfpathlineto{\pgfqpoint{1.140985in}{1.038709in}}%
\pgfpathlineto{\pgfqpoint{1.176061in}{1.033301in}}%
\pgfpathlineto{\pgfqpoint{1.179959in}{1.059125in}}%
\pgfpathlineto{\pgfqpoint{1.250210in}{1.048975in}}%
\pgfpathlineto{\pgfqpoint{1.257348in}{1.100532in}}%
\pgfpathlineto{\pgfqpoint{1.238228in}{1.103295in}}%
\pgfpathlineto{\pgfqpoint{1.240643in}{1.119582in}}%
\pgfpathlineto{\pgfqpoint{1.284940in}{1.113328in}}%
\pgfpathlineto{\pgfqpoint{1.282794in}{1.097145in}}%
\pgfpathlineto{\pgfqpoint{1.321229in}{1.092420in}}%
\pgfpathlineto{\pgfqpoint{1.311663in}{1.090483in}}%
\pgfpathlineto{\pgfqpoint{1.310426in}{1.080623in}}%
\pgfpathlineto{\pgfqpoint{1.318889in}{1.074107in}}%
\pgfpathlineto{\pgfqpoint{1.312520in}{1.020777in}}%
\pgfpathlineto{\pgfqpoint{1.344853in}{1.016966in}}%
\pgfpathclose%
\pgfusepath{fill}%
\end{pgfscope}%
\begin{pgfscope}%
\pgfpathrectangle{\pgfqpoint{0.100000in}{0.100000in}}{\pgfqpoint{3.420221in}{2.189500in}}%
\pgfusepath{clip}%
\pgfsetbuttcap%
\pgfsetmiterjoin%
\definecolor{currentfill}{rgb}{0.000000,0.427451,0.786275}%
\pgfsetfillcolor{currentfill}%
\pgfsetlinewidth{0.000000pt}%
\definecolor{currentstroke}{rgb}{0.000000,0.000000,0.000000}%
\pgfsetstrokecolor{currentstroke}%
\pgfsetstrokeopacity{0.000000}%
\pgfsetdash{}{0pt}%
\pgfpathmoveto{\pgfqpoint{1.894784in}{0.621395in}}%
\pgfpathlineto{\pgfqpoint{1.888049in}{0.623285in}}%
\pgfpathlineto{\pgfqpoint{1.880110in}{0.632064in}}%
\pgfpathlineto{\pgfqpoint{1.877588in}{0.638837in}}%
\pgfpathlineto{\pgfqpoint{1.874072in}{0.641534in}}%
\pgfpathlineto{\pgfqpoint{1.896622in}{0.654234in}}%
\pgfpathlineto{\pgfqpoint{1.895025in}{0.661414in}}%
\pgfpathlineto{\pgfqpoint{1.888100in}{0.669322in}}%
\pgfpathlineto{\pgfqpoint{1.886405in}{0.679461in}}%
\pgfpathlineto{\pgfqpoint{1.883685in}{0.682871in}}%
\pgfpathlineto{\pgfqpoint{1.898873in}{0.691284in}}%
\pgfpathlineto{\pgfqpoint{1.916975in}{0.701293in}}%
\pgfpathlineto{\pgfqpoint{1.916335in}{0.695976in}}%
\pgfpathlineto{\pgfqpoint{1.921621in}{0.672218in}}%
\pgfpathlineto{\pgfqpoint{1.926193in}{0.660768in}}%
\pgfpathlineto{\pgfqpoint{1.946054in}{0.663658in}}%
\pgfpathlineto{\pgfqpoint{1.948024in}{0.645939in}}%
\pgfpathlineto{\pgfqpoint{1.949456in}{0.616792in}}%
\pgfpathlineto{\pgfqpoint{1.930808in}{0.615701in}}%
\pgfpathlineto{\pgfqpoint{1.930568in}{0.619250in}}%
\pgfpathlineto{\pgfqpoint{1.925117in}{0.628424in}}%
\pgfpathlineto{\pgfqpoint{1.916315in}{0.627683in}}%
\pgfpathlineto{\pgfqpoint{1.909089in}{0.624664in}}%
\pgfpathclose%
\pgfusepath{fill}%
\end{pgfscope}%
\begin{pgfscope}%
\pgfpathrectangle{\pgfqpoint{0.100000in}{0.100000in}}{\pgfqpoint{3.420221in}{2.189500in}}%
\pgfusepath{clip}%
\pgfsetbuttcap%
\pgfsetmiterjoin%
\definecolor{currentfill}{rgb}{0.000000,0.745098,0.627451}%
\pgfsetfillcolor{currentfill}%
\pgfsetlinewidth{0.000000pt}%
\definecolor{currentstroke}{rgb}{0.000000,0.000000,0.000000}%
\pgfsetstrokecolor{currentstroke}%
\pgfsetstrokeopacity{0.000000}%
\pgfsetdash{}{0pt}%
\pgfpathmoveto{\pgfqpoint{2.840274in}{1.222647in}}%
\pgfpathlineto{\pgfqpoint{2.824652in}{1.232270in}}%
\pgfpathlineto{\pgfqpoint{2.818904in}{1.230411in}}%
\pgfpathlineto{\pgfqpoint{2.818035in}{1.225335in}}%
\pgfpathlineto{\pgfqpoint{2.812107in}{1.229451in}}%
\pgfpathlineto{\pgfqpoint{2.803271in}{1.241265in}}%
\pgfpathlineto{\pgfqpoint{2.797754in}{1.243552in}}%
\pgfpathlineto{\pgfqpoint{2.792892in}{1.253169in}}%
\pgfpathlineto{\pgfqpoint{2.795978in}{1.258228in}}%
\pgfpathlineto{\pgfqpoint{2.803576in}{1.257907in}}%
\pgfpathlineto{\pgfqpoint{2.804192in}{1.272571in}}%
\pgfpathlineto{\pgfqpoint{2.808953in}{1.281024in}}%
\pgfpathlineto{\pgfqpoint{2.818862in}{1.277446in}}%
\pgfpathlineto{\pgfqpoint{2.823750in}{1.280225in}}%
\pgfpathlineto{\pgfqpoint{2.831160in}{1.272330in}}%
\pgfpathlineto{\pgfqpoint{2.841330in}{1.272091in}}%
\pgfpathlineto{\pgfqpoint{2.855121in}{1.288580in}}%
\pgfpathlineto{\pgfqpoint{2.860060in}{1.286231in}}%
\pgfpathlineto{\pgfqpoint{2.861211in}{1.277953in}}%
\pgfpathlineto{\pgfqpoint{2.867768in}{1.273882in}}%
\pgfpathlineto{\pgfqpoint{2.873466in}{1.274019in}}%
\pgfpathlineto{\pgfqpoint{2.885302in}{1.278077in}}%
\pgfpathlineto{\pgfqpoint{2.883849in}{1.272609in}}%
\pgfpathlineto{\pgfqpoint{2.873323in}{1.256808in}}%
\pgfpathlineto{\pgfqpoint{2.870218in}{1.247340in}}%
\pgfpathlineto{\pgfqpoint{2.861206in}{1.245276in}}%
\pgfpathlineto{\pgfqpoint{2.848802in}{1.246585in}}%
\pgfpathclose%
\pgfusepath{fill}%
\end{pgfscope}%
\begin{pgfscope}%
\pgfpathrectangle{\pgfqpoint{0.100000in}{0.100000in}}{\pgfqpoint{3.420221in}{2.189500in}}%
\pgfusepath{clip}%
\pgfsetbuttcap%
\pgfsetmiterjoin%
\definecolor{currentfill}{rgb}{0.000000,0.509804,0.745098}%
\pgfsetfillcolor{currentfill}%
\pgfsetlinewidth{0.000000pt}%
\definecolor{currentstroke}{rgb}{0.000000,0.000000,0.000000}%
\pgfsetstrokecolor{currentstroke}%
\pgfsetstrokeopacity{0.000000}%
\pgfsetdash{}{0pt}%
\pgfpathmoveto{\pgfqpoint{3.134996in}{1.641143in}}%
\pgfpathlineto{\pgfqpoint{3.102490in}{1.609314in}}%
\pgfpathlineto{\pgfqpoint{3.095767in}{1.609352in}}%
\pgfpathlineto{\pgfqpoint{3.093833in}{1.614161in}}%
\pgfpathlineto{\pgfqpoint{3.088473in}{1.617631in}}%
\pgfpathlineto{\pgfqpoint{3.084411in}{1.619978in}}%
\pgfpathlineto{\pgfqpoint{3.082132in}{1.631262in}}%
\pgfpathlineto{\pgfqpoint{3.070245in}{1.628722in}}%
\pgfpathlineto{\pgfqpoint{3.069339in}{1.632598in}}%
\pgfpathlineto{\pgfqpoint{3.058008in}{1.631007in}}%
\pgfpathlineto{\pgfqpoint{3.054463in}{1.642220in}}%
\pgfpathlineto{\pgfqpoint{3.048237in}{1.664540in}}%
\pgfpathlineto{\pgfqpoint{3.079934in}{1.672972in}}%
\pgfpathlineto{\pgfqpoint{3.082685in}{1.682223in}}%
\pgfpathlineto{\pgfqpoint{3.087730in}{1.687327in}}%
\pgfpathlineto{\pgfqpoint{3.099530in}{1.683548in}}%
\pgfpathlineto{\pgfqpoint{3.099821in}{1.689257in}}%
\pgfpathlineto{\pgfqpoint{3.106600in}{1.688090in}}%
\pgfpathlineto{\pgfqpoint{3.113333in}{1.687026in}}%
\pgfpathlineto{\pgfqpoint{3.113614in}{1.681019in}}%
\pgfpathlineto{\pgfqpoint{3.117667in}{1.672393in}}%
\pgfpathlineto{\pgfqpoint{3.115117in}{1.663432in}}%
\pgfpathlineto{\pgfqpoint{3.121675in}{1.657735in}}%
\pgfpathlineto{\pgfqpoint{3.133241in}{1.654587in}}%
\pgfpathlineto{\pgfqpoint{3.129817in}{1.642433in}}%
\pgfpathclose%
\pgfusepath{fill}%
\end{pgfscope}%
\begin{pgfscope}%
\pgfpathrectangle{\pgfqpoint{0.100000in}{0.100000in}}{\pgfqpoint{3.420221in}{2.189500in}}%
\pgfusepath{clip}%
\pgfsetbuttcap%
\pgfsetmiterjoin%
\definecolor{currentfill}{rgb}{0.000000,0.254902,0.872549}%
\pgfsetfillcolor{currentfill}%
\pgfsetlinewidth{0.000000pt}%
\definecolor{currentstroke}{rgb}{0.000000,0.000000,0.000000}%
\pgfsetstrokecolor{currentstroke}%
\pgfsetstrokeopacity{0.000000}%
\pgfsetdash{}{0pt}%
\pgfpathmoveto{\pgfqpoint{1.822533in}{1.904442in}}%
\pgfpathlineto{\pgfqpoint{1.783199in}{1.906173in}}%
\pgfpathlineto{\pgfqpoint{1.784444in}{1.932325in}}%
\pgfpathlineto{\pgfqpoint{1.783464in}{1.945588in}}%
\pgfpathlineto{\pgfqpoint{1.822718in}{1.943858in}}%
\pgfpathlineto{\pgfqpoint{1.823502in}{1.930624in}}%
\pgfpathclose%
\pgfusepath{fill}%
\end{pgfscope}%
\begin{pgfscope}%
\pgfpathrectangle{\pgfqpoint{0.100000in}{0.100000in}}{\pgfqpoint{3.420221in}{2.189500in}}%
\pgfusepath{clip}%
\pgfsetbuttcap%
\pgfsetmiterjoin%
\definecolor{currentfill}{rgb}{0.000000,0.282353,0.858824}%
\pgfsetfillcolor{currentfill}%
\pgfsetlinewidth{0.000000pt}%
\definecolor{currentstroke}{rgb}{0.000000,0.000000,0.000000}%
\pgfsetstrokecolor{currentstroke}%
\pgfsetstrokeopacity{0.000000}%
\pgfsetdash{}{0pt}%
\pgfpathmoveto{\pgfqpoint{1.837680in}{0.966588in}}%
\pgfpathlineto{\pgfqpoint{1.837270in}{0.953538in}}%
\pgfpathlineto{\pgfqpoint{1.811293in}{0.954343in}}%
\pgfpathlineto{\pgfqpoint{1.811707in}{0.967396in}}%
\pgfpathlineto{\pgfqpoint{1.779045in}{0.968604in}}%
\pgfpathlineto{\pgfqpoint{1.779694in}{0.986799in}}%
\pgfpathlineto{\pgfqpoint{1.780925in}{1.020880in}}%
\pgfpathlineto{\pgfqpoint{1.799835in}{1.020152in}}%
\pgfpathlineto{\pgfqpoint{1.799778in}{1.007059in}}%
\pgfpathlineto{\pgfqpoint{1.817251in}{1.006476in}}%
\pgfpathlineto{\pgfqpoint{1.828097in}{1.002826in}}%
\pgfpathlineto{\pgfqpoint{1.838519in}{1.002686in}}%
\pgfpathclose%
\pgfusepath{fill}%
\end{pgfscope}%
\begin{pgfscope}%
\pgfpathrectangle{\pgfqpoint{0.100000in}{0.100000in}}{\pgfqpoint{3.420221in}{2.189500in}}%
\pgfusepath{clip}%
\pgfsetbuttcap%
\pgfsetmiterjoin%
\definecolor{currentfill}{rgb}{0.000000,0.400000,0.800000}%
\pgfsetfillcolor{currentfill}%
\pgfsetlinewidth{0.000000pt}%
\definecolor{currentstroke}{rgb}{0.000000,0.000000,0.000000}%
\pgfsetstrokecolor{currentstroke}%
\pgfsetstrokeopacity{0.000000}%
\pgfsetdash{}{0pt}%
\pgfpathmoveto{\pgfqpoint{2.058209in}{1.612731in}}%
\pgfpathlineto{\pgfqpoint{2.058667in}{1.568271in}}%
\pgfpathlineto{\pgfqpoint{2.006833in}{1.568114in}}%
\pgfpathlineto{\pgfqpoint{1.980763in}{1.568157in}}%
\pgfpathlineto{\pgfqpoint{1.980877in}{1.594089in}}%
\pgfpathlineto{\pgfqpoint{1.980970in}{1.612505in}}%
\pgfpathlineto{\pgfqpoint{2.006748in}{1.612479in}}%
\pgfpathlineto{\pgfqpoint{2.006822in}{1.594042in}}%
\pgfpathlineto{\pgfqpoint{2.032579in}{1.594129in}}%
\pgfpathlineto{\pgfqpoint{2.032478in}{1.612580in}}%
\pgfpathclose%
\pgfusepath{fill}%
\end{pgfscope}%
\begin{pgfscope}%
\pgfpathrectangle{\pgfqpoint{0.100000in}{0.100000in}}{\pgfqpoint{3.420221in}{2.189500in}}%
\pgfusepath{clip}%
\pgfsetbuttcap%
\pgfsetmiterjoin%
\definecolor{currentfill}{rgb}{0.000000,0.647059,0.676471}%
\pgfsetfillcolor{currentfill}%
\pgfsetlinewidth{0.000000pt}%
\definecolor{currentstroke}{rgb}{0.000000,0.000000,0.000000}%
\pgfsetstrokecolor{currentstroke}%
\pgfsetstrokeopacity{0.000000}%
\pgfsetdash{}{0pt}%
\pgfpathmoveto{\pgfqpoint{2.682851in}{1.245988in}}%
\pgfpathlineto{\pgfqpoint{2.683114in}{1.239099in}}%
\pgfpathlineto{\pgfqpoint{2.678833in}{1.229736in}}%
\pgfpathlineto{\pgfqpoint{2.673255in}{1.229010in}}%
\pgfpathlineto{\pgfqpoint{2.665224in}{1.233614in}}%
\pgfpathlineto{\pgfqpoint{2.652964in}{1.234692in}}%
\pgfpathlineto{\pgfqpoint{2.644253in}{1.247307in}}%
\pgfpathlineto{\pgfqpoint{2.656168in}{1.262476in}}%
\pgfpathlineto{\pgfqpoint{2.663005in}{1.262200in}}%
\pgfpathlineto{\pgfqpoint{2.669625in}{1.255899in}}%
\pgfpathclose%
\pgfusepath{fill}%
\end{pgfscope}%
\begin{pgfscope}%
\pgfpathrectangle{\pgfqpoint{0.100000in}{0.100000in}}{\pgfqpoint{3.420221in}{2.189500in}}%
\pgfusepath{clip}%
\pgfsetbuttcap%
\pgfsetmiterjoin%
\definecolor{currentfill}{rgb}{0.000000,0.478431,0.760784}%
\pgfsetfillcolor{currentfill}%
\pgfsetlinewidth{0.000000pt}%
\definecolor{currentstroke}{rgb}{0.000000,0.000000,0.000000}%
\pgfsetstrokecolor{currentstroke}%
\pgfsetstrokeopacity{0.000000}%
\pgfsetdash{}{0pt}%
\pgfpathmoveto{\pgfqpoint{2.456869in}{1.457965in}}%
\pgfpathlineto{\pgfqpoint{2.458575in}{1.438561in}}%
\pgfpathlineto{\pgfqpoint{2.455922in}{1.432713in}}%
\pgfpathlineto{\pgfqpoint{2.449521in}{1.432129in}}%
\pgfpathlineto{\pgfqpoint{2.450247in}{1.424587in}}%
\pgfpathlineto{\pgfqpoint{2.426052in}{1.422567in}}%
\pgfpathlineto{\pgfqpoint{2.422561in}{1.464491in}}%
\pgfpathlineto{\pgfqpoint{2.420189in}{1.495153in}}%
\pgfpathlineto{\pgfqpoint{2.426744in}{1.490786in}}%
\pgfpathlineto{\pgfqpoint{2.438005in}{1.490368in}}%
\pgfpathlineto{\pgfqpoint{2.453168in}{1.498102in}}%
\pgfpathclose%
\pgfusepath{fill}%
\end{pgfscope}%
\begin{pgfscope}%
\pgfpathrectangle{\pgfqpoint{0.100000in}{0.100000in}}{\pgfqpoint{3.420221in}{2.189500in}}%
\pgfusepath{clip}%
\pgfsetbuttcap%
\pgfsetmiterjoin%
\definecolor{currentfill}{rgb}{0.000000,0.349020,0.825490}%
\pgfsetfillcolor{currentfill}%
\pgfsetlinewidth{0.000000pt}%
\definecolor{currentstroke}{rgb}{0.000000,0.000000,0.000000}%
\pgfsetstrokecolor{currentstroke}%
\pgfsetstrokeopacity{0.000000}%
\pgfsetdash{}{0pt}%
\pgfpathmoveto{\pgfqpoint{1.753701in}{0.859974in}}%
\pgfpathlineto{\pgfqpoint{1.787184in}{0.858514in}}%
\pgfpathlineto{\pgfqpoint{1.787337in}{0.863827in}}%
\pgfpathlineto{\pgfqpoint{1.815104in}{0.862873in}}%
\pgfpathlineto{\pgfqpoint{1.818835in}{0.860248in}}%
\pgfpathlineto{\pgfqpoint{1.817571in}{0.827696in}}%
\pgfpathlineto{\pgfqpoint{1.785712in}{0.829309in}}%
\pgfpathlineto{\pgfqpoint{1.785512in}{0.825124in}}%
\pgfpathlineto{\pgfqpoint{1.776144in}{0.825640in}}%
\pgfpathlineto{\pgfqpoint{1.719594in}{0.828347in}}%
\pgfpathlineto{\pgfqpoint{1.721156in}{0.861597in}}%
\pgfpathclose%
\pgfusepath{fill}%
\end{pgfscope}%
\begin{pgfscope}%
\pgfpathrectangle{\pgfqpoint{0.100000in}{0.100000in}}{\pgfqpoint{3.420221in}{2.189500in}}%
\pgfusepath{clip}%
\pgfsetbuttcap%
\pgfsetmiterjoin%
\definecolor{currentfill}{rgb}{0.000000,0.556863,0.721569}%
\pgfsetfillcolor{currentfill}%
\pgfsetlinewidth{0.000000pt}%
\definecolor{currentstroke}{rgb}{0.000000,0.000000,0.000000}%
\pgfsetstrokecolor{currentstroke}%
\pgfsetstrokeopacity{0.000000}%
\pgfsetdash{}{0pt}%
\pgfpathmoveto{\pgfqpoint{2.192229in}{1.303820in}}%
\pgfpathlineto{\pgfqpoint{2.157906in}{1.303342in}}%
\pgfpathlineto{\pgfqpoint{2.157893in}{1.295761in}}%
\pgfpathlineto{\pgfqpoint{2.151174in}{1.295727in}}%
\pgfpathlineto{\pgfqpoint{2.134964in}{1.300751in}}%
\pgfpathlineto{\pgfqpoint{2.135292in}{1.328973in}}%
\pgfpathlineto{\pgfqpoint{2.125573in}{1.328878in}}%
\pgfpathlineto{\pgfqpoint{2.125576in}{1.354158in}}%
\pgfpathlineto{\pgfqpoint{2.154375in}{1.354917in}}%
\pgfpathlineto{\pgfqpoint{2.154585in}{1.348627in}}%
\pgfpathlineto{\pgfqpoint{2.177103in}{1.349044in}}%
\pgfpathlineto{\pgfqpoint{2.183555in}{1.349175in}}%
\pgfpathlineto{\pgfqpoint{2.183941in}{1.327471in}}%
\pgfpathlineto{\pgfqpoint{2.191541in}{1.327674in}}%
\pgfpathclose%
\pgfusepath{fill}%
\end{pgfscope}%
\begin{pgfscope}%
\pgfpathrectangle{\pgfqpoint{0.100000in}{0.100000in}}{\pgfqpoint{3.420221in}{2.189500in}}%
\pgfusepath{clip}%
\pgfsetbuttcap%
\pgfsetmiterjoin%
\definecolor{currentfill}{rgb}{0.000000,0.658824,0.670588}%
\pgfsetfillcolor{currentfill}%
\pgfsetlinewidth{0.000000pt}%
\definecolor{currentstroke}{rgb}{0.000000,0.000000,0.000000}%
\pgfsetstrokecolor{currentstroke}%
\pgfsetstrokeopacity{0.000000}%
\pgfsetdash{}{0pt}%
\pgfpathmoveto{\pgfqpoint{1.941718in}{0.986762in}}%
\pgfpathlineto{\pgfqpoint{1.941973in}{0.997147in}}%
\pgfpathlineto{\pgfqpoint{1.942047in}{1.003672in}}%
\pgfpathlineto{\pgfqpoint{1.948516in}{1.003596in}}%
\pgfpathlineto{\pgfqpoint{1.951772in}{1.010118in}}%
\pgfpathlineto{\pgfqpoint{1.951897in}{1.016704in}}%
\pgfpathlineto{\pgfqpoint{1.958575in}{1.016653in}}%
\pgfpathlineto{\pgfqpoint{1.958670in}{1.029717in}}%
\pgfpathlineto{\pgfqpoint{1.955433in}{1.029731in}}%
\pgfpathlineto{\pgfqpoint{1.955518in}{1.039538in}}%
\pgfpathlineto{\pgfqpoint{1.962557in}{1.039509in}}%
\pgfpathlineto{\pgfqpoint{1.966270in}{1.032556in}}%
\pgfpathlineto{\pgfqpoint{1.973048in}{1.035501in}}%
\pgfpathlineto{\pgfqpoint{1.985943in}{1.036128in}}%
\pgfpathlineto{\pgfqpoint{1.985185in}{1.047448in}}%
\pgfpathlineto{\pgfqpoint{1.989501in}{1.055735in}}%
\pgfpathlineto{\pgfqpoint{1.994943in}{1.055704in}}%
\pgfpathlineto{\pgfqpoint{1.994936in}{1.062246in}}%
\pgfpathlineto{\pgfqpoint{2.001425in}{1.062254in}}%
\pgfpathlineto{\pgfqpoint{2.028586in}{1.062314in}}%
\pgfpathlineto{\pgfqpoint{2.032974in}{1.032132in}}%
\pgfpathlineto{\pgfqpoint{2.034273in}{1.023088in}}%
\pgfpathlineto{\pgfqpoint{1.994335in}{1.023025in}}%
\pgfpathlineto{\pgfqpoint{1.994020in}{1.014567in}}%
\pgfpathlineto{\pgfqpoint{1.997190in}{1.007485in}}%
\pgfpathlineto{\pgfqpoint{1.992268in}{1.002797in}}%
\pgfpathlineto{\pgfqpoint{1.988065in}{0.995129in}}%
\pgfpathlineto{\pgfqpoint{1.978482in}{0.999590in}}%
\pgfpathlineto{\pgfqpoint{1.967137in}{0.993598in}}%
\pgfpathlineto{\pgfqpoint{1.956564in}{0.989696in}}%
\pgfpathlineto{\pgfqpoint{1.948731in}{0.989322in}}%
\pgfpathclose%
\pgfusepath{fill}%
\end{pgfscope}%
\begin{pgfscope}%
\pgfpathrectangle{\pgfqpoint{0.100000in}{0.100000in}}{\pgfqpoint{3.420221in}{2.189500in}}%
\pgfusepath{clip}%
\pgfsetbuttcap%
\pgfsetmiterjoin%
\definecolor{currentfill}{rgb}{0.000000,0.482353,0.758824}%
\pgfsetfillcolor{currentfill}%
\pgfsetlinewidth{0.000000pt}%
\definecolor{currentstroke}{rgb}{0.000000,0.000000,0.000000}%
\pgfsetstrokecolor{currentstroke}%
\pgfsetstrokeopacity{0.000000}%
\pgfsetdash{}{0pt}%
\pgfpathmoveto{\pgfqpoint{1.984415in}{1.349932in}}%
\pgfpathlineto{\pgfqpoint{1.977930in}{1.354375in}}%
\pgfpathlineto{\pgfqpoint{1.978942in}{1.359621in}}%
\pgfpathlineto{\pgfqpoint{1.974679in}{1.363885in}}%
\pgfpathlineto{\pgfqpoint{1.974067in}{1.368632in}}%
\pgfpathlineto{\pgfqpoint{1.966804in}{1.373529in}}%
\pgfpathlineto{\pgfqpoint{1.958536in}{1.393937in}}%
\pgfpathlineto{\pgfqpoint{2.023056in}{1.392746in}}%
\pgfpathlineto{\pgfqpoint{2.025026in}{1.378930in}}%
\pgfpathlineto{\pgfqpoint{2.024686in}{1.359404in}}%
\pgfpathlineto{\pgfqpoint{1.999646in}{1.359667in}}%
\pgfpathlineto{\pgfqpoint{2.002489in}{1.343536in}}%
\pgfpathlineto{\pgfqpoint{1.997221in}{1.339556in}}%
\pgfpathlineto{\pgfqpoint{1.990503in}{1.342726in}}%
\pgfpathclose%
\pgfusepath{fill}%
\end{pgfscope}%
\begin{pgfscope}%
\pgfpathrectangle{\pgfqpoint{0.100000in}{0.100000in}}{\pgfqpoint{3.420221in}{2.189500in}}%
\pgfusepath{clip}%
\pgfsetbuttcap%
\pgfsetmiterjoin%
\definecolor{currentfill}{rgb}{0.000000,0.529412,0.735294}%
\pgfsetfillcolor{currentfill}%
\pgfsetlinewidth{0.000000pt}%
\definecolor{currentstroke}{rgb}{0.000000,0.000000,0.000000}%
\pgfsetstrokecolor{currentstroke}%
\pgfsetstrokeopacity{0.000000}%
\pgfsetdash{}{0pt}%
\pgfpathmoveto{\pgfqpoint{2.585662in}{1.048432in}}%
\pgfpathlineto{\pgfqpoint{2.577812in}{1.045009in}}%
\pgfpathlineto{\pgfqpoint{2.562173in}{1.041928in}}%
\pgfpathlineto{\pgfqpoint{2.558469in}{1.042357in}}%
\pgfpathlineto{\pgfqpoint{2.553746in}{1.045623in}}%
\pgfpathlineto{\pgfqpoint{2.550889in}{1.051856in}}%
\pgfpathlineto{\pgfqpoint{2.550346in}{1.057416in}}%
\pgfpathlineto{\pgfqpoint{2.553970in}{1.068182in}}%
\pgfpathlineto{\pgfqpoint{2.546852in}{1.074149in}}%
\pgfpathlineto{\pgfqpoint{2.543615in}{1.081234in}}%
\pgfpathlineto{\pgfqpoint{2.549041in}{1.085668in}}%
\pgfpathlineto{\pgfqpoint{2.554936in}{1.084956in}}%
\pgfpathlineto{\pgfqpoint{2.558094in}{1.088134in}}%
\pgfpathlineto{\pgfqpoint{2.560394in}{1.084352in}}%
\pgfpathlineto{\pgfqpoint{2.565748in}{1.084237in}}%
\pgfpathlineto{\pgfqpoint{2.568136in}{1.080556in}}%
\pgfpathlineto{\pgfqpoint{2.576534in}{1.086301in}}%
\pgfpathlineto{\pgfqpoint{2.585940in}{1.084568in}}%
\pgfpathlineto{\pgfqpoint{2.592049in}{1.080185in}}%
\pgfpathlineto{\pgfqpoint{2.595392in}{1.072845in}}%
\pgfpathlineto{\pgfqpoint{2.594053in}{1.060946in}}%
\pgfpathclose%
\pgfusepath{fill}%
\end{pgfscope}%
\begin{pgfscope}%
\pgfpathrectangle{\pgfqpoint{0.100000in}{0.100000in}}{\pgfqpoint{3.420221in}{2.189500in}}%
\pgfusepath{clip}%
\pgfsetbuttcap%
\pgfsetmiterjoin%
\definecolor{currentfill}{rgb}{0.000000,0.443137,0.778431}%
\pgfsetfillcolor{currentfill}%
\pgfsetlinewidth{0.000000pt}%
\definecolor{currentstroke}{rgb}{0.000000,0.000000,0.000000}%
\pgfsetstrokecolor{currentstroke}%
\pgfsetstrokeopacity{0.000000}%
\pgfsetdash{}{0pt}%
\pgfpathmoveto{\pgfqpoint{1.343500in}{2.076042in}}%
\pgfpathlineto{\pgfqpoint{1.402235in}{2.067844in}}%
\pgfpathlineto{\pgfqpoint{1.448618in}{2.061832in}}%
\pgfpathlineto{\pgfqpoint{1.445012in}{2.048848in}}%
\pgfpathlineto{\pgfqpoint{1.451504in}{2.048053in}}%
\pgfpathlineto{\pgfqpoint{1.449108in}{2.028377in}}%
\pgfpathlineto{\pgfqpoint{1.459939in}{2.027077in}}%
\pgfpathlineto{\pgfqpoint{1.456842in}{2.001118in}}%
\pgfpathlineto{\pgfqpoint{1.454876in}{2.001343in}}%
\pgfpathlineto{\pgfqpoint{1.452974in}{1.985616in}}%
\pgfpathlineto{\pgfqpoint{1.433700in}{1.989199in}}%
\pgfpathlineto{\pgfqpoint{1.425762in}{1.993121in}}%
\pgfpathlineto{\pgfqpoint{1.423514in}{1.985198in}}%
\pgfpathlineto{\pgfqpoint{1.419696in}{1.983495in}}%
\pgfpathlineto{\pgfqpoint{1.414677in}{1.975388in}}%
\pgfpathlineto{\pgfqpoint{1.404536in}{1.970220in}}%
\pgfpathlineto{\pgfqpoint{1.398708in}{1.971402in}}%
\pgfpathlineto{\pgfqpoint{1.396916in}{1.967157in}}%
\pgfpathlineto{\pgfqpoint{1.378733in}{1.968423in}}%
\pgfpathlineto{\pgfqpoint{1.370589in}{1.971852in}}%
\pgfpathlineto{\pgfqpoint{1.368086in}{1.966871in}}%
\pgfpathlineto{\pgfqpoint{1.358860in}{1.970238in}}%
\pgfpathlineto{\pgfqpoint{1.353212in}{1.965014in}}%
\pgfpathlineto{\pgfqpoint{1.346121in}{1.961824in}}%
\pgfpathlineto{\pgfqpoint{1.343039in}{1.957270in}}%
\pgfpathlineto{\pgfqpoint{1.343297in}{1.966187in}}%
\pgfpathlineto{\pgfqpoint{1.336774in}{1.969487in}}%
\pgfpathlineto{\pgfqpoint{1.321989in}{1.969961in}}%
\pgfpathlineto{\pgfqpoint{1.317972in}{1.973322in}}%
\pgfpathlineto{\pgfqpoint{1.311200in}{1.974108in}}%
\pgfpathlineto{\pgfqpoint{1.301793in}{1.978174in}}%
\pgfpathlineto{\pgfqpoint{1.296549in}{1.985918in}}%
\pgfpathlineto{\pgfqpoint{1.298867in}{2.000012in}}%
\pgfpathlineto{\pgfqpoint{1.312034in}{1.997856in}}%
\pgfpathlineto{\pgfqpoint{1.313718in}{2.002992in}}%
\pgfpathlineto{\pgfqpoint{1.322296in}{2.000591in}}%
\pgfpathlineto{\pgfqpoint{1.325017in}{2.018824in}}%
\pgfpathlineto{\pgfqpoint{1.328665in}{2.035591in}}%
\pgfpathlineto{\pgfqpoint{1.331671in}{2.033412in}}%
\pgfpathlineto{\pgfqpoint{1.334643in}{2.044198in}}%
\pgfpathlineto{\pgfqpoint{1.336558in}{2.057276in}}%
\pgfpathlineto{\pgfqpoint{1.339524in}{2.056832in}}%
\pgfpathclose%
\pgfusepath{fill}%
\end{pgfscope}%
\begin{pgfscope}%
\pgfpathrectangle{\pgfqpoint{0.100000in}{0.100000in}}{\pgfqpoint{3.420221in}{2.189500in}}%
\pgfusepath{clip}%
\pgfsetbuttcap%
\pgfsetmiterjoin%
\definecolor{currentfill}{rgb}{0.000000,0.603922,0.698039}%
\pgfsetfillcolor{currentfill}%
\pgfsetlinewidth{0.000000pt}%
\definecolor{currentstroke}{rgb}{0.000000,0.000000,0.000000}%
\pgfsetstrokecolor{currentstroke}%
\pgfsetstrokeopacity{0.000000}%
\pgfsetdash{}{0pt}%
\pgfpathmoveto{\pgfqpoint{2.991825in}{0.492145in}}%
\pgfpathlineto{\pgfqpoint{2.961260in}{0.487369in}}%
\pgfpathlineto{\pgfqpoint{2.960195in}{0.493731in}}%
\pgfpathlineto{\pgfqpoint{2.953850in}{0.492665in}}%
\pgfpathlineto{\pgfqpoint{2.946201in}{0.545912in}}%
\pgfpathlineto{\pgfqpoint{2.942658in}{0.551963in}}%
\pgfpathlineto{\pgfqpoint{2.942819in}{0.557881in}}%
\pgfpathlineto{\pgfqpoint{2.934858in}{0.564607in}}%
\pgfpathlineto{\pgfqpoint{2.934062in}{0.578133in}}%
\pgfpathlineto{\pgfqpoint{2.949616in}{0.580603in}}%
\pgfpathlineto{\pgfqpoint{2.962352in}{0.566902in}}%
\pgfpathlineto{\pgfqpoint{2.967260in}{0.557714in}}%
\pgfpathlineto{\pgfqpoint{2.963191in}{0.549874in}}%
\pgfpathlineto{\pgfqpoint{2.964102in}{0.543028in}}%
\pgfpathlineto{\pgfqpoint{2.969520in}{0.528810in}}%
\pgfpathlineto{\pgfqpoint{2.985351in}{0.505245in}}%
\pgfpathclose%
\pgfusepath{fill}%
\end{pgfscope}%
\begin{pgfscope}%
\pgfpathrectangle{\pgfqpoint{0.100000in}{0.100000in}}{\pgfqpoint{3.420221in}{2.189500in}}%
\pgfusepath{clip}%
\pgfsetbuttcap%
\pgfsetmiterjoin%
\definecolor{currentfill}{rgb}{0.000000,0.552941,0.723529}%
\pgfsetfillcolor{currentfill}%
\pgfsetlinewidth{0.000000pt}%
\definecolor{currentstroke}{rgb}{0.000000,0.000000,0.000000}%
\pgfsetstrokecolor{currentstroke}%
\pgfsetstrokeopacity{0.000000}%
\pgfsetdash{}{0pt}%
\pgfpathmoveto{\pgfqpoint{2.107760in}{0.959013in}}%
\pgfpathlineto{\pgfqpoint{2.101228in}{0.956655in}}%
\pgfpathlineto{\pgfqpoint{2.081721in}{0.956487in}}%
\pgfpathlineto{\pgfqpoint{2.081761in}{0.977102in}}%
\pgfpathlineto{\pgfqpoint{2.059034in}{0.977340in}}%
\pgfpathlineto{\pgfqpoint{2.054867in}{0.979485in}}%
\pgfpathlineto{\pgfqpoint{2.054956in}{0.986070in}}%
\pgfpathlineto{\pgfqpoint{2.058373in}{0.991477in}}%
\pgfpathlineto{\pgfqpoint{2.058680in}{1.008785in}}%
\pgfpathlineto{\pgfqpoint{2.061428in}{1.013443in}}%
\pgfpathlineto{\pgfqpoint{2.058500in}{1.018549in}}%
\pgfpathlineto{\pgfqpoint{2.058602in}{1.025058in}}%
\pgfpathlineto{\pgfqpoint{2.064045in}{1.025010in}}%
\pgfpathlineto{\pgfqpoint{2.068429in}{1.032488in}}%
\pgfpathlineto{\pgfqpoint{2.092270in}{1.033108in}}%
\pgfpathlineto{\pgfqpoint{2.092282in}{1.030925in}}%
\pgfpathlineto{\pgfqpoint{2.135532in}{1.030896in}}%
\pgfpathlineto{\pgfqpoint{2.135623in}{1.017880in}}%
\pgfpathlineto{\pgfqpoint{2.133504in}{1.011369in}}%
\pgfpathlineto{\pgfqpoint{2.133750in}{0.989583in}}%
\pgfpathlineto{\pgfqpoint{2.129349in}{0.989036in}}%
\pgfpathlineto{\pgfqpoint{2.115323in}{0.974397in}}%
\pgfpathlineto{\pgfqpoint{2.106511in}{0.966683in}}%
\pgfpathclose%
\pgfusepath{fill}%
\end{pgfscope}%
\begin{pgfscope}%
\pgfpathrectangle{\pgfqpoint{0.100000in}{0.100000in}}{\pgfqpoint{3.420221in}{2.189500in}}%
\pgfusepath{clip}%
\pgfsetbuttcap%
\pgfsetmiterjoin%
\definecolor{currentfill}{rgb}{0.000000,0.333333,0.833333}%
\pgfsetfillcolor{currentfill}%
\pgfsetlinewidth{0.000000pt}%
\definecolor{currentstroke}{rgb}{0.000000,0.000000,0.000000}%
\pgfsetstrokecolor{currentstroke}%
\pgfsetstrokeopacity{0.000000}%
\pgfsetdash{}{0pt}%
\pgfpathmoveto{\pgfqpoint{1.024476in}{0.871303in}}%
\pgfpathlineto{\pgfqpoint{0.954398in}{0.883239in}}%
\pgfpathlineto{\pgfqpoint{0.891374in}{0.895100in}}%
\pgfpathlineto{\pgfqpoint{0.844233in}{0.904518in}}%
\pgfpathlineto{\pgfqpoint{0.857403in}{0.968918in}}%
\pgfpathlineto{\pgfqpoint{0.866788in}{1.014753in}}%
\pgfpathlineto{\pgfqpoint{0.902848in}{1.007551in}}%
\pgfpathlineto{\pgfqpoint{0.929810in}{0.993281in}}%
\pgfpathlineto{\pgfqpoint{0.938991in}{1.004219in}}%
\pgfpathlineto{\pgfqpoint{0.965217in}{0.995576in}}%
\pgfpathlineto{\pgfqpoint{0.979350in}{0.992961in}}%
\pgfpathlineto{\pgfqpoint{0.982440in}{1.003961in}}%
\pgfpathlineto{\pgfqpoint{0.968103in}{1.006621in}}%
\pgfpathlineto{\pgfqpoint{0.974552in}{1.023403in}}%
\pgfpathlineto{\pgfqpoint{0.982827in}{1.028356in}}%
\pgfpathlineto{\pgfqpoint{0.990916in}{1.023161in}}%
\pgfpathlineto{\pgfqpoint{0.999559in}{1.023134in}}%
\pgfpathlineto{\pgfqpoint{1.005688in}{1.017601in}}%
\pgfpathlineto{\pgfqpoint{1.013615in}{1.014403in}}%
\pgfpathlineto{\pgfqpoint{1.022533in}{1.005502in}}%
\pgfpathlineto{\pgfqpoint{1.028431in}{1.004289in}}%
\pgfpathlineto{\pgfqpoint{1.025018in}{0.984790in}}%
\pgfpathlineto{\pgfqpoint{1.071111in}{0.976916in}}%
\pgfpathlineto{\pgfqpoint{1.065882in}{0.945522in}}%
\pgfpathlineto{\pgfqpoint{1.064508in}{0.937276in}}%
\pgfpathlineto{\pgfqpoint{1.054048in}{0.939013in}}%
\pgfpathlineto{\pgfqpoint{1.044265in}{0.938176in}}%
\pgfpathlineto{\pgfqpoint{1.036145in}{0.935622in}}%
\pgfpathlineto{\pgfqpoint{1.033293in}{0.921832in}}%
\pgfpathclose%
\pgfusepath{fill}%
\end{pgfscope}%
\begin{pgfscope}%
\pgfpathrectangle{\pgfqpoint{0.100000in}{0.100000in}}{\pgfqpoint{3.420221in}{2.189500in}}%
\pgfusepath{clip}%
\pgfsetbuttcap%
\pgfsetmiterjoin%
\definecolor{currentfill}{rgb}{0.000000,0.403922,0.798039}%
\pgfsetfillcolor{currentfill}%
\pgfsetlinewidth{0.000000pt}%
\definecolor{currentstroke}{rgb}{0.000000,0.000000,0.000000}%
\pgfsetstrokecolor{currentstroke}%
\pgfsetstrokeopacity{0.000000}%
\pgfsetdash{}{0pt}%
\pgfpathmoveto{\pgfqpoint{1.931934in}{2.028569in}}%
\pgfpathlineto{\pgfqpoint{1.994431in}{2.027951in}}%
\pgfpathlineto{\pgfqpoint{1.994501in}{2.057308in}}%
\pgfpathlineto{\pgfqpoint{1.999208in}{2.054925in}}%
\pgfpathlineto{\pgfqpoint{2.004218in}{2.056222in}}%
\pgfpathlineto{\pgfqpoint{2.010794in}{2.050446in}}%
\pgfpathlineto{\pgfqpoint{2.017960in}{2.019235in}}%
\pgfpathlineto{\pgfqpoint{2.017598in}{2.011169in}}%
\pgfpathlineto{\pgfqpoint{2.022819in}{2.006592in}}%
\pgfpathlineto{\pgfqpoint{2.030697in}{2.005365in}}%
\pgfpathlineto{\pgfqpoint{2.030936in}{1.980015in}}%
\pgfpathlineto{\pgfqpoint{1.991462in}{1.979833in}}%
\pgfpathlineto{\pgfqpoint{1.991441in}{1.993053in}}%
\pgfpathlineto{\pgfqpoint{1.971800in}{1.993054in}}%
\pgfpathlineto{\pgfqpoint{1.932284in}{1.993898in}}%
\pgfpathlineto{\pgfqpoint{1.931624in}{2.007059in}}%
\pgfpathclose%
\pgfusepath{fill}%
\end{pgfscope}%
\begin{pgfscope}%
\pgfpathrectangle{\pgfqpoint{0.100000in}{0.100000in}}{\pgfqpoint{3.420221in}{2.189500in}}%
\pgfusepath{clip}%
\pgfsetbuttcap%
\pgfsetmiterjoin%
\definecolor{currentfill}{rgb}{0.000000,0.431373,0.784314}%
\pgfsetfillcolor{currentfill}%
\pgfsetlinewidth{0.000000pt}%
\definecolor{currentstroke}{rgb}{0.000000,0.000000,0.000000}%
\pgfsetstrokecolor{currentstroke}%
\pgfsetstrokeopacity{0.000000}%
\pgfsetdash{}{0pt}%
\pgfpathmoveto{\pgfqpoint{0.954903in}{2.031949in}}%
\pgfpathlineto{\pgfqpoint{0.949737in}{2.038496in}}%
\pgfpathlineto{\pgfqpoint{0.953933in}{2.041118in}}%
\pgfpathlineto{\pgfqpoint{0.953909in}{2.048944in}}%
\pgfpathlineto{\pgfqpoint{0.949823in}{2.054332in}}%
\pgfpathlineto{\pgfqpoint{0.942612in}{2.073554in}}%
\pgfpathlineto{\pgfqpoint{0.946711in}{2.091138in}}%
\pgfpathlineto{\pgfqpoint{0.947803in}{2.086498in}}%
\pgfpathlineto{\pgfqpoint{0.964080in}{2.089642in}}%
\pgfpathlineto{\pgfqpoint{0.964515in}{2.076365in}}%
\pgfpathlineto{\pgfqpoint{0.967967in}{2.067970in}}%
\pgfpathlineto{\pgfqpoint{0.967344in}{2.062481in}}%
\pgfpathlineto{\pgfqpoint{0.979492in}{2.058425in}}%
\pgfpathlineto{\pgfqpoint{0.984851in}{2.060321in}}%
\pgfpathlineto{\pgfqpoint{0.987549in}{2.066343in}}%
\pgfpathlineto{\pgfqpoint{0.994661in}{2.064735in}}%
\pgfpathlineto{\pgfqpoint{0.997658in}{2.080271in}}%
\pgfpathlineto{\pgfqpoint{1.006159in}{2.078392in}}%
\pgfpathlineto{\pgfqpoint{1.011792in}{2.104174in}}%
\pgfpathlineto{\pgfqpoint{1.011280in}{2.110902in}}%
\pgfpathlineto{\pgfqpoint{1.021348in}{2.110711in}}%
\pgfpathlineto{\pgfqpoint{1.025060in}{2.117243in}}%
\pgfpathlineto{\pgfqpoint{1.023944in}{2.122412in}}%
\pgfpathlineto{\pgfqpoint{1.024736in}{2.134554in}}%
\pgfpathlineto{\pgfqpoint{1.056936in}{2.127465in}}%
\pgfpathlineto{\pgfqpoint{1.056811in}{2.118321in}}%
\pgfpathlineto{\pgfqpoint{1.059550in}{2.112814in}}%
\pgfpathlineto{\pgfqpoint{1.065792in}{2.114467in}}%
\pgfpathlineto{\pgfqpoint{1.070453in}{2.102910in}}%
\pgfpathlineto{\pgfqpoint{1.066357in}{2.095923in}}%
\pgfpathlineto{\pgfqpoint{1.079370in}{2.087695in}}%
\pgfpathlineto{\pgfqpoint{1.077353in}{2.081292in}}%
\pgfpathlineto{\pgfqpoint{1.082889in}{2.073174in}}%
\pgfpathlineto{\pgfqpoint{1.080178in}{2.071594in}}%
\pgfpathlineto{\pgfqpoint{1.086189in}{2.063017in}}%
\pgfpathlineto{\pgfqpoint{1.085291in}{2.057259in}}%
\pgfpathlineto{\pgfqpoint{1.095369in}{2.052264in}}%
\pgfpathlineto{\pgfqpoint{1.099271in}{2.043502in}}%
\pgfpathlineto{\pgfqpoint{1.094304in}{2.038665in}}%
\pgfpathlineto{\pgfqpoint{1.089384in}{2.034350in}}%
\pgfpathlineto{\pgfqpoint{1.086452in}{2.024455in}}%
\pgfpathlineto{\pgfqpoint{1.081767in}{2.023146in}}%
\pgfpathlineto{\pgfqpoint{1.080946in}{2.013694in}}%
\pgfpathlineto{\pgfqpoint{1.064900in}{2.017251in}}%
\pgfpathlineto{\pgfqpoint{1.041716in}{2.021843in}}%
\pgfpathlineto{\pgfqpoint{1.038710in}{2.013123in}}%
\pgfpathlineto{\pgfqpoint{1.041422in}{2.006055in}}%
\pgfpathlineto{\pgfqpoint{1.039745in}{1.998406in}}%
\pgfpathlineto{\pgfqpoint{1.040598in}{1.989940in}}%
\pgfpathlineto{\pgfqpoint{1.034294in}{1.987915in}}%
\pgfpathlineto{\pgfqpoint{1.021519in}{1.990629in}}%
\pgfpathlineto{\pgfqpoint{1.017863in}{1.990055in}}%
\pgfpathlineto{\pgfqpoint{1.010779in}{1.998519in}}%
\pgfpathlineto{\pgfqpoint{0.998622in}{2.006162in}}%
\pgfpathlineto{\pgfqpoint{0.990720in}{2.007460in}}%
\pgfpathlineto{\pgfqpoint{0.985087in}{2.012119in}}%
\pgfpathlineto{\pgfqpoint{0.985938in}{2.018362in}}%
\pgfpathlineto{\pgfqpoint{0.972309in}{2.028527in}}%
\pgfpathlineto{\pgfqpoint{0.963906in}{2.028834in}}%
\pgfpathclose%
\pgfusepath{fill}%
\end{pgfscope}%
\begin{pgfscope}%
\pgfpathrectangle{\pgfqpoint{0.100000in}{0.100000in}}{\pgfqpoint{3.420221in}{2.189500in}}%
\pgfusepath{clip}%
\pgfsetbuttcap%
\pgfsetmiterjoin%
\definecolor{currentfill}{rgb}{0.000000,0.478431,0.760784}%
\pgfsetfillcolor{currentfill}%
\pgfsetlinewidth{0.000000pt}%
\definecolor{currentstroke}{rgb}{0.000000,0.000000,0.000000}%
\pgfsetstrokecolor{currentstroke}%
\pgfsetstrokeopacity{0.000000}%
\pgfsetdash{}{0pt}%
\pgfpathmoveto{\pgfqpoint{1.506106in}{0.622590in}}%
\pgfpathlineto{\pgfqpoint{1.521140in}{0.608133in}}%
\pgfpathlineto{\pgfqpoint{1.515634in}{0.600210in}}%
\pgfpathlineto{\pgfqpoint{1.496946in}{0.599727in}}%
\pgfpathlineto{\pgfqpoint{1.491378in}{0.588846in}}%
\pgfpathlineto{\pgfqpoint{1.486838in}{0.583840in}}%
\pgfpathlineto{\pgfqpoint{1.483663in}{0.571686in}}%
\pgfpathlineto{\pgfqpoint{1.470573in}{0.555007in}}%
\pgfpathlineto{\pgfqpoint{1.464522in}{0.550583in}}%
\pgfpathlineto{\pgfqpoint{1.463018in}{0.545035in}}%
\pgfpathlineto{\pgfqpoint{1.454792in}{0.545432in}}%
\pgfpathlineto{\pgfqpoint{1.440777in}{0.553334in}}%
\pgfpathlineto{\pgfqpoint{1.435112in}{0.560708in}}%
\pgfpathlineto{\pgfqpoint{1.424660in}{0.563479in}}%
\pgfpathlineto{\pgfqpoint{1.420817in}{0.570282in}}%
\pgfpathlineto{\pgfqpoint{1.408372in}{0.573861in}}%
\pgfpathlineto{\pgfqpoint{1.398040in}{0.581313in}}%
\pgfpathlineto{\pgfqpoint{1.392757in}{0.590749in}}%
\pgfpathlineto{\pgfqpoint{1.386437in}{0.593121in}}%
\pgfpathlineto{\pgfqpoint{1.375505in}{0.603544in}}%
\pgfpathlineto{\pgfqpoint{1.372999in}{0.614281in}}%
\pgfpathlineto{\pgfqpoint{1.367315in}{0.624990in}}%
\pgfpathlineto{\pgfqpoint{1.366621in}{0.636755in}}%
\pgfpathlineto{\pgfqpoint{1.367812in}{0.649639in}}%
\pgfpathlineto{\pgfqpoint{1.359152in}{0.662392in}}%
\pgfpathlineto{\pgfqpoint{1.359500in}{0.670308in}}%
\pgfpathlineto{\pgfqpoint{1.356508in}{0.679070in}}%
\pgfpathlineto{\pgfqpoint{1.353269in}{0.681268in}}%
\pgfpathlineto{\pgfqpoint{1.357623in}{0.683419in}}%
\pgfpathlineto{\pgfqpoint{1.413730in}{0.711118in}}%
\pgfpathlineto{\pgfqpoint{1.444647in}{0.682354in}}%
\pgfpathclose%
\pgfusepath{fill}%
\end{pgfscope}%
\begin{pgfscope}%
\pgfpathrectangle{\pgfqpoint{0.100000in}{0.100000in}}{\pgfqpoint{3.420221in}{2.189500in}}%
\pgfusepath{clip}%
\pgfsetbuttcap%
\pgfsetmiterjoin%
\definecolor{currentfill}{rgb}{0.000000,0.501961,0.749020}%
\pgfsetfillcolor{currentfill}%
\pgfsetlinewidth{0.000000pt}%
\definecolor{currentstroke}{rgb}{0.000000,0.000000,0.000000}%
\pgfsetstrokecolor{currentstroke}%
\pgfsetstrokeopacity{0.000000}%
\pgfsetdash{}{0pt}%
\pgfpathmoveto{\pgfqpoint{2.578746in}{0.926936in}}%
\pgfpathlineto{\pgfqpoint{2.570022in}{0.924678in}}%
\pgfpathlineto{\pgfqpoint{2.559427in}{0.915451in}}%
\pgfpathlineto{\pgfqpoint{2.554196in}{0.919259in}}%
\pgfpathlineto{\pgfqpoint{2.550048in}{0.925622in}}%
\pgfpathlineto{\pgfqpoint{2.544427in}{0.924686in}}%
\pgfpathlineto{\pgfqpoint{2.541354in}{0.929132in}}%
\pgfpathlineto{\pgfqpoint{2.542768in}{0.933242in}}%
\pgfpathlineto{\pgfqpoint{2.530547in}{0.947463in}}%
\pgfpathlineto{\pgfqpoint{2.524078in}{0.947004in}}%
\pgfpathlineto{\pgfqpoint{2.525691in}{0.962109in}}%
\pgfpathlineto{\pgfqpoint{2.532588in}{0.960995in}}%
\pgfpathlineto{\pgfqpoint{2.539121in}{0.966130in}}%
\pgfpathlineto{\pgfqpoint{2.535519in}{0.972835in}}%
\pgfpathlineto{\pgfqpoint{2.536104in}{0.987774in}}%
\pgfpathlineto{\pgfqpoint{2.534274in}{0.990908in}}%
\pgfpathlineto{\pgfqpoint{2.535887in}{0.999923in}}%
\pgfpathlineto{\pgfqpoint{2.563217in}{1.002398in}}%
\pgfpathlineto{\pgfqpoint{2.579166in}{1.003760in}}%
\pgfpathlineto{\pgfqpoint{2.587030in}{0.974669in}}%
\pgfpathlineto{\pgfqpoint{2.588368in}{0.969935in}}%
\pgfpathlineto{\pgfqpoint{2.584794in}{0.966559in}}%
\pgfpathlineto{\pgfqpoint{2.582016in}{0.957368in}}%
\pgfpathlineto{\pgfqpoint{2.574347in}{0.948455in}}%
\pgfpathlineto{\pgfqpoint{2.570155in}{0.946901in}}%
\pgfpathclose%
\pgfusepath{fill}%
\end{pgfscope}%
\begin{pgfscope}%
\pgfpathrectangle{\pgfqpoint{0.100000in}{0.100000in}}{\pgfqpoint{3.420221in}{2.189500in}}%
\pgfusepath{clip}%
\pgfsetbuttcap%
\pgfsetmiterjoin%
\definecolor{currentfill}{rgb}{0.000000,0.913725,0.543137}%
\pgfsetfillcolor{currentfill}%
\pgfsetlinewidth{0.000000pt}%
\definecolor{currentstroke}{rgb}{0.000000,0.000000,0.000000}%
\pgfsetstrokecolor{currentstroke}%
\pgfsetstrokeopacity{0.000000}%
\pgfsetdash{}{0pt}%
\pgfpathmoveto{\pgfqpoint{0.579546in}{1.526027in}}%
\pgfpathlineto{\pgfqpoint{0.574372in}{1.530867in}}%
\pgfpathlineto{\pgfqpoint{0.578026in}{1.543361in}}%
\pgfpathlineto{\pgfqpoint{0.574431in}{1.550144in}}%
\pgfpathlineto{\pgfqpoint{0.575457in}{1.555502in}}%
\pgfpathlineto{\pgfqpoint{0.568799in}{1.559682in}}%
\pgfpathlineto{\pgfqpoint{0.559376in}{1.575663in}}%
\pgfpathlineto{\pgfqpoint{0.549720in}{1.580851in}}%
\pgfpathlineto{\pgfqpoint{0.542266in}{1.577222in}}%
\pgfpathlineto{\pgfqpoint{0.534587in}{1.577862in}}%
\pgfpathlineto{\pgfqpoint{0.532169in}{1.581343in}}%
\pgfpathlineto{\pgfqpoint{0.536202in}{1.595006in}}%
\pgfpathlineto{\pgfqpoint{0.521612in}{1.599241in}}%
\pgfpathlineto{\pgfqpoint{0.531898in}{1.632190in}}%
\pgfpathlineto{\pgfqpoint{0.537210in}{1.652351in}}%
\pgfpathlineto{\pgfqpoint{0.474232in}{1.671474in}}%
\pgfpathlineto{\pgfqpoint{0.477517in}{1.683039in}}%
\pgfpathlineto{\pgfqpoint{0.473506in}{1.685552in}}%
\pgfpathlineto{\pgfqpoint{0.457833in}{1.678178in}}%
\pgfpathlineto{\pgfqpoint{0.450450in}{1.678603in}}%
\pgfpathlineto{\pgfqpoint{0.446966in}{1.674323in}}%
\pgfpathlineto{\pgfqpoint{0.449282in}{1.667261in}}%
\pgfpathlineto{\pgfqpoint{0.441182in}{1.667869in}}%
\pgfpathlineto{\pgfqpoint{0.438930in}{1.674194in}}%
\pgfpathlineto{\pgfqpoint{0.431662in}{1.676724in}}%
\pgfpathlineto{\pgfqpoint{0.431177in}{1.680416in}}%
\pgfpathlineto{\pgfqpoint{0.425646in}{1.685217in}}%
\pgfpathlineto{\pgfqpoint{0.424944in}{1.702887in}}%
\pgfpathlineto{\pgfqpoint{0.416363in}{1.705641in}}%
\pgfpathlineto{\pgfqpoint{0.419641in}{1.708558in}}%
\pgfpathlineto{\pgfqpoint{0.420630in}{1.716310in}}%
\pgfpathlineto{\pgfqpoint{0.419134in}{1.723110in}}%
\pgfpathlineto{\pgfqpoint{0.424146in}{1.729287in}}%
\pgfpathlineto{\pgfqpoint{0.424521in}{1.738272in}}%
\pgfpathlineto{\pgfqpoint{0.433956in}{1.741230in}}%
\pgfpathlineto{\pgfqpoint{0.438361in}{1.747283in}}%
\pgfpathlineto{\pgfqpoint{0.503786in}{1.727081in}}%
\pgfpathlineto{\pgfqpoint{0.526142in}{1.797850in}}%
\pgfpathlineto{\pgfqpoint{0.527696in}{1.802948in}}%
\pgfpathlineto{\pgfqpoint{0.537902in}{1.800390in}}%
\pgfpathlineto{\pgfqpoint{0.540923in}{1.805575in}}%
\pgfpathlineto{\pgfqpoint{0.547870in}{1.812283in}}%
\pgfpathlineto{\pgfqpoint{0.547842in}{1.819890in}}%
\pgfpathlineto{\pgfqpoint{0.543810in}{1.827290in}}%
\pgfpathlineto{\pgfqpoint{0.546447in}{1.835654in}}%
\pgfpathlineto{\pgfqpoint{0.554402in}{1.837778in}}%
\pgfpathlineto{\pgfqpoint{0.589314in}{1.827339in}}%
\pgfpathlineto{\pgfqpoint{0.582916in}{1.808864in}}%
\pgfpathlineto{\pgfqpoint{0.569749in}{1.764981in}}%
\pgfpathlineto{\pgfqpoint{0.594355in}{1.757492in}}%
\pgfpathlineto{\pgfqpoint{0.578745in}{1.703409in}}%
\pgfpathlineto{\pgfqpoint{0.625927in}{1.689905in}}%
\pgfpathlineto{\pgfqpoint{0.609379in}{1.631416in}}%
\pgfpathlineto{\pgfqpoint{0.595667in}{1.582010in}}%
\pgfpathclose%
\pgfusepath{fill}%
\end{pgfscope}%
\begin{pgfscope}%
\pgfpathrectangle{\pgfqpoint{0.100000in}{0.100000in}}{\pgfqpoint{3.420221in}{2.189500in}}%
\pgfusepath{clip}%
\pgfsetbuttcap%
\pgfsetmiterjoin%
\definecolor{currentfill}{rgb}{0.000000,0.458824,0.770588}%
\pgfsetfillcolor{currentfill}%
\pgfsetlinewidth{0.000000pt}%
\definecolor{currentstroke}{rgb}{0.000000,0.000000,0.000000}%
\pgfsetstrokecolor{currentstroke}%
\pgfsetstrokeopacity{0.000000}%
\pgfsetdash{}{0pt}%
\pgfpathmoveto{\pgfqpoint{2.503884in}{1.482216in}}%
\pgfpathlineto{\pgfqpoint{2.506093in}{1.462644in}}%
\pgfpathlineto{\pgfqpoint{2.482907in}{1.460294in}}%
\pgfpathlineto{\pgfqpoint{2.456869in}{1.457965in}}%
\pgfpathlineto{\pgfqpoint{2.453168in}{1.498102in}}%
\pgfpathlineto{\pgfqpoint{2.466135in}{1.509097in}}%
\pgfpathlineto{\pgfqpoint{2.471097in}{1.517259in}}%
\pgfpathlineto{\pgfqpoint{2.475060in}{1.530708in}}%
\pgfpathlineto{\pgfqpoint{2.480973in}{1.540837in}}%
\pgfpathlineto{\pgfqpoint{2.488833in}{1.541607in}}%
\pgfpathlineto{\pgfqpoint{2.490110in}{1.528776in}}%
\pgfpathlineto{\pgfqpoint{2.515623in}{1.531150in}}%
\pgfpathlineto{\pgfqpoint{2.517831in}{1.510992in}}%
\pgfpathlineto{\pgfqpoint{2.516398in}{1.507830in}}%
\pgfpathlineto{\pgfqpoint{2.501285in}{1.506433in}}%
\pgfpathclose%
\pgfusepath{fill}%
\end{pgfscope}%
\begin{pgfscope}%
\pgfpathrectangle{\pgfqpoint{0.100000in}{0.100000in}}{\pgfqpoint{3.420221in}{2.189500in}}%
\pgfusepath{clip}%
\pgfsetbuttcap%
\pgfsetmiterjoin%
\definecolor{currentfill}{rgb}{0.000000,0.427451,0.786275}%
\pgfsetfillcolor{currentfill}%
\pgfsetlinewidth{0.000000pt}%
\definecolor{currentstroke}{rgb}{0.000000,0.000000,0.000000}%
\pgfsetstrokecolor{currentstroke}%
\pgfsetstrokeopacity{0.000000}%
\pgfsetdash{}{0pt}%
\pgfpathmoveto{\pgfqpoint{2.275940in}{0.732891in}}%
\pgfpathlineto{\pgfqpoint{2.276385in}{0.754011in}}%
\pgfpathlineto{\pgfqpoint{2.274598in}{0.772623in}}%
\pgfpathlineto{\pgfqpoint{2.275685in}{0.781902in}}%
\pgfpathlineto{\pgfqpoint{2.279961in}{0.784207in}}%
\pgfpathlineto{\pgfqpoint{2.281604in}{0.792301in}}%
\pgfpathlineto{\pgfqpoint{2.284699in}{0.794436in}}%
\pgfpathlineto{\pgfqpoint{2.291991in}{0.800715in}}%
\pgfpathlineto{\pgfqpoint{2.306701in}{0.807375in}}%
\pgfpathlineto{\pgfqpoint{2.316751in}{0.815147in}}%
\pgfpathlineto{\pgfqpoint{2.320479in}{0.824381in}}%
\pgfpathlineto{\pgfqpoint{2.335357in}{0.825618in}}%
\pgfpathlineto{\pgfqpoint{2.335151in}{0.829013in}}%
\pgfpathlineto{\pgfqpoint{2.361084in}{0.830708in}}%
\pgfpathlineto{\pgfqpoint{2.364194in}{0.777408in}}%
\pgfpathlineto{\pgfqpoint{2.338229in}{0.775716in}}%
\pgfpathlineto{\pgfqpoint{2.339014in}{0.762555in}}%
\pgfpathlineto{\pgfqpoint{2.343993in}{0.760321in}}%
\pgfpathlineto{\pgfqpoint{2.345111in}{0.742764in}}%
\pgfpathlineto{\pgfqpoint{2.324590in}{0.740155in}}%
\pgfpathlineto{\pgfqpoint{2.314895in}{0.738933in}}%
\pgfpathlineto{\pgfqpoint{2.305255in}{0.734445in}}%
\pgfpathclose%
\pgfusepath{fill}%
\end{pgfscope}%
\begin{pgfscope}%
\pgfpathrectangle{\pgfqpoint{0.100000in}{0.100000in}}{\pgfqpoint{3.420221in}{2.189500in}}%
\pgfusepath{clip}%
\pgfsetbuttcap%
\pgfsetmiterjoin%
\definecolor{currentfill}{rgb}{0.000000,0.537255,0.731373}%
\pgfsetfillcolor{currentfill}%
\pgfsetlinewidth{0.000000pt}%
\definecolor{currentstroke}{rgb}{0.000000,0.000000,0.000000}%
\pgfsetstrokecolor{currentstroke}%
\pgfsetstrokeopacity{0.000000}%
\pgfsetdash{}{0pt}%
\pgfpathmoveto{\pgfqpoint{2.132292in}{0.728706in}}%
\pgfpathlineto{\pgfqpoint{2.136757in}{0.722363in}}%
\pgfpathlineto{\pgfqpoint{2.143551in}{0.720512in}}%
\pgfpathlineto{\pgfqpoint{2.149003in}{0.714645in}}%
\pgfpathlineto{\pgfqpoint{2.136769in}{0.703061in}}%
\pgfpathlineto{\pgfqpoint{2.132390in}{0.701330in}}%
\pgfpathlineto{\pgfqpoint{2.129080in}{0.702592in}}%
\pgfpathlineto{\pgfqpoint{2.115931in}{0.702337in}}%
\pgfpathlineto{\pgfqpoint{2.115640in}{0.715470in}}%
\pgfpathlineto{\pgfqpoint{2.109152in}{0.715330in}}%
\pgfpathlineto{\pgfqpoint{2.108811in}{0.728632in}}%
\pgfpathlineto{\pgfqpoint{2.102209in}{0.728531in}}%
\pgfpathlineto{\pgfqpoint{2.102077in}{0.738404in}}%
\pgfpathlineto{\pgfqpoint{2.073998in}{0.737919in}}%
\pgfpathlineto{\pgfqpoint{2.065129in}{0.748051in}}%
\pgfpathlineto{\pgfqpoint{2.063388in}{0.749102in}}%
\pgfpathlineto{\pgfqpoint{2.062668in}{0.815983in}}%
\pgfpathlineto{\pgfqpoint{2.062553in}{0.826398in}}%
\pgfpathlineto{\pgfqpoint{2.077586in}{0.826587in}}%
\pgfpathlineto{\pgfqpoint{2.129133in}{0.827351in}}%
\pgfpathlineto{\pgfqpoint{2.145790in}{0.827512in}}%
\pgfpathlineto{\pgfqpoint{2.146242in}{0.808318in}}%
\pgfpathlineto{\pgfqpoint{2.139710in}{0.808103in}}%
\pgfpathlineto{\pgfqpoint{2.136622in}{0.796059in}}%
\pgfpathlineto{\pgfqpoint{2.137034in}{0.785064in}}%
\pgfpathlineto{\pgfqpoint{2.143528in}{0.785169in}}%
\pgfpathlineto{\pgfqpoint{2.144163in}{0.768701in}}%
\pgfpathlineto{\pgfqpoint{2.141701in}{0.761995in}}%
\pgfpathlineto{\pgfqpoint{2.133784in}{0.761888in}}%
\pgfpathlineto{\pgfqpoint{2.137596in}{0.746243in}}%
\pgfpathlineto{\pgfqpoint{2.134084in}{0.739706in}}%
\pgfpathclose%
\pgfusepath{fill}%
\end{pgfscope}%
\begin{pgfscope}%
\pgfpathrectangle{\pgfqpoint{0.100000in}{0.100000in}}{\pgfqpoint{3.420221in}{2.189500in}}%
\pgfusepath{clip}%
\pgfsetbuttcap%
\pgfsetmiterjoin%
\definecolor{currentfill}{rgb}{0.000000,0.549020,0.725490}%
\pgfsetfillcolor{currentfill}%
\pgfsetlinewidth{0.000000pt}%
\definecolor{currentstroke}{rgb}{0.000000,0.000000,0.000000}%
\pgfsetstrokecolor{currentstroke}%
\pgfsetstrokeopacity{0.000000}%
\pgfsetdash{}{0pt}%
\pgfpathmoveto{\pgfqpoint{2.715961in}{1.073488in}}%
\pgfpathlineto{\pgfqpoint{2.702464in}{1.061702in}}%
\pgfpathlineto{\pgfqpoint{2.692435in}{1.060901in}}%
\pgfpathlineto{\pgfqpoint{2.691310in}{1.069948in}}%
\pgfpathlineto{\pgfqpoint{2.687386in}{1.073776in}}%
\pgfpathlineto{\pgfqpoint{2.681543in}{1.083589in}}%
\pgfpathlineto{\pgfqpoint{2.689178in}{1.090955in}}%
\pgfpathlineto{\pgfqpoint{2.682654in}{1.106375in}}%
\pgfpathlineto{\pgfqpoint{2.684932in}{1.118580in}}%
\pgfpathlineto{\pgfqpoint{2.693843in}{1.121300in}}%
\pgfpathlineto{\pgfqpoint{2.703105in}{1.126591in}}%
\pgfpathlineto{\pgfqpoint{2.707627in}{1.125277in}}%
\pgfpathlineto{\pgfqpoint{2.710162in}{1.117620in}}%
\pgfpathlineto{\pgfqpoint{2.715454in}{1.121623in}}%
\pgfpathlineto{\pgfqpoint{2.720990in}{1.115497in}}%
\pgfpathlineto{\pgfqpoint{2.716392in}{1.108500in}}%
\pgfpathlineto{\pgfqpoint{2.723922in}{1.104353in}}%
\pgfpathlineto{\pgfqpoint{2.734971in}{1.094864in}}%
\pgfpathlineto{\pgfqpoint{2.732227in}{1.082731in}}%
\pgfpathlineto{\pgfqpoint{2.721030in}{1.079378in}}%
\pgfpathclose%
\pgfusepath{fill}%
\end{pgfscope}%
\begin{pgfscope}%
\pgfpathrectangle{\pgfqpoint{0.100000in}{0.100000in}}{\pgfqpoint{3.420221in}{2.189500in}}%
\pgfusepath{clip}%
\pgfsetbuttcap%
\pgfsetmiterjoin%
\definecolor{currentfill}{rgb}{0.000000,0.258824,0.870588}%
\pgfsetfillcolor{currentfill}%
\pgfsetlinewidth{0.000000pt}%
\definecolor{currentstroke}{rgb}{0.000000,0.000000,0.000000}%
\pgfsetstrokecolor{currentstroke}%
\pgfsetstrokeopacity{0.000000}%
\pgfsetdash{}{0pt}%
\pgfpathmoveto{\pgfqpoint{1.228882in}{1.697003in}}%
\pgfpathlineto{\pgfqpoint{1.240747in}{1.695356in}}%
\pgfpathlineto{\pgfqpoint{1.241733in}{1.701918in}}%
\pgfpathlineto{\pgfqpoint{1.251377in}{1.700442in}}%
\pgfpathlineto{\pgfqpoint{1.252192in}{1.706867in}}%
\pgfpathlineto{\pgfqpoint{1.258837in}{1.705861in}}%
\pgfpathlineto{\pgfqpoint{1.259817in}{1.712449in}}%
\pgfpathlineto{\pgfqpoint{1.272716in}{1.710478in}}%
\pgfpathlineto{\pgfqpoint{1.273721in}{1.717100in}}%
\pgfpathlineto{\pgfqpoint{1.348673in}{1.706219in}}%
\pgfpathlineto{\pgfqpoint{1.343956in}{1.656380in}}%
\pgfpathlineto{\pgfqpoint{1.321060in}{1.659586in}}%
\pgfpathlineto{\pgfqpoint{1.317505in}{1.657873in}}%
\pgfpathlineto{\pgfqpoint{1.287196in}{1.662072in}}%
\pgfpathlineto{\pgfqpoint{1.277280in}{1.662703in}}%
\pgfpathlineto{\pgfqpoint{1.258831in}{1.670975in}}%
\pgfpathlineto{\pgfqpoint{1.259344in}{1.674204in}}%
\pgfpathlineto{\pgfqpoint{1.246955in}{1.679402in}}%
\pgfpathlineto{\pgfqpoint{1.234194in}{1.681436in}}%
\pgfpathlineto{\pgfqpoint{1.235133in}{1.687338in}}%
\pgfpathclose%
\pgfusepath{fill}%
\end{pgfscope}%
\begin{pgfscope}%
\pgfpathrectangle{\pgfqpoint{0.100000in}{0.100000in}}{\pgfqpoint{3.420221in}{2.189500in}}%
\pgfusepath{clip}%
\pgfsetbuttcap%
\pgfsetmiterjoin%
\definecolor{currentfill}{rgb}{0.000000,0.247059,0.876471}%
\pgfsetfillcolor{currentfill}%
\pgfsetlinewidth{0.000000pt}%
\definecolor{currentstroke}{rgb}{0.000000,0.000000,0.000000}%
\pgfsetstrokecolor{currentstroke}%
\pgfsetstrokeopacity{0.000000}%
\pgfsetdash{}{0pt}%
\pgfpathmoveto{\pgfqpoint{2.396586in}{1.578630in}}%
\pgfpathlineto{\pgfqpoint{2.344408in}{1.574809in}}%
\pgfpathlineto{\pgfqpoint{2.331404in}{1.574268in}}%
\pgfpathlineto{\pgfqpoint{2.327879in}{1.633172in}}%
\pgfpathlineto{\pgfqpoint{2.360764in}{1.635303in}}%
\pgfpathlineto{\pgfqpoint{2.361278in}{1.628752in}}%
\pgfpathlineto{\pgfqpoint{2.374277in}{1.629687in}}%
\pgfpathlineto{\pgfqpoint{2.394235in}{1.631207in}}%
\pgfpathlineto{\pgfqpoint{2.394072in}{1.625247in}}%
\pgfpathlineto{\pgfqpoint{2.391215in}{1.619643in}}%
\pgfpathlineto{\pgfqpoint{2.389597in}{1.608176in}}%
\pgfpathclose%
\pgfusepath{fill}%
\end{pgfscope}%
\begin{pgfscope}%
\pgfpathrectangle{\pgfqpoint{0.100000in}{0.100000in}}{\pgfqpoint{3.420221in}{2.189500in}}%
\pgfusepath{clip}%
\pgfsetbuttcap%
\pgfsetmiterjoin%
\definecolor{currentfill}{rgb}{0.000000,0.274510,0.862745}%
\pgfsetfillcolor{currentfill}%
\pgfsetlinewidth{0.000000pt}%
\definecolor{currentstroke}{rgb}{0.000000,0.000000,0.000000}%
\pgfsetstrokecolor{currentstroke}%
\pgfsetstrokeopacity{0.000000}%
\pgfsetdash{}{0pt}%
\pgfpathmoveto{\pgfqpoint{0.689353in}{1.009685in}}%
\pgfpathlineto{\pgfqpoint{0.683524in}{0.983897in}}%
\pgfpathlineto{\pgfqpoint{0.682160in}{0.984206in}}%
\pgfpathlineto{\pgfqpoint{0.674070in}{0.950804in}}%
\pgfpathlineto{\pgfqpoint{0.609682in}{0.959925in}}%
\pgfpathlineto{\pgfqpoint{0.610488in}{0.965893in}}%
\pgfpathlineto{\pgfqpoint{0.604580in}{0.971423in}}%
\pgfpathlineto{\pgfqpoint{0.607871in}{0.986142in}}%
\pgfpathlineto{\pgfqpoint{0.608748in}{0.996332in}}%
\pgfpathlineto{\pgfqpoint{0.606894in}{1.008889in}}%
\pgfpathlineto{\pgfqpoint{0.601201in}{1.023639in}}%
\pgfpathlineto{\pgfqpoint{0.596428in}{1.029491in}}%
\pgfpathlineto{\pgfqpoint{0.598753in}{1.034048in}}%
\pgfpathlineto{\pgfqpoint{0.603881in}{1.036716in}}%
\pgfpathlineto{\pgfqpoint{0.612074in}{1.033541in}}%
\pgfpathlineto{\pgfqpoint{0.618875in}{1.027275in}}%
\pgfpathlineto{\pgfqpoint{0.631624in}{1.023719in}}%
\pgfpathclose%
\pgfusepath{fill}%
\end{pgfscope}%
\begin{pgfscope}%
\pgfpathrectangle{\pgfqpoint{0.100000in}{0.100000in}}{\pgfqpoint{3.420221in}{2.189500in}}%
\pgfusepath{clip}%
\pgfsetbuttcap%
\pgfsetmiterjoin%
\definecolor{currentfill}{rgb}{0.000000,0.725490,0.637255}%
\pgfsetfillcolor{currentfill}%
\pgfsetlinewidth{0.000000pt}%
\definecolor{currentstroke}{rgb}{0.000000,0.000000,0.000000}%
\pgfsetstrokecolor{currentstroke}%
\pgfsetstrokeopacity{0.000000}%
\pgfsetdash{}{0pt}%
\pgfpathmoveto{\pgfqpoint{2.261925in}{1.806421in}}%
\pgfpathlineto{\pgfqpoint{2.248409in}{1.805657in}}%
\pgfpathlineto{\pgfqpoint{2.247679in}{1.818735in}}%
\pgfpathlineto{\pgfqpoint{2.241246in}{1.818346in}}%
\pgfpathlineto{\pgfqpoint{2.240934in}{1.824864in}}%
\pgfpathlineto{\pgfqpoint{2.234348in}{1.824567in}}%
\pgfpathlineto{\pgfqpoint{2.233206in}{1.850454in}}%
\pgfpathlineto{\pgfqpoint{2.239010in}{1.849078in}}%
\pgfpathlineto{\pgfqpoint{2.243266in}{1.851966in}}%
\pgfpathlineto{\pgfqpoint{2.259766in}{1.858554in}}%
\pgfpathlineto{\pgfqpoint{2.271630in}{1.870217in}}%
\pgfpathlineto{\pgfqpoint{2.290979in}{1.873319in}}%
\pgfpathlineto{\pgfqpoint{2.299164in}{1.878176in}}%
\pgfpathlineto{\pgfqpoint{2.304066in}{1.884877in}}%
\pgfpathlineto{\pgfqpoint{2.312145in}{1.886292in}}%
\pgfpathlineto{\pgfqpoint{2.314771in}{1.889216in}}%
\pgfpathlineto{\pgfqpoint{2.316175in}{1.868973in}}%
\pgfpathlineto{\pgfqpoint{2.320114in}{1.862758in}}%
\pgfpathlineto{\pgfqpoint{2.313651in}{1.862384in}}%
\pgfpathlineto{\pgfqpoint{2.314865in}{1.842736in}}%
\pgfpathlineto{\pgfqpoint{2.316486in}{1.818440in}}%
\pgfpathlineto{\pgfqpoint{2.311017in}{1.821199in}}%
\pgfpathlineto{\pgfqpoint{2.266599in}{1.830712in}}%
\pgfpathlineto{\pgfqpoint{2.267954in}{1.806755in}}%
\pgfpathclose%
\pgfusepath{fill}%
\end{pgfscope}%
\begin{pgfscope}%
\pgfpathrectangle{\pgfqpoint{0.100000in}{0.100000in}}{\pgfqpoint{3.420221in}{2.189500in}}%
\pgfusepath{clip}%
\pgfsetbuttcap%
\pgfsetmiterjoin%
\definecolor{currentfill}{rgb}{0.000000,0.427451,0.786275}%
\pgfsetfillcolor{currentfill}%
\pgfsetlinewidth{0.000000pt}%
\definecolor{currentstroke}{rgb}{0.000000,0.000000,0.000000}%
\pgfsetstrokecolor{currentstroke}%
\pgfsetstrokeopacity{0.000000}%
\pgfsetdash{}{0pt}%
\pgfpathmoveto{\pgfqpoint{1.500023in}{1.341478in}}%
\pgfpathlineto{\pgfqpoint{1.531803in}{1.338478in}}%
\pgfpathlineto{\pgfqpoint{1.528358in}{1.305826in}}%
\pgfpathlineto{\pgfqpoint{1.525354in}{1.273196in}}%
\pgfpathlineto{\pgfqpoint{1.523738in}{1.260933in}}%
\pgfpathlineto{\pgfqpoint{1.504310in}{1.262061in}}%
\pgfpathlineto{\pgfqpoint{1.472309in}{1.265613in}}%
\pgfpathlineto{\pgfqpoint{1.474804in}{1.291448in}}%
\pgfpathlineto{\pgfqpoint{1.494279in}{1.289386in}}%
\pgfpathclose%
\pgfusepath{fill}%
\end{pgfscope}%
\begin{pgfscope}%
\pgfpathrectangle{\pgfqpoint{0.100000in}{0.100000in}}{\pgfqpoint{3.420221in}{2.189500in}}%
\pgfusepath{clip}%
\pgfsetbuttcap%
\pgfsetmiterjoin%
\definecolor{currentfill}{rgb}{0.000000,0.384314,0.807843}%
\pgfsetfillcolor{currentfill}%
\pgfsetlinewidth{0.000000pt}%
\definecolor{currentstroke}{rgb}{0.000000,0.000000,0.000000}%
\pgfsetstrokecolor{currentstroke}%
\pgfsetstrokeopacity{0.000000}%
\pgfsetdash{}{0pt}%
\pgfpathmoveto{\pgfqpoint{1.997524in}{1.202886in}}%
\pgfpathlineto{\pgfqpoint{1.996892in}{1.180023in}}%
\pgfpathlineto{\pgfqpoint{1.945166in}{1.180450in}}%
\pgfpathlineto{\pgfqpoint{1.945500in}{1.203383in}}%
\pgfpathlineto{\pgfqpoint{1.971468in}{1.203012in}}%
\pgfpathlineto{\pgfqpoint{1.972284in}{1.232687in}}%
\pgfpathlineto{\pgfqpoint{1.972831in}{1.255474in}}%
\pgfpathlineto{\pgfqpoint{1.998810in}{1.255392in}}%
\pgfpathlineto{\pgfqpoint{1.998265in}{1.252126in}}%
\pgfpathlineto{\pgfqpoint{1.998280in}{1.219448in}}%
\pgfpathclose%
\pgfusepath{fill}%
\end{pgfscope}%
\begin{pgfscope}%
\pgfpathrectangle{\pgfqpoint{0.100000in}{0.100000in}}{\pgfqpoint{3.420221in}{2.189500in}}%
\pgfusepath{clip}%
\pgfsetbuttcap%
\pgfsetmiterjoin%
\definecolor{currentfill}{rgb}{0.000000,0.407843,0.796078}%
\pgfsetfillcolor{currentfill}%
\pgfsetlinewidth{0.000000pt}%
\definecolor{currentstroke}{rgb}{0.000000,0.000000,0.000000}%
\pgfsetstrokecolor{currentstroke}%
\pgfsetstrokeopacity{0.000000}%
\pgfsetdash{}{0pt}%
\pgfpathmoveto{\pgfqpoint{2.238385in}{1.612287in}}%
\pgfpathlineto{\pgfqpoint{2.209111in}{1.610975in}}%
\pgfpathlineto{\pgfqpoint{2.208205in}{1.616773in}}%
\pgfpathlineto{\pgfqpoint{2.186791in}{1.615926in}}%
\pgfpathlineto{\pgfqpoint{2.180284in}{1.615699in}}%
\pgfpathlineto{\pgfqpoint{2.179344in}{1.641708in}}%
\pgfpathlineto{\pgfqpoint{2.203523in}{1.642647in}}%
\pgfpathlineto{\pgfqpoint{2.195475in}{1.652653in}}%
\pgfpathlineto{\pgfqpoint{2.201046in}{1.652886in}}%
\pgfpathlineto{\pgfqpoint{2.201593in}{1.658963in}}%
\pgfpathlineto{\pgfqpoint{2.219627in}{1.660171in}}%
\pgfpathlineto{\pgfqpoint{2.222945in}{1.666910in}}%
\pgfpathlineto{\pgfqpoint{2.254950in}{1.668229in}}%
\pgfpathlineto{\pgfqpoint{2.254616in}{1.675253in}}%
\pgfpathlineto{\pgfqpoint{2.276646in}{1.676494in}}%
\pgfpathlineto{\pgfqpoint{2.270513in}{1.664213in}}%
\pgfpathlineto{\pgfqpoint{2.275239in}{1.653732in}}%
\pgfpathlineto{\pgfqpoint{2.275283in}{1.647074in}}%
\pgfpathlineto{\pgfqpoint{2.280622in}{1.641596in}}%
\pgfpathlineto{\pgfqpoint{2.285551in}{1.631176in}}%
\pgfpathlineto{\pgfqpoint{2.256952in}{1.629612in}}%
\pgfpathlineto{\pgfqpoint{2.257368in}{1.623082in}}%
\pgfpathlineto{\pgfqpoint{2.237751in}{1.622035in}}%
\pgfpathclose%
\pgfusepath{fill}%
\end{pgfscope}%
\begin{pgfscope}%
\pgfpathrectangle{\pgfqpoint{0.100000in}{0.100000in}}{\pgfqpoint{3.420221in}{2.189500in}}%
\pgfusepath{clip}%
\pgfsetbuttcap%
\pgfsetmiterjoin%
\definecolor{currentfill}{rgb}{0.000000,0.490196,0.754902}%
\pgfsetfillcolor{currentfill}%
\pgfsetlinewidth{0.000000pt}%
\definecolor{currentstroke}{rgb}{0.000000,0.000000,0.000000}%
\pgfsetstrokecolor{currentstroke}%
\pgfsetstrokeopacity{0.000000}%
\pgfsetdash{}{0pt}%
\pgfpathmoveto{\pgfqpoint{1.175357in}{2.036904in}}%
\pgfpathlineto{\pgfqpoint{1.173407in}{2.026068in}}%
\pgfpathlineto{\pgfqpoint{1.144655in}{2.031235in}}%
\pgfpathlineto{\pgfqpoint{1.142932in}{2.039449in}}%
\pgfpathlineto{\pgfqpoint{1.137065in}{2.043970in}}%
\pgfpathlineto{\pgfqpoint{1.095369in}{2.052264in}}%
\pgfpathlineto{\pgfqpoint{1.085291in}{2.057259in}}%
\pgfpathlineto{\pgfqpoint{1.086189in}{2.063017in}}%
\pgfpathlineto{\pgfqpoint{1.080178in}{2.071594in}}%
\pgfpathlineto{\pgfqpoint{1.082889in}{2.073174in}}%
\pgfpathlineto{\pgfqpoint{1.077353in}{2.081292in}}%
\pgfpathlineto{\pgfqpoint{1.079370in}{2.087695in}}%
\pgfpathlineto{\pgfqpoint{1.066357in}{2.095923in}}%
\pgfpathlineto{\pgfqpoint{1.070453in}{2.102910in}}%
\pgfpathlineto{\pgfqpoint{1.065792in}{2.114467in}}%
\pgfpathlineto{\pgfqpoint{1.059550in}{2.112814in}}%
\pgfpathlineto{\pgfqpoint{1.056811in}{2.118321in}}%
\pgfpathlineto{\pgfqpoint{1.056936in}{2.127465in}}%
\pgfpathlineto{\pgfqpoint{1.102109in}{2.118042in}}%
\pgfpathlineto{\pgfqpoint{1.136154in}{2.111320in}}%
\pgfpathlineto{\pgfqpoint{1.193962in}{2.100364in}}%
\pgfpathlineto{\pgfqpoint{1.183231in}{2.042209in}}%
\pgfpathlineto{\pgfqpoint{1.176553in}{2.043426in}}%
\pgfpathclose%
\pgfusepath{fill}%
\end{pgfscope}%
\begin{pgfscope}%
\pgfpathrectangle{\pgfqpoint{0.100000in}{0.100000in}}{\pgfqpoint{3.420221in}{2.189500in}}%
\pgfusepath{clip}%
\pgfsetbuttcap%
\pgfsetmiterjoin%
\definecolor{currentfill}{rgb}{0.000000,0.258824,0.870588}%
\pgfsetfillcolor{currentfill}%
\pgfsetlinewidth{0.000000pt}%
\definecolor{currentstroke}{rgb}{0.000000,0.000000,0.000000}%
\pgfsetstrokecolor{currentstroke}%
\pgfsetstrokeopacity{0.000000}%
\pgfsetdash{}{0pt}%
\pgfpathmoveto{\pgfqpoint{1.718020in}{1.949196in}}%
\pgfpathlineto{\pgfqpoint{1.652699in}{1.953699in}}%
\pgfpathlineto{\pgfqpoint{1.653702in}{1.966967in}}%
\pgfpathlineto{\pgfqpoint{1.651134in}{1.967171in}}%
\pgfpathlineto{\pgfqpoint{1.653161in}{1.993596in}}%
\pgfpathlineto{\pgfqpoint{1.650585in}{1.993775in}}%
\pgfpathlineto{\pgfqpoint{1.651609in}{2.007035in}}%
\pgfpathlineto{\pgfqpoint{1.638517in}{2.007966in}}%
\pgfpathlineto{\pgfqpoint{1.639598in}{2.021165in}}%
\pgfpathlineto{\pgfqpoint{1.643650in}{2.020850in}}%
\pgfpathlineto{\pgfqpoint{1.644195in}{2.027434in}}%
\pgfpathlineto{\pgfqpoint{1.650706in}{2.026917in}}%
\pgfpathlineto{\pgfqpoint{1.651861in}{2.041486in}}%
\pgfpathlineto{\pgfqpoint{1.701879in}{2.037908in}}%
\pgfpathlineto{\pgfqpoint{1.743456in}{2.035343in}}%
\pgfpathlineto{\pgfqpoint{1.742226in}{2.014058in}}%
\pgfpathlineto{\pgfqpoint{1.743331in}{2.000767in}}%
\pgfpathlineto{\pgfqpoint{1.736759in}{2.001110in}}%
\pgfpathlineto{\pgfqpoint{1.737983in}{1.987834in}}%
\pgfpathlineto{\pgfqpoint{1.736475in}{1.961354in}}%
\pgfpathlineto{\pgfqpoint{1.737696in}{1.948036in}}%
\pgfpathclose%
\pgfusepath{fill}%
\end{pgfscope}%
\begin{pgfscope}%
\pgfpathrectangle{\pgfqpoint{0.100000in}{0.100000in}}{\pgfqpoint{3.420221in}{2.189500in}}%
\pgfusepath{clip}%
\pgfsetbuttcap%
\pgfsetmiterjoin%
\definecolor{currentfill}{rgb}{0.000000,0.427451,0.786275}%
\pgfsetfillcolor{currentfill}%
\pgfsetlinewidth{0.000000pt}%
\definecolor{currentstroke}{rgb}{0.000000,0.000000,0.000000}%
\pgfsetstrokecolor{currentstroke}%
\pgfsetstrokeopacity{0.000000}%
\pgfsetdash{}{0pt}%
\pgfpathmoveto{\pgfqpoint{1.795993in}{1.649473in}}%
\pgfpathlineto{\pgfqpoint{1.795333in}{1.616714in}}%
\pgfpathlineto{\pgfqpoint{1.767928in}{1.617953in}}%
\pgfpathlineto{\pgfqpoint{1.750202in}{1.618833in}}%
\pgfpathlineto{\pgfqpoint{1.751135in}{1.637119in}}%
\pgfpathlineto{\pgfqpoint{1.739185in}{1.633624in}}%
\pgfpathlineto{\pgfqpoint{1.727787in}{1.634509in}}%
\pgfpathlineto{\pgfqpoint{1.718078in}{1.636688in}}%
\pgfpathlineto{\pgfqpoint{1.712227in}{1.637261in}}%
\pgfpathlineto{\pgfqpoint{1.711329in}{1.647107in}}%
\pgfpathlineto{\pgfqpoint{1.712700in}{1.671249in}}%
\pgfpathlineto{\pgfqpoint{1.712814in}{1.673209in}}%
\pgfpathlineto{\pgfqpoint{1.735918in}{1.671942in}}%
\pgfpathlineto{\pgfqpoint{1.740485in}{1.666348in}}%
\pgfpathlineto{\pgfqpoint{1.745114in}{1.666933in}}%
\pgfpathlineto{\pgfqpoint{1.752167in}{1.663762in}}%
\pgfpathlineto{\pgfqpoint{1.750793in}{1.672835in}}%
\pgfpathlineto{\pgfqpoint{1.755377in}{1.670718in}}%
\pgfpathlineto{\pgfqpoint{1.790424in}{1.669387in}}%
\pgfpathlineto{\pgfqpoint{1.789530in}{1.649742in}}%
\pgfpathclose%
\pgfusepath{fill}%
\end{pgfscope}%
\begin{pgfscope}%
\pgfpathrectangle{\pgfqpoint{0.100000in}{0.100000in}}{\pgfqpoint{3.420221in}{2.189500in}}%
\pgfusepath{clip}%
\pgfsetbuttcap%
\pgfsetmiterjoin%
\definecolor{currentfill}{rgb}{0.000000,0.917647,0.541176}%
\pgfsetfillcolor{currentfill}%
\pgfsetlinewidth{0.000000pt}%
\definecolor{currentstroke}{rgb}{0.000000,0.000000,0.000000}%
\pgfsetstrokecolor{currentstroke}%
\pgfsetstrokeopacity{0.000000}%
\pgfsetdash{}{0pt}%
\pgfpathmoveto{\pgfqpoint{2.749742in}{1.248131in}}%
\pgfpathlineto{\pgfqpoint{2.756185in}{1.251398in}}%
\pgfpathlineto{\pgfqpoint{2.761558in}{1.243339in}}%
\pgfpathlineto{\pgfqpoint{2.762895in}{1.235158in}}%
\pgfpathlineto{\pgfqpoint{2.769279in}{1.230057in}}%
\pgfpathlineto{\pgfqpoint{2.782260in}{1.231360in}}%
\pgfpathlineto{\pgfqpoint{2.780353in}{1.222630in}}%
\pgfpathlineto{\pgfqpoint{2.776519in}{1.219287in}}%
\pgfpathlineto{\pgfqpoint{2.773866in}{1.220848in}}%
\pgfpathlineto{\pgfqpoint{2.756011in}{1.199988in}}%
\pgfpathlineto{\pgfqpoint{2.742229in}{1.190524in}}%
\pgfpathlineto{\pgfqpoint{2.735619in}{1.191972in}}%
\pgfpathlineto{\pgfqpoint{2.729982in}{1.197519in}}%
\pgfpathlineto{\pgfqpoint{2.727914in}{1.206079in}}%
\pgfpathlineto{\pgfqpoint{2.716523in}{1.210261in}}%
\pgfpathlineto{\pgfqpoint{2.706346in}{1.218031in}}%
\pgfpathlineto{\pgfqpoint{2.696544in}{1.221009in}}%
\pgfpathlineto{\pgfqpoint{2.695042in}{1.227310in}}%
\pgfpathlineto{\pgfqpoint{2.704471in}{1.237923in}}%
\pgfpathlineto{\pgfqpoint{2.711636in}{1.239856in}}%
\pgfpathlineto{\pgfqpoint{2.707210in}{1.247120in}}%
\pgfpathlineto{\pgfqpoint{2.713971in}{1.255931in}}%
\pgfpathlineto{\pgfqpoint{2.711546in}{1.260678in}}%
\pgfpathlineto{\pgfqpoint{2.717713in}{1.266128in}}%
\pgfpathlineto{\pgfqpoint{2.729310in}{1.268475in}}%
\pgfpathlineto{\pgfqpoint{2.728388in}{1.259951in}}%
\pgfpathlineto{\pgfqpoint{2.740080in}{1.249814in}}%
\pgfpathlineto{\pgfqpoint{2.738831in}{1.245353in}}%
\pgfpathlineto{\pgfqpoint{2.744881in}{1.241341in}}%
\pgfpathclose%
\pgfusepath{fill}%
\end{pgfscope}%
\begin{pgfscope}%
\pgfpathrectangle{\pgfqpoint{0.100000in}{0.100000in}}{\pgfqpoint{3.420221in}{2.189500in}}%
\pgfusepath{clip}%
\pgfsetbuttcap%
\pgfsetmiterjoin%
\definecolor{currentfill}{rgb}{0.000000,0.349020,0.825490}%
\pgfsetfillcolor{currentfill}%
\pgfsetlinewidth{0.000000pt}%
\definecolor{currentstroke}{rgb}{0.000000,0.000000,0.000000}%
\pgfsetstrokecolor{currentstroke}%
\pgfsetstrokeopacity{0.000000}%
\pgfsetdash{}{0pt}%
\pgfpathmoveto{\pgfqpoint{2.084673in}{1.568690in}}%
\pgfpathlineto{\pgfqpoint{2.058667in}{1.568271in}}%
\pgfpathlineto{\pgfqpoint{2.058209in}{1.612731in}}%
\pgfpathlineto{\pgfqpoint{2.075764in}{1.612957in}}%
\pgfpathlineto{\pgfqpoint{2.109782in}{1.613581in}}%
\pgfpathlineto{\pgfqpoint{2.110721in}{1.569135in}}%
\pgfpathclose%
\pgfusepath{fill}%
\end{pgfscope}%
\begin{pgfscope}%
\pgfpathrectangle{\pgfqpoint{0.100000in}{0.100000in}}{\pgfqpoint{3.420221in}{2.189500in}}%
\pgfusepath{clip}%
\pgfsetbuttcap%
\pgfsetmiterjoin%
\definecolor{currentfill}{rgb}{0.000000,0.745098,0.627451}%
\pgfsetfillcolor{currentfill}%
\pgfsetlinewidth{0.000000pt}%
\definecolor{currentstroke}{rgb}{0.000000,0.000000,0.000000}%
\pgfsetstrokecolor{currentstroke}%
\pgfsetstrokeopacity{0.000000}%
\pgfsetdash{}{0pt}%
\pgfpathmoveto{\pgfqpoint{2.788194in}{1.304363in}}%
\pgfpathlineto{\pgfqpoint{2.782440in}{1.299093in}}%
\pgfpathlineto{\pgfqpoint{2.772938in}{1.307263in}}%
\pgfpathlineto{\pgfqpoint{2.763451in}{1.319695in}}%
\pgfpathlineto{\pgfqpoint{2.769799in}{1.332126in}}%
\pgfpathlineto{\pgfqpoint{2.768483in}{1.344419in}}%
\pgfpathlineto{\pgfqpoint{2.769478in}{1.351324in}}%
\pgfpathlineto{\pgfqpoint{2.761394in}{1.353807in}}%
\pgfpathlineto{\pgfqpoint{2.760459in}{1.363673in}}%
\pgfpathlineto{\pgfqpoint{2.761224in}{1.367123in}}%
\pgfpathlineto{\pgfqpoint{2.767926in}{1.367099in}}%
\pgfpathlineto{\pgfqpoint{2.767850in}{1.375398in}}%
\pgfpathlineto{\pgfqpoint{2.780231in}{1.376695in}}%
\pgfpathlineto{\pgfqpoint{2.790847in}{1.380001in}}%
\pgfpathlineto{\pgfqpoint{2.793219in}{1.378335in}}%
\pgfpathlineto{\pgfqpoint{2.805285in}{1.379664in}}%
\pgfpathlineto{\pgfqpoint{2.803000in}{1.373584in}}%
\pgfpathlineto{\pgfqpoint{2.813098in}{1.366811in}}%
\pgfpathlineto{\pgfqpoint{2.816952in}{1.360416in}}%
\pgfpathlineto{\pgfqpoint{2.815761in}{1.358611in}}%
\pgfpathlineto{\pgfqpoint{2.823445in}{1.347414in}}%
\pgfpathlineto{\pgfqpoint{2.818111in}{1.341570in}}%
\pgfpathlineto{\pgfqpoint{2.811916in}{1.338045in}}%
\pgfpathlineto{\pgfqpoint{2.805595in}{1.339763in}}%
\pgfpathlineto{\pgfqpoint{2.795340in}{1.326870in}}%
\pgfpathlineto{\pgfqpoint{2.788819in}{1.329372in}}%
\pgfpathlineto{\pgfqpoint{2.784613in}{1.325204in}}%
\pgfpathlineto{\pgfqpoint{2.786119in}{1.308571in}}%
\pgfpathclose%
\pgfusepath{fill}%
\end{pgfscope}%
\begin{pgfscope}%
\pgfpathrectangle{\pgfqpoint{0.100000in}{0.100000in}}{\pgfqpoint{3.420221in}{2.189500in}}%
\pgfusepath{clip}%
\pgfsetbuttcap%
\pgfsetmiterjoin%
\definecolor{currentfill}{rgb}{0.000000,0.313725,0.843137}%
\pgfsetfillcolor{currentfill}%
\pgfsetlinewidth{0.000000pt}%
\definecolor{currentstroke}{rgb}{0.000000,0.000000,0.000000}%
\pgfsetstrokecolor{currentstroke}%
\pgfsetstrokeopacity{0.000000}%
\pgfsetdash{}{0pt}%
\pgfpathmoveto{\pgfqpoint{2.146061in}{1.692855in}}%
\pgfpathlineto{\pgfqpoint{2.155763in}{1.693044in}}%
\pgfpathlineto{\pgfqpoint{2.154340in}{1.736830in}}%
\pgfpathlineto{\pgfqpoint{2.153048in}{1.743437in}}%
\pgfpathlineto{\pgfqpoint{2.185649in}{1.744345in}}%
\pgfpathlineto{\pgfqpoint{2.185452in}{1.750784in}}%
\pgfpathlineto{\pgfqpoint{2.218087in}{1.752145in}}%
\pgfpathlineto{\pgfqpoint{2.220845in}{1.693291in}}%
\pgfpathlineto{\pgfqpoint{2.227347in}{1.693562in}}%
\pgfpathlineto{\pgfqpoint{2.227636in}{1.687026in}}%
\pgfpathlineto{\pgfqpoint{2.253718in}{1.688461in}}%
\pgfpathlineto{\pgfqpoint{2.254616in}{1.675253in}}%
\pgfpathlineto{\pgfqpoint{2.254950in}{1.668229in}}%
\pgfpathlineto{\pgfqpoint{2.222945in}{1.666910in}}%
\pgfpathlineto{\pgfqpoint{2.219627in}{1.660171in}}%
\pgfpathlineto{\pgfqpoint{2.201593in}{1.658963in}}%
\pgfpathlineto{\pgfqpoint{2.201046in}{1.652886in}}%
\pgfpathlineto{\pgfqpoint{2.195475in}{1.652653in}}%
\pgfpathlineto{\pgfqpoint{2.203523in}{1.642647in}}%
\pgfpathlineto{\pgfqpoint{2.179344in}{1.641708in}}%
\pgfpathlineto{\pgfqpoint{2.160439in}{1.641074in}}%
\pgfpathlineto{\pgfqpoint{2.159599in}{1.667078in}}%
\pgfpathlineto{\pgfqpoint{2.171519in}{1.667473in}}%
\pgfpathlineto{\pgfqpoint{2.167531in}{1.677916in}}%
\pgfpathlineto{\pgfqpoint{2.160322in}{1.682886in}}%
\pgfpathlineto{\pgfqpoint{2.151714in}{1.685307in}}%
\pgfpathclose%
\pgfusepath{fill}%
\end{pgfscope}%
\begin{pgfscope}%
\pgfpathrectangle{\pgfqpoint{0.100000in}{0.100000in}}{\pgfqpoint{3.420221in}{2.189500in}}%
\pgfusepath{clip}%
\pgfsetbuttcap%
\pgfsetmiterjoin%
\definecolor{currentfill}{rgb}{0.000000,0.600000,0.700000}%
\pgfsetfillcolor{currentfill}%
\pgfsetlinewidth{0.000000pt}%
\definecolor{currentstroke}{rgb}{0.000000,0.000000,0.000000}%
\pgfsetstrokecolor{currentstroke}%
\pgfsetstrokeopacity{0.000000}%
\pgfsetdash{}{0pt}%
\pgfpathmoveto{\pgfqpoint{1.842914in}{0.477536in}}%
\pgfpathlineto{\pgfqpoint{1.829500in}{0.458168in}}%
\pgfpathlineto{\pgfqpoint{1.812796in}{0.462255in}}%
\pgfpathlineto{\pgfqpoint{1.813420in}{0.462742in}}%
\pgfpathlineto{\pgfqpoint{1.795792in}{0.499865in}}%
\pgfpathlineto{\pgfqpoint{1.801502in}{0.501749in}}%
\pgfpathlineto{\pgfqpoint{1.789621in}{0.516780in}}%
\pgfpathlineto{\pgfqpoint{1.821187in}{0.541571in}}%
\pgfpathlineto{\pgfqpoint{1.828587in}{0.532838in}}%
\pgfpathlineto{\pgfqpoint{1.818925in}{0.525316in}}%
\pgfpathlineto{\pgfqpoint{1.830503in}{0.510228in}}%
\pgfpathlineto{\pgfqpoint{1.816568in}{0.499575in}}%
\pgfpathlineto{\pgfqpoint{1.821765in}{0.489803in}}%
\pgfpathlineto{\pgfqpoint{1.830925in}{0.487448in}}%
\pgfpathlineto{\pgfqpoint{1.831242in}{0.482620in}}%
\pgfpathclose%
\pgfusepath{fill}%
\end{pgfscope}%
\begin{pgfscope}%
\pgfpathrectangle{\pgfqpoint{0.100000in}{0.100000in}}{\pgfqpoint{3.420221in}{2.189500in}}%
\pgfusepath{clip}%
\pgfsetbuttcap%
\pgfsetmiterjoin%
\definecolor{currentfill}{rgb}{0.000000,0.462745,0.768627}%
\pgfsetfillcolor{currentfill}%
\pgfsetlinewidth{0.000000pt}%
\definecolor{currentstroke}{rgb}{0.000000,0.000000,0.000000}%
\pgfsetstrokecolor{currentstroke}%
\pgfsetstrokeopacity{0.000000}%
\pgfsetdash{}{0pt}%
\pgfpathmoveto{\pgfqpoint{2.848715in}{0.890873in}}%
\pgfpathlineto{\pgfqpoint{2.836100in}{0.878627in}}%
\pgfpathlineto{\pgfqpoint{2.829978in}{0.881128in}}%
\pgfpathlineto{\pgfqpoint{2.816374in}{0.876340in}}%
\pgfpathlineto{\pgfqpoint{2.813246in}{0.868503in}}%
\pgfpathlineto{\pgfqpoint{2.809963in}{0.866837in}}%
\pgfpathlineto{\pgfqpoint{2.802257in}{0.868513in}}%
\pgfpathlineto{\pgfqpoint{2.795897in}{0.861685in}}%
\pgfpathlineto{\pgfqpoint{2.789398in}{0.866209in}}%
\pgfpathlineto{\pgfqpoint{2.789476in}{0.872624in}}%
\pgfpathlineto{\pgfqpoint{2.785489in}{0.879401in}}%
\pgfpathlineto{\pgfqpoint{2.777926in}{0.887090in}}%
\pgfpathlineto{\pgfqpoint{2.772549in}{0.889816in}}%
\pgfpathlineto{\pgfqpoint{2.771349in}{0.894513in}}%
\pgfpathlineto{\pgfqpoint{2.762814in}{0.908932in}}%
\pgfpathlineto{\pgfqpoint{2.763936in}{0.913553in}}%
\pgfpathlineto{\pgfqpoint{2.768566in}{0.914838in}}%
\pgfpathlineto{\pgfqpoint{2.771940in}{0.922077in}}%
\pgfpathlineto{\pgfqpoint{2.775653in}{0.925144in}}%
\pgfpathlineto{\pgfqpoint{2.784669in}{0.926708in}}%
\pgfpathlineto{\pgfqpoint{2.794075in}{0.932781in}}%
\pgfpathlineto{\pgfqpoint{2.799666in}{0.932480in}}%
\pgfpathlineto{\pgfqpoint{2.805960in}{0.926095in}}%
\pgfpathlineto{\pgfqpoint{2.807851in}{0.929674in}}%
\pgfpathlineto{\pgfqpoint{2.801484in}{0.935623in}}%
\pgfpathlineto{\pgfqpoint{2.804628in}{0.944072in}}%
\pgfpathlineto{\pgfqpoint{2.800993in}{0.951131in}}%
\pgfpathlineto{\pgfqpoint{2.809775in}{0.954853in}}%
\pgfpathlineto{\pgfqpoint{2.816623in}{0.956882in}}%
\pgfpathlineto{\pgfqpoint{2.821373in}{0.949289in}}%
\pgfpathlineto{\pgfqpoint{2.833375in}{0.947100in}}%
\pgfpathlineto{\pgfqpoint{2.837647in}{0.952515in}}%
\pgfpathlineto{\pgfqpoint{2.848095in}{0.943859in}}%
\pgfpathlineto{\pgfqpoint{2.863931in}{0.939400in}}%
\pgfpathlineto{\pgfqpoint{2.854258in}{0.925569in}}%
\pgfpathlineto{\pgfqpoint{2.833606in}{0.900247in}}%
\pgfpathlineto{\pgfqpoint{2.837770in}{0.894641in}}%
\pgfpathclose%
\pgfusepath{fill}%
\end{pgfscope}%
\begin{pgfscope}%
\pgfpathrectangle{\pgfqpoint{0.100000in}{0.100000in}}{\pgfqpoint{3.420221in}{2.189500in}}%
\pgfusepath{clip}%
\pgfsetbuttcap%
\pgfsetmiterjoin%
\definecolor{currentfill}{rgb}{0.000000,0.329412,0.835294}%
\pgfsetfillcolor{currentfill}%
\pgfsetlinewidth{0.000000pt}%
\definecolor{currentstroke}{rgb}{0.000000,0.000000,0.000000}%
\pgfsetstrokecolor{currentstroke}%
\pgfsetstrokeopacity{0.000000}%
\pgfsetdash{}{0pt}%
\pgfpathmoveto{\pgfqpoint{3.079049in}{1.036839in}}%
\pgfpathlineto{\pgfqpoint{3.074132in}{1.040535in}}%
\pgfpathlineto{\pgfqpoint{3.065131in}{1.055667in}}%
\pgfpathlineto{\pgfqpoint{3.064024in}{1.071002in}}%
\pgfpathlineto{\pgfqpoint{3.056886in}{1.077170in}}%
\pgfpathlineto{\pgfqpoint{3.053733in}{1.085984in}}%
\pgfpathlineto{\pgfqpoint{3.049364in}{1.087605in}}%
\pgfpathlineto{\pgfqpoint{3.044436in}{1.116845in}}%
\pgfpathlineto{\pgfqpoint{3.066652in}{1.141039in}}%
\pgfpathlineto{\pgfqpoint{3.069212in}{1.140495in}}%
\pgfpathlineto{\pgfqpoint{3.080520in}{1.136733in}}%
\pgfpathlineto{\pgfqpoint{3.082447in}{1.129067in}}%
\pgfpathlineto{\pgfqpoint{3.086444in}{1.123595in}}%
\pgfpathlineto{\pgfqpoint{3.084017in}{1.120469in}}%
\pgfpathlineto{\pgfqpoint{3.084622in}{1.112919in}}%
\pgfpathlineto{\pgfqpoint{3.100224in}{1.107012in}}%
\pgfpathlineto{\pgfqpoint{3.108335in}{1.102137in}}%
\pgfpathlineto{\pgfqpoint{3.120844in}{1.106154in}}%
\pgfpathlineto{\pgfqpoint{3.120643in}{1.113708in}}%
\pgfpathlineto{\pgfqpoint{3.128530in}{1.113790in}}%
\pgfpathlineto{\pgfqpoint{3.130372in}{1.110405in}}%
\pgfpathlineto{\pgfqpoint{3.126684in}{1.096734in}}%
\pgfpathlineto{\pgfqpoint{3.122271in}{1.091634in}}%
\pgfpathlineto{\pgfqpoint{3.125754in}{1.088032in}}%
\pgfpathlineto{\pgfqpoint{3.135993in}{1.091885in}}%
\pgfpathlineto{\pgfqpoint{3.141039in}{1.090493in}}%
\pgfpathlineto{\pgfqpoint{3.145111in}{1.093781in}}%
\pgfpathlineto{\pgfqpoint{3.152390in}{1.093564in}}%
\pgfpathlineto{\pgfqpoint{3.138594in}{1.067336in}}%
\pgfpathlineto{\pgfqpoint{3.133807in}{1.064142in}}%
\pgfpathlineto{\pgfqpoint{3.126545in}{1.065556in}}%
\pgfpathlineto{\pgfqpoint{3.119008in}{1.064408in}}%
\pgfpathlineto{\pgfqpoint{3.098391in}{1.055120in}}%
\pgfpathclose%
\pgfusepath{fill}%
\end{pgfscope}%
\begin{pgfscope}%
\pgfpathrectangle{\pgfqpoint{0.100000in}{0.100000in}}{\pgfqpoint{3.420221in}{2.189500in}}%
\pgfusepath{clip}%
\pgfsetbuttcap%
\pgfsetmiterjoin%
\definecolor{currentfill}{rgb}{0.000000,0.305882,0.847059}%
\pgfsetfillcolor{currentfill}%
\pgfsetlinewidth{0.000000pt}%
\definecolor{currentstroke}{rgb}{0.000000,0.000000,0.000000}%
\pgfsetstrokecolor{currentstroke}%
\pgfsetstrokeopacity{0.000000}%
\pgfsetdash{}{0pt}%
\pgfpathmoveto{\pgfqpoint{1.671009in}{0.689648in}}%
\pgfpathlineto{\pgfqpoint{1.669335in}{0.661125in}}%
\pgfpathlineto{\pgfqpoint{1.614467in}{0.664147in}}%
\pgfpathlineto{\pgfqpoint{1.616232in}{0.692573in}}%
\pgfpathlineto{\pgfqpoint{1.596014in}{0.693701in}}%
\pgfpathlineto{\pgfqpoint{1.598797in}{0.727554in}}%
\pgfpathlineto{\pgfqpoint{1.601942in}{0.769605in}}%
\pgfpathlineto{\pgfqpoint{1.607058in}{0.769259in}}%
\pgfpathlineto{\pgfqpoint{1.640138in}{0.766963in}}%
\pgfpathlineto{\pgfqpoint{1.667580in}{0.765086in}}%
\pgfpathlineto{\pgfqpoint{1.665435in}{0.727322in}}%
\pgfpathlineto{\pgfqpoint{1.673358in}{0.726783in}}%
\pgfpathclose%
\pgfusepath{fill}%
\end{pgfscope}%
\begin{pgfscope}%
\pgfpathrectangle{\pgfqpoint{0.100000in}{0.100000in}}{\pgfqpoint{3.420221in}{2.189500in}}%
\pgfusepath{clip}%
\pgfsetbuttcap%
\pgfsetmiterjoin%
\definecolor{currentfill}{rgb}{0.000000,0.415686,0.792157}%
\pgfsetfillcolor{currentfill}%
\pgfsetlinewidth{0.000000pt}%
\definecolor{currentstroke}{rgb}{0.000000,0.000000,0.000000}%
\pgfsetstrokecolor{currentstroke}%
\pgfsetstrokeopacity{0.000000}%
\pgfsetdash{}{0pt}%
\pgfpathmoveto{\pgfqpoint{2.696353in}{0.952258in}}%
\pgfpathlineto{\pgfqpoint{2.693513in}{0.949691in}}%
\pgfpathlineto{\pgfqpoint{2.685111in}{0.956009in}}%
\pgfpathlineto{\pgfqpoint{2.688203in}{0.962761in}}%
\pgfpathlineto{\pgfqpoint{2.683704in}{0.972759in}}%
\pgfpathlineto{\pgfqpoint{2.676351in}{0.974994in}}%
\pgfpathlineto{\pgfqpoint{2.669449in}{0.980294in}}%
\pgfpathlineto{\pgfqpoint{2.669197in}{0.984930in}}%
\pgfpathlineto{\pgfqpoint{2.670692in}{0.988606in}}%
\pgfpathlineto{\pgfqpoint{2.678182in}{0.988987in}}%
\pgfpathlineto{\pgfqpoint{2.683256in}{0.997095in}}%
\pgfpathlineto{\pgfqpoint{2.688489in}{0.996322in}}%
\pgfpathlineto{\pgfqpoint{2.690669in}{1.001428in}}%
\pgfpathlineto{\pgfqpoint{2.700065in}{1.005393in}}%
\pgfpathlineto{\pgfqpoint{2.704040in}{1.005711in}}%
\pgfpathlineto{\pgfqpoint{2.706921in}{1.000730in}}%
\pgfpathlineto{\pgfqpoint{2.719427in}{0.999752in}}%
\pgfpathlineto{\pgfqpoint{2.720569in}{0.997703in}}%
\pgfpathlineto{\pgfqpoint{2.719033in}{0.996619in}}%
\pgfpathlineto{\pgfqpoint{2.715081in}{0.981409in}}%
\pgfpathlineto{\pgfqpoint{2.720306in}{0.978149in}}%
\pgfpathlineto{\pgfqpoint{2.720633in}{0.970266in}}%
\pgfpathlineto{\pgfqpoint{2.723978in}{0.963056in}}%
\pgfpathlineto{\pgfqpoint{2.721313in}{0.960704in}}%
\pgfpathlineto{\pgfqpoint{2.724901in}{0.955002in}}%
\pgfpathlineto{\pgfqpoint{2.723685in}{0.949844in}}%
\pgfpathlineto{\pgfqpoint{2.715193in}{0.942368in}}%
\pgfpathlineto{\pgfqpoint{2.712974in}{0.947085in}}%
\pgfpathlineto{\pgfqpoint{2.700168in}{0.948015in}}%
\pgfpathclose%
\pgfusepath{fill}%
\end{pgfscope}%
\begin{pgfscope}%
\pgfpathrectangle{\pgfqpoint{0.100000in}{0.100000in}}{\pgfqpoint{3.420221in}{2.189500in}}%
\pgfusepath{clip}%
\pgfsetbuttcap%
\pgfsetmiterjoin%
\definecolor{currentfill}{rgb}{0.000000,0.321569,0.839216}%
\pgfsetfillcolor{currentfill}%
\pgfsetlinewidth{0.000000pt}%
\definecolor{currentstroke}{rgb}{0.000000,0.000000,0.000000}%
\pgfsetstrokecolor{currentstroke}%
\pgfsetstrokeopacity{0.000000}%
\pgfsetdash{}{0pt}%
\pgfpathmoveto{\pgfqpoint{1.972284in}{1.232687in}}%
\pgfpathlineto{\pgfqpoint{1.946339in}{1.232897in}}%
\pgfpathlineto{\pgfqpoint{1.946836in}{1.255712in}}%
\pgfpathlineto{\pgfqpoint{1.923073in}{1.256037in}}%
\pgfpathlineto{\pgfqpoint{1.920956in}{1.262554in}}%
\pgfpathlineto{\pgfqpoint{1.914474in}{1.262689in}}%
\pgfpathlineto{\pgfqpoint{1.914770in}{1.278975in}}%
\pgfpathlineto{\pgfqpoint{1.921193in}{1.278845in}}%
\pgfpathlineto{\pgfqpoint{1.921450in}{1.288539in}}%
\pgfpathlineto{\pgfqpoint{1.929888in}{1.291060in}}%
\pgfpathlineto{\pgfqpoint{1.933785in}{1.289556in}}%
\pgfpathlineto{\pgfqpoint{1.942005in}{1.291546in}}%
\pgfpathlineto{\pgfqpoint{1.942281in}{1.317735in}}%
\pgfpathlineto{\pgfqpoint{1.956522in}{1.317588in}}%
\pgfpathlineto{\pgfqpoint{1.956599in}{1.324101in}}%
\pgfpathlineto{\pgfqpoint{1.969567in}{1.324003in}}%
\pgfpathlineto{\pgfqpoint{1.969091in}{1.306486in}}%
\pgfpathlineto{\pgfqpoint{1.991707in}{1.306381in}}%
\pgfpathlineto{\pgfqpoint{1.991648in}{1.284651in}}%
\pgfpathlineto{\pgfqpoint{1.992674in}{1.273579in}}%
\pgfpathlineto{\pgfqpoint{1.998833in}{1.273645in}}%
\pgfpathlineto{\pgfqpoint{1.998810in}{1.255392in}}%
\pgfpathlineto{\pgfqpoint{1.972831in}{1.255474in}}%
\pgfpathclose%
\pgfusepath{fill}%
\end{pgfscope}%
\begin{pgfscope}%
\pgfpathrectangle{\pgfqpoint{0.100000in}{0.100000in}}{\pgfqpoint{3.420221in}{2.189500in}}%
\pgfusepath{clip}%
\pgfsetbuttcap%
\pgfsetmiterjoin%
\definecolor{currentfill}{rgb}{0.000000,0.478431,0.760784}%
\pgfsetfillcolor{currentfill}%
\pgfsetlinewidth{0.000000pt}%
\definecolor{currentstroke}{rgb}{0.000000,0.000000,0.000000}%
\pgfsetstrokecolor{currentstroke}%
\pgfsetstrokeopacity{0.000000}%
\pgfsetdash{}{0pt}%
\pgfpathmoveto{\pgfqpoint{1.924552in}{0.570650in}}%
\pgfpathlineto{\pgfqpoint{1.915707in}{0.561716in}}%
\pgfpathlineto{\pgfqpoint{1.912881in}{0.553688in}}%
\pgfpathlineto{\pgfqpoint{1.893366in}{0.541816in}}%
\pgfpathlineto{\pgfqpoint{1.898767in}{0.548381in}}%
\pgfpathlineto{\pgfqpoint{1.883739in}{0.567024in}}%
\pgfpathlineto{\pgfqpoint{1.878464in}{0.571343in}}%
\pgfpathlineto{\pgfqpoint{1.860710in}{0.571385in}}%
\pgfpathlineto{\pgfqpoint{1.849494in}{0.583546in}}%
\pgfpathlineto{\pgfqpoint{1.869261in}{0.603314in}}%
\pgfpathlineto{\pgfqpoint{1.881080in}{0.607961in}}%
\pgfpathlineto{\pgfqpoint{1.884510in}{0.611267in}}%
\pgfpathlineto{\pgfqpoint{1.891780in}{0.610408in}}%
\pgfpathlineto{\pgfqpoint{1.901146in}{0.591759in}}%
\pgfpathlineto{\pgfqpoint{1.909084in}{0.585378in}}%
\pgfpathlineto{\pgfqpoint{1.914520in}{0.584539in}}%
\pgfpathlineto{\pgfqpoint{1.919172in}{0.573261in}}%
\pgfpathclose%
\pgfusepath{fill}%
\end{pgfscope}%
\begin{pgfscope}%
\pgfpathrectangle{\pgfqpoint{0.100000in}{0.100000in}}{\pgfqpoint{3.420221in}{2.189500in}}%
\pgfusepath{clip}%
\pgfsetbuttcap%
\pgfsetmiterjoin%
\definecolor{currentfill}{rgb}{0.000000,0.580392,0.709804}%
\pgfsetfillcolor{currentfill}%
\pgfsetlinewidth{0.000000pt}%
\definecolor{currentstroke}{rgb}{0.000000,0.000000,0.000000}%
\pgfsetstrokecolor{currentstroke}%
\pgfsetstrokeopacity{0.000000}%
\pgfsetdash{}{0pt}%
\pgfpathmoveto{\pgfqpoint{0.721574in}{1.509614in}}%
\pgfpathlineto{\pgfqpoint{0.662413in}{1.524753in}}%
\pgfpathlineto{\pgfqpoint{0.621773in}{1.535754in}}%
\pgfpathlineto{\pgfqpoint{0.632112in}{1.573260in}}%
\pgfpathlineto{\pgfqpoint{0.633521in}{1.572913in}}%
\pgfpathlineto{\pgfqpoint{0.642468in}{1.604815in}}%
\pgfpathlineto{\pgfqpoint{0.641370in}{1.605121in}}%
\pgfpathlineto{\pgfqpoint{0.650723in}{1.637863in}}%
\pgfpathlineto{\pgfqpoint{0.662174in}{1.679793in}}%
\pgfpathlineto{\pgfqpoint{0.695911in}{1.670479in}}%
\pgfpathlineto{\pgfqpoint{0.722921in}{1.663756in}}%
\pgfpathlineto{\pgfqpoint{0.786248in}{1.647919in}}%
\pgfpathlineto{\pgfqpoint{0.786661in}{1.647747in}}%
\pgfpathlineto{\pgfqpoint{0.768679in}{1.575019in}}%
\pgfpathlineto{\pgfqpoint{0.762241in}{1.549030in}}%
\pgfpathlineto{\pgfqpoint{0.749641in}{1.552115in}}%
\pgfpathclose%
\pgfusepath{fill}%
\end{pgfscope}%
\begin{pgfscope}%
\pgfpathrectangle{\pgfqpoint{0.100000in}{0.100000in}}{\pgfqpoint{3.420221in}{2.189500in}}%
\pgfusepath{clip}%
\pgfsetbuttcap%
\pgfsetmiterjoin%
\definecolor{currentfill}{rgb}{0.000000,0.149020,0.925490}%
\pgfsetfillcolor{currentfill}%
\pgfsetlinewidth{0.000000pt}%
\definecolor{currentstroke}{rgb}{0.000000,0.000000,0.000000}%
\pgfsetstrokecolor{currentstroke}%
\pgfsetstrokeopacity{0.000000}%
\pgfsetdash{}{0pt}%
\pgfpathmoveto{\pgfqpoint{1.891017in}{1.558192in}}%
\pgfpathlineto{\pgfqpoint{1.890231in}{1.527418in}}%
\pgfpathlineto{\pgfqpoint{1.870822in}{1.527937in}}%
\pgfpathlineto{\pgfqpoint{1.871003in}{1.534471in}}%
\pgfpathlineto{\pgfqpoint{1.819375in}{1.535927in}}%
\pgfpathlineto{\pgfqpoint{1.820061in}{1.569379in}}%
\pgfpathlineto{\pgfqpoint{1.827563in}{1.565694in}}%
\pgfpathlineto{\pgfqpoint{1.830527in}{1.568438in}}%
\pgfpathlineto{\pgfqpoint{1.832222in}{1.575813in}}%
\pgfpathlineto{\pgfqpoint{1.831514in}{1.583314in}}%
\pgfpathlineto{\pgfqpoint{1.833414in}{1.590379in}}%
\pgfpathlineto{\pgfqpoint{1.870695in}{1.589338in}}%
\pgfpathlineto{\pgfqpoint{1.883782in}{1.589055in}}%
\pgfpathlineto{\pgfqpoint{1.883118in}{1.561255in}}%
\pgfpathclose%
\pgfusepath{fill}%
\end{pgfscope}%
\begin{pgfscope}%
\pgfpathrectangle{\pgfqpoint{0.100000in}{0.100000in}}{\pgfqpoint{3.420221in}{2.189500in}}%
\pgfusepath{clip}%
\pgfsetbuttcap%
\pgfsetmiterjoin%
\definecolor{currentfill}{rgb}{0.000000,0.494118,0.752941}%
\pgfsetfillcolor{currentfill}%
\pgfsetlinewidth{0.000000pt}%
\definecolor{currentstroke}{rgb}{0.000000,0.000000,0.000000}%
\pgfsetstrokecolor{currentstroke}%
\pgfsetstrokeopacity{0.000000}%
\pgfsetdash{}{0pt}%
\pgfpathmoveto{\pgfqpoint{3.029806in}{1.115037in}}%
\pgfpathlineto{\pgfqpoint{3.044436in}{1.116845in}}%
\pgfpathlineto{\pgfqpoint{3.049364in}{1.087605in}}%
\pgfpathlineto{\pgfqpoint{3.053733in}{1.085984in}}%
\pgfpathlineto{\pgfqpoint{3.056886in}{1.077170in}}%
\pgfpathlineto{\pgfqpoint{3.064024in}{1.071002in}}%
\pgfpathlineto{\pgfqpoint{3.065131in}{1.055667in}}%
\pgfpathlineto{\pgfqpoint{3.038870in}{1.050770in}}%
\pgfpathlineto{\pgfqpoint{3.032635in}{1.037417in}}%
\pgfpathlineto{\pgfqpoint{3.021614in}{1.049679in}}%
\pgfpathlineto{\pgfqpoint{3.013882in}{1.056385in}}%
\pgfpathlineto{\pgfqpoint{3.003062in}{1.064569in}}%
\pgfpathlineto{\pgfqpoint{3.001020in}{1.072920in}}%
\pgfpathlineto{\pgfqpoint{3.001215in}{1.083802in}}%
\pgfpathlineto{\pgfqpoint{3.004846in}{1.089773in}}%
\pgfpathlineto{\pgfqpoint{3.008560in}{1.086495in}}%
\pgfpathlineto{\pgfqpoint{3.013544in}{1.086877in}}%
\pgfpathlineto{\pgfqpoint{3.020139in}{1.092579in}}%
\pgfpathlineto{\pgfqpoint{3.027731in}{1.096697in}}%
\pgfpathlineto{\pgfqpoint{3.026711in}{1.101640in}}%
\pgfpathclose%
\pgfusepath{fill}%
\end{pgfscope}%
\begin{pgfscope}%
\pgfpathrectangle{\pgfqpoint{0.100000in}{0.100000in}}{\pgfqpoint{3.420221in}{2.189500in}}%
\pgfusepath{clip}%
\pgfsetbuttcap%
\pgfsetmiterjoin%
\definecolor{currentfill}{rgb}{0.000000,0.529412,0.735294}%
\pgfsetfillcolor{currentfill}%
\pgfsetlinewidth{0.000000pt}%
\definecolor{currentstroke}{rgb}{0.000000,0.000000,0.000000}%
\pgfsetstrokecolor{currentstroke}%
\pgfsetstrokeopacity{0.000000}%
\pgfsetdash{}{0pt}%
\pgfpathmoveto{\pgfqpoint{2.646694in}{0.714784in}}%
\pgfpathlineto{\pgfqpoint{2.649003in}{0.709343in}}%
\pgfpathlineto{\pgfqpoint{2.582467in}{0.701816in}}%
\pgfpathlineto{\pgfqpoint{2.572640in}{0.700905in}}%
\pgfpathlineto{\pgfqpoint{2.568235in}{0.741184in}}%
\pgfpathlineto{\pgfqpoint{2.570723in}{0.748090in}}%
\pgfpathlineto{\pgfqpoint{2.596171in}{0.750656in}}%
\pgfpathlineto{\pgfqpoint{2.617388in}{0.752971in}}%
\pgfpathlineto{\pgfqpoint{2.616720in}{0.759523in}}%
\pgfpathlineto{\pgfqpoint{2.635374in}{0.760639in}}%
\pgfpathlineto{\pgfqpoint{2.640352in}{0.755383in}}%
\pgfpathlineto{\pgfqpoint{2.642010in}{0.749780in}}%
\pgfpathlineto{\pgfqpoint{2.639738in}{0.729706in}}%
\pgfpathlineto{\pgfqpoint{2.641342in}{0.721040in}}%
\pgfpathclose%
\pgfusepath{fill}%
\end{pgfscope}%
\begin{pgfscope}%
\pgfpathrectangle{\pgfqpoint{0.100000in}{0.100000in}}{\pgfqpoint{3.420221in}{2.189500in}}%
\pgfusepath{clip}%
\pgfsetbuttcap%
\pgfsetmiterjoin%
\definecolor{currentfill}{rgb}{0.000000,0.541176,0.729412}%
\pgfsetfillcolor{currentfill}%
\pgfsetlinewidth{0.000000pt}%
\definecolor{currentstroke}{rgb}{0.000000,0.000000,0.000000}%
\pgfsetstrokecolor{currentstroke}%
\pgfsetstrokeopacity{0.000000}%
\pgfsetdash{}{0pt}%
\pgfpathmoveto{\pgfqpoint{0.834795in}{2.044232in}}%
\pgfpathlineto{\pgfqpoint{0.784172in}{2.057684in}}%
\pgfpathlineto{\pgfqpoint{0.797951in}{2.107511in}}%
\pgfpathlineto{\pgfqpoint{0.804944in}{2.106385in}}%
\pgfpathlineto{\pgfqpoint{0.817519in}{2.101389in}}%
\pgfpathlineto{\pgfqpoint{0.822460in}{2.092825in}}%
\pgfpathlineto{\pgfqpoint{0.833521in}{2.098020in}}%
\pgfpathlineto{\pgfqpoint{0.837653in}{2.092309in}}%
\pgfpathlineto{\pgfqpoint{0.839236in}{2.085302in}}%
\pgfpathlineto{\pgfqpoint{0.849853in}{2.086301in}}%
\pgfpathlineto{\pgfqpoint{0.853029in}{2.082786in}}%
\pgfpathlineto{\pgfqpoint{0.860521in}{2.083830in}}%
\pgfpathlineto{\pgfqpoint{0.865871in}{2.078102in}}%
\pgfpathlineto{\pgfqpoint{0.870608in}{2.096681in}}%
\pgfpathlineto{\pgfqpoint{0.873835in}{2.102676in}}%
\pgfpathlineto{\pgfqpoint{0.883670in}{2.141040in}}%
\pgfpathlineto{\pgfqpoint{0.877357in}{2.142611in}}%
\pgfpathlineto{\pgfqpoint{0.882174in}{2.148183in}}%
\pgfpathlineto{\pgfqpoint{0.883931in}{2.155132in}}%
\pgfpathlineto{\pgfqpoint{0.893481in}{2.165550in}}%
\pgfpathlineto{\pgfqpoint{0.912703in}{2.160672in}}%
\pgfpathlineto{\pgfqpoint{0.909910in}{2.149399in}}%
\pgfpathlineto{\pgfqpoint{0.921948in}{2.146507in}}%
\pgfpathlineto{\pgfqpoint{0.915701in}{2.120890in}}%
\pgfpathlineto{\pgfqpoint{0.951744in}{2.112346in}}%
\pgfpathlineto{\pgfqpoint{0.946711in}{2.091138in}}%
\pgfpathlineto{\pgfqpoint{0.942612in}{2.073554in}}%
\pgfpathlineto{\pgfqpoint{0.949823in}{2.054332in}}%
\pgfpathlineto{\pgfqpoint{0.953909in}{2.048944in}}%
\pgfpathlineto{\pgfqpoint{0.953933in}{2.041118in}}%
\pgfpathlineto{\pgfqpoint{0.949737in}{2.038496in}}%
\pgfpathlineto{\pgfqpoint{0.954903in}{2.031949in}}%
\pgfpathlineto{\pgfqpoint{0.948286in}{2.030531in}}%
\pgfpathlineto{\pgfqpoint{0.956571in}{2.021476in}}%
\pgfpathlineto{\pgfqpoint{0.956801in}{2.017491in}}%
\pgfpathlineto{\pgfqpoint{0.966567in}{2.011955in}}%
\pgfpathlineto{\pgfqpoint{0.974880in}{1.989787in}}%
\pgfpathlineto{\pgfqpoint{0.979308in}{1.983961in}}%
\pgfpathlineto{\pgfqpoint{0.946515in}{1.991426in}}%
\pgfpathlineto{\pgfqpoint{0.910368in}{2.000022in}}%
\pgfpathlineto{\pgfqpoint{0.911915in}{2.006493in}}%
\pgfpathlineto{\pgfqpoint{0.905695in}{2.009090in}}%
\pgfpathlineto{\pgfqpoint{0.890041in}{2.012951in}}%
\pgfpathlineto{\pgfqpoint{0.878226in}{2.022853in}}%
\pgfpathlineto{\pgfqpoint{0.880612in}{2.032558in}}%
\pgfpathclose%
\pgfusepath{fill}%
\end{pgfscope}%
\begin{pgfscope}%
\pgfpathrectangle{\pgfqpoint{0.100000in}{0.100000in}}{\pgfqpoint{3.420221in}{2.189500in}}%
\pgfusepath{clip}%
\pgfsetbuttcap%
\pgfsetmiterjoin%
\definecolor{currentfill}{rgb}{0.000000,0.501961,0.749020}%
\pgfsetfillcolor{currentfill}%
\pgfsetlinewidth{0.000000pt}%
\definecolor{currentstroke}{rgb}{0.000000,0.000000,0.000000}%
\pgfsetstrokecolor{currentstroke}%
\pgfsetstrokeopacity{0.000000}%
\pgfsetdash{}{0pt}%
\pgfpathmoveto{\pgfqpoint{2.078696in}{0.574988in}}%
\pgfpathlineto{\pgfqpoint{2.061847in}{0.572417in}}%
\pgfpathlineto{\pgfqpoint{2.044751in}{0.564816in}}%
\pgfpathlineto{\pgfqpoint{2.044330in}{0.589549in}}%
\pgfpathlineto{\pgfqpoint{2.038788in}{0.589670in}}%
\pgfpathlineto{\pgfqpoint{2.038495in}{0.606615in}}%
\pgfpathlineto{\pgfqpoint{2.028579in}{0.606493in}}%
\pgfpathlineto{\pgfqpoint{2.019551in}{0.635100in}}%
\pgfpathlineto{\pgfqpoint{2.004343in}{0.641627in}}%
\pgfpathlineto{\pgfqpoint{2.000571in}{0.647961in}}%
\pgfpathlineto{\pgfqpoint{1.994620in}{0.648023in}}%
\pgfpathlineto{\pgfqpoint{1.993536in}{0.657106in}}%
\pgfpathlineto{\pgfqpoint{1.989101in}{0.660413in}}%
\pgfpathlineto{\pgfqpoint{2.006054in}{0.676853in}}%
\pgfpathlineto{\pgfqpoint{2.012282in}{0.684777in}}%
\pgfpathlineto{\pgfqpoint{2.018206in}{0.681698in}}%
\pgfpathlineto{\pgfqpoint{2.026559in}{0.680335in}}%
\pgfpathlineto{\pgfqpoint{2.037251in}{0.676307in}}%
\pgfpathlineto{\pgfqpoint{2.058468in}{0.681450in}}%
\pgfpathlineto{\pgfqpoint{2.072562in}{0.686048in}}%
\pgfpathlineto{\pgfqpoint{2.092651in}{0.687691in}}%
\pgfpathlineto{\pgfqpoint{2.095722in}{0.688451in}}%
\pgfpathlineto{\pgfqpoint{2.095179in}{0.673629in}}%
\pgfpathlineto{\pgfqpoint{2.097749in}{0.669154in}}%
\pgfpathlineto{\pgfqpoint{2.094929in}{0.664959in}}%
\pgfpathlineto{\pgfqpoint{2.091203in}{0.649972in}}%
\pgfpathlineto{\pgfqpoint{2.087882in}{0.647107in}}%
\pgfpathlineto{\pgfqpoint{2.084698in}{0.638772in}}%
\pgfpathlineto{\pgfqpoint{2.087142in}{0.631936in}}%
\pgfpathlineto{\pgfqpoint{2.082811in}{0.623705in}}%
\pgfpathlineto{\pgfqpoint{2.086854in}{0.620494in}}%
\pgfpathlineto{\pgfqpoint{2.086794in}{0.602613in}}%
\pgfpathlineto{\pgfqpoint{2.081616in}{0.597541in}}%
\pgfpathlineto{\pgfqpoint{2.079746in}{0.591980in}}%
\pgfpathlineto{\pgfqpoint{2.072820in}{0.584301in}}%
\pgfpathclose%
\pgfusepath{fill}%
\end{pgfscope}%
\begin{pgfscope}%
\pgfpathrectangle{\pgfqpoint{0.100000in}{0.100000in}}{\pgfqpoint{3.420221in}{2.189500in}}%
\pgfusepath{clip}%
\pgfsetbuttcap%
\pgfsetmiterjoin%
\definecolor{currentfill}{rgb}{0.000000,0.172549,0.913725}%
\pgfsetfillcolor{currentfill}%
\pgfsetlinewidth{0.000000pt}%
\definecolor{currentstroke}{rgb}{0.000000,0.000000,0.000000}%
\pgfsetstrokecolor{currentstroke}%
\pgfsetstrokeopacity{0.000000}%
\pgfsetdash{}{0pt}%
\pgfpathmoveto{\pgfqpoint{3.054420in}{1.301559in}}%
\pgfpathlineto{\pgfqpoint{3.048193in}{1.299274in}}%
\pgfpathlineto{\pgfqpoint{3.040722in}{1.285463in}}%
\pgfpathlineto{\pgfqpoint{3.035720in}{1.287292in}}%
\pgfpathlineto{\pgfqpoint{3.033718in}{1.292115in}}%
\pgfpathlineto{\pgfqpoint{3.018032in}{1.297713in}}%
\pgfpathlineto{\pgfqpoint{3.008087in}{1.302911in}}%
\pgfpathlineto{\pgfqpoint{3.001405in}{1.303080in}}%
\pgfpathlineto{\pgfqpoint{3.015325in}{1.327855in}}%
\pgfpathlineto{\pgfqpoint{3.014703in}{1.331820in}}%
\pgfpathlineto{\pgfqpoint{3.021830in}{1.343049in}}%
\pgfpathlineto{\pgfqpoint{3.032848in}{1.337829in}}%
\pgfpathlineto{\pgfqpoint{3.034858in}{1.329046in}}%
\pgfpathlineto{\pgfqpoint{3.037759in}{1.326390in}}%
\pgfpathlineto{\pgfqpoint{3.047639in}{1.331046in}}%
\pgfpathlineto{\pgfqpoint{3.052203in}{1.329223in}}%
\pgfpathlineto{\pgfqpoint{3.053997in}{1.325414in}}%
\pgfpathlineto{\pgfqpoint{3.052475in}{1.317800in}}%
\pgfpathlineto{\pgfqpoint{3.046999in}{1.308770in}}%
\pgfpathclose%
\pgfusepath{fill}%
\end{pgfscope}%
\begin{pgfscope}%
\pgfpathrectangle{\pgfqpoint{0.100000in}{0.100000in}}{\pgfqpoint{3.420221in}{2.189500in}}%
\pgfusepath{clip}%
\pgfsetbuttcap%
\pgfsetmiterjoin%
\definecolor{currentfill}{rgb}{0.000000,0.188235,0.905882}%
\pgfsetfillcolor{currentfill}%
\pgfsetlinewidth{0.000000pt}%
\definecolor{currentstroke}{rgb}{0.000000,0.000000,0.000000}%
\pgfsetstrokecolor{currentstroke}%
\pgfsetstrokeopacity{0.000000}%
\pgfsetdash{}{0pt}%
\pgfpathmoveto{\pgfqpoint{1.731658in}{1.160443in}}%
\pgfpathlineto{\pgfqpoint{1.730353in}{1.131874in}}%
\pgfpathlineto{\pgfqpoint{1.702774in}{1.133481in}}%
\pgfpathlineto{\pgfqpoint{1.697580in}{1.133815in}}%
\pgfpathlineto{\pgfqpoint{1.699076in}{1.162603in}}%
\pgfpathlineto{\pgfqpoint{1.698514in}{1.169238in}}%
\pgfpathlineto{\pgfqpoint{1.666112in}{1.171164in}}%
\pgfpathlineto{\pgfqpoint{1.667840in}{1.197625in}}%
\pgfpathlineto{\pgfqpoint{1.667903in}{1.210673in}}%
\pgfpathlineto{\pgfqpoint{1.693687in}{1.209009in}}%
\pgfpathlineto{\pgfqpoint{1.694779in}{1.228547in}}%
\pgfpathlineto{\pgfqpoint{1.732547in}{1.226504in}}%
\pgfpathlineto{\pgfqpoint{1.733349in}{1.226463in}}%
\pgfpathlineto{\pgfqpoint{1.731718in}{1.193782in}}%
\pgfpathlineto{\pgfqpoint{1.732013in}{1.186971in}}%
\pgfpathclose%
\pgfusepath{fill}%
\end{pgfscope}%
\begin{pgfscope}%
\pgfpathrectangle{\pgfqpoint{0.100000in}{0.100000in}}{\pgfqpoint{3.420221in}{2.189500in}}%
\pgfusepath{clip}%
\pgfsetbuttcap%
\pgfsetmiterjoin%
\definecolor{currentfill}{rgb}{0.000000,0.486275,0.756863}%
\pgfsetfillcolor{currentfill}%
\pgfsetlinewidth{0.000000pt}%
\definecolor{currentstroke}{rgb}{0.000000,0.000000,0.000000}%
\pgfsetstrokecolor{currentstroke}%
\pgfsetstrokeopacity{0.000000}%
\pgfsetdash{}{0pt}%
\pgfpathmoveto{\pgfqpoint{1.672894in}{0.764801in}}%
\pgfpathlineto{\pgfqpoint{1.667580in}{0.765086in}}%
\pgfpathlineto{\pgfqpoint{1.640138in}{0.766963in}}%
\pgfpathlineto{\pgfqpoint{1.644861in}{0.833002in}}%
\pgfpathlineto{\pgfqpoint{1.653480in}{0.832431in}}%
\pgfpathlineto{\pgfqpoint{1.677128in}{0.830819in}}%
\pgfpathclose%
\pgfusepath{fill}%
\end{pgfscope}%
\begin{pgfscope}%
\pgfpathrectangle{\pgfqpoint{0.100000in}{0.100000in}}{\pgfqpoint{3.420221in}{2.189500in}}%
\pgfusepath{clip}%
\pgfsetbuttcap%
\pgfsetmiterjoin%
\definecolor{currentfill}{rgb}{0.000000,0.188235,0.905882}%
\pgfsetfillcolor{currentfill}%
\pgfsetlinewidth{0.000000pt}%
\definecolor{currentstroke}{rgb}{0.000000,0.000000,0.000000}%
\pgfsetstrokecolor{currentstroke}%
\pgfsetstrokeopacity{0.000000}%
\pgfsetdash{}{0pt}%
\pgfpathmoveto{\pgfqpoint{1.666112in}{1.171164in}}%
\pgfpathlineto{\pgfqpoint{1.665720in}{1.164627in}}%
\pgfpathlineto{\pgfqpoint{1.641044in}{1.166240in}}%
\pgfpathlineto{\pgfqpoint{1.639752in}{1.166320in}}%
\pgfpathlineto{\pgfqpoint{1.641500in}{1.192399in}}%
\pgfpathlineto{\pgfqpoint{1.585215in}{1.196628in}}%
\pgfpathlineto{\pgfqpoint{1.588122in}{1.235779in}}%
\pgfpathlineto{\pgfqpoint{1.616100in}{1.233729in}}%
\pgfpathlineto{\pgfqpoint{1.667922in}{1.230313in}}%
\pgfpathlineto{\pgfqpoint{1.694779in}{1.228547in}}%
\pgfpathlineto{\pgfqpoint{1.693687in}{1.209009in}}%
\pgfpathlineto{\pgfqpoint{1.667903in}{1.210673in}}%
\pgfpathlineto{\pgfqpoint{1.667840in}{1.197625in}}%
\pgfpathclose%
\pgfusepath{fill}%
\end{pgfscope}%
\begin{pgfscope}%
\pgfpathrectangle{\pgfqpoint{0.100000in}{0.100000in}}{\pgfqpoint{3.420221in}{2.189500in}}%
\pgfusepath{clip}%
\pgfsetbuttcap%
\pgfsetmiterjoin%
\definecolor{currentfill}{rgb}{0.000000,0.470588,0.764706}%
\pgfsetfillcolor{currentfill}%
\pgfsetlinewidth{0.000000pt}%
\definecolor{currentstroke}{rgb}{0.000000,0.000000,0.000000}%
\pgfsetstrokecolor{currentstroke}%
\pgfsetstrokeopacity{0.000000}%
\pgfsetdash{}{0pt}%
\pgfpathmoveto{\pgfqpoint{2.166987in}{1.224499in}}%
\pgfpathlineto{\pgfqpoint{2.199375in}{1.225317in}}%
\pgfpathlineto{\pgfqpoint{2.199738in}{1.215222in}}%
\pgfpathlineto{\pgfqpoint{2.206799in}{1.215394in}}%
\pgfpathlineto{\pgfqpoint{2.207617in}{1.188172in}}%
\pgfpathlineto{\pgfqpoint{2.220669in}{1.188552in}}%
\pgfpathlineto{\pgfqpoint{2.220795in}{1.182022in}}%
\pgfpathlineto{\pgfqpoint{2.230123in}{1.182205in}}%
\pgfpathlineto{\pgfqpoint{2.230366in}{1.174021in}}%
\pgfpathlineto{\pgfqpoint{2.240282in}{1.175633in}}%
\pgfpathlineto{\pgfqpoint{2.253236in}{1.176157in}}%
\pgfpathlineto{\pgfqpoint{2.253783in}{1.158975in}}%
\pgfpathlineto{\pgfqpoint{2.256375in}{1.151396in}}%
\pgfpathlineto{\pgfqpoint{2.256642in}{1.143558in}}%
\pgfpathlineto{\pgfqpoint{2.254515in}{1.134747in}}%
\pgfpathlineto{\pgfqpoint{2.243453in}{1.134489in}}%
\pgfpathlineto{\pgfqpoint{2.240130in}{1.137500in}}%
\pgfpathlineto{\pgfqpoint{2.240024in}{1.141086in}}%
\pgfpathlineto{\pgfqpoint{2.233331in}{1.144995in}}%
\pgfpathlineto{\pgfqpoint{2.229984in}{1.153339in}}%
\pgfpathlineto{\pgfqpoint{2.229707in}{1.160995in}}%
\pgfpathlineto{\pgfqpoint{2.195204in}{1.160350in}}%
\pgfpathlineto{\pgfqpoint{2.194751in}{1.173438in}}%
\pgfpathlineto{\pgfqpoint{2.165384in}{1.172896in}}%
\pgfpathlineto{\pgfqpoint{2.165320in}{1.176185in}}%
\pgfpathlineto{\pgfqpoint{2.155406in}{1.180805in}}%
\pgfpathlineto{\pgfqpoint{2.155359in}{1.191691in}}%
\pgfpathlineto{\pgfqpoint{2.155187in}{1.203831in}}%
\pgfpathlineto{\pgfqpoint{2.168232in}{1.203856in}}%
\pgfpathclose%
\pgfusepath{fill}%
\end{pgfscope}%
\begin{pgfscope}%
\pgfpathrectangle{\pgfqpoint{0.100000in}{0.100000in}}{\pgfqpoint{3.420221in}{2.189500in}}%
\pgfusepath{clip}%
\pgfsetbuttcap%
\pgfsetmiterjoin%
\definecolor{currentfill}{rgb}{0.000000,0.521569,0.739216}%
\pgfsetfillcolor{currentfill}%
\pgfsetlinewidth{0.000000pt}%
\definecolor{currentstroke}{rgb}{0.000000,0.000000,0.000000}%
\pgfsetstrokecolor{currentstroke}%
\pgfsetstrokeopacity{0.000000}%
\pgfsetdash{}{0pt}%
\pgfpathmoveto{\pgfqpoint{2.688645in}{0.719806in}}%
\pgfpathlineto{\pgfqpoint{2.691790in}{0.690555in}}%
\pgfpathlineto{\pgfqpoint{2.660309in}{0.688587in}}%
\pgfpathlineto{\pgfqpoint{2.654690in}{0.696287in}}%
\pgfpathlineto{\pgfqpoint{2.653797in}{0.701334in}}%
\pgfpathlineto{\pgfqpoint{2.649003in}{0.709343in}}%
\pgfpathlineto{\pgfqpoint{2.646694in}{0.714784in}}%
\pgfpathlineto{\pgfqpoint{2.653553in}{0.715320in}}%
\pgfpathlineto{\pgfqpoint{2.652328in}{0.729137in}}%
\pgfpathlineto{\pgfqpoint{2.670020in}{0.731361in}}%
\pgfpathlineto{\pgfqpoint{2.676760in}{0.731907in}}%
\pgfpathlineto{\pgfqpoint{2.677939in}{0.718577in}}%
\pgfpathclose%
\pgfusepath{fill}%
\end{pgfscope}%
\begin{pgfscope}%
\pgfpathrectangle{\pgfqpoint{0.100000in}{0.100000in}}{\pgfqpoint{3.420221in}{2.189500in}}%
\pgfusepath{clip}%
\pgfsetbuttcap%
\pgfsetmiterjoin%
\definecolor{currentfill}{rgb}{0.000000,0.250980,0.874510}%
\pgfsetfillcolor{currentfill}%
\pgfsetlinewidth{0.000000pt}%
\definecolor{currentstroke}{rgb}{0.000000,0.000000,0.000000}%
\pgfsetstrokecolor{currentstroke}%
\pgfsetstrokeopacity{0.000000}%
\pgfsetdash{}{0pt}%
\pgfpathmoveto{\pgfqpoint{1.924552in}{0.570650in}}%
\pgfpathlineto{\pgfqpoint{1.919172in}{0.573261in}}%
\pgfpathlineto{\pgfqpoint{1.914520in}{0.584539in}}%
\pgfpathlineto{\pgfqpoint{1.909084in}{0.585378in}}%
\pgfpathlineto{\pgfqpoint{1.901146in}{0.591759in}}%
\pgfpathlineto{\pgfqpoint{1.891780in}{0.610408in}}%
\pgfpathlineto{\pgfqpoint{1.884510in}{0.611267in}}%
\pgfpathlineto{\pgfqpoint{1.888676in}{0.617453in}}%
\pgfpathlineto{\pgfqpoint{1.894784in}{0.621395in}}%
\pgfpathlineto{\pgfqpoint{1.909089in}{0.624664in}}%
\pgfpathlineto{\pgfqpoint{1.916315in}{0.627683in}}%
\pgfpathlineto{\pgfqpoint{1.925117in}{0.628424in}}%
\pgfpathlineto{\pgfqpoint{1.930568in}{0.619250in}}%
\pgfpathlineto{\pgfqpoint{1.930808in}{0.615701in}}%
\pgfpathlineto{\pgfqpoint{1.949456in}{0.616792in}}%
\pgfpathlineto{\pgfqpoint{1.948024in}{0.645939in}}%
\pgfpathlineto{\pgfqpoint{1.963031in}{0.636636in}}%
\pgfpathlineto{\pgfqpoint{1.978502in}{0.636551in}}%
\pgfpathlineto{\pgfqpoint{1.980838in}{0.663083in}}%
\pgfpathlineto{\pgfqpoint{1.985798in}{0.666621in}}%
\pgfpathlineto{\pgfqpoint{1.989101in}{0.660413in}}%
\pgfpathlineto{\pgfqpoint{1.993536in}{0.657106in}}%
\pgfpathlineto{\pgfqpoint{1.994620in}{0.648023in}}%
\pgfpathlineto{\pgfqpoint{2.000571in}{0.647961in}}%
\pgfpathlineto{\pgfqpoint{2.004343in}{0.641627in}}%
\pgfpathlineto{\pgfqpoint{2.019551in}{0.635100in}}%
\pgfpathlineto{\pgfqpoint{2.028579in}{0.606493in}}%
\pgfpathlineto{\pgfqpoint{2.038495in}{0.606615in}}%
\pgfpathlineto{\pgfqpoint{2.038788in}{0.589670in}}%
\pgfpathlineto{\pgfqpoint{2.044330in}{0.589549in}}%
\pgfpathlineto{\pgfqpoint{2.044751in}{0.564816in}}%
\pgfpathlineto{\pgfqpoint{2.027872in}{0.557097in}}%
\pgfpathlineto{\pgfqpoint{2.031200in}{0.565672in}}%
\pgfpathlineto{\pgfqpoint{2.019227in}{0.561927in}}%
\pgfpathlineto{\pgfqpoint{2.022552in}{0.575094in}}%
\pgfpathlineto{\pgfqpoint{2.019407in}{0.581734in}}%
\pgfpathlineto{\pgfqpoint{2.014265in}{0.579513in}}%
\pgfpathlineto{\pgfqpoint{2.010504in}{0.572980in}}%
\pgfpathlineto{\pgfqpoint{2.004918in}{0.575060in}}%
\pgfpathlineto{\pgfqpoint{2.001058in}{0.570467in}}%
\pgfpathlineto{\pgfqpoint{2.001036in}{0.563243in}}%
\pgfpathlineto{\pgfqpoint{2.008102in}{0.559745in}}%
\pgfpathlineto{\pgfqpoint{2.009089in}{0.547650in}}%
\pgfpathlineto{\pgfqpoint{2.020145in}{0.547418in}}%
\pgfpathlineto{\pgfqpoint{1.994185in}{0.527920in}}%
\pgfpathlineto{\pgfqpoint{1.998384in}{0.537149in}}%
\pgfpathlineto{\pgfqpoint{1.986697in}{0.557401in}}%
\pgfpathlineto{\pgfqpoint{1.987807in}{0.564254in}}%
\pgfpathlineto{\pgfqpoint{1.984840in}{0.567089in}}%
\pgfpathlineto{\pgfqpoint{1.974109in}{0.566150in}}%
\pgfpathlineto{\pgfqpoint{1.971520in}{0.555457in}}%
\pgfpathlineto{\pgfqpoint{1.965767in}{0.555426in}}%
\pgfpathlineto{\pgfqpoint{1.963805in}{0.547988in}}%
\pgfpathlineto{\pgfqpoint{1.958781in}{0.544216in}}%
\pgfpathlineto{\pgfqpoint{1.952246in}{0.546869in}}%
\pgfpathlineto{\pgfqpoint{1.945927in}{0.542225in}}%
\pgfpathlineto{\pgfqpoint{1.939227in}{0.547806in}}%
\pgfpathlineto{\pgfqpoint{1.936508in}{0.555327in}}%
\pgfpathlineto{\pgfqpoint{1.932052in}{0.558403in}}%
\pgfpathlineto{\pgfqpoint{1.932043in}{0.567296in}}%
\pgfpathclose%
\pgfusepath{fill}%
\end{pgfscope}%
\begin{pgfscope}%
\pgfpathrectangle{\pgfqpoint{0.100000in}{0.100000in}}{\pgfqpoint{3.420221in}{2.189500in}}%
\pgfusepath{clip}%
\pgfsetbuttcap%
\pgfsetmiterjoin%
\definecolor{currentfill}{rgb}{0.000000,0.474510,0.762745}%
\pgfsetfillcolor{currentfill}%
\pgfsetlinewidth{0.000000pt}%
\definecolor{currentstroke}{rgb}{0.000000,0.000000,0.000000}%
\pgfsetstrokecolor{currentstroke}%
\pgfsetstrokeopacity{0.000000}%
\pgfsetdash{}{0pt}%
\pgfpathmoveto{\pgfqpoint{1.842257in}{0.894605in}}%
\pgfpathlineto{\pgfqpoint{1.835570in}{0.901713in}}%
\pgfpathlineto{\pgfqpoint{1.823111in}{0.891395in}}%
\pgfpathlineto{\pgfqpoint{1.816212in}{0.894668in}}%
\pgfpathlineto{\pgfqpoint{1.818444in}{0.902148in}}%
\pgfpathlineto{\pgfqpoint{1.812092in}{0.902672in}}%
\pgfpathlineto{\pgfqpoint{1.804759in}{0.911929in}}%
\pgfpathlineto{\pgfqpoint{1.792759in}{0.915583in}}%
\pgfpathlineto{\pgfqpoint{1.790476in}{0.911378in}}%
\pgfpathlineto{\pgfqpoint{1.784983in}{0.908773in}}%
\pgfpathlineto{\pgfqpoint{1.777557in}{0.916175in}}%
\pgfpathlineto{\pgfqpoint{1.778125in}{0.929439in}}%
\pgfpathlineto{\pgfqpoint{1.774886in}{0.929552in}}%
\pgfpathlineto{\pgfqpoint{1.775410in}{0.942610in}}%
\pgfpathlineto{\pgfqpoint{1.765243in}{0.943022in}}%
\pgfpathlineto{\pgfqpoint{1.765510in}{0.949553in}}%
\pgfpathlineto{\pgfqpoint{1.766330in}{0.969160in}}%
\pgfpathlineto{\pgfqpoint{1.779045in}{0.968604in}}%
\pgfpathlineto{\pgfqpoint{1.811707in}{0.967396in}}%
\pgfpathlineto{\pgfqpoint{1.811293in}{0.954343in}}%
\pgfpathlineto{\pgfqpoint{1.837270in}{0.953538in}}%
\pgfpathlineto{\pgfqpoint{1.843796in}{0.953335in}}%
\pgfpathclose%
\pgfusepath{fill}%
\end{pgfscope}%
\begin{pgfscope}%
\pgfpathrectangle{\pgfqpoint{0.100000in}{0.100000in}}{\pgfqpoint{3.420221in}{2.189500in}}%
\pgfusepath{clip}%
\pgfsetbuttcap%
\pgfsetmiterjoin%
\definecolor{currentfill}{rgb}{0.000000,0.341176,0.829412}%
\pgfsetfillcolor{currentfill}%
\pgfsetlinewidth{0.000000pt}%
\definecolor{currentstroke}{rgb}{0.000000,0.000000,0.000000}%
\pgfsetstrokecolor{currentstroke}%
\pgfsetstrokeopacity{0.000000}%
\pgfsetdash{}{0pt}%
\pgfpathmoveto{\pgfqpoint{3.188777in}{1.644598in}}%
\pgfpathlineto{\pgfqpoint{3.188401in}{1.638975in}}%
\pgfpathlineto{\pgfqpoint{3.177952in}{1.638592in}}%
\pgfpathlineto{\pgfqpoint{3.166263in}{1.641419in}}%
\pgfpathlineto{\pgfqpoint{3.164573in}{1.644933in}}%
\pgfpathlineto{\pgfqpoint{3.157145in}{1.646380in}}%
\pgfpathlineto{\pgfqpoint{3.156654in}{1.640578in}}%
\pgfpathlineto{\pgfqpoint{3.143734in}{1.638935in}}%
\pgfpathlineto{\pgfqpoint{3.134996in}{1.641143in}}%
\pgfpathlineto{\pgfqpoint{3.129817in}{1.642433in}}%
\pgfpathlineto{\pgfqpoint{3.133241in}{1.654587in}}%
\pgfpathlineto{\pgfqpoint{3.121675in}{1.657735in}}%
\pgfpathlineto{\pgfqpoint{3.115117in}{1.663432in}}%
\pgfpathlineto{\pgfqpoint{3.117667in}{1.672393in}}%
\pgfpathlineto{\pgfqpoint{3.113614in}{1.681019in}}%
\pgfpathlineto{\pgfqpoint{3.113333in}{1.687026in}}%
\pgfpathlineto{\pgfqpoint{3.124665in}{1.685262in}}%
\pgfpathlineto{\pgfqpoint{3.133365in}{1.688106in}}%
\pgfpathlineto{\pgfqpoint{3.142447in}{1.698988in}}%
\pgfpathlineto{\pgfqpoint{3.121288in}{1.758173in}}%
\pgfpathlineto{\pgfqpoint{3.129353in}{1.761278in}}%
\pgfpathlineto{\pgfqpoint{3.164020in}{1.774485in}}%
\pgfpathlineto{\pgfqpoint{3.164792in}{1.765623in}}%
\pgfpathlineto{\pgfqpoint{3.175442in}{1.757573in}}%
\pgfpathlineto{\pgfqpoint{3.179384in}{1.740036in}}%
\pgfpathlineto{\pgfqpoint{3.186610in}{1.704915in}}%
\pgfpathlineto{\pgfqpoint{3.188954in}{1.698406in}}%
\pgfpathlineto{\pgfqpoint{3.187524in}{1.647119in}}%
\pgfpathclose%
\pgfusepath{fill}%
\end{pgfscope}%
\begin{pgfscope}%
\pgfpathrectangle{\pgfqpoint{0.100000in}{0.100000in}}{\pgfqpoint{3.420221in}{2.189500in}}%
\pgfusepath{clip}%
\pgfsetbuttcap%
\pgfsetmiterjoin%
\definecolor{currentfill}{rgb}{0.000000,0.384314,0.807843}%
\pgfsetfillcolor{currentfill}%
\pgfsetlinewidth{0.000000pt}%
\definecolor{currentstroke}{rgb}{0.000000,0.000000,0.000000}%
\pgfsetstrokecolor{currentstroke}%
\pgfsetstrokeopacity{0.000000}%
\pgfsetdash{}{0pt}%
\pgfpathmoveto{\pgfqpoint{1.298884in}{1.776601in}}%
\pgfpathlineto{\pgfqpoint{1.254982in}{1.783265in}}%
\pgfpathlineto{\pgfqpoint{1.253980in}{1.783901in}}%
\pgfpathlineto{\pgfqpoint{1.217375in}{1.789489in}}%
\pgfpathlineto{\pgfqpoint{1.219475in}{1.802103in}}%
\pgfpathlineto{\pgfqpoint{1.205555in}{1.804425in}}%
\pgfpathlineto{\pgfqpoint{1.207777in}{1.817552in}}%
\pgfpathlineto{\pgfqpoint{1.214626in}{1.816397in}}%
\pgfpathlineto{\pgfqpoint{1.216783in}{1.829340in}}%
\pgfpathlineto{\pgfqpoint{1.230693in}{1.833649in}}%
\pgfpathlineto{\pgfqpoint{1.237118in}{1.832622in}}%
\pgfpathlineto{\pgfqpoint{1.239096in}{1.845591in}}%
\pgfpathlineto{\pgfqpoint{1.244193in}{1.858230in}}%
\pgfpathlineto{\pgfqpoint{1.248530in}{1.857616in}}%
\pgfpathlineto{\pgfqpoint{1.249844in}{1.863848in}}%
\pgfpathlineto{\pgfqpoint{1.240146in}{1.865478in}}%
\pgfpathlineto{\pgfqpoint{1.238656in}{1.872362in}}%
\pgfpathlineto{\pgfqpoint{1.239704in}{1.878862in}}%
\pgfpathlineto{\pgfqpoint{1.252676in}{1.876778in}}%
\pgfpathlineto{\pgfqpoint{1.259031in}{1.912046in}}%
\pgfpathlineto{\pgfqpoint{1.259726in}{1.916326in}}%
\pgfpathlineto{\pgfqpoint{1.298514in}{1.910176in}}%
\pgfpathlineto{\pgfqpoint{1.339766in}{1.904622in}}%
\pgfpathlineto{\pgfqpoint{1.340743in}{1.898702in}}%
\pgfpathlineto{\pgfqpoint{1.338590in}{1.884933in}}%
\pgfpathlineto{\pgfqpoint{1.342376in}{1.882222in}}%
\pgfpathlineto{\pgfqpoint{1.346101in}{1.869544in}}%
\pgfpathlineto{\pgfqpoint{1.352320in}{1.858733in}}%
\pgfpathlineto{\pgfqpoint{1.348591in}{1.849014in}}%
\pgfpathlineto{\pgfqpoint{1.340149in}{1.850219in}}%
\pgfpathlineto{\pgfqpoint{1.339532in}{1.845889in}}%
\pgfpathlineto{\pgfqpoint{1.333113in}{1.846795in}}%
\pgfpathlineto{\pgfqpoint{1.327852in}{1.840933in}}%
\pgfpathlineto{\pgfqpoint{1.319235in}{1.842198in}}%
\pgfpathlineto{\pgfqpoint{1.316859in}{1.833719in}}%
\pgfpathlineto{\pgfqpoint{1.313956in}{1.813916in}}%
\pgfpathlineto{\pgfqpoint{1.300425in}{1.811495in}}%
\pgfpathlineto{\pgfqpoint{1.283280in}{1.814330in}}%
\pgfpathlineto{\pgfqpoint{1.280987in}{1.812314in}}%
\pgfpathlineto{\pgfqpoint{1.277639in}{1.796419in}}%
\pgfpathlineto{\pgfqpoint{1.309242in}{1.791914in}}%
\pgfpathlineto{\pgfqpoint{1.306452in}{1.790243in}}%
\pgfpathclose%
\pgfusepath{fill}%
\end{pgfscope}%
\begin{pgfscope}%
\pgfpathrectangle{\pgfqpoint{0.100000in}{0.100000in}}{\pgfqpoint{3.420221in}{2.189500in}}%
\pgfusepath{clip}%
\pgfsetbuttcap%
\pgfsetmiterjoin%
\definecolor{currentfill}{rgb}{0.000000,0.592157,0.703922}%
\pgfsetfillcolor{currentfill}%
\pgfsetlinewidth{0.000000pt}%
\definecolor{currentstroke}{rgb}{0.000000,0.000000,0.000000}%
\pgfsetstrokecolor{currentstroke}%
\pgfsetstrokeopacity{0.000000}%
\pgfsetdash{}{0pt}%
\pgfpathmoveto{\pgfqpoint{2.747861in}{0.841676in}}%
\pgfpathlineto{\pgfqpoint{2.733474in}{0.863678in}}%
\pgfpathlineto{\pgfqpoint{2.736311in}{0.866162in}}%
\pgfpathlineto{\pgfqpoint{2.729356in}{0.884961in}}%
\pgfpathlineto{\pgfqpoint{2.722210in}{0.883017in}}%
\pgfpathlineto{\pgfqpoint{2.720451in}{0.902727in}}%
\pgfpathlineto{\pgfqpoint{2.735793in}{0.908412in}}%
\pgfpathlineto{\pgfqpoint{2.744234in}{0.899763in}}%
\pgfpathlineto{\pgfqpoint{2.752540in}{0.909520in}}%
\pgfpathlineto{\pgfqpoint{2.762814in}{0.908932in}}%
\pgfpathlineto{\pgfqpoint{2.771349in}{0.894513in}}%
\pgfpathlineto{\pgfqpoint{2.765166in}{0.890527in}}%
\pgfpathlineto{\pgfqpoint{2.763576in}{0.885891in}}%
\pgfpathlineto{\pgfqpoint{2.758800in}{0.883019in}}%
\pgfpathlineto{\pgfqpoint{2.753524in}{0.876171in}}%
\pgfpathlineto{\pgfqpoint{2.753920in}{0.870009in}}%
\pgfpathlineto{\pgfqpoint{2.757692in}{0.863683in}}%
\pgfpathlineto{\pgfqpoint{2.762846in}{0.860171in}}%
\pgfpathlineto{\pgfqpoint{2.763599in}{0.853183in}}%
\pgfpathclose%
\pgfusepath{fill}%
\end{pgfscope}%
\begin{pgfscope}%
\pgfpathrectangle{\pgfqpoint{0.100000in}{0.100000in}}{\pgfqpoint{3.420221in}{2.189500in}}%
\pgfusepath{clip}%
\pgfsetbuttcap%
\pgfsetmiterjoin%
\definecolor{currentfill}{rgb}{0.000000,0.443137,0.778431}%
\pgfsetfillcolor{currentfill}%
\pgfsetlinewidth{0.000000pt}%
\definecolor{currentstroke}{rgb}{0.000000,0.000000,0.000000}%
\pgfsetstrokecolor{currentstroke}%
\pgfsetstrokeopacity{0.000000}%
\pgfsetdash{}{0pt}%
\pgfpathmoveto{\pgfqpoint{1.423514in}{1.985198in}}%
\pgfpathlineto{\pgfqpoint{1.425711in}{1.978117in}}%
\pgfpathlineto{\pgfqpoint{1.430617in}{1.977495in}}%
\pgfpathlineto{\pgfqpoint{1.427464in}{1.951967in}}%
\pgfpathlineto{\pgfqpoint{1.423915in}{1.939080in}}%
\pgfpathlineto{\pgfqpoint{1.430409in}{1.938297in}}%
\pgfpathlineto{\pgfqpoint{1.428809in}{1.925236in}}%
\pgfpathlineto{\pgfqpoint{1.433071in}{1.924714in}}%
\pgfpathlineto{\pgfqpoint{1.430010in}{1.899550in}}%
\pgfpathlineto{\pgfqpoint{1.397477in}{1.903600in}}%
\pgfpathlineto{\pgfqpoint{1.397628in}{1.904675in}}%
\pgfpathlineto{\pgfqpoint{1.337478in}{1.912255in}}%
\pgfpathlineto{\pgfqpoint{1.335317in}{1.914996in}}%
\pgfpathlineto{\pgfqpoint{1.339267in}{1.924153in}}%
\pgfpathlineto{\pgfqpoint{1.337684in}{1.932883in}}%
\pgfpathlineto{\pgfqpoint{1.337886in}{1.942708in}}%
\pgfpathlineto{\pgfqpoint{1.343039in}{1.957270in}}%
\pgfpathlineto{\pgfqpoint{1.346121in}{1.961824in}}%
\pgfpathlineto{\pgfqpoint{1.353212in}{1.965014in}}%
\pgfpathlineto{\pgfqpoint{1.358860in}{1.970238in}}%
\pgfpathlineto{\pgfqpoint{1.368086in}{1.966871in}}%
\pgfpathlineto{\pgfqpoint{1.370589in}{1.971852in}}%
\pgfpathlineto{\pgfqpoint{1.378733in}{1.968423in}}%
\pgfpathlineto{\pgfqpoint{1.396916in}{1.967157in}}%
\pgfpathlineto{\pgfqpoint{1.398708in}{1.971402in}}%
\pgfpathlineto{\pgfqpoint{1.404536in}{1.970220in}}%
\pgfpathlineto{\pgfqpoint{1.414677in}{1.975388in}}%
\pgfpathlineto{\pgfqpoint{1.419696in}{1.983495in}}%
\pgfpathclose%
\pgfusepath{fill}%
\end{pgfscope}%
\begin{pgfscope}%
\pgfpathrectangle{\pgfqpoint{0.100000in}{0.100000in}}{\pgfqpoint{3.420221in}{2.189500in}}%
\pgfusepath{clip}%
\pgfsetbuttcap%
\pgfsetmiterjoin%
\definecolor{currentfill}{rgb}{0.000000,0.360784,0.819608}%
\pgfsetfillcolor{currentfill}%
\pgfsetlinewidth{0.000000pt}%
\definecolor{currentstroke}{rgb}{0.000000,0.000000,0.000000}%
\pgfsetstrokecolor{currentstroke}%
\pgfsetstrokeopacity{0.000000}%
\pgfsetdash{}{0pt}%
\pgfpathmoveto{\pgfqpoint{2.994042in}{1.378885in}}%
\pgfpathlineto{\pgfqpoint{2.959852in}{1.397886in}}%
\pgfpathlineto{\pgfqpoint{2.952251in}{1.400351in}}%
\pgfpathlineto{\pgfqpoint{2.953464in}{1.407492in}}%
\pgfpathlineto{\pgfqpoint{2.958679in}{1.414878in}}%
\pgfpathlineto{\pgfqpoint{2.954465in}{1.416446in}}%
\pgfpathlineto{\pgfqpoint{2.957679in}{1.424882in}}%
\pgfpathlineto{\pgfqpoint{2.959851in}{1.438076in}}%
\pgfpathlineto{\pgfqpoint{2.959299in}{1.445059in}}%
\pgfpathlineto{\pgfqpoint{2.962266in}{1.451636in}}%
\pgfpathlineto{\pgfqpoint{2.974915in}{1.449194in}}%
\pgfpathlineto{\pgfqpoint{2.978934in}{1.446884in}}%
\pgfpathlineto{\pgfqpoint{2.985139in}{1.463500in}}%
\pgfpathlineto{\pgfqpoint{2.987668in}{1.464488in}}%
\pgfpathlineto{\pgfqpoint{3.002720in}{1.442543in}}%
\pgfpathlineto{\pgfqpoint{3.006635in}{1.426164in}}%
\pgfpathlineto{\pgfqpoint{3.001814in}{1.417574in}}%
\pgfpathlineto{\pgfqpoint{2.999922in}{1.394659in}}%
\pgfpathlineto{\pgfqpoint{2.996817in}{1.393154in}}%
\pgfpathclose%
\pgfusepath{fill}%
\end{pgfscope}%
\begin{pgfscope}%
\pgfpathrectangle{\pgfqpoint{0.100000in}{0.100000in}}{\pgfqpoint{3.420221in}{2.189500in}}%
\pgfusepath{clip}%
\pgfsetbuttcap%
\pgfsetmiterjoin%
\definecolor{currentfill}{rgb}{0.000000,0.768627,0.615686}%
\pgfsetfillcolor{currentfill}%
\pgfsetlinewidth{0.000000pt}%
\definecolor{currentstroke}{rgb}{0.000000,0.000000,0.000000}%
\pgfsetstrokecolor{currentstroke}%
\pgfsetstrokeopacity{0.000000}%
\pgfsetdash{}{0pt}%
\pgfpathmoveto{\pgfqpoint{0.702190in}{0.453432in}}%
\pgfpathlineto{\pgfqpoint{0.704870in}{0.456358in}}%
\pgfpathlineto{\pgfqpoint{0.709956in}{0.455232in}}%
\pgfpathlineto{\pgfqpoint{0.707170in}{0.453436in}}%
\pgfpathclose%
\pgfusepath{fill}%
\end{pgfscope}%
\begin{pgfscope}%
\pgfpathrectangle{\pgfqpoint{0.100000in}{0.100000in}}{\pgfqpoint{3.420221in}{2.189500in}}%
\pgfusepath{clip}%
\pgfsetbuttcap%
\pgfsetmiterjoin%
\definecolor{currentfill}{rgb}{0.000000,0.768627,0.615686}%
\pgfsetfillcolor{currentfill}%
\pgfsetlinewidth{0.000000pt}%
\definecolor{currentstroke}{rgb}{0.000000,0.000000,0.000000}%
\pgfsetstrokecolor{currentstroke}%
\pgfsetstrokeopacity{0.000000}%
\pgfsetdash{}{0pt}%
\pgfpathmoveto{\pgfqpoint{0.824868in}{0.436985in}}%
\pgfpathlineto{\pgfqpoint{0.823896in}{0.434760in}}%
\pgfpathlineto{\pgfqpoint{0.818084in}{0.426988in}}%
\pgfpathlineto{\pgfqpoint{0.816621in}{0.428064in}}%
\pgfpathlineto{\pgfqpoint{0.810796in}{0.420377in}}%
\pgfpathlineto{\pgfqpoint{0.809404in}{0.421390in}}%
\pgfpathlineto{\pgfqpoint{0.803852in}{0.413980in}}%
\pgfpathlineto{\pgfqpoint{0.802441in}{0.415049in}}%
\pgfpathlineto{\pgfqpoint{0.800969in}{0.413117in}}%
\pgfpathlineto{\pgfqpoint{0.799058in}{0.414546in}}%
\pgfpathlineto{\pgfqpoint{0.796156in}{0.410686in}}%
\pgfpathlineto{\pgfqpoint{0.794266in}{0.412104in}}%
\pgfpathlineto{\pgfqpoint{0.792835in}{0.410153in}}%
\pgfpathlineto{\pgfqpoint{0.789544in}{0.412576in}}%
\pgfpathlineto{\pgfqpoint{0.786705in}{0.408749in}}%
\pgfpathlineto{\pgfqpoint{0.782927in}{0.411601in}}%
\pgfpathlineto{\pgfqpoint{0.780011in}{0.407770in}}%
\pgfpathlineto{\pgfqpoint{0.780586in}{0.407291in}}%
\pgfpathlineto{\pgfqpoint{0.774891in}{0.399921in}}%
\pgfpathlineto{\pgfqpoint{0.774053in}{0.397548in}}%
\pgfpathlineto{\pgfqpoint{0.777778in}{0.394690in}}%
\pgfpathlineto{\pgfqpoint{0.774725in}{0.390638in}}%
\pgfpathlineto{\pgfqpoint{0.772585in}{0.390859in}}%
\pgfpathlineto{\pgfqpoint{0.771254in}{0.389905in}}%
\pgfpathlineto{\pgfqpoint{0.767323in}{0.390986in}}%
\pgfpathlineto{\pgfqpoint{0.763959in}{0.393118in}}%
\pgfpathlineto{\pgfqpoint{0.762489in}{0.392379in}}%
\pgfpathlineto{\pgfqpoint{0.760646in}{0.393822in}}%
\pgfpathlineto{\pgfqpoint{0.758681in}{0.393872in}}%
\pgfpathlineto{\pgfqpoint{0.753756in}{0.387567in}}%
\pgfpathlineto{\pgfqpoint{0.750244in}{0.387954in}}%
\pgfpathlineto{\pgfqpoint{0.748505in}{0.389329in}}%
\pgfpathlineto{\pgfqpoint{0.745535in}{0.387287in}}%
\pgfpathlineto{\pgfqpoint{0.741947in}{0.386331in}}%
\pgfpathlineto{\pgfqpoint{0.741792in}{0.386394in}}%
\pgfpathlineto{\pgfqpoint{0.742303in}{0.387463in}}%
\pgfpathlineto{\pgfqpoint{0.738133in}{0.388189in}}%
\pgfpathlineto{\pgfqpoint{0.736357in}{0.389624in}}%
\pgfpathlineto{\pgfqpoint{0.733358in}{0.388997in}}%
\pgfpathlineto{\pgfqpoint{0.732199in}{0.387456in}}%
\pgfpathlineto{\pgfqpoint{0.730314in}{0.387945in}}%
\pgfpathlineto{\pgfqpoint{0.727100in}{0.387065in}}%
\pgfpathlineto{\pgfqpoint{0.725882in}{0.385544in}}%
\pgfpathlineto{\pgfqpoint{0.729654in}{0.382439in}}%
\pgfpathlineto{\pgfqpoint{0.725392in}{0.380062in}}%
\pgfpathlineto{\pgfqpoint{0.724266in}{0.382186in}}%
\pgfpathlineto{\pgfqpoint{0.720824in}{0.380694in}}%
\pgfpathlineto{\pgfqpoint{0.719621in}{0.382045in}}%
\pgfpathlineto{\pgfqpoint{0.709550in}{0.383764in}}%
\pgfpathlineto{\pgfqpoint{0.708289in}{0.381050in}}%
\pgfpathlineto{\pgfqpoint{0.705982in}{0.382702in}}%
\pgfpathlineto{\pgfqpoint{0.705905in}{0.384958in}}%
\pgfpathlineto{\pgfqpoint{0.703307in}{0.385490in}}%
\pgfpathlineto{\pgfqpoint{0.702959in}{0.383374in}}%
\pgfpathlineto{\pgfqpoint{0.701334in}{0.381838in}}%
\pgfpathlineto{\pgfqpoint{0.697450in}{0.383063in}}%
\pgfpathlineto{\pgfqpoint{0.694776in}{0.385515in}}%
\pgfpathlineto{\pgfqpoint{0.691332in}{0.384080in}}%
\pgfpathlineto{\pgfqpoint{0.693895in}{0.380479in}}%
\pgfpathlineto{\pgfqpoint{0.688409in}{0.379224in}}%
\pgfpathlineto{\pgfqpoint{0.686100in}{0.376708in}}%
\pgfpathlineto{\pgfqpoint{0.685287in}{0.379238in}}%
\pgfpathlineto{\pgfqpoint{0.682662in}{0.378892in}}%
\pgfpathlineto{\pgfqpoint{0.680686in}{0.380309in}}%
\pgfpathlineto{\pgfqpoint{0.678983in}{0.380346in}}%
\pgfpathlineto{\pgfqpoint{0.673650in}{0.382881in}}%
\pgfpathlineto{\pgfqpoint{0.671344in}{0.381770in}}%
\pgfpathlineto{\pgfqpoint{0.669585in}{0.381817in}}%
\pgfpathlineto{\pgfqpoint{0.667048in}{0.380157in}}%
\pgfpathlineto{\pgfqpoint{0.671407in}{0.385231in}}%
\pgfpathlineto{\pgfqpoint{0.667965in}{0.388203in}}%
\pgfpathlineto{\pgfqpoint{0.670323in}{0.389489in}}%
\pgfpathlineto{\pgfqpoint{0.676211in}{0.396397in}}%
\pgfpathlineto{\pgfqpoint{0.675171in}{0.397330in}}%
\pgfpathlineto{\pgfqpoint{0.678194in}{0.400721in}}%
\pgfpathlineto{\pgfqpoint{0.681667in}{0.397810in}}%
\pgfpathlineto{\pgfqpoint{0.683188in}{0.399559in}}%
\pgfpathlineto{\pgfqpoint{0.686747in}{0.396507in}}%
\pgfpathlineto{\pgfqpoint{0.688332in}{0.398361in}}%
\pgfpathlineto{\pgfqpoint{0.691064in}{0.395949in}}%
\pgfpathlineto{\pgfqpoint{0.692536in}{0.397682in}}%
\pgfpathlineto{\pgfqpoint{0.694924in}{0.397371in}}%
\pgfpathlineto{\pgfqpoint{0.697527in}{0.394913in}}%
\pgfpathlineto{\pgfqpoint{0.700397in}{0.399205in}}%
\pgfpathlineto{\pgfqpoint{0.705858in}{0.401326in}}%
\pgfpathlineto{\pgfqpoint{0.711507in}{0.402263in}}%
\pgfpathlineto{\pgfqpoint{0.714525in}{0.401133in}}%
\pgfpathlineto{\pgfqpoint{0.717676in}{0.402687in}}%
\pgfpathlineto{\pgfqpoint{0.722068in}{0.402353in}}%
\pgfpathlineto{\pgfqpoint{0.723436in}{0.404493in}}%
\pgfpathlineto{\pgfqpoint{0.733578in}{0.413469in}}%
\pgfpathlineto{\pgfqpoint{0.735768in}{0.413576in}}%
\pgfpathlineto{\pgfqpoint{0.735632in}{0.415404in}}%
\pgfpathlineto{\pgfqpoint{0.738934in}{0.418862in}}%
\pgfpathlineto{\pgfqpoint{0.744119in}{0.420039in}}%
\pgfpathlineto{\pgfqpoint{0.754853in}{0.411634in}}%
\pgfpathlineto{\pgfqpoint{0.759703in}{0.417699in}}%
\pgfpathlineto{\pgfqpoint{0.751367in}{0.424210in}}%
\pgfpathlineto{\pgfqpoint{0.736991in}{0.427359in}}%
\pgfpathlineto{\pgfqpoint{0.734276in}{0.428473in}}%
\pgfpathlineto{\pgfqpoint{0.733695in}{0.430657in}}%
\pgfpathlineto{\pgfqpoint{0.734527in}{0.433641in}}%
\pgfpathlineto{\pgfqpoint{0.733338in}{0.434961in}}%
\pgfpathlineto{\pgfqpoint{0.736772in}{0.438253in}}%
\pgfpathlineto{\pgfqpoint{0.735601in}{0.438814in}}%
\pgfpathlineto{\pgfqpoint{0.728787in}{0.436898in}}%
\pgfpathlineto{\pgfqpoint{0.726569in}{0.431612in}}%
\pgfpathlineto{\pgfqpoint{0.724984in}{0.429829in}}%
\pgfpathlineto{\pgfqpoint{0.722848in}{0.430170in}}%
\pgfpathlineto{\pgfqpoint{0.721744in}{0.432778in}}%
\pgfpathlineto{\pgfqpoint{0.722681in}{0.433600in}}%
\pgfpathlineto{\pgfqpoint{0.722723in}{0.436792in}}%
\pgfpathlineto{\pgfqpoint{0.724070in}{0.444132in}}%
\pgfpathlineto{\pgfqpoint{0.723277in}{0.445609in}}%
\pgfpathlineto{\pgfqpoint{0.725148in}{0.448048in}}%
\pgfpathlineto{\pgfqpoint{0.723152in}{0.449039in}}%
\pgfpathlineto{\pgfqpoint{0.722032in}{0.447990in}}%
\pgfpathlineto{\pgfqpoint{0.718614in}{0.447097in}}%
\pgfpathlineto{\pgfqpoint{0.718906in}{0.449313in}}%
\pgfpathlineto{\pgfqpoint{0.717551in}{0.453993in}}%
\pgfpathlineto{\pgfqpoint{0.719474in}{0.456863in}}%
\pgfpathlineto{\pgfqpoint{0.716996in}{0.457421in}}%
\pgfpathlineto{\pgfqpoint{0.711991in}{0.457624in}}%
\pgfpathlineto{\pgfqpoint{0.707299in}{0.458956in}}%
\pgfpathlineto{\pgfqpoint{0.710745in}{0.462233in}}%
\pgfpathlineto{\pgfqpoint{0.712466in}{0.460737in}}%
\pgfpathlineto{\pgfqpoint{0.713896in}{0.462362in}}%
\pgfpathlineto{\pgfqpoint{0.715663in}{0.460844in}}%
\pgfpathlineto{\pgfqpoint{0.718816in}{0.464400in}}%
\pgfpathlineto{\pgfqpoint{0.720992in}{0.465660in}}%
\pgfpathlineto{\pgfqpoint{0.722723in}{0.464078in}}%
\pgfpathlineto{\pgfqpoint{0.724914in}{0.465385in}}%
\pgfpathlineto{\pgfqpoint{0.728070in}{0.468974in}}%
\pgfpathlineto{\pgfqpoint{0.729876in}{0.467393in}}%
\pgfpathlineto{\pgfqpoint{0.733007in}{0.470950in}}%
\pgfpathlineto{\pgfqpoint{0.735431in}{0.468920in}}%
\pgfpathlineto{\pgfqpoint{0.738553in}{0.472489in}}%
\pgfpathlineto{\pgfqpoint{0.740281in}{0.471011in}}%
\pgfpathlineto{\pgfqpoint{0.743293in}{0.474389in}}%
\pgfpathlineto{\pgfqpoint{0.745516in}{0.475618in}}%
\pgfpathlineto{\pgfqpoint{0.747320in}{0.474107in}}%
\pgfpathlineto{\pgfqpoint{0.750410in}{0.477654in}}%
\pgfpathlineto{\pgfqpoint{0.752217in}{0.476095in}}%
\pgfpathlineto{\pgfqpoint{0.753782in}{0.477906in}}%
\pgfpathlineto{\pgfqpoint{0.755809in}{0.476145in}}%
\pgfpathlineto{\pgfqpoint{0.758926in}{0.479765in}}%
\pgfpathlineto{\pgfqpoint{0.760979in}{0.481161in}}%
\pgfpathlineto{\pgfqpoint{0.762737in}{0.479737in}}%
\pgfpathlineto{\pgfqpoint{0.764377in}{0.481657in}}%
\pgfpathlineto{\pgfqpoint{0.765790in}{0.480531in}}%
\pgfpathlineto{\pgfqpoint{0.767347in}{0.482344in}}%
\pgfpathlineto{\pgfqpoint{0.777662in}{0.473881in}}%
\pgfpathlineto{\pgfqpoint{0.776181in}{0.472015in}}%
\pgfpathlineto{\pgfqpoint{0.778072in}{0.470436in}}%
\pgfpathlineto{\pgfqpoint{0.779586in}{0.472327in}}%
\pgfpathlineto{\pgfqpoint{0.781520in}{0.470904in}}%
\pgfpathlineto{\pgfqpoint{0.783177in}{0.472910in}}%
\pgfpathlineto{\pgfqpoint{0.786829in}{0.469921in}}%
\pgfpathlineto{\pgfqpoint{0.785188in}{0.467882in}}%
\pgfpathlineto{\pgfqpoint{0.797891in}{0.457757in}}%
\pgfpathlineto{\pgfqpoint{0.814246in}{0.444980in}}%
\pgfpathclose%
\pgfusepath{fill}%
\end{pgfscope}%
\begin{pgfscope}%
\pgfpathrectangle{\pgfqpoint{0.100000in}{0.100000in}}{\pgfqpoint{3.420221in}{2.189500in}}%
\pgfusepath{clip}%
\pgfsetbuttcap%
\pgfsetmiterjoin%
\definecolor{currentfill}{rgb}{0.000000,0.768627,0.615686}%
\pgfsetfillcolor{currentfill}%
\pgfsetlinewidth{0.000000pt}%
\definecolor{currentstroke}{rgb}{0.000000,0.000000,0.000000}%
\pgfsetstrokecolor{currentstroke}%
\pgfsetstrokeopacity{0.000000}%
\pgfsetdash{}{0pt}%
\pgfpathmoveto{\pgfqpoint{0.702788in}{0.459696in}}%
\pgfpathlineto{\pgfqpoint{0.704030in}{0.461752in}}%
\pgfpathlineto{\pgfqpoint{0.707125in}{0.458976in}}%
\pgfpathclose%
\pgfusepath{fill}%
\end{pgfscope}%
\begin{pgfscope}%
\pgfpathrectangle{\pgfqpoint{0.100000in}{0.100000in}}{\pgfqpoint{3.420221in}{2.189500in}}%
\pgfusepath{clip}%
\pgfsetbuttcap%
\pgfsetmiterjoin%
\definecolor{currentfill}{rgb}{0.000000,0.380392,0.809804}%
\pgfsetfillcolor{currentfill}%
\pgfsetlinewidth{0.000000pt}%
\definecolor{currentstroke}{rgb}{0.000000,0.000000,0.000000}%
\pgfsetstrokecolor{currentstroke}%
\pgfsetstrokeopacity{0.000000}%
\pgfsetdash{}{0pt}%
\pgfpathmoveto{\pgfqpoint{1.075098in}{1.862908in}}%
\pgfpathlineto{\pgfqpoint{1.072884in}{1.852000in}}%
\pgfpathlineto{\pgfqpoint{1.077952in}{1.847349in}}%
\pgfpathlineto{\pgfqpoint{1.081849in}{1.847585in}}%
\pgfpathlineto{\pgfqpoint{1.082983in}{1.840215in}}%
\pgfpathlineto{\pgfqpoint{1.079267in}{1.820871in}}%
\pgfpathlineto{\pgfqpoint{1.085694in}{1.819623in}}%
\pgfpathlineto{\pgfqpoint{1.084411in}{1.813124in}}%
\pgfpathlineto{\pgfqpoint{1.091900in}{1.811670in}}%
\pgfpathlineto{\pgfqpoint{1.090696in}{1.798580in}}%
\pgfpathlineto{\pgfqpoint{1.095402in}{1.794389in}}%
\pgfpathlineto{\pgfqpoint{1.110285in}{1.791384in}}%
\pgfpathlineto{\pgfqpoint{1.109291in}{1.786033in}}%
\pgfpathlineto{\pgfqpoint{1.125542in}{1.783054in}}%
\pgfpathlineto{\pgfqpoint{1.121392in}{1.774149in}}%
\pgfpathlineto{\pgfqpoint{1.111322in}{1.774272in}}%
\pgfpathlineto{\pgfqpoint{1.104434in}{1.771659in}}%
\pgfpathlineto{\pgfqpoint{1.102695in}{1.776274in}}%
\pgfpathlineto{\pgfqpoint{1.090137in}{1.775425in}}%
\pgfpathlineto{\pgfqpoint{1.080619in}{1.780656in}}%
\pgfpathlineto{\pgfqpoint{1.076378in}{1.778497in}}%
\pgfpathlineto{\pgfqpoint{1.073743in}{1.772782in}}%
\pgfpathlineto{\pgfqpoint{1.069657in}{1.776059in}}%
\pgfpathlineto{\pgfqpoint{1.053495in}{1.779606in}}%
\pgfpathlineto{\pgfqpoint{1.050199in}{1.771971in}}%
\pgfpathlineto{\pgfqpoint{1.048288in}{1.770922in}}%
\pgfpathlineto{\pgfqpoint{1.041288in}{1.779426in}}%
\pgfpathlineto{\pgfqpoint{1.042216in}{1.785000in}}%
\pgfpathlineto{\pgfqpoint{1.040240in}{1.800998in}}%
\pgfpathlineto{\pgfqpoint{1.036558in}{1.807322in}}%
\pgfpathlineto{\pgfqpoint{1.028302in}{1.810400in}}%
\pgfpathlineto{\pgfqpoint{1.023830in}{1.814795in}}%
\pgfpathlineto{\pgfqpoint{1.026921in}{1.829095in}}%
\pgfpathlineto{\pgfqpoint{1.022384in}{1.834159in}}%
\pgfpathlineto{\pgfqpoint{1.016206in}{1.851908in}}%
\pgfpathlineto{\pgfqpoint{1.017377in}{1.856546in}}%
\pgfpathlineto{\pgfqpoint{1.014146in}{1.867281in}}%
\pgfpathlineto{\pgfqpoint{1.016847in}{1.872988in}}%
\pgfpathlineto{\pgfqpoint{1.011530in}{1.881323in}}%
\pgfpathlineto{\pgfqpoint{1.019033in}{1.884505in}}%
\pgfpathlineto{\pgfqpoint{1.026000in}{1.889654in}}%
\pgfpathlineto{\pgfqpoint{1.034820in}{1.890409in}}%
\pgfpathlineto{\pgfqpoint{1.036983in}{1.894726in}}%
\pgfpathlineto{\pgfqpoint{1.040548in}{1.880815in}}%
\pgfpathlineto{\pgfqpoint{1.050918in}{1.886535in}}%
\pgfpathlineto{\pgfqpoint{1.057529in}{1.886121in}}%
\pgfpathlineto{\pgfqpoint{1.060925in}{1.880603in}}%
\pgfpathlineto{\pgfqpoint{1.073154in}{1.872670in}}%
\pgfpathclose%
\pgfusepath{fill}%
\end{pgfscope}%
\begin{pgfscope}%
\pgfpathrectangle{\pgfqpoint{0.100000in}{0.100000in}}{\pgfqpoint{3.420221in}{2.189500in}}%
\pgfusepath{clip}%
\pgfsetbuttcap%
\pgfsetmiterjoin%
\definecolor{currentfill}{rgb}{0.000000,0.627451,0.686275}%
\pgfsetfillcolor{currentfill}%
\pgfsetlinewidth{0.000000pt}%
\definecolor{currentstroke}{rgb}{0.000000,0.000000,0.000000}%
\pgfsetstrokecolor{currentstroke}%
\pgfsetstrokeopacity{0.000000}%
\pgfsetdash{}{0pt}%
\pgfpathmoveto{\pgfqpoint{0.829088in}{1.947246in}}%
\pgfpathlineto{\pgfqpoint{0.863981in}{1.938253in}}%
\pgfpathlineto{\pgfqpoint{0.867725in}{1.924123in}}%
\pgfpathlineto{\pgfqpoint{0.871998in}{1.922673in}}%
\pgfpathlineto{\pgfqpoint{0.878729in}{1.914372in}}%
\pgfpathlineto{\pgfqpoint{0.880335in}{1.904794in}}%
\pgfpathlineto{\pgfqpoint{0.873537in}{1.897775in}}%
\pgfpathlineto{\pgfqpoint{0.862672in}{1.882079in}}%
\pgfpathlineto{\pgfqpoint{0.858234in}{1.873406in}}%
\pgfpathlineto{\pgfqpoint{0.854302in}{1.869474in}}%
\pgfpathlineto{\pgfqpoint{0.829409in}{1.875766in}}%
\pgfpathlineto{\pgfqpoint{0.830933in}{1.881991in}}%
\pgfpathlineto{\pgfqpoint{0.820112in}{1.884668in}}%
\pgfpathlineto{\pgfqpoint{0.819693in}{1.892738in}}%
\pgfpathlineto{\pgfqpoint{0.814718in}{1.899400in}}%
\pgfpathlineto{\pgfqpoint{0.813742in}{1.913459in}}%
\pgfpathlineto{\pgfqpoint{0.815662in}{1.920952in}}%
\pgfpathlineto{\pgfqpoint{0.813924in}{1.927026in}}%
\pgfpathlineto{\pgfqpoint{0.819187in}{1.939030in}}%
\pgfpathlineto{\pgfqpoint{0.807697in}{1.941951in}}%
\pgfpathlineto{\pgfqpoint{0.810096in}{1.952255in}}%
\pgfpathclose%
\pgfusepath{fill}%
\end{pgfscope}%
\begin{pgfscope}%
\pgfpathrectangle{\pgfqpoint{0.100000in}{0.100000in}}{\pgfqpoint{3.420221in}{2.189500in}}%
\pgfusepath{clip}%
\pgfsetbuttcap%
\pgfsetmiterjoin%
\definecolor{currentfill}{rgb}{0.000000,0.278431,0.860784}%
\pgfsetfillcolor{currentfill}%
\pgfsetlinewidth{0.000000pt}%
\definecolor{currentstroke}{rgb}{0.000000,0.000000,0.000000}%
\pgfsetstrokecolor{currentstroke}%
\pgfsetstrokeopacity{0.000000}%
\pgfsetdash{}{0pt}%
\pgfpathmoveto{\pgfqpoint{1.623298in}{1.266030in}}%
\pgfpathlineto{\pgfqpoint{1.590598in}{1.268282in}}%
\pgfpathlineto{\pgfqpoint{1.595388in}{1.333331in}}%
\pgfpathlineto{\pgfqpoint{1.632008in}{1.330624in}}%
\pgfpathlineto{\pgfqpoint{1.633440in}{1.330519in}}%
\pgfpathlineto{\pgfqpoint{1.631005in}{1.298160in}}%
\pgfpathlineto{\pgfqpoint{1.625992in}{1.298488in}}%
\pgfpathclose%
\pgfusepath{fill}%
\end{pgfscope}%
\begin{pgfscope}%
\pgfpathrectangle{\pgfqpoint{0.100000in}{0.100000in}}{\pgfqpoint{3.420221in}{2.189500in}}%
\pgfusepath{clip}%
\pgfsetbuttcap%
\pgfsetmiterjoin%
\definecolor{currentfill}{rgb}{0.000000,0.086275,0.956863}%
\pgfsetfillcolor{currentfill}%
\pgfsetlinewidth{0.000000pt}%
\definecolor{currentstroke}{rgb}{0.000000,0.000000,0.000000}%
\pgfsetstrokecolor{currentstroke}%
\pgfsetstrokeopacity{0.000000}%
\pgfsetdash{}{0pt}%
\pgfpathmoveto{\pgfqpoint{1.933940in}{0.812869in}}%
\pgfpathlineto{\pgfqpoint{1.933568in}{0.776610in}}%
\pgfpathlineto{\pgfqpoint{1.913946in}{0.774695in}}%
\pgfpathlineto{\pgfqpoint{1.880869in}{0.756078in}}%
\pgfpathlineto{\pgfqpoint{1.877992in}{0.754486in}}%
\pgfpathlineto{\pgfqpoint{1.869055in}{0.770787in}}%
\pgfpathlineto{\pgfqpoint{1.869506in}{0.792199in}}%
\pgfpathlineto{\pgfqpoint{1.872583in}{0.792069in}}%
\pgfpathlineto{\pgfqpoint{1.873699in}{0.824910in}}%
\pgfpathlineto{\pgfqpoint{1.850583in}{0.825894in}}%
\pgfpathlineto{\pgfqpoint{1.852435in}{0.859193in}}%
\pgfpathlineto{\pgfqpoint{1.845903in}{0.859439in}}%
\pgfpathlineto{\pgfqpoint{1.847076in}{0.895655in}}%
\pgfpathlineto{\pgfqpoint{1.849384in}{0.888699in}}%
\pgfpathlineto{\pgfqpoint{1.853827in}{0.888256in}}%
\pgfpathlineto{\pgfqpoint{1.861979in}{0.894343in}}%
\pgfpathlineto{\pgfqpoint{1.865861in}{0.893835in}}%
\pgfpathlineto{\pgfqpoint{1.864260in}{0.886641in}}%
\pgfpathlineto{\pgfqpoint{1.867322in}{0.880731in}}%
\pgfpathlineto{\pgfqpoint{1.870898in}{0.880939in}}%
\pgfpathlineto{\pgfqpoint{1.880797in}{0.897379in}}%
\pgfpathlineto{\pgfqpoint{1.879987in}{0.857296in}}%
\pgfpathlineto{\pgfqpoint{1.886420in}{0.856348in}}%
\pgfpathlineto{\pgfqpoint{1.915101in}{0.855254in}}%
\pgfpathlineto{\pgfqpoint{1.920651in}{0.851717in}}%
\pgfpathlineto{\pgfqpoint{1.919984in}{0.813294in}}%
\pgfpathclose%
\pgfusepath{fill}%
\end{pgfscope}%
\begin{pgfscope}%
\pgfpathrectangle{\pgfqpoint{0.100000in}{0.100000in}}{\pgfqpoint{3.420221in}{2.189500in}}%
\pgfusepath{clip}%
\pgfsetbuttcap%
\pgfsetmiterjoin%
\definecolor{currentfill}{rgb}{0.000000,0.419608,0.790196}%
\pgfsetfillcolor{currentfill}%
\pgfsetlinewidth{0.000000pt}%
\definecolor{currentstroke}{rgb}{0.000000,0.000000,0.000000}%
\pgfsetstrokecolor{currentstroke}%
\pgfsetstrokeopacity{0.000000}%
\pgfsetdash{}{0pt}%
\pgfpathmoveto{\pgfqpoint{1.880797in}{0.897379in}}%
\pgfpathlineto{\pgfqpoint{1.870898in}{0.880939in}}%
\pgfpathlineto{\pgfqpoint{1.867322in}{0.880731in}}%
\pgfpathlineto{\pgfqpoint{1.864260in}{0.886641in}}%
\pgfpathlineto{\pgfqpoint{1.865861in}{0.893835in}}%
\pgfpathlineto{\pgfqpoint{1.861979in}{0.894343in}}%
\pgfpathlineto{\pgfqpoint{1.853827in}{0.888256in}}%
\pgfpathlineto{\pgfqpoint{1.849384in}{0.888699in}}%
\pgfpathlineto{\pgfqpoint{1.847076in}{0.895655in}}%
\pgfpathlineto{\pgfqpoint{1.842257in}{0.894605in}}%
\pgfpathlineto{\pgfqpoint{1.843796in}{0.953335in}}%
\pgfpathlineto{\pgfqpoint{1.837270in}{0.953538in}}%
\pgfpathlineto{\pgfqpoint{1.837680in}{0.966588in}}%
\pgfpathlineto{\pgfqpoint{1.882992in}{0.965391in}}%
\pgfpathlineto{\pgfqpoint{1.882587in}{0.945758in}}%
\pgfpathlineto{\pgfqpoint{1.889089in}{0.945628in}}%
\pgfpathlineto{\pgfqpoint{1.888957in}{0.939078in}}%
\pgfpathlineto{\pgfqpoint{1.908363in}{0.938676in}}%
\pgfpathlineto{\pgfqpoint{1.908260in}{0.932231in}}%
\pgfpathlineto{\pgfqpoint{1.914858in}{0.932044in}}%
\pgfpathlineto{\pgfqpoint{1.914535in}{0.912409in}}%
\pgfpathlineto{\pgfqpoint{1.900911in}{0.900986in}}%
\pgfpathlineto{\pgfqpoint{1.895114in}{0.888244in}}%
\pgfpathlineto{\pgfqpoint{1.884513in}{0.891192in}}%
\pgfpathclose%
\pgfusepath{fill}%
\end{pgfscope}%
\begin{pgfscope}%
\pgfpathrectangle{\pgfqpoint{0.100000in}{0.100000in}}{\pgfqpoint{3.420221in}{2.189500in}}%
\pgfusepath{clip}%
\pgfsetbuttcap%
\pgfsetmiterjoin%
\definecolor{currentfill}{rgb}{0.000000,0.764706,0.617647}%
\pgfsetfillcolor{currentfill}%
\pgfsetlinewidth{0.000000pt}%
\definecolor{currentstroke}{rgb}{0.000000,0.000000,0.000000}%
\pgfsetstrokecolor{currentstroke}%
\pgfsetstrokeopacity{0.000000}%
\pgfsetdash{}{0pt}%
\pgfpathmoveto{\pgfqpoint{1.290004in}{0.905403in}}%
\pgfpathlineto{\pgfqpoint{1.290863in}{0.898520in}}%
\pgfpathlineto{\pgfqpoint{1.288030in}{0.873223in}}%
\pgfpathlineto{\pgfqpoint{1.251495in}{0.860451in}}%
\pgfpathlineto{\pgfqpoint{1.251075in}{0.857198in}}%
\pgfpathlineto{\pgfqpoint{1.225235in}{0.860730in}}%
\pgfpathlineto{\pgfqpoint{1.223452in}{0.847680in}}%
\pgfpathlineto{\pgfqpoint{1.204024in}{0.850348in}}%
\pgfpathlineto{\pgfqpoint{1.196181in}{0.853300in}}%
\pgfpathlineto{\pgfqpoint{1.198710in}{0.864291in}}%
\pgfpathlineto{\pgfqpoint{1.196265in}{0.870442in}}%
\pgfpathlineto{\pgfqpoint{1.197373in}{0.876820in}}%
\pgfpathlineto{\pgfqpoint{1.193456in}{0.879904in}}%
\pgfpathlineto{\pgfqpoint{1.194839in}{0.887306in}}%
\pgfpathlineto{\pgfqpoint{1.193569in}{0.897266in}}%
\pgfpathlineto{\pgfqpoint{1.185762in}{0.898306in}}%
\pgfpathlineto{\pgfqpoint{1.188729in}{0.918907in}}%
\pgfpathlineto{\pgfqpoint{1.192521in}{0.918086in}}%
\pgfpathlineto{\pgfqpoint{1.254494in}{0.909662in}}%
\pgfpathclose%
\pgfusepath{fill}%
\end{pgfscope}%
\begin{pgfscope}%
\pgfpathrectangle{\pgfqpoint{0.100000in}{0.100000in}}{\pgfqpoint{3.420221in}{2.189500in}}%
\pgfusepath{clip}%
\pgfsetbuttcap%
\pgfsetmiterjoin%
\definecolor{currentfill}{rgb}{0.000000,0.129412,0.935294}%
\pgfsetfillcolor{currentfill}%
\pgfsetlinewidth{0.000000pt}%
\definecolor{currentstroke}{rgb}{0.000000,0.000000,0.000000}%
\pgfsetstrokecolor{currentstroke}%
\pgfsetstrokeopacity{0.000000}%
\pgfsetdash{}{0pt}%
\pgfpathmoveto{\pgfqpoint{1.941226in}{1.855330in}}%
\pgfpathlineto{\pgfqpoint{1.941260in}{1.848732in}}%
\pgfpathlineto{\pgfqpoint{1.935618in}{1.848817in}}%
\pgfpathlineto{\pgfqpoint{1.935295in}{1.822483in}}%
\pgfpathlineto{\pgfqpoint{1.935770in}{1.802787in}}%
\pgfpathlineto{\pgfqpoint{1.919474in}{1.803061in}}%
\pgfpathlineto{\pgfqpoint{1.920061in}{1.796513in}}%
\pgfpathlineto{\pgfqpoint{1.885162in}{1.797233in}}%
\pgfpathlineto{\pgfqpoint{1.883613in}{1.801948in}}%
\pgfpathlineto{\pgfqpoint{1.883078in}{1.823500in}}%
\pgfpathlineto{\pgfqpoint{1.883765in}{1.849712in}}%
\pgfpathlineto{\pgfqpoint{1.862904in}{1.850337in}}%
\pgfpathlineto{\pgfqpoint{1.863696in}{1.876696in}}%
\pgfpathlineto{\pgfqpoint{1.863028in}{1.896506in}}%
\pgfpathlineto{\pgfqpoint{1.907894in}{1.895201in}}%
\pgfpathlineto{\pgfqpoint{1.907479in}{1.915012in}}%
\pgfpathlineto{\pgfqpoint{1.947469in}{1.914387in}}%
\pgfpathlineto{\pgfqpoint{1.947184in}{1.888071in}}%
\pgfpathlineto{\pgfqpoint{1.940614in}{1.888108in}}%
\pgfpathlineto{\pgfqpoint{1.941510in}{1.874960in}}%
\pgfpathclose%
\pgfusepath{fill}%
\end{pgfscope}%
\begin{pgfscope}%
\pgfpathrectangle{\pgfqpoint{0.100000in}{0.100000in}}{\pgfqpoint{3.420221in}{2.189500in}}%
\pgfusepath{clip}%
\pgfsetbuttcap%
\pgfsetmiterjoin%
\definecolor{currentfill}{rgb}{0.000000,0.870588,0.564706}%
\pgfsetfillcolor{currentfill}%
\pgfsetlinewidth{0.000000pt}%
\definecolor{currentstroke}{rgb}{0.000000,0.000000,0.000000}%
\pgfsetstrokecolor{currentstroke}%
\pgfsetstrokeopacity{0.000000}%
\pgfsetdash{}{0pt}%
\pgfpathmoveto{\pgfqpoint{2.600403in}{1.698834in}}%
\pgfpathlineto{\pgfqpoint{2.561638in}{1.694127in}}%
\pgfpathlineto{\pgfqpoint{2.548562in}{1.692702in}}%
\pgfpathlineto{\pgfqpoint{2.545681in}{1.718849in}}%
\pgfpathlineto{\pgfqpoint{2.542941in}{1.744869in}}%
\pgfpathlineto{\pgfqpoint{2.540293in}{1.764119in}}%
\pgfpathlineto{\pgfqpoint{2.546753in}{1.764724in}}%
\pgfpathlineto{\pgfqpoint{2.546043in}{1.771263in}}%
\pgfpathlineto{\pgfqpoint{2.571775in}{1.774012in}}%
\pgfpathlineto{\pgfqpoint{2.591004in}{1.776583in}}%
\pgfpathlineto{\pgfqpoint{2.592142in}{1.763691in}}%
\pgfpathclose%
\pgfusepath{fill}%
\end{pgfscope}%
\begin{pgfscope}%
\pgfpathrectangle{\pgfqpoint{0.100000in}{0.100000in}}{\pgfqpoint{3.420221in}{2.189500in}}%
\pgfusepath{clip}%
\pgfsetbuttcap%
\pgfsetmiterjoin%
\definecolor{currentfill}{rgb}{0.000000,0.760784,0.619608}%
\pgfsetfillcolor{currentfill}%
\pgfsetlinewidth{0.000000pt}%
\definecolor{currentstroke}{rgb}{0.000000,0.000000,0.000000}%
\pgfsetstrokecolor{currentstroke}%
\pgfsetstrokeopacity{0.000000}%
\pgfsetdash{}{0pt}%
\pgfpathmoveto{\pgfqpoint{2.914591in}{0.439577in}}%
\pgfpathlineto{\pgfqpoint{2.917346in}{0.419570in}}%
\pgfpathlineto{\pgfqpoint{2.883802in}{0.414752in}}%
\pgfpathlineto{\pgfqpoint{2.883282in}{0.422326in}}%
\pgfpathlineto{\pgfqpoint{2.877810in}{0.427505in}}%
\pgfpathlineto{\pgfqpoint{2.874700in}{0.423728in}}%
\pgfpathlineto{\pgfqpoint{2.877773in}{0.415272in}}%
\pgfpathlineto{\pgfqpoint{2.871767in}{0.414114in}}%
\pgfpathlineto{\pgfqpoint{2.869385in}{0.414143in}}%
\pgfpathlineto{\pgfqpoint{2.854744in}{0.433068in}}%
\pgfpathlineto{\pgfqpoint{2.844060in}{0.449880in}}%
\pgfpathlineto{\pgfqpoint{2.834225in}{0.459861in}}%
\pgfpathlineto{\pgfqpoint{2.836266in}{0.467028in}}%
\pgfpathlineto{\pgfqpoint{2.841260in}{0.476284in}}%
\pgfpathlineto{\pgfqpoint{2.874791in}{0.481116in}}%
\pgfpathlineto{\pgfqpoint{2.878039in}{0.457800in}}%
\pgfpathlineto{\pgfqpoint{2.911068in}{0.462818in}}%
\pgfpathclose%
\pgfusepath{fill}%
\end{pgfscope}%
\begin{pgfscope}%
\pgfpathrectangle{\pgfqpoint{0.100000in}{0.100000in}}{\pgfqpoint{3.420221in}{2.189500in}}%
\pgfusepath{clip}%
\pgfsetbuttcap%
\pgfsetmiterjoin%
\definecolor{currentfill}{rgb}{0.000000,0.388235,0.805882}%
\pgfsetfillcolor{currentfill}%
\pgfsetlinewidth{0.000000pt}%
\definecolor{currentstroke}{rgb}{0.000000,0.000000,0.000000}%
\pgfsetstrokecolor{currentstroke}%
\pgfsetstrokeopacity{0.000000}%
\pgfsetdash{}{0pt}%
\pgfpathmoveto{\pgfqpoint{2.892820in}{1.528533in}}%
\pgfpathlineto{\pgfqpoint{2.891827in}{1.534635in}}%
\pgfpathlineto{\pgfqpoint{2.897925in}{1.537886in}}%
\pgfpathlineto{\pgfqpoint{2.896798in}{1.550794in}}%
\pgfpathlineto{\pgfqpoint{2.922787in}{1.555960in}}%
\pgfpathlineto{\pgfqpoint{2.938379in}{1.557955in}}%
\pgfpathlineto{\pgfqpoint{2.948792in}{1.549023in}}%
\pgfpathlineto{\pgfqpoint{2.952182in}{1.549615in}}%
\pgfpathlineto{\pgfqpoint{2.953702in}{1.541710in}}%
\pgfpathlineto{\pgfqpoint{2.950006in}{1.529528in}}%
\pgfpathlineto{\pgfqpoint{2.913797in}{1.526351in}}%
\pgfpathlineto{\pgfqpoint{2.900067in}{1.531444in}}%
\pgfpathclose%
\pgfusepath{fill}%
\end{pgfscope}%
\begin{pgfscope}%
\pgfpathrectangle{\pgfqpoint{0.100000in}{0.100000in}}{\pgfqpoint{3.420221in}{2.189500in}}%
\pgfusepath{clip}%
\pgfsetbuttcap%
\pgfsetmiterjoin%
\definecolor{currentfill}{rgb}{0.000000,0.529412,0.735294}%
\pgfsetfillcolor{currentfill}%
\pgfsetlinewidth{0.000000pt}%
\definecolor{currentstroke}{rgb}{0.000000,0.000000,0.000000}%
\pgfsetstrokecolor{currentstroke}%
\pgfsetstrokeopacity{0.000000}%
\pgfsetdash{}{0pt}%
\pgfpathmoveto{\pgfqpoint{2.453887in}{1.287376in}}%
\pgfpathlineto{\pgfqpoint{2.437529in}{1.285590in}}%
\pgfpathlineto{\pgfqpoint{2.433957in}{1.291594in}}%
\pgfpathlineto{\pgfqpoint{2.433697in}{1.296918in}}%
\pgfpathlineto{\pgfqpoint{2.429834in}{1.304131in}}%
\pgfpathlineto{\pgfqpoint{2.432062in}{1.307542in}}%
\pgfpathlineto{\pgfqpoint{2.430219in}{1.315711in}}%
\pgfpathlineto{\pgfqpoint{2.434283in}{1.318811in}}%
\pgfpathlineto{\pgfqpoint{2.430901in}{1.358769in}}%
\pgfpathlineto{\pgfqpoint{2.429374in}{1.378558in}}%
\pgfpathlineto{\pgfqpoint{2.436624in}{1.377617in}}%
\pgfpathlineto{\pgfqpoint{2.436942in}{1.364482in}}%
\pgfpathlineto{\pgfqpoint{2.455779in}{1.366133in}}%
\pgfpathlineto{\pgfqpoint{2.454005in}{1.385668in}}%
\pgfpathlineto{\pgfqpoint{2.476602in}{1.387657in}}%
\pgfpathlineto{\pgfqpoint{2.476844in}{1.384992in}}%
\pgfpathlineto{\pgfqpoint{2.481074in}{1.344123in}}%
\pgfpathlineto{\pgfqpoint{2.483536in}{1.338454in}}%
\pgfpathlineto{\pgfqpoint{2.482202in}{1.332178in}}%
\pgfpathlineto{\pgfqpoint{2.467603in}{1.331098in}}%
\pgfpathlineto{\pgfqpoint{2.468330in}{1.321287in}}%
\pgfpathlineto{\pgfqpoint{2.461908in}{1.320763in}}%
\pgfpathlineto{\pgfqpoint{2.462998in}{1.307733in}}%
\pgfpathlineto{\pgfqpoint{2.452203in}{1.307076in}}%
\pgfpathclose%
\pgfusepath{fill}%
\end{pgfscope}%
\begin{pgfscope}%
\pgfpathrectangle{\pgfqpoint{0.100000in}{0.100000in}}{\pgfqpoint{3.420221in}{2.189500in}}%
\pgfusepath{clip}%
\pgfsetbuttcap%
\pgfsetmiterjoin%
\definecolor{currentfill}{rgb}{0.000000,0.258824,0.870588}%
\pgfsetfillcolor{currentfill}%
\pgfsetlinewidth{0.000000pt}%
\definecolor{currentstroke}{rgb}{0.000000,0.000000,0.000000}%
\pgfsetstrokecolor{currentstroke}%
\pgfsetstrokeopacity{0.000000}%
\pgfsetdash{}{0pt}%
\pgfpathmoveto{\pgfqpoint{1.502857in}{1.570607in}}%
\pgfpathlineto{\pgfqpoint{1.495057in}{1.492423in}}%
\pgfpathlineto{\pgfqpoint{1.461553in}{1.495939in}}%
\pgfpathlineto{\pgfqpoint{1.462234in}{1.502502in}}%
\pgfpathlineto{\pgfqpoint{1.427460in}{1.506606in}}%
\pgfpathlineto{\pgfqpoint{1.430608in}{1.531116in}}%
\pgfpathlineto{\pgfqpoint{1.433707in}{1.564391in}}%
\pgfpathlineto{\pgfqpoint{1.435043in}{1.577414in}}%
\pgfpathlineto{\pgfqpoint{1.456705in}{1.575235in}}%
\pgfpathclose%
\pgfusepath{fill}%
\end{pgfscope}%
\begin{pgfscope}%
\pgfpathrectangle{\pgfqpoint{0.100000in}{0.100000in}}{\pgfqpoint{3.420221in}{2.189500in}}%
\pgfusepath{clip}%
\pgfsetbuttcap%
\pgfsetmiterjoin%
\definecolor{currentfill}{rgb}{0.000000,0.478431,0.760784}%
\pgfsetfillcolor{currentfill}%
\pgfsetlinewidth{0.000000pt}%
\definecolor{currentstroke}{rgb}{0.000000,0.000000,0.000000}%
\pgfsetstrokecolor{currentstroke}%
\pgfsetstrokeopacity{0.000000}%
\pgfsetdash{}{0pt}%
\pgfpathmoveto{\pgfqpoint{1.570538in}{0.972766in}}%
\pgfpathlineto{\pgfqpoint{1.602969in}{0.970304in}}%
\pgfpathlineto{\pgfqpoint{1.600642in}{0.937683in}}%
\pgfpathlineto{\pgfqpoint{1.627283in}{0.935857in}}%
\pgfpathlineto{\pgfqpoint{1.624911in}{0.899929in}}%
\pgfpathlineto{\pgfqpoint{1.559762in}{0.903967in}}%
\pgfpathlineto{\pgfqpoint{1.562332in}{0.940610in}}%
\pgfpathlineto{\pgfqpoint{1.568050in}{0.940159in}}%
\pgfpathclose%
\pgfusepath{fill}%
\end{pgfscope}%
\begin{pgfscope}%
\pgfpathrectangle{\pgfqpoint{0.100000in}{0.100000in}}{\pgfqpoint{3.420221in}{2.189500in}}%
\pgfusepath{clip}%
\pgfsetbuttcap%
\pgfsetmiterjoin%
\definecolor{currentfill}{rgb}{0.000000,0.184314,0.907843}%
\pgfsetfillcolor{currentfill}%
\pgfsetlinewidth{0.000000pt}%
\definecolor{currentstroke}{rgb}{0.000000,0.000000,0.000000}%
\pgfsetstrokecolor{currentstroke}%
\pgfsetstrokeopacity{0.000000}%
\pgfsetdash{}{0pt}%
\pgfpathmoveto{\pgfqpoint{1.328624in}{1.317869in}}%
\pgfpathlineto{\pgfqpoint{1.325819in}{1.321737in}}%
\pgfpathlineto{\pgfqpoint{1.320384in}{1.322298in}}%
\pgfpathlineto{\pgfqpoint{1.316870in}{1.318005in}}%
\pgfpathlineto{\pgfqpoint{1.310732in}{1.319787in}}%
\pgfpathlineto{\pgfqpoint{1.302857in}{1.330371in}}%
\pgfpathlineto{\pgfqpoint{1.290404in}{1.332043in}}%
\pgfpathlineto{\pgfqpoint{1.286871in}{1.336633in}}%
\pgfpathlineto{\pgfqpoint{1.285356in}{1.343136in}}%
\pgfpathlineto{\pgfqpoint{1.281886in}{1.348003in}}%
\pgfpathlineto{\pgfqpoint{1.284383in}{1.351579in}}%
\pgfpathlineto{\pgfqpoint{1.228273in}{1.359595in}}%
\pgfpathlineto{\pgfqpoint{1.191570in}{1.365236in}}%
\pgfpathlineto{\pgfqpoint{1.193046in}{1.374938in}}%
\pgfpathlineto{\pgfqpoint{1.194936in}{1.386988in}}%
\pgfpathlineto{\pgfqpoint{1.223129in}{1.381891in}}%
\pgfpathlineto{\pgfqpoint{1.223608in}{1.385113in}}%
\pgfpathlineto{\pgfqpoint{1.258760in}{1.380015in}}%
\pgfpathlineto{\pgfqpoint{1.260151in}{1.389695in}}%
\pgfpathlineto{\pgfqpoint{1.289058in}{1.385889in}}%
\pgfpathlineto{\pgfqpoint{1.289977in}{1.392521in}}%
\pgfpathlineto{\pgfqpoint{1.296377in}{1.391435in}}%
\pgfpathlineto{\pgfqpoint{1.298257in}{1.404472in}}%
\pgfpathlineto{\pgfqpoint{1.314150in}{1.402388in}}%
\pgfpathlineto{\pgfqpoint{1.312761in}{1.389525in}}%
\pgfpathlineto{\pgfqpoint{1.335903in}{1.386939in}}%
\pgfpathlineto{\pgfqpoint{1.357433in}{1.383303in}}%
\pgfpathlineto{\pgfqpoint{1.365751in}{1.374270in}}%
\pgfpathlineto{\pgfqpoint{1.367969in}{1.365158in}}%
\pgfpathlineto{\pgfqpoint{1.373896in}{1.365152in}}%
\pgfpathlineto{\pgfqpoint{1.381376in}{1.359045in}}%
\pgfpathlineto{\pgfqpoint{1.369774in}{1.344170in}}%
\pgfpathlineto{\pgfqpoint{1.355473in}{1.337996in}}%
\pgfpathlineto{\pgfqpoint{1.354540in}{1.325234in}}%
\pgfpathlineto{\pgfqpoint{1.352738in}{1.319215in}}%
\pgfpathlineto{\pgfqpoint{1.330421in}{1.322164in}}%
\pgfpathclose%
\pgfusepath{fill}%
\end{pgfscope}%
\begin{pgfscope}%
\pgfpathrectangle{\pgfqpoint{0.100000in}{0.100000in}}{\pgfqpoint{3.420221in}{2.189500in}}%
\pgfusepath{clip}%
\pgfsetbuttcap%
\pgfsetmiterjoin%
\definecolor{currentfill}{rgb}{0.000000,0.545098,0.727451}%
\pgfsetfillcolor{currentfill}%
\pgfsetlinewidth{0.000000pt}%
\definecolor{currentstroke}{rgb}{0.000000,0.000000,0.000000}%
\pgfsetstrokecolor{currentstroke}%
\pgfsetstrokeopacity{0.000000}%
\pgfsetdash{}{0pt}%
\pgfpathmoveto{\pgfqpoint{2.935857in}{0.489944in}}%
\pgfpathlineto{\pgfqpoint{2.932570in}{0.480238in}}%
\pgfpathlineto{\pgfqpoint{2.940828in}{0.470730in}}%
\pgfpathlineto{\pgfqpoint{2.944208in}{0.470901in}}%
\pgfpathlineto{\pgfqpoint{2.952340in}{0.459572in}}%
\pgfpathlineto{\pgfqpoint{2.939092in}{0.456922in}}%
\pgfpathlineto{\pgfqpoint{2.940156in}{0.450279in}}%
\pgfpathlineto{\pgfqpoint{2.933473in}{0.449208in}}%
\pgfpathlineto{\pgfqpoint{2.934530in}{0.442510in}}%
\pgfpathlineto{\pgfqpoint{2.914591in}{0.439577in}}%
\pgfpathlineto{\pgfqpoint{2.911068in}{0.462818in}}%
\pgfpathlineto{\pgfqpoint{2.878039in}{0.457800in}}%
\pgfpathlineto{\pgfqpoint{2.874791in}{0.481116in}}%
\pgfpathlineto{\pgfqpoint{2.868975in}{0.520764in}}%
\pgfpathlineto{\pgfqpoint{2.864710in}{0.526890in}}%
\pgfpathlineto{\pgfqpoint{2.867475in}{0.531354in}}%
\pgfpathlineto{\pgfqpoint{2.873642in}{0.534802in}}%
\pgfpathlineto{\pgfqpoint{2.893550in}{0.537863in}}%
\pgfpathlineto{\pgfqpoint{2.894535in}{0.531236in}}%
\pgfpathlineto{\pgfqpoint{2.901156in}{0.532230in}}%
\pgfpathlineto{\pgfqpoint{2.904675in}{0.523772in}}%
\pgfpathlineto{\pgfqpoint{2.915296in}{0.515661in}}%
\pgfpathlineto{\pgfqpoint{2.921269in}{0.509192in}}%
\pgfpathlineto{\pgfqpoint{2.922481in}{0.504860in}}%
\pgfpathlineto{\pgfqpoint{2.928816in}{0.503272in}}%
\pgfpathlineto{\pgfqpoint{2.934427in}{0.495471in}}%
\pgfpathclose%
\pgfusepath{fill}%
\end{pgfscope}%
\begin{pgfscope}%
\pgfpathrectangle{\pgfqpoint{0.100000in}{0.100000in}}{\pgfqpoint{3.420221in}{2.189500in}}%
\pgfusepath{clip}%
\pgfsetbuttcap%
\pgfsetmiterjoin%
\definecolor{currentfill}{rgb}{0.000000,0.898039,0.550980}%
\pgfsetfillcolor{currentfill}%
\pgfsetlinewidth{0.000000pt}%
\definecolor{currentstroke}{rgb}{0.000000,0.000000,0.000000}%
\pgfsetstrokecolor{currentstroke}%
\pgfsetstrokeopacity{0.000000}%
\pgfsetdash{}{0pt}%
\pgfpathmoveto{\pgfqpoint{0.488169in}{1.245512in}}%
\pgfpathlineto{\pgfqpoint{0.526622in}{1.234749in}}%
\pgfpathlineto{\pgfqpoint{0.580156in}{1.220219in}}%
\pgfpathlineto{\pgfqpoint{0.616577in}{1.210635in}}%
\pgfpathlineto{\pgfqpoint{0.638838in}{1.205551in}}%
\pgfpathlineto{\pgfqpoint{0.620855in}{1.134593in}}%
\pgfpathlineto{\pgfqpoint{0.572078in}{1.146879in}}%
\pgfpathlineto{\pgfqpoint{0.524599in}{1.159354in}}%
\pgfpathlineto{\pgfqpoint{0.523917in}{1.164641in}}%
\pgfpathlineto{\pgfqpoint{0.514520in}{1.168905in}}%
\pgfpathlineto{\pgfqpoint{0.516225in}{1.182117in}}%
\pgfpathlineto{\pgfqpoint{0.511214in}{1.184343in}}%
\pgfpathlineto{\pgfqpoint{0.513498in}{1.190902in}}%
\pgfpathlineto{\pgfqpoint{0.506687in}{1.192390in}}%
\pgfpathlineto{\pgfqpoint{0.508477in}{1.198730in}}%
\pgfpathlineto{\pgfqpoint{0.500067in}{1.201159in}}%
\pgfpathlineto{\pgfqpoint{0.501840in}{1.207467in}}%
\pgfpathlineto{\pgfqpoint{0.497671in}{1.208663in}}%
\pgfpathlineto{\pgfqpoint{0.499463in}{1.215016in}}%
\pgfpathlineto{\pgfqpoint{0.493159in}{1.219096in}}%
\pgfpathlineto{\pgfqpoint{0.489090in}{1.224759in}}%
\pgfpathlineto{\pgfqpoint{0.490933in}{1.231100in}}%
\pgfpathlineto{\pgfqpoint{0.484591in}{1.232887in}}%
\pgfpathclose%
\pgfusepath{fill}%
\end{pgfscope}%
\begin{pgfscope}%
\pgfpathrectangle{\pgfqpoint{0.100000in}{0.100000in}}{\pgfqpoint{3.420221in}{2.189500in}}%
\pgfusepath{clip}%
\pgfsetbuttcap%
\pgfsetmiterjoin%
\definecolor{currentfill}{rgb}{0.000000,0.368627,0.815686}%
\pgfsetfillcolor{currentfill}%
\pgfsetlinewidth{0.000000pt}%
\definecolor{currentstroke}{rgb}{0.000000,0.000000,0.000000}%
\pgfsetstrokecolor{currentstroke}%
\pgfsetstrokeopacity{0.000000}%
\pgfsetdash{}{0pt}%
\pgfpathmoveto{\pgfqpoint{2.084243in}{1.394033in}}%
\pgfpathlineto{\pgfqpoint{2.071871in}{1.393650in}}%
\pgfpathlineto{\pgfqpoint{2.032210in}{1.392764in}}%
\pgfpathlineto{\pgfqpoint{2.031989in}{1.436739in}}%
\pgfpathlineto{\pgfqpoint{2.064270in}{1.436940in}}%
\pgfpathlineto{\pgfqpoint{2.083525in}{1.437583in}}%
\pgfpathclose%
\pgfusepath{fill}%
\end{pgfscope}%
\begin{pgfscope}%
\pgfpathrectangle{\pgfqpoint{0.100000in}{0.100000in}}{\pgfqpoint{3.420221in}{2.189500in}}%
\pgfusepath{clip}%
\pgfsetbuttcap%
\pgfsetmiterjoin%
\definecolor{currentfill}{rgb}{0.000000,0.643137,0.678431}%
\pgfsetfillcolor{currentfill}%
\pgfsetlinewidth{0.000000pt}%
\definecolor{currentstroke}{rgb}{0.000000,0.000000,0.000000}%
\pgfsetstrokecolor{currentstroke}%
\pgfsetstrokeopacity{0.000000}%
\pgfsetdash{}{0pt}%
\pgfpathmoveto{\pgfqpoint{2.587935in}{1.145769in}}%
\pgfpathlineto{\pgfqpoint{2.586385in}{1.150927in}}%
\pgfpathlineto{\pgfqpoint{2.573177in}{1.150506in}}%
\pgfpathlineto{\pgfqpoint{2.569575in}{1.155323in}}%
\pgfpathlineto{\pgfqpoint{2.567375in}{1.162795in}}%
\pgfpathlineto{\pgfqpoint{2.560946in}{1.163313in}}%
\pgfpathlineto{\pgfqpoint{2.557312in}{1.167238in}}%
\pgfpathlineto{\pgfqpoint{2.555754in}{1.176086in}}%
\pgfpathlineto{\pgfqpoint{2.557212in}{1.185310in}}%
\pgfpathlineto{\pgfqpoint{2.558186in}{1.189136in}}%
\pgfpathlineto{\pgfqpoint{2.568187in}{1.189724in}}%
\pgfpathlineto{\pgfqpoint{2.575075in}{1.190934in}}%
\pgfpathlineto{\pgfqpoint{2.581575in}{1.187757in}}%
\pgfpathlineto{\pgfqpoint{2.585616in}{1.191554in}}%
\pgfpathlineto{\pgfqpoint{2.593547in}{1.188427in}}%
\pgfpathlineto{\pgfqpoint{2.592758in}{1.198374in}}%
\pgfpathlineto{\pgfqpoint{2.603681in}{1.198329in}}%
\pgfpathlineto{\pgfqpoint{2.604897in}{1.190815in}}%
\pgfpathlineto{\pgfqpoint{2.613044in}{1.186694in}}%
\pgfpathlineto{\pgfqpoint{2.614187in}{1.177617in}}%
\pgfpathlineto{\pgfqpoint{2.604501in}{1.167336in}}%
\pgfpathlineto{\pgfqpoint{2.604795in}{1.162160in}}%
\pgfpathlineto{\pgfqpoint{2.609417in}{1.158920in}}%
\pgfpathlineto{\pgfqpoint{2.593943in}{1.144379in}}%
\pgfpathclose%
\pgfusepath{fill}%
\end{pgfscope}%
\begin{pgfscope}%
\pgfpathrectangle{\pgfqpoint{0.100000in}{0.100000in}}{\pgfqpoint{3.420221in}{2.189500in}}%
\pgfusepath{clip}%
\pgfsetbuttcap%
\pgfsetmiterjoin%
\definecolor{currentfill}{rgb}{0.000000,0.415686,0.792157}%
\pgfsetfillcolor{currentfill}%
\pgfsetlinewidth{0.000000pt}%
\definecolor{currentstroke}{rgb}{0.000000,0.000000,0.000000}%
\pgfsetstrokecolor{currentstroke}%
\pgfsetstrokeopacity{0.000000}%
\pgfsetdash{}{0pt}%
\pgfpathmoveto{\pgfqpoint{1.627283in}{0.935857in}}%
\pgfpathlineto{\pgfqpoint{1.600642in}{0.937683in}}%
\pgfpathlineto{\pgfqpoint{1.602969in}{0.970304in}}%
\pgfpathlineto{\pgfqpoint{1.626424in}{0.968739in}}%
\pgfpathlineto{\pgfqpoint{1.628880in}{1.001237in}}%
\pgfpathlineto{\pgfqpoint{1.662404in}{0.999165in}}%
\pgfpathlineto{\pgfqpoint{1.660304in}{0.966529in}}%
\pgfpathlineto{\pgfqpoint{1.667994in}{0.966050in}}%
\pgfpathlineto{\pgfqpoint{1.665935in}{0.933499in}}%
\pgfpathlineto{\pgfqpoint{1.659767in}{0.933914in}}%
\pgfpathclose%
\pgfusepath{fill}%
\end{pgfscope}%
\begin{pgfscope}%
\pgfpathrectangle{\pgfqpoint{0.100000in}{0.100000in}}{\pgfqpoint{3.420221in}{2.189500in}}%
\pgfusepath{clip}%
\pgfsetbuttcap%
\pgfsetmiterjoin%
\definecolor{currentfill}{rgb}{0.000000,0.635294,0.682353}%
\pgfsetfillcolor{currentfill}%
\pgfsetlinewidth{0.000000pt}%
\definecolor{currentstroke}{rgb}{0.000000,0.000000,0.000000}%
\pgfsetstrokecolor{currentstroke}%
\pgfsetstrokeopacity{0.000000}%
\pgfsetdash{}{0pt}%
\pgfpathmoveto{\pgfqpoint{1.584701in}{1.189583in}}%
\pgfpathlineto{\pgfqpoint{1.583218in}{1.170522in}}%
\pgfpathlineto{\pgfqpoint{1.580884in}{1.140923in}}%
\pgfpathlineto{\pgfqpoint{1.541755in}{1.144259in}}%
\pgfpathlineto{\pgfqpoint{1.523651in}{1.146199in}}%
\pgfpathlineto{\pgfqpoint{1.518648in}{1.146655in}}%
\pgfpathlineto{\pgfqpoint{1.523533in}{1.194617in}}%
\pgfpathclose%
\pgfusepath{fill}%
\end{pgfscope}%
\begin{pgfscope}%
\pgfpathrectangle{\pgfqpoint{0.100000in}{0.100000in}}{\pgfqpoint{3.420221in}{2.189500in}}%
\pgfusepath{clip}%
\pgfsetbuttcap%
\pgfsetmiterjoin%
\definecolor{currentfill}{rgb}{0.000000,0.117647,0.941176}%
\pgfsetfillcolor{currentfill}%
\pgfsetlinewidth{0.000000pt}%
\definecolor{currentstroke}{rgb}{0.000000,0.000000,0.000000}%
\pgfsetstrokecolor{currentstroke}%
\pgfsetstrokeopacity{0.000000}%
\pgfsetdash{}{0pt}%
\pgfpathmoveto{\pgfqpoint{2.550889in}{1.051856in}}%
\pgfpathlineto{\pgfqpoint{2.546871in}{1.050242in}}%
\pgfpathlineto{\pgfqpoint{2.534883in}{1.052457in}}%
\pgfpathlineto{\pgfqpoint{2.521360in}{1.047953in}}%
\pgfpathlineto{\pgfqpoint{2.518497in}{1.050860in}}%
\pgfpathlineto{\pgfqpoint{2.510993in}{1.049975in}}%
\pgfpathlineto{\pgfqpoint{2.500291in}{1.051602in}}%
\pgfpathlineto{\pgfqpoint{2.474985in}{1.059179in}}%
\pgfpathlineto{\pgfqpoint{2.474945in}{1.067350in}}%
\pgfpathlineto{\pgfqpoint{2.476009in}{1.082158in}}%
\pgfpathlineto{\pgfqpoint{2.474318in}{1.086247in}}%
\pgfpathlineto{\pgfqpoint{2.475574in}{1.093646in}}%
\pgfpathlineto{\pgfqpoint{2.467660in}{1.094038in}}%
\pgfpathlineto{\pgfqpoint{2.469501in}{1.098410in}}%
\pgfpathlineto{\pgfqpoint{2.476254in}{1.105317in}}%
\pgfpathlineto{\pgfqpoint{2.475905in}{1.118871in}}%
\pgfpathlineto{\pgfqpoint{2.479149in}{1.119212in}}%
\pgfpathlineto{\pgfqpoint{2.507298in}{1.122451in}}%
\pgfpathlineto{\pgfqpoint{2.522784in}{1.123629in}}%
\pgfpathlineto{\pgfqpoint{2.544205in}{1.124211in}}%
\pgfpathlineto{\pgfqpoint{2.555100in}{1.124163in}}%
\pgfpathlineto{\pgfqpoint{2.554920in}{1.115547in}}%
\pgfpathlineto{\pgfqpoint{2.554559in}{1.099582in}}%
\pgfpathlineto{\pgfqpoint{2.558008in}{1.093760in}}%
\pgfpathlineto{\pgfqpoint{2.558094in}{1.088134in}}%
\pgfpathlineto{\pgfqpoint{2.554936in}{1.084956in}}%
\pgfpathlineto{\pgfqpoint{2.549041in}{1.085668in}}%
\pgfpathlineto{\pgfqpoint{2.543615in}{1.081234in}}%
\pgfpathlineto{\pgfqpoint{2.546852in}{1.074149in}}%
\pgfpathlineto{\pgfqpoint{2.553970in}{1.068182in}}%
\pgfpathlineto{\pgfqpoint{2.550346in}{1.057416in}}%
\pgfpathclose%
\pgfusepath{fill}%
\end{pgfscope}%
\begin{pgfscope}%
\pgfpathrectangle{\pgfqpoint{0.100000in}{0.100000in}}{\pgfqpoint{3.420221in}{2.189500in}}%
\pgfusepath{clip}%
\pgfsetbuttcap%
\pgfsetmiterjoin%
\definecolor{currentfill}{rgb}{0.000000,0.654902,0.672549}%
\pgfsetfillcolor{currentfill}%
\pgfsetlinewidth{0.000000pt}%
\definecolor{currentstroke}{rgb}{0.000000,0.000000,0.000000}%
\pgfsetstrokecolor{currentstroke}%
\pgfsetstrokeopacity{0.000000}%
\pgfsetdash{}{0pt}%
\pgfpathmoveto{\pgfqpoint{0.465872in}{1.326215in}}%
\pgfpathlineto{\pgfqpoint{0.455080in}{1.336918in}}%
\pgfpathlineto{\pgfqpoint{0.451650in}{1.344932in}}%
\pgfpathlineto{\pgfqpoint{0.456099in}{1.354724in}}%
\pgfpathlineto{\pgfqpoint{0.456423in}{1.361584in}}%
\pgfpathlineto{\pgfqpoint{0.446758in}{1.364592in}}%
\pgfpathlineto{\pgfqpoint{0.446735in}{1.374396in}}%
\pgfpathlineto{\pgfqpoint{0.451367in}{1.381255in}}%
\pgfpathlineto{\pgfqpoint{0.449934in}{1.388939in}}%
\pgfpathlineto{\pgfqpoint{0.449968in}{1.388899in}}%
\pgfpathlineto{\pgfqpoint{0.469645in}{1.399676in}}%
\pgfpathlineto{\pgfqpoint{0.483174in}{1.400853in}}%
\pgfpathlineto{\pgfqpoint{0.486813in}{1.398059in}}%
\pgfpathlineto{\pgfqpoint{0.493631in}{1.422507in}}%
\pgfpathlineto{\pgfqpoint{0.503949in}{1.400364in}}%
\pgfpathlineto{\pgfqpoint{0.514360in}{1.406799in}}%
\pgfpathlineto{\pgfqpoint{0.526617in}{1.418655in}}%
\pgfpathlineto{\pgfqpoint{0.542200in}{1.431593in}}%
\pgfpathlineto{\pgfqpoint{0.552257in}{1.433384in}}%
\pgfpathlineto{\pgfqpoint{0.558333in}{1.425664in}}%
\pgfpathlineto{\pgfqpoint{0.567128in}{1.427943in}}%
\pgfpathlineto{\pgfqpoint{0.571639in}{1.419667in}}%
\pgfpathlineto{\pgfqpoint{0.570130in}{1.416943in}}%
\pgfpathlineto{\pgfqpoint{0.571803in}{1.406463in}}%
\pgfpathlineto{\pgfqpoint{0.583420in}{1.397463in}}%
\pgfpathlineto{\pgfqpoint{0.582854in}{1.387029in}}%
\pgfpathlineto{\pgfqpoint{0.586345in}{1.383239in}}%
\pgfpathlineto{\pgfqpoint{0.585849in}{1.374410in}}%
\pgfpathlineto{\pgfqpoint{0.580995in}{1.371409in}}%
\pgfpathlineto{\pgfqpoint{0.579476in}{1.374837in}}%
\pgfpathlineto{\pgfqpoint{0.559438in}{1.363441in}}%
\pgfpathlineto{\pgfqpoint{0.558087in}{1.358710in}}%
\pgfpathlineto{\pgfqpoint{0.552707in}{1.354226in}}%
\pgfpathlineto{\pgfqpoint{0.546350in}{1.355942in}}%
\pgfpathlineto{\pgfqpoint{0.524944in}{1.343740in}}%
\pgfpathlineto{\pgfqpoint{0.510656in}{1.345426in}}%
\pgfpathlineto{\pgfqpoint{0.498761in}{1.344453in}}%
\pgfpathclose%
\pgfusepath{fill}%
\end{pgfscope}%
\begin{pgfscope}%
\pgfpathrectangle{\pgfqpoint{0.100000in}{0.100000in}}{\pgfqpoint{3.420221in}{2.189500in}}%
\pgfusepath{clip}%
\pgfsetbuttcap%
\pgfsetmiterjoin%
\definecolor{currentfill}{rgb}{0.000000,0.286275,0.856863}%
\pgfsetfillcolor{currentfill}%
\pgfsetlinewidth{0.000000pt}%
\definecolor{currentstroke}{rgb}{0.000000,0.000000,0.000000}%
\pgfsetstrokecolor{currentstroke}%
\pgfsetstrokeopacity{0.000000}%
\pgfsetdash{}{0pt}%
\pgfpathmoveto{\pgfqpoint{1.583050in}{1.814998in}}%
\pgfpathlineto{\pgfqpoint{1.528185in}{1.820168in}}%
\pgfpathlineto{\pgfqpoint{1.532672in}{1.864929in}}%
\pgfpathlineto{\pgfqpoint{1.545306in}{1.863670in}}%
\pgfpathlineto{\pgfqpoint{1.545954in}{1.870352in}}%
\pgfpathlineto{\pgfqpoint{1.555861in}{1.869418in}}%
\pgfpathlineto{\pgfqpoint{1.591078in}{1.866225in}}%
\pgfpathlineto{\pgfqpoint{1.588963in}{1.839973in}}%
\pgfpathlineto{\pgfqpoint{1.585270in}{1.840235in}}%
\pgfpathclose%
\pgfusepath{fill}%
\end{pgfscope}%
\begin{pgfscope}%
\pgfpathrectangle{\pgfqpoint{0.100000in}{0.100000in}}{\pgfqpoint{3.420221in}{2.189500in}}%
\pgfusepath{clip}%
\pgfsetbuttcap%
\pgfsetmiterjoin%
\definecolor{currentfill}{rgb}{0.000000,0.262745,0.868627}%
\pgfsetfillcolor{currentfill}%
\pgfsetlinewidth{0.000000pt}%
\definecolor{currentstroke}{rgb}{0.000000,0.000000,0.000000}%
\pgfsetstrokecolor{currentstroke}%
\pgfsetstrokeopacity{0.000000}%
\pgfsetdash{}{0pt}%
\pgfpathmoveto{\pgfqpoint{3.116341in}{1.335878in}}%
\pgfpathlineto{\pgfqpoint{3.112511in}{1.336515in}}%
\pgfpathlineto{\pgfqpoint{3.101933in}{1.332853in}}%
\pgfpathlineto{\pgfqpoint{3.089310in}{1.347652in}}%
\pgfpathlineto{\pgfqpoint{3.091325in}{1.350419in}}%
\pgfpathlineto{\pgfqpoint{3.089982in}{1.359434in}}%
\pgfpathlineto{\pgfqpoint{3.097053in}{1.362338in}}%
\pgfpathlineto{\pgfqpoint{3.084619in}{1.369658in}}%
\pgfpathlineto{\pgfqpoint{3.089777in}{1.382305in}}%
\pgfpathlineto{\pgfqpoint{3.086019in}{1.387091in}}%
\pgfpathlineto{\pgfqpoint{3.085726in}{1.393533in}}%
\pgfpathlineto{\pgfqpoint{3.082023in}{1.398128in}}%
\pgfpathlineto{\pgfqpoint{3.084766in}{1.411509in}}%
\pgfpathlineto{\pgfqpoint{3.091678in}{1.417486in}}%
\pgfpathlineto{\pgfqpoint{3.101074in}{1.418013in}}%
\pgfpathlineto{\pgfqpoint{3.107585in}{1.420655in}}%
\pgfpathlineto{\pgfqpoint{3.109232in}{1.414832in}}%
\pgfpathlineto{\pgfqpoint{3.124163in}{1.361653in}}%
\pgfpathlineto{\pgfqpoint{3.118418in}{1.354310in}}%
\pgfpathlineto{\pgfqpoint{3.115322in}{1.340266in}}%
\pgfpathclose%
\pgfusepath{fill}%
\end{pgfscope}%
\begin{pgfscope}%
\pgfpathrectangle{\pgfqpoint{0.100000in}{0.100000in}}{\pgfqpoint{3.420221in}{2.189500in}}%
\pgfusepath{clip}%
\pgfsetbuttcap%
\pgfsetmiterjoin%
\definecolor{currentfill}{rgb}{0.000000,0.164706,0.917647}%
\pgfsetfillcolor{currentfill}%
\pgfsetlinewidth{0.000000pt}%
\definecolor{currentstroke}{rgb}{0.000000,0.000000,0.000000}%
\pgfsetstrokecolor{currentstroke}%
\pgfsetstrokeopacity{0.000000}%
\pgfsetdash{}{0pt}%
\pgfpathmoveto{\pgfqpoint{2.403073in}{1.691155in}}%
\pgfpathlineto{\pgfqpoint{2.384514in}{1.689643in}}%
\pgfpathlineto{\pgfqpoint{2.385006in}{1.683097in}}%
\pgfpathlineto{\pgfqpoint{2.376664in}{1.682492in}}%
\pgfpathlineto{\pgfqpoint{2.368634in}{1.681962in}}%
\pgfpathlineto{\pgfqpoint{2.366799in}{1.707844in}}%
\pgfpathlineto{\pgfqpoint{2.363892in}{1.707475in}}%
\pgfpathlineto{\pgfqpoint{2.362220in}{1.727568in}}%
\pgfpathlineto{\pgfqpoint{2.349391in}{1.726840in}}%
\pgfpathlineto{\pgfqpoint{2.348267in}{1.746512in}}%
\pgfpathlineto{\pgfqpoint{2.339916in}{1.745954in}}%
\pgfpathlineto{\pgfqpoint{2.337393in}{1.752399in}}%
\pgfpathlineto{\pgfqpoint{2.336602in}{1.765467in}}%
\pgfpathlineto{\pgfqpoint{2.349801in}{1.766279in}}%
\pgfpathlineto{\pgfqpoint{2.356273in}{1.766568in}}%
\pgfpathlineto{\pgfqpoint{2.357061in}{1.753552in}}%
\pgfpathlineto{\pgfqpoint{2.363568in}{1.753779in}}%
\pgfpathlineto{\pgfqpoint{2.367646in}{1.747446in}}%
\pgfpathlineto{\pgfqpoint{2.368022in}{1.740926in}}%
\pgfpathlineto{\pgfqpoint{2.387405in}{1.738040in}}%
\pgfpathlineto{\pgfqpoint{2.377121in}{1.718773in}}%
\pgfpathlineto{\pgfqpoint{2.374865in}{1.706919in}}%
\pgfpathlineto{\pgfqpoint{2.380232in}{1.704523in}}%
\pgfpathlineto{\pgfqpoint{2.384115in}{1.711345in}}%
\pgfpathlineto{\pgfqpoint{2.388673in}{1.713416in}}%
\pgfpathlineto{\pgfqpoint{2.396205in}{1.728996in}}%
\pgfpathlineto{\pgfqpoint{2.403116in}{1.731402in}}%
\pgfpathlineto{\pgfqpoint{2.406648in}{1.735207in}}%
\pgfpathlineto{\pgfqpoint{2.413251in}{1.749391in}}%
\pgfpathlineto{\pgfqpoint{2.413911in}{1.755468in}}%
\pgfpathlineto{\pgfqpoint{2.424505in}{1.759732in}}%
\pgfpathlineto{\pgfqpoint{2.424452in}{1.751614in}}%
\pgfpathlineto{\pgfqpoint{2.417839in}{1.740671in}}%
\pgfpathlineto{\pgfqpoint{2.417676in}{1.733635in}}%
\pgfpathlineto{\pgfqpoint{2.414346in}{1.731010in}}%
\pgfpathlineto{\pgfqpoint{2.406931in}{1.712403in}}%
\pgfpathclose%
\pgfusepath{fill}%
\end{pgfscope}%
\begin{pgfscope}%
\pgfpathrectangle{\pgfqpoint{0.100000in}{0.100000in}}{\pgfqpoint{3.420221in}{2.189500in}}%
\pgfusepath{clip}%
\pgfsetbuttcap%
\pgfsetmiterjoin%
\definecolor{currentfill}{rgb}{0.000000,0.478431,0.760784}%
\pgfsetfillcolor{currentfill}%
\pgfsetlinewidth{0.000000pt}%
\definecolor{currentstroke}{rgb}{0.000000,0.000000,0.000000}%
\pgfsetstrokecolor{currentstroke}%
\pgfsetstrokeopacity{0.000000}%
\pgfsetdash{}{0pt}%
\pgfpathmoveto{\pgfqpoint{2.985139in}{1.463500in}}%
\pgfpathlineto{\pgfqpoint{2.978934in}{1.446884in}}%
\pgfpathlineto{\pgfqpoint{2.974915in}{1.449194in}}%
\pgfpathlineto{\pgfqpoint{2.962266in}{1.451636in}}%
\pgfpathlineto{\pgfqpoint{2.955736in}{1.454929in}}%
\pgfpathlineto{\pgfqpoint{2.953538in}{1.460106in}}%
\pgfpathlineto{\pgfqpoint{2.958140in}{1.475005in}}%
\pgfpathlineto{\pgfqpoint{2.953333in}{1.479934in}}%
\pgfpathlineto{\pgfqpoint{2.954186in}{1.484678in}}%
\pgfpathlineto{\pgfqpoint{2.950343in}{1.488226in}}%
\pgfpathlineto{\pgfqpoint{2.942478in}{1.490597in}}%
\pgfpathlineto{\pgfqpoint{2.917269in}{1.485969in}}%
\pgfpathlineto{\pgfqpoint{2.914664in}{1.499317in}}%
\pgfpathlineto{\pgfqpoint{2.891719in}{1.495647in}}%
\pgfpathlineto{\pgfqpoint{2.889929in}{1.505930in}}%
\pgfpathlineto{\pgfqpoint{2.886621in}{1.526756in}}%
\pgfpathlineto{\pgfqpoint{2.892820in}{1.528533in}}%
\pgfpathlineto{\pgfqpoint{2.900067in}{1.531444in}}%
\pgfpathlineto{\pgfqpoint{2.913797in}{1.526351in}}%
\pgfpathlineto{\pgfqpoint{2.950006in}{1.529528in}}%
\pgfpathlineto{\pgfqpoint{2.953945in}{1.525441in}}%
\pgfpathlineto{\pgfqpoint{2.967395in}{1.529799in}}%
\pgfpathlineto{\pgfqpoint{2.970791in}{1.525044in}}%
\pgfpathlineto{\pgfqpoint{2.975310in}{1.526118in}}%
\pgfpathlineto{\pgfqpoint{2.986028in}{1.517022in}}%
\pgfpathlineto{\pgfqpoint{2.994878in}{1.520659in}}%
\pgfpathlineto{\pgfqpoint{3.005121in}{1.526964in}}%
\pgfpathlineto{\pgfqpoint{2.995770in}{1.510110in}}%
\pgfpathlineto{\pgfqpoint{2.981135in}{1.499447in}}%
\pgfpathlineto{\pgfqpoint{2.973347in}{1.483154in}}%
\pgfpathlineto{\pgfqpoint{2.975479in}{1.479645in}}%
\pgfpathlineto{\pgfqpoint{2.972163in}{1.472905in}}%
\pgfpathlineto{\pgfqpoint{2.975079in}{1.469307in}}%
\pgfpathlineto{\pgfqpoint{2.980367in}{1.470586in}}%
\pgfpathclose%
\pgfusepath{fill}%
\end{pgfscope}%
\begin{pgfscope}%
\pgfpathrectangle{\pgfqpoint{0.100000in}{0.100000in}}{\pgfqpoint{3.420221in}{2.189500in}}%
\pgfusepath{clip}%
\pgfsetbuttcap%
\pgfsetmiterjoin%
\definecolor{currentfill}{rgb}{0.000000,0.286275,0.856863}%
\pgfsetfillcolor{currentfill}%
\pgfsetlinewidth{0.000000pt}%
\definecolor{currentstroke}{rgb}{0.000000,0.000000,0.000000}%
\pgfsetstrokecolor{currentstroke}%
\pgfsetstrokeopacity{0.000000}%
\pgfsetdash{}{0pt}%
\pgfpathmoveto{\pgfqpoint{2.087519in}{1.516249in}}%
\pgfpathlineto{\pgfqpoint{2.048535in}{1.515790in}}%
\pgfpathlineto{\pgfqpoint{2.009974in}{1.515561in}}%
\pgfpathlineto{\pgfqpoint{2.010033in}{1.535401in}}%
\pgfpathlineto{\pgfqpoint{2.006829in}{1.535419in}}%
\pgfpathlineto{\pgfqpoint{2.006833in}{1.568114in}}%
\pgfpathlineto{\pgfqpoint{2.058667in}{1.568271in}}%
\pgfpathlineto{\pgfqpoint{2.084673in}{1.568690in}}%
\pgfpathlineto{\pgfqpoint{2.085190in}{1.535851in}}%
\pgfpathlineto{\pgfqpoint{2.087208in}{1.535874in}}%
\pgfpathclose%
\pgfusepath{fill}%
\end{pgfscope}%
\begin{pgfscope}%
\pgfpathrectangle{\pgfqpoint{0.100000in}{0.100000in}}{\pgfqpoint{3.420221in}{2.189500in}}%
\pgfusepath{clip}%
\pgfsetbuttcap%
\pgfsetmiterjoin%
\definecolor{currentfill}{rgb}{0.000000,0.333333,0.833333}%
\pgfsetfillcolor{currentfill}%
\pgfsetlinewidth{0.000000pt}%
\definecolor{currentstroke}{rgb}{0.000000,0.000000,0.000000}%
\pgfsetstrokecolor{currentstroke}%
\pgfsetstrokeopacity{0.000000}%
\pgfsetdash{}{0pt}%
\pgfpathmoveto{\pgfqpoint{3.291027in}{1.613899in}}%
\pgfpathlineto{\pgfqpoint{3.286458in}{1.613999in}}%
\pgfpathlineto{\pgfqpoint{3.277654in}{1.608733in}}%
\pgfpathlineto{\pgfqpoint{3.272434in}{1.608434in}}%
\pgfpathlineto{\pgfqpoint{3.263286in}{1.602080in}}%
\pgfpathlineto{\pgfqpoint{3.259480in}{1.602516in}}%
\pgfpathlineto{\pgfqpoint{3.233652in}{1.595034in}}%
\pgfpathlineto{\pgfqpoint{3.198653in}{1.564938in}}%
\pgfpathlineto{\pgfqpoint{3.192775in}{1.572376in}}%
\pgfpathlineto{\pgfqpoint{3.204266in}{1.583772in}}%
\pgfpathlineto{\pgfqpoint{3.199075in}{1.588896in}}%
\pgfpathlineto{\pgfqpoint{3.196152in}{1.606066in}}%
\pgfpathlineto{\pgfqpoint{3.189295in}{1.644720in}}%
\pgfpathlineto{\pgfqpoint{3.212962in}{1.649734in}}%
\pgfpathlineto{\pgfqpoint{3.262715in}{1.661511in}}%
\pgfpathlineto{\pgfqpoint{3.281284in}{1.664602in}}%
\pgfpathlineto{\pgfqpoint{3.288616in}{1.639224in}}%
\pgfpathlineto{\pgfqpoint{3.292576in}{1.621702in}}%
\pgfpathclose%
\pgfusepath{fill}%
\end{pgfscope}%
\begin{pgfscope}%
\pgfpathrectangle{\pgfqpoint{0.100000in}{0.100000in}}{\pgfqpoint{3.420221in}{2.189500in}}%
\pgfusepath{clip}%
\pgfsetbuttcap%
\pgfsetmiterjoin%
\definecolor{currentfill}{rgb}{0.000000,0.439216,0.780392}%
\pgfsetfillcolor{currentfill}%
\pgfsetlinewidth{0.000000pt}%
\definecolor{currentstroke}{rgb}{0.000000,0.000000,0.000000}%
\pgfsetstrokecolor{currentstroke}%
\pgfsetstrokeopacity{0.000000}%
\pgfsetdash{}{0pt}%
\pgfpathmoveto{\pgfqpoint{2.235823in}{1.376468in}}%
\pgfpathlineto{\pgfqpoint{2.236009in}{1.369620in}}%
\pgfpathlineto{\pgfqpoint{2.202058in}{1.368766in}}%
\pgfpathlineto{\pgfqpoint{2.202382in}{1.372399in}}%
\pgfpathlineto{\pgfqpoint{2.176440in}{1.372100in}}%
\pgfpathlineto{\pgfqpoint{2.176000in}{1.398222in}}%
\pgfpathlineto{\pgfqpoint{2.188147in}{1.399231in}}%
\pgfpathlineto{\pgfqpoint{2.187412in}{1.440395in}}%
\pgfpathlineto{\pgfqpoint{2.206902in}{1.441020in}}%
\pgfpathlineto{\pgfqpoint{2.207141in}{1.434494in}}%
\pgfpathlineto{\pgfqpoint{2.230936in}{1.435391in}}%
\pgfpathlineto{\pgfqpoint{2.240214in}{1.435516in}}%
\pgfpathlineto{\pgfqpoint{2.241474in}{1.403083in}}%
\pgfpathlineto{\pgfqpoint{2.234943in}{1.403048in}}%
\pgfpathclose%
\pgfusepath{fill}%
\end{pgfscope}%
\begin{pgfscope}%
\pgfpathrectangle{\pgfqpoint{0.100000in}{0.100000in}}{\pgfqpoint{3.420221in}{2.189500in}}%
\pgfusepath{clip}%
\pgfsetbuttcap%
\pgfsetmiterjoin%
\definecolor{currentfill}{rgb}{0.000000,0.533333,0.733333}%
\pgfsetfillcolor{currentfill}%
\pgfsetlinewidth{0.000000pt}%
\definecolor{currentstroke}{rgb}{0.000000,0.000000,0.000000}%
\pgfsetstrokecolor{currentstroke}%
\pgfsetstrokeopacity{0.000000}%
\pgfsetdash{}{0pt}%
\pgfpathmoveto{\pgfqpoint{2.657615in}{1.012741in}}%
\pgfpathlineto{\pgfqpoint{2.639305in}{1.010604in}}%
\pgfpathlineto{\pgfqpoint{2.617575in}{1.008104in}}%
\pgfpathlineto{\pgfqpoint{2.613415in}{1.018205in}}%
\pgfpathlineto{\pgfqpoint{2.617237in}{1.030317in}}%
\pgfpathlineto{\pgfqpoint{2.611697in}{1.039306in}}%
\pgfpathlineto{\pgfqpoint{2.617100in}{1.042026in}}%
\pgfpathlineto{\pgfqpoint{2.624644in}{1.055340in}}%
\pgfpathlineto{\pgfqpoint{2.623727in}{1.060974in}}%
\pgfpathlineto{\pgfqpoint{2.626571in}{1.066921in}}%
\pgfpathlineto{\pgfqpoint{2.633023in}{1.064552in}}%
\pgfpathlineto{\pgfqpoint{2.636043in}{1.058954in}}%
\pgfpathlineto{\pgfqpoint{2.649679in}{1.063255in}}%
\pgfpathlineto{\pgfqpoint{2.664650in}{1.058409in}}%
\pgfpathlineto{\pgfqpoint{2.675333in}{1.050814in}}%
\pgfpathlineto{\pgfqpoint{2.672407in}{1.045621in}}%
\pgfpathlineto{\pgfqpoint{2.672754in}{1.037526in}}%
\pgfpathlineto{\pgfqpoint{2.666516in}{1.033241in}}%
\pgfpathlineto{\pgfqpoint{2.661119in}{1.034380in}}%
\pgfpathlineto{\pgfqpoint{2.657454in}{1.030655in}}%
\pgfpathclose%
\pgfusepath{fill}%
\end{pgfscope}%
\begin{pgfscope}%
\pgfpathrectangle{\pgfqpoint{0.100000in}{0.100000in}}{\pgfqpoint{3.420221in}{2.189500in}}%
\pgfusepath{clip}%
\pgfsetbuttcap%
\pgfsetmiterjoin%
\definecolor{currentfill}{rgb}{0.000000,0.568627,0.715686}%
\pgfsetfillcolor{currentfill}%
\pgfsetlinewidth{0.000000pt}%
\definecolor{currentstroke}{rgb}{0.000000,0.000000,0.000000}%
\pgfsetstrokecolor{currentstroke}%
\pgfsetstrokeopacity{0.000000}%
\pgfsetdash{}{0pt}%
\pgfpathmoveto{\pgfqpoint{1.442768in}{1.013837in}}%
\pgfpathlineto{\pgfqpoint{1.443279in}{1.019332in}}%
\pgfpathlineto{\pgfqpoint{1.372180in}{1.027007in}}%
\pgfpathlineto{\pgfqpoint{1.370729in}{1.014003in}}%
\pgfpathlineto{\pgfqpoint{1.344853in}{1.016966in}}%
\pgfpathlineto{\pgfqpoint{1.352954in}{1.086847in}}%
\pgfpathlineto{\pgfqpoint{1.361641in}{1.085349in}}%
\pgfpathlineto{\pgfqpoint{1.372199in}{1.096151in}}%
\pgfpathlineto{\pgfqpoint{1.375200in}{1.104952in}}%
\pgfpathlineto{\pgfqpoint{1.378071in}{1.105333in}}%
\pgfpathlineto{\pgfqpoint{1.380097in}{1.116128in}}%
\pgfpathlineto{\pgfqpoint{1.379836in}{1.135903in}}%
\pgfpathlineto{\pgfqpoint{1.387357in}{1.138627in}}%
\pgfpathlineto{\pgfqpoint{1.391399in}{1.153256in}}%
\pgfpathlineto{\pgfqpoint{1.391467in}{1.159123in}}%
\pgfpathlineto{\pgfqpoint{1.395369in}{1.158691in}}%
\pgfpathlineto{\pgfqpoint{1.439814in}{1.153754in}}%
\pgfpathlineto{\pgfqpoint{1.463680in}{1.151536in}}%
\pgfpathlineto{\pgfqpoint{1.457840in}{1.093473in}}%
\pgfpathlineto{\pgfqpoint{1.432115in}{1.096148in}}%
\pgfpathlineto{\pgfqpoint{1.435028in}{1.082670in}}%
\pgfpathlineto{\pgfqpoint{1.434656in}{1.071635in}}%
\pgfpathlineto{\pgfqpoint{1.432893in}{1.063080in}}%
\pgfpathlineto{\pgfqpoint{1.453849in}{1.060957in}}%
\pgfpathlineto{\pgfqpoint{1.456691in}{1.062199in}}%
\pgfpathlineto{\pgfqpoint{1.469241in}{1.032356in}}%
\pgfpathlineto{\pgfqpoint{1.474561in}{1.031831in}}%
\pgfpathlineto{\pgfqpoint{1.473279in}{1.018293in}}%
\pgfpathlineto{\pgfqpoint{1.459832in}{1.019690in}}%
\pgfpathclose%
\pgfusepath{fill}%
\end{pgfscope}%
\begin{pgfscope}%
\pgfpathrectangle{\pgfqpoint{0.100000in}{0.100000in}}{\pgfqpoint{3.420221in}{2.189500in}}%
\pgfusepath{clip}%
\pgfsetbuttcap%
\pgfsetmiterjoin%
\definecolor{currentfill}{rgb}{0.000000,0.341176,0.829412}%
\pgfsetfillcolor{currentfill}%
\pgfsetlinewidth{0.000000pt}%
\definecolor{currentstroke}{rgb}{0.000000,0.000000,0.000000}%
\pgfsetstrokecolor{currentstroke}%
\pgfsetstrokeopacity{0.000000}%
\pgfsetdash{}{0pt}%
\pgfpathmoveto{\pgfqpoint{2.153835in}{1.465307in}}%
\pgfpathlineto{\pgfqpoint{2.095955in}{1.463778in}}%
\pgfpathlineto{\pgfqpoint{2.094709in}{1.470727in}}%
\pgfpathlineto{\pgfqpoint{2.094386in}{1.490398in}}%
\pgfpathlineto{\pgfqpoint{2.100868in}{1.490491in}}%
\pgfpathlineto{\pgfqpoint{2.100376in}{1.516456in}}%
\pgfpathlineto{\pgfqpoint{2.113161in}{1.516746in}}%
\pgfpathlineto{\pgfqpoint{2.126163in}{1.517114in}}%
\pgfpathlineto{\pgfqpoint{2.125969in}{1.523603in}}%
\pgfpathlineto{\pgfqpoint{2.152004in}{1.524370in}}%
\pgfpathclose%
\pgfusepath{fill}%
\end{pgfscope}%
\begin{pgfscope}%
\pgfpathrectangle{\pgfqpoint{0.100000in}{0.100000in}}{\pgfqpoint{3.420221in}{2.189500in}}%
\pgfusepath{clip}%
\pgfsetbuttcap%
\pgfsetmiterjoin%
\definecolor{currentfill}{rgb}{0.000000,0.580392,0.709804}%
\pgfsetfillcolor{currentfill}%
\pgfsetlinewidth{0.000000pt}%
\definecolor{currentstroke}{rgb}{0.000000,0.000000,0.000000}%
\pgfsetstrokecolor{currentstroke}%
\pgfsetstrokeopacity{0.000000}%
\pgfsetdash{}{0pt}%
\pgfpathmoveto{\pgfqpoint{0.686294in}{1.446910in}}%
\pgfpathlineto{\pgfqpoint{0.666255in}{1.446394in}}%
\pgfpathlineto{\pgfqpoint{0.664727in}{1.440531in}}%
\pgfpathlineto{\pgfqpoint{0.684999in}{1.400852in}}%
\pgfpathlineto{\pgfqpoint{0.636451in}{1.368523in}}%
\pgfpathlineto{\pgfqpoint{0.632115in}{1.369859in}}%
\pgfpathlineto{\pgfqpoint{0.600695in}{1.418501in}}%
\pgfpathlineto{\pgfqpoint{0.614865in}{1.414658in}}%
\pgfpathlineto{\pgfqpoint{0.623655in}{1.446234in}}%
\pgfpathlineto{\pgfqpoint{0.617307in}{1.447881in}}%
\pgfpathlineto{\pgfqpoint{0.619091in}{1.454541in}}%
\pgfpathlineto{\pgfqpoint{0.626420in}{1.462678in}}%
\pgfpathclose%
\pgfusepath{fill}%
\end{pgfscope}%
\begin{pgfscope}%
\pgfpathrectangle{\pgfqpoint{0.100000in}{0.100000in}}{\pgfqpoint{3.420221in}{2.189500in}}%
\pgfusepath{clip}%
\pgfsetbuttcap%
\pgfsetmiterjoin%
\definecolor{currentfill}{rgb}{0.000000,0.658824,0.670588}%
\pgfsetfillcolor{currentfill}%
\pgfsetlinewidth{0.000000pt}%
\definecolor{currentstroke}{rgb}{0.000000,0.000000,0.000000}%
\pgfsetstrokecolor{currentstroke}%
\pgfsetstrokeopacity{0.000000}%
\pgfsetdash{}{0pt}%
\pgfpathmoveto{\pgfqpoint{2.845785in}{1.145513in}}%
\pgfpathlineto{\pgfqpoint{2.837275in}{1.146680in}}%
\pgfpathlineto{\pgfqpoint{2.830662in}{1.141269in}}%
\pgfpathlineto{\pgfqpoint{2.821021in}{1.152797in}}%
\pgfpathlineto{\pgfqpoint{2.820818in}{1.154808in}}%
\pgfpathlineto{\pgfqpoint{2.801362in}{1.152916in}}%
\pgfpathlineto{\pgfqpoint{2.802926in}{1.154951in}}%
\pgfpathlineto{\pgfqpoint{2.808356in}{1.159169in}}%
\pgfpathlineto{\pgfqpoint{2.809112in}{1.163141in}}%
\pgfpathlineto{\pgfqpoint{2.824150in}{1.169603in}}%
\pgfpathlineto{\pgfqpoint{2.833766in}{1.171524in}}%
\pgfpathlineto{\pgfqpoint{2.835866in}{1.174412in}}%
\pgfpathlineto{\pgfqpoint{2.853723in}{1.182986in}}%
\pgfpathlineto{\pgfqpoint{2.859407in}{1.187895in}}%
\pgfpathlineto{\pgfqpoint{2.872118in}{1.172995in}}%
\pgfpathlineto{\pgfqpoint{2.872435in}{1.170339in}}%
\pgfpathlineto{\pgfqpoint{2.863805in}{1.164921in}}%
\pgfpathlineto{\pgfqpoint{2.865120in}{1.160414in}}%
\pgfpathlineto{\pgfqpoint{2.847846in}{1.157982in}}%
\pgfpathlineto{\pgfqpoint{2.844764in}{1.148659in}}%
\pgfpathclose%
\pgfusepath{fill}%
\end{pgfscope}%
\begin{pgfscope}%
\pgfpathrectangle{\pgfqpoint{0.100000in}{0.100000in}}{\pgfqpoint{3.420221in}{2.189500in}}%
\pgfusepath{clip}%
\pgfsetbuttcap%
\pgfsetmiterjoin%
\definecolor{currentfill}{rgb}{0.000000,0.517647,0.741176}%
\pgfsetfillcolor{currentfill}%
\pgfsetlinewidth{0.000000pt}%
\definecolor{currentstroke}{rgb}{0.000000,0.000000,0.000000}%
\pgfsetstrokecolor{currentstroke}%
\pgfsetstrokeopacity{0.000000}%
\pgfsetdash{}{0pt}%
\pgfpathmoveto{\pgfqpoint{2.724745in}{1.480691in}}%
\pgfpathlineto{\pgfqpoint{2.727854in}{1.475723in}}%
\pgfpathlineto{\pgfqpoint{2.731463in}{1.451719in}}%
\pgfpathlineto{\pgfqpoint{2.726153in}{1.450916in}}%
\pgfpathlineto{\pgfqpoint{2.727229in}{1.443546in}}%
\pgfpathlineto{\pgfqpoint{2.718684in}{1.441032in}}%
\pgfpathlineto{\pgfqpoint{2.703569in}{1.438875in}}%
\pgfpathlineto{\pgfqpoint{2.705453in}{1.423633in}}%
\pgfpathlineto{\pgfqpoint{2.689280in}{1.422441in}}%
\pgfpathlineto{\pgfqpoint{2.689013in}{1.426690in}}%
\pgfpathlineto{\pgfqpoint{2.670417in}{1.426431in}}%
\pgfpathlineto{\pgfqpoint{2.669838in}{1.431126in}}%
\pgfpathlineto{\pgfqpoint{2.660228in}{1.429749in}}%
\pgfpathlineto{\pgfqpoint{2.658416in}{1.443238in}}%
\pgfpathlineto{\pgfqpoint{2.675643in}{1.446694in}}%
\pgfpathlineto{\pgfqpoint{2.672731in}{1.468279in}}%
\pgfpathlineto{\pgfqpoint{2.688576in}{1.470632in}}%
\pgfpathlineto{\pgfqpoint{2.710846in}{1.473413in}}%
\pgfpathlineto{\pgfqpoint{2.709871in}{1.478759in}}%
\pgfpathclose%
\pgfusepath{fill}%
\end{pgfscope}%
\begin{pgfscope}%
\pgfpathrectangle{\pgfqpoint{0.100000in}{0.100000in}}{\pgfqpoint{3.420221in}{2.189500in}}%
\pgfusepath{clip}%
\pgfsetbuttcap%
\pgfsetmiterjoin%
\definecolor{currentfill}{rgb}{0.000000,0.670588,0.664706}%
\pgfsetfillcolor{currentfill}%
\pgfsetlinewidth{0.000000pt}%
\definecolor{currentstroke}{rgb}{0.000000,0.000000,0.000000}%
\pgfsetstrokecolor{currentstroke}%
\pgfsetstrokeopacity{0.000000}%
\pgfsetdash{}{0pt}%
\pgfpathmoveto{\pgfqpoint{2.585662in}{1.048432in}}%
\pgfpathlineto{\pgfqpoint{2.594053in}{1.060946in}}%
\pgfpathlineto{\pgfqpoint{2.595392in}{1.072845in}}%
\pgfpathlineto{\pgfqpoint{2.592049in}{1.080185in}}%
\pgfpathlineto{\pgfqpoint{2.592741in}{1.089813in}}%
\pgfpathlineto{\pgfqpoint{2.597800in}{1.089892in}}%
\pgfpathlineto{\pgfqpoint{2.601243in}{1.097560in}}%
\pgfpathlineto{\pgfqpoint{2.597969in}{1.105382in}}%
\pgfpathlineto{\pgfqpoint{2.597242in}{1.113042in}}%
\pgfpathlineto{\pgfqpoint{2.600590in}{1.124847in}}%
\pgfpathlineto{\pgfqpoint{2.606437in}{1.127189in}}%
\pgfpathlineto{\pgfqpoint{2.619560in}{1.124392in}}%
\pgfpathlineto{\pgfqpoint{2.625016in}{1.115245in}}%
\pgfpathlineto{\pgfqpoint{2.622724in}{1.113145in}}%
\pgfpathlineto{\pgfqpoint{2.612819in}{1.105194in}}%
\pgfpathlineto{\pgfqpoint{2.612102in}{1.095756in}}%
\pgfpathlineto{\pgfqpoint{2.624710in}{1.084986in}}%
\pgfpathlineto{\pgfqpoint{2.627863in}{1.078816in}}%
\pgfpathlineto{\pgfqpoint{2.622280in}{1.071714in}}%
\pgfpathlineto{\pgfqpoint{2.626571in}{1.066921in}}%
\pgfpathlineto{\pgfqpoint{2.623727in}{1.060974in}}%
\pgfpathlineto{\pgfqpoint{2.624644in}{1.055340in}}%
\pgfpathlineto{\pgfqpoint{2.617100in}{1.042026in}}%
\pgfpathlineto{\pgfqpoint{2.611697in}{1.039306in}}%
\pgfpathlineto{\pgfqpoint{2.602084in}{1.041763in}}%
\pgfpathlineto{\pgfqpoint{2.599396in}{1.033810in}}%
\pgfpathlineto{\pgfqpoint{2.587506in}{1.042954in}}%
\pgfpathclose%
\pgfusepath{fill}%
\end{pgfscope}%
\begin{pgfscope}%
\pgfpathrectangle{\pgfqpoint{0.100000in}{0.100000in}}{\pgfqpoint{3.420221in}{2.189500in}}%
\pgfusepath{clip}%
\pgfsetbuttcap%
\pgfsetmiterjoin%
\definecolor{currentfill}{rgb}{0.000000,0.647059,0.676471}%
\pgfsetfillcolor{currentfill}%
\pgfsetlinewidth{0.000000pt}%
\definecolor{currentstroke}{rgb}{0.000000,0.000000,0.000000}%
\pgfsetstrokecolor{currentstroke}%
\pgfsetstrokeopacity{0.000000}%
\pgfsetdash{}{0pt}%
\pgfpathmoveto{\pgfqpoint{3.105841in}{1.164131in}}%
\pgfpathlineto{\pgfqpoint{3.106548in}{1.157625in}}%
\pgfpathlineto{\pgfqpoint{3.104318in}{1.151334in}}%
\pgfpathlineto{\pgfqpoint{3.100727in}{1.151736in}}%
\pgfpathlineto{\pgfqpoint{3.097203in}{1.145499in}}%
\pgfpathlineto{\pgfqpoint{3.093602in}{1.151382in}}%
\pgfpathlineto{\pgfqpoint{3.089264in}{1.147479in}}%
\pgfpathlineto{\pgfqpoint{3.083033in}{1.151299in}}%
\pgfpathlineto{\pgfqpoint{3.076193in}{1.151685in}}%
\pgfpathlineto{\pgfqpoint{3.075057in}{1.156021in}}%
\pgfpathlineto{\pgfqpoint{3.067515in}{1.158318in}}%
\pgfpathlineto{\pgfqpoint{3.063421in}{1.153559in}}%
\pgfpathlineto{\pgfqpoint{3.055560in}{1.154512in}}%
\pgfpathlineto{\pgfqpoint{3.054402in}{1.158355in}}%
\pgfpathlineto{\pgfqpoint{3.044870in}{1.161366in}}%
\pgfpathlineto{\pgfqpoint{3.034061in}{1.158889in}}%
\pgfpathlineto{\pgfqpoint{3.024677in}{1.161202in}}%
\pgfpathlineto{\pgfqpoint{3.027780in}{1.175502in}}%
\pgfpathlineto{\pgfqpoint{3.026229in}{1.187604in}}%
\pgfpathlineto{\pgfqpoint{3.034068in}{1.189160in}}%
\pgfpathlineto{\pgfqpoint{3.038259in}{1.202805in}}%
\pgfpathlineto{\pgfqpoint{3.035557in}{1.216103in}}%
\pgfpathlineto{\pgfqpoint{3.040973in}{1.213872in}}%
\pgfpathlineto{\pgfqpoint{3.047743in}{1.216061in}}%
\pgfpathlineto{\pgfqpoint{3.047937in}{1.205255in}}%
\pgfpathlineto{\pgfqpoint{3.051661in}{1.204962in}}%
\pgfpathlineto{\pgfqpoint{3.058776in}{1.194345in}}%
\pgfpathlineto{\pgfqpoint{3.084398in}{1.198942in}}%
\pgfpathlineto{\pgfqpoint{3.085581in}{1.188260in}}%
\pgfpathlineto{\pgfqpoint{3.094922in}{1.187227in}}%
\pgfpathlineto{\pgfqpoint{3.101069in}{1.183376in}}%
\pgfpathlineto{\pgfqpoint{3.100694in}{1.172877in}}%
\pgfpathclose%
\pgfusepath{fill}%
\end{pgfscope}%
\begin{pgfscope}%
\pgfpathrectangle{\pgfqpoint{0.100000in}{0.100000in}}{\pgfqpoint{3.420221in}{2.189500in}}%
\pgfusepath{clip}%
\pgfsetbuttcap%
\pgfsetmiterjoin%
\definecolor{currentfill}{rgb}{0.000000,0.360784,0.819608}%
\pgfsetfillcolor{currentfill}%
\pgfsetlinewidth{0.000000pt}%
\definecolor{currentstroke}{rgb}{0.000000,0.000000,0.000000}%
\pgfsetstrokecolor{currentstroke}%
\pgfsetstrokeopacity{0.000000}%
\pgfsetdash{}{0pt}%
\pgfpathmoveto{\pgfqpoint{1.736360in}{1.057495in}}%
\pgfpathlineto{\pgfqpoint{1.729066in}{1.054891in}}%
\pgfpathlineto{\pgfqpoint{1.725051in}{1.047014in}}%
\pgfpathlineto{\pgfqpoint{1.715395in}{1.047040in}}%
\pgfpathlineto{\pgfqpoint{1.710015in}{1.052516in}}%
\pgfpathlineto{\pgfqpoint{1.704007in}{1.055319in}}%
\pgfpathlineto{\pgfqpoint{1.698280in}{1.049547in}}%
\pgfpathlineto{\pgfqpoint{1.699001in}{1.062637in}}%
\pgfpathlineto{\pgfqpoint{1.700635in}{1.095898in}}%
\pgfpathlineto{\pgfqpoint{1.702774in}{1.133481in}}%
\pgfpathlineto{\pgfqpoint{1.730353in}{1.131874in}}%
\pgfpathlineto{\pgfqpoint{1.735430in}{1.131613in}}%
\pgfpathlineto{\pgfqpoint{1.739867in}{1.122447in}}%
\pgfpathlineto{\pgfqpoint{1.746816in}{1.114997in}}%
\pgfpathlineto{\pgfqpoint{1.753609in}{1.114508in}}%
\pgfpathlineto{\pgfqpoint{1.763799in}{1.100240in}}%
\pgfpathlineto{\pgfqpoint{1.762618in}{1.067453in}}%
\pgfpathlineto{\pgfqpoint{1.736839in}{1.068841in}}%
\pgfpathclose%
\pgfusepath{fill}%
\end{pgfscope}%
\begin{pgfscope}%
\pgfpathrectangle{\pgfqpoint{0.100000in}{0.100000in}}{\pgfqpoint{3.420221in}{2.189500in}}%
\pgfusepath{clip}%
\pgfsetbuttcap%
\pgfsetmiterjoin%
\definecolor{currentfill}{rgb}{0.000000,0.188235,0.905882}%
\pgfsetfillcolor{currentfill}%
\pgfsetlinewidth{0.000000pt}%
\definecolor{currentstroke}{rgb}{0.000000,0.000000,0.000000}%
\pgfsetstrokecolor{currentstroke}%
\pgfsetstrokeopacity{0.000000}%
\pgfsetdash{}{0pt}%
\pgfpathmoveto{\pgfqpoint{2.927511in}{1.037437in}}%
\pgfpathlineto{\pgfqpoint{2.888787in}{1.031926in}}%
\pgfpathlineto{\pgfqpoint{2.900556in}{1.018396in}}%
\pgfpathlineto{\pgfqpoint{2.892526in}{1.012941in}}%
\pgfpathlineto{\pgfqpoint{2.893441in}{1.007997in}}%
\pgfpathlineto{\pgfqpoint{2.887564in}{1.006957in}}%
\pgfpathlineto{\pgfqpoint{2.880980in}{1.009566in}}%
\pgfpathlineto{\pgfqpoint{2.873627in}{1.002160in}}%
\pgfpathlineto{\pgfqpoint{2.872740in}{1.008515in}}%
\pgfpathlineto{\pgfqpoint{2.839061in}{1.005538in}}%
\pgfpathlineto{\pgfqpoint{2.835983in}{1.012914in}}%
\pgfpathlineto{\pgfqpoint{2.834117in}{1.025686in}}%
\pgfpathlineto{\pgfqpoint{2.829974in}{1.039333in}}%
\pgfpathlineto{\pgfqpoint{2.834554in}{1.040897in}}%
\pgfpathlineto{\pgfqpoint{2.835831in}{1.050089in}}%
\pgfpathlineto{\pgfqpoint{2.856814in}{1.052049in}}%
\pgfpathlineto{\pgfqpoint{2.855967in}{1.061094in}}%
\pgfpathlineto{\pgfqpoint{2.858078in}{1.069488in}}%
\pgfpathlineto{\pgfqpoint{2.856683in}{1.080134in}}%
\pgfpathlineto{\pgfqpoint{2.870008in}{1.081374in}}%
\pgfpathlineto{\pgfqpoint{2.865965in}{1.093485in}}%
\pgfpathlineto{\pgfqpoint{2.867714in}{1.107272in}}%
\pgfpathlineto{\pgfqpoint{2.872372in}{1.107473in}}%
\pgfpathlineto{\pgfqpoint{2.883974in}{1.101519in}}%
\pgfpathlineto{\pgfqpoint{2.892493in}{1.098029in}}%
\pgfpathlineto{\pgfqpoint{2.901022in}{1.091945in}}%
\pgfpathlineto{\pgfqpoint{2.903485in}{1.086564in}}%
\pgfpathlineto{\pgfqpoint{2.908402in}{1.084077in}}%
\pgfpathlineto{\pgfqpoint{2.913042in}{1.078039in}}%
\pgfpathlineto{\pgfqpoint{2.911583in}{1.069164in}}%
\pgfpathlineto{\pgfqpoint{2.915289in}{1.063729in}}%
\pgfpathlineto{\pgfqpoint{2.915231in}{1.058641in}}%
\pgfpathlineto{\pgfqpoint{2.922759in}{1.060429in}}%
\pgfpathlineto{\pgfqpoint{2.927719in}{1.054942in}}%
\pgfpathlineto{\pgfqpoint{2.931099in}{1.045504in}}%
\pgfpathclose%
\pgfusepath{fill}%
\end{pgfscope}%
\begin{pgfscope}%
\pgfpathrectangle{\pgfqpoint{0.100000in}{0.100000in}}{\pgfqpoint{3.420221in}{2.189500in}}%
\pgfusepath{clip}%
\pgfsetbuttcap%
\pgfsetmiterjoin%
\definecolor{currentfill}{rgb}{0.000000,0.356863,0.821569}%
\pgfsetfillcolor{currentfill}%
\pgfsetlinewidth{0.000000pt}%
\definecolor{currentstroke}{rgb}{0.000000,0.000000,0.000000}%
\pgfsetstrokecolor{currentstroke}%
\pgfsetstrokeopacity{0.000000}%
\pgfsetdash{}{0pt}%
\pgfpathmoveto{\pgfqpoint{1.933940in}{0.812869in}}%
\pgfpathlineto{\pgfqpoint{1.919984in}{0.813294in}}%
\pgfpathlineto{\pgfqpoint{1.920651in}{0.851717in}}%
\pgfpathlineto{\pgfqpoint{1.948151in}{0.855746in}}%
\pgfpathlineto{\pgfqpoint{1.948662in}{0.887163in}}%
\pgfpathlineto{\pgfqpoint{1.954519in}{0.890543in}}%
\pgfpathlineto{\pgfqpoint{1.965546in}{0.895090in}}%
\pgfpathlineto{\pgfqpoint{1.967638in}{0.891613in}}%
\pgfpathlineto{\pgfqpoint{1.976639in}{0.889902in}}%
\pgfpathlineto{\pgfqpoint{1.983689in}{0.890342in}}%
\pgfpathlineto{\pgfqpoint{1.988426in}{0.897026in}}%
\pgfpathlineto{\pgfqpoint{2.005015in}{0.886097in}}%
\pgfpathlineto{\pgfqpoint{2.010254in}{0.880804in}}%
\pgfpathlineto{\pgfqpoint{2.018001in}{0.877582in}}%
\pgfpathlineto{\pgfqpoint{2.018028in}{0.849427in}}%
\pgfpathlineto{\pgfqpoint{2.009891in}{0.854342in}}%
\pgfpathlineto{\pgfqpoint{1.996243in}{0.854575in}}%
\pgfpathlineto{\pgfqpoint{1.982713in}{0.853040in}}%
\pgfpathlineto{\pgfqpoint{1.982581in}{0.821847in}}%
\pgfpathlineto{\pgfqpoint{1.960052in}{0.821807in}}%
\pgfpathlineto{\pgfqpoint{1.961832in}{0.803699in}}%
\pgfpathlineto{\pgfqpoint{1.951758in}{0.807978in}}%
\pgfpathlineto{\pgfqpoint{1.948644in}{0.807489in}}%
\pgfpathlineto{\pgfqpoint{1.942951in}{0.812673in}}%
\pgfpathclose%
\pgfusepath{fill}%
\end{pgfscope}%
\begin{pgfscope}%
\pgfpathrectangle{\pgfqpoint{0.100000in}{0.100000in}}{\pgfqpoint{3.420221in}{2.189500in}}%
\pgfusepath{clip}%
\pgfsetbuttcap%
\pgfsetmiterjoin%
\definecolor{currentfill}{rgb}{0.000000,0.835294,0.582353}%
\pgfsetfillcolor{currentfill}%
\pgfsetlinewidth{0.000000pt}%
\definecolor{currentstroke}{rgb}{0.000000,0.000000,0.000000}%
\pgfsetstrokecolor{currentstroke}%
\pgfsetstrokeopacity{0.000000}%
\pgfsetdash{}{0pt}%
\pgfpathmoveto{\pgfqpoint{2.811916in}{1.338045in}}%
\pgfpathlineto{\pgfqpoint{2.815571in}{1.326240in}}%
\pgfpathlineto{\pgfqpoint{2.813065in}{1.321828in}}%
\pgfpathlineto{\pgfqpoint{2.818073in}{1.316886in}}%
\pgfpathlineto{\pgfqpoint{2.815948in}{1.312761in}}%
\pgfpathlineto{\pgfqpoint{2.808147in}{1.300931in}}%
\pgfpathlineto{\pgfqpoint{2.792181in}{1.300199in}}%
\pgfpathlineto{\pgfqpoint{2.788194in}{1.304363in}}%
\pgfpathlineto{\pgfqpoint{2.786119in}{1.308571in}}%
\pgfpathlineto{\pgfqpoint{2.784613in}{1.325204in}}%
\pgfpathlineto{\pgfqpoint{2.788819in}{1.329372in}}%
\pgfpathlineto{\pgfqpoint{2.795340in}{1.326870in}}%
\pgfpathlineto{\pgfqpoint{2.805595in}{1.339763in}}%
\pgfpathclose%
\pgfusepath{fill}%
\end{pgfscope}%
\begin{pgfscope}%
\pgfpathrectangle{\pgfqpoint{0.100000in}{0.100000in}}{\pgfqpoint{3.420221in}{2.189500in}}%
\pgfusepath{clip}%
\pgfsetbuttcap%
\pgfsetmiterjoin%
\definecolor{currentfill}{rgb}{0.000000,0.282353,0.858824}%
\pgfsetfillcolor{currentfill}%
\pgfsetlinewidth{0.000000pt}%
\definecolor{currentstroke}{rgb}{0.000000,0.000000,0.000000}%
\pgfsetstrokecolor{currentstroke}%
\pgfsetstrokeopacity{0.000000}%
\pgfsetdash{}{0pt}%
\pgfpathmoveto{\pgfqpoint{1.543671in}{1.974567in}}%
\pgfpathlineto{\pgfqpoint{1.546498in}{2.004294in}}%
\pgfpathlineto{\pgfqpoint{1.551020in}{2.050490in}}%
\pgfpathlineto{\pgfqpoint{1.606198in}{2.045276in}}%
\pgfpathlineto{\pgfqpoint{1.651861in}{2.041486in}}%
\pgfpathlineto{\pgfqpoint{1.650706in}{2.026917in}}%
\pgfpathlineto{\pgfqpoint{1.644195in}{2.027434in}}%
\pgfpathlineto{\pgfqpoint{1.643650in}{2.020850in}}%
\pgfpathlineto{\pgfqpoint{1.639598in}{2.021165in}}%
\pgfpathlineto{\pgfqpoint{1.638517in}{2.007966in}}%
\pgfpathlineto{\pgfqpoint{1.605842in}{2.010721in}}%
\pgfpathlineto{\pgfqpoint{1.604724in}{1.997573in}}%
\pgfpathlineto{\pgfqpoint{1.607608in}{1.997345in}}%
\pgfpathlineto{\pgfqpoint{1.604114in}{1.979668in}}%
\pgfpathlineto{\pgfqpoint{1.587192in}{1.981029in}}%
\pgfpathlineto{\pgfqpoint{1.582856in}{1.975541in}}%
\pgfpathlineto{\pgfqpoint{1.571998in}{1.972636in}}%
\pgfpathlineto{\pgfqpoint{1.567945in}{1.975228in}}%
\pgfpathlineto{\pgfqpoint{1.566570in}{1.981365in}}%
\pgfpathlineto{\pgfqpoint{1.562829in}{1.982271in}}%
\pgfpathlineto{\pgfqpoint{1.552230in}{1.971795in}}%
\pgfpathclose%
\pgfusepath{fill}%
\end{pgfscope}%
\begin{pgfscope}%
\pgfpathrectangle{\pgfqpoint{0.100000in}{0.100000in}}{\pgfqpoint{3.420221in}{2.189500in}}%
\pgfusepath{clip}%
\pgfsetbuttcap%
\pgfsetmiterjoin%
\definecolor{currentfill}{rgb}{0.000000,0.376471,0.811765}%
\pgfsetfillcolor{currentfill}%
\pgfsetlinewidth{0.000000pt}%
\definecolor{currentstroke}{rgb}{0.000000,0.000000,0.000000}%
\pgfsetstrokecolor{currentstroke}%
\pgfsetstrokeopacity{0.000000}%
\pgfsetdash{}{0pt}%
\pgfpathmoveto{\pgfqpoint{1.499877in}{0.908964in}}%
\pgfpathlineto{\pgfqpoint{1.559762in}{0.903967in}}%
\pgfpathlineto{\pgfqpoint{1.624911in}{0.899929in}}%
\pgfpathlineto{\pgfqpoint{1.622877in}{0.867151in}}%
\pgfpathlineto{\pgfqpoint{1.620750in}{0.835066in}}%
\pgfpathlineto{\pgfqpoint{1.612221in}{0.835139in}}%
\pgfpathlineto{\pgfqpoint{1.555288in}{0.839003in}}%
\pgfpathlineto{\pgfqpoint{1.557824in}{0.871277in}}%
\pgfpathlineto{\pgfqpoint{1.496390in}{0.876315in}}%
\pgfpathlineto{\pgfqpoint{1.497857in}{0.889950in}}%
\pgfpathclose%
\pgfusepath{fill}%
\end{pgfscope}%
\begin{pgfscope}%
\pgfpathrectangle{\pgfqpoint{0.100000in}{0.100000in}}{\pgfqpoint{3.420221in}{2.189500in}}%
\pgfusepath{clip}%
\pgfsetbuttcap%
\pgfsetmiterjoin%
\definecolor{currentfill}{rgb}{0.000000,0.458824,0.770588}%
\pgfsetfillcolor{currentfill}%
\pgfsetlinewidth{0.000000pt}%
\definecolor{currentstroke}{rgb}{0.000000,0.000000,0.000000}%
\pgfsetstrokecolor{currentstroke}%
\pgfsetstrokeopacity{0.000000}%
\pgfsetdash{}{0pt}%
\pgfpathmoveto{\pgfqpoint{2.986217in}{1.008864in}}%
\pgfpathlineto{\pgfqpoint{2.958924in}{1.028641in}}%
\pgfpathlineto{\pgfqpoint{2.942880in}{1.039893in}}%
\pgfpathlineto{\pgfqpoint{2.927511in}{1.037437in}}%
\pgfpathlineto{\pgfqpoint{2.931099in}{1.045504in}}%
\pgfpathlineto{\pgfqpoint{2.927719in}{1.054942in}}%
\pgfpathlineto{\pgfqpoint{2.922759in}{1.060429in}}%
\pgfpathlineto{\pgfqpoint{2.915231in}{1.058641in}}%
\pgfpathlineto{\pgfqpoint{2.915289in}{1.063729in}}%
\pgfpathlineto{\pgfqpoint{2.911583in}{1.069164in}}%
\pgfpathlineto{\pgfqpoint{2.913042in}{1.078039in}}%
\pgfpathlineto{\pgfqpoint{2.908402in}{1.084077in}}%
\pgfpathlineto{\pgfqpoint{2.903485in}{1.086564in}}%
\pgfpathlineto{\pgfqpoint{2.941115in}{1.093711in}}%
\pgfpathlineto{\pgfqpoint{2.953469in}{1.096061in}}%
\pgfpathlineto{\pgfqpoint{2.953638in}{1.092737in}}%
\pgfpathlineto{\pgfqpoint{2.964354in}{1.078920in}}%
\pgfpathlineto{\pgfqpoint{2.973010in}{1.074654in}}%
\pgfpathlineto{\pgfqpoint{2.990246in}{1.083365in}}%
\pgfpathlineto{\pgfqpoint{3.001215in}{1.083802in}}%
\pgfpathlineto{\pgfqpoint{3.001020in}{1.072920in}}%
\pgfpathlineto{\pgfqpoint{3.003062in}{1.064569in}}%
\pgfpathlineto{\pgfqpoint{3.013882in}{1.056385in}}%
\pgfpathlineto{\pgfqpoint{2.995435in}{1.052555in}}%
\pgfpathlineto{\pgfqpoint{2.989523in}{1.050329in}}%
\pgfpathlineto{\pgfqpoint{2.997200in}{1.040699in}}%
\pgfpathlineto{\pgfqpoint{2.997384in}{1.026746in}}%
\pgfpathlineto{\pgfqpoint{2.991145in}{1.020679in}}%
\pgfpathclose%
\pgfusepath{fill}%
\end{pgfscope}%
\begin{pgfscope}%
\pgfpathrectangle{\pgfqpoint{0.100000in}{0.100000in}}{\pgfqpoint{3.420221in}{2.189500in}}%
\pgfusepath{clip}%
\pgfsetbuttcap%
\pgfsetmiterjoin%
\definecolor{currentfill}{rgb}{0.000000,0.247059,0.876471}%
\pgfsetfillcolor{currentfill}%
\pgfsetlinewidth{0.000000pt}%
\definecolor{currentstroke}{rgb}{0.000000,0.000000,0.000000}%
\pgfsetstrokecolor{currentstroke}%
\pgfsetstrokeopacity{0.000000}%
\pgfsetdash{}{0pt}%
\pgfpathmoveto{\pgfqpoint{1.866876in}{1.377916in}}%
\pgfpathlineto{\pgfqpoint{1.866155in}{1.351828in}}%
\pgfpathlineto{\pgfqpoint{1.814244in}{1.353473in}}%
\pgfpathlineto{\pgfqpoint{1.815212in}{1.379526in}}%
\pgfpathlineto{\pgfqpoint{1.840861in}{1.378665in}}%
\pgfpathclose%
\pgfusepath{fill}%
\end{pgfscope}%
\begin{pgfscope}%
\pgfpathrectangle{\pgfqpoint{0.100000in}{0.100000in}}{\pgfqpoint{3.420221in}{2.189500in}}%
\pgfusepath{clip}%
\pgfsetbuttcap%
\pgfsetmiterjoin%
\definecolor{currentfill}{rgb}{0.000000,0.576471,0.711765}%
\pgfsetfillcolor{currentfill}%
\pgfsetlinewidth{0.000000pt}%
\definecolor{currentstroke}{rgb}{0.000000,0.000000,0.000000}%
\pgfsetstrokecolor{currentstroke}%
\pgfsetstrokeopacity{0.000000}%
\pgfsetdash{}{0pt}%
\pgfpathmoveto{\pgfqpoint{2.384289in}{1.068094in}}%
\pgfpathlineto{\pgfqpoint{2.373557in}{1.072996in}}%
\pgfpathlineto{\pgfqpoint{2.367355in}{1.077383in}}%
\pgfpathlineto{\pgfqpoint{2.355665in}{1.076848in}}%
\pgfpathlineto{\pgfqpoint{2.335796in}{1.076210in}}%
\pgfpathlineto{\pgfqpoint{2.338842in}{1.079921in}}%
\pgfpathlineto{\pgfqpoint{2.342291in}{1.091378in}}%
\pgfpathlineto{\pgfqpoint{2.338433in}{1.097884in}}%
\pgfpathlineto{\pgfqpoint{2.341047in}{1.107577in}}%
\pgfpathlineto{\pgfqpoint{2.349528in}{1.103863in}}%
\pgfpathlineto{\pgfqpoint{2.353162in}{1.111382in}}%
\pgfpathlineto{\pgfqpoint{2.350500in}{1.114945in}}%
\pgfpathlineto{\pgfqpoint{2.354720in}{1.120421in}}%
\pgfpathlineto{\pgfqpoint{2.373379in}{1.120747in}}%
\pgfpathlineto{\pgfqpoint{2.372510in}{1.133719in}}%
\pgfpathlineto{\pgfqpoint{2.392163in}{1.134769in}}%
\pgfpathlineto{\pgfqpoint{2.393118in}{1.120413in}}%
\pgfpathlineto{\pgfqpoint{2.394245in}{1.101756in}}%
\pgfpathlineto{\pgfqpoint{2.392607in}{1.101666in}}%
\pgfpathlineto{\pgfqpoint{2.393715in}{1.072719in}}%
\pgfpathlineto{\pgfqpoint{2.386255in}{1.072336in}}%
\pgfpathclose%
\pgfusepath{fill}%
\end{pgfscope}%
\begin{pgfscope}%
\pgfpathrectangle{\pgfqpoint{0.100000in}{0.100000in}}{\pgfqpoint{3.420221in}{2.189500in}}%
\pgfusepath{clip}%
\pgfsetbuttcap%
\pgfsetmiterjoin%
\definecolor{currentfill}{rgb}{0.000000,0.403922,0.798039}%
\pgfsetfillcolor{currentfill}%
\pgfsetlinewidth{0.000000pt}%
\definecolor{currentstroke}{rgb}{0.000000,0.000000,0.000000}%
\pgfsetstrokecolor{currentstroke}%
\pgfsetstrokeopacity{0.000000}%
\pgfsetdash{}{0pt}%
\pgfpathmoveto{\pgfqpoint{1.333239in}{1.468741in}}%
\pgfpathlineto{\pgfqpoint{1.363314in}{1.464617in}}%
\pgfpathlineto{\pgfqpoint{1.366907in}{1.494088in}}%
\pgfpathlineto{\pgfqpoint{1.381110in}{1.492321in}}%
\pgfpathlineto{\pgfqpoint{1.387052in}{1.543851in}}%
\pgfpathlineto{\pgfqpoint{1.390297in}{1.569654in}}%
\pgfpathlineto{\pgfqpoint{1.418726in}{1.566054in}}%
\pgfpathlineto{\pgfqpoint{1.415036in}{1.562470in}}%
\pgfpathlineto{\pgfqpoint{1.414442in}{1.557026in}}%
\pgfpathlineto{\pgfqpoint{1.427457in}{1.554435in}}%
\pgfpathlineto{\pgfqpoint{1.431023in}{1.564674in}}%
\pgfpathlineto{\pgfqpoint{1.433707in}{1.564391in}}%
\pgfpathlineto{\pgfqpoint{1.430608in}{1.531116in}}%
\pgfpathlineto{\pgfqpoint{1.427460in}{1.506606in}}%
\pgfpathlineto{\pgfqpoint{1.421950in}{1.457544in}}%
\pgfpathlineto{\pgfqpoint{1.370632in}{1.463589in}}%
\pgfpathlineto{\pgfqpoint{1.370341in}{1.458821in}}%
\pgfpathlineto{\pgfqpoint{1.376450in}{1.449245in}}%
\pgfpathlineto{\pgfqpoint{1.384913in}{1.423241in}}%
\pgfpathlineto{\pgfqpoint{1.377138in}{1.413905in}}%
\pgfpathlineto{\pgfqpoint{1.374628in}{1.415959in}}%
\pgfpathlineto{\pgfqpoint{1.366839in}{1.413793in}}%
\pgfpathlineto{\pgfqpoint{1.357591in}{1.414807in}}%
\pgfpathlineto{\pgfqpoint{1.342606in}{1.420639in}}%
\pgfpathlineto{\pgfqpoint{1.339434in}{1.425804in}}%
\pgfpathlineto{\pgfqpoint{1.337871in}{1.436041in}}%
\pgfpathlineto{\pgfqpoint{1.343665in}{1.451458in}}%
\pgfpathlineto{\pgfqpoint{1.343167in}{1.455843in}}%
\pgfpathlineto{\pgfqpoint{1.332812in}{1.462916in}}%
\pgfpathclose%
\pgfusepath{fill}%
\end{pgfscope}%
\begin{pgfscope}%
\pgfpathrectangle{\pgfqpoint{0.100000in}{0.100000in}}{\pgfqpoint{3.420221in}{2.189500in}}%
\pgfusepath{clip}%
\pgfsetbuttcap%
\pgfsetmiterjoin%
\definecolor{currentfill}{rgb}{0.000000,0.568627,0.715686}%
\pgfsetfillcolor{currentfill}%
\pgfsetlinewidth{0.000000pt}%
\definecolor{currentstroke}{rgb}{0.000000,0.000000,0.000000}%
\pgfsetstrokecolor{currentstroke}%
\pgfsetstrokeopacity{0.000000}%
\pgfsetdash{}{0pt}%
\pgfpathmoveto{\pgfqpoint{2.673579in}{1.672594in}}%
\pgfpathlineto{\pgfqpoint{2.646202in}{1.667792in}}%
\pgfpathlineto{\pgfqpoint{2.650369in}{1.641957in}}%
\pgfpathlineto{\pgfqpoint{2.636829in}{1.639811in}}%
\pgfpathlineto{\pgfqpoint{2.637809in}{1.633333in}}%
\pgfpathlineto{\pgfqpoint{2.632017in}{1.631677in}}%
\pgfpathlineto{\pgfqpoint{2.606439in}{1.628258in}}%
\pgfpathlineto{\pgfqpoint{2.607408in}{1.621761in}}%
\pgfpathlineto{\pgfqpoint{2.583571in}{1.618567in}}%
\pgfpathlineto{\pgfqpoint{2.580495in}{1.643758in}}%
\pgfpathlineto{\pgfqpoint{2.567621in}{1.642252in}}%
\pgfpathlineto{\pgfqpoint{2.561638in}{1.694127in}}%
\pgfpathlineto{\pgfqpoint{2.600403in}{1.698834in}}%
\pgfpathlineto{\pgfqpoint{2.617490in}{1.701080in}}%
\pgfpathlineto{\pgfqpoint{2.617855in}{1.696772in}}%
\pgfpathlineto{\pgfqpoint{2.608189in}{1.686395in}}%
\pgfpathlineto{\pgfqpoint{2.604928in}{1.686422in}}%
\pgfpathlineto{\pgfqpoint{2.601124in}{1.678868in}}%
\pgfpathlineto{\pgfqpoint{2.600622in}{1.667250in}}%
\pgfpathlineto{\pgfqpoint{2.603413in}{1.662259in}}%
\pgfpathlineto{\pgfqpoint{2.617508in}{1.657858in}}%
\pgfpathlineto{\pgfqpoint{2.635540in}{1.689083in}}%
\pgfpathlineto{\pgfqpoint{2.646802in}{1.693646in}}%
\pgfpathlineto{\pgfqpoint{2.652457in}{1.698522in}}%
\pgfpathlineto{\pgfqpoint{2.662982in}{1.694392in}}%
\pgfpathlineto{\pgfqpoint{2.670741in}{1.682901in}}%
\pgfpathclose%
\pgfusepath{fill}%
\end{pgfscope}%
\begin{pgfscope}%
\pgfpathrectangle{\pgfqpoint{0.100000in}{0.100000in}}{\pgfqpoint{3.420221in}{2.189500in}}%
\pgfusepath{clip}%
\pgfsetbuttcap%
\pgfsetmiterjoin%
\definecolor{currentfill}{rgb}{0.000000,0.603922,0.698039}%
\pgfsetfillcolor{currentfill}%
\pgfsetlinewidth{0.000000pt}%
\definecolor{currentstroke}{rgb}{0.000000,0.000000,0.000000}%
\pgfsetstrokecolor{currentstroke}%
\pgfsetstrokeopacity{0.000000}%
\pgfsetdash{}{0pt}%
\pgfpathmoveto{\pgfqpoint{2.896731in}{1.320111in}}%
\pgfpathlineto{\pgfqpoint{2.896398in}{1.307889in}}%
\pgfpathlineto{\pgfqpoint{2.891005in}{1.297370in}}%
\pgfpathlineto{\pgfqpoint{2.892673in}{1.294867in}}%
\pgfpathlineto{\pgfqpoint{2.886103in}{1.287105in}}%
\pgfpathlineto{\pgfqpoint{2.885302in}{1.278077in}}%
\pgfpathlineto{\pgfqpoint{2.873466in}{1.274019in}}%
\pgfpathlineto{\pgfqpoint{2.867768in}{1.273882in}}%
\pgfpathlineto{\pgfqpoint{2.861211in}{1.277953in}}%
\pgfpathlineto{\pgfqpoint{2.860060in}{1.286231in}}%
\pgfpathlineto{\pgfqpoint{2.860470in}{1.294643in}}%
\pgfpathlineto{\pgfqpoint{2.864743in}{1.299340in}}%
\pgfpathlineto{\pgfqpoint{2.866647in}{1.310049in}}%
\pgfpathlineto{\pgfqpoint{2.859039in}{1.321669in}}%
\pgfpathlineto{\pgfqpoint{2.860941in}{1.325692in}}%
\pgfpathlineto{\pgfqpoint{2.867299in}{1.326897in}}%
\pgfpathlineto{\pgfqpoint{2.866416in}{1.338229in}}%
\pgfpathlineto{\pgfqpoint{2.868718in}{1.344562in}}%
\pgfpathlineto{\pgfqpoint{2.857024in}{1.353066in}}%
\pgfpathlineto{\pgfqpoint{2.858934in}{1.363211in}}%
\pgfpathlineto{\pgfqpoint{2.867050in}{1.364980in}}%
\pgfpathlineto{\pgfqpoint{2.873162in}{1.370169in}}%
\pgfpathlineto{\pgfqpoint{2.879492in}{1.365745in}}%
\pgfpathlineto{\pgfqpoint{2.885984in}{1.369951in}}%
\pgfpathlineto{\pgfqpoint{2.898397in}{1.366262in}}%
\pgfpathlineto{\pgfqpoint{2.906313in}{1.369093in}}%
\pgfpathlineto{\pgfqpoint{2.909371in}{1.367679in}}%
\pgfpathlineto{\pgfqpoint{2.907645in}{1.360172in}}%
\pgfpathlineto{\pgfqpoint{2.910776in}{1.359156in}}%
\pgfpathlineto{\pgfqpoint{2.909299in}{1.350100in}}%
\pgfpathlineto{\pgfqpoint{2.901161in}{1.342328in}}%
\pgfpathlineto{\pgfqpoint{2.902420in}{1.336455in}}%
\pgfpathclose%
\pgfusepath{fill}%
\end{pgfscope}%
\begin{pgfscope}%
\pgfpathrectangle{\pgfqpoint{0.100000in}{0.100000in}}{\pgfqpoint{3.420221in}{2.189500in}}%
\pgfusepath{clip}%
\pgfsetbuttcap%
\pgfsetmiterjoin%
\definecolor{currentfill}{rgb}{0.000000,0.458824,0.770588}%
\pgfsetfillcolor{currentfill}%
\pgfsetlinewidth{0.000000pt}%
\definecolor{currentstroke}{rgb}{0.000000,0.000000,0.000000}%
\pgfsetstrokecolor{currentstroke}%
\pgfsetstrokeopacity{0.000000}%
\pgfsetdash{}{0pt}%
\pgfpathmoveto{\pgfqpoint{3.054463in}{1.642220in}}%
\pgfpathlineto{\pgfqpoint{3.033485in}{1.637159in}}%
\pgfpathlineto{\pgfqpoint{3.029503in}{1.652890in}}%
\pgfpathlineto{\pgfqpoint{3.007819in}{1.648395in}}%
\pgfpathlineto{\pgfqpoint{3.003101in}{1.654285in}}%
\pgfpathlineto{\pgfqpoint{2.998823in}{1.673135in}}%
\pgfpathlineto{\pgfqpoint{2.999025in}{1.681601in}}%
\pgfpathlineto{\pgfqpoint{2.994024in}{1.700763in}}%
\pgfpathlineto{\pgfqpoint{2.997851in}{1.706915in}}%
\pgfpathlineto{\pgfqpoint{3.007496in}{1.717117in}}%
\pgfpathlineto{\pgfqpoint{3.017427in}{1.720008in}}%
\pgfpathlineto{\pgfqpoint{3.016741in}{1.731302in}}%
\pgfpathlineto{\pgfqpoint{3.025632in}{1.735237in}}%
\pgfpathlineto{\pgfqpoint{3.039232in}{1.736744in}}%
\pgfpathlineto{\pgfqpoint{3.043659in}{1.720900in}}%
\pgfpathlineto{\pgfqpoint{3.056336in}{1.719750in}}%
\pgfpathlineto{\pgfqpoint{3.075711in}{1.739052in}}%
\pgfpathlineto{\pgfqpoint{3.064684in}{1.773704in}}%
\pgfpathlineto{\pgfqpoint{3.071108in}{1.771545in}}%
\pgfpathlineto{\pgfqpoint{3.081784in}{1.775456in}}%
\pgfpathlineto{\pgfqpoint{3.095604in}{1.733614in}}%
\pgfpathlineto{\pgfqpoint{3.093130in}{1.721745in}}%
\pgfpathlineto{\pgfqpoint{3.102289in}{1.719681in}}%
\pgfpathlineto{\pgfqpoint{3.105060in}{1.711795in}}%
\pgfpathlineto{\pgfqpoint{3.102390in}{1.704087in}}%
\pgfpathlineto{\pgfqpoint{3.105603in}{1.698264in}}%
\pgfpathlineto{\pgfqpoint{3.106600in}{1.688090in}}%
\pgfpathlineto{\pgfqpoint{3.099821in}{1.689257in}}%
\pgfpathlineto{\pgfqpoint{3.099530in}{1.683548in}}%
\pgfpathlineto{\pgfqpoint{3.087730in}{1.687327in}}%
\pgfpathlineto{\pgfqpoint{3.082685in}{1.682223in}}%
\pgfpathlineto{\pgfqpoint{3.079934in}{1.672972in}}%
\pgfpathlineto{\pgfqpoint{3.048237in}{1.664540in}}%
\pgfpathclose%
\pgfusepath{fill}%
\end{pgfscope}%
\begin{pgfscope}%
\pgfpathrectangle{\pgfqpoint{0.100000in}{0.100000in}}{\pgfqpoint{3.420221in}{2.189500in}}%
\pgfusepath{clip}%
\pgfsetbuttcap%
\pgfsetmiterjoin%
\definecolor{currentfill}{rgb}{0.000000,0.345098,0.827451}%
\pgfsetfillcolor{currentfill}%
\pgfsetlinewidth{0.000000pt}%
\definecolor{currentstroke}{rgb}{0.000000,0.000000,0.000000}%
\pgfsetstrokecolor{currentstroke}%
\pgfsetstrokeopacity{0.000000}%
\pgfsetdash{}{0pt}%
\pgfpathmoveto{\pgfqpoint{2.009974in}{1.515561in}}%
\pgfpathlineto{\pgfqpoint{2.048535in}{1.515790in}}%
\pgfpathlineto{\pgfqpoint{2.048799in}{1.489785in}}%
\pgfpathlineto{\pgfqpoint{2.016364in}{1.489528in}}%
\pgfpathlineto{\pgfqpoint{1.970982in}{1.489681in}}%
\pgfpathlineto{\pgfqpoint{1.964532in}{1.489705in}}%
\pgfpathlineto{\pgfqpoint{1.964818in}{1.515815in}}%
\pgfpathclose%
\pgfusepath{fill}%
\end{pgfscope}%
\begin{pgfscope}%
\pgfpathrectangle{\pgfqpoint{0.100000in}{0.100000in}}{\pgfqpoint{3.420221in}{2.189500in}}%
\pgfusepath{clip}%
\pgfsetbuttcap%
\pgfsetmiterjoin%
\definecolor{currentfill}{rgb}{0.000000,0.541176,0.729412}%
\pgfsetfillcolor{currentfill}%
\pgfsetlinewidth{0.000000pt}%
\definecolor{currentstroke}{rgb}{0.000000,0.000000,0.000000}%
\pgfsetstrokecolor{currentstroke}%
\pgfsetstrokeopacity{0.000000}%
\pgfsetdash{}{0pt}%
\pgfpathmoveto{\pgfqpoint{2.514329in}{0.822121in}}%
\pgfpathlineto{\pgfqpoint{2.515789in}{0.813145in}}%
\pgfpathlineto{\pgfqpoint{2.522971in}{0.798418in}}%
\pgfpathlineto{\pgfqpoint{2.517810in}{0.789186in}}%
\pgfpathlineto{\pgfqpoint{2.519094in}{0.775887in}}%
\pgfpathlineto{\pgfqpoint{2.501822in}{0.774336in}}%
\pgfpathlineto{\pgfqpoint{2.491110in}{0.784824in}}%
\pgfpathlineto{\pgfqpoint{2.481717in}{0.789097in}}%
\pgfpathlineto{\pgfqpoint{2.478144in}{0.792008in}}%
\pgfpathlineto{\pgfqpoint{2.477049in}{0.805141in}}%
\pgfpathlineto{\pgfqpoint{2.464106in}{0.804005in}}%
\pgfpathlineto{\pgfqpoint{2.458532in}{0.806861in}}%
\pgfpathlineto{\pgfqpoint{2.459757in}{0.813063in}}%
\pgfpathlineto{\pgfqpoint{2.456424in}{0.824943in}}%
\pgfpathlineto{\pgfqpoint{2.456787in}{0.834914in}}%
\pgfpathlineto{\pgfqpoint{2.461210in}{0.839185in}}%
\pgfpathlineto{\pgfqpoint{2.461989in}{0.844422in}}%
\pgfpathlineto{\pgfqpoint{2.454360in}{0.843778in}}%
\pgfpathlineto{\pgfqpoint{2.453219in}{0.853992in}}%
\pgfpathlineto{\pgfqpoint{2.450820in}{0.881836in}}%
\pgfpathlineto{\pgfqpoint{2.461724in}{0.882461in}}%
\pgfpathlineto{\pgfqpoint{2.463370in}{0.889258in}}%
\pgfpathlineto{\pgfqpoint{2.479698in}{0.889995in}}%
\pgfpathlineto{\pgfqpoint{2.486773in}{0.883890in}}%
\pgfpathlineto{\pgfqpoint{2.486144in}{0.878371in}}%
\pgfpathlineto{\pgfqpoint{2.493471in}{0.869972in}}%
\pgfpathlineto{\pgfqpoint{2.503539in}{0.865207in}}%
\pgfpathlineto{\pgfqpoint{2.504151in}{0.859163in}}%
\pgfpathlineto{\pgfqpoint{2.508133in}{0.854419in}}%
\pgfpathlineto{\pgfqpoint{2.514268in}{0.851300in}}%
\pgfpathlineto{\pgfqpoint{2.515752in}{0.835271in}}%
\pgfpathlineto{\pgfqpoint{2.506727in}{0.834528in}}%
\pgfpathlineto{\pgfqpoint{2.507970in}{0.821491in}}%
\pgfpathclose%
\pgfusepath{fill}%
\end{pgfscope}%
\begin{pgfscope}%
\pgfpathrectangle{\pgfqpoint{0.100000in}{0.100000in}}{\pgfqpoint{3.420221in}{2.189500in}}%
\pgfusepath{clip}%
\pgfsetbuttcap%
\pgfsetmiterjoin%
\definecolor{currentfill}{rgb}{0.000000,0.384314,0.807843}%
\pgfsetfillcolor{currentfill}%
\pgfsetlinewidth{0.000000pt}%
\definecolor{currentstroke}{rgb}{0.000000,0.000000,0.000000}%
\pgfsetstrokecolor{currentstroke}%
\pgfsetstrokeopacity{0.000000}%
\pgfsetdash{}{0pt}%
\pgfpathmoveto{\pgfqpoint{2.113161in}{1.516746in}}%
\pgfpathlineto{\pgfqpoint{2.100376in}{1.516456in}}%
\pgfpathlineto{\pgfqpoint{2.087519in}{1.516249in}}%
\pgfpathlineto{\pgfqpoint{2.087208in}{1.535874in}}%
\pgfpathlineto{\pgfqpoint{2.085190in}{1.535851in}}%
\pgfpathlineto{\pgfqpoint{2.084673in}{1.568690in}}%
\pgfpathlineto{\pgfqpoint{2.110721in}{1.569135in}}%
\pgfpathlineto{\pgfqpoint{2.111313in}{1.536342in}}%
\pgfpathlineto{\pgfqpoint{2.112659in}{1.536363in}}%
\pgfpathclose%
\pgfusepath{fill}%
\end{pgfscope}%
\begin{pgfscope}%
\pgfpathrectangle{\pgfqpoint{0.100000in}{0.100000in}}{\pgfqpoint{3.420221in}{2.189500in}}%
\pgfusepath{clip}%
\pgfsetbuttcap%
\pgfsetmiterjoin%
\definecolor{currentfill}{rgb}{0.000000,0.788235,0.605882}%
\pgfsetfillcolor{currentfill}%
\pgfsetlinewidth{0.000000pt}%
\definecolor{currentstroke}{rgb}{0.000000,0.000000,0.000000}%
\pgfsetstrokecolor{currentstroke}%
\pgfsetstrokeopacity{0.000000}%
\pgfsetdash{}{0pt}%
\pgfpathmoveto{\pgfqpoint{2.583571in}{1.618567in}}%
\pgfpathlineto{\pgfqpoint{2.583661in}{1.617791in}}%
\pgfpathlineto{\pgfqpoint{2.558113in}{1.614925in}}%
\pgfpathlineto{\pgfqpoint{2.554700in}{1.640793in}}%
\pgfpathlineto{\pgfqpoint{2.541751in}{1.639369in}}%
\pgfpathlineto{\pgfqpoint{2.535922in}{1.691526in}}%
\pgfpathlineto{\pgfqpoint{2.548562in}{1.692702in}}%
\pgfpathlineto{\pgfqpoint{2.561638in}{1.694127in}}%
\pgfpathlineto{\pgfqpoint{2.567621in}{1.642252in}}%
\pgfpathlineto{\pgfqpoint{2.580495in}{1.643758in}}%
\pgfpathclose%
\pgfusepath{fill}%
\end{pgfscope}%
\begin{pgfscope}%
\pgfpathrectangle{\pgfqpoint{0.100000in}{0.100000in}}{\pgfqpoint{3.420221in}{2.189500in}}%
\pgfusepath{clip}%
\pgfsetbuttcap%
\pgfsetmiterjoin%
\definecolor{currentfill}{rgb}{0.000000,0.466667,0.766667}%
\pgfsetfillcolor{currentfill}%
\pgfsetlinewidth{0.000000pt}%
\definecolor{currentstroke}{rgb}{0.000000,0.000000,0.000000}%
\pgfsetstrokecolor{currentstroke}%
\pgfsetstrokeopacity{0.000000}%
\pgfsetdash{}{0pt}%
\pgfpathmoveto{\pgfqpoint{2.737398in}{1.025703in}}%
\pgfpathlineto{\pgfqpoint{2.734645in}{1.027887in}}%
\pgfpathlineto{\pgfqpoint{2.740082in}{1.041314in}}%
\pgfpathlineto{\pgfqpoint{2.739610in}{1.046866in}}%
\pgfpathlineto{\pgfqpoint{2.722510in}{1.061157in}}%
\pgfpathlineto{\pgfqpoint{2.720573in}{1.072221in}}%
\pgfpathlineto{\pgfqpoint{2.715961in}{1.073488in}}%
\pgfpathlineto{\pgfqpoint{2.721030in}{1.079378in}}%
\pgfpathlineto{\pgfqpoint{2.732227in}{1.082731in}}%
\pgfpathlineto{\pgfqpoint{2.734971in}{1.094864in}}%
\pgfpathlineto{\pgfqpoint{2.749769in}{1.105999in}}%
\pgfpathlineto{\pgfqpoint{2.752055in}{1.098869in}}%
\pgfpathlineto{\pgfqpoint{2.758296in}{1.100507in}}%
\pgfpathlineto{\pgfqpoint{2.759202in}{1.097016in}}%
\pgfpathlineto{\pgfqpoint{2.769449in}{1.089454in}}%
\pgfpathlineto{\pgfqpoint{2.777985in}{1.071585in}}%
\pgfpathlineto{\pgfqpoint{2.783269in}{1.069921in}}%
\pgfpathlineto{\pgfqpoint{2.779477in}{1.068751in}}%
\pgfpathlineto{\pgfqpoint{2.777423in}{1.062602in}}%
\pgfpathlineto{\pgfqpoint{2.779092in}{1.059106in}}%
\pgfpathlineto{\pgfqpoint{2.775085in}{1.050472in}}%
\pgfpathlineto{\pgfqpoint{2.775613in}{1.043290in}}%
\pgfpathlineto{\pgfqpoint{2.756928in}{1.035146in}}%
\pgfpathclose%
\pgfusepath{fill}%
\end{pgfscope}%
\begin{pgfscope}%
\pgfpathrectangle{\pgfqpoint{0.100000in}{0.100000in}}{\pgfqpoint{3.420221in}{2.189500in}}%
\pgfusepath{clip}%
\pgfsetbuttcap%
\pgfsetmiterjoin%
\definecolor{currentfill}{rgb}{0.000000,0.584314,0.707843}%
\pgfsetfillcolor{currentfill}%
\pgfsetlinewidth{0.000000pt}%
\definecolor{currentstroke}{rgb}{0.000000,0.000000,0.000000}%
\pgfsetstrokecolor{currentstroke}%
\pgfsetstrokeopacity{0.000000}%
\pgfsetdash{}{0pt}%
\pgfpathmoveto{\pgfqpoint{2.159074in}{1.050540in}}%
\pgfpathlineto{\pgfqpoint{2.159328in}{1.036490in}}%
\pgfpathlineto{\pgfqpoint{2.146357in}{1.036304in}}%
\pgfpathlineto{\pgfqpoint{2.146414in}{1.030866in}}%
\pgfpathlineto{\pgfqpoint{2.135532in}{1.030896in}}%
\pgfpathlineto{\pgfqpoint{2.092282in}{1.030925in}}%
\pgfpathlineto{\pgfqpoint{2.092270in}{1.033108in}}%
\pgfpathlineto{\pgfqpoint{2.091855in}{1.041962in}}%
\pgfpathlineto{\pgfqpoint{2.096178in}{1.048480in}}%
\pgfpathlineto{\pgfqpoint{2.096342in}{1.056981in}}%
\pgfpathlineto{\pgfqpoint{2.089902in}{1.060281in}}%
\pgfpathlineto{\pgfqpoint{2.085631in}{1.065848in}}%
\pgfpathlineto{\pgfqpoint{2.083243in}{1.073452in}}%
\pgfpathlineto{\pgfqpoint{2.070461in}{1.073606in}}%
\pgfpathlineto{\pgfqpoint{2.070421in}{1.087952in}}%
\pgfpathlineto{\pgfqpoint{2.131453in}{1.089002in}}%
\pgfpathlineto{\pgfqpoint{2.129611in}{1.088422in}}%
\pgfpathlineto{\pgfqpoint{2.129921in}{1.060163in}}%
\pgfpathlineto{\pgfqpoint{2.133205in}{1.056938in}}%
\pgfpathlineto{\pgfqpoint{2.159059in}{1.057061in}}%
\pgfpathclose%
\pgfusepath{fill}%
\end{pgfscope}%
\begin{pgfscope}%
\pgfpathrectangle{\pgfqpoint{0.100000in}{0.100000in}}{\pgfqpoint{3.420221in}{2.189500in}}%
\pgfusepath{clip}%
\pgfsetbuttcap%
\pgfsetmiterjoin%
\definecolor{currentfill}{rgb}{0.000000,0.337255,0.831373}%
\pgfsetfillcolor{currentfill}%
\pgfsetlinewidth{0.000000pt}%
\definecolor{currentstroke}{rgb}{0.000000,0.000000,0.000000}%
\pgfsetstrokecolor{currentstroke}%
\pgfsetstrokeopacity{0.000000}%
\pgfsetdash{}{0pt}%
\pgfpathmoveto{\pgfqpoint{2.399978in}{1.552493in}}%
\pgfpathlineto{\pgfqpoint{2.350052in}{1.548913in}}%
\pgfpathlineto{\pgfqpoint{2.313651in}{1.547024in}}%
\pgfpathlineto{\pgfqpoint{2.311883in}{1.572845in}}%
\pgfpathlineto{\pgfqpoint{2.331404in}{1.574268in}}%
\pgfpathlineto{\pgfqpoint{2.344408in}{1.574809in}}%
\pgfpathlineto{\pgfqpoint{2.396586in}{1.578630in}}%
\pgfpathlineto{\pgfqpoint{2.400722in}{1.574206in}}%
\pgfpathlineto{\pgfqpoint{2.398341in}{1.561728in}}%
\pgfpathclose%
\pgfusepath{fill}%
\end{pgfscope}%
\begin{pgfscope}%
\pgfpathrectangle{\pgfqpoint{0.100000in}{0.100000in}}{\pgfqpoint{3.420221in}{2.189500in}}%
\pgfusepath{clip}%
\pgfsetbuttcap%
\pgfsetmiterjoin%
\definecolor{currentfill}{rgb}{0.000000,0.400000,0.800000}%
\pgfsetfillcolor{currentfill}%
\pgfsetlinewidth{0.000000pt}%
\definecolor{currentstroke}{rgb}{0.000000,0.000000,0.000000}%
\pgfsetstrokecolor{currentstroke}%
\pgfsetstrokeopacity{0.000000}%
\pgfsetdash{}{0pt}%
\pgfpathmoveto{\pgfqpoint{1.686831in}{0.896323in}}%
\pgfpathlineto{\pgfqpoint{1.657499in}{0.898006in}}%
\pgfpathlineto{\pgfqpoint{1.659767in}{0.933914in}}%
\pgfpathlineto{\pgfqpoint{1.665935in}{0.933499in}}%
\pgfpathlineto{\pgfqpoint{1.667994in}{0.966050in}}%
\pgfpathlineto{\pgfqpoint{1.693596in}{0.964501in}}%
\pgfpathlineto{\pgfqpoint{1.693006in}{0.950545in}}%
\pgfpathlineto{\pgfqpoint{1.691610in}{0.925312in}}%
\pgfpathlineto{\pgfqpoint{1.688565in}{0.925888in}}%
\pgfpathclose%
\pgfusepath{fill}%
\end{pgfscope}%
\begin{pgfscope}%
\pgfpathrectangle{\pgfqpoint{0.100000in}{0.100000in}}{\pgfqpoint{3.420221in}{2.189500in}}%
\pgfusepath{clip}%
\pgfsetbuttcap%
\pgfsetmiterjoin%
\definecolor{currentfill}{rgb}{0.000000,0.964706,0.517647}%
\pgfsetfillcolor{currentfill}%
\pgfsetlinewidth{0.000000pt}%
\definecolor{currentstroke}{rgb}{0.000000,0.000000,0.000000}%
\pgfsetstrokecolor{currentstroke}%
\pgfsetstrokeopacity{0.000000}%
\pgfsetdash{}{0pt}%
\pgfpathmoveto{\pgfqpoint{2.221561in}{0.744501in}}%
\pgfpathlineto{\pgfqpoint{2.221631in}{0.738133in}}%
\pgfpathlineto{\pgfqpoint{2.217924in}{0.732702in}}%
\pgfpathlineto{\pgfqpoint{2.214481in}{0.733261in}}%
\pgfpathlineto{\pgfqpoint{2.209278in}{0.717683in}}%
\pgfpathlineto{\pgfqpoint{2.204170in}{0.712402in}}%
\pgfpathlineto{\pgfqpoint{2.206618in}{0.697375in}}%
\pgfpathlineto{\pgfqpoint{2.195679in}{0.700885in}}%
\pgfpathlineto{\pgfqpoint{2.185955in}{0.707221in}}%
\pgfpathlineto{\pgfqpoint{2.185084in}{0.711748in}}%
\pgfpathlineto{\pgfqpoint{2.177624in}{0.716120in}}%
\pgfpathlineto{\pgfqpoint{2.172359in}{0.723291in}}%
\pgfpathlineto{\pgfqpoint{2.171335in}{0.736334in}}%
\pgfpathlineto{\pgfqpoint{2.174251in}{0.746294in}}%
\pgfpathlineto{\pgfqpoint{2.200391in}{0.747140in}}%
\pgfpathlineto{\pgfqpoint{2.203626in}{0.749572in}}%
\pgfpathlineto{\pgfqpoint{2.208482in}{0.743229in}}%
\pgfpathlineto{\pgfqpoint{2.219956in}{0.751027in}}%
\pgfpathclose%
\pgfusepath{fill}%
\end{pgfscope}%
\begin{pgfscope}%
\pgfpathrectangle{\pgfqpoint{0.100000in}{0.100000in}}{\pgfqpoint{3.420221in}{2.189500in}}%
\pgfusepath{clip}%
\pgfsetbuttcap%
\pgfsetmiterjoin%
\definecolor{currentfill}{rgb}{0.000000,0.513725,0.743137}%
\pgfsetfillcolor{currentfill}%
\pgfsetlinewidth{0.000000pt}%
\definecolor{currentstroke}{rgb}{0.000000,0.000000,0.000000}%
\pgfsetstrokecolor{currentstroke}%
\pgfsetstrokeopacity{0.000000}%
\pgfsetdash{}{0pt}%
\pgfpathmoveto{\pgfqpoint{2.486773in}{0.883890in}}%
\pgfpathlineto{\pgfqpoint{2.479698in}{0.889995in}}%
\pgfpathlineto{\pgfqpoint{2.463370in}{0.889258in}}%
\pgfpathlineto{\pgfqpoint{2.461724in}{0.882461in}}%
\pgfpathlineto{\pgfqpoint{2.450820in}{0.881836in}}%
\pgfpathlineto{\pgfqpoint{2.423624in}{0.880447in}}%
\pgfpathlineto{\pgfqpoint{2.424054in}{0.896398in}}%
\pgfpathlineto{\pgfqpoint{2.425499in}{0.939950in}}%
\pgfpathlineto{\pgfqpoint{2.465452in}{0.941858in}}%
\pgfpathlineto{\pgfqpoint{2.491461in}{0.943662in}}%
\pgfpathlineto{\pgfqpoint{2.493708in}{0.918031in}}%
\pgfpathlineto{\pgfqpoint{2.495766in}{0.914405in}}%
\pgfpathlineto{\pgfqpoint{2.503456in}{0.911430in}}%
\pgfpathlineto{\pgfqpoint{2.500291in}{0.904684in}}%
\pgfpathlineto{\pgfqpoint{2.499838in}{0.898039in}}%
\pgfpathlineto{\pgfqpoint{2.496912in}{0.893445in}}%
\pgfpathlineto{\pgfqpoint{2.491591in}{0.891922in}}%
\pgfpathclose%
\pgfusepath{fill}%
\end{pgfscope}%
\begin{pgfscope}%
\pgfpathrectangle{\pgfqpoint{0.100000in}{0.100000in}}{\pgfqpoint{3.420221in}{2.189500in}}%
\pgfusepath{clip}%
\pgfsetbuttcap%
\pgfsetmiterjoin%
\definecolor{currentfill}{rgb}{0.000000,0.478431,0.760784}%
\pgfsetfillcolor{currentfill}%
\pgfsetlinewidth{0.000000pt}%
\definecolor{currentstroke}{rgb}{0.000000,0.000000,0.000000}%
\pgfsetstrokecolor{currentstroke}%
\pgfsetstrokeopacity{0.000000}%
\pgfsetdash{}{0pt}%
\pgfpathmoveto{\pgfqpoint{2.800436in}{1.105317in}}%
\pgfpathlineto{\pgfqpoint{2.807211in}{1.099719in}}%
\pgfpathlineto{\pgfqpoint{2.817566in}{1.093835in}}%
\pgfpathlineto{\pgfqpoint{2.824078in}{1.095407in}}%
\pgfpathlineto{\pgfqpoint{2.828518in}{1.093957in}}%
\pgfpathlineto{\pgfqpoint{2.821177in}{1.078551in}}%
\pgfpathlineto{\pgfqpoint{2.814323in}{1.076986in}}%
\pgfpathlineto{\pgfqpoint{2.807396in}{1.079812in}}%
\pgfpathlineto{\pgfqpoint{2.802846in}{1.073825in}}%
\pgfpathlineto{\pgfqpoint{2.795123in}{1.071370in}}%
\pgfpathlineto{\pgfqpoint{2.783269in}{1.069921in}}%
\pgfpathlineto{\pgfqpoint{2.777985in}{1.071585in}}%
\pgfpathlineto{\pgfqpoint{2.769449in}{1.089454in}}%
\pgfpathlineto{\pgfqpoint{2.759202in}{1.097016in}}%
\pgfpathlineto{\pgfqpoint{2.758296in}{1.100507in}}%
\pgfpathlineto{\pgfqpoint{2.763226in}{1.109518in}}%
\pgfpathlineto{\pgfqpoint{2.773447in}{1.116203in}}%
\pgfpathlineto{\pgfqpoint{2.782511in}{1.113410in}}%
\pgfpathlineto{\pgfqpoint{2.787002in}{1.101858in}}%
\pgfpathlineto{\pgfqpoint{2.790484in}{1.100040in}}%
\pgfpathlineto{\pgfqpoint{2.794469in}{1.107334in}}%
\pgfpathclose%
\pgfusepath{fill}%
\end{pgfscope}%
\begin{pgfscope}%
\pgfpathrectangle{\pgfqpoint{0.100000in}{0.100000in}}{\pgfqpoint{3.420221in}{2.189500in}}%
\pgfusepath{clip}%
\pgfsetbuttcap%
\pgfsetmiterjoin%
\definecolor{currentfill}{rgb}{0.000000,0.466667,0.766667}%
\pgfsetfillcolor{currentfill}%
\pgfsetlinewidth{0.000000pt}%
\definecolor{currentstroke}{rgb}{0.000000,0.000000,0.000000}%
\pgfsetstrokecolor{currentstroke}%
\pgfsetstrokeopacity{0.000000}%
\pgfsetdash{}{0pt}%
\pgfpathmoveto{\pgfqpoint{0.621751in}{2.196819in}}%
\pgfpathlineto{\pgfqpoint{0.611896in}{2.203063in}}%
\pgfpathlineto{\pgfqpoint{0.607348in}{2.210263in}}%
\pgfpathlineto{\pgfqpoint{0.608674in}{2.217058in}}%
\pgfpathlineto{\pgfqpoint{0.614826in}{2.212767in}}%
\pgfpathlineto{\pgfqpoint{0.622189in}{2.219783in}}%
\pgfpathlineto{\pgfqpoint{0.630549in}{2.213165in}}%
\pgfpathclose%
\pgfusepath{fill}%
\end{pgfscope}%
\begin{pgfscope}%
\pgfpathrectangle{\pgfqpoint{0.100000in}{0.100000in}}{\pgfqpoint{3.420221in}{2.189500in}}%
\pgfusepath{clip}%
\pgfsetbuttcap%
\pgfsetmiterjoin%
\definecolor{currentfill}{rgb}{0.000000,0.466667,0.766667}%
\pgfsetfillcolor{currentfill}%
\pgfsetlinewidth{0.000000pt}%
\definecolor{currentstroke}{rgb}{0.000000,0.000000,0.000000}%
\pgfsetstrokecolor{currentstroke}%
\pgfsetstrokeopacity{0.000000}%
\pgfsetdash{}{0pt}%
\pgfpathmoveto{\pgfqpoint{1.563625in}{0.696062in}}%
\pgfpathlineto{\pgfqpoint{1.596014in}{0.693701in}}%
\pgfpathlineto{\pgfqpoint{1.616232in}{0.692573in}}%
\pgfpathlineto{\pgfqpoint{1.614467in}{0.664147in}}%
\pgfpathlineto{\pgfqpoint{1.612387in}{0.632536in}}%
\pgfpathlineto{\pgfqpoint{1.560320in}{0.636205in}}%
\pgfpathlineto{\pgfqpoint{1.567162in}{0.641054in}}%
\pgfpathlineto{\pgfqpoint{1.562231in}{0.646219in}}%
\pgfpathlineto{\pgfqpoint{1.568703in}{0.655417in}}%
\pgfpathlineto{\pgfqpoint{1.568005in}{0.662236in}}%
\pgfpathlineto{\pgfqpoint{1.563804in}{0.663750in}}%
\pgfpathlineto{\pgfqpoint{1.537757in}{0.665884in}}%
\pgfpathlineto{\pgfqpoint{1.537369in}{0.661475in}}%
\pgfpathlineto{\pgfqpoint{1.524085in}{0.662629in}}%
\pgfpathlineto{\pgfqpoint{1.522156in}{0.638874in}}%
\pgfpathlineto{\pgfqpoint{1.507553in}{0.640011in}}%
\pgfpathlineto{\pgfqpoint{1.506106in}{0.622590in}}%
\pgfpathlineto{\pgfqpoint{1.444647in}{0.682354in}}%
\pgfpathlineto{\pgfqpoint{1.485843in}{0.724518in}}%
\pgfpathlineto{\pgfqpoint{1.490676in}{0.722476in}}%
\pgfpathlineto{\pgfqpoint{1.497408in}{0.715681in}}%
\pgfpathlineto{\pgfqpoint{1.501013in}{0.717314in}}%
\pgfpathlineto{\pgfqpoint{1.507081in}{0.719306in}}%
\pgfpathlineto{\pgfqpoint{1.521440in}{0.709449in}}%
\pgfpathlineto{\pgfqpoint{1.522991in}{0.699845in}}%
\pgfpathclose%
\pgfusepath{fill}%
\end{pgfscope}%
\begin{pgfscope}%
\pgfpathrectangle{\pgfqpoint{0.100000in}{0.100000in}}{\pgfqpoint{3.420221in}{2.189500in}}%
\pgfusepath{clip}%
\pgfsetbuttcap%
\pgfsetmiterjoin%
\definecolor{currentfill}{rgb}{0.000000,0.478431,0.760784}%
\pgfsetfillcolor{currentfill}%
\pgfsetlinewidth{0.000000pt}%
\definecolor{currentstroke}{rgb}{0.000000,0.000000,0.000000}%
\pgfsetstrokecolor{currentstroke}%
\pgfsetstrokeopacity{0.000000}%
\pgfsetdash{}{0pt}%
\pgfpathmoveto{\pgfqpoint{2.182528in}{0.839749in}}%
\pgfpathlineto{\pgfqpoint{2.184673in}{0.829947in}}%
\pgfpathlineto{\pgfqpoint{2.187216in}{0.828233in}}%
\pgfpathlineto{\pgfqpoint{2.145790in}{0.827512in}}%
\pgfpathlineto{\pgfqpoint{2.129133in}{0.827351in}}%
\pgfpathlineto{\pgfqpoint{2.129177in}{0.854458in}}%
\pgfpathlineto{\pgfqpoint{2.120339in}{0.854495in}}%
\pgfpathlineto{\pgfqpoint{2.120392in}{0.859963in}}%
\pgfpathlineto{\pgfqpoint{2.120642in}{0.884382in}}%
\pgfpathlineto{\pgfqpoint{2.130230in}{0.886062in}}%
\pgfpathlineto{\pgfqpoint{2.133286in}{0.890644in}}%
\pgfpathlineto{\pgfqpoint{2.133608in}{0.913152in}}%
\pgfpathlineto{\pgfqpoint{2.146551in}{0.912997in}}%
\pgfpathlineto{\pgfqpoint{2.158058in}{0.912898in}}%
\pgfpathlineto{\pgfqpoint{2.168009in}{0.906817in}}%
\pgfpathlineto{\pgfqpoint{2.159603in}{0.906383in}}%
\pgfpathlineto{\pgfqpoint{2.160747in}{0.900052in}}%
\pgfpathlineto{\pgfqpoint{2.168350in}{0.889971in}}%
\pgfpathlineto{\pgfqpoint{2.170843in}{0.864760in}}%
\pgfpathlineto{\pgfqpoint{2.167008in}{0.855383in}}%
\pgfpathlineto{\pgfqpoint{2.168337in}{0.848254in}}%
\pgfpathlineto{\pgfqpoint{2.172246in}{0.848314in}}%
\pgfpathclose%
\pgfusepath{fill}%
\end{pgfscope}%
\begin{pgfscope}%
\pgfpathrectangle{\pgfqpoint{0.100000in}{0.100000in}}{\pgfqpoint{3.420221in}{2.189500in}}%
\pgfusepath{clip}%
\pgfsetbuttcap%
\pgfsetmiterjoin%
\definecolor{currentfill}{rgb}{0.000000,0.266667,0.866667}%
\pgfsetfillcolor{currentfill}%
\pgfsetlinewidth{0.000000pt}%
\definecolor{currentstroke}{rgb}{0.000000,0.000000,0.000000}%
\pgfsetstrokecolor{currentstroke}%
\pgfsetstrokeopacity{0.000000}%
\pgfsetdash{}{0pt}%
\pgfpathmoveto{\pgfqpoint{1.706936in}{1.436820in}}%
\pgfpathlineto{\pgfqpoint{1.705498in}{1.410906in}}%
\pgfpathlineto{\pgfqpoint{1.674048in}{1.412800in}}%
\pgfpathlineto{\pgfqpoint{1.647383in}{1.414389in}}%
\pgfpathlineto{\pgfqpoint{1.649067in}{1.440539in}}%
\pgfpathlineto{\pgfqpoint{1.647918in}{1.440620in}}%
\pgfpathlineto{\pgfqpoint{1.649754in}{1.466609in}}%
\pgfpathlineto{\pgfqpoint{1.642038in}{1.467184in}}%
\pgfpathlineto{\pgfqpoint{1.643919in}{1.493196in}}%
\pgfpathlineto{\pgfqpoint{1.675394in}{1.490861in}}%
\pgfpathlineto{\pgfqpoint{1.708415in}{1.488792in}}%
\pgfpathclose%
\pgfusepath{fill}%
\end{pgfscope}%
\begin{pgfscope}%
\pgfpathrectangle{\pgfqpoint{0.100000in}{0.100000in}}{\pgfqpoint{3.420221in}{2.189500in}}%
\pgfusepath{clip}%
\pgfsetbuttcap%
\pgfsetmiterjoin%
\definecolor{currentfill}{rgb}{0.000000,0.396078,0.801961}%
\pgfsetfillcolor{currentfill}%
\pgfsetlinewidth{0.000000pt}%
\definecolor{currentstroke}{rgb}{0.000000,0.000000,0.000000}%
\pgfsetstrokecolor{currentstroke}%
\pgfsetstrokeopacity{0.000000}%
\pgfsetdash{}{0pt}%
\pgfpathmoveto{\pgfqpoint{2.654303in}{1.369454in}}%
\pgfpathlineto{\pgfqpoint{2.654891in}{1.356947in}}%
\pgfpathlineto{\pgfqpoint{2.637130in}{1.356179in}}%
\pgfpathlineto{\pgfqpoint{2.619275in}{1.355542in}}%
\pgfpathlineto{\pgfqpoint{2.608353in}{1.352724in}}%
\pgfpathlineto{\pgfqpoint{2.589083in}{1.350403in}}%
\pgfpathlineto{\pgfqpoint{2.585651in}{1.383015in}}%
\pgfpathlineto{\pgfqpoint{2.583486in}{1.405784in}}%
\pgfpathlineto{\pgfqpoint{2.583116in}{1.408965in}}%
\pgfpathlineto{\pgfqpoint{2.604063in}{1.411475in}}%
\pgfpathlineto{\pgfqpoint{2.609292in}{1.413955in}}%
\pgfpathlineto{\pgfqpoint{2.608369in}{1.421563in}}%
\pgfpathlineto{\pgfqpoint{2.627905in}{1.424023in}}%
\pgfpathlineto{\pgfqpoint{2.627438in}{1.427917in}}%
\pgfpathlineto{\pgfqpoint{2.633828in}{1.428964in}}%
\pgfpathlineto{\pgfqpoint{2.654491in}{1.428927in}}%
\pgfpathlineto{\pgfqpoint{2.655308in}{1.408283in}}%
\pgfpathlineto{\pgfqpoint{2.658594in}{1.408417in}}%
\pgfpathlineto{\pgfqpoint{2.659831in}{1.392332in}}%
\pgfpathlineto{\pgfqpoint{2.657550in}{1.373782in}}%
\pgfpathclose%
\pgfusepath{fill}%
\end{pgfscope}%
\begin{pgfscope}%
\pgfpathrectangle{\pgfqpoint{0.100000in}{0.100000in}}{\pgfqpoint{3.420221in}{2.189500in}}%
\pgfusepath{clip}%
\pgfsetbuttcap%
\pgfsetmiterjoin%
\definecolor{currentfill}{rgb}{0.000000,0.600000,0.700000}%
\pgfsetfillcolor{currentfill}%
\pgfsetlinewidth{0.000000pt}%
\definecolor{currentstroke}{rgb}{0.000000,0.000000,0.000000}%
\pgfsetstrokecolor{currentstroke}%
\pgfsetstrokeopacity{0.000000}%
\pgfsetdash{}{0pt}%
\pgfpathmoveto{\pgfqpoint{2.176440in}{1.372100in}}%
\pgfpathlineto{\pgfqpoint{2.177103in}{1.349044in}}%
\pgfpathlineto{\pgfqpoint{2.154585in}{1.348627in}}%
\pgfpathlineto{\pgfqpoint{2.154375in}{1.354917in}}%
\pgfpathlineto{\pgfqpoint{2.125576in}{1.354158in}}%
\pgfpathlineto{\pgfqpoint{2.125076in}{1.354142in}}%
\pgfpathlineto{\pgfqpoint{2.124569in}{1.377019in}}%
\pgfpathlineto{\pgfqpoint{2.134358in}{1.377321in}}%
\pgfpathlineto{\pgfqpoint{2.134992in}{1.382551in}}%
\pgfpathlineto{\pgfqpoint{2.132169in}{1.395705in}}%
\pgfpathlineto{\pgfqpoint{2.136517in}{1.395922in}}%
\pgfpathlineto{\pgfqpoint{2.176000in}{1.398222in}}%
\pgfpathclose%
\pgfusepath{fill}%
\end{pgfscope}%
\begin{pgfscope}%
\pgfpathrectangle{\pgfqpoint{0.100000in}{0.100000in}}{\pgfqpoint{3.420221in}{2.189500in}}%
\pgfusepath{clip}%
\pgfsetbuttcap%
\pgfsetmiterjoin%
\definecolor{currentfill}{rgb}{0.000000,0.513725,0.743137}%
\pgfsetfillcolor{currentfill}%
\pgfsetlinewidth{0.000000pt}%
\definecolor{currentstroke}{rgb}{0.000000,0.000000,0.000000}%
\pgfsetstrokecolor{currentstroke}%
\pgfsetstrokeopacity{0.000000}%
\pgfsetdash{}{0pt}%
\pgfpathmoveto{\pgfqpoint{2.081721in}{0.956487in}}%
\pgfpathlineto{\pgfqpoint{2.071881in}{0.953906in}}%
\pgfpathlineto{\pgfqpoint{2.068199in}{0.950323in}}%
\pgfpathlineto{\pgfqpoint{2.062760in}{0.952281in}}%
\pgfpathlineto{\pgfqpoint{2.044257in}{0.952482in}}%
\pgfpathlineto{\pgfqpoint{2.035787in}{0.954795in}}%
\pgfpathlineto{\pgfqpoint{2.035479in}{0.938149in}}%
\pgfpathlineto{\pgfqpoint{2.005973in}{0.938009in}}%
\pgfpathlineto{\pgfqpoint{2.005955in}{0.951083in}}%
\pgfpathlineto{\pgfqpoint{1.970321in}{0.951191in}}%
\pgfpathlineto{\pgfqpoint{1.970297in}{0.944655in}}%
\pgfpathlineto{\pgfqpoint{1.960545in}{0.944696in}}%
\pgfpathlineto{\pgfqpoint{1.947544in}{0.944783in}}%
\pgfpathlineto{\pgfqpoint{1.947612in}{0.951316in}}%
\pgfpathlineto{\pgfqpoint{1.934634in}{0.951465in}}%
\pgfpathlineto{\pgfqpoint{1.935172in}{0.979200in}}%
\pgfpathlineto{\pgfqpoint{1.941718in}{0.986762in}}%
\pgfpathlineto{\pgfqpoint{1.948731in}{0.989322in}}%
\pgfpathlineto{\pgfqpoint{1.956564in}{0.989696in}}%
\pgfpathlineto{\pgfqpoint{1.967137in}{0.993598in}}%
\pgfpathlineto{\pgfqpoint{1.978482in}{0.999590in}}%
\pgfpathlineto{\pgfqpoint{1.988065in}{0.995129in}}%
\pgfpathlineto{\pgfqpoint{1.992268in}{1.002797in}}%
\pgfpathlineto{\pgfqpoint{1.997190in}{1.007485in}}%
\pgfpathlineto{\pgfqpoint{1.994020in}{1.014567in}}%
\pgfpathlineto{\pgfqpoint{1.994335in}{1.023025in}}%
\pgfpathlineto{\pgfqpoint{2.034273in}{1.023088in}}%
\pgfpathlineto{\pgfqpoint{2.032974in}{1.032132in}}%
\pgfpathlineto{\pgfqpoint{2.055403in}{1.031647in}}%
\pgfpathlineto{\pgfqpoint{2.068429in}{1.032488in}}%
\pgfpathlineto{\pgfqpoint{2.064045in}{1.025010in}}%
\pgfpathlineto{\pgfqpoint{2.058602in}{1.025058in}}%
\pgfpathlineto{\pgfqpoint{2.058500in}{1.018549in}}%
\pgfpathlineto{\pgfqpoint{2.061428in}{1.013443in}}%
\pgfpathlineto{\pgfqpoint{2.058680in}{1.008785in}}%
\pgfpathlineto{\pgfqpoint{2.058373in}{0.991477in}}%
\pgfpathlineto{\pgfqpoint{2.054956in}{0.986070in}}%
\pgfpathlineto{\pgfqpoint{2.054867in}{0.979485in}}%
\pgfpathlineto{\pgfqpoint{2.059034in}{0.977340in}}%
\pgfpathlineto{\pgfqpoint{2.081761in}{0.977102in}}%
\pgfpathclose%
\pgfusepath{fill}%
\end{pgfscope}%
\begin{pgfscope}%
\pgfpathrectangle{\pgfqpoint{0.100000in}{0.100000in}}{\pgfqpoint{3.420221in}{2.189500in}}%
\pgfusepath{clip}%
\pgfsetbuttcap%
\pgfsetmiterjoin%
\definecolor{currentfill}{rgb}{0.000000,0.490196,0.754902}%
\pgfsetfillcolor{currentfill}%
\pgfsetlinewidth{0.000000pt}%
\definecolor{currentstroke}{rgb}{0.000000,0.000000,0.000000}%
\pgfsetstrokecolor{currentstroke}%
\pgfsetstrokeopacity{0.000000}%
\pgfsetdash{}{0pt}%
\pgfpathmoveto{\pgfqpoint{3.391841in}{1.860205in}}%
\pgfpathlineto{\pgfqpoint{3.398923in}{1.859355in}}%
\pgfpathlineto{\pgfqpoint{3.396021in}{1.853659in}}%
\pgfpathclose%
\pgfusepath{fill}%
\end{pgfscope}%
\begin{pgfscope}%
\pgfpathrectangle{\pgfqpoint{0.100000in}{0.100000in}}{\pgfqpoint{3.420221in}{2.189500in}}%
\pgfusepath{clip}%
\pgfsetbuttcap%
\pgfsetmiterjoin%
\definecolor{currentfill}{rgb}{0.000000,0.490196,0.754902}%
\pgfsetfillcolor{currentfill}%
\pgfsetlinewidth{0.000000pt}%
\definecolor{currentstroke}{rgb}{0.000000,0.000000,0.000000}%
\pgfsetstrokecolor{currentstroke}%
\pgfsetstrokeopacity{0.000000}%
\pgfsetdash{}{0pt}%
\pgfpathmoveto{\pgfqpoint{3.370709in}{1.841180in}}%
\pgfpathlineto{\pgfqpoint{3.373752in}{1.849957in}}%
\pgfpathlineto{\pgfqpoint{3.367483in}{1.859839in}}%
\pgfpathlineto{\pgfqpoint{3.363923in}{1.858911in}}%
\pgfpathlineto{\pgfqpoint{3.359421in}{1.865700in}}%
\pgfpathlineto{\pgfqpoint{3.355948in}{1.866956in}}%
\pgfpathlineto{\pgfqpoint{3.356592in}{1.876617in}}%
\pgfpathlineto{\pgfqpoint{3.355601in}{1.890161in}}%
\pgfpathlineto{\pgfqpoint{3.353129in}{1.897669in}}%
\pgfpathlineto{\pgfqpoint{3.360381in}{1.897946in}}%
\pgfpathlineto{\pgfqpoint{3.348436in}{1.922049in}}%
\pgfpathlineto{\pgfqpoint{3.336174in}{1.913600in}}%
\pgfpathlineto{\pgfqpoint{3.317021in}{1.949824in}}%
\pgfpathlineto{\pgfqpoint{3.319320in}{1.957167in}}%
\pgfpathlineto{\pgfqpoint{3.313839in}{1.960523in}}%
\pgfpathlineto{\pgfqpoint{3.314371in}{1.973467in}}%
\pgfpathlineto{\pgfqpoint{3.310630in}{1.981972in}}%
\pgfpathlineto{\pgfqpoint{3.301852in}{2.012260in}}%
\pgfpathlineto{\pgfqpoint{3.298672in}{2.025538in}}%
\pgfpathlineto{\pgfqpoint{3.343558in}{2.038485in}}%
\pgfpathlineto{\pgfqpoint{3.347484in}{2.025715in}}%
\pgfpathlineto{\pgfqpoint{3.366904in}{2.030425in}}%
\pgfpathlineto{\pgfqpoint{3.376860in}{1.998945in}}%
\pgfpathlineto{\pgfqpoint{3.384510in}{1.972619in}}%
\pgfpathlineto{\pgfqpoint{3.385814in}{1.972367in}}%
\pgfpathlineto{\pgfqpoint{3.402633in}{1.982740in}}%
\pgfpathlineto{\pgfqpoint{3.416375in}{1.957568in}}%
\pgfpathlineto{\pgfqpoint{3.410721in}{1.954879in}}%
\pgfpathlineto{\pgfqpoint{3.421034in}{1.934432in}}%
\pgfpathlineto{\pgfqpoint{3.415119in}{1.931322in}}%
\pgfpathlineto{\pgfqpoint{3.427359in}{1.907488in}}%
\pgfpathlineto{\pgfqpoint{3.434518in}{1.895708in}}%
\pgfpathlineto{\pgfqpoint{3.430044in}{1.888725in}}%
\pgfpathlineto{\pgfqpoint{3.424197in}{1.897916in}}%
\pgfpathlineto{\pgfqpoint{3.417238in}{1.894133in}}%
\pgfpathlineto{\pgfqpoint{3.425050in}{1.886280in}}%
\pgfpathlineto{\pgfqpoint{3.418873in}{1.875996in}}%
\pgfpathlineto{\pgfqpoint{3.412146in}{1.881153in}}%
\pgfpathlineto{\pgfqpoint{3.412438in}{1.886214in}}%
\pgfpathlineto{\pgfqpoint{3.407695in}{1.890775in}}%
\pgfpathlineto{\pgfqpoint{3.405138in}{1.885652in}}%
\pgfpathlineto{\pgfqpoint{3.404181in}{1.875710in}}%
\pgfpathlineto{\pgfqpoint{3.394732in}{1.877536in}}%
\pgfpathlineto{\pgfqpoint{3.383600in}{1.883806in}}%
\pgfpathlineto{\pgfqpoint{3.381001in}{1.880773in}}%
\pgfpathlineto{\pgfqpoint{3.384764in}{1.872635in}}%
\pgfpathlineto{\pgfqpoint{3.381877in}{1.859937in}}%
\pgfpathlineto{\pgfqpoint{3.384637in}{1.852048in}}%
\pgfpathlineto{\pgfqpoint{3.380352in}{1.843199in}}%
\pgfpathclose%
\pgfusepath{fill}%
\end{pgfscope}%
\begin{pgfscope}%
\pgfpathrectangle{\pgfqpoint{0.100000in}{0.100000in}}{\pgfqpoint{3.420221in}{2.189500in}}%
\pgfusepath{clip}%
\pgfsetbuttcap%
\pgfsetmiterjoin%
\definecolor{currentfill}{rgb}{0.000000,0.564706,0.717647}%
\pgfsetfillcolor{currentfill}%
\pgfsetlinewidth{0.000000pt}%
\definecolor{currentstroke}{rgb}{0.000000,0.000000,0.000000}%
\pgfsetstrokecolor{currentstroke}%
\pgfsetstrokeopacity{0.000000}%
\pgfsetdash{}{0pt}%
\pgfpathmoveto{\pgfqpoint{1.767804in}{0.640448in}}%
\pgfpathlineto{\pgfqpoint{1.743511in}{0.641354in}}%
\pgfpathlineto{\pgfqpoint{1.744882in}{0.673335in}}%
\pgfpathlineto{\pgfqpoint{1.778527in}{0.671968in}}%
\pgfpathlineto{\pgfqpoint{1.779231in}{0.680125in}}%
\pgfpathlineto{\pgfqpoint{1.813444in}{0.679336in}}%
\pgfpathlineto{\pgfqpoint{1.818539in}{0.669780in}}%
\pgfpathlineto{\pgfqpoint{1.809514in}{0.660601in}}%
\pgfpathlineto{\pgfqpoint{1.798798in}{0.639328in}}%
\pgfpathlineto{\pgfqpoint{1.798006in}{0.633746in}}%
\pgfpathlineto{\pgfqpoint{1.779559in}{0.640164in}}%
\pgfpathclose%
\pgfusepath{fill}%
\end{pgfscope}%
\begin{pgfscope}%
\pgfpathrectangle{\pgfqpoint{0.100000in}{0.100000in}}{\pgfqpoint{3.420221in}{2.189500in}}%
\pgfusepath{clip}%
\pgfsetbuttcap%
\pgfsetmiterjoin%
\definecolor{currentfill}{rgb}{0.000000,0.800000,0.600000}%
\pgfsetfillcolor{currentfill}%
\pgfsetlinewidth{0.000000pt}%
\definecolor{currentstroke}{rgb}{0.000000,0.000000,0.000000}%
\pgfsetstrokecolor{currentstroke}%
\pgfsetstrokeopacity{0.000000}%
\pgfsetdash{}{0pt}%
\pgfpathmoveto{\pgfqpoint{2.244232in}{0.830051in}}%
\pgfpathlineto{\pgfqpoint{2.249326in}{0.828169in}}%
\pgfpathlineto{\pgfqpoint{2.250633in}{0.821619in}}%
\pgfpathlineto{\pgfqpoint{2.245299in}{0.816386in}}%
\pgfpathlineto{\pgfqpoint{2.245071in}{0.811827in}}%
\pgfpathlineto{\pgfqpoint{2.251781in}{0.809512in}}%
\pgfpathlineto{\pgfqpoint{2.246235in}{0.802730in}}%
\pgfpathlineto{\pgfqpoint{2.247981in}{0.798710in}}%
\pgfpathlineto{\pgfqpoint{2.253368in}{0.798171in}}%
\pgfpathlineto{\pgfqpoint{2.247770in}{0.795038in}}%
\pgfpathlineto{\pgfqpoint{2.227198in}{0.794188in}}%
\pgfpathlineto{\pgfqpoint{2.227852in}{0.797545in}}%
\pgfpathlineto{\pgfqpoint{2.218284in}{0.797196in}}%
\pgfpathlineto{\pgfqpoint{2.215228in}{0.806156in}}%
\pgfpathlineto{\pgfqpoint{2.220920in}{0.819497in}}%
\pgfpathlineto{\pgfqpoint{2.223821in}{0.820682in}}%
\pgfpathlineto{\pgfqpoint{2.227203in}{0.829507in}}%
\pgfpathclose%
\pgfusepath{fill}%
\end{pgfscope}%
\begin{pgfscope}%
\pgfpathrectangle{\pgfqpoint{0.100000in}{0.100000in}}{\pgfqpoint{3.420221in}{2.189500in}}%
\pgfusepath{clip}%
\pgfsetbuttcap%
\pgfsetmiterjoin%
\definecolor{currentfill}{rgb}{0.000000,0.600000,0.700000}%
\pgfsetfillcolor{currentfill}%
\pgfsetlinewidth{0.000000pt}%
\definecolor{currentstroke}{rgb}{0.000000,0.000000,0.000000}%
\pgfsetstrokecolor{currentstroke}%
\pgfsetstrokeopacity{0.000000}%
\pgfsetdash{}{0pt}%
\pgfpathmoveto{\pgfqpoint{1.989101in}{0.660413in}}%
\pgfpathlineto{\pgfqpoint{1.985798in}{0.666621in}}%
\pgfpathlineto{\pgfqpoint{1.980838in}{0.663083in}}%
\pgfpathlineto{\pgfqpoint{1.978502in}{0.636551in}}%
\pgfpathlineto{\pgfqpoint{1.963031in}{0.636636in}}%
\pgfpathlineto{\pgfqpoint{1.948024in}{0.645939in}}%
\pgfpathlineto{\pgfqpoint{1.946054in}{0.663658in}}%
\pgfpathlineto{\pgfqpoint{1.926193in}{0.660768in}}%
\pgfpathlineto{\pgfqpoint{1.921621in}{0.672218in}}%
\pgfpathlineto{\pgfqpoint{1.938901in}{0.680954in}}%
\pgfpathlineto{\pgfqpoint{1.952608in}{0.681005in}}%
\pgfpathlineto{\pgfqpoint{1.951930in}{0.684231in}}%
\pgfpathlineto{\pgfqpoint{1.955232in}{0.694323in}}%
\pgfpathlineto{\pgfqpoint{1.958759in}{0.696206in}}%
\pgfpathlineto{\pgfqpoint{1.954447in}{0.711960in}}%
\pgfpathlineto{\pgfqpoint{1.960145in}{0.714779in}}%
\pgfpathlineto{\pgfqpoint{1.984517in}{0.718513in}}%
\pgfpathlineto{\pgfqpoint{1.992469in}{0.716729in}}%
\pgfpathlineto{\pgfqpoint{1.995278in}{0.710187in}}%
\pgfpathlineto{\pgfqpoint{2.001901in}{0.705887in}}%
\pgfpathlineto{\pgfqpoint{2.010751in}{0.713514in}}%
\pgfpathlineto{\pgfqpoint{2.019601in}{0.708285in}}%
\pgfpathlineto{\pgfqpoint{2.034131in}{0.704820in}}%
\pgfpathlineto{\pgfqpoint{2.038397in}{0.697422in}}%
\pgfpathlineto{\pgfqpoint{2.045075in}{0.691189in}}%
\pgfpathlineto{\pgfqpoint{2.058468in}{0.681450in}}%
\pgfpathlineto{\pgfqpoint{2.037251in}{0.676307in}}%
\pgfpathlineto{\pgfqpoint{2.026559in}{0.680335in}}%
\pgfpathlineto{\pgfqpoint{2.018206in}{0.681698in}}%
\pgfpathlineto{\pgfqpoint{2.012282in}{0.684777in}}%
\pgfpathlineto{\pgfqpoint{2.006054in}{0.676853in}}%
\pgfpathclose%
\pgfusepath{fill}%
\end{pgfscope}%
\begin{pgfscope}%
\pgfpathrectangle{\pgfqpoint{0.100000in}{0.100000in}}{\pgfqpoint{3.420221in}{2.189500in}}%
\pgfusepath{clip}%
\pgfsetbuttcap%
\pgfsetmiterjoin%
\definecolor{currentfill}{rgb}{0.000000,0.517647,0.741176}%
\pgfsetfillcolor{currentfill}%
\pgfsetlinewidth{0.000000pt}%
\definecolor{currentstroke}{rgb}{0.000000,0.000000,0.000000}%
\pgfsetstrokecolor{currentstroke}%
\pgfsetstrokeopacity{0.000000}%
\pgfsetdash{}{0pt}%
\pgfpathmoveto{\pgfqpoint{2.758296in}{1.100507in}}%
\pgfpathlineto{\pgfqpoint{2.752055in}{1.098869in}}%
\pgfpathlineto{\pgfqpoint{2.749769in}{1.105999in}}%
\pgfpathlineto{\pgfqpoint{2.734971in}{1.094864in}}%
\pgfpathlineto{\pgfqpoint{2.723922in}{1.104353in}}%
\pgfpathlineto{\pgfqpoint{2.716392in}{1.108500in}}%
\pgfpathlineto{\pgfqpoint{2.720990in}{1.115497in}}%
\pgfpathlineto{\pgfqpoint{2.715454in}{1.121623in}}%
\pgfpathlineto{\pgfqpoint{2.710162in}{1.117620in}}%
\pgfpathlineto{\pgfqpoint{2.707627in}{1.125277in}}%
\pgfpathlineto{\pgfqpoint{2.712529in}{1.128154in}}%
\pgfpathlineto{\pgfqpoint{2.715776in}{1.135184in}}%
\pgfpathlineto{\pgfqpoint{2.723344in}{1.142396in}}%
\pgfpathlineto{\pgfqpoint{2.724773in}{1.148797in}}%
\pgfpathlineto{\pgfqpoint{2.731610in}{1.154061in}}%
\pgfpathlineto{\pgfqpoint{2.735060in}{1.162053in}}%
\pgfpathlineto{\pgfqpoint{2.741367in}{1.166373in}}%
\pgfpathlineto{\pgfqpoint{2.754925in}{1.168084in}}%
\pgfpathlineto{\pgfqpoint{2.761092in}{1.156303in}}%
\pgfpathlineto{\pgfqpoint{2.773721in}{1.164860in}}%
\pgfpathlineto{\pgfqpoint{2.774670in}{1.169257in}}%
\pgfpathlineto{\pgfqpoint{2.785778in}{1.173488in}}%
\pgfpathlineto{\pgfqpoint{2.789038in}{1.178342in}}%
\pgfpathlineto{\pgfqpoint{2.795226in}{1.181477in}}%
\pgfpathlineto{\pgfqpoint{2.810971in}{1.186335in}}%
\pgfpathlineto{\pgfqpoint{2.818751in}{1.179126in}}%
\pgfpathlineto{\pgfqpoint{2.824150in}{1.169603in}}%
\pgfpathlineto{\pgfqpoint{2.809112in}{1.163141in}}%
\pgfpathlineto{\pgfqpoint{2.808356in}{1.159169in}}%
\pgfpathlineto{\pgfqpoint{2.802926in}{1.154951in}}%
\pgfpathlineto{\pgfqpoint{2.792191in}{1.153548in}}%
\pgfpathlineto{\pgfqpoint{2.786655in}{1.149234in}}%
\pgfpathlineto{\pgfqpoint{2.785919in}{1.142179in}}%
\pgfpathlineto{\pgfqpoint{2.782179in}{1.134821in}}%
\pgfpathlineto{\pgfqpoint{2.788542in}{1.126307in}}%
\pgfpathlineto{\pgfqpoint{2.785358in}{1.115371in}}%
\pgfpathlineto{\pgfqpoint{2.782511in}{1.113410in}}%
\pgfpathlineto{\pgfqpoint{2.773447in}{1.116203in}}%
\pgfpathlineto{\pgfqpoint{2.763226in}{1.109518in}}%
\pgfpathclose%
\pgfusepath{fill}%
\end{pgfscope}%
\begin{pgfscope}%
\pgfpathrectangle{\pgfqpoint{0.100000in}{0.100000in}}{\pgfqpoint{3.420221in}{2.189500in}}%
\pgfusepath{clip}%
\pgfsetbuttcap%
\pgfsetmiterjoin%
\definecolor{currentfill}{rgb}{0.000000,0.592157,0.703922}%
\pgfsetfillcolor{currentfill}%
\pgfsetlinewidth{0.000000pt}%
\definecolor{currentstroke}{rgb}{0.000000,0.000000,0.000000}%
\pgfsetstrokecolor{currentstroke}%
\pgfsetstrokeopacity{0.000000}%
\pgfsetdash{}{0pt}%
\pgfpathmoveto{\pgfqpoint{2.410143in}{1.784162in}}%
\pgfpathlineto{\pgfqpoint{2.402128in}{1.763423in}}%
\pgfpathlineto{\pgfqpoint{2.395629in}{1.753906in}}%
\pgfpathlineto{\pgfqpoint{2.394124in}{1.739391in}}%
\pgfpathlineto{\pgfqpoint{2.387405in}{1.738040in}}%
\pgfpathlineto{\pgfqpoint{2.368022in}{1.740926in}}%
\pgfpathlineto{\pgfqpoint{2.367646in}{1.747446in}}%
\pgfpathlineto{\pgfqpoint{2.363568in}{1.753779in}}%
\pgfpathlineto{\pgfqpoint{2.357061in}{1.753552in}}%
\pgfpathlineto{\pgfqpoint{2.356273in}{1.766568in}}%
\pgfpathlineto{\pgfqpoint{2.349801in}{1.766279in}}%
\pgfpathlineto{\pgfqpoint{2.348064in}{1.792270in}}%
\pgfpathlineto{\pgfqpoint{2.334895in}{1.791365in}}%
\pgfpathlineto{\pgfqpoint{2.333622in}{1.810795in}}%
\pgfpathlineto{\pgfqpoint{2.316486in}{1.818440in}}%
\pgfpathlineto{\pgfqpoint{2.314865in}{1.842736in}}%
\pgfpathlineto{\pgfqpoint{2.360277in}{1.845958in}}%
\pgfpathlineto{\pgfqpoint{2.361254in}{1.832926in}}%
\pgfpathlineto{\pgfqpoint{2.387357in}{1.834992in}}%
\pgfpathlineto{\pgfqpoint{2.388897in}{1.815352in}}%
\pgfpathlineto{\pgfqpoint{2.401932in}{1.816384in}}%
\pgfpathlineto{\pgfqpoint{2.404677in}{1.810052in}}%
\pgfpathlineto{\pgfqpoint{2.406801in}{1.783876in}}%
\pgfpathclose%
\pgfusepath{fill}%
\end{pgfscope}%
\begin{pgfscope}%
\pgfpathrectangle{\pgfqpoint{0.100000in}{0.100000in}}{\pgfqpoint{3.420221in}{2.189500in}}%
\pgfusepath{clip}%
\pgfsetbuttcap%
\pgfsetmiterjoin%
\definecolor{currentfill}{rgb}{0.000000,0.509804,0.745098}%
\pgfsetfillcolor{currentfill}%
\pgfsetlinewidth{0.000000pt}%
\definecolor{currentstroke}{rgb}{0.000000,0.000000,0.000000}%
\pgfsetstrokecolor{currentstroke}%
\pgfsetstrokeopacity{0.000000}%
\pgfsetdash{}{0pt}%
\pgfpathmoveto{\pgfqpoint{2.381832in}{0.987606in}}%
\pgfpathlineto{\pgfqpoint{2.358813in}{0.986031in}}%
\pgfpathlineto{\pgfqpoint{2.357760in}{1.016259in}}%
\pgfpathlineto{\pgfqpoint{2.364008in}{1.019243in}}%
\pgfpathlineto{\pgfqpoint{2.363299in}{1.038853in}}%
\pgfpathlineto{\pgfqpoint{2.345956in}{1.045562in}}%
\pgfpathlineto{\pgfqpoint{2.355985in}{1.063440in}}%
\pgfpathlineto{\pgfqpoint{2.355665in}{1.076848in}}%
\pgfpathlineto{\pgfqpoint{2.367355in}{1.077383in}}%
\pgfpathlineto{\pgfqpoint{2.373557in}{1.072996in}}%
\pgfpathlineto{\pgfqpoint{2.384289in}{1.068094in}}%
\pgfpathlineto{\pgfqpoint{2.384846in}{1.047700in}}%
\pgfpathlineto{\pgfqpoint{2.390924in}{1.047996in}}%
\pgfpathlineto{\pgfqpoint{2.391575in}{1.032908in}}%
\pgfpathlineto{\pgfqpoint{2.398573in}{1.027673in}}%
\pgfpathlineto{\pgfqpoint{2.404266in}{1.025849in}}%
\pgfpathlineto{\pgfqpoint{2.407898in}{1.021344in}}%
\pgfpathlineto{\pgfqpoint{2.407991in}{1.018513in}}%
\pgfpathlineto{\pgfqpoint{2.395215in}{1.017654in}}%
\pgfpathlineto{\pgfqpoint{2.391109in}{1.015740in}}%
\pgfpathlineto{\pgfqpoint{2.386877in}{1.006945in}}%
\pgfpathlineto{\pgfqpoint{2.383083in}{1.006687in}}%
\pgfpathlineto{\pgfqpoint{2.384082in}{0.987761in}}%
\pgfpathclose%
\pgfusepath{fill}%
\end{pgfscope}%
\begin{pgfscope}%
\pgfpathrectangle{\pgfqpoint{0.100000in}{0.100000in}}{\pgfqpoint{3.420221in}{2.189500in}}%
\pgfusepath{clip}%
\pgfsetbuttcap%
\pgfsetmiterjoin%
\definecolor{currentfill}{rgb}{0.000000,0.270588,0.864706}%
\pgfsetfillcolor{currentfill}%
\pgfsetlinewidth{0.000000pt}%
\definecolor{currentstroke}{rgb}{0.000000,0.000000,0.000000}%
\pgfsetstrokecolor{currentstroke}%
\pgfsetstrokeopacity{0.000000}%
\pgfsetdash{}{0pt}%
\pgfpathmoveto{\pgfqpoint{1.863264in}{1.247528in}}%
\pgfpathlineto{\pgfqpoint{1.830928in}{1.248509in}}%
\pgfpathlineto{\pgfqpoint{1.831301in}{1.268083in}}%
\pgfpathlineto{\pgfqpoint{1.798902in}{1.269198in}}%
\pgfpathlineto{\pgfqpoint{1.799364in}{1.288818in}}%
\pgfpathlineto{\pgfqpoint{1.799608in}{1.295325in}}%
\pgfpathlineto{\pgfqpoint{1.832074in}{1.294144in}}%
\pgfpathlineto{\pgfqpoint{1.832327in}{1.300676in}}%
\pgfpathlineto{\pgfqpoint{1.864761in}{1.299711in}}%
\pgfpathclose%
\pgfusepath{fill}%
\end{pgfscope}%
\begin{pgfscope}%
\pgfpathrectangle{\pgfqpoint{0.100000in}{0.100000in}}{\pgfqpoint{3.420221in}{2.189500in}}%
\pgfusepath{clip}%
\pgfsetbuttcap%
\pgfsetmiterjoin%
\definecolor{currentfill}{rgb}{0.000000,0.407843,0.796078}%
\pgfsetfillcolor{currentfill}%
\pgfsetlinewidth{0.000000pt}%
\definecolor{currentstroke}{rgb}{0.000000,0.000000,0.000000}%
\pgfsetstrokecolor{currentstroke}%
\pgfsetstrokeopacity{0.000000}%
\pgfsetdash{}{0pt}%
\pgfpathmoveto{\pgfqpoint{2.420478in}{1.103406in}}%
\pgfpathlineto{\pgfqpoint{2.421473in}{1.106757in}}%
\pgfpathlineto{\pgfqpoint{2.415601in}{1.121818in}}%
\pgfpathlineto{\pgfqpoint{2.412131in}{1.130599in}}%
\pgfpathlineto{\pgfqpoint{2.427538in}{1.138774in}}%
\pgfpathlineto{\pgfqpoint{2.434954in}{1.140453in}}%
\pgfpathlineto{\pgfqpoint{2.437904in}{1.144214in}}%
\pgfpathlineto{\pgfqpoint{2.438930in}{1.153919in}}%
\pgfpathlineto{\pgfqpoint{2.448686in}{1.151389in}}%
\pgfpathlineto{\pgfqpoint{2.459552in}{1.156183in}}%
\pgfpathlineto{\pgfqpoint{2.465269in}{1.147902in}}%
\pgfpathlineto{\pgfqpoint{2.476867in}{1.150453in}}%
\pgfpathlineto{\pgfqpoint{2.479149in}{1.119212in}}%
\pgfpathlineto{\pgfqpoint{2.475905in}{1.118871in}}%
\pgfpathlineto{\pgfqpoint{2.476254in}{1.105317in}}%
\pgfpathlineto{\pgfqpoint{2.469501in}{1.098410in}}%
\pgfpathlineto{\pgfqpoint{2.467660in}{1.094038in}}%
\pgfpathlineto{\pgfqpoint{2.453911in}{1.093847in}}%
\pgfpathlineto{\pgfqpoint{2.453832in}{1.087382in}}%
\pgfpathlineto{\pgfqpoint{2.450386in}{1.083246in}}%
\pgfpathlineto{\pgfqpoint{2.441402in}{1.085362in}}%
\pgfpathlineto{\pgfqpoint{2.428181in}{1.084896in}}%
\pgfpathlineto{\pgfqpoint{2.427307in}{1.090235in}}%
\pgfpathlineto{\pgfqpoint{2.420583in}{1.098954in}}%
\pgfpathclose%
\pgfusepath{fill}%
\end{pgfscope}%
\begin{pgfscope}%
\pgfpathrectangle{\pgfqpoint{0.100000in}{0.100000in}}{\pgfqpoint{3.420221in}{2.189500in}}%
\pgfusepath{clip}%
\pgfsetbuttcap%
\pgfsetmiterjoin%
\definecolor{currentfill}{rgb}{0.000000,0.227451,0.886275}%
\pgfsetfillcolor{currentfill}%
\pgfsetlinewidth{0.000000pt}%
\definecolor{currentstroke}{rgb}{0.000000,0.000000,0.000000}%
\pgfsetstrokecolor{currentstroke}%
\pgfsetstrokeopacity{0.000000}%
\pgfsetdash{}{0pt}%
\pgfpathmoveto{\pgfqpoint{3.081378in}{1.441459in}}%
\pgfpathlineto{\pgfqpoint{3.101762in}{1.445809in}}%
\pgfpathlineto{\pgfqpoint{3.105599in}{1.453697in}}%
\pgfpathlineto{\pgfqpoint{3.109958in}{1.456419in}}%
\pgfpathlineto{\pgfqpoint{3.118210in}{1.456923in}}%
\pgfpathlineto{\pgfqpoint{3.118614in}{1.452479in}}%
\pgfpathlineto{\pgfqpoint{3.115396in}{1.442094in}}%
\pgfpathlineto{\pgfqpoint{3.118967in}{1.437711in}}%
\pgfpathlineto{\pgfqpoint{3.119163in}{1.432425in}}%
\pgfpathlineto{\pgfqpoint{3.116152in}{1.429141in}}%
\pgfpathlineto{\pgfqpoint{3.122136in}{1.422988in}}%
\pgfpathlineto{\pgfqpoint{3.115476in}{1.415806in}}%
\pgfpathlineto{\pgfqpoint{3.109232in}{1.414832in}}%
\pgfpathlineto{\pgfqpoint{3.107585in}{1.420655in}}%
\pgfpathlineto{\pgfqpoint{3.101074in}{1.418013in}}%
\pgfpathlineto{\pgfqpoint{3.091678in}{1.417486in}}%
\pgfpathlineto{\pgfqpoint{3.093682in}{1.421815in}}%
\pgfpathlineto{\pgfqpoint{3.093118in}{1.429170in}}%
\pgfpathlineto{\pgfqpoint{3.087545in}{1.429119in}}%
\pgfpathlineto{\pgfqpoint{3.075868in}{1.440278in}}%
\pgfpathclose%
\pgfusepath{fill}%
\end{pgfscope}%
\begin{pgfscope}%
\pgfpathrectangle{\pgfqpoint{0.100000in}{0.100000in}}{\pgfqpoint{3.420221in}{2.189500in}}%
\pgfusepath{clip}%
\pgfsetbuttcap%
\pgfsetmiterjoin%
\definecolor{currentfill}{rgb}{0.000000,0.278431,0.860784}%
\pgfsetfillcolor{currentfill}%
\pgfsetlinewidth{0.000000pt}%
\definecolor{currentstroke}{rgb}{0.000000,0.000000,0.000000}%
\pgfsetstrokecolor{currentstroke}%
\pgfsetstrokeopacity{0.000000}%
\pgfsetdash{}{0pt}%
\pgfpathmoveto{\pgfqpoint{1.703121in}{1.293354in}}%
\pgfpathlineto{\pgfqpoint{1.702166in}{1.293388in}}%
\pgfpathlineto{\pgfqpoint{1.704176in}{1.325944in}}%
\pgfpathlineto{\pgfqpoint{1.734953in}{1.324296in}}%
\pgfpathlineto{\pgfqpoint{1.768636in}{1.322719in}}%
\pgfpathlineto{\pgfqpoint{1.766978in}{1.290183in}}%
\pgfpathclose%
\pgfusepath{fill}%
\end{pgfscope}%
\begin{pgfscope}%
\pgfpathrectangle{\pgfqpoint{0.100000in}{0.100000in}}{\pgfqpoint{3.420221in}{2.189500in}}%
\pgfusepath{clip}%
\pgfsetbuttcap%
\pgfsetmiterjoin%
\definecolor{currentfill}{rgb}{0.000000,0.329412,0.835294}%
\pgfsetfillcolor{currentfill}%
\pgfsetlinewidth{0.000000pt}%
\definecolor{currentstroke}{rgb}{0.000000,0.000000,0.000000}%
\pgfsetstrokecolor{currentstroke}%
\pgfsetstrokeopacity{0.000000}%
\pgfsetdash{}{0pt}%
\pgfpathmoveto{\pgfqpoint{1.801759in}{1.128650in}}%
\pgfpathlineto{\pgfqpoint{1.802669in}{1.157580in}}%
\pgfpathlineto{\pgfqpoint{1.795819in}{1.157836in}}%
\pgfpathlineto{\pgfqpoint{1.796820in}{1.183949in}}%
\pgfpathlineto{\pgfqpoint{1.842202in}{1.182595in}}%
\pgfpathlineto{\pgfqpoint{1.842398in}{1.195912in}}%
\pgfpathlineto{\pgfqpoint{1.874879in}{1.195020in}}%
\pgfpathlineto{\pgfqpoint{1.875150in}{1.208129in}}%
\pgfpathlineto{\pgfqpoint{1.893548in}{1.207568in}}%
\pgfpathlineto{\pgfqpoint{1.912311in}{1.207284in}}%
\pgfpathlineto{\pgfqpoint{1.911373in}{1.161584in}}%
\pgfpathlineto{\pgfqpoint{1.874054in}{1.162275in}}%
\pgfpathlineto{\pgfqpoint{1.873540in}{1.126566in}}%
\pgfpathlineto{\pgfqpoint{1.854713in}{1.127012in}}%
\pgfpathclose%
\pgfusepath{fill}%
\end{pgfscope}%
\begin{pgfscope}%
\pgfpathrectangle{\pgfqpoint{0.100000in}{0.100000in}}{\pgfqpoint{3.420221in}{2.189500in}}%
\pgfusepath{clip}%
\pgfsetbuttcap%
\pgfsetmiterjoin%
\definecolor{currentfill}{rgb}{0.000000,0.341176,0.829412}%
\pgfsetfillcolor{currentfill}%
\pgfsetlinewidth{0.000000pt}%
\definecolor{currentstroke}{rgb}{0.000000,0.000000,0.000000}%
\pgfsetstrokecolor{currentstroke}%
\pgfsetstrokeopacity{0.000000}%
\pgfsetdash{}{0pt}%
\pgfpathmoveto{\pgfqpoint{1.722266in}{0.886745in}}%
\pgfpathlineto{\pgfqpoint{1.721171in}{0.889409in}}%
\pgfpathlineto{\pgfqpoint{1.708710in}{0.894375in}}%
\pgfpathlineto{\pgfqpoint{1.686831in}{0.896323in}}%
\pgfpathlineto{\pgfqpoint{1.688565in}{0.925888in}}%
\pgfpathlineto{\pgfqpoint{1.691610in}{0.925312in}}%
\pgfpathlineto{\pgfqpoint{1.693006in}{0.950545in}}%
\pgfpathlineto{\pgfqpoint{1.697408in}{0.951473in}}%
\pgfpathlineto{\pgfqpoint{1.710993in}{0.935868in}}%
\pgfpathlineto{\pgfqpoint{1.716862in}{0.935293in}}%
\pgfpathlineto{\pgfqpoint{1.722254in}{0.938006in}}%
\pgfpathlineto{\pgfqpoint{1.728701in}{0.934562in}}%
\pgfpathlineto{\pgfqpoint{1.730730in}{0.940797in}}%
\pgfpathlineto{\pgfqpoint{1.740838in}{0.931281in}}%
\pgfpathlineto{\pgfqpoint{1.741743in}{0.922019in}}%
\pgfpathlineto{\pgfqpoint{1.756457in}{0.921234in}}%
\pgfpathlineto{\pgfqpoint{1.755182in}{0.892795in}}%
\pgfpathlineto{\pgfqpoint{1.722569in}{0.894304in}}%
\pgfpathclose%
\pgfusepath{fill}%
\end{pgfscope}%
\begin{pgfscope}%
\pgfpathrectangle{\pgfqpoint{0.100000in}{0.100000in}}{\pgfqpoint{3.420221in}{2.189500in}}%
\pgfusepath{clip}%
\pgfsetbuttcap%
\pgfsetmiterjoin%
\definecolor{currentfill}{rgb}{0.000000,0.396078,0.801961}%
\pgfsetfillcolor{currentfill}%
\pgfsetlinewidth{0.000000pt}%
\definecolor{currentstroke}{rgb}{0.000000,0.000000,0.000000}%
\pgfsetstrokecolor{currentstroke}%
\pgfsetstrokeopacity{0.000000}%
\pgfsetdash{}{0pt}%
\pgfpathmoveto{\pgfqpoint{0.864088in}{0.378009in}}%
\pgfpathlineto{\pgfqpoint{0.860297in}{0.379879in}}%
\pgfpathlineto{\pgfqpoint{0.858530in}{0.382216in}}%
\pgfpathlineto{\pgfqpoint{0.856075in}{0.382828in}}%
\pgfpathlineto{\pgfqpoint{0.856600in}{0.385546in}}%
\pgfpathlineto{\pgfqpoint{0.855725in}{0.386468in}}%
\pgfpathlineto{\pgfqpoint{0.853403in}{0.384433in}}%
\pgfpathlineto{\pgfqpoint{0.851320in}{0.383658in}}%
\pgfpathlineto{\pgfqpoint{0.849895in}{0.385287in}}%
\pgfpathlineto{\pgfqpoint{0.848210in}{0.384666in}}%
\pgfpathlineto{\pgfqpoint{0.846977in}{0.381173in}}%
\pgfpathlineto{\pgfqpoint{0.844878in}{0.385984in}}%
\pgfpathlineto{\pgfqpoint{0.844309in}{0.382180in}}%
\pgfpathlineto{\pgfqpoint{0.840379in}{0.381991in}}%
\pgfpathlineto{\pgfqpoint{0.839810in}{0.380606in}}%
\pgfpathlineto{\pgfqpoint{0.836427in}{0.380714in}}%
\pgfpathlineto{\pgfqpoint{0.832538in}{0.381977in}}%
\pgfpathlineto{\pgfqpoint{0.826726in}{0.381079in}}%
\pgfpathlineto{\pgfqpoint{0.824192in}{0.382154in}}%
\pgfpathlineto{\pgfqpoint{0.823213in}{0.380505in}}%
\pgfpathlineto{\pgfqpoint{0.820948in}{0.381922in}}%
\pgfpathlineto{\pgfqpoint{0.820829in}{0.384312in}}%
\pgfpathlineto{\pgfqpoint{0.818446in}{0.383242in}}%
\pgfpathlineto{\pgfqpoint{0.815104in}{0.384288in}}%
\pgfpathlineto{\pgfqpoint{0.814846in}{0.389826in}}%
\pgfpathlineto{\pgfqpoint{0.817666in}{0.391483in}}%
\pgfpathlineto{\pgfqpoint{0.821432in}{0.390956in}}%
\pgfpathlineto{\pgfqpoint{0.823183in}{0.389137in}}%
\pgfpathlineto{\pgfqpoint{0.825213in}{0.390029in}}%
\pgfpathlineto{\pgfqpoint{0.827821in}{0.388945in}}%
\pgfpathlineto{\pgfqpoint{0.828719in}{0.390525in}}%
\pgfpathlineto{\pgfqpoint{0.834229in}{0.391928in}}%
\pgfpathlineto{\pgfqpoint{0.831739in}{0.393028in}}%
\pgfpathlineto{\pgfqpoint{0.828778in}{0.392747in}}%
\pgfpathlineto{\pgfqpoint{0.825874in}{0.391476in}}%
\pgfpathlineto{\pgfqpoint{0.823671in}{0.395471in}}%
\pgfpathlineto{\pgfqpoint{0.823244in}{0.398390in}}%
\pgfpathlineto{\pgfqpoint{0.829643in}{0.403520in}}%
\pgfpathlineto{\pgfqpoint{0.834229in}{0.404849in}}%
\pgfpathlineto{\pgfqpoint{0.839609in}{0.408025in}}%
\pgfpathlineto{\pgfqpoint{0.842572in}{0.410824in}}%
\pgfpathlineto{\pgfqpoint{0.843558in}{0.416054in}}%
\pgfpathlineto{\pgfqpoint{0.846165in}{0.416317in}}%
\pgfpathlineto{\pgfqpoint{0.848769in}{0.415397in}}%
\pgfpathlineto{\pgfqpoint{0.854708in}{0.415990in}}%
\pgfpathlineto{\pgfqpoint{0.860476in}{0.415557in}}%
\pgfpathlineto{\pgfqpoint{0.860219in}{0.412566in}}%
\pgfpathlineto{\pgfqpoint{0.861939in}{0.410155in}}%
\pgfpathlineto{\pgfqpoint{0.867071in}{0.410118in}}%
\pgfpathlineto{\pgfqpoint{0.865539in}{0.416051in}}%
\pgfpathlineto{\pgfqpoint{0.867715in}{0.417422in}}%
\pgfpathlineto{\pgfqpoint{0.863045in}{0.420891in}}%
\pgfpathlineto{\pgfqpoint{0.861435in}{0.423132in}}%
\pgfpathlineto{\pgfqpoint{0.856270in}{0.423837in}}%
\pgfpathlineto{\pgfqpoint{0.852711in}{0.422439in}}%
\pgfpathlineto{\pgfqpoint{0.847311in}{0.423726in}}%
\pgfpathlineto{\pgfqpoint{0.842097in}{0.422862in}}%
\pgfpathlineto{\pgfqpoint{0.840286in}{0.418846in}}%
\pgfpathlineto{\pgfqpoint{0.839233in}{0.420379in}}%
\pgfpathlineto{\pgfqpoint{0.836425in}{0.420345in}}%
\pgfpathlineto{\pgfqpoint{0.831037in}{0.419128in}}%
\pgfpathlineto{\pgfqpoint{0.828067in}{0.415903in}}%
\pgfpathlineto{\pgfqpoint{0.825745in}{0.414910in}}%
\pgfpathlineto{\pgfqpoint{0.822974in}{0.414754in}}%
\pgfpathlineto{\pgfqpoint{0.821152in}{0.416144in}}%
\pgfpathlineto{\pgfqpoint{0.820535in}{0.411343in}}%
\pgfpathlineto{\pgfqpoint{0.816302in}{0.409041in}}%
\pgfpathlineto{\pgfqpoint{0.814299in}{0.409360in}}%
\pgfpathlineto{\pgfqpoint{0.812745in}{0.410826in}}%
\pgfpathlineto{\pgfqpoint{0.810009in}{0.407720in}}%
\pgfpathlineto{\pgfqpoint{0.807635in}{0.407238in}}%
\pgfpathlineto{\pgfqpoint{0.801657in}{0.409844in}}%
\pgfpathlineto{\pgfqpoint{0.800297in}{0.408770in}}%
\pgfpathlineto{\pgfqpoint{0.798270in}{0.408923in}}%
\pgfpathlineto{\pgfqpoint{0.797142in}{0.406932in}}%
\pgfpathlineto{\pgfqpoint{0.792995in}{0.408544in}}%
\pgfpathlineto{\pgfqpoint{0.789072in}{0.405423in}}%
\pgfpathlineto{\pgfqpoint{0.786281in}{0.404755in}}%
\pgfpathlineto{\pgfqpoint{0.787696in}{0.401710in}}%
\pgfpathlineto{\pgfqpoint{0.790829in}{0.398878in}}%
\pgfpathlineto{\pgfqpoint{0.793089in}{0.393058in}}%
\pgfpathlineto{\pgfqpoint{0.793097in}{0.390258in}}%
\pgfpathlineto{\pgfqpoint{0.790804in}{0.391968in}}%
\pgfpathlineto{\pgfqpoint{0.788860in}{0.389307in}}%
\pgfpathlineto{\pgfqpoint{0.784734in}{0.392394in}}%
\pgfpathlineto{\pgfqpoint{0.783322in}{0.390490in}}%
\pgfpathlineto{\pgfqpoint{0.777778in}{0.394690in}}%
\pgfpathlineto{\pgfqpoint{0.774053in}{0.397548in}}%
\pgfpathlineto{\pgfqpoint{0.774891in}{0.399921in}}%
\pgfpathlineto{\pgfqpoint{0.780586in}{0.407291in}}%
\pgfpathlineto{\pgfqpoint{0.780011in}{0.407770in}}%
\pgfpathlineto{\pgfqpoint{0.782927in}{0.411601in}}%
\pgfpathlineto{\pgfqpoint{0.786705in}{0.408749in}}%
\pgfpathlineto{\pgfqpoint{0.789544in}{0.412576in}}%
\pgfpathlineto{\pgfqpoint{0.792835in}{0.410153in}}%
\pgfpathlineto{\pgfqpoint{0.794266in}{0.412104in}}%
\pgfpathlineto{\pgfqpoint{0.796156in}{0.410686in}}%
\pgfpathlineto{\pgfqpoint{0.799058in}{0.414546in}}%
\pgfpathlineto{\pgfqpoint{0.800969in}{0.413117in}}%
\pgfpathlineto{\pgfqpoint{0.802441in}{0.415049in}}%
\pgfpathlineto{\pgfqpoint{0.803852in}{0.413980in}}%
\pgfpathlineto{\pgfqpoint{0.809404in}{0.421390in}}%
\pgfpathlineto{\pgfqpoint{0.810796in}{0.420377in}}%
\pgfpathlineto{\pgfqpoint{0.816621in}{0.428064in}}%
\pgfpathlineto{\pgfqpoint{0.818084in}{0.426988in}}%
\pgfpathlineto{\pgfqpoint{0.823896in}{0.434760in}}%
\pgfpathlineto{\pgfqpoint{0.824868in}{0.436985in}}%
\pgfpathlineto{\pgfqpoint{0.829091in}{0.442571in}}%
\pgfpathlineto{\pgfqpoint{0.830174in}{0.445346in}}%
\pgfpathlineto{\pgfqpoint{0.833858in}{0.450233in}}%
\pgfpathlineto{\pgfqpoint{0.837771in}{0.447122in}}%
\pgfpathlineto{\pgfqpoint{0.838124in}{0.445034in}}%
\pgfpathlineto{\pgfqpoint{0.853197in}{0.465288in}}%
\pgfpathlineto{\pgfqpoint{0.860358in}{0.474731in}}%
\pgfpathlineto{\pgfqpoint{0.871589in}{0.466101in}}%
\pgfpathlineto{\pgfqpoint{0.872819in}{0.467673in}}%
\pgfpathlineto{\pgfqpoint{0.896369in}{0.463780in}}%
\pgfpathlineto{\pgfqpoint{0.905818in}{0.462274in}}%
\pgfpathlineto{\pgfqpoint{0.921474in}{0.451616in}}%
\pgfpathlineto{\pgfqpoint{0.923645in}{0.454885in}}%
\pgfpathlineto{\pgfqpoint{0.927328in}{0.452402in}}%
\pgfpathlineto{\pgfqpoint{0.940211in}{0.444350in}}%
\pgfpathlineto{\pgfqpoint{0.933697in}{0.433889in}}%
\pgfpathlineto{\pgfqpoint{0.931589in}{0.429229in}}%
\pgfpathlineto{\pgfqpoint{0.922510in}{0.414648in}}%
\pgfpathlineto{\pgfqpoint{0.916427in}{0.418626in}}%
\pgfpathlineto{\pgfqpoint{0.915345in}{0.416446in}}%
\pgfpathlineto{\pgfqpoint{0.905234in}{0.400269in}}%
\pgfpathlineto{\pgfqpoint{0.902337in}{0.402174in}}%
\pgfpathlineto{\pgfqpoint{0.901579in}{0.401011in}}%
\pgfpathlineto{\pgfqpoint{0.887640in}{0.410227in}}%
\pgfpathlineto{\pgfqpoint{0.883377in}{0.404310in}}%
\pgfpathlineto{\pgfqpoint{0.878600in}{0.397005in}}%
\pgfpathlineto{\pgfqpoint{0.875473in}{0.398948in}}%
\pgfpathlineto{\pgfqpoint{0.873744in}{0.396362in}}%
\pgfpathlineto{\pgfqpoint{0.874767in}{0.395699in}}%
\pgfpathlineto{\pgfqpoint{0.869845in}{0.388628in}}%
\pgfpathlineto{\pgfqpoint{0.871019in}{0.387827in}}%
\pgfpathlineto{\pgfqpoint{0.866480in}{0.381088in}}%
\pgfpathclose%
\pgfusepath{fill}%
\end{pgfscope}%
\begin{pgfscope}%
\pgfpathrectangle{\pgfqpoint{0.100000in}{0.100000in}}{\pgfqpoint{3.420221in}{2.189500in}}%
\pgfusepath{clip}%
\pgfsetbuttcap%
\pgfsetmiterjoin%
\definecolor{currentfill}{rgb}{0.000000,0.776471,0.611765}%
\pgfsetfillcolor{currentfill}%
\pgfsetlinewidth{0.000000pt}%
\definecolor{currentstroke}{rgb}{0.000000,0.000000,0.000000}%
\pgfsetstrokecolor{currentstroke}%
\pgfsetstrokeopacity{0.000000}%
\pgfsetdash{}{0pt}%
\pgfpathmoveto{\pgfqpoint{2.754925in}{1.168084in}}%
\pgfpathlineto{\pgfqpoint{2.761038in}{1.171140in}}%
\pgfpathlineto{\pgfqpoint{2.758610in}{1.175757in}}%
\pgfpathlineto{\pgfqpoint{2.769107in}{1.183255in}}%
\pgfpathlineto{\pgfqpoint{2.761543in}{1.194371in}}%
\pgfpathlineto{\pgfqpoint{2.756011in}{1.199988in}}%
\pgfpathlineto{\pgfqpoint{2.773866in}{1.220848in}}%
\pgfpathlineto{\pgfqpoint{2.776519in}{1.219287in}}%
\pgfpathlineto{\pgfqpoint{2.776238in}{1.213801in}}%
\pgfpathlineto{\pgfqpoint{2.778873in}{1.207287in}}%
\pgfpathlineto{\pgfqpoint{2.787570in}{1.203491in}}%
\pgfpathlineto{\pgfqpoint{2.794605in}{1.198401in}}%
\pgfpathlineto{\pgfqpoint{2.801838in}{1.199960in}}%
\pgfpathlineto{\pgfqpoint{2.811229in}{1.209761in}}%
\pgfpathlineto{\pgfqpoint{2.816028in}{1.224152in}}%
\pgfpathlineto{\pgfqpoint{2.818035in}{1.225335in}}%
\pgfpathlineto{\pgfqpoint{2.818904in}{1.230411in}}%
\pgfpathlineto{\pgfqpoint{2.824652in}{1.232270in}}%
\pgfpathlineto{\pgfqpoint{2.840274in}{1.222647in}}%
\pgfpathlineto{\pgfqpoint{2.841829in}{1.216619in}}%
\pgfpathlineto{\pgfqpoint{2.832935in}{1.209806in}}%
\pgfpathlineto{\pgfqpoint{2.843841in}{1.201805in}}%
\pgfpathlineto{\pgfqpoint{2.839730in}{1.198703in}}%
\pgfpathlineto{\pgfqpoint{2.842953in}{1.192554in}}%
\pgfpathlineto{\pgfqpoint{2.853723in}{1.182986in}}%
\pgfpathlineto{\pgfqpoint{2.835866in}{1.174412in}}%
\pgfpathlineto{\pgfqpoint{2.833766in}{1.171524in}}%
\pgfpathlineto{\pgfqpoint{2.824150in}{1.169603in}}%
\pgfpathlineto{\pgfqpoint{2.818751in}{1.179126in}}%
\pgfpathlineto{\pgfqpoint{2.810971in}{1.186335in}}%
\pgfpathlineto{\pgfqpoint{2.795226in}{1.181477in}}%
\pgfpathlineto{\pgfqpoint{2.789038in}{1.178342in}}%
\pgfpathlineto{\pgfqpoint{2.785778in}{1.173488in}}%
\pgfpathlineto{\pgfqpoint{2.774670in}{1.169257in}}%
\pgfpathlineto{\pgfqpoint{2.773721in}{1.164860in}}%
\pgfpathlineto{\pgfqpoint{2.761092in}{1.156303in}}%
\pgfpathclose%
\pgfusepath{fill}%
\end{pgfscope}%
\begin{pgfscope}%
\pgfpathrectangle{\pgfqpoint{0.100000in}{0.100000in}}{\pgfqpoint{3.420221in}{2.189500in}}%
\pgfusepath{clip}%
\pgfsetbuttcap%
\pgfsetmiterjoin%
\definecolor{currentfill}{rgb}{0.000000,0.258824,0.870588}%
\pgfsetfillcolor{currentfill}%
\pgfsetlinewidth{0.000000pt}%
\definecolor{currentstroke}{rgb}{0.000000,0.000000,0.000000}%
\pgfsetstrokecolor{currentstroke}%
\pgfsetstrokeopacity{0.000000}%
\pgfsetdash{}{0pt}%
\pgfpathmoveto{\pgfqpoint{1.772202in}{1.906702in}}%
\pgfpathlineto{\pgfqpoint{1.770893in}{1.880507in}}%
\pgfpathlineto{\pgfqpoint{1.772681in}{1.880434in}}%
\pgfpathlineto{\pgfqpoint{1.771221in}{1.854064in}}%
\pgfpathlineto{\pgfqpoint{1.738489in}{1.855910in}}%
\pgfpathlineto{\pgfqpoint{1.740281in}{1.882184in}}%
\pgfpathlineto{\pgfqpoint{1.738251in}{1.882303in}}%
\pgfpathlineto{\pgfqpoint{1.739775in}{1.908408in}}%
\pgfpathclose%
\pgfusepath{fill}%
\end{pgfscope}%
\begin{pgfscope}%
\pgfpathrectangle{\pgfqpoint{0.100000in}{0.100000in}}{\pgfqpoint{3.420221in}{2.189500in}}%
\pgfusepath{clip}%
\pgfsetbuttcap%
\pgfsetmiterjoin%
\definecolor{currentfill}{rgb}{0.000000,0.415686,0.792157}%
\pgfsetfillcolor{currentfill}%
\pgfsetlinewidth{0.000000pt}%
\definecolor{currentstroke}{rgb}{0.000000,0.000000,0.000000}%
\pgfsetstrokecolor{currentstroke}%
\pgfsetstrokeopacity{0.000000}%
\pgfsetdash{}{0pt}%
\pgfpathmoveto{\pgfqpoint{1.643444in}{1.099348in}}%
\pgfpathlineto{\pgfqpoint{1.700635in}{1.095898in}}%
\pgfpathlineto{\pgfqpoint{1.699001in}{1.062637in}}%
\pgfpathlineto{\pgfqpoint{1.666321in}{1.064596in}}%
\pgfpathlineto{\pgfqpoint{1.600767in}{1.068907in}}%
\pgfpathlineto{\pgfqpoint{1.603165in}{1.102134in}}%
\pgfpathclose%
\pgfusepath{fill}%
\end{pgfscope}%
\begin{pgfscope}%
\pgfpathrectangle{\pgfqpoint{0.100000in}{0.100000in}}{\pgfqpoint{3.420221in}{2.189500in}}%
\pgfusepath{clip}%
\pgfsetbuttcap%
\pgfsetmiterjoin%
\definecolor{currentfill}{rgb}{0.000000,0.403922,0.798039}%
\pgfsetfillcolor{currentfill}%
\pgfsetlinewidth{0.000000pt}%
\definecolor{currentstroke}{rgb}{0.000000,0.000000,0.000000}%
\pgfsetstrokecolor{currentstroke}%
\pgfsetstrokeopacity{0.000000}%
\pgfsetdash{}{0pt}%
\pgfpathmoveto{\pgfqpoint{2.484750in}{1.554511in}}%
\pgfpathlineto{\pgfqpoint{2.486386in}{1.576531in}}%
\pgfpathlineto{\pgfqpoint{2.484439in}{1.591531in}}%
\pgfpathlineto{\pgfqpoint{2.480583in}{1.605246in}}%
\pgfpathlineto{\pgfqpoint{2.469716in}{1.624828in}}%
\pgfpathlineto{\pgfqpoint{2.462467in}{1.640964in}}%
\pgfpathlineto{\pgfqpoint{2.461912in}{1.647254in}}%
\pgfpathlineto{\pgfqpoint{2.466224in}{1.658370in}}%
\pgfpathlineto{\pgfqpoint{2.487428in}{1.660113in}}%
\pgfpathlineto{\pgfqpoint{2.513143in}{1.662713in}}%
\pgfpathlineto{\pgfqpoint{2.515852in}{1.636769in}}%
\pgfpathlineto{\pgfqpoint{2.541751in}{1.639369in}}%
\pgfpathlineto{\pgfqpoint{2.554700in}{1.640793in}}%
\pgfpathlineto{\pgfqpoint{2.558113in}{1.614925in}}%
\pgfpathlineto{\pgfqpoint{2.561011in}{1.588939in}}%
\pgfpathlineto{\pgfqpoint{2.548030in}{1.587505in}}%
\pgfpathlineto{\pgfqpoint{2.522223in}{1.584549in}}%
\pgfpathlineto{\pgfqpoint{2.525050in}{1.558657in}}%
\pgfpathclose%
\pgfusepath{fill}%
\end{pgfscope}%
\begin{pgfscope}%
\pgfpathrectangle{\pgfqpoint{0.100000in}{0.100000in}}{\pgfqpoint{3.420221in}{2.189500in}}%
\pgfusepath{clip}%
\pgfsetbuttcap%
\pgfsetmiterjoin%
\definecolor{currentfill}{rgb}{0.000000,0.254902,0.872549}%
\pgfsetfillcolor{currentfill}%
\pgfsetlinewidth{0.000000pt}%
\definecolor{currentstroke}{rgb}{0.000000,0.000000,0.000000}%
\pgfsetstrokecolor{currentstroke}%
\pgfsetstrokeopacity{0.000000}%
\pgfsetdash{}{0pt}%
\pgfpathmoveto{\pgfqpoint{1.093336in}{1.422040in}}%
\pgfpathlineto{\pgfqpoint{1.082517in}{1.424000in}}%
\pgfpathlineto{\pgfqpoint{1.083058in}{1.426908in}}%
\pgfpathlineto{\pgfqpoint{1.075183in}{1.437137in}}%
\pgfpathlineto{\pgfqpoint{1.077995in}{1.445072in}}%
\pgfpathlineto{\pgfqpoint{1.075510in}{1.453491in}}%
\pgfpathlineto{\pgfqpoint{1.066969in}{1.456748in}}%
\pgfpathlineto{\pgfqpoint{1.059590in}{1.470024in}}%
\pgfpathlineto{\pgfqpoint{1.063893in}{1.476679in}}%
\pgfpathlineto{\pgfqpoint{1.052700in}{1.477124in}}%
\pgfpathlineto{\pgfqpoint{1.038853in}{1.471193in}}%
\pgfpathlineto{\pgfqpoint{1.036407in}{1.475513in}}%
\pgfpathlineto{\pgfqpoint{1.026582in}{1.476974in}}%
\pgfpathlineto{\pgfqpoint{1.024958in}{1.462399in}}%
\pgfpathlineto{\pgfqpoint{1.026340in}{1.455124in}}%
\pgfpathlineto{\pgfqpoint{1.022099in}{1.443683in}}%
\pgfpathlineto{\pgfqpoint{1.011444in}{1.437538in}}%
\pgfpathlineto{\pgfqpoint{0.985709in}{1.442499in}}%
\pgfpathlineto{\pgfqpoint{0.915423in}{1.457160in}}%
\pgfpathlineto{\pgfqpoint{0.932755in}{1.537283in}}%
\pgfpathlineto{\pgfqpoint{1.001627in}{1.523152in}}%
\pgfpathlineto{\pgfqpoint{1.019705in}{1.525441in}}%
\pgfpathlineto{\pgfqpoint{1.037555in}{1.541841in}}%
\pgfpathlineto{\pgfqpoint{1.049135in}{1.539609in}}%
\pgfpathlineto{\pgfqpoint{1.057759in}{1.544217in}}%
\pgfpathlineto{\pgfqpoint{1.064315in}{1.539244in}}%
\pgfpathlineto{\pgfqpoint{1.068615in}{1.542100in}}%
\pgfpathlineto{\pgfqpoint{1.078948in}{1.540654in}}%
\pgfpathlineto{\pgfqpoint{1.082482in}{1.535352in}}%
\pgfpathlineto{\pgfqpoint{1.089637in}{1.531640in}}%
\pgfpathlineto{\pgfqpoint{1.089498in}{1.525118in}}%
\pgfpathlineto{\pgfqpoint{1.092376in}{1.518376in}}%
\pgfpathlineto{\pgfqpoint{1.102376in}{1.523354in}}%
\pgfpathlineto{\pgfqpoint{1.099025in}{1.504628in}}%
\pgfpathlineto{\pgfqpoint{1.129408in}{1.499057in}}%
\pgfpathlineto{\pgfqpoint{1.157331in}{1.494524in}}%
\pgfpathlineto{\pgfqpoint{1.155070in}{1.480949in}}%
\pgfpathlineto{\pgfqpoint{1.145712in}{1.482426in}}%
\pgfpathlineto{\pgfqpoint{1.140808in}{1.484998in}}%
\pgfpathlineto{\pgfqpoint{1.133812in}{1.482595in}}%
\pgfpathlineto{\pgfqpoint{1.117404in}{1.481801in}}%
\pgfpathlineto{\pgfqpoint{1.112235in}{1.483241in}}%
\pgfpathlineto{\pgfqpoint{1.102917in}{1.479593in}}%
\pgfpathclose%
\pgfusepath{fill}%
\end{pgfscope}%
\begin{pgfscope}%
\pgfpathrectangle{\pgfqpoint{0.100000in}{0.100000in}}{\pgfqpoint{3.420221in}{2.189500in}}%
\pgfusepath{clip}%
\pgfsetbuttcap%
\pgfsetmiterjoin%
\definecolor{currentfill}{rgb}{0.000000,0.133333,0.933333}%
\pgfsetfillcolor{currentfill}%
\pgfsetlinewidth{0.000000pt}%
\definecolor{currentstroke}{rgb}{0.000000,0.000000,0.000000}%
\pgfsetstrokecolor{currentstroke}%
\pgfsetstrokeopacity{0.000000}%
\pgfsetdash{}{0pt}%
\pgfpathmoveto{\pgfqpoint{1.954775in}{1.408853in}}%
\pgfpathlineto{\pgfqpoint{1.919043in}{1.409319in}}%
\pgfpathlineto{\pgfqpoint{1.919370in}{1.428885in}}%
\pgfpathlineto{\pgfqpoint{1.894224in}{1.429423in}}%
\pgfpathlineto{\pgfqpoint{1.895456in}{1.481643in}}%
\pgfpathlineto{\pgfqpoint{1.921296in}{1.481092in}}%
\pgfpathlineto{\pgfqpoint{1.921199in}{1.476739in}}%
\pgfpathlineto{\pgfqpoint{1.939287in}{1.476419in}}%
\pgfpathlineto{\pgfqpoint{1.942581in}{1.484796in}}%
\pgfpathlineto{\pgfqpoint{1.938486in}{1.490187in}}%
\pgfpathlineto{\pgfqpoint{1.964532in}{1.489705in}}%
\pgfpathlineto{\pgfqpoint{1.970982in}{1.489681in}}%
\pgfpathlineto{\pgfqpoint{1.970816in}{1.470092in}}%
\pgfpathlineto{\pgfqpoint{1.974130in}{1.462854in}}%
\pgfpathlineto{\pgfqpoint{1.993414in}{1.462723in}}%
\pgfpathlineto{\pgfqpoint{1.993314in}{1.436796in}}%
\pgfpathlineto{\pgfqpoint{1.980478in}{1.436807in}}%
\pgfpathlineto{\pgfqpoint{1.980301in}{1.417509in}}%
\pgfpathlineto{\pgfqpoint{1.956026in}{1.417680in}}%
\pgfpathclose%
\pgfusepath{fill}%
\end{pgfscope}%
\begin{pgfscope}%
\pgfpathrectangle{\pgfqpoint{0.100000in}{0.100000in}}{\pgfqpoint{3.420221in}{2.189500in}}%
\pgfusepath{clip}%
\pgfsetbuttcap%
\pgfsetmiterjoin%
\definecolor{currentfill}{rgb}{0.000000,0.282353,0.858824}%
\pgfsetfillcolor{currentfill}%
\pgfsetlinewidth{0.000000pt}%
\definecolor{currentstroke}{rgb}{0.000000,0.000000,0.000000}%
\pgfsetstrokecolor{currentstroke}%
\pgfsetstrokeopacity{0.000000}%
\pgfsetdash{}{0pt}%
\pgfpathmoveto{\pgfqpoint{2.179344in}{1.641708in}}%
\pgfpathlineto{\pgfqpoint{2.180284in}{1.615699in}}%
\pgfpathlineto{\pgfqpoint{2.141137in}{1.614429in}}%
\pgfpathlineto{\pgfqpoint{2.140428in}{1.639493in}}%
\pgfpathlineto{\pgfqpoint{2.127337in}{1.640225in}}%
\pgfpathlineto{\pgfqpoint{2.108054in}{1.639769in}}%
\pgfpathlineto{\pgfqpoint{2.107400in}{1.686648in}}%
\pgfpathlineto{\pgfqpoint{2.107323in}{1.689965in}}%
\pgfpathlineto{\pgfqpoint{2.120524in}{1.692360in}}%
\pgfpathlineto{\pgfqpoint{2.120340in}{1.698846in}}%
\pgfpathlineto{\pgfqpoint{2.129435in}{1.698288in}}%
\pgfpathlineto{\pgfqpoint{2.135150in}{1.695118in}}%
\pgfpathlineto{\pgfqpoint{2.146061in}{1.692855in}}%
\pgfpathlineto{\pgfqpoint{2.151714in}{1.685307in}}%
\pgfpathlineto{\pgfqpoint{2.160322in}{1.682886in}}%
\pgfpathlineto{\pgfqpoint{2.167531in}{1.677916in}}%
\pgfpathlineto{\pgfqpoint{2.171519in}{1.667473in}}%
\pgfpathlineto{\pgfqpoint{2.159599in}{1.667078in}}%
\pgfpathlineto{\pgfqpoint{2.160439in}{1.641074in}}%
\pgfpathclose%
\pgfusepath{fill}%
\end{pgfscope}%
\begin{pgfscope}%
\pgfpathrectangle{\pgfqpoint{0.100000in}{0.100000in}}{\pgfqpoint{3.420221in}{2.189500in}}%
\pgfusepath{clip}%
\pgfsetbuttcap%
\pgfsetmiterjoin%
\definecolor{currentfill}{rgb}{0.000000,0.207843,0.896078}%
\pgfsetfillcolor{currentfill}%
\pgfsetlinewidth{0.000000pt}%
\definecolor{currentstroke}{rgb}{0.000000,0.000000,0.000000}%
\pgfsetstrokecolor{currentstroke}%
\pgfsetstrokeopacity{0.000000}%
\pgfsetdash{}{0pt}%
\pgfpathmoveto{\pgfqpoint{1.848459in}{1.667117in}}%
\pgfpathlineto{\pgfqpoint{1.900226in}{1.665818in}}%
\pgfpathlineto{\pgfqpoint{1.899719in}{1.639679in}}%
\pgfpathlineto{\pgfqpoint{1.860708in}{1.640699in}}%
\pgfpathlineto{\pgfqpoint{1.847550in}{1.641167in}}%
\pgfpathclose%
\pgfusepath{fill}%
\end{pgfscope}%
\begin{pgfscope}%
\pgfpathrectangle{\pgfqpoint{0.100000in}{0.100000in}}{\pgfqpoint{3.420221in}{2.189500in}}%
\pgfusepath{clip}%
\pgfsetbuttcap%
\pgfsetmiterjoin%
\definecolor{currentfill}{rgb}{0.000000,0.207843,0.896078}%
\pgfsetfillcolor{currentfill}%
\pgfsetlinewidth{0.000000pt}%
\definecolor{currentstroke}{rgb}{0.000000,0.000000,0.000000}%
\pgfsetstrokecolor{currentstroke}%
\pgfsetstrokeopacity{0.000000}%
\pgfsetdash{}{0pt}%
\pgfpathmoveto{\pgfqpoint{1.982601in}{1.349938in}}%
\pgfpathlineto{\pgfqpoint{1.982509in}{1.323938in}}%
\pgfpathlineto{\pgfqpoint{1.969567in}{1.324003in}}%
\pgfpathlineto{\pgfqpoint{1.956599in}{1.324101in}}%
\pgfpathlineto{\pgfqpoint{1.956522in}{1.317588in}}%
\pgfpathlineto{\pgfqpoint{1.942281in}{1.317735in}}%
\pgfpathlineto{\pgfqpoint{1.930558in}{1.317893in}}%
\pgfpathlineto{\pgfqpoint{1.930990in}{1.350438in}}%
\pgfpathclose%
\pgfusepath{fill}%
\end{pgfscope}%
\begin{pgfscope}%
\pgfpathrectangle{\pgfqpoint{0.100000in}{0.100000in}}{\pgfqpoint{3.420221in}{2.189500in}}%
\pgfusepath{clip}%
\pgfsetbuttcap%
\pgfsetmiterjoin%
\definecolor{currentfill}{rgb}{0.000000,0.317647,0.841176}%
\pgfsetfillcolor{currentfill}%
\pgfsetlinewidth{0.000000pt}%
\definecolor{currentstroke}{rgb}{0.000000,0.000000,0.000000}%
\pgfsetstrokecolor{currentstroke}%
\pgfsetstrokeopacity{0.000000}%
\pgfsetdash{}{0pt}%
\pgfpathmoveto{\pgfqpoint{2.294226in}{1.657377in}}%
\pgfpathlineto{\pgfqpoint{2.295635in}{1.631881in}}%
\pgfpathlineto{\pgfqpoint{2.285551in}{1.631176in}}%
\pgfpathlineto{\pgfqpoint{2.280622in}{1.641596in}}%
\pgfpathlineto{\pgfqpoint{2.275283in}{1.647074in}}%
\pgfpathlineto{\pgfqpoint{2.275239in}{1.653732in}}%
\pgfpathlineto{\pgfqpoint{2.270513in}{1.664213in}}%
\pgfpathlineto{\pgfqpoint{2.276646in}{1.676494in}}%
\pgfpathlineto{\pgfqpoint{2.254616in}{1.675253in}}%
\pgfpathlineto{\pgfqpoint{2.253718in}{1.688461in}}%
\pgfpathlineto{\pgfqpoint{2.227636in}{1.687026in}}%
\pgfpathlineto{\pgfqpoint{2.227347in}{1.693562in}}%
\pgfpathlineto{\pgfqpoint{2.220845in}{1.693291in}}%
\pgfpathlineto{\pgfqpoint{2.218087in}{1.752145in}}%
\pgfpathlineto{\pgfqpoint{2.217798in}{1.758721in}}%
\pgfpathlineto{\pgfqpoint{2.230839in}{1.759197in}}%
\pgfpathlineto{\pgfqpoint{2.264481in}{1.761226in}}%
\pgfpathlineto{\pgfqpoint{2.263712in}{1.774146in}}%
\pgfpathlineto{\pgfqpoint{2.296135in}{1.776165in}}%
\pgfpathlineto{\pgfqpoint{2.296723in}{1.769546in}}%
\pgfpathlineto{\pgfqpoint{2.298264in}{1.743355in}}%
\pgfpathlineto{\pgfqpoint{2.308979in}{1.743978in}}%
\pgfpathlineto{\pgfqpoint{2.311095in}{1.711121in}}%
\pgfpathlineto{\pgfqpoint{2.313127in}{1.678231in}}%
\pgfpathlineto{\pgfqpoint{2.293049in}{1.677160in}}%
\pgfpathclose%
\pgfusepath{fill}%
\end{pgfscope}%
\begin{pgfscope}%
\pgfpathrectangle{\pgfqpoint{0.100000in}{0.100000in}}{\pgfqpoint{3.420221in}{2.189500in}}%
\pgfusepath{clip}%
\pgfsetbuttcap%
\pgfsetmiterjoin%
\definecolor{currentfill}{rgb}{0.000000,0.768627,0.615686}%
\pgfsetfillcolor{currentfill}%
\pgfsetlinewidth{0.000000pt}%
\definecolor{currentstroke}{rgb}{0.000000,0.000000,0.000000}%
\pgfsetstrokecolor{currentstroke}%
\pgfsetstrokeopacity{0.000000}%
\pgfsetdash{}{0pt}%
\pgfpathmoveto{\pgfqpoint{0.509357in}{1.579952in}}%
\pgfpathlineto{\pgfqpoint{0.500034in}{1.578731in}}%
\pgfpathlineto{\pgfqpoint{0.494206in}{1.571252in}}%
\pgfpathlineto{\pgfqpoint{0.491100in}{1.572340in}}%
\pgfpathlineto{\pgfqpoint{0.483386in}{1.566845in}}%
\pgfpathlineto{\pgfqpoint{0.470128in}{1.570850in}}%
\pgfpathlineto{\pgfqpoint{0.468151in}{1.564704in}}%
\pgfpathlineto{\pgfqpoint{0.419194in}{1.580023in}}%
\pgfpathlineto{\pgfqpoint{0.423474in}{1.592812in}}%
\pgfpathlineto{\pgfqpoint{0.424225in}{1.605703in}}%
\pgfpathlineto{\pgfqpoint{0.426794in}{1.613035in}}%
\pgfpathlineto{\pgfqpoint{0.423337in}{1.616417in}}%
\pgfpathlineto{\pgfqpoint{0.424640in}{1.621297in}}%
\pgfpathlineto{\pgfqpoint{0.429934in}{1.625355in}}%
\pgfpathlineto{\pgfqpoint{0.435814in}{1.625621in}}%
\pgfpathlineto{\pgfqpoint{0.440206in}{1.628939in}}%
\pgfpathlineto{\pgfqpoint{0.450219in}{1.631697in}}%
\pgfpathlineto{\pgfqpoint{0.449506in}{1.638594in}}%
\pgfpathlineto{\pgfqpoint{0.455755in}{1.645066in}}%
\pgfpathlineto{\pgfqpoint{0.464920in}{1.658890in}}%
\pgfpathlineto{\pgfqpoint{0.468887in}{1.659928in}}%
\pgfpathlineto{\pgfqpoint{0.474232in}{1.671474in}}%
\pgfpathlineto{\pgfqpoint{0.537210in}{1.652351in}}%
\pgfpathlineto{\pgfqpoint{0.531898in}{1.632190in}}%
\pgfpathlineto{\pgfqpoint{0.521612in}{1.599241in}}%
\pgfpathlineto{\pgfqpoint{0.512340in}{1.602032in}}%
\pgfpathlineto{\pgfqpoint{0.511873in}{1.594676in}}%
\pgfpathlineto{\pgfqpoint{0.517411in}{1.590350in}}%
\pgfpathlineto{\pgfqpoint{0.514525in}{1.583265in}}%
\pgfpathclose%
\pgfusepath{fill}%
\end{pgfscope}%
\begin{pgfscope}%
\pgfpathrectangle{\pgfqpoint{0.100000in}{0.100000in}}{\pgfqpoint{3.420221in}{2.189500in}}%
\pgfusepath{clip}%
\pgfsetbuttcap%
\pgfsetmiterjoin%
\definecolor{currentfill}{rgb}{0.000000,0.239216,0.880392}%
\pgfsetfillcolor{currentfill}%
\pgfsetlinewidth{0.000000pt}%
\definecolor{currentstroke}{rgb}{0.000000,0.000000,0.000000}%
\pgfsetstrokecolor{currentstroke}%
\pgfsetstrokeopacity{0.000000}%
\pgfsetdash{}{0pt}%
\pgfpathmoveto{\pgfqpoint{2.553189in}{1.274098in}}%
\pgfpathlineto{\pgfqpoint{2.552572in}{1.279491in}}%
\pgfpathlineto{\pgfqpoint{2.545588in}{1.283144in}}%
\pgfpathlineto{\pgfqpoint{2.544925in}{1.289637in}}%
\pgfpathlineto{\pgfqpoint{2.549140in}{1.290097in}}%
\pgfpathlineto{\pgfqpoint{2.558061in}{1.297713in}}%
\pgfpathlineto{\pgfqpoint{2.572066in}{1.299273in}}%
\pgfpathlineto{\pgfqpoint{2.573957in}{1.282710in}}%
\pgfpathlineto{\pgfqpoint{2.583686in}{1.289084in}}%
\pgfpathlineto{\pgfqpoint{2.589723in}{1.282241in}}%
\pgfpathlineto{\pgfqpoint{2.582127in}{1.276486in}}%
\pgfpathlineto{\pgfqpoint{2.576744in}{1.275058in}}%
\pgfpathlineto{\pgfqpoint{2.571009in}{1.267726in}}%
\pgfpathlineto{\pgfqpoint{2.561806in}{1.268799in}}%
\pgfpathlineto{\pgfqpoint{2.561605in}{1.273482in}}%
\pgfpathclose%
\pgfusepath{fill}%
\end{pgfscope}%
\begin{pgfscope}%
\pgfpathrectangle{\pgfqpoint{0.100000in}{0.100000in}}{\pgfqpoint{3.420221in}{2.189500in}}%
\pgfusepath{clip}%
\pgfsetbuttcap%
\pgfsetmiterjoin%
\definecolor{currentfill}{rgb}{0.000000,0.129412,0.935294}%
\pgfsetfillcolor{currentfill}%
\pgfsetlinewidth{0.000000pt}%
\definecolor{currentstroke}{rgb}{0.000000,0.000000,0.000000}%
\pgfsetstrokecolor{currentstroke}%
\pgfsetstrokeopacity{0.000000}%
\pgfsetdash{}{0pt}%
\pgfpathmoveto{\pgfqpoint{1.817723in}{1.483763in}}%
\pgfpathlineto{\pgfqpoint{1.817016in}{1.457714in}}%
\pgfpathlineto{\pgfqpoint{1.791568in}{1.458730in}}%
\pgfpathlineto{\pgfqpoint{1.765780in}{1.459855in}}%
\pgfpathlineto{\pgfqpoint{1.766492in}{1.485831in}}%
\pgfpathlineto{\pgfqpoint{1.739953in}{1.487110in}}%
\pgfpathlineto{\pgfqpoint{1.741283in}{1.513010in}}%
\pgfpathlineto{\pgfqpoint{1.766442in}{1.511909in}}%
\pgfpathlineto{\pgfqpoint{1.792767in}{1.510793in}}%
\pgfpathlineto{\pgfqpoint{1.791810in}{1.484703in}}%
\pgfpathclose%
\pgfusepath{fill}%
\end{pgfscope}%
\begin{pgfscope}%
\pgfpathrectangle{\pgfqpoint{0.100000in}{0.100000in}}{\pgfqpoint{3.420221in}{2.189500in}}%
\pgfusepath{clip}%
\pgfsetbuttcap%
\pgfsetmiterjoin%
\definecolor{currentfill}{rgb}{0.000000,0.207843,0.896078}%
\pgfsetfillcolor{currentfill}%
\pgfsetlinewidth{0.000000pt}%
\definecolor{currentstroke}{rgb}{0.000000,0.000000,0.000000}%
\pgfsetstrokecolor{currentstroke}%
\pgfsetstrokeopacity{0.000000}%
\pgfsetdash{}{0pt}%
\pgfpathmoveto{\pgfqpoint{1.870822in}{1.527937in}}%
\pgfpathlineto{\pgfqpoint{1.890231in}{1.527418in}}%
\pgfpathlineto{\pgfqpoint{1.891017in}{1.558192in}}%
\pgfpathlineto{\pgfqpoint{1.902479in}{1.553657in}}%
\pgfpathlineto{\pgfqpoint{1.903076in}{1.582126in}}%
\pgfpathlineto{\pgfqpoint{1.922357in}{1.581778in}}%
\pgfpathlineto{\pgfqpoint{1.919199in}{1.579302in}}%
\pgfpathlineto{\pgfqpoint{1.919739in}{1.572387in}}%
\pgfpathlineto{\pgfqpoint{1.917514in}{1.568759in}}%
\pgfpathlineto{\pgfqpoint{1.954833in}{1.568286in}}%
\pgfpathlineto{\pgfqpoint{1.954558in}{1.542144in}}%
\pgfpathlineto{\pgfqpoint{1.961113in}{1.542111in}}%
\pgfpathlineto{\pgfqpoint{1.965036in}{1.535579in}}%
\pgfpathlineto{\pgfqpoint{1.964818in}{1.515815in}}%
\pgfpathlineto{\pgfqpoint{1.964532in}{1.489705in}}%
\pgfpathlineto{\pgfqpoint{1.938486in}{1.490187in}}%
\pgfpathlineto{\pgfqpoint{1.939074in}{1.498184in}}%
\pgfpathlineto{\pgfqpoint{1.935640in}{1.499098in}}%
\pgfpathlineto{\pgfqpoint{1.927364in}{1.512541in}}%
\pgfpathlineto{\pgfqpoint{1.926810in}{1.521132in}}%
\pgfpathlineto{\pgfqpoint{1.906258in}{1.521595in}}%
\pgfpathlineto{\pgfqpoint{1.900916in}{1.520671in}}%
\pgfpathlineto{\pgfqpoint{1.900574in}{1.507631in}}%
\pgfpathlineto{\pgfqpoint{1.889699in}{1.507861in}}%
\pgfpathlineto{\pgfqpoint{1.870278in}{1.508356in}}%
\pgfpathclose%
\pgfusepath{fill}%
\end{pgfscope}%
\begin{pgfscope}%
\pgfpathrectangle{\pgfqpoint{0.100000in}{0.100000in}}{\pgfqpoint{3.420221in}{2.189500in}}%
\pgfusepath{clip}%
\pgfsetbuttcap%
\pgfsetmiterjoin%
\definecolor{currentfill}{rgb}{0.000000,0.788235,0.605882}%
\pgfsetfillcolor{currentfill}%
\pgfsetlinewidth{0.000000pt}%
\definecolor{currentstroke}{rgb}{0.000000,0.000000,0.000000}%
\pgfsetstrokecolor{currentstroke}%
\pgfsetstrokeopacity{0.000000}%
\pgfsetdash{}{0pt}%
\pgfpathmoveto{\pgfqpoint{2.897367in}{0.872307in}}%
\pgfpathlineto{\pgfqpoint{2.893560in}{0.865395in}}%
\pgfpathlineto{\pgfqpoint{2.885003in}{0.874094in}}%
\pgfpathlineto{\pgfqpoint{2.870130in}{0.856407in}}%
\pgfpathlineto{\pgfqpoint{2.863299in}{0.858754in}}%
\pgfpathlineto{\pgfqpoint{2.859707in}{0.866329in}}%
\pgfpathlineto{\pgfqpoint{2.858215in}{0.876688in}}%
\pgfpathlineto{\pgfqpoint{2.852405in}{0.883023in}}%
\pgfpathlineto{\pgfqpoint{2.852261in}{0.888580in}}%
\pgfpathlineto{\pgfqpoint{2.848715in}{0.890873in}}%
\pgfpathlineto{\pgfqpoint{2.837770in}{0.894641in}}%
\pgfpathlineto{\pgfqpoint{2.833606in}{0.900247in}}%
\pgfpathlineto{\pgfqpoint{2.854258in}{0.925569in}}%
\pgfpathlineto{\pgfqpoint{2.859091in}{0.925505in}}%
\pgfpathlineto{\pgfqpoint{2.866576in}{0.921396in}}%
\pgfpathlineto{\pgfqpoint{2.881371in}{0.916643in}}%
\pgfpathlineto{\pgfqpoint{2.893603in}{0.907760in}}%
\pgfpathlineto{\pgfqpoint{2.877600in}{0.893793in}}%
\pgfpathlineto{\pgfqpoint{2.885005in}{0.885412in}}%
\pgfpathlineto{\pgfqpoint{2.891197in}{0.881189in}}%
\pgfpathclose%
\pgfusepath{fill}%
\end{pgfscope}%
\begin{pgfscope}%
\pgfpathrectangle{\pgfqpoint{0.100000in}{0.100000in}}{\pgfqpoint{3.420221in}{2.189500in}}%
\pgfusepath{clip}%
\pgfsetbuttcap%
\pgfsetmiterjoin%
\definecolor{currentfill}{rgb}{0.000000,0.521569,0.739216}%
\pgfsetfillcolor{currentfill}%
\pgfsetlinewidth{0.000000pt}%
\definecolor{currentstroke}{rgb}{0.000000,0.000000,0.000000}%
\pgfsetstrokecolor{currentstroke}%
\pgfsetstrokeopacity{0.000000}%
\pgfsetdash{}{0pt}%
\pgfpathmoveto{\pgfqpoint{0.996739in}{0.172056in}}%
\pgfpathlineto{\pgfqpoint{0.996879in}{0.175358in}}%
\pgfpathlineto{\pgfqpoint{0.999164in}{0.173563in}}%
\pgfpathclose%
\pgfusepath{fill}%
\end{pgfscope}%
\begin{pgfscope}%
\pgfpathrectangle{\pgfqpoint{0.100000in}{0.100000in}}{\pgfqpoint{3.420221in}{2.189500in}}%
\pgfusepath{clip}%
\pgfsetbuttcap%
\pgfsetmiterjoin%
\definecolor{currentfill}{rgb}{0.000000,0.521569,0.739216}%
\pgfsetfillcolor{currentfill}%
\pgfsetlinewidth{0.000000pt}%
\definecolor{currentstroke}{rgb}{0.000000,0.000000,0.000000}%
\pgfsetstrokecolor{currentstroke}%
\pgfsetstrokeopacity{0.000000}%
\pgfsetdash{}{0pt}%
\pgfpathmoveto{\pgfqpoint{0.993285in}{0.159291in}}%
\pgfpathlineto{\pgfqpoint{0.994105in}{0.161791in}}%
\pgfpathlineto{\pgfqpoint{0.995672in}{0.162042in}}%
\pgfpathlineto{\pgfqpoint{0.995077in}{0.158591in}}%
\pgfpathclose%
\pgfusepath{fill}%
\end{pgfscope}%
\begin{pgfscope}%
\pgfpathrectangle{\pgfqpoint{0.100000in}{0.100000in}}{\pgfqpoint{3.420221in}{2.189500in}}%
\pgfusepath{clip}%
\pgfsetbuttcap%
\pgfsetmiterjoin%
\definecolor{currentfill}{rgb}{0.000000,0.521569,0.739216}%
\pgfsetfillcolor{currentfill}%
\pgfsetlinewidth{0.000000pt}%
\definecolor{currentstroke}{rgb}{0.000000,0.000000,0.000000}%
\pgfsetstrokecolor{currentstroke}%
\pgfsetstrokeopacity{0.000000}%
\pgfsetdash{}{0pt}%
\pgfpathmoveto{\pgfqpoint{0.996598in}{0.142673in}}%
\pgfpathlineto{\pgfqpoint{0.995200in}{0.148205in}}%
\pgfpathlineto{\pgfqpoint{0.996049in}{0.149787in}}%
\pgfpathlineto{\pgfqpoint{0.994212in}{0.151227in}}%
\pgfpathlineto{\pgfqpoint{0.995859in}{0.154916in}}%
\pgfpathlineto{\pgfqpoint{0.995752in}{0.157225in}}%
\pgfpathlineto{\pgfqpoint{0.997783in}{0.158082in}}%
\pgfpathlineto{\pgfqpoint{0.998233in}{0.154747in}}%
\pgfpathlineto{\pgfqpoint{0.996410in}{0.153186in}}%
\pgfpathlineto{\pgfqpoint{0.997585in}{0.152080in}}%
\pgfpathlineto{\pgfqpoint{0.997610in}{0.146652in}}%
\pgfpathlineto{\pgfqpoint{1.000421in}{0.145954in}}%
\pgfpathclose%
\pgfusepath{fill}%
\end{pgfscope}%
\begin{pgfscope}%
\pgfpathrectangle{\pgfqpoint{0.100000in}{0.100000in}}{\pgfqpoint{3.420221in}{2.189500in}}%
\pgfusepath{clip}%
\pgfsetbuttcap%
\pgfsetmiterjoin%
\definecolor{currentfill}{rgb}{0.000000,0.521569,0.739216}%
\pgfsetfillcolor{currentfill}%
\pgfsetlinewidth{0.000000pt}%
\definecolor{currentstroke}{rgb}{0.000000,0.000000,0.000000}%
\pgfsetstrokecolor{currentstroke}%
\pgfsetstrokeopacity{0.000000}%
\pgfsetdash{}{0pt}%
\pgfpathmoveto{\pgfqpoint{1.021116in}{0.147916in}}%
\pgfpathlineto{\pgfqpoint{1.021319in}{0.145104in}}%
\pgfpathlineto{\pgfqpoint{1.020107in}{0.142486in}}%
\pgfpathlineto{\pgfqpoint{1.018763in}{0.142460in}}%
\pgfpathlineto{\pgfqpoint{1.017546in}{0.145423in}}%
\pgfpathlineto{\pgfqpoint{1.018884in}{0.146117in}}%
\pgfpathlineto{\pgfqpoint{1.019042in}{0.149031in}}%
\pgfpathclose%
\pgfusepath{fill}%
\end{pgfscope}%
\begin{pgfscope}%
\pgfpathrectangle{\pgfqpoint{0.100000in}{0.100000in}}{\pgfqpoint{3.420221in}{2.189500in}}%
\pgfusepath{clip}%
\pgfsetbuttcap%
\pgfsetmiterjoin%
\definecolor{currentfill}{rgb}{0.000000,0.521569,0.739216}%
\pgfsetfillcolor{currentfill}%
\pgfsetlinewidth{0.000000pt}%
\definecolor{currentstroke}{rgb}{0.000000,0.000000,0.000000}%
\pgfsetstrokecolor{currentstroke}%
\pgfsetstrokeopacity{0.000000}%
\pgfsetdash{}{0pt}%
\pgfpathmoveto{\pgfqpoint{1.015118in}{0.165753in}}%
\pgfpathlineto{\pgfqpoint{1.016941in}{0.161937in}}%
\pgfpathlineto{\pgfqpoint{1.018756in}{0.162209in}}%
\pgfpathlineto{\pgfqpoint{1.020191in}{0.164491in}}%
\pgfpathlineto{\pgfqpoint{1.019647in}{0.166703in}}%
\pgfpathlineto{\pgfqpoint{1.022191in}{0.167275in}}%
\pgfpathlineto{\pgfqpoint{1.022469in}{0.168543in}}%
\pgfpathlineto{\pgfqpoint{1.024595in}{0.169689in}}%
\pgfpathlineto{\pgfqpoint{1.027282in}{0.170030in}}%
\pgfpathlineto{\pgfqpoint{1.028725in}{0.166915in}}%
\pgfpathlineto{\pgfqpoint{1.030811in}{0.167866in}}%
\pgfpathlineto{\pgfqpoint{1.030590in}{0.169668in}}%
\pgfpathlineto{\pgfqpoint{1.033749in}{0.170216in}}%
\pgfpathlineto{\pgfqpoint{1.036326in}{0.169278in}}%
\pgfpathlineto{\pgfqpoint{1.038283in}{0.171601in}}%
\pgfpathlineto{\pgfqpoint{1.038404in}{0.174086in}}%
\pgfpathlineto{\pgfqpoint{1.041219in}{0.172037in}}%
\pgfpathlineto{\pgfqpoint{1.044056in}{0.167154in}}%
\pgfpathlineto{\pgfqpoint{1.044497in}{0.163903in}}%
\pgfpathlineto{\pgfqpoint{1.046522in}{0.161755in}}%
\pgfpathlineto{\pgfqpoint{1.048599in}{0.161592in}}%
\pgfpathlineto{\pgfqpoint{1.048918in}{0.158822in}}%
\pgfpathlineto{\pgfqpoint{1.047864in}{0.156322in}}%
\pgfpathlineto{\pgfqpoint{1.044149in}{0.153441in}}%
\pgfpathlineto{\pgfqpoint{1.043895in}{0.150944in}}%
\pgfpathlineto{\pgfqpoint{1.042699in}{0.148543in}}%
\pgfpathlineto{\pgfqpoint{1.042377in}{0.144280in}}%
\pgfpathlineto{\pgfqpoint{1.041325in}{0.140436in}}%
\pgfpathlineto{\pgfqpoint{1.038957in}{0.139000in}}%
\pgfpathlineto{\pgfqpoint{1.032962in}{0.133699in}}%
\pgfpathlineto{\pgfqpoint{1.030748in}{0.134196in}}%
\pgfpathlineto{\pgfqpoint{1.028300in}{0.132940in}}%
\pgfpathlineto{\pgfqpoint{1.026119in}{0.134622in}}%
\pgfpathlineto{\pgfqpoint{1.023917in}{0.134387in}}%
\pgfpathlineto{\pgfqpoint{1.025563in}{0.141854in}}%
\pgfpathlineto{\pgfqpoint{1.025436in}{0.144034in}}%
\pgfpathlineto{\pgfqpoint{1.028789in}{0.146368in}}%
\pgfpathlineto{\pgfqpoint{1.030608in}{0.145895in}}%
\pgfpathlineto{\pgfqpoint{1.030638in}{0.148840in}}%
\pgfpathlineto{\pgfqpoint{1.032506in}{0.152254in}}%
\pgfpathlineto{\pgfqpoint{1.034280in}{0.158976in}}%
\pgfpathlineto{\pgfqpoint{1.031886in}{0.158594in}}%
\pgfpathlineto{\pgfqpoint{1.032383in}{0.156123in}}%
\pgfpathlineto{\pgfqpoint{1.031384in}{0.152743in}}%
\pgfpathlineto{\pgfqpoint{1.027086in}{0.146968in}}%
\pgfpathlineto{\pgfqpoint{1.023683in}{0.145655in}}%
\pgfpathlineto{\pgfqpoint{1.023125in}{0.147762in}}%
\pgfpathlineto{\pgfqpoint{1.019935in}{0.150577in}}%
\pgfpathlineto{\pgfqpoint{1.017129in}{0.149924in}}%
\pgfpathlineto{\pgfqpoint{1.015342in}{0.148739in}}%
\pgfpathlineto{\pgfqpoint{1.016584in}{0.153561in}}%
\pgfpathlineto{\pgfqpoint{1.018424in}{0.155924in}}%
\pgfpathlineto{\pgfqpoint{1.020271in}{0.156386in}}%
\pgfpathlineto{\pgfqpoint{1.021486in}{0.158876in}}%
\pgfpathlineto{\pgfqpoint{1.023596in}{0.160464in}}%
\pgfpathlineto{\pgfqpoint{1.023648in}{0.162517in}}%
\pgfpathlineto{\pgfqpoint{1.026579in}{0.163184in}}%
\pgfpathlineto{\pgfqpoint{1.026545in}{0.165208in}}%
\pgfpathlineto{\pgfqpoint{1.023929in}{0.166623in}}%
\pgfpathlineto{\pgfqpoint{1.021226in}{0.165567in}}%
\pgfpathlineto{\pgfqpoint{1.022726in}{0.163840in}}%
\pgfpathlineto{\pgfqpoint{1.020338in}{0.160752in}}%
\pgfpathlineto{\pgfqpoint{1.016640in}{0.157635in}}%
\pgfpathlineto{\pgfqpoint{1.014718in}{0.161659in}}%
\pgfpathclose%
\pgfusepath{fill}%
\end{pgfscope}%
\begin{pgfscope}%
\pgfpathrectangle{\pgfqpoint{0.100000in}{0.100000in}}{\pgfqpoint{3.420221in}{2.189500in}}%
\pgfusepath{clip}%
\pgfsetbuttcap%
\pgfsetmiterjoin%
\definecolor{currentfill}{rgb}{0.000000,0.521569,0.739216}%
\pgfsetfillcolor{currentfill}%
\pgfsetlinewidth{0.000000pt}%
\definecolor{currentstroke}{rgb}{0.000000,0.000000,0.000000}%
\pgfsetstrokecolor{currentstroke}%
\pgfsetstrokeopacity{0.000000}%
\pgfsetdash{}{0pt}%
\pgfpathmoveto{\pgfqpoint{0.992807in}{0.166188in}}%
\pgfpathlineto{\pgfqpoint{0.990511in}{0.167159in}}%
\pgfpathlineto{\pgfqpoint{0.993762in}{0.169053in}}%
\pgfpathlineto{\pgfqpoint{0.994718in}{0.167062in}}%
\pgfpathlineto{\pgfqpoint{0.997959in}{0.167896in}}%
\pgfpathlineto{\pgfqpoint{0.998842in}{0.165988in}}%
\pgfpathlineto{\pgfqpoint{0.996293in}{0.165950in}}%
\pgfpathlineto{\pgfqpoint{0.994935in}{0.163760in}}%
\pgfpathlineto{\pgfqpoint{0.992797in}{0.162964in}}%
\pgfpathlineto{\pgfqpoint{0.991256in}{0.163793in}}%
\pgfpathclose%
\pgfusepath{fill}%
\end{pgfscope}%
\begin{pgfscope}%
\pgfpathrectangle{\pgfqpoint{0.100000in}{0.100000in}}{\pgfqpoint{3.420221in}{2.189500in}}%
\pgfusepath{clip}%
\pgfsetbuttcap%
\pgfsetmiterjoin%
\definecolor{currentfill}{rgb}{0.000000,0.521569,0.739216}%
\pgfsetfillcolor{currentfill}%
\pgfsetlinewidth{0.000000pt}%
\definecolor{currentstroke}{rgb}{0.000000,0.000000,0.000000}%
\pgfsetstrokecolor{currentstroke}%
\pgfsetstrokeopacity{0.000000}%
\pgfsetdash{}{0pt}%
\pgfpathmoveto{\pgfqpoint{1.001794in}{0.154652in}}%
\pgfpathlineto{\pgfqpoint{1.001882in}{0.156880in}}%
\pgfpathlineto{\pgfqpoint{0.997128in}{0.160446in}}%
\pgfpathlineto{\pgfqpoint{1.001069in}{0.162412in}}%
\pgfpathlineto{\pgfqpoint{1.000298in}{0.164317in}}%
\pgfpathlineto{\pgfqpoint{1.001245in}{0.166382in}}%
\pgfpathlineto{\pgfqpoint{0.999489in}{0.167141in}}%
\pgfpathlineto{\pgfqpoint{0.998389in}{0.169518in}}%
\pgfpathlineto{\pgfqpoint{1.001799in}{0.171871in}}%
\pgfpathlineto{\pgfqpoint{1.002193in}{0.174691in}}%
\pgfpathlineto{\pgfqpoint{1.004111in}{0.175593in}}%
\pgfpathlineto{\pgfqpoint{1.004054in}{0.178408in}}%
\pgfpathlineto{\pgfqpoint{1.001644in}{0.179346in}}%
\pgfpathlineto{\pgfqpoint{0.996142in}{0.178580in}}%
\pgfpathlineto{\pgfqpoint{0.997864in}{0.180678in}}%
\pgfpathlineto{\pgfqpoint{0.999980in}{0.181708in}}%
\pgfpathlineto{\pgfqpoint{1.002381in}{0.181825in}}%
\pgfpathlineto{\pgfqpoint{1.001969in}{0.185499in}}%
\pgfpathlineto{\pgfqpoint{1.003156in}{0.187842in}}%
\pgfpathlineto{\pgfqpoint{1.004982in}{0.187059in}}%
\pgfpathlineto{\pgfqpoint{1.009081in}{0.184172in}}%
\pgfpathlineto{\pgfqpoint{1.009123in}{0.180211in}}%
\pgfpathlineto{\pgfqpoint{1.007184in}{0.178684in}}%
\pgfpathlineto{\pgfqpoint{1.008015in}{0.175216in}}%
\pgfpathlineto{\pgfqpoint{1.009443in}{0.175478in}}%
\pgfpathlineto{\pgfqpoint{1.011211in}{0.173192in}}%
\pgfpathlineto{\pgfqpoint{1.012432in}{0.168917in}}%
\pgfpathlineto{\pgfqpoint{1.012159in}{0.166611in}}%
\pgfpathlineto{\pgfqpoint{1.010802in}{0.165392in}}%
\pgfpathlineto{\pgfqpoint{1.012656in}{0.163892in}}%
\pgfpathlineto{\pgfqpoint{1.012484in}{0.161528in}}%
\pgfpathlineto{\pgfqpoint{1.010747in}{0.161156in}}%
\pgfpathlineto{\pgfqpoint{1.009687in}{0.163425in}}%
\pgfpathlineto{\pgfqpoint{1.008318in}{0.161265in}}%
\pgfpathlineto{\pgfqpoint{1.011027in}{0.159187in}}%
\pgfpathlineto{\pgfqpoint{1.011033in}{0.157293in}}%
\pgfpathlineto{\pgfqpoint{1.012303in}{0.154368in}}%
\pgfpathlineto{\pgfqpoint{1.010807in}{0.150895in}}%
\pgfpathlineto{\pgfqpoint{1.013730in}{0.151868in}}%
\pgfpathlineto{\pgfqpoint{1.013203in}{0.149786in}}%
\pgfpathlineto{\pgfqpoint{1.011395in}{0.148520in}}%
\pgfpathlineto{\pgfqpoint{1.009947in}{0.149237in}}%
\pgfpathlineto{\pgfqpoint{1.008922in}{0.145927in}}%
\pgfpathlineto{\pgfqpoint{1.011477in}{0.146229in}}%
\pgfpathlineto{\pgfqpoint{1.009876in}{0.142915in}}%
\pgfpathlineto{\pgfqpoint{1.009354in}{0.140374in}}%
\pgfpathlineto{\pgfqpoint{1.008010in}{0.138779in}}%
\pgfpathlineto{\pgfqpoint{1.005237in}{0.139742in}}%
\pgfpathlineto{\pgfqpoint{1.004754in}{0.144770in}}%
\pgfpathlineto{\pgfqpoint{1.002987in}{0.148297in}}%
\pgfpathlineto{\pgfqpoint{1.002813in}{0.150522in}}%
\pgfpathlineto{\pgfqpoint{1.004574in}{0.154392in}}%
\pgfpathclose%
\pgfusepath{fill}%
\end{pgfscope}%
\begin{pgfscope}%
\pgfpathrectangle{\pgfqpoint{0.100000in}{0.100000in}}{\pgfqpoint{3.420221in}{2.189500in}}%
\pgfusepath{clip}%
\pgfsetbuttcap%
\pgfsetmiterjoin%
\definecolor{currentfill}{rgb}{0.000000,0.670588,0.664706}%
\pgfsetfillcolor{currentfill}%
\pgfsetlinewidth{0.000000pt}%
\definecolor{currentstroke}{rgb}{0.000000,0.000000,0.000000}%
\pgfsetstrokecolor{currentstroke}%
\pgfsetstrokeopacity{0.000000}%
\pgfsetdash{}{0pt}%
\pgfpathmoveto{\pgfqpoint{0.834795in}{2.044232in}}%
\pgfpathlineto{\pgfqpoint{0.828322in}{2.018891in}}%
\pgfpathlineto{\pgfqpoint{0.823294in}{2.010548in}}%
\pgfpathlineto{\pgfqpoint{0.812406in}{2.009276in}}%
\pgfpathlineto{\pgfqpoint{0.785388in}{2.016141in}}%
\pgfpathlineto{\pgfqpoint{0.754324in}{2.024708in}}%
\pgfpathlineto{\pgfqpoint{0.748980in}{2.021627in}}%
\pgfpathlineto{\pgfqpoint{0.746829in}{2.025999in}}%
\pgfpathlineto{\pgfqpoint{0.739642in}{2.021524in}}%
\pgfpathlineto{\pgfqpoint{0.726916in}{2.023792in}}%
\pgfpathlineto{\pgfqpoint{0.724324in}{2.028028in}}%
\pgfpathlineto{\pgfqpoint{0.727687in}{2.036842in}}%
\pgfpathlineto{\pgfqpoint{0.728787in}{2.047988in}}%
\pgfpathlineto{\pgfqpoint{0.727916in}{2.059083in}}%
\pgfpathlineto{\pgfqpoint{0.732427in}{2.068726in}}%
\pgfpathlineto{\pgfqpoint{0.734598in}{2.075981in}}%
\pgfpathlineto{\pgfqpoint{0.739753in}{2.078024in}}%
\pgfpathlineto{\pgfqpoint{0.743371in}{2.082741in}}%
\pgfpathlineto{\pgfqpoint{0.760341in}{2.078322in}}%
\pgfpathlineto{\pgfqpoint{0.762120in}{2.084615in}}%
\pgfpathlineto{\pgfqpoint{0.770125in}{2.089076in}}%
\pgfpathlineto{\pgfqpoint{0.773250in}{2.088207in}}%
\pgfpathlineto{\pgfqpoint{0.777689in}{2.096102in}}%
\pgfpathlineto{\pgfqpoint{0.783377in}{2.100236in}}%
\pgfpathlineto{\pgfqpoint{0.785130in}{2.106594in}}%
\pgfpathlineto{\pgfqpoint{0.791910in}{2.110578in}}%
\pgfpathlineto{\pgfqpoint{0.797880in}{2.108980in}}%
\pgfpathlineto{\pgfqpoint{0.797951in}{2.107511in}}%
\pgfpathlineto{\pgfqpoint{0.784172in}{2.057684in}}%
\pgfpathclose%
\pgfusepath{fill}%
\end{pgfscope}%
\begin{pgfscope}%
\pgfpathrectangle{\pgfqpoint{0.100000in}{0.100000in}}{\pgfqpoint{3.420221in}{2.189500in}}%
\pgfusepath{clip}%
\pgfsetbuttcap%
\pgfsetmiterjoin%
\definecolor{currentfill}{rgb}{0.000000,0.501961,0.749020}%
\pgfsetfillcolor{currentfill}%
\pgfsetlinewidth{0.000000pt}%
\definecolor{currentstroke}{rgb}{0.000000,0.000000,0.000000}%
\pgfsetstrokecolor{currentstroke}%
\pgfsetstrokeopacity{0.000000}%
\pgfsetdash{}{0pt}%
\pgfpathmoveto{\pgfqpoint{2.816952in}{1.360416in}}%
\pgfpathlineto{\pgfqpoint{2.813098in}{1.366811in}}%
\pgfpathlineto{\pgfqpoint{2.803000in}{1.373584in}}%
\pgfpathlineto{\pgfqpoint{2.805285in}{1.379664in}}%
\pgfpathlineto{\pgfqpoint{2.793219in}{1.378335in}}%
\pgfpathlineto{\pgfqpoint{2.790847in}{1.380001in}}%
\pgfpathlineto{\pgfqpoint{2.787649in}{1.387212in}}%
\pgfpathlineto{\pgfqpoint{2.786142in}{1.399268in}}%
\pgfpathlineto{\pgfqpoint{2.790497in}{1.399806in}}%
\pgfpathlineto{\pgfqpoint{2.787724in}{1.422292in}}%
\pgfpathlineto{\pgfqpoint{2.807256in}{1.424511in}}%
\pgfpathlineto{\pgfqpoint{2.827828in}{1.427788in}}%
\pgfpathlineto{\pgfqpoint{2.833025in}{1.395364in}}%
\pgfpathlineto{\pgfqpoint{2.838600in}{1.396246in}}%
\pgfpathlineto{\pgfqpoint{2.841325in}{1.390874in}}%
\pgfpathlineto{\pgfqpoint{2.836118in}{1.383833in}}%
\pgfpathlineto{\pgfqpoint{2.838583in}{1.379703in}}%
\pgfpathlineto{\pgfqpoint{2.836472in}{1.374185in}}%
\pgfpathlineto{\pgfqpoint{2.830577in}{1.374407in}}%
\pgfpathclose%
\pgfusepath{fill}%
\end{pgfscope}%
\begin{pgfscope}%
\pgfpathrectangle{\pgfqpoint{0.100000in}{0.100000in}}{\pgfqpoint{3.420221in}{2.189500in}}%
\pgfusepath{clip}%
\pgfsetbuttcap%
\pgfsetmiterjoin%
\definecolor{currentfill}{rgb}{0.000000,0.278431,0.860784}%
\pgfsetfillcolor{currentfill}%
\pgfsetlinewidth{0.000000pt}%
\definecolor{currentstroke}{rgb}{0.000000,0.000000,0.000000}%
\pgfsetstrokecolor{currentstroke}%
\pgfsetstrokeopacity{0.000000}%
\pgfsetdash{}{0pt}%
\pgfpathmoveto{\pgfqpoint{3.291027in}{1.613899in}}%
\pgfpathlineto{\pgfqpoint{3.292576in}{1.621702in}}%
\pgfpathlineto{\pgfqpoint{3.288616in}{1.639224in}}%
\pgfpathlineto{\pgfqpoint{3.281284in}{1.664602in}}%
\pgfpathlineto{\pgfqpoint{3.303598in}{1.671313in}}%
\pgfpathlineto{\pgfqpoint{3.305174in}{1.669132in}}%
\pgfpathlineto{\pgfqpoint{3.318313in}{1.681224in}}%
\pgfpathlineto{\pgfqpoint{3.322521in}{1.672082in}}%
\pgfpathlineto{\pgfqpoint{3.335318in}{1.659744in}}%
\pgfpathlineto{\pgfqpoint{3.340283in}{1.650836in}}%
\pgfpathlineto{\pgfqpoint{3.335898in}{1.648116in}}%
\pgfpathlineto{\pgfqpoint{3.336680in}{1.643015in}}%
\pgfpathlineto{\pgfqpoint{3.332262in}{1.638354in}}%
\pgfpathlineto{\pgfqpoint{3.317499in}{1.633086in}}%
\pgfpathlineto{\pgfqpoint{3.317045in}{1.643875in}}%
\pgfpathlineto{\pgfqpoint{3.313476in}{1.650663in}}%
\pgfpathlineto{\pgfqpoint{3.308977in}{1.645110in}}%
\pgfpathlineto{\pgfqpoint{3.308967in}{1.637702in}}%
\pgfpathlineto{\pgfqpoint{3.312140in}{1.631209in}}%
\pgfpathlineto{\pgfqpoint{3.310526in}{1.622893in}}%
\pgfpathlineto{\pgfqpoint{3.306526in}{1.622002in}}%
\pgfpathclose%
\pgfusepath{fill}%
\end{pgfscope}%
\begin{pgfscope}%
\pgfpathrectangle{\pgfqpoint{0.100000in}{0.100000in}}{\pgfqpoint{3.420221in}{2.189500in}}%
\pgfusepath{clip}%
\pgfsetbuttcap%
\pgfsetmiterjoin%
\definecolor{currentfill}{rgb}{0.000000,0.282353,0.858824}%
\pgfsetfillcolor{currentfill}%
\pgfsetlinewidth{0.000000pt}%
\definecolor{currentstroke}{rgb}{0.000000,0.000000,0.000000}%
\pgfsetstrokecolor{currentstroke}%
\pgfsetstrokeopacity{0.000000}%
\pgfsetdash{}{0pt}%
\pgfpathmoveto{\pgfqpoint{2.041308in}{1.746848in}}%
\pgfpathlineto{\pgfqpoint{2.034777in}{1.750024in}}%
\pgfpathlineto{\pgfqpoint{2.014648in}{1.749880in}}%
\pgfpathlineto{\pgfqpoint{2.014643in}{1.756416in}}%
\pgfpathlineto{\pgfqpoint{1.995139in}{1.756374in}}%
\pgfpathlineto{\pgfqpoint{1.994758in}{1.783602in}}%
\pgfpathlineto{\pgfqpoint{2.040430in}{1.783885in}}%
\pgfpathlineto{\pgfqpoint{2.036134in}{1.787383in}}%
\pgfpathlineto{\pgfqpoint{2.067103in}{1.787715in}}%
\pgfpathlineto{\pgfqpoint{2.066679in}{1.799816in}}%
\pgfpathlineto{\pgfqpoint{2.064505in}{1.799823in}}%
\pgfpathlineto{\pgfqpoint{2.064332in}{1.812925in}}%
\pgfpathlineto{\pgfqpoint{2.064218in}{1.819460in}}%
\pgfpathlineto{\pgfqpoint{2.084080in}{1.819988in}}%
\pgfpathlineto{\pgfqpoint{2.084131in}{1.813030in}}%
\pgfpathlineto{\pgfqpoint{2.084321in}{1.800001in}}%
\pgfpathlineto{\pgfqpoint{2.079833in}{1.799955in}}%
\pgfpathlineto{\pgfqpoint{2.080363in}{1.781293in}}%
\pgfpathlineto{\pgfqpoint{2.081054in}{1.744447in}}%
\pgfpathlineto{\pgfqpoint{2.077326in}{1.747872in}}%
\pgfpathlineto{\pgfqpoint{2.068172in}{1.747944in}}%
\pgfpathlineto{\pgfqpoint{2.052546in}{1.757374in}}%
\pgfpathlineto{\pgfqpoint{2.048992in}{1.750305in}}%
\pgfpathclose%
\pgfusepath{fill}%
\end{pgfscope}%
\begin{pgfscope}%
\pgfpathrectangle{\pgfqpoint{0.100000in}{0.100000in}}{\pgfqpoint{3.420221in}{2.189500in}}%
\pgfusepath{clip}%
\pgfsetbuttcap%
\pgfsetmiterjoin%
\definecolor{currentfill}{rgb}{0.000000,0.647059,0.676471}%
\pgfsetfillcolor{currentfill}%
\pgfsetlinewidth{0.000000pt}%
\definecolor{currentstroke}{rgb}{0.000000,0.000000,0.000000}%
\pgfsetstrokecolor{currentstroke}%
\pgfsetstrokeopacity{0.000000}%
\pgfsetdash{}{0pt}%
\pgfpathmoveto{\pgfqpoint{1.086956in}{1.583137in}}%
\pgfpathlineto{\pgfqpoint{1.087197in}{1.588806in}}%
\pgfpathlineto{\pgfqpoint{1.083457in}{1.590954in}}%
\pgfpathlineto{\pgfqpoint{1.083291in}{1.599763in}}%
\pgfpathlineto{\pgfqpoint{1.087583in}{1.603916in}}%
\pgfpathlineto{\pgfqpoint{1.089287in}{1.609600in}}%
\pgfpathlineto{\pgfqpoint{1.087622in}{1.614775in}}%
\pgfpathlineto{\pgfqpoint{1.072544in}{1.617623in}}%
\pgfpathlineto{\pgfqpoint{1.073082in}{1.632862in}}%
\pgfpathlineto{\pgfqpoint{1.071637in}{1.640838in}}%
\pgfpathlineto{\pgfqpoint{1.064001in}{1.640846in}}%
\pgfpathlineto{\pgfqpoint{1.063138in}{1.648000in}}%
\pgfpathlineto{\pgfqpoint{1.064401in}{1.655947in}}%
\pgfpathlineto{\pgfqpoint{1.070829in}{1.664247in}}%
\pgfpathlineto{\pgfqpoint{1.125812in}{1.653943in}}%
\pgfpathlineto{\pgfqpoint{1.112247in}{1.578714in}}%
\pgfpathclose%
\pgfusepath{fill}%
\end{pgfscope}%
\begin{pgfscope}%
\pgfpathrectangle{\pgfqpoint{0.100000in}{0.100000in}}{\pgfqpoint{3.420221in}{2.189500in}}%
\pgfusepath{clip}%
\pgfsetbuttcap%
\pgfsetmiterjoin%
\definecolor{currentfill}{rgb}{0.000000,0.650980,0.674510}%
\pgfsetfillcolor{currentfill}%
\pgfsetlinewidth{0.000000pt}%
\definecolor{currentstroke}{rgb}{0.000000,0.000000,0.000000}%
\pgfsetstrokecolor{currentstroke}%
\pgfsetstrokeopacity{0.000000}%
\pgfsetdash{}{0pt}%
\pgfpathmoveto{\pgfqpoint{2.084056in}{1.146860in}}%
\pgfpathlineto{\pgfqpoint{2.058615in}{1.147148in}}%
\pgfpathlineto{\pgfqpoint{2.056798in}{1.151566in}}%
\pgfpathlineto{\pgfqpoint{2.024892in}{1.152471in}}%
\pgfpathlineto{\pgfqpoint{2.024890in}{1.202876in}}%
\pgfpathlineto{\pgfqpoint{2.024878in}{1.204610in}}%
\pgfpathlineto{\pgfqpoint{2.046951in}{1.204051in}}%
\pgfpathlineto{\pgfqpoint{2.057682in}{1.203082in}}%
\pgfpathlineto{\pgfqpoint{2.057351in}{1.192924in}}%
\pgfpathlineto{\pgfqpoint{2.072567in}{1.192484in}}%
\pgfpathlineto{\pgfqpoint{2.072437in}{1.188130in}}%
\pgfpathlineto{\pgfqpoint{2.083284in}{1.187891in}}%
\pgfpathlineto{\pgfqpoint{2.084502in}{1.181336in}}%
\pgfpathclose%
\pgfusepath{fill}%
\end{pgfscope}%
\begin{pgfscope}%
\pgfpathrectangle{\pgfqpoint{0.100000in}{0.100000in}}{\pgfqpoint{3.420221in}{2.189500in}}%
\pgfusepath{clip}%
\pgfsetbuttcap%
\pgfsetmiterjoin%
\definecolor{currentfill}{rgb}{0.000000,0.447059,0.776471}%
\pgfsetfillcolor{currentfill}%
\pgfsetlinewidth{0.000000pt}%
\definecolor{currentstroke}{rgb}{0.000000,0.000000,0.000000}%
\pgfsetstrokecolor{currentstroke}%
\pgfsetstrokeopacity{0.000000}%
\pgfsetdash{}{0pt}%
\pgfpathmoveto{\pgfqpoint{2.941115in}{1.093711in}}%
\pgfpathlineto{\pgfqpoint{2.903485in}{1.086564in}}%
\pgfpathlineto{\pgfqpoint{2.901022in}{1.091945in}}%
\pgfpathlineto{\pgfqpoint{2.892493in}{1.098029in}}%
\pgfpathlineto{\pgfqpoint{2.883974in}{1.101519in}}%
\pgfpathlineto{\pgfqpoint{2.881990in}{1.107689in}}%
\pgfpathlineto{\pgfqpoint{2.886411in}{1.109411in}}%
\pgfpathlineto{\pgfqpoint{2.886154in}{1.118609in}}%
\pgfpathlineto{\pgfqpoint{2.895251in}{1.124938in}}%
\pgfpathlineto{\pgfqpoint{2.905672in}{1.125420in}}%
\pgfpathlineto{\pgfqpoint{2.900079in}{1.164945in}}%
\pgfpathlineto{\pgfqpoint{2.918648in}{1.168023in}}%
\pgfpathlineto{\pgfqpoint{2.919091in}{1.191996in}}%
\pgfpathlineto{\pgfqpoint{2.919493in}{1.206218in}}%
\pgfpathlineto{\pgfqpoint{2.925981in}{1.209025in}}%
\pgfpathlineto{\pgfqpoint{2.933306in}{1.215168in}}%
\pgfpathlineto{\pgfqpoint{2.939060in}{1.215848in}}%
\pgfpathlineto{\pgfqpoint{2.942969in}{1.211233in}}%
\pgfpathlineto{\pgfqpoint{2.948821in}{1.212787in}}%
\pgfpathlineto{\pgfqpoint{2.948111in}{1.173085in}}%
\pgfpathlineto{\pgfqpoint{2.952898in}{1.173935in}}%
\pgfpathlineto{\pgfqpoint{2.955859in}{1.151639in}}%
\pgfpathlineto{\pgfqpoint{2.949686in}{1.150689in}}%
\pgfpathlineto{\pgfqpoint{2.953579in}{1.125751in}}%
\pgfpathlineto{\pgfqpoint{2.956026in}{1.121353in}}%
\pgfpathlineto{\pgfqpoint{2.937751in}{1.118126in}}%
\pgfpathclose%
\pgfusepath{fill}%
\end{pgfscope}%
\begin{pgfscope}%
\pgfpathrectangle{\pgfqpoint{0.100000in}{0.100000in}}{\pgfqpoint{3.420221in}{2.189500in}}%
\pgfusepath{clip}%
\pgfsetbuttcap%
\pgfsetmiterjoin%
\definecolor{currentfill}{rgb}{0.000000,0.309804,0.845098}%
\pgfsetfillcolor{currentfill}%
\pgfsetlinewidth{0.000000pt}%
\definecolor{currentstroke}{rgb}{0.000000,0.000000,0.000000}%
\pgfsetstrokecolor{currentstroke}%
\pgfsetstrokeopacity{0.000000}%
\pgfsetdash{}{0pt}%
\pgfpathmoveto{\pgfqpoint{1.313517in}{0.429646in}}%
\pgfpathlineto{\pgfqpoint{1.316923in}{0.438981in}}%
\pgfpathlineto{\pgfqpoint{1.317096in}{0.444806in}}%
\pgfpathlineto{\pgfqpoint{1.325626in}{0.438841in}}%
\pgfpathlineto{\pgfqpoint{1.327349in}{0.430957in}}%
\pgfpathlineto{\pgfqpoint{1.322320in}{0.425507in}}%
\pgfpathclose%
\pgfusepath{fill}%
\end{pgfscope}%
\begin{pgfscope}%
\pgfpathrectangle{\pgfqpoint{0.100000in}{0.100000in}}{\pgfqpoint{3.420221in}{2.189500in}}%
\pgfusepath{clip}%
\pgfsetbuttcap%
\pgfsetmiterjoin%
\definecolor{currentfill}{rgb}{0.000000,0.309804,0.845098}%
\pgfsetfillcolor{currentfill}%
\pgfsetlinewidth{0.000000pt}%
\definecolor{currentstroke}{rgb}{0.000000,0.000000,0.000000}%
\pgfsetstrokecolor{currentstroke}%
\pgfsetstrokeopacity{0.000000}%
\pgfsetdash{}{0pt}%
\pgfpathmoveto{\pgfqpoint{1.318733in}{0.402030in}}%
\pgfpathlineto{\pgfqpoint{1.319408in}{0.405315in}}%
\pgfpathlineto{\pgfqpoint{1.329864in}{0.403406in}}%
\pgfpathlineto{\pgfqpoint{1.327292in}{0.398639in}}%
\pgfpathclose%
\pgfusepath{fill}%
\end{pgfscope}%
\begin{pgfscope}%
\pgfpathrectangle{\pgfqpoint{0.100000in}{0.100000in}}{\pgfqpoint{3.420221in}{2.189500in}}%
\pgfusepath{clip}%
\pgfsetbuttcap%
\pgfsetmiterjoin%
\definecolor{currentfill}{rgb}{0.000000,0.309804,0.845098}%
\pgfsetfillcolor{currentfill}%
\pgfsetlinewidth{0.000000pt}%
\definecolor{currentstroke}{rgb}{0.000000,0.000000,0.000000}%
\pgfsetstrokecolor{currentstroke}%
\pgfsetstrokeopacity{0.000000}%
\pgfsetdash{}{0pt}%
\pgfpathmoveto{\pgfqpoint{1.339731in}{0.392640in}}%
\pgfpathlineto{\pgfqpoint{1.337166in}{0.397724in}}%
\pgfpathlineto{\pgfqpoint{1.344615in}{0.409434in}}%
\pgfpathlineto{\pgfqpoint{1.336751in}{0.418559in}}%
\pgfpathlineto{\pgfqpoint{1.336742in}{0.426037in}}%
\pgfpathlineto{\pgfqpoint{1.344156in}{0.432355in}}%
\pgfpathlineto{\pgfqpoint{1.349709in}{0.430610in}}%
\pgfpathlineto{\pgfqpoint{1.351041in}{0.424964in}}%
\pgfpathlineto{\pgfqpoint{1.349416in}{0.417487in}}%
\pgfpathlineto{\pgfqpoint{1.363032in}{0.411323in}}%
\pgfpathlineto{\pgfqpoint{1.366797in}{0.396133in}}%
\pgfpathlineto{\pgfqpoint{1.371700in}{0.389050in}}%
\pgfpathlineto{\pgfqpoint{1.368383in}{0.383304in}}%
\pgfpathlineto{\pgfqpoint{1.362293in}{0.382490in}}%
\pgfpathlineto{\pgfqpoint{1.352146in}{0.388016in}}%
\pgfpathclose%
\pgfusepath{fill}%
\end{pgfscope}%
\begin{pgfscope}%
\pgfpathrectangle{\pgfqpoint{0.100000in}{0.100000in}}{\pgfqpoint{3.420221in}{2.189500in}}%
\pgfusepath{clip}%
\pgfsetbuttcap%
\pgfsetmiterjoin%
\definecolor{currentfill}{rgb}{0.000000,0.309804,0.845098}%
\pgfsetfillcolor{currentfill}%
\pgfsetlinewidth{0.000000pt}%
\definecolor{currentstroke}{rgb}{0.000000,0.000000,0.000000}%
\pgfsetstrokecolor{currentstroke}%
\pgfsetstrokeopacity{0.000000}%
\pgfsetdash{}{0pt}%
\pgfpathmoveto{\pgfqpoint{1.337918in}{0.453862in}}%
\pgfpathlineto{\pgfqpoint{1.348123in}{0.446212in}}%
\pgfpathlineto{\pgfqpoint{1.344397in}{0.442387in}}%
\pgfpathlineto{\pgfqpoint{1.336751in}{0.443404in}}%
\pgfpathlineto{\pgfqpoint{1.330730in}{0.449228in}}%
\pgfpathlineto{\pgfqpoint{1.326217in}{0.456739in}}%
\pgfpathlineto{\pgfqpoint{1.312163in}{0.467564in}}%
\pgfpathlineto{\pgfqpoint{1.321994in}{0.471957in}}%
\pgfpathclose%
\pgfusepath{fill}%
\end{pgfscope}%
\begin{pgfscope}%
\pgfpathrectangle{\pgfqpoint{0.100000in}{0.100000in}}{\pgfqpoint{3.420221in}{2.189500in}}%
\pgfusepath{clip}%
\pgfsetbuttcap%
\pgfsetmiterjoin%
\definecolor{currentfill}{rgb}{0.000000,0.533333,0.733333}%
\pgfsetfillcolor{currentfill}%
\pgfsetlinewidth{0.000000pt}%
\definecolor{currentstroke}{rgb}{0.000000,0.000000,0.000000}%
\pgfsetstrokecolor{currentstroke}%
\pgfsetstrokeopacity{0.000000}%
\pgfsetdash{}{0pt}%
\pgfpathmoveto{\pgfqpoint{1.499877in}{0.908964in}}%
\pgfpathlineto{\pgfqpoint{1.497857in}{0.889950in}}%
\pgfpathlineto{\pgfqpoint{1.469277in}{0.892536in}}%
\pgfpathlineto{\pgfqpoint{1.469923in}{0.899045in}}%
\pgfpathlineto{\pgfqpoint{1.456913in}{0.900203in}}%
\pgfpathlineto{\pgfqpoint{1.458181in}{0.912414in}}%
\pgfpathlineto{\pgfqpoint{1.450397in}{0.913165in}}%
\pgfpathlineto{\pgfqpoint{1.452332in}{0.932808in}}%
\pgfpathlineto{\pgfqpoint{1.445874in}{0.933525in}}%
\pgfpathlineto{\pgfqpoint{1.449555in}{0.972610in}}%
\pgfpathlineto{\pgfqpoint{1.464474in}{0.971123in}}%
\pgfpathlineto{\pgfqpoint{1.465158in}{0.977690in}}%
\pgfpathlineto{\pgfqpoint{1.478048in}{0.976397in}}%
\pgfpathlineto{\pgfqpoint{1.478688in}{0.982915in}}%
\pgfpathlineto{\pgfqpoint{1.485145in}{0.982282in}}%
\pgfpathlineto{\pgfqpoint{1.485799in}{0.988783in}}%
\pgfpathlineto{\pgfqpoint{1.492265in}{0.988159in}}%
\pgfpathlineto{\pgfqpoint{1.492943in}{0.994844in}}%
\pgfpathlineto{\pgfqpoint{1.507670in}{0.993501in}}%
\pgfpathlineto{\pgfqpoint{1.506292in}{0.978032in}}%
\pgfpathlineto{\pgfqpoint{1.538122in}{0.975267in}}%
\pgfpathlineto{\pgfqpoint{1.535397in}{0.942773in}}%
\pgfpathlineto{\pgfqpoint{1.503330in}{0.945506in}}%
\pgfpathclose%
\pgfusepath{fill}%
\end{pgfscope}%
\begin{pgfscope}%
\pgfpathrectangle{\pgfqpoint{0.100000in}{0.100000in}}{\pgfqpoint{3.420221in}{2.189500in}}%
\pgfusepath{clip}%
\pgfsetbuttcap%
\pgfsetmiterjoin%
\definecolor{currentfill}{rgb}{0.000000,0.525490,0.737255}%
\pgfsetfillcolor{currentfill}%
\pgfsetlinewidth{0.000000pt}%
\definecolor{currentstroke}{rgb}{0.000000,0.000000,0.000000}%
\pgfsetstrokecolor{currentstroke}%
\pgfsetstrokeopacity{0.000000}%
\pgfsetdash{}{0pt}%
\pgfpathmoveto{\pgfqpoint{2.932992in}{0.938520in}}%
\pgfpathlineto{\pgfqpoint{2.926173in}{0.933340in}}%
\pgfpathlineto{\pgfqpoint{2.918286in}{0.931024in}}%
\pgfpathlineto{\pgfqpoint{2.910722in}{0.935339in}}%
\pgfpathlineto{\pgfqpoint{2.904896in}{0.944603in}}%
\pgfpathlineto{\pgfqpoint{2.897625in}{0.950640in}}%
\pgfpathlineto{\pgfqpoint{2.898575in}{0.956373in}}%
\pgfpathlineto{\pgfqpoint{2.895848in}{0.959538in}}%
\pgfpathlineto{\pgfqpoint{2.895643in}{0.973761in}}%
\pgfpathlineto{\pgfqpoint{2.892801in}{0.977782in}}%
\pgfpathlineto{\pgfqpoint{2.887990in}{0.974832in}}%
\pgfpathlineto{\pgfqpoint{2.877414in}{0.981799in}}%
\pgfpathlineto{\pgfqpoint{2.881263in}{0.997223in}}%
\pgfpathlineto{\pgfqpoint{2.873627in}{1.002160in}}%
\pgfpathlineto{\pgfqpoint{2.880980in}{1.009566in}}%
\pgfpathlineto{\pgfqpoint{2.887564in}{1.006957in}}%
\pgfpathlineto{\pgfqpoint{2.893441in}{1.007997in}}%
\pgfpathlineto{\pgfqpoint{2.892526in}{1.012941in}}%
\pgfpathlineto{\pgfqpoint{2.900556in}{1.018396in}}%
\pgfpathlineto{\pgfqpoint{2.906453in}{1.010891in}}%
\pgfpathlineto{\pgfqpoint{2.910879in}{1.001149in}}%
\pgfpathlineto{\pgfqpoint{2.918317in}{0.999328in}}%
\pgfpathlineto{\pgfqpoint{2.923392in}{0.992722in}}%
\pgfpathlineto{\pgfqpoint{2.921338in}{0.987279in}}%
\pgfpathlineto{\pgfqpoint{2.939759in}{0.977192in}}%
\pgfpathlineto{\pgfqpoint{2.935293in}{0.973196in}}%
\pgfpathlineto{\pgfqpoint{2.940531in}{0.964321in}}%
\pgfpathlineto{\pgfqpoint{2.936207in}{0.959726in}}%
\pgfpathlineto{\pgfqpoint{2.938091in}{0.956508in}}%
\pgfpathclose%
\pgfusepath{fill}%
\end{pgfscope}%
\begin{pgfscope}%
\pgfpathrectangle{\pgfqpoint{0.100000in}{0.100000in}}{\pgfqpoint{3.420221in}{2.189500in}}%
\pgfusepath{clip}%
\pgfsetbuttcap%
\pgfsetmiterjoin%
\definecolor{currentfill}{rgb}{0.000000,0.443137,0.778431}%
\pgfsetfillcolor{currentfill}%
\pgfsetlinewidth{0.000000pt}%
\definecolor{currentstroke}{rgb}{0.000000,0.000000,0.000000}%
\pgfsetstrokecolor{currentstroke}%
\pgfsetstrokeopacity{0.000000}%
\pgfsetdash{}{0pt}%
\pgfpathmoveto{\pgfqpoint{1.485889in}{1.839116in}}%
\pgfpathlineto{\pgfqpoint{1.483103in}{1.839413in}}%
\pgfpathlineto{\pgfqpoint{1.480231in}{1.813181in}}%
\pgfpathlineto{\pgfqpoint{1.477993in}{1.813430in}}%
\pgfpathlineto{\pgfqpoint{1.414789in}{1.820947in}}%
\pgfpathlineto{\pgfqpoint{1.418041in}{1.847044in}}%
\pgfpathlineto{\pgfqpoint{1.420025in}{1.846812in}}%
\pgfpathlineto{\pgfqpoint{1.423124in}{1.872785in}}%
\pgfpathlineto{\pgfqpoint{1.424925in}{1.872572in}}%
\pgfpathlineto{\pgfqpoint{1.428108in}{1.898667in}}%
\pgfpathlineto{\pgfqpoint{1.430010in}{1.899550in}}%
\pgfpathlineto{\pgfqpoint{1.433071in}{1.924714in}}%
\pgfpathlineto{\pgfqpoint{1.439553in}{1.923934in}}%
\pgfpathlineto{\pgfqpoint{1.438748in}{1.917310in}}%
\pgfpathlineto{\pgfqpoint{1.445261in}{1.916541in}}%
\pgfpathlineto{\pgfqpoint{1.446050in}{1.923144in}}%
\pgfpathlineto{\pgfqpoint{1.467857in}{1.920627in}}%
\pgfpathlineto{\pgfqpoint{1.472018in}{1.920141in}}%
\pgfpathlineto{\pgfqpoint{1.470261in}{1.904780in}}%
\pgfpathlineto{\pgfqpoint{1.476835in}{1.904036in}}%
\pgfpathlineto{\pgfqpoint{1.476332in}{1.899677in}}%
\pgfpathlineto{\pgfqpoint{1.482877in}{1.898940in}}%
\pgfpathlineto{\pgfqpoint{1.484552in}{1.894368in}}%
\pgfpathlineto{\pgfqpoint{1.506335in}{1.891959in}}%
\pgfpathlineto{\pgfqpoint{1.504764in}{1.878745in}}%
\pgfpathlineto{\pgfqpoint{1.498043in}{1.877284in}}%
\pgfpathlineto{\pgfqpoint{1.496625in}{1.864207in}}%
\pgfpathlineto{\pgfqpoint{1.487686in}{1.865172in}}%
\pgfpathclose%
\pgfusepath{fill}%
\end{pgfscope}%
\begin{pgfscope}%
\pgfpathrectangle{\pgfqpoint{0.100000in}{0.100000in}}{\pgfqpoint{3.420221in}{2.189500in}}%
\pgfusepath{clip}%
\pgfsetbuttcap%
\pgfsetmiterjoin%
\definecolor{currentfill}{rgb}{0.000000,0.180392,0.909804}%
\pgfsetfillcolor{currentfill}%
\pgfsetlinewidth{0.000000pt}%
\definecolor{currentstroke}{rgb}{0.000000,0.000000,0.000000}%
\pgfsetstrokecolor{currentstroke}%
\pgfsetstrokeopacity{0.000000}%
\pgfsetdash{}{0pt}%
\pgfpathmoveto{\pgfqpoint{1.913740in}{0.495413in}}%
\pgfpathlineto{\pgfqpoint{1.915005in}{0.519834in}}%
\pgfpathlineto{\pgfqpoint{1.893366in}{0.541816in}}%
\pgfpathlineto{\pgfqpoint{1.912881in}{0.553688in}}%
\pgfpathlineto{\pgfqpoint{1.915707in}{0.561716in}}%
\pgfpathlineto{\pgfqpoint{1.924552in}{0.570650in}}%
\pgfpathlineto{\pgfqpoint{1.932043in}{0.567296in}}%
\pgfpathlineto{\pgfqpoint{1.932052in}{0.558403in}}%
\pgfpathlineto{\pgfqpoint{1.936508in}{0.555327in}}%
\pgfpathlineto{\pgfqpoint{1.939227in}{0.547806in}}%
\pgfpathlineto{\pgfqpoint{1.945927in}{0.542225in}}%
\pgfpathlineto{\pgfqpoint{1.952246in}{0.546869in}}%
\pgfpathlineto{\pgfqpoint{1.958781in}{0.544216in}}%
\pgfpathlineto{\pgfqpoint{1.963805in}{0.547988in}}%
\pgfpathlineto{\pgfqpoint{1.965767in}{0.555426in}}%
\pgfpathlineto{\pgfqpoint{1.971520in}{0.555457in}}%
\pgfpathlineto{\pgfqpoint{1.974109in}{0.566150in}}%
\pgfpathlineto{\pgfqpoint{1.984840in}{0.567089in}}%
\pgfpathlineto{\pgfqpoint{1.987807in}{0.564254in}}%
\pgfpathlineto{\pgfqpoint{1.986697in}{0.557401in}}%
\pgfpathlineto{\pgfqpoint{1.998384in}{0.537149in}}%
\pgfpathlineto{\pgfqpoint{1.994185in}{0.527920in}}%
\pgfpathlineto{\pgfqpoint{1.976546in}{0.511820in}}%
\pgfpathlineto{\pgfqpoint{1.971348in}{0.510629in}}%
\pgfpathlineto{\pgfqpoint{1.947847in}{0.496749in}}%
\pgfpathlineto{\pgfqpoint{1.935911in}{0.492404in}}%
\pgfpathlineto{\pgfqpoint{1.932863in}{0.495924in}}%
\pgfpathlineto{\pgfqpoint{1.920663in}{0.490442in}}%
\pgfpathlineto{\pgfqpoint{1.920812in}{0.499528in}}%
\pgfpathclose%
\pgfusepath{fill}%
\end{pgfscope}%
\begin{pgfscope}%
\pgfpathrectangle{\pgfqpoint{0.100000in}{0.100000in}}{\pgfqpoint{3.420221in}{2.189500in}}%
\pgfusepath{clip}%
\pgfsetbuttcap%
\pgfsetmiterjoin%
\definecolor{currentfill}{rgb}{0.000000,0.545098,0.727451}%
\pgfsetfillcolor{currentfill}%
\pgfsetlinewidth{0.000000pt}%
\definecolor{currentstroke}{rgb}{0.000000,0.000000,0.000000}%
\pgfsetstrokecolor{currentstroke}%
\pgfsetstrokeopacity{0.000000}%
\pgfsetdash{}{0pt}%
\pgfpathmoveto{\pgfqpoint{2.954170in}{1.365655in}}%
\pgfpathlineto{\pgfqpoint{2.947296in}{1.353566in}}%
\pgfpathlineto{\pgfqpoint{2.942576in}{1.350644in}}%
\pgfpathlineto{\pgfqpoint{2.939441in}{1.340626in}}%
\pgfpathlineto{\pgfqpoint{2.931126in}{1.345770in}}%
\pgfpathlineto{\pgfqpoint{2.928652in}{1.338582in}}%
\pgfpathlineto{\pgfqpoint{2.909299in}{1.350100in}}%
\pgfpathlineto{\pgfqpoint{2.910776in}{1.359156in}}%
\pgfpathlineto{\pgfqpoint{2.907645in}{1.360172in}}%
\pgfpathlineto{\pgfqpoint{2.909371in}{1.367679in}}%
\pgfpathlineto{\pgfqpoint{2.906313in}{1.369093in}}%
\pgfpathlineto{\pgfqpoint{2.898397in}{1.366262in}}%
\pgfpathlineto{\pgfqpoint{2.892296in}{1.405213in}}%
\pgfpathlineto{\pgfqpoint{2.897071in}{1.406077in}}%
\pgfpathlineto{\pgfqpoint{2.954465in}{1.416446in}}%
\pgfpathlineto{\pgfqpoint{2.958679in}{1.414878in}}%
\pgfpathlineto{\pgfqpoint{2.953464in}{1.407492in}}%
\pgfpathlineto{\pgfqpoint{2.952251in}{1.400351in}}%
\pgfpathlineto{\pgfqpoint{2.941259in}{1.399741in}}%
\pgfpathlineto{\pgfqpoint{2.935228in}{1.390228in}}%
\pgfpathlineto{\pgfqpoint{2.932977in}{1.382690in}}%
\pgfpathlineto{\pgfqpoint{2.926844in}{1.374496in}}%
\pgfpathlineto{\pgfqpoint{2.931978in}{1.372464in}}%
\pgfpathlineto{\pgfqpoint{2.948792in}{1.369383in}}%
\pgfpathclose%
\pgfusepath{fill}%
\end{pgfscope}%
\begin{pgfscope}%
\pgfpathrectangle{\pgfqpoint{0.100000in}{0.100000in}}{\pgfqpoint{3.420221in}{2.189500in}}%
\pgfusepath{clip}%
\pgfsetbuttcap%
\pgfsetmiterjoin%
\definecolor{currentfill}{rgb}{0.000000,0.513725,0.743137}%
\pgfsetfillcolor{currentfill}%
\pgfsetlinewidth{0.000000pt}%
\definecolor{currentstroke}{rgb}{0.000000,0.000000,0.000000}%
\pgfsetstrokecolor{currentstroke}%
\pgfsetstrokeopacity{0.000000}%
\pgfsetdash{}{0pt}%
\pgfpathmoveto{\pgfqpoint{3.113333in}{1.687026in}}%
\pgfpathlineto{\pgfqpoint{3.106600in}{1.688090in}}%
\pgfpathlineto{\pgfqpoint{3.105603in}{1.698264in}}%
\pgfpathlineto{\pgfqpoint{3.102390in}{1.704087in}}%
\pgfpathlineto{\pgfqpoint{3.105060in}{1.711795in}}%
\pgfpathlineto{\pgfqpoint{3.102289in}{1.719681in}}%
\pgfpathlineto{\pgfqpoint{3.093130in}{1.721745in}}%
\pgfpathlineto{\pgfqpoint{3.095604in}{1.733614in}}%
\pgfpathlineto{\pgfqpoint{3.081784in}{1.775456in}}%
\pgfpathlineto{\pgfqpoint{3.098104in}{1.781414in}}%
\pgfpathlineto{\pgfqpoint{3.111072in}{1.786047in}}%
\pgfpathlineto{\pgfqpoint{3.114998in}{1.775320in}}%
\pgfpathlineto{\pgfqpoint{3.111492in}{1.771066in}}%
\pgfpathlineto{\pgfqpoint{3.119939in}{1.764197in}}%
\pgfpathlineto{\pgfqpoint{3.128989in}{1.765195in}}%
\pgfpathlineto{\pgfqpoint{3.129353in}{1.761278in}}%
\pgfpathlineto{\pgfqpoint{3.121288in}{1.758173in}}%
\pgfpathlineto{\pgfqpoint{3.142447in}{1.698988in}}%
\pgfpathlineto{\pgfqpoint{3.133365in}{1.688106in}}%
\pgfpathlineto{\pgfqpoint{3.124665in}{1.685262in}}%
\pgfpathclose%
\pgfusepath{fill}%
\end{pgfscope}%
\begin{pgfscope}%
\pgfpathrectangle{\pgfqpoint{0.100000in}{0.100000in}}{\pgfqpoint{3.420221in}{2.189500in}}%
\pgfusepath{clip}%
\pgfsetbuttcap%
\pgfsetmiterjoin%
\definecolor{currentfill}{rgb}{0.000000,0.403922,0.798039}%
\pgfsetfillcolor{currentfill}%
\pgfsetlinewidth{0.000000pt}%
\definecolor{currentstroke}{rgb}{0.000000,0.000000,0.000000}%
\pgfsetstrokecolor{currentstroke}%
\pgfsetstrokeopacity{0.000000}%
\pgfsetdash{}{0pt}%
\pgfpathmoveto{\pgfqpoint{2.429834in}{1.304131in}}%
\pgfpathlineto{\pgfqpoint{2.422921in}{1.305214in}}%
\pgfpathlineto{\pgfqpoint{2.380979in}{1.301424in}}%
\pgfpathlineto{\pgfqpoint{2.380743in}{1.304684in}}%
\pgfpathlineto{\pgfqpoint{2.378335in}{1.337307in}}%
\pgfpathlineto{\pgfqpoint{2.401832in}{1.339131in}}%
\pgfpathlineto{\pgfqpoint{2.407207in}{1.342053in}}%
\pgfpathlineto{\pgfqpoint{2.407745in}{1.356666in}}%
\pgfpathlineto{\pgfqpoint{2.430901in}{1.358769in}}%
\pgfpathlineto{\pgfqpoint{2.434283in}{1.318811in}}%
\pgfpathlineto{\pgfqpoint{2.430219in}{1.315711in}}%
\pgfpathlineto{\pgfqpoint{2.432062in}{1.307542in}}%
\pgfpathclose%
\pgfusepath{fill}%
\end{pgfscope}%
\begin{pgfscope}%
\pgfpathrectangle{\pgfqpoint{0.100000in}{0.100000in}}{\pgfqpoint{3.420221in}{2.189500in}}%
\pgfusepath{clip}%
\pgfsetbuttcap%
\pgfsetmiterjoin%
\definecolor{currentfill}{rgb}{0.000000,0.156863,0.921569}%
\pgfsetfillcolor{currentfill}%
\pgfsetlinewidth{0.000000pt}%
\definecolor{currentstroke}{rgb}{0.000000,0.000000,0.000000}%
\pgfsetstrokecolor{currentstroke}%
\pgfsetstrokeopacity{0.000000}%
\pgfsetdash{}{0pt}%
\pgfpathmoveto{\pgfqpoint{1.752338in}{1.750273in}}%
\pgfpathlineto{\pgfqpoint{1.723158in}{1.751959in}}%
\pgfpathlineto{\pgfqpoint{1.720482in}{1.757483in}}%
\pgfpathlineto{\pgfqpoint{1.723020in}{1.762111in}}%
\pgfpathlineto{\pgfqpoint{1.719850in}{1.769218in}}%
\pgfpathlineto{\pgfqpoint{1.713200in}{1.773159in}}%
\pgfpathlineto{\pgfqpoint{1.718238in}{1.783426in}}%
\pgfpathlineto{\pgfqpoint{1.722595in}{1.786818in}}%
\pgfpathlineto{\pgfqpoint{1.719230in}{1.799199in}}%
\pgfpathlineto{\pgfqpoint{1.713684in}{1.805164in}}%
\pgfpathlineto{\pgfqpoint{1.754598in}{1.802660in}}%
\pgfpathclose%
\pgfusepath{fill}%
\end{pgfscope}%
\begin{pgfscope}%
\pgfpathrectangle{\pgfqpoint{0.100000in}{0.100000in}}{\pgfqpoint{3.420221in}{2.189500in}}%
\pgfusepath{clip}%
\pgfsetbuttcap%
\pgfsetmiterjoin%
\definecolor{currentfill}{rgb}{0.000000,0.768627,0.615686}%
\pgfsetfillcolor{currentfill}%
\pgfsetlinewidth{0.000000pt}%
\definecolor{currentstroke}{rgb}{0.000000,0.000000,0.000000}%
\pgfsetstrokecolor{currentstroke}%
\pgfsetstrokeopacity{0.000000}%
\pgfsetdash{}{0pt}%
\pgfpathmoveto{\pgfqpoint{0.751367in}{0.424210in}}%
\pgfpathlineto{\pgfqpoint{0.759703in}{0.417699in}}%
\pgfpathlineto{\pgfqpoint{0.754853in}{0.411634in}}%
\pgfpathlineto{\pgfqpoint{0.744119in}{0.420039in}}%
\pgfpathlineto{\pgfqpoint{0.748019in}{0.420657in}}%
\pgfpathlineto{\pgfqpoint{0.751209in}{0.422586in}}%
\pgfpathclose%
\pgfusepath{fill}%
\end{pgfscope}%
\begin{pgfscope}%
\pgfpathrectangle{\pgfqpoint{0.100000in}{0.100000in}}{\pgfqpoint{3.420221in}{2.189500in}}%
\pgfusepath{clip}%
\pgfsetbuttcap%
\pgfsetmiterjoin%
\definecolor{currentfill}{rgb}{0.000000,0.458824,0.770588}%
\pgfsetfillcolor{currentfill}%
\pgfsetlinewidth{0.000000pt}%
\definecolor{currentstroke}{rgb}{0.000000,0.000000,0.000000}%
\pgfsetstrokecolor{currentstroke}%
\pgfsetstrokeopacity{0.000000}%
\pgfsetdash{}{0pt}%
\pgfpathmoveto{\pgfqpoint{1.673547in}{1.649089in}}%
\pgfpathlineto{\pgfqpoint{1.664443in}{1.645920in}}%
\pgfpathlineto{\pgfqpoint{1.662326in}{1.615766in}}%
\pgfpathlineto{\pgfqpoint{1.614273in}{1.619347in}}%
\pgfpathlineto{\pgfqpoint{1.613248in}{1.625968in}}%
\pgfpathlineto{\pgfqpoint{1.614564in}{1.642760in}}%
\pgfpathlineto{\pgfqpoint{1.621113in}{1.642783in}}%
\pgfpathlineto{\pgfqpoint{1.623407in}{1.664221in}}%
\pgfpathlineto{\pgfqpoint{1.626772in}{1.702970in}}%
\pgfpathlineto{\pgfqpoint{1.638598in}{1.703926in}}%
\pgfpathlineto{\pgfqpoint{1.644782in}{1.707049in}}%
\pgfpathlineto{\pgfqpoint{1.652199in}{1.705328in}}%
\pgfpathlineto{\pgfqpoint{1.661991in}{1.713775in}}%
\pgfpathlineto{\pgfqpoint{1.668661in}{1.713429in}}%
\pgfpathlineto{\pgfqpoint{1.674179in}{1.717315in}}%
\pgfpathlineto{\pgfqpoint{1.672963in}{1.713088in}}%
\pgfpathlineto{\pgfqpoint{1.671160in}{1.687049in}}%
\pgfpathlineto{\pgfqpoint{1.669646in}{1.673951in}}%
\pgfpathlineto{\pgfqpoint{1.676099in}{1.673547in}}%
\pgfpathclose%
\pgfusepath{fill}%
\end{pgfscope}%
\begin{pgfscope}%
\pgfpathrectangle{\pgfqpoint{0.100000in}{0.100000in}}{\pgfqpoint{3.420221in}{2.189500in}}%
\pgfusepath{clip}%
\pgfsetbuttcap%
\pgfsetmiterjoin%
\definecolor{currentfill}{rgb}{0.000000,0.376471,0.811765}%
\pgfsetfillcolor{currentfill}%
\pgfsetlinewidth{0.000000pt}%
\definecolor{currentstroke}{rgb}{0.000000,0.000000,0.000000}%
\pgfsetstrokecolor{currentstroke}%
\pgfsetstrokeopacity{0.000000}%
\pgfsetdash{}{0pt}%
\pgfpathmoveto{\pgfqpoint{2.285737in}{1.430772in}}%
\pgfpathlineto{\pgfqpoint{2.285074in}{1.443849in}}%
\pgfpathlineto{\pgfqpoint{2.259526in}{1.442641in}}%
\pgfpathlineto{\pgfqpoint{2.259756in}{1.436105in}}%
\pgfpathlineto{\pgfqpoint{2.240214in}{1.435516in}}%
\pgfpathlineto{\pgfqpoint{2.230936in}{1.435391in}}%
\pgfpathlineto{\pgfqpoint{2.228388in}{1.441507in}}%
\pgfpathlineto{\pgfqpoint{2.221160in}{1.447515in}}%
\pgfpathlineto{\pgfqpoint{2.224375in}{1.460720in}}%
\pgfpathlineto{\pgfqpoint{2.238823in}{1.464283in}}%
\pgfpathlineto{\pgfqpoint{2.238462in}{1.475046in}}%
\pgfpathlineto{\pgfqpoint{2.231957in}{1.474796in}}%
\pgfpathlineto{\pgfqpoint{2.231453in}{1.487855in}}%
\pgfpathlineto{\pgfqpoint{2.257301in}{1.488279in}}%
\pgfpathlineto{\pgfqpoint{2.261417in}{1.486033in}}%
\pgfpathlineto{\pgfqpoint{2.268056in}{1.490509in}}%
\pgfpathlineto{\pgfqpoint{2.272014in}{1.475825in}}%
\pgfpathlineto{\pgfqpoint{2.290116in}{1.476751in}}%
\pgfpathlineto{\pgfqpoint{2.291854in}{1.450612in}}%
\pgfpathlineto{\pgfqpoint{2.304181in}{1.451287in}}%
\pgfpathlineto{\pgfqpoint{2.305338in}{1.431824in}}%
\pgfpathclose%
\pgfusepath{fill}%
\end{pgfscope}%
\begin{pgfscope}%
\pgfpathrectangle{\pgfqpoint{0.100000in}{0.100000in}}{\pgfqpoint{3.420221in}{2.189500in}}%
\pgfusepath{clip}%
\pgfsetbuttcap%
\pgfsetmiterjoin%
\definecolor{currentfill}{rgb}{0.000000,0.443137,0.778431}%
\pgfsetfillcolor{currentfill}%
\pgfsetlinewidth{0.000000pt}%
\definecolor{currentstroke}{rgb}{0.000000,0.000000,0.000000}%
\pgfsetstrokecolor{currentstroke}%
\pgfsetstrokeopacity{0.000000}%
\pgfsetdash{}{0pt}%
\pgfpathmoveto{\pgfqpoint{2.572640in}{0.700905in}}%
\pgfpathlineto{\pgfqpoint{2.582467in}{0.701816in}}%
\pgfpathlineto{\pgfqpoint{2.584454in}{0.680056in}}%
\pgfpathlineto{\pgfqpoint{2.596130in}{0.680767in}}%
\pgfpathlineto{\pgfqpoint{2.592887in}{0.675713in}}%
\pgfpathlineto{\pgfqpoint{2.597600in}{0.665053in}}%
\pgfpathlineto{\pgfqpoint{2.589728in}{0.656602in}}%
\pgfpathlineto{\pgfqpoint{2.590388in}{0.647524in}}%
\pgfpathlineto{\pgfqpoint{2.574603in}{0.651791in}}%
\pgfpathlineto{\pgfqpoint{2.563484in}{0.653226in}}%
\pgfpathlineto{\pgfqpoint{2.548029in}{0.653122in}}%
\pgfpathlineto{\pgfqpoint{2.532734in}{0.650371in}}%
\pgfpathlineto{\pgfqpoint{2.512822in}{0.644506in}}%
\pgfpathlineto{\pgfqpoint{2.503176in}{0.643060in}}%
\pgfpathlineto{\pgfqpoint{2.491118in}{0.639334in}}%
\pgfpathlineto{\pgfqpoint{2.495486in}{0.648141in}}%
\pgfpathlineto{\pgfqpoint{2.499982in}{0.651786in}}%
\pgfpathlineto{\pgfqpoint{2.494249in}{0.658342in}}%
\pgfpathlineto{\pgfqpoint{2.497042in}{0.665279in}}%
\pgfpathlineto{\pgfqpoint{2.495889in}{0.669731in}}%
\pgfpathlineto{\pgfqpoint{2.487498in}{0.674027in}}%
\pgfpathlineto{\pgfqpoint{2.479928in}{0.682899in}}%
\pgfpathlineto{\pgfqpoint{2.481449in}{0.693010in}}%
\pgfpathlineto{\pgfqpoint{2.540291in}{0.697885in}}%
\pgfpathclose%
\pgfusepath{fill}%
\end{pgfscope}%
\begin{pgfscope}%
\pgfpathrectangle{\pgfqpoint{0.100000in}{0.100000in}}{\pgfqpoint{3.420221in}{2.189500in}}%
\pgfusepath{clip}%
\pgfsetbuttcap%
\pgfsetmiterjoin%
\definecolor{currentfill}{rgb}{0.000000,0.278431,0.860784}%
\pgfsetfillcolor{currentfill}%
\pgfsetlinewidth{0.000000pt}%
\definecolor{currentstroke}{rgb}{0.000000,0.000000,0.000000}%
\pgfsetstrokecolor{currentstroke}%
\pgfsetstrokeopacity{0.000000}%
\pgfsetdash{}{0pt}%
\pgfpathmoveto{\pgfqpoint{1.736723in}{1.356796in}}%
\pgfpathlineto{\pgfqpoint{1.734953in}{1.324296in}}%
\pgfpathlineto{\pgfqpoint{1.704176in}{1.325944in}}%
\pgfpathlineto{\pgfqpoint{1.703097in}{1.325976in}}%
\pgfpathlineto{\pgfqpoint{1.705097in}{1.358483in}}%
\pgfpathclose%
\pgfusepath{fill}%
\end{pgfscope}%
\begin{pgfscope}%
\pgfpathrectangle{\pgfqpoint{0.100000in}{0.100000in}}{\pgfqpoint{3.420221in}{2.189500in}}%
\pgfusepath{clip}%
\pgfsetbuttcap%
\pgfsetmiterjoin%
\definecolor{currentfill}{rgb}{0.000000,0.776471,0.611765}%
\pgfsetfillcolor{currentfill}%
\pgfsetlinewidth{0.000000pt}%
\definecolor{currentstroke}{rgb}{0.000000,0.000000,0.000000}%
\pgfsetstrokecolor{currentstroke}%
\pgfsetstrokeopacity{0.000000}%
\pgfsetdash{}{0pt}%
\pgfpathmoveto{\pgfqpoint{2.233111in}{1.046699in}}%
\pgfpathlineto{\pgfqpoint{2.226624in}{1.037089in}}%
\pgfpathlineto{\pgfqpoint{2.223522in}{1.027529in}}%
\pgfpathlineto{\pgfqpoint{2.223815in}{1.019004in}}%
\pgfpathlineto{\pgfqpoint{2.197889in}{1.018611in}}%
\pgfpathlineto{\pgfqpoint{2.197670in}{1.031942in}}%
\pgfpathlineto{\pgfqpoint{2.192475in}{1.039127in}}%
\pgfpathlineto{\pgfqpoint{2.193644in}{1.041681in}}%
\pgfpathlineto{\pgfqpoint{2.187155in}{1.048123in}}%
\pgfpathlineto{\pgfqpoint{2.178890in}{1.050162in}}%
\pgfpathlineto{\pgfqpoint{2.180602in}{1.055599in}}%
\pgfpathlineto{\pgfqpoint{2.171980in}{1.062716in}}%
\pgfpathlineto{\pgfqpoint{2.174022in}{1.072405in}}%
\pgfpathlineto{\pgfqpoint{2.209770in}{1.072792in}}%
\pgfpathlineto{\pgfqpoint{2.209534in}{1.079347in}}%
\pgfpathlineto{\pgfqpoint{2.216266in}{1.079461in}}%
\pgfpathlineto{\pgfqpoint{2.216060in}{1.091625in}}%
\pgfpathlineto{\pgfqpoint{2.218643in}{1.091687in}}%
\pgfpathlineto{\pgfqpoint{2.226905in}{1.082822in}}%
\pgfpathlineto{\pgfqpoint{2.228322in}{1.073038in}}%
\pgfpathlineto{\pgfqpoint{2.222896in}{1.071879in}}%
\pgfpathlineto{\pgfqpoint{2.223473in}{1.046349in}}%
\pgfpathclose%
\pgfusepath{fill}%
\end{pgfscope}%
\begin{pgfscope}%
\pgfpathrectangle{\pgfqpoint{0.100000in}{0.100000in}}{\pgfqpoint{3.420221in}{2.189500in}}%
\pgfusepath{clip}%
\pgfsetbuttcap%
\pgfsetmiterjoin%
\definecolor{currentfill}{rgb}{0.000000,0.309804,0.845098}%
\pgfsetfillcolor{currentfill}%
\pgfsetlinewidth{0.000000pt}%
\definecolor{currentstroke}{rgb}{0.000000,0.000000,0.000000}%
\pgfsetstrokecolor{currentstroke}%
\pgfsetstrokeopacity{0.000000}%
\pgfsetdash{}{0pt}%
\pgfpathmoveto{\pgfqpoint{2.198482in}{1.256419in}}%
\pgfpathlineto{\pgfqpoint{2.199375in}{1.225317in}}%
\pgfpathlineto{\pgfqpoint{2.166987in}{1.224499in}}%
\pgfpathlineto{\pgfqpoint{2.160602in}{1.227633in}}%
\pgfpathlineto{\pgfqpoint{2.154378in}{1.227806in}}%
\pgfpathlineto{\pgfqpoint{2.154487in}{1.234104in}}%
\pgfpathlineto{\pgfqpoint{2.141469in}{1.234157in}}%
\pgfpathlineto{\pgfqpoint{2.128460in}{1.246274in}}%
\pgfpathlineto{\pgfqpoint{2.128427in}{1.252806in}}%
\pgfpathlineto{\pgfqpoint{2.115532in}{1.253198in}}%
\pgfpathlineto{\pgfqpoint{2.115718in}{1.270824in}}%
\pgfpathlineto{\pgfqpoint{2.115763in}{1.274135in}}%
\pgfpathlineto{\pgfqpoint{2.122238in}{1.281182in}}%
\pgfpathlineto{\pgfqpoint{2.121693in}{1.284091in}}%
\pgfpathlineto{\pgfqpoint{2.126904in}{1.293377in}}%
\pgfpathlineto{\pgfqpoint{2.130115in}{1.302277in}}%
\pgfpathlineto{\pgfqpoint{2.134964in}{1.300751in}}%
\pgfpathlineto{\pgfqpoint{2.151174in}{1.295727in}}%
\pgfpathlineto{\pgfqpoint{2.157893in}{1.295761in}}%
\pgfpathlineto{\pgfqpoint{2.170080in}{1.295630in}}%
\pgfpathlineto{\pgfqpoint{2.170166in}{1.282459in}}%
\pgfpathlineto{\pgfqpoint{2.197868in}{1.283028in}}%
\pgfpathclose%
\pgfusepath{fill}%
\end{pgfscope}%
\begin{pgfscope}%
\pgfpathrectangle{\pgfqpoint{0.100000in}{0.100000in}}{\pgfqpoint{3.420221in}{2.189500in}}%
\pgfusepath{clip}%
\pgfsetbuttcap%
\pgfsetmiterjoin%
\definecolor{currentfill}{rgb}{0.000000,0.474510,0.762745}%
\pgfsetfillcolor{currentfill}%
\pgfsetlinewidth{0.000000pt}%
\definecolor{currentstroke}{rgb}{0.000000,0.000000,0.000000}%
\pgfsetstrokecolor{currentstroke}%
\pgfsetstrokeopacity{0.000000}%
\pgfsetdash{}{0pt}%
\pgfpathmoveto{\pgfqpoint{2.835475in}{0.761421in}}%
\pgfpathlineto{\pgfqpoint{2.833414in}{0.767835in}}%
\pgfpathlineto{\pgfqpoint{2.828555in}{0.767904in}}%
\pgfpathlineto{\pgfqpoint{2.825306in}{0.790493in}}%
\pgfpathlineto{\pgfqpoint{2.828880in}{0.795948in}}%
\pgfpathlineto{\pgfqpoint{2.825742in}{0.798969in}}%
\pgfpathlineto{\pgfqpoint{2.817994in}{0.800107in}}%
\pgfpathlineto{\pgfqpoint{2.818056in}{0.819380in}}%
\pgfpathlineto{\pgfqpoint{2.813198in}{0.830343in}}%
\pgfpathlineto{\pgfqpoint{2.840395in}{0.828065in}}%
\pgfpathlineto{\pgfqpoint{2.843479in}{0.822023in}}%
\pgfpathlineto{\pgfqpoint{2.845603in}{0.818704in}}%
\pgfpathlineto{\pgfqpoint{2.853574in}{0.815697in}}%
\pgfpathlineto{\pgfqpoint{2.859351in}{0.809674in}}%
\pgfpathlineto{\pgfqpoint{2.868518in}{0.809395in}}%
\pgfpathlineto{\pgfqpoint{2.872457in}{0.803146in}}%
\pgfpathlineto{\pgfqpoint{2.880736in}{0.797338in}}%
\pgfpathlineto{\pgfqpoint{2.886313in}{0.795377in}}%
\pgfpathlineto{\pgfqpoint{2.890614in}{0.789094in}}%
\pgfpathlineto{\pgfqpoint{2.887327in}{0.783872in}}%
\pgfpathlineto{\pgfqpoint{2.882045in}{0.789449in}}%
\pgfpathlineto{\pgfqpoint{2.871330in}{0.787092in}}%
\pgfpathlineto{\pgfqpoint{2.866872in}{0.790921in}}%
\pgfpathlineto{\pgfqpoint{2.863488in}{0.781076in}}%
\pgfpathlineto{\pgfqpoint{2.857466in}{0.777143in}}%
\pgfpathlineto{\pgfqpoint{2.861733in}{0.772666in}}%
\pgfpathlineto{\pgfqpoint{2.856140in}{0.760891in}}%
\pgfpathlineto{\pgfqpoint{2.850887in}{0.762924in}}%
\pgfpathlineto{\pgfqpoint{2.843287in}{0.760451in}}%
\pgfpathclose%
\pgfusepath{fill}%
\end{pgfscope}%
\begin{pgfscope}%
\pgfpathrectangle{\pgfqpoint{0.100000in}{0.100000in}}{\pgfqpoint{3.420221in}{2.189500in}}%
\pgfusepath{clip}%
\pgfsetbuttcap%
\pgfsetmiterjoin%
\definecolor{currentfill}{rgb}{0.000000,0.207843,0.896078}%
\pgfsetfillcolor{currentfill}%
\pgfsetlinewidth{0.000000pt}%
\definecolor{currentstroke}{rgb}{0.000000,0.000000,0.000000}%
\pgfsetstrokecolor{currentstroke}%
\pgfsetstrokeopacity{0.000000}%
\pgfsetdash{}{0pt}%
\pgfpathmoveto{\pgfqpoint{1.924742in}{1.724212in}}%
\pgfpathlineto{\pgfqpoint{1.924343in}{1.698139in}}%
\pgfpathlineto{\pgfqpoint{1.924224in}{1.691542in}}%
\pgfpathlineto{\pgfqpoint{1.887960in}{1.692218in}}%
\pgfpathlineto{\pgfqpoint{1.849082in}{1.693376in}}%
\pgfpathlineto{\pgfqpoint{1.849126in}{1.699917in}}%
\pgfpathlineto{\pgfqpoint{1.842625in}{1.700167in}}%
\pgfpathlineto{\pgfqpoint{1.843697in}{1.739436in}}%
\pgfpathlineto{\pgfqpoint{1.883810in}{1.738175in}}%
\pgfpathlineto{\pgfqpoint{1.902028in}{1.737645in}}%
\pgfpathlineto{\pgfqpoint{1.901819in}{1.724583in}}%
\pgfpathclose%
\pgfusepath{fill}%
\end{pgfscope}%
\begin{pgfscope}%
\pgfpathrectangle{\pgfqpoint{0.100000in}{0.100000in}}{\pgfqpoint{3.420221in}{2.189500in}}%
\pgfusepath{clip}%
\pgfsetbuttcap%
\pgfsetmiterjoin%
\definecolor{currentfill}{rgb}{0.000000,0.462745,0.768627}%
\pgfsetfillcolor{currentfill}%
\pgfsetlinewidth{0.000000pt}%
\definecolor{currentstroke}{rgb}{0.000000,0.000000,0.000000}%
\pgfsetstrokecolor{currentstroke}%
\pgfsetstrokeopacity{0.000000}%
\pgfsetdash{}{0pt}%
\pgfpathmoveto{\pgfqpoint{1.824320in}{0.727207in}}%
\pgfpathlineto{\pgfqpoint{1.791205in}{0.708951in}}%
\pgfpathlineto{\pgfqpoint{1.777808in}{0.732674in}}%
\pgfpathlineto{\pgfqpoint{1.766463in}{0.731376in}}%
\pgfpathlineto{\pgfqpoint{1.751262in}{0.760536in}}%
\pgfpathlineto{\pgfqpoint{1.780628in}{0.776032in}}%
\pgfpathlineto{\pgfqpoint{1.781217in}{0.792157in}}%
\pgfpathlineto{\pgfqpoint{1.807050in}{0.791120in}}%
\pgfpathlineto{\pgfqpoint{1.814180in}{0.769894in}}%
\pgfpathlineto{\pgfqpoint{1.818986in}{0.758709in}}%
\pgfpathlineto{\pgfqpoint{1.809836in}{0.753750in}}%
\pgfpathclose%
\pgfusepath{fill}%
\end{pgfscope}%
\begin{pgfscope}%
\pgfpathrectangle{\pgfqpoint{0.100000in}{0.100000in}}{\pgfqpoint{3.420221in}{2.189500in}}%
\pgfusepath{clip}%
\pgfsetbuttcap%
\pgfsetmiterjoin%
\definecolor{currentfill}{rgb}{0.000000,0.341176,0.829412}%
\pgfsetfillcolor{currentfill}%
\pgfsetlinewidth{0.000000pt}%
\definecolor{currentstroke}{rgb}{0.000000,0.000000,0.000000}%
\pgfsetstrokecolor{currentstroke}%
\pgfsetstrokeopacity{0.000000}%
\pgfsetdash{}{0pt}%
\pgfpathmoveto{\pgfqpoint{2.954170in}{1.365655in}}%
\pgfpathlineto{\pgfqpoint{2.959744in}{1.364453in}}%
\pgfpathlineto{\pgfqpoint{2.961721in}{1.369893in}}%
\pgfpathlineto{\pgfqpoint{2.968001in}{1.364489in}}%
\pgfpathlineto{\pgfqpoint{2.969651in}{1.359892in}}%
\pgfpathlineto{\pgfqpoint{2.966261in}{1.350625in}}%
\pgfpathlineto{\pgfqpoint{2.973074in}{1.346484in}}%
\pgfpathlineto{\pgfqpoint{2.971807in}{1.336194in}}%
\pgfpathlineto{\pgfqpoint{2.967394in}{1.326906in}}%
\pgfpathlineto{\pgfqpoint{2.965897in}{1.318284in}}%
\pgfpathlineto{\pgfqpoint{2.958482in}{1.305816in}}%
\pgfpathlineto{\pgfqpoint{2.953750in}{1.300847in}}%
\pgfpathlineto{\pgfqpoint{2.922709in}{1.315961in}}%
\pgfpathlineto{\pgfqpoint{2.918478in}{1.310204in}}%
\pgfpathlineto{\pgfqpoint{2.908393in}{1.311904in}}%
\pgfpathlineto{\pgfqpoint{2.903722in}{1.318222in}}%
\pgfpathlineto{\pgfqpoint{2.896731in}{1.320111in}}%
\pgfpathlineto{\pgfqpoint{2.902420in}{1.336455in}}%
\pgfpathlineto{\pgfqpoint{2.901161in}{1.342328in}}%
\pgfpathlineto{\pgfqpoint{2.909299in}{1.350100in}}%
\pgfpathlineto{\pgfqpoint{2.928652in}{1.338582in}}%
\pgfpathlineto{\pgfqpoint{2.931126in}{1.345770in}}%
\pgfpathlineto{\pgfqpoint{2.939441in}{1.340626in}}%
\pgfpathlineto{\pgfqpoint{2.942576in}{1.350644in}}%
\pgfpathlineto{\pgfqpoint{2.947296in}{1.353566in}}%
\pgfpathclose%
\pgfusepath{fill}%
\end{pgfscope}%
\begin{pgfscope}%
\pgfpathrectangle{\pgfqpoint{0.100000in}{0.100000in}}{\pgfqpoint{3.420221in}{2.189500in}}%
\pgfusepath{clip}%
\pgfsetbuttcap%
\pgfsetmiterjoin%
\definecolor{currentfill}{rgb}{0.000000,0.705882,0.647059}%
\pgfsetfillcolor{currentfill}%
\pgfsetlinewidth{0.000000pt}%
\definecolor{currentstroke}{rgb}{0.000000,0.000000,0.000000}%
\pgfsetstrokecolor{currentstroke}%
\pgfsetstrokeopacity{0.000000}%
\pgfsetdash{}{0pt}%
\pgfpathmoveto{\pgfqpoint{2.293907in}{1.105211in}}%
\pgfpathlineto{\pgfqpoint{2.290087in}{1.095036in}}%
\pgfpathlineto{\pgfqpoint{2.235543in}{1.092426in}}%
\pgfpathlineto{\pgfqpoint{2.235137in}{1.116873in}}%
\pgfpathlineto{\pgfqpoint{2.228601in}{1.116748in}}%
\pgfpathlineto{\pgfqpoint{2.228249in}{1.136281in}}%
\pgfpathlineto{\pgfqpoint{2.240130in}{1.137500in}}%
\pgfpathlineto{\pgfqpoint{2.243453in}{1.134489in}}%
\pgfpathlineto{\pgfqpoint{2.254515in}{1.134747in}}%
\pgfpathlineto{\pgfqpoint{2.256642in}{1.143558in}}%
\pgfpathlineto{\pgfqpoint{2.256375in}{1.151396in}}%
\pgfpathlineto{\pgfqpoint{2.267382in}{1.152042in}}%
\pgfpathlineto{\pgfqpoint{2.267269in}{1.155412in}}%
\pgfpathlineto{\pgfqpoint{2.287048in}{1.156221in}}%
\pgfpathlineto{\pgfqpoint{2.287615in}{1.139098in}}%
\pgfpathlineto{\pgfqpoint{2.294358in}{1.139414in}}%
\pgfpathlineto{\pgfqpoint{2.286141in}{1.126680in}}%
\pgfpathlineto{\pgfqpoint{2.292963in}{1.112090in}}%
\pgfpathclose%
\pgfusepath{fill}%
\end{pgfscope}%
\begin{pgfscope}%
\pgfpathrectangle{\pgfqpoint{0.100000in}{0.100000in}}{\pgfqpoint{3.420221in}{2.189500in}}%
\pgfusepath{clip}%
\pgfsetbuttcap%
\pgfsetmiterjoin%
\definecolor{currentfill}{rgb}{0.000000,0.443137,0.778431}%
\pgfsetfillcolor{currentfill}%
\pgfsetlinewidth{0.000000pt}%
\definecolor{currentstroke}{rgb}{0.000000,0.000000,0.000000}%
\pgfsetstrokecolor{currentstroke}%
\pgfsetstrokeopacity{0.000000}%
\pgfsetdash{}{0pt}%
\pgfpathmoveto{\pgfqpoint{1.500921in}{2.055722in}}%
\pgfpathlineto{\pgfqpoint{1.551020in}{2.050490in}}%
\pgfpathlineto{\pgfqpoint{1.546498in}{2.004294in}}%
\pgfpathlineto{\pgfqpoint{1.517336in}{2.007286in}}%
\pgfpathlineto{\pgfqpoint{1.518050in}{2.013865in}}%
\pgfpathlineto{\pgfqpoint{1.511536in}{2.014579in}}%
\pgfpathlineto{\pgfqpoint{1.512254in}{2.021118in}}%
\pgfpathlineto{\pgfqpoint{1.501467in}{2.022301in}}%
\pgfpathlineto{\pgfqpoint{1.498935in}{2.029236in}}%
\pgfpathlineto{\pgfqpoint{1.501131in}{2.048913in}}%
\pgfpathclose%
\pgfusepath{fill}%
\end{pgfscope}%
\begin{pgfscope}%
\pgfpathrectangle{\pgfqpoint{0.100000in}{0.100000in}}{\pgfqpoint{3.420221in}{2.189500in}}%
\pgfusepath{clip}%
\pgfsetbuttcap%
\pgfsetmiterjoin%
\definecolor{currentfill}{rgb}{0.000000,0.682353,0.658824}%
\pgfsetfillcolor{currentfill}%
\pgfsetlinewidth{0.000000pt}%
\definecolor{currentstroke}{rgb}{0.000000,0.000000,0.000000}%
\pgfsetstrokecolor{currentstroke}%
\pgfsetstrokeopacity{0.000000}%
\pgfsetdash{}{0pt}%
\pgfpathmoveto{\pgfqpoint{2.013137in}{1.861460in}}%
\pgfpathlineto{\pgfqpoint{1.993682in}{1.861514in}}%
\pgfpathlineto{\pgfqpoint{1.993634in}{1.854921in}}%
\pgfpathlineto{\pgfqpoint{1.941226in}{1.855330in}}%
\pgfpathlineto{\pgfqpoint{1.941510in}{1.874960in}}%
\pgfpathlineto{\pgfqpoint{1.940614in}{1.888108in}}%
\pgfpathlineto{\pgfqpoint{1.947184in}{1.888071in}}%
\pgfpathlineto{\pgfqpoint{1.947469in}{1.914387in}}%
\pgfpathlineto{\pgfqpoint{1.973742in}{1.914239in}}%
\pgfpathlineto{\pgfqpoint{1.972565in}{1.927375in}}%
\pgfpathlineto{\pgfqpoint{1.972522in}{1.947123in}}%
\pgfpathlineto{\pgfqpoint{1.972039in}{1.953696in}}%
\pgfpathlineto{\pgfqpoint{1.971800in}{1.993054in}}%
\pgfpathlineto{\pgfqpoint{1.991441in}{1.993053in}}%
\pgfpathlineto{\pgfqpoint{1.991462in}{1.979833in}}%
\pgfpathlineto{\pgfqpoint{2.030936in}{1.980015in}}%
\pgfpathlineto{\pgfqpoint{2.031703in}{1.940437in}}%
\pgfpathlineto{\pgfqpoint{2.032024in}{1.907545in}}%
\pgfpathlineto{\pgfqpoint{2.019014in}{1.907352in}}%
\pgfpathlineto{\pgfqpoint{2.019715in}{1.861437in}}%
\pgfpathclose%
\pgfusepath{fill}%
\end{pgfscope}%
\begin{pgfscope}%
\pgfpathrectangle{\pgfqpoint{0.100000in}{0.100000in}}{\pgfqpoint{3.420221in}{2.189500in}}%
\pgfusepath{clip}%
\pgfsetbuttcap%
\pgfsetmiterjoin%
\definecolor{currentfill}{rgb}{0.000000,0.486275,0.756863}%
\pgfsetfillcolor{currentfill}%
\pgfsetlinewidth{0.000000pt}%
\definecolor{currentstroke}{rgb}{0.000000,0.000000,0.000000}%
\pgfsetstrokecolor{currentstroke}%
\pgfsetstrokeopacity{0.000000}%
\pgfsetdash{}{0pt}%
\pgfpathmoveto{\pgfqpoint{2.159436in}{0.568114in}}%
\pgfpathlineto{\pgfqpoint{2.158596in}{0.602883in}}%
\pgfpathlineto{\pgfqpoint{2.156522in}{0.636574in}}%
\pgfpathlineto{\pgfqpoint{2.158706in}{0.644078in}}%
\pgfpathlineto{\pgfqpoint{2.158051in}{0.667872in}}%
\pgfpathlineto{\pgfqpoint{2.165133in}{0.668959in}}%
\pgfpathlineto{\pgfqpoint{2.171349in}{0.676213in}}%
\pgfpathlineto{\pgfqpoint{2.178535in}{0.671466in}}%
\pgfpathlineto{\pgfqpoint{2.181526in}{0.664915in}}%
\pgfpathlineto{\pgfqpoint{2.208955in}{0.665805in}}%
\pgfpathlineto{\pgfqpoint{2.214650in}{0.654922in}}%
\pgfpathlineto{\pgfqpoint{2.213484in}{0.644465in}}%
\pgfpathlineto{\pgfqpoint{2.215029in}{0.640276in}}%
\pgfpathlineto{\pgfqpoint{2.221461in}{0.635545in}}%
\pgfpathlineto{\pgfqpoint{2.223078in}{0.625539in}}%
\pgfpathlineto{\pgfqpoint{2.227083in}{0.620533in}}%
\pgfpathlineto{\pgfqpoint{2.232841in}{0.619809in}}%
\pgfpathlineto{\pgfqpoint{2.233966in}{0.610219in}}%
\pgfpathlineto{\pgfqpoint{2.249902in}{0.605006in}}%
\pgfpathlineto{\pgfqpoint{2.247683in}{0.603007in}}%
\pgfpathlineto{\pgfqpoint{2.253086in}{0.591581in}}%
\pgfpathlineto{\pgfqpoint{2.257979in}{0.590256in}}%
\pgfpathlineto{\pgfqpoint{2.259629in}{0.584935in}}%
\pgfpathlineto{\pgfqpoint{2.253669in}{0.582638in}}%
\pgfpathlineto{\pgfqpoint{2.239996in}{0.586258in}}%
\pgfpathlineto{\pgfqpoint{2.240688in}{0.590137in}}%
\pgfpathlineto{\pgfqpoint{2.233865in}{0.599098in}}%
\pgfpathlineto{\pgfqpoint{2.223865in}{0.597905in}}%
\pgfpathlineto{\pgfqpoint{2.218461in}{0.590176in}}%
\pgfpathlineto{\pgfqpoint{2.207653in}{0.580053in}}%
\pgfpathlineto{\pgfqpoint{2.207271in}{0.588912in}}%
\pgfpathlineto{\pgfqpoint{2.197167in}{0.584528in}}%
\pgfpathlineto{\pgfqpoint{2.190261in}{0.579344in}}%
\pgfpathlineto{\pgfqpoint{2.192396in}{0.572614in}}%
\pgfpathlineto{\pgfqpoint{2.200259in}{0.571735in}}%
\pgfpathlineto{\pgfqpoint{2.210569in}{0.573434in}}%
\pgfpathlineto{\pgfqpoint{2.219394in}{0.568134in}}%
\pgfpathlineto{\pgfqpoint{2.212338in}{0.561782in}}%
\pgfpathlineto{\pgfqpoint{2.196304in}{0.569285in}}%
\pgfpathlineto{\pgfqpoint{2.189924in}{0.569190in}}%
\pgfpathlineto{\pgfqpoint{2.180048in}{0.565111in}}%
\pgfpathclose%
\pgfusepath{fill}%
\end{pgfscope}%
\begin{pgfscope}%
\pgfpathrectangle{\pgfqpoint{0.100000in}{0.100000in}}{\pgfqpoint{3.420221in}{2.189500in}}%
\pgfusepath{clip}%
\pgfsetbuttcap%
\pgfsetmiterjoin%
\definecolor{currentfill}{rgb}{0.000000,0.596078,0.701961}%
\pgfsetfillcolor{currentfill}%
\pgfsetlinewidth{0.000000pt}%
\definecolor{currentstroke}{rgb}{0.000000,0.000000,0.000000}%
\pgfsetstrokecolor{currentstroke}%
\pgfsetstrokeopacity{0.000000}%
\pgfsetdash{}{0pt}%
\pgfpathmoveto{\pgfqpoint{1.746239in}{0.455579in}}%
\pgfpathlineto{\pgfqpoint{1.706770in}{0.455293in}}%
\pgfpathlineto{\pgfqpoint{1.707060in}{0.468578in}}%
\pgfpathlineto{\pgfqpoint{1.708563in}{0.501769in}}%
\pgfpathlineto{\pgfqpoint{1.707675in}{0.501795in}}%
\pgfpathlineto{\pgfqpoint{1.709016in}{0.536062in}}%
\pgfpathlineto{\pgfqpoint{1.749425in}{0.534165in}}%
\pgfpathlineto{\pgfqpoint{1.748286in}{0.500487in}}%
\pgfpathclose%
\pgfusepath{fill}%
\end{pgfscope}%
\begin{pgfscope}%
\pgfpathrectangle{\pgfqpoint{0.100000in}{0.100000in}}{\pgfqpoint{3.420221in}{2.189500in}}%
\pgfusepath{clip}%
\pgfsetbuttcap%
\pgfsetmiterjoin%
\definecolor{currentfill}{rgb}{0.000000,0.545098,0.727451}%
\pgfsetfillcolor{currentfill}%
\pgfsetlinewidth{0.000000pt}%
\definecolor{currentstroke}{rgb}{0.000000,0.000000,0.000000}%
\pgfsetstrokecolor{currentstroke}%
\pgfsetstrokeopacity{0.000000}%
\pgfsetdash{}{0pt}%
\pgfpathmoveto{\pgfqpoint{2.816932in}{0.698517in}}%
\pgfpathlineto{\pgfqpoint{2.808819in}{0.697974in}}%
\pgfpathlineto{\pgfqpoint{2.806281in}{0.706921in}}%
\pgfpathlineto{\pgfqpoint{2.795571in}{0.706923in}}%
\pgfpathlineto{\pgfqpoint{2.789415in}{0.713911in}}%
\pgfpathlineto{\pgfqpoint{2.781026in}{0.715406in}}%
\pgfpathlineto{\pgfqpoint{2.777889in}{0.739034in}}%
\pgfpathlineto{\pgfqpoint{2.773048in}{0.738346in}}%
\pgfpathlineto{\pgfqpoint{2.772985in}{0.745298in}}%
\pgfpathlineto{\pgfqpoint{2.765122in}{0.753714in}}%
\pgfpathlineto{\pgfqpoint{2.763856in}{0.759120in}}%
\pgfpathlineto{\pgfqpoint{2.766010in}{0.759858in}}%
\pgfpathlineto{\pgfqpoint{2.768039in}{0.770276in}}%
\pgfpathlineto{\pgfqpoint{2.771381in}{0.774463in}}%
\pgfpathlineto{\pgfqpoint{2.770382in}{0.783477in}}%
\pgfpathlineto{\pgfqpoint{2.778813in}{0.784810in}}%
\pgfpathlineto{\pgfqpoint{2.781599in}{0.776663in}}%
\pgfpathlineto{\pgfqpoint{2.804649in}{0.780507in}}%
\pgfpathlineto{\pgfqpoint{2.820739in}{0.773264in}}%
\pgfpathlineto{\pgfqpoint{2.828555in}{0.767904in}}%
\pgfpathlineto{\pgfqpoint{2.833414in}{0.767835in}}%
\pgfpathlineto{\pgfqpoint{2.835475in}{0.761421in}}%
\pgfpathlineto{\pgfqpoint{2.839273in}{0.756963in}}%
\pgfpathlineto{\pgfqpoint{2.833319in}{0.753762in}}%
\pgfpathlineto{\pgfqpoint{2.827447in}{0.747545in}}%
\pgfpathlineto{\pgfqpoint{2.827837in}{0.738300in}}%
\pgfpathlineto{\pgfqpoint{2.833520in}{0.733453in}}%
\pgfpathlineto{\pgfqpoint{2.815263in}{0.731093in}}%
\pgfpathlineto{\pgfqpoint{2.817254in}{0.714675in}}%
\pgfpathlineto{\pgfqpoint{2.834781in}{0.716309in}}%
\pgfpathlineto{\pgfqpoint{2.832894in}{0.699556in}}%
\pgfpathclose%
\pgfusepath{fill}%
\end{pgfscope}%
\begin{pgfscope}%
\pgfpathrectangle{\pgfqpoint{0.100000in}{0.100000in}}{\pgfqpoint{3.420221in}{2.189500in}}%
\pgfusepath{clip}%
\pgfsetbuttcap%
\pgfsetmiterjoin%
\definecolor{currentfill}{rgb}{0.000000,0.603922,0.698039}%
\pgfsetfillcolor{currentfill}%
\pgfsetlinewidth{0.000000pt}%
\definecolor{currentstroke}{rgb}{0.000000,0.000000,0.000000}%
\pgfsetstrokecolor{currentstroke}%
\pgfsetstrokeopacity{0.000000}%
\pgfsetdash{}{0pt}%
\pgfpathmoveto{\pgfqpoint{0.554402in}{1.837778in}}%
\pgfpathlineto{\pgfqpoint{0.546447in}{1.835654in}}%
\pgfpathlineto{\pgfqpoint{0.543810in}{1.827290in}}%
\pgfpathlineto{\pgfqpoint{0.512029in}{1.836928in}}%
\pgfpathlineto{\pgfqpoint{0.514402in}{1.844642in}}%
\pgfpathlineto{\pgfqpoint{0.495312in}{1.850341in}}%
\pgfpathlineto{\pgfqpoint{0.495287in}{1.855554in}}%
\pgfpathlineto{\pgfqpoint{0.499234in}{1.867999in}}%
\pgfpathlineto{\pgfqpoint{0.488394in}{1.871520in}}%
\pgfpathlineto{\pgfqpoint{0.480168in}{1.878077in}}%
\pgfpathlineto{\pgfqpoint{0.477629in}{1.886070in}}%
\pgfpathlineto{\pgfqpoint{0.467490in}{1.891357in}}%
\pgfpathlineto{\pgfqpoint{0.459487in}{1.887476in}}%
\pgfpathlineto{\pgfqpoint{0.448546in}{1.891138in}}%
\pgfpathlineto{\pgfqpoint{0.456062in}{1.907572in}}%
\pgfpathlineto{\pgfqpoint{0.460516in}{1.919845in}}%
\pgfpathlineto{\pgfqpoint{0.478103in}{1.914672in}}%
\pgfpathlineto{\pgfqpoint{0.482985in}{1.919170in}}%
\pgfpathlineto{\pgfqpoint{0.484414in}{1.924441in}}%
\pgfpathlineto{\pgfqpoint{0.490703in}{1.922376in}}%
\pgfpathlineto{\pgfqpoint{0.497148in}{1.943101in}}%
\pgfpathlineto{\pgfqpoint{0.491343in}{1.946410in}}%
\pgfpathlineto{\pgfqpoint{0.499270in}{1.970525in}}%
\pgfpathlineto{\pgfqpoint{0.496213in}{1.971544in}}%
\pgfpathlineto{\pgfqpoint{0.499511in}{1.981542in}}%
\pgfpathlineto{\pgfqpoint{0.515725in}{1.976277in}}%
\pgfpathlineto{\pgfqpoint{0.520695in}{1.991907in}}%
\pgfpathlineto{\pgfqpoint{0.537282in}{1.986525in}}%
\pgfpathlineto{\pgfqpoint{0.541838in}{1.982818in}}%
\pgfpathlineto{\pgfqpoint{0.543609in}{1.977629in}}%
\pgfpathlineto{\pgfqpoint{0.551340in}{1.969463in}}%
\pgfpathlineto{\pgfqpoint{0.550890in}{1.962307in}}%
\pgfpathlineto{\pgfqpoint{0.547891in}{1.959164in}}%
\pgfpathlineto{\pgfqpoint{0.550476in}{1.952763in}}%
\pgfpathlineto{\pgfqpoint{0.555164in}{1.948162in}}%
\pgfpathlineto{\pgfqpoint{0.556365in}{1.941287in}}%
\pgfpathlineto{\pgfqpoint{0.562381in}{1.935369in}}%
\pgfpathlineto{\pgfqpoint{0.596123in}{1.925186in}}%
\pgfpathlineto{\pgfqpoint{0.593489in}{1.921058in}}%
\pgfpathlineto{\pgfqpoint{0.589123in}{1.914813in}}%
\pgfpathlineto{\pgfqpoint{0.586720in}{1.904890in}}%
\pgfpathlineto{\pgfqpoint{0.581754in}{1.898328in}}%
\pgfpathlineto{\pgfqpoint{0.579024in}{1.889342in}}%
\pgfpathlineto{\pgfqpoint{0.579348in}{1.880020in}}%
\pgfpathlineto{\pgfqpoint{0.576768in}{1.868288in}}%
\pgfpathlineto{\pgfqpoint{0.569421in}{1.860413in}}%
\pgfpathlineto{\pgfqpoint{0.561131in}{1.854715in}}%
\pgfpathlineto{\pgfqpoint{0.559843in}{1.847793in}}%
\pgfpathclose%
\pgfusepath{fill}%
\end{pgfscope}%
\begin{pgfscope}%
\pgfpathrectangle{\pgfqpoint{0.100000in}{0.100000in}}{\pgfqpoint{3.420221in}{2.189500in}}%
\pgfusepath{clip}%
\pgfsetbuttcap%
\pgfsetmiterjoin%
\definecolor{currentfill}{rgb}{0.000000,0.447059,0.776471}%
\pgfsetfillcolor{currentfill}%
\pgfsetlinewidth{0.000000pt}%
\definecolor{currentstroke}{rgb}{0.000000,0.000000,0.000000}%
\pgfsetstrokecolor{currentstroke}%
\pgfsetstrokeopacity{0.000000}%
\pgfsetdash{}{0pt}%
\pgfpathmoveto{\pgfqpoint{2.626172in}{0.839440in}}%
\pgfpathlineto{\pgfqpoint{2.616847in}{0.837234in}}%
\pgfpathlineto{\pgfqpoint{2.597438in}{0.835018in}}%
\pgfpathlineto{\pgfqpoint{2.594488in}{0.863405in}}%
\pgfpathlineto{\pgfqpoint{2.590699in}{0.862960in}}%
\pgfpathlineto{\pgfqpoint{2.588324in}{0.892178in}}%
\pgfpathlineto{\pgfqpoint{2.580690in}{0.891599in}}%
\pgfpathlineto{\pgfqpoint{2.578210in}{0.897922in}}%
\pgfpathlineto{\pgfqpoint{2.565704in}{0.896738in}}%
\pgfpathlineto{\pgfqpoint{2.561555in}{0.902991in}}%
\pgfpathlineto{\pgfqpoint{2.555629in}{0.902748in}}%
\pgfpathlineto{\pgfqpoint{2.560929in}{0.910799in}}%
\pgfpathlineto{\pgfqpoint{2.559427in}{0.915451in}}%
\pgfpathlineto{\pgfqpoint{2.570022in}{0.924678in}}%
\pgfpathlineto{\pgfqpoint{2.578746in}{0.926936in}}%
\pgfpathlineto{\pgfqpoint{2.591844in}{0.927347in}}%
\pgfpathlineto{\pgfqpoint{2.599881in}{0.928810in}}%
\pgfpathlineto{\pgfqpoint{2.601137in}{0.924223in}}%
\pgfpathlineto{\pgfqpoint{2.621985in}{0.926685in}}%
\pgfpathlineto{\pgfqpoint{2.623515in}{0.919850in}}%
\pgfpathlineto{\pgfqpoint{2.632240in}{0.918452in}}%
\pgfpathlineto{\pgfqpoint{2.633704in}{0.902920in}}%
\pgfpathlineto{\pgfqpoint{2.639090in}{0.898974in}}%
\pgfpathlineto{\pgfqpoint{2.651572in}{0.900217in}}%
\pgfpathlineto{\pgfqpoint{2.655481in}{0.889099in}}%
\pgfpathlineto{\pgfqpoint{2.661897in}{0.882196in}}%
\pgfpathlineto{\pgfqpoint{2.661915in}{0.879426in}}%
\pgfpathlineto{\pgfqpoint{2.639977in}{0.877124in}}%
\pgfpathlineto{\pgfqpoint{2.642360in}{0.850476in}}%
\pgfpathlineto{\pgfqpoint{2.622072in}{0.848394in}}%
\pgfpathclose%
\pgfusepath{fill}%
\end{pgfscope}%
\begin{pgfscope}%
\pgfpathrectangle{\pgfqpoint{0.100000in}{0.100000in}}{\pgfqpoint{3.420221in}{2.189500in}}%
\pgfusepath{clip}%
\pgfsetbuttcap%
\pgfsetmiterjoin%
\definecolor{currentfill}{rgb}{0.000000,0.376471,0.811765}%
\pgfsetfillcolor{currentfill}%
\pgfsetlinewidth{0.000000pt}%
\definecolor{currentstroke}{rgb}{0.000000,0.000000,0.000000}%
\pgfsetstrokecolor{currentstroke}%
\pgfsetstrokeopacity{0.000000}%
\pgfsetdash{}{0pt}%
\pgfpathmoveto{\pgfqpoint{1.242917in}{1.237047in}}%
\pgfpathlineto{\pgfqpoint{1.236242in}{1.244152in}}%
\pgfpathlineto{\pgfqpoint{1.222934in}{1.243386in}}%
\pgfpathlineto{\pgfqpoint{1.221021in}{1.249232in}}%
\pgfpathlineto{\pgfqpoint{1.188265in}{1.254218in}}%
\pgfpathlineto{\pgfqpoint{1.174971in}{1.255182in}}%
\pgfpathlineto{\pgfqpoint{1.178225in}{1.276176in}}%
\pgfpathlineto{\pgfqpoint{1.178440in}{1.284544in}}%
\pgfpathlineto{\pgfqpoint{1.181037in}{1.301162in}}%
\pgfpathlineto{\pgfqpoint{1.191570in}{1.365236in}}%
\pgfpathlineto{\pgfqpoint{1.228273in}{1.359595in}}%
\pgfpathlineto{\pgfqpoint{1.284383in}{1.351579in}}%
\pgfpathlineto{\pgfqpoint{1.281886in}{1.348003in}}%
\pgfpathlineto{\pgfqpoint{1.285356in}{1.343136in}}%
\pgfpathlineto{\pgfqpoint{1.278839in}{1.341129in}}%
\pgfpathlineto{\pgfqpoint{1.269489in}{1.273146in}}%
\pgfpathlineto{\pgfqpoint{1.261626in}{1.274238in}}%
\pgfpathlineto{\pgfqpoint{1.264081in}{1.268519in}}%
\pgfpathlineto{\pgfqpoint{1.264044in}{1.257078in}}%
\pgfpathlineto{\pgfqpoint{1.261999in}{1.248700in}}%
\pgfpathlineto{\pgfqpoint{1.251464in}{1.245668in}}%
\pgfpathclose%
\pgfusepath{fill}%
\end{pgfscope}%
\begin{pgfscope}%
\pgfpathrectangle{\pgfqpoint{0.100000in}{0.100000in}}{\pgfqpoint{3.420221in}{2.189500in}}%
\pgfusepath{clip}%
\pgfsetbuttcap%
\pgfsetmiterjoin%
\definecolor{currentfill}{rgb}{0.000000,0.768627,0.615686}%
\pgfsetfillcolor{currentfill}%
\pgfsetlinewidth{0.000000pt}%
\definecolor{currentstroke}{rgb}{0.000000,0.000000,0.000000}%
\pgfsetstrokecolor{currentstroke}%
\pgfsetstrokeopacity{0.000000}%
\pgfsetdash{}{0pt}%
\pgfpathmoveto{\pgfqpoint{1.070657in}{0.535615in}}%
\pgfpathlineto{\pgfqpoint{1.057682in}{0.543496in}}%
\pgfpathlineto{\pgfqpoint{1.031807in}{0.559415in}}%
\pgfpathlineto{\pgfqpoint{1.023374in}{0.564348in}}%
\pgfpathlineto{\pgfqpoint{1.015804in}{0.553076in}}%
\pgfpathlineto{\pgfqpoint{1.006548in}{0.558967in}}%
\pgfpathlineto{\pgfqpoint{1.002494in}{0.560994in}}%
\pgfpathlineto{\pgfqpoint{0.978272in}{0.577564in}}%
\pgfpathlineto{\pgfqpoint{0.951054in}{0.597366in}}%
\pgfpathlineto{\pgfqpoint{0.930585in}{0.613220in}}%
\pgfpathlineto{\pgfqpoint{0.931090in}{0.613884in}}%
\pgfpathlineto{\pgfqpoint{0.922894in}{0.620349in}}%
\pgfpathlineto{\pgfqpoint{0.924458in}{0.622307in}}%
\pgfpathlineto{\pgfqpoint{0.916545in}{0.628787in}}%
\pgfpathlineto{\pgfqpoint{0.918248in}{0.630876in}}%
\pgfpathlineto{\pgfqpoint{0.909903in}{0.637743in}}%
\pgfpathlineto{\pgfqpoint{0.896331in}{0.649277in}}%
\pgfpathlineto{\pgfqpoint{0.897659in}{0.650815in}}%
\pgfpathlineto{\pgfqpoint{0.894073in}{0.653926in}}%
\pgfpathlineto{\pgfqpoint{0.892735in}{0.652391in}}%
\pgfpathlineto{\pgfqpoint{0.884608in}{0.659555in}}%
\pgfpathlineto{\pgfqpoint{0.885972in}{0.661087in}}%
\pgfpathlineto{\pgfqpoint{0.882484in}{0.664207in}}%
\pgfpathlineto{\pgfqpoint{0.881114in}{0.662682in}}%
\pgfpathlineto{\pgfqpoint{0.877126in}{0.666282in}}%
\pgfpathlineto{\pgfqpoint{0.878506in}{0.667801in}}%
\pgfpathlineto{\pgfqpoint{0.872610in}{0.673199in}}%
\pgfpathlineto{\pgfqpoint{0.871218in}{0.671691in}}%
\pgfpathlineto{\pgfqpoint{0.856807in}{0.685255in}}%
\pgfpathlineto{\pgfqpoint{0.852720in}{0.680997in}}%
\pgfpathlineto{\pgfqpoint{0.845701in}{0.687817in}}%
\pgfpathlineto{\pgfqpoint{0.842989in}{0.694539in}}%
\pgfpathlineto{\pgfqpoint{0.843814in}{0.697671in}}%
\pgfpathlineto{\pgfqpoint{0.843267in}{0.700900in}}%
\pgfpathlineto{\pgfqpoint{0.840962in}{0.705449in}}%
\pgfpathlineto{\pgfqpoint{0.844026in}{0.704249in}}%
\pgfpathlineto{\pgfqpoint{0.847267in}{0.704057in}}%
\pgfpathlineto{\pgfqpoint{0.849876in}{0.705569in}}%
\pgfpathlineto{\pgfqpoint{0.855660in}{0.711122in}}%
\pgfpathlineto{\pgfqpoint{0.857691in}{0.710456in}}%
\pgfpathlineto{\pgfqpoint{0.862133in}{0.705665in}}%
\pgfpathlineto{\pgfqpoint{0.872110in}{0.699038in}}%
\pgfpathlineto{\pgfqpoint{0.875133in}{0.697835in}}%
\pgfpathlineto{\pgfqpoint{0.881677in}{0.697442in}}%
\pgfpathlineto{\pgfqpoint{0.890696in}{0.699266in}}%
\pgfpathlineto{\pgfqpoint{0.895224in}{0.702680in}}%
\pgfpathlineto{\pgfqpoint{0.896335in}{0.704481in}}%
\pgfpathlineto{\pgfqpoint{0.899531in}{0.707163in}}%
\pgfpathlineto{\pgfqpoint{0.901673in}{0.708032in}}%
\pgfpathlineto{\pgfqpoint{0.908408in}{0.707972in}}%
\pgfpathlineto{\pgfqpoint{0.910461in}{0.709994in}}%
\pgfpathlineto{\pgfqpoint{0.914912in}{0.710348in}}%
\pgfpathlineto{\pgfqpoint{0.920226in}{0.711506in}}%
\pgfpathlineto{\pgfqpoint{0.923264in}{0.707203in}}%
\pgfpathlineto{\pgfqpoint{0.928934in}{0.705881in}}%
\pgfpathlineto{\pgfqpoint{0.938985in}{0.706510in}}%
\pgfpathlineto{\pgfqpoint{0.949250in}{0.708287in}}%
\pgfpathlineto{\pgfqpoint{0.952611in}{0.706755in}}%
\pgfpathlineto{\pgfqpoint{0.953018in}{0.704412in}}%
\pgfpathlineto{\pgfqpoint{0.957026in}{0.701189in}}%
\pgfpathlineto{\pgfqpoint{0.960854in}{0.699035in}}%
\pgfpathlineto{\pgfqpoint{0.964076in}{0.698021in}}%
\pgfpathlineto{\pgfqpoint{0.970846in}{0.698536in}}%
\pgfpathlineto{\pgfqpoint{0.979809in}{0.701857in}}%
\pgfpathlineto{\pgfqpoint{0.982886in}{0.702325in}}%
\pgfpathlineto{\pgfqpoint{0.984401in}{0.697651in}}%
\pgfpathlineto{\pgfqpoint{0.983583in}{0.694683in}}%
\pgfpathlineto{\pgfqpoint{0.985921in}{0.694735in}}%
\pgfpathlineto{\pgfqpoint{0.987178in}{0.692462in}}%
\pgfpathlineto{\pgfqpoint{0.987192in}{0.689773in}}%
\pgfpathlineto{\pgfqpoint{0.982178in}{0.687830in}}%
\pgfpathlineto{\pgfqpoint{0.978325in}{0.689224in}}%
\pgfpathlineto{\pgfqpoint{0.978929in}{0.684819in}}%
\pgfpathlineto{\pgfqpoint{0.982839in}{0.683294in}}%
\pgfpathlineto{\pgfqpoint{0.985116in}{0.685133in}}%
\pgfpathlineto{\pgfqpoint{0.990295in}{0.685030in}}%
\pgfpathlineto{\pgfqpoint{0.993015in}{0.681852in}}%
\pgfpathlineto{\pgfqpoint{0.989953in}{0.678364in}}%
\pgfpathlineto{\pgfqpoint{0.989380in}{0.676333in}}%
\pgfpathlineto{\pgfqpoint{0.991662in}{0.674651in}}%
\pgfpathlineto{\pgfqpoint{0.992049in}{0.672318in}}%
\pgfpathlineto{\pgfqpoint{0.995276in}{0.673190in}}%
\pgfpathlineto{\pgfqpoint{0.999790in}{0.670932in}}%
\pgfpathlineto{\pgfqpoint{1.002962in}{0.669995in}}%
\pgfpathlineto{\pgfqpoint{1.004980in}{0.664980in}}%
\pgfpathlineto{\pgfqpoint{1.006988in}{0.665220in}}%
\pgfpathlineto{\pgfqpoint{1.008722in}{0.662376in}}%
\pgfpathlineto{\pgfqpoint{1.002922in}{0.658343in}}%
\pgfpathlineto{\pgfqpoint{1.001193in}{0.656232in}}%
\pgfpathlineto{\pgfqpoint{1.007335in}{0.652017in}}%
\pgfpathlineto{\pgfqpoint{1.006610in}{0.650487in}}%
\pgfpathlineto{\pgfqpoint{1.003605in}{0.650027in}}%
\pgfpathlineto{\pgfqpoint{1.008174in}{0.646561in}}%
\pgfpathlineto{\pgfqpoint{1.008647in}{0.643699in}}%
\pgfpathlineto{\pgfqpoint{1.011571in}{0.644963in}}%
\pgfpathlineto{\pgfqpoint{1.016249in}{0.643597in}}%
\pgfpathlineto{\pgfqpoint{1.016849in}{0.641836in}}%
\pgfpathlineto{\pgfqpoint{1.014585in}{0.640054in}}%
\pgfpathlineto{\pgfqpoint{1.019376in}{0.638710in}}%
\pgfpathlineto{\pgfqpoint{1.022612in}{0.639263in}}%
\pgfpathlineto{\pgfqpoint{1.026479in}{0.635205in}}%
\pgfpathlineto{\pgfqpoint{1.027632in}{0.635159in}}%
\pgfpathlineto{\pgfqpoint{1.028549in}{0.630380in}}%
\pgfpathlineto{\pgfqpoint{1.031525in}{0.628809in}}%
\pgfpathlineto{\pgfqpoint{1.031642in}{0.626294in}}%
\pgfpathlineto{\pgfqpoint{1.033816in}{0.625045in}}%
\pgfpathlineto{\pgfqpoint{1.034288in}{0.620465in}}%
\pgfpathlineto{\pgfqpoint{1.040607in}{0.612852in}}%
\pgfpathlineto{\pgfqpoint{1.043157in}{0.612771in}}%
\pgfpathlineto{\pgfqpoint{1.046740in}{0.610437in}}%
\pgfpathlineto{\pgfqpoint{1.048979in}{0.607352in}}%
\pgfpathlineto{\pgfqpoint{1.051732in}{0.605722in}}%
\pgfpathlineto{\pgfqpoint{1.054061in}{0.598655in}}%
\pgfpathlineto{\pgfqpoint{1.055967in}{0.596060in}}%
\pgfpathlineto{\pgfqpoint{1.061045in}{0.594770in}}%
\pgfpathlineto{\pgfqpoint{1.067724in}{0.592487in}}%
\pgfpathlineto{\pgfqpoint{1.070424in}{0.592265in}}%
\pgfpathlineto{\pgfqpoint{1.073491in}{0.590359in}}%
\pgfpathlineto{\pgfqpoint{1.076796in}{0.585626in}}%
\pgfpathlineto{\pgfqpoint{1.078845in}{0.580263in}}%
\pgfpathlineto{\pgfqpoint{1.078812in}{0.578701in}}%
\pgfpathlineto{\pgfqpoint{1.080849in}{0.574704in}}%
\pgfpathlineto{\pgfqpoint{1.083034in}{0.572004in}}%
\pgfpathlineto{\pgfqpoint{1.083835in}{0.569359in}}%
\pgfpathlineto{\pgfqpoint{1.086706in}{0.565972in}}%
\pgfpathlineto{\pgfqpoint{1.085507in}{0.562404in}}%
\pgfpathclose%
\pgfusepath{fill}%
\end{pgfscope}%
\begin{pgfscope}%
\pgfpathrectangle{\pgfqpoint{0.100000in}{0.100000in}}{\pgfqpoint{3.420221in}{2.189500in}}%
\pgfusepath{clip}%
\pgfsetbuttcap%
\pgfsetmiterjoin%
\definecolor{currentfill}{rgb}{0.000000,0.513725,0.743137}%
\pgfsetfillcolor{currentfill}%
\pgfsetlinewidth{0.000000pt}%
\definecolor{currentstroke}{rgb}{0.000000,0.000000,0.000000}%
\pgfsetstrokecolor{currentstroke}%
\pgfsetstrokeopacity{0.000000}%
\pgfsetdash{}{0pt}%
\pgfpathmoveto{\pgfqpoint{2.397160in}{1.169901in}}%
\pgfpathlineto{\pgfqpoint{2.393689in}{1.171067in}}%
\pgfpathlineto{\pgfqpoint{2.392914in}{1.184249in}}%
\pgfpathlineto{\pgfqpoint{2.375361in}{1.182955in}}%
\pgfpathlineto{\pgfqpoint{2.373872in}{1.205956in}}%
\pgfpathlineto{\pgfqpoint{2.393456in}{1.207469in}}%
\pgfpathlineto{\pgfqpoint{2.407485in}{1.208449in}}%
\pgfpathlineto{\pgfqpoint{2.410202in}{1.207405in}}%
\pgfpathlineto{\pgfqpoint{2.414510in}{1.200841in}}%
\pgfpathlineto{\pgfqpoint{2.407560in}{1.190110in}}%
\pgfpathlineto{\pgfqpoint{2.412858in}{1.176050in}}%
\pgfpathlineto{\pgfqpoint{2.401453in}{1.173805in}}%
\pgfpathclose%
\pgfusepath{fill}%
\end{pgfscope}%
\begin{pgfscope}%
\pgfpathrectangle{\pgfqpoint{0.100000in}{0.100000in}}{\pgfqpoint{3.420221in}{2.189500in}}%
\pgfusepath{clip}%
\pgfsetbuttcap%
\pgfsetmiterjoin%
\definecolor{currentfill}{rgb}{0.000000,0.619608,0.690196}%
\pgfsetfillcolor{currentfill}%
\pgfsetlinewidth{0.000000pt}%
\definecolor{currentstroke}{rgb}{0.000000,0.000000,0.000000}%
\pgfsetstrokecolor{currentstroke}%
\pgfsetstrokeopacity{0.000000}%
\pgfsetdash{}{0pt}%
\pgfpathmoveto{\pgfqpoint{2.344952in}{0.900296in}}%
\pgfpathlineto{\pgfqpoint{2.328628in}{0.899322in}}%
\pgfpathlineto{\pgfqpoint{2.318714in}{0.901066in}}%
\pgfpathlineto{\pgfqpoint{2.319013in}{0.895500in}}%
\pgfpathlineto{\pgfqpoint{2.305955in}{0.893690in}}%
\pgfpathlineto{\pgfqpoint{2.306289in}{0.887103in}}%
\pgfpathlineto{\pgfqpoint{2.295669in}{0.888124in}}%
\pgfpathlineto{\pgfqpoint{2.303078in}{0.893537in}}%
\pgfpathlineto{\pgfqpoint{2.286275in}{0.892826in}}%
\pgfpathlineto{\pgfqpoint{2.285679in}{0.905923in}}%
\pgfpathlineto{\pgfqpoint{2.272613in}{0.905427in}}%
\pgfpathlineto{\pgfqpoint{2.272198in}{0.915290in}}%
\pgfpathlineto{\pgfqpoint{2.253756in}{0.914534in}}%
\pgfpathlineto{\pgfqpoint{2.265059in}{0.927121in}}%
\pgfpathlineto{\pgfqpoint{2.265311in}{0.933660in}}%
\pgfpathlineto{\pgfqpoint{2.276030in}{0.937429in}}%
\pgfpathlineto{\pgfqpoint{2.276314in}{0.946045in}}%
\pgfpathlineto{\pgfqpoint{2.287201in}{0.938981in}}%
\pgfpathlineto{\pgfqpoint{2.293243in}{0.939346in}}%
\pgfpathlineto{\pgfqpoint{2.292761in}{0.945724in}}%
\pgfpathlineto{\pgfqpoint{2.299118in}{0.949428in}}%
\pgfpathlineto{\pgfqpoint{2.343730in}{0.951946in}}%
\pgfpathlineto{\pgfqpoint{2.358247in}{0.948440in}}%
\pgfpathlineto{\pgfqpoint{2.358787in}{0.939731in}}%
\pgfpathlineto{\pgfqpoint{2.359878in}{0.923352in}}%
\pgfpathlineto{\pgfqpoint{2.343465in}{0.922403in}}%
\pgfpathclose%
\pgfusepath{fill}%
\end{pgfscope}%
\begin{pgfscope}%
\pgfpathrectangle{\pgfqpoint{0.100000in}{0.100000in}}{\pgfqpoint{3.420221in}{2.189500in}}%
\pgfusepath{clip}%
\pgfsetbuttcap%
\pgfsetmiterjoin%
\definecolor{currentfill}{rgb}{0.000000,0.325490,0.837255}%
\pgfsetfillcolor{currentfill}%
\pgfsetlinewidth{0.000000pt}%
\definecolor{currentstroke}{rgb}{0.000000,0.000000,0.000000}%
\pgfsetstrokecolor{currentstroke}%
\pgfsetstrokeopacity{0.000000}%
\pgfsetdash{}{0pt}%
\pgfpathmoveto{\pgfqpoint{3.328383in}{1.796058in}}%
\pgfpathlineto{\pgfqpoint{3.319674in}{1.799207in}}%
\pgfpathlineto{\pgfqpoint{3.320763in}{1.802272in}}%
\pgfpathlineto{\pgfqpoint{3.314451in}{1.806262in}}%
\pgfpathlineto{\pgfqpoint{3.308989in}{1.803799in}}%
\pgfpathlineto{\pgfqpoint{3.307642in}{1.809971in}}%
\pgfpathlineto{\pgfqpoint{3.298667in}{1.809391in}}%
\pgfpathlineto{\pgfqpoint{3.289937in}{1.805540in}}%
\pgfpathlineto{\pgfqpoint{3.279179in}{1.840986in}}%
\pgfpathlineto{\pgfqpoint{3.267982in}{1.876426in}}%
\pgfpathlineto{\pgfqpoint{3.255324in}{1.914147in}}%
\pgfpathlineto{\pgfqpoint{3.261298in}{1.918329in}}%
\pgfpathlineto{\pgfqpoint{3.266044in}{1.912268in}}%
\pgfpathlineto{\pgfqpoint{3.269184in}{1.917343in}}%
\pgfpathlineto{\pgfqpoint{3.268321in}{1.927503in}}%
\pgfpathlineto{\pgfqpoint{3.272659in}{1.925520in}}%
\pgfpathlineto{\pgfqpoint{3.276357in}{1.928931in}}%
\pgfpathlineto{\pgfqpoint{3.269706in}{1.934439in}}%
\pgfpathlineto{\pgfqpoint{3.271789in}{1.942206in}}%
\pgfpathlineto{\pgfqpoint{3.282272in}{1.954605in}}%
\pgfpathlineto{\pgfqpoint{3.280773in}{1.961561in}}%
\pgfpathlineto{\pgfqpoint{3.284750in}{1.966533in}}%
\pgfpathlineto{\pgfqpoint{3.285522in}{1.972366in}}%
\pgfpathlineto{\pgfqpoint{3.280074in}{1.976784in}}%
\pgfpathlineto{\pgfqpoint{3.281594in}{1.987062in}}%
\pgfpathlineto{\pgfqpoint{3.277889in}{1.989618in}}%
\pgfpathlineto{\pgfqpoint{3.279277in}{2.000829in}}%
\pgfpathlineto{\pgfqpoint{3.285150in}{2.009268in}}%
\pgfpathlineto{\pgfqpoint{3.283579in}{2.021263in}}%
\pgfpathlineto{\pgfqpoint{3.298672in}{2.025538in}}%
\pgfpathlineto{\pgfqpoint{3.301852in}{2.012260in}}%
\pgfpathlineto{\pgfqpoint{3.310630in}{1.981972in}}%
\pgfpathlineto{\pgfqpoint{3.314371in}{1.973467in}}%
\pgfpathlineto{\pgfqpoint{3.313839in}{1.960523in}}%
\pgfpathlineto{\pgfqpoint{3.319320in}{1.957167in}}%
\pgfpathlineto{\pgfqpoint{3.317021in}{1.949824in}}%
\pgfpathlineto{\pgfqpoint{3.336174in}{1.913600in}}%
\pgfpathlineto{\pgfqpoint{3.348436in}{1.922049in}}%
\pgfpathlineto{\pgfqpoint{3.360381in}{1.897946in}}%
\pgfpathlineto{\pgfqpoint{3.353129in}{1.897669in}}%
\pgfpathlineto{\pgfqpoint{3.355601in}{1.890161in}}%
\pgfpathlineto{\pgfqpoint{3.356592in}{1.876617in}}%
\pgfpathlineto{\pgfqpoint{3.355948in}{1.866956in}}%
\pgfpathlineto{\pgfqpoint{3.359421in}{1.865700in}}%
\pgfpathlineto{\pgfqpoint{3.363923in}{1.858911in}}%
\pgfpathlineto{\pgfqpoint{3.367483in}{1.859839in}}%
\pgfpathlineto{\pgfqpoint{3.373752in}{1.849957in}}%
\pgfpathlineto{\pgfqpoint{3.370709in}{1.841180in}}%
\pgfpathlineto{\pgfqpoint{3.367221in}{1.840660in}}%
\pgfpathlineto{\pgfqpoint{3.366698in}{1.831215in}}%
\pgfpathlineto{\pgfqpoint{3.356482in}{1.825377in}}%
\pgfpathlineto{\pgfqpoint{3.355573in}{1.819997in}}%
\pgfpathlineto{\pgfqpoint{3.351037in}{1.816601in}}%
\pgfpathlineto{\pgfqpoint{3.338472in}{1.822219in}}%
\pgfpathlineto{\pgfqpoint{3.332306in}{1.815802in}}%
\pgfpathlineto{\pgfqpoint{3.330692in}{1.808299in}}%
\pgfpathlineto{\pgfqpoint{3.335975in}{1.800438in}}%
\pgfpathclose%
\pgfusepath{fill}%
\end{pgfscope}%
\begin{pgfscope}%
\pgfpathrectangle{\pgfqpoint{0.100000in}{0.100000in}}{\pgfqpoint{3.420221in}{2.189500in}}%
\pgfusepath{clip}%
\pgfsetbuttcap%
\pgfsetmiterjoin%
\definecolor{currentfill}{rgb}{0.000000,0.631373,0.684314}%
\pgfsetfillcolor{currentfill}%
\pgfsetlinewidth{0.000000pt}%
\definecolor{currentstroke}{rgb}{0.000000,0.000000,0.000000}%
\pgfsetstrokecolor{currentstroke}%
\pgfsetstrokeopacity{0.000000}%
\pgfsetdash{}{0pt}%
\pgfpathmoveto{\pgfqpoint{2.393064in}{0.832373in}}%
\pgfpathlineto{\pgfqpoint{2.361084in}{0.830708in}}%
\pgfpathlineto{\pgfqpoint{2.359592in}{0.857326in}}%
\pgfpathlineto{\pgfqpoint{2.350959in}{0.856816in}}%
\pgfpathlineto{\pgfqpoint{2.350267in}{0.867886in}}%
\pgfpathlineto{\pgfqpoint{2.346858in}{0.869680in}}%
\pgfpathlineto{\pgfqpoint{2.345624in}{0.889359in}}%
\pgfpathlineto{\pgfqpoint{2.365312in}{0.890611in}}%
\pgfpathlineto{\pgfqpoint{2.381406in}{0.894088in}}%
\pgfpathlineto{\pgfqpoint{2.381141in}{0.898448in}}%
\pgfpathlineto{\pgfqpoint{2.394314in}{0.899243in}}%
\pgfpathlineto{\pgfqpoint{2.395019in}{0.889429in}}%
\pgfpathlineto{\pgfqpoint{2.408029in}{0.888072in}}%
\pgfpathlineto{\pgfqpoint{2.410713in}{0.884733in}}%
\pgfpathlineto{\pgfqpoint{2.407531in}{0.879522in}}%
\pgfpathlineto{\pgfqpoint{2.398952in}{0.876570in}}%
\pgfpathlineto{\pgfqpoint{2.400237in}{0.860115in}}%
\pgfpathlineto{\pgfqpoint{2.391200in}{0.859454in}}%
\pgfpathclose%
\pgfusepath{fill}%
\end{pgfscope}%
\begin{pgfscope}%
\pgfpathrectangle{\pgfqpoint{0.100000in}{0.100000in}}{\pgfqpoint{3.420221in}{2.189500in}}%
\pgfusepath{clip}%
\pgfsetbuttcap%
\pgfsetmiterjoin%
\definecolor{currentfill}{rgb}{0.000000,0.262745,0.868627}%
\pgfsetfillcolor{currentfill}%
\pgfsetlinewidth{0.000000pt}%
\definecolor{currentstroke}{rgb}{0.000000,0.000000,0.000000}%
\pgfsetstrokecolor{currentstroke}%
\pgfsetstrokeopacity{0.000000}%
\pgfsetdash{}{0pt}%
\pgfpathmoveto{\pgfqpoint{0.596123in}{1.925186in}}%
\pgfpathlineto{\pgfqpoint{0.562381in}{1.935369in}}%
\pgfpathlineto{\pgfqpoint{0.556365in}{1.941287in}}%
\pgfpathlineto{\pgfqpoint{0.555164in}{1.948162in}}%
\pgfpathlineto{\pgfqpoint{0.550476in}{1.952763in}}%
\pgfpathlineto{\pgfqpoint{0.547891in}{1.959164in}}%
\pgfpathlineto{\pgfqpoint{0.550890in}{1.962307in}}%
\pgfpathlineto{\pgfqpoint{0.551340in}{1.969463in}}%
\pgfpathlineto{\pgfqpoint{0.543609in}{1.977629in}}%
\pgfpathlineto{\pgfqpoint{0.541838in}{1.982818in}}%
\pgfpathlineto{\pgfqpoint{0.537282in}{1.986525in}}%
\pgfpathlineto{\pgfqpoint{0.520695in}{1.991907in}}%
\pgfpathlineto{\pgfqpoint{0.526542in}{1.999227in}}%
\pgfpathlineto{\pgfqpoint{0.527488in}{2.005744in}}%
\pgfpathlineto{\pgfqpoint{0.532297in}{2.009940in}}%
\pgfpathlineto{\pgfqpoint{0.533937in}{2.015037in}}%
\pgfpathlineto{\pgfqpoint{0.542302in}{2.041419in}}%
\pgfpathlineto{\pgfqpoint{0.553107in}{2.041328in}}%
\pgfpathlineto{\pgfqpoint{0.563769in}{2.029590in}}%
\pgfpathlineto{\pgfqpoint{0.567345in}{2.013820in}}%
\pgfpathlineto{\pgfqpoint{0.582909in}{2.016004in}}%
\pgfpathlineto{\pgfqpoint{0.588299in}{2.011855in}}%
\pgfpathlineto{\pgfqpoint{0.596012in}{2.017138in}}%
\pgfpathlineto{\pgfqpoint{0.603598in}{2.040987in}}%
\pgfpathlineto{\pgfqpoint{0.639458in}{2.030290in}}%
\pgfpathlineto{\pgfqpoint{0.631964in}{2.005400in}}%
\pgfpathlineto{\pgfqpoint{0.627515in}{2.006723in}}%
\pgfpathlineto{\pgfqpoint{0.622004in}{1.987791in}}%
\pgfpathlineto{\pgfqpoint{0.624607in}{1.982592in}}%
\pgfpathlineto{\pgfqpoint{0.613712in}{1.983318in}}%
\pgfpathlineto{\pgfqpoint{0.607038in}{1.985286in}}%
\pgfpathlineto{\pgfqpoint{0.603289in}{1.982947in}}%
\pgfpathlineto{\pgfqpoint{0.601240in}{1.973430in}}%
\pgfpathlineto{\pgfqpoint{0.608445in}{1.960387in}}%
\pgfpathlineto{\pgfqpoint{0.606772in}{1.950475in}}%
\pgfpathlineto{\pgfqpoint{0.602060in}{1.946847in}}%
\pgfpathlineto{\pgfqpoint{0.602386in}{1.937372in}}%
\pgfpathlineto{\pgfqpoint{0.595144in}{1.934937in}}%
\pgfpathclose%
\pgfusepath{fill}%
\end{pgfscope}%
\begin{pgfscope}%
\pgfpathrectangle{\pgfqpoint{0.100000in}{0.100000in}}{\pgfqpoint{3.420221in}{2.189500in}}%
\pgfusepath{clip}%
\pgfsetbuttcap%
\pgfsetmiterjoin%
\definecolor{currentfill}{rgb}{0.000000,0.352941,0.823529}%
\pgfsetfillcolor{currentfill}%
\pgfsetlinewidth{0.000000pt}%
\definecolor{currentstroke}{rgb}{0.000000,0.000000,0.000000}%
\pgfsetstrokecolor{currentstroke}%
\pgfsetstrokeopacity{0.000000}%
\pgfsetdash{}{0pt}%
\pgfpathmoveto{\pgfqpoint{2.874791in}{0.481116in}}%
\pgfpathlineto{\pgfqpoint{2.841260in}{0.476284in}}%
\pgfpathlineto{\pgfqpoint{2.846831in}{0.489507in}}%
\pgfpathlineto{\pgfqpoint{2.841412in}{0.490828in}}%
\pgfpathlineto{\pgfqpoint{2.835279in}{0.486366in}}%
\pgfpathlineto{\pgfqpoint{2.834232in}{0.479547in}}%
\pgfpathlineto{\pgfqpoint{2.830129in}{0.478034in}}%
\pgfpathlineto{\pgfqpoint{2.819060in}{0.490905in}}%
\pgfpathlineto{\pgfqpoint{2.821693in}{0.505139in}}%
\pgfpathlineto{\pgfqpoint{2.819408in}{0.514935in}}%
\pgfpathlineto{\pgfqpoint{2.821853in}{0.518603in}}%
\pgfpathlineto{\pgfqpoint{2.825326in}{0.545994in}}%
\pgfpathlineto{\pgfqpoint{2.824824in}{0.554806in}}%
\pgfpathlineto{\pgfqpoint{2.839382in}{0.556836in}}%
\pgfpathlineto{\pgfqpoint{2.849939in}{0.556223in}}%
\pgfpathlineto{\pgfqpoint{2.854622in}{0.549533in}}%
\pgfpathlineto{\pgfqpoint{2.865300in}{0.547175in}}%
\pgfpathlineto{\pgfqpoint{2.867475in}{0.531354in}}%
\pgfpathlineto{\pgfqpoint{2.864710in}{0.526890in}}%
\pgfpathlineto{\pgfqpoint{2.868975in}{0.520764in}}%
\pgfpathclose%
\pgfusepath{fill}%
\end{pgfscope}%
\begin{pgfscope}%
\pgfpathrectangle{\pgfqpoint{0.100000in}{0.100000in}}{\pgfqpoint{3.420221in}{2.189500in}}%
\pgfusepath{clip}%
\pgfsetbuttcap%
\pgfsetmiterjoin%
\definecolor{currentfill}{rgb}{0.000000,0.184314,0.907843}%
\pgfsetfillcolor{currentfill}%
\pgfsetlinewidth{0.000000pt}%
\definecolor{currentstroke}{rgb}{0.000000,0.000000,0.000000}%
\pgfsetstrokecolor{currentstroke}%
\pgfsetstrokeopacity{0.000000}%
\pgfsetdash{}{0pt}%
\pgfpathmoveto{\pgfqpoint{3.247365in}{1.742708in}}%
\pgfpathlineto{\pgfqpoint{3.247802in}{1.747403in}}%
\pgfpathlineto{\pgfqpoint{3.240982in}{1.756093in}}%
\pgfpathlineto{\pgfqpoint{3.242994in}{1.765491in}}%
\pgfpathlineto{\pgfqpoint{3.248237in}{1.780405in}}%
\pgfpathlineto{\pgfqpoint{3.251189in}{1.776475in}}%
\pgfpathlineto{\pgfqpoint{3.258505in}{1.783840in}}%
\pgfpathlineto{\pgfqpoint{3.257404in}{1.788597in}}%
\pgfpathlineto{\pgfqpoint{3.263819in}{1.791144in}}%
\pgfpathlineto{\pgfqpoint{3.257833in}{1.804798in}}%
\pgfpathlineto{\pgfqpoint{3.268017in}{1.809242in}}%
\pgfpathlineto{\pgfqpoint{3.264796in}{1.820324in}}%
\pgfpathlineto{\pgfqpoint{3.259472in}{1.830046in}}%
\pgfpathlineto{\pgfqpoint{3.266644in}{1.826128in}}%
\pgfpathlineto{\pgfqpoint{3.267361in}{1.834652in}}%
\pgfpathlineto{\pgfqpoint{3.278411in}{1.837120in}}%
\pgfpathlineto{\pgfqpoint{3.279179in}{1.840986in}}%
\pgfpathlineto{\pgfqpoint{3.289937in}{1.805540in}}%
\pgfpathlineto{\pgfqpoint{3.298667in}{1.809391in}}%
\pgfpathlineto{\pgfqpoint{3.307642in}{1.809971in}}%
\pgfpathlineto{\pgfqpoint{3.308989in}{1.803799in}}%
\pgfpathlineto{\pgfqpoint{3.314451in}{1.806262in}}%
\pgfpathlineto{\pgfqpoint{3.320763in}{1.802272in}}%
\pgfpathlineto{\pgfqpoint{3.319674in}{1.799207in}}%
\pgfpathlineto{\pgfqpoint{3.328383in}{1.796058in}}%
\pgfpathlineto{\pgfqpoint{3.329577in}{1.788547in}}%
\pgfpathlineto{\pgfqpoint{3.327580in}{1.782067in}}%
\pgfpathlineto{\pgfqpoint{3.322580in}{1.778868in}}%
\pgfpathlineto{\pgfqpoint{3.321912in}{1.761965in}}%
\pgfpathlineto{\pgfqpoint{3.316753in}{1.744319in}}%
\pgfpathlineto{\pgfqpoint{3.317116in}{1.741270in}}%
\pgfpathlineto{\pgfqpoint{3.310843in}{1.740538in}}%
\pgfpathlineto{\pgfqpoint{3.301361in}{1.733063in}}%
\pgfpathlineto{\pgfqpoint{3.295169in}{1.721735in}}%
\pgfpathlineto{\pgfqpoint{3.261023in}{1.713932in}}%
\pgfpathlineto{\pgfqpoint{3.257139in}{1.718909in}}%
\pgfpathlineto{\pgfqpoint{3.252544in}{1.729704in}}%
\pgfpathlineto{\pgfqpoint{3.249530in}{1.728998in}}%
\pgfpathclose%
\pgfusepath{fill}%
\end{pgfscope}%
\begin{pgfscope}%
\pgfpathrectangle{\pgfqpoint{0.100000in}{0.100000in}}{\pgfqpoint{3.420221in}{2.189500in}}%
\pgfusepath{clip}%
\pgfsetbuttcap%
\pgfsetmiterjoin%
\definecolor{currentfill}{rgb}{0.000000,0.568627,0.715686}%
\pgfsetfillcolor{currentfill}%
\pgfsetlinewidth{0.000000pt}%
\definecolor{currentstroke}{rgb}{0.000000,0.000000,0.000000}%
\pgfsetstrokecolor{currentstroke}%
\pgfsetstrokeopacity{0.000000}%
\pgfsetdash{}{0pt}%
\pgfpathmoveto{\pgfqpoint{2.354720in}{1.120421in}}%
\pgfpathlineto{\pgfqpoint{2.353555in}{1.125007in}}%
\pgfpathlineto{\pgfqpoint{2.355333in}{1.133690in}}%
\pgfpathlineto{\pgfqpoint{2.353219in}{1.135116in}}%
\pgfpathlineto{\pgfqpoint{2.349972in}{1.139300in}}%
\pgfpathlineto{\pgfqpoint{2.358229in}{1.152641in}}%
\pgfpathlineto{\pgfqpoint{2.364225in}{1.154194in}}%
\pgfpathlineto{\pgfqpoint{2.364834in}{1.162429in}}%
\pgfpathlineto{\pgfqpoint{2.376595in}{1.163344in}}%
\pgfpathlineto{\pgfqpoint{2.375361in}{1.182955in}}%
\pgfpathlineto{\pgfqpoint{2.392914in}{1.184249in}}%
\pgfpathlineto{\pgfqpoint{2.393689in}{1.171067in}}%
\pgfpathlineto{\pgfqpoint{2.397160in}{1.169901in}}%
\pgfpathlineto{\pgfqpoint{2.406206in}{1.160670in}}%
\pgfpathlineto{\pgfqpoint{2.408730in}{1.152021in}}%
\pgfpathlineto{\pgfqpoint{2.406441in}{1.146645in}}%
\pgfpathlineto{\pgfqpoint{2.412131in}{1.130599in}}%
\pgfpathlineto{\pgfqpoint{2.415601in}{1.121818in}}%
\pgfpathlineto{\pgfqpoint{2.393118in}{1.120413in}}%
\pgfpathlineto{\pgfqpoint{2.392163in}{1.134769in}}%
\pgfpathlineto{\pgfqpoint{2.372510in}{1.133719in}}%
\pgfpathlineto{\pgfqpoint{2.373379in}{1.120747in}}%
\pgfpathclose%
\pgfusepath{fill}%
\end{pgfscope}%
\begin{pgfscope}%
\pgfpathrectangle{\pgfqpoint{0.100000in}{0.100000in}}{\pgfqpoint{3.420221in}{2.189500in}}%
\pgfusepath{clip}%
\pgfsetbuttcap%
\pgfsetmiterjoin%
\definecolor{currentfill}{rgb}{0.000000,0.490196,0.754902}%
\pgfsetfillcolor{currentfill}%
\pgfsetlinewidth{0.000000pt}%
\definecolor{currentstroke}{rgb}{0.000000,0.000000,0.000000}%
\pgfsetstrokecolor{currentstroke}%
\pgfsetstrokeopacity{0.000000}%
\pgfsetdash{}{0pt}%
\pgfpathmoveto{\pgfqpoint{3.005121in}{1.526964in}}%
\pgfpathlineto{\pgfqpoint{2.994878in}{1.520659in}}%
\pgfpathlineto{\pgfqpoint{2.986028in}{1.517022in}}%
\pgfpathlineto{\pgfqpoint{2.975310in}{1.526118in}}%
\pgfpathlineto{\pgfqpoint{2.970791in}{1.525044in}}%
\pgfpathlineto{\pgfqpoint{2.967395in}{1.529799in}}%
\pgfpathlineto{\pgfqpoint{2.953945in}{1.525441in}}%
\pgfpathlineto{\pgfqpoint{2.950006in}{1.529528in}}%
\pgfpathlineto{\pgfqpoint{2.953702in}{1.541710in}}%
\pgfpathlineto{\pgfqpoint{2.952182in}{1.549615in}}%
\pgfpathlineto{\pgfqpoint{2.973705in}{1.554026in}}%
\pgfpathlineto{\pgfqpoint{2.972715in}{1.558699in}}%
\pgfpathlineto{\pgfqpoint{3.007775in}{1.566319in}}%
\pgfpathlineto{\pgfqpoint{3.011820in}{1.570591in}}%
\pgfpathlineto{\pgfqpoint{3.002935in}{1.599794in}}%
\pgfpathlineto{\pgfqpoint{3.023112in}{1.603787in}}%
\pgfpathlineto{\pgfqpoint{3.045578in}{1.608356in}}%
\pgfpathlineto{\pgfqpoint{3.052654in}{1.583281in}}%
\pgfpathlineto{\pgfqpoint{3.048128in}{1.581912in}}%
\pgfpathlineto{\pgfqpoint{3.048598in}{1.571517in}}%
\pgfpathlineto{\pgfqpoint{3.047204in}{1.555823in}}%
\pgfpathlineto{\pgfqpoint{3.041840in}{1.554670in}}%
\pgfpathlineto{\pgfqpoint{3.033936in}{1.541521in}}%
\pgfpathlineto{\pgfqpoint{3.031305in}{1.540824in}}%
\pgfpathlineto{\pgfqpoint{3.022826in}{1.540629in}}%
\pgfpathlineto{\pgfqpoint{3.013747in}{1.536730in}}%
\pgfpathlineto{\pgfqpoint{3.013627in}{1.532040in}}%
\pgfpathclose%
\pgfusepath{fill}%
\end{pgfscope}%
\begin{pgfscope}%
\pgfpathrectangle{\pgfqpoint{0.100000in}{0.100000in}}{\pgfqpoint{3.420221in}{2.189500in}}%
\pgfusepath{clip}%
\pgfsetbuttcap%
\pgfsetmiterjoin%
\definecolor{currentfill}{rgb}{0.000000,0.768627,0.615686}%
\pgfsetfillcolor{currentfill}%
\pgfsetlinewidth{0.000000pt}%
\definecolor{currentstroke}{rgb}{0.000000,0.000000,0.000000}%
\pgfsetstrokecolor{currentstroke}%
\pgfsetstrokeopacity{0.000000}%
\pgfsetdash{}{0pt}%
\pgfpathmoveto{\pgfqpoint{2.540291in}{0.697885in}}%
\pgfpathlineto{\pgfqpoint{2.481449in}{0.693010in}}%
\pgfpathlineto{\pgfqpoint{2.478837in}{0.711538in}}%
\pgfpathlineto{\pgfqpoint{2.472149in}{0.715370in}}%
\pgfpathlineto{\pgfqpoint{2.470973in}{0.720707in}}%
\pgfpathlineto{\pgfqpoint{2.474548in}{0.724742in}}%
\pgfpathlineto{\pgfqpoint{2.480437in}{0.726286in}}%
\pgfpathlineto{\pgfqpoint{2.479202in}{0.745886in}}%
\pgfpathlineto{\pgfqpoint{2.482446in}{0.746207in}}%
\pgfpathlineto{\pgfqpoint{2.482601in}{0.756178in}}%
\pgfpathlineto{\pgfqpoint{2.520557in}{0.759546in}}%
\pgfpathlineto{\pgfqpoint{2.521894in}{0.744657in}}%
\pgfpathlineto{\pgfqpoint{2.526938in}{0.736958in}}%
\pgfpathlineto{\pgfqpoint{2.535789in}{0.737656in}}%
\pgfpathlineto{\pgfqpoint{2.539481in}{0.698260in}}%
\pgfpathclose%
\pgfusepath{fill}%
\end{pgfscope}%
\begin{pgfscope}%
\pgfpathrectangle{\pgfqpoint{0.100000in}{0.100000in}}{\pgfqpoint{3.420221in}{2.189500in}}%
\pgfusepath{clip}%
\pgfsetbuttcap%
\pgfsetmiterjoin%
\definecolor{currentfill}{rgb}{0.000000,0.764706,0.617647}%
\pgfsetfillcolor{currentfill}%
\pgfsetlinewidth{0.000000pt}%
\definecolor{currentstroke}{rgb}{0.000000,0.000000,0.000000}%
\pgfsetstrokecolor{currentstroke}%
\pgfsetstrokeopacity{0.000000}%
\pgfsetdash{}{0pt}%
\pgfpathmoveto{\pgfqpoint{1.185762in}{0.898306in}}%
\pgfpathlineto{\pgfqpoint{1.120630in}{0.908635in}}%
\pgfpathlineto{\pgfqpoint{1.127245in}{0.950968in}}%
\pgfpathlineto{\pgfqpoint{1.136566in}{1.010472in}}%
\pgfpathlineto{\pgfqpoint{1.177571in}{1.004184in}}%
\pgfpathlineto{\pgfqpoint{1.217413in}{0.998383in}}%
\pgfpathlineto{\pgfqpoint{1.269062in}{0.991454in}}%
\pgfpathlineto{\pgfqpoint{1.275604in}{0.986945in}}%
\pgfpathlineto{\pgfqpoint{1.296314in}{0.977358in}}%
\pgfpathlineto{\pgfqpoint{1.294687in}{0.963920in}}%
\pgfpathlineto{\pgfqpoint{1.325022in}{0.960197in}}%
\pgfpathlineto{\pgfqpoint{1.321059in}{0.927763in}}%
\pgfpathlineto{\pgfqpoint{1.313113in}{0.928753in}}%
\pgfpathlineto{\pgfqpoint{1.311512in}{0.915642in}}%
\pgfpathlineto{\pgfqpoint{1.291645in}{0.918872in}}%
\pgfpathlineto{\pgfqpoint{1.290004in}{0.905403in}}%
\pgfpathlineto{\pgfqpoint{1.254494in}{0.909662in}}%
\pgfpathlineto{\pgfqpoint{1.192521in}{0.918086in}}%
\pgfpathlineto{\pgfqpoint{1.188729in}{0.918907in}}%
\pgfpathclose%
\pgfusepath{fill}%
\end{pgfscope}%
\begin{pgfscope}%
\pgfpathrectangle{\pgfqpoint{0.100000in}{0.100000in}}{\pgfqpoint{3.420221in}{2.189500in}}%
\pgfusepath{clip}%
\pgfsetbuttcap%
\pgfsetmiterjoin%
\definecolor{currentfill}{rgb}{0.000000,0.619608,0.690196}%
\pgfsetfillcolor{currentfill}%
\pgfsetlinewidth{0.000000pt}%
\definecolor{currentstroke}{rgb}{0.000000,0.000000,0.000000}%
\pgfsetstrokecolor{currentstroke}%
\pgfsetstrokeopacity{0.000000}%
\pgfsetdash{}{0pt}%
\pgfpathmoveto{\pgfqpoint{2.541751in}{1.639369in}}%
\pgfpathlineto{\pgfqpoint{2.515852in}{1.636769in}}%
\pgfpathlineto{\pgfqpoint{2.513143in}{1.662713in}}%
\pgfpathlineto{\pgfqpoint{2.487428in}{1.660113in}}%
\pgfpathlineto{\pgfqpoint{2.484632in}{1.686465in}}%
\pgfpathlineto{\pgfqpoint{2.496525in}{1.687374in}}%
\pgfpathlineto{\pgfqpoint{2.494079in}{1.713495in}}%
\pgfpathlineto{\pgfqpoint{2.545681in}{1.718849in}}%
\pgfpathlineto{\pgfqpoint{2.548562in}{1.692702in}}%
\pgfpathlineto{\pgfqpoint{2.535922in}{1.691526in}}%
\pgfpathclose%
\pgfusepath{fill}%
\end{pgfscope}%
\begin{pgfscope}%
\pgfpathrectangle{\pgfqpoint{0.100000in}{0.100000in}}{\pgfqpoint{3.420221in}{2.189500in}}%
\pgfusepath{clip}%
\pgfsetbuttcap%
\pgfsetmiterjoin%
\definecolor{currentfill}{rgb}{0.000000,0.372549,0.813725}%
\pgfsetfillcolor{currentfill}%
\pgfsetlinewidth{0.000000pt}%
\definecolor{currentstroke}{rgb}{0.000000,0.000000,0.000000}%
\pgfsetstrokecolor{currentstroke}%
\pgfsetstrokeopacity{0.000000}%
\pgfsetdash{}{0pt}%
\pgfpathmoveto{\pgfqpoint{2.446738in}{1.029172in}}%
\pgfpathlineto{\pgfqpoint{2.450300in}{1.037559in}}%
\pgfpathlineto{\pgfqpoint{2.444061in}{1.042556in}}%
\pgfpathlineto{\pgfqpoint{2.444652in}{1.055763in}}%
\pgfpathlineto{\pgfqpoint{2.444841in}{1.060810in}}%
\pgfpathlineto{\pgfqpoint{2.454737in}{1.068227in}}%
\pgfpathlineto{\pgfqpoint{2.474945in}{1.067350in}}%
\pgfpathlineto{\pgfqpoint{2.474985in}{1.059179in}}%
\pgfpathlineto{\pgfqpoint{2.500291in}{1.051602in}}%
\pgfpathlineto{\pgfqpoint{2.510993in}{1.049975in}}%
\pgfpathlineto{\pgfqpoint{2.511575in}{1.031138in}}%
\pgfpathlineto{\pgfqpoint{2.515642in}{1.026211in}}%
\pgfpathlineto{\pgfqpoint{2.515714in}{1.023357in}}%
\pgfpathlineto{\pgfqpoint{2.507418in}{1.017006in}}%
\pgfpathlineto{\pgfqpoint{2.502375in}{1.017363in}}%
\pgfpathlineto{\pgfqpoint{2.503714in}{0.996982in}}%
\pgfpathlineto{\pgfqpoint{2.480721in}{0.995506in}}%
\pgfpathlineto{\pgfqpoint{2.479546in}{0.999164in}}%
\pgfpathlineto{\pgfqpoint{2.478005in}{1.024083in}}%
\pgfpathlineto{\pgfqpoint{2.476897in}{1.027609in}}%
\pgfpathlineto{\pgfqpoint{2.470777in}{1.029126in}}%
\pgfpathlineto{\pgfqpoint{2.455922in}{1.023523in}}%
\pgfpathlineto{\pgfqpoint{2.451169in}{1.028731in}}%
\pgfpathclose%
\pgfusepath{fill}%
\end{pgfscope}%
\begin{pgfscope}%
\pgfpathrectangle{\pgfqpoint{0.100000in}{0.100000in}}{\pgfqpoint{3.420221in}{2.189500in}}%
\pgfusepath{clip}%
\pgfsetbuttcap%
\pgfsetmiterjoin%
\definecolor{currentfill}{rgb}{0.000000,0.674510,0.662745}%
\pgfsetfillcolor{currentfill}%
\pgfsetlinewidth{0.000000pt}%
\definecolor{currentstroke}{rgb}{0.000000,0.000000,0.000000}%
\pgfsetstrokecolor{currentstroke}%
\pgfsetstrokeopacity{0.000000}%
\pgfsetdash{}{0pt}%
\pgfpathmoveto{\pgfqpoint{2.859039in}{1.321669in}}%
\pgfpathlineto{\pgfqpoint{2.866647in}{1.310049in}}%
\pgfpathlineto{\pgfqpoint{2.864743in}{1.299340in}}%
\pgfpathlineto{\pgfqpoint{2.860470in}{1.294643in}}%
\pgfpathlineto{\pgfqpoint{2.860060in}{1.286231in}}%
\pgfpathlineto{\pgfqpoint{2.855121in}{1.288580in}}%
\pgfpathlineto{\pgfqpoint{2.841330in}{1.272091in}}%
\pgfpathlineto{\pgfqpoint{2.831160in}{1.272330in}}%
\pgfpathlineto{\pgfqpoint{2.823750in}{1.280225in}}%
\pgfpathlineto{\pgfqpoint{2.818862in}{1.277446in}}%
\pgfpathlineto{\pgfqpoint{2.808953in}{1.281024in}}%
\pgfpathlineto{\pgfqpoint{2.825830in}{1.295111in}}%
\pgfpathlineto{\pgfqpoint{2.826687in}{1.302013in}}%
\pgfpathlineto{\pgfqpoint{2.823059in}{1.307156in}}%
\pgfpathlineto{\pgfqpoint{2.815948in}{1.312761in}}%
\pgfpathlineto{\pgfqpoint{2.818073in}{1.316886in}}%
\pgfpathlineto{\pgfqpoint{2.813065in}{1.321828in}}%
\pgfpathlineto{\pgfqpoint{2.815571in}{1.326240in}}%
\pgfpathlineto{\pgfqpoint{2.811916in}{1.338045in}}%
\pgfpathlineto{\pgfqpoint{2.818111in}{1.341570in}}%
\pgfpathlineto{\pgfqpoint{2.823445in}{1.347414in}}%
\pgfpathlineto{\pgfqpoint{2.828466in}{1.347165in}}%
\pgfpathlineto{\pgfqpoint{2.830150in}{1.340814in}}%
\pgfpathlineto{\pgfqpoint{2.838371in}{1.333586in}}%
\pgfpathlineto{\pgfqpoint{2.846207in}{1.330088in}}%
\pgfpathlineto{\pgfqpoint{2.849782in}{1.322776in}}%
\pgfpathclose%
\pgfusepath{fill}%
\end{pgfscope}%
\begin{pgfscope}%
\pgfpathrectangle{\pgfqpoint{0.100000in}{0.100000in}}{\pgfqpoint{3.420221in}{2.189500in}}%
\pgfusepath{clip}%
\pgfsetbuttcap%
\pgfsetmiterjoin%
\definecolor{currentfill}{rgb}{0.000000,0.713725,0.643137}%
\pgfsetfillcolor{currentfill}%
\pgfsetlinewidth{0.000000pt}%
\definecolor{currentstroke}{rgb}{0.000000,0.000000,0.000000}%
\pgfsetstrokecolor{currentstroke}%
\pgfsetstrokeopacity{0.000000}%
\pgfsetdash{}{0pt}%
\pgfpathmoveto{\pgfqpoint{0.937424in}{1.230332in}}%
\pgfpathlineto{\pgfqpoint{0.869922in}{1.244248in}}%
\pgfpathlineto{\pgfqpoint{0.882667in}{1.303530in}}%
\pgfpathlineto{\pgfqpoint{0.894470in}{1.359506in}}%
\pgfpathlineto{\pgfqpoint{0.953379in}{1.347413in}}%
\pgfpathlineto{\pgfqpoint{0.982552in}{1.341497in}}%
\pgfpathlineto{\pgfqpoint{0.982076in}{1.332373in}}%
\pgfpathlineto{\pgfqpoint{0.986009in}{1.329158in}}%
\pgfpathlineto{\pgfqpoint{0.989491in}{1.321881in}}%
\pgfpathlineto{\pgfqpoint{0.984234in}{1.313214in}}%
\pgfpathlineto{\pgfqpoint{0.978565in}{1.309833in}}%
\pgfpathlineto{\pgfqpoint{0.975495in}{1.290822in}}%
\pgfpathlineto{\pgfqpoint{0.968978in}{1.291459in}}%
\pgfpathlineto{\pgfqpoint{0.967852in}{1.285754in}}%
\pgfpathlineto{\pgfqpoint{0.961440in}{1.287111in}}%
\pgfpathlineto{\pgfqpoint{0.957985in}{1.267733in}}%
\pgfpathlineto{\pgfqpoint{0.945283in}{1.270267in}}%
\pgfpathclose%
\pgfusepath{fill}%
\end{pgfscope}%
\begin{pgfscope}%
\pgfpathrectangle{\pgfqpoint{0.100000in}{0.100000in}}{\pgfqpoint{3.420221in}{2.189500in}}%
\pgfusepath{clip}%
\pgfsetbuttcap%
\pgfsetmiterjoin%
\definecolor{currentfill}{rgb}{0.000000,0.384314,0.807843}%
\pgfsetfillcolor{currentfill}%
\pgfsetlinewidth{0.000000pt}%
\definecolor{currentstroke}{rgb}{0.000000,0.000000,0.000000}%
\pgfsetstrokecolor{currentstroke}%
\pgfsetstrokeopacity{0.000000}%
\pgfsetdash{}{0pt}%
\pgfpathmoveto{\pgfqpoint{2.549465in}{1.213876in}}%
\pgfpathlineto{\pgfqpoint{2.556238in}{1.205886in}}%
\pgfpathlineto{\pgfqpoint{2.558961in}{1.205950in}}%
\pgfpathlineto{\pgfqpoint{2.559193in}{1.195010in}}%
\pgfpathlineto{\pgfqpoint{2.565682in}{1.195350in}}%
\pgfpathlineto{\pgfqpoint{2.568187in}{1.189724in}}%
\pgfpathlineto{\pgfqpoint{2.558186in}{1.189136in}}%
\pgfpathlineto{\pgfqpoint{2.557212in}{1.185310in}}%
\pgfpathlineto{\pgfqpoint{2.533899in}{1.185051in}}%
\pgfpathlineto{\pgfqpoint{2.532023in}{1.177429in}}%
\pgfpathlineto{\pgfqpoint{2.528212in}{1.175891in}}%
\pgfpathlineto{\pgfqpoint{2.509846in}{1.173025in}}%
\pgfpathlineto{\pgfqpoint{2.508477in}{1.176013in}}%
\pgfpathlineto{\pgfqpoint{2.500840in}{1.177777in}}%
\pgfpathlineto{\pgfqpoint{2.496901in}{1.188062in}}%
\pgfpathlineto{\pgfqpoint{2.498444in}{1.197106in}}%
\pgfpathlineto{\pgfqpoint{2.494206in}{1.207082in}}%
\pgfpathlineto{\pgfqpoint{2.495202in}{1.210912in}}%
\pgfpathlineto{\pgfqpoint{2.498588in}{1.217143in}}%
\pgfpathlineto{\pgfqpoint{2.503286in}{1.218158in}}%
\pgfpathlineto{\pgfqpoint{2.501935in}{1.226596in}}%
\pgfpathlineto{\pgfqpoint{2.504621in}{1.232676in}}%
\pgfpathlineto{\pgfqpoint{2.508081in}{1.231929in}}%
\pgfpathlineto{\pgfqpoint{2.510924in}{1.238837in}}%
\pgfpathlineto{\pgfqpoint{2.515758in}{1.234671in}}%
\pgfpathlineto{\pgfqpoint{2.516902in}{1.228778in}}%
\pgfpathlineto{\pgfqpoint{2.522445in}{1.226093in}}%
\pgfpathlineto{\pgfqpoint{2.535607in}{1.227157in}}%
\pgfpathlineto{\pgfqpoint{2.541238in}{1.223259in}}%
\pgfpathlineto{\pgfqpoint{2.544423in}{1.216866in}}%
\pgfpathclose%
\pgfusepath{fill}%
\end{pgfscope}%
\begin{pgfscope}%
\pgfpathrectangle{\pgfqpoint{0.100000in}{0.100000in}}{\pgfqpoint{3.420221in}{2.189500in}}%
\pgfusepath{clip}%
\pgfsetbuttcap%
\pgfsetmiterjoin%
\definecolor{currentfill}{rgb}{0.000000,0.258824,0.870588}%
\pgfsetfillcolor{currentfill}%
\pgfsetlinewidth{0.000000pt}%
\definecolor{currentstroke}{rgb}{0.000000,0.000000,0.000000}%
\pgfsetstrokecolor{currentstroke}%
\pgfsetstrokeopacity{0.000000}%
\pgfsetdash{}{0pt}%
\pgfpathmoveto{\pgfqpoint{1.802451in}{2.032421in}}%
\pgfpathlineto{\pgfqpoint{1.801504in}{2.010992in}}%
\pgfpathlineto{\pgfqpoint{1.802403in}{1.997722in}}%
\pgfpathlineto{\pgfqpoint{1.801799in}{1.984536in}}%
\pgfpathlineto{\pgfqpoint{1.775492in}{1.985799in}}%
\pgfpathlineto{\pgfqpoint{1.776128in}{1.998996in}}%
\pgfpathlineto{\pgfqpoint{1.743331in}{2.000767in}}%
\pgfpathlineto{\pgfqpoint{1.742226in}{2.014058in}}%
\pgfpathlineto{\pgfqpoint{1.743456in}{2.035343in}}%
\pgfpathclose%
\pgfusepath{fill}%
\end{pgfscope}%
\begin{pgfscope}%
\pgfpathrectangle{\pgfqpoint{0.100000in}{0.100000in}}{\pgfqpoint{3.420221in}{2.189500in}}%
\pgfusepath{clip}%
\pgfsetbuttcap%
\pgfsetmiterjoin%
\definecolor{currentfill}{rgb}{0.000000,0.270588,0.864706}%
\pgfsetfillcolor{currentfill}%
\pgfsetlinewidth{0.000000pt}%
\definecolor{currentstroke}{rgb}{0.000000,0.000000,0.000000}%
\pgfsetstrokecolor{currentstroke}%
\pgfsetstrokeopacity{0.000000}%
\pgfsetdash{}{0pt}%
\pgfpathmoveto{\pgfqpoint{1.913786in}{0.997539in}}%
\pgfpathlineto{\pgfqpoint{1.913581in}{0.984453in}}%
\pgfpathlineto{\pgfqpoint{1.910566in}{0.984511in}}%
\pgfpathlineto{\pgfqpoint{1.910292in}{0.969092in}}%
\pgfpathlineto{\pgfqpoint{1.900660in}{0.968627in}}%
\pgfpathlineto{\pgfqpoint{1.895921in}{0.965304in}}%
\pgfpathlineto{\pgfqpoint{1.890325in}{0.972096in}}%
\pgfpathlineto{\pgfqpoint{1.885751in}{0.970896in}}%
\pgfpathlineto{\pgfqpoint{1.882992in}{0.965391in}}%
\pgfpathlineto{\pgfqpoint{1.837680in}{0.966588in}}%
\pgfpathlineto{\pgfqpoint{1.838519in}{1.002686in}}%
\pgfpathlineto{\pgfqpoint{1.828097in}{1.002826in}}%
\pgfpathlineto{\pgfqpoint{1.817251in}{1.006476in}}%
\pgfpathlineto{\pgfqpoint{1.799778in}{1.007059in}}%
\pgfpathlineto{\pgfqpoint{1.799835in}{1.020152in}}%
\pgfpathlineto{\pgfqpoint{1.800303in}{1.033179in}}%
\pgfpathlineto{\pgfqpoint{1.806768in}{1.032943in}}%
\pgfpathlineto{\pgfqpoint{1.807728in}{1.065923in}}%
\pgfpathlineto{\pgfqpoint{1.853111in}{1.064475in}}%
\pgfpathlineto{\pgfqpoint{1.859580in}{1.064290in}}%
\pgfpathlineto{\pgfqpoint{1.859232in}{1.050839in}}%
\pgfpathlineto{\pgfqpoint{1.872104in}{1.047284in}}%
\pgfpathlineto{\pgfqpoint{1.903647in}{1.046608in}}%
\pgfpathlineto{\pgfqpoint{1.903269in}{1.023911in}}%
\pgfpathlineto{\pgfqpoint{1.902764in}{1.006049in}}%
\pgfpathlineto{\pgfqpoint{1.910517in}{1.007993in}}%
\pgfpathclose%
\pgfusepath{fill}%
\end{pgfscope}%
\begin{pgfscope}%
\pgfpathrectangle{\pgfqpoint{0.100000in}{0.100000in}}{\pgfqpoint{3.420221in}{2.189500in}}%
\pgfusepath{clip}%
\pgfsetbuttcap%
\pgfsetmiterjoin%
\definecolor{currentfill}{rgb}{0.000000,0.364706,0.817647}%
\pgfsetfillcolor{currentfill}%
\pgfsetlinewidth{0.000000pt}%
\definecolor{currentstroke}{rgb}{0.000000,0.000000,0.000000}%
\pgfsetstrokecolor{currentstroke}%
\pgfsetstrokeopacity{0.000000}%
\pgfsetdash{}{0pt}%
\pgfpathmoveto{\pgfqpoint{2.006748in}{1.612479in}}%
\pgfpathlineto{\pgfqpoint{2.010044in}{1.612494in}}%
\pgfpathlineto{\pgfqpoint{2.010002in}{1.638606in}}%
\pgfpathlineto{\pgfqpoint{2.068917in}{1.639049in}}%
\pgfpathlineto{\pgfqpoint{2.075381in}{1.639155in}}%
\pgfpathlineto{\pgfqpoint{2.075764in}{1.612957in}}%
\pgfpathlineto{\pgfqpoint{2.058209in}{1.612731in}}%
\pgfpathlineto{\pgfqpoint{2.032478in}{1.612580in}}%
\pgfpathlineto{\pgfqpoint{2.032579in}{1.594129in}}%
\pgfpathlineto{\pgfqpoint{2.006822in}{1.594042in}}%
\pgfpathclose%
\pgfusepath{fill}%
\end{pgfscope}%
\begin{pgfscope}%
\pgfpathrectangle{\pgfqpoint{0.100000in}{0.100000in}}{\pgfqpoint{3.420221in}{2.189500in}}%
\pgfusepath{clip}%
\pgfsetbuttcap%
\pgfsetmiterjoin%
\definecolor{currentfill}{rgb}{0.000000,0.768627,0.615686}%
\pgfsetfillcolor{currentfill}%
\pgfsetlinewidth{0.000000pt}%
\definecolor{currentstroke}{rgb}{0.000000,0.000000,0.000000}%
\pgfsetstrokecolor{currentstroke}%
\pgfsetstrokeopacity{0.000000}%
\pgfsetdash{}{0pt}%
\pgfpathmoveto{\pgfqpoint{2.663498in}{1.134462in}}%
\pgfpathlineto{\pgfqpoint{2.649158in}{1.132929in}}%
\pgfpathlineto{\pgfqpoint{2.643863in}{1.138251in}}%
\pgfpathlineto{\pgfqpoint{2.639531in}{1.148355in}}%
\pgfpathlineto{\pgfqpoint{2.641750in}{1.156121in}}%
\pgfpathlineto{\pgfqpoint{2.639155in}{1.167640in}}%
\pgfpathlineto{\pgfqpoint{2.640541in}{1.174149in}}%
\pgfpathlineto{\pgfqpoint{2.641954in}{1.181008in}}%
\pgfpathlineto{\pgfqpoint{2.648455in}{1.188870in}}%
\pgfpathlineto{\pgfqpoint{2.656427in}{1.183734in}}%
\pgfpathlineto{\pgfqpoint{2.660150in}{1.183974in}}%
\pgfpathlineto{\pgfqpoint{2.668623in}{1.192429in}}%
\pgfpathlineto{\pgfqpoint{2.674932in}{1.192333in}}%
\pgfpathlineto{\pgfqpoint{2.684730in}{1.187635in}}%
\pgfpathlineto{\pgfqpoint{2.684127in}{1.180690in}}%
\pgfpathlineto{\pgfqpoint{2.688809in}{1.163964in}}%
\pgfpathlineto{\pgfqpoint{2.673572in}{1.151623in}}%
\pgfpathlineto{\pgfqpoint{2.671216in}{1.144736in}}%
\pgfpathlineto{\pgfqpoint{2.664871in}{1.138828in}}%
\pgfpathclose%
\pgfusepath{fill}%
\end{pgfscope}%
\begin{pgfscope}%
\pgfpathrectangle{\pgfqpoint{0.100000in}{0.100000in}}{\pgfqpoint{3.420221in}{2.189500in}}%
\pgfusepath{clip}%
\pgfsetbuttcap%
\pgfsetmiterjoin%
\definecolor{currentfill}{rgb}{0.000000,0.552941,0.723529}%
\pgfsetfillcolor{currentfill}%
\pgfsetlinewidth{0.000000pt}%
\definecolor{currentstroke}{rgb}{0.000000,0.000000,0.000000}%
\pgfsetstrokecolor{currentstroke}%
\pgfsetstrokeopacity{0.000000}%
\pgfsetdash{}{0pt}%
\pgfpathmoveto{\pgfqpoint{2.626317in}{1.185821in}}%
\pgfpathlineto{\pgfqpoint{2.622914in}{1.187490in}}%
\pgfpathlineto{\pgfqpoint{2.614187in}{1.177617in}}%
\pgfpathlineto{\pgfqpoint{2.613044in}{1.186694in}}%
\pgfpathlineto{\pgfqpoint{2.604897in}{1.190815in}}%
\pgfpathlineto{\pgfqpoint{2.603681in}{1.198329in}}%
\pgfpathlineto{\pgfqpoint{2.592758in}{1.198374in}}%
\pgfpathlineto{\pgfqpoint{2.592522in}{1.221667in}}%
\pgfpathlineto{\pgfqpoint{2.590455in}{1.224248in}}%
\pgfpathlineto{\pgfqpoint{2.595668in}{1.229524in}}%
\pgfpathlineto{\pgfqpoint{2.602425in}{1.230209in}}%
\pgfpathlineto{\pgfqpoint{2.618781in}{1.215147in}}%
\pgfpathlineto{\pgfqpoint{2.621176in}{1.218506in}}%
\pgfpathlineto{\pgfqpoint{2.633654in}{1.202565in}}%
\pgfpathlineto{\pgfqpoint{2.628273in}{1.197919in}}%
\pgfpathclose%
\pgfusepath{fill}%
\end{pgfscope}%
\begin{pgfscope}%
\pgfpathrectangle{\pgfqpoint{0.100000in}{0.100000in}}{\pgfqpoint{3.420221in}{2.189500in}}%
\pgfusepath{clip}%
\pgfsetbuttcap%
\pgfsetmiterjoin%
\definecolor{currentfill}{rgb}{0.000000,0.427451,0.786275}%
\pgfsetfillcolor{currentfill}%
\pgfsetlinewidth{0.000000pt}%
\definecolor{currentstroke}{rgb}{0.000000,0.000000,0.000000}%
\pgfsetstrokecolor{currentstroke}%
\pgfsetstrokeopacity{0.000000}%
\pgfsetdash{}{0pt}%
\pgfpathmoveto{\pgfqpoint{1.599819in}{1.391683in}}%
\pgfpathlineto{\pgfqpoint{1.595419in}{1.333771in}}%
\pgfpathlineto{\pgfqpoint{1.531803in}{1.338478in}}%
\pgfpathlineto{\pgfqpoint{1.500023in}{1.341478in}}%
\pgfpathlineto{\pgfqpoint{1.503220in}{1.373918in}}%
\pgfpathlineto{\pgfqpoint{1.516574in}{1.372652in}}%
\pgfpathlineto{\pgfqpoint{1.520475in}{1.411561in}}%
\pgfpathlineto{\pgfqpoint{1.513984in}{1.412184in}}%
\pgfpathlineto{\pgfqpoint{1.517776in}{1.447837in}}%
\pgfpathlineto{\pgfqpoint{1.569677in}{1.443236in}}%
\pgfpathlineto{\pgfqpoint{1.568209in}{1.424322in}}%
\pgfpathlineto{\pgfqpoint{1.602161in}{1.421596in}}%
\pgfpathlineto{\pgfqpoint{1.601856in}{1.417708in}}%
\pgfpathclose%
\pgfusepath{fill}%
\end{pgfscope}%
\begin{pgfscope}%
\pgfpathrectangle{\pgfqpoint{0.100000in}{0.100000in}}{\pgfqpoint{3.420221in}{2.189500in}}%
\pgfusepath{clip}%
\pgfsetbuttcap%
\pgfsetmiterjoin%
\definecolor{currentfill}{rgb}{0.000000,0.341176,0.829412}%
\pgfsetfillcolor{currentfill}%
\pgfsetlinewidth{0.000000pt}%
\definecolor{currentstroke}{rgb}{0.000000,0.000000,0.000000}%
\pgfsetstrokecolor{currentstroke}%
\pgfsetstrokeopacity{0.000000}%
\pgfsetdash{}{0pt}%
\pgfpathmoveto{\pgfqpoint{1.755182in}{0.892795in}}%
\pgfpathlineto{\pgfqpoint{1.753701in}{0.859974in}}%
\pgfpathlineto{\pgfqpoint{1.721156in}{0.861597in}}%
\pgfpathlineto{\pgfqpoint{1.722266in}{0.886745in}}%
\pgfpathlineto{\pgfqpoint{1.722569in}{0.894304in}}%
\pgfpathclose%
\pgfusepath{fill}%
\end{pgfscope}%
\begin{pgfscope}%
\pgfpathrectangle{\pgfqpoint{0.100000in}{0.100000in}}{\pgfqpoint{3.420221in}{2.189500in}}%
\pgfusepath{clip}%
\pgfsetbuttcap%
\pgfsetmiterjoin%
\definecolor{currentfill}{rgb}{0.000000,0.415686,0.792157}%
\pgfsetfillcolor{currentfill}%
\pgfsetlinewidth{0.000000pt}%
\definecolor{currentstroke}{rgb}{0.000000,0.000000,0.000000}%
\pgfsetstrokecolor{currentstroke}%
\pgfsetstrokeopacity{0.000000}%
\pgfsetdash{}{0pt}%
\pgfpathmoveto{\pgfqpoint{2.126904in}{1.293377in}}%
\pgfpathlineto{\pgfqpoint{2.121693in}{1.284091in}}%
\pgfpathlineto{\pgfqpoint{2.122238in}{1.281182in}}%
\pgfpathlineto{\pgfqpoint{2.115763in}{1.274135in}}%
\pgfpathlineto{\pgfqpoint{2.115718in}{1.270824in}}%
\pgfpathlineto{\pgfqpoint{2.089714in}{1.271428in}}%
\pgfpathlineto{\pgfqpoint{2.089153in}{1.292032in}}%
\pgfpathlineto{\pgfqpoint{2.080017in}{1.293987in}}%
\pgfpathlineto{\pgfqpoint{2.074177in}{1.290983in}}%
\pgfpathlineto{\pgfqpoint{2.073810in}{1.314766in}}%
\pgfpathlineto{\pgfqpoint{2.073222in}{1.347352in}}%
\pgfpathlineto{\pgfqpoint{2.096159in}{1.348314in}}%
\pgfpathlineto{\pgfqpoint{2.096356in}{1.328557in}}%
\pgfpathlineto{\pgfqpoint{2.100965in}{1.327249in}}%
\pgfpathlineto{\pgfqpoint{2.102722in}{1.317540in}}%
\pgfpathlineto{\pgfqpoint{2.101749in}{1.312344in}}%
\pgfpathlineto{\pgfqpoint{2.105149in}{1.308234in}}%
\pgfpathlineto{\pgfqpoint{2.110870in}{1.307415in}}%
\pgfpathlineto{\pgfqpoint{2.113610in}{1.301438in}}%
\pgfpathlineto{\pgfqpoint{2.119768in}{1.299919in}}%
\pgfpathlineto{\pgfqpoint{2.122094in}{1.293487in}}%
\pgfpathclose%
\pgfusepath{fill}%
\end{pgfscope}%
\begin{pgfscope}%
\pgfpathrectangle{\pgfqpoint{0.100000in}{0.100000in}}{\pgfqpoint{3.420221in}{2.189500in}}%
\pgfusepath{clip}%
\pgfsetbuttcap%
\pgfsetmiterjoin%
\definecolor{currentfill}{rgb}{0.000000,0.180392,0.909804}%
\pgfsetfillcolor{currentfill}%
\pgfsetlinewidth{0.000000pt}%
\definecolor{currentstroke}{rgb}{0.000000,0.000000,0.000000}%
\pgfsetstrokecolor{currentstroke}%
\pgfsetstrokeopacity{0.000000}%
\pgfsetdash{}{0pt}%
\pgfpathmoveto{\pgfqpoint{2.741341in}{1.368968in}}%
\pgfpathlineto{\pgfqpoint{2.741822in}{1.362216in}}%
\pgfpathlineto{\pgfqpoint{2.735295in}{1.361776in}}%
\pgfpathlineto{\pgfqpoint{2.735692in}{1.355144in}}%
\pgfpathlineto{\pgfqpoint{2.722664in}{1.354204in}}%
\pgfpathlineto{\pgfqpoint{2.722756in}{1.352228in}}%
\pgfpathlineto{\pgfqpoint{2.709514in}{1.350316in}}%
\pgfpathlineto{\pgfqpoint{2.708967in}{1.357806in}}%
\pgfpathlineto{\pgfqpoint{2.694600in}{1.358789in}}%
\pgfpathlineto{\pgfqpoint{2.678338in}{1.357370in}}%
\pgfpathlineto{\pgfqpoint{2.677427in}{1.370789in}}%
\pgfpathlineto{\pgfqpoint{2.654303in}{1.369454in}}%
\pgfpathlineto{\pgfqpoint{2.657550in}{1.373782in}}%
\pgfpathlineto{\pgfqpoint{2.659831in}{1.392332in}}%
\pgfpathlineto{\pgfqpoint{2.658594in}{1.408417in}}%
\pgfpathlineto{\pgfqpoint{2.655308in}{1.408283in}}%
\pgfpathlineto{\pgfqpoint{2.654491in}{1.428927in}}%
\pgfpathlineto{\pgfqpoint{2.660228in}{1.429749in}}%
\pgfpathlineto{\pgfqpoint{2.669838in}{1.431126in}}%
\pgfpathlineto{\pgfqpoint{2.670417in}{1.426431in}}%
\pgfpathlineto{\pgfqpoint{2.689013in}{1.426690in}}%
\pgfpathlineto{\pgfqpoint{2.689280in}{1.422441in}}%
\pgfpathlineto{\pgfqpoint{2.705453in}{1.423633in}}%
\pgfpathlineto{\pgfqpoint{2.703569in}{1.438875in}}%
\pgfpathlineto{\pgfqpoint{2.718684in}{1.441032in}}%
\pgfpathlineto{\pgfqpoint{2.727229in}{1.443546in}}%
\pgfpathlineto{\pgfqpoint{2.729508in}{1.444244in}}%
\pgfpathlineto{\pgfqpoint{2.732055in}{1.419319in}}%
\pgfpathlineto{\pgfqpoint{2.733296in}{1.414033in}}%
\pgfpathlineto{\pgfqpoint{2.734936in}{1.397863in}}%
\pgfpathlineto{\pgfqpoint{2.733318in}{1.394842in}}%
\pgfpathlineto{\pgfqpoint{2.720118in}{1.394288in}}%
\pgfpathlineto{\pgfqpoint{2.720525in}{1.387297in}}%
\pgfpathlineto{\pgfqpoint{2.724921in}{1.387583in}}%
\pgfpathlineto{\pgfqpoint{2.727575in}{1.381116in}}%
\pgfpathlineto{\pgfqpoint{2.728297in}{1.370188in}}%
\pgfpathclose%
\pgfusepath{fill}%
\end{pgfscope}%
\begin{pgfscope}%
\pgfpathrectangle{\pgfqpoint{0.100000in}{0.100000in}}{\pgfqpoint{3.420221in}{2.189500in}}%
\pgfusepath{clip}%
\pgfsetbuttcap%
\pgfsetmiterjoin%
\definecolor{currentfill}{rgb}{0.000000,0.482353,0.758824}%
\pgfsetfillcolor{currentfill}%
\pgfsetlinewidth{0.000000pt}%
\definecolor{currentstroke}{rgb}{0.000000,0.000000,0.000000}%
\pgfsetstrokecolor{currentstroke}%
\pgfsetstrokeopacity{0.000000}%
\pgfsetdash{}{0pt}%
\pgfpathmoveto{\pgfqpoint{2.024686in}{1.359404in}}%
\pgfpathlineto{\pgfqpoint{2.025026in}{1.378930in}}%
\pgfpathlineto{\pgfqpoint{2.023056in}{1.392746in}}%
\pgfpathlineto{\pgfqpoint{2.032210in}{1.392764in}}%
\pgfpathlineto{\pgfqpoint{2.071871in}{1.393650in}}%
\pgfpathlineto{\pgfqpoint{2.072911in}{1.360341in}}%
\pgfpathlineto{\pgfqpoint{2.046994in}{1.360312in}}%
\pgfpathlineto{\pgfqpoint{2.046924in}{1.352708in}}%
\pgfpathlineto{\pgfqpoint{2.024746in}{1.352890in}}%
\pgfpathclose%
\pgfusepath{fill}%
\end{pgfscope}%
\begin{pgfscope}%
\pgfpathrectangle{\pgfqpoint{0.100000in}{0.100000in}}{\pgfqpoint{3.420221in}{2.189500in}}%
\pgfusepath{clip}%
\pgfsetbuttcap%
\pgfsetmiterjoin%
\definecolor{currentfill}{rgb}{0.000000,0.254902,0.872549}%
\pgfsetfillcolor{currentfill}%
\pgfsetlinewidth{0.000000pt}%
\definecolor{currentstroke}{rgb}{0.000000,0.000000,0.000000}%
\pgfsetstrokecolor{currentstroke}%
\pgfsetstrokeopacity{0.000000}%
\pgfsetdash{}{0pt}%
\pgfpathmoveto{\pgfqpoint{1.970197in}{1.690924in}}%
\pgfpathlineto{\pgfqpoint{1.969756in}{1.697528in}}%
\pgfpathlineto{\pgfqpoint{1.924343in}{1.698139in}}%
\pgfpathlineto{\pgfqpoint{1.924742in}{1.724212in}}%
\pgfpathlineto{\pgfqpoint{1.925048in}{1.746180in}}%
\pgfpathlineto{\pgfqpoint{1.934000in}{1.744389in}}%
\pgfpathlineto{\pgfqpoint{1.939838in}{1.740212in}}%
\pgfpathlineto{\pgfqpoint{1.942783in}{1.743728in}}%
\pgfpathlineto{\pgfqpoint{1.943001in}{1.756699in}}%
\pgfpathlineto{\pgfqpoint{1.995139in}{1.756374in}}%
\pgfpathlineto{\pgfqpoint{2.014643in}{1.756416in}}%
\pgfpathlineto{\pgfqpoint{2.014648in}{1.749880in}}%
\pgfpathlineto{\pgfqpoint{2.015011in}{1.717178in}}%
\pgfpathlineto{\pgfqpoint{1.976316in}{1.717134in}}%
\pgfpathlineto{\pgfqpoint{1.976292in}{1.706766in}}%
\pgfpathlineto{\pgfqpoint{1.982835in}{1.702622in}}%
\pgfpathlineto{\pgfqpoint{1.982776in}{1.690811in}}%
\pgfpathclose%
\pgfusepath{fill}%
\end{pgfscope}%
\begin{pgfscope}%
\pgfpathrectangle{\pgfqpoint{0.100000in}{0.100000in}}{\pgfqpoint{3.420221in}{2.189500in}}%
\pgfusepath{clip}%
\pgfsetbuttcap%
\pgfsetmiterjoin%
\definecolor{currentfill}{rgb}{0.000000,0.576471,0.711765}%
\pgfsetfillcolor{currentfill}%
\pgfsetlinewidth{0.000000pt}%
\definecolor{currentstroke}{rgb}{0.000000,0.000000,0.000000}%
\pgfsetstrokecolor{currentstroke}%
\pgfsetstrokeopacity{0.000000}%
\pgfsetdash{}{0pt}%
\pgfpathmoveto{\pgfqpoint{2.136270in}{1.112219in}}%
\pgfpathlineto{\pgfqpoint{2.136386in}{1.089101in}}%
\pgfpathlineto{\pgfqpoint{2.131453in}{1.089002in}}%
\pgfpathlineto{\pgfqpoint{2.070421in}{1.087952in}}%
\pgfpathlineto{\pgfqpoint{2.057743in}{1.087809in}}%
\pgfpathlineto{\pgfqpoint{2.058049in}{1.106494in}}%
\pgfpathlineto{\pgfqpoint{2.058615in}{1.147148in}}%
\pgfpathlineto{\pgfqpoint{2.084056in}{1.146860in}}%
\pgfpathlineto{\pgfqpoint{2.085293in}{1.146851in}}%
\pgfpathlineto{\pgfqpoint{2.085321in}{1.125434in}}%
\pgfpathlineto{\pgfqpoint{2.101648in}{1.125459in}}%
\pgfpathlineto{\pgfqpoint{2.101613in}{1.112322in}}%
\pgfpathclose%
\pgfusepath{fill}%
\end{pgfscope}%
\begin{pgfscope}%
\pgfpathrectangle{\pgfqpoint{0.100000in}{0.100000in}}{\pgfqpoint{3.420221in}{2.189500in}}%
\pgfusepath{clip}%
\pgfsetbuttcap%
\pgfsetmiterjoin%
\definecolor{currentfill}{rgb}{0.000000,0.580392,0.709804}%
\pgfsetfillcolor{currentfill}%
\pgfsetlinewidth{0.000000pt}%
\definecolor{currentstroke}{rgb}{0.000000,0.000000,0.000000}%
\pgfsetstrokecolor{currentstroke}%
\pgfsetstrokeopacity{0.000000}%
\pgfsetdash{}{0pt}%
\pgfpathmoveto{\pgfqpoint{2.586134in}{1.128384in}}%
\pgfpathlineto{\pgfqpoint{2.615694in}{1.129893in}}%
\pgfpathlineto{\pgfqpoint{2.619560in}{1.124392in}}%
\pgfpathlineto{\pgfqpoint{2.606437in}{1.127189in}}%
\pgfpathlineto{\pgfqpoint{2.600590in}{1.124847in}}%
\pgfpathlineto{\pgfqpoint{2.597242in}{1.113042in}}%
\pgfpathlineto{\pgfqpoint{2.597969in}{1.105382in}}%
\pgfpathlineto{\pgfqpoint{2.601243in}{1.097560in}}%
\pgfpathlineto{\pgfqpoint{2.597800in}{1.089892in}}%
\pgfpathlineto{\pgfqpoint{2.592741in}{1.089813in}}%
\pgfpathlineto{\pgfqpoint{2.592049in}{1.080185in}}%
\pgfpathlineto{\pgfqpoint{2.585940in}{1.084568in}}%
\pgfpathlineto{\pgfqpoint{2.576534in}{1.086301in}}%
\pgfpathlineto{\pgfqpoint{2.568136in}{1.080556in}}%
\pgfpathlineto{\pgfqpoint{2.565748in}{1.084237in}}%
\pgfpathlineto{\pgfqpoint{2.560394in}{1.084352in}}%
\pgfpathlineto{\pgfqpoint{2.558094in}{1.088134in}}%
\pgfpathlineto{\pgfqpoint{2.558008in}{1.093760in}}%
\pgfpathlineto{\pgfqpoint{2.554559in}{1.099582in}}%
\pgfpathlineto{\pgfqpoint{2.554920in}{1.115547in}}%
\pgfpathlineto{\pgfqpoint{2.561330in}{1.117929in}}%
\pgfpathlineto{\pgfqpoint{2.569396in}{1.114925in}}%
\pgfpathlineto{\pgfqpoint{2.574770in}{1.110337in}}%
\pgfpathlineto{\pgfqpoint{2.577397in}{1.117996in}}%
\pgfpathlineto{\pgfqpoint{2.585532in}{1.120825in}}%
\pgfpathclose%
\pgfusepath{fill}%
\end{pgfscope}%
\begin{pgfscope}%
\pgfpathrectangle{\pgfqpoint{0.100000in}{0.100000in}}{\pgfqpoint{3.420221in}{2.189500in}}%
\pgfusepath{clip}%
\pgfsetbuttcap%
\pgfsetmiterjoin%
\definecolor{currentfill}{rgb}{0.000000,0.258824,0.870588}%
\pgfsetfillcolor{currentfill}%
\pgfsetlinewidth{0.000000pt}%
\definecolor{currentstroke}{rgb}{0.000000,0.000000,0.000000}%
\pgfsetstrokecolor{currentstroke}%
\pgfsetstrokeopacity{0.000000}%
\pgfsetdash{}{0pt}%
\pgfpathmoveto{\pgfqpoint{1.509406in}{1.637237in}}%
\pgfpathlineto{\pgfqpoint{1.463565in}{1.641775in}}%
\pgfpathlineto{\pgfqpoint{1.403019in}{1.648421in}}%
\pgfpathlineto{\pgfqpoint{1.409200in}{1.698771in}}%
\pgfpathlineto{\pgfqpoint{1.414273in}{1.737835in}}%
\pgfpathlineto{\pgfqpoint{1.416260in}{1.760382in}}%
\pgfpathlineto{\pgfqpoint{1.422219in}{1.760219in}}%
\pgfpathlineto{\pgfqpoint{1.468508in}{1.754814in}}%
\pgfpathlineto{\pgfqpoint{1.520434in}{1.749117in}}%
\pgfpathlineto{\pgfqpoint{1.514487in}{1.687903in}}%
\pgfpathclose%
\pgfusepath{fill}%
\end{pgfscope}%
\begin{pgfscope}%
\pgfpathrectangle{\pgfqpoint{0.100000in}{0.100000in}}{\pgfqpoint{3.420221in}{2.189500in}}%
\pgfusepath{clip}%
\pgfsetbuttcap%
\pgfsetmiterjoin%
\definecolor{currentfill}{rgb}{0.000000,0.501961,0.749020}%
\pgfsetfillcolor{currentfill}%
\pgfsetlinewidth{0.000000pt}%
\definecolor{currentstroke}{rgb}{0.000000,0.000000,0.000000}%
\pgfsetstrokecolor{currentstroke}%
\pgfsetstrokeopacity{0.000000}%
\pgfsetdash{}{0pt}%
\pgfpathmoveto{\pgfqpoint{1.299547in}{0.257510in}}%
\pgfpathlineto{\pgfqpoint{1.298329in}{0.263168in}}%
\pgfpathlineto{\pgfqpoint{1.292456in}{0.274839in}}%
\pgfpathlineto{\pgfqpoint{1.294345in}{0.281755in}}%
\pgfpathlineto{\pgfqpoint{1.306903in}{0.293442in}}%
\pgfpathlineto{\pgfqpoint{1.310538in}{0.305783in}}%
\pgfpathlineto{\pgfqpoint{1.314940in}{0.314445in}}%
\pgfpathlineto{\pgfqpoint{1.314113in}{0.320379in}}%
\pgfpathlineto{\pgfqpoint{1.318830in}{0.328382in}}%
\pgfpathlineto{\pgfqpoint{1.342022in}{0.331117in}}%
\pgfpathlineto{\pgfqpoint{1.344378in}{0.333389in}}%
\pgfpathlineto{\pgfqpoint{1.345341in}{0.342901in}}%
\pgfpathlineto{\pgfqpoint{1.350536in}{0.350013in}}%
\pgfpathlineto{\pgfqpoint{1.355573in}{0.349103in}}%
\pgfpathlineto{\pgfqpoint{1.359074in}{0.341379in}}%
\pgfpathlineto{\pgfqpoint{1.362400in}{0.329035in}}%
\pgfpathlineto{\pgfqpoint{1.369321in}{0.321370in}}%
\pgfpathlineto{\pgfqpoint{1.375964in}{0.307838in}}%
\pgfpathlineto{\pgfqpoint{1.379288in}{0.289559in}}%
\pgfpathlineto{\pgfqpoint{1.372941in}{0.281272in}}%
\pgfpathlineto{\pgfqpoint{1.378260in}{0.278025in}}%
\pgfpathlineto{\pgfqpoint{1.375006in}{0.271294in}}%
\pgfpathlineto{\pgfqpoint{1.376031in}{0.262996in}}%
\pgfpathlineto{\pgfqpoint{1.379430in}{0.256189in}}%
\pgfpathlineto{\pgfqpoint{1.376457in}{0.253478in}}%
\pgfpathlineto{\pgfqpoint{1.360842in}{0.252484in}}%
\pgfpathlineto{\pgfqpoint{1.348262in}{0.254919in}}%
\pgfpathlineto{\pgfqpoint{1.338167in}{0.261810in}}%
\pgfpathlineto{\pgfqpoint{1.332443in}{0.261158in}}%
\pgfpathlineto{\pgfqpoint{1.320226in}{0.263331in}}%
\pgfpathclose%
\pgfusepath{fill}%
\end{pgfscope}%
\begin{pgfscope}%
\pgfpathrectangle{\pgfqpoint{0.100000in}{0.100000in}}{\pgfqpoint{3.420221in}{2.189500in}}%
\pgfusepath{clip}%
\pgfsetbuttcap%
\pgfsetmiterjoin%
\definecolor{currentfill}{rgb}{0.000000,0.235294,0.882353}%
\pgfsetfillcolor{currentfill}%
\pgfsetlinewidth{0.000000pt}%
\definecolor{currentstroke}{rgb}{0.000000,0.000000,0.000000}%
\pgfsetstrokecolor{currentstroke}%
\pgfsetstrokeopacity{0.000000}%
\pgfsetdash{}{0pt}%
\pgfpathmoveto{\pgfqpoint{2.060534in}{1.691242in}}%
\pgfpathlineto{\pgfqpoint{2.059616in}{1.684719in}}%
\pgfpathlineto{\pgfqpoint{2.022294in}{1.684387in}}%
\pgfpathlineto{\pgfqpoint{2.021917in}{1.704006in}}%
\pgfpathlineto{\pgfqpoint{2.028962in}{1.704071in}}%
\pgfpathlineto{\pgfqpoint{2.028909in}{1.717250in}}%
\pgfpathlineto{\pgfqpoint{2.015011in}{1.717178in}}%
\pgfpathlineto{\pgfqpoint{2.014648in}{1.749880in}}%
\pgfpathlineto{\pgfqpoint{2.034777in}{1.750024in}}%
\pgfpathlineto{\pgfqpoint{2.041308in}{1.746848in}}%
\pgfpathlineto{\pgfqpoint{2.041756in}{1.723906in}}%
\pgfpathlineto{\pgfqpoint{2.054739in}{1.723973in}}%
\pgfpathlineto{\pgfqpoint{2.055057in}{1.704294in}}%
\pgfpathlineto{\pgfqpoint{2.061565in}{1.704354in}}%
\pgfpathclose%
\pgfusepath{fill}%
\end{pgfscope}%
\begin{pgfscope}%
\pgfpathrectangle{\pgfqpoint{0.100000in}{0.100000in}}{\pgfqpoint{3.420221in}{2.189500in}}%
\pgfusepath{clip}%
\pgfsetbuttcap%
\pgfsetmiterjoin%
\definecolor{currentfill}{rgb}{0.000000,0.321569,0.839216}%
\pgfsetfillcolor{currentfill}%
\pgfsetlinewidth{0.000000pt}%
\definecolor{currentstroke}{rgb}{0.000000,0.000000,0.000000}%
\pgfsetstrokecolor{currentstroke}%
\pgfsetstrokeopacity{0.000000}%
\pgfsetdash{}{0pt}%
\pgfpathmoveto{\pgfqpoint{2.504621in}{1.232676in}}%
\pgfpathlineto{\pgfqpoint{2.501935in}{1.226596in}}%
\pgfpathlineto{\pgfqpoint{2.503286in}{1.218158in}}%
\pgfpathlineto{\pgfqpoint{2.498588in}{1.217143in}}%
\pgfpathlineto{\pgfqpoint{2.495202in}{1.210912in}}%
\pgfpathlineto{\pgfqpoint{2.495231in}{1.215889in}}%
\pgfpathlineto{\pgfqpoint{2.489382in}{1.215287in}}%
\pgfpathlineto{\pgfqpoint{2.485977in}{1.221108in}}%
\pgfpathlineto{\pgfqpoint{2.471851in}{1.212375in}}%
\pgfpathlineto{\pgfqpoint{2.471182in}{1.206165in}}%
\pgfpathlineto{\pgfqpoint{2.456709in}{1.211714in}}%
\pgfpathlineto{\pgfqpoint{2.448956in}{1.214546in}}%
\pgfpathlineto{\pgfqpoint{2.441905in}{1.212484in}}%
\pgfpathlineto{\pgfqpoint{2.438734in}{1.205268in}}%
\pgfpathlineto{\pgfqpoint{2.434375in}{1.210233in}}%
\pgfpathlineto{\pgfqpoint{2.425709in}{1.207524in}}%
\pgfpathlineto{\pgfqpoint{2.419270in}{1.208092in}}%
\pgfpathlineto{\pgfqpoint{2.421617in}{1.202024in}}%
\pgfpathlineto{\pgfqpoint{2.414510in}{1.200841in}}%
\pgfpathlineto{\pgfqpoint{2.410202in}{1.207405in}}%
\pgfpathlineto{\pgfqpoint{2.417481in}{1.225676in}}%
\pgfpathlineto{\pgfqpoint{2.414061in}{1.235413in}}%
\pgfpathlineto{\pgfqpoint{2.416324in}{1.239224in}}%
\pgfpathlineto{\pgfqpoint{2.414162in}{1.244298in}}%
\pgfpathlineto{\pgfqpoint{2.414665in}{1.253298in}}%
\pgfpathlineto{\pgfqpoint{2.416854in}{1.258989in}}%
\pgfpathlineto{\pgfqpoint{2.432080in}{1.260021in}}%
\pgfpathlineto{\pgfqpoint{2.436766in}{1.254047in}}%
\pgfpathlineto{\pgfqpoint{2.443307in}{1.258338in}}%
\pgfpathlineto{\pgfqpoint{2.456130in}{1.260278in}}%
\pgfpathlineto{\pgfqpoint{2.464023in}{1.260416in}}%
\pgfpathlineto{\pgfqpoint{2.477661in}{1.258403in}}%
\pgfpathlineto{\pgfqpoint{2.479862in}{1.260974in}}%
\pgfpathlineto{\pgfqpoint{2.488925in}{1.261781in}}%
\pgfpathlineto{\pgfqpoint{2.489880in}{1.251976in}}%
\pgfpathlineto{\pgfqpoint{2.490894in}{1.242161in}}%
\pgfpathlineto{\pgfqpoint{2.497278in}{1.242995in}}%
\pgfpathlineto{\pgfqpoint{2.497597in}{1.238601in}}%
\pgfpathlineto{\pgfqpoint{2.504122in}{1.239158in}}%
\pgfpathclose%
\pgfusepath{fill}%
\end{pgfscope}%
\begin{pgfscope}%
\pgfpathrectangle{\pgfqpoint{0.100000in}{0.100000in}}{\pgfqpoint{3.420221in}{2.189500in}}%
\pgfusepath{clip}%
\pgfsetbuttcap%
\pgfsetmiterjoin%
\definecolor{currentfill}{rgb}{0.000000,0.482353,0.758824}%
\pgfsetfillcolor{currentfill}%
\pgfsetlinewidth{0.000000pt}%
\definecolor{currentstroke}{rgb}{0.000000,0.000000,0.000000}%
\pgfsetstrokecolor{currentstroke}%
\pgfsetstrokeopacity{0.000000}%
\pgfsetdash{}{0pt}%
\pgfpathmoveto{\pgfqpoint{2.180748in}{1.861235in}}%
\pgfpathlineto{\pgfqpoint{2.182449in}{1.815976in}}%
\pgfpathlineto{\pgfqpoint{2.156396in}{1.815109in}}%
\pgfpathlineto{\pgfqpoint{2.143638in}{1.814683in}}%
\pgfpathlineto{\pgfqpoint{2.143098in}{1.834306in}}%
\pgfpathlineto{\pgfqpoint{2.103417in}{1.833445in}}%
\pgfpathlineto{\pgfqpoint{2.102592in}{1.849385in}}%
\pgfpathlineto{\pgfqpoint{2.102378in}{1.879342in}}%
\pgfpathlineto{\pgfqpoint{2.101084in}{1.907111in}}%
\pgfpathlineto{\pgfqpoint{2.100981in}{1.931873in}}%
\pgfpathlineto{\pgfqpoint{2.099691in}{1.944868in}}%
\pgfpathlineto{\pgfqpoint{2.098632in}{1.958199in}}%
\pgfpathlineto{\pgfqpoint{2.098144in}{2.000716in}}%
\pgfpathlineto{\pgfqpoint{2.103349in}{2.000535in}}%
\pgfpathlineto{\pgfqpoint{2.116416in}{1.994424in}}%
\pgfpathlineto{\pgfqpoint{2.121093in}{1.994816in}}%
\pgfpathlineto{\pgfqpoint{2.120240in}{1.986724in}}%
\pgfpathlineto{\pgfqpoint{2.127711in}{1.987771in}}%
\pgfpathlineto{\pgfqpoint{2.135144in}{1.970700in}}%
\pgfpathlineto{\pgfqpoint{2.139587in}{1.972641in}}%
\pgfpathlineto{\pgfqpoint{2.138945in}{1.980122in}}%
\pgfpathlineto{\pgfqpoint{2.150700in}{1.981721in}}%
\pgfpathlineto{\pgfqpoint{2.153390in}{1.974682in}}%
\pgfpathlineto{\pgfqpoint{2.160726in}{1.970493in}}%
\pgfpathlineto{\pgfqpoint{2.168301in}{1.970107in}}%
\pgfpathlineto{\pgfqpoint{2.168692in}{1.963754in}}%
\pgfpathlineto{\pgfqpoint{2.182740in}{1.959347in}}%
\pgfpathlineto{\pgfqpoint{2.192168in}{1.962346in}}%
\pgfpathlineto{\pgfqpoint{2.202883in}{1.970731in}}%
\pgfpathlineto{\pgfqpoint{2.204439in}{1.947977in}}%
\pgfpathlineto{\pgfqpoint{2.205721in}{1.915932in}}%
\pgfpathlineto{\pgfqpoint{2.194510in}{1.901552in}}%
\pgfpathlineto{\pgfqpoint{2.183476in}{1.889257in}}%
\pgfpathlineto{\pgfqpoint{2.178707in}{1.886248in}}%
\pgfpathlineto{\pgfqpoint{2.168375in}{1.875002in}}%
\pgfpathlineto{\pgfqpoint{2.152798in}{1.862881in}}%
\pgfpathlineto{\pgfqpoint{2.156246in}{1.856938in}}%
\pgfpathlineto{\pgfqpoint{2.160645in}{1.854730in}}%
\pgfpathlineto{\pgfqpoint{2.168530in}{1.856117in}}%
\pgfpathclose%
\pgfusepath{fill}%
\end{pgfscope}%
\begin{pgfscope}%
\pgfpathrectangle{\pgfqpoint{0.100000in}{0.100000in}}{\pgfqpoint{3.420221in}{2.189500in}}%
\pgfusepath{clip}%
\pgfsetbuttcap%
\pgfsetmiterjoin%
\definecolor{currentfill}{rgb}{0.000000,0.305882,0.847059}%
\pgfsetfillcolor{currentfill}%
\pgfsetlinewidth{0.000000pt}%
\definecolor{currentstroke}{rgb}{0.000000,0.000000,0.000000}%
\pgfsetstrokecolor{currentstroke}%
\pgfsetstrokeopacity{0.000000}%
\pgfsetdash{}{0pt}%
\pgfpathmoveto{\pgfqpoint{2.173554in}{1.466014in}}%
\pgfpathlineto{\pgfqpoint{2.153835in}{1.465307in}}%
\pgfpathlineto{\pgfqpoint{2.152004in}{1.524370in}}%
\pgfpathlineto{\pgfqpoint{2.190939in}{1.525590in}}%
\pgfpathlineto{\pgfqpoint{2.229698in}{1.527114in}}%
\pgfpathlineto{\pgfqpoint{2.230851in}{1.500944in}}%
\pgfpathlineto{\pgfqpoint{2.204802in}{1.499922in}}%
\pgfpathlineto{\pgfqpoint{2.205050in}{1.493383in}}%
\pgfpathlineto{\pgfqpoint{2.179060in}{1.492544in}}%
\pgfpathlineto{\pgfqpoint{2.180099in}{1.466291in}}%
\pgfpathclose%
\pgfusepath{fill}%
\end{pgfscope}%
\begin{pgfscope}%
\pgfpathrectangle{\pgfqpoint{0.100000in}{0.100000in}}{\pgfqpoint{3.420221in}{2.189500in}}%
\pgfusepath{clip}%
\pgfsetbuttcap%
\pgfsetmiterjoin%
\definecolor{currentfill}{rgb}{0.000000,0.349020,0.825490}%
\pgfsetfillcolor{currentfill}%
\pgfsetlinewidth{0.000000pt}%
\definecolor{currentstroke}{rgb}{0.000000,0.000000,0.000000}%
\pgfsetstrokecolor{currentstroke}%
\pgfsetstrokeopacity{0.000000}%
\pgfsetdash{}{0pt}%
\pgfpathmoveto{\pgfqpoint{2.886313in}{0.795377in}}%
\pgfpathlineto{\pgfqpoint{2.880736in}{0.797338in}}%
\pgfpathlineto{\pgfqpoint{2.872457in}{0.803146in}}%
\pgfpathlineto{\pgfqpoint{2.868518in}{0.809395in}}%
\pgfpathlineto{\pgfqpoint{2.859351in}{0.809674in}}%
\pgfpathlineto{\pgfqpoint{2.853574in}{0.815697in}}%
\pgfpathlineto{\pgfqpoint{2.845603in}{0.818704in}}%
\pgfpathlineto{\pgfqpoint{2.843479in}{0.822023in}}%
\pgfpathlineto{\pgfqpoint{2.864350in}{0.831907in}}%
\pgfpathlineto{\pgfqpoint{2.862519in}{0.836609in}}%
\pgfpathlineto{\pgfqpoint{2.856504in}{0.842230in}}%
\pgfpathlineto{\pgfqpoint{2.854883in}{0.850905in}}%
\pgfpathlineto{\pgfqpoint{2.863299in}{0.858754in}}%
\pgfpathlineto{\pgfqpoint{2.870130in}{0.856407in}}%
\pgfpathlineto{\pgfqpoint{2.885003in}{0.874094in}}%
\pgfpathlineto{\pgfqpoint{2.893560in}{0.865395in}}%
\pgfpathlineto{\pgfqpoint{2.897367in}{0.872307in}}%
\pgfpathlineto{\pgfqpoint{2.906620in}{0.871372in}}%
\pgfpathlineto{\pgfqpoint{2.911697in}{0.865179in}}%
\pgfpathlineto{\pgfqpoint{2.916233in}{0.864727in}}%
\pgfpathlineto{\pgfqpoint{2.921904in}{0.859461in}}%
\pgfpathlineto{\pgfqpoint{2.926000in}{0.852956in}}%
\pgfpathlineto{\pgfqpoint{2.925318in}{0.847692in}}%
\pgfpathlineto{\pgfqpoint{2.914820in}{0.840816in}}%
\pgfpathlineto{\pgfqpoint{2.910187in}{0.832455in}}%
\pgfpathlineto{\pgfqpoint{2.901091in}{0.824920in}}%
\pgfpathlineto{\pgfqpoint{2.903674in}{0.817195in}}%
\pgfpathlineto{\pgfqpoint{2.889607in}{0.796158in}}%
\pgfpathclose%
\pgfusepath{fill}%
\end{pgfscope}%
\begin{pgfscope}%
\pgfpathrectangle{\pgfqpoint{0.100000in}{0.100000in}}{\pgfqpoint{3.420221in}{2.189500in}}%
\pgfusepath{clip}%
\pgfsetbuttcap%
\pgfsetmiterjoin%
\definecolor{currentfill}{rgb}{0.000000,0.764706,0.617647}%
\pgfsetfillcolor{currentfill}%
\pgfsetlinewidth{0.000000pt}%
\definecolor{currentstroke}{rgb}{0.000000,0.000000,0.000000}%
\pgfsetstrokecolor{currentstroke}%
\pgfsetstrokeopacity{0.000000}%
\pgfsetdash{}{0pt}%
\pgfpathmoveto{\pgfqpoint{0.683805in}{1.826899in}}%
\pgfpathlineto{\pgfqpoint{0.727730in}{1.814970in}}%
\pgfpathlineto{\pgfqpoint{0.729459in}{1.821334in}}%
\pgfpathlineto{\pgfqpoint{0.760080in}{1.812628in}}%
\pgfpathlineto{\pgfqpoint{0.754882in}{1.793703in}}%
\pgfpathlineto{\pgfqpoint{0.738615in}{1.730832in}}%
\pgfpathlineto{\pgfqpoint{0.739271in}{1.730658in}}%
\pgfpathlineto{\pgfqpoint{0.727264in}{1.684258in}}%
\pgfpathlineto{\pgfqpoint{0.722921in}{1.663756in}}%
\pgfpathlineto{\pgfqpoint{0.695911in}{1.670479in}}%
\pgfpathlineto{\pgfqpoint{0.662174in}{1.679793in}}%
\pgfpathlineto{\pgfqpoint{0.660242in}{1.680338in}}%
\pgfpathlineto{\pgfqpoint{0.674970in}{1.735022in}}%
\pgfpathlineto{\pgfqpoint{0.644185in}{1.743373in}}%
\pgfpathlineto{\pgfqpoint{0.653595in}{1.774476in}}%
\pgfpathlineto{\pgfqpoint{0.655476in}{1.773949in}}%
\pgfpathlineto{\pgfqpoint{0.664264in}{1.805137in}}%
\pgfpathlineto{\pgfqpoint{0.665960in}{1.811535in}}%
\pgfpathlineto{\pgfqpoint{0.672268in}{1.809748in}}%
\pgfpathlineto{\pgfqpoint{0.677650in}{1.828618in}}%
\pgfpathclose%
\pgfusepath{fill}%
\end{pgfscope}%
\begin{pgfscope}%
\pgfpathrectangle{\pgfqpoint{0.100000in}{0.100000in}}{\pgfqpoint{3.420221in}{2.189500in}}%
\pgfusepath{clip}%
\pgfsetbuttcap%
\pgfsetmiterjoin%
\definecolor{currentfill}{rgb}{0.000000,0.443137,0.778431}%
\pgfsetfillcolor{currentfill}%
\pgfsetlinewidth{0.000000pt}%
\definecolor{currentstroke}{rgb}{0.000000,0.000000,0.000000}%
\pgfsetstrokecolor{currentstroke}%
\pgfsetstrokeopacity{0.000000}%
\pgfsetdash{}{0pt}%
\pgfpathmoveto{\pgfqpoint{2.969786in}{0.388022in}}%
\pgfpathlineto{\pgfqpoint{2.943425in}{0.383453in}}%
\pgfpathlineto{\pgfqpoint{2.940159in}{0.403438in}}%
\pgfpathlineto{\pgfqpoint{2.920396in}{0.400172in}}%
\pgfpathlineto{\pgfqpoint{2.917346in}{0.419570in}}%
\pgfpathlineto{\pgfqpoint{2.914591in}{0.439577in}}%
\pgfpathlineto{\pgfqpoint{2.934530in}{0.442510in}}%
\pgfpathlineto{\pgfqpoint{2.933473in}{0.449208in}}%
\pgfpathlineto{\pgfqpoint{2.940156in}{0.450279in}}%
\pgfpathlineto{\pgfqpoint{2.939092in}{0.456922in}}%
\pgfpathlineto{\pgfqpoint{2.952340in}{0.459572in}}%
\pgfpathlineto{\pgfqpoint{2.944208in}{0.470901in}}%
\pgfpathlineto{\pgfqpoint{2.940828in}{0.470730in}}%
\pgfpathlineto{\pgfqpoint{2.932570in}{0.480238in}}%
\pgfpathlineto{\pgfqpoint{2.935857in}{0.489944in}}%
\pgfpathlineto{\pgfqpoint{2.953850in}{0.492665in}}%
\pgfpathlineto{\pgfqpoint{2.960195in}{0.493731in}}%
\pgfpathlineto{\pgfqpoint{2.961260in}{0.487369in}}%
\pgfpathlineto{\pgfqpoint{2.991825in}{0.492145in}}%
\pgfpathlineto{\pgfqpoint{2.997072in}{0.481741in}}%
\pgfpathlineto{\pgfqpoint{3.007016in}{0.466339in}}%
\pgfpathlineto{\pgfqpoint{3.013118in}{0.455036in}}%
\pgfpathlineto{\pgfqpoint{3.020672in}{0.437924in}}%
\pgfpathlineto{\pgfqpoint{3.022871in}{0.422603in}}%
\pgfpathlineto{\pgfqpoint{3.023709in}{0.401480in}}%
\pgfpathlineto{\pgfqpoint{3.013221in}{0.400828in}}%
\pgfpathlineto{\pgfqpoint{2.968807in}{0.393642in}}%
\pgfpathclose%
\pgfusepath{fill}%
\end{pgfscope}%
\begin{pgfscope}%
\pgfpathrectangle{\pgfqpoint{0.100000in}{0.100000in}}{\pgfqpoint{3.420221in}{2.189500in}}%
\pgfusepath{clip}%
\pgfsetbuttcap%
\pgfsetmiterjoin%
\definecolor{currentfill}{rgb}{0.000000,0.470588,0.764706}%
\pgfsetfillcolor{currentfill}%
\pgfsetlinewidth{0.000000pt}%
\definecolor{currentstroke}{rgb}{0.000000,0.000000,0.000000}%
\pgfsetstrokecolor{currentstroke}%
\pgfsetstrokeopacity{0.000000}%
\pgfsetdash{}{0pt}%
\pgfpathmoveto{\pgfqpoint{2.489880in}{1.251976in}}%
\pgfpathlineto{\pgfqpoint{2.488925in}{1.261781in}}%
\pgfpathlineto{\pgfqpoint{2.479862in}{1.260974in}}%
\pgfpathlineto{\pgfqpoint{2.477661in}{1.258403in}}%
\pgfpathlineto{\pgfqpoint{2.464023in}{1.260416in}}%
\pgfpathlineto{\pgfqpoint{2.456130in}{1.260278in}}%
\pgfpathlineto{\pgfqpoint{2.453011in}{1.271112in}}%
\pgfpathlineto{\pgfqpoint{2.454557in}{1.280800in}}%
\pgfpathlineto{\pgfqpoint{2.460123in}{1.283074in}}%
\pgfpathlineto{\pgfqpoint{2.460331in}{1.287890in}}%
\pgfpathlineto{\pgfqpoint{2.453887in}{1.287376in}}%
\pgfpathlineto{\pgfqpoint{2.452203in}{1.307076in}}%
\pgfpathlineto{\pgfqpoint{2.462998in}{1.307733in}}%
\pgfpathlineto{\pgfqpoint{2.461908in}{1.320763in}}%
\pgfpathlineto{\pgfqpoint{2.468330in}{1.321287in}}%
\pgfpathlineto{\pgfqpoint{2.467603in}{1.331098in}}%
\pgfpathlineto{\pgfqpoint{2.482202in}{1.332178in}}%
\pgfpathlineto{\pgfqpoint{2.485353in}{1.332481in}}%
\pgfpathlineto{\pgfqpoint{2.486320in}{1.322820in}}%
\pgfpathlineto{\pgfqpoint{2.500717in}{1.324094in}}%
\pgfpathlineto{\pgfqpoint{2.503441in}{1.302426in}}%
\pgfpathlineto{\pgfqpoint{2.506445in}{1.302745in}}%
\pgfpathlineto{\pgfqpoint{2.509044in}{1.298646in}}%
\pgfpathlineto{\pgfqpoint{2.510973in}{1.281729in}}%
\pgfpathlineto{\pgfqpoint{2.509170in}{1.279248in}}%
\pgfpathlineto{\pgfqpoint{2.511665in}{1.253968in}}%
\pgfpathclose%
\pgfusepath{fill}%
\end{pgfscope}%
\begin{pgfscope}%
\pgfpathrectangle{\pgfqpoint{0.100000in}{0.100000in}}{\pgfqpoint{3.420221in}{2.189500in}}%
\pgfusepath{clip}%
\pgfsetbuttcap%
\pgfsetmiterjoin%
\definecolor{currentfill}{rgb}{0.000000,0.819608,0.590196}%
\pgfsetfillcolor{currentfill}%
\pgfsetlinewidth{0.000000pt}%
\definecolor{currentstroke}{rgb}{0.000000,0.000000,0.000000}%
\pgfsetstrokecolor{currentstroke}%
\pgfsetstrokeopacity{0.000000}%
\pgfsetdash{}{0pt}%
\pgfpathmoveto{\pgfqpoint{2.146242in}{0.808318in}}%
\pgfpathlineto{\pgfqpoint{2.154579in}{0.808246in}}%
\pgfpathlineto{\pgfqpoint{2.159545in}{0.801966in}}%
\pgfpathlineto{\pgfqpoint{2.166098in}{0.802102in}}%
\pgfpathlineto{\pgfqpoint{2.166636in}{0.782338in}}%
\pgfpathlineto{\pgfqpoint{2.173423in}{0.775966in}}%
\pgfpathlineto{\pgfqpoint{2.174251in}{0.746294in}}%
\pgfpathlineto{\pgfqpoint{2.171335in}{0.736334in}}%
\pgfpathlineto{\pgfqpoint{2.154873in}{0.735918in}}%
\pgfpathlineto{\pgfqpoint{2.154998in}{0.729305in}}%
\pgfpathlineto{\pgfqpoint{2.132292in}{0.728706in}}%
\pgfpathlineto{\pgfqpoint{2.134084in}{0.739706in}}%
\pgfpathlineto{\pgfqpoint{2.137596in}{0.746243in}}%
\pgfpathlineto{\pgfqpoint{2.133784in}{0.761888in}}%
\pgfpathlineto{\pgfqpoint{2.141701in}{0.761995in}}%
\pgfpathlineto{\pgfqpoint{2.144163in}{0.768701in}}%
\pgfpathlineto{\pgfqpoint{2.143528in}{0.785169in}}%
\pgfpathlineto{\pgfqpoint{2.137034in}{0.785064in}}%
\pgfpathlineto{\pgfqpoint{2.136622in}{0.796059in}}%
\pgfpathlineto{\pgfqpoint{2.139710in}{0.808103in}}%
\pgfpathclose%
\pgfusepath{fill}%
\end{pgfscope}%
\begin{pgfscope}%
\pgfpathrectangle{\pgfqpoint{0.100000in}{0.100000in}}{\pgfqpoint{3.420221in}{2.189500in}}%
\pgfusepath{clip}%
\pgfsetbuttcap%
\pgfsetmiterjoin%
\definecolor{currentfill}{rgb}{0.000000,0.270588,0.864706}%
\pgfsetfillcolor{currentfill}%
\pgfsetlinewidth{0.000000pt}%
\definecolor{currentstroke}{rgb}{0.000000,0.000000,0.000000}%
\pgfsetstrokecolor{currentstroke}%
\pgfsetstrokeopacity{0.000000}%
\pgfsetdash{}{0pt}%
\pgfpathmoveto{\pgfqpoint{1.228882in}{1.697003in}}%
\pgfpathlineto{\pgfqpoint{1.235133in}{1.687338in}}%
\pgfpathlineto{\pgfqpoint{1.234194in}{1.681436in}}%
\pgfpathlineto{\pgfqpoint{1.246955in}{1.679402in}}%
\pgfpathlineto{\pgfqpoint{1.259344in}{1.674204in}}%
\pgfpathlineto{\pgfqpoint{1.258831in}{1.670975in}}%
\pgfpathlineto{\pgfqpoint{1.277280in}{1.662703in}}%
\pgfpathlineto{\pgfqpoint{1.287196in}{1.662072in}}%
\pgfpathlineto{\pgfqpoint{1.317505in}{1.657873in}}%
\pgfpathlineto{\pgfqpoint{1.321060in}{1.659586in}}%
\pgfpathlineto{\pgfqpoint{1.321728in}{1.657315in}}%
\pgfpathlineto{\pgfqpoint{1.318972in}{1.631563in}}%
\pgfpathlineto{\pgfqpoint{1.315452in}{1.605775in}}%
\pgfpathlineto{\pgfqpoint{1.313155in}{1.606101in}}%
\pgfpathlineto{\pgfqpoint{1.309652in}{1.580276in}}%
\pgfpathlineto{\pgfqpoint{1.310701in}{1.580131in}}%
\pgfpathlineto{\pgfqpoint{1.308933in}{1.567300in}}%
\pgfpathlineto{\pgfqpoint{1.282658in}{1.571037in}}%
\pgfpathlineto{\pgfqpoint{1.225389in}{1.579745in}}%
\pgfpathlineto{\pgfqpoint{1.225948in}{1.592553in}}%
\pgfpathlineto{\pgfqpoint{1.228774in}{1.611866in}}%
\pgfpathlineto{\pgfqpoint{1.220782in}{1.619446in}}%
\pgfpathlineto{\pgfqpoint{1.214602in}{1.632752in}}%
\pgfpathlineto{\pgfqpoint{1.209149in}{1.639075in}}%
\pgfpathlineto{\pgfqpoint{1.204086in}{1.648719in}}%
\pgfpathlineto{\pgfqpoint{1.202330in}{1.656427in}}%
\pgfpathlineto{\pgfqpoint{1.203048in}{1.667065in}}%
\pgfpathlineto{\pgfqpoint{1.200971in}{1.674853in}}%
\pgfpathlineto{\pgfqpoint{1.185018in}{1.677614in}}%
\pgfpathlineto{\pgfqpoint{1.191665in}{1.717904in}}%
\pgfpathlineto{\pgfqpoint{1.194304in}{1.712969in}}%
\pgfpathlineto{\pgfqpoint{1.199941in}{1.711997in}}%
\pgfpathlineto{\pgfqpoint{1.203328in}{1.700206in}}%
\pgfpathlineto{\pgfqpoint{1.210289in}{1.704167in}}%
\pgfpathlineto{\pgfqpoint{1.212589in}{1.707959in}}%
\pgfpathlineto{\pgfqpoint{1.218369in}{1.709960in}}%
\pgfpathlineto{\pgfqpoint{1.223344in}{1.706806in}}%
\pgfpathclose%
\pgfusepath{fill}%
\end{pgfscope}%
\begin{pgfscope}%
\pgfpathrectangle{\pgfqpoint{0.100000in}{0.100000in}}{\pgfqpoint{3.420221in}{2.189500in}}%
\pgfusepath{clip}%
\pgfsetbuttcap%
\pgfsetmiterjoin%
\definecolor{currentfill}{rgb}{0.000000,0.454902,0.772549}%
\pgfsetfillcolor{currentfill}%
\pgfsetlinewidth{0.000000pt}%
\definecolor{currentstroke}{rgb}{0.000000,0.000000,0.000000}%
\pgfsetstrokecolor{currentstroke}%
\pgfsetstrokeopacity{0.000000}%
\pgfsetdash{}{0pt}%
\pgfpathmoveto{\pgfqpoint{2.296723in}{1.769546in}}%
\pgfpathlineto{\pgfqpoint{2.296135in}{1.776165in}}%
\pgfpathlineto{\pgfqpoint{2.263712in}{1.774146in}}%
\pgfpathlineto{\pgfqpoint{2.261925in}{1.806421in}}%
\pgfpathlineto{\pgfqpoint{2.267954in}{1.806755in}}%
\pgfpathlineto{\pgfqpoint{2.266599in}{1.830712in}}%
\pgfpathlineto{\pgfqpoint{2.311017in}{1.821199in}}%
\pgfpathlineto{\pgfqpoint{2.316486in}{1.818440in}}%
\pgfpathlineto{\pgfqpoint{2.333622in}{1.810795in}}%
\pgfpathlineto{\pgfqpoint{2.334895in}{1.791365in}}%
\pgfpathlineto{\pgfqpoint{2.348064in}{1.792270in}}%
\pgfpathlineto{\pgfqpoint{2.349801in}{1.766279in}}%
\pgfpathlineto{\pgfqpoint{2.336602in}{1.765467in}}%
\pgfpathlineto{\pgfqpoint{2.323491in}{1.764568in}}%
\pgfpathlineto{\pgfqpoint{2.323211in}{1.771085in}}%
\pgfpathclose%
\pgfusepath{fill}%
\end{pgfscope}%
\begin{pgfscope}%
\pgfpathrectangle{\pgfqpoint{0.100000in}{0.100000in}}{\pgfqpoint{3.420221in}{2.189500in}}%
\pgfusepath{clip}%
\pgfsetbuttcap%
\pgfsetmiterjoin%
\definecolor{currentfill}{rgb}{0.000000,0.627451,0.686275}%
\pgfsetfillcolor{currentfill}%
\pgfsetlinewidth{0.000000pt}%
\definecolor{currentstroke}{rgb}{0.000000,0.000000,0.000000}%
\pgfsetstrokecolor{currentstroke}%
\pgfsetstrokeopacity{0.000000}%
\pgfsetdash{}{0pt}%
\pgfpathmoveto{\pgfqpoint{1.072544in}{1.617623in}}%
\pgfpathlineto{\pgfqpoint{1.069149in}{1.606068in}}%
\pgfpathlineto{\pgfqpoint{1.064801in}{1.609097in}}%
\pgfpathlineto{\pgfqpoint{1.057055in}{1.610550in}}%
\pgfpathlineto{\pgfqpoint{1.049680in}{1.620534in}}%
\pgfpathlineto{\pgfqpoint{1.046079in}{1.629285in}}%
\pgfpathlineto{\pgfqpoint{1.039911in}{1.630495in}}%
\pgfpathlineto{\pgfqpoint{1.038681in}{1.624077in}}%
\pgfpathlineto{\pgfqpoint{1.030189in}{1.625741in}}%
\pgfpathlineto{\pgfqpoint{1.028927in}{1.619326in}}%
\pgfpathlineto{\pgfqpoint{1.010036in}{1.622975in}}%
\pgfpathlineto{\pgfqpoint{1.013776in}{1.642167in}}%
\pgfpathlineto{\pgfqpoint{1.004430in}{1.644154in}}%
\pgfpathlineto{\pgfqpoint{1.001651in}{1.647471in}}%
\pgfpathlineto{\pgfqpoint{1.005535in}{1.650532in}}%
\pgfpathlineto{\pgfqpoint{1.003703in}{1.657554in}}%
\pgfpathlineto{\pgfqpoint{1.008569in}{1.683288in}}%
\pgfpathlineto{\pgfqpoint{1.021244in}{1.680720in}}%
\pgfpathlineto{\pgfqpoint{1.023833in}{1.693513in}}%
\pgfpathlineto{\pgfqpoint{1.004826in}{1.697357in}}%
\pgfpathlineto{\pgfqpoint{1.006071in}{1.703435in}}%
\pgfpathlineto{\pgfqpoint{0.991445in}{1.706479in}}%
\pgfpathlineto{\pgfqpoint{0.992760in}{1.712732in}}%
\pgfpathlineto{\pgfqpoint{0.989920in}{1.723862in}}%
\pgfpathlineto{\pgfqpoint{0.985859in}{1.723095in}}%
\pgfpathlineto{\pgfqpoint{0.988761in}{1.726669in}}%
\pgfpathlineto{\pgfqpoint{0.999763in}{1.729917in}}%
\pgfpathlineto{\pgfqpoint{1.007249in}{1.733966in}}%
\pgfpathlineto{\pgfqpoint{1.010918in}{1.749688in}}%
\pgfpathlineto{\pgfqpoint{1.013627in}{1.754982in}}%
\pgfpathlineto{\pgfqpoint{1.016239in}{1.767806in}}%
\pgfpathlineto{\pgfqpoint{1.048049in}{1.761168in}}%
\pgfpathlineto{\pgfqpoint{1.050199in}{1.771971in}}%
\pgfpathlineto{\pgfqpoint{1.053495in}{1.779606in}}%
\pgfpathlineto{\pgfqpoint{1.069657in}{1.776059in}}%
\pgfpathlineto{\pgfqpoint{1.073743in}{1.772782in}}%
\pgfpathlineto{\pgfqpoint{1.076378in}{1.778497in}}%
\pgfpathlineto{\pgfqpoint{1.080619in}{1.780656in}}%
\pgfpathlineto{\pgfqpoint{1.090137in}{1.775425in}}%
\pgfpathlineto{\pgfqpoint{1.102695in}{1.776274in}}%
\pgfpathlineto{\pgfqpoint{1.104434in}{1.771659in}}%
\pgfpathlineto{\pgfqpoint{1.111322in}{1.774272in}}%
\pgfpathlineto{\pgfqpoint{1.121392in}{1.774149in}}%
\pgfpathlineto{\pgfqpoint{1.125542in}{1.783054in}}%
\pgfpathlineto{\pgfqpoint{1.131128in}{1.785288in}}%
\pgfpathlineto{\pgfqpoint{1.134748in}{1.780845in}}%
\pgfpathlineto{\pgfqpoint{1.140741in}{1.764287in}}%
\pgfpathlineto{\pgfqpoint{1.144753in}{1.761592in}}%
\pgfpathlineto{\pgfqpoint{1.138371in}{1.725234in}}%
\pgfpathlineto{\pgfqpoint{1.129473in}{1.723048in}}%
\pgfpathlineto{\pgfqpoint{1.118825in}{1.724160in}}%
\pgfpathlineto{\pgfqpoint{1.114707in}{1.701900in}}%
\pgfpathlineto{\pgfqpoint{1.123055in}{1.700360in}}%
\pgfpathlineto{\pgfqpoint{1.124361in}{1.693859in}}%
\pgfpathlineto{\pgfqpoint{1.132087in}{1.689533in}}%
\pgfpathlineto{\pgfqpoint{1.129687in}{1.675810in}}%
\pgfpathlineto{\pgfqpoint{1.125812in}{1.653943in}}%
\pgfpathlineto{\pgfqpoint{1.070829in}{1.664247in}}%
\pgfpathlineto{\pgfqpoint{1.064401in}{1.655947in}}%
\pgfpathlineto{\pgfqpoint{1.063138in}{1.648000in}}%
\pgfpathlineto{\pgfqpoint{1.064001in}{1.640846in}}%
\pgfpathlineto{\pgfqpoint{1.071637in}{1.640838in}}%
\pgfpathlineto{\pgfqpoint{1.073082in}{1.632862in}}%
\pgfpathclose%
\pgfusepath{fill}%
\end{pgfscope}%
\begin{pgfscope}%
\pgfpathrectangle{\pgfqpoint{0.100000in}{0.100000in}}{\pgfqpoint{3.420221in}{2.189500in}}%
\pgfusepath{clip}%
\pgfsetbuttcap%
\pgfsetmiterjoin%
\definecolor{currentfill}{rgb}{0.000000,0.596078,0.701961}%
\pgfsetfillcolor{currentfill}%
\pgfsetlinewidth{0.000000pt}%
\definecolor{currentstroke}{rgb}{0.000000,0.000000,0.000000}%
\pgfsetstrokecolor{currentstroke}%
\pgfsetstrokeopacity{0.000000}%
\pgfsetdash{}{0pt}%
\pgfpathmoveto{\pgfqpoint{0.685861in}{2.111912in}}%
\pgfpathlineto{\pgfqpoint{0.689924in}{2.104506in}}%
\pgfpathlineto{\pgfqpoint{0.691651in}{2.097276in}}%
\pgfpathlineto{\pgfqpoint{0.700209in}{2.091621in}}%
\pgfpathlineto{\pgfqpoint{0.702501in}{2.087001in}}%
\pgfpathlineto{\pgfqpoint{0.709387in}{2.083201in}}%
\pgfpathlineto{\pgfqpoint{0.714388in}{2.077072in}}%
\pgfpathlineto{\pgfqpoint{0.728971in}{2.073016in}}%
\pgfpathlineto{\pgfqpoint{0.732427in}{2.068726in}}%
\pgfpathlineto{\pgfqpoint{0.727916in}{2.059083in}}%
\pgfpathlineto{\pgfqpoint{0.728787in}{2.047988in}}%
\pgfpathlineto{\pgfqpoint{0.727687in}{2.036842in}}%
\pgfpathlineto{\pgfqpoint{0.724324in}{2.028028in}}%
\pgfpathlineto{\pgfqpoint{0.726916in}{2.023792in}}%
\pgfpathlineto{\pgfqpoint{0.715303in}{1.980922in}}%
\pgfpathlineto{\pgfqpoint{0.686950in}{1.988930in}}%
\pgfpathlineto{\pgfqpoint{0.631964in}{2.005400in}}%
\pgfpathlineto{\pgfqpoint{0.639458in}{2.030290in}}%
\pgfpathlineto{\pgfqpoint{0.645492in}{2.028467in}}%
\pgfpathlineto{\pgfqpoint{0.645679in}{2.037449in}}%
\pgfpathlineto{\pgfqpoint{0.651655in}{2.046186in}}%
\pgfpathlineto{\pgfqpoint{0.652269in}{2.058401in}}%
\pgfpathlineto{\pgfqpoint{0.650027in}{2.065403in}}%
\pgfpathlineto{\pgfqpoint{0.665990in}{2.081257in}}%
\pgfpathlineto{\pgfqpoint{0.667884in}{2.092315in}}%
\pgfpathlineto{\pgfqpoint{0.663638in}{2.094531in}}%
\pgfpathlineto{\pgfqpoint{0.663410in}{2.099346in}}%
\pgfpathlineto{\pgfqpoint{0.669728in}{2.105412in}}%
\pgfpathlineto{\pgfqpoint{0.680228in}{2.111398in}}%
\pgfpathclose%
\pgfusepath{fill}%
\end{pgfscope}%
\begin{pgfscope}%
\pgfpathrectangle{\pgfqpoint{0.100000in}{0.100000in}}{\pgfqpoint{3.420221in}{2.189500in}}%
\pgfusepath{clip}%
\pgfsetbuttcap%
\pgfsetmiterjoin%
\definecolor{currentfill}{rgb}{0.000000,0.760784,0.619608}%
\pgfsetfillcolor{currentfill}%
\pgfsetlinewidth{0.000000pt}%
\definecolor{currentstroke}{rgb}{0.000000,0.000000,0.000000}%
\pgfsetstrokecolor{currentstroke}%
\pgfsetstrokeopacity{0.000000}%
\pgfsetdash{}{0pt}%
\pgfpathmoveto{\pgfqpoint{0.516018in}{1.518528in}}%
\pgfpathlineto{\pgfqpoint{0.509905in}{1.519445in}}%
\pgfpathlineto{\pgfqpoint{0.498294in}{1.511210in}}%
\pgfpathlineto{\pgfqpoint{0.492841in}{1.498925in}}%
\pgfpathlineto{\pgfqpoint{0.485875in}{1.497103in}}%
\pgfpathlineto{\pgfqpoint{0.481190in}{1.492511in}}%
\pgfpathlineto{\pgfqpoint{0.477422in}{1.479975in}}%
\pgfpathlineto{\pgfqpoint{0.469625in}{1.481111in}}%
\pgfpathlineto{\pgfqpoint{0.464362in}{1.488016in}}%
\pgfpathlineto{\pgfqpoint{0.465159in}{1.492099in}}%
\pgfpathlineto{\pgfqpoint{0.460710in}{1.498576in}}%
\pgfpathlineto{\pgfqpoint{0.432606in}{1.507192in}}%
\pgfpathlineto{\pgfqpoint{0.430694in}{1.514620in}}%
\pgfpathlineto{\pgfqpoint{0.427040in}{1.519065in}}%
\pgfpathlineto{\pgfqpoint{0.429225in}{1.530605in}}%
\pgfpathlineto{\pgfqpoint{0.422884in}{1.533660in}}%
\pgfpathlineto{\pgfqpoint{0.417527in}{1.540172in}}%
\pgfpathlineto{\pgfqpoint{0.422628in}{1.551639in}}%
\pgfpathlineto{\pgfqpoint{0.425385in}{1.560976in}}%
\pgfpathlineto{\pgfqpoint{0.417130in}{1.563539in}}%
\pgfpathlineto{\pgfqpoint{0.419641in}{1.572866in}}%
\pgfpathlineto{\pgfqpoint{0.419194in}{1.580023in}}%
\pgfpathlineto{\pgfqpoint{0.468151in}{1.564704in}}%
\pgfpathlineto{\pgfqpoint{0.470128in}{1.570850in}}%
\pgfpathlineto{\pgfqpoint{0.483386in}{1.566845in}}%
\pgfpathlineto{\pgfqpoint{0.491100in}{1.572340in}}%
\pgfpathlineto{\pgfqpoint{0.494206in}{1.571252in}}%
\pgfpathlineto{\pgfqpoint{0.500034in}{1.578731in}}%
\pgfpathlineto{\pgfqpoint{0.509357in}{1.579952in}}%
\pgfpathlineto{\pgfqpoint{0.511734in}{1.575858in}}%
\pgfpathlineto{\pgfqpoint{0.507376in}{1.569897in}}%
\pgfpathlineto{\pgfqpoint{0.504692in}{1.559411in}}%
\pgfpathlineto{\pgfqpoint{0.507145in}{1.557431in}}%
\pgfpathlineto{\pgfqpoint{0.517948in}{1.527591in}}%
\pgfpathclose%
\pgfusepath{fill}%
\end{pgfscope}%
\begin{pgfscope}%
\pgfpathrectangle{\pgfqpoint{0.100000in}{0.100000in}}{\pgfqpoint{3.420221in}{2.189500in}}%
\pgfusepath{clip}%
\pgfsetbuttcap%
\pgfsetmiterjoin%
\definecolor{currentfill}{rgb}{0.000000,0.368627,0.815686}%
\pgfsetfillcolor{currentfill}%
\pgfsetlinewidth{0.000000pt}%
\definecolor{currentstroke}{rgb}{0.000000,0.000000,0.000000}%
\pgfsetstrokecolor{currentstroke}%
\pgfsetstrokeopacity{0.000000}%
\pgfsetdash{}{0pt}%
\pgfpathmoveto{\pgfqpoint{1.972284in}{1.232687in}}%
\pgfpathlineto{\pgfqpoint{1.971468in}{1.203012in}}%
\pgfpathlineto{\pgfqpoint{1.945500in}{1.203383in}}%
\pgfpathlineto{\pgfqpoint{1.945616in}{1.213179in}}%
\pgfpathlineto{\pgfqpoint{1.922167in}{1.213597in}}%
\pgfpathlineto{\pgfqpoint{1.922021in}{1.207088in}}%
\pgfpathlineto{\pgfqpoint{1.912311in}{1.207284in}}%
\pgfpathlineto{\pgfqpoint{1.893548in}{1.207568in}}%
\pgfpathlineto{\pgfqpoint{1.894125in}{1.227274in}}%
\pgfpathlineto{\pgfqpoint{1.895457in}{1.240273in}}%
\pgfpathlineto{\pgfqpoint{1.922761in}{1.239736in}}%
\pgfpathlineto{\pgfqpoint{1.923073in}{1.256037in}}%
\pgfpathlineto{\pgfqpoint{1.946836in}{1.255712in}}%
\pgfpathlineto{\pgfqpoint{1.946339in}{1.232897in}}%
\pgfpathclose%
\pgfusepath{fill}%
\end{pgfscope}%
\begin{pgfscope}%
\pgfpathrectangle{\pgfqpoint{0.100000in}{0.100000in}}{\pgfqpoint{3.420221in}{2.189500in}}%
\pgfusepath{clip}%
\pgfsetbuttcap%
\pgfsetmiterjoin%
\definecolor{currentfill}{rgb}{0.000000,0.490196,0.754902}%
\pgfsetfillcolor{currentfill}%
\pgfsetlinewidth{0.000000pt}%
\definecolor{currentstroke}{rgb}{0.000000,0.000000,0.000000}%
\pgfsetstrokecolor{currentstroke}%
\pgfsetstrokeopacity{0.000000}%
\pgfsetdash{}{0pt}%
\pgfpathmoveto{\pgfqpoint{1.240781in}{1.915007in}}%
\pgfpathlineto{\pgfqpoint{1.237015in}{1.915587in}}%
\pgfpathlineto{\pgfqpoint{1.230051in}{1.921291in}}%
\pgfpathlineto{\pgfqpoint{1.204201in}{1.919987in}}%
\pgfpathlineto{\pgfqpoint{1.199482in}{1.925029in}}%
\pgfpathlineto{\pgfqpoint{1.191532in}{1.938875in}}%
\pgfpathlineto{\pgfqpoint{1.184875in}{1.946552in}}%
\pgfpathlineto{\pgfqpoint{1.172823in}{1.951056in}}%
\pgfpathlineto{\pgfqpoint{1.153351in}{1.953743in}}%
\pgfpathlineto{\pgfqpoint{1.152149in}{1.947275in}}%
\pgfpathlineto{\pgfqpoint{1.146113in}{1.948385in}}%
\pgfpathlineto{\pgfqpoint{1.139417in}{1.949675in}}%
\pgfpathlineto{\pgfqpoint{1.142445in}{1.965665in}}%
\pgfpathlineto{\pgfqpoint{1.135051in}{1.968034in}}%
\pgfpathlineto{\pgfqpoint{1.130410in}{1.972843in}}%
\pgfpathlineto{\pgfqpoint{1.134856in}{1.996920in}}%
\pgfpathlineto{\pgfqpoint{1.123097in}{1.998390in}}%
\pgfpathlineto{\pgfqpoint{1.117341in}{2.003240in}}%
\pgfpathlineto{\pgfqpoint{1.117293in}{2.006994in}}%
\pgfpathlineto{\pgfqpoint{1.109896in}{2.010783in}}%
\pgfpathlineto{\pgfqpoint{1.097469in}{2.011264in}}%
\pgfpathlineto{\pgfqpoint{1.093969in}{2.018698in}}%
\pgfpathlineto{\pgfqpoint{1.096036in}{2.028452in}}%
\pgfpathlineto{\pgfqpoint{1.094304in}{2.038665in}}%
\pgfpathlineto{\pgfqpoint{1.099271in}{2.043502in}}%
\pgfpathlineto{\pgfqpoint{1.095369in}{2.052264in}}%
\pgfpathlineto{\pgfqpoint{1.137065in}{2.043970in}}%
\pgfpathlineto{\pgfqpoint{1.142932in}{2.039449in}}%
\pgfpathlineto{\pgfqpoint{1.144655in}{2.031235in}}%
\pgfpathlineto{\pgfqpoint{1.173407in}{2.026068in}}%
\pgfpathlineto{\pgfqpoint{1.175357in}{2.036904in}}%
\pgfpathlineto{\pgfqpoint{1.204389in}{2.031780in}}%
\pgfpathlineto{\pgfqpoint{1.205528in}{2.038247in}}%
\pgfpathlineto{\pgfqpoint{1.215569in}{2.036488in}}%
\pgfpathlineto{\pgfqpoint{1.216708in}{2.042984in}}%
\pgfpathlineto{\pgfqpoint{1.255528in}{2.036375in}}%
\pgfpathlineto{\pgfqpoint{1.254757in}{2.030011in}}%
\pgfpathlineto{\pgfqpoint{1.260414in}{2.029067in}}%
\pgfpathlineto{\pgfqpoint{1.259321in}{2.022415in}}%
\pgfpathlineto{\pgfqpoint{1.268989in}{2.021035in}}%
\pgfpathlineto{\pgfqpoint{1.263993in}{1.991396in}}%
\pgfpathlineto{\pgfqpoint{1.249223in}{1.992063in}}%
\pgfpathlineto{\pgfqpoint{1.246036in}{1.979217in}}%
\pgfpathlineto{\pgfqpoint{1.230863in}{1.973266in}}%
\pgfpathlineto{\pgfqpoint{1.228479in}{1.961394in}}%
\pgfpathlineto{\pgfqpoint{1.234410in}{1.957089in}}%
\pgfpathlineto{\pgfqpoint{1.240895in}{1.956022in}}%
\pgfpathlineto{\pgfqpoint{1.246223in}{1.951816in}}%
\pgfpathlineto{\pgfqpoint{1.242759in}{1.930195in}}%
\pgfpathlineto{\pgfqpoint{1.240262in}{1.928370in}}%
\pgfpathclose%
\pgfusepath{fill}%
\end{pgfscope}%
\begin{pgfscope}%
\pgfpathrectangle{\pgfqpoint{0.100000in}{0.100000in}}{\pgfqpoint{3.420221in}{2.189500in}}%
\pgfusepath{clip}%
\pgfsetbuttcap%
\pgfsetmiterjoin%
\definecolor{currentfill}{rgb}{0.000000,0.258824,0.870588}%
\pgfsetfillcolor{currentfill}%
\pgfsetlinewidth{0.000000pt}%
\definecolor{currentstroke}{rgb}{0.000000,0.000000,0.000000}%
\pgfsetstrokecolor{currentstroke}%
\pgfsetstrokeopacity{0.000000}%
\pgfsetdash{}{0pt}%
\pgfpathmoveto{\pgfqpoint{1.718020in}{1.949196in}}%
\pgfpathlineto{\pgfqpoint{1.737696in}{1.948036in}}%
\pgfpathlineto{\pgfqpoint{1.736475in}{1.961354in}}%
\pgfpathlineto{\pgfqpoint{1.737983in}{1.987834in}}%
\pgfpathlineto{\pgfqpoint{1.736759in}{2.001110in}}%
\pgfpathlineto{\pgfqpoint{1.743331in}{2.000767in}}%
\pgfpathlineto{\pgfqpoint{1.776128in}{1.998996in}}%
\pgfpathlineto{\pgfqpoint{1.775492in}{1.985799in}}%
\pgfpathlineto{\pgfqpoint{1.757669in}{1.986742in}}%
\pgfpathlineto{\pgfqpoint{1.756178in}{1.960223in}}%
\pgfpathlineto{\pgfqpoint{1.757312in}{1.946954in}}%
\pgfpathlineto{\pgfqpoint{1.783464in}{1.945588in}}%
\pgfpathlineto{\pgfqpoint{1.784444in}{1.932325in}}%
\pgfpathlineto{\pgfqpoint{1.783199in}{1.906173in}}%
\pgfpathlineto{\pgfqpoint{1.772202in}{1.906702in}}%
\pgfpathlineto{\pgfqpoint{1.739775in}{1.908408in}}%
\pgfpathlineto{\pgfqpoint{1.711080in}{1.910137in}}%
\pgfpathlineto{\pgfqpoint{1.712770in}{1.936344in}}%
\pgfpathlineto{\pgfqpoint{1.717215in}{1.936060in}}%
\pgfpathclose%
\pgfusepath{fill}%
\end{pgfscope}%
\begin{pgfscope}%
\pgfpathrectangle{\pgfqpoint{0.100000in}{0.100000in}}{\pgfqpoint{3.420221in}{2.189500in}}%
\pgfusepath{clip}%
\pgfsetbuttcap%
\pgfsetmiterjoin%
\definecolor{currentfill}{rgb}{0.000000,0.478431,0.760784}%
\pgfsetfillcolor{currentfill}%
\pgfsetlinewidth{0.000000pt}%
\definecolor{currentstroke}{rgb}{0.000000,0.000000,0.000000}%
\pgfsetstrokecolor{currentstroke}%
\pgfsetstrokeopacity{0.000000}%
\pgfsetdash{}{0pt}%
\pgfpathmoveto{\pgfqpoint{2.288796in}{1.365916in}}%
\pgfpathlineto{\pgfqpoint{2.277542in}{1.361124in}}%
\pgfpathlineto{\pgfqpoint{2.271841in}{1.361627in}}%
\pgfpathlineto{\pgfqpoint{2.270478in}{1.365500in}}%
\pgfpathlineto{\pgfqpoint{2.276776in}{1.370938in}}%
\pgfpathlineto{\pgfqpoint{2.281084in}{1.374849in}}%
\pgfpathlineto{\pgfqpoint{2.285232in}{1.385733in}}%
\pgfpathlineto{\pgfqpoint{2.292313in}{1.391397in}}%
\pgfpathlineto{\pgfqpoint{2.294081in}{1.396589in}}%
\pgfpathlineto{\pgfqpoint{2.293551in}{1.404925in}}%
\pgfpathlineto{\pgfqpoint{2.286981in}{1.404646in}}%
\pgfpathlineto{\pgfqpoint{2.285737in}{1.430772in}}%
\pgfpathlineto{\pgfqpoint{2.305338in}{1.431824in}}%
\pgfpathlineto{\pgfqpoint{2.304181in}{1.451287in}}%
\pgfpathlineto{\pgfqpoint{2.291854in}{1.450612in}}%
\pgfpathlineto{\pgfqpoint{2.290116in}{1.476751in}}%
\pgfpathlineto{\pgfqpoint{2.329072in}{1.479199in}}%
\pgfpathlineto{\pgfqpoint{2.328824in}{1.482466in}}%
\pgfpathlineto{\pgfqpoint{2.360395in}{1.484794in}}%
\pgfpathlineto{\pgfqpoint{2.364070in}{1.445705in}}%
\pgfpathlineto{\pgfqpoint{2.344650in}{1.444212in}}%
\pgfpathlineto{\pgfqpoint{2.346504in}{1.417855in}}%
\pgfpathlineto{\pgfqpoint{2.343810in}{1.411041in}}%
\pgfpathlineto{\pgfqpoint{2.335685in}{1.405385in}}%
\pgfpathlineto{\pgfqpoint{2.327998in}{1.404709in}}%
\pgfpathlineto{\pgfqpoint{2.329581in}{1.384603in}}%
\pgfpathlineto{\pgfqpoint{2.310271in}{1.383031in}}%
\pgfpathlineto{\pgfqpoint{2.311224in}{1.368257in}}%
\pgfpathlineto{\pgfqpoint{2.306264in}{1.370307in}}%
\pgfpathlineto{\pgfqpoint{2.297925in}{1.367631in}}%
\pgfpathlineto{\pgfqpoint{2.290038in}{1.368516in}}%
\pgfpathclose%
\pgfusepath{fill}%
\end{pgfscope}%
\begin{pgfscope}%
\pgfpathrectangle{\pgfqpoint{0.100000in}{0.100000in}}{\pgfqpoint{3.420221in}{2.189500in}}%
\pgfusepath{clip}%
\pgfsetbuttcap%
\pgfsetmiterjoin%
\definecolor{currentfill}{rgb}{0.000000,0.243137,0.878431}%
\pgfsetfillcolor{currentfill}%
\pgfsetlinewidth{0.000000pt}%
\definecolor{currentstroke}{rgb}{0.000000,0.000000,0.000000}%
\pgfsetstrokecolor{currentstroke}%
\pgfsetstrokeopacity{0.000000}%
\pgfsetdash{}{0pt}%
\pgfpathmoveto{\pgfqpoint{1.752841in}{1.723989in}}%
\pgfpathlineto{\pgfqpoint{1.758248in}{1.723697in}}%
\pgfpathlineto{\pgfqpoint{1.804479in}{1.721565in}}%
\pgfpathlineto{\pgfqpoint{1.803654in}{1.701784in}}%
\pgfpathlineto{\pgfqpoint{1.802581in}{1.668886in}}%
\pgfpathlineto{\pgfqpoint{1.790424in}{1.669387in}}%
\pgfpathlineto{\pgfqpoint{1.755377in}{1.670718in}}%
\pgfpathlineto{\pgfqpoint{1.750793in}{1.672835in}}%
\pgfpathlineto{\pgfqpoint{1.751380in}{1.697708in}}%
\pgfpathclose%
\pgfusepath{fill}%
\end{pgfscope}%
\begin{pgfscope}%
\pgfpathrectangle{\pgfqpoint{0.100000in}{0.100000in}}{\pgfqpoint{3.420221in}{2.189500in}}%
\pgfusepath{clip}%
\pgfsetbuttcap%
\pgfsetmiterjoin%
\definecolor{currentfill}{rgb}{0.000000,0.792157,0.603922}%
\pgfsetfillcolor{currentfill}%
\pgfsetlinewidth{0.000000pt}%
\definecolor{currentstroke}{rgb}{0.000000,0.000000,0.000000}%
\pgfsetstrokecolor{currentstroke}%
\pgfsetstrokeopacity{0.000000}%
\pgfsetdash{}{0pt}%
\pgfpathmoveto{\pgfqpoint{2.220108in}{0.677662in}}%
\pgfpathlineto{\pgfqpoint{2.224632in}{0.682934in}}%
\pgfpathlineto{\pgfqpoint{2.220555in}{0.687844in}}%
\pgfpathlineto{\pgfqpoint{2.224382in}{0.695696in}}%
\pgfpathlineto{\pgfqpoint{2.229053in}{0.695193in}}%
\pgfpathlineto{\pgfqpoint{2.230085in}{0.702267in}}%
\pgfpathlineto{\pgfqpoint{2.244798in}{0.706728in}}%
\pgfpathlineto{\pgfqpoint{2.253112in}{0.703237in}}%
\pgfpathlineto{\pgfqpoint{2.259051in}{0.705578in}}%
\pgfpathlineto{\pgfqpoint{2.283848in}{0.706845in}}%
\pgfpathlineto{\pgfqpoint{2.317666in}{0.708697in}}%
\pgfpathlineto{\pgfqpoint{2.321878in}{0.712241in}}%
\pgfpathlineto{\pgfqpoint{2.327148in}{0.712214in}}%
\pgfpathlineto{\pgfqpoint{2.335884in}{0.712671in}}%
\pgfpathlineto{\pgfqpoint{2.335621in}{0.715992in}}%
\pgfpathlineto{\pgfqpoint{2.346581in}{0.716605in}}%
\pgfpathlineto{\pgfqpoint{2.348509in}{0.684053in}}%
\pgfpathlineto{\pgfqpoint{2.343715in}{0.683758in}}%
\pgfpathlineto{\pgfqpoint{2.303644in}{0.681420in}}%
\pgfpathlineto{\pgfqpoint{2.272640in}{0.679838in}}%
\pgfpathclose%
\pgfusepath{fill}%
\end{pgfscope}%
\begin{pgfscope}%
\pgfpathrectangle{\pgfqpoint{0.100000in}{0.100000in}}{\pgfqpoint{3.420221in}{2.189500in}}%
\pgfusepath{clip}%
\pgfsetbuttcap%
\pgfsetmiterjoin%
\definecolor{currentfill}{rgb}{0.000000,0.317647,0.841176}%
\pgfsetfillcolor{currentfill}%
\pgfsetlinewidth{0.000000pt}%
\definecolor{currentstroke}{rgb}{0.000000,0.000000,0.000000}%
\pgfsetstrokecolor{currentstroke}%
\pgfsetstrokeopacity{0.000000}%
\pgfsetdash{}{0pt}%
\pgfpathmoveto{\pgfqpoint{2.110721in}{1.569135in}}%
\pgfpathlineto{\pgfqpoint{2.109782in}{1.613581in}}%
\pgfpathlineto{\pgfqpoint{2.141137in}{1.614429in}}%
\pgfpathlineto{\pgfqpoint{2.180284in}{1.615699in}}%
\pgfpathlineto{\pgfqpoint{2.186791in}{1.615926in}}%
\pgfpathlineto{\pgfqpoint{2.188271in}{1.584492in}}%
\pgfpathlineto{\pgfqpoint{2.162184in}{1.583671in}}%
\pgfpathlineto{\pgfqpoint{2.162575in}{1.570476in}}%
\pgfpathclose%
\pgfusepath{fill}%
\end{pgfscope}%
\begin{pgfscope}%
\pgfpathrectangle{\pgfqpoint{0.100000in}{0.100000in}}{\pgfqpoint{3.420221in}{2.189500in}}%
\pgfusepath{clip}%
\pgfsetbuttcap%
\pgfsetmiterjoin%
\definecolor{currentfill}{rgb}{0.000000,0.376471,0.811765}%
\pgfsetfillcolor{currentfill}%
\pgfsetlinewidth{0.000000pt}%
\definecolor{currentstroke}{rgb}{0.000000,0.000000,0.000000}%
\pgfsetstrokecolor{currentstroke}%
\pgfsetstrokeopacity{0.000000}%
\pgfsetdash{}{0pt}%
\pgfpathmoveto{\pgfqpoint{2.952749in}{1.625817in}}%
\pgfpathlineto{\pgfqpoint{2.945693in}{1.628021in}}%
\pgfpathlineto{\pgfqpoint{2.911749in}{1.621762in}}%
\pgfpathlineto{\pgfqpoint{2.910243in}{1.622669in}}%
\pgfpathlineto{\pgfqpoint{2.895489in}{1.613932in}}%
\pgfpathlineto{\pgfqpoint{2.889402in}{1.611633in}}%
\pgfpathlineto{\pgfqpoint{2.882861in}{1.617368in}}%
\pgfpathlineto{\pgfqpoint{2.879054in}{1.617283in}}%
\pgfpathlineto{\pgfqpoint{2.874496in}{1.618908in}}%
\pgfpathlineto{\pgfqpoint{2.877758in}{1.628786in}}%
\pgfpathlineto{\pgfqpoint{2.884156in}{1.633338in}}%
\pgfpathlineto{\pgfqpoint{2.887050in}{1.637996in}}%
\pgfpathlineto{\pgfqpoint{2.880526in}{1.649419in}}%
\pgfpathlineto{\pgfqpoint{2.875855in}{1.650829in}}%
\pgfpathlineto{\pgfqpoint{2.875035in}{1.656823in}}%
\pgfpathlineto{\pgfqpoint{2.869923in}{1.670288in}}%
\pgfpathlineto{\pgfqpoint{2.880936in}{1.677100in}}%
\pgfpathlineto{\pgfqpoint{2.891133in}{1.681967in}}%
\pgfpathlineto{\pgfqpoint{2.904002in}{1.685749in}}%
\pgfpathlineto{\pgfqpoint{2.926389in}{1.689315in}}%
\pgfpathlineto{\pgfqpoint{2.938416in}{1.689468in}}%
\pgfpathlineto{\pgfqpoint{2.949629in}{1.684227in}}%
\pgfpathlineto{\pgfqpoint{2.960325in}{1.688976in}}%
\pgfpathlineto{\pgfqpoint{2.983157in}{1.693138in}}%
\pgfpathlineto{\pgfqpoint{2.994024in}{1.700763in}}%
\pgfpathlineto{\pgfqpoint{2.999025in}{1.681601in}}%
\pgfpathlineto{\pgfqpoint{2.998823in}{1.673135in}}%
\pgfpathlineto{\pgfqpoint{3.003101in}{1.654285in}}%
\pgfpathlineto{\pgfqpoint{3.007819in}{1.648395in}}%
\pgfpathlineto{\pgfqpoint{3.013288in}{1.643881in}}%
\pgfpathlineto{\pgfqpoint{2.995851in}{1.639685in}}%
\pgfpathlineto{\pgfqpoint{2.997396in}{1.633934in}}%
\pgfpathlineto{\pgfqpoint{2.983479in}{1.634037in}}%
\pgfpathlineto{\pgfqpoint{2.982664in}{1.639752in}}%
\pgfpathlineto{\pgfqpoint{2.954622in}{1.634518in}}%
\pgfpathclose%
\pgfusepath{fill}%
\end{pgfscope}%
\begin{pgfscope}%
\pgfpathrectangle{\pgfqpoint{0.100000in}{0.100000in}}{\pgfqpoint{3.420221in}{2.189500in}}%
\pgfusepath{clip}%
\pgfsetbuttcap%
\pgfsetmiterjoin%
\definecolor{currentfill}{rgb}{0.000000,0.286275,0.856863}%
\pgfsetfillcolor{currentfill}%
\pgfsetlinewidth{0.000000pt}%
\definecolor{currentstroke}{rgb}{0.000000,0.000000,0.000000}%
\pgfsetstrokecolor{currentstroke}%
\pgfsetstrokeopacity{0.000000}%
\pgfsetdash{}{0pt}%
\pgfpathmoveto{\pgfqpoint{2.838600in}{1.396246in}}%
\pgfpathlineto{\pgfqpoint{2.833025in}{1.395364in}}%
\pgfpathlineto{\pgfqpoint{2.827828in}{1.427788in}}%
\pgfpathlineto{\pgfqpoint{2.822137in}{1.463239in}}%
\pgfpathlineto{\pgfqpoint{2.819608in}{1.478966in}}%
\pgfpathlineto{\pgfqpoint{2.839733in}{1.482531in}}%
\pgfpathlineto{\pgfqpoint{2.837585in}{1.493217in}}%
\pgfpathlineto{\pgfqpoint{2.844711in}{1.507420in}}%
\pgfpathlineto{\pgfqpoint{2.861658in}{1.510354in}}%
\pgfpathlineto{\pgfqpoint{2.868011in}{1.497128in}}%
\pgfpathlineto{\pgfqpoint{2.875359in}{1.497930in}}%
\pgfpathlineto{\pgfqpoint{2.886670in}{1.502243in}}%
\pgfpathlineto{\pgfqpoint{2.889929in}{1.505930in}}%
\pgfpathlineto{\pgfqpoint{2.891719in}{1.495647in}}%
\pgfpathlineto{\pgfqpoint{2.914664in}{1.499317in}}%
\pgfpathlineto{\pgfqpoint{2.917269in}{1.485969in}}%
\pgfpathlineto{\pgfqpoint{2.913987in}{1.468265in}}%
\pgfpathlineto{\pgfqpoint{2.907018in}{1.445573in}}%
\pgfpathlineto{\pgfqpoint{2.900073in}{1.434495in}}%
\pgfpathlineto{\pgfqpoint{2.894424in}{1.419233in}}%
\pgfpathlineto{\pgfqpoint{2.897321in}{1.413647in}}%
\pgfpathlineto{\pgfqpoint{2.897071in}{1.406077in}}%
\pgfpathlineto{\pgfqpoint{2.892296in}{1.405213in}}%
\pgfpathclose%
\pgfusepath{fill}%
\end{pgfscope}%
\begin{pgfscope}%
\pgfpathrectangle{\pgfqpoint{0.100000in}{0.100000in}}{\pgfqpoint{3.420221in}{2.189500in}}%
\pgfusepath{clip}%
\pgfsetbuttcap%
\pgfsetmiterjoin%
\definecolor{currentfill}{rgb}{0.000000,0.568627,0.715686}%
\pgfsetfillcolor{currentfill}%
\pgfsetlinewidth{0.000000pt}%
\definecolor{currentstroke}{rgb}{0.000000,0.000000,0.000000}%
\pgfsetstrokecolor{currentstroke}%
\pgfsetstrokeopacity{0.000000}%
\pgfsetdash{}{0pt}%
\pgfpathmoveto{\pgfqpoint{1.699618in}{0.598582in}}%
\pgfpathlineto{\pgfqpoint{1.693811in}{0.598966in}}%
\pgfpathlineto{\pgfqpoint{1.694552in}{0.611485in}}%
\pgfpathlineto{\pgfqpoint{1.690112in}{0.611725in}}%
\pgfpathlineto{\pgfqpoint{1.691200in}{0.628101in}}%
\pgfpathlineto{\pgfqpoint{1.720734in}{0.626327in}}%
\pgfpathlineto{\pgfqpoint{1.721367in}{0.642463in}}%
\pgfpathlineto{\pgfqpoint{1.743511in}{0.641354in}}%
\pgfpathlineto{\pgfqpoint{1.767804in}{0.640448in}}%
\pgfpathlineto{\pgfqpoint{1.766971in}{0.613112in}}%
\pgfpathlineto{\pgfqpoint{1.745207in}{0.613960in}}%
\pgfpathlineto{\pgfqpoint{1.744205in}{0.586924in}}%
\pgfpathlineto{\pgfqpoint{1.727528in}{0.596275in}}%
\pgfpathclose%
\pgfusepath{fill}%
\end{pgfscope}%
\begin{pgfscope}%
\pgfpathrectangle{\pgfqpoint{0.100000in}{0.100000in}}{\pgfqpoint{3.420221in}{2.189500in}}%
\pgfusepath{clip}%
\pgfsetbuttcap%
\pgfsetmiterjoin%
\definecolor{currentfill}{rgb}{0.000000,0.505882,0.747059}%
\pgfsetfillcolor{currentfill}%
\pgfsetlinewidth{0.000000pt}%
\definecolor{currentstroke}{rgb}{0.000000,0.000000,0.000000}%
\pgfsetstrokecolor{currentstroke}%
\pgfsetstrokeopacity{0.000000}%
\pgfsetdash{}{0pt}%
\pgfpathmoveto{\pgfqpoint{2.725880in}{0.923443in}}%
\pgfpathlineto{\pgfqpoint{2.733043in}{0.929172in}}%
\pgfpathlineto{\pgfqpoint{2.743710in}{0.925776in}}%
\pgfpathlineto{\pgfqpoint{2.750216in}{0.926076in}}%
\pgfpathlineto{\pgfqpoint{2.751051in}{0.932978in}}%
\pgfpathlineto{\pgfqpoint{2.756679in}{0.944479in}}%
\pgfpathlineto{\pgfqpoint{2.762024in}{0.948882in}}%
\pgfpathlineto{\pgfqpoint{2.769484in}{0.951418in}}%
\pgfpathlineto{\pgfqpoint{2.775451in}{0.949570in}}%
\pgfpathlineto{\pgfqpoint{2.793412in}{0.940284in}}%
\pgfpathlineto{\pgfqpoint{2.799666in}{0.932480in}}%
\pgfpathlineto{\pgfqpoint{2.794075in}{0.932781in}}%
\pgfpathlineto{\pgfqpoint{2.784669in}{0.926708in}}%
\pgfpathlineto{\pgfqpoint{2.775653in}{0.925144in}}%
\pgfpathlineto{\pgfqpoint{2.771940in}{0.922077in}}%
\pgfpathlineto{\pgfqpoint{2.768566in}{0.914838in}}%
\pgfpathlineto{\pgfqpoint{2.763936in}{0.913553in}}%
\pgfpathlineto{\pgfqpoint{2.762814in}{0.908932in}}%
\pgfpathlineto{\pgfqpoint{2.752540in}{0.909520in}}%
\pgfpathlineto{\pgfqpoint{2.744234in}{0.899763in}}%
\pgfpathlineto{\pgfqpoint{2.735793in}{0.908412in}}%
\pgfpathclose%
\pgfusepath{fill}%
\end{pgfscope}%
\begin{pgfscope}%
\pgfpathrectangle{\pgfqpoint{0.100000in}{0.100000in}}{\pgfqpoint{3.420221in}{2.189500in}}%
\pgfusepath{clip}%
\pgfsetbuttcap%
\pgfsetmiterjoin%
\definecolor{currentfill}{rgb}{0.000000,0.482353,0.758824}%
\pgfsetfillcolor{currentfill}%
\pgfsetlinewidth{0.000000pt}%
\definecolor{currentstroke}{rgb}{0.000000,0.000000,0.000000}%
\pgfsetstrokecolor{currentstroke}%
\pgfsetstrokeopacity{0.000000}%
\pgfsetdash{}{0pt}%
\pgfpathmoveto{\pgfqpoint{2.096079in}{1.353229in}}%
\pgfpathlineto{\pgfqpoint{2.096159in}{1.348314in}}%
\pgfpathlineto{\pgfqpoint{2.073222in}{1.347352in}}%
\pgfpathlineto{\pgfqpoint{2.072911in}{1.360341in}}%
\pgfpathlineto{\pgfqpoint{2.071871in}{1.393650in}}%
\pgfpathlineto{\pgfqpoint{2.084243in}{1.394033in}}%
\pgfpathlineto{\pgfqpoint{2.094629in}{1.394208in}}%
\pgfpathclose%
\pgfusepath{fill}%
\end{pgfscope}%
\begin{pgfscope}%
\pgfpathrectangle{\pgfqpoint{0.100000in}{0.100000in}}{\pgfqpoint{3.420221in}{2.189500in}}%
\pgfusepath{clip}%
\pgfsetbuttcap%
\pgfsetmiterjoin%
\definecolor{currentfill}{rgb}{0.000000,0.396078,0.801961}%
\pgfsetfillcolor{currentfill}%
\pgfsetlinewidth{0.000000pt}%
\definecolor{currentstroke}{rgb}{0.000000,0.000000,0.000000}%
\pgfsetstrokecolor{currentstroke}%
\pgfsetstrokeopacity{0.000000}%
\pgfsetdash{}{0pt}%
\pgfpathmoveto{\pgfqpoint{1.736668in}{0.675187in}}%
\pgfpathlineto{\pgfqpoint{1.711232in}{0.676357in}}%
\pgfpathlineto{\pgfqpoint{1.703529in}{0.676752in}}%
\pgfpathlineto{\pgfqpoint{1.704116in}{0.687823in}}%
\pgfpathlineto{\pgfqpoint{1.671009in}{0.689648in}}%
\pgfpathlineto{\pgfqpoint{1.673358in}{0.726783in}}%
\pgfpathlineto{\pgfqpoint{1.665435in}{0.727322in}}%
\pgfpathlineto{\pgfqpoint{1.667580in}{0.765086in}}%
\pgfpathlineto{\pgfqpoint{1.672894in}{0.764801in}}%
\pgfpathlineto{\pgfqpoint{1.700850in}{0.763241in}}%
\pgfpathlineto{\pgfqpoint{1.698297in}{0.724109in}}%
\pgfpathlineto{\pgfqpoint{1.704148in}{0.716603in}}%
\pgfpathlineto{\pgfqpoint{1.716028in}{0.716581in}}%
\pgfpathlineto{\pgfqpoint{1.723953in}{0.711250in}}%
\pgfpathlineto{\pgfqpoint{1.729253in}{0.716524in}}%
\pgfpathlineto{\pgfqpoint{1.738547in}{0.714445in}}%
\pgfpathclose%
\pgfusepath{fill}%
\end{pgfscope}%
\begin{pgfscope}%
\pgfpathrectangle{\pgfqpoint{0.100000in}{0.100000in}}{\pgfqpoint{3.420221in}{2.189500in}}%
\pgfusepath{clip}%
\pgfsetbuttcap%
\pgfsetmiterjoin%
\definecolor{currentfill}{rgb}{0.000000,0.635294,0.682353}%
\pgfsetfillcolor{currentfill}%
\pgfsetlinewidth{0.000000pt}%
\definecolor{currentstroke}{rgb}{0.000000,0.000000,0.000000}%
\pgfsetstrokecolor{currentstroke}%
\pgfsetstrokeopacity{0.000000}%
\pgfsetdash{}{0pt}%
\pgfpathmoveto{\pgfqpoint{2.348509in}{0.684053in}}%
\pgfpathlineto{\pgfqpoint{2.346581in}{0.716605in}}%
\pgfpathlineto{\pgfqpoint{2.335621in}{0.715992in}}%
\pgfpathlineto{\pgfqpoint{2.335884in}{0.712671in}}%
\pgfpathlineto{\pgfqpoint{2.327148in}{0.712214in}}%
\pgfpathlineto{\pgfqpoint{2.325494in}{0.722112in}}%
\pgfpathlineto{\pgfqpoint{2.324590in}{0.740155in}}%
\pgfpathlineto{\pgfqpoint{2.345111in}{0.742764in}}%
\pgfpathlineto{\pgfqpoint{2.361174in}{0.744960in}}%
\pgfpathlineto{\pgfqpoint{2.362994in}{0.717612in}}%
\pgfpathlineto{\pgfqpoint{2.398862in}{0.719911in}}%
\pgfpathlineto{\pgfqpoint{2.402010in}{0.680459in}}%
\pgfpathlineto{\pgfqpoint{2.369214in}{0.678290in}}%
\pgfpathlineto{\pgfqpoint{2.368794in}{0.685172in}}%
\pgfpathlineto{\pgfqpoint{2.358335in}{0.685504in}}%
\pgfpathclose%
\pgfusepath{fill}%
\end{pgfscope}%
\begin{pgfscope}%
\pgfpathrectangle{\pgfqpoint{0.100000in}{0.100000in}}{\pgfqpoint{3.420221in}{2.189500in}}%
\pgfusepath{clip}%
\pgfsetbuttcap%
\pgfsetmiterjoin%
\definecolor{currentfill}{rgb}{0.000000,0.000000,1.000000}%
\pgfsetfillcolor{currentfill}%
\pgfsetlinewidth{0.000000pt}%
\definecolor{currentstroke}{rgb}{0.000000,0.000000,0.000000}%
\pgfsetstrokecolor{currentstroke}%
\pgfsetstrokeopacity{0.000000}%
\pgfsetdash{}{0pt}%
\pgfpathmoveto{\pgfqpoint{2.971807in}{1.336194in}}%
\pgfpathlineto{\pgfqpoint{2.973074in}{1.346484in}}%
\pgfpathlineto{\pgfqpoint{2.966261in}{1.350625in}}%
\pgfpathlineto{\pgfqpoint{2.969651in}{1.359892in}}%
\pgfpathlineto{\pgfqpoint{2.968001in}{1.364489in}}%
\pgfpathlineto{\pgfqpoint{2.982442in}{1.368041in}}%
\pgfpathlineto{\pgfqpoint{2.986124in}{1.365721in}}%
\pgfpathlineto{\pgfqpoint{2.992810in}{1.374739in}}%
\pgfpathlineto{\pgfqpoint{2.994042in}{1.378885in}}%
\pgfpathlineto{\pgfqpoint{2.996817in}{1.393154in}}%
\pgfpathlineto{\pgfqpoint{2.999922in}{1.394659in}}%
\pgfpathlineto{\pgfqpoint{3.001814in}{1.417574in}}%
\pgfpathlineto{\pgfqpoint{3.006635in}{1.426164in}}%
\pgfpathlineto{\pgfqpoint{3.020353in}{1.428862in}}%
\pgfpathlineto{\pgfqpoint{3.017193in}{1.420020in}}%
\pgfpathlineto{\pgfqpoint{3.025502in}{1.417174in}}%
\pgfpathlineto{\pgfqpoint{3.029837in}{1.413515in}}%
\pgfpathlineto{\pgfqpoint{3.027620in}{1.401142in}}%
\pgfpathlineto{\pgfqpoint{3.031535in}{1.396849in}}%
\pgfpathlineto{\pgfqpoint{3.039623in}{1.393507in}}%
\pgfpathlineto{\pgfqpoint{3.042415in}{1.389690in}}%
\pgfpathlineto{\pgfqpoint{3.060686in}{1.380709in}}%
\pgfpathlineto{\pgfqpoint{3.063513in}{1.375170in}}%
\pgfpathlineto{\pgfqpoint{3.062430in}{1.368136in}}%
\pgfpathlineto{\pgfqpoint{3.064245in}{1.363575in}}%
\pgfpathlineto{\pgfqpoint{3.074443in}{1.362826in}}%
\pgfpathlineto{\pgfqpoint{3.077719in}{1.350204in}}%
\pgfpathlineto{\pgfqpoint{3.087027in}{1.340728in}}%
\pgfpathlineto{\pgfqpoint{3.090293in}{1.328067in}}%
\pgfpathlineto{\pgfqpoint{3.095209in}{1.323111in}}%
\pgfpathlineto{\pgfqpoint{3.093932in}{1.316917in}}%
\pgfpathlineto{\pgfqpoint{3.088105in}{1.321389in}}%
\pgfpathlineto{\pgfqpoint{3.073570in}{1.326003in}}%
\pgfpathlineto{\pgfqpoint{3.067291in}{1.324279in}}%
\pgfpathlineto{\pgfqpoint{3.052203in}{1.329223in}}%
\pgfpathlineto{\pgfqpoint{3.047639in}{1.331046in}}%
\pgfpathlineto{\pgfqpoint{3.037759in}{1.326390in}}%
\pgfpathlineto{\pgfqpoint{3.034858in}{1.329046in}}%
\pgfpathlineto{\pgfqpoint{3.032848in}{1.337829in}}%
\pgfpathlineto{\pgfqpoint{3.021830in}{1.343049in}}%
\pgfpathlineto{\pgfqpoint{3.014703in}{1.331820in}}%
\pgfpathlineto{\pgfqpoint{3.015325in}{1.327855in}}%
\pgfpathlineto{\pgfqpoint{3.009477in}{1.327012in}}%
\pgfpathlineto{\pgfqpoint{2.998994in}{1.338041in}}%
\pgfpathlineto{\pgfqpoint{2.994009in}{1.345576in}}%
\pgfpathlineto{\pgfqpoint{2.982931in}{1.330309in}}%
\pgfpathlineto{\pgfqpoint{2.979254in}{1.330335in}}%
\pgfpathclose%
\pgfusepath{fill}%
\end{pgfscope}%
\begin{pgfscope}%
\pgfpathrectangle{\pgfqpoint{0.100000in}{0.100000in}}{\pgfqpoint{3.420221in}{2.189500in}}%
\pgfusepath{clip}%
\pgfsetbuttcap%
\pgfsetmiterjoin%
\definecolor{currentfill}{rgb}{0.000000,0.807843,0.596078}%
\pgfsetfillcolor{currentfill}%
\pgfsetlinewidth{0.000000pt}%
\definecolor{currentstroke}{rgb}{0.000000,0.000000,0.000000}%
\pgfsetstrokecolor{currentstroke}%
\pgfsetstrokeopacity{0.000000}%
\pgfsetdash{}{0pt}%
\pgfpathmoveto{\pgfqpoint{0.579546in}{1.526027in}}%
\pgfpathlineto{\pgfqpoint{0.573802in}{1.506061in}}%
\pgfpathlineto{\pgfqpoint{0.545879in}{1.514102in}}%
\pgfpathlineto{\pgfqpoint{0.543468in}{1.520815in}}%
\pgfpathlineto{\pgfqpoint{0.539243in}{1.522304in}}%
\pgfpathlineto{\pgfqpoint{0.533038in}{1.519030in}}%
\pgfpathlineto{\pgfqpoint{0.516018in}{1.518528in}}%
\pgfpathlineto{\pgfqpoint{0.517948in}{1.527591in}}%
\pgfpathlineto{\pgfqpoint{0.507145in}{1.557431in}}%
\pgfpathlineto{\pgfqpoint{0.504692in}{1.559411in}}%
\pgfpathlineto{\pgfqpoint{0.507376in}{1.569897in}}%
\pgfpathlineto{\pgfqpoint{0.511734in}{1.575858in}}%
\pgfpathlineto{\pgfqpoint{0.509357in}{1.579952in}}%
\pgfpathlineto{\pgfqpoint{0.514525in}{1.583265in}}%
\pgfpathlineto{\pgfqpoint{0.517411in}{1.590350in}}%
\pgfpathlineto{\pgfqpoint{0.511873in}{1.594676in}}%
\pgfpathlineto{\pgfqpoint{0.512340in}{1.602032in}}%
\pgfpathlineto{\pgfqpoint{0.521612in}{1.599241in}}%
\pgfpathlineto{\pgfqpoint{0.536202in}{1.595006in}}%
\pgfpathlineto{\pgfqpoint{0.532169in}{1.581343in}}%
\pgfpathlineto{\pgfqpoint{0.534587in}{1.577862in}}%
\pgfpathlineto{\pgfqpoint{0.542266in}{1.577222in}}%
\pgfpathlineto{\pgfqpoint{0.549720in}{1.580851in}}%
\pgfpathlineto{\pgfqpoint{0.559376in}{1.575663in}}%
\pgfpathlineto{\pgfqpoint{0.568799in}{1.559682in}}%
\pgfpathlineto{\pgfqpoint{0.575457in}{1.555502in}}%
\pgfpathlineto{\pgfqpoint{0.574431in}{1.550144in}}%
\pgfpathlineto{\pgfqpoint{0.578026in}{1.543361in}}%
\pgfpathlineto{\pgfqpoint{0.574372in}{1.530867in}}%
\pgfpathclose%
\pgfusepath{fill}%
\end{pgfscope}%
\begin{pgfscope}%
\pgfpathrectangle{\pgfqpoint{0.100000in}{0.100000in}}{\pgfqpoint{3.420221in}{2.189500in}}%
\pgfusepath{clip}%
\pgfsetbuttcap%
\pgfsetmiterjoin%
\definecolor{currentfill}{rgb}{0.000000,0.427451,0.786275}%
\pgfsetfillcolor{currentfill}%
\pgfsetlinewidth{0.000000pt}%
\definecolor{currentstroke}{rgb}{0.000000,0.000000,0.000000}%
\pgfsetstrokecolor{currentstroke}%
\pgfsetstrokeopacity{0.000000}%
\pgfsetdash{}{0pt}%
\pgfpathmoveto{\pgfqpoint{1.795333in}{1.616714in}}%
\pgfpathlineto{\pgfqpoint{1.795993in}{1.649473in}}%
\pgfpathlineto{\pgfqpoint{1.821765in}{1.648660in}}%
\pgfpathlineto{\pgfqpoint{1.822463in}{1.668113in}}%
\pgfpathlineto{\pgfqpoint{1.848235in}{1.667121in}}%
\pgfpathlineto{\pgfqpoint{1.848459in}{1.667117in}}%
\pgfpathlineto{\pgfqpoint{1.847550in}{1.641167in}}%
\pgfpathlineto{\pgfqpoint{1.860708in}{1.640699in}}%
\pgfpathlineto{\pgfqpoint{1.860027in}{1.614455in}}%
\pgfpathclose%
\pgfusepath{fill}%
\end{pgfscope}%
\begin{pgfscope}%
\pgfpathrectangle{\pgfqpoint{0.100000in}{0.100000in}}{\pgfqpoint{3.420221in}{2.189500in}}%
\pgfusepath{clip}%
\pgfsetbuttcap%
\pgfsetmiterjoin%
\definecolor{currentfill}{rgb}{0.000000,0.647059,0.676471}%
\pgfsetfillcolor{currentfill}%
\pgfsetlinewidth{0.000000pt}%
\definecolor{currentstroke}{rgb}{0.000000,0.000000,0.000000}%
\pgfsetstrokecolor{currentstroke}%
\pgfsetstrokeopacity{0.000000}%
\pgfsetdash{}{0pt}%
\pgfpathmoveto{\pgfqpoint{0.893481in}{2.165550in}}%
\pgfpathlineto{\pgfqpoint{0.883931in}{2.155132in}}%
\pgfpathlineto{\pgfqpoint{0.882174in}{2.148183in}}%
\pgfpathlineto{\pgfqpoint{0.877357in}{2.142611in}}%
\pgfpathlineto{\pgfqpoint{0.883670in}{2.141040in}}%
\pgfpathlineto{\pgfqpoint{0.873835in}{2.102676in}}%
\pgfpathlineto{\pgfqpoint{0.870608in}{2.096681in}}%
\pgfpathlineto{\pgfqpoint{0.865871in}{2.078102in}}%
\pgfpathlineto{\pgfqpoint{0.860521in}{2.083830in}}%
\pgfpathlineto{\pgfqpoint{0.853029in}{2.082786in}}%
\pgfpathlineto{\pgfqpoint{0.849853in}{2.086301in}}%
\pgfpathlineto{\pgfqpoint{0.839236in}{2.085302in}}%
\pgfpathlineto{\pgfqpoint{0.837653in}{2.092309in}}%
\pgfpathlineto{\pgfqpoint{0.833521in}{2.098020in}}%
\pgfpathlineto{\pgfqpoint{0.822460in}{2.092825in}}%
\pgfpathlineto{\pgfqpoint{0.817519in}{2.101389in}}%
\pgfpathlineto{\pgfqpoint{0.804944in}{2.106385in}}%
\pgfpathlineto{\pgfqpoint{0.797951in}{2.107511in}}%
\pgfpathlineto{\pgfqpoint{0.797880in}{2.108980in}}%
\pgfpathlineto{\pgfqpoint{0.797843in}{2.122472in}}%
\pgfpathlineto{\pgfqpoint{0.794397in}{2.124782in}}%
\pgfpathlineto{\pgfqpoint{0.782060in}{2.120169in}}%
\pgfpathlineto{\pgfqpoint{0.777734in}{2.123614in}}%
\pgfpathlineto{\pgfqpoint{0.765691in}{2.120904in}}%
\pgfpathlineto{\pgfqpoint{0.763502in}{2.129377in}}%
\pgfpathlineto{\pgfqpoint{0.755449in}{2.128147in}}%
\pgfpathlineto{\pgfqpoint{0.754359in}{2.120835in}}%
\pgfpathlineto{\pgfqpoint{0.745410in}{2.123458in}}%
\pgfpathlineto{\pgfqpoint{0.744680in}{2.128230in}}%
\pgfpathlineto{\pgfqpoint{0.734545in}{2.142147in}}%
\pgfpathlineto{\pgfqpoint{0.736080in}{2.144858in}}%
\pgfpathlineto{\pgfqpoint{0.730387in}{2.155398in}}%
\pgfpathlineto{\pgfqpoint{0.725746in}{2.167661in}}%
\pgfpathlineto{\pgfqpoint{0.725860in}{2.174090in}}%
\pgfpathlineto{\pgfqpoint{0.723634in}{2.182177in}}%
\pgfpathlineto{\pgfqpoint{0.726097in}{2.183989in}}%
\pgfpathlineto{\pgfqpoint{0.730468in}{2.185708in}}%
\pgfpathlineto{\pgfqpoint{0.728634in}{2.210416in}}%
\pgfpathlineto{\pgfqpoint{0.764871in}{2.199936in}}%
\pgfpathlineto{\pgfqpoint{0.808906in}{2.187735in}}%
\pgfpathlineto{\pgfqpoint{0.841674in}{2.178937in}}%
\pgfpathclose%
\pgfusepath{fill}%
\end{pgfscope}%
\begin{pgfscope}%
\pgfpathrectangle{\pgfqpoint{0.100000in}{0.100000in}}{\pgfqpoint{3.420221in}{2.189500in}}%
\pgfusepath{clip}%
\pgfsetbuttcap%
\pgfsetmiterjoin%
\definecolor{currentfill}{rgb}{0.000000,0.415686,0.792157}%
\pgfsetfillcolor{currentfill}%
\pgfsetlinewidth{0.000000pt}%
\definecolor{currentstroke}{rgb}{0.000000,0.000000,0.000000}%
\pgfsetstrokecolor{currentstroke}%
\pgfsetstrokeopacity{0.000000}%
\pgfsetdash{}{0pt}%
\pgfpathmoveto{\pgfqpoint{2.679603in}{0.845589in}}%
\pgfpathlineto{\pgfqpoint{2.672679in}{0.847767in}}%
\pgfpathlineto{\pgfqpoint{2.664364in}{0.854200in}}%
\pgfpathlineto{\pgfqpoint{2.661467in}{0.863462in}}%
\pgfpathlineto{\pgfqpoint{2.661915in}{0.879426in}}%
\pgfpathlineto{\pgfqpoint{2.661897in}{0.882196in}}%
\pgfpathlineto{\pgfqpoint{2.665953in}{0.882593in}}%
\pgfpathlineto{\pgfqpoint{2.667908in}{0.890104in}}%
\pgfpathlineto{\pgfqpoint{2.676698in}{0.891144in}}%
\pgfpathlineto{\pgfqpoint{2.688011in}{0.888652in}}%
\pgfpathlineto{\pgfqpoint{2.696002in}{0.900954in}}%
\pgfpathlineto{\pgfqpoint{2.700344in}{0.895885in}}%
\pgfpathlineto{\pgfqpoint{2.704743in}{0.886634in}}%
\pgfpathlineto{\pgfqpoint{2.704826in}{0.881484in}}%
\pgfpathlineto{\pgfqpoint{2.701749in}{0.882747in}}%
\pgfpathlineto{\pgfqpoint{2.690899in}{0.881475in}}%
\pgfpathlineto{\pgfqpoint{2.693043in}{0.862407in}}%
\pgfpathlineto{\pgfqpoint{2.688212in}{0.860632in}}%
\pgfpathlineto{\pgfqpoint{2.687756in}{0.850637in}}%
\pgfpathlineto{\pgfqpoint{2.685397in}{0.841915in}}%
\pgfpathclose%
\pgfusepath{fill}%
\end{pgfscope}%
\begin{pgfscope}%
\pgfpathrectangle{\pgfqpoint{0.100000in}{0.100000in}}{\pgfqpoint{3.420221in}{2.189500in}}%
\pgfusepath{clip}%
\pgfsetbuttcap%
\pgfsetmiterjoin%
\definecolor{currentfill}{rgb}{0.000000,0.188235,0.905882}%
\pgfsetfillcolor{currentfill}%
\pgfsetlinewidth{0.000000pt}%
\definecolor{currentstroke}{rgb}{0.000000,0.000000,0.000000}%
\pgfsetstrokecolor{currentstroke}%
\pgfsetstrokeopacity{0.000000}%
\pgfsetdash{}{0pt}%
\pgfpathmoveto{\pgfqpoint{3.074443in}{1.362826in}}%
\pgfpathlineto{\pgfqpoint{3.064245in}{1.363575in}}%
\pgfpathlineto{\pgfqpoint{3.062430in}{1.368136in}}%
\pgfpathlineto{\pgfqpoint{3.063513in}{1.375170in}}%
\pgfpathlineto{\pgfqpoint{3.060686in}{1.380709in}}%
\pgfpathlineto{\pgfqpoint{3.042415in}{1.389690in}}%
\pgfpathlineto{\pgfqpoint{3.039623in}{1.393507in}}%
\pgfpathlineto{\pgfqpoint{3.031535in}{1.396849in}}%
\pgfpathlineto{\pgfqpoint{3.027620in}{1.401142in}}%
\pgfpathlineto{\pgfqpoint{3.029837in}{1.413515in}}%
\pgfpathlineto{\pgfqpoint{3.025502in}{1.417174in}}%
\pgfpathlineto{\pgfqpoint{3.017193in}{1.420020in}}%
\pgfpathlineto{\pgfqpoint{3.020353in}{1.428862in}}%
\pgfpathlineto{\pgfqpoint{3.075488in}{1.440206in}}%
\pgfpathlineto{\pgfqpoint{3.075868in}{1.440278in}}%
\pgfpathlineto{\pgfqpoint{3.087545in}{1.429119in}}%
\pgfpathlineto{\pgfqpoint{3.087690in}{1.420766in}}%
\pgfpathlineto{\pgfqpoint{3.084199in}{1.415707in}}%
\pgfpathlineto{\pgfqpoint{3.074593in}{1.408367in}}%
\pgfpathlineto{\pgfqpoint{3.072045in}{1.399230in}}%
\pgfpathlineto{\pgfqpoint{3.074879in}{1.388561in}}%
\pgfpathlineto{\pgfqpoint{3.075439in}{1.377866in}}%
\pgfpathlineto{\pgfqpoint{3.072374in}{1.364672in}}%
\pgfpathclose%
\pgfusepath{fill}%
\end{pgfscope}%
\begin{pgfscope}%
\pgfpathrectangle{\pgfqpoint{0.100000in}{0.100000in}}{\pgfqpoint{3.420221in}{2.189500in}}%
\pgfusepath{clip}%
\pgfsetbuttcap%
\pgfsetmiterjoin%
\definecolor{currentfill}{rgb}{0.000000,0.435294,0.782353}%
\pgfsetfillcolor{currentfill}%
\pgfsetlinewidth{0.000000pt}%
\definecolor{currentstroke}{rgb}{0.000000,0.000000,0.000000}%
\pgfsetstrokecolor{currentstroke}%
\pgfsetstrokeopacity{0.000000}%
\pgfsetdash{}{0pt}%
\pgfpathmoveto{\pgfqpoint{1.732013in}{1.186971in}}%
\pgfpathlineto{\pgfqpoint{1.731718in}{1.193782in}}%
\pgfpathlineto{\pgfqpoint{1.733349in}{1.226463in}}%
\pgfpathlineto{\pgfqpoint{1.732547in}{1.226504in}}%
\pgfpathlineto{\pgfqpoint{1.734136in}{1.259007in}}%
\pgfpathlineto{\pgfqpoint{1.798329in}{1.256189in}}%
\pgfpathlineto{\pgfqpoint{1.796636in}{1.190847in}}%
\pgfpathlineto{\pgfqpoint{1.764647in}{1.192197in}}%
\pgfpathlineto{\pgfqpoint{1.764329in}{1.185302in}}%
\pgfpathclose%
\pgfusepath{fill}%
\end{pgfscope}%
\begin{pgfscope}%
\pgfpathrectangle{\pgfqpoint{0.100000in}{0.100000in}}{\pgfqpoint{3.420221in}{2.189500in}}%
\pgfusepath{clip}%
\pgfsetbuttcap%
\pgfsetmiterjoin%
\definecolor{currentfill}{rgb}{0.000000,0.615686,0.692157}%
\pgfsetfillcolor{currentfill}%
\pgfsetlinewidth{0.000000pt}%
\definecolor{currentstroke}{rgb}{0.000000,0.000000,0.000000}%
\pgfsetstrokecolor{currentstroke}%
\pgfsetstrokeopacity{0.000000}%
\pgfsetdash{}{0pt}%
\pgfpathmoveto{\pgfqpoint{2.494079in}{1.713495in}}%
\pgfpathlineto{\pgfqpoint{2.471779in}{1.711703in}}%
\pgfpathlineto{\pgfqpoint{2.469896in}{1.725364in}}%
\pgfpathlineto{\pgfqpoint{2.477849in}{1.729099in}}%
\pgfpathlineto{\pgfqpoint{2.477676in}{1.740225in}}%
\pgfpathlineto{\pgfqpoint{2.482006in}{1.741604in}}%
\pgfpathlineto{\pgfqpoint{2.484272in}{1.746584in}}%
\pgfpathlineto{\pgfqpoint{2.491408in}{1.746402in}}%
\pgfpathlineto{\pgfqpoint{2.493789in}{1.754303in}}%
\pgfpathlineto{\pgfqpoint{2.500148in}{1.760515in}}%
\pgfpathlineto{\pgfqpoint{2.502178in}{1.756619in}}%
\pgfpathlineto{\pgfqpoint{2.502410in}{1.745530in}}%
\pgfpathlineto{\pgfqpoint{2.500313in}{1.742159in}}%
\pgfpathlineto{\pgfqpoint{2.501498in}{1.734100in}}%
\pgfpathlineto{\pgfqpoint{2.507749in}{1.732656in}}%
\pgfpathlineto{\pgfqpoint{2.512817in}{1.745346in}}%
\pgfpathlineto{\pgfqpoint{2.513578in}{1.760752in}}%
\pgfpathlineto{\pgfqpoint{2.511485in}{1.767799in}}%
\pgfpathlineto{\pgfqpoint{2.520221in}{1.768684in}}%
\pgfpathlineto{\pgfqpoint{2.520879in}{1.762090in}}%
\pgfpathlineto{\pgfqpoint{2.540293in}{1.764119in}}%
\pgfpathlineto{\pgfqpoint{2.542941in}{1.744869in}}%
\pgfpathlineto{\pgfqpoint{2.545681in}{1.718849in}}%
\pgfpathclose%
\pgfusepath{fill}%
\end{pgfscope}%
\begin{pgfscope}%
\pgfpathrectangle{\pgfqpoint{0.100000in}{0.100000in}}{\pgfqpoint{3.420221in}{2.189500in}}%
\pgfusepath{clip}%
\pgfsetbuttcap%
\pgfsetmiterjoin%
\definecolor{currentfill}{rgb}{0.000000,0.266667,0.866667}%
\pgfsetfillcolor{currentfill}%
\pgfsetlinewidth{0.000000pt}%
\definecolor{currentstroke}{rgb}{0.000000,0.000000,0.000000}%
\pgfsetstrokecolor{currentstroke}%
\pgfsetstrokeopacity{0.000000}%
\pgfsetdash{}{0pt}%
\pgfpathmoveto{\pgfqpoint{1.739775in}{1.908408in}}%
\pgfpathlineto{\pgfqpoint{1.738251in}{1.882303in}}%
\pgfpathlineto{\pgfqpoint{1.740281in}{1.882184in}}%
\pgfpathlineto{\pgfqpoint{1.738489in}{1.855910in}}%
\pgfpathlineto{\pgfqpoint{1.708404in}{1.857771in}}%
\pgfpathlineto{\pgfqpoint{1.713074in}{1.854420in}}%
\pgfpathlineto{\pgfqpoint{1.714019in}{1.849799in}}%
\pgfpathlineto{\pgfqpoint{1.711159in}{1.841959in}}%
\pgfpathlineto{\pgfqpoint{1.699558in}{1.839401in}}%
\pgfpathlineto{\pgfqpoint{1.695956in}{1.841534in}}%
\pgfpathlineto{\pgfqpoint{1.687807in}{1.832577in}}%
\pgfpathlineto{\pgfqpoint{1.687055in}{1.839227in}}%
\pgfpathlineto{\pgfqpoint{1.674029in}{1.840117in}}%
\pgfpathlineto{\pgfqpoint{1.675386in}{1.859635in}}%
\pgfpathlineto{\pgfqpoint{1.653677in}{1.861242in}}%
\pgfpathlineto{\pgfqpoint{1.654170in}{1.867802in}}%
\pgfpathlineto{\pgfqpoint{1.634626in}{1.869267in}}%
\pgfpathlineto{\pgfqpoint{1.636156in}{1.889136in}}%
\pgfpathlineto{\pgfqpoint{1.633690in}{1.889320in}}%
\pgfpathlineto{\pgfqpoint{1.635765in}{1.915512in}}%
\pgfpathlineto{\pgfqpoint{1.632633in}{1.915750in}}%
\pgfpathlineto{\pgfqpoint{1.634124in}{1.934364in}}%
\pgfpathlineto{\pgfqpoint{1.631368in}{1.938249in}}%
\pgfpathlineto{\pgfqpoint{1.632542in}{1.950224in}}%
\pgfpathlineto{\pgfqpoint{1.626060in}{1.948968in}}%
\pgfpathlineto{\pgfqpoint{1.626616in}{1.955744in}}%
\pgfpathlineto{\pgfqpoint{1.652699in}{1.953699in}}%
\pgfpathlineto{\pgfqpoint{1.718020in}{1.949196in}}%
\pgfpathlineto{\pgfqpoint{1.717215in}{1.936060in}}%
\pgfpathlineto{\pgfqpoint{1.712770in}{1.936344in}}%
\pgfpathlineto{\pgfqpoint{1.711080in}{1.910137in}}%
\pgfpathclose%
\pgfusepath{fill}%
\end{pgfscope}%
\begin{pgfscope}%
\pgfpathrectangle{\pgfqpoint{0.100000in}{0.100000in}}{\pgfqpoint{3.420221in}{2.189500in}}%
\pgfusepath{clip}%
\pgfsetbuttcap%
\pgfsetmiterjoin%
\definecolor{currentfill}{rgb}{0.000000,0.290196,0.854902}%
\pgfsetfillcolor{currentfill}%
\pgfsetlinewidth{0.000000pt}%
\definecolor{currentstroke}{rgb}{0.000000,0.000000,0.000000}%
\pgfsetstrokecolor{currentstroke}%
\pgfsetstrokeopacity{0.000000}%
\pgfsetdash{}{0pt}%
\pgfpathmoveto{\pgfqpoint{3.119163in}{1.432425in}}%
\pgfpathlineto{\pgfqpoint{3.118967in}{1.437711in}}%
\pgfpathlineto{\pgfqpoint{3.115396in}{1.442094in}}%
\pgfpathlineto{\pgfqpoint{3.118614in}{1.452479in}}%
\pgfpathlineto{\pgfqpoint{3.118210in}{1.456923in}}%
\pgfpathlineto{\pgfqpoint{3.109958in}{1.456419in}}%
\pgfpathlineto{\pgfqpoint{3.105599in}{1.453697in}}%
\pgfpathlineto{\pgfqpoint{3.101762in}{1.445809in}}%
\pgfpathlineto{\pgfqpoint{3.081378in}{1.441459in}}%
\pgfpathlineto{\pgfqpoint{3.084659in}{1.445088in}}%
\pgfpathlineto{\pgfqpoint{3.086567in}{1.452593in}}%
\pgfpathlineto{\pgfqpoint{3.085737in}{1.460138in}}%
\pgfpathlineto{\pgfqpoint{3.087727in}{1.467166in}}%
\pgfpathlineto{\pgfqpoint{3.086169in}{1.472092in}}%
\pgfpathlineto{\pgfqpoint{3.097884in}{1.484823in}}%
\pgfpathlineto{\pgfqpoint{3.103958in}{1.501884in}}%
\pgfpathlineto{\pgfqpoint{3.109921in}{1.506258in}}%
\pgfpathlineto{\pgfqpoint{3.119879in}{1.517769in}}%
\pgfpathlineto{\pgfqpoint{3.128081in}{1.514497in}}%
\pgfpathlineto{\pgfqpoint{3.130649in}{1.505537in}}%
\pgfpathlineto{\pgfqpoint{3.136068in}{1.505250in}}%
\pgfpathlineto{\pgfqpoint{3.145465in}{1.495992in}}%
\pgfpathlineto{\pgfqpoint{3.161529in}{1.491002in}}%
\pgfpathlineto{\pgfqpoint{3.178732in}{1.466911in}}%
\pgfpathlineto{\pgfqpoint{3.183866in}{1.446964in}}%
\pgfpathlineto{\pgfqpoint{3.188696in}{1.446475in}}%
\pgfpathlineto{\pgfqpoint{3.184391in}{1.436461in}}%
\pgfpathlineto{\pgfqpoint{3.175117in}{1.425772in}}%
\pgfpathlineto{\pgfqpoint{3.168982in}{1.404608in}}%
\pgfpathlineto{\pgfqpoint{3.165800in}{1.399958in}}%
\pgfpathlineto{\pgfqpoint{3.159705in}{1.398600in}}%
\pgfpathlineto{\pgfqpoint{3.160697in}{1.415992in}}%
\pgfpathlineto{\pgfqpoint{3.143097in}{1.418094in}}%
\pgfpathlineto{\pgfqpoint{3.134331in}{1.421203in}}%
\pgfpathlineto{\pgfqpoint{3.122188in}{1.428680in}}%
\pgfpathclose%
\pgfusepath{fill}%
\end{pgfscope}%
\begin{pgfscope}%
\pgfpathrectangle{\pgfqpoint{0.100000in}{0.100000in}}{\pgfqpoint{3.420221in}{2.189500in}}%
\pgfusepath{clip}%
\pgfsetbuttcap%
\pgfsetmiterjoin%
\definecolor{currentfill}{rgb}{0.000000,0.188235,0.905882}%
\pgfsetfillcolor{currentfill}%
\pgfsetlinewidth{0.000000pt}%
\definecolor{currentstroke}{rgb}{0.000000,0.000000,0.000000}%
\pgfsetstrokecolor{currentstroke}%
\pgfsetstrokeopacity{0.000000}%
\pgfsetdash{}{0pt}%
\pgfpathmoveto{\pgfqpoint{1.641044in}{1.166240in}}%
\pgfpathlineto{\pgfqpoint{1.639164in}{1.137076in}}%
\pgfpathlineto{\pgfqpoint{1.609973in}{1.138918in}}%
\pgfpathlineto{\pgfqpoint{1.612016in}{1.168311in}}%
\pgfpathlineto{\pgfqpoint{1.583218in}{1.170522in}}%
\pgfpathlineto{\pgfqpoint{1.584701in}{1.189583in}}%
\pgfpathlineto{\pgfqpoint{1.585215in}{1.196628in}}%
\pgfpathlineto{\pgfqpoint{1.641500in}{1.192399in}}%
\pgfpathlineto{\pgfqpoint{1.639752in}{1.166320in}}%
\pgfpathclose%
\pgfusepath{fill}%
\end{pgfscope}%
\begin{pgfscope}%
\pgfpathrectangle{\pgfqpoint{0.100000in}{0.100000in}}{\pgfqpoint{3.420221in}{2.189500in}}%
\pgfusepath{clip}%
\pgfsetbuttcap%
\pgfsetmiterjoin%
\definecolor{currentfill}{rgb}{0.000000,0.521569,0.739216}%
\pgfsetfillcolor{currentfill}%
\pgfsetlinewidth{0.000000pt}%
\definecolor{currentstroke}{rgb}{0.000000,0.000000,0.000000}%
\pgfsetstrokecolor{currentstroke}%
\pgfsetstrokeopacity{0.000000}%
\pgfsetdash{}{0pt}%
\pgfpathmoveto{\pgfqpoint{2.712617in}{0.722552in}}%
\pgfpathlineto{\pgfqpoint{2.712929in}{0.719879in}}%
\pgfpathlineto{\pgfqpoint{2.730176in}{0.721700in}}%
\pgfpathlineto{\pgfqpoint{2.733239in}{0.693158in}}%
\pgfpathlineto{\pgfqpoint{2.691790in}{0.690555in}}%
\pgfpathlineto{\pgfqpoint{2.688645in}{0.719806in}}%
\pgfpathclose%
\pgfusepath{fill}%
\end{pgfscope}%
\begin{pgfscope}%
\pgfpathrectangle{\pgfqpoint{0.100000in}{0.100000in}}{\pgfqpoint{3.420221in}{2.189500in}}%
\pgfusepath{clip}%
\pgfsetbuttcap%
\pgfsetmiterjoin%
\definecolor{currentfill}{rgb}{0.000000,0.403922,0.798039}%
\pgfsetfillcolor{currentfill}%
\pgfsetlinewidth{0.000000pt}%
\definecolor{currentstroke}{rgb}{0.000000,0.000000,0.000000}%
\pgfsetstrokecolor{currentstroke}%
\pgfsetstrokeopacity{0.000000}%
\pgfsetdash{}{0pt}%
\pgfpathmoveto{\pgfqpoint{2.393456in}{1.207469in}}%
\pgfpathlineto{\pgfqpoint{2.373872in}{1.205956in}}%
\pgfpathlineto{\pgfqpoint{2.372849in}{1.222302in}}%
\pgfpathlineto{\pgfqpoint{2.346690in}{1.220515in}}%
\pgfpathlineto{\pgfqpoint{2.346403in}{1.227100in}}%
\pgfpathlineto{\pgfqpoint{2.313716in}{1.225575in}}%
\pgfpathlineto{\pgfqpoint{2.311593in}{1.258160in}}%
\pgfpathlineto{\pgfqpoint{2.318114in}{1.258602in}}%
\pgfpathlineto{\pgfqpoint{2.317603in}{1.265127in}}%
\pgfpathlineto{\pgfqpoint{2.337622in}{1.266263in}}%
\pgfpathlineto{\pgfqpoint{2.344404in}{1.266271in}}%
\pgfpathlineto{\pgfqpoint{2.343993in}{1.272843in}}%
\pgfpathlineto{\pgfqpoint{2.369769in}{1.274725in}}%
\pgfpathlineto{\pgfqpoint{2.370731in}{1.258264in}}%
\pgfpathlineto{\pgfqpoint{2.372315in}{1.232121in}}%
\pgfpathlineto{\pgfqpoint{2.391798in}{1.233394in}}%
\pgfpathclose%
\pgfusepath{fill}%
\end{pgfscope}%
\begin{pgfscope}%
\pgfpathrectangle{\pgfqpoint{0.100000in}{0.100000in}}{\pgfqpoint{3.420221in}{2.189500in}}%
\pgfusepath{clip}%
\pgfsetbuttcap%
\pgfsetmiterjoin%
\definecolor{currentfill}{rgb}{0.000000,0.313725,0.843137}%
\pgfsetfillcolor{currentfill}%
\pgfsetlinewidth{0.000000pt}%
\definecolor{currentstroke}{rgb}{0.000000,0.000000,0.000000}%
\pgfsetstrokecolor{currentstroke}%
\pgfsetstrokeopacity{0.000000}%
\pgfsetdash{}{0pt}%
\pgfpathmoveto{\pgfqpoint{2.535789in}{0.737656in}}%
\pgfpathlineto{\pgfqpoint{2.526938in}{0.736958in}}%
\pgfpathlineto{\pgfqpoint{2.521894in}{0.744657in}}%
\pgfpathlineto{\pgfqpoint{2.520557in}{0.759546in}}%
\pgfpathlineto{\pgfqpoint{2.519094in}{0.775887in}}%
\pgfpathlineto{\pgfqpoint{2.517810in}{0.789186in}}%
\pgfpathlineto{\pgfqpoint{2.522971in}{0.798418in}}%
\pgfpathlineto{\pgfqpoint{2.515789in}{0.813145in}}%
\pgfpathlineto{\pgfqpoint{2.514329in}{0.822121in}}%
\pgfpathlineto{\pgfqpoint{2.527210in}{0.823101in}}%
\pgfpathlineto{\pgfqpoint{2.526856in}{0.826396in}}%
\pgfpathlineto{\pgfqpoint{2.545857in}{0.828293in}}%
\pgfpathlineto{\pgfqpoint{2.547789in}{0.832030in}}%
\pgfpathlineto{\pgfqpoint{2.579159in}{0.835118in}}%
\pgfpathlineto{\pgfqpoint{2.580736in}{0.815466in}}%
\pgfpathlineto{\pgfqpoint{2.572867in}{0.808389in}}%
\pgfpathlineto{\pgfqpoint{2.575447in}{0.802931in}}%
\pgfpathlineto{\pgfqpoint{2.577028in}{0.781586in}}%
\pgfpathlineto{\pgfqpoint{2.577747in}{0.775324in}}%
\pgfpathlineto{\pgfqpoint{2.558209in}{0.773227in}}%
\pgfpathlineto{\pgfqpoint{2.557495in}{0.779725in}}%
\pgfpathlineto{\pgfqpoint{2.550966in}{0.779031in}}%
\pgfpathlineto{\pgfqpoint{2.548906in}{0.772308in}}%
\pgfpathlineto{\pgfqpoint{2.551071in}{0.749076in}}%
\pgfpathlineto{\pgfqpoint{2.547816in}{0.748742in}}%
\pgfpathlineto{\pgfqpoint{2.548753in}{0.738972in}}%
\pgfpathclose%
\pgfusepath{fill}%
\end{pgfscope}%
\begin{pgfscope}%
\pgfpathrectangle{\pgfqpoint{0.100000in}{0.100000in}}{\pgfqpoint{3.420221in}{2.189500in}}%
\pgfusepath{clip}%
\pgfsetbuttcap%
\pgfsetmiterjoin%
\definecolor{currentfill}{rgb}{0.000000,0.192157,0.903922}%
\pgfsetfillcolor{currentfill}%
\pgfsetlinewidth{0.000000pt}%
\definecolor{currentstroke}{rgb}{0.000000,0.000000,0.000000}%
\pgfsetstrokecolor{currentstroke}%
\pgfsetstrokeopacity{0.000000}%
\pgfsetdash{}{0pt}%
\pgfpathmoveto{\pgfqpoint{0.629351in}{2.193256in}}%
\pgfpathlineto{\pgfqpoint{0.633664in}{2.185093in}}%
\pgfpathlineto{\pgfqpoint{0.628029in}{2.184767in}}%
\pgfpathlineto{\pgfqpoint{0.625117in}{2.180370in}}%
\pgfpathlineto{\pgfqpoint{0.629759in}{2.168518in}}%
\pgfpathlineto{\pgfqpoint{0.633996in}{2.162291in}}%
\pgfpathlineto{\pgfqpoint{0.633575in}{2.156368in}}%
\pgfpathlineto{\pgfqpoint{0.628593in}{2.154454in}}%
\pgfpathlineto{\pgfqpoint{0.627716in}{2.160792in}}%
\pgfpathlineto{\pgfqpoint{0.622707in}{2.165585in}}%
\pgfpathlineto{\pgfqpoint{0.624816in}{2.170964in}}%
\pgfpathlineto{\pgfqpoint{0.619295in}{2.182246in}}%
\pgfpathclose%
\pgfusepath{fill}%
\end{pgfscope}%
\begin{pgfscope}%
\pgfpathrectangle{\pgfqpoint{0.100000in}{0.100000in}}{\pgfqpoint{3.420221in}{2.189500in}}%
\pgfusepath{clip}%
\pgfsetbuttcap%
\pgfsetmiterjoin%
\definecolor{currentfill}{rgb}{0.000000,0.192157,0.903922}%
\pgfsetfillcolor{currentfill}%
\pgfsetlinewidth{0.000000pt}%
\definecolor{currentstroke}{rgb}{0.000000,0.000000,0.000000}%
\pgfsetstrokecolor{currentstroke}%
\pgfsetstrokeopacity{0.000000}%
\pgfsetdash{}{0pt}%
\pgfpathmoveto{\pgfqpoint{0.639458in}{2.030290in}}%
\pgfpathlineto{\pgfqpoint{0.603598in}{2.040987in}}%
\pgfpathlineto{\pgfqpoint{0.548087in}{2.058433in}}%
\pgfpathlineto{\pgfqpoint{0.556943in}{2.087997in}}%
\pgfpathlineto{\pgfqpoint{0.567327in}{2.084740in}}%
\pgfpathlineto{\pgfqpoint{0.572111in}{2.099317in}}%
\pgfpathlineto{\pgfqpoint{0.572015in}{2.106393in}}%
\pgfpathlineto{\pgfqpoint{0.557886in}{2.110776in}}%
\pgfpathlineto{\pgfqpoint{0.561151in}{2.123661in}}%
\pgfpathlineto{\pgfqpoint{0.567299in}{2.142345in}}%
\pgfpathlineto{\pgfqpoint{0.569389in}{2.148789in}}%
\pgfpathlineto{\pgfqpoint{0.596321in}{2.140085in}}%
\pgfpathlineto{\pgfqpoint{0.607311in}{2.142824in}}%
\pgfpathlineto{\pgfqpoint{0.611584in}{2.149681in}}%
\pgfpathlineto{\pgfqpoint{0.619623in}{2.154193in}}%
\pgfpathlineto{\pgfqpoint{0.619068in}{2.158988in}}%
\pgfpathlineto{\pgfqpoint{0.623951in}{2.155477in}}%
\pgfpathlineto{\pgfqpoint{0.622407in}{2.136019in}}%
\pgfpathlineto{\pgfqpoint{0.627316in}{2.140148in}}%
\pgfpathlineto{\pgfqpoint{0.628103in}{2.146368in}}%
\pgfpathlineto{\pgfqpoint{0.635486in}{2.155212in}}%
\pgfpathlineto{\pgfqpoint{0.640864in}{2.159386in}}%
\pgfpathlineto{\pgfqpoint{0.637064in}{2.166311in}}%
\pgfpathlineto{\pgfqpoint{0.638097in}{2.172731in}}%
\pgfpathlineto{\pgfqpoint{0.631327in}{2.180235in}}%
\pgfpathlineto{\pgfqpoint{0.639760in}{2.181745in}}%
\pgfpathlineto{\pgfqpoint{0.633034in}{2.194010in}}%
\pgfpathlineto{\pgfqpoint{0.644137in}{2.204452in}}%
\pgfpathlineto{\pgfqpoint{0.642178in}{2.208304in}}%
\pgfpathlineto{\pgfqpoint{0.695982in}{2.191609in}}%
\pgfpathlineto{\pgfqpoint{0.718101in}{2.185011in}}%
\pgfpathlineto{\pgfqpoint{0.726097in}{2.183989in}}%
\pgfpathlineto{\pgfqpoint{0.723634in}{2.182177in}}%
\pgfpathlineto{\pgfqpoint{0.725860in}{2.174090in}}%
\pgfpathlineto{\pgfqpoint{0.718709in}{2.177531in}}%
\pgfpathlineto{\pgfqpoint{0.706856in}{2.173979in}}%
\pgfpathlineto{\pgfqpoint{0.703739in}{2.164769in}}%
\pgfpathlineto{\pgfqpoint{0.708077in}{2.152301in}}%
\pgfpathlineto{\pgfqpoint{0.704909in}{2.147368in}}%
\pgfpathlineto{\pgfqpoint{0.694594in}{2.144668in}}%
\pgfpathlineto{\pgfqpoint{0.689070in}{2.134810in}}%
\pgfpathlineto{\pgfqpoint{0.690480in}{2.119171in}}%
\pgfpathlineto{\pgfqpoint{0.685861in}{2.111912in}}%
\pgfpathlineto{\pgfqpoint{0.680228in}{2.111398in}}%
\pgfpathlineto{\pgfqpoint{0.669728in}{2.105412in}}%
\pgfpathlineto{\pgfqpoint{0.663410in}{2.099346in}}%
\pgfpathlineto{\pgfqpoint{0.663638in}{2.094531in}}%
\pgfpathlineto{\pgfqpoint{0.667884in}{2.092315in}}%
\pgfpathlineto{\pgfqpoint{0.665990in}{2.081257in}}%
\pgfpathlineto{\pgfqpoint{0.650027in}{2.065403in}}%
\pgfpathlineto{\pgfqpoint{0.652269in}{2.058401in}}%
\pgfpathlineto{\pgfqpoint{0.651655in}{2.046186in}}%
\pgfpathlineto{\pgfqpoint{0.645679in}{2.037449in}}%
\pgfpathlineto{\pgfqpoint{0.645492in}{2.028467in}}%
\pgfpathclose%
\pgfusepath{fill}%
\end{pgfscope}%
\begin{pgfscope}%
\pgfpathrectangle{\pgfqpoint{0.100000in}{0.100000in}}{\pgfqpoint{3.420221in}{2.189500in}}%
\pgfusepath{clip}%
\pgfsetbuttcap%
\pgfsetmiterjoin%
\definecolor{currentfill}{rgb}{0.000000,0.423529,0.788235}%
\pgfsetfillcolor{currentfill}%
\pgfsetlinewidth{0.000000pt}%
\definecolor{currentstroke}{rgb}{0.000000,0.000000,0.000000}%
\pgfsetstrokecolor{currentstroke}%
\pgfsetstrokeopacity{0.000000}%
\pgfsetdash{}{0pt}%
\pgfpathmoveto{\pgfqpoint{2.484750in}{1.554511in}}%
\pgfpathlineto{\pgfqpoint{2.525050in}{1.558657in}}%
\pgfpathlineto{\pgfqpoint{2.522223in}{1.584549in}}%
\pgfpathlineto{\pgfqpoint{2.548030in}{1.587505in}}%
\pgfpathlineto{\pgfqpoint{2.551025in}{1.561460in}}%
\pgfpathlineto{\pgfqpoint{2.570463in}{1.563673in}}%
\pgfpathlineto{\pgfqpoint{2.573929in}{1.537514in}}%
\pgfpathlineto{\pgfqpoint{2.567450in}{1.536947in}}%
\pgfpathlineto{\pgfqpoint{2.515623in}{1.531150in}}%
\pgfpathlineto{\pgfqpoint{2.490110in}{1.528776in}}%
\pgfpathlineto{\pgfqpoint{2.488833in}{1.541607in}}%
\pgfpathlineto{\pgfqpoint{2.480973in}{1.540837in}}%
\pgfpathclose%
\pgfusepath{fill}%
\end{pgfscope}%
\begin{pgfscope}%
\pgfpathrectangle{\pgfqpoint{0.100000in}{0.100000in}}{\pgfqpoint{3.420221in}{2.189500in}}%
\pgfusepath{clip}%
\pgfsetbuttcap%
\pgfsetmiterjoin%
\definecolor{currentfill}{rgb}{0.000000,0.407843,0.796078}%
\pgfsetfillcolor{currentfill}%
\pgfsetlinewidth{0.000000pt}%
\definecolor{currentstroke}{rgb}{0.000000,0.000000,0.000000}%
\pgfsetstrokecolor{currentstroke}%
\pgfsetstrokeopacity{0.000000}%
\pgfsetdash{}{0pt}%
\pgfpathmoveto{\pgfqpoint{1.240781in}{1.915007in}}%
\pgfpathlineto{\pgfqpoint{1.259031in}{1.912046in}}%
\pgfpathlineto{\pgfqpoint{1.252676in}{1.876778in}}%
\pgfpathlineto{\pgfqpoint{1.239704in}{1.878862in}}%
\pgfpathlineto{\pgfqpoint{1.207370in}{1.884322in}}%
\pgfpathlineto{\pgfqpoint{1.200448in}{1.883307in}}%
\pgfpathlineto{\pgfqpoint{1.166816in}{1.889269in}}%
\pgfpathlineto{\pgfqpoint{1.165617in}{1.899000in}}%
\pgfpathlineto{\pgfqpoint{1.169833in}{1.904480in}}%
\pgfpathlineto{\pgfqpoint{1.165834in}{1.909249in}}%
\pgfpathlineto{\pgfqpoint{1.161477in}{1.908319in}}%
\pgfpathlineto{\pgfqpoint{1.158654in}{1.919655in}}%
\pgfpathlineto{\pgfqpoint{1.154959in}{1.926267in}}%
\pgfpathlineto{\pgfqpoint{1.152129in}{1.939198in}}%
\pgfpathlineto{\pgfqpoint{1.145784in}{1.943085in}}%
\pgfpathlineto{\pgfqpoint{1.146113in}{1.948385in}}%
\pgfpathlineto{\pgfqpoint{1.152149in}{1.947275in}}%
\pgfpathlineto{\pgfqpoint{1.153351in}{1.953743in}}%
\pgfpathlineto{\pgfqpoint{1.172823in}{1.951056in}}%
\pgfpathlineto{\pgfqpoint{1.184875in}{1.946552in}}%
\pgfpathlineto{\pgfqpoint{1.191532in}{1.938875in}}%
\pgfpathlineto{\pgfqpoint{1.199482in}{1.925029in}}%
\pgfpathlineto{\pgfqpoint{1.204201in}{1.919987in}}%
\pgfpathlineto{\pgfqpoint{1.230051in}{1.921291in}}%
\pgfpathlineto{\pgfqpoint{1.237015in}{1.915587in}}%
\pgfpathclose%
\pgfusepath{fill}%
\end{pgfscope}%
\begin{pgfscope}%
\pgfpathrectangle{\pgfqpoint{0.100000in}{0.100000in}}{\pgfqpoint{3.420221in}{2.189500in}}%
\pgfusepath{clip}%
\pgfsetbuttcap%
\pgfsetmiterjoin%
\definecolor{currentfill}{rgb}{0.000000,0.168627,0.915686}%
\pgfsetfillcolor{currentfill}%
\pgfsetlinewidth{0.000000pt}%
\definecolor{currentstroke}{rgb}{0.000000,0.000000,0.000000}%
\pgfsetstrokecolor{currentstroke}%
\pgfsetstrokeopacity{0.000000}%
\pgfsetdash{}{0pt}%
\pgfpathmoveto{\pgfqpoint{3.173989in}{1.521185in}}%
\pgfpathlineto{\pgfqpoint{3.176568in}{1.518622in}}%
\pgfpathlineto{\pgfqpoint{3.170324in}{1.504143in}}%
\pgfpathlineto{\pgfqpoint{3.159358in}{1.493926in}}%
\pgfpathlineto{\pgfqpoint{3.161529in}{1.491002in}}%
\pgfpathlineto{\pgfqpoint{3.145465in}{1.495992in}}%
\pgfpathlineto{\pgfqpoint{3.136068in}{1.505250in}}%
\pgfpathlineto{\pgfqpoint{3.130649in}{1.505537in}}%
\pgfpathlineto{\pgfqpoint{3.128081in}{1.514497in}}%
\pgfpathlineto{\pgfqpoint{3.119879in}{1.517769in}}%
\pgfpathlineto{\pgfqpoint{3.118536in}{1.530457in}}%
\pgfpathlineto{\pgfqpoint{3.121813in}{1.532210in}}%
\pgfpathlineto{\pgfqpoint{3.123503in}{1.538927in}}%
\pgfpathlineto{\pgfqpoint{3.118270in}{1.545029in}}%
\pgfpathlineto{\pgfqpoint{3.127933in}{1.563516in}}%
\pgfpathlineto{\pgfqpoint{3.129045in}{1.571941in}}%
\pgfpathlineto{\pgfqpoint{3.135304in}{1.578757in}}%
\pgfpathlineto{\pgfqpoint{3.164200in}{1.568896in}}%
\pgfpathlineto{\pgfqpoint{3.174987in}{1.585419in}}%
\pgfpathlineto{\pgfqpoint{3.176888in}{1.579951in}}%
\pgfpathlineto{\pgfqpoint{3.183350in}{1.573034in}}%
\pgfpathlineto{\pgfqpoint{3.185556in}{1.561494in}}%
\pgfpathlineto{\pgfqpoint{3.182432in}{1.535324in}}%
\pgfpathlineto{\pgfqpoint{3.175317in}{1.533250in}}%
\pgfpathclose%
\pgfusepath{fill}%
\end{pgfscope}%
\begin{pgfscope}%
\pgfpathrectangle{\pgfqpoint{0.100000in}{0.100000in}}{\pgfqpoint{3.420221in}{2.189500in}}%
\pgfusepath{clip}%
\pgfsetbuttcap%
\pgfsetmiterjoin%
\definecolor{currentfill}{rgb}{0.000000,0.623529,0.688235}%
\pgfsetfillcolor{currentfill}%
\pgfsetlinewidth{0.000000pt}%
\definecolor{currentstroke}{rgb}{0.000000,0.000000,0.000000}%
\pgfsetstrokecolor{currentstroke}%
\pgfsetstrokeopacity{0.000000}%
\pgfsetdash{}{0pt}%
\pgfpathmoveto{\pgfqpoint{2.747861in}{0.841676in}}%
\pgfpathlineto{\pgfqpoint{2.763599in}{0.853183in}}%
\pgfpathlineto{\pgfqpoint{2.762846in}{0.860171in}}%
\pgfpathlineto{\pgfqpoint{2.757692in}{0.863683in}}%
\pgfpathlineto{\pgfqpoint{2.753920in}{0.870009in}}%
\pgfpathlineto{\pgfqpoint{2.753524in}{0.876171in}}%
\pgfpathlineto{\pgfqpoint{2.758800in}{0.883019in}}%
\pgfpathlineto{\pgfqpoint{2.763576in}{0.885891in}}%
\pgfpathlineto{\pgfqpoint{2.765166in}{0.890527in}}%
\pgfpathlineto{\pgfqpoint{2.771349in}{0.894513in}}%
\pgfpathlineto{\pgfqpoint{2.772549in}{0.889816in}}%
\pgfpathlineto{\pgfqpoint{2.777926in}{0.887090in}}%
\pgfpathlineto{\pgfqpoint{2.785489in}{0.879401in}}%
\pgfpathlineto{\pgfqpoint{2.789476in}{0.872624in}}%
\pgfpathlineto{\pgfqpoint{2.789398in}{0.866209in}}%
\pgfpathlineto{\pgfqpoint{2.795897in}{0.861685in}}%
\pgfpathlineto{\pgfqpoint{2.792572in}{0.855434in}}%
\pgfpathlineto{\pgfqpoint{2.794958in}{0.852801in}}%
\pgfpathlineto{\pgfqpoint{2.787720in}{0.846628in}}%
\pgfpathlineto{\pgfqpoint{2.785056in}{0.841197in}}%
\pgfpathlineto{\pgfqpoint{2.781297in}{0.827579in}}%
\pgfpathlineto{\pgfqpoint{2.782427in}{0.825358in}}%
\pgfpathlineto{\pgfqpoint{2.774969in}{0.815604in}}%
\pgfpathlineto{\pgfqpoint{2.767042in}{0.810958in}}%
\pgfpathlineto{\pgfqpoint{2.754960in}{0.830347in}}%
\pgfpathclose%
\pgfusepath{fill}%
\end{pgfscope}%
\begin{pgfscope}%
\pgfpathrectangle{\pgfqpoint{0.100000in}{0.100000in}}{\pgfqpoint{3.420221in}{2.189500in}}%
\pgfusepath{clip}%
\pgfsetbuttcap%
\pgfsetmiterjoin%
\definecolor{currentfill}{rgb}{0.000000,0.368627,0.815686}%
\pgfsetfillcolor{currentfill}%
\pgfsetlinewidth{0.000000pt}%
\definecolor{currentstroke}{rgb}{0.000000,0.000000,0.000000}%
\pgfsetstrokecolor{currentstroke}%
\pgfsetstrokeopacity{0.000000}%
\pgfsetdash{}{0pt}%
\pgfpathmoveto{\pgfqpoint{2.389677in}{0.951522in}}%
\pgfpathlineto{\pgfqpoint{2.386476in}{0.950242in}}%
\pgfpathlineto{\pgfqpoint{2.386974in}{0.942656in}}%
\pgfpathlineto{\pgfqpoint{2.378437in}{0.940997in}}%
\pgfpathlineto{\pgfqpoint{2.358787in}{0.939731in}}%
\pgfpathlineto{\pgfqpoint{2.358247in}{0.948440in}}%
\pgfpathlineto{\pgfqpoint{2.354609in}{0.954784in}}%
\pgfpathlineto{\pgfqpoint{2.353322in}{0.974388in}}%
\pgfpathlineto{\pgfqpoint{2.350032in}{0.974184in}}%
\pgfpathlineto{\pgfqpoint{2.349336in}{0.985386in}}%
\pgfpathlineto{\pgfqpoint{2.358813in}{0.986031in}}%
\pgfpathlineto{\pgfqpoint{2.381832in}{0.987606in}}%
\pgfpathlineto{\pgfqpoint{2.382516in}{0.977368in}}%
\pgfpathlineto{\pgfqpoint{2.386851in}{0.977676in}}%
\pgfpathlineto{\pgfqpoint{2.389499in}{0.970179in}}%
\pgfpathclose%
\pgfusepath{fill}%
\end{pgfscope}%
\begin{pgfscope}%
\pgfpathrectangle{\pgfqpoint{0.100000in}{0.100000in}}{\pgfqpoint{3.420221in}{2.189500in}}%
\pgfusepath{clip}%
\pgfsetbuttcap%
\pgfsetmiterjoin%
\definecolor{currentfill}{rgb}{0.000000,0.568627,0.715686}%
\pgfsetfillcolor{currentfill}%
\pgfsetlinewidth{0.000000pt}%
\definecolor{currentstroke}{rgb}{0.000000,0.000000,0.000000}%
\pgfsetstrokecolor{currentstroke}%
\pgfsetstrokeopacity{0.000000}%
\pgfsetdash{}{0pt}%
\pgfpathmoveto{\pgfqpoint{2.465452in}{0.941858in}}%
\pgfpathlineto{\pgfqpoint{2.425499in}{0.939950in}}%
\pgfpathlineto{\pgfqpoint{2.425769in}{0.950686in}}%
\pgfpathlineto{\pgfqpoint{2.426932in}{0.983099in}}%
\pgfpathlineto{\pgfqpoint{2.419827in}{0.991287in}}%
\pgfpathlineto{\pgfqpoint{2.433212in}{0.992148in}}%
\pgfpathlineto{\pgfqpoint{2.431738in}{1.013857in}}%
\pgfpathlineto{\pgfqpoint{2.428545in}{1.020886in}}%
\pgfpathlineto{\pgfqpoint{2.431260in}{1.026207in}}%
\pgfpathlineto{\pgfqpoint{2.446738in}{1.029172in}}%
\pgfpathlineto{\pgfqpoint{2.451169in}{1.028731in}}%
\pgfpathlineto{\pgfqpoint{2.455922in}{1.023523in}}%
\pgfpathlineto{\pgfqpoint{2.470777in}{1.029126in}}%
\pgfpathlineto{\pgfqpoint{2.476897in}{1.027609in}}%
\pgfpathlineto{\pgfqpoint{2.478005in}{1.024083in}}%
\pgfpathlineto{\pgfqpoint{2.479546in}{0.999164in}}%
\pgfpathlineto{\pgfqpoint{2.480721in}{0.995506in}}%
\pgfpathlineto{\pgfqpoint{2.481643in}{0.981788in}}%
\pgfpathlineto{\pgfqpoint{2.476434in}{0.977234in}}%
\pgfpathlineto{\pgfqpoint{2.468740in}{0.979489in}}%
\pgfpathlineto{\pgfqpoint{2.469308in}{0.972666in}}%
\pgfpathlineto{\pgfqpoint{2.463838in}{0.961521in}}%
\pgfpathclose%
\pgfusepath{fill}%
\end{pgfscope}%
\begin{pgfscope}%
\pgfpathrectangle{\pgfqpoint{0.100000in}{0.100000in}}{\pgfqpoint{3.420221in}{2.189500in}}%
\pgfusepath{clip}%
\pgfsetbuttcap%
\pgfsetmiterjoin%
\definecolor{currentfill}{rgb}{0.000000,0.498039,0.750980}%
\pgfsetfillcolor{currentfill}%
\pgfsetlinewidth{0.000000pt}%
\definecolor{currentstroke}{rgb}{0.000000,0.000000,0.000000}%
\pgfsetstrokecolor{currentstroke}%
\pgfsetstrokeopacity{0.000000}%
\pgfsetdash{}{0pt}%
\pgfpathmoveto{\pgfqpoint{1.877454in}{0.451812in}}%
\pgfpathlineto{\pgfqpoint{1.879080in}{0.455758in}}%
\pgfpathlineto{\pgfqpoint{1.880147in}{0.460335in}}%
\pgfpathlineto{\pgfqpoint{1.885897in}{0.461530in}}%
\pgfpathlineto{\pgfqpoint{1.895549in}{0.469604in}}%
\pgfpathlineto{\pgfqpoint{1.897502in}{0.466982in}}%
\pgfpathlineto{\pgfqpoint{1.886411in}{0.459399in}}%
\pgfpathclose%
\pgfusepath{fill}%
\end{pgfscope}%
\begin{pgfscope}%
\pgfpathrectangle{\pgfqpoint{0.100000in}{0.100000in}}{\pgfqpoint{3.420221in}{2.189500in}}%
\pgfusepath{clip}%
\pgfsetbuttcap%
\pgfsetmiterjoin%
\definecolor{currentfill}{rgb}{0.000000,0.498039,0.750980}%
\pgfsetfillcolor{currentfill}%
\pgfsetlinewidth{0.000000pt}%
\definecolor{currentstroke}{rgb}{0.000000,0.000000,0.000000}%
\pgfsetstrokecolor{currentstroke}%
\pgfsetstrokeopacity{0.000000}%
\pgfsetdash{}{0pt}%
\pgfpathmoveto{\pgfqpoint{1.913740in}{0.495413in}}%
\pgfpathlineto{\pgfqpoint{1.903573in}{0.490321in}}%
\pgfpathlineto{\pgfqpoint{1.897610in}{0.495729in}}%
\pgfpathlineto{\pgfqpoint{1.896296in}{0.502002in}}%
\pgfpathlineto{\pgfqpoint{1.891006in}{0.499754in}}%
\pgfpathlineto{\pgfqpoint{1.895357in}{0.491264in}}%
\pgfpathlineto{\pgfqpoint{1.908612in}{0.480070in}}%
\pgfpathlineto{\pgfqpoint{1.896554in}{0.474005in}}%
\pgfpathlineto{\pgfqpoint{1.893361in}{0.471048in}}%
\pgfpathlineto{\pgfqpoint{1.882920in}{0.477565in}}%
\pgfpathlineto{\pgfqpoint{1.876238in}{0.485555in}}%
\pgfpathlineto{\pgfqpoint{1.869224in}{0.485387in}}%
\pgfpathlineto{\pgfqpoint{1.865089in}{0.489288in}}%
\pgfpathlineto{\pgfqpoint{1.857623in}{0.489736in}}%
\pgfpathlineto{\pgfqpoint{1.842914in}{0.477536in}}%
\pgfpathlineto{\pgfqpoint{1.831242in}{0.482620in}}%
\pgfpathlineto{\pgfqpoint{1.830925in}{0.487448in}}%
\pgfpathlineto{\pgfqpoint{1.821765in}{0.489803in}}%
\pgfpathlineto{\pgfqpoint{1.816568in}{0.499575in}}%
\pgfpathlineto{\pgfqpoint{1.830503in}{0.510228in}}%
\pgfpathlineto{\pgfqpoint{1.818925in}{0.525316in}}%
\pgfpathlineto{\pgfqpoint{1.828587in}{0.532838in}}%
\pgfpathlineto{\pgfqpoint{1.853866in}{0.553047in}}%
\pgfpathlineto{\pgfqpoint{1.856610in}{0.567877in}}%
\pgfpathlineto{\pgfqpoint{1.860710in}{0.571385in}}%
\pgfpathlineto{\pgfqpoint{1.878464in}{0.571343in}}%
\pgfpathlineto{\pgfqpoint{1.883739in}{0.567024in}}%
\pgfpathlineto{\pgfqpoint{1.898767in}{0.548381in}}%
\pgfpathlineto{\pgfqpoint{1.893366in}{0.541816in}}%
\pgfpathlineto{\pgfqpoint{1.915005in}{0.519834in}}%
\pgfpathclose%
\pgfusepath{fill}%
\end{pgfscope}%
\begin{pgfscope}%
\pgfpathrectangle{\pgfqpoint{0.100000in}{0.100000in}}{\pgfqpoint{3.420221in}{2.189500in}}%
\pgfusepath{clip}%
\pgfsetbuttcap%
\pgfsetmiterjoin%
\definecolor{currentfill}{rgb}{0.000000,0.678431,0.660784}%
\pgfsetfillcolor{currentfill}%
\pgfsetlinewidth{0.000000pt}%
\definecolor{currentstroke}{rgb}{0.000000,0.000000,0.000000}%
\pgfsetstrokecolor{currentstroke}%
\pgfsetstrokeopacity{0.000000}%
\pgfsetdash{}{0pt}%
\pgfpathmoveto{\pgfqpoint{2.244232in}{0.830051in}}%
\pgfpathlineto{\pgfqpoint{2.227203in}{0.829507in}}%
\pgfpathlineto{\pgfqpoint{2.225649in}{0.829433in}}%
\pgfpathlineto{\pgfqpoint{2.224716in}{0.858265in}}%
\pgfpathlineto{\pgfqpoint{2.191227in}{0.857753in}}%
\pgfpathlineto{\pgfqpoint{2.192005in}{0.850284in}}%
\pgfpathlineto{\pgfqpoint{2.183006in}{0.842900in}}%
\pgfpathlineto{\pgfqpoint{2.182528in}{0.839749in}}%
\pgfpathlineto{\pgfqpoint{2.172246in}{0.848314in}}%
\pgfpathlineto{\pgfqpoint{2.168337in}{0.848254in}}%
\pgfpathlineto{\pgfqpoint{2.167008in}{0.855383in}}%
\pgfpathlineto{\pgfqpoint{2.170843in}{0.864760in}}%
\pgfpathlineto{\pgfqpoint{2.168350in}{0.889971in}}%
\pgfpathlineto{\pgfqpoint{2.160747in}{0.900052in}}%
\pgfpathlineto{\pgfqpoint{2.159603in}{0.906383in}}%
\pgfpathlineto{\pgfqpoint{2.168009in}{0.906817in}}%
\pgfpathlineto{\pgfqpoint{2.174430in}{0.907210in}}%
\pgfpathlineto{\pgfqpoint{2.175317in}{0.919876in}}%
\pgfpathlineto{\pgfqpoint{2.175034in}{0.939603in}}%
\pgfpathlineto{\pgfqpoint{2.206170in}{0.939887in}}%
\pgfpathlineto{\pgfqpoint{2.206437in}{0.921187in}}%
\pgfpathlineto{\pgfqpoint{2.217089in}{0.920892in}}%
\pgfpathlineto{\pgfqpoint{2.218894in}{0.914810in}}%
\pgfpathlineto{\pgfqpoint{2.222834in}{0.914295in}}%
\pgfpathlineto{\pgfqpoint{2.224170in}{0.906301in}}%
\pgfpathlineto{\pgfqpoint{2.232275in}{0.901572in}}%
\pgfpathlineto{\pgfqpoint{2.243296in}{0.901753in}}%
\pgfpathlineto{\pgfqpoint{2.239012in}{0.908751in}}%
\pgfpathlineto{\pgfqpoint{2.243689in}{0.914018in}}%
\pgfpathlineto{\pgfqpoint{2.253810in}{0.914439in}}%
\pgfpathlineto{\pgfqpoint{2.259131in}{0.911482in}}%
\pgfpathlineto{\pgfqpoint{2.257596in}{0.907455in}}%
\pgfpathlineto{\pgfqpoint{2.246019in}{0.903296in}}%
\pgfpathlineto{\pgfqpoint{2.250986in}{0.899765in}}%
\pgfpathlineto{\pgfqpoint{2.247433in}{0.895180in}}%
\pgfpathlineto{\pgfqpoint{2.252172in}{0.888041in}}%
\pgfpathlineto{\pgfqpoint{2.243563in}{0.888479in}}%
\pgfpathlineto{\pgfqpoint{2.239025in}{0.879265in}}%
\pgfpathlineto{\pgfqpoint{2.244363in}{0.874547in}}%
\pgfpathlineto{\pgfqpoint{2.238807in}{0.872271in}}%
\pgfpathlineto{\pgfqpoint{2.244507in}{0.858382in}}%
\pgfpathlineto{\pgfqpoint{2.244845in}{0.852296in}}%
\pgfpathlineto{\pgfqpoint{2.248223in}{0.846557in}}%
\pgfpathlineto{\pgfqpoint{2.248526in}{0.840348in}}%
\pgfpathclose%
\pgfusepath{fill}%
\end{pgfscope}%
\begin{pgfscope}%
\pgfpathrectangle{\pgfqpoint{0.100000in}{0.100000in}}{\pgfqpoint{3.420221in}{2.189500in}}%
\pgfusepath{clip}%
\pgfsetbuttcap%
\pgfsetmiterjoin%
\definecolor{currentfill}{rgb}{0.000000,0.643137,0.678431}%
\pgfsetfillcolor{currentfill}%
\pgfsetlinewidth{0.000000pt}%
\definecolor{currentstroke}{rgb}{0.000000,0.000000,0.000000}%
\pgfsetstrokecolor{currentstroke}%
\pgfsetstrokeopacity{0.000000}%
\pgfsetdash{}{0pt}%
\pgfpathmoveto{\pgfqpoint{3.164020in}{1.774485in}}%
\pgfpathlineto{\pgfqpoint{3.129353in}{1.761278in}}%
\pgfpathlineto{\pgfqpoint{3.128989in}{1.765195in}}%
\pgfpathlineto{\pgfqpoint{3.119939in}{1.764197in}}%
\pgfpathlineto{\pgfqpoint{3.111492in}{1.771066in}}%
\pgfpathlineto{\pgfqpoint{3.114998in}{1.775320in}}%
\pgfpathlineto{\pgfqpoint{3.111072in}{1.786047in}}%
\pgfpathlineto{\pgfqpoint{3.098104in}{1.781414in}}%
\pgfpathlineto{\pgfqpoint{3.097403in}{1.786761in}}%
\pgfpathlineto{\pgfqpoint{3.078217in}{1.842756in}}%
\pgfpathlineto{\pgfqpoint{3.073128in}{1.844861in}}%
\pgfpathlineto{\pgfqpoint{3.103090in}{1.851488in}}%
\pgfpathlineto{\pgfqpoint{3.144567in}{1.862878in}}%
\pgfpathlineto{\pgfqpoint{3.146427in}{1.856239in}}%
\pgfpathlineto{\pgfqpoint{3.145755in}{1.849771in}}%
\pgfpathlineto{\pgfqpoint{3.148662in}{1.847850in}}%
\pgfpathlineto{\pgfqpoint{3.149091in}{1.833562in}}%
\pgfpathlineto{\pgfqpoint{3.155540in}{1.825986in}}%
\pgfpathlineto{\pgfqpoint{3.156415in}{1.816282in}}%
\pgfpathlineto{\pgfqpoint{3.159337in}{1.809401in}}%
\pgfpathlineto{\pgfqpoint{3.156649in}{1.802296in}}%
\pgfpathlineto{\pgfqpoint{3.156756in}{1.791032in}}%
\pgfpathclose%
\pgfusepath{fill}%
\end{pgfscope}%
\begin{pgfscope}%
\pgfpathrectangle{\pgfqpoint{0.100000in}{0.100000in}}{\pgfqpoint{3.420221in}{2.189500in}}%
\pgfusepath{clip}%
\pgfsetbuttcap%
\pgfsetmiterjoin%
\definecolor{currentfill}{rgb}{0.000000,0.396078,0.801961}%
\pgfsetfillcolor{currentfill}%
\pgfsetlinewidth{0.000000pt}%
\definecolor{currentstroke}{rgb}{0.000000,0.000000,0.000000}%
\pgfsetstrokecolor{currentstroke}%
\pgfsetstrokeopacity{0.000000}%
\pgfsetdash{}{0pt}%
\pgfpathmoveto{\pgfqpoint{1.596014in}{0.693701in}}%
\pgfpathlineto{\pgfqpoint{1.563625in}{0.696062in}}%
\pgfpathlineto{\pgfqpoint{1.569304in}{0.771922in}}%
\pgfpathlineto{\pgfqpoint{1.574471in}{0.771596in}}%
\pgfpathlineto{\pgfqpoint{1.577299in}{0.804493in}}%
\pgfpathlineto{\pgfqpoint{1.544712in}{0.806841in}}%
\pgfpathlineto{\pgfqpoint{1.546947in}{0.839603in}}%
\pgfpathlineto{\pgfqpoint{1.555288in}{0.839003in}}%
\pgfpathlineto{\pgfqpoint{1.612221in}{0.835139in}}%
\pgfpathlineto{\pgfqpoint{1.609897in}{0.802098in}}%
\pgfpathlineto{\pgfqpoint{1.607058in}{0.769259in}}%
\pgfpathlineto{\pgfqpoint{1.601942in}{0.769605in}}%
\pgfpathlineto{\pgfqpoint{1.598797in}{0.727554in}}%
\pgfpathclose%
\pgfusepath{fill}%
\end{pgfscope}%
\begin{pgfscope}%
\pgfpathrectangle{\pgfqpoint{0.100000in}{0.100000in}}{\pgfqpoint{3.420221in}{2.189500in}}%
\pgfusepath{clip}%
\pgfsetbuttcap%
\pgfsetmiterjoin%
\definecolor{currentfill}{rgb}{0.000000,0.768627,0.615686}%
\pgfsetfillcolor{currentfill}%
\pgfsetlinewidth{0.000000pt}%
\definecolor{currentstroke}{rgb}{0.000000,0.000000,0.000000}%
\pgfsetstrokecolor{currentstroke}%
\pgfsetstrokeopacity{0.000000}%
\pgfsetdash{}{0pt}%
\pgfpathmoveto{\pgfqpoint{0.743871in}{0.348716in}}%
\pgfpathlineto{\pgfqpoint{0.743625in}{0.351060in}}%
\pgfpathlineto{\pgfqpoint{0.746213in}{0.350627in}}%
\pgfpathlineto{\pgfqpoint{0.748546in}{0.346263in}}%
\pgfpathclose%
\pgfusepath{fill}%
\end{pgfscope}%
\begin{pgfscope}%
\pgfpathrectangle{\pgfqpoint{0.100000in}{0.100000in}}{\pgfqpoint{3.420221in}{2.189500in}}%
\pgfusepath{clip}%
\pgfsetbuttcap%
\pgfsetmiterjoin%
\definecolor{currentfill}{rgb}{0.000000,0.768627,0.615686}%
\pgfsetfillcolor{currentfill}%
\pgfsetlinewidth{0.000000pt}%
\definecolor{currentstroke}{rgb}{0.000000,0.000000,0.000000}%
\pgfsetstrokecolor{currentstroke}%
\pgfsetstrokeopacity{0.000000}%
\pgfsetdash{}{0pt}%
\pgfpathmoveto{\pgfqpoint{0.736358in}{0.351504in}}%
\pgfpathlineto{\pgfqpoint{0.738074in}{0.353223in}}%
\pgfpathlineto{\pgfqpoint{0.742067in}{0.353511in}}%
\pgfpathlineto{\pgfqpoint{0.741270in}{0.351825in}}%
\pgfpathclose%
\pgfusepath{fill}%
\end{pgfscope}%
\begin{pgfscope}%
\pgfpathrectangle{\pgfqpoint{0.100000in}{0.100000in}}{\pgfqpoint{3.420221in}{2.189500in}}%
\pgfusepath{clip}%
\pgfsetbuttcap%
\pgfsetmiterjoin%
\definecolor{currentfill}{rgb}{0.000000,0.768627,0.615686}%
\pgfsetfillcolor{currentfill}%
\pgfsetlinewidth{0.000000pt}%
\definecolor{currentstroke}{rgb}{0.000000,0.000000,0.000000}%
\pgfsetstrokecolor{currentstroke}%
\pgfsetstrokeopacity{0.000000}%
\pgfsetdash{}{0pt}%
\pgfpathmoveto{\pgfqpoint{0.729654in}{0.382439in}}%
\pgfpathlineto{\pgfqpoint{0.725882in}{0.385544in}}%
\pgfpathlineto{\pgfqpoint{0.727100in}{0.387065in}}%
\pgfpathlineto{\pgfqpoint{0.730314in}{0.387945in}}%
\pgfpathlineto{\pgfqpoint{0.732199in}{0.387456in}}%
\pgfpathlineto{\pgfqpoint{0.733358in}{0.388997in}}%
\pgfpathlineto{\pgfqpoint{0.736357in}{0.389624in}}%
\pgfpathlineto{\pgfqpoint{0.738133in}{0.388189in}}%
\pgfpathlineto{\pgfqpoint{0.742303in}{0.387463in}}%
\pgfpathlineto{\pgfqpoint{0.741792in}{0.386394in}}%
\pgfpathlineto{\pgfqpoint{0.737640in}{0.387163in}}%
\pgfpathlineto{\pgfqpoint{0.733002in}{0.386842in}}%
\pgfpathlineto{\pgfqpoint{0.732661in}{0.384092in}}%
\pgfpathclose%
\pgfusepath{fill}%
\end{pgfscope}%
\begin{pgfscope}%
\pgfpathrectangle{\pgfqpoint{0.100000in}{0.100000in}}{\pgfqpoint{3.420221in}{2.189500in}}%
\pgfusepath{clip}%
\pgfsetbuttcap%
\pgfsetmiterjoin%
\definecolor{currentfill}{rgb}{0.000000,0.768627,0.615686}%
\pgfsetfillcolor{currentfill}%
\pgfsetlinewidth{0.000000pt}%
\definecolor{currentstroke}{rgb}{0.000000,0.000000,0.000000}%
\pgfsetstrokecolor{currentstroke}%
\pgfsetstrokeopacity{0.000000}%
\pgfsetdash{}{0pt}%
\pgfpathmoveto{\pgfqpoint{0.794914in}{0.374951in}}%
\pgfpathlineto{\pgfqpoint{0.793540in}{0.372806in}}%
\pgfpathlineto{\pgfqpoint{0.795936in}{0.372449in}}%
\pgfpathlineto{\pgfqpoint{0.796306in}{0.370848in}}%
\pgfpathlineto{\pgfqpoint{0.798313in}{0.370324in}}%
\pgfpathlineto{\pgfqpoint{0.798809in}{0.367224in}}%
\pgfpathlineto{\pgfqpoint{0.797016in}{0.365495in}}%
\pgfpathlineto{\pgfqpoint{0.794739in}{0.365160in}}%
\pgfpathlineto{\pgfqpoint{0.789114in}{0.368409in}}%
\pgfpathlineto{\pgfqpoint{0.784394in}{0.367501in}}%
\pgfpathlineto{\pgfqpoint{0.780834in}{0.371048in}}%
\pgfpathlineto{\pgfqpoint{0.780377in}{0.373643in}}%
\pgfpathlineto{\pgfqpoint{0.782244in}{0.374954in}}%
\pgfpathlineto{\pgfqpoint{0.785598in}{0.374338in}}%
\pgfpathlineto{\pgfqpoint{0.786103in}{0.376325in}}%
\pgfpathlineto{\pgfqpoint{0.788649in}{0.372460in}}%
\pgfpathlineto{\pgfqpoint{0.790689in}{0.376724in}}%
\pgfpathlineto{\pgfqpoint{0.794385in}{0.376449in}}%
\pgfpathlineto{\pgfqpoint{0.796205in}{0.379022in}}%
\pgfpathlineto{\pgfqpoint{0.797578in}{0.377572in}}%
\pgfpathlineto{\pgfqpoint{0.797613in}{0.375876in}}%
\pgfpathclose%
\pgfusepath{fill}%
\end{pgfscope}%
\begin{pgfscope}%
\pgfpathrectangle{\pgfqpoint{0.100000in}{0.100000in}}{\pgfqpoint{3.420221in}{2.189500in}}%
\pgfusepath{clip}%
\pgfsetbuttcap%
\pgfsetmiterjoin%
\definecolor{currentfill}{rgb}{0.000000,0.768627,0.615686}%
\pgfsetfillcolor{currentfill}%
\pgfsetlinewidth{0.000000pt}%
\definecolor{currentstroke}{rgb}{0.000000,0.000000,0.000000}%
\pgfsetstrokecolor{currentstroke}%
\pgfsetstrokeopacity{0.000000}%
\pgfsetdash{}{0pt}%
\pgfpathmoveto{\pgfqpoint{0.786441in}{0.362878in}}%
\pgfpathlineto{\pgfqpoint{0.787067in}{0.359899in}}%
\pgfpathlineto{\pgfqpoint{0.785029in}{0.359922in}}%
\pgfpathlineto{\pgfqpoint{0.782181in}{0.356059in}}%
\pgfpathlineto{\pgfqpoint{0.785937in}{0.354070in}}%
\pgfpathlineto{\pgfqpoint{0.781231in}{0.351895in}}%
\pgfpathlineto{\pgfqpoint{0.779247in}{0.352710in}}%
\pgfpathlineto{\pgfqpoint{0.778032in}{0.355176in}}%
\pgfpathlineto{\pgfqpoint{0.776553in}{0.354392in}}%
\pgfpathlineto{\pgfqpoint{0.776230in}{0.352473in}}%
\pgfpathlineto{\pgfqpoint{0.774319in}{0.351903in}}%
\pgfpathlineto{\pgfqpoint{0.766957in}{0.354149in}}%
\pgfpathlineto{\pgfqpoint{0.767310in}{0.351942in}}%
\pgfpathlineto{\pgfqpoint{0.764878in}{0.351299in}}%
\pgfpathlineto{\pgfqpoint{0.761570in}{0.354176in}}%
\pgfpathlineto{\pgfqpoint{0.759768in}{0.354554in}}%
\pgfpathlineto{\pgfqpoint{0.755262in}{0.352049in}}%
\pgfpathlineto{\pgfqpoint{0.750715in}{0.351732in}}%
\pgfpathlineto{\pgfqpoint{0.751365in}{0.353951in}}%
\pgfpathlineto{\pgfqpoint{0.756368in}{0.355088in}}%
\pgfpathlineto{\pgfqpoint{0.756625in}{0.358328in}}%
\pgfpathlineto{\pgfqpoint{0.753764in}{0.358664in}}%
\pgfpathlineto{\pgfqpoint{0.750218in}{0.356800in}}%
\pgfpathlineto{\pgfqpoint{0.749257in}{0.362525in}}%
\pgfpathlineto{\pgfqpoint{0.751769in}{0.366938in}}%
\pgfpathlineto{\pgfqpoint{0.750571in}{0.370130in}}%
\pgfpathlineto{\pgfqpoint{0.756159in}{0.373552in}}%
\pgfpathlineto{\pgfqpoint{0.760557in}{0.373596in}}%
\pgfpathlineto{\pgfqpoint{0.763233in}{0.372821in}}%
\pgfpathlineto{\pgfqpoint{0.764976in}{0.370863in}}%
\pgfpathlineto{\pgfqpoint{0.762944in}{0.365806in}}%
\pgfpathlineto{\pgfqpoint{0.766337in}{0.367668in}}%
\pgfpathlineto{\pgfqpoint{0.767862in}{0.372947in}}%
\pgfpathlineto{\pgfqpoint{0.770149in}{0.374135in}}%
\pgfpathlineto{\pgfqpoint{0.772286in}{0.373761in}}%
\pgfpathlineto{\pgfqpoint{0.772576in}{0.370087in}}%
\pgfpathlineto{\pgfqpoint{0.774779in}{0.369583in}}%
\pgfpathlineto{\pgfqpoint{0.774939in}{0.372797in}}%
\pgfpathlineto{\pgfqpoint{0.779469in}{0.370602in}}%
\pgfpathlineto{\pgfqpoint{0.780735in}{0.368248in}}%
\pgfpathlineto{\pgfqpoint{0.782941in}{0.366025in}}%
\pgfpathlineto{\pgfqpoint{0.780808in}{0.364126in}}%
\pgfpathlineto{\pgfqpoint{0.786651in}{0.363748in}}%
\pgfpathclose%
\pgfusepath{fill}%
\end{pgfscope}%
\begin{pgfscope}%
\pgfpathrectangle{\pgfqpoint{0.100000in}{0.100000in}}{\pgfqpoint{3.420221in}{2.189500in}}%
\pgfusepath{clip}%
\pgfsetbuttcap%
\pgfsetmiterjoin%
\definecolor{currentfill}{rgb}{0.000000,0.768627,0.615686}%
\pgfsetfillcolor{currentfill}%
\pgfsetlinewidth{0.000000pt}%
\definecolor{currentstroke}{rgb}{0.000000,0.000000,0.000000}%
\pgfsetstrokecolor{currentstroke}%
\pgfsetstrokeopacity{0.000000}%
\pgfsetdash{}{0pt}%
\pgfpathmoveto{\pgfqpoint{0.793097in}{0.390258in}}%
\pgfpathlineto{\pgfqpoint{0.791393in}{0.390605in}}%
\pgfpathlineto{\pgfqpoint{0.789051in}{0.388882in}}%
\pgfpathlineto{\pgfqpoint{0.783943in}{0.388798in}}%
\pgfpathlineto{\pgfqpoint{0.782561in}{0.389841in}}%
\pgfpathlineto{\pgfqpoint{0.779715in}{0.388621in}}%
\pgfpathlineto{\pgfqpoint{0.777081in}{0.388323in}}%
\pgfpathlineto{\pgfqpoint{0.775799in}{0.386259in}}%
\pgfpathlineto{\pgfqpoint{0.774068in}{0.387437in}}%
\pgfpathlineto{\pgfqpoint{0.773381in}{0.385036in}}%
\pgfpathlineto{\pgfqpoint{0.766635in}{0.383810in}}%
\pgfpathlineto{\pgfqpoint{0.763862in}{0.384305in}}%
\pgfpathlineto{\pgfqpoint{0.761056in}{0.386333in}}%
\pgfpathlineto{\pgfqpoint{0.757811in}{0.387020in}}%
\pgfpathlineto{\pgfqpoint{0.757044in}{0.385590in}}%
\pgfpathlineto{\pgfqpoint{0.753020in}{0.385713in}}%
\pgfpathlineto{\pgfqpoint{0.750532in}{0.384592in}}%
\pgfpathlineto{\pgfqpoint{0.749927in}{0.387419in}}%
\pgfpathlineto{\pgfqpoint{0.747714in}{0.385692in}}%
\pgfpathlineto{\pgfqpoint{0.745185in}{0.386473in}}%
\pgfpathlineto{\pgfqpoint{0.744126in}{0.384769in}}%
\pgfpathlineto{\pgfqpoint{0.741947in}{0.386331in}}%
\pgfpathlineto{\pgfqpoint{0.745535in}{0.387287in}}%
\pgfpathlineto{\pgfqpoint{0.748505in}{0.389329in}}%
\pgfpathlineto{\pgfqpoint{0.750244in}{0.387954in}}%
\pgfpathlineto{\pgfqpoint{0.753756in}{0.387567in}}%
\pgfpathlineto{\pgfqpoint{0.758681in}{0.393872in}}%
\pgfpathlineto{\pgfqpoint{0.760646in}{0.393822in}}%
\pgfpathlineto{\pgfqpoint{0.762489in}{0.392379in}}%
\pgfpathlineto{\pgfqpoint{0.763959in}{0.393118in}}%
\pgfpathlineto{\pgfqpoint{0.767323in}{0.390986in}}%
\pgfpathlineto{\pgfqpoint{0.771254in}{0.389905in}}%
\pgfpathlineto{\pgfqpoint{0.772585in}{0.390859in}}%
\pgfpathlineto{\pgfqpoint{0.774725in}{0.390638in}}%
\pgfpathlineto{\pgfqpoint{0.777778in}{0.394690in}}%
\pgfpathlineto{\pgfqpoint{0.783322in}{0.390490in}}%
\pgfpathlineto{\pgfqpoint{0.784734in}{0.392394in}}%
\pgfpathlineto{\pgfqpoint{0.788860in}{0.389307in}}%
\pgfpathlineto{\pgfqpoint{0.790804in}{0.391968in}}%
\pgfpathclose%
\pgfusepath{fill}%
\end{pgfscope}%
\begin{pgfscope}%
\pgfpathrectangle{\pgfqpoint{0.100000in}{0.100000in}}{\pgfqpoint{3.420221in}{2.189500in}}%
\pgfusepath{clip}%
\pgfsetbuttcap%
\pgfsetmiterjoin%
\definecolor{currentfill}{rgb}{0.000000,0.380392,0.809804}%
\pgfsetfillcolor{currentfill}%
\pgfsetlinewidth{0.000000pt}%
\definecolor{currentstroke}{rgb}{0.000000,0.000000,0.000000}%
\pgfsetstrokecolor{currentstroke}%
\pgfsetstrokeopacity{0.000000}%
\pgfsetdash{}{0pt}%
\pgfpathmoveto{\pgfqpoint{1.146960in}{1.775893in}}%
\pgfpathlineto{\pgfqpoint{1.144753in}{1.761592in}}%
\pgfpathlineto{\pgfqpoint{1.140741in}{1.764287in}}%
\pgfpathlineto{\pgfqpoint{1.134748in}{1.780845in}}%
\pgfpathlineto{\pgfqpoint{1.131128in}{1.785288in}}%
\pgfpathlineto{\pgfqpoint{1.125542in}{1.783054in}}%
\pgfpathlineto{\pgfqpoint{1.109291in}{1.786033in}}%
\pgfpathlineto{\pgfqpoint{1.110285in}{1.791384in}}%
\pgfpathlineto{\pgfqpoint{1.095402in}{1.794389in}}%
\pgfpathlineto{\pgfqpoint{1.090696in}{1.798580in}}%
\pgfpathlineto{\pgfqpoint{1.091900in}{1.811670in}}%
\pgfpathlineto{\pgfqpoint{1.084411in}{1.813124in}}%
\pgfpathlineto{\pgfqpoint{1.085694in}{1.819623in}}%
\pgfpathlineto{\pgfqpoint{1.079267in}{1.820871in}}%
\pgfpathlineto{\pgfqpoint{1.082983in}{1.840215in}}%
\pgfpathlineto{\pgfqpoint{1.081849in}{1.847585in}}%
\pgfpathlineto{\pgfqpoint{1.077952in}{1.847349in}}%
\pgfpathlineto{\pgfqpoint{1.072884in}{1.852000in}}%
\pgfpathlineto{\pgfqpoint{1.075098in}{1.862908in}}%
\pgfpathlineto{\pgfqpoint{1.083024in}{1.865892in}}%
\pgfpathlineto{\pgfqpoint{1.085771in}{1.870248in}}%
\pgfpathlineto{\pgfqpoint{1.102559in}{1.867035in}}%
\pgfpathlineto{\pgfqpoint{1.110257in}{1.873573in}}%
\pgfpathlineto{\pgfqpoint{1.117614in}{1.872511in}}%
\pgfpathlineto{\pgfqpoint{1.120262in}{1.869223in}}%
\pgfpathlineto{\pgfqpoint{1.128740in}{1.866106in}}%
\pgfpathlineto{\pgfqpoint{1.135614in}{1.872312in}}%
\pgfpathlineto{\pgfqpoint{1.142509in}{1.875146in}}%
\pgfpathlineto{\pgfqpoint{1.146405in}{1.882816in}}%
\pgfpathlineto{\pgfqpoint{1.151109in}{1.886731in}}%
\pgfpathlineto{\pgfqpoint{1.152147in}{1.891545in}}%
\pgfpathlineto{\pgfqpoint{1.166816in}{1.889269in}}%
\pgfpathlineto{\pgfqpoint{1.200448in}{1.883307in}}%
\pgfpathlineto{\pgfqpoint{1.207370in}{1.884322in}}%
\pgfpathlineto{\pgfqpoint{1.239704in}{1.878862in}}%
\pgfpathlineto{\pgfqpoint{1.238656in}{1.872362in}}%
\pgfpathlineto{\pgfqpoint{1.240146in}{1.865478in}}%
\pgfpathlineto{\pgfqpoint{1.249844in}{1.863848in}}%
\pgfpathlineto{\pgfqpoint{1.248530in}{1.857616in}}%
\pgfpathlineto{\pgfqpoint{1.244193in}{1.858230in}}%
\pgfpathlineto{\pgfqpoint{1.239096in}{1.845591in}}%
\pgfpathlineto{\pgfqpoint{1.237118in}{1.832622in}}%
\pgfpathlineto{\pgfqpoint{1.230693in}{1.833649in}}%
\pgfpathlineto{\pgfqpoint{1.216783in}{1.829340in}}%
\pgfpathlineto{\pgfqpoint{1.214626in}{1.816397in}}%
\pgfpathlineto{\pgfqpoint{1.207777in}{1.817552in}}%
\pgfpathlineto{\pgfqpoint{1.205555in}{1.804425in}}%
\pgfpathlineto{\pgfqpoint{1.219475in}{1.802103in}}%
\pgfpathlineto{\pgfqpoint{1.217375in}{1.789489in}}%
\pgfpathlineto{\pgfqpoint{1.169639in}{1.796823in}}%
\pgfpathlineto{\pgfqpoint{1.151401in}{1.800731in}}%
\pgfpathclose%
\pgfusepath{fill}%
\end{pgfscope}%
\begin{pgfscope}%
\pgfpathrectangle{\pgfqpoint{0.100000in}{0.100000in}}{\pgfqpoint{3.420221in}{2.189500in}}%
\pgfusepath{clip}%
\pgfsetbuttcap%
\pgfsetmiterjoin%
\definecolor{currentfill}{rgb}{0.000000,0.247059,0.876471}%
\pgfsetfillcolor{currentfill}%
\pgfsetlinewidth{0.000000pt}%
\definecolor{currentstroke}{rgb}{0.000000,0.000000,0.000000}%
\pgfsetstrokecolor{currentstroke}%
\pgfsetstrokeopacity{0.000000}%
\pgfsetdash{}{0pt}%
\pgfpathmoveto{\pgfqpoint{1.814244in}{1.353473in}}%
\pgfpathlineto{\pgfqpoint{1.801027in}{1.353950in}}%
\pgfpathlineto{\pgfqpoint{1.788305in}{1.354451in}}%
\pgfpathlineto{\pgfqpoint{1.790722in}{1.406614in}}%
\pgfpathlineto{\pgfqpoint{1.841579in}{1.404755in}}%
\pgfpathlineto{\pgfqpoint{1.840861in}{1.378665in}}%
\pgfpathlineto{\pgfqpoint{1.815212in}{1.379526in}}%
\pgfpathclose%
\pgfusepath{fill}%
\end{pgfscope}%
\begin{pgfscope}%
\pgfpathrectangle{\pgfqpoint{0.100000in}{0.100000in}}{\pgfqpoint{3.420221in}{2.189500in}}%
\pgfusepath{clip}%
\pgfsetbuttcap%
\pgfsetmiterjoin%
\definecolor{currentfill}{rgb}{0.000000,0.572549,0.713725}%
\pgfsetfillcolor{currentfill}%
\pgfsetlinewidth{0.000000pt}%
\definecolor{currentstroke}{rgb}{0.000000,0.000000,0.000000}%
\pgfsetstrokecolor{currentstroke}%
\pgfsetstrokeopacity{0.000000}%
\pgfsetdash{}{0pt}%
\pgfpathmoveto{\pgfqpoint{0.664264in}{1.805137in}}%
\pgfpathlineto{\pgfqpoint{0.639058in}{1.812355in}}%
\pgfpathlineto{\pgfqpoint{0.589314in}{1.827339in}}%
\pgfpathlineto{\pgfqpoint{0.554402in}{1.837778in}}%
\pgfpathlineto{\pgfqpoint{0.559843in}{1.847793in}}%
\pgfpathlineto{\pgfqpoint{0.561131in}{1.854715in}}%
\pgfpathlineto{\pgfqpoint{0.569421in}{1.860413in}}%
\pgfpathlineto{\pgfqpoint{0.576768in}{1.868288in}}%
\pgfpathlineto{\pgfqpoint{0.579348in}{1.880020in}}%
\pgfpathlineto{\pgfqpoint{0.579024in}{1.889342in}}%
\pgfpathlineto{\pgfqpoint{0.581754in}{1.898328in}}%
\pgfpathlineto{\pgfqpoint{0.586720in}{1.904890in}}%
\pgfpathlineto{\pgfqpoint{0.589123in}{1.914813in}}%
\pgfpathlineto{\pgfqpoint{0.593489in}{1.921058in}}%
\pgfpathlineto{\pgfqpoint{0.627712in}{1.910636in}}%
\pgfpathlineto{\pgfqpoint{0.664497in}{1.899801in}}%
\pgfpathlineto{\pgfqpoint{0.662554in}{1.895639in}}%
\pgfpathlineto{\pgfqpoint{0.655798in}{1.872215in}}%
\pgfpathlineto{\pgfqpoint{0.674628in}{1.866933in}}%
\pgfpathlineto{\pgfqpoint{0.679982in}{1.861592in}}%
\pgfpathlineto{\pgfqpoint{0.678305in}{1.855586in}}%
\pgfpathlineto{\pgfqpoint{0.690922in}{1.852082in}}%
\pgfpathlineto{\pgfqpoint{0.683805in}{1.826899in}}%
\pgfpathlineto{\pgfqpoint{0.677650in}{1.828618in}}%
\pgfpathlineto{\pgfqpoint{0.672268in}{1.809748in}}%
\pgfpathlineto{\pgfqpoint{0.665960in}{1.811535in}}%
\pgfpathclose%
\pgfusepath{fill}%
\end{pgfscope}%
\begin{pgfscope}%
\pgfpathrectangle{\pgfqpoint{0.100000in}{0.100000in}}{\pgfqpoint{3.420221in}{2.189500in}}%
\pgfusepath{clip}%
\pgfsetbuttcap%
\pgfsetmiterjoin%
\definecolor{currentfill}{rgb}{0.000000,0.270588,0.864706}%
\pgfsetfillcolor{currentfill}%
\pgfsetlinewidth{0.000000pt}%
\definecolor{currentstroke}{rgb}{0.000000,0.000000,0.000000}%
\pgfsetstrokecolor{currentstroke}%
\pgfsetstrokeopacity{0.000000}%
\pgfsetdash{}{0pt}%
\pgfpathmoveto{\pgfqpoint{1.632008in}{1.330624in}}%
\pgfpathlineto{\pgfqpoint{1.595388in}{1.333331in}}%
\pgfpathlineto{\pgfqpoint{1.595419in}{1.333771in}}%
\pgfpathlineto{\pgfqpoint{1.599819in}{1.391683in}}%
\pgfpathlineto{\pgfqpoint{1.641298in}{1.388721in}}%
\pgfpathlineto{\pgfqpoint{1.639449in}{1.362727in}}%
\pgfpathlineto{\pgfqpoint{1.634466in}{1.363052in}}%
\pgfpathclose%
\pgfusepath{fill}%
\end{pgfscope}%
\begin{pgfscope}%
\pgfpathrectangle{\pgfqpoint{0.100000in}{0.100000in}}{\pgfqpoint{3.420221in}{2.189500in}}%
\pgfusepath{clip}%
\pgfsetbuttcap%
\pgfsetmiterjoin%
\definecolor{currentfill}{rgb}{0.000000,0.843137,0.578431}%
\pgfsetfillcolor{currentfill}%
\pgfsetlinewidth{0.000000pt}%
\definecolor{currentstroke}{rgb}{0.000000,0.000000,0.000000}%
\pgfsetstrokecolor{currentstroke}%
\pgfsetstrokeopacity{0.000000}%
\pgfsetdash{}{0pt}%
\pgfpathmoveto{\pgfqpoint{0.866788in}{1.014753in}}%
\pgfpathlineto{\pgfqpoint{0.871604in}{1.038223in}}%
\pgfpathlineto{\pgfqpoint{0.889779in}{1.127288in}}%
\pgfpathlineto{\pgfqpoint{0.893922in}{1.147506in}}%
\pgfpathlineto{\pgfqpoint{0.897283in}{1.158903in}}%
\pgfpathlineto{\pgfqpoint{0.896293in}{1.165197in}}%
\pgfpathlineto{\pgfqpoint{0.899610in}{1.169085in}}%
\pgfpathlineto{\pgfqpoint{0.904734in}{1.167363in}}%
\pgfpathlineto{\pgfqpoint{0.907012in}{1.171373in}}%
\pgfpathlineto{\pgfqpoint{0.916713in}{1.173988in}}%
\pgfpathlineto{\pgfqpoint{0.932897in}{1.176449in}}%
\pgfpathlineto{\pgfqpoint{0.940664in}{1.179335in}}%
\pgfpathlineto{\pgfqpoint{0.944986in}{1.183466in}}%
\pgfpathlineto{\pgfqpoint{0.944684in}{1.191769in}}%
\pgfpathlineto{\pgfqpoint{0.948035in}{1.203324in}}%
\pgfpathlineto{\pgfqpoint{0.958514in}{1.226188in}}%
\pgfpathlineto{\pgfqpoint{0.937424in}{1.230332in}}%
\pgfpathlineto{\pgfqpoint{0.945283in}{1.270267in}}%
\pgfpathlineto{\pgfqpoint{0.957985in}{1.267733in}}%
\pgfpathlineto{\pgfqpoint{0.976668in}{1.263573in}}%
\pgfpathlineto{\pgfqpoint{1.052426in}{1.249751in}}%
\pgfpathlineto{\pgfqpoint{1.076953in}{1.245506in}}%
\pgfpathlineto{\pgfqpoint{1.068804in}{1.231965in}}%
\pgfpathlineto{\pgfqpoint{1.060752in}{1.229263in}}%
\pgfpathlineto{\pgfqpoint{1.057265in}{1.221112in}}%
\pgfpathlineto{\pgfqpoint{1.047066in}{1.217159in}}%
\pgfpathlineto{\pgfqpoint{1.040007in}{1.218670in}}%
\pgfpathlineto{\pgfqpoint{1.036119in}{1.214492in}}%
\pgfpathlineto{\pgfqpoint{1.027530in}{1.213272in}}%
\pgfpathlineto{\pgfqpoint{1.063895in}{1.206860in}}%
\pgfpathlineto{\pgfqpoint{1.028431in}{1.004289in}}%
\pgfpathlineto{\pgfqpoint{1.022533in}{1.005502in}}%
\pgfpathlineto{\pgfqpoint{1.013615in}{1.014403in}}%
\pgfpathlineto{\pgfqpoint{1.005688in}{1.017601in}}%
\pgfpathlineto{\pgfqpoint{0.999559in}{1.023134in}}%
\pgfpathlineto{\pgfqpoint{0.990916in}{1.023161in}}%
\pgfpathlineto{\pgfqpoint{0.982827in}{1.028356in}}%
\pgfpathlineto{\pgfqpoint{0.974552in}{1.023403in}}%
\pgfpathlineto{\pgfqpoint{0.968103in}{1.006621in}}%
\pgfpathlineto{\pgfqpoint{0.982440in}{1.003961in}}%
\pgfpathlineto{\pgfqpoint{0.979350in}{0.992961in}}%
\pgfpathlineto{\pgfqpoint{0.965217in}{0.995576in}}%
\pgfpathlineto{\pgfqpoint{0.938991in}{1.004219in}}%
\pgfpathlineto{\pgfqpoint{0.929810in}{0.993281in}}%
\pgfpathlineto{\pgfqpoint{0.902848in}{1.007551in}}%
\pgfpathclose%
\pgfusepath{fill}%
\end{pgfscope}%
\begin{pgfscope}%
\pgfpathrectangle{\pgfqpoint{0.100000in}{0.100000in}}{\pgfqpoint{3.420221in}{2.189500in}}%
\pgfusepath{clip}%
\pgfsetbuttcap%
\pgfsetmiterjoin%
\definecolor{currentfill}{rgb}{0.000000,0.062745,0.968627}%
\pgfsetfillcolor{currentfill}%
\pgfsetlinewidth{0.000000pt}%
\definecolor{currentstroke}{rgb}{0.000000,0.000000,0.000000}%
\pgfsetstrokecolor{currentstroke}%
\pgfsetstrokeopacity{0.000000}%
\pgfsetdash{}{0pt}%
\pgfpathmoveto{\pgfqpoint{1.500023in}{1.341478in}}%
\pgfpathlineto{\pgfqpoint{1.494279in}{1.289386in}}%
\pgfpathlineto{\pgfqpoint{1.474804in}{1.291448in}}%
\pgfpathlineto{\pgfqpoint{1.476962in}{1.310767in}}%
\pgfpathlineto{\pgfqpoint{1.403074in}{1.318781in}}%
\pgfpathlineto{\pgfqpoint{1.399475in}{1.286581in}}%
\pgfpathlineto{\pgfqpoint{1.362244in}{1.290688in}}%
\pgfpathlineto{\pgfqpoint{1.366927in}{1.298680in}}%
\pgfpathlineto{\pgfqpoint{1.366438in}{1.303660in}}%
\pgfpathlineto{\pgfqpoint{1.361115in}{1.309640in}}%
\pgfpathlineto{\pgfqpoint{1.352268in}{1.316407in}}%
\pgfpathlineto{\pgfqpoint{1.352738in}{1.319215in}}%
\pgfpathlineto{\pgfqpoint{1.354540in}{1.325234in}}%
\pgfpathlineto{\pgfqpoint{1.355473in}{1.337996in}}%
\pgfpathlineto{\pgfqpoint{1.369774in}{1.344170in}}%
\pgfpathlineto{\pgfqpoint{1.381376in}{1.359045in}}%
\pgfpathlineto{\pgfqpoint{1.373896in}{1.365152in}}%
\pgfpathlineto{\pgfqpoint{1.380485in}{1.370897in}}%
\pgfpathlineto{\pgfqpoint{1.388650in}{1.374935in}}%
\pgfpathlineto{\pgfqpoint{1.390311in}{1.385501in}}%
\pgfpathlineto{\pgfqpoint{1.393229in}{1.388487in}}%
\pgfpathlineto{\pgfqpoint{1.394344in}{1.405081in}}%
\pgfpathlineto{\pgfqpoint{1.428340in}{1.401313in}}%
\pgfpathlineto{\pgfqpoint{1.426155in}{1.381804in}}%
\pgfpathlineto{\pgfqpoint{1.477819in}{1.376381in}}%
\pgfpathlineto{\pgfqpoint{1.503220in}{1.373918in}}%
\pgfpathclose%
\pgfusepath{fill}%
\end{pgfscope}%
\begin{pgfscope}%
\pgfpathrectangle{\pgfqpoint{0.100000in}{0.100000in}}{\pgfqpoint{3.420221in}{2.189500in}}%
\pgfusepath{clip}%
\pgfsetbuttcap%
\pgfsetmiterjoin%
\definecolor{currentfill}{rgb}{0.000000,0.666667,0.666667}%
\pgfsetfillcolor{currentfill}%
\pgfsetlinewidth{0.000000pt}%
\definecolor{currentstroke}{rgb}{0.000000,0.000000,0.000000}%
\pgfsetstrokecolor{currentstroke}%
\pgfsetstrokeopacity{0.000000}%
\pgfsetdash{}{0pt}%
\pgfpathmoveto{\pgfqpoint{1.709016in}{0.536062in}}%
\pgfpathlineto{\pgfqpoint{1.662723in}{0.538108in}}%
\pgfpathlineto{\pgfqpoint{1.664970in}{0.578836in}}%
\pgfpathlineto{\pgfqpoint{1.626255in}{0.581190in}}%
\pgfpathlineto{\pgfqpoint{1.629383in}{0.631481in}}%
\pgfpathlineto{\pgfqpoint{1.612387in}{0.632536in}}%
\pgfpathlineto{\pgfqpoint{1.614467in}{0.664147in}}%
\pgfpathlineto{\pgfqpoint{1.669335in}{0.661125in}}%
\pgfpathlineto{\pgfqpoint{1.667531in}{0.629362in}}%
\pgfpathlineto{\pgfqpoint{1.691200in}{0.628101in}}%
\pgfpathlineto{\pgfqpoint{1.690112in}{0.611725in}}%
\pgfpathlineto{\pgfqpoint{1.694552in}{0.611485in}}%
\pgfpathlineto{\pgfqpoint{1.693811in}{0.598966in}}%
\pgfpathlineto{\pgfqpoint{1.699618in}{0.598582in}}%
\pgfpathlineto{\pgfqpoint{1.698521in}{0.577338in}}%
\pgfpathlineto{\pgfqpoint{1.711129in}{0.576738in}}%
\pgfpathclose%
\pgfusepath{fill}%
\end{pgfscope}%
\begin{pgfscope}%
\pgfpathrectangle{\pgfqpoint{0.100000in}{0.100000in}}{\pgfqpoint{3.420221in}{2.189500in}}%
\pgfusepath{clip}%
\pgfsetbuttcap%
\pgfsetmiterjoin%
\definecolor{currentfill}{rgb}{0.000000,0.764706,0.617647}%
\pgfsetfillcolor{currentfill}%
\pgfsetlinewidth{0.000000pt}%
\definecolor{currentstroke}{rgb}{0.000000,0.000000,0.000000}%
\pgfsetstrokecolor{currentstroke}%
\pgfsetstrokeopacity{0.000000}%
\pgfsetdash{}{0pt}%
\pgfpathmoveto{\pgfqpoint{1.111527in}{0.850369in}}%
\pgfpathlineto{\pgfqpoint{1.120630in}{0.908635in}}%
\pgfpathlineto{\pgfqpoint{1.185762in}{0.898306in}}%
\pgfpathlineto{\pgfqpoint{1.193569in}{0.897266in}}%
\pgfpathlineto{\pgfqpoint{1.194839in}{0.887306in}}%
\pgfpathlineto{\pgfqpoint{1.193456in}{0.879904in}}%
\pgfpathlineto{\pgfqpoint{1.197373in}{0.876820in}}%
\pgfpathlineto{\pgfqpoint{1.196265in}{0.870442in}}%
\pgfpathlineto{\pgfqpoint{1.198710in}{0.864291in}}%
\pgfpathlineto{\pgfqpoint{1.196181in}{0.853300in}}%
\pgfpathlineto{\pgfqpoint{1.204024in}{0.850348in}}%
\pgfpathlineto{\pgfqpoint{1.223452in}{0.847680in}}%
\pgfpathlineto{\pgfqpoint{1.215257in}{0.786241in}}%
\pgfpathlineto{\pgfqpoint{1.157319in}{0.794395in}}%
\pgfpathlineto{\pgfqpoint{1.152382in}{0.760743in}}%
\pgfpathlineto{\pgfqpoint{1.098657in}{0.768794in}}%
\pgfpathclose%
\pgfusepath{fill}%
\end{pgfscope}%
\begin{pgfscope}%
\pgfpathrectangle{\pgfqpoint{0.100000in}{0.100000in}}{\pgfqpoint{3.420221in}{2.189500in}}%
\pgfusepath{clip}%
\pgfsetbuttcap%
\pgfsetmiterjoin%
\definecolor{currentfill}{rgb}{0.000000,0.254902,0.872549}%
\pgfsetfillcolor{currentfill}%
\pgfsetlinewidth{0.000000pt}%
\definecolor{currentstroke}{rgb}{0.000000,0.000000,0.000000}%
\pgfsetstrokecolor{currentstroke}%
\pgfsetstrokeopacity{0.000000}%
\pgfsetdash{}{0pt}%
\pgfpathmoveto{\pgfqpoint{1.885162in}{1.797233in}}%
\pgfpathlineto{\pgfqpoint{1.845789in}{1.798416in}}%
\pgfpathlineto{\pgfqpoint{1.791943in}{1.800793in}}%
\pgfpathlineto{\pgfqpoint{1.793212in}{1.826664in}}%
\pgfpathlineto{\pgfqpoint{1.843740in}{1.824676in}}%
\pgfpathlineto{\pgfqpoint{1.844653in}{1.850966in}}%
\pgfpathlineto{\pgfqpoint{1.862904in}{1.850337in}}%
\pgfpathlineto{\pgfqpoint{1.883765in}{1.849712in}}%
\pgfpathlineto{\pgfqpoint{1.883078in}{1.823500in}}%
\pgfpathlineto{\pgfqpoint{1.883613in}{1.801948in}}%
\pgfpathclose%
\pgfusepath{fill}%
\end{pgfscope}%
\begin{pgfscope}%
\pgfpathrectangle{\pgfqpoint{0.100000in}{0.100000in}}{\pgfqpoint{3.420221in}{2.189500in}}%
\pgfusepath{clip}%
\pgfsetbuttcap%
\pgfsetmiterjoin%
\definecolor{currentfill}{rgb}{0.000000,0.411765,0.794118}%
\pgfsetfillcolor{currentfill}%
\pgfsetlinewidth{0.000000pt}%
\definecolor{currentstroke}{rgb}{0.000000,0.000000,0.000000}%
\pgfsetstrokecolor{currentstroke}%
\pgfsetstrokeopacity{0.000000}%
\pgfsetdash{}{0pt}%
\pgfpathmoveto{\pgfqpoint{3.200881in}{1.837697in}}%
\pgfpathlineto{\pgfqpoint{3.202734in}{1.844896in}}%
\pgfpathlineto{\pgfqpoint{3.196024in}{1.846393in}}%
\pgfpathlineto{\pgfqpoint{3.197752in}{1.853740in}}%
\pgfpathlineto{\pgfqpoint{3.188331in}{1.855917in}}%
\pgfpathlineto{\pgfqpoint{3.189732in}{1.860303in}}%
\pgfpathlineto{\pgfqpoint{3.185390in}{1.872783in}}%
\pgfpathlineto{\pgfqpoint{3.198040in}{1.875649in}}%
\pgfpathlineto{\pgfqpoint{3.239531in}{1.887164in}}%
\pgfpathlineto{\pgfqpoint{3.242160in}{1.906168in}}%
\pgfpathlineto{\pgfqpoint{3.254058in}{1.909133in}}%
\pgfpathlineto{\pgfqpoint{3.255324in}{1.914147in}}%
\pgfpathlineto{\pgfqpoint{3.267982in}{1.876426in}}%
\pgfpathlineto{\pgfqpoint{3.279179in}{1.840986in}}%
\pgfpathlineto{\pgfqpoint{3.278411in}{1.837120in}}%
\pgfpathlineto{\pgfqpoint{3.267361in}{1.834652in}}%
\pgfpathlineto{\pgfqpoint{3.266644in}{1.826128in}}%
\pgfpathlineto{\pgfqpoint{3.259472in}{1.830046in}}%
\pgfpathlineto{\pgfqpoint{3.257321in}{1.832372in}}%
\pgfpathlineto{\pgfqpoint{3.250318in}{1.830941in}}%
\pgfpathlineto{\pgfqpoint{3.248766in}{1.834429in}}%
\pgfpathlineto{\pgfqpoint{3.242536in}{1.834934in}}%
\pgfpathlineto{\pgfqpoint{3.237577in}{1.839415in}}%
\pgfpathlineto{\pgfqpoint{3.233333in}{1.832866in}}%
\pgfpathlineto{\pgfqpoint{3.227465in}{1.831313in}}%
\pgfpathlineto{\pgfqpoint{3.224309in}{1.825639in}}%
\pgfpathlineto{\pgfqpoint{3.225967in}{1.819545in}}%
\pgfpathlineto{\pgfqpoint{3.213614in}{1.815862in}}%
\pgfpathlineto{\pgfqpoint{3.209914in}{1.816543in}}%
\pgfpathlineto{\pgfqpoint{3.213344in}{1.834548in}}%
\pgfpathclose%
\pgfusepath{fill}%
\end{pgfscope}%
\begin{pgfscope}%
\pgfpathrectangle{\pgfqpoint{0.100000in}{0.100000in}}{\pgfqpoint{3.420221in}{2.189500in}}%
\pgfusepath{clip}%
\pgfsetbuttcap%
\pgfsetmiterjoin%
\definecolor{currentfill}{rgb}{0.000000,0.105882,0.947059}%
\pgfsetfillcolor{currentfill}%
\pgfsetlinewidth{0.000000pt}%
\definecolor{currentstroke}{rgb}{0.000000,0.000000,0.000000}%
\pgfsetstrokecolor{currentstroke}%
\pgfsetstrokeopacity{0.000000}%
\pgfsetdash{}{0pt}%
\pgfpathmoveto{\pgfqpoint{1.818204in}{1.496781in}}%
\pgfpathlineto{\pgfqpoint{1.819375in}{1.535927in}}%
\pgfpathlineto{\pgfqpoint{1.871003in}{1.534471in}}%
\pgfpathlineto{\pgfqpoint{1.870822in}{1.527937in}}%
\pgfpathlineto{\pgfqpoint{1.870278in}{1.508356in}}%
\pgfpathlineto{\pgfqpoint{1.889699in}{1.507861in}}%
\pgfpathlineto{\pgfqpoint{1.889090in}{1.481804in}}%
\pgfpathlineto{\pgfqpoint{1.843731in}{1.482953in}}%
\pgfpathlineto{\pgfqpoint{1.844000in}{1.496017in}}%
\pgfpathclose%
\pgfusepath{fill}%
\end{pgfscope}%
\begin{pgfscope}%
\pgfpathrectangle{\pgfqpoint{0.100000in}{0.100000in}}{\pgfqpoint{3.420221in}{2.189500in}}%
\pgfusepath{clip}%
\pgfsetbuttcap%
\pgfsetmiterjoin%
\definecolor{currentfill}{rgb}{0.000000,0.305882,0.847059}%
\pgfsetfillcolor{currentfill}%
\pgfsetlinewidth{0.000000pt}%
\definecolor{currentstroke}{rgb}{0.000000,0.000000,0.000000}%
\pgfsetstrokecolor{currentstroke}%
\pgfsetstrokeopacity{0.000000}%
\pgfsetdash{}{0pt}%
\pgfpathmoveto{\pgfqpoint{1.503220in}{1.373918in}}%
\pgfpathlineto{\pgfqpoint{1.477819in}{1.376381in}}%
\pgfpathlineto{\pgfqpoint{1.426155in}{1.381804in}}%
\pgfpathlineto{\pgfqpoint{1.428340in}{1.401313in}}%
\pgfpathlineto{\pgfqpoint{1.394344in}{1.405081in}}%
\pgfpathlineto{\pgfqpoint{1.388729in}{1.413239in}}%
\pgfpathlineto{\pgfqpoint{1.384913in}{1.423241in}}%
\pgfpathlineto{\pgfqpoint{1.376450in}{1.449245in}}%
\pgfpathlineto{\pgfqpoint{1.370341in}{1.458821in}}%
\pgfpathlineto{\pgfqpoint{1.370632in}{1.463589in}}%
\pgfpathlineto{\pgfqpoint{1.421950in}{1.457544in}}%
\pgfpathlineto{\pgfqpoint{1.458619in}{1.453758in}}%
\pgfpathlineto{\pgfqpoint{1.517776in}{1.447837in}}%
\pgfpathlineto{\pgfqpoint{1.513984in}{1.412184in}}%
\pgfpathlineto{\pgfqpoint{1.520475in}{1.411561in}}%
\pgfpathlineto{\pgfqpoint{1.516574in}{1.372652in}}%
\pgfpathclose%
\pgfusepath{fill}%
\end{pgfscope}%
\begin{pgfscope}%
\pgfpathrectangle{\pgfqpoint{0.100000in}{0.100000in}}{\pgfqpoint{3.420221in}{2.189500in}}%
\pgfusepath{clip}%
\pgfsetbuttcap%
\pgfsetmiterjoin%
\definecolor{currentfill}{rgb}{0.000000,0.619608,0.690196}%
\pgfsetfillcolor{currentfill}%
\pgfsetlinewidth{0.000000pt}%
\definecolor{currentstroke}{rgb}{0.000000,0.000000,0.000000}%
\pgfsetstrokecolor{currentstroke}%
\pgfsetstrokeopacity{0.000000}%
\pgfsetdash{}{0pt}%
\pgfpathmoveto{\pgfqpoint{0.829088in}{1.947246in}}%
\pgfpathlineto{\pgfqpoint{0.834963in}{1.972213in}}%
\pgfpathlineto{\pgfqpoint{0.831748in}{1.972963in}}%
\pgfpathlineto{\pgfqpoint{0.831107in}{1.983555in}}%
\pgfpathlineto{\pgfqpoint{0.824750in}{1.985151in}}%
\pgfpathlineto{\pgfqpoint{0.828218in}{1.996264in}}%
\pgfpathlineto{\pgfqpoint{0.837956in}{1.999890in}}%
\pgfpathlineto{\pgfqpoint{0.841715in}{1.996554in}}%
\pgfpathlineto{\pgfqpoint{0.849461in}{1.996056in}}%
\pgfpathlineto{\pgfqpoint{0.851556in}{1.989763in}}%
\pgfpathlineto{\pgfqpoint{0.856857in}{1.982769in}}%
\pgfpathlineto{\pgfqpoint{0.857332in}{1.972934in}}%
\pgfpathlineto{\pgfqpoint{0.865475in}{1.971333in}}%
\pgfpathlineto{\pgfqpoint{0.867594in}{1.979867in}}%
\pgfpathlineto{\pgfqpoint{0.883858in}{1.975951in}}%
\pgfpathlineto{\pgfqpoint{0.890276in}{1.981136in}}%
\pgfpathlineto{\pgfqpoint{0.904991in}{1.977468in}}%
\pgfpathlineto{\pgfqpoint{0.910368in}{2.000022in}}%
\pgfpathlineto{\pgfqpoint{0.946515in}{1.991426in}}%
\pgfpathlineto{\pgfqpoint{0.979308in}{1.983961in}}%
\pgfpathlineto{\pgfqpoint{0.979536in}{1.975757in}}%
\pgfpathlineto{\pgfqpoint{0.992726in}{1.963526in}}%
\pgfpathlineto{\pgfqpoint{0.991947in}{1.958416in}}%
\pgfpathlineto{\pgfqpoint{1.005194in}{1.957799in}}%
\pgfpathlineto{\pgfqpoint{1.006951in}{1.955930in}}%
\pgfpathlineto{\pgfqpoint{0.999885in}{1.939013in}}%
\pgfpathlineto{\pgfqpoint{0.993643in}{1.929733in}}%
\pgfpathlineto{\pgfqpoint{0.992907in}{1.921757in}}%
\pgfpathlineto{\pgfqpoint{0.988643in}{1.913763in}}%
\pgfpathlineto{\pgfqpoint{0.991478in}{1.900635in}}%
\pgfpathlineto{\pgfqpoint{0.984507in}{1.898684in}}%
\pgfpathlineto{\pgfqpoint{0.980431in}{1.894039in}}%
\pgfpathlineto{\pgfqpoint{0.982168in}{1.885566in}}%
\pgfpathlineto{\pgfqpoint{0.977017in}{1.878098in}}%
\pgfpathlineto{\pgfqpoint{0.970527in}{1.872842in}}%
\pgfpathlineto{\pgfqpoint{0.963651in}{1.874783in}}%
\pgfpathlineto{\pgfqpoint{0.970766in}{1.859237in}}%
\pgfpathlineto{\pgfqpoint{0.964504in}{1.853012in}}%
\pgfpathlineto{\pgfqpoint{0.898770in}{1.867828in}}%
\pgfpathlineto{\pgfqpoint{0.887868in}{1.863610in}}%
\pgfpathlineto{\pgfqpoint{0.880987in}{1.866921in}}%
\pgfpathlineto{\pgfqpoint{0.880422in}{1.877811in}}%
\pgfpathlineto{\pgfqpoint{0.862672in}{1.882079in}}%
\pgfpathlineto{\pgfqpoint{0.873537in}{1.897775in}}%
\pgfpathlineto{\pgfqpoint{0.880335in}{1.904794in}}%
\pgfpathlineto{\pgfqpoint{0.878729in}{1.914372in}}%
\pgfpathlineto{\pgfqpoint{0.871998in}{1.922673in}}%
\pgfpathlineto{\pgfqpoint{0.867725in}{1.924123in}}%
\pgfpathlineto{\pgfqpoint{0.863981in}{1.938253in}}%
\pgfpathclose%
\pgfusepath{fill}%
\end{pgfscope}%
\begin{pgfscope}%
\pgfpathrectangle{\pgfqpoint{0.100000in}{0.100000in}}{\pgfqpoint{3.420221in}{2.189500in}}%
\pgfusepath{clip}%
\pgfsetbuttcap%
\pgfsetmiterjoin%
\definecolor{currentfill}{rgb}{0.000000,0.674510,0.662745}%
\pgfsetfillcolor{currentfill}%
\pgfsetlinewidth{0.000000pt}%
\definecolor{currentstroke}{rgb}{0.000000,0.000000,0.000000}%
\pgfsetstrokecolor{currentstroke}%
\pgfsetstrokeopacity{0.000000}%
\pgfsetdash{}{0pt}%
\pgfpathmoveto{\pgfqpoint{2.821673in}{0.664921in}}%
\pgfpathlineto{\pgfqpoint{2.848148in}{0.669171in}}%
\pgfpathlineto{\pgfqpoint{2.852731in}{0.637305in}}%
\pgfpathlineto{\pgfqpoint{2.854959in}{0.618679in}}%
\pgfpathlineto{\pgfqpoint{2.845245in}{0.614077in}}%
\pgfpathlineto{\pgfqpoint{2.844620in}{0.618162in}}%
\pgfpathlineto{\pgfqpoint{2.831860in}{0.616441in}}%
\pgfpathlineto{\pgfqpoint{2.834914in}{0.596202in}}%
\pgfpathlineto{\pgfqpoint{2.826200in}{0.594925in}}%
\pgfpathlineto{\pgfqpoint{2.827989in}{0.582136in}}%
\pgfpathlineto{\pgfqpoint{2.823282in}{0.578757in}}%
\pgfpathlineto{\pgfqpoint{2.816463in}{0.579447in}}%
\pgfpathlineto{\pgfqpoint{2.813898in}{0.576352in}}%
\pgfpathlineto{\pgfqpoint{2.813078in}{0.581007in}}%
\pgfpathlineto{\pgfqpoint{2.808078in}{0.588241in}}%
\pgfpathlineto{\pgfqpoint{2.794627in}{0.586993in}}%
\pgfpathlineto{\pgfqpoint{2.789823in}{0.593822in}}%
\pgfpathlineto{\pgfqpoint{2.782411in}{0.599279in}}%
\pgfpathlineto{\pgfqpoint{2.779382in}{0.604364in}}%
\pgfpathlineto{\pgfqpoint{2.773961in}{0.605520in}}%
\pgfpathlineto{\pgfqpoint{2.766845in}{0.610114in}}%
\pgfpathlineto{\pgfqpoint{2.764031in}{0.621193in}}%
\pgfpathlineto{\pgfqpoint{2.766774in}{0.621574in}}%
\pgfpathlineto{\pgfqpoint{2.768815in}{0.633824in}}%
\pgfpathlineto{\pgfqpoint{2.796564in}{0.637847in}}%
\pgfpathlineto{\pgfqpoint{2.797672in}{0.644415in}}%
\pgfpathlineto{\pgfqpoint{2.802590in}{0.646729in}}%
\pgfpathlineto{\pgfqpoint{2.806718in}{0.641452in}}%
\pgfpathlineto{\pgfqpoint{2.812968in}{0.640082in}}%
\pgfpathlineto{\pgfqpoint{2.815772in}{0.643097in}}%
\pgfpathlineto{\pgfqpoint{2.816146in}{0.651817in}}%
\pgfpathclose%
\pgfusepath{fill}%
\end{pgfscope}%
\begin{pgfscope}%
\pgfpathrectangle{\pgfqpoint{0.100000in}{0.100000in}}{\pgfqpoint{3.420221in}{2.189500in}}%
\pgfusepath{clip}%
\pgfsetbuttcap%
\pgfsetmiterjoin%
\definecolor{currentfill}{rgb}{0.000000,0.274510,0.862745}%
\pgfsetfillcolor{currentfill}%
\pgfsetlinewidth{0.000000pt}%
\definecolor{currentstroke}{rgb}{0.000000,0.000000,0.000000}%
\pgfsetstrokecolor{currentstroke}%
\pgfsetstrokeopacity{0.000000}%
\pgfsetdash{}{0pt}%
\pgfpathmoveto{\pgfqpoint{2.675925in}{1.322230in}}%
\pgfpathlineto{\pgfqpoint{2.682927in}{1.320166in}}%
\pgfpathlineto{\pgfqpoint{2.686944in}{1.290644in}}%
\pgfpathlineto{\pgfqpoint{2.671440in}{1.295060in}}%
\pgfpathlineto{\pgfqpoint{2.666174in}{1.292986in}}%
\pgfpathlineto{\pgfqpoint{2.663718in}{1.288412in}}%
\pgfpathlineto{\pgfqpoint{2.657473in}{1.289695in}}%
\pgfpathlineto{\pgfqpoint{2.650464in}{1.296921in}}%
\pgfpathlineto{\pgfqpoint{2.644114in}{1.298006in}}%
\pgfpathlineto{\pgfqpoint{2.637696in}{1.296257in}}%
\pgfpathlineto{\pgfqpoint{2.628882in}{1.298824in}}%
\pgfpathlineto{\pgfqpoint{2.632795in}{1.281175in}}%
\pgfpathlineto{\pgfqpoint{2.621640in}{1.278242in}}%
\pgfpathlineto{\pgfqpoint{2.611910in}{1.270491in}}%
\pgfpathlineto{\pgfqpoint{2.599389in}{1.280039in}}%
\pgfpathlineto{\pgfqpoint{2.597872in}{1.287524in}}%
\pgfpathlineto{\pgfqpoint{2.589723in}{1.282241in}}%
\pgfpathlineto{\pgfqpoint{2.583686in}{1.289084in}}%
\pgfpathlineto{\pgfqpoint{2.591336in}{1.292416in}}%
\pgfpathlineto{\pgfqpoint{2.596571in}{1.299433in}}%
\pgfpathlineto{\pgfqpoint{2.590871in}{1.301278in}}%
\pgfpathlineto{\pgfqpoint{2.592613in}{1.307196in}}%
\pgfpathlineto{\pgfqpoint{2.588644in}{1.311542in}}%
\pgfpathlineto{\pgfqpoint{2.592652in}{1.315991in}}%
\pgfpathlineto{\pgfqpoint{2.589440in}{1.347028in}}%
\pgfpathlineto{\pgfqpoint{2.589083in}{1.350403in}}%
\pgfpathlineto{\pgfqpoint{2.608353in}{1.352724in}}%
\pgfpathlineto{\pgfqpoint{2.619275in}{1.355542in}}%
\pgfpathlineto{\pgfqpoint{2.637130in}{1.356179in}}%
\pgfpathlineto{\pgfqpoint{2.638259in}{1.332657in}}%
\pgfpathlineto{\pgfqpoint{2.646458in}{1.333031in}}%
\pgfpathlineto{\pgfqpoint{2.648150in}{1.316177in}}%
\pgfpathlineto{\pgfqpoint{2.663225in}{1.317872in}}%
\pgfpathclose%
\pgfusepath{fill}%
\end{pgfscope}%
\begin{pgfscope}%
\pgfpathrectangle{\pgfqpoint{0.100000in}{0.100000in}}{\pgfqpoint{3.420221in}{2.189500in}}%
\pgfusepath{clip}%
\pgfsetbuttcap%
\pgfsetmiterjoin%
\definecolor{currentfill}{rgb}{0.000000,0.678431,0.660784}%
\pgfsetfillcolor{currentfill}%
\pgfsetlinewidth{0.000000pt}%
\definecolor{currentstroke}{rgb}{0.000000,0.000000,0.000000}%
\pgfsetstrokecolor{currentstroke}%
\pgfsetstrokeopacity{0.000000}%
\pgfsetdash{}{0pt}%
\pgfpathmoveto{\pgfqpoint{2.760459in}{1.363673in}}%
\pgfpathlineto{\pgfqpoint{2.748261in}{1.362666in}}%
\pgfpathlineto{\pgfqpoint{2.747799in}{1.369459in}}%
\pgfpathlineto{\pgfqpoint{2.741341in}{1.368968in}}%
\pgfpathlineto{\pgfqpoint{2.728297in}{1.370188in}}%
\pgfpathlineto{\pgfqpoint{2.727575in}{1.381116in}}%
\pgfpathlineto{\pgfqpoint{2.724921in}{1.387583in}}%
\pgfpathlineto{\pgfqpoint{2.720525in}{1.387297in}}%
\pgfpathlineto{\pgfqpoint{2.720118in}{1.394288in}}%
\pgfpathlineto{\pgfqpoint{2.733318in}{1.394842in}}%
\pgfpathlineto{\pgfqpoint{2.734936in}{1.397863in}}%
\pgfpathlineto{\pgfqpoint{2.733296in}{1.414033in}}%
\pgfpathlineto{\pgfqpoint{2.762739in}{1.417091in}}%
\pgfpathlineto{\pgfqpoint{2.762144in}{1.422418in}}%
\pgfpathlineto{\pgfqpoint{2.780881in}{1.424599in}}%
\pgfpathlineto{\pgfqpoint{2.787724in}{1.422292in}}%
\pgfpathlineto{\pgfqpoint{2.790497in}{1.399806in}}%
\pgfpathlineto{\pgfqpoint{2.786142in}{1.399268in}}%
\pgfpathlineto{\pgfqpoint{2.787649in}{1.387212in}}%
\pgfpathlineto{\pgfqpoint{2.790847in}{1.380001in}}%
\pgfpathlineto{\pgfqpoint{2.780231in}{1.376695in}}%
\pgfpathlineto{\pgfqpoint{2.767850in}{1.375398in}}%
\pgfpathlineto{\pgfqpoint{2.767926in}{1.367099in}}%
\pgfpathlineto{\pgfqpoint{2.761224in}{1.367123in}}%
\pgfpathclose%
\pgfusepath{fill}%
\end{pgfscope}%
\begin{pgfscope}%
\pgfpathrectangle{\pgfqpoint{0.100000in}{0.100000in}}{\pgfqpoint{3.420221in}{2.189500in}}%
\pgfusepath{clip}%
\pgfsetbuttcap%
\pgfsetmiterjoin%
\definecolor{currentfill}{rgb}{0.000000,0.278431,0.860784}%
\pgfsetfillcolor{currentfill}%
\pgfsetlinewidth{0.000000pt}%
\definecolor{currentstroke}{rgb}{0.000000,0.000000,0.000000}%
\pgfsetstrokecolor{currentstroke}%
\pgfsetstrokeopacity{0.000000}%
\pgfsetdash{}{0pt}%
\pgfpathmoveto{\pgfqpoint{1.924742in}{1.724212in}}%
\pgfpathlineto{\pgfqpoint{1.901819in}{1.724583in}}%
\pgfpathlineto{\pgfqpoint{1.902028in}{1.737645in}}%
\pgfpathlineto{\pgfqpoint{1.883810in}{1.738175in}}%
\pgfpathlineto{\pgfqpoint{1.884078in}{1.749170in}}%
\pgfpathlineto{\pgfqpoint{1.895683in}{1.748835in}}%
\pgfpathlineto{\pgfqpoint{1.896507in}{1.751080in}}%
\pgfpathlineto{\pgfqpoint{1.924188in}{1.750527in}}%
\pgfpathlineto{\pgfqpoint{1.921532in}{1.754272in}}%
\pgfpathlineto{\pgfqpoint{1.913179in}{1.757047in}}%
\pgfpathlineto{\pgfqpoint{1.904100in}{1.771966in}}%
\pgfpathlineto{\pgfqpoint{1.905171in}{1.775078in}}%
\pgfpathlineto{\pgfqpoint{1.915135in}{1.782379in}}%
\pgfpathlineto{\pgfqpoint{1.918871in}{1.787838in}}%
\pgfpathlineto{\pgfqpoint{1.920061in}{1.796513in}}%
\pgfpathlineto{\pgfqpoint{1.919474in}{1.803061in}}%
\pgfpathlineto{\pgfqpoint{1.935770in}{1.802787in}}%
\pgfpathlineto{\pgfqpoint{1.936310in}{1.796229in}}%
\pgfpathlineto{\pgfqpoint{1.935975in}{1.769873in}}%
\pgfpathlineto{\pgfqpoint{1.936420in}{1.756837in}}%
\pgfpathlineto{\pgfqpoint{1.943001in}{1.756699in}}%
\pgfpathlineto{\pgfqpoint{1.942783in}{1.743728in}}%
\pgfpathlineto{\pgfqpoint{1.939838in}{1.740212in}}%
\pgfpathlineto{\pgfqpoint{1.934000in}{1.744389in}}%
\pgfpathlineto{\pgfqpoint{1.925048in}{1.746180in}}%
\pgfpathclose%
\pgfusepath{fill}%
\end{pgfscope}%
\begin{pgfscope}%
\pgfpathrectangle{\pgfqpoint{0.100000in}{0.100000in}}{\pgfqpoint{3.420221in}{2.189500in}}%
\pgfusepath{clip}%
\pgfsetbuttcap%
\pgfsetmiterjoin%
\definecolor{currentfill}{rgb}{0.000000,0.439216,0.780392}%
\pgfsetfillcolor{currentfill}%
\pgfsetlinewidth{0.000000pt}%
\definecolor{currentstroke}{rgb}{0.000000,0.000000,0.000000}%
\pgfsetstrokecolor{currentstroke}%
\pgfsetstrokeopacity{0.000000}%
\pgfsetdash{}{0pt}%
\pgfpathmoveto{\pgfqpoint{2.527732in}{1.473996in}}%
\pgfpathlineto{\pgfqpoint{2.526546in}{1.484582in}}%
\pgfpathlineto{\pgfqpoint{2.503884in}{1.482216in}}%
\pgfpathlineto{\pgfqpoint{2.501285in}{1.506433in}}%
\pgfpathlineto{\pgfqpoint{2.516398in}{1.507830in}}%
\pgfpathlineto{\pgfqpoint{2.517831in}{1.510992in}}%
\pgfpathlineto{\pgfqpoint{2.515623in}{1.531150in}}%
\pgfpathlineto{\pgfqpoint{2.567450in}{1.536947in}}%
\pgfpathlineto{\pgfqpoint{2.570134in}{1.513654in}}%
\pgfpathlineto{\pgfqpoint{2.549482in}{1.511355in}}%
\pgfpathlineto{\pgfqpoint{2.553736in}{1.474448in}}%
\pgfpathlineto{\pgfqpoint{2.534384in}{1.472530in}}%
\pgfpathclose%
\pgfusepath{fill}%
\end{pgfscope}%
\begin{pgfscope}%
\pgfpathrectangle{\pgfqpoint{0.100000in}{0.100000in}}{\pgfqpoint{3.420221in}{2.189500in}}%
\pgfusepath{clip}%
\pgfsetbuttcap%
\pgfsetmiterjoin%
\definecolor{currentfill}{rgb}{0.000000,0.101961,0.949020}%
\pgfsetfillcolor{currentfill}%
\pgfsetlinewidth{0.000000pt}%
\definecolor{currentstroke}{rgb}{0.000000,0.000000,0.000000}%
\pgfsetstrokecolor{currentstroke}%
\pgfsetstrokeopacity{0.000000}%
\pgfsetdash{}{0pt}%
\pgfpathmoveto{\pgfqpoint{1.384913in}{1.423241in}}%
\pgfpathlineto{\pgfqpoint{1.388729in}{1.413239in}}%
\pgfpathlineto{\pgfqpoint{1.394344in}{1.405081in}}%
\pgfpathlineto{\pgfqpoint{1.393229in}{1.388487in}}%
\pgfpathlineto{\pgfqpoint{1.390311in}{1.385501in}}%
\pgfpathlineto{\pgfqpoint{1.388650in}{1.374935in}}%
\pgfpathlineto{\pgfqpoint{1.380485in}{1.370897in}}%
\pgfpathlineto{\pgfqpoint{1.373896in}{1.365152in}}%
\pgfpathlineto{\pgfqpoint{1.367969in}{1.365158in}}%
\pgfpathlineto{\pgfqpoint{1.365751in}{1.374270in}}%
\pgfpathlineto{\pgfqpoint{1.357433in}{1.383303in}}%
\pgfpathlineto{\pgfqpoint{1.335903in}{1.386939in}}%
\pgfpathlineto{\pgfqpoint{1.336047in}{1.392762in}}%
\pgfpathlineto{\pgfqpoint{1.339565in}{1.417954in}}%
\pgfpathlineto{\pgfqpoint{1.339434in}{1.425804in}}%
\pgfpathlineto{\pgfqpoint{1.342606in}{1.420639in}}%
\pgfpathlineto{\pgfqpoint{1.357591in}{1.414807in}}%
\pgfpathlineto{\pgfqpoint{1.366839in}{1.413793in}}%
\pgfpathlineto{\pgfqpoint{1.374628in}{1.415959in}}%
\pgfpathlineto{\pgfqpoint{1.377138in}{1.413905in}}%
\pgfpathclose%
\pgfusepath{fill}%
\end{pgfscope}%
\begin{pgfscope}%
\pgfpathrectangle{\pgfqpoint{0.100000in}{0.100000in}}{\pgfqpoint{3.420221in}{2.189500in}}%
\pgfusepath{clip}%
\pgfsetbuttcap%
\pgfsetmiterjoin%
\definecolor{currentfill}{rgb}{0.000000,0.215686,0.892157}%
\pgfsetfillcolor{currentfill}%
\pgfsetlinewidth{0.000000pt}%
\definecolor{currentstroke}{rgb}{0.000000,0.000000,0.000000}%
\pgfsetstrokecolor{currentstroke}%
\pgfsetstrokeopacity{0.000000}%
\pgfsetdash{}{0pt}%
\pgfpathmoveto{\pgfqpoint{2.506445in}{1.302745in}}%
\pgfpathlineto{\pgfqpoint{2.503441in}{1.302426in}}%
\pgfpathlineto{\pgfqpoint{2.500717in}{1.324094in}}%
\pgfpathlineto{\pgfqpoint{2.486320in}{1.322820in}}%
\pgfpathlineto{\pgfqpoint{2.485353in}{1.332481in}}%
\pgfpathlineto{\pgfqpoint{2.482202in}{1.332178in}}%
\pgfpathlineto{\pgfqpoint{2.483536in}{1.338454in}}%
\pgfpathlineto{\pgfqpoint{2.481074in}{1.344123in}}%
\pgfpathlineto{\pgfqpoint{2.476844in}{1.384992in}}%
\pgfpathlineto{\pgfqpoint{2.502677in}{1.387583in}}%
\pgfpathlineto{\pgfqpoint{2.502400in}{1.390197in}}%
\pgfpathlineto{\pgfqpoint{2.524058in}{1.392572in}}%
\pgfpathlineto{\pgfqpoint{2.526098in}{1.372042in}}%
\pgfpathlineto{\pgfqpoint{2.542430in}{1.373883in}}%
\pgfpathlineto{\pgfqpoint{2.542491in}{1.361890in}}%
\pgfpathlineto{\pgfqpoint{2.559393in}{1.363745in}}%
\pgfpathlineto{\pgfqpoint{2.561678in}{1.344236in}}%
\pgfpathlineto{\pgfqpoint{2.562288in}{1.338860in}}%
\pgfpathlineto{\pgfqpoint{2.543185in}{1.336838in}}%
\pgfpathlineto{\pgfqpoint{2.540808in}{1.328822in}}%
\pgfpathlineto{\pgfqpoint{2.517744in}{1.326075in}}%
\pgfpathlineto{\pgfqpoint{2.520208in}{1.304192in}}%
\pgfpathclose%
\pgfusepath{fill}%
\end{pgfscope}%
\begin{pgfscope}%
\pgfpathrectangle{\pgfqpoint{0.100000in}{0.100000in}}{\pgfqpoint{3.420221in}{2.189500in}}%
\pgfusepath{clip}%
\pgfsetbuttcap%
\pgfsetmiterjoin%
\definecolor{currentfill}{rgb}{0.000000,0.949020,0.525490}%
\pgfsetfillcolor{currentfill}%
\pgfsetlinewidth{0.000000pt}%
\definecolor{currentstroke}{rgb}{0.000000,0.000000,0.000000}%
\pgfsetstrokecolor{currentstroke}%
\pgfsetstrokeopacity{0.000000}%
\pgfsetdash{}{0pt}%
\pgfpathmoveto{\pgfqpoint{0.857403in}{0.968918in}}%
\pgfpathlineto{\pgfqpoint{0.844233in}{0.904518in}}%
\pgfpathlineto{\pgfqpoint{0.837200in}{0.870088in}}%
\pgfpathlineto{\pgfqpoint{0.752037in}{0.923335in}}%
\pgfpathlineto{\pgfqpoint{0.754507in}{0.932517in}}%
\pgfpathlineto{\pgfqpoint{0.761516in}{0.938572in}}%
\pgfpathlineto{\pgfqpoint{0.674070in}{0.950804in}}%
\pgfpathlineto{\pgfqpoint{0.682160in}{0.984206in}}%
\pgfpathlineto{\pgfqpoint{0.683524in}{0.983897in}}%
\pgfpathlineto{\pgfqpoint{0.689353in}{1.009685in}}%
\pgfpathlineto{\pgfqpoint{0.751952in}{0.995434in}}%
\pgfpathlineto{\pgfqpoint{0.772267in}{0.989060in}}%
\pgfpathlineto{\pgfqpoint{0.772920in}{0.978857in}}%
\pgfpathlineto{\pgfqpoint{0.768616in}{0.966832in}}%
\pgfpathlineto{\pgfqpoint{0.770439in}{0.960916in}}%
\pgfpathlineto{\pgfqpoint{0.794383in}{0.955399in}}%
\pgfpathlineto{\pgfqpoint{0.801314in}{0.987316in}}%
\pgfpathlineto{\pgfqpoint{0.820352in}{0.983335in}}%
\pgfpathlineto{\pgfqpoint{0.818995in}{0.976922in}}%
\pgfpathclose%
\pgfusepath{fill}%
\end{pgfscope}%
\begin{pgfscope}%
\pgfpathrectangle{\pgfqpoint{0.100000in}{0.100000in}}{\pgfqpoint{3.420221in}{2.189500in}}%
\pgfusepath{clip}%
\pgfsetbuttcap%
\pgfsetroundjoin%
\pgfsetlinewidth{0.050187pt}%
\definecolor{currentstroke}{rgb}{1.000000,1.000000,1.000000}%
\pgfsetstrokecolor{currentstroke}%
\pgfsetdash{}{0pt}%
\pgfusepath{stroke}%
\end{pgfscope}%
\begin{pgfscope}%
\pgfpathrectangle{\pgfqpoint{0.100000in}{0.100000in}}{\pgfqpoint{3.420221in}{2.189500in}}%
\pgfusepath{clip}%
\pgfsetbuttcap%
\pgfsetroundjoin%
\pgfsetlinewidth{0.050187pt}%
\definecolor{currentstroke}{rgb}{1.000000,1.000000,1.000000}%
\pgfsetstrokecolor{currentstroke}%
\pgfsetdash{}{0pt}%
\pgfusepath{stroke}%
\end{pgfscope}%
\begin{pgfscope}%
\pgfpathrectangle{\pgfqpoint{0.100000in}{0.100000in}}{\pgfqpoint{3.420221in}{2.189500in}}%
\pgfusepath{clip}%
\pgfsetbuttcap%
\pgfsetroundjoin%
\pgfsetlinewidth{0.050187pt}%
\definecolor{currentstroke}{rgb}{1.000000,1.000000,1.000000}%
\pgfsetstrokecolor{currentstroke}%
\pgfsetdash{}{0pt}%
\pgfusepath{stroke}%
\end{pgfscope}%
\begin{pgfscope}%
\pgfpathrectangle{\pgfqpoint{0.100000in}{0.100000in}}{\pgfqpoint{3.420221in}{2.189500in}}%
\pgfusepath{clip}%
\pgfsetbuttcap%
\pgfsetroundjoin%
\pgfsetlinewidth{0.050187pt}%
\definecolor{currentstroke}{rgb}{1.000000,1.000000,1.000000}%
\pgfsetstrokecolor{currentstroke}%
\pgfsetdash{}{0pt}%
\pgfusepath{stroke}%
\end{pgfscope}%
\begin{pgfscope}%
\pgfpathrectangle{\pgfqpoint{0.100000in}{0.100000in}}{\pgfqpoint{3.420221in}{2.189500in}}%
\pgfusepath{clip}%
\pgfsetbuttcap%
\pgfsetroundjoin%
\pgfsetlinewidth{0.050187pt}%
\definecolor{currentstroke}{rgb}{1.000000,1.000000,1.000000}%
\pgfsetstrokecolor{currentstroke}%
\pgfsetdash{}{0pt}%
\pgfusepath{stroke}%
\end{pgfscope}%
\begin{pgfscope}%
\pgfpathrectangle{\pgfqpoint{0.100000in}{0.100000in}}{\pgfqpoint{3.420221in}{2.189500in}}%
\pgfusepath{clip}%
\pgfsetbuttcap%
\pgfsetroundjoin%
\pgfsetlinewidth{0.050187pt}%
\definecolor{currentstroke}{rgb}{1.000000,1.000000,1.000000}%
\pgfsetstrokecolor{currentstroke}%
\pgfsetdash{}{0pt}%
\pgfusepath{stroke}%
\end{pgfscope}%
\begin{pgfscope}%
\pgfpathrectangle{\pgfqpoint{0.100000in}{0.100000in}}{\pgfqpoint{3.420221in}{2.189500in}}%
\pgfusepath{clip}%
\pgfsetbuttcap%
\pgfsetroundjoin%
\pgfsetlinewidth{0.050187pt}%
\definecolor{currentstroke}{rgb}{1.000000,1.000000,1.000000}%
\pgfsetstrokecolor{currentstroke}%
\pgfsetdash{}{0pt}%
\pgfpathmoveto{\pgfqpoint{0.912678in}{2.160731in}}%
\pgfpathlineto{\pgfqpoint{0.894941in}{2.090452in}}%
\pgfpathlineto{\pgfqpoint{0.882329in}{2.040095in}}%
\pgfpathlineto{\pgfqpoint{0.867334in}{1.979518in}}%
\pgfpathlineto{\pgfqpoint{0.864390in}{1.964587in}}%
\pgfpathlineto{\pgfqpoint{0.866313in}{1.950893in}}%
\pgfpathlineto{\pgfqpoint{0.863764in}{1.938278in}}%
\pgfpathlineto{\pgfqpoint{0.759091in}{1.965593in}}%
\pgfpathlineto{\pgfqpoint{0.749635in}{1.962420in}}%
\pgfpathlineto{\pgfqpoint{0.741543in}{1.965174in}}%
\pgfpathlineto{\pgfqpoint{0.704079in}{1.966073in}}%
\pgfpathlineto{\pgfqpoint{0.691407in}{1.962501in}}%
\pgfpathlineto{\pgfqpoint{0.678824in}{1.963708in}}%
\pgfpathlineto{\pgfqpoint{0.673491in}{1.969300in}}%
\pgfpathlineto{\pgfqpoint{0.639949in}{1.968081in}}%
\pgfpathlineto{\pgfqpoint{0.634652in}{1.977218in}}%
\pgfpathlineto{\pgfqpoint{0.624597in}{1.982033in}}%
\pgfpathlineto{\pgfqpoint{0.609965in}{1.984966in}}%
\pgfpathlineto{\pgfqpoint{0.584712in}{1.980671in}}%
\pgfpathlineto{\pgfqpoint{0.575389in}{1.984879in}}%
\pgfpathlineto{\pgfqpoint{0.561016in}{1.996072in}}%
\pgfpathlineto{\pgfqpoint{0.565336in}{2.018009in}}%
\pgfpathlineto{\pgfqpoint{0.563822in}{2.029212in}}%
\pgfpathlineto{\pgfqpoint{0.552430in}{2.041104in}}%
\pgfpathlineto{\pgfqpoint{0.545133in}{2.040367in}}%
\pgfpathlineto{\pgfqpoint{0.539899in}{2.052397in}}%
\pgfpathlineto{\pgfqpoint{0.527499in}{2.057259in}}%
\pgfpathlineto{\pgfqpoint{0.518486in}{2.056640in}}%
\pgfpathlineto{\pgfqpoint{0.515570in}{2.068995in}}%
\pgfpathlineto{\pgfqpoint{0.524549in}{2.067680in}}%
\pgfpathlineto{\pgfqpoint{0.523836in}{2.084852in}}%
\pgfpathlineto{\pgfqpoint{0.519923in}{2.095049in}}%
\pgfpathlineto{\pgfqpoint{0.521648in}{2.108100in}}%
\pgfpathlineto{\pgfqpoint{0.526346in}{2.117916in}}%
\pgfpathlineto{\pgfqpoint{0.526081in}{2.136543in}}%
\pgfpathlineto{\pgfqpoint{0.523573in}{2.143301in}}%
\pgfpathlineto{\pgfqpoint{0.527904in}{2.164984in}}%
\pgfpathlineto{\pgfqpoint{0.526654in}{2.178958in}}%
\pgfpathlineto{\pgfqpoint{0.522324in}{2.185658in}}%
\pgfpathlineto{\pgfqpoint{0.522984in}{2.207565in}}%
\pgfpathlineto{\pgfqpoint{0.529283in}{2.223676in}}%
\pgfpathlineto{\pgfqpoint{0.536133in}{2.219733in}}%
\pgfpathlineto{\pgfqpoint{0.558816in}{2.196302in}}%
\pgfpathlineto{\pgfqpoint{0.586302in}{2.183480in}}%
\pgfpathlineto{\pgfqpoint{0.600375in}{2.181962in}}%
\pgfpathlineto{\pgfqpoint{0.613634in}{2.176503in}}%
\pgfpathlineto{\pgfqpoint{0.617398in}{2.158133in}}%
\pgfpathlineto{\pgfqpoint{0.605778in}{2.154653in}}%
\pgfpathlineto{\pgfqpoint{0.598078in}{2.140176in}}%
\pgfpathlineto{\pgfqpoint{0.607119in}{2.140980in}}%
\pgfpathlineto{\pgfqpoint{0.610765in}{2.147479in}}%
\pgfpathlineto{\pgfqpoint{0.623514in}{2.155605in}}%
\pgfpathlineto{\pgfqpoint{0.622856in}{2.143537in}}%
\pgfpathlineto{\pgfqpoint{0.614343in}{2.141591in}}%
\pgfpathlineto{\pgfqpoint{0.615756in}{2.126080in}}%
\pgfpathlineto{\pgfqpoint{0.605390in}{2.110542in}}%
\pgfpathlineto{\pgfqpoint{0.598755in}{2.117930in}}%
\pgfpathlineto{\pgfqpoint{0.579038in}{2.113913in}}%
\pgfpathlineto{\pgfqpoint{0.578001in}{2.104874in}}%
\pgfpathlineto{\pgfqpoint{0.584798in}{2.099367in}}%
\pgfpathlineto{\pgfqpoint{0.595167in}{2.098826in}}%
\pgfpathlineto{\pgfqpoint{0.609505in}{2.110593in}}%
\pgfpathlineto{\pgfqpoint{0.620854in}{2.111571in}}%
\pgfpathlineto{\pgfqpoint{0.621289in}{2.124596in}}%
\pgfpathlineto{\pgfqpoint{0.627138in}{2.143728in}}%
\pgfpathlineto{\pgfqpoint{0.640998in}{2.157957in}}%
\pgfpathlineto{\pgfqpoint{0.636376in}{2.168977in}}%
\pgfpathlineto{\pgfqpoint{0.639533in}{2.180843in}}%
\pgfpathlineto{\pgfqpoint{0.633384in}{2.192668in}}%
\pgfpathlineto{\pgfqpoint{0.642147in}{2.207790in}}%
\pgfpathlineto{\pgfqpoint{0.643460in}{2.216888in}}%
\pgfpathlineto{\pgfqpoint{0.635851in}{2.222867in}}%
\pgfpathlineto{\pgfqpoint{0.637118in}{2.238162in}}%
\pgfpathlineto{\pgfqpoint{0.728280in}{2.210465in}}%
\pgfpathlineto{\pgfqpoint{0.825086in}{2.183319in}}%
\pgfpathlineto{\pgfqpoint{0.912678in}{2.160731in}}%
\pgfusepath{stroke}%
\end{pgfscope}%
\begin{pgfscope}%
\pgfpathrectangle{\pgfqpoint{0.100000in}{0.100000in}}{\pgfqpoint{3.420221in}{2.189500in}}%
\pgfusepath{clip}%
\pgfsetbuttcap%
\pgfsetroundjoin%
\pgfsetlinewidth{0.050187pt}%
\definecolor{currentstroke}{rgb}{1.000000,1.000000,1.000000}%
\pgfsetstrokecolor{currentstroke}%
\pgfsetdash{}{0pt}%
\pgfpathmoveto{\pgfqpoint{0.622122in}{2.185250in}}%
\pgfpathlineto{\pgfqpoint{0.626611in}{2.167944in}}%
\pgfpathlineto{\pgfqpoint{0.633650in}{2.162195in}}%
\pgfpathlineto{\pgfqpoint{0.628033in}{2.154878in}}%
\pgfpathlineto{\pgfqpoint{0.622559in}{2.165600in}}%
\pgfpathlineto{\pgfqpoint{0.622122in}{2.185250in}}%
\pgfusepath{stroke}%
\end{pgfscope}%
\begin{pgfscope}%
\pgfpathrectangle{\pgfqpoint{0.100000in}{0.100000in}}{\pgfqpoint{3.420221in}{2.189500in}}%
\pgfusepath{clip}%
\pgfsetbuttcap%
\pgfsetroundjoin%
\pgfsetlinewidth{0.050187pt}%
\definecolor{currentstroke}{rgb}{1.000000,1.000000,1.000000}%
\pgfsetstrokecolor{currentstroke}%
\pgfsetdash{}{0pt}%
\pgfpathmoveto{\pgfqpoint{0.959687in}{2.149372in}}%
\pgfpathlineto{\pgfqpoint{1.057134in}{2.127504in}}%
\pgfpathlineto{\pgfqpoint{1.148908in}{2.108954in}}%
\pgfpathlineto{\pgfqpoint{1.219517in}{2.096016in}}%
\pgfpathlineto{\pgfqpoint{1.281075in}{2.085675in}}%
\pgfpathlineto{\pgfqpoint{1.342770in}{2.076181in}}%
\pgfpathlineto{\pgfqpoint{1.395309in}{2.068780in}}%
\pgfpathlineto{\pgfqpoint{1.447931in}{2.061993in}}%
\pgfpathlineto{\pgfqpoint{1.500628in}{2.055822in}}%
\pgfpathlineto{\pgfqpoint{1.550288in}{2.050577in}}%
\pgfpathlineto{\pgfqpoint{1.543459in}{1.974893in}}%
\pgfpathlineto{\pgfqpoint{1.533293in}{1.872617in}}%
\pgfpathlineto{\pgfqpoint{1.527961in}{1.820015in}}%
\pgfpathlineto{\pgfqpoint{1.520293in}{1.749123in}}%
\pgfpathlineto{\pgfqpoint{1.466009in}{1.755050in}}%
\pgfpathlineto{\pgfqpoint{1.403859in}{1.762105in}}%
\pgfpathlineto{\pgfqpoint{1.317564in}{1.773916in}}%
\pgfpathlineto{\pgfqpoint{1.279022in}{1.779414in}}%
\pgfpathlineto{\pgfqpoint{1.184066in}{1.794319in}}%
\pgfpathlineto{\pgfqpoint{1.151378in}{1.800299in}}%
\pgfpathlineto{\pgfqpoint{1.144565in}{1.761563in}}%
\pgfpathlineto{\pgfqpoint{1.140840in}{1.764327in}}%
\pgfpathlineto{\pgfqpoint{1.133841in}{1.782962in}}%
\pgfpathlineto{\pgfqpoint{1.124843in}{1.781814in}}%
\pgfpathlineto{\pgfqpoint{1.118402in}{1.772394in}}%
\pgfpathlineto{\pgfqpoint{1.105037in}{1.771133in}}%
\pgfpathlineto{\pgfqpoint{1.102352in}{1.775640in}}%
\pgfpathlineto{\pgfqpoint{1.089833in}{1.775095in}}%
\pgfpathlineto{\pgfqpoint{1.083540in}{1.779460in}}%
\pgfpathlineto{\pgfqpoint{1.074740in}{1.772707in}}%
\pgfpathlineto{\pgfqpoint{1.053364in}{1.778783in}}%
\pgfpathlineto{\pgfqpoint{1.044036in}{1.775520in}}%
\pgfpathlineto{\pgfqpoint{1.039018in}{1.789999in}}%
\pgfpathlineto{\pgfqpoint{1.038783in}{1.803776in}}%
\pgfpathlineto{\pgfqpoint{1.024049in}{1.813684in}}%
\pgfpathlineto{\pgfqpoint{1.026451in}{1.827963in}}%
\pgfpathlineto{\pgfqpoint{1.015960in}{1.851550in}}%
\pgfpathlineto{\pgfqpoint{1.016666in}{1.871600in}}%
\pgfpathlineto{\pgfqpoint{1.007649in}{1.881413in}}%
\pgfpathlineto{\pgfqpoint{0.999189in}{1.872731in}}%
\pgfpathlineto{\pgfqpoint{0.987673in}{1.868036in}}%
\pgfpathlineto{\pgfqpoint{0.978907in}{1.877714in}}%
\pgfpathlineto{\pgfqpoint{0.980333in}{1.893169in}}%
\pgfpathlineto{\pgfqpoint{0.990846in}{1.899258in}}%
\pgfpathlineto{\pgfqpoint{0.988051in}{1.909050in}}%
\pgfpathlineto{\pgfqpoint{1.006325in}{1.956444in}}%
\pgfpathlineto{\pgfqpoint{0.991916in}{1.957632in}}%
\pgfpathlineto{\pgfqpoint{0.990434in}{1.966137in}}%
\pgfpathlineto{\pgfqpoint{0.979938in}{1.973471in}}%
\pgfpathlineto{\pgfqpoint{0.980609in}{1.981670in}}%
\pgfpathlineto{\pgfqpoint{0.975072in}{1.988040in}}%
\pgfpathlineto{\pgfqpoint{0.966059in}{2.011166in}}%
\pgfpathlineto{\pgfqpoint{0.957823in}{2.015882in}}%
\pgfpathlineto{\pgfqpoint{0.951925in}{2.035841in}}%
\pgfpathlineto{\pgfqpoint{0.953316in}{2.049107in}}%
\pgfpathlineto{\pgfqpoint{0.942313in}{2.073541in}}%
\pgfpathlineto{\pgfqpoint{0.959687in}{2.149372in}}%
\pgfusepath{stroke}%
\end{pgfscope}%
\begin{pgfscope}%
\pgfpathrectangle{\pgfqpoint{0.100000in}{0.100000in}}{\pgfqpoint{3.420221in}{2.189500in}}%
\pgfusepath{clip}%
\pgfsetbuttcap%
\pgfsetroundjoin%
\pgfsetlinewidth{0.050187pt}%
\definecolor{currentstroke}{rgb}{1.000000,1.000000,1.000000}%
\pgfsetstrokecolor{currentstroke}%
\pgfsetdash{}{0pt}%
\pgfpathmoveto{\pgfqpoint{3.312135in}{1.759362in}}%
\pgfpathlineto{\pgfqpoint{3.310184in}{1.767655in}}%
\pgfpathlineto{\pgfqpoint{3.299345in}{1.774920in}}%
\pgfpathlineto{\pgfqpoint{3.270700in}{1.868618in}}%
\pgfpathlineto{\pgfqpoint{3.255260in}{1.913835in}}%
\pgfpathlineto{\pgfqpoint{3.267673in}{1.926861in}}%
\pgfpathlineto{\pgfqpoint{3.279739in}{1.958742in}}%
\pgfpathlineto{\pgfqpoint{3.285757in}{1.968930in}}%
\pgfpathlineto{\pgfqpoint{3.281494in}{1.973209in}}%
\pgfpathlineto{\pgfqpoint{3.280066in}{2.001509in}}%
\pgfpathlineto{\pgfqpoint{3.285450in}{2.010184in}}%
\pgfpathlineto{\pgfqpoint{3.283090in}{2.030349in}}%
\pgfpathlineto{\pgfqpoint{3.304576in}{2.096507in}}%
\pgfpathlineto{\pgfqpoint{3.314258in}{2.096868in}}%
\pgfpathlineto{\pgfqpoint{3.318277in}{2.085051in}}%
\pgfpathlineto{\pgfqpoint{3.326880in}{2.081657in}}%
\pgfpathlineto{\pgfqpoint{3.343134in}{2.095558in}}%
\pgfpathlineto{\pgfqpoint{3.355878in}{2.103768in}}%
\pgfpathlineto{\pgfqpoint{3.383958in}{2.089310in}}%
\pgfpathlineto{\pgfqpoint{3.409326in}{2.009036in}}%
\pgfpathlineto{\pgfqpoint{3.414148in}{1.989286in}}%
\pgfpathlineto{\pgfqpoint{3.425260in}{1.986970in}}%
\pgfpathlineto{\pgfqpoint{3.440330in}{1.973556in}}%
\pgfpathlineto{\pgfqpoint{3.439477in}{1.965741in}}%
\pgfpathlineto{\pgfqpoint{3.449743in}{1.956478in}}%
\pgfpathlineto{\pgfqpoint{3.458082in}{1.961793in}}%
\pgfpathlineto{\pgfqpoint{3.475533in}{1.941441in}}%
\pgfpathlineto{\pgfqpoint{3.467734in}{1.925153in}}%
\pgfpathlineto{\pgfqpoint{3.457275in}{1.924831in}}%
\pgfpathlineto{\pgfqpoint{3.448885in}{1.910291in}}%
\pgfpathlineto{\pgfqpoint{3.438720in}{1.908230in}}%
\pgfpathlineto{\pgfqpoint{3.431258in}{1.900487in}}%
\pgfpathlineto{\pgfqpoint{3.408971in}{1.892185in}}%
\pgfpathlineto{\pgfqpoint{3.395481in}{1.878805in}}%
\pgfpathlineto{\pgfqpoint{3.388629in}{1.888409in}}%
\pgfpathlineto{\pgfqpoint{3.382402in}{1.881526in}}%
\pgfpathlineto{\pgfqpoint{3.384107in}{1.853699in}}%
\pgfpathlineto{\pgfqpoint{3.379177in}{1.842677in}}%
\pgfpathlineto{\pgfqpoint{3.368423in}{1.845687in}}%
\pgfpathlineto{\pgfqpoint{3.366687in}{1.834387in}}%
\pgfpathlineto{\pgfqpoint{3.362040in}{1.829741in}}%
\pgfpathlineto{\pgfqpoint{3.352741in}{1.830873in}}%
\pgfpathlineto{\pgfqpoint{3.353327in}{1.820308in}}%
\pgfpathlineto{\pgfqpoint{3.339231in}{1.823308in}}%
\pgfpathlineto{\pgfqpoint{3.331533in}{1.808670in}}%
\pgfpathlineto{\pgfqpoint{3.334440in}{1.801011in}}%
\pgfpathlineto{\pgfqpoint{3.329861in}{1.788272in}}%
\pgfpathlineto{\pgfqpoint{3.322652in}{1.778902in}}%
\pgfpathlineto{\pgfqpoint{3.320838in}{1.759342in}}%
\pgfpathlineto{\pgfqpoint{3.312135in}{1.759362in}}%
\pgfusepath{stroke}%
\end{pgfscope}%
\begin{pgfscope}%
\pgfpathrectangle{\pgfqpoint{0.100000in}{0.100000in}}{\pgfqpoint{3.420221in}{2.189500in}}%
\pgfusepath{clip}%
\pgfsetbuttcap%
\pgfsetroundjoin%
\pgfsetlinewidth{0.050187pt}%
\definecolor{currentstroke}{rgb}{1.000000,1.000000,1.000000}%
\pgfsetstrokecolor{currentstroke}%
\pgfsetdash{}{0pt}%
\pgfpathmoveto{\pgfqpoint{3.412974in}{1.886471in}}%
\pgfpathlineto{\pgfqpoint{3.419338in}{1.893137in}}%
\pgfpathlineto{\pgfqpoint{3.425396in}{1.886871in}}%
\pgfpathlineto{\pgfqpoint{3.414514in}{1.878570in}}%
\pgfpathlineto{\pgfqpoint{3.412974in}{1.886471in}}%
\pgfusepath{stroke}%
\end{pgfscope}%
\begin{pgfscope}%
\pgfpathrectangle{\pgfqpoint{0.100000in}{0.100000in}}{\pgfqpoint{3.420221in}{2.189500in}}%
\pgfusepath{clip}%
\pgfsetbuttcap%
\pgfsetroundjoin%
\pgfsetlinewidth{0.050187pt}%
\definecolor{currentstroke}{rgb}{1.000000,1.000000,1.000000}%
\pgfsetstrokecolor{currentstroke}%
\pgfsetdash{}{0pt}%
\pgfpathmoveto{\pgfqpoint{1.527961in}{1.820015in}}%
\pgfpathlineto{\pgfqpoint{1.533293in}{1.872617in}}%
\pgfpathlineto{\pgfqpoint{1.543459in}{1.974893in}}%
\pgfpathlineto{\pgfqpoint{1.550288in}{2.050577in}}%
\pgfpathlineto{\pgfqpoint{1.606220in}{2.045330in}}%
\pgfpathlineto{\pgfqpoint{1.677775in}{2.039633in}}%
\pgfpathlineto{\pgfqpoint{1.743180in}{2.035420in}}%
\pgfpathlineto{\pgfqpoint{1.802401in}{2.032424in}}%
\pgfpathlineto{\pgfqpoint{1.890780in}{2.029392in}}%
\pgfpathlineto{\pgfqpoint{1.896812in}{2.004483in}}%
\pgfpathlineto{\pgfqpoint{1.894653in}{1.992579in}}%
\pgfpathlineto{\pgfqpoint{1.893942in}{1.968112in}}%
\pgfpathlineto{\pgfqpoint{1.898046in}{1.949788in}}%
\pgfpathlineto{\pgfqpoint{1.907447in}{1.922754in}}%
\pgfpathlineto{\pgfqpoint{1.907425in}{1.888719in}}%
\pgfpathlineto{\pgfqpoint{1.909165in}{1.849181in}}%
\pgfpathlineto{\pgfqpoint{1.911553in}{1.838525in}}%
\pgfpathlineto{\pgfqpoint{1.918541in}{1.826826in}}%
\pgfpathlineto{\pgfqpoint{1.920845in}{1.808598in}}%
\pgfpathlineto{\pgfqpoint{1.919850in}{1.796428in}}%
\pgfpathlineto{\pgfqpoint{1.845759in}{1.798033in}}%
\pgfpathlineto{\pgfqpoint{1.791859in}{1.800766in}}%
\pgfpathlineto{\pgfqpoint{1.712845in}{1.804972in}}%
\pgfpathlineto{\pgfqpoint{1.634926in}{1.810507in}}%
\pgfpathlineto{\pgfqpoint{1.583027in}{1.814693in}}%
\pgfpathlineto{\pgfqpoint{1.527961in}{1.820015in}}%
\pgfusepath{stroke}%
\end{pgfscope}%
\begin{pgfscope}%
\pgfpathrectangle{\pgfqpoint{0.100000in}{0.100000in}}{\pgfqpoint{3.420221in}{2.189500in}}%
\pgfusepath{clip}%
\pgfsetbuttcap%
\pgfsetroundjoin%
\pgfsetlinewidth{0.050187pt}%
\definecolor{currentstroke}{rgb}{1.000000,1.000000,1.000000}%
\pgfsetstrokecolor{currentstroke}%
\pgfsetdash{}{0pt}%
\pgfpathmoveto{\pgfqpoint{1.505565in}{1.599871in}}%
\pgfpathlineto{\pgfqpoint{1.514120in}{1.688048in}}%
\pgfpathlineto{\pgfqpoint{1.520293in}{1.749123in}}%
\pgfpathlineto{\pgfqpoint{1.527961in}{1.820015in}}%
\pgfpathlineto{\pgfqpoint{1.583027in}{1.814693in}}%
\pgfpathlineto{\pgfqpoint{1.634926in}{1.810507in}}%
\pgfpathlineto{\pgfqpoint{1.712845in}{1.804972in}}%
\pgfpathlineto{\pgfqpoint{1.791859in}{1.800766in}}%
\pgfpathlineto{\pgfqpoint{1.845759in}{1.798033in}}%
\pgfpathlineto{\pgfqpoint{1.919850in}{1.796428in}}%
\pgfpathlineto{\pgfqpoint{1.914834in}{1.781790in}}%
\pgfpathlineto{\pgfqpoint{1.904821in}{1.770298in}}%
\pgfpathlineto{\pgfqpoint{1.912489in}{1.757068in}}%
\pgfpathlineto{\pgfqpoint{1.920943in}{1.754241in}}%
\pgfpathlineto{\pgfqpoint{1.924955in}{1.746641in}}%
\pgfpathlineto{\pgfqpoint{1.924034in}{1.691151in}}%
\pgfpathlineto{\pgfqpoint{1.922492in}{1.613058in}}%
\pgfpathlineto{\pgfqpoint{1.916790in}{1.592547in}}%
\pgfpathlineto{\pgfqpoint{1.921890in}{1.581201in}}%
\pgfpathlineto{\pgfqpoint{1.911650in}{1.556821in}}%
\pgfpathlineto{\pgfqpoint{1.922438in}{1.537167in}}%
\pgfpathlineto{\pgfqpoint{1.913273in}{1.538671in}}%
\pgfpathlineto{\pgfqpoint{1.907019in}{1.550899in}}%
\pgfpathlineto{\pgfqpoint{1.884701in}{1.559281in}}%
\pgfpathlineto{\pgfqpoint{1.870626in}{1.566654in}}%
\pgfpathlineto{\pgfqpoint{1.847007in}{1.567267in}}%
\pgfpathlineto{\pgfqpoint{1.838809in}{1.560550in}}%
\pgfpathlineto{\pgfqpoint{1.812075in}{1.573776in}}%
\pgfpathlineto{\pgfqpoint{1.810024in}{1.577959in}}%
\pgfpathlineto{\pgfqpoint{1.716726in}{1.582375in}}%
\pgfpathlineto{\pgfqpoint{1.660043in}{1.585624in}}%
\pgfpathlineto{\pgfqpoint{1.613222in}{1.589291in}}%
\pgfpathlineto{\pgfqpoint{1.535857in}{1.596600in}}%
\pgfpathlineto{\pgfqpoint{1.505565in}{1.599871in}}%
\pgfusepath{stroke}%
\end{pgfscope}%
\begin{pgfscope}%
\pgfpathrectangle{\pgfqpoint{0.100000in}{0.100000in}}{\pgfqpoint{3.420221in}{2.189500in}}%
\pgfusepath{clip}%
\pgfsetbuttcap%
\pgfsetroundjoin%
\pgfsetlinewidth{0.050187pt}%
\definecolor{currentstroke}{rgb}{1.000000,1.000000,1.000000}%
\pgfsetstrokecolor{currentstroke}%
\pgfsetdash{}{0pt}%
\pgfpathmoveto{\pgfqpoint{1.490892in}{1.450563in}}%
\pgfpathlineto{\pgfqpoint{1.441153in}{1.455105in}}%
\pgfpathlineto{\pgfqpoint{1.332734in}{1.468487in}}%
\pgfpathlineto{\pgfqpoint{1.273755in}{1.476952in}}%
\pgfpathlineto{\pgfqpoint{1.210497in}{1.486032in}}%
\pgfpathlineto{\pgfqpoint{1.157213in}{1.494559in}}%
\pgfpathlineto{\pgfqpoint{1.098726in}{1.504573in}}%
\pgfpathlineto{\pgfqpoint{1.112024in}{1.578340in}}%
\pgfpathlineto{\pgfqpoint{1.125493in}{1.653981in}}%
\pgfpathlineto{\pgfqpoint{1.144565in}{1.761563in}}%
\pgfpathlineto{\pgfqpoint{1.151378in}{1.800299in}}%
\pgfpathlineto{\pgfqpoint{1.184066in}{1.794319in}}%
\pgfpathlineto{\pgfqpoint{1.279022in}{1.779414in}}%
\pgfpathlineto{\pgfqpoint{1.317564in}{1.773916in}}%
\pgfpathlineto{\pgfqpoint{1.403859in}{1.762105in}}%
\pgfpathlineto{\pgfqpoint{1.466009in}{1.755050in}}%
\pgfpathlineto{\pgfqpoint{1.520293in}{1.749123in}}%
\pgfpathlineto{\pgfqpoint{1.514120in}{1.688048in}}%
\pgfpathlineto{\pgfqpoint{1.505565in}{1.599871in}}%
\pgfpathlineto{\pgfqpoint{1.498223in}{1.524933in}}%
\pgfpathlineto{\pgfqpoint{1.490892in}{1.450563in}}%
\pgfusepath{stroke}%
\end{pgfscope}%
\begin{pgfscope}%
\pgfpathrectangle{\pgfqpoint{0.100000in}{0.100000in}}{\pgfqpoint{3.420221in}{2.189500in}}%
\pgfusepath{clip}%
\pgfsetbuttcap%
\pgfsetroundjoin%
\pgfsetlinewidth{0.050187pt}%
\definecolor{currentstroke}{rgb}{1.000000,1.000000,1.000000}%
\pgfsetstrokecolor{currentstroke}%
\pgfsetdash{}{0pt}%
\pgfpathmoveto{\pgfqpoint{2.400293in}{1.552318in}}%
\pgfpathlineto{\pgfqpoint{2.337262in}{1.547840in}}%
\pgfpathlineto{\pgfqpoint{2.243311in}{1.543836in}}%
\pgfpathlineto{\pgfqpoint{2.239741in}{1.553326in}}%
\pgfpathlineto{\pgfqpoint{2.218907in}{1.560420in}}%
\pgfpathlineto{\pgfqpoint{2.214311in}{1.573824in}}%
\pgfpathlineto{\pgfqpoint{2.212387in}{1.590405in}}%
\pgfpathlineto{\pgfqpoint{2.217082in}{1.598900in}}%
\pgfpathlineto{\pgfqpoint{2.209667in}{1.607053in}}%
\pgfpathlineto{\pgfqpoint{2.207878in}{1.616777in}}%
\pgfpathlineto{\pgfqpoint{2.205488in}{1.638287in}}%
\pgfpathlineto{\pgfqpoint{2.198382in}{1.649971in}}%
\pgfpathlineto{\pgfqpoint{2.185761in}{1.656529in}}%
\pgfpathlineto{\pgfqpoint{2.172027in}{1.667351in}}%
\pgfpathlineto{\pgfqpoint{2.164928in}{1.680155in}}%
\pgfpathlineto{\pgfqpoint{2.152189in}{1.685311in}}%
\pgfpathlineto{\pgfqpoint{2.144701in}{1.693703in}}%
\pgfpathlineto{\pgfqpoint{2.135629in}{1.695127in}}%
\pgfpathlineto{\pgfqpoint{2.119433in}{1.707587in}}%
\pgfpathlineto{\pgfqpoint{2.122064in}{1.721926in}}%
\pgfpathlineto{\pgfqpoint{2.121557in}{1.749193in}}%
\pgfpathlineto{\pgfqpoint{2.126552in}{1.756699in}}%
\pgfpathlineto{\pgfqpoint{2.122064in}{1.768033in}}%
\pgfpathlineto{\pgfqpoint{2.114158in}{1.770225in}}%
\pgfpathlineto{\pgfqpoint{2.114812in}{1.780179in}}%
\pgfpathlineto{\pgfqpoint{2.124620in}{1.795903in}}%
\pgfpathlineto{\pgfqpoint{2.144044in}{1.808338in}}%
\pgfpathlineto{\pgfqpoint{2.142834in}{1.852560in}}%
\pgfpathlineto{\pgfqpoint{2.152553in}{1.859202in}}%
\pgfpathlineto{\pgfqpoint{2.161751in}{1.854777in}}%
\pgfpathlineto{\pgfqpoint{2.180487in}{1.861255in}}%
\pgfpathlineto{\pgfqpoint{2.215739in}{1.877536in}}%
\pgfpathlineto{\pgfqpoint{2.220331in}{1.872495in}}%
\pgfpathlineto{\pgfqpoint{2.213654in}{1.849662in}}%
\pgfpathlineto{\pgfqpoint{2.223587in}{1.854673in}}%
\pgfpathlineto{\pgfqpoint{2.240600in}{1.849669in}}%
\pgfpathlineto{\pgfqpoint{2.251054in}{1.845493in}}%
\pgfpathlineto{\pgfqpoint{2.256915in}{1.833255in}}%
\pgfpathlineto{\pgfqpoint{2.310548in}{1.821705in}}%
\pgfpathlineto{\pgfqpoint{2.326580in}{1.813781in}}%
\pgfpathlineto{\pgfqpoint{2.342879in}{1.813875in}}%
\pgfpathlineto{\pgfqpoint{2.359627in}{1.810606in}}%
\pgfpathlineto{\pgfqpoint{2.370497in}{1.799441in}}%
\pgfpathlineto{\pgfqpoint{2.380883in}{1.793924in}}%
\pgfpathlineto{\pgfqpoint{2.382791in}{1.778011in}}%
\pgfpathlineto{\pgfqpoint{2.379723in}{1.768020in}}%
\pgfpathlineto{\pgfqpoint{2.391298in}{1.767281in}}%
\pgfpathlineto{\pgfqpoint{2.387399in}{1.755693in}}%
\pgfpathlineto{\pgfqpoint{2.391108in}{1.751577in}}%
\pgfpathlineto{\pgfqpoint{2.394797in}{1.740638in}}%
\pgfpathlineto{\pgfqpoint{2.383516in}{1.734847in}}%
\pgfpathlineto{\pgfqpoint{2.376966in}{1.718711in}}%
\pgfpathlineto{\pgfqpoint{2.374911in}{1.707300in}}%
\pgfpathlineto{\pgfqpoint{2.381195in}{1.705347in}}%
\pgfpathlineto{\pgfqpoint{2.389240in}{1.713904in}}%
\pgfpathlineto{\pgfqpoint{2.396081in}{1.728732in}}%
\pgfpathlineto{\pgfqpoint{2.405335in}{1.733881in}}%
\pgfpathlineto{\pgfqpoint{2.412291in}{1.727182in}}%
\pgfpathlineto{\pgfqpoint{2.405445in}{1.706907in}}%
\pgfpathlineto{\pgfqpoint{2.403297in}{1.691168in}}%
\pgfpathlineto{\pgfqpoint{2.405320in}{1.679854in}}%
\pgfpathlineto{\pgfqpoint{2.398962in}{1.673453in}}%
\pgfpathlineto{\pgfqpoint{2.395787in}{1.657812in}}%
\pgfpathlineto{\pgfqpoint{2.398381in}{1.641373in}}%
\pgfpathlineto{\pgfqpoint{2.390879in}{1.617063in}}%
\pgfpathlineto{\pgfqpoint{2.391037in}{1.604915in}}%
\pgfpathlineto{\pgfqpoint{2.396965in}{1.578591in}}%
\pgfpathlineto{\pgfqpoint{2.400808in}{1.574075in}}%
\pgfpathlineto{\pgfqpoint{2.400293in}{1.552318in}}%
\pgfusepath{stroke}%
\end{pgfscope}%
\begin{pgfscope}%
\pgfpathrectangle{\pgfqpoint{0.100000in}{0.100000in}}{\pgfqpoint{3.420221in}{2.189500in}}%
\pgfusepath{clip}%
\pgfsetbuttcap%
\pgfsetroundjoin%
\pgfsetlinewidth{0.050187pt}%
\definecolor{currentstroke}{rgb}{1.000000,1.000000,1.000000}%
\pgfsetstrokecolor{currentstroke}%
\pgfsetdash{}{0pt}%
\pgfpathmoveto{\pgfqpoint{2.423925in}{1.765669in}}%
\pgfpathlineto{\pgfqpoint{2.425332in}{1.755168in}}%
\pgfpathlineto{\pgfqpoint{2.412443in}{1.727502in}}%
\pgfpathlineto{\pgfqpoint{2.406715in}{1.735516in}}%
\pgfpathlineto{\pgfqpoint{2.423925in}{1.765669in}}%
\pgfusepath{stroke}%
\end{pgfscope}%
\begin{pgfscope}%
\pgfpathrectangle{\pgfqpoint{0.100000in}{0.100000in}}{\pgfqpoint{3.420221in}{2.189500in}}%
\pgfusepath{clip}%
\pgfsetbuttcap%
\pgfsetroundjoin%
\pgfsetlinewidth{0.050187pt}%
\definecolor{currentstroke}{rgb}{1.000000,1.000000,1.000000}%
\pgfsetstrokecolor{currentstroke}%
\pgfsetdash{}{0pt}%
\pgfpathmoveto{\pgfqpoint{0.863764in}{1.938278in}}%
\pgfpathlineto{\pgfqpoint{0.866313in}{1.950893in}}%
\pgfpathlineto{\pgfqpoint{0.864390in}{1.964587in}}%
\pgfpathlineto{\pgfqpoint{0.867334in}{1.979518in}}%
\pgfpathlineto{\pgfqpoint{0.882329in}{2.040095in}}%
\pgfpathlineto{\pgfqpoint{0.894941in}{2.090452in}}%
\pgfpathlineto{\pgfqpoint{0.912678in}{2.160731in}}%
\pgfpathlineto{\pgfqpoint{0.959687in}{2.149372in}}%
\pgfpathlineto{\pgfqpoint{0.942313in}{2.073541in}}%
\pgfpathlineto{\pgfqpoint{0.953316in}{2.049107in}}%
\pgfpathlineto{\pgfqpoint{0.951925in}{2.035841in}}%
\pgfpathlineto{\pgfqpoint{0.957823in}{2.015882in}}%
\pgfpathlineto{\pgfqpoint{0.966059in}{2.011166in}}%
\pgfpathlineto{\pgfqpoint{0.975072in}{1.988040in}}%
\pgfpathlineto{\pgfqpoint{0.980609in}{1.981670in}}%
\pgfpathlineto{\pgfqpoint{0.979938in}{1.973471in}}%
\pgfpathlineto{\pgfqpoint{0.990434in}{1.966137in}}%
\pgfpathlineto{\pgfqpoint{0.991916in}{1.957632in}}%
\pgfpathlineto{\pgfqpoint{1.006325in}{1.956444in}}%
\pgfpathlineto{\pgfqpoint{0.988051in}{1.909050in}}%
\pgfpathlineto{\pgfqpoint{0.990846in}{1.899258in}}%
\pgfpathlineto{\pgfqpoint{0.980333in}{1.893169in}}%
\pgfpathlineto{\pgfqpoint{0.978907in}{1.877714in}}%
\pgfpathlineto{\pgfqpoint{0.987673in}{1.868036in}}%
\pgfpathlineto{\pgfqpoint{0.999189in}{1.872731in}}%
\pgfpathlineto{\pgfqpoint{1.007649in}{1.881413in}}%
\pgfpathlineto{\pgfqpoint{1.016666in}{1.871600in}}%
\pgfpathlineto{\pgfqpoint{1.015960in}{1.851550in}}%
\pgfpathlineto{\pgfqpoint{1.026451in}{1.827963in}}%
\pgfpathlineto{\pgfqpoint{1.024049in}{1.813684in}}%
\pgfpathlineto{\pgfqpoint{1.038783in}{1.803776in}}%
\pgfpathlineto{\pgfqpoint{1.039018in}{1.789999in}}%
\pgfpathlineto{\pgfqpoint{1.044036in}{1.775520in}}%
\pgfpathlineto{\pgfqpoint{1.053364in}{1.778783in}}%
\pgfpathlineto{\pgfqpoint{1.074740in}{1.772707in}}%
\pgfpathlineto{\pgfqpoint{1.083540in}{1.779460in}}%
\pgfpathlineto{\pgfqpoint{1.089833in}{1.775095in}}%
\pgfpathlineto{\pgfqpoint{1.102352in}{1.775640in}}%
\pgfpathlineto{\pgfqpoint{1.105037in}{1.771133in}}%
\pgfpathlineto{\pgfqpoint{1.118402in}{1.772394in}}%
\pgfpathlineto{\pgfqpoint{1.124843in}{1.781814in}}%
\pgfpathlineto{\pgfqpoint{1.133841in}{1.782962in}}%
\pgfpathlineto{\pgfqpoint{1.140840in}{1.764327in}}%
\pgfpathlineto{\pgfqpoint{1.144565in}{1.761563in}}%
\pgfpathlineto{\pgfqpoint{1.125493in}{1.653981in}}%
\pgfpathlineto{\pgfqpoint{1.112024in}{1.578340in}}%
\pgfpathlineto{\pgfqpoint{1.005787in}{1.598849in}}%
\pgfpathlineto{\pgfqpoint{0.948402in}{1.610263in}}%
\pgfpathlineto{\pgfqpoint{0.894726in}{1.622066in}}%
\pgfpathlineto{\pgfqpoint{0.844556in}{1.633425in}}%
\pgfpathlineto{\pgfqpoint{0.786494in}{1.647428in}}%
\pgfpathlineto{\pgfqpoint{0.816442in}{1.770288in}}%
\pgfpathlineto{\pgfqpoint{0.818057in}{1.779267in}}%
\pgfpathlineto{\pgfqpoint{0.831375in}{1.802794in}}%
\pgfpathlineto{\pgfqpoint{0.817585in}{1.816935in}}%
\pgfpathlineto{\pgfqpoint{0.820449in}{1.830821in}}%
\pgfpathlineto{\pgfqpoint{0.825698in}{1.834296in}}%
\pgfpathlineto{\pgfqpoint{0.835074in}{1.848605in}}%
\pgfpathlineto{\pgfqpoint{0.848740in}{1.858504in}}%
\pgfpathlineto{\pgfqpoint{0.849487in}{1.865829in}}%
\pgfpathlineto{\pgfqpoint{0.857735in}{1.873147in}}%
\pgfpathlineto{\pgfqpoint{0.864584in}{1.886845in}}%
\pgfpathlineto{\pgfqpoint{0.878632in}{1.901338in}}%
\pgfpathlineto{\pgfqpoint{0.877635in}{1.915651in}}%
\pgfpathlineto{\pgfqpoint{0.868030in}{1.923603in}}%
\pgfpathlineto{\pgfqpoint{0.863764in}{1.938278in}}%
\pgfusepath{stroke}%
\end{pgfscope}%
\begin{pgfscope}%
\pgfpathrectangle{\pgfqpoint{0.100000in}{0.100000in}}{\pgfqpoint{3.420221in}{2.189500in}}%
\pgfusepath{clip}%
\pgfsetbuttcap%
\pgfsetroundjoin%
\pgfsetlinewidth{0.050187pt}%
\definecolor{currentstroke}{rgb}{1.000000,1.000000,1.000000}%
\pgfsetstrokecolor{currentstroke}%
\pgfsetdash{}{0pt}%
\pgfpathmoveto{\pgfqpoint{3.189344in}{1.698517in}}%
\pgfpathlineto{\pgfqpoint{3.184930in}{1.712451in}}%
\pgfpathlineto{\pgfqpoint{3.176738in}{1.754721in}}%
\pgfpathlineto{\pgfqpoint{3.166147in}{1.767726in}}%
\pgfpathlineto{\pgfqpoint{3.156857in}{1.791113in}}%
\pgfpathlineto{\pgfqpoint{3.159907in}{1.808452in}}%
\pgfpathlineto{\pgfqpoint{3.157461in}{1.821231in}}%
\pgfpathlineto{\pgfqpoint{3.149524in}{1.833783in}}%
\pgfpathlineto{\pgfqpoint{3.149198in}{1.847610in}}%
\pgfpathlineto{\pgfqpoint{3.144589in}{1.862522in}}%
\pgfpathlineto{\pgfqpoint{3.185834in}{1.872673in}}%
\pgfpathlineto{\pgfqpoint{3.239407in}{1.887133in}}%
\pgfpathlineto{\pgfqpoint{3.241545in}{1.878841in}}%
\pgfpathlineto{\pgfqpoint{3.238134in}{1.865640in}}%
\pgfpathlineto{\pgfqpoint{3.246141in}{1.855078in}}%
\pgfpathlineto{\pgfqpoint{3.241888in}{1.841693in}}%
\pgfpathlineto{\pgfqpoint{3.224988in}{1.824869in}}%
\pgfpathlineto{\pgfqpoint{3.229633in}{1.812230in}}%
\pgfpathlineto{\pgfqpoint{3.226190in}{1.785517in}}%
\pgfpathlineto{\pgfqpoint{3.221371in}{1.770455in}}%
\pgfpathlineto{\pgfqpoint{3.227520in}{1.727623in}}%
\pgfpathlineto{\pgfqpoint{3.226691in}{1.712554in}}%
\pgfpathlineto{\pgfqpoint{3.232641in}{1.707717in}}%
\pgfpathlineto{\pgfqpoint{3.189344in}{1.698517in}}%
\pgfusepath{stroke}%
\end{pgfscope}%
\begin{pgfscope}%
\pgfpathrectangle{\pgfqpoint{0.100000in}{0.100000in}}{\pgfqpoint{3.420221in}{2.189500in}}%
\pgfusepath{clip}%
\pgfsetbuttcap%
\pgfsetroundjoin%
\pgfsetlinewidth{0.050187pt}%
\definecolor{currentstroke}{rgb}{1.000000,1.000000,1.000000}%
\pgfsetstrokecolor{currentstroke}%
\pgfsetdash{}{0pt}%
\pgfpathmoveto{\pgfqpoint{1.922492in}{1.613058in}}%
\pgfpathlineto{\pgfqpoint{1.924034in}{1.691151in}}%
\pgfpathlineto{\pgfqpoint{1.924955in}{1.746641in}}%
\pgfpathlineto{\pgfqpoint{1.920943in}{1.754241in}}%
\pgfpathlineto{\pgfqpoint{1.912489in}{1.757068in}}%
\pgfpathlineto{\pgfqpoint{1.904821in}{1.770298in}}%
\pgfpathlineto{\pgfqpoint{1.914834in}{1.781790in}}%
\pgfpathlineto{\pgfqpoint{1.919850in}{1.796428in}}%
\pgfpathlineto{\pgfqpoint{1.920845in}{1.808598in}}%
\pgfpathlineto{\pgfqpoint{1.918541in}{1.826826in}}%
\pgfpathlineto{\pgfqpoint{1.911553in}{1.838525in}}%
\pgfpathlineto{\pgfqpoint{1.909165in}{1.849181in}}%
\pgfpathlineto{\pgfqpoint{1.907425in}{1.888719in}}%
\pgfpathlineto{\pgfqpoint{1.907447in}{1.922754in}}%
\pgfpathlineto{\pgfqpoint{1.898046in}{1.949788in}}%
\pgfpathlineto{\pgfqpoint{1.893942in}{1.968112in}}%
\pgfpathlineto{\pgfqpoint{1.894653in}{1.992579in}}%
\pgfpathlineto{\pgfqpoint{1.896812in}{2.004483in}}%
\pgfpathlineto{\pgfqpoint{1.890780in}{2.029392in}}%
\pgfpathlineto{\pgfqpoint{1.931844in}{2.028570in}}%
\pgfpathlineto{\pgfqpoint{1.994218in}{2.028034in}}%
\pgfpathlineto{\pgfqpoint{1.994559in}{2.056345in}}%
\pgfpathlineto{\pgfqpoint{2.010436in}{2.053227in}}%
\pgfpathlineto{\pgfqpoint{2.018045in}{2.018707in}}%
\pgfpathlineto{\pgfqpoint{2.023655in}{2.006295in}}%
\pgfpathlineto{\pgfqpoint{2.037605in}{2.005929in}}%
\pgfpathlineto{\pgfqpoint{2.040727in}{2.001717in}}%
\pgfpathlineto{\pgfqpoint{2.060186in}{1.999852in}}%
\pgfpathlineto{\pgfqpoint{2.063457in}{1.991294in}}%
\pgfpathlineto{\pgfqpoint{2.076867in}{1.993217in}}%
\pgfpathlineto{\pgfqpoint{2.087283in}{2.001222in}}%
\pgfpathlineto{\pgfqpoint{2.105245in}{2.000922in}}%
\pgfpathlineto{\pgfqpoint{2.116359in}{1.994485in}}%
\pgfpathlineto{\pgfqpoint{2.117637in}{1.988448in}}%
\pgfpathlineto{\pgfqpoint{2.128212in}{1.987184in}}%
\pgfpathlineto{\pgfqpoint{2.135113in}{1.970717in}}%
\pgfpathlineto{\pgfqpoint{2.139566in}{1.980842in}}%
\pgfpathlineto{\pgfqpoint{2.151714in}{1.981465in}}%
\pgfpathlineto{\pgfqpoint{2.154787in}{1.973579in}}%
\pgfpathlineto{\pgfqpoint{2.168456in}{1.969980in}}%
\pgfpathlineto{\pgfqpoint{2.175994in}{1.958622in}}%
\pgfpathlineto{\pgfqpoint{2.192712in}{1.962149in}}%
\pgfpathlineto{\pgfqpoint{2.211108in}{1.976086in}}%
\pgfpathlineto{\pgfqpoint{2.217815in}{1.963800in}}%
\pgfpathlineto{\pgfqpoint{2.248000in}{1.967163in}}%
\pgfpathlineto{\pgfqpoint{2.260903in}{1.958730in}}%
\pgfpathlineto{\pgfqpoint{2.268413in}{1.961740in}}%
\pgfpathlineto{\pgfqpoint{2.274435in}{1.956993in}}%
\pgfpathlineto{\pgfqpoint{2.256570in}{1.945730in}}%
\pgfpathlineto{\pgfqpoint{2.231069in}{1.935694in}}%
\pgfpathlineto{\pgfqpoint{2.205812in}{1.915634in}}%
\pgfpathlineto{\pgfqpoint{2.183916in}{1.889254in}}%
\pgfpathlineto{\pgfqpoint{2.167340in}{1.873658in}}%
\pgfpathlineto{\pgfqpoint{2.152821in}{1.862939in}}%
\pgfpathlineto{\pgfqpoint{2.142834in}{1.852560in}}%
\pgfpathlineto{\pgfqpoint{2.144044in}{1.808338in}}%
\pgfpathlineto{\pgfqpoint{2.124620in}{1.795903in}}%
\pgfpathlineto{\pgfqpoint{2.114812in}{1.780179in}}%
\pgfpathlineto{\pgfqpoint{2.114158in}{1.770225in}}%
\pgfpathlineto{\pgfqpoint{2.122064in}{1.768033in}}%
\pgfpathlineto{\pgfqpoint{2.126552in}{1.756699in}}%
\pgfpathlineto{\pgfqpoint{2.121557in}{1.749193in}}%
\pgfpathlineto{\pgfqpoint{2.122064in}{1.721926in}}%
\pgfpathlineto{\pgfqpoint{2.119433in}{1.707587in}}%
\pgfpathlineto{\pgfqpoint{2.135629in}{1.695127in}}%
\pgfpathlineto{\pgfqpoint{2.144701in}{1.693703in}}%
\pgfpathlineto{\pgfqpoint{2.152189in}{1.685311in}}%
\pgfpathlineto{\pgfqpoint{2.164928in}{1.680155in}}%
\pgfpathlineto{\pgfqpoint{2.172027in}{1.667351in}}%
\pgfpathlineto{\pgfqpoint{2.185761in}{1.656529in}}%
\pgfpathlineto{\pgfqpoint{2.198382in}{1.649971in}}%
\pgfpathlineto{\pgfqpoint{2.205488in}{1.638287in}}%
\pgfpathlineto{\pgfqpoint{2.207878in}{1.616777in}}%
\pgfpathlineto{\pgfqpoint{2.140900in}{1.614344in}}%
\pgfpathlineto{\pgfqpoint{2.083807in}{1.613150in}}%
\pgfpathlineto{\pgfqpoint{2.031790in}{1.612386in}}%
\pgfpathlineto{\pgfqpoint{1.976761in}{1.612470in}}%
\pgfpathlineto{\pgfqpoint{1.922492in}{1.613058in}}%
\pgfusepath{stroke}%
\end{pgfscope}%
\begin{pgfscope}%
\pgfpathrectangle{\pgfqpoint{0.100000in}{0.100000in}}{\pgfqpoint{3.420221in}{2.189500in}}%
\pgfusepath{clip}%
\pgfsetbuttcap%
\pgfsetroundjoin%
\pgfsetlinewidth{0.050187pt}%
\definecolor{currentstroke}{rgb}{1.000000,1.000000,1.000000}%
\pgfsetstrokecolor{currentstroke}%
\pgfsetdash{}{0pt}%
\pgfpathmoveto{\pgfqpoint{0.401756in}{1.759050in}}%
\pgfpathlineto{\pgfqpoint{0.396451in}{1.768811in}}%
\pgfpathlineto{\pgfqpoint{0.399838in}{1.793864in}}%
\pgfpathlineto{\pgfqpoint{0.406375in}{1.806957in}}%
\pgfpathlineto{\pgfqpoint{0.403107in}{1.824669in}}%
\pgfpathlineto{\pgfqpoint{0.409927in}{1.832118in}}%
\pgfpathlineto{\pgfqpoint{0.422369in}{1.852194in}}%
\pgfpathlineto{\pgfqpoint{0.432918in}{1.864313in}}%
\pgfpathlineto{\pgfqpoint{0.448338in}{1.890708in}}%
\pgfpathlineto{\pgfqpoint{0.460165in}{1.919462in}}%
\pgfpathlineto{\pgfqpoint{0.472720in}{1.946429in}}%
\pgfpathlineto{\pgfqpoint{0.475257in}{1.957692in}}%
\pgfpathlineto{\pgfqpoint{0.492578in}{1.989912in}}%
\pgfpathlineto{\pgfqpoint{0.495912in}{2.004063in}}%
\pgfpathlineto{\pgfqpoint{0.503287in}{2.018923in}}%
\pgfpathlineto{\pgfqpoint{0.505923in}{2.037050in}}%
\pgfpathlineto{\pgfqpoint{0.520001in}{2.045890in}}%
\pgfpathlineto{\pgfqpoint{0.536694in}{2.050345in}}%
\pgfpathlineto{\pgfqpoint{0.545133in}{2.040367in}}%
\pgfpathlineto{\pgfqpoint{0.552430in}{2.041104in}}%
\pgfpathlineto{\pgfqpoint{0.563822in}{2.029212in}}%
\pgfpathlineto{\pgfqpoint{0.565336in}{2.018009in}}%
\pgfpathlineto{\pgfqpoint{0.561016in}{1.996072in}}%
\pgfpathlineto{\pgfqpoint{0.575389in}{1.984879in}}%
\pgfpathlineto{\pgfqpoint{0.584712in}{1.980671in}}%
\pgfpathlineto{\pgfqpoint{0.609965in}{1.984966in}}%
\pgfpathlineto{\pgfqpoint{0.624597in}{1.982033in}}%
\pgfpathlineto{\pgfqpoint{0.634652in}{1.977218in}}%
\pgfpathlineto{\pgfqpoint{0.639949in}{1.968081in}}%
\pgfpathlineto{\pgfqpoint{0.673491in}{1.969300in}}%
\pgfpathlineto{\pgfqpoint{0.678824in}{1.963708in}}%
\pgfpathlineto{\pgfqpoint{0.691407in}{1.962501in}}%
\pgfpathlineto{\pgfqpoint{0.704079in}{1.966073in}}%
\pgfpathlineto{\pgfqpoint{0.741543in}{1.965174in}}%
\pgfpathlineto{\pgfqpoint{0.749635in}{1.962420in}}%
\pgfpathlineto{\pgfqpoint{0.759091in}{1.965593in}}%
\pgfpathlineto{\pgfqpoint{0.863764in}{1.938278in}}%
\pgfpathlineto{\pgfqpoint{0.868030in}{1.923603in}}%
\pgfpathlineto{\pgfqpoint{0.877635in}{1.915651in}}%
\pgfpathlineto{\pgfqpoint{0.878632in}{1.901338in}}%
\pgfpathlineto{\pgfqpoint{0.864584in}{1.886845in}}%
\pgfpathlineto{\pgfqpoint{0.857735in}{1.873147in}}%
\pgfpathlineto{\pgfqpoint{0.849487in}{1.865829in}}%
\pgfpathlineto{\pgfqpoint{0.848740in}{1.858504in}}%
\pgfpathlineto{\pgfqpoint{0.835074in}{1.848605in}}%
\pgfpathlineto{\pgfqpoint{0.825698in}{1.834296in}}%
\pgfpathlineto{\pgfqpoint{0.820449in}{1.830821in}}%
\pgfpathlineto{\pgfqpoint{0.817585in}{1.816935in}}%
\pgfpathlineto{\pgfqpoint{0.831375in}{1.802794in}}%
\pgfpathlineto{\pgfqpoint{0.818057in}{1.779267in}}%
\pgfpathlineto{\pgfqpoint{0.816442in}{1.770288in}}%
\pgfpathlineto{\pgfqpoint{0.786494in}{1.647428in}}%
\pgfpathlineto{\pgfqpoint{0.723499in}{1.663569in}}%
\pgfpathlineto{\pgfqpoint{0.662728in}{1.679239in}}%
\pgfpathlineto{\pgfqpoint{0.626067in}{1.689441in}}%
\pgfpathlineto{\pgfqpoint{0.578961in}{1.702859in}}%
\pgfpathlineto{\pgfqpoint{0.503824in}{1.726509in}}%
\pgfpathlineto{\pgfqpoint{0.422146in}{1.751938in}}%
\pgfpathlineto{\pgfqpoint{0.401756in}{1.759050in}}%
\pgfusepath{stroke}%
\end{pgfscope}%
\begin{pgfscope}%
\pgfpathrectangle{\pgfqpoint{0.100000in}{0.100000in}}{\pgfqpoint{3.420221in}{2.189500in}}%
\pgfusepath{clip}%
\pgfsetbuttcap%
\pgfsetroundjoin%
\pgfsetlinewidth{0.050187pt}%
\definecolor{currentstroke}{rgb}{1.000000,1.000000,1.000000}%
\pgfsetstrokecolor{currentstroke}%
\pgfsetdash{}{0pt}%
\pgfpathmoveto{\pgfqpoint{3.232641in}{1.707717in}}%
\pgfpathlineto{\pgfqpoint{3.226691in}{1.712554in}}%
\pgfpathlineto{\pgfqpoint{3.227520in}{1.727623in}}%
\pgfpathlineto{\pgfqpoint{3.221371in}{1.770455in}}%
\pgfpathlineto{\pgfqpoint{3.226190in}{1.785517in}}%
\pgfpathlineto{\pgfqpoint{3.229633in}{1.812230in}}%
\pgfpathlineto{\pgfqpoint{3.224988in}{1.824869in}}%
\pgfpathlineto{\pgfqpoint{3.241888in}{1.841693in}}%
\pgfpathlineto{\pgfqpoint{3.246141in}{1.855078in}}%
\pgfpathlineto{\pgfqpoint{3.238134in}{1.865640in}}%
\pgfpathlineto{\pgfqpoint{3.241545in}{1.878841in}}%
\pgfpathlineto{\pgfqpoint{3.239407in}{1.887133in}}%
\pgfpathlineto{\pgfqpoint{3.241243in}{1.904892in}}%
\pgfpathlineto{\pgfqpoint{3.244676in}{1.910374in}}%
\pgfpathlineto{\pgfqpoint{3.255260in}{1.913835in}}%
\pgfpathlineto{\pgfqpoint{3.270700in}{1.868618in}}%
\pgfpathlineto{\pgfqpoint{3.299345in}{1.774920in}}%
\pgfpathlineto{\pgfqpoint{3.310184in}{1.767655in}}%
\pgfpathlineto{\pgfqpoint{3.312135in}{1.759362in}}%
\pgfpathlineto{\pgfqpoint{3.317856in}{1.756011in}}%
\pgfpathlineto{\pgfqpoint{3.317421in}{1.740970in}}%
\pgfpathlineto{\pgfqpoint{3.311359in}{1.740729in}}%
\pgfpathlineto{\pgfqpoint{3.299078in}{1.731362in}}%
\pgfpathlineto{\pgfqpoint{3.295536in}{1.721966in}}%
\pgfpathlineto{\pgfqpoint{3.262665in}{1.713833in}}%
\pgfpathlineto{\pgfqpoint{3.232641in}{1.707717in}}%
\pgfusepath{stroke}%
\end{pgfscope}%
\begin{pgfscope}%
\pgfpathrectangle{\pgfqpoint{0.100000in}{0.100000in}}{\pgfqpoint{3.420221in}{2.189500in}}%
\pgfusepath{clip}%
\pgfsetbuttcap%
\pgfsetroundjoin%
\pgfsetlinewidth{0.050187pt}%
\definecolor{currentstroke}{rgb}{1.000000,1.000000,1.000000}%
\pgfsetstrokecolor{currentstroke}%
\pgfsetdash{}{0pt}%
\pgfpathmoveto{\pgfqpoint{2.204785in}{1.381733in}}%
\pgfpathlineto{\pgfqpoint{2.187432in}{1.398918in}}%
\pgfpathlineto{\pgfqpoint{2.131962in}{1.395719in}}%
\pgfpathlineto{\pgfqpoint{2.045450in}{1.392868in}}%
\pgfpathlineto{\pgfqpoint{1.958418in}{1.394226in}}%
\pgfpathlineto{\pgfqpoint{1.952309in}{1.404878in}}%
\pgfpathlineto{\pgfqpoint{1.954803in}{1.415338in}}%
\pgfpathlineto{\pgfqpoint{1.953619in}{1.433256in}}%
\pgfpathlineto{\pgfqpoint{1.949019in}{1.450612in}}%
\pgfpathlineto{\pgfqpoint{1.949554in}{1.459777in}}%
\pgfpathlineto{\pgfqpoint{1.939777in}{1.470133in}}%
\pgfpathlineto{\pgfqpoint{1.941905in}{1.484542in}}%
\pgfpathlineto{\pgfqpoint{1.926907in}{1.513004in}}%
\pgfpathlineto{\pgfqpoint{1.922438in}{1.537167in}}%
\pgfpathlineto{\pgfqpoint{1.911650in}{1.556821in}}%
\pgfpathlineto{\pgfqpoint{1.921890in}{1.581201in}}%
\pgfpathlineto{\pgfqpoint{1.916790in}{1.592547in}}%
\pgfpathlineto{\pgfqpoint{1.922492in}{1.613058in}}%
\pgfpathlineto{\pgfqpoint{1.976761in}{1.612470in}}%
\pgfpathlineto{\pgfqpoint{2.031790in}{1.612386in}}%
\pgfpathlineto{\pgfqpoint{2.083807in}{1.613150in}}%
\pgfpathlineto{\pgfqpoint{2.140900in}{1.614344in}}%
\pgfpathlineto{\pgfqpoint{2.207878in}{1.616777in}}%
\pgfpathlineto{\pgfqpoint{2.209667in}{1.607053in}}%
\pgfpathlineto{\pgfqpoint{2.217082in}{1.598900in}}%
\pgfpathlineto{\pgfqpoint{2.212387in}{1.590405in}}%
\pgfpathlineto{\pgfqpoint{2.214311in}{1.573824in}}%
\pgfpathlineto{\pgfqpoint{2.218907in}{1.560420in}}%
\pgfpathlineto{\pgfqpoint{2.239741in}{1.553326in}}%
\pgfpathlineto{\pgfqpoint{2.243311in}{1.543836in}}%
\pgfpathlineto{\pgfqpoint{2.254744in}{1.533181in}}%
\pgfpathlineto{\pgfqpoint{2.259407in}{1.522158in}}%
\pgfpathlineto{\pgfqpoint{2.270992in}{1.514764in}}%
\pgfpathlineto{\pgfqpoint{2.272804in}{1.505862in}}%
\pgfpathlineto{\pgfqpoint{2.270549in}{1.492385in}}%
\pgfpathlineto{\pgfqpoint{2.264653in}{1.488346in}}%
\pgfpathlineto{\pgfqpoint{2.262874in}{1.475518in}}%
\pgfpathlineto{\pgfqpoint{2.245930in}{1.465330in}}%
\pgfpathlineto{\pgfqpoint{2.223843in}{1.459748in}}%
\pgfpathlineto{\pgfqpoint{2.221819in}{1.446914in}}%
\pgfpathlineto{\pgfqpoint{2.230340in}{1.437745in}}%
\pgfpathlineto{\pgfqpoint{2.230688in}{1.426216in}}%
\pgfpathlineto{\pgfqpoint{2.223813in}{1.417154in}}%
\pgfpathlineto{\pgfqpoint{2.220205in}{1.403691in}}%
\pgfpathlineto{\pgfqpoint{2.208256in}{1.399234in}}%
\pgfpathlineto{\pgfqpoint{2.209018in}{1.384231in}}%
\pgfpathlineto{\pgfqpoint{2.204785in}{1.381733in}}%
\pgfusepath{stroke}%
\end{pgfscope}%
\begin{pgfscope}%
\pgfpathrectangle{\pgfqpoint{0.100000in}{0.100000in}}{\pgfqpoint{3.420221in}{2.189500in}}%
\pgfusepath{clip}%
\pgfsetbuttcap%
\pgfsetroundjoin%
\pgfsetlinewidth{0.050187pt}%
\definecolor{currentstroke}{rgb}{1.000000,1.000000,1.000000}%
\pgfsetstrokecolor{currentstroke}%
\pgfsetdash{}{0pt}%
\pgfpathmoveto{\pgfqpoint{3.281456in}{1.664360in}}%
\pgfpathlineto{\pgfqpoint{3.265020in}{1.661767in}}%
\pgfpathlineto{\pgfqpoint{3.189524in}{1.644612in}}%
\pgfpathlineto{\pgfqpoint{3.188204in}{1.646611in}}%
\pgfpathlineto{\pgfqpoint{3.189344in}{1.698517in}}%
\pgfpathlineto{\pgfqpoint{3.232641in}{1.707717in}}%
\pgfpathlineto{\pgfqpoint{3.262665in}{1.713833in}}%
\pgfpathlineto{\pgfqpoint{3.295536in}{1.721966in}}%
\pgfpathlineto{\pgfqpoint{3.299078in}{1.731362in}}%
\pgfpathlineto{\pgfqpoint{3.311359in}{1.740729in}}%
\pgfpathlineto{\pgfqpoint{3.317421in}{1.740970in}}%
\pgfpathlineto{\pgfqpoint{3.325390in}{1.727306in}}%
\pgfpathlineto{\pgfqpoint{3.318159in}{1.707359in}}%
\pgfpathlineto{\pgfqpoint{3.317098in}{1.695663in}}%
\pgfpathlineto{\pgfqpoint{3.331734in}{1.696762in}}%
\pgfpathlineto{\pgfqpoint{3.338372in}{1.691151in}}%
\pgfpathlineto{\pgfqpoint{3.353257in}{1.668217in}}%
\pgfpathlineto{\pgfqpoint{3.360656in}{1.665428in}}%
\pgfpathlineto{\pgfqpoint{3.373049in}{1.666462in}}%
\pgfpathlineto{\pgfqpoint{3.381672in}{1.674256in}}%
\pgfpathlineto{\pgfqpoint{3.387405in}{1.667301in}}%
\pgfpathlineto{\pgfqpoint{3.351355in}{1.648277in}}%
\pgfpathlineto{\pgfqpoint{3.350229in}{1.661928in}}%
\pgfpathlineto{\pgfqpoint{3.333853in}{1.640712in}}%
\pgfpathlineto{\pgfqpoint{3.328095in}{1.637057in}}%
\pgfpathlineto{\pgfqpoint{3.320064in}{1.649299in}}%
\pgfpathlineto{\pgfqpoint{3.317865in}{1.650976in}}%
\pgfpathlineto{\pgfqpoint{3.310389in}{1.654936in}}%
\pgfpathlineto{\pgfqpoint{3.303870in}{1.670998in}}%
\pgfpathlineto{\pgfqpoint{3.281456in}{1.664360in}}%
\pgfusepath{stroke}%
\end{pgfscope}%
\begin{pgfscope}%
\pgfpathrectangle{\pgfqpoint{0.100000in}{0.100000in}}{\pgfqpoint{3.420221in}{2.189500in}}%
\pgfusepath{clip}%
\pgfsetbuttcap%
\pgfsetroundjoin%
\pgfsetlinewidth{0.050187pt}%
\definecolor{currentstroke}{rgb}{1.000000,1.000000,1.000000}%
\pgfsetstrokecolor{currentstroke}%
\pgfsetdash{}{0pt}%
\pgfpathmoveto{\pgfqpoint{1.597761in}{1.365516in}}%
\pgfpathlineto{\pgfqpoint{1.603779in}{1.440151in}}%
\pgfpathlineto{\pgfqpoint{1.569691in}{1.442918in}}%
\pgfpathlineto{\pgfqpoint{1.490892in}{1.450563in}}%
\pgfpathlineto{\pgfqpoint{1.498223in}{1.524933in}}%
\pgfpathlineto{\pgfqpoint{1.505565in}{1.599871in}}%
\pgfpathlineto{\pgfqpoint{1.535857in}{1.596600in}}%
\pgfpathlineto{\pgfqpoint{1.613222in}{1.589291in}}%
\pgfpathlineto{\pgfqpoint{1.660043in}{1.585624in}}%
\pgfpathlineto{\pgfqpoint{1.716726in}{1.582375in}}%
\pgfpathlineto{\pgfqpoint{1.810024in}{1.577959in}}%
\pgfpathlineto{\pgfqpoint{1.812075in}{1.573776in}}%
\pgfpathlineto{\pgfqpoint{1.838809in}{1.560550in}}%
\pgfpathlineto{\pgfqpoint{1.847007in}{1.567267in}}%
\pgfpathlineto{\pgfqpoint{1.870626in}{1.566654in}}%
\pgfpathlineto{\pgfqpoint{1.884701in}{1.559281in}}%
\pgfpathlineto{\pgfqpoint{1.907019in}{1.550899in}}%
\pgfpathlineto{\pgfqpoint{1.913273in}{1.538671in}}%
\pgfpathlineto{\pgfqpoint{1.922438in}{1.537167in}}%
\pgfpathlineto{\pgfqpoint{1.926907in}{1.513004in}}%
\pgfpathlineto{\pgfqpoint{1.941905in}{1.484542in}}%
\pgfpathlineto{\pgfqpoint{1.939777in}{1.470133in}}%
\pgfpathlineto{\pgfqpoint{1.949554in}{1.459777in}}%
\pgfpathlineto{\pgfqpoint{1.949019in}{1.450612in}}%
\pgfpathlineto{\pgfqpoint{1.953619in}{1.433256in}}%
\pgfpathlineto{\pgfqpoint{1.954803in}{1.415338in}}%
\pgfpathlineto{\pgfqpoint{1.952309in}{1.404878in}}%
\pgfpathlineto{\pgfqpoint{1.958418in}{1.394226in}}%
\pgfpathlineto{\pgfqpoint{1.966797in}{1.374855in}}%
\pgfpathlineto{\pgfqpoint{1.974817in}{1.366971in}}%
\pgfpathlineto{\pgfqpoint{1.984377in}{1.349888in}}%
\pgfpathlineto{\pgfqpoint{1.957282in}{1.349607in}}%
\pgfpathlineto{\pgfqpoint{1.898698in}{1.350513in}}%
\pgfpathlineto{\pgfqpoint{1.833972in}{1.352496in}}%
\pgfpathlineto{\pgfqpoint{1.768865in}{1.354994in}}%
\pgfpathlineto{\pgfqpoint{1.673137in}{1.360223in}}%
\pgfpathlineto{\pgfqpoint{1.597761in}{1.365516in}}%
\pgfusepath{stroke}%
\end{pgfscope}%
\begin{pgfscope}%
\pgfpathrectangle{\pgfqpoint{0.100000in}{0.100000in}}{\pgfqpoint{3.420221in}{2.189500in}}%
\pgfusepath{clip}%
\pgfsetbuttcap%
\pgfsetroundjoin%
\pgfsetlinewidth{0.050187pt}%
\definecolor{currentstroke}{rgb}{1.000000,1.000000,1.000000}%
\pgfsetstrokecolor{currentstroke}%
\pgfsetdash{}{0pt}%
\pgfpathmoveto{\pgfqpoint{2.844176in}{1.590642in}}%
\pgfpathlineto{\pgfqpoint{2.874151in}{1.619223in}}%
\pgfpathlineto{\pgfqpoint{2.877877in}{1.629382in}}%
\pgfpathlineto{\pgfqpoint{2.886660in}{1.638080in}}%
\pgfpathlineto{\pgfqpoint{2.880069in}{1.650743in}}%
\pgfpathlineto{\pgfqpoint{2.871806in}{1.658150in}}%
\pgfpathlineto{\pgfqpoint{2.869418in}{1.671275in}}%
\pgfpathlineto{\pgfqpoint{2.900174in}{1.684752in}}%
\pgfpathlineto{\pgfqpoint{2.925639in}{1.689015in}}%
\pgfpathlineto{\pgfqpoint{2.939322in}{1.689300in}}%
\pgfpathlineto{\pgfqpoint{2.949745in}{1.684150in}}%
\pgfpathlineto{\pgfqpoint{2.959906in}{1.688749in}}%
\pgfpathlineto{\pgfqpoint{2.984687in}{1.693902in}}%
\pgfpathlineto{\pgfqpoint{2.993250in}{1.700554in}}%
\pgfpathlineto{\pgfqpoint{3.005947in}{1.715279in}}%
\pgfpathlineto{\pgfqpoint{3.017497in}{1.721786in}}%
\pgfpathlineto{\pgfqpoint{3.018312in}{1.728027in}}%
\pgfpathlineto{\pgfqpoint{3.012250in}{1.742265in}}%
\pgfpathlineto{\pgfqpoint{3.016632in}{1.750644in}}%
\pgfpathlineto{\pgfqpoint{3.010731in}{1.759654in}}%
\pgfpathlineto{\pgfqpoint{3.001695in}{1.760282in}}%
\pgfpathlineto{\pgfqpoint{3.024311in}{1.787497in}}%
\pgfpathlineto{\pgfqpoint{3.026997in}{1.797868in}}%
\pgfpathlineto{\pgfqpoint{3.044797in}{1.824319in}}%
\pgfpathlineto{\pgfqpoint{3.061320in}{1.838633in}}%
\pgfpathlineto{\pgfqpoint{3.072666in}{1.844610in}}%
\pgfpathlineto{\pgfqpoint{3.109792in}{1.853022in}}%
\pgfpathlineto{\pgfqpoint{3.144589in}{1.862522in}}%
\pgfpathlineto{\pgfqpoint{3.149198in}{1.847610in}}%
\pgfpathlineto{\pgfqpoint{3.149524in}{1.833783in}}%
\pgfpathlineto{\pgfqpoint{3.157461in}{1.821231in}}%
\pgfpathlineto{\pgfqpoint{3.159907in}{1.808452in}}%
\pgfpathlineto{\pgfqpoint{3.156857in}{1.791113in}}%
\pgfpathlineto{\pgfqpoint{3.166147in}{1.767726in}}%
\pgfpathlineto{\pgfqpoint{3.176738in}{1.754721in}}%
\pgfpathlineto{\pgfqpoint{3.184930in}{1.712451in}}%
\pgfpathlineto{\pgfqpoint{3.189344in}{1.698517in}}%
\pgfpathlineto{\pgfqpoint{3.188204in}{1.646611in}}%
\pgfpathlineto{\pgfqpoint{3.189524in}{1.644612in}}%
\pgfpathlineto{\pgfqpoint{3.199167in}{1.588786in}}%
\pgfpathlineto{\pgfqpoint{3.204575in}{1.583699in}}%
\pgfpathlineto{\pgfqpoint{3.192943in}{1.572397in}}%
\pgfpathlineto{\pgfqpoint{3.198682in}{1.565910in}}%
\pgfpathlineto{\pgfqpoint{3.193637in}{1.556098in}}%
\pgfpathlineto{\pgfqpoint{3.193680in}{1.551921in}}%
\pgfpathlineto{\pgfqpoint{3.187373in}{1.548152in}}%
\pgfpathlineto{\pgfqpoint{3.184302in}{1.539820in}}%
\pgfpathlineto{\pgfqpoint{3.185276in}{1.562731in}}%
\pgfpathlineto{\pgfqpoint{3.165727in}{1.567775in}}%
\pgfpathlineto{\pgfqpoint{3.135155in}{1.578203in}}%
\pgfpathlineto{\pgfqpoint{3.130831in}{1.583340in}}%
\pgfpathlineto{\pgfqpoint{3.118056in}{1.584571in}}%
\pgfpathlineto{\pgfqpoint{3.110707in}{1.592216in}}%
\pgfpathlineto{\pgfqpoint{3.106742in}{1.607443in}}%
\pgfpathlineto{\pgfqpoint{3.096324in}{1.609337in}}%
\pgfpathlineto{\pgfqpoint{3.089357in}{1.617324in}}%
\pgfpathlineto{\pgfqpoint{3.000664in}{1.599443in}}%
\pgfpathlineto{\pgfqpoint{2.958276in}{1.590677in}}%
\pgfpathlineto{\pgfqpoint{2.893917in}{1.578930in}}%
\pgfpathlineto{\pgfqpoint{2.847571in}{1.571112in}}%
\pgfpathlineto{\pgfqpoint{2.844176in}{1.590642in}}%
\pgfusepath{stroke}%
\end{pgfscope}%
\begin{pgfscope}%
\pgfpathrectangle{\pgfqpoint{0.100000in}{0.100000in}}{\pgfqpoint{3.420221in}{2.189500in}}%
\pgfusepath{clip}%
\pgfsetbuttcap%
\pgfsetroundjoin%
\pgfsetlinewidth{0.050187pt}%
\definecolor{currentstroke}{rgb}{1.000000,1.000000,1.000000}%
\pgfsetstrokecolor{currentstroke}%
\pgfsetdash{}{0pt}%
\pgfpathmoveto{\pgfqpoint{3.200243in}{1.535167in}}%
\pgfpathlineto{\pgfqpoint{3.186527in}{1.530894in}}%
\pgfpathlineto{\pgfqpoint{3.184219in}{1.534822in}}%
\pgfpathlineto{\pgfqpoint{3.188622in}{1.548009in}}%
\pgfpathlineto{\pgfqpoint{3.196776in}{1.549308in}}%
\pgfpathlineto{\pgfqpoint{3.196046in}{1.553460in}}%
\pgfpathlineto{\pgfqpoint{3.203367in}{1.559691in}}%
\pgfpathlineto{\pgfqpoint{3.224534in}{1.564655in}}%
\pgfpathlineto{\pgfqpoint{3.233957in}{1.572171in}}%
\pgfpathlineto{\pgfqpoint{3.255125in}{1.578433in}}%
\pgfpathlineto{\pgfqpoint{3.264779in}{1.576142in}}%
\pgfpathlineto{\pgfqpoint{3.272901in}{1.586239in}}%
\pgfpathlineto{\pgfqpoint{3.285176in}{1.587573in}}%
\pgfpathlineto{\pgfqpoint{3.264234in}{1.567878in}}%
\pgfpathlineto{\pgfqpoint{3.200243in}{1.535167in}}%
\pgfusepath{stroke}%
\end{pgfscope}%
\begin{pgfscope}%
\pgfpathrectangle{\pgfqpoint{0.100000in}{0.100000in}}{\pgfqpoint{3.420221in}{2.189500in}}%
\pgfusepath{clip}%
\pgfsetbuttcap%
\pgfsetroundjoin%
\pgfsetlinewidth{0.050187pt}%
\definecolor{currentstroke}{rgb}{1.000000,1.000000,1.000000}%
\pgfsetstrokecolor{currentstroke}%
\pgfsetdash{}{0pt}%
\pgfpathmoveto{\pgfqpoint{2.892074in}{1.405120in}}%
\pgfpathlineto{\pgfqpoint{2.832758in}{1.395312in}}%
\pgfpathlineto{\pgfqpoint{2.821998in}{1.463100in}}%
\pgfpathlineto{\pgfqpoint{2.806022in}{1.563023in}}%
\pgfpathlineto{\pgfqpoint{2.844176in}{1.590642in}}%
\pgfpathlineto{\pgfqpoint{2.847571in}{1.571112in}}%
\pgfpathlineto{\pgfqpoint{2.893917in}{1.578930in}}%
\pgfpathlineto{\pgfqpoint{2.958276in}{1.590677in}}%
\pgfpathlineto{\pgfqpoint{3.000664in}{1.599443in}}%
\pgfpathlineto{\pgfqpoint{3.089357in}{1.617324in}}%
\pgfpathlineto{\pgfqpoint{3.096324in}{1.609337in}}%
\pgfpathlineto{\pgfqpoint{3.106742in}{1.607443in}}%
\pgfpathlineto{\pgfqpoint{3.110707in}{1.592216in}}%
\pgfpathlineto{\pgfqpoint{3.118056in}{1.584571in}}%
\pgfpathlineto{\pgfqpoint{3.130831in}{1.583340in}}%
\pgfpathlineto{\pgfqpoint{3.135155in}{1.578203in}}%
\pgfpathlineto{\pgfqpoint{3.130765in}{1.574240in}}%
\pgfpathlineto{\pgfqpoint{3.126810in}{1.560218in}}%
\pgfpathlineto{\pgfqpoint{3.117543in}{1.544463in}}%
\pgfpathlineto{\pgfqpoint{3.123764in}{1.537611in}}%
\pgfpathlineto{\pgfqpoint{3.118673in}{1.530270in}}%
\pgfpathlineto{\pgfqpoint{3.121545in}{1.514171in}}%
\pgfpathlineto{\pgfqpoint{3.130712in}{1.505720in}}%
\pgfpathlineto{\pgfqpoint{3.152464in}{1.491991in}}%
\pgfpathlineto{\pgfqpoint{3.134947in}{1.472619in}}%
\pgfpathlineto{\pgfqpoint{3.134701in}{1.465266in}}%
\pgfpathlineto{\pgfqpoint{3.120440in}{1.455784in}}%
\pgfpathlineto{\pgfqpoint{3.104672in}{1.454004in}}%
\pgfpathlineto{\pgfqpoint{3.100771in}{1.445763in}}%
\pgfpathlineto{\pgfqpoint{3.056910in}{1.436272in}}%
\pgfpathlineto{\pgfqpoint{3.005730in}{1.425964in}}%
\pgfpathlineto{\pgfqpoint{2.970550in}{1.419665in}}%
\pgfpathlineto{\pgfqpoint{2.892074in}{1.405120in}}%
\pgfusepath{stroke}%
\end{pgfscope}%
\begin{pgfscope}%
\pgfpathrectangle{\pgfqpoint{0.100000in}{0.100000in}}{\pgfqpoint{3.420221in}{2.189500in}}%
\pgfusepath{clip}%
\pgfsetbuttcap%
\pgfsetroundjoin%
\pgfsetlinewidth{0.050187pt}%
\definecolor{currentstroke}{rgb}{1.000000,1.000000,1.000000}%
\pgfsetstrokecolor{currentstroke}%
\pgfsetdash{}{0pt}%
\pgfpathmoveto{\pgfqpoint{3.189524in}{1.644612in}}%
\pgfpathlineto{\pgfqpoint{3.265020in}{1.661767in}}%
\pgfpathlineto{\pgfqpoint{3.281456in}{1.664360in}}%
\pgfpathlineto{\pgfqpoint{3.292334in}{1.621590in}}%
\pgfpathlineto{\pgfqpoint{3.290612in}{1.613931in}}%
\pgfpathlineto{\pgfqpoint{3.255668in}{1.600402in}}%
\pgfpathlineto{\pgfqpoint{3.234807in}{1.595674in}}%
\pgfpathlineto{\pgfqpoint{3.225942in}{1.585067in}}%
\pgfpathlineto{\pgfqpoint{3.198682in}{1.565910in}}%
\pgfpathlineto{\pgfqpoint{3.192943in}{1.572397in}}%
\pgfpathlineto{\pgfqpoint{3.204575in}{1.583699in}}%
\pgfpathlineto{\pgfqpoint{3.199167in}{1.588786in}}%
\pgfpathlineto{\pgfqpoint{3.189524in}{1.644612in}}%
\pgfusepath{stroke}%
\end{pgfscope}%
\begin{pgfscope}%
\pgfpathrectangle{\pgfqpoint{0.100000in}{0.100000in}}{\pgfqpoint{3.420221in}{2.189500in}}%
\pgfusepath{clip}%
\pgfsetbuttcap%
\pgfsetroundjoin%
\pgfsetlinewidth{0.050187pt}%
\definecolor{currentstroke}{rgb}{1.000000,1.000000,1.000000}%
\pgfsetstrokecolor{currentstroke}%
\pgfsetdash{}{0pt}%
\pgfpathmoveto{\pgfqpoint{3.281456in}{1.664360in}}%
\pgfpathlineto{\pgfqpoint{3.303870in}{1.670998in}}%
\pgfpathlineto{\pgfqpoint{3.310389in}{1.654936in}}%
\pgfpathlineto{\pgfqpoint{3.317865in}{1.650976in}}%
\pgfpathlineto{\pgfqpoint{3.308640in}{1.644211in}}%
\pgfpathlineto{\pgfqpoint{3.309804in}{1.624340in}}%
\pgfpathlineto{\pgfqpoint{3.290612in}{1.613931in}}%
\pgfpathlineto{\pgfqpoint{3.292334in}{1.621590in}}%
\pgfpathlineto{\pgfqpoint{3.281456in}{1.664360in}}%
\pgfusepath{stroke}%
\end{pgfscope}%
\begin{pgfscope}%
\pgfpathrectangle{\pgfqpoint{0.100000in}{0.100000in}}{\pgfqpoint{3.420221in}{2.189500in}}%
\pgfusepath{clip}%
\pgfsetbuttcap%
\pgfsetroundjoin%
\pgfsetlinewidth{0.050187pt}%
\definecolor{currentstroke}{rgb}{1.000000,1.000000,1.000000}%
\pgfsetstrokecolor{currentstroke}%
\pgfsetdash{}{0pt}%
\pgfpathmoveto{\pgfqpoint{3.117913in}{1.448780in}}%
\pgfpathlineto{\pgfqpoint{3.120440in}{1.455784in}}%
\pgfpathlineto{\pgfqpoint{3.134701in}{1.465266in}}%
\pgfpathlineto{\pgfqpoint{3.134947in}{1.472619in}}%
\pgfpathlineto{\pgfqpoint{3.152464in}{1.491991in}}%
\pgfpathlineto{\pgfqpoint{3.130712in}{1.505720in}}%
\pgfpathlineto{\pgfqpoint{3.121545in}{1.514171in}}%
\pgfpathlineto{\pgfqpoint{3.118673in}{1.530270in}}%
\pgfpathlineto{\pgfqpoint{3.123764in}{1.537611in}}%
\pgfpathlineto{\pgfqpoint{3.117543in}{1.544463in}}%
\pgfpathlineto{\pgfqpoint{3.126810in}{1.560218in}}%
\pgfpathlineto{\pgfqpoint{3.130765in}{1.574240in}}%
\pgfpathlineto{\pgfqpoint{3.135155in}{1.578203in}}%
\pgfpathlineto{\pgfqpoint{3.165727in}{1.567775in}}%
\pgfpathlineto{\pgfqpoint{3.185276in}{1.562731in}}%
\pgfpathlineto{\pgfqpoint{3.184302in}{1.539820in}}%
\pgfpathlineto{\pgfqpoint{3.172426in}{1.522446in}}%
\pgfpathlineto{\pgfqpoint{3.182212in}{1.519890in}}%
\pgfpathlineto{\pgfqpoint{3.192371in}{1.512435in}}%
\pgfpathlineto{\pgfqpoint{3.192971in}{1.492042in}}%
\pgfpathlineto{\pgfqpoint{3.189896in}{1.477606in}}%
\pgfpathlineto{\pgfqpoint{3.191942in}{1.465748in}}%
\pgfpathlineto{\pgfqpoint{3.174281in}{1.425734in}}%
\pgfpathlineto{\pgfqpoint{3.167934in}{1.407050in}}%
\pgfpathlineto{\pgfqpoint{3.159072in}{1.416132in}}%
\pgfpathlineto{\pgfqpoint{3.147320in}{1.414585in}}%
\pgfpathlineto{\pgfqpoint{3.117908in}{1.431578in}}%
\pgfpathlineto{\pgfqpoint{3.114898in}{1.440677in}}%
\pgfpathlineto{\pgfqpoint{3.117913in}{1.448780in}}%
\pgfusepath{stroke}%
\end{pgfscope}%
\begin{pgfscope}%
\pgfpathrectangle{\pgfqpoint{0.100000in}{0.100000in}}{\pgfqpoint{3.420221in}{2.189500in}}%
\pgfusepath{clip}%
\pgfsetbuttcap%
\pgfsetroundjoin%
\pgfsetlinewidth{0.050187pt}%
\definecolor{currentstroke}{rgb}{1.000000,1.000000,1.000000}%
\pgfsetstrokecolor{currentstroke}%
\pgfsetdash{}{0pt}%
\pgfpathmoveto{\pgfqpoint{2.414023in}{1.201296in}}%
\pgfpathlineto{\pgfqpoint{2.412648in}{1.219858in}}%
\pgfpathlineto{\pgfqpoint{2.417993in}{1.229068in}}%
\pgfpathlineto{\pgfqpoint{2.414470in}{1.233580in}}%
\pgfpathlineto{\pgfqpoint{2.427315in}{1.248355in}}%
\pgfpathlineto{\pgfqpoint{2.426579in}{1.251916in}}%
\pgfpathlineto{\pgfqpoint{2.439060in}{1.277692in}}%
\pgfpathlineto{\pgfqpoint{2.436537in}{1.290124in}}%
\pgfpathlineto{\pgfqpoint{2.427487in}{1.303139in}}%
\pgfpathlineto{\pgfqpoint{2.433769in}{1.318973in}}%
\pgfpathlineto{\pgfqpoint{2.425626in}{1.423173in}}%
\pgfpathlineto{\pgfqpoint{2.419763in}{1.496273in}}%
\pgfpathlineto{\pgfqpoint{2.427863in}{1.490216in}}%
\pgfpathlineto{\pgfqpoint{2.436900in}{1.490380in}}%
\pgfpathlineto{\pgfqpoint{2.458244in}{1.502744in}}%
\pgfpathlineto{\pgfqpoint{2.523712in}{1.508853in}}%
\pgfpathlineto{\pgfqpoint{2.572169in}{1.513960in}}%
\pgfpathlineto{\pgfqpoint{2.572597in}{1.509218in}}%
\pgfpathlineto{\pgfqpoint{2.583659in}{1.409045in}}%
\pgfpathlineto{\pgfqpoint{2.593180in}{1.315837in}}%
\pgfpathlineto{\pgfqpoint{2.589076in}{1.311461in}}%
\pgfpathlineto{\pgfqpoint{2.595380in}{1.300600in}}%
\pgfpathlineto{\pgfqpoint{2.595348in}{1.292772in}}%
\pgfpathlineto{\pgfqpoint{2.586352in}{1.290806in}}%
\pgfpathlineto{\pgfqpoint{2.576285in}{1.283260in}}%
\pgfpathlineto{\pgfqpoint{2.569470in}{1.286231in}}%
\pgfpathlineto{\pgfqpoint{2.559269in}{1.281400in}}%
\pgfpathlineto{\pgfqpoint{2.562425in}{1.271699in}}%
\pgfpathlineto{\pgfqpoint{2.551940in}{1.261952in}}%
\pgfpathlineto{\pgfqpoint{2.549045in}{1.250678in}}%
\pgfpathlineto{\pgfqpoint{2.541836in}{1.248828in}}%
\pgfpathlineto{\pgfqpoint{2.536456in}{1.240289in}}%
\pgfpathlineto{\pgfqpoint{2.537160in}{1.231691in}}%
\pgfpathlineto{\pgfqpoint{2.530830in}{1.225644in}}%
\pgfpathlineto{\pgfqpoint{2.521306in}{1.226577in}}%
\pgfpathlineto{\pgfqpoint{2.514066in}{1.235855in}}%
\pgfpathlineto{\pgfqpoint{2.501799in}{1.226921in}}%
\pgfpathlineto{\pgfqpoint{2.502660in}{1.219114in}}%
\pgfpathlineto{\pgfqpoint{2.489017in}{1.214552in}}%
\pgfpathlineto{\pgfqpoint{2.485589in}{1.220293in}}%
\pgfpathlineto{\pgfqpoint{2.472307in}{1.213784in}}%
\pgfpathlineto{\pgfqpoint{2.467454in}{1.204464in}}%
\pgfpathlineto{\pgfqpoint{2.451458in}{1.214020in}}%
\pgfpathlineto{\pgfqpoint{2.426106in}{1.207689in}}%
\pgfpathlineto{\pgfqpoint{2.414023in}{1.201296in}}%
\pgfusepath{stroke}%
\end{pgfscope}%
\begin{pgfscope}%
\pgfpathrectangle{\pgfqpoint{0.100000in}{0.100000in}}{\pgfqpoint{3.420221in}{2.189500in}}%
\pgfusepath{clip}%
\pgfsetbuttcap%
\pgfsetroundjoin%
\pgfsetlinewidth{0.050187pt}%
\definecolor{currentstroke}{rgb}{1.000000,1.000000,1.000000}%
\pgfsetstrokecolor{currentstroke}%
\pgfsetdash{}{0pt}%
\pgfpathmoveto{\pgfqpoint{0.626067in}{1.689441in}}%
\pgfpathlineto{\pgfqpoint{0.662728in}{1.679239in}}%
\pgfpathlineto{\pgfqpoint{0.723499in}{1.663569in}}%
\pgfpathlineto{\pgfqpoint{0.786494in}{1.647428in}}%
\pgfpathlineto{\pgfqpoint{0.844556in}{1.633425in}}%
\pgfpathlineto{\pgfqpoint{0.894726in}{1.622066in}}%
\pgfpathlineto{\pgfqpoint{0.948402in}{1.610263in}}%
\pgfpathlineto{\pgfqpoint{0.932894in}{1.537076in}}%
\pgfpathlineto{\pgfqpoint{0.919078in}{1.472082in}}%
\pgfpathlineto{\pgfqpoint{0.896413in}{1.367214in}}%
\pgfpathlineto{\pgfqpoint{0.879397in}{1.288061in}}%
\pgfpathlineto{\pgfqpoint{0.870204in}{1.243890in}}%
\pgfpathlineto{\pgfqpoint{0.858416in}{1.186560in}}%
\pgfpathlineto{\pgfqpoint{0.850261in}{1.174930in}}%
\pgfpathlineto{\pgfqpoint{0.843697in}{1.174541in}}%
\pgfpathlineto{\pgfqpoint{0.839015in}{1.184694in}}%
\pgfpathlineto{\pgfqpoint{0.828265in}{1.188381in}}%
\pgfpathlineto{\pgfqpoint{0.816681in}{1.187089in}}%
\pgfpathlineto{\pgfqpoint{0.813393in}{1.178784in}}%
\pgfpathlineto{\pgfqpoint{0.813050in}{1.149947in}}%
\pgfpathlineto{\pgfqpoint{0.809581in}{1.143363in}}%
\pgfpathlineto{\pgfqpoint{0.811564in}{1.120212in}}%
\pgfpathlineto{\pgfqpoint{0.804320in}{1.104789in}}%
\pgfpathlineto{\pgfqpoint{0.745577in}{1.195244in}}%
\pgfpathlineto{\pgfqpoint{0.687725in}{1.283141in}}%
\pgfpathlineto{\pgfqpoint{0.657611in}{1.329232in}}%
\pgfpathlineto{\pgfqpoint{0.632490in}{1.369015in}}%
\pgfpathlineto{\pgfqpoint{0.600836in}{1.418227in}}%
\pgfpathlineto{\pgfqpoint{0.565005in}{1.473429in}}%
\pgfpathlineto{\pgfqpoint{0.579736in}{1.525826in}}%
\pgfpathlineto{\pgfqpoint{0.609382in}{1.630908in}}%
\pgfpathlineto{\pgfqpoint{0.626067in}{1.689441in}}%
\pgfusepath{stroke}%
\end{pgfscope}%
\begin{pgfscope}%
\pgfpathrectangle{\pgfqpoint{0.100000in}{0.100000in}}{\pgfqpoint{3.420221in}{2.189500in}}%
\pgfusepath{clip}%
\pgfsetbuttcap%
\pgfsetroundjoin%
\pgfsetlinewidth{0.050187pt}%
\definecolor{currentstroke}{rgb}{1.000000,1.000000,1.000000}%
\pgfsetstrokecolor{currentstroke}%
\pgfsetdash{}{0pt}%
\pgfpathmoveto{\pgfqpoint{0.948402in}{1.610263in}}%
\pgfpathlineto{\pgfqpoint{1.005787in}{1.598849in}}%
\pgfpathlineto{\pgfqpoint{1.112024in}{1.578340in}}%
\pgfpathlineto{\pgfqpoint{1.098726in}{1.504573in}}%
\pgfpathlineto{\pgfqpoint{1.157213in}{1.494559in}}%
\pgfpathlineto{\pgfqpoint{1.210497in}{1.486032in}}%
\pgfpathlineto{\pgfqpoint{1.201239in}{1.427713in}}%
\pgfpathlineto{\pgfqpoint{1.191428in}{1.364823in}}%
\pgfpathlineto{\pgfqpoint{1.178292in}{1.282248in}}%
\pgfpathlineto{\pgfqpoint{1.177952in}{1.275326in}}%
\pgfpathlineto{\pgfqpoint{1.164318in}{1.189748in}}%
\pgfpathlineto{\pgfqpoint{1.108190in}{1.198448in}}%
\pgfpathlineto{\pgfqpoint{1.079589in}{1.204191in}}%
\pgfpathlineto{\pgfqpoint{0.976211in}{1.222364in}}%
\pgfpathlineto{\pgfqpoint{0.937281in}{1.230041in}}%
\pgfpathlineto{\pgfqpoint{0.870204in}{1.243890in}}%
\pgfpathlineto{\pgfqpoint{0.879397in}{1.288061in}}%
\pgfpathlineto{\pgfqpoint{0.896413in}{1.367214in}}%
\pgfpathlineto{\pgfqpoint{0.919078in}{1.472082in}}%
\pgfpathlineto{\pgfqpoint{0.932894in}{1.537076in}}%
\pgfpathlineto{\pgfqpoint{0.948402in}{1.610263in}}%
\pgfusepath{stroke}%
\end{pgfscope}%
\begin{pgfscope}%
\pgfpathrectangle{\pgfqpoint{0.100000in}{0.100000in}}{\pgfqpoint{3.420221in}{2.189500in}}%
\pgfusepath{clip}%
\pgfsetbuttcap%
\pgfsetroundjoin%
\pgfsetlinewidth{0.050187pt}%
\definecolor{currentstroke}{rgb}{1.000000,1.000000,1.000000}%
\pgfsetstrokecolor{currentstroke}%
\pgfsetdash{}{0pt}%
\pgfpathmoveto{\pgfqpoint{0.401756in}{1.759050in}}%
\pgfpathlineto{\pgfqpoint{0.422146in}{1.751938in}}%
\pgfpathlineto{\pgfqpoint{0.503824in}{1.726509in}}%
\pgfpathlineto{\pgfqpoint{0.578961in}{1.702859in}}%
\pgfpathlineto{\pgfqpoint{0.626067in}{1.689441in}}%
\pgfpathlineto{\pgfqpoint{0.609382in}{1.630908in}}%
\pgfpathlineto{\pgfqpoint{0.579736in}{1.525826in}}%
\pgfpathlineto{\pgfqpoint{0.565005in}{1.473429in}}%
\pgfpathlineto{\pgfqpoint{0.600836in}{1.418227in}}%
\pgfpathlineto{\pgfqpoint{0.632490in}{1.369015in}}%
\pgfpathlineto{\pgfqpoint{0.657611in}{1.329232in}}%
\pgfpathlineto{\pgfqpoint{0.687725in}{1.283141in}}%
\pgfpathlineto{\pgfqpoint{0.745577in}{1.195244in}}%
\pgfpathlineto{\pgfqpoint{0.804320in}{1.104789in}}%
\pgfpathlineto{\pgfqpoint{0.801962in}{1.095819in}}%
\pgfpathlineto{\pgfqpoint{0.809047in}{1.081533in}}%
\pgfpathlineto{\pgfqpoint{0.810448in}{1.061990in}}%
\pgfpathlineto{\pgfqpoint{0.820739in}{1.052510in}}%
\pgfpathlineto{\pgfqpoint{0.821146in}{1.044872in}}%
\pgfpathlineto{\pgfqpoint{0.802721in}{1.036203in}}%
\pgfpathlineto{\pgfqpoint{0.793942in}{1.027523in}}%
\pgfpathlineto{\pgfqpoint{0.791204in}{1.008357in}}%
\pgfpathlineto{\pgfqpoint{0.786761in}{0.997904in}}%
\pgfpathlineto{\pgfqpoint{0.777399in}{0.989093in}}%
\pgfpathlineto{\pgfqpoint{0.768088in}{0.966183in}}%
\pgfpathlineto{\pgfqpoint{0.781180in}{0.954249in}}%
\pgfpathlineto{\pgfqpoint{0.779490in}{0.944411in}}%
\pgfpathlineto{\pgfqpoint{0.768755in}{0.937565in}}%
\pgfpathlineto{\pgfqpoint{0.761353in}{0.938773in}}%
\pgfpathlineto{\pgfqpoint{0.673986in}{0.950897in}}%
\pgfpathlineto{\pgfqpoint{0.609457in}{0.960052in}}%
\pgfpathlineto{\pgfqpoint{0.612279in}{0.970470in}}%
\pgfpathlineto{\pgfqpoint{0.608078in}{0.987767in}}%
\pgfpathlineto{\pgfqpoint{0.607654in}{1.005214in}}%
\pgfpathlineto{\pgfqpoint{0.604909in}{1.015423in}}%
\pgfpathlineto{\pgfqpoint{0.596452in}{1.030021in}}%
\pgfpathlineto{\pgfqpoint{0.572127in}{1.063701in}}%
\pgfpathlineto{\pgfqpoint{0.564160in}{1.067827in}}%
\pgfpathlineto{\pgfqpoint{0.553895in}{1.067762in}}%
\pgfpathlineto{\pgfqpoint{0.556240in}{1.078387in}}%
\pgfpathlineto{\pgfqpoint{0.551379in}{1.091673in}}%
\pgfpathlineto{\pgfqpoint{0.527504in}{1.098277in}}%
\pgfpathlineto{\pgfqpoint{0.512950in}{1.110504in}}%
\pgfpathlineto{\pgfqpoint{0.511743in}{1.117986in}}%
\pgfpathlineto{\pgfqpoint{0.494985in}{1.136509in}}%
\pgfpathlineto{\pgfqpoint{0.479034in}{1.140054in}}%
\pgfpathlineto{\pgfqpoint{0.464254in}{1.149467in}}%
\pgfpathlineto{\pgfqpoint{0.444816in}{1.152724in}}%
\pgfpathlineto{\pgfqpoint{0.436511in}{1.165279in}}%
\pgfpathlineto{\pgfqpoint{0.444398in}{1.185152in}}%
\pgfpathlineto{\pgfqpoint{0.441986in}{1.189620in}}%
\pgfpathlineto{\pgfqpoint{0.448554in}{1.206187in}}%
\pgfpathlineto{\pgfqpoint{0.436870in}{1.215003in}}%
\pgfpathlineto{\pgfqpoint{0.440661in}{1.230986in}}%
\pgfpathlineto{\pgfqpoint{0.434420in}{1.235072in}}%
\pgfpathlineto{\pgfqpoint{0.429032in}{1.250179in}}%
\pgfpathlineto{\pgfqpoint{0.422529in}{1.254786in}}%
\pgfpathlineto{\pgfqpoint{0.422040in}{1.265698in}}%
\pgfpathlineto{\pgfqpoint{0.416945in}{1.273393in}}%
\pgfpathlineto{\pgfqpoint{0.409269in}{1.299307in}}%
\pgfpathlineto{\pgfqpoint{0.400869in}{1.311719in}}%
\pgfpathlineto{\pgfqpoint{0.402695in}{1.332790in}}%
\pgfpathlineto{\pgfqpoint{0.412572in}{1.334910in}}%
\pgfpathlineto{\pgfqpoint{0.419004in}{1.346339in}}%
\pgfpathlineto{\pgfqpoint{0.415118in}{1.358734in}}%
\pgfpathlineto{\pgfqpoint{0.404612in}{1.360803in}}%
\pgfpathlineto{\pgfqpoint{0.399371in}{1.366601in}}%
\pgfpathlineto{\pgfqpoint{0.390883in}{1.387926in}}%
\pgfpathlineto{\pgfqpoint{0.394850in}{1.395585in}}%
\pgfpathlineto{\pgfqpoint{0.392036in}{1.409845in}}%
\pgfpathlineto{\pgfqpoint{0.398267in}{1.428315in}}%
\pgfpathlineto{\pgfqpoint{0.405566in}{1.421513in}}%
\pgfpathlineto{\pgfqpoint{0.402253in}{1.413483in}}%
\pgfpathlineto{\pgfqpoint{0.414902in}{1.400754in}}%
\pgfpathlineto{\pgfqpoint{0.414095in}{1.419656in}}%
\pgfpathlineto{\pgfqpoint{0.408675in}{1.424726in}}%
\pgfpathlineto{\pgfqpoint{0.414870in}{1.441346in}}%
\pgfpathlineto{\pgfqpoint{0.438303in}{1.438699in}}%
\pgfpathlineto{\pgfqpoint{0.435149in}{1.444879in}}%
\pgfpathlineto{\pgfqpoint{0.419684in}{1.444271in}}%
\pgfpathlineto{\pgfqpoint{0.412327in}{1.453632in}}%
\pgfpathlineto{\pgfqpoint{0.403066in}{1.445319in}}%
\pgfpathlineto{\pgfqpoint{0.398152in}{1.431425in}}%
\pgfpathlineto{\pgfqpoint{0.385037in}{1.450122in}}%
\pgfpathlineto{\pgfqpoint{0.379985in}{1.453525in}}%
\pgfpathlineto{\pgfqpoint{0.381910in}{1.473874in}}%
\pgfpathlineto{\pgfqpoint{0.377912in}{1.485873in}}%
\pgfpathlineto{\pgfqpoint{0.370665in}{1.497146in}}%
\pgfpathlineto{\pgfqpoint{0.355825in}{1.531689in}}%
\pgfpathlineto{\pgfqpoint{0.360677in}{1.539327in}}%
\pgfpathlineto{\pgfqpoint{0.360621in}{1.563467in}}%
\pgfpathlineto{\pgfqpoint{0.368630in}{1.576931in}}%
\pgfpathlineto{\pgfqpoint{0.370466in}{1.597973in}}%
\pgfpathlineto{\pgfqpoint{0.362914in}{1.622083in}}%
\pgfpathlineto{\pgfqpoint{0.352895in}{1.637429in}}%
\pgfpathlineto{\pgfqpoint{0.354682in}{1.651277in}}%
\pgfpathlineto{\pgfqpoint{0.382818in}{1.684791in}}%
\pgfpathlineto{\pgfqpoint{0.384216in}{1.696224in}}%
\pgfpathlineto{\pgfqpoint{0.396894in}{1.718032in}}%
\pgfpathlineto{\pgfqpoint{0.398653in}{1.738704in}}%
\pgfpathlineto{\pgfqpoint{0.394577in}{1.743978in}}%
\pgfpathlineto{\pgfqpoint{0.401756in}{1.759050in}}%
\pgfusepath{stroke}%
\end{pgfscope}%
\begin{pgfscope}%
\pgfpathrectangle{\pgfqpoint{0.100000in}{0.100000in}}{\pgfqpoint{3.420221in}{2.189500in}}%
\pgfusepath{clip}%
\pgfsetbuttcap%
\pgfsetroundjoin%
\pgfsetlinewidth{0.050187pt}%
\definecolor{currentstroke}{rgb}{1.000000,1.000000,1.000000}%
\pgfsetstrokecolor{currentstroke}%
\pgfsetdash{}{0pt}%
\pgfpathmoveto{\pgfqpoint{2.593180in}{1.315837in}}%
\pgfpathlineto{\pgfqpoint{2.583659in}{1.409045in}}%
\pgfpathlineto{\pgfqpoint{2.572597in}{1.509218in}}%
\pgfpathlineto{\pgfqpoint{2.645038in}{1.520007in}}%
\pgfpathlineto{\pgfqpoint{2.664264in}{1.515002in}}%
\pgfpathlineto{\pgfqpoint{2.673485in}{1.509571in}}%
\pgfpathlineto{\pgfqpoint{2.685039in}{1.511077in}}%
\pgfpathlineto{\pgfqpoint{2.700274in}{1.502084in}}%
\pgfpathlineto{\pgfqpoint{2.728651in}{1.515469in}}%
\pgfpathlineto{\pgfqpoint{2.744311in}{1.515916in}}%
\pgfpathlineto{\pgfqpoint{2.762598in}{1.536318in}}%
\pgfpathlineto{\pgfqpoint{2.781204in}{1.548728in}}%
\pgfpathlineto{\pgfqpoint{2.806022in}{1.563023in}}%
\pgfpathlineto{\pgfqpoint{2.821998in}{1.463100in}}%
\pgfpathlineto{\pgfqpoint{2.814575in}{1.456683in}}%
\pgfpathlineto{\pgfqpoint{2.819368in}{1.450788in}}%
\pgfpathlineto{\pgfqpoint{2.821277in}{1.437867in}}%
\pgfpathlineto{\pgfqpoint{2.817558in}{1.425730in}}%
\pgfpathlineto{\pgfqpoint{2.817067in}{1.407990in}}%
\pgfpathlineto{\pgfqpoint{2.813574in}{1.384901in}}%
\pgfpathlineto{\pgfqpoint{2.795906in}{1.364308in}}%
\pgfpathlineto{\pgfqpoint{2.788468in}{1.359937in}}%
\pgfpathlineto{\pgfqpoint{2.782684in}{1.363700in}}%
\pgfpathlineto{\pgfqpoint{2.768369in}{1.344073in}}%
\pgfpathlineto{\pgfqpoint{2.769710in}{1.325174in}}%
\pgfpathlineto{\pgfqpoint{2.753768in}{1.329719in}}%
\pgfpathlineto{\pgfqpoint{2.746219in}{1.310844in}}%
\pgfpathlineto{\pgfqpoint{2.750035in}{1.297470in}}%
\pgfpathlineto{\pgfqpoint{2.744062in}{1.295494in}}%
\pgfpathlineto{\pgfqpoint{2.743226in}{1.284926in}}%
\pgfpathlineto{\pgfqpoint{2.728564in}{1.280663in}}%
\pgfpathlineto{\pgfqpoint{2.720945in}{1.289196in}}%
\pgfpathlineto{\pgfqpoint{2.711142in}{1.292504in}}%
\pgfpathlineto{\pgfqpoint{2.708838in}{1.301156in}}%
\pgfpathlineto{\pgfqpoint{2.699974in}{1.299634in}}%
\pgfpathlineto{\pgfqpoint{2.694167in}{1.291677in}}%
\pgfpathlineto{\pgfqpoint{2.685859in}{1.288878in}}%
\pgfpathlineto{\pgfqpoint{2.671185in}{1.294510in}}%
\pgfpathlineto{\pgfqpoint{2.663068in}{1.287809in}}%
\pgfpathlineto{\pgfqpoint{2.651533in}{1.295737in}}%
\pgfpathlineto{\pgfqpoint{2.629404in}{1.298170in}}%
\pgfpathlineto{\pgfqpoint{2.622729in}{1.312571in}}%
\pgfpathlineto{\pgfqpoint{2.614272in}{1.318952in}}%
\pgfpathlineto{\pgfqpoint{2.606073in}{1.314867in}}%
\pgfpathlineto{\pgfqpoint{2.593180in}{1.315837in}}%
\pgfusepath{stroke}%
\end{pgfscope}%
\begin{pgfscope}%
\pgfpathrectangle{\pgfqpoint{0.100000in}{0.100000in}}{\pgfqpoint{3.420221in}{2.189500in}}%
\pgfusepath{clip}%
\pgfsetbuttcap%
\pgfsetroundjoin%
\pgfsetlinewidth{0.050187pt}%
\definecolor{currentstroke}{rgb}{1.000000,1.000000,1.000000}%
\pgfsetstrokecolor{currentstroke}%
\pgfsetdash{}{0pt}%
\pgfpathmoveto{\pgfqpoint{2.353350in}{1.135577in}}%
\pgfpathlineto{\pgfqpoint{2.344863in}{1.142471in}}%
\pgfpathlineto{\pgfqpoint{2.337942in}{1.139191in}}%
\pgfpathlineto{\pgfqpoint{2.329095in}{1.155711in}}%
\pgfpathlineto{\pgfqpoint{2.333615in}{1.166095in}}%
\pgfpathlineto{\pgfqpoint{2.327101in}{1.177782in}}%
\pgfpathlineto{\pgfqpoint{2.326751in}{1.186984in}}%
\pgfpathlineto{\pgfqpoint{2.317949in}{1.190262in}}%
\pgfpathlineto{\pgfqpoint{2.313866in}{1.197195in}}%
\pgfpathlineto{\pgfqpoint{2.296656in}{1.205854in}}%
\pgfpathlineto{\pgfqpoint{2.281709in}{1.216522in}}%
\pgfpathlineto{\pgfqpoint{2.274764in}{1.224576in}}%
\pgfpathlineto{\pgfqpoint{2.274620in}{1.234399in}}%
\pgfpathlineto{\pgfqpoint{2.283914in}{1.253258in}}%
\pgfpathlineto{\pgfqpoint{2.286772in}{1.267684in}}%
\pgfpathlineto{\pgfqpoint{2.279201in}{1.275834in}}%
\pgfpathlineto{\pgfqpoint{2.269173in}{1.278906in}}%
\pgfpathlineto{\pgfqpoint{2.257008in}{1.272190in}}%
\pgfpathlineto{\pgfqpoint{2.251858in}{1.283719in}}%
\pgfpathlineto{\pgfqpoint{2.249260in}{1.299348in}}%
\pgfpathlineto{\pgfqpoint{2.231331in}{1.313280in}}%
\pgfpathlineto{\pgfqpoint{2.227748in}{1.319462in}}%
\pgfpathlineto{\pgfqpoint{2.211373in}{1.333458in}}%
\pgfpathlineto{\pgfqpoint{2.206240in}{1.343633in}}%
\pgfpathlineto{\pgfqpoint{2.201613in}{1.363811in}}%
\pgfpathlineto{\pgfqpoint{2.204785in}{1.381733in}}%
\pgfpathlineto{\pgfqpoint{2.209018in}{1.384231in}}%
\pgfpathlineto{\pgfqpoint{2.208256in}{1.399234in}}%
\pgfpathlineto{\pgfqpoint{2.220205in}{1.403691in}}%
\pgfpathlineto{\pgfqpoint{2.223813in}{1.417154in}}%
\pgfpathlineto{\pgfqpoint{2.230688in}{1.426216in}}%
\pgfpathlineto{\pgfqpoint{2.230340in}{1.437745in}}%
\pgfpathlineto{\pgfqpoint{2.221819in}{1.446914in}}%
\pgfpathlineto{\pgfqpoint{2.223843in}{1.459748in}}%
\pgfpathlineto{\pgfqpoint{2.245930in}{1.465330in}}%
\pgfpathlineto{\pgfqpoint{2.262874in}{1.475518in}}%
\pgfpathlineto{\pgfqpoint{2.264653in}{1.488346in}}%
\pgfpathlineto{\pgfqpoint{2.270549in}{1.492385in}}%
\pgfpathlineto{\pgfqpoint{2.272804in}{1.505862in}}%
\pgfpathlineto{\pgfqpoint{2.270992in}{1.514764in}}%
\pgfpathlineto{\pgfqpoint{2.259407in}{1.522158in}}%
\pgfpathlineto{\pgfqpoint{2.254744in}{1.533181in}}%
\pgfpathlineto{\pgfqpoint{2.243311in}{1.543836in}}%
\pgfpathlineto{\pgfqpoint{2.337262in}{1.547840in}}%
\pgfpathlineto{\pgfqpoint{2.400293in}{1.552318in}}%
\pgfpathlineto{\pgfqpoint{2.399140in}{1.539059in}}%
\pgfpathlineto{\pgfqpoint{2.409882in}{1.520770in}}%
\pgfpathlineto{\pgfqpoint{2.414392in}{1.505145in}}%
\pgfpathlineto{\pgfqpoint{2.419763in}{1.496273in}}%
\pgfpathlineto{\pgfqpoint{2.425626in}{1.423173in}}%
\pgfpathlineto{\pgfqpoint{2.433769in}{1.318973in}}%
\pgfpathlineto{\pgfqpoint{2.427487in}{1.303139in}}%
\pgfpathlineto{\pgfqpoint{2.436537in}{1.290124in}}%
\pgfpathlineto{\pgfqpoint{2.439060in}{1.277692in}}%
\pgfpathlineto{\pgfqpoint{2.426579in}{1.251916in}}%
\pgfpathlineto{\pgfqpoint{2.427315in}{1.248355in}}%
\pgfpathlineto{\pgfqpoint{2.414470in}{1.233580in}}%
\pgfpathlineto{\pgfqpoint{2.417993in}{1.229068in}}%
\pgfpathlineto{\pgfqpoint{2.412648in}{1.219858in}}%
\pgfpathlineto{\pgfqpoint{2.414023in}{1.201296in}}%
\pgfpathlineto{\pgfqpoint{2.407532in}{1.189905in}}%
\pgfpathlineto{\pgfqpoint{2.412811in}{1.176445in}}%
\pgfpathlineto{\pgfqpoint{2.390692in}{1.169131in}}%
\pgfpathlineto{\pgfqpoint{2.388653in}{1.161177in}}%
\pgfpathlineto{\pgfqpoint{2.394693in}{1.151096in}}%
\pgfpathlineto{\pgfqpoint{2.391914in}{1.144525in}}%
\pgfpathlineto{\pgfqpoint{2.368195in}{1.152642in}}%
\pgfpathlineto{\pgfqpoint{2.360370in}{1.153458in}}%
\pgfpathlineto{\pgfqpoint{2.350613in}{1.141119in}}%
\pgfpathlineto{\pgfqpoint{2.353350in}{1.135577in}}%
\pgfusepath{stroke}%
\end{pgfscope}%
\begin{pgfscope}%
\pgfpathrectangle{\pgfqpoint{0.100000in}{0.100000in}}{\pgfqpoint{3.420221in}{2.189500in}}%
\pgfusepath{clip}%
\pgfsetbuttcap%
\pgfsetroundjoin%
\pgfsetlinewidth{0.050187pt}%
\definecolor{currentstroke}{rgb}{1.000000,1.000000,1.000000}%
\pgfsetstrokecolor{currentstroke}%
\pgfsetdash{}{0pt}%
\pgfpathmoveto{\pgfqpoint{3.043772in}{1.362358in}}%
\pgfpathlineto{\pgfqpoint{3.037245in}{1.372050in}}%
\pgfpathlineto{\pgfqpoint{3.040928in}{1.377474in}}%
\pgfpathlineto{\pgfqpoint{3.049949in}{1.371383in}}%
\pgfpathlineto{\pgfqpoint{3.043772in}{1.362358in}}%
\pgfusepath{stroke}%
\end{pgfscope}%
\begin{pgfscope}%
\pgfpathrectangle{\pgfqpoint{0.100000in}{0.100000in}}{\pgfqpoint{3.420221in}{2.189500in}}%
\pgfusepath{clip}%
\pgfsetbuttcap%
\pgfsetroundjoin%
\pgfsetlinewidth{0.050187pt}%
\definecolor{currentstroke}{rgb}{1.000000,1.000000,1.000000}%
\pgfsetstrokecolor{currentstroke}%
\pgfsetdash{}{0pt}%
\pgfpathmoveto{\pgfqpoint{3.100771in}{1.445763in}}%
\pgfpathlineto{\pgfqpoint{3.104672in}{1.454004in}}%
\pgfpathlineto{\pgfqpoint{3.120440in}{1.455784in}}%
\pgfpathlineto{\pgfqpoint{3.117913in}{1.448780in}}%
\pgfpathlineto{\pgfqpoint{3.112711in}{1.439833in}}%
\pgfpathlineto{\pgfqpoint{3.116236in}{1.429173in}}%
\pgfpathlineto{\pgfqpoint{3.130149in}{1.416400in}}%
\pgfpathlineto{\pgfqpoint{3.133378in}{1.402950in}}%
\pgfpathlineto{\pgfqpoint{3.149407in}{1.386201in}}%
\pgfpathlineto{\pgfqpoint{3.155695in}{1.386918in}}%
\pgfpathlineto{\pgfqpoint{3.163529in}{1.361774in}}%
\pgfpathlineto{\pgfqpoint{3.162244in}{1.361524in}}%
\pgfpathlineto{\pgfqpoint{3.160816in}{1.361243in}}%
\pgfpathlineto{\pgfqpoint{3.125879in}{1.354555in}}%
\pgfpathlineto{\pgfqpoint{3.107188in}{1.421030in}}%
\pgfpathlineto{\pgfqpoint{3.100771in}{1.445763in}}%
\pgfusepath{stroke}%
\end{pgfscope}%
\begin{pgfscope}%
\pgfpathrectangle{\pgfqpoint{0.100000in}{0.100000in}}{\pgfqpoint{3.420221in}{2.189500in}}%
\pgfusepath{clip}%
\pgfsetbuttcap%
\pgfsetroundjoin%
\pgfsetlinewidth{0.050187pt}%
\definecolor{currentstroke}{rgb}{1.000000,1.000000,1.000000}%
\pgfsetstrokecolor{currentstroke}%
\pgfsetdash{}{0pt}%
\pgfpathmoveto{\pgfqpoint{2.774458in}{1.220440in}}%
\pgfpathlineto{\pgfqpoint{2.763418in}{1.220846in}}%
\pgfpathlineto{\pgfqpoint{2.753246in}{1.227853in}}%
\pgfpathlineto{\pgfqpoint{2.739503in}{1.249150in}}%
\pgfpathlineto{\pgfqpoint{2.727805in}{1.260429in}}%
\pgfpathlineto{\pgfqpoint{2.730862in}{1.269140in}}%
\pgfpathlineto{\pgfqpoint{2.728564in}{1.280663in}}%
\pgfpathlineto{\pgfqpoint{2.743226in}{1.284926in}}%
\pgfpathlineto{\pgfqpoint{2.744062in}{1.295494in}}%
\pgfpathlineto{\pgfqpoint{2.750035in}{1.297470in}}%
\pgfpathlineto{\pgfqpoint{2.746219in}{1.310844in}}%
\pgfpathlineto{\pgfqpoint{2.753768in}{1.329719in}}%
\pgfpathlineto{\pgfqpoint{2.769710in}{1.325174in}}%
\pgfpathlineto{\pgfqpoint{2.768369in}{1.344073in}}%
\pgfpathlineto{\pgfqpoint{2.782684in}{1.363700in}}%
\pgfpathlineto{\pgfqpoint{2.788468in}{1.359937in}}%
\pgfpathlineto{\pgfqpoint{2.795906in}{1.364308in}}%
\pgfpathlineto{\pgfqpoint{2.813574in}{1.384901in}}%
\pgfpathlineto{\pgfqpoint{2.817067in}{1.407990in}}%
\pgfpathlineto{\pgfqpoint{2.817558in}{1.425730in}}%
\pgfpathlineto{\pgfqpoint{2.821277in}{1.437867in}}%
\pgfpathlineto{\pgfqpoint{2.819368in}{1.450788in}}%
\pgfpathlineto{\pgfqpoint{2.814575in}{1.456683in}}%
\pgfpathlineto{\pgfqpoint{2.821998in}{1.463100in}}%
\pgfpathlineto{\pgfqpoint{2.832758in}{1.395312in}}%
\pgfpathlineto{\pgfqpoint{2.892074in}{1.405120in}}%
\pgfpathlineto{\pgfqpoint{2.898222in}{1.366429in}}%
\pgfpathlineto{\pgfqpoint{2.919697in}{1.391967in}}%
\pgfpathlineto{\pgfqpoint{2.924757in}{1.389420in}}%
\pgfpathlineto{\pgfqpoint{2.932200in}{1.404218in}}%
\pgfpathlineto{\pgfqpoint{2.942512in}{1.399578in}}%
\pgfpathlineto{\pgfqpoint{2.951504in}{1.400460in}}%
\pgfpathlineto{\pgfqpoint{2.957436in}{1.410747in}}%
\pgfpathlineto{\pgfqpoint{2.966046in}{1.416470in}}%
\pgfpathlineto{\pgfqpoint{2.978017in}{1.411430in}}%
\pgfpathlineto{\pgfqpoint{2.985894in}{1.413171in}}%
\pgfpathlineto{\pgfqpoint{2.997179in}{1.393642in}}%
\pgfpathlineto{\pgfqpoint{2.993915in}{1.378857in}}%
\pgfpathlineto{\pgfqpoint{2.964442in}{1.395503in}}%
\pgfpathlineto{\pgfqpoint{2.958921in}{1.381838in}}%
\pgfpathlineto{\pgfqpoint{2.960743in}{1.375550in}}%
\pgfpathlineto{\pgfqpoint{2.948199in}{1.354296in}}%
\pgfpathlineto{\pgfqpoint{2.942279in}{1.350068in}}%
\pgfpathlineto{\pgfqpoint{2.939605in}{1.340695in}}%
\pgfpathlineto{\pgfqpoint{2.929512in}{1.341676in}}%
\pgfpathlineto{\pgfqpoint{2.922264in}{1.316094in}}%
\pgfpathlineto{\pgfqpoint{2.918207in}{1.310224in}}%
\pgfpathlineto{\pgfqpoint{2.907781in}{1.312181in}}%
\pgfpathlineto{\pgfqpoint{2.897117in}{1.320250in}}%
\pgfpathlineto{\pgfqpoint{2.896748in}{1.307866in}}%
\pgfpathlineto{\pgfqpoint{2.886425in}{1.287053in}}%
\pgfpathlineto{\pgfqpoint{2.883857in}{1.272237in}}%
\pgfpathlineto{\pgfqpoint{2.875936in}{1.262369in}}%
\pgfpathlineto{\pgfqpoint{2.869931in}{1.246611in}}%
\pgfpathlineto{\pgfqpoint{2.869636in}{1.232022in}}%
\pgfpathlineto{\pgfqpoint{2.860362in}{1.229298in}}%
\pgfpathlineto{\pgfqpoint{2.849842in}{1.221040in}}%
\pgfpathlineto{\pgfqpoint{2.839711in}{1.219467in}}%
\pgfpathlineto{\pgfqpoint{2.837377in}{1.212477in}}%
\pgfpathlineto{\pgfqpoint{2.821055in}{1.205319in}}%
\pgfpathlineto{\pgfqpoint{2.811934in}{1.211418in}}%
\pgfpathlineto{\pgfqpoint{2.801739in}{1.199837in}}%
\pgfpathlineto{\pgfqpoint{2.784213in}{1.203251in}}%
\pgfpathlineto{\pgfqpoint{2.773464in}{1.215406in}}%
\pgfpathlineto{\pgfqpoint{2.774458in}{1.220440in}}%
\pgfusepath{stroke}%
\end{pgfscope}%
\begin{pgfscope}%
\pgfpathrectangle{\pgfqpoint{0.100000in}{0.100000in}}{\pgfqpoint{3.420221in}{2.189500in}}%
\pgfusepath{clip}%
\pgfsetbuttcap%
\pgfsetroundjoin%
\pgfsetlinewidth{0.050187pt}%
\definecolor{currentstroke}{rgb}{1.000000,1.000000,1.000000}%
\pgfsetstrokecolor{currentstroke}%
\pgfsetdash{}{0pt}%
\pgfpathmoveto{\pgfqpoint{3.160816in}{1.361243in}}%
\pgfpathlineto{\pgfqpoint{3.160368in}{1.347583in}}%
\pgfpathlineto{\pgfqpoint{3.155111in}{1.340868in}}%
\pgfpathlineto{\pgfqpoint{3.151756in}{1.325959in}}%
\pgfpathlineto{\pgfqpoint{3.136606in}{1.319088in}}%
\pgfpathlineto{\pgfqpoint{3.123895in}{1.317085in}}%
\pgfpathlineto{\pgfqpoint{3.127599in}{1.326890in}}%
\pgfpathlineto{\pgfqpoint{3.108027in}{1.335105in}}%
\pgfpathlineto{\pgfqpoint{3.092127in}{1.345392in}}%
\pgfpathlineto{\pgfqpoint{3.101971in}{1.360768in}}%
\pgfpathlineto{\pgfqpoint{3.094082in}{1.372720in}}%
\pgfpathlineto{\pgfqpoint{3.095010in}{1.383029in}}%
\pgfpathlineto{\pgfqpoint{3.089306in}{1.385859in}}%
\pgfpathlineto{\pgfqpoint{3.084677in}{1.402608in}}%
\pgfpathlineto{\pgfqpoint{3.095624in}{1.425313in}}%
\pgfpathlineto{\pgfqpoint{3.087400in}{1.429021in}}%
\pgfpathlineto{\pgfqpoint{3.085268in}{1.417820in}}%
\pgfpathlineto{\pgfqpoint{3.073529in}{1.414701in}}%
\pgfpathlineto{\pgfqpoint{3.074763in}{1.377847in}}%
\pgfpathlineto{\pgfqpoint{3.072617in}{1.365986in}}%
\pgfpathlineto{\pgfqpoint{3.078505in}{1.349081in}}%
\pgfpathlineto{\pgfqpoint{3.087564in}{1.340947in}}%
\pgfpathlineto{\pgfqpoint{3.083456in}{1.335840in}}%
\pgfpathlineto{\pgfqpoint{3.092702in}{1.328378in}}%
\pgfpathlineto{\pgfqpoint{3.096045in}{1.316265in}}%
\pgfpathlineto{\pgfqpoint{3.079097in}{1.326296in}}%
\pgfpathlineto{\pgfqpoint{3.068371in}{1.324993in}}%
\pgfpathlineto{\pgfqpoint{3.060043in}{1.335311in}}%
\pgfpathlineto{\pgfqpoint{3.051562in}{1.336316in}}%
\pgfpathlineto{\pgfqpoint{3.039511in}{1.331148in}}%
\pgfpathlineto{\pgfqpoint{3.034837in}{1.337588in}}%
\pgfpathlineto{\pgfqpoint{3.040980in}{1.351105in}}%
\pgfpathlineto{\pgfqpoint{3.043772in}{1.362358in}}%
\pgfpathlineto{\pgfqpoint{3.049949in}{1.371383in}}%
\pgfpathlineto{\pgfqpoint{3.040928in}{1.377474in}}%
\pgfpathlineto{\pgfqpoint{3.037245in}{1.372050in}}%
\pgfpathlineto{\pgfqpoint{3.028233in}{1.377556in}}%
\pgfpathlineto{\pgfqpoint{3.012276in}{1.381224in}}%
\pgfpathlineto{\pgfqpoint{3.013723in}{1.389291in}}%
\pgfpathlineto{\pgfqpoint{3.006490in}{1.393969in}}%
\pgfpathlineto{\pgfqpoint{2.997179in}{1.393642in}}%
\pgfpathlineto{\pgfqpoint{2.985894in}{1.413171in}}%
\pgfpathlineto{\pgfqpoint{2.978017in}{1.411430in}}%
\pgfpathlineto{\pgfqpoint{2.966046in}{1.416470in}}%
\pgfpathlineto{\pgfqpoint{2.957436in}{1.410747in}}%
\pgfpathlineto{\pgfqpoint{2.951504in}{1.400460in}}%
\pgfpathlineto{\pgfqpoint{2.942512in}{1.399578in}}%
\pgfpathlineto{\pgfqpoint{2.932200in}{1.404218in}}%
\pgfpathlineto{\pgfqpoint{2.924757in}{1.389420in}}%
\pgfpathlineto{\pgfqpoint{2.919697in}{1.391967in}}%
\pgfpathlineto{\pgfqpoint{2.898222in}{1.366429in}}%
\pgfpathlineto{\pgfqpoint{2.892074in}{1.405120in}}%
\pgfpathlineto{\pgfqpoint{2.970550in}{1.419665in}}%
\pgfpathlineto{\pgfqpoint{3.005730in}{1.425964in}}%
\pgfpathlineto{\pgfqpoint{3.056910in}{1.436272in}}%
\pgfpathlineto{\pgfqpoint{3.100771in}{1.445763in}}%
\pgfpathlineto{\pgfqpoint{3.107188in}{1.421030in}}%
\pgfpathlineto{\pgfqpoint{3.125879in}{1.354555in}}%
\pgfpathlineto{\pgfqpoint{3.160816in}{1.361243in}}%
\pgfusepath{stroke}%
\end{pgfscope}%
\begin{pgfscope}%
\pgfpathrectangle{\pgfqpoint{0.100000in}{0.100000in}}{\pgfqpoint{3.420221in}{2.189500in}}%
\pgfusepath{clip}%
\pgfsetbuttcap%
\pgfsetroundjoin%
\pgfsetlinewidth{0.050187pt}%
\definecolor{currentstroke}{rgb}{1.000000,1.000000,1.000000}%
\pgfsetstrokecolor{currentstroke}%
\pgfsetdash{}{0pt}%
\pgfpathmoveto{\pgfqpoint{1.581134in}{1.140598in}}%
\pgfpathlineto{\pgfqpoint{1.523879in}{1.146061in}}%
\pgfpathlineto{\pgfqpoint{1.464482in}{1.151325in}}%
\pgfpathlineto{\pgfqpoint{1.395847in}{1.158485in}}%
\pgfpathlineto{\pgfqpoint{1.293872in}{1.170613in}}%
\pgfpathlineto{\pgfqpoint{1.257697in}{1.176187in}}%
\pgfpathlineto{\pgfqpoint{1.164318in}{1.189748in}}%
\pgfpathlineto{\pgfqpoint{1.177952in}{1.275326in}}%
\pgfpathlineto{\pgfqpoint{1.178292in}{1.282248in}}%
\pgfpathlineto{\pgfqpoint{1.191428in}{1.364823in}}%
\pgfpathlineto{\pgfqpoint{1.201239in}{1.427713in}}%
\pgfpathlineto{\pgfqpoint{1.210497in}{1.486032in}}%
\pgfpathlineto{\pgfqpoint{1.273755in}{1.476952in}}%
\pgfpathlineto{\pgfqpoint{1.332734in}{1.468487in}}%
\pgfpathlineto{\pgfqpoint{1.441153in}{1.455105in}}%
\pgfpathlineto{\pgfqpoint{1.490892in}{1.450563in}}%
\pgfpathlineto{\pgfqpoint{1.569691in}{1.442918in}}%
\pgfpathlineto{\pgfqpoint{1.603779in}{1.440151in}}%
\pgfpathlineto{\pgfqpoint{1.597761in}{1.365516in}}%
\pgfpathlineto{\pgfqpoint{1.592323in}{1.293655in}}%
\pgfpathlineto{\pgfqpoint{1.587947in}{1.235143in}}%
\pgfpathlineto{\pgfqpoint{1.581134in}{1.140598in}}%
\pgfusepath{stroke}%
\end{pgfscope}%
\begin{pgfscope}%
\pgfpathrectangle{\pgfqpoint{0.100000in}{0.100000in}}{\pgfqpoint{3.420221in}{2.189500in}}%
\pgfusepath{clip}%
\pgfsetbuttcap%
\pgfsetroundjoin%
\pgfsetlinewidth{0.050187pt}%
\definecolor{currentstroke}{rgb}{1.000000,1.000000,1.000000}%
\pgfsetstrokecolor{currentstroke}%
\pgfsetdash{}{0pt}%
\pgfpathmoveto{\pgfqpoint{2.338563in}{1.098170in}}%
\pgfpathlineto{\pgfqpoint{2.340493in}{1.106822in}}%
\pgfpathlineto{\pgfqpoint{2.350497in}{1.104874in}}%
\pgfpathlineto{\pgfqpoint{2.353350in}{1.135577in}}%
\pgfpathlineto{\pgfqpoint{2.350613in}{1.141119in}}%
\pgfpathlineto{\pgfqpoint{2.360370in}{1.153458in}}%
\pgfpathlineto{\pgfqpoint{2.368195in}{1.152642in}}%
\pgfpathlineto{\pgfqpoint{2.391914in}{1.144525in}}%
\pgfpathlineto{\pgfqpoint{2.394693in}{1.151096in}}%
\pgfpathlineto{\pgfqpoint{2.388653in}{1.161177in}}%
\pgfpathlineto{\pgfqpoint{2.390692in}{1.169131in}}%
\pgfpathlineto{\pgfqpoint{2.412811in}{1.176445in}}%
\pgfpathlineto{\pgfqpoint{2.407532in}{1.189905in}}%
\pgfpathlineto{\pgfqpoint{2.414023in}{1.201296in}}%
\pgfpathlineto{\pgfqpoint{2.426106in}{1.207689in}}%
\pgfpathlineto{\pgfqpoint{2.451458in}{1.214020in}}%
\pgfpathlineto{\pgfqpoint{2.467454in}{1.204464in}}%
\pgfpathlineto{\pgfqpoint{2.472307in}{1.213784in}}%
\pgfpathlineto{\pgfqpoint{2.485589in}{1.220293in}}%
\pgfpathlineto{\pgfqpoint{2.489017in}{1.214552in}}%
\pgfpathlineto{\pgfqpoint{2.502660in}{1.219114in}}%
\pgfpathlineto{\pgfqpoint{2.501799in}{1.226921in}}%
\pgfpathlineto{\pgfqpoint{2.514066in}{1.235855in}}%
\pgfpathlineto{\pgfqpoint{2.521306in}{1.226577in}}%
\pgfpathlineto{\pgfqpoint{2.530830in}{1.225644in}}%
\pgfpathlineto{\pgfqpoint{2.537160in}{1.231691in}}%
\pgfpathlineto{\pgfqpoint{2.536456in}{1.240289in}}%
\pgfpathlineto{\pgfqpoint{2.541836in}{1.248828in}}%
\pgfpathlineto{\pgfqpoint{2.549045in}{1.250678in}}%
\pgfpathlineto{\pgfqpoint{2.551940in}{1.261952in}}%
\pgfpathlineto{\pgfqpoint{2.562425in}{1.271699in}}%
\pgfpathlineto{\pgfqpoint{2.559269in}{1.281400in}}%
\pgfpathlineto{\pgfqpoint{2.569470in}{1.286231in}}%
\pgfpathlineto{\pgfqpoint{2.576285in}{1.283260in}}%
\pgfpathlineto{\pgfqpoint{2.586352in}{1.290806in}}%
\pgfpathlineto{\pgfqpoint{2.595348in}{1.292772in}}%
\pgfpathlineto{\pgfqpoint{2.595380in}{1.300600in}}%
\pgfpathlineto{\pgfqpoint{2.589076in}{1.311461in}}%
\pgfpathlineto{\pgfqpoint{2.593180in}{1.315837in}}%
\pgfpathlineto{\pgfqpoint{2.606073in}{1.314867in}}%
\pgfpathlineto{\pgfqpoint{2.614272in}{1.318952in}}%
\pgfpathlineto{\pgfqpoint{2.622729in}{1.312571in}}%
\pgfpathlineto{\pgfqpoint{2.629404in}{1.298170in}}%
\pgfpathlineto{\pgfqpoint{2.651533in}{1.295737in}}%
\pgfpathlineto{\pgfqpoint{2.663068in}{1.287809in}}%
\pgfpathlineto{\pgfqpoint{2.671185in}{1.294510in}}%
\pgfpathlineto{\pgfqpoint{2.685859in}{1.288878in}}%
\pgfpathlineto{\pgfqpoint{2.694167in}{1.291677in}}%
\pgfpathlineto{\pgfqpoint{2.699974in}{1.299634in}}%
\pgfpathlineto{\pgfqpoint{2.708838in}{1.301156in}}%
\pgfpathlineto{\pgfqpoint{2.711142in}{1.292504in}}%
\pgfpathlineto{\pgfqpoint{2.720945in}{1.289196in}}%
\pgfpathlineto{\pgfqpoint{2.728564in}{1.280663in}}%
\pgfpathlineto{\pgfqpoint{2.730862in}{1.269140in}}%
\pgfpathlineto{\pgfqpoint{2.727805in}{1.260429in}}%
\pgfpathlineto{\pgfqpoint{2.739503in}{1.249150in}}%
\pgfpathlineto{\pgfqpoint{2.753246in}{1.227853in}}%
\pgfpathlineto{\pgfqpoint{2.763418in}{1.220846in}}%
\pgfpathlineto{\pgfqpoint{2.774458in}{1.220440in}}%
\pgfpathlineto{\pgfqpoint{2.754073in}{1.197050in}}%
\pgfpathlineto{\pgfqpoint{2.734008in}{1.182893in}}%
\pgfpathlineto{\pgfqpoint{2.726624in}{1.171650in}}%
\pgfpathlineto{\pgfqpoint{2.726749in}{1.165548in}}%
\pgfpathlineto{\pgfqpoint{2.715881in}{1.160872in}}%
\pgfpathlineto{\pgfqpoint{2.712766in}{1.152073in}}%
\pgfpathlineto{\pgfqpoint{2.690126in}{1.143210in}}%
\pgfpathlineto{\pgfqpoint{2.682101in}{1.137453in}}%
\pgfpathlineto{\pgfqpoint{2.681015in}{1.136225in}}%
\pgfpathlineto{\pgfqpoint{2.615890in}{1.130050in}}%
\pgfpathlineto{\pgfqpoint{2.576562in}{1.126742in}}%
\pgfpathlineto{\pgfqpoint{2.512026in}{1.123090in}}%
\pgfpathlineto{\pgfqpoint{2.431626in}{1.115137in}}%
\pgfpathlineto{\pgfqpoint{2.418343in}{1.116976in}}%
\pgfpathlineto{\pgfqpoint{2.421106in}{1.103416in}}%
\pgfpathlineto{\pgfqpoint{2.338563in}{1.098170in}}%
\pgfusepath{stroke}%
\end{pgfscope}%
\begin{pgfscope}%
\pgfpathrectangle{\pgfqpoint{0.100000in}{0.100000in}}{\pgfqpoint{3.420221in}{2.189500in}}%
\pgfusepath{clip}%
\pgfsetbuttcap%
\pgfsetroundjoin%
\pgfsetlinewidth{0.050187pt}%
\definecolor{currentstroke}{rgb}{1.000000,1.000000,1.000000}%
\pgfsetstrokecolor{currentstroke}%
\pgfsetdash{}{0pt}%
\pgfpathmoveto{\pgfqpoint{1.581134in}{1.140598in}}%
\pgfpathlineto{\pgfqpoint{1.587947in}{1.235143in}}%
\pgfpathlineto{\pgfqpoint{1.592323in}{1.293655in}}%
\pgfpathlineto{\pgfqpoint{1.597761in}{1.365516in}}%
\pgfpathlineto{\pgfqpoint{1.673137in}{1.360223in}}%
\pgfpathlineto{\pgfqpoint{1.768865in}{1.354994in}}%
\pgfpathlineto{\pgfqpoint{1.833972in}{1.352496in}}%
\pgfpathlineto{\pgfqpoint{1.898698in}{1.350513in}}%
\pgfpathlineto{\pgfqpoint{1.957282in}{1.349607in}}%
\pgfpathlineto{\pgfqpoint{1.984377in}{1.349888in}}%
\pgfpathlineto{\pgfqpoint{1.996303in}{1.340155in}}%
\pgfpathlineto{\pgfqpoint{2.005645in}{1.342117in}}%
\pgfpathlineto{\pgfqpoint{2.005937in}{1.333628in}}%
\pgfpathlineto{\pgfqpoint{1.999000in}{1.318943in}}%
\pgfpathlineto{\pgfqpoint{1.999752in}{1.309665in}}%
\pgfpathlineto{\pgfqpoint{2.006395in}{1.303548in}}%
\pgfpathlineto{\pgfqpoint{2.012496in}{1.290801in}}%
\pgfpathlineto{\pgfqpoint{2.024873in}{1.283482in}}%
\pgfpathlineto{\pgfqpoint{2.024460in}{1.235433in}}%
\pgfpathlineto{\pgfqpoint{2.024824in}{1.124942in}}%
\pgfpathlineto{\pgfqpoint{1.941863in}{1.125324in}}%
\pgfpathlineto{\pgfqpoint{1.854502in}{1.126853in}}%
\pgfpathlineto{\pgfqpoint{1.790191in}{1.129122in}}%
\pgfpathlineto{\pgfqpoint{1.736509in}{1.131193in}}%
\pgfpathlineto{\pgfqpoint{1.646071in}{1.136526in}}%
\pgfpathlineto{\pgfqpoint{1.581134in}{1.140598in}}%
\pgfusepath{stroke}%
\end{pgfscope}%
\begin{pgfscope}%
\pgfpathrectangle{\pgfqpoint{0.100000in}{0.100000in}}{\pgfqpoint{3.420221in}{2.189500in}}%
\pgfusepath{clip}%
\pgfsetbuttcap%
\pgfsetroundjoin%
\pgfsetlinewidth{0.050187pt}%
\definecolor{currentstroke}{rgb}{1.000000,1.000000,1.000000}%
\pgfsetstrokecolor{currentstroke}%
\pgfsetdash{}{0pt}%
\pgfpathmoveto{\pgfqpoint{2.801811in}{1.153116in}}%
\pgfpathlineto{\pgfqpoint{2.709926in}{1.140209in}}%
\pgfpathlineto{\pgfqpoint{2.682101in}{1.137453in}}%
\pgfpathlineto{\pgfqpoint{2.690126in}{1.143210in}}%
\pgfpathlineto{\pgfqpoint{2.712766in}{1.152073in}}%
\pgfpathlineto{\pgfqpoint{2.715881in}{1.160872in}}%
\pgfpathlineto{\pgfqpoint{2.726749in}{1.165548in}}%
\pgfpathlineto{\pgfqpoint{2.726624in}{1.171650in}}%
\pgfpathlineto{\pgfqpoint{2.734008in}{1.182893in}}%
\pgfpathlineto{\pgfqpoint{2.754073in}{1.197050in}}%
\pgfpathlineto{\pgfqpoint{2.774458in}{1.220440in}}%
\pgfpathlineto{\pgfqpoint{2.773464in}{1.215406in}}%
\pgfpathlineto{\pgfqpoint{2.784213in}{1.203251in}}%
\pgfpathlineto{\pgfqpoint{2.801739in}{1.199837in}}%
\pgfpathlineto{\pgfqpoint{2.811934in}{1.211418in}}%
\pgfpathlineto{\pgfqpoint{2.821055in}{1.205319in}}%
\pgfpathlineto{\pgfqpoint{2.837377in}{1.212477in}}%
\pgfpathlineto{\pgfqpoint{2.839711in}{1.219467in}}%
\pgfpathlineto{\pgfqpoint{2.849842in}{1.221040in}}%
\pgfpathlineto{\pgfqpoint{2.860362in}{1.229298in}}%
\pgfpathlineto{\pgfqpoint{2.869636in}{1.232022in}}%
\pgfpathlineto{\pgfqpoint{2.869931in}{1.246611in}}%
\pgfpathlineto{\pgfqpoint{2.875936in}{1.262369in}}%
\pgfpathlineto{\pgfqpoint{2.883857in}{1.272237in}}%
\pgfpathlineto{\pgfqpoint{2.886425in}{1.287053in}}%
\pgfpathlineto{\pgfqpoint{2.896748in}{1.307866in}}%
\pgfpathlineto{\pgfqpoint{2.897117in}{1.320250in}}%
\pgfpathlineto{\pgfqpoint{2.907781in}{1.312181in}}%
\pgfpathlineto{\pgfqpoint{2.918207in}{1.310224in}}%
\pgfpathlineto{\pgfqpoint{2.922264in}{1.316094in}}%
\pgfpathlineto{\pgfqpoint{2.929512in}{1.341676in}}%
\pgfpathlineto{\pgfqpoint{2.939605in}{1.340695in}}%
\pgfpathlineto{\pgfqpoint{2.942279in}{1.350068in}}%
\pgfpathlineto{\pgfqpoint{2.948199in}{1.354296in}}%
\pgfpathlineto{\pgfqpoint{2.960743in}{1.375550in}}%
\pgfpathlineto{\pgfqpoint{2.958921in}{1.381838in}}%
\pgfpathlineto{\pgfqpoint{2.964442in}{1.395503in}}%
\pgfpathlineto{\pgfqpoint{2.993915in}{1.378857in}}%
\pgfpathlineto{\pgfqpoint{2.997179in}{1.393642in}}%
\pgfpathlineto{\pgfqpoint{3.006490in}{1.393969in}}%
\pgfpathlineto{\pgfqpoint{3.013723in}{1.389291in}}%
\pgfpathlineto{\pgfqpoint{3.012276in}{1.381224in}}%
\pgfpathlineto{\pgfqpoint{3.028233in}{1.377556in}}%
\pgfpathlineto{\pgfqpoint{3.037245in}{1.372050in}}%
\pgfpathlineto{\pgfqpoint{3.043772in}{1.362358in}}%
\pgfpathlineto{\pgfqpoint{3.040980in}{1.351105in}}%
\pgfpathlineto{\pgfqpoint{3.035340in}{1.350183in}}%
\pgfpathlineto{\pgfqpoint{3.032070in}{1.333203in}}%
\pgfpathlineto{\pgfqpoint{3.039234in}{1.326564in}}%
\pgfpathlineto{\pgfqpoint{3.049319in}{1.331931in}}%
\pgfpathlineto{\pgfqpoint{3.058676in}{1.320599in}}%
\pgfpathlineto{\pgfqpoint{3.079583in}{1.318562in}}%
\pgfpathlineto{\pgfqpoint{3.083185in}{1.312044in}}%
\pgfpathlineto{\pgfqpoint{3.102533in}{1.305704in}}%
\pgfpathlineto{\pgfqpoint{3.100145in}{1.298223in}}%
\pgfpathlineto{\pgfqpoint{3.102662in}{1.277080in}}%
\pgfpathlineto{\pgfqpoint{3.110472in}{1.269096in}}%
\pgfpathlineto{\pgfqpoint{3.098948in}{1.265726in}}%
\pgfpathlineto{\pgfqpoint{3.100476in}{1.256700in}}%
\pgfpathlineto{\pgfqpoint{3.112769in}{1.249048in}}%
\pgfpathlineto{\pgfqpoint{3.113877in}{1.241498in}}%
\pgfpathlineto{\pgfqpoint{3.106937in}{1.235826in}}%
\pgfpathlineto{\pgfqpoint{3.092933in}{1.249265in}}%
\pgfpathlineto{\pgfqpoint{3.091549in}{1.239459in}}%
\pgfpathlineto{\pgfqpoint{3.103272in}{1.234797in}}%
\pgfpathlineto{\pgfqpoint{3.104445in}{1.229975in}}%
\pgfpathlineto{\pgfqpoint{3.120524in}{1.236348in}}%
\pgfpathlineto{\pgfqpoint{3.132845in}{1.238033in}}%
\pgfpathlineto{\pgfqpoint{3.145467in}{1.212567in}}%
\pgfpathlineto{\pgfqpoint{3.144059in}{1.212292in}}%
\pgfpathlineto{\pgfqpoint{3.138356in}{1.211113in}}%
\pgfpathlineto{\pgfqpoint{3.136675in}{1.210762in}}%
\pgfpathlineto{\pgfqpoint{3.135564in}{1.210544in}}%
\pgfpathlineto{\pgfqpoint{3.060414in}{1.194956in}}%
\pgfpathlineto{\pgfqpoint{2.993198in}{1.181198in}}%
\pgfpathlineto{\pgfqpoint{2.900263in}{1.165189in}}%
\pgfpathlineto{\pgfqpoint{2.821335in}{1.154764in}}%
\pgfpathlineto{\pgfqpoint{2.801811in}{1.153116in}}%
\pgfusepath{stroke}%
\end{pgfscope}%
\begin{pgfscope}%
\pgfpathrectangle{\pgfqpoint{0.100000in}{0.100000in}}{\pgfqpoint{3.420221in}{2.189500in}}%
\pgfusepath{clip}%
\pgfsetbuttcap%
\pgfsetroundjoin%
\pgfsetlinewidth{0.050187pt}%
\definecolor{currentstroke}{rgb}{1.000000,1.000000,1.000000}%
\pgfsetstrokecolor{currentstroke}%
\pgfsetdash{}{0pt}%
\pgfpathmoveto{\pgfqpoint{3.136606in}{1.319088in}}%
\pgfpathlineto{\pgfqpoint{3.151756in}{1.325959in}}%
\pgfpathlineto{\pgfqpoint{3.139657in}{1.290568in}}%
\pgfpathlineto{\pgfqpoint{3.131656in}{1.271826in}}%
\pgfpathlineto{\pgfqpoint{3.125893in}{1.282433in}}%
\pgfpathlineto{\pgfqpoint{3.136142in}{1.307825in}}%
\pgfpathlineto{\pgfqpoint{3.136606in}{1.319088in}}%
\pgfusepath{stroke}%
\end{pgfscope}%
\begin{pgfscope}%
\pgfpathrectangle{\pgfqpoint{0.100000in}{0.100000in}}{\pgfqpoint{3.420221in}{2.189500in}}%
\pgfusepath{clip}%
\pgfsetbuttcap%
\pgfsetroundjoin%
\pgfsetlinewidth{0.050187pt}%
\definecolor{currentstroke}{rgb}{1.000000,1.000000,1.000000}%
\pgfsetstrokecolor{currentstroke}%
\pgfsetdash{}{0pt}%
\pgfpathmoveto{\pgfqpoint{2.338563in}{1.098170in}}%
\pgfpathlineto{\pgfqpoint{2.334904in}{1.097640in}}%
\pgfpathlineto{\pgfqpoint{2.331454in}{1.097399in}}%
\pgfpathlineto{\pgfqpoint{2.332931in}{1.086806in}}%
\pgfpathlineto{\pgfqpoint{2.324030in}{1.076128in}}%
\pgfpathlineto{\pgfqpoint{2.322276in}{1.059420in}}%
\pgfpathlineto{\pgfqpoint{2.282483in}{1.056477in}}%
\pgfpathlineto{\pgfqpoint{2.285952in}{1.064322in}}%
\pgfpathlineto{\pgfqpoint{2.300299in}{1.078649in}}%
\pgfpathlineto{\pgfqpoint{2.300696in}{1.086954in}}%
\pgfpathlineto{\pgfqpoint{2.294345in}{1.094817in}}%
\pgfpathlineto{\pgfqpoint{2.235119in}{1.091685in}}%
\pgfpathlineto{\pgfqpoint{2.131572in}{1.088398in}}%
\pgfpathlineto{\pgfqpoint{2.070975in}{1.087293in}}%
\pgfpathlineto{\pgfqpoint{2.025171in}{1.086881in}}%
\pgfpathlineto{\pgfqpoint{2.024824in}{1.124942in}}%
\pgfpathlineto{\pgfqpoint{2.024460in}{1.235433in}}%
\pgfpathlineto{\pgfqpoint{2.024873in}{1.283482in}}%
\pgfpathlineto{\pgfqpoint{2.012496in}{1.290801in}}%
\pgfpathlineto{\pgfqpoint{2.006395in}{1.303548in}}%
\pgfpathlineto{\pgfqpoint{1.999752in}{1.309665in}}%
\pgfpathlineto{\pgfqpoint{1.999000in}{1.318943in}}%
\pgfpathlineto{\pgfqpoint{2.005937in}{1.333628in}}%
\pgfpathlineto{\pgfqpoint{2.005645in}{1.342117in}}%
\pgfpathlineto{\pgfqpoint{1.996303in}{1.340155in}}%
\pgfpathlineto{\pgfqpoint{1.984377in}{1.349888in}}%
\pgfpathlineto{\pgfqpoint{1.974817in}{1.366971in}}%
\pgfpathlineto{\pgfqpoint{1.966797in}{1.374855in}}%
\pgfpathlineto{\pgfqpoint{1.958418in}{1.394226in}}%
\pgfpathlineto{\pgfqpoint{2.045450in}{1.392868in}}%
\pgfpathlineto{\pgfqpoint{2.131962in}{1.395719in}}%
\pgfpathlineto{\pgfqpoint{2.187432in}{1.398918in}}%
\pgfpathlineto{\pgfqpoint{2.204785in}{1.381733in}}%
\pgfpathlineto{\pgfqpoint{2.201613in}{1.363811in}}%
\pgfpathlineto{\pgfqpoint{2.206240in}{1.343633in}}%
\pgfpathlineto{\pgfqpoint{2.211373in}{1.333458in}}%
\pgfpathlineto{\pgfqpoint{2.227748in}{1.319462in}}%
\pgfpathlineto{\pgfqpoint{2.231331in}{1.313280in}}%
\pgfpathlineto{\pgfqpoint{2.249260in}{1.299348in}}%
\pgfpathlineto{\pgfqpoint{2.251858in}{1.283719in}}%
\pgfpathlineto{\pgfqpoint{2.257008in}{1.272190in}}%
\pgfpathlineto{\pgfqpoint{2.269173in}{1.278906in}}%
\pgfpathlineto{\pgfqpoint{2.279201in}{1.275834in}}%
\pgfpathlineto{\pgfqpoint{2.286772in}{1.267684in}}%
\pgfpathlineto{\pgfqpoint{2.283914in}{1.253258in}}%
\pgfpathlineto{\pgfqpoint{2.274620in}{1.234399in}}%
\pgfpathlineto{\pgfqpoint{2.274764in}{1.224576in}}%
\pgfpathlineto{\pgfqpoint{2.281709in}{1.216522in}}%
\pgfpathlineto{\pgfqpoint{2.296656in}{1.205854in}}%
\pgfpathlineto{\pgfqpoint{2.313866in}{1.197195in}}%
\pgfpathlineto{\pgfqpoint{2.317949in}{1.190262in}}%
\pgfpathlineto{\pgfqpoint{2.326751in}{1.186984in}}%
\pgfpathlineto{\pgfqpoint{2.327101in}{1.177782in}}%
\pgfpathlineto{\pgfqpoint{2.333615in}{1.166095in}}%
\pgfpathlineto{\pgfqpoint{2.329095in}{1.155711in}}%
\pgfpathlineto{\pgfqpoint{2.337942in}{1.139191in}}%
\pgfpathlineto{\pgfqpoint{2.344863in}{1.142471in}}%
\pgfpathlineto{\pgfqpoint{2.353350in}{1.135577in}}%
\pgfpathlineto{\pgfqpoint{2.350497in}{1.104874in}}%
\pgfpathlineto{\pgfqpoint{2.340493in}{1.106822in}}%
\pgfpathlineto{\pgfqpoint{2.338563in}{1.098170in}}%
\pgfusepath{stroke}%
\end{pgfscope}%
\begin{pgfscope}%
\pgfpathrectangle{\pgfqpoint{0.100000in}{0.100000in}}{\pgfqpoint{3.420221in}{2.189500in}}%
\pgfusepath{clip}%
\pgfsetbuttcap%
\pgfsetroundjoin%
\pgfsetlinewidth{0.050187pt}%
\definecolor{currentstroke}{rgb}{1.000000,1.000000,1.000000}%
\pgfsetstrokecolor{currentstroke}%
\pgfsetdash{}{0pt}%
\pgfpathmoveto{\pgfqpoint{0.804320in}{1.104789in}}%
\pgfpathlineto{\pgfqpoint{0.811564in}{1.120212in}}%
\pgfpathlineto{\pgfqpoint{0.809581in}{1.143363in}}%
\pgfpathlineto{\pgfqpoint{0.813050in}{1.149947in}}%
\pgfpathlineto{\pgfqpoint{0.813393in}{1.178784in}}%
\pgfpathlineto{\pgfqpoint{0.816681in}{1.187089in}}%
\pgfpathlineto{\pgfqpoint{0.828265in}{1.188381in}}%
\pgfpathlineto{\pgfqpoint{0.839015in}{1.184694in}}%
\pgfpathlineto{\pgfqpoint{0.843697in}{1.174541in}}%
\pgfpathlineto{\pgfqpoint{0.850261in}{1.174930in}}%
\pgfpathlineto{\pgfqpoint{0.858416in}{1.186560in}}%
\pgfpathlineto{\pgfqpoint{0.870204in}{1.243890in}}%
\pgfpathlineto{\pgfqpoint{0.937281in}{1.230041in}}%
\pgfpathlineto{\pgfqpoint{0.976211in}{1.222364in}}%
\pgfpathlineto{\pgfqpoint{1.079589in}{1.204191in}}%
\pgfpathlineto{\pgfqpoint{1.108190in}{1.198448in}}%
\pgfpathlineto{\pgfqpoint{1.164318in}{1.189748in}}%
\pgfpathlineto{\pgfqpoint{1.152809in}{1.115645in}}%
\pgfpathlineto{\pgfqpoint{1.140839in}{1.038349in}}%
\pgfpathlineto{\pgfqpoint{1.127055in}{0.951388in}}%
\pgfpathlineto{\pgfqpoint{1.111553in}{0.851566in}}%
\pgfpathlineto{\pgfqpoint{1.099030in}{0.769581in}}%
\pgfpathlineto{\pgfqpoint{1.009328in}{0.783825in}}%
\pgfpathlineto{\pgfqpoint{0.969915in}{0.790575in}}%
\pgfpathlineto{\pgfqpoint{0.952309in}{0.801116in}}%
\pgfpathlineto{\pgfqpoint{0.837515in}{0.870377in}}%
\pgfpathlineto{\pgfqpoint{0.751367in}{0.922939in}}%
\pgfpathlineto{\pgfqpoint{0.754236in}{0.932255in}}%
\pgfpathlineto{\pgfqpoint{0.761353in}{0.938773in}}%
\pgfpathlineto{\pgfqpoint{0.768755in}{0.937565in}}%
\pgfpathlineto{\pgfqpoint{0.779490in}{0.944411in}}%
\pgfpathlineto{\pgfqpoint{0.781180in}{0.954249in}}%
\pgfpathlineto{\pgfqpoint{0.768088in}{0.966183in}}%
\pgfpathlineto{\pgfqpoint{0.777399in}{0.989093in}}%
\pgfpathlineto{\pgfqpoint{0.786761in}{0.997904in}}%
\pgfpathlineto{\pgfqpoint{0.791204in}{1.008357in}}%
\pgfpathlineto{\pgfqpoint{0.793942in}{1.027523in}}%
\pgfpathlineto{\pgfqpoint{0.802721in}{1.036203in}}%
\pgfpathlineto{\pgfqpoint{0.821146in}{1.044872in}}%
\pgfpathlineto{\pgfqpoint{0.820739in}{1.052510in}}%
\pgfpathlineto{\pgfqpoint{0.810448in}{1.061990in}}%
\pgfpathlineto{\pgfqpoint{0.809047in}{1.081533in}}%
\pgfpathlineto{\pgfqpoint{0.801962in}{1.095819in}}%
\pgfpathlineto{\pgfqpoint{0.804320in}{1.104789in}}%
\pgfusepath{stroke}%
\end{pgfscope}%
\begin{pgfscope}%
\pgfpathrectangle{\pgfqpoint{0.100000in}{0.100000in}}{\pgfqpoint{3.420221in}{2.189500in}}%
\pgfusepath{clip}%
\pgfsetbuttcap%
\pgfsetroundjoin%
\pgfsetlinewidth{0.050187pt}%
\definecolor{currentstroke}{rgb}{1.000000,1.000000,1.000000}%
\pgfsetstrokecolor{currentstroke}%
\pgfsetdash{}{0pt}%
\pgfpathmoveto{\pgfqpoint{2.034882in}{0.872325in}}%
\pgfpathlineto{\pgfqpoint{2.018253in}{0.877468in}}%
\pgfpathlineto{\pgfqpoint{1.996946in}{0.893785in}}%
\pgfpathlineto{\pgfqpoint{1.987482in}{0.897292in}}%
\pgfpathlineto{\pgfqpoint{1.981468in}{0.890245in}}%
\pgfpathlineto{\pgfqpoint{1.973874in}{0.889889in}}%
\pgfpathlineto{\pgfqpoint{1.964265in}{0.895872in}}%
\pgfpathlineto{\pgfqpoint{1.949188in}{0.888206in}}%
\pgfpathlineto{\pgfqpoint{1.943820in}{0.891981in}}%
\pgfpathlineto{\pgfqpoint{1.928984in}{0.883037in}}%
\pgfpathlineto{\pgfqpoint{1.913332in}{0.884687in}}%
\pgfpathlineto{\pgfqpoint{1.901260in}{0.890500in}}%
\pgfpathlineto{\pgfqpoint{1.892805in}{0.888312in}}%
\pgfpathlineto{\pgfqpoint{1.879272in}{0.896532in}}%
\pgfpathlineto{\pgfqpoint{1.871744in}{0.882035in}}%
\pgfpathlineto{\pgfqpoint{1.864045in}{0.894502in}}%
\pgfpathlineto{\pgfqpoint{1.848839in}{0.889665in}}%
\pgfpathlineto{\pgfqpoint{1.835539in}{0.901507in}}%
\pgfpathlineto{\pgfqpoint{1.823916in}{0.891962in}}%
\pgfpathlineto{\pgfqpoint{1.818096in}{0.900734in}}%
\pgfpathlineto{\pgfqpoint{1.809707in}{0.903582in}}%
\pgfpathlineto{\pgfqpoint{1.804597in}{0.912039in}}%
\pgfpathlineto{\pgfqpoint{1.793624in}{0.914446in}}%
\pgfpathlineto{\pgfqpoint{1.787293in}{0.908080in}}%
\pgfpathlineto{\pgfqpoint{1.776535in}{0.916330in}}%
\pgfpathlineto{\pgfqpoint{1.771520in}{0.914446in}}%
\pgfpathlineto{\pgfqpoint{1.753691in}{0.921130in}}%
\pgfpathlineto{\pgfqpoint{1.742524in}{0.921876in}}%
\pgfpathlineto{\pgfqpoint{1.737525in}{0.936072in}}%
\pgfpathlineto{\pgfqpoint{1.728573in}{0.934304in}}%
\pgfpathlineto{\pgfqpoint{1.718320in}{0.937807in}}%
\pgfpathlineto{\pgfqpoint{1.711568in}{0.935783in}}%
\pgfpathlineto{\pgfqpoint{1.697088in}{0.951726in}}%
\pgfpathlineto{\pgfqpoint{1.693054in}{0.950683in}}%
\pgfpathlineto{\pgfqpoint{1.696718in}{1.015338in}}%
\pgfpathlineto{\pgfqpoint{1.700710in}{1.095364in}}%
\pgfpathlineto{\pgfqpoint{1.635182in}{1.099029in}}%
\pgfpathlineto{\pgfqpoint{1.570519in}{1.103918in}}%
\pgfpathlineto{\pgfqpoint{1.520562in}{1.108256in}}%
\pgfpathlineto{\pgfqpoint{1.523879in}{1.146061in}}%
\pgfpathlineto{\pgfqpoint{1.581134in}{1.140598in}}%
\pgfpathlineto{\pgfqpoint{1.646071in}{1.136526in}}%
\pgfpathlineto{\pgfqpoint{1.736509in}{1.131193in}}%
\pgfpathlineto{\pgfqpoint{1.790191in}{1.129122in}}%
\pgfpathlineto{\pgfqpoint{1.854502in}{1.126853in}}%
\pgfpathlineto{\pgfqpoint{1.941863in}{1.125324in}}%
\pgfpathlineto{\pgfqpoint{2.024824in}{1.124942in}}%
\pgfpathlineto{\pgfqpoint{2.025171in}{1.086881in}}%
\pgfpathlineto{\pgfqpoint{2.029828in}{1.058207in}}%
\pgfpathlineto{\pgfqpoint{2.037062in}{1.005246in}}%
\pgfpathlineto{\pgfqpoint{2.035570in}{0.914803in}}%
\pgfpathlineto{\pgfqpoint{2.034882in}{0.872325in}}%
\pgfusepath{stroke}%
\end{pgfscope}%
\begin{pgfscope}%
\pgfpathrectangle{\pgfqpoint{0.100000in}{0.100000in}}{\pgfqpoint{3.420221in}{2.189500in}}%
\pgfusepath{clip}%
\pgfsetbuttcap%
\pgfsetroundjoin%
\pgfsetlinewidth{0.050187pt}%
\definecolor{currentstroke}{rgb}{1.000000,1.000000,1.000000}%
\pgfsetstrokecolor{currentstroke}%
\pgfsetdash{}{0pt}%
\pgfpathmoveto{\pgfqpoint{2.657497in}{1.012784in}}%
\pgfpathlineto{\pgfqpoint{2.657547in}{1.029542in}}%
\pgfpathlineto{\pgfqpoint{2.672109in}{1.035977in}}%
\pgfpathlineto{\pgfqpoint{2.672720in}{1.046264in}}%
\pgfpathlineto{\pgfqpoint{2.685758in}{1.058830in}}%
\pgfpathlineto{\pgfqpoint{2.702057in}{1.061404in}}%
\pgfpathlineto{\pgfqpoint{2.716385in}{1.075253in}}%
\pgfpathlineto{\pgfqpoint{2.731349in}{1.081443in}}%
\pgfpathlineto{\pgfqpoint{2.742580in}{1.100027in}}%
\pgfpathlineto{\pgfqpoint{2.752808in}{1.098677in}}%
\pgfpathlineto{\pgfqpoint{2.774426in}{1.115609in}}%
\pgfpathlineto{\pgfqpoint{2.785841in}{1.115928in}}%
\pgfpathlineto{\pgfqpoint{2.790662in}{1.128830in}}%
\pgfpathlineto{\pgfqpoint{2.799735in}{1.137812in}}%
\pgfpathlineto{\pgfqpoint{2.801811in}{1.153116in}}%
\pgfpathlineto{\pgfqpoint{2.821335in}{1.154764in}}%
\pgfpathlineto{\pgfqpoint{2.900263in}{1.165189in}}%
\pgfpathlineto{\pgfqpoint{2.993198in}{1.181198in}}%
\pgfpathlineto{\pgfqpoint{3.060414in}{1.194956in}}%
\pgfpathlineto{\pgfqpoint{3.135564in}{1.210544in}}%
\pgfpathlineto{\pgfqpoint{3.157037in}{1.181062in}}%
\pgfpathlineto{\pgfqpoint{3.145406in}{1.182944in}}%
\pgfpathlineto{\pgfqpoint{3.130474in}{1.179306in}}%
\pgfpathlineto{\pgfqpoint{3.115762in}{1.164281in}}%
\pgfpathlineto{\pgfqpoint{3.105187in}{1.165362in}}%
\pgfpathlineto{\pgfqpoint{3.103850in}{1.156438in}}%
\pgfpathlineto{\pgfqpoint{3.122962in}{1.163495in}}%
\pgfpathlineto{\pgfqpoint{3.142195in}{1.166405in}}%
\pgfpathlineto{\pgfqpoint{3.149336in}{1.162533in}}%
\pgfpathlineto{\pgfqpoint{3.158947in}{1.166947in}}%
\pgfpathlineto{\pgfqpoint{3.163906in}{1.163854in}}%
\pgfpathlineto{\pgfqpoint{3.168295in}{1.149127in}}%
\pgfpathlineto{\pgfqpoint{3.159161in}{1.144567in}}%
\pgfpathlineto{\pgfqpoint{3.152891in}{1.126609in}}%
\pgfpathlineto{\pgfqpoint{3.146318in}{1.119617in}}%
\pgfpathlineto{\pgfqpoint{3.126168in}{1.121147in}}%
\pgfpathlineto{\pgfqpoint{3.127250in}{1.131691in}}%
\pgfpathlineto{\pgfqpoint{3.116246in}{1.127085in}}%
\pgfpathlineto{\pgfqpoint{3.113852in}{1.118291in}}%
\pgfpathlineto{\pgfqpoint{3.121395in}{1.109181in}}%
\pgfpathlineto{\pgfqpoint{3.123958in}{1.093720in}}%
\pgfpathlineto{\pgfqpoint{3.111615in}{1.084910in}}%
\pgfpathlineto{\pgfqpoint{3.124964in}{1.081811in}}%
\pgfpathlineto{\pgfqpoint{3.131005in}{1.088289in}}%
\pgfpathlineto{\pgfqpoint{3.144361in}{1.089078in}}%
\pgfpathlineto{\pgfqpoint{3.137497in}{1.075090in}}%
\pgfpathlineto{\pgfqpoint{3.129090in}{1.068349in}}%
\pgfpathlineto{\pgfqpoint{3.103661in}{1.061601in}}%
\pgfpathlineto{\pgfqpoint{3.077551in}{1.037921in}}%
\pgfpathlineto{\pgfqpoint{3.061451in}{1.014588in}}%
\pgfpathlineto{\pgfqpoint{3.061356in}{1.005112in}}%
\pgfpathlineto{\pgfqpoint{3.054925in}{0.992036in}}%
\pgfpathlineto{\pgfqpoint{3.021920in}{0.983433in}}%
\pgfpathlineto{\pgfqpoint{2.943428in}{1.039700in}}%
\pgfpathlineto{\pgfqpoint{2.874304in}{1.029543in}}%
\pgfpathlineto{\pgfqpoint{2.873725in}{1.038921in}}%
\pgfpathlineto{\pgfqpoint{2.863219in}{1.049482in}}%
\pgfpathlineto{\pgfqpoint{2.855265in}{1.052067in}}%
\pgfpathlineto{\pgfqpoint{2.780161in}{1.044263in}}%
\pgfpathlineto{\pgfqpoint{2.762899in}{1.038409in}}%
\pgfpathlineto{\pgfqpoint{2.731730in}{1.022904in}}%
\pgfpathlineto{\pgfqpoint{2.704793in}{1.018612in}}%
\pgfpathlineto{\pgfqpoint{2.657497in}{1.012784in}}%
\pgfusepath{stroke}%
\end{pgfscope}%
\begin{pgfscope}%
\pgfpathrectangle{\pgfqpoint{0.100000in}{0.100000in}}{\pgfqpoint{3.420221in}{2.189500in}}%
\pgfusepath{clip}%
\pgfsetbuttcap%
\pgfsetroundjoin%
\pgfsetlinewidth{0.050187pt}%
\definecolor{currentstroke}{rgb}{1.000000,1.000000,1.000000}%
\pgfsetstrokecolor{currentstroke}%
\pgfsetdash{}{0pt}%
\pgfpathmoveto{\pgfqpoint{2.657497in}{1.012784in}}%
\pgfpathlineto{\pgfqpoint{2.578893in}{1.004151in}}%
\pgfpathlineto{\pgfqpoint{2.506986in}{0.997685in}}%
\pgfpathlineto{\pgfqpoint{2.433080in}{0.992905in}}%
\pgfpathlineto{\pgfqpoint{2.420394in}{0.991066in}}%
\pgfpathlineto{\pgfqpoint{2.370574in}{0.987235in}}%
\pgfpathlineto{\pgfqpoint{2.290779in}{0.982578in}}%
\pgfpathlineto{\pgfqpoint{2.304976in}{0.994362in}}%
\pgfpathlineto{\pgfqpoint{2.301736in}{1.006767in}}%
\pgfpathlineto{\pgfqpoint{2.304874in}{1.021803in}}%
\pgfpathlineto{\pgfqpoint{2.309668in}{1.028882in}}%
\pgfpathlineto{\pgfqpoint{2.309488in}{1.038731in}}%
\pgfpathlineto{\pgfqpoint{2.322254in}{1.044926in}}%
\pgfpathlineto{\pgfqpoint{2.322276in}{1.059420in}}%
\pgfpathlineto{\pgfqpoint{2.324030in}{1.076128in}}%
\pgfpathlineto{\pgfqpoint{2.332931in}{1.086806in}}%
\pgfpathlineto{\pgfqpoint{2.331454in}{1.097399in}}%
\pgfpathlineto{\pgfqpoint{2.334904in}{1.097640in}}%
\pgfpathlineto{\pgfqpoint{2.338563in}{1.098170in}}%
\pgfpathlineto{\pgfqpoint{2.421106in}{1.103416in}}%
\pgfpathlineto{\pgfqpoint{2.418343in}{1.116976in}}%
\pgfpathlineto{\pgfqpoint{2.431626in}{1.115137in}}%
\pgfpathlineto{\pgfqpoint{2.512026in}{1.123090in}}%
\pgfpathlineto{\pgfqpoint{2.576562in}{1.126742in}}%
\pgfpathlineto{\pgfqpoint{2.615890in}{1.130050in}}%
\pgfpathlineto{\pgfqpoint{2.681015in}{1.136225in}}%
\pgfpathlineto{\pgfqpoint{2.682101in}{1.137453in}}%
\pgfpathlineto{\pgfqpoint{2.709926in}{1.140209in}}%
\pgfpathlineto{\pgfqpoint{2.801811in}{1.153116in}}%
\pgfpathlineto{\pgfqpoint{2.799735in}{1.137812in}}%
\pgfpathlineto{\pgfqpoint{2.790662in}{1.128830in}}%
\pgfpathlineto{\pgfqpoint{2.785841in}{1.115928in}}%
\pgfpathlineto{\pgfqpoint{2.774426in}{1.115609in}}%
\pgfpathlineto{\pgfqpoint{2.752808in}{1.098677in}}%
\pgfpathlineto{\pgfqpoint{2.742580in}{1.100027in}}%
\pgfpathlineto{\pgfqpoint{2.731349in}{1.081443in}}%
\pgfpathlineto{\pgfqpoint{2.716385in}{1.075253in}}%
\pgfpathlineto{\pgfqpoint{2.702057in}{1.061404in}}%
\pgfpathlineto{\pgfqpoint{2.685758in}{1.058830in}}%
\pgfpathlineto{\pgfqpoint{2.672720in}{1.046264in}}%
\pgfpathlineto{\pgfqpoint{2.672109in}{1.035977in}}%
\pgfpathlineto{\pgfqpoint{2.657547in}{1.029542in}}%
\pgfpathlineto{\pgfqpoint{2.657497in}{1.012784in}}%
\pgfusepath{stroke}%
\end{pgfscope}%
\begin{pgfscope}%
\pgfpathrectangle{\pgfqpoint{0.100000in}{0.100000in}}{\pgfqpoint{3.420221in}{2.189500in}}%
\pgfusepath{clip}%
\pgfsetbuttcap%
\pgfsetroundjoin%
\pgfsetlinewidth{0.050187pt}%
\definecolor{currentstroke}{rgb}{1.000000,1.000000,1.000000}%
\pgfsetstrokecolor{currentstroke}%
\pgfsetdash{}{0pt}%
\pgfpathmoveto{\pgfqpoint{1.263435in}{0.780071in}}%
\pgfpathlineto{\pgfqpoint{1.257828in}{0.789079in}}%
\pgfpathlineto{\pgfqpoint{1.260148in}{0.796838in}}%
\pgfpathlineto{\pgfqpoint{1.372826in}{0.783628in}}%
\pgfpathlineto{\pgfqpoint{1.486996in}{0.772248in}}%
\pgfpathlineto{\pgfqpoint{1.490325in}{0.810842in}}%
\pgfpathlineto{\pgfqpoint{1.507485in}{0.977778in}}%
\pgfpathlineto{\pgfqpoint{1.520562in}{1.108256in}}%
\pgfpathlineto{\pgfqpoint{1.635182in}{1.099029in}}%
\pgfpathlineto{\pgfqpoint{1.700710in}{1.095364in}}%
\pgfpathlineto{\pgfqpoint{1.693054in}{0.950683in}}%
\pgfpathlineto{\pgfqpoint{1.697088in}{0.951726in}}%
\pgfpathlineto{\pgfqpoint{1.711568in}{0.935783in}}%
\pgfpathlineto{\pgfqpoint{1.718320in}{0.937807in}}%
\pgfpathlineto{\pgfqpoint{1.728573in}{0.934304in}}%
\pgfpathlineto{\pgfqpoint{1.737525in}{0.936072in}}%
\pgfpathlineto{\pgfqpoint{1.742524in}{0.921876in}}%
\pgfpathlineto{\pgfqpoint{1.753691in}{0.921130in}}%
\pgfpathlineto{\pgfqpoint{1.771520in}{0.914446in}}%
\pgfpathlineto{\pgfqpoint{1.776535in}{0.916330in}}%
\pgfpathlineto{\pgfqpoint{1.787293in}{0.908080in}}%
\pgfpathlineto{\pgfqpoint{1.793624in}{0.914446in}}%
\pgfpathlineto{\pgfqpoint{1.804597in}{0.912039in}}%
\pgfpathlineto{\pgfqpoint{1.809707in}{0.903582in}}%
\pgfpathlineto{\pgfqpoint{1.818096in}{0.900734in}}%
\pgfpathlineto{\pgfqpoint{1.823916in}{0.891962in}}%
\pgfpathlineto{\pgfqpoint{1.835539in}{0.901507in}}%
\pgfpathlineto{\pgfqpoint{1.848839in}{0.889665in}}%
\pgfpathlineto{\pgfqpoint{1.864045in}{0.894502in}}%
\pgfpathlineto{\pgfqpoint{1.871744in}{0.882035in}}%
\pgfpathlineto{\pgfqpoint{1.879272in}{0.896532in}}%
\pgfpathlineto{\pgfqpoint{1.892805in}{0.888312in}}%
\pgfpathlineto{\pgfqpoint{1.901260in}{0.890500in}}%
\pgfpathlineto{\pgfqpoint{1.913332in}{0.884687in}}%
\pgfpathlineto{\pgfqpoint{1.928984in}{0.883037in}}%
\pgfpathlineto{\pgfqpoint{1.943820in}{0.891981in}}%
\pgfpathlineto{\pgfqpoint{1.949188in}{0.888206in}}%
\pgfpathlineto{\pgfqpoint{1.964265in}{0.895872in}}%
\pgfpathlineto{\pgfqpoint{1.973874in}{0.889889in}}%
\pgfpathlineto{\pgfqpoint{1.981468in}{0.890245in}}%
\pgfpathlineto{\pgfqpoint{1.987482in}{0.897292in}}%
\pgfpathlineto{\pgfqpoint{1.996946in}{0.893785in}}%
\pgfpathlineto{\pgfqpoint{2.018253in}{0.877468in}}%
\pgfpathlineto{\pgfqpoint{2.034882in}{0.872325in}}%
\pgfpathlineto{\pgfqpoint{2.041550in}{0.866025in}}%
\pgfpathlineto{\pgfqpoint{2.049897in}{0.869456in}}%
\pgfpathlineto{\pgfqpoint{2.062545in}{0.866828in}}%
\pgfpathlineto{\pgfqpoint{2.063848in}{0.749168in}}%
\pgfpathlineto{\pgfqpoint{2.072632in}{0.741715in}}%
\pgfpathlineto{\pgfqpoint{2.079673in}{0.727982in}}%
\pgfpathlineto{\pgfqpoint{2.077198in}{0.718794in}}%
\pgfpathlineto{\pgfqpoint{2.082727in}{0.714903in}}%
\pgfpathlineto{\pgfqpoint{2.087220in}{0.697960in}}%
\pgfpathlineto{\pgfqpoint{2.095826in}{0.688886in}}%
\pgfpathlineto{\pgfqpoint{2.097754in}{0.669621in}}%
\pgfpathlineto{\pgfqpoint{2.096257in}{0.661464in}}%
\pgfpathlineto{\pgfqpoint{2.084560in}{0.639878in}}%
\pgfpathlineto{\pgfqpoint{2.083215in}{0.624358in}}%
\pgfpathlineto{\pgfqpoint{2.087183in}{0.621118in}}%
\pgfpathlineto{\pgfqpoint{2.087528in}{0.607521in}}%
\pgfpathlineto{\pgfqpoint{2.083494in}{0.598975in}}%
\pgfpathlineto{\pgfqpoint{2.077222in}{0.595756in}}%
\pgfpathlineto{\pgfqpoint{2.071080in}{0.584589in}}%
\pgfpathlineto{\pgfqpoint{2.078908in}{0.573772in}}%
\pgfpathlineto{\pgfqpoint{2.063713in}{0.573559in}}%
\pgfpathlineto{\pgfqpoint{2.023082in}{0.554970in}}%
\pgfpathlineto{\pgfqpoint{2.030839in}{0.566095in}}%
\pgfpathlineto{\pgfqpoint{2.021446in}{0.572093in}}%
\pgfpathlineto{\pgfqpoint{2.019483in}{0.582291in}}%
\pgfpathlineto{\pgfqpoint{2.001149in}{0.564528in}}%
\pgfpathlineto{\pgfqpoint{2.009286in}{0.552387in}}%
\pgfpathlineto{\pgfqpoint{1.997679in}{0.536928in}}%
\pgfpathlineto{\pgfqpoint{1.991433in}{0.537251in}}%
\pgfpathlineto{\pgfqpoint{1.985556in}{0.520416in}}%
\pgfpathlineto{\pgfqpoint{1.966973in}{0.507173in}}%
\pgfpathlineto{\pgfqpoint{1.949611in}{0.502401in}}%
\pgfpathlineto{\pgfqpoint{1.919330in}{0.489985in}}%
\pgfpathlineto{\pgfqpoint{1.916192in}{0.496911in}}%
\pgfpathlineto{\pgfqpoint{1.897329in}{0.490536in}}%
\pgfpathlineto{\pgfqpoint{1.908931in}{0.479681in}}%
\pgfpathlineto{\pgfqpoint{1.890631in}{0.470252in}}%
\pgfpathlineto{\pgfqpoint{1.870892in}{0.456014in}}%
\pgfpathlineto{\pgfqpoint{1.866152in}{0.462633in}}%
\pgfpathlineto{\pgfqpoint{1.849998in}{0.452713in}}%
\pgfpathlineto{\pgfqpoint{1.865831in}{0.448946in}}%
\pgfpathlineto{\pgfqpoint{1.853893in}{0.433350in}}%
\pgfpathlineto{\pgfqpoint{1.848067in}{0.437991in}}%
\pgfpathlineto{\pgfqpoint{1.840240in}{0.430536in}}%
\pgfpathlineto{\pgfqpoint{1.849994in}{0.424024in}}%
\pgfpathlineto{\pgfqpoint{1.844227in}{0.414506in}}%
\pgfpathlineto{\pgfqpoint{1.837139in}{0.391974in}}%
\pgfpathlineto{\pgfqpoint{1.826915in}{0.370272in}}%
\pgfpathlineto{\pgfqpoint{1.831505in}{0.356047in}}%
\pgfpathlineto{\pgfqpoint{1.839307in}{0.322510in}}%
\pgfpathlineto{\pgfqpoint{1.840009in}{0.308937in}}%
\pgfpathlineto{\pgfqpoint{1.852063in}{0.291135in}}%
\pgfpathlineto{\pgfqpoint{1.842768in}{0.292175in}}%
\pgfpathlineto{\pgfqpoint{1.833739in}{0.283194in}}%
\pgfpathlineto{\pgfqpoint{1.823165in}{0.290253in}}%
\pgfpathlineto{\pgfqpoint{1.819369in}{0.297294in}}%
\pgfpathlineto{\pgfqpoint{1.804320in}{0.300570in}}%
\pgfpathlineto{\pgfqpoint{1.781332in}{0.300972in}}%
\pgfpathlineto{\pgfqpoint{1.764394in}{0.314307in}}%
\pgfpathlineto{\pgfqpoint{1.749014in}{0.316534in}}%
\pgfpathlineto{\pgfqpoint{1.739684in}{0.327132in}}%
\pgfpathlineto{\pgfqpoint{1.720143in}{0.331397in}}%
\pgfpathlineto{\pgfqpoint{1.709459in}{0.365473in}}%
\pgfpathlineto{\pgfqpoint{1.698534in}{0.379124in}}%
\pgfpathlineto{\pgfqpoint{1.700390in}{0.392095in}}%
\pgfpathlineto{\pgfqpoint{1.693616in}{0.401574in}}%
\pgfpathlineto{\pgfqpoint{1.697867in}{0.414531in}}%
\pgfpathlineto{\pgfqpoint{1.694360in}{0.424025in}}%
\pgfpathlineto{\pgfqpoint{1.683382in}{0.428327in}}%
\pgfpathlineto{\pgfqpoint{1.673117in}{0.439275in}}%
\pgfpathlineto{\pgfqpoint{1.669376in}{0.453933in}}%
\pgfpathlineto{\pgfqpoint{1.659663in}{0.467254in}}%
\pgfpathlineto{\pgfqpoint{1.646734in}{0.477619in}}%
\pgfpathlineto{\pgfqpoint{1.635089in}{0.507351in}}%
\pgfpathlineto{\pgfqpoint{1.625686in}{0.539868in}}%
\pgfpathlineto{\pgfqpoint{1.617970in}{0.552725in}}%
\pgfpathlineto{\pgfqpoint{1.604595in}{0.563550in}}%
\pgfpathlineto{\pgfqpoint{1.601252in}{0.571406in}}%
\pgfpathlineto{\pgfqpoint{1.588728in}{0.576285in}}%
\pgfpathlineto{\pgfqpoint{1.576367in}{0.597102in}}%
\pgfpathlineto{\pgfqpoint{1.565079in}{0.595504in}}%
\pgfpathlineto{\pgfqpoint{1.553616in}{0.600703in}}%
\pgfpathlineto{\pgfqpoint{1.537368in}{0.599698in}}%
\pgfpathlineto{\pgfqpoint{1.520861in}{0.608278in}}%
\pgfpathlineto{\pgfqpoint{1.516207in}{0.600118in}}%
\pgfpathlineto{\pgfqpoint{1.496916in}{0.599914in}}%
\pgfpathlineto{\pgfqpoint{1.487095in}{0.584444in}}%
\pgfpathlineto{\pgfqpoint{1.478554in}{0.565291in}}%
\pgfpathlineto{\pgfqpoint{1.460390in}{0.544731in}}%
\pgfpathlineto{\pgfqpoint{1.439812in}{0.553754in}}%
\pgfpathlineto{\pgfqpoint{1.436891in}{0.559717in}}%
\pgfpathlineto{\pgfqpoint{1.424386in}{0.564265in}}%
\pgfpathlineto{\pgfqpoint{1.420544in}{0.570486in}}%
\pgfpathlineto{\pgfqpoint{1.403942in}{0.576766in}}%
\pgfpathlineto{\pgfqpoint{1.394655in}{0.589604in}}%
\pgfpathlineto{\pgfqpoint{1.383809in}{0.595798in}}%
\pgfpathlineto{\pgfqpoint{1.374477in}{0.606606in}}%
\pgfpathlineto{\pgfqpoint{1.367207in}{0.624911in}}%
\pgfpathlineto{\pgfqpoint{1.368021in}{0.649925in}}%
\pgfpathlineto{\pgfqpoint{1.359492in}{0.662565in}}%
\pgfpathlineto{\pgfqpoint{1.358511in}{0.676255in}}%
\pgfpathlineto{\pgfqpoint{1.339589in}{0.696752in}}%
\pgfpathlineto{\pgfqpoint{1.328579in}{0.701128in}}%
\pgfpathlineto{\pgfqpoint{1.316895in}{0.720253in}}%
\pgfpathlineto{\pgfqpoint{1.306939in}{0.727852in}}%
\pgfpathlineto{\pgfqpoint{1.294289in}{0.746356in}}%
\pgfpathlineto{\pgfqpoint{1.281345in}{0.754364in}}%
\pgfpathlineto{\pgfqpoint{1.272869in}{0.774883in}}%
\pgfpathlineto{\pgfqpoint{1.263435in}{0.780071in}}%
\pgfpathlineto{\pgfqpoint{1.263435in}{0.780071in}}%
\pgfusepath{stroke}%
\end{pgfscope}%
\begin{pgfscope}%
\pgfpathrectangle{\pgfqpoint{0.100000in}{0.100000in}}{\pgfqpoint{3.420221in}{2.189500in}}%
\pgfusepath{clip}%
\pgfsetbuttcap%
\pgfsetroundjoin%
\pgfsetlinewidth{0.050187pt}%
\definecolor{currentstroke}{rgb}{1.000000,1.000000,1.000000}%
\pgfsetstrokecolor{currentstroke}%
\pgfsetdash{}{0pt}%
\pgfpathmoveto{\pgfqpoint{1.164318in}{1.189748in}}%
\pgfpathlineto{\pgfqpoint{1.257697in}{1.176187in}}%
\pgfpathlineto{\pgfqpoint{1.293872in}{1.170613in}}%
\pgfpathlineto{\pgfqpoint{1.395847in}{1.158485in}}%
\pgfpathlineto{\pgfqpoint{1.464482in}{1.151325in}}%
\pgfpathlineto{\pgfqpoint{1.523879in}{1.146061in}}%
\pgfpathlineto{\pgfqpoint{1.520562in}{1.108256in}}%
\pgfpathlineto{\pgfqpoint{1.518762in}{1.108357in}}%
\pgfpathlineto{\pgfqpoint{1.513322in}{1.043470in}}%
\pgfpathlineto{\pgfqpoint{1.507485in}{0.977778in}}%
\pgfpathlineto{\pgfqpoint{1.500733in}{0.909008in}}%
\pgfpathlineto{\pgfqpoint{1.490325in}{0.810842in}}%
\pgfpathlineto{\pgfqpoint{1.486996in}{0.772248in}}%
\pgfpathlineto{\pgfqpoint{1.425807in}{0.778422in}}%
\pgfpathlineto{\pgfqpoint{1.372826in}{0.783628in}}%
\pgfpathlineto{\pgfqpoint{1.299554in}{0.791951in}}%
\pgfpathlineto{\pgfqpoint{1.260148in}{0.796838in}}%
\pgfpathlineto{\pgfqpoint{1.257828in}{0.789079in}}%
\pgfpathlineto{\pgfqpoint{1.263435in}{0.780071in}}%
\pgfpathlineto{\pgfqpoint{1.216076in}{0.786221in}}%
\pgfpathlineto{\pgfqpoint{1.157645in}{0.794607in}}%
\pgfpathlineto{\pgfqpoint{1.152328in}{0.761559in}}%
\pgfpathlineto{\pgfqpoint{1.099030in}{0.769581in}}%
\pgfpathlineto{\pgfqpoint{1.111553in}{0.851566in}}%
\pgfpathlineto{\pgfqpoint{1.127055in}{0.951388in}}%
\pgfpathlineto{\pgfqpoint{1.140839in}{1.038349in}}%
\pgfpathlineto{\pgfqpoint{1.152809in}{1.115645in}}%
\pgfpathlineto{\pgfqpoint{1.164318in}{1.189748in}}%
\pgfusepath{stroke}%
\end{pgfscope}%
\begin{pgfscope}%
\pgfpathrectangle{\pgfqpoint{0.100000in}{0.100000in}}{\pgfqpoint{3.420221in}{2.189500in}}%
\pgfusepath{clip}%
\pgfsetbuttcap%
\pgfsetroundjoin%
\pgfsetlinewidth{0.050187pt}%
\definecolor{currentstroke}{rgb}{1.000000,1.000000,1.000000}%
\pgfsetstrokecolor{currentstroke}%
\pgfsetdash{}{0pt}%
\pgfpathmoveto{\pgfqpoint{2.649050in}{0.709406in}}%
\pgfpathlineto{\pgfqpoint{2.559966in}{0.699511in}}%
\pgfpathlineto{\pgfqpoint{2.481440in}{0.693415in}}%
\pgfpathlineto{\pgfqpoint{2.480457in}{0.683792in}}%
\pgfpathlineto{\pgfqpoint{2.487670in}{0.674633in}}%
\pgfpathlineto{\pgfqpoint{2.496474in}{0.669252in}}%
\pgfpathlineto{\pgfqpoint{2.496340in}{0.655084in}}%
\pgfpathlineto{\pgfqpoint{2.494008in}{0.645617in}}%
\pgfpathlineto{\pgfqpoint{2.486238in}{0.638796in}}%
\pgfpathlineto{\pgfqpoint{2.475407in}{0.639518in}}%
\pgfpathlineto{\pgfqpoint{2.465168in}{0.647971in}}%
\pgfpathlineto{\pgfqpoint{2.463343in}{0.663012in}}%
\pgfpathlineto{\pgfqpoint{2.455716in}{0.671760in}}%
\pgfpathlineto{\pgfqpoint{2.450525in}{0.640451in}}%
\pgfpathlineto{\pgfqpoint{2.432887in}{0.643426in}}%
\pgfpathlineto{\pgfqpoint{2.426753in}{0.698174in}}%
\pgfpathlineto{\pgfqpoint{2.420110in}{0.755872in}}%
\pgfpathlineto{\pgfqpoint{2.422577in}{0.839116in}}%
\pgfpathlineto{\pgfqpoint{2.424108in}{0.896240in}}%
\pgfpathlineto{\pgfqpoint{2.427365in}{0.983397in}}%
\pgfpathlineto{\pgfqpoint{2.420394in}{0.991066in}}%
\pgfpathlineto{\pgfqpoint{2.433080in}{0.992905in}}%
\pgfpathlineto{\pgfqpoint{2.506986in}{0.997685in}}%
\pgfpathlineto{\pgfqpoint{2.578893in}{1.004151in}}%
\pgfpathlineto{\pgfqpoint{2.597811in}{0.937884in}}%
\pgfpathlineto{\pgfqpoint{2.610714in}{0.889244in}}%
\pgfpathlineto{\pgfqpoint{2.622279in}{0.848523in}}%
\pgfpathlineto{\pgfqpoint{2.628968in}{0.832126in}}%
\pgfpathlineto{\pgfqpoint{2.639499in}{0.816867in}}%
\pgfpathlineto{\pgfqpoint{2.637798in}{0.809102in}}%
\pgfpathlineto{\pgfqpoint{2.645322in}{0.805319in}}%
\pgfpathlineto{\pgfqpoint{2.636431in}{0.793548in}}%
\pgfpathlineto{\pgfqpoint{2.633428in}{0.772586in}}%
\pgfpathlineto{\pgfqpoint{2.642117in}{0.748069in}}%
\pgfpathlineto{\pgfqpoint{2.640910in}{0.723398in}}%
\pgfpathlineto{\pgfqpoint{2.649050in}{0.709406in}}%
\pgfusepath{stroke}%
\end{pgfscope}%
\begin{pgfscope}%
\pgfpathrectangle{\pgfqpoint{0.100000in}{0.100000in}}{\pgfqpoint{3.420221in}{2.189500in}}%
\pgfusepath{clip}%
\pgfsetbuttcap%
\pgfsetroundjoin%
\pgfsetlinewidth{0.050187pt}%
\definecolor{currentstroke}{rgb}{1.000000,1.000000,1.000000}%
\pgfsetstrokecolor{currentstroke}%
\pgfsetdash{}{0pt}%
\pgfpathmoveto{\pgfqpoint{2.432887in}{0.643426in}}%
\pgfpathlineto{\pgfqpoint{2.414786in}{0.638247in}}%
\pgfpathlineto{\pgfqpoint{2.401294in}{0.641572in}}%
\pgfpathlineto{\pgfqpoint{2.376238in}{0.633603in}}%
\pgfpathlineto{\pgfqpoint{2.357340in}{0.623338in}}%
\pgfpathlineto{\pgfqpoint{2.349587in}{0.641886in}}%
\pgfpathlineto{\pgfqpoint{2.341594in}{0.649660in}}%
\pgfpathlineto{\pgfqpoint{2.337738in}{0.658963in}}%
\pgfpathlineto{\pgfqpoint{2.343390in}{0.684142in}}%
\pgfpathlineto{\pgfqpoint{2.289837in}{0.680845in}}%
\pgfpathlineto{\pgfqpoint{2.220395in}{0.677834in}}%
\pgfpathlineto{\pgfqpoint{2.224526in}{0.684102in}}%
\pgfpathlineto{\pgfqpoint{2.219420in}{0.695940in}}%
\pgfpathlineto{\pgfqpoint{2.224509in}{0.698352in}}%
\pgfpathlineto{\pgfqpoint{2.223380in}{0.709725in}}%
\pgfpathlineto{\pgfqpoint{2.232249in}{0.720755in}}%
\pgfpathlineto{\pgfqpoint{2.237101in}{0.742166in}}%
\pgfpathlineto{\pgfqpoint{2.253333in}{0.756273in}}%
\pgfpathlineto{\pgfqpoint{2.247317in}{0.769973in}}%
\pgfpathlineto{\pgfqpoint{2.258827in}{0.777859in}}%
\pgfpathlineto{\pgfqpoint{2.248899in}{0.792136in}}%
\pgfpathlineto{\pgfqpoint{2.241712in}{0.824818in}}%
\pgfpathlineto{\pgfqpoint{2.244438in}{0.830752in}}%
\pgfpathlineto{\pgfqpoint{2.248741in}{0.842135in}}%
\pgfpathlineto{\pgfqpoint{2.244736in}{0.854081in}}%
\pgfpathlineto{\pgfqpoint{2.246694in}{0.859516in}}%
\pgfpathlineto{\pgfqpoint{2.239648in}{0.880048in}}%
\pgfpathlineto{\pgfqpoint{2.260448in}{0.916916in}}%
\pgfpathlineto{\pgfqpoint{2.261415in}{0.926729in}}%
\pgfpathlineto{\pgfqpoint{2.271435in}{0.933864in}}%
\pgfpathlineto{\pgfqpoint{2.282126in}{0.957419in}}%
\pgfpathlineto{\pgfqpoint{2.281618in}{0.966994in}}%
\pgfpathlineto{\pgfqpoint{2.294934in}{0.976781in}}%
\pgfpathlineto{\pgfqpoint{2.290779in}{0.982578in}}%
\pgfpathlineto{\pgfqpoint{2.370574in}{0.987235in}}%
\pgfpathlineto{\pgfqpoint{2.420394in}{0.991066in}}%
\pgfpathlineto{\pgfqpoint{2.427365in}{0.983397in}}%
\pgfpathlineto{\pgfqpoint{2.424108in}{0.896240in}}%
\pgfpathlineto{\pgfqpoint{2.422577in}{0.839116in}}%
\pgfpathlineto{\pgfqpoint{2.420110in}{0.755872in}}%
\pgfpathlineto{\pgfqpoint{2.426753in}{0.698174in}}%
\pgfpathlineto{\pgfqpoint{2.432887in}{0.643426in}}%
\pgfusepath{stroke}%
\end{pgfscope}%
\begin{pgfscope}%
\pgfpathrectangle{\pgfqpoint{0.100000in}{0.100000in}}{\pgfqpoint{3.420221in}{2.189500in}}%
\pgfusepath{clip}%
\pgfsetbuttcap%
\pgfsetroundjoin%
\pgfsetlinewidth{0.050187pt}%
\definecolor{currentstroke}{rgb}{1.000000,1.000000,1.000000}%
\pgfsetstrokecolor{currentstroke}%
\pgfsetdash{}{0pt}%
\pgfpathmoveto{\pgfqpoint{2.578893in}{1.004151in}}%
\pgfpathlineto{\pgfqpoint{2.657497in}{1.012784in}}%
\pgfpathlineto{\pgfqpoint{2.704793in}{1.018612in}}%
\pgfpathlineto{\pgfqpoint{2.731730in}{1.022904in}}%
\pgfpathlineto{\pgfqpoint{2.720572in}{1.005304in}}%
\pgfpathlineto{\pgfqpoint{2.720579in}{0.997006in}}%
\pgfpathlineto{\pgfqpoint{2.732064in}{0.992532in}}%
\pgfpathlineto{\pgfqpoint{2.739875in}{0.985308in}}%
\pgfpathlineto{\pgfqpoint{2.751679in}{0.984415in}}%
\pgfpathlineto{\pgfqpoint{2.762712in}{0.964078in}}%
\pgfpathlineto{\pgfqpoint{2.774680in}{0.949737in}}%
\pgfpathlineto{\pgfqpoint{2.795934in}{0.937684in}}%
\pgfpathlineto{\pgfqpoint{2.801975in}{0.928070in}}%
\pgfpathlineto{\pgfqpoint{2.820929in}{0.917673in}}%
\pgfpathlineto{\pgfqpoint{2.822354in}{0.910133in}}%
\pgfpathlineto{\pgfqpoint{2.837730in}{0.895140in}}%
\pgfpathlineto{\pgfqpoint{2.849507in}{0.890871in}}%
\pgfpathlineto{\pgfqpoint{2.857847in}{0.876714in}}%
\pgfpathlineto{\pgfqpoint{2.861539in}{0.860802in}}%
\pgfpathlineto{\pgfqpoint{2.873772in}{0.854651in}}%
\pgfpathlineto{\pgfqpoint{2.883621in}{0.837547in}}%
\pgfpathlineto{\pgfqpoint{2.886797in}{0.824968in}}%
\pgfpathlineto{\pgfqpoint{2.901229in}{0.819613in}}%
\pgfpathlineto{\pgfqpoint{2.889119in}{0.796426in}}%
\pgfpathlineto{\pgfqpoint{2.884561in}{0.782723in}}%
\pgfpathlineto{\pgfqpoint{2.887535in}{0.776300in}}%
\pgfpathlineto{\pgfqpoint{2.882343in}{0.765625in}}%
\pgfpathlineto{\pgfqpoint{2.880136in}{0.750866in}}%
\pgfpathlineto{\pgfqpoint{2.870903in}{0.748117in}}%
\pgfpathlineto{\pgfqpoint{2.875778in}{0.734604in}}%
\pgfpathlineto{\pgfqpoint{2.875467in}{0.717461in}}%
\pgfpathlineto{\pgfqpoint{2.859056in}{0.718104in}}%
\pgfpathlineto{\pgfqpoint{2.847623in}{0.721232in}}%
\pgfpathlineto{\pgfqpoint{2.842676in}{0.715417in}}%
\pgfpathlineto{\pgfqpoint{2.846367in}{0.701681in}}%
\pgfpathlineto{\pgfqpoint{2.845565in}{0.685711in}}%
\pgfpathlineto{\pgfqpoint{2.838352in}{0.684490in}}%
\pgfpathlineto{\pgfqpoint{2.832517in}{0.699404in}}%
\pgfpathlineto{\pgfqpoint{2.773106in}{0.695445in}}%
\pgfpathlineto{\pgfqpoint{2.698194in}{0.691358in}}%
\pgfpathlineto{\pgfqpoint{2.660413in}{0.688698in}}%
\pgfpathlineto{\pgfqpoint{2.649050in}{0.709406in}}%
\pgfpathlineto{\pgfqpoint{2.640910in}{0.723398in}}%
\pgfpathlineto{\pgfqpoint{2.642117in}{0.748069in}}%
\pgfpathlineto{\pgfqpoint{2.633428in}{0.772586in}}%
\pgfpathlineto{\pgfqpoint{2.636431in}{0.793548in}}%
\pgfpathlineto{\pgfqpoint{2.645322in}{0.805319in}}%
\pgfpathlineto{\pgfqpoint{2.637798in}{0.809102in}}%
\pgfpathlineto{\pgfqpoint{2.639499in}{0.816867in}}%
\pgfpathlineto{\pgfqpoint{2.628968in}{0.832126in}}%
\pgfpathlineto{\pgfqpoint{2.622279in}{0.848523in}}%
\pgfpathlineto{\pgfqpoint{2.610714in}{0.889244in}}%
\pgfpathlineto{\pgfqpoint{2.597811in}{0.937884in}}%
\pgfpathlineto{\pgfqpoint{2.578893in}{1.004151in}}%
\pgfusepath{stroke}%
\end{pgfscope}%
\begin{pgfscope}%
\pgfpathrectangle{\pgfqpoint{0.100000in}{0.100000in}}{\pgfqpoint{3.420221in}{2.189500in}}%
\pgfusepath{clip}%
\pgfsetbuttcap%
\pgfsetroundjoin%
\pgfsetlinewidth{0.050187pt}%
\definecolor{currentstroke}{rgb}{1.000000,1.000000,1.000000}%
\pgfsetstrokecolor{currentstroke}%
\pgfsetdash{}{0pt}%
\pgfpathmoveto{\pgfqpoint{2.731730in}{1.022904in}}%
\pgfpathlineto{\pgfqpoint{2.762899in}{1.038409in}}%
\pgfpathlineto{\pgfqpoint{2.780161in}{1.044263in}}%
\pgfpathlineto{\pgfqpoint{2.855265in}{1.052067in}}%
\pgfpathlineto{\pgfqpoint{2.863219in}{1.049482in}}%
\pgfpathlineto{\pgfqpoint{2.873725in}{1.038921in}}%
\pgfpathlineto{\pgfqpoint{2.874304in}{1.029543in}}%
\pgfpathlineto{\pgfqpoint{2.943428in}{1.039700in}}%
\pgfpathlineto{\pgfqpoint{3.021920in}{0.983433in}}%
\pgfpathlineto{\pgfqpoint{3.007200in}{0.968091in}}%
\pgfpathlineto{\pgfqpoint{2.987002in}{0.932461in}}%
\pgfpathlineto{\pgfqpoint{2.992752in}{0.924803in}}%
\pgfpathlineto{\pgfqpoint{2.981977in}{0.909940in}}%
\pgfpathlineto{\pgfqpoint{2.971259in}{0.908258in}}%
\pgfpathlineto{\pgfqpoint{2.971220in}{0.899326in}}%
\pgfpathlineto{\pgfqpoint{2.963467in}{0.889981in}}%
\pgfpathlineto{\pgfqpoint{2.953818in}{0.888085in}}%
\pgfpathlineto{\pgfqpoint{2.955924in}{0.879781in}}%
\pgfpathlineto{\pgfqpoint{2.950539in}{0.873406in}}%
\pgfpathlineto{\pgfqpoint{2.937660in}{0.867900in}}%
\pgfpathlineto{\pgfqpoint{2.921962in}{0.855372in}}%
\pgfpathlineto{\pgfqpoint{2.924925in}{0.847268in}}%
\pgfpathlineto{\pgfqpoint{2.915046in}{0.842179in}}%
\pgfpathlineto{\pgfqpoint{2.906713in}{0.847530in}}%
\pgfpathlineto{\pgfqpoint{2.900619in}{0.824268in}}%
\pgfpathlineto{\pgfqpoint{2.886797in}{0.824968in}}%
\pgfpathlineto{\pgfqpoint{2.883621in}{0.837547in}}%
\pgfpathlineto{\pgfqpoint{2.873772in}{0.854651in}}%
\pgfpathlineto{\pgfqpoint{2.861539in}{0.860802in}}%
\pgfpathlineto{\pgfqpoint{2.857847in}{0.876714in}}%
\pgfpathlineto{\pgfqpoint{2.849507in}{0.890871in}}%
\pgfpathlineto{\pgfqpoint{2.837730in}{0.895140in}}%
\pgfpathlineto{\pgfqpoint{2.822354in}{0.910133in}}%
\pgfpathlineto{\pgfqpoint{2.820929in}{0.917673in}}%
\pgfpathlineto{\pgfqpoint{2.801975in}{0.928070in}}%
\pgfpathlineto{\pgfqpoint{2.795934in}{0.937684in}}%
\pgfpathlineto{\pgfqpoint{2.774680in}{0.949737in}}%
\pgfpathlineto{\pgfqpoint{2.762712in}{0.964078in}}%
\pgfpathlineto{\pgfqpoint{2.751679in}{0.984415in}}%
\pgfpathlineto{\pgfqpoint{2.739875in}{0.985308in}}%
\pgfpathlineto{\pgfqpoint{2.732064in}{0.992532in}}%
\pgfpathlineto{\pgfqpoint{2.720579in}{0.997006in}}%
\pgfpathlineto{\pgfqpoint{2.720572in}{1.005304in}}%
\pgfpathlineto{\pgfqpoint{2.731730in}{1.022904in}}%
\pgfusepath{stroke}%
\end{pgfscope}%
\begin{pgfscope}%
\pgfpathrectangle{\pgfqpoint{0.100000in}{0.100000in}}{\pgfqpoint{3.420221in}{2.189500in}}%
\pgfusepath{clip}%
\pgfsetbuttcap%
\pgfsetroundjoin%
\pgfsetlinewidth{0.050187pt}%
\definecolor{currentstroke}{rgb}{1.000000,1.000000,1.000000}%
\pgfsetstrokecolor{currentstroke}%
\pgfsetdash{}{0pt}%
\pgfpathmoveto{\pgfqpoint{2.025171in}{1.086881in}}%
\pgfpathlineto{\pgfqpoint{2.070975in}{1.087293in}}%
\pgfpathlineto{\pgfqpoint{2.131572in}{1.088398in}}%
\pgfpathlineto{\pgfqpoint{2.235119in}{1.091685in}}%
\pgfpathlineto{\pgfqpoint{2.294345in}{1.094817in}}%
\pgfpathlineto{\pgfqpoint{2.300696in}{1.086954in}}%
\pgfpathlineto{\pgfqpoint{2.300299in}{1.078649in}}%
\pgfpathlineto{\pgfqpoint{2.285952in}{1.064322in}}%
\pgfpathlineto{\pgfqpoint{2.282483in}{1.056477in}}%
\pgfpathlineto{\pgfqpoint{2.322276in}{1.059420in}}%
\pgfpathlineto{\pgfqpoint{2.322254in}{1.044926in}}%
\pgfpathlineto{\pgfqpoint{2.309488in}{1.038731in}}%
\pgfpathlineto{\pgfqpoint{2.309668in}{1.028882in}}%
\pgfpathlineto{\pgfqpoint{2.304874in}{1.021803in}}%
\pgfpathlineto{\pgfqpoint{2.301736in}{1.006767in}}%
\pgfpathlineto{\pgfqpoint{2.304976in}{0.994362in}}%
\pgfpathlineto{\pgfqpoint{2.290779in}{0.982578in}}%
\pgfpathlineto{\pgfqpoint{2.294934in}{0.976781in}}%
\pgfpathlineto{\pgfqpoint{2.281618in}{0.966994in}}%
\pgfpathlineto{\pgfqpoint{2.282126in}{0.957419in}}%
\pgfpathlineto{\pgfqpoint{2.271435in}{0.933864in}}%
\pgfpathlineto{\pgfqpoint{2.261415in}{0.926729in}}%
\pgfpathlineto{\pgfqpoint{2.260448in}{0.916916in}}%
\pgfpathlineto{\pgfqpoint{2.239648in}{0.880048in}}%
\pgfpathlineto{\pgfqpoint{2.246694in}{0.859516in}}%
\pgfpathlineto{\pgfqpoint{2.244736in}{0.854081in}}%
\pgfpathlineto{\pgfqpoint{2.248741in}{0.842135in}}%
\pgfpathlineto{\pgfqpoint{2.244438in}{0.830752in}}%
\pgfpathlineto{\pgfqpoint{2.187560in}{0.828404in}}%
\pgfpathlineto{\pgfqpoint{2.113722in}{0.827187in}}%
\pgfpathlineto{\pgfqpoint{2.062793in}{0.826728in}}%
\pgfpathlineto{\pgfqpoint{2.062545in}{0.866828in}}%
\pgfpathlineto{\pgfqpoint{2.049897in}{0.869456in}}%
\pgfpathlineto{\pgfqpoint{2.041550in}{0.866025in}}%
\pgfpathlineto{\pgfqpoint{2.034882in}{0.872325in}}%
\pgfpathlineto{\pgfqpoint{2.035570in}{0.914803in}}%
\pgfpathlineto{\pgfqpoint{2.037062in}{1.005246in}}%
\pgfpathlineto{\pgfqpoint{2.029828in}{1.058207in}}%
\pgfpathlineto{\pgfqpoint{2.025171in}{1.086881in}}%
\pgfusepath{stroke}%
\end{pgfscope}%
\begin{pgfscope}%
\pgfpathrectangle{\pgfqpoint{0.100000in}{0.100000in}}{\pgfqpoint{3.420221in}{2.189500in}}%
\pgfusepath{clip}%
\pgfsetbuttcap%
\pgfsetroundjoin%
\pgfsetlinewidth{0.050187pt}%
\definecolor{currentstroke}{rgb}{1.000000,1.000000,1.000000}%
\pgfsetstrokecolor{currentstroke}%
\pgfsetdash{}{0pt}%
\pgfpathmoveto{\pgfqpoint{2.062793in}{0.826728in}}%
\pgfpathlineto{\pgfqpoint{2.113722in}{0.827187in}}%
\pgfpathlineto{\pgfqpoint{2.187560in}{0.828404in}}%
\pgfpathlineto{\pgfqpoint{2.244438in}{0.830752in}}%
\pgfpathlineto{\pgfqpoint{2.241712in}{0.824818in}}%
\pgfpathlineto{\pgfqpoint{2.248899in}{0.792136in}}%
\pgfpathlineto{\pgfqpoint{2.258827in}{0.777859in}}%
\pgfpathlineto{\pgfqpoint{2.247317in}{0.769973in}}%
\pgfpathlineto{\pgfqpoint{2.253333in}{0.756273in}}%
\pgfpathlineto{\pgfqpoint{2.237101in}{0.742166in}}%
\pgfpathlineto{\pgfqpoint{2.232249in}{0.720755in}}%
\pgfpathlineto{\pgfqpoint{2.223380in}{0.709725in}}%
\pgfpathlineto{\pgfqpoint{2.224509in}{0.698352in}}%
\pgfpathlineto{\pgfqpoint{2.219420in}{0.695940in}}%
\pgfpathlineto{\pgfqpoint{2.224526in}{0.684102in}}%
\pgfpathlineto{\pgfqpoint{2.220395in}{0.677834in}}%
\pgfpathlineto{\pgfqpoint{2.289837in}{0.680845in}}%
\pgfpathlineto{\pgfqpoint{2.343390in}{0.684142in}}%
\pgfpathlineto{\pgfqpoint{2.337738in}{0.658963in}}%
\pgfpathlineto{\pgfqpoint{2.341594in}{0.649660in}}%
\pgfpathlineto{\pgfqpoint{2.349587in}{0.641886in}}%
\pgfpathlineto{\pgfqpoint{2.357340in}{0.623338in}}%
\pgfpathlineto{\pgfqpoint{2.332850in}{0.627611in}}%
\pgfpathlineto{\pgfqpoint{2.323815in}{0.634636in}}%
\pgfpathlineto{\pgfqpoint{2.313051in}{0.634971in}}%
\pgfpathlineto{\pgfqpoint{2.301735in}{0.619588in}}%
\pgfpathlineto{\pgfqpoint{2.303990in}{0.612583in}}%
\pgfpathlineto{\pgfqpoint{2.340142in}{0.608345in}}%
\pgfpathlineto{\pgfqpoint{2.349606in}{0.600293in}}%
\pgfpathlineto{\pgfqpoint{2.365008in}{0.596150in}}%
\pgfpathlineto{\pgfqpoint{2.358281in}{0.586617in}}%
\pgfpathlineto{\pgfqpoint{2.346949in}{0.578303in}}%
\pgfpathlineto{\pgfqpoint{2.376861in}{0.559607in}}%
\pgfpathlineto{\pgfqpoint{2.390795in}{0.556686in}}%
\pgfpathlineto{\pgfqpoint{2.396589in}{0.541404in}}%
\pgfpathlineto{\pgfqpoint{2.383470in}{0.538592in}}%
\pgfpathlineto{\pgfqpoint{2.358814in}{0.557813in}}%
\pgfpathlineto{\pgfqpoint{2.349173in}{0.560485in}}%
\pgfpathlineto{\pgfqpoint{2.344509in}{0.568078in}}%
\pgfpathlineto{\pgfqpoint{2.330209in}{0.564961in}}%
\pgfpathlineto{\pgfqpoint{2.325732in}{0.555177in}}%
\pgfpathlineto{\pgfqpoint{2.328568in}{0.544275in}}%
\pgfpathlineto{\pgfqpoint{2.318941in}{0.537841in}}%
\pgfpathlineto{\pgfqpoint{2.310155in}{0.553691in}}%
\pgfpathlineto{\pgfqpoint{2.294767in}{0.548965in}}%
\pgfpathlineto{\pgfqpoint{2.289001in}{0.539508in}}%
\pgfpathlineto{\pgfqpoint{2.278073in}{0.542198in}}%
\pgfpathlineto{\pgfqpoint{2.271086in}{0.554141in}}%
\pgfpathlineto{\pgfqpoint{2.252549in}{0.558084in}}%
\pgfpathlineto{\pgfqpoint{2.249049in}{0.564299in}}%
\pgfpathlineto{\pgfqpoint{2.229736in}{0.575124in}}%
\pgfpathlineto{\pgfqpoint{2.224942in}{0.584589in}}%
\pgfpathlineto{\pgfqpoint{2.208765in}{0.580721in}}%
\pgfpathlineto{\pgfqpoint{2.210820in}{0.589387in}}%
\pgfpathlineto{\pgfqpoint{2.190714in}{0.580497in}}%
\pgfpathlineto{\pgfqpoint{2.196115in}{0.571274in}}%
\pgfpathlineto{\pgfqpoint{2.180588in}{0.565821in}}%
\pgfpathlineto{\pgfqpoint{2.160022in}{0.568819in}}%
\pgfpathlineto{\pgfqpoint{2.118397in}{0.583076in}}%
\pgfpathlineto{\pgfqpoint{2.086278in}{0.580247in}}%
\pgfpathlineto{\pgfqpoint{2.083494in}{0.598975in}}%
\pgfpathlineto{\pgfqpoint{2.087528in}{0.607521in}}%
\pgfpathlineto{\pgfqpoint{2.087183in}{0.621118in}}%
\pgfpathlineto{\pgfqpoint{2.083215in}{0.624358in}}%
\pgfpathlineto{\pgfqpoint{2.084560in}{0.639878in}}%
\pgfpathlineto{\pgfqpoint{2.096257in}{0.661464in}}%
\pgfpathlineto{\pgfqpoint{2.097754in}{0.669621in}}%
\pgfpathlineto{\pgfqpoint{2.095826in}{0.688886in}}%
\pgfpathlineto{\pgfqpoint{2.087220in}{0.697960in}}%
\pgfpathlineto{\pgfqpoint{2.082727in}{0.714903in}}%
\pgfpathlineto{\pgfqpoint{2.077198in}{0.718794in}}%
\pgfpathlineto{\pgfqpoint{2.079673in}{0.727982in}}%
\pgfpathlineto{\pgfqpoint{2.072632in}{0.741715in}}%
\pgfpathlineto{\pgfqpoint{2.063848in}{0.749168in}}%
\pgfpathlineto{\pgfqpoint{2.062793in}{0.826728in}}%
\pgfusepath{stroke}%
\end{pgfscope}%
\begin{pgfscope}%
\pgfpathrectangle{\pgfqpoint{0.100000in}{0.100000in}}{\pgfqpoint{3.420221in}{2.189500in}}%
\pgfusepath{clip}%
\pgfsetbuttcap%
\pgfsetroundjoin%
\pgfsetlinewidth{0.050187pt}%
\definecolor{currentstroke}{rgb}{1.000000,1.000000,1.000000}%
\pgfsetstrokecolor{currentstroke}%
\pgfsetdash{}{0pt}%
\pgfpathmoveto{\pgfqpoint{2.199003in}{0.570568in}}%
\pgfpathlineto{\pgfqpoint{2.206371in}{0.574951in}}%
\pgfpathlineto{\pgfqpoint{2.215314in}{0.569769in}}%
\pgfpathlineto{\pgfqpoint{2.210327in}{0.562640in}}%
\pgfpathlineto{\pgfqpoint{2.199003in}{0.570568in}}%
\pgfusepath{stroke}%
\end{pgfscope}%
\begin{pgfscope}%
\pgfpathrectangle{\pgfqpoint{0.100000in}{0.100000in}}{\pgfqpoint{3.420221in}{2.189500in}}%
\pgfusepath{clip}%
\pgfsetbuttcap%
\pgfsetroundjoin%
\pgfsetlinewidth{0.050187pt}%
\definecolor{currentstroke}{rgb}{1.000000,1.000000,1.000000}%
\pgfsetstrokecolor{currentstroke}%
\pgfsetdash{}{0pt}%
\pgfpathmoveto{\pgfqpoint{2.496340in}{0.655084in}}%
\pgfpathlineto{\pgfqpoint{2.496474in}{0.669252in}}%
\pgfpathlineto{\pgfqpoint{2.487670in}{0.674633in}}%
\pgfpathlineto{\pgfqpoint{2.480457in}{0.683792in}}%
\pgfpathlineto{\pgfqpoint{2.481440in}{0.693415in}}%
\pgfpathlineto{\pgfqpoint{2.559966in}{0.699511in}}%
\pgfpathlineto{\pgfqpoint{2.649050in}{0.709406in}}%
\pgfpathlineto{\pgfqpoint{2.660413in}{0.688698in}}%
\pgfpathlineto{\pgfqpoint{2.698194in}{0.691358in}}%
\pgfpathlineto{\pgfqpoint{2.773106in}{0.695445in}}%
\pgfpathlineto{\pgfqpoint{2.832517in}{0.699404in}}%
\pgfpathlineto{\pgfqpoint{2.838352in}{0.684490in}}%
\pgfpathlineto{\pgfqpoint{2.845565in}{0.685711in}}%
\pgfpathlineto{\pgfqpoint{2.846367in}{0.701681in}}%
\pgfpathlineto{\pgfqpoint{2.842676in}{0.715417in}}%
\pgfpathlineto{\pgfqpoint{2.847623in}{0.721232in}}%
\pgfpathlineto{\pgfqpoint{2.859056in}{0.718104in}}%
\pgfpathlineto{\pgfqpoint{2.875467in}{0.717461in}}%
\pgfpathlineto{\pgfqpoint{2.877991in}{0.705189in}}%
\pgfpathlineto{\pgfqpoint{2.883039in}{0.698177in}}%
\pgfpathlineto{\pgfqpoint{2.886981in}{0.682835in}}%
\pgfpathlineto{\pgfqpoint{2.899196in}{0.659092in}}%
\pgfpathlineto{\pgfqpoint{2.899257in}{0.652671in}}%
\pgfpathlineto{\pgfqpoint{2.917309in}{0.624793in}}%
\pgfpathlineto{\pgfqpoint{2.919056in}{0.619034in}}%
\pgfpathlineto{\pgfqpoint{2.946138in}{0.579660in}}%
\pgfpathlineto{\pgfqpoint{2.941847in}{0.579024in}}%
\pgfpathlineto{\pgfqpoint{2.953265in}{0.551018in}}%
\pgfpathlineto{\pgfqpoint{2.984629in}{0.502332in}}%
\pgfpathlineto{\pgfqpoint{3.002839in}{0.466535in}}%
\pgfpathlineto{\pgfqpoint{3.008925in}{0.460267in}}%
\pgfpathlineto{\pgfqpoint{3.019355in}{0.437798in}}%
\pgfpathlineto{\pgfqpoint{3.022972in}{0.401820in}}%
\pgfpathlineto{\pgfqpoint{3.024412in}{0.374918in}}%
\pgfpathlineto{\pgfqpoint{3.022673in}{0.357693in}}%
\pgfpathlineto{\pgfqpoint{3.017082in}{0.345379in}}%
\pgfpathlineto{\pgfqpoint{3.019680in}{0.329232in}}%
\pgfpathlineto{\pgfqpoint{3.013659in}{0.316451in}}%
\pgfpathlineto{\pgfqpoint{2.995784in}{0.305970in}}%
\pgfpathlineto{\pgfqpoint{2.984158in}{0.306736in}}%
\pgfpathlineto{\pgfqpoint{2.976608in}{0.301257in}}%
\pgfpathlineto{\pgfqpoint{2.974445in}{0.315885in}}%
\pgfpathlineto{\pgfqpoint{2.961948in}{0.319746in}}%
\pgfpathlineto{\pgfqpoint{2.950745in}{0.340157in}}%
\pgfpathlineto{\pgfqpoint{2.949480in}{0.349452in}}%
\pgfpathlineto{\pgfqpoint{2.929430in}{0.355096in}}%
\pgfpathlineto{\pgfqpoint{2.916511in}{0.353874in}}%
\pgfpathlineto{\pgfqpoint{2.909200in}{0.367361in}}%
\pgfpathlineto{\pgfqpoint{2.900831in}{0.391602in}}%
\pgfpathlineto{\pgfqpoint{2.889216in}{0.396515in}}%
\pgfpathlineto{\pgfqpoint{2.882920in}{0.410407in}}%
\pgfpathlineto{\pgfqpoint{2.867520in}{0.418593in}}%
\pgfpathlineto{\pgfqpoint{2.858619in}{0.428834in}}%
\pgfpathlineto{\pgfqpoint{2.843264in}{0.456674in}}%
\pgfpathlineto{\pgfqpoint{2.841458in}{0.476270in}}%
\pgfpathlineto{\pgfqpoint{2.849826in}{0.488807in}}%
\pgfpathlineto{\pgfqpoint{2.848989in}{0.497519in}}%
\pgfpathlineto{\pgfqpoint{2.839320in}{0.498450in}}%
\pgfpathlineto{\pgfqpoint{2.831241in}{0.504510in}}%
\pgfpathlineto{\pgfqpoint{2.826817in}{0.497110in}}%
\pgfpathlineto{\pgfqpoint{2.834570in}{0.491081in}}%
\pgfpathlineto{\pgfqpoint{2.828927in}{0.480239in}}%
\pgfpathlineto{\pgfqpoint{2.819798in}{0.489238in}}%
\pgfpathlineto{\pgfqpoint{2.820857in}{0.514192in}}%
\pgfpathlineto{\pgfqpoint{2.825260in}{0.534437in}}%
\pgfpathlineto{\pgfqpoint{2.822997in}{0.569176in}}%
\pgfpathlineto{\pgfqpoint{2.813897in}{0.577450in}}%
\pgfpathlineto{\pgfqpoint{2.809320in}{0.588067in}}%
\pgfpathlineto{\pgfqpoint{2.793645in}{0.587831in}}%
\pgfpathlineto{\pgfqpoint{2.778149in}{0.605301in}}%
\pgfpathlineto{\pgfqpoint{2.767759in}{0.610537in}}%
\pgfpathlineto{\pgfqpoint{2.764678in}{0.621601in}}%
\pgfpathlineto{\pgfqpoint{2.754515in}{0.625487in}}%
\pgfpathlineto{\pgfqpoint{2.746094in}{0.637703in}}%
\pgfpathlineto{\pgfqpoint{2.723845in}{0.647674in}}%
\pgfpathlineto{\pgfqpoint{2.706549in}{0.647928in}}%
\pgfpathlineto{\pgfqpoint{2.699019in}{0.644099in}}%
\pgfpathlineto{\pgfqpoint{2.700890in}{0.632144in}}%
\pgfpathlineto{\pgfqpoint{2.693039in}{0.632710in}}%
\pgfpathlineto{\pgfqpoint{2.668886in}{0.615995in}}%
\pgfpathlineto{\pgfqpoint{2.639874in}{0.609370in}}%
\pgfpathlineto{\pgfqpoint{2.639398in}{0.617567in}}%
\pgfpathlineto{\pgfqpoint{2.632969in}{0.625578in}}%
\pgfpathlineto{\pgfqpoint{2.615719in}{0.636639in}}%
\pgfpathlineto{\pgfqpoint{2.590958in}{0.647979in}}%
\pgfpathlineto{\pgfqpoint{2.564063in}{0.653982in}}%
\pgfpathlineto{\pgfqpoint{2.559015in}{0.662145in}}%
\pgfpathlineto{\pgfqpoint{2.549317in}{0.655337in}}%
\pgfpathlineto{\pgfqpoint{2.537641in}{0.653835in}}%
\pgfpathlineto{\pgfqpoint{2.496995in}{0.643126in}}%
\pgfpathlineto{\pgfqpoint{2.496340in}{0.655084in}}%
\pgfusepath{stroke}%
\end{pgfscope}%
\begin{pgfscope}%
\pgfpathrectangle{\pgfqpoint{0.100000in}{0.100000in}}{\pgfqpoint{3.420221in}{2.189500in}}%
\pgfusepath{clip}%
\pgfsetbuttcap%
\pgfsetroundjoin%
\pgfsetlinewidth{0.050187pt}%
\definecolor{currentstroke}{rgb}{1.000000,1.000000,1.000000}%
\pgfsetstrokecolor{currentstroke}%
\pgfsetdash{}{0pt}%
\pgfpathmoveto{\pgfqpoint{2.950084in}{0.580220in}}%
\pgfpathlineto{\pgfqpoint{2.960600in}{0.567987in}}%
\pgfpathlineto{\pgfqpoint{2.948453in}{0.567217in}}%
\pgfpathlineto{\pgfqpoint{2.950084in}{0.580220in}}%
\pgfusepath{stroke}%
\end{pgfscope}%
\begin{pgfscope}%
\pgfpathrectangle{\pgfqpoint{0.100000in}{0.100000in}}{\pgfqpoint{3.420221in}{2.189500in}}%
\pgfusepath{clip}%
\pgfsetbuttcap%
\pgfsetroundjoin%
\pgfsetlinewidth{0.050187pt}%
\definecolor{currentstroke}{rgb}{1.000000,1.000000,1.000000}%
\pgfsetstrokecolor{currentstroke}%
\pgfsetdash{}{0pt}%
\pgfpathmoveto{\pgfqpoint{2.331007in}{1.976998in}}%
\pgfpathlineto{\pgfqpoint{2.325301in}{1.965892in}}%
\pgfpathlineto{\pgfqpoint{2.311689in}{1.959411in}}%
\pgfpathlineto{\pgfqpoint{2.305793in}{1.950687in}}%
\pgfpathlineto{\pgfqpoint{2.298887in}{1.956970in}}%
\pgfpathlineto{\pgfqpoint{2.331007in}{1.976998in}}%
\pgfusepath{stroke}%
\end{pgfscope}%
\begin{pgfscope}%
\pgfpathrectangle{\pgfqpoint{0.100000in}{0.100000in}}{\pgfqpoint{3.420221in}{2.189500in}}%
\pgfusepath{clip}%
\pgfsetbuttcap%
\pgfsetroundjoin%
\pgfsetlinewidth{0.050187pt}%
\definecolor{currentstroke}{rgb}{1.000000,1.000000,1.000000}%
\pgfsetstrokecolor{currentstroke}%
\pgfsetdash{}{0pt}%
\pgfpathmoveto{\pgfqpoint{2.335634in}{1.910164in}}%
\pgfpathlineto{\pgfqpoint{2.349562in}{1.923156in}}%
\pgfpathlineto{\pgfqpoint{2.371045in}{1.926559in}}%
\pgfpathlineto{\pgfqpoint{2.365128in}{1.917523in}}%
\pgfpathlineto{\pgfqpoint{2.350400in}{1.904455in}}%
\pgfpathlineto{\pgfqpoint{2.341790in}{1.887682in}}%
\pgfpathlineto{\pgfqpoint{2.338143in}{1.896780in}}%
\pgfpathlineto{\pgfqpoint{2.331669in}{1.898085in}}%
\pgfpathlineto{\pgfqpoint{2.331018in}{1.906299in}}%
\pgfpathlineto{\pgfqpoint{2.335634in}{1.910164in}}%
\pgfusepath{stroke}%
\end{pgfscope}%
\begin{pgfscope}%
\pgfpathrectangle{\pgfqpoint{0.100000in}{0.100000in}}{\pgfqpoint{3.420221in}{2.189500in}}%
\pgfusepath{clip}%
\pgfsetbuttcap%
\pgfsetroundjoin%
\pgfsetlinewidth{0.050187pt}%
\definecolor{currentstroke}{rgb}{1.000000,1.000000,1.000000}%
\pgfsetstrokecolor{currentstroke}%
\pgfsetdash{}{0pt}%
\pgfpathmoveto{\pgfqpoint{2.391108in}{1.751577in}}%
\pgfpathlineto{\pgfqpoint{2.387399in}{1.755693in}}%
\pgfpathlineto{\pgfqpoint{2.391298in}{1.767281in}}%
\pgfpathlineto{\pgfqpoint{2.379723in}{1.768020in}}%
\pgfpathlineto{\pgfqpoint{2.382791in}{1.778011in}}%
\pgfpathlineto{\pgfqpoint{2.380883in}{1.793924in}}%
\pgfpathlineto{\pgfqpoint{2.370497in}{1.799441in}}%
\pgfpathlineto{\pgfqpoint{2.359627in}{1.810606in}}%
\pgfpathlineto{\pgfqpoint{2.342879in}{1.813875in}}%
\pgfpathlineto{\pgfqpoint{2.326580in}{1.813781in}}%
\pgfpathlineto{\pgfqpoint{2.310548in}{1.821705in}}%
\pgfpathlineto{\pgfqpoint{2.256915in}{1.833255in}}%
\pgfpathlineto{\pgfqpoint{2.251054in}{1.845493in}}%
\pgfpathlineto{\pgfqpoint{2.240600in}{1.849669in}}%
\pgfpathlineto{\pgfqpoint{2.260348in}{1.859039in}}%
\pgfpathlineto{\pgfqpoint{2.271503in}{1.870729in}}%
\pgfpathlineto{\pgfqpoint{2.292282in}{1.873899in}}%
\pgfpathlineto{\pgfqpoint{2.305068in}{1.885806in}}%
\pgfpathlineto{\pgfqpoint{2.311776in}{1.886282in}}%
\pgfpathlineto{\pgfqpoint{2.316896in}{1.894774in}}%
\pgfpathlineto{\pgfqpoint{2.329376in}{1.904834in}}%
\pgfpathlineto{\pgfqpoint{2.330463in}{1.897723in}}%
\pgfpathlineto{\pgfqpoint{2.336087in}{1.896255in}}%
\pgfpathlineto{\pgfqpoint{2.340318in}{1.887776in}}%
\pgfpathlineto{\pgfqpoint{2.341080in}{1.873319in}}%
\pgfpathlineto{\pgfqpoint{2.353796in}{1.881958in}}%
\pgfpathlineto{\pgfqpoint{2.368614in}{1.883762in}}%
\pgfpathlineto{\pgfqpoint{2.381263in}{1.879237in}}%
\pgfpathlineto{\pgfqpoint{2.398416in}{1.855704in}}%
\pgfpathlineto{\pgfqpoint{2.417150in}{1.859466in}}%
\pgfpathlineto{\pgfqpoint{2.424763in}{1.853168in}}%
\pgfpathlineto{\pgfqpoint{2.437007in}{1.852605in}}%
\pgfpathlineto{\pgfqpoint{2.445141in}{1.863900in}}%
\pgfpathlineto{\pgfqpoint{2.460576in}{1.873884in}}%
\pgfpathlineto{\pgfqpoint{2.475411in}{1.877011in}}%
\pgfpathlineto{\pgfqpoint{2.493819in}{1.877337in}}%
\pgfpathlineto{\pgfqpoint{2.507271in}{1.885050in}}%
\pgfpathlineto{\pgfqpoint{2.518249in}{1.881497in}}%
\pgfpathlineto{\pgfqpoint{2.520587in}{1.865185in}}%
\pgfpathlineto{\pgfqpoint{2.544139in}{1.862631in}}%
\pgfpathlineto{\pgfqpoint{2.556931in}{1.870266in}}%
\pgfpathlineto{\pgfqpoint{2.565799in}{1.853065in}}%
\pgfpathlineto{\pgfqpoint{2.582720in}{1.833175in}}%
\pgfpathlineto{\pgfqpoint{2.559037in}{1.833290in}}%
\pgfpathlineto{\pgfqpoint{2.551554in}{1.830831in}}%
\pgfpathlineto{\pgfqpoint{2.541297in}{1.834026in}}%
\pgfpathlineto{\pgfqpoint{2.540607in}{1.820256in}}%
\pgfpathlineto{\pgfqpoint{2.521983in}{1.831017in}}%
\pgfpathlineto{\pgfqpoint{2.498032in}{1.834306in}}%
\pgfpathlineto{\pgfqpoint{2.491439in}{1.823826in}}%
\pgfpathlineto{\pgfqpoint{2.460102in}{1.818730in}}%
\pgfpathlineto{\pgfqpoint{2.456496in}{1.809852in}}%
\pgfpathlineto{\pgfqpoint{2.434689in}{1.807188in}}%
\pgfpathlineto{\pgfqpoint{2.428097in}{1.798163in}}%
\pgfpathlineto{\pgfqpoint{2.416562in}{1.795743in}}%
\pgfpathlineto{\pgfqpoint{2.407349in}{1.774343in}}%
\pgfpathlineto{\pgfqpoint{2.395685in}{1.753617in}}%
\pgfpathlineto{\pgfqpoint{2.391108in}{1.751577in}}%
\pgfusepath{stroke}%
\end{pgfscope}%
\begin{pgfscope}%
\pgfpathrectangle{\pgfqpoint{0.100000in}{0.100000in}}{\pgfqpoint{3.420221in}{2.189500in}}%
\pgfusepath{clip}%
\pgfsetbuttcap%
\pgfsetroundjoin%
\pgfsetlinewidth{0.050187pt}%
\definecolor{currentstroke}{rgb}{1.000000,1.000000,1.000000}%
\pgfsetstrokecolor{currentstroke}%
\pgfsetdash{}{0pt}%
\pgfpathmoveto{\pgfqpoint{2.458244in}{1.502744in}}%
\pgfpathlineto{\pgfqpoint{2.469365in}{1.514450in}}%
\pgfpathlineto{\pgfqpoint{2.474432in}{1.531422in}}%
\pgfpathlineto{\pgfqpoint{2.480457in}{1.541259in}}%
\pgfpathlineto{\pgfqpoint{2.484152in}{1.554646in}}%
\pgfpathlineto{\pgfqpoint{2.485300in}{1.581339in}}%
\pgfpathlineto{\pgfqpoint{2.479737in}{1.606920in}}%
\pgfpathlineto{\pgfqpoint{2.461354in}{1.646093in}}%
\pgfpathlineto{\pgfqpoint{2.466333in}{1.658409in}}%
\pgfpathlineto{\pgfqpoint{2.459847in}{1.675434in}}%
\pgfpathlineto{\pgfqpoint{2.470972in}{1.698978in}}%
\pgfpathlineto{\pgfqpoint{2.469159in}{1.725231in}}%
\pgfpathlineto{\pgfqpoint{2.476909in}{1.728535in}}%
\pgfpathlineto{\pgfqpoint{2.477882in}{1.741026in}}%
\pgfpathlineto{\pgfqpoint{2.491637in}{1.749043in}}%
\pgfpathlineto{\pgfqpoint{2.499416in}{1.747749in}}%
\pgfpathlineto{\pgfqpoint{2.501603in}{1.734353in}}%
\pgfpathlineto{\pgfqpoint{2.507675in}{1.733828in}}%
\pgfpathlineto{\pgfqpoint{2.513215in}{1.753151in}}%
\pgfpathlineto{\pgfqpoint{2.511332in}{1.768180in}}%
\pgfpathlineto{\pgfqpoint{2.514934in}{1.776810in}}%
\pgfpathlineto{\pgfqpoint{2.534408in}{1.785725in}}%
\pgfpathlineto{\pgfqpoint{2.525534in}{1.788897in}}%
\pgfpathlineto{\pgfqpoint{2.522643in}{1.796556in}}%
\pgfpathlineto{\pgfqpoint{2.529036in}{1.810005in}}%
\pgfpathlineto{\pgfqpoint{2.541643in}{1.814650in}}%
\pgfpathlineto{\pgfqpoint{2.556283in}{1.806715in}}%
\pgfpathlineto{\pgfqpoint{2.570085in}{1.806616in}}%
\pgfpathlineto{\pgfqpoint{2.576495in}{1.797346in}}%
\pgfpathlineto{\pgfqpoint{2.586161in}{1.797991in}}%
\pgfpathlineto{\pgfqpoint{2.616000in}{1.785106in}}%
\pgfpathlineto{\pgfqpoint{2.621962in}{1.772632in}}%
\pgfpathlineto{\pgfqpoint{2.615451in}{1.768312in}}%
\pgfpathlineto{\pgfqpoint{2.617458in}{1.758964in}}%
\pgfpathlineto{\pgfqpoint{2.623890in}{1.754806in}}%
\pgfpathlineto{\pgfqpoint{2.627475in}{1.743314in}}%
\pgfpathlineto{\pgfqpoint{2.626973in}{1.715318in}}%
\pgfpathlineto{\pgfqpoint{2.618480in}{1.708630in}}%
\pgfpathlineto{\pgfqpoint{2.616574in}{1.693924in}}%
\pgfpathlineto{\pgfqpoint{2.600821in}{1.680334in}}%
\pgfpathlineto{\pgfqpoint{2.601756in}{1.663883in}}%
\pgfpathlineto{\pgfqpoint{2.615563in}{1.658120in}}%
\pgfpathlineto{\pgfqpoint{2.631139in}{1.678645in}}%
\pgfpathlineto{\pgfqpoint{2.632427in}{1.686090in}}%
\pgfpathlineto{\pgfqpoint{2.651853in}{1.698463in}}%
\pgfpathlineto{\pgfqpoint{2.664206in}{1.692732in}}%
\pgfpathlineto{\pgfqpoint{2.671958in}{1.679779in}}%
\pgfpathlineto{\pgfqpoint{2.684464in}{1.634806in}}%
\pgfpathlineto{\pgfqpoint{2.691091in}{1.620578in}}%
\pgfpathlineto{\pgfqpoint{2.689016in}{1.614696in}}%
\pgfpathlineto{\pgfqpoint{2.689218in}{1.594667in}}%
\pgfpathlineto{\pgfqpoint{2.677163in}{1.596589in}}%
\pgfpathlineto{\pgfqpoint{2.670354in}{1.581623in}}%
\pgfpathlineto{\pgfqpoint{2.669410in}{1.571448in}}%
\pgfpathlineto{\pgfqpoint{2.660304in}{1.564916in}}%
\pgfpathlineto{\pgfqpoint{2.658280in}{1.545074in}}%
\pgfpathlineto{\pgfqpoint{2.645038in}{1.520007in}}%
\pgfpathlineto{\pgfqpoint{2.572597in}{1.509218in}}%
\pgfpathlineto{\pgfqpoint{2.572169in}{1.513960in}}%
\pgfpathlineto{\pgfqpoint{2.523712in}{1.508853in}}%
\pgfpathlineto{\pgfqpoint{2.458244in}{1.502744in}}%
\pgfusepath{stroke}%
\end{pgfscope}%
\begin{pgfscope}%
\pgfpathrectangle{\pgfqpoint{0.100000in}{0.100000in}}{\pgfqpoint{3.420221in}{2.189500in}}%
\pgfusepath{clip}%
\pgfsetbuttcap%
\pgfsetroundjoin%
\pgfsetlinewidth{0.050187pt}%
\definecolor{currentstroke}{rgb}{1.000000,1.000000,1.000000}%
\pgfsetstrokecolor{currentstroke}%
\pgfsetdash{}{0pt}%
\pgfpathmoveto{\pgfqpoint{0.000000in}{0.000000in}}%
\pgfusepath{stroke}%
\end{pgfscope}%
\begin{pgfscope}%
\pgfpathrectangle{\pgfqpoint{0.100000in}{0.100000in}}{\pgfqpoint{3.420221in}{2.189500in}}%
\pgfusepath{clip}%
\pgfsetbuttcap%
\pgfsetroundjoin%
\pgfsetlinewidth{0.050187pt}%
\definecolor{currentstroke}{rgb}{1.000000,1.000000,1.000000}%
\pgfsetstrokecolor{currentstroke}%
\pgfsetdash{}{0pt}%
\pgfusepath{stroke}%
\end{pgfscope}%
\begin{pgfscope}%
\pgfpathrectangle{\pgfqpoint{0.100000in}{0.100000in}}{\pgfqpoint{3.420221in}{2.189500in}}%
\pgfusepath{clip}%
\pgfsetbuttcap%
\pgfsetroundjoin%
\pgfsetlinewidth{0.050187pt}%
\definecolor{currentstroke}{rgb}{1.000000,1.000000,1.000000}%
\pgfsetstrokecolor{currentstroke}%
\pgfsetdash{}{0pt}%
\pgfusepath{stroke}%
\end{pgfscope}%
\begin{pgfscope}%
\pgfpathrectangle{\pgfqpoint{0.100000in}{0.100000in}}{\pgfqpoint{3.420221in}{2.189500in}}%
\pgfusepath{clip}%
\pgfsetbuttcap%
\pgfsetroundjoin%
\pgfsetlinewidth{0.050187pt}%
\definecolor{currentstroke}{rgb}{1.000000,1.000000,1.000000}%
\pgfsetstrokecolor{currentstroke}%
\pgfsetdash{}{0pt}%
\pgfusepath{stroke}%
\end{pgfscope}%
\begin{pgfscope}%
\pgfpathrectangle{\pgfqpoint{0.100000in}{0.100000in}}{\pgfqpoint{3.420221in}{2.189500in}}%
\pgfusepath{clip}%
\pgfsetbuttcap%
\pgfsetroundjoin%
\pgfsetlinewidth{0.050187pt}%
\definecolor{currentstroke}{rgb}{1.000000,1.000000,1.000000}%
\pgfsetstrokecolor{currentstroke}%
\pgfsetdash{}{0pt}%
\pgfusepath{stroke}%
\end{pgfscope}%
\begin{pgfscope}%
\pgfpathrectangle{\pgfqpoint{0.100000in}{0.100000in}}{\pgfqpoint{3.420221in}{2.189500in}}%
\pgfusepath{clip}%
\pgfsetbuttcap%
\pgfsetroundjoin%
\pgfsetlinewidth{0.050187pt}%
\definecolor{currentstroke}{rgb}{1.000000,1.000000,1.000000}%
\pgfsetstrokecolor{currentstroke}%
\pgfsetdash{}{0pt}%
\pgfusepath{stroke}%
\end{pgfscope}%
\begin{pgfscope}%
\pgfpathrectangle{\pgfqpoint{0.100000in}{0.100000in}}{\pgfqpoint{3.420221in}{2.189500in}}%
\pgfusepath{clip}%
\pgfsetbuttcap%
\pgfsetroundjoin%
\pgfsetlinewidth{0.050187pt}%
\definecolor{currentstroke}{rgb}{1.000000,1.000000,1.000000}%
\pgfsetstrokecolor{currentstroke}%
\pgfsetdash{}{0pt}%
\pgfusepath{stroke}%
\end{pgfscope}%
\begin{pgfscope}%
\pgfpathrectangle{\pgfqpoint{0.100000in}{0.100000in}}{\pgfqpoint{3.420221in}{2.189500in}}%
\pgfusepath{clip}%
\pgfsetbuttcap%
\pgfsetroundjoin%
\pgfsetlinewidth{0.050187pt}%
\definecolor{currentstroke}{rgb}{1.000000,1.000000,1.000000}%
\pgfsetstrokecolor{currentstroke}%
\pgfsetdash{}{0pt}%
\pgfusepath{stroke}%
\end{pgfscope}%
\begin{pgfscope}%
\pgfpathrectangle{\pgfqpoint{0.100000in}{0.100000in}}{\pgfqpoint{3.420221in}{2.189500in}}%
\pgfusepath{clip}%
\pgfsetbuttcap%
\pgfsetroundjoin%
\pgfsetlinewidth{0.050187pt}%
\definecolor{currentstroke}{rgb}{1.000000,1.000000,1.000000}%
\pgfsetstrokecolor{currentstroke}%
\pgfsetdash{}{0pt}%
\pgfusepath{stroke}%
\end{pgfscope}%
\begin{pgfscope}%
\pgfpathrectangle{\pgfqpoint{0.100000in}{0.100000in}}{\pgfqpoint{3.420221in}{2.189500in}}%
\pgfusepath{clip}%
\pgfsetbuttcap%
\pgfsetroundjoin%
\pgfsetlinewidth{0.050187pt}%
\definecolor{currentstroke}{rgb}{1.000000,1.000000,1.000000}%
\pgfsetstrokecolor{currentstroke}%
\pgfsetdash{}{0pt}%
\pgfusepath{stroke}%
\end{pgfscope}%
\begin{pgfscope}%
\pgfpathrectangle{\pgfqpoint{0.100000in}{0.100000in}}{\pgfqpoint{3.420221in}{2.189500in}}%
\pgfusepath{clip}%
\pgfsetbuttcap%
\pgfsetroundjoin%
\pgfsetlinewidth{0.050187pt}%
\definecolor{currentstroke}{rgb}{1.000000,1.000000,1.000000}%
\pgfsetstrokecolor{currentstroke}%
\pgfsetdash{}{0pt}%
\pgfusepath{stroke}%
\end{pgfscope}%
\begin{pgfscope}%
\pgfpathrectangle{\pgfqpoint{0.100000in}{0.100000in}}{\pgfqpoint{3.420221in}{2.189500in}}%
\pgfusepath{clip}%
\pgfsetbuttcap%
\pgfsetroundjoin%
\pgfsetlinewidth{0.050187pt}%
\definecolor{currentstroke}{rgb}{1.000000,1.000000,1.000000}%
\pgfsetstrokecolor{currentstroke}%
\pgfsetdash{}{0pt}%
\pgfusepath{stroke}%
\end{pgfscope}%
\begin{pgfscope}%
\pgfpathrectangle{\pgfqpoint{0.100000in}{0.100000in}}{\pgfqpoint{3.420221in}{2.189500in}}%
\pgfusepath{clip}%
\pgfsetbuttcap%
\pgfsetroundjoin%
\pgfsetlinewidth{0.050187pt}%
\definecolor{currentstroke}{rgb}{1.000000,1.000000,1.000000}%
\pgfsetstrokecolor{currentstroke}%
\pgfsetdash{}{0pt}%
\pgfusepath{stroke}%
\end{pgfscope}%
\begin{pgfscope}%
\pgfpathrectangle{\pgfqpoint{0.100000in}{0.100000in}}{\pgfqpoint{3.420221in}{2.189500in}}%
\pgfusepath{clip}%
\pgfsetbuttcap%
\pgfsetroundjoin%
\pgfsetlinewidth{0.050187pt}%
\definecolor{currentstroke}{rgb}{1.000000,1.000000,1.000000}%
\pgfsetstrokecolor{currentstroke}%
\pgfsetdash{}{0pt}%
\pgfusepath{stroke}%
\end{pgfscope}%
\begin{pgfscope}%
\pgfpathrectangle{\pgfqpoint{0.100000in}{0.100000in}}{\pgfqpoint{3.420221in}{2.189500in}}%
\pgfusepath{clip}%
\pgfsetbuttcap%
\pgfsetroundjoin%
\pgfsetlinewidth{0.050187pt}%
\definecolor{currentstroke}{rgb}{1.000000,1.000000,1.000000}%
\pgfsetstrokecolor{currentstroke}%
\pgfsetdash{}{0pt}%
\pgfusepath{stroke}%
\end{pgfscope}%
\begin{pgfscope}%
\pgfpathrectangle{\pgfqpoint{0.100000in}{0.100000in}}{\pgfqpoint{3.420221in}{2.189500in}}%
\pgfusepath{clip}%
\pgfsetbuttcap%
\pgfsetroundjoin%
\pgfsetlinewidth{0.050187pt}%
\definecolor{currentstroke}{rgb}{1.000000,1.000000,1.000000}%
\pgfsetstrokecolor{currentstroke}%
\pgfsetdash{}{0pt}%
\pgfusepath{stroke}%
\end{pgfscope}%
\begin{pgfscope}%
\pgfpathrectangle{\pgfqpoint{0.100000in}{0.100000in}}{\pgfqpoint{3.420221in}{2.189500in}}%
\pgfusepath{clip}%
\pgfsetbuttcap%
\pgfsetroundjoin%
\pgfsetlinewidth{0.050187pt}%
\definecolor{currentstroke}{rgb}{1.000000,1.000000,1.000000}%
\pgfsetstrokecolor{currentstroke}%
\pgfsetdash{}{0pt}%
\pgfusepath{stroke}%
\end{pgfscope}%
\begin{pgfscope}%
\pgfpathrectangle{\pgfqpoint{0.100000in}{0.100000in}}{\pgfqpoint{3.420221in}{2.189500in}}%
\pgfusepath{clip}%
\pgfsetbuttcap%
\pgfsetroundjoin%
\pgfsetlinewidth{0.050187pt}%
\definecolor{currentstroke}{rgb}{1.000000,1.000000,1.000000}%
\pgfsetstrokecolor{currentstroke}%
\pgfsetdash{}{0pt}%
\pgfusepath{stroke}%
\end{pgfscope}%
\begin{pgfscope}%
\pgfpathrectangle{\pgfqpoint{0.100000in}{0.100000in}}{\pgfqpoint{3.420221in}{2.189500in}}%
\pgfusepath{clip}%
\pgfsetbuttcap%
\pgfsetroundjoin%
\pgfsetlinewidth{0.050187pt}%
\definecolor{currentstroke}{rgb}{1.000000,1.000000,1.000000}%
\pgfsetstrokecolor{currentstroke}%
\pgfsetdash{}{0pt}%
\pgfusepath{stroke}%
\end{pgfscope}%
\begin{pgfscope}%
\pgfpathrectangle{\pgfqpoint{0.100000in}{0.100000in}}{\pgfqpoint{3.420221in}{2.189500in}}%
\pgfusepath{clip}%
\pgfsetbuttcap%
\pgfsetroundjoin%
\pgfsetlinewidth{0.050187pt}%
\definecolor{currentstroke}{rgb}{1.000000,1.000000,1.000000}%
\pgfsetstrokecolor{currentstroke}%
\pgfsetdash{}{0pt}%
\pgfusepath{stroke}%
\end{pgfscope}%
\begin{pgfscope}%
\pgfpathrectangle{\pgfqpoint{0.100000in}{0.100000in}}{\pgfqpoint{3.420221in}{2.189500in}}%
\pgfusepath{clip}%
\pgfsetbuttcap%
\pgfsetroundjoin%
\pgfsetlinewidth{0.050187pt}%
\definecolor{currentstroke}{rgb}{1.000000,1.000000,1.000000}%
\pgfsetstrokecolor{currentstroke}%
\pgfsetdash{}{0pt}%
\pgfusepath{stroke}%
\end{pgfscope}%
\begin{pgfscope}%
\pgfpathrectangle{\pgfqpoint{0.100000in}{0.100000in}}{\pgfqpoint{3.420221in}{2.189500in}}%
\pgfusepath{clip}%
\pgfsetbuttcap%
\pgfsetroundjoin%
\pgfsetlinewidth{0.050187pt}%
\definecolor{currentstroke}{rgb}{1.000000,1.000000,1.000000}%
\pgfsetstrokecolor{currentstroke}%
\pgfsetdash{}{0pt}%
\pgfusepath{stroke}%
\end{pgfscope}%
\begin{pgfscope}%
\pgfpathrectangle{\pgfqpoint{0.100000in}{0.100000in}}{\pgfqpoint{3.420221in}{2.189500in}}%
\pgfusepath{clip}%
\pgfsetbuttcap%
\pgfsetroundjoin%
\pgfsetlinewidth{0.050187pt}%
\definecolor{currentstroke}{rgb}{1.000000,1.000000,1.000000}%
\pgfsetstrokecolor{currentstroke}%
\pgfsetdash{}{0pt}%
\pgfusepath{stroke}%
\end{pgfscope}%
\begin{pgfscope}%
\pgfpathrectangle{\pgfqpoint{0.100000in}{0.100000in}}{\pgfqpoint{3.420221in}{2.189500in}}%
\pgfusepath{clip}%
\pgfsetbuttcap%
\pgfsetroundjoin%
\pgfsetlinewidth{0.050187pt}%
\definecolor{currentstroke}{rgb}{1.000000,1.000000,1.000000}%
\pgfsetstrokecolor{currentstroke}%
\pgfsetdash{}{0pt}%
\pgfusepath{stroke}%
\end{pgfscope}%
\begin{pgfscope}%
\pgfpathrectangle{\pgfqpoint{0.100000in}{0.100000in}}{\pgfqpoint{3.420221in}{2.189500in}}%
\pgfusepath{clip}%
\pgfsetbuttcap%
\pgfsetroundjoin%
\pgfsetlinewidth{0.050187pt}%
\definecolor{currentstroke}{rgb}{1.000000,1.000000,1.000000}%
\pgfsetstrokecolor{currentstroke}%
\pgfsetdash{}{0pt}%
\pgfusepath{stroke}%
\end{pgfscope}%
\begin{pgfscope}%
\pgfpathrectangle{\pgfqpoint{0.100000in}{0.100000in}}{\pgfqpoint{3.420221in}{2.189500in}}%
\pgfusepath{clip}%
\pgfsetbuttcap%
\pgfsetroundjoin%
\pgfsetlinewidth{0.050187pt}%
\definecolor{currentstroke}{rgb}{1.000000,1.000000,1.000000}%
\pgfsetstrokecolor{currentstroke}%
\pgfsetdash{}{0pt}%
\pgfusepath{stroke}%
\end{pgfscope}%
\begin{pgfscope}%
\pgfpathrectangle{\pgfqpoint{0.100000in}{0.100000in}}{\pgfqpoint{3.420221in}{2.189500in}}%
\pgfusepath{clip}%
\pgfsetbuttcap%
\pgfsetroundjoin%
\pgfsetlinewidth{0.050187pt}%
\definecolor{currentstroke}{rgb}{1.000000,1.000000,1.000000}%
\pgfsetstrokecolor{currentstroke}%
\pgfsetdash{}{0pt}%
\pgfusepath{stroke}%
\end{pgfscope}%
\begin{pgfscope}%
\pgfpathrectangle{\pgfqpoint{0.100000in}{0.100000in}}{\pgfqpoint{3.420221in}{2.189500in}}%
\pgfusepath{clip}%
\pgfsetbuttcap%
\pgfsetroundjoin%
\pgfsetlinewidth{0.050187pt}%
\definecolor{currentstroke}{rgb}{1.000000,1.000000,1.000000}%
\pgfsetstrokecolor{currentstroke}%
\pgfsetdash{}{0pt}%
\pgfusepath{stroke}%
\end{pgfscope}%
\begin{pgfscope}%
\pgfpathrectangle{\pgfqpoint{3.131332in}{0.488889in}}{\pgfqpoint{0.342022in}{0.875800in}}%
\pgfusepath{clip}%
\pgfsetbuttcap%
\pgfsetmiterjoin%
\definecolor{currentfill}{rgb}{0.000000,0.843137,0.578431}%
\pgfsetfillcolor{currentfill}%
\pgfsetlinewidth{0.000000pt}%
\definecolor{currentstroke}{rgb}{0.000000,0.000000,0.000000}%
\pgfsetstrokecolor{currentstroke}%
\pgfsetstrokeopacity{0.000000}%
\pgfsetdash{}{0pt}%
\pgfpathmoveto{\pgfqpoint{3.385501in}{0.488889in}}%
\pgfpathlineto{\pgfqpoint{3.473354in}{0.488889in}}%
\pgfpathlineto{\pgfqpoint{3.473354in}{0.576469in}}%
\pgfpathlineto{\pgfqpoint{3.385501in}{0.576469in}}%
\pgfpathclose%
\pgfusepath{fill}%
\end{pgfscope}%
\begin{pgfscope}%
\pgfpathrectangle{\pgfqpoint{3.131332in}{0.488889in}}{\pgfqpoint{0.342022in}{0.875800in}}%
\pgfusepath{clip}%
\pgfsetbuttcap%
\pgfsetmiterjoin%
\definecolor{currentfill}{rgb}{0.000000,0.760784,0.619608}%
\pgfsetfillcolor{currentfill}%
\pgfsetlinewidth{0.000000pt}%
\definecolor{currentstroke}{rgb}{0.000000,0.000000,0.000000}%
\pgfsetstrokecolor{currentstroke}%
\pgfsetstrokeopacity{0.000000}%
\pgfsetdash{}{0pt}%
\pgfpathmoveto{\pgfqpoint{3.380139in}{0.497647in}}%
\pgfpathlineto{\pgfqpoint{3.473354in}{0.497647in}}%
\pgfpathlineto{\pgfqpoint{3.473354in}{0.585227in}}%
\pgfpathlineto{\pgfqpoint{3.380139in}{0.585227in}}%
\pgfpathclose%
\pgfusepath{fill}%
\end{pgfscope}%
\begin{pgfscope}%
\pgfpathrectangle{\pgfqpoint{3.131332in}{0.488889in}}{\pgfqpoint{0.342022in}{0.875800in}}%
\pgfusepath{clip}%
\pgfsetbuttcap%
\pgfsetmiterjoin%
\definecolor{currentfill}{rgb}{0.000000,0.725490,0.637255}%
\pgfsetfillcolor{currentfill}%
\pgfsetlinewidth{0.000000pt}%
\definecolor{currentstroke}{rgb}{0.000000,0.000000,0.000000}%
\pgfsetstrokecolor{currentstroke}%
\pgfsetstrokeopacity{0.000000}%
\pgfsetdash{}{0pt}%
\pgfpathmoveto{\pgfqpoint{3.378010in}{0.506405in}}%
\pgfpathlineto{\pgfqpoint{3.473354in}{0.506405in}}%
\pgfpathlineto{\pgfqpoint{3.473354in}{0.593985in}}%
\pgfpathlineto{\pgfqpoint{3.378010in}{0.593985in}}%
\pgfpathclose%
\pgfusepath{fill}%
\end{pgfscope}%
\begin{pgfscope}%
\pgfpathrectangle{\pgfqpoint{3.131332in}{0.488889in}}{\pgfqpoint{0.342022in}{0.875800in}}%
\pgfusepath{clip}%
\pgfsetbuttcap%
\pgfsetmiterjoin%
\definecolor{currentfill}{rgb}{0.000000,0.678431,0.660784}%
\pgfsetfillcolor{currentfill}%
\pgfsetlinewidth{0.000000pt}%
\definecolor{currentstroke}{rgb}{0.000000,0.000000,0.000000}%
\pgfsetstrokecolor{currentstroke}%
\pgfsetstrokeopacity{0.000000}%
\pgfsetdash{}{0pt}%
\pgfpathmoveto{\pgfqpoint{3.374999in}{0.515163in}}%
\pgfpathlineto{\pgfqpoint{3.473354in}{0.515163in}}%
\pgfpathlineto{\pgfqpoint{3.473354in}{0.602743in}}%
\pgfpathlineto{\pgfqpoint{3.374999in}{0.602743in}}%
\pgfpathclose%
\pgfusepath{fill}%
\end{pgfscope}%
\begin{pgfscope}%
\pgfpathrectangle{\pgfqpoint{3.131332in}{0.488889in}}{\pgfqpoint{0.342022in}{0.875800in}}%
\pgfusepath{clip}%
\pgfsetbuttcap%
\pgfsetmiterjoin%
\definecolor{currentfill}{rgb}{0.000000,0.658824,0.670588}%
\pgfsetfillcolor{currentfill}%
\pgfsetlinewidth{0.000000pt}%
\definecolor{currentstroke}{rgb}{0.000000,0.000000,0.000000}%
\pgfsetstrokecolor{currentstroke}%
\pgfsetstrokeopacity{0.000000}%
\pgfsetdash{}{0pt}%
\pgfpathmoveto{\pgfqpoint{3.373653in}{0.523921in}}%
\pgfpathlineto{\pgfqpoint{3.473354in}{0.523921in}}%
\pgfpathlineto{\pgfqpoint{3.473354in}{0.611501in}}%
\pgfpathlineto{\pgfqpoint{3.373653in}{0.611501in}}%
\pgfpathclose%
\pgfusepath{fill}%
\end{pgfscope}%
\begin{pgfscope}%
\pgfpathrectangle{\pgfqpoint{3.131332in}{0.488889in}}{\pgfqpoint{0.342022in}{0.875800in}}%
\pgfusepath{clip}%
\pgfsetbuttcap%
\pgfsetmiterjoin%
\definecolor{currentfill}{rgb}{0.000000,0.631373,0.684314}%
\pgfsetfillcolor{currentfill}%
\pgfsetlinewidth{0.000000pt}%
\definecolor{currentstroke}{rgb}{0.000000,0.000000,0.000000}%
\pgfsetstrokecolor{currentstroke}%
\pgfsetstrokeopacity{0.000000}%
\pgfsetdash{}{0pt}%
\pgfpathmoveto{\pgfqpoint{3.371845in}{0.532679in}}%
\pgfpathlineto{\pgfqpoint{3.473354in}{0.532679in}}%
\pgfpathlineto{\pgfqpoint{3.473354in}{0.620259in}}%
\pgfpathlineto{\pgfqpoint{3.371845in}{0.620259in}}%
\pgfpathclose%
\pgfusepath{fill}%
\end{pgfscope}%
\begin{pgfscope}%
\pgfpathrectangle{\pgfqpoint{3.131332in}{0.488889in}}{\pgfqpoint{0.342022in}{0.875800in}}%
\pgfusepath{clip}%
\pgfsetbuttcap%
\pgfsetmiterjoin%
\definecolor{currentfill}{rgb}{0.000000,0.615686,0.692157}%
\pgfsetfillcolor{currentfill}%
\pgfsetlinewidth{0.000000pt}%
\definecolor{currentstroke}{rgb}{0.000000,0.000000,0.000000}%
\pgfsetstrokecolor{currentstroke}%
\pgfsetstrokeopacity{0.000000}%
\pgfsetdash{}{0pt}%
\pgfpathmoveto{\pgfqpoint{3.370684in}{0.541437in}}%
\pgfpathlineto{\pgfqpoint{3.473354in}{0.541437in}}%
\pgfpathlineto{\pgfqpoint{3.473354in}{0.629017in}}%
\pgfpathlineto{\pgfqpoint{3.370684in}{0.629017in}}%
\pgfpathclose%
\pgfusepath{fill}%
\end{pgfscope}%
\begin{pgfscope}%
\pgfpathrectangle{\pgfqpoint{3.131332in}{0.488889in}}{\pgfqpoint{0.342022in}{0.875800in}}%
\pgfusepath{clip}%
\pgfsetbuttcap%
\pgfsetmiterjoin%
\definecolor{currentfill}{rgb}{0.000000,0.600000,0.700000}%
\pgfsetfillcolor{currentfill}%
\pgfsetlinewidth{0.000000pt}%
\definecolor{currentstroke}{rgb}{0.000000,0.000000,0.000000}%
\pgfsetstrokecolor{currentstroke}%
\pgfsetstrokeopacity{0.000000}%
\pgfsetdash{}{0pt}%
\pgfpathmoveto{\pgfqpoint{3.369885in}{0.550195in}}%
\pgfpathlineto{\pgfqpoint{3.473354in}{0.550195in}}%
\pgfpathlineto{\pgfqpoint{3.473354in}{0.637775in}}%
\pgfpathlineto{\pgfqpoint{3.369885in}{0.637775in}}%
\pgfpathclose%
\pgfusepath{fill}%
\end{pgfscope}%
\begin{pgfscope}%
\pgfpathrectangle{\pgfqpoint{3.131332in}{0.488889in}}{\pgfqpoint{0.342022in}{0.875800in}}%
\pgfusepath{clip}%
\pgfsetbuttcap%
\pgfsetmiterjoin%
\definecolor{currentfill}{rgb}{0.000000,0.584314,0.707843}%
\pgfsetfillcolor{currentfill}%
\pgfsetlinewidth{0.000000pt}%
\definecolor{currentstroke}{rgb}{0.000000,0.000000,0.000000}%
\pgfsetstrokecolor{currentstroke}%
\pgfsetstrokeopacity{0.000000}%
\pgfsetdash{}{0pt}%
\pgfpathmoveto{\pgfqpoint{3.368858in}{0.558953in}}%
\pgfpathlineto{\pgfqpoint{3.473354in}{0.558953in}}%
\pgfpathlineto{\pgfqpoint{3.473354in}{0.646533in}}%
\pgfpathlineto{\pgfqpoint{3.368858in}{0.646533in}}%
\pgfpathclose%
\pgfusepath{fill}%
\end{pgfscope}%
\begin{pgfscope}%
\pgfpathrectangle{\pgfqpoint{3.131332in}{0.488889in}}{\pgfqpoint{0.342022in}{0.875800in}}%
\pgfusepath{clip}%
\pgfsetbuttcap%
\pgfsetmiterjoin%
\definecolor{currentfill}{rgb}{0.000000,0.568627,0.715686}%
\pgfsetfillcolor{currentfill}%
\pgfsetlinewidth{0.000000pt}%
\definecolor{currentstroke}{rgb}{0.000000,0.000000,0.000000}%
\pgfsetstrokecolor{currentstroke}%
\pgfsetstrokeopacity{0.000000}%
\pgfsetdash{}{0pt}%
\pgfpathmoveto{\pgfqpoint{3.367795in}{0.567711in}}%
\pgfpathlineto{\pgfqpoint{3.473354in}{0.567711in}}%
\pgfpathlineto{\pgfqpoint{3.473354in}{0.655291in}}%
\pgfpathlineto{\pgfqpoint{3.367795in}{0.655291in}}%
\pgfpathclose%
\pgfusepath{fill}%
\end{pgfscope}%
\begin{pgfscope}%
\pgfpathrectangle{\pgfqpoint{3.131332in}{0.488889in}}{\pgfqpoint{0.342022in}{0.875800in}}%
\pgfusepath{clip}%
\pgfsetbuttcap%
\pgfsetmiterjoin%
\definecolor{currentfill}{rgb}{0.000000,0.560784,0.719608}%
\pgfsetfillcolor{currentfill}%
\pgfsetlinewidth{0.000000pt}%
\definecolor{currentstroke}{rgb}{0.000000,0.000000,0.000000}%
\pgfsetstrokecolor{currentstroke}%
\pgfsetstrokeopacity{0.000000}%
\pgfsetdash{}{0pt}%
\pgfpathmoveto{\pgfqpoint{3.367247in}{0.576469in}}%
\pgfpathlineto{\pgfqpoint{3.473354in}{0.576469in}}%
\pgfpathlineto{\pgfqpoint{3.473354in}{0.664049in}}%
\pgfpathlineto{\pgfqpoint{3.367247in}{0.664049in}}%
\pgfpathclose%
\pgfusepath{fill}%
\end{pgfscope}%
\begin{pgfscope}%
\pgfpathrectangle{\pgfqpoint{3.131332in}{0.488889in}}{\pgfqpoint{0.342022in}{0.875800in}}%
\pgfusepath{clip}%
\pgfsetbuttcap%
\pgfsetmiterjoin%
\definecolor{currentfill}{rgb}{0.000000,0.549020,0.725490}%
\pgfsetfillcolor{currentfill}%
\pgfsetlinewidth{0.000000pt}%
\definecolor{currentstroke}{rgb}{0.000000,0.000000,0.000000}%
\pgfsetstrokecolor{currentstroke}%
\pgfsetstrokeopacity{0.000000}%
\pgfsetdash{}{0pt}%
\pgfpathmoveto{\pgfqpoint{3.366460in}{0.585227in}}%
\pgfpathlineto{\pgfqpoint{3.473354in}{0.585227in}}%
\pgfpathlineto{\pgfqpoint{3.473354in}{0.672807in}}%
\pgfpathlineto{\pgfqpoint{3.366460in}{0.672807in}}%
\pgfpathclose%
\pgfusepath{fill}%
\end{pgfscope}%
\begin{pgfscope}%
\pgfpathrectangle{\pgfqpoint{3.131332in}{0.488889in}}{\pgfqpoint{0.342022in}{0.875800in}}%
\pgfusepath{clip}%
\pgfsetbuttcap%
\pgfsetmiterjoin%
\definecolor{currentfill}{rgb}{0.000000,0.541176,0.729412}%
\pgfsetfillcolor{currentfill}%
\pgfsetlinewidth{0.000000pt}%
\definecolor{currentstroke}{rgb}{0.000000,0.000000,0.000000}%
\pgfsetstrokecolor{currentstroke}%
\pgfsetstrokeopacity{0.000000}%
\pgfsetdash{}{0pt}%
\pgfpathmoveto{\pgfqpoint{3.365994in}{0.593985in}}%
\pgfpathlineto{\pgfqpoint{3.473354in}{0.593985in}}%
\pgfpathlineto{\pgfqpoint{3.473354in}{0.681565in}}%
\pgfpathlineto{\pgfqpoint{3.365994in}{0.681565in}}%
\pgfpathclose%
\pgfusepath{fill}%
\end{pgfscope}%
\begin{pgfscope}%
\pgfpathrectangle{\pgfqpoint{3.131332in}{0.488889in}}{\pgfqpoint{0.342022in}{0.875800in}}%
\pgfusepath{clip}%
\pgfsetbuttcap%
\pgfsetmiterjoin%
\definecolor{currentfill}{rgb}{0.000000,0.529412,0.735294}%
\pgfsetfillcolor{currentfill}%
\pgfsetlinewidth{0.000000pt}%
\definecolor{currentstroke}{rgb}{0.000000,0.000000,0.000000}%
\pgfsetstrokecolor{currentstroke}%
\pgfsetstrokeopacity{0.000000}%
\pgfsetdash{}{0pt}%
\pgfpathmoveto{\pgfqpoint{3.365204in}{0.602743in}}%
\pgfpathlineto{\pgfqpoint{3.473354in}{0.602743in}}%
\pgfpathlineto{\pgfqpoint{3.473354in}{0.690323in}}%
\pgfpathlineto{\pgfqpoint{3.365204in}{0.690323in}}%
\pgfpathclose%
\pgfusepath{fill}%
\end{pgfscope}%
\begin{pgfscope}%
\pgfpathrectangle{\pgfqpoint{3.131332in}{0.488889in}}{\pgfqpoint{0.342022in}{0.875800in}}%
\pgfusepath{clip}%
\pgfsetbuttcap%
\pgfsetmiterjoin%
\definecolor{currentfill}{rgb}{0.000000,0.517647,0.741176}%
\pgfsetfillcolor{currentfill}%
\pgfsetlinewidth{0.000000pt}%
\definecolor{currentstroke}{rgb}{0.000000,0.000000,0.000000}%
\pgfsetstrokecolor{currentstroke}%
\pgfsetstrokeopacity{0.000000}%
\pgfsetdash{}{0pt}%
\pgfpathmoveto{\pgfqpoint{3.364341in}{0.611501in}}%
\pgfpathlineto{\pgfqpoint{3.473354in}{0.611501in}}%
\pgfpathlineto{\pgfqpoint{3.473354in}{0.699081in}}%
\pgfpathlineto{\pgfqpoint{3.364341in}{0.699081in}}%
\pgfpathclose%
\pgfusepath{fill}%
\end{pgfscope}%
\begin{pgfscope}%
\pgfpathrectangle{\pgfqpoint{3.131332in}{0.488889in}}{\pgfqpoint{0.342022in}{0.875800in}}%
\pgfusepath{clip}%
\pgfsetbuttcap%
\pgfsetmiterjoin%
\definecolor{currentfill}{rgb}{0.000000,0.505882,0.747059}%
\pgfsetfillcolor{currentfill}%
\pgfsetlinewidth{0.000000pt}%
\definecolor{currentstroke}{rgb}{0.000000,0.000000,0.000000}%
\pgfsetstrokecolor{currentstroke}%
\pgfsetstrokeopacity{0.000000}%
\pgfsetdash{}{0pt}%
\pgfpathmoveto{\pgfqpoint{3.363616in}{0.620259in}}%
\pgfpathlineto{\pgfqpoint{3.473354in}{0.620259in}}%
\pgfpathlineto{\pgfqpoint{3.473354in}{0.707839in}}%
\pgfpathlineto{\pgfqpoint{3.363616in}{0.707839in}}%
\pgfpathclose%
\pgfusepath{fill}%
\end{pgfscope}%
\begin{pgfscope}%
\pgfpathrectangle{\pgfqpoint{3.131332in}{0.488889in}}{\pgfqpoint{0.342022in}{0.875800in}}%
\pgfusepath{clip}%
\pgfsetbuttcap%
\pgfsetmiterjoin%
\definecolor{currentfill}{rgb}{0.000000,0.494118,0.752941}%
\pgfsetfillcolor{currentfill}%
\pgfsetlinewidth{0.000000pt}%
\definecolor{currentstroke}{rgb}{0.000000,0.000000,0.000000}%
\pgfsetstrokecolor{currentstroke}%
\pgfsetstrokeopacity{0.000000}%
\pgfsetdash{}{0pt}%
\pgfpathmoveto{\pgfqpoint{3.362781in}{0.629017in}}%
\pgfpathlineto{\pgfqpoint{3.473354in}{0.629017in}}%
\pgfpathlineto{\pgfqpoint{3.473354in}{0.716597in}}%
\pgfpathlineto{\pgfqpoint{3.362781in}{0.716597in}}%
\pgfpathclose%
\pgfusepath{fill}%
\end{pgfscope}%
\begin{pgfscope}%
\pgfpathrectangle{\pgfqpoint{3.131332in}{0.488889in}}{\pgfqpoint{0.342022in}{0.875800in}}%
\pgfusepath{clip}%
\pgfsetbuttcap%
\pgfsetmiterjoin%
\definecolor{currentfill}{rgb}{0.000000,0.486275,0.756863}%
\pgfsetfillcolor{currentfill}%
\pgfsetlinewidth{0.000000pt}%
\definecolor{currentstroke}{rgb}{0.000000,0.000000,0.000000}%
\pgfsetstrokecolor{currentstroke}%
\pgfsetstrokeopacity{0.000000}%
\pgfsetdash{}{0pt}%
\pgfpathmoveto{\pgfqpoint{3.362319in}{0.637775in}}%
\pgfpathlineto{\pgfqpoint{3.473354in}{0.637775in}}%
\pgfpathlineto{\pgfqpoint{3.473354in}{0.725355in}}%
\pgfpathlineto{\pgfqpoint{3.362319in}{0.725355in}}%
\pgfpathclose%
\pgfusepath{fill}%
\end{pgfscope}%
\begin{pgfscope}%
\pgfpathrectangle{\pgfqpoint{3.131332in}{0.488889in}}{\pgfqpoint{0.342022in}{0.875800in}}%
\pgfusepath{clip}%
\pgfsetbuttcap%
\pgfsetmiterjoin%
\definecolor{currentfill}{rgb}{0.000000,0.478431,0.760784}%
\pgfsetfillcolor{currentfill}%
\pgfsetlinewidth{0.000000pt}%
\definecolor{currentstroke}{rgb}{0.000000,0.000000,0.000000}%
\pgfsetstrokecolor{currentstroke}%
\pgfsetstrokeopacity{0.000000}%
\pgfsetdash{}{0pt}%
\pgfpathmoveto{\pgfqpoint{3.361848in}{0.646533in}}%
\pgfpathlineto{\pgfqpoint{3.473354in}{0.646533in}}%
\pgfpathlineto{\pgfqpoint{3.473354in}{0.734113in}}%
\pgfpathlineto{\pgfqpoint{3.361848in}{0.734113in}}%
\pgfpathclose%
\pgfusepath{fill}%
\end{pgfscope}%
\begin{pgfscope}%
\pgfpathrectangle{\pgfqpoint{3.131332in}{0.488889in}}{\pgfqpoint{0.342022in}{0.875800in}}%
\pgfusepath{clip}%
\pgfsetbuttcap%
\pgfsetmiterjoin%
\definecolor{currentfill}{rgb}{0.000000,0.474510,0.762745}%
\pgfsetfillcolor{currentfill}%
\pgfsetlinewidth{0.000000pt}%
\definecolor{currentstroke}{rgb}{0.000000,0.000000,0.000000}%
\pgfsetstrokecolor{currentstroke}%
\pgfsetstrokeopacity{0.000000}%
\pgfsetdash{}{0pt}%
\pgfpathmoveto{\pgfqpoint{3.361650in}{0.655291in}}%
\pgfpathlineto{\pgfqpoint{3.473354in}{0.655291in}}%
\pgfpathlineto{\pgfqpoint{3.473354in}{0.742871in}}%
\pgfpathlineto{\pgfqpoint{3.361650in}{0.742871in}}%
\pgfpathclose%
\pgfusepath{fill}%
\end{pgfscope}%
\begin{pgfscope}%
\pgfpathrectangle{\pgfqpoint{3.131332in}{0.488889in}}{\pgfqpoint{0.342022in}{0.875800in}}%
\pgfusepath{clip}%
\pgfsetbuttcap%
\pgfsetmiterjoin%
\definecolor{currentfill}{rgb}{0.000000,0.462745,0.768627}%
\pgfsetfillcolor{currentfill}%
\pgfsetlinewidth{0.000000pt}%
\definecolor{currentstroke}{rgb}{0.000000,0.000000,0.000000}%
\pgfsetstrokecolor{currentstroke}%
\pgfsetstrokeopacity{0.000000}%
\pgfsetdash{}{0pt}%
\pgfpathmoveto{\pgfqpoint{3.360818in}{0.664049in}}%
\pgfpathlineto{\pgfqpoint{3.473354in}{0.664049in}}%
\pgfpathlineto{\pgfqpoint{3.473354in}{0.751629in}}%
\pgfpathlineto{\pgfqpoint{3.360818in}{0.751629in}}%
\pgfpathclose%
\pgfusepath{fill}%
\end{pgfscope}%
\begin{pgfscope}%
\pgfpathrectangle{\pgfqpoint{3.131332in}{0.488889in}}{\pgfqpoint{0.342022in}{0.875800in}}%
\pgfusepath{clip}%
\pgfsetbuttcap%
\pgfsetmiterjoin%
\definecolor{currentfill}{rgb}{0.000000,0.454902,0.772549}%
\pgfsetfillcolor{currentfill}%
\pgfsetlinewidth{0.000000pt}%
\definecolor{currentstroke}{rgb}{0.000000,0.000000,0.000000}%
\pgfsetstrokecolor{currentstroke}%
\pgfsetstrokeopacity{0.000000}%
\pgfsetdash{}{0pt}%
\pgfpathmoveto{\pgfqpoint{3.360438in}{0.672807in}}%
\pgfpathlineto{\pgfqpoint{3.473354in}{0.672807in}}%
\pgfpathlineto{\pgfqpoint{3.473354in}{0.760387in}}%
\pgfpathlineto{\pgfqpoint{3.360438in}{0.760387in}}%
\pgfpathclose%
\pgfusepath{fill}%
\end{pgfscope}%
\begin{pgfscope}%
\pgfpathrectangle{\pgfqpoint{3.131332in}{0.488889in}}{\pgfqpoint{0.342022in}{0.875800in}}%
\pgfusepath{clip}%
\pgfsetbuttcap%
\pgfsetmiterjoin%
\definecolor{currentfill}{rgb}{0.000000,0.447059,0.776471}%
\pgfsetfillcolor{currentfill}%
\pgfsetlinewidth{0.000000pt}%
\definecolor{currentstroke}{rgb}{0.000000,0.000000,0.000000}%
\pgfsetstrokecolor{currentstroke}%
\pgfsetstrokeopacity{0.000000}%
\pgfsetdash{}{0pt}%
\pgfpathmoveto{\pgfqpoint{3.359811in}{0.681565in}}%
\pgfpathlineto{\pgfqpoint{3.473354in}{0.681565in}}%
\pgfpathlineto{\pgfqpoint{3.473354in}{0.769145in}}%
\pgfpathlineto{\pgfqpoint{3.359811in}{0.769145in}}%
\pgfpathclose%
\pgfusepath{fill}%
\end{pgfscope}%
\begin{pgfscope}%
\pgfpathrectangle{\pgfqpoint{3.131332in}{0.488889in}}{\pgfqpoint{0.342022in}{0.875800in}}%
\pgfusepath{clip}%
\pgfsetbuttcap%
\pgfsetmiterjoin%
\definecolor{currentfill}{rgb}{0.000000,0.443137,0.778431}%
\pgfsetfillcolor{currentfill}%
\pgfsetlinewidth{0.000000pt}%
\definecolor{currentstroke}{rgb}{0.000000,0.000000,0.000000}%
\pgfsetstrokecolor{currentstroke}%
\pgfsetstrokeopacity{0.000000}%
\pgfsetdash{}{0pt}%
\pgfpathmoveto{\pgfqpoint{3.359476in}{0.690323in}}%
\pgfpathlineto{\pgfqpoint{3.473354in}{0.690323in}}%
\pgfpathlineto{\pgfqpoint{3.473354in}{0.777903in}}%
\pgfpathlineto{\pgfqpoint{3.359476in}{0.777903in}}%
\pgfpathclose%
\pgfusepath{fill}%
\end{pgfscope}%
\begin{pgfscope}%
\pgfpathrectangle{\pgfqpoint{3.131332in}{0.488889in}}{\pgfqpoint{0.342022in}{0.875800in}}%
\pgfusepath{clip}%
\pgfsetbuttcap%
\pgfsetmiterjoin%
\definecolor{currentfill}{rgb}{0.000000,0.431373,0.784314}%
\pgfsetfillcolor{currentfill}%
\pgfsetlinewidth{0.000000pt}%
\definecolor{currentstroke}{rgb}{0.000000,0.000000,0.000000}%
\pgfsetstrokecolor{currentstroke}%
\pgfsetstrokeopacity{0.000000}%
\pgfsetdash{}{0pt}%
\pgfpathmoveto{\pgfqpoint{3.358891in}{0.699081in}}%
\pgfpathlineto{\pgfqpoint{3.473354in}{0.699081in}}%
\pgfpathlineto{\pgfqpoint{3.473354in}{0.786661in}}%
\pgfpathlineto{\pgfqpoint{3.358891in}{0.786661in}}%
\pgfpathclose%
\pgfusepath{fill}%
\end{pgfscope}%
\begin{pgfscope}%
\pgfpathrectangle{\pgfqpoint{3.131332in}{0.488889in}}{\pgfqpoint{0.342022in}{0.875800in}}%
\pgfusepath{clip}%
\pgfsetbuttcap%
\pgfsetmiterjoin%
\definecolor{currentfill}{rgb}{0.000000,0.423529,0.788235}%
\pgfsetfillcolor{currentfill}%
\pgfsetlinewidth{0.000000pt}%
\definecolor{currentstroke}{rgb}{0.000000,0.000000,0.000000}%
\pgfsetstrokecolor{currentstroke}%
\pgfsetstrokeopacity{0.000000}%
\pgfsetdash{}{0pt}%
\pgfpathmoveto{\pgfqpoint{3.358228in}{0.707839in}}%
\pgfpathlineto{\pgfqpoint{3.473354in}{0.707839in}}%
\pgfpathlineto{\pgfqpoint{3.473354in}{0.795419in}}%
\pgfpathlineto{\pgfqpoint{3.358228in}{0.795419in}}%
\pgfpathclose%
\pgfusepath{fill}%
\end{pgfscope}%
\begin{pgfscope}%
\pgfpathrectangle{\pgfqpoint{3.131332in}{0.488889in}}{\pgfqpoint{0.342022in}{0.875800in}}%
\pgfusepath{clip}%
\pgfsetbuttcap%
\pgfsetmiterjoin%
\definecolor{currentfill}{rgb}{0.000000,0.415686,0.792157}%
\pgfsetfillcolor{currentfill}%
\pgfsetlinewidth{0.000000pt}%
\definecolor{currentstroke}{rgb}{0.000000,0.000000,0.000000}%
\pgfsetstrokecolor{currentstroke}%
\pgfsetstrokeopacity{0.000000}%
\pgfsetdash{}{0pt}%
\pgfpathmoveto{\pgfqpoint{3.357808in}{0.716597in}}%
\pgfpathlineto{\pgfqpoint{3.473354in}{0.716597in}}%
\pgfpathlineto{\pgfqpoint{3.473354in}{0.804177in}}%
\pgfpathlineto{\pgfqpoint{3.357808in}{0.804177in}}%
\pgfpathclose%
\pgfusepath{fill}%
\end{pgfscope}%
\begin{pgfscope}%
\pgfpathrectangle{\pgfqpoint{3.131332in}{0.488889in}}{\pgfqpoint{0.342022in}{0.875800in}}%
\pgfusepath{clip}%
\pgfsetbuttcap%
\pgfsetmiterjoin%
\definecolor{currentfill}{rgb}{0.000000,0.407843,0.796078}%
\pgfsetfillcolor{currentfill}%
\pgfsetlinewidth{0.000000pt}%
\definecolor{currentstroke}{rgb}{0.000000,0.000000,0.000000}%
\pgfsetstrokecolor{currentstroke}%
\pgfsetstrokeopacity{0.000000}%
\pgfsetdash{}{0pt}%
\pgfpathmoveto{\pgfqpoint{3.357325in}{0.725355in}}%
\pgfpathlineto{\pgfqpoint{3.473354in}{0.725355in}}%
\pgfpathlineto{\pgfqpoint{3.473354in}{0.812935in}}%
\pgfpathlineto{\pgfqpoint{3.357325in}{0.812935in}}%
\pgfpathclose%
\pgfusepath{fill}%
\end{pgfscope}%
\begin{pgfscope}%
\pgfpathrectangle{\pgfqpoint{3.131332in}{0.488889in}}{\pgfqpoint{0.342022in}{0.875800in}}%
\pgfusepath{clip}%
\pgfsetbuttcap%
\pgfsetmiterjoin%
\definecolor{currentfill}{rgb}{0.000000,0.407843,0.796078}%
\pgfsetfillcolor{currentfill}%
\pgfsetlinewidth{0.000000pt}%
\definecolor{currentstroke}{rgb}{0.000000,0.000000,0.000000}%
\pgfsetstrokecolor{currentstroke}%
\pgfsetstrokeopacity{0.000000}%
\pgfsetdash{}{0pt}%
\pgfpathmoveto{\pgfqpoint{3.357325in}{0.734113in}}%
\pgfpathlineto{\pgfqpoint{3.473354in}{0.734113in}}%
\pgfpathlineto{\pgfqpoint{3.473354in}{0.821693in}}%
\pgfpathlineto{\pgfqpoint{3.357325in}{0.821693in}}%
\pgfpathclose%
\pgfusepath{fill}%
\end{pgfscope}%
\begin{pgfscope}%
\pgfpathrectangle{\pgfqpoint{3.131332in}{0.488889in}}{\pgfqpoint{0.342022in}{0.875800in}}%
\pgfusepath{clip}%
\pgfsetbuttcap%
\pgfsetmiterjoin%
\definecolor{currentfill}{rgb}{0.000000,0.403922,0.798039}%
\pgfsetfillcolor{currentfill}%
\pgfsetlinewidth{0.000000pt}%
\definecolor{currentstroke}{rgb}{0.000000,0.000000,0.000000}%
\pgfsetstrokecolor{currentstroke}%
\pgfsetstrokeopacity{0.000000}%
\pgfsetdash{}{0pt}%
\pgfpathmoveto{\pgfqpoint{3.357098in}{0.742871in}}%
\pgfpathlineto{\pgfqpoint{3.473354in}{0.742871in}}%
\pgfpathlineto{\pgfqpoint{3.473354in}{0.830451in}}%
\pgfpathlineto{\pgfqpoint{3.357098in}{0.830451in}}%
\pgfpathclose%
\pgfusepath{fill}%
\end{pgfscope}%
\begin{pgfscope}%
\pgfpathrectangle{\pgfqpoint{3.131332in}{0.488889in}}{\pgfqpoint{0.342022in}{0.875800in}}%
\pgfusepath{clip}%
\pgfsetbuttcap%
\pgfsetmiterjoin%
\definecolor{currentfill}{rgb}{0.000000,0.396078,0.801961}%
\pgfsetfillcolor{currentfill}%
\pgfsetlinewidth{0.000000pt}%
\definecolor{currentstroke}{rgb}{0.000000,0.000000,0.000000}%
\pgfsetstrokecolor{currentstroke}%
\pgfsetstrokeopacity{0.000000}%
\pgfsetdash{}{0pt}%
\pgfpathmoveto{\pgfqpoint{3.356540in}{0.751629in}}%
\pgfpathlineto{\pgfqpoint{3.473354in}{0.751629in}}%
\pgfpathlineto{\pgfqpoint{3.473354in}{0.839209in}}%
\pgfpathlineto{\pgfqpoint{3.356540in}{0.839209in}}%
\pgfpathclose%
\pgfusepath{fill}%
\end{pgfscope}%
\begin{pgfscope}%
\pgfpathrectangle{\pgfqpoint{3.131332in}{0.488889in}}{\pgfqpoint{0.342022in}{0.875800in}}%
\pgfusepath{clip}%
\pgfsetbuttcap%
\pgfsetmiterjoin%
\definecolor{currentfill}{rgb}{0.000000,0.392157,0.803922}%
\pgfsetfillcolor{currentfill}%
\pgfsetlinewidth{0.000000pt}%
\definecolor{currentstroke}{rgb}{0.000000,0.000000,0.000000}%
\pgfsetstrokecolor{currentstroke}%
\pgfsetstrokeopacity{0.000000}%
\pgfsetdash{}{0pt}%
\pgfpathmoveto{\pgfqpoint{3.356223in}{0.760387in}}%
\pgfpathlineto{\pgfqpoint{3.473354in}{0.760387in}}%
\pgfpathlineto{\pgfqpoint{3.473354in}{0.847967in}}%
\pgfpathlineto{\pgfqpoint{3.356223in}{0.847967in}}%
\pgfpathclose%
\pgfusepath{fill}%
\end{pgfscope}%
\begin{pgfscope}%
\pgfpathrectangle{\pgfqpoint{3.131332in}{0.488889in}}{\pgfqpoint{0.342022in}{0.875800in}}%
\pgfusepath{clip}%
\pgfsetbuttcap%
\pgfsetmiterjoin%
\definecolor{currentfill}{rgb}{0.000000,0.376471,0.811765}%
\pgfsetfillcolor{currentfill}%
\pgfsetlinewidth{0.000000pt}%
\definecolor{currentstroke}{rgb}{0.000000,0.000000,0.000000}%
\pgfsetstrokecolor{currentstroke}%
\pgfsetstrokeopacity{0.000000}%
\pgfsetdash{}{0pt}%
\pgfpathmoveto{\pgfqpoint{3.355316in}{0.769145in}}%
\pgfpathlineto{\pgfqpoint{3.473354in}{0.769145in}}%
\pgfpathlineto{\pgfqpoint{3.473354in}{0.856725in}}%
\pgfpathlineto{\pgfqpoint{3.355316in}{0.856725in}}%
\pgfpathclose%
\pgfusepath{fill}%
\end{pgfscope}%
\begin{pgfscope}%
\pgfpathrectangle{\pgfqpoint{3.131332in}{0.488889in}}{\pgfqpoint{0.342022in}{0.875800in}}%
\pgfusepath{clip}%
\pgfsetbuttcap%
\pgfsetmiterjoin%
\definecolor{currentfill}{rgb}{0.000000,0.372549,0.813725}%
\pgfsetfillcolor{currentfill}%
\pgfsetlinewidth{0.000000pt}%
\definecolor{currentstroke}{rgb}{0.000000,0.000000,0.000000}%
\pgfsetstrokecolor{currentstroke}%
\pgfsetstrokeopacity{0.000000}%
\pgfsetdash{}{0pt}%
\pgfpathmoveto{\pgfqpoint{3.355016in}{0.777903in}}%
\pgfpathlineto{\pgfqpoint{3.473354in}{0.777903in}}%
\pgfpathlineto{\pgfqpoint{3.473354in}{0.865483in}}%
\pgfpathlineto{\pgfqpoint{3.355016in}{0.865483in}}%
\pgfpathclose%
\pgfusepath{fill}%
\end{pgfscope}%
\begin{pgfscope}%
\pgfpathrectangle{\pgfqpoint{3.131332in}{0.488889in}}{\pgfqpoint{0.342022in}{0.875800in}}%
\pgfusepath{clip}%
\pgfsetbuttcap%
\pgfsetmiterjoin%
\definecolor{currentfill}{rgb}{0.000000,0.364706,0.817647}%
\pgfsetfillcolor{currentfill}%
\pgfsetlinewidth{0.000000pt}%
\definecolor{currentstroke}{rgb}{0.000000,0.000000,0.000000}%
\pgfsetstrokecolor{currentstroke}%
\pgfsetstrokeopacity{0.000000}%
\pgfsetdash{}{0pt}%
\pgfpathmoveto{\pgfqpoint{3.354495in}{0.786661in}}%
\pgfpathlineto{\pgfqpoint{3.473354in}{0.786661in}}%
\pgfpathlineto{\pgfqpoint{3.473354in}{0.874241in}}%
\pgfpathlineto{\pgfqpoint{3.354495in}{0.874241in}}%
\pgfpathclose%
\pgfusepath{fill}%
\end{pgfscope}%
\begin{pgfscope}%
\pgfpathrectangle{\pgfqpoint{3.131332in}{0.488889in}}{\pgfqpoint{0.342022in}{0.875800in}}%
\pgfusepath{clip}%
\pgfsetbuttcap%
\pgfsetmiterjoin%
\definecolor{currentfill}{rgb}{0.000000,0.360784,0.819608}%
\pgfsetfillcolor{currentfill}%
\pgfsetlinewidth{0.000000pt}%
\definecolor{currentstroke}{rgb}{0.000000,0.000000,0.000000}%
\pgfsetstrokecolor{currentstroke}%
\pgfsetstrokeopacity{0.000000}%
\pgfsetdash{}{0pt}%
\pgfpathmoveto{\pgfqpoint{3.354237in}{0.795419in}}%
\pgfpathlineto{\pgfqpoint{3.473354in}{0.795419in}}%
\pgfpathlineto{\pgfqpoint{3.473354in}{0.882999in}}%
\pgfpathlineto{\pgfqpoint{3.354237in}{0.882999in}}%
\pgfpathclose%
\pgfusepath{fill}%
\end{pgfscope}%
\begin{pgfscope}%
\pgfpathrectangle{\pgfqpoint{3.131332in}{0.488889in}}{\pgfqpoint{0.342022in}{0.875800in}}%
\pgfusepath{clip}%
\pgfsetbuttcap%
\pgfsetmiterjoin%
\definecolor{currentfill}{rgb}{0.000000,0.356863,0.821569}%
\pgfsetfillcolor{currentfill}%
\pgfsetlinewidth{0.000000pt}%
\definecolor{currentstroke}{rgb}{0.000000,0.000000,0.000000}%
\pgfsetstrokecolor{currentstroke}%
\pgfsetstrokeopacity{0.000000}%
\pgfsetdash{}{0pt}%
\pgfpathmoveto{\pgfqpoint{3.353878in}{0.804177in}}%
\pgfpathlineto{\pgfqpoint{3.473354in}{0.804177in}}%
\pgfpathlineto{\pgfqpoint{3.473354in}{0.891757in}}%
\pgfpathlineto{\pgfqpoint{3.353878in}{0.891757in}}%
\pgfpathclose%
\pgfusepath{fill}%
\end{pgfscope}%
\begin{pgfscope}%
\pgfpathrectangle{\pgfqpoint{3.131332in}{0.488889in}}{\pgfqpoint{0.342022in}{0.875800in}}%
\pgfusepath{clip}%
\pgfsetbuttcap%
\pgfsetmiterjoin%
\definecolor{currentfill}{rgb}{0.000000,0.356863,0.821569}%
\pgfsetfillcolor{currentfill}%
\pgfsetlinewidth{0.000000pt}%
\definecolor{currentstroke}{rgb}{0.000000,0.000000,0.000000}%
\pgfsetstrokecolor{currentstroke}%
\pgfsetstrokeopacity{0.000000}%
\pgfsetdash{}{0pt}%
\pgfpathmoveto{\pgfqpoint{3.353878in}{0.812935in}}%
\pgfpathlineto{\pgfqpoint{3.473354in}{0.812935in}}%
\pgfpathlineto{\pgfqpoint{3.473354in}{0.900515in}}%
\pgfpathlineto{\pgfqpoint{3.353878in}{0.900515in}}%
\pgfpathclose%
\pgfusepath{fill}%
\end{pgfscope}%
\begin{pgfscope}%
\pgfpathrectangle{\pgfqpoint{3.131332in}{0.488889in}}{\pgfqpoint{0.342022in}{0.875800in}}%
\pgfusepath{clip}%
\pgfsetbuttcap%
\pgfsetmiterjoin%
\definecolor{currentfill}{rgb}{0.000000,0.356863,0.821569}%
\pgfsetfillcolor{currentfill}%
\pgfsetlinewidth{0.000000pt}%
\definecolor{currentstroke}{rgb}{0.000000,0.000000,0.000000}%
\pgfsetstrokecolor{currentstroke}%
\pgfsetstrokeopacity{0.000000}%
\pgfsetdash{}{0pt}%
\pgfpathmoveto{\pgfqpoint{3.353878in}{0.821693in}}%
\pgfpathlineto{\pgfqpoint{3.473354in}{0.821693in}}%
\pgfpathlineto{\pgfqpoint{3.473354in}{0.909273in}}%
\pgfpathlineto{\pgfqpoint{3.353878in}{0.909273in}}%
\pgfpathclose%
\pgfusepath{fill}%
\end{pgfscope}%
\begin{pgfscope}%
\pgfpathrectangle{\pgfqpoint{3.131332in}{0.488889in}}{\pgfqpoint{0.342022in}{0.875800in}}%
\pgfusepath{clip}%
\pgfsetbuttcap%
\pgfsetmiterjoin%
\definecolor{currentfill}{rgb}{0.000000,0.356863,0.821569}%
\pgfsetfillcolor{currentfill}%
\pgfsetlinewidth{0.000000pt}%
\definecolor{currentstroke}{rgb}{0.000000,0.000000,0.000000}%
\pgfsetstrokecolor{currentstroke}%
\pgfsetstrokeopacity{0.000000}%
\pgfsetdash{}{0pt}%
\pgfpathmoveto{\pgfqpoint{3.353878in}{0.830451in}}%
\pgfpathlineto{\pgfqpoint{3.473354in}{0.830451in}}%
\pgfpathlineto{\pgfqpoint{3.473354in}{0.918031in}}%
\pgfpathlineto{\pgfqpoint{3.353878in}{0.918031in}}%
\pgfpathclose%
\pgfusepath{fill}%
\end{pgfscope}%
\begin{pgfscope}%
\pgfpathrectangle{\pgfqpoint{3.131332in}{0.488889in}}{\pgfqpoint{0.342022in}{0.875800in}}%
\pgfusepath{clip}%
\pgfsetbuttcap%
\pgfsetmiterjoin%
\definecolor{currentfill}{rgb}{0.000000,0.356863,0.821569}%
\pgfsetfillcolor{currentfill}%
\pgfsetlinewidth{0.000000pt}%
\definecolor{currentstroke}{rgb}{0.000000,0.000000,0.000000}%
\pgfsetstrokecolor{currentstroke}%
\pgfsetstrokeopacity{0.000000}%
\pgfsetdash{}{0pt}%
\pgfpathmoveto{\pgfqpoint{3.353878in}{0.839209in}}%
\pgfpathlineto{\pgfqpoint{3.473354in}{0.839209in}}%
\pgfpathlineto{\pgfqpoint{3.473354in}{0.926789in}}%
\pgfpathlineto{\pgfqpoint{3.353878in}{0.926789in}}%
\pgfpathclose%
\pgfusepath{fill}%
\end{pgfscope}%
\begin{pgfscope}%
\pgfpathrectangle{\pgfqpoint{3.131332in}{0.488889in}}{\pgfqpoint{0.342022in}{0.875800in}}%
\pgfusepath{clip}%
\pgfsetbuttcap%
\pgfsetmiterjoin%
\definecolor{currentfill}{rgb}{0.000000,0.356863,0.821569}%
\pgfsetfillcolor{currentfill}%
\pgfsetlinewidth{0.000000pt}%
\definecolor{currentstroke}{rgb}{0.000000,0.000000,0.000000}%
\pgfsetstrokecolor{currentstroke}%
\pgfsetstrokeopacity{0.000000}%
\pgfsetdash{}{0pt}%
\pgfpathmoveto{\pgfqpoint{3.353878in}{0.847967in}}%
\pgfpathlineto{\pgfqpoint{3.473354in}{0.847967in}}%
\pgfpathlineto{\pgfqpoint{3.473354in}{0.935547in}}%
\pgfpathlineto{\pgfqpoint{3.353878in}{0.935547in}}%
\pgfpathclose%
\pgfusepath{fill}%
\end{pgfscope}%
\begin{pgfscope}%
\pgfpathrectangle{\pgfqpoint{3.131332in}{0.488889in}}{\pgfqpoint{0.342022in}{0.875800in}}%
\pgfusepath{clip}%
\pgfsetbuttcap%
\pgfsetmiterjoin%
\definecolor{currentfill}{rgb}{0.000000,0.352941,0.823529}%
\pgfsetfillcolor{currentfill}%
\pgfsetlinewidth{0.000000pt}%
\definecolor{currentstroke}{rgb}{0.000000,0.000000,0.000000}%
\pgfsetstrokecolor{currentstroke}%
\pgfsetstrokeopacity{0.000000}%
\pgfsetdash{}{0pt}%
\pgfpathmoveto{\pgfqpoint{3.353839in}{0.856725in}}%
\pgfpathlineto{\pgfqpoint{3.473354in}{0.856725in}}%
\pgfpathlineto{\pgfqpoint{3.473354in}{0.944305in}}%
\pgfpathlineto{\pgfqpoint{3.353839in}{0.944305in}}%
\pgfpathclose%
\pgfusepath{fill}%
\end{pgfscope}%
\begin{pgfscope}%
\pgfpathrectangle{\pgfqpoint{3.131332in}{0.488889in}}{\pgfqpoint{0.342022in}{0.875800in}}%
\pgfusepath{clip}%
\pgfsetbuttcap%
\pgfsetmiterjoin%
\definecolor{currentfill}{rgb}{0.000000,0.352941,0.823529}%
\pgfsetfillcolor{currentfill}%
\pgfsetlinewidth{0.000000pt}%
\definecolor{currentstroke}{rgb}{0.000000,0.000000,0.000000}%
\pgfsetstrokecolor{currentstroke}%
\pgfsetstrokeopacity{0.000000}%
\pgfsetdash{}{0pt}%
\pgfpathmoveto{\pgfqpoint{3.353711in}{0.865483in}}%
\pgfpathlineto{\pgfqpoint{3.473354in}{0.865483in}}%
\pgfpathlineto{\pgfqpoint{3.473354in}{0.953063in}}%
\pgfpathlineto{\pgfqpoint{3.353711in}{0.953063in}}%
\pgfpathclose%
\pgfusepath{fill}%
\end{pgfscope}%
\begin{pgfscope}%
\pgfpathrectangle{\pgfqpoint{3.131332in}{0.488889in}}{\pgfqpoint{0.342022in}{0.875800in}}%
\pgfusepath{clip}%
\pgfsetbuttcap%
\pgfsetmiterjoin%
\definecolor{currentfill}{rgb}{0.000000,0.341176,0.829412}%
\pgfsetfillcolor{currentfill}%
\pgfsetlinewidth{0.000000pt}%
\definecolor{currentstroke}{rgb}{0.000000,0.000000,0.000000}%
\pgfsetstrokecolor{currentstroke}%
\pgfsetstrokeopacity{0.000000}%
\pgfsetdash{}{0pt}%
\pgfpathmoveto{\pgfqpoint{3.353029in}{0.874241in}}%
\pgfpathlineto{\pgfqpoint{3.473354in}{0.874241in}}%
\pgfpathlineto{\pgfqpoint{3.473354in}{0.961821in}}%
\pgfpathlineto{\pgfqpoint{3.353029in}{0.961821in}}%
\pgfpathclose%
\pgfusepath{fill}%
\end{pgfscope}%
\begin{pgfscope}%
\pgfpathrectangle{\pgfqpoint{3.131332in}{0.488889in}}{\pgfqpoint{0.342022in}{0.875800in}}%
\pgfusepath{clip}%
\pgfsetbuttcap%
\pgfsetmiterjoin%
\definecolor{currentfill}{rgb}{0.000000,0.337255,0.831373}%
\pgfsetfillcolor{currentfill}%
\pgfsetlinewidth{0.000000pt}%
\definecolor{currentstroke}{rgb}{0.000000,0.000000,0.000000}%
\pgfsetstrokecolor{currentstroke}%
\pgfsetstrokeopacity{0.000000}%
\pgfsetdash{}{0pt}%
\pgfpathmoveto{\pgfqpoint{3.352639in}{0.882999in}}%
\pgfpathlineto{\pgfqpoint{3.473354in}{0.882999in}}%
\pgfpathlineto{\pgfqpoint{3.473354in}{0.970579in}}%
\pgfpathlineto{\pgfqpoint{3.352639in}{0.970579in}}%
\pgfpathclose%
\pgfusepath{fill}%
\end{pgfscope}%
\begin{pgfscope}%
\pgfpathrectangle{\pgfqpoint{3.131332in}{0.488889in}}{\pgfqpoint{0.342022in}{0.875800in}}%
\pgfusepath{clip}%
\pgfsetbuttcap%
\pgfsetmiterjoin%
\definecolor{currentfill}{rgb}{0.000000,0.333333,0.833333}%
\pgfsetfillcolor{currentfill}%
\pgfsetlinewidth{0.000000pt}%
\definecolor{currentstroke}{rgb}{0.000000,0.000000,0.000000}%
\pgfsetstrokecolor{currentstroke}%
\pgfsetstrokeopacity{0.000000}%
\pgfsetdash{}{0pt}%
\pgfpathmoveto{\pgfqpoint{3.352400in}{0.891757in}}%
\pgfpathlineto{\pgfqpoint{3.473354in}{0.891757in}}%
\pgfpathlineto{\pgfqpoint{3.473354in}{0.979337in}}%
\pgfpathlineto{\pgfqpoint{3.352400in}{0.979337in}}%
\pgfpathclose%
\pgfusepath{fill}%
\end{pgfscope}%
\begin{pgfscope}%
\pgfpathrectangle{\pgfqpoint{3.131332in}{0.488889in}}{\pgfqpoint{0.342022in}{0.875800in}}%
\pgfusepath{clip}%
\pgfsetbuttcap%
\pgfsetmiterjoin%
\definecolor{currentfill}{rgb}{0.000000,0.333333,0.833333}%
\pgfsetfillcolor{currentfill}%
\pgfsetlinewidth{0.000000pt}%
\definecolor{currentstroke}{rgb}{0.000000,0.000000,0.000000}%
\pgfsetstrokecolor{currentstroke}%
\pgfsetstrokeopacity{0.000000}%
\pgfsetdash{}{0pt}%
\pgfpathmoveto{\pgfqpoint{3.352389in}{0.900515in}}%
\pgfpathlineto{\pgfqpoint{3.473354in}{0.900515in}}%
\pgfpathlineto{\pgfqpoint{3.473354in}{0.988095in}}%
\pgfpathlineto{\pgfqpoint{3.352389in}{0.988095in}}%
\pgfpathclose%
\pgfusepath{fill}%
\end{pgfscope}%
\begin{pgfscope}%
\pgfpathrectangle{\pgfqpoint{3.131332in}{0.488889in}}{\pgfqpoint{0.342022in}{0.875800in}}%
\pgfusepath{clip}%
\pgfsetbuttcap%
\pgfsetmiterjoin%
\definecolor{currentfill}{rgb}{0.000000,0.333333,0.833333}%
\pgfsetfillcolor{currentfill}%
\pgfsetlinewidth{0.000000pt}%
\definecolor{currentstroke}{rgb}{0.000000,0.000000,0.000000}%
\pgfsetstrokecolor{currentstroke}%
\pgfsetstrokeopacity{0.000000}%
\pgfsetdash{}{0pt}%
\pgfpathmoveto{\pgfqpoint{3.352343in}{0.909273in}}%
\pgfpathlineto{\pgfqpoint{3.473354in}{0.909273in}}%
\pgfpathlineto{\pgfqpoint{3.473354in}{0.996853in}}%
\pgfpathlineto{\pgfqpoint{3.352343in}{0.996853in}}%
\pgfpathclose%
\pgfusepath{fill}%
\end{pgfscope}%
\begin{pgfscope}%
\pgfpathrectangle{\pgfqpoint{3.131332in}{0.488889in}}{\pgfqpoint{0.342022in}{0.875800in}}%
\pgfusepath{clip}%
\pgfsetbuttcap%
\pgfsetmiterjoin%
\definecolor{currentfill}{rgb}{0.000000,0.325490,0.837255}%
\pgfsetfillcolor{currentfill}%
\pgfsetlinewidth{0.000000pt}%
\definecolor{currentstroke}{rgb}{0.000000,0.000000,0.000000}%
\pgfsetstrokecolor{currentstroke}%
\pgfsetstrokeopacity{0.000000}%
\pgfsetdash{}{0pt}%
\pgfpathmoveto{\pgfqpoint{3.352005in}{0.918031in}}%
\pgfpathlineto{\pgfqpoint{3.473354in}{0.918031in}}%
\pgfpathlineto{\pgfqpoint{3.473354in}{1.005611in}}%
\pgfpathlineto{\pgfqpoint{3.352005in}{1.005611in}}%
\pgfpathclose%
\pgfusepath{fill}%
\end{pgfscope}%
\begin{pgfscope}%
\pgfpathrectangle{\pgfqpoint{3.131332in}{0.488889in}}{\pgfqpoint{0.342022in}{0.875800in}}%
\pgfusepath{clip}%
\pgfsetbuttcap%
\pgfsetmiterjoin%
\definecolor{currentfill}{rgb}{0.000000,0.317647,0.841176}%
\pgfsetfillcolor{currentfill}%
\pgfsetlinewidth{0.000000pt}%
\definecolor{currentstroke}{rgb}{0.000000,0.000000,0.000000}%
\pgfsetstrokecolor{currentstroke}%
\pgfsetstrokeopacity{0.000000}%
\pgfsetdash{}{0pt}%
\pgfpathmoveto{\pgfqpoint{3.351508in}{0.926789in}}%
\pgfpathlineto{\pgfqpoint{3.473354in}{0.926789in}}%
\pgfpathlineto{\pgfqpoint{3.473354in}{1.014369in}}%
\pgfpathlineto{\pgfqpoint{3.351508in}{1.014369in}}%
\pgfpathclose%
\pgfusepath{fill}%
\end{pgfscope}%
\begin{pgfscope}%
\pgfpathrectangle{\pgfqpoint{3.131332in}{0.488889in}}{\pgfqpoint{0.342022in}{0.875800in}}%
\pgfusepath{clip}%
\pgfsetbuttcap%
\pgfsetmiterjoin%
\definecolor{currentfill}{rgb}{0.000000,0.309804,0.845098}%
\pgfsetfillcolor{currentfill}%
\pgfsetlinewidth{0.000000pt}%
\definecolor{currentstroke}{rgb}{0.000000,0.000000,0.000000}%
\pgfsetstrokecolor{currentstroke}%
\pgfsetstrokeopacity{0.000000}%
\pgfsetdash{}{0pt}%
\pgfpathmoveto{\pgfqpoint{3.350994in}{0.935547in}}%
\pgfpathlineto{\pgfqpoint{3.473354in}{0.935547in}}%
\pgfpathlineto{\pgfqpoint{3.473354in}{1.023127in}}%
\pgfpathlineto{\pgfqpoint{3.350994in}{1.023127in}}%
\pgfpathclose%
\pgfusepath{fill}%
\end{pgfscope}%
\begin{pgfscope}%
\pgfpathrectangle{\pgfqpoint{3.131332in}{0.488889in}}{\pgfqpoint{0.342022in}{0.875800in}}%
\pgfusepath{clip}%
\pgfsetbuttcap%
\pgfsetmiterjoin%
\definecolor{currentfill}{rgb}{0.000000,0.294118,0.852941}%
\pgfsetfillcolor{currentfill}%
\pgfsetlinewidth{0.000000pt}%
\definecolor{currentstroke}{rgb}{0.000000,0.000000,0.000000}%
\pgfsetstrokecolor{currentstroke}%
\pgfsetstrokeopacity{0.000000}%
\pgfsetdash{}{0pt}%
\pgfpathmoveto{\pgfqpoint{3.350009in}{0.944305in}}%
\pgfpathlineto{\pgfqpoint{3.473354in}{0.944305in}}%
\pgfpathlineto{\pgfqpoint{3.473354in}{1.031885in}}%
\pgfpathlineto{\pgfqpoint{3.350009in}{1.031885in}}%
\pgfpathclose%
\pgfusepath{fill}%
\end{pgfscope}%
\begin{pgfscope}%
\pgfpathrectangle{\pgfqpoint{3.131332in}{0.488889in}}{\pgfqpoint{0.342022in}{0.875800in}}%
\pgfusepath{clip}%
\pgfsetbuttcap%
\pgfsetmiterjoin%
\definecolor{currentfill}{rgb}{0.000000,0.290196,0.854902}%
\pgfsetfillcolor{currentfill}%
\pgfsetlinewidth{0.000000pt}%
\definecolor{currentstroke}{rgb}{0.000000,0.000000,0.000000}%
\pgfsetstrokecolor{currentstroke}%
\pgfsetstrokeopacity{0.000000}%
\pgfsetdash{}{0pt}%
\pgfpathmoveto{\pgfqpoint{3.349695in}{0.953063in}}%
\pgfpathlineto{\pgfqpoint{3.473354in}{0.953063in}}%
\pgfpathlineto{\pgfqpoint{3.473354in}{1.040643in}}%
\pgfpathlineto{\pgfqpoint{3.349695in}{1.040643in}}%
\pgfpathclose%
\pgfusepath{fill}%
\end{pgfscope}%
\begin{pgfscope}%
\pgfpathrectangle{\pgfqpoint{3.131332in}{0.488889in}}{\pgfqpoint{0.342022in}{0.875800in}}%
\pgfusepath{clip}%
\pgfsetbuttcap%
\pgfsetmiterjoin%
\definecolor{currentfill}{rgb}{0.000000,0.290196,0.854902}%
\pgfsetfillcolor{currentfill}%
\pgfsetlinewidth{0.000000pt}%
\definecolor{currentstroke}{rgb}{0.000000,0.000000,0.000000}%
\pgfsetstrokecolor{currentstroke}%
\pgfsetstrokeopacity{0.000000}%
\pgfsetdash{}{0pt}%
\pgfpathmoveto{\pgfqpoint{3.349679in}{0.961821in}}%
\pgfpathlineto{\pgfqpoint{3.473354in}{0.961821in}}%
\pgfpathlineto{\pgfqpoint{3.473354in}{1.049401in}}%
\pgfpathlineto{\pgfqpoint{3.349679in}{1.049401in}}%
\pgfpathclose%
\pgfusepath{fill}%
\end{pgfscope}%
\begin{pgfscope}%
\pgfpathrectangle{\pgfqpoint{3.131332in}{0.488889in}}{\pgfqpoint{0.342022in}{0.875800in}}%
\pgfusepath{clip}%
\pgfsetbuttcap%
\pgfsetmiterjoin%
\definecolor{currentfill}{rgb}{0.000000,0.290196,0.854902}%
\pgfsetfillcolor{currentfill}%
\pgfsetlinewidth{0.000000pt}%
\definecolor{currentstroke}{rgb}{0.000000,0.000000,0.000000}%
\pgfsetstrokecolor{currentstroke}%
\pgfsetstrokeopacity{0.000000}%
\pgfsetdash{}{0pt}%
\pgfpathmoveto{\pgfqpoint{3.349679in}{0.970579in}}%
\pgfpathlineto{\pgfqpoint{3.473354in}{0.970579in}}%
\pgfpathlineto{\pgfqpoint{3.473354in}{1.058159in}}%
\pgfpathlineto{\pgfqpoint{3.349679in}{1.058159in}}%
\pgfpathclose%
\pgfusepath{fill}%
\end{pgfscope}%
\begin{pgfscope}%
\pgfpathrectangle{\pgfqpoint{3.131332in}{0.488889in}}{\pgfqpoint{0.342022in}{0.875800in}}%
\pgfusepath{clip}%
\pgfsetbuttcap%
\pgfsetmiterjoin%
\definecolor{currentfill}{rgb}{0.000000,0.286275,0.856863}%
\pgfsetfillcolor{currentfill}%
\pgfsetlinewidth{0.000000pt}%
\definecolor{currentstroke}{rgb}{0.000000,0.000000,0.000000}%
\pgfsetstrokecolor{currentstroke}%
\pgfsetstrokeopacity{0.000000}%
\pgfsetdash{}{0pt}%
\pgfpathmoveto{\pgfqpoint{3.349346in}{0.979337in}}%
\pgfpathlineto{\pgfqpoint{3.473354in}{0.979337in}}%
\pgfpathlineto{\pgfqpoint{3.473354in}{1.066917in}}%
\pgfpathlineto{\pgfqpoint{3.349346in}{1.066917in}}%
\pgfpathclose%
\pgfusepath{fill}%
\end{pgfscope}%
\begin{pgfscope}%
\pgfpathrectangle{\pgfqpoint{3.131332in}{0.488889in}}{\pgfqpoint{0.342022in}{0.875800in}}%
\pgfusepath{clip}%
\pgfsetbuttcap%
\pgfsetmiterjoin%
\definecolor{currentfill}{rgb}{0.000000,0.282353,0.858824}%
\pgfsetfillcolor{currentfill}%
\pgfsetlinewidth{0.000000pt}%
\definecolor{currentstroke}{rgb}{0.000000,0.000000,0.000000}%
\pgfsetstrokecolor{currentstroke}%
\pgfsetstrokeopacity{0.000000}%
\pgfsetdash{}{0pt}%
\pgfpathmoveto{\pgfqpoint{3.349215in}{0.988095in}}%
\pgfpathlineto{\pgfqpoint{3.473354in}{0.988095in}}%
\pgfpathlineto{\pgfqpoint{3.473354in}{1.075675in}}%
\pgfpathlineto{\pgfqpoint{3.349215in}{1.075675in}}%
\pgfpathclose%
\pgfusepath{fill}%
\end{pgfscope}%
\begin{pgfscope}%
\pgfpathrectangle{\pgfqpoint{3.131332in}{0.488889in}}{\pgfqpoint{0.342022in}{0.875800in}}%
\pgfusepath{clip}%
\pgfsetbuttcap%
\pgfsetmiterjoin%
\definecolor{currentfill}{rgb}{0.000000,0.278431,0.860784}%
\pgfsetfillcolor{currentfill}%
\pgfsetlinewidth{0.000000pt}%
\definecolor{currentstroke}{rgb}{0.000000,0.000000,0.000000}%
\pgfsetstrokecolor{currentstroke}%
\pgfsetstrokeopacity{0.000000}%
\pgfsetdash{}{0pt}%
\pgfpathmoveto{\pgfqpoint{3.348806in}{0.996853in}}%
\pgfpathlineto{\pgfqpoint{3.473354in}{0.996853in}}%
\pgfpathlineto{\pgfqpoint{3.473354in}{1.084433in}}%
\pgfpathlineto{\pgfqpoint{3.348806in}{1.084433in}}%
\pgfpathclose%
\pgfusepath{fill}%
\end{pgfscope}%
\begin{pgfscope}%
\pgfpathrectangle{\pgfqpoint{3.131332in}{0.488889in}}{\pgfqpoint{0.342022in}{0.875800in}}%
\pgfusepath{clip}%
\pgfsetbuttcap%
\pgfsetmiterjoin%
\definecolor{currentfill}{rgb}{0.000000,0.274510,0.862745}%
\pgfsetfillcolor{currentfill}%
\pgfsetlinewidth{0.000000pt}%
\definecolor{currentstroke}{rgb}{0.000000,0.000000,0.000000}%
\pgfsetstrokecolor{currentstroke}%
\pgfsetstrokeopacity{0.000000}%
\pgfsetdash{}{0pt}%
\pgfpathmoveto{\pgfqpoint{3.348611in}{1.005611in}}%
\pgfpathlineto{\pgfqpoint{3.473354in}{1.005611in}}%
\pgfpathlineto{\pgfqpoint{3.473354in}{1.093191in}}%
\pgfpathlineto{\pgfqpoint{3.348611in}{1.093191in}}%
\pgfpathclose%
\pgfusepath{fill}%
\end{pgfscope}%
\begin{pgfscope}%
\pgfpathrectangle{\pgfqpoint{3.131332in}{0.488889in}}{\pgfqpoint{0.342022in}{0.875800in}}%
\pgfusepath{clip}%
\pgfsetbuttcap%
\pgfsetmiterjoin%
\definecolor{currentfill}{rgb}{0.000000,0.270588,0.864706}%
\pgfsetfillcolor{currentfill}%
\pgfsetlinewidth{0.000000pt}%
\definecolor{currentstroke}{rgb}{0.000000,0.000000,0.000000}%
\pgfsetstrokecolor{currentstroke}%
\pgfsetstrokeopacity{0.000000}%
\pgfsetdash{}{0pt}%
\pgfpathmoveto{\pgfqpoint{3.348498in}{1.014369in}}%
\pgfpathlineto{\pgfqpoint{3.473354in}{1.014369in}}%
\pgfpathlineto{\pgfqpoint{3.473354in}{1.101949in}}%
\pgfpathlineto{\pgfqpoint{3.348498in}{1.101949in}}%
\pgfpathclose%
\pgfusepath{fill}%
\end{pgfscope}%
\begin{pgfscope}%
\pgfpathrectangle{\pgfqpoint{3.131332in}{0.488889in}}{\pgfqpoint{0.342022in}{0.875800in}}%
\pgfusepath{clip}%
\pgfsetbuttcap%
\pgfsetmiterjoin%
\definecolor{currentfill}{rgb}{0.000000,0.266667,0.866667}%
\pgfsetfillcolor{currentfill}%
\pgfsetlinewidth{0.000000pt}%
\definecolor{currentstroke}{rgb}{0.000000,0.000000,0.000000}%
\pgfsetstrokecolor{currentstroke}%
\pgfsetstrokeopacity{0.000000}%
\pgfsetdash{}{0pt}%
\pgfpathmoveto{\pgfqpoint{3.348185in}{1.023127in}}%
\pgfpathlineto{\pgfqpoint{3.473354in}{1.023127in}}%
\pgfpathlineto{\pgfqpoint{3.473354in}{1.110707in}}%
\pgfpathlineto{\pgfqpoint{3.348185in}{1.110707in}}%
\pgfpathclose%
\pgfusepath{fill}%
\end{pgfscope}%
\begin{pgfscope}%
\pgfpathrectangle{\pgfqpoint{3.131332in}{0.488889in}}{\pgfqpoint{0.342022in}{0.875800in}}%
\pgfusepath{clip}%
\pgfsetbuttcap%
\pgfsetmiterjoin%
\definecolor{currentfill}{rgb}{0.000000,0.254902,0.872549}%
\pgfsetfillcolor{currentfill}%
\pgfsetlinewidth{0.000000pt}%
\definecolor{currentstroke}{rgb}{0.000000,0.000000,0.000000}%
\pgfsetstrokecolor{currentstroke}%
\pgfsetstrokeopacity{0.000000}%
\pgfsetdash{}{0pt}%
\pgfpathmoveto{\pgfqpoint{3.347360in}{1.031885in}}%
\pgfpathlineto{\pgfqpoint{3.473354in}{1.031885in}}%
\pgfpathlineto{\pgfqpoint{3.473354in}{1.119465in}}%
\pgfpathlineto{\pgfqpoint{3.347360in}{1.119465in}}%
\pgfpathclose%
\pgfusepath{fill}%
\end{pgfscope}%
\begin{pgfscope}%
\pgfpathrectangle{\pgfqpoint{3.131332in}{0.488889in}}{\pgfqpoint{0.342022in}{0.875800in}}%
\pgfusepath{clip}%
\pgfsetbuttcap%
\pgfsetmiterjoin%
\definecolor{currentfill}{rgb}{0.000000,0.250980,0.874510}%
\pgfsetfillcolor{currentfill}%
\pgfsetlinewidth{0.000000pt}%
\definecolor{currentstroke}{rgb}{0.000000,0.000000,0.000000}%
\pgfsetstrokecolor{currentstroke}%
\pgfsetstrokeopacity{0.000000}%
\pgfsetdash{}{0pt}%
\pgfpathmoveto{\pgfqpoint{3.347043in}{1.040643in}}%
\pgfpathlineto{\pgfqpoint{3.473354in}{1.040643in}}%
\pgfpathlineto{\pgfqpoint{3.473354in}{1.128223in}}%
\pgfpathlineto{\pgfqpoint{3.347043in}{1.128223in}}%
\pgfpathclose%
\pgfusepath{fill}%
\end{pgfscope}%
\begin{pgfscope}%
\pgfpathrectangle{\pgfqpoint{3.131332in}{0.488889in}}{\pgfqpoint{0.342022in}{0.875800in}}%
\pgfusepath{clip}%
\pgfsetbuttcap%
\pgfsetmiterjoin%
\definecolor{currentfill}{rgb}{0.000000,0.250980,0.874510}%
\pgfsetfillcolor{currentfill}%
\pgfsetlinewidth{0.000000pt}%
\definecolor{currentstroke}{rgb}{0.000000,0.000000,0.000000}%
\pgfsetstrokecolor{currentstroke}%
\pgfsetstrokeopacity{0.000000}%
\pgfsetdash{}{0pt}%
\pgfpathmoveto{\pgfqpoint{3.347043in}{1.049401in}}%
\pgfpathlineto{\pgfqpoint{3.473354in}{1.049401in}}%
\pgfpathlineto{\pgfqpoint{3.473354in}{1.136981in}}%
\pgfpathlineto{\pgfqpoint{3.347043in}{1.136981in}}%
\pgfpathclose%
\pgfusepath{fill}%
\end{pgfscope}%
\begin{pgfscope}%
\pgfpathrectangle{\pgfqpoint{3.131332in}{0.488889in}}{\pgfqpoint{0.342022in}{0.875800in}}%
\pgfusepath{clip}%
\pgfsetbuttcap%
\pgfsetmiterjoin%
\definecolor{currentfill}{rgb}{0.000000,0.247059,0.876471}%
\pgfsetfillcolor{currentfill}%
\pgfsetlinewidth{0.000000pt}%
\definecolor{currentstroke}{rgb}{0.000000,0.000000,0.000000}%
\pgfsetstrokecolor{currentstroke}%
\pgfsetstrokeopacity{0.000000}%
\pgfsetdash{}{0pt}%
\pgfpathmoveto{\pgfqpoint{3.346799in}{1.058159in}}%
\pgfpathlineto{\pgfqpoint{3.473354in}{1.058159in}}%
\pgfpathlineto{\pgfqpoint{3.473354in}{1.145739in}}%
\pgfpathlineto{\pgfqpoint{3.346799in}{1.145739in}}%
\pgfpathclose%
\pgfusepath{fill}%
\end{pgfscope}%
\begin{pgfscope}%
\pgfpathrectangle{\pgfqpoint{3.131332in}{0.488889in}}{\pgfqpoint{0.342022in}{0.875800in}}%
\pgfusepath{clip}%
\pgfsetbuttcap%
\pgfsetmiterjoin%
\definecolor{currentfill}{rgb}{0.000000,0.239216,0.880392}%
\pgfsetfillcolor{currentfill}%
\pgfsetlinewidth{0.000000pt}%
\definecolor{currentstroke}{rgb}{0.000000,0.000000,0.000000}%
\pgfsetstrokecolor{currentstroke}%
\pgfsetstrokeopacity{0.000000}%
\pgfsetdash{}{0pt}%
\pgfpathmoveto{\pgfqpoint{3.346372in}{1.066917in}}%
\pgfpathlineto{\pgfqpoint{3.473354in}{1.066917in}}%
\pgfpathlineto{\pgfqpoint{3.473354in}{1.154497in}}%
\pgfpathlineto{\pgfqpoint{3.346372in}{1.154497in}}%
\pgfpathclose%
\pgfusepath{fill}%
\end{pgfscope}%
\begin{pgfscope}%
\pgfpathrectangle{\pgfqpoint{3.131332in}{0.488889in}}{\pgfqpoint{0.342022in}{0.875800in}}%
\pgfusepath{clip}%
\pgfsetbuttcap%
\pgfsetmiterjoin%
\definecolor{currentfill}{rgb}{0.000000,0.239216,0.880392}%
\pgfsetfillcolor{currentfill}%
\pgfsetlinewidth{0.000000pt}%
\definecolor{currentstroke}{rgb}{0.000000,0.000000,0.000000}%
\pgfsetstrokecolor{currentstroke}%
\pgfsetstrokeopacity{0.000000}%
\pgfsetdash{}{0pt}%
\pgfpathmoveto{\pgfqpoint{3.346372in}{1.075675in}}%
\pgfpathlineto{\pgfqpoint{3.473354in}{1.075675in}}%
\pgfpathlineto{\pgfqpoint{3.473354in}{1.163255in}}%
\pgfpathlineto{\pgfqpoint{3.346372in}{1.163255in}}%
\pgfpathclose%
\pgfusepath{fill}%
\end{pgfscope}%
\begin{pgfscope}%
\pgfpathrectangle{\pgfqpoint{3.131332in}{0.488889in}}{\pgfqpoint{0.342022in}{0.875800in}}%
\pgfusepath{clip}%
\pgfsetbuttcap%
\pgfsetmiterjoin%
\definecolor{currentfill}{rgb}{0.000000,0.239216,0.880392}%
\pgfsetfillcolor{currentfill}%
\pgfsetlinewidth{0.000000pt}%
\definecolor{currentstroke}{rgb}{0.000000,0.000000,0.000000}%
\pgfsetstrokecolor{currentstroke}%
\pgfsetstrokeopacity{0.000000}%
\pgfsetdash{}{0pt}%
\pgfpathmoveto{\pgfqpoint{3.346372in}{1.084433in}}%
\pgfpathlineto{\pgfqpoint{3.473354in}{1.084433in}}%
\pgfpathlineto{\pgfqpoint{3.473354in}{1.172013in}}%
\pgfpathlineto{\pgfqpoint{3.346372in}{1.172013in}}%
\pgfpathclose%
\pgfusepath{fill}%
\end{pgfscope}%
\begin{pgfscope}%
\pgfpathrectangle{\pgfqpoint{3.131332in}{0.488889in}}{\pgfqpoint{0.342022in}{0.875800in}}%
\pgfusepath{clip}%
\pgfsetbuttcap%
\pgfsetmiterjoin%
\definecolor{currentfill}{rgb}{0.000000,0.239216,0.880392}%
\pgfsetfillcolor{currentfill}%
\pgfsetlinewidth{0.000000pt}%
\definecolor{currentstroke}{rgb}{0.000000,0.000000,0.000000}%
\pgfsetstrokecolor{currentstroke}%
\pgfsetstrokeopacity{0.000000}%
\pgfsetdash{}{0pt}%
\pgfpathmoveto{\pgfqpoint{3.346372in}{1.093191in}}%
\pgfpathlineto{\pgfqpoint{3.473354in}{1.093191in}}%
\pgfpathlineto{\pgfqpoint{3.473354in}{1.180771in}}%
\pgfpathlineto{\pgfqpoint{3.346372in}{1.180771in}}%
\pgfpathclose%
\pgfusepath{fill}%
\end{pgfscope}%
\begin{pgfscope}%
\pgfpathrectangle{\pgfqpoint{3.131332in}{0.488889in}}{\pgfqpoint{0.342022in}{0.875800in}}%
\pgfusepath{clip}%
\pgfsetbuttcap%
\pgfsetmiterjoin%
\definecolor{currentfill}{rgb}{0.000000,0.239216,0.880392}%
\pgfsetfillcolor{currentfill}%
\pgfsetlinewidth{0.000000pt}%
\definecolor{currentstroke}{rgb}{0.000000,0.000000,0.000000}%
\pgfsetstrokecolor{currentstroke}%
\pgfsetstrokeopacity{0.000000}%
\pgfsetdash{}{0pt}%
\pgfpathmoveto{\pgfqpoint{3.346314in}{1.101949in}}%
\pgfpathlineto{\pgfqpoint{3.473354in}{1.101949in}}%
\pgfpathlineto{\pgfqpoint{3.473354in}{1.189529in}}%
\pgfpathlineto{\pgfqpoint{3.346314in}{1.189529in}}%
\pgfpathclose%
\pgfusepath{fill}%
\end{pgfscope}%
\begin{pgfscope}%
\pgfpathrectangle{\pgfqpoint{3.131332in}{0.488889in}}{\pgfqpoint{0.342022in}{0.875800in}}%
\pgfusepath{clip}%
\pgfsetbuttcap%
\pgfsetmiterjoin%
\definecolor{currentfill}{rgb}{0.000000,0.227451,0.886275}%
\pgfsetfillcolor{currentfill}%
\pgfsetlinewidth{0.000000pt}%
\definecolor{currentstroke}{rgb}{0.000000,0.000000,0.000000}%
\pgfsetstrokecolor{currentstroke}%
\pgfsetstrokeopacity{0.000000}%
\pgfsetdash{}{0pt}%
\pgfpathmoveto{\pgfqpoint{3.345462in}{1.110707in}}%
\pgfpathlineto{\pgfqpoint{3.473354in}{1.110707in}}%
\pgfpathlineto{\pgfqpoint{3.473354in}{1.198287in}}%
\pgfpathlineto{\pgfqpoint{3.345462in}{1.198287in}}%
\pgfpathclose%
\pgfusepath{fill}%
\end{pgfscope}%
\begin{pgfscope}%
\pgfpathrectangle{\pgfqpoint{3.131332in}{0.488889in}}{\pgfqpoint{0.342022in}{0.875800in}}%
\pgfusepath{clip}%
\pgfsetbuttcap%
\pgfsetmiterjoin%
\definecolor{currentfill}{rgb}{0.000000,0.215686,0.892157}%
\pgfsetfillcolor{currentfill}%
\pgfsetlinewidth{0.000000pt}%
\definecolor{currentstroke}{rgb}{0.000000,0.000000,0.000000}%
\pgfsetstrokecolor{currentstroke}%
\pgfsetstrokeopacity{0.000000}%
\pgfsetdash{}{0pt}%
\pgfpathmoveto{\pgfqpoint{3.344749in}{1.119465in}}%
\pgfpathlineto{\pgfqpoint{3.473354in}{1.119465in}}%
\pgfpathlineto{\pgfqpoint{3.473354in}{1.207045in}}%
\pgfpathlineto{\pgfqpoint{3.344749in}{1.207045in}}%
\pgfpathclose%
\pgfusepath{fill}%
\end{pgfscope}%
\begin{pgfscope}%
\pgfpathrectangle{\pgfqpoint{3.131332in}{0.488889in}}{\pgfqpoint{0.342022in}{0.875800in}}%
\pgfusepath{clip}%
\pgfsetbuttcap%
\pgfsetmiterjoin%
\definecolor{currentfill}{rgb}{0.000000,0.211765,0.894118}%
\pgfsetfillcolor{currentfill}%
\pgfsetlinewidth{0.000000pt}%
\definecolor{currentstroke}{rgb}{0.000000,0.000000,0.000000}%
\pgfsetstrokecolor{currentstroke}%
\pgfsetstrokeopacity{0.000000}%
\pgfsetdash{}{0pt}%
\pgfpathmoveto{\pgfqpoint{3.344519in}{1.128223in}}%
\pgfpathlineto{\pgfqpoint{3.473354in}{1.128223in}}%
\pgfpathlineto{\pgfqpoint{3.473354in}{1.215803in}}%
\pgfpathlineto{\pgfqpoint{3.344519in}{1.215803in}}%
\pgfpathclose%
\pgfusepath{fill}%
\end{pgfscope}%
\begin{pgfscope}%
\pgfpathrectangle{\pgfqpoint{3.131332in}{0.488889in}}{\pgfqpoint{0.342022in}{0.875800in}}%
\pgfusepath{clip}%
\pgfsetbuttcap%
\pgfsetmiterjoin%
\definecolor{currentfill}{rgb}{0.000000,0.192157,0.903922}%
\pgfsetfillcolor{currentfill}%
\pgfsetlinewidth{0.000000pt}%
\definecolor{currentstroke}{rgb}{0.000000,0.000000,0.000000}%
\pgfsetstrokecolor{currentstroke}%
\pgfsetstrokeopacity{0.000000}%
\pgfsetdash{}{0pt}%
\pgfpathmoveto{\pgfqpoint{3.343260in}{1.136981in}}%
\pgfpathlineto{\pgfqpoint{3.473354in}{1.136981in}}%
\pgfpathlineto{\pgfqpoint{3.473354in}{1.224561in}}%
\pgfpathlineto{\pgfqpoint{3.343260in}{1.224561in}}%
\pgfpathclose%
\pgfusepath{fill}%
\end{pgfscope}%
\begin{pgfscope}%
\pgfpathrectangle{\pgfqpoint{3.131332in}{0.488889in}}{\pgfqpoint{0.342022in}{0.875800in}}%
\pgfusepath{clip}%
\pgfsetbuttcap%
\pgfsetmiterjoin%
\definecolor{currentfill}{rgb}{0.000000,0.192157,0.903922}%
\pgfsetfillcolor{currentfill}%
\pgfsetlinewidth{0.000000pt}%
\definecolor{currentstroke}{rgb}{0.000000,0.000000,0.000000}%
\pgfsetstrokecolor{currentstroke}%
\pgfsetstrokeopacity{0.000000}%
\pgfsetdash{}{0pt}%
\pgfpathmoveto{\pgfqpoint{3.343260in}{1.145739in}}%
\pgfpathlineto{\pgfqpoint{3.473354in}{1.145739in}}%
\pgfpathlineto{\pgfqpoint{3.473354in}{1.233319in}}%
\pgfpathlineto{\pgfqpoint{3.343260in}{1.233319in}}%
\pgfpathclose%
\pgfusepath{fill}%
\end{pgfscope}%
\begin{pgfscope}%
\pgfpathrectangle{\pgfqpoint{3.131332in}{0.488889in}}{\pgfqpoint{0.342022in}{0.875800in}}%
\pgfusepath{clip}%
\pgfsetbuttcap%
\pgfsetmiterjoin%
\definecolor{currentfill}{rgb}{0.000000,0.192157,0.903922}%
\pgfsetfillcolor{currentfill}%
\pgfsetlinewidth{0.000000pt}%
\definecolor{currentstroke}{rgb}{0.000000,0.000000,0.000000}%
\pgfsetstrokecolor{currentstroke}%
\pgfsetstrokeopacity{0.000000}%
\pgfsetdash{}{0pt}%
\pgfpathmoveto{\pgfqpoint{3.343260in}{1.154497in}}%
\pgfpathlineto{\pgfqpoint{3.473354in}{1.154497in}}%
\pgfpathlineto{\pgfqpoint{3.473354in}{1.242077in}}%
\pgfpathlineto{\pgfqpoint{3.343260in}{1.242077in}}%
\pgfpathclose%
\pgfusepath{fill}%
\end{pgfscope}%
\begin{pgfscope}%
\pgfpathrectangle{\pgfqpoint{3.131332in}{0.488889in}}{\pgfqpoint{0.342022in}{0.875800in}}%
\pgfusepath{clip}%
\pgfsetbuttcap%
\pgfsetmiterjoin%
\definecolor{currentfill}{rgb}{0.000000,0.192157,0.903922}%
\pgfsetfillcolor{currentfill}%
\pgfsetlinewidth{0.000000pt}%
\definecolor{currentstroke}{rgb}{0.000000,0.000000,0.000000}%
\pgfsetstrokecolor{currentstroke}%
\pgfsetstrokeopacity{0.000000}%
\pgfsetdash{}{0pt}%
\pgfpathmoveto{\pgfqpoint{3.343221in}{1.163255in}}%
\pgfpathlineto{\pgfqpoint{3.473354in}{1.163255in}}%
\pgfpathlineto{\pgfqpoint{3.473354in}{1.250835in}}%
\pgfpathlineto{\pgfqpoint{3.343221in}{1.250835in}}%
\pgfpathclose%
\pgfusepath{fill}%
\end{pgfscope}%
\begin{pgfscope}%
\pgfpathrectangle{\pgfqpoint{3.131332in}{0.488889in}}{\pgfqpoint{0.342022in}{0.875800in}}%
\pgfusepath{clip}%
\pgfsetbuttcap%
\pgfsetmiterjoin%
\definecolor{currentfill}{rgb}{0.000000,0.192157,0.903922}%
\pgfsetfillcolor{currentfill}%
\pgfsetlinewidth{0.000000pt}%
\definecolor{currentstroke}{rgb}{0.000000,0.000000,0.000000}%
\pgfsetstrokecolor{currentstroke}%
\pgfsetstrokeopacity{0.000000}%
\pgfsetdash{}{0pt}%
\pgfpathmoveto{\pgfqpoint{3.343221in}{1.172013in}}%
\pgfpathlineto{\pgfqpoint{3.473354in}{1.172013in}}%
\pgfpathlineto{\pgfqpoint{3.473354in}{1.259593in}}%
\pgfpathlineto{\pgfqpoint{3.343221in}{1.259593in}}%
\pgfpathclose%
\pgfusepath{fill}%
\end{pgfscope}%
\begin{pgfscope}%
\pgfpathrectangle{\pgfqpoint{3.131332in}{0.488889in}}{\pgfqpoint{0.342022in}{0.875800in}}%
\pgfusepath{clip}%
\pgfsetbuttcap%
\pgfsetmiterjoin%
\definecolor{currentfill}{rgb}{0.000000,0.188235,0.905882}%
\pgfsetfillcolor{currentfill}%
\pgfsetlinewidth{0.000000pt}%
\definecolor{currentstroke}{rgb}{0.000000,0.000000,0.000000}%
\pgfsetstrokecolor{currentstroke}%
\pgfsetstrokeopacity{0.000000}%
\pgfsetdash{}{0pt}%
\pgfpathmoveto{\pgfqpoint{3.343046in}{1.180771in}}%
\pgfpathlineto{\pgfqpoint{3.473354in}{1.180771in}}%
\pgfpathlineto{\pgfqpoint{3.473354in}{1.268351in}}%
\pgfpathlineto{\pgfqpoint{3.343046in}{1.268351in}}%
\pgfpathclose%
\pgfusepath{fill}%
\end{pgfscope}%
\begin{pgfscope}%
\pgfpathrectangle{\pgfqpoint{3.131332in}{0.488889in}}{\pgfqpoint{0.342022in}{0.875800in}}%
\pgfusepath{clip}%
\pgfsetbuttcap%
\pgfsetmiterjoin%
\definecolor{currentfill}{rgb}{0.000000,0.188235,0.905882}%
\pgfsetfillcolor{currentfill}%
\pgfsetlinewidth{0.000000pt}%
\definecolor{currentstroke}{rgb}{0.000000,0.000000,0.000000}%
\pgfsetstrokecolor{currentstroke}%
\pgfsetstrokeopacity{0.000000}%
\pgfsetdash{}{0pt}%
\pgfpathmoveto{\pgfqpoint{3.342919in}{1.189529in}}%
\pgfpathlineto{\pgfqpoint{3.473354in}{1.189529in}}%
\pgfpathlineto{\pgfqpoint{3.473354in}{1.277109in}}%
\pgfpathlineto{\pgfqpoint{3.342919in}{1.277109in}}%
\pgfpathclose%
\pgfusepath{fill}%
\end{pgfscope}%
\begin{pgfscope}%
\pgfpathrectangle{\pgfqpoint{3.131332in}{0.488889in}}{\pgfqpoint{0.342022in}{0.875800in}}%
\pgfusepath{clip}%
\pgfsetbuttcap%
\pgfsetmiterjoin%
\definecolor{currentfill}{rgb}{0.000000,0.184314,0.907843}%
\pgfsetfillcolor{currentfill}%
\pgfsetlinewidth{0.000000pt}%
\definecolor{currentstroke}{rgb}{0.000000,0.000000,0.000000}%
\pgfsetstrokecolor{currentstroke}%
\pgfsetstrokeopacity{0.000000}%
\pgfsetdash{}{0pt}%
\pgfpathmoveto{\pgfqpoint{3.342664in}{1.198287in}}%
\pgfpathlineto{\pgfqpoint{3.473354in}{1.198287in}}%
\pgfpathlineto{\pgfqpoint{3.473354in}{1.285867in}}%
\pgfpathlineto{\pgfqpoint{3.342664in}{1.285867in}}%
\pgfpathclose%
\pgfusepath{fill}%
\end{pgfscope}%
\begin{pgfscope}%
\pgfpathrectangle{\pgfqpoint{3.131332in}{0.488889in}}{\pgfqpoint{0.342022in}{0.875800in}}%
\pgfusepath{clip}%
\pgfsetbuttcap%
\pgfsetmiterjoin%
\definecolor{currentfill}{rgb}{0.000000,0.172549,0.913725}%
\pgfsetfillcolor{currentfill}%
\pgfsetlinewidth{0.000000pt}%
\definecolor{currentstroke}{rgb}{0.000000,0.000000,0.000000}%
\pgfsetstrokecolor{currentstroke}%
\pgfsetstrokeopacity{0.000000}%
\pgfsetdash{}{0pt}%
\pgfpathmoveto{\pgfqpoint{3.341994in}{1.207045in}}%
\pgfpathlineto{\pgfqpoint{3.473354in}{1.207045in}}%
\pgfpathlineto{\pgfqpoint{3.473354in}{1.294625in}}%
\pgfpathlineto{\pgfqpoint{3.341994in}{1.294625in}}%
\pgfpathclose%
\pgfusepath{fill}%
\end{pgfscope}%
\begin{pgfscope}%
\pgfpathrectangle{\pgfqpoint{3.131332in}{0.488889in}}{\pgfqpoint{0.342022in}{0.875800in}}%
\pgfusepath{clip}%
\pgfsetbuttcap%
\pgfsetmiterjoin%
\definecolor{currentfill}{rgb}{0.000000,0.172549,0.913725}%
\pgfsetfillcolor{currentfill}%
\pgfsetlinewidth{0.000000pt}%
\definecolor{currentstroke}{rgb}{0.000000,0.000000,0.000000}%
\pgfsetstrokecolor{currentstroke}%
\pgfsetstrokeopacity{0.000000}%
\pgfsetdash{}{0pt}%
\pgfpathmoveto{\pgfqpoint{3.341994in}{1.215803in}}%
\pgfpathlineto{\pgfqpoint{3.473354in}{1.215803in}}%
\pgfpathlineto{\pgfqpoint{3.473354in}{1.303383in}}%
\pgfpathlineto{\pgfqpoint{3.341994in}{1.303383in}}%
\pgfpathclose%
\pgfusepath{fill}%
\end{pgfscope}%
\begin{pgfscope}%
\pgfpathrectangle{\pgfqpoint{3.131332in}{0.488889in}}{\pgfqpoint{0.342022in}{0.875800in}}%
\pgfusepath{clip}%
\pgfsetbuttcap%
\pgfsetmiterjoin%
\definecolor{currentfill}{rgb}{0.000000,0.168627,0.915686}%
\pgfsetfillcolor{currentfill}%
\pgfsetlinewidth{0.000000pt}%
\definecolor{currentstroke}{rgb}{0.000000,0.000000,0.000000}%
\pgfsetstrokecolor{currentstroke}%
\pgfsetstrokeopacity{0.000000}%
\pgfsetdash{}{0pt}%
\pgfpathmoveto{\pgfqpoint{3.341676in}{1.224561in}}%
\pgfpathlineto{\pgfqpoint{3.473354in}{1.224561in}}%
\pgfpathlineto{\pgfqpoint{3.473354in}{1.312141in}}%
\pgfpathlineto{\pgfqpoint{3.341676in}{1.312141in}}%
\pgfpathclose%
\pgfusepath{fill}%
\end{pgfscope}%
\begin{pgfscope}%
\pgfpathrectangle{\pgfqpoint{3.131332in}{0.488889in}}{\pgfqpoint{0.342022in}{0.875800in}}%
\pgfusepath{clip}%
\pgfsetbuttcap%
\pgfsetmiterjoin%
\definecolor{currentfill}{rgb}{0.000000,0.168627,0.915686}%
\pgfsetfillcolor{currentfill}%
\pgfsetlinewidth{0.000000pt}%
\definecolor{currentstroke}{rgb}{0.000000,0.000000,0.000000}%
\pgfsetstrokecolor{currentstroke}%
\pgfsetstrokeopacity{0.000000}%
\pgfsetdash{}{0pt}%
\pgfpathmoveto{\pgfqpoint{3.341676in}{1.233319in}}%
\pgfpathlineto{\pgfqpoint{3.473354in}{1.233319in}}%
\pgfpathlineto{\pgfqpoint{3.473354in}{1.320899in}}%
\pgfpathlineto{\pgfqpoint{3.341676in}{1.320899in}}%
\pgfpathclose%
\pgfusepath{fill}%
\end{pgfscope}%
\begin{pgfscope}%
\pgfpathrectangle{\pgfqpoint{3.131332in}{0.488889in}}{\pgfqpoint{0.342022in}{0.875800in}}%
\pgfusepath{clip}%
\pgfsetbuttcap%
\pgfsetmiterjoin%
\definecolor{currentfill}{rgb}{0.000000,0.160784,0.919608}%
\pgfsetfillcolor{currentfill}%
\pgfsetlinewidth{0.000000pt}%
\definecolor{currentstroke}{rgb}{0.000000,0.000000,0.000000}%
\pgfsetstrokecolor{currentstroke}%
\pgfsetstrokeopacity{0.000000}%
\pgfsetdash{}{0pt}%
\pgfpathmoveto{\pgfqpoint{3.341237in}{1.242077in}}%
\pgfpathlineto{\pgfqpoint{3.473354in}{1.242077in}}%
\pgfpathlineto{\pgfqpoint{3.473354in}{1.329657in}}%
\pgfpathlineto{\pgfqpoint{3.341237in}{1.329657in}}%
\pgfpathclose%
\pgfusepath{fill}%
\end{pgfscope}%
\begin{pgfscope}%
\pgfpathrectangle{\pgfqpoint{3.131332in}{0.488889in}}{\pgfqpoint{0.342022in}{0.875800in}}%
\pgfusepath{clip}%
\pgfsetbuttcap%
\pgfsetmiterjoin%
\definecolor{currentfill}{rgb}{0.000000,0.156863,0.921569}%
\pgfsetfillcolor{currentfill}%
\pgfsetlinewidth{0.000000pt}%
\definecolor{currentstroke}{rgb}{0.000000,0.000000,0.000000}%
\pgfsetstrokecolor{currentstroke}%
\pgfsetstrokeopacity{0.000000}%
\pgfsetdash{}{0pt}%
\pgfpathmoveto{\pgfqpoint{3.341068in}{1.250835in}}%
\pgfpathlineto{\pgfqpoint{3.473354in}{1.250835in}}%
\pgfpathlineto{\pgfqpoint{3.473354in}{1.338415in}}%
\pgfpathlineto{\pgfqpoint{3.341068in}{1.338415in}}%
\pgfpathclose%
\pgfusepath{fill}%
\end{pgfscope}%
\begin{pgfscope}%
\pgfpathrectangle{\pgfqpoint{3.131332in}{0.488889in}}{\pgfqpoint{0.342022in}{0.875800in}}%
\pgfusepath{clip}%
\pgfsetbuttcap%
\pgfsetmiterjoin%
\definecolor{currentfill}{rgb}{0.000000,0.152941,0.923529}%
\pgfsetfillcolor{currentfill}%
\pgfsetlinewidth{0.000000pt}%
\definecolor{currentstroke}{rgb}{0.000000,0.000000,0.000000}%
\pgfsetstrokecolor{currentstroke}%
\pgfsetstrokeopacity{0.000000}%
\pgfsetdash{}{0pt}%
\pgfpathmoveto{\pgfqpoint{3.340850in}{1.259593in}}%
\pgfpathlineto{\pgfqpoint{3.473354in}{1.259593in}}%
\pgfpathlineto{\pgfqpoint{3.473354in}{1.347173in}}%
\pgfpathlineto{\pgfqpoint{3.340850in}{1.347173in}}%
\pgfpathclose%
\pgfusepath{fill}%
\end{pgfscope}%
\begin{pgfscope}%
\pgfpathrectangle{\pgfqpoint{3.131332in}{0.488889in}}{\pgfqpoint{0.342022in}{0.875800in}}%
\pgfusepath{clip}%
\pgfsetbuttcap%
\pgfsetmiterjoin%
\definecolor{currentfill}{rgb}{0.000000,0.152941,0.923529}%
\pgfsetfillcolor{currentfill}%
\pgfsetlinewidth{0.000000pt}%
\definecolor{currentstroke}{rgb}{0.000000,0.000000,0.000000}%
\pgfsetstrokecolor{currentstroke}%
\pgfsetstrokeopacity{0.000000}%
\pgfsetdash{}{0pt}%
\pgfpathmoveto{\pgfqpoint{3.340850in}{1.268351in}}%
\pgfpathlineto{\pgfqpoint{3.473354in}{1.268351in}}%
\pgfpathlineto{\pgfqpoint{3.473354in}{1.355931in}}%
\pgfpathlineto{\pgfqpoint{3.340850in}{1.355931in}}%
\pgfpathclose%
\pgfusepath{fill}%
\end{pgfscope}%
\begin{pgfscope}%
\pgfpathrectangle{\pgfqpoint{3.131332in}{0.488889in}}{\pgfqpoint{0.342022in}{0.875800in}}%
\pgfusepath{clip}%
\pgfsetbuttcap%
\pgfsetmiterjoin%
\definecolor{currentfill}{rgb}{0.000000,0.149020,0.925490}%
\pgfsetfillcolor{currentfill}%
\pgfsetlinewidth{0.000000pt}%
\definecolor{currentstroke}{rgb}{0.000000,0.000000,0.000000}%
\pgfsetstrokecolor{currentstroke}%
\pgfsetstrokeopacity{0.000000}%
\pgfsetdash{}{0pt}%
\pgfpathmoveto{\pgfqpoint{3.340490in}{1.277109in}}%
\pgfpathlineto{\pgfqpoint{3.473354in}{1.277109in}}%
\pgfpathlineto{\pgfqpoint{3.473354in}{1.364689in}}%
\pgfpathlineto{\pgfqpoint{3.340490in}{1.364689in}}%
\pgfpathclose%
\pgfusepath{fill}%
\end{pgfscope}%
\begin{pgfscope}%
\pgfpathrectangle{\pgfqpoint{3.131332in}{0.488889in}}{\pgfqpoint{0.342022in}{0.875800in}}%
\pgfusepath{clip}%
\pgfsetbuttcap%
\pgfsetmiterjoin%
\definecolor{currentfill}{rgb}{0.000000,0.137255,0.931373}%
\pgfsetfillcolor{currentfill}%
\pgfsetlinewidth{0.000000pt}%
\definecolor{currentstroke}{rgb}{0.000000,0.000000,0.000000}%
\pgfsetstrokecolor{currentstroke}%
\pgfsetstrokeopacity{0.000000}%
\pgfsetdash{}{0pt}%
\pgfpathmoveto{\pgfqpoint{3.339686in}{1.285867in}}%
\pgfpathlineto{\pgfqpoint{3.473354in}{1.285867in}}%
\pgfpathlineto{\pgfqpoint{3.473354in}{1.373447in}}%
\pgfpathlineto{\pgfqpoint{3.339686in}{1.373447in}}%
\pgfpathclose%
\pgfusepath{fill}%
\end{pgfscope}%
\begin{pgfscope}%
\pgfpathrectangle{\pgfqpoint{3.131332in}{0.488889in}}{\pgfqpoint{0.342022in}{0.875800in}}%
\pgfusepath{clip}%
\pgfsetbuttcap%
\pgfsetmiterjoin%
\definecolor{currentfill}{rgb}{0.000000,0.137255,0.931373}%
\pgfsetfillcolor{currentfill}%
\pgfsetlinewidth{0.000000pt}%
\definecolor{currentstroke}{rgb}{0.000000,0.000000,0.000000}%
\pgfsetstrokecolor{currentstroke}%
\pgfsetstrokeopacity{0.000000}%
\pgfsetdash{}{0pt}%
\pgfpathmoveto{\pgfqpoint{3.339686in}{1.294625in}}%
\pgfpathlineto{\pgfqpoint{3.473354in}{1.294625in}}%
\pgfpathlineto{\pgfqpoint{3.473354in}{1.382205in}}%
\pgfpathlineto{\pgfqpoint{3.339686in}{1.382205in}}%
\pgfpathclose%
\pgfusepath{fill}%
\end{pgfscope}%
\begin{pgfscope}%
\pgfpathrectangle{\pgfqpoint{3.131332in}{0.488889in}}{\pgfqpoint{0.342022in}{0.875800in}}%
\pgfusepath{clip}%
\pgfsetbuttcap%
\pgfsetmiterjoin%
\definecolor{currentfill}{rgb}{0.000000,0.117647,0.941176}%
\pgfsetfillcolor{currentfill}%
\pgfsetlinewidth{0.000000pt}%
\definecolor{currentstroke}{rgb}{0.000000,0.000000,0.000000}%
\pgfsetstrokecolor{currentstroke}%
\pgfsetstrokeopacity{0.000000}%
\pgfsetdash{}{0pt}%
\pgfpathmoveto{\pgfqpoint{3.338502in}{1.303383in}}%
\pgfpathlineto{\pgfqpoint{3.473354in}{1.303383in}}%
\pgfpathlineto{\pgfqpoint{3.473354in}{1.390963in}}%
\pgfpathlineto{\pgfqpoint{3.338502in}{1.390963in}}%
\pgfpathclose%
\pgfusepath{fill}%
\end{pgfscope}%
\begin{pgfscope}%
\pgfpathrectangle{\pgfqpoint{3.131332in}{0.488889in}}{\pgfqpoint{0.342022in}{0.875800in}}%
\pgfusepath{clip}%
\pgfsetbuttcap%
\pgfsetmiterjoin%
\definecolor{currentfill}{rgb}{0.000000,0.086275,0.956863}%
\pgfsetfillcolor{currentfill}%
\pgfsetlinewidth{0.000000pt}%
\definecolor{currentstroke}{rgb}{0.000000,0.000000,0.000000}%
\pgfsetstrokecolor{currentstroke}%
\pgfsetstrokeopacity{0.000000}%
\pgfsetdash{}{0pt}%
\pgfpathmoveto{\pgfqpoint{3.336307in}{1.312141in}}%
\pgfpathlineto{\pgfqpoint{3.473354in}{1.312141in}}%
\pgfpathlineto{\pgfqpoint{3.473354in}{1.399721in}}%
\pgfpathlineto{\pgfqpoint{3.336307in}{1.399721in}}%
\pgfpathclose%
\pgfusepath{fill}%
\end{pgfscope}%
\begin{pgfscope}%
\pgfpathrectangle{\pgfqpoint{3.131332in}{0.488889in}}{\pgfqpoint{0.342022in}{0.875800in}}%
\pgfusepath{clip}%
\pgfsetbuttcap%
\pgfsetmiterjoin%
\definecolor{currentfill}{rgb}{0.000000,0.086275,0.956863}%
\pgfsetfillcolor{currentfill}%
\pgfsetlinewidth{0.000000pt}%
\definecolor{currentstroke}{rgb}{0.000000,0.000000,0.000000}%
\pgfsetstrokecolor{currentstroke}%
\pgfsetstrokeopacity{0.000000}%
\pgfsetdash{}{0pt}%
\pgfpathmoveto{\pgfqpoint{3.336307in}{1.320899in}}%
\pgfpathlineto{\pgfqpoint{3.473354in}{1.320899in}}%
\pgfpathlineto{\pgfqpoint{3.473354in}{1.408479in}}%
\pgfpathlineto{\pgfqpoint{3.336307in}{1.408479in}}%
\pgfpathclose%
\pgfusepath{fill}%
\end{pgfscope}%
\begin{pgfscope}%
\pgfpathrectangle{\pgfqpoint{3.131332in}{0.488889in}}{\pgfqpoint{0.342022in}{0.875800in}}%
\pgfusepath{clip}%
\pgfsetbuttcap%
\pgfsetmiterjoin%
\definecolor{currentfill}{rgb}{0.000000,0.062745,0.968627}%
\pgfsetfillcolor{currentfill}%
\pgfsetlinewidth{0.000000pt}%
\definecolor{currentstroke}{rgb}{0.000000,0.000000,0.000000}%
\pgfsetstrokecolor{currentstroke}%
\pgfsetstrokeopacity{0.000000}%
\pgfsetdash{}{0pt}%
\pgfpathmoveto{\pgfqpoint{3.334971in}{1.329657in}}%
\pgfpathlineto{\pgfqpoint{3.473354in}{1.329657in}}%
\pgfpathlineto{\pgfqpoint{3.473354in}{1.417237in}}%
\pgfpathlineto{\pgfqpoint{3.334971in}{1.417237in}}%
\pgfpathclose%
\pgfusepath{fill}%
\end{pgfscope}%
\begin{pgfscope}%
\pgfpathrectangle{\pgfqpoint{3.131332in}{0.488889in}}{\pgfqpoint{0.342022in}{0.875800in}}%
\pgfusepath{clip}%
\pgfsetbuttcap%
\pgfsetmiterjoin%
\definecolor{currentfill}{rgb}{0.000000,0.039216,0.980392}%
\pgfsetfillcolor{currentfill}%
\pgfsetlinewidth{0.000000pt}%
\definecolor{currentstroke}{rgb}{0.000000,0.000000,0.000000}%
\pgfsetstrokecolor{currentstroke}%
\pgfsetstrokeopacity{0.000000}%
\pgfsetdash{}{0pt}%
\pgfpathmoveto{\pgfqpoint{3.333238in}{1.338415in}}%
\pgfpathlineto{\pgfqpoint{3.473354in}{1.338415in}}%
\pgfpathlineto{\pgfqpoint{3.473354in}{1.425995in}}%
\pgfpathlineto{\pgfqpoint{3.333238in}{1.425995in}}%
\pgfpathclose%
\pgfusepath{fill}%
\end{pgfscope}%
\begin{pgfscope}%
\pgfpathrectangle{\pgfqpoint{3.131332in}{0.488889in}}{\pgfqpoint{0.342022in}{0.875800in}}%
\pgfusepath{clip}%
\pgfsetbuttcap%
\pgfsetmiterjoin%
\definecolor{currentfill}{rgb}{0.000000,0.000000,1.000000}%
\pgfsetfillcolor{currentfill}%
\pgfsetlinewidth{0.000000pt}%
\definecolor{currentstroke}{rgb}{0.000000,0.000000,0.000000}%
\pgfsetstrokecolor{currentstroke}%
\pgfsetstrokeopacity{0.000000}%
\pgfsetdash{}{0pt}%
\pgfpathmoveto{\pgfqpoint{3.330656in}{1.347173in}}%
\pgfpathlineto{\pgfqpoint{3.473354in}{1.347173in}}%
\pgfpathlineto{\pgfqpoint{3.473354in}{1.434753in}}%
\pgfpathlineto{\pgfqpoint{3.330656in}{1.434753in}}%
\pgfpathclose%
\pgfusepath{fill}%
\end{pgfscope}%
\begin{pgfscope}%
\pgfpathrectangle{\pgfqpoint{3.131332in}{0.488889in}}{\pgfqpoint{0.342022in}{0.875800in}}%
\pgfusepath{clip}%
\pgfsetbuttcap%
\pgfsetmiterjoin%
\definecolor{currentfill}{rgb}{0.000000,0.000000,1.000000}%
\pgfsetfillcolor{currentfill}%
\pgfsetlinewidth{0.000000pt}%
\definecolor{currentstroke}{rgb}{0.000000,0.000000,0.000000}%
\pgfsetstrokecolor{currentstroke}%
\pgfsetstrokeopacity{0.000000}%
\pgfsetdash{}{0pt}%
\pgfpathmoveto{\pgfqpoint{3.330656in}{1.355931in}}%
\pgfpathlineto{\pgfqpoint{3.473354in}{1.355931in}}%
\pgfpathlineto{\pgfqpoint{3.473354in}{1.443511in}}%
\pgfpathlineto{\pgfqpoint{3.330656in}{1.443511in}}%
\pgfpathclose%
\pgfusepath{fill}%
\end{pgfscope}%
\begin{pgfscope}%
\definecolor{textcolor}{rgb}{0.000000,0.000000,0.000000}%
\pgfsetstrokecolor{textcolor}%
\pgfsetfillcolor{textcolor}%
\pgftext[x=3.274982in,y=1.417237in,left,base]{\color{textcolor}\setmainfont{Lato}\rmfamily\fontsize{7.000000}{8.400000}\selectfont 41.7\%}%
\end{pgfscope}%
\begin{pgfscope}%
\definecolor{textcolor}{rgb}{0.000000,0.000000,0.000000}%
\pgfsetstrokecolor{textcolor}%
\pgfsetfillcolor{textcolor}%
\pgftext[x=3.274982in,y=0.366277in,left,base]{\color{textcolor}\setmainfont{Lato}\rmfamily\fontsize{7.000000}{8.400000}\selectfont 22.6\%}%
\end{pgfscope}%
\end{pgfpicture}%
\makeatother%
\endgroup%


\vspace{-4mm}
\footnotesize{Source: American Community Survey, Dorn, Author's Calculations}
\end{minipage} \hspace{3mm}
\begin{minipage}{0.24\textwidth}
\vspace{-1mm}

\footnotesize \input{text/acs_ftfy.txt}
\end{minipage}
\newpage
\begin{minipage}{0.76\textwidth} \index{poverty}
\small The Census Bureau \href{https://www.census.gov/library/publications/2021/demo/p60-275.html}{report} \textbf{the number of people taken out of poverty by various programs}, along with how many people are put in poverty by various expenses. In 2021, Social Security payments lift income above the poverty line for 26.3 million people, by far the most effective program for reducing poverty.

The third round of pandemic-related economic stimulus payments prevented 8.9 million people from being in poverty. Unemployment benefits removed 2.3 million people from poverty. Refundable tax credits, which include the expanded child tax credit, remove 9.6 million people from poverty, including 4.9 million children.

Several elements add to the number of people in poverty. Medical expenses are the most significant, and cause the disposable income of 4.6 million people to fall below the poverty line. Work expenses additionally put 2.2 million people in poverty.

\vspace{5mm}

\normalsize \textbf{Effect of Individual Elements on Poverty Headcount}\\
\footnotesize{\textit{individual element effect on number of people in poverty, millions, 2021}}\\
\vspace{0.5mm}
\hspace*{-12mm}  \begin{tikzpicture}
	\begin{axis}[clip=false,
            xbar stacked, \bbar{x}{0},
            xmin=-31, xmax=7.5, ymin=-0.5, ymax=17.5,
            stack negative=separate, ymajorgrids=false,
            yticklabels from table={\spm}{name},
			y axis line style={opacity=0},   
		    x axis line style={opacity=0}, 
			yticklabel style={black, align=left, anchor=east},            
            ytick=data, tickwidth=0pt,
            legend style={
                at={(0.0, 1.05)},
                anchor=north,legend
                columns=-1, draw=none},
            legend cell align=left, extra x ticks={0},
            extra x tick style={
                grid style={black},
                xticklabel=\empty},
            enlarge y limits={0.02}, enlarge x limits={0.02},
            width=10.4cm, height=10.0cm,
            \barylab{3.8cm}{1.5ex}]
            \addplot[fill=green!75!blue!75!black, draw opacity=0] table 
            	[y expr=\coordindex, x=kids, col sep=comma] {\spm};
            \addplot[fill=cyan!95!blue!70!white, draw opacity=0] table 
            	[y expr=\coordindex, x=adults, col sep=comma] {\spm};
            \addplot[fill=blue!80!purple, draw opacity=0] table 
            	[y expr=\coordindex, x=elderly, col sep=comma] {\spm};
            \legend{Under 18, 18 to 64, 65 or older}
        \end{axis}
        \begin{axis}[xmin=-31, xmax=7.5, ymin=-0.5, 
            ymax=17.5, enlarge y limits={0.02}, enlarge x limits={0.02}, xticklabels=\empty,
            width=10.4cm, height=10.0cm, ymajorgrids=false, yticklabels=\empty,
			y axis line style={opacity=0}, x axis line style={opacity=0}, ]
            \addplot[scatter, no marks, draw=none, nodes near coords*={\Label},
            	nodes near coords align={horizontal},
        		visualization depends on={value \thisrow{label} \as \Label}] table 		
        		[x=xloc,y=yloc,col sep=comma] {data/spmtbl21.csv};
        \end{axis}
    \end{tikzpicture}\\
\footnotesize{Source: Census Bureau Supplemental Poverty Measure} \hfill \tbllink{spmtbl21.csv}
\end{minipage}
\newpage
\printindex
\end{document}
