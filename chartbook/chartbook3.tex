% % % % % % % % % % % % % % 
%
%	U.S. Chartbook
%	Brian W. Dew (brianwdew@gmail.com)
%	Updated: December 15, 2019
%	GitHub repo contains to do list (issues)
%   https://github.com/bdecon/US-chartbook
%
% % % % % % % % % % % % % %
\PassOptionsToPackage{table}{xcolor}
\documentclass{report}

%
% % % % % % Packages % % % % % % % % % 
%
	
	\usepackage[letterpaper, margin=1.18in]{geometry}
	\usepackage{microtype}
	\usepackage[default]{lato}
	\usepackage{pgfplots, pgfplotstable}
	\usepackage[eulergreek]{sansmath}
	\usepackage{xcolor}
	\usepackage{array}
	\usepackage{fontawesome5}
	\usepackage{titlesec}
	\usepackage{imakeidx}
	\usepackage{fancyhdr}
	\usepackage[colorlinks, linkcolor=blue, filecolor=blue, 
		citecolor=blue, urlcolor=blue, linktoc=all, 
		pdfencoding=auto]{hyperref}
	\usetikzlibrary{pgfplots.dateplot, pgfplots.fillbetween, patterns,
	                pgfplots.groupplots, shapes.geometric}

%
% % % % % Document Settings % % % % % % % 
%

	% Paragraph spacing
	\usepackage{parskip}
	\setlength\parindent{0pt}
	\setlength{\parskip}{8pt}
	\makeatletter
		\newcommand{\@minipagerestore}{\setlength{\parskip}{8pt}}
	\makeatother
	
	% Section and Subsection Headings
	\titleformat{\section}
  		{\color{darkgray} \LARGE \seriffont \bfseries}
  		{\thesection}{1em}{}
	\titleformat{\subsection}
  		{\color{black!70} \seriffont \bfseries \large}
  		{\thesection}{1em}{}
	\titleformat{\subsubsection}
  		{\color{black!70} \seriffont \bfseries \normalsize}
  		{\thesection}{1em}{}		
%
% % % % % Graph Settings % % % % % % % 
%
	
	% Header and footer
	
	\pagestyle{fancy}
	\fancyhf{}
	\renewcommand{\headrulewidth}{0pt}
	\fancyhead[R]{\rightmark}
	\rfoot{\hyperlink{toc}{\faList}}
	\cfoot{\thepage}	
	
	
	% Index
	\indexsetup{level=\section*,noclearpage}
	\makeindex
	
	% Color square
	\newcommand{\cbox}[1]{
		\begin{tikzpicture} \draw [#1, line width=6](0,0) -- (.2,0);  
		\end{tikzpicture}}
	\newcommand{\colorline}[2]{
		\begin{tikzpicture} \draw [#1, line width=1.8](0,0.2) -- +(0.6,0) node[right, black!80] {#2}; 
		\end{tikzpicture}}
		
	% Table link
	\newcommand{\tbllink}[1]{\href{https://raw.githubusercontent.com/bdecon/US-chartbook/master/chartbook/data/#1}{\faTable}}
	
	% Last two digits of year
	\makeatletter
	\newcommand*\short[1]{\expandafter\@gobbletwo\number\numexpr#1\relax}
	\makeatother	
	
	% Column width and alignment
	\newcolumntype{R}[1]{>{\raggedleft\let\newline\\\arraybackslash\hspace{0pt}}m{#1}}	
	\newcolumntype{C}[1]{>{\centering\let\newline\\\arraybackslash\hspace{0pt}}m{#1}}
	
	% Style for date plots
	\pgfplotsset{compat=newest, 
		scaled y ticks=false,
		axis line style={black!20}, 
		xtick style={black!20}, ytick style={draw=none},
		every tick label/.style={black!50, font=\scriptsize,
			/pgf/number format/assume math mode=true},
		width=12.8cm, height=4.8cm, 
		xticklabel style={align=left}, 
		yticklabel style={text width=0.9em, align=right},       
		axis x line*=bottom, x axis line style={black!50},
	    axis y line=left, y axis line style={opacity=0},
	    ymajorgrids, grid style={very thin, black!10},	        
	    every node near coord/.style={/pgf/number format/fixed,
	    	font=\scriptsize, style={black!70}},
	    legend style={legend columns=-1, draw=none, fill=none,
	    	/tikz/every even column/.append style={column sep=0.3cm}}}
	    	
	
	% stacked diverging bar
	\newcommand{\sbar}[4]{
		\addplot[ybar stacked, bar width=2.4pt, draw opacity=0, fill=#1] 
			table [x=#2, y=#3, col sep=comma]{#4};}

	% stacked diverging bar area legend
	\newcommand{\sbaral}[4]{
		\addplot[ybar stacked, bar width=2.4pt, draw opacity=0, fill=#1, area legend] 
			table [x=#2, y=#3, col sep=comma]{#4};}
			
	% thin stacked diverging bar
	\newcommand{\tsbar}[4]{
		\addplot[ybar stacked, bar width=2.2pt, draw opacity=0, fill=#1] 
			table [x=#2, y=#3, col sep=comma]{#4};}
			
	% custom width stacked diverging bar
	\newcommand{\ctsbar}[5]{
		\addplot[ybar stacked, bar width=#5, draw opacity=0, fill=#1] 
			table [x=#2, y=#3, col sep=comma]{#4};}
			
	% area plot segment
	\newcommand{\abar}[4]{
		\addplot[stack plots=y, area style, draw=none, fill=#1] 
			table [x=#2, y=#3, col sep=comma]{#4}\closedcycle;}
					
	% text node
	\newcommand{\stdnode}[3]{\node[below, align=left, shift=({#1,#2})]{#3};}	
	
	% text node located by data	 
	\newcommand{\absnode}[3]{\node[below right, align=left] at (axis cs: #1,#2) {#3};}   
	
	% multiline text node located by data	 
	\newcommand{\absnodeml}[4]{\node[below right, align=left, text width=#4cm] 
		at (axis cs: #1,#2) {#3};}       
		        
	% Date (X) Axis Tick Marks, one tick per year, every even year labeled
	\newcommand{\dateaxisticks}{
		date coordinates in=x, axis line style={draw=none},
		xmax={2023-03-31},
		max space between ticks=40,	    
		xtick={{1990-01-01}, {1992-01-01}, {1994-01-01}, 
			{1996-01-01}, {1998-01-01}, {2000-01-01}, 
			{2002-01-01}, {2004-01-01}, {2006-01-01},
			{2008-01-01}, {2010-01-01}, {2012-01-01}, {2014-01-01},
		    {2016-01-01}, {2018-01-01}, {2020-01-01}, {2022-01-01}, 
		    {2024-01-01}, {2026-01-01}},
		minor xtick={{1989-01-01}, {1991-01-01}, {1993-01-01},
			{1995-01-01}, {1997-01-01}, {1999-01-01}, 
			{2001-01-01}, {2003-01-01}, {2005-01-01}, {2007-01-01},
		    {2009-01-01}, {2011-01-01}, {2013-01-01}, {2015-01-01},
		    {2017-01-01}, {2019-01-01}, {2021-01-01}, {2023-01-01}, 
		    {2025-01-01}, {2027-01-01}},
		enlarge y limits={0.06}, enlarge x limits={0.01},
		}
		
	% Date (X) Axis Tick Marks, one tick per year, every even year labeled
	\newcommand{\shdateaxisticks}{
		date coordinates in=x, axis line style={draw=none},
		xmax={2023-03-31},
		max space between ticks=40,	    
		xtick={{1990-01-01}, {1995-01-01}, {2000-01-01}, 
			{2005-01-01}, {2010-01-01}, {2015-01-01}, {2020-01-01}},
		minor xtick={},
		enlarge y limits={0.06}, enlarge x limits={0.01},
		}
		
	% Date (X) Axis Tick Marks, one tick per year, every even year labeled
	\newcommand{\ltdateaxisticks}{
		date coordinates in=x, axis line style={draw=none},
		xmax={2023-03-31},
		max space between ticks=40,	    
		xtick={{2013-01-01}, {2014-01-01}, {2015-01-01}, {2016-01-01}, {2017-01-01}, {2018-01-01}, 
		    {2019-01-01}, {2020-01-01}, {2021-01-01}, {2022-01-01}, {2023-01-01}},
		enlarge y limits={0.06}, enlarge x limits={0.01},
		}
		
	% Date (X) Axis Tick Marks, one tick per year, every even year labeled
	\newcommand{\lfdateaxisticks}{
		date coordinates in=x, axis line style={draw=none},
		xmin={2018-01-01}, xmax={2023-03-31},   
		xtick={{2018-01-01}, {2019-01-01}, {2020-01-01}, {2021-01-01}, {2022-01-01}, {2023-01-01}},
		enlarge y limits={value=0.12, upper}, enlarge x limits={0.02}, ymin=0,
		yticklabel style={text width=1.0em},
		height=3.8cm, width=6.4cm,
		}
		
	% Date (X) Axis Tick Marks, one tick per year, every even year labeled
	\newcommand{\tydateaxisticks}{
		date coordinates in=x, axis line style={draw=none},
		xmax={2023-03-31}, max space between ticks=40,	    
		xtick={{2011-01-01}, {2012-01-01}, {2013-01-01}, {2014-01-01}, {2015-01-01}, {2016-01-01}, 
			{2017-01-01}, {2018-01-01}, {2019-01-01}, {2020-01-01}, {2021-01-01}, {2022-01-01}, {2023-01-01}},
		enlarge y limits={0.06}, enlarge x limits={0.01},
		}
		
	% Date (X) Axis Tick Marks, very short term monthly ticks
	\newcommand{\shticks}{
		date coordinates in=x, axis line style={draw=none},
		xmax={2023-03-31},
		}
		
	% Settings for y label text in horizontal bar charts
	\newcommand{\barylab}[2]{yticklabel style={text width=#1, align=right, 
		style={black!70}, text height=#2},}
	
	% Solid bars at significant  x or y values
	\newcommand{\bbar}[2]{extra #1 ticks = {{#2}}, extra #1 tick labels = ,
		extra #1 tick style = {grid=major, grid style={thick, black!25}},}
		
	% Dashed line at significant  x or y values
	\newcommand{\dbar}[2]{extra #1 ticks = {{#2}}, extra #1 tick labels = ,
		extra #1 tick style = {grid=major, grid style={dashed, thick, black!50}},}
		
	% Standard line
	\newcommand{\stdline}[4]{\addplot[very thick, no markers, color=#1] 
		table [x=#2, y=#3, col sep=comma] {#4};	}
		
	% Thin line
	\newcommand{\thinline}[4]{\addplot[no markers, color=#1] 
		table [x=#2, y=#3, col sep=comma] {#4};	}
		
	% Dashed line
	\newcommand{\dashline}[4]{\addplot[very thick, dashed, no markers, color=#1] 
		table [x=#2, y=#3, col sep=comma] {#4};	}
		
	% Thicker line
	\newcommand{\thickline}[4]{\addplot[ultra thick, no markers, color=#1] 
		table [x=#2, y=#3, col sep=comma] {#4};	}
		
	% Style for bar plots legend symbol		
	\pgfplotsset{/pgfplots/area legend/.style={/pgfplots/legend image code/.code={
		\fill[##1] (0cm, -0.1cm) rectangle (0.6cm, 0.1cm);}},}		
		
	% Additional bar plot settings
	\newcommand{\barplotnogrid}{xbar=0pt, axis line style={draw=none},
	    yticklabel style={align=left, anchor=east},
      		xmajorticks=false, ymajorgrids=false,   
	    ytick=data, tickwidth=0pt, area legend, reverse legend,
	    nodes near coords align={horizontal},}  
		
	% Recession bars		
	\newcommand{\rbars}{
		\fill[color=black!10] (axis cs:{1990-07-01},\pgfkeysvalueof{/pgfplots/ymin}) rectangle 
			(axis cs:{1991-03-01}, \pgfkeysvalueof{/pgfplots/ymax});
		\fill[color=black!10] (axis cs:{2007-12-01},\pgfkeysvalueof{/pgfplots/ymin}) rectangle 
			(axis cs:{2009-07-01}, \pgfkeysvalueof{/pgfplots/ymax});
		\fill[color=black!10] (axis cs:{2001-03-01},\pgfkeysvalueof{/pgfplots/ymin}) rectangle 
			(axis cs:{2001-11-01}, \pgfkeysvalueof{/pgfplots/ymax});
		\fill[color=black!10] (axis cs:{2020-02-01},\pgfkeysvalueof{/pgfplots/ymin}) rectangle 
			(axis cs:{2020-05-01}, \pgfkeysvalueof{/pgfplots/ymax});}
			
	\newcommand{\rebars}{
		\fill[color=black!10] (axis cs:{2007-12-01},\pgfkeysvalueof{/pgfplots/ymin}) rectangle 
			(axis cs:{2009-07-01}, \pgfkeysvalueof{/pgfplots/ymax});
		\fill[color=black!10] (axis cs:{2001-03-01},\pgfkeysvalueof{/pgfplots/ymin}) rectangle 
			(axis cs:{2001-11-01}, \pgfkeysvalueof{/pgfplots/ymax});
		\fill[color=black!10] (axis cs:{2020-02-01},\pgfkeysvalueof{/pgfplots/ymin}) rectangle 
			(axis cs:{2020-05-01}, \pgfkeysvalueof{/pgfplots/ymax});}
			
	\newcommand{\recbars}{
		\fill[color=black!10] (axis cs:{2007-12-01},\pgfkeysvalueof{/pgfplots/ymin}) rectangle 
			(axis cs:{2009-07-01}, \pgfkeysvalueof{/pgfplots/ymax});
		\fill[color=black!10] (axis cs:{2020-02-01},\pgfkeysvalueof{/pgfplots/ymin}) rectangle 
			(axis cs:{2020-05-01}, \pgfkeysvalueof{/pgfplots/ymax});}
			
	\newcommand{\rbar}{
		\fill[color=black!10] (axis cs:{2020-02-01},\pgfkeysvalueof{/pgfplots/ymin}) rectangle 
			(axis cs:{2020-05-01}, \pgfkeysvalueof{/pgfplots/ymax});}
	
	\newfontfamily\seriffont{RobotoSlab}	
	
	\pgfplotstableread[header=true, col sep=comma]{data/cpi_comp.csv}\cpi
	\pgfplotstableread[header=true, col sep=semicolon]{data/ip_comp.csv}\ip
	\pgfplotstableread[header=true, col sep=comma]{data/rs_comp.csv}\rs
	\pgfplotstableread[header=true, col sep=comma]{data/ahe_ind.csv}\ahe
	\pgfplotstableread[header=true, col sep=comma]{data/poor.csv}\poor
	\pgfplotstableread[header=true, col sep=comma]{data/poor2.csv}\pvrt
	\pgfplotstableread[header=true, col sep=comma]{data/spmtbl21.csv}\spm
	\pgfplotstableread[header=true, col sep=semicolon]{data/occs.csv}\occ
	\pgfplotstableread[header=true, col sep=comma]{data/empgroups.csv}\emp
	\pgfplotstableread[header=true, col sep=comma]{data/empgroups2.csv}\empt
	\pgfplotstableread[header=true, col sep=comma]{data/unempgroups.csv}\unemp
	\pgfplotstableread[header=true, col sep=comma]{data/unempgroups2.csv}\unempt
	\pgfplotstableread[header=true, col sep=comma]{data/unempgroups3.csv}\unemptt
	\pgfplotstableread[header=true, col sep=semicolon]{data/cps_educ.csv}\edsh
	\pgfplotstableread[header=true, col sep=comma]{data/cps_educ_tot.csv}\edtot
	\pgfplotstableread[header=true, col sep=comma]{data/cps_age.csv}\agesh
	\pgfplotstableread[header=true, col sep=semicolon]{data/union_ind.csv}\unmem
	\pgfplotstableread[header=true, col sep=semicolon]{data/quits_ind.csv}\quits
	\pgfplotstableread[header=true, col sep=semicolon]{data/state_pa_epop.csv}\paepop
	\pgfplotstableread[header=true, col sep=semicolon]{data/state_pa_epop2.csv}\paepopt
	\pgfplotstableread[header=true, col sep=semicolon]{data/state_pa_epop3.csv}\paepoptt
	\pgfplotstableread[header=true, col sep=semicolon]{data/openings_ind.csv}\opens
	\pgfplotstableread[header=true, col sep=comma]{data/nilf_comp.csv}\nilf
	\pgfplotstableread[header=true, col sep=comma]{data/pinc.csv}\pinc
	\pgfplotstableread[header=true, col sep=comma]{data/unemp_grp.csv}\ungrp
	\pgfplotstableread[header=true, col sep=comma]{data/unemp_grpsh.csv}\ungrpsh
	\pgfplotstableread[header=true, col sep=comma]{data/ce_age.csv}\ceage
	\pgfplotstableread[header=true, col sep=comma]{data/ce_inc.csv}\ceinc
	\pgfplotstableread[header=true, col sep=comma]{data/cpi_monthly.csv}\cpimo
	\pgfplotstableread[header=true, col sep=comma]{data/ppi_monthly.csv}\ppimo
	\pgfplotstableread[header=true, col sep=comma]{data/ccdebtbar.csv}\ccbar
	\pgfplotstableread[header=true, col sep=comma]{data/gdp_rec.csv}\gdprec
	\pgfplotstableread[header=true, col sep=comma]{data/educ_wage_bar.csv}\web
	\pgfplotstableread[header=true, col sep=comma]{data/unemp_reason_mon.csv}\unrsn
	\pgfplotstableread[header=true, col sep=comma]{data/inf_exp_ch.csv}\rych
	\pgfplotstableread[header=true, col sep=comma]{data/inf_exp_ch2.csv}\rytch
	\pgfplotstableread[header=true, col sep=comma]{data/jobs_tercile.csv}\jobt
	\pgfplotstableread[header=true, col sep=comma]{data/jobs_ss.csv}\jobss
	\pgfplotstableread[header=true, col sep=comma]{data/emp_lt.csv}\emplt
	\pgfplotstableread[header=true, col sep=comma]{data/icsa_mon.csv}\icsamon
	\pgfplotstableread[header=true, col sep=comma]{data/ccsa_mon.csv}\ccsamon
	\pgfplotstableread[header=true, col sep=comma]{data/uwe_cps_sh.csv}\uwecps
	
	% Required for bar plots with individual bar colors for categories
	\pgfplotsset{discard if not/.style 2 args={
        x filter/.code={
            \edef\tempa{\thisrow{#1}}
            \edef\tempb{#2}
            \ifx\tempa\tempb
            \else
                \def\pgfmathresult{inf}
            \fi}}}	
	
% % % % % % % %
%
%  Begin Document
%
% % % % % % % %		
\begin{document}
\chapter*{
		\textcolor{blue!70}{\rule[-1pt]{6pt}{20pt}}
		\textcolor{green!70!blue}{\rule[-1pt]{6pt}{32pt}} \ \color{darkgray} US Chartbook}
\vspace*{-16mm}

\footnotesize \hspace{11mm} v0.0; Last updated: \today \normalsize 

\vspace{5mm}

\small \textit{Open source notes on the United States economy}

\vspace{4mm}

\thispagestyle{empty}

\begin{minipage}{0.36\textwidth}
\subsection*{ {\color{red} \faExclamationTriangle} \seriffont Warning}

\small {\color{red} \textbf{Early stage draft!}} \\
This early draft contains many errors! 
\vfill

\end{minipage} \hspace{18mm}
\begin{minipage}{0.36\textwidth}
\subsection*{{\color{gray} \faUser} Contact}

\textbf{Brian Dew} \  \\
\small {\color{gray} \faEnvelope} \ brian.w.dew@gmail.com \ \\
{\color{gray} \faTwitter} \ @bd\_econ \ \\
{\color{gray} \faGithub} \ \ \href{https://github.com/bdecon/US-chartbook}{bdecon/US-chartbook}
\end{minipage}
\vspace{5mm}

\begin{minipage}{0.76\textwidth}
\subsection*{\textcolor{blue!70}{\rule[-0.5pt]{3pt}{7.5pt}}
		\textcolor{green!70!blue}{\rule[-0.5pt]{3pt}{12pt}} \ About the Chartbook}
\vspace{2mm}

\small \textit{I like a place with a lot of items on a menu. Because you know they do them all beautifully.} \textbf{Will Ferrell}

\vspace{2mm}

This chartbook offers a big menu of US economic and social indicators. To keep the data fresh and cover a wide-variety of topics, shortcuts are made on the back end. Most of the text is generated by simple scripts. Likewise, the charts are standardized with each other in ways that reduce how well they represent a topic. 

As a result of these shortcuts, it is unlikely that you will be completely satisfied with the content of the chartbook. To sweeten the deal, I've added links to the data and \href{https://github.com/bdecon/US-chartbook}{source code}. Hopefully the end result can inspire and facilitate further exploration of topics of interest.

%Please be aware that this chartbook is an early-stage draft. Content is being added, removed, and improved. In the meantime, the current draft contains many errors and is not particularly comprehensible without lots of patience. I'm correcting the errors as I find them and gradually editing the text for clarity. 
%
\textbf{Version 0.1 release planned for Spring 2023}
\end{minipage}
\newpage
\section*{\hyperlink{toc}{\faList} \ {Contents}}
\markright{\seriffont Contents}
\hypertarget{toc}{}
\vspace{3mm}
\small

\begin{minipage}{0.4\textwidth}
\begin{description}
\item {\hyperlink{oea}{Overall Economic Activity}}
\begin{description}
\item {\hyperlink{oety}{Types of Activity}}
\item {\hyperlink{oegr}{Economic Growth}}
\item {\hyperlink{oegc}{Components of Growth}}
\end{description}
\item {\hyperlink{ofa}{Overall Financial Activity}}
\begin{description}
\item {\hyperlink{ofl}{Liabilities}}
\item {\hyperlink{ofsb}{Sectoral Balances}}
\item {\hyperlink{ofw}{Wealth}}
\item {\hyperlink{ofi}{Investment}}
\end{description}
\item {\hyperlink{hh}{Households}}
\begin{description}
\item {\hyperlink{hhdem}{Demographics}}
\item {\hyperlink{hhinc}{Income}}
\item {\hyperlink{hhss}{Spending and Saving}}
\item {\hyperlink{hhbs}{Balance Sheets}}
\item {\hyperlink{hhh}{Housing}}
\item {\hyperlink{hhpov}{Poverty}}
\end{description}
\item {\hyperlink{bus}{Businesses}}
\begin{description}
\item {\hyperlink{busin}{Investment}}
\item {\hyperlink{buspr}{Corporate Profits}}
\item {Balance Sheets}
\item {\hyperlink{busip}{Industrial Production}}
\item {\hyperlink{busrs}{Retail Sales}}
\end{description}
\item {\hyperlink{gov}{Government}}
\begin{description}
\item Spending and Investment
\item Revenue
\item {\hyperlink{govbs}{Balance Sheets}}
\end{description}
\item {\hyperlink{ext}{External Sector}}
\begin{description}
\item {\hyperlink{exbop}{Balance of Payments}}
\item {\hyperlink{extt}{Trade}}
\item {\hyperlink{exiip}{International Investment Position}}
\item {\hyperlink{excf}{Capital Flows}}
\item {\hyperlink{extfx}{Exchange Rates}}
\end{description}
\end{description}
\end{minipage} \hspace{10mm} 
\begin{minipage}{0.4\textwidth}
\begin{description}
\item {\hyperlink{lab}{Labor Markets}}
\begin{description}
\item {\hyperlink{labe}{Employment}}
\item {\hyperlink{labu}{Unemployment}}
\item {\hyperlink{labp}{Participation}}
\item {\hyperlink{labf}{Labor Force Flows}}
\item {\hyperlink{labh}{Hours}}
\item {\hyperlink{labns}{Nonstandard Work Arrangements}}
\item {\hyperlink{labw}{Wages}}
\item {\hyperlink{labprod}{Productivity}}
\item {\hyperlink{labun}{Union Membership}}
\end{description}
\item {\hyperlink{cap}{Capital Markets}}
\begin{description}
\item {\hyperlink{capeq}{Equity Markets}}
\item {\hyperlink{capint}{Interest Rates}}
\item {\hyperlink{capmm}{Money and Monetary Policy}}
\end{description}
\item {\hyperlink{pr}{Prices}}
\begin{description}
\item {\hyperlink{prin}{Consumer Price Index}}
\item {\hyperlink{prie}{Inflation Expectations}}
\item {\hyperlink{prpce}{PCE Price Index}}
\item {\hyperlink{prp}{Producer Prices}}
\item {\hyperlink{prex}{Import and Export Prices}}
\item {\hyperlink{prco}{Commodities}}
\end{description}
\item {\hyperlink{index}{Index}}
\end{description}
\vspace{4.6cm}
\end{minipage}
\newpage
\begin{minipage}{0.76\textwidth}
\subsubsection*{Jobless Claims}
\small Each week, the Department of Labor \href{https://www.dol.gov/ui/data.pdf}{present} the unemployment insurance (UI) claims reported by state unemployment offices. An initial claim for UI is filed by an unemployed person, after a separation from an employer, to determine eligibility for benefits.
\end{minipage}
\vspace{1mm}

\begin{minipage}{0.42\textwidth}
%\index{unemployment!initial claims}
\normalsize \textbf{New Jobless Claims}\\
\footnotesize{\textit{initial claims per week, thousands, seasonally adjusted}}

\hspace*{-2mm} \begin{tikzpicture}
	\begin{axis}[\shticks
	\bbar{y}{0}, height=5.2cm, width=6.65cm, yticklabel style={text width=1.6em}, 
		ymin=0, enlarge y limits={lower, 0.0}, enlarge x limits={0.01}, xtick=data,
		minor xtick={{2022-01-01}, {2023-01-01}, {2024-01-01}},
		clip=false, xticklabel style={align=center}, minor tick length=7pt, 
		xticklabels from table={\icsamon}{label}, ]
	\stdline{blue!0}{date}{icsa}{\icsamon}
	%\stdline{black!12}{date}{V12M}{data/icsa.csv}
	\stdline{orange!80!yellow}{date}{icsa}{data/icsa.csv}
	\input{text/icsa_node.txt}
	\end{axis}
\end{tikzpicture}\\
\footnotesize{Source: Department of Labor} \hfill \tbllink{icsa.csv} \hspace{1mm}
\end{minipage} \hspace{4mm} \begin{minipage}{0.3\textwidth}
\small \input{text/icsa.txt}

Initial claims are considered a leading indicator of labor market conditions. An increase in jobless claims suggests a deterioration in economic conditions. 
\end{minipage}
\vspace{1mm}

\begin{minipage}{0.76\textwidth}
\small The Labor Department additionally report continued claims for UI, also referred to as insured unemployment. Insured unemployment is the number of people receiving UI benefits during a given week. 
\end{minipage}
\vspace{0.5mm}

\begin{minipage}{0.42\textwidth}
%\index{unemployment!initial claims}
\normalsize \textbf{Insured Unemployed}\\
\footnotesize{\textit{continuing claims, thousands, seasonally adjusted}}

\hspace*{-2mm} \begin{tikzpicture}
	\begin{axis}[\shticks
	\bbar{y}{0}, height=5.4cm, width=6.5cm, yticklabel style={text width=2.2em}, 
		ymin=0, enlarge y limits={lower, 0.0}, enlarge x limits={0.01}, xtick=data,
		minor xtick={{2022-01-01}, {2023-01-01}, {2024-01-01}},
		clip=false, xticklabel style={align=center}, minor tick length=7pt, 
		xticklabels from table={\ccsamon}{label}, ]
	\stdline{blue!0}{date}{ccsa}{\ccsamon}
	%\stdline{black!12}{date}{V12M}{data/icsa.csv}
	\stdline{green!75!black}{date}{ccsa}{data/ccsa.csv}
	\input{text/ccsa_node.txt}
	\end{axis}
\end{tikzpicture}\\
\footnotesize{Source: Department of Labor} \hfill \tbllink{ccsa.csv} \hspace{1mm}
\end{minipage} \hspace{4mm} \begin{minipage}{0.3\textwidth}
\small \input{text/ccsa.txt}

\input{text/ccsa_alt.txt}
\end{minipage}
\vspace{5mm}

\begin{minipage}{0.76\textwidth}
\normalsize \textbf{Jobless Claims}\\
\footnotesize{\textit{thousands per week \hspace{52mm} period averages}}\\
\noindent \hspace*{-2mm} \rowcolors{1}{}{black!5} \setlength{\tabcolsep}{4.1pt} \color{black!90}
		{\renewcommand{\arraystretch}{1.6}
		 \begin{tabular}{p{32mm} R{10mm} R{10mm} R{10mm} R{10mm} R{8mm} R{8mm} R{8mm} }
			 \input data/jobless_claims.tex \hline
		\end{tabular}}\vspace{-2mm}
		
\footnotesize{Source: Department of Labor}
\end{minipage}
\newpage
\begin{minipage}{0.76\textwidth}
\subsubsection*{Jobless Claims}
\small Each week, the Department of Labor \href{https://www.dol.gov/ui/data.pdf}{present} the unemployment insurance (UI) claims reported by state unemployment offices. Initial claims are filed by an unemployed person, after a separation from an employer, to determine eligibility for benefits. Initial claims are considered a leading indicator of labor market conditions.

\input{text/icnsa.txt}

\input{text/ccnsa.txt}
\end{minipage}
\vspace{1mm}

\begin{minipage}{0.345\textwidth}
\normalsize \textbf{New UI Claims}\\
\footnotesize{\textit{initial claims per week, in millions,}}\\
\footnotesize{\textit{not seasonally adjusted}}\\
\hspace*{-2mm} \begin{tikzpicture}
	\begin{axis}[\bbar{y}{0}, \ltdateaxisticks ytick={0, 1, 2, 3, 4, 5, 6}, 
		enlarge y limits={0.05}, ymin=0.25, xmin={2018-01-01},
		xticklabel={`\short{\year}}, height=7.2cm, width=6.5cm]
	\rbars
	\thinline{blue!50!purple!80!black}{date}{pua_ic}{data/fed_uic.csv}
	\stdline{cyan!80!blue}{date}{VALUE}{data/icnsa.csv}
	\stdnode{3.4cm}{0.35cm}{\footnotesize \color{blue!50!purple!80!black}PUA}
	\stdnode{2.45cm}{0.7cm}{\footnotesize \color{cyan!80!blue}State}
	\end{axis}
\end{tikzpicture}\\
\footnotesize{Source: Department of Labor} \hfill \tbllink{icnsa.csv}
\end{minipage} \hspace{9mm}
\begin{minipage}{0.35\textwidth}
\normalsize \textbf{Continued UI Claims}\\
\footnotesize{\textit{insured unemployed, in millions,}}\\
\footnotesize{\textit{not seasonally adjusted}}\\
\hspace*{-2mm} \begin{tikzpicture}
	\begin{axis}[\bbar{y}{0}, \ltdateaxisticks ytick={0, 5, 10, 15, 20}, 
		enlarge y limits={0.05}, ymin=1, xmin={2018-01-01},
		yticklabel style={text width=1.0em},
		xticklabel={`\short{\year}}, height=7.2cm, width=6.5cm]
	\rbars
	\thinline{green!50!blue}{date}{fed_cc}{data/fed_uic.csv}
	\stdline{green!90!blue}{date}{VALUE}{data/ccnsa.csv}
	\stdnode{4.3cm}{4.0cm}{\footnotesize \color{green!50!blue}PUA+\\ \footnotesize \color{green!50!blue}PEUC}
	\stdnode{2.4cm}{0.85cm}{\footnotesize \color{green!90!blue}State}
	\end{axis}
\end{tikzpicture}\\
\footnotesize{Source: Department of Labor} \hfill \tbllink{ccnsa.csv}
\end{minipage}
\vspace{1mm}

\begin{minipage}{0.76\textwidth}
\small In response to the COVID-19 pandemic, traditional state-run unemployment insurance was temporarily boosted by federal programs that expanded eligibility for benefits and increased the amount of benefit payments. These programs were ended on September 6, 2021. 

%\input{text/fed_uic.txt}

\end{minipage}
\newpage
\begin{minipage}{0.76\textwidth}
\small The \textbf{Treasury yield curve} \href{https://www.treasury.gov/resource-center/data-chart-center/interest-rates/Pages/TextView.aspx?data=yield}{shows} the interest rates on different maturities of US Treasury bonds and bills, at a given point in time. The yield curve summarizes the term structure of interest rates, how much it costs to borrow for different periods of time, and has traditionally been considered an indicator of how markets view short-term economic conditions relative to longer-term conditions. 

The yield curve is normally upward sloping as investors expect to be compensated for lending for a longer period of time. The shape of the yield curve changes over time and is affected by several factors, including the term premium, the monetary policy of the Federal Reserve, and expectations about future inflation. The curve can become steeper, for example, if interest rates or inflation is expected to be higher in the future. 
\end{minipage}
\vspace{1mm}

\begin{minipage}{0.43\textwidth}
\index{treasuries!yield curve}
\normalsize \textbf{Treasury Yield Curve}\\
\footnotesize{\textit{constant maturity yield, percent}}\\
\hspace*{-2mm} \begin{tikzpicture}[trim axis right]
	\begin{axis}[height=7.2cm, width=7.7cm,	xmajorgrids, ymax=8.1, ymin=0.35,
		axis line style={draw=none}, nodes near coords style={style={black!80, fill=white,
		 yshift=1.5mm, inner sep=0.5}, /pgf/number format/.cd, fixed zerofill, 
		 precision=2, assume math mode}, legend style={at={(0.5, 0.92)}, 
		 legend columns=1},
		xtick=data, xticklabels={1M, 3M, 6M, 1Y, 2Y, 5Y, 10Y, 20Y, 30Y}, 
		legend cell align={left},
		reverse legend, xticklabel style={black!70, font=\small\bfseries}, 
		enlarge y limits={0.08}, enlarge x limits={0.02}, \bbar{y}{0}]
	\addplot[thick, mark=*, color=blue!65!black!30] 
		table [x=number, y=fiveyear, col sep=comma] {data/yc.csv};
	\addplot[thick, mark=*, color=blue!65!black!60] 
		table [x=number, y=oneyear, col sep=comma] {data/yc.csv};			
	\addplot[ultra thick, mark=*, nodes near coords, 
    		color=blue!60!black] table [x=number, y=value, col sep=comma] {data/yc.csv};
    \input{text/yc_date.txt}
    \legend{Five Years Ago, One Year Ago, Most Recent}	
	\end{axis}
\end{tikzpicture}\\
\footnotesize{Source: Federal Reserve} 
\end{minipage} \hspace{4mm}
\begin{minipage}{0.29\textwidth}
\small The yield curve can also become \textit{inverted} when yields on shorter-term debt are higher than yields on longer-term debt. An inverted yield curve can be a sign of worsening economic conditions. For example, short term rates may exceed longer-term rates if the Federal Reserve is expected to lower interest rates in the future, or if inflation is expected to fall due to weakened economic conditions. 

\input{text/yc_inversion.txt}
\end{minipage}
\vspace{3mm}

\begin{minipage}{0.76\textwidth}
\index{treasuries!yield spread}
\small Another measure of the term structure of interest rates is the \textit{spread} between treasuries with different maturities. \textbf{Treasury yield spreads} can be used to track changes in the term structure over time.

\input{text/spread_basic.txt}
\vspace{2mm}

\normalsize \textbf{Treasury Yield Spreads}\\
\footnotesize{\textit{percentage points}}\\
\hspace*{-3mm} \begin{tikzpicture}
    \begin{groupplot}[group style={group size=2 by 1, horizontal sep=52pt,}]
    \nextgroupplot[\bbar{y}{0}, \ltdateaxisticks ytick={-1, 0, 1, 2, 3}, ymin=-1.45, ymax=2.3,
    	yticklabel style={text width=1.2em}, 
     xticklabel={`\short{\year}},  height=5.6cm, width=5.9cm, clip=false]
    \rbar
	\thinline{blue!70!cyan!80!white}{date}{Ten-3M}{data/spread.csv}
	\node[text width=3.8cm, anchor=west] at (axis description cs: 0, 0.95) 
		{\small \color{blue!70!cyan!80!white}\textbf{10-Year - 3-Month}};
	\input{text/spread_node.txt}
    \nextgroupplot[\bbar{y}{0}, \ltdateaxisticks ytick={-1, 0, 1, 2, 3}, ymin=-1.45, ymax=2.3,
    	yticklabel style={text width=1.2em},
     xticklabel={`\short{\year}}, height=5.6cm, width=5.9cm, clip=false]
	\rbar
	\thinline{red!60!violet!90!white}{date}{Ten-2Y}{data/spread.csv}
	\node[text width=3.8cm, anchor=west] at (axis description cs: 0, 0.95) 
		{\small \color{red!60!violet!90!white}\textbf{10-Year - 2-Year}};
	\input{text/spread_node2.txt}
	\end{groupplot}
	\end{tikzpicture}\\
\footnotesize{Source: Federal Reserve} \hfill \tbllink{spread.csv} 
\end{minipage}
\newpage 
\hypertarget{pr}{} \markright{\seriffont Prices} \index{prices!consumer price index}
\begin{minipage}{0.76\textwidth} 
\section*{Prices}
\vspace*{-2mm}

\small The price of goods and services determine how much can be purchased by a fixed income. Researchers are interested in the prices of specific goods, as well as changes in overall purchasing power, more generally.

To understand the overall change in prices paid or charged by a group, such as consumers or manufacturers, researchers create a representative ``basket'' of the goods and services relevant to the group, and track the changes in the basket, and the price of the basket, over time. The end result of these methods is a price index. Researchers can then use the price index to calculate the rate of inflation.

Inflation is typically calculated as the 12-month percent change in the price index. This annual inflation rate measures how prices in a given month compare to prices during the same month, one year prior. 
\vspace{1mm}

\normalsize \textbf{Price Growth, Various Measures}\\
\footnotesize{\textit{one-year growth, percent}}
\vspace*{-4mm}

\hspace*{-2mm} \rowcolors{1}{}{black!5} \setlength{\tabcolsep}{3.1pt} \color{black!90}
	{\renewcommand{\arraystretch}{1.5}
		\begin{tabular}{p{30mm} R{7.6mm} R{7.6mm} R{7.6mm} R{7.6mm} R{7.6mm} R{7.6mm} 
		   R{9.2mm} R{8.5mm}}
			 \input data/prices_12m.tex \hline
		\end{tabular}}\vspace{-1mm}
		
\footnotesize{Source: BLS, BEA, Federal Reserve Bank of Dallas}
\vspace{2mm}

\small In effect, the 12-month percent change in prices is smoothed, relative to the one-month change, by including information on price changes that happened over the past year. While the chartbook uses less-volatile 12-month inflation rates in most cases, the \textbf{one-month rate} can be more useful for examining short-term trends, for example by eliminating the base effects from changes in prices a year ago. 

\input{text/cpi_monthly.txt} The Cleveland Fed \href{https://www.clevelandfed.org/indicators-and-data/inflation-nowcasting}{nowcasts} current inflation by combining recent inflation data with current oil and gasoline prices. \input{text/cpinow.txt}
\vspace{1mm}

\normalsize \textbf{CPI One-Month Change}\\
\footnotesize{\textit{percent change from previous month}}
\vspace*{-4mm}

\hspace*{-3mm} \begin{tikzpicture} 
	\begin{axis}[\bbar{y}{0}, date coordinates in=x, axis line style={draw=none},
		enlarge y limits={0.24}, enlarge x limits={0.02}, 
		width=13.0cm, height=4.7cm,
		yticklabel style={text width=1.0em}, clip=false,
		nodes near coords style={/pgf/number format/.cd, fixed zerofill,
			precision=1, assume math mode}, xtick=data,
		xticklabels from table={\cpimo}{label}]
	\addplot[thick, draw opacity=0] table [x=date, y=FILL, col sep=comma]{\cpimo};
	\addplot[ybar, fill=blue, draw opacity=0, nodes near coords] table [x=date, y=ALL_S, col sep=comma]{\cpimo};
	\draw [black!40!white, dashed] (rel axis cs:0.35,0) -- 
		(rel axis cs:0.35, 1.06);
	\node[below right, align=left] at (rel axis cs:0.35, 1.08) {\color{black!40!white}\footnotesize {latest 12 months}};
	\draw [black!40!white, dashed] (rel axis cs:0.96,0) -- 
		(rel axis cs:0.96, 1.06);
	\node[below right, align=left] at (rel axis cs:0.96, 1.08) {\color{black!40!white}\footnotesize {now-}};
	\node[below right, align=left] at (rel axis cs:0.96, 1.02) {\color{black!40!white}\footnotesize {cast}};
	\input{text/cpinow_node.txt}
	\end{axis}
\end{tikzpicture}\\
\footnotesize{Source: Bureau of Labor Statistics, Federal Reserve Bank of Cleveland} \hfill \tbllink{cpi_monthly.csv}
\end{minipage}
\newpage
\vspace{-9mm}

\begin{minipage}{0.76\textwidth} \index{usual weekly earnings} \index{wages} \index{median wage}
\subsection*{Wages} \hypertarget{labw}{}
\small Wages are an important indicator and are closely monitored by economists. Wages are the majority of income in the economy and the main cost for businesses. Wage growth is particularly closely monitored as it affects quality of life and can affect inflation rates. 

This subsection covers several wage measures. First, the distribution of usual weekly earnings provides the median, or typical, full-time wage, and insight into wage growth for low-wage workers and other groups. Next, the section discusses average hourly earnings, including by industry, and employee and benefit costs. Finally, we measure the wages of the same individuals, over time. 

\subsubsection*{Usual Weekly Earnings}
\small The Bureau of Labor Statistics (BLS) \href{https://www.bls.gov/webapps/legacy/cpswktab5.htm}{report} the \textbf{usual wages of full-time workers} at various points in the income distribution, including by decile and by quartile. The most commonly used of these measures is the median usual weekly earnings, which represents the middle wage; half of wages are above and half are below.

\input{text/uwe_median.txt}
\vspace{1mm}

\normalsize \textbf{Median Usual Weekly Earnings}\\
\footnotesize{\textit{one-year growth, percent, full-time, wage and salary earners, age 16+}}
\vspace{2.8cm}

\hspace*{4mm} \begin{tikzpicture}[overlay]
	\begin{axis}[\bbar{y}{0}, \dateaxisticks ytick={-5, 0, 5, 10, 15}, 
		enlarge y limits={0.05}, xticklabel={`\short{\year}}, ymax=10.2,
		clip=false, width=12.7cm, height=4.8cm, xmin=1989-07-01, 
		legend style={at={(0.44, 1.05)}, legend columns=1}, 
		legend cell align={left}]
	\rbars
	\thickline{cyan!60!white}{date}{p50uwe}{data/uwe_bls_gr.csv}
	\thinline{violet!80!blue}{date}{p50_3M_gr}{data/uwe_cps.csv}
    \legend{BLS (quarterly), CPS (3-month moving average)}	
	\end{axis}
\end{tikzpicture}

\footnotesize{Source: Bureau of Labor Statistics, Author's Calculations} \hfill \tbllink{uwe_bls_gr.csv} \ \ \tbllink{uwe_cps.csv}
\vspace{2mm}

\small The primary source for BLS quarterly estimates of usual weekly earnings is the \href{https://www.census.gov/data/datasets/time-series/demo/cps/cps-basic.html}{Current Population Survey} (CPS). Using the CPS, more-volatile monthly estimates can be calculated before the next BLS quarterly estimate is available. 
\end{minipage}
\vspace{1mm}

\begin{minipage}{0.35\textwidth}
\small \input{text/uwe_median_cps.txt}
\end{minipage} \hspace{6mm} \begin{minipage}{0.35\textwidth}
\normalsize \textbf{Median Usual Weekly Earnings}\\
\footnotesize{\textit{one-year growth, percent}}
\vspace*{-2mm}

\hspace*{-3mm} \begin{tikzpicture} 
	\begin{axis}[\bbar{y}{0}, date coordinates in=x, axis line style={draw=none},
		enlarge y limits={0.16}, enlarge x limits={0.01}, 
		width=6.6cm, height=5.0cm, 
		yticklabel style={text width=1.0em}, 
		minor xtick={{2023-01-01}, {2024-01-01}},
		xticklabel style={align=center}, minor tick length=7pt, 
		nodes near coords style={black, /pgf/number format/.cd, fixed zerofill,
			precision=1, assume math mode}, xtick=data,
		xticklabels from table={\uwecps}{label}]
	\addplot[thick, draw opacity=0] table [x=date, y=FILL, col sep=comma]{\uwecps};
	\addplot[ybar, bar width=50.0pt, fill=cyan!30!white, draw=cyan!90!blue] table [x=date, y=p50uwe, col sep=comma]{data/uwe_bls_sh.csv};
	\addplot[ybar, bar width=9.6pt, fill=violet!80!blue, draw opacity=0, nodes near coords] table [x=date, y=p50_gr, col sep=comma]{data/uwe_cps_shift.csv};
	\end{axis}
\end{tikzpicture}\\
\footnotesize{Source: BLS, Author} \hfill \tbllink{uwe_cps.csv}

\end{minipage}
\newpage
\vspace{-8mm}

\begin{minipage}{0.76\textwidth} \index{usual weekly earnings} \index{wages} \index{first decile wage}
\small The income distribution also tells us the earnings of low-wage workers, represented here by the first decile. Only ten percent of workers earn less than the first decile wage. \input{text/uwe_p10_basic.txt}
\vspace{1mm}

\normalsize \textbf{First Decile Usual Weekly Earnings}\\
\footnotesize{\textit{one-year growth, percent, full-time, wage and salary earners, age 16+}}
\vspace{2.6cm}

\hspace*{4mm} \begin{tikzpicture}[overlay]
	\begin{axis}[\bbar{y}{0}, \dateaxisticks ytick={-5, 0, 5, 10, 15}, 
		enlarge y limits={0.05}, xticklabel={`\short{\year}}, ymax=9.8,
		clip=false, width=12.7cm, height=4.6cm, xmin=1990-01-01, reverse legend,
		legend style={at={(0.72, 1.02)}, legend columns=-1}, 
		legend cell align={left}]
	\rbars
	\thinline{lime!65!green!90!black}{date}{p10_3M_gr}{data/uwe_cps.csv}
	\thickline{blue!65!black}{date}{p10uwe}{data/uwe_bls_gr.csv}
    \legend{CPS (3-month moving average), BLS (quarterly)}	
	\end{axis}
\end{tikzpicture}

\footnotesize{Source: Bureau of Labor Statistics, Author's Calculations} \hfill \tbllink{uwe_bls_gr.csv} \ \ \tbllink{uwe_cps.csv}
\vspace{2mm}

\small The following tables present the BLS published estimates for usual weekly earnings of full-time wage and salary earnings. The first table presents the earnings in levels, and the second table shows the one-year percent change.
\vspace{2mm}

\normalsize \textbf{Usual Weekly Earnings}\\
\footnotesize{\textit{full-time, wage and salary earners, age 16+, nominal USD}}\\
\rowcolors{1}{}{black!5} \setlength{\tabcolsep}{3.1pt} \color{black!90}
		{\renewcommand{\arraystretch}{1.54}
		 \begin{tabular}{p{19mm} R{8.8mm} R{8.8mm} R{8.8mm} R{8.8mm} R{8.8mm} R{8.8mm} 
		   R{8.8mm} R{8.8mm} R{8.8mm} R{8.8mm}}
			 \input data/wage_dist_bls2.tex \hline
		\end{tabular}}
\vspace{-3mm}
		
\footnotesize{Source: Bureau of Labor Statistics}
\vspace{3mm}

\normalsize \textbf{Weekly Earnings Growth}\\
\footnotesize{\textit{full-time, wage and salary earners, age 16+, one-year growth, percent}}\\
\rowcolors{1}{}{black!5} \setlength{\tabcolsep}{3.1pt} \color{black!90}
	{\renewcommand{\arraystretch}{1.54}
		\begin{tabular}{p{19mm} R{8.8mm} R{8.8mm} R{8.8mm} R{8.8mm} R{8.8mm} R{8.8mm} 
		   R{8.8mm} R{8.8mm} R{8.8mm} R{8.8mm}}
			 \input data/wage_dist_bls.tex \hline
		\end{tabular}}\vspace{-2mm}
				
\footnotesize{Source: Bureau of Labor Statistics}
\end{minipage}
\newpage
\vspace{-10mm}

\begin{minipage}{0.76\textwidth} \index{prices!consumer price index}
\small \input{text/cpi_monthly_rel.txt}
\end{minipage}
\vspace{1mm}

\normalsize \textbf{Selected CPI Categories, Monthly Rate}\\
\footnotesize{\textit{one-month growth, seasonally adjusted, percent}\\
\hspace*{-3mm} \rowcolors{1}{}{black!5} \setlength{\tabcolsep}{2.6pt} \color{black!90}
		{\renewcommand{\arraystretch}{1.4}
\begin{tabular}{p{38mm} R{8.2mm} R{8.2mm} R{8.2mm} R{8.2mm} R{8.2mm} R{8.2mm} 
		R{8.2mm} R{8.2mm}} % 
			 \input data/cpi_comp_mo.tex \hline
		\end{tabular}}}
\vspace{-2mm}		
		
\footnotesize{Source: Bureau of Labor Statistics; *not seasonally adjusted}
\newpage
\vspace*{-8mm}
\hypertarget{extt}{}
\begin{minipage}{0.76\textwidth}
\subsection*{International Trade}
\index{trade!overview}
\small Each month, the Census Bureau \href{https://www.census.gov/foreign-trade/Press-Release/current\_press\_release/index.html}{report} \textbf{goods and services trade} between the US and the rest of the world. US purchases of foreign goods and services are classified as imports and foreign purchases of US goods and services are exports. The trade of goods includes consumer goods, industrial equipment, and agricultural products. Services trade includes travel and tourism, business services, and charges for the use of intellectual property, among other services. 
\end{minipage}
\vspace*{-1mm}

\begin{minipage}{0.325\textwidth}
\normalsize \textbf{US Imports and Exports}\\
\footnotesize{\textit{billions of US dollars, seasonally adjusted}}\\
\hspace*{-3mm} \begin{tikzpicture}
	\begin{axis}[\ltdateaxisticks 
		enlarge y limits={0.05}, clip=false, max space between ticks=30,
		yticklabel style={text width=1.3em},
		xticklabel={`\short{\year}}, height=5.8cm, width=5.4cm]
	\rbar
	\stdline{green!80!blue}{date}{EXP}{data/tradelt.csv}
	\stdline{blue!80!violet}{date}{IMP}{data/tradelt.csv}
	\input{text/tradelt_nodes.txt}
	\absnode{{2017-01-01}}{190}{\small \color{green!80!blue}Exports}
	\absnode{{2015-02-01}}{265}{\small \color{blue!80!violet}Imports}
	\end{axis}
\end{tikzpicture}

\normalsize \textbf{Trade Balance}\footnotesize\\
\hspace*{-5mm} \begin{tikzpicture}
	\begin{axis}[\ltdateaxisticks 
		enlarge y limits={0.06}, clip=false, 
		yticklabel style={text width=2.0em},
		xticklabel={`\short{\year}}, height=4.5cm, width=5.4cm]
	\rbar
	\stdline{red}{date}{BAL}{data/tradelt.csv}
	\input{text/ballt.txt}
	\end{axis}
\end{tikzpicture}\\
\footnotesize{Source: Census Bureau} \hfill \tbllink{tradelt.csv}
\end{minipage} \hspace{5mm}
\begin{minipage}{0.39\textwidth}
\small \input{text/tradeltlevels.txt}
\end{minipage}
\vspace{3mm}

\begin{minipage}{0.76\textwidth}
\normalsize \textbf{International Trade}\\
\footnotesize{\textit{millions of US dollars, seasonally adjusted}\\
\hspace*{-3mm} \rowcolors{1}{}{black!5} \setlength{\tabcolsep}{2.6pt} \color{black!90}
		{\renewcommand{\arraystretch}{1.6}
\begin{tabular}{p{22mm} R{14mm} R{14mm} R{14mm} R{14mm} R{14mm} R{14mm}} % 
			 \input data/trade_mo_summary.tex \hline
		\end{tabular}}}
\vspace{-1mm}		
		
\footnotesize{Source: Census Bureau} \hfill \tbllink{pinc08.csv}
\end{minipage}
\newpage
\begin{minipage}{0.76\textwidth}
\subsection*{Union Membership}
\index{unions}
\hypertarget{labun}{}
\small Membership in \textbf{unions and employee associations} has diminished in the United States over the past fifty years. Unionized jobs typically offer higher wages and better benefits and union membership tends to increase wages and benefits even in nonunion jobs. Many researchers argue that lower union membership increases income inequality. 

\input{text/union.txt}
\vspace{1mm}

\normalsize \textbf{Union Membership and Coverage}\\
\footnotesize{\textit{union or employee association share of jobs, percent, one-year moving average}}\\
\hspace*{-2mm} \begin{tikzpicture}
	\begin{axis}[\bbar{y}{0}, \dateaxisticks ytick={0, 5, 10, 15, 20}, ymin=1,
		height=5.2cm, width=12.2cm, enlarge y limits={0.09},
		yticklabel style={text width=1.0em}, 
		xticklabel={`\short{\year}}, clip=false]
	\rbars
	\ctsbar{violet}{date}{Membership}{data/union.csv}{1.8pt}
	\node[text width=3.0cm, anchor=west] at 
		({1995-01-01}, 7){\color{white}\textbf{Union/Employee \\ Association Members}};
	\ctsbar{magenta!70!purple}{date}{Diff}{data/union.csv}{1.8pt}	
	\node[text width=2.2cm, anchor=west] at 
		({2010-02-01}, 15.4){\color{magenta!70!purple}\textbf{Covered Non-Members}};
	\input{text/union_nodes.txt}
	\end{axis}
\end{tikzpicture}\\
\footnotesize{Source: Author's Calculations from Current Population Survey} \hfill \tbllink{union.csv}
\end{minipage}
\vspace{3mm}

\begin{minipage}{0.43\textwidth}
\normalsize \textbf{Union Membership Rate by Industry}\\
\footnotesize{\textit{union or employee association member, percent}}\\
\hspace*{-3mm} \begin{tikzpicture}
	\begin{axis}[clip=false, stack negative=separate, ymajorgrids=false,
            yticklabels from table={\unmem}{name},
			y axis line style={opacity=0}, 
		    x axis line style={opacity=0}, 
			yticklabel style={black, font=\footnotesize, 
			align=left, anchor=east},  
			nodes near coords align={above},    
			nodes near coords style={/pgf/number format/.cd,fixed zerofill, 
				precision=1, assume math mode},  
            ytick=data, tickwidth=0pt, xmin=0,
            bar width=2.2ex, 
            extra x ticks={0}, extra x tick style={
                grid style={black}, xticklabel=\empty,},
            enlarge y limits={0.08}, enlarge x limits={0.02},
            width=5.75cm, height=8.2cm, legend style={at={(1.0, 1.07)}},
            \barylab{2.15cm}{1.5ex}]
	\addplot[xbar stacked, bar shift=0pt, fill=none, draw=none] 
		table [y expr=-\coordindex, x index=1] {\unmem};
	\addplot[xbar stacked, bar shift=0pt, fill=violet!60!blue!90, draw=none] 
		table [y expr=-\coordindex, x index=2] {\unmem};	
	\addplot[only marks, mark=square*, mark options={fill=white, 
		draw=magenta, scale=1.4}] 
		table [y expr=-\coordindex, x index=4] {\unmem};	        
	\addplot[only marks, mark=diamond*, mark options={fill=orange!75!yellow!90!white, 
		draw=magenta, scale=2.4}] 
		table [y expr=-\coordindex, x index=3] {\unmem};
	\addplot [nodes near coords={\pgfmathprintnumber{\pgfkeysvalueof{/data point/x}}}, only marks, mark=none] table[y=y, x index=3] {\unmem};
	\legend{, 30-year range,\input{text/union_yrdt.txt},\input{text/union_ltdt.txt}}%
	\end{axis}
\end{tikzpicture}\\
\footnotesize{Source: Author's Calculations from CPS} \hspace{12mm} \tbllink{union_ind.csv}
\end{minipage}\hspace{4mm}
\begin{minipage}{0.3\textwidth}
\small Union membership rates vary substantially by industry. \input{text/union_ind.txt}
\end{minipage}
\newpage
\vspace*{-6mm}
\begin{minipage}{0.76\textwidth} \index{employment!employment rate}
\subsubsection*{Employment Rates} 
\vspace{-0.5mm}
\small The \textbf{employment rate}, or the employment-population ratio, is the share of a group that is employed. Employment rates can provide useful insight into macroeconomic conditions. A high employment rate means available labor are being utilized in the productive process. All else equal, higher employment results in both increased supply, as the result of more labor being used for production, and increased demand, as the result of higher levels of income.

Economists are interested in both the overall employment rate and in the employment rates for individual groups of people. The overall employment rate provides insight into the overall utilization of labor of a society and is affected by demographic and macroeconomic factors. Employment rates for individual groups can tell us about macroeconomic conditions and even tell us about differences in local economic conditions. 

\input{text/epop_text2.txt} 
\vspace{1mm}

\normalsize \textbf{Employment Rate, Age 16 and Older}\\
\footnotesize{\textit{employed share of age 16 and older population, percent, seasonally adjusted}}\\
\hspace*{-2mm} \begin{tikzpicture}
	\begin{axis}[\dateaxisticks ytick={50, 55, 60, 65, 70}, 
		yticklabel style={text width=1.0em}, 
		width=12.2cm, enlarge y limits={0.1},
		xticklabel={`\short{\year}}, clip=false, height=4.8cm]
	\rbars
	\stdline{green!60!black}{date}{EPOP}{data/epop.csv}
	\input{text/epop_node2.txt}
	\end{axis}
\end{tikzpicture}\\
\footnotesize{Source: Bureau of Labor Statistics} \hfill \tbllink{epop.csv}
\vspace{3mm} \index{employment!prime-age}

\small Importantly, a larger share of the US population is of retirement age, reducing the overall US employment rate. To examine macroeconomic conditions separate from demographic developments, BLS \href{LNS12300060}{report} the employment rate for a more-narrow age group, specifically, those age 25 to 54. This group has the highest employment rate and are sometimes considered the ``prime'' age for labor market purposes. 

The \textbf{age 25 to 54 employment rate} is an important measure of labor market utilization. In a tight labor market, the age group is employed at a very high rate. \input{text/epop_text.txt} 
\vspace{1mm}

\normalsize \textbf{Employment Rate, Age 25 to 54}\\
\footnotesize{\textit{employed share of age 25 to 54 population, percent, seasonally adjusted}}\\
\hspace*{-2mm} \begin{tikzpicture}
	\begin{axis}[\dateaxisticks ytick={70, 75, 80}, 
		yticklabel style={text width=1.0em}, 
		width=12.2cm, enlarge y limits={0.1},
		xticklabel={`\short{\year}}, clip=false, height=4.8cm]
	\rbars
	\stdline{blue!90!cyan}{date}{PA_EPOP}{data/epop.csv}
	\input{text/epop_node.txt}
	\end{axis}
\end{tikzpicture}\\
\footnotesize{Source: Bureau of Labor Statistics} \hfill \tbllink{epop.csv}
\end{minipage}
\newpage
\vspace*{-6mm}
\index{employment!with disability}
\begin{minipage}{0.76\textwidth} 
\small Next, BLS also \href{https://data.bls.gov/timeseries/LNU02374597}{report} the \textbf{employment rate for people with disabilities}. People with disabilities may be limited in their ability to participate in labor markets and can also face discrimination during hiring. Labor market prospects for the group are also affected by economic conditions. A tight labor market pushes businesses to accommodate disabilities and to discriminate less in hiring.

In June 2008, the Current Population Survey (CPS), started to ask respondents age 16 and older whether they have difficulty with any of the following: hearing, seeing (even while wearing glasses), walking or climbing stairs, concentrating, remembering, making decisions, dressing or bathing, or running errands alone. In the latest data, covering March 2023, around XX.X million people or XX.X percent of those age 16 and older report at least one such disability. The rate of disability is relatively stable over time, and averages XX.X percent since 2008. 


\end{minipage}
\vspace{1mm}

\begin{minipage}{0.43\textwidth} 
\normalsize \textbf{Employment Rate, with Disability}\\
\footnotesize{\textit{employed share of age group, persons with disabilities,}}\\
\footnotesize{\textit{percent, not seasonally adjusted}}\\
\hspace*{-2mm} \begin{tikzpicture}
	\begin{axis}[\bbar{y}{0}, \dateaxisticks ymin=3, 
		yticklabel style={text width=1.0em}, xmin={2007-10-01},
		width=7.0cm, enlarge y limits={0.08},
		xticklabel={`\short{\year}}, clip=false, height=6.4cm]
	\recbars
	\stdline{red}{date}{bls}{data/dis_emp_rate.csv}
	\stdline{cyan}{date}{cps}{data/dis_emp_rate.csv}
	\absnode{{2012-07-01}}{16}{\footnotesize \color{red}\textbf{Age 16 and older}}
	\absnode{{2010-07-01}}{36}{\footnotesize \color{cyan}\textbf{Age 25 to 54}}
	\input{text/dis_emp_nodes.txt}
	\end{axis}
\end{tikzpicture}\\
\footnotesize{Source: Bureau of Labor Statistics; Author} \hfill \tbllink{dis_emp_rate.csv}


\end{minipage}
\newpage
\begin{minipage}{0.76\textwidth} \index{employment!employment rate}
\small The monthly jobs report describes employment at a given point in time, by asking about activities during a specific week of the previous month. To instead examine activities over a period of time, annual data on weeks worked per year and hours worked per week can be combined to identify the \textit{fully-employed}, or \textit{full-time, full-year workers}, who usually work 35 hours per week or more for 50 weeks per year or more. The Census Bureau \href{https://www.census.gov/data/tables/time-series/demo/income-poverty/cps-pinc/pinc-01.html#par_textimage_14}{report} \input{text/asec_ftfy_tot.txt}

Employment rates vary dramatically by location. \input{text/acs_ftfy_text.txt}The top ten and bottom ten commuter zones by fully-employed rate are listed below.
\end{minipage}
\vspace{1mm}

\begin{minipage}{0.55\textwidth}
\normalsize \textbf{Commuter Zone Fully-Employed Rate}\\
\footnotesize{\textit{full-time, full-year worker share of population, \input{text/acs_ftfy_year.txt}}}\\
\vspace*{-6mm}

\hspace{-11mm} %% Creator: Matplotlib, PGF backend
%%
%% To include the figure in your LaTeX document, write
%%   \input{<filename>.pgf}
%%
%% Make sure the required packages are loaded in your preamble
%%   \usepackage{pgf}
%%
%% Also ensure that all the required font packages are loaded; for instance,
%% the lmodern package is sometimes necessary when using math font.
%%   \usepackage{lmodern}
%%
%% Figures using additional raster images can only be included by \input if
%% they are in the same directory as the main LaTeX file. For loading figures
%% from other directories you can use the `import` package
%%   \usepackage{import}
%%
%% and then include the figures with
%%   \import{<path to file>}{<filename>.pgf}
%%
%% Matplotlib used the following preamble
%%   
%%   \usepackage{fontspec}
%%   \setmainfont{DejaVuSerif.ttf}[Path=\detokenize{/home/brian/miniconda3/lib/python3.8/site-packages/matplotlib/mpl-data/fonts/ttf/}]
%%   \setsansfont{DejaVuSans.ttf}[Path=\detokenize{/home/brian/miniconda3/lib/python3.8/site-packages/matplotlib/mpl-data/fonts/ttf/}]
%%   \setmonofont{DejaVuSansMono.ttf}[Path=\detokenize{/home/brian/miniconda3/lib/python3.8/site-packages/matplotlib/mpl-data/fonts/ttf/}]
%%   \makeatletter\@ifpackageloaded{underscore}{}{\usepackage[strings]{underscore}}\makeatother
%%
\begingroup%
\makeatletter%
\begin{pgfpicture}%
\pgfpathrectangle{\pgfpointorigin}{\pgfqpoint{3.841155in}{2.510000in}}%
\pgfusepath{use as bounding box, clip}%
\begin{pgfscope}%
\pgfsetbuttcap%
\pgfsetmiterjoin%
\pgfsetlinewidth{0.000000pt}%
\definecolor{currentstroke}{rgb}{0.000000,0.000000,0.000000}%
\pgfsetstrokecolor{currentstroke}%
\pgfsetstrokeopacity{0.000000}%
\pgfsetdash{}{0pt}%
\pgfpathmoveto{\pgfqpoint{0.000000in}{0.000000in}}%
\pgfpathlineto{\pgfqpoint{3.841155in}{0.000000in}}%
\pgfpathlineto{\pgfqpoint{3.841155in}{2.510000in}}%
\pgfpathlineto{\pgfqpoint{0.000000in}{2.510000in}}%
\pgfpathlineto{\pgfqpoint{0.000000in}{0.000000in}}%
\pgfpathclose%
\pgfusepath{}%
\end{pgfscope}%
\begin{pgfscope}%
\pgfpathrectangle{\pgfqpoint{0.100000in}{0.100000in}}{\pgfqpoint{3.608454in}{2.310000in}}%
\pgfusepath{clip}%
\pgfsetbuttcap%
\pgfsetroundjoin%
\pgfsetlinewidth{0.010037pt}%
\definecolor{currentstroke}{rgb}{1.000000,1.000000,1.000000}%
\pgfsetstrokecolor{currentstroke}%
\pgfsetdash{}{0pt}%
\pgfusepath{stroke}%
\end{pgfscope}%
\begin{pgfscope}%
\pgfpathrectangle{\pgfqpoint{0.100000in}{0.100000in}}{\pgfqpoint{3.608454in}{2.310000in}}%
\pgfusepath{clip}%
\pgfsetbuttcap%
\pgfsetroundjoin%
\pgfsetlinewidth{0.010037pt}%
\definecolor{currentstroke}{rgb}{1.000000,1.000000,1.000000}%
\pgfsetstrokecolor{currentstroke}%
\pgfsetdash{}{0pt}%
\pgfusepath{stroke}%
\end{pgfscope}%
\begin{pgfscope}%
\pgfpathrectangle{\pgfqpoint{0.100000in}{0.100000in}}{\pgfqpoint{3.608454in}{2.310000in}}%
\pgfusepath{clip}%
\pgfsetbuttcap%
\pgfsetroundjoin%
\pgfsetlinewidth{0.010037pt}%
\definecolor{currentstroke}{rgb}{1.000000,1.000000,1.000000}%
\pgfsetstrokecolor{currentstroke}%
\pgfsetdash{}{0pt}%
\pgfusepath{stroke}%
\end{pgfscope}%
\begin{pgfscope}%
\pgfpathrectangle{\pgfqpoint{0.100000in}{0.100000in}}{\pgfqpoint{3.608454in}{2.310000in}}%
\pgfusepath{clip}%
\pgfsetbuttcap%
\pgfsetroundjoin%
\pgfsetlinewidth{0.010037pt}%
\definecolor{currentstroke}{rgb}{1.000000,1.000000,1.000000}%
\pgfsetstrokecolor{currentstroke}%
\pgfsetdash{}{0pt}%
\pgfusepath{stroke}%
\end{pgfscope}%
\begin{pgfscope}%
\pgfpathrectangle{\pgfqpoint{0.100000in}{0.100000in}}{\pgfqpoint{3.608454in}{2.310000in}}%
\pgfusepath{clip}%
\pgfsetbuttcap%
\pgfsetroundjoin%
\pgfsetlinewidth{0.010037pt}%
\definecolor{currentstroke}{rgb}{1.000000,1.000000,1.000000}%
\pgfsetstrokecolor{currentstroke}%
\pgfsetdash{}{0pt}%
\pgfusepath{stroke}%
\end{pgfscope}%
\begin{pgfscope}%
\pgfpathrectangle{\pgfqpoint{0.100000in}{0.100000in}}{\pgfqpoint{3.608454in}{2.310000in}}%
\pgfusepath{clip}%
\pgfsetbuttcap%
\pgfsetroundjoin%
\pgfsetlinewidth{0.010037pt}%
\definecolor{currentstroke}{rgb}{1.000000,1.000000,1.000000}%
\pgfsetstrokecolor{currentstroke}%
\pgfsetdash{}{0pt}%
\pgfusepath{stroke}%
\end{pgfscope}%
\begin{pgfscope}%
\pgfpathrectangle{\pgfqpoint{0.100000in}{0.100000in}}{\pgfqpoint{3.608454in}{2.310000in}}%
\pgfusepath{clip}%
\pgfsetbuttcap%
\pgfsetroundjoin%
\pgfsetlinewidth{0.010037pt}%
\definecolor{currentstroke}{rgb}{1.000000,1.000000,1.000000}%
\pgfsetstrokecolor{currentstroke}%
\pgfsetdash{}{0pt}%
\pgfpathmoveto{\pgfqpoint{0.957404in}{2.274144in}}%
\pgfpathlineto{\pgfqpoint{0.938691in}{2.199997in}}%
\pgfpathlineto{\pgfqpoint{0.925385in}{2.146869in}}%
\pgfpathlineto{\pgfqpoint{0.909565in}{2.082958in}}%
\pgfpathlineto{\pgfqpoint{0.906458in}{2.067206in}}%
\pgfpathlineto{\pgfqpoint{0.908487in}{2.052757in}}%
\pgfpathlineto{\pgfqpoint{0.905799in}{2.039449in}}%
\pgfpathlineto{\pgfqpoint{0.795364in}{2.068266in}}%
\pgfpathlineto{\pgfqpoint{0.785388in}{2.064919in}}%
\pgfpathlineto{\pgfqpoint{0.776851in}{2.067824in}}%
\pgfpathlineto{\pgfqpoint{0.737324in}{2.068774in}}%
\pgfpathlineto{\pgfqpoint{0.723956in}{2.065005in}}%
\pgfpathlineto{\pgfqpoint{0.710680in}{2.066278in}}%
\pgfpathlineto{\pgfqpoint{0.705053in}{2.072177in}}%
\pgfpathlineto{\pgfqpoint{0.669666in}{2.070891in}}%
\pgfpathlineto{\pgfqpoint{0.664077in}{2.080532in}}%
\pgfpathlineto{\pgfqpoint{0.653468in}{2.085611in}}%
\pgfpathlineto{\pgfqpoint{0.638032in}{2.088706in}}%
\pgfpathlineto{\pgfqpoint{0.611388in}{2.084174in}}%
\pgfpathlineto{\pgfqpoint{0.601553in}{2.088614in}}%
\pgfpathlineto{\pgfqpoint{0.586388in}{2.100423in}}%
\pgfpathlineto{\pgfqpoint{0.590946in}{2.123567in}}%
\pgfpathlineto{\pgfqpoint{0.589348in}{2.135386in}}%
\pgfpathlineto{\pgfqpoint{0.577330in}{2.147934in}}%
\pgfpathlineto{\pgfqpoint{0.569631in}{2.147155in}}%
\pgfpathlineto{\pgfqpoint{0.564109in}{2.159848in}}%
\pgfpathlineto{\pgfqpoint{0.551027in}{2.164978in}}%
\pgfpathlineto{\pgfqpoint{0.541518in}{2.164324in}}%
\pgfpathlineto{\pgfqpoint{0.538442in}{2.177359in}}%
\pgfpathlineto{\pgfqpoint{0.547915in}{2.175972in}}%
\pgfpathlineto{\pgfqpoint{0.547162in}{2.194090in}}%
\pgfpathlineto{\pgfqpoint{0.543033in}{2.204847in}}%
\pgfpathlineto{\pgfqpoint{0.544853in}{2.218617in}}%
\pgfpathlineto{\pgfqpoint{0.549810in}{2.228973in}}%
\pgfpathlineto{\pgfqpoint{0.549530in}{2.248625in}}%
\pgfpathlineto{\pgfqpoint{0.546884in}{2.255755in}}%
\pgfpathlineto{\pgfqpoint{0.551454in}{2.278631in}}%
\pgfpathlineto{\pgfqpoint{0.550135in}{2.293374in}}%
\pgfpathlineto{\pgfqpoint{0.545566in}{2.300443in}}%
\pgfpathlineto{\pgfqpoint{0.546263in}{2.323556in}}%
\pgfpathlineto{\pgfqpoint{0.552909in}{2.340554in}}%
\pgfpathlineto{\pgfqpoint{0.560136in}{2.336394in}}%
\pgfpathlineto{\pgfqpoint{0.584067in}{2.311673in}}%
\pgfpathlineto{\pgfqpoint{0.613066in}{2.298145in}}%
\pgfpathlineto{\pgfqpoint{0.627914in}{2.296544in}}%
\pgfpathlineto{\pgfqpoint{0.641902in}{2.290785in}}%
\pgfpathlineto{\pgfqpoint{0.645873in}{2.271403in}}%
\pgfpathlineto{\pgfqpoint{0.633614in}{2.267732in}}%
\pgfpathlineto{\pgfqpoint{0.625490in}{2.252457in}}%
\pgfpathlineto{\pgfqpoint{0.635028in}{2.253306in}}%
\pgfpathlineto{\pgfqpoint{0.638875in}{2.260162in}}%
\pgfpathlineto{\pgfqpoint{0.652326in}{2.268736in}}%
\pgfpathlineto{\pgfqpoint{0.651631in}{2.256004in}}%
\pgfpathlineto{\pgfqpoint{0.642650in}{2.253951in}}%
\pgfpathlineto{\pgfqpoint{0.644141in}{2.237586in}}%
\pgfpathlineto{\pgfqpoint{0.633204in}{2.221193in}}%
\pgfpathlineto{\pgfqpoint{0.626204in}{2.228988in}}%
\pgfpathlineto{\pgfqpoint{0.605402in}{2.224750in}}%
\pgfpathlineto{\pgfqpoint{0.604308in}{2.215213in}}%
\pgfpathlineto{\pgfqpoint{0.611479in}{2.209403in}}%
\pgfpathlineto{\pgfqpoint{0.622419in}{2.208832in}}%
\pgfpathlineto{\pgfqpoint{0.637546in}{2.221247in}}%
\pgfpathlineto{\pgfqpoint{0.649519in}{2.222278in}}%
\pgfpathlineto{\pgfqpoint{0.649978in}{2.236021in}}%
\pgfpathlineto{\pgfqpoint{0.656149in}{2.256205in}}%
\pgfpathlineto{\pgfqpoint{0.670772in}{2.271217in}}%
\pgfpathlineto{\pgfqpoint{0.665896in}{2.282844in}}%
\pgfpathlineto{\pgfqpoint{0.669227in}{2.295363in}}%
\pgfpathlineto{\pgfqpoint{0.662739in}{2.307839in}}%
\pgfpathlineto{\pgfqpoint{0.671985in}{2.323793in}}%
\pgfpathlineto{\pgfqpoint{0.673370in}{2.333392in}}%
\pgfpathlineto{\pgfqpoint{0.665342in}{2.339700in}}%
\pgfpathlineto{\pgfqpoint{0.666679in}{2.355837in}}%
\pgfpathlineto{\pgfqpoint{0.762858in}{2.326616in}}%
\pgfpathlineto{\pgfqpoint{0.864991in}{2.297975in}}%
\pgfpathlineto{\pgfqpoint{0.957404in}{2.274144in}}%
\pgfusepath{stroke}%
\end{pgfscope}%
\begin{pgfscope}%
\pgfpathrectangle{\pgfqpoint{0.100000in}{0.100000in}}{\pgfqpoint{3.608454in}{2.310000in}}%
\pgfusepath{clip}%
\pgfsetbuttcap%
\pgfsetroundjoin%
\pgfsetlinewidth{0.010037pt}%
\definecolor{currentstroke}{rgb}{1.000000,1.000000,1.000000}%
\pgfsetstrokecolor{currentstroke}%
\pgfsetdash{}{0pt}%
\pgfpathmoveto{\pgfqpoint{0.650857in}{2.300012in}}%
\pgfpathlineto{\pgfqpoint{0.655593in}{2.281754in}}%
\pgfpathlineto{\pgfqpoint{0.663020in}{2.275689in}}%
\pgfpathlineto{\pgfqpoint{0.657094in}{2.267969in}}%
\pgfpathlineto{\pgfqpoint{0.651318in}{2.279281in}}%
\pgfpathlineto{\pgfqpoint{0.650857in}{2.300012in}}%
\pgfusepath{stroke}%
\end{pgfscope}%
\begin{pgfscope}%
\pgfpathrectangle{\pgfqpoint{0.100000in}{0.100000in}}{\pgfqpoint{3.608454in}{2.310000in}}%
\pgfusepath{clip}%
\pgfsetbuttcap%
\pgfsetroundjoin%
\pgfsetlinewidth{0.010037pt}%
\definecolor{currentstroke}{rgb}{1.000000,1.000000,1.000000}%
\pgfsetstrokecolor{currentstroke}%
\pgfsetdash{}{0pt}%
\pgfpathmoveto{\pgfqpoint{1.007001in}{2.262160in}}%
\pgfpathlineto{\pgfqpoint{1.109811in}{2.239088in}}%
\pgfpathlineto{\pgfqpoint{1.206635in}{2.219517in}}%
\pgfpathlineto{\pgfqpoint{1.281130in}{2.205868in}}%
\pgfpathlineto{\pgfqpoint{1.346076in}{2.194957in}}%
\pgfpathlineto{\pgfqpoint{1.411166in}{2.184941in}}%
\pgfpathlineto{\pgfqpoint{1.466597in}{2.177132in}}%
\pgfpathlineto{\pgfqpoint{1.522115in}{2.169972in}}%
\pgfpathlineto{\pgfqpoint{1.577712in}{2.163461in}}%
\pgfpathlineto{\pgfqpoint{1.630106in}{2.157928in}}%
\pgfpathlineto{\pgfqpoint{1.622900in}{2.078078in}}%
\pgfpathlineto{\pgfqpoint{1.612175in}{1.970174in}}%
\pgfpathlineto{\pgfqpoint{1.606550in}{1.914676in}}%
\pgfpathlineto{\pgfqpoint{1.598460in}{1.839883in}}%
\pgfpathlineto{\pgfqpoint{1.541187in}{1.846137in}}%
\pgfpathlineto{\pgfqpoint{1.475618in}{1.853580in}}%
\pgfpathlineto{\pgfqpoint{1.384573in}{1.866040in}}%
\pgfpathlineto{\pgfqpoint{1.343910in}{1.871841in}}%
\pgfpathlineto{\pgfqpoint{1.243728in}{1.887567in}}%
\pgfpathlineto{\pgfqpoint{1.209241in}{1.893876in}}%
\pgfpathlineto{\pgfqpoint{1.202053in}{1.853008in}}%
\pgfpathlineto{\pgfqpoint{1.198123in}{1.855924in}}%
\pgfpathlineto{\pgfqpoint{1.190738in}{1.875584in}}%
\pgfpathlineto{\pgfqpoint{1.181245in}{1.874373in}}%
\pgfpathlineto{\pgfqpoint{1.174451in}{1.864434in}}%
\pgfpathlineto{\pgfqpoint{1.160350in}{1.863105in}}%
\pgfpathlineto{\pgfqpoint{1.157517in}{1.867860in}}%
\pgfpathlineto{\pgfqpoint{1.144309in}{1.867284in}}%
\pgfpathlineto{\pgfqpoint{1.137669in}{1.871889in}}%
\pgfpathlineto{\pgfqpoint{1.128385in}{1.864765in}}%
\pgfpathlineto{\pgfqpoint{1.105833in}{1.871176in}}%
\pgfpathlineto{\pgfqpoint{1.095991in}{1.867733in}}%
\pgfpathlineto{\pgfqpoint{1.090697in}{1.883009in}}%
\pgfpathlineto{\pgfqpoint{1.090449in}{1.897544in}}%
\pgfpathlineto{\pgfqpoint{1.074904in}{1.907997in}}%
\pgfpathlineto{\pgfqpoint{1.077439in}{1.923062in}}%
\pgfpathlineto{\pgfqpoint{1.066370in}{1.947947in}}%
\pgfpathlineto{\pgfqpoint{1.067116in}{1.969101in}}%
\pgfpathlineto{\pgfqpoint{1.057602in}{1.979454in}}%
\pgfpathlineto{\pgfqpoint{1.048676in}{1.970294in}}%
\pgfpathlineto{\pgfqpoint{1.036527in}{1.965340in}}%
\pgfpathlineto{\pgfqpoint{1.027278in}{1.975551in}}%
\pgfpathlineto{\pgfqpoint{1.028782in}{1.991857in}}%
\pgfpathlineto{\pgfqpoint{1.039874in}{1.998281in}}%
\pgfpathlineto{\pgfqpoint{1.036925in}{2.008611in}}%
\pgfpathlineto{\pgfqpoint{1.056205in}{2.058614in}}%
\pgfpathlineto{\pgfqpoint{1.041003in}{2.059867in}}%
\pgfpathlineto{\pgfqpoint{1.039440in}{2.068841in}}%
\pgfpathlineto{\pgfqpoint{1.028366in}{2.076578in}}%
\pgfpathlineto{\pgfqpoint{1.029074in}{2.085229in}}%
\pgfpathlineto{\pgfqpoint{1.023232in}{2.091949in}}%
\pgfpathlineto{\pgfqpoint{1.013723in}{2.116347in}}%
\pgfpathlineto{\pgfqpoint{1.005033in}{2.121323in}}%
\pgfpathlineto{\pgfqpoint{0.998811in}{2.142381in}}%
\pgfpathlineto{\pgfqpoint{1.000279in}{2.156377in}}%
\pgfpathlineto{\pgfqpoint{0.988670in}{2.182155in}}%
\pgfpathlineto{\pgfqpoint{1.007001in}{2.262160in}}%
\pgfusepath{stroke}%
\end{pgfscope}%
\begin{pgfscope}%
\pgfpathrectangle{\pgfqpoint{0.100000in}{0.100000in}}{\pgfqpoint{3.608454in}{2.310000in}}%
\pgfusepath{clip}%
\pgfsetbuttcap%
\pgfsetroundjoin%
\pgfsetlinewidth{0.010037pt}%
\definecolor{currentstroke}{rgb}{1.000000,1.000000,1.000000}%
\pgfsetstrokecolor{currentstroke}%
\pgfsetdash{}{0pt}%
\pgfpathmoveto{\pgfqpoint{3.488916in}{1.850686in}}%
\pgfpathlineto{\pgfqpoint{3.486858in}{1.859435in}}%
\pgfpathlineto{\pgfqpoint{3.475422in}{1.867100in}}%
\pgfpathlineto{\pgfqpoint{3.445201in}{1.965955in}}%
\pgfpathlineto{\pgfqpoint{3.428911in}{2.013660in}}%
\pgfpathlineto{\pgfqpoint{3.442007in}{2.027403in}}%
\pgfpathlineto{\pgfqpoint{3.454737in}{2.061039in}}%
\pgfpathlineto{\pgfqpoint{3.461086in}{2.071788in}}%
\pgfpathlineto{\pgfqpoint{3.456588in}{2.076302in}}%
\pgfpathlineto{\pgfqpoint{3.455082in}{2.106159in}}%
\pgfpathlineto{\pgfqpoint{3.460762in}{2.115312in}}%
\pgfpathlineto{\pgfqpoint{3.458272in}{2.136586in}}%
\pgfpathlineto{\pgfqpoint{3.480942in}{2.206386in}}%
\pgfpathlineto{\pgfqpoint{3.491156in}{2.206767in}}%
\pgfpathlineto{\pgfqpoint{3.495396in}{2.194299in}}%
\pgfpathlineto{\pgfqpoint{3.504473in}{2.190719in}}%
\pgfpathlineto{\pgfqpoint{3.521621in}{2.205384in}}%
\pgfpathlineto{\pgfqpoint{3.535066in}{2.214046in}}%
\pgfpathlineto{\pgfqpoint{3.564692in}{2.198793in}}%
\pgfpathlineto{\pgfqpoint{3.591456in}{2.114100in}}%
\pgfpathlineto{\pgfqpoint{3.596544in}{2.093264in}}%
\pgfpathlineto{\pgfqpoint{3.608267in}{2.090820in}}%
\pgfpathlineto{\pgfqpoint{3.624167in}{2.076668in}}%
\pgfpathlineto{\pgfqpoint{3.623267in}{2.068423in}}%
\pgfpathlineto{\pgfqpoint{3.634098in}{2.058650in}}%
\pgfpathlineto{\pgfqpoint{3.642896in}{2.064257in}}%
\pgfpathlineto{\pgfqpoint{3.661307in}{2.042785in}}%
\pgfpathlineto{\pgfqpoint{3.653078in}{2.025601in}}%
\pgfpathlineto{\pgfqpoint{3.642044in}{2.025261in}}%
\pgfpathlineto{\pgfqpoint{3.633192in}{2.009922in}}%
\pgfpathlineto{\pgfqpoint{3.622468in}{2.007746in}}%
\pgfpathlineto{\pgfqpoint{3.614595in}{1.999578in}}%
\pgfpathlineto{\pgfqpoint{3.591081in}{1.990819in}}%
\pgfpathlineto{\pgfqpoint{3.576849in}{1.976702in}}%
\pgfpathlineto{\pgfqpoint{3.569620in}{1.986835in}}%
\pgfpathlineto{\pgfqpoint{3.563050in}{1.979573in}}%
\pgfpathlineto{\pgfqpoint{3.564849in}{1.950215in}}%
\pgfpathlineto{\pgfqpoint{3.559648in}{1.938586in}}%
\pgfpathlineto{\pgfqpoint{3.548302in}{1.941762in}}%
\pgfpathlineto{\pgfqpoint{3.546471in}{1.929840in}}%
\pgfpathlineto{\pgfqpoint{3.541568in}{1.924938in}}%
\pgfpathlineto{\pgfqpoint{3.531757in}{1.926132in}}%
\pgfpathlineto{\pgfqpoint{3.532375in}{1.914985in}}%
\pgfpathlineto{\pgfqpoint{3.517503in}{1.918151in}}%
\pgfpathlineto{\pgfqpoint{3.509381in}{1.902707in}}%
\pgfpathlineto{\pgfqpoint{3.512448in}{1.894627in}}%
\pgfpathlineto{\pgfqpoint{3.507617in}{1.881187in}}%
\pgfpathlineto{\pgfqpoint{3.500012in}{1.871301in}}%
\pgfpathlineto{\pgfqpoint{3.498098in}{1.850665in}}%
\pgfpathlineto{\pgfqpoint{3.488916in}{1.850686in}}%
\pgfusepath{stroke}%
\end{pgfscope}%
\begin{pgfscope}%
\pgfpathrectangle{\pgfqpoint{0.100000in}{0.100000in}}{\pgfqpoint{3.608454in}{2.310000in}}%
\pgfusepath{clip}%
\pgfsetbuttcap%
\pgfsetroundjoin%
\pgfsetlinewidth{0.010037pt}%
\definecolor{currentstroke}{rgb}{1.000000,1.000000,1.000000}%
\pgfsetstrokecolor{currentstroke}%
\pgfsetdash{}{0pt}%
\pgfpathmoveto{\pgfqpoint{3.595305in}{1.984790in}}%
\pgfpathlineto{\pgfqpoint{3.602019in}{1.991823in}}%
\pgfpathlineto{\pgfqpoint{3.608411in}{1.985213in}}%
\pgfpathlineto{\pgfqpoint{3.596929in}{1.976454in}}%
\pgfpathlineto{\pgfqpoint{3.595305in}{1.984790in}}%
\pgfusepath{stroke}%
\end{pgfscope}%
\begin{pgfscope}%
\pgfpathrectangle{\pgfqpoint{0.100000in}{0.100000in}}{\pgfqpoint{3.608454in}{2.310000in}}%
\pgfusepath{clip}%
\pgfsetbuttcap%
\pgfsetroundjoin%
\pgfsetlinewidth{0.010037pt}%
\definecolor{currentstroke}{rgb}{1.000000,1.000000,1.000000}%
\pgfsetstrokecolor{currentstroke}%
\pgfsetdash{}{0pt}%
\pgfpathmoveto{\pgfqpoint{1.606550in}{1.914676in}}%
\pgfpathlineto{\pgfqpoint{1.612175in}{1.970174in}}%
\pgfpathlineto{\pgfqpoint{1.622900in}{2.078078in}}%
\pgfpathlineto{\pgfqpoint{1.630106in}{2.157928in}}%
\pgfpathlineto{\pgfqpoint{1.689116in}{2.152392in}}%
\pgfpathlineto{\pgfqpoint{1.764609in}{2.146381in}}%
\pgfpathlineto{\pgfqpoint{1.833613in}{2.141936in}}%
\pgfpathlineto{\pgfqpoint{1.896094in}{2.138775in}}%
\pgfpathlineto{\pgfqpoint{1.989337in}{2.135576in}}%
\pgfpathlineto{\pgfqpoint{1.995700in}{2.109297in}}%
\pgfpathlineto{\pgfqpoint{1.993422in}{2.096738in}}%
\pgfpathlineto{\pgfqpoint{1.992673in}{2.070924in}}%
\pgfpathlineto{\pgfqpoint{1.997002in}{2.051591in}}%
\pgfpathlineto{\pgfqpoint{2.006920in}{2.023070in}}%
\pgfpathlineto{\pgfqpoint{2.006897in}{1.987162in}}%
\pgfpathlineto{\pgfqpoint{2.008733in}{1.945448in}}%
\pgfpathlineto{\pgfqpoint{2.011253in}{1.934205in}}%
\pgfpathlineto{\pgfqpoint{2.018625in}{1.921863in}}%
\pgfpathlineto{\pgfqpoint{2.021056in}{1.902632in}}%
\pgfpathlineto{\pgfqpoint{2.020006in}{1.889792in}}%
\pgfpathlineto{\pgfqpoint{1.941838in}{1.891485in}}%
\pgfpathlineto{\pgfqpoint{1.884971in}{1.894368in}}%
\pgfpathlineto{\pgfqpoint{1.801609in}{1.898805in}}%
\pgfpathlineto{\pgfqpoint{1.719401in}{1.904645in}}%
\pgfpathlineto{\pgfqpoint{1.664646in}{1.909062in}}%
\pgfpathlineto{\pgfqpoint{1.606550in}{1.914676in}}%
\pgfusepath{stroke}%
\end{pgfscope}%
\begin{pgfscope}%
\pgfpathrectangle{\pgfqpoint{0.100000in}{0.100000in}}{\pgfqpoint{3.608454in}{2.310000in}}%
\pgfusepath{clip}%
\pgfsetbuttcap%
\pgfsetroundjoin%
\pgfsetlinewidth{0.010037pt}%
\definecolor{currentstroke}{rgb}{1.000000,1.000000,1.000000}%
\pgfsetstrokecolor{currentstroke}%
\pgfsetdash{}{0pt}%
\pgfpathmoveto{\pgfqpoint{1.582921in}{1.682417in}}%
\pgfpathlineto{\pgfqpoint{1.591946in}{1.775447in}}%
\pgfpathlineto{\pgfqpoint{1.598460in}{1.839883in}}%
\pgfpathlineto{\pgfqpoint{1.606550in}{1.914676in}}%
\pgfpathlineto{\pgfqpoint{1.664646in}{1.909062in}}%
\pgfpathlineto{\pgfqpoint{1.719401in}{1.904645in}}%
\pgfpathlineto{\pgfqpoint{1.801609in}{1.898805in}}%
\pgfpathlineto{\pgfqpoint{1.884971in}{1.894368in}}%
\pgfpathlineto{\pgfqpoint{1.941838in}{1.891485in}}%
\pgfpathlineto{\pgfqpoint{2.020006in}{1.889792in}}%
\pgfpathlineto{\pgfqpoint{2.014714in}{1.874348in}}%
\pgfpathlineto{\pgfqpoint{2.004150in}{1.862224in}}%
\pgfpathlineto{\pgfqpoint{2.012240in}{1.848265in}}%
\pgfpathlineto{\pgfqpoint{2.021160in}{1.845282in}}%
\pgfpathlineto{\pgfqpoint{2.025393in}{1.837264in}}%
\pgfpathlineto{\pgfqpoint{2.024421in}{1.778721in}}%
\pgfpathlineto{\pgfqpoint{2.022794in}{1.696330in}}%
\pgfpathlineto{\pgfqpoint{2.016778in}{1.674690in}}%
\pgfpathlineto{\pgfqpoint{2.022159in}{1.662720in}}%
\pgfpathlineto{\pgfqpoint{2.011355in}{1.636997in}}%
\pgfpathlineto{\pgfqpoint{2.022737in}{1.616262in}}%
\pgfpathlineto{\pgfqpoint{2.013067in}{1.617849in}}%
\pgfpathlineto{\pgfqpoint{2.006469in}{1.630750in}}%
\pgfpathlineto{\pgfqpoint{1.982922in}{1.639593in}}%
\pgfpathlineto{\pgfqpoint{1.968073in}{1.647372in}}%
\pgfpathlineto{\pgfqpoint{1.943154in}{1.648019in}}%
\pgfpathlineto{\pgfqpoint{1.934505in}{1.640932in}}%
\pgfpathlineto{\pgfqpoint{1.906300in}{1.654886in}}%
\pgfpathlineto{\pgfqpoint{1.904136in}{1.659299in}}%
\pgfpathlineto{\pgfqpoint{1.805703in}{1.663958in}}%
\pgfpathlineto{\pgfqpoint{1.745900in}{1.667386in}}%
\pgfpathlineto{\pgfqpoint{1.696503in}{1.671254in}}%
\pgfpathlineto{\pgfqpoint{1.614880in}{1.678966in}}%
\pgfpathlineto{\pgfqpoint{1.582921in}{1.682417in}}%
\pgfusepath{stroke}%
\end{pgfscope}%
\begin{pgfscope}%
\pgfpathrectangle{\pgfqpoint{0.100000in}{0.100000in}}{\pgfqpoint{3.608454in}{2.310000in}}%
\pgfusepath{clip}%
\pgfsetbuttcap%
\pgfsetroundjoin%
\pgfsetlinewidth{0.010037pt}%
\definecolor{currentstroke}{rgb}{1.000000,1.000000,1.000000}%
\pgfsetstrokecolor{currentstroke}%
\pgfsetdash{}{0pt}%
\pgfpathmoveto{\pgfqpoint{1.567441in}{1.524892in}}%
\pgfpathlineto{\pgfqpoint{1.514964in}{1.529684in}}%
\pgfpathlineto{\pgfqpoint{1.400578in}{1.543802in}}%
\pgfpathlineto{\pgfqpoint{1.338353in}{1.552733in}}%
\pgfpathlineto{\pgfqpoint{1.271614in}{1.562313in}}%
\pgfpathlineto{\pgfqpoint{1.215397in}{1.571309in}}%
\pgfpathlineto{\pgfqpoint{1.153691in}{1.581874in}}%
\pgfpathlineto{\pgfqpoint{1.167721in}{1.659701in}}%
\pgfpathlineto{\pgfqpoint{1.181932in}{1.739505in}}%
\pgfpathlineto{\pgfqpoint{1.202053in}{1.853008in}}%
\pgfpathlineto{\pgfqpoint{1.209241in}{1.893876in}}%
\pgfpathlineto{\pgfqpoint{1.243728in}{1.887567in}}%
\pgfpathlineto{\pgfqpoint{1.343910in}{1.871841in}}%
\pgfpathlineto{\pgfqpoint{1.384573in}{1.866040in}}%
\pgfpathlineto{\pgfqpoint{1.475618in}{1.853580in}}%
\pgfpathlineto{\pgfqpoint{1.541187in}{1.846137in}}%
\pgfpathlineto{\pgfqpoint{1.598460in}{1.839883in}}%
\pgfpathlineto{\pgfqpoint{1.591946in}{1.775447in}}%
\pgfpathlineto{\pgfqpoint{1.582921in}{1.682417in}}%
\pgfpathlineto{\pgfqpoint{1.575175in}{1.603355in}}%
\pgfpathlineto{\pgfqpoint{1.567441in}{1.524892in}}%
\pgfusepath{stroke}%
\end{pgfscope}%
\begin{pgfscope}%
\pgfpathrectangle{\pgfqpoint{0.100000in}{0.100000in}}{\pgfqpoint{3.608454in}{2.310000in}}%
\pgfusepath{clip}%
\pgfsetbuttcap%
\pgfsetroundjoin%
\pgfsetlinewidth{0.010037pt}%
\definecolor{currentstroke}{rgb}{1.000000,1.000000,1.000000}%
\pgfsetstrokecolor{currentstroke}%
\pgfsetdash{}{0pt}%
\pgfpathmoveto{\pgfqpoint{2.526890in}{1.632246in}}%
\pgfpathlineto{\pgfqpoint{2.460390in}{1.627522in}}%
\pgfpathlineto{\pgfqpoint{2.361269in}{1.623298in}}%
\pgfpathlineto{\pgfqpoint{2.357502in}{1.633311in}}%
\pgfpathlineto{\pgfqpoint{2.335521in}{1.640795in}}%
\pgfpathlineto{\pgfqpoint{2.330673in}{1.654936in}}%
\pgfpathlineto{\pgfqpoint{2.328643in}{1.672430in}}%
\pgfpathlineto{\pgfqpoint{2.333596in}{1.681393in}}%
\pgfpathlineto{\pgfqpoint{2.325773in}{1.689994in}}%
\pgfpathlineto{\pgfqpoint{2.323886in}{1.700253in}}%
\pgfpathlineto{\pgfqpoint{2.321364in}{1.722947in}}%
\pgfpathlineto{\pgfqpoint{2.313867in}{1.735275in}}%
\pgfpathlineto{\pgfqpoint{2.300552in}{1.742194in}}%
\pgfpathlineto{\pgfqpoint{2.286062in}{1.753611in}}%
\pgfpathlineto{\pgfqpoint{2.278572in}{1.767120in}}%
\pgfpathlineto{\pgfqpoint{2.265132in}{1.772560in}}%
\pgfpathlineto{\pgfqpoint{2.257232in}{1.781413in}}%
\pgfpathlineto{\pgfqpoint{2.247661in}{1.782916in}}%
\pgfpathlineto{\pgfqpoint{2.230573in}{1.796062in}}%
\pgfpathlineto{\pgfqpoint{2.233349in}{1.811190in}}%
\pgfpathlineto{\pgfqpoint{2.232814in}{1.839958in}}%
\pgfpathlineto{\pgfqpoint{2.238084in}{1.847876in}}%
\pgfpathlineto{\pgfqpoint{2.233349in}{1.859834in}}%
\pgfpathlineto{\pgfqpoint{2.225008in}{1.862146in}}%
\pgfpathlineto{\pgfqpoint{2.225698in}{1.872648in}}%
\pgfpathlineto{\pgfqpoint{2.236046in}{1.889238in}}%
\pgfpathlineto{\pgfqpoint{2.256539in}{1.902357in}}%
\pgfpathlineto{\pgfqpoint{2.255262in}{1.949012in}}%
\pgfpathlineto{\pgfqpoint{2.265516in}{1.956021in}}%
\pgfpathlineto{\pgfqpoint{2.275220in}{1.951352in}}%
\pgfpathlineto{\pgfqpoint{2.294988in}{1.958187in}}%
\pgfpathlineto{\pgfqpoint{2.332179in}{1.975363in}}%
\pgfpathlineto{\pgfqpoint{2.337024in}{1.970045in}}%
\pgfpathlineto{\pgfqpoint{2.329980in}{1.945955in}}%
\pgfpathlineto{\pgfqpoint{2.340460in}{1.951242in}}%
\pgfpathlineto{\pgfqpoint{2.358409in}{1.945963in}}%
\pgfpathlineto{\pgfqpoint{2.369439in}{1.941557in}}%
\pgfpathlineto{\pgfqpoint{2.375622in}{1.928645in}}%
\pgfpathlineto{\pgfqpoint{2.432207in}{1.916460in}}%
\pgfpathlineto{\pgfqpoint{2.449120in}{1.908100in}}%
\pgfpathlineto{\pgfqpoint{2.466317in}{1.908198in}}%
\pgfpathlineto{\pgfqpoint{2.483987in}{1.904750in}}%
\pgfpathlineto{\pgfqpoint{2.495454in}{1.892970in}}%
\pgfpathlineto{\pgfqpoint{2.506412in}{1.887150in}}%
\pgfpathlineto{\pgfqpoint{2.508426in}{1.870361in}}%
\pgfpathlineto{\pgfqpoint{2.505188in}{1.859820in}}%
\pgfpathlineto{\pgfqpoint{2.517401in}{1.859041in}}%
\pgfpathlineto{\pgfqpoint{2.513287in}{1.846815in}}%
\pgfpathlineto{\pgfqpoint{2.517200in}{1.842472in}}%
\pgfpathlineto{\pgfqpoint{2.521092in}{1.830931in}}%
\pgfpathlineto{\pgfqpoint{2.509190in}{1.824821in}}%
\pgfpathlineto{\pgfqpoint{2.502280in}{1.807797in}}%
\pgfpathlineto{\pgfqpoint{2.500111in}{1.795759in}}%
\pgfpathlineto{\pgfqpoint{2.506741in}{1.793698in}}%
\pgfpathlineto{\pgfqpoint{2.515230in}{1.802725in}}%
\pgfpathlineto{\pgfqpoint{2.522446in}{1.818369in}}%
\pgfpathlineto{\pgfqpoint{2.532211in}{1.823802in}}%
\pgfpathlineto{\pgfqpoint{2.539548in}{1.816735in}}%
\pgfpathlineto{\pgfqpoint{2.532326in}{1.795344in}}%
\pgfpathlineto{\pgfqpoint{2.530060in}{1.778738in}}%
\pgfpathlineto{\pgfqpoint{2.532194in}{1.766801in}}%
\pgfpathlineto{\pgfqpoint{2.525487in}{1.760048in}}%
\pgfpathlineto{\pgfqpoint{2.522137in}{1.743546in}}%
\pgfpathlineto{\pgfqpoint{2.524874in}{1.726204in}}%
\pgfpathlineto{\pgfqpoint{2.516958in}{1.700555in}}%
\pgfpathlineto{\pgfqpoint{2.517125in}{1.687738in}}%
\pgfpathlineto{\pgfqpoint{2.523380in}{1.659966in}}%
\pgfpathlineto{\pgfqpoint{2.527434in}{1.655201in}}%
\pgfpathlineto{\pgfqpoint{2.526890in}{1.632246in}}%
\pgfusepath{stroke}%
\end{pgfscope}%
\begin{pgfscope}%
\pgfpathrectangle{\pgfqpoint{0.100000in}{0.100000in}}{\pgfqpoint{3.608454in}{2.310000in}}%
\pgfusepath{clip}%
\pgfsetbuttcap%
\pgfsetroundjoin%
\pgfsetlinewidth{0.010037pt}%
\definecolor{currentstroke}{rgb}{1.000000,1.000000,1.000000}%
\pgfsetstrokecolor{currentstroke}%
\pgfsetdash{}{0pt}%
\pgfpathmoveto{\pgfqpoint{2.551823in}{1.857339in}}%
\pgfpathlineto{\pgfqpoint{2.553308in}{1.846261in}}%
\pgfpathlineto{\pgfqpoint{2.539709in}{1.817072in}}%
\pgfpathlineto{\pgfqpoint{2.533666in}{1.825527in}}%
\pgfpathlineto{\pgfqpoint{2.551823in}{1.857339in}}%
\pgfusepath{stroke}%
\end{pgfscope}%
\begin{pgfscope}%
\pgfpathrectangle{\pgfqpoint{0.100000in}{0.100000in}}{\pgfqpoint{3.608454in}{2.310000in}}%
\pgfusepath{clip}%
\pgfsetbuttcap%
\pgfsetroundjoin%
\pgfsetlinewidth{0.010037pt}%
\definecolor{currentstroke}{rgb}{1.000000,1.000000,1.000000}%
\pgfsetstrokecolor{currentstroke}%
\pgfsetdash{}{0pt}%
\pgfpathmoveto{\pgfqpoint{0.905799in}{2.039449in}}%
\pgfpathlineto{\pgfqpoint{0.908487in}{2.052757in}}%
\pgfpathlineto{\pgfqpoint{0.906458in}{2.067206in}}%
\pgfpathlineto{\pgfqpoint{0.909565in}{2.082958in}}%
\pgfpathlineto{\pgfqpoint{0.925385in}{2.146869in}}%
\pgfpathlineto{\pgfqpoint{0.938691in}{2.199997in}}%
\pgfpathlineto{\pgfqpoint{0.957404in}{2.274144in}}%
\pgfpathlineto{\pgfqpoint{1.007001in}{2.262160in}}%
\pgfpathlineto{\pgfqpoint{0.988670in}{2.182155in}}%
\pgfpathlineto{\pgfqpoint{1.000279in}{2.156377in}}%
\pgfpathlineto{\pgfqpoint{0.998811in}{2.142381in}}%
\pgfpathlineto{\pgfqpoint{1.005033in}{2.121323in}}%
\pgfpathlineto{\pgfqpoint{1.013723in}{2.116347in}}%
\pgfpathlineto{\pgfqpoint{1.023232in}{2.091949in}}%
\pgfpathlineto{\pgfqpoint{1.029074in}{2.085229in}}%
\pgfpathlineto{\pgfqpoint{1.028366in}{2.076578in}}%
\pgfpathlineto{\pgfqpoint{1.039440in}{2.068841in}}%
\pgfpathlineto{\pgfqpoint{1.041003in}{2.059867in}}%
\pgfpathlineto{\pgfqpoint{1.056205in}{2.058614in}}%
\pgfpathlineto{\pgfqpoint{1.036925in}{2.008611in}}%
\pgfpathlineto{\pgfqpoint{1.039874in}{1.998281in}}%
\pgfpathlineto{\pgfqpoint{1.028782in}{1.991857in}}%
\pgfpathlineto{\pgfqpoint{1.027278in}{1.975551in}}%
\pgfpathlineto{\pgfqpoint{1.036527in}{1.965340in}}%
\pgfpathlineto{\pgfqpoint{1.048676in}{1.970294in}}%
\pgfpathlineto{\pgfqpoint{1.057602in}{1.979454in}}%
\pgfpathlineto{\pgfqpoint{1.067116in}{1.969101in}}%
\pgfpathlineto{\pgfqpoint{1.066370in}{1.947947in}}%
\pgfpathlineto{\pgfqpoint{1.077439in}{1.923062in}}%
\pgfpathlineto{\pgfqpoint{1.074904in}{1.907997in}}%
\pgfpathlineto{\pgfqpoint{1.090449in}{1.897544in}}%
\pgfpathlineto{\pgfqpoint{1.090697in}{1.883009in}}%
\pgfpathlineto{\pgfqpoint{1.095991in}{1.867733in}}%
\pgfpathlineto{\pgfqpoint{1.105833in}{1.871176in}}%
\pgfpathlineto{\pgfqpoint{1.128385in}{1.864765in}}%
\pgfpathlineto{\pgfqpoint{1.137669in}{1.871889in}}%
\pgfpathlineto{\pgfqpoint{1.144309in}{1.867284in}}%
\pgfpathlineto{\pgfqpoint{1.157517in}{1.867860in}}%
\pgfpathlineto{\pgfqpoint{1.160350in}{1.863105in}}%
\pgfpathlineto{\pgfqpoint{1.174451in}{1.864434in}}%
\pgfpathlineto{\pgfqpoint{1.181245in}{1.874373in}}%
\pgfpathlineto{\pgfqpoint{1.190738in}{1.875584in}}%
\pgfpathlineto{\pgfqpoint{1.198123in}{1.855924in}}%
\pgfpathlineto{\pgfqpoint{1.202053in}{1.853008in}}%
\pgfpathlineto{\pgfqpoint{1.181932in}{1.739505in}}%
\pgfpathlineto{\pgfqpoint{1.167721in}{1.659701in}}%
\pgfpathlineto{\pgfqpoint{1.055637in}{1.681339in}}%
\pgfpathlineto{\pgfqpoint{0.995094in}{1.693381in}}%
\pgfpathlineto{\pgfqpoint{0.938464in}{1.705834in}}%
\pgfpathlineto{\pgfqpoint{0.885533in}{1.717818in}}%
\pgfpathlineto{\pgfqpoint{0.824275in}{1.732591in}}%
\pgfpathlineto{\pgfqpoint{0.855872in}{1.862213in}}%
\pgfpathlineto{\pgfqpoint{0.857576in}{1.871686in}}%
\pgfpathlineto{\pgfqpoint{0.871627in}{1.896508in}}%
\pgfpathlineto{\pgfqpoint{0.857078in}{1.911427in}}%
\pgfpathlineto{\pgfqpoint{0.860099in}{1.926077in}}%
\pgfpathlineto{\pgfqpoint{0.865637in}{1.929744in}}%
\pgfpathlineto{\pgfqpoint{0.875529in}{1.944840in}}%
\pgfpathlineto{\pgfqpoint{0.889947in}{1.955284in}}%
\pgfpathlineto{\pgfqpoint{0.890735in}{1.963012in}}%
\pgfpathlineto{\pgfqpoint{0.899437in}{1.970733in}}%
\pgfpathlineto{\pgfqpoint{0.906663in}{1.985185in}}%
\pgfpathlineto{\pgfqpoint{0.921484in}{2.000475in}}%
\pgfpathlineto{\pgfqpoint{0.920432in}{2.015576in}}%
\pgfpathlineto{\pgfqpoint{0.910299in}{2.023965in}}%
\pgfpathlineto{\pgfqpoint{0.905799in}{2.039449in}}%
\pgfusepath{stroke}%
\end{pgfscope}%
\begin{pgfscope}%
\pgfpathrectangle{\pgfqpoint{0.100000in}{0.100000in}}{\pgfqpoint{3.608454in}{2.310000in}}%
\pgfusepath{clip}%
\pgfsetbuttcap%
\pgfsetroundjoin%
\pgfsetlinewidth{0.010037pt}%
\definecolor{currentstroke}{rgb}{1.000000,1.000000,1.000000}%
\pgfsetstrokecolor{currentstroke}%
\pgfsetdash{}{0pt}%
\pgfpathmoveto{\pgfqpoint{3.359367in}{1.786492in}}%
\pgfpathlineto{\pgfqpoint{3.354710in}{1.801193in}}%
\pgfpathlineto{\pgfqpoint{3.346067in}{1.845790in}}%
\pgfpathlineto{\pgfqpoint{3.334894in}{1.859510in}}%
\pgfpathlineto{\pgfqpoint{3.325093in}{1.884184in}}%
\pgfpathlineto{\pgfqpoint{3.328311in}{1.902477in}}%
\pgfpathlineto{\pgfqpoint{3.325730in}{1.915960in}}%
\pgfpathlineto{\pgfqpoint{3.317356in}{1.929202in}}%
\pgfpathlineto{\pgfqpoint{3.317011in}{1.943790in}}%
\pgfpathlineto{\pgfqpoint{3.312149in}{1.959523in}}%
\pgfpathlineto{\pgfqpoint{3.355664in}{1.970232in}}%
\pgfpathlineto{\pgfqpoint{3.412185in}{1.985489in}}%
\pgfpathlineto{\pgfqpoint{3.414441in}{1.976741in}}%
\pgfpathlineto{\pgfqpoint{3.410842in}{1.962813in}}%
\pgfpathlineto{\pgfqpoint{3.419291in}{1.951669in}}%
\pgfpathlineto{\pgfqpoint{3.414804in}{1.937548in}}%
\pgfpathlineto{\pgfqpoint{3.396973in}{1.919797in}}%
\pgfpathlineto{\pgfqpoint{3.401874in}{1.906464in}}%
\pgfpathlineto{\pgfqpoint{3.398242in}{1.878280in}}%
\pgfpathlineto{\pgfqpoint{3.393157in}{1.862389in}}%
\pgfpathlineto{\pgfqpoint{3.399644in}{1.817200in}}%
\pgfpathlineto{\pgfqpoint{3.398770in}{1.801301in}}%
\pgfpathlineto{\pgfqpoint{3.405047in}{1.796199in}}%
\pgfpathlineto{\pgfqpoint{3.359367in}{1.786492in}}%
\pgfusepath{stroke}%
\end{pgfscope}%
\begin{pgfscope}%
\pgfpathrectangle{\pgfqpoint{0.100000in}{0.100000in}}{\pgfqpoint{3.608454in}{2.310000in}}%
\pgfusepath{clip}%
\pgfsetbuttcap%
\pgfsetroundjoin%
\pgfsetlinewidth{0.010037pt}%
\definecolor{currentstroke}{rgb}{1.000000,1.000000,1.000000}%
\pgfsetstrokecolor{currentstroke}%
\pgfsetdash{}{0pt}%
\pgfpathmoveto{\pgfqpoint{2.022794in}{1.696330in}}%
\pgfpathlineto{\pgfqpoint{2.024421in}{1.778721in}}%
\pgfpathlineto{\pgfqpoint{2.025393in}{1.837264in}}%
\pgfpathlineto{\pgfqpoint{2.021160in}{1.845282in}}%
\pgfpathlineto{\pgfqpoint{2.012240in}{1.848265in}}%
\pgfpathlineto{\pgfqpoint{2.004150in}{1.862224in}}%
\pgfpathlineto{\pgfqpoint{2.014714in}{1.874348in}}%
\pgfpathlineto{\pgfqpoint{2.020006in}{1.889792in}}%
\pgfpathlineto{\pgfqpoint{2.021056in}{1.902632in}}%
\pgfpathlineto{\pgfqpoint{2.018625in}{1.921863in}}%
\pgfpathlineto{\pgfqpoint{2.011253in}{1.934205in}}%
\pgfpathlineto{\pgfqpoint{2.008733in}{1.945448in}}%
\pgfpathlineto{\pgfqpoint{2.006897in}{1.987162in}}%
\pgfpathlineto{\pgfqpoint{2.006920in}{2.023070in}}%
\pgfpathlineto{\pgfqpoint{1.997002in}{2.051591in}}%
\pgfpathlineto{\pgfqpoint{1.992673in}{2.070924in}}%
\pgfpathlineto{\pgfqpoint{1.993422in}{2.096738in}}%
\pgfpathlineto{\pgfqpoint{1.995700in}{2.109297in}}%
\pgfpathlineto{\pgfqpoint{1.989337in}{2.135576in}}%
\pgfpathlineto{\pgfqpoint{2.032661in}{2.134709in}}%
\pgfpathlineto{\pgfqpoint{2.098467in}{2.134144in}}%
\pgfpathlineto{\pgfqpoint{2.098827in}{2.164013in}}%
\pgfpathlineto{\pgfqpoint{2.115577in}{2.160724in}}%
\pgfpathlineto{\pgfqpoint{2.123605in}{2.124304in}}%
\pgfpathlineto{\pgfqpoint{2.129524in}{2.111209in}}%
\pgfpathlineto{\pgfqpoint{2.144242in}{2.110823in}}%
\pgfpathlineto{\pgfqpoint{2.147536in}{2.106379in}}%
\pgfpathlineto{\pgfqpoint{2.168066in}{2.104411in}}%
\pgfpathlineto{\pgfqpoint{2.171516in}{2.095382in}}%
\pgfpathlineto{\pgfqpoint{2.185665in}{2.097411in}}%
\pgfpathlineto{\pgfqpoint{2.196654in}{2.105856in}}%
\pgfpathlineto{\pgfqpoint{2.215604in}{2.105540in}}%
\pgfpathlineto{\pgfqpoint{2.227330in}{2.098749in}}%
\pgfpathlineto{\pgfqpoint{2.228678in}{2.092380in}}%
\pgfpathlineto{\pgfqpoint{2.239835in}{2.091046in}}%
\pgfpathlineto{\pgfqpoint{2.247116in}{2.073673in}}%
\pgfpathlineto{\pgfqpoint{2.251814in}{2.084355in}}%
\pgfpathlineto{\pgfqpoint{2.264631in}{2.085012in}}%
\pgfpathlineto{\pgfqpoint{2.267873in}{2.076692in}}%
\pgfpathlineto{\pgfqpoint{2.282295in}{2.072895in}}%
\pgfpathlineto{\pgfqpoint{2.290247in}{2.060912in}}%
\pgfpathlineto{\pgfqpoint{2.307885in}{2.064633in}}%
\pgfpathlineto{\pgfqpoint{2.327294in}{2.079338in}}%
\pgfpathlineto{\pgfqpoint{2.334370in}{2.066375in}}%
\pgfpathlineto{\pgfqpoint{2.366216in}{2.069923in}}%
\pgfpathlineto{\pgfqpoint{2.379829in}{2.061026in}}%
\pgfpathlineto{\pgfqpoint{2.387752in}{2.064202in}}%
\pgfpathlineto{\pgfqpoint{2.394106in}{2.059194in}}%
\pgfpathlineto{\pgfqpoint{2.375257in}{2.047310in}}%
\pgfpathlineto{\pgfqpoint{2.348353in}{2.036722in}}%
\pgfpathlineto{\pgfqpoint{2.321706in}{2.015558in}}%
\pgfpathlineto{\pgfqpoint{2.298605in}{1.987726in}}%
\pgfpathlineto{\pgfqpoint{2.281117in}{1.971272in}}%
\pgfpathlineto{\pgfqpoint{2.265799in}{1.959963in}}%
\pgfpathlineto{\pgfqpoint{2.255262in}{1.949012in}}%
\pgfpathlineto{\pgfqpoint{2.256539in}{1.902357in}}%
\pgfpathlineto{\pgfqpoint{2.236046in}{1.889238in}}%
\pgfpathlineto{\pgfqpoint{2.225698in}{1.872648in}}%
\pgfpathlineto{\pgfqpoint{2.225008in}{1.862146in}}%
\pgfpathlineto{\pgfqpoint{2.233349in}{1.859834in}}%
\pgfpathlineto{\pgfqpoint{2.238084in}{1.847876in}}%
\pgfpathlineto{\pgfqpoint{2.232814in}{1.839958in}}%
\pgfpathlineto{\pgfqpoint{2.233349in}{1.811190in}}%
\pgfpathlineto{\pgfqpoint{2.230573in}{1.796062in}}%
\pgfpathlineto{\pgfqpoint{2.247661in}{1.782916in}}%
\pgfpathlineto{\pgfqpoint{2.257232in}{1.781413in}}%
\pgfpathlineto{\pgfqpoint{2.265132in}{1.772560in}}%
\pgfpathlineto{\pgfqpoint{2.278572in}{1.767120in}}%
\pgfpathlineto{\pgfqpoint{2.286062in}{1.753611in}}%
\pgfpathlineto{\pgfqpoint{2.300552in}{1.742194in}}%
\pgfpathlineto{\pgfqpoint{2.313867in}{1.735275in}}%
\pgfpathlineto{\pgfqpoint{2.321364in}{1.722947in}}%
\pgfpathlineto{\pgfqpoint{2.323886in}{1.700253in}}%
\pgfpathlineto{\pgfqpoint{2.253222in}{1.697687in}}%
\pgfpathlineto{\pgfqpoint{2.192987in}{1.696427in}}%
\pgfpathlineto{\pgfqpoint{2.138106in}{1.695621in}}%
\pgfpathlineto{\pgfqpoint{2.080050in}{1.695709in}}%
\pgfpathlineto{\pgfqpoint{2.022794in}{1.696330in}}%
\pgfusepath{stroke}%
\end{pgfscope}%
\begin{pgfscope}%
\pgfpathrectangle{\pgfqpoint{0.100000in}{0.100000in}}{\pgfqpoint{3.608454in}{2.310000in}}%
\pgfusepath{clip}%
\pgfsetbuttcap%
\pgfsetroundjoin%
\pgfsetlinewidth{0.010037pt}%
\definecolor{currentstroke}{rgb}{1.000000,1.000000,1.000000}%
\pgfsetstrokecolor{currentstroke}%
\pgfsetdash{}{0pt}%
\pgfpathmoveto{\pgfqpoint{0.418363in}{1.850356in}}%
\pgfpathlineto{\pgfqpoint{0.412766in}{1.860654in}}%
\pgfpathlineto{\pgfqpoint{0.416340in}{1.887087in}}%
\pgfpathlineto{\pgfqpoint{0.423237in}{1.900901in}}%
\pgfpathlineto{\pgfqpoint{0.419789in}{1.919587in}}%
\pgfpathlineto{\pgfqpoint{0.426984in}{1.927445in}}%
\pgfpathlineto{\pgfqpoint{0.440110in}{1.948627in}}%
\pgfpathlineto{\pgfqpoint{0.451240in}{1.961413in}}%
\pgfpathlineto{\pgfqpoint{0.467509in}{1.989261in}}%
\pgfpathlineto{\pgfqpoint{0.479986in}{2.019596in}}%
\pgfpathlineto{\pgfqpoint{0.493232in}{2.048048in}}%
\pgfpathlineto{\pgfqpoint{0.495910in}{2.059930in}}%
\pgfpathlineto{\pgfqpoint{0.514183in}{2.093924in}}%
\pgfpathlineto{\pgfqpoint{0.517702in}{2.108854in}}%
\pgfpathlineto{\pgfqpoint{0.525482in}{2.124532in}}%
\pgfpathlineto{\pgfqpoint{0.528263in}{2.143657in}}%
\pgfpathlineto{\pgfqpoint{0.543116in}{2.152982in}}%
\pgfpathlineto{\pgfqpoint{0.560727in}{2.157683in}}%
\pgfpathlineto{\pgfqpoint{0.569631in}{2.147155in}}%
\pgfpathlineto{\pgfqpoint{0.577330in}{2.147934in}}%
\pgfpathlineto{\pgfqpoint{0.589348in}{2.135386in}}%
\pgfpathlineto{\pgfqpoint{0.590946in}{2.123567in}}%
\pgfpathlineto{\pgfqpoint{0.586388in}{2.100423in}}%
\pgfpathlineto{\pgfqpoint{0.601553in}{2.088614in}}%
\pgfpathlineto{\pgfqpoint{0.611388in}{2.084174in}}%
\pgfpathlineto{\pgfqpoint{0.638032in}{2.088706in}}%
\pgfpathlineto{\pgfqpoint{0.653468in}{2.085611in}}%
\pgfpathlineto{\pgfqpoint{0.664077in}{2.080532in}}%
\pgfpathlineto{\pgfqpoint{0.669666in}{2.070891in}}%
\pgfpathlineto{\pgfqpoint{0.705053in}{2.072177in}}%
\pgfpathlineto{\pgfqpoint{0.710680in}{2.066278in}}%
\pgfpathlineto{\pgfqpoint{0.723956in}{2.065005in}}%
\pgfpathlineto{\pgfqpoint{0.737324in}{2.068774in}}%
\pgfpathlineto{\pgfqpoint{0.776851in}{2.067824in}}%
\pgfpathlineto{\pgfqpoint{0.785388in}{2.064919in}}%
\pgfpathlineto{\pgfqpoint{0.795364in}{2.068266in}}%
\pgfpathlineto{\pgfqpoint{0.905799in}{2.039449in}}%
\pgfpathlineto{\pgfqpoint{0.910299in}{2.023965in}}%
\pgfpathlineto{\pgfqpoint{0.920432in}{2.015576in}}%
\pgfpathlineto{\pgfqpoint{0.921484in}{2.000475in}}%
\pgfpathlineto{\pgfqpoint{0.906663in}{1.985185in}}%
\pgfpathlineto{\pgfqpoint{0.899437in}{1.970733in}}%
\pgfpathlineto{\pgfqpoint{0.890735in}{1.963012in}}%
\pgfpathlineto{\pgfqpoint{0.889947in}{1.955284in}}%
\pgfpathlineto{\pgfqpoint{0.875529in}{1.944840in}}%
\pgfpathlineto{\pgfqpoint{0.865637in}{1.929744in}}%
\pgfpathlineto{\pgfqpoint{0.860099in}{1.926077in}}%
\pgfpathlineto{\pgfqpoint{0.857078in}{1.911427in}}%
\pgfpathlineto{\pgfqpoint{0.871627in}{1.896508in}}%
\pgfpathlineto{\pgfqpoint{0.857576in}{1.871686in}}%
\pgfpathlineto{\pgfqpoint{0.855872in}{1.862213in}}%
\pgfpathlineto{\pgfqpoint{0.824275in}{1.732591in}}%
\pgfpathlineto{\pgfqpoint{0.757813in}{1.749621in}}%
\pgfpathlineto{\pgfqpoint{0.693698in}{1.766153in}}%
\pgfpathlineto{\pgfqpoint{0.655019in}{1.776916in}}%
\pgfpathlineto{\pgfqpoint{0.605321in}{1.791073in}}%
\pgfpathlineto{\pgfqpoint{0.526049in}{1.816024in}}%
\pgfpathlineto{\pgfqpoint{0.439875in}{1.842853in}}%
\pgfpathlineto{\pgfqpoint{0.418363in}{1.850356in}}%
\pgfusepath{stroke}%
\end{pgfscope}%
\begin{pgfscope}%
\pgfpathrectangle{\pgfqpoint{0.100000in}{0.100000in}}{\pgfqpoint{3.608454in}{2.310000in}}%
\pgfusepath{clip}%
\pgfsetbuttcap%
\pgfsetroundjoin%
\pgfsetlinewidth{0.010037pt}%
\definecolor{currentstroke}{rgb}{1.000000,1.000000,1.000000}%
\pgfsetstrokecolor{currentstroke}%
\pgfsetdash{}{0pt}%
\pgfpathmoveto{\pgfqpoint{3.405047in}{1.796199in}}%
\pgfpathlineto{\pgfqpoint{3.398770in}{1.801301in}}%
\pgfpathlineto{\pgfqpoint{3.399644in}{1.817200in}}%
\pgfpathlineto{\pgfqpoint{3.393157in}{1.862389in}}%
\pgfpathlineto{\pgfqpoint{3.398242in}{1.878280in}}%
\pgfpathlineto{\pgfqpoint{3.401874in}{1.906464in}}%
\pgfpathlineto{\pgfqpoint{3.396973in}{1.919797in}}%
\pgfpathlineto{\pgfqpoint{3.414804in}{1.937548in}}%
\pgfpathlineto{\pgfqpoint{3.419291in}{1.951669in}}%
\pgfpathlineto{\pgfqpoint{3.410842in}{1.962813in}}%
\pgfpathlineto{\pgfqpoint{3.414441in}{1.976741in}}%
\pgfpathlineto{\pgfqpoint{3.412185in}{1.985489in}}%
\pgfpathlineto{\pgfqpoint{3.414122in}{2.004225in}}%
\pgfpathlineto{\pgfqpoint{3.417744in}{2.010009in}}%
\pgfpathlineto{\pgfqpoint{3.428911in}{2.013660in}}%
\pgfpathlineto{\pgfqpoint{3.445201in}{1.965955in}}%
\pgfpathlineto{\pgfqpoint{3.475422in}{1.867100in}}%
\pgfpathlineto{\pgfqpoint{3.486858in}{1.859435in}}%
\pgfpathlineto{\pgfqpoint{3.488916in}{1.850686in}}%
\pgfpathlineto{\pgfqpoint{3.494952in}{1.847150in}}%
\pgfpathlineto{\pgfqpoint{3.494493in}{1.831282in}}%
\pgfpathlineto{\pgfqpoint{3.488098in}{1.831027in}}%
\pgfpathlineto{\pgfqpoint{3.475140in}{1.821145in}}%
\pgfpathlineto{\pgfqpoint{3.471404in}{1.811231in}}%
\pgfpathlineto{\pgfqpoint{3.436723in}{1.802651in}}%
\pgfpathlineto{\pgfqpoint{3.405047in}{1.796199in}}%
\pgfusepath{stroke}%
\end{pgfscope}%
\begin{pgfscope}%
\pgfpathrectangle{\pgfqpoint{0.100000in}{0.100000in}}{\pgfqpoint{3.608454in}{2.310000in}}%
\pgfusepath{clip}%
\pgfsetbuttcap%
\pgfsetroundjoin%
\pgfsetlinewidth{0.010037pt}%
\definecolor{currentstroke}{rgb}{1.000000,1.000000,1.000000}%
\pgfsetstrokecolor{currentstroke}%
\pgfsetdash{}{0pt}%
\pgfpathmoveto{\pgfqpoint{2.320622in}{1.452274in}}%
\pgfpathlineto{\pgfqpoint{2.302315in}{1.470405in}}%
\pgfpathlineto{\pgfqpoint{2.243792in}{1.467029in}}%
\pgfpathlineto{\pgfqpoint{2.152519in}{1.464021in}}%
\pgfpathlineto{\pgfqpoint{2.060697in}{1.465454in}}%
\pgfpathlineto{\pgfqpoint{2.054252in}{1.476692in}}%
\pgfpathlineto{\pgfqpoint{2.056883in}{1.487728in}}%
\pgfpathlineto{\pgfqpoint{2.055634in}{1.506632in}}%
\pgfpathlineto{\pgfqpoint{2.050781in}{1.524944in}}%
\pgfpathlineto{\pgfqpoint{2.051345in}{1.534613in}}%
\pgfpathlineto{\pgfqpoint{2.041030in}{1.545539in}}%
\pgfpathlineto{\pgfqpoint{2.043274in}{1.560741in}}%
\pgfpathlineto{\pgfqpoint{2.027451in}{1.590769in}}%
\pgfpathlineto{\pgfqpoint{2.022737in}{1.616262in}}%
\pgfpathlineto{\pgfqpoint{2.011355in}{1.636997in}}%
\pgfpathlineto{\pgfqpoint{2.022159in}{1.662720in}}%
\pgfpathlineto{\pgfqpoint{2.016778in}{1.674690in}}%
\pgfpathlineto{\pgfqpoint{2.022794in}{1.696330in}}%
\pgfpathlineto{\pgfqpoint{2.080050in}{1.695709in}}%
\pgfpathlineto{\pgfqpoint{2.138106in}{1.695621in}}%
\pgfpathlineto{\pgfqpoint{2.192987in}{1.696427in}}%
\pgfpathlineto{\pgfqpoint{2.253222in}{1.697687in}}%
\pgfpathlineto{\pgfqpoint{2.323886in}{1.700253in}}%
\pgfpathlineto{\pgfqpoint{2.325773in}{1.689994in}}%
\pgfpathlineto{\pgfqpoint{2.333596in}{1.681393in}}%
\pgfpathlineto{\pgfqpoint{2.328643in}{1.672430in}}%
\pgfpathlineto{\pgfqpoint{2.330673in}{1.654936in}}%
\pgfpathlineto{\pgfqpoint{2.335521in}{1.640795in}}%
\pgfpathlineto{\pgfqpoint{2.357502in}{1.633311in}}%
\pgfpathlineto{\pgfqpoint{2.361269in}{1.623298in}}%
\pgfpathlineto{\pgfqpoint{2.373331in}{1.612057in}}%
\pgfpathlineto{\pgfqpoint{2.378251in}{1.600427in}}%
\pgfpathlineto{\pgfqpoint{2.390473in}{1.592626in}}%
\pgfpathlineto{\pgfqpoint{2.392386in}{1.583234in}}%
\pgfpathlineto{\pgfqpoint{2.390006in}{1.569016in}}%
\pgfpathlineto{\pgfqpoint{2.383785in}{1.564755in}}%
\pgfpathlineto{\pgfqpoint{2.381909in}{1.551220in}}%
\pgfpathlineto{\pgfqpoint{2.364032in}{1.540471in}}%
\pgfpathlineto{\pgfqpoint{2.340730in}{1.534582in}}%
\pgfpathlineto{\pgfqpoint{2.338595in}{1.521042in}}%
\pgfpathlineto{\pgfqpoint{2.347584in}{1.511368in}}%
\pgfpathlineto{\pgfqpoint{2.347951in}{1.499205in}}%
\pgfpathlineto{\pgfqpoint{2.340698in}{1.489644in}}%
\pgfpathlineto{\pgfqpoint{2.336891in}{1.475440in}}%
\pgfpathlineto{\pgfqpoint{2.324285in}{1.470738in}}%
\pgfpathlineto{\pgfqpoint{2.325089in}{1.454910in}}%
\pgfpathlineto{\pgfqpoint{2.320622in}{1.452274in}}%
\pgfusepath{stroke}%
\end{pgfscope}%
\begin{pgfscope}%
\pgfpathrectangle{\pgfqpoint{0.100000in}{0.100000in}}{\pgfqpoint{3.608454in}{2.310000in}}%
\pgfusepath{clip}%
\pgfsetbuttcap%
\pgfsetroundjoin%
\pgfsetlinewidth{0.010037pt}%
\definecolor{currentstroke}{rgb}{1.000000,1.000000,1.000000}%
\pgfsetstrokecolor{currentstroke}%
\pgfsetdash{}{0pt}%
\pgfpathmoveto{\pgfqpoint{3.456549in}{1.750456in}}%
\pgfpathlineto{\pgfqpoint{3.439208in}{1.747720in}}%
\pgfpathlineto{\pgfqpoint{3.359557in}{1.729620in}}%
\pgfpathlineto{\pgfqpoint{3.358165in}{1.731729in}}%
\pgfpathlineto{\pgfqpoint{3.359367in}{1.786492in}}%
\pgfpathlineto{\pgfqpoint{3.405047in}{1.796199in}}%
\pgfpathlineto{\pgfqpoint{3.436723in}{1.802651in}}%
\pgfpathlineto{\pgfqpoint{3.471404in}{1.811231in}}%
\pgfpathlineto{\pgfqpoint{3.475140in}{1.821145in}}%
\pgfpathlineto{\pgfqpoint{3.488098in}{1.831027in}}%
\pgfpathlineto{\pgfqpoint{3.494493in}{1.831282in}}%
\pgfpathlineto{\pgfqpoint{3.502901in}{1.816865in}}%
\pgfpathlineto{\pgfqpoint{3.495271in}{1.795821in}}%
\pgfpathlineto{\pgfqpoint{3.494152in}{1.783481in}}%
\pgfpathlineto{\pgfqpoint{3.509594in}{1.784640in}}%
\pgfpathlineto{\pgfqpoint{3.516597in}{1.778721in}}%
\pgfpathlineto{\pgfqpoint{3.532302in}{1.754525in}}%
\pgfpathlineto{\pgfqpoint{3.540107in}{1.751582in}}%
\pgfpathlineto{\pgfqpoint{3.553182in}{1.752673in}}%
\pgfpathlineto{\pgfqpoint{3.562281in}{1.760896in}}%
\pgfpathlineto{\pgfqpoint{3.568329in}{1.753558in}}%
\pgfpathlineto{\pgfqpoint{3.530295in}{1.733487in}}%
\pgfpathlineto{\pgfqpoint{3.529106in}{1.747889in}}%
\pgfpathlineto{\pgfqpoint{3.511829in}{1.725505in}}%
\pgfpathlineto{\pgfqpoint{3.505754in}{1.721649in}}%
\pgfpathlineto{\pgfqpoint{3.497281in}{1.734565in}}%
\pgfpathlineto{\pgfqpoint{3.494962in}{1.736335in}}%
\pgfpathlineto{\pgfqpoint{3.487074in}{1.740513in}}%
\pgfpathlineto{\pgfqpoint{3.480196in}{1.757459in}}%
\pgfpathlineto{\pgfqpoint{3.456549in}{1.750456in}}%
\pgfusepath{stroke}%
\end{pgfscope}%
\begin{pgfscope}%
\pgfpathrectangle{\pgfqpoint{0.100000in}{0.100000in}}{\pgfqpoint{3.608454in}{2.310000in}}%
\pgfusepath{clip}%
\pgfsetbuttcap%
\pgfsetroundjoin%
\pgfsetlinewidth{0.010037pt}%
\definecolor{currentstroke}{rgb}{1.000000,1.000000,1.000000}%
\pgfsetstrokecolor{currentstroke}%
\pgfsetdash{}{0pt}%
\pgfpathmoveto{\pgfqpoint{1.680190in}{1.435164in}}%
\pgfpathlineto{\pgfqpoint{1.686540in}{1.513906in}}%
\pgfpathlineto{\pgfqpoint{1.650576in}{1.516826in}}%
\pgfpathlineto{\pgfqpoint{1.567441in}{1.524892in}}%
\pgfpathlineto{\pgfqpoint{1.575175in}{1.603355in}}%
\pgfpathlineto{\pgfqpoint{1.582921in}{1.682417in}}%
\pgfpathlineto{\pgfqpoint{1.614880in}{1.678966in}}%
\pgfpathlineto{\pgfqpoint{1.696503in}{1.671254in}}%
\pgfpathlineto{\pgfqpoint{1.745900in}{1.667386in}}%
\pgfpathlineto{\pgfqpoint{1.805703in}{1.663958in}}%
\pgfpathlineto{\pgfqpoint{1.904136in}{1.659299in}}%
\pgfpathlineto{\pgfqpoint{1.906300in}{1.654886in}}%
\pgfpathlineto{\pgfqpoint{1.934505in}{1.640932in}}%
\pgfpathlineto{\pgfqpoint{1.943154in}{1.648019in}}%
\pgfpathlineto{\pgfqpoint{1.968073in}{1.647372in}}%
\pgfpathlineto{\pgfqpoint{1.982922in}{1.639593in}}%
\pgfpathlineto{\pgfqpoint{2.006469in}{1.630750in}}%
\pgfpathlineto{\pgfqpoint{2.013067in}{1.617849in}}%
\pgfpathlineto{\pgfqpoint{2.022737in}{1.616262in}}%
\pgfpathlineto{\pgfqpoint{2.027451in}{1.590769in}}%
\pgfpathlineto{\pgfqpoint{2.043274in}{1.560741in}}%
\pgfpathlineto{\pgfqpoint{2.041030in}{1.545539in}}%
\pgfpathlineto{\pgfqpoint{2.051345in}{1.534613in}}%
\pgfpathlineto{\pgfqpoint{2.050781in}{1.524944in}}%
\pgfpathlineto{\pgfqpoint{2.055634in}{1.506632in}}%
\pgfpathlineto{\pgfqpoint{2.056883in}{1.487728in}}%
\pgfpathlineto{\pgfqpoint{2.054252in}{1.476692in}}%
\pgfpathlineto{\pgfqpoint{2.060697in}{1.465454in}}%
\pgfpathlineto{\pgfqpoint{2.069537in}{1.445017in}}%
\pgfpathlineto{\pgfqpoint{2.077998in}{1.436700in}}%
\pgfpathlineto{\pgfqpoint{2.088084in}{1.418676in}}%
\pgfpathlineto{\pgfqpoint{2.059498in}{1.418379in}}%
\pgfpathlineto{\pgfqpoint{1.997690in}{1.419335in}}%
\pgfpathlineto{\pgfqpoint{1.929402in}{1.421428in}}%
\pgfpathlineto{\pgfqpoint{1.860711in}{1.424063in}}%
\pgfpathlineto{\pgfqpoint{1.759716in}{1.429580in}}%
\pgfpathlineto{\pgfqpoint{1.680190in}{1.435164in}}%
\pgfusepath{stroke}%
\end{pgfscope}%
\begin{pgfscope}%
\pgfpathrectangle{\pgfqpoint{0.100000in}{0.100000in}}{\pgfqpoint{3.608454in}{2.310000in}}%
\pgfusepath{clip}%
\pgfsetbuttcap%
\pgfsetroundjoin%
\pgfsetlinewidth{0.010037pt}%
\definecolor{currentstroke}{rgb}{1.000000,1.000000,1.000000}%
\pgfsetstrokecolor{currentstroke}%
\pgfsetdash{}{0pt}%
\pgfpathmoveto{\pgfqpoint{2.995203in}{1.672680in}}%
\pgfpathlineto{\pgfqpoint{3.026828in}{1.702834in}}%
\pgfpathlineto{\pgfqpoint{3.030759in}{1.713553in}}%
\pgfpathlineto{\pgfqpoint{3.040025in}{1.722729in}}%
\pgfpathlineto{\pgfqpoint{3.033072in}{1.736089in}}%
\pgfpathlineto{\pgfqpoint{3.024354in}{1.743903in}}%
\pgfpathlineto{\pgfqpoint{3.021834in}{1.757751in}}%
\pgfpathlineto{\pgfqpoint{3.054283in}{1.771969in}}%
\pgfpathlineto{\pgfqpoint{3.081149in}{1.776467in}}%
\pgfpathlineto{\pgfqpoint{3.095585in}{1.776768in}}%
\pgfpathlineto{\pgfqpoint{3.106582in}{1.771334in}}%
\pgfpathlineto{\pgfqpoint{3.117302in}{1.776186in}}%
\pgfpathlineto{\pgfqpoint{3.143447in}{1.781623in}}%
\pgfpathlineto{\pgfqpoint{3.152481in}{1.788641in}}%
\pgfpathlineto{\pgfqpoint{3.165877in}{1.804177in}}%
\pgfpathlineto{\pgfqpoint{3.178063in}{1.811042in}}%
\pgfpathlineto{\pgfqpoint{3.178922in}{1.817626in}}%
\pgfpathlineto{\pgfqpoint{3.172527in}{1.832648in}}%
\pgfpathlineto{\pgfqpoint{3.177150in}{1.841488in}}%
\pgfpathlineto{\pgfqpoint{3.170925in}{1.850993in}}%
\pgfpathlineto{\pgfqpoint{3.161391in}{1.851657in}}%
\pgfpathlineto{\pgfqpoint{3.185252in}{1.880369in}}%
\pgfpathlineto{\pgfqpoint{3.188086in}{1.891311in}}%
\pgfpathlineto{\pgfqpoint{3.206865in}{1.919218in}}%
\pgfpathlineto{\pgfqpoint{3.224297in}{1.934319in}}%
\pgfpathlineto{\pgfqpoint{3.236268in}{1.940625in}}%
\pgfpathlineto{\pgfqpoint{3.275437in}{1.949500in}}%
\pgfpathlineto{\pgfqpoint{3.312149in}{1.959523in}}%
\pgfpathlineto{\pgfqpoint{3.317011in}{1.943790in}}%
\pgfpathlineto{\pgfqpoint{3.317356in}{1.929202in}}%
\pgfpathlineto{\pgfqpoint{3.325730in}{1.915960in}}%
\pgfpathlineto{\pgfqpoint{3.328311in}{1.902477in}}%
\pgfpathlineto{\pgfqpoint{3.325093in}{1.884184in}}%
\pgfpathlineto{\pgfqpoint{3.334894in}{1.859510in}}%
\pgfpathlineto{\pgfqpoint{3.346067in}{1.845790in}}%
\pgfpathlineto{\pgfqpoint{3.354710in}{1.801193in}}%
\pgfpathlineto{\pgfqpoint{3.359367in}{1.786492in}}%
\pgfpathlineto{\pgfqpoint{3.358165in}{1.731729in}}%
\pgfpathlineto{\pgfqpoint{3.359557in}{1.729620in}}%
\pgfpathlineto{\pgfqpoint{3.369731in}{1.670722in}}%
\pgfpathlineto{\pgfqpoint{3.375437in}{1.665355in}}%
\pgfpathlineto{\pgfqpoint{3.363164in}{1.653431in}}%
\pgfpathlineto{\pgfqpoint{3.369219in}{1.646587in}}%
\pgfpathlineto{\pgfqpoint{3.363897in}{1.636235in}}%
\pgfpathlineto{\pgfqpoint{3.363942in}{1.631828in}}%
\pgfpathlineto{\pgfqpoint{3.357288in}{1.627852in}}%
\pgfpathlineto{\pgfqpoint{3.354047in}{1.619061in}}%
\pgfpathlineto{\pgfqpoint{3.355075in}{1.643233in}}%
\pgfpathlineto{\pgfqpoint{3.334450in}{1.648555in}}%
\pgfpathlineto{\pgfqpoint{3.302196in}{1.659557in}}%
\pgfpathlineto{\pgfqpoint{3.297634in}{1.664976in}}%
\pgfpathlineto{\pgfqpoint{3.284155in}{1.666275in}}%
\pgfpathlineto{\pgfqpoint{3.276402in}{1.674341in}}%
\pgfpathlineto{\pgfqpoint{3.272219in}{1.690405in}}%
\pgfpathlineto{\pgfqpoint{3.261228in}{1.692404in}}%
\pgfpathlineto{\pgfqpoint{3.253878in}{1.700831in}}%
\pgfpathlineto{\pgfqpoint{3.160304in}{1.681966in}}%
\pgfpathlineto{\pgfqpoint{3.115582in}{1.672717in}}%
\pgfpathlineto{\pgfqpoint{3.047681in}{1.660324in}}%
\pgfpathlineto{\pgfqpoint{2.998785in}{1.652075in}}%
\pgfpathlineto{\pgfqpoint{2.995203in}{1.672680in}}%
\pgfusepath{stroke}%
\end{pgfscope}%
\begin{pgfscope}%
\pgfpathrectangle{\pgfqpoint{0.100000in}{0.100000in}}{\pgfqpoint{3.608454in}{2.310000in}}%
\pgfusepath{clip}%
\pgfsetbuttcap%
\pgfsetroundjoin%
\pgfsetlinewidth{0.010037pt}%
\definecolor{currentstroke}{rgb}{1.000000,1.000000,1.000000}%
\pgfsetstrokecolor{currentstroke}%
\pgfsetdash{}{0pt}%
\pgfpathmoveto{\pgfqpoint{3.370867in}{1.614152in}}%
\pgfpathlineto{\pgfqpoint{3.356395in}{1.609644in}}%
\pgfpathlineto{\pgfqpoint{3.353960in}{1.613788in}}%
\pgfpathlineto{\pgfqpoint{3.358606in}{1.627701in}}%
\pgfpathlineto{\pgfqpoint{3.367208in}{1.629071in}}%
\pgfpathlineto{\pgfqpoint{3.366438in}{1.633451in}}%
\pgfpathlineto{\pgfqpoint{3.374162in}{1.640026in}}%
\pgfpathlineto{\pgfqpoint{3.396494in}{1.645263in}}%
\pgfpathlineto{\pgfqpoint{3.406435in}{1.653192in}}%
\pgfpathlineto{\pgfqpoint{3.428768in}{1.659799in}}%
\pgfpathlineto{\pgfqpoint{3.438954in}{1.657382in}}%
\pgfpathlineto{\pgfqpoint{3.447523in}{1.668034in}}%
\pgfpathlineto{\pgfqpoint{3.460473in}{1.669442in}}%
\pgfpathlineto{\pgfqpoint{3.438379in}{1.648663in}}%
\pgfpathlineto{\pgfqpoint{3.370867in}{1.614152in}}%
\pgfusepath{stroke}%
\end{pgfscope}%
\begin{pgfscope}%
\pgfpathrectangle{\pgfqpoint{0.100000in}{0.100000in}}{\pgfqpoint{3.608454in}{2.310000in}}%
\pgfusepath{clip}%
\pgfsetbuttcap%
\pgfsetroundjoin%
\pgfsetlinewidth{0.010037pt}%
\definecolor{currentstroke}{rgb}{1.000000,1.000000,1.000000}%
\pgfsetstrokecolor{currentstroke}%
\pgfsetdash{}{0pt}%
\pgfpathmoveto{\pgfqpoint{3.045737in}{1.476948in}}%
\pgfpathlineto{\pgfqpoint{2.983156in}{1.466600in}}%
\pgfpathlineto{\pgfqpoint{2.971804in}{1.538119in}}%
\pgfpathlineto{\pgfqpoint{2.954949in}{1.643541in}}%
\pgfpathlineto{\pgfqpoint{2.995203in}{1.672680in}}%
\pgfpathlineto{\pgfqpoint{2.998785in}{1.652075in}}%
\pgfpathlineto{\pgfqpoint{3.047681in}{1.660324in}}%
\pgfpathlineto{\pgfqpoint{3.115582in}{1.672717in}}%
\pgfpathlineto{\pgfqpoint{3.160304in}{1.681966in}}%
\pgfpathlineto{\pgfqpoint{3.253878in}{1.700831in}}%
\pgfpathlineto{\pgfqpoint{3.261228in}{1.692404in}}%
\pgfpathlineto{\pgfqpoint{3.272219in}{1.690405in}}%
\pgfpathlineto{\pgfqpoint{3.276402in}{1.674341in}}%
\pgfpathlineto{\pgfqpoint{3.284155in}{1.666275in}}%
\pgfpathlineto{\pgfqpoint{3.297634in}{1.664976in}}%
\pgfpathlineto{\pgfqpoint{3.302196in}{1.659557in}}%
\pgfpathlineto{\pgfqpoint{3.297564in}{1.655376in}}%
\pgfpathlineto{\pgfqpoint{3.293392in}{1.640582in}}%
\pgfpathlineto{\pgfqpoint{3.283615in}{1.623960in}}%
\pgfpathlineto{\pgfqpoint{3.290178in}{1.616731in}}%
\pgfpathlineto{\pgfqpoint{3.284807in}{1.608986in}}%
\pgfpathlineto{\pgfqpoint{3.287837in}{1.592001in}}%
\pgfpathlineto{\pgfqpoint{3.297508in}{1.583085in}}%
\pgfpathlineto{\pgfqpoint{3.320457in}{1.568600in}}%
\pgfpathlineto{\pgfqpoint{3.301976in}{1.548162in}}%
\pgfpathlineto{\pgfqpoint{3.301717in}{1.540404in}}%
\pgfpathlineto{\pgfqpoint{3.286671in}{1.530400in}}%
\pgfpathlineto{\pgfqpoint{3.270035in}{1.528522in}}%
\pgfpathlineto{\pgfqpoint{3.265920in}{1.519827in}}%
\pgfpathlineto{\pgfqpoint{3.219645in}{1.509814in}}%
\pgfpathlineto{\pgfqpoint{3.165648in}{1.498939in}}%
\pgfpathlineto{\pgfqpoint{3.128532in}{1.492293in}}%
\pgfpathlineto{\pgfqpoint{3.045737in}{1.476948in}}%
\pgfusepath{stroke}%
\end{pgfscope}%
\begin{pgfscope}%
\pgfpathrectangle{\pgfqpoint{0.100000in}{0.100000in}}{\pgfqpoint{3.608454in}{2.310000in}}%
\pgfusepath{clip}%
\pgfsetbuttcap%
\pgfsetroundjoin%
\pgfsetlinewidth{0.010037pt}%
\definecolor{currentstroke}{rgb}{1.000000,1.000000,1.000000}%
\pgfsetstrokecolor{currentstroke}%
\pgfsetdash{}{0pt}%
\pgfpathmoveto{\pgfqpoint{3.359557in}{1.729620in}}%
\pgfpathlineto{\pgfqpoint{3.439208in}{1.747720in}}%
\pgfpathlineto{\pgfqpoint{3.456549in}{1.750456in}}%
\pgfpathlineto{\pgfqpoint{3.468026in}{1.705331in}}%
\pgfpathlineto{\pgfqpoint{3.466209in}{1.697251in}}%
\pgfpathlineto{\pgfqpoint{3.429341in}{1.682977in}}%
\pgfpathlineto{\pgfqpoint{3.407332in}{1.677989in}}%
\pgfpathlineto{\pgfqpoint{3.397980in}{1.666798in}}%
\pgfpathlineto{\pgfqpoint{3.369219in}{1.646587in}}%
\pgfpathlineto{\pgfqpoint{3.363164in}{1.653431in}}%
\pgfpathlineto{\pgfqpoint{3.375437in}{1.665355in}}%
\pgfpathlineto{\pgfqpoint{3.369731in}{1.670722in}}%
\pgfpathlineto{\pgfqpoint{3.359557in}{1.729620in}}%
\pgfusepath{stroke}%
\end{pgfscope}%
\begin{pgfscope}%
\pgfpathrectangle{\pgfqpoint{0.100000in}{0.100000in}}{\pgfqpoint{3.608454in}{2.310000in}}%
\pgfusepath{clip}%
\pgfsetbuttcap%
\pgfsetroundjoin%
\pgfsetlinewidth{0.010037pt}%
\definecolor{currentstroke}{rgb}{1.000000,1.000000,1.000000}%
\pgfsetstrokecolor{currentstroke}%
\pgfsetdash{}{0pt}%
\pgfpathmoveto{\pgfqpoint{3.456549in}{1.750456in}}%
\pgfpathlineto{\pgfqpoint{3.480196in}{1.757459in}}%
\pgfpathlineto{\pgfqpoint{3.487074in}{1.740513in}}%
\pgfpathlineto{\pgfqpoint{3.494962in}{1.736335in}}%
\pgfpathlineto{\pgfqpoint{3.485228in}{1.729197in}}%
\pgfpathlineto{\pgfqpoint{3.486456in}{1.708233in}}%
\pgfpathlineto{\pgfqpoint{3.466209in}{1.697251in}}%
\pgfpathlineto{\pgfqpoint{3.468026in}{1.705331in}}%
\pgfpathlineto{\pgfqpoint{3.456549in}{1.750456in}}%
\pgfusepath{stroke}%
\end{pgfscope}%
\begin{pgfscope}%
\pgfpathrectangle{\pgfqpoint{0.100000in}{0.100000in}}{\pgfqpoint{3.608454in}{2.310000in}}%
\pgfusepath{clip}%
\pgfsetbuttcap%
\pgfsetroundjoin%
\pgfsetlinewidth{0.010037pt}%
\definecolor{currentstroke}{rgb}{1.000000,1.000000,1.000000}%
\pgfsetstrokecolor{currentstroke}%
\pgfsetdash{}{0pt}%
\pgfpathmoveto{\pgfqpoint{3.284005in}{1.523011in}}%
\pgfpathlineto{\pgfqpoint{3.286671in}{1.530400in}}%
\pgfpathlineto{\pgfqpoint{3.301717in}{1.540404in}}%
\pgfpathlineto{\pgfqpoint{3.301976in}{1.548162in}}%
\pgfpathlineto{\pgfqpoint{3.320457in}{1.568600in}}%
\pgfpathlineto{\pgfqpoint{3.297508in}{1.583085in}}%
\pgfpathlineto{\pgfqpoint{3.287837in}{1.592001in}}%
\pgfpathlineto{\pgfqpoint{3.284807in}{1.608986in}}%
\pgfpathlineto{\pgfqpoint{3.290178in}{1.616731in}}%
\pgfpathlineto{\pgfqpoint{3.283615in}{1.623960in}}%
\pgfpathlineto{\pgfqpoint{3.293392in}{1.640582in}}%
\pgfpathlineto{\pgfqpoint{3.297564in}{1.655376in}}%
\pgfpathlineto{\pgfqpoint{3.302196in}{1.659557in}}%
\pgfpathlineto{\pgfqpoint{3.334450in}{1.648555in}}%
\pgfpathlineto{\pgfqpoint{3.355075in}{1.643233in}}%
\pgfpathlineto{\pgfqpoint{3.354047in}{1.619061in}}%
\pgfpathlineto{\pgfqpoint{3.341519in}{1.600731in}}%
\pgfpathlineto{\pgfqpoint{3.351842in}{1.598035in}}%
\pgfpathlineto{\pgfqpoint{3.362561in}{1.590168in}}%
\pgfpathlineto{\pgfqpoint{3.363194in}{1.568654in}}%
\pgfpathlineto{\pgfqpoint{3.359949in}{1.553423in}}%
\pgfpathlineto{\pgfqpoint{3.362108in}{1.540913in}}%
\pgfpathlineto{\pgfqpoint{3.343475in}{1.498697in}}%
\pgfpathlineto{\pgfqpoint{3.336779in}{1.478984in}}%
\pgfpathlineto{\pgfqpoint{3.327429in}{1.488566in}}%
\pgfpathlineto{\pgfqpoint{3.315030in}{1.486933in}}%
\pgfpathlineto{\pgfqpoint{3.284000in}{1.504862in}}%
\pgfpathlineto{\pgfqpoint{3.280825in}{1.514462in}}%
\pgfpathlineto{\pgfqpoint{3.284005in}{1.523011in}}%
\pgfusepath{stroke}%
\end{pgfscope}%
\begin{pgfscope}%
\pgfpathrectangle{\pgfqpoint{0.100000in}{0.100000in}}{\pgfqpoint{3.608454in}{2.310000in}}%
\pgfusepath{clip}%
\pgfsetbuttcap%
\pgfsetroundjoin%
\pgfsetlinewidth{0.010037pt}%
\definecolor{currentstroke}{rgb}{1.000000,1.000000,1.000000}%
\pgfsetstrokecolor{currentstroke}%
\pgfsetdash{}{0pt}%
\pgfpathmoveto{\pgfqpoint{2.541376in}{1.261906in}}%
\pgfpathlineto{\pgfqpoint{2.539926in}{1.281490in}}%
\pgfpathlineto{\pgfqpoint{2.545564in}{1.291207in}}%
\pgfpathlineto{\pgfqpoint{2.541848in}{1.295967in}}%
\pgfpathlineto{\pgfqpoint{2.555400in}{1.311555in}}%
\pgfpathlineto{\pgfqpoint{2.554624in}{1.315312in}}%
\pgfpathlineto{\pgfqpoint{2.567791in}{1.342507in}}%
\pgfpathlineto{\pgfqpoint{2.565129in}{1.355623in}}%
\pgfpathlineto{\pgfqpoint{2.555581in}{1.369354in}}%
\pgfpathlineto{\pgfqpoint{2.562209in}{1.386059in}}%
\pgfpathlineto{\pgfqpoint{2.553618in}{1.495995in}}%
\pgfpathlineto{\pgfqpoint{2.547432in}{1.573118in}}%
\pgfpathlineto{\pgfqpoint{2.555978in}{1.566728in}}%
\pgfpathlineto{\pgfqpoint{2.565512in}{1.566900in}}%
\pgfpathlineto{\pgfqpoint{2.588031in}{1.579945in}}%
\pgfpathlineto{\pgfqpoint{2.657102in}{1.586389in}}%
\pgfpathlineto{\pgfqpoint{2.708226in}{1.591778in}}%
\pgfpathlineto{\pgfqpoint{2.708677in}{1.586775in}}%
\pgfpathlineto{\pgfqpoint{2.720348in}{1.481089in}}%
\pgfpathlineto{\pgfqpoint{2.730393in}{1.382751in}}%
\pgfpathlineto{\pgfqpoint{2.726063in}{1.378135in}}%
\pgfpathlineto{\pgfqpoint{2.732714in}{1.366675in}}%
\pgfpathlineto{\pgfqpoint{2.732680in}{1.358417in}}%
\pgfpathlineto{\pgfqpoint{2.723189in}{1.356342in}}%
\pgfpathlineto{\pgfqpoint{2.712569in}{1.348381in}}%
\pgfpathlineto{\pgfqpoint{2.705378in}{1.351515in}}%
\pgfpathlineto{\pgfqpoint{2.694616in}{1.346419in}}%
\pgfpathlineto{\pgfqpoint{2.697946in}{1.336184in}}%
\pgfpathlineto{\pgfqpoint{2.686884in}{1.325901in}}%
\pgfpathlineto{\pgfqpoint{2.683829in}{1.314006in}}%
\pgfpathlineto{\pgfqpoint{2.676223in}{1.312054in}}%
\pgfpathlineto{\pgfqpoint{2.670547in}{1.303046in}}%
\pgfpathlineto{\pgfqpoint{2.671290in}{1.293974in}}%
\pgfpathlineto{\pgfqpoint{2.664612in}{1.287594in}}%
\pgfpathlineto{\pgfqpoint{2.654564in}{1.288579in}}%
\pgfpathlineto{\pgfqpoint{2.646926in}{1.298367in}}%
\pgfpathlineto{\pgfqpoint{2.633983in}{1.288941in}}%
\pgfpathlineto{\pgfqpoint{2.634892in}{1.280705in}}%
\pgfpathlineto{\pgfqpoint{2.620497in}{1.275892in}}%
\pgfpathlineto{\pgfqpoint{2.616881in}{1.281949in}}%
\pgfpathlineto{\pgfqpoint{2.602868in}{1.275081in}}%
\pgfpathlineto{\pgfqpoint{2.597748in}{1.265248in}}%
\pgfpathlineto{\pgfqpoint{2.580872in}{1.275331in}}%
\pgfpathlineto{\pgfqpoint{2.554124in}{1.268651in}}%
\pgfpathlineto{\pgfqpoint{2.541376in}{1.261906in}}%
\pgfusepath{stroke}%
\end{pgfscope}%
\begin{pgfscope}%
\pgfpathrectangle{\pgfqpoint{0.100000in}{0.100000in}}{\pgfqpoint{3.608454in}{2.310000in}}%
\pgfusepath{clip}%
\pgfsetbuttcap%
\pgfsetroundjoin%
\pgfsetlinewidth{0.010037pt}%
\definecolor{currentstroke}{rgb}{1.000000,1.000000,1.000000}%
\pgfsetstrokecolor{currentstroke}%
\pgfsetdash{}{0pt}%
\pgfpathmoveto{\pgfqpoint{0.655019in}{1.776916in}}%
\pgfpathlineto{\pgfqpoint{0.693698in}{1.766153in}}%
\pgfpathlineto{\pgfqpoint{0.757813in}{1.749621in}}%
\pgfpathlineto{\pgfqpoint{0.824275in}{1.732591in}}%
\pgfpathlineto{\pgfqpoint{0.885533in}{1.717818in}}%
\pgfpathlineto{\pgfqpoint{0.938464in}{1.705834in}}%
\pgfpathlineto{\pgfqpoint{0.995094in}{1.693381in}}%
\pgfpathlineto{\pgfqpoint{0.978732in}{1.616166in}}%
\pgfpathlineto{\pgfqpoint{0.964156in}{1.547595in}}%
\pgfpathlineto{\pgfqpoint{0.940244in}{1.436956in}}%
\pgfpathlineto{\pgfqpoint{0.922291in}{1.353447in}}%
\pgfpathlineto{\pgfqpoint{0.912593in}{1.306845in}}%
\pgfpathlineto{\pgfqpoint{0.900156in}{1.246360in}}%
\pgfpathlineto{\pgfqpoint{0.891552in}{1.234089in}}%
\pgfpathlineto{\pgfqpoint{0.884626in}{1.233679in}}%
\pgfpathlineto{\pgfqpoint{0.879687in}{1.244390in}}%
\pgfpathlineto{\pgfqpoint{0.868345in}{1.248280in}}%
\pgfpathlineto{\pgfqpoint{0.856124in}{1.246917in}}%
\pgfpathlineto{\pgfqpoint{0.852655in}{1.238156in}}%
\pgfpathlineto{\pgfqpoint{0.852293in}{1.207731in}}%
\pgfpathlineto{\pgfqpoint{0.848633in}{1.200785in}}%
\pgfpathlineto{\pgfqpoint{0.850725in}{1.176360in}}%
\pgfpathlineto{\pgfqpoint{0.843082in}{1.160088in}}%
\pgfpathlineto{\pgfqpoint{0.781106in}{1.255521in}}%
\pgfpathlineto{\pgfqpoint{0.720071in}{1.348255in}}%
\pgfpathlineto{\pgfqpoint{0.688300in}{1.396883in}}%
\pgfpathlineto{\pgfqpoint{0.661796in}{1.438856in}}%
\pgfpathlineto{\pgfqpoint{0.628400in}{1.490776in}}%
\pgfpathlineto{\pgfqpoint{0.590596in}{1.549016in}}%
\pgfpathlineto{\pgfqpoint{0.606139in}{1.604297in}}%
\pgfpathlineto{\pgfqpoint{0.637416in}{1.715162in}}%
\pgfpathlineto{\pgfqpoint{0.655019in}{1.776916in}}%
\pgfusepath{stroke}%
\end{pgfscope}%
\begin{pgfscope}%
\pgfpathrectangle{\pgfqpoint{0.100000in}{0.100000in}}{\pgfqpoint{3.608454in}{2.310000in}}%
\pgfusepath{clip}%
\pgfsetbuttcap%
\pgfsetroundjoin%
\pgfsetlinewidth{0.010037pt}%
\definecolor{currentstroke}{rgb}{1.000000,1.000000,1.000000}%
\pgfsetstrokecolor{currentstroke}%
\pgfsetdash{}{0pt}%
\pgfpathmoveto{\pgfqpoint{0.995094in}{1.693381in}}%
\pgfpathlineto{\pgfqpoint{1.055637in}{1.681339in}}%
\pgfpathlineto{\pgfqpoint{1.167721in}{1.659701in}}%
\pgfpathlineto{\pgfqpoint{1.153691in}{1.581874in}}%
\pgfpathlineto{\pgfqpoint{1.215397in}{1.571309in}}%
\pgfpathlineto{\pgfqpoint{1.271614in}{1.562313in}}%
\pgfpathlineto{\pgfqpoint{1.261846in}{1.500784in}}%
\pgfpathlineto{\pgfqpoint{1.251495in}{1.434433in}}%
\pgfpathlineto{\pgfqpoint{1.237637in}{1.347314in}}%
\pgfpathlineto{\pgfqpoint{1.237278in}{1.340011in}}%
\pgfpathlineto{\pgfqpoint{1.222893in}{1.249723in}}%
\pgfpathlineto{\pgfqpoint{1.163676in}{1.258901in}}%
\pgfpathlineto{\pgfqpoint{1.133501in}{1.264961in}}%
\pgfpathlineto{\pgfqpoint{1.024434in}{1.284134in}}%
\pgfpathlineto{\pgfqpoint{0.983361in}{1.292233in}}%
\pgfpathlineto{\pgfqpoint{0.912593in}{1.306845in}}%
\pgfpathlineto{\pgfqpoint{0.922291in}{1.353447in}}%
\pgfpathlineto{\pgfqpoint{0.940244in}{1.436956in}}%
\pgfpathlineto{\pgfqpoint{0.964156in}{1.547595in}}%
\pgfpathlineto{\pgfqpoint{0.978732in}{1.616166in}}%
\pgfpathlineto{\pgfqpoint{0.995094in}{1.693381in}}%
\pgfusepath{stroke}%
\end{pgfscope}%
\begin{pgfscope}%
\pgfpathrectangle{\pgfqpoint{0.100000in}{0.100000in}}{\pgfqpoint{3.608454in}{2.310000in}}%
\pgfusepath{clip}%
\pgfsetbuttcap%
\pgfsetroundjoin%
\pgfsetlinewidth{0.010037pt}%
\definecolor{currentstroke}{rgb}{1.000000,1.000000,1.000000}%
\pgfsetstrokecolor{currentstroke}%
\pgfsetdash{}{0pt}%
\pgfpathmoveto{\pgfqpoint{0.418363in}{1.850356in}}%
\pgfpathlineto{\pgfqpoint{0.439875in}{1.842853in}}%
\pgfpathlineto{\pgfqpoint{0.526049in}{1.816024in}}%
\pgfpathlineto{\pgfqpoint{0.605321in}{1.791073in}}%
\pgfpathlineto{\pgfqpoint{0.655019in}{1.776916in}}%
\pgfpathlineto{\pgfqpoint{0.637416in}{1.715162in}}%
\pgfpathlineto{\pgfqpoint{0.606139in}{1.604297in}}%
\pgfpathlineto{\pgfqpoint{0.590596in}{1.549016in}}%
\pgfpathlineto{\pgfqpoint{0.628400in}{1.490776in}}%
\pgfpathlineto{\pgfqpoint{0.661796in}{1.438856in}}%
\pgfpathlineto{\pgfqpoint{0.688300in}{1.396883in}}%
\pgfpathlineto{\pgfqpoint{0.720071in}{1.348255in}}%
\pgfpathlineto{\pgfqpoint{0.781106in}{1.255521in}}%
\pgfpathlineto{\pgfqpoint{0.843082in}{1.160088in}}%
\pgfpathlineto{\pgfqpoint{0.840594in}{1.150624in}}%
\pgfpathlineto{\pgfqpoint{0.848070in}{1.135552in}}%
\pgfpathlineto{\pgfqpoint{0.849548in}{1.114934in}}%
\pgfpathlineto{\pgfqpoint{0.860405in}{1.104932in}}%
\pgfpathlineto{\pgfqpoint{0.860834in}{1.096874in}}%
\pgfpathlineto{\pgfqpoint{0.841395in}{1.087727in}}%
\pgfpathlineto{\pgfqpoint{0.832133in}{1.078570in}}%
\pgfpathlineto{\pgfqpoint{0.829245in}{1.058349in}}%
\pgfpathlineto{\pgfqpoint{0.824557in}{1.047321in}}%
\pgfpathlineto{\pgfqpoint{0.814680in}{1.038025in}}%
\pgfpathlineto{\pgfqpoint{0.804856in}{1.013854in}}%
\pgfpathlineto{\pgfqpoint{0.818669in}{1.001263in}}%
\pgfpathlineto{\pgfqpoint{0.816886in}{0.990883in}}%
\pgfpathlineto{\pgfqpoint{0.805560in}{0.983661in}}%
\pgfpathlineto{\pgfqpoint{0.797751in}{0.984935in}}%
\pgfpathlineto{\pgfqpoint{0.705575in}{0.997727in}}%
\pgfpathlineto{\pgfqpoint{0.637495in}{1.007385in}}%
\pgfpathlineto{\pgfqpoint{0.640472in}{1.018376in}}%
\pgfpathlineto{\pgfqpoint{0.636040in}{1.036625in}}%
\pgfpathlineto{\pgfqpoint{0.635593in}{1.055032in}}%
\pgfpathlineto{\pgfqpoint{0.632697in}{1.065804in}}%
\pgfpathlineto{\pgfqpoint{0.623775in}{1.081205in}}%
\pgfpathlineto{\pgfqpoint{0.598111in}{1.116738in}}%
\pgfpathlineto{\pgfqpoint{0.589705in}{1.121092in}}%
\pgfpathlineto{\pgfqpoint{0.578875in}{1.121023in}}%
\pgfpathlineto{\pgfqpoint{0.581349in}{1.132233in}}%
\pgfpathlineto{\pgfqpoint{0.576221in}{1.146250in}}%
\pgfpathlineto{\pgfqpoint{0.551032in}{1.153217in}}%
\pgfpathlineto{\pgfqpoint{0.535677in}{1.166118in}}%
\pgfpathlineto{\pgfqpoint{0.534404in}{1.174011in}}%
\pgfpathlineto{\pgfqpoint{0.516724in}{1.193554in}}%
\pgfpathlineto{\pgfqpoint{0.499894in}{1.197294in}}%
\pgfpathlineto{\pgfqpoint{0.484301in}{1.207225in}}%
\pgfpathlineto{\pgfqpoint{0.463793in}{1.210661in}}%
\pgfpathlineto{\pgfqpoint{0.455031in}{1.223907in}}%
\pgfpathlineto{\pgfqpoint{0.463352in}{1.244873in}}%
\pgfpathlineto{\pgfqpoint{0.460807in}{1.249588in}}%
\pgfpathlineto{\pgfqpoint{0.467737in}{1.267066in}}%
\pgfpathlineto{\pgfqpoint{0.455409in}{1.276367in}}%
\pgfpathlineto{\pgfqpoint{0.459410in}{1.293230in}}%
\pgfpathlineto{\pgfqpoint{0.452825in}{1.297541in}}%
\pgfpathlineto{\pgfqpoint{0.447140in}{1.313479in}}%
\pgfpathlineto{\pgfqpoint{0.440280in}{1.318340in}}%
\pgfpathlineto{\pgfqpoint{0.439763in}{1.329853in}}%
\pgfpathlineto{\pgfqpoint{0.434389in}{1.337971in}}%
\pgfpathlineto{\pgfqpoint{0.426289in}{1.365312in}}%
\pgfpathlineto{\pgfqpoint{0.417427in}{1.378407in}}%
\pgfpathlineto{\pgfqpoint{0.419354in}{1.400637in}}%
\pgfpathlineto{\pgfqpoint{0.429775in}{1.402874in}}%
\pgfpathlineto{\pgfqpoint{0.436561in}{1.414932in}}%
\pgfpathlineto{\pgfqpoint{0.432461in}{1.428009in}}%
\pgfpathlineto{\pgfqpoint{0.421377in}{1.430192in}}%
\pgfpathlineto{\pgfqpoint{0.415847in}{1.436309in}}%
\pgfpathlineto{\pgfqpoint{0.406892in}{1.458808in}}%
\pgfpathlineto{\pgfqpoint{0.411078in}{1.466888in}}%
\pgfpathlineto{\pgfqpoint{0.408109in}{1.481933in}}%
\pgfpathlineto{\pgfqpoint{0.414682in}{1.501419in}}%
\pgfpathlineto{\pgfqpoint{0.422383in}{1.494243in}}%
\pgfpathlineto{\pgfqpoint{0.418888in}{1.485771in}}%
\pgfpathlineto{\pgfqpoint{0.432233in}{1.472341in}}%
\pgfpathlineto{\pgfqpoint{0.431381in}{1.492284in}}%
\pgfpathlineto{\pgfqpoint{0.425663in}{1.497632in}}%
\pgfpathlineto{\pgfqpoint{0.432199in}{1.515167in}}%
\pgfpathlineto{\pgfqpoint{0.456921in}{1.512375in}}%
\pgfpathlineto{\pgfqpoint{0.453594in}{1.518895in}}%
\pgfpathlineto{\pgfqpoint{0.437278in}{1.518254in}}%
\pgfpathlineto{\pgfqpoint{0.429516in}{1.528130in}}%
\pgfpathlineto{\pgfqpoint{0.419745in}{1.519360in}}%
\pgfpathlineto{\pgfqpoint{0.414561in}{1.504701in}}%
\pgfpathlineto{\pgfqpoint{0.400724in}{1.524426in}}%
\pgfpathlineto{\pgfqpoint{0.395395in}{1.528017in}}%
\pgfpathlineto{\pgfqpoint{0.397425in}{1.549485in}}%
\pgfpathlineto{\pgfqpoint{0.393207in}{1.562145in}}%
\pgfpathlineto{\pgfqpoint{0.385562in}{1.574039in}}%
\pgfpathlineto{\pgfqpoint{0.369905in}{1.610482in}}%
\pgfpathlineto{\pgfqpoint{0.375024in}{1.618541in}}%
\pgfpathlineto{\pgfqpoint{0.374964in}{1.644010in}}%
\pgfpathlineto{\pgfqpoint{0.383414in}{1.658215in}}%
\pgfpathlineto{\pgfqpoint{0.385352in}{1.680415in}}%
\pgfpathlineto{\pgfqpoint{0.377384in}{1.705851in}}%
\pgfpathlineto{\pgfqpoint{0.366813in}{1.722042in}}%
\pgfpathlineto{\pgfqpoint{0.368698in}{1.736652in}}%
\pgfpathlineto{\pgfqpoint{0.398383in}{1.772011in}}%
\pgfpathlineto{\pgfqpoint{0.399858in}{1.784073in}}%
\pgfpathlineto{\pgfqpoint{0.413234in}{1.807081in}}%
\pgfpathlineto{\pgfqpoint{0.415090in}{1.828890in}}%
\pgfpathlineto{\pgfqpoint{0.410789in}{1.834455in}}%
\pgfpathlineto{\pgfqpoint{0.418363in}{1.850356in}}%
\pgfusepath{stroke}%
\end{pgfscope}%
\begin{pgfscope}%
\pgfpathrectangle{\pgfqpoint{0.100000in}{0.100000in}}{\pgfqpoint{3.608454in}{2.310000in}}%
\pgfusepath{clip}%
\pgfsetbuttcap%
\pgfsetroundjoin%
\pgfsetlinewidth{0.010037pt}%
\definecolor{currentstroke}{rgb}{1.000000,1.000000,1.000000}%
\pgfsetstrokecolor{currentstroke}%
\pgfsetdash{}{0pt}%
\pgfpathmoveto{\pgfqpoint{2.730393in}{1.382751in}}%
\pgfpathlineto{\pgfqpoint{2.720348in}{1.481089in}}%
\pgfpathlineto{\pgfqpoint{2.708677in}{1.586775in}}%
\pgfpathlineto{\pgfqpoint{2.785105in}{1.598157in}}%
\pgfpathlineto{\pgfqpoint{2.805389in}{1.592878in}}%
\pgfpathlineto{\pgfqpoint{2.815117in}{1.587147in}}%
\pgfpathlineto{\pgfqpoint{2.827308in}{1.588736in}}%
\pgfpathlineto{\pgfqpoint{2.843381in}{1.579248in}}%
\pgfpathlineto{\pgfqpoint{2.873320in}{1.593370in}}%
\pgfpathlineto{\pgfqpoint{2.889841in}{1.593841in}}%
\pgfpathlineto{\pgfqpoint{2.909135in}{1.615366in}}%
\pgfpathlineto{\pgfqpoint{2.928765in}{1.628459in}}%
\pgfpathlineto{\pgfqpoint{2.954949in}{1.643541in}}%
\pgfpathlineto{\pgfqpoint{2.971804in}{1.538119in}}%
\pgfpathlineto{\pgfqpoint{2.963973in}{1.531348in}}%
\pgfpathlineto{\pgfqpoint{2.969029in}{1.525129in}}%
\pgfpathlineto{\pgfqpoint{2.971043in}{1.511497in}}%
\pgfpathlineto{\pgfqpoint{2.967120in}{1.498692in}}%
\pgfpathlineto{\pgfqpoint{2.966601in}{1.479976in}}%
\pgfpathlineto{\pgfqpoint{2.962917in}{1.455616in}}%
\pgfpathlineto{\pgfqpoint{2.944276in}{1.433890in}}%
\pgfpathlineto{\pgfqpoint{2.936429in}{1.429278in}}%
\pgfpathlineto{\pgfqpoint{2.930326in}{1.433249in}}%
\pgfpathlineto{\pgfqpoint{2.915224in}{1.412541in}}%
\pgfpathlineto{\pgfqpoint{2.916638in}{1.392602in}}%
\pgfpathlineto{\pgfqpoint{2.899819in}{1.397397in}}%
\pgfpathlineto{\pgfqpoint{2.891855in}{1.377484in}}%
\pgfpathlineto{\pgfqpoint{2.895881in}{1.363373in}}%
\pgfpathlineto{\pgfqpoint{2.889579in}{1.361288in}}%
\pgfpathlineto{\pgfqpoint{2.888697in}{1.350139in}}%
\pgfpathlineto{\pgfqpoint{2.873228in}{1.345641in}}%
\pgfpathlineto{\pgfqpoint{2.865190in}{1.354643in}}%
\pgfpathlineto{\pgfqpoint{2.854847in}{1.358134in}}%
\pgfpathlineto{\pgfqpoint{2.852417in}{1.367262in}}%
\pgfpathlineto{\pgfqpoint{2.843065in}{1.365657in}}%
\pgfpathlineto{\pgfqpoint{2.836938in}{1.357261in}}%
\pgfpathlineto{\pgfqpoint{2.828173in}{1.354308in}}%
\pgfpathlineto{\pgfqpoint{2.812691in}{1.360250in}}%
\pgfpathlineto{\pgfqpoint{2.804128in}{1.353181in}}%
\pgfpathlineto{\pgfqpoint{2.791958in}{1.361544in}}%
\pgfpathlineto{\pgfqpoint{2.768610in}{1.364112in}}%
\pgfpathlineto{\pgfqpoint{2.761568in}{1.379305in}}%
\pgfpathlineto{\pgfqpoint{2.752646in}{1.386037in}}%
\pgfpathlineto{\pgfqpoint{2.743996in}{1.381728in}}%
\pgfpathlineto{\pgfqpoint{2.730393in}{1.382751in}}%
\pgfusepath{stroke}%
\end{pgfscope}%
\begin{pgfscope}%
\pgfpathrectangle{\pgfqpoint{0.100000in}{0.100000in}}{\pgfqpoint{3.608454in}{2.310000in}}%
\pgfusepath{clip}%
\pgfsetbuttcap%
\pgfsetroundjoin%
\pgfsetlinewidth{0.010037pt}%
\definecolor{currentstroke}{rgb}{1.000000,1.000000,1.000000}%
\pgfsetstrokecolor{currentstroke}%
\pgfsetdash{}{0pt}%
\pgfpathmoveto{\pgfqpoint{2.477364in}{1.192570in}}%
\pgfpathlineto{\pgfqpoint{2.468410in}{1.199844in}}%
\pgfpathlineto{\pgfqpoint{2.461108in}{1.196383in}}%
\pgfpathlineto{\pgfqpoint{2.451774in}{1.213813in}}%
\pgfpathlineto{\pgfqpoint{2.456543in}{1.224768in}}%
\pgfpathlineto{\pgfqpoint{2.449670in}{1.237099in}}%
\pgfpathlineto{\pgfqpoint{2.449301in}{1.246806in}}%
\pgfpathlineto{\pgfqpoint{2.440014in}{1.250265in}}%
\pgfpathlineto{\pgfqpoint{2.435708in}{1.257580in}}%
\pgfpathlineto{\pgfqpoint{2.417550in}{1.266715in}}%
\pgfpathlineto{\pgfqpoint{2.401780in}{1.277970in}}%
\pgfpathlineto{\pgfqpoint{2.394453in}{1.286467in}}%
\pgfpathlineto{\pgfqpoint{2.394301in}{1.296831in}}%
\pgfpathlineto{\pgfqpoint{2.404106in}{1.316728in}}%
\pgfpathlineto{\pgfqpoint{2.407122in}{1.331947in}}%
\pgfpathlineto{\pgfqpoint{2.399134in}{1.340547in}}%
\pgfpathlineto{\pgfqpoint{2.388555in}{1.343787in}}%
\pgfpathlineto{\pgfqpoint{2.375719in}{1.336702in}}%
\pgfpathlineto{\pgfqpoint{2.370287in}{1.348866in}}%
\pgfpathlineto{\pgfqpoint{2.367545in}{1.365354in}}%
\pgfpathlineto{\pgfqpoint{2.348630in}{1.380053in}}%
\pgfpathlineto{\pgfqpoint{2.344850in}{1.386576in}}%
\pgfpathlineto{\pgfqpoint{2.327573in}{1.401342in}}%
\pgfpathlineto{\pgfqpoint{2.322158in}{1.412077in}}%
\pgfpathlineto{\pgfqpoint{2.317277in}{1.433365in}}%
\pgfpathlineto{\pgfqpoint{2.320622in}{1.452274in}}%
\pgfpathlineto{\pgfqpoint{2.325089in}{1.454910in}}%
\pgfpathlineto{\pgfqpoint{2.324285in}{1.470738in}}%
\pgfpathlineto{\pgfqpoint{2.336891in}{1.475440in}}%
\pgfpathlineto{\pgfqpoint{2.340698in}{1.489644in}}%
\pgfpathlineto{\pgfqpoint{2.347951in}{1.499205in}}%
\pgfpathlineto{\pgfqpoint{2.347584in}{1.511368in}}%
\pgfpathlineto{\pgfqpoint{2.338595in}{1.521042in}}%
\pgfpathlineto{\pgfqpoint{2.340730in}{1.534582in}}%
\pgfpathlineto{\pgfqpoint{2.364032in}{1.540471in}}%
\pgfpathlineto{\pgfqpoint{2.381909in}{1.551220in}}%
\pgfpathlineto{\pgfqpoint{2.383785in}{1.564755in}}%
\pgfpathlineto{\pgfqpoint{2.390006in}{1.569016in}}%
\pgfpathlineto{\pgfqpoint{2.392386in}{1.583234in}}%
\pgfpathlineto{\pgfqpoint{2.390473in}{1.592626in}}%
\pgfpathlineto{\pgfqpoint{2.378251in}{1.600427in}}%
\pgfpathlineto{\pgfqpoint{2.373331in}{1.612057in}}%
\pgfpathlineto{\pgfqpoint{2.361269in}{1.623298in}}%
\pgfpathlineto{\pgfqpoint{2.460390in}{1.627522in}}%
\pgfpathlineto{\pgfqpoint{2.526890in}{1.632246in}}%
\pgfpathlineto{\pgfqpoint{2.525674in}{1.618258in}}%
\pgfpathlineto{\pgfqpoint{2.537007in}{1.598963in}}%
\pgfpathlineto{\pgfqpoint{2.541765in}{1.582478in}}%
\pgfpathlineto{\pgfqpoint{2.547432in}{1.573118in}}%
\pgfpathlineto{\pgfqpoint{2.553618in}{1.495995in}}%
\pgfpathlineto{\pgfqpoint{2.562209in}{1.386059in}}%
\pgfpathlineto{\pgfqpoint{2.555581in}{1.369354in}}%
\pgfpathlineto{\pgfqpoint{2.565129in}{1.355623in}}%
\pgfpathlineto{\pgfqpoint{2.567791in}{1.342507in}}%
\pgfpathlineto{\pgfqpoint{2.554624in}{1.315312in}}%
\pgfpathlineto{\pgfqpoint{2.555400in}{1.311555in}}%
\pgfpathlineto{\pgfqpoint{2.541848in}{1.295967in}}%
\pgfpathlineto{\pgfqpoint{2.545564in}{1.291207in}}%
\pgfpathlineto{\pgfqpoint{2.539926in}{1.281490in}}%
\pgfpathlineto{\pgfqpoint{2.541376in}{1.261906in}}%
\pgfpathlineto{\pgfqpoint{2.534528in}{1.249888in}}%
\pgfpathlineto{\pgfqpoint{2.540098in}{1.235688in}}%
\pgfpathlineto{\pgfqpoint{2.516761in}{1.227971in}}%
\pgfpathlineto{\pgfqpoint{2.514610in}{1.219580in}}%
\pgfpathlineto{\pgfqpoint{2.520983in}{1.208943in}}%
\pgfpathlineto{\pgfqpoint{2.518051in}{1.202010in}}%
\pgfpathlineto{\pgfqpoint{2.493026in}{1.210574in}}%
\pgfpathlineto{\pgfqpoint{2.484770in}{1.211436in}}%
\pgfpathlineto{\pgfqpoint{2.474476in}{1.198417in}}%
\pgfpathlineto{\pgfqpoint{2.477364in}{1.192570in}}%
\pgfusepath{stroke}%
\end{pgfscope}%
\begin{pgfscope}%
\pgfpathrectangle{\pgfqpoint{0.100000in}{0.100000in}}{\pgfqpoint{3.608454in}{2.310000in}}%
\pgfusepath{clip}%
\pgfsetbuttcap%
\pgfsetroundjoin%
\pgfsetlinewidth{0.010037pt}%
\definecolor{currentstroke}{rgb}{1.000000,1.000000,1.000000}%
\pgfsetstrokecolor{currentstroke}%
\pgfsetdash{}{0pt}%
\pgfpathmoveto{\pgfqpoint{3.205784in}{1.431832in}}%
\pgfpathlineto{\pgfqpoint{3.198897in}{1.442058in}}%
\pgfpathlineto{\pgfqpoint{3.202783in}{1.447781in}}%
\pgfpathlineto{\pgfqpoint{3.212300in}{1.441354in}}%
\pgfpathlineto{\pgfqpoint{3.205784in}{1.431832in}}%
\pgfusepath{stroke}%
\end{pgfscope}%
\begin{pgfscope}%
\pgfpathrectangle{\pgfqpoint{0.100000in}{0.100000in}}{\pgfqpoint{3.608454in}{2.310000in}}%
\pgfusepath{clip}%
\pgfsetbuttcap%
\pgfsetroundjoin%
\pgfsetlinewidth{0.010037pt}%
\definecolor{currentstroke}{rgb}{1.000000,1.000000,1.000000}%
\pgfsetstrokecolor{currentstroke}%
\pgfsetdash{}{0pt}%
\pgfpathmoveto{\pgfqpoint{3.265920in}{1.519827in}}%
\pgfpathlineto{\pgfqpoint{3.270035in}{1.528522in}}%
\pgfpathlineto{\pgfqpoint{3.286671in}{1.530400in}}%
\pgfpathlineto{\pgfqpoint{3.284005in}{1.523011in}}%
\pgfpathlineto{\pgfqpoint{3.278517in}{1.513571in}}%
\pgfpathlineto{\pgfqpoint{3.282236in}{1.502324in}}%
\pgfpathlineto{\pgfqpoint{3.296914in}{1.488849in}}%
\pgfpathlineto{\pgfqpoint{3.300321in}{1.474658in}}%
\pgfpathlineto{\pgfqpoint{3.317232in}{1.456987in}}%
\pgfpathlineto{\pgfqpoint{3.323867in}{1.457744in}}%
\pgfpathlineto{\pgfqpoint{3.332131in}{1.431216in}}%
\pgfpathlineto{\pgfqpoint{3.330776in}{1.430952in}}%
\pgfpathlineto{\pgfqpoint{3.329269in}{1.430656in}}%
\pgfpathlineto{\pgfqpoint{3.292410in}{1.423600in}}%
\pgfpathlineto{\pgfqpoint{3.272690in}{1.493733in}}%
\pgfpathlineto{\pgfqpoint{3.265920in}{1.519827in}}%
\pgfusepath{stroke}%
\end{pgfscope}%
\begin{pgfscope}%
\pgfpathrectangle{\pgfqpoint{0.100000in}{0.100000in}}{\pgfqpoint{3.608454in}{2.310000in}}%
\pgfusepath{clip}%
\pgfsetbuttcap%
\pgfsetroundjoin%
\pgfsetlinewidth{0.010037pt}%
\definecolor{currentstroke}{rgb}{1.000000,1.000000,1.000000}%
\pgfsetstrokecolor{currentstroke}%
\pgfsetdash{}{0pt}%
\pgfpathmoveto{\pgfqpoint{2.921647in}{1.282104in}}%
\pgfpathlineto{\pgfqpoint{2.910000in}{1.282532in}}%
\pgfpathlineto{\pgfqpoint{2.899268in}{1.289925in}}%
\pgfpathlineto{\pgfqpoint{2.884769in}{1.312394in}}%
\pgfpathlineto{\pgfqpoint{2.872427in}{1.324293in}}%
\pgfpathlineto{\pgfqpoint{2.875653in}{1.333484in}}%
\pgfpathlineto{\pgfqpoint{2.873228in}{1.345641in}}%
\pgfpathlineto{\pgfqpoint{2.888697in}{1.350139in}}%
\pgfpathlineto{\pgfqpoint{2.889579in}{1.361288in}}%
\pgfpathlineto{\pgfqpoint{2.895881in}{1.363373in}}%
\pgfpathlineto{\pgfqpoint{2.891855in}{1.377484in}}%
\pgfpathlineto{\pgfqpoint{2.899819in}{1.397397in}}%
\pgfpathlineto{\pgfqpoint{2.916638in}{1.392602in}}%
\pgfpathlineto{\pgfqpoint{2.915224in}{1.412541in}}%
\pgfpathlineto{\pgfqpoint{2.930326in}{1.433249in}}%
\pgfpathlineto{\pgfqpoint{2.936429in}{1.429278in}}%
\pgfpathlineto{\pgfqpoint{2.944276in}{1.433890in}}%
\pgfpathlineto{\pgfqpoint{2.962917in}{1.455616in}}%
\pgfpathlineto{\pgfqpoint{2.966601in}{1.479976in}}%
\pgfpathlineto{\pgfqpoint{2.967120in}{1.498692in}}%
\pgfpathlineto{\pgfqpoint{2.971043in}{1.511497in}}%
\pgfpathlineto{\pgfqpoint{2.969029in}{1.525129in}}%
\pgfpathlineto{\pgfqpoint{2.963973in}{1.531348in}}%
\pgfpathlineto{\pgfqpoint{2.971804in}{1.538119in}}%
\pgfpathlineto{\pgfqpoint{2.983156in}{1.466600in}}%
\pgfpathlineto{\pgfqpoint{3.045737in}{1.476948in}}%
\pgfpathlineto{\pgfqpoint{3.052224in}{1.436127in}}%
\pgfpathlineto{\pgfqpoint{3.074880in}{1.463071in}}%
\pgfpathlineto{\pgfqpoint{3.080218in}{1.460384in}}%
\pgfpathlineto{\pgfqpoint{3.088071in}{1.475996in}}%
\pgfpathlineto{\pgfqpoint{3.098951in}{1.471101in}}%
\pgfpathlineto{\pgfqpoint{3.108437in}{1.472032in}}%
\pgfpathlineto{\pgfqpoint{3.114696in}{1.482885in}}%
\pgfpathlineto{\pgfqpoint{3.123780in}{1.488922in}}%
\pgfpathlineto{\pgfqpoint{3.136410in}{1.483605in}}%
\pgfpathlineto{\pgfqpoint{3.144720in}{1.485442in}}%
\pgfpathlineto{\pgfqpoint{3.156626in}{1.464838in}}%
\pgfpathlineto{\pgfqpoint{3.153183in}{1.449239in}}%
\pgfpathlineto{\pgfqpoint{3.122088in}{1.466802in}}%
\pgfpathlineto{\pgfqpoint{3.116263in}{1.452384in}}%
\pgfpathlineto{\pgfqpoint{3.118185in}{1.445751in}}%
\pgfpathlineto{\pgfqpoint{3.104950in}{1.423326in}}%
\pgfpathlineto{\pgfqpoint{3.098705in}{1.418866in}}%
\pgfpathlineto{\pgfqpoint{3.095884in}{1.408977in}}%
\pgfpathlineto{\pgfqpoint{3.085235in}{1.410012in}}%
\pgfpathlineto{\pgfqpoint{3.077588in}{1.383022in}}%
\pgfpathlineto{\pgfqpoint{3.073308in}{1.376829in}}%
\pgfpathlineto{\pgfqpoint{3.062309in}{1.378894in}}%
\pgfpathlineto{\pgfqpoint{3.051058in}{1.387407in}}%
\pgfpathlineto{\pgfqpoint{3.050668in}{1.374341in}}%
\pgfpathlineto{\pgfqpoint{3.039777in}{1.352383in}}%
\pgfpathlineto{\pgfqpoint{3.037068in}{1.336751in}}%
\pgfpathlineto{\pgfqpoint{3.028711in}{1.326341in}}%
\pgfpathlineto{\pgfqpoint{3.022375in}{1.309715in}}%
\pgfpathlineto{\pgfqpoint{3.022064in}{1.294324in}}%
\pgfpathlineto{\pgfqpoint{3.012280in}{1.291449in}}%
\pgfpathlineto{\pgfqpoint{3.001180in}{1.282737in}}%
\pgfpathlineto{\pgfqpoint{2.990493in}{1.281077in}}%
\pgfpathlineto{\pgfqpoint{2.988030in}{1.273703in}}%
\pgfpathlineto{\pgfqpoint{2.970810in}{1.266151in}}%
\pgfpathlineto{\pgfqpoint{2.961186in}{1.272585in}}%
\pgfpathlineto{\pgfqpoint{2.950430in}{1.260367in}}%
\pgfpathlineto{\pgfqpoint{2.931940in}{1.263968in}}%
\pgfpathlineto{\pgfqpoint{2.920600in}{1.276793in}}%
\pgfpathlineto{\pgfqpoint{2.921647in}{1.282104in}}%
\pgfusepath{stroke}%
\end{pgfscope}%
\begin{pgfscope}%
\pgfpathrectangle{\pgfqpoint{0.100000in}{0.100000in}}{\pgfqpoint{3.608454in}{2.310000in}}%
\pgfusepath{clip}%
\pgfsetbuttcap%
\pgfsetroundjoin%
\pgfsetlinewidth{0.010037pt}%
\definecolor{currentstroke}{rgb}{1.000000,1.000000,1.000000}%
\pgfsetstrokecolor{currentstroke}%
\pgfsetdash{}{0pt}%
\pgfpathmoveto{\pgfqpoint{3.329269in}{1.430656in}}%
\pgfpathlineto{\pgfqpoint{3.328797in}{1.416244in}}%
\pgfpathlineto{\pgfqpoint{3.323250in}{1.409159in}}%
\pgfpathlineto{\pgfqpoint{3.319710in}{1.393431in}}%
\pgfpathlineto{\pgfqpoint{3.303727in}{1.386181in}}%
\pgfpathlineto{\pgfqpoint{3.290316in}{1.384068in}}%
\pgfpathlineto{\pgfqpoint{3.294224in}{1.394412in}}%
\pgfpathlineto{\pgfqpoint{3.273575in}{1.403080in}}%
\pgfpathlineto{\pgfqpoint{3.256800in}{1.413932in}}%
\pgfpathlineto{\pgfqpoint{3.267185in}{1.430155in}}%
\pgfpathlineto{\pgfqpoint{3.258862in}{1.442765in}}%
\pgfpathlineto{\pgfqpoint{3.259842in}{1.453641in}}%
\pgfpathlineto{\pgfqpoint{3.253824in}{1.456627in}}%
\pgfpathlineto{\pgfqpoint{3.248940in}{1.474298in}}%
\pgfpathlineto{\pgfqpoint{3.260489in}{1.498252in}}%
\pgfpathlineto{\pgfqpoint{3.251813in}{1.502164in}}%
\pgfpathlineto{\pgfqpoint{3.249564in}{1.490346in}}%
\pgfpathlineto{\pgfqpoint{3.237178in}{1.487056in}}%
\pgfpathlineto{\pgfqpoint{3.238481in}{1.448173in}}%
\pgfpathlineto{\pgfqpoint{3.236216in}{1.435660in}}%
\pgfpathlineto{\pgfqpoint{3.242428in}{1.417824in}}%
\pgfpathlineto{\pgfqpoint{3.251985in}{1.409243in}}%
\pgfpathlineto{\pgfqpoint{3.247651in}{1.403855in}}%
\pgfpathlineto{\pgfqpoint{3.257407in}{1.395982in}}%
\pgfpathlineto{\pgfqpoint{3.260933in}{1.383203in}}%
\pgfpathlineto{\pgfqpoint{3.243052in}{1.393785in}}%
\pgfpathlineto{\pgfqpoint{3.231736in}{1.392411in}}%
\pgfpathlineto{\pgfqpoint{3.222950in}{1.403297in}}%
\pgfpathlineto{\pgfqpoint{3.214003in}{1.404357in}}%
\pgfpathlineto{\pgfqpoint{3.201288in}{1.398904in}}%
\pgfpathlineto{\pgfqpoint{3.196357in}{1.405699in}}%
\pgfpathlineto{\pgfqpoint{3.202838in}{1.419960in}}%
\pgfpathlineto{\pgfqpoint{3.205784in}{1.431832in}}%
\pgfpathlineto{\pgfqpoint{3.212300in}{1.441354in}}%
\pgfpathlineto{\pgfqpoint{3.202783in}{1.447781in}}%
\pgfpathlineto{\pgfqpoint{3.198897in}{1.442058in}}%
\pgfpathlineto{\pgfqpoint{3.189390in}{1.447867in}}%
\pgfpathlineto{\pgfqpoint{3.172555in}{1.451737in}}%
\pgfpathlineto{\pgfqpoint{3.174081in}{1.460247in}}%
\pgfpathlineto{\pgfqpoint{3.166450in}{1.465183in}}%
\pgfpathlineto{\pgfqpoint{3.156626in}{1.464838in}}%
\pgfpathlineto{\pgfqpoint{3.144720in}{1.485442in}}%
\pgfpathlineto{\pgfqpoint{3.136410in}{1.483605in}}%
\pgfpathlineto{\pgfqpoint{3.123780in}{1.488922in}}%
\pgfpathlineto{\pgfqpoint{3.114696in}{1.482885in}}%
\pgfpathlineto{\pgfqpoint{3.108437in}{1.472032in}}%
\pgfpathlineto{\pgfqpoint{3.098951in}{1.471101in}}%
\pgfpathlineto{\pgfqpoint{3.088071in}{1.475996in}}%
\pgfpathlineto{\pgfqpoint{3.080218in}{1.460384in}}%
\pgfpathlineto{\pgfqpoint{3.074880in}{1.463071in}}%
\pgfpathlineto{\pgfqpoint{3.052224in}{1.436127in}}%
\pgfpathlineto{\pgfqpoint{3.045737in}{1.476948in}}%
\pgfpathlineto{\pgfqpoint{3.128532in}{1.492293in}}%
\pgfpathlineto{\pgfqpoint{3.165648in}{1.498939in}}%
\pgfpathlineto{\pgfqpoint{3.219645in}{1.509814in}}%
\pgfpathlineto{\pgfqpoint{3.265920in}{1.519827in}}%
\pgfpathlineto{\pgfqpoint{3.272690in}{1.493733in}}%
\pgfpathlineto{\pgfqpoint{3.292410in}{1.423600in}}%
\pgfpathlineto{\pgfqpoint{3.329269in}{1.430656in}}%
\pgfusepath{stroke}%
\end{pgfscope}%
\begin{pgfscope}%
\pgfpathrectangle{\pgfqpoint{0.100000in}{0.100000in}}{\pgfqpoint{3.608454in}{2.310000in}}%
\pgfusepath{clip}%
\pgfsetbuttcap%
\pgfsetroundjoin%
\pgfsetlinewidth{0.010037pt}%
\definecolor{currentstroke}{rgb}{1.000000,1.000000,1.000000}%
\pgfsetstrokecolor{currentstroke}%
\pgfsetdash{}{0pt}%
\pgfpathmoveto{\pgfqpoint{1.662648in}{1.197867in}}%
\pgfpathlineto{\pgfqpoint{1.602243in}{1.203632in}}%
\pgfpathlineto{\pgfqpoint{1.539577in}{1.209185in}}%
\pgfpathlineto{\pgfqpoint{1.467164in}{1.216739in}}%
\pgfpathlineto{\pgfqpoint{1.359578in}{1.229535in}}%
\pgfpathlineto{\pgfqpoint{1.321411in}{1.235415in}}%
\pgfpathlineto{\pgfqpoint{1.222893in}{1.249723in}}%
\pgfpathlineto{\pgfqpoint{1.237278in}{1.340011in}}%
\pgfpathlineto{\pgfqpoint{1.237637in}{1.347314in}}%
\pgfpathlineto{\pgfqpoint{1.251495in}{1.434433in}}%
\pgfpathlineto{\pgfqpoint{1.261846in}{1.500784in}}%
\pgfpathlineto{\pgfqpoint{1.271614in}{1.562313in}}%
\pgfpathlineto{\pgfqpoint{1.338353in}{1.552733in}}%
\pgfpathlineto{\pgfqpoint{1.400578in}{1.543802in}}%
\pgfpathlineto{\pgfqpoint{1.514964in}{1.529684in}}%
\pgfpathlineto{\pgfqpoint{1.567441in}{1.524892in}}%
\pgfpathlineto{\pgfqpoint{1.650576in}{1.516826in}}%
\pgfpathlineto{\pgfqpoint{1.686540in}{1.513906in}}%
\pgfpathlineto{\pgfqpoint{1.680190in}{1.435164in}}%
\pgfpathlineto{\pgfqpoint{1.674454in}{1.359348in}}%
\pgfpathlineto{\pgfqpoint{1.669837in}{1.297616in}}%
\pgfpathlineto{\pgfqpoint{1.662648in}{1.197867in}}%
\pgfusepath{stroke}%
\end{pgfscope}%
\begin{pgfscope}%
\pgfpathrectangle{\pgfqpoint{0.100000in}{0.100000in}}{\pgfqpoint{3.608454in}{2.310000in}}%
\pgfusepath{clip}%
\pgfsetbuttcap%
\pgfsetroundjoin%
\pgfsetlinewidth{0.010037pt}%
\definecolor{currentstroke}{rgb}{1.000000,1.000000,1.000000}%
\pgfsetstrokecolor{currentstroke}%
\pgfsetdash{}{0pt}%
\pgfpathmoveto{\pgfqpoint{2.461763in}{1.153105in}}%
\pgfpathlineto{\pgfqpoint{2.463799in}{1.162233in}}%
\pgfpathlineto{\pgfqpoint{2.474354in}{1.160178in}}%
\pgfpathlineto{\pgfqpoint{2.477364in}{1.192570in}}%
\pgfpathlineto{\pgfqpoint{2.474476in}{1.198417in}}%
\pgfpathlineto{\pgfqpoint{2.484770in}{1.211436in}}%
\pgfpathlineto{\pgfqpoint{2.493026in}{1.210574in}}%
\pgfpathlineto{\pgfqpoint{2.518051in}{1.202010in}}%
\pgfpathlineto{\pgfqpoint{2.520983in}{1.208943in}}%
\pgfpathlineto{\pgfqpoint{2.514610in}{1.219580in}}%
\pgfpathlineto{\pgfqpoint{2.516761in}{1.227971in}}%
\pgfpathlineto{\pgfqpoint{2.540098in}{1.235688in}}%
\pgfpathlineto{\pgfqpoint{2.534528in}{1.249888in}}%
\pgfpathlineto{\pgfqpoint{2.541376in}{1.261906in}}%
\pgfpathlineto{\pgfqpoint{2.554124in}{1.268651in}}%
\pgfpathlineto{\pgfqpoint{2.580872in}{1.275331in}}%
\pgfpathlineto{\pgfqpoint{2.597748in}{1.265248in}}%
\pgfpathlineto{\pgfqpoint{2.602868in}{1.275081in}}%
\pgfpathlineto{\pgfqpoint{2.616881in}{1.281949in}}%
\pgfpathlineto{\pgfqpoint{2.620497in}{1.275892in}}%
\pgfpathlineto{\pgfqpoint{2.634892in}{1.280705in}}%
\pgfpathlineto{\pgfqpoint{2.633983in}{1.288941in}}%
\pgfpathlineto{\pgfqpoint{2.646926in}{1.298367in}}%
\pgfpathlineto{\pgfqpoint{2.654564in}{1.288579in}}%
\pgfpathlineto{\pgfqpoint{2.664612in}{1.287594in}}%
\pgfpathlineto{\pgfqpoint{2.671290in}{1.293974in}}%
\pgfpathlineto{\pgfqpoint{2.670547in}{1.303046in}}%
\pgfpathlineto{\pgfqpoint{2.676223in}{1.312054in}}%
\pgfpathlineto{\pgfqpoint{2.683829in}{1.314006in}}%
\pgfpathlineto{\pgfqpoint{2.686884in}{1.325901in}}%
\pgfpathlineto{\pgfqpoint{2.697946in}{1.336184in}}%
\pgfpathlineto{\pgfqpoint{2.694616in}{1.346419in}}%
\pgfpathlineto{\pgfqpoint{2.705378in}{1.351515in}}%
\pgfpathlineto{\pgfqpoint{2.712569in}{1.348381in}}%
\pgfpathlineto{\pgfqpoint{2.723189in}{1.356342in}}%
\pgfpathlineto{\pgfqpoint{2.732680in}{1.358417in}}%
\pgfpathlineto{\pgfqpoint{2.732714in}{1.366675in}}%
\pgfpathlineto{\pgfqpoint{2.726063in}{1.378135in}}%
\pgfpathlineto{\pgfqpoint{2.730393in}{1.382751in}}%
\pgfpathlineto{\pgfqpoint{2.743996in}{1.381728in}}%
\pgfpathlineto{\pgfqpoint{2.752646in}{1.386037in}}%
\pgfpathlineto{\pgfqpoint{2.761568in}{1.379305in}}%
\pgfpathlineto{\pgfqpoint{2.768610in}{1.364112in}}%
\pgfpathlineto{\pgfqpoint{2.791958in}{1.361544in}}%
\pgfpathlineto{\pgfqpoint{2.804128in}{1.353181in}}%
\pgfpathlineto{\pgfqpoint{2.812691in}{1.360250in}}%
\pgfpathlineto{\pgfqpoint{2.828173in}{1.354308in}}%
\pgfpathlineto{\pgfqpoint{2.836938in}{1.357261in}}%
\pgfpathlineto{\pgfqpoint{2.843065in}{1.365657in}}%
\pgfpathlineto{\pgfqpoint{2.852417in}{1.367262in}}%
\pgfpathlineto{\pgfqpoint{2.854847in}{1.358134in}}%
\pgfpathlineto{\pgfqpoint{2.865190in}{1.354643in}}%
\pgfpathlineto{\pgfqpoint{2.873228in}{1.345641in}}%
\pgfpathlineto{\pgfqpoint{2.875653in}{1.333484in}}%
\pgfpathlineto{\pgfqpoint{2.872427in}{1.324293in}}%
\pgfpathlineto{\pgfqpoint{2.884769in}{1.312394in}}%
\pgfpathlineto{\pgfqpoint{2.899268in}{1.289925in}}%
\pgfpathlineto{\pgfqpoint{2.910000in}{1.282532in}}%
\pgfpathlineto{\pgfqpoint{2.921647in}{1.282104in}}%
\pgfpathlineto{\pgfqpoint{2.900141in}{1.257427in}}%
\pgfpathlineto{\pgfqpoint{2.878972in}{1.242490in}}%
\pgfpathlineto{\pgfqpoint{2.871181in}{1.230629in}}%
\pgfpathlineto{\pgfqpoint{2.871313in}{1.224191in}}%
\pgfpathlineto{\pgfqpoint{2.859847in}{1.219258in}}%
\pgfpathlineto{\pgfqpoint{2.856561in}{1.209974in}}%
\pgfpathlineto{\pgfqpoint{2.832674in}{1.200623in}}%
\pgfpathlineto{\pgfqpoint{2.824207in}{1.194550in}}%
\pgfpathlineto{\pgfqpoint{2.823062in}{1.193254in}}%
\pgfpathlineto{\pgfqpoint{2.754353in}{1.186739in}}%
\pgfpathlineto{\pgfqpoint{2.712861in}{1.183250in}}%
\pgfpathlineto{\pgfqpoint{2.644773in}{1.179396in}}%
\pgfpathlineto{\pgfqpoint{2.559948in}{1.171005in}}%
\pgfpathlineto{\pgfqpoint{2.545934in}{1.172945in}}%
\pgfpathlineto{\pgfqpoint{2.548849in}{1.158640in}}%
\pgfpathlineto{\pgfqpoint{2.461763in}{1.153105in}}%
\pgfusepath{stroke}%
\end{pgfscope}%
\begin{pgfscope}%
\pgfpathrectangle{\pgfqpoint{0.100000in}{0.100000in}}{\pgfqpoint{3.608454in}{2.310000in}}%
\pgfusepath{clip}%
\pgfsetbuttcap%
\pgfsetroundjoin%
\pgfsetlinewidth{0.010037pt}%
\definecolor{currentstroke}{rgb}{1.000000,1.000000,1.000000}%
\pgfsetstrokecolor{currentstroke}%
\pgfsetdash{}{0pt}%
\pgfpathmoveto{\pgfqpoint{1.662648in}{1.197867in}}%
\pgfpathlineto{\pgfqpoint{1.669837in}{1.297616in}}%
\pgfpathlineto{\pgfqpoint{1.674454in}{1.359348in}}%
\pgfpathlineto{\pgfqpoint{1.680190in}{1.435164in}}%
\pgfpathlineto{\pgfqpoint{1.759716in}{1.429580in}}%
\pgfpathlineto{\pgfqpoint{1.860711in}{1.424063in}}%
\pgfpathlineto{\pgfqpoint{1.929402in}{1.421428in}}%
\pgfpathlineto{\pgfqpoint{1.997690in}{1.419335in}}%
\pgfpathlineto{\pgfqpoint{2.059498in}{1.418379in}}%
\pgfpathlineto{\pgfqpoint{2.088084in}{1.418676in}}%
\pgfpathlineto{\pgfqpoint{2.100667in}{1.408407in}}%
\pgfpathlineto{\pgfqpoint{2.110523in}{1.410477in}}%
\pgfpathlineto{\pgfqpoint{2.110832in}{1.401521in}}%
\pgfpathlineto{\pgfqpoint{2.103512in}{1.386028in}}%
\pgfpathlineto{\pgfqpoint{2.104306in}{1.376240in}}%
\pgfpathlineto{\pgfqpoint{2.111314in}{1.369785in}}%
\pgfpathlineto{\pgfqpoint{2.117751in}{1.356337in}}%
\pgfpathlineto{\pgfqpoint{2.130810in}{1.348616in}}%
\pgfpathlineto{\pgfqpoint{2.130373in}{1.297922in}}%
\pgfpathlineto{\pgfqpoint{2.130758in}{1.181350in}}%
\pgfpathlineto{\pgfqpoint{2.043230in}{1.181753in}}%
\pgfpathlineto{\pgfqpoint{1.951062in}{1.183366in}}%
\pgfpathlineto{\pgfqpoint{1.883211in}{1.185760in}}%
\pgfpathlineto{\pgfqpoint{1.826575in}{1.187945in}}%
\pgfpathlineto{\pgfqpoint{1.731159in}{1.193571in}}%
\pgfpathlineto{\pgfqpoint{1.662648in}{1.197867in}}%
\pgfusepath{stroke}%
\end{pgfscope}%
\begin{pgfscope}%
\pgfpathrectangle{\pgfqpoint{0.100000in}{0.100000in}}{\pgfqpoint{3.608454in}{2.310000in}}%
\pgfusepath{clip}%
\pgfsetbuttcap%
\pgfsetroundjoin%
\pgfsetlinewidth{0.010037pt}%
\definecolor{currentstroke}{rgb}{1.000000,1.000000,1.000000}%
\pgfsetstrokecolor{currentstroke}%
\pgfsetdash{}{0pt}%
\pgfpathmoveto{\pgfqpoint{2.950506in}{1.211075in}}%
\pgfpathlineto{\pgfqpoint{2.853565in}{1.197457in}}%
\pgfpathlineto{\pgfqpoint{2.824207in}{1.194550in}}%
\pgfpathlineto{\pgfqpoint{2.832674in}{1.200623in}}%
\pgfpathlineto{\pgfqpoint{2.856561in}{1.209974in}}%
\pgfpathlineto{\pgfqpoint{2.859847in}{1.219258in}}%
\pgfpathlineto{\pgfqpoint{2.871313in}{1.224191in}}%
\pgfpathlineto{\pgfqpoint{2.871181in}{1.230629in}}%
\pgfpathlineto{\pgfqpoint{2.878972in}{1.242490in}}%
\pgfpathlineto{\pgfqpoint{2.900141in}{1.257427in}}%
\pgfpathlineto{\pgfqpoint{2.921647in}{1.282104in}}%
\pgfpathlineto{\pgfqpoint{2.920600in}{1.276793in}}%
\pgfpathlineto{\pgfqpoint{2.931940in}{1.263968in}}%
\pgfpathlineto{\pgfqpoint{2.950430in}{1.260367in}}%
\pgfpathlineto{\pgfqpoint{2.961186in}{1.272585in}}%
\pgfpathlineto{\pgfqpoint{2.970810in}{1.266151in}}%
\pgfpathlineto{\pgfqpoint{2.988030in}{1.273703in}}%
\pgfpathlineto{\pgfqpoint{2.990493in}{1.281077in}}%
\pgfpathlineto{\pgfqpoint{3.001180in}{1.282737in}}%
\pgfpathlineto{\pgfqpoint{3.012280in}{1.291449in}}%
\pgfpathlineto{\pgfqpoint{3.022064in}{1.294324in}}%
\pgfpathlineto{\pgfqpoint{3.022375in}{1.309715in}}%
\pgfpathlineto{\pgfqpoint{3.028711in}{1.326341in}}%
\pgfpathlineto{\pgfqpoint{3.037068in}{1.336751in}}%
\pgfpathlineto{\pgfqpoint{3.039777in}{1.352383in}}%
\pgfpathlineto{\pgfqpoint{3.050668in}{1.374341in}}%
\pgfpathlineto{\pgfqpoint{3.051058in}{1.387407in}}%
\pgfpathlineto{\pgfqpoint{3.062309in}{1.378894in}}%
\pgfpathlineto{\pgfqpoint{3.073308in}{1.376829in}}%
\pgfpathlineto{\pgfqpoint{3.077588in}{1.383022in}}%
\pgfpathlineto{\pgfqpoint{3.085235in}{1.410012in}}%
\pgfpathlineto{\pgfqpoint{3.095884in}{1.408977in}}%
\pgfpathlineto{\pgfqpoint{3.098705in}{1.418866in}}%
\pgfpathlineto{\pgfqpoint{3.104950in}{1.423326in}}%
\pgfpathlineto{\pgfqpoint{3.118185in}{1.445751in}}%
\pgfpathlineto{\pgfqpoint{3.116263in}{1.452384in}}%
\pgfpathlineto{\pgfqpoint{3.122088in}{1.466802in}}%
\pgfpathlineto{\pgfqpoint{3.153183in}{1.449239in}}%
\pgfpathlineto{\pgfqpoint{3.156626in}{1.464838in}}%
\pgfpathlineto{\pgfqpoint{3.166450in}{1.465183in}}%
\pgfpathlineto{\pgfqpoint{3.174081in}{1.460247in}}%
\pgfpathlineto{\pgfqpoint{3.172555in}{1.451737in}}%
\pgfpathlineto{\pgfqpoint{3.189390in}{1.447867in}}%
\pgfpathlineto{\pgfqpoint{3.198897in}{1.442058in}}%
\pgfpathlineto{\pgfqpoint{3.205784in}{1.431832in}}%
\pgfpathlineto{\pgfqpoint{3.202838in}{1.419960in}}%
\pgfpathlineto{\pgfqpoint{3.196888in}{1.418988in}}%
\pgfpathlineto{\pgfqpoint{3.193438in}{1.401073in}}%
\pgfpathlineto{\pgfqpoint{3.200996in}{1.394069in}}%
\pgfpathlineto{\pgfqpoint{3.211636in}{1.399731in}}%
\pgfpathlineto{\pgfqpoint{3.221508in}{1.387775in}}%
\pgfpathlineto{\pgfqpoint{3.243566in}{1.385626in}}%
\pgfpathlineto{\pgfqpoint{3.247366in}{1.378749in}}%
\pgfpathlineto{\pgfqpoint{3.267778in}{1.372060in}}%
\pgfpathlineto{\pgfqpoint{3.265259in}{1.364168in}}%
\pgfpathlineto{\pgfqpoint{3.267915in}{1.341861in}}%
\pgfpathlineto{\pgfqpoint{3.276154in}{1.333438in}}%
\pgfpathlineto{\pgfqpoint{3.263997in}{1.329883in}}%
\pgfpathlineto{\pgfqpoint{3.265608in}{1.320360in}}%
\pgfpathlineto{\pgfqpoint{3.278578in}{1.312286in}}%
\pgfpathlineto{\pgfqpoint{3.279747in}{1.304321in}}%
\pgfpathlineto{\pgfqpoint{3.272425in}{1.298336in}}%
\pgfpathlineto{\pgfqpoint{3.257650in}{1.312515in}}%
\pgfpathlineto{\pgfqpoint{3.256190in}{1.302170in}}%
\pgfpathlineto{\pgfqpoint{3.268558in}{1.297251in}}%
\pgfpathlineto{\pgfqpoint{3.269796in}{1.292163in}}%
\pgfpathlineto{\pgfqpoint{3.286760in}{1.298887in}}%
\pgfpathlineto{\pgfqpoint{3.299759in}{1.300665in}}%
\pgfpathlineto{\pgfqpoint{3.313076in}{1.273798in}}%
\pgfpathlineto{\pgfqpoint{3.311590in}{1.273507in}}%
\pgfpathlineto{\pgfqpoint{3.305573in}{1.272263in}}%
\pgfpathlineto{\pgfqpoint{3.303799in}{1.271893in}}%
\pgfpathlineto{\pgfqpoint{3.302627in}{1.271664in}}%
\pgfpathlineto{\pgfqpoint{3.223341in}{1.255217in}}%
\pgfpathlineto{\pgfqpoint{3.152426in}{1.240702in}}%
\pgfpathlineto{\pgfqpoint{3.054377in}{1.223812in}}%
\pgfpathlineto{\pgfqpoint{2.971105in}{1.212813in}}%
\pgfpathlineto{\pgfqpoint{2.950506in}{1.211075in}}%
\pgfusepath{stroke}%
\end{pgfscope}%
\begin{pgfscope}%
\pgfpathrectangle{\pgfqpoint{0.100000in}{0.100000in}}{\pgfqpoint{3.608454in}{2.310000in}}%
\pgfusepath{clip}%
\pgfsetbuttcap%
\pgfsetroundjoin%
\pgfsetlinewidth{0.010037pt}%
\definecolor{currentstroke}{rgb}{1.000000,1.000000,1.000000}%
\pgfsetstrokecolor{currentstroke}%
\pgfsetdash{}{0pt}%
\pgfpathmoveto{\pgfqpoint{3.303727in}{1.386181in}}%
\pgfpathlineto{\pgfqpoint{3.319710in}{1.393431in}}%
\pgfpathlineto{\pgfqpoint{3.306946in}{1.356091in}}%
\pgfpathlineto{\pgfqpoint{3.298505in}{1.336318in}}%
\pgfpathlineto{\pgfqpoint{3.292424in}{1.347508in}}%
\pgfpathlineto{\pgfqpoint{3.303237in}{1.374298in}}%
\pgfpathlineto{\pgfqpoint{3.303727in}{1.386181in}}%
\pgfusepath{stroke}%
\end{pgfscope}%
\begin{pgfscope}%
\pgfpathrectangle{\pgfqpoint{0.100000in}{0.100000in}}{\pgfqpoint{3.608454in}{2.310000in}}%
\pgfusepath{clip}%
\pgfsetbuttcap%
\pgfsetroundjoin%
\pgfsetlinewidth{0.010037pt}%
\definecolor{currentstroke}{rgb}{1.000000,1.000000,1.000000}%
\pgfsetstrokecolor{currentstroke}%
\pgfsetdash{}{0pt}%
\pgfpathmoveto{\pgfqpoint{2.461763in}{1.153105in}}%
\pgfpathlineto{\pgfqpoint{2.457903in}{1.152546in}}%
\pgfpathlineto{\pgfqpoint{2.454263in}{1.152291in}}%
\pgfpathlineto{\pgfqpoint{2.455821in}{1.141115in}}%
\pgfpathlineto{\pgfqpoint{2.446430in}{1.129849in}}%
\pgfpathlineto{\pgfqpoint{2.444579in}{1.112222in}}%
\pgfpathlineto{\pgfqpoint{2.402597in}{1.109117in}}%
\pgfpathlineto{\pgfqpoint{2.406257in}{1.117394in}}%
\pgfpathlineto{\pgfqpoint{2.421394in}{1.132509in}}%
\pgfpathlineto{\pgfqpoint{2.421813in}{1.141271in}}%
\pgfpathlineto{\pgfqpoint{2.415111in}{1.149567in}}%
\pgfpathlineto{\pgfqpoint{2.352626in}{1.146263in}}%
\pgfpathlineto{\pgfqpoint{2.243380in}{1.142795in}}%
\pgfpathlineto{\pgfqpoint{2.179448in}{1.141630in}}%
\pgfpathlineto{\pgfqpoint{2.131124in}{1.141195in}}%
\pgfpathlineto{\pgfqpoint{2.130758in}{1.181350in}}%
\pgfpathlineto{\pgfqpoint{2.130373in}{1.297922in}}%
\pgfpathlineto{\pgfqpoint{2.130810in}{1.348616in}}%
\pgfpathlineto{\pgfqpoint{2.117751in}{1.356337in}}%
\pgfpathlineto{\pgfqpoint{2.111314in}{1.369785in}}%
\pgfpathlineto{\pgfqpoint{2.104306in}{1.376240in}}%
\pgfpathlineto{\pgfqpoint{2.103512in}{1.386028in}}%
\pgfpathlineto{\pgfqpoint{2.110832in}{1.401521in}}%
\pgfpathlineto{\pgfqpoint{2.110523in}{1.410477in}}%
\pgfpathlineto{\pgfqpoint{2.100667in}{1.408407in}}%
\pgfpathlineto{\pgfqpoint{2.088084in}{1.418676in}}%
\pgfpathlineto{\pgfqpoint{2.077998in}{1.436700in}}%
\pgfpathlineto{\pgfqpoint{2.069537in}{1.445017in}}%
\pgfpathlineto{\pgfqpoint{2.060697in}{1.465454in}}%
\pgfpathlineto{\pgfqpoint{2.152519in}{1.464021in}}%
\pgfpathlineto{\pgfqpoint{2.243792in}{1.467029in}}%
\pgfpathlineto{\pgfqpoint{2.302315in}{1.470405in}}%
\pgfpathlineto{\pgfqpoint{2.320622in}{1.452274in}}%
\pgfpathlineto{\pgfqpoint{2.317277in}{1.433365in}}%
\pgfpathlineto{\pgfqpoint{2.322158in}{1.412077in}}%
\pgfpathlineto{\pgfqpoint{2.327573in}{1.401342in}}%
\pgfpathlineto{\pgfqpoint{2.344850in}{1.386576in}}%
\pgfpathlineto{\pgfqpoint{2.348630in}{1.380053in}}%
\pgfpathlineto{\pgfqpoint{2.367545in}{1.365354in}}%
\pgfpathlineto{\pgfqpoint{2.370287in}{1.348866in}}%
\pgfpathlineto{\pgfqpoint{2.375719in}{1.336702in}}%
\pgfpathlineto{\pgfqpoint{2.388555in}{1.343787in}}%
\pgfpathlineto{\pgfqpoint{2.399134in}{1.340547in}}%
\pgfpathlineto{\pgfqpoint{2.407122in}{1.331947in}}%
\pgfpathlineto{\pgfqpoint{2.404106in}{1.316728in}}%
\pgfpathlineto{\pgfqpoint{2.394301in}{1.296831in}}%
\pgfpathlineto{\pgfqpoint{2.394453in}{1.286467in}}%
\pgfpathlineto{\pgfqpoint{2.401780in}{1.277970in}}%
\pgfpathlineto{\pgfqpoint{2.417550in}{1.266715in}}%
\pgfpathlineto{\pgfqpoint{2.435708in}{1.257580in}}%
\pgfpathlineto{\pgfqpoint{2.440014in}{1.250265in}}%
\pgfpathlineto{\pgfqpoint{2.449301in}{1.246806in}}%
\pgfpathlineto{\pgfqpoint{2.449670in}{1.237099in}}%
\pgfpathlineto{\pgfqpoint{2.456543in}{1.224768in}}%
\pgfpathlineto{\pgfqpoint{2.451774in}{1.213813in}}%
\pgfpathlineto{\pgfqpoint{2.461108in}{1.196383in}}%
\pgfpathlineto{\pgfqpoint{2.468410in}{1.199844in}}%
\pgfpathlineto{\pgfqpoint{2.477364in}{1.192570in}}%
\pgfpathlineto{\pgfqpoint{2.474354in}{1.160178in}}%
\pgfpathlineto{\pgfqpoint{2.463799in}{1.162233in}}%
\pgfpathlineto{\pgfqpoint{2.461763in}{1.153105in}}%
\pgfusepath{stroke}%
\end{pgfscope}%
\begin{pgfscope}%
\pgfpathrectangle{\pgfqpoint{0.100000in}{0.100000in}}{\pgfqpoint{3.608454in}{2.310000in}}%
\pgfusepath{clip}%
\pgfsetbuttcap%
\pgfsetroundjoin%
\pgfsetlinewidth{0.010037pt}%
\definecolor{currentstroke}{rgb}{1.000000,1.000000,1.000000}%
\pgfsetstrokecolor{currentstroke}%
\pgfsetdash{}{0pt}%
\pgfpathmoveto{\pgfqpoint{0.843082in}{1.160088in}}%
\pgfpathlineto{\pgfqpoint{0.850725in}{1.176360in}}%
\pgfpathlineto{\pgfqpoint{0.848633in}{1.200785in}}%
\pgfpathlineto{\pgfqpoint{0.852293in}{1.207731in}}%
\pgfpathlineto{\pgfqpoint{0.852655in}{1.238156in}}%
\pgfpathlineto{\pgfqpoint{0.856124in}{1.246917in}}%
\pgfpathlineto{\pgfqpoint{0.868345in}{1.248280in}}%
\pgfpathlineto{\pgfqpoint{0.879687in}{1.244390in}}%
\pgfpathlineto{\pgfqpoint{0.884626in}{1.233679in}}%
\pgfpathlineto{\pgfqpoint{0.891552in}{1.234089in}}%
\pgfpathlineto{\pgfqpoint{0.900156in}{1.246360in}}%
\pgfpathlineto{\pgfqpoint{0.912593in}{1.306845in}}%
\pgfpathlineto{\pgfqpoint{0.983361in}{1.292233in}}%
\pgfpathlineto{\pgfqpoint{1.024434in}{1.284134in}}%
\pgfpathlineto{\pgfqpoint{1.133501in}{1.264961in}}%
\pgfpathlineto{\pgfqpoint{1.163676in}{1.258901in}}%
\pgfpathlineto{\pgfqpoint{1.222893in}{1.249723in}}%
\pgfpathlineto{\pgfqpoint{1.210751in}{1.171542in}}%
\pgfpathlineto{\pgfqpoint{1.198122in}{1.089991in}}%
\pgfpathlineto{\pgfqpoint{1.183580in}{0.998244in}}%
\pgfpathlineto{\pgfqpoint{1.167224in}{0.892929in}}%
\pgfpathlineto{\pgfqpoint{1.154011in}{0.806432in}}%
\pgfpathlineto{\pgfqpoint{1.059373in}{0.821460in}}%
\pgfpathlineto{\pgfqpoint{1.017791in}{0.828581in}}%
\pgfpathlineto{\pgfqpoint{0.999216in}{0.839702in}}%
\pgfpathlineto{\pgfqpoint{0.878105in}{0.912775in}}%
\pgfpathlineto{\pgfqpoint{0.787216in}{0.968230in}}%
\pgfpathlineto{\pgfqpoint{0.790243in}{0.978058in}}%
\pgfpathlineto{\pgfqpoint{0.797751in}{0.984935in}}%
\pgfpathlineto{\pgfqpoint{0.805560in}{0.983661in}}%
\pgfpathlineto{\pgfqpoint{0.816886in}{0.990883in}}%
\pgfpathlineto{\pgfqpoint{0.818669in}{1.001263in}}%
\pgfpathlineto{\pgfqpoint{0.804856in}{1.013854in}}%
\pgfpathlineto{\pgfqpoint{0.814680in}{1.038025in}}%
\pgfpathlineto{\pgfqpoint{0.824557in}{1.047321in}}%
\pgfpathlineto{\pgfqpoint{0.829245in}{1.058349in}}%
\pgfpathlineto{\pgfqpoint{0.832133in}{1.078570in}}%
\pgfpathlineto{\pgfqpoint{0.841395in}{1.087727in}}%
\pgfpathlineto{\pgfqpoint{0.860834in}{1.096874in}}%
\pgfpathlineto{\pgfqpoint{0.860405in}{1.104932in}}%
\pgfpathlineto{\pgfqpoint{0.849548in}{1.114934in}}%
\pgfpathlineto{\pgfqpoint{0.848070in}{1.135552in}}%
\pgfpathlineto{\pgfqpoint{0.840594in}{1.150624in}}%
\pgfpathlineto{\pgfqpoint{0.843082in}{1.160088in}}%
\pgfusepath{stroke}%
\end{pgfscope}%
\begin{pgfscope}%
\pgfpathrectangle{\pgfqpoint{0.100000in}{0.100000in}}{\pgfqpoint{3.608454in}{2.310000in}}%
\pgfusepath{clip}%
\pgfsetbuttcap%
\pgfsetroundjoin%
\pgfsetlinewidth{0.010037pt}%
\definecolor{currentstroke}{rgb}{1.000000,1.000000,1.000000}%
\pgfsetstrokecolor{currentstroke}%
\pgfsetdash{}{0pt}%
\pgfpathmoveto{\pgfqpoint{2.141369in}{0.914830in}}%
\pgfpathlineto{\pgfqpoint{2.123825in}{0.920256in}}%
\pgfpathlineto{\pgfqpoint{2.101346in}{0.937472in}}%
\pgfpathlineto{\pgfqpoint{2.091360in}{0.941172in}}%
\pgfpathlineto{\pgfqpoint{2.085016in}{0.933736in}}%
\pgfpathlineto{\pgfqpoint{2.077003in}{0.933361in}}%
\pgfpathlineto{\pgfqpoint{2.066866in}{0.939673in}}%
\pgfpathlineto{\pgfqpoint{2.050959in}{0.931585in}}%
\pgfpathlineto{\pgfqpoint{2.045296in}{0.935568in}}%
\pgfpathlineto{\pgfqpoint{2.029643in}{0.926132in}}%
\pgfpathlineto{\pgfqpoint{2.013130in}{0.927873in}}%
\pgfpathlineto{\pgfqpoint{2.000393in}{0.934005in}}%
\pgfpathlineto{\pgfqpoint{1.991473in}{0.931697in}}%
\pgfpathlineto{\pgfqpoint{1.977195in}{0.940369in}}%
\pgfpathlineto{\pgfqpoint{1.969253in}{0.925074in}}%
\pgfpathlineto{\pgfqpoint{1.961130in}{0.938228in}}%
\pgfpathlineto{\pgfqpoint{1.945087in}{0.933125in}}%
\pgfpathlineto{\pgfqpoint{1.931055in}{0.945619in}}%
\pgfpathlineto{\pgfqpoint{1.918793in}{0.935548in}}%
\pgfpathlineto{\pgfqpoint{1.912652in}{0.944802in}}%
\pgfpathlineto{\pgfqpoint{1.903801in}{0.947808in}}%
\pgfpathlineto{\pgfqpoint{1.898410in}{0.956730in}}%
\pgfpathlineto{\pgfqpoint{1.886834in}{0.959269in}}%
\pgfpathlineto{\pgfqpoint{1.880154in}{0.952553in}}%
\pgfpathlineto{\pgfqpoint{1.868804in}{0.961257in}}%
\pgfpathlineto{\pgfqpoint{1.863513in}{0.959270in}}%
\pgfpathlineto{\pgfqpoint{1.844702in}{0.966322in}}%
\pgfpathlineto{\pgfqpoint{1.832921in}{0.967109in}}%
\pgfpathlineto{\pgfqpoint{1.827647in}{0.982085in}}%
\pgfpathlineto{\pgfqpoint{1.818203in}{0.980220in}}%
\pgfpathlineto{\pgfqpoint{1.807385in}{0.983916in}}%
\pgfpathlineto{\pgfqpoint{1.800261in}{0.981780in}}%
\pgfpathlineto{\pgfqpoint{1.784985in}{0.998601in}}%
\pgfpathlineto{\pgfqpoint{1.780728in}{0.997501in}}%
\pgfpathlineto{\pgfqpoint{1.784594in}{1.065714in}}%
\pgfpathlineto{\pgfqpoint{1.788806in}{1.150144in}}%
\pgfpathlineto{\pgfqpoint{1.719672in}{1.154011in}}%
\pgfpathlineto{\pgfqpoint{1.651450in}{1.159169in}}%
\pgfpathlineto{\pgfqpoint{1.598743in}{1.163746in}}%
\pgfpathlineto{\pgfqpoint{1.602243in}{1.203632in}}%
\pgfpathlineto{\pgfqpoint{1.662648in}{1.197867in}}%
\pgfpathlineto{\pgfqpoint{1.731159in}{1.193571in}}%
\pgfpathlineto{\pgfqpoint{1.826575in}{1.187945in}}%
\pgfpathlineto{\pgfqpoint{1.883211in}{1.185760in}}%
\pgfpathlineto{\pgfqpoint{1.951062in}{1.183366in}}%
\pgfpathlineto{\pgfqpoint{2.043230in}{1.181753in}}%
\pgfpathlineto{\pgfqpoint{2.130758in}{1.181350in}}%
\pgfpathlineto{\pgfqpoint{2.131124in}{1.141195in}}%
\pgfpathlineto{\pgfqpoint{2.136037in}{1.110943in}}%
\pgfpathlineto{\pgfqpoint{2.143669in}{1.055067in}}%
\pgfpathlineto{\pgfqpoint{2.142095in}{0.959646in}}%
\pgfpathlineto{\pgfqpoint{2.141369in}{0.914830in}}%
\pgfusepath{stroke}%
\end{pgfscope}%
\begin{pgfscope}%
\pgfpathrectangle{\pgfqpoint{0.100000in}{0.100000in}}{\pgfqpoint{3.608454in}{2.310000in}}%
\pgfusepath{clip}%
\pgfsetbuttcap%
\pgfsetroundjoin%
\pgfsetlinewidth{0.010037pt}%
\definecolor{currentstroke}{rgb}{1.000000,1.000000,1.000000}%
\pgfsetstrokecolor{currentstroke}%
\pgfsetdash{}{0pt}%
\pgfpathmoveto{\pgfqpoint{2.798250in}{1.063020in}}%
\pgfpathlineto{\pgfqpoint{2.798302in}{1.080700in}}%
\pgfpathlineto{\pgfqpoint{2.813666in}{1.087489in}}%
\pgfpathlineto{\pgfqpoint{2.814310in}{1.098342in}}%
\pgfpathlineto{\pgfqpoint{2.828067in}{1.111600in}}%
\pgfpathlineto{\pgfqpoint{2.845262in}{1.114316in}}%
\pgfpathlineto{\pgfqpoint{2.860379in}{1.128927in}}%
\pgfpathlineto{\pgfqpoint{2.876166in}{1.135457in}}%
\pgfpathlineto{\pgfqpoint{2.888015in}{1.155063in}}%
\pgfpathlineto{\pgfqpoint{2.898807in}{1.153640in}}%
\pgfpathlineto{\pgfqpoint{2.921614in}{1.171504in}}%
\pgfpathlineto{\pgfqpoint{2.933658in}{1.171840in}}%
\pgfpathlineto{\pgfqpoint{2.938743in}{1.185452in}}%
\pgfpathlineto{\pgfqpoint{2.948316in}{1.194928in}}%
\pgfpathlineto{\pgfqpoint{2.950506in}{1.211075in}}%
\pgfpathlineto{\pgfqpoint{2.971105in}{1.212813in}}%
\pgfpathlineto{\pgfqpoint{3.054377in}{1.223812in}}%
\pgfpathlineto{\pgfqpoint{3.152426in}{1.240702in}}%
\pgfpathlineto{\pgfqpoint{3.223341in}{1.255217in}}%
\pgfpathlineto{\pgfqpoint{3.302627in}{1.271664in}}%
\pgfpathlineto{\pgfqpoint{3.325283in}{1.240558in}}%
\pgfpathlineto{\pgfqpoint{3.313011in}{1.242544in}}%
\pgfpathlineto{\pgfqpoint{3.297258in}{1.238706in}}%
\pgfpathlineto{\pgfqpoint{3.281735in}{1.222854in}}%
\pgfpathlineto{\pgfqpoint{3.270579in}{1.223994in}}%
\pgfpathlineto{\pgfqpoint{3.269168in}{1.214579in}}%
\pgfpathlineto{\pgfqpoint{3.289332in}{1.222025in}}%
\pgfpathlineto{\pgfqpoint{3.309624in}{1.225095in}}%
\pgfpathlineto{\pgfqpoint{3.317158in}{1.221010in}}%
\pgfpathlineto{\pgfqpoint{3.327297in}{1.225667in}}%
\pgfpathlineto{\pgfqpoint{3.332529in}{1.222404in}}%
\pgfpathlineto{\pgfqpoint{3.337160in}{1.206867in}}%
\pgfpathlineto{\pgfqpoint{3.327523in}{1.202055in}}%
\pgfpathlineto{\pgfqpoint{3.320908in}{1.183109in}}%
\pgfpathlineto{\pgfqpoint{3.313973in}{1.175732in}}%
\pgfpathlineto{\pgfqpoint{3.292714in}{1.177346in}}%
\pgfpathlineto{\pgfqpoint{3.293856in}{1.188471in}}%
\pgfpathlineto{\pgfqpoint{3.282246in}{1.183611in}}%
\pgfpathlineto{\pgfqpoint{3.279721in}{1.174333in}}%
\pgfpathlineto{\pgfqpoint{3.287679in}{1.164722in}}%
\pgfpathlineto{\pgfqpoint{3.290383in}{1.148409in}}%
\pgfpathlineto{\pgfqpoint{3.277360in}{1.139115in}}%
\pgfpathlineto{\pgfqpoint{3.291444in}{1.135846in}}%
\pgfpathlineto{\pgfqpoint{3.297817in}{1.142680in}}%
\pgfpathlineto{\pgfqpoint{3.311909in}{1.143512in}}%
\pgfpathlineto{\pgfqpoint{3.304667in}{1.128754in}}%
\pgfpathlineto{\pgfqpoint{3.295798in}{1.121643in}}%
\pgfpathlineto{\pgfqpoint{3.268968in}{1.114523in}}%
\pgfpathlineto{\pgfqpoint{3.241422in}{1.089539in}}%
\pgfpathlineto{\pgfqpoint{3.224436in}{1.064923in}}%
\pgfpathlineto{\pgfqpoint{3.224336in}{1.054925in}}%
\pgfpathlineto{\pgfqpoint{3.217551in}{1.041130in}}%
\pgfpathlineto{\pgfqpoint{3.182729in}{1.032053in}}%
\pgfpathlineto{\pgfqpoint{3.099917in}{1.091416in}}%
\pgfpathlineto{\pgfqpoint{3.026989in}{1.080701in}}%
\pgfpathlineto{\pgfqpoint{3.026378in}{1.090595in}}%
\pgfpathlineto{\pgfqpoint{3.015293in}{1.101737in}}%
\pgfpathlineto{\pgfqpoint{3.006902in}{1.104464in}}%
\pgfpathlineto{\pgfqpoint{2.927665in}{1.096231in}}%
\pgfpathlineto{\pgfqpoint{2.909453in}{1.090054in}}%
\pgfpathlineto{\pgfqpoint{2.876568in}{1.073696in}}%
\pgfpathlineto{\pgfqpoint{2.848149in}{1.069168in}}%
\pgfpathlineto{\pgfqpoint{2.798250in}{1.063020in}}%
\pgfusepath{stroke}%
\end{pgfscope}%
\begin{pgfscope}%
\pgfpathrectangle{\pgfqpoint{0.100000in}{0.100000in}}{\pgfqpoint{3.608454in}{2.310000in}}%
\pgfusepath{clip}%
\pgfsetbuttcap%
\pgfsetroundjoin%
\pgfsetlinewidth{0.010037pt}%
\definecolor{currentstroke}{rgb}{1.000000,1.000000,1.000000}%
\pgfsetstrokecolor{currentstroke}%
\pgfsetdash{}{0pt}%
\pgfpathmoveto{\pgfqpoint{2.798250in}{1.063020in}}%
\pgfpathlineto{\pgfqpoint{2.715320in}{1.053912in}}%
\pgfpathlineto{\pgfqpoint{2.639455in}{1.047089in}}%
\pgfpathlineto{\pgfqpoint{2.561482in}{1.042047in}}%
\pgfpathlineto{\pgfqpoint{2.548098in}{1.040107in}}%
\pgfpathlineto{\pgfqpoint{2.495536in}{1.036064in}}%
\pgfpathlineto{\pgfqpoint{2.411349in}{1.031151in}}%
\pgfpathlineto{\pgfqpoint{2.426327in}{1.043584in}}%
\pgfpathlineto{\pgfqpoint{2.422910in}{1.056671in}}%
\pgfpathlineto{\pgfqpoint{2.426220in}{1.072535in}}%
\pgfpathlineto{\pgfqpoint{2.431278in}{1.080004in}}%
\pgfpathlineto{\pgfqpoint{2.431088in}{1.090395in}}%
\pgfpathlineto{\pgfqpoint{2.444556in}{1.096930in}}%
\pgfpathlineto{\pgfqpoint{2.444579in}{1.112222in}}%
\pgfpathlineto{\pgfqpoint{2.446430in}{1.129849in}}%
\pgfpathlineto{\pgfqpoint{2.455821in}{1.141115in}}%
\pgfpathlineto{\pgfqpoint{2.454263in}{1.152291in}}%
\pgfpathlineto{\pgfqpoint{2.457903in}{1.152546in}}%
\pgfpathlineto{\pgfqpoint{2.461763in}{1.153105in}}%
\pgfpathlineto{\pgfqpoint{2.548849in}{1.158640in}}%
\pgfpathlineto{\pgfqpoint{2.545934in}{1.172945in}}%
\pgfpathlineto{\pgfqpoint{2.559948in}{1.171005in}}%
\pgfpathlineto{\pgfqpoint{2.644773in}{1.179396in}}%
\pgfpathlineto{\pgfqpoint{2.712861in}{1.183250in}}%
\pgfpathlineto{\pgfqpoint{2.754353in}{1.186739in}}%
\pgfpathlineto{\pgfqpoint{2.823062in}{1.193254in}}%
\pgfpathlineto{\pgfqpoint{2.824207in}{1.194550in}}%
\pgfpathlineto{\pgfqpoint{2.853565in}{1.197457in}}%
\pgfpathlineto{\pgfqpoint{2.950506in}{1.211075in}}%
\pgfpathlineto{\pgfqpoint{2.948316in}{1.194928in}}%
\pgfpathlineto{\pgfqpoint{2.938743in}{1.185452in}}%
\pgfpathlineto{\pgfqpoint{2.933658in}{1.171840in}}%
\pgfpathlineto{\pgfqpoint{2.921614in}{1.171504in}}%
\pgfpathlineto{\pgfqpoint{2.898807in}{1.153640in}}%
\pgfpathlineto{\pgfqpoint{2.888015in}{1.155063in}}%
\pgfpathlineto{\pgfqpoint{2.876166in}{1.135457in}}%
\pgfpathlineto{\pgfqpoint{2.860379in}{1.128927in}}%
\pgfpathlineto{\pgfqpoint{2.845262in}{1.114316in}}%
\pgfpathlineto{\pgfqpoint{2.828067in}{1.111600in}}%
\pgfpathlineto{\pgfqpoint{2.814310in}{1.098342in}}%
\pgfpathlineto{\pgfqpoint{2.813666in}{1.087489in}}%
\pgfpathlineto{\pgfqpoint{2.798302in}{1.080700in}}%
\pgfpathlineto{\pgfqpoint{2.798250in}{1.063020in}}%
\pgfusepath{stroke}%
\end{pgfscope}%
\begin{pgfscope}%
\pgfpathrectangle{\pgfqpoint{0.100000in}{0.100000in}}{\pgfqpoint{3.608454in}{2.310000in}}%
\pgfusepath{clip}%
\pgfsetbuttcap%
\pgfsetroundjoin%
\pgfsetlinewidth{0.010037pt}%
\definecolor{currentstroke}{rgb}{1.000000,1.000000,1.000000}%
\pgfsetstrokecolor{currentstroke}%
\pgfsetdash{}{0pt}%
\pgfpathmoveto{\pgfqpoint{1.327465in}{0.817498in}}%
\pgfpathlineto{\pgfqpoint{1.321550in}{0.827003in}}%
\pgfpathlineto{\pgfqpoint{1.323997in}{0.835189in}}%
\pgfpathlineto{\pgfqpoint{1.442877in}{0.821252in}}%
\pgfpathlineto{\pgfqpoint{1.563330in}{0.809245in}}%
\pgfpathlineto{\pgfqpoint{1.566842in}{0.849964in}}%
\pgfpathlineto{\pgfqpoint{1.584946in}{1.026087in}}%
\pgfpathlineto{\pgfqpoint{1.596844in}{1.163852in}}%
\pgfpathlineto{\pgfqpoint{1.598743in}{1.163746in}}%
\pgfpathlineto{\pgfqpoint{1.719672in}{1.154011in}}%
\pgfpathlineto{\pgfqpoint{1.788806in}{1.150144in}}%
\pgfpathlineto{\pgfqpoint{1.780728in}{0.997501in}}%
\pgfpathlineto{\pgfqpoint{1.784985in}{0.998601in}}%
\pgfpathlineto{\pgfqpoint{1.800261in}{0.981780in}}%
\pgfpathlineto{\pgfqpoint{1.807385in}{0.983916in}}%
\pgfpathlineto{\pgfqpoint{1.818203in}{0.980220in}}%
\pgfpathlineto{\pgfqpoint{1.827647in}{0.982085in}}%
\pgfpathlineto{\pgfqpoint{1.832921in}{0.967109in}}%
\pgfpathlineto{\pgfqpoint{1.844702in}{0.966322in}}%
\pgfpathlineto{\pgfqpoint{1.863513in}{0.959270in}}%
\pgfpathlineto{\pgfqpoint{1.868804in}{0.961257in}}%
\pgfpathlineto{\pgfqpoint{1.880154in}{0.952553in}}%
\pgfpathlineto{\pgfqpoint{1.886834in}{0.959269in}}%
\pgfpathlineto{\pgfqpoint{1.898410in}{0.956730in}}%
\pgfpathlineto{\pgfqpoint{1.903801in}{0.947808in}}%
\pgfpathlineto{\pgfqpoint{1.912652in}{0.944802in}}%
\pgfpathlineto{\pgfqpoint{1.918793in}{0.935548in}}%
\pgfpathlineto{\pgfqpoint{1.931055in}{0.945619in}}%
\pgfpathlineto{\pgfqpoint{1.945087in}{0.933125in}}%
\pgfpathlineto{\pgfqpoint{1.961130in}{0.938228in}}%
\pgfpathlineto{\pgfqpoint{1.969253in}{0.925074in}}%
\pgfpathlineto{\pgfqpoint{1.977195in}{0.940369in}}%
\pgfpathlineto{\pgfqpoint{1.991473in}{0.931697in}}%
\pgfpathlineto{\pgfqpoint{2.000393in}{0.934005in}}%
\pgfpathlineto{\pgfqpoint{2.013130in}{0.927873in}}%
\pgfpathlineto{\pgfqpoint{2.029643in}{0.926132in}}%
\pgfpathlineto{\pgfqpoint{2.045296in}{0.935568in}}%
\pgfpathlineto{\pgfqpoint{2.050959in}{0.931585in}}%
\pgfpathlineto{\pgfqpoint{2.066866in}{0.939673in}}%
\pgfpathlineto{\pgfqpoint{2.077003in}{0.933361in}}%
\pgfpathlineto{\pgfqpoint{2.085016in}{0.933736in}}%
\pgfpathlineto{\pgfqpoint{2.091360in}{0.941172in}}%
\pgfpathlineto{\pgfqpoint{2.101346in}{0.937472in}}%
\pgfpathlineto{\pgfqpoint{2.123825in}{0.920256in}}%
\pgfpathlineto{\pgfqpoint{2.141369in}{0.914830in}}%
\pgfpathlineto{\pgfqpoint{2.148404in}{0.908183in}}%
\pgfpathlineto{\pgfqpoint{2.157210in}{0.911803in}}%
\pgfpathlineto{\pgfqpoint{2.170554in}{0.909031in}}%
\pgfpathlineto{\pgfqpoint{2.171929in}{0.784895in}}%
\pgfpathlineto{\pgfqpoint{2.181197in}{0.777032in}}%
\pgfpathlineto{\pgfqpoint{2.188625in}{0.762543in}}%
\pgfpathlineto{\pgfqpoint{2.186014in}{0.752849in}}%
\pgfpathlineto{\pgfqpoint{2.191847in}{0.748745in}}%
\pgfpathlineto{\pgfqpoint{2.196588in}{0.730869in}}%
\pgfpathlineto{\pgfqpoint{2.205667in}{0.721296in}}%
\pgfpathlineto{\pgfqpoint{2.207701in}{0.700970in}}%
\pgfpathlineto{\pgfqpoint{2.206122in}{0.692365in}}%
\pgfpathlineto{\pgfqpoint{2.193781in}{0.669590in}}%
\pgfpathlineto{\pgfqpoint{2.192362in}{0.653216in}}%
\pgfpathlineto{\pgfqpoint{2.196548in}{0.649798in}}%
\pgfpathlineto{\pgfqpoint{2.196912in}{0.635452in}}%
\pgfpathlineto{\pgfqpoint{2.192657in}{0.626436in}}%
\pgfpathlineto{\pgfqpoint{2.186039in}{0.623040in}}%
\pgfpathlineto{\pgfqpoint{2.179559in}{0.611258in}}%
\pgfpathlineto{\pgfqpoint{2.187818in}{0.599847in}}%
\pgfpathlineto{\pgfqpoint{2.171786in}{0.599622in}}%
\pgfpathlineto{\pgfqpoint{2.128920in}{0.580009in}}%
\pgfpathlineto{\pgfqpoint{2.137104in}{0.591747in}}%
\pgfpathlineto{\pgfqpoint{2.127193in}{0.598075in}}%
\pgfpathlineto{\pgfqpoint{2.125123in}{0.608834in}}%
\pgfpathlineto{\pgfqpoint{2.105780in}{0.590093in}}%
\pgfpathlineto{\pgfqpoint{2.114364in}{0.577284in}}%
\pgfpathlineto{\pgfqpoint{2.102118in}{0.560974in}}%
\pgfpathlineto{\pgfqpoint{2.095528in}{0.561316in}}%
\pgfpathlineto{\pgfqpoint{2.089328in}{0.543553in}}%
\pgfpathlineto{\pgfqpoint{2.069722in}{0.529582in}}%
\pgfpathlineto{\pgfqpoint{2.051405in}{0.524547in}}%
\pgfpathlineto{\pgfqpoint{2.019457in}{0.511448in}}%
\pgfpathlineto{\pgfqpoint{2.016147in}{0.518755in}}%
\pgfpathlineto{\pgfqpoint{1.996246in}{0.512029in}}%
\pgfpathlineto{\pgfqpoint{2.008486in}{0.500577in}}%
\pgfpathlineto{\pgfqpoint{1.989179in}{0.490629in}}%
\pgfpathlineto{\pgfqpoint{1.968354in}{0.475608in}}%
\pgfpathlineto{\pgfqpoint{1.963353in}{0.482590in}}%
\pgfpathlineto{\pgfqpoint{1.946310in}{0.472125in}}%
\pgfpathlineto{\pgfqpoint{1.963014in}{0.468151in}}%
\pgfpathlineto{\pgfqpoint{1.950420in}{0.451696in}}%
\pgfpathlineto{\pgfqpoint{1.944272in}{0.456593in}}%
\pgfpathlineto{\pgfqpoint{1.936015in}{0.448727in}}%
\pgfpathlineto{\pgfqpoint{1.946306in}{0.441857in}}%
\pgfpathlineto{\pgfqpoint{1.940221in}{0.431815in}}%
\pgfpathlineto{\pgfqpoint{1.932743in}{0.408043in}}%
\pgfpathlineto{\pgfqpoint{1.921956in}{0.385147in}}%
\pgfpathlineto{\pgfqpoint{1.926799in}{0.370139in}}%
\pgfpathlineto{\pgfqpoint{1.935030in}{0.334756in}}%
\pgfpathlineto{\pgfqpoint{1.935771in}{0.320436in}}%
\pgfpathlineto{\pgfqpoint{1.948489in}{0.301654in}}%
\pgfpathlineto{\pgfqpoint{1.938682in}{0.302751in}}%
\pgfpathlineto{\pgfqpoint{1.929156in}{0.293276in}}%
\pgfpathlineto{\pgfqpoint{1.918000in}{0.300723in}}%
\pgfpathlineto{\pgfqpoint{1.913996in}{0.308152in}}%
\pgfpathlineto{\pgfqpoint{1.898118in}{0.311608in}}%
\pgfpathlineto{\pgfqpoint{1.873865in}{0.312032in}}%
\pgfpathlineto{\pgfqpoint{1.855994in}{0.326101in}}%
\pgfpathlineto{\pgfqpoint{1.839768in}{0.328451in}}%
\pgfpathlineto{\pgfqpoint{1.829925in}{0.339632in}}%
\pgfpathlineto{\pgfqpoint{1.809309in}{0.344132in}}%
\pgfpathlineto{\pgfqpoint{1.798036in}{0.380084in}}%
\pgfpathlineto{\pgfqpoint{1.786510in}{0.394486in}}%
\pgfpathlineto{\pgfqpoint{1.788468in}{0.408170in}}%
\pgfpathlineto{\pgfqpoint{1.781321in}{0.418171in}}%
\pgfpathlineto{\pgfqpoint{1.785806in}{0.431841in}}%
\pgfpathlineto{\pgfqpoint{1.782106in}{0.441858in}}%
\pgfpathlineto{\pgfqpoint{1.770524in}{0.446396in}}%
\pgfpathlineto{\pgfqpoint{1.759694in}{0.457947in}}%
\pgfpathlineto{\pgfqpoint{1.755748in}{0.473412in}}%
\pgfpathlineto{\pgfqpoint{1.745500in}{0.487466in}}%
\pgfpathlineto{\pgfqpoint{1.731859in}{0.498402in}}%
\pgfpathlineto{\pgfqpoint{1.719573in}{0.529770in}}%
\pgfpathlineto{\pgfqpoint{1.709652in}{0.564077in}}%
\pgfpathlineto{\pgfqpoint{1.701513in}{0.577641in}}%
\pgfpathlineto{\pgfqpoint{1.687401in}{0.589061in}}%
\pgfpathlineto{\pgfqpoint{1.683874in}{0.597350in}}%
\pgfpathlineto{\pgfqpoint{1.670661in}{0.602497in}}%
\pgfpathlineto{\pgfqpoint{1.657620in}{0.624460in}}%
\pgfpathlineto{\pgfqpoint{1.645710in}{0.622774in}}%
\pgfpathlineto{\pgfqpoint{1.633616in}{0.628259in}}%
\pgfpathlineto{\pgfqpoint{1.616474in}{0.627199in}}%
\pgfpathlineto{\pgfqpoint{1.599059in}{0.636251in}}%
\pgfpathlineto{\pgfqpoint{1.594148in}{0.627642in}}%
\pgfpathlineto{\pgfqpoint{1.573796in}{0.627427in}}%
\pgfpathlineto{\pgfqpoint{1.563435in}{0.611106in}}%
\pgfpathlineto{\pgfqpoint{1.554423in}{0.590899in}}%
\pgfpathlineto{\pgfqpoint{1.535259in}{0.569207in}}%
\pgfpathlineto{\pgfqpoint{1.513549in}{0.578727in}}%
\pgfpathlineto{\pgfqpoint{1.510468in}{0.585018in}}%
\pgfpathlineto{\pgfqpoint{1.497274in}{0.589816in}}%
\pgfpathlineto{\pgfqpoint{1.493221in}{0.596380in}}%
\pgfpathlineto{\pgfqpoint{1.475705in}{0.603005in}}%
\pgfpathlineto{\pgfqpoint{1.465907in}{0.616550in}}%
\pgfpathlineto{\pgfqpoint{1.454464in}{0.623084in}}%
\pgfpathlineto{\pgfqpoint{1.444618in}{0.634488in}}%
\pgfpathlineto{\pgfqpoint{1.436949in}{0.653800in}}%
\pgfpathlineto{\pgfqpoint{1.437807in}{0.680190in}}%
\pgfpathlineto{\pgfqpoint{1.428809in}{0.693526in}}%
\pgfpathlineto{\pgfqpoint{1.427774in}{0.707969in}}%
\pgfpathlineto{\pgfqpoint{1.407810in}{0.729594in}}%
\pgfpathlineto{\pgfqpoint{1.396194in}{0.734211in}}%
\pgfpathlineto{\pgfqpoint{1.383867in}{0.754389in}}%
\pgfpathlineto{\pgfqpoint{1.373364in}{0.762406in}}%
\pgfpathlineto{\pgfqpoint{1.360017in}{0.781928in}}%
\pgfpathlineto{\pgfqpoint{1.346361in}{0.790377in}}%
\pgfpathlineto{\pgfqpoint{1.337418in}{0.812026in}}%
\pgfpathlineto{\pgfqpoint{1.327465in}{0.817498in}}%
\pgfpathlineto{\pgfqpoint{1.327465in}{0.817498in}}%
\pgfusepath{stroke}%
\end{pgfscope}%
\begin{pgfscope}%
\pgfpathrectangle{\pgfqpoint{0.100000in}{0.100000in}}{\pgfqpoint{3.608454in}{2.310000in}}%
\pgfusepath{clip}%
\pgfsetbuttcap%
\pgfsetroundjoin%
\pgfsetlinewidth{0.010037pt}%
\definecolor{currentstroke}{rgb}{1.000000,1.000000,1.000000}%
\pgfsetstrokecolor{currentstroke}%
\pgfsetdash{}{0pt}%
\pgfpathmoveto{\pgfqpoint{1.222893in}{1.249723in}}%
\pgfpathlineto{\pgfqpoint{1.321411in}{1.235415in}}%
\pgfpathlineto{\pgfqpoint{1.359578in}{1.229535in}}%
\pgfpathlineto{\pgfqpoint{1.467164in}{1.216739in}}%
\pgfpathlineto{\pgfqpoint{1.539577in}{1.209185in}}%
\pgfpathlineto{\pgfqpoint{1.602243in}{1.203632in}}%
\pgfpathlineto{\pgfqpoint{1.598743in}{1.163746in}}%
\pgfpathlineto{\pgfqpoint{1.596844in}{1.163852in}}%
\pgfpathlineto{\pgfqpoint{1.591104in}{1.095394in}}%
\pgfpathlineto{\pgfqpoint{1.584946in}{1.026087in}}%
\pgfpathlineto{\pgfqpoint{1.577823in}{0.953532in}}%
\pgfpathlineto{\pgfqpoint{1.566842in}{0.849964in}}%
\pgfpathlineto{\pgfqpoint{1.563330in}{0.809245in}}%
\pgfpathlineto{\pgfqpoint{1.498773in}{0.815759in}}%
\pgfpathlineto{\pgfqpoint{1.442877in}{0.821252in}}%
\pgfpathlineto{\pgfqpoint{1.365572in}{0.830032in}}%
\pgfpathlineto{\pgfqpoint{1.323997in}{0.835189in}}%
\pgfpathlineto{\pgfqpoint{1.321550in}{0.827003in}}%
\pgfpathlineto{\pgfqpoint{1.327465in}{0.817498in}}%
\pgfpathlineto{\pgfqpoint{1.277499in}{0.823987in}}%
\pgfpathlineto{\pgfqpoint{1.215853in}{0.832835in}}%
\pgfpathlineto{\pgfqpoint{1.210243in}{0.797968in}}%
\pgfpathlineto{\pgfqpoint{1.154011in}{0.806432in}}%
\pgfpathlineto{\pgfqpoint{1.167224in}{0.892929in}}%
\pgfpathlineto{\pgfqpoint{1.183580in}{0.998244in}}%
\pgfpathlineto{\pgfqpoint{1.198122in}{1.089991in}}%
\pgfpathlineto{\pgfqpoint{1.210751in}{1.171542in}}%
\pgfpathlineto{\pgfqpoint{1.222893in}{1.249723in}}%
\pgfusepath{stroke}%
\end{pgfscope}%
\begin{pgfscope}%
\pgfpathrectangle{\pgfqpoint{0.100000in}{0.100000in}}{\pgfqpoint{3.608454in}{2.310000in}}%
\pgfusepath{clip}%
\pgfsetbuttcap%
\pgfsetroundjoin%
\pgfsetlinewidth{0.010037pt}%
\definecolor{currentstroke}{rgb}{1.000000,1.000000,1.000000}%
\pgfsetstrokecolor{currentstroke}%
\pgfsetdash{}{0pt}%
\pgfpathmoveto{\pgfqpoint{2.789338in}{0.742945in}}%
\pgfpathlineto{\pgfqpoint{2.695351in}{0.732506in}}%
\pgfpathlineto{\pgfqpoint{2.612503in}{0.726074in}}%
\pgfpathlineto{\pgfqpoint{2.611467in}{0.715921in}}%
\pgfpathlineto{\pgfqpoint{2.619077in}{0.706258in}}%
\pgfpathlineto{\pgfqpoint{2.628365in}{0.700581in}}%
\pgfpathlineto{\pgfqpoint{2.628224in}{0.685633in}}%
\pgfpathlineto{\pgfqpoint{2.625763in}{0.675646in}}%
\pgfpathlineto{\pgfqpoint{2.617565in}{0.668449in}}%
\pgfpathlineto{\pgfqpoint{2.606139in}{0.669211in}}%
\pgfpathlineto{\pgfqpoint{2.595336in}{0.678128in}}%
\pgfpathlineto{\pgfqpoint{2.593410in}{0.693997in}}%
\pgfpathlineto{\pgfqpoint{2.585364in}{0.703227in}}%
\pgfpathlineto{\pgfqpoint{2.579887in}{0.670195in}}%
\pgfpathlineto{\pgfqpoint{2.561278in}{0.673334in}}%
\pgfpathlineto{\pgfqpoint{2.554807in}{0.731095in}}%
\pgfpathlineto{\pgfqpoint{2.547798in}{0.791968in}}%
\pgfpathlineto{\pgfqpoint{2.550401in}{0.879794in}}%
\pgfpathlineto{\pgfqpoint{2.552017in}{0.940061in}}%
\pgfpathlineto{\pgfqpoint{2.555453in}{1.032015in}}%
\pgfpathlineto{\pgfqpoint{2.548098in}{1.040107in}}%
\pgfpathlineto{\pgfqpoint{2.561482in}{1.042047in}}%
\pgfpathlineto{\pgfqpoint{2.639455in}{1.047089in}}%
\pgfpathlineto{\pgfqpoint{2.715320in}{1.053912in}}%
\pgfpathlineto{\pgfqpoint{2.735280in}{0.983998in}}%
\pgfpathlineto{\pgfqpoint{2.748892in}{0.932681in}}%
\pgfpathlineto{\pgfqpoint{2.761093in}{0.889718in}}%
\pgfpathlineto{\pgfqpoint{2.768151in}{0.872419in}}%
\pgfpathlineto{\pgfqpoint{2.779261in}{0.856321in}}%
\pgfpathlineto{\pgfqpoint{2.777467in}{0.848128in}}%
\pgfpathlineto{\pgfqpoint{2.785404in}{0.844137in}}%
\pgfpathlineto{\pgfqpoint{2.776025in}{0.831718in}}%
\pgfpathlineto{\pgfqpoint{2.772856in}{0.809602in}}%
\pgfpathlineto{\pgfqpoint{2.782023in}{0.783735in}}%
\pgfpathlineto{\pgfqpoint{2.780750in}{0.757706in}}%
\pgfpathlineto{\pgfqpoint{2.789338in}{0.742945in}}%
\pgfusepath{stroke}%
\end{pgfscope}%
\begin{pgfscope}%
\pgfpathrectangle{\pgfqpoint{0.100000in}{0.100000in}}{\pgfqpoint{3.608454in}{2.310000in}}%
\pgfusepath{clip}%
\pgfsetbuttcap%
\pgfsetroundjoin%
\pgfsetlinewidth{0.010037pt}%
\definecolor{currentstroke}{rgb}{1.000000,1.000000,1.000000}%
\pgfsetstrokecolor{currentstroke}%
\pgfsetdash{}{0pt}%
\pgfpathmoveto{\pgfqpoint{2.561278in}{0.673334in}}%
\pgfpathlineto{\pgfqpoint{2.542181in}{0.667869in}}%
\pgfpathlineto{\pgfqpoint{2.527946in}{0.671378in}}%
\pgfpathlineto{\pgfqpoint{2.501511in}{0.662970in}}%
\pgfpathlineto{\pgfqpoint{2.481574in}{0.652140in}}%
\pgfpathlineto{\pgfqpoint{2.473394in}{0.671709in}}%
\pgfpathlineto{\pgfqpoint{2.464961in}{0.679911in}}%
\pgfpathlineto{\pgfqpoint{2.460893in}{0.689726in}}%
\pgfpathlineto{\pgfqpoint{2.466856in}{0.716291in}}%
\pgfpathlineto{\pgfqpoint{2.410356in}{0.712812in}}%
\pgfpathlineto{\pgfqpoint{2.337092in}{0.709636in}}%
\pgfpathlineto{\pgfqpoint{2.341451in}{0.716248in}}%
\pgfpathlineto{\pgfqpoint{2.336063in}{0.728737in}}%
\pgfpathlineto{\pgfqpoint{2.341432in}{0.731282in}}%
\pgfpathlineto{\pgfqpoint{2.340241in}{0.743282in}}%
\pgfpathlineto{\pgfqpoint{2.349598in}{0.754918in}}%
\pgfpathlineto{\pgfqpoint{2.354717in}{0.777508in}}%
\pgfpathlineto{\pgfqpoint{2.371843in}{0.792391in}}%
\pgfpathlineto{\pgfqpoint{2.365495in}{0.806845in}}%
\pgfpathlineto{\pgfqpoint{2.377639in}{0.815165in}}%
\pgfpathlineto{\pgfqpoint{2.367164in}{0.830227in}}%
\pgfpathlineto{\pgfqpoint{2.359582in}{0.864708in}}%
\pgfpathlineto{\pgfqpoint{2.362458in}{0.870969in}}%
\pgfpathlineto{\pgfqpoint{2.366997in}{0.882978in}}%
\pgfpathlineto{\pgfqpoint{2.362772in}{0.895582in}}%
\pgfpathlineto{\pgfqpoint{2.364838in}{0.901316in}}%
\pgfpathlineto{\pgfqpoint{2.357405in}{0.922979in}}%
\pgfpathlineto{\pgfqpoint{2.379349in}{0.961875in}}%
\pgfpathlineto{\pgfqpoint{2.380370in}{0.972228in}}%
\pgfpathlineto{\pgfqpoint{2.390941in}{0.979756in}}%
\pgfpathlineto{\pgfqpoint{2.402220in}{1.004607in}}%
\pgfpathlineto{\pgfqpoint{2.401684in}{1.014709in}}%
\pgfpathlineto{\pgfqpoint{2.415733in}{1.025035in}}%
\pgfpathlineto{\pgfqpoint{2.411349in}{1.031151in}}%
\pgfpathlineto{\pgfqpoint{2.495536in}{1.036064in}}%
\pgfpathlineto{\pgfqpoint{2.548098in}{1.040107in}}%
\pgfpathlineto{\pgfqpoint{2.555453in}{1.032015in}}%
\pgfpathlineto{\pgfqpoint{2.552017in}{0.940061in}}%
\pgfpathlineto{\pgfqpoint{2.550401in}{0.879794in}}%
\pgfpathlineto{\pgfqpoint{2.547798in}{0.791968in}}%
\pgfpathlineto{\pgfqpoint{2.554807in}{0.731095in}}%
\pgfpathlineto{\pgfqpoint{2.561278in}{0.673334in}}%
\pgfusepath{stroke}%
\end{pgfscope}%
\begin{pgfscope}%
\pgfpathrectangle{\pgfqpoint{0.100000in}{0.100000in}}{\pgfqpoint{3.608454in}{2.310000in}}%
\pgfusepath{clip}%
\pgfsetbuttcap%
\pgfsetroundjoin%
\pgfsetlinewidth{0.010037pt}%
\definecolor{currentstroke}{rgb}{1.000000,1.000000,1.000000}%
\pgfsetstrokecolor{currentstroke}%
\pgfsetdash{}{0pt}%
\pgfpathmoveto{\pgfqpoint{2.715320in}{1.053912in}}%
\pgfpathlineto{\pgfqpoint{2.798250in}{1.063020in}}%
\pgfpathlineto{\pgfqpoint{2.848149in}{1.069168in}}%
\pgfpathlineto{\pgfqpoint{2.876568in}{1.073696in}}%
\pgfpathlineto{\pgfqpoint{2.864796in}{1.055128in}}%
\pgfpathlineto{\pgfqpoint{2.864803in}{1.046374in}}%
\pgfpathlineto{\pgfqpoint{2.876921in}{1.041653in}}%
\pgfpathlineto{\pgfqpoint{2.885162in}{1.034031in}}%
\pgfpathlineto{\pgfqpoint{2.897615in}{1.033089in}}%
\pgfpathlineto{\pgfqpoint{2.909255in}{1.011633in}}%
\pgfpathlineto{\pgfqpoint{2.921882in}{0.996503in}}%
\pgfpathlineto{\pgfqpoint{2.944306in}{0.983786in}}%
\pgfpathlineto{\pgfqpoint{2.950679in}{0.973643in}}%
\pgfpathlineto{\pgfqpoint{2.970676in}{0.962674in}}%
\pgfpathlineto{\pgfqpoint{2.972180in}{0.954719in}}%
\pgfpathlineto{\pgfqpoint{2.988402in}{0.938901in}}%
\pgfpathlineto{\pgfqpoint{3.000827in}{0.934397in}}%
\pgfpathlineto{\pgfqpoint{3.009626in}{0.919461in}}%
\pgfpathlineto{\pgfqpoint{3.013522in}{0.902673in}}%
\pgfpathlineto{\pgfqpoint{3.026428in}{0.896184in}}%
\pgfpathlineto{\pgfqpoint{3.036819in}{0.878138in}}%
\pgfpathlineto{\pgfqpoint{3.040169in}{0.864866in}}%
\pgfpathlineto{\pgfqpoint{3.055395in}{0.859218in}}%
\pgfpathlineto{\pgfqpoint{3.042619in}{0.834754in}}%
\pgfpathlineto{\pgfqpoint{3.037811in}{0.820297in}}%
\pgfpathlineto{\pgfqpoint{3.040948in}{0.813520in}}%
\pgfpathlineto{\pgfqpoint{3.035470in}{0.802258in}}%
\pgfpathlineto{\pgfqpoint{3.033142in}{0.786687in}}%
\pgfpathlineto{\pgfqpoint{3.023401in}{0.783787in}}%
\pgfpathlineto{\pgfqpoint{3.028545in}{0.769530in}}%
\pgfpathlineto{\pgfqpoint{3.028216in}{0.751443in}}%
\pgfpathlineto{\pgfqpoint{3.010902in}{0.752121in}}%
\pgfpathlineto{\pgfqpoint{2.998840in}{0.755421in}}%
\pgfpathlineto{\pgfqpoint{2.993620in}{0.749287in}}%
\pgfpathlineto{\pgfqpoint{2.997515in}{0.734795in}}%
\pgfpathlineto{\pgfqpoint{2.996668in}{0.717946in}}%
\pgfpathlineto{\pgfqpoint{2.989059in}{0.716657in}}%
\pgfpathlineto{\pgfqpoint{2.982903in}{0.732393in}}%
\pgfpathlineto{\pgfqpoint{2.920222in}{0.728215in}}%
\pgfpathlineto{\pgfqpoint{2.841186in}{0.723903in}}%
\pgfpathlineto{\pgfqpoint{2.801326in}{0.721097in}}%
\pgfpathlineto{\pgfqpoint{2.789338in}{0.742945in}}%
\pgfpathlineto{\pgfqpoint{2.780750in}{0.757706in}}%
\pgfpathlineto{\pgfqpoint{2.782023in}{0.783735in}}%
\pgfpathlineto{\pgfqpoint{2.772856in}{0.809602in}}%
\pgfpathlineto{\pgfqpoint{2.776025in}{0.831718in}}%
\pgfpathlineto{\pgfqpoint{2.785404in}{0.844137in}}%
\pgfpathlineto{\pgfqpoint{2.777467in}{0.848128in}}%
\pgfpathlineto{\pgfqpoint{2.779261in}{0.856321in}}%
\pgfpathlineto{\pgfqpoint{2.768151in}{0.872419in}}%
\pgfpathlineto{\pgfqpoint{2.761093in}{0.889718in}}%
\pgfpathlineto{\pgfqpoint{2.748892in}{0.932681in}}%
\pgfpathlineto{\pgfqpoint{2.735280in}{0.983998in}}%
\pgfpathlineto{\pgfqpoint{2.715320in}{1.053912in}}%
\pgfusepath{stroke}%
\end{pgfscope}%
\begin{pgfscope}%
\pgfpathrectangle{\pgfqpoint{0.100000in}{0.100000in}}{\pgfqpoint{3.608454in}{2.310000in}}%
\pgfusepath{clip}%
\pgfsetbuttcap%
\pgfsetroundjoin%
\pgfsetlinewidth{0.010037pt}%
\definecolor{currentstroke}{rgb}{1.000000,1.000000,1.000000}%
\pgfsetstrokecolor{currentstroke}%
\pgfsetdash{}{0pt}%
\pgfpathmoveto{\pgfqpoint{2.876568in}{1.073696in}}%
\pgfpathlineto{\pgfqpoint{2.909453in}{1.090054in}}%
\pgfpathlineto{\pgfqpoint{2.927665in}{1.096231in}}%
\pgfpathlineto{\pgfqpoint{3.006902in}{1.104464in}}%
\pgfpathlineto{\pgfqpoint{3.015293in}{1.101737in}}%
\pgfpathlineto{\pgfqpoint{3.026378in}{1.090595in}}%
\pgfpathlineto{\pgfqpoint{3.026989in}{1.080701in}}%
\pgfpathlineto{\pgfqpoint{3.099917in}{1.091416in}}%
\pgfpathlineto{\pgfqpoint{3.182729in}{1.032053in}}%
\pgfpathlineto{\pgfqpoint{3.167199in}{1.015867in}}%
\pgfpathlineto{\pgfqpoint{3.145890in}{0.978275in}}%
\pgfpathlineto{\pgfqpoint{3.151955in}{0.970196in}}%
\pgfpathlineto{\pgfqpoint{3.140588in}{0.954516in}}%
\pgfpathlineto{\pgfqpoint{3.129280in}{0.952741in}}%
\pgfpathlineto{\pgfqpoint{3.129238in}{0.943317in}}%
\pgfpathlineto{\pgfqpoint{3.121059in}{0.933457in}}%
\pgfpathlineto{\pgfqpoint{3.110879in}{0.931458in}}%
\pgfpathlineto{\pgfqpoint{3.113100in}{0.922697in}}%
\pgfpathlineto{\pgfqpoint{3.107420in}{0.915971in}}%
\pgfpathlineto{\pgfqpoint{3.093832in}{0.910161in}}%
\pgfpathlineto{\pgfqpoint{3.077270in}{0.896945in}}%
\pgfpathlineto{\pgfqpoint{3.080396in}{0.888394in}}%
\pgfpathlineto{\pgfqpoint{3.069973in}{0.883025in}}%
\pgfpathlineto{\pgfqpoint{3.061181in}{0.888670in}}%
\pgfpathlineto{\pgfqpoint{3.054753in}{0.864128in}}%
\pgfpathlineto{\pgfqpoint{3.040169in}{0.864866in}}%
\pgfpathlineto{\pgfqpoint{3.036819in}{0.878138in}}%
\pgfpathlineto{\pgfqpoint{3.026428in}{0.896184in}}%
\pgfpathlineto{\pgfqpoint{3.013522in}{0.902673in}}%
\pgfpathlineto{\pgfqpoint{3.009626in}{0.919461in}}%
\pgfpathlineto{\pgfqpoint{3.000827in}{0.934397in}}%
\pgfpathlineto{\pgfqpoint{2.988402in}{0.938901in}}%
\pgfpathlineto{\pgfqpoint{2.972180in}{0.954719in}}%
\pgfpathlineto{\pgfqpoint{2.970676in}{0.962674in}}%
\pgfpathlineto{\pgfqpoint{2.950679in}{0.973643in}}%
\pgfpathlineto{\pgfqpoint{2.944306in}{0.983786in}}%
\pgfpathlineto{\pgfqpoint{2.921882in}{0.996503in}}%
\pgfpathlineto{\pgfqpoint{2.909255in}{1.011633in}}%
\pgfpathlineto{\pgfqpoint{2.897615in}{1.033089in}}%
\pgfpathlineto{\pgfqpoint{2.885162in}{1.034031in}}%
\pgfpathlineto{\pgfqpoint{2.876921in}{1.041653in}}%
\pgfpathlineto{\pgfqpoint{2.864803in}{1.046374in}}%
\pgfpathlineto{\pgfqpoint{2.864796in}{1.055128in}}%
\pgfpathlineto{\pgfqpoint{2.876568in}{1.073696in}}%
\pgfusepath{stroke}%
\end{pgfscope}%
\begin{pgfscope}%
\pgfpathrectangle{\pgfqpoint{0.100000in}{0.100000in}}{\pgfqpoint{3.608454in}{2.310000in}}%
\pgfusepath{clip}%
\pgfsetbuttcap%
\pgfsetroundjoin%
\pgfsetlinewidth{0.010037pt}%
\definecolor{currentstroke}{rgb}{1.000000,1.000000,1.000000}%
\pgfsetstrokecolor{currentstroke}%
\pgfsetdash{}{0pt}%
\pgfpathmoveto{\pgfqpoint{2.131124in}{1.141195in}}%
\pgfpathlineto{\pgfqpoint{2.179448in}{1.141630in}}%
\pgfpathlineto{\pgfqpoint{2.243380in}{1.142795in}}%
\pgfpathlineto{\pgfqpoint{2.352626in}{1.146263in}}%
\pgfpathlineto{\pgfqpoint{2.415111in}{1.149567in}}%
\pgfpathlineto{\pgfqpoint{2.421813in}{1.141271in}}%
\pgfpathlineto{\pgfqpoint{2.421394in}{1.132509in}}%
\pgfpathlineto{\pgfqpoint{2.406257in}{1.117394in}}%
\pgfpathlineto{\pgfqpoint{2.402597in}{1.109117in}}%
\pgfpathlineto{\pgfqpoint{2.444579in}{1.112222in}}%
\pgfpathlineto{\pgfqpoint{2.444556in}{1.096930in}}%
\pgfpathlineto{\pgfqpoint{2.431088in}{1.090395in}}%
\pgfpathlineto{\pgfqpoint{2.431278in}{1.080004in}}%
\pgfpathlineto{\pgfqpoint{2.426220in}{1.072535in}}%
\pgfpathlineto{\pgfqpoint{2.422910in}{1.056671in}}%
\pgfpathlineto{\pgfqpoint{2.426327in}{1.043584in}}%
\pgfpathlineto{\pgfqpoint{2.411349in}{1.031151in}}%
\pgfpathlineto{\pgfqpoint{2.415733in}{1.025035in}}%
\pgfpathlineto{\pgfqpoint{2.401684in}{1.014709in}}%
\pgfpathlineto{\pgfqpoint{2.402220in}{1.004607in}}%
\pgfpathlineto{\pgfqpoint{2.390941in}{0.979756in}}%
\pgfpathlineto{\pgfqpoint{2.380370in}{0.972228in}}%
\pgfpathlineto{\pgfqpoint{2.379349in}{0.961875in}}%
\pgfpathlineto{\pgfqpoint{2.357405in}{0.922979in}}%
\pgfpathlineto{\pgfqpoint{2.364838in}{0.901316in}}%
\pgfpathlineto{\pgfqpoint{2.362772in}{0.895582in}}%
\pgfpathlineto{\pgfqpoint{2.366997in}{0.882978in}}%
\pgfpathlineto{\pgfqpoint{2.362458in}{0.870969in}}%
\pgfpathlineto{\pgfqpoint{2.302450in}{0.868492in}}%
\pgfpathlineto{\pgfqpoint{2.224548in}{0.867208in}}%
\pgfpathlineto{\pgfqpoint{2.170816in}{0.866723in}}%
\pgfpathlineto{\pgfqpoint{2.170554in}{0.909031in}}%
\pgfpathlineto{\pgfqpoint{2.157210in}{0.911803in}}%
\pgfpathlineto{\pgfqpoint{2.148404in}{0.908183in}}%
\pgfpathlineto{\pgfqpoint{2.141369in}{0.914830in}}%
\pgfpathlineto{\pgfqpoint{2.142095in}{0.959646in}}%
\pgfpathlineto{\pgfqpoint{2.143669in}{1.055067in}}%
\pgfpathlineto{\pgfqpoint{2.136037in}{1.110943in}}%
\pgfpathlineto{\pgfqpoint{2.131124in}{1.141195in}}%
\pgfusepath{stroke}%
\end{pgfscope}%
\begin{pgfscope}%
\pgfpathrectangle{\pgfqpoint{0.100000in}{0.100000in}}{\pgfqpoint{3.608454in}{2.310000in}}%
\pgfusepath{clip}%
\pgfsetbuttcap%
\pgfsetroundjoin%
\pgfsetlinewidth{0.010037pt}%
\definecolor{currentstroke}{rgb}{1.000000,1.000000,1.000000}%
\pgfsetstrokecolor{currentstroke}%
\pgfsetdash{}{0pt}%
\pgfpathmoveto{\pgfqpoint{2.170816in}{0.866723in}}%
\pgfpathlineto{\pgfqpoint{2.224548in}{0.867208in}}%
\pgfpathlineto{\pgfqpoint{2.302450in}{0.868492in}}%
\pgfpathlineto{\pgfqpoint{2.362458in}{0.870969in}}%
\pgfpathlineto{\pgfqpoint{2.359582in}{0.864708in}}%
\pgfpathlineto{\pgfqpoint{2.367164in}{0.830227in}}%
\pgfpathlineto{\pgfqpoint{2.377639in}{0.815165in}}%
\pgfpathlineto{\pgfqpoint{2.365495in}{0.806845in}}%
\pgfpathlineto{\pgfqpoint{2.371843in}{0.792391in}}%
\pgfpathlineto{\pgfqpoint{2.354717in}{0.777508in}}%
\pgfpathlineto{\pgfqpoint{2.349598in}{0.754918in}}%
\pgfpathlineto{\pgfqpoint{2.340241in}{0.743282in}}%
\pgfpathlineto{\pgfqpoint{2.341432in}{0.731282in}}%
\pgfpathlineto{\pgfqpoint{2.336063in}{0.728737in}}%
\pgfpathlineto{\pgfqpoint{2.341451in}{0.716248in}}%
\pgfpathlineto{\pgfqpoint{2.337092in}{0.709636in}}%
\pgfpathlineto{\pgfqpoint{2.410356in}{0.712812in}}%
\pgfpathlineto{\pgfqpoint{2.466856in}{0.716291in}}%
\pgfpathlineto{\pgfqpoint{2.460893in}{0.689726in}}%
\pgfpathlineto{\pgfqpoint{2.464961in}{0.679911in}}%
\pgfpathlineto{\pgfqpoint{2.473394in}{0.671709in}}%
\pgfpathlineto{\pgfqpoint{2.481574in}{0.652140in}}%
\pgfpathlineto{\pgfqpoint{2.455735in}{0.656649in}}%
\pgfpathlineto{\pgfqpoint{2.446204in}{0.664060in}}%
\pgfpathlineto{\pgfqpoint{2.434847in}{0.664413in}}%
\pgfpathlineto{\pgfqpoint{2.422908in}{0.648183in}}%
\pgfpathlineto{\pgfqpoint{2.425288in}{0.640793in}}%
\pgfpathlineto{\pgfqpoint{2.463429in}{0.636322in}}%
\pgfpathlineto{\pgfqpoint{2.473414in}{0.627827in}}%
\pgfpathlineto{\pgfqpoint{2.489663in}{0.623456in}}%
\pgfpathlineto{\pgfqpoint{2.482567in}{0.613398in}}%
\pgfpathlineto{\pgfqpoint{2.470611in}{0.604627in}}%
\pgfpathlineto{\pgfqpoint{2.502169in}{0.584902in}}%
\pgfpathlineto{\pgfqpoint{2.516869in}{0.581820in}}%
\pgfpathlineto{\pgfqpoint{2.522982in}{0.565696in}}%
\pgfpathlineto{\pgfqpoint{2.509142in}{0.562730in}}%
\pgfpathlineto{\pgfqpoint{2.483128in}{0.583009in}}%
\pgfpathlineto{\pgfqpoint{2.472958in}{0.585828in}}%
\pgfpathlineto{\pgfqpoint{2.468036in}{0.593839in}}%
\pgfpathlineto{\pgfqpoint{2.452950in}{0.590551in}}%
\pgfpathlineto{\pgfqpoint{2.448226in}{0.580228in}}%
\pgfpathlineto{\pgfqpoint{2.451218in}{0.568725in}}%
\pgfpathlineto{\pgfqpoint{2.441061in}{0.561938in}}%
\pgfpathlineto{\pgfqpoint{2.431792in}{0.578660in}}%
\pgfpathlineto{\pgfqpoint{2.415557in}{0.573674in}}%
\pgfpathlineto{\pgfqpoint{2.409474in}{0.563696in}}%
\pgfpathlineto{\pgfqpoint{2.397944in}{0.566534in}}%
\pgfpathlineto{\pgfqpoint{2.390573in}{0.579135in}}%
\pgfpathlineto{\pgfqpoint{2.371016in}{0.583295in}}%
\pgfpathlineto{\pgfqpoint{2.367323in}{0.589852in}}%
\pgfpathlineto{\pgfqpoint{2.346947in}{0.601273in}}%
\pgfpathlineto{\pgfqpoint{2.341889in}{0.611259in}}%
\pgfpathlineto{\pgfqpoint{2.324822in}{0.607178in}}%
\pgfpathlineto{\pgfqpoint{2.326990in}{0.616320in}}%
\pgfpathlineto{\pgfqpoint{2.305777in}{0.606942in}}%
\pgfpathlineto{\pgfqpoint{2.311475in}{0.597211in}}%
\pgfpathlineto{\pgfqpoint{2.295094in}{0.591458in}}%
\pgfpathlineto{\pgfqpoint{2.273396in}{0.594620in}}%
\pgfpathlineto{\pgfqpoint{2.229480in}{0.609663in}}%
\pgfpathlineto{\pgfqpoint{2.195594in}{0.606677in}}%
\pgfpathlineto{\pgfqpoint{2.192657in}{0.626436in}}%
\pgfpathlineto{\pgfqpoint{2.196912in}{0.635452in}}%
\pgfpathlineto{\pgfqpoint{2.196548in}{0.649798in}}%
\pgfpathlineto{\pgfqpoint{2.192362in}{0.653216in}}%
\pgfpathlineto{\pgfqpoint{2.193781in}{0.669590in}}%
\pgfpathlineto{\pgfqpoint{2.206122in}{0.692365in}}%
\pgfpathlineto{\pgfqpoint{2.207701in}{0.700970in}}%
\pgfpathlineto{\pgfqpoint{2.205667in}{0.721296in}}%
\pgfpathlineto{\pgfqpoint{2.196588in}{0.730869in}}%
\pgfpathlineto{\pgfqpoint{2.191847in}{0.748745in}}%
\pgfpathlineto{\pgfqpoint{2.186014in}{0.752849in}}%
\pgfpathlineto{\pgfqpoint{2.188625in}{0.762543in}}%
\pgfpathlineto{\pgfqpoint{2.181197in}{0.777032in}}%
\pgfpathlineto{\pgfqpoint{2.171929in}{0.784895in}}%
\pgfpathlineto{\pgfqpoint{2.170816in}{0.866723in}}%
\pgfusepath{stroke}%
\end{pgfscope}%
\begin{pgfscope}%
\pgfpathrectangle{\pgfqpoint{0.100000in}{0.100000in}}{\pgfqpoint{3.608454in}{2.310000in}}%
\pgfusepath{clip}%
\pgfsetbuttcap%
\pgfsetroundjoin%
\pgfsetlinewidth{0.010037pt}%
\definecolor{currentstroke}{rgb}{1.000000,1.000000,1.000000}%
\pgfsetstrokecolor{currentstroke}%
\pgfsetdash{}{0pt}%
\pgfpathmoveto{\pgfqpoint{2.314523in}{0.596466in}}%
\pgfpathlineto{\pgfqpoint{2.322296in}{0.601090in}}%
\pgfpathlineto{\pgfqpoint{2.331731in}{0.595623in}}%
\pgfpathlineto{\pgfqpoint{2.326469in}{0.588102in}}%
\pgfpathlineto{\pgfqpoint{2.314523in}{0.596466in}}%
\pgfusepath{stroke}%
\end{pgfscope}%
\begin{pgfscope}%
\pgfpathrectangle{\pgfqpoint{0.100000in}{0.100000in}}{\pgfqpoint{3.608454in}{2.310000in}}%
\pgfusepath{clip}%
\pgfsetbuttcap%
\pgfsetroundjoin%
\pgfsetlinewidth{0.010037pt}%
\definecolor{currentstroke}{rgb}{1.000000,1.000000,1.000000}%
\pgfsetstrokecolor{currentstroke}%
\pgfsetdash{}{0pt}%
\pgfpathmoveto{\pgfqpoint{2.628224in}{0.685633in}}%
\pgfpathlineto{\pgfqpoint{2.628365in}{0.700581in}}%
\pgfpathlineto{\pgfqpoint{2.619077in}{0.706258in}}%
\pgfpathlineto{\pgfqpoint{2.611467in}{0.715921in}}%
\pgfpathlineto{\pgfqpoint{2.612503in}{0.726074in}}%
\pgfpathlineto{\pgfqpoint{2.695351in}{0.732506in}}%
\pgfpathlineto{\pgfqpoint{2.789338in}{0.742945in}}%
\pgfpathlineto{\pgfqpoint{2.801326in}{0.721097in}}%
\pgfpathlineto{\pgfqpoint{2.841186in}{0.723903in}}%
\pgfpathlineto{\pgfqpoint{2.920222in}{0.728215in}}%
\pgfpathlineto{\pgfqpoint{2.982903in}{0.732393in}}%
\pgfpathlineto{\pgfqpoint{2.989059in}{0.716657in}}%
\pgfpathlineto{\pgfqpoint{2.996668in}{0.717946in}}%
\pgfpathlineto{\pgfqpoint{2.997515in}{0.734795in}}%
\pgfpathlineto{\pgfqpoint{2.993620in}{0.749287in}}%
\pgfpathlineto{\pgfqpoint{2.998840in}{0.755421in}}%
\pgfpathlineto{\pgfqpoint{3.010902in}{0.752121in}}%
\pgfpathlineto{\pgfqpoint{3.028216in}{0.751443in}}%
\pgfpathlineto{\pgfqpoint{3.030878in}{0.738496in}}%
\pgfpathlineto{\pgfqpoint{3.036204in}{0.731098in}}%
\pgfpathlineto{\pgfqpoint{3.040364in}{0.714912in}}%
\pgfpathlineto{\pgfqpoint{3.053250in}{0.689862in}}%
\pgfpathlineto{\pgfqpoint{3.053315in}{0.683088in}}%
\pgfpathlineto{\pgfqpoint{3.072361in}{0.653675in}}%
\pgfpathlineto{\pgfqpoint{3.074204in}{0.647599in}}%
\pgfpathlineto{\pgfqpoint{3.102776in}{0.606058in}}%
\pgfpathlineto{\pgfqpoint{3.098249in}{0.605387in}}%
\pgfpathlineto{\pgfqpoint{3.110295in}{0.575840in}}%
\pgfpathlineto{\pgfqpoint{3.143385in}{0.524475in}}%
\pgfpathlineto{\pgfqpoint{3.162598in}{0.486707in}}%
\pgfpathlineto{\pgfqpoint{3.169019in}{0.480095in}}%
\pgfpathlineto{\pgfqpoint{3.180022in}{0.456389in}}%
\pgfpathlineto{\pgfqpoint{3.183839in}{0.418430in}}%
\pgfpathlineto{\pgfqpoint{3.185359in}{0.390048in}}%
\pgfpathlineto{\pgfqpoint{3.183524in}{0.371875in}}%
\pgfpathlineto{\pgfqpoint{3.177624in}{0.358884in}}%
\pgfpathlineto{\pgfqpoint{3.180365in}{0.341848in}}%
\pgfpathlineto{\pgfqpoint{3.174013in}{0.328363in}}%
\pgfpathlineto{\pgfqpoint{3.155154in}{0.317306in}}%
\pgfpathlineto{\pgfqpoint{3.142888in}{0.318113in}}%
\pgfpathlineto{\pgfqpoint{3.134923in}{0.312333in}}%
\pgfpathlineto{\pgfqpoint{3.132641in}{0.327766in}}%
\pgfpathlineto{\pgfqpoint{3.119456in}{0.331840in}}%
\pgfpathlineto{\pgfqpoint{3.107637in}{0.353374in}}%
\pgfpathlineto{\pgfqpoint{3.106303in}{0.363180in}}%
\pgfpathlineto{\pgfqpoint{3.085149in}{0.369136in}}%
\pgfpathlineto{\pgfqpoint{3.071519in}{0.367846in}}%
\pgfpathlineto{\pgfqpoint{3.063806in}{0.382075in}}%
\pgfpathlineto{\pgfqpoint{3.054976in}{0.407650in}}%
\pgfpathlineto{\pgfqpoint{3.042722in}{0.412834in}}%
\pgfpathlineto{\pgfqpoint{3.036079in}{0.427490in}}%
\pgfpathlineto{\pgfqpoint{3.019832in}{0.436127in}}%
\pgfpathlineto{\pgfqpoint{3.010441in}{0.446931in}}%
\pgfpathlineto{\pgfqpoint{2.994241in}{0.476304in}}%
\pgfpathlineto{\pgfqpoint{2.992335in}{0.496978in}}%
\pgfpathlineto{\pgfqpoint{3.001163in}{0.510205in}}%
\pgfpathlineto{\pgfqpoint{3.000281in}{0.519397in}}%
\pgfpathlineto{\pgfqpoint{2.990079in}{0.520379in}}%
\pgfpathlineto{\pgfqpoint{2.981556in}{0.526772in}}%
\pgfpathlineto{\pgfqpoint{2.976889in}{0.518966in}}%
\pgfpathlineto{\pgfqpoint{2.985069in}{0.512605in}}%
\pgfpathlineto{\pgfqpoint{2.979115in}{0.501166in}}%
\pgfpathlineto{\pgfqpoint{2.969483in}{0.510660in}}%
\pgfpathlineto{\pgfqpoint{2.970600in}{0.536987in}}%
\pgfpathlineto{\pgfqpoint{2.975246in}{0.558347in}}%
\pgfpathlineto{\pgfqpoint{2.972858in}{0.594997in}}%
\pgfpathlineto{\pgfqpoint{2.963257in}{0.603726in}}%
\pgfpathlineto{\pgfqpoint{2.958429in}{0.614928in}}%
\pgfpathlineto{\pgfqpoint{2.941890in}{0.614679in}}%
\pgfpathlineto{\pgfqpoint{2.925541in}{0.633111in}}%
\pgfpathlineto{\pgfqpoint{2.914580in}{0.638635in}}%
\pgfpathlineto{\pgfqpoint{2.911329in}{0.650308in}}%
\pgfpathlineto{\pgfqpoint{2.900608in}{0.654407in}}%
\pgfpathlineto{\pgfqpoint{2.891722in}{0.667295in}}%
\pgfpathlineto{\pgfqpoint{2.868250in}{0.677815in}}%
\pgfpathlineto{\pgfqpoint{2.850001in}{0.678084in}}%
\pgfpathlineto{\pgfqpoint{2.842057in}{0.674044in}}%
\pgfpathlineto{\pgfqpoint{2.844031in}{0.661431in}}%
\pgfpathlineto{\pgfqpoint{2.835747in}{0.662028in}}%
\pgfpathlineto{\pgfqpoint{2.810265in}{0.644393in}}%
\pgfpathlineto{\pgfqpoint{2.779657in}{0.637403in}}%
\pgfpathlineto{\pgfqpoint{2.779155in}{0.646052in}}%
\pgfpathlineto{\pgfqpoint{2.772372in}{0.654504in}}%
\pgfpathlineto{\pgfqpoint{2.754172in}{0.666173in}}%
\pgfpathlineto{\pgfqpoint{2.728049in}{0.678137in}}%
\pgfpathlineto{\pgfqpoint{2.699674in}{0.684471in}}%
\pgfpathlineto{\pgfqpoint{2.694348in}{0.693083in}}%
\pgfpathlineto{\pgfqpoint{2.684117in}{0.685900in}}%
\pgfpathlineto{\pgfqpoint{2.671797in}{0.684316in}}%
\pgfpathlineto{\pgfqpoint{2.628914in}{0.673017in}}%
\pgfpathlineto{\pgfqpoint{2.628224in}{0.685633in}}%
\pgfusepath{stroke}%
\end{pgfscope}%
\begin{pgfscope}%
\pgfpathrectangle{\pgfqpoint{0.100000in}{0.100000in}}{\pgfqpoint{3.608454in}{2.310000in}}%
\pgfusepath{clip}%
\pgfsetbuttcap%
\pgfsetroundjoin%
\pgfsetlinewidth{0.010037pt}%
\definecolor{currentstroke}{rgb}{1.000000,1.000000,1.000000}%
\pgfsetstrokecolor{currentstroke}%
\pgfsetdash{}{0pt}%
\pgfpathmoveto{\pgfqpoint{3.106940in}{0.606649in}}%
\pgfpathlineto{\pgfqpoint{3.118035in}{0.593742in}}%
\pgfpathlineto{\pgfqpoint{3.105219in}{0.592930in}}%
\pgfpathlineto{\pgfqpoint{3.106940in}{0.606649in}}%
\pgfusepath{stroke}%
\end{pgfscope}%
\begin{pgfscope}%
\pgfpathrectangle{\pgfqpoint{0.100000in}{0.100000in}}{\pgfqpoint{3.608454in}{2.310000in}}%
\pgfusepath{clip}%
\pgfsetbuttcap%
\pgfsetroundjoin%
\pgfsetlinewidth{0.010037pt}%
\definecolor{currentstroke}{rgb}{1.000000,1.000000,1.000000}%
\pgfsetstrokecolor{currentstroke}%
\pgfsetdash{}{0pt}%
\pgfpathmoveto{\pgfqpoint{2.453791in}{2.080299in}}%
\pgfpathlineto{\pgfqpoint{2.447771in}{2.068582in}}%
\pgfpathlineto{\pgfqpoint{2.433410in}{2.061744in}}%
\pgfpathlineto{\pgfqpoint{2.427189in}{2.052541in}}%
\pgfpathlineto{\pgfqpoint{2.419903in}{2.059169in}}%
\pgfpathlineto{\pgfqpoint{2.453791in}{2.080299in}}%
\pgfusepath{stroke}%
\end{pgfscope}%
\begin{pgfscope}%
\pgfpathrectangle{\pgfqpoint{0.100000in}{0.100000in}}{\pgfqpoint{3.608454in}{2.310000in}}%
\pgfusepath{clip}%
\pgfsetbuttcap%
\pgfsetroundjoin%
\pgfsetlinewidth{0.010037pt}%
\definecolor{currentstroke}{rgb}{1.000000,1.000000,1.000000}%
\pgfsetstrokecolor{currentstroke}%
\pgfsetdash{}{0pt}%
\pgfpathmoveto{\pgfqpoint{2.458673in}{2.009787in}}%
\pgfpathlineto{\pgfqpoint{2.473368in}{2.023495in}}%
\pgfpathlineto{\pgfqpoint{2.496033in}{2.027084in}}%
\pgfpathlineto{\pgfqpoint{2.489790in}{2.017551in}}%
\pgfpathlineto{\pgfqpoint{2.474252in}{2.003764in}}%
\pgfpathlineto{\pgfqpoint{2.465168in}{1.986068in}}%
\pgfpathlineto{\pgfqpoint{2.461321in}{1.995666in}}%
\pgfpathlineto{\pgfqpoint{2.454489in}{1.997044in}}%
\pgfpathlineto{\pgfqpoint{2.453803in}{2.005709in}}%
\pgfpathlineto{\pgfqpoint{2.458673in}{2.009787in}}%
\pgfusepath{stroke}%
\end{pgfscope}%
\begin{pgfscope}%
\pgfpathrectangle{\pgfqpoint{0.100000in}{0.100000in}}{\pgfqpoint{3.608454in}{2.310000in}}%
\pgfusepath{clip}%
\pgfsetbuttcap%
\pgfsetroundjoin%
\pgfsetlinewidth{0.010037pt}%
\definecolor{currentstroke}{rgb}{1.000000,1.000000,1.000000}%
\pgfsetstrokecolor{currentstroke}%
\pgfsetdash{}{0pt}%
\pgfpathmoveto{\pgfqpoint{2.517200in}{1.842472in}}%
\pgfpathlineto{\pgfqpoint{2.513287in}{1.846815in}}%
\pgfpathlineto{\pgfqpoint{2.517401in}{1.859041in}}%
\pgfpathlineto{\pgfqpoint{2.505188in}{1.859820in}}%
\pgfpathlineto{\pgfqpoint{2.508426in}{1.870361in}}%
\pgfpathlineto{\pgfqpoint{2.506412in}{1.887150in}}%
\pgfpathlineto{\pgfqpoint{2.495454in}{1.892970in}}%
\pgfpathlineto{\pgfqpoint{2.483987in}{1.904750in}}%
\pgfpathlineto{\pgfqpoint{2.466317in}{1.908198in}}%
\pgfpathlineto{\pgfqpoint{2.449120in}{1.908100in}}%
\pgfpathlineto{\pgfqpoint{2.432207in}{1.916460in}}%
\pgfpathlineto{\pgfqpoint{2.375622in}{1.928645in}}%
\pgfpathlineto{\pgfqpoint{2.369439in}{1.941557in}}%
\pgfpathlineto{\pgfqpoint{2.358409in}{1.945963in}}%
\pgfpathlineto{\pgfqpoint{2.379243in}{1.955848in}}%
\pgfpathlineto{\pgfqpoint{2.391013in}{1.968181in}}%
\pgfpathlineto{\pgfqpoint{2.412935in}{1.971526in}}%
\pgfpathlineto{\pgfqpoint{2.426425in}{1.984088in}}%
\pgfpathlineto{\pgfqpoint{2.433502in}{1.984591in}}%
\pgfpathlineto{\pgfqpoint{2.438904in}{1.993550in}}%
\pgfpathlineto{\pgfqpoint{2.452070in}{2.004163in}}%
\pgfpathlineto{\pgfqpoint{2.453218in}{1.996662in}}%
\pgfpathlineto{\pgfqpoint{2.459151in}{1.995113in}}%
\pgfpathlineto{\pgfqpoint{2.463615in}{1.986167in}}%
\pgfpathlineto{\pgfqpoint{2.464419in}{1.970914in}}%
\pgfpathlineto{\pgfqpoint{2.477835in}{1.980029in}}%
\pgfpathlineto{\pgfqpoint{2.493468in}{1.981933in}}%
\pgfpathlineto{\pgfqpoint{2.506813in}{1.977158in}}%
\pgfpathlineto{\pgfqpoint{2.524910in}{1.952330in}}%
\pgfpathlineto{\pgfqpoint{2.544676in}{1.956298in}}%
\pgfpathlineto{\pgfqpoint{2.552707in}{1.949654in}}%
\pgfpathlineto{\pgfqpoint{2.565625in}{1.949060in}}%
\pgfpathlineto{\pgfqpoint{2.574207in}{1.960977in}}%
\pgfpathlineto{\pgfqpoint{2.590492in}{1.971511in}}%
\pgfpathlineto{\pgfqpoint{2.606143in}{1.974810in}}%
\pgfpathlineto{\pgfqpoint{2.625564in}{1.975153in}}%
\pgfpathlineto{\pgfqpoint{2.639756in}{1.983291in}}%
\pgfpathlineto{\pgfqpoint{2.651338in}{1.979543in}}%
\pgfpathlineto{\pgfqpoint{2.653805in}{1.962333in}}%
\pgfpathlineto{\pgfqpoint{2.678653in}{1.959638in}}%
\pgfpathlineto{\pgfqpoint{2.692149in}{1.967694in}}%
\pgfpathlineto{\pgfqpoint{2.701505in}{1.949546in}}%
\pgfpathlineto{\pgfqpoint{2.719357in}{1.928560in}}%
\pgfpathlineto{\pgfqpoint{2.694371in}{1.928682in}}%
\pgfpathlineto{\pgfqpoint{2.686477in}{1.926088in}}%
\pgfpathlineto{\pgfqpoint{2.675655in}{1.929459in}}%
\pgfpathlineto{\pgfqpoint{2.674927in}{1.914931in}}%
\pgfpathlineto{\pgfqpoint{2.655278in}{1.926284in}}%
\pgfpathlineto{\pgfqpoint{2.630009in}{1.929754in}}%
\pgfpathlineto{\pgfqpoint{2.623052in}{1.918697in}}%
\pgfpathlineto{\pgfqpoint{2.589991in}{1.913321in}}%
\pgfpathlineto{\pgfqpoint{2.586187in}{1.903954in}}%
\pgfpathlineto{\pgfqpoint{2.563180in}{1.901144in}}%
\pgfpathlineto{\pgfqpoint{2.556225in}{1.891622in}}%
\pgfpathlineto{\pgfqpoint{2.544055in}{1.889069in}}%
\pgfpathlineto{\pgfqpoint{2.534335in}{1.866491in}}%
\pgfpathlineto{\pgfqpoint{2.522029in}{1.844624in}}%
\pgfpathlineto{\pgfqpoint{2.517200in}{1.842472in}}%
\pgfusepath{stroke}%
\end{pgfscope}%
\begin{pgfscope}%
\pgfpathrectangle{\pgfqpoint{0.100000in}{0.100000in}}{\pgfqpoint{3.608454in}{2.310000in}}%
\pgfusepath{clip}%
\pgfsetbuttcap%
\pgfsetroundjoin%
\pgfsetlinewidth{0.010037pt}%
\definecolor{currentstroke}{rgb}{1.000000,1.000000,1.000000}%
\pgfsetstrokecolor{currentstroke}%
\pgfsetdash{}{0pt}%
\pgfpathmoveto{\pgfqpoint{2.588031in}{1.579945in}}%
\pgfpathlineto{\pgfqpoint{2.599764in}{1.592295in}}%
\pgfpathlineto{\pgfqpoint{2.605110in}{1.610201in}}%
\pgfpathlineto{\pgfqpoint{2.611467in}{1.620579in}}%
\pgfpathlineto{\pgfqpoint{2.615365in}{1.634703in}}%
\pgfpathlineto{\pgfqpoint{2.616576in}{1.662865in}}%
\pgfpathlineto{\pgfqpoint{2.610707in}{1.689854in}}%
\pgfpathlineto{\pgfqpoint{2.591312in}{1.731182in}}%
\pgfpathlineto{\pgfqpoint{2.596565in}{1.744177in}}%
\pgfpathlineto{\pgfqpoint{2.589723in}{1.762138in}}%
\pgfpathlineto{\pgfqpoint{2.601460in}{1.786979in}}%
\pgfpathlineto{\pgfqpoint{2.599547in}{1.814676in}}%
\pgfpathlineto{\pgfqpoint{2.607723in}{1.818162in}}%
\pgfpathlineto{\pgfqpoint{2.608750in}{1.831341in}}%
\pgfpathlineto{\pgfqpoint{2.623262in}{1.839799in}}%
\pgfpathlineto{\pgfqpoint{2.631469in}{1.838434in}}%
\pgfpathlineto{\pgfqpoint{2.633776in}{1.824300in}}%
\pgfpathlineto{\pgfqpoint{2.640182in}{1.823746in}}%
\pgfpathlineto{\pgfqpoint{2.646027in}{1.844133in}}%
\pgfpathlineto{\pgfqpoint{2.644040in}{1.859989in}}%
\pgfpathlineto{\pgfqpoint{2.647841in}{1.869093in}}%
\pgfpathlineto{\pgfqpoint{2.668387in}{1.878499in}}%
\pgfpathlineto{\pgfqpoint{2.659024in}{1.881847in}}%
\pgfpathlineto{\pgfqpoint{2.655974in}{1.889927in}}%
\pgfpathlineto{\pgfqpoint{2.662719in}{1.904115in}}%
\pgfpathlineto{\pgfqpoint{2.676020in}{1.909016in}}%
\pgfpathlineto{\pgfqpoint{2.691465in}{1.900645in}}%
\pgfpathlineto{\pgfqpoint{2.706027in}{1.900540in}}%
\pgfpathlineto{\pgfqpoint{2.712790in}{1.890761in}}%
\pgfpathlineto{\pgfqpoint{2.722988in}{1.891440in}}%
\pgfpathlineto{\pgfqpoint{2.754470in}{1.877846in}}%
\pgfpathlineto{\pgfqpoint{2.760759in}{1.864686in}}%
\pgfpathlineto{\pgfqpoint{2.753890in}{1.860128in}}%
\pgfpathlineto{\pgfqpoint{2.756007in}{1.850265in}}%
\pgfpathlineto{\pgfqpoint{2.762793in}{1.845879in}}%
\pgfpathlineto{\pgfqpoint{2.766575in}{1.833755in}}%
\pgfpathlineto{\pgfqpoint{2.766045in}{1.804218in}}%
\pgfpathlineto{\pgfqpoint{2.757086in}{1.797162in}}%
\pgfpathlineto{\pgfqpoint{2.755075in}{1.781646in}}%
\pgfpathlineto{\pgfqpoint{2.738454in}{1.767308in}}%
\pgfpathlineto{\pgfqpoint{2.739441in}{1.749952in}}%
\pgfpathlineto{\pgfqpoint{2.754008in}{1.743872in}}%
\pgfpathlineto{\pgfqpoint{2.770441in}{1.765527in}}%
\pgfpathlineto{\pgfqpoint{2.771800in}{1.773381in}}%
\pgfpathlineto{\pgfqpoint{2.792295in}{1.786435in}}%
\pgfpathlineto{\pgfqpoint{2.805328in}{1.780388in}}%
\pgfpathlineto{\pgfqpoint{2.813507in}{1.766723in}}%
\pgfpathlineto{\pgfqpoint{2.826701in}{1.719275in}}%
\pgfpathlineto{\pgfqpoint{2.833693in}{1.704263in}}%
\pgfpathlineto{\pgfqpoint{2.831504in}{1.698058in}}%
\pgfpathlineto{\pgfqpoint{2.831717in}{1.676927in}}%
\pgfpathlineto{\pgfqpoint{2.818998in}{1.678955in}}%
\pgfpathlineto{\pgfqpoint{2.811814in}{1.663165in}}%
\pgfpathlineto{\pgfqpoint{2.810819in}{1.652430in}}%
\pgfpathlineto{\pgfqpoint{2.801211in}{1.645539in}}%
\pgfpathlineto{\pgfqpoint{2.799076in}{1.624604in}}%
\pgfpathlineto{\pgfqpoint{2.785105in}{1.598157in}}%
\pgfpathlineto{\pgfqpoint{2.708677in}{1.586775in}}%
\pgfpathlineto{\pgfqpoint{2.708226in}{1.591778in}}%
\pgfpathlineto{\pgfqpoint{2.657102in}{1.586389in}}%
\pgfpathlineto{\pgfqpoint{2.588031in}{1.579945in}}%
\pgfusepath{stroke}%
\end{pgfscope}%
\begin{pgfscope}%
\pgfpathrectangle{\pgfqpoint{0.100000in}{0.100000in}}{\pgfqpoint{3.608454in}{2.310000in}}%
\pgfusepath{clip}%
\pgfsetbuttcap%
\pgfsetroundjoin%
\pgfsetlinewidth{0.010037pt}%
\definecolor{currentstroke}{rgb}{1.000000,1.000000,1.000000}%
\pgfsetstrokecolor{currentstroke}%
\pgfsetdash{}{0pt}%
\pgfpathmoveto{\pgfqpoint{0.000000in}{0.000000in}}%
\pgfusepath{stroke}%
\end{pgfscope}%
\begin{pgfscope}%
\pgfpathrectangle{\pgfqpoint{0.100000in}{0.100000in}}{\pgfqpoint{3.608454in}{2.310000in}}%
\pgfusepath{clip}%
\pgfsetbuttcap%
\pgfsetroundjoin%
\pgfsetlinewidth{0.010037pt}%
\definecolor{currentstroke}{rgb}{1.000000,1.000000,1.000000}%
\pgfsetstrokecolor{currentstroke}%
\pgfsetdash{}{0pt}%
\pgfusepath{stroke}%
\end{pgfscope}%
\begin{pgfscope}%
\pgfpathrectangle{\pgfqpoint{0.100000in}{0.100000in}}{\pgfqpoint{3.608454in}{2.310000in}}%
\pgfusepath{clip}%
\pgfsetbuttcap%
\pgfsetroundjoin%
\pgfsetlinewidth{0.010037pt}%
\definecolor{currentstroke}{rgb}{1.000000,1.000000,1.000000}%
\pgfsetstrokecolor{currentstroke}%
\pgfsetdash{}{0pt}%
\pgfusepath{stroke}%
\end{pgfscope}%
\begin{pgfscope}%
\pgfpathrectangle{\pgfqpoint{0.100000in}{0.100000in}}{\pgfqpoint{3.608454in}{2.310000in}}%
\pgfusepath{clip}%
\pgfsetbuttcap%
\pgfsetroundjoin%
\pgfsetlinewidth{0.010037pt}%
\definecolor{currentstroke}{rgb}{1.000000,1.000000,1.000000}%
\pgfsetstrokecolor{currentstroke}%
\pgfsetdash{}{0pt}%
\pgfusepath{stroke}%
\end{pgfscope}%
\begin{pgfscope}%
\pgfpathrectangle{\pgfqpoint{0.100000in}{0.100000in}}{\pgfqpoint{3.608454in}{2.310000in}}%
\pgfusepath{clip}%
\pgfsetbuttcap%
\pgfsetroundjoin%
\pgfsetlinewidth{0.010037pt}%
\definecolor{currentstroke}{rgb}{1.000000,1.000000,1.000000}%
\pgfsetstrokecolor{currentstroke}%
\pgfsetdash{}{0pt}%
\pgfusepath{stroke}%
\end{pgfscope}%
\begin{pgfscope}%
\pgfpathrectangle{\pgfqpoint{0.100000in}{0.100000in}}{\pgfqpoint{3.608454in}{2.310000in}}%
\pgfusepath{clip}%
\pgfsetbuttcap%
\pgfsetroundjoin%
\pgfsetlinewidth{0.010037pt}%
\definecolor{currentstroke}{rgb}{1.000000,1.000000,1.000000}%
\pgfsetstrokecolor{currentstroke}%
\pgfsetdash{}{0pt}%
\pgfusepath{stroke}%
\end{pgfscope}%
\begin{pgfscope}%
\pgfpathrectangle{\pgfqpoint{0.100000in}{0.100000in}}{\pgfqpoint{3.608454in}{2.310000in}}%
\pgfusepath{clip}%
\pgfsetbuttcap%
\pgfsetroundjoin%
\pgfsetlinewidth{0.010037pt}%
\definecolor{currentstroke}{rgb}{1.000000,1.000000,1.000000}%
\pgfsetstrokecolor{currentstroke}%
\pgfsetdash{}{0pt}%
\pgfusepath{stroke}%
\end{pgfscope}%
\begin{pgfscope}%
\pgfpathrectangle{\pgfqpoint{0.100000in}{0.100000in}}{\pgfqpoint{3.608454in}{2.310000in}}%
\pgfusepath{clip}%
\pgfsetbuttcap%
\pgfsetroundjoin%
\pgfsetlinewidth{0.010037pt}%
\definecolor{currentstroke}{rgb}{1.000000,1.000000,1.000000}%
\pgfsetstrokecolor{currentstroke}%
\pgfsetdash{}{0pt}%
\pgfusepath{stroke}%
\end{pgfscope}%
\begin{pgfscope}%
\pgfpathrectangle{\pgfqpoint{0.100000in}{0.100000in}}{\pgfqpoint{3.608454in}{2.310000in}}%
\pgfusepath{clip}%
\pgfsetbuttcap%
\pgfsetroundjoin%
\pgfsetlinewidth{0.010037pt}%
\definecolor{currentstroke}{rgb}{1.000000,1.000000,1.000000}%
\pgfsetstrokecolor{currentstroke}%
\pgfsetdash{}{0pt}%
\pgfusepath{stroke}%
\end{pgfscope}%
\begin{pgfscope}%
\pgfpathrectangle{\pgfqpoint{0.100000in}{0.100000in}}{\pgfqpoint{3.608454in}{2.310000in}}%
\pgfusepath{clip}%
\pgfsetbuttcap%
\pgfsetroundjoin%
\pgfsetlinewidth{0.010037pt}%
\definecolor{currentstroke}{rgb}{1.000000,1.000000,1.000000}%
\pgfsetstrokecolor{currentstroke}%
\pgfsetdash{}{0pt}%
\pgfusepath{stroke}%
\end{pgfscope}%
\begin{pgfscope}%
\pgfpathrectangle{\pgfqpoint{0.100000in}{0.100000in}}{\pgfqpoint{3.608454in}{2.310000in}}%
\pgfusepath{clip}%
\pgfsetbuttcap%
\pgfsetroundjoin%
\pgfsetlinewidth{0.010037pt}%
\definecolor{currentstroke}{rgb}{1.000000,1.000000,1.000000}%
\pgfsetstrokecolor{currentstroke}%
\pgfsetdash{}{0pt}%
\pgfusepath{stroke}%
\end{pgfscope}%
\begin{pgfscope}%
\pgfpathrectangle{\pgfqpoint{0.100000in}{0.100000in}}{\pgfqpoint{3.608454in}{2.310000in}}%
\pgfusepath{clip}%
\pgfsetbuttcap%
\pgfsetroundjoin%
\pgfsetlinewidth{0.010037pt}%
\definecolor{currentstroke}{rgb}{1.000000,1.000000,1.000000}%
\pgfsetstrokecolor{currentstroke}%
\pgfsetdash{}{0pt}%
\pgfusepath{stroke}%
\end{pgfscope}%
\begin{pgfscope}%
\pgfpathrectangle{\pgfqpoint{0.100000in}{0.100000in}}{\pgfqpoint{3.608454in}{2.310000in}}%
\pgfusepath{clip}%
\pgfsetbuttcap%
\pgfsetroundjoin%
\pgfsetlinewidth{0.010037pt}%
\definecolor{currentstroke}{rgb}{1.000000,1.000000,1.000000}%
\pgfsetstrokecolor{currentstroke}%
\pgfsetdash{}{0pt}%
\pgfusepath{stroke}%
\end{pgfscope}%
\begin{pgfscope}%
\pgfpathrectangle{\pgfqpoint{0.100000in}{0.100000in}}{\pgfqpoint{3.608454in}{2.310000in}}%
\pgfusepath{clip}%
\pgfsetbuttcap%
\pgfsetroundjoin%
\pgfsetlinewidth{0.010037pt}%
\definecolor{currentstroke}{rgb}{1.000000,1.000000,1.000000}%
\pgfsetstrokecolor{currentstroke}%
\pgfsetdash{}{0pt}%
\pgfusepath{stroke}%
\end{pgfscope}%
\begin{pgfscope}%
\pgfpathrectangle{\pgfqpoint{0.100000in}{0.100000in}}{\pgfqpoint{3.608454in}{2.310000in}}%
\pgfusepath{clip}%
\pgfsetbuttcap%
\pgfsetroundjoin%
\pgfsetlinewidth{0.010037pt}%
\definecolor{currentstroke}{rgb}{1.000000,1.000000,1.000000}%
\pgfsetstrokecolor{currentstroke}%
\pgfsetdash{}{0pt}%
\pgfusepath{stroke}%
\end{pgfscope}%
\begin{pgfscope}%
\pgfpathrectangle{\pgfqpoint{0.100000in}{0.100000in}}{\pgfqpoint{3.608454in}{2.310000in}}%
\pgfusepath{clip}%
\pgfsetbuttcap%
\pgfsetroundjoin%
\pgfsetlinewidth{0.010037pt}%
\definecolor{currentstroke}{rgb}{1.000000,1.000000,1.000000}%
\pgfsetstrokecolor{currentstroke}%
\pgfsetdash{}{0pt}%
\pgfusepath{stroke}%
\end{pgfscope}%
\begin{pgfscope}%
\pgfpathrectangle{\pgfqpoint{0.100000in}{0.100000in}}{\pgfqpoint{3.608454in}{2.310000in}}%
\pgfusepath{clip}%
\pgfsetbuttcap%
\pgfsetroundjoin%
\pgfsetlinewidth{0.010037pt}%
\definecolor{currentstroke}{rgb}{1.000000,1.000000,1.000000}%
\pgfsetstrokecolor{currentstroke}%
\pgfsetdash{}{0pt}%
\pgfusepath{stroke}%
\end{pgfscope}%
\begin{pgfscope}%
\pgfpathrectangle{\pgfqpoint{0.100000in}{0.100000in}}{\pgfqpoint{3.608454in}{2.310000in}}%
\pgfusepath{clip}%
\pgfsetbuttcap%
\pgfsetroundjoin%
\pgfsetlinewidth{0.010037pt}%
\definecolor{currentstroke}{rgb}{1.000000,1.000000,1.000000}%
\pgfsetstrokecolor{currentstroke}%
\pgfsetdash{}{0pt}%
\pgfusepath{stroke}%
\end{pgfscope}%
\begin{pgfscope}%
\pgfpathrectangle{\pgfqpoint{0.100000in}{0.100000in}}{\pgfqpoint{3.608454in}{2.310000in}}%
\pgfusepath{clip}%
\pgfsetbuttcap%
\pgfsetroundjoin%
\pgfsetlinewidth{0.010037pt}%
\definecolor{currentstroke}{rgb}{1.000000,1.000000,1.000000}%
\pgfsetstrokecolor{currentstroke}%
\pgfsetdash{}{0pt}%
\pgfusepath{stroke}%
\end{pgfscope}%
\begin{pgfscope}%
\pgfpathrectangle{\pgfqpoint{0.100000in}{0.100000in}}{\pgfqpoint{3.608454in}{2.310000in}}%
\pgfusepath{clip}%
\pgfsetbuttcap%
\pgfsetroundjoin%
\pgfsetlinewidth{0.010037pt}%
\definecolor{currentstroke}{rgb}{1.000000,1.000000,1.000000}%
\pgfsetstrokecolor{currentstroke}%
\pgfsetdash{}{0pt}%
\pgfusepath{stroke}%
\end{pgfscope}%
\begin{pgfscope}%
\pgfpathrectangle{\pgfqpoint{0.100000in}{0.100000in}}{\pgfqpoint{3.608454in}{2.310000in}}%
\pgfusepath{clip}%
\pgfsetbuttcap%
\pgfsetroundjoin%
\pgfsetlinewidth{0.010037pt}%
\definecolor{currentstroke}{rgb}{1.000000,1.000000,1.000000}%
\pgfsetstrokecolor{currentstroke}%
\pgfsetdash{}{0pt}%
\pgfusepath{stroke}%
\end{pgfscope}%
\begin{pgfscope}%
\pgfpathrectangle{\pgfqpoint{0.100000in}{0.100000in}}{\pgfqpoint{3.608454in}{2.310000in}}%
\pgfusepath{clip}%
\pgfsetbuttcap%
\pgfsetroundjoin%
\pgfsetlinewidth{0.010037pt}%
\definecolor{currentstroke}{rgb}{1.000000,1.000000,1.000000}%
\pgfsetstrokecolor{currentstroke}%
\pgfsetdash{}{0pt}%
\pgfusepath{stroke}%
\end{pgfscope}%
\begin{pgfscope}%
\pgfpathrectangle{\pgfqpoint{0.100000in}{0.100000in}}{\pgfqpoint{3.608454in}{2.310000in}}%
\pgfusepath{clip}%
\pgfsetbuttcap%
\pgfsetroundjoin%
\pgfsetlinewidth{0.010037pt}%
\definecolor{currentstroke}{rgb}{1.000000,1.000000,1.000000}%
\pgfsetstrokecolor{currentstroke}%
\pgfsetdash{}{0pt}%
\pgfusepath{stroke}%
\end{pgfscope}%
\begin{pgfscope}%
\pgfpathrectangle{\pgfqpoint{0.100000in}{0.100000in}}{\pgfqpoint{3.608454in}{2.310000in}}%
\pgfusepath{clip}%
\pgfsetbuttcap%
\pgfsetroundjoin%
\pgfsetlinewidth{0.010037pt}%
\definecolor{currentstroke}{rgb}{1.000000,1.000000,1.000000}%
\pgfsetstrokecolor{currentstroke}%
\pgfsetdash{}{0pt}%
\pgfusepath{stroke}%
\end{pgfscope}%
\begin{pgfscope}%
\pgfpathrectangle{\pgfqpoint{0.100000in}{0.100000in}}{\pgfqpoint{3.608454in}{2.310000in}}%
\pgfusepath{clip}%
\pgfsetbuttcap%
\pgfsetroundjoin%
\pgfsetlinewidth{0.010037pt}%
\definecolor{currentstroke}{rgb}{1.000000,1.000000,1.000000}%
\pgfsetstrokecolor{currentstroke}%
\pgfsetdash{}{0pt}%
\pgfusepath{stroke}%
\end{pgfscope}%
\begin{pgfscope}%
\pgfpathrectangle{\pgfqpoint{0.100000in}{0.100000in}}{\pgfqpoint{3.608454in}{2.310000in}}%
\pgfusepath{clip}%
\pgfsetbuttcap%
\pgfsetroundjoin%
\pgfsetlinewidth{0.010037pt}%
\definecolor{currentstroke}{rgb}{1.000000,1.000000,1.000000}%
\pgfsetstrokecolor{currentstroke}%
\pgfsetdash{}{0pt}%
\pgfusepath{stroke}%
\end{pgfscope}%
\begin{pgfscope}%
\pgfpathrectangle{\pgfqpoint{0.100000in}{0.100000in}}{\pgfqpoint{3.608454in}{2.310000in}}%
\pgfusepath{clip}%
\pgfsetbuttcap%
\pgfsetroundjoin%
\pgfsetlinewidth{0.010037pt}%
\definecolor{currentstroke}{rgb}{1.000000,1.000000,1.000000}%
\pgfsetstrokecolor{currentstroke}%
\pgfsetdash{}{0pt}%
\pgfusepath{stroke}%
\end{pgfscope}%
\begin{pgfscope}%
\pgfpathrectangle{\pgfqpoint{0.100000in}{0.100000in}}{\pgfqpoint{3.608454in}{2.310000in}}%
\pgfusepath{clip}%
\pgfsetbuttcap%
\pgfsetroundjoin%
\pgfsetlinewidth{0.010037pt}%
\definecolor{currentstroke}{rgb}{1.000000,1.000000,1.000000}%
\pgfsetstrokecolor{currentstroke}%
\pgfsetdash{}{0pt}%
\pgfusepath{stroke}%
\end{pgfscope}%
\begin{pgfscope}%
\pgfpathrectangle{\pgfqpoint{0.100000in}{0.100000in}}{\pgfqpoint{3.608454in}{2.310000in}}%
\pgfusepath{clip}%
\pgfsetbuttcap%
\pgfsetmiterjoin%
\pgfsetlinewidth{0.000000pt}%
\definecolor{currentstroke}{rgb}{0.000000,0.000000,0.000000}%
\pgfsetstrokecolor{currentstroke}%
\pgfsetstrokeopacity{0.000000}%
\pgfsetdash{}{0pt}%
\pgfpathmoveto{\pgfqpoint{0.100000in}{0.100000in}}%
\pgfpathlineto{\pgfqpoint{3.708454in}{0.100000in}}%
\pgfpathlineto{\pgfqpoint{3.708454in}{2.410000in}}%
\pgfpathlineto{\pgfqpoint{0.100000in}{2.410000in}}%
\pgfpathlineto{\pgfqpoint{0.100000in}{0.100000in}}%
\pgfpathclose%
\pgfusepath{}%
\end{pgfscope}%
\begin{pgfscope}%
\pgfpathrectangle{\pgfqpoint{0.100000in}{0.100000in}}{\pgfqpoint{3.608454in}{2.310000in}}%
\pgfusepath{clip}%
\pgfsetbuttcap%
\pgfsetmiterjoin%
\definecolor{currentfill}{rgb}{0.000000,0.364706,0.817647}%
\pgfsetfillcolor{currentfill}%
\pgfsetlinewidth{0.000000pt}%
\definecolor{currentstroke}{rgb}{0.000000,0.000000,0.000000}%
\pgfsetstrokecolor{currentstroke}%
\pgfsetstrokeopacity{0.000000}%
\pgfsetdash{}{0pt}%
\pgfpathmoveto{\pgfqpoint{2.124622in}{1.510289in}}%
\pgfpathlineto{\pgfqpoint{2.097514in}{1.510367in}}%
\pgfpathlineto{\pgfqpoint{2.097619in}{1.537721in}}%
\pgfpathlineto{\pgfqpoint{2.077274in}{1.537859in}}%
\pgfpathlineto{\pgfqpoint{2.073777in}{1.545495in}}%
\pgfpathlineto{\pgfqpoint{2.073952in}{1.566162in}}%
\pgfpathlineto{\pgfqpoint{2.121832in}{1.566001in}}%
\pgfpathlineto{\pgfqpoint{2.121931in}{1.545408in}}%
\pgfpathlineto{\pgfqpoint{2.124501in}{1.545418in}}%
\pgfpathlineto{\pgfqpoint{2.124622in}{1.510289in}}%
\pgfpathclose%
\pgfusepath{fill}%
\end{pgfscope}%
\begin{pgfscope}%
\pgfpathrectangle{\pgfqpoint{0.100000in}{0.100000in}}{\pgfqpoint{3.608454in}{2.310000in}}%
\pgfusepath{clip}%
\pgfsetbuttcap%
\pgfsetmiterjoin%
\definecolor{currentfill}{rgb}{0.000000,0.850980,0.574510}%
\pgfsetfillcolor{currentfill}%
\pgfsetlinewidth{0.000000pt}%
\definecolor{currentstroke}{rgb}{0.000000,0.000000,0.000000}%
\pgfsetstrokecolor{currentstroke}%
\pgfsetstrokeopacity{0.000000}%
\pgfsetdash{}{0pt}%
\pgfpathmoveto{\pgfqpoint{2.724860in}{1.203323in}}%
\pgfpathlineto{\pgfqpoint{2.731198in}{1.201857in}}%
\pgfpathlineto{\pgfqpoint{2.747524in}{1.217198in}}%
\pgfpathlineto{\pgfqpoint{2.742647in}{1.220616in}}%
\pgfpathlineto{\pgfqpoint{2.742337in}{1.226077in}}%
\pgfpathlineto{\pgfqpoint{2.752556in}{1.236924in}}%
\pgfpathlineto{\pgfqpoint{2.761764in}{1.247341in}}%
\pgfpathlineto{\pgfqpoint{2.765354in}{1.245580in}}%
\pgfpathlineto{\pgfqpoint{2.780361in}{1.233265in}}%
\pgfpathlineto{\pgfqpoint{2.778898in}{1.226398in}}%
\pgfpathlineto{\pgfqpoint{2.781636in}{1.214245in}}%
\pgfpathlineto{\pgfqpoint{2.779295in}{1.206052in}}%
\pgfpathlineto{\pgfqpoint{2.783865in}{1.195392in}}%
\pgfpathlineto{\pgfqpoint{2.789452in}{1.189777in}}%
\pgfpathlineto{\pgfqpoint{2.788196in}{1.183891in}}%
\pgfpathlineto{\pgfqpoint{2.783231in}{1.181649in}}%
\pgfpathlineto{\pgfqpoint{2.785832in}{1.164847in}}%
\pgfpathlineto{\pgfqpoint{2.781495in}{1.155039in}}%
\pgfpathlineto{\pgfqpoint{2.771140in}{1.159342in}}%
\pgfpathlineto{\pgfqpoint{2.761563in}{1.168904in}}%
\pgfpathlineto{\pgfqpoint{2.763981in}{1.171120in}}%
\pgfpathlineto{\pgfqpoint{2.758225in}{1.180770in}}%
\pgfpathlineto{\pgfqpoint{2.754147in}{1.186574in}}%
\pgfpathlineto{\pgfqpoint{2.722959in}{1.184982in}}%
\pgfpathlineto{\pgfqpoint{2.724860in}{1.203323in}}%
\pgfpathclose%
\pgfusepath{fill}%
\end{pgfscope}%
\begin{pgfscope}%
\pgfpathrectangle{\pgfqpoint{0.100000in}{0.100000in}}{\pgfqpoint{3.608454in}{2.310000in}}%
\pgfusepath{clip}%
\pgfsetbuttcap%
\pgfsetmiterjoin%
\definecolor{currentfill}{rgb}{0.000000,0.635294,0.682353}%
\pgfsetfillcolor{currentfill}%
\pgfsetlinewidth{0.000000pt}%
\definecolor{currentstroke}{rgb}{0.000000,0.000000,0.000000}%
\pgfsetstrokecolor{currentstroke}%
\pgfsetstrokeopacity{0.000000}%
\pgfsetdash{}{0pt}%
\pgfpathmoveto{\pgfqpoint{1.745610in}{0.979809in}}%
\pgfpathlineto{\pgfqpoint{1.743217in}{0.941925in}}%
\pgfpathlineto{\pgfqpoint{1.708835in}{0.943953in}}%
\pgfpathlineto{\pgfqpoint{1.711338in}{0.981859in}}%
\pgfpathlineto{\pgfqpoint{1.745610in}{0.979809in}}%
\pgfpathclose%
\pgfusepath{fill}%
\end{pgfscope}%
\begin{pgfscope}%
\pgfpathrectangle{\pgfqpoint{0.100000in}{0.100000in}}{\pgfqpoint{3.608454in}{2.310000in}}%
\pgfusepath{clip}%
\pgfsetbuttcap%
\pgfsetmiterjoin%
\definecolor{currentfill}{rgb}{0.000000,0.713725,0.643137}%
\pgfsetfillcolor{currentfill}%
\pgfsetlinewidth{0.000000pt}%
\definecolor{currentstroke}{rgb}{0.000000,0.000000,0.000000}%
\pgfsetstrokecolor{currentstroke}%
\pgfsetstrokeopacity{0.000000}%
\pgfsetdash{}{0pt}%
\pgfpathmoveto{\pgfqpoint{1.670049in}{1.298787in}}%
\pgfpathlineto{\pgfqpoint{1.670021in}{1.298287in}}%
\pgfpathlineto{\pgfqpoint{1.666954in}{1.256981in}}%
\pgfpathlineto{\pgfqpoint{1.666412in}{1.249548in}}%
\pgfpathlineto{\pgfqpoint{1.601877in}{1.254860in}}%
\pgfpathlineto{\pgfqpoint{1.540663in}{1.260697in}}%
\pgfpathlineto{\pgfqpoint{1.541303in}{1.267794in}}%
\pgfpathlineto{\pgfqpoint{1.506183in}{1.284643in}}%
\pgfpathlineto{\pgfqpoint{1.483235in}{1.285604in}}%
\pgfpathlineto{\pgfqpoint{1.481289in}{1.288540in}}%
\pgfpathlineto{\pgfqpoint{1.484320in}{1.315446in}}%
\pgfpathlineto{\pgfqpoint{1.491038in}{1.314728in}}%
\pgfpathlineto{\pgfqpoint{1.493221in}{1.335267in}}%
\pgfpathlineto{\pgfqpoint{1.547834in}{1.329763in}}%
\pgfpathlineto{\pgfqpoint{1.581597in}{1.326016in}}%
\pgfpathlineto{\pgfqpoint{1.579920in}{1.306214in}}%
\pgfpathlineto{\pgfqpoint{1.626877in}{1.302163in}}%
\pgfpathlineto{\pgfqpoint{1.670049in}{1.298787in}}%
\pgfpathclose%
\pgfusepath{fill}%
\end{pgfscope}%
\begin{pgfscope}%
\pgfpathrectangle{\pgfqpoint{0.100000in}{0.100000in}}{\pgfqpoint{3.608454in}{2.310000in}}%
\pgfusepath{clip}%
\pgfsetbuttcap%
\pgfsetmiterjoin%
\definecolor{currentfill}{rgb}{0.000000,0.431373,0.784314}%
\pgfsetfillcolor{currentfill}%
\pgfsetlinewidth{0.000000pt}%
\definecolor{currentstroke}{rgb}{0.000000,0.000000,0.000000}%
\pgfsetstrokecolor{currentstroke}%
\pgfsetstrokeopacity{0.000000}%
\pgfsetdash{}{0pt}%
\pgfpathmoveto{\pgfqpoint{2.103336in}{1.338237in}}%
\pgfpathlineto{\pgfqpoint{2.096838in}{1.338167in}}%
\pgfpathlineto{\pgfqpoint{2.095756in}{1.349849in}}%
\pgfpathlineto{\pgfqpoint{2.095818in}{1.372775in}}%
\pgfpathlineto{\pgfqpoint{2.071958in}{1.372885in}}%
\pgfpathlineto{\pgfqpoint{2.072459in}{1.391367in}}%
\pgfpathlineto{\pgfqpoint{2.086114in}{1.391298in}}%
\pgfpathlineto{\pgfqpoint{2.099783in}{1.391231in}}%
\pgfpathlineto{\pgfqpoint{2.103446in}{1.388933in}}%
\pgfpathlineto{\pgfqpoint{2.100521in}{1.381808in}}%
\pgfpathlineto{\pgfqpoint{2.107556in}{1.374791in}}%
\pgfpathlineto{\pgfqpoint{2.116197in}{1.360877in}}%
\pgfpathlineto{\pgfqpoint{2.112893in}{1.355625in}}%
\pgfpathlineto{\pgfqpoint{2.112374in}{1.338940in}}%
\pgfpathlineto{\pgfqpoint{2.103336in}{1.338237in}}%
\pgfpathclose%
\pgfusepath{fill}%
\end{pgfscope}%
\begin{pgfscope}%
\pgfpathrectangle{\pgfqpoint{0.100000in}{0.100000in}}{\pgfqpoint{3.608454in}{2.310000in}}%
\pgfusepath{clip}%
\pgfsetbuttcap%
\pgfsetmiterjoin%
\definecolor{currentfill}{rgb}{0.000000,0.517647,0.741176}%
\pgfsetfillcolor{currentfill}%
\pgfsetlinewidth{0.000000pt}%
\definecolor{currentstroke}{rgb}{0.000000,0.000000,0.000000}%
\pgfsetstrokecolor{currentstroke}%
\pgfsetstrokeopacity{0.000000}%
\pgfsetdash{}{0pt}%
\pgfpathmoveto{\pgfqpoint{2.760784in}{0.972182in}}%
\pgfpathlineto{\pgfqpoint{2.738788in}{0.969584in}}%
\pgfpathlineto{\pgfqpoint{2.737463in}{0.974424in}}%
\pgfpathlineto{\pgfqpoint{2.728983in}{0.972881in}}%
\pgfpathlineto{\pgfqpoint{2.715165in}{0.972447in}}%
\pgfpathlineto{\pgfqpoint{2.706101in}{0.993510in}}%
\pgfpathlineto{\pgfqpoint{2.710524in}{0.995150in}}%
\pgfpathlineto{\pgfqpoint{2.718615in}{1.004553in}}%
\pgfpathlineto{\pgfqpoint{2.721545in}{1.014250in}}%
\pgfpathlineto{\pgfqpoint{2.725316in}{1.017812in}}%
\pgfpathlineto{\pgfqpoint{2.723904in}{1.022807in}}%
\pgfpathlineto{\pgfqpoint{2.754206in}{1.025936in}}%
\pgfpathlineto{\pgfqpoint{2.753682in}{1.036469in}}%
\pgfpathlineto{\pgfqpoint{2.746037in}{1.035657in}}%
\pgfpathlineto{\pgfqpoint{2.747587in}{1.043751in}}%
\pgfpathlineto{\pgfqpoint{2.752122in}{1.044231in}}%
\pgfpathlineto{\pgfqpoint{2.756131in}{1.058082in}}%
\pgfpathlineto{\pgfqpoint{2.779057in}{1.060719in}}%
\pgfpathlineto{\pgfqpoint{2.780160in}{1.050381in}}%
\pgfpathlineto{\pgfqpoint{2.782890in}{1.048115in}}%
\pgfpathlineto{\pgfqpoint{2.779095in}{1.040030in}}%
\pgfpathlineto{\pgfqpoint{2.780892in}{1.025876in}}%
\pgfpathlineto{\pgfqpoint{2.784817in}{0.988730in}}%
\pgfpathlineto{\pgfqpoint{2.767572in}{0.987144in}}%
\pgfpathlineto{\pgfqpoint{2.765064in}{0.976399in}}%
\pgfpathlineto{\pgfqpoint{2.760784in}{0.972182in}}%
\pgfpathclose%
\pgfusepath{fill}%
\end{pgfscope}%
\begin{pgfscope}%
\pgfpathrectangle{\pgfqpoint{0.100000in}{0.100000in}}{\pgfqpoint{3.608454in}{2.310000in}}%
\pgfusepath{clip}%
\pgfsetbuttcap%
\pgfsetmiterjoin%
\definecolor{currentfill}{rgb}{0.000000,0.431373,0.784314}%
\pgfsetfillcolor{currentfill}%
\pgfsetlinewidth{0.000000pt}%
\definecolor{currentstroke}{rgb}{0.000000,0.000000,0.000000}%
\pgfsetstrokecolor{currentstroke}%
\pgfsetstrokeopacity{0.000000}%
\pgfsetdash{}{0pt}%
\pgfpathmoveto{\pgfqpoint{1.716695in}{1.867335in}}%
\pgfpathlineto{\pgfqpoint{1.714041in}{1.832923in}}%
\pgfpathlineto{\pgfqpoint{1.660454in}{1.837408in}}%
\pgfpathlineto{\pgfqpoint{1.661648in}{1.851140in}}%
\pgfpathlineto{\pgfqpoint{1.667609in}{1.909112in}}%
\pgfpathlineto{\pgfqpoint{1.664670in}{1.909384in}}%
\pgfpathlineto{\pgfqpoint{1.667012in}{1.936010in}}%
\pgfpathlineto{\pgfqpoint{1.670909in}{1.935733in}}%
\pgfpathlineto{\pgfqpoint{1.673140in}{1.963430in}}%
\pgfpathlineto{\pgfqpoint{1.718524in}{1.959750in}}%
\pgfpathlineto{\pgfqpoint{1.721015in}{1.959541in}}%
\pgfpathlineto{\pgfqpoint{1.719106in}{1.931926in}}%
\pgfpathlineto{\pgfqpoint{1.721787in}{1.931717in}}%
\pgfpathlineto{\pgfqpoint{1.720336in}{1.913435in}}%
\pgfpathlineto{\pgfqpoint{1.716695in}{1.867335in}}%
\pgfpathclose%
\pgfusepath{fill}%
\end{pgfscope}%
\begin{pgfscope}%
\pgfpathrectangle{\pgfqpoint{0.100000in}{0.100000in}}{\pgfqpoint{3.608454in}{2.310000in}}%
\pgfusepath{clip}%
\pgfsetbuttcap%
\pgfsetmiterjoin%
\definecolor{currentfill}{rgb}{0.000000,0.647059,0.676471}%
\pgfsetfillcolor{currentfill}%
\pgfsetlinewidth{0.000000pt}%
\definecolor{currentstroke}{rgb}{0.000000,0.000000,0.000000}%
\pgfsetstrokecolor{currentstroke}%
\pgfsetstrokeopacity{0.000000}%
\pgfsetdash{}{0pt}%
\pgfpathmoveto{\pgfqpoint{3.298349in}{1.383037in}}%
\pgfpathlineto{\pgfqpoint{3.304739in}{1.388327in}}%
\pgfpathlineto{\pgfqpoint{3.327417in}{1.395926in}}%
\pgfpathlineto{\pgfqpoint{3.322504in}{1.381252in}}%
\pgfpathlineto{\pgfqpoint{3.318671in}{1.381735in}}%
\pgfpathlineto{\pgfqpoint{3.314476in}{1.374248in}}%
\pgfpathlineto{\pgfqpoint{3.312084in}{1.358499in}}%
\pgfpathlineto{\pgfqpoint{3.313688in}{1.355781in}}%
\pgfpathlineto{\pgfqpoint{3.309070in}{1.336218in}}%
\pgfpathlineto{\pgfqpoint{3.307238in}{1.324528in}}%
\pgfpathlineto{\pgfqpoint{3.302528in}{1.316927in}}%
\pgfpathlineto{\pgfqpoint{3.298046in}{1.315923in}}%
\pgfpathlineto{\pgfqpoint{3.292330in}{1.326062in}}%
\pgfpathlineto{\pgfqpoint{3.292495in}{1.350793in}}%
\pgfpathlineto{\pgfqpoint{3.295985in}{1.361073in}}%
\pgfpathlineto{\pgfqpoint{3.296227in}{1.370099in}}%
\pgfpathlineto{\pgfqpoint{3.301816in}{1.371758in}}%
\pgfpathlineto{\pgfqpoint{3.303055in}{1.378748in}}%
\pgfpathlineto{\pgfqpoint{3.298349in}{1.383037in}}%
\pgfpathclose%
\pgfusepath{fill}%
\end{pgfscope}%
\begin{pgfscope}%
\pgfpathrectangle{\pgfqpoint{0.100000in}{0.100000in}}{\pgfqpoint{3.608454in}{2.310000in}}%
\pgfusepath{clip}%
\pgfsetbuttcap%
\pgfsetmiterjoin%
\definecolor{currentfill}{rgb}{0.000000,0.603922,0.698039}%
\pgfsetfillcolor{currentfill}%
\pgfsetlinewidth{0.000000pt}%
\definecolor{currentstroke}{rgb}{0.000000,0.000000,0.000000}%
\pgfsetstrokecolor{currentstroke}%
\pgfsetstrokeopacity{0.000000}%
\pgfsetdash{}{0pt}%
\pgfpathmoveto{\pgfqpoint{2.488486in}{1.795943in}}%
\pgfpathlineto{\pgfqpoint{2.468096in}{1.794922in}}%
\pgfpathlineto{\pgfqpoint{2.467658in}{1.801883in}}%
\pgfpathlineto{\pgfqpoint{2.432784in}{1.799789in}}%
\pgfpathlineto{\pgfqpoint{2.430551in}{1.834455in}}%
\pgfpathlineto{\pgfqpoint{2.419246in}{1.833798in}}%
\pgfpathlineto{\pgfqpoint{2.417621in}{1.861430in}}%
\pgfpathlineto{\pgfqpoint{2.445567in}{1.863054in}}%
\pgfpathlineto{\pgfqpoint{2.445861in}{1.856178in}}%
\pgfpathlineto{\pgfqpoint{2.459694in}{1.857127in}}%
\pgfpathlineto{\pgfqpoint{2.460529in}{1.843339in}}%
\pgfpathlineto{\pgfqpoint{2.463191in}{1.836540in}}%
\pgfpathlineto{\pgfqpoint{2.472001in}{1.837128in}}%
\pgfpathlineto{\pgfqpoint{2.473187in}{1.816374in}}%
\pgfpathlineto{\pgfqpoint{2.486722in}{1.817142in}}%
\pgfpathlineto{\pgfqpoint{2.488486in}{1.795943in}}%
\pgfpathclose%
\pgfusepath{fill}%
\end{pgfscope}%
\begin{pgfscope}%
\pgfpathrectangle{\pgfqpoint{0.100000in}{0.100000in}}{\pgfqpoint{3.608454in}{2.310000in}}%
\pgfusepath{clip}%
\pgfsetbuttcap%
\pgfsetmiterjoin%
\definecolor{currentfill}{rgb}{0.000000,0.556863,0.721569}%
\pgfsetfillcolor{currentfill}%
\pgfsetlinewidth{0.000000pt}%
\definecolor{currentstroke}{rgb}{0.000000,0.000000,0.000000}%
\pgfsetstrokecolor{currentstroke}%
\pgfsetstrokeopacity{0.000000}%
\pgfsetdash{}{0pt}%
\pgfpathmoveto{\pgfqpoint{3.384285in}{1.735025in}}%
\pgfpathlineto{\pgfqpoint{3.359316in}{1.729735in}}%
\pgfpathlineto{\pgfqpoint{3.358769in}{1.729605in}}%
\pgfpathlineto{\pgfqpoint{3.357447in}{1.732265in}}%
\pgfpathlineto{\pgfqpoint{3.358955in}{1.786375in}}%
\pgfpathlineto{\pgfqpoint{3.356483in}{1.793242in}}%
\pgfpathlineto{\pgfqpoint{3.348859in}{1.830296in}}%
\pgfpathlineto{\pgfqpoint{3.370781in}{1.834556in}}%
\pgfpathlineto{\pgfqpoint{3.374364in}{1.831752in}}%
\pgfpathlineto{\pgfqpoint{3.374319in}{1.820118in}}%
\pgfpathlineto{\pgfqpoint{3.367008in}{1.818791in}}%
\pgfpathlineto{\pgfqpoint{3.369972in}{1.804097in}}%
\pgfpathlineto{\pgfqpoint{3.374916in}{1.805051in}}%
\pgfpathlineto{\pgfqpoint{3.377987in}{1.790511in}}%
\pgfpathlineto{\pgfqpoint{3.373418in}{1.786388in}}%
\pgfpathlineto{\pgfqpoint{3.378669in}{1.782710in}}%
\pgfpathlineto{\pgfqpoint{3.380209in}{1.761732in}}%
\pgfpathlineto{\pgfqpoint{3.384285in}{1.735025in}}%
\pgfpathclose%
\pgfusepath{fill}%
\end{pgfscope}%
\begin{pgfscope}%
\pgfpathrectangle{\pgfqpoint{0.100000in}{0.100000in}}{\pgfqpoint{3.608454in}{2.310000in}}%
\pgfusepath{clip}%
\pgfsetbuttcap%
\pgfsetmiterjoin%
\definecolor{currentfill}{rgb}{0.000000,0.752941,0.623529}%
\pgfsetfillcolor{currentfill}%
\pgfsetlinewidth{0.000000pt}%
\definecolor{currentstroke}{rgb}{0.000000,0.000000,0.000000}%
\pgfsetstrokecolor{currentstroke}%
\pgfsetstrokeopacity{0.000000}%
\pgfsetdash{}{0pt}%
\pgfpathmoveto{\pgfqpoint{2.895571in}{1.311319in}}%
\pgfpathlineto{\pgfqpoint{2.890443in}{1.304155in}}%
\pgfpathlineto{\pgfqpoint{2.884060in}{1.308388in}}%
\pgfpathlineto{\pgfqpoint{2.885378in}{1.313095in}}%
\pgfpathlineto{\pgfqpoint{2.873043in}{1.323790in}}%
\pgfpathlineto{\pgfqpoint{2.874015in}{1.332782in}}%
\pgfpathlineto{\pgfqpoint{2.861780in}{1.330306in}}%
\pgfpathlineto{\pgfqpoint{2.855274in}{1.324556in}}%
\pgfpathlineto{\pgfqpoint{2.857831in}{1.319548in}}%
\pgfpathlineto{\pgfqpoint{2.850698in}{1.310252in}}%
\pgfpathlineto{\pgfqpoint{2.840076in}{1.309133in}}%
\pgfpathlineto{\pgfqpoint{2.834982in}{1.312343in}}%
\pgfpathlineto{\pgfqpoint{2.824999in}{1.309058in}}%
\pgfpathlineto{\pgfqpoint{2.811045in}{1.319515in}}%
\pgfpathlineto{\pgfqpoint{2.818045in}{1.327218in}}%
\pgfpathlineto{\pgfqpoint{2.817804in}{1.333230in}}%
\pgfpathlineto{\pgfqpoint{2.822416in}{1.337584in}}%
\pgfpathlineto{\pgfqpoint{2.826872in}{1.332234in}}%
\pgfpathlineto{\pgfqpoint{2.834430in}{1.334621in}}%
\pgfpathlineto{\pgfqpoint{2.833231in}{1.340587in}}%
\pgfpathlineto{\pgfqpoint{2.835681in}{1.347268in}}%
\pgfpathlineto{\pgfqpoint{2.841606in}{1.355934in}}%
\pgfpathlineto{\pgfqpoint{2.842704in}{1.366486in}}%
\pgfpathlineto{\pgfqpoint{2.850733in}{1.370027in}}%
\pgfpathlineto{\pgfqpoint{2.854867in}{1.358657in}}%
\pgfpathlineto{\pgfqpoint{2.861822in}{1.358808in}}%
\pgfpathlineto{\pgfqpoint{2.864952in}{1.371264in}}%
\pgfpathlineto{\pgfqpoint{2.864652in}{1.379289in}}%
\pgfpathlineto{\pgfqpoint{2.869245in}{1.379552in}}%
\pgfpathlineto{\pgfqpoint{2.869516in}{1.374371in}}%
\pgfpathlineto{\pgfqpoint{2.875638in}{1.374674in}}%
\pgfpathlineto{\pgfqpoint{2.877293in}{1.367722in}}%
\pgfpathlineto{\pgfqpoint{2.884573in}{1.368240in}}%
\pgfpathlineto{\pgfqpoint{2.885129in}{1.361064in}}%
\pgfpathlineto{\pgfqpoint{2.893788in}{1.362802in}}%
\pgfpathlineto{\pgfqpoint{2.905122in}{1.355064in}}%
\pgfpathlineto{\pgfqpoint{2.906738in}{1.347296in}}%
\pgfpathlineto{\pgfqpoint{2.898128in}{1.341949in}}%
\pgfpathlineto{\pgfqpoint{2.894444in}{1.332079in}}%
\pgfpathlineto{\pgfqpoint{2.894629in}{1.327193in}}%
\pgfpathlineto{\pgfqpoint{2.900793in}{1.318863in}}%
\pgfpathlineto{\pgfqpoint{2.895571in}{1.311319in}}%
\pgfpathclose%
\pgfusepath{fill}%
\end{pgfscope}%
\begin{pgfscope}%
\pgfpathrectangle{\pgfqpoint{0.100000in}{0.100000in}}{\pgfqpoint{3.608454in}{2.310000in}}%
\pgfusepath{clip}%
\pgfsetbuttcap%
\pgfsetmiterjoin%
\definecolor{currentfill}{rgb}{0.000000,0.623529,0.688235}%
\pgfsetfillcolor{currentfill}%
\pgfsetlinewidth{0.000000pt}%
\definecolor{currentstroke}{rgb}{0.000000,0.000000,0.000000}%
\pgfsetstrokecolor{currentstroke}%
\pgfsetstrokeopacity{0.000000}%
\pgfsetdash{}{0pt}%
\pgfpathmoveto{\pgfqpoint{1.413364in}{1.067432in}}%
\pgfpathlineto{\pgfqpoint{1.379252in}{1.071453in}}%
\pgfpathlineto{\pgfqpoint{1.385971in}{1.127718in}}%
\pgfpathlineto{\pgfqpoint{1.377042in}{1.134592in}}%
\pgfpathlineto{\pgfqpoint{1.378347in}{1.144995in}}%
\pgfpathlineto{\pgfqpoint{1.388440in}{1.147038in}}%
\pgfpathlineto{\pgfqpoint{1.347890in}{1.152024in}}%
\pgfpathlineto{\pgfqpoint{1.350154in}{1.169097in}}%
\pgfpathlineto{\pgfqpoint{1.303419in}{1.175695in}}%
\pgfpathlineto{\pgfqpoint{1.310197in}{1.221478in}}%
\pgfpathlineto{\pgfqpoint{1.322223in}{1.228643in}}%
\pgfpathlineto{\pgfqpoint{1.324687in}{1.235061in}}%
\pgfpathlineto{\pgfqpoint{1.383780in}{1.226748in}}%
\pgfpathlineto{\pgfqpoint{1.462544in}{1.217413in}}%
\pgfpathlineto{\pgfqpoint{1.462472in}{1.211222in}}%
\pgfpathlineto{\pgfqpoint{1.458207in}{1.195789in}}%
\pgfpathlineto{\pgfqpoint{1.450272in}{1.192914in}}%
\pgfpathlineto{\pgfqpoint{1.450548in}{1.172051in}}%
\pgfpathlineto{\pgfqpoint{1.448410in}{1.160662in}}%
\pgfpathlineto{\pgfqpoint{1.445381in}{1.160260in}}%
\pgfpathlineto{\pgfqpoint{1.442215in}{1.150974in}}%
\pgfpathlineto{\pgfqpoint{1.431076in}{1.139578in}}%
\pgfpathlineto{\pgfqpoint{1.421911in}{1.141159in}}%
\pgfpathlineto{\pgfqpoint{1.413364in}{1.067432in}}%
\pgfpathclose%
\pgfusepath{fill}%
\end{pgfscope}%
\begin{pgfscope}%
\pgfpathrectangle{\pgfqpoint{0.100000in}{0.100000in}}{\pgfqpoint{3.608454in}{2.310000in}}%
\pgfusepath{clip}%
\pgfsetbuttcap%
\pgfsetmiterjoin%
\definecolor{currentfill}{rgb}{0.000000,0.847059,0.576471}%
\pgfsetfillcolor{currentfill}%
\pgfsetlinewidth{0.000000pt}%
\definecolor{currentstroke}{rgb}{0.000000,0.000000,0.000000}%
\pgfsetstrokecolor{currentstroke}%
\pgfsetstrokeopacity{0.000000}%
\pgfsetdash{}{0pt}%
\pgfpathmoveto{\pgfqpoint{2.407413in}{1.214350in}}%
\pgfpathlineto{\pgfqpoint{2.386546in}{1.213497in}}%
\pgfpathlineto{\pgfqpoint{2.386664in}{1.209942in}}%
\pgfpathlineto{\pgfqpoint{2.375052in}{1.209261in}}%
\pgfpathlineto{\pgfqpoint{2.372317in}{1.217256in}}%
\pgfpathlineto{\pgfqpoint{2.371740in}{1.235384in}}%
\pgfpathlineto{\pgfqpoint{2.358074in}{1.234831in}}%
\pgfpathlineto{\pgfqpoint{2.347612in}{1.233131in}}%
\pgfpathlineto{\pgfqpoint{2.347355in}{1.241765in}}%
\pgfpathlineto{\pgfqpoint{2.350544in}{1.245278in}}%
\pgfpathlineto{\pgfqpoint{2.349233in}{1.281942in}}%
\pgfpathlineto{\pgfqpoint{2.368816in}{1.282820in}}%
\pgfpathlineto{\pgfqpoint{2.381801in}{1.267561in}}%
\pgfpathlineto{\pgfqpoint{2.392056in}{1.271141in}}%
\pgfpathlineto{\pgfqpoint{2.397241in}{1.275533in}}%
\pgfpathlineto{\pgfqpoint{2.404880in}{1.275417in}}%
\pgfpathlineto{\pgfqpoint{2.409704in}{1.273645in}}%
\pgfpathlineto{\pgfqpoint{2.418250in}{1.266369in}}%
\pgfpathlineto{\pgfqpoint{2.421525in}{1.260176in}}%
\pgfpathlineto{\pgfqpoint{2.428315in}{1.262231in}}%
\pgfpathlineto{\pgfqpoint{2.434982in}{1.258000in}}%
\pgfpathlineto{\pgfqpoint{2.445368in}{1.248025in}}%
\pgfpathlineto{\pgfqpoint{2.449790in}{1.246442in}}%
\pgfpathlineto{\pgfqpoint{2.452467in}{1.239392in}}%
\pgfpathlineto{\pgfqpoint{2.449905in}{1.236650in}}%
\pgfpathlineto{\pgfqpoint{2.444353in}{1.238826in}}%
\pgfpathlineto{\pgfqpoint{2.410673in}{1.236890in}}%
\pgfpathlineto{\pgfqpoint{2.411944in}{1.214363in}}%
\pgfpathlineto{\pgfqpoint{2.407413in}{1.214350in}}%
\pgfpathclose%
\pgfusepath{fill}%
\end{pgfscope}%
\begin{pgfscope}%
\pgfpathrectangle{\pgfqpoint{0.100000in}{0.100000in}}{\pgfqpoint{3.608454in}{2.310000in}}%
\pgfusepath{clip}%
\pgfsetbuttcap%
\pgfsetmiterjoin%
\definecolor{currentfill}{rgb}{0.000000,0.647059,0.676471}%
\pgfsetfillcolor{currentfill}%
\pgfsetlinewidth{0.000000pt}%
\definecolor{currentstroke}{rgb}{0.000000,0.000000,0.000000}%
\pgfsetstrokecolor{currentstroke}%
\pgfsetstrokeopacity{0.000000}%
\pgfsetdash{}{0pt}%
\pgfpathmoveto{\pgfqpoint{2.751930in}{0.830708in}}%
\pgfpathlineto{\pgfqpoint{2.751810in}{0.820152in}}%
\pgfpathlineto{\pgfqpoint{2.741287in}{0.817640in}}%
\pgfpathlineto{\pgfqpoint{2.737520in}{0.807875in}}%
\pgfpathlineto{\pgfqpoint{2.728545in}{0.806963in}}%
\pgfpathlineto{\pgfqpoint{2.727912in}{0.813855in}}%
\pgfpathlineto{\pgfqpoint{2.721575in}{0.813214in}}%
\pgfpathlineto{\pgfqpoint{2.720311in}{0.819497in}}%
\pgfpathlineto{\pgfqpoint{2.713352in}{0.819097in}}%
\pgfpathlineto{\pgfqpoint{2.711685in}{0.841617in}}%
\pgfpathlineto{\pgfqpoint{2.708963in}{0.847375in}}%
\pgfpathlineto{\pgfqpoint{2.717264in}{0.854842in}}%
\pgfpathlineto{\pgfqpoint{2.723195in}{0.855536in}}%
\pgfpathlineto{\pgfqpoint{2.722392in}{0.862394in}}%
\pgfpathlineto{\pgfqpoint{2.729130in}{0.864274in}}%
\pgfpathlineto{\pgfqpoint{2.728245in}{0.872298in}}%
\pgfpathlineto{\pgfqpoint{2.734886in}{0.875470in}}%
\pgfpathlineto{\pgfqpoint{2.755362in}{0.877808in}}%
\pgfpathlineto{\pgfqpoint{2.765201in}{0.880136in}}%
\pgfpathlineto{\pgfqpoint{2.768066in}{0.872209in}}%
\pgfpathlineto{\pgfqpoint{2.776169in}{0.862453in}}%
\pgfpathlineto{\pgfqpoint{2.772613in}{0.859054in}}%
\pgfpathlineto{\pgfqpoint{2.754330in}{0.856732in}}%
\pgfpathlineto{\pgfqpoint{2.755065in}{0.852187in}}%
\pgfpathlineto{\pgfqpoint{2.748134in}{0.851370in}}%
\pgfpathlineto{\pgfqpoint{2.749632in}{0.837509in}}%
\pgfpathlineto{\pgfqpoint{2.751930in}{0.830708in}}%
\pgfpathclose%
\pgfusepath{fill}%
\end{pgfscope}%
\begin{pgfscope}%
\pgfpathrectangle{\pgfqpoint{0.100000in}{0.100000in}}{\pgfqpoint{3.608454in}{2.310000in}}%
\pgfusepath{clip}%
\pgfsetbuttcap%
\pgfsetmiterjoin%
\definecolor{currentfill}{rgb}{0.000000,0.474510,0.762745}%
\pgfsetfillcolor{currentfill}%
\pgfsetlinewidth{0.000000pt}%
\definecolor{currentstroke}{rgb}{0.000000,0.000000,0.000000}%
\pgfsetstrokecolor{currentstroke}%
\pgfsetstrokeopacity{0.000000}%
\pgfsetdash{}{0pt}%
\pgfpathmoveto{\pgfqpoint{0.975129in}{1.855275in}}%
\pgfpathlineto{\pgfqpoint{0.973897in}{1.850229in}}%
\pgfpathlineto{\pgfqpoint{0.967810in}{1.841171in}}%
\pgfpathlineto{\pgfqpoint{0.968859in}{1.837111in}}%
\pgfpathlineto{\pgfqpoint{0.963091in}{1.830834in}}%
\pgfpathlineto{\pgfqpoint{0.950896in}{1.777540in}}%
\pgfpathlineto{\pgfqpoint{0.953662in}{1.776726in}}%
\pgfpathlineto{\pgfqpoint{0.937739in}{1.705964in}}%
\pgfpathlineto{\pgfqpoint{0.861529in}{1.723655in}}%
\pgfpathlineto{\pgfqpoint{0.824452in}{1.732928in}}%
\pgfpathlineto{\pgfqpoint{0.824016in}{1.733109in}}%
\pgfpathlineto{\pgfqpoint{0.855951in}{1.862354in}}%
\pgfpathlineto{\pgfqpoint{0.860241in}{1.876128in}}%
\pgfpathlineto{\pgfqpoint{0.863735in}{1.876688in}}%
\pgfpathlineto{\pgfqpoint{0.867712in}{1.868584in}}%
\pgfpathlineto{\pgfqpoint{0.875796in}{1.867766in}}%
\pgfpathlineto{\pgfqpoint{0.879074in}{1.881212in}}%
\pgfpathlineto{\pgfqpoint{0.886843in}{1.879251in}}%
\pgfpathlineto{\pgfqpoint{0.890663in}{1.885325in}}%
\pgfpathlineto{\pgfqpoint{0.895063in}{1.884202in}}%
\pgfpathlineto{\pgfqpoint{0.896690in}{1.890839in}}%
\pgfpathlineto{\pgfqpoint{0.901859in}{1.889570in}}%
\pgfpathlineto{\pgfqpoint{0.905607in}{1.902658in}}%
\pgfpathlineto{\pgfqpoint{0.910516in}{1.911417in}}%
\pgfpathlineto{\pgfqpoint{0.919645in}{1.913720in}}%
\pgfpathlineto{\pgfqpoint{0.916455in}{1.900234in}}%
\pgfpathlineto{\pgfqpoint{0.913150in}{1.901010in}}%
\pgfpathlineto{\pgfqpoint{0.909972in}{1.887658in}}%
\pgfpathlineto{\pgfqpoint{0.915455in}{1.886073in}}%
\pgfpathlineto{\pgfqpoint{0.917584in}{1.892808in}}%
\pgfpathlineto{\pgfqpoint{0.949827in}{1.885175in}}%
\pgfpathlineto{\pgfqpoint{0.953430in}{1.884971in}}%
\pgfpathlineto{\pgfqpoint{0.964170in}{1.890384in}}%
\pgfpathlineto{\pgfqpoint{0.969814in}{1.885604in}}%
\pgfpathlineto{\pgfqpoint{0.968833in}{1.878204in}}%
\pgfpathlineto{\pgfqpoint{0.975886in}{1.873015in}}%
\pgfpathlineto{\pgfqpoint{0.972986in}{1.865357in}}%
\pgfpathlineto{\pgfqpoint{0.975129in}{1.855275in}}%
\pgfpathclose%
\pgfusepath{fill}%
\end{pgfscope}%
\begin{pgfscope}%
\pgfpathrectangle{\pgfqpoint{0.100000in}{0.100000in}}{\pgfqpoint{3.608454in}{2.310000in}}%
\pgfusepath{clip}%
\pgfsetbuttcap%
\pgfsetmiterjoin%
\definecolor{currentfill}{rgb}{0.000000,0.607843,0.696078}%
\pgfsetfillcolor{currentfill}%
\pgfsetlinewidth{0.000000pt}%
\definecolor{currentstroke}{rgb}{0.000000,0.000000,0.000000}%
\pgfsetstrokecolor{currentstroke}%
\pgfsetstrokeopacity{0.000000}%
\pgfsetdash{}{0pt}%
\pgfpathmoveto{\pgfqpoint{2.272778in}{0.593877in}}%
\pgfpathlineto{\pgfqpoint{2.264140in}{0.596489in}}%
\pgfpathlineto{\pgfqpoint{2.246043in}{0.604519in}}%
\pgfpathlineto{\pgfqpoint{2.232132in}{0.608488in}}%
\pgfpathlineto{\pgfqpoint{2.221611in}{0.607031in}}%
\pgfpathlineto{\pgfqpoint{2.210990in}{0.607615in}}%
\pgfpathlineto{\pgfqpoint{2.192546in}{0.604264in}}%
\pgfpathlineto{\pgfqpoint{2.187595in}{0.601129in}}%
\pgfpathlineto{\pgfqpoint{2.181395in}{0.610955in}}%
\pgfpathlineto{\pgfqpoint{2.188702in}{0.619056in}}%
\pgfpathlineto{\pgfqpoint{2.190675in}{0.624923in}}%
\pgfpathlineto{\pgfqpoint{2.196138in}{0.630275in}}%
\pgfpathlineto{\pgfqpoint{2.196201in}{0.649140in}}%
\pgfpathlineto{\pgfqpoint{2.191936in}{0.652528in}}%
\pgfpathlineto{\pgfqpoint{2.196505in}{0.661211in}}%
\pgfpathlineto{\pgfqpoint{2.193927in}{0.668423in}}%
\pgfpathlineto{\pgfqpoint{2.197286in}{0.677217in}}%
\pgfpathlineto{\pgfqpoint{2.200790in}{0.680240in}}%
\pgfpathlineto{\pgfqpoint{2.204721in}{0.696052in}}%
\pgfpathlineto{\pgfqpoint{2.207696in}{0.700477in}}%
\pgfpathlineto{\pgfqpoint{2.204984in}{0.705199in}}%
\pgfpathlineto{\pgfqpoint{2.205558in}{0.720837in}}%
\pgfpathlineto{\pgfqpoint{2.206617in}{0.728100in}}%
\pgfpathlineto{\pgfqpoint{2.212945in}{0.728230in}}%
\pgfpathlineto{\pgfqpoint{2.216369in}{0.735263in}}%
\pgfpathlineto{\pgfqpoint{2.226879in}{0.735487in}}%
\pgfpathlineto{\pgfqpoint{2.240752in}{0.735755in}}%
\pgfpathlineto{\pgfqpoint{2.244244in}{0.734424in}}%
\pgfpathlineto{\pgfqpoint{2.251222in}{0.732248in}}%
\pgfpathlineto{\pgfqpoint{2.254758in}{0.727139in}}%
\pgfpathlineto{\pgfqpoint{2.255860in}{0.698248in}}%
\pgfpathlineto{\pgfqpoint{2.271317in}{0.699125in}}%
\pgfpathlineto{\pgfqpoint{2.272007in}{0.674022in}}%
\pgfpathlineto{\pgfqpoint{2.269704in}{0.666105in}}%
\pgfpathlineto{\pgfqpoint{2.271892in}{0.630559in}}%
\pgfpathlineto{\pgfqpoint{2.272778in}{0.593877in}}%
\pgfpathclose%
\pgfusepath{fill}%
\end{pgfscope}%
\begin{pgfscope}%
\pgfpathrectangle{\pgfqpoint{0.100000in}{0.100000in}}{\pgfqpoint{3.608454in}{2.310000in}}%
\pgfusepath{clip}%
\pgfsetbuttcap%
\pgfsetmiterjoin%
\definecolor{currentfill}{rgb}{0.000000,0.850980,0.574510}%
\pgfsetfillcolor{currentfill}%
\pgfsetlinewidth{0.000000pt}%
\definecolor{currentstroke}{rgb}{0.000000,0.000000,0.000000}%
\pgfsetstrokecolor{currentstroke}%
\pgfsetstrokeopacity{0.000000}%
\pgfsetdash{}{0pt}%
\pgfpathmoveto{\pgfqpoint{0.792611in}{1.237224in}}%
\pgfpathlineto{\pgfqpoint{0.786789in}{1.237324in}}%
\pgfpathlineto{\pgfqpoint{0.765878in}{1.242208in}}%
\pgfpathlineto{\pgfqpoint{0.692238in}{1.260226in}}%
\pgfpathlineto{\pgfqpoint{0.668493in}{1.266395in}}%
\pgfpathlineto{\pgfqpoint{0.645007in}{1.271759in}}%
\pgfpathlineto{\pgfqpoint{0.648713in}{1.279430in}}%
\pgfpathlineto{\pgfqpoint{0.647370in}{1.296028in}}%
\pgfpathlineto{\pgfqpoint{0.649389in}{1.301776in}}%
\pgfpathlineto{\pgfqpoint{0.647795in}{1.312826in}}%
\pgfpathlineto{\pgfqpoint{0.650356in}{1.315914in}}%
\pgfpathlineto{\pgfqpoint{0.644042in}{1.329081in}}%
\pgfpathlineto{\pgfqpoint{0.644925in}{1.333356in}}%
\pgfpathlineto{\pgfqpoint{0.641670in}{1.344858in}}%
\pgfpathlineto{\pgfqpoint{0.642344in}{1.357942in}}%
\pgfpathlineto{\pgfqpoint{0.645186in}{1.361551in}}%
\pgfpathlineto{\pgfqpoint{0.643441in}{1.374382in}}%
\pgfpathlineto{\pgfqpoint{0.636091in}{1.384439in}}%
\pgfpathlineto{\pgfqpoint{0.632393in}{1.385724in}}%
\pgfpathlineto{\pgfqpoint{0.632454in}{1.400962in}}%
\pgfpathlineto{\pgfqpoint{0.628697in}{1.402513in}}%
\pgfpathlineto{\pgfqpoint{0.632163in}{1.410294in}}%
\pgfpathlineto{\pgfqpoint{0.627031in}{1.414709in}}%
\pgfpathlineto{\pgfqpoint{0.624814in}{1.421196in}}%
\pgfpathlineto{\pgfqpoint{0.619173in}{1.425619in}}%
\pgfpathlineto{\pgfqpoint{0.618630in}{1.433861in}}%
\pgfpathlineto{\pgfqpoint{0.616083in}{1.438599in}}%
\pgfpathlineto{\pgfqpoint{0.607467in}{1.441382in}}%
\pgfpathlineto{\pgfqpoint{0.612588in}{1.444548in}}%
\pgfpathlineto{\pgfqpoint{0.613111in}{1.453862in}}%
\pgfpathlineto{\pgfqpoint{0.609428in}{1.457861in}}%
\pgfpathlineto{\pgfqpoint{0.610025in}{1.468870in}}%
\pgfpathlineto{\pgfqpoint{0.597769in}{1.478364in}}%
\pgfpathlineto{\pgfqpoint{0.596004in}{1.489421in}}%
\pgfpathlineto{\pgfqpoint{0.597596in}{1.492295in}}%
\pgfpathlineto{\pgfqpoint{0.603382in}{1.497016in}}%
\pgfpathlineto{\pgfqpoint{0.606996in}{1.503863in}}%
\pgfpathlineto{\pgfqpoint{0.604651in}{1.513031in}}%
\pgfpathlineto{\pgfqpoint{0.609163in}{1.520157in}}%
\pgfpathlineto{\pgfqpoint{0.620520in}{1.503015in}}%
\pgfpathlineto{\pgfqpoint{0.628251in}{1.491065in}}%
\pgfpathlineto{\pgfqpoint{0.661400in}{1.439746in}}%
\pgfpathlineto{\pgfqpoint{0.688870in}{1.397366in}}%
\pgfpathlineto{\pgfqpoint{0.720005in}{1.349216in}}%
\pgfpathlineto{\pgfqpoint{0.780428in}{1.255922in}}%
\pgfpathlineto{\pgfqpoint{0.792611in}{1.237224in}}%
\pgfpathclose%
\pgfusepath{fill}%
\end{pgfscope}%
\begin{pgfscope}%
\pgfpathrectangle{\pgfqpoint{0.100000in}{0.100000in}}{\pgfqpoint{3.608454in}{2.310000in}}%
\pgfusepath{clip}%
\pgfsetbuttcap%
\pgfsetmiterjoin%
\definecolor{currentfill}{rgb}{0.000000,0.666667,0.666667}%
\pgfsetfillcolor{currentfill}%
\pgfsetlinewidth{0.000000pt}%
\definecolor{currentstroke}{rgb}{0.000000,0.000000,0.000000}%
\pgfsetstrokecolor{currentstroke}%
\pgfsetstrokeopacity{0.000000}%
\pgfsetdash{}{0pt}%
\pgfpathmoveto{\pgfqpoint{2.910462in}{0.795395in}}%
\pgfpathlineto{\pgfqpoint{2.894796in}{0.794134in}}%
\pgfpathlineto{\pgfqpoint{2.891131in}{0.799434in}}%
\pgfpathlineto{\pgfqpoint{2.885535in}{0.800050in}}%
\pgfpathlineto{\pgfqpoint{2.884962in}{0.810094in}}%
\pgfpathlineto{\pgfqpoint{2.887379in}{0.811341in}}%
\pgfpathlineto{\pgfqpoint{2.884148in}{0.822135in}}%
\pgfpathlineto{\pgfqpoint{2.904439in}{0.824809in}}%
\pgfpathlineto{\pgfqpoint{2.917347in}{0.821092in}}%
\pgfpathlineto{\pgfqpoint{2.918401in}{0.811582in}}%
\pgfpathlineto{\pgfqpoint{2.914876in}{0.807165in}}%
\pgfpathlineto{\pgfqpoint{2.912735in}{0.796174in}}%
\pgfpathlineto{\pgfqpoint{2.910462in}{0.795395in}}%
\pgfpathclose%
\pgfusepath{fill}%
\end{pgfscope}%
\begin{pgfscope}%
\pgfpathrectangle{\pgfqpoint{0.100000in}{0.100000in}}{\pgfqpoint{3.608454in}{2.310000in}}%
\pgfusepath{clip}%
\pgfsetbuttcap%
\pgfsetmiterjoin%
\definecolor{currentfill}{rgb}{0.000000,0.788235,0.605882}%
\pgfsetfillcolor{currentfill}%
\pgfsetlinewidth{0.000000pt}%
\definecolor{currentstroke}{rgb}{0.000000,0.000000,0.000000}%
\pgfsetstrokecolor{currentstroke}%
\pgfsetstrokeopacity{0.000000}%
\pgfsetdash{}{0pt}%
\pgfpathmoveto{\pgfqpoint{0.465159in}{1.900474in}}%
\pgfpathlineto{\pgfqpoint{0.478228in}{1.889504in}}%
\pgfpathlineto{\pgfqpoint{0.483261in}{1.890283in}}%
\pgfpathlineto{\pgfqpoint{0.491499in}{1.886385in}}%
\pgfpathlineto{\pgfqpoint{0.495724in}{1.888283in}}%
\pgfpathlineto{\pgfqpoint{0.503506in}{1.885466in}}%
\pgfpathlineto{\pgfqpoint{0.516046in}{1.883757in}}%
\pgfpathlineto{\pgfqpoint{0.532832in}{1.891746in}}%
\pgfpathlineto{\pgfqpoint{0.538396in}{1.890064in}}%
\pgfpathlineto{\pgfqpoint{0.543027in}{1.893318in}}%
\pgfpathlineto{\pgfqpoint{0.549594in}{1.891292in}}%
\pgfpathlineto{\pgfqpoint{0.526008in}{1.816628in}}%
\pgfpathlineto{\pgfqpoint{0.456983in}{1.837942in}}%
\pgfpathlineto{\pgfqpoint{0.439878in}{1.843077in}}%
\pgfpathlineto{\pgfqpoint{0.441504in}{1.854677in}}%
\pgfpathlineto{\pgfqpoint{0.447509in}{1.860643in}}%
\pgfpathlineto{\pgfqpoint{0.444876in}{1.871177in}}%
\pgfpathlineto{\pgfqpoint{0.437497in}{1.874169in}}%
\pgfpathlineto{\pgfqpoint{0.441915in}{1.883291in}}%
\pgfpathlineto{\pgfqpoint{0.448521in}{1.882309in}}%
\pgfpathlineto{\pgfqpoint{0.455038in}{1.891185in}}%
\pgfpathlineto{\pgfqpoint{0.465159in}{1.900474in}}%
\pgfpathclose%
\pgfusepath{fill}%
\end{pgfscope}%
\begin{pgfscope}%
\pgfpathrectangle{\pgfqpoint{0.100000in}{0.100000in}}{\pgfqpoint{3.608454in}{2.310000in}}%
\pgfusepath{clip}%
\pgfsetbuttcap%
\pgfsetmiterjoin%
\definecolor{currentfill}{rgb}{0.000000,0.662745,0.668627}%
\pgfsetfillcolor{currentfill}%
\pgfsetlinewidth{0.000000pt}%
\definecolor{currentstroke}{rgb}{0.000000,0.000000,0.000000}%
\pgfsetstrokecolor{currentstroke}%
\pgfsetstrokeopacity{0.000000}%
\pgfsetdash{}{0pt}%
\pgfpathmoveto{\pgfqpoint{2.734886in}{0.875470in}}%
\pgfpathlineto{\pgfqpoint{2.728245in}{0.872298in}}%
\pgfpathlineto{\pgfqpoint{2.729130in}{0.864274in}}%
\pgfpathlineto{\pgfqpoint{2.722392in}{0.862394in}}%
\pgfpathlineto{\pgfqpoint{2.723195in}{0.855536in}}%
\pgfpathlineto{\pgfqpoint{2.717264in}{0.854842in}}%
\pgfpathlineto{\pgfqpoint{2.715600in}{0.875575in}}%
\pgfpathlineto{\pgfqpoint{2.682505in}{0.872318in}}%
\pgfpathlineto{\pgfqpoint{2.676747in}{0.876531in}}%
\pgfpathlineto{\pgfqpoint{2.672768in}{0.884091in}}%
\pgfpathlineto{\pgfqpoint{2.670991in}{0.904805in}}%
\pgfpathlineto{\pgfqpoint{2.673491in}{0.910352in}}%
\pgfpathlineto{\pgfqpoint{2.679726in}{0.915262in}}%
\pgfpathlineto{\pgfqpoint{2.677615in}{0.922675in}}%
\pgfpathlineto{\pgfqpoint{2.688862in}{0.935943in}}%
\pgfpathlineto{\pgfqpoint{2.686513in}{0.939240in}}%
\pgfpathlineto{\pgfqpoint{2.690775in}{0.946928in}}%
\pgfpathlineto{\pgfqpoint{2.697028in}{0.947184in}}%
\pgfpathlineto{\pgfqpoint{2.701405in}{0.940587in}}%
\pgfpathlineto{\pgfqpoint{2.714599in}{0.941836in}}%
\pgfpathlineto{\pgfqpoint{2.717216in}{0.935165in}}%
\pgfpathlineto{\pgfqpoint{2.725270in}{0.935776in}}%
\pgfpathlineto{\pgfqpoint{2.727775in}{0.904950in}}%
\pgfpathlineto{\pgfqpoint{2.731773in}{0.905420in}}%
\pgfpathlineto{\pgfqpoint{2.734886in}{0.875470in}}%
\pgfpathclose%
\pgfusepath{fill}%
\end{pgfscope}%
\begin{pgfscope}%
\pgfpathrectangle{\pgfqpoint{0.100000in}{0.100000in}}{\pgfqpoint{3.608454in}{2.310000in}}%
\pgfusepath{clip}%
\pgfsetbuttcap%
\pgfsetmiterjoin%
\definecolor{currentfill}{rgb}{0.000000,0.533333,0.733333}%
\pgfsetfillcolor{currentfill}%
\pgfsetlinewidth{0.000000pt}%
\definecolor{currentstroke}{rgb}{0.000000,0.000000,0.000000}%
\pgfsetstrokecolor{currentstroke}%
\pgfsetstrokeopacity{0.000000}%
\pgfsetdash{}{0pt}%
\pgfpathmoveto{\pgfqpoint{2.460863in}{1.433950in}}%
\pgfpathlineto{\pgfqpoint{2.459428in}{1.452340in}}%
\pgfpathlineto{\pgfqpoint{2.452287in}{1.455302in}}%
\pgfpathlineto{\pgfqpoint{2.450616in}{1.476514in}}%
\pgfpathlineto{\pgfqpoint{2.458726in}{1.477228in}}%
\pgfpathlineto{\pgfqpoint{2.467299in}{1.483195in}}%
\pgfpathlineto{\pgfqpoint{2.470141in}{1.490384in}}%
\pgfpathlineto{\pgfqpoint{2.468185in}{1.518191in}}%
\pgfpathlineto{\pgfqpoint{2.488674in}{1.519766in}}%
\pgfpathlineto{\pgfqpoint{2.508522in}{1.521688in}}%
\pgfpathlineto{\pgfqpoint{2.509486in}{1.512307in}}%
\pgfpathlineto{\pgfqpoint{2.512434in}{1.482660in}}%
\pgfpathlineto{\pgfqpoint{2.498999in}{1.481611in}}%
\pgfpathlineto{\pgfqpoint{2.500893in}{1.455137in}}%
\pgfpathlineto{\pgfqpoint{2.493952in}{1.454636in}}%
\pgfpathlineto{\pgfqpoint{2.484922in}{1.436043in}}%
\pgfpathlineto{\pgfqpoint{2.460863in}{1.433950in}}%
\pgfpathclose%
\pgfusepath{fill}%
\end{pgfscope}%
\begin{pgfscope}%
\pgfpathrectangle{\pgfqpoint{0.100000in}{0.100000in}}{\pgfqpoint{3.608454in}{2.310000in}}%
\pgfusepath{clip}%
\pgfsetbuttcap%
\pgfsetmiterjoin%
\definecolor{currentfill}{rgb}{0.000000,0.643137,0.678431}%
\pgfsetfillcolor{currentfill}%
\pgfsetlinewidth{0.000000pt}%
\definecolor{currentstroke}{rgb}{0.000000,0.000000,0.000000}%
\pgfsetstrokecolor{currentstroke}%
\pgfsetstrokeopacity{0.000000}%
\pgfsetdash{}{0pt}%
\pgfpathmoveto{\pgfqpoint{2.322557in}{0.730252in}}%
\pgfpathlineto{\pgfqpoint{2.333480in}{0.724393in}}%
\pgfpathlineto{\pgfqpoint{2.330790in}{0.720030in}}%
\pgfpathlineto{\pgfqpoint{2.330722in}{0.712813in}}%
\pgfpathlineto{\pgfqpoint{2.325328in}{0.707042in}}%
\pgfpathlineto{\pgfqpoint{2.325022in}{0.696944in}}%
\pgfpathlineto{\pgfqpoint{2.296084in}{0.696005in}}%
\pgfpathlineto{\pgfqpoint{2.292928in}{0.702917in}}%
\pgfpathlineto{\pgfqpoint{2.285347in}{0.707925in}}%
\pgfpathlineto{\pgfqpoint{2.278789in}{0.700272in}}%
\pgfpathlineto{\pgfqpoint{2.271317in}{0.699125in}}%
\pgfpathlineto{\pgfqpoint{2.255860in}{0.698248in}}%
\pgfpathlineto{\pgfqpoint{2.254758in}{0.727139in}}%
\pgfpathlineto{\pgfqpoint{2.251222in}{0.732248in}}%
\pgfpathlineto{\pgfqpoint{2.244244in}{0.734424in}}%
\pgfpathlineto{\pgfqpoint{2.248864in}{0.736251in}}%
\pgfpathlineto{\pgfqpoint{2.261771in}{0.748472in}}%
\pgfpathlineto{\pgfqpoint{2.256019in}{0.754662in}}%
\pgfpathlineto{\pgfqpoint{2.248851in}{0.756615in}}%
\pgfpathlineto{\pgfqpoint{2.244140in}{0.763307in}}%
\pgfpathlineto{\pgfqpoint{2.268095in}{0.763939in}}%
\pgfpathlineto{\pgfqpoint{2.267964in}{0.770915in}}%
\pgfpathlineto{\pgfqpoint{2.285332in}{0.771355in}}%
\pgfpathlineto{\pgfqpoint{2.286412in}{0.757594in}}%
\pgfpathlineto{\pgfqpoint{2.291967in}{0.750029in}}%
\pgfpathlineto{\pgfqpoint{2.299838in}{0.745415in}}%
\pgfpathlineto{\pgfqpoint{2.300756in}{0.740640in}}%
\pgfpathlineto{\pgfqpoint{2.311015in}{0.733955in}}%
\pgfpathlineto{\pgfqpoint{2.322557in}{0.730252in}}%
\pgfpathclose%
\pgfusepath{fill}%
\end{pgfscope}%
\begin{pgfscope}%
\pgfpathrectangle{\pgfqpoint{0.100000in}{0.100000in}}{\pgfqpoint{3.608454in}{2.310000in}}%
\pgfusepath{clip}%
\pgfsetbuttcap%
\pgfsetmiterjoin%
\definecolor{currentfill}{rgb}{0.000000,0.623529,0.688235}%
\pgfsetfillcolor{currentfill}%
\pgfsetlinewidth{0.000000pt}%
\definecolor{currentstroke}{rgb}{0.000000,0.000000,0.000000}%
\pgfsetstrokecolor{currentstroke}%
\pgfsetstrokeopacity{0.000000}%
\pgfsetdash{}{0pt}%
\pgfpathmoveto{\pgfqpoint{2.537345in}{1.268351in}}%
\pgfpathlineto{\pgfqpoint{2.534479in}{1.269453in}}%
\pgfpathlineto{\pgfqpoint{2.519677in}{1.268419in}}%
\pgfpathlineto{\pgfqpoint{2.517928in}{1.295771in}}%
\pgfpathlineto{\pgfqpoint{2.497373in}{1.294428in}}%
\pgfpathlineto{\pgfqpoint{2.495702in}{1.322010in}}%
\pgfpathlineto{\pgfqpoint{2.523132in}{1.323450in}}%
\pgfpathlineto{\pgfqpoint{2.521612in}{1.327886in}}%
\pgfpathlineto{\pgfqpoint{2.521388in}{1.342978in}}%
\pgfpathlineto{\pgfqpoint{2.542895in}{1.344839in}}%
\pgfpathlineto{\pgfqpoint{2.544363in}{1.322774in}}%
\pgfpathlineto{\pgfqpoint{2.542054in}{1.316771in}}%
\pgfpathlineto{\pgfqpoint{2.541523in}{1.307274in}}%
\pgfpathlineto{\pgfqpoint{2.543804in}{1.301922in}}%
\pgfpathlineto{\pgfqpoint{2.541416in}{1.297901in}}%
\pgfpathlineto{\pgfqpoint{2.545025in}{1.287628in}}%
\pgfpathlineto{\pgfqpoint{2.537345in}{1.268351in}}%
\pgfpathclose%
\pgfusepath{fill}%
\end{pgfscope}%
\begin{pgfscope}%
\pgfpathrectangle{\pgfqpoint{0.100000in}{0.100000in}}{\pgfqpoint{3.608454in}{2.310000in}}%
\pgfusepath{clip}%
\pgfsetbuttcap%
\pgfsetmiterjoin%
\definecolor{currentfill}{rgb}{0.000000,0.741176,0.629412}%
\pgfsetfillcolor{currentfill}%
\pgfsetlinewidth{0.000000pt}%
\definecolor{currentstroke}{rgb}{0.000000,0.000000,0.000000}%
\pgfsetstrokecolor{currentstroke}%
\pgfsetstrokeopacity{0.000000}%
\pgfsetdash{}{0pt}%
\pgfpathmoveto{\pgfqpoint{2.920044in}{1.373705in}}%
\pgfpathlineto{\pgfqpoint{2.909141in}{1.370242in}}%
\pgfpathlineto{\pgfqpoint{2.903048in}{1.365422in}}%
\pgfpathlineto{\pgfqpoint{2.905122in}{1.355064in}}%
\pgfpathlineto{\pgfqpoint{2.893788in}{1.362802in}}%
\pgfpathlineto{\pgfqpoint{2.885129in}{1.361064in}}%
\pgfpathlineto{\pgfqpoint{2.884573in}{1.368240in}}%
\pgfpathlineto{\pgfqpoint{2.877293in}{1.367722in}}%
\pgfpathlineto{\pgfqpoint{2.875638in}{1.374674in}}%
\pgfpathlineto{\pgfqpoint{2.869516in}{1.374371in}}%
\pgfpathlineto{\pgfqpoint{2.869245in}{1.379552in}}%
\pgfpathlineto{\pgfqpoint{2.876585in}{1.380575in}}%
\pgfpathlineto{\pgfqpoint{2.875723in}{1.395664in}}%
\pgfpathlineto{\pgfqpoint{2.882640in}{1.396033in}}%
\pgfpathlineto{\pgfqpoint{2.881668in}{1.410091in}}%
\pgfpathlineto{\pgfqpoint{2.883523in}{1.417325in}}%
\pgfpathlineto{\pgfqpoint{2.880748in}{1.424222in}}%
\pgfpathlineto{\pgfqpoint{2.880330in}{1.431219in}}%
\pgfpathlineto{\pgfqpoint{2.887216in}{1.431682in}}%
\pgfpathlineto{\pgfqpoint{2.886708in}{1.438806in}}%
\pgfpathlineto{\pgfqpoint{2.893522in}{1.439324in}}%
\pgfpathlineto{\pgfqpoint{2.894009in}{1.432157in}}%
\pgfpathlineto{\pgfqpoint{2.906878in}{1.433220in}}%
\pgfpathlineto{\pgfqpoint{2.907865in}{1.422811in}}%
\pgfpathlineto{\pgfqpoint{2.916394in}{1.420191in}}%
\pgfpathlineto{\pgfqpoint{2.915344in}{1.412906in}}%
\pgfpathlineto{\pgfqpoint{2.916733in}{1.399937in}}%
\pgfpathlineto{\pgfqpoint{2.910035in}{1.386821in}}%
\pgfpathlineto{\pgfqpoint{2.920044in}{1.373705in}}%
\pgfpathclose%
\pgfusepath{fill}%
\end{pgfscope}%
\begin{pgfscope}%
\pgfpathrectangle{\pgfqpoint{0.100000in}{0.100000in}}{\pgfqpoint{3.608454in}{2.310000in}}%
\pgfusepath{clip}%
\pgfsetbuttcap%
\pgfsetmiterjoin%
\definecolor{currentfill}{rgb}{0.000000,0.443137,0.778431}%
\pgfsetfillcolor{currentfill}%
\pgfsetlinewidth{0.000000pt}%
\definecolor{currentstroke}{rgb}{0.000000,0.000000,0.000000}%
\pgfsetstrokecolor{currentstroke}%
\pgfsetstrokeopacity{0.000000}%
\pgfsetdash{}{0pt}%
\pgfpathmoveto{\pgfqpoint{1.893270in}{1.388628in}}%
\pgfpathlineto{\pgfqpoint{1.928379in}{1.387333in}}%
\pgfpathlineto{\pgfqpoint{1.927666in}{1.366756in}}%
\pgfpathlineto{\pgfqpoint{1.927400in}{1.359864in}}%
\pgfpathlineto{\pgfqpoint{1.893147in}{1.361110in}}%
\pgfpathlineto{\pgfqpoint{1.892889in}{1.354245in}}%
\pgfpathlineto{\pgfqpoint{1.858721in}{1.355685in}}%
\pgfpathlineto{\pgfqpoint{1.860470in}{1.390012in}}%
\pgfpathlineto{\pgfqpoint{1.893270in}{1.388628in}}%
\pgfpathclose%
\pgfusepath{fill}%
\end{pgfscope}%
\begin{pgfscope}%
\pgfpathrectangle{\pgfqpoint{0.100000in}{0.100000in}}{\pgfqpoint{3.608454in}{2.310000in}}%
\pgfusepath{clip}%
\pgfsetbuttcap%
\pgfsetmiterjoin%
\definecolor{currentfill}{rgb}{0.000000,0.341176,0.829412}%
\pgfsetfillcolor{currentfill}%
\pgfsetlinewidth{0.000000pt}%
\definecolor{currentstroke}{rgb}{0.000000,0.000000,0.000000}%
\pgfsetstrokecolor{currentstroke}%
\pgfsetstrokeopacity{0.000000}%
\pgfsetdash{}{0pt}%
\pgfpathmoveto{\pgfqpoint{3.214106in}{1.265774in}}%
\pgfpathlineto{\pgfqpoint{3.210178in}{1.266083in}}%
\pgfpathlineto{\pgfqpoint{3.209973in}{1.277484in}}%
\pgfpathlineto{\pgfqpoint{3.202830in}{1.275174in}}%
\pgfpathlineto{\pgfqpoint{3.197116in}{1.277528in}}%
\pgfpathlineto{\pgfqpoint{3.189244in}{1.282467in}}%
\pgfpathlineto{\pgfqpoint{3.181322in}{1.282092in}}%
\pgfpathlineto{\pgfqpoint{3.178373in}{1.294019in}}%
\pgfpathlineto{\pgfqpoint{3.184110in}{1.299032in}}%
\pgfpathlineto{\pgfqpoint{3.155665in}{1.301994in}}%
\pgfpathlineto{\pgfqpoint{3.154386in}{1.307543in}}%
\pgfpathlineto{\pgfqpoint{3.159486in}{1.315467in}}%
\pgfpathlineto{\pgfqpoint{3.159782in}{1.331466in}}%
\pgfpathlineto{\pgfqpoint{3.153524in}{1.337961in}}%
\pgfpathlineto{\pgfqpoint{3.157131in}{1.351262in}}%
\pgfpathlineto{\pgfqpoint{3.165049in}{1.347998in}}%
\pgfpathlineto{\pgfqpoint{3.169983in}{1.342070in}}%
\pgfpathlineto{\pgfqpoint{3.176043in}{1.340697in}}%
\pgfpathlineto{\pgfqpoint{3.178627in}{1.363629in}}%
\pgfpathlineto{\pgfqpoint{3.195177in}{1.357724in}}%
\pgfpathlineto{\pgfqpoint{3.197288in}{1.352635in}}%
\pgfpathlineto{\pgfqpoint{3.202566in}{1.350706in}}%
\pgfpathlineto{\pgfqpoint{3.210448in}{1.365277in}}%
\pgfpathlineto{\pgfqpoint{3.217018in}{1.367687in}}%
\pgfpathlineto{\pgfqpoint{3.222015in}{1.359018in}}%
\pgfpathlineto{\pgfqpoint{3.228182in}{1.355944in}}%
\pgfpathlineto{\pgfqpoint{3.236077in}{1.357801in}}%
\pgfpathlineto{\pgfqpoint{3.240232in}{1.353321in}}%
\pgfpathlineto{\pgfqpoint{3.248164in}{1.344820in}}%
\pgfpathlineto{\pgfqpoint{3.249638in}{1.333200in}}%
\pgfpathlineto{\pgfqpoint{3.236059in}{1.328225in}}%
\pgfpathlineto{\pgfqpoint{3.240056in}{1.312499in}}%
\pgfpathlineto{\pgfqpoint{3.234147in}{1.313361in}}%
\pgfpathlineto{\pgfqpoint{3.224896in}{1.300661in}}%
\pgfpathlineto{\pgfqpoint{3.236491in}{1.297792in}}%
\pgfpathlineto{\pgfqpoint{3.240057in}{1.290021in}}%
\pgfpathlineto{\pgfqpoint{3.214106in}{1.265774in}}%
\pgfpathclose%
\pgfusepath{fill}%
\end{pgfscope}%
\begin{pgfscope}%
\pgfpathrectangle{\pgfqpoint{0.100000in}{0.100000in}}{\pgfqpoint{3.608454in}{2.310000in}}%
\pgfusepath{clip}%
\pgfsetbuttcap%
\pgfsetmiterjoin%
\definecolor{currentfill}{rgb}{0.000000,0.603922,0.698039}%
\pgfsetfillcolor{currentfill}%
\pgfsetlinewidth{0.000000pt}%
\definecolor{currentstroke}{rgb}{0.000000,0.000000,0.000000}%
\pgfsetstrokecolor{currentstroke}%
\pgfsetstrokeopacity{0.000000}%
\pgfsetdash{}{0pt}%
\pgfpathmoveto{\pgfqpoint{1.997875in}{0.723825in}}%
\pgfpathlineto{\pgfqpoint{1.981851in}{0.714949in}}%
\pgfpathlineto{\pgfqpoint{1.965169in}{0.705757in}}%
\pgfpathlineto{\pgfqpoint{1.951524in}{0.729506in}}%
\pgfpathlineto{\pgfqpoint{1.947012in}{0.726789in}}%
\pgfpathlineto{\pgfqpoint{1.942049in}{0.732981in}}%
\pgfpathlineto{\pgfqpoint{1.924502in}{0.764640in}}%
\pgfpathlineto{\pgfqpoint{1.919219in}{0.761725in}}%
\pgfpathlineto{\pgfqpoint{1.903937in}{0.789730in}}%
\pgfpathlineto{\pgfqpoint{1.913591in}{0.794961in}}%
\pgfpathlineto{\pgfqpoint{1.929646in}{0.804003in}}%
\pgfpathlineto{\pgfqpoint{1.934684in}{0.799256in}}%
\pgfpathlineto{\pgfqpoint{1.940012in}{0.801698in}}%
\pgfpathlineto{\pgfqpoint{1.966416in}{0.807704in}}%
\pgfpathlineto{\pgfqpoint{1.975845in}{0.790506in}}%
\pgfpathlineto{\pgfqpoint{1.978879in}{0.792186in}}%
\pgfpathlineto{\pgfqpoint{1.990434in}{0.771318in}}%
\pgfpathlineto{\pgfqpoint{1.975832in}{0.763164in}}%
\pgfpathlineto{\pgfqpoint{1.997875in}{0.723825in}}%
\pgfpathclose%
\pgfusepath{fill}%
\end{pgfscope}%
\begin{pgfscope}%
\pgfpathrectangle{\pgfqpoint{0.100000in}{0.100000in}}{\pgfqpoint{3.608454in}{2.310000in}}%
\pgfusepath{clip}%
\pgfsetbuttcap%
\pgfsetmiterjoin%
\definecolor{currentfill}{rgb}{0.000000,0.592157,0.703922}%
\pgfsetfillcolor{currentfill}%
\pgfsetlinewidth{0.000000pt}%
\definecolor{currentstroke}{rgb}{0.000000,0.000000,0.000000}%
\pgfsetstrokecolor{currentstroke}%
\pgfsetstrokeopacity{0.000000}%
\pgfsetdash{}{0pt}%
\pgfpathmoveto{\pgfqpoint{2.978628in}{1.415804in}}%
\pgfpathlineto{\pgfqpoint{2.973331in}{1.416066in}}%
\pgfpathlineto{\pgfqpoint{2.965224in}{1.427879in}}%
\pgfpathlineto{\pgfqpoint{2.966480in}{1.429783in}}%
\pgfpathlineto{\pgfqpoint{2.980855in}{1.444545in}}%
\pgfpathlineto{\pgfqpoint{2.987075in}{1.444310in}}%
\pgfpathlineto{\pgfqpoint{2.989302in}{1.450132in}}%
\pgfpathlineto{\pgfqpoint{2.986701in}{1.454490in}}%
\pgfpathlineto{\pgfqpoint{2.992195in}{1.461918in}}%
\pgfpathlineto{\pgfqpoint{2.989320in}{1.467586in}}%
\pgfpathlineto{\pgfqpoint{3.045971in}{1.477046in}}%
\pgfpathlineto{\pgfqpoint{3.052407in}{1.435951in}}%
\pgfpathlineto{\pgfqpoint{3.039311in}{1.439843in}}%
\pgfpathlineto{\pgfqpoint{3.032463in}{1.435406in}}%
\pgfpathlineto{\pgfqpoint{3.025784in}{1.440073in}}%
\pgfpathlineto{\pgfqpoint{3.019335in}{1.434599in}}%
\pgfpathlineto{\pgfqpoint{3.010773in}{1.432733in}}%
\pgfpathlineto{\pgfqpoint{3.008758in}{1.422029in}}%
\pgfpathlineto{\pgfqpoint{3.004458in}{1.420613in}}%
\pgfpathlineto{\pgfqpoint{2.986872in}{1.422883in}}%
\pgfpathlineto{\pgfqpoint{2.978628in}{1.415804in}}%
\pgfpathclose%
\pgfusepath{fill}%
\end{pgfscope}%
\begin{pgfscope}%
\pgfpathrectangle{\pgfqpoint{0.100000in}{0.100000in}}{\pgfqpoint{3.608454in}{2.310000in}}%
\pgfusepath{clip}%
\pgfsetbuttcap%
\pgfsetmiterjoin%
\definecolor{currentfill}{rgb}{0.000000,0.607843,0.696078}%
\pgfsetfillcolor{currentfill}%
\pgfsetlinewidth{0.000000pt}%
\definecolor{currentstroke}{rgb}{0.000000,0.000000,0.000000}%
\pgfsetstrokecolor{currentstroke}%
\pgfsetstrokeopacity{0.000000}%
\pgfsetdash{}{0pt}%
\pgfpathmoveto{\pgfqpoint{3.192630in}{1.620121in}}%
\pgfpathlineto{\pgfqpoint{3.195360in}{1.613257in}}%
\pgfpathlineto{\pgfqpoint{3.203466in}{1.605423in}}%
\pgfpathlineto{\pgfqpoint{3.206320in}{1.596141in}}%
\pgfpathlineto{\pgfqpoint{3.211501in}{1.598059in}}%
\pgfpathlineto{\pgfqpoint{3.213947in}{1.593817in}}%
\pgfpathlineto{\pgfqpoint{3.196959in}{1.580835in}}%
\pgfpathlineto{\pgfqpoint{3.184525in}{1.574369in}}%
\pgfpathlineto{\pgfqpoint{3.176843in}{1.578501in}}%
\pgfpathlineto{\pgfqpoint{3.157897in}{1.576523in}}%
\pgfpathlineto{\pgfqpoint{3.155139in}{1.587716in}}%
\pgfpathlineto{\pgfqpoint{3.165006in}{1.605498in}}%
\pgfpathlineto{\pgfqpoint{3.173979in}{1.610853in}}%
\pgfpathlineto{\pgfqpoint{3.174106in}{1.615801in}}%
\pgfpathlineto{\pgfqpoint{3.183685in}{1.619914in}}%
\pgfpathlineto{\pgfqpoint{3.192630in}{1.620121in}}%
\pgfpathclose%
\pgfusepath{fill}%
\end{pgfscope}%
\begin{pgfscope}%
\pgfpathrectangle{\pgfqpoint{0.100000in}{0.100000in}}{\pgfqpoint{3.608454in}{2.310000in}}%
\pgfusepath{clip}%
\pgfsetbuttcap%
\pgfsetmiterjoin%
\definecolor{currentfill}{rgb}{0.000000,0.784314,0.607843}%
\pgfsetfillcolor{currentfill}%
\pgfsetlinewidth{0.000000pt}%
\definecolor{currentstroke}{rgb}{0.000000,0.000000,0.000000}%
\pgfsetstrokecolor{currentstroke}%
\pgfsetstrokeopacity{0.000000}%
\pgfsetdash{}{0pt}%
\pgfpathmoveto{\pgfqpoint{2.549469in}{0.776821in}}%
\pgfpathlineto{\pgfqpoint{2.547754in}{0.792351in}}%
\pgfpathlineto{\pgfqpoint{2.549490in}{0.847081in}}%
\pgfpathlineto{\pgfqpoint{2.550184in}{0.875112in}}%
\pgfpathlineto{\pgfqpoint{2.550318in}{0.880073in}}%
\pgfpathlineto{\pgfqpoint{2.561497in}{0.881275in}}%
\pgfpathlineto{\pgfqpoint{2.564922in}{0.885225in}}%
\pgfpathlineto{\pgfqpoint{2.570549in}{0.885685in}}%
\pgfpathlineto{\pgfqpoint{2.575754in}{0.891315in}}%
\pgfpathlineto{\pgfqpoint{2.582729in}{0.895488in}}%
\pgfpathlineto{\pgfqpoint{2.583933in}{0.884712in}}%
\pgfpathlineto{\pgfqpoint{2.591982in}{0.885391in}}%
\pgfpathlineto{\pgfqpoint{2.591160in}{0.879866in}}%
\pgfpathlineto{\pgfqpoint{2.586494in}{0.875360in}}%
\pgfpathlineto{\pgfqpoint{2.586111in}{0.864841in}}%
\pgfpathlineto{\pgfqpoint{2.589627in}{0.852306in}}%
\pgfpathlineto{\pgfqpoint{2.588335in}{0.845764in}}%
\pgfpathlineto{\pgfqpoint{2.594215in}{0.842750in}}%
\pgfpathlineto{\pgfqpoint{2.607870in}{0.843949in}}%
\pgfpathlineto{\pgfqpoint{2.609026in}{0.830093in}}%
\pgfpathlineto{\pgfqpoint{2.612796in}{0.827022in}}%
\pgfpathlineto{\pgfqpoint{2.609428in}{0.825937in}}%
\pgfpathlineto{\pgfqpoint{2.610325in}{0.816266in}}%
\pgfpathlineto{\pgfqpoint{2.603471in}{0.815667in}}%
\pgfpathlineto{\pgfqpoint{2.604373in}{0.805606in}}%
\pgfpathlineto{\pgfqpoint{2.601434in}{0.804239in}}%
\pgfpathlineto{\pgfqpoint{2.574126in}{0.801999in}}%
\pgfpathlineto{\pgfqpoint{2.567924in}{0.787803in}}%
\pgfpathlineto{\pgfqpoint{2.573394in}{0.783619in}}%
\pgfpathlineto{\pgfqpoint{2.574889in}{0.778819in}}%
\pgfpathlineto{\pgfqpoint{2.549469in}{0.776821in}}%
\pgfpathclose%
\pgfusepath{fill}%
\end{pgfscope}%
\begin{pgfscope}%
\pgfpathrectangle{\pgfqpoint{0.100000in}{0.100000in}}{\pgfqpoint{3.608454in}{2.310000in}}%
\pgfusepath{clip}%
\pgfsetbuttcap%
\pgfsetmiterjoin%
\definecolor{currentfill}{rgb}{0.000000,0.800000,0.600000}%
\pgfsetfillcolor{currentfill}%
\pgfsetlinewidth{0.000000pt}%
\definecolor{currentstroke}{rgb}{0.000000,0.000000,0.000000}%
\pgfsetstrokecolor{currentstroke}%
\pgfsetstrokeopacity{0.000000}%
\pgfsetdash{}{0pt}%
\pgfpathmoveto{\pgfqpoint{2.361668in}{0.958818in}}%
\pgfpathlineto{\pgfqpoint{2.356734in}{0.953261in}}%
\pgfpathlineto{\pgfqpoint{2.361253in}{0.945877in}}%
\pgfpathlineto{\pgfqpoint{2.349626in}{0.945687in}}%
\pgfpathlineto{\pgfqpoint{2.341075in}{0.950676in}}%
\pgfpathlineto{\pgfqpoint{2.339665in}{0.959110in}}%
\pgfpathlineto{\pgfqpoint{2.335508in}{0.959654in}}%
\pgfpathlineto{\pgfqpoint{2.333604in}{0.966070in}}%
\pgfpathlineto{\pgfqpoint{2.322365in}{0.966382in}}%
\pgfpathlineto{\pgfqpoint{2.322083in}{0.986111in}}%
\pgfpathlineto{\pgfqpoint{2.326278in}{0.991006in}}%
\pgfpathlineto{\pgfqpoint{2.326016in}{0.997543in}}%
\pgfpathlineto{\pgfqpoint{2.321823in}{1.001132in}}%
\pgfpathlineto{\pgfqpoint{2.321574in}{1.020760in}}%
\pgfpathlineto{\pgfqpoint{2.314414in}{1.020654in}}%
\pgfpathlineto{\pgfqpoint{2.314200in}{1.029274in}}%
\pgfpathlineto{\pgfqpoint{2.320051in}{1.032034in}}%
\pgfpathlineto{\pgfqpoint{2.328311in}{1.029359in}}%
\pgfpathlineto{\pgfqpoint{2.328181in}{1.034618in}}%
\pgfpathlineto{\pgfqpoint{2.335892in}{1.034757in}}%
\pgfpathlineto{\pgfqpoint{2.342574in}{1.032044in}}%
\pgfpathlineto{\pgfqpoint{2.347372in}{1.026940in}}%
\pgfpathlineto{\pgfqpoint{2.356606in}{1.028702in}}%
\pgfpathlineto{\pgfqpoint{2.356746in}{1.021522in}}%
\pgfpathlineto{\pgfqpoint{2.360312in}{1.018136in}}%
\pgfpathlineto{\pgfqpoint{2.360588in}{1.007650in}}%
\pgfpathlineto{\pgfqpoint{2.364049in}{1.007736in}}%
\pgfpathlineto{\pgfqpoint{2.364599in}{0.986870in}}%
\pgfpathlineto{\pgfqpoint{2.368151in}{0.986916in}}%
\pgfpathlineto{\pgfqpoint{2.368481in}{0.976354in}}%
\pgfpathlineto{\pgfqpoint{2.364963in}{0.976228in}}%
\pgfpathlineto{\pgfqpoint{2.363796in}{0.962251in}}%
\pgfpathlineto{\pgfqpoint{2.361668in}{0.958818in}}%
\pgfpathclose%
\pgfusepath{fill}%
\end{pgfscope}%
\begin{pgfscope}%
\pgfpathrectangle{\pgfqpoint{0.100000in}{0.100000in}}{\pgfqpoint{3.608454in}{2.310000in}}%
\pgfusepath{clip}%
\pgfsetbuttcap%
\pgfsetmiterjoin%
\definecolor{currentfill}{rgb}{0.000000,0.403922,0.798039}%
\pgfsetfillcolor{currentfill}%
\pgfsetlinewidth{0.000000pt}%
\definecolor{currentstroke}{rgb}{0.000000,0.000000,0.000000}%
\pgfsetstrokecolor{currentstroke}%
\pgfsetstrokeopacity{0.000000}%
\pgfsetdash{}{0pt}%
\pgfpathmoveto{\pgfqpoint{1.670049in}{1.298787in}}%
\pgfpathlineto{\pgfqpoint{1.626877in}{1.302163in}}%
\pgfpathlineto{\pgfqpoint{1.579920in}{1.306214in}}%
\pgfpathlineto{\pgfqpoint{1.581597in}{1.326016in}}%
\pgfpathlineto{\pgfqpoint{1.602094in}{1.324825in}}%
\pgfpathlineto{\pgfqpoint{1.603799in}{1.337764in}}%
\pgfpathlineto{\pgfqpoint{1.606968in}{1.372189in}}%
\pgfpathlineto{\pgfqpoint{1.610603in}{1.406638in}}%
\pgfpathlineto{\pgfqpoint{1.677720in}{1.401672in}}%
\pgfpathlineto{\pgfqpoint{1.677687in}{1.401208in}}%
\pgfpathlineto{\pgfqpoint{1.672634in}{1.332579in}}%
\pgfpathlineto{\pgfqpoint{1.670049in}{1.298787in}}%
\pgfpathclose%
\pgfusepath{fill}%
\end{pgfscope}%
\begin{pgfscope}%
\pgfpathrectangle{\pgfqpoint{0.100000in}{0.100000in}}{\pgfqpoint{3.608454in}{2.310000in}}%
\pgfusepath{clip}%
\pgfsetbuttcap%
\pgfsetmiterjoin%
\definecolor{currentfill}{rgb}{0.000000,0.760784,0.619608}%
\pgfsetfillcolor{currentfill}%
\pgfsetlinewidth{0.000000pt}%
\definecolor{currentstroke}{rgb}{0.000000,0.000000,0.000000}%
\pgfsetstrokecolor{currentstroke}%
\pgfsetstrokeopacity{0.000000}%
\pgfsetdash{}{0pt}%
\pgfpathmoveto{\pgfqpoint{2.396088in}{0.992607in}}%
\pgfpathlineto{\pgfqpoint{2.395789in}{0.983518in}}%
\pgfpathlineto{\pgfqpoint{2.384480in}{0.979541in}}%
\pgfpathlineto{\pgfqpoint{2.384214in}{0.972642in}}%
\pgfpathlineto{\pgfqpoint{2.372289in}{0.959362in}}%
\pgfpathlineto{\pgfqpoint{2.372346in}{0.959262in}}%
\pgfpathlineto{\pgfqpoint{2.361668in}{0.958818in}}%
\pgfpathlineto{\pgfqpoint{2.363796in}{0.962251in}}%
\pgfpathlineto{\pgfqpoint{2.364963in}{0.976228in}}%
\pgfpathlineto{\pgfqpoint{2.368481in}{0.976354in}}%
\pgfpathlineto{\pgfqpoint{2.368151in}{0.986916in}}%
\pgfpathlineto{\pgfqpoint{2.364599in}{0.986870in}}%
\pgfpathlineto{\pgfqpoint{2.364049in}{1.007736in}}%
\pgfpathlineto{\pgfqpoint{2.360588in}{1.007650in}}%
\pgfpathlineto{\pgfqpoint{2.360312in}{1.018136in}}%
\pgfpathlineto{\pgfqpoint{2.356746in}{1.021522in}}%
\pgfpathlineto{\pgfqpoint{2.356606in}{1.028702in}}%
\pgfpathlineto{\pgfqpoint{2.363393in}{1.028869in}}%
\pgfpathlineto{\pgfqpoint{2.362249in}{1.063873in}}%
\pgfpathlineto{\pgfqpoint{2.389852in}{1.065601in}}%
\pgfpathlineto{\pgfqpoint{2.410748in}{1.065840in}}%
\pgfpathlineto{\pgfqpoint{2.420217in}{1.066185in}}%
\pgfpathlineto{\pgfqpoint{2.424647in}{1.062261in}}%
\pgfpathlineto{\pgfqpoint{2.420989in}{1.051604in}}%
\pgfpathlineto{\pgfqpoint{2.425407in}{1.049147in}}%
\pgfpathlineto{\pgfqpoint{2.424139in}{1.041016in}}%
\pgfpathlineto{\pgfqpoint{2.419408in}{1.040872in}}%
\pgfpathlineto{\pgfqpoint{2.417154in}{1.033431in}}%
\pgfpathlineto{\pgfqpoint{2.412041in}{1.029240in}}%
\pgfpathlineto{\pgfqpoint{2.415331in}{1.023934in}}%
\pgfpathlineto{\pgfqpoint{2.404445in}{1.017531in}}%
\pgfpathlineto{\pgfqpoint{2.401872in}{1.004184in}}%
\pgfpathlineto{\pgfqpoint{2.394572in}{0.999746in}}%
\pgfpathlineto{\pgfqpoint{2.396088in}{0.992607in}}%
\pgfpathclose%
\pgfusepath{fill}%
\end{pgfscope}%
\begin{pgfscope}%
\pgfpathrectangle{\pgfqpoint{0.100000in}{0.100000in}}{\pgfqpoint{3.608454in}{2.310000in}}%
\pgfusepath{clip}%
\pgfsetbuttcap%
\pgfsetmiterjoin%
\definecolor{currentfill}{rgb}{0.000000,0.470588,0.764706}%
\pgfsetfillcolor{currentfill}%
\pgfsetlinewidth{0.000000pt}%
\definecolor{currentstroke}{rgb}{0.000000,0.000000,0.000000}%
\pgfsetstrokecolor{currentstroke}%
\pgfsetstrokeopacity{0.000000}%
\pgfsetdash{}{0pt}%
\pgfpathmoveto{\pgfqpoint{2.720376in}{1.354526in}}%
\pgfpathlineto{\pgfqpoint{2.710112in}{1.347801in}}%
\pgfpathlineto{\pgfqpoint{2.708117in}{1.365276in}}%
\pgfpathlineto{\pgfqpoint{2.693342in}{1.363630in}}%
\pgfpathlineto{\pgfqpoint{2.691293in}{1.385879in}}%
\pgfpathlineto{\pgfqpoint{2.684115in}{1.380130in}}%
\pgfpathlineto{\pgfqpoint{2.676781in}{1.379202in}}%
\pgfpathlineto{\pgfqpoint{2.675138in}{1.396451in}}%
\pgfpathlineto{\pgfqpoint{2.677647in}{1.404907in}}%
\pgfpathlineto{\pgfqpoint{2.697801in}{1.407041in}}%
\pgfpathlineto{\pgfqpoint{2.697158in}{1.412713in}}%
\pgfpathlineto{\pgfqpoint{2.726448in}{1.415659in}}%
\pgfpathlineto{\pgfqpoint{2.729836in}{1.382914in}}%
\pgfpathlineto{\pgfqpoint{2.725607in}{1.378219in}}%
\pgfpathlineto{\pgfqpoint{2.729795in}{1.373635in}}%
\pgfpathlineto{\pgfqpoint{2.727957in}{1.367391in}}%
\pgfpathlineto{\pgfqpoint{2.733971in}{1.365444in}}%
\pgfpathlineto{\pgfqpoint{2.728448in}{1.358041in}}%
\pgfpathlineto{\pgfqpoint{2.720376in}{1.354526in}}%
\pgfpathclose%
\pgfusepath{fill}%
\end{pgfscope}%
\begin{pgfscope}%
\pgfpathrectangle{\pgfqpoint{0.100000in}{0.100000in}}{\pgfqpoint{3.608454in}{2.310000in}}%
\pgfusepath{clip}%
\pgfsetbuttcap%
\pgfsetmiterjoin%
\definecolor{currentfill}{rgb}{0.000000,0.670588,0.664706}%
\pgfsetfillcolor{currentfill}%
\pgfsetlinewidth{0.000000pt}%
\definecolor{currentstroke}{rgb}{0.000000,0.000000,0.000000}%
\pgfsetstrokecolor{currentstroke}%
\pgfsetstrokeopacity{0.000000}%
\pgfsetdash{}{0pt}%
\pgfpathmoveto{\pgfqpoint{1.708835in}{0.943953in}}%
\pgfpathlineto{\pgfqpoint{1.743217in}{0.941925in}}%
\pgfpathlineto{\pgfqpoint{1.774163in}{0.940149in}}%
\pgfpathlineto{\pgfqpoint{1.797246in}{0.938094in}}%
\pgfpathlineto{\pgfqpoint{1.810392in}{0.932854in}}%
\pgfpathlineto{\pgfqpoint{1.811548in}{0.930044in}}%
\pgfpathlineto{\pgfqpoint{1.810377in}{0.903512in}}%
\pgfpathlineto{\pgfqpoint{1.808729in}{0.868432in}}%
\pgfpathlineto{\pgfqpoint{1.763926in}{0.871040in}}%
\pgfpathlineto{\pgfqpoint{1.738977in}{0.872740in}}%
\pgfpathlineto{\pgfqpoint{1.741190in}{0.907227in}}%
\pgfpathlineto{\pgfqpoint{1.706689in}{0.909371in}}%
\pgfpathlineto{\pgfqpoint{1.708835in}{0.943953in}}%
\pgfpathclose%
\pgfusepath{fill}%
\end{pgfscope}%
\begin{pgfscope}%
\pgfpathrectangle{\pgfqpoint{0.100000in}{0.100000in}}{\pgfqpoint{3.608454in}{2.310000in}}%
\pgfusepath{clip}%
\pgfsetbuttcap%
\pgfsetmiterjoin%
\definecolor{currentfill}{rgb}{0.000000,0.835294,0.582353}%
\pgfsetfillcolor{currentfill}%
\pgfsetlinewidth{0.000000pt}%
\definecolor{currentstroke}{rgb}{0.000000,0.000000,0.000000}%
\pgfsetstrokecolor{currentstroke}%
\pgfsetstrokeopacity{0.000000}%
\pgfsetdash{}{0pt}%
\pgfpathmoveto{\pgfqpoint{2.336789in}{0.709454in}}%
\pgfpathlineto{\pgfqpoint{2.330722in}{0.712813in}}%
\pgfpathlineto{\pgfqpoint{2.330790in}{0.720030in}}%
\pgfpathlineto{\pgfqpoint{2.333480in}{0.724393in}}%
\pgfpathlineto{\pgfqpoint{2.322557in}{0.730252in}}%
\pgfpathlineto{\pgfqpoint{2.319974in}{0.746106in}}%
\pgfpathlineto{\pgfqpoint{2.325362in}{0.751678in}}%
\pgfpathlineto{\pgfqpoint{2.330853in}{0.768113in}}%
\pgfpathlineto{\pgfqpoint{2.334485in}{0.767523in}}%
\pgfpathlineto{\pgfqpoint{2.338396in}{0.773253in}}%
\pgfpathlineto{\pgfqpoint{2.338322in}{0.779971in}}%
\pgfpathlineto{\pgfqpoint{2.343388in}{0.795089in}}%
\pgfpathlineto{\pgfqpoint{2.342946in}{0.805709in}}%
\pgfpathlineto{\pgfqpoint{2.354522in}{0.806095in}}%
\pgfpathlineto{\pgfqpoint{2.366944in}{0.810346in}}%
\pgfpathlineto{\pgfqpoint{2.365200in}{0.801106in}}%
\pgfpathlineto{\pgfqpoint{2.374090in}{0.798791in}}%
\pgfpathlineto{\pgfqpoint{2.371635in}{0.792020in}}%
\pgfpathlineto{\pgfqpoint{2.364252in}{0.787285in}}%
\pgfpathlineto{\pgfqpoint{2.363437in}{0.782994in}}%
\pgfpathlineto{\pgfqpoint{2.353993in}{0.777417in}}%
\pgfpathlineto{\pgfqpoint{2.357930in}{0.763773in}}%
\pgfpathlineto{\pgfqpoint{2.367747in}{0.759421in}}%
\pgfpathlineto{\pgfqpoint{2.395986in}{0.760766in}}%
\pgfpathlineto{\pgfqpoint{2.403042in}{0.761124in}}%
\pgfpathlineto{\pgfqpoint{2.404037in}{0.740243in}}%
\pgfpathlineto{\pgfqpoint{2.377876in}{0.738907in}}%
\pgfpathlineto{\pgfqpoint{2.371609in}{0.736436in}}%
\pgfpathlineto{\pgfqpoint{2.362838in}{0.740120in}}%
\pgfpathlineto{\pgfqpoint{2.347315in}{0.735413in}}%
\pgfpathlineto{\pgfqpoint{2.346226in}{0.727949in}}%
\pgfpathlineto{\pgfqpoint{2.341299in}{0.728481in}}%
\pgfpathlineto{\pgfqpoint{2.337261in}{0.720197in}}%
\pgfpathlineto{\pgfqpoint{2.341562in}{0.715016in}}%
\pgfpathlineto{\pgfqpoint{2.336789in}{0.709454in}}%
\pgfpathclose%
\pgfusepath{fill}%
\end{pgfscope}%
\begin{pgfscope}%
\pgfpathrectangle{\pgfqpoint{0.100000in}{0.100000in}}{\pgfqpoint{3.608454in}{2.310000in}}%
\pgfusepath{clip}%
\pgfsetbuttcap%
\pgfsetmiterjoin%
\definecolor{currentfill}{rgb}{0.000000,0.533333,0.733333}%
\pgfsetfillcolor{currentfill}%
\pgfsetlinewidth{0.000000pt}%
\definecolor{currentstroke}{rgb}{0.000000,0.000000,0.000000}%
\pgfsetstrokecolor{currentstroke}%
\pgfsetstrokeopacity{0.000000}%
\pgfsetdash{}{0pt}%
\pgfpathmoveto{\pgfqpoint{2.368642in}{1.363523in}}%
\pgfpathlineto{\pgfqpoint{2.340427in}{1.362513in}}%
\pgfpathlineto{\pgfqpoint{2.340562in}{1.355530in}}%
\pgfpathlineto{\pgfqpoint{2.335984in}{1.355415in}}%
\pgfpathlineto{\pgfqpoint{2.326861in}{1.355261in}}%
\pgfpathlineto{\pgfqpoint{2.324527in}{1.369104in}}%
\pgfpathlineto{\pgfqpoint{2.307376in}{1.370072in}}%
\pgfpathlineto{\pgfqpoint{2.306650in}{1.395239in}}%
\pgfpathlineto{\pgfqpoint{2.298632in}{1.395025in}}%
\pgfpathlineto{\pgfqpoint{2.298225in}{1.417924in}}%
\pgfpathlineto{\pgfqpoint{2.291417in}{1.417786in}}%
\pgfpathlineto{\pgfqpoint{2.290718in}{1.442111in}}%
\pgfpathlineto{\pgfqpoint{2.318087in}{1.442426in}}%
\pgfpathlineto{\pgfqpoint{2.317746in}{1.438593in}}%
\pgfpathlineto{\pgfqpoint{2.353565in}{1.439495in}}%
\pgfpathlineto{\pgfqpoint{2.354524in}{1.412008in}}%
\pgfpathlineto{\pgfqpoint{2.375453in}{1.412534in}}%
\pgfpathlineto{\pgfqpoint{2.371479in}{1.401180in}}%
\pgfpathlineto{\pgfqpoint{2.376576in}{1.388834in}}%
\pgfpathlineto{\pgfqpoint{2.374602in}{1.377391in}}%
\pgfpathlineto{\pgfqpoint{2.354879in}{1.376769in}}%
\pgfpathlineto{\pgfqpoint{2.363949in}{1.370060in}}%
\pgfpathlineto{\pgfqpoint{2.368642in}{1.363523in}}%
\pgfpathclose%
\pgfusepath{fill}%
\end{pgfscope}%
\begin{pgfscope}%
\pgfpathrectangle{\pgfqpoint{0.100000in}{0.100000in}}{\pgfqpoint{3.608454in}{2.310000in}}%
\pgfusepath{clip}%
\pgfsetbuttcap%
\pgfsetmiterjoin%
\definecolor{currentfill}{rgb}{0.000000,0.858824,0.570588}%
\pgfsetfillcolor{currentfill}%
\pgfsetlinewidth{0.000000pt}%
\definecolor{currentstroke}{rgb}{0.000000,0.000000,0.000000}%
\pgfsetstrokecolor{currentstroke}%
\pgfsetstrokeopacity{0.000000}%
\pgfsetdash{}{0pt}%
\pgfpathmoveto{\pgfqpoint{1.007808in}{2.262053in}}%
\pgfpathlineto{\pgfqpoint{0.998620in}{2.223097in}}%
\pgfpathlineto{\pgfqpoint{0.960593in}{2.232110in}}%
\pgfpathlineto{\pgfqpoint{0.967184in}{2.259137in}}%
\pgfpathlineto{\pgfqpoint{0.954484in}{2.262189in}}%
\pgfpathlineto{\pgfqpoint{0.957430in}{2.274082in}}%
\pgfpathlineto{\pgfqpoint{1.007808in}{2.262053in}}%
\pgfpathclose%
\pgfusepath{fill}%
\end{pgfscope}%
\begin{pgfscope}%
\pgfpathrectangle{\pgfqpoint{0.100000in}{0.100000in}}{\pgfqpoint{3.608454in}{2.310000in}}%
\pgfusepath{clip}%
\pgfsetbuttcap%
\pgfsetmiterjoin%
\definecolor{currentfill}{rgb}{0.000000,0.537255,0.731373}%
\pgfsetfillcolor{currentfill}%
\pgfsetlinewidth{0.000000pt}%
\definecolor{currentstroke}{rgb}{0.000000,0.000000,0.000000}%
\pgfsetstrokecolor{currentstroke}%
\pgfsetstrokeopacity{0.000000}%
\pgfsetdash{}{0pt}%
\pgfpathmoveto{\pgfqpoint{2.799445in}{1.624219in}}%
\pgfpathlineto{\pgfqpoint{2.792334in}{1.628535in}}%
\pgfpathlineto{\pgfqpoint{2.764552in}{1.624067in}}%
\pgfpathlineto{\pgfqpoint{2.743822in}{1.620714in}}%
\pgfpathlineto{\pgfqpoint{2.739935in}{1.648352in}}%
\pgfpathlineto{\pgfqpoint{2.735608in}{1.675908in}}%
\pgfpathlineto{\pgfqpoint{2.723746in}{1.674452in}}%
\pgfpathlineto{\pgfqpoint{2.720350in}{1.701323in}}%
\pgfpathlineto{\pgfqpoint{2.720255in}{1.702142in}}%
\pgfpathlineto{\pgfqpoint{2.745404in}{1.705512in}}%
\pgfpathlineto{\pgfqpoint{2.744382in}{1.712367in}}%
\pgfpathlineto{\pgfqpoint{2.771368in}{1.715974in}}%
\pgfpathlineto{\pgfqpoint{2.777478in}{1.717721in}}%
\pgfpathlineto{\pgfqpoint{2.776445in}{1.724555in}}%
\pgfpathlineto{\pgfqpoint{2.790730in}{1.726819in}}%
\pgfpathlineto{\pgfqpoint{2.786333in}{1.754076in}}%
\pgfpathlineto{\pgfqpoint{2.815217in}{1.759142in}}%
\pgfpathlineto{\pgfqpoint{2.817633in}{1.749127in}}%
\pgfpathlineto{\pgfqpoint{2.821949in}{1.738559in}}%
\pgfpathlineto{\pgfqpoint{2.825537in}{1.722337in}}%
\pgfpathlineto{\pgfqpoint{2.828453in}{1.713989in}}%
\pgfpathlineto{\pgfqpoint{2.833465in}{1.706759in}}%
\pgfpathlineto{\pgfqpoint{2.831898in}{1.690908in}}%
\pgfpathlineto{\pgfqpoint{2.833245in}{1.687359in}}%
\pgfpathlineto{\pgfqpoint{2.831702in}{1.674798in}}%
\pgfpathlineto{\pgfqpoint{2.820874in}{1.679961in}}%
\pgfpathlineto{\pgfqpoint{2.815658in}{1.676414in}}%
\pgfpathlineto{\pgfqpoint{2.812091in}{1.663621in}}%
\pgfpathlineto{\pgfqpoint{2.813316in}{1.659807in}}%
\pgfpathlineto{\pgfqpoint{2.810815in}{1.651423in}}%
\pgfpathlineto{\pgfqpoint{2.803471in}{1.647833in}}%
\pgfpathlineto{\pgfqpoint{2.800576in}{1.641151in}}%
\pgfpathlineto{\pgfqpoint{2.801791in}{1.629277in}}%
\pgfpathlineto{\pgfqpoint{2.799445in}{1.624219in}}%
\pgfpathclose%
\pgfusepath{fill}%
\end{pgfscope}%
\begin{pgfscope}%
\pgfpathrectangle{\pgfqpoint{0.100000in}{0.100000in}}{\pgfqpoint{3.608454in}{2.310000in}}%
\pgfusepath{clip}%
\pgfsetbuttcap%
\pgfsetmiterjoin%
\definecolor{currentfill}{rgb}{0.000000,0.603922,0.698039}%
\pgfsetfillcolor{currentfill}%
\pgfsetlinewidth{0.000000pt}%
\definecolor{currentstroke}{rgb}{0.000000,0.000000,0.000000}%
\pgfsetstrokecolor{currentstroke}%
\pgfsetstrokeopacity{0.000000}%
\pgfsetdash{}{0pt}%
\pgfpathmoveto{\pgfqpoint{0.763231in}{2.326563in}}%
\pgfpathlineto{\pgfqpoint{0.765166in}{2.300496in}}%
\pgfpathlineto{\pgfqpoint{0.760555in}{2.298682in}}%
\pgfpathlineto{\pgfqpoint{0.752119in}{2.299760in}}%
\pgfpathlineto{\pgfqpoint{0.728782in}{2.306722in}}%
\pgfpathlineto{\pgfqpoint{0.672017in}{2.324335in}}%
\pgfpathlineto{\pgfqpoint{0.672629in}{2.335131in}}%
\pgfpathlineto{\pgfqpoint{0.667385in}{2.333960in}}%
\pgfpathlineto{\pgfqpoint{0.662898in}{2.347488in}}%
\pgfpathlineto{\pgfqpoint{0.667167in}{2.355917in}}%
\pgfpathlineto{\pgfqpoint{0.717719in}{2.339867in}}%
\pgfpathlineto{\pgfqpoint{0.763231in}{2.326563in}}%
\pgfpathclose%
\pgfusepath{fill}%
\end{pgfscope}%
\begin{pgfscope}%
\pgfpathrectangle{\pgfqpoint{0.100000in}{0.100000in}}{\pgfqpoint{3.608454in}{2.310000in}}%
\pgfusepath{clip}%
\pgfsetbuttcap%
\pgfsetmiterjoin%
\definecolor{currentfill}{rgb}{0.000000,0.984314,0.507843}%
\pgfsetfillcolor{currentfill}%
\pgfsetlinewidth{0.000000pt}%
\definecolor{currentstroke}{rgb}{0.000000,0.000000,0.000000}%
\pgfsetstrokecolor{currentstroke}%
\pgfsetstrokeopacity{0.000000}%
\pgfsetdash{}{0pt}%
\pgfpathmoveto{\pgfqpoint{1.223167in}{1.249876in}}%
\pgfpathlineto{\pgfqpoint{1.210980in}{1.171960in}}%
\pgfpathlineto{\pgfqpoint{1.300870in}{1.158512in}}%
\pgfpathlineto{\pgfqpoint{1.321043in}{1.155597in}}%
\pgfpathlineto{\pgfqpoint{1.313512in}{1.101203in}}%
\pgfpathlineto{\pgfqpoint{1.239395in}{1.111911in}}%
\pgfpathlineto{\pgfqpoint{1.235282in}{1.084666in}}%
\pgfpathlineto{\pgfqpoint{1.198276in}{1.090371in}}%
\pgfpathlineto{\pgfqpoint{1.193614in}{1.060581in}}%
\pgfpathlineto{\pgfqpoint{1.183780in}{0.997802in}}%
\pgfpathlineto{\pgfqpoint{1.164171in}{1.000833in}}%
\pgfpathlineto{\pgfqpoint{1.154422in}{0.995185in}}%
\pgfpathlineto{\pgfqpoint{1.150108in}{0.990389in}}%
\pgfpathlineto{\pgfqpoint{1.145330in}{0.990568in}}%
\pgfpathlineto{\pgfqpoint{1.133001in}{0.981587in}}%
\pgfpathlineto{\pgfqpoint{1.125550in}{0.990854in}}%
\pgfpathlineto{\pgfqpoint{1.119039in}{0.992055in}}%
\pgfpathlineto{\pgfqpoint{1.124557in}{1.025178in}}%
\pgfpathlineto{\pgfqpoint{1.075926in}{1.033485in}}%
\pgfpathlineto{\pgfqpoint{1.079528in}{1.054057in}}%
\pgfpathlineto{\pgfqpoint{1.116943in}{1.267777in}}%
\pgfpathlineto{\pgfqpoint{1.163571in}{1.259392in}}%
\pgfpathlineto{\pgfqpoint{1.223167in}{1.249876in}}%
\pgfpathclose%
\pgfusepath{fill}%
\end{pgfscope}%
\begin{pgfscope}%
\pgfpathrectangle{\pgfqpoint{0.100000in}{0.100000in}}{\pgfqpoint{3.608454in}{2.310000in}}%
\pgfusepath{clip}%
\pgfsetbuttcap%
\pgfsetmiterjoin%
\definecolor{currentfill}{rgb}{0.000000,0.541176,0.729412}%
\pgfsetfillcolor{currentfill}%
\pgfsetlinewidth{0.000000pt}%
\definecolor{currentstroke}{rgb}{0.000000,0.000000,0.000000}%
\pgfsetstrokecolor{currentstroke}%
\pgfsetstrokeopacity{0.000000}%
\pgfsetdash{}{0pt}%
\pgfpathmoveto{\pgfqpoint{2.269571in}{1.909501in}}%
\pgfpathlineto{\pgfqpoint{2.297058in}{1.910416in}}%
\pgfpathlineto{\pgfqpoint{2.331620in}{1.911661in}}%
\pgfpathlineto{\pgfqpoint{2.332207in}{1.897841in}}%
\pgfpathlineto{\pgfqpoint{2.345925in}{1.898515in}}%
\pgfpathlineto{\pgfqpoint{2.348110in}{1.850512in}}%
\pgfpathlineto{\pgfqpoint{2.334351in}{1.850010in}}%
\pgfpathlineto{\pgfqpoint{2.334657in}{1.843071in}}%
\pgfpathlineto{\pgfqpoint{2.300225in}{1.841636in}}%
\pgfpathlineto{\pgfqpoint{2.300433in}{1.834842in}}%
\pgfpathlineto{\pgfqpoint{2.266038in}{1.833884in}}%
\pgfpathlineto{\pgfqpoint{2.265057in}{1.868094in}}%
\pgfpathlineto{\pgfqpoint{2.271935in}{1.868329in}}%
\pgfpathlineto{\pgfqpoint{2.270918in}{1.895690in}}%
\pgfpathlineto{\pgfqpoint{2.269571in}{1.909501in}}%
\pgfpathclose%
\pgfusepath{fill}%
\end{pgfscope}%
\begin{pgfscope}%
\pgfpathrectangle{\pgfqpoint{0.100000in}{0.100000in}}{\pgfqpoint{3.608454in}{2.310000in}}%
\pgfusepath{clip}%
\pgfsetbuttcap%
\pgfsetmiterjoin%
\definecolor{currentfill}{rgb}{0.000000,0.674510,0.662745}%
\pgfsetfillcolor{currentfill}%
\pgfsetlinewidth{0.000000pt}%
\definecolor{currentstroke}{rgb}{0.000000,0.000000,0.000000}%
\pgfsetstrokecolor{currentstroke}%
\pgfsetstrokeopacity{0.000000}%
\pgfsetdash{}{0pt}%
\pgfpathmoveto{\pgfqpoint{1.724900in}{0.803670in}}%
\pgfpathlineto{\pgfqpoint{1.690000in}{0.806092in}}%
\pgfpathlineto{\pgfqpoint{1.692995in}{0.840738in}}%
\pgfpathlineto{\pgfqpoint{1.695447in}{0.875598in}}%
\pgfpathlineto{\pgfqpoint{1.704446in}{0.875520in}}%
\pgfpathlineto{\pgfqpoint{1.706689in}{0.909371in}}%
\pgfpathlineto{\pgfqpoint{1.741190in}{0.907227in}}%
\pgfpathlineto{\pgfqpoint{1.738977in}{0.872740in}}%
\pgfpathlineto{\pgfqpoint{1.729883in}{0.873343in}}%
\pgfpathlineto{\pgfqpoint{1.724900in}{0.803670in}}%
\pgfpathclose%
\pgfusepath{fill}%
\end{pgfscope}%
\begin{pgfscope}%
\pgfpathrectangle{\pgfqpoint{0.100000in}{0.100000in}}{\pgfqpoint{3.608454in}{2.310000in}}%
\pgfusepath{clip}%
\pgfsetbuttcap%
\pgfsetmiterjoin%
\definecolor{currentfill}{rgb}{0.000000,0.827451,0.586275}%
\pgfsetfillcolor{currentfill}%
\pgfsetlinewidth{0.000000pt}%
\definecolor{currentstroke}{rgb}{0.000000,0.000000,0.000000}%
\pgfsetstrokecolor{currentstroke}%
\pgfsetstrokeopacity{0.000000}%
\pgfsetdash{}{0pt}%
\pgfpathmoveto{\pgfqpoint{2.474841in}{0.898468in}}%
\pgfpathlineto{\pgfqpoint{2.483949in}{0.899006in}}%
\pgfpathlineto{\pgfqpoint{2.485523in}{0.870923in}}%
\pgfpathlineto{\pgfqpoint{2.458164in}{0.869135in}}%
\pgfpathlineto{\pgfqpoint{2.458381in}{0.865552in}}%
\pgfpathlineto{\pgfqpoint{2.442684in}{0.864247in}}%
\pgfpathlineto{\pgfqpoint{2.417323in}{0.873285in}}%
\pgfpathlineto{\pgfqpoint{2.410750in}{0.879777in}}%
\pgfpathlineto{\pgfqpoint{2.417347in}{0.883236in}}%
\pgfpathlineto{\pgfqpoint{2.416578in}{0.896534in}}%
\pgfpathlineto{\pgfqpoint{2.419871in}{0.899622in}}%
\pgfpathlineto{\pgfqpoint{2.426856in}{0.899501in}}%
\pgfpathlineto{\pgfqpoint{2.455570in}{0.891636in}}%
\pgfpathlineto{\pgfqpoint{2.458419in}{0.897453in}}%
\pgfpathlineto{\pgfqpoint{2.474841in}{0.898468in}}%
\pgfpathclose%
\pgfusepath{fill}%
\end{pgfscope}%
\begin{pgfscope}%
\pgfpathrectangle{\pgfqpoint{0.100000in}{0.100000in}}{\pgfqpoint{3.608454in}{2.310000in}}%
\pgfusepath{clip}%
\pgfsetbuttcap%
\pgfsetmiterjoin%
\definecolor{currentfill}{rgb}{0.000000,0.552941,0.723529}%
\pgfsetfillcolor{currentfill}%
\pgfsetlinewidth{0.000000pt}%
\definecolor{currentstroke}{rgb}{0.000000,0.000000,0.000000}%
\pgfsetstrokecolor{currentstroke}%
\pgfsetstrokeopacity{0.000000}%
\pgfsetdash{}{0pt}%
\pgfpathmoveto{\pgfqpoint{2.348213in}{1.508885in}}%
\pgfpathlineto{\pgfqpoint{2.323109in}{1.507938in}}%
\pgfpathlineto{\pgfqpoint{2.322856in}{1.514824in}}%
\pgfpathlineto{\pgfqpoint{2.302293in}{1.514165in}}%
\pgfpathlineto{\pgfqpoint{2.288611in}{1.513740in}}%
\pgfpathlineto{\pgfqpoint{2.287673in}{1.541193in}}%
\pgfpathlineto{\pgfqpoint{2.294578in}{1.541485in}}%
\pgfpathlineto{\pgfqpoint{2.293482in}{1.569183in}}%
\pgfpathlineto{\pgfqpoint{2.320902in}{1.570069in}}%
\pgfpathlineto{\pgfqpoint{2.320640in}{1.576968in}}%
\pgfpathlineto{\pgfqpoint{2.348123in}{1.578046in}}%
\pgfpathlineto{\pgfqpoint{2.348758in}{1.564236in}}%
\pgfpathlineto{\pgfqpoint{2.349290in}{1.550458in}}%
\pgfpathlineto{\pgfqpoint{2.356154in}{1.550723in}}%
\pgfpathlineto{\pgfqpoint{2.356534in}{1.539366in}}%
\pgfpathlineto{\pgfqpoint{2.341290in}{1.535608in}}%
\pgfpathlineto{\pgfqpoint{2.337899in}{1.521676in}}%
\pgfpathlineto{\pgfqpoint{2.345525in}{1.515337in}}%
\pgfpathlineto{\pgfqpoint{2.348213in}{1.508885in}}%
\pgfpathclose%
\pgfusepath{fill}%
\end{pgfscope}%
\begin{pgfscope}%
\pgfpathrectangle{\pgfqpoint{0.100000in}{0.100000in}}{\pgfqpoint{3.608454in}{2.310000in}}%
\pgfusepath{clip}%
\pgfsetbuttcap%
\pgfsetmiterjoin%
\definecolor{currentfill}{rgb}{0.000000,0.513725,0.743137}%
\pgfsetfillcolor{currentfill}%
\pgfsetlinewidth{0.000000pt}%
\definecolor{currentstroke}{rgb}{0.000000,0.000000,0.000000}%
\pgfsetstrokecolor{currentstroke}%
\pgfsetstrokeopacity{0.000000}%
\pgfsetdash{}{0pt}%
\pgfpathmoveto{\pgfqpoint{2.099037in}{1.876260in}}%
\pgfpathlineto{\pgfqpoint{2.099439in}{1.847533in}}%
\pgfpathlineto{\pgfqpoint{2.044431in}{1.847876in}}%
\pgfpathlineto{\pgfqpoint{2.037488in}{1.848022in}}%
\pgfpathlineto{\pgfqpoint{2.037019in}{1.861775in}}%
\pgfpathlineto{\pgfqpoint{2.037372in}{1.889582in}}%
\pgfpathlineto{\pgfqpoint{2.036803in}{1.896501in}}%
\pgfpathlineto{\pgfqpoint{2.036301in}{1.917281in}}%
\pgfpathlineto{\pgfqpoint{2.036642in}{1.945064in}}%
\pgfpathlineto{\pgfqpoint{2.042594in}{1.944974in}}%
\pgfpathlineto{\pgfqpoint{2.042559in}{1.951936in}}%
\pgfpathlineto{\pgfqpoint{2.097851in}{1.951504in}}%
\pgfpathlineto{\pgfqpoint{2.098775in}{1.916792in}}%
\pgfpathlineto{\pgfqpoint{2.099037in}{1.876260in}}%
\pgfpathclose%
\pgfusepath{fill}%
\end{pgfscope}%
\begin{pgfscope}%
\pgfpathrectangle{\pgfqpoint{0.100000in}{0.100000in}}{\pgfqpoint{3.608454in}{2.310000in}}%
\pgfusepath{clip}%
\pgfsetbuttcap%
\pgfsetmiterjoin%
\definecolor{currentfill}{rgb}{0.000000,0.549020,0.725490}%
\pgfsetfillcolor{currentfill}%
\pgfsetlinewidth{0.000000pt}%
\definecolor{currentstroke}{rgb}{0.000000,0.000000,0.000000}%
\pgfsetstrokecolor{currentstroke}%
\pgfsetstrokeopacity{0.000000}%
\pgfsetdash{}{0pt}%
\pgfpathmoveto{\pgfqpoint{2.817692in}{1.389496in}}%
\pgfpathlineto{\pgfqpoint{2.804293in}{1.384898in}}%
\pgfpathlineto{\pgfqpoint{2.788388in}{1.383110in}}%
\pgfpathlineto{\pgfqpoint{2.786603in}{1.400891in}}%
\pgfpathlineto{\pgfqpoint{2.777953in}{1.400497in}}%
\pgfpathlineto{\pgfqpoint{2.776762in}{1.425313in}}%
\pgfpathlineto{\pgfqpoint{2.795500in}{1.426123in}}%
\pgfpathlineto{\pgfqpoint{2.794880in}{1.439319in}}%
\pgfpathlineto{\pgfqpoint{2.819277in}{1.440727in}}%
\pgfpathlineto{\pgfqpoint{2.820238in}{1.426570in}}%
\pgfpathlineto{\pgfqpoint{2.815226in}{1.414852in}}%
\pgfpathlineto{\pgfqpoint{2.816007in}{1.404773in}}%
\pgfpathlineto{\pgfqpoint{2.818458in}{1.403774in}}%
\pgfpathlineto{\pgfqpoint{2.817692in}{1.389496in}}%
\pgfpathclose%
\pgfusepath{fill}%
\end{pgfscope}%
\begin{pgfscope}%
\pgfpathrectangle{\pgfqpoint{0.100000in}{0.100000in}}{\pgfqpoint{3.608454in}{2.310000in}}%
\pgfusepath{clip}%
\pgfsetbuttcap%
\pgfsetmiterjoin%
\definecolor{currentfill}{rgb}{0.000000,0.670588,0.664706}%
\pgfsetfillcolor{currentfill}%
\pgfsetlinewidth{0.000000pt}%
\definecolor{currentstroke}{rgb}{0.000000,0.000000,0.000000}%
\pgfsetstrokecolor{currentstroke}%
\pgfsetstrokeopacity{0.000000}%
\pgfsetdash{}{0pt}%
\pgfpathmoveto{\pgfqpoint{3.596315in}{2.093414in}}%
\pgfpathlineto{\pgfqpoint{3.619860in}{2.089848in}}%
\pgfpathlineto{\pgfqpoint{3.617148in}{2.084070in}}%
\pgfpathlineto{\pgfqpoint{3.624313in}{2.074071in}}%
\pgfpathlineto{\pgfqpoint{3.623386in}{2.068081in}}%
\pgfpathlineto{\pgfqpoint{3.634273in}{2.058968in}}%
\pgfpathlineto{\pgfqpoint{3.636385in}{2.065102in}}%
\pgfpathlineto{\pgfqpoint{3.643097in}{2.064783in}}%
\pgfpathlineto{\pgfqpoint{3.657645in}{2.048872in}}%
\pgfpathlineto{\pgfqpoint{3.660907in}{2.041358in}}%
\pgfpathlineto{\pgfqpoint{3.656677in}{2.034155in}}%
\pgfpathlineto{\pgfqpoint{3.653836in}{2.025053in}}%
\pgfpathlineto{\pgfqpoint{3.642794in}{2.024053in}}%
\pgfpathlineto{\pgfqpoint{3.643360in}{2.017403in}}%
\pgfpathlineto{\pgfqpoint{3.634988in}{2.017176in}}%
\pgfpathlineto{\pgfqpoint{3.633250in}{2.008704in}}%
\pgfpathlineto{\pgfqpoint{3.624318in}{2.007131in}}%
\pgfpathlineto{\pgfqpoint{3.621576in}{1.995429in}}%
\pgfpathlineto{\pgfqpoint{3.618035in}{1.994535in}}%
\pgfpathlineto{\pgfqpoint{3.610482in}{2.006964in}}%
\pgfpathlineto{\pgfqpoint{3.597567in}{2.032110in}}%
\pgfpathlineto{\pgfqpoint{3.603808in}{2.035391in}}%
\pgfpathlineto{\pgfqpoint{3.592928in}{2.056963in}}%
\pgfpathlineto{\pgfqpoint{3.598893in}{2.059799in}}%
\pgfpathlineto{\pgfqpoint{3.584395in}{2.086358in}}%
\pgfpathlineto{\pgfqpoint{3.596315in}{2.093414in}}%
\pgfpathclose%
\pgfusepath{fill}%
\end{pgfscope}%
\begin{pgfscope}%
\pgfpathrectangle{\pgfqpoint{0.100000in}{0.100000in}}{\pgfqpoint{3.608454in}{2.310000in}}%
\pgfusepath{clip}%
\pgfsetbuttcap%
\pgfsetmiterjoin%
\definecolor{currentfill}{rgb}{0.000000,0.529412,0.735294}%
\pgfsetfillcolor{currentfill}%
\pgfsetlinewidth{0.000000pt}%
\definecolor{currentstroke}{rgb}{0.000000,0.000000,0.000000}%
\pgfsetstrokecolor{currentstroke}%
\pgfsetstrokeopacity{0.000000}%
\pgfsetdash{}{0pt}%
\pgfpathmoveto{\pgfqpoint{2.833971in}{1.586462in}}%
\pgfpathlineto{\pgfqpoint{2.818905in}{1.581252in}}%
\pgfpathlineto{\pgfqpoint{2.796288in}{1.578242in}}%
\pgfpathlineto{\pgfqpoint{2.791356in}{1.581015in}}%
\pgfpathlineto{\pgfqpoint{2.794328in}{1.554792in}}%
\pgfpathlineto{\pgfqpoint{2.767138in}{1.551464in}}%
\pgfpathlineto{\pgfqpoint{2.746637in}{1.548837in}}%
\pgfpathlineto{\pgfqpoint{2.744178in}{1.569410in}}%
\pgfpathlineto{\pgfqpoint{2.737491in}{1.568598in}}%
\pgfpathlineto{\pgfqpoint{2.734410in}{1.575123in}}%
\pgfpathlineto{\pgfqpoint{2.731544in}{1.590059in}}%
\pgfpathlineto{\pgfqpoint{2.768691in}{1.595961in}}%
\pgfpathlineto{\pgfqpoint{2.764552in}{1.624067in}}%
\pgfpathlineto{\pgfqpoint{2.792334in}{1.628535in}}%
\pgfpathlineto{\pgfqpoint{2.799445in}{1.624219in}}%
\pgfpathlineto{\pgfqpoint{2.792747in}{1.616003in}}%
\pgfpathlineto{\pgfqpoint{2.786784in}{1.605434in}}%
\pgfpathlineto{\pgfqpoint{2.786107in}{1.596021in}}%
\pgfpathlineto{\pgfqpoint{2.793257in}{1.596373in}}%
\pgfpathlineto{\pgfqpoint{2.810519in}{1.591373in}}%
\pgfpathlineto{\pgfqpoint{2.817182in}{1.586236in}}%
\pgfpathlineto{\pgfqpoint{2.831148in}{1.589793in}}%
\pgfpathlineto{\pgfqpoint{2.833971in}{1.586462in}}%
\pgfpathclose%
\pgfusepath{fill}%
\end{pgfscope}%
\begin{pgfscope}%
\pgfpathrectangle{\pgfqpoint{0.100000in}{0.100000in}}{\pgfqpoint{3.608454in}{2.310000in}}%
\pgfusepath{clip}%
\pgfsetbuttcap%
\pgfsetmiterjoin%
\definecolor{currentfill}{rgb}{0.000000,0.556863,0.721569}%
\pgfsetfillcolor{currentfill}%
\pgfsetlinewidth{0.000000pt}%
\definecolor{currentstroke}{rgb}{0.000000,0.000000,0.000000}%
\pgfsetstrokecolor{currentstroke}%
\pgfsetstrokeopacity{0.000000}%
\pgfsetdash{}{0pt}%
\pgfpathmoveto{\pgfqpoint{2.380907in}{1.900334in}}%
\pgfpathlineto{\pgfqpoint{2.382793in}{1.866283in}}%
\pgfpathlineto{\pgfqpoint{2.383604in}{1.852652in}}%
\pgfpathlineto{\pgfqpoint{2.348110in}{1.850512in}}%
\pgfpathlineto{\pgfqpoint{2.345925in}{1.898515in}}%
\pgfpathlineto{\pgfqpoint{2.332207in}{1.897841in}}%
\pgfpathlineto{\pgfqpoint{2.331620in}{1.911661in}}%
\pgfpathlineto{\pgfqpoint{2.297058in}{1.910416in}}%
\pgfpathlineto{\pgfqpoint{2.295263in}{1.958166in}}%
\pgfpathlineto{\pgfqpoint{2.305024in}{1.961414in}}%
\pgfpathlineto{\pgfqpoint{2.313626in}{1.967707in}}%
\pgfpathlineto{\pgfqpoint{2.319119in}{1.967177in}}%
\pgfpathlineto{\pgfqpoint{2.326155in}{1.974190in}}%
\pgfpathlineto{\pgfqpoint{2.332310in}{1.976143in}}%
\pgfpathlineto{\pgfqpoint{2.338487in}{1.970865in}}%
\pgfpathlineto{\pgfqpoint{2.331663in}{1.959426in}}%
\pgfpathlineto{\pgfqpoint{2.331044in}{1.945780in}}%
\pgfpathlineto{\pgfqpoint{2.342366in}{1.952806in}}%
\pgfpathlineto{\pgfqpoint{2.350608in}{1.946791in}}%
\pgfpathlineto{\pgfqpoint{2.351813in}{1.919479in}}%
\pgfpathlineto{\pgfqpoint{2.358761in}{1.919792in}}%
\pgfpathlineto{\pgfqpoint{2.359091in}{1.912916in}}%
\pgfpathlineto{\pgfqpoint{2.365878in}{1.913327in}}%
\pgfpathlineto{\pgfqpoint{2.366647in}{1.899528in}}%
\pgfpathlineto{\pgfqpoint{2.380907in}{1.900334in}}%
\pgfpathclose%
\pgfusepath{fill}%
\end{pgfscope}%
\begin{pgfscope}%
\pgfpathrectangle{\pgfqpoint{0.100000in}{0.100000in}}{\pgfqpoint{3.608454in}{2.310000in}}%
\pgfusepath{clip}%
\pgfsetbuttcap%
\pgfsetmiterjoin%
\definecolor{currentfill}{rgb}{0.000000,0.611765,0.694118}%
\pgfsetfillcolor{currentfill}%
\pgfsetlinewidth{0.000000pt}%
\definecolor{currentstroke}{rgb}{0.000000,0.000000,0.000000}%
\pgfsetstrokecolor{currentstroke}%
\pgfsetstrokeopacity{0.000000}%
\pgfsetdash{}{0pt}%
\pgfpathmoveto{\pgfqpoint{2.634617in}{1.461203in}}%
\pgfpathlineto{\pgfqpoint{2.634909in}{1.458446in}}%
\pgfpathlineto{\pgfqpoint{2.607655in}{1.455712in}}%
\pgfpathlineto{\pgfqpoint{2.607399in}{1.458524in}}%
\pgfpathlineto{\pgfqpoint{2.583559in}{1.456426in}}%
\pgfpathlineto{\pgfqpoint{2.585430in}{1.435815in}}%
\pgfpathlineto{\pgfqpoint{2.565557in}{1.434073in}}%
\pgfpathlineto{\pgfqpoint{2.565221in}{1.447932in}}%
\pgfpathlineto{\pgfqpoint{2.557572in}{1.448924in}}%
\pgfpathlineto{\pgfqpoint{2.559183in}{1.428045in}}%
\pgfpathlineto{\pgfqpoint{2.534753in}{1.425828in}}%
\pgfpathlineto{\pgfqpoint{2.532347in}{1.453054in}}%
\pgfpathlineto{\pgfqpoint{2.531157in}{1.473630in}}%
\pgfpathlineto{\pgfqpoint{2.555646in}{1.476018in}}%
\pgfpathlineto{\pgfqpoint{2.554067in}{1.495355in}}%
\pgfpathlineto{\pgfqpoint{2.579594in}{1.497487in}}%
\pgfpathlineto{\pgfqpoint{2.578828in}{1.505444in}}%
\pgfpathlineto{\pgfqpoint{2.585581in}{1.506060in}}%
\pgfpathlineto{\pgfqpoint{2.588380in}{1.512229in}}%
\pgfpathlineto{\pgfqpoint{2.609172in}{1.513996in}}%
\pgfpathlineto{\pgfqpoint{2.610441in}{1.500202in}}%
\pgfpathlineto{\pgfqpoint{2.614139in}{1.497101in}}%
\pgfpathlineto{\pgfqpoint{2.623185in}{1.497972in}}%
\pgfpathlineto{\pgfqpoint{2.625004in}{1.477436in}}%
\pgfpathlineto{\pgfqpoint{2.633399in}{1.473636in}}%
\pgfpathlineto{\pgfqpoint{2.634617in}{1.461203in}}%
\pgfpathclose%
\pgfusepath{fill}%
\end{pgfscope}%
\begin{pgfscope}%
\pgfpathrectangle{\pgfqpoint{0.100000in}{0.100000in}}{\pgfqpoint{3.608454in}{2.310000in}}%
\pgfusepath{clip}%
\pgfsetbuttcap%
\pgfsetmiterjoin%
\definecolor{currentfill}{rgb}{0.000000,0.788235,0.605882}%
\pgfsetfillcolor{currentfill}%
\pgfsetlinewidth{0.000000pt}%
\definecolor{currentstroke}{rgb}{0.000000,0.000000,0.000000}%
\pgfsetstrokecolor{currentstroke}%
\pgfsetstrokeopacity{0.000000}%
\pgfsetdash{}{0pt}%
\pgfpathmoveto{\pgfqpoint{2.202317in}{0.720035in}}%
\pgfpathlineto{\pgfqpoint{2.181123in}{0.718301in}}%
\pgfpathlineto{\pgfqpoint{2.166253in}{0.713450in}}%
\pgfpathlineto{\pgfqpoint{2.152123in}{0.723725in}}%
\pgfpathlineto{\pgfqpoint{2.145077in}{0.730301in}}%
\pgfpathlineto{\pgfqpoint{2.140576in}{0.738107in}}%
\pgfpathlineto{\pgfqpoint{2.125247in}{0.741762in}}%
\pgfpathlineto{\pgfqpoint{2.115910in}{0.747279in}}%
\pgfpathlineto{\pgfqpoint{2.111897in}{0.751748in}}%
\pgfpathlineto{\pgfqpoint{2.108517in}{0.767596in}}%
\pgfpathlineto{\pgfqpoint{2.111050in}{0.772661in}}%
\pgfpathlineto{\pgfqpoint{2.143853in}{0.772670in}}%
\pgfpathlineto{\pgfqpoint{2.139795in}{0.782977in}}%
\pgfpathlineto{\pgfqpoint{2.173280in}{0.783716in}}%
\pgfpathlineto{\pgfqpoint{2.182637in}{0.773027in}}%
\pgfpathlineto{\pgfqpoint{2.186530in}{0.767441in}}%
\pgfpathlineto{\pgfqpoint{2.187852in}{0.760329in}}%
\pgfpathlineto{\pgfqpoint{2.185571in}{0.753524in}}%
\pgfpathlineto{\pgfqpoint{2.191816in}{0.743213in}}%
\pgfpathlineto{\pgfqpoint{2.197023in}{0.737625in}}%
\pgfpathlineto{\pgfqpoint{2.196211in}{0.730685in}}%
\pgfpathlineto{\pgfqpoint{2.202317in}{0.720035in}}%
\pgfpathclose%
\pgfusepath{fill}%
\end{pgfscope}%
\begin{pgfscope}%
\pgfpathrectangle{\pgfqpoint{0.100000in}{0.100000in}}{\pgfqpoint{3.608454in}{2.310000in}}%
\pgfusepath{clip}%
\pgfsetbuttcap%
\pgfsetmiterjoin%
\definecolor{currentfill}{rgb}{0.000000,0.725490,0.637255}%
\pgfsetfillcolor{currentfill}%
\pgfsetlinewidth{0.000000pt}%
\definecolor{currentstroke}{rgb}{0.000000,0.000000,0.000000}%
\pgfsetstrokecolor{currentstroke}%
\pgfsetstrokeopacity{0.000000}%
\pgfsetdash{}{0pt}%
\pgfpathmoveto{\pgfqpoint{3.147047in}{0.962987in}}%
\pgfpathlineto{\pgfqpoint{3.134423in}{0.971285in}}%
\pgfpathlineto{\pgfqpoint{3.115214in}{0.974844in}}%
\pgfpathlineto{\pgfqpoint{3.097264in}{0.986382in}}%
\pgfpathlineto{\pgfqpoint{3.088906in}{0.984668in}}%
\pgfpathlineto{\pgfqpoint{3.094287in}{1.003646in}}%
\pgfpathlineto{\pgfqpoint{3.092298in}{1.007042in}}%
\pgfpathlineto{\pgfqpoint{3.096860in}{1.011889in}}%
\pgfpathlineto{\pgfqpoint{3.091334in}{1.021253in}}%
\pgfpathlineto{\pgfqpoint{3.096046in}{1.025469in}}%
\pgfpathlineto{\pgfqpoint{3.076612in}{1.036110in}}%
\pgfpathlineto{\pgfqpoint{3.078779in}{1.041854in}}%
\pgfpathlineto{\pgfqpoint{3.073425in}{1.048823in}}%
\pgfpathlineto{\pgfqpoint{3.065576in}{1.050744in}}%
\pgfpathlineto{\pgfqpoint{3.079593in}{1.063190in}}%
\pgfpathlineto{\pgfqpoint{3.095167in}{1.065172in}}%
\pgfpathlineto{\pgfqpoint{3.108787in}{1.052476in}}%
\pgfpathlineto{\pgfqpoint{3.116266in}{1.079750in}}%
\pgfpathlineto{\pgfqpoint{3.145061in}{1.058883in}}%
\pgfpathlineto{\pgfqpoint{3.185535in}{1.030022in}}%
\pgfpathlineto{\pgfqpoint{3.175718in}{1.024509in}}%
\pgfpathlineto{\pgfqpoint{3.169582in}{1.018517in}}%
\pgfpathlineto{\pgfqpoint{3.162304in}{1.007975in}}%
\pgfpathlineto{\pgfqpoint{3.155468in}{0.994985in}}%
\pgfpathlineto{\pgfqpoint{3.152816in}{0.985825in}}%
\pgfpathlineto{\pgfqpoint{3.152996in}{0.969508in}}%
\pgfpathlineto{\pgfqpoint{3.147047in}{0.962987in}}%
\pgfpathclose%
\pgfusepath{fill}%
\end{pgfscope}%
\begin{pgfscope}%
\pgfpathrectangle{\pgfqpoint{0.100000in}{0.100000in}}{\pgfqpoint{3.608454in}{2.310000in}}%
\pgfusepath{clip}%
\pgfsetbuttcap%
\pgfsetmiterjoin%
\definecolor{currentfill}{rgb}{0.000000,0.796078,0.601961}%
\pgfsetfillcolor{currentfill}%
\pgfsetlinewidth{0.000000pt}%
\definecolor{currentstroke}{rgb}{0.000000,0.000000,0.000000}%
\pgfsetstrokecolor{currentstroke}%
\pgfsetstrokeopacity{0.000000}%
\pgfsetdash{}{0pt}%
\pgfpathmoveto{\pgfqpoint{2.611522in}{1.917144in}}%
\pgfpathlineto{\pgfqpoint{2.605875in}{1.974408in}}%
\pgfpathlineto{\pgfqpoint{2.625687in}{1.975346in}}%
\pgfpathlineto{\pgfqpoint{2.636425in}{1.981889in}}%
\pgfpathlineto{\pgfqpoint{2.653804in}{1.986050in}}%
\pgfpathlineto{\pgfqpoint{2.651220in}{1.978697in}}%
\pgfpathlineto{\pgfqpoint{2.652749in}{1.962987in}}%
\pgfpathlineto{\pgfqpoint{2.664373in}{1.960858in}}%
\pgfpathlineto{\pgfqpoint{2.672255in}{1.965247in}}%
\pgfpathlineto{\pgfqpoint{2.677033in}{1.960330in}}%
\pgfpathlineto{\pgfqpoint{2.686113in}{1.967978in}}%
\pgfpathlineto{\pgfqpoint{2.702927in}{1.970218in}}%
\pgfpathlineto{\pgfqpoint{2.701836in}{1.963213in}}%
\pgfpathlineto{\pgfqpoint{2.704934in}{1.955834in}}%
\pgfpathlineto{\pgfqpoint{2.703228in}{1.945319in}}%
\pgfpathlineto{\pgfqpoint{2.711142in}{1.941318in}}%
\pgfpathlineto{\pgfqpoint{2.714091in}{1.933704in}}%
\pgfpathlineto{\pgfqpoint{2.724603in}{1.932074in}}%
\pgfpathlineto{\pgfqpoint{2.735969in}{1.940892in}}%
\pgfpathlineto{\pgfqpoint{2.742949in}{1.933510in}}%
\pgfpathlineto{\pgfqpoint{2.741240in}{1.928257in}}%
\pgfpathlineto{\pgfqpoint{2.733601in}{1.928846in}}%
\pgfpathlineto{\pgfqpoint{2.725778in}{1.927125in}}%
\pgfpathlineto{\pgfqpoint{2.720067in}{1.929651in}}%
\pgfpathlineto{\pgfqpoint{2.706856in}{1.928149in}}%
\pgfpathlineto{\pgfqpoint{2.693720in}{1.923132in}}%
\pgfpathlineto{\pgfqpoint{2.684872in}{1.925062in}}%
\pgfpathlineto{\pgfqpoint{2.682294in}{1.929947in}}%
\pgfpathlineto{\pgfqpoint{2.675578in}{1.929116in}}%
\pgfpathlineto{\pgfqpoint{2.673548in}{1.913649in}}%
\pgfpathlineto{\pgfqpoint{2.664389in}{1.919036in}}%
\pgfpathlineto{\pgfqpoint{2.658981in}{1.924851in}}%
\pgfpathlineto{\pgfqpoint{2.650299in}{1.927597in}}%
\pgfpathlineto{\pgfqpoint{2.635790in}{1.929952in}}%
\pgfpathlineto{\pgfqpoint{2.629423in}{1.928847in}}%
\pgfpathlineto{\pgfqpoint{2.622249in}{1.917995in}}%
\pgfpathlineto{\pgfqpoint{2.611522in}{1.917144in}}%
\pgfpathclose%
\pgfusepath{fill}%
\end{pgfscope}%
\begin{pgfscope}%
\pgfpathrectangle{\pgfqpoint{0.100000in}{0.100000in}}{\pgfqpoint{3.608454in}{2.310000in}}%
\pgfusepath{clip}%
\pgfsetbuttcap%
\pgfsetmiterjoin%
\definecolor{currentfill}{rgb}{0.000000,0.662745,0.668627}%
\pgfsetfillcolor{currentfill}%
\pgfsetlinewidth{0.000000pt}%
\definecolor{currentstroke}{rgb}{0.000000,0.000000,0.000000}%
\pgfsetstrokecolor{currentstroke}%
\pgfsetstrokeopacity{0.000000}%
\pgfsetdash{}{0pt}%
\pgfpathmoveto{\pgfqpoint{2.482016in}{1.066685in}}%
\pgfpathlineto{\pgfqpoint{2.463288in}{1.065984in}}%
\pgfpathlineto{\pgfqpoint{2.461856in}{1.079471in}}%
\pgfpathlineto{\pgfqpoint{2.453807in}{1.083497in}}%
\pgfpathlineto{\pgfqpoint{2.441604in}{1.083493in}}%
\pgfpathlineto{\pgfqpoint{2.434616in}{1.075007in}}%
\pgfpathlineto{\pgfqpoint{2.432045in}{1.079916in}}%
\pgfpathlineto{\pgfqpoint{2.431130in}{1.090720in}}%
\pgfpathlineto{\pgfqpoint{2.439328in}{1.092627in}}%
\pgfpathlineto{\pgfqpoint{2.442664in}{1.097005in}}%
\pgfpathlineto{\pgfqpoint{2.442276in}{1.103468in}}%
\pgfpathlineto{\pgfqpoint{2.445220in}{1.109326in}}%
\pgfpathlineto{\pgfqpoint{2.446875in}{1.118895in}}%
\pgfpathlineto{\pgfqpoint{2.452158in}{1.122751in}}%
\pgfpathlineto{\pgfqpoint{2.445412in}{1.130104in}}%
\pgfpathlineto{\pgfqpoint{2.455346in}{1.132898in}}%
\pgfpathlineto{\pgfqpoint{2.454276in}{1.139497in}}%
\pgfpathlineto{\pgfqpoint{2.453898in}{1.152266in}}%
\pgfpathlineto{\pgfqpoint{2.457328in}{1.152433in}}%
\pgfpathlineto{\pgfqpoint{2.461626in}{1.152803in}}%
\pgfpathlineto{\pgfqpoint{2.465696in}{1.145939in}}%
\pgfpathlineto{\pgfqpoint{2.462058in}{1.133851in}}%
\pgfpathlineto{\pgfqpoint{2.458843in}{1.129936in}}%
\pgfpathlineto{\pgfqpoint{2.479806in}{1.130609in}}%
\pgfpathlineto{\pgfqpoint{2.480144in}{1.116463in}}%
\pgfpathlineto{\pgfqpoint{2.469563in}{1.097602in}}%
\pgfpathlineto{\pgfqpoint{2.487860in}{1.090523in}}%
\pgfpathlineto{\pgfqpoint{2.488608in}{1.069834in}}%
\pgfpathlineto{\pgfqpoint{2.482016in}{1.066685in}}%
\pgfpathclose%
\pgfusepath{fill}%
\end{pgfscope}%
\begin{pgfscope}%
\pgfpathrectangle{\pgfqpoint{0.100000in}{0.100000in}}{\pgfqpoint{3.608454in}{2.310000in}}%
\pgfusepath{clip}%
\pgfsetbuttcap%
\pgfsetmiterjoin%
\definecolor{currentfill}{rgb}{0.000000,0.470588,0.764706}%
\pgfsetfillcolor{currentfill}%
\pgfsetlinewidth{0.000000pt}%
\definecolor{currentstroke}{rgb}{0.000000,0.000000,0.000000}%
\pgfsetstrokecolor{currentstroke}%
\pgfsetstrokeopacity{0.000000}%
\pgfsetdash{}{0pt}%
\pgfpathmoveto{\pgfqpoint{2.970938in}{1.132406in}}%
\pgfpathlineto{\pgfqpoint{2.978683in}{1.148660in}}%
\pgfpathlineto{\pgfqpoint{2.973998in}{1.150189in}}%
\pgfpathlineto{\pgfqpoint{2.967129in}{1.148531in}}%
\pgfpathlineto{\pgfqpoint{2.956204in}{1.154739in}}%
\pgfpathlineto{\pgfqpoint{2.949055in}{1.160645in}}%
\pgfpathlineto{\pgfqpoint{2.955589in}{1.174438in}}%
\pgfpathlineto{\pgfqpoint{2.963826in}{1.175254in}}%
\pgfpathlineto{\pgfqpoint{2.972912in}{1.172989in}}%
\pgfpathlineto{\pgfqpoint{2.978982in}{1.167861in}}%
\pgfpathlineto{\pgfqpoint{2.987201in}{1.172347in}}%
\pgfpathlineto{\pgfqpoint{2.993174in}{1.172573in}}%
\pgfpathlineto{\pgfqpoint{3.006498in}{1.176840in}}%
\pgfpathlineto{\pgfqpoint{3.018531in}{1.178366in}}%
\pgfpathlineto{\pgfqpoint{3.020036in}{1.162708in}}%
\pgfpathlineto{\pgfqpoint{3.018191in}{1.148162in}}%
\pgfpathlineto{\pgfqpoint{3.022456in}{1.135384in}}%
\pgfpathlineto{\pgfqpoint{3.008399in}{1.134076in}}%
\pgfpathlineto{\pgfqpoint{3.007763in}{1.136339in}}%
\pgfpathlineto{\pgfqpoint{2.970938in}{1.132406in}}%
\pgfpathclose%
\pgfusepath{fill}%
\end{pgfscope}%
\begin{pgfscope}%
\pgfpathrectangle{\pgfqpoint{0.100000in}{0.100000in}}{\pgfqpoint{3.608454in}{2.310000in}}%
\pgfusepath{clip}%
\pgfsetbuttcap%
\pgfsetmiterjoin%
\definecolor{currentfill}{rgb}{0.000000,0.517647,0.741176}%
\pgfsetfillcolor{currentfill}%
\pgfsetlinewidth{0.000000pt}%
\definecolor{currentstroke}{rgb}{0.000000,0.000000,0.000000}%
\pgfsetstrokecolor{currentstroke}%
\pgfsetstrokeopacity{0.000000}%
\pgfsetdash{}{0pt}%
\pgfpathmoveto{\pgfqpoint{2.801132in}{1.502933in}}%
\pgfpathlineto{\pgfqpoint{2.795078in}{1.502065in}}%
\pgfpathlineto{\pgfqpoint{2.773279in}{1.502104in}}%
\pgfpathlineto{\pgfqpoint{2.767138in}{1.551464in}}%
\pgfpathlineto{\pgfqpoint{2.794328in}{1.554792in}}%
\pgfpathlineto{\pgfqpoint{2.791356in}{1.581015in}}%
\pgfpathlineto{\pgfqpoint{2.796288in}{1.578242in}}%
\pgfpathlineto{\pgfqpoint{2.818905in}{1.581252in}}%
\pgfpathlineto{\pgfqpoint{2.825402in}{1.579904in}}%
\pgfpathlineto{\pgfqpoint{2.831040in}{1.546066in}}%
\pgfpathlineto{\pgfqpoint{2.814322in}{1.543583in}}%
\pgfpathlineto{\pgfqpoint{2.817395in}{1.520810in}}%
\pgfpathlineto{\pgfqpoint{2.799219in}{1.517163in}}%
\pgfpathlineto{\pgfqpoint{2.801132in}{1.502933in}}%
\pgfpathclose%
\pgfusepath{fill}%
\end{pgfscope}%
\begin{pgfscope}%
\pgfpathrectangle{\pgfqpoint{0.100000in}{0.100000in}}{\pgfqpoint{3.608454in}{2.310000in}}%
\pgfusepath{clip}%
\pgfsetbuttcap%
\pgfsetmiterjoin%
\definecolor{currentfill}{rgb}{0.000000,0.709804,0.645098}%
\pgfsetfillcolor{currentfill}%
\pgfsetlinewidth{0.000000pt}%
\definecolor{currentstroke}{rgb}{0.000000,0.000000,0.000000}%
\pgfsetstrokecolor{currentstroke}%
\pgfsetstrokeopacity{0.000000}%
\pgfsetdash{}{0pt}%
\pgfpathmoveto{\pgfqpoint{3.436777in}{1.747450in}}%
\pgfpathlineto{\pgfqpoint{3.384285in}{1.735025in}}%
\pgfpathlineto{\pgfqpoint{3.380209in}{1.761732in}}%
\pgfpathlineto{\pgfqpoint{3.378669in}{1.782710in}}%
\pgfpathlineto{\pgfqpoint{3.373418in}{1.786388in}}%
\pgfpathlineto{\pgfqpoint{3.377987in}{1.790511in}}%
\pgfpathlineto{\pgfqpoint{3.414849in}{1.798333in}}%
\pgfpathlineto{\pgfqpoint{3.419015in}{1.794601in}}%
\pgfpathlineto{\pgfqpoint{3.418010in}{1.787350in}}%
\pgfpathlineto{\pgfqpoint{3.420462in}{1.768872in}}%
\pgfpathlineto{\pgfqpoint{3.427810in}{1.762986in}}%
\pgfpathlineto{\pgfqpoint{3.426466in}{1.757361in}}%
\pgfpathlineto{\pgfqpoint{3.434236in}{1.757543in}}%
\pgfpathlineto{\pgfqpoint{3.436777in}{1.747450in}}%
\pgfpathclose%
\pgfusepath{fill}%
\end{pgfscope}%
\begin{pgfscope}%
\pgfpathrectangle{\pgfqpoint{0.100000in}{0.100000in}}{\pgfqpoint{3.608454in}{2.310000in}}%
\pgfusepath{clip}%
\pgfsetbuttcap%
\pgfsetmiterjoin%
\definecolor{currentfill}{rgb}{0.000000,0.470588,0.764706}%
\pgfsetfillcolor{currentfill}%
\pgfsetlinewidth{0.000000pt}%
\definecolor{currentstroke}{rgb}{0.000000,0.000000,0.000000}%
\pgfsetstrokecolor{currentstroke}%
\pgfsetstrokeopacity{0.000000}%
\pgfsetdash{}{0pt}%
\pgfpathmoveto{\pgfqpoint{1.938292in}{1.256226in}}%
\pgfpathlineto{\pgfqpoint{1.938903in}{1.276826in}}%
\pgfpathlineto{\pgfqpoint{1.925218in}{1.277253in}}%
\pgfpathlineto{\pgfqpoint{1.926190in}{1.311718in}}%
\pgfpathlineto{\pgfqpoint{1.960306in}{1.310682in}}%
\pgfpathlineto{\pgfqpoint{1.987581in}{1.310037in}}%
\pgfpathlineto{\pgfqpoint{1.987451in}{1.303171in}}%
\pgfpathlineto{\pgfqpoint{1.994270in}{1.303029in}}%
\pgfpathlineto{\pgfqpoint{1.992866in}{1.289314in}}%
\pgfpathlineto{\pgfqpoint{1.992256in}{1.268523in}}%
\pgfpathlineto{\pgfqpoint{1.972846in}{1.269115in}}%
\pgfpathlineto{\pgfqpoint{1.972560in}{1.255285in}}%
\pgfpathlineto{\pgfqpoint{1.938292in}{1.256226in}}%
\pgfpathclose%
\pgfusepath{fill}%
\end{pgfscope}%
\begin{pgfscope}%
\pgfpathrectangle{\pgfqpoint{0.100000in}{0.100000in}}{\pgfqpoint{3.608454in}{2.310000in}}%
\pgfusepath{clip}%
\pgfsetbuttcap%
\pgfsetmiterjoin%
\definecolor{currentfill}{rgb}{0.000000,0.431373,0.784314}%
\pgfsetfillcolor{currentfill}%
\pgfsetlinewidth{0.000000pt}%
\definecolor{currentstroke}{rgb}{0.000000,0.000000,0.000000}%
\pgfsetstrokecolor{currentstroke}%
\pgfsetstrokeopacity{0.000000}%
\pgfsetdash{}{0pt}%
\pgfpathmoveto{\pgfqpoint{1.719085in}{1.966639in}}%
\pgfpathlineto{\pgfqpoint{1.718524in}{1.959750in}}%
\pgfpathlineto{\pgfqpoint{1.673140in}{1.963430in}}%
\pgfpathlineto{\pgfqpoint{1.635984in}{1.966799in}}%
\pgfpathlineto{\pgfqpoint{1.638609in}{1.994610in}}%
\pgfpathlineto{\pgfqpoint{1.635493in}{1.994898in}}%
\pgfpathlineto{\pgfqpoint{1.638167in}{2.022688in}}%
\pgfpathlineto{\pgfqpoint{1.668766in}{2.019808in}}%
\pgfpathlineto{\pgfqpoint{1.671188in}{2.047334in}}%
\pgfpathlineto{\pgfqpoint{1.695708in}{2.045165in}}%
\pgfpathlineto{\pgfqpoint{1.700751in}{2.053267in}}%
\pgfpathlineto{\pgfqpoint{1.710047in}{2.050726in}}%
\pgfpathlineto{\pgfqpoint{1.716886in}{2.052052in}}%
\pgfpathlineto{\pgfqpoint{1.715648in}{2.039417in}}%
\pgfpathlineto{\pgfqpoint{1.718555in}{2.035319in}}%
\pgfpathlineto{\pgfqpoint{1.716982in}{2.015681in}}%
\pgfpathlineto{\pgfqpoint{1.720286in}{2.015429in}}%
\pgfpathlineto{\pgfqpoint{1.718098in}{1.987796in}}%
\pgfpathlineto{\pgfqpoint{1.720698in}{1.987602in}}%
\pgfpathlineto{\pgfqpoint{1.719085in}{1.966639in}}%
\pgfpathclose%
\pgfusepath{fill}%
\end{pgfscope}%
\begin{pgfscope}%
\pgfpathrectangle{\pgfqpoint{0.100000in}{0.100000in}}{\pgfqpoint{3.608454in}{2.310000in}}%
\pgfusepath{clip}%
\pgfsetbuttcap%
\pgfsetmiterjoin%
\definecolor{currentfill}{rgb}{0.000000,0.368627,0.815686}%
\pgfsetfillcolor{currentfill}%
\pgfsetlinewidth{0.000000pt}%
\definecolor{currentstroke}{rgb}{0.000000,0.000000,0.000000}%
\pgfsetstrokecolor{currentstroke}%
\pgfsetstrokeopacity{0.000000}%
\pgfsetdash{}{0pt}%
\pgfpathmoveto{\pgfqpoint{0.934361in}{0.385742in}}%
\pgfpathlineto{\pgfqpoint{0.935737in}{0.388869in}}%
\pgfpathlineto{\pgfqpoint{0.937444in}{0.389743in}}%
\pgfpathlineto{\pgfqpoint{0.940486in}{0.388027in}}%
\pgfpathlineto{\pgfqpoint{0.940016in}{0.386614in}}%
\pgfpathlineto{\pgfqpoint{0.942968in}{0.383955in}}%
\pgfpathlineto{\pgfqpoint{0.936317in}{0.384289in}}%
\pgfpathlineto{\pgfqpoint{0.934361in}{0.385742in}}%
\pgfpathclose%
\pgfusepath{fill}%
\end{pgfscope}%
\begin{pgfscope}%
\pgfpathrectangle{\pgfqpoint{0.100000in}{0.100000in}}{\pgfqpoint{3.608454in}{2.310000in}}%
\pgfusepath{clip}%
\pgfsetbuttcap%
\pgfsetmiterjoin%
\definecolor{currentfill}{rgb}{0.000000,0.368627,0.815686}%
\pgfsetfillcolor{currentfill}%
\pgfsetlinewidth{0.000000pt}%
\definecolor{currentstroke}{rgb}{0.000000,0.000000,0.000000}%
\pgfsetstrokecolor{currentstroke}%
\pgfsetstrokeopacity{0.000000}%
\pgfsetdash{}{0pt}%
\pgfpathmoveto{\pgfqpoint{0.911727in}{0.383608in}}%
\pgfpathlineto{\pgfqpoint{0.914559in}{0.385588in}}%
\pgfpathlineto{\pgfqpoint{0.917234in}{0.385734in}}%
\pgfpathlineto{\pgfqpoint{0.927384in}{0.388734in}}%
\pgfpathlineto{\pgfqpoint{0.928405in}{0.390996in}}%
\pgfpathlineto{\pgfqpoint{0.930625in}{0.389633in}}%
\pgfpathlineto{\pgfqpoint{0.930606in}{0.387345in}}%
\pgfpathlineto{\pgfqpoint{0.924754in}{0.385978in}}%
\pgfpathlineto{\pgfqpoint{0.920480in}{0.383050in}}%
\pgfpathlineto{\pgfqpoint{0.918337in}{0.383207in}}%
\pgfpathlineto{\pgfqpoint{0.917432in}{0.381621in}}%
\pgfpathlineto{\pgfqpoint{0.912959in}{0.382564in}}%
\pgfpathlineto{\pgfqpoint{0.911727in}{0.383608in}}%
\pgfpathclose%
\pgfusepath{fill}%
\end{pgfscope}%
\begin{pgfscope}%
\pgfpathrectangle{\pgfqpoint{0.100000in}{0.100000in}}{\pgfqpoint{3.608454in}{2.310000in}}%
\pgfusepath{clip}%
\pgfsetbuttcap%
\pgfsetmiterjoin%
\definecolor{currentfill}{rgb}{0.000000,0.368627,0.815686}%
\pgfsetfillcolor{currentfill}%
\pgfsetlinewidth{0.000000pt}%
\definecolor{currentstroke}{rgb}{0.000000,0.000000,0.000000}%
\pgfsetstrokecolor{currentstroke}%
\pgfsetstrokeopacity{0.000000}%
\pgfsetdash{}{0pt}%
\pgfpathmoveto{\pgfqpoint{0.918694in}{0.390985in}}%
\pgfpathlineto{\pgfqpoint{0.922043in}{0.397743in}}%
\pgfpathlineto{\pgfqpoint{0.922804in}{0.395503in}}%
\pgfpathlineto{\pgfqpoint{0.918694in}{0.390985in}}%
\pgfpathclose%
\pgfusepath{fill}%
\end{pgfscope}%
\begin{pgfscope}%
\pgfpathrectangle{\pgfqpoint{0.100000in}{0.100000in}}{\pgfqpoint{3.608454in}{2.310000in}}%
\pgfusepath{clip}%
\pgfsetbuttcap%
\pgfsetmiterjoin%
\definecolor{currentfill}{rgb}{0.000000,0.368627,0.815686}%
\pgfsetfillcolor{currentfill}%
\pgfsetlinewidth{0.000000pt}%
\definecolor{currentstroke}{rgb}{0.000000,0.000000,0.000000}%
\pgfsetstrokecolor{currentstroke}%
\pgfsetstrokeopacity{0.000000}%
\pgfsetdash{}{0pt}%
\pgfpathmoveto{\pgfqpoint{0.979580in}{0.452265in}}%
\pgfpathlineto{\pgfqpoint{0.983915in}{0.449521in}}%
\pgfpathlineto{\pgfqpoint{0.986635in}{0.453959in}}%
\pgfpathlineto{\pgfqpoint{0.997579in}{0.447110in}}%
\pgfpathlineto{\pgfqpoint{0.996218in}{0.444880in}}%
\pgfpathlineto{\pgfqpoint{1.002410in}{0.441046in}}%
\pgfpathlineto{\pgfqpoint{1.019938in}{0.430914in}}%
\pgfpathlineto{\pgfqpoint{1.017896in}{0.425014in}}%
\pgfpathlineto{\pgfqpoint{1.014239in}{0.423539in}}%
\pgfpathlineto{\pgfqpoint{1.015064in}{0.421049in}}%
\pgfpathlineto{\pgfqpoint{1.012216in}{0.417308in}}%
\pgfpathlineto{\pgfqpoint{1.015043in}{0.415179in}}%
\pgfpathlineto{\pgfqpoint{1.018284in}{0.417154in}}%
\pgfpathlineto{\pgfqpoint{1.018953in}{0.415851in}}%
\pgfpathlineto{\pgfqpoint{1.022698in}{0.413642in}}%
\pgfpathlineto{\pgfqpoint{1.021385in}{0.411488in}}%
\pgfpathlineto{\pgfqpoint{1.023662in}{0.410127in}}%
\pgfpathlineto{\pgfqpoint{1.022393in}{0.408007in}}%
\pgfpathlineto{\pgfqpoint{1.024400in}{0.406832in}}%
\pgfpathlineto{\pgfqpoint{1.019250in}{0.398079in}}%
\pgfpathlineto{\pgfqpoint{1.020524in}{0.396033in}}%
\pgfpathlineto{\pgfqpoint{1.017303in}{0.390307in}}%
\pgfpathlineto{\pgfqpoint{1.027182in}{0.384801in}}%
\pgfpathlineto{\pgfqpoint{1.016055in}{0.364729in}}%
\pgfpathlineto{\pgfqpoint{1.004608in}{0.344079in}}%
\pgfpathlineto{\pgfqpoint{0.997055in}{0.330455in}}%
\pgfpathlineto{\pgfqpoint{0.993111in}{0.336827in}}%
\pgfpathlineto{\pgfqpoint{0.996872in}{0.337073in}}%
\pgfpathlineto{\pgfqpoint{0.997859in}{0.339919in}}%
\pgfpathlineto{\pgfqpoint{0.995627in}{0.341376in}}%
\pgfpathlineto{\pgfqpoint{0.995160in}{0.339365in}}%
\pgfpathlineto{\pgfqpoint{0.990363in}{0.340190in}}%
\pgfpathlineto{\pgfqpoint{0.988940in}{0.342964in}}%
\pgfpathlineto{\pgfqpoint{0.979357in}{0.351803in}}%
\pgfpathlineto{\pgfqpoint{0.975077in}{0.352806in}}%
\pgfpathlineto{\pgfqpoint{0.969268in}{0.355424in}}%
\pgfpathlineto{\pgfqpoint{0.963603in}{0.357154in}}%
\pgfpathlineto{\pgfqpoint{0.962385in}{0.358792in}}%
\pgfpathlineto{\pgfqpoint{0.962290in}{0.362125in}}%
\pgfpathlineto{\pgfqpoint{0.961040in}{0.364752in}}%
\pgfpathlineto{\pgfqpoint{0.957865in}{0.367144in}}%
\pgfpathlineto{\pgfqpoint{0.958163in}{0.370011in}}%
\pgfpathlineto{\pgfqpoint{0.956916in}{0.372017in}}%
\pgfpathlineto{\pgfqpoint{0.959883in}{0.375282in}}%
\pgfpathlineto{\pgfqpoint{0.960378in}{0.377067in}}%
\pgfpathlineto{\pgfqpoint{0.958642in}{0.378412in}}%
\pgfpathlineto{\pgfqpoint{0.956108in}{0.376672in}}%
\pgfpathlineto{\pgfqpoint{0.953407in}{0.377974in}}%
\pgfpathlineto{\pgfqpoint{0.952146in}{0.377061in}}%
\pgfpathlineto{\pgfqpoint{0.949343in}{0.382116in}}%
\pgfpathlineto{\pgfqpoint{0.947228in}{0.383148in}}%
\pgfpathlineto{\pgfqpoint{0.949313in}{0.385149in}}%
\pgfpathlineto{\pgfqpoint{0.949856in}{0.389896in}}%
\pgfpathlineto{\pgfqpoint{0.949598in}{0.393135in}}%
\pgfpathlineto{\pgfqpoint{0.945394in}{0.394108in}}%
\pgfpathlineto{\pgfqpoint{0.947389in}{0.396132in}}%
\pgfpathlineto{\pgfqpoint{0.942948in}{0.398138in}}%
\pgfpathlineto{\pgfqpoint{0.943546in}{0.402478in}}%
\pgfpathlineto{\pgfqpoint{0.941428in}{0.403633in}}%
\pgfpathlineto{\pgfqpoint{0.940901in}{0.405559in}}%
\pgfpathlineto{\pgfqpoint{0.938307in}{0.405202in}}%
\pgfpathlineto{\pgfqpoint{0.937360in}{0.403019in}}%
\pgfpathlineto{\pgfqpoint{0.933986in}{0.407135in}}%
\pgfpathlineto{\pgfqpoint{0.935525in}{0.410297in}}%
\pgfpathlineto{\pgfqpoint{0.930641in}{0.409132in}}%
\pgfpathlineto{\pgfqpoint{0.924871in}{0.408830in}}%
\pgfpathlineto{\pgfqpoint{0.926703in}{0.411773in}}%
\pgfpathlineto{\pgfqpoint{0.928482in}{0.410664in}}%
\pgfpathlineto{\pgfqpoint{0.928700in}{0.414487in}}%
\pgfpathlineto{\pgfqpoint{0.926958in}{0.414550in}}%
\pgfpathlineto{\pgfqpoint{0.922100in}{0.411389in}}%
\pgfpathlineto{\pgfqpoint{0.918534in}{0.408022in}}%
\pgfpathlineto{\pgfqpoint{0.919688in}{0.407473in}}%
\pgfpathlineto{\pgfqpoint{0.921994in}{0.409270in}}%
\pgfpathlineto{\pgfqpoint{0.923912in}{0.407532in}}%
\pgfpathlineto{\pgfqpoint{0.921409in}{0.405514in}}%
\pgfpathlineto{\pgfqpoint{0.919314in}{0.405095in}}%
\pgfpathlineto{\pgfqpoint{0.916088in}{0.406347in}}%
\pgfpathlineto{\pgfqpoint{0.917614in}{0.402235in}}%
\pgfpathlineto{\pgfqpoint{0.919867in}{0.400817in}}%
\pgfpathlineto{\pgfqpoint{0.913772in}{0.398471in}}%
\pgfpathlineto{\pgfqpoint{0.912875in}{0.397284in}}%
\pgfpathlineto{\pgfqpoint{0.916456in}{0.393000in}}%
\pgfpathlineto{\pgfqpoint{0.915325in}{0.388126in}}%
\pgfpathlineto{\pgfqpoint{0.913318in}{0.388120in}}%
\pgfpathlineto{\pgfqpoint{0.911580in}{0.390660in}}%
\pgfpathlineto{\pgfqpoint{0.912461in}{0.393615in}}%
\pgfpathlineto{\pgfqpoint{0.910952in}{0.394996in}}%
\pgfpathlineto{\pgfqpoint{0.908291in}{0.391990in}}%
\pgfpathlineto{\pgfqpoint{0.906140in}{0.393309in}}%
\pgfpathlineto{\pgfqpoint{0.908663in}{0.396558in}}%
\pgfpathlineto{\pgfqpoint{0.913452in}{0.403668in}}%
\pgfpathlineto{\pgfqpoint{0.912214in}{0.404513in}}%
\pgfpathlineto{\pgfqpoint{0.917407in}{0.411973in}}%
\pgfpathlineto{\pgfqpoint{0.916327in}{0.412673in}}%
\pgfpathlineto{\pgfqpoint{0.918151in}{0.415401in}}%
\pgfpathlineto{\pgfqpoint{0.921451in}{0.413351in}}%
\pgfpathlineto{\pgfqpoint{0.926490in}{0.421058in}}%
\pgfpathlineto{\pgfqpoint{0.930988in}{0.427301in}}%
\pgfpathlineto{\pgfqpoint{0.945694in}{0.417578in}}%
\pgfpathlineto{\pgfqpoint{0.946494in}{0.418805in}}%
\pgfpathlineto{\pgfqpoint{0.949551in}{0.416794in}}%
\pgfpathlineto{\pgfqpoint{0.960218in}{0.433861in}}%
\pgfpathlineto{\pgfqpoint{0.961359in}{0.436162in}}%
\pgfpathlineto{\pgfqpoint{0.967777in}{0.431965in}}%
\pgfpathlineto{\pgfqpoint{0.977356in}{0.447349in}}%
\pgfpathlineto{\pgfqpoint{0.979580in}{0.452265in}}%
\pgfpathclose%
\pgfusepath{fill}%
\end{pgfscope}%
\begin{pgfscope}%
\pgfpathrectangle{\pgfqpoint{0.100000in}{0.100000in}}{\pgfqpoint{3.608454in}{2.310000in}}%
\pgfusepath{clip}%
\pgfsetbuttcap%
\pgfsetmiterjoin%
\definecolor{currentfill}{rgb}{0.000000,0.941176,0.529412}%
\pgfsetfillcolor{currentfill}%
\pgfsetlinewidth{0.000000pt}%
\definecolor{currentstroke}{rgb}{0.000000,0.000000,0.000000}%
\pgfsetstrokecolor{currentstroke}%
\pgfsetstrokeopacity{0.000000}%
\pgfsetdash{}{0pt}%
\pgfpathmoveto{\pgfqpoint{2.923822in}{1.280887in}}%
\pgfpathlineto{\pgfqpoint{2.927867in}{1.284414in}}%
\pgfpathlineto{\pgfqpoint{2.929879in}{1.293625in}}%
\pgfpathlineto{\pgfqpoint{2.935674in}{1.297744in}}%
\pgfpathlineto{\pgfqpoint{2.931210in}{1.301107in}}%
\pgfpathlineto{\pgfqpoint{2.946226in}{1.306488in}}%
\pgfpathlineto{\pgfqpoint{2.952046in}{1.304075in}}%
\pgfpathlineto{\pgfqpoint{2.961369in}{1.291611in}}%
\pgfpathlineto{\pgfqpoint{2.967623in}{1.287269in}}%
\pgfpathlineto{\pgfqpoint{2.965505in}{1.286020in}}%
\pgfpathlineto{\pgfqpoint{2.960443in}{1.270838in}}%
\pgfpathlineto{\pgfqpoint{2.950534in}{1.260497in}}%
\pgfpathlineto{\pgfqpoint{2.942904in}{1.258852in}}%
\pgfpathlineto{\pgfqpoint{2.935481in}{1.264222in}}%
\pgfpathlineto{\pgfqpoint{2.926305in}{1.268226in}}%
\pgfpathlineto{\pgfqpoint{2.923526in}{1.275100in}}%
\pgfpathlineto{\pgfqpoint{2.923822in}{1.280887in}}%
\pgfpathclose%
\pgfusepath{fill}%
\end{pgfscope}%
\begin{pgfscope}%
\pgfpathrectangle{\pgfqpoint{0.100000in}{0.100000in}}{\pgfqpoint{3.608454in}{2.310000in}}%
\pgfusepath{clip}%
\pgfsetbuttcap%
\pgfsetmiterjoin%
\definecolor{currentfill}{rgb}{0.000000,0.443137,0.778431}%
\pgfsetfillcolor{currentfill}%
\pgfsetlinewidth{0.000000pt}%
\definecolor{currentstroke}{rgb}{0.000000,0.000000,0.000000}%
\pgfsetstrokecolor{currentstroke}%
\pgfsetstrokeopacity{0.000000}%
\pgfsetdash{}{0pt}%
\pgfpathmoveto{\pgfqpoint{3.019784in}{0.555093in}}%
\pgfpathlineto{\pgfqpoint{3.017489in}{0.571785in}}%
\pgfpathlineto{\pgfqpoint{3.006224in}{0.574274in}}%
\pgfpathlineto{\pgfqpoint{3.001283in}{0.581332in}}%
\pgfpathlineto{\pgfqpoint{3.006411in}{0.592107in}}%
\pgfpathlineto{\pgfqpoint{2.994626in}{0.604123in}}%
\pgfpathlineto{\pgfqpoint{3.039989in}{0.610672in}}%
\pgfpathlineto{\pgfqpoint{3.040352in}{0.617807in}}%
\pgfpathlineto{\pgfqpoint{3.037437in}{0.635958in}}%
\pgfpathlineto{\pgfqpoint{3.040376in}{0.630406in}}%
\pgfpathlineto{\pgfqpoint{3.046213in}{0.629170in}}%
\pgfpathlineto{\pgfqpoint{3.049410in}{0.623089in}}%
\pgfpathlineto{\pgfqpoint{3.060692in}{0.615927in}}%
\pgfpathlineto{\pgfqpoint{3.065010in}{0.604224in}}%
\pgfpathlineto{\pgfqpoint{3.071577in}{0.605034in}}%
\pgfpathlineto{\pgfqpoint{3.078312in}{0.602797in}}%
\pgfpathlineto{\pgfqpoint{3.080990in}{0.605844in}}%
\pgfpathlineto{\pgfqpoint{3.090875in}{0.590177in}}%
\pgfpathlineto{\pgfqpoint{3.099275in}{0.583081in}}%
\pgfpathlineto{\pgfqpoint{3.099105in}{0.576837in}}%
\pgfpathlineto{\pgfqpoint{3.102843in}{0.570453in}}%
\pgfpathlineto{\pgfqpoint{3.110913in}{0.514275in}}%
\pgfpathlineto{\pgfqpoint{3.091930in}{0.511405in}}%
\pgfpathlineto{\pgfqpoint{3.090421in}{0.517236in}}%
\pgfpathlineto{\pgfqpoint{3.084501in}{0.525466in}}%
\pgfpathlineto{\pgfqpoint{3.077818in}{0.527142in}}%
\pgfpathlineto{\pgfqpoint{3.076539in}{0.531713in}}%
\pgfpathlineto{\pgfqpoint{3.070237in}{0.538537in}}%
\pgfpathlineto{\pgfqpoint{3.059031in}{0.547094in}}%
\pgfpathlineto{\pgfqpoint{3.055318in}{0.556018in}}%
\pgfpathlineto{\pgfqpoint{3.048333in}{0.554969in}}%
\pgfpathlineto{\pgfqpoint{3.047294in}{0.561961in}}%
\pgfpathlineto{\pgfqpoint{3.026291in}{0.558732in}}%
\pgfpathlineto{\pgfqpoint{3.019784in}{0.555093in}}%
\pgfpathclose%
\pgfusepath{fill}%
\end{pgfscope}%
\begin{pgfscope}%
\pgfpathrectangle{\pgfqpoint{0.100000in}{0.100000in}}{\pgfqpoint{3.608454in}{2.310000in}}%
\pgfusepath{clip}%
\pgfsetbuttcap%
\pgfsetmiterjoin%
\definecolor{currentfill}{rgb}{0.000000,0.474510,0.762745}%
\pgfsetfillcolor{currentfill}%
\pgfsetlinewidth{0.000000pt}%
\definecolor{currentstroke}{rgb}{0.000000,0.000000,0.000000}%
\pgfsetstrokecolor{currentstroke}%
\pgfsetstrokeopacity{0.000000}%
\pgfsetdash{}{0pt}%
\pgfpathmoveto{\pgfqpoint{1.723873in}{1.194151in}}%
\pgfpathlineto{\pgfqpoint{1.731511in}{1.193664in}}%
\pgfpathlineto{\pgfqpoint{1.728388in}{1.154348in}}%
\pgfpathlineto{\pgfqpoint{1.685892in}{1.157286in}}%
\pgfpathlineto{\pgfqpoint{1.683362in}{1.122232in}}%
\pgfpathlineto{\pgfqpoint{1.648937in}{1.124836in}}%
\pgfpathlineto{\pgfqpoint{1.651744in}{1.159945in}}%
\pgfpathlineto{\pgfqpoint{1.598511in}{1.164402in}}%
\pgfpathlineto{\pgfqpoint{1.602002in}{1.203777in}}%
\pgfpathlineto{\pgfqpoint{1.621102in}{1.201730in}}%
\pgfpathlineto{\pgfqpoint{1.662385in}{1.198210in}}%
\pgfpathlineto{\pgfqpoint{1.664847in}{1.229439in}}%
\pgfpathlineto{\pgfqpoint{1.695230in}{1.227106in}}%
\pgfpathlineto{\pgfqpoint{1.693075in}{1.196095in}}%
\pgfpathlineto{\pgfqpoint{1.723873in}{1.194151in}}%
\pgfpathclose%
\pgfusepath{fill}%
\end{pgfscope}%
\begin{pgfscope}%
\pgfpathrectangle{\pgfqpoint{0.100000in}{0.100000in}}{\pgfqpoint{3.608454in}{2.310000in}}%
\pgfusepath{clip}%
\pgfsetbuttcap%
\pgfsetmiterjoin%
\definecolor{currentfill}{rgb}{0.000000,0.607843,0.696078}%
\pgfsetfillcolor{currentfill}%
\pgfsetlinewidth{0.000000pt}%
\definecolor{currentstroke}{rgb}{0.000000,0.000000,0.000000}%
\pgfsetstrokecolor{currentstroke}%
\pgfsetstrokeopacity{0.000000}%
\pgfsetdash{}{0pt}%
\pgfpathmoveto{\pgfqpoint{2.642473in}{1.232080in}}%
\pgfpathlineto{\pgfqpoint{2.647960in}{1.220595in}}%
\pgfpathlineto{\pgfqpoint{2.639824in}{1.220359in}}%
\pgfpathlineto{\pgfqpoint{2.634641in}{1.217991in}}%
\pgfpathlineto{\pgfqpoint{2.631929in}{1.205576in}}%
\pgfpathlineto{\pgfqpoint{2.620792in}{1.209072in}}%
\pgfpathlineto{\pgfqpoint{2.607679in}{1.208265in}}%
\pgfpathlineto{\pgfqpoint{2.595443in}{1.205574in}}%
\pgfpathlineto{\pgfqpoint{2.589411in}{1.214310in}}%
\pgfpathlineto{\pgfqpoint{2.585378in}{1.221768in}}%
\pgfpathlineto{\pgfqpoint{2.589982in}{1.231634in}}%
\pgfpathlineto{\pgfqpoint{2.586421in}{1.235460in}}%
\pgfpathlineto{\pgfqpoint{2.585128in}{1.245024in}}%
\pgfpathlineto{\pgfqpoint{2.577210in}{1.248460in}}%
\pgfpathlineto{\pgfqpoint{2.585709in}{1.266483in}}%
\pgfpathlineto{\pgfqpoint{2.586412in}{1.272897in}}%
\pgfpathlineto{\pgfqpoint{2.601681in}{1.267043in}}%
\pgfpathlineto{\pgfqpoint{2.602387in}{1.273595in}}%
\pgfpathlineto{\pgfqpoint{2.617290in}{1.282808in}}%
\pgfpathlineto{\pgfqpoint{2.620883in}{1.276667in}}%
\pgfpathlineto{\pgfqpoint{2.627054in}{1.277303in}}%
\pgfpathlineto{\pgfqpoint{2.627023in}{1.272051in}}%
\pgfpathlineto{\pgfqpoint{2.625972in}{1.268011in}}%
\pgfpathlineto{\pgfqpoint{2.630443in}{1.257485in}}%
\pgfpathlineto{\pgfqpoint{2.628815in}{1.247944in}}%
\pgfpathlineto{\pgfqpoint{2.632971in}{1.237093in}}%
\pgfpathlineto{\pgfqpoint{2.641029in}{1.235231in}}%
\pgfpathlineto{\pgfqpoint{2.642473in}{1.232080in}}%
\pgfpathclose%
\pgfusepath{fill}%
\end{pgfscope}%
\begin{pgfscope}%
\pgfpathrectangle{\pgfqpoint{0.100000in}{0.100000in}}{\pgfqpoint{3.608454in}{2.310000in}}%
\pgfusepath{clip}%
\pgfsetbuttcap%
\pgfsetmiterjoin%
\definecolor{currentfill}{rgb}{0.000000,0.415686,0.792157}%
\pgfsetfillcolor{currentfill}%
\pgfsetlinewidth{0.000000pt}%
\definecolor{currentstroke}{rgb}{0.000000,0.000000,0.000000}%
\pgfsetstrokecolor{currentstroke}%
\pgfsetstrokeopacity{0.000000}%
\pgfsetdash{}{0pt}%
\pgfpathmoveto{\pgfqpoint{2.218568in}{1.724511in}}%
\pgfpathlineto{\pgfqpoint{2.238913in}{1.724992in}}%
\pgfpathlineto{\pgfqpoint{2.252724in}{1.724220in}}%
\pgfpathlineto{\pgfqpoint{2.253472in}{1.697776in}}%
\pgfpathlineto{\pgfqpoint{2.220391in}{1.696881in}}%
\pgfpathlineto{\pgfqpoint{2.184501in}{1.696223in}}%
\pgfpathlineto{\pgfqpoint{2.184096in}{1.723863in}}%
\pgfpathlineto{\pgfqpoint{2.218568in}{1.724511in}}%
\pgfpathclose%
\pgfusepath{fill}%
\end{pgfscope}%
\begin{pgfscope}%
\pgfpathrectangle{\pgfqpoint{0.100000in}{0.100000in}}{\pgfqpoint{3.608454in}{2.310000in}}%
\pgfusepath{clip}%
\pgfsetbuttcap%
\pgfsetmiterjoin%
\definecolor{currentfill}{rgb}{0.000000,0.207843,0.896078}%
\pgfsetfillcolor{currentfill}%
\pgfsetlinewidth{0.000000pt}%
\definecolor{currentstroke}{rgb}{0.000000,0.000000,0.000000}%
\pgfsetstrokecolor{currentstroke}%
\pgfsetstrokeopacity{0.000000}%
\pgfsetdash{}{0pt}%
\pgfpathmoveto{\pgfqpoint{1.989587in}{1.638444in}}%
\pgfpathlineto{\pgfqpoint{1.981253in}{1.641675in}}%
\pgfpathlineto{\pgfqpoint{1.981953in}{1.671006in}}%
\pgfpathlineto{\pgfqpoint{1.968146in}{1.671305in}}%
\pgfpathlineto{\pgfqpoint{1.968709in}{1.697472in}}%
\pgfpathlineto{\pgfqpoint{1.956891in}{1.697804in}}%
\pgfpathlineto{\pgfqpoint{1.957610in}{1.725492in}}%
\pgfpathlineto{\pgfqpoint{1.998767in}{1.724416in}}%
\pgfpathlineto{\pgfqpoint{2.023670in}{1.724067in}}%
\pgfpathlineto{\pgfqpoint{2.023239in}{1.696380in}}%
\pgfpathlineto{\pgfqpoint{2.014851in}{1.696524in}}%
\pgfpathlineto{\pgfqpoint{2.015077in}{1.691199in}}%
\pgfpathlineto{\pgfqpoint{2.019001in}{1.688123in}}%
\pgfpathlineto{\pgfqpoint{2.018524in}{1.680571in}}%
\pgfpathlineto{\pgfqpoint{2.021565in}{1.674298in}}%
\pgfpathlineto{\pgfqpoint{2.022652in}{1.663329in}}%
\pgfpathlineto{\pgfqpoint{2.002309in}{1.663696in}}%
\pgfpathlineto{\pgfqpoint{2.001679in}{1.633660in}}%
\pgfpathlineto{\pgfqpoint{1.989587in}{1.638444in}}%
\pgfpathclose%
\pgfusepath{fill}%
\end{pgfscope}%
\begin{pgfscope}%
\pgfpathrectangle{\pgfqpoint{0.100000in}{0.100000in}}{\pgfqpoint{3.608454in}{2.310000in}}%
\pgfusepath{clip}%
\pgfsetbuttcap%
\pgfsetmiterjoin%
\definecolor{currentfill}{rgb}{0.000000,0.490196,0.754902}%
\pgfsetfillcolor{currentfill}%
\pgfsetlinewidth{0.000000pt}%
\definecolor{currentstroke}{rgb}{0.000000,0.000000,0.000000}%
\pgfsetstrokecolor{currentstroke}%
\pgfsetstrokeopacity{0.000000}%
\pgfsetdash{}{0pt}%
\pgfpathmoveto{\pgfqpoint{2.018170in}{1.419481in}}%
\pgfpathlineto{\pgfqpoint{2.031759in}{1.419257in}}%
\pgfpathlineto{\pgfqpoint{2.031304in}{1.384920in}}%
\pgfpathlineto{\pgfqpoint{1.962491in}{1.386340in}}%
\pgfpathlineto{\pgfqpoint{1.963356in}{1.420723in}}%
\pgfpathlineto{\pgfqpoint{2.018170in}{1.419481in}}%
\pgfpathclose%
\pgfusepath{fill}%
\end{pgfscope}%
\begin{pgfscope}%
\pgfpathrectangle{\pgfqpoint{0.100000in}{0.100000in}}{\pgfqpoint{3.608454in}{2.310000in}}%
\pgfusepath{clip}%
\pgfsetbuttcap%
\pgfsetmiterjoin%
\definecolor{currentfill}{rgb}{0.000000,0.592157,0.703922}%
\pgfsetfillcolor{currentfill}%
\pgfsetlinewidth{0.000000pt}%
\definecolor{currentstroke}{rgb}{0.000000,0.000000,0.000000}%
\pgfsetstrokecolor{currentstroke}%
\pgfsetstrokeopacity{0.000000}%
\pgfsetdash{}{0pt}%
\pgfpathmoveto{\pgfqpoint{2.396575in}{1.440884in}}%
\pgfpathlineto{\pgfqpoint{2.381360in}{1.440493in}}%
\pgfpathlineto{\pgfqpoint{2.381085in}{1.447410in}}%
\pgfpathlineto{\pgfqpoint{2.353369in}{1.446719in}}%
\pgfpathlineto{\pgfqpoint{2.352440in}{1.474762in}}%
\pgfpathlineto{\pgfqpoint{2.359331in}{1.474799in}}%
\pgfpathlineto{\pgfqpoint{2.358001in}{1.509016in}}%
\pgfpathlineto{\pgfqpoint{2.378619in}{1.509638in}}%
\pgfpathlineto{\pgfqpoint{2.378376in}{1.516534in}}%
\pgfpathlineto{\pgfqpoint{2.405330in}{1.517808in}}%
\pgfpathlineto{\pgfqpoint{2.406029in}{1.504012in}}%
\pgfpathlineto{\pgfqpoint{2.407342in}{1.476448in}}%
\pgfpathlineto{\pgfqpoint{2.414274in}{1.476742in}}%
\pgfpathlineto{\pgfqpoint{2.414833in}{1.467947in}}%
\pgfpathlineto{\pgfqpoint{2.412968in}{1.462469in}}%
\pgfpathlineto{\pgfqpoint{2.405497in}{1.456494in}}%
\pgfpathlineto{\pgfqpoint{2.401121in}{1.445010in}}%
\pgfpathlineto{\pgfqpoint{2.396575in}{1.440884in}}%
\pgfpathclose%
\pgfusepath{fill}%
\end{pgfscope}%
\begin{pgfscope}%
\pgfpathrectangle{\pgfqpoint{0.100000in}{0.100000in}}{\pgfqpoint{3.608454in}{2.310000in}}%
\pgfusepath{clip}%
\pgfsetbuttcap%
\pgfsetmiterjoin%
\definecolor{currentfill}{rgb}{0.000000,0.572549,0.713725}%
\pgfsetfillcolor{currentfill}%
\pgfsetlinewidth{0.000000pt}%
\definecolor{currentstroke}{rgb}{0.000000,0.000000,0.000000}%
\pgfsetstrokecolor{currentstroke}%
\pgfsetstrokeopacity{0.000000}%
\pgfsetdash{}{0pt}%
\pgfpathmoveto{\pgfqpoint{2.729948in}{1.258824in}}%
\pgfpathlineto{\pgfqpoint{2.730780in}{1.248329in}}%
\pgfpathlineto{\pgfqpoint{2.722412in}{1.251628in}}%
\pgfpathlineto{\pgfqpoint{2.718150in}{1.247622in}}%
\pgfpathlineto{\pgfqpoint{2.711292in}{1.250974in}}%
\pgfpathlineto{\pgfqpoint{2.704025in}{1.249697in}}%
\pgfpathlineto{\pgfqpoint{2.701382in}{1.255633in}}%
\pgfpathlineto{\pgfqpoint{2.694535in}{1.255275in}}%
\pgfpathlineto{\pgfqpoint{2.694291in}{1.266816in}}%
\pgfpathlineto{\pgfqpoint{2.691418in}{1.266749in}}%
\pgfpathlineto{\pgfqpoint{2.684273in}{1.275179in}}%
\pgfpathlineto{\pgfqpoint{2.690035in}{1.277558in}}%
\pgfpathlineto{\pgfqpoint{2.698255in}{1.290856in}}%
\pgfpathlineto{\pgfqpoint{2.704197in}{1.287428in}}%
\pgfpathlineto{\pgfqpoint{2.718323in}{1.291481in}}%
\pgfpathlineto{\pgfqpoint{2.719889in}{1.285770in}}%
\pgfpathlineto{\pgfqpoint{2.727518in}{1.286122in}}%
\pgfpathlineto{\pgfqpoint{2.729699in}{1.283399in}}%
\pgfpathlineto{\pgfqpoint{2.729948in}{1.258824in}}%
\pgfpathclose%
\pgfusepath{fill}%
\end{pgfscope}%
\begin{pgfscope}%
\pgfpathrectangle{\pgfqpoint{0.100000in}{0.100000in}}{\pgfqpoint{3.608454in}{2.310000in}}%
\pgfusepath{clip}%
\pgfsetbuttcap%
\pgfsetmiterjoin%
\definecolor{currentfill}{rgb}{0.000000,0.529412,0.735294}%
\pgfsetfillcolor{currentfill}%
\pgfsetlinewidth{0.000000pt}%
\definecolor{currentstroke}{rgb}{0.000000,0.000000,0.000000}%
\pgfsetstrokecolor{currentstroke}%
\pgfsetstrokeopacity{0.000000}%
\pgfsetdash{}{0pt}%
\pgfpathmoveto{\pgfqpoint{2.868390in}{1.010555in}}%
\pgfpathlineto{\pgfqpoint{2.871596in}{1.014173in}}%
\pgfpathlineto{\pgfqpoint{2.886412in}{1.013136in}}%
\pgfpathlineto{\pgfqpoint{2.885344in}{1.008668in}}%
\pgfpathlineto{\pgfqpoint{2.897463in}{0.997613in}}%
\pgfpathlineto{\pgfqpoint{2.908530in}{0.995600in}}%
\pgfpathlineto{\pgfqpoint{2.902891in}{0.990956in}}%
\pgfpathlineto{\pgfqpoint{2.896953in}{0.978821in}}%
\pgfpathlineto{\pgfqpoint{2.896071in}{0.971539in}}%
\pgfpathlineto{\pgfqpoint{2.889208in}{0.971223in}}%
\pgfpathlineto{\pgfqpoint{2.877953in}{0.974806in}}%
\pgfpathlineto{\pgfqpoint{2.870396in}{0.968762in}}%
\pgfpathlineto{\pgfqpoint{2.864261in}{0.976571in}}%
\pgfpathlineto{\pgfqpoint{2.850509in}{0.983786in}}%
\pgfpathlineto{\pgfqpoint{2.844938in}{0.981568in}}%
\pgfpathlineto{\pgfqpoint{2.837086in}{0.989053in}}%
\pgfpathlineto{\pgfqpoint{2.839244in}{0.999162in}}%
\pgfpathlineto{\pgfqpoint{2.843269in}{0.994686in}}%
\pgfpathlineto{\pgfqpoint{2.856780in}{0.993705in}}%
\pgfpathlineto{\pgfqpoint{2.859121in}{0.988728in}}%
\pgfpathlineto{\pgfqpoint{2.868081in}{0.996615in}}%
\pgfpathlineto{\pgfqpoint{2.869364in}{1.002057in}}%
\pgfpathlineto{\pgfqpoint{2.865578in}{1.008074in}}%
\pgfpathlineto{\pgfqpoint{2.868390in}{1.010555in}}%
\pgfpathclose%
\pgfusepath{fill}%
\end{pgfscope}%
\begin{pgfscope}%
\pgfpathrectangle{\pgfqpoint{0.100000in}{0.100000in}}{\pgfqpoint{3.608454in}{2.310000in}}%
\pgfusepath{clip}%
\pgfsetbuttcap%
\pgfsetmiterjoin%
\definecolor{currentfill}{rgb}{0.000000,0.364706,0.817647}%
\pgfsetfillcolor{currentfill}%
\pgfsetlinewidth{0.000000pt}%
\definecolor{currentstroke}{rgb}{0.000000,0.000000,0.000000}%
\pgfsetstrokecolor{currentstroke}%
\pgfsetstrokeopacity{0.000000}%
\pgfsetdash{}{0pt}%
\pgfpathmoveto{\pgfqpoint{1.858155in}{1.589614in}}%
\pgfpathlineto{\pgfqpoint{1.831612in}{1.590776in}}%
\pgfpathlineto{\pgfqpoint{1.803419in}{1.592288in}}%
\pgfpathlineto{\pgfqpoint{1.804075in}{1.619812in}}%
\pgfpathlineto{\pgfqpoint{1.805638in}{1.647292in}}%
\pgfpathlineto{\pgfqpoint{1.805743in}{1.664546in}}%
\pgfpathlineto{\pgfqpoint{1.860373in}{1.661671in}}%
\pgfpathlineto{\pgfqpoint{1.858155in}{1.589614in}}%
\pgfpathclose%
\pgfusepath{fill}%
\end{pgfscope}%
\begin{pgfscope}%
\pgfpathrectangle{\pgfqpoint{0.100000in}{0.100000in}}{\pgfqpoint{3.608454in}{2.310000in}}%
\pgfusepath{clip}%
\pgfsetbuttcap%
\pgfsetmiterjoin%
\definecolor{currentfill}{rgb}{0.000000,0.325490,0.837255}%
\pgfsetfillcolor{currentfill}%
\pgfsetlinewidth{0.000000pt}%
\definecolor{currentstroke}{rgb}{0.000000,0.000000,0.000000}%
\pgfsetstrokecolor{currentstroke}%
\pgfsetstrokeopacity{0.000000}%
\pgfsetdash{}{0pt}%
\pgfpathmoveto{\pgfqpoint{2.039668in}{1.566696in}}%
\pgfpathlineto{\pgfqpoint{2.043988in}{1.561009in}}%
\pgfpathlineto{\pgfqpoint{2.040513in}{1.552171in}}%
\pgfpathlineto{\pgfqpoint{2.021430in}{1.552509in}}%
\pgfpathlineto{\pgfqpoint{2.021531in}{1.557101in}}%
\pgfpathlineto{\pgfqpoint{1.994270in}{1.557682in}}%
\pgfpathlineto{\pgfqpoint{1.987553in}{1.557852in}}%
\pgfpathlineto{\pgfqpoint{1.988196in}{1.585344in}}%
\pgfpathlineto{\pgfqpoint{1.999669in}{1.585100in}}%
\pgfpathlineto{\pgfqpoint{2.000030in}{1.598859in}}%
\pgfpathlineto{\pgfqpoint{2.005666in}{1.599833in}}%
\pgfpathlineto{\pgfqpoint{2.027350in}{1.599345in}}%
\pgfpathlineto{\pgfqpoint{2.027934in}{1.590280in}}%
\pgfpathlineto{\pgfqpoint{2.036665in}{1.576098in}}%
\pgfpathlineto{\pgfqpoint{2.040288in}{1.575134in}}%
\pgfpathlineto{\pgfqpoint{2.039668in}{1.566696in}}%
\pgfpathclose%
\pgfusepath{fill}%
\end{pgfscope}%
\begin{pgfscope}%
\pgfpathrectangle{\pgfqpoint{0.100000in}{0.100000in}}{\pgfqpoint{3.608454in}{2.310000in}}%
\pgfusepath{clip}%
\pgfsetbuttcap%
\pgfsetmiterjoin%
\definecolor{currentfill}{rgb}{0.000000,0.352941,0.823529}%
\pgfsetfillcolor{currentfill}%
\pgfsetlinewidth{0.000000pt}%
\definecolor{currentstroke}{rgb}{0.000000,0.000000,0.000000}%
\pgfsetstrokecolor{currentstroke}%
\pgfsetstrokeopacity{0.000000}%
\pgfsetdash{}{0pt}%
\pgfpathmoveto{\pgfqpoint{2.258667in}{1.780519in}}%
\pgfpathlineto{\pgfqpoint{2.247155in}{1.782906in}}%
\pgfpathlineto{\pgfqpoint{2.241126in}{1.786250in}}%
\pgfpathlineto{\pgfqpoint{2.231530in}{1.786839in}}%
\pgfpathlineto{\pgfqpoint{2.231724in}{1.779996in}}%
\pgfpathlineto{\pgfqpoint{2.217797in}{1.777469in}}%
\pgfpathlineto{\pgfqpoint{2.217878in}{1.773970in}}%
\pgfpathlineto{\pgfqpoint{2.204133in}{1.773697in}}%
\pgfpathlineto{\pgfqpoint{2.204050in}{1.779422in}}%
\pgfpathlineto{\pgfqpoint{2.190305in}{1.779149in}}%
\pgfpathlineto{\pgfqpoint{2.168433in}{1.778817in}}%
\pgfpathlineto{\pgfqpoint{2.169520in}{1.792650in}}%
\pgfpathlineto{\pgfqpoint{2.162655in}{1.792587in}}%
\pgfpathlineto{\pgfqpoint{2.162319in}{1.813349in}}%
\pgfpathlineto{\pgfqpoint{2.148622in}{1.813278in}}%
\pgfpathlineto{\pgfqpoint{2.148148in}{1.837483in}}%
\pgfpathlineto{\pgfqpoint{2.156256in}{1.841131in}}%
\pgfpathlineto{\pgfqpoint{2.160005in}{1.848588in}}%
\pgfpathlineto{\pgfqpoint{2.176492in}{1.838639in}}%
\pgfpathlineto{\pgfqpoint{2.186149in}{1.838564in}}%
\pgfpathlineto{\pgfqpoint{2.190082in}{1.834950in}}%
\pgfpathlineto{\pgfqpoint{2.189353in}{1.873824in}}%
\pgfpathlineto{\pgfqpoint{2.226726in}{1.874225in}}%
\pgfpathlineto{\pgfqpoint{2.224641in}{1.861874in}}%
\pgfpathlineto{\pgfqpoint{2.230979in}{1.861346in}}%
\pgfpathlineto{\pgfqpoint{2.238078in}{1.851709in}}%
\pgfpathlineto{\pgfqpoint{2.237996in}{1.848119in}}%
\pgfpathlineto{\pgfqpoint{2.232161in}{1.839416in}}%
\pgfpathlineto{\pgfqpoint{2.232360in}{1.832947in}}%
\pgfpathlineto{\pgfqpoint{2.266038in}{1.833884in}}%
\pgfpathlineto{\pgfqpoint{2.267401in}{1.826914in}}%
\pgfpathlineto{\pgfqpoint{2.268903in}{1.780718in}}%
\pgfpathlineto{\pgfqpoint{2.258667in}{1.780519in}}%
\pgfpathclose%
\pgfusepath{fill}%
\end{pgfscope}%
\begin{pgfscope}%
\pgfpathrectangle{\pgfqpoint{0.100000in}{0.100000in}}{\pgfqpoint{3.608454in}{2.310000in}}%
\pgfusepath{clip}%
\pgfsetbuttcap%
\pgfsetmiterjoin%
\definecolor{currentfill}{rgb}{0.000000,0.482353,0.758824}%
\pgfsetfillcolor{currentfill}%
\pgfsetlinewidth{0.000000pt}%
\definecolor{currentstroke}{rgb}{0.000000,0.000000,0.000000}%
\pgfsetstrokecolor{currentstroke}%
\pgfsetstrokeopacity{0.000000}%
\pgfsetdash{}{0pt}%
\pgfpathmoveto{\pgfqpoint{2.424922in}{0.713419in}}%
\pgfpathlineto{\pgfqpoint{2.432451in}{0.691509in}}%
\pgfpathlineto{\pgfqpoint{2.435277in}{0.651886in}}%
\pgfpathlineto{\pgfqpoint{2.431222in}{0.657222in}}%
\pgfpathlineto{\pgfqpoint{2.419484in}{0.654679in}}%
\pgfpathlineto{\pgfqpoint{2.415540in}{0.649843in}}%
\pgfpathlineto{\pgfqpoint{2.407987in}{0.645878in}}%
\pgfpathlineto{\pgfqpoint{2.389280in}{0.640751in}}%
\pgfpathlineto{\pgfqpoint{2.386115in}{0.636858in}}%
\pgfpathlineto{\pgfqpoint{2.383161in}{0.637744in}}%
\pgfpathlineto{\pgfqpoint{2.368222in}{0.632799in}}%
\pgfpathlineto{\pgfqpoint{2.351410in}{0.638299in}}%
\pgfpathlineto{\pgfqpoint{2.350223in}{0.648417in}}%
\pgfpathlineto{\pgfqpoint{2.344148in}{0.649181in}}%
\pgfpathlineto{\pgfqpoint{2.339922in}{0.654462in}}%
\pgfpathlineto{\pgfqpoint{2.338216in}{0.665019in}}%
\pgfpathlineto{\pgfqpoint{2.331431in}{0.670010in}}%
\pgfpathlineto{\pgfqpoint{2.329801in}{0.674430in}}%
\pgfpathlineto{\pgfqpoint{2.331030in}{0.685463in}}%
\pgfpathlineto{\pgfqpoint{2.325022in}{0.696944in}}%
\pgfpathlineto{\pgfqpoint{2.325328in}{0.707042in}}%
\pgfpathlineto{\pgfqpoint{2.330722in}{0.712813in}}%
\pgfpathlineto{\pgfqpoint{2.336789in}{0.709454in}}%
\pgfpathlineto{\pgfqpoint{2.392212in}{0.711750in}}%
\pgfpathlineto{\pgfqpoint{2.424922in}{0.713419in}}%
\pgfpathclose%
\pgfusepath{fill}%
\end{pgfscope}%
\begin{pgfscope}%
\pgfpathrectangle{\pgfqpoint{0.100000in}{0.100000in}}{\pgfqpoint{3.608454in}{2.310000in}}%
\pgfusepath{clip}%
\pgfsetbuttcap%
\pgfsetmiterjoin%
\definecolor{currentfill}{rgb}{0.000000,0.592157,0.703922}%
\pgfsetfillcolor{currentfill}%
\pgfsetlinewidth{0.000000pt}%
\definecolor{currentstroke}{rgb}{0.000000,0.000000,0.000000}%
\pgfsetstrokecolor{currentstroke}%
\pgfsetstrokeopacity{0.000000}%
\pgfsetdash{}{0pt}%
\pgfpathmoveto{\pgfqpoint{1.895416in}{1.185262in}}%
\pgfpathlineto{\pgfqpoint{1.854168in}{1.187060in}}%
\pgfpathlineto{\pgfqpoint{1.855479in}{1.217455in}}%
\pgfpathlineto{\pgfqpoint{1.821457in}{1.218805in}}%
\pgfpathlineto{\pgfqpoint{1.821831in}{1.246792in}}%
\pgfpathlineto{\pgfqpoint{1.855927in}{1.245032in}}%
\pgfpathlineto{\pgfqpoint{1.856262in}{1.252307in}}%
\pgfpathlineto{\pgfqpoint{1.890012in}{1.250882in}}%
\pgfpathlineto{\pgfqpoint{1.890205in}{1.243605in}}%
\pgfpathlineto{\pgfqpoint{1.889149in}{1.216054in}}%
\pgfpathlineto{\pgfqpoint{1.896376in}{1.215785in}}%
\pgfpathlineto{\pgfqpoint{1.895416in}{1.185262in}}%
\pgfpathclose%
\pgfusepath{fill}%
\end{pgfscope}%
\begin{pgfscope}%
\pgfpathrectangle{\pgfqpoint{0.100000in}{0.100000in}}{\pgfqpoint{3.608454in}{2.310000in}}%
\pgfusepath{clip}%
\pgfsetbuttcap%
\pgfsetmiterjoin%
\definecolor{currentfill}{rgb}{0.000000,0.427451,0.786275}%
\pgfsetfillcolor{currentfill}%
\pgfsetlinewidth{0.000000pt}%
\definecolor{currentstroke}{rgb}{0.000000,0.000000,0.000000}%
\pgfsetstrokecolor{currentstroke}%
\pgfsetstrokeopacity{0.000000}%
\pgfsetdash{}{0pt}%
\pgfpathmoveto{\pgfqpoint{2.499443in}{1.713874in}}%
\pgfpathlineto{\pgfqpoint{2.485729in}{1.712888in}}%
\pgfpathlineto{\pgfqpoint{2.485186in}{1.719799in}}%
\pgfpathlineto{\pgfqpoint{2.450492in}{1.717550in}}%
\pgfpathlineto{\pgfqpoint{2.436777in}{1.717456in}}%
\pgfpathlineto{\pgfqpoint{2.436156in}{1.727226in}}%
\pgfpathlineto{\pgfqpoint{2.440017in}{1.736230in}}%
\pgfpathlineto{\pgfqpoint{2.439466in}{1.744650in}}%
\pgfpathlineto{\pgfqpoint{2.414986in}{1.743088in}}%
\pgfpathlineto{\pgfqpoint{2.413744in}{1.763959in}}%
\pgfpathlineto{\pgfqpoint{2.434927in}{1.765090in}}%
\pgfpathlineto{\pgfqpoint{2.432784in}{1.799789in}}%
\pgfpathlineto{\pgfqpoint{2.467658in}{1.801883in}}%
\pgfpathlineto{\pgfqpoint{2.468096in}{1.794922in}}%
\pgfpathlineto{\pgfqpoint{2.488486in}{1.795943in}}%
\pgfpathlineto{\pgfqpoint{2.491553in}{1.796332in}}%
\pgfpathlineto{\pgfqpoint{2.493489in}{1.769026in}}%
\pgfpathlineto{\pgfqpoint{2.501961in}{1.769586in}}%
\pgfpathlineto{\pgfqpoint{2.504166in}{1.741961in}}%
\pgfpathlineto{\pgfqpoint{2.497309in}{1.741424in}}%
\pgfpathlineto{\pgfqpoint{2.499443in}{1.713874in}}%
\pgfpathclose%
\pgfusepath{fill}%
\end{pgfscope}%
\begin{pgfscope}%
\pgfpathrectangle{\pgfqpoint{0.100000in}{0.100000in}}{\pgfqpoint{3.608454in}{2.310000in}}%
\pgfusepath{clip}%
\pgfsetbuttcap%
\pgfsetmiterjoin%
\definecolor{currentfill}{rgb}{0.000000,0.364706,0.817647}%
\pgfsetfillcolor{currentfill}%
\pgfsetlinewidth{0.000000pt}%
\definecolor{currentstroke}{rgb}{0.000000,0.000000,0.000000}%
\pgfsetstrokecolor{currentstroke}%
\pgfsetstrokeopacity{0.000000}%
\pgfsetdash{}{0pt}%
\pgfpathmoveto{\pgfqpoint{3.147047in}{0.962987in}}%
\pgfpathlineto{\pgfqpoint{3.144033in}{0.954523in}}%
\pgfpathlineto{\pgfqpoint{3.134057in}{0.954802in}}%
\pgfpathlineto{\pgfqpoint{3.127866in}{0.947248in}}%
\pgfpathlineto{\pgfqpoint{3.131225in}{0.944243in}}%
\pgfpathlineto{\pgfqpoint{3.123999in}{0.938426in}}%
\pgfpathlineto{\pgfqpoint{3.122700in}{0.934730in}}%
\pgfpathlineto{\pgfqpoint{3.114162in}{0.927752in}}%
\pgfpathlineto{\pgfqpoint{3.113774in}{0.923333in}}%
\pgfpathlineto{\pgfqpoint{3.106625in}{0.916011in}}%
\pgfpathlineto{\pgfqpoint{3.099237in}{0.913335in}}%
\pgfpathlineto{\pgfqpoint{3.086793in}{0.902214in}}%
\pgfpathlineto{\pgfqpoint{3.077208in}{0.901258in}}%
\pgfpathlineto{\pgfqpoint{3.071225in}{0.906814in}}%
\pgfpathlineto{\pgfqpoint{3.066440in}{0.907291in}}%
\pgfpathlineto{\pgfqpoint{3.061084in}{0.913825in}}%
\pgfpathlineto{\pgfqpoint{3.051321in}{0.914811in}}%
\pgfpathlineto{\pgfqpoint{3.044811in}{0.924182in}}%
\pgfpathlineto{\pgfqpoint{3.038279in}{0.928637in}}%
\pgfpathlineto{\pgfqpoint{3.030466in}{0.937480in}}%
\pgfpathlineto{\pgfqpoint{3.047350in}{0.952216in}}%
\pgfpathlineto{\pgfqpoint{3.064743in}{0.967659in}}%
\pgfpathlineto{\pgfqpoint{3.070676in}{0.962442in}}%
\pgfpathlineto{\pgfqpoint{3.081498in}{0.967552in}}%
\pgfpathlineto{\pgfqpoint{3.081713in}{0.979204in}}%
\pgfpathlineto{\pgfqpoint{3.088906in}{0.984668in}}%
\pgfpathlineto{\pgfqpoint{3.097264in}{0.986382in}}%
\pgfpathlineto{\pgfqpoint{3.115214in}{0.974844in}}%
\pgfpathlineto{\pgfqpoint{3.134423in}{0.971285in}}%
\pgfpathlineto{\pgfqpoint{3.147047in}{0.962987in}}%
\pgfpathclose%
\pgfusepath{fill}%
\end{pgfscope}%
\begin{pgfscope}%
\pgfpathrectangle{\pgfqpoint{0.100000in}{0.100000in}}{\pgfqpoint{3.608454in}{2.310000in}}%
\pgfusepath{clip}%
\pgfsetbuttcap%
\pgfsetmiterjoin%
\definecolor{currentfill}{rgb}{0.000000,0.356863,0.821569}%
\pgfsetfillcolor{currentfill}%
\pgfsetlinewidth{0.000000pt}%
\definecolor{currentstroke}{rgb}{0.000000,0.000000,0.000000}%
\pgfsetstrokecolor{currentstroke}%
\pgfsetstrokeopacity{0.000000}%
\pgfsetdash{}{0pt}%
\pgfpathmoveto{\pgfqpoint{3.191049in}{1.170900in}}%
\pgfpathlineto{\pgfqpoint{3.187783in}{1.156766in}}%
\pgfpathlineto{\pgfqpoint{3.188860in}{1.151551in}}%
\pgfpathlineto{\pgfqpoint{3.180850in}{1.147206in}}%
\pgfpathlineto{\pgfqpoint{3.173892in}{1.141190in}}%
\pgfpathlineto{\pgfqpoint{3.168634in}{1.140787in}}%
\pgfpathlineto{\pgfqpoint{3.164716in}{1.144245in}}%
\pgfpathlineto{\pgfqpoint{3.160885in}{1.137946in}}%
\pgfpathlineto{\pgfqpoint{3.149312in}{1.137485in}}%
\pgfpathlineto{\pgfqpoint{3.131128in}{1.128295in}}%
\pgfpathlineto{\pgfqpoint{3.121994in}{1.132796in}}%
\pgfpathlineto{\pgfqpoint{3.110689in}{1.147372in}}%
\pgfpathlineto{\pgfqpoint{3.110510in}{1.150879in}}%
\pgfpathlineto{\pgfqpoint{3.097477in}{1.148400in}}%
\pgfpathlineto{\pgfqpoint{3.093928in}{1.174159in}}%
\pgfpathlineto{\pgfqpoint{3.113209in}{1.177564in}}%
\pgfpathlineto{\pgfqpoint{3.110627in}{1.182204in}}%
\pgfpathlineto{\pgfqpoint{3.106519in}{1.208514in}}%
\pgfpathlineto{\pgfqpoint{3.113032in}{1.209517in}}%
\pgfpathlineto{\pgfqpoint{3.109908in}{1.233039in}}%
\pgfpathlineto{\pgfqpoint{3.152496in}{1.240772in}}%
\pgfpathlineto{\pgfqpoint{3.150361in}{1.233286in}}%
\pgfpathlineto{\pgfqpoint{3.154402in}{1.212187in}}%
\pgfpathlineto{\pgfqpoint{3.160658in}{1.212280in}}%
\pgfpathlineto{\pgfqpoint{3.165876in}{1.221007in}}%
\pgfpathlineto{\pgfqpoint{3.177079in}{1.221508in}}%
\pgfpathlineto{\pgfqpoint{3.185638in}{1.219606in}}%
\pgfpathlineto{\pgfqpoint{3.175634in}{1.186715in}}%
\pgfpathlineto{\pgfqpoint{3.181671in}{1.180528in}}%
\pgfpathlineto{\pgfqpoint{3.186933in}{1.171546in}}%
\pgfpathlineto{\pgfqpoint{3.191049in}{1.170900in}}%
\pgfpathclose%
\pgfusepath{fill}%
\end{pgfscope}%
\begin{pgfscope}%
\pgfpathrectangle{\pgfqpoint{0.100000in}{0.100000in}}{\pgfqpoint{3.608454in}{2.310000in}}%
\pgfusepath{clip}%
\pgfsetbuttcap%
\pgfsetmiterjoin%
\definecolor{currentfill}{rgb}{0.000000,0.505882,0.747059}%
\pgfsetfillcolor{currentfill}%
\pgfsetlinewidth{0.000000pt}%
\definecolor{currentstroke}{rgb}{0.000000,0.000000,0.000000}%
\pgfsetstrokecolor{currentstroke}%
\pgfsetstrokeopacity{0.000000}%
\pgfsetdash{}{0pt}%
\pgfpathmoveto{\pgfqpoint{1.881222in}{1.423490in}}%
\pgfpathlineto{\pgfqpoint{1.853837in}{1.424663in}}%
\pgfpathlineto{\pgfqpoint{1.856208in}{1.476676in}}%
\pgfpathlineto{\pgfqpoint{1.841990in}{1.478161in}}%
\pgfpathlineto{\pgfqpoint{1.842917in}{1.507856in}}%
\pgfpathlineto{\pgfqpoint{1.855925in}{1.507277in}}%
\pgfpathlineto{\pgfqpoint{1.857457in}{1.534695in}}%
\pgfpathlineto{\pgfqpoint{1.884664in}{1.533508in}}%
\pgfpathlineto{\pgfqpoint{1.883282in}{1.506053in}}%
\pgfpathlineto{\pgfqpoint{1.884864in}{1.505981in}}%
\pgfpathlineto{\pgfqpoint{1.883772in}{1.478524in}}%
\pgfpathlineto{\pgfqpoint{1.881222in}{1.423490in}}%
\pgfpathclose%
\pgfusepath{fill}%
\end{pgfscope}%
\begin{pgfscope}%
\pgfpathrectangle{\pgfqpoint{0.100000in}{0.100000in}}{\pgfqpoint{3.608454in}{2.310000in}}%
\pgfusepath{clip}%
\pgfsetbuttcap%
\pgfsetmiterjoin%
\definecolor{currentfill}{rgb}{0.000000,0.423529,0.788235}%
\pgfsetfillcolor{currentfill}%
\pgfsetlinewidth{0.000000pt}%
\definecolor{currentstroke}{rgb}{0.000000,0.000000,0.000000}%
\pgfsetstrokecolor{currentstroke}%
\pgfsetstrokeopacity{0.000000}%
\pgfsetdash{}{0pt}%
\pgfpathmoveto{\pgfqpoint{0.836288in}{2.219545in}}%
\pgfpathlineto{\pgfqpoint{0.829989in}{2.221231in}}%
\pgfpathlineto{\pgfqpoint{0.822836in}{2.217028in}}%
\pgfpathlineto{\pgfqpoint{0.820986in}{2.210320in}}%
\pgfpathlineto{\pgfqpoint{0.814986in}{2.205959in}}%
\pgfpathlineto{\pgfqpoint{0.810303in}{2.197629in}}%
\pgfpathlineto{\pgfqpoint{0.807006in}{2.198546in}}%
\pgfpathlineto{\pgfqpoint{0.798560in}{2.193839in}}%
\pgfpathlineto{\pgfqpoint{0.796683in}{2.187200in}}%
\pgfpathlineto{\pgfqpoint{0.778779in}{2.191862in}}%
\pgfpathlineto{\pgfqpoint{0.774962in}{2.186885in}}%
\pgfpathlineto{\pgfqpoint{0.769524in}{2.184730in}}%
\pgfpathlineto{\pgfqpoint{0.767233in}{2.177075in}}%
\pgfpathlineto{\pgfqpoint{0.763587in}{2.181602in}}%
\pgfpathlineto{\pgfqpoint{0.748201in}{2.185881in}}%
\pgfpathlineto{\pgfqpoint{0.742925in}{2.192347in}}%
\pgfpathlineto{\pgfqpoint{0.735660in}{2.196356in}}%
\pgfpathlineto{\pgfqpoint{0.733242in}{2.201231in}}%
\pgfpathlineto{\pgfqpoint{0.724213in}{2.207197in}}%
\pgfpathlineto{\pgfqpoint{0.722390in}{2.214825in}}%
\pgfpathlineto{\pgfqpoint{0.718104in}{2.222638in}}%
\pgfpathlineto{\pgfqpoint{0.722977in}{2.230297in}}%
\pgfpathlineto{\pgfqpoint{0.721489in}{2.246797in}}%
\pgfpathlineto{\pgfqpoint{0.727318in}{2.257197in}}%
\pgfpathlineto{\pgfqpoint{0.738201in}{2.260046in}}%
\pgfpathlineto{\pgfqpoint{0.741543in}{2.265250in}}%
\pgfpathlineto{\pgfqpoint{0.736966in}{2.278404in}}%
\pgfpathlineto{\pgfqpoint{0.740255in}{2.288121in}}%
\pgfpathlineto{\pgfqpoint{0.752759in}{2.291869in}}%
\pgfpathlineto{\pgfqpoint{0.760304in}{2.288239in}}%
\pgfpathlineto{\pgfqpoint{0.760184in}{2.281456in}}%
\pgfpathlineto{\pgfqpoint{0.765081in}{2.268518in}}%
\pgfpathlineto{\pgfqpoint{0.771087in}{2.257398in}}%
\pgfpathlineto{\pgfqpoint{0.769468in}{2.254538in}}%
\pgfpathlineto{\pgfqpoint{0.780160in}{2.239854in}}%
\pgfpathlineto{\pgfqpoint{0.780931in}{2.234819in}}%
\pgfpathlineto{\pgfqpoint{0.790372in}{2.232053in}}%
\pgfpathlineto{\pgfqpoint{0.791522in}{2.239767in}}%
\pgfpathlineto{\pgfqpoint{0.800018in}{2.241064in}}%
\pgfpathlineto{\pgfqpoint{0.802328in}{2.232125in}}%
\pgfpathlineto{\pgfqpoint{0.815034in}{2.234984in}}%
\pgfpathlineto{\pgfqpoint{0.819597in}{2.231350in}}%
\pgfpathlineto{\pgfqpoint{0.832614in}{2.236217in}}%
\pgfpathlineto{\pgfqpoint{0.836249in}{2.233779in}}%
\pgfpathlineto{\pgfqpoint{0.836288in}{2.219545in}}%
\pgfpathclose%
\pgfusepath{fill}%
\end{pgfscope}%
\begin{pgfscope}%
\pgfpathrectangle{\pgfqpoint{0.100000in}{0.100000in}}{\pgfqpoint{3.608454in}{2.310000in}}%
\pgfusepath{clip}%
\pgfsetbuttcap%
\pgfsetmiterjoin%
\definecolor{currentfill}{rgb}{0.000000,0.631373,0.684314}%
\pgfsetfillcolor{currentfill}%
\pgfsetlinewidth{0.000000pt}%
\definecolor{currentstroke}{rgb}{0.000000,0.000000,0.000000}%
\pgfsetstrokecolor{currentstroke}%
\pgfsetstrokeopacity{0.000000}%
\pgfsetdash{}{0pt}%
\pgfpathmoveto{\pgfqpoint{2.485040in}{0.651337in}}%
\pgfpathlineto{\pgfqpoint{2.478549in}{0.654208in}}%
\pgfpathlineto{\pgfqpoint{2.472822in}{0.672176in}}%
\pgfpathlineto{\pgfqpoint{2.466743in}{0.676753in}}%
\pgfpathlineto{\pgfqpoint{2.460731in}{0.688378in}}%
\pgfpathlineto{\pgfqpoint{2.461305in}{0.698015in}}%
\pgfpathlineto{\pgfqpoint{2.465118in}{0.704833in}}%
\pgfpathlineto{\pgfqpoint{2.467199in}{0.715885in}}%
\pgfpathlineto{\pgfqpoint{2.472257in}{0.716197in}}%
\pgfpathlineto{\pgfqpoint{2.482623in}{0.717728in}}%
\pgfpathlineto{\pgfqpoint{2.493658in}{0.717377in}}%
\pgfpathlineto{\pgfqpoint{2.494101in}{0.710117in}}%
\pgfpathlineto{\pgfqpoint{2.528702in}{0.712404in}}%
\pgfpathlineto{\pgfqpoint{2.525381in}{0.754028in}}%
\pgfpathlineto{\pgfqpoint{2.551986in}{0.756063in}}%
\pgfpathlineto{\pgfqpoint{2.556138in}{0.721372in}}%
\pgfpathlineto{\pgfqpoint{2.561760in}{0.672494in}}%
\pgfpathlineto{\pgfqpoint{2.556857in}{0.667105in}}%
\pgfpathlineto{\pgfqpoint{2.547096in}{0.670673in}}%
\pgfpathlineto{\pgfqpoint{2.541328in}{0.668013in}}%
\pgfpathlineto{\pgfqpoint{2.525544in}{0.670845in}}%
\pgfpathlineto{\pgfqpoint{2.515743in}{0.668541in}}%
\pgfpathlineto{\pgfqpoint{2.497858in}{0.661623in}}%
\pgfpathlineto{\pgfqpoint{2.485040in}{0.651337in}}%
\pgfpathclose%
\pgfusepath{fill}%
\end{pgfscope}%
\begin{pgfscope}%
\pgfpathrectangle{\pgfqpoint{0.100000in}{0.100000in}}{\pgfqpoint{3.608454in}{2.310000in}}%
\pgfusepath{clip}%
\pgfsetbuttcap%
\pgfsetmiterjoin%
\definecolor{currentfill}{rgb}{0.000000,0.690196,0.654902}%
\pgfsetfillcolor{currentfill}%
\pgfsetlinewidth{0.000000pt}%
\definecolor{currentstroke}{rgb}{0.000000,0.000000,0.000000}%
\pgfsetstrokecolor{currentstroke}%
\pgfsetstrokeopacity{0.000000}%
\pgfsetdash{}{0pt}%
\pgfpathmoveto{\pgfqpoint{3.568956in}{1.730471in}}%
\pgfpathlineto{\pgfqpoint{3.576435in}{1.723494in}}%
\pgfpathlineto{\pgfqpoint{3.574246in}{1.719609in}}%
\pgfpathlineto{\pgfqpoint{3.565620in}{1.718239in}}%
\pgfpathlineto{\pgfqpoint{3.570850in}{1.726339in}}%
\pgfpathlineto{\pgfqpoint{3.568956in}{1.730471in}}%
\pgfpathclose%
\pgfusepath{fill}%
\end{pgfscope}%
\begin{pgfscope}%
\pgfpathrectangle{\pgfqpoint{0.100000in}{0.100000in}}{\pgfqpoint{3.608454in}{2.310000in}}%
\pgfusepath{clip}%
\pgfsetbuttcap%
\pgfsetmiterjoin%
\definecolor{currentfill}{rgb}{0.000000,0.525490,0.737255}%
\pgfsetfillcolor{currentfill}%
\pgfsetlinewidth{0.000000pt}%
\definecolor{currentstroke}{rgb}{0.000000,0.000000,0.000000}%
\pgfsetstrokecolor{currentstroke}%
\pgfsetstrokeopacity{0.000000}%
\pgfsetdash{}{0pt}%
\pgfpathmoveto{\pgfqpoint{3.024683in}{1.232048in}}%
\pgfpathlineto{\pgfqpoint{3.011272in}{1.247768in}}%
\pgfpathlineto{\pgfqpoint{3.005275in}{1.242589in}}%
\pgfpathlineto{\pgfqpoint{2.993913in}{1.252683in}}%
\pgfpathlineto{\pgfqpoint{2.990512in}{1.259171in}}%
\pgfpathlineto{\pgfqpoint{2.994849in}{1.262443in}}%
\pgfpathlineto{\pgfqpoint{2.983343in}{1.270885in}}%
\pgfpathlineto{\pgfqpoint{2.992726in}{1.278073in}}%
\pgfpathlineto{\pgfqpoint{2.991086in}{1.284432in}}%
\pgfpathlineto{\pgfqpoint{3.000083in}{1.309688in}}%
\pgfpathlineto{\pgfqpoint{3.013170in}{1.308307in}}%
\pgfpathlineto{\pgfqpoint{3.022678in}{1.310484in}}%
\pgfpathlineto{\pgfqpoint{3.033267in}{1.304579in}}%
\pgfpathlineto{\pgfqpoint{3.045914in}{1.315133in}}%
\pgfpathlineto{\pgfqpoint{3.050832in}{1.323908in}}%
\pgfpathlineto{\pgfqpoint{3.060124in}{1.322440in}}%
\pgfpathlineto{\pgfqpoint{3.060878in}{1.314437in}}%
\pgfpathlineto{\pgfqpoint{3.074021in}{1.306357in}}%
\pgfpathlineto{\pgfqpoint{3.067687in}{1.299266in}}%
\pgfpathlineto{\pgfqpoint{3.064782in}{1.300929in}}%
\pgfpathlineto{\pgfqpoint{3.057612in}{1.293606in}}%
\pgfpathlineto{\pgfqpoint{3.055501in}{1.285981in}}%
\pgfpathlineto{\pgfqpoint{3.057510in}{1.278132in}}%
\pgfpathlineto{\pgfqpoint{3.069985in}{1.273083in}}%
\pgfpathlineto{\pgfqpoint{3.074664in}{1.267099in}}%
\pgfpathlineto{\pgfqpoint{3.074241in}{1.252094in}}%
\pgfpathlineto{\pgfqpoint{3.067580in}{1.245522in}}%
\pgfpathlineto{\pgfqpoint{3.036908in}{1.247132in}}%
\pgfpathlineto{\pgfqpoint{3.032148in}{1.244073in}}%
\pgfpathlineto{\pgfqpoint{3.024683in}{1.232048in}}%
\pgfpathclose%
\pgfusepath{fill}%
\end{pgfscope}%
\begin{pgfscope}%
\pgfpathrectangle{\pgfqpoint{0.100000in}{0.100000in}}{\pgfqpoint{3.608454in}{2.310000in}}%
\pgfusepath{clip}%
\pgfsetbuttcap%
\pgfsetmiterjoin%
\definecolor{currentfill}{rgb}{0.000000,0.560784,0.719608}%
\pgfsetfillcolor{currentfill}%
\pgfsetlinewidth{0.000000pt}%
\definecolor{currentstroke}{rgb}{0.000000,0.000000,0.000000}%
\pgfsetstrokecolor{currentstroke}%
\pgfsetstrokeopacity{0.000000}%
\pgfsetdash{}{0pt}%
\pgfpathmoveto{\pgfqpoint{1.267460in}{1.535366in}}%
\pgfpathlineto{\pgfqpoint{1.261144in}{1.538773in}}%
\pgfpathlineto{\pgfqpoint{1.262249in}{1.546769in}}%
\pgfpathlineto{\pgfqpoint{1.260798in}{1.552822in}}%
\pgfpathlineto{\pgfqpoint{1.249496in}{1.554569in}}%
\pgfpathlineto{\pgfqpoint{1.248664in}{1.549321in}}%
\pgfpathlineto{\pgfqpoint{1.242723in}{1.546157in}}%
\pgfpathlineto{\pgfqpoint{1.241154in}{1.553467in}}%
\pgfpathlineto{\pgfqpoint{1.220332in}{1.551758in}}%
\pgfpathlineto{\pgfqpoint{1.213136in}{1.556950in}}%
\pgfpathlineto{\pgfqpoint{1.215522in}{1.571272in}}%
\pgfpathlineto{\pgfqpoint{1.186062in}{1.576054in}}%
\pgfpathlineto{\pgfqpoint{1.154007in}{1.581932in}}%
\pgfpathlineto{\pgfqpoint{1.157543in}{1.601688in}}%
\pgfpathlineto{\pgfqpoint{1.146992in}{1.596437in}}%
\pgfpathlineto{\pgfqpoint{1.143956in}{1.603549in}}%
\pgfpathlineto{\pgfqpoint{1.144102in}{1.610430in}}%
\pgfpathlineto{\pgfqpoint{1.136553in}{1.614347in}}%
\pgfpathlineto{\pgfqpoint{1.132825in}{1.619941in}}%
\pgfpathlineto{\pgfqpoint{1.138324in}{1.627635in}}%
\pgfpathlineto{\pgfqpoint{1.142480in}{1.636751in}}%
\pgfpathlineto{\pgfqpoint{1.141060in}{1.650733in}}%
\pgfpathlineto{\pgfqpoint{1.139137in}{1.653332in}}%
\pgfpathlineto{\pgfqpoint{1.142493in}{1.660281in}}%
\pgfpathlineto{\pgfqpoint{1.141273in}{1.664762in}}%
\pgfpathlineto{\pgfqpoint{1.167956in}{1.660095in}}%
\pgfpathlineto{\pgfqpoint{1.182268in}{1.739464in}}%
\pgfpathlineto{\pgfqpoint{1.186356in}{1.762535in}}%
\pgfpathlineto{\pgfqpoint{1.199226in}{1.758954in}}%
\pgfpathlineto{\pgfqpoint{1.198346in}{1.753929in}}%
\pgfpathlineto{\pgfqpoint{1.211732in}{1.751597in}}%
\pgfpathlineto{\pgfqpoint{1.212511in}{1.756000in}}%
\pgfpathlineto{\pgfqpoint{1.225832in}{1.753673in}}%
\pgfpathlineto{\pgfqpoint{1.226972in}{1.760454in}}%
\pgfpathlineto{\pgfqpoint{1.243269in}{1.757690in}}%
\pgfpathlineto{\pgfqpoint{1.244733in}{1.764439in}}%
\pgfpathlineto{\pgfqpoint{1.261563in}{1.761526in}}%
\pgfpathlineto{\pgfqpoint{1.263755in}{1.753309in}}%
\pgfpathlineto{\pgfqpoint{1.262997in}{1.742086in}}%
\pgfpathlineto{\pgfqpoint{1.264849in}{1.733953in}}%
\pgfpathlineto{\pgfqpoint{1.270191in}{1.723778in}}%
\pgfpathlineto{\pgfqpoint{1.275945in}{1.717107in}}%
\pgfpathlineto{\pgfqpoint{1.282465in}{1.703069in}}%
\pgfpathlineto{\pgfqpoint{1.290897in}{1.695072in}}%
\pgfpathlineto{\pgfqpoint{1.287915in}{1.674697in}}%
\pgfpathlineto{\pgfqpoint{1.287325in}{1.661183in}}%
\pgfpathlineto{\pgfqpoint{1.347746in}{1.651996in}}%
\pgfpathlineto{\pgfqpoint{1.375467in}{1.648053in}}%
\pgfpathlineto{\pgfqpoint{1.374676in}{1.634274in}}%
\pgfpathlineto{\pgfqpoint{1.369658in}{1.600576in}}%
\pgfpathlineto{\pgfqpoint{1.345097in}{1.604208in}}%
\pgfpathlineto{\pgfqpoint{1.342910in}{1.583654in}}%
\pgfpathlineto{\pgfqpoint{1.338390in}{1.552623in}}%
\pgfpathlineto{\pgfqpoint{1.302483in}{1.557690in}}%
\pgfpathlineto{\pgfqpoint{1.271578in}{1.562500in}}%
\pgfpathlineto{\pgfqpoint{1.267460in}{1.535366in}}%
\pgfpathclose%
\pgfusepath{fill}%
\end{pgfscope}%
\begin{pgfscope}%
\pgfpathrectangle{\pgfqpoint{0.100000in}{0.100000in}}{\pgfqpoint{3.608454in}{2.310000in}}%
\pgfusepath{clip}%
\pgfsetbuttcap%
\pgfsetmiterjoin%
\definecolor{currentfill}{rgb}{0.000000,0.439216,0.780392}%
\pgfsetfillcolor{currentfill}%
\pgfsetlinewidth{0.000000pt}%
\definecolor{currentstroke}{rgb}{0.000000,0.000000,0.000000}%
\pgfsetstrokecolor{currentstroke}%
\pgfsetstrokeopacity{0.000000}%
\pgfsetdash{}{0pt}%
\pgfpathmoveto{\pgfqpoint{2.520499in}{1.715478in}}%
\pgfpathlineto{\pgfqpoint{2.499443in}{1.713874in}}%
\pgfpathlineto{\pgfqpoint{2.497309in}{1.741424in}}%
\pgfpathlineto{\pgfqpoint{2.504166in}{1.741961in}}%
\pgfpathlineto{\pgfqpoint{2.501961in}{1.769586in}}%
\pgfpathlineto{\pgfqpoint{2.510762in}{1.770224in}}%
\pgfpathlineto{\pgfqpoint{2.510243in}{1.777129in}}%
\pgfpathlineto{\pgfqpoint{2.529824in}{1.778725in}}%
\pgfpathlineto{\pgfqpoint{2.532257in}{1.767832in}}%
\pgfpathlineto{\pgfqpoint{2.525321in}{1.760582in}}%
\pgfpathlineto{\pgfqpoint{2.521671in}{1.742517in}}%
\pgfpathlineto{\pgfqpoint{2.524790in}{1.727344in}}%
\pgfpathlineto{\pgfqpoint{2.520499in}{1.715478in}}%
\pgfpathclose%
\pgfusepath{fill}%
\end{pgfscope}%
\begin{pgfscope}%
\pgfpathrectangle{\pgfqpoint{0.100000in}{0.100000in}}{\pgfqpoint{3.608454in}{2.310000in}}%
\pgfusepath{clip}%
\pgfsetbuttcap%
\pgfsetmiterjoin%
\definecolor{currentfill}{rgb}{0.000000,0.600000,0.700000}%
\pgfsetfillcolor{currentfill}%
\pgfsetlinewidth{0.000000pt}%
\definecolor{currentstroke}{rgb}{0.000000,0.000000,0.000000}%
\pgfsetstrokecolor{currentstroke}%
\pgfsetstrokeopacity{0.000000}%
\pgfsetdash{}{0pt}%
\pgfpathmoveto{\pgfqpoint{1.576920in}{0.953485in}}%
\pgfpathlineto{\pgfqpoint{1.580563in}{0.992038in}}%
\pgfpathlineto{\pgfqpoint{1.614395in}{0.989155in}}%
\pgfpathlineto{\pgfqpoint{1.617270in}{1.023437in}}%
\pgfpathlineto{\pgfqpoint{1.651469in}{1.020799in}}%
\pgfpathlineto{\pgfqpoint{1.648845in}{0.986397in}}%
\pgfpathlineto{\pgfqpoint{1.642812in}{0.986873in}}%
\pgfpathlineto{\pgfqpoint{1.640100in}{0.948213in}}%
\pgfpathlineto{\pgfqpoint{1.576920in}{0.953485in}}%
\pgfpathclose%
\pgfusepath{fill}%
\end{pgfscope}%
\begin{pgfscope}%
\pgfpathrectangle{\pgfqpoint{0.100000in}{0.100000in}}{\pgfqpoint{3.608454in}{2.310000in}}%
\pgfusepath{clip}%
\pgfsetbuttcap%
\pgfsetmiterjoin%
\definecolor{currentfill}{rgb}{0.000000,0.792157,0.603922}%
\pgfsetfillcolor{currentfill}%
\pgfsetlinewidth{0.000000pt}%
\definecolor{currentstroke}{rgb}{0.000000,0.000000,0.000000}%
\pgfsetstrokecolor{currentstroke}%
\pgfsetstrokeopacity{0.000000}%
\pgfsetdash{}{0pt}%
\pgfpathmoveto{\pgfqpoint{0.557819in}{2.120432in}}%
\pgfpathlineto{\pgfqpoint{0.538739in}{2.126084in}}%
\pgfpathlineto{\pgfqpoint{0.525797in}{2.131179in}}%
\pgfpathlineto{\pgfqpoint{0.534887in}{2.152356in}}%
\pgfpathlineto{\pgfqpoint{0.534769in}{2.164166in}}%
\pgfpathlineto{\pgfqpoint{0.540718in}{2.157678in}}%
\pgfpathlineto{\pgfqpoint{0.547629in}{2.160026in}}%
\pgfpathlineto{\pgfqpoint{0.555579in}{2.159647in}}%
\pgfpathlineto{\pgfqpoint{0.552341in}{2.166159in}}%
\pgfpathlineto{\pgfqpoint{0.542113in}{2.164053in}}%
\pgfpathlineto{\pgfqpoint{0.534487in}{2.174240in}}%
\pgfpathlineto{\pgfqpoint{0.538478in}{2.184732in}}%
\pgfpathlineto{\pgfqpoint{0.548088in}{2.185551in}}%
\pgfpathlineto{\pgfqpoint{0.547127in}{2.201192in}}%
\pgfpathlineto{\pgfqpoint{0.543078in}{2.205524in}}%
\pgfpathlineto{\pgfqpoint{0.545221in}{2.218928in}}%
\pgfpathlineto{\pgfqpoint{0.549339in}{2.216444in}}%
\pgfpathlineto{\pgfqpoint{0.559409in}{2.219592in}}%
\pgfpathlineto{\pgfqpoint{0.547351in}{2.228134in}}%
\pgfpathlineto{\pgfqpoint{0.549613in}{2.249680in}}%
\pgfpathlineto{\pgfqpoint{0.547461in}{2.260324in}}%
\pgfpathlineto{\pgfqpoint{0.549782in}{2.270356in}}%
\pgfpathlineto{\pgfqpoint{0.565186in}{2.263908in}}%
\pgfpathlineto{\pgfqpoint{0.593017in}{2.254746in}}%
\pgfpathlineto{\pgfqpoint{0.586531in}{2.235034in}}%
\pgfpathlineto{\pgfqpoint{0.583086in}{2.221440in}}%
\pgfpathlineto{\pgfqpoint{0.597993in}{2.216816in}}%
\pgfpathlineto{\pgfqpoint{0.598094in}{2.209351in}}%
\pgfpathlineto{\pgfqpoint{0.593046in}{2.193971in}}%
\pgfpathlineto{\pgfqpoint{0.582091in}{2.197407in}}%
\pgfpathlineto{\pgfqpoint{0.572748in}{2.166216in}}%
\pgfpathlineto{\pgfqpoint{0.631314in}{2.147810in}}%
\pgfpathlineto{\pgfqpoint{0.623310in}{2.122649in}}%
\pgfpathlineto{\pgfqpoint{0.615173in}{2.117075in}}%
\pgfpathlineto{\pgfqpoint{0.609486in}{2.121452in}}%
\pgfpathlineto{\pgfqpoint{0.593066in}{2.119148in}}%
\pgfpathlineto{\pgfqpoint{0.589293in}{2.135786in}}%
\pgfpathlineto{\pgfqpoint{0.578043in}{2.148170in}}%
\pgfpathlineto{\pgfqpoint{0.566644in}{2.148266in}}%
\pgfpathlineto{\pgfqpoint{0.557819in}{2.120432in}}%
\pgfpathclose%
\pgfusepath{fill}%
\end{pgfscope}%
\begin{pgfscope}%
\pgfpathrectangle{\pgfqpoint{0.100000in}{0.100000in}}{\pgfqpoint{3.608454in}{2.310000in}}%
\pgfusepath{clip}%
\pgfsetbuttcap%
\pgfsetmiterjoin%
\definecolor{currentfill}{rgb}{0.000000,0.839216,0.580392}%
\pgfsetfillcolor{currentfill}%
\pgfsetlinewidth{0.000000pt}%
\definecolor{currentstroke}{rgb}{0.000000,0.000000,0.000000}%
\pgfsetstrokecolor{currentstroke}%
\pgfsetstrokeopacity{0.000000}%
\pgfsetdash{}{0pt}%
\pgfpathmoveto{\pgfqpoint{0.954967in}{2.104591in}}%
\pgfpathlineto{\pgfqpoint{0.949294in}{2.080795in}}%
\pgfpathlineto{\pgfqpoint{0.933769in}{2.084665in}}%
\pgfpathlineto{\pgfqpoint{0.926998in}{2.079195in}}%
\pgfpathlineto{\pgfqpoint{0.909838in}{2.083326in}}%
\pgfpathlineto{\pgfqpoint{0.907603in}{2.074322in}}%
\pgfpathlineto{\pgfqpoint{0.899012in}{2.076012in}}%
\pgfpathlineto{\pgfqpoint{0.898511in}{2.086388in}}%
\pgfpathlineto{\pgfqpoint{0.892918in}{2.093767in}}%
\pgfpathlineto{\pgfqpoint{0.890708in}{2.100407in}}%
\pgfpathlineto{\pgfqpoint{0.882536in}{2.100932in}}%
\pgfpathlineto{\pgfqpoint{0.878570in}{2.104452in}}%
\pgfpathlineto{\pgfqpoint{0.868296in}{2.100626in}}%
\pgfpathlineto{\pgfqpoint{0.849927in}{2.099792in}}%
\pgfpathlineto{\pgfqpoint{0.849420in}{2.114482in}}%
\pgfpathlineto{\pgfqpoint{0.851613in}{2.114354in}}%
\pgfpathlineto{\pgfqpoint{0.863100in}{2.115696in}}%
\pgfpathlineto{\pgfqpoint{0.868405in}{2.124498in}}%
\pgfpathlineto{\pgfqpoint{0.875234in}{2.151233in}}%
\pgfpathlineto{\pgfqpoint{0.923573in}{2.138917in}}%
\pgfpathlineto{\pgfqpoint{0.921056in}{2.128678in}}%
\pgfpathlineto{\pgfqpoint{0.933521in}{2.118231in}}%
\pgfpathlineto{\pgfqpoint{0.950036in}{2.114158in}}%
\pgfpathlineto{\pgfqpoint{0.956600in}{2.111418in}}%
\pgfpathlineto{\pgfqpoint{0.954967in}{2.104591in}}%
\pgfpathclose%
\pgfusepath{fill}%
\end{pgfscope}%
\begin{pgfscope}%
\pgfpathrectangle{\pgfqpoint{0.100000in}{0.100000in}}{\pgfqpoint{3.608454in}{2.310000in}}%
\pgfusepath{clip}%
\pgfsetbuttcap%
\pgfsetmiterjoin%
\definecolor{currentfill}{rgb}{0.000000,0.698039,0.650980}%
\pgfsetfillcolor{currentfill}%
\pgfsetlinewidth{0.000000pt}%
\definecolor{currentstroke}{rgb}{0.000000,0.000000,0.000000}%
\pgfsetstrokecolor{currentstroke}%
\pgfsetstrokeopacity{0.000000}%
\pgfsetdash{}{0pt}%
\pgfpathmoveto{\pgfqpoint{0.877772in}{0.912470in}}%
\pgfpathlineto{\pgfqpoint{0.885192in}{0.948795in}}%
\pgfpathlineto{\pgfqpoint{0.934927in}{0.938858in}}%
\pgfpathlineto{\pgfqpoint{1.001421in}{0.926345in}}%
\pgfpathlineto{\pgfqpoint{1.075355in}{0.913752in}}%
\pgfpathlineto{\pgfqpoint{1.074172in}{0.906976in}}%
\pgfpathlineto{\pgfqpoint{1.092774in}{0.903733in}}%
\pgfpathlineto{\pgfqpoint{1.167197in}{0.891666in}}%
\pgfpathlineto{\pgfqpoint{1.153619in}{0.805601in}}%
\pgfpathlineto{\pgfqpoint{1.089707in}{0.815956in}}%
\pgfpathlineto{\pgfqpoint{1.017594in}{0.828333in}}%
\pgfpathlineto{\pgfqpoint{0.999378in}{0.839223in}}%
\pgfpathlineto{\pgfqpoint{0.900066in}{0.898746in}}%
\pgfpathlineto{\pgfqpoint{0.877772in}{0.912470in}}%
\pgfpathclose%
\pgfusepath{fill}%
\end{pgfscope}%
\begin{pgfscope}%
\pgfpathrectangle{\pgfqpoint{0.100000in}{0.100000in}}{\pgfqpoint{3.608454in}{2.310000in}}%
\pgfusepath{clip}%
\pgfsetbuttcap%
\pgfsetmiterjoin%
\definecolor{currentfill}{rgb}{0.000000,0.439216,0.780392}%
\pgfsetfillcolor{currentfill}%
\pgfsetlinewidth{0.000000pt}%
\definecolor{currentstroke}{rgb}{0.000000,0.000000,0.000000}%
\pgfsetstrokecolor{currentstroke}%
\pgfsetstrokeopacity{0.000000}%
\pgfsetdash{}{0pt}%
\pgfpathmoveto{\pgfqpoint{3.153317in}{1.449269in}}%
\pgfpathlineto{\pgfqpoint{3.152017in}{1.444895in}}%
\pgfpathlineto{\pgfqpoint{3.144962in}{1.435381in}}%
\pgfpathlineto{\pgfqpoint{3.141078in}{1.437828in}}%
\pgfpathlineto{\pgfqpoint{3.125843in}{1.434080in}}%
\pgfpathlineto{\pgfqpoint{3.119217in}{1.439782in}}%
\pgfpathlineto{\pgfqpoint{3.117131in}{1.434042in}}%
\pgfpathlineto{\pgfqpoint{3.111250in}{1.435310in}}%
\pgfpathlineto{\pgfqpoint{3.105576in}{1.439244in}}%
\pgfpathlineto{\pgfqpoint{3.087837in}{1.442494in}}%
\pgfpathlineto{\pgfqpoint{3.082421in}{1.444639in}}%
\pgfpathlineto{\pgfqpoint{3.088891in}{1.453283in}}%
\pgfpathlineto{\pgfqpoint{3.091266in}{1.461236in}}%
\pgfpathlineto{\pgfqpoint{3.097628in}{1.471273in}}%
\pgfpathlineto{\pgfqpoint{3.109226in}{1.471916in}}%
\pgfpathlineto{\pgfqpoint{3.117245in}{1.469315in}}%
\pgfpathlineto{\pgfqpoint{3.153317in}{1.449269in}}%
\pgfpathclose%
\pgfusepath{fill}%
\end{pgfscope}%
\begin{pgfscope}%
\pgfpathrectangle{\pgfqpoint{0.100000in}{0.100000in}}{\pgfqpoint{3.608454in}{2.310000in}}%
\pgfusepath{clip}%
\pgfsetbuttcap%
\pgfsetmiterjoin%
\definecolor{currentfill}{rgb}{0.000000,0.501961,0.749020}%
\pgfsetfillcolor{currentfill}%
\pgfsetlinewidth{0.000000pt}%
\definecolor{currentstroke}{rgb}{0.000000,0.000000,0.000000}%
\pgfsetstrokecolor{currentstroke}%
\pgfsetstrokeopacity{0.000000}%
\pgfsetdash{}{0pt}%
\pgfpathmoveto{\pgfqpoint{1.965169in}{0.705757in}}%
\pgfpathlineto{\pgfqpoint{1.952059in}{0.698355in}}%
\pgfpathlineto{\pgfqpoint{1.947915in}{0.687499in}}%
\pgfpathlineto{\pgfqpoint{1.926963in}{0.697491in}}%
\pgfpathlineto{\pgfqpoint{1.913119in}{0.701138in}}%
\pgfpathlineto{\pgfqpoint{1.907744in}{0.711220in}}%
\pgfpathlineto{\pgfqpoint{1.871649in}{0.712052in}}%
\pgfpathlineto{\pgfqpoint{1.866023in}{0.717001in}}%
\pgfpathlineto{\pgfqpoint{1.863869in}{0.728404in}}%
\pgfpathlineto{\pgfqpoint{1.884281in}{0.742465in}}%
\pgfpathlineto{\pgfqpoint{1.919219in}{0.761725in}}%
\pgfpathlineto{\pgfqpoint{1.924502in}{0.764640in}}%
\pgfpathlineto{\pgfqpoint{1.942049in}{0.732981in}}%
\pgfpathlineto{\pgfqpoint{1.947012in}{0.726789in}}%
\pgfpathlineto{\pgfqpoint{1.951524in}{0.729506in}}%
\pgfpathlineto{\pgfqpoint{1.965169in}{0.705757in}}%
\pgfpathclose%
\pgfusepath{fill}%
\end{pgfscope}%
\begin{pgfscope}%
\pgfpathrectangle{\pgfqpoint{0.100000in}{0.100000in}}{\pgfqpoint{3.608454in}{2.310000in}}%
\pgfusepath{clip}%
\pgfsetbuttcap%
\pgfsetmiterjoin%
\definecolor{currentfill}{rgb}{0.000000,0.800000,0.600000}%
\pgfsetfillcolor{currentfill}%
\pgfsetlinewidth{0.000000pt}%
\definecolor{currentstroke}{rgb}{0.000000,0.000000,0.000000}%
\pgfsetstrokecolor{currentstroke}%
\pgfsetstrokeopacity{0.000000}%
\pgfsetdash{}{0pt}%
\pgfpathmoveto{\pgfqpoint{1.805378in}{0.356576in}}%
\pgfpathlineto{\pgfqpoint{1.817184in}{0.364409in}}%
\pgfpathlineto{\pgfqpoint{1.821677in}{0.373163in}}%
\pgfpathlineto{\pgfqpoint{1.859982in}{0.371433in}}%
\pgfpathlineto{\pgfqpoint{1.891484in}{0.370110in}}%
\pgfpathlineto{\pgfqpoint{1.891060in}{0.356772in}}%
\pgfpathlineto{\pgfqpoint{1.899944in}{0.355045in}}%
\pgfpathlineto{\pgfqpoint{1.930477in}{0.354309in}}%
\pgfpathlineto{\pgfqpoint{1.932540in}{0.346180in}}%
\pgfpathlineto{\pgfqpoint{1.930157in}{0.343218in}}%
\pgfpathlineto{\pgfqpoint{1.936961in}{0.334615in}}%
\pgfpathlineto{\pgfqpoint{1.936672in}{0.327595in}}%
\pgfpathlineto{\pgfqpoint{1.940091in}{0.321228in}}%
\pgfpathlineto{\pgfqpoint{1.940482in}{0.313585in}}%
\pgfpathlineto{\pgfqpoint{1.950108in}{0.309249in}}%
\pgfpathlineto{\pgfqpoint{1.950498in}{0.302891in}}%
\pgfpathlineto{\pgfqpoint{1.934207in}{0.298800in}}%
\pgfpathlineto{\pgfqpoint{1.929813in}{0.292862in}}%
\pgfpathlineto{\pgfqpoint{1.923685in}{0.296794in}}%
\pgfpathlineto{\pgfqpoint{1.911823in}{0.308379in}}%
\pgfpathlineto{\pgfqpoint{1.896783in}{0.312095in}}%
\pgfpathlineto{\pgfqpoint{1.883885in}{0.310848in}}%
\pgfpathlineto{\pgfqpoint{1.870440in}{0.313575in}}%
\pgfpathlineto{\pgfqpoint{1.855483in}{0.326486in}}%
\pgfpathlineto{\pgfqpoint{1.840888in}{0.327815in}}%
\pgfpathlineto{\pgfqpoint{1.830852in}{0.339080in}}%
\pgfpathlineto{\pgfqpoint{1.828260in}{0.338364in}}%
\pgfpathlineto{\pgfqpoint{1.809077in}{0.344635in}}%
\pgfpathlineto{\pgfqpoint{1.810652in}{0.348669in}}%
\pgfpathlineto{\pgfqpoint{1.805378in}{0.356576in}}%
\pgfpathclose%
\pgfusepath{fill}%
\end{pgfscope}%
\begin{pgfscope}%
\pgfpathrectangle{\pgfqpoint{0.100000in}{0.100000in}}{\pgfqpoint{3.608454in}{2.310000in}}%
\pgfusepath{clip}%
\pgfsetbuttcap%
\pgfsetmiterjoin%
\definecolor{currentfill}{rgb}{0.000000,0.525490,0.737255}%
\pgfsetfillcolor{currentfill}%
\pgfsetlinewidth{0.000000pt}%
\definecolor{currentstroke}{rgb}{0.000000,0.000000,0.000000}%
\pgfsetstrokecolor{currentstroke}%
\pgfsetstrokeopacity{0.000000}%
\pgfsetdash{}{0pt}%
\pgfpathmoveto{\pgfqpoint{2.495702in}{1.322010in}}%
\pgfpathlineto{\pgfqpoint{2.494687in}{1.339376in}}%
\pgfpathlineto{\pgfqpoint{2.467492in}{1.337391in}}%
\pgfpathlineto{\pgfqpoint{2.467926in}{1.330458in}}%
\pgfpathlineto{\pgfqpoint{2.460771in}{1.330448in}}%
\pgfpathlineto{\pgfqpoint{2.458623in}{1.367966in}}%
\pgfpathlineto{\pgfqpoint{2.465371in}{1.368430in}}%
\pgfpathlineto{\pgfqpoint{2.506265in}{1.370984in}}%
\pgfpathlineto{\pgfqpoint{2.506514in}{1.367545in}}%
\pgfpathlineto{\pgfqpoint{2.550764in}{1.371543in}}%
\pgfpathlineto{\pgfqpoint{2.558058in}{1.370400in}}%
\pgfpathlineto{\pgfqpoint{2.562133in}{1.362791in}}%
\pgfpathlineto{\pgfqpoint{2.562407in}{1.357174in}}%
\pgfpathlineto{\pgfqpoint{2.566176in}{1.350839in}}%
\pgfpathlineto{\pgfqpoint{2.565996in}{1.346868in}}%
\pgfpathlineto{\pgfqpoint{2.542895in}{1.344839in}}%
\pgfpathlineto{\pgfqpoint{2.521388in}{1.342978in}}%
\pgfpathlineto{\pgfqpoint{2.521612in}{1.327886in}}%
\pgfpathlineto{\pgfqpoint{2.523132in}{1.323450in}}%
\pgfpathlineto{\pgfqpoint{2.495702in}{1.322010in}}%
\pgfpathclose%
\pgfusepath{fill}%
\end{pgfscope}%
\begin{pgfscope}%
\pgfpathrectangle{\pgfqpoint{0.100000in}{0.100000in}}{\pgfqpoint{3.608454in}{2.310000in}}%
\pgfusepath{clip}%
\pgfsetbuttcap%
\pgfsetmiterjoin%
\definecolor{currentfill}{rgb}{0.000000,0.568627,0.715686}%
\pgfsetfillcolor{currentfill}%
\pgfsetlinewidth{0.000000pt}%
\definecolor{currentstroke}{rgb}{0.000000,0.000000,0.000000}%
\pgfsetstrokecolor{currentstroke}%
\pgfsetstrokeopacity{0.000000}%
\pgfsetdash{}{0pt}%
\pgfpathmoveto{\pgfqpoint{3.037191in}{1.156638in}}%
\pgfpathlineto{\pgfqpoint{3.024951in}{1.162919in}}%
\pgfpathlineto{\pgfqpoint{3.020036in}{1.162708in}}%
\pgfpathlineto{\pgfqpoint{3.018531in}{1.178366in}}%
\pgfpathlineto{\pgfqpoint{3.006498in}{1.176840in}}%
\pgfpathlineto{\pgfqpoint{3.004020in}{1.198138in}}%
\pgfpathlineto{\pgfqpoint{2.996900in}{1.203053in}}%
\pgfpathlineto{\pgfqpoint{2.995823in}{1.206373in}}%
\pgfpathlineto{\pgfqpoint{2.999075in}{1.216208in}}%
\pgfpathlineto{\pgfqpoint{3.017299in}{1.218774in}}%
\pgfpathlineto{\pgfqpoint{3.037358in}{1.220901in}}%
\pgfpathlineto{\pgfqpoint{3.054182in}{1.223554in}}%
\pgfpathlineto{\pgfqpoint{3.060084in}{1.181854in}}%
\pgfpathlineto{\pgfqpoint{3.049089in}{1.181346in}}%
\pgfpathlineto{\pgfqpoint{3.039491in}{1.174669in}}%
\pgfpathlineto{\pgfqpoint{3.039762in}{1.164964in}}%
\pgfpathlineto{\pgfqpoint{3.035098in}{1.163147in}}%
\pgfpathlineto{\pgfqpoint{3.037191in}{1.156638in}}%
\pgfpathclose%
\pgfusepath{fill}%
\end{pgfscope}%
\begin{pgfscope}%
\pgfpathrectangle{\pgfqpoint{0.100000in}{0.100000in}}{\pgfqpoint{3.608454in}{2.310000in}}%
\pgfusepath{clip}%
\pgfsetbuttcap%
\pgfsetmiterjoin%
\definecolor{currentfill}{rgb}{0.000000,0.607843,0.696078}%
\pgfsetfillcolor{currentfill}%
\pgfsetlinewidth{0.000000pt}%
\definecolor{currentstroke}{rgb}{0.000000,0.000000,0.000000}%
\pgfsetstrokecolor{currentstroke}%
\pgfsetstrokeopacity{0.000000}%
\pgfsetdash{}{0pt}%
\pgfpathmoveto{\pgfqpoint{3.358684in}{1.520578in}}%
\pgfpathlineto{\pgfqpoint{3.353587in}{1.521095in}}%
\pgfpathlineto{\pgfqpoint{3.348171in}{1.542140in}}%
\pgfpathlineto{\pgfqpoint{3.330022in}{1.567556in}}%
\pgfpathlineto{\pgfqpoint{3.327731in}{1.570641in}}%
\pgfpathlineto{\pgfqpoint{3.339300in}{1.581420in}}%
\pgfpathlineto{\pgfqpoint{3.345889in}{1.596696in}}%
\pgfpathlineto{\pgfqpoint{3.350906in}{1.598176in}}%
\pgfpathlineto{\pgfqpoint{3.359788in}{1.596585in}}%
\pgfpathlineto{\pgfqpoint{3.363035in}{1.588297in}}%
\pgfpathlineto{\pgfqpoint{3.363118in}{1.574131in}}%
\pgfpathlineto{\pgfqpoint{3.365563in}{1.546396in}}%
\pgfpathlineto{\pgfqpoint{3.364102in}{1.538156in}}%
\pgfpathlineto{\pgfqpoint{3.358684in}{1.520578in}}%
\pgfpathclose%
\pgfusepath{fill}%
\end{pgfscope}%
\begin{pgfscope}%
\pgfpathrectangle{\pgfqpoint{0.100000in}{0.100000in}}{\pgfqpoint{3.608454in}{2.310000in}}%
\pgfusepath{clip}%
\pgfsetbuttcap%
\pgfsetmiterjoin%
\definecolor{currentfill}{rgb}{0.000000,0.419608,0.790196}%
\pgfsetfillcolor{currentfill}%
\pgfsetlinewidth{0.000000pt}%
\definecolor{currentstroke}{rgb}{0.000000,0.000000,0.000000}%
\pgfsetstrokecolor{currentstroke}%
\pgfsetstrokeopacity{0.000000}%
\pgfsetdash{}{0pt}%
\pgfpathmoveto{\pgfqpoint{1.939662in}{1.829664in}}%
\pgfpathlineto{\pgfqpoint{1.938531in}{1.788233in}}%
\pgfpathlineto{\pgfqpoint{1.897415in}{1.789939in}}%
\pgfpathlineto{\pgfqpoint{1.898285in}{1.810809in}}%
\pgfpathlineto{\pgfqpoint{1.849510in}{1.813058in}}%
\pgfpathlineto{\pgfqpoint{1.850854in}{1.840684in}}%
\pgfpathlineto{\pgfqpoint{1.843275in}{1.841097in}}%
\pgfpathlineto{\pgfqpoint{1.845659in}{1.896366in}}%
\pgfpathlineto{\pgfqpoint{1.885060in}{1.894397in}}%
\pgfpathlineto{\pgfqpoint{1.941869in}{1.891889in}}%
\pgfpathlineto{\pgfqpoint{1.939662in}{1.829664in}}%
\pgfpathclose%
\pgfusepath{fill}%
\end{pgfscope}%
\begin{pgfscope}%
\pgfpathrectangle{\pgfqpoint{0.100000in}{0.100000in}}{\pgfqpoint{3.608454in}{2.310000in}}%
\pgfusepath{clip}%
\pgfsetbuttcap%
\pgfsetmiterjoin%
\definecolor{currentfill}{rgb}{0.000000,0.600000,0.700000}%
\pgfsetfillcolor{currentfill}%
\pgfsetlinewidth{0.000000pt}%
\definecolor{currentstroke}{rgb}{0.000000,0.000000,0.000000}%
\pgfsetstrokecolor{currentstroke}%
\pgfsetstrokeopacity{0.000000}%
\pgfsetdash{}{0pt}%
\pgfpathmoveto{\pgfqpoint{2.173280in}{0.783716in}}%
\pgfpathlineto{\pgfqpoint{2.139795in}{0.782977in}}%
\pgfpathlineto{\pgfqpoint{2.143853in}{0.772670in}}%
\pgfpathlineto{\pgfqpoint{2.111050in}{0.772661in}}%
\pgfpathlineto{\pgfqpoint{2.107805in}{0.772710in}}%
\pgfpathlineto{\pgfqpoint{2.107625in}{0.827720in}}%
\pgfpathlineto{\pgfqpoint{2.096510in}{0.830344in}}%
\pgfpathlineto{\pgfqpoint{2.096641in}{0.856769in}}%
\pgfpathlineto{\pgfqpoint{2.125475in}{0.856951in}}%
\pgfpathlineto{\pgfqpoint{2.126427in}{0.854941in}}%
\pgfpathlineto{\pgfqpoint{2.170684in}{0.855388in}}%
\pgfpathlineto{\pgfqpoint{2.171444in}{0.784825in}}%
\pgfpathlineto{\pgfqpoint{2.173280in}{0.783716in}}%
\pgfpathclose%
\pgfusepath{fill}%
\end{pgfscope}%
\begin{pgfscope}%
\pgfpathrectangle{\pgfqpoint{0.100000in}{0.100000in}}{\pgfqpoint{3.608454in}{2.310000in}}%
\pgfusepath{clip}%
\pgfsetbuttcap%
\pgfsetmiterjoin%
\definecolor{currentfill}{rgb}{0.000000,0.827451,0.586275}%
\pgfsetfillcolor{currentfill}%
\pgfsetlinewidth{0.000000pt}%
\definecolor{currentstroke}{rgb}{0.000000,0.000000,0.000000}%
\pgfsetstrokecolor{currentstroke}%
\pgfsetstrokeopacity{0.000000}%
\pgfsetdash{}{0pt}%
\pgfpathmoveto{\pgfqpoint{0.976297in}{0.641465in}}%
\pgfpathlineto{\pgfqpoint{0.971445in}{0.636313in}}%
\pgfpathlineto{\pgfqpoint{0.973534in}{0.634639in}}%
\pgfpathlineto{\pgfqpoint{0.971825in}{0.632484in}}%
\pgfpathlineto{\pgfqpoint{0.974033in}{0.630789in}}%
\pgfpathlineto{\pgfqpoint{0.972344in}{0.628719in}}%
\pgfpathlineto{\pgfqpoint{0.974510in}{0.627028in}}%
\pgfpathlineto{\pgfqpoint{0.972240in}{0.625296in}}%
\pgfpathlineto{\pgfqpoint{0.967277in}{0.618907in}}%
\pgfpathlineto{\pgfqpoint{0.971558in}{0.615632in}}%
\pgfpathlineto{\pgfqpoint{0.969910in}{0.613404in}}%
\pgfpathlineto{\pgfqpoint{0.971435in}{0.612298in}}%
\pgfpathlineto{\pgfqpoint{0.964303in}{0.603251in}}%
\pgfpathlineto{\pgfqpoint{0.961995in}{0.601456in}}%
\pgfpathlineto{\pgfqpoint{0.955683in}{0.606572in}}%
\pgfpathlineto{\pgfqpoint{0.952323in}{0.602325in}}%
\pgfpathlineto{\pgfqpoint{0.946326in}{0.607097in}}%
\pgfpathlineto{\pgfqpoint{0.940885in}{0.601118in}}%
\pgfpathlineto{\pgfqpoint{0.936932in}{0.604209in}}%
\pgfpathlineto{\pgfqpoint{0.938627in}{0.606373in}}%
\pgfpathlineto{\pgfqpoint{0.936545in}{0.608078in}}%
\pgfpathlineto{\pgfqpoint{0.938173in}{0.610139in}}%
\pgfpathlineto{\pgfqpoint{0.934029in}{0.613492in}}%
\pgfpathlineto{\pgfqpoint{0.930663in}{0.609300in}}%
\pgfpathlineto{\pgfqpoint{0.926478in}{0.612658in}}%
\pgfpathlineto{\pgfqpoint{0.929884in}{0.616921in}}%
\pgfpathlineto{\pgfqpoint{0.921542in}{0.623737in}}%
\pgfpathlineto{\pgfqpoint{0.914790in}{0.615304in}}%
\pgfpathlineto{\pgfqpoint{0.904546in}{0.623960in}}%
\pgfpathlineto{\pgfqpoint{0.901313in}{0.620035in}}%
\pgfpathlineto{\pgfqpoint{0.895296in}{0.625196in}}%
\pgfpathlineto{\pgfqpoint{0.886576in}{0.615036in}}%
\pgfpathlineto{\pgfqpoint{0.884580in}{0.616759in}}%
\pgfpathlineto{\pgfqpoint{0.882807in}{0.614773in}}%
\pgfpathlineto{\pgfqpoint{0.870840in}{0.625356in}}%
\pgfpathlineto{\pgfqpoint{0.853614in}{0.640940in}}%
\pgfpathlineto{\pgfqpoint{0.845457in}{0.648542in}}%
\pgfpathlineto{\pgfqpoint{0.849066in}{0.652382in}}%
\pgfpathlineto{\pgfqpoint{0.847118in}{0.654238in}}%
\pgfpathlineto{\pgfqpoint{0.850812in}{0.658129in}}%
\pgfpathlineto{\pgfqpoint{0.848280in}{0.660517in}}%
\pgfpathlineto{\pgfqpoint{0.855680in}{0.668231in}}%
\pgfpathlineto{\pgfqpoint{0.855048in}{0.668896in}}%
\pgfpathlineto{\pgfqpoint{0.862419in}{0.676643in}}%
\pgfpathlineto{\pgfqpoint{0.863943in}{0.679699in}}%
\pgfpathlineto{\pgfqpoint{0.867167in}{0.677080in}}%
\pgfpathlineto{\pgfqpoint{0.868881in}{0.673318in}}%
\pgfpathlineto{\pgfqpoint{0.866790in}{0.671377in}}%
\pgfpathlineto{\pgfqpoint{0.864000in}{0.670092in}}%
\pgfpathlineto{\pgfqpoint{0.862316in}{0.667806in}}%
\pgfpathlineto{\pgfqpoint{0.859533in}{0.667863in}}%
\pgfpathlineto{\pgfqpoint{0.858589in}{0.662654in}}%
\pgfpathlineto{\pgfqpoint{0.861141in}{0.660797in}}%
\pgfpathlineto{\pgfqpoint{0.865258in}{0.656612in}}%
\pgfpathlineto{\pgfqpoint{0.868725in}{0.654793in}}%
\pgfpathlineto{\pgfqpoint{0.868360in}{0.652816in}}%
\pgfpathlineto{\pgfqpoint{0.870641in}{0.649968in}}%
\pgfpathlineto{\pgfqpoint{0.873832in}{0.649148in}}%
\pgfpathlineto{\pgfqpoint{0.875177in}{0.647013in}}%
\pgfpathlineto{\pgfqpoint{0.873862in}{0.644273in}}%
\pgfpathlineto{\pgfqpoint{0.879648in}{0.646559in}}%
\pgfpathlineto{\pgfqpoint{0.883020in}{0.648596in}}%
\pgfpathlineto{\pgfqpoint{0.884548in}{0.645448in}}%
\pgfpathlineto{\pgfqpoint{0.888156in}{0.644278in}}%
\pgfpathlineto{\pgfqpoint{0.888264in}{0.647509in}}%
\pgfpathlineto{\pgfqpoint{0.885826in}{0.651944in}}%
\pgfpathlineto{\pgfqpoint{0.882189in}{0.653954in}}%
\pgfpathlineto{\pgfqpoint{0.885088in}{0.656299in}}%
\pgfpathlineto{\pgfqpoint{0.885204in}{0.660563in}}%
\pgfpathlineto{\pgfqpoint{0.886202in}{0.663888in}}%
\pgfpathlineto{\pgfqpoint{0.884247in}{0.666557in}}%
\pgfpathlineto{\pgfqpoint{0.886027in}{0.671213in}}%
\pgfpathlineto{\pgfqpoint{0.888440in}{0.671601in}}%
\pgfpathlineto{\pgfqpoint{0.890129in}{0.670612in}}%
\pgfpathlineto{\pgfqpoint{0.889298in}{0.664217in}}%
\pgfpathlineto{\pgfqpoint{0.886556in}{0.660681in}}%
\pgfpathlineto{\pgfqpoint{0.886858in}{0.653674in}}%
\pgfpathlineto{\pgfqpoint{0.891233in}{0.651664in}}%
\pgfpathlineto{\pgfqpoint{0.892089in}{0.649695in}}%
\pgfpathlineto{\pgfqpoint{0.891874in}{0.645714in}}%
\pgfpathlineto{\pgfqpoint{0.897602in}{0.640349in}}%
\pgfpathlineto{\pgfqpoint{0.900250in}{0.642484in}}%
\pgfpathlineto{\pgfqpoint{0.901360in}{0.646766in}}%
\pgfpathlineto{\pgfqpoint{0.898368in}{0.648567in}}%
\pgfpathlineto{\pgfqpoint{0.897419in}{0.652291in}}%
\pgfpathlineto{\pgfqpoint{0.893693in}{0.653344in}}%
\pgfpathlineto{\pgfqpoint{0.891532in}{0.652766in}}%
\pgfpathlineto{\pgfqpoint{0.889610in}{0.655212in}}%
\pgfpathlineto{\pgfqpoint{0.889414in}{0.661725in}}%
\pgfpathlineto{\pgfqpoint{0.891701in}{0.663459in}}%
\pgfpathlineto{\pgfqpoint{0.897189in}{0.666384in}}%
\pgfpathlineto{\pgfqpoint{0.896490in}{0.669062in}}%
\pgfpathlineto{\pgfqpoint{0.890332in}{0.672207in}}%
\pgfpathlineto{\pgfqpoint{0.891668in}{0.673920in}}%
\pgfpathlineto{\pgfqpoint{0.887467in}{0.677256in}}%
\pgfpathlineto{\pgfqpoint{0.884828in}{0.680240in}}%
\pgfpathlineto{\pgfqpoint{0.881627in}{0.686225in}}%
\pgfpathlineto{\pgfqpoint{0.885981in}{0.691916in}}%
\pgfpathlineto{\pgfqpoint{0.888139in}{0.698608in}}%
\pgfpathlineto{\pgfqpoint{0.888448in}{0.701521in}}%
\pgfpathlineto{\pgfqpoint{0.887314in}{0.706916in}}%
\pgfpathlineto{\pgfqpoint{0.887408in}{0.709755in}}%
\pgfpathlineto{\pgfqpoint{0.886741in}{0.720168in}}%
\pgfpathlineto{\pgfqpoint{0.894146in}{0.712972in}}%
\pgfpathlineto{\pgfqpoint{0.898458in}{0.717465in}}%
\pgfpathlineto{\pgfqpoint{0.913662in}{0.703154in}}%
\pgfpathlineto{\pgfqpoint{0.915131in}{0.704745in}}%
\pgfpathlineto{\pgfqpoint{0.921351in}{0.699050in}}%
\pgfpathlineto{\pgfqpoint{0.919895in}{0.697448in}}%
\pgfpathlineto{\pgfqpoint{0.924102in}{0.693649in}}%
\pgfpathlineto{\pgfqpoint{0.925549in}{0.695258in}}%
\pgfpathlineto{\pgfqpoint{0.929228in}{0.691967in}}%
\pgfpathlineto{\pgfqpoint{0.927789in}{0.690351in}}%
\pgfpathlineto{\pgfqpoint{0.936363in}{0.682792in}}%
\pgfpathlineto{\pgfqpoint{0.937775in}{0.684411in}}%
\pgfpathlineto{\pgfqpoint{0.941559in}{0.681130in}}%
\pgfpathlineto{\pgfqpoint{0.940158in}{0.679506in}}%
\pgfpathlineto{\pgfqpoint{0.954477in}{0.667338in}}%
\pgfpathlineto{\pgfqpoint{0.963280in}{0.660093in}}%
\pgfpathlineto{\pgfqpoint{0.961484in}{0.657889in}}%
\pgfpathlineto{\pgfqpoint{0.969832in}{0.651052in}}%
\pgfpathlineto{\pgfqpoint{0.968183in}{0.648986in}}%
\pgfpathlineto{\pgfqpoint{0.976829in}{0.642165in}}%
\pgfpathlineto{\pgfqpoint{0.976297in}{0.641465in}}%
\pgfpathclose%
\pgfusepath{fill}%
\end{pgfscope}%
\begin{pgfscope}%
\pgfpathrectangle{\pgfqpoint{0.100000in}{0.100000in}}{\pgfqpoint{3.608454in}{2.310000in}}%
\pgfusepath{clip}%
\pgfsetbuttcap%
\pgfsetmiterjoin%
\definecolor{currentfill}{rgb}{0.000000,0.796078,0.601961}%
\pgfsetfillcolor{currentfill}%
\pgfsetlinewidth{0.000000pt}%
\definecolor{currentstroke}{rgb}{0.000000,0.000000,0.000000}%
\pgfsetstrokecolor{currentstroke}%
\pgfsetstrokeopacity{0.000000}%
\pgfsetdash{}{0pt}%
\pgfpathmoveto{\pgfqpoint{1.801456in}{1.757724in}}%
\pgfpathlineto{\pgfqpoint{1.800009in}{1.732253in}}%
\pgfpathlineto{\pgfqpoint{1.800957in}{1.721865in}}%
\pgfpathlineto{\pgfqpoint{1.793255in}{1.722620in}}%
\pgfpathlineto{\pgfqpoint{1.787775in}{1.725501in}}%
\pgfpathlineto{\pgfqpoint{1.775409in}{1.724494in}}%
\pgfpathlineto{\pgfqpoint{1.768321in}{1.732235in}}%
\pgfpathlineto{\pgfqpoint{1.760148in}{1.734344in}}%
\pgfpathlineto{\pgfqpoint{1.762841in}{1.760148in}}%
\pgfpathlineto{\pgfqpoint{1.801456in}{1.757724in}}%
\pgfpathclose%
\pgfusepath{fill}%
\end{pgfscope}%
\begin{pgfscope}%
\pgfpathrectangle{\pgfqpoint{0.100000in}{0.100000in}}{\pgfqpoint{3.608454in}{2.310000in}}%
\pgfusepath{clip}%
\pgfsetbuttcap%
\pgfsetmiterjoin%
\definecolor{currentfill}{rgb}{0.000000,0.462745,0.768627}%
\pgfsetfillcolor{currentfill}%
\pgfsetlinewidth{0.000000pt}%
\definecolor{currentstroke}{rgb}{0.000000,0.000000,0.000000}%
\pgfsetstrokecolor{currentstroke}%
\pgfsetstrokeopacity{0.000000}%
\pgfsetdash{}{0pt}%
\pgfpathmoveto{\pgfqpoint{2.387375in}{1.567036in}}%
\pgfpathlineto{\pgfqpoint{2.380372in}{1.562314in}}%
\pgfpathlineto{\pgfqpoint{2.376029in}{1.564684in}}%
\pgfpathlineto{\pgfqpoint{2.348758in}{1.564236in}}%
\pgfpathlineto{\pgfqpoint{2.348123in}{1.578046in}}%
\pgfpathlineto{\pgfqpoint{2.346907in}{1.605656in}}%
\pgfpathlineto{\pgfqpoint{2.360507in}{1.606246in}}%
\pgfpathlineto{\pgfqpoint{2.360207in}{1.613167in}}%
\pgfpathlineto{\pgfqpoint{2.371380in}{1.613652in}}%
\pgfpathlineto{\pgfqpoint{2.374950in}{1.609365in}}%
\pgfpathlineto{\pgfqpoint{2.376710in}{1.601691in}}%
\pgfpathlineto{\pgfqpoint{2.381293in}{1.599262in}}%
\pgfpathlineto{\pgfqpoint{2.390701in}{1.593517in}}%
\pgfpathlineto{\pgfqpoint{2.392472in}{1.584764in}}%
\pgfpathlineto{\pgfqpoint{2.390834in}{1.569129in}}%
\pgfpathlineto{\pgfqpoint{2.387375in}{1.567036in}}%
\pgfpathclose%
\pgfusepath{fill}%
\end{pgfscope}%
\begin{pgfscope}%
\pgfpathrectangle{\pgfqpoint{0.100000in}{0.100000in}}{\pgfqpoint{3.608454in}{2.310000in}}%
\pgfusepath{clip}%
\pgfsetbuttcap%
\pgfsetmiterjoin%
\definecolor{currentfill}{rgb}{0.000000,0.364706,0.817647}%
\pgfsetfillcolor{currentfill}%
\pgfsetlinewidth{0.000000pt}%
\definecolor{currentstroke}{rgb}{0.000000,0.000000,0.000000}%
\pgfsetstrokecolor{currentstroke}%
\pgfsetstrokeopacity{0.000000}%
\pgfsetdash{}{0pt}%
\pgfpathmoveto{\pgfqpoint{1.853837in}{1.424663in}}%
\pgfpathlineto{\pgfqpoint{1.881222in}{1.423490in}}%
\pgfpathlineto{\pgfqpoint{1.894644in}{1.422961in}}%
\pgfpathlineto{\pgfqpoint{1.893270in}{1.388628in}}%
\pgfpathlineto{\pgfqpoint{1.860470in}{1.390012in}}%
\pgfpathlineto{\pgfqpoint{1.824934in}{1.391676in}}%
\pgfpathlineto{\pgfqpoint{1.826801in}{1.425965in}}%
\pgfpathlineto{\pgfqpoint{1.853837in}{1.424663in}}%
\pgfpathclose%
\pgfusepath{fill}%
\end{pgfscope}%
\begin{pgfscope}%
\pgfpathrectangle{\pgfqpoint{0.100000in}{0.100000in}}{\pgfqpoint{3.608454in}{2.310000in}}%
\pgfusepath{clip}%
\pgfsetbuttcap%
\pgfsetmiterjoin%
\definecolor{currentfill}{rgb}{0.000000,0.494118,0.752941}%
\pgfsetfillcolor{currentfill}%
\pgfsetlinewidth{0.000000pt}%
\definecolor{currentstroke}{rgb}{0.000000,0.000000,0.000000}%
\pgfsetstrokecolor{currentstroke}%
\pgfsetstrokeopacity{0.000000}%
\pgfsetdash{}{0pt}%
\pgfpathmoveto{\pgfqpoint{2.374602in}{1.377391in}}%
\pgfpathlineto{\pgfqpoint{2.376576in}{1.388834in}}%
\pgfpathlineto{\pgfqpoint{2.371479in}{1.401180in}}%
\pgfpathlineto{\pgfqpoint{2.375453in}{1.412534in}}%
\pgfpathlineto{\pgfqpoint{2.354524in}{1.412008in}}%
\pgfpathlineto{\pgfqpoint{2.353565in}{1.439495in}}%
\pgfpathlineto{\pgfqpoint{2.353369in}{1.446719in}}%
\pgfpathlineto{\pgfqpoint{2.381085in}{1.447410in}}%
\pgfpathlineto{\pgfqpoint{2.381360in}{1.440493in}}%
\pgfpathlineto{\pgfqpoint{2.396575in}{1.440884in}}%
\pgfpathlineto{\pgfqpoint{2.389932in}{1.435148in}}%
\pgfpathlineto{\pgfqpoint{2.391369in}{1.431061in}}%
\pgfpathlineto{\pgfqpoint{2.397384in}{1.430530in}}%
\pgfpathlineto{\pgfqpoint{2.409257in}{1.435587in}}%
\pgfpathlineto{\pgfqpoint{2.410721in}{1.410092in}}%
\pgfpathlineto{\pgfqpoint{2.415980in}{1.389550in}}%
\pgfpathlineto{\pgfqpoint{2.402134in}{1.388642in}}%
\pgfpathlineto{\pgfqpoint{2.403570in}{1.368279in}}%
\pgfpathlineto{\pgfqpoint{2.393816in}{1.360813in}}%
\pgfpathlineto{\pgfqpoint{2.383166in}{1.360351in}}%
\pgfpathlineto{\pgfqpoint{2.376719in}{1.362091in}}%
\pgfpathlineto{\pgfqpoint{2.374602in}{1.377391in}}%
\pgfpathclose%
\pgfusepath{fill}%
\end{pgfscope}%
\begin{pgfscope}%
\pgfpathrectangle{\pgfqpoint{0.100000in}{0.100000in}}{\pgfqpoint{3.608454in}{2.310000in}}%
\pgfusepath{clip}%
\pgfsetbuttcap%
\pgfsetmiterjoin%
\definecolor{currentfill}{rgb}{0.000000,0.639216,0.680392}%
\pgfsetfillcolor{currentfill}%
\pgfsetlinewidth{0.000000pt}%
\definecolor{currentstroke}{rgb}{0.000000,0.000000,0.000000}%
\pgfsetstrokecolor{currentstroke}%
\pgfsetstrokeopacity{0.000000}%
\pgfsetdash{}{0pt}%
\pgfpathmoveto{\pgfqpoint{3.177265in}{1.821081in}}%
\pgfpathlineto{\pgfqpoint{3.173354in}{1.830395in}}%
\pgfpathlineto{\pgfqpoint{3.177753in}{1.838567in}}%
\pgfpathlineto{\pgfqpoint{3.175101in}{1.843253in}}%
\pgfpathlineto{\pgfqpoint{3.166217in}{1.845402in}}%
\pgfpathlineto{\pgfqpoint{3.162379in}{1.854420in}}%
\pgfpathlineto{\pgfqpoint{3.171115in}{1.863602in}}%
\pgfpathlineto{\pgfqpoint{3.169608in}{1.868162in}}%
\pgfpathlineto{\pgfqpoint{3.181998in}{1.877869in}}%
\pgfpathlineto{\pgfqpoint{3.186045in}{1.883942in}}%
\pgfpathlineto{\pgfqpoint{3.187678in}{1.891082in}}%
\pgfpathlineto{\pgfqpoint{3.198874in}{1.908865in}}%
\pgfpathlineto{\pgfqpoint{3.215748in}{1.929678in}}%
\pgfpathlineto{\pgfqpoint{3.223456in}{1.936959in}}%
\pgfpathlineto{\pgfqpoint{3.230888in}{1.941273in}}%
\pgfpathlineto{\pgfqpoint{3.236756in}{1.940890in}}%
\pgfpathlineto{\pgfqpoint{3.242124in}{1.938669in}}%
\pgfpathlineto{\pgfqpoint{3.262366in}{1.879593in}}%
\pgfpathlineto{\pgfqpoint{3.263105in}{1.873951in}}%
\pgfpathlineto{\pgfqpoint{3.245888in}{1.867665in}}%
\pgfpathlineto{\pgfqpoint{3.234624in}{1.863539in}}%
\pgfpathlineto{\pgfqpoint{3.227847in}{1.865817in}}%
\pgfpathlineto{\pgfqpoint{3.239480in}{1.829258in}}%
\pgfpathlineto{\pgfqpoint{3.219039in}{1.808893in}}%
\pgfpathlineto{\pgfqpoint{3.205664in}{1.810106in}}%
\pgfpathlineto{\pgfqpoint{3.200994in}{1.826823in}}%
\pgfpathlineto{\pgfqpoint{3.186645in}{1.825233in}}%
\pgfpathlineto{\pgfqpoint{3.177265in}{1.821081in}}%
\pgfpathclose%
\pgfusepath{fill}%
\end{pgfscope}%
\begin{pgfscope}%
\pgfpathrectangle{\pgfqpoint{0.100000in}{0.100000in}}{\pgfqpoint{3.608454in}{2.310000in}}%
\pgfusepath{clip}%
\pgfsetbuttcap%
\pgfsetmiterjoin%
\definecolor{currentfill}{rgb}{0.000000,0.647059,0.676471}%
\pgfsetfillcolor{currentfill}%
\pgfsetlinewidth{0.000000pt}%
\definecolor{currentstroke}{rgb}{0.000000,0.000000,0.000000}%
\pgfsetstrokecolor{currentstroke}%
\pgfsetstrokeopacity{0.000000}%
\pgfsetdash{}{0pt}%
\pgfpathmoveto{\pgfqpoint{2.268476in}{1.251773in}}%
\pgfpathlineto{\pgfqpoint{2.260153in}{1.246809in}}%
\pgfpathlineto{\pgfqpoint{2.255244in}{1.254033in}}%
\pgfpathlineto{\pgfqpoint{2.227021in}{1.254178in}}%
\pgfpathlineto{\pgfqpoint{2.227204in}{1.266822in}}%
\pgfpathlineto{\pgfqpoint{2.225942in}{1.276603in}}%
\pgfpathlineto{\pgfqpoint{2.226458in}{1.316665in}}%
\pgfpathlineto{\pgfqpoint{2.240062in}{1.316251in}}%
\pgfpathlineto{\pgfqpoint{2.240098in}{1.309360in}}%
\pgfpathlineto{\pgfqpoint{2.253822in}{1.296576in}}%
\pgfpathlineto{\pgfqpoint{2.267557in}{1.296520in}}%
\pgfpathlineto{\pgfqpoint{2.267441in}{1.289875in}}%
\pgfpathlineto{\pgfqpoint{2.274008in}{1.289693in}}%
\pgfpathlineto{\pgfqpoint{2.280744in}{1.286387in}}%
\pgfpathlineto{\pgfqpoint{2.282058in}{1.264607in}}%
\pgfpathlineto{\pgfqpoint{2.268296in}{1.264580in}}%
\pgfpathlineto{\pgfqpoint{2.268476in}{1.251773in}}%
\pgfpathclose%
\pgfusepath{fill}%
\end{pgfscope}%
\begin{pgfscope}%
\pgfpathrectangle{\pgfqpoint{0.100000in}{0.100000in}}{\pgfqpoint{3.608454in}{2.310000in}}%
\pgfusepath{clip}%
\pgfsetbuttcap%
\pgfsetmiterjoin%
\definecolor{currentfill}{rgb}{0.000000,0.709804,0.645098}%
\pgfsetfillcolor{currentfill}%
\pgfsetlinewidth{0.000000pt}%
\definecolor{currentstroke}{rgb}{0.000000,0.000000,0.000000}%
\pgfsetstrokecolor{currentstroke}%
\pgfsetstrokeopacity{0.000000}%
\pgfsetdash{}{0pt}%
\pgfpathmoveto{\pgfqpoint{2.434616in}{1.075007in}}%
\pgfpathlineto{\pgfqpoint{2.426249in}{1.075405in}}%
\pgfpathlineto{\pgfqpoint{2.427918in}{1.068838in}}%
\pgfpathlineto{\pgfqpoint{2.420217in}{1.066185in}}%
\pgfpathlineto{\pgfqpoint{2.410748in}{1.065840in}}%
\pgfpathlineto{\pgfqpoint{2.408349in}{1.109943in}}%
\pgfpathlineto{\pgfqpoint{2.406026in}{1.117238in}}%
\pgfpathlineto{\pgfqpoint{2.411938in}{1.125058in}}%
\pgfpathlineto{\pgfqpoint{2.420239in}{1.132410in}}%
\pgfpathlineto{\pgfqpoint{2.421101in}{1.141357in}}%
\pgfpathlineto{\pgfqpoint{2.410619in}{1.149798in}}%
\pgfpathlineto{\pgfqpoint{2.414649in}{1.160533in}}%
\pgfpathlineto{\pgfqpoint{2.426666in}{1.161074in}}%
\pgfpathlineto{\pgfqpoint{2.426853in}{1.142086in}}%
\pgfpathlineto{\pgfqpoint{2.440887in}{1.142638in}}%
\pgfpathlineto{\pgfqpoint{2.450987in}{1.144593in}}%
\pgfpathlineto{\pgfqpoint{2.454276in}{1.139497in}}%
\pgfpathlineto{\pgfqpoint{2.455346in}{1.132898in}}%
\pgfpathlineto{\pgfqpoint{2.445412in}{1.130104in}}%
\pgfpathlineto{\pgfqpoint{2.452158in}{1.122751in}}%
\pgfpathlineto{\pgfqpoint{2.446875in}{1.118895in}}%
\pgfpathlineto{\pgfqpoint{2.445220in}{1.109326in}}%
\pgfpathlineto{\pgfqpoint{2.442276in}{1.103468in}}%
\pgfpathlineto{\pgfqpoint{2.442664in}{1.097005in}}%
\pgfpathlineto{\pgfqpoint{2.439328in}{1.092627in}}%
\pgfpathlineto{\pgfqpoint{2.431130in}{1.090720in}}%
\pgfpathlineto{\pgfqpoint{2.432045in}{1.079916in}}%
\pgfpathlineto{\pgfqpoint{2.434616in}{1.075007in}}%
\pgfpathclose%
\pgfusepath{fill}%
\end{pgfscope}%
\begin{pgfscope}%
\pgfpathrectangle{\pgfqpoint{0.100000in}{0.100000in}}{\pgfqpoint{3.608454in}{2.310000in}}%
\pgfusepath{clip}%
\pgfsetbuttcap%
\pgfsetmiterjoin%
\definecolor{currentfill}{rgb}{0.000000,0.552941,0.723529}%
\pgfsetfillcolor{currentfill}%
\pgfsetlinewidth{0.000000pt}%
\definecolor{currentstroke}{rgb}{0.000000,0.000000,0.000000}%
\pgfsetstrokecolor{currentstroke}%
\pgfsetstrokeopacity{0.000000}%
\pgfsetdash{}{0pt}%
\pgfpathmoveto{\pgfqpoint{2.994468in}{0.861760in}}%
\pgfpathlineto{\pgfqpoint{2.991213in}{0.868134in}}%
\pgfpathlineto{\pgfqpoint{2.962520in}{0.870538in}}%
\pgfpathlineto{\pgfqpoint{2.954746in}{0.867311in}}%
\pgfpathlineto{\pgfqpoint{2.950142in}{0.872561in}}%
\pgfpathlineto{\pgfqpoint{2.951258in}{0.878733in}}%
\pgfpathlineto{\pgfqpoint{2.943431in}{0.882150in}}%
\pgfpathlineto{\pgfqpoint{2.932829in}{0.881990in}}%
\pgfpathlineto{\pgfqpoint{2.935640in}{0.887719in}}%
\pgfpathlineto{\pgfqpoint{2.943276in}{0.894232in}}%
\pgfpathlineto{\pgfqpoint{2.940759in}{0.897010in}}%
\pgfpathlineto{\pgfqpoint{2.944266in}{0.903605in}}%
\pgfpathlineto{\pgfqpoint{2.950977in}{0.910809in}}%
\pgfpathlineto{\pgfqpoint{2.959107in}{0.909040in}}%
\pgfpathlineto{\pgfqpoint{2.962570in}{0.910798in}}%
\pgfpathlineto{\pgfqpoint{2.965871in}{0.919067in}}%
\pgfpathlineto{\pgfqpoint{2.980224in}{0.924118in}}%
\pgfpathlineto{\pgfqpoint{2.986683in}{0.921479in}}%
\pgfpathlineto{\pgfqpoint{2.999992in}{0.934399in}}%
\pgfpathlineto{\pgfqpoint{3.003733in}{0.931980in}}%
\pgfpathlineto{\pgfqpoint{3.003885in}{0.926116in}}%
\pgfpathlineto{\pgfqpoint{3.010014in}{0.919433in}}%
\pgfpathlineto{\pgfqpoint{3.011589in}{0.908504in}}%
\pgfpathlineto{\pgfqpoint{3.015378in}{0.900512in}}%
\pgfpathlineto{\pgfqpoint{3.006499in}{0.892231in}}%
\pgfpathlineto{\pgfqpoint{3.008209in}{0.883079in}}%
\pgfpathlineto{\pgfqpoint{3.014555in}{0.877148in}}%
\pgfpathlineto{\pgfqpoint{3.016488in}{0.872188in}}%
\pgfpathlineto{\pgfqpoint{2.994468in}{0.861760in}}%
\pgfpathclose%
\pgfusepath{fill}%
\end{pgfscope}%
\begin{pgfscope}%
\pgfpathrectangle{\pgfqpoint{0.100000in}{0.100000in}}{\pgfqpoint{3.608454in}{2.310000in}}%
\pgfusepath{clip}%
\pgfsetbuttcap%
\pgfsetmiterjoin%
\definecolor{currentfill}{rgb}{0.000000,0.631373,0.684314}%
\pgfsetfillcolor{currentfill}%
\pgfsetlinewidth{0.000000pt}%
\definecolor{currentstroke}{rgb}{0.000000,0.000000,0.000000}%
\pgfsetstrokecolor{currentstroke}%
\pgfsetstrokeopacity{0.000000}%
\pgfsetdash{}{0pt}%
\pgfpathmoveto{\pgfqpoint{1.438374in}{0.821513in}}%
\pgfpathlineto{\pgfqpoint{1.435480in}{0.791748in}}%
\pgfpathlineto{\pgfqpoint{1.429579in}{0.734241in}}%
\pgfpathlineto{\pgfqpoint{1.426837in}{0.715528in}}%
\pgfpathlineto{\pgfqpoint{1.422243in}{0.713258in}}%
\pgfpathlineto{\pgfqpoint{1.413714in}{0.723890in}}%
\pgfpathlineto{\pgfqpoint{1.407758in}{0.729084in}}%
\pgfpathlineto{\pgfqpoint{1.396001in}{0.734660in}}%
\pgfpathlineto{\pgfqpoint{1.395928in}{0.737162in}}%
\pgfpathlineto{\pgfqpoint{1.386189in}{0.746307in}}%
\pgfpathlineto{\pgfqpoint{1.383892in}{0.754060in}}%
\pgfpathlineto{\pgfqpoint{1.373173in}{0.762125in}}%
\pgfpathlineto{\pgfqpoint{1.362767in}{0.779271in}}%
\pgfpathlineto{\pgfqpoint{1.354361in}{0.783205in}}%
\pgfpathlineto{\pgfqpoint{1.346623in}{0.789592in}}%
\pgfpathlineto{\pgfqpoint{1.337607in}{0.811852in}}%
\pgfpathlineto{\pgfqpoint{1.328213in}{0.817202in}}%
\pgfpathlineto{\pgfqpoint{1.276635in}{0.824008in}}%
\pgfpathlineto{\pgfqpoint{1.285282in}{0.888829in}}%
\pgfpathlineto{\pgfqpoint{1.287163in}{0.902597in}}%
\pgfpathlineto{\pgfqpoint{1.314425in}{0.898871in}}%
\pgfpathlineto{\pgfqpoint{1.314868in}{0.902303in}}%
\pgfpathlineto{\pgfqpoint{1.353414in}{0.915777in}}%
\pgfpathlineto{\pgfqpoint{1.350085in}{0.908598in}}%
\pgfpathlineto{\pgfqpoint{1.340564in}{0.833080in}}%
\pgfpathlineto{\pgfqpoint{1.404085in}{0.825375in}}%
\pgfpathlineto{\pgfqpoint{1.438374in}{0.821513in}}%
\pgfpathclose%
\pgfusepath{fill}%
\end{pgfscope}%
\begin{pgfscope}%
\pgfpathrectangle{\pgfqpoint{0.100000in}{0.100000in}}{\pgfqpoint{3.608454in}{2.310000in}}%
\pgfusepath{clip}%
\pgfsetbuttcap%
\pgfsetmiterjoin%
\definecolor{currentfill}{rgb}{0.000000,0.564706,0.717647}%
\pgfsetfillcolor{currentfill}%
\pgfsetlinewidth{0.000000pt}%
\definecolor{currentstroke}{rgb}{0.000000,0.000000,0.000000}%
\pgfsetstrokecolor{currentstroke}%
\pgfsetstrokeopacity{0.000000}%
\pgfsetdash{}{0pt}%
\pgfpathmoveto{\pgfqpoint{2.765201in}{0.880136in}}%
\pgfpathlineto{\pgfqpoint{2.760875in}{0.889582in}}%
\pgfpathlineto{\pgfqpoint{2.782280in}{0.891779in}}%
\pgfpathlineto{\pgfqpoint{2.779766in}{0.919893in}}%
\pgfpathlineto{\pgfqpoint{2.802911in}{0.922322in}}%
\pgfpathlineto{\pgfqpoint{2.802438in}{0.905479in}}%
\pgfpathlineto{\pgfqpoint{2.805495in}{0.895708in}}%
\pgfpathlineto{\pgfqpoint{2.814267in}{0.888921in}}%
\pgfpathlineto{\pgfqpoint{2.821573in}{0.886623in}}%
\pgfpathlineto{\pgfqpoint{2.812799in}{0.870744in}}%
\pgfpathlineto{\pgfqpoint{2.800195in}{0.867107in}}%
\pgfpathlineto{\pgfqpoint{2.799223in}{0.863800in}}%
\pgfpathlineto{\pgfqpoint{2.801485in}{0.843067in}}%
\pgfpathlineto{\pgfqpoint{2.783833in}{0.840916in}}%
\pgfpathlineto{\pgfqpoint{2.775432in}{0.832235in}}%
\pgfpathlineto{\pgfqpoint{2.776392in}{0.826653in}}%
\pgfpathlineto{\pgfqpoint{2.767748in}{0.825602in}}%
\pgfpathlineto{\pgfqpoint{2.762136in}{0.831926in}}%
\pgfpathlineto{\pgfqpoint{2.751930in}{0.830708in}}%
\pgfpathlineto{\pgfqpoint{2.749632in}{0.837509in}}%
\pgfpathlineto{\pgfqpoint{2.748134in}{0.851370in}}%
\pgfpathlineto{\pgfqpoint{2.755065in}{0.852187in}}%
\pgfpathlineto{\pgfqpoint{2.754330in}{0.856732in}}%
\pgfpathlineto{\pgfqpoint{2.772613in}{0.859054in}}%
\pgfpathlineto{\pgfqpoint{2.776169in}{0.862453in}}%
\pgfpathlineto{\pgfqpoint{2.768066in}{0.872209in}}%
\pgfpathlineto{\pgfqpoint{2.765201in}{0.880136in}}%
\pgfpathclose%
\pgfusepath{fill}%
\end{pgfscope}%
\begin{pgfscope}%
\pgfpathrectangle{\pgfqpoint{0.100000in}{0.100000in}}{\pgfqpoint{3.608454in}{2.310000in}}%
\pgfusepath{clip}%
\pgfsetbuttcap%
\pgfsetmiterjoin%
\definecolor{currentfill}{rgb}{0.000000,0.494118,0.752941}%
\pgfsetfillcolor{currentfill}%
\pgfsetlinewidth{0.000000pt}%
\definecolor{currentstroke}{rgb}{0.000000,0.000000,0.000000}%
\pgfsetstrokecolor{currentstroke}%
\pgfsetstrokeopacity{0.000000}%
\pgfsetdash{}{0pt}%
\pgfpathmoveto{\pgfqpoint{2.682505in}{0.872318in}}%
\pgfpathlineto{\pgfqpoint{2.680466in}{0.868375in}}%
\pgfpathlineto{\pgfqpoint{2.660419in}{0.866374in}}%
\pgfpathlineto{\pgfqpoint{2.660793in}{0.862897in}}%
\pgfpathlineto{\pgfqpoint{2.647203in}{0.861864in}}%
\pgfpathlineto{\pgfqpoint{2.640493in}{0.861198in}}%
\pgfpathlineto{\pgfqpoint{2.639182in}{0.874953in}}%
\pgfpathlineto{\pgfqpoint{2.648704in}{0.875737in}}%
\pgfpathlineto{\pgfqpoint{2.647138in}{0.892648in}}%
\pgfpathlineto{\pgfqpoint{2.640666in}{0.895938in}}%
\pgfpathlineto{\pgfqpoint{2.636464in}{0.900943in}}%
\pgfpathlineto{\pgfqpoint{2.635819in}{0.907320in}}%
\pgfpathlineto{\pgfqpoint{2.625197in}{0.912348in}}%
\pgfpathlineto{\pgfqpoint{2.617467in}{0.921208in}}%
\pgfpathlineto{\pgfqpoint{2.618130in}{0.927031in}}%
\pgfpathlineto{\pgfqpoint{2.623213in}{0.935505in}}%
\pgfpathlineto{\pgfqpoint{2.628827in}{0.937113in}}%
\pgfpathlineto{\pgfqpoint{2.631914in}{0.941959in}}%
\pgfpathlineto{\pgfqpoint{2.632392in}{0.948970in}}%
\pgfpathlineto{\pgfqpoint{2.635731in}{0.956087in}}%
\pgfpathlineto{\pgfqpoint{2.627618in}{0.959226in}}%
\pgfpathlineto{\pgfqpoint{2.625447in}{0.963052in}}%
\pgfpathlineto{\pgfqpoint{2.623076in}{0.990093in}}%
\pgfpathlineto{\pgfqpoint{2.622963in}{0.991238in}}%
\pgfpathlineto{\pgfqpoint{2.657488in}{0.993619in}}%
\pgfpathlineto{\pgfqpoint{2.664313in}{0.994103in}}%
\pgfpathlineto{\pgfqpoint{2.677206in}{0.979100in}}%
\pgfpathlineto{\pgfqpoint{2.675715in}{0.974764in}}%
\pgfpathlineto{\pgfqpoint{2.678957in}{0.970072in}}%
\pgfpathlineto{\pgfqpoint{2.684888in}{0.971060in}}%
\pgfpathlineto{\pgfqpoint{2.689264in}{0.964347in}}%
\pgfpathlineto{\pgfqpoint{2.694783in}{0.960330in}}%
\pgfpathlineto{\pgfqpoint{2.696367in}{0.955422in}}%
\pgfpathlineto{\pgfqpoint{2.690775in}{0.946928in}}%
\pgfpathlineto{\pgfqpoint{2.686513in}{0.939240in}}%
\pgfpathlineto{\pgfqpoint{2.688862in}{0.935943in}}%
\pgfpathlineto{\pgfqpoint{2.677615in}{0.922675in}}%
\pgfpathlineto{\pgfqpoint{2.679726in}{0.915262in}}%
\pgfpathlineto{\pgfqpoint{2.673491in}{0.910352in}}%
\pgfpathlineto{\pgfqpoint{2.670991in}{0.904805in}}%
\pgfpathlineto{\pgfqpoint{2.672768in}{0.884091in}}%
\pgfpathlineto{\pgfqpoint{2.676747in}{0.876531in}}%
\pgfpathlineto{\pgfqpoint{2.682505in}{0.872318in}}%
\pgfpathclose%
\pgfusepath{fill}%
\end{pgfscope}%
\begin{pgfscope}%
\pgfpathrectangle{\pgfqpoint{0.100000in}{0.100000in}}{\pgfqpoint{3.608454in}{2.310000in}}%
\pgfusepath{clip}%
\pgfsetbuttcap%
\pgfsetmiterjoin%
\definecolor{currentfill}{rgb}{0.000000,0.458824,0.770588}%
\pgfsetfillcolor{currentfill}%
\pgfsetlinewidth{0.000000pt}%
\definecolor{currentstroke}{rgb}{0.000000,0.000000,0.000000}%
\pgfsetstrokecolor{currentstroke}%
\pgfsetstrokeopacity{0.000000}%
\pgfsetdash{}{0pt}%
\pgfpathmoveto{\pgfqpoint{2.032756in}{2.134709in}}%
\pgfpathlineto{\pgfqpoint{2.032428in}{2.112015in}}%
\pgfpathlineto{\pgfqpoint{2.033124in}{2.098129in}}%
\pgfpathlineto{\pgfqpoint{1.991978in}{2.098902in}}%
\pgfpathlineto{\pgfqpoint{1.993550in}{2.092414in}}%
\pgfpathlineto{\pgfqpoint{1.991966in}{2.086284in}}%
\pgfpathlineto{\pgfqpoint{1.993953in}{2.077757in}}%
\pgfpathlineto{\pgfqpoint{1.992478in}{2.070813in}}%
\pgfpathlineto{\pgfqpoint{1.951799in}{2.072044in}}%
\pgfpathlineto{\pgfqpoint{1.931043in}{2.072760in}}%
\pgfpathlineto{\pgfqpoint{1.931568in}{2.086739in}}%
\pgfpathlineto{\pgfqpoint{1.930695in}{2.100743in}}%
\pgfpathlineto{\pgfqpoint{1.896095in}{2.102164in}}%
\pgfpathlineto{\pgfqpoint{1.895146in}{2.116164in}}%
\pgfpathlineto{\pgfqpoint{1.896146in}{2.138772in}}%
\pgfpathlineto{\pgfqpoint{1.946219in}{2.136867in}}%
\pgfpathlineto{\pgfqpoint{1.989377in}{2.135633in}}%
\pgfpathlineto{\pgfqpoint{2.032756in}{2.134709in}}%
\pgfpathclose%
\pgfusepath{fill}%
\end{pgfscope}%
\begin{pgfscope}%
\pgfpathrectangle{\pgfqpoint{0.100000in}{0.100000in}}{\pgfqpoint{3.608454in}{2.310000in}}%
\pgfusepath{clip}%
\pgfsetbuttcap%
\pgfsetmiterjoin%
\definecolor{currentfill}{rgb}{0.000000,0.592157,0.703922}%
\pgfsetfillcolor{currentfill}%
\pgfsetlinewidth{0.000000pt}%
\definecolor{currentstroke}{rgb}{0.000000,0.000000,0.000000}%
\pgfsetstrokecolor{currentstroke}%
\pgfsetstrokeopacity{0.000000}%
\pgfsetdash{}{0pt}%
\pgfpathmoveto{\pgfqpoint{1.805378in}{0.356576in}}%
\pgfpathlineto{\pgfqpoint{1.803098in}{0.369070in}}%
\pgfpathlineto{\pgfqpoint{1.799243in}{0.378831in}}%
\pgfpathlineto{\pgfqpoint{1.791844in}{0.386485in}}%
\pgfpathlineto{\pgfqpoint{1.787001in}{0.397445in}}%
\pgfpathlineto{\pgfqpoint{1.789197in}{0.406267in}}%
\pgfpathlineto{\pgfqpoint{1.784237in}{0.419600in}}%
\pgfpathlineto{\pgfqpoint{1.786564in}{0.431356in}}%
\pgfpathlineto{\pgfqpoint{1.783350in}{0.432395in}}%
\pgfpathlineto{\pgfqpoint{1.782660in}{0.441230in}}%
\pgfpathlineto{\pgfqpoint{1.770087in}{0.446743in}}%
\pgfpathlineto{\pgfqpoint{1.764022in}{0.455756in}}%
\pgfpathlineto{\pgfqpoint{1.759560in}{0.458124in}}%
\pgfpathlineto{\pgfqpoint{1.758847in}{0.466132in}}%
\pgfpathlineto{\pgfqpoint{1.752364in}{0.474146in}}%
\pgfpathlineto{\pgfqpoint{1.746457in}{0.486543in}}%
\pgfpathlineto{\pgfqpoint{1.737580in}{0.491300in}}%
\pgfpathlineto{\pgfqpoint{1.744553in}{0.490983in}}%
\pgfpathlineto{\pgfqpoint{1.795506in}{0.488863in}}%
\pgfpathlineto{\pgfqpoint{1.795200in}{0.474847in}}%
\pgfpathlineto{\pgfqpoint{1.836840in}{0.475148in}}%
\pgfpathlineto{\pgfqpoint{1.834828in}{0.418520in}}%
\pgfpathlineto{\pgfqpoint{1.852427in}{0.418217in}}%
\pgfpathlineto{\pgfqpoint{1.856313in}{0.408047in}}%
\pgfpathlineto{\pgfqpoint{1.857651in}{0.393438in}}%
\pgfpathlineto{\pgfqpoint{1.861148in}{0.393352in}}%
\pgfpathlineto{\pgfqpoint{1.859982in}{0.371433in}}%
\pgfpathlineto{\pgfqpoint{1.821677in}{0.373163in}}%
\pgfpathlineto{\pgfqpoint{1.817184in}{0.364409in}}%
\pgfpathlineto{\pgfqpoint{1.805378in}{0.356576in}}%
\pgfpathclose%
\pgfusepath{fill}%
\end{pgfscope}%
\begin{pgfscope}%
\pgfpathrectangle{\pgfqpoint{0.100000in}{0.100000in}}{\pgfqpoint{3.608454in}{2.310000in}}%
\pgfusepath{clip}%
\pgfsetbuttcap%
\pgfsetmiterjoin%
\definecolor{currentfill}{rgb}{0.000000,0.596078,0.701961}%
\pgfsetfillcolor{currentfill}%
\pgfsetlinewidth{0.000000pt}%
\definecolor{currentstroke}{rgb}{0.000000,0.000000,0.000000}%
\pgfsetstrokecolor{currentstroke}%
\pgfsetstrokeopacity{0.000000}%
\pgfsetdash{}{0pt}%
\pgfpathmoveto{\pgfqpoint{2.983960in}{0.768316in}}%
\pgfpathlineto{\pgfqpoint{2.977965in}{0.773429in}}%
\pgfpathlineto{\pgfqpoint{2.977554in}{0.783183in}}%
\pgfpathlineto{\pgfqpoint{2.983748in}{0.789742in}}%
\pgfpathlineto{\pgfqpoint{2.990030in}{0.793119in}}%
\pgfpathlineto{\pgfqpoint{2.986023in}{0.797823in}}%
\pgfpathlineto{\pgfqpoint{2.994265in}{0.796799in}}%
\pgfpathlineto{\pgfqpoint{3.002283in}{0.799408in}}%
\pgfpathlineto{\pgfqpoint{3.007825in}{0.797264in}}%
\pgfpathlineto{\pgfqpoint{3.013726in}{0.809687in}}%
\pgfpathlineto{\pgfqpoint{3.009224in}{0.814410in}}%
\pgfpathlineto{\pgfqpoint{3.015577in}{0.818559in}}%
\pgfpathlineto{\pgfqpoint{3.019148in}{0.828947in}}%
\pgfpathlineto{\pgfqpoint{3.023851in}{0.824906in}}%
\pgfpathlineto{\pgfqpoint{3.035156in}{0.827393in}}%
\pgfpathlineto{\pgfqpoint{3.040729in}{0.821509in}}%
\pgfpathlineto{\pgfqpoint{3.042224in}{0.818383in}}%
\pgfpathlineto{\pgfqpoint{3.038091in}{0.806725in}}%
\pgfpathlineto{\pgfqpoint{3.040053in}{0.796620in}}%
\pgfpathlineto{\pgfqpoint{3.031881in}{0.783770in}}%
\pgfpathlineto{\pgfqpoint{3.030760in}{0.775066in}}%
\pgfpathlineto{\pgfqpoint{3.020788in}{0.780741in}}%
\pgfpathlineto{\pgfqpoint{3.005604in}{0.782838in}}%
\pgfpathlineto{\pgfqpoint{2.996804in}{0.773153in}}%
\pgfpathlineto{\pgfqpoint{2.988034in}{0.774133in}}%
\pgfpathlineto{\pgfqpoint{2.983960in}{0.768316in}}%
\pgfpathclose%
\pgfusepath{fill}%
\end{pgfscope}%
\begin{pgfscope}%
\pgfpathrectangle{\pgfqpoint{0.100000in}{0.100000in}}{\pgfqpoint{3.608454in}{2.310000in}}%
\pgfusepath{clip}%
\pgfsetbuttcap%
\pgfsetmiterjoin%
\definecolor{currentfill}{rgb}{0.000000,0.364706,0.817647}%
\pgfsetfillcolor{currentfill}%
\pgfsetlinewidth{0.000000pt}%
\definecolor{currentstroke}{rgb}{0.000000,0.000000,0.000000}%
\pgfsetstrokecolor{currentstroke}%
\pgfsetstrokeopacity{0.000000}%
\pgfsetdash{}{0pt}%
\pgfpathmoveto{\pgfqpoint{1.571835in}{1.569055in}}%
\pgfpathlineto{\pgfqpoint{1.580064in}{1.651542in}}%
\pgfpathlineto{\pgfqpoint{1.583107in}{1.682203in}}%
\pgfpathlineto{\pgfqpoint{1.643808in}{1.676503in}}%
\pgfpathlineto{\pgfqpoint{1.655854in}{1.675393in}}%
\pgfpathlineto{\pgfqpoint{1.655483in}{1.658354in}}%
\pgfpathlineto{\pgfqpoint{1.653148in}{1.631106in}}%
\pgfpathlineto{\pgfqpoint{1.654543in}{1.630993in}}%
\pgfpathlineto{\pgfqpoint{1.652617in}{1.603533in}}%
\pgfpathlineto{\pgfqpoint{1.655822in}{1.596346in}}%
\pgfpathlineto{\pgfqpoint{1.654522in}{1.581965in}}%
\pgfpathlineto{\pgfqpoint{1.656181in}{1.573226in}}%
\pgfpathlineto{\pgfqpoint{1.654457in}{1.551499in}}%
\pgfpathlineto{\pgfqpoint{1.611187in}{1.555209in}}%
\pgfpathlineto{\pgfqpoint{1.610795in}{1.551818in}}%
\pgfpathlineto{\pgfqpoint{1.570505in}{1.555595in}}%
\pgfpathlineto{\pgfqpoint{1.571835in}{1.569055in}}%
\pgfpathclose%
\pgfusepath{fill}%
\end{pgfscope}%
\begin{pgfscope}%
\pgfpathrectangle{\pgfqpoint{0.100000in}{0.100000in}}{\pgfqpoint{3.608454in}{2.310000in}}%
\pgfusepath{clip}%
\pgfsetbuttcap%
\pgfsetmiterjoin%
\definecolor{currentfill}{rgb}{0.000000,0.937255,0.531373}%
\pgfsetfillcolor{currentfill}%
\pgfsetlinewidth{0.000000pt}%
\definecolor{currentstroke}{rgb}{0.000000,0.000000,0.000000}%
\pgfsetstrokecolor{currentstroke}%
\pgfsetstrokeopacity{0.000000}%
\pgfsetdash{}{0pt}%
\pgfpathmoveto{\pgfqpoint{2.393376in}{0.840961in}}%
\pgfpathlineto{\pgfqpoint{2.393171in}{0.837522in}}%
\pgfpathlineto{\pgfqpoint{2.384135in}{0.837054in}}%
\pgfpathlineto{\pgfqpoint{2.384564in}{0.826478in}}%
\pgfpathlineto{\pgfqpoint{2.378401in}{0.826111in}}%
\pgfpathlineto{\pgfqpoint{2.377206in}{0.836777in}}%
\pgfpathlineto{\pgfqpoint{2.371879in}{0.836595in}}%
\pgfpathlineto{\pgfqpoint{2.366196in}{0.837164in}}%
\pgfpathlineto{\pgfqpoint{2.364353in}{0.841406in}}%
\pgfpathlineto{\pgfqpoint{2.370206in}{0.848561in}}%
\pgfpathlineto{\pgfqpoint{2.363126in}{0.851002in}}%
\pgfpathlineto{\pgfqpoint{2.363366in}{0.855813in}}%
\pgfpathlineto{\pgfqpoint{2.368994in}{0.861333in}}%
\pgfpathlineto{\pgfqpoint{2.367614in}{0.868244in}}%
\pgfpathlineto{\pgfqpoint{2.362240in}{0.870230in}}%
\pgfpathlineto{\pgfqpoint{2.366771in}{0.881093in}}%
\pgfpathlineto{\pgfqpoint{2.366452in}{0.887645in}}%
\pgfpathlineto{\pgfqpoint{2.362887in}{0.893699in}}%
\pgfpathlineto{\pgfqpoint{2.362531in}{0.900120in}}%
\pgfpathlineto{\pgfqpoint{2.356517in}{0.914773in}}%
\pgfpathlineto{\pgfqpoint{2.362379in}{0.917175in}}%
\pgfpathlineto{\pgfqpoint{2.356747in}{0.922152in}}%
\pgfpathlineto{\pgfqpoint{2.361535in}{0.931873in}}%
\pgfpathlineto{\pgfqpoint{2.370618in}{0.931411in}}%
\pgfpathlineto{\pgfqpoint{2.365618in}{0.938943in}}%
\pgfpathlineto{\pgfqpoint{2.369367in}{0.943780in}}%
\pgfpathlineto{\pgfqpoint{2.364126in}{0.947506in}}%
\pgfpathlineto{\pgfqpoint{2.376340in}{0.951894in}}%
\pgfpathlineto{\pgfqpoint{2.377959in}{0.956142in}}%
\pgfpathlineto{\pgfqpoint{2.372346in}{0.959262in}}%
\pgfpathlineto{\pgfqpoint{2.372289in}{0.959362in}}%
\pgfpathlineto{\pgfqpoint{2.391745in}{0.960160in}}%
\pgfpathlineto{\pgfqpoint{2.392184in}{0.949754in}}%
\pgfpathlineto{\pgfqpoint{2.405968in}{0.950277in}}%
\pgfpathlineto{\pgfqpoint{2.406598in}{0.936460in}}%
\pgfpathlineto{\pgfqpoint{2.408409in}{0.898240in}}%
\pgfpathlineto{\pgfqpoint{2.401625in}{0.897879in}}%
\pgfpathlineto{\pgfqpoint{2.401868in}{0.893227in}}%
\pgfpathlineto{\pgfqpoint{2.391193in}{0.892674in}}%
\pgfpathlineto{\pgfqpoint{2.395715in}{0.879087in}}%
\pgfpathlineto{\pgfqpoint{2.396470in}{0.865278in}}%
\pgfpathlineto{\pgfqpoint{2.389913in}{0.858059in}}%
\pgfpathlineto{\pgfqpoint{2.393315in}{0.855864in}}%
\pgfpathlineto{\pgfqpoint{2.393376in}{0.840961in}}%
\pgfpathclose%
\pgfusepath{fill}%
\end{pgfscope}%
\begin{pgfscope}%
\pgfpathrectangle{\pgfqpoint{0.100000in}{0.100000in}}{\pgfqpoint{3.608454in}{2.310000in}}%
\pgfusepath{clip}%
\pgfsetbuttcap%
\pgfsetmiterjoin%
\definecolor{currentfill}{rgb}{0.000000,0.733333,0.633333}%
\pgfsetfillcolor{currentfill}%
\pgfsetlinewidth{0.000000pt}%
\definecolor{currentstroke}{rgb}{0.000000,0.000000,0.000000}%
\pgfsetstrokecolor{currentstroke}%
\pgfsetstrokeopacity{0.000000}%
\pgfsetdash{}{0pt}%
\pgfpathmoveto{\pgfqpoint{0.480357in}{2.020001in}}%
\pgfpathlineto{\pgfqpoint{0.492958in}{2.047705in}}%
\pgfpathlineto{\pgfqpoint{0.494929in}{2.056692in}}%
\pgfpathlineto{\pgfqpoint{0.510061in}{2.083946in}}%
\pgfpathlineto{\pgfqpoint{0.514910in}{2.098885in}}%
\pgfpathlineto{\pgfqpoint{0.524396in}{2.121590in}}%
\pgfpathlineto{\pgfqpoint{0.525797in}{2.131179in}}%
\pgfpathlineto{\pgfqpoint{0.538739in}{2.126084in}}%
\pgfpathlineto{\pgfqpoint{0.557819in}{2.120432in}}%
\pgfpathlineto{\pgfqpoint{0.556089in}{2.115055in}}%
\pgfpathlineto{\pgfqpoint{0.551015in}{2.110628in}}%
\pgfpathlineto{\pgfqpoint{0.550017in}{2.103751in}}%
\pgfpathlineto{\pgfqpoint{0.543848in}{2.096029in}}%
\pgfpathlineto{\pgfqpoint{0.538605in}{2.079539in}}%
\pgfpathlineto{\pgfqpoint{0.521499in}{2.085093in}}%
\pgfpathlineto{\pgfqpoint{0.518018in}{2.074545in}}%
\pgfpathlineto{\pgfqpoint{0.521244in}{2.073470in}}%
\pgfpathlineto{\pgfqpoint{0.512881in}{2.048028in}}%
\pgfpathlineto{\pgfqpoint{0.519006in}{2.044536in}}%
\pgfpathlineto{\pgfqpoint{0.512206in}{2.022672in}}%
\pgfpathlineto{\pgfqpoint{0.505570in}{2.024850in}}%
\pgfpathlineto{\pgfqpoint{0.504063in}{2.019289in}}%
\pgfpathlineto{\pgfqpoint{0.498912in}{2.014543in}}%
\pgfpathlineto{\pgfqpoint{0.480357in}{2.020001in}}%
\pgfpathclose%
\pgfusepath{fill}%
\end{pgfscope}%
\begin{pgfscope}%
\pgfpathrectangle{\pgfqpoint{0.100000in}{0.100000in}}{\pgfqpoint{3.608454in}{2.310000in}}%
\pgfusepath{clip}%
\pgfsetbuttcap%
\pgfsetmiterjoin%
\definecolor{currentfill}{rgb}{0.000000,0.827451,0.586275}%
\pgfsetfillcolor{currentfill}%
\pgfsetlinewidth{0.000000pt}%
\definecolor{currentstroke}{rgb}{0.000000,0.000000,0.000000}%
\pgfsetstrokecolor{currentstroke}%
\pgfsetstrokeopacity{0.000000}%
\pgfsetdash{}{0pt}%
\pgfpathmoveto{\pgfqpoint{0.375397in}{0.571894in}}%
\pgfpathlineto{\pgfqpoint{0.374645in}{0.576884in}}%
\pgfpathlineto{\pgfqpoint{0.375793in}{0.577086in}}%
\pgfpathlineto{\pgfqpoint{0.375397in}{0.571894in}}%
\pgfpathclose%
\pgfusepath{fill}%
\end{pgfscope}%
\begin{pgfscope}%
\pgfpathrectangle{\pgfqpoint{0.100000in}{0.100000in}}{\pgfqpoint{3.608454in}{2.310000in}}%
\pgfusepath{clip}%
\pgfsetbuttcap%
\pgfsetmiterjoin%
\definecolor{currentfill}{rgb}{0.000000,0.827451,0.586275}%
\pgfsetfillcolor{currentfill}%
\pgfsetlinewidth{0.000000pt}%
\definecolor{currentstroke}{rgb}{0.000000,0.000000,0.000000}%
\pgfsetstrokecolor{currentstroke}%
\pgfsetstrokeopacity{0.000000}%
\pgfsetdash{}{0pt}%
\pgfpathmoveto{\pgfqpoint{0.409283in}{0.538102in}}%
\pgfpathlineto{\pgfqpoint{0.408841in}{0.542212in}}%
\pgfpathlineto{\pgfqpoint{0.410919in}{0.540658in}}%
\pgfpathlineto{\pgfqpoint{0.412176in}{0.541182in}}%
\pgfpathlineto{\pgfqpoint{0.411443in}{0.546443in}}%
\pgfpathlineto{\pgfqpoint{0.412787in}{0.546879in}}%
\pgfpathlineto{\pgfqpoint{0.415145in}{0.543509in}}%
\pgfpathlineto{\pgfqpoint{0.414661in}{0.542347in}}%
\pgfpathlineto{\pgfqpoint{0.415729in}{0.539263in}}%
\pgfpathlineto{\pgfqpoint{0.410458in}{0.539171in}}%
\pgfpathlineto{\pgfqpoint{0.409283in}{0.538102in}}%
\pgfpathclose%
\pgfusepath{fill}%
\end{pgfscope}%
\begin{pgfscope}%
\pgfpathrectangle{\pgfqpoint{0.100000in}{0.100000in}}{\pgfqpoint{3.608454in}{2.310000in}}%
\pgfusepath{clip}%
\pgfsetbuttcap%
\pgfsetmiterjoin%
\definecolor{currentfill}{rgb}{0.000000,0.827451,0.586275}%
\pgfsetfillcolor{currentfill}%
\pgfsetlinewidth{0.000000pt}%
\definecolor{currentstroke}{rgb}{0.000000,0.000000,0.000000}%
\pgfsetstrokecolor{currentstroke}%
\pgfsetstrokeopacity{0.000000}%
\pgfsetdash{}{0pt}%
\pgfpathmoveto{\pgfqpoint{0.420907in}{0.523072in}}%
\pgfpathlineto{\pgfqpoint{0.420394in}{0.526482in}}%
\pgfpathlineto{\pgfqpoint{0.425789in}{0.526498in}}%
\pgfpathlineto{\pgfqpoint{0.427996in}{0.528757in}}%
\pgfpathlineto{\pgfqpoint{0.429378in}{0.528347in}}%
\pgfpathlineto{\pgfqpoint{0.431337in}{0.526034in}}%
\pgfpathlineto{\pgfqpoint{0.429229in}{0.525606in}}%
\pgfpathlineto{\pgfqpoint{0.428074in}{0.524089in}}%
\pgfpathlineto{\pgfqpoint{0.429993in}{0.521225in}}%
\pgfpathlineto{\pgfqpoint{0.430475in}{0.519141in}}%
\pgfpathlineto{\pgfqpoint{0.428313in}{0.519015in}}%
\pgfpathlineto{\pgfqpoint{0.427037in}{0.520588in}}%
\pgfpathlineto{\pgfqpoint{0.423736in}{0.522918in}}%
\pgfpathlineto{\pgfqpoint{0.420907in}{0.523072in}}%
\pgfpathclose%
\pgfusepath{fill}%
\end{pgfscope}%
\begin{pgfscope}%
\pgfpathrectangle{\pgfqpoint{0.100000in}{0.100000in}}{\pgfqpoint{3.608454in}{2.310000in}}%
\pgfusepath{clip}%
\pgfsetbuttcap%
\pgfsetmiterjoin%
\definecolor{currentfill}{rgb}{0.000000,0.827451,0.586275}%
\pgfsetfillcolor{currentfill}%
\pgfsetlinewidth{0.000000pt}%
\definecolor{currentstroke}{rgb}{0.000000,0.000000,0.000000}%
\pgfsetstrokecolor{currentstroke}%
\pgfsetstrokeopacity{0.000000}%
\pgfsetdash{}{0pt}%
\pgfpathmoveto{\pgfqpoint{0.417557in}{0.530445in}}%
\pgfpathlineto{\pgfqpoint{0.417550in}{0.532867in}}%
\pgfpathlineto{\pgfqpoint{0.412931in}{0.535157in}}%
\pgfpathlineto{\pgfqpoint{0.414886in}{0.536599in}}%
\pgfpathlineto{\pgfqpoint{0.417623in}{0.533713in}}%
\pgfpathlineto{\pgfqpoint{0.421451in}{0.532232in}}%
\pgfpathlineto{\pgfqpoint{0.423992in}{0.533859in}}%
\pgfpathlineto{\pgfqpoint{0.424555in}{0.531246in}}%
\pgfpathlineto{\pgfqpoint{0.421848in}{0.531121in}}%
\pgfpathlineto{\pgfqpoint{0.419783in}{0.529378in}}%
\pgfpathlineto{\pgfqpoint{0.417557in}{0.530445in}}%
\pgfpathclose%
\pgfusepath{fill}%
\end{pgfscope}%
\begin{pgfscope}%
\pgfpathrectangle{\pgfqpoint{0.100000in}{0.100000in}}{\pgfqpoint{3.608454in}{2.310000in}}%
\pgfusepath{clip}%
\pgfsetbuttcap%
\pgfsetmiterjoin%
\definecolor{currentfill}{rgb}{0.000000,0.827451,0.586275}%
\pgfsetfillcolor{currentfill}%
\pgfsetlinewidth{0.000000pt}%
\definecolor{currentstroke}{rgb}{0.000000,0.000000,0.000000}%
\pgfsetstrokecolor{currentstroke}%
\pgfsetstrokeopacity{0.000000}%
\pgfsetdash{}{0pt}%
\pgfpathmoveto{\pgfqpoint{0.389946in}{0.575682in}}%
\pgfpathlineto{\pgfqpoint{0.389432in}{0.578330in}}%
\pgfpathlineto{\pgfqpoint{0.392641in}{0.578848in}}%
\pgfpathlineto{\pgfqpoint{0.393455in}{0.577648in}}%
\pgfpathlineto{\pgfqpoint{0.392694in}{0.575332in}}%
\pgfpathlineto{\pgfqpoint{0.389946in}{0.575682in}}%
\pgfpathclose%
\pgfusepath{fill}%
\end{pgfscope}%
\begin{pgfscope}%
\pgfpathrectangle{\pgfqpoint{0.100000in}{0.100000in}}{\pgfqpoint{3.608454in}{2.310000in}}%
\pgfusepath{clip}%
\pgfsetbuttcap%
\pgfsetmiterjoin%
\definecolor{currentfill}{rgb}{0.000000,0.827451,0.586275}%
\pgfsetfillcolor{currentfill}%
\pgfsetlinewidth{0.000000pt}%
\definecolor{currentstroke}{rgb}{0.000000,0.000000,0.000000}%
\pgfsetstrokecolor{currentstroke}%
\pgfsetstrokeopacity{0.000000}%
\pgfsetdash{}{0pt}%
\pgfpathmoveto{\pgfqpoint{0.432376in}{0.516958in}}%
\pgfpathlineto{\pgfqpoint{0.431679in}{0.519437in}}%
\pgfpathlineto{\pgfqpoint{0.436217in}{0.517285in}}%
\pgfpathlineto{\pgfqpoint{0.434890in}{0.515977in}}%
\pgfpathlineto{\pgfqpoint{0.432376in}{0.516958in}}%
\pgfpathclose%
\pgfusepath{fill}%
\end{pgfscope}%
\begin{pgfscope}%
\pgfpathrectangle{\pgfqpoint{0.100000in}{0.100000in}}{\pgfqpoint{3.608454in}{2.310000in}}%
\pgfusepath{clip}%
\pgfsetbuttcap%
\pgfsetmiterjoin%
\definecolor{currentfill}{rgb}{0.000000,0.827451,0.586275}%
\pgfsetfillcolor{currentfill}%
\pgfsetlinewidth{0.000000pt}%
\definecolor{currentstroke}{rgb}{0.000000,0.000000,0.000000}%
\pgfsetstrokecolor{currentstroke}%
\pgfsetstrokeopacity{0.000000}%
\pgfsetdash{}{0pt}%
\pgfpathmoveto{\pgfqpoint{0.378939in}{0.591628in}}%
\pgfpathlineto{\pgfqpoint{0.379145in}{0.593528in}}%
\pgfpathlineto{\pgfqpoint{0.381451in}{0.592746in}}%
\pgfpathlineto{\pgfqpoint{0.380887in}{0.590969in}}%
\pgfpathlineto{\pgfqpoint{0.378939in}{0.591628in}}%
\pgfpathclose%
\pgfusepath{fill}%
\end{pgfscope}%
\begin{pgfscope}%
\pgfpathrectangle{\pgfqpoint{0.100000in}{0.100000in}}{\pgfqpoint{3.608454in}{2.310000in}}%
\pgfusepath{clip}%
\pgfsetbuttcap%
\pgfsetmiterjoin%
\definecolor{currentfill}{rgb}{0.000000,0.827451,0.586275}%
\pgfsetfillcolor{currentfill}%
\pgfsetlinewidth{0.000000pt}%
\definecolor{currentstroke}{rgb}{0.000000,0.000000,0.000000}%
\pgfsetstrokecolor{currentstroke}%
\pgfsetstrokeopacity{0.000000}%
\pgfsetdash{}{0pt}%
\pgfpathmoveto{\pgfqpoint{0.436549in}{0.519300in}}%
\pgfpathlineto{\pgfqpoint{0.436374in}{0.521884in}}%
\pgfpathlineto{\pgfqpoint{0.438554in}{0.523113in}}%
\pgfpathlineto{\pgfqpoint{0.439378in}{0.520715in}}%
\pgfpathlineto{\pgfqpoint{0.436549in}{0.519300in}}%
\pgfpathclose%
\pgfusepath{fill}%
\end{pgfscope}%
\begin{pgfscope}%
\pgfpathrectangle{\pgfqpoint{0.100000in}{0.100000in}}{\pgfqpoint{3.608454in}{2.310000in}}%
\pgfusepath{clip}%
\pgfsetbuttcap%
\pgfsetmiterjoin%
\definecolor{currentfill}{rgb}{0.000000,0.827451,0.586275}%
\pgfsetfillcolor{currentfill}%
\pgfsetlinewidth{0.000000pt}%
\definecolor{currentstroke}{rgb}{0.000000,0.000000,0.000000}%
\pgfsetstrokecolor{currentstroke}%
\pgfsetstrokeopacity{0.000000}%
\pgfsetdash{}{0pt}%
\pgfpathmoveto{\pgfqpoint{0.365932in}{0.607246in}}%
\pgfpathlineto{\pgfqpoint{0.366004in}{0.609622in}}%
\pgfpathlineto{\pgfqpoint{0.369338in}{0.609095in}}%
\pgfpathlineto{\pgfqpoint{0.371550in}{0.607762in}}%
\pgfpathlineto{\pgfqpoint{0.374186in}{0.608460in}}%
\pgfpathlineto{\pgfqpoint{0.375594in}{0.606756in}}%
\pgfpathlineto{\pgfqpoint{0.371710in}{0.606454in}}%
\pgfpathlineto{\pgfqpoint{0.370752in}{0.604856in}}%
\pgfpathlineto{\pgfqpoint{0.368700in}{0.607913in}}%
\pgfpathlineto{\pgfqpoint{0.365932in}{0.607246in}}%
\pgfpathclose%
\pgfusepath{fill}%
\end{pgfscope}%
\begin{pgfscope}%
\pgfpathrectangle{\pgfqpoint{0.100000in}{0.100000in}}{\pgfqpoint{3.608454in}{2.310000in}}%
\pgfusepath{clip}%
\pgfsetbuttcap%
\pgfsetmiterjoin%
\definecolor{currentfill}{rgb}{0.000000,0.827451,0.586275}%
\pgfsetfillcolor{currentfill}%
\pgfsetlinewidth{0.000000pt}%
\definecolor{currentstroke}{rgb}{0.000000,0.000000,0.000000}%
\pgfsetstrokecolor{currentstroke}%
\pgfsetstrokeopacity{0.000000}%
\pgfsetdash{}{0pt}%
\pgfpathmoveto{\pgfqpoint{0.480883in}{0.478483in}}%
\pgfpathlineto{\pgfqpoint{0.480302in}{0.480201in}}%
\pgfpathlineto{\pgfqpoint{0.484465in}{0.480081in}}%
\pgfpathlineto{\pgfqpoint{0.485718in}{0.477885in}}%
\pgfpathlineto{\pgfqpoint{0.484820in}{0.476951in}}%
\pgfpathlineto{\pgfqpoint{0.480883in}{0.478483in}}%
\pgfpathclose%
\pgfusepath{fill}%
\end{pgfscope}%
\begin{pgfscope}%
\pgfpathrectangle{\pgfqpoint{0.100000in}{0.100000in}}{\pgfqpoint{3.608454in}{2.310000in}}%
\pgfusepath{clip}%
\pgfsetbuttcap%
\pgfsetmiterjoin%
\definecolor{currentfill}{rgb}{0.000000,0.827451,0.586275}%
\pgfsetfillcolor{currentfill}%
\pgfsetlinewidth{0.000000pt}%
\definecolor{currentstroke}{rgb}{0.000000,0.000000,0.000000}%
\pgfsetstrokecolor{currentstroke}%
\pgfsetstrokeopacity{0.000000}%
\pgfsetdash{}{0pt}%
\pgfpathmoveto{\pgfqpoint{0.344293in}{0.666210in}}%
\pgfpathlineto{\pgfqpoint{0.343816in}{0.669244in}}%
\pgfpathlineto{\pgfqpoint{0.344625in}{0.670409in}}%
\pgfpathlineto{\pgfqpoint{0.346843in}{0.669814in}}%
\pgfpathlineto{\pgfqpoint{0.346503in}{0.668280in}}%
\pgfpathlineto{\pgfqpoint{0.344293in}{0.666210in}}%
\pgfpathclose%
\pgfusepath{fill}%
\end{pgfscope}%
\begin{pgfscope}%
\pgfpathrectangle{\pgfqpoint{0.100000in}{0.100000in}}{\pgfqpoint{3.608454in}{2.310000in}}%
\pgfusepath{clip}%
\pgfsetbuttcap%
\pgfsetmiterjoin%
\definecolor{currentfill}{rgb}{0.000000,0.827451,0.586275}%
\pgfsetfillcolor{currentfill}%
\pgfsetlinewidth{0.000000pt}%
\definecolor{currentstroke}{rgb}{0.000000,0.000000,0.000000}%
\pgfsetstrokecolor{currentstroke}%
\pgfsetstrokeopacity{0.000000}%
\pgfsetdash{}{0pt}%
\pgfpathmoveto{\pgfqpoint{0.506806in}{0.461794in}}%
\pgfpathlineto{\pgfqpoint{0.507677in}{0.463806in}}%
\pgfpathlineto{\pgfqpoint{0.511473in}{0.462938in}}%
\pgfpathlineto{\pgfqpoint{0.510522in}{0.460924in}}%
\pgfpathlineto{\pgfqpoint{0.506806in}{0.461794in}}%
\pgfpathclose%
\pgfusepath{fill}%
\end{pgfscope}%
\begin{pgfscope}%
\pgfpathrectangle{\pgfqpoint{0.100000in}{0.100000in}}{\pgfqpoint{3.608454in}{2.310000in}}%
\pgfusepath{clip}%
\pgfsetbuttcap%
\pgfsetmiterjoin%
\definecolor{currentfill}{rgb}{0.000000,0.827451,0.586275}%
\pgfsetfillcolor{currentfill}%
\pgfsetlinewidth{0.000000pt}%
\definecolor{currentstroke}{rgb}{0.000000,0.000000,0.000000}%
\pgfsetstrokecolor{currentstroke}%
\pgfsetstrokeopacity{0.000000}%
\pgfsetdash{}{0pt}%
\pgfpathmoveto{\pgfqpoint{0.349460in}{0.680635in}}%
\pgfpathlineto{\pgfqpoint{0.349236in}{0.682931in}}%
\pgfpathlineto{\pgfqpoint{0.345635in}{0.686541in}}%
\pgfpathlineto{\pgfqpoint{0.348354in}{0.687196in}}%
\pgfpathlineto{\pgfqpoint{0.346567in}{0.689336in}}%
\pgfpathlineto{\pgfqpoint{0.347637in}{0.690563in}}%
\pgfpathlineto{\pgfqpoint{0.350022in}{0.690962in}}%
\pgfpathlineto{\pgfqpoint{0.351654in}{0.687198in}}%
\pgfpathlineto{\pgfqpoint{0.353748in}{0.683821in}}%
\pgfpathlineto{\pgfqpoint{0.353994in}{0.680863in}}%
\pgfpathlineto{\pgfqpoint{0.352204in}{0.679863in}}%
\pgfpathlineto{\pgfqpoint{0.349460in}{0.680635in}}%
\pgfpathclose%
\pgfusepath{fill}%
\end{pgfscope}%
\begin{pgfscope}%
\pgfpathrectangle{\pgfqpoint{0.100000in}{0.100000in}}{\pgfqpoint{3.608454in}{2.310000in}}%
\pgfusepath{clip}%
\pgfsetbuttcap%
\pgfsetmiterjoin%
\definecolor{currentfill}{rgb}{0.000000,0.827451,0.586275}%
\pgfsetfillcolor{currentfill}%
\pgfsetlinewidth{0.000000pt}%
\definecolor{currentstroke}{rgb}{0.000000,0.000000,0.000000}%
\pgfsetstrokecolor{currentstroke}%
\pgfsetstrokeopacity{0.000000}%
\pgfsetdash{}{0pt}%
\pgfpathmoveto{\pgfqpoint{0.556547in}{0.436939in}}%
\pgfpathlineto{\pgfqpoint{0.555985in}{0.439788in}}%
\pgfpathlineto{\pgfqpoint{0.557578in}{0.440221in}}%
\pgfpathlineto{\pgfqpoint{0.561322in}{0.439031in}}%
\pgfpathlineto{\pgfqpoint{0.563123in}{0.437521in}}%
\pgfpathlineto{\pgfqpoint{0.565195in}{0.437153in}}%
\pgfpathlineto{\pgfqpoint{0.566888in}{0.435455in}}%
\pgfpathlineto{\pgfqpoint{0.569188in}{0.437256in}}%
\pgfpathlineto{\pgfqpoint{0.572877in}{0.438751in}}%
\pgfpathlineto{\pgfqpoint{0.575932in}{0.435745in}}%
\pgfpathlineto{\pgfqpoint{0.574169in}{0.442233in}}%
\pgfpathlineto{\pgfqpoint{0.576542in}{0.443329in}}%
\pgfpathlineto{\pgfqpoint{0.579175in}{0.442765in}}%
\pgfpathlineto{\pgfqpoint{0.583426in}{0.440455in}}%
\pgfpathlineto{\pgfqpoint{0.582448in}{0.438627in}}%
\pgfpathlineto{\pgfqpoint{0.584561in}{0.436213in}}%
\pgfpathlineto{\pgfqpoint{0.586754in}{0.436810in}}%
\pgfpathlineto{\pgfqpoint{0.586008in}{0.433801in}}%
\pgfpathlineto{\pgfqpoint{0.582411in}{0.433473in}}%
\pgfpathlineto{\pgfqpoint{0.580948in}{0.429661in}}%
\pgfpathlineto{\pgfqpoint{0.574828in}{0.431855in}}%
\pgfpathlineto{\pgfqpoint{0.572584in}{0.432073in}}%
\pgfpathlineto{\pgfqpoint{0.571112in}{0.430136in}}%
\pgfpathlineto{\pgfqpoint{0.569377in}{0.433000in}}%
\pgfpathlineto{\pgfqpoint{0.567679in}{0.433841in}}%
\pgfpathlineto{\pgfqpoint{0.562725in}{0.434707in}}%
\pgfpathlineto{\pgfqpoint{0.560495in}{0.435926in}}%
\pgfpathlineto{\pgfqpoint{0.556547in}{0.436939in}}%
\pgfpathclose%
\pgfusepath{fill}%
\end{pgfscope}%
\begin{pgfscope}%
\pgfpathrectangle{\pgfqpoint{0.100000in}{0.100000in}}{\pgfqpoint{3.608454in}{2.310000in}}%
\pgfusepath{clip}%
\pgfsetbuttcap%
\pgfsetmiterjoin%
\definecolor{currentfill}{rgb}{0.000000,0.827451,0.586275}%
\pgfsetfillcolor{currentfill}%
\pgfsetlinewidth{0.000000pt}%
\definecolor{currentstroke}{rgb}{0.000000,0.000000,0.000000}%
\pgfsetstrokecolor{currentstroke}%
\pgfsetstrokeopacity{0.000000}%
\pgfsetdash{}{0pt}%
\pgfpathmoveto{\pgfqpoint{0.523468in}{0.452981in}}%
\pgfpathlineto{\pgfqpoint{0.523015in}{0.455703in}}%
\pgfpathlineto{\pgfqpoint{0.525225in}{0.455976in}}%
\pgfpathlineto{\pgfqpoint{0.525280in}{0.453766in}}%
\pgfpathlineto{\pgfqpoint{0.523468in}{0.452981in}}%
\pgfpathclose%
\pgfusepath{fill}%
\end{pgfscope}%
\begin{pgfscope}%
\pgfpathrectangle{\pgfqpoint{0.100000in}{0.100000in}}{\pgfqpoint{3.608454in}{2.310000in}}%
\pgfusepath{clip}%
\pgfsetbuttcap%
\pgfsetmiterjoin%
\definecolor{currentfill}{rgb}{0.000000,0.827451,0.586275}%
\pgfsetfillcolor{currentfill}%
\pgfsetlinewidth{0.000000pt}%
\definecolor{currentstroke}{rgb}{0.000000,0.000000,0.000000}%
\pgfsetstrokecolor{currentstroke}%
\pgfsetstrokeopacity{0.000000}%
\pgfsetdash{}{0pt}%
\pgfpathmoveto{\pgfqpoint{0.532273in}{0.445921in}}%
\pgfpathlineto{\pgfqpoint{0.532818in}{0.446891in}}%
\pgfpathlineto{\pgfqpoint{0.538427in}{0.447049in}}%
\pgfpathlineto{\pgfqpoint{0.541753in}{0.449221in}}%
\pgfpathlineto{\pgfqpoint{0.546018in}{0.448961in}}%
\pgfpathlineto{\pgfqpoint{0.548669in}{0.445595in}}%
\pgfpathlineto{\pgfqpoint{0.551094in}{0.448774in}}%
\pgfpathlineto{\pgfqpoint{0.553730in}{0.449651in}}%
\pgfpathlineto{\pgfqpoint{0.556658in}{0.449030in}}%
\pgfpathlineto{\pgfqpoint{0.558881in}{0.447475in}}%
\pgfpathlineto{\pgfqpoint{0.559652in}{0.444922in}}%
\pgfpathlineto{\pgfqpoint{0.559201in}{0.443279in}}%
\pgfpathlineto{\pgfqpoint{0.556808in}{0.441953in}}%
\pgfpathlineto{\pgfqpoint{0.548779in}{0.444295in}}%
\pgfpathlineto{\pgfqpoint{0.543555in}{0.442792in}}%
\pgfpathlineto{\pgfqpoint{0.541563in}{0.443699in}}%
\pgfpathlineto{\pgfqpoint{0.532273in}{0.445921in}}%
\pgfpathclose%
\pgfusepath{fill}%
\end{pgfscope}%
\begin{pgfscope}%
\pgfpathrectangle{\pgfqpoint{0.100000in}{0.100000in}}{\pgfqpoint{3.608454in}{2.310000in}}%
\pgfusepath{clip}%
\pgfsetbuttcap%
\pgfsetmiterjoin%
\definecolor{currentfill}{rgb}{0.000000,0.827451,0.586275}%
\pgfsetfillcolor{currentfill}%
\pgfsetlinewidth{0.000000pt}%
\definecolor{currentstroke}{rgb}{0.000000,0.000000,0.000000}%
\pgfsetstrokecolor{currentstroke}%
\pgfsetstrokeopacity{0.000000}%
\pgfsetdash{}{0pt}%
\pgfpathmoveto{\pgfqpoint{0.467157in}{0.485560in}}%
\pgfpathlineto{\pgfqpoint{0.465100in}{0.487760in}}%
\pgfpathlineto{\pgfqpoint{0.463877in}{0.490340in}}%
\pgfpathlineto{\pgfqpoint{0.461910in}{0.492281in}}%
\pgfpathlineto{\pgfqpoint{0.462381in}{0.494960in}}%
\pgfpathlineto{\pgfqpoint{0.466531in}{0.490940in}}%
\pgfpathlineto{\pgfqpoint{0.468975in}{0.485923in}}%
\pgfpathlineto{\pgfqpoint{0.467157in}{0.485560in}}%
\pgfpathclose%
\pgfusepath{fill}%
\end{pgfscope}%
\begin{pgfscope}%
\pgfpathrectangle{\pgfqpoint{0.100000in}{0.100000in}}{\pgfqpoint{3.608454in}{2.310000in}}%
\pgfusepath{clip}%
\pgfsetbuttcap%
\pgfsetmiterjoin%
\definecolor{currentfill}{rgb}{0.000000,0.827451,0.586275}%
\pgfsetfillcolor{currentfill}%
\pgfsetlinewidth{0.000000pt}%
\definecolor{currentstroke}{rgb}{0.000000,0.000000,0.000000}%
\pgfsetstrokecolor{currentstroke}%
\pgfsetstrokeopacity{0.000000}%
\pgfsetdash{}{0pt}%
\pgfpathmoveto{\pgfqpoint{0.456040in}{0.499043in}}%
\pgfpathlineto{\pgfqpoint{0.454673in}{0.501491in}}%
\pgfpathlineto{\pgfqpoint{0.451914in}{0.502371in}}%
\pgfpathlineto{\pgfqpoint{0.451045in}{0.506296in}}%
\pgfpathlineto{\pgfqpoint{0.454103in}{0.504564in}}%
\pgfpathlineto{\pgfqpoint{0.455369in}{0.502279in}}%
\pgfpathlineto{\pgfqpoint{0.457486in}{0.502766in}}%
\pgfpathlineto{\pgfqpoint{0.461584in}{0.502365in}}%
\pgfpathlineto{\pgfqpoint{0.463798in}{0.503703in}}%
\pgfpathlineto{\pgfqpoint{0.465884in}{0.502834in}}%
\pgfpathlineto{\pgfqpoint{0.466975in}{0.500406in}}%
\pgfpathlineto{\pgfqpoint{0.466593in}{0.498917in}}%
\pgfpathlineto{\pgfqpoint{0.464307in}{0.497731in}}%
\pgfpathlineto{\pgfqpoint{0.462347in}{0.499412in}}%
\pgfpathlineto{\pgfqpoint{0.461307in}{0.496536in}}%
\pgfpathlineto{\pgfqpoint{0.459853in}{0.497056in}}%
\pgfpathlineto{\pgfqpoint{0.457626in}{0.499895in}}%
\pgfpathlineto{\pgfqpoint{0.456040in}{0.499043in}}%
\pgfpathclose%
\pgfusepath{fill}%
\end{pgfscope}%
\begin{pgfscope}%
\pgfpathrectangle{\pgfqpoint{0.100000in}{0.100000in}}{\pgfqpoint{3.608454in}{2.310000in}}%
\pgfusepath{clip}%
\pgfsetbuttcap%
\pgfsetmiterjoin%
\definecolor{currentfill}{rgb}{0.000000,0.776471,0.611765}%
\pgfsetfillcolor{currentfill}%
\pgfsetlinewidth{0.000000pt}%
\definecolor{currentstroke}{rgb}{0.000000,0.000000,0.000000}%
\pgfsetstrokecolor{currentstroke}%
\pgfsetstrokeopacity{0.000000}%
\pgfsetdash{}{0pt}%
\pgfpathmoveto{\pgfqpoint{2.231585in}{0.901788in}}%
\pgfpathlineto{\pgfqpoint{2.224654in}{0.900658in}}%
\pgfpathlineto{\pgfqpoint{2.207240in}{0.900881in}}%
\pgfpathlineto{\pgfqpoint{2.207261in}{0.903173in}}%
\pgfpathlineto{\pgfqpoint{2.191374in}{0.903396in}}%
\pgfpathlineto{\pgfqpoint{2.183884in}{0.913246in}}%
\pgfpathlineto{\pgfqpoint{2.175110in}{0.918121in}}%
\pgfpathlineto{\pgfqpoint{2.175503in}{0.924555in}}%
\pgfpathlineto{\pgfqpoint{2.174128in}{0.929756in}}%
\pgfpathlineto{\pgfqpoint{2.169177in}{0.935220in}}%
\pgfpathlineto{\pgfqpoint{2.166804in}{0.946907in}}%
\pgfpathlineto{\pgfqpoint{2.165791in}{0.959401in}}%
\pgfpathlineto{\pgfqpoint{2.156410in}{0.959401in}}%
\pgfpathlineto{\pgfqpoint{2.155587in}{0.972321in}}%
\pgfpathlineto{\pgfqpoint{2.176495in}{0.972103in}}%
\pgfpathlineto{\pgfqpoint{2.176519in}{0.997121in}}%
\pgfpathlineto{\pgfqpoint{2.180404in}{1.000901in}}%
\pgfpathlineto{\pgfqpoint{2.190786in}{1.003624in}}%
\pgfpathlineto{\pgfqpoint{2.211366in}{1.003801in}}%
\pgfpathlineto{\pgfqpoint{2.218257in}{1.006289in}}%
\pgfpathlineto{\pgfqpoint{2.238992in}{1.006643in}}%
\pgfpathlineto{\pgfqpoint{2.238537in}{0.995926in}}%
\pgfpathlineto{\pgfqpoint{2.245351in}{0.995915in}}%
\pgfpathlineto{\pgfqpoint{2.249833in}{0.992475in}}%
\pgfpathlineto{\pgfqpoint{2.250959in}{0.985619in}}%
\pgfpathlineto{\pgfqpoint{2.258094in}{0.983370in}}%
\pgfpathlineto{\pgfqpoint{2.259281in}{0.978738in}}%
\pgfpathlineto{\pgfqpoint{2.259184in}{0.957741in}}%
\pgfpathlineto{\pgfqpoint{2.245528in}{0.957904in}}%
\pgfpathlineto{\pgfqpoint{2.245189in}{0.934157in}}%
\pgfpathlineto{\pgfqpoint{2.241965in}{0.929323in}}%
\pgfpathlineto{\pgfqpoint{2.231849in}{0.927551in}}%
\pgfpathlineto{\pgfqpoint{2.231585in}{0.901788in}}%
\pgfpathclose%
\pgfusepath{fill}%
\end{pgfscope}%
\begin{pgfscope}%
\pgfpathrectangle{\pgfqpoint{0.100000in}{0.100000in}}{\pgfqpoint{3.608454in}{2.310000in}}%
\pgfusepath{clip}%
\pgfsetbuttcap%
\pgfsetmiterjoin%
\definecolor{currentfill}{rgb}{0.000000,0.721569,0.639216}%
\pgfsetfillcolor{currentfill}%
\pgfsetlinewidth{0.000000pt}%
\definecolor{currentstroke}{rgb}{0.000000,0.000000,0.000000}%
\pgfsetstrokecolor{currentstroke}%
\pgfsetstrokeopacity{0.000000}%
\pgfsetdash{}{0pt}%
\pgfpathmoveto{\pgfqpoint{0.695564in}{1.998853in}}%
\pgfpathlineto{\pgfqpoint{0.656754in}{2.010285in}}%
\pgfpathlineto{\pgfqpoint{0.620648in}{2.021281in}}%
\pgfpathlineto{\pgfqpoint{0.623427in}{2.025636in}}%
\pgfpathlineto{\pgfqpoint{0.622394in}{2.035924in}}%
\pgfpathlineto{\pgfqpoint{0.630035in}{2.038493in}}%
\pgfpathlineto{\pgfqpoint{0.629691in}{2.048489in}}%
\pgfpathlineto{\pgfqpoint{0.634662in}{2.052316in}}%
\pgfpathlineto{\pgfqpoint{0.636428in}{2.062774in}}%
\pgfpathlineto{\pgfqpoint{0.628826in}{2.076535in}}%
\pgfpathlineto{\pgfqpoint{0.630988in}{2.086576in}}%
\pgfpathlineto{\pgfqpoint{0.634943in}{2.089043in}}%
\pgfpathlineto{\pgfqpoint{0.641984in}{2.086967in}}%
\pgfpathlineto{\pgfqpoint{0.653479in}{2.086201in}}%
\pgfpathlineto{\pgfqpoint{0.650732in}{2.091686in}}%
\pgfpathlineto{\pgfqpoint{0.656547in}{2.111660in}}%
\pgfpathlineto{\pgfqpoint{0.661241in}{2.110265in}}%
\pgfpathlineto{\pgfqpoint{0.719253in}{2.092888in}}%
\pgfpathlineto{\pgfqpoint{0.749167in}{2.084439in}}%
\pgfpathlineto{\pgfqpoint{0.744663in}{2.068773in}}%
\pgfpathlineto{\pgfqpoint{0.737141in}{2.068948in}}%
\pgfpathlineto{\pgfqpoint{0.723978in}{2.065521in}}%
\pgfpathlineto{\pgfqpoint{0.713071in}{2.066429in}}%
\pgfpathlineto{\pgfqpoint{0.707552in}{2.064386in}}%
\pgfpathlineto{\pgfqpoint{0.710428in}{2.058581in}}%
\pgfpathlineto{\pgfqpoint{0.710674in}{2.050338in}}%
\pgfpathlineto{\pgfqpoint{0.702993in}{2.049946in}}%
\pgfpathlineto{\pgfqpoint{0.695191in}{2.034615in}}%
\pgfpathlineto{\pgfqpoint{0.697834in}{2.026472in}}%
\pgfpathlineto{\pgfqpoint{0.692055in}{2.010580in}}%
\pgfpathlineto{\pgfqpoint{0.692032in}{2.003466in}}%
\pgfpathlineto{\pgfqpoint{0.695564in}{1.998853in}}%
\pgfpathclose%
\pgfusepath{fill}%
\end{pgfscope}%
\begin{pgfscope}%
\pgfpathrectangle{\pgfqpoint{0.100000in}{0.100000in}}{\pgfqpoint{3.608454in}{2.310000in}}%
\pgfusepath{clip}%
\pgfsetbuttcap%
\pgfsetmiterjoin%
\definecolor{currentfill}{rgb}{0.000000,0.505882,0.747059}%
\pgfsetfillcolor{currentfill}%
\pgfsetlinewidth{0.000000pt}%
\definecolor{currentstroke}{rgb}{0.000000,0.000000,0.000000}%
\pgfsetstrokecolor{currentstroke}%
\pgfsetstrokeopacity{0.000000}%
\pgfsetdash{}{0pt}%
\pgfpathmoveto{\pgfqpoint{1.748391in}{1.048651in}}%
\pgfpathlineto{\pgfqpoint{1.713023in}{1.050837in}}%
\pgfpathlineto{\pgfqpoint{1.678367in}{1.053341in}}%
\pgfpathlineto{\pgfqpoint{1.680896in}{1.087866in}}%
\pgfpathlineto{\pgfqpoint{1.683362in}{1.122232in}}%
\pgfpathlineto{\pgfqpoint{1.752524in}{1.117682in}}%
\pgfpathlineto{\pgfqpoint{1.787003in}{1.115616in}}%
\pgfpathlineto{\pgfqpoint{1.786242in}{1.101806in}}%
\pgfpathlineto{\pgfqpoint{1.785102in}{1.081107in}}%
\pgfpathlineto{\pgfqpoint{1.750416in}{1.083136in}}%
\pgfpathlineto{\pgfqpoint{1.748391in}{1.048651in}}%
\pgfpathclose%
\pgfusepath{fill}%
\end{pgfscope}%
\begin{pgfscope}%
\pgfpathrectangle{\pgfqpoint{0.100000in}{0.100000in}}{\pgfqpoint{3.608454in}{2.310000in}}%
\pgfusepath{clip}%
\pgfsetbuttcap%
\pgfsetmiterjoin%
\definecolor{currentfill}{rgb}{0.000000,0.654902,0.672549}%
\pgfsetfillcolor{currentfill}%
\pgfsetlinewidth{0.000000pt}%
\definecolor{currentstroke}{rgb}{0.000000,0.000000,0.000000}%
\pgfsetstrokecolor{currentstroke}%
\pgfsetstrokeopacity{0.000000}%
\pgfsetdash{}{0pt}%
\pgfpathmoveto{\pgfqpoint{1.884281in}{0.742465in}}%
\pgfpathlineto{\pgfqpoint{1.863869in}{0.728404in}}%
\pgfpathlineto{\pgfqpoint{1.866023in}{0.717001in}}%
\pgfpathlineto{\pgfqpoint{1.871649in}{0.712052in}}%
\pgfpathlineto{\pgfqpoint{1.870905in}{0.703446in}}%
\pgfpathlineto{\pgfqpoint{1.835408in}{0.704889in}}%
\pgfpathlineto{\pgfqpoint{1.826743in}{0.706842in}}%
\pgfpathlineto{\pgfqpoint{1.828725in}{0.748261in}}%
\pgfpathlineto{\pgfqpoint{1.818920in}{0.750454in}}%
\pgfpathlineto{\pgfqpoint{1.813328in}{0.744891in}}%
\pgfpathlineto{\pgfqpoint{1.804967in}{0.750514in}}%
\pgfpathlineto{\pgfqpoint{1.792433in}{0.750538in}}%
\pgfpathlineto{\pgfqpoint{1.786260in}{0.758457in}}%
\pgfpathlineto{\pgfqpoint{1.788954in}{0.799743in}}%
\pgfpathlineto{\pgfqpoint{1.842140in}{0.796889in}}%
\pgfpathlineto{\pgfqpoint{1.858178in}{0.766124in}}%
\pgfpathlineto{\pgfqpoint{1.870147in}{0.767494in}}%
\pgfpathlineto{\pgfqpoint{1.884281in}{0.742465in}}%
\pgfpathclose%
\pgfusepath{fill}%
\end{pgfscope}%
\begin{pgfscope}%
\pgfpathrectangle{\pgfqpoint{0.100000in}{0.100000in}}{\pgfqpoint{3.608454in}{2.310000in}}%
\pgfusepath{clip}%
\pgfsetbuttcap%
\pgfsetmiterjoin%
\definecolor{currentfill}{rgb}{0.000000,0.905882,0.547059}%
\pgfsetfillcolor{currentfill}%
\pgfsetlinewidth{0.000000pt}%
\definecolor{currentstroke}{rgb}{0.000000,0.000000,0.000000}%
\pgfsetstrokecolor{currentstroke}%
\pgfsetstrokeopacity{0.000000}%
\pgfsetdash{}{0pt}%
\pgfpathmoveto{\pgfqpoint{2.910647in}{0.649877in}}%
\pgfpathlineto{\pgfqpoint{2.900528in}{0.654212in}}%
\pgfpathlineto{\pgfqpoint{2.893044in}{0.665397in}}%
\pgfpathlineto{\pgfqpoint{2.882864in}{0.671272in}}%
\pgfpathlineto{\pgfqpoint{2.871699in}{0.674920in}}%
\pgfpathlineto{\pgfqpoint{2.866769in}{0.678489in}}%
\pgfpathlineto{\pgfqpoint{2.870453in}{0.691569in}}%
\pgfpathlineto{\pgfqpoint{2.878191in}{0.698489in}}%
\pgfpathlineto{\pgfqpoint{2.876684in}{0.701966in}}%
\pgfpathlineto{\pgfqpoint{2.881876in}{0.708207in}}%
\pgfpathlineto{\pgfqpoint{2.880133in}{0.713968in}}%
\pgfpathlineto{\pgfqpoint{2.888448in}{0.720934in}}%
\pgfpathlineto{\pgfqpoint{2.887245in}{0.726366in}}%
\pgfpathlineto{\pgfqpoint{2.957900in}{0.730883in}}%
\pgfpathlineto{\pgfqpoint{2.966459in}{0.731456in}}%
\pgfpathlineto{\pgfqpoint{2.971461in}{0.696012in}}%
\pgfpathlineto{\pgfqpoint{2.965631in}{0.682186in}}%
\pgfpathlineto{\pgfqpoint{2.965236in}{0.672986in}}%
\pgfpathlineto{\pgfqpoint{2.962277in}{0.669805in}}%
\pgfpathlineto{\pgfqpoint{2.955684in}{0.671251in}}%
\pgfpathlineto{\pgfqpoint{2.951329in}{0.676818in}}%
\pgfpathlineto{\pgfqpoint{2.946140in}{0.674377in}}%
\pgfpathlineto{\pgfqpoint{2.944970in}{0.667447in}}%
\pgfpathlineto{\pgfqpoint{2.915694in}{0.663204in}}%
\pgfpathlineto{\pgfqpoint{2.913541in}{0.650279in}}%
\pgfpathlineto{\pgfqpoint{2.910647in}{0.649877in}}%
\pgfpathclose%
\pgfusepath{fill}%
\end{pgfscope}%
\begin{pgfscope}%
\pgfpathrectangle{\pgfqpoint{0.100000in}{0.100000in}}{\pgfqpoint{3.608454in}{2.310000in}}%
\pgfusepath{clip}%
\pgfsetbuttcap%
\pgfsetmiterjoin%
\definecolor{currentfill}{rgb}{0.000000,0.686275,0.656863}%
\pgfsetfillcolor{currentfill}%
\pgfsetlinewidth{0.000000pt}%
\definecolor{currentstroke}{rgb}{0.000000,0.000000,0.000000}%
\pgfsetstrokecolor{currentstroke}%
\pgfsetstrokeopacity{0.000000}%
\pgfsetdash{}{0pt}%
\pgfpathmoveto{\pgfqpoint{2.543041in}{1.178054in}}%
\pgfpathlineto{\pgfqpoint{2.549236in}{1.162164in}}%
\pgfpathlineto{\pgfqpoint{2.548186in}{1.158629in}}%
\pgfpathlineto{\pgfqpoint{2.520509in}{1.156888in}}%
\pgfpathlineto{\pgfqpoint{2.519321in}{1.176572in}}%
\pgfpathlineto{\pgfqpoint{2.543041in}{1.178054in}}%
\pgfpathclose%
\pgfusepath{fill}%
\end{pgfscope}%
\begin{pgfscope}%
\pgfpathrectangle{\pgfqpoint{0.100000in}{0.100000in}}{\pgfqpoint{3.608454in}{2.310000in}}%
\pgfusepath{clip}%
\pgfsetbuttcap%
\pgfsetmiterjoin%
\definecolor{currentfill}{rgb}{0.000000,0.729412,0.635294}%
\pgfsetfillcolor{currentfill}%
\pgfsetlinewidth{0.000000pt}%
\definecolor{currentstroke}{rgb}{0.000000,0.000000,0.000000}%
\pgfsetstrokecolor{currentstroke}%
\pgfsetstrokeopacity{0.000000}%
\pgfsetdash{}{0pt}%
\pgfpathmoveto{\pgfqpoint{2.213676in}{1.928846in}}%
\pgfpathlineto{\pgfqpoint{2.214175in}{1.908040in}}%
\pgfpathlineto{\pgfqpoint{2.193329in}{1.907307in}}%
\pgfpathlineto{\pgfqpoint{2.193274in}{1.914649in}}%
\pgfpathlineto{\pgfqpoint{2.172320in}{1.914091in}}%
\pgfpathlineto{\pgfqpoint{2.172440in}{1.907196in}}%
\pgfpathlineto{\pgfqpoint{2.141306in}{1.906925in}}%
\pgfpathlineto{\pgfqpoint{2.142555in}{1.917955in}}%
\pgfpathlineto{\pgfqpoint{2.138352in}{1.920870in}}%
\pgfpathlineto{\pgfqpoint{2.132410in}{1.918482in}}%
\pgfpathlineto{\pgfqpoint{2.118937in}{1.925718in}}%
\pgfpathlineto{\pgfqpoint{2.118427in}{1.958403in}}%
\pgfpathlineto{\pgfqpoint{2.125367in}{1.958378in}}%
\pgfpathlineto{\pgfqpoint{2.124628in}{2.006820in}}%
\pgfpathlineto{\pgfqpoint{2.138354in}{2.007024in}}%
\pgfpathlineto{\pgfqpoint{2.138015in}{2.041726in}}%
\pgfpathlineto{\pgfqpoint{2.172503in}{2.042117in}}%
\pgfpathlineto{\pgfqpoint{2.172471in}{2.046322in}}%
\pgfpathlineto{\pgfqpoint{2.209745in}{2.046401in}}%
\pgfpathlineto{\pgfqpoint{2.211106in}{2.032691in}}%
\pgfpathlineto{\pgfqpoint{2.211215in}{2.006566in}}%
\pgfpathlineto{\pgfqpoint{2.212580in}{1.977269in}}%
\pgfpathlineto{\pgfqpoint{2.212805in}{1.945663in}}%
\pgfpathlineto{\pgfqpoint{2.213676in}{1.928846in}}%
\pgfpathclose%
\pgfusepath{fill}%
\end{pgfscope}%
\begin{pgfscope}%
\pgfpathrectangle{\pgfqpoint{0.100000in}{0.100000in}}{\pgfqpoint{3.608454in}{2.310000in}}%
\pgfusepath{clip}%
\pgfsetbuttcap%
\pgfsetmiterjoin%
\definecolor{currentfill}{rgb}{0.000000,0.521569,0.739216}%
\pgfsetfillcolor{currentfill}%
\pgfsetlinewidth{0.000000pt}%
\definecolor{currentstroke}{rgb}{0.000000,0.000000,0.000000}%
\pgfsetstrokecolor{currentstroke}%
\pgfsetstrokeopacity{0.000000}%
\pgfsetdash{}{0pt}%
\pgfpathmoveto{\pgfqpoint{1.598608in}{1.839877in}}%
\pgfpathlineto{\pgfqpoint{1.543825in}{1.845887in}}%
\pgfpathlineto{\pgfqpoint{1.546917in}{1.873677in}}%
\pgfpathlineto{\pgfqpoint{1.549754in}{1.873361in}}%
\pgfpathlineto{\pgfqpoint{1.553831in}{1.907730in}}%
\pgfpathlineto{\pgfqpoint{1.556192in}{1.907466in}}%
\pgfpathlineto{\pgfqpoint{1.559222in}{1.935142in}}%
\pgfpathlineto{\pgfqpoint{1.562162in}{1.934829in}}%
\pgfpathlineto{\pgfqpoint{1.582466in}{1.932625in}}%
\pgfpathlineto{\pgfqpoint{1.581763in}{1.926153in}}%
\pgfpathlineto{\pgfqpoint{1.588691in}{1.925458in}}%
\pgfpathlineto{\pgfqpoint{1.587963in}{1.918755in}}%
\pgfpathlineto{\pgfqpoint{1.602057in}{1.917205in}}%
\pgfpathlineto{\pgfqpoint{1.606786in}{1.914838in}}%
\pgfpathlineto{\pgfqpoint{1.664670in}{1.909384in}}%
\pgfpathlineto{\pgfqpoint{1.667609in}{1.909112in}}%
\pgfpathlineto{\pgfqpoint{1.661648in}{1.851140in}}%
\pgfpathlineto{\pgfqpoint{1.601301in}{1.856823in}}%
\pgfpathlineto{\pgfqpoint{1.598608in}{1.839877in}}%
\pgfpathclose%
\pgfusepath{fill}%
\end{pgfscope}%
\begin{pgfscope}%
\pgfpathrectangle{\pgfqpoint{0.100000in}{0.100000in}}{\pgfqpoint{3.608454in}{2.310000in}}%
\pgfusepath{clip}%
\pgfsetbuttcap%
\pgfsetmiterjoin%
\definecolor{currentfill}{rgb}{0.000000,0.768627,0.615686}%
\pgfsetfillcolor{currentfill}%
\pgfsetlinewidth{0.000000pt}%
\definecolor{currentstroke}{rgb}{0.000000,0.000000,0.000000}%
\pgfsetstrokecolor{currentstroke}%
\pgfsetstrokeopacity{0.000000}%
\pgfsetdash{}{0pt}%
\pgfpathmoveto{\pgfqpoint{2.618822in}{1.745974in}}%
\pgfpathlineto{\pgfqpoint{2.596450in}{1.744135in}}%
\pgfpathlineto{\pgfqpoint{2.593518in}{1.755812in}}%
\pgfpathlineto{\pgfqpoint{2.590087in}{1.761390in}}%
\pgfpathlineto{\pgfqpoint{2.594695in}{1.768652in}}%
\pgfpathlineto{\pgfqpoint{2.601730in}{1.786502in}}%
\pgfpathlineto{\pgfqpoint{2.602311in}{1.800403in}}%
\pgfpathlineto{\pgfqpoint{2.625838in}{1.802295in}}%
\pgfpathlineto{\pgfqpoint{2.628418in}{1.774736in}}%
\pgfpathlineto{\pgfqpoint{2.615872in}{1.773777in}}%
\pgfpathlineto{\pgfqpoint{2.618822in}{1.745974in}}%
\pgfpathclose%
\pgfusepath{fill}%
\end{pgfscope}%
\begin{pgfscope}%
\pgfpathrectangle{\pgfqpoint{0.100000in}{0.100000in}}{\pgfqpoint{3.608454in}{2.310000in}}%
\pgfusepath{clip}%
\pgfsetbuttcap%
\pgfsetmiterjoin%
\definecolor{currentfill}{rgb}{0.000000,0.682353,0.658824}%
\pgfsetfillcolor{currentfill}%
\pgfsetlinewidth{0.000000pt}%
\definecolor{currentstroke}{rgb}{0.000000,0.000000,0.000000}%
\pgfsetstrokecolor{currentstroke}%
\pgfsetstrokeopacity{0.000000}%
\pgfsetdash{}{0pt}%
\pgfpathmoveto{\pgfqpoint{1.194163in}{1.451751in}}%
\pgfpathlineto{\pgfqpoint{1.192855in}{1.443180in}}%
\pgfpathlineto{\pgfqpoint{1.188609in}{1.438112in}}%
\pgfpathlineto{\pgfqpoint{1.189342in}{1.433929in}}%
\pgfpathlineto{\pgfqpoint{1.182094in}{1.423685in}}%
\pgfpathlineto{\pgfqpoint{1.177987in}{1.406555in}}%
\pgfpathlineto{\pgfqpoint{1.181137in}{1.398548in}}%
\pgfpathlineto{\pgfqpoint{1.179215in}{1.395609in}}%
\pgfpathlineto{\pgfqpoint{1.183363in}{1.379999in}}%
\pgfpathlineto{\pgfqpoint{1.181575in}{1.376785in}}%
\pgfpathlineto{\pgfqpoint{1.141855in}{1.383558in}}%
\pgfpathlineto{\pgfqpoint{1.103543in}{1.390370in}}%
\pgfpathlineto{\pgfqpoint{1.103689in}{1.391163in}}%
\pgfpathlineto{\pgfqpoint{1.111444in}{1.431696in}}%
\pgfpathlineto{\pgfqpoint{1.077995in}{1.437925in}}%
\pgfpathlineto{\pgfqpoint{1.068484in}{1.440713in}}%
\pgfpathlineto{\pgfqpoint{1.072686in}{1.463362in}}%
\pgfpathlineto{\pgfqpoint{1.078936in}{1.465774in}}%
\pgfpathlineto{\pgfqpoint{1.089487in}{1.463907in}}%
\pgfpathlineto{\pgfqpoint{1.092637in}{1.469533in}}%
\pgfpathlineto{\pgfqpoint{1.095705in}{1.487818in}}%
\pgfpathlineto{\pgfqpoint{1.104984in}{1.493265in}}%
\pgfpathlineto{\pgfqpoint{1.101981in}{1.496172in}}%
\pgfpathlineto{\pgfqpoint{1.148850in}{1.487724in}}%
\pgfpathlineto{\pgfqpoint{1.169072in}{1.483661in}}%
\pgfpathlineto{\pgfqpoint{1.206990in}{1.477188in}}%
\pgfpathlineto{\pgfqpoint{1.199560in}{1.472867in}}%
\pgfpathlineto{\pgfqpoint{1.199187in}{1.467135in}}%
\pgfpathlineto{\pgfqpoint{1.194698in}{1.460074in}}%
\pgfpathlineto{\pgfqpoint{1.194163in}{1.451751in}}%
\pgfpathclose%
\pgfusepath{fill}%
\end{pgfscope}%
\begin{pgfscope}%
\pgfpathrectangle{\pgfqpoint{0.100000in}{0.100000in}}{\pgfqpoint{3.608454in}{2.310000in}}%
\pgfusepath{clip}%
\pgfsetbuttcap%
\pgfsetmiterjoin%
\definecolor{currentfill}{rgb}{0.000000,0.945098,0.527451}%
\pgfsetfillcolor{currentfill}%
\pgfsetlinewidth{0.000000pt}%
\definecolor{currentstroke}{rgb}{0.000000,0.000000,0.000000}%
\pgfsetstrokecolor{currentstroke}%
\pgfsetstrokeopacity{0.000000}%
\pgfsetdash{}{0pt}%
\pgfpathmoveto{\pgfqpoint{0.456983in}{1.837942in}}%
\pgfpathlineto{\pgfqpoint{0.452336in}{1.831556in}}%
\pgfpathlineto{\pgfqpoint{0.442382in}{1.828434in}}%
\pgfpathlineto{\pgfqpoint{0.441986in}{1.818955in}}%
\pgfpathlineto{\pgfqpoint{0.436698in}{1.812439in}}%
\pgfpathlineto{\pgfqpoint{0.438276in}{1.805265in}}%
\pgfpathlineto{\pgfqpoint{0.437233in}{1.797086in}}%
\pgfpathlineto{\pgfqpoint{0.433774in}{1.794008in}}%
\pgfpathlineto{\pgfqpoint{0.427599in}{1.795918in}}%
\pgfpathlineto{\pgfqpoint{0.429636in}{1.802194in}}%
\pgfpathlineto{\pgfqpoint{0.413061in}{1.807713in}}%
\pgfpathlineto{\pgfqpoint{0.414756in}{1.827792in}}%
\pgfpathlineto{\pgfqpoint{0.410295in}{1.835155in}}%
\pgfpathlineto{\pgfqpoint{0.417067in}{1.845633in}}%
\pgfpathlineto{\pgfqpoint{0.418133in}{1.851097in}}%
\pgfpathlineto{\pgfqpoint{0.412925in}{1.860819in}}%
\pgfpathlineto{\pgfqpoint{0.414420in}{1.867082in}}%
\pgfpathlineto{\pgfqpoint{0.414023in}{1.879041in}}%
\pgfpathlineto{\pgfqpoint{0.418855in}{1.891440in}}%
\pgfpathlineto{\pgfqpoint{0.422736in}{1.897083in}}%
\pgfpathlineto{\pgfqpoint{0.423493in}{1.904021in}}%
\pgfpathlineto{\pgfqpoint{0.420036in}{1.911157in}}%
\pgfpathlineto{\pgfqpoint{0.420507in}{1.920108in}}%
\pgfpathlineto{\pgfqpoint{0.427334in}{1.927274in}}%
\pgfpathlineto{\pgfqpoint{0.439586in}{1.923181in}}%
\pgfpathlineto{\pgfqpoint{0.443285in}{1.914029in}}%
\pgfpathlineto{\pgfqpoint{0.438998in}{1.899672in}}%
\pgfpathlineto{\pgfqpoint{0.447212in}{1.898187in}}%
\pgfpathlineto{\pgfqpoint{0.456477in}{1.902564in}}%
\pgfpathlineto{\pgfqpoint{0.465159in}{1.900474in}}%
\pgfpathlineto{\pgfqpoint{0.455038in}{1.891185in}}%
\pgfpathlineto{\pgfqpoint{0.448521in}{1.882309in}}%
\pgfpathlineto{\pgfqpoint{0.441915in}{1.883291in}}%
\pgfpathlineto{\pgfqpoint{0.437497in}{1.874169in}}%
\pgfpathlineto{\pgfqpoint{0.444876in}{1.871177in}}%
\pgfpathlineto{\pgfqpoint{0.447509in}{1.860643in}}%
\pgfpathlineto{\pgfqpoint{0.441504in}{1.854677in}}%
\pgfpathlineto{\pgfqpoint{0.439878in}{1.843077in}}%
\pgfpathlineto{\pgfqpoint{0.456983in}{1.837942in}}%
\pgfpathclose%
\pgfusepath{fill}%
\end{pgfscope}%
\begin{pgfscope}%
\pgfpathrectangle{\pgfqpoint{0.100000in}{0.100000in}}{\pgfqpoint{3.608454in}{2.310000in}}%
\pgfusepath{clip}%
\pgfsetbuttcap%
\pgfsetmiterjoin%
\definecolor{currentfill}{rgb}{0.000000,0.639216,0.680392}%
\pgfsetfillcolor{currentfill}%
\pgfsetlinewidth{0.000000pt}%
\definecolor{currentstroke}{rgb}{0.000000,0.000000,0.000000}%
\pgfsetstrokecolor{currentstroke}%
\pgfsetstrokeopacity{0.000000}%
\pgfsetdash{}{0pt}%
\pgfpathmoveto{\pgfqpoint{2.720165in}{1.477649in}}%
\pgfpathlineto{\pgfqpoint{2.722450in}{1.453627in}}%
\pgfpathlineto{\pgfqpoint{2.698943in}{1.451000in}}%
\pgfpathlineto{\pgfqpoint{2.699598in}{1.433892in}}%
\pgfpathlineto{\pgfqpoint{2.694747in}{1.433296in}}%
\pgfpathlineto{\pgfqpoint{2.676915in}{1.431338in}}%
\pgfpathlineto{\pgfqpoint{2.676851in}{1.443992in}}%
\pgfpathlineto{\pgfqpoint{2.659619in}{1.442049in}}%
\pgfpathlineto{\pgfqpoint{2.657467in}{1.463709in}}%
\pgfpathlineto{\pgfqpoint{2.653860in}{1.497742in}}%
\pgfpathlineto{\pgfqpoint{2.667350in}{1.499209in}}%
\pgfpathlineto{\pgfqpoint{2.678668in}{1.500441in}}%
\pgfpathlineto{\pgfqpoint{2.679497in}{1.493619in}}%
\pgfpathlineto{\pgfqpoint{2.694173in}{1.495218in}}%
\pgfpathlineto{\pgfqpoint{2.695336in}{1.474638in}}%
\pgfpathlineto{\pgfqpoint{2.720165in}{1.477649in}}%
\pgfpathclose%
\pgfusepath{fill}%
\end{pgfscope}%
\begin{pgfscope}%
\pgfpathrectangle{\pgfqpoint{0.100000in}{0.100000in}}{\pgfqpoint{3.608454in}{2.310000in}}%
\pgfusepath{clip}%
\pgfsetbuttcap%
\pgfsetmiterjoin%
\definecolor{currentfill}{rgb}{0.000000,0.635294,0.682353}%
\pgfsetfillcolor{currentfill}%
\pgfsetlinewidth{0.000000pt}%
\definecolor{currentstroke}{rgb}{0.000000,0.000000,0.000000}%
\pgfsetstrokecolor{currentstroke}%
\pgfsetstrokeopacity{0.000000}%
\pgfsetdash{}{0pt}%
\pgfpathmoveto{\pgfqpoint{2.809988in}{1.252552in}}%
\pgfpathlineto{\pgfqpoint{2.801049in}{1.243631in}}%
\pgfpathlineto{\pgfqpoint{2.797121in}{1.243378in}}%
\pgfpathlineto{\pgfqpoint{2.788710in}{1.248796in}}%
\pgfpathlineto{\pgfqpoint{2.781851in}{1.240501in}}%
\pgfpathlineto{\pgfqpoint{2.780361in}{1.233265in}}%
\pgfpathlineto{\pgfqpoint{2.765354in}{1.245580in}}%
\pgfpathlineto{\pgfqpoint{2.767417in}{1.258344in}}%
\pgfpathlineto{\pgfqpoint{2.773095in}{1.263245in}}%
\pgfpathlineto{\pgfqpoint{2.759930in}{1.280063in}}%
\pgfpathlineto{\pgfqpoint{2.774318in}{1.293523in}}%
\pgfpathlineto{\pgfqpoint{2.791570in}{1.288910in}}%
\pgfpathlineto{\pgfqpoint{2.801059in}{1.283583in}}%
\pgfpathlineto{\pgfqpoint{2.810307in}{1.281879in}}%
\pgfpathlineto{\pgfqpoint{2.804559in}{1.276722in}}%
\pgfpathlineto{\pgfqpoint{2.800586in}{1.266861in}}%
\pgfpathlineto{\pgfqpoint{2.802573in}{1.261875in}}%
\pgfpathlineto{\pgfqpoint{2.809988in}{1.252552in}}%
\pgfpathclose%
\pgfusepath{fill}%
\end{pgfscope}%
\begin{pgfscope}%
\pgfpathrectangle{\pgfqpoint{0.100000in}{0.100000in}}{\pgfqpoint{3.608454in}{2.310000in}}%
\pgfusepath{clip}%
\pgfsetbuttcap%
\pgfsetmiterjoin%
\definecolor{currentfill}{rgb}{0.000000,0.964706,0.517647}%
\pgfsetfillcolor{currentfill}%
\pgfsetlinewidth{0.000000pt}%
\definecolor{currentstroke}{rgb}{0.000000,0.000000,0.000000}%
\pgfsetstrokecolor{currentstroke}%
\pgfsetstrokeopacity{0.000000}%
\pgfsetdash{}{0pt}%
\pgfpathmoveto{\pgfqpoint{2.182637in}{0.773027in}}%
\pgfpathlineto{\pgfqpoint{2.212262in}{0.773539in}}%
\pgfpathlineto{\pgfqpoint{2.212402in}{0.763123in}}%
\pgfpathlineto{\pgfqpoint{2.219367in}{0.763229in}}%
\pgfpathlineto{\pgfqpoint{2.219726in}{0.749195in}}%
\pgfpathlineto{\pgfqpoint{2.226572in}{0.749343in}}%
\pgfpathlineto{\pgfqpoint{2.226879in}{0.735487in}}%
\pgfpathlineto{\pgfqpoint{2.216369in}{0.735263in}}%
\pgfpathlineto{\pgfqpoint{2.212945in}{0.728230in}}%
\pgfpathlineto{\pgfqpoint{2.206617in}{0.728100in}}%
\pgfpathlineto{\pgfqpoint{2.205558in}{0.720837in}}%
\pgfpathlineto{\pgfqpoint{2.202317in}{0.720035in}}%
\pgfpathlineto{\pgfqpoint{2.196211in}{0.730685in}}%
\pgfpathlineto{\pgfqpoint{2.197023in}{0.737625in}}%
\pgfpathlineto{\pgfqpoint{2.191816in}{0.743213in}}%
\pgfpathlineto{\pgfqpoint{2.185571in}{0.753524in}}%
\pgfpathlineto{\pgfqpoint{2.187852in}{0.760329in}}%
\pgfpathlineto{\pgfqpoint{2.186530in}{0.767441in}}%
\pgfpathlineto{\pgfqpoint{2.182637in}{0.773027in}}%
\pgfpathclose%
\pgfusepath{fill}%
\end{pgfscope}%
\begin{pgfscope}%
\pgfpathrectangle{\pgfqpoint{0.100000in}{0.100000in}}{\pgfqpoint{3.608454in}{2.310000in}}%
\pgfusepath{clip}%
\pgfsetbuttcap%
\pgfsetmiterjoin%
\definecolor{currentfill}{rgb}{0.000000,0.486275,0.756863}%
\pgfsetfillcolor{currentfill}%
\pgfsetlinewidth{0.000000pt}%
\definecolor{currentstroke}{rgb}{0.000000,0.000000,0.000000}%
\pgfsetstrokecolor{currentstroke}%
\pgfsetstrokeopacity{0.000000}%
\pgfsetdash{}{0pt}%
\pgfpathmoveto{\pgfqpoint{1.732544in}{1.486727in}}%
\pgfpathlineto{\pgfqpoint{1.684512in}{1.490228in}}%
\pgfpathlineto{\pgfqpoint{1.684833in}{1.494331in}}%
\pgfpathlineto{\pgfqpoint{1.649013in}{1.497206in}}%
\pgfpathlineto{\pgfqpoint{1.650561in}{1.517162in}}%
\pgfpathlineto{\pgfqpoint{1.653438in}{1.520561in}}%
\pgfpathlineto{\pgfqpoint{1.654575in}{1.534236in}}%
\pgfpathlineto{\pgfqpoint{1.687474in}{1.531563in}}%
\pgfpathlineto{\pgfqpoint{1.688556in}{1.545253in}}%
\pgfpathlineto{\pgfqpoint{1.692731in}{1.544894in}}%
\pgfpathlineto{\pgfqpoint{1.726904in}{1.542427in}}%
\pgfpathlineto{\pgfqpoint{1.735045in}{1.541820in}}%
\pgfpathlineto{\pgfqpoint{1.733109in}{1.514401in}}%
\pgfpathlineto{\pgfqpoint{1.734320in}{1.514317in}}%
\pgfpathlineto{\pgfqpoint{1.732544in}{1.486727in}}%
\pgfpathclose%
\pgfusepath{fill}%
\end{pgfscope}%
\begin{pgfscope}%
\pgfpathrectangle{\pgfqpoint{0.100000in}{0.100000in}}{\pgfqpoint{3.608454in}{2.310000in}}%
\pgfusepath{clip}%
\pgfsetbuttcap%
\pgfsetmiterjoin%
\definecolor{currentfill}{rgb}{0.000000,0.596078,0.701961}%
\pgfsetfillcolor{currentfill}%
\pgfsetlinewidth{0.000000pt}%
\definecolor{currentstroke}{rgb}{0.000000,0.000000,0.000000}%
\pgfsetstrokecolor{currentstroke}%
\pgfsetstrokeopacity{0.000000}%
\pgfsetdash{}{0pt}%
\pgfpathmoveto{\pgfqpoint{2.610086in}{1.175304in}}%
\pgfpathlineto{\pgfqpoint{2.607679in}{1.208265in}}%
\pgfpathlineto{\pgfqpoint{2.620792in}{1.209072in}}%
\pgfpathlineto{\pgfqpoint{2.631929in}{1.205576in}}%
\pgfpathlineto{\pgfqpoint{2.634641in}{1.217991in}}%
\pgfpathlineto{\pgfqpoint{2.639824in}{1.220359in}}%
\pgfpathlineto{\pgfqpoint{2.647960in}{1.220595in}}%
\pgfpathlineto{\pgfqpoint{2.642473in}{1.232080in}}%
\pgfpathlineto{\pgfqpoint{2.661850in}{1.235103in}}%
\pgfpathlineto{\pgfqpoint{2.669571in}{1.226318in}}%
\pgfpathlineto{\pgfqpoint{2.669252in}{1.213390in}}%
\pgfpathlineto{\pgfqpoint{2.664358in}{1.203424in}}%
\pgfpathlineto{\pgfqpoint{2.677443in}{1.189404in}}%
\pgfpathlineto{\pgfqpoint{2.678723in}{1.180579in}}%
\pgfpathlineto{\pgfqpoint{2.656122in}{1.179965in}}%
\pgfpathlineto{\pgfqpoint{2.639785in}{1.178722in}}%
\pgfpathlineto{\pgfqpoint{2.610086in}{1.175304in}}%
\pgfpathclose%
\pgfusepath{fill}%
\end{pgfscope}%
\begin{pgfscope}%
\pgfpathrectangle{\pgfqpoint{0.100000in}{0.100000in}}{\pgfqpoint{3.608454in}{2.310000in}}%
\pgfusepath{clip}%
\pgfsetbuttcap%
\pgfsetmiterjoin%
\definecolor{currentfill}{rgb}{0.000000,0.564706,0.717647}%
\pgfsetfillcolor{currentfill}%
\pgfsetlinewidth{0.000000pt}%
\definecolor{currentstroke}{rgb}{0.000000,0.000000,0.000000}%
\pgfsetstrokecolor{currentstroke}%
\pgfsetstrokeopacity{0.000000}%
\pgfsetdash{}{0pt}%
\pgfpathmoveto{\pgfqpoint{0.975129in}{1.855275in}}%
\pgfpathlineto{\pgfqpoint{0.981526in}{1.858299in}}%
\pgfpathlineto{\pgfqpoint{0.983732in}{1.852523in}}%
\pgfpathlineto{\pgfqpoint{0.991130in}{1.850884in}}%
\pgfpathlineto{\pgfqpoint{0.999848in}{1.846301in}}%
\pgfpathlineto{\pgfqpoint{1.012788in}{1.842181in}}%
\pgfpathlineto{\pgfqpoint{1.013709in}{1.836495in}}%
\pgfpathlineto{\pgfqpoint{1.027024in}{1.825238in}}%
\pgfpathlineto{\pgfqpoint{1.034613in}{1.812423in}}%
\pgfpathlineto{\pgfqpoint{1.038897in}{1.813232in}}%
\pgfpathlineto{\pgfqpoint{1.041894in}{1.801489in}}%
\pgfpathlineto{\pgfqpoint{1.040506in}{1.794892in}}%
\pgfpathlineto{\pgfqpoint{1.055937in}{1.791680in}}%
\pgfpathlineto{\pgfqpoint{1.054623in}{1.785268in}}%
\pgfpathlineto{\pgfqpoint{1.074676in}{1.781213in}}%
\pgfpathlineto{\pgfqpoint{1.071945in}{1.767715in}}%
\pgfpathlineto{\pgfqpoint{1.058573in}{1.770425in}}%
\pgfpathlineto{\pgfqpoint{1.053439in}{1.743275in}}%
\pgfpathlineto{\pgfqpoint{1.055372in}{1.735866in}}%
\pgfpathlineto{\pgfqpoint{1.051274in}{1.732636in}}%
\pgfpathlineto{\pgfqpoint{1.044474in}{1.739072in}}%
\pgfpathlineto{\pgfqpoint{1.038549in}{1.738796in}}%
\pgfpathlineto{\pgfqpoint{1.041428in}{1.752757in}}%
\pgfpathlineto{\pgfqpoint{1.044792in}{1.752067in}}%
\pgfpathlineto{\pgfqpoint{1.050343in}{1.779280in}}%
\pgfpathlineto{\pgfqpoint{1.033321in}{1.782826in}}%
\pgfpathlineto{\pgfqpoint{1.027689in}{1.755650in}}%
\pgfpathlineto{\pgfqpoint{1.023421in}{1.749624in}}%
\pgfpathlineto{\pgfqpoint{1.013857in}{1.751700in}}%
\pgfpathlineto{\pgfqpoint{1.010156in}{1.733884in}}%
\pgfpathlineto{\pgfqpoint{1.002092in}{1.733915in}}%
\pgfpathlineto{\pgfqpoint{1.000366in}{1.726400in}}%
\pgfpathlineto{\pgfqpoint{0.988011in}{1.729060in}}%
\pgfpathlineto{\pgfqpoint{0.981161in}{1.696275in}}%
\pgfpathlineto{\pgfqpoint{0.949739in}{1.703791in}}%
\pgfpathlineto{\pgfqpoint{0.937739in}{1.705964in}}%
\pgfpathlineto{\pgfqpoint{0.953662in}{1.776726in}}%
\pgfpathlineto{\pgfqpoint{0.950896in}{1.777540in}}%
\pgfpathlineto{\pgfqpoint{0.963091in}{1.830834in}}%
\pgfpathlineto{\pgfqpoint{0.968859in}{1.837111in}}%
\pgfpathlineto{\pgfqpoint{0.967810in}{1.841171in}}%
\pgfpathlineto{\pgfqpoint{0.973897in}{1.850229in}}%
\pgfpathlineto{\pgfqpoint{0.975129in}{1.855275in}}%
\pgfpathclose%
\pgfusepath{fill}%
\end{pgfscope}%
\begin{pgfscope}%
\pgfpathrectangle{\pgfqpoint{0.100000in}{0.100000in}}{\pgfqpoint{3.608454in}{2.310000in}}%
\pgfusepath{clip}%
\pgfsetbuttcap%
\pgfsetmiterjoin%
\definecolor{currentfill}{rgb}{0.000000,0.909804,0.545098}%
\pgfsetfillcolor{currentfill}%
\pgfsetlinewidth{0.000000pt}%
\definecolor{currentstroke}{rgb}{0.000000,0.000000,0.000000}%
\pgfsetstrokecolor{currentstroke}%
\pgfsetstrokeopacity{0.000000}%
\pgfsetdash{}{0pt}%
\pgfpathmoveto{\pgfqpoint{0.494828in}{1.757961in}}%
\pgfpathlineto{\pgfqpoint{0.489189in}{1.745779in}}%
\pgfpathlineto{\pgfqpoint{0.485004in}{1.744684in}}%
\pgfpathlineto{\pgfqpoint{0.475334in}{1.730099in}}%
\pgfpathlineto{\pgfqpoint{0.468741in}{1.723271in}}%
\pgfpathlineto{\pgfqpoint{0.469494in}{1.715995in}}%
\pgfpathlineto{\pgfqpoint{0.458930in}{1.713085in}}%
\pgfpathlineto{\pgfqpoint{0.454296in}{1.709584in}}%
\pgfpathlineto{\pgfqpoint{0.448092in}{1.709303in}}%
\pgfpathlineto{\pgfqpoint{0.442506in}{1.705022in}}%
\pgfpathlineto{\pgfqpoint{0.441132in}{1.699873in}}%
\pgfpathlineto{\pgfqpoint{0.444780in}{1.696306in}}%
\pgfpathlineto{\pgfqpoint{0.442069in}{1.688570in}}%
\pgfpathlineto{\pgfqpoint{0.441276in}{1.674970in}}%
\pgfpathlineto{\pgfqpoint{0.406038in}{1.686194in}}%
\pgfpathlineto{\pgfqpoint{0.406652in}{1.688065in}}%
\pgfpathlineto{\pgfqpoint{0.379033in}{1.697172in}}%
\pgfpathlineto{\pgfqpoint{0.376641in}{1.705935in}}%
\pgfpathlineto{\pgfqpoint{0.372917in}{1.709765in}}%
\pgfpathlineto{\pgfqpoint{0.365968in}{1.722668in}}%
\pgfpathlineto{\pgfqpoint{0.368189in}{1.726700in}}%
\pgfpathlineto{\pgfqpoint{0.367902in}{1.737177in}}%
\pgfpathlineto{\pgfqpoint{0.378459in}{1.750564in}}%
\pgfpathlineto{\pgfqpoint{0.394964in}{1.767791in}}%
\pgfpathlineto{\pgfqpoint{0.399441in}{1.775849in}}%
\pgfpathlineto{\pgfqpoint{0.397857in}{1.779799in}}%
\pgfpathlineto{\pgfqpoint{0.406645in}{1.794088in}}%
\pgfpathlineto{\pgfqpoint{0.413061in}{1.807713in}}%
\pgfpathlineto{\pgfqpoint{0.429636in}{1.802194in}}%
\pgfpathlineto{\pgfqpoint{0.427599in}{1.795918in}}%
\pgfpathlineto{\pgfqpoint{0.433774in}{1.794008in}}%
\pgfpathlineto{\pgfqpoint{0.442827in}{1.791103in}}%
\pgfpathlineto{\pgfqpoint{0.443568in}{1.772460in}}%
\pgfpathlineto{\pgfqpoint{0.449403in}{1.767395in}}%
\pgfpathlineto{\pgfqpoint{0.449915in}{1.763500in}}%
\pgfpathlineto{\pgfqpoint{0.457583in}{1.760831in}}%
\pgfpathlineto{\pgfqpoint{0.459959in}{1.754157in}}%
\pgfpathlineto{\pgfqpoint{0.468505in}{1.753515in}}%
\pgfpathlineto{\pgfqpoint{0.466061in}{1.760966in}}%
\pgfpathlineto{\pgfqpoint{0.469737in}{1.765482in}}%
\pgfpathlineto{\pgfqpoint{0.477526in}{1.765034in}}%
\pgfpathlineto{\pgfqpoint{0.494062in}{1.772814in}}%
\pgfpathlineto{\pgfqpoint{0.498294in}{1.770162in}}%
\pgfpathlineto{\pgfqpoint{0.494828in}{1.757961in}}%
\pgfpathclose%
\pgfusepath{fill}%
\end{pgfscope}%
\begin{pgfscope}%
\pgfpathrectangle{\pgfqpoint{0.100000in}{0.100000in}}{\pgfqpoint{3.608454in}{2.310000in}}%
\pgfusepath{clip}%
\pgfsetbuttcap%
\pgfsetmiterjoin%
\definecolor{currentfill}{rgb}{0.000000,0.603922,0.698039}%
\pgfsetfillcolor{currentfill}%
\pgfsetlinewidth{0.000000pt}%
\definecolor{currentstroke}{rgb}{0.000000,0.000000,0.000000}%
\pgfsetstrokecolor{currentstroke}%
\pgfsetstrokeopacity{0.000000}%
\pgfsetdash{}{0pt}%
\pgfpathmoveto{\pgfqpoint{2.246959in}{1.366834in}}%
\pgfpathlineto{\pgfqpoint{2.241843in}{1.368445in}}%
\pgfpathlineto{\pgfqpoint{2.238456in}{1.359055in}}%
\pgfpathlineto{\pgfqpoint{2.233381in}{1.359171in}}%
\pgfpathlineto{\pgfqpoint{2.230926in}{1.365957in}}%
\pgfpathlineto{\pgfqpoint{2.224430in}{1.367560in}}%
\pgfpathlineto{\pgfqpoint{2.221539in}{1.373866in}}%
\pgfpathlineto{\pgfqpoint{2.215503in}{1.374729in}}%
\pgfpathlineto{\pgfqpoint{2.211917in}{1.379066in}}%
\pgfpathlineto{\pgfqpoint{2.212943in}{1.384548in}}%
\pgfpathlineto{\pgfqpoint{2.211089in}{1.394791in}}%
\pgfpathlineto{\pgfqpoint{2.206226in}{1.396171in}}%
\pgfpathlineto{\pgfqpoint{2.206018in}{1.417015in}}%
\pgfpathlineto{\pgfqpoint{2.205934in}{1.422201in}}%
\pgfpathlineto{\pgfqpoint{2.236527in}{1.423164in}}%
\pgfpathlineto{\pgfqpoint{2.237055in}{1.423181in}}%
\pgfpathlineto{\pgfqpoint{2.237052in}{1.396510in}}%
\pgfpathlineto{\pgfqpoint{2.247305in}{1.396610in}}%
\pgfpathlineto{\pgfqpoint{2.246959in}{1.366834in}}%
\pgfpathclose%
\pgfusepath{fill}%
\end{pgfscope}%
\begin{pgfscope}%
\pgfpathrectangle{\pgfqpoint{0.100000in}{0.100000in}}{\pgfqpoint{3.608454in}{2.310000in}}%
\pgfusepath{clip}%
\pgfsetbuttcap%
\pgfsetmiterjoin%
\definecolor{currentfill}{rgb}{0.000000,0.572549,0.713725}%
\pgfsetfillcolor{currentfill}%
\pgfsetlinewidth{0.000000pt}%
\definecolor{currentstroke}{rgb}{0.000000,0.000000,0.000000}%
\pgfsetstrokecolor{currentstroke}%
\pgfsetstrokeopacity{0.000000}%
\pgfsetdash{}{0pt}%
\pgfpathmoveto{\pgfqpoint{1.857172in}{0.996308in}}%
\pgfpathlineto{\pgfqpoint{1.856890in}{0.989419in}}%
\pgfpathlineto{\pgfqpoint{1.867617in}{0.988984in}}%
\pgfpathlineto{\pgfqpoint{1.867064in}{0.975207in}}%
\pgfpathlineto{\pgfqpoint{1.870481in}{0.975087in}}%
\pgfpathlineto{\pgfqpoint{1.869882in}{0.961093in}}%
\pgfpathlineto{\pgfqpoint{1.864563in}{0.959406in}}%
\pgfpathlineto{\pgfqpoint{1.852516in}{0.961694in}}%
\pgfpathlineto{\pgfqpoint{1.847621in}{0.966431in}}%
\pgfpathlineto{\pgfqpoint{1.832097in}{0.967259in}}%
\pgfpathlineto{\pgfqpoint{1.831143in}{0.977031in}}%
\pgfpathlineto{\pgfqpoint{1.831022in}{0.989497in}}%
\pgfpathlineto{\pgfqpoint{1.835171in}{0.996785in}}%
\pgfpathlineto{\pgfqpoint{1.857172in}{0.996308in}}%
\pgfpathclose%
\pgfusepath{fill}%
\end{pgfscope}%
\begin{pgfscope}%
\pgfpathrectangle{\pgfqpoint{0.100000in}{0.100000in}}{\pgfqpoint{3.608454in}{2.310000in}}%
\pgfusepath{clip}%
\pgfsetbuttcap%
\pgfsetmiterjoin%
\definecolor{currentfill}{rgb}{0.000000,0.533333,0.733333}%
\pgfsetfillcolor{currentfill}%
\pgfsetlinewidth{0.000000pt}%
\definecolor{currentstroke}{rgb}{0.000000,0.000000,0.000000}%
\pgfsetstrokecolor{currentstroke}%
\pgfsetstrokeopacity{0.000000}%
\pgfsetdash{}{0pt}%
\pgfpathmoveto{\pgfqpoint{2.046244in}{1.696054in}}%
\pgfpathlineto{\pgfqpoint{2.023239in}{1.696380in}}%
\pgfpathlineto{\pgfqpoint{2.023670in}{1.724067in}}%
\pgfpathlineto{\pgfqpoint{2.024621in}{1.779133in}}%
\pgfpathlineto{\pgfqpoint{2.024747in}{1.786093in}}%
\pgfpathlineto{\pgfqpoint{2.072659in}{1.785449in}}%
\pgfpathlineto{\pgfqpoint{2.073124in}{1.778481in}}%
\pgfpathlineto{\pgfqpoint{2.073025in}{1.750974in}}%
\pgfpathlineto{\pgfqpoint{2.046268in}{1.751283in}}%
\pgfpathlineto{\pgfqpoint{2.046612in}{1.723715in}}%
\pgfpathlineto{\pgfqpoint{2.046244in}{1.696054in}}%
\pgfpathclose%
\pgfusepath{fill}%
\end{pgfscope}%
\begin{pgfscope}%
\pgfpathrectangle{\pgfqpoint{0.100000in}{0.100000in}}{\pgfqpoint{3.608454in}{2.310000in}}%
\pgfusepath{clip}%
\pgfsetbuttcap%
\pgfsetmiterjoin%
\definecolor{currentfill}{rgb}{0.000000,0.490196,0.754902}%
\pgfsetfillcolor{currentfill}%
\pgfsetlinewidth{0.000000pt}%
\definecolor{currentstroke}{rgb}{0.000000,0.000000,0.000000}%
\pgfsetstrokecolor{currentstroke}%
\pgfsetstrokeopacity{0.000000}%
\pgfsetdash{}{0pt}%
\pgfpathmoveto{\pgfqpoint{1.461309in}{1.650537in}}%
\pgfpathlineto{\pgfqpoint{1.457886in}{1.623314in}}%
\pgfpathlineto{\pgfqpoint{1.451617in}{1.568947in}}%
\pgfpathlineto{\pgfqpoint{1.436631in}{1.570812in}}%
\pgfpathlineto{\pgfqpoint{1.432841in}{1.539719in}}%
\pgfpathlineto{\pgfqpoint{1.401111in}{1.544070in}}%
\pgfpathlineto{\pgfqpoint{1.338390in}{1.552623in}}%
\pgfpathlineto{\pgfqpoint{1.342910in}{1.583654in}}%
\pgfpathlineto{\pgfqpoint{1.345097in}{1.604208in}}%
\pgfpathlineto{\pgfqpoint{1.369658in}{1.600576in}}%
\pgfpathlineto{\pgfqpoint{1.374676in}{1.634274in}}%
\pgfpathlineto{\pgfqpoint{1.375467in}{1.648053in}}%
\pgfpathlineto{\pgfqpoint{1.377332in}{1.661590in}}%
\pgfpathlineto{\pgfqpoint{1.429302in}{1.654336in}}%
\pgfpathlineto{\pgfqpoint{1.461309in}{1.650537in}}%
\pgfpathclose%
\pgfusepath{fill}%
\end{pgfscope}%
\begin{pgfscope}%
\pgfpathrectangle{\pgfqpoint{0.100000in}{0.100000in}}{\pgfqpoint{3.608454in}{2.310000in}}%
\pgfusepath{clip}%
\pgfsetbuttcap%
\pgfsetmiterjoin%
\definecolor{currentfill}{rgb}{0.000000,0.301961,0.849020}%
\pgfsetfillcolor{currentfill}%
\pgfsetlinewidth{0.000000pt}%
\definecolor{currentstroke}{rgb}{0.000000,0.000000,0.000000}%
\pgfsetstrokecolor{currentstroke}%
\pgfsetstrokeopacity{0.000000}%
\pgfsetdash{}{0pt}%
\pgfpathmoveto{\pgfqpoint{1.963356in}{1.420723in}}%
\pgfpathlineto{\pgfqpoint{1.964117in}{1.448246in}}%
\pgfpathlineto{\pgfqpoint{1.965564in}{1.503301in}}%
\pgfpathlineto{\pgfqpoint{1.992970in}{1.502588in}}%
\pgfpathlineto{\pgfqpoint{2.019500in}{1.502021in}}%
\pgfpathlineto{\pgfqpoint{2.019155in}{1.481378in}}%
\pgfpathlineto{\pgfqpoint{2.018170in}{1.419481in}}%
\pgfpathlineto{\pgfqpoint{1.963356in}{1.420723in}}%
\pgfpathclose%
\pgfusepath{fill}%
\end{pgfscope}%
\begin{pgfscope}%
\pgfpathrectangle{\pgfqpoint{0.100000in}{0.100000in}}{\pgfqpoint{3.608454in}{2.310000in}}%
\pgfusepath{clip}%
\pgfsetbuttcap%
\pgfsetmiterjoin%
\definecolor{currentfill}{rgb}{0.000000,0.458824,0.770588}%
\pgfsetfillcolor{currentfill}%
\pgfsetlinewidth{0.000000pt}%
\definecolor{currentstroke}{rgb}{0.000000,0.000000,0.000000}%
\pgfsetstrokecolor{currentstroke}%
\pgfsetstrokeopacity{0.000000}%
\pgfsetdash{}{0pt}%
\pgfpathmoveto{\pgfqpoint{1.960057in}{1.995378in}}%
\pgfpathlineto{\pgfqpoint{1.946235in}{1.995849in}}%
\pgfpathlineto{\pgfqpoint{1.945081in}{2.002729in}}%
\pgfpathlineto{\pgfqpoint{1.946085in}{2.030374in}}%
\pgfpathlineto{\pgfqpoint{1.951584in}{2.030204in}}%
\pgfpathlineto{\pgfqpoint{1.952493in}{2.057946in}}%
\pgfpathlineto{\pgfqpoint{1.951799in}{2.072044in}}%
\pgfpathlineto{\pgfqpoint{1.992478in}{2.070813in}}%
\pgfpathlineto{\pgfqpoint{1.992147in}{2.069188in}}%
\pgfpathlineto{\pgfqpoint{2.026639in}{2.068568in}}%
\pgfpathlineto{\pgfqpoint{2.027218in}{2.042403in}}%
\pgfpathlineto{\pgfqpoint{2.034152in}{2.042276in}}%
\pgfpathlineto{\pgfqpoint{2.034103in}{2.035335in}}%
\pgfpathlineto{\pgfqpoint{2.061781in}{2.035022in}}%
\pgfpathlineto{\pgfqpoint{2.061858in}{2.042025in}}%
\pgfpathlineto{\pgfqpoint{2.068716in}{2.041994in}}%
\pgfpathlineto{\pgfqpoint{2.068749in}{2.048976in}}%
\pgfpathlineto{\pgfqpoint{2.075577in}{2.048780in}}%
\pgfpathlineto{\pgfqpoint{2.075622in}{2.027945in}}%
\pgfpathlineto{\pgfqpoint{2.076864in}{2.014087in}}%
\pgfpathlineto{\pgfqpoint{2.049145in}{2.014243in}}%
\pgfpathlineto{\pgfqpoint{2.006954in}{2.014902in}}%
\pgfpathlineto{\pgfqpoint{2.007393in}{1.994001in}}%
\pgfpathlineto{\pgfqpoint{1.960057in}{1.995378in}}%
\pgfpathclose%
\pgfusepath{fill}%
\end{pgfscope}%
\begin{pgfscope}%
\pgfpathrectangle{\pgfqpoint{0.100000in}{0.100000in}}{\pgfqpoint{3.608454in}{2.310000in}}%
\pgfusepath{clip}%
\pgfsetbuttcap%
\pgfsetmiterjoin%
\definecolor{currentfill}{rgb}{0.000000,0.454902,0.772549}%
\pgfsetfillcolor{currentfill}%
\pgfsetlinewidth{0.000000pt}%
\definecolor{currentstroke}{rgb}{0.000000,0.000000,0.000000}%
\pgfsetstrokecolor{currentstroke}%
\pgfsetstrokeopacity{0.000000}%
\pgfsetdash{}{0pt}%
\pgfpathmoveto{\pgfqpoint{2.100603in}{1.750815in}}%
\pgfpathlineto{\pgfqpoint{2.073025in}{1.750974in}}%
\pgfpathlineto{\pgfqpoint{2.073124in}{1.778481in}}%
\pgfpathlineto{\pgfqpoint{2.086396in}{1.778361in}}%
\pgfpathlineto{\pgfqpoint{2.086458in}{1.790823in}}%
\pgfpathlineto{\pgfqpoint{2.079554in}{1.795195in}}%
\pgfpathlineto{\pgfqpoint{2.079580in}{1.806133in}}%
\pgfpathlineto{\pgfqpoint{2.120405in}{1.806180in}}%
\pgfpathlineto{\pgfqpoint{2.135067in}{1.806256in}}%
\pgfpathlineto{\pgfqpoint{2.135124in}{1.792351in}}%
\pgfpathlineto{\pgfqpoint{2.127691in}{1.792284in}}%
\pgfpathlineto{\pgfqpoint{2.128088in}{1.771585in}}%
\pgfpathlineto{\pgfqpoint{2.118639in}{1.771585in}}%
\pgfpathlineto{\pgfqpoint{2.114392in}{1.774876in}}%
\pgfpathlineto{\pgfqpoint{2.114393in}{1.757738in}}%
\pgfpathlineto{\pgfqpoint{2.100687in}{1.757698in}}%
\pgfpathlineto{\pgfqpoint{2.100603in}{1.750815in}}%
\pgfpathclose%
\pgfusepath{fill}%
\end{pgfscope}%
\begin{pgfscope}%
\pgfpathrectangle{\pgfqpoint{0.100000in}{0.100000in}}{\pgfqpoint{3.608454in}{2.310000in}}%
\pgfusepath{clip}%
\pgfsetbuttcap%
\pgfsetmiterjoin%
\definecolor{currentfill}{rgb}{0.000000,0.796078,0.601961}%
\pgfsetfillcolor{currentfill}%
\pgfsetlinewidth{0.000000pt}%
\definecolor{currentstroke}{rgb}{0.000000,0.000000,0.000000}%
\pgfsetstrokecolor{currentstroke}%
\pgfsetstrokeopacity{0.000000}%
\pgfsetdash{}{0pt}%
\pgfpathmoveto{\pgfqpoint{2.318616in}{2.073687in}}%
\pgfpathlineto{\pgfqpoint{2.328248in}{2.078608in}}%
\pgfpathlineto{\pgfqpoint{2.333925in}{2.066565in}}%
\pgfpathlineto{\pgfqpoint{2.354068in}{2.067548in}}%
\pgfpathlineto{\pgfqpoint{2.366718in}{2.069902in}}%
\pgfpathlineto{\pgfqpoint{2.372839in}{2.068040in}}%
\pgfpathlineto{\pgfqpoint{2.374681in}{2.063610in}}%
\pgfpathlineto{\pgfqpoint{2.381315in}{2.060550in}}%
\pgfpathlineto{\pgfqpoint{2.386736in}{2.063805in}}%
\pgfpathlineto{\pgfqpoint{2.394855in}{2.063381in}}%
\pgfpathlineto{\pgfqpoint{2.394022in}{2.058782in}}%
\pgfpathlineto{\pgfqpoint{2.376788in}{2.048043in}}%
\pgfpathlineto{\pgfqpoint{2.365376in}{2.042875in}}%
\pgfpathlineto{\pgfqpoint{2.346958in}{2.036101in}}%
\pgfpathlineto{\pgfqpoint{2.336678in}{2.029365in}}%
\pgfpathlineto{\pgfqpoint{2.321610in}{2.015873in}}%
\pgfpathlineto{\pgfqpoint{2.320258in}{2.049681in}}%
\pgfpathlineto{\pgfqpoint{2.318616in}{2.073687in}}%
\pgfpathclose%
\pgfusepath{fill}%
\end{pgfscope}%
\begin{pgfscope}%
\pgfpathrectangle{\pgfqpoint{0.100000in}{0.100000in}}{\pgfqpoint{3.608454in}{2.310000in}}%
\pgfusepath{clip}%
\pgfsetbuttcap%
\pgfsetmiterjoin%
\definecolor{currentfill}{rgb}{0.000000,0.737255,0.631373}%
\pgfsetfillcolor{currentfill}%
\pgfsetlinewidth{0.000000pt}%
\definecolor{currentstroke}{rgb}{0.000000,0.000000,0.000000}%
\pgfsetstrokecolor{currentstroke}%
\pgfsetstrokeopacity{0.000000}%
\pgfsetdash{}{0pt}%
\pgfpathmoveto{\pgfqpoint{2.842704in}{1.366486in}}%
\pgfpathlineto{\pgfqpoint{2.836077in}{1.357238in}}%
\pgfpathlineto{\pgfqpoint{2.829317in}{1.356171in}}%
\pgfpathlineto{\pgfqpoint{2.825080in}{1.387318in}}%
\pgfpathlineto{\pgfqpoint{2.817692in}{1.389496in}}%
\pgfpathlineto{\pgfqpoint{2.818458in}{1.403774in}}%
\pgfpathlineto{\pgfqpoint{2.816007in}{1.404773in}}%
\pgfpathlineto{\pgfqpoint{2.815226in}{1.414852in}}%
\pgfpathlineto{\pgfqpoint{2.820238in}{1.426570in}}%
\pgfpathlineto{\pgfqpoint{2.837395in}{1.428067in}}%
\pgfpathlineto{\pgfqpoint{2.852553in}{1.427030in}}%
\pgfpathlineto{\pgfqpoint{2.853130in}{1.419128in}}%
\pgfpathlineto{\pgfqpoint{2.867100in}{1.421144in}}%
\pgfpathlineto{\pgfqpoint{2.867003in}{1.423229in}}%
\pgfpathlineto{\pgfqpoint{2.880748in}{1.424222in}}%
\pgfpathlineto{\pgfqpoint{2.883523in}{1.417325in}}%
\pgfpathlineto{\pgfqpoint{2.881668in}{1.410091in}}%
\pgfpathlineto{\pgfqpoint{2.882640in}{1.396033in}}%
\pgfpathlineto{\pgfqpoint{2.875723in}{1.395664in}}%
\pgfpathlineto{\pgfqpoint{2.876585in}{1.380575in}}%
\pgfpathlineto{\pgfqpoint{2.869245in}{1.379552in}}%
\pgfpathlineto{\pgfqpoint{2.864652in}{1.379289in}}%
\pgfpathlineto{\pgfqpoint{2.864952in}{1.371264in}}%
\pgfpathlineto{\pgfqpoint{2.861822in}{1.358808in}}%
\pgfpathlineto{\pgfqpoint{2.854867in}{1.358657in}}%
\pgfpathlineto{\pgfqpoint{2.850733in}{1.370027in}}%
\pgfpathlineto{\pgfqpoint{2.842704in}{1.366486in}}%
\pgfpathclose%
\pgfusepath{fill}%
\end{pgfscope}%
\begin{pgfscope}%
\pgfpathrectangle{\pgfqpoint{0.100000in}{0.100000in}}{\pgfqpoint{3.608454in}{2.310000in}}%
\pgfusepath{clip}%
\pgfsetbuttcap%
\pgfsetmiterjoin%
\definecolor{currentfill}{rgb}{0.000000,0.537255,0.731373}%
\pgfsetfillcolor{currentfill}%
\pgfsetlinewidth{0.000000pt}%
\definecolor{currentstroke}{rgb}{0.000000,0.000000,0.000000}%
\pgfsetstrokecolor{currentstroke}%
\pgfsetstrokeopacity{0.000000}%
\pgfsetdash{}{0pt}%
\pgfpathmoveto{\pgfqpoint{2.402794in}{1.625171in}}%
\pgfpathlineto{\pgfqpoint{2.404561in}{1.600729in}}%
\pgfpathlineto{\pgfqpoint{2.381293in}{1.599262in}}%
\pgfpathlineto{\pgfqpoint{2.376710in}{1.601691in}}%
\pgfpathlineto{\pgfqpoint{2.374950in}{1.609365in}}%
\pgfpathlineto{\pgfqpoint{2.371380in}{1.613652in}}%
\pgfpathlineto{\pgfqpoint{2.360207in}{1.613167in}}%
\pgfpathlineto{\pgfqpoint{2.360507in}{1.606246in}}%
\pgfpathlineto{\pgfqpoint{2.346907in}{1.605656in}}%
\pgfpathlineto{\pgfqpoint{2.306014in}{1.604048in}}%
\pgfpathlineto{\pgfqpoint{2.304826in}{1.617701in}}%
\pgfpathlineto{\pgfqpoint{2.303199in}{1.666191in}}%
\pgfpathlineto{\pgfqpoint{2.301639in}{1.699356in}}%
\pgfpathlineto{\pgfqpoint{2.324231in}{1.700249in}}%
\pgfpathlineto{\pgfqpoint{2.325187in}{1.694132in}}%
\pgfpathlineto{\pgfqpoint{2.325644in}{1.688292in}}%
\pgfpathlineto{\pgfqpoint{2.334254in}{1.681222in}}%
\pgfpathlineto{\pgfqpoint{2.327890in}{1.671376in}}%
\pgfpathlineto{\pgfqpoint{2.330505in}{1.653247in}}%
\pgfpathlineto{\pgfqpoint{2.333060in}{1.651644in}}%
\pgfpathlineto{\pgfqpoint{2.336274in}{1.640299in}}%
\pgfpathlineto{\pgfqpoint{2.342986in}{1.636283in}}%
\pgfpathlineto{\pgfqpoint{2.356020in}{1.633465in}}%
\pgfpathlineto{\pgfqpoint{2.361136in}{1.623207in}}%
\pgfpathlineto{\pgfqpoint{2.402794in}{1.625171in}}%
\pgfpathclose%
\pgfusepath{fill}%
\end{pgfscope}%
\begin{pgfscope}%
\pgfpathrectangle{\pgfqpoint{0.100000in}{0.100000in}}{\pgfqpoint{3.608454in}{2.310000in}}%
\pgfusepath{clip}%
\pgfsetbuttcap%
\pgfsetmiterjoin%
\definecolor{currentfill}{rgb}{0.000000,0.576471,0.711765}%
\pgfsetfillcolor{currentfill}%
\pgfsetlinewidth{0.000000pt}%
\definecolor{currentstroke}{rgb}{0.000000,0.000000,0.000000}%
\pgfsetstrokecolor{currentstroke}%
\pgfsetstrokeopacity{0.000000}%
\pgfsetdash{}{0pt}%
\pgfpathmoveto{\pgfqpoint{3.105607in}{1.274029in}}%
\pgfpathlineto{\pgfqpoint{3.099433in}{1.272390in}}%
\pgfpathlineto{\pgfqpoint{3.095309in}{1.277259in}}%
\pgfpathlineto{\pgfqpoint{3.089238in}{1.276542in}}%
\pgfpathlineto{\pgfqpoint{3.081510in}{1.270060in}}%
\pgfpathlineto{\pgfqpoint{3.074664in}{1.267099in}}%
\pgfpathlineto{\pgfqpoint{3.069985in}{1.273083in}}%
\pgfpathlineto{\pgfqpoint{3.057510in}{1.278132in}}%
\pgfpathlineto{\pgfqpoint{3.055501in}{1.285981in}}%
\pgfpathlineto{\pgfqpoint{3.057612in}{1.293606in}}%
\pgfpathlineto{\pgfqpoint{3.064782in}{1.300929in}}%
\pgfpathlineto{\pgfqpoint{3.067687in}{1.299266in}}%
\pgfpathlineto{\pgfqpoint{3.074021in}{1.306357in}}%
\pgfpathlineto{\pgfqpoint{3.075683in}{1.311970in}}%
\pgfpathlineto{\pgfqpoint{3.081757in}{1.317949in}}%
\pgfpathlineto{\pgfqpoint{3.084921in}{1.329255in}}%
\pgfpathlineto{\pgfqpoint{3.092431in}{1.332135in}}%
\pgfpathlineto{\pgfqpoint{3.098792in}{1.325084in}}%
\pgfpathlineto{\pgfqpoint{3.103381in}{1.324700in}}%
\pgfpathlineto{\pgfqpoint{3.110914in}{1.313734in}}%
\pgfpathlineto{\pgfqpoint{3.115324in}{1.315206in}}%
\pgfpathlineto{\pgfqpoint{3.125802in}{1.307188in}}%
\pgfpathlineto{\pgfqpoint{3.131753in}{1.305762in}}%
\pgfpathlineto{\pgfqpoint{3.128475in}{1.293170in}}%
\pgfpathlineto{\pgfqpoint{3.120247in}{1.288198in}}%
\pgfpathlineto{\pgfqpoint{3.117815in}{1.273052in}}%
\pgfpathlineto{\pgfqpoint{3.105607in}{1.274029in}}%
\pgfpathclose%
\pgfusepath{fill}%
\end{pgfscope}%
\begin{pgfscope}%
\pgfpathrectangle{\pgfqpoint{0.100000in}{0.100000in}}{\pgfqpoint{3.608454in}{2.310000in}}%
\pgfusepath{clip}%
\pgfsetbuttcap%
\pgfsetmiterjoin%
\definecolor{currentfill}{rgb}{0.000000,0.909804,0.545098}%
\pgfsetfillcolor{currentfill}%
\pgfsetlinewidth{0.000000pt}%
\definecolor{currentstroke}{rgb}{0.000000,0.000000,0.000000}%
\pgfsetstrokecolor{currentstroke}%
\pgfsetstrokeopacity{0.000000}%
\pgfsetdash{}{0pt}%
\pgfpathmoveto{\pgfqpoint{0.616244in}{1.922404in}}%
\pgfpathlineto{\pgfqpoint{0.668725in}{1.906595in}}%
\pgfpathlineto{\pgfqpoint{0.695319in}{1.898980in}}%
\pgfpathlineto{\pgfqpoint{0.686047in}{1.866076in}}%
\pgfpathlineto{\pgfqpoint{0.684062in}{1.866632in}}%
\pgfpathlineto{\pgfqpoint{0.674134in}{1.833816in}}%
\pgfpathlineto{\pgfqpoint{0.706614in}{1.825006in}}%
\pgfpathlineto{\pgfqpoint{0.691075in}{1.767312in}}%
\pgfpathlineto{\pgfqpoint{0.654872in}{1.777406in}}%
\pgfpathlineto{\pgfqpoint{0.605093in}{1.791654in}}%
\pgfpathlineto{\pgfqpoint{0.621562in}{1.848713in}}%
\pgfpathlineto{\pgfqpoint{0.595602in}{1.856614in}}%
\pgfpathlineto{\pgfqpoint{0.609494in}{1.902912in}}%
\pgfpathlineto{\pgfqpoint{0.616244in}{1.922404in}}%
\pgfpathclose%
\pgfusepath{fill}%
\end{pgfscope}%
\begin{pgfscope}%
\pgfpathrectangle{\pgfqpoint{0.100000in}{0.100000in}}{\pgfqpoint{3.608454in}{2.310000in}}%
\pgfusepath{clip}%
\pgfsetbuttcap%
\pgfsetmiterjoin%
\definecolor{currentfill}{rgb}{0.000000,0.403922,0.798039}%
\pgfsetfillcolor{currentfill}%
\pgfsetlinewidth{0.000000pt}%
\definecolor{currentstroke}{rgb}{0.000000,0.000000,0.000000}%
\pgfsetstrokecolor{currentstroke}%
\pgfsetstrokeopacity{0.000000}%
\pgfsetdash{}{0pt}%
\pgfpathmoveto{\pgfqpoint{2.139356in}{1.083433in}}%
\pgfpathlineto{\pgfqpoint{2.134726in}{1.115276in}}%
\pgfpathlineto{\pgfqpoint{2.106071in}{1.115212in}}%
\pgfpathlineto{\pgfqpoint{2.106820in}{1.155478in}}%
\pgfpathlineto{\pgfqpoint{2.131026in}{1.155307in}}%
\pgfpathlineto{\pgfqpoint{2.130979in}{1.163114in}}%
\pgfpathlineto{\pgfqpoint{2.165811in}{1.161887in}}%
\pgfpathlineto{\pgfqpoint{2.165488in}{1.142173in}}%
\pgfpathlineto{\pgfqpoint{2.178864in}{1.142325in}}%
\pgfpathlineto{\pgfqpoint{2.178906in}{1.127188in}}%
\pgfpathlineto{\pgfqpoint{2.192391in}{1.127027in}}%
\pgfpathlineto{\pgfqpoint{2.194911in}{1.119004in}}%
\pgfpathlineto{\pgfqpoint{2.199417in}{1.113130in}}%
\pgfpathlineto{\pgfqpoint{2.206212in}{1.109649in}}%
\pgfpathlineto{\pgfqpoint{2.206038in}{1.100680in}}%
\pgfpathlineto{\pgfqpoint{2.201478in}{1.093804in}}%
\pgfpathlineto{\pgfqpoint{2.201915in}{1.084462in}}%
\pgfpathlineto{\pgfqpoint{2.176762in}{1.083808in}}%
\pgfpathlineto{\pgfqpoint{2.163020in}{1.082921in}}%
\pgfpathlineto{\pgfqpoint{2.139356in}{1.083433in}}%
\pgfpathclose%
\pgfusepath{fill}%
\end{pgfscope}%
\begin{pgfscope}%
\pgfpathrectangle{\pgfqpoint{0.100000in}{0.100000in}}{\pgfqpoint{3.608454in}{2.310000in}}%
\pgfusepath{clip}%
\pgfsetbuttcap%
\pgfsetmiterjoin%
\definecolor{currentfill}{rgb}{0.000000,0.533333,0.733333}%
\pgfsetfillcolor{currentfill}%
\pgfsetlinewidth{0.000000pt}%
\definecolor{currentstroke}{rgb}{0.000000,0.000000,0.000000}%
\pgfsetstrokecolor{currentstroke}%
\pgfsetstrokeopacity{0.000000}%
\pgfsetdash{}{0pt}%
\pgfpathmoveto{\pgfqpoint{3.081713in}{0.979204in}}%
\pgfpathlineto{\pgfqpoint{3.081498in}{0.967552in}}%
\pgfpathlineto{\pgfqpoint{3.070676in}{0.962442in}}%
\pgfpathlineto{\pgfqpoint{3.064743in}{0.967659in}}%
\pgfpathlineto{\pgfqpoint{3.047350in}{0.952216in}}%
\pgfpathlineto{\pgfqpoint{3.034445in}{0.961587in}}%
\pgfpathlineto{\pgfqpoint{3.018836in}{0.966601in}}%
\pgfpathlineto{\pgfqpoint{3.010939in}{0.970937in}}%
\pgfpathlineto{\pgfqpoint{3.005840in}{0.971004in}}%
\pgfpathlineto{\pgfqpoint{3.016045in}{0.985596in}}%
\pgfpathlineto{\pgfqpoint{2.999338in}{0.990301in}}%
\pgfpathlineto{\pgfqpoint{2.988314in}{0.999433in}}%
\pgfpathlineto{\pgfqpoint{2.983807in}{0.993720in}}%
\pgfpathlineto{\pgfqpoint{2.971145in}{0.996030in}}%
\pgfpathlineto{\pgfqpoint{2.966134in}{1.004041in}}%
\pgfpathlineto{\pgfqpoint{2.958909in}{1.001900in}}%
\pgfpathlineto{\pgfqpoint{2.963991in}{1.017466in}}%
\pgfpathlineto{\pgfqpoint{2.961263in}{1.022919in}}%
\pgfpathlineto{\pgfqpoint{2.962380in}{1.029555in}}%
\pgfpathlineto{\pgfqpoint{2.968957in}{1.035717in}}%
\pgfpathlineto{\pgfqpoint{2.975999in}{1.050234in}}%
\pgfpathlineto{\pgfqpoint{2.982517in}{1.046654in}}%
\pgfpathlineto{\pgfqpoint{2.990720in}{1.049995in}}%
\pgfpathlineto{\pgfqpoint{2.989806in}{1.055375in}}%
\pgfpathlineto{\pgfqpoint{3.025339in}{1.058515in}}%
\pgfpathlineto{\pgfqpoint{3.026275in}{1.051810in}}%
\pgfpathlineto{\pgfqpoint{3.034331in}{1.046602in}}%
\pgfpathlineto{\pgfqpoint{3.030270in}{1.030329in}}%
\pgfpathlineto{\pgfqpoint{3.041428in}{1.022979in}}%
\pgfpathlineto{\pgfqpoint{3.046504in}{1.026091in}}%
\pgfpathlineto{\pgfqpoint{3.049502in}{1.021849in}}%
\pgfpathlineto{\pgfqpoint{3.049719in}{1.006843in}}%
\pgfpathlineto{\pgfqpoint{3.052596in}{1.003504in}}%
\pgfpathlineto{\pgfqpoint{3.051594in}{0.997455in}}%
\pgfpathlineto{\pgfqpoint{3.059264in}{0.991086in}}%
\pgfpathlineto{\pgfqpoint{3.065412in}{0.981312in}}%
\pgfpathlineto{\pgfqpoint{3.073391in}{0.976759in}}%
\pgfpathlineto{\pgfqpoint{3.081713in}{0.979204in}}%
\pgfpathclose%
\pgfusepath{fill}%
\end{pgfscope}%
\begin{pgfscope}%
\pgfpathrectangle{\pgfqpoint{0.100000in}{0.100000in}}{\pgfqpoint{3.608454in}{2.310000in}}%
\pgfusepath{clip}%
\pgfsetbuttcap%
\pgfsetmiterjoin%
\definecolor{currentfill}{rgb}{0.000000,0.811765,0.594118}%
\pgfsetfillcolor{currentfill}%
\pgfsetlinewidth{0.000000pt}%
\definecolor{currentstroke}{rgb}{0.000000,0.000000,0.000000}%
\pgfsetstrokecolor{currentstroke}%
\pgfsetstrokeopacity{0.000000}%
\pgfsetdash{}{0pt}%
\pgfpathmoveto{\pgfqpoint{2.876403in}{1.073700in}}%
\pgfpathlineto{\pgfqpoint{2.876200in}{1.069013in}}%
\pgfpathlineto{\pgfqpoint{2.869918in}{1.063176in}}%
\pgfpathlineto{\pgfqpoint{2.864683in}{1.055271in}}%
\pgfpathlineto{\pgfqpoint{2.863588in}{1.049271in}}%
\pgfpathlineto{\pgfqpoint{2.850394in}{1.050302in}}%
\pgfpathlineto{\pgfqpoint{2.847354in}{1.055558in}}%
\pgfpathlineto{\pgfqpoint{2.843160in}{1.055221in}}%
\pgfpathlineto{\pgfqpoint{2.842470in}{1.059416in}}%
\pgfpathlineto{\pgfqpoint{2.850115in}{1.070383in}}%
\pgfpathlineto{\pgfqpoint{2.840206in}{1.081206in}}%
\pgfpathlineto{\pgfqpoint{2.834354in}{1.080705in}}%
\pgfpathlineto{\pgfqpoint{2.835727in}{1.088275in}}%
\pgfpathlineto{\pgfqpoint{2.837326in}{1.091183in}}%
\pgfpathlineto{\pgfqpoint{2.851025in}{1.094550in}}%
\pgfpathlineto{\pgfqpoint{2.858178in}{1.097707in}}%
\pgfpathlineto{\pgfqpoint{2.867388in}{1.085037in}}%
\pgfpathlineto{\pgfqpoint{2.872545in}{1.080983in}}%
\pgfpathlineto{\pgfqpoint{2.876403in}{1.073700in}}%
\pgfpathclose%
\pgfusepath{fill}%
\end{pgfscope}%
\begin{pgfscope}%
\pgfpathrectangle{\pgfqpoint{0.100000in}{0.100000in}}{\pgfqpoint{3.608454in}{2.310000in}}%
\pgfusepath{clip}%
\pgfsetbuttcap%
\pgfsetmiterjoin%
\definecolor{currentfill}{rgb}{0.000000,0.768627,0.615686}%
\pgfsetfillcolor{currentfill}%
\pgfsetlinewidth{0.000000pt}%
\definecolor{currentstroke}{rgb}{0.000000,0.000000,0.000000}%
\pgfsetstrokecolor{currentstroke}%
\pgfsetstrokeopacity{0.000000}%
\pgfsetdash{}{0pt}%
\pgfpathmoveto{\pgfqpoint{2.288166in}{1.125922in}}%
\pgfpathlineto{\pgfqpoint{2.286012in}{1.115699in}}%
\pgfpathlineto{\pgfqpoint{2.286066in}{1.109865in}}%
\pgfpathlineto{\pgfqpoint{2.279199in}{1.109799in}}%
\pgfpathlineto{\pgfqpoint{2.279258in}{1.102907in}}%
\pgfpathlineto{\pgfqpoint{2.272396in}{1.102853in}}%
\pgfpathlineto{\pgfqpoint{2.272381in}{1.109733in}}%
\pgfpathlineto{\pgfqpoint{2.245103in}{1.109604in}}%
\pgfpathlineto{\pgfqpoint{2.241638in}{1.113006in}}%
\pgfpathlineto{\pgfqpoint{2.241311in}{1.142821in}}%
\pgfpathlineto{\pgfqpoint{2.243255in}{1.143432in}}%
\pgfpathlineto{\pgfqpoint{2.248459in}{1.143536in}}%
\pgfpathlineto{\pgfqpoint{2.248336in}{1.167927in}}%
\pgfpathlineto{\pgfqpoint{2.289673in}{1.168159in}}%
\pgfpathlineto{\pgfqpoint{2.289909in}{1.144788in}}%
\pgfpathlineto{\pgfqpoint{2.288004in}{1.144708in}}%
\pgfpathlineto{\pgfqpoint{2.288166in}{1.125922in}}%
\pgfpathclose%
\pgfusepath{fill}%
\end{pgfscope}%
\begin{pgfscope}%
\pgfpathrectangle{\pgfqpoint{0.100000in}{0.100000in}}{\pgfqpoint{3.608454in}{2.310000in}}%
\pgfusepath{clip}%
\pgfsetbuttcap%
\pgfsetmiterjoin%
\definecolor{currentfill}{rgb}{0.000000,0.847059,0.576471}%
\pgfsetfillcolor{currentfill}%
\pgfsetlinewidth{0.000000pt}%
\definecolor{currentstroke}{rgb}{0.000000,0.000000,0.000000}%
\pgfsetstrokecolor{currentstroke}%
\pgfsetstrokeopacity{0.000000}%
\pgfsetdash{}{0pt}%
\pgfpathmoveto{\pgfqpoint{3.197116in}{1.277528in}}%
\pgfpathlineto{\pgfqpoint{3.199968in}{1.263498in}}%
\pgfpathlineto{\pgfqpoint{3.195545in}{1.249102in}}%
\pgfpathlineto{\pgfqpoint{3.187275in}{1.247460in}}%
\pgfpathlineto{\pgfqpoint{3.152496in}{1.240772in}}%
\pgfpathlineto{\pgfqpoint{3.109908in}{1.233039in}}%
\pgfpathlineto{\pgfqpoint{3.104858in}{1.232143in}}%
\pgfpathlineto{\pgfqpoint{3.105607in}{1.274029in}}%
\pgfpathlineto{\pgfqpoint{3.117815in}{1.273052in}}%
\pgfpathlineto{\pgfqpoint{3.120247in}{1.288198in}}%
\pgfpathlineto{\pgfqpoint{3.128475in}{1.293170in}}%
\pgfpathlineto{\pgfqpoint{3.131753in}{1.305762in}}%
\pgfpathlineto{\pgfqpoint{3.140579in}{1.302693in}}%
\pgfpathlineto{\pgfqpoint{3.149648in}{1.328164in}}%
\pgfpathlineto{\pgfqpoint{3.149422in}{1.332537in}}%
\pgfpathlineto{\pgfqpoint{3.153524in}{1.337961in}}%
\pgfpathlineto{\pgfqpoint{3.159782in}{1.331466in}}%
\pgfpathlineto{\pgfqpoint{3.159486in}{1.315467in}}%
\pgfpathlineto{\pgfqpoint{3.154386in}{1.307543in}}%
\pgfpathlineto{\pgfqpoint{3.155665in}{1.301994in}}%
\pgfpathlineto{\pgfqpoint{3.184110in}{1.299032in}}%
\pgfpathlineto{\pgfqpoint{3.178373in}{1.294019in}}%
\pgfpathlineto{\pgfqpoint{3.181322in}{1.282092in}}%
\pgfpathlineto{\pgfqpoint{3.189244in}{1.282467in}}%
\pgfpathlineto{\pgfqpoint{3.197116in}{1.277528in}}%
\pgfpathclose%
\pgfusepath{fill}%
\end{pgfscope}%
\begin{pgfscope}%
\pgfpathrectangle{\pgfqpoint{0.100000in}{0.100000in}}{\pgfqpoint{3.608454in}{2.310000in}}%
\pgfusepath{clip}%
\pgfsetbuttcap%
\pgfsetmiterjoin%
\definecolor{currentfill}{rgb}{0.000000,0.713725,0.643137}%
\pgfsetfillcolor{currentfill}%
\pgfsetlinewidth{0.000000pt}%
\definecolor{currentstroke}{rgb}{0.000000,0.000000,0.000000}%
\pgfsetstrokecolor{currentstroke}%
\pgfsetstrokeopacity{0.000000}%
\pgfsetdash{}{0pt}%
\pgfpathmoveto{\pgfqpoint{1.034613in}{1.812423in}}%
\pgfpathlineto{\pgfqpoint{1.027024in}{1.825238in}}%
\pgfpathlineto{\pgfqpoint{1.013709in}{1.836495in}}%
\pgfpathlineto{\pgfqpoint{1.012788in}{1.842181in}}%
\pgfpathlineto{\pgfqpoint{0.999848in}{1.846301in}}%
\pgfpathlineto{\pgfqpoint{0.991130in}{1.850884in}}%
\pgfpathlineto{\pgfqpoint{0.983732in}{1.852523in}}%
\pgfpathlineto{\pgfqpoint{0.981526in}{1.858299in}}%
\pgfpathlineto{\pgfqpoint{0.975129in}{1.855275in}}%
\pgfpathlineto{\pgfqpoint{0.972986in}{1.865357in}}%
\pgfpathlineto{\pgfqpoint{0.975886in}{1.873015in}}%
\pgfpathlineto{\pgfqpoint{0.968833in}{1.878204in}}%
\pgfpathlineto{\pgfqpoint{0.969814in}{1.885604in}}%
\pgfpathlineto{\pgfqpoint{0.964170in}{1.890384in}}%
\pgfpathlineto{\pgfqpoint{0.969006in}{1.895451in}}%
\pgfpathlineto{\pgfqpoint{0.970494in}{1.901193in}}%
\pgfpathlineto{\pgfqpoint{0.967552in}{1.905305in}}%
\pgfpathlineto{\pgfqpoint{0.969772in}{1.911307in}}%
\pgfpathlineto{\pgfqpoint{0.976747in}{1.912867in}}%
\pgfpathlineto{\pgfqpoint{0.981235in}{1.920433in}}%
\pgfpathlineto{\pgfqpoint{0.989615in}{1.915702in}}%
\pgfpathlineto{\pgfqpoint{1.004445in}{1.925280in}}%
\pgfpathlineto{\pgfqpoint{1.009235in}{1.946422in}}%
\pgfpathlineto{\pgfqpoint{1.012082in}{1.949489in}}%
\pgfpathlineto{\pgfqpoint{1.018689in}{1.956057in}}%
\pgfpathlineto{\pgfqpoint{1.011182in}{1.972459in}}%
\pgfpathlineto{\pgfqpoint{1.018436in}{1.970411in}}%
\pgfpathlineto{\pgfqpoint{1.025284in}{1.975956in}}%
\pgfpathlineto{\pgfqpoint{1.028023in}{1.976205in}}%
\pgfpathlineto{\pgfqpoint{1.034877in}{1.968576in}}%
\pgfpathlineto{\pgfqpoint{1.048671in}{1.970878in}}%
\pgfpathlineto{\pgfqpoint{1.053520in}{1.975559in}}%
\pgfpathlineto{\pgfqpoint{1.061696in}{1.979359in}}%
\pgfpathlineto{\pgfqpoint{1.067306in}{1.970565in}}%
\pgfpathlineto{\pgfqpoint{1.064456in}{1.964544in}}%
\pgfpathlineto{\pgfqpoint{1.067866in}{1.953218in}}%
\pgfpathlineto{\pgfqpoint{1.066629in}{1.948325in}}%
\pgfpathlineto{\pgfqpoint{1.073148in}{1.929599in}}%
\pgfpathlineto{\pgfqpoint{1.077934in}{1.924256in}}%
\pgfpathlineto{\pgfqpoint{1.074674in}{1.909170in}}%
\pgfpathlineto{\pgfqpoint{1.079391in}{1.904532in}}%
\pgfpathlineto{\pgfqpoint{1.088102in}{1.901286in}}%
\pgfpathlineto{\pgfqpoint{1.091986in}{1.894613in}}%
\pgfpathlineto{\pgfqpoint{1.094071in}{1.877735in}}%
\pgfpathlineto{\pgfqpoint{1.093092in}{1.871853in}}%
\pgfpathlineto{\pgfqpoint{1.100477in}{1.862882in}}%
\pgfpathlineto{\pgfqpoint{1.102494in}{1.863989in}}%
\pgfpathlineto{\pgfqpoint{1.100225in}{1.852591in}}%
\pgfpathlineto{\pgfqpoint{1.066665in}{1.859594in}}%
\pgfpathlineto{\pgfqpoint{1.063909in}{1.846064in}}%
\pgfpathlineto{\pgfqpoint{1.061051in}{1.840480in}}%
\pgfpathlineto{\pgfqpoint{1.057180in}{1.823892in}}%
\pgfpathlineto{\pgfqpoint{1.049281in}{1.819620in}}%
\pgfpathlineto{\pgfqpoint{1.037674in}{1.816193in}}%
\pgfpathlineto{\pgfqpoint{1.034613in}{1.812423in}}%
\pgfpathclose%
\pgfusepath{fill}%
\end{pgfscope}%
\begin{pgfscope}%
\pgfpathrectangle{\pgfqpoint{0.100000in}{0.100000in}}{\pgfqpoint{3.608454in}{2.310000in}}%
\pgfusepath{clip}%
\pgfsetbuttcap%
\pgfsetmiterjoin%
\definecolor{currentfill}{rgb}{0.000000,0.666667,0.666667}%
\pgfsetfillcolor{currentfill}%
\pgfsetlinewidth{0.000000pt}%
\definecolor{currentstroke}{rgb}{0.000000,0.000000,0.000000}%
\pgfsetstrokecolor{currentstroke}%
\pgfsetstrokeopacity{0.000000}%
\pgfsetdash{}{0pt}%
\pgfpathmoveto{\pgfqpoint{1.183026in}{1.348649in}}%
\pgfpathlineto{\pgfqpoint{1.177278in}{1.346155in}}%
\pgfpathlineto{\pgfqpoint{1.170873in}{1.338075in}}%
\pgfpathlineto{\pgfqpoint{1.168304in}{1.339366in}}%
\pgfpathlineto{\pgfqpoint{1.159137in}{1.332405in}}%
\pgfpathlineto{\pgfqpoint{1.150177in}{1.332380in}}%
\pgfpathlineto{\pgfqpoint{1.141396in}{1.315621in}}%
\pgfpathlineto{\pgfqpoint{1.134341in}{1.312779in}}%
\pgfpathlineto{\pgfqpoint{1.130720in}{1.308550in}}%
\pgfpathlineto{\pgfqpoint{1.104844in}{1.313028in}}%
\pgfpathlineto{\pgfqpoint{1.024916in}{1.327610in}}%
\pgfpathlineto{\pgfqpoint{1.005205in}{1.331999in}}%
\pgfpathlineto{\pgfqpoint{1.008850in}{1.352444in}}%
\pgfpathlineto{\pgfqpoint{1.015614in}{1.351012in}}%
\pgfpathlineto{\pgfqpoint{1.016803in}{1.357032in}}%
\pgfpathlineto{\pgfqpoint{1.023679in}{1.356359in}}%
\pgfpathlineto{\pgfqpoint{1.026917in}{1.376417in}}%
\pgfpathlineto{\pgfqpoint{1.032898in}{1.379983in}}%
\pgfpathlineto{\pgfqpoint{1.038444in}{1.389127in}}%
\pgfpathlineto{\pgfqpoint{1.034771in}{1.396805in}}%
\pgfpathlineto{\pgfqpoint{1.030622in}{1.400197in}}%
\pgfpathlineto{\pgfqpoint{1.031123in}{1.409823in}}%
\pgfpathlineto{\pgfqpoint{1.036855in}{1.417471in}}%
\pgfpathlineto{\pgfqpoint{1.045136in}{1.415471in}}%
\pgfpathlineto{\pgfqpoint{1.051426in}{1.418402in}}%
\pgfpathlineto{\pgfqpoint{1.053099in}{1.429145in}}%
\pgfpathlineto{\pgfqpoint{1.060054in}{1.435184in}}%
\pgfpathlineto{\pgfqpoint{1.064107in}{1.434490in}}%
\pgfpathlineto{\pgfqpoint{1.068484in}{1.440713in}}%
\pgfpathlineto{\pgfqpoint{1.077995in}{1.437925in}}%
\pgfpathlineto{\pgfqpoint{1.111444in}{1.431696in}}%
\pgfpathlineto{\pgfqpoint{1.103689in}{1.391163in}}%
\pgfpathlineto{\pgfqpoint{1.076288in}{1.396187in}}%
\pgfpathlineto{\pgfqpoint{1.069417in}{1.390629in}}%
\pgfpathlineto{\pgfqpoint{1.065800in}{1.369302in}}%
\pgfpathlineto{\pgfqpoint{1.136475in}{1.356525in}}%
\pgfpathlineto{\pgfqpoint{1.183026in}{1.348649in}}%
\pgfpathclose%
\pgfusepath{fill}%
\end{pgfscope}%
\begin{pgfscope}%
\pgfpathrectangle{\pgfqpoint{0.100000in}{0.100000in}}{\pgfqpoint{3.608454in}{2.310000in}}%
\pgfusepath{clip}%
\pgfsetbuttcap%
\pgfsetmiterjoin%
\definecolor{currentfill}{rgb}{0.000000,0.631373,0.684314}%
\pgfsetfillcolor{currentfill}%
\pgfsetlinewidth{0.000000pt}%
\definecolor{currentstroke}{rgb}{0.000000,0.000000,0.000000}%
\pgfsetstrokecolor{currentstroke}%
\pgfsetstrokeopacity{0.000000}%
\pgfsetdash{}{0pt}%
\pgfpathmoveto{\pgfqpoint{2.102150in}{1.181491in}}%
\pgfpathlineto{\pgfqpoint{2.081042in}{1.181525in}}%
\pgfpathlineto{\pgfqpoint{2.079502in}{1.177066in}}%
\pgfpathlineto{\pgfqpoint{2.079386in}{1.149757in}}%
\pgfpathlineto{\pgfqpoint{2.055192in}{1.149919in}}%
\pgfpathlineto{\pgfqpoint{2.055083in}{1.136146in}}%
\pgfpathlineto{\pgfqpoint{2.043055in}{1.136287in}}%
\pgfpathlineto{\pgfqpoint{2.043578in}{1.181785in}}%
\pgfpathlineto{\pgfqpoint{2.045886in}{1.181780in}}%
\pgfpathlineto{\pgfqpoint{2.046716in}{1.239913in}}%
\pgfpathlineto{\pgfqpoint{2.101288in}{1.239463in}}%
\pgfpathlineto{\pgfqpoint{2.102082in}{1.211883in}}%
\pgfpathlineto{\pgfqpoint{2.102150in}{1.181491in}}%
\pgfpathclose%
\pgfusepath{fill}%
\end{pgfscope}%
\begin{pgfscope}%
\pgfpathrectangle{\pgfqpoint{0.100000in}{0.100000in}}{\pgfqpoint{3.608454in}{2.310000in}}%
\pgfusepath{clip}%
\pgfsetbuttcap%
\pgfsetmiterjoin%
\definecolor{currentfill}{rgb}{0.000000,0.654902,0.672549}%
\pgfsetfillcolor{currentfill}%
\pgfsetlinewidth{0.000000pt}%
\definecolor{currentstroke}{rgb}{0.000000,0.000000,0.000000}%
\pgfsetstrokecolor{currentstroke}%
\pgfsetstrokeopacity{0.000000}%
\pgfsetdash{}{0pt}%
\pgfpathmoveto{\pgfqpoint{1.411450in}{2.059486in}}%
\pgfpathlineto{\pgfqpoint{1.406013in}{2.044122in}}%
\pgfpathlineto{\pgfqpoint{1.405800in}{2.033757in}}%
\pgfpathlineto{\pgfqpoint{1.407470in}{2.024546in}}%
\pgfpathlineto{\pgfqpoint{1.403304in}{2.014885in}}%
\pgfpathlineto{\pgfqpoint{1.405583in}{2.011993in}}%
\pgfpathlineto{\pgfqpoint{1.407997in}{2.003940in}}%
\pgfpathlineto{\pgfqpoint{1.364475in}{2.009800in}}%
\pgfpathlineto{\pgfqpoint{1.323552in}{2.016288in}}%
\pgfpathlineto{\pgfqpoint{1.322819in}{2.011773in}}%
\pgfpathlineto{\pgfqpoint{1.303565in}{2.014897in}}%
\pgfpathlineto{\pgfqpoint{1.303017in}{2.028995in}}%
\pgfpathlineto{\pgfqpoint{1.305651in}{2.030921in}}%
\pgfpathlineto{\pgfqpoint{1.309306in}{2.053731in}}%
\pgfpathlineto{\pgfqpoint{1.303684in}{2.058169in}}%
\pgfpathlineto{\pgfqpoint{1.296843in}{2.059295in}}%
\pgfpathlineto{\pgfqpoint{1.290585in}{2.063837in}}%
\pgfpathlineto{\pgfqpoint{1.293100in}{2.076361in}}%
\pgfpathlineto{\pgfqpoint{1.309109in}{2.082641in}}%
\pgfpathlineto{\pgfqpoint{1.312471in}{2.096193in}}%
\pgfpathlineto{\pgfqpoint{1.328054in}{2.095490in}}%
\pgfpathlineto{\pgfqpoint{1.332093in}{2.093598in}}%
\pgfpathlineto{\pgfqpoint{1.342674in}{2.097531in}}%
\pgfpathlineto{\pgfqpoint{1.350200in}{2.095416in}}%
\pgfpathlineto{\pgfqpoint{1.358594in}{2.095823in}}%
\pgfpathlineto{\pgfqpoint{1.362402in}{2.089710in}}%
\pgfpathlineto{\pgfqpoint{1.367934in}{2.081540in}}%
\pgfpathlineto{\pgfqpoint{1.377859in}{2.077250in}}%
\pgfpathlineto{\pgfqpoint{1.385004in}{2.076421in}}%
\pgfpathlineto{\pgfqpoint{1.389242in}{2.072875in}}%
\pgfpathlineto{\pgfqpoint{1.404841in}{2.072375in}}%
\pgfpathlineto{\pgfqpoint{1.411723in}{2.068893in}}%
\pgfpathlineto{\pgfqpoint{1.411450in}{2.059486in}}%
\pgfpathclose%
\pgfusepath{fill}%
\end{pgfscope}%
\begin{pgfscope}%
\pgfpathrectangle{\pgfqpoint{0.100000in}{0.100000in}}{\pgfqpoint{3.608454in}{2.310000in}}%
\pgfusepath{clip}%
\pgfsetbuttcap%
\pgfsetmiterjoin%
\definecolor{currentfill}{rgb}{0.000000,0.521569,0.739216}%
\pgfsetfillcolor{currentfill}%
\pgfsetlinewidth{0.000000pt}%
\definecolor{currentstroke}{rgb}{0.000000,0.000000,0.000000}%
\pgfsetstrokecolor{currentstroke}%
\pgfsetstrokeopacity{0.000000}%
\pgfsetdash{}{0pt}%
\pgfpathmoveto{\pgfqpoint{2.002513in}{1.074759in}}%
\pgfpathlineto{\pgfqpoint{2.002912in}{1.098705in}}%
\pgfpathlineto{\pgfqpoint{2.003198in}{1.115990in}}%
\pgfpathlineto{\pgfqpoint{1.990571in}{1.116149in}}%
\pgfpathlineto{\pgfqpoint{1.990717in}{1.123080in}}%
\pgfpathlineto{\pgfqpoint{1.983944in}{1.123221in}}%
\pgfpathlineto{\pgfqpoint{1.984100in}{1.130082in}}%
\pgfpathlineto{\pgfqpoint{1.977272in}{1.130240in}}%
\pgfpathlineto{\pgfqpoint{1.977572in}{1.143978in}}%
\pgfpathlineto{\pgfqpoint{1.985971in}{1.147335in}}%
\pgfpathlineto{\pgfqpoint{1.984832in}{1.151230in}}%
\pgfpathlineto{\pgfqpoint{1.950456in}{1.151489in}}%
\pgfpathlineto{\pgfqpoint{1.951284in}{1.183534in}}%
\pgfpathlineto{\pgfqpoint{1.971148in}{1.183063in}}%
\pgfpathlineto{\pgfqpoint{2.043578in}{1.181785in}}%
\pgfpathlineto{\pgfqpoint{2.043055in}{1.136287in}}%
\pgfpathlineto{\pgfqpoint{2.055083in}{1.136146in}}%
\pgfpathlineto{\pgfqpoint{2.055192in}{1.149919in}}%
\pgfpathlineto{\pgfqpoint{2.079386in}{1.149757in}}%
\pgfpathlineto{\pgfqpoint{2.079502in}{1.177066in}}%
\pgfpathlineto{\pgfqpoint{2.081042in}{1.181525in}}%
\pgfpathlineto{\pgfqpoint{2.102150in}{1.181491in}}%
\pgfpathlineto{\pgfqpoint{2.106837in}{1.177041in}}%
\pgfpathlineto{\pgfqpoint{2.106820in}{1.155478in}}%
\pgfpathlineto{\pgfqpoint{2.106071in}{1.115212in}}%
\pgfpathlineto{\pgfqpoint{2.099224in}{1.115204in}}%
\pgfpathlineto{\pgfqpoint{2.099232in}{1.108302in}}%
\pgfpathlineto{\pgfqpoint{2.093490in}{1.108334in}}%
\pgfpathlineto{\pgfqpoint{2.088937in}{1.099591in}}%
\pgfpathlineto{\pgfqpoint{2.089737in}{1.087648in}}%
\pgfpathlineto{\pgfqpoint{2.076132in}{1.086986in}}%
\pgfpathlineto{\pgfqpoint{2.068981in}{1.083880in}}%
\pgfpathlineto{\pgfqpoint{2.065063in}{1.091216in}}%
\pgfpathlineto{\pgfqpoint{2.057637in}{1.091246in}}%
\pgfpathlineto{\pgfqpoint{2.030288in}{1.091590in}}%
\pgfpathlineto{\pgfqpoint{2.030056in}{1.074337in}}%
\pgfpathlineto{\pgfqpoint{2.002513in}{1.074759in}}%
\pgfpathclose%
\pgfusepath{fill}%
\end{pgfscope}%
\begin{pgfscope}%
\pgfpathrectangle{\pgfqpoint{0.100000in}{0.100000in}}{\pgfqpoint{3.608454in}{2.310000in}}%
\pgfusepath{clip}%
\pgfsetbuttcap%
\pgfsetmiterjoin%
\definecolor{currentfill}{rgb}{0.000000,0.796078,0.601961}%
\pgfsetfillcolor{currentfill}%
\pgfsetlinewidth{0.000000pt}%
\definecolor{currentstroke}{rgb}{0.000000,0.000000,0.000000}%
\pgfsetstrokecolor{currentstroke}%
\pgfsetstrokeopacity{0.000000}%
\pgfsetdash{}{0pt}%
\pgfpathmoveto{\pgfqpoint{1.841646in}{1.759396in}}%
\pgfpathlineto{\pgfqpoint{1.843095in}{1.749824in}}%
\pgfpathlineto{\pgfqpoint{1.835653in}{1.753170in}}%
\pgfpathlineto{\pgfqpoint{1.830770in}{1.752552in}}%
\pgfpathlineto{\pgfqpoint{1.825951in}{1.758455in}}%
\pgfpathlineto{\pgfqpoint{1.801576in}{1.759791in}}%
\pgfpathlineto{\pgfqpoint{1.801456in}{1.757724in}}%
\pgfpathlineto{\pgfqpoint{1.762841in}{1.760148in}}%
\pgfpathlineto{\pgfqpoint{1.756032in}{1.760574in}}%
\pgfpathlineto{\pgfqpoint{1.757629in}{1.774393in}}%
\pgfpathlineto{\pgfqpoint{1.759531in}{1.801865in}}%
\pgfpathlineto{\pgfqpoint{1.760814in}{1.806325in}}%
\pgfpathlineto{\pgfqpoint{1.767130in}{1.804067in}}%
\pgfpathlineto{\pgfqpoint{1.778292in}{1.808049in}}%
\pgfpathlineto{\pgfqpoint{1.781507in}{1.805854in}}%
\pgfpathlineto{\pgfqpoint{1.787641in}{1.810142in}}%
\pgfpathlineto{\pgfqpoint{1.795074in}{1.805632in}}%
\pgfpathlineto{\pgfqpoint{1.803220in}{1.813204in}}%
\pgfpathlineto{\pgfqpoint{1.802764in}{1.815650in}}%
\pgfpathlineto{\pgfqpoint{1.843805in}{1.813366in}}%
\pgfpathlineto{\pgfqpoint{1.842265in}{1.785638in}}%
\pgfpathlineto{\pgfqpoint{1.841646in}{1.759396in}}%
\pgfpathclose%
\pgfusepath{fill}%
\end{pgfscope}%
\begin{pgfscope}%
\pgfpathrectangle{\pgfqpoint{0.100000in}{0.100000in}}{\pgfqpoint{3.608454in}{2.310000in}}%
\pgfusepath{clip}%
\pgfsetbuttcap%
\pgfsetmiterjoin%
\definecolor{currentfill}{rgb}{0.000000,0.850980,0.574510}%
\pgfsetfillcolor{currentfill}%
\pgfsetlinewidth{0.000000pt}%
\definecolor{currentstroke}{rgb}{0.000000,0.000000,0.000000}%
\pgfsetstrokecolor{currentstroke}%
\pgfsetstrokeopacity{0.000000}%
\pgfsetdash{}{0pt}%
\pgfpathmoveto{\pgfqpoint{2.404037in}{0.740243in}}%
\pgfpathlineto{\pgfqpoint{2.403042in}{0.761124in}}%
\pgfpathlineto{\pgfqpoint{2.395986in}{0.760766in}}%
\pgfpathlineto{\pgfqpoint{2.395693in}{0.767722in}}%
\pgfpathlineto{\pgfqpoint{2.426622in}{0.769362in}}%
\pgfpathlineto{\pgfqpoint{2.436793in}{0.774096in}}%
\pgfpathlineto{\pgfqpoint{2.447021in}{0.775387in}}%
\pgfpathlineto{\pgfqpoint{2.447975in}{0.756350in}}%
\pgfpathlineto{\pgfqpoint{2.449720in}{0.745908in}}%
\pgfpathlineto{\pgfqpoint{2.444160in}{0.745936in}}%
\pgfpathlineto{\pgfqpoint{2.439716in}{0.742196in}}%
\pgfpathlineto{\pgfqpoint{2.404037in}{0.740243in}}%
\pgfpathclose%
\pgfusepath{fill}%
\end{pgfscope}%
\begin{pgfscope}%
\pgfpathrectangle{\pgfqpoint{0.100000in}{0.100000in}}{\pgfqpoint{3.608454in}{2.310000in}}%
\pgfusepath{clip}%
\pgfsetbuttcap%
\pgfsetmiterjoin%
\definecolor{currentfill}{rgb}{0.000000,0.427451,0.786275}%
\pgfsetfillcolor{currentfill}%
\pgfsetlinewidth{0.000000pt}%
\definecolor{currentstroke}{rgb}{0.000000,0.000000,0.000000}%
\pgfsetstrokecolor{currentstroke}%
\pgfsetstrokeopacity{0.000000}%
\pgfsetdash{}{0pt}%
\pgfpathmoveto{\pgfqpoint{1.644176in}{0.728867in}}%
\pgfpathlineto{\pgfqpoint{1.601306in}{0.732857in}}%
\pgfpathlineto{\pgfqpoint{1.599669in}{0.742990in}}%
\pgfpathlineto{\pgfqpoint{1.584520in}{0.753390in}}%
\pgfpathlineto{\pgfqpoint{1.578119in}{0.751288in}}%
\pgfpathlineto{\pgfqpoint{1.580562in}{0.779720in}}%
\pgfpathlineto{\pgfqpoint{1.578420in}{0.779909in}}%
\pgfpathlineto{\pgfqpoint{1.581375in}{0.814353in}}%
\pgfpathlineto{\pgfqpoint{1.563559in}{0.816012in}}%
\pgfpathlineto{\pgfqpoint{1.566607in}{0.850506in}}%
\pgfpathlineto{\pgfqpoint{1.624223in}{0.845743in}}%
\pgfpathlineto{\pgfqpoint{1.658603in}{0.843265in}}%
\pgfpathlineto{\pgfqpoint{1.655619in}{0.808558in}}%
\pgfpathlineto{\pgfqpoint{1.650167in}{0.808902in}}%
\pgfpathlineto{\pgfqpoint{1.644176in}{0.728867in}}%
\pgfpathclose%
\pgfusepath{fill}%
\end{pgfscope}%
\begin{pgfscope}%
\pgfpathrectangle{\pgfqpoint{0.100000in}{0.100000in}}{\pgfqpoint{3.608454in}{2.310000in}}%
\pgfusepath{clip}%
\pgfsetbuttcap%
\pgfsetmiterjoin%
\definecolor{currentfill}{rgb}{0.000000,0.521569,0.739216}%
\pgfsetfillcolor{currentfill}%
\pgfsetlinewidth{0.000000pt}%
\definecolor{currentstroke}{rgb}{0.000000,0.000000,0.000000}%
\pgfsetstrokecolor{currentstroke}%
\pgfsetstrokeopacity{0.000000}%
\pgfsetdash{}{0pt}%
\pgfpathmoveto{\pgfqpoint{2.433309in}{1.321900in}}%
\pgfpathlineto{\pgfqpoint{2.404058in}{1.320682in}}%
\pgfpathlineto{\pgfqpoint{2.402286in}{1.326271in}}%
\pgfpathlineto{\pgfqpoint{2.407445in}{1.335820in}}%
\pgfpathlineto{\pgfqpoint{2.398503in}{1.340979in}}%
\pgfpathlineto{\pgfqpoint{2.386232in}{1.344093in}}%
\pgfpathlineto{\pgfqpoint{2.380869in}{1.336490in}}%
\pgfpathlineto{\pgfqpoint{2.373739in}{1.339722in}}%
\pgfpathlineto{\pgfqpoint{2.369779in}{1.350114in}}%
\pgfpathlineto{\pgfqpoint{2.371543in}{1.352918in}}%
\pgfpathlineto{\pgfqpoint{2.368642in}{1.363523in}}%
\pgfpathlineto{\pgfqpoint{2.363949in}{1.370060in}}%
\pgfpathlineto{\pgfqpoint{2.354879in}{1.376769in}}%
\pgfpathlineto{\pgfqpoint{2.374602in}{1.377391in}}%
\pgfpathlineto{\pgfqpoint{2.376719in}{1.362091in}}%
\pgfpathlineto{\pgfqpoint{2.383166in}{1.360351in}}%
\pgfpathlineto{\pgfqpoint{2.393816in}{1.360813in}}%
\pgfpathlineto{\pgfqpoint{2.403570in}{1.368279in}}%
\pgfpathlineto{\pgfqpoint{2.402134in}{1.388642in}}%
\pgfpathlineto{\pgfqpoint{2.415980in}{1.389550in}}%
\pgfpathlineto{\pgfqpoint{2.439852in}{1.391129in}}%
\pgfpathlineto{\pgfqpoint{2.440862in}{1.377267in}}%
\pgfpathlineto{\pgfqpoint{2.464725in}{1.378777in}}%
\pgfpathlineto{\pgfqpoint{2.465371in}{1.368430in}}%
\pgfpathlineto{\pgfqpoint{2.458623in}{1.367966in}}%
\pgfpathlineto{\pgfqpoint{2.460771in}{1.330448in}}%
\pgfpathlineto{\pgfqpoint{2.439649in}{1.329250in}}%
\pgfpathlineto{\pgfqpoint{2.440188in}{1.322366in}}%
\pgfpathlineto{\pgfqpoint{2.433309in}{1.321900in}}%
\pgfpathclose%
\pgfusepath{fill}%
\end{pgfscope}%
\begin{pgfscope}%
\pgfpathrectangle{\pgfqpoint{0.100000in}{0.100000in}}{\pgfqpoint{3.608454in}{2.310000in}}%
\pgfusepath{clip}%
\pgfsetbuttcap%
\pgfsetmiterjoin%
\definecolor{currentfill}{rgb}{0.000000,0.882353,0.558824}%
\pgfsetfillcolor{currentfill}%
\pgfsetlinewidth{0.000000pt}%
\definecolor{currentstroke}{rgb}{0.000000,0.000000,0.000000}%
\pgfsetstrokecolor{currentstroke}%
\pgfsetstrokeopacity{0.000000}%
\pgfsetdash{}{0pt}%
\pgfpathmoveto{\pgfqpoint{2.342635in}{0.869578in}}%
\pgfpathlineto{\pgfqpoint{2.302086in}{0.868312in}}%
\pgfpathlineto{\pgfqpoint{2.299404in}{0.870120in}}%
\pgfpathlineto{\pgfqpoint{2.297141in}{0.880461in}}%
\pgfpathlineto{\pgfqpoint{2.297645in}{0.883786in}}%
\pgfpathlineto{\pgfqpoint{2.307139in}{0.891576in}}%
\pgfpathlineto{\pgfqpoint{2.306319in}{0.899456in}}%
\pgfpathlineto{\pgfqpoint{2.341651in}{0.899996in}}%
\pgfpathlineto{\pgfqpoint{2.342635in}{0.869578in}}%
\pgfpathclose%
\pgfusepath{fill}%
\end{pgfscope}%
\begin{pgfscope}%
\pgfpathrectangle{\pgfqpoint{0.100000in}{0.100000in}}{\pgfqpoint{3.608454in}{2.310000in}}%
\pgfusepath{clip}%
\pgfsetbuttcap%
\pgfsetmiterjoin%
\definecolor{currentfill}{rgb}{0.000000,0.403922,0.798039}%
\pgfsetfillcolor{currentfill}%
\pgfsetlinewidth{0.000000pt}%
\definecolor{currentstroke}{rgb}{0.000000,0.000000,0.000000}%
\pgfsetstrokecolor{currentstroke}%
\pgfsetstrokeopacity{0.000000}%
\pgfsetdash{}{0pt}%
\pgfpathmoveto{\pgfqpoint{1.726904in}{1.542427in}}%
\pgfpathlineto{\pgfqpoint{1.692731in}{1.544894in}}%
\pgfpathlineto{\pgfqpoint{1.694791in}{1.572316in}}%
\pgfpathlineto{\pgfqpoint{1.689939in}{1.572676in}}%
\pgfpathlineto{\pgfqpoint{1.691656in}{1.593744in}}%
\pgfpathlineto{\pgfqpoint{1.693738in}{1.600532in}}%
\pgfpathlineto{\pgfqpoint{1.695984in}{1.627874in}}%
\pgfpathlineto{\pgfqpoint{1.694322in}{1.628015in}}%
\pgfpathlineto{\pgfqpoint{1.696414in}{1.655020in}}%
\pgfpathlineto{\pgfqpoint{1.696825in}{1.671982in}}%
\pgfpathlineto{\pgfqpoint{1.746207in}{1.668241in}}%
\pgfpathlineto{\pgfqpoint{1.805743in}{1.664546in}}%
\pgfpathlineto{\pgfqpoint{1.805638in}{1.647292in}}%
\pgfpathlineto{\pgfqpoint{1.804075in}{1.619812in}}%
\pgfpathlineto{\pgfqpoint{1.803419in}{1.592288in}}%
\pgfpathlineto{\pgfqpoint{1.763649in}{1.594885in}}%
\pgfpathlineto{\pgfqpoint{1.762096in}{1.567407in}}%
\pgfpathlineto{\pgfqpoint{1.728889in}{1.569871in}}%
\pgfpathlineto{\pgfqpoint{1.726904in}{1.542427in}}%
\pgfpathclose%
\pgfusepath{fill}%
\end{pgfscope}%
\begin{pgfscope}%
\pgfpathrectangle{\pgfqpoint{0.100000in}{0.100000in}}{\pgfqpoint{3.608454in}{2.310000in}}%
\pgfusepath{clip}%
\pgfsetbuttcap%
\pgfsetmiterjoin%
\definecolor{currentfill}{rgb}{0.000000,0.639216,0.680392}%
\pgfsetfillcolor{currentfill}%
\pgfsetlinewidth{0.000000pt}%
\definecolor{currentstroke}{rgb}{0.000000,0.000000,0.000000}%
\pgfsetstrokecolor{currentstroke}%
\pgfsetstrokeopacity{0.000000}%
\pgfsetdash{}{0pt}%
\pgfpathmoveto{\pgfqpoint{2.731544in}{1.590059in}}%
\pgfpathlineto{\pgfqpoint{2.707766in}{1.586556in}}%
\pgfpathlineto{\pgfqpoint{2.706079in}{1.591455in}}%
\pgfpathlineto{\pgfqpoint{2.703247in}{1.616030in}}%
\pgfpathlineto{\pgfqpoint{2.710082in}{1.616628in}}%
\pgfpathlineto{\pgfqpoint{2.706426in}{1.644227in}}%
\pgfpathlineto{\pgfqpoint{2.739935in}{1.648352in}}%
\pgfpathlineto{\pgfqpoint{2.743822in}{1.620714in}}%
\pgfpathlineto{\pgfqpoint{2.764552in}{1.624067in}}%
\pgfpathlineto{\pgfqpoint{2.768691in}{1.595961in}}%
\pgfpathlineto{\pgfqpoint{2.731544in}{1.590059in}}%
\pgfpathclose%
\pgfusepath{fill}%
\end{pgfscope}%
\begin{pgfscope}%
\pgfpathrectangle{\pgfqpoint{0.100000in}{0.100000in}}{\pgfqpoint{3.608454in}{2.310000in}}%
\pgfusepath{clip}%
\pgfsetbuttcap%
\pgfsetmiterjoin%
\definecolor{currentfill}{rgb}{0.000000,0.470588,0.764706}%
\pgfsetfillcolor{currentfill}%
\pgfsetlinewidth{0.000000pt}%
\definecolor{currentstroke}{rgb}{0.000000,0.000000,0.000000}%
\pgfsetstrokecolor{currentstroke}%
\pgfsetstrokeopacity{0.000000}%
\pgfsetdash{}{0pt}%
\pgfpathmoveto{\pgfqpoint{3.269282in}{1.579037in}}%
\pgfpathlineto{\pgfqpoint{3.244201in}{1.592250in}}%
\pgfpathlineto{\pgfqpoint{3.251074in}{1.598426in}}%
\pgfpathlineto{\pgfqpoint{3.234122in}{1.609162in}}%
\pgfpathlineto{\pgfqpoint{3.248038in}{1.619832in}}%
\pgfpathlineto{\pgfqpoint{3.244631in}{1.624617in}}%
\pgfpathlineto{\pgfqpoint{3.248335in}{1.629894in}}%
\pgfpathlineto{\pgfqpoint{3.252558in}{1.629161in}}%
\pgfpathlineto{\pgfqpoint{3.262436in}{1.621007in}}%
\pgfpathlineto{\pgfqpoint{3.258629in}{1.617098in}}%
\pgfpathlineto{\pgfqpoint{3.266342in}{1.608149in}}%
\pgfpathlineto{\pgfqpoint{3.276398in}{1.614345in}}%
\pgfpathlineto{\pgfqpoint{3.284382in}{1.624557in}}%
\pgfpathlineto{\pgfqpoint{3.289903in}{1.618119in}}%
\pgfpathlineto{\pgfqpoint{3.288120in}{1.611032in}}%
\pgfpathlineto{\pgfqpoint{3.284662in}{1.609183in}}%
\pgfpathlineto{\pgfqpoint{3.286080in}{1.595797in}}%
\pgfpathlineto{\pgfqpoint{3.275573in}{1.583652in}}%
\pgfpathlineto{\pgfqpoint{3.269282in}{1.579037in}}%
\pgfpathclose%
\pgfusepath{fill}%
\end{pgfscope}%
\begin{pgfscope}%
\pgfpathrectangle{\pgfqpoint{0.100000in}{0.100000in}}{\pgfqpoint{3.608454in}{2.310000in}}%
\pgfusepath{clip}%
\pgfsetbuttcap%
\pgfsetmiterjoin%
\definecolor{currentfill}{rgb}{0.000000,0.556863,0.721569}%
\pgfsetfillcolor{currentfill}%
\pgfsetlinewidth{0.000000pt}%
\definecolor{currentstroke}{rgb}{0.000000,0.000000,0.000000}%
\pgfsetstrokecolor{currentstroke}%
\pgfsetstrokeopacity{0.000000}%
\pgfsetdash{}{0pt}%
\pgfpathmoveto{\pgfqpoint{1.826743in}{0.706842in}}%
\pgfpathlineto{\pgfqpoint{1.835408in}{0.704889in}}%
\pgfpathlineto{\pgfqpoint{1.833962in}{0.671147in}}%
\pgfpathlineto{\pgfqpoint{1.810600in}{0.672318in}}%
\pgfpathlineto{\pgfqpoint{1.798187in}{0.672904in}}%
\pgfpathlineto{\pgfqpoint{1.799907in}{0.708077in}}%
\pgfpathlineto{\pgfqpoint{1.826743in}{0.706842in}}%
\pgfpathclose%
\pgfusepath{fill}%
\end{pgfscope}%
\begin{pgfscope}%
\pgfpathrectangle{\pgfqpoint{0.100000in}{0.100000in}}{\pgfqpoint{3.608454in}{2.310000in}}%
\pgfusepath{clip}%
\pgfsetbuttcap%
\pgfsetmiterjoin%
\definecolor{currentfill}{rgb}{0.000000,0.627451,0.686275}%
\pgfsetfillcolor{currentfill}%
\pgfsetlinewidth{0.000000pt}%
\definecolor{currentstroke}{rgb}{0.000000,0.000000,0.000000}%
\pgfsetstrokecolor{currentstroke}%
\pgfsetstrokeopacity{0.000000}%
\pgfsetdash{}{0pt}%
\pgfpathmoveto{\pgfqpoint{2.236527in}{1.423164in}}%
\pgfpathlineto{\pgfqpoint{2.205934in}{1.422201in}}%
\pgfpathlineto{\pgfqpoint{2.204405in}{1.465435in}}%
\pgfpathlineto{\pgfqpoint{2.244011in}{1.467014in}}%
\pgfpathlineto{\pgfqpoint{2.246988in}{1.453136in}}%
\pgfpathlineto{\pgfqpoint{2.246319in}{1.447619in}}%
\pgfpathlineto{\pgfqpoint{2.235992in}{1.447300in}}%
\pgfpathlineto{\pgfqpoint{2.236527in}{1.423164in}}%
\pgfpathclose%
\pgfusepath{fill}%
\end{pgfscope}%
\begin{pgfscope}%
\pgfpathrectangle{\pgfqpoint{0.100000in}{0.100000in}}{\pgfqpoint{3.608454in}{2.310000in}}%
\pgfusepath{clip}%
\pgfsetbuttcap%
\pgfsetmiterjoin%
\definecolor{currentfill}{rgb}{0.000000,0.513725,0.743137}%
\pgfsetfillcolor{currentfill}%
\pgfsetlinewidth{0.000000pt}%
\definecolor{currentstroke}{rgb}{0.000000,0.000000,0.000000}%
\pgfsetstrokecolor{currentstroke}%
\pgfsetstrokeopacity{0.000000}%
\pgfsetdash{}{0pt}%
\pgfpathmoveto{\pgfqpoint{3.115324in}{1.315206in}}%
\pgfpathlineto{\pgfqpoint{3.110914in}{1.313734in}}%
\pgfpathlineto{\pgfqpoint{3.103381in}{1.324700in}}%
\pgfpathlineto{\pgfqpoint{3.098792in}{1.325084in}}%
\pgfpathlineto{\pgfqpoint{3.092431in}{1.332135in}}%
\pgfpathlineto{\pgfqpoint{3.084921in}{1.329255in}}%
\pgfpathlineto{\pgfqpoint{3.081757in}{1.317949in}}%
\pgfpathlineto{\pgfqpoint{3.075683in}{1.311970in}}%
\pgfpathlineto{\pgfqpoint{3.074021in}{1.306357in}}%
\pgfpathlineto{\pgfqpoint{3.060878in}{1.314437in}}%
\pgfpathlineto{\pgfqpoint{3.060124in}{1.322440in}}%
\pgfpathlineto{\pgfqpoint{3.050832in}{1.323908in}}%
\pgfpathlineto{\pgfqpoint{3.045914in}{1.315133in}}%
\pgfpathlineto{\pgfqpoint{3.033267in}{1.304579in}}%
\pgfpathlineto{\pgfqpoint{3.022678in}{1.310484in}}%
\pgfpathlineto{\pgfqpoint{3.025954in}{1.320474in}}%
\pgfpathlineto{\pgfqpoint{3.037059in}{1.337144in}}%
\pgfpathlineto{\pgfqpoint{3.038592in}{1.342913in}}%
\pgfpathlineto{\pgfqpoint{3.039437in}{1.352438in}}%
\pgfpathlineto{\pgfqpoint{3.046369in}{1.360627in}}%
\pgfpathlineto{\pgfqpoint{3.044609in}{1.363268in}}%
\pgfpathlineto{\pgfqpoint{3.050299in}{1.374365in}}%
\pgfpathlineto{\pgfqpoint{3.050650in}{1.387261in}}%
\pgfpathlineto{\pgfqpoint{3.058026in}{1.385268in}}%
\pgfpathlineto{\pgfqpoint{3.062954in}{1.378601in}}%
\pgfpathlineto{\pgfqpoint{3.073594in}{1.376808in}}%
\pgfpathlineto{\pgfqpoint{3.078058in}{1.382882in}}%
\pgfpathlineto{\pgfqpoint{3.110807in}{1.366936in}}%
\pgfpathlineto{\pgfqpoint{3.110708in}{1.357752in}}%
\pgfpathlineto{\pgfqpoint{3.107539in}{1.353570in}}%
\pgfpathlineto{\pgfqpoint{3.120128in}{1.335691in}}%
\pgfpathlineto{\pgfqpoint{3.121341in}{1.327517in}}%
\pgfpathlineto{\pgfqpoint{3.116356in}{1.322582in}}%
\pgfpathlineto{\pgfqpoint{3.115324in}{1.315206in}}%
\pgfpathclose%
\pgfusepath{fill}%
\end{pgfscope}%
\begin{pgfscope}%
\pgfpathrectangle{\pgfqpoint{0.100000in}{0.100000in}}{\pgfqpoint{3.608454in}{2.310000in}}%
\pgfusepath{clip}%
\pgfsetbuttcap%
\pgfsetmiterjoin%
\definecolor{currentfill}{rgb}{0.000000,0.556863,0.721569}%
\pgfsetfillcolor{currentfill}%
\pgfsetlinewidth{0.000000pt}%
\definecolor{currentstroke}{rgb}{0.000000,0.000000,0.000000}%
\pgfsetstrokecolor{currentstroke}%
\pgfsetstrokeopacity{0.000000}%
\pgfsetdash{}{0pt}%
\pgfpathmoveto{\pgfqpoint{1.810600in}{0.672318in}}%
\pgfpathlineto{\pgfqpoint{1.809932in}{0.655294in}}%
\pgfpathlineto{\pgfqpoint{1.778772in}{0.657165in}}%
\pgfpathlineto{\pgfqpoint{1.753800in}{0.658495in}}%
\pgfpathlineto{\pgfqpoint{1.755704in}{0.692007in}}%
\pgfpathlineto{\pgfqpoint{1.757470in}{0.722100in}}%
\pgfpathlineto{\pgfqpoint{1.792400in}{0.720174in}}%
\pgfpathlineto{\pgfqpoint{1.791780in}{0.708494in}}%
\pgfpathlineto{\pgfqpoint{1.799907in}{0.708077in}}%
\pgfpathlineto{\pgfqpoint{1.798187in}{0.672904in}}%
\pgfpathlineto{\pgfqpoint{1.810600in}{0.672318in}}%
\pgfpathclose%
\pgfusepath{fill}%
\end{pgfscope}%
\begin{pgfscope}%
\pgfpathrectangle{\pgfqpoint{0.100000in}{0.100000in}}{\pgfqpoint{3.608454in}{2.310000in}}%
\pgfusepath{clip}%
\pgfsetbuttcap%
\pgfsetmiterjoin%
\definecolor{currentfill}{rgb}{0.000000,0.521569,0.739216}%
\pgfsetfillcolor{currentfill}%
\pgfsetlinewidth{0.000000pt}%
\definecolor{currentstroke}{rgb}{0.000000,0.000000,0.000000}%
\pgfsetstrokecolor{currentstroke}%
\pgfsetstrokeopacity{0.000000}%
\pgfsetdash{}{0pt}%
\pgfpathmoveto{\pgfqpoint{3.185535in}{1.030022in}}%
\pgfpathlineto{\pgfqpoint{3.145061in}{1.058883in}}%
\pgfpathlineto{\pgfqpoint{3.150261in}{1.071349in}}%
\pgfpathlineto{\pgfqpoint{3.156843in}{1.077749in}}%
\pgfpathlineto{\pgfqpoint{3.156648in}{1.092471in}}%
\pgfpathlineto{\pgfqpoint{3.148549in}{1.102631in}}%
\pgfpathlineto{\pgfqpoint{3.154787in}{1.104979in}}%
\pgfpathlineto{\pgfqpoint{3.174249in}{1.109020in}}%
\pgfpathlineto{\pgfqpoint{3.182406in}{1.101945in}}%
\pgfpathlineto{\pgfqpoint{3.194034in}{1.089009in}}%
\pgfpathlineto{\pgfqpoint{3.200611in}{1.103096in}}%
\pgfpathlineto{\pgfqpoint{3.228319in}{1.108263in}}%
\pgfpathlineto{\pgfqpoint{3.237815in}{1.092298in}}%
\pgfpathlineto{\pgfqpoint{3.243003in}{1.088398in}}%
\pgfpathlineto{\pgfqpoint{3.231067in}{1.072331in}}%
\pgfpathlineto{\pgfqpoint{3.226781in}{1.061809in}}%
\pgfpathlineto{\pgfqpoint{3.223101in}{1.039243in}}%
\pgfpathlineto{\pgfqpoint{3.209280in}{1.039463in}}%
\pgfpathlineto{\pgfqpoint{3.196751in}{1.036307in}}%
\pgfpathlineto{\pgfqpoint{3.185535in}{1.030022in}}%
\pgfpathclose%
\pgfusepath{fill}%
\end{pgfscope}%
\begin{pgfscope}%
\pgfpathrectangle{\pgfqpoint{0.100000in}{0.100000in}}{\pgfqpoint{3.608454in}{2.310000in}}%
\pgfusepath{clip}%
\pgfsetbuttcap%
\pgfsetmiterjoin%
\definecolor{currentfill}{rgb}{0.000000,0.490196,0.754902}%
\pgfsetfillcolor{currentfill}%
\pgfsetlinewidth{0.000000pt}%
\definecolor{currentstroke}{rgb}{0.000000,0.000000,0.000000}%
\pgfsetstrokecolor{currentstroke}%
\pgfsetstrokeopacity{0.000000}%
\pgfsetdash{}{0pt}%
\pgfpathmoveto{\pgfqpoint{0.628251in}{1.491065in}}%
\pgfpathlineto{\pgfqpoint{0.620520in}{1.503015in}}%
\pgfpathlineto{\pgfqpoint{0.623308in}{1.518183in}}%
\pgfpathlineto{\pgfqpoint{0.620041in}{1.519423in}}%
\pgfpathlineto{\pgfqpoint{0.623821in}{1.536529in}}%
\pgfpathlineto{\pgfqpoint{0.631100in}{1.537480in}}%
\pgfpathlineto{\pgfqpoint{0.632981in}{1.544578in}}%
\pgfpathlineto{\pgfqpoint{0.606969in}{1.551881in}}%
\pgfpathlineto{\pgfqpoint{0.607050in}{1.554247in}}%
\pgfpathlineto{\pgfqpoint{0.592721in}{1.558163in}}%
\pgfpathlineto{\pgfqpoint{0.599878in}{1.583444in}}%
\pgfpathlineto{\pgfqpoint{0.605938in}{1.604509in}}%
\pgfpathlineto{\pgfqpoint{0.622947in}{1.663573in}}%
\pgfpathlineto{\pgfqpoint{0.637413in}{1.715698in}}%
\pgfpathlineto{\pgfqpoint{0.654872in}{1.777406in}}%
\pgfpathlineto{\pgfqpoint{0.691075in}{1.767312in}}%
\pgfpathlineto{\pgfqpoint{0.693113in}{1.766738in}}%
\pgfpathlineto{\pgfqpoint{0.681032in}{1.722500in}}%
\pgfpathlineto{\pgfqpoint{0.671164in}{1.687956in}}%
\pgfpathlineto{\pgfqpoint{0.672323in}{1.687633in}}%
\pgfpathlineto{\pgfqpoint{0.662884in}{1.653976in}}%
\pgfpathlineto{\pgfqpoint{0.661398in}{1.654341in}}%
\pgfpathlineto{\pgfqpoint{0.650489in}{1.614771in}}%
\pgfpathlineto{\pgfqpoint{0.693366in}{1.603165in}}%
\pgfpathlineto{\pgfqpoint{0.755783in}{1.587192in}}%
\pgfpathlineto{\pgfqpoint{0.758589in}{1.580048in}}%
\pgfpathlineto{\pgfqpoint{0.753615in}{1.567143in}}%
\pgfpathlineto{\pgfqpoint{0.753176in}{1.557006in}}%
\pgfpathlineto{\pgfqpoint{0.749642in}{1.550822in}}%
\pgfpathlineto{\pgfqpoint{0.738497in}{1.547647in}}%
\pgfpathlineto{\pgfqpoint{0.730097in}{1.540859in}}%
\pgfpathlineto{\pgfqpoint{0.730363in}{1.531286in}}%
\pgfpathlineto{\pgfqpoint{0.725481in}{1.527601in}}%
\pgfpathlineto{\pgfqpoint{0.724276in}{1.521184in}}%
\pgfpathlineto{\pgfqpoint{0.718561in}{1.521037in}}%
\pgfpathlineto{\pgfqpoint{0.655392in}{1.537674in}}%
\pgfpathlineto{\pgfqpoint{0.647659in}{1.529088in}}%
\pgfpathlineto{\pgfqpoint{0.645777in}{1.522062in}}%
\pgfpathlineto{\pgfqpoint{0.652474in}{1.520325in}}%
\pgfpathlineto{\pgfqpoint{0.643201in}{1.487011in}}%
\pgfpathlineto{\pgfqpoint{0.628251in}{1.491065in}}%
\pgfpathclose%
\pgfusepath{fill}%
\end{pgfscope}%
\begin{pgfscope}%
\pgfpathrectangle{\pgfqpoint{0.100000in}{0.100000in}}{\pgfqpoint{3.608454in}{2.310000in}}%
\pgfusepath{clip}%
\pgfsetbuttcap%
\pgfsetmiterjoin%
\definecolor{currentfill}{rgb}{0.000000,0.470588,0.764706}%
\pgfsetfillcolor{currentfill}%
\pgfsetlinewidth{0.000000pt}%
\definecolor{currentstroke}{rgb}{0.000000,0.000000,0.000000}%
\pgfsetstrokecolor{currentstroke}%
\pgfsetstrokeopacity{0.000000}%
\pgfsetdash{}{0pt}%
\pgfpathmoveto{\pgfqpoint{1.889333in}{1.734748in}}%
\pgfpathlineto{\pgfqpoint{1.882514in}{1.735032in}}%
\pgfpathlineto{\pgfqpoint{1.883458in}{1.755759in}}%
\pgfpathlineto{\pgfqpoint{1.896283in}{1.755231in}}%
\pgfpathlineto{\pgfqpoint{1.897415in}{1.789939in}}%
\pgfpathlineto{\pgfqpoint{1.938531in}{1.788233in}}%
\pgfpathlineto{\pgfqpoint{1.945390in}{1.787969in}}%
\pgfpathlineto{\pgfqpoint{1.945343in}{1.781068in}}%
\pgfpathlineto{\pgfqpoint{1.944449in}{1.753368in}}%
\pgfpathlineto{\pgfqpoint{1.917260in}{1.754415in}}%
\pgfpathlineto{\pgfqpoint{1.916523in}{1.733891in}}%
\pgfpathlineto{\pgfqpoint{1.889333in}{1.734748in}}%
\pgfpathclose%
\pgfusepath{fill}%
\end{pgfscope}%
\begin{pgfscope}%
\pgfpathrectangle{\pgfqpoint{0.100000in}{0.100000in}}{\pgfqpoint{3.608454in}{2.310000in}}%
\pgfusepath{clip}%
\pgfsetbuttcap%
\pgfsetmiterjoin%
\definecolor{currentfill}{rgb}{0.000000,0.647059,0.676471}%
\pgfsetfillcolor{currentfill}%
\pgfsetlinewidth{0.000000pt}%
\definecolor{currentstroke}{rgb}{0.000000,0.000000,0.000000}%
\pgfsetstrokecolor{currentstroke}%
\pgfsetstrokeopacity{0.000000}%
\pgfsetdash{}{0pt}%
\pgfpathmoveto{\pgfqpoint{3.262571in}{1.349107in}}%
\pgfpathlineto{\pgfqpoint{3.255578in}{1.348500in}}%
\pgfpathlineto{\pgfqpoint{3.250135in}{1.356031in}}%
\pgfpathlineto{\pgfqpoint{3.244523in}{1.358215in}}%
\pgfpathlineto{\pgfqpoint{3.240232in}{1.353321in}}%
\pgfpathlineto{\pgfqpoint{3.236077in}{1.357801in}}%
\pgfpathlineto{\pgfqpoint{3.228182in}{1.355944in}}%
\pgfpathlineto{\pgfqpoint{3.222015in}{1.359018in}}%
\pgfpathlineto{\pgfqpoint{3.217018in}{1.367687in}}%
\pgfpathlineto{\pgfqpoint{3.209188in}{1.375295in}}%
\pgfpathlineto{\pgfqpoint{3.214965in}{1.384823in}}%
\pgfpathlineto{\pgfqpoint{3.216571in}{1.392855in}}%
\pgfpathlineto{\pgfqpoint{3.225674in}{1.386154in}}%
\pgfpathlineto{\pgfqpoint{3.242132in}{1.387601in}}%
\pgfpathlineto{\pgfqpoint{3.253966in}{1.377009in}}%
\pgfpathlineto{\pgfqpoint{3.269135in}{1.372212in}}%
\pgfpathlineto{\pgfqpoint{3.266819in}{1.363023in}}%
\pgfpathlineto{\pgfqpoint{3.268336in}{1.356399in}}%
\pgfpathlineto{\pgfqpoint{3.266028in}{1.348977in}}%
\pgfpathlineto{\pgfqpoint{3.262571in}{1.349107in}}%
\pgfpathclose%
\pgfusepath{fill}%
\end{pgfscope}%
\begin{pgfscope}%
\pgfpathrectangle{\pgfqpoint{0.100000in}{0.100000in}}{\pgfqpoint{3.608454in}{2.310000in}}%
\pgfusepath{clip}%
\pgfsetbuttcap%
\pgfsetmiterjoin%
\definecolor{currentfill}{rgb}{0.000000,0.498039,0.750980}%
\pgfsetfillcolor{currentfill}%
\pgfsetlinewidth{0.000000pt}%
\definecolor{currentstroke}{rgb}{0.000000,0.000000,0.000000}%
\pgfsetstrokecolor{currentstroke}%
\pgfsetstrokeopacity{0.000000}%
\pgfsetdash{}{0pt}%
\pgfpathmoveto{\pgfqpoint{1.938292in}{1.256226in}}%
\pgfpathlineto{\pgfqpoint{1.938084in}{1.242176in}}%
\pgfpathlineto{\pgfqpoint{1.890205in}{1.243605in}}%
\pgfpathlineto{\pgfqpoint{1.890012in}{1.250882in}}%
\pgfpathlineto{\pgfqpoint{1.891797in}{1.319820in}}%
\pgfpathlineto{\pgfqpoint{1.892401in}{1.333545in}}%
\pgfpathlineto{\pgfqpoint{1.926584in}{1.332368in}}%
\pgfpathlineto{\pgfqpoint{1.926190in}{1.311718in}}%
\pgfpathlineto{\pgfqpoint{1.925218in}{1.277253in}}%
\pgfpathlineto{\pgfqpoint{1.938903in}{1.276826in}}%
\pgfpathlineto{\pgfqpoint{1.938292in}{1.256226in}}%
\pgfpathclose%
\pgfusepath{fill}%
\end{pgfscope}%
\begin{pgfscope}%
\pgfpathrectangle{\pgfqpoint{0.100000in}{0.100000in}}{\pgfqpoint{3.608454in}{2.310000in}}%
\pgfusepath{clip}%
\pgfsetbuttcap%
\pgfsetmiterjoin%
\definecolor{currentfill}{rgb}{0.000000,0.631373,0.684314}%
\pgfsetfillcolor{currentfill}%
\pgfsetlinewidth{0.000000pt}%
\definecolor{currentstroke}{rgb}{0.000000,0.000000,0.000000}%
\pgfsetstrokecolor{currentstroke}%
\pgfsetstrokeopacity{0.000000}%
\pgfsetdash{}{0pt}%
\pgfpathmoveto{\pgfqpoint{2.856403in}{0.756814in}}%
\pgfpathlineto{\pgfqpoint{2.831112in}{0.753917in}}%
\pgfpathlineto{\pgfqpoint{2.819817in}{0.752621in}}%
\pgfpathlineto{\pgfqpoint{2.818573in}{0.766684in}}%
\pgfpathlineto{\pgfqpoint{2.811462in}{0.766109in}}%
\pgfpathlineto{\pgfqpoint{2.810152in}{0.780002in}}%
\pgfpathlineto{\pgfqpoint{2.820126in}{0.780667in}}%
\pgfpathlineto{\pgfqpoint{2.825145in}{0.783605in}}%
\pgfpathlineto{\pgfqpoint{2.821443in}{0.790502in}}%
\pgfpathlineto{\pgfqpoint{2.821071in}{0.796344in}}%
\pgfpathlineto{\pgfqpoint{2.814786in}{0.800405in}}%
\pgfpathlineto{\pgfqpoint{2.809408in}{0.807037in}}%
\pgfpathlineto{\pgfqpoint{2.808283in}{0.818788in}}%
\pgfpathlineto{\pgfqpoint{2.853827in}{0.823500in}}%
\pgfpathlineto{\pgfqpoint{2.854299in}{0.818491in}}%
\pgfpathlineto{\pgfqpoint{2.862925in}{0.815951in}}%
\pgfpathlineto{\pgfqpoint{2.865088in}{0.801722in}}%
\pgfpathlineto{\pgfqpoint{2.870385in}{0.802308in}}%
\pgfpathlineto{\pgfqpoint{2.875554in}{0.798633in}}%
\pgfpathlineto{\pgfqpoint{2.877599in}{0.779820in}}%
\pgfpathlineto{\pgfqpoint{2.854226in}{0.777256in}}%
\pgfpathlineto{\pgfqpoint{2.856403in}{0.756814in}}%
\pgfpathclose%
\pgfusepath{fill}%
\end{pgfscope}%
\begin{pgfscope}%
\pgfpathrectangle{\pgfqpoint{0.100000in}{0.100000in}}{\pgfqpoint{3.608454in}{2.310000in}}%
\pgfusepath{clip}%
\pgfsetbuttcap%
\pgfsetmiterjoin%
\definecolor{currentfill}{rgb}{0.000000,0.619608,0.690196}%
\pgfsetfillcolor{currentfill}%
\pgfsetlinewidth{0.000000pt}%
\definecolor{currentstroke}{rgb}{0.000000,0.000000,0.000000}%
\pgfsetstrokecolor{currentstroke}%
\pgfsetstrokeopacity{0.000000}%
\pgfsetdash{}{0pt}%
\pgfpathmoveto{\pgfqpoint{2.631561in}{1.887489in}}%
\pgfpathlineto{\pgfqpoint{2.627523in}{1.889173in}}%
\pgfpathlineto{\pgfqpoint{2.629695in}{1.902348in}}%
\pgfpathlineto{\pgfqpoint{2.633726in}{1.900071in}}%
\pgfpathlineto{\pgfqpoint{2.635138in}{1.890834in}}%
\pgfpathlineto{\pgfqpoint{2.631561in}{1.887489in}}%
\pgfpathclose%
\pgfusepath{fill}%
\end{pgfscope}%
\begin{pgfscope}%
\pgfpathrectangle{\pgfqpoint{0.100000in}{0.100000in}}{\pgfqpoint{3.608454in}{2.310000in}}%
\pgfusepath{clip}%
\pgfsetbuttcap%
\pgfsetmiterjoin%
\definecolor{currentfill}{rgb}{0.000000,0.619608,0.690196}%
\pgfsetfillcolor{currentfill}%
\pgfsetlinewidth{0.000000pt}%
\definecolor{currentstroke}{rgb}{0.000000,0.000000,0.000000}%
\pgfsetstrokecolor{currentstroke}%
\pgfsetstrokeopacity{0.000000}%
\pgfsetdash{}{0pt}%
\pgfpathmoveto{\pgfqpoint{2.706092in}{1.900170in}}%
\pgfpathlineto{\pgfqpoint{2.703507in}{1.899857in}}%
\pgfpathlineto{\pgfqpoint{2.707810in}{1.866142in}}%
\pgfpathlineto{\pgfqpoint{2.680662in}{1.863242in}}%
\pgfpathlineto{\pgfqpoint{2.681411in}{1.856342in}}%
\pgfpathlineto{\pgfqpoint{2.674595in}{1.855705in}}%
\pgfpathlineto{\pgfqpoint{2.654113in}{1.853564in}}%
\pgfpathlineto{\pgfqpoint{2.653419in}{1.860520in}}%
\pgfpathlineto{\pgfqpoint{2.644203in}{1.859587in}}%
\pgfpathlineto{\pgfqpoint{2.648170in}{1.869172in}}%
\pgfpathlineto{\pgfqpoint{2.656698in}{1.874267in}}%
\pgfpathlineto{\pgfqpoint{2.661603in}{1.879931in}}%
\pgfpathlineto{\pgfqpoint{2.658078in}{1.883394in}}%
\pgfpathlineto{\pgfqpoint{2.656143in}{1.891002in}}%
\pgfpathlineto{\pgfqpoint{2.664544in}{1.902846in}}%
\pgfpathlineto{\pgfqpoint{2.671993in}{1.905904in}}%
\pgfpathlineto{\pgfqpoint{2.673199in}{1.909457in}}%
\pgfpathlineto{\pgfqpoint{2.691843in}{1.900527in}}%
\pgfpathlineto{\pgfqpoint{2.698822in}{1.902391in}}%
\pgfpathlineto{\pgfqpoint{2.706092in}{1.900170in}}%
\pgfpathclose%
\pgfusepath{fill}%
\end{pgfscope}%
\begin{pgfscope}%
\pgfpathrectangle{\pgfqpoint{0.100000in}{0.100000in}}{\pgfqpoint{3.608454in}{2.310000in}}%
\pgfusepath{clip}%
\pgfsetbuttcap%
\pgfsetmiterjoin%
\definecolor{currentfill}{rgb}{0.000000,0.670588,0.664706}%
\pgfsetfillcolor{currentfill}%
\pgfsetlinewidth{0.000000pt}%
\definecolor{currentstroke}{rgb}{0.000000,0.000000,0.000000}%
\pgfsetstrokecolor{currentstroke}%
\pgfsetstrokeopacity{0.000000}%
\pgfsetdash{}{0pt}%
\pgfpathmoveto{\pgfqpoint{2.878161in}{0.725802in}}%
\pgfpathlineto{\pgfqpoint{2.887245in}{0.726366in}}%
\pgfpathlineto{\pgfqpoint{2.888448in}{0.720934in}}%
\pgfpathlineto{\pgfqpoint{2.880133in}{0.713968in}}%
\pgfpathlineto{\pgfqpoint{2.881876in}{0.708207in}}%
\pgfpathlineto{\pgfqpoint{2.876684in}{0.701966in}}%
\pgfpathlineto{\pgfqpoint{2.878191in}{0.698489in}}%
\pgfpathlineto{\pgfqpoint{2.870453in}{0.691569in}}%
\pgfpathlineto{\pgfqpoint{2.866769in}{0.678489in}}%
\pgfpathlineto{\pgfqpoint{2.854948in}{0.675589in}}%
\pgfpathlineto{\pgfqpoint{2.847384in}{0.676389in}}%
\pgfpathlineto{\pgfqpoint{2.841677in}{0.670279in}}%
\pgfpathlineto{\pgfqpoint{2.844033in}{0.665474in}}%
\pgfpathlineto{\pgfqpoint{2.829477in}{0.667841in}}%
\pgfpathlineto{\pgfqpoint{2.797575in}{0.664470in}}%
\pgfpathlineto{\pgfqpoint{2.788514in}{0.665867in}}%
\pgfpathlineto{\pgfqpoint{2.787376in}{0.676778in}}%
\pgfpathlineto{\pgfqpoint{2.769691in}{0.676534in}}%
\pgfpathlineto{\pgfqpoint{2.766984in}{0.705619in}}%
\pgfpathlineto{\pgfqpoint{2.760006in}{0.704933in}}%
\pgfpathlineto{\pgfqpoint{2.760979in}{0.694551in}}%
\pgfpathlineto{\pgfqpoint{2.735873in}{0.692265in}}%
\pgfpathlineto{\pgfqpoint{2.726751in}{0.687235in}}%
\pgfpathlineto{\pgfqpoint{2.735056in}{0.696150in}}%
\pgfpathlineto{\pgfqpoint{2.730084in}{0.707397in}}%
\pgfpathlineto{\pgfqpoint{2.733506in}{0.712730in}}%
\pgfpathlineto{\pgfqpoint{2.721186in}{0.711980in}}%
\pgfpathlineto{\pgfqpoint{2.719091in}{0.734938in}}%
\pgfpathlineto{\pgfqpoint{2.789288in}{0.742878in}}%
\pgfpathlineto{\pgfqpoint{2.794346in}{0.734429in}}%
\pgfpathlineto{\pgfqpoint{2.795289in}{0.729104in}}%
\pgfpathlineto{\pgfqpoint{2.801217in}{0.720980in}}%
\pgfpathlineto{\pgfqpoint{2.834431in}{0.723057in}}%
\pgfpathlineto{\pgfqpoint{2.878161in}{0.725802in}}%
\pgfpathclose%
\pgfusepath{fill}%
\end{pgfscope}%
\begin{pgfscope}%
\pgfpathrectangle{\pgfqpoint{0.100000in}{0.100000in}}{\pgfqpoint{3.608454in}{2.310000in}}%
\pgfusepath{clip}%
\pgfsetbuttcap%
\pgfsetmiterjoin%
\definecolor{currentfill}{rgb}{0.000000,0.431373,0.784314}%
\pgfsetfillcolor{currentfill}%
\pgfsetlinewidth{0.000000pt}%
\definecolor{currentstroke}{rgb}{0.000000,0.000000,0.000000}%
\pgfsetstrokecolor{currentstroke}%
\pgfsetstrokeopacity{0.000000}%
\pgfsetdash{}{0pt}%
\pgfpathmoveto{\pgfqpoint{1.718524in}{1.959750in}}%
\pgfpathlineto{\pgfqpoint{1.719085in}{1.966639in}}%
\pgfpathlineto{\pgfqpoint{1.739705in}{1.965094in}}%
\pgfpathlineto{\pgfqpoint{1.739184in}{1.958172in}}%
\pgfpathlineto{\pgfqpoint{1.762088in}{1.956478in}}%
\pgfpathlineto{\pgfqpoint{1.760656in}{1.935885in}}%
\pgfpathlineto{\pgfqpoint{1.774399in}{1.934946in}}%
\pgfpathlineto{\pgfqpoint{1.775193in}{1.927930in}}%
\pgfpathlineto{\pgfqpoint{1.759368in}{1.916843in}}%
\pgfpathlineto{\pgfqpoint{1.756437in}{1.912852in}}%
\pgfpathlineto{\pgfqpoint{1.744713in}{1.908214in}}%
\pgfpathlineto{\pgfqpoint{1.730862in}{1.910799in}}%
\pgfpathlineto{\pgfqpoint{1.726123in}{1.914246in}}%
\pgfpathlineto{\pgfqpoint{1.720336in}{1.913435in}}%
\pgfpathlineto{\pgfqpoint{1.721787in}{1.931717in}}%
\pgfpathlineto{\pgfqpoint{1.719106in}{1.931926in}}%
\pgfpathlineto{\pgfqpoint{1.721015in}{1.959541in}}%
\pgfpathlineto{\pgfqpoint{1.718524in}{1.959750in}}%
\pgfpathclose%
\pgfusepath{fill}%
\end{pgfscope}%
\begin{pgfscope}%
\pgfpathrectangle{\pgfqpoint{0.100000in}{0.100000in}}{\pgfqpoint{3.608454in}{2.310000in}}%
\pgfusepath{clip}%
\pgfsetbuttcap%
\pgfsetmiterjoin%
\definecolor{currentfill}{rgb}{0.000000,0.364706,0.817647}%
\pgfsetfillcolor{currentfill}%
\pgfsetlinewidth{0.000000pt}%
\definecolor{currentstroke}{rgb}{0.000000,0.000000,0.000000}%
\pgfsetstrokecolor{currentstroke}%
\pgfsetstrokeopacity{0.000000}%
\pgfsetdash{}{0pt}%
\pgfpathmoveto{\pgfqpoint{1.670021in}{1.298287in}}%
\pgfpathlineto{\pgfqpoint{1.670049in}{1.298787in}}%
\pgfpathlineto{\pgfqpoint{1.672634in}{1.332579in}}%
\pgfpathlineto{\pgfqpoint{1.707133in}{1.330203in}}%
\pgfpathlineto{\pgfqpoint{1.756162in}{1.326912in}}%
\pgfpathlineto{\pgfqpoint{1.754213in}{1.292521in}}%
\pgfpathlineto{\pgfqpoint{1.699539in}{1.296124in}}%
\pgfpathlineto{\pgfqpoint{1.670021in}{1.298287in}}%
\pgfpathclose%
\pgfusepath{fill}%
\end{pgfscope}%
\begin{pgfscope}%
\pgfpathrectangle{\pgfqpoint{0.100000in}{0.100000in}}{\pgfqpoint{3.608454in}{2.310000in}}%
\pgfusepath{clip}%
\pgfsetbuttcap%
\pgfsetmiterjoin%
\definecolor{currentfill}{rgb}{0.000000,0.454902,0.772549}%
\pgfsetfillcolor{currentfill}%
\pgfsetlinewidth{0.000000pt}%
\definecolor{currentstroke}{rgb}{0.000000,0.000000,0.000000}%
\pgfsetstrokecolor{currentstroke}%
\pgfsetstrokeopacity{0.000000}%
\pgfsetdash{}{0pt}%
\pgfpathmoveto{\pgfqpoint{2.084392in}{1.676317in}}%
\pgfpathlineto{\pgfqpoint{2.084271in}{1.648958in}}%
\pgfpathlineto{\pgfqpoint{2.056914in}{1.649093in}}%
\pgfpathlineto{\pgfqpoint{2.017541in}{1.649592in}}%
\pgfpathlineto{\pgfqpoint{2.019889in}{1.653420in}}%
\pgfpathlineto{\pgfqpoint{2.019320in}{1.660716in}}%
\pgfpathlineto{\pgfqpoint{2.022652in}{1.663329in}}%
\pgfpathlineto{\pgfqpoint{2.021565in}{1.674298in}}%
\pgfpathlineto{\pgfqpoint{2.018524in}{1.680571in}}%
\pgfpathlineto{\pgfqpoint{2.019001in}{1.688123in}}%
\pgfpathlineto{\pgfqpoint{2.015077in}{1.691199in}}%
\pgfpathlineto{\pgfqpoint{2.014851in}{1.696524in}}%
\pgfpathlineto{\pgfqpoint{2.023239in}{1.696380in}}%
\pgfpathlineto{\pgfqpoint{2.046244in}{1.696054in}}%
\pgfpathlineto{\pgfqpoint{2.057273in}{1.695912in}}%
\pgfpathlineto{\pgfqpoint{2.057034in}{1.676704in}}%
\pgfpathlineto{\pgfqpoint{2.084392in}{1.676317in}}%
\pgfpathclose%
\pgfusepath{fill}%
\end{pgfscope}%
\begin{pgfscope}%
\pgfpathrectangle{\pgfqpoint{0.100000in}{0.100000in}}{\pgfqpoint{3.608454in}{2.310000in}}%
\pgfusepath{clip}%
\pgfsetbuttcap%
\pgfsetmiterjoin%
\definecolor{currentfill}{rgb}{0.000000,0.439216,0.780392}%
\pgfsetfillcolor{currentfill}%
\pgfsetlinewidth{0.000000pt}%
\definecolor{currentstroke}{rgb}{0.000000,0.000000,0.000000}%
\pgfsetstrokecolor{currentstroke}%
\pgfsetstrokeopacity{0.000000}%
\pgfsetdash{}{0pt}%
\pgfpathmoveto{\pgfqpoint{3.420582in}{1.833115in}}%
\pgfpathlineto{\pgfqpoint{3.422865in}{1.818650in}}%
\pgfpathlineto{\pgfqpoint{3.426046in}{1.819395in}}%
\pgfpathlineto{\pgfqpoint{3.430894in}{1.808006in}}%
\pgfpathlineto{\pgfqpoint{3.434992in}{1.802755in}}%
\pgfpathlineto{\pgfqpoint{3.414849in}{1.798333in}}%
\pgfpathlineto{\pgfqpoint{3.377987in}{1.790511in}}%
\pgfpathlineto{\pgfqpoint{3.374916in}{1.805051in}}%
\pgfpathlineto{\pgfqpoint{3.369972in}{1.804097in}}%
\pgfpathlineto{\pgfqpoint{3.367008in}{1.818791in}}%
\pgfpathlineto{\pgfqpoint{3.374319in}{1.820118in}}%
\pgfpathlineto{\pgfqpoint{3.374364in}{1.831752in}}%
\pgfpathlineto{\pgfqpoint{3.382485in}{1.831082in}}%
\pgfpathlineto{\pgfqpoint{3.396386in}{1.835470in}}%
\pgfpathlineto{\pgfqpoint{3.401069in}{1.830482in}}%
\pgfpathlineto{\pgfqpoint{3.412904in}{1.835488in}}%
\pgfpathlineto{\pgfqpoint{3.420582in}{1.833115in}}%
\pgfpathclose%
\pgfusepath{fill}%
\end{pgfscope}%
\begin{pgfscope}%
\pgfpathrectangle{\pgfqpoint{0.100000in}{0.100000in}}{\pgfqpoint{3.608454in}{2.310000in}}%
\pgfusepath{clip}%
\pgfsetbuttcap%
\pgfsetmiterjoin%
\definecolor{currentfill}{rgb}{0.000000,0.403922,0.798039}%
\pgfsetfillcolor{currentfill}%
\pgfsetlinewidth{0.000000pt}%
\definecolor{currentstroke}{rgb}{0.000000,0.000000,0.000000}%
\pgfsetstrokecolor{currentstroke}%
\pgfsetstrokeopacity{0.000000}%
\pgfsetdash{}{0pt}%
\pgfpathmoveto{\pgfqpoint{1.357331in}{0.522521in}}%
\pgfpathlineto{\pgfqpoint{1.350294in}{0.534978in}}%
\pgfpathlineto{\pgfqpoint{1.341731in}{0.539953in}}%
\pgfpathlineto{\pgfqpoint{1.343780in}{0.546075in}}%
\pgfpathlineto{\pgfqpoint{1.344206in}{0.557855in}}%
\pgfpathlineto{\pgfqpoint{1.348048in}{0.565231in}}%
\pgfpathlineto{\pgfqpoint{1.356655in}{0.558900in}}%
\pgfpathlineto{\pgfqpoint{1.368635in}{0.560958in}}%
\pgfpathlineto{\pgfqpoint{1.371871in}{0.559296in}}%
\pgfpathlineto{\pgfqpoint{1.370602in}{0.544014in}}%
\pgfpathlineto{\pgfqpoint{1.366190in}{0.537636in}}%
\pgfpathlineto{\pgfqpoint{1.369907in}{0.525625in}}%
\pgfpathlineto{\pgfqpoint{1.369026in}{0.517805in}}%
\pgfpathlineto{\pgfqpoint{1.357331in}{0.522521in}}%
\pgfpathclose%
\pgfusepath{fill}%
\end{pgfscope}%
\begin{pgfscope}%
\pgfpathrectangle{\pgfqpoint{0.100000in}{0.100000in}}{\pgfqpoint{3.608454in}{2.310000in}}%
\pgfusepath{clip}%
\pgfsetbuttcap%
\pgfsetmiterjoin%
\definecolor{currentfill}{rgb}{0.000000,0.521569,0.739216}%
\pgfsetfillcolor{currentfill}%
\pgfsetlinewidth{0.000000pt}%
\definecolor{currentstroke}{rgb}{0.000000,0.000000,0.000000}%
\pgfsetstrokecolor{currentstroke}%
\pgfsetstrokeopacity{0.000000}%
\pgfsetdash{}{0pt}%
\pgfpathmoveto{\pgfqpoint{0.575043in}{1.085831in}}%
\pgfpathlineto{\pgfqpoint{0.566660in}{1.090402in}}%
\pgfpathlineto{\pgfqpoint{0.566840in}{1.100543in}}%
\pgfpathlineto{\pgfqpoint{0.574437in}{1.095234in}}%
\pgfpathlineto{\pgfqpoint{0.575043in}{1.085831in}}%
\pgfpathclose%
\pgfusepath{fill}%
\end{pgfscope}%
\begin{pgfscope}%
\pgfpathrectangle{\pgfqpoint{0.100000in}{0.100000in}}{\pgfqpoint{3.608454in}{2.310000in}}%
\pgfusepath{clip}%
\pgfsetbuttcap%
\pgfsetmiterjoin%
\definecolor{currentfill}{rgb}{0.000000,0.521569,0.739216}%
\pgfsetfillcolor{currentfill}%
\pgfsetlinewidth{0.000000pt}%
\definecolor{currentstroke}{rgb}{0.000000,0.000000,0.000000}%
\pgfsetstrokecolor{currentstroke}%
\pgfsetstrokeopacity{0.000000}%
\pgfsetdash{}{0pt}%
\pgfpathmoveto{\pgfqpoint{0.623749in}{1.080645in}}%
\pgfpathlineto{\pgfqpoint{0.614676in}{1.095570in}}%
\pgfpathlineto{\pgfqpoint{0.606778in}{1.102859in}}%
\pgfpathlineto{\pgfqpoint{0.598881in}{1.114997in}}%
\pgfpathlineto{\pgfqpoint{0.593720in}{1.119151in}}%
\pgfpathlineto{\pgfqpoint{0.585255in}{1.116816in}}%
\pgfpathlineto{\pgfqpoint{0.578599in}{1.121274in}}%
\pgfpathlineto{\pgfqpoint{0.581839in}{1.128446in}}%
\pgfpathlineto{\pgfqpoint{0.579418in}{1.141321in}}%
\pgfpathlineto{\pgfqpoint{0.576385in}{1.146196in}}%
\pgfpathlineto{\pgfqpoint{0.563289in}{1.149271in}}%
\pgfpathlineto{\pgfqpoint{0.558328in}{1.148636in}}%
\pgfpathlineto{\pgfqpoint{0.535794in}{1.165997in}}%
\pgfpathlineto{\pgfqpoint{0.534238in}{1.176440in}}%
\pgfpathlineto{\pgfqpoint{0.524045in}{1.187970in}}%
\pgfpathlineto{\pgfqpoint{0.527921in}{1.193369in}}%
\pgfpathlineto{\pgfqpoint{0.537333in}{1.227732in}}%
\pgfpathlineto{\pgfqpoint{0.547247in}{1.223234in}}%
\pgfpathlineto{\pgfqpoint{0.547967in}{1.217656in}}%
\pgfpathlineto{\pgfqpoint{0.598059in}{1.204495in}}%
\pgfpathlineto{\pgfqpoint{0.649520in}{1.191532in}}%
\pgfpathlineto{\pgfqpoint{0.668493in}{1.266395in}}%
\pgfpathlineto{\pgfqpoint{0.692238in}{1.260226in}}%
\pgfpathlineto{\pgfqpoint{0.765878in}{1.242208in}}%
\pgfpathlineto{\pgfqpoint{0.786789in}{1.237324in}}%
\pgfpathlineto{\pgfqpoint{0.792611in}{1.237224in}}%
\pgfpathlineto{\pgfqpoint{0.842429in}{1.160488in}}%
\pgfpathlineto{\pgfqpoint{0.840274in}{1.151795in}}%
\pgfpathlineto{\pgfqpoint{0.843543in}{1.141162in}}%
\pgfpathlineto{\pgfqpoint{0.847819in}{1.135796in}}%
\pgfpathlineto{\pgfqpoint{0.848041in}{1.126325in}}%
\pgfpathlineto{\pgfqpoint{0.851945in}{1.113704in}}%
\pgfpathlineto{\pgfqpoint{0.862231in}{1.099514in}}%
\pgfpathlineto{\pgfqpoint{0.868397in}{1.094701in}}%
\pgfpathlineto{\pgfqpoint{0.895182in}{1.086583in}}%
\pgfpathlineto{\pgfqpoint{0.900636in}{1.091652in}}%
\pgfpathlineto{\pgfqpoint{0.914070in}{1.089859in}}%
\pgfpathlineto{\pgfqpoint{0.908989in}{1.065097in}}%
\pgfpathlineto{\pgfqpoint{0.899087in}{1.016739in}}%
\pgfpathlineto{\pgfqpoint{0.858565in}{1.025184in}}%
\pgfpathlineto{\pgfqpoint{0.859997in}{1.031950in}}%
\pgfpathlineto{\pgfqpoint{0.839911in}{1.036150in}}%
\pgfpathlineto{\pgfqpoint{0.832599in}{1.002476in}}%
\pgfpathlineto{\pgfqpoint{0.807337in}{1.008296in}}%
\pgfpathlineto{\pgfqpoint{0.805414in}{1.014538in}}%
\pgfpathlineto{\pgfqpoint{0.809954in}{1.027225in}}%
\pgfpathlineto{\pgfqpoint{0.809266in}{1.037990in}}%
\pgfpathlineto{\pgfqpoint{0.787832in}{1.044715in}}%
\pgfpathlineto{\pgfqpoint{0.721788in}{1.059750in}}%
\pgfpathlineto{\pgfqpoint{0.660882in}{1.074556in}}%
\pgfpathlineto{\pgfqpoint{0.647432in}{1.078308in}}%
\pgfpathlineto{\pgfqpoint{0.640257in}{1.084919in}}%
\pgfpathlineto{\pgfqpoint{0.631612in}{1.088268in}}%
\pgfpathlineto{\pgfqpoint{0.626202in}{1.085453in}}%
\pgfpathlineto{\pgfqpoint{0.623749in}{1.080645in}}%
\pgfpathclose%
\pgfusepath{fill}%
\end{pgfscope}%
\begin{pgfscope}%
\pgfpathrectangle{\pgfqpoint{0.100000in}{0.100000in}}{\pgfqpoint{3.608454in}{2.310000in}}%
\pgfusepath{clip}%
\pgfsetbuttcap%
\pgfsetmiterjoin%
\definecolor{currentfill}{rgb}{0.000000,0.764706,0.617647}%
\pgfsetfillcolor{currentfill}%
\pgfsetlinewidth{0.000000pt}%
\definecolor{currentstroke}{rgb}{0.000000,0.000000,0.000000}%
\pgfsetstrokecolor{currentstroke}%
\pgfsetstrokeopacity{0.000000}%
\pgfsetdash{}{0pt}%
\pgfpathmoveto{\pgfqpoint{2.669844in}{0.772749in}}%
\pgfpathlineto{\pgfqpoint{2.683521in}{0.774139in}}%
\pgfpathlineto{\pgfqpoint{2.682532in}{0.784446in}}%
\pgfpathlineto{\pgfqpoint{2.685966in}{0.784799in}}%
\pgfpathlineto{\pgfqpoint{2.683683in}{0.809309in}}%
\pgfpathlineto{\pgfqpoint{2.685856in}{0.816401in}}%
\pgfpathlineto{\pgfqpoint{2.692745in}{0.817134in}}%
\pgfpathlineto{\pgfqpoint{2.693497in}{0.810278in}}%
\pgfpathlineto{\pgfqpoint{2.714111in}{0.812491in}}%
\pgfpathlineto{\pgfqpoint{2.713352in}{0.819097in}}%
\pgfpathlineto{\pgfqpoint{2.720311in}{0.819497in}}%
\pgfpathlineto{\pgfqpoint{2.721575in}{0.813214in}}%
\pgfpathlineto{\pgfqpoint{2.727912in}{0.813855in}}%
\pgfpathlineto{\pgfqpoint{2.728545in}{0.806963in}}%
\pgfpathlineto{\pgfqpoint{2.737520in}{0.807875in}}%
\pgfpathlineto{\pgfqpoint{2.737974in}{0.800305in}}%
\pgfpathlineto{\pgfqpoint{2.733549in}{0.786465in}}%
\pgfpathlineto{\pgfqpoint{2.706700in}{0.783758in}}%
\pgfpathlineto{\pgfqpoint{2.704075in}{0.776471in}}%
\pgfpathlineto{\pgfqpoint{2.708723in}{0.733977in}}%
\pgfpathlineto{\pgfqpoint{2.674593in}{0.730790in}}%
\pgfpathlineto{\pgfqpoint{2.673739in}{0.731186in}}%
\pgfpathlineto{\pgfqpoint{2.669844in}{0.772749in}}%
\pgfpathclose%
\pgfusepath{fill}%
\end{pgfscope}%
\begin{pgfscope}%
\pgfpathrectangle{\pgfqpoint{0.100000in}{0.100000in}}{\pgfqpoint{3.608454in}{2.310000in}}%
\pgfusepath{clip}%
\pgfsetbuttcap%
\pgfsetmiterjoin%
\definecolor{currentfill}{rgb}{0.000000,0.431373,0.784314}%
\pgfsetfillcolor{currentfill}%
\pgfsetlinewidth{0.000000pt}%
\definecolor{currentstroke}{rgb}{0.000000,0.000000,0.000000}%
\pgfsetstrokecolor{currentstroke}%
\pgfsetstrokeopacity{0.000000}%
\pgfsetdash{}{0pt}%
\pgfpathmoveto{\pgfqpoint{2.420217in}{1.066185in}}%
\pgfpathlineto{\pgfqpoint{2.427918in}{1.068838in}}%
\pgfpathlineto{\pgfqpoint{2.426249in}{1.075405in}}%
\pgfpathlineto{\pgfqpoint{2.434616in}{1.075007in}}%
\pgfpathlineto{\pgfqpoint{2.441604in}{1.083493in}}%
\pgfpathlineto{\pgfqpoint{2.453807in}{1.083497in}}%
\pgfpathlineto{\pgfqpoint{2.461856in}{1.079471in}}%
\pgfpathlineto{\pgfqpoint{2.463288in}{1.065984in}}%
\pgfpathlineto{\pgfqpoint{2.482016in}{1.066685in}}%
\pgfpathlineto{\pgfqpoint{2.483127in}{1.034794in}}%
\pgfpathlineto{\pgfqpoint{2.473129in}{1.034113in}}%
\pgfpathlineto{\pgfqpoint{2.473863in}{1.022295in}}%
\pgfpathlineto{\pgfqpoint{2.477335in}{1.022510in}}%
\pgfpathlineto{\pgfqpoint{2.478692in}{1.001827in}}%
\pgfpathlineto{\pgfqpoint{2.482530in}{0.995135in}}%
\pgfpathlineto{\pgfqpoint{2.467215in}{0.998833in}}%
\pgfpathlineto{\pgfqpoint{2.420148in}{0.996176in}}%
\pgfpathlineto{\pgfqpoint{2.413440in}{0.992269in}}%
\pgfpathlineto{\pgfqpoint{2.413949in}{0.985540in}}%
\pgfpathlineto{\pgfqpoint{2.407574in}{0.985155in}}%
\pgfpathlineto{\pgfqpoint{2.396088in}{0.992607in}}%
\pgfpathlineto{\pgfqpoint{2.394572in}{0.999746in}}%
\pgfpathlineto{\pgfqpoint{2.401872in}{1.004184in}}%
\pgfpathlineto{\pgfqpoint{2.404445in}{1.017531in}}%
\pgfpathlineto{\pgfqpoint{2.415331in}{1.023934in}}%
\pgfpathlineto{\pgfqpoint{2.412041in}{1.029240in}}%
\pgfpathlineto{\pgfqpoint{2.417154in}{1.033431in}}%
\pgfpathlineto{\pgfqpoint{2.419408in}{1.040872in}}%
\pgfpathlineto{\pgfqpoint{2.424139in}{1.041016in}}%
\pgfpathlineto{\pgfqpoint{2.425407in}{1.049147in}}%
\pgfpathlineto{\pgfqpoint{2.420989in}{1.051604in}}%
\pgfpathlineto{\pgfqpoint{2.424647in}{1.062261in}}%
\pgfpathlineto{\pgfqpoint{2.420217in}{1.066185in}}%
\pgfpathclose%
\pgfusepath{fill}%
\end{pgfscope}%
\begin{pgfscope}%
\pgfpathrectangle{\pgfqpoint{0.100000in}{0.100000in}}{\pgfqpoint{3.608454in}{2.310000in}}%
\pgfusepath{clip}%
\pgfsetbuttcap%
\pgfsetmiterjoin%
\definecolor{currentfill}{rgb}{0.000000,0.937255,0.531373}%
\pgfsetfillcolor{currentfill}%
\pgfsetlinewidth{0.000000pt}%
\definecolor{currentstroke}{rgb}{0.000000,0.000000,0.000000}%
\pgfsetstrokecolor{currentstroke}%
\pgfsetstrokeopacity{0.000000}%
\pgfsetdash{}{0pt}%
\pgfpathmoveto{\pgfqpoint{2.826981in}{1.247494in}}%
\pgfpathlineto{\pgfqpoint{2.816644in}{1.252449in}}%
\pgfpathlineto{\pgfqpoint{2.809988in}{1.252552in}}%
\pgfpathlineto{\pgfqpoint{2.802573in}{1.261875in}}%
\pgfpathlineto{\pgfqpoint{2.800586in}{1.266861in}}%
\pgfpathlineto{\pgfqpoint{2.804559in}{1.276722in}}%
\pgfpathlineto{\pgfqpoint{2.810307in}{1.281879in}}%
\pgfpathlineto{\pgfqpoint{2.814875in}{1.291146in}}%
\pgfpathlineto{\pgfqpoint{2.820760in}{1.291911in}}%
\pgfpathlineto{\pgfqpoint{2.825277in}{1.301789in}}%
\pgfpathlineto{\pgfqpoint{2.824999in}{1.309058in}}%
\pgfpathlineto{\pgfqpoint{2.834982in}{1.312343in}}%
\pgfpathlineto{\pgfqpoint{2.840076in}{1.309133in}}%
\pgfpathlineto{\pgfqpoint{2.850698in}{1.310252in}}%
\pgfpathlineto{\pgfqpoint{2.855369in}{1.302588in}}%
\pgfpathlineto{\pgfqpoint{2.847809in}{1.300549in}}%
\pgfpathlineto{\pgfqpoint{2.837862in}{1.289352in}}%
\pgfpathlineto{\pgfqpoint{2.839446in}{1.282704in}}%
\pgfpathlineto{\pgfqpoint{2.849787in}{1.279563in}}%
\pgfpathlineto{\pgfqpoint{2.860524in}{1.271364in}}%
\pgfpathlineto{\pgfqpoint{2.851560in}{1.271717in}}%
\pgfpathlineto{\pgfqpoint{2.850587in}{1.262241in}}%
\pgfpathlineto{\pgfqpoint{2.841576in}{1.260538in}}%
\pgfpathlineto{\pgfqpoint{2.833788in}{1.254755in}}%
\pgfpathlineto{\pgfqpoint{2.830889in}{1.256941in}}%
\pgfpathlineto{\pgfqpoint{2.825042in}{1.252479in}}%
\pgfpathlineto{\pgfqpoint{2.826981in}{1.247494in}}%
\pgfpathclose%
\pgfusepath{fill}%
\end{pgfscope}%
\begin{pgfscope}%
\pgfpathrectangle{\pgfqpoint{0.100000in}{0.100000in}}{\pgfqpoint{3.608454in}{2.310000in}}%
\pgfusepath{clip}%
\pgfsetbuttcap%
\pgfsetmiterjoin%
\definecolor{currentfill}{rgb}{0.000000,0.678431,0.660784}%
\pgfsetfillcolor{currentfill}%
\pgfsetlinewidth{0.000000pt}%
\definecolor{currentstroke}{rgb}{0.000000,0.000000,0.000000}%
\pgfsetstrokecolor{currentstroke}%
\pgfsetstrokeopacity{0.000000}%
\pgfsetdash{}{0pt}%
\pgfpathmoveto{\pgfqpoint{3.127726in}{0.403873in}}%
\pgfpathlineto{\pgfqpoint{3.133900in}{0.367620in}}%
\pgfpathlineto{\pgfqpoint{3.099940in}{0.362168in}}%
\pgfpathlineto{\pgfqpoint{3.093963in}{0.366222in}}%
\pgfpathlineto{\pgfqpoint{3.076512in}{0.366235in}}%
\pgfpathlineto{\pgfqpoint{3.068767in}{0.369764in}}%
\pgfpathlineto{\pgfqpoint{3.063694in}{0.379993in}}%
\pgfpathlineto{\pgfqpoint{3.058788in}{0.396588in}}%
\pgfpathlineto{\pgfqpoint{3.055285in}{0.402609in}}%
\pgfpathlineto{\pgfqpoint{3.046296in}{0.409678in}}%
\pgfpathlineto{\pgfqpoint{3.037249in}{0.409747in}}%
\pgfpathlineto{\pgfqpoint{3.032422in}{0.421403in}}%
\pgfpathlineto{\pgfqpoint{3.036040in}{0.423473in}}%
\pgfpathlineto{\pgfqpoint{3.037010in}{0.432074in}}%
\pgfpathlineto{\pgfqpoint{3.072400in}{0.437158in}}%
\pgfpathlineto{\pgfqpoint{3.075618in}{0.416692in}}%
\pgfpathlineto{\pgfqpoint{3.096468in}{0.420138in}}%
\pgfpathlineto{\pgfqpoint{3.099914in}{0.399053in}}%
\pgfpathlineto{\pgfqpoint{3.127726in}{0.403873in}}%
\pgfpathclose%
\pgfusepath{fill}%
\end{pgfscope}%
\begin{pgfscope}%
\pgfpathrectangle{\pgfqpoint{0.100000in}{0.100000in}}{\pgfqpoint{3.608454in}{2.310000in}}%
\pgfusepath{clip}%
\pgfsetbuttcap%
\pgfsetmiterjoin%
\definecolor{currentfill}{rgb}{0.000000,0.427451,0.786275}%
\pgfsetfillcolor{currentfill}%
\pgfsetlinewidth{0.000000pt}%
\definecolor{currentstroke}{rgb}{0.000000,0.000000,0.000000}%
\pgfsetstrokecolor{currentstroke}%
\pgfsetstrokeopacity{0.000000}%
\pgfsetdash{}{0pt}%
\pgfpathmoveto{\pgfqpoint{0.508309in}{1.308909in}}%
\pgfpathlineto{\pgfqpoint{0.468241in}{1.320678in}}%
\pgfpathlineto{\pgfqpoint{0.438474in}{1.329831in}}%
\pgfpathlineto{\pgfqpoint{0.433387in}{1.338858in}}%
\pgfpathlineto{\pgfqpoint{0.433303in}{1.348727in}}%
\pgfpathlineto{\pgfqpoint{0.430147in}{1.350732in}}%
\pgfpathlineto{\pgfqpoint{0.425066in}{1.366592in}}%
\pgfpathlineto{\pgfqpoint{0.420021in}{1.371609in}}%
\pgfpathlineto{\pgfqpoint{0.416793in}{1.379069in}}%
\pgfpathlineto{\pgfqpoint{0.418273in}{1.400974in}}%
\pgfpathlineto{\pgfqpoint{0.425399in}{1.400805in}}%
\pgfpathlineto{\pgfqpoint{0.429729in}{1.404759in}}%
\pgfpathlineto{\pgfqpoint{0.434644in}{1.414437in}}%
\pgfpathlineto{\pgfqpoint{0.433004in}{1.425553in}}%
\pgfpathlineto{\pgfqpoint{0.418942in}{1.431643in}}%
\pgfpathlineto{\pgfqpoint{0.413490in}{1.439050in}}%
\pgfpathlineto{\pgfqpoint{0.411162in}{1.446533in}}%
\pgfpathlineto{\pgfqpoint{0.411566in}{1.453040in}}%
\pgfpathlineto{\pgfqpoint{0.422155in}{1.452127in}}%
\pgfpathlineto{\pgfqpoint{0.422616in}{1.464232in}}%
\pgfpathlineto{\pgfqpoint{0.424905in}{1.469119in}}%
\pgfpathlineto{\pgfqpoint{0.432540in}{1.470642in}}%
\pgfpathlineto{\pgfqpoint{0.441374in}{1.466010in}}%
\pgfpathlineto{\pgfqpoint{0.445670in}{1.467188in}}%
\pgfpathlineto{\pgfqpoint{0.469193in}{1.459876in}}%
\pgfpathlineto{\pgfqpoint{0.470704in}{1.451769in}}%
\pgfpathlineto{\pgfqpoint{0.465818in}{1.444533in}}%
\pgfpathlineto{\pgfqpoint{0.465842in}{1.434190in}}%
\pgfpathlineto{\pgfqpoint{0.476039in}{1.431016in}}%
\pgfpathlineto{\pgfqpoint{0.475697in}{1.423778in}}%
\pgfpathlineto{\pgfqpoint{0.471004in}{1.413448in}}%
\pgfpathlineto{\pgfqpoint{0.474622in}{1.404992in}}%
\pgfpathlineto{\pgfqpoint{0.486008in}{1.393700in}}%
\pgfpathlineto{\pgfqpoint{0.499634in}{1.368954in}}%
\pgfpathlineto{\pgfqpoint{0.496557in}{1.356685in}}%
\pgfpathlineto{\pgfqpoint{0.492171in}{1.356794in}}%
\pgfpathlineto{\pgfqpoint{0.488481in}{1.345190in}}%
\pgfpathlineto{\pgfqpoint{0.493971in}{1.332372in}}%
\pgfpathlineto{\pgfqpoint{0.498712in}{1.326448in}}%
\pgfpathlineto{\pgfqpoint{0.503087in}{1.324821in}}%
\pgfpathlineto{\pgfqpoint{0.508309in}{1.308909in}}%
\pgfpathclose%
\pgfusepath{fill}%
\end{pgfscope}%
\begin{pgfscope}%
\pgfpathrectangle{\pgfqpoint{0.100000in}{0.100000in}}{\pgfqpoint{3.608454in}{2.310000in}}%
\pgfusepath{clip}%
\pgfsetbuttcap%
\pgfsetmiterjoin%
\definecolor{currentfill}{rgb}{0.000000,0.364706,0.817647}%
\pgfsetfillcolor{currentfill}%
\pgfsetlinewidth{0.000000pt}%
\definecolor{currentstroke}{rgb}{0.000000,0.000000,0.000000}%
\pgfsetstrokecolor{currentstroke}%
\pgfsetstrokeopacity{0.000000}%
\pgfsetdash{}{0pt}%
\pgfpathmoveto{\pgfqpoint{1.082084in}{0.210879in}}%
\pgfpathlineto{\pgfqpoint{1.080487in}{0.211762in}}%
\pgfpathlineto{\pgfqpoint{1.079012in}{0.214678in}}%
\pgfpathlineto{\pgfqpoint{1.073207in}{0.214262in}}%
\pgfpathlineto{\pgfqpoint{1.071517in}{0.212096in}}%
\pgfpathlineto{\pgfqpoint{1.069651in}{0.211374in}}%
\pgfpathlineto{\pgfqpoint{1.067915in}{0.213771in}}%
\pgfpathlineto{\pgfqpoint{1.064056in}{0.213557in}}%
\pgfpathlineto{\pgfqpoint{1.066029in}{0.216398in}}%
\pgfpathlineto{\pgfqpoint{1.068443in}{0.216507in}}%
\pgfpathlineto{\pgfqpoint{1.067670in}{0.218823in}}%
\pgfpathlineto{\pgfqpoint{1.067684in}{0.221839in}}%
\pgfpathlineto{\pgfqpoint{1.068559in}{0.223929in}}%
\pgfpathlineto{\pgfqpoint{1.067133in}{0.226051in}}%
\pgfpathlineto{\pgfqpoint{1.068011in}{0.227410in}}%
\pgfpathlineto{\pgfqpoint{1.070899in}{0.228330in}}%
\pgfpathlineto{\pgfqpoint{1.071501in}{0.231767in}}%
\pgfpathlineto{\pgfqpoint{1.068523in}{0.231093in}}%
\pgfpathlineto{\pgfqpoint{1.068667in}{0.235893in}}%
\pgfpathlineto{\pgfqpoint{1.067104in}{0.238920in}}%
\pgfpathlineto{\pgfqpoint{1.067789in}{0.242382in}}%
\pgfpathlineto{\pgfqpoint{1.067367in}{0.244154in}}%
\pgfpathlineto{\pgfqpoint{1.065528in}{0.244746in}}%
\pgfpathlineto{\pgfqpoint{1.061828in}{0.247605in}}%
\pgfpathlineto{\pgfqpoint{1.061314in}{0.250040in}}%
\pgfpathlineto{\pgfqpoint{1.062351in}{0.251212in}}%
\pgfpathlineto{\pgfqpoint{1.061213in}{0.253041in}}%
\pgfpathlineto{\pgfqpoint{1.062006in}{0.255168in}}%
\pgfpathlineto{\pgfqpoint{1.061710in}{0.260818in}}%
\pgfpathlineto{\pgfqpoint{1.062710in}{0.261351in}}%
\pgfpathlineto{\pgfqpoint{1.063933in}{0.265257in}}%
\pgfpathlineto{\pgfqpoint{1.061849in}{0.265034in}}%
\pgfpathlineto{\pgfqpoint{1.062080in}{0.267893in}}%
\pgfpathlineto{\pgfqpoint{1.064059in}{0.276502in}}%
\pgfpathlineto{\pgfqpoint{1.064827in}{0.282112in}}%
\pgfpathlineto{\pgfqpoint{1.062541in}{0.281407in}}%
\pgfpathlineto{\pgfqpoint{1.062850in}{0.278917in}}%
\pgfpathlineto{\pgfqpoint{1.061375in}{0.278552in}}%
\pgfpathlineto{\pgfqpoint{1.061437in}{0.275880in}}%
\pgfpathlineto{\pgfqpoint{1.059564in}{0.269832in}}%
\pgfpathlineto{\pgfqpoint{1.059156in}{0.263608in}}%
\pgfpathlineto{\pgfqpoint{1.057760in}{0.259141in}}%
\pgfpathlineto{\pgfqpoint{1.056698in}{0.253549in}}%
\pgfpathlineto{\pgfqpoint{1.054233in}{0.250993in}}%
\pgfpathlineto{\pgfqpoint{1.052026in}{0.254747in}}%
\pgfpathlineto{\pgfqpoint{1.052338in}{0.257691in}}%
\pgfpathlineto{\pgfqpoint{1.050359in}{0.258935in}}%
\pgfpathlineto{\pgfqpoint{1.045539in}{0.260432in}}%
\pgfpathlineto{\pgfqpoint{1.048292in}{0.263786in}}%
\pgfpathlineto{\pgfqpoint{1.048887in}{0.269049in}}%
\pgfpathlineto{\pgfqpoint{1.048759in}{0.271883in}}%
\pgfpathlineto{\pgfqpoint{1.045964in}{0.271941in}}%
\pgfpathlineto{\pgfqpoint{1.044789in}{0.274978in}}%
\pgfpathlineto{\pgfqpoint{1.043552in}{0.275786in}}%
\pgfpathlineto{\pgfqpoint{1.044076in}{0.279452in}}%
\pgfpathlineto{\pgfqpoint{1.040938in}{0.280527in}}%
\pgfpathlineto{\pgfqpoint{1.038980in}{0.283005in}}%
\pgfpathlineto{\pgfqpoint{1.037883in}{0.281325in}}%
\pgfpathlineto{\pgfqpoint{1.040576in}{0.278511in}}%
\pgfpathlineto{\pgfqpoint{1.040732in}{0.274755in}}%
\pgfpathlineto{\pgfqpoint{1.043542in}{0.270946in}}%
\pgfpathlineto{\pgfqpoint{1.040999in}{0.270087in}}%
\pgfpathlineto{\pgfqpoint{1.042205in}{0.268402in}}%
\pgfpathlineto{\pgfqpoint{1.044236in}{0.268878in}}%
\pgfpathlineto{\pgfqpoint{1.043891in}{0.261248in}}%
\pgfpathlineto{\pgfqpoint{1.039661in}{0.261025in}}%
\pgfpathlineto{\pgfqpoint{1.040400in}{0.263502in}}%
\pgfpathlineto{\pgfqpoint{1.037395in}{0.262751in}}%
\pgfpathlineto{\pgfqpoint{1.033326in}{0.260670in}}%
\pgfpathlineto{\pgfqpoint{1.033334in}{0.264446in}}%
\pgfpathlineto{\pgfqpoint{1.032173in}{0.267309in}}%
\pgfpathlineto{\pgfqpoint{1.029605in}{0.268061in}}%
\pgfpathlineto{\pgfqpoint{1.025973in}{0.276463in}}%
\pgfpathlineto{\pgfqpoint{1.026377in}{0.278219in}}%
\pgfpathlineto{\pgfqpoint{1.024142in}{0.282697in}}%
\pgfpathlineto{\pgfqpoint{1.024419in}{0.288147in}}%
\pgfpathlineto{\pgfqpoint{1.022440in}{0.293955in}}%
\pgfpathlineto{\pgfqpoint{1.018989in}{0.298401in}}%
\pgfpathlineto{\pgfqpoint{1.017453in}{0.301716in}}%
\pgfpathlineto{\pgfqpoint{1.012827in}{0.307991in}}%
\pgfpathlineto{\pgfqpoint{1.008318in}{0.316535in}}%
\pgfpathlineto{\pgfqpoint{1.011237in}{0.316069in}}%
\pgfpathlineto{\pgfqpoint{1.015372in}{0.318061in}}%
\pgfpathlineto{\pgfqpoint{1.015809in}{0.323540in}}%
\pgfpathlineto{\pgfqpoint{1.016829in}{0.325554in}}%
\pgfpathlineto{\pgfqpoint{1.011575in}{0.323679in}}%
\pgfpathlineto{\pgfqpoint{1.009826in}{0.326603in}}%
\pgfpathlineto{\pgfqpoint{1.006161in}{0.325696in}}%
\pgfpathlineto{\pgfqpoint{1.005156in}{0.323648in}}%
\pgfpathlineto{\pgfqpoint{1.000051in}{0.326986in}}%
\pgfpathlineto{\pgfqpoint{0.997055in}{0.330455in}}%
\pgfpathlineto{\pgfqpoint{1.004608in}{0.344079in}}%
\pgfpathlineto{\pgfqpoint{1.009161in}{0.338796in}}%
\pgfpathlineto{\pgfqpoint{1.012055in}{0.339391in}}%
\pgfpathlineto{\pgfqpoint{1.015589in}{0.334105in}}%
\pgfpathlineto{\pgfqpoint{1.021713in}{0.335942in}}%
\pgfpathlineto{\pgfqpoint{1.027923in}{0.333090in}}%
\pgfpathlineto{\pgfqpoint{1.029124in}{0.331408in}}%
\pgfpathlineto{\pgfqpoint{1.024345in}{0.326303in}}%
\pgfpathlineto{\pgfqpoint{1.025043in}{0.322901in}}%
\pgfpathlineto{\pgfqpoint{1.028234in}{0.318423in}}%
\pgfpathlineto{\pgfqpoint{1.027542in}{0.314145in}}%
\pgfpathlineto{\pgfqpoint{1.033899in}{0.293981in}}%
\pgfpathlineto{\pgfqpoint{1.032438in}{0.287802in}}%
\pgfpathlineto{\pgfqpoint{1.030648in}{0.284485in}}%
\pgfpathlineto{\pgfqpoint{1.031736in}{0.284026in}}%
\pgfpathlineto{\pgfqpoint{1.035421in}{0.285214in}}%
\pgfpathlineto{\pgfqpoint{1.041713in}{0.285984in}}%
\pgfpathlineto{\pgfqpoint{1.046557in}{0.284922in}}%
\pgfpathlineto{\pgfqpoint{1.049608in}{0.287280in}}%
\pgfpathlineto{\pgfqpoint{1.051192in}{0.290700in}}%
\pgfpathlineto{\pgfqpoint{1.054086in}{0.290900in}}%
\pgfpathlineto{\pgfqpoint{1.058160in}{0.294663in}}%
\pgfpathlineto{\pgfqpoint{1.061301in}{0.293799in}}%
\pgfpathlineto{\pgfqpoint{1.068256in}{0.294295in}}%
\pgfpathlineto{\pgfqpoint{1.069607in}{0.293456in}}%
\pgfpathlineto{\pgfqpoint{1.071897in}{0.287157in}}%
\pgfpathlineto{\pgfqpoint{1.071746in}{0.284296in}}%
\pgfpathlineto{\pgfqpoint{1.069635in}{0.282181in}}%
\pgfpathlineto{\pgfqpoint{1.069594in}{0.277534in}}%
\pgfpathlineto{\pgfqpoint{1.072600in}{0.274944in}}%
\pgfpathlineto{\pgfqpoint{1.072637in}{0.271740in}}%
\pgfpathlineto{\pgfqpoint{1.074084in}{0.269420in}}%
\pgfpathlineto{\pgfqpoint{1.073539in}{0.266568in}}%
\pgfpathlineto{\pgfqpoint{1.073899in}{0.262067in}}%
\pgfpathlineto{\pgfqpoint{1.077857in}{0.256052in}}%
\pgfpathlineto{\pgfqpoint{1.078297in}{0.250259in}}%
\pgfpathlineto{\pgfqpoint{1.080003in}{0.246871in}}%
\pgfpathlineto{\pgfqpoint{1.078915in}{0.244757in}}%
\pgfpathlineto{\pgfqpoint{1.079981in}{0.235727in}}%
\pgfpathlineto{\pgfqpoint{1.079690in}{0.232569in}}%
\pgfpathlineto{\pgfqpoint{1.080751in}{0.226649in}}%
\pgfpathlineto{\pgfqpoint{1.081182in}{0.216121in}}%
\pgfpathlineto{\pgfqpoint{1.082084in}{0.210879in}}%
\pgfpathclose%
\pgfusepath{fill}%
\end{pgfscope}%
\begin{pgfscope}%
\pgfpathrectangle{\pgfqpoint{0.100000in}{0.100000in}}{\pgfqpoint{3.608454in}{2.310000in}}%
\pgfusepath{clip}%
\pgfsetbuttcap%
\pgfsetmiterjoin%
\definecolor{currentfill}{rgb}{0.000000,0.364706,0.817647}%
\pgfsetfillcolor{currentfill}%
\pgfsetlinewidth{0.000000pt}%
\definecolor{currentstroke}{rgb}{0.000000,0.000000,0.000000}%
\pgfsetstrokecolor{currentstroke}%
\pgfsetstrokeopacity{0.000000}%
\pgfsetdash{}{0pt}%
\pgfpathmoveto{\pgfqpoint{1.046772in}{0.216481in}}%
\pgfpathlineto{\pgfqpoint{1.049473in}{0.221987in}}%
\pgfpathlineto{\pgfqpoint{1.051491in}{0.223563in}}%
\pgfpathlineto{\pgfqpoint{1.053235in}{0.226047in}}%
\pgfpathlineto{\pgfqpoint{1.053669in}{0.234273in}}%
\pgfpathlineto{\pgfqpoint{1.055631in}{0.239433in}}%
\pgfpathlineto{\pgfqpoint{1.056457in}{0.243986in}}%
\pgfpathlineto{\pgfqpoint{1.058002in}{0.245736in}}%
\pgfpathlineto{\pgfqpoint{1.057331in}{0.248039in}}%
\pgfpathlineto{\pgfqpoint{1.057875in}{0.253294in}}%
\pgfpathlineto{\pgfqpoint{1.059869in}{0.251052in}}%
\pgfpathlineto{\pgfqpoint{1.059422in}{0.246810in}}%
\pgfpathlineto{\pgfqpoint{1.061246in}{0.245939in}}%
\pgfpathlineto{\pgfqpoint{1.065594in}{0.242418in}}%
\pgfpathlineto{\pgfqpoint{1.066004in}{0.236870in}}%
\pgfpathlineto{\pgfqpoint{1.065636in}{0.230146in}}%
\pgfpathlineto{\pgfqpoint{1.064005in}{0.231270in}}%
\pgfpathlineto{\pgfqpoint{1.064341in}{0.239679in}}%
\pgfpathlineto{\pgfqpoint{1.062886in}{0.239330in}}%
\pgfpathlineto{\pgfqpoint{1.060924in}{0.235715in}}%
\pgfpathlineto{\pgfqpoint{1.062789in}{0.235055in}}%
\pgfpathlineto{\pgfqpoint{1.062708in}{0.232793in}}%
\pgfpathlineto{\pgfqpoint{1.061477in}{0.229384in}}%
\pgfpathlineto{\pgfqpoint{1.062971in}{0.228484in}}%
\pgfpathlineto{\pgfqpoint{1.062440in}{0.223965in}}%
\pgfpathlineto{\pgfqpoint{1.059310in}{0.224988in}}%
\pgfpathlineto{\pgfqpoint{1.061475in}{0.221010in}}%
\pgfpathlineto{\pgfqpoint{1.059297in}{0.219771in}}%
\pgfpathlineto{\pgfqpoint{1.058337in}{0.221103in}}%
\pgfpathlineto{\pgfqpoint{1.056457in}{0.219297in}}%
\pgfpathlineto{\pgfqpoint{1.053559in}{0.218893in}}%
\pgfpathlineto{\pgfqpoint{1.050357in}{0.216700in}}%
\pgfpathlineto{\pgfqpoint{1.046772in}{0.216481in}}%
\pgfpathclose%
\pgfusepath{fill}%
\end{pgfscope}%
\begin{pgfscope}%
\pgfpathrectangle{\pgfqpoint{0.100000in}{0.100000in}}{\pgfqpoint{3.608454in}{2.310000in}}%
\pgfusepath{clip}%
\pgfsetbuttcap%
\pgfsetmiterjoin%
\definecolor{currentfill}{rgb}{0.000000,0.364706,0.817647}%
\pgfsetfillcolor{currentfill}%
\pgfsetlinewidth{0.000000pt}%
\definecolor{currentstroke}{rgb}{0.000000,0.000000,0.000000}%
\pgfsetstrokecolor{currentstroke}%
\pgfsetstrokeopacity{0.000000}%
\pgfsetdash{}{0pt}%
\pgfpathmoveto{\pgfqpoint{1.031898in}{0.249479in}}%
\pgfpathlineto{\pgfqpoint{1.031917in}{0.252366in}}%
\pgfpathlineto{\pgfqpoint{1.033185in}{0.256544in}}%
\pgfpathlineto{\pgfqpoint{1.037420in}{0.256450in}}%
\pgfpathlineto{\pgfqpoint{1.039053in}{0.255636in}}%
\pgfpathlineto{\pgfqpoint{1.040317in}{0.257461in}}%
\pgfpathlineto{\pgfqpoint{1.041757in}{0.255742in}}%
\pgfpathlineto{\pgfqpoint{1.045494in}{0.257061in}}%
\pgfpathlineto{\pgfqpoint{1.048001in}{0.253354in}}%
\pgfpathlineto{\pgfqpoint{1.048579in}{0.249693in}}%
\pgfpathlineto{\pgfqpoint{1.052991in}{0.247043in}}%
\pgfpathlineto{\pgfqpoint{1.054422in}{0.245099in}}%
\pgfpathlineto{\pgfqpoint{1.053698in}{0.242447in}}%
\pgfpathlineto{\pgfqpoint{1.052033in}{0.241302in}}%
\pgfpathlineto{\pgfqpoint{1.050384in}{0.238572in}}%
\pgfpathlineto{\pgfqpoint{1.047084in}{0.239857in}}%
\pgfpathlineto{\pgfqpoint{1.044554in}{0.244818in}}%
\pgfpathlineto{\pgfqpoint{1.042604in}{0.247304in}}%
\pgfpathlineto{\pgfqpoint{1.041099in}{0.250574in}}%
\pgfpathlineto{\pgfqpoint{1.037351in}{0.246961in}}%
\pgfpathlineto{\pgfqpoint{1.031898in}{0.249479in}}%
\pgfpathclose%
\pgfusepath{fill}%
\end{pgfscope}%
\begin{pgfscope}%
\pgfpathrectangle{\pgfqpoint{0.100000in}{0.100000in}}{\pgfqpoint{3.608454in}{2.310000in}}%
\pgfusepath{clip}%
\pgfsetbuttcap%
\pgfsetmiterjoin%
\definecolor{currentfill}{rgb}{0.000000,0.482353,0.758824}%
\pgfsetfillcolor{currentfill}%
\pgfsetlinewidth{0.000000pt}%
\definecolor{currentstroke}{rgb}{0.000000,0.000000,0.000000}%
\pgfsetstrokecolor{currentstroke}%
\pgfsetstrokeopacity{0.000000}%
\pgfsetdash{}{0pt}%
\pgfpathmoveto{\pgfqpoint{3.344220in}{1.667169in}}%
\pgfpathlineto{\pgfqpoint{3.344000in}{1.680413in}}%
\pgfpathlineto{\pgfqpoint{3.366550in}{1.688953in}}%
\pgfpathlineto{\pgfqpoint{3.369634in}{1.670838in}}%
\pgfpathlineto{\pgfqpoint{3.375111in}{1.665432in}}%
\pgfpathlineto{\pgfqpoint{3.362988in}{1.653409in}}%
\pgfpathlineto{\pgfqpoint{3.369189in}{1.645562in}}%
\pgfpathlineto{\pgfqpoint{3.372252in}{1.639364in}}%
\pgfpathlineto{\pgfqpoint{3.379684in}{1.642975in}}%
\pgfpathlineto{\pgfqpoint{3.395641in}{1.645504in}}%
\pgfpathlineto{\pgfqpoint{3.404650in}{1.652435in}}%
\pgfpathlineto{\pgfqpoint{3.420950in}{1.656635in}}%
\pgfpathlineto{\pgfqpoint{3.428750in}{1.659772in}}%
\pgfpathlineto{\pgfqpoint{3.439665in}{1.671536in}}%
\pgfpathlineto{\pgfqpoint{3.441705in}{1.675928in}}%
\pgfpathlineto{\pgfqpoint{3.448912in}{1.669829in}}%
\pgfpathlineto{\pgfqpoint{3.454605in}{1.672223in}}%
\pgfpathlineto{\pgfqpoint{3.458853in}{1.668402in}}%
\pgfpathlineto{\pgfqpoint{3.465859in}{1.676647in}}%
\pgfpathlineto{\pgfqpoint{3.471136in}{1.676378in}}%
\pgfpathlineto{\pgfqpoint{3.438928in}{1.650105in}}%
\pgfpathlineto{\pgfqpoint{3.418976in}{1.636559in}}%
\pgfpathlineto{\pgfqpoint{3.411351in}{1.630261in}}%
\pgfpathlineto{\pgfqpoint{3.399067in}{1.623101in}}%
\pgfpathlineto{\pgfqpoint{3.394098in}{1.622800in}}%
\pgfpathlineto{\pgfqpoint{3.382218in}{1.615957in}}%
\pgfpathlineto{\pgfqpoint{3.369738in}{1.613583in}}%
\pgfpathlineto{\pgfqpoint{3.361694in}{1.608438in}}%
\pgfpathlineto{\pgfqpoint{3.353119in}{1.610609in}}%
\pgfpathlineto{\pgfqpoint{3.350697in}{1.605499in}}%
\pgfpathlineto{\pgfqpoint{3.343167in}{1.599400in}}%
\pgfpathlineto{\pgfqpoint{3.344569in}{1.612129in}}%
\pgfpathlineto{\pgfqpoint{3.352075in}{1.614317in}}%
\pgfpathlineto{\pgfqpoint{3.355371in}{1.641928in}}%
\pgfpathlineto{\pgfqpoint{3.353043in}{1.654103in}}%
\pgfpathlineto{\pgfqpoint{3.346226in}{1.661401in}}%
\pgfpathlineto{\pgfqpoint{3.344220in}{1.667169in}}%
\pgfpathclose%
\pgfusepath{fill}%
\end{pgfscope}%
\begin{pgfscope}%
\pgfpathrectangle{\pgfqpoint{0.100000in}{0.100000in}}{\pgfqpoint{3.608454in}{2.310000in}}%
\pgfusepath{clip}%
\pgfsetbuttcap%
\pgfsetmiterjoin%
\definecolor{currentfill}{rgb}{0.000000,0.478431,0.760784}%
\pgfsetfillcolor{currentfill}%
\pgfsetlinewidth{0.000000pt}%
\definecolor{currentstroke}{rgb}{0.000000,0.000000,0.000000}%
\pgfsetstrokecolor{currentstroke}%
\pgfsetstrokeopacity{0.000000}%
\pgfsetdash{}{0pt}%
\pgfpathmoveto{\pgfqpoint{1.157736in}{1.964285in}}%
\pgfpathlineto{\pgfqpoint{1.146770in}{1.973621in}}%
\pgfpathlineto{\pgfqpoint{1.147664in}{1.994687in}}%
\pgfpathlineto{\pgfqpoint{1.143228in}{2.002013in}}%
\pgfpathlineto{\pgfqpoint{1.145294in}{2.011278in}}%
\pgfpathlineto{\pgfqpoint{1.148015in}{2.013748in}}%
\pgfpathlineto{\pgfqpoint{1.158552in}{2.015787in}}%
\pgfpathlineto{\pgfqpoint{1.162054in}{2.020418in}}%
\pgfpathlineto{\pgfqpoint{1.164980in}{2.032458in}}%
\pgfpathlineto{\pgfqpoint{1.160126in}{2.041022in}}%
\pgfpathlineto{\pgfqpoint{1.153421in}{2.042231in}}%
\pgfpathlineto{\pgfqpoint{1.155484in}{2.052872in}}%
\pgfpathlineto{\pgfqpoint{1.141533in}{2.055406in}}%
\pgfpathlineto{\pgfqpoint{1.146831in}{2.082471in}}%
\pgfpathlineto{\pgfqpoint{1.132903in}{2.085429in}}%
\pgfpathlineto{\pgfqpoint{1.137742in}{2.109618in}}%
\pgfpathlineto{\pgfqpoint{1.134932in}{2.119015in}}%
\pgfpathlineto{\pgfqpoint{1.135799in}{2.128987in}}%
\pgfpathlineto{\pgfqpoint{1.140741in}{2.130368in}}%
\pgfpathlineto{\pgfqpoint{1.143835in}{2.140807in}}%
\pgfpathlineto{\pgfqpoint{1.149026in}{2.145360in}}%
\pgfpathlineto{\pgfqpoint{1.150853in}{2.134585in}}%
\pgfpathlineto{\pgfqpoint{1.148672in}{2.124295in}}%
\pgfpathlineto{\pgfqpoint{1.152365in}{2.116451in}}%
\pgfpathlineto{\pgfqpoint{1.165476in}{2.115944in}}%
\pgfpathlineto{\pgfqpoint{1.173281in}{2.111946in}}%
\pgfpathlineto{\pgfqpoint{1.173331in}{2.107985in}}%
\pgfpathlineto{\pgfqpoint{1.179403in}{2.102869in}}%
\pgfpathlineto{\pgfqpoint{1.191810in}{2.101317in}}%
\pgfpathlineto{\pgfqpoint{1.187119in}{2.075916in}}%
\pgfpathlineto{\pgfqpoint{1.192015in}{2.070842in}}%
\pgfpathlineto{\pgfqpoint{1.199816in}{2.068342in}}%
\pgfpathlineto{\pgfqpoint{1.196622in}{2.051472in}}%
\pgfpathlineto{\pgfqpoint{1.203686in}{2.050112in}}%
\pgfpathlineto{\pgfqpoint{1.203339in}{2.044520in}}%
\pgfpathlineto{\pgfqpoint{1.210034in}{2.040419in}}%
\pgfpathlineto{\pgfqpoint{1.213019in}{2.026776in}}%
\pgfpathlineto{\pgfqpoint{1.216917in}{2.019801in}}%
\pgfpathlineto{\pgfqpoint{1.219896in}{2.007840in}}%
\pgfpathlineto{\pgfqpoint{1.224493in}{2.008821in}}%
\pgfpathlineto{\pgfqpoint{1.228711in}{2.003790in}}%
\pgfpathlineto{\pgfqpoint{1.224264in}{1.998009in}}%
\pgfpathlineto{\pgfqpoint{1.225529in}{1.987742in}}%
\pgfpathlineto{\pgfqpoint{1.210052in}{1.990143in}}%
\pgfpathlineto{\pgfqpoint{1.208958in}{1.985065in}}%
\pgfpathlineto{\pgfqpoint{1.203995in}{1.980934in}}%
\pgfpathlineto{\pgfqpoint{1.199884in}{1.972842in}}%
\pgfpathlineto{\pgfqpoint{1.192609in}{1.969852in}}%
\pgfpathlineto{\pgfqpoint{1.185357in}{1.963304in}}%
\pgfpathlineto{\pgfqpoint{1.176413in}{1.966592in}}%
\pgfpathlineto{\pgfqpoint{1.173618in}{1.970061in}}%
\pgfpathlineto{\pgfqpoint{1.165857in}{1.971182in}}%
\pgfpathlineto{\pgfqpoint{1.157736in}{1.964285in}}%
\pgfpathclose%
\pgfusepath{fill}%
\end{pgfscope}%
\begin{pgfscope}%
\pgfpathrectangle{\pgfqpoint{0.100000in}{0.100000in}}{\pgfqpoint{3.608454in}{2.310000in}}%
\pgfusepath{clip}%
\pgfsetbuttcap%
\pgfsetmiterjoin%
\definecolor{currentfill}{rgb}{0.000000,0.372549,0.813725}%
\pgfsetfillcolor{currentfill}%
\pgfsetlinewidth{0.000000pt}%
\definecolor{currentstroke}{rgb}{0.000000,0.000000,0.000000}%
\pgfsetstrokecolor{currentstroke}%
\pgfsetstrokeopacity{0.000000}%
\pgfsetdash{}{0pt}%
\pgfpathmoveto{\pgfqpoint{1.964117in}{1.448246in}}%
\pgfpathlineto{\pgfqpoint{1.936670in}{1.449037in}}%
\pgfpathlineto{\pgfqpoint{1.937428in}{1.476563in}}%
\pgfpathlineto{\pgfqpoint{1.938437in}{1.514106in}}%
\pgfpathlineto{\pgfqpoint{1.942299in}{1.518835in}}%
\pgfpathlineto{\pgfqpoint{1.952480in}{1.526297in}}%
\pgfpathlineto{\pgfqpoint{1.966281in}{1.530825in}}%
\pgfpathlineto{\pgfqpoint{1.965564in}{1.503301in}}%
\pgfpathlineto{\pgfqpoint{1.964117in}{1.448246in}}%
\pgfpathclose%
\pgfusepath{fill}%
\end{pgfscope}%
\begin{pgfscope}%
\pgfpathrectangle{\pgfqpoint{0.100000in}{0.100000in}}{\pgfqpoint{3.608454in}{2.310000in}}%
\pgfusepath{clip}%
\pgfsetbuttcap%
\pgfsetmiterjoin%
\definecolor{currentfill}{rgb}{0.000000,0.678431,0.660784}%
\pgfsetfillcolor{currentfill}%
\pgfsetlinewidth{0.000000pt}%
\definecolor{currentstroke}{rgb}{0.000000,0.000000,0.000000}%
\pgfsetstrokecolor{currentstroke}%
\pgfsetstrokeopacity{0.000000}%
\pgfsetdash{}{0pt}%
\pgfpathmoveto{\pgfqpoint{0.737189in}{2.001359in}}%
\pgfpathlineto{\pgfqpoint{0.734618in}{1.987950in}}%
\pgfpathlineto{\pgfqpoint{0.723444in}{1.948508in}}%
\pgfpathlineto{\pgfqpoint{0.710132in}{1.952205in}}%
\pgfpathlineto{\pgfqpoint{0.711901in}{1.958542in}}%
\pgfpathlineto{\pgfqpoint{0.706253in}{1.964177in}}%
\pgfpathlineto{\pgfqpoint{0.686387in}{1.969750in}}%
\pgfpathlineto{\pgfqpoint{0.693514in}{1.994463in}}%
\pgfpathlineto{\pgfqpoint{0.695564in}{1.998853in}}%
\pgfpathlineto{\pgfqpoint{0.692032in}{2.003466in}}%
\pgfpathlineto{\pgfqpoint{0.692055in}{2.010580in}}%
\pgfpathlineto{\pgfqpoint{0.697834in}{2.026472in}}%
\pgfpathlineto{\pgfqpoint{0.695191in}{2.034615in}}%
\pgfpathlineto{\pgfqpoint{0.702993in}{2.049946in}}%
\pgfpathlineto{\pgfqpoint{0.710674in}{2.050338in}}%
\pgfpathlineto{\pgfqpoint{0.710428in}{2.058581in}}%
\pgfpathlineto{\pgfqpoint{0.707552in}{2.064386in}}%
\pgfpathlineto{\pgfqpoint{0.713071in}{2.066429in}}%
\pgfpathlineto{\pgfqpoint{0.723978in}{2.065521in}}%
\pgfpathlineto{\pgfqpoint{0.737141in}{2.068948in}}%
\pgfpathlineto{\pgfqpoint{0.724757in}{2.026561in}}%
\pgfpathlineto{\pgfqpoint{0.731442in}{2.024554in}}%
\pgfpathlineto{\pgfqpoint{0.729534in}{2.017963in}}%
\pgfpathlineto{\pgfqpoint{0.736122in}{2.015946in}}%
\pgfpathlineto{\pgfqpoint{0.730749in}{2.003174in}}%
\pgfpathlineto{\pgfqpoint{0.737189in}{2.001359in}}%
\pgfpathclose%
\pgfusepath{fill}%
\end{pgfscope}%
\begin{pgfscope}%
\pgfpathrectangle{\pgfqpoint{0.100000in}{0.100000in}}{\pgfqpoint{3.608454in}{2.310000in}}%
\pgfusepath{clip}%
\pgfsetbuttcap%
\pgfsetmiterjoin%
\definecolor{currentfill}{rgb}{0.000000,0.521569,0.739216}%
\pgfsetfillcolor{currentfill}%
\pgfsetlinewidth{0.000000pt}%
\definecolor{currentstroke}{rgb}{0.000000,0.000000,0.000000}%
\pgfsetstrokecolor{currentstroke}%
\pgfsetstrokeopacity{0.000000}%
\pgfsetdash{}{0pt}%
\pgfpathmoveto{\pgfqpoint{1.792481in}{1.427797in}}%
\pgfpathlineto{\pgfqpoint{1.724173in}{1.432221in}}%
\pgfpathlineto{\pgfqpoint{1.726124in}{1.459646in}}%
\pgfpathlineto{\pgfqpoint{1.682362in}{1.462771in}}%
\pgfpathlineto{\pgfqpoint{1.684512in}{1.490228in}}%
\pgfpathlineto{\pgfqpoint{1.732544in}{1.486727in}}%
\pgfpathlineto{\pgfqpoint{1.760676in}{1.485050in}}%
\pgfpathlineto{\pgfqpoint{1.758946in}{1.457374in}}%
\pgfpathlineto{\pgfqpoint{1.793800in}{1.455314in}}%
\pgfpathlineto{\pgfqpoint{1.792481in}{1.427797in}}%
\pgfpathclose%
\pgfusepath{fill}%
\end{pgfscope}%
\begin{pgfscope}%
\pgfpathrectangle{\pgfqpoint{0.100000in}{0.100000in}}{\pgfqpoint{3.608454in}{2.310000in}}%
\pgfusepath{clip}%
\pgfsetbuttcap%
\pgfsetmiterjoin%
\definecolor{currentfill}{rgb}{0.000000,0.505882,0.747059}%
\pgfsetfillcolor{currentfill}%
\pgfsetlinewidth{0.000000pt}%
\definecolor{currentstroke}{rgb}{0.000000,0.000000,0.000000}%
\pgfsetstrokecolor{currentstroke}%
\pgfsetstrokeopacity{0.000000}%
\pgfsetdash{}{0pt}%
\pgfpathmoveto{\pgfqpoint{1.223167in}{1.249876in}}%
\pgfpathlineto{\pgfqpoint{1.163571in}{1.259392in}}%
\pgfpathlineto{\pgfqpoint{1.116943in}{1.267777in}}%
\pgfpathlineto{\pgfqpoint{1.078577in}{1.274542in}}%
\pgfpathlineto{\pgfqpoint{1.087638in}{1.275829in}}%
\pgfpathlineto{\pgfqpoint{1.091741in}{1.280236in}}%
\pgfpathlineto{\pgfqpoint{1.099188in}{1.278642in}}%
\pgfpathlineto{\pgfqpoint{1.109948in}{1.282813in}}%
\pgfpathlineto{\pgfqpoint{1.113627in}{1.291412in}}%
\pgfpathlineto{\pgfqpoint{1.122123in}{1.294263in}}%
\pgfpathlineto{\pgfqpoint{1.130720in}{1.308550in}}%
\pgfpathlineto{\pgfqpoint{1.134341in}{1.312779in}}%
\pgfpathlineto{\pgfqpoint{1.141396in}{1.315621in}}%
\pgfpathlineto{\pgfqpoint{1.150177in}{1.332380in}}%
\pgfpathlineto{\pgfqpoint{1.159137in}{1.332405in}}%
\pgfpathlineto{\pgfqpoint{1.168304in}{1.339366in}}%
\pgfpathlineto{\pgfqpoint{1.170873in}{1.338075in}}%
\pgfpathlineto{\pgfqpoint{1.177278in}{1.346155in}}%
\pgfpathlineto{\pgfqpoint{1.183026in}{1.348649in}}%
\pgfpathlineto{\pgfqpoint{1.186021in}{1.352257in}}%
\pgfpathlineto{\pgfqpoint{1.183516in}{1.359664in}}%
\pgfpathlineto{\pgfqpoint{1.181575in}{1.376785in}}%
\pgfpathlineto{\pgfqpoint{1.240533in}{1.367269in}}%
\pgfpathlineto{\pgfqpoint{1.237792in}{1.349736in}}%
\pgfpathlineto{\pgfqpoint{1.237566in}{1.340907in}}%
\pgfpathlineto{\pgfqpoint{1.234133in}{1.318758in}}%
\pgfpathlineto{\pgfqpoint{1.248158in}{1.317741in}}%
\pgfpathlineto{\pgfqpoint{1.282717in}{1.312481in}}%
\pgfpathlineto{\pgfqpoint{1.284736in}{1.306313in}}%
\pgfpathlineto{\pgfqpoint{1.298775in}{1.307121in}}%
\pgfpathlineto{\pgfqpoint{1.305818in}{1.299625in}}%
\pgfpathlineto{\pgfqpoint{1.303897in}{1.295339in}}%
\pgfpathlineto{\pgfqpoint{1.293840in}{1.286121in}}%
\pgfpathlineto{\pgfqpoint{1.291651in}{1.276850in}}%
\pgfpathlineto{\pgfqpoint{1.286167in}{1.271003in}}%
\pgfpathlineto{\pgfqpoint{1.280183in}{1.269745in}}%
\pgfpathlineto{\pgfqpoint{1.272888in}{1.260292in}}%
\pgfpathlineto{\pgfqpoint{1.268205in}{1.247503in}}%
\pgfpathlineto{\pgfqpoint{1.264762in}{1.243586in}}%
\pgfpathlineto{\pgfqpoint{1.223167in}{1.249876in}}%
\pgfpathclose%
\pgfusepath{fill}%
\end{pgfscope}%
\begin{pgfscope}%
\pgfpathrectangle{\pgfqpoint{0.100000in}{0.100000in}}{\pgfqpoint{3.608454in}{2.310000in}}%
\pgfusepath{clip}%
\pgfsetbuttcap%
\pgfsetmiterjoin%
\definecolor{currentfill}{rgb}{0.000000,0.423529,0.788235}%
\pgfsetfillcolor{currentfill}%
\pgfsetlinewidth{0.000000pt}%
\definecolor{currentstroke}{rgb}{0.000000,0.000000,0.000000}%
\pgfsetstrokecolor{currentstroke}%
\pgfsetstrokeopacity{0.000000}%
\pgfsetdash{}{0pt}%
\pgfpathmoveto{\pgfqpoint{2.669947in}{1.049451in}}%
\pgfpathlineto{\pgfqpoint{2.668245in}{1.039940in}}%
\pgfpathlineto{\pgfqpoint{2.670176in}{1.036633in}}%
\pgfpathlineto{\pgfqpoint{2.669558in}{1.020872in}}%
\pgfpathlineto{\pgfqpoint{2.673359in}{1.013797in}}%
\pgfpathlineto{\pgfqpoint{2.666467in}{1.008380in}}%
\pgfpathlineto{\pgfqpoint{2.659190in}{1.009555in}}%
\pgfpathlineto{\pgfqpoint{2.657488in}{0.993619in}}%
\pgfpathlineto{\pgfqpoint{2.622963in}{0.991238in}}%
\pgfpathlineto{\pgfqpoint{2.623076in}{0.990093in}}%
\pgfpathlineto{\pgfqpoint{2.595636in}{0.988190in}}%
\pgfpathlineto{\pgfqpoint{2.593933in}{1.008935in}}%
\pgfpathlineto{\pgfqpoint{2.599703in}{1.020694in}}%
\pgfpathlineto{\pgfqpoint{2.599104in}{1.027892in}}%
\pgfpathlineto{\pgfqpoint{2.607222in}{1.025513in}}%
\pgfpathlineto{\pgfqpoint{2.612717in}{1.030317in}}%
\pgfpathlineto{\pgfqpoint{2.611745in}{1.044790in}}%
\pgfpathlineto{\pgfqpoint{2.636003in}{1.046348in}}%
\pgfpathlineto{\pgfqpoint{2.634590in}{1.067850in}}%
\pgfpathlineto{\pgfqpoint{2.639911in}{1.067474in}}%
\pgfpathlineto{\pgfqpoint{2.648663in}{1.074175in}}%
\pgfpathlineto{\pgfqpoint{2.648588in}{1.077186in}}%
\pgfpathlineto{\pgfqpoint{2.662888in}{1.068219in}}%
\pgfpathlineto{\pgfqpoint{2.665061in}{1.060409in}}%
\pgfpathlineto{\pgfqpoint{2.668473in}{1.060141in}}%
\pgfpathlineto{\pgfqpoint{2.669947in}{1.049451in}}%
\pgfpathclose%
\pgfusepath{fill}%
\end{pgfscope}%
\begin{pgfscope}%
\pgfpathrectangle{\pgfqpoint{0.100000in}{0.100000in}}{\pgfqpoint{3.608454in}{2.310000in}}%
\pgfusepath{clip}%
\pgfsetbuttcap%
\pgfsetmiterjoin%
\definecolor{currentfill}{rgb}{0.000000,0.349020,0.825490}%
\pgfsetfillcolor{currentfill}%
\pgfsetlinewidth{0.000000pt}%
\definecolor{currentstroke}{rgb}{0.000000,0.000000,0.000000}%
\pgfsetstrokecolor{currentstroke}%
\pgfsetstrokeopacity{0.000000}%
\pgfsetdash{}{0pt}%
\pgfpathmoveto{\pgfqpoint{1.255196in}{1.457818in}}%
\pgfpathlineto{\pgfqpoint{1.267460in}{1.535366in}}%
\pgfpathlineto{\pgfqpoint{1.271578in}{1.562500in}}%
\pgfpathlineto{\pgfqpoint{1.302483in}{1.557690in}}%
\pgfpathlineto{\pgfqpoint{1.338390in}{1.552623in}}%
\pgfpathlineto{\pgfqpoint{1.401111in}{1.544070in}}%
\pgfpathlineto{\pgfqpoint{1.400660in}{1.537925in}}%
\pgfpathlineto{\pgfqpoint{1.411585in}{1.530462in}}%
\pgfpathlineto{\pgfqpoint{1.412111in}{1.525836in}}%
\pgfpathlineto{\pgfqpoint{1.405998in}{1.509570in}}%
\pgfpathlineto{\pgfqpoint{1.407647in}{1.498770in}}%
\pgfpathlineto{\pgfqpoint{1.407785in}{1.490488in}}%
\pgfpathlineto{\pgfqpoint{1.404073in}{1.463909in}}%
\pgfpathlineto{\pgfqpoint{1.403922in}{1.457766in}}%
\pgfpathlineto{\pgfqpoint{1.379506in}{1.460495in}}%
\pgfpathlineto{\pgfqpoint{1.380971in}{1.474066in}}%
\pgfpathlineto{\pgfqpoint{1.364204in}{1.476264in}}%
\pgfpathlineto{\pgfqpoint{1.362220in}{1.462510in}}%
\pgfpathlineto{\pgfqpoint{1.355468in}{1.463655in}}%
\pgfpathlineto{\pgfqpoint{1.354499in}{1.456659in}}%
\pgfpathlineto{\pgfqpoint{1.324000in}{1.460674in}}%
\pgfpathlineto{\pgfqpoint{1.322533in}{1.450461in}}%
\pgfpathlineto{\pgfqpoint{1.285446in}{1.455840in}}%
\pgfpathlineto{\pgfqpoint{1.284941in}{1.452441in}}%
\pgfpathlineto{\pgfqpoint{1.255196in}{1.457818in}}%
\pgfpathclose%
\pgfusepath{fill}%
\end{pgfscope}%
\begin{pgfscope}%
\pgfpathrectangle{\pgfqpoint{0.100000in}{0.100000in}}{\pgfqpoint{3.608454in}{2.310000in}}%
\pgfusepath{clip}%
\pgfsetbuttcap%
\pgfsetmiterjoin%
\definecolor{currentfill}{rgb}{0.000000,0.717647,0.641176}%
\pgfsetfillcolor{currentfill}%
\pgfsetlinewidth{0.000000pt}%
\definecolor{currentstroke}{rgb}{0.000000,0.000000,0.000000}%
\pgfsetstrokecolor{currentstroke}%
\pgfsetstrokeopacity{0.000000}%
\pgfsetdash{}{0pt}%
\pgfpathmoveto{\pgfqpoint{1.709808in}{0.564368in}}%
\pgfpathlineto{\pgfqpoint{1.748728in}{0.562219in}}%
\pgfpathlineto{\pgfqpoint{1.744553in}{0.490983in}}%
\pgfpathlineto{\pgfqpoint{1.737580in}{0.491300in}}%
\pgfpathlineto{\pgfqpoint{1.732274in}{0.498247in}}%
\pgfpathlineto{\pgfqpoint{1.729908in}{0.510537in}}%
\pgfpathlineto{\pgfqpoint{1.725203in}{0.520544in}}%
\pgfpathlineto{\pgfqpoint{1.726134in}{0.523344in}}%
\pgfpathlineto{\pgfqpoint{1.719531in}{0.529734in}}%
\pgfpathlineto{\pgfqpoint{1.716987in}{0.543109in}}%
\pgfpathlineto{\pgfqpoint{1.710227in}{0.553019in}}%
\pgfpathlineto{\pgfqpoint{1.709808in}{0.564368in}}%
\pgfpathclose%
\pgfusepath{fill}%
\end{pgfscope}%
\begin{pgfscope}%
\pgfpathrectangle{\pgfqpoint{0.100000in}{0.100000in}}{\pgfqpoint{3.608454in}{2.310000in}}%
\pgfusepath{clip}%
\pgfsetbuttcap%
\pgfsetmiterjoin%
\definecolor{currentfill}{rgb}{0.000000,0.631373,0.684314}%
\pgfsetfillcolor{currentfill}%
\pgfsetlinewidth{0.000000pt}%
\definecolor{currentstroke}{rgb}{0.000000,0.000000,0.000000}%
\pgfsetstrokecolor{currentstroke}%
\pgfsetstrokeopacity{0.000000}%
\pgfsetdash{}{0pt}%
\pgfpathmoveto{\pgfqpoint{2.111050in}{0.772661in}}%
\pgfpathlineto{\pgfqpoint{2.108517in}{0.767596in}}%
\pgfpathlineto{\pgfqpoint{2.111897in}{0.751748in}}%
\pgfpathlineto{\pgfqpoint{2.115910in}{0.747279in}}%
\pgfpathlineto{\pgfqpoint{2.106573in}{0.739232in}}%
\pgfpathlineto{\pgfqpoint{2.099586in}{0.743769in}}%
\pgfpathlineto{\pgfqpoint{2.096621in}{0.750671in}}%
\pgfpathlineto{\pgfqpoint{2.088232in}{0.752553in}}%
\pgfpathlineto{\pgfqpoint{2.062519in}{0.748614in}}%
\pgfpathlineto{\pgfqpoint{2.056507in}{0.745639in}}%
\pgfpathlineto{\pgfqpoint{2.058523in}{0.753655in}}%
\pgfpathlineto{\pgfqpoint{2.048151in}{0.760702in}}%
\pgfpathlineto{\pgfqpoint{2.047446in}{0.765794in}}%
\pgfpathlineto{\pgfqpoint{2.042310in}{0.767446in}}%
\pgfpathlineto{\pgfqpoint{2.034850in}{0.788420in}}%
\pgfpathlineto{\pgfqpoint{2.023652in}{0.805506in}}%
\pgfpathlineto{\pgfqpoint{2.013777in}{0.811827in}}%
\pgfpathlineto{\pgfqpoint{2.034479in}{0.813847in}}%
\pgfpathlineto{\pgfqpoint{2.034871in}{0.852102in}}%
\pgfpathlineto{\pgfqpoint{2.044379in}{0.851895in}}%
\pgfpathlineto{\pgfqpoint{2.050385in}{0.846426in}}%
\pgfpathlineto{\pgfqpoint{2.053670in}{0.846942in}}%
\pgfpathlineto{\pgfqpoint{2.064298in}{0.842428in}}%
\pgfpathlineto{\pgfqpoint{2.062420in}{0.861532in}}%
\pgfpathlineto{\pgfqpoint{2.086189in}{0.861574in}}%
\pgfpathlineto{\pgfqpoint{2.095246in}{0.861494in}}%
\pgfpathlineto{\pgfqpoint{2.096641in}{0.856769in}}%
\pgfpathlineto{\pgfqpoint{2.096510in}{0.830344in}}%
\pgfpathlineto{\pgfqpoint{2.107625in}{0.827720in}}%
\pgfpathlineto{\pgfqpoint{2.107805in}{0.772710in}}%
\pgfpathlineto{\pgfqpoint{2.111050in}{0.772661in}}%
\pgfpathclose%
\pgfusepath{fill}%
\end{pgfscope}%
\begin{pgfscope}%
\pgfpathrectangle{\pgfqpoint{0.100000in}{0.100000in}}{\pgfqpoint{3.608454in}{2.310000in}}%
\pgfusepath{clip}%
\pgfsetbuttcap%
\pgfsetmiterjoin%
\definecolor{currentfill}{rgb}{0.000000,0.600000,0.700000}%
\pgfsetfillcolor{currentfill}%
\pgfsetlinewidth{0.000000pt}%
\definecolor{currentstroke}{rgb}{0.000000,0.000000,0.000000}%
\pgfsetstrokecolor{currentstroke}%
\pgfsetstrokeopacity{0.000000}%
\pgfsetdash{}{0pt}%
\pgfpathmoveto{\pgfqpoint{0.469229in}{1.459834in}}%
\pgfpathlineto{\pgfqpoint{0.465525in}{1.465978in}}%
\pgfpathlineto{\pgfqpoint{0.471844in}{1.486991in}}%
\pgfpathlineto{\pgfqpoint{0.471692in}{1.491720in}}%
\pgfpathlineto{\pgfqpoint{0.476612in}{1.508153in}}%
\pgfpathlineto{\pgfqpoint{0.471851in}{1.509726in}}%
\pgfpathlineto{\pgfqpoint{0.471900in}{1.514999in}}%
\pgfpathlineto{\pgfqpoint{0.477138in}{1.516663in}}%
\pgfpathlineto{\pgfqpoint{0.480918in}{1.524945in}}%
\pgfpathlineto{\pgfqpoint{0.474910in}{1.526813in}}%
\pgfpathlineto{\pgfqpoint{0.479765in}{1.542960in}}%
\pgfpathlineto{\pgfqpoint{0.465266in}{1.547913in}}%
\pgfpathlineto{\pgfqpoint{0.460110in}{1.545792in}}%
\pgfpathlineto{\pgfqpoint{0.455284in}{1.549289in}}%
\pgfpathlineto{\pgfqpoint{0.454748in}{1.561199in}}%
\pgfpathlineto{\pgfqpoint{0.452551in}{1.565246in}}%
\pgfpathlineto{\pgfqpoint{0.452005in}{1.577321in}}%
\pgfpathlineto{\pgfqpoint{0.446281in}{1.581110in}}%
\pgfpathlineto{\pgfqpoint{0.450911in}{1.584637in}}%
\pgfpathlineto{\pgfqpoint{0.480562in}{1.575547in}}%
\pgfpathlineto{\pgfqpoint{0.485256in}{1.568714in}}%
\pgfpathlineto{\pgfqpoint{0.484414in}{1.564406in}}%
\pgfpathlineto{\pgfqpoint{0.489967in}{1.557121in}}%
\pgfpathlineto{\pgfqpoint{0.498193in}{1.555922in}}%
\pgfpathlineto{\pgfqpoint{0.502169in}{1.569148in}}%
\pgfpathlineto{\pgfqpoint{0.507112in}{1.573994in}}%
\pgfpathlineto{\pgfqpoint{0.514461in}{1.575916in}}%
\pgfpathlineto{\pgfqpoint{0.520214in}{1.588877in}}%
\pgfpathlineto{\pgfqpoint{0.532464in}{1.597564in}}%
\pgfpathlineto{\pgfqpoint{0.538914in}{1.596597in}}%
\pgfpathlineto{\pgfqpoint{0.556871in}{1.597127in}}%
\pgfpathlineto{\pgfqpoint{0.563417in}{1.600581in}}%
\pgfpathlineto{\pgfqpoint{0.567874in}{1.599010in}}%
\pgfpathlineto{\pgfqpoint{0.570419in}{1.591928in}}%
\pgfpathlineto{\pgfqpoint{0.599878in}{1.583444in}}%
\pgfpathlineto{\pgfqpoint{0.592721in}{1.558163in}}%
\pgfpathlineto{\pgfqpoint{0.607050in}{1.554247in}}%
\pgfpathlineto{\pgfqpoint{0.606969in}{1.551881in}}%
\pgfpathlineto{\pgfqpoint{0.632981in}{1.544578in}}%
\pgfpathlineto{\pgfqpoint{0.631100in}{1.537480in}}%
\pgfpathlineto{\pgfqpoint{0.623821in}{1.536529in}}%
\pgfpathlineto{\pgfqpoint{0.620041in}{1.519423in}}%
\pgfpathlineto{\pgfqpoint{0.623308in}{1.518183in}}%
\pgfpathlineto{\pgfqpoint{0.620520in}{1.503015in}}%
\pgfpathlineto{\pgfqpoint{0.609163in}{1.520157in}}%
\pgfpathlineto{\pgfqpoint{0.604651in}{1.513031in}}%
\pgfpathlineto{\pgfqpoint{0.606996in}{1.503863in}}%
\pgfpathlineto{\pgfqpoint{0.603382in}{1.497016in}}%
\pgfpathlineto{\pgfqpoint{0.597596in}{1.492295in}}%
\pgfpathlineto{\pgfqpoint{0.592836in}{1.501027in}}%
\pgfpathlineto{\pgfqpoint{0.583557in}{1.498623in}}%
\pgfpathlineto{\pgfqpoint{0.577147in}{1.506768in}}%
\pgfpathlineto{\pgfqpoint{0.566537in}{1.504878in}}%
\pgfpathlineto{\pgfqpoint{0.550096in}{1.491227in}}%
\pgfpathlineto{\pgfqpoint{0.537164in}{1.478720in}}%
\pgfpathlineto{\pgfqpoint{0.526181in}{1.471930in}}%
\pgfpathlineto{\pgfqpoint{0.515295in}{1.495292in}}%
\pgfpathlineto{\pgfqpoint{0.508101in}{1.469498in}}%
\pgfpathlineto{\pgfqpoint{0.504262in}{1.472446in}}%
\pgfpathlineto{\pgfqpoint{0.489989in}{1.471204in}}%
\pgfpathlineto{\pgfqpoint{0.469229in}{1.459834in}}%
\pgfpathclose%
\pgfusepath{fill}%
\end{pgfscope}%
\begin{pgfscope}%
\pgfpathrectangle{\pgfqpoint{0.100000in}{0.100000in}}{\pgfqpoint{3.608454in}{2.310000in}}%
\pgfusepath{clip}%
\pgfsetbuttcap%
\pgfsetmiterjoin%
\definecolor{currentfill}{rgb}{0.000000,0.478431,0.760784}%
\pgfsetfillcolor{currentfill}%
\pgfsetlinewidth{0.000000pt}%
\definecolor{currentstroke}{rgb}{0.000000,0.000000,0.000000}%
\pgfsetstrokecolor{currentstroke}%
\pgfsetstrokeopacity{0.000000}%
\pgfsetdash{}{0pt}%
\pgfpathmoveto{\pgfqpoint{1.886398in}{1.921692in}}%
\pgfpathlineto{\pgfqpoint{1.884519in}{1.921788in}}%
\pgfpathlineto{\pgfqpoint{1.885796in}{1.949443in}}%
\pgfpathlineto{\pgfqpoint{1.863197in}{1.950599in}}%
\pgfpathlineto{\pgfqpoint{1.864737in}{1.978421in}}%
\pgfpathlineto{\pgfqpoint{1.862851in}{1.978498in}}%
\pgfpathlineto{\pgfqpoint{1.864232in}{2.006135in}}%
\pgfpathlineto{\pgfqpoint{1.875835in}{2.005576in}}%
\pgfpathlineto{\pgfqpoint{1.917334in}{2.003750in}}%
\pgfpathlineto{\pgfqpoint{1.918791in}{1.996815in}}%
\pgfpathlineto{\pgfqpoint{1.946235in}{1.995849in}}%
\pgfpathlineto{\pgfqpoint{1.960057in}{1.995378in}}%
\pgfpathlineto{\pgfqpoint{1.960762in}{1.974477in}}%
\pgfpathlineto{\pgfqpoint{1.959926in}{1.946668in}}%
\pgfpathlineto{\pgfqpoint{1.940671in}{1.947331in}}%
\pgfpathlineto{\pgfqpoint{1.939707in}{1.919594in}}%
\pgfpathlineto{\pgfqpoint{1.886398in}{1.921692in}}%
\pgfpathclose%
\pgfusepath{fill}%
\end{pgfscope}%
\begin{pgfscope}%
\pgfpathrectangle{\pgfqpoint{0.100000in}{0.100000in}}{\pgfqpoint{3.608454in}{2.310000in}}%
\pgfusepath{clip}%
\pgfsetbuttcap%
\pgfsetmiterjoin%
\definecolor{currentfill}{rgb}{0.000000,0.529412,0.735294}%
\pgfsetfillcolor{currentfill}%
\pgfsetlinewidth{0.000000pt}%
\definecolor{currentstroke}{rgb}{0.000000,0.000000,0.000000}%
\pgfsetstrokecolor{currentstroke}%
\pgfsetstrokeopacity{0.000000}%
\pgfsetdash{}{0pt}%
\pgfpathmoveto{\pgfqpoint{3.420582in}{1.833115in}}%
\pgfpathlineto{\pgfqpoint{3.412904in}{1.835488in}}%
\pgfpathlineto{\pgfqpoint{3.401069in}{1.830482in}}%
\pgfpathlineto{\pgfqpoint{3.396386in}{1.835470in}}%
\pgfpathlineto{\pgfqpoint{3.382485in}{1.831082in}}%
\pgfpathlineto{\pgfqpoint{3.374364in}{1.831752in}}%
\pgfpathlineto{\pgfqpoint{3.370781in}{1.834556in}}%
\pgfpathlineto{\pgfqpoint{3.375120in}{1.840353in}}%
\pgfpathlineto{\pgfqpoint{3.375426in}{1.849437in}}%
\pgfpathlineto{\pgfqpoint{3.371356in}{1.850568in}}%
\pgfpathlineto{\pgfqpoint{3.372767in}{1.866271in}}%
\pgfpathlineto{\pgfqpoint{3.365489in}{1.867233in}}%
\pgfpathlineto{\pgfqpoint{3.365877in}{1.874467in}}%
\pgfpathlineto{\pgfqpoint{3.354162in}{1.876806in}}%
\pgfpathlineto{\pgfqpoint{3.358819in}{1.880708in}}%
\pgfpathlineto{\pgfqpoint{3.363839in}{1.892385in}}%
\pgfpathlineto{\pgfqpoint{3.353338in}{1.893246in}}%
\pgfpathlineto{\pgfqpoint{3.350605in}{1.900061in}}%
\pgfpathlineto{\pgfqpoint{3.349620in}{1.909475in}}%
\pgfpathlineto{\pgfqpoint{3.352226in}{1.924022in}}%
\pgfpathlineto{\pgfqpoint{3.361540in}{1.921976in}}%
\pgfpathlineto{\pgfqpoint{3.363137in}{1.929015in}}%
\pgfpathlineto{\pgfqpoint{3.370017in}{1.926361in}}%
\pgfpathlineto{\pgfqpoint{3.371539in}{1.933332in}}%
\pgfpathlineto{\pgfqpoint{3.384688in}{1.930010in}}%
\pgfpathlineto{\pgfqpoint{3.381069in}{1.911013in}}%
\pgfpathlineto{\pgfqpoint{3.384973in}{1.910295in}}%
\pgfpathlineto{\pgfqpoint{3.398006in}{1.914180in}}%
\pgfpathlineto{\pgfqpoint{3.396256in}{1.920610in}}%
\pgfpathlineto{\pgfqpoint{3.399587in}{1.926596in}}%
\pgfpathlineto{\pgfqpoint{3.405777in}{1.928235in}}%
\pgfpathlineto{\pgfqpoint{3.410255in}{1.935145in}}%
\pgfpathlineto{\pgfqpoint{3.415486in}{1.930416in}}%
\pgfpathlineto{\pgfqpoint{3.422059in}{1.929884in}}%
\pgfpathlineto{\pgfqpoint{3.423697in}{1.926204in}}%
\pgfpathlineto{\pgfqpoint{3.431085in}{1.927713in}}%
\pgfpathlineto{\pgfqpoint{3.433355in}{1.925260in}}%
\pgfpathlineto{\pgfqpoint{3.438972in}{1.915003in}}%
\pgfpathlineto{\pgfqpoint{3.442370in}{1.903310in}}%
\pgfpathlineto{\pgfqpoint{3.431626in}{1.898622in}}%
\pgfpathlineto{\pgfqpoint{3.437941in}{1.884217in}}%
\pgfpathlineto{\pgfqpoint{3.431173in}{1.881530in}}%
\pgfpathlineto{\pgfqpoint{3.432335in}{1.876511in}}%
\pgfpathlineto{\pgfqpoint{3.424616in}{1.868740in}}%
\pgfpathlineto{\pgfqpoint{3.421501in}{1.872886in}}%
\pgfpathlineto{\pgfqpoint{3.415970in}{1.857152in}}%
\pgfpathlineto{\pgfqpoint{3.413847in}{1.847236in}}%
\pgfpathlineto{\pgfqpoint{3.421042in}{1.838069in}}%
\pgfpathlineto{\pgfqpoint{3.420582in}{1.833115in}}%
\pgfpathclose%
\pgfusepath{fill}%
\end{pgfscope}%
\begin{pgfscope}%
\pgfpathrectangle{\pgfqpoint{0.100000in}{0.100000in}}{\pgfqpoint{3.608454in}{2.310000in}}%
\pgfusepath{clip}%
\pgfsetbuttcap%
\pgfsetmiterjoin%
\definecolor{currentfill}{rgb}{0.000000,0.666667,0.666667}%
\pgfsetfillcolor{currentfill}%
\pgfsetlinewidth{0.000000pt}%
\definecolor{currentstroke}{rgb}{0.000000,0.000000,0.000000}%
\pgfsetstrokecolor{currentstroke}%
\pgfsetstrokeopacity{0.000000}%
\pgfsetdash{}{0pt}%
\pgfpathmoveto{\pgfqpoint{3.043474in}{1.583306in}}%
\pgfpathlineto{\pgfqpoint{3.040035in}{1.579416in}}%
\pgfpathlineto{\pgfqpoint{3.028102in}{1.574866in}}%
\pgfpathlineto{\pgfqpoint{3.020349in}{1.574020in}}%
\pgfpathlineto{\pgfqpoint{3.013646in}{1.587974in}}%
\pgfpathlineto{\pgfqpoint{2.995768in}{1.584878in}}%
\pgfpathlineto{\pgfqpoint{2.991632in}{1.609704in}}%
\pgfpathlineto{\pgfqpoint{2.984410in}{1.607734in}}%
\pgfpathlineto{\pgfqpoint{2.961131in}{1.605506in}}%
\pgfpathlineto{\pgfqpoint{2.955120in}{1.642813in}}%
\pgfpathlineto{\pgfqpoint{2.973125in}{1.655287in}}%
\pgfpathlineto{\pgfqpoint{2.989337in}{1.668857in}}%
\pgfpathlineto{\pgfqpoint{2.999470in}{1.675879in}}%
\pgfpathlineto{\pgfqpoint{3.011435in}{1.687328in}}%
\pgfpathlineto{\pgfqpoint{3.021181in}{1.698369in}}%
\pgfpathlineto{\pgfqpoint{3.027192in}{1.702502in}}%
\pgfpathlineto{\pgfqpoint{3.032000in}{1.700788in}}%
\pgfpathlineto{\pgfqpoint{3.039413in}{1.658799in}}%
\pgfpathlineto{\pgfqpoint{3.047674in}{1.660178in}}%
\pgfpathlineto{\pgfqpoint{3.050043in}{1.647962in}}%
\pgfpathlineto{\pgfqpoint{3.048278in}{1.646809in}}%
\pgfpathlineto{\pgfqpoint{3.050721in}{1.630639in}}%
\pgfpathlineto{\pgfqpoint{3.051910in}{1.617021in}}%
\pgfpathlineto{\pgfqpoint{3.045476in}{1.613591in}}%
\pgfpathlineto{\pgfqpoint{3.046524in}{1.607153in}}%
\pgfpathlineto{\pgfqpoint{3.039983in}{1.605278in}}%
\pgfpathlineto{\pgfqpoint{3.043474in}{1.583306in}}%
\pgfpathclose%
\pgfusepath{fill}%
\end{pgfscope}%
\begin{pgfscope}%
\pgfpathrectangle{\pgfqpoint{0.100000in}{0.100000in}}{\pgfqpoint{3.608454in}{2.310000in}}%
\pgfusepath{clip}%
\pgfsetbuttcap%
\pgfsetmiterjoin%
\definecolor{currentfill}{rgb}{0.000000,0.333333,0.833333}%
\pgfsetfillcolor{currentfill}%
\pgfsetlinewidth{0.000000pt}%
\definecolor{currentstroke}{rgb}{0.000000,0.000000,0.000000}%
\pgfsetstrokecolor{currentstroke}%
\pgfsetstrokeopacity{0.000000}%
\pgfsetdash{}{0pt}%
\pgfpathmoveto{\pgfqpoint{1.992970in}{1.502588in}}%
\pgfpathlineto{\pgfqpoint{1.965564in}{1.503301in}}%
\pgfpathlineto{\pgfqpoint{1.966281in}{1.530825in}}%
\pgfpathlineto{\pgfqpoint{1.952480in}{1.526297in}}%
\pgfpathlineto{\pgfqpoint{1.952621in}{1.531206in}}%
\pgfpathlineto{\pgfqpoint{1.946376in}{1.531379in}}%
\pgfpathlineto{\pgfqpoint{1.946070in}{1.524600in}}%
\pgfpathlineto{\pgfqpoint{1.925386in}{1.523978in}}%
\pgfpathlineto{\pgfqpoint{1.922421in}{1.521780in}}%
\pgfpathlineto{\pgfqpoint{1.912100in}{1.522170in}}%
\pgfpathlineto{\pgfqpoint{1.911513in}{1.532437in}}%
\pgfpathlineto{\pgfqpoint{1.912258in}{1.559919in}}%
\pgfpathlineto{\pgfqpoint{1.912766in}{1.573654in}}%
\pgfpathlineto{\pgfqpoint{1.939982in}{1.572847in}}%
\pgfpathlineto{\pgfqpoint{1.939698in}{1.559064in}}%
\pgfpathlineto{\pgfqpoint{1.987553in}{1.557852in}}%
\pgfpathlineto{\pgfqpoint{1.994270in}{1.557682in}}%
\pgfpathlineto{\pgfqpoint{1.992970in}{1.502588in}}%
\pgfpathclose%
\pgfusepath{fill}%
\end{pgfscope}%
\begin{pgfscope}%
\pgfpathrectangle{\pgfqpoint{0.100000in}{0.100000in}}{\pgfqpoint{3.608454in}{2.310000in}}%
\pgfusepath{clip}%
\pgfsetbuttcap%
\pgfsetmiterjoin%
\definecolor{currentfill}{rgb}{0.000000,0.462745,0.768627}%
\pgfsetfillcolor{currentfill}%
\pgfsetlinewidth{0.000000pt}%
\definecolor{currentstroke}{rgb}{0.000000,0.000000,0.000000}%
\pgfsetstrokecolor{currentstroke}%
\pgfsetstrokeopacity{0.000000}%
\pgfsetdash{}{0pt}%
\pgfpathmoveto{\pgfqpoint{2.115120in}{1.723283in}}%
\pgfpathlineto{\pgfqpoint{2.114776in}{1.743905in}}%
\pgfpathlineto{\pgfqpoint{2.101083in}{1.743915in}}%
\pgfpathlineto{\pgfqpoint{2.100603in}{1.750815in}}%
\pgfpathlineto{\pgfqpoint{2.100687in}{1.757698in}}%
\pgfpathlineto{\pgfqpoint{2.114393in}{1.757738in}}%
\pgfpathlineto{\pgfqpoint{2.114392in}{1.774876in}}%
\pgfpathlineto{\pgfqpoint{2.118639in}{1.771585in}}%
\pgfpathlineto{\pgfqpoint{2.128088in}{1.771585in}}%
\pgfpathlineto{\pgfqpoint{2.167464in}{1.771934in}}%
\pgfpathlineto{\pgfqpoint{2.168433in}{1.778817in}}%
\pgfpathlineto{\pgfqpoint{2.190305in}{1.779149in}}%
\pgfpathlineto{\pgfqpoint{2.190701in}{1.751566in}}%
\pgfpathlineto{\pgfqpoint{2.176891in}{1.751349in}}%
\pgfpathlineto{\pgfqpoint{2.177278in}{1.723751in}}%
\pgfpathlineto{\pgfqpoint{2.115120in}{1.723283in}}%
\pgfpathclose%
\pgfusepath{fill}%
\end{pgfscope}%
\begin{pgfscope}%
\pgfpathrectangle{\pgfqpoint{0.100000in}{0.100000in}}{\pgfqpoint{3.608454in}{2.310000in}}%
\pgfusepath{clip}%
\pgfsetbuttcap%
\pgfsetmiterjoin%
\definecolor{currentfill}{rgb}{0.000000,0.529412,0.735294}%
\pgfsetfillcolor{currentfill}%
\pgfsetlinewidth{0.000000pt}%
\definecolor{currentstroke}{rgb}{0.000000,0.000000,0.000000}%
\pgfsetstrokecolor{currentstroke}%
\pgfsetstrokeopacity{0.000000}%
\pgfsetdash{}{0pt}%
\pgfpathmoveto{\pgfqpoint{2.248598in}{1.467244in}}%
\pgfpathlineto{\pgfqpoint{2.247888in}{1.491592in}}%
\pgfpathlineto{\pgfqpoint{2.220496in}{1.490915in}}%
\pgfpathlineto{\pgfqpoint{2.219965in}{1.511654in}}%
\pgfpathlineto{\pgfqpoint{2.206358in}{1.511376in}}%
\pgfpathlineto{\pgfqpoint{2.205803in}{1.538835in}}%
\pgfpathlineto{\pgfqpoint{2.266869in}{1.540448in}}%
\pgfpathlineto{\pgfqpoint{2.287673in}{1.541193in}}%
\pgfpathlineto{\pgfqpoint{2.288611in}{1.513740in}}%
\pgfpathlineto{\pgfqpoint{2.302293in}{1.514165in}}%
\pgfpathlineto{\pgfqpoint{2.303069in}{1.470734in}}%
\pgfpathlineto{\pgfqpoint{2.290253in}{1.469670in}}%
\pgfpathlineto{\pgfqpoint{2.248598in}{1.467244in}}%
\pgfpathclose%
\pgfusepath{fill}%
\end{pgfscope}%
\begin{pgfscope}%
\pgfpathrectangle{\pgfqpoint{0.100000in}{0.100000in}}{\pgfqpoint{3.608454in}{2.310000in}}%
\pgfusepath{clip}%
\pgfsetbuttcap%
\pgfsetmiterjoin%
\definecolor{currentfill}{rgb}{0.000000,0.627451,0.686275}%
\pgfsetfillcolor{currentfill}%
\pgfsetlinewidth{0.000000pt}%
\definecolor{currentstroke}{rgb}{0.000000,0.000000,0.000000}%
\pgfsetstrokecolor{currentstroke}%
\pgfsetstrokeopacity{0.000000}%
\pgfsetdash{}{0pt}%
\pgfpathmoveto{\pgfqpoint{1.141273in}{1.664762in}}%
\pgfpathlineto{\pgfqpoint{1.142493in}{1.660281in}}%
\pgfpathlineto{\pgfqpoint{1.139137in}{1.653332in}}%
\pgfpathlineto{\pgfqpoint{1.141060in}{1.650733in}}%
\pgfpathlineto{\pgfqpoint{1.142480in}{1.636751in}}%
\pgfpathlineto{\pgfqpoint{1.138324in}{1.627635in}}%
\pgfpathlineto{\pgfqpoint{1.132825in}{1.619941in}}%
\pgfpathlineto{\pgfqpoint{1.121923in}{1.621467in}}%
\pgfpathlineto{\pgfqpoint{1.117387in}{1.618454in}}%
\pgfpathlineto{\pgfqpoint{1.110470in}{1.623700in}}%
\pgfpathlineto{\pgfqpoint{1.101371in}{1.618838in}}%
\pgfpathlineto{\pgfqpoint{1.089153in}{1.621193in}}%
\pgfpathlineto{\pgfqpoint{1.070321in}{1.603891in}}%
\pgfpathlineto{\pgfqpoint{1.051249in}{1.601475in}}%
\pgfpathlineto{\pgfqpoint{0.978586in}{1.616384in}}%
\pgfpathlineto{\pgfqpoint{0.994964in}{1.693281in}}%
\pgfpathlineto{\pgfqpoint{1.004416in}{1.690817in}}%
\pgfpathlineto{\pgfqpoint{1.054992in}{1.681239in}}%
\pgfpathlineto{\pgfqpoint{1.060120in}{1.706792in}}%
\pgfpathlineto{\pgfqpoint{1.080051in}{1.702943in}}%
\pgfpathlineto{\pgfqpoint{1.081382in}{1.709710in}}%
\pgfpathlineto{\pgfqpoint{1.090342in}{1.707956in}}%
\pgfpathlineto{\pgfqpoint{1.091640in}{1.714726in}}%
\pgfpathlineto{\pgfqpoint{1.098147in}{1.713450in}}%
\pgfpathlineto{\pgfqpoint{1.101946in}{1.704218in}}%
\pgfpathlineto{\pgfqpoint{1.109727in}{1.693683in}}%
\pgfpathlineto{\pgfqpoint{1.117899in}{1.692151in}}%
\pgfpathlineto{\pgfqpoint{1.122487in}{1.688955in}}%
\pgfpathlineto{\pgfqpoint{1.126068in}{1.701146in}}%
\pgfpathlineto{\pgfqpoint{1.141976in}{1.698141in}}%
\pgfpathlineto{\pgfqpoint{1.143732in}{1.692681in}}%
\pgfpathlineto{\pgfqpoint{1.141935in}{1.686685in}}%
\pgfpathlineto{\pgfqpoint{1.137407in}{1.682303in}}%
\pgfpathlineto{\pgfqpoint{1.137582in}{1.673009in}}%
\pgfpathlineto{\pgfqpoint{1.141528in}{1.670743in}}%
\pgfpathlineto{\pgfqpoint{1.141273in}{1.664762in}}%
\pgfpathclose%
\pgfusepath{fill}%
\end{pgfscope}%
\begin{pgfscope}%
\pgfpathrectangle{\pgfqpoint{0.100000in}{0.100000in}}{\pgfqpoint{3.608454in}{2.310000in}}%
\pgfusepath{clip}%
\pgfsetbuttcap%
\pgfsetmiterjoin%
\definecolor{currentfill}{rgb}{0.000000,0.713725,0.643137}%
\pgfsetfillcolor{currentfill}%
\pgfsetlinewidth{0.000000pt}%
\definecolor{currentstroke}{rgb}{0.000000,0.000000,0.000000}%
\pgfsetstrokecolor{currentstroke}%
\pgfsetstrokeopacity{0.000000}%
\pgfsetdash{}{0pt}%
\pgfpathmoveto{\pgfqpoint{2.811045in}{1.319515in}}%
\pgfpathlineto{\pgfqpoint{2.804061in}{1.326162in}}%
\pgfpathlineto{\pgfqpoint{2.796848in}{1.326453in}}%
\pgfpathlineto{\pgfqpoint{2.787097in}{1.337635in}}%
\pgfpathlineto{\pgfqpoint{2.789928in}{1.341177in}}%
\pgfpathlineto{\pgfqpoint{2.786720in}{1.347970in}}%
\pgfpathlineto{\pgfqpoint{2.779558in}{1.348364in}}%
\pgfpathlineto{\pgfqpoint{2.772188in}{1.346181in}}%
\pgfpathlineto{\pgfqpoint{2.768060in}{1.364802in}}%
\pgfpathlineto{\pgfqpoint{2.777359in}{1.362093in}}%
\pgfpathlineto{\pgfqpoint{2.784131in}{1.363939in}}%
\pgfpathlineto{\pgfqpoint{2.790830in}{1.362794in}}%
\pgfpathlineto{\pgfqpoint{2.798224in}{1.355170in}}%
\pgfpathlineto{\pgfqpoint{2.804813in}{1.353816in}}%
\pgfpathlineto{\pgfqpoint{2.807405in}{1.358642in}}%
\pgfpathlineto{\pgfqpoint{2.812960in}{1.360831in}}%
\pgfpathlineto{\pgfqpoint{2.829317in}{1.356171in}}%
\pgfpathlineto{\pgfqpoint{2.836077in}{1.357238in}}%
\pgfpathlineto{\pgfqpoint{2.842704in}{1.366486in}}%
\pgfpathlineto{\pgfqpoint{2.841606in}{1.355934in}}%
\pgfpathlineto{\pgfqpoint{2.835681in}{1.347268in}}%
\pgfpathlineto{\pgfqpoint{2.833231in}{1.340587in}}%
\pgfpathlineto{\pgfqpoint{2.834430in}{1.334621in}}%
\pgfpathlineto{\pgfqpoint{2.826872in}{1.332234in}}%
\pgfpathlineto{\pgfqpoint{2.822416in}{1.337584in}}%
\pgfpathlineto{\pgfqpoint{2.817804in}{1.333230in}}%
\pgfpathlineto{\pgfqpoint{2.818045in}{1.327218in}}%
\pgfpathlineto{\pgfqpoint{2.811045in}{1.319515in}}%
\pgfpathclose%
\pgfusepath{fill}%
\end{pgfscope}%
\begin{pgfscope}%
\pgfpathrectangle{\pgfqpoint{0.100000in}{0.100000in}}{\pgfqpoint{3.608454in}{2.310000in}}%
\pgfusepath{clip}%
\pgfsetbuttcap%
\pgfsetmiterjoin%
\definecolor{currentfill}{rgb}{0.000000,0.560784,0.719608}%
\pgfsetfillcolor{currentfill}%
\pgfsetlinewidth{0.000000pt}%
\definecolor{currentstroke}{rgb}{0.000000,0.000000,0.000000}%
\pgfsetstrokecolor{currentstroke}%
\pgfsetstrokeopacity{0.000000}%
\pgfsetdash{}{0pt}%
\pgfpathmoveto{\pgfqpoint{2.839212in}{0.870674in}}%
\pgfpathlineto{\pgfqpoint{2.834772in}{0.868856in}}%
\pgfpathlineto{\pgfqpoint{2.830726in}{0.859944in}}%
\pgfpathlineto{\pgfqpoint{2.822592in}{0.856703in}}%
\pgfpathlineto{\pgfqpoint{2.816271in}{0.861716in}}%
\pgfpathlineto{\pgfqpoint{2.812799in}{0.870744in}}%
\pgfpathlineto{\pgfqpoint{2.821573in}{0.886623in}}%
\pgfpathlineto{\pgfqpoint{2.827685in}{0.882746in}}%
\pgfpathlineto{\pgfqpoint{2.830174in}{0.891948in}}%
\pgfpathlineto{\pgfqpoint{2.830655in}{0.902493in}}%
\pgfpathlineto{\pgfqpoint{2.835752in}{0.904367in}}%
\pgfpathlineto{\pgfqpoint{2.833490in}{0.924483in}}%
\pgfpathlineto{\pgfqpoint{2.844937in}{0.925826in}}%
\pgfpathlineto{\pgfqpoint{2.848183in}{0.924494in}}%
\pgfpathlineto{\pgfqpoint{2.849065in}{0.920723in}}%
\pgfpathlineto{\pgfqpoint{2.866524in}{0.926111in}}%
\pgfpathlineto{\pgfqpoint{2.874063in}{0.928161in}}%
\pgfpathlineto{\pgfqpoint{2.881402in}{0.908328in}}%
\pgfpathlineto{\pgfqpoint{2.878409in}{0.905708in}}%
\pgfpathlineto{\pgfqpoint{2.893588in}{0.882495in}}%
\pgfpathlineto{\pgfqpoint{2.901077in}{0.870542in}}%
\pgfpathlineto{\pgfqpoint{2.898491in}{0.872568in}}%
\pgfpathlineto{\pgfqpoint{2.890867in}{0.862703in}}%
\pgfpathlineto{\pgfqpoint{2.887187in}{0.854824in}}%
\pgfpathlineto{\pgfqpoint{2.893436in}{0.849951in}}%
\pgfpathlineto{\pgfqpoint{2.891100in}{0.843787in}}%
\pgfpathlineto{\pgfqpoint{2.872842in}{0.842457in}}%
\pgfpathlineto{\pgfqpoint{2.871284in}{0.855909in}}%
\pgfpathlineto{\pgfqpoint{2.855107in}{0.854082in}}%
\pgfpathlineto{\pgfqpoint{2.854444in}{0.865881in}}%
\pgfpathlineto{\pgfqpoint{2.851231in}{0.869522in}}%
\pgfpathlineto{\pgfqpoint{2.839212in}{0.870674in}}%
\pgfpathclose%
\pgfusepath{fill}%
\end{pgfscope}%
\begin{pgfscope}%
\pgfpathrectangle{\pgfqpoint{0.100000in}{0.100000in}}{\pgfqpoint{3.608454in}{2.310000in}}%
\pgfusepath{clip}%
\pgfsetbuttcap%
\pgfsetmiterjoin%
\definecolor{currentfill}{rgb}{0.000000,0.419608,0.790196}%
\pgfsetfillcolor{currentfill}%
\pgfsetlinewidth{0.000000pt}%
\definecolor{currentstroke}{rgb}{0.000000,0.000000,0.000000}%
\pgfsetstrokecolor{currentstroke}%
\pgfsetstrokeopacity{0.000000}%
\pgfsetdash{}{0pt}%
\pgfpathmoveto{\pgfqpoint{1.981983in}{1.828333in}}%
\pgfpathlineto{\pgfqpoint{1.939662in}{1.829664in}}%
\pgfpathlineto{\pgfqpoint{1.941869in}{1.891889in}}%
\pgfpathlineto{\pgfqpoint{1.983409in}{1.890641in}}%
\pgfpathlineto{\pgfqpoint{2.020229in}{1.889881in}}%
\pgfpathlineto{\pgfqpoint{2.018973in}{1.880728in}}%
\pgfpathlineto{\pgfqpoint{2.015031in}{1.874970in}}%
\pgfpathlineto{\pgfqpoint{2.004519in}{1.867267in}}%
\pgfpathlineto{\pgfqpoint{2.003389in}{1.863983in}}%
\pgfpathlineto{\pgfqpoint{2.012969in}{1.848244in}}%
\pgfpathlineto{\pgfqpoint{2.021780in}{1.845315in}}%
\pgfpathlineto{\pgfqpoint{2.024583in}{1.841364in}}%
\pgfpathlineto{\pgfqpoint{1.995378in}{1.841947in}}%
\pgfpathlineto{\pgfqpoint{1.994509in}{1.839579in}}%
\pgfpathlineto{\pgfqpoint{1.982265in}{1.839933in}}%
\pgfpathlineto{\pgfqpoint{1.981983in}{1.828333in}}%
\pgfpathclose%
\pgfusepath{fill}%
\end{pgfscope}%
\begin{pgfscope}%
\pgfpathrectangle{\pgfqpoint{0.100000in}{0.100000in}}{\pgfqpoint{3.608454in}{2.310000in}}%
\pgfusepath{clip}%
\pgfsetbuttcap%
\pgfsetmiterjoin%
\definecolor{currentfill}{rgb}{0.000000,0.682353,0.658824}%
\pgfsetfillcolor{currentfill}%
\pgfsetlinewidth{0.000000pt}%
\definecolor{currentstroke}{rgb}{0.000000,0.000000,0.000000}%
\pgfsetstrokecolor{currentstroke}%
\pgfsetstrokeopacity{0.000000}%
\pgfsetdash{}{0pt}%
\pgfpathmoveto{\pgfqpoint{2.971951in}{1.538265in}}%
\pgfpathlineto{\pgfqpoint{2.963825in}{1.532458in}}%
\pgfpathlineto{\pgfqpoint{2.952109in}{1.531961in}}%
\pgfpathlineto{\pgfqpoint{2.948242in}{1.534882in}}%
\pgfpathlineto{\pgfqpoint{2.947280in}{1.541412in}}%
\pgfpathlineto{\pgfqpoint{2.937196in}{1.539921in}}%
\pgfpathlineto{\pgfqpoint{2.934085in}{1.560268in}}%
\pgfpathlineto{\pgfqpoint{2.939221in}{1.561031in}}%
\pgfpathlineto{\pgfqpoint{2.932782in}{1.601168in}}%
\pgfpathlineto{\pgfqpoint{2.961131in}{1.605506in}}%
\pgfpathlineto{\pgfqpoint{2.984410in}{1.607734in}}%
\pgfpathlineto{\pgfqpoint{2.991632in}{1.609704in}}%
\pgfpathlineto{\pgfqpoint{2.995768in}{1.584878in}}%
\pgfpathlineto{\pgfqpoint{2.988249in}{1.569893in}}%
\pgfpathlineto{\pgfqpoint{2.990516in}{1.558619in}}%
\pgfpathlineto{\pgfqpoint{2.969282in}{1.554858in}}%
\pgfpathlineto{\pgfqpoint{2.971951in}{1.538265in}}%
\pgfpathclose%
\pgfusepath{fill}%
\end{pgfscope}%
\begin{pgfscope}%
\pgfpathrectangle{\pgfqpoint{0.100000in}{0.100000in}}{\pgfqpoint{3.608454in}{2.310000in}}%
\pgfusepath{clip}%
\pgfsetbuttcap%
\pgfsetmiterjoin%
\definecolor{currentfill}{rgb}{0.000000,0.443137,0.778431}%
\pgfsetfillcolor{currentfill}%
\pgfsetlinewidth{0.000000pt}%
\definecolor{currentstroke}{rgb}{0.000000,0.000000,0.000000}%
\pgfsetstrokecolor{currentstroke}%
\pgfsetstrokeopacity{0.000000}%
\pgfsetdash{}{0pt}%
\pgfpathmoveto{\pgfqpoint{1.643808in}{1.676503in}}%
\pgfpathlineto{\pgfqpoint{1.583107in}{1.682203in}}%
\pgfpathlineto{\pgfqpoint{1.586973in}{1.721839in}}%
\pgfpathlineto{\pgfqpoint{1.592333in}{1.775294in}}%
\pgfpathlineto{\pgfqpoint{1.598608in}{1.839877in}}%
\pgfpathlineto{\pgfqpoint{1.601301in}{1.856823in}}%
\pgfpathlineto{\pgfqpoint{1.661648in}{1.851140in}}%
\pgfpathlineto{\pgfqpoint{1.660454in}{1.837408in}}%
\pgfpathlineto{\pgfqpoint{1.714041in}{1.832923in}}%
\pgfpathlineto{\pgfqpoint{1.710799in}{1.791190in}}%
\pgfpathlineto{\pgfqpoint{1.707249in}{1.750308in}}%
\pgfpathlineto{\pgfqpoint{1.704828in}{1.727690in}}%
\pgfpathlineto{\pgfqpoint{1.697918in}{1.727667in}}%
\pgfpathlineto{\pgfqpoint{1.695616in}{1.726789in}}%
\pgfpathlineto{\pgfqpoint{1.659487in}{1.729827in}}%
\pgfpathlineto{\pgfqpoint{1.654175in}{1.729234in}}%
\pgfpathlineto{\pgfqpoint{1.648016in}{1.724287in}}%
\pgfpathlineto{\pgfqpoint{1.643808in}{1.676503in}}%
\pgfpathclose%
\pgfusepath{fill}%
\end{pgfscope}%
\begin{pgfscope}%
\pgfpathrectangle{\pgfqpoint{0.100000in}{0.100000in}}{\pgfqpoint{3.608454in}{2.310000in}}%
\pgfusepath{clip}%
\pgfsetbuttcap%
\pgfsetmiterjoin%
\definecolor{currentfill}{rgb}{0.000000,0.768627,0.615686}%
\pgfsetfillcolor{currentfill}%
\pgfsetlinewidth{0.000000pt}%
\definecolor{currentstroke}{rgb}{0.000000,0.000000,0.000000}%
\pgfsetstrokecolor{currentstroke}%
\pgfsetstrokeopacity{0.000000}%
\pgfsetdash{}{0pt}%
\pgfpathmoveto{\pgfqpoint{2.867721in}{1.199764in}}%
\pgfpathlineto{\pgfqpoint{2.859737in}{1.192156in}}%
\pgfpathlineto{\pgfqpoint{2.856311in}{1.184739in}}%
\pgfpathlineto{\pgfqpoint{2.851139in}{1.181704in}}%
\pgfpathlineto{\pgfqpoint{2.846368in}{1.183090in}}%
\pgfpathlineto{\pgfqpoint{2.836596in}{1.177507in}}%
\pgfpathlineto{\pgfqpoint{2.827195in}{1.174638in}}%
\pgfpathlineto{\pgfqpoint{2.814063in}{1.180206in}}%
\pgfpathlineto{\pgfqpoint{2.810250in}{1.179188in}}%
\pgfpathlineto{\pgfqpoint{2.804581in}{1.191394in}}%
\pgfpathlineto{\pgfqpoint{2.806029in}{1.196000in}}%
\pgfpathlineto{\pgfqpoint{2.812724in}{1.202234in}}%
\pgfpathlineto{\pgfqpoint{2.815210in}{1.209500in}}%
\pgfpathlineto{\pgfqpoint{2.831285in}{1.222519in}}%
\pgfpathlineto{\pgfqpoint{2.833882in}{1.215723in}}%
\pgfpathlineto{\pgfqpoint{2.833845in}{1.206501in}}%
\pgfpathlineto{\pgfqpoint{2.837048in}{1.201444in}}%
\pgfpathlineto{\pgfqpoint{2.828062in}{1.198088in}}%
\pgfpathlineto{\pgfqpoint{2.827558in}{1.194906in}}%
\pgfpathlineto{\pgfqpoint{2.867721in}{1.199764in}}%
\pgfpathclose%
\pgfusepath{fill}%
\end{pgfscope}%
\begin{pgfscope}%
\pgfpathrectangle{\pgfqpoint{0.100000in}{0.100000in}}{\pgfqpoint{3.608454in}{2.310000in}}%
\pgfusepath{clip}%
\pgfsetbuttcap%
\pgfsetmiterjoin%
\definecolor{currentfill}{rgb}{0.000000,0.784314,0.607843}%
\pgfsetfillcolor{currentfill}%
\pgfsetlinewidth{0.000000pt}%
\definecolor{currentstroke}{rgb}{0.000000,0.000000,0.000000}%
\pgfsetstrokecolor{currentstroke}%
\pgfsetstrokeopacity{0.000000}%
\pgfsetdash{}{0pt}%
\pgfpathmoveto{\pgfqpoint{2.843160in}{1.055221in}}%
\pgfpathlineto{\pgfqpoint{2.833248in}{1.051039in}}%
\pgfpathlineto{\pgfqpoint{2.830948in}{1.045651in}}%
\pgfpathlineto{\pgfqpoint{2.825426in}{1.046467in}}%
\pgfpathlineto{\pgfqpoint{2.820073in}{1.037913in}}%
\pgfpathlineto{\pgfqpoint{2.812171in}{1.037511in}}%
\pgfpathlineto{\pgfqpoint{2.814885in}{1.050035in}}%
\pgfpathlineto{\pgfqpoint{2.808303in}{1.061863in}}%
\pgfpathlineto{\pgfqpoint{2.798375in}{1.062974in}}%
\pgfpathlineto{\pgfqpoint{2.798205in}{1.081874in}}%
\pgfpathlineto{\pgfqpoint{2.802072in}{1.085804in}}%
\pgfpathlineto{\pgfqpoint{2.807766in}{1.084602in}}%
\pgfpathlineto{\pgfqpoint{2.814347in}{1.089123in}}%
\pgfpathlineto{\pgfqpoint{2.819581in}{1.083755in}}%
\pgfpathlineto{\pgfqpoint{2.827437in}{1.087830in}}%
\pgfpathlineto{\pgfqpoint{2.835727in}{1.088275in}}%
\pgfpathlineto{\pgfqpoint{2.834354in}{1.080705in}}%
\pgfpathlineto{\pgfqpoint{2.840206in}{1.081206in}}%
\pgfpathlineto{\pgfqpoint{2.850115in}{1.070383in}}%
\pgfpathlineto{\pgfqpoint{2.842470in}{1.059416in}}%
\pgfpathlineto{\pgfqpoint{2.843160in}{1.055221in}}%
\pgfpathclose%
\pgfusepath{fill}%
\end{pgfscope}%
\begin{pgfscope}%
\pgfpathrectangle{\pgfqpoint{0.100000in}{0.100000in}}{\pgfqpoint{3.608454in}{2.310000in}}%
\pgfusepath{clip}%
\pgfsetbuttcap%
\pgfsetmiterjoin%
\definecolor{currentfill}{rgb}{0.000000,0.478431,0.760784}%
\pgfsetfillcolor{currentfill}%
\pgfsetlinewidth{0.000000pt}%
\definecolor{currentstroke}{rgb}{0.000000,0.000000,0.000000}%
\pgfsetstrokecolor{currentstroke}%
\pgfsetstrokeopacity{0.000000}%
\pgfsetdash{}{0pt}%
\pgfpathmoveto{\pgfqpoint{2.872479in}{1.551436in}}%
\pgfpathlineto{\pgfqpoint{2.869199in}{1.556678in}}%
\pgfpathlineto{\pgfqpoint{2.868477in}{1.562448in}}%
\pgfpathlineto{\pgfqpoint{2.873516in}{1.568184in}}%
\pgfpathlineto{\pgfqpoint{2.879290in}{1.569013in}}%
\pgfpathlineto{\pgfqpoint{2.878552in}{1.574944in}}%
\pgfpathlineto{\pgfqpoint{2.884117in}{1.575727in}}%
\pgfpathlineto{\pgfqpoint{2.883315in}{1.581642in}}%
\pgfpathlineto{\pgfqpoint{2.877765in}{1.580902in}}%
\pgfpathlineto{\pgfqpoint{2.876219in}{1.593055in}}%
\pgfpathlineto{\pgfqpoint{2.889848in}{1.593630in}}%
\pgfpathlineto{\pgfqpoint{2.897159in}{1.600476in}}%
\pgfpathlineto{\pgfqpoint{2.905051in}{1.610697in}}%
\pgfpathlineto{\pgfqpoint{2.928077in}{1.628592in}}%
\pgfpathlineto{\pgfqpoint{2.934104in}{1.630691in}}%
\pgfpathlineto{\pgfqpoint{2.955120in}{1.642813in}}%
\pgfpathlineto{\pgfqpoint{2.961131in}{1.605506in}}%
\pgfpathlineto{\pgfqpoint{2.932782in}{1.601168in}}%
\pgfpathlineto{\pgfqpoint{2.939221in}{1.561031in}}%
\pgfpathlineto{\pgfqpoint{2.934085in}{1.560268in}}%
\pgfpathlineto{\pgfqpoint{2.914636in}{1.557327in}}%
\pgfpathlineto{\pgfqpoint{2.915375in}{1.550896in}}%
\pgfpathlineto{\pgfqpoint{2.901826in}{1.549489in}}%
\pgfpathlineto{\pgfqpoint{2.901003in}{1.555251in}}%
\pgfpathlineto{\pgfqpoint{2.872479in}{1.551436in}}%
\pgfpathclose%
\pgfusepath{fill}%
\end{pgfscope}%
\begin{pgfscope}%
\pgfpathrectangle{\pgfqpoint{0.100000in}{0.100000in}}{\pgfqpoint{3.608454in}{2.310000in}}%
\pgfusepath{clip}%
\pgfsetbuttcap%
\pgfsetmiterjoin%
\definecolor{currentfill}{rgb}{0.000000,0.639216,0.680392}%
\pgfsetfillcolor{currentfill}%
\pgfsetlinewidth{0.000000pt}%
\definecolor{currentstroke}{rgb}{0.000000,0.000000,0.000000}%
\pgfsetstrokecolor{currentstroke}%
\pgfsetstrokeopacity{0.000000}%
\pgfsetdash{}{0pt}%
\pgfpathmoveto{\pgfqpoint{0.737189in}{2.001359in}}%
\pgfpathlineto{\pgfqpoint{0.730749in}{2.003174in}}%
\pgfpathlineto{\pgfqpoint{0.736122in}{2.015946in}}%
\pgfpathlineto{\pgfqpoint{0.729534in}{2.017963in}}%
\pgfpathlineto{\pgfqpoint{0.731442in}{2.024554in}}%
\pgfpathlineto{\pgfqpoint{0.724757in}{2.026561in}}%
\pgfpathlineto{\pgfqpoint{0.737141in}{2.068948in}}%
\pgfpathlineto{\pgfqpoint{0.744663in}{2.068773in}}%
\pgfpathlineto{\pgfqpoint{0.749167in}{2.084439in}}%
\pgfpathlineto{\pgfqpoint{0.761418in}{2.129669in}}%
\pgfpathlineto{\pgfqpoint{0.774845in}{2.127276in}}%
\pgfpathlineto{\pgfqpoint{0.782427in}{2.131997in}}%
\pgfpathlineto{\pgfqpoint{0.784697in}{2.127385in}}%
\pgfpathlineto{\pgfqpoint{0.790335in}{2.130635in}}%
\pgfpathlineto{\pgfqpoint{0.823108in}{2.121596in}}%
\pgfpathlineto{\pgfqpoint{0.851613in}{2.114354in}}%
\pgfpathlineto{\pgfqpoint{0.849420in}{2.114482in}}%
\pgfpathlineto{\pgfqpoint{0.849927in}{2.099792in}}%
\pgfpathlineto{\pgfqpoint{0.868296in}{2.100626in}}%
\pgfpathlineto{\pgfqpoint{0.864637in}{2.088901in}}%
\pgfpathlineto{\pgfqpoint{0.871344in}{2.087217in}}%
\pgfpathlineto{\pgfqpoint{0.872020in}{2.076043in}}%
\pgfpathlineto{\pgfqpoint{0.875412in}{2.075250in}}%
\pgfpathlineto{\pgfqpoint{0.869214in}{2.048910in}}%
\pgfpathlineto{\pgfqpoint{0.849176in}{2.054195in}}%
\pgfpathlineto{\pgfqpoint{0.846645in}{2.043323in}}%
\pgfpathlineto{\pgfqpoint{0.841907in}{2.040973in}}%
\pgfpathlineto{\pgfqpoint{0.838162in}{2.031358in}}%
\pgfpathlineto{\pgfqpoint{0.835457in}{2.032072in}}%
\pgfpathlineto{\pgfqpoint{0.831103in}{2.015303in}}%
\pgfpathlineto{\pgfqpoint{0.825859in}{2.013085in}}%
\pgfpathlineto{\pgfqpoint{0.813450in}{2.016488in}}%
\pgfpathlineto{\pgfqpoint{0.811964in}{2.010809in}}%
\pgfpathlineto{\pgfqpoint{0.797207in}{2.013901in}}%
\pgfpathlineto{\pgfqpoint{0.796435in}{2.001902in}}%
\pgfpathlineto{\pgfqpoint{0.802344in}{2.000341in}}%
\pgfpathlineto{\pgfqpoint{0.799750in}{1.984499in}}%
\pgfpathlineto{\pgfqpoint{0.737189in}{2.001359in}}%
\pgfpathclose%
\pgfusepath{fill}%
\end{pgfscope}%
\begin{pgfscope}%
\pgfpathrectangle{\pgfqpoint{0.100000in}{0.100000in}}{\pgfqpoint{3.608454in}{2.310000in}}%
\pgfusepath{clip}%
\pgfsetbuttcap%
\pgfsetmiterjoin%
\definecolor{currentfill}{rgb}{0.000000,0.670588,0.664706}%
\pgfsetfillcolor{currentfill}%
\pgfsetlinewidth{0.000000pt}%
\definecolor{currentstroke}{rgb}{0.000000,0.000000,0.000000}%
\pgfsetstrokecolor{currentstroke}%
\pgfsetstrokeopacity{0.000000}%
\pgfsetdash{}{0pt}%
\pgfpathmoveto{\pgfqpoint{1.601877in}{1.254860in}}%
\pgfpathlineto{\pgfqpoint{1.596724in}{1.204258in}}%
\pgfpathlineto{\pgfqpoint{1.538731in}{1.209408in}}%
\pgfpathlineto{\pgfqpoint{1.513551in}{1.211748in}}%
\pgfpathlineto{\pgfqpoint{1.466660in}{1.216957in}}%
\pgfpathlineto{\pgfqpoint{1.469341in}{1.240362in}}%
\pgfpathlineto{\pgfqpoint{1.471264in}{1.248705in}}%
\pgfpathlineto{\pgfqpoint{1.469792in}{1.261228in}}%
\pgfpathlineto{\pgfqpoint{1.463748in}{1.269672in}}%
\pgfpathlineto{\pgfqpoint{1.450742in}{1.268060in}}%
\pgfpathlineto{\pgfqpoint{1.457227in}{1.286192in}}%
\pgfpathlineto{\pgfqpoint{1.454881in}{1.289984in}}%
\pgfpathlineto{\pgfqpoint{1.458524in}{1.289130in}}%
\pgfpathlineto{\pgfqpoint{1.470019in}{1.293572in}}%
\pgfpathlineto{\pgfqpoint{1.474843in}{1.297491in}}%
\pgfpathlineto{\pgfqpoint{1.481289in}{1.288540in}}%
\pgfpathlineto{\pgfqpoint{1.483235in}{1.285604in}}%
\pgfpathlineto{\pgfqpoint{1.506183in}{1.284643in}}%
\pgfpathlineto{\pgfqpoint{1.541303in}{1.267794in}}%
\pgfpathlineto{\pgfqpoint{1.540663in}{1.260697in}}%
\pgfpathlineto{\pgfqpoint{1.601877in}{1.254860in}}%
\pgfpathclose%
\pgfusepath{fill}%
\end{pgfscope}%
\begin{pgfscope}%
\pgfpathrectangle{\pgfqpoint{0.100000in}{0.100000in}}{\pgfqpoint{3.608454in}{2.310000in}}%
\pgfusepath{clip}%
\pgfsetbuttcap%
\pgfsetmiterjoin%
\definecolor{currentfill}{rgb}{0.000000,0.380392,0.809804}%
\pgfsetfillcolor{currentfill}%
\pgfsetlinewidth{0.000000pt}%
\definecolor{currentstroke}{rgb}{0.000000,0.000000,0.000000}%
\pgfsetstrokecolor{currentstroke}%
\pgfsetstrokeopacity{0.000000}%
\pgfsetdash{}{0pt}%
\pgfpathmoveto{\pgfqpoint{2.347355in}{1.241765in}}%
\pgfpathlineto{\pgfqpoint{2.337514in}{1.241571in}}%
\pgfpathlineto{\pgfqpoint{2.337381in}{1.248460in}}%
\pgfpathlineto{\pgfqpoint{2.323611in}{1.248060in}}%
\pgfpathlineto{\pgfqpoint{2.322748in}{1.276780in}}%
\pgfpathlineto{\pgfqpoint{2.315298in}{1.276599in}}%
\pgfpathlineto{\pgfqpoint{2.314915in}{1.287249in}}%
\pgfpathlineto{\pgfqpoint{2.313973in}{1.320063in}}%
\pgfpathlineto{\pgfqpoint{2.319587in}{1.318096in}}%
\pgfpathlineto{\pgfqpoint{2.327621in}{1.321058in}}%
\pgfpathlineto{\pgfqpoint{2.327371in}{1.331768in}}%
\pgfpathlineto{\pgfqpoint{2.336652in}{1.332026in}}%
\pgfpathlineto{\pgfqpoint{2.335984in}{1.355415in}}%
\pgfpathlineto{\pgfqpoint{2.340562in}{1.355530in}}%
\pgfpathlineto{\pgfqpoint{2.340427in}{1.362513in}}%
\pgfpathlineto{\pgfqpoint{2.368642in}{1.363523in}}%
\pgfpathlineto{\pgfqpoint{2.371543in}{1.352918in}}%
\pgfpathlineto{\pgfqpoint{2.369779in}{1.350114in}}%
\pgfpathlineto{\pgfqpoint{2.373739in}{1.339722in}}%
\pgfpathlineto{\pgfqpoint{2.380869in}{1.336490in}}%
\pgfpathlineto{\pgfqpoint{2.386232in}{1.344093in}}%
\pgfpathlineto{\pgfqpoint{2.398503in}{1.340979in}}%
\pgfpathlineto{\pgfqpoint{2.407445in}{1.335820in}}%
\pgfpathlineto{\pgfqpoint{2.402286in}{1.326271in}}%
\pgfpathlineto{\pgfqpoint{2.404058in}{1.320682in}}%
\pgfpathlineto{\pgfqpoint{2.433309in}{1.321900in}}%
\pgfpathlineto{\pgfqpoint{2.435549in}{1.287521in}}%
\pgfpathlineto{\pgfqpoint{2.414877in}{1.286654in}}%
\pgfpathlineto{\pgfqpoint{2.415279in}{1.279749in}}%
\pgfpathlineto{\pgfqpoint{2.404880in}{1.275417in}}%
\pgfpathlineto{\pgfqpoint{2.397241in}{1.275533in}}%
\pgfpathlineto{\pgfqpoint{2.392056in}{1.271141in}}%
\pgfpathlineto{\pgfqpoint{2.381801in}{1.267561in}}%
\pgfpathlineto{\pgfqpoint{2.368816in}{1.282820in}}%
\pgfpathlineto{\pgfqpoint{2.349233in}{1.281942in}}%
\pgfpathlineto{\pgfqpoint{2.350544in}{1.245278in}}%
\pgfpathlineto{\pgfqpoint{2.347355in}{1.241765in}}%
\pgfpathclose%
\pgfusepath{fill}%
\end{pgfscope}%
\begin{pgfscope}%
\pgfpathrectangle{\pgfqpoint{0.100000in}{0.100000in}}{\pgfqpoint{3.608454in}{2.310000in}}%
\pgfusepath{clip}%
\pgfsetbuttcap%
\pgfsetmiterjoin%
\definecolor{currentfill}{rgb}{0.000000,0.301961,0.849020}%
\pgfsetfillcolor{currentfill}%
\pgfsetlinewidth{0.000000pt}%
\definecolor{currentstroke}{rgb}{0.000000,0.000000,0.000000}%
\pgfsetstrokecolor{currentstroke}%
\pgfsetstrokeopacity{0.000000}%
\pgfsetdash{}{0pt}%
\pgfpathmoveto{\pgfqpoint{2.130827in}{1.263573in}}%
\pgfpathlineto{\pgfqpoint{2.101955in}{1.263584in}}%
\pgfpathlineto{\pgfqpoint{2.102752in}{1.281057in}}%
\pgfpathlineto{\pgfqpoint{2.102737in}{1.315533in}}%
\pgfpathlineto{\pgfqpoint{2.103312in}{1.318980in}}%
\pgfpathlineto{\pgfqpoint{2.103336in}{1.338237in}}%
\pgfpathlineto{\pgfqpoint{2.112374in}{1.338940in}}%
\pgfpathlineto{\pgfqpoint{2.112893in}{1.355625in}}%
\pgfpathlineto{\pgfqpoint{2.116197in}{1.360877in}}%
\pgfpathlineto{\pgfqpoint{2.107556in}{1.374791in}}%
\pgfpathlineto{\pgfqpoint{2.100521in}{1.381808in}}%
\pgfpathlineto{\pgfqpoint{2.131100in}{1.381764in}}%
\pgfpathlineto{\pgfqpoint{2.131152in}{1.375734in}}%
\pgfpathlineto{\pgfqpoint{2.154965in}{1.375795in}}%
\pgfpathlineto{\pgfqpoint{2.155044in}{1.381527in}}%
\pgfpathlineto{\pgfqpoint{2.182439in}{1.381621in}}%
\pgfpathlineto{\pgfqpoint{2.182827in}{1.356530in}}%
\pgfpathlineto{\pgfqpoint{2.188988in}{1.359699in}}%
\pgfpathlineto{\pgfqpoint{2.198627in}{1.357636in}}%
\pgfpathlineto{\pgfqpoint{2.199219in}{1.335898in}}%
\pgfpathlineto{\pgfqpoint{2.198808in}{1.305319in}}%
\pgfpathlineto{\pgfqpoint{2.164617in}{1.305750in}}%
\pgfpathlineto{\pgfqpoint{2.164228in}{1.277853in}}%
\pgfpathlineto{\pgfqpoint{2.165424in}{1.263791in}}%
\pgfpathlineto{\pgfqpoint{2.154102in}{1.264813in}}%
\pgfpathlineto{\pgfqpoint{2.130815in}{1.265402in}}%
\pgfpathlineto{\pgfqpoint{2.130827in}{1.263573in}}%
\pgfpathclose%
\pgfusepath{fill}%
\end{pgfscope}%
\begin{pgfscope}%
\pgfpathrectangle{\pgfqpoint{0.100000in}{0.100000in}}{\pgfqpoint{3.608454in}{2.310000in}}%
\pgfusepath{clip}%
\pgfsetbuttcap%
\pgfsetmiterjoin%
\definecolor{currentfill}{rgb}{0.000000,0.611765,0.694118}%
\pgfsetfillcolor{currentfill}%
\pgfsetlinewidth{0.000000pt}%
\definecolor{currentstroke}{rgb}{0.000000,0.000000,0.000000}%
\pgfsetstrokecolor{currentstroke}%
\pgfsetstrokeopacity{0.000000}%
\pgfsetdash{}{0pt}%
\pgfpathmoveto{\pgfqpoint{1.522840in}{2.169802in}}%
\pgfpathlineto{\pgfqpoint{1.578022in}{2.163356in}}%
\pgfpathlineto{\pgfqpoint{1.578243in}{2.156172in}}%
\pgfpathlineto{\pgfqpoint{1.575926in}{2.135412in}}%
\pgfpathlineto{\pgfqpoint{1.578597in}{2.128096in}}%
\pgfpathlineto{\pgfqpoint{1.534783in}{2.133134in}}%
\pgfpathlineto{\pgfqpoint{1.523357in}{2.134506in}}%
\pgfpathlineto{\pgfqpoint{1.525884in}{2.155265in}}%
\pgfpathlineto{\pgfqpoint{1.519036in}{2.156103in}}%
\pgfpathlineto{\pgfqpoint{1.522840in}{2.169802in}}%
\pgfpathclose%
\pgfusepath{fill}%
\end{pgfscope}%
\begin{pgfscope}%
\pgfpathrectangle{\pgfqpoint{0.100000in}{0.100000in}}{\pgfqpoint{3.608454in}{2.310000in}}%
\pgfusepath{clip}%
\pgfsetbuttcap%
\pgfsetmiterjoin%
\definecolor{currentfill}{rgb}{0.000000,0.419608,0.790196}%
\pgfsetfillcolor{currentfill}%
\pgfsetlinewidth{0.000000pt}%
\definecolor{currentstroke}{rgb}{0.000000,0.000000,0.000000}%
\pgfsetstrokecolor{currentstroke}%
\pgfsetstrokeopacity{0.000000}%
\pgfsetdash{}{0pt}%
\pgfpathmoveto{\pgfqpoint{3.062330in}{0.671592in}}%
\pgfpathlineto{\pgfqpoint{3.055227in}{0.666862in}}%
\pgfpathlineto{\pgfqpoint{3.041523in}{0.664546in}}%
\pgfpathlineto{\pgfqpoint{3.039796in}{0.675377in}}%
\pgfpathlineto{\pgfqpoint{3.035034in}{0.681162in}}%
\pgfpathlineto{\pgfqpoint{3.019158in}{0.678567in}}%
\pgfpathlineto{\pgfqpoint{3.011416in}{0.670241in}}%
\pgfpathlineto{\pgfqpoint{3.004229in}{0.666876in}}%
\pgfpathlineto{\pgfqpoint{2.999393in}{0.700496in}}%
\pgfpathlineto{\pgfqpoint{2.971461in}{0.696012in}}%
\pgfpathlineto{\pgfqpoint{2.966459in}{0.731456in}}%
\pgfpathlineto{\pgfqpoint{2.983300in}{0.732552in}}%
\pgfpathlineto{\pgfqpoint{2.985291in}{0.750228in}}%
\pgfpathlineto{\pgfqpoint{2.966799in}{0.748504in}}%
\pgfpathlineto{\pgfqpoint{2.964698in}{0.765826in}}%
\pgfpathlineto{\pgfqpoint{2.983960in}{0.768316in}}%
\pgfpathlineto{\pgfqpoint{2.988034in}{0.774133in}}%
\pgfpathlineto{\pgfqpoint{2.996804in}{0.773153in}}%
\pgfpathlineto{\pgfqpoint{3.005604in}{0.782838in}}%
\pgfpathlineto{\pgfqpoint{3.020788in}{0.780741in}}%
\pgfpathlineto{\pgfqpoint{3.030760in}{0.775066in}}%
\pgfpathlineto{\pgfqpoint{3.034315in}{0.767379in}}%
\pgfpathlineto{\pgfqpoint{3.032529in}{0.754216in}}%
\pgfpathlineto{\pgfqpoint{3.035355in}{0.750593in}}%
\pgfpathlineto{\pgfqpoint{3.035557in}{0.741809in}}%
\pgfpathlineto{\pgfqpoint{3.041274in}{0.727152in}}%
\pgfpathlineto{\pgfqpoint{3.042489in}{0.719793in}}%
\pgfpathlineto{\pgfqpoint{3.052515in}{0.693343in}}%
\pgfpathlineto{\pgfqpoint{3.062330in}{0.671592in}}%
\pgfpathclose%
\pgfusepath{fill}%
\end{pgfscope}%
\begin{pgfscope}%
\pgfpathrectangle{\pgfqpoint{0.100000in}{0.100000in}}{\pgfqpoint{3.608454in}{2.310000in}}%
\pgfusepath{clip}%
\pgfsetbuttcap%
\pgfsetmiterjoin%
\definecolor{currentfill}{rgb}{0.000000,0.721569,0.639216}%
\pgfsetfillcolor{currentfill}%
\pgfsetlinewidth{0.000000pt}%
\definecolor{currentstroke}{rgb}{0.000000,0.000000,0.000000}%
\pgfsetstrokecolor{currentstroke}%
\pgfsetstrokeopacity{0.000000}%
\pgfsetdash{}{0pt}%
\pgfpathmoveto{\pgfqpoint{1.590719in}{1.104616in}}%
\pgfpathlineto{\pgfqpoint{1.589893in}{1.095406in}}%
\pgfpathlineto{\pgfqpoint{1.585141in}{1.042675in}}%
\pgfpathlineto{\pgfqpoint{1.569604in}{1.044093in}}%
\pgfpathlineto{\pgfqpoint{1.568889in}{1.037039in}}%
\pgfpathlineto{\pgfqpoint{1.562067in}{1.037697in}}%
\pgfpathlineto{\pgfqpoint{1.561377in}{1.030839in}}%
\pgfpathlineto{\pgfqpoint{1.554565in}{1.031506in}}%
\pgfpathlineto{\pgfqpoint{1.553889in}{1.024630in}}%
\pgfpathlineto{\pgfqpoint{1.540290in}{1.025994in}}%
\pgfpathlineto{\pgfqpoint{1.539569in}{1.019066in}}%
\pgfpathlineto{\pgfqpoint{1.523829in}{1.020634in}}%
\pgfpathlineto{\pgfqpoint{1.512139in}{1.021816in}}%
\pgfpathlineto{\pgfqpoint{1.516667in}{1.064131in}}%
\pgfpathlineto{\pgfqpoint{1.534671in}{1.070306in}}%
\pgfpathlineto{\pgfqpoint{1.548858in}{1.068831in}}%
\pgfpathlineto{\pgfqpoint{1.550211in}{1.083114in}}%
\pgfpathlineto{\pgfqpoint{1.544598in}{1.083669in}}%
\pgfpathlineto{\pgfqpoint{1.531357in}{1.115154in}}%
\pgfpathlineto{\pgfqpoint{1.528359in}{1.113844in}}%
\pgfpathlineto{\pgfqpoint{1.506250in}{1.116084in}}%
\pgfpathlineto{\pgfqpoint{1.508110in}{1.125109in}}%
\pgfpathlineto{\pgfqpoint{1.508502in}{1.136752in}}%
\pgfpathlineto{\pgfqpoint{1.505428in}{1.150972in}}%
\pgfpathlineto{\pgfqpoint{1.532570in}{1.148149in}}%
\pgfpathlineto{\pgfqpoint{1.532225in}{1.144723in}}%
\pgfpathlineto{\pgfqpoint{1.545891in}{1.143339in}}%
\pgfpathlineto{\pgfqpoint{1.545193in}{1.136393in}}%
\pgfpathlineto{\pgfqpoint{1.572537in}{1.133851in}}%
\pgfpathlineto{\pgfqpoint{1.569376in}{1.106608in}}%
\pgfpathlineto{\pgfqpoint{1.590719in}{1.104616in}}%
\pgfpathclose%
\pgfusepath{fill}%
\end{pgfscope}%
\begin{pgfscope}%
\pgfpathrectangle{\pgfqpoint{0.100000in}{0.100000in}}{\pgfqpoint{3.608454in}{2.310000in}}%
\pgfusepath{clip}%
\pgfsetbuttcap%
\pgfsetmiterjoin%
\definecolor{currentfill}{rgb}{0.000000,0.631373,0.684314}%
\pgfsetfillcolor{currentfill}%
\pgfsetlinewidth{0.000000pt}%
\definecolor{currentstroke}{rgb}{0.000000,0.000000,0.000000}%
\pgfsetstrokecolor{currentstroke}%
\pgfsetstrokeopacity{0.000000}%
\pgfsetdash{}{0pt}%
\pgfpathmoveto{\pgfqpoint{3.130816in}{1.638979in}}%
\pgfpathlineto{\pgfqpoint{3.123343in}{1.674373in}}%
\pgfpathlineto{\pgfqpoint{3.115261in}{1.672766in}}%
\pgfpathlineto{\pgfqpoint{3.109751in}{1.709791in}}%
\pgfpathlineto{\pgfqpoint{3.111728in}{1.718970in}}%
\pgfpathlineto{\pgfqpoint{3.141312in}{1.724493in}}%
\pgfpathlineto{\pgfqpoint{3.142173in}{1.718464in}}%
\pgfpathlineto{\pgfqpoint{3.156855in}{1.718354in}}%
\pgfpathlineto{\pgfqpoint{3.155225in}{1.724423in}}%
\pgfpathlineto{\pgfqpoint{3.173622in}{1.728849in}}%
\pgfpathlineto{\pgfqpoint{3.167852in}{1.733612in}}%
\pgfpathlineto{\pgfqpoint{3.190729in}{1.738354in}}%
\pgfpathlineto{\pgfqpoint{3.194930in}{1.721757in}}%
\pgfpathlineto{\pgfqpoint{3.196916in}{1.713203in}}%
\pgfpathlineto{\pgfqpoint{3.187096in}{1.712911in}}%
\pgfpathlineto{\pgfqpoint{3.180627in}{1.708614in}}%
\pgfpathlineto{\pgfqpoint{3.183986in}{1.686549in}}%
\pgfpathlineto{\pgfqpoint{3.162699in}{1.682336in}}%
\pgfpathlineto{\pgfqpoint{3.172073in}{1.651526in}}%
\pgfpathlineto{\pgfqpoint{3.167806in}{1.647018in}}%
\pgfpathlineto{\pgfqpoint{3.130816in}{1.638979in}}%
\pgfpathclose%
\pgfusepath{fill}%
\end{pgfscope}%
\begin{pgfscope}%
\pgfpathrectangle{\pgfqpoint{0.100000in}{0.100000in}}{\pgfqpoint{3.608454in}{2.310000in}}%
\pgfusepath{clip}%
\pgfsetbuttcap%
\pgfsetmiterjoin%
\definecolor{currentfill}{rgb}{0.000000,0.796078,0.601961}%
\pgfsetfillcolor{currentfill}%
\pgfsetlinewidth{0.000000pt}%
\definecolor{currentstroke}{rgb}{0.000000,0.000000,0.000000}%
\pgfsetstrokecolor{currentstroke}%
\pgfsetstrokeopacity{0.000000}%
\pgfsetdash{}{0pt}%
\pgfpathmoveto{\pgfqpoint{0.800486in}{1.923543in}}%
\pgfpathlineto{\pgfqpoint{0.805150in}{1.923524in}}%
\pgfpathlineto{\pgfqpoint{0.814693in}{1.932896in}}%
\pgfpathlineto{\pgfqpoint{0.818538in}{1.934210in}}%
\pgfpathlineto{\pgfqpoint{0.839459in}{1.928946in}}%
\pgfpathlineto{\pgfqpoint{0.844130in}{1.923115in}}%
\pgfpathlineto{\pgfqpoint{0.842438in}{1.916478in}}%
\pgfpathlineto{\pgfqpoint{0.857097in}{1.912795in}}%
\pgfpathlineto{\pgfqpoint{0.860164in}{1.926417in}}%
\pgfpathlineto{\pgfqpoint{0.866297in}{1.932367in}}%
\pgfpathlineto{\pgfqpoint{0.875640in}{1.944745in}}%
\pgfpathlineto{\pgfqpoint{0.884981in}{1.950005in}}%
\pgfpathlineto{\pgfqpoint{0.899991in}{1.946190in}}%
\pgfpathlineto{\pgfqpoint{0.896730in}{1.932620in}}%
\pgfpathlineto{\pgfqpoint{0.901098in}{1.927625in}}%
\pgfpathlineto{\pgfqpoint{0.898851in}{1.918021in}}%
\pgfpathlineto{\pgfqpoint{0.904574in}{1.912817in}}%
\pgfpathlineto{\pgfqpoint{0.910516in}{1.911417in}}%
\pgfpathlineto{\pgfqpoint{0.905607in}{1.902658in}}%
\pgfpathlineto{\pgfqpoint{0.901859in}{1.889570in}}%
\pgfpathlineto{\pgfqpoint{0.896690in}{1.890839in}}%
\pgfpathlineto{\pgfqpoint{0.895063in}{1.884202in}}%
\pgfpathlineto{\pgfqpoint{0.890663in}{1.885325in}}%
\pgfpathlineto{\pgfqpoint{0.886843in}{1.879251in}}%
\pgfpathlineto{\pgfqpoint{0.879074in}{1.881212in}}%
\pgfpathlineto{\pgfqpoint{0.875796in}{1.867766in}}%
\pgfpathlineto{\pgfqpoint{0.867712in}{1.868584in}}%
\pgfpathlineto{\pgfqpoint{0.863735in}{1.876688in}}%
\pgfpathlineto{\pgfqpoint{0.860241in}{1.876128in}}%
\pgfpathlineto{\pgfqpoint{0.855951in}{1.862354in}}%
\pgfpathlineto{\pgfqpoint{0.824016in}{1.733109in}}%
\pgfpathlineto{\pgfqpoint{0.757203in}{1.749818in}}%
\pgfpathlineto{\pgfqpoint{0.761786in}{1.771448in}}%
\pgfpathlineto{\pgfqpoint{0.774453in}{1.820402in}}%
\pgfpathlineto{\pgfqpoint{0.773761in}{1.820585in}}%
\pgfpathlineto{\pgfqpoint{0.790924in}{1.886917in}}%
\pgfpathlineto{\pgfqpoint{0.796408in}{1.906884in}}%
\pgfpathlineto{\pgfqpoint{0.800486in}{1.923543in}}%
\pgfpathclose%
\pgfusepath{fill}%
\end{pgfscope}%
\begin{pgfscope}%
\pgfpathrectangle{\pgfqpoint{0.100000in}{0.100000in}}{\pgfqpoint{3.608454in}{2.310000in}}%
\pgfusepath{clip}%
\pgfsetbuttcap%
\pgfsetmiterjoin%
\definecolor{currentfill}{rgb}{0.000000,0.635294,0.682353}%
\pgfsetfillcolor{currentfill}%
\pgfsetlinewidth{0.000000pt}%
\definecolor{currentstroke}{rgb}{0.000000,0.000000,0.000000}%
\pgfsetstrokecolor{currentstroke}%
\pgfsetstrokeopacity{0.000000}%
\pgfsetdash{}{0pt}%
\pgfpathmoveto{\pgfqpoint{2.862925in}{0.815951in}}%
\pgfpathlineto{\pgfqpoint{2.854299in}{0.818491in}}%
\pgfpathlineto{\pgfqpoint{2.853827in}{0.823500in}}%
\pgfpathlineto{\pgfqpoint{2.847520in}{0.837484in}}%
\pgfpathlineto{\pgfqpoint{2.844094in}{0.841666in}}%
\pgfpathlineto{\pgfqpoint{2.847916in}{0.849972in}}%
\pgfpathlineto{\pgfqpoint{2.855107in}{0.854082in}}%
\pgfpathlineto{\pgfqpoint{2.871284in}{0.855909in}}%
\pgfpathlineto{\pgfqpoint{2.872842in}{0.842457in}}%
\pgfpathlineto{\pgfqpoint{2.891100in}{0.843787in}}%
\pgfpathlineto{\pgfqpoint{2.894217in}{0.842012in}}%
\pgfpathlineto{\pgfqpoint{2.895939in}{0.833978in}}%
\pgfpathlineto{\pgfqpoint{2.900223in}{0.831517in}}%
\pgfpathlineto{\pgfqpoint{2.904439in}{0.824809in}}%
\pgfpathlineto{\pgfqpoint{2.884148in}{0.822135in}}%
\pgfpathlineto{\pgfqpoint{2.875215in}{0.821548in}}%
\pgfpathlineto{\pgfqpoint{2.875686in}{0.817596in}}%
\pgfpathlineto{\pgfqpoint{2.862925in}{0.815951in}}%
\pgfpathclose%
\pgfusepath{fill}%
\end{pgfscope}%
\begin{pgfscope}%
\pgfpathrectangle{\pgfqpoint{0.100000in}{0.100000in}}{\pgfqpoint{3.608454in}{2.310000in}}%
\pgfusepath{clip}%
\pgfsetbuttcap%
\pgfsetmiterjoin%
\definecolor{currentfill}{rgb}{0.000000,0.517647,0.741176}%
\pgfsetfillcolor{currentfill}%
\pgfsetlinewidth{0.000000pt}%
\definecolor{currentstroke}{rgb}{0.000000,0.000000,0.000000}%
\pgfsetstrokecolor{currentstroke}%
\pgfsetstrokeopacity{0.000000}%
\pgfsetdash{}{0pt}%
\pgfpathmoveto{\pgfqpoint{1.793434in}{1.427744in}}%
\pgfpathlineto{\pgfqpoint{1.792481in}{1.427797in}}%
\pgfpathlineto{\pgfqpoint{1.793800in}{1.455314in}}%
\pgfpathlineto{\pgfqpoint{1.758946in}{1.457374in}}%
\pgfpathlineto{\pgfqpoint{1.760676in}{1.485050in}}%
\pgfpathlineto{\pgfqpoint{1.793857in}{1.483052in}}%
\pgfpathlineto{\pgfqpoint{1.795374in}{1.510393in}}%
\pgfpathlineto{\pgfqpoint{1.842917in}{1.507856in}}%
\pgfpathlineto{\pgfqpoint{1.841990in}{1.478161in}}%
\pgfpathlineto{\pgfqpoint{1.856208in}{1.476676in}}%
\pgfpathlineto{\pgfqpoint{1.853837in}{1.424663in}}%
\pgfpathlineto{\pgfqpoint{1.826801in}{1.425965in}}%
\pgfpathlineto{\pgfqpoint{1.793434in}{1.427744in}}%
\pgfpathclose%
\pgfusepath{fill}%
\end{pgfscope}%
\begin{pgfscope}%
\pgfpathrectangle{\pgfqpoint{0.100000in}{0.100000in}}{\pgfqpoint{3.608454in}{2.310000in}}%
\pgfusepath{clip}%
\pgfsetbuttcap%
\pgfsetmiterjoin%
\definecolor{currentfill}{rgb}{0.000000,0.341176,0.829412}%
\pgfsetfillcolor{currentfill}%
\pgfsetlinewidth{0.000000pt}%
\definecolor{currentstroke}{rgb}{0.000000,0.000000,0.000000}%
\pgfsetstrokecolor{currentstroke}%
\pgfsetstrokeopacity{0.000000}%
\pgfsetdash{}{0pt}%
\pgfpathmoveto{\pgfqpoint{1.538610in}{1.726627in}}%
\pgfpathlineto{\pgfqpoint{1.586973in}{1.721839in}}%
\pgfpathlineto{\pgfqpoint{1.583107in}{1.682203in}}%
\pgfpathlineto{\pgfqpoint{1.580064in}{1.651542in}}%
\pgfpathlineto{\pgfqpoint{1.531372in}{1.656425in}}%
\pgfpathlineto{\pgfqpoint{1.532653in}{1.671867in}}%
\pgfpathlineto{\pgfqpoint{1.538610in}{1.726627in}}%
\pgfpathclose%
\pgfusepath{fill}%
\end{pgfscope}%
\begin{pgfscope}%
\pgfpathrectangle{\pgfqpoint{0.100000in}{0.100000in}}{\pgfqpoint{3.608454in}{2.310000in}}%
\pgfusepath{clip}%
\pgfsetbuttcap%
\pgfsetmiterjoin%
\definecolor{currentfill}{rgb}{0.000000,0.392157,0.803922}%
\pgfsetfillcolor{currentfill}%
\pgfsetlinewidth{0.000000pt}%
\definecolor{currentstroke}{rgb}{0.000000,0.000000,0.000000}%
\pgfsetstrokecolor{currentstroke}%
\pgfsetstrokeopacity{0.000000}%
\pgfsetdash{}{0pt}%
\pgfpathmoveto{\pgfqpoint{0.429033in}{1.527264in}}%
\pgfpathlineto{\pgfqpoint{0.432587in}{1.529018in}}%
\pgfpathlineto{\pgfqpoint{0.433169in}{1.538861in}}%
\pgfpathlineto{\pgfqpoint{0.431374in}{1.550880in}}%
\pgfpathlineto{\pgfqpoint{0.425226in}{1.565689in}}%
\pgfpathlineto{\pgfqpoint{0.427977in}{1.570534in}}%
\pgfpathlineto{\pgfqpoint{0.438495in}{1.570367in}}%
\pgfpathlineto{\pgfqpoint{0.446440in}{1.577149in}}%
\pgfpathlineto{\pgfqpoint{0.446281in}{1.581110in}}%
\pgfpathlineto{\pgfqpoint{0.452005in}{1.577321in}}%
\pgfpathlineto{\pgfqpoint{0.452551in}{1.565246in}}%
\pgfpathlineto{\pgfqpoint{0.454748in}{1.561199in}}%
\pgfpathlineto{\pgfqpoint{0.455284in}{1.549289in}}%
\pgfpathlineto{\pgfqpoint{0.460110in}{1.545792in}}%
\pgfpathlineto{\pgfqpoint{0.465266in}{1.547913in}}%
\pgfpathlineto{\pgfqpoint{0.479765in}{1.542960in}}%
\pgfpathlineto{\pgfqpoint{0.474910in}{1.526813in}}%
\pgfpathlineto{\pgfqpoint{0.480918in}{1.524945in}}%
\pgfpathlineto{\pgfqpoint{0.477138in}{1.516663in}}%
\pgfpathlineto{\pgfqpoint{0.471900in}{1.514999in}}%
\pgfpathlineto{\pgfqpoint{0.471851in}{1.509726in}}%
\pgfpathlineto{\pgfqpoint{0.476612in}{1.508153in}}%
\pgfpathlineto{\pgfqpoint{0.471692in}{1.491720in}}%
\pgfpathlineto{\pgfqpoint{0.471844in}{1.486991in}}%
\pgfpathlineto{\pgfqpoint{0.465525in}{1.465978in}}%
\pgfpathlineto{\pgfqpoint{0.469229in}{1.459834in}}%
\pgfpathlineto{\pgfqpoint{0.469193in}{1.459876in}}%
\pgfpathlineto{\pgfqpoint{0.445670in}{1.467188in}}%
\pgfpathlineto{\pgfqpoint{0.441374in}{1.466010in}}%
\pgfpathlineto{\pgfqpoint{0.432540in}{1.470642in}}%
\pgfpathlineto{\pgfqpoint{0.424905in}{1.469119in}}%
\pgfpathlineto{\pgfqpoint{0.422616in}{1.464232in}}%
\pgfpathlineto{\pgfqpoint{0.422155in}{1.452127in}}%
\pgfpathlineto{\pgfqpoint{0.411566in}{1.453040in}}%
\pgfpathlineto{\pgfqpoint{0.411162in}{1.446533in}}%
\pgfpathlineto{\pgfqpoint{0.406795in}{1.454089in}}%
\pgfpathlineto{\pgfqpoint{0.406757in}{1.459003in}}%
\pgfpathlineto{\pgfqpoint{0.410394in}{1.466958in}}%
\pgfpathlineto{\pgfqpoint{0.410387in}{1.477682in}}%
\pgfpathlineto{\pgfqpoint{0.407424in}{1.482308in}}%
\pgfpathlineto{\pgfqpoint{0.411743in}{1.490790in}}%
\pgfpathlineto{\pgfqpoint{0.413684in}{1.501555in}}%
\pgfpathlineto{\pgfqpoint{0.420869in}{1.501805in}}%
\pgfpathlineto{\pgfqpoint{0.418472in}{1.484385in}}%
\pgfpathlineto{\pgfqpoint{0.423806in}{1.481158in}}%
\pgfpathlineto{\pgfqpoint{0.428604in}{1.473723in}}%
\pgfpathlineto{\pgfqpoint{0.431932in}{1.475097in}}%
\pgfpathlineto{\pgfqpoint{0.432089in}{1.485171in}}%
\pgfpathlineto{\pgfqpoint{0.424993in}{1.499408in}}%
\pgfpathlineto{\pgfqpoint{0.428477in}{1.506604in}}%
\pgfpathlineto{\pgfqpoint{0.424773in}{1.512850in}}%
\pgfpathlineto{\pgfqpoint{0.435086in}{1.517401in}}%
\pgfpathlineto{\pgfqpoint{0.433615in}{1.522888in}}%
\pgfpathlineto{\pgfqpoint{0.429033in}{1.527264in}}%
\pgfpathclose%
\pgfusepath{fill}%
\end{pgfscope}%
\begin{pgfscope}%
\pgfpathrectangle{\pgfqpoint{0.100000in}{0.100000in}}{\pgfqpoint{3.608454in}{2.310000in}}%
\pgfusepath{clip}%
\pgfsetbuttcap%
\pgfsetmiterjoin%
\definecolor{currentfill}{rgb}{0.000000,0.392157,0.803922}%
\pgfsetfillcolor{currentfill}%
\pgfsetlinewidth{0.000000pt}%
\definecolor{currentstroke}{rgb}{0.000000,0.000000,0.000000}%
\pgfsetstrokecolor{currentstroke}%
\pgfsetstrokeopacity{0.000000}%
\pgfsetdash{}{0pt}%
\pgfpathmoveto{\pgfqpoint{0.422980in}{1.525827in}}%
\pgfpathlineto{\pgfqpoint{0.419309in}{1.509781in}}%
\pgfpathlineto{\pgfqpoint{0.414223in}{1.505038in}}%
\pgfpathlineto{\pgfqpoint{0.408463in}{1.513232in}}%
\pgfpathlineto{\pgfqpoint{0.405158in}{1.513449in}}%
\pgfpathlineto{\pgfqpoint{0.399143in}{1.525491in}}%
\pgfpathlineto{\pgfqpoint{0.393060in}{1.528517in}}%
\pgfpathlineto{\pgfqpoint{0.396639in}{1.537475in}}%
\pgfpathlineto{\pgfqpoint{0.396978in}{1.549508in}}%
\pgfpathlineto{\pgfqpoint{0.403577in}{1.549156in}}%
\pgfpathlineto{\pgfqpoint{0.410415in}{1.537899in}}%
\pgfpathlineto{\pgfqpoint{0.419978in}{1.533247in}}%
\pgfpathlineto{\pgfqpoint{0.422980in}{1.525827in}}%
\pgfpathclose%
\pgfusepath{fill}%
\end{pgfscope}%
\begin{pgfscope}%
\pgfpathrectangle{\pgfqpoint{0.100000in}{0.100000in}}{\pgfqpoint{3.608454in}{2.310000in}}%
\pgfusepath{clip}%
\pgfsetbuttcap%
\pgfsetmiterjoin%
\definecolor{currentfill}{rgb}{0.000000,0.592157,0.703922}%
\pgfsetfillcolor{currentfill}%
\pgfsetlinewidth{0.000000pt}%
\definecolor{currentstroke}{rgb}{0.000000,0.000000,0.000000}%
\pgfsetstrokecolor{currentstroke}%
\pgfsetstrokeopacity{0.000000}%
\pgfsetdash{}{0pt}%
\pgfpathmoveto{\pgfqpoint{2.410650in}{1.552521in}}%
\pgfpathlineto{\pgfqpoint{2.391551in}{1.551544in}}%
\pgfpathlineto{\pgfqpoint{2.387375in}{1.567036in}}%
\pgfpathlineto{\pgfqpoint{2.390834in}{1.569129in}}%
\pgfpathlineto{\pgfqpoint{2.392472in}{1.584764in}}%
\pgfpathlineto{\pgfqpoint{2.390701in}{1.593517in}}%
\pgfpathlineto{\pgfqpoint{2.381293in}{1.599262in}}%
\pgfpathlineto{\pgfqpoint{2.404561in}{1.600729in}}%
\pgfpathlineto{\pgfqpoint{2.402794in}{1.625171in}}%
\pgfpathlineto{\pgfqpoint{2.435480in}{1.626662in}}%
\pgfpathlineto{\pgfqpoint{2.473885in}{1.628654in}}%
\pgfpathlineto{\pgfqpoint{2.475849in}{1.601806in}}%
\pgfpathlineto{\pgfqpoint{2.482705in}{1.602290in}}%
\pgfpathlineto{\pgfqpoint{2.482413in}{1.595355in}}%
\pgfpathlineto{\pgfqpoint{2.484797in}{1.561007in}}%
\pgfpathlineto{\pgfqpoint{2.451488in}{1.558550in}}%
\pgfpathlineto{\pgfqpoint{2.451750in}{1.555104in}}%
\pgfpathlineto{\pgfqpoint{2.410650in}{1.552521in}}%
\pgfpathclose%
\pgfusepath{fill}%
\end{pgfscope}%
\begin{pgfscope}%
\pgfpathrectangle{\pgfqpoint{0.100000in}{0.100000in}}{\pgfqpoint{3.608454in}{2.310000in}}%
\pgfusepath{clip}%
\pgfsetbuttcap%
\pgfsetmiterjoin%
\definecolor{currentfill}{rgb}{0.000000,0.462745,0.768627}%
\pgfsetfillcolor{currentfill}%
\pgfsetlinewidth{0.000000pt}%
\definecolor{currentstroke}{rgb}{0.000000,0.000000,0.000000}%
\pgfsetstrokecolor{currentstroke}%
\pgfsetstrokeopacity{0.000000}%
\pgfsetdash{}{0pt}%
\pgfpathmoveto{\pgfqpoint{2.583434in}{1.352723in}}%
\pgfpathlineto{\pgfqpoint{2.590232in}{1.353266in}}%
\pgfpathlineto{\pgfqpoint{2.590013in}{1.348185in}}%
\pgfpathlineto{\pgfqpoint{2.584141in}{1.345785in}}%
\pgfpathlineto{\pgfqpoint{2.582510in}{1.335565in}}%
\pgfpathlineto{\pgfqpoint{2.585800in}{1.324135in}}%
\pgfpathlineto{\pgfqpoint{2.572272in}{1.322088in}}%
\pgfpathlineto{\pgfqpoint{2.565371in}{1.317560in}}%
\pgfpathlineto{\pgfqpoint{2.560427in}{1.323864in}}%
\pgfpathlineto{\pgfqpoint{2.544363in}{1.322774in}}%
\pgfpathlineto{\pgfqpoint{2.542895in}{1.344839in}}%
\pgfpathlineto{\pgfqpoint{2.565996in}{1.346868in}}%
\pgfpathlineto{\pgfqpoint{2.566176in}{1.350839in}}%
\pgfpathlineto{\pgfqpoint{2.583434in}{1.352723in}}%
\pgfpathclose%
\pgfusepath{fill}%
\end{pgfscope}%
\begin{pgfscope}%
\pgfpathrectangle{\pgfqpoint{0.100000in}{0.100000in}}{\pgfqpoint{3.608454in}{2.310000in}}%
\pgfusepath{clip}%
\pgfsetbuttcap%
\pgfsetmiterjoin%
\definecolor{currentfill}{rgb}{0.000000,0.701961,0.649020}%
\pgfsetfillcolor{currentfill}%
\pgfsetlinewidth{0.000000pt}%
\definecolor{currentstroke}{rgb}{0.000000,0.000000,0.000000}%
\pgfsetstrokecolor{currentstroke}%
\pgfsetstrokeopacity{0.000000}%
\pgfsetdash{}{0pt}%
\pgfpathmoveto{\pgfqpoint{2.062941in}{0.991185in}}%
\pgfpathlineto{\pgfqpoint{2.062908in}{0.984282in}}%
\pgfpathlineto{\pgfqpoint{2.055870in}{0.984320in}}%
\pgfpathlineto{\pgfqpoint{2.055634in}{0.956575in}}%
\pgfpathlineto{\pgfqpoint{2.041706in}{0.956720in}}%
\pgfpathlineto{\pgfqpoint{2.014398in}{0.957120in}}%
\pgfpathlineto{\pgfqpoint{2.014740in}{0.977835in}}%
\pgfpathlineto{\pgfqpoint{2.007778in}{0.978034in}}%
\pgfpathlineto{\pgfqpoint{2.007887in}{0.984833in}}%
\pgfpathlineto{\pgfqpoint{1.987413in}{0.985257in}}%
\pgfpathlineto{\pgfqpoint{1.987552in}{0.992168in}}%
\pgfpathlineto{\pgfqpoint{1.980692in}{0.992304in}}%
\pgfpathlineto{\pgfqpoint{1.981120in}{1.013018in}}%
\pgfpathlineto{\pgfqpoint{1.984031in}{1.018827in}}%
\pgfpathlineto{\pgfqpoint{1.988856in}{1.020093in}}%
\pgfpathlineto{\pgfqpoint{1.994760in}{1.012926in}}%
\pgfpathlineto{\pgfqpoint{1.999760in}{1.016432in}}%
\pgfpathlineto{\pgfqpoint{2.009922in}{1.016922in}}%
\pgfpathlineto{\pgfqpoint{2.010212in}{1.033190in}}%
\pgfpathlineto{\pgfqpoint{2.013392in}{1.033129in}}%
\pgfpathlineto{\pgfqpoint{2.013608in}{1.046935in}}%
\pgfpathlineto{\pgfqpoint{2.043347in}{1.046522in}}%
\pgfpathlineto{\pgfqpoint{2.043077in}{1.035565in}}%
\pgfpathlineto{\pgfqpoint{2.036171in}{1.027587in}}%
\pgfpathlineto{\pgfqpoint{2.035603in}{0.998326in}}%
\pgfpathlineto{\pgfqpoint{2.049296in}{0.998169in}}%
\pgfpathlineto{\pgfqpoint{2.049224in}{0.991276in}}%
\pgfpathlineto{\pgfqpoint{2.062941in}{0.991185in}}%
\pgfpathclose%
\pgfusepath{fill}%
\end{pgfscope}%
\begin{pgfscope}%
\pgfpathrectangle{\pgfqpoint{0.100000in}{0.100000in}}{\pgfqpoint{3.608454in}{2.310000in}}%
\pgfusepath{clip}%
\pgfsetbuttcap%
\pgfsetmiterjoin%
\definecolor{currentfill}{rgb}{0.000000,0.580392,0.709804}%
\pgfsetfillcolor{currentfill}%
\pgfsetlinewidth{0.000000pt}%
\definecolor{currentstroke}{rgb}{0.000000,0.000000,0.000000}%
\pgfsetstrokecolor{currentstroke}%
\pgfsetstrokeopacity{0.000000}%
\pgfsetdash{}{0pt}%
\pgfpathmoveto{\pgfqpoint{3.152496in}{1.240772in}}%
\pgfpathlineto{\pgfqpoint{3.187275in}{1.247460in}}%
\pgfpathlineto{\pgfqpoint{3.188912in}{1.234693in}}%
\pgfpathlineto{\pgfqpoint{3.185638in}{1.219606in}}%
\pgfpathlineto{\pgfqpoint{3.177079in}{1.221508in}}%
\pgfpathlineto{\pgfqpoint{3.165876in}{1.221007in}}%
\pgfpathlineto{\pgfqpoint{3.160658in}{1.212280in}}%
\pgfpathlineto{\pgfqpoint{3.154402in}{1.212187in}}%
\pgfpathlineto{\pgfqpoint{3.150361in}{1.233286in}}%
\pgfpathlineto{\pgfqpoint{3.152496in}{1.240772in}}%
\pgfpathclose%
\pgfusepath{fill}%
\end{pgfscope}%
\begin{pgfscope}%
\pgfpathrectangle{\pgfqpoint{0.100000in}{0.100000in}}{\pgfqpoint{3.608454in}{2.310000in}}%
\pgfusepath{clip}%
\pgfsetbuttcap%
\pgfsetmiterjoin%
\definecolor{currentfill}{rgb}{0.000000,0.792157,0.603922}%
\pgfsetfillcolor{currentfill}%
\pgfsetlinewidth{0.000000pt}%
\definecolor{currentstroke}{rgb}{0.000000,0.000000,0.000000}%
\pgfsetstrokecolor{currentstroke}%
\pgfsetstrokeopacity{0.000000}%
\pgfsetdash{}{0pt}%
\pgfpathmoveto{\pgfqpoint{1.183780in}{0.997802in}}%
\pgfpathlineto{\pgfqpoint{1.176800in}{0.953138in}}%
\pgfpathlineto{\pgfqpoint{1.167197in}{0.891666in}}%
\pgfpathlineto{\pgfqpoint{1.092774in}{0.903733in}}%
\pgfpathlineto{\pgfqpoint{1.074172in}{0.906976in}}%
\pgfpathlineto{\pgfqpoint{1.075355in}{0.913752in}}%
\pgfpathlineto{\pgfqpoint{1.084657in}{0.967062in}}%
\pgfpathlineto{\pgfqpoint{1.087666in}{0.981611in}}%
\pgfpathlineto{\pgfqpoint{1.096233in}{0.984306in}}%
\pgfpathlineto{\pgfqpoint{1.106555in}{0.985188in}}%
\pgfpathlineto{\pgfqpoint{1.117590in}{0.983356in}}%
\pgfpathlineto{\pgfqpoint{1.119039in}{0.992055in}}%
\pgfpathlineto{\pgfqpoint{1.125550in}{0.990854in}}%
\pgfpathlineto{\pgfqpoint{1.133001in}{0.981587in}}%
\pgfpathlineto{\pgfqpoint{1.145330in}{0.990568in}}%
\pgfpathlineto{\pgfqpoint{1.150108in}{0.990389in}}%
\pgfpathlineto{\pgfqpoint{1.154422in}{0.995185in}}%
\pgfpathlineto{\pgfqpoint{1.164171in}{1.000833in}}%
\pgfpathlineto{\pgfqpoint{1.183780in}{0.997802in}}%
\pgfpathclose%
\pgfusepath{fill}%
\end{pgfscope}%
\begin{pgfscope}%
\pgfpathrectangle{\pgfqpoint{0.100000in}{0.100000in}}{\pgfqpoint{3.608454in}{2.310000in}}%
\pgfusepath{clip}%
\pgfsetbuttcap%
\pgfsetmiterjoin%
\definecolor{currentfill}{rgb}{0.000000,0.760784,0.619608}%
\pgfsetfillcolor{currentfill}%
\pgfsetlinewidth{0.000000pt}%
\definecolor{currentstroke}{rgb}{0.000000,0.000000,0.000000}%
\pgfsetstrokecolor{currentstroke}%
\pgfsetstrokeopacity{0.000000}%
\pgfsetdash{}{0pt}%
\pgfpathmoveto{\pgfqpoint{2.056507in}{0.745639in}}%
\pgfpathlineto{\pgfqpoint{2.061056in}{0.729018in}}%
\pgfpathlineto{\pgfqpoint{2.057335in}{0.727032in}}%
\pgfpathlineto{\pgfqpoint{2.053852in}{0.716384in}}%
\pgfpathlineto{\pgfqpoint{2.054567in}{0.712981in}}%
\pgfpathlineto{\pgfqpoint{2.040106in}{0.712927in}}%
\pgfpathlineto{\pgfqpoint{2.021875in}{0.703710in}}%
\pgfpathlineto{\pgfqpoint{2.016298in}{0.728776in}}%
\pgfpathlineto{\pgfqpoint{2.016972in}{0.734385in}}%
\pgfpathlineto{\pgfqpoint{1.997875in}{0.723825in}}%
\pgfpathlineto{\pgfqpoint{1.975832in}{0.763164in}}%
\pgfpathlineto{\pgfqpoint{1.990434in}{0.771318in}}%
\pgfpathlineto{\pgfqpoint{1.978879in}{0.792186in}}%
\pgfpathlineto{\pgfqpoint{2.013777in}{0.811827in}}%
\pgfpathlineto{\pgfqpoint{2.023652in}{0.805506in}}%
\pgfpathlineto{\pgfqpoint{2.034850in}{0.788420in}}%
\pgfpathlineto{\pgfqpoint{2.042310in}{0.767446in}}%
\pgfpathlineto{\pgfqpoint{2.047446in}{0.765794in}}%
\pgfpathlineto{\pgfqpoint{2.048151in}{0.760702in}}%
\pgfpathlineto{\pgfqpoint{2.058523in}{0.753655in}}%
\pgfpathlineto{\pgfqpoint{2.056507in}{0.745639in}}%
\pgfpathclose%
\pgfusepath{fill}%
\end{pgfscope}%
\begin{pgfscope}%
\pgfpathrectangle{\pgfqpoint{0.100000in}{0.100000in}}{\pgfqpoint{3.608454in}{2.310000in}}%
\pgfusepath{clip}%
\pgfsetbuttcap%
\pgfsetmiterjoin%
\definecolor{currentfill}{rgb}{0.000000,0.466667,0.766667}%
\pgfsetfillcolor{currentfill}%
\pgfsetlinewidth{0.000000pt}%
\definecolor{currentstroke}{rgb}{0.000000,0.000000,0.000000}%
\pgfsetstrokecolor{currentstroke}%
\pgfsetstrokeopacity{0.000000}%
\pgfsetdash{}{0pt}%
\pgfpathmoveto{\pgfqpoint{1.951799in}{2.072044in}}%
\pgfpathlineto{\pgfqpoint{1.952493in}{2.057946in}}%
\pgfpathlineto{\pgfqpoint{1.951584in}{2.030204in}}%
\pgfpathlineto{\pgfqpoint{1.946085in}{2.030374in}}%
\pgfpathlineto{\pgfqpoint{1.918355in}{2.031373in}}%
\pgfpathlineto{\pgfqpoint{1.917529in}{2.045335in}}%
\pgfpathlineto{\pgfqpoint{1.876114in}{2.047161in}}%
\pgfpathlineto{\pgfqpoint{1.848523in}{2.048601in}}%
\pgfpathlineto{\pgfqpoint{1.847326in}{2.062601in}}%
\pgfpathlineto{\pgfqpoint{1.848899in}{2.090580in}}%
\pgfpathlineto{\pgfqpoint{1.867703in}{2.089585in}}%
\pgfpathlineto{\pgfqpoint{1.895459in}{2.088252in}}%
\pgfpathlineto{\pgfqpoint{1.896095in}{2.102164in}}%
\pgfpathlineto{\pgfqpoint{1.930695in}{2.100743in}}%
\pgfpathlineto{\pgfqpoint{1.931568in}{2.086739in}}%
\pgfpathlineto{\pgfqpoint{1.931043in}{2.072760in}}%
\pgfpathlineto{\pgfqpoint{1.951799in}{2.072044in}}%
\pgfpathclose%
\pgfusepath{fill}%
\end{pgfscope}%
\begin{pgfscope}%
\pgfpathrectangle{\pgfqpoint{0.100000in}{0.100000in}}{\pgfqpoint{3.608454in}{2.310000in}}%
\pgfusepath{clip}%
\pgfsetbuttcap%
\pgfsetmiterjoin%
\definecolor{currentfill}{rgb}{0.000000,0.654902,0.672549}%
\pgfsetfillcolor{currentfill}%
\pgfsetlinewidth{0.000000pt}%
\definecolor{currentstroke}{rgb}{0.000000,0.000000,0.000000}%
\pgfsetstrokecolor{currentstroke}%
\pgfsetstrokeopacity{0.000000}%
\pgfsetdash{}{0pt}%
\pgfpathmoveto{\pgfqpoint{2.452854in}{0.589739in}}%
\pgfpathlineto{\pgfqpoint{2.452510in}{0.580153in}}%
\pgfpathlineto{\pgfqpoint{2.447694in}{0.578303in}}%
\pgfpathlineto{\pgfqpoint{2.452088in}{0.571325in}}%
\pgfpathlineto{\pgfqpoint{2.450488in}{0.567507in}}%
\pgfpathlineto{\pgfqpoint{2.441407in}{0.560816in}}%
\pgfpathlineto{\pgfqpoint{2.437817in}{0.565645in}}%
\pgfpathlineto{\pgfqpoint{2.435406in}{0.574170in}}%
\pgfpathlineto{\pgfqpoint{2.431293in}{0.576903in}}%
\pgfpathlineto{\pgfqpoint{2.428352in}{0.570405in}}%
\pgfpathlineto{\pgfqpoint{2.421368in}{0.576035in}}%
\pgfpathlineto{\pgfqpoint{2.411945in}{0.568422in}}%
\pgfpathlineto{\pgfqpoint{2.411612in}{0.562816in}}%
\pgfpathlineto{\pgfqpoint{2.405732in}{0.562002in}}%
\pgfpathlineto{\pgfqpoint{2.397767in}{0.557468in}}%
\pgfpathlineto{\pgfqpoint{2.389463in}{0.565906in}}%
\pgfpathlineto{\pgfqpoint{2.380049in}{0.566008in}}%
\pgfpathlineto{\pgfqpoint{2.367175in}{0.570371in}}%
\pgfpathlineto{\pgfqpoint{2.363150in}{0.573969in}}%
\pgfpathlineto{\pgfqpoint{2.369537in}{0.580614in}}%
\pgfpathlineto{\pgfqpoint{2.366976in}{0.588543in}}%
\pgfpathlineto{\pgfqpoint{2.347924in}{0.595251in}}%
\pgfpathlineto{\pgfqpoint{2.341614in}{0.599740in}}%
\pgfpathlineto{\pgfqpoint{2.341637in}{0.608452in}}%
\pgfpathlineto{\pgfqpoint{2.330757in}{0.608626in}}%
\pgfpathlineto{\pgfqpoint{2.323648in}{0.606472in}}%
\pgfpathlineto{\pgfqpoint{2.335052in}{0.617153in}}%
\pgfpathlineto{\pgfqpoint{2.340753in}{0.625307in}}%
\pgfpathlineto{\pgfqpoint{2.351303in}{0.626566in}}%
\pgfpathlineto{\pgfqpoint{2.358501in}{0.617112in}}%
\pgfpathlineto{\pgfqpoint{2.357771in}{0.613019in}}%
\pgfpathlineto{\pgfqpoint{2.372197in}{0.609200in}}%
\pgfpathlineto{\pgfqpoint{2.378486in}{0.611623in}}%
\pgfpathlineto{\pgfqpoint{2.376744in}{0.617237in}}%
\pgfpathlineto{\pgfqpoint{2.371581in}{0.618635in}}%
\pgfpathlineto{\pgfqpoint{2.365882in}{0.630690in}}%
\pgfpathlineto{\pgfqpoint{2.368222in}{0.632799in}}%
\pgfpathlineto{\pgfqpoint{2.383161in}{0.637744in}}%
\pgfpathlineto{\pgfqpoint{2.386115in}{0.636858in}}%
\pgfpathlineto{\pgfqpoint{2.386695in}{0.629800in}}%
\pgfpathlineto{\pgfqpoint{2.392099in}{0.624223in}}%
\pgfpathlineto{\pgfqpoint{2.416070in}{0.624294in}}%
\pgfpathlineto{\pgfqpoint{2.420890in}{0.618181in}}%
\pgfpathlineto{\pgfqpoint{2.428180in}{0.614445in}}%
\pgfpathlineto{\pgfqpoint{2.430825in}{0.609081in}}%
\pgfpathlineto{\pgfqpoint{2.440322in}{0.609463in}}%
\pgfpathlineto{\pgfqpoint{2.443757in}{0.607123in}}%
\pgfpathlineto{\pgfqpoint{2.442006in}{0.599263in}}%
\pgfpathlineto{\pgfqpoint{2.445754in}{0.592557in}}%
\pgfpathlineto{\pgfqpoint{2.452854in}{0.589739in}}%
\pgfpathclose%
\pgfusepath{fill}%
\end{pgfscope}%
\begin{pgfscope}%
\pgfpathrectangle{\pgfqpoint{0.100000in}{0.100000in}}{\pgfqpoint{3.608454in}{2.310000in}}%
\pgfusepath{clip}%
\pgfsetbuttcap%
\pgfsetmiterjoin%
\definecolor{currentfill}{rgb}{0.000000,0.568627,0.715686}%
\pgfsetfillcolor{currentfill}%
\pgfsetlinewidth{0.000000pt}%
\definecolor{currentstroke}{rgb}{0.000000,0.000000,0.000000}%
\pgfsetstrokecolor{currentstroke}%
\pgfsetstrokeopacity{0.000000}%
\pgfsetdash{}{0pt}%
\pgfpathmoveto{\pgfqpoint{1.782538in}{1.034546in}}%
\pgfpathlineto{\pgfqpoint{1.785102in}{1.081107in}}%
\pgfpathlineto{\pgfqpoint{1.786242in}{1.101806in}}%
\pgfpathlineto{\pgfqpoint{1.792285in}{1.107896in}}%
\pgfpathlineto{\pgfqpoint{1.798623in}{1.104938in}}%
\pgfpathlineto{\pgfqpoint{1.804299in}{1.099161in}}%
\pgfpathlineto{\pgfqpoint{1.814486in}{1.099133in}}%
\pgfpathlineto{\pgfqpoint{1.818722in}{1.107444in}}%
\pgfpathlineto{\pgfqpoint{1.826418in}{1.110191in}}%
\pgfpathlineto{\pgfqpoint{1.825972in}{1.094304in}}%
\pgfpathlineto{\pgfqpoint{1.873684in}{1.092215in}}%
\pgfpathlineto{\pgfqpoint{1.873435in}{1.071561in}}%
\pgfpathlineto{\pgfqpoint{1.872137in}{1.035604in}}%
\pgfpathlineto{\pgfqpoint{1.863914in}{1.037452in}}%
\pgfpathlineto{\pgfqpoint{1.796992in}{1.040622in}}%
\pgfpathlineto{\pgfqpoint{1.796623in}{1.033765in}}%
\pgfpathlineto{\pgfqpoint{1.782538in}{1.034546in}}%
\pgfpathclose%
\pgfusepath{fill}%
\end{pgfscope}%
\begin{pgfscope}%
\pgfpathrectangle{\pgfqpoint{0.100000in}{0.100000in}}{\pgfqpoint{3.608454in}{2.310000in}}%
\pgfusepath{clip}%
\pgfsetbuttcap%
\pgfsetmiterjoin%
\definecolor{currentfill}{rgb}{0.000000,0.682353,0.658824}%
\pgfsetfillcolor{currentfill}%
\pgfsetlinewidth{0.000000pt}%
\definecolor{currentstroke}{rgb}{0.000000,0.000000,0.000000}%
\pgfsetstrokecolor{currentstroke}%
\pgfsetstrokeopacity{0.000000}%
\pgfsetdash{}{0pt}%
\pgfpathmoveto{\pgfqpoint{3.008758in}{1.422029in}}%
\pgfpathlineto{\pgfqpoint{3.021096in}{1.413057in}}%
\pgfpathlineto{\pgfqpoint{3.018667in}{1.406375in}}%
\pgfpathlineto{\pgfqpoint{3.019598in}{1.394420in}}%
\pgfpathlineto{\pgfqpoint{3.012891in}{1.393149in}}%
\pgfpathlineto{\pgfqpoint{3.010884in}{1.388904in}}%
\pgfpathlineto{\pgfqpoint{3.001117in}{1.390072in}}%
\pgfpathlineto{\pgfqpoint{2.997346in}{1.397787in}}%
\pgfpathlineto{\pgfqpoint{2.989078in}{1.401477in}}%
\pgfpathlineto{\pgfqpoint{2.980404in}{1.409102in}}%
\pgfpathlineto{\pgfqpoint{2.978628in}{1.415804in}}%
\pgfpathlineto{\pgfqpoint{2.986872in}{1.422883in}}%
\pgfpathlineto{\pgfqpoint{3.004458in}{1.420613in}}%
\pgfpathlineto{\pgfqpoint{3.008758in}{1.422029in}}%
\pgfpathclose%
\pgfusepath{fill}%
\end{pgfscope}%
\begin{pgfscope}%
\pgfpathrectangle{\pgfqpoint{0.100000in}{0.100000in}}{\pgfqpoint{3.608454in}{2.310000in}}%
\pgfusepath{clip}%
\pgfsetbuttcap%
\pgfsetmiterjoin%
\definecolor{currentfill}{rgb}{0.000000,0.478431,0.760784}%
\pgfsetfillcolor{currentfill}%
\pgfsetlinewidth{0.000000pt}%
\definecolor{currentstroke}{rgb}{0.000000,0.000000,0.000000}%
\pgfsetstrokecolor{currentstroke}%
\pgfsetstrokeopacity{0.000000}%
\pgfsetdash{}{0pt}%
\pgfpathmoveto{\pgfqpoint{2.814875in}{1.291146in}}%
\pgfpathlineto{\pgfqpoint{2.810307in}{1.281879in}}%
\pgfpathlineto{\pgfqpoint{2.801059in}{1.283583in}}%
\pgfpathlineto{\pgfqpoint{2.791570in}{1.288910in}}%
\pgfpathlineto{\pgfqpoint{2.774318in}{1.293523in}}%
\pgfpathlineto{\pgfqpoint{2.759930in}{1.280063in}}%
\pgfpathlineto{\pgfqpoint{2.757404in}{1.276520in}}%
\pgfpathlineto{\pgfqpoint{2.740147in}{1.292411in}}%
\pgfpathlineto{\pgfqpoint{2.733018in}{1.291688in}}%
\pgfpathlineto{\pgfqpoint{2.727518in}{1.286122in}}%
\pgfpathlineto{\pgfqpoint{2.719889in}{1.285770in}}%
\pgfpathlineto{\pgfqpoint{2.718323in}{1.291481in}}%
\pgfpathlineto{\pgfqpoint{2.724571in}{1.300390in}}%
\pgfpathlineto{\pgfqpoint{2.726400in}{1.307715in}}%
\pgfpathlineto{\pgfqpoint{2.726160in}{1.316456in}}%
\pgfpathlineto{\pgfqpoint{2.730992in}{1.324932in}}%
\pgfpathlineto{\pgfqpoint{2.724859in}{1.333241in}}%
\pgfpathlineto{\pgfqpoint{2.722897in}{1.338929in}}%
\pgfpathlineto{\pgfqpoint{2.718732in}{1.341235in}}%
\pgfpathlineto{\pgfqpoint{2.726746in}{1.347306in}}%
\pgfpathlineto{\pgfqpoint{2.735344in}{1.352880in}}%
\pgfpathlineto{\pgfqpoint{2.736943in}{1.344983in}}%
\pgfpathlineto{\pgfqpoint{2.750154in}{1.334910in}}%
\pgfpathlineto{\pgfqpoint{2.760419in}{1.343088in}}%
\pgfpathlineto{\pgfqpoint{2.772188in}{1.346181in}}%
\pgfpathlineto{\pgfqpoint{2.779558in}{1.348364in}}%
\pgfpathlineto{\pgfqpoint{2.786720in}{1.347970in}}%
\pgfpathlineto{\pgfqpoint{2.789928in}{1.341177in}}%
\pgfpathlineto{\pgfqpoint{2.787097in}{1.337635in}}%
\pgfpathlineto{\pgfqpoint{2.796848in}{1.326453in}}%
\pgfpathlineto{\pgfqpoint{2.784277in}{1.310449in}}%
\pgfpathlineto{\pgfqpoint{2.793467in}{1.297141in}}%
\pgfpathlineto{\pgfqpoint{2.806402in}{1.296003in}}%
\pgfpathlineto{\pgfqpoint{2.814875in}{1.291146in}}%
\pgfpathclose%
\pgfusepath{fill}%
\end{pgfscope}%
\begin{pgfscope}%
\pgfpathrectangle{\pgfqpoint{0.100000in}{0.100000in}}{\pgfqpoint{3.608454in}{2.310000in}}%
\pgfusepath{clip}%
\pgfsetbuttcap%
\pgfsetmiterjoin%
\definecolor{currentfill}{rgb}{0.000000,0.619608,0.690196}%
\pgfsetfillcolor{currentfill}%
\pgfsetlinewidth{0.000000pt}%
\definecolor{currentstroke}{rgb}{0.000000,0.000000,0.000000}%
\pgfsetstrokecolor{currentstroke}%
\pgfsetstrokeopacity{0.000000}%
\pgfsetdash{}{0pt}%
\pgfpathmoveto{\pgfqpoint{2.657467in}{1.463709in}}%
\pgfpathlineto{\pgfqpoint{2.634617in}{1.461203in}}%
\pgfpathlineto{\pgfqpoint{2.633399in}{1.473636in}}%
\pgfpathlineto{\pgfqpoint{2.625004in}{1.477436in}}%
\pgfpathlineto{\pgfqpoint{2.623185in}{1.497972in}}%
\pgfpathlineto{\pgfqpoint{2.614139in}{1.497101in}}%
\pgfpathlineto{\pgfqpoint{2.610441in}{1.500202in}}%
\pgfpathlineto{\pgfqpoint{2.609172in}{1.513996in}}%
\pgfpathlineto{\pgfqpoint{2.588380in}{1.512229in}}%
\pgfpathlineto{\pgfqpoint{2.586580in}{1.532701in}}%
\pgfpathlineto{\pgfqpoint{2.614051in}{1.535158in}}%
\pgfpathlineto{\pgfqpoint{2.615905in}{1.514539in}}%
\pgfpathlineto{\pgfqpoint{2.633720in}{1.516230in}}%
\pgfpathlineto{\pgfqpoint{2.633101in}{1.523054in}}%
\pgfpathlineto{\pgfqpoint{2.646259in}{1.524584in}}%
\pgfpathlineto{\pgfqpoint{2.649395in}{1.497238in}}%
\pgfpathlineto{\pgfqpoint{2.653860in}{1.497742in}}%
\pgfpathlineto{\pgfqpoint{2.657467in}{1.463709in}}%
\pgfpathclose%
\pgfusepath{fill}%
\end{pgfscope}%
\begin{pgfscope}%
\pgfpathrectangle{\pgfqpoint{0.100000in}{0.100000in}}{\pgfqpoint{3.608454in}{2.310000in}}%
\pgfusepath{clip}%
\pgfsetbuttcap%
\pgfsetmiterjoin%
\definecolor{currentfill}{rgb}{0.000000,0.713725,0.643137}%
\pgfsetfillcolor{currentfill}%
\pgfsetlinewidth{0.000000pt}%
\definecolor{currentstroke}{rgb}{0.000000,0.000000,0.000000}%
\pgfsetstrokecolor{currentstroke}%
\pgfsetstrokeopacity{0.000000}%
\pgfsetdash{}{0pt}%
\pgfpathmoveto{\pgfqpoint{2.356606in}{1.028702in}}%
\pgfpathlineto{\pgfqpoint{2.347372in}{1.026940in}}%
\pgfpathlineto{\pgfqpoint{2.342574in}{1.032044in}}%
\pgfpathlineto{\pgfqpoint{2.335892in}{1.034757in}}%
\pgfpathlineto{\pgfqpoint{2.328181in}{1.034618in}}%
\pgfpathlineto{\pgfqpoint{2.328311in}{1.029359in}}%
\pgfpathlineto{\pgfqpoint{2.320051in}{1.032034in}}%
\pgfpathlineto{\pgfqpoint{2.314200in}{1.029274in}}%
\pgfpathlineto{\pgfqpoint{2.309489in}{1.028555in}}%
\pgfpathlineto{\pgfqpoint{2.300760in}{1.031780in}}%
\pgfpathlineto{\pgfqpoint{2.293540in}{1.031438in}}%
\pgfpathlineto{\pgfqpoint{2.293330in}{1.054914in}}%
\pgfpathlineto{\pgfqpoint{2.310094in}{1.055119in}}%
\pgfpathlineto{\pgfqpoint{2.313444in}{1.062948in}}%
\pgfpathlineto{\pgfqpoint{2.313347in}{1.069167in}}%
\pgfpathlineto{\pgfqpoint{2.340700in}{1.069582in}}%
\pgfpathlineto{\pgfqpoint{2.340391in}{1.078576in}}%
\pgfpathlineto{\pgfqpoint{2.343663in}{1.088662in}}%
\pgfpathlineto{\pgfqpoint{2.350507in}{1.098801in}}%
\pgfpathlineto{\pgfqpoint{2.361164in}{1.098864in}}%
\pgfpathlineto{\pgfqpoint{2.362249in}{1.063873in}}%
\pgfpathlineto{\pgfqpoint{2.363393in}{1.028869in}}%
\pgfpathlineto{\pgfqpoint{2.356606in}{1.028702in}}%
\pgfpathclose%
\pgfusepath{fill}%
\end{pgfscope}%
\begin{pgfscope}%
\pgfpathrectangle{\pgfqpoint{0.100000in}{0.100000in}}{\pgfqpoint{3.608454in}{2.310000in}}%
\pgfusepath{clip}%
\pgfsetbuttcap%
\pgfsetmiterjoin%
\definecolor{currentfill}{rgb}{0.000000,0.337255,0.831373}%
\pgfsetfillcolor{currentfill}%
\pgfsetlinewidth{0.000000pt}%
\definecolor{currentstroke}{rgb}{0.000000,0.000000,0.000000}%
\pgfsetstrokecolor{currentstroke}%
\pgfsetstrokeopacity{0.000000}%
\pgfsetdash{}{0pt}%
\pgfpathmoveto{\pgfqpoint{1.844714in}{0.901800in}}%
\pgfpathlineto{\pgfqpoint{1.846276in}{0.936427in}}%
\pgfpathlineto{\pgfqpoint{1.847621in}{0.966431in}}%
\pgfpathlineto{\pgfqpoint{1.852516in}{0.961694in}}%
\pgfpathlineto{\pgfqpoint{1.864563in}{0.959406in}}%
\pgfpathlineto{\pgfqpoint{1.869882in}{0.961093in}}%
\pgfpathlineto{\pgfqpoint{1.877716in}{0.953284in}}%
\pgfpathlineto{\pgfqpoint{1.883512in}{0.956032in}}%
\pgfpathlineto{\pgfqpoint{1.885921in}{0.960469in}}%
\pgfpathlineto{\pgfqpoint{1.898581in}{0.956614in}}%
\pgfpathlineto{\pgfqpoint{1.906317in}{0.946847in}}%
\pgfpathlineto{\pgfqpoint{1.913020in}{0.946294in}}%
\pgfpathlineto{\pgfqpoint{1.910664in}{0.938402in}}%
\pgfpathlineto{\pgfqpoint{1.909495in}{0.904858in}}%
\pgfpathlineto{\pgfqpoint{1.880200in}{0.905864in}}%
\pgfpathlineto{\pgfqpoint{1.880039in}{0.900259in}}%
\pgfpathlineto{\pgfqpoint{1.844714in}{0.901800in}}%
\pgfpathclose%
\pgfusepath{fill}%
\end{pgfscope}%
\begin{pgfscope}%
\pgfpathrectangle{\pgfqpoint{0.100000in}{0.100000in}}{\pgfqpoint{3.608454in}{2.310000in}}%
\pgfusepath{clip}%
\pgfsetbuttcap%
\pgfsetmiterjoin%
\definecolor{currentfill}{rgb}{0.000000,0.627451,0.686275}%
\pgfsetfillcolor{currentfill}%
\pgfsetlinewidth{0.000000pt}%
\definecolor{currentstroke}{rgb}{0.000000,0.000000,0.000000}%
\pgfsetstrokecolor{currentstroke}%
\pgfsetstrokeopacity{0.000000}%
\pgfsetdash{}{0pt}%
\pgfpathmoveto{\pgfqpoint{2.227204in}{1.266822in}}%
\pgfpathlineto{\pgfqpoint{2.199885in}{1.267142in}}%
\pgfpathlineto{\pgfqpoint{2.199587in}{1.254527in}}%
\pgfpathlineto{\pgfqpoint{2.196058in}{1.254577in}}%
\pgfpathlineto{\pgfqpoint{2.192435in}{1.247763in}}%
\pgfpathlineto{\pgfqpoint{2.180991in}{1.248016in}}%
\pgfpathlineto{\pgfqpoint{2.181129in}{1.252610in}}%
\pgfpathlineto{\pgfqpoint{2.165075in}{1.253073in}}%
\pgfpathlineto{\pgfqpoint{2.165424in}{1.263791in}}%
\pgfpathlineto{\pgfqpoint{2.164228in}{1.277853in}}%
\pgfpathlineto{\pgfqpoint{2.164617in}{1.305750in}}%
\pgfpathlineto{\pgfqpoint{2.198808in}{1.305319in}}%
\pgfpathlineto{\pgfqpoint{2.199219in}{1.335898in}}%
\pgfpathlineto{\pgfqpoint{2.226654in}{1.335261in}}%
\pgfpathlineto{\pgfqpoint{2.226458in}{1.316665in}}%
\pgfpathlineto{\pgfqpoint{2.225942in}{1.276603in}}%
\pgfpathlineto{\pgfqpoint{2.227204in}{1.266822in}}%
\pgfpathclose%
\pgfusepath{fill}%
\end{pgfscope}%
\begin{pgfscope}%
\pgfpathrectangle{\pgfqpoint{0.100000in}{0.100000in}}{\pgfqpoint{3.608454in}{2.310000in}}%
\pgfusepath{clip}%
\pgfsetbuttcap%
\pgfsetmiterjoin%
\definecolor{currentfill}{rgb}{0.000000,0.674510,0.662745}%
\pgfsetfillcolor{currentfill}%
\pgfsetlinewidth{0.000000pt}%
\definecolor{currentstroke}{rgb}{0.000000,0.000000,0.000000}%
\pgfsetstrokecolor{currentstroke}%
\pgfsetstrokeopacity{0.000000}%
\pgfsetdash{}{0pt}%
\pgfpathmoveto{\pgfqpoint{0.499716in}{1.160624in}}%
\pgfpathlineto{\pgfqpoint{0.490524in}{1.164969in}}%
\pgfpathlineto{\pgfqpoint{0.493536in}{1.171597in}}%
\pgfpathlineto{\pgfqpoint{0.503853in}{1.164499in}}%
\pgfpathlineto{\pgfqpoint{0.499716in}{1.160624in}}%
\pgfpathclose%
\pgfusepath{fill}%
\end{pgfscope}%
\begin{pgfscope}%
\pgfpathrectangle{\pgfqpoint{0.100000in}{0.100000in}}{\pgfqpoint{3.608454in}{2.310000in}}%
\pgfusepath{clip}%
\pgfsetbuttcap%
\pgfsetmiterjoin%
\definecolor{currentfill}{rgb}{0.000000,0.674510,0.662745}%
\pgfsetfillcolor{currentfill}%
\pgfsetlinewidth{0.000000pt}%
\definecolor{currentstroke}{rgb}{0.000000,0.000000,0.000000}%
\pgfsetstrokecolor{currentstroke}%
\pgfsetstrokeopacity{0.000000}%
\pgfsetdash{}{0pt}%
\pgfpathmoveto{\pgfqpoint{0.473802in}{1.162641in}}%
\pgfpathlineto{\pgfqpoint{0.468569in}{1.171802in}}%
\pgfpathlineto{\pgfqpoint{0.474440in}{1.173296in}}%
\pgfpathlineto{\pgfqpoint{0.483849in}{1.167231in}}%
\pgfpathlineto{\pgfqpoint{0.482039in}{1.164194in}}%
\pgfpathlineto{\pgfqpoint{0.473802in}{1.162641in}}%
\pgfpathclose%
\pgfusepath{fill}%
\end{pgfscope}%
\begin{pgfscope}%
\pgfpathrectangle{\pgfqpoint{0.100000in}{0.100000in}}{\pgfqpoint{3.608454in}{2.310000in}}%
\pgfusepath{clip}%
\pgfsetbuttcap%
\pgfsetmiterjoin%
\definecolor{currentfill}{rgb}{0.000000,0.674510,0.662745}%
\pgfsetfillcolor{currentfill}%
\pgfsetlinewidth{0.000000pt}%
\definecolor{currentstroke}{rgb}{0.000000,0.000000,0.000000}%
\pgfsetstrokecolor{currentstroke}%
\pgfsetstrokeopacity{0.000000}%
\pgfsetdash{}{0pt}%
\pgfpathmoveto{\pgfqpoint{0.537333in}{1.227732in}}%
\pgfpathlineto{\pgfqpoint{0.527921in}{1.193369in}}%
\pgfpathlineto{\pgfqpoint{0.524045in}{1.187970in}}%
\pgfpathlineto{\pgfqpoint{0.516406in}{1.193887in}}%
\pgfpathlineto{\pgfqpoint{0.508386in}{1.194136in}}%
\pgfpathlineto{\pgfqpoint{0.499604in}{1.197528in}}%
\pgfpathlineto{\pgfqpoint{0.493082in}{1.203453in}}%
\pgfpathlineto{\pgfqpoint{0.479238in}{1.208664in}}%
\pgfpathlineto{\pgfqpoint{0.464027in}{1.210445in}}%
\pgfpathlineto{\pgfqpoint{0.462181in}{1.217763in}}%
\pgfpathlineto{\pgfqpoint{0.454827in}{1.224336in}}%
\pgfpathlineto{\pgfqpoint{0.460433in}{1.233714in}}%
\pgfpathlineto{\pgfqpoint{0.459304in}{1.237786in}}%
\pgfpathlineto{\pgfqpoint{0.463136in}{1.244048in}}%
\pgfpathlineto{\pgfqpoint{0.460490in}{1.249495in}}%
\pgfpathlineto{\pgfqpoint{0.464659in}{1.256580in}}%
\pgfpathlineto{\pgfqpoint{0.467179in}{1.267774in}}%
\pgfpathlineto{\pgfqpoint{0.460247in}{1.270875in}}%
\pgfpathlineto{\pgfqpoint{0.454219in}{1.280640in}}%
\pgfpathlineto{\pgfqpoint{0.458963in}{1.289133in}}%
\pgfpathlineto{\pgfqpoint{0.457966in}{1.295353in}}%
\pgfpathlineto{\pgfqpoint{0.452130in}{1.297956in}}%
\pgfpathlineto{\pgfqpoint{0.446221in}{1.313638in}}%
\pgfpathlineto{\pgfqpoint{0.439451in}{1.318713in}}%
\pgfpathlineto{\pgfqpoint{0.438474in}{1.329831in}}%
\pgfpathlineto{\pgfqpoint{0.468241in}{1.320678in}}%
\pgfpathlineto{\pgfqpoint{0.508309in}{1.308909in}}%
\pgfpathlineto{\pgfqpoint{0.509532in}{1.308555in}}%
\pgfpathlineto{\pgfqpoint{0.505758in}{1.295236in}}%
\pgfpathlineto{\pgfqpoint{0.512448in}{1.293350in}}%
\pgfpathlineto{\pgfqpoint{0.510504in}{1.286661in}}%
\pgfpathlineto{\pgfqpoint{0.514797in}{1.280686in}}%
\pgfpathlineto{\pgfqpoint{0.521448in}{1.276381in}}%
\pgfpathlineto{\pgfqpoint{0.519557in}{1.269679in}}%
\pgfpathlineto{\pgfqpoint{0.523955in}{1.268417in}}%
\pgfpathlineto{\pgfqpoint{0.522085in}{1.261761in}}%
\pgfpathlineto{\pgfqpoint{0.530958in}{1.259199in}}%
\pgfpathlineto{\pgfqpoint{0.529070in}{1.252510in}}%
\pgfpathlineto{\pgfqpoint{0.536255in}{1.250941in}}%
\pgfpathlineto{\pgfqpoint{0.533846in}{1.244020in}}%
\pgfpathlineto{\pgfqpoint{0.539132in}{1.241671in}}%
\pgfpathlineto{\pgfqpoint{0.537333in}{1.227732in}}%
\pgfpathclose%
\pgfusepath{fill}%
\end{pgfscope}%
\begin{pgfscope}%
\pgfpathrectangle{\pgfqpoint{0.100000in}{0.100000in}}{\pgfqpoint{3.608454in}{2.310000in}}%
\pgfusepath{clip}%
\pgfsetbuttcap%
\pgfsetmiterjoin%
\definecolor{currentfill}{rgb}{0.000000,0.415686,0.792157}%
\pgfsetfillcolor{currentfill}%
\pgfsetlinewidth{0.000000pt}%
\definecolor{currentstroke}{rgb}{0.000000,0.000000,0.000000}%
\pgfsetstrokecolor{currentstroke}%
\pgfsetstrokeopacity{0.000000}%
\pgfsetdash{}{0pt}%
\pgfpathmoveto{\pgfqpoint{3.370781in}{1.834556in}}%
\pgfpathlineto{\pgfqpoint{3.348859in}{1.830296in}}%
\pgfpathlineto{\pgfqpoint{3.344700in}{1.848799in}}%
\pgfpathlineto{\pgfqpoint{3.333464in}{1.857292in}}%
\pgfpathlineto{\pgfqpoint{3.332649in}{1.866641in}}%
\pgfpathlineto{\pgfqpoint{3.324985in}{1.884098in}}%
\pgfpathlineto{\pgfqpoint{3.324873in}{1.895982in}}%
\pgfpathlineto{\pgfqpoint{3.327709in}{1.903478in}}%
\pgfpathlineto{\pgfqpoint{3.324626in}{1.910739in}}%
\pgfpathlineto{\pgfqpoint{3.323703in}{1.920977in}}%
\pgfpathlineto{\pgfqpoint{3.316899in}{1.928969in}}%
\pgfpathlineto{\pgfqpoint{3.316446in}{1.944043in}}%
\pgfpathlineto{\pgfqpoint{3.313379in}{1.946070in}}%
\pgfpathlineto{\pgfqpoint{3.314088in}{1.952894in}}%
\pgfpathlineto{\pgfqpoint{3.312126in}{1.959898in}}%
\pgfpathlineto{\pgfqpoint{3.351137in}{1.969833in}}%
\pgfpathlineto{\pgfqpoint{3.355196in}{1.970349in}}%
\pgfpathlineto{\pgfqpoint{3.359777in}{1.957182in}}%
\pgfpathlineto{\pgfqpoint{3.358298in}{1.952554in}}%
\pgfpathlineto{\pgfqpoint{3.368238in}{1.950258in}}%
\pgfpathlineto{\pgfqpoint{3.366415in}{1.942507in}}%
\pgfpathlineto{\pgfqpoint{3.373495in}{1.940927in}}%
\pgfpathlineto{\pgfqpoint{3.371539in}{1.933332in}}%
\pgfpathlineto{\pgfqpoint{3.370017in}{1.926361in}}%
\pgfpathlineto{\pgfqpoint{3.363137in}{1.929015in}}%
\pgfpathlineto{\pgfqpoint{3.361540in}{1.921976in}}%
\pgfpathlineto{\pgfqpoint{3.352226in}{1.924022in}}%
\pgfpathlineto{\pgfqpoint{3.349620in}{1.909475in}}%
\pgfpathlineto{\pgfqpoint{3.350605in}{1.900061in}}%
\pgfpathlineto{\pgfqpoint{3.353338in}{1.893246in}}%
\pgfpathlineto{\pgfqpoint{3.363839in}{1.892385in}}%
\pgfpathlineto{\pgfqpoint{3.358819in}{1.880708in}}%
\pgfpathlineto{\pgfqpoint{3.354162in}{1.876806in}}%
\pgfpathlineto{\pgfqpoint{3.365877in}{1.874467in}}%
\pgfpathlineto{\pgfqpoint{3.365489in}{1.867233in}}%
\pgfpathlineto{\pgfqpoint{3.372767in}{1.866271in}}%
\pgfpathlineto{\pgfqpoint{3.371356in}{1.850568in}}%
\pgfpathlineto{\pgfqpoint{3.375426in}{1.849437in}}%
\pgfpathlineto{\pgfqpoint{3.375120in}{1.840353in}}%
\pgfpathlineto{\pgfqpoint{3.370781in}{1.834556in}}%
\pgfpathclose%
\pgfusepath{fill}%
\end{pgfscope}%
\begin{pgfscope}%
\pgfpathrectangle{\pgfqpoint{0.100000in}{0.100000in}}{\pgfqpoint{3.608454in}{2.310000in}}%
\pgfusepath{clip}%
\pgfsetbuttcap%
\pgfsetmiterjoin%
\definecolor{currentfill}{rgb}{0.000000,0.745098,0.627451}%
\pgfsetfillcolor{currentfill}%
\pgfsetlinewidth{0.000000pt}%
\definecolor{currentstroke}{rgb}{0.000000,0.000000,0.000000}%
\pgfsetstrokecolor{currentstroke}%
\pgfsetstrokeopacity{0.000000}%
\pgfsetdash{}{0pt}%
\pgfpathmoveto{\pgfqpoint{2.342946in}{0.805709in}}%
\pgfpathlineto{\pgfqpoint{2.340198in}{0.812791in}}%
\pgfpathlineto{\pgfqpoint{2.344014in}{0.814843in}}%
\pgfpathlineto{\pgfqpoint{2.342569in}{0.828917in}}%
\pgfpathlineto{\pgfqpoint{2.344269in}{0.832393in}}%
\pgfpathlineto{\pgfqpoint{2.365973in}{0.833290in}}%
\pgfpathlineto{\pgfqpoint{2.371879in}{0.836595in}}%
\pgfpathlineto{\pgfqpoint{2.377206in}{0.836777in}}%
\pgfpathlineto{\pgfqpoint{2.378401in}{0.826111in}}%
\pgfpathlineto{\pgfqpoint{2.384564in}{0.826478in}}%
\pgfpathlineto{\pgfqpoint{2.384135in}{0.837054in}}%
\pgfpathlineto{\pgfqpoint{2.393171in}{0.837522in}}%
\pgfpathlineto{\pgfqpoint{2.393376in}{0.840961in}}%
\pgfpathlineto{\pgfqpoint{2.404935in}{0.832654in}}%
\pgfpathlineto{\pgfqpoint{2.401669in}{0.830402in}}%
\pgfpathlineto{\pgfqpoint{2.399936in}{0.821863in}}%
\pgfpathlineto{\pgfqpoint{2.395425in}{0.819431in}}%
\pgfpathlineto{\pgfqpoint{2.394278in}{0.809641in}}%
\pgfpathlineto{\pgfqpoint{2.396163in}{0.790004in}}%
\pgfpathlineto{\pgfqpoint{2.395693in}{0.767722in}}%
\pgfpathlineto{\pgfqpoint{2.395986in}{0.760766in}}%
\pgfpathlineto{\pgfqpoint{2.367747in}{0.759421in}}%
\pgfpathlineto{\pgfqpoint{2.357930in}{0.763773in}}%
\pgfpathlineto{\pgfqpoint{2.353993in}{0.777417in}}%
\pgfpathlineto{\pgfqpoint{2.363437in}{0.782994in}}%
\pgfpathlineto{\pgfqpoint{2.364252in}{0.787285in}}%
\pgfpathlineto{\pgfqpoint{2.371635in}{0.792020in}}%
\pgfpathlineto{\pgfqpoint{2.374090in}{0.798791in}}%
\pgfpathlineto{\pgfqpoint{2.365200in}{0.801106in}}%
\pgfpathlineto{\pgfqpoint{2.366944in}{0.810346in}}%
\pgfpathlineto{\pgfqpoint{2.354522in}{0.806095in}}%
\pgfpathlineto{\pgfqpoint{2.342946in}{0.805709in}}%
\pgfpathclose%
\pgfusepath{fill}%
\end{pgfscope}%
\begin{pgfscope}%
\pgfpathrectangle{\pgfqpoint{0.100000in}{0.100000in}}{\pgfqpoint{3.608454in}{2.310000in}}%
\pgfusepath{clip}%
\pgfsetbuttcap%
\pgfsetmiterjoin%
\definecolor{currentfill}{rgb}{0.000000,0.760784,0.619608}%
\pgfsetfillcolor{currentfill}%
\pgfsetlinewidth{0.000000pt}%
\definecolor{currentstroke}{rgb}{0.000000,0.000000,0.000000}%
\pgfsetstrokecolor{currentstroke}%
\pgfsetstrokeopacity{0.000000}%
\pgfsetdash{}{0pt}%
\pgfpathmoveto{\pgfqpoint{0.509532in}{1.308555in}}%
\pgfpathlineto{\pgfqpoint{0.508309in}{1.308909in}}%
\pgfpathlineto{\pgfqpoint{0.503087in}{1.324821in}}%
\pgfpathlineto{\pgfqpoint{0.498712in}{1.326448in}}%
\pgfpathlineto{\pgfqpoint{0.493971in}{1.332372in}}%
\pgfpathlineto{\pgfqpoint{0.488481in}{1.345190in}}%
\pgfpathlineto{\pgfqpoint{0.492171in}{1.356794in}}%
\pgfpathlineto{\pgfqpoint{0.496557in}{1.356685in}}%
\pgfpathlineto{\pgfqpoint{0.499634in}{1.368954in}}%
\pgfpathlineto{\pgfqpoint{0.486008in}{1.393700in}}%
\pgfpathlineto{\pgfqpoint{0.520707in}{1.412942in}}%
\pgfpathlineto{\pgfqpoint{0.533257in}{1.413969in}}%
\pgfpathlineto{\pgfqpoint{0.548331in}{1.412190in}}%
\pgfpathlineto{\pgfqpoint{0.570915in}{1.425063in}}%
\pgfpathlineto{\pgfqpoint{0.577622in}{1.423252in}}%
\pgfpathlineto{\pgfqpoint{0.583298in}{1.427984in}}%
\pgfpathlineto{\pgfqpoint{0.584724in}{1.432975in}}%
\pgfpathlineto{\pgfqpoint{0.605864in}{1.444998in}}%
\pgfpathlineto{\pgfqpoint{0.607467in}{1.441382in}}%
\pgfpathlineto{\pgfqpoint{0.616083in}{1.438599in}}%
\pgfpathlineto{\pgfqpoint{0.618630in}{1.433861in}}%
\pgfpathlineto{\pgfqpoint{0.619173in}{1.425619in}}%
\pgfpathlineto{\pgfqpoint{0.624814in}{1.421196in}}%
\pgfpathlineto{\pgfqpoint{0.627031in}{1.414709in}}%
\pgfpathlineto{\pgfqpoint{0.632163in}{1.410294in}}%
\pgfpathlineto{\pgfqpoint{0.628697in}{1.402513in}}%
\pgfpathlineto{\pgfqpoint{0.632454in}{1.400962in}}%
\pgfpathlineto{\pgfqpoint{0.632393in}{1.385724in}}%
\pgfpathlineto{\pgfqpoint{0.636091in}{1.384439in}}%
\pgfpathlineto{\pgfqpoint{0.643441in}{1.374382in}}%
\pgfpathlineto{\pgfqpoint{0.645186in}{1.361551in}}%
\pgfpathlineto{\pgfqpoint{0.642344in}{1.357942in}}%
\pgfpathlineto{\pgfqpoint{0.641670in}{1.344858in}}%
\pgfpathlineto{\pgfqpoint{0.644925in}{1.333356in}}%
\pgfpathlineto{\pgfqpoint{0.644042in}{1.329081in}}%
\pgfpathlineto{\pgfqpoint{0.650356in}{1.315914in}}%
\pgfpathlineto{\pgfqpoint{0.647795in}{1.312826in}}%
\pgfpathlineto{\pgfqpoint{0.649389in}{1.301776in}}%
\pgfpathlineto{\pgfqpoint{0.647370in}{1.296028in}}%
\pgfpathlineto{\pgfqpoint{0.648713in}{1.279430in}}%
\pgfpathlineto{\pgfqpoint{0.645007in}{1.271759in}}%
\pgfpathlineto{\pgfqpoint{0.606581in}{1.281871in}}%
\pgfpathlineto{\pgfqpoint{0.550101in}{1.297200in}}%
\pgfpathlineto{\pgfqpoint{0.509532in}{1.308555in}}%
\pgfpathclose%
\pgfusepath{fill}%
\end{pgfscope}%
\begin{pgfscope}%
\pgfpathrectangle{\pgfqpoint{0.100000in}{0.100000in}}{\pgfqpoint{3.608454in}{2.310000in}}%
\pgfusepath{clip}%
\pgfsetbuttcap%
\pgfsetmiterjoin%
\definecolor{currentfill}{rgb}{0.000000,0.564706,0.717647}%
\pgfsetfillcolor{currentfill}%
\pgfsetlinewidth{0.000000pt}%
\definecolor{currentstroke}{rgb}{0.000000,0.000000,0.000000}%
\pgfsetstrokecolor{currentstroke}%
\pgfsetstrokeopacity{0.000000}%
\pgfsetdash{}{0pt}%
\pgfpathmoveto{\pgfqpoint{1.007808in}{2.262053in}}%
\pgfpathlineto{\pgfqpoint{1.050960in}{2.251947in}}%
\pgfpathlineto{\pgfqpoint{1.075629in}{2.246526in}}%
\pgfpathlineto{\pgfqpoint{1.074794in}{2.233716in}}%
\pgfpathlineto{\pgfqpoint{1.075971in}{2.228263in}}%
\pgfpathlineto{\pgfqpoint{1.072055in}{2.221371in}}%
\pgfpathlineto{\pgfqpoint{1.061432in}{2.221573in}}%
\pgfpathlineto{\pgfqpoint{1.061972in}{2.214474in}}%
\pgfpathlineto{\pgfqpoint{1.056030in}{2.187273in}}%
\pgfpathlineto{\pgfqpoint{1.047061in}{2.189256in}}%
\pgfpathlineto{\pgfqpoint{1.043899in}{2.172864in}}%
\pgfpathlineto{\pgfqpoint{1.036396in}{2.174561in}}%
\pgfpathlineto{\pgfqpoint{1.033549in}{2.168208in}}%
\pgfpathlineto{\pgfqpoint{1.027896in}{2.166208in}}%
\pgfpathlineto{\pgfqpoint{1.015079in}{2.170487in}}%
\pgfpathlineto{\pgfqpoint{1.015735in}{2.176278in}}%
\pgfpathlineto{\pgfqpoint{1.012094in}{2.185135in}}%
\pgfpathlineto{\pgfqpoint{1.011635in}{2.199142in}}%
\pgfpathlineto{\pgfqpoint{0.994462in}{2.195826in}}%
\pgfpathlineto{\pgfqpoint{0.993310in}{2.200722in}}%
\pgfpathlineto{\pgfqpoint{0.998620in}{2.223097in}}%
\pgfpathlineto{\pgfqpoint{1.007808in}{2.262053in}}%
\pgfpathclose%
\pgfusepath{fill}%
\end{pgfscope}%
\begin{pgfscope}%
\pgfpathrectangle{\pgfqpoint{0.100000in}{0.100000in}}{\pgfqpoint{3.608454in}{2.310000in}}%
\pgfusepath{clip}%
\pgfsetbuttcap%
\pgfsetmiterjoin%
\definecolor{currentfill}{rgb}{0.000000,0.631373,0.684314}%
\pgfsetfillcolor{currentfill}%
\pgfsetlinewidth{0.000000pt}%
\definecolor{currentstroke}{rgb}{0.000000,0.000000,0.000000}%
\pgfsetstrokecolor{currentstroke}%
\pgfsetstrokeopacity{0.000000}%
\pgfsetdash{}{0pt}%
\pgfpathmoveto{\pgfqpoint{2.515690in}{0.998385in}}%
\pgfpathlineto{\pgfqpoint{2.515503in}{1.018069in}}%
\pgfpathlineto{\pgfqpoint{2.512709in}{1.025980in}}%
\pgfpathlineto{\pgfqpoint{2.508135in}{1.025655in}}%
\pgfpathlineto{\pgfqpoint{2.507414in}{1.036455in}}%
\pgfpathlineto{\pgfqpoint{2.509787in}{1.036620in}}%
\pgfpathlineto{\pgfqpoint{2.508734in}{1.056586in}}%
\pgfpathlineto{\pgfqpoint{2.512736in}{1.056859in}}%
\pgfpathlineto{\pgfqpoint{2.517201in}{1.066138in}}%
\pgfpathlineto{\pgfqpoint{2.521534in}{1.068158in}}%
\pgfpathlineto{\pgfqpoint{2.535012in}{1.069064in}}%
\pgfpathlineto{\pgfqpoint{2.534914in}{1.072050in}}%
\pgfpathlineto{\pgfqpoint{2.550818in}{1.070538in}}%
\pgfpathlineto{\pgfqpoint{2.556697in}{1.071568in}}%
\pgfpathlineto{\pgfqpoint{2.560066in}{1.064151in}}%
\pgfpathlineto{\pgfqpoint{2.561622in}{1.041248in}}%
\pgfpathlineto{\pgfqpoint{2.547499in}{1.040339in}}%
\pgfpathlineto{\pgfqpoint{2.554995in}{1.031701in}}%
\pgfpathlineto{\pgfqpoint{2.553769in}{0.997504in}}%
\pgfpathlineto{\pgfqpoint{2.528607in}{0.995823in}}%
\pgfpathlineto{\pgfqpoint{2.528379in}{0.999297in}}%
\pgfpathlineto{\pgfqpoint{2.515690in}{0.998385in}}%
\pgfpathclose%
\pgfusepath{fill}%
\end{pgfscope}%
\begin{pgfscope}%
\pgfpathrectangle{\pgfqpoint{0.100000in}{0.100000in}}{\pgfqpoint{3.608454in}{2.310000in}}%
\pgfusepath{clip}%
\pgfsetbuttcap%
\pgfsetmiterjoin%
\definecolor{currentfill}{rgb}{0.000000,0.992157,0.503922}%
\pgfsetfillcolor{currentfill}%
\pgfsetlinewidth{0.000000pt}%
\definecolor{currentstroke}{rgb}{0.000000,0.000000,0.000000}%
\pgfsetstrokecolor{currentstroke}%
\pgfsetstrokeopacity{0.000000}%
\pgfsetdash{}{0pt}%
\pgfpathmoveto{\pgfqpoint{0.647635in}{2.272305in}}%
\pgfpathlineto{\pgfqpoint{0.648220in}{2.267247in}}%
\pgfpathlineto{\pgfqpoint{0.639739in}{2.262486in}}%
\pgfpathlineto{\pgfqpoint{0.635231in}{2.255252in}}%
\pgfpathlineto{\pgfqpoint{0.623636in}{2.252362in}}%
\pgfpathlineto{\pgfqpoint{0.595222in}{2.261545in}}%
\pgfpathlineto{\pgfqpoint{0.593017in}{2.254746in}}%
\pgfpathlineto{\pgfqpoint{0.565186in}{2.263908in}}%
\pgfpathlineto{\pgfqpoint{0.549782in}{2.270356in}}%
\pgfpathlineto{\pgfqpoint{0.551434in}{2.287191in}}%
\pgfpathlineto{\pgfqpoint{0.549994in}{2.294555in}}%
\pgfpathlineto{\pgfqpoint{0.544785in}{2.302357in}}%
\pgfpathlineto{\pgfqpoint{0.544644in}{2.308702in}}%
\pgfpathlineto{\pgfqpoint{0.548618in}{2.325631in}}%
\pgfpathlineto{\pgfqpoint{0.557234in}{2.339937in}}%
\pgfpathlineto{\pgfqpoint{0.566992in}{2.328262in}}%
\pgfpathlineto{\pgfqpoint{0.573972in}{2.323699in}}%
\pgfpathlineto{\pgfqpoint{0.582903in}{2.313221in}}%
\pgfpathlineto{\pgfqpoint{0.591705in}{2.308913in}}%
\pgfpathlineto{\pgfqpoint{0.598948in}{2.307790in}}%
\pgfpathlineto{\pgfqpoint{0.604150in}{2.302970in}}%
\pgfpathlineto{\pgfqpoint{0.620717in}{2.296209in}}%
\pgfpathlineto{\pgfqpoint{0.627218in}{2.298959in}}%
\pgfpathlineto{\pgfqpoint{0.630726in}{2.290276in}}%
\pgfpathlineto{\pgfqpoint{0.646614in}{2.290179in}}%
\pgfpathlineto{\pgfqpoint{0.648837in}{2.285967in}}%
\pgfpathlineto{\pgfqpoint{0.646729in}{2.278658in}}%
\pgfpathlineto{\pgfqpoint{0.647635in}{2.272305in}}%
\pgfpathclose%
\pgfusepath{fill}%
\end{pgfscope}%
\begin{pgfscope}%
\pgfpathrectangle{\pgfqpoint{0.100000in}{0.100000in}}{\pgfqpoint{3.608454in}{2.310000in}}%
\pgfusepath{clip}%
\pgfsetbuttcap%
\pgfsetmiterjoin%
\definecolor{currentfill}{rgb}{0.000000,0.639216,0.680392}%
\pgfsetfillcolor{currentfill}%
\pgfsetlinewidth{0.000000pt}%
\definecolor{currentstroke}{rgb}{0.000000,0.000000,0.000000}%
\pgfsetstrokecolor{currentstroke}%
\pgfsetstrokeopacity{0.000000}%
\pgfsetdash{}{0pt}%
\pgfpathmoveto{\pgfqpoint{2.726071in}{1.419220in}}%
\pgfpathlineto{\pgfqpoint{2.726448in}{1.415659in}}%
\pgfpathlineto{\pgfqpoint{2.697158in}{1.412713in}}%
\pgfpathlineto{\pgfqpoint{2.694747in}{1.433296in}}%
\pgfpathlineto{\pgfqpoint{2.699598in}{1.433892in}}%
\pgfpathlineto{\pgfqpoint{2.698943in}{1.451000in}}%
\pgfpathlineto{\pgfqpoint{2.722450in}{1.453627in}}%
\pgfpathlineto{\pgfqpoint{2.726071in}{1.419220in}}%
\pgfpathclose%
\pgfusepath{fill}%
\end{pgfscope}%
\begin{pgfscope}%
\pgfpathrectangle{\pgfqpoint{0.100000in}{0.100000in}}{\pgfqpoint{3.608454in}{2.310000in}}%
\pgfusepath{clip}%
\pgfsetbuttcap%
\pgfsetmiterjoin%
\definecolor{currentfill}{rgb}{0.000000,0.529412,0.735294}%
\pgfsetfillcolor{currentfill}%
\pgfsetlinewidth{0.000000pt}%
\definecolor{currentstroke}{rgb}{0.000000,0.000000,0.000000}%
\pgfsetstrokecolor{currentstroke}%
\pgfsetstrokeopacity{0.000000}%
\pgfsetdash{}{0pt}%
\pgfpathmoveto{\pgfqpoint{2.266038in}{1.833884in}}%
\pgfpathlineto{\pgfqpoint{2.232360in}{1.832947in}}%
\pgfpathlineto{\pgfqpoint{2.232161in}{1.839416in}}%
\pgfpathlineto{\pgfqpoint{2.237996in}{1.848119in}}%
\pgfpathlineto{\pgfqpoint{2.238078in}{1.851709in}}%
\pgfpathlineto{\pgfqpoint{2.230979in}{1.861346in}}%
\pgfpathlineto{\pgfqpoint{2.224641in}{1.861874in}}%
\pgfpathlineto{\pgfqpoint{2.226726in}{1.874225in}}%
\pgfpathlineto{\pgfqpoint{2.229731in}{1.876962in}}%
\pgfpathlineto{\pgfqpoint{2.233501in}{1.887214in}}%
\pgfpathlineto{\pgfqpoint{2.237467in}{1.890615in}}%
\pgfpathlineto{\pgfqpoint{2.253310in}{1.897735in}}%
\pgfpathlineto{\pgfqpoint{2.256290in}{1.902446in}}%
\pgfpathlineto{\pgfqpoint{2.256111in}{1.909051in}}%
\pgfpathlineto{\pgfqpoint{2.269571in}{1.909501in}}%
\pgfpathlineto{\pgfqpoint{2.270918in}{1.895690in}}%
\pgfpathlineto{\pgfqpoint{2.271935in}{1.868329in}}%
\pgfpathlineto{\pgfqpoint{2.265057in}{1.868094in}}%
\pgfpathlineto{\pgfqpoint{2.266038in}{1.833884in}}%
\pgfpathclose%
\pgfusepath{fill}%
\end{pgfscope}%
\begin{pgfscope}%
\pgfpathrectangle{\pgfqpoint{0.100000in}{0.100000in}}{\pgfqpoint{3.608454in}{2.310000in}}%
\pgfusepath{clip}%
\pgfsetbuttcap%
\pgfsetmiterjoin%
\definecolor{currentfill}{rgb}{0.000000,0.560784,0.719608}%
\pgfsetfillcolor{currentfill}%
\pgfsetlinewidth{0.000000pt}%
\definecolor{currentstroke}{rgb}{0.000000,0.000000,0.000000}%
\pgfsetstrokecolor{currentstroke}%
\pgfsetstrokeopacity{0.000000}%
\pgfsetdash{}{0pt}%
\pgfpathmoveto{\pgfqpoint{2.442684in}{0.864247in}}%
\pgfpathlineto{\pgfqpoint{2.438751in}{0.854506in}}%
\pgfpathlineto{\pgfqpoint{2.428148in}{0.846306in}}%
\pgfpathlineto{\pgfqpoint{2.412628in}{0.839279in}}%
\pgfpathlineto{\pgfqpoint{2.404935in}{0.832654in}}%
\pgfpathlineto{\pgfqpoint{2.393376in}{0.840961in}}%
\pgfpathlineto{\pgfqpoint{2.393315in}{0.855864in}}%
\pgfpathlineto{\pgfqpoint{2.389913in}{0.858059in}}%
\pgfpathlineto{\pgfqpoint{2.396470in}{0.865278in}}%
\pgfpathlineto{\pgfqpoint{2.395715in}{0.879087in}}%
\pgfpathlineto{\pgfqpoint{2.391193in}{0.892674in}}%
\pgfpathlineto{\pgfqpoint{2.401868in}{0.893227in}}%
\pgfpathlineto{\pgfqpoint{2.401625in}{0.897879in}}%
\pgfpathlineto{\pgfqpoint{2.408409in}{0.898240in}}%
\pgfpathlineto{\pgfqpoint{2.408616in}{0.893304in}}%
\pgfpathlineto{\pgfqpoint{2.416578in}{0.896534in}}%
\pgfpathlineto{\pgfqpoint{2.417347in}{0.883236in}}%
\pgfpathlineto{\pgfqpoint{2.410750in}{0.879777in}}%
\pgfpathlineto{\pgfqpoint{2.417323in}{0.873285in}}%
\pgfpathlineto{\pgfqpoint{2.442684in}{0.864247in}}%
\pgfpathclose%
\pgfusepath{fill}%
\end{pgfscope}%
\begin{pgfscope}%
\pgfpathrectangle{\pgfqpoint{0.100000in}{0.100000in}}{\pgfqpoint{3.608454in}{2.310000in}}%
\pgfusepath{clip}%
\pgfsetbuttcap%
\pgfsetmiterjoin%
\definecolor{currentfill}{rgb}{0.000000,0.741176,0.629412}%
\pgfsetfillcolor{currentfill}%
\pgfsetlinewidth{0.000000pt}%
\definecolor{currentstroke}{rgb}{0.000000,0.000000,0.000000}%
\pgfsetstrokecolor{currentstroke}%
\pgfsetstrokeopacity{0.000000}%
\pgfsetdash{}{0pt}%
\pgfpathmoveto{\pgfqpoint{2.172440in}{1.907196in}}%
\pgfpathlineto{\pgfqpoint{2.172623in}{1.893374in}}%
\pgfpathlineto{\pgfqpoint{2.174916in}{1.893366in}}%
\pgfpathlineto{\pgfqpoint{2.175363in}{1.880599in}}%
\pgfpathlineto{\pgfqpoint{2.142690in}{1.880249in}}%
\pgfpathlineto{\pgfqpoint{2.147223in}{1.876558in}}%
\pgfpathlineto{\pgfqpoint{2.099037in}{1.876260in}}%
\pgfpathlineto{\pgfqpoint{2.098775in}{1.916792in}}%
\pgfpathlineto{\pgfqpoint{2.097851in}{1.951504in}}%
\pgfpathlineto{\pgfqpoint{2.097902in}{1.958459in}}%
\pgfpathlineto{\pgfqpoint{2.118427in}{1.958403in}}%
\pgfpathlineto{\pgfqpoint{2.118937in}{1.925718in}}%
\pgfpathlineto{\pgfqpoint{2.132410in}{1.918482in}}%
\pgfpathlineto{\pgfqpoint{2.138352in}{1.920870in}}%
\pgfpathlineto{\pgfqpoint{2.142555in}{1.917955in}}%
\pgfpathlineto{\pgfqpoint{2.141306in}{1.906925in}}%
\pgfpathlineto{\pgfqpoint{2.172440in}{1.907196in}}%
\pgfpathclose%
\pgfusepath{fill}%
\end{pgfscope}%
\begin{pgfscope}%
\pgfpathrectangle{\pgfqpoint{0.100000in}{0.100000in}}{\pgfqpoint{3.608454in}{2.310000in}}%
\pgfusepath{clip}%
\pgfsetbuttcap%
\pgfsetmiterjoin%
\definecolor{currentfill}{rgb}{0.000000,0.745098,0.627451}%
\pgfsetfillcolor{currentfill}%
\pgfsetlinewidth{0.000000pt}%
\definecolor{currentstroke}{rgb}{0.000000,0.000000,0.000000}%
\pgfsetstrokecolor{currentstroke}%
\pgfsetstrokeopacity{0.000000}%
\pgfsetdash{}{0pt}%
\pgfpathmoveto{\pgfqpoint{2.231585in}{0.901788in}}%
\pgfpathlineto{\pgfqpoint{2.231529in}{0.896018in}}%
\pgfpathlineto{\pgfqpoint{2.240853in}{0.895980in}}%
\pgfpathlineto{\pgfqpoint{2.240807in}{0.867381in}}%
\pgfpathlineto{\pgfqpoint{2.186423in}{0.866575in}}%
\pgfpathlineto{\pgfqpoint{2.182676in}{0.872245in}}%
\pgfpathlineto{\pgfqpoint{2.183288in}{0.876547in}}%
\pgfpathlineto{\pgfqpoint{2.189397in}{0.878197in}}%
\pgfpathlineto{\pgfqpoint{2.195345in}{0.891554in}}%
\pgfpathlineto{\pgfqpoint{2.190887in}{0.898874in}}%
\pgfpathlineto{\pgfqpoint{2.191374in}{0.903396in}}%
\pgfpathlineto{\pgfqpoint{2.207261in}{0.903173in}}%
\pgfpathlineto{\pgfqpoint{2.207240in}{0.900881in}}%
\pgfpathlineto{\pgfqpoint{2.224654in}{0.900658in}}%
\pgfpathlineto{\pgfqpoint{2.231585in}{0.901788in}}%
\pgfpathclose%
\pgfusepath{fill}%
\end{pgfscope}%
\begin{pgfscope}%
\pgfpathrectangle{\pgfqpoint{0.100000in}{0.100000in}}{\pgfqpoint{3.608454in}{2.310000in}}%
\pgfusepath{clip}%
\pgfsetbuttcap%
\pgfsetmiterjoin%
\definecolor{currentfill}{rgb}{0.000000,0.282353,0.858824}%
\pgfsetfillcolor{currentfill}%
\pgfsetlinewidth{0.000000pt}%
\definecolor{currentstroke}{rgb}{0.000000,0.000000,0.000000}%
\pgfsetstrokecolor{currentstroke}%
\pgfsetstrokeopacity{0.000000}%
\pgfsetdash{}{0pt}%
\pgfpathmoveto{\pgfqpoint{2.454210in}{1.655405in}}%
\pgfpathlineto{\pgfqpoint{2.433615in}{1.653903in}}%
\pgfpathlineto{\pgfqpoint{2.406360in}{1.653264in}}%
\pgfpathlineto{\pgfqpoint{2.406543in}{1.649815in}}%
\pgfpathlineto{\pgfqpoint{2.372395in}{1.647911in}}%
\pgfpathlineto{\pgfqpoint{2.370685in}{1.678611in}}%
\pgfpathlineto{\pgfqpoint{2.363976in}{1.679112in}}%
\pgfpathlineto{\pgfqpoint{2.357123in}{1.675633in}}%
\pgfpathlineto{\pgfqpoint{2.356072in}{1.695517in}}%
\pgfpathlineto{\pgfqpoint{2.355403in}{1.705801in}}%
\pgfpathlineto{\pgfqpoint{2.376100in}{1.706906in}}%
\pgfpathlineto{\pgfqpoint{2.375661in}{1.713794in}}%
\pgfpathlineto{\pgfqpoint{2.405834in}{1.715444in}}%
\pgfpathlineto{\pgfqpoint{2.416473in}{1.716189in}}%
\pgfpathlineto{\pgfqpoint{2.414986in}{1.743088in}}%
\pgfpathlineto{\pgfqpoint{2.439466in}{1.744650in}}%
\pgfpathlineto{\pgfqpoint{2.440017in}{1.736230in}}%
\pgfpathlineto{\pgfqpoint{2.436156in}{1.727226in}}%
\pgfpathlineto{\pgfqpoint{2.436777in}{1.717456in}}%
\pgfpathlineto{\pgfqpoint{2.450492in}{1.717550in}}%
\pgfpathlineto{\pgfqpoint{2.454210in}{1.655405in}}%
\pgfpathclose%
\pgfusepath{fill}%
\end{pgfscope}%
\begin{pgfscope}%
\pgfpathrectangle{\pgfqpoint{0.100000in}{0.100000in}}{\pgfqpoint{3.608454in}{2.310000in}}%
\pgfusepath{clip}%
\pgfsetbuttcap%
\pgfsetmiterjoin%
\definecolor{currentfill}{rgb}{0.000000,0.388235,0.805882}%
\pgfsetfillcolor{currentfill}%
\pgfsetlinewidth{0.000000pt}%
\definecolor{currentstroke}{rgb}{0.000000,0.000000,0.000000}%
\pgfsetstrokecolor{currentstroke}%
\pgfsetstrokeopacity{0.000000}%
\pgfsetdash{}{0pt}%
\pgfpathmoveto{\pgfqpoint{1.278870in}{1.882470in}}%
\pgfpathlineto{\pgfqpoint{1.317490in}{1.876575in}}%
\pgfpathlineto{\pgfqpoint{1.318547in}{1.875904in}}%
\pgfpathlineto{\pgfqpoint{1.364865in}{1.868873in}}%
\pgfpathlineto{\pgfqpoint{1.383661in}{1.866297in}}%
\pgfpathlineto{\pgfqpoint{1.384830in}{1.855882in}}%
\pgfpathlineto{\pgfqpoint{1.389944in}{1.843550in}}%
\pgfpathlineto{\pgfqpoint{1.396824in}{1.840362in}}%
\pgfpathlineto{\pgfqpoint{1.406562in}{1.831802in}}%
\pgfpathlineto{\pgfqpoint{1.412375in}{1.821775in}}%
\pgfpathlineto{\pgfqpoint{1.417230in}{1.817131in}}%
\pgfpathlineto{\pgfqpoint{1.419171in}{1.807060in}}%
\pgfpathlineto{\pgfqpoint{1.417395in}{1.794618in}}%
\pgfpathlineto{\pgfqpoint{1.338318in}{1.806097in}}%
\pgfpathlineto{\pgfqpoint{1.337257in}{1.799111in}}%
\pgfpathlineto{\pgfqpoint{1.323648in}{1.801191in}}%
\pgfpathlineto{\pgfqpoint{1.322614in}{1.794241in}}%
\pgfpathlineto{\pgfqpoint{1.315604in}{1.795301in}}%
\pgfpathlineto{\pgfqpoint{1.314743in}{1.788523in}}%
\pgfpathlineto{\pgfqpoint{1.304569in}{1.790081in}}%
\pgfpathlineto{\pgfqpoint{1.303529in}{1.783157in}}%
\pgfpathlineto{\pgfqpoint{1.291011in}{1.784895in}}%
\pgfpathlineto{\pgfqpoint{1.285167in}{1.795237in}}%
\pgfpathlineto{\pgfqpoint{1.279919in}{1.798565in}}%
\pgfpathlineto{\pgfqpoint{1.273821in}{1.796453in}}%
\pgfpathlineto{\pgfqpoint{1.271394in}{1.792453in}}%
\pgfpathlineto{\pgfqpoint{1.264050in}{1.788274in}}%
\pgfpathlineto{\pgfqpoint{1.260477in}{1.800714in}}%
\pgfpathlineto{\pgfqpoint{1.254530in}{1.801740in}}%
\pgfpathlineto{\pgfqpoint{1.251746in}{1.806946in}}%
\pgfpathlineto{\pgfqpoint{1.253366in}{1.816700in}}%
\pgfpathlineto{\pgfqpoint{1.250644in}{1.821729in}}%
\pgfpathlineto{\pgfqpoint{1.249318in}{1.830814in}}%
\pgfpathlineto{\pgfqpoint{1.243670in}{1.842268in}}%
\pgfpathlineto{\pgfqpoint{1.245880in}{1.848777in}}%
\pgfpathlineto{\pgfqpoint{1.241305in}{1.854999in}}%
\pgfpathlineto{\pgfqpoint{1.224995in}{1.857794in}}%
\pgfpathlineto{\pgfqpoint{1.226134in}{1.864328in}}%
\pgfpathlineto{\pgfqpoint{1.204580in}{1.868127in}}%
\pgfpathlineto{\pgfqpoint{1.209266in}{1.894331in}}%
\pgfpathlineto{\pgfqpoint{1.228507in}{1.890208in}}%
\pgfpathlineto{\pgfqpoint{1.278870in}{1.882470in}}%
\pgfpathclose%
\pgfusepath{fill}%
\end{pgfscope}%
\begin{pgfscope}%
\pgfpathrectangle{\pgfqpoint{0.100000in}{0.100000in}}{\pgfqpoint{3.608454in}{2.310000in}}%
\pgfusepath{clip}%
\pgfsetbuttcap%
\pgfsetmiterjoin%
\definecolor{currentfill}{rgb}{0.000000,0.470588,0.764706}%
\pgfsetfillcolor{currentfill}%
\pgfsetlinewidth{0.000000pt}%
\definecolor{currentstroke}{rgb}{0.000000,0.000000,0.000000}%
\pgfsetstrokecolor{currentstroke}%
\pgfsetstrokeopacity{0.000000}%
\pgfsetdash{}{0pt}%
\pgfpathmoveto{\pgfqpoint{2.551959in}{0.940228in}}%
\pgfpathlineto{\pgfqpoint{2.546945in}{0.939900in}}%
\pgfpathlineto{\pgfqpoint{2.544297in}{0.934956in}}%
\pgfpathlineto{\pgfqpoint{2.535053in}{0.931443in}}%
\pgfpathlineto{\pgfqpoint{2.521327in}{0.932876in}}%
\pgfpathlineto{\pgfqpoint{2.520583in}{0.943229in}}%
\pgfpathlineto{\pgfqpoint{2.506684in}{0.942391in}}%
\pgfpathlineto{\pgfqpoint{2.506964in}{0.937791in}}%
\pgfpathlineto{\pgfqpoint{2.489985in}{0.934122in}}%
\pgfpathlineto{\pgfqpoint{2.469213in}{0.932802in}}%
\pgfpathlineto{\pgfqpoint{2.468504in}{0.944341in}}%
\pgfpathlineto{\pgfqpoint{2.466935in}{0.967664in}}%
\pgfpathlineto{\pgfqpoint{2.484251in}{0.968666in}}%
\pgfpathlineto{\pgfqpoint{2.483100in}{0.985946in}}%
\pgfpathlineto{\pgfqpoint{2.503832in}{0.987282in}}%
\pgfpathlineto{\pgfqpoint{2.512839in}{0.989031in}}%
\pgfpathlineto{\pgfqpoint{2.512313in}{0.997036in}}%
\pgfpathlineto{\pgfqpoint{2.515690in}{0.998385in}}%
\pgfpathlineto{\pgfqpoint{2.528379in}{0.999297in}}%
\pgfpathlineto{\pgfqpoint{2.528607in}{0.995823in}}%
\pgfpathlineto{\pgfqpoint{2.553769in}{0.997504in}}%
\pgfpathlineto{\pgfqpoint{2.553484in}{0.986177in}}%
\pgfpathlineto{\pgfqpoint{2.551959in}{0.940228in}}%
\pgfpathclose%
\pgfusepath{fill}%
\end{pgfscope}%
\begin{pgfscope}%
\pgfpathrectangle{\pgfqpoint{0.100000in}{0.100000in}}{\pgfqpoint{3.608454in}{2.310000in}}%
\pgfusepath{clip}%
\pgfsetbuttcap%
\pgfsetmiterjoin%
\definecolor{currentfill}{rgb}{0.000000,0.670588,0.664706}%
\pgfsetfillcolor{currentfill}%
\pgfsetlinewidth{0.000000pt}%
\definecolor{currentstroke}{rgb}{0.000000,0.000000,0.000000}%
\pgfsetstrokecolor{currentstroke}%
\pgfsetstrokeopacity{0.000000}%
\pgfsetdash{}{0pt}%
\pgfpathmoveto{\pgfqpoint{3.458788in}{2.127000in}}%
\pgfpathlineto{\pgfqpoint{3.457596in}{2.136892in}}%
\pgfpathlineto{\pgfqpoint{3.481080in}{2.206952in}}%
\pgfpathlineto{\pgfqpoint{3.491936in}{2.206386in}}%
\pgfpathlineto{\pgfqpoint{3.494641in}{2.194072in}}%
\pgfpathlineto{\pgfqpoint{3.504213in}{2.190328in}}%
\pgfpathlineto{\pgfqpoint{3.518463in}{2.203609in}}%
\pgfpathlineto{\pgfqpoint{3.528567in}{2.206604in}}%
\pgfpathlineto{\pgfqpoint{3.527922in}{2.211919in}}%
\pgfpathlineto{\pgfqpoint{3.535054in}{2.214135in}}%
\pgfpathlineto{\pgfqpoint{3.552913in}{2.206672in}}%
\pgfpathlineto{\pgfqpoint{3.558787in}{2.200716in}}%
\pgfpathlineto{\pgfqpoint{3.564560in}{2.199223in}}%
\pgfpathlineto{\pgfqpoint{3.575635in}{2.163717in}}%
\pgfpathlineto{\pgfqpoint{3.591481in}{2.113768in}}%
\pgfpathlineto{\pgfqpoint{3.591685in}{2.108859in}}%
\pgfpathlineto{\pgfqpoint{3.596315in}{2.093414in}}%
\pgfpathlineto{\pgfqpoint{3.584395in}{2.086358in}}%
\pgfpathlineto{\pgfqpoint{3.566650in}{2.075414in}}%
\pgfpathlineto{\pgfqpoint{3.565274in}{2.075679in}}%
\pgfpathlineto{\pgfqpoint{3.557203in}{2.103454in}}%
\pgfpathlineto{\pgfqpoint{3.546699in}{2.136666in}}%
\pgfpathlineto{\pgfqpoint{3.526210in}{2.131697in}}%
\pgfpathlineto{\pgfqpoint{3.522068in}{2.145170in}}%
\pgfpathlineto{\pgfqpoint{3.474712in}{2.131510in}}%
\pgfpathlineto{\pgfqpoint{3.458788in}{2.127000in}}%
\pgfpathclose%
\pgfusepath{fill}%
\end{pgfscope}%
\begin{pgfscope}%
\pgfpathrectangle{\pgfqpoint{0.100000in}{0.100000in}}{\pgfqpoint{3.608454in}{2.310000in}}%
\pgfusepath{clip}%
\pgfsetbuttcap%
\pgfsetmiterjoin%
\definecolor{currentfill}{rgb}{0.000000,0.482353,0.758824}%
\pgfsetfillcolor{currentfill}%
\pgfsetlinewidth{0.000000pt}%
\definecolor{currentstroke}{rgb}{0.000000,0.000000,0.000000}%
\pgfsetstrokecolor{currentstroke}%
\pgfsetstrokeopacity{0.000000}%
\pgfsetdash{}{0pt}%
\pgfpathmoveto{\pgfqpoint{2.714189in}{1.531011in}}%
\pgfpathlineto{\pgfqpoint{2.707766in}{1.586556in}}%
\pgfpathlineto{\pgfqpoint{2.731544in}{1.590059in}}%
\pgfpathlineto{\pgfqpoint{2.734410in}{1.575123in}}%
\pgfpathlineto{\pgfqpoint{2.737491in}{1.568598in}}%
\pgfpathlineto{\pgfqpoint{2.744178in}{1.569410in}}%
\pgfpathlineto{\pgfqpoint{2.746637in}{1.548837in}}%
\pgfpathlineto{\pgfqpoint{2.739953in}{1.548013in}}%
\pgfpathlineto{\pgfqpoint{2.741540in}{1.534284in}}%
\pgfpathlineto{\pgfqpoint{2.714189in}{1.531011in}}%
\pgfpathclose%
\pgfusepath{fill}%
\end{pgfscope}%
\begin{pgfscope}%
\pgfpathrectangle{\pgfqpoint{0.100000in}{0.100000in}}{\pgfqpoint{3.608454in}{2.310000in}}%
\pgfusepath{clip}%
\pgfsetbuttcap%
\pgfsetmiterjoin%
\definecolor{currentfill}{rgb}{0.000000,0.650980,0.674510}%
\pgfsetfillcolor{currentfill}%
\pgfsetlinewidth{0.000000pt}%
\definecolor{currentstroke}{rgb}{0.000000,0.000000,0.000000}%
\pgfsetstrokecolor{currentstroke}%
\pgfsetstrokeopacity{0.000000}%
\pgfsetdash{}{0pt}%
\pgfpathmoveto{\pgfqpoint{1.303419in}{1.175695in}}%
\pgfpathlineto{\pgfqpoint{1.300870in}{1.158512in}}%
\pgfpathlineto{\pgfqpoint{1.210980in}{1.171960in}}%
\pgfpathlineto{\pgfqpoint{1.223167in}{1.249876in}}%
\pgfpathlineto{\pgfqpoint{1.264762in}{1.243586in}}%
\pgfpathlineto{\pgfqpoint{1.268205in}{1.247503in}}%
\pgfpathlineto{\pgfqpoint{1.272888in}{1.260292in}}%
\pgfpathlineto{\pgfqpoint{1.280183in}{1.269745in}}%
\pgfpathlineto{\pgfqpoint{1.286167in}{1.271003in}}%
\pgfpathlineto{\pgfqpoint{1.291651in}{1.276850in}}%
\pgfpathlineto{\pgfqpoint{1.293840in}{1.286121in}}%
\pgfpathlineto{\pgfqpoint{1.303897in}{1.295339in}}%
\pgfpathlineto{\pgfqpoint{1.305818in}{1.299625in}}%
\pgfpathlineto{\pgfqpoint{1.314836in}{1.308720in}}%
\pgfpathlineto{\pgfqpoint{1.325950in}{1.311919in}}%
\pgfpathlineto{\pgfqpoint{1.327961in}{1.309954in}}%
\pgfpathlineto{\pgfqpoint{1.329166in}{1.295757in}}%
\pgfpathlineto{\pgfqpoint{1.325433in}{1.268700in}}%
\pgfpathlineto{\pgfqpoint{1.347428in}{1.265725in}}%
\pgfpathlineto{\pgfqpoint{1.347103in}{1.263333in}}%
\pgfpathlineto{\pgfqpoint{1.375196in}{1.259964in}}%
\pgfpathlineto{\pgfqpoint{1.373495in}{1.246788in}}%
\pgfpathlineto{\pgfqpoint{1.383780in}{1.226748in}}%
\pgfpathlineto{\pgfqpoint{1.324687in}{1.235061in}}%
\pgfpathlineto{\pgfqpoint{1.322223in}{1.228643in}}%
\pgfpathlineto{\pgfqpoint{1.310197in}{1.221478in}}%
\pgfpathlineto{\pgfqpoint{1.303419in}{1.175695in}}%
\pgfpathclose%
\pgfusepath{fill}%
\end{pgfscope}%
\begin{pgfscope}%
\pgfpathrectangle{\pgfqpoint{0.100000in}{0.100000in}}{\pgfqpoint{3.608454in}{2.310000in}}%
\pgfusepath{clip}%
\pgfsetbuttcap%
\pgfsetmiterjoin%
\definecolor{currentfill}{rgb}{0.000000,0.760784,0.619608}%
\pgfsetfillcolor{currentfill}%
\pgfsetlinewidth{0.000000pt}%
\definecolor{currentstroke}{rgb}{0.000000,0.000000,0.000000}%
\pgfsetstrokecolor{currentstroke}%
\pgfsetstrokeopacity{0.000000}%
\pgfsetdash{}{0pt}%
\pgfpathmoveto{\pgfqpoint{2.420115in}{2.049767in}}%
\pgfpathlineto{\pgfqpoint{2.414517in}{2.052884in}}%
\pgfpathlineto{\pgfqpoint{2.417513in}{2.057828in}}%
\pgfpathlineto{\pgfqpoint{2.438907in}{2.070311in}}%
\pgfpathlineto{\pgfqpoint{2.446529in}{2.077612in}}%
\pgfpathlineto{\pgfqpoint{2.455726in}{2.081379in}}%
\pgfpathlineto{\pgfqpoint{2.450327in}{2.070558in}}%
\pgfpathlineto{\pgfqpoint{2.444429in}{2.065660in}}%
\pgfpathlineto{\pgfqpoint{2.430864in}{2.058882in}}%
\pgfpathlineto{\pgfqpoint{2.433030in}{2.055826in}}%
\pgfpathlineto{\pgfqpoint{2.420115in}{2.049767in}}%
\pgfpathclose%
\pgfusepath{fill}%
\end{pgfscope}%
\begin{pgfscope}%
\pgfpathrectangle{\pgfqpoint{0.100000in}{0.100000in}}{\pgfqpoint{3.608454in}{2.310000in}}%
\pgfusepath{clip}%
\pgfsetbuttcap%
\pgfsetmiterjoin%
\definecolor{currentfill}{rgb}{0.000000,0.760784,0.619608}%
\pgfsetfillcolor{currentfill}%
\pgfsetlinewidth{0.000000pt}%
\definecolor{currentstroke}{rgb}{0.000000,0.000000,0.000000}%
\pgfsetstrokecolor{currentstroke}%
\pgfsetstrokeopacity{0.000000}%
\pgfsetdash{}{0pt}%
\pgfpathmoveto{\pgfqpoint{2.485656in}{1.981514in}}%
\pgfpathlineto{\pgfqpoint{2.486521in}{1.969842in}}%
\pgfpathlineto{\pgfqpoint{2.489953in}{1.963218in}}%
\pgfpathlineto{\pgfqpoint{2.483177in}{1.962621in}}%
\pgfpathlineto{\pgfqpoint{2.484672in}{1.942048in}}%
\pgfpathlineto{\pgfqpoint{2.436761in}{1.938649in}}%
\pgfpathlineto{\pgfqpoint{2.435480in}{1.959378in}}%
\pgfpathlineto{\pgfqpoint{2.442299in}{1.959772in}}%
\pgfpathlineto{\pgfqpoint{2.438144in}{1.966329in}}%
\pgfpathlineto{\pgfqpoint{2.436662in}{1.987686in}}%
\pgfpathlineto{\pgfqpoint{2.438695in}{1.993269in}}%
\pgfpathlineto{\pgfqpoint{2.450631in}{2.004021in}}%
\pgfpathlineto{\pgfqpoint{2.454724in}{2.005602in}}%
\pgfpathlineto{\pgfqpoint{2.462516in}{2.016619in}}%
\pgfpathlineto{\pgfqpoint{2.472955in}{2.023672in}}%
\pgfpathlineto{\pgfqpoint{2.487300in}{2.027022in}}%
\pgfpathlineto{\pgfqpoint{2.495202in}{2.026998in}}%
\pgfpathlineto{\pgfqpoint{2.500592in}{2.021625in}}%
\pgfpathlineto{\pgfqpoint{2.489998in}{2.020382in}}%
\pgfpathlineto{\pgfqpoint{2.488577in}{2.015517in}}%
\pgfpathlineto{\pgfqpoint{2.482568in}{2.011992in}}%
\pgfpathlineto{\pgfqpoint{2.473948in}{2.003663in}}%
\pgfpathlineto{\pgfqpoint{2.473922in}{1.999171in}}%
\pgfpathlineto{\pgfqpoint{2.463650in}{1.984401in}}%
\pgfpathlineto{\pgfqpoint{2.462316in}{1.973885in}}%
\pgfpathlineto{\pgfqpoint{2.465262in}{1.970695in}}%
\pgfpathlineto{\pgfqpoint{2.474637in}{1.982064in}}%
\pgfpathlineto{\pgfqpoint{2.479239in}{1.985245in}}%
\pgfpathlineto{\pgfqpoint{2.485656in}{1.981514in}}%
\pgfpathclose%
\pgfusepath{fill}%
\end{pgfscope}%
\begin{pgfscope}%
\pgfpathrectangle{\pgfqpoint{0.100000in}{0.100000in}}{\pgfqpoint{3.608454in}{2.310000in}}%
\pgfusepath{clip}%
\pgfsetbuttcap%
\pgfsetmiterjoin%
\definecolor{currentfill}{rgb}{0.000000,0.372549,0.813725}%
\pgfsetfillcolor{currentfill}%
\pgfsetlinewidth{0.000000pt}%
\definecolor{currentstroke}{rgb}{0.000000,0.000000,0.000000}%
\pgfsetstrokecolor{currentstroke}%
\pgfsetstrokeopacity{0.000000}%
\pgfsetdash{}{0pt}%
\pgfpathmoveto{\pgfqpoint{1.481289in}{1.288540in}}%
\pgfpathlineto{\pgfqpoint{1.474843in}{1.297491in}}%
\pgfpathlineto{\pgfqpoint{1.470019in}{1.293572in}}%
\pgfpathlineto{\pgfqpoint{1.458524in}{1.289130in}}%
\pgfpathlineto{\pgfqpoint{1.454881in}{1.289984in}}%
\pgfpathlineto{\pgfqpoint{1.448675in}{1.296375in}}%
\pgfpathlineto{\pgfqpoint{1.444613in}{1.311073in}}%
\pgfpathlineto{\pgfqpoint{1.426840in}{1.337144in}}%
\pgfpathlineto{\pgfqpoint{1.435665in}{1.345243in}}%
\pgfpathlineto{\pgfqpoint{1.436154in}{1.349875in}}%
\pgfpathlineto{\pgfqpoint{1.431712in}{1.356218in}}%
\pgfpathlineto{\pgfqpoint{1.470992in}{1.351885in}}%
\pgfpathlineto{\pgfqpoint{1.474789in}{1.385857in}}%
\pgfpathlineto{\pgfqpoint{1.552744in}{1.377402in}}%
\pgfpathlineto{\pgfqpoint{1.550467in}{1.357020in}}%
\pgfpathlineto{\pgfqpoint{1.547834in}{1.329763in}}%
\pgfpathlineto{\pgfqpoint{1.493221in}{1.335267in}}%
\pgfpathlineto{\pgfqpoint{1.491038in}{1.314728in}}%
\pgfpathlineto{\pgfqpoint{1.484320in}{1.315446in}}%
\pgfpathlineto{\pgfqpoint{1.481289in}{1.288540in}}%
\pgfpathclose%
\pgfusepath{fill}%
\end{pgfscope}%
\begin{pgfscope}%
\pgfpathrectangle{\pgfqpoint{0.100000in}{0.100000in}}{\pgfqpoint{3.608454in}{2.310000in}}%
\pgfusepath{clip}%
\pgfsetbuttcap%
\pgfsetmiterjoin%
\definecolor{currentfill}{rgb}{0.000000,0.717647,0.641176}%
\pgfsetfillcolor{currentfill}%
\pgfsetlinewidth{0.000000pt}%
\definecolor{currentstroke}{rgb}{0.000000,0.000000,0.000000}%
\pgfsetstrokecolor{currentstroke}%
\pgfsetstrokeopacity{0.000000}%
\pgfsetdash{}{0pt}%
\pgfpathmoveto{\pgfqpoint{2.977955in}{1.500863in}}%
\pgfpathlineto{\pgfqpoint{2.956251in}{1.497406in}}%
\pgfpathlineto{\pgfqpoint{2.954062in}{1.518144in}}%
\pgfpathlineto{\pgfqpoint{2.949522in}{1.517575in}}%
\pgfpathlineto{\pgfqpoint{2.949140in}{1.528048in}}%
\pgfpathlineto{\pgfqpoint{2.952109in}{1.531961in}}%
\pgfpathlineto{\pgfqpoint{2.963825in}{1.532458in}}%
\pgfpathlineto{\pgfqpoint{2.971951in}{1.538265in}}%
\pgfpathlineto{\pgfqpoint{2.977955in}{1.500863in}}%
\pgfpathclose%
\pgfusepath{fill}%
\end{pgfscope}%
\begin{pgfscope}%
\pgfpathrectangle{\pgfqpoint{0.100000in}{0.100000in}}{\pgfqpoint{3.608454in}{2.310000in}}%
\pgfusepath{clip}%
\pgfsetbuttcap%
\pgfsetmiterjoin%
\definecolor{currentfill}{rgb}{0.000000,0.760784,0.619608}%
\pgfsetfillcolor{currentfill}%
\pgfsetlinewidth{0.000000pt}%
\definecolor{currentstroke}{rgb}{0.000000,0.000000,0.000000}%
\pgfsetstrokecolor{currentstroke}%
\pgfsetstrokeopacity{0.000000}%
\pgfsetdash{}{0pt}%
\pgfpathmoveto{\pgfqpoint{2.573691in}{1.108364in}}%
\pgfpathlineto{\pgfqpoint{2.573068in}{1.094430in}}%
\pgfpathlineto{\pgfqpoint{2.579650in}{1.089158in}}%
\pgfpathlineto{\pgfqpoint{2.575892in}{1.080309in}}%
\pgfpathlineto{\pgfqpoint{2.559562in}{1.077181in}}%
\pgfpathlineto{\pgfqpoint{2.556697in}{1.071568in}}%
\pgfpathlineto{\pgfqpoint{2.550818in}{1.070538in}}%
\pgfpathlineto{\pgfqpoint{2.534914in}{1.072050in}}%
\pgfpathlineto{\pgfqpoint{2.531082in}{1.076803in}}%
\pgfpathlineto{\pgfqpoint{2.525076in}{1.078728in}}%
\pgfpathlineto{\pgfqpoint{2.517692in}{1.084251in}}%
\pgfpathlineto{\pgfqpoint{2.517006in}{1.100169in}}%
\pgfpathlineto{\pgfqpoint{2.519200in}{1.103140in}}%
\pgfpathlineto{\pgfqpoint{2.544235in}{1.104329in}}%
\pgfpathlineto{\pgfqpoint{2.552190in}{1.107568in}}%
\pgfpathlineto{\pgfqpoint{2.553934in}{1.104609in}}%
\pgfpathlineto{\pgfqpoint{2.561312in}{1.107350in}}%
\pgfpathlineto{\pgfqpoint{2.573691in}{1.108364in}}%
\pgfpathclose%
\pgfusepath{fill}%
\end{pgfscope}%
\begin{pgfscope}%
\pgfpathrectangle{\pgfqpoint{0.100000in}{0.100000in}}{\pgfqpoint{3.608454in}{2.310000in}}%
\pgfusepath{clip}%
\pgfsetbuttcap%
\pgfsetmiterjoin%
\definecolor{currentfill}{rgb}{0.000000,0.662745,0.668627}%
\pgfsetfillcolor{currentfill}%
\pgfsetlinewidth{0.000000pt}%
\definecolor{currentstroke}{rgb}{0.000000,0.000000,0.000000}%
\pgfsetstrokecolor{currentstroke}%
\pgfsetstrokeopacity{0.000000}%
\pgfsetdash{}{0pt}%
\pgfpathmoveto{\pgfqpoint{1.518670in}{0.813225in}}%
\pgfpathlineto{\pgfqpoint{1.498395in}{0.815195in}}%
\pgfpathlineto{\pgfqpoint{1.443108in}{0.820998in}}%
\pgfpathlineto{\pgfqpoint{1.447334in}{0.862167in}}%
\pgfpathlineto{\pgfqpoint{1.413830in}{0.865768in}}%
\pgfpathlineto{\pgfqpoint{1.417927in}{0.900770in}}%
\pgfpathlineto{\pgfqpoint{1.420339in}{0.900498in}}%
\pgfpathlineto{\pgfqpoint{1.421822in}{0.913943in}}%
\pgfpathlineto{\pgfqpoint{1.449063in}{0.911411in}}%
\pgfpathlineto{\pgfqpoint{1.452184in}{0.931773in}}%
\pgfpathlineto{\pgfqpoint{1.458090in}{0.986273in}}%
\pgfpathlineto{\pgfqpoint{1.465068in}{0.985473in}}%
\pgfpathlineto{\pgfqpoint{1.464338in}{0.978551in}}%
\pgfpathlineto{\pgfqpoint{1.505497in}{0.973876in}}%
\pgfpathlineto{\pgfqpoint{1.506226in}{0.980812in}}%
\pgfpathlineto{\pgfqpoint{1.519945in}{0.979398in}}%
\pgfpathlineto{\pgfqpoint{1.526759in}{0.978642in}}%
\pgfpathlineto{\pgfqpoint{1.524717in}{0.957918in}}%
\pgfpathlineto{\pgfqpoint{1.532929in}{0.957126in}}%
\pgfpathlineto{\pgfqpoint{1.531592in}{0.944243in}}%
\pgfpathlineto{\pgfqpoint{1.545317in}{0.943021in}}%
\pgfpathlineto{\pgfqpoint{1.544636in}{0.936153in}}%
\pgfpathlineto{\pgfqpoint{1.530771in}{0.937426in}}%
\pgfpathlineto{\pgfqpoint{1.529406in}{0.923693in}}%
\pgfpathlineto{\pgfqpoint{1.526429in}{0.923948in}}%
\pgfpathlineto{\pgfqpoint{1.523143in}{0.889921in}}%
\pgfpathlineto{\pgfqpoint{1.519940in}{0.890229in}}%
\pgfpathlineto{\pgfqpoint{1.516573in}{0.855187in}}%
\pgfpathlineto{\pgfqpoint{1.522650in}{0.854609in}}%
\pgfpathlineto{\pgfqpoint{1.518670in}{0.813225in}}%
\pgfpathclose%
\pgfusepath{fill}%
\end{pgfscope}%
\begin{pgfscope}%
\pgfpathrectangle{\pgfqpoint{0.100000in}{0.100000in}}{\pgfqpoint{3.608454in}{2.310000in}}%
\pgfusepath{clip}%
\pgfsetbuttcap%
\pgfsetmiterjoin%
\definecolor{currentfill}{rgb}{0.000000,0.654902,0.672549}%
\pgfsetfillcolor{currentfill}%
\pgfsetlinewidth{0.000000pt}%
\definecolor{currentstroke}{rgb}{0.000000,0.000000,0.000000}%
\pgfsetstrokecolor{currentstroke}%
\pgfsetstrokeopacity{0.000000}%
\pgfsetdash{}{0pt}%
\pgfpathmoveto{\pgfqpoint{1.411936in}{2.184794in}}%
\pgfpathlineto{\pgfqpoint{1.407741in}{2.164527in}}%
\pgfpathlineto{\pgfqpoint{1.404613in}{2.164996in}}%
\pgfpathlineto{\pgfqpoint{1.402592in}{2.151198in}}%
\pgfpathlineto{\pgfqpoint{1.399456in}{2.139818in}}%
\pgfpathlineto{\pgfqpoint{1.396285in}{2.142117in}}%
\pgfpathlineto{\pgfqpoint{1.392437in}{2.124428in}}%
\pgfpathlineto{\pgfqpoint{1.389566in}{2.105191in}}%
\pgfpathlineto{\pgfqpoint{1.380515in}{2.107723in}}%
\pgfpathlineto{\pgfqpoint{1.378739in}{2.102305in}}%
\pgfpathlineto{\pgfqpoint{1.364847in}{2.104580in}}%
\pgfpathlineto{\pgfqpoint{1.362402in}{2.089710in}}%
\pgfpathlineto{\pgfqpoint{1.358594in}{2.095823in}}%
\pgfpathlineto{\pgfqpoint{1.350200in}{2.095416in}}%
\pgfpathlineto{\pgfqpoint{1.342674in}{2.097531in}}%
\pgfpathlineto{\pgfqpoint{1.332093in}{2.093598in}}%
\pgfpathlineto{\pgfqpoint{1.328054in}{2.095490in}}%
\pgfpathlineto{\pgfqpoint{1.333325in}{2.126759in}}%
\pgfpathlineto{\pgfqpoint{1.323124in}{2.128216in}}%
\pgfpathlineto{\pgfqpoint{1.324278in}{2.135234in}}%
\pgfpathlineto{\pgfqpoint{1.318309in}{2.136230in}}%
\pgfpathlineto{\pgfqpoint{1.319123in}{2.142944in}}%
\pgfpathlineto{\pgfqpoint{1.278167in}{2.149917in}}%
\pgfpathlineto{\pgfqpoint{1.276965in}{2.143064in}}%
\pgfpathlineto{\pgfqpoint{1.266372in}{2.144919in}}%
\pgfpathlineto{\pgfqpoint{1.265169in}{2.138097in}}%
\pgfpathlineto{\pgfqpoint{1.234540in}{2.143503in}}%
\pgfpathlineto{\pgfqpoint{1.235802in}{2.150383in}}%
\pgfpathlineto{\pgfqpoint{1.242847in}{2.149099in}}%
\pgfpathlineto{\pgfqpoint{1.254169in}{2.210455in}}%
\pgfpathlineto{\pgfqpoint{1.308831in}{2.201031in}}%
\pgfpathlineto{\pgfqpoint{1.372332in}{2.190734in}}%
\pgfpathlineto{\pgfqpoint{1.411936in}{2.184794in}}%
\pgfpathclose%
\pgfusepath{fill}%
\end{pgfscope}%
\begin{pgfscope}%
\pgfpathrectangle{\pgfqpoint{0.100000in}{0.100000in}}{\pgfqpoint{3.608454in}{2.310000in}}%
\pgfusepath{clip}%
\pgfsetbuttcap%
\pgfsetmiterjoin%
\definecolor{currentfill}{rgb}{0.000000,0.541176,0.729412}%
\pgfsetfillcolor{currentfill}%
\pgfsetlinewidth{0.000000pt}%
\definecolor{currentstroke}{rgb}{0.000000,0.000000,0.000000}%
\pgfsetstrokecolor{currentstroke}%
\pgfsetstrokeopacity{0.000000}%
\pgfsetdash{}{0pt}%
\pgfpathmoveto{\pgfqpoint{3.286885in}{1.169498in}}%
\pgfpathlineto{\pgfqpoint{3.287098in}{1.161529in}}%
\pgfpathlineto{\pgfqpoint{3.273900in}{1.157290in}}%
\pgfpathlineto{\pgfqpoint{3.265343in}{1.162433in}}%
\pgfpathlineto{\pgfqpoint{3.248882in}{1.168666in}}%
\pgfpathlineto{\pgfqpoint{3.248243in}{1.176631in}}%
\pgfpathlineto{\pgfqpoint{3.250804in}{1.179929in}}%
\pgfpathlineto{\pgfqpoint{3.246587in}{1.185703in}}%
\pgfpathlineto{\pgfqpoint{3.244554in}{1.193790in}}%
\pgfpathlineto{\pgfqpoint{3.232624in}{1.197760in}}%
\pgfpathlineto{\pgfqpoint{3.231820in}{1.204552in}}%
\pgfpathlineto{\pgfqpoint{3.226514in}{1.211542in}}%
\pgfpathlineto{\pgfqpoint{3.230833in}{1.216563in}}%
\pgfpathlineto{\pgfqpoint{3.238791in}{1.214139in}}%
\pgfpathlineto{\pgfqpoint{3.239989in}{1.209565in}}%
\pgfpathlineto{\pgfqpoint{3.247206in}{1.209157in}}%
\pgfpathlineto{\pgfqpoint{3.253779in}{1.205127in}}%
\pgfpathlineto{\pgfqpoint{3.258356in}{1.209245in}}%
\pgfpathlineto{\pgfqpoint{3.262155in}{1.203038in}}%
\pgfpathlineto{\pgfqpoint{3.265874in}{1.209619in}}%
\pgfpathlineto{\pgfqpoint{3.269662in}{1.209194in}}%
\pgfpathlineto{\pgfqpoint{3.272014in}{1.215831in}}%
\pgfpathlineto{\pgfqpoint{3.282173in}{1.217740in}}%
\pgfpathlineto{\pgfqpoint{3.289627in}{1.222451in}}%
\pgfpathlineto{\pgfqpoint{3.293663in}{1.220232in}}%
\pgfpathlineto{\pgfqpoint{3.301128in}{1.225302in}}%
\pgfpathlineto{\pgfqpoint{3.310778in}{1.227788in}}%
\pgfpathlineto{\pgfqpoint{3.318505in}{1.221809in}}%
\pgfpathlineto{\pgfqpoint{3.321566in}{1.228593in}}%
\pgfpathlineto{\pgfqpoint{3.327494in}{1.228385in}}%
\pgfpathlineto{\pgfqpoint{3.332202in}{1.223602in}}%
\pgfpathlineto{\pgfqpoint{3.337579in}{1.209759in}}%
\pgfpathlineto{\pgfqpoint{3.336122in}{1.199785in}}%
\pgfpathlineto{\pgfqpoint{3.328969in}{1.197798in}}%
\pgfpathlineto{\pgfqpoint{3.323739in}{1.188050in}}%
\pgfpathlineto{\pgfqpoint{3.323024in}{1.183127in}}%
\pgfpathlineto{\pgfqpoint{3.315978in}{1.174823in}}%
\pgfpathlineto{\pgfqpoint{3.307317in}{1.174597in}}%
\pgfpathlineto{\pgfqpoint{3.297293in}{1.177393in}}%
\pgfpathlineto{\pgfqpoint{3.295040in}{1.173975in}}%
\pgfpathlineto{\pgfqpoint{3.285428in}{1.174379in}}%
\pgfpathlineto{\pgfqpoint{3.286885in}{1.169498in}}%
\pgfpathclose%
\pgfusepath{fill}%
\end{pgfscope}%
\begin{pgfscope}%
\pgfpathrectangle{\pgfqpoint{0.100000in}{0.100000in}}{\pgfqpoint{3.608454in}{2.310000in}}%
\pgfusepath{clip}%
\pgfsetbuttcap%
\pgfsetmiterjoin%
\definecolor{currentfill}{rgb}{0.000000,0.541176,0.729412}%
\pgfsetfillcolor{currentfill}%
\pgfsetlinewidth{0.000000pt}%
\definecolor{currentstroke}{rgb}{0.000000,0.000000,0.000000}%
\pgfsetstrokecolor{currentstroke}%
\pgfsetstrokeopacity{0.000000}%
\pgfsetdash{}{0pt}%
\pgfpathmoveto{\pgfqpoint{3.324933in}{1.250084in}}%
\pgfpathlineto{\pgfqpoint{3.338120in}{1.231991in}}%
\pgfpathlineto{\pgfqpoint{3.331617in}{1.232924in}}%
\pgfpathlineto{\pgfqpoint{3.324107in}{1.249673in}}%
\pgfpathlineto{\pgfqpoint{3.324933in}{1.250084in}}%
\pgfpathclose%
\pgfusepath{fill}%
\end{pgfscope}%
\begin{pgfscope}%
\pgfpathrectangle{\pgfqpoint{0.100000in}{0.100000in}}{\pgfqpoint{3.608454in}{2.310000in}}%
\pgfusepath{clip}%
\pgfsetbuttcap%
\pgfsetmiterjoin%
\definecolor{currentfill}{rgb}{0.000000,0.505882,0.747059}%
\pgfsetfillcolor{currentfill}%
\pgfsetlinewidth{0.000000pt}%
\definecolor{currentstroke}{rgb}{0.000000,0.000000,0.000000}%
\pgfsetstrokecolor{currentstroke}%
\pgfsetstrokeopacity{0.000000}%
\pgfsetdash{}{0pt}%
\pgfpathmoveto{\pgfqpoint{1.923573in}{0.599973in}}%
\pgfpathlineto{\pgfqpoint{1.925990in}{0.598363in}}%
\pgfpathlineto{\pgfqpoint{1.945778in}{0.610158in}}%
\pgfpathlineto{\pgfqpoint{1.957611in}{0.597327in}}%
\pgfpathlineto{\pgfqpoint{1.953286in}{0.593627in}}%
\pgfpathlineto{\pgfqpoint{1.950391in}{0.577980in}}%
\pgfpathlineto{\pgfqpoint{1.923720in}{0.556660in}}%
\pgfpathlineto{\pgfqpoint{1.915913in}{0.565873in}}%
\pgfpathlineto{\pgfqpoint{1.907128in}{0.576661in}}%
\pgfpathlineto{\pgfqpoint{1.923573in}{0.599973in}}%
\pgfpathclose%
\pgfusepath{fill}%
\end{pgfscope}%
\begin{pgfscope}%
\pgfpathrectangle{\pgfqpoint{0.100000in}{0.100000in}}{\pgfqpoint{3.608454in}{2.310000in}}%
\pgfusepath{clip}%
\pgfsetbuttcap%
\pgfsetmiterjoin%
\definecolor{currentfill}{rgb}{0.000000,0.623529,0.688235}%
\pgfsetfillcolor{currentfill}%
\pgfsetlinewidth{0.000000pt}%
\definecolor{currentstroke}{rgb}{0.000000,0.000000,0.000000}%
\pgfsetstrokecolor{currentstroke}%
\pgfsetstrokeopacity{0.000000}%
\pgfsetdash{}{0pt}%
\pgfpathmoveto{\pgfqpoint{1.888636in}{1.700186in}}%
\pgfpathlineto{\pgfqpoint{1.956891in}{1.697804in}}%
\pgfpathlineto{\pgfqpoint{1.968709in}{1.697472in}}%
\pgfpathlineto{\pgfqpoint{1.968146in}{1.671305in}}%
\pgfpathlineto{\pgfqpoint{1.928813in}{1.672403in}}%
\pgfpathlineto{\pgfqpoint{1.926808in}{1.664949in}}%
\pgfpathlineto{\pgfqpoint{1.927556in}{1.657034in}}%
\pgfpathlineto{\pgfqpoint{1.925767in}{1.649254in}}%
\pgfpathlineto{\pgfqpoint{1.922640in}{1.646360in}}%
\pgfpathlineto{\pgfqpoint{1.914725in}{1.650247in}}%
\pgfpathlineto{\pgfqpoint{1.905777in}{1.655731in}}%
\pgfpathlineto{\pgfqpoint{1.904126in}{1.659838in}}%
\pgfpathlineto{\pgfqpoint{1.902600in}{1.663282in}}%
\pgfpathlineto{\pgfqpoint{1.893266in}{1.666452in}}%
\pgfpathlineto{\pgfqpoint{1.888697in}{1.672141in}}%
\pgfpathlineto{\pgfqpoint{1.884096in}{1.673023in}}%
\pgfpathlineto{\pgfqpoint{1.881583in}{1.680508in}}%
\pgfpathlineto{\pgfqpoint{1.872226in}{1.686715in}}%
\pgfpathlineto{\pgfqpoint{1.867137in}{1.695946in}}%
\pgfpathlineto{\pgfqpoint{1.862167in}{1.696367in}}%
\pgfpathlineto{\pgfqpoint{1.859723in}{1.701494in}}%
\pgfpathlineto{\pgfqpoint{1.888636in}{1.700186in}}%
\pgfpathclose%
\pgfusepath{fill}%
\end{pgfscope}%
\begin{pgfscope}%
\pgfpathrectangle{\pgfqpoint{0.100000in}{0.100000in}}{\pgfqpoint{3.608454in}{2.310000in}}%
\pgfusepath{clip}%
\pgfsetbuttcap%
\pgfsetmiterjoin%
\definecolor{currentfill}{rgb}{0.000000,0.486275,0.756863}%
\pgfsetfillcolor{currentfill}%
\pgfsetlinewidth{0.000000pt}%
\definecolor{currentstroke}{rgb}{0.000000,0.000000,0.000000}%
\pgfsetstrokecolor{currentstroke}%
\pgfsetstrokeopacity{0.000000}%
\pgfsetdash{}{0pt}%
\pgfpathmoveto{\pgfqpoint{2.688201in}{1.338715in}}%
\pgfpathlineto{\pgfqpoint{2.674475in}{1.337189in}}%
\pgfpathlineto{\pgfqpoint{2.669050in}{1.334303in}}%
\pgfpathlineto{\pgfqpoint{2.667619in}{1.346789in}}%
\pgfpathlineto{\pgfqpoint{2.662944in}{1.348688in}}%
\pgfpathlineto{\pgfqpoint{2.643662in}{1.346766in}}%
\pgfpathlineto{\pgfqpoint{2.641627in}{1.364614in}}%
\pgfpathlineto{\pgfqpoint{2.638885in}{1.368939in}}%
\pgfpathlineto{\pgfqpoint{2.653405in}{1.370465in}}%
\pgfpathlineto{\pgfqpoint{2.650805in}{1.393553in}}%
\pgfpathlineto{\pgfqpoint{2.675138in}{1.396451in}}%
\pgfpathlineto{\pgfqpoint{2.676781in}{1.379202in}}%
\pgfpathlineto{\pgfqpoint{2.684115in}{1.380130in}}%
\pgfpathlineto{\pgfqpoint{2.691293in}{1.385879in}}%
\pgfpathlineto{\pgfqpoint{2.693342in}{1.363630in}}%
\pgfpathlineto{\pgfqpoint{2.683930in}{1.355595in}}%
\pgfpathlineto{\pgfqpoint{2.679482in}{1.355109in}}%
\pgfpathlineto{\pgfqpoint{2.680182in}{1.348259in}}%
\pgfpathlineto{\pgfqpoint{2.687551in}{1.344405in}}%
\pgfpathlineto{\pgfqpoint{2.688201in}{1.338715in}}%
\pgfpathclose%
\pgfusepath{fill}%
\end{pgfscope}%
\begin{pgfscope}%
\pgfpathrectangle{\pgfqpoint{0.100000in}{0.100000in}}{\pgfqpoint{3.608454in}{2.310000in}}%
\pgfusepath{clip}%
\pgfsetbuttcap%
\pgfsetmiterjoin%
\definecolor{currentfill}{rgb}{0.000000,0.596078,0.701961}%
\pgfsetfillcolor{currentfill}%
\pgfsetlinewidth{0.000000pt}%
\definecolor{currentstroke}{rgb}{0.000000,0.000000,0.000000}%
\pgfsetstrokecolor{currentstroke}%
\pgfsetstrokeopacity{0.000000}%
\pgfsetdash{}{0pt}%
\pgfpathmoveto{\pgfqpoint{2.043578in}{1.181785in}}%
\pgfpathlineto{\pgfqpoint{1.971148in}{1.183063in}}%
\pgfpathlineto{\pgfqpoint{1.971690in}{1.220737in}}%
\pgfpathlineto{\pgfqpoint{2.011063in}{1.220009in}}%
\pgfpathlineto{\pgfqpoint{2.012052in}{1.268224in}}%
\pgfpathlineto{\pgfqpoint{2.022297in}{1.268017in}}%
\pgfpathlineto{\pgfqpoint{2.022450in}{1.274884in}}%
\pgfpathlineto{\pgfqpoint{2.047190in}{1.274443in}}%
\pgfpathlineto{\pgfqpoint{2.047067in}{1.264108in}}%
\pgfpathlineto{\pgfqpoint{2.046716in}{1.239913in}}%
\pgfpathlineto{\pgfqpoint{2.045886in}{1.181780in}}%
\pgfpathlineto{\pgfqpoint{2.043578in}{1.181785in}}%
\pgfpathclose%
\pgfusepath{fill}%
\end{pgfscope}%
\begin{pgfscope}%
\pgfpathrectangle{\pgfqpoint{0.100000in}{0.100000in}}{\pgfqpoint{3.608454in}{2.310000in}}%
\pgfusepath{clip}%
\pgfsetbuttcap%
\pgfsetmiterjoin%
\definecolor{currentfill}{rgb}{0.000000,0.901961,0.549020}%
\pgfsetfillcolor{currentfill}%
\pgfsetlinewidth{0.000000pt}%
\definecolor{currentstroke}{rgb}{0.000000,0.000000,0.000000}%
\pgfsetstrokecolor{currentstroke}%
\pgfsetstrokeopacity{0.000000}%
\pgfsetdash{}{0pt}%
\pgfpathmoveto{\pgfqpoint{2.867721in}{1.199764in}}%
\pgfpathlineto{\pgfqpoint{2.827558in}{1.194906in}}%
\pgfpathlineto{\pgfqpoint{2.828062in}{1.198088in}}%
\pgfpathlineto{\pgfqpoint{2.837048in}{1.201444in}}%
\pgfpathlineto{\pgfqpoint{2.856686in}{1.210206in}}%
\pgfpathlineto{\pgfqpoint{2.859365in}{1.219154in}}%
\pgfpathlineto{\pgfqpoint{2.871264in}{1.223809in}}%
\pgfpathlineto{\pgfqpoint{2.870950in}{1.230065in}}%
\pgfpathlineto{\pgfqpoint{2.878218in}{1.237030in}}%
\pgfpathlineto{\pgfqpoint{2.878670in}{1.243251in}}%
\pgfpathlineto{\pgfqpoint{2.887645in}{1.250541in}}%
\pgfpathlineto{\pgfqpoint{2.902185in}{1.260526in}}%
\pgfpathlineto{\pgfqpoint{2.908022in}{1.254600in}}%
\pgfpathlineto{\pgfqpoint{2.916003in}{1.242873in}}%
\pgfpathlineto{\pgfqpoint{2.904928in}{1.234962in}}%
\pgfpathlineto{\pgfqpoint{2.907489in}{1.230091in}}%
\pgfpathlineto{\pgfqpoint{2.901040in}{1.226866in}}%
\pgfpathlineto{\pgfqpoint{2.886736in}{1.225061in}}%
\pgfpathlineto{\pgfqpoint{2.880082in}{1.220503in}}%
\pgfpathlineto{\pgfqpoint{2.876442in}{1.212071in}}%
\pgfpathlineto{\pgfqpoint{2.869228in}{1.206518in}}%
\pgfpathlineto{\pgfqpoint{2.867721in}{1.199764in}}%
\pgfpathclose%
\pgfusepath{fill}%
\end{pgfscope}%
\begin{pgfscope}%
\pgfpathrectangle{\pgfqpoint{0.100000in}{0.100000in}}{\pgfqpoint{3.608454in}{2.310000in}}%
\pgfusepath{clip}%
\pgfsetbuttcap%
\pgfsetmiterjoin%
\definecolor{currentfill}{rgb}{0.000000,0.505882,0.747059}%
\pgfsetfillcolor{currentfill}%
\pgfsetlinewidth{0.000000pt}%
\definecolor{currentstroke}{rgb}{0.000000,0.000000,0.000000}%
\pgfsetstrokecolor{currentstroke}%
\pgfsetstrokeopacity{0.000000}%
\pgfsetdash{}{0pt}%
\pgfpathmoveto{\pgfqpoint{1.683362in}{1.122232in}}%
\pgfpathlineto{\pgfqpoint{1.680896in}{1.087866in}}%
\pgfpathlineto{\pgfqpoint{1.646284in}{1.091077in}}%
\pgfpathlineto{\pgfqpoint{1.589893in}{1.095406in}}%
\pgfpathlineto{\pgfqpoint{1.590719in}{1.104616in}}%
\pgfpathlineto{\pgfqpoint{1.596009in}{1.164627in}}%
\pgfpathlineto{\pgfqpoint{1.598511in}{1.164402in}}%
\pgfpathlineto{\pgfqpoint{1.651744in}{1.159945in}}%
\pgfpathlineto{\pgfqpoint{1.648937in}{1.124836in}}%
\pgfpathlineto{\pgfqpoint{1.683362in}{1.122232in}}%
\pgfpathclose%
\pgfusepath{fill}%
\end{pgfscope}%
\begin{pgfscope}%
\pgfpathrectangle{\pgfqpoint{0.100000in}{0.100000in}}{\pgfqpoint{3.608454in}{2.310000in}}%
\pgfusepath{clip}%
\pgfsetbuttcap%
\pgfsetmiterjoin%
\definecolor{currentfill}{rgb}{0.000000,0.447059,0.776471}%
\pgfsetfillcolor{currentfill}%
\pgfsetlinewidth{0.000000pt}%
\definecolor{currentstroke}{rgb}{0.000000,0.000000,0.000000}%
\pgfsetstrokecolor{currentstroke}%
\pgfsetstrokeopacity{0.000000}%
\pgfsetdash{}{0pt}%
\pgfpathmoveto{\pgfqpoint{1.907060in}{0.482191in}}%
\pgfpathlineto{\pgfqpoint{1.900788in}{0.477356in}}%
\pgfpathlineto{\pgfqpoint{1.902126in}{0.472699in}}%
\pgfpathlineto{\pgfqpoint{1.836840in}{0.475148in}}%
\pgfpathlineto{\pgfqpoint{1.839000in}{0.522528in}}%
\pgfpathlineto{\pgfqpoint{1.840202in}{0.558059in}}%
\pgfpathlineto{\pgfqpoint{1.797569in}{0.560061in}}%
\pgfpathlineto{\pgfqpoint{1.799798in}{0.602976in}}%
\pgfpathlineto{\pgfqpoint{1.786496in}{0.603609in}}%
\pgfpathlineto{\pgfqpoint{1.787654in}{0.626021in}}%
\pgfpathlineto{\pgfqpoint{1.817099in}{0.623588in}}%
\pgfpathlineto{\pgfqpoint{1.834695in}{0.613722in}}%
\pgfpathlineto{\pgfqpoint{1.835751in}{0.642247in}}%
\pgfpathlineto{\pgfqpoint{1.858713in}{0.641352in}}%
\pgfpathlineto{\pgfqpoint{1.870125in}{0.624788in}}%
\pgfpathlineto{\pgfqpoint{1.878483in}{0.632521in}}%
\pgfpathlineto{\pgfqpoint{1.896461in}{0.616726in}}%
\pgfpathlineto{\pgfqpoint{1.898375in}{0.608966in}}%
\pgfpathlineto{\pgfqpoint{1.907278in}{0.617155in}}%
\pgfpathlineto{\pgfqpoint{1.918807in}{0.601623in}}%
\pgfpathlineto{\pgfqpoint{1.923573in}{0.599973in}}%
\pgfpathlineto{\pgfqpoint{1.907128in}{0.576661in}}%
\pgfpathlineto{\pgfqpoint{1.915913in}{0.565873in}}%
\pgfpathlineto{\pgfqpoint{1.882610in}{0.539717in}}%
\pgfpathlineto{\pgfqpoint{1.895145in}{0.523860in}}%
\pgfpathlineto{\pgfqpoint{1.889121in}{0.521871in}}%
\pgfpathlineto{\pgfqpoint{1.907719in}{0.482706in}}%
\pgfpathlineto{\pgfqpoint{1.907060in}{0.482191in}}%
\pgfpathclose%
\pgfusepath{fill}%
\end{pgfscope}%
\begin{pgfscope}%
\pgfpathrectangle{\pgfqpoint{0.100000in}{0.100000in}}{\pgfqpoint{3.608454in}{2.310000in}}%
\pgfusepath{clip}%
\pgfsetbuttcap%
\pgfsetmiterjoin%
\definecolor{currentfill}{rgb}{0.000000,0.572549,0.713725}%
\pgfsetfillcolor{currentfill}%
\pgfsetlinewidth{0.000000pt}%
\definecolor{currentstroke}{rgb}{0.000000,0.000000,0.000000}%
\pgfsetstrokecolor{currentstroke}%
\pgfsetstrokeopacity{0.000000}%
\pgfsetdash{}{0pt}%
\pgfpathmoveto{\pgfqpoint{1.831143in}{0.977031in}}%
\pgfpathlineto{\pgfqpoint{1.820478in}{0.987070in}}%
\pgfpathlineto{\pgfqpoint{1.818337in}{0.980492in}}%
\pgfpathlineto{\pgfqpoint{1.811536in}{0.984126in}}%
\pgfpathlineto{\pgfqpoint{1.805846in}{0.981264in}}%
\pgfpathlineto{\pgfqpoint{1.799655in}{0.981871in}}%
\pgfpathlineto{\pgfqpoint{1.785322in}{0.998334in}}%
\pgfpathlineto{\pgfqpoint{1.780677in}{0.997355in}}%
\pgfpathlineto{\pgfqpoint{1.781300in}{1.012079in}}%
\pgfpathlineto{\pgfqpoint{1.782538in}{1.034546in}}%
\pgfpathlineto{\pgfqpoint{1.796623in}{1.033765in}}%
\pgfpathlineto{\pgfqpoint{1.796992in}{1.040622in}}%
\pgfpathlineto{\pgfqpoint{1.863914in}{1.037452in}}%
\pgfpathlineto{\pgfqpoint{1.872137in}{1.035604in}}%
\pgfpathlineto{\pgfqpoint{1.871452in}{1.016408in}}%
\pgfpathlineto{\pgfqpoint{1.858037in}{1.016994in}}%
\pgfpathlineto{\pgfqpoint{1.857172in}{0.996308in}}%
\pgfpathlineto{\pgfqpoint{1.835171in}{0.996785in}}%
\pgfpathlineto{\pgfqpoint{1.831022in}{0.989497in}}%
\pgfpathlineto{\pgfqpoint{1.831143in}{0.977031in}}%
\pgfpathclose%
\pgfusepath{fill}%
\end{pgfscope}%
\begin{pgfscope}%
\pgfpathrectangle{\pgfqpoint{0.100000in}{0.100000in}}{\pgfqpoint{3.608454in}{2.310000in}}%
\pgfusepath{clip}%
\pgfsetbuttcap%
\pgfsetmiterjoin%
\definecolor{currentfill}{rgb}{0.000000,0.349020,0.825490}%
\pgfsetfillcolor{currentfill}%
\pgfsetlinewidth{0.000000pt}%
\definecolor{currentstroke}{rgb}{0.000000,0.000000,0.000000}%
\pgfsetstrokecolor{currentstroke}%
\pgfsetstrokeopacity{0.000000}%
\pgfsetdash{}{0pt}%
\pgfpathmoveto{\pgfqpoint{2.023407in}{1.319659in}}%
\pgfpathlineto{\pgfqpoint{2.023077in}{1.302462in}}%
\pgfpathlineto{\pgfqpoint{1.994270in}{1.303029in}}%
\pgfpathlineto{\pgfqpoint{1.987451in}{1.303171in}}%
\pgfpathlineto{\pgfqpoint{1.987581in}{1.310037in}}%
\pgfpathlineto{\pgfqpoint{1.960306in}{1.310682in}}%
\pgfpathlineto{\pgfqpoint{1.961886in}{1.365738in}}%
\pgfpathlineto{\pgfqpoint{1.962491in}{1.386340in}}%
\pgfpathlineto{\pgfqpoint{2.031304in}{1.384920in}}%
\pgfpathlineto{\pgfqpoint{2.043672in}{1.384754in}}%
\pgfpathlineto{\pgfqpoint{2.043381in}{1.357123in}}%
\pgfpathlineto{\pgfqpoint{2.034708in}{1.355023in}}%
\pgfpathlineto{\pgfqpoint{2.030596in}{1.356610in}}%
\pgfpathlineto{\pgfqpoint{2.021694in}{1.353951in}}%
\pgfpathlineto{\pgfqpoint{2.021423in}{1.343723in}}%
\pgfpathlineto{\pgfqpoint{2.014646in}{1.343860in}}%
\pgfpathlineto{\pgfqpoint{2.014334in}{1.326679in}}%
\pgfpathlineto{\pgfqpoint{2.021173in}{1.326536in}}%
\pgfpathlineto{\pgfqpoint{2.023407in}{1.319659in}}%
\pgfpathclose%
\pgfusepath{fill}%
\end{pgfscope}%
\begin{pgfscope}%
\pgfpathrectangle{\pgfqpoint{0.100000in}{0.100000in}}{\pgfqpoint{3.608454in}{2.310000in}}%
\pgfusepath{clip}%
\pgfsetbuttcap%
\pgfsetmiterjoin%
\definecolor{currentfill}{rgb}{0.000000,0.482353,0.758824}%
\pgfsetfillcolor{currentfill}%
\pgfsetlinewidth{0.000000pt}%
\definecolor{currentstroke}{rgb}{0.000000,0.000000,0.000000}%
\pgfsetstrokecolor{currentstroke}%
\pgfsetstrokeopacity{0.000000}%
\pgfsetdash{}{0pt}%
\pgfpathmoveto{\pgfqpoint{2.986399in}{1.102377in}}%
\pgfpathlineto{\pgfqpoint{2.947537in}{1.098430in}}%
\pgfpathlineto{\pgfqpoint{2.946773in}{1.103937in}}%
\pgfpathlineto{\pgfqpoint{2.932868in}{1.114078in}}%
\pgfpathlineto{\pgfqpoint{2.926537in}{1.111891in}}%
\pgfpathlineto{\pgfqpoint{2.924776in}{1.115579in}}%
\pgfpathlineto{\pgfqpoint{2.926944in}{1.122067in}}%
\pgfpathlineto{\pgfqpoint{2.930944in}{1.123301in}}%
\pgfpathlineto{\pgfqpoint{2.943450in}{1.124830in}}%
\pgfpathlineto{\pgfqpoint{2.951598in}{1.127420in}}%
\pgfpathlineto{\pgfqpoint{2.956399in}{1.133737in}}%
\pgfpathlineto{\pgfqpoint{2.963707in}{1.130755in}}%
\pgfpathlineto{\pgfqpoint{2.970938in}{1.132406in}}%
\pgfpathlineto{\pgfqpoint{3.007763in}{1.136339in}}%
\pgfpathlineto{\pgfqpoint{3.008399in}{1.134076in}}%
\pgfpathlineto{\pgfqpoint{3.009870in}{1.122844in}}%
\pgfpathlineto{\pgfqpoint{3.007642in}{1.113989in}}%
\pgfpathlineto{\pgfqpoint{3.008536in}{1.104446in}}%
\pgfpathlineto{\pgfqpoint{2.986399in}{1.102377in}}%
\pgfpathclose%
\pgfusepath{fill}%
\end{pgfscope}%
\begin{pgfscope}%
\pgfpathrectangle{\pgfqpoint{0.100000in}{0.100000in}}{\pgfqpoint{3.608454in}{2.310000in}}%
\pgfusepath{clip}%
\pgfsetbuttcap%
\pgfsetmiterjoin%
\definecolor{currentfill}{rgb}{0.000000,0.631373,0.684314}%
\pgfsetfillcolor{currentfill}%
\pgfsetlinewidth{0.000000pt}%
\definecolor{currentstroke}{rgb}{0.000000,0.000000,0.000000}%
\pgfsetstrokecolor{currentstroke}%
\pgfsetstrokeopacity{0.000000}%
\pgfsetdash{}{0pt}%
\pgfpathmoveto{\pgfqpoint{2.922696in}{0.996327in}}%
\pgfpathlineto{\pgfqpoint{2.916400in}{0.998276in}}%
\pgfpathlineto{\pgfqpoint{2.908530in}{0.995600in}}%
\pgfpathlineto{\pgfqpoint{2.897463in}{0.997613in}}%
\pgfpathlineto{\pgfqpoint{2.885344in}{1.008668in}}%
\pgfpathlineto{\pgfqpoint{2.886412in}{1.013136in}}%
\pgfpathlineto{\pgfqpoint{2.871596in}{1.014173in}}%
\pgfpathlineto{\pgfqpoint{2.868390in}{1.010555in}}%
\pgfpathlineto{\pgfqpoint{2.864861in}{1.018162in}}%
\pgfpathlineto{\pgfqpoint{2.864516in}{1.026478in}}%
\pgfpathlineto{\pgfqpoint{2.859003in}{1.029918in}}%
\pgfpathlineto{\pgfqpoint{2.863172in}{1.045964in}}%
\pgfpathlineto{\pgfqpoint{2.864794in}{1.047108in}}%
\pgfpathlineto{\pgfqpoint{2.872404in}{1.042068in}}%
\pgfpathlineto{\pgfqpoint{2.876324in}{1.042276in}}%
\pgfpathlineto{\pgfqpoint{2.884093in}{1.034847in}}%
\pgfpathlineto{\pgfqpoint{2.888625in}{1.033073in}}%
\pgfpathlineto{\pgfqpoint{2.895066in}{1.035008in}}%
\pgfpathlineto{\pgfqpoint{2.907921in}{1.017944in}}%
\pgfpathlineto{\pgfqpoint{2.910642in}{1.010268in}}%
\pgfpathlineto{\pgfqpoint{2.919954in}{1.001789in}}%
\pgfpathlineto{\pgfqpoint{2.922696in}{0.996327in}}%
\pgfpathclose%
\pgfusepath{fill}%
\end{pgfscope}%
\begin{pgfscope}%
\pgfpathrectangle{\pgfqpoint{0.100000in}{0.100000in}}{\pgfqpoint{3.608454in}{2.310000in}}%
\pgfusepath{clip}%
\pgfsetbuttcap%
\pgfsetmiterjoin%
\definecolor{currentfill}{rgb}{0.000000,0.647059,0.676471}%
\pgfsetfillcolor{currentfill}%
\pgfsetlinewidth{0.000000pt}%
\definecolor{currentstroke}{rgb}{0.000000,0.000000,0.000000}%
\pgfsetstrokecolor{currentstroke}%
\pgfsetstrokeopacity{0.000000}%
\pgfsetdash{}{0pt}%
\pgfpathmoveto{\pgfqpoint{0.960301in}{1.531851in}}%
\pgfpathlineto{\pgfqpoint{0.954290in}{1.503771in}}%
\pgfpathlineto{\pgfqpoint{0.939931in}{1.436894in}}%
\pgfpathlineto{\pgfqpoint{0.882599in}{1.449402in}}%
\pgfpathlineto{\pgfqpoint{0.836825in}{1.499170in}}%
\pgfpathlineto{\pgfqpoint{0.795453in}{1.508995in}}%
\pgfpathlineto{\pgfqpoint{0.752002in}{1.519867in}}%
\pgfpathlineto{\pgfqpoint{0.724276in}{1.521184in}}%
\pgfpathlineto{\pgfqpoint{0.725481in}{1.527601in}}%
\pgfpathlineto{\pgfqpoint{0.730363in}{1.531286in}}%
\pgfpathlineto{\pgfqpoint{0.730097in}{1.540859in}}%
\pgfpathlineto{\pgfqpoint{0.738497in}{1.547647in}}%
\pgfpathlineto{\pgfqpoint{0.749642in}{1.550822in}}%
\pgfpathlineto{\pgfqpoint{0.753176in}{1.557006in}}%
\pgfpathlineto{\pgfqpoint{0.753615in}{1.567143in}}%
\pgfpathlineto{\pgfqpoint{0.758589in}{1.580048in}}%
\pgfpathlineto{\pgfqpoint{0.755783in}{1.587192in}}%
\pgfpathlineto{\pgfqpoint{0.785394in}{1.632033in}}%
\pgfpathlineto{\pgfqpoint{0.798688in}{1.628778in}}%
\pgfpathlineto{\pgfqpoint{0.805480in}{1.656197in}}%
\pgfpathlineto{\pgfqpoint{0.824452in}{1.732928in}}%
\pgfpathlineto{\pgfqpoint{0.861529in}{1.723655in}}%
\pgfpathlineto{\pgfqpoint{0.937739in}{1.705964in}}%
\pgfpathlineto{\pgfqpoint{0.949739in}{1.703791in}}%
\pgfpathlineto{\pgfqpoint{0.981161in}{1.696275in}}%
\pgfpathlineto{\pgfqpoint{0.994964in}{1.693281in}}%
\pgfpathlineto{\pgfqpoint{0.978586in}{1.616384in}}%
\pgfpathlineto{\pgfqpoint{0.960301in}{1.531851in}}%
\pgfpathclose%
\pgfusepath{fill}%
\end{pgfscope}%
\begin{pgfscope}%
\pgfpathrectangle{\pgfqpoint{0.100000in}{0.100000in}}{\pgfqpoint{3.608454in}{2.310000in}}%
\pgfusepath{clip}%
\pgfsetbuttcap%
\pgfsetmiterjoin%
\definecolor{currentfill}{rgb}{0.000000,0.796078,0.601961}%
\pgfsetfillcolor{currentfill}%
\pgfsetlinewidth{0.000000pt}%
\definecolor{currentstroke}{rgb}{0.000000,0.000000,0.000000}%
\pgfsetstrokecolor{currentstroke}%
\pgfsetstrokeopacity{0.000000}%
\pgfsetdash{}{0pt}%
\pgfpathmoveto{\pgfqpoint{1.859723in}{1.701494in}}%
\pgfpathlineto{\pgfqpoint{1.862167in}{1.696367in}}%
\pgfpathlineto{\pgfqpoint{1.867137in}{1.695946in}}%
\pgfpathlineto{\pgfqpoint{1.872226in}{1.686715in}}%
\pgfpathlineto{\pgfqpoint{1.881583in}{1.680508in}}%
\pgfpathlineto{\pgfqpoint{1.884096in}{1.673023in}}%
\pgfpathlineto{\pgfqpoint{1.888697in}{1.672141in}}%
\pgfpathlineto{\pgfqpoint{1.893266in}{1.666452in}}%
\pgfpathlineto{\pgfqpoint{1.902600in}{1.663282in}}%
\pgfpathlineto{\pgfqpoint{1.904126in}{1.659838in}}%
\pgfpathlineto{\pgfqpoint{1.860373in}{1.661671in}}%
\pgfpathlineto{\pgfqpoint{1.805743in}{1.664546in}}%
\pgfpathlineto{\pgfqpoint{1.805445in}{1.674881in}}%
\pgfpathlineto{\pgfqpoint{1.806996in}{1.702523in}}%
\pgfpathlineto{\pgfqpoint{1.807130in}{1.721260in}}%
\pgfpathlineto{\pgfqpoint{1.817373in}{1.718961in}}%
\pgfpathlineto{\pgfqpoint{1.829398in}{1.718027in}}%
\pgfpathlineto{\pgfqpoint{1.842006in}{1.721715in}}%
\pgfpathlineto{\pgfqpoint{1.841021in}{1.702423in}}%
\pgfpathlineto{\pgfqpoint{1.859723in}{1.701494in}}%
\pgfpathclose%
\pgfusepath{fill}%
\end{pgfscope}%
\begin{pgfscope}%
\pgfpathrectangle{\pgfqpoint{0.100000in}{0.100000in}}{\pgfqpoint{3.608454in}{2.310000in}}%
\pgfusepath{clip}%
\pgfsetbuttcap%
\pgfsetmiterjoin%
\definecolor{currentfill}{rgb}{0.000000,0.709804,0.645098}%
\pgfsetfillcolor{currentfill}%
\pgfsetlinewidth{0.000000pt}%
\definecolor{currentstroke}{rgb}{0.000000,0.000000,0.000000}%
\pgfsetstrokecolor{currentstroke}%
\pgfsetstrokeopacity{0.000000}%
\pgfsetdash{}{0pt}%
\pgfpathmoveto{\pgfqpoint{2.811462in}{0.766109in}}%
\pgfpathlineto{\pgfqpoint{2.792796in}{0.763762in}}%
\pgfpathlineto{\pgfqpoint{2.794089in}{0.749184in}}%
\pgfpathlineto{\pgfqpoint{2.786852in}{0.748618in}}%
\pgfpathlineto{\pgfqpoint{2.781206in}{0.755220in}}%
\pgfpathlineto{\pgfqpoint{2.779513in}{0.764362in}}%
\pgfpathlineto{\pgfqpoint{2.781911in}{0.785540in}}%
\pgfpathlineto{\pgfqpoint{2.780161in}{0.791452in}}%
\pgfpathlineto{\pgfqpoint{2.774910in}{0.796997in}}%
\pgfpathlineto{\pgfqpoint{2.755229in}{0.795820in}}%
\pgfpathlineto{\pgfqpoint{2.755934in}{0.788907in}}%
\pgfpathlineto{\pgfqpoint{2.733549in}{0.786465in}}%
\pgfpathlineto{\pgfqpoint{2.737974in}{0.800305in}}%
\pgfpathlineto{\pgfqpoint{2.737520in}{0.807875in}}%
\pgfpathlineto{\pgfqpoint{2.741287in}{0.817640in}}%
\pgfpathlineto{\pgfqpoint{2.751810in}{0.820152in}}%
\pgfpathlineto{\pgfqpoint{2.751930in}{0.830708in}}%
\pgfpathlineto{\pgfqpoint{2.762136in}{0.831926in}}%
\pgfpathlineto{\pgfqpoint{2.767748in}{0.825602in}}%
\pgfpathlineto{\pgfqpoint{2.776392in}{0.826653in}}%
\pgfpathlineto{\pgfqpoint{2.778092in}{0.819839in}}%
\pgfpathlineto{\pgfqpoint{2.786918in}{0.816730in}}%
\pgfpathlineto{\pgfqpoint{2.808283in}{0.818788in}}%
\pgfpathlineto{\pgfqpoint{2.809408in}{0.807037in}}%
\pgfpathlineto{\pgfqpoint{2.814786in}{0.800405in}}%
\pgfpathlineto{\pgfqpoint{2.821071in}{0.796344in}}%
\pgfpathlineto{\pgfqpoint{2.821443in}{0.790502in}}%
\pgfpathlineto{\pgfqpoint{2.825145in}{0.783605in}}%
\pgfpathlineto{\pgfqpoint{2.820126in}{0.780667in}}%
\pgfpathlineto{\pgfqpoint{2.810152in}{0.780002in}}%
\pgfpathlineto{\pgfqpoint{2.811462in}{0.766109in}}%
\pgfpathclose%
\pgfusepath{fill}%
\end{pgfscope}%
\begin{pgfscope}%
\pgfpathrectangle{\pgfqpoint{0.100000in}{0.100000in}}{\pgfqpoint{3.608454in}{2.310000in}}%
\pgfusepath{clip}%
\pgfsetbuttcap%
\pgfsetmiterjoin%
\definecolor{currentfill}{rgb}{0.000000,0.635294,0.682353}%
\pgfsetfillcolor{currentfill}%
\pgfsetlinewidth{0.000000pt}%
\definecolor{currentstroke}{rgb}{0.000000,0.000000,0.000000}%
\pgfsetstrokecolor{currentstroke}%
\pgfsetstrokeopacity{0.000000}%
\pgfsetdash{}{0pt}%
\pgfpathmoveto{\pgfqpoint{1.088551in}{1.993499in}}%
\pgfpathlineto{\pgfqpoint{1.086269in}{1.988945in}}%
\pgfpathlineto{\pgfqpoint{1.076963in}{1.988148in}}%
\pgfpathlineto{\pgfqpoint{1.069612in}{1.982716in}}%
\pgfpathlineto{\pgfqpoint{1.061696in}{1.979359in}}%
\pgfpathlineto{\pgfqpoint{1.053520in}{1.975559in}}%
\pgfpathlineto{\pgfqpoint{1.048671in}{1.970878in}}%
\pgfpathlineto{\pgfqpoint{1.034877in}{1.968576in}}%
\pgfpathlineto{\pgfqpoint{1.028023in}{1.976205in}}%
\pgfpathlineto{\pgfqpoint{1.025284in}{1.975956in}}%
\pgfpathlineto{\pgfqpoint{1.030719in}{1.983835in}}%
\pgfpathlineto{\pgfqpoint{1.028886in}{1.992775in}}%
\pgfpathlineto{\pgfqpoint{1.033186in}{1.997675in}}%
\pgfpathlineto{\pgfqpoint{1.040540in}{1.999733in}}%
\pgfpathlineto{\pgfqpoint{1.037550in}{2.013584in}}%
\pgfpathlineto{\pgfqpoint{1.042048in}{2.022019in}}%
\pgfpathlineto{\pgfqpoint{1.042825in}{2.030433in}}%
\pgfpathlineto{\pgfqpoint{1.049410in}{2.040224in}}%
\pgfpathlineto{\pgfqpoint{1.056865in}{2.058071in}}%
\pgfpathlineto{\pgfqpoint{1.055012in}{2.060043in}}%
\pgfpathlineto{\pgfqpoint{1.041035in}{2.060695in}}%
\pgfpathlineto{\pgfqpoint{1.041857in}{2.066086in}}%
\pgfpathlineto{\pgfqpoint{1.027942in}{2.078990in}}%
\pgfpathlineto{\pgfqpoint{1.027701in}{2.087646in}}%
\pgfpathlineto{\pgfqpoint{1.023030in}{2.093793in}}%
\pgfpathlineto{\pgfqpoint{1.014258in}{2.117180in}}%
\pgfpathlineto{\pgfqpoint{1.003956in}{2.123021in}}%
\pgfpathlineto{\pgfqpoint{1.003713in}{2.127225in}}%
\pgfpathlineto{\pgfqpoint{0.994972in}{2.136779in}}%
\pgfpathlineto{\pgfqpoint{1.001953in}{2.138275in}}%
\pgfpathlineto{\pgfqpoint{1.011451in}{2.134988in}}%
\pgfpathlineto{\pgfqpoint{1.020317in}{2.134665in}}%
\pgfpathlineto{\pgfqpoint{1.034696in}{2.123940in}}%
\pgfpathlineto{\pgfqpoint{1.033798in}{2.117353in}}%
\pgfpathlineto{\pgfqpoint{1.039741in}{2.112438in}}%
\pgfpathlineto{\pgfqpoint{1.048078in}{2.111068in}}%
\pgfpathlineto{\pgfqpoint{1.060904in}{2.103004in}}%
\pgfpathlineto{\pgfqpoint{1.068378in}{2.094075in}}%
\pgfpathlineto{\pgfqpoint{1.072235in}{2.094680in}}%
\pgfpathlineto{\pgfqpoint{1.085713in}{2.091817in}}%
\pgfpathlineto{\pgfqpoint{1.092364in}{2.093954in}}%
\pgfpathlineto{\pgfqpoint{1.091464in}{2.102885in}}%
\pgfpathlineto{\pgfqpoint{1.093234in}{2.110956in}}%
\pgfpathlineto{\pgfqpoint{1.090372in}{2.118413in}}%
\pgfpathlineto{\pgfqpoint{1.093543in}{2.127613in}}%
\pgfpathlineto{\pgfqpoint{1.118004in}{2.122768in}}%
\pgfpathlineto{\pgfqpoint{1.111233in}{2.089809in}}%
\pgfpathlineto{\pgfqpoint{1.119922in}{2.088067in}}%
\pgfpathlineto{\pgfqpoint{1.114370in}{2.060903in}}%
\pgfpathlineto{\pgfqpoint{1.108312in}{2.062127in}}%
\pgfpathlineto{\pgfqpoint{1.103510in}{2.056028in}}%
\pgfpathlineto{\pgfqpoint{1.092981in}{2.055856in}}%
\pgfpathlineto{\pgfqpoint{1.092070in}{2.051353in}}%
\pgfpathlineto{\pgfqpoint{1.085312in}{2.052758in}}%
\pgfpathlineto{\pgfqpoint{1.078479in}{2.042039in}}%
\pgfpathlineto{\pgfqpoint{1.077963in}{2.034819in}}%
\pgfpathlineto{\pgfqpoint{1.081054in}{2.029697in}}%
\pgfpathlineto{\pgfqpoint{1.081296in}{2.021948in}}%
\pgfpathlineto{\pgfqpoint{1.076933in}{2.013289in}}%
\pgfpathlineto{\pgfqpoint{1.076031in}{2.007092in}}%
\pgfpathlineto{\pgfqpoint{1.080846in}{2.002972in}}%
\pgfpathlineto{\pgfqpoint{1.081829in}{1.997042in}}%
\pgfpathlineto{\pgfqpoint{1.088551in}{1.993499in}}%
\pgfpathclose%
\pgfusepath{fill}%
\end{pgfscope}%
\begin{pgfscope}%
\pgfpathrectangle{\pgfqpoint{0.100000in}{0.100000in}}{\pgfqpoint{3.608454in}{2.310000in}}%
\pgfusepath{clip}%
\pgfsetbuttcap%
\pgfsetmiterjoin%
\definecolor{currentfill}{rgb}{0.000000,0.701961,0.649020}%
\pgfsetfillcolor{currentfill}%
\pgfsetlinewidth{0.000000pt}%
\definecolor{currentstroke}{rgb}{0.000000,0.000000,0.000000}%
\pgfsetstrokecolor{currentstroke}%
\pgfsetstrokeopacity{0.000000}%
\pgfsetdash{}{0pt}%
\pgfpathmoveto{\pgfqpoint{2.193329in}{1.907307in}}%
\pgfpathlineto{\pgfqpoint{2.214175in}{1.908040in}}%
\pgfpathlineto{\pgfqpoint{2.213676in}{1.928846in}}%
\pgfpathlineto{\pgfqpoint{2.255541in}{1.929754in}}%
\pgfpathlineto{\pgfqpoint{2.256111in}{1.909051in}}%
\pgfpathlineto{\pgfqpoint{2.256290in}{1.902446in}}%
\pgfpathlineto{\pgfqpoint{2.253310in}{1.897735in}}%
\pgfpathlineto{\pgfqpoint{2.237467in}{1.890615in}}%
\pgfpathlineto{\pgfqpoint{2.233501in}{1.887214in}}%
\pgfpathlineto{\pgfqpoint{2.229731in}{1.876962in}}%
\pgfpathlineto{\pgfqpoint{2.226726in}{1.874225in}}%
\pgfpathlineto{\pgfqpoint{2.189353in}{1.873824in}}%
\pgfpathlineto{\pgfqpoint{2.188794in}{1.893513in}}%
\pgfpathlineto{\pgfqpoint{2.193529in}{1.893561in}}%
\pgfpathlineto{\pgfqpoint{2.193329in}{1.907307in}}%
\pgfpathclose%
\pgfusepath{fill}%
\end{pgfscope}%
\begin{pgfscope}%
\pgfpathrectangle{\pgfqpoint{0.100000in}{0.100000in}}{\pgfqpoint{3.608454in}{2.310000in}}%
\pgfusepath{clip}%
\pgfsetbuttcap%
\pgfsetmiterjoin%
\definecolor{currentfill}{rgb}{0.000000,0.592157,0.703922}%
\pgfsetfillcolor{currentfill}%
\pgfsetlinewidth{0.000000pt}%
\definecolor{currentstroke}{rgb}{0.000000,0.000000,0.000000}%
\pgfsetstrokecolor{currentstroke}%
\pgfsetstrokeopacity{0.000000}%
\pgfsetdash{}{0pt}%
\pgfpathmoveto{\pgfqpoint{1.854168in}{1.187060in}}%
\pgfpathlineto{\pgfqpoint{1.825437in}{1.188388in}}%
\pgfpathlineto{\pgfqpoint{1.820080in}{1.188664in}}%
\pgfpathlineto{\pgfqpoint{1.821457in}{1.218805in}}%
\pgfpathlineto{\pgfqpoint{1.855479in}{1.217455in}}%
\pgfpathlineto{\pgfqpoint{1.854168in}{1.187060in}}%
\pgfpathclose%
\pgfusepath{fill}%
\end{pgfscope}%
\begin{pgfscope}%
\pgfpathrectangle{\pgfqpoint{0.100000in}{0.100000in}}{\pgfqpoint{3.608454in}{2.310000in}}%
\pgfusepath{clip}%
\pgfsetbuttcap%
\pgfsetmiterjoin%
\definecolor{currentfill}{rgb}{0.000000,0.701961,0.649020}%
\pgfsetfillcolor{currentfill}%
\pgfsetlinewidth{0.000000pt}%
\definecolor{currentstroke}{rgb}{0.000000,0.000000,0.000000}%
\pgfsetstrokecolor{currentstroke}%
\pgfsetstrokeopacity{0.000000}%
\pgfsetdash{}{0pt}%
\pgfpathmoveto{\pgfqpoint{2.589411in}{1.214310in}}%
\pgfpathlineto{\pgfqpoint{2.577947in}{1.209252in}}%
\pgfpathlineto{\pgfqpoint{2.567654in}{1.211922in}}%
\pgfpathlineto{\pgfqpoint{2.566571in}{1.201683in}}%
\pgfpathlineto{\pgfqpoint{2.563459in}{1.197714in}}%
\pgfpathlineto{\pgfqpoint{2.555635in}{1.195943in}}%
\pgfpathlineto{\pgfqpoint{2.539380in}{1.187318in}}%
\pgfpathlineto{\pgfqpoint{2.533377in}{1.204248in}}%
\pgfpathlineto{\pgfqpoint{2.535791in}{1.209919in}}%
\pgfpathlineto{\pgfqpoint{2.533129in}{1.219044in}}%
\pgfpathlineto{\pgfqpoint{2.523585in}{1.228784in}}%
\pgfpathlineto{\pgfqpoint{2.528115in}{1.232902in}}%
\pgfpathlineto{\pgfqpoint{2.540147in}{1.235271in}}%
\pgfpathlineto{\pgfqpoint{2.534558in}{1.250105in}}%
\pgfpathlineto{\pgfqpoint{2.541890in}{1.261426in}}%
\pgfpathlineto{\pgfqpoint{2.549388in}{1.262674in}}%
\pgfpathlineto{\pgfqpoint{2.546912in}{1.269077in}}%
\pgfpathlineto{\pgfqpoint{2.553706in}{1.268477in}}%
\pgfpathlineto{\pgfqpoint{2.562848in}{1.271335in}}%
\pgfpathlineto{\pgfqpoint{2.567447in}{1.266096in}}%
\pgfpathlineto{\pgfqpoint{2.570793in}{1.273710in}}%
\pgfpathlineto{\pgfqpoint{2.578231in}{1.275886in}}%
\pgfpathlineto{\pgfqpoint{2.586412in}{1.272897in}}%
\pgfpathlineto{\pgfqpoint{2.585709in}{1.266483in}}%
\pgfpathlineto{\pgfqpoint{2.577210in}{1.248460in}}%
\pgfpathlineto{\pgfqpoint{2.585128in}{1.245024in}}%
\pgfpathlineto{\pgfqpoint{2.586421in}{1.235460in}}%
\pgfpathlineto{\pgfqpoint{2.589982in}{1.231634in}}%
\pgfpathlineto{\pgfqpoint{2.585378in}{1.221768in}}%
\pgfpathlineto{\pgfqpoint{2.589411in}{1.214310in}}%
\pgfpathclose%
\pgfusepath{fill}%
\end{pgfscope}%
\begin{pgfscope}%
\pgfpathrectangle{\pgfqpoint{0.100000in}{0.100000in}}{\pgfqpoint{3.608454in}{2.310000in}}%
\pgfusepath{clip}%
\pgfsetbuttcap%
\pgfsetmiterjoin%
\definecolor{currentfill}{rgb}{0.000000,0.588235,0.705882}%
\pgfsetfillcolor{currentfill}%
\pgfsetlinewidth{0.000000pt}%
\definecolor{currentstroke}{rgb}{0.000000,0.000000,0.000000}%
\pgfsetstrokecolor{currentstroke}%
\pgfsetstrokeopacity{0.000000}%
\pgfsetdash{}{0pt}%
\pgfpathmoveto{\pgfqpoint{1.638167in}{2.022688in}}%
\pgfpathlineto{\pgfqpoint{1.635493in}{1.994898in}}%
\pgfpathlineto{\pgfqpoint{1.638609in}{1.994610in}}%
\pgfpathlineto{\pgfqpoint{1.635984in}{1.966799in}}%
\pgfpathlineto{\pgfqpoint{1.625532in}{1.967784in}}%
\pgfpathlineto{\pgfqpoint{1.624849in}{1.960734in}}%
\pgfpathlineto{\pgfqpoint{1.611520in}{1.962062in}}%
\pgfpathlineto{\pgfqpoint{1.612308in}{1.970040in}}%
\pgfpathlineto{\pgfqpoint{1.595495in}{1.971747in}}%
\pgfpathlineto{\pgfqpoint{1.592373in}{1.975555in}}%
\pgfpathlineto{\pgfqpoint{1.582076in}{1.976639in}}%
\pgfpathlineto{\pgfqpoint{1.583733in}{1.990580in}}%
\pgfpathlineto{\pgfqpoint{1.560751in}{1.993122in}}%
\pgfpathlineto{\pgfqpoint{1.558985in}{1.997946in}}%
\pgfpathlineto{\pgfqpoint{1.552079in}{1.998723in}}%
\pgfpathlineto{\pgfqpoint{1.552609in}{2.003322in}}%
\pgfpathlineto{\pgfqpoint{1.545674in}{2.004107in}}%
\pgfpathlineto{\pgfqpoint{1.547528in}{2.020313in}}%
\pgfpathlineto{\pgfqpoint{1.543137in}{2.020826in}}%
\pgfpathlineto{\pgfqpoint{1.546296in}{2.048425in}}%
\pgfpathlineto{\pgfqpoint{1.548607in}{2.048174in}}%
\pgfpathlineto{\pgfqpoint{1.551007in}{2.068911in}}%
\pgfpathlineto{\pgfqpoint{1.557865in}{2.068157in}}%
\pgfpathlineto{\pgfqpoint{1.557068in}{2.061233in}}%
\pgfpathlineto{\pgfqpoint{1.570807in}{2.059681in}}%
\pgfpathlineto{\pgfqpoint{1.570016in}{2.052759in}}%
\pgfpathlineto{\pgfqpoint{1.590639in}{2.050503in}}%
\pgfpathlineto{\pgfqpoint{1.589896in}{2.043588in}}%
\pgfpathlineto{\pgfqpoint{1.594243in}{2.043152in}}%
\pgfpathlineto{\pgfqpoint{1.592763in}{2.029281in}}%
\pgfpathlineto{\pgfqpoint{1.613315in}{2.027119in}}%
\pgfpathlineto{\pgfqpoint{1.617764in}{2.024744in}}%
\pgfpathlineto{\pgfqpoint{1.638167in}{2.022688in}}%
\pgfpathclose%
\pgfusepath{fill}%
\end{pgfscope}%
\begin{pgfscope}%
\pgfpathrectangle{\pgfqpoint{0.100000in}{0.100000in}}{\pgfqpoint{3.608454in}{2.310000in}}%
\pgfusepath{clip}%
\pgfsetbuttcap%
\pgfsetmiterjoin%
\definecolor{currentfill}{rgb}{0.000000,0.647059,0.676471}%
\pgfsetfillcolor{currentfill}%
\pgfsetlinewidth{0.000000pt}%
\definecolor{currentstroke}{rgb}{0.000000,0.000000,0.000000}%
\pgfsetstrokecolor{currentstroke}%
\pgfsetstrokeopacity{0.000000}%
\pgfsetdash{}{0pt}%
\pgfpathmoveto{\pgfqpoint{2.534753in}{1.425828in}}%
\pgfpathlineto{\pgfqpoint{2.534185in}{1.410410in}}%
\pgfpathlineto{\pgfqpoint{2.528514in}{1.407327in}}%
\pgfpathlineto{\pgfqpoint{2.503724in}{1.405402in}}%
\pgfpathlineto{\pgfqpoint{2.506265in}{1.370984in}}%
\pgfpathlineto{\pgfqpoint{2.465371in}{1.368430in}}%
\pgfpathlineto{\pgfqpoint{2.464725in}{1.378777in}}%
\pgfpathlineto{\pgfqpoint{2.471713in}{1.379003in}}%
\pgfpathlineto{\pgfqpoint{2.470095in}{1.403294in}}%
\pgfpathlineto{\pgfqpoint{2.463205in}{1.402934in}}%
\pgfpathlineto{\pgfqpoint{2.462297in}{1.414401in}}%
\pgfpathlineto{\pgfqpoint{2.457647in}{1.415090in}}%
\pgfpathlineto{\pgfqpoint{2.457108in}{1.423280in}}%
\pgfpathlineto{\pgfqpoint{2.461611in}{1.423638in}}%
\pgfpathlineto{\pgfqpoint{2.460863in}{1.433950in}}%
\pgfpathlineto{\pgfqpoint{2.484922in}{1.436043in}}%
\pgfpathlineto{\pgfqpoint{2.493952in}{1.454636in}}%
\pgfpathlineto{\pgfqpoint{2.500893in}{1.455137in}}%
\pgfpathlineto{\pgfqpoint{2.498999in}{1.481611in}}%
\pgfpathlineto{\pgfqpoint{2.512434in}{1.482660in}}%
\pgfpathlineto{\pgfqpoint{2.509486in}{1.512307in}}%
\pgfpathlineto{\pgfqpoint{2.516336in}{1.513050in}}%
\pgfpathlineto{\pgfqpoint{2.520205in}{1.472936in}}%
\pgfpathlineto{\pgfqpoint{2.531157in}{1.473630in}}%
\pgfpathlineto{\pgfqpoint{2.532347in}{1.453054in}}%
\pgfpathlineto{\pgfqpoint{2.534753in}{1.425828in}}%
\pgfpathclose%
\pgfusepath{fill}%
\end{pgfscope}%
\begin{pgfscope}%
\pgfpathrectangle{\pgfqpoint{0.100000in}{0.100000in}}{\pgfqpoint{3.608454in}{2.310000in}}%
\pgfusepath{clip}%
\pgfsetbuttcap%
\pgfsetmiterjoin%
\definecolor{currentfill}{rgb}{0.000000,0.478431,0.760784}%
\pgfsetfillcolor{currentfill}%
\pgfsetlinewidth{0.000000pt}%
\definecolor{currentstroke}{rgb}{0.000000,0.000000,0.000000}%
\pgfsetstrokecolor{currentstroke}%
\pgfsetstrokeopacity{0.000000}%
\pgfsetdash{}{0pt}%
\pgfpathmoveto{\pgfqpoint{1.638167in}{2.022688in}}%
\pgfpathlineto{\pgfqpoint{1.617764in}{2.024744in}}%
\pgfpathlineto{\pgfqpoint{1.613315in}{2.027119in}}%
\pgfpathlineto{\pgfqpoint{1.592763in}{2.029281in}}%
\pgfpathlineto{\pgfqpoint{1.594243in}{2.043152in}}%
\pgfpathlineto{\pgfqpoint{1.589896in}{2.043588in}}%
\pgfpathlineto{\pgfqpoint{1.590639in}{2.050503in}}%
\pgfpathlineto{\pgfqpoint{1.570016in}{2.052759in}}%
\pgfpathlineto{\pgfqpoint{1.570807in}{2.059681in}}%
\pgfpathlineto{\pgfqpoint{1.557068in}{2.061233in}}%
\pgfpathlineto{\pgfqpoint{1.557865in}{2.068157in}}%
\pgfpathlineto{\pgfqpoint{1.560678in}{2.074073in}}%
\pgfpathlineto{\pgfqpoint{1.562458in}{2.089780in}}%
\pgfpathlineto{\pgfqpoint{1.569948in}{2.088334in}}%
\pgfpathlineto{\pgfqpoint{1.582871in}{2.091842in}}%
\pgfpathlineto{\pgfqpoint{1.590181in}{2.089744in}}%
\pgfpathlineto{\pgfqpoint{1.601867in}{2.089959in}}%
\pgfpathlineto{\pgfqpoint{1.607083in}{2.083595in}}%
\pgfpathlineto{\pgfqpoint{1.619223in}{2.081466in}}%
\pgfpathlineto{\pgfqpoint{1.623124in}{2.077735in}}%
\pgfpathlineto{\pgfqpoint{1.632154in}{2.074810in}}%
\pgfpathlineto{\pgfqpoint{1.643336in}{2.085862in}}%
\pgfpathlineto{\pgfqpoint{1.647283in}{2.084906in}}%
\pgfpathlineto{\pgfqpoint{1.648734in}{2.078432in}}%
\pgfpathlineto{\pgfqpoint{1.653010in}{2.075697in}}%
\pgfpathlineto{\pgfqpoint{1.664466in}{2.078762in}}%
\pgfpathlineto{\pgfqpoint{1.669040in}{2.084553in}}%
\pgfpathlineto{\pgfqpoint{1.686894in}{2.083116in}}%
\pgfpathlineto{\pgfqpoint{1.690580in}{2.101766in}}%
\pgfpathlineto{\pgfqpoint{1.687537in}{2.102006in}}%
\pgfpathlineto{\pgfqpoint{1.688717in}{2.115878in}}%
\pgfpathlineto{\pgfqpoint{1.723190in}{2.112972in}}%
\pgfpathlineto{\pgfqpoint{1.737002in}{2.111989in}}%
\pgfpathlineto{\pgfqpoint{1.735922in}{2.098000in}}%
\pgfpathlineto{\pgfqpoint{1.738640in}{2.097811in}}%
\pgfpathlineto{\pgfqpoint{1.736501in}{2.069931in}}%
\pgfpathlineto{\pgfqpoint{1.739211in}{2.069716in}}%
\pgfpathlineto{\pgfqpoint{1.738152in}{2.055718in}}%
\pgfpathlineto{\pgfqpoint{1.710634in}{2.057876in}}%
\pgfpathlineto{\pgfqpoint{1.710047in}{2.050726in}}%
\pgfpathlineto{\pgfqpoint{1.700751in}{2.053267in}}%
\pgfpathlineto{\pgfqpoint{1.695708in}{2.045165in}}%
\pgfpathlineto{\pgfqpoint{1.671188in}{2.047334in}}%
\pgfpathlineto{\pgfqpoint{1.668766in}{2.019808in}}%
\pgfpathlineto{\pgfqpoint{1.638167in}{2.022688in}}%
\pgfpathclose%
\pgfusepath{fill}%
\end{pgfscope}%
\begin{pgfscope}%
\pgfpathrectangle{\pgfqpoint{0.100000in}{0.100000in}}{\pgfqpoint{3.608454in}{2.310000in}}%
\pgfusepath{clip}%
\pgfsetbuttcap%
\pgfsetmiterjoin%
\definecolor{currentfill}{rgb}{0.000000,0.690196,0.654902}%
\pgfsetfillcolor{currentfill}%
\pgfsetlinewidth{0.000000pt}%
\definecolor{currentstroke}{rgb}{0.000000,0.000000,0.000000}%
\pgfsetstrokecolor{currentstroke}%
\pgfsetstrokeopacity{0.000000}%
\pgfsetdash{}{0pt}%
\pgfpathmoveto{\pgfqpoint{3.535136in}{1.728569in}}%
\pgfpathlineto{\pgfqpoint{3.546887in}{1.721036in}}%
\pgfpathlineto{\pgfqpoint{3.532181in}{1.716409in}}%
\pgfpathlineto{\pgfqpoint{3.530762in}{1.723556in}}%
\pgfpathlineto{\pgfqpoint{3.535136in}{1.728569in}}%
\pgfpathclose%
\pgfusepath{fill}%
\end{pgfscope}%
\begin{pgfscope}%
\pgfpathrectangle{\pgfqpoint{0.100000in}{0.100000in}}{\pgfqpoint{3.608454in}{2.310000in}}%
\pgfusepath{clip}%
\pgfsetbuttcap%
\pgfsetmiterjoin%
\definecolor{currentfill}{rgb}{0.000000,0.494118,0.752941}%
\pgfsetfillcolor{currentfill}%
\pgfsetlinewidth{0.000000pt}%
\definecolor{currentstroke}{rgb}{0.000000,0.000000,0.000000}%
\pgfsetstrokecolor{currentstroke}%
\pgfsetstrokeopacity{0.000000}%
\pgfsetdash{}{0pt}%
\pgfpathmoveto{\pgfqpoint{3.153524in}{1.337961in}}%
\pgfpathlineto{\pgfqpoint{3.149422in}{1.332537in}}%
\pgfpathlineto{\pgfqpoint{3.149648in}{1.328164in}}%
\pgfpathlineto{\pgfqpoint{3.140579in}{1.302693in}}%
\pgfpathlineto{\pgfqpoint{3.131753in}{1.305762in}}%
\pgfpathlineto{\pgfqpoint{3.125802in}{1.307188in}}%
\pgfpathlineto{\pgfqpoint{3.115324in}{1.315206in}}%
\pgfpathlineto{\pgfqpoint{3.116356in}{1.322582in}}%
\pgfpathlineto{\pgfqpoint{3.121341in}{1.327517in}}%
\pgfpathlineto{\pgfqpoint{3.120128in}{1.335691in}}%
\pgfpathlineto{\pgfqpoint{3.107539in}{1.353570in}}%
\pgfpathlineto{\pgfqpoint{3.110708in}{1.357752in}}%
\pgfpathlineto{\pgfqpoint{3.110807in}{1.366936in}}%
\pgfpathlineto{\pgfqpoint{3.115799in}{1.372179in}}%
\pgfpathlineto{\pgfqpoint{3.123623in}{1.385333in}}%
\pgfpathlineto{\pgfqpoint{3.125203in}{1.394430in}}%
\pgfpathlineto{\pgfqpoint{3.129858in}{1.404228in}}%
\pgfpathlineto{\pgfqpoint{3.137714in}{1.398047in}}%
\pgfpathlineto{\pgfqpoint{3.141594in}{1.398020in}}%
\pgfpathlineto{\pgfqpoint{3.153282in}{1.414127in}}%
\pgfpathlineto{\pgfqpoint{3.158541in}{1.406177in}}%
\pgfpathlineto{\pgfqpoint{3.169602in}{1.394541in}}%
\pgfpathlineto{\pgfqpoint{3.175771in}{1.395430in}}%
\pgfpathlineto{\pgfqpoint{3.161085in}{1.369292in}}%
\pgfpathlineto{\pgfqpoint{3.168135in}{1.369114in}}%
\pgfpathlineto{\pgfqpoint{3.178627in}{1.363629in}}%
\pgfpathlineto{\pgfqpoint{3.176043in}{1.340697in}}%
\pgfpathlineto{\pgfqpoint{3.169983in}{1.342070in}}%
\pgfpathlineto{\pgfqpoint{3.165049in}{1.347998in}}%
\pgfpathlineto{\pgfqpoint{3.157131in}{1.351262in}}%
\pgfpathlineto{\pgfqpoint{3.153524in}{1.337961in}}%
\pgfpathclose%
\pgfusepath{fill}%
\end{pgfscope}%
\begin{pgfscope}%
\pgfpathrectangle{\pgfqpoint{0.100000in}{0.100000in}}{\pgfqpoint{3.608454in}{2.310000in}}%
\pgfusepath{clip}%
\pgfsetbuttcap%
\pgfsetmiterjoin%
\definecolor{currentfill}{rgb}{0.000000,0.427451,0.786275}%
\pgfsetfillcolor{currentfill}%
\pgfsetlinewidth{0.000000pt}%
\definecolor{currentstroke}{rgb}{0.000000,0.000000,0.000000}%
\pgfsetstrokecolor{currentstroke}%
\pgfsetstrokeopacity{0.000000}%
\pgfsetdash{}{0pt}%
\pgfpathmoveto{\pgfqpoint{1.713023in}{1.050837in}}%
\pgfpathlineto{\pgfqpoint{1.710432in}{1.016551in}}%
\pgfpathlineto{\pgfqpoint{1.685686in}{1.018201in}}%
\pgfpathlineto{\pgfqpoint{1.651469in}{1.020799in}}%
\pgfpathlineto{\pgfqpoint{1.617270in}{1.023437in}}%
\pgfpathlineto{\pgfqpoint{1.583688in}{1.026354in}}%
\pgfpathlineto{\pgfqpoint{1.585141in}{1.042675in}}%
\pgfpathlineto{\pgfqpoint{1.589893in}{1.095406in}}%
\pgfpathlineto{\pgfqpoint{1.646284in}{1.091077in}}%
\pgfpathlineto{\pgfqpoint{1.680896in}{1.087866in}}%
\pgfpathlineto{\pgfqpoint{1.678367in}{1.053341in}}%
\pgfpathlineto{\pgfqpoint{1.713023in}{1.050837in}}%
\pgfpathclose%
\pgfusepath{fill}%
\end{pgfscope}%
\begin{pgfscope}%
\pgfpathrectangle{\pgfqpoint{0.100000in}{0.100000in}}{\pgfqpoint{3.608454in}{2.310000in}}%
\pgfusepath{clip}%
\pgfsetbuttcap%
\pgfsetmiterjoin%
\definecolor{currentfill}{rgb}{0.000000,0.470588,0.764706}%
\pgfsetfillcolor{currentfill}%
\pgfsetlinewidth{0.000000pt}%
\definecolor{currentstroke}{rgb}{0.000000,0.000000,0.000000}%
\pgfsetstrokecolor{currentstroke}%
\pgfsetstrokeopacity{0.000000}%
\pgfsetdash{}{0pt}%
\pgfpathmoveto{\pgfqpoint{2.281822in}{0.951221in}}%
\pgfpathlineto{\pgfqpoint{2.271324in}{0.957636in}}%
\pgfpathlineto{\pgfqpoint{2.259184in}{0.957741in}}%
\pgfpathlineto{\pgfqpoint{2.259281in}{0.978738in}}%
\pgfpathlineto{\pgfqpoint{2.258094in}{0.983370in}}%
\pgfpathlineto{\pgfqpoint{2.250959in}{0.985619in}}%
\pgfpathlineto{\pgfqpoint{2.249833in}{0.992475in}}%
\pgfpathlineto{\pgfqpoint{2.245351in}{0.995915in}}%
\pgfpathlineto{\pgfqpoint{2.238537in}{0.995926in}}%
\pgfpathlineto{\pgfqpoint{2.238992in}{1.006643in}}%
\pgfpathlineto{\pgfqpoint{2.218257in}{1.006289in}}%
\pgfpathlineto{\pgfqpoint{2.216940in}{1.014382in}}%
\pgfpathlineto{\pgfqpoint{2.226237in}{1.022520in}}%
\pgfpathlineto{\pgfqpoint{2.241035in}{1.037964in}}%
\pgfpathlineto{\pgfqpoint{2.245679in}{1.038542in}}%
\pgfpathlineto{\pgfqpoint{2.245419in}{1.061527in}}%
\pgfpathlineto{\pgfqpoint{2.269359in}{1.061552in}}%
\pgfpathlineto{\pgfqpoint{2.269431in}{1.054652in}}%
\pgfpathlineto{\pgfqpoint{2.293330in}{1.054914in}}%
\pgfpathlineto{\pgfqpoint{2.293540in}{1.031438in}}%
\pgfpathlineto{\pgfqpoint{2.300760in}{1.031780in}}%
\pgfpathlineto{\pgfqpoint{2.309489in}{1.028555in}}%
\pgfpathlineto{\pgfqpoint{2.314200in}{1.029274in}}%
\pgfpathlineto{\pgfqpoint{2.314414in}{1.020654in}}%
\pgfpathlineto{\pgfqpoint{2.321574in}{1.020760in}}%
\pgfpathlineto{\pgfqpoint{2.321823in}{1.001132in}}%
\pgfpathlineto{\pgfqpoint{2.326016in}{0.997543in}}%
\pgfpathlineto{\pgfqpoint{2.326278in}{0.991006in}}%
\pgfpathlineto{\pgfqpoint{2.322083in}{0.986111in}}%
\pgfpathlineto{\pgfqpoint{2.289234in}{0.985811in}}%
\pgfpathlineto{\pgfqpoint{2.289533in}{0.964998in}}%
\pgfpathlineto{\pgfqpoint{2.288597in}{0.951635in}}%
\pgfpathlineto{\pgfqpoint{2.281822in}{0.951221in}}%
\pgfpathclose%
\pgfusepath{fill}%
\end{pgfscope}%
\begin{pgfscope}%
\pgfpathrectangle{\pgfqpoint{0.100000in}{0.100000in}}{\pgfqpoint{3.608454in}{2.310000in}}%
\pgfusepath{clip}%
\pgfsetbuttcap%
\pgfsetmiterjoin%
\definecolor{currentfill}{rgb}{0.000000,0.611765,0.694118}%
\pgfsetfillcolor{currentfill}%
\pgfsetlinewidth{0.000000pt}%
\definecolor{currentstroke}{rgb}{0.000000,0.000000,0.000000}%
\pgfsetstrokecolor{currentstroke}%
\pgfsetstrokeopacity{0.000000}%
\pgfsetdash{}{0pt}%
\pgfpathmoveto{\pgfqpoint{1.553831in}{1.907730in}}%
\pgfpathlineto{\pgfqpoint{1.549754in}{1.873361in}}%
\pgfpathlineto{\pgfqpoint{1.546917in}{1.873677in}}%
\pgfpathlineto{\pgfqpoint{1.543825in}{1.845887in}}%
\pgfpathlineto{\pgfqpoint{1.494987in}{1.851589in}}%
\pgfpathlineto{\pgfqpoint{1.488701in}{1.851762in}}%
\pgfpathlineto{\pgfqpoint{1.475384in}{1.853420in}}%
\pgfpathlineto{\pgfqpoint{1.477206in}{1.868092in}}%
\pgfpathlineto{\pgfqpoint{1.449171in}{1.871656in}}%
\pgfpathlineto{\pgfqpoint{1.453065in}{1.884941in}}%
\pgfpathlineto{\pgfqpoint{1.456350in}{1.910291in}}%
\pgfpathlineto{\pgfqpoint{1.446358in}{1.912489in}}%
\pgfpathlineto{\pgfqpoint{1.446859in}{1.927356in}}%
\pgfpathlineto{\pgfqpoint{1.434747in}{1.935937in}}%
\pgfpathlineto{\pgfqpoint{1.421111in}{1.937794in}}%
\pgfpathlineto{\pgfqpoint{1.422053in}{1.944628in}}%
\pgfpathlineto{\pgfqpoint{1.417308in}{1.945271in}}%
\pgfpathlineto{\pgfqpoint{1.421242in}{1.955525in}}%
\pgfpathlineto{\pgfqpoint{1.414681in}{1.966931in}}%
\pgfpathlineto{\pgfqpoint{1.410750in}{1.980307in}}%
\pgfpathlineto{\pgfqpoint{1.406757in}{1.983167in}}%
\pgfpathlineto{\pgfqpoint{1.409028in}{1.997695in}}%
\pgfpathlineto{\pgfqpoint{1.407997in}{2.003940in}}%
\pgfpathlineto{\pgfqpoint{1.405583in}{2.011993in}}%
\pgfpathlineto{\pgfqpoint{1.469044in}{2.003996in}}%
\pgfpathlineto{\pgfqpoint{1.468884in}{2.002862in}}%
\pgfpathlineto{\pgfqpoint{1.503208in}{1.998589in}}%
\pgfpathlineto{\pgfqpoint{1.501201in}{1.997658in}}%
\pgfpathlineto{\pgfqpoint{1.497843in}{1.970127in}}%
\pgfpathlineto{\pgfqpoint{1.495943in}{1.970351in}}%
\pgfpathlineto{\pgfqpoint{1.492673in}{1.942949in}}%
\pgfpathlineto{\pgfqpoint{1.490580in}{1.943193in}}%
\pgfpathlineto{\pgfqpoint{1.487149in}{1.915660in}}%
\pgfpathlineto{\pgfqpoint{1.553831in}{1.907730in}}%
\pgfpathclose%
\pgfusepath{fill}%
\end{pgfscope}%
\begin{pgfscope}%
\pgfpathrectangle{\pgfqpoint{0.100000in}{0.100000in}}{\pgfqpoint{3.608454in}{2.310000in}}%
\pgfusepath{clip}%
\pgfsetbuttcap%
\pgfsetmiterjoin%
\definecolor{currentfill}{rgb}{0.000000,0.776471,0.611765}%
\pgfsetfillcolor{currentfill}%
\pgfsetlinewidth{0.000000pt}%
\definecolor{currentstroke}{rgb}{0.000000,0.000000,0.000000}%
\pgfsetstrokecolor{currentstroke}%
\pgfsetstrokeopacity{0.000000}%
\pgfsetdash{}{0pt}%
\pgfpathmoveto{\pgfqpoint{2.978722in}{0.804662in}}%
\pgfpathlineto{\pgfqpoint{2.970476in}{0.810318in}}%
\pgfpathlineto{\pgfqpoint{2.953500in}{0.817959in}}%
\pgfpathlineto{\pgfqpoint{2.929182in}{0.813904in}}%
\pgfpathlineto{\pgfqpoint{2.926243in}{0.822499in}}%
\pgfpathlineto{\pgfqpoint{2.917347in}{0.821092in}}%
\pgfpathlineto{\pgfqpoint{2.904439in}{0.824809in}}%
\pgfpathlineto{\pgfqpoint{2.900223in}{0.831517in}}%
\pgfpathlineto{\pgfqpoint{2.895939in}{0.833978in}}%
\pgfpathlineto{\pgfqpoint{2.894217in}{0.842012in}}%
\pgfpathlineto{\pgfqpoint{2.891100in}{0.843787in}}%
\pgfpathlineto{\pgfqpoint{2.893436in}{0.849951in}}%
\pgfpathlineto{\pgfqpoint{2.887187in}{0.854824in}}%
\pgfpathlineto{\pgfqpoint{2.890867in}{0.862703in}}%
\pgfpathlineto{\pgfqpoint{2.898491in}{0.872568in}}%
\pgfpathlineto{\pgfqpoint{2.901077in}{0.870542in}}%
\pgfpathlineto{\pgfqpoint{2.913824in}{0.850086in}}%
\pgfpathlineto{\pgfqpoint{2.922187in}{0.854988in}}%
\pgfpathlineto{\pgfqpoint{2.930056in}{0.865279in}}%
\pgfpathlineto{\pgfqpoint{2.928863in}{0.867621in}}%
\pgfpathlineto{\pgfqpoint{2.932829in}{0.881990in}}%
\pgfpathlineto{\pgfqpoint{2.943431in}{0.882150in}}%
\pgfpathlineto{\pgfqpoint{2.951258in}{0.878733in}}%
\pgfpathlineto{\pgfqpoint{2.950142in}{0.872561in}}%
\pgfpathlineto{\pgfqpoint{2.954746in}{0.867311in}}%
\pgfpathlineto{\pgfqpoint{2.962520in}{0.870538in}}%
\pgfpathlineto{\pgfqpoint{2.967645in}{0.858971in}}%
\pgfpathlineto{\pgfqpoint{2.967579in}{0.838638in}}%
\pgfpathlineto{\pgfqpoint{2.975754in}{0.837437in}}%
\pgfpathlineto{\pgfqpoint{2.979065in}{0.834250in}}%
\pgfpathlineto{\pgfqpoint{2.975294in}{0.828495in}}%
\pgfpathlineto{\pgfqpoint{2.978722in}{0.804662in}}%
\pgfpathclose%
\pgfusepath{fill}%
\end{pgfscope}%
\begin{pgfscope}%
\pgfpathrectangle{\pgfqpoint{0.100000in}{0.100000in}}{\pgfqpoint{3.608454in}{2.310000in}}%
\pgfusepath{clip}%
\pgfsetbuttcap%
\pgfsetmiterjoin%
\definecolor{currentfill}{rgb}{0.000000,0.458824,0.770588}%
\pgfsetfillcolor{currentfill}%
\pgfsetlinewidth{0.000000pt}%
\definecolor{currentstroke}{rgb}{0.000000,0.000000,0.000000}%
\pgfsetstrokecolor{currentstroke}%
\pgfsetstrokeopacity{0.000000}%
\pgfsetdash{}{0pt}%
\pgfpathmoveto{\pgfqpoint{3.269493in}{1.294748in}}%
\pgfpathlineto{\pgfqpoint{3.266535in}{1.293082in}}%
\pgfpathlineto{\pgfqpoint{3.248223in}{1.267423in}}%
\pgfpathlineto{\pgfqpoint{3.248646in}{1.259422in}}%
\pgfpathlineto{\pgfqpoint{3.221614in}{1.254572in}}%
\pgfpathlineto{\pgfqpoint{3.214106in}{1.265774in}}%
\pgfpathlineto{\pgfqpoint{3.240057in}{1.290021in}}%
\pgfpathlineto{\pgfqpoint{3.236491in}{1.297792in}}%
\pgfpathlineto{\pgfqpoint{3.224896in}{1.300661in}}%
\pgfpathlineto{\pgfqpoint{3.234147in}{1.313361in}}%
\pgfpathlineto{\pgfqpoint{3.240056in}{1.312499in}}%
\pgfpathlineto{\pgfqpoint{3.236059in}{1.328225in}}%
\pgfpathlineto{\pgfqpoint{3.249638in}{1.333200in}}%
\pgfpathlineto{\pgfqpoint{3.248164in}{1.344820in}}%
\pgfpathlineto{\pgfqpoint{3.240232in}{1.353321in}}%
\pgfpathlineto{\pgfqpoint{3.244523in}{1.358215in}}%
\pgfpathlineto{\pgfqpoint{3.250135in}{1.356031in}}%
\pgfpathlineto{\pgfqpoint{3.255578in}{1.348500in}}%
\pgfpathlineto{\pgfqpoint{3.262571in}{1.349107in}}%
\pgfpathlineto{\pgfqpoint{3.267940in}{1.341269in}}%
\pgfpathlineto{\pgfqpoint{3.273671in}{1.340685in}}%
\pgfpathlineto{\pgfqpoint{3.276753in}{1.332527in}}%
\pgfpathlineto{\pgfqpoint{3.273954in}{1.327806in}}%
\pgfpathlineto{\pgfqpoint{3.268757in}{1.331100in}}%
\pgfpathlineto{\pgfqpoint{3.270994in}{1.316465in}}%
\pgfpathlineto{\pgfqpoint{3.280061in}{1.309564in}}%
\pgfpathlineto{\pgfqpoint{3.278895in}{1.302449in}}%
\pgfpathlineto{\pgfqpoint{3.274672in}{1.301154in}}%
\pgfpathlineto{\pgfqpoint{3.269493in}{1.294748in}}%
\pgfpathclose%
\pgfusepath{fill}%
\end{pgfscope}%
\begin{pgfscope}%
\pgfpathrectangle{\pgfqpoint{0.100000in}{0.100000in}}{\pgfqpoint{3.608454in}{2.310000in}}%
\pgfusepath{clip}%
\pgfsetbuttcap%
\pgfsetmiterjoin%
\definecolor{currentfill}{rgb}{0.000000,0.447059,0.776471}%
\pgfsetfillcolor{currentfill}%
\pgfsetlinewidth{0.000000pt}%
\definecolor{currentstroke}{rgb}{0.000000,0.000000,0.000000}%
\pgfsetstrokecolor{currentstroke}%
\pgfsetstrokeopacity{0.000000}%
\pgfsetdash{}{0pt}%
\pgfpathmoveto{\pgfqpoint{2.844033in}{0.665474in}}%
\pgfpathlineto{\pgfqpoint{2.844132in}{0.660367in}}%
\pgfpathlineto{\pgfqpoint{2.835832in}{0.661752in}}%
\pgfpathlineto{\pgfqpoint{2.827136in}{0.657037in}}%
\pgfpathlineto{\pgfqpoint{2.809853in}{0.644061in}}%
\pgfpathlineto{\pgfqpoint{2.799185in}{0.640400in}}%
\pgfpathlineto{\pgfqpoint{2.793287in}{0.640173in}}%
\pgfpathlineto{\pgfqpoint{2.783205in}{0.636006in}}%
\pgfpathlineto{\pgfqpoint{2.779514in}{0.637469in}}%
\pgfpathlineto{\pgfqpoint{2.779023in}{0.645884in}}%
\pgfpathlineto{\pgfqpoint{2.769443in}{0.656196in}}%
\pgfpathlineto{\pgfqpoint{2.764564in}{0.656808in}}%
\pgfpathlineto{\pgfqpoint{2.757459in}{0.662818in}}%
\pgfpathlineto{\pgfqpoint{2.742727in}{0.670681in}}%
\pgfpathlineto{\pgfqpoint{2.727447in}{0.677658in}}%
\pgfpathlineto{\pgfqpoint{2.726751in}{0.687235in}}%
\pgfpathlineto{\pgfqpoint{2.735873in}{0.692265in}}%
\pgfpathlineto{\pgfqpoint{2.760979in}{0.694551in}}%
\pgfpathlineto{\pgfqpoint{2.760006in}{0.704933in}}%
\pgfpathlineto{\pgfqpoint{2.766984in}{0.705619in}}%
\pgfpathlineto{\pgfqpoint{2.769691in}{0.676534in}}%
\pgfpathlineto{\pgfqpoint{2.787376in}{0.676778in}}%
\pgfpathlineto{\pgfqpoint{2.788514in}{0.665867in}}%
\pgfpathlineto{\pgfqpoint{2.797575in}{0.664470in}}%
\pgfpathlineto{\pgfqpoint{2.829477in}{0.667841in}}%
\pgfpathlineto{\pgfqpoint{2.844033in}{0.665474in}}%
\pgfpathclose%
\pgfusepath{fill}%
\end{pgfscope}%
\begin{pgfscope}%
\pgfpathrectangle{\pgfqpoint{0.100000in}{0.100000in}}{\pgfqpoint{3.608454in}{2.310000in}}%
\pgfusepath{clip}%
\pgfsetbuttcap%
\pgfsetmiterjoin%
\definecolor{currentfill}{rgb}{0.000000,0.737255,0.631373}%
\pgfsetfillcolor{currentfill}%
\pgfsetlinewidth{0.000000pt}%
\definecolor{currentstroke}{rgb}{0.000000,0.000000,0.000000}%
\pgfsetstrokecolor{currentstroke}%
\pgfsetstrokeopacity{0.000000}%
\pgfsetdash{}{0pt}%
\pgfpathmoveto{\pgfqpoint{3.074241in}{1.252094in}}%
\pgfpathlineto{\pgfqpoint{3.073773in}{1.226802in}}%
\pgfpathlineto{\pgfqpoint{3.054182in}{1.223554in}}%
\pgfpathlineto{\pgfqpoint{3.037358in}{1.220901in}}%
\pgfpathlineto{\pgfqpoint{3.017299in}{1.218774in}}%
\pgfpathlineto{\pgfqpoint{3.015913in}{1.223529in}}%
\pgfpathlineto{\pgfqpoint{3.025017in}{1.229246in}}%
\pgfpathlineto{\pgfqpoint{3.024683in}{1.232048in}}%
\pgfpathlineto{\pgfqpoint{3.032148in}{1.244073in}}%
\pgfpathlineto{\pgfqpoint{3.036908in}{1.247132in}}%
\pgfpathlineto{\pgfqpoint{3.067580in}{1.245522in}}%
\pgfpathlineto{\pgfqpoint{3.074241in}{1.252094in}}%
\pgfpathclose%
\pgfusepath{fill}%
\end{pgfscope}%
\begin{pgfscope}%
\pgfpathrectangle{\pgfqpoint{0.100000in}{0.100000in}}{\pgfqpoint{3.608454in}{2.310000in}}%
\pgfusepath{clip}%
\pgfsetbuttcap%
\pgfsetmiterjoin%
\definecolor{currentfill}{rgb}{0.000000,0.721569,0.639216}%
\pgfsetfillcolor{currentfill}%
\pgfsetlinewidth{0.000000pt}%
\definecolor{currentstroke}{rgb}{0.000000,0.000000,0.000000}%
\pgfsetstrokecolor{currentstroke}%
\pgfsetstrokeopacity{0.000000}%
\pgfsetdash{}{0pt}%
\pgfpathmoveto{\pgfqpoint{1.523829in}{1.020634in}}%
\pgfpathlineto{\pgfqpoint{1.519945in}{0.979398in}}%
\pgfpathlineto{\pgfqpoint{1.506226in}{0.980812in}}%
\pgfpathlineto{\pgfqpoint{1.505497in}{0.973876in}}%
\pgfpathlineto{\pgfqpoint{1.464338in}{0.978551in}}%
\pgfpathlineto{\pgfqpoint{1.465068in}{0.985473in}}%
\pgfpathlineto{\pgfqpoint{1.458090in}{0.986273in}}%
\pgfpathlineto{\pgfqpoint{1.460404in}{1.006662in}}%
\pgfpathlineto{\pgfqpoint{1.432998in}{1.009699in}}%
\pgfpathlineto{\pgfqpoint{1.435358in}{1.030206in}}%
\pgfpathlineto{\pgfqpoint{1.436822in}{1.030070in}}%
\pgfpathlineto{\pgfqpoint{1.440664in}{1.064305in}}%
\pgfpathlineto{\pgfqpoint{1.442195in}{1.078025in}}%
\pgfpathlineto{\pgfqpoint{1.517207in}{1.069928in}}%
\pgfpathlineto{\pgfqpoint{1.516667in}{1.064131in}}%
\pgfpathlineto{\pgfqpoint{1.512139in}{1.021816in}}%
\pgfpathlineto{\pgfqpoint{1.523829in}{1.020634in}}%
\pgfpathclose%
\pgfusepath{fill}%
\end{pgfscope}%
\begin{pgfscope}%
\pgfpathrectangle{\pgfqpoint{0.100000in}{0.100000in}}{\pgfqpoint{3.608454in}{2.310000in}}%
\pgfusepath{clip}%
\pgfsetbuttcap%
\pgfsetmiterjoin%
\definecolor{currentfill}{rgb}{0.000000,0.352941,0.823529}%
\pgfsetfillcolor{currentfill}%
\pgfsetlinewidth{0.000000pt}%
\definecolor{currentstroke}{rgb}{0.000000,0.000000,0.000000}%
\pgfsetstrokecolor{currentstroke}%
\pgfsetstrokeopacity{0.000000}%
\pgfsetdash{}{0pt}%
\pgfpathmoveto{\pgfqpoint{3.530808in}{1.730215in}}%
\pgfpathlineto{\pgfqpoint{3.529206in}{1.730530in}}%
\pgfpathlineto{\pgfqpoint{3.531215in}{1.735382in}}%
\pgfpathlineto{\pgfqpoint{3.526862in}{1.745686in}}%
\pgfpathlineto{\pgfqpoint{3.518613in}{1.736187in}}%
\pgfpathlineto{\pgfqpoint{3.513375in}{1.745585in}}%
\pgfpathlineto{\pgfqpoint{3.499874in}{1.758602in}}%
\pgfpathlineto{\pgfqpoint{3.495434in}{1.768247in}}%
\pgfpathlineto{\pgfqpoint{3.481572in}{1.755490in}}%
\pgfpathlineto{\pgfqpoint{3.479909in}{1.757791in}}%
\pgfpathlineto{\pgfqpoint{3.456367in}{1.750710in}}%
\pgfpathlineto{\pgfqpoint{3.436777in}{1.747450in}}%
\pgfpathlineto{\pgfqpoint{3.434236in}{1.757543in}}%
\pgfpathlineto{\pgfqpoint{3.426466in}{1.757361in}}%
\pgfpathlineto{\pgfqpoint{3.427810in}{1.762986in}}%
\pgfpathlineto{\pgfqpoint{3.420462in}{1.768872in}}%
\pgfpathlineto{\pgfqpoint{3.418010in}{1.787350in}}%
\pgfpathlineto{\pgfqpoint{3.419015in}{1.794601in}}%
\pgfpathlineto{\pgfqpoint{3.414849in}{1.798333in}}%
\pgfpathlineto{\pgfqpoint{3.434992in}{1.802755in}}%
\pgfpathlineto{\pgfqpoint{3.471016in}{1.810988in}}%
\pgfpathlineto{\pgfqpoint{3.477550in}{1.822939in}}%
\pgfpathlineto{\pgfqpoint{3.487553in}{1.830826in}}%
\pgfpathlineto{\pgfqpoint{3.494171in}{1.831598in}}%
\pgfpathlineto{\pgfqpoint{3.501460in}{1.818237in}}%
\pgfpathlineto{\pgfqpoint{3.511382in}{1.816812in}}%
\pgfpathlineto{\pgfqpoint{3.507584in}{1.811929in}}%
\pgfpathlineto{\pgfqpoint{3.498548in}{1.805841in}}%
\pgfpathlineto{\pgfqpoint{3.500415in}{1.803471in}}%
\pgfpathlineto{\pgfqpoint{3.494052in}{1.793429in}}%
\pgfpathlineto{\pgfqpoint{3.496118in}{1.786298in}}%
\pgfpathlineto{\pgfqpoint{3.502227in}{1.787338in}}%
\pgfpathlineto{\pgfqpoint{3.513856in}{1.782235in}}%
\pgfpathlineto{\pgfqpoint{3.520466in}{1.774625in}}%
\pgfpathlineto{\pgfqpoint{3.519312in}{1.765836in}}%
\pgfpathlineto{\pgfqpoint{3.529828in}{1.763255in}}%
\pgfpathlineto{\pgfqpoint{3.531850in}{1.755199in}}%
\pgfpathlineto{\pgfqpoint{3.540059in}{1.751718in}}%
\pgfpathlineto{\pgfqpoint{3.547764in}{1.752941in}}%
\pgfpathlineto{\pgfqpoint{3.566780in}{1.763298in}}%
\pgfpathlineto{\pgfqpoint{3.569480in}{1.754165in}}%
\pgfpathlineto{\pgfqpoint{3.542220in}{1.740804in}}%
\pgfpathlineto{\pgfqpoint{3.540912in}{1.736515in}}%
\pgfpathlineto{\pgfqpoint{3.530808in}{1.730215in}}%
\pgfpathclose%
\pgfusepath{fill}%
\end{pgfscope}%
\begin{pgfscope}%
\pgfpathrectangle{\pgfqpoint{0.100000in}{0.100000in}}{\pgfqpoint{3.608454in}{2.310000in}}%
\pgfusepath{clip}%
\pgfsetbuttcap%
\pgfsetmiterjoin%
\definecolor{currentfill}{rgb}{0.000000,0.827451,0.586275}%
\pgfsetfillcolor{currentfill}%
\pgfsetlinewidth{0.000000pt}%
\definecolor{currentstroke}{rgb}{0.000000,0.000000,0.000000}%
\pgfsetstrokecolor{currentstroke}%
\pgfsetstrokeopacity{0.000000}%
\pgfsetdash{}{0pt}%
\pgfpathmoveto{\pgfqpoint{0.737385in}{0.650268in}}%
\pgfpathlineto{\pgfqpoint{0.738818in}{0.653776in}}%
\pgfpathlineto{\pgfqpoint{0.738884in}{0.657312in}}%
\pgfpathlineto{\pgfqpoint{0.737658in}{0.662474in}}%
\pgfpathlineto{\pgfqpoint{0.738418in}{0.665625in}}%
\pgfpathlineto{\pgfqpoint{0.736741in}{0.671963in}}%
\pgfpathlineto{\pgfqpoint{0.734805in}{0.674605in}}%
\pgfpathlineto{\pgfqpoint{0.729012in}{0.675764in}}%
\pgfpathlineto{\pgfqpoint{0.728310in}{0.678102in}}%
\pgfpathlineto{\pgfqpoint{0.728866in}{0.682342in}}%
\pgfpathlineto{\pgfqpoint{0.731069in}{0.684627in}}%
\pgfpathlineto{\pgfqpoint{0.733927in}{0.685526in}}%
\pgfpathlineto{\pgfqpoint{0.736427in}{0.687461in}}%
\pgfpathlineto{\pgfqpoint{0.737267in}{0.686485in}}%
\pgfpathlineto{\pgfqpoint{0.736221in}{0.683768in}}%
\pgfpathlineto{\pgfqpoint{0.738962in}{0.676360in}}%
\pgfpathlineto{\pgfqpoint{0.740767in}{0.675472in}}%
\pgfpathlineto{\pgfqpoint{0.746631in}{0.673961in}}%
\pgfpathlineto{\pgfqpoint{0.747962in}{0.671116in}}%
\pgfpathlineto{\pgfqpoint{0.747857in}{0.668032in}}%
\pgfpathlineto{\pgfqpoint{0.745828in}{0.665738in}}%
\pgfpathlineto{\pgfqpoint{0.747273in}{0.661310in}}%
\pgfpathlineto{\pgfqpoint{0.747476in}{0.658092in}}%
\pgfpathlineto{\pgfqpoint{0.749575in}{0.653902in}}%
\pgfpathlineto{\pgfqpoint{0.752259in}{0.652120in}}%
\pgfpathlineto{\pgfqpoint{0.753791in}{0.649042in}}%
\pgfpathlineto{\pgfqpoint{0.752833in}{0.647643in}}%
\pgfpathlineto{\pgfqpoint{0.749601in}{0.646731in}}%
\pgfpathlineto{\pgfqpoint{0.748012in}{0.649373in}}%
\pgfpathlineto{\pgfqpoint{0.745554in}{0.651542in}}%
\pgfpathlineto{\pgfqpoint{0.742391in}{0.652099in}}%
\pgfpathlineto{\pgfqpoint{0.737385in}{0.650268in}}%
\pgfpathclose%
\pgfusepath{fill}%
\end{pgfscope}%
\begin{pgfscope}%
\pgfpathrectangle{\pgfqpoint{0.100000in}{0.100000in}}{\pgfqpoint{3.608454in}{2.310000in}}%
\pgfusepath{clip}%
\pgfsetbuttcap%
\pgfsetmiterjoin%
\definecolor{currentfill}{rgb}{0.000000,0.827451,0.586275}%
\pgfsetfillcolor{currentfill}%
\pgfsetlinewidth{0.000000pt}%
\definecolor{currentstroke}{rgb}{0.000000,0.000000,0.000000}%
\pgfsetstrokecolor{currentstroke}%
\pgfsetstrokeopacity{0.000000}%
\pgfsetdash{}{0pt}%
\pgfpathmoveto{\pgfqpoint{0.863943in}{0.679699in}}%
\pgfpathlineto{\pgfqpoint{0.862419in}{0.676643in}}%
\pgfpathlineto{\pgfqpoint{0.855048in}{0.668896in}}%
\pgfpathlineto{\pgfqpoint{0.855680in}{0.668231in}}%
\pgfpathlineto{\pgfqpoint{0.848280in}{0.660517in}}%
\pgfpathlineto{\pgfqpoint{0.850812in}{0.658129in}}%
\pgfpathlineto{\pgfqpoint{0.847118in}{0.654238in}}%
\pgfpathlineto{\pgfqpoint{0.849066in}{0.652382in}}%
\pgfpathlineto{\pgfqpoint{0.845457in}{0.648542in}}%
\pgfpathlineto{\pgfqpoint{0.853614in}{0.640940in}}%
\pgfpathlineto{\pgfqpoint{0.870840in}{0.625356in}}%
\pgfpathlineto{\pgfqpoint{0.882807in}{0.614773in}}%
\pgfpathlineto{\pgfqpoint{0.884580in}{0.616759in}}%
\pgfpathlineto{\pgfqpoint{0.886576in}{0.615036in}}%
\pgfpathlineto{\pgfqpoint{0.888579in}{0.613334in}}%
\pgfpathlineto{\pgfqpoint{0.883388in}{0.607238in}}%
\pgfpathlineto{\pgfqpoint{0.881527in}{0.608877in}}%
\pgfpathlineto{\pgfqpoint{0.874796in}{0.601023in}}%
\pgfpathlineto{\pgfqpoint{0.872825in}{0.602697in}}%
\pgfpathlineto{\pgfqpoint{0.869420in}{0.598690in}}%
\pgfpathlineto{\pgfqpoint{0.867382in}{0.600383in}}%
\pgfpathlineto{\pgfqpoint{0.853750in}{0.584249in}}%
\pgfpathlineto{\pgfqpoint{0.856024in}{0.582334in}}%
\pgfpathlineto{\pgfqpoint{0.850691in}{0.576622in}}%
\pgfpathlineto{\pgfqpoint{0.848929in}{0.578173in}}%
\pgfpathlineto{\pgfqpoint{0.847165in}{0.576109in}}%
\pgfpathlineto{\pgfqpoint{0.845387in}{0.577604in}}%
\pgfpathlineto{\pgfqpoint{0.843712in}{0.575563in}}%
\pgfpathlineto{\pgfqpoint{0.839724in}{0.579022in}}%
\pgfpathlineto{\pgfqpoint{0.838193in}{0.577309in}}%
\pgfpathlineto{\pgfqpoint{0.836205in}{0.579038in}}%
\pgfpathlineto{\pgfqpoint{0.831146in}{0.572881in}}%
\pgfpathlineto{\pgfqpoint{0.829339in}{0.574413in}}%
\pgfpathlineto{\pgfqpoint{0.824219in}{0.568707in}}%
\pgfpathlineto{\pgfqpoint{0.812726in}{0.579011in}}%
\pgfpathlineto{\pgfqpoint{0.814280in}{0.580959in}}%
\pgfpathlineto{\pgfqpoint{0.812542in}{0.582561in}}%
\pgfpathlineto{\pgfqpoint{0.814282in}{0.584612in}}%
\pgfpathlineto{\pgfqpoint{0.812396in}{0.586335in}}%
\pgfpathlineto{\pgfqpoint{0.814106in}{0.588275in}}%
\pgfpathlineto{\pgfqpoint{0.812082in}{0.590252in}}%
\pgfpathlineto{\pgfqpoint{0.818724in}{0.592259in}}%
\pgfpathlineto{\pgfqpoint{0.818875in}{0.594792in}}%
\pgfpathlineto{\pgfqpoint{0.816853in}{0.596475in}}%
\pgfpathlineto{\pgfqpoint{0.818540in}{0.598214in}}%
\pgfpathlineto{\pgfqpoint{0.821571in}{0.596416in}}%
\pgfpathlineto{\pgfqpoint{0.820998in}{0.592054in}}%
\pgfpathlineto{\pgfqpoint{0.823073in}{0.586861in}}%
\pgfpathlineto{\pgfqpoint{0.829242in}{0.582849in}}%
\pgfpathlineto{\pgfqpoint{0.831926in}{0.582649in}}%
\pgfpathlineto{\pgfqpoint{0.836286in}{0.583592in}}%
\pgfpathlineto{\pgfqpoint{0.839848in}{0.584937in}}%
\pgfpathlineto{\pgfqpoint{0.841321in}{0.586303in}}%
\pgfpathlineto{\pgfqpoint{0.844401in}{0.593367in}}%
\pgfpathlineto{\pgfqpoint{0.847330in}{0.597336in}}%
\pgfpathlineto{\pgfqpoint{0.848129in}{0.603625in}}%
\pgfpathlineto{\pgfqpoint{0.845939in}{0.606018in}}%
\pgfpathlineto{\pgfqpoint{0.848681in}{0.607604in}}%
\pgfpathlineto{\pgfqpoint{0.851564in}{0.603948in}}%
\pgfpathlineto{\pgfqpoint{0.856618in}{0.604295in}}%
\pgfpathlineto{\pgfqpoint{0.858912in}{0.606588in}}%
\pgfpathlineto{\pgfqpoint{0.860360in}{0.609378in}}%
\pgfpathlineto{\pgfqpoint{0.859005in}{0.614759in}}%
\pgfpathlineto{\pgfqpoint{0.854769in}{0.613300in}}%
\pgfpathlineto{\pgfqpoint{0.852074in}{0.613498in}}%
\pgfpathlineto{\pgfqpoint{0.851004in}{0.616189in}}%
\pgfpathlineto{\pgfqpoint{0.847338in}{0.616442in}}%
\pgfpathlineto{\pgfqpoint{0.843639in}{0.617453in}}%
\pgfpathlineto{\pgfqpoint{0.840644in}{0.617353in}}%
\pgfpathlineto{\pgfqpoint{0.831603in}{0.615210in}}%
\pgfpathlineto{\pgfqpoint{0.833848in}{0.619214in}}%
\pgfpathlineto{\pgfqpoint{0.834393in}{0.625065in}}%
\pgfpathlineto{\pgfqpoint{0.831455in}{0.625440in}}%
\pgfpathlineto{\pgfqpoint{0.831880in}{0.622625in}}%
\pgfpathlineto{\pgfqpoint{0.829678in}{0.621808in}}%
\pgfpathlineto{\pgfqpoint{0.829726in}{0.624782in}}%
\pgfpathlineto{\pgfqpoint{0.828624in}{0.627896in}}%
\pgfpathlineto{\pgfqpoint{0.822394in}{0.633936in}}%
\pgfpathlineto{\pgfqpoint{0.815333in}{0.636154in}}%
\pgfpathlineto{\pgfqpoint{0.813484in}{0.637133in}}%
\pgfpathlineto{\pgfqpoint{0.805786in}{0.649884in}}%
\pgfpathlineto{\pgfqpoint{0.804377in}{0.652802in}}%
\pgfpathlineto{\pgfqpoint{0.804337in}{0.655328in}}%
\pgfpathlineto{\pgfqpoint{0.806949in}{0.658877in}}%
\pgfpathlineto{\pgfqpoint{0.809295in}{0.660793in}}%
\pgfpathlineto{\pgfqpoint{0.809700in}{0.668848in}}%
\pgfpathlineto{\pgfqpoint{0.813648in}{0.667058in}}%
\pgfpathlineto{\pgfqpoint{0.818571in}{0.669230in}}%
\pgfpathlineto{\pgfqpoint{0.813780in}{0.677479in}}%
\pgfpathlineto{\pgfqpoint{0.811754in}{0.680104in}}%
\pgfpathlineto{\pgfqpoint{0.810600in}{0.686441in}}%
\pgfpathlineto{\pgfqpoint{0.809362in}{0.690533in}}%
\pgfpathlineto{\pgfqpoint{0.811858in}{0.693024in}}%
\pgfpathlineto{\pgfqpoint{0.822834in}{0.689897in}}%
\pgfpathlineto{\pgfqpoint{0.823528in}{0.687901in}}%
\pgfpathlineto{\pgfqpoint{0.827552in}{0.687211in}}%
\pgfpathlineto{\pgfqpoint{0.827406in}{0.689342in}}%
\pgfpathlineto{\pgfqpoint{0.838119in}{0.687842in}}%
\pgfpathlineto{\pgfqpoint{0.840746in}{0.682200in}}%
\pgfpathlineto{\pgfqpoint{0.842608in}{0.680338in}}%
\pgfpathlineto{\pgfqpoint{0.844777in}{0.680608in}}%
\pgfpathlineto{\pgfqpoint{0.843442in}{0.683454in}}%
\pgfpathlineto{\pgfqpoint{0.846822in}{0.686101in}}%
\pgfpathlineto{\pgfqpoint{0.855456in}{0.683423in}}%
\pgfpathlineto{\pgfqpoint{0.860934in}{0.681377in}}%
\pgfpathlineto{\pgfqpoint{0.863943in}{0.679699in}}%
\pgfpathclose%
\pgfusepath{fill}%
\end{pgfscope}%
\begin{pgfscope}%
\pgfpathrectangle{\pgfqpoint{0.100000in}{0.100000in}}{\pgfqpoint{3.608454in}{2.310000in}}%
\pgfusepath{clip}%
\pgfsetbuttcap%
\pgfsetmiterjoin%
\definecolor{currentfill}{rgb}{0.000000,0.796078,0.601961}%
\pgfsetfillcolor{currentfill}%
\pgfsetlinewidth{0.000000pt}%
\definecolor{currentstroke}{rgb}{0.000000,0.000000,0.000000}%
\pgfsetstrokecolor{currentstroke}%
\pgfsetstrokeopacity{0.000000}%
\pgfsetdash{}{0pt}%
\pgfpathmoveto{\pgfqpoint{1.805743in}{1.664546in}}%
\pgfpathlineto{\pgfqpoint{1.746207in}{1.668241in}}%
\pgfpathlineto{\pgfqpoint{1.748309in}{1.699187in}}%
\pgfpathlineto{\pgfqpoint{1.750543in}{1.731001in}}%
\pgfpathlineto{\pgfqpoint{1.760148in}{1.734344in}}%
\pgfpathlineto{\pgfqpoint{1.768321in}{1.732235in}}%
\pgfpathlineto{\pgfqpoint{1.775409in}{1.724494in}}%
\pgfpathlineto{\pgfqpoint{1.787775in}{1.725501in}}%
\pgfpathlineto{\pgfqpoint{1.793255in}{1.722620in}}%
\pgfpathlineto{\pgfqpoint{1.800957in}{1.721865in}}%
\pgfpathlineto{\pgfqpoint{1.807130in}{1.721260in}}%
\pgfpathlineto{\pgfqpoint{1.806996in}{1.702523in}}%
\pgfpathlineto{\pgfqpoint{1.805445in}{1.674881in}}%
\pgfpathlineto{\pgfqpoint{1.805743in}{1.664546in}}%
\pgfpathclose%
\pgfusepath{fill}%
\end{pgfscope}%
\begin{pgfscope}%
\pgfpathrectangle{\pgfqpoint{0.100000in}{0.100000in}}{\pgfqpoint{3.608454in}{2.310000in}}%
\pgfusepath{clip}%
\pgfsetbuttcap%
\pgfsetmiterjoin%
\definecolor{currentfill}{rgb}{0.000000,0.607843,0.696078}%
\pgfsetfillcolor{currentfill}%
\pgfsetlinewidth{0.000000pt}%
\definecolor{currentstroke}{rgb}{0.000000,0.000000,0.000000}%
\pgfsetstrokecolor{currentstroke}%
\pgfsetstrokeopacity{0.000000}%
\pgfsetdash{}{0pt}%
\pgfpathmoveto{\pgfqpoint{0.895815in}{1.966858in}}%
\pgfpathlineto{\pgfqpoint{0.891257in}{1.963356in}}%
\pgfpathlineto{\pgfqpoint{0.890375in}{1.956866in}}%
\pgfpathlineto{\pgfqpoint{0.884981in}{1.950005in}}%
\pgfpathlineto{\pgfqpoint{0.875640in}{1.944745in}}%
\pgfpathlineto{\pgfqpoint{0.866297in}{1.932367in}}%
\pgfpathlineto{\pgfqpoint{0.860164in}{1.926417in}}%
\pgfpathlineto{\pgfqpoint{0.857097in}{1.912795in}}%
\pgfpathlineto{\pgfqpoint{0.842438in}{1.916478in}}%
\pgfpathlineto{\pgfqpoint{0.844130in}{1.923115in}}%
\pgfpathlineto{\pgfqpoint{0.839459in}{1.928946in}}%
\pgfpathlineto{\pgfqpoint{0.818538in}{1.934210in}}%
\pgfpathlineto{\pgfqpoint{0.814693in}{1.932896in}}%
\pgfpathlineto{\pgfqpoint{0.805150in}{1.923524in}}%
\pgfpathlineto{\pgfqpoint{0.800486in}{1.923543in}}%
\pgfpathlineto{\pgfqpoint{0.785890in}{1.927298in}}%
\pgfpathlineto{\pgfqpoint{0.792882in}{1.935165in}}%
\pgfpathlineto{\pgfqpoint{0.793934in}{1.941045in}}%
\pgfpathlineto{\pgfqpoint{0.803077in}{1.950037in}}%
\pgfpathlineto{\pgfqpoint{0.801936in}{1.953999in}}%
\pgfpathlineto{\pgfqpoint{0.793832in}{1.958919in}}%
\pgfpathlineto{\pgfqpoint{0.793910in}{1.961931in}}%
\pgfpathlineto{\pgfqpoint{0.807519in}{1.962306in}}%
\pgfpathlineto{\pgfqpoint{0.807404in}{1.969358in}}%
\pgfpathlineto{\pgfqpoint{0.813336in}{1.971025in}}%
\pgfpathlineto{\pgfqpoint{0.814622in}{1.975723in}}%
\pgfpathlineto{\pgfqpoint{0.808158in}{1.981911in}}%
\pgfpathlineto{\pgfqpoint{0.799750in}{1.984499in}}%
\pgfpathlineto{\pgfqpoint{0.802344in}{2.000341in}}%
\pgfpathlineto{\pgfqpoint{0.796435in}{2.001902in}}%
\pgfpathlineto{\pgfqpoint{0.797207in}{2.013901in}}%
\pgfpathlineto{\pgfqpoint{0.811964in}{2.010809in}}%
\pgfpathlineto{\pgfqpoint{0.813450in}{2.016488in}}%
\pgfpathlineto{\pgfqpoint{0.825859in}{2.013085in}}%
\pgfpathlineto{\pgfqpoint{0.831103in}{2.015303in}}%
\pgfpathlineto{\pgfqpoint{0.835457in}{2.032072in}}%
\pgfpathlineto{\pgfqpoint{0.838162in}{2.031358in}}%
\pgfpathlineto{\pgfqpoint{0.841907in}{2.040973in}}%
\pgfpathlineto{\pgfqpoint{0.846645in}{2.043323in}}%
\pgfpathlineto{\pgfqpoint{0.858767in}{2.040242in}}%
\pgfpathlineto{\pgfqpoint{0.853215in}{2.027577in}}%
\pgfpathlineto{\pgfqpoint{0.855049in}{2.021169in}}%
\pgfpathlineto{\pgfqpoint{0.853023in}{2.013264in}}%
\pgfpathlineto{\pgfqpoint{0.854053in}{1.998430in}}%
\pgfpathlineto{\pgfqpoint{0.859301in}{1.991402in}}%
\pgfpathlineto{\pgfqpoint{0.859744in}{1.982888in}}%
\pgfpathlineto{\pgfqpoint{0.871160in}{1.980064in}}%
\pgfpathlineto{\pgfqpoint{0.869552in}{1.973496in}}%
\pgfpathlineto{\pgfqpoint{0.895815in}{1.966858in}}%
\pgfpathclose%
\pgfusepath{fill}%
\end{pgfscope}%
\begin{pgfscope}%
\pgfpathrectangle{\pgfqpoint{0.100000in}{0.100000in}}{\pgfqpoint{3.608454in}{2.310000in}}%
\pgfusepath{clip}%
\pgfsetbuttcap%
\pgfsetmiterjoin%
\definecolor{currentfill}{rgb}{0.000000,0.364706,0.817647}%
\pgfsetfillcolor{currentfill}%
\pgfsetlinewidth{0.000000pt}%
\definecolor{currentstroke}{rgb}{0.000000,0.000000,0.000000}%
\pgfsetstrokecolor{currentstroke}%
\pgfsetstrokeopacity{0.000000}%
\pgfsetdash{}{0pt}%
\pgfpathmoveto{\pgfqpoint{1.791350in}{1.359031in}}%
\pgfpathlineto{\pgfqpoint{1.789033in}{1.324734in}}%
\pgfpathlineto{\pgfqpoint{1.756162in}{1.326912in}}%
\pgfpathlineto{\pgfqpoint{1.707133in}{1.330203in}}%
\pgfpathlineto{\pgfqpoint{1.709976in}{1.364447in}}%
\pgfpathlineto{\pgfqpoint{1.715265in}{1.364101in}}%
\pgfpathlineto{\pgfqpoint{1.717833in}{1.398242in}}%
\pgfpathlineto{\pgfqpoint{1.791324in}{1.393448in}}%
\pgfpathlineto{\pgfqpoint{1.792462in}{1.393414in}}%
\pgfpathlineto{\pgfqpoint{1.790342in}{1.359067in}}%
\pgfpathlineto{\pgfqpoint{1.791350in}{1.359031in}}%
\pgfpathclose%
\pgfusepath{fill}%
\end{pgfscope}%
\begin{pgfscope}%
\pgfpathrectangle{\pgfqpoint{0.100000in}{0.100000in}}{\pgfqpoint{3.608454in}{2.310000in}}%
\pgfusepath{clip}%
\pgfsetbuttcap%
\pgfsetmiterjoin%
\definecolor{currentfill}{rgb}{0.000000,0.486275,0.756863}%
\pgfsetfillcolor{currentfill}%
\pgfsetlinewidth{0.000000pt}%
\definecolor{currentstroke}{rgb}{0.000000,0.000000,0.000000}%
\pgfsetstrokecolor{currentstroke}%
\pgfsetstrokeopacity{0.000000}%
\pgfsetdash{}{0pt}%
\pgfpathmoveto{\pgfqpoint{2.464725in}{1.378777in}}%
\pgfpathlineto{\pgfqpoint{2.440862in}{1.377267in}}%
\pgfpathlineto{\pgfqpoint{2.439852in}{1.391129in}}%
\pgfpathlineto{\pgfqpoint{2.415980in}{1.389550in}}%
\pgfpathlineto{\pgfqpoint{2.410721in}{1.410092in}}%
\pgfpathlineto{\pgfqpoint{2.409257in}{1.435587in}}%
\pgfpathlineto{\pgfqpoint{2.410568in}{1.438329in}}%
\pgfpathlineto{\pgfqpoint{2.418888in}{1.437396in}}%
\pgfpathlineto{\pgfqpoint{2.427687in}{1.440218in}}%
\pgfpathlineto{\pgfqpoint{2.432920in}{1.438056in}}%
\pgfpathlineto{\pgfqpoint{2.431914in}{1.453643in}}%
\pgfpathlineto{\pgfqpoint{2.452287in}{1.455302in}}%
\pgfpathlineto{\pgfqpoint{2.459428in}{1.452340in}}%
\pgfpathlineto{\pgfqpoint{2.460863in}{1.433950in}}%
\pgfpathlineto{\pgfqpoint{2.461611in}{1.423638in}}%
\pgfpathlineto{\pgfqpoint{2.457108in}{1.423280in}}%
\pgfpathlineto{\pgfqpoint{2.457647in}{1.415090in}}%
\pgfpathlineto{\pgfqpoint{2.462297in}{1.414401in}}%
\pgfpathlineto{\pgfqpoint{2.463205in}{1.402934in}}%
\pgfpathlineto{\pgfqpoint{2.470095in}{1.403294in}}%
\pgfpathlineto{\pgfqpoint{2.471713in}{1.379003in}}%
\pgfpathlineto{\pgfqpoint{2.464725in}{1.378777in}}%
\pgfpathclose%
\pgfusepath{fill}%
\end{pgfscope}%
\begin{pgfscope}%
\pgfpathrectangle{\pgfqpoint{0.100000in}{0.100000in}}{\pgfqpoint{3.608454in}{2.310000in}}%
\pgfusepath{clip}%
\pgfsetbuttcap%
\pgfsetmiterjoin%
\definecolor{currentfill}{rgb}{0.000000,0.556863,0.721569}%
\pgfsetfillcolor{currentfill}%
\pgfsetlinewidth{0.000000pt}%
\definecolor{currentstroke}{rgb}{0.000000,0.000000,0.000000}%
\pgfsetstrokecolor{currentstroke}%
\pgfsetstrokeopacity{0.000000}%
\pgfsetdash{}{0pt}%
\pgfpathmoveto{\pgfqpoint{3.344220in}{1.667169in}}%
\pgfpathlineto{\pgfqpoint{3.332839in}{1.649737in}}%
\pgfpathlineto{\pgfqpoint{3.302353in}{1.660141in}}%
\pgfpathlineto{\pgfqpoint{3.298905in}{1.664653in}}%
\pgfpathlineto{\pgfqpoint{3.292838in}{1.664005in}}%
\pgfpathlineto{\pgfqpoint{3.282783in}{1.667998in}}%
\pgfpathlineto{\pgfqpoint{3.277583in}{1.676067in}}%
\pgfpathlineto{\pgfqpoint{3.274806in}{1.685909in}}%
\pgfpathlineto{\pgfqpoint{3.267733in}{1.692380in}}%
\pgfpathlineto{\pgfqpoint{3.302028in}{1.725960in}}%
\pgfpathlineto{\pgfqpoint{3.311248in}{1.723631in}}%
\pgfpathlineto{\pgfqpoint{3.324878in}{1.725364in}}%
\pgfpathlineto{\pgfqpoint{3.325396in}{1.731486in}}%
\pgfpathlineto{\pgfqpoint{3.333233in}{1.729959in}}%
\pgfpathlineto{\pgfqpoint{3.335015in}{1.726252in}}%
\pgfpathlineto{\pgfqpoint{3.347348in}{1.723269in}}%
\pgfpathlineto{\pgfqpoint{3.358373in}{1.723673in}}%
\pgfpathlineto{\pgfqpoint{3.358769in}{1.729605in}}%
\pgfpathlineto{\pgfqpoint{3.359316in}{1.729735in}}%
\pgfpathlineto{\pgfqpoint{3.366550in}{1.688953in}}%
\pgfpathlineto{\pgfqpoint{3.344000in}{1.680413in}}%
\pgfpathlineto{\pgfqpoint{3.344220in}{1.667169in}}%
\pgfpathclose%
\pgfusepath{fill}%
\end{pgfscope}%
\begin{pgfscope}%
\pgfpathrectangle{\pgfqpoint{0.100000in}{0.100000in}}{\pgfqpoint{3.608454in}{2.310000in}}%
\pgfusepath{clip}%
\pgfsetbuttcap%
\pgfsetmiterjoin%
\definecolor{currentfill}{rgb}{0.000000,0.674510,0.662745}%
\pgfsetfillcolor{currentfill}%
\pgfsetlinewidth{0.000000pt}%
\definecolor{currentstroke}{rgb}{0.000000,0.000000,0.000000}%
\pgfsetstrokecolor{currentstroke}%
\pgfsetstrokeopacity{0.000000}%
\pgfsetdash{}{0pt}%
\pgfpathmoveto{\pgfqpoint{1.443108in}{0.820998in}}%
\pgfpathlineto{\pgfqpoint{1.438374in}{0.821513in}}%
\pgfpathlineto{\pgfqpoint{1.404085in}{0.825375in}}%
\pgfpathlineto{\pgfqpoint{1.340564in}{0.833080in}}%
\pgfpathlineto{\pgfqpoint{1.350085in}{0.908598in}}%
\pgfpathlineto{\pgfqpoint{1.353414in}{0.915777in}}%
\pgfpathlineto{\pgfqpoint{1.356403in}{0.942467in}}%
\pgfpathlineto{\pgfqpoint{1.355496in}{0.949729in}}%
\pgfpathlineto{\pgfqpoint{1.357228in}{0.963939in}}%
\pgfpathlineto{\pgfqpoint{1.378188in}{0.960531in}}%
\pgfpathlineto{\pgfqpoint{1.379878in}{0.974364in}}%
\pgfpathlineto{\pgfqpoint{1.388260in}{0.973319in}}%
\pgfpathlineto{\pgfqpoint{1.392442in}{1.007538in}}%
\pgfpathlineto{\pgfqpoint{1.432205in}{1.002872in}}%
\pgfpathlineto{\pgfqpoint{1.432998in}{1.009699in}}%
\pgfpathlineto{\pgfqpoint{1.460404in}{1.006662in}}%
\pgfpathlineto{\pgfqpoint{1.458090in}{0.986273in}}%
\pgfpathlineto{\pgfqpoint{1.452184in}{0.931773in}}%
\pgfpathlineto{\pgfqpoint{1.449063in}{0.911411in}}%
\pgfpathlineto{\pgfqpoint{1.421822in}{0.913943in}}%
\pgfpathlineto{\pgfqpoint{1.420339in}{0.900498in}}%
\pgfpathlineto{\pgfqpoint{1.417927in}{0.900770in}}%
\pgfpathlineto{\pgfqpoint{1.413830in}{0.865768in}}%
\pgfpathlineto{\pgfqpoint{1.447334in}{0.862167in}}%
\pgfpathlineto{\pgfqpoint{1.443108in}{0.820998in}}%
\pgfpathclose%
\pgfusepath{fill}%
\end{pgfscope}%
\begin{pgfscope}%
\pgfpathrectangle{\pgfqpoint{0.100000in}{0.100000in}}{\pgfqpoint{3.608454in}{2.310000in}}%
\pgfusepath{clip}%
\pgfsetbuttcap%
\pgfsetmiterjoin%
\definecolor{currentfill}{rgb}{0.000000,0.627451,0.686275}%
\pgfsetfillcolor{currentfill}%
\pgfsetlinewidth{0.000000pt}%
\definecolor{currentstroke}{rgb}{0.000000,0.000000,0.000000}%
\pgfsetstrokecolor{currentstroke}%
\pgfsetstrokeopacity{0.000000}%
\pgfsetdash{}{0pt}%
\pgfpathmoveto{\pgfqpoint{2.853827in}{0.823500in}}%
\pgfpathlineto{\pgfqpoint{2.808283in}{0.818788in}}%
\pgfpathlineto{\pgfqpoint{2.786918in}{0.816730in}}%
\pgfpathlineto{\pgfqpoint{2.778092in}{0.819839in}}%
\pgfpathlineto{\pgfqpoint{2.776392in}{0.826653in}}%
\pgfpathlineto{\pgfqpoint{2.775432in}{0.832235in}}%
\pgfpathlineto{\pgfqpoint{2.783833in}{0.840916in}}%
\pgfpathlineto{\pgfqpoint{2.801485in}{0.843067in}}%
\pgfpathlineto{\pgfqpoint{2.799223in}{0.863800in}}%
\pgfpathlineto{\pgfqpoint{2.800195in}{0.867107in}}%
\pgfpathlineto{\pgfqpoint{2.812799in}{0.870744in}}%
\pgfpathlineto{\pgfqpoint{2.816271in}{0.861716in}}%
\pgfpathlineto{\pgfqpoint{2.822592in}{0.856703in}}%
\pgfpathlineto{\pgfqpoint{2.830726in}{0.859944in}}%
\pgfpathlineto{\pgfqpoint{2.834772in}{0.868856in}}%
\pgfpathlineto{\pgfqpoint{2.839212in}{0.870674in}}%
\pgfpathlineto{\pgfqpoint{2.851231in}{0.869522in}}%
\pgfpathlineto{\pgfqpoint{2.854444in}{0.865881in}}%
\pgfpathlineto{\pgfqpoint{2.855107in}{0.854082in}}%
\pgfpathlineto{\pgfqpoint{2.847916in}{0.849972in}}%
\pgfpathlineto{\pgfqpoint{2.844094in}{0.841666in}}%
\pgfpathlineto{\pgfqpoint{2.847520in}{0.837484in}}%
\pgfpathlineto{\pgfqpoint{2.853827in}{0.823500in}}%
\pgfpathclose%
\pgfusepath{fill}%
\end{pgfscope}%
\begin{pgfscope}%
\pgfpathrectangle{\pgfqpoint{0.100000in}{0.100000in}}{\pgfqpoint{3.608454in}{2.310000in}}%
\pgfusepath{clip}%
\pgfsetbuttcap%
\pgfsetmiterjoin%
\definecolor{currentfill}{rgb}{0.000000,0.843137,0.578431}%
\pgfsetfillcolor{currentfill}%
\pgfsetlinewidth{0.000000pt}%
\definecolor{currentstroke}{rgb}{0.000000,0.000000,0.000000}%
\pgfsetstrokecolor{currentstroke}%
\pgfsetstrokeopacity{0.000000}%
\pgfsetdash{}{0pt}%
\pgfpathmoveto{\pgfqpoint{2.756041in}{1.789197in}}%
\pgfpathlineto{\pgfqpoint{2.738013in}{1.786827in}}%
\pgfpathlineto{\pgfqpoint{2.729298in}{1.855252in}}%
\pgfpathlineto{\pgfqpoint{2.728097in}{1.868854in}}%
\pgfpathlineto{\pgfqpoint{2.707810in}{1.866142in}}%
\pgfpathlineto{\pgfqpoint{2.703507in}{1.899857in}}%
\pgfpathlineto{\pgfqpoint{2.706092in}{1.900170in}}%
\pgfpathlineto{\pgfqpoint{2.712457in}{1.891199in}}%
\pgfpathlineto{\pgfqpoint{2.723164in}{1.891127in}}%
\pgfpathlineto{\pgfqpoint{2.731458in}{1.886319in}}%
\pgfpathlineto{\pgfqpoint{2.749951in}{1.882041in}}%
\pgfpathlineto{\pgfqpoint{2.753811in}{1.874987in}}%
\pgfpathlineto{\pgfqpoint{2.760125in}{1.868000in}}%
\pgfpathlineto{\pgfqpoint{2.764215in}{1.858878in}}%
\pgfpathlineto{\pgfqpoint{2.754329in}{1.860433in}}%
\pgfpathlineto{\pgfqpoint{2.755769in}{1.850748in}}%
\pgfpathlineto{\pgfqpoint{2.762952in}{1.847681in}}%
\pgfpathlineto{\pgfqpoint{2.767020in}{1.834435in}}%
\pgfpathlineto{\pgfqpoint{2.765648in}{1.826069in}}%
\pgfpathlineto{\pgfqpoint{2.767351in}{1.804576in}}%
\pgfpathlineto{\pgfqpoint{2.762782in}{1.798736in}}%
\pgfpathlineto{\pgfqpoint{2.757883in}{1.797924in}}%
\pgfpathlineto{\pgfqpoint{2.756041in}{1.789197in}}%
\pgfpathclose%
\pgfusepath{fill}%
\end{pgfscope}%
\begin{pgfscope}%
\pgfpathrectangle{\pgfqpoint{0.100000in}{0.100000in}}{\pgfqpoint{3.608454in}{2.310000in}}%
\pgfusepath{clip}%
\pgfsetbuttcap%
\pgfsetmiterjoin%
\definecolor{currentfill}{rgb}{0.000000,0.513725,0.743137}%
\pgfsetfillcolor{currentfill}%
\pgfsetlinewidth{0.000000pt}%
\definecolor{currentstroke}{rgb}{0.000000,0.000000,0.000000}%
\pgfsetstrokecolor{currentstroke}%
\pgfsetstrokeopacity{0.000000}%
\pgfsetdash{}{0pt}%
\pgfpathmoveto{\pgfqpoint{2.773279in}{1.502104in}}%
\pgfpathlineto{\pgfqpoint{2.766536in}{1.500999in}}%
\pgfpathlineto{\pgfqpoint{2.767029in}{1.496892in}}%
\pgfpathlineto{\pgfqpoint{2.746418in}{1.494296in}}%
\pgfpathlineto{\pgfqpoint{2.747392in}{1.486269in}}%
\pgfpathlineto{\pgfqpoint{2.741875in}{1.483652in}}%
\pgfpathlineto{\pgfqpoint{2.719775in}{1.481004in}}%
\pgfpathlineto{\pgfqpoint{2.714189in}{1.531011in}}%
\pgfpathlineto{\pgfqpoint{2.741540in}{1.534284in}}%
\pgfpathlineto{\pgfqpoint{2.739953in}{1.548013in}}%
\pgfpathlineto{\pgfqpoint{2.746637in}{1.548837in}}%
\pgfpathlineto{\pgfqpoint{2.767138in}{1.551464in}}%
\pgfpathlineto{\pgfqpoint{2.773279in}{1.502104in}}%
\pgfpathclose%
\pgfusepath{fill}%
\end{pgfscope}%
\begin{pgfscope}%
\pgfpathrectangle{\pgfqpoint{0.100000in}{0.100000in}}{\pgfqpoint{3.608454in}{2.310000in}}%
\pgfusepath{clip}%
\pgfsetbuttcap%
\pgfsetmiterjoin%
\definecolor{currentfill}{rgb}{0.000000,0.768627,0.615686}%
\pgfsetfillcolor{currentfill}%
\pgfsetlinewidth{0.000000pt}%
\definecolor{currentstroke}{rgb}{0.000000,0.000000,0.000000}%
\pgfsetstrokecolor{currentstroke}%
\pgfsetstrokeopacity{0.000000}%
\pgfsetdash{}{0pt}%
\pgfpathmoveto{\pgfqpoint{3.130816in}{1.638979in}}%
\pgfpathlineto{\pgfqpoint{3.131860in}{1.634049in}}%
\pgfpathlineto{\pgfqpoint{3.109153in}{1.629395in}}%
\pgfpathlineto{\pgfqpoint{3.105576in}{1.628770in}}%
\pgfpathlineto{\pgfqpoint{3.094591in}{1.638194in}}%
\pgfpathlineto{\pgfqpoint{3.078140in}{1.636089in}}%
\pgfpathlineto{\pgfqpoint{3.050721in}{1.630639in}}%
\pgfpathlineto{\pgfqpoint{3.048278in}{1.646809in}}%
\pgfpathlineto{\pgfqpoint{3.050043in}{1.647962in}}%
\pgfpathlineto{\pgfqpoint{3.047674in}{1.660178in}}%
\pgfpathlineto{\pgfqpoint{3.039413in}{1.658799in}}%
\pgfpathlineto{\pgfqpoint{3.032000in}{1.700788in}}%
\pgfpathlineto{\pgfqpoint{3.036017in}{1.700877in}}%
\pgfpathlineto{\pgfqpoint{3.042918in}{1.694827in}}%
\pgfpathlineto{\pgfqpoint{3.049340in}{1.697252in}}%
\pgfpathlineto{\pgfqpoint{3.064906in}{1.706470in}}%
\pgfpathlineto{\pgfqpoint{3.066495in}{1.705513in}}%
\pgfpathlineto{\pgfqpoint{3.102306in}{1.712116in}}%
\pgfpathlineto{\pgfqpoint{3.109751in}{1.709791in}}%
\pgfpathlineto{\pgfqpoint{3.115261in}{1.672766in}}%
\pgfpathlineto{\pgfqpoint{3.123343in}{1.674373in}}%
\pgfpathlineto{\pgfqpoint{3.130816in}{1.638979in}}%
\pgfpathclose%
\pgfusepath{fill}%
\end{pgfscope}%
\begin{pgfscope}%
\pgfpathrectangle{\pgfqpoint{0.100000in}{0.100000in}}{\pgfqpoint{3.608454in}{2.310000in}}%
\pgfusepath{clip}%
\pgfsetbuttcap%
\pgfsetmiterjoin%
\definecolor{currentfill}{rgb}{0.000000,0.552941,0.723529}%
\pgfsetfillcolor{currentfill}%
\pgfsetlinewidth{0.000000pt}%
\definecolor{currentstroke}{rgb}{0.000000,0.000000,0.000000}%
\pgfsetstrokecolor{currentstroke}%
\pgfsetstrokeopacity{0.000000}%
\pgfsetdash{}{0pt}%
\pgfpathmoveto{\pgfqpoint{2.682759in}{1.669371in}}%
\pgfpathlineto{\pgfqpoint{2.696454in}{1.670883in}}%
\pgfpathlineto{\pgfqpoint{2.693396in}{1.698299in}}%
\pgfpathlineto{\pgfqpoint{2.720350in}{1.701323in}}%
\pgfpathlineto{\pgfqpoint{2.723746in}{1.674452in}}%
\pgfpathlineto{\pgfqpoint{2.735608in}{1.675908in}}%
\pgfpathlineto{\pgfqpoint{2.739935in}{1.648352in}}%
\pgfpathlineto{\pgfqpoint{2.706426in}{1.644227in}}%
\pgfpathlineto{\pgfqpoint{2.685918in}{1.641892in}}%
\pgfpathlineto{\pgfqpoint{2.682759in}{1.669371in}}%
\pgfpathclose%
\pgfusepath{fill}%
\end{pgfscope}%
\begin{pgfscope}%
\pgfpathrectangle{\pgfqpoint{0.100000in}{0.100000in}}{\pgfqpoint{3.608454in}{2.310000in}}%
\pgfusepath{clip}%
\pgfsetbuttcap%
\pgfsetmiterjoin%
\definecolor{currentfill}{rgb}{0.000000,0.447059,0.776471}%
\pgfsetfillcolor{currentfill}%
\pgfsetlinewidth{0.000000pt}%
\definecolor{currentstroke}{rgb}{0.000000,0.000000,0.000000}%
\pgfsetstrokecolor{currentstroke}%
\pgfsetstrokeopacity{0.000000}%
\pgfsetdash{}{0pt}%
\pgfpathmoveto{\pgfqpoint{1.992478in}{2.070813in}}%
\pgfpathlineto{\pgfqpoint{1.993953in}{2.077757in}}%
\pgfpathlineto{\pgfqpoint{1.991966in}{2.086284in}}%
\pgfpathlineto{\pgfqpoint{1.993550in}{2.092414in}}%
\pgfpathlineto{\pgfqpoint{1.991978in}{2.098902in}}%
\pgfpathlineto{\pgfqpoint{2.033124in}{2.098129in}}%
\pgfpathlineto{\pgfqpoint{2.074816in}{2.097239in}}%
\pgfpathlineto{\pgfqpoint{2.075068in}{2.055715in}}%
\pgfpathlineto{\pgfqpoint{2.075577in}{2.048780in}}%
\pgfpathlineto{\pgfqpoint{2.068749in}{2.048976in}}%
\pgfpathlineto{\pgfqpoint{2.068716in}{2.041994in}}%
\pgfpathlineto{\pgfqpoint{2.061858in}{2.042025in}}%
\pgfpathlineto{\pgfqpoint{2.061781in}{2.035022in}}%
\pgfpathlineto{\pgfqpoint{2.034103in}{2.035335in}}%
\pgfpathlineto{\pgfqpoint{2.034152in}{2.042276in}}%
\pgfpathlineto{\pgfqpoint{2.027218in}{2.042403in}}%
\pgfpathlineto{\pgfqpoint{2.026639in}{2.068568in}}%
\pgfpathlineto{\pgfqpoint{1.992147in}{2.069188in}}%
\pgfpathlineto{\pgfqpoint{1.992478in}{2.070813in}}%
\pgfpathclose%
\pgfusepath{fill}%
\end{pgfscope}%
\begin{pgfscope}%
\pgfpathrectangle{\pgfqpoint{0.100000in}{0.100000in}}{\pgfqpoint{3.608454in}{2.310000in}}%
\pgfusepath{clip}%
\pgfsetbuttcap%
\pgfsetmiterjoin%
\definecolor{currentfill}{rgb}{0.000000,0.439216,0.780392}%
\pgfsetfillcolor{currentfill}%
\pgfsetlinewidth{0.000000pt}%
\definecolor{currentstroke}{rgb}{0.000000,0.000000,0.000000}%
\pgfsetstrokecolor{currentstroke}%
\pgfsetstrokeopacity{0.000000}%
\pgfsetdash{}{0pt}%
\pgfpathmoveto{\pgfqpoint{3.181076in}{1.501996in}}%
\pgfpathlineto{\pgfqpoint{3.166603in}{1.499150in}}%
\pgfpathlineto{\pgfqpoint{3.162472in}{1.516430in}}%
\pgfpathlineto{\pgfqpoint{3.146592in}{1.539583in}}%
\pgfpathlineto{\pgfqpoint{3.143924in}{1.538541in}}%
\pgfpathlineto{\pgfqpoint{3.138889in}{1.546016in}}%
\pgfpathlineto{\pgfqpoint{3.133310in}{1.544667in}}%
\pgfpathlineto{\pgfqpoint{3.130234in}{1.548464in}}%
\pgfpathlineto{\pgfqpoint{3.133732in}{1.555574in}}%
\pgfpathlineto{\pgfqpoint{3.131483in}{1.559276in}}%
\pgfpathlineto{\pgfqpoint{3.139700in}{1.576466in}}%
\pgfpathlineto{\pgfqpoint{3.155139in}{1.587716in}}%
\pgfpathlineto{\pgfqpoint{3.157897in}{1.576523in}}%
\pgfpathlineto{\pgfqpoint{3.176843in}{1.578501in}}%
\pgfpathlineto{\pgfqpoint{3.184525in}{1.574369in}}%
\pgfpathlineto{\pgfqpoint{3.196959in}{1.580835in}}%
\pgfpathlineto{\pgfqpoint{3.208359in}{1.574869in}}%
\pgfpathlineto{\pgfqpoint{3.201240in}{1.566916in}}%
\pgfpathlineto{\pgfqpoint{3.212062in}{1.546662in}}%
\pgfpathlineto{\pgfqpoint{3.204878in}{1.540940in}}%
\pgfpathlineto{\pgfqpoint{3.208552in}{1.535223in}}%
\pgfpathlineto{\pgfqpoint{3.216418in}{1.535851in}}%
\pgfpathlineto{\pgfqpoint{3.221777in}{1.528537in}}%
\pgfpathlineto{\pgfqpoint{3.227209in}{1.526852in}}%
\pgfpathlineto{\pgfqpoint{3.239245in}{1.513965in}}%
\pgfpathlineto{\pgfqpoint{3.181076in}{1.501996in}}%
\pgfpathclose%
\pgfusepath{fill}%
\end{pgfscope}%
\begin{pgfscope}%
\pgfpathrectangle{\pgfqpoint{0.100000in}{0.100000in}}{\pgfqpoint{3.608454in}{2.310000in}}%
\pgfusepath{clip}%
\pgfsetbuttcap%
\pgfsetmiterjoin%
\definecolor{currentfill}{rgb}{0.000000,0.278431,0.860784}%
\pgfsetfillcolor{currentfill}%
\pgfsetlinewidth{0.000000pt}%
\definecolor{currentstroke}{rgb}{0.000000,0.000000,0.000000}%
\pgfsetstrokecolor{currentstroke}%
\pgfsetstrokeopacity{0.000000}%
\pgfsetdash{}{0pt}%
\pgfpathmoveto{\pgfqpoint{1.654457in}{1.551499in}}%
\pgfpathlineto{\pgfqpoint{1.655781in}{1.547900in}}%
\pgfpathlineto{\pgfqpoint{1.654575in}{1.534236in}}%
\pgfpathlineto{\pgfqpoint{1.653438in}{1.520561in}}%
\pgfpathlineto{\pgfqpoint{1.650561in}{1.517162in}}%
\pgfpathlineto{\pgfqpoint{1.595804in}{1.522015in}}%
\pgfpathlineto{\pgfqpoint{1.533391in}{1.528263in}}%
\pgfpathlineto{\pgfqpoint{1.494704in}{1.532257in}}%
\pgfpathlineto{\pgfqpoint{1.500517in}{1.584019in}}%
\pgfpathlineto{\pgfqpoint{1.537205in}{1.579689in}}%
\pgfpathlineto{\pgfqpoint{1.536486in}{1.572765in}}%
\pgfpathlineto{\pgfqpoint{1.571835in}{1.569055in}}%
\pgfpathlineto{\pgfqpoint{1.570505in}{1.555595in}}%
\pgfpathlineto{\pgfqpoint{1.610795in}{1.551818in}}%
\pgfpathlineto{\pgfqpoint{1.611187in}{1.555209in}}%
\pgfpathlineto{\pgfqpoint{1.654457in}{1.551499in}}%
\pgfpathclose%
\pgfusepath{fill}%
\end{pgfscope}%
\begin{pgfscope}%
\pgfpathrectangle{\pgfqpoint{0.100000in}{0.100000in}}{\pgfqpoint{3.608454in}{2.310000in}}%
\pgfusepath{clip}%
\pgfsetbuttcap%
\pgfsetmiterjoin%
\definecolor{currentfill}{rgb}{0.000000,0.713725,0.643137}%
\pgfsetfillcolor{currentfill}%
\pgfsetlinewidth{0.000000pt}%
\definecolor{currentstroke}{rgb}{0.000000,0.000000,0.000000}%
\pgfsetstrokecolor{currentstroke}%
\pgfsetstrokeopacity{0.000000}%
\pgfsetdash{}{0pt}%
\pgfpathmoveto{\pgfqpoint{1.421683in}{1.386315in}}%
\pgfpathlineto{\pgfqpoint{1.421188in}{1.383353in}}%
\pgfpathlineto{\pgfqpoint{1.430521in}{1.376213in}}%
\pgfpathlineto{\pgfqpoint{1.436137in}{1.369904in}}%
\pgfpathlineto{\pgfqpoint{1.436653in}{1.364650in}}%
\pgfpathlineto{\pgfqpoint{1.431712in}{1.356218in}}%
\pgfpathlineto{\pgfqpoint{1.436154in}{1.349875in}}%
\pgfpathlineto{\pgfqpoint{1.435665in}{1.345243in}}%
\pgfpathlineto{\pgfqpoint{1.426840in}{1.337144in}}%
\pgfpathlineto{\pgfqpoint{1.422616in}{1.335846in}}%
\pgfpathlineto{\pgfqpoint{1.412097in}{1.337063in}}%
\pgfpathlineto{\pgfqpoint{1.406993in}{1.344082in}}%
\pgfpathlineto{\pgfqpoint{1.404621in}{1.362130in}}%
\pgfpathlineto{\pgfqpoint{1.405672in}{1.369838in}}%
\pgfpathlineto{\pgfqpoint{1.412208in}{1.374642in}}%
\pgfpathlineto{\pgfqpoint{1.403546in}{1.376980in}}%
\pgfpathlineto{\pgfqpoint{1.396241in}{1.384895in}}%
\pgfpathlineto{\pgfqpoint{1.398138in}{1.389426in}}%
\pgfpathlineto{\pgfqpoint{1.421683in}{1.386315in}}%
\pgfpathclose%
\pgfusepath{fill}%
\end{pgfscope}%
\begin{pgfscope}%
\pgfpathrectangle{\pgfqpoint{0.100000in}{0.100000in}}{\pgfqpoint{3.608454in}{2.310000in}}%
\pgfusepath{clip}%
\pgfsetbuttcap%
\pgfsetmiterjoin%
\definecolor{currentfill}{rgb}{0.000000,0.572549,0.713725}%
\pgfsetfillcolor{currentfill}%
\pgfsetlinewidth{0.000000pt}%
\definecolor{currentstroke}{rgb}{0.000000,0.000000,0.000000}%
\pgfsetstrokecolor{currentstroke}%
\pgfsetstrokeopacity{0.000000}%
\pgfsetdash{}{0pt}%
\pgfpathmoveto{\pgfqpoint{2.517006in}{1.100169in}}%
\pgfpathlineto{\pgfqpoint{2.510594in}{1.099857in}}%
\pgfpathlineto{\pgfqpoint{2.510006in}{1.121373in}}%
\pgfpathlineto{\pgfqpoint{2.512080in}{1.125849in}}%
\pgfpathlineto{\pgfqpoint{2.519951in}{1.126253in}}%
\pgfpathlineto{\pgfqpoint{2.518781in}{1.156793in}}%
\pgfpathlineto{\pgfqpoint{2.520509in}{1.156888in}}%
\pgfpathlineto{\pgfqpoint{2.548186in}{1.158629in}}%
\pgfpathlineto{\pgfqpoint{2.548297in}{1.153932in}}%
\pgfpathlineto{\pgfqpoint{2.555391in}{1.144733in}}%
\pgfpathlineto{\pgfqpoint{2.556313in}{1.139100in}}%
\pgfpathlineto{\pgfqpoint{2.570262in}{1.139592in}}%
\pgfpathlineto{\pgfqpoint{2.579740in}{1.137359in}}%
\pgfpathlineto{\pgfqpoint{2.583376in}{1.141723in}}%
\pgfpathlineto{\pgfqpoint{2.583459in}{1.148544in}}%
\pgfpathlineto{\pgfqpoint{2.597965in}{1.148745in}}%
\pgfpathlineto{\pgfqpoint{2.606315in}{1.148332in}}%
\pgfpathlineto{\pgfqpoint{2.604989in}{1.140525in}}%
\pgfpathlineto{\pgfqpoint{2.606773in}{1.136211in}}%
\pgfpathlineto{\pgfqpoint{2.605651in}{1.120589in}}%
\pgfpathlineto{\pgfqpoint{2.584331in}{1.121514in}}%
\pgfpathlineto{\pgfqpoint{2.573891in}{1.113688in}}%
\pgfpathlineto{\pgfqpoint{2.573691in}{1.108364in}}%
\pgfpathlineto{\pgfqpoint{2.561312in}{1.107350in}}%
\pgfpathlineto{\pgfqpoint{2.553934in}{1.104609in}}%
\pgfpathlineto{\pgfqpoint{2.552190in}{1.107568in}}%
\pgfpathlineto{\pgfqpoint{2.544235in}{1.104329in}}%
\pgfpathlineto{\pgfqpoint{2.519200in}{1.103140in}}%
\pgfpathlineto{\pgfqpoint{2.517006in}{1.100169in}}%
\pgfpathclose%
\pgfusepath{fill}%
\end{pgfscope}%
\begin{pgfscope}%
\pgfpathrectangle{\pgfqpoint{0.100000in}{0.100000in}}{\pgfqpoint{3.608454in}{2.310000in}}%
\pgfusepath{clip}%
\pgfsetbuttcap%
\pgfsetmiterjoin%
\definecolor{currentfill}{rgb}{0.000000,0.360784,0.819608}%
\pgfsetfillcolor{currentfill}%
\pgfsetlinewidth{0.000000pt}%
\definecolor{currentstroke}{rgb}{0.000000,0.000000,0.000000}%
\pgfsetstrokecolor{currentstroke}%
\pgfsetstrokeopacity{0.000000}%
\pgfsetdash{}{0pt}%
\pgfpathmoveto{\pgfqpoint{1.027182in}{0.384801in}}%
\pgfpathlineto{\pgfqpoint{1.017303in}{0.390307in}}%
\pgfpathlineto{\pgfqpoint{1.020524in}{0.396033in}}%
\pgfpathlineto{\pgfqpoint{1.019250in}{0.398079in}}%
\pgfpathlineto{\pgfqpoint{1.024400in}{0.406832in}}%
\pgfpathlineto{\pgfqpoint{1.022393in}{0.408007in}}%
\pgfpathlineto{\pgfqpoint{1.023662in}{0.410127in}}%
\pgfpathlineto{\pgfqpoint{1.021385in}{0.411488in}}%
\pgfpathlineto{\pgfqpoint{1.022698in}{0.413642in}}%
\pgfpathlineto{\pgfqpoint{1.018953in}{0.415851in}}%
\pgfpathlineto{\pgfqpoint{1.018284in}{0.417154in}}%
\pgfpathlineto{\pgfqpoint{1.015043in}{0.415179in}}%
\pgfpathlineto{\pgfqpoint{1.012216in}{0.417308in}}%
\pgfpathlineto{\pgfqpoint{1.015064in}{0.421049in}}%
\pgfpathlineto{\pgfqpoint{1.014239in}{0.423539in}}%
\pgfpathlineto{\pgfqpoint{1.017896in}{0.425014in}}%
\pgfpathlineto{\pgfqpoint{1.019938in}{0.430914in}}%
\pgfpathlineto{\pgfqpoint{1.002410in}{0.441046in}}%
\pgfpathlineto{\pgfqpoint{0.996218in}{0.444880in}}%
\pgfpathlineto{\pgfqpoint{0.997579in}{0.447110in}}%
\pgfpathlineto{\pgfqpoint{0.986635in}{0.453959in}}%
\pgfpathlineto{\pgfqpoint{0.983915in}{0.449521in}}%
\pgfpathlineto{\pgfqpoint{0.979580in}{0.452265in}}%
\pgfpathlineto{\pgfqpoint{0.986452in}{0.463301in}}%
\pgfpathlineto{\pgfqpoint{0.972860in}{0.471797in}}%
\pgfpathlineto{\pgfqpoint{0.968975in}{0.474416in}}%
\pgfpathlineto{\pgfqpoint{0.966684in}{0.470967in}}%
\pgfpathlineto{\pgfqpoint{0.950166in}{0.482212in}}%
\pgfpathlineto{\pgfqpoint{0.940198in}{0.483801in}}%
\pgfpathlineto{\pgfqpoint{0.915352in}{0.487908in}}%
\pgfpathlineto{\pgfqpoint{0.914054in}{0.486250in}}%
\pgfpathlineto{\pgfqpoint{0.902205in}{0.495355in}}%
\pgfpathlineto{\pgfqpoint{0.894650in}{0.485392in}}%
\pgfpathlineto{\pgfqpoint{0.892390in}{0.483966in}}%
\pgfpathlineto{\pgfqpoint{0.888822in}{0.486756in}}%
\pgfpathlineto{\pgfqpoint{0.887256in}{0.484755in}}%
\pgfpathlineto{\pgfqpoint{0.883274in}{0.487912in}}%
\pgfpathlineto{\pgfqpoint{0.880946in}{0.484915in}}%
\pgfpathlineto{\pgfqpoint{0.878680in}{0.486403in}}%
\pgfpathlineto{\pgfqpoint{0.867623in}{0.494883in}}%
\pgfpathlineto{\pgfqpoint{0.854046in}{0.505533in}}%
\pgfpathlineto{\pgfqpoint{0.845087in}{0.513012in}}%
\pgfpathlineto{\pgfqpoint{0.846928in}{0.515267in}}%
\pgfpathlineto{\pgfqpoint{0.831654in}{0.528086in}}%
\pgfpathlineto{\pgfqpoint{0.830011in}{0.526141in}}%
\pgfpathlineto{\pgfqpoint{0.821946in}{0.532976in}}%
\pgfpathlineto{\pgfqpoint{0.819842in}{0.530526in}}%
\pgfpathlineto{\pgfqpoint{0.812784in}{0.536599in}}%
\pgfpathlineto{\pgfqpoint{0.802699in}{0.545595in}}%
\pgfpathlineto{\pgfqpoint{0.807682in}{0.551257in}}%
\pgfpathlineto{\pgfqpoint{0.806268in}{0.552525in}}%
\pgfpathlineto{\pgfqpoint{0.813107in}{0.560166in}}%
\pgfpathlineto{\pgfqpoint{0.813592in}{0.559787in}}%
\pgfpathlineto{\pgfqpoint{0.820507in}{0.567425in}}%
\pgfpathlineto{\pgfqpoint{0.823025in}{0.569214in}}%
\pgfpathlineto{\pgfqpoint{0.824219in}{0.568707in}}%
\pgfpathlineto{\pgfqpoint{0.829339in}{0.574413in}}%
\pgfpathlineto{\pgfqpoint{0.831146in}{0.572881in}}%
\pgfpathlineto{\pgfqpoint{0.836205in}{0.579038in}}%
\pgfpathlineto{\pgfqpoint{0.838193in}{0.577309in}}%
\pgfpathlineto{\pgfqpoint{0.839724in}{0.579022in}}%
\pgfpathlineto{\pgfqpoint{0.843712in}{0.575563in}}%
\pgfpathlineto{\pgfqpoint{0.845387in}{0.577604in}}%
\pgfpathlineto{\pgfqpoint{0.847165in}{0.576109in}}%
\pgfpathlineto{\pgfqpoint{0.848929in}{0.578173in}}%
\pgfpathlineto{\pgfqpoint{0.850691in}{0.576622in}}%
\pgfpathlineto{\pgfqpoint{0.856024in}{0.582334in}}%
\pgfpathlineto{\pgfqpoint{0.853750in}{0.584249in}}%
\pgfpathlineto{\pgfqpoint{0.867382in}{0.600383in}}%
\pgfpathlineto{\pgfqpoint{0.869420in}{0.598690in}}%
\pgfpathlineto{\pgfqpoint{0.872825in}{0.602697in}}%
\pgfpathlineto{\pgfqpoint{0.874796in}{0.601023in}}%
\pgfpathlineto{\pgfqpoint{0.881527in}{0.608877in}}%
\pgfpathlineto{\pgfqpoint{0.883388in}{0.607238in}}%
\pgfpathlineto{\pgfqpoint{0.888579in}{0.613334in}}%
\pgfpathlineto{\pgfqpoint{0.886576in}{0.615036in}}%
\pgfpathlineto{\pgfqpoint{0.895296in}{0.625196in}}%
\pgfpathlineto{\pgfqpoint{0.901313in}{0.620035in}}%
\pgfpathlineto{\pgfqpoint{0.904546in}{0.623960in}}%
\pgfpathlineto{\pgfqpoint{0.914790in}{0.615304in}}%
\pgfpathlineto{\pgfqpoint{0.921542in}{0.623737in}}%
\pgfpathlineto{\pgfqpoint{0.929884in}{0.616921in}}%
\pgfpathlineto{\pgfqpoint{0.926478in}{0.612658in}}%
\pgfpathlineto{\pgfqpoint{0.930663in}{0.609300in}}%
\pgfpathlineto{\pgfqpoint{0.934029in}{0.613492in}}%
\pgfpathlineto{\pgfqpoint{0.938173in}{0.610139in}}%
\pgfpathlineto{\pgfqpoint{0.936545in}{0.608078in}}%
\pgfpathlineto{\pgfqpoint{0.938627in}{0.606373in}}%
\pgfpathlineto{\pgfqpoint{0.936932in}{0.604209in}}%
\pgfpathlineto{\pgfqpoint{0.940885in}{0.601118in}}%
\pgfpathlineto{\pgfqpoint{0.946326in}{0.607097in}}%
\pgfpathlineto{\pgfqpoint{0.952323in}{0.602325in}}%
\pgfpathlineto{\pgfqpoint{0.955683in}{0.606572in}}%
\pgfpathlineto{\pgfqpoint{0.961995in}{0.601456in}}%
\pgfpathlineto{\pgfqpoint{0.964303in}{0.603251in}}%
\pgfpathlineto{\pgfqpoint{0.971435in}{0.612298in}}%
\pgfpathlineto{\pgfqpoint{0.969910in}{0.613404in}}%
\pgfpathlineto{\pgfqpoint{0.971558in}{0.615632in}}%
\pgfpathlineto{\pgfqpoint{0.967277in}{0.618907in}}%
\pgfpathlineto{\pgfqpoint{0.972240in}{0.625296in}}%
\pgfpathlineto{\pgfqpoint{0.974510in}{0.627028in}}%
\pgfpathlineto{\pgfqpoint{0.972344in}{0.628719in}}%
\pgfpathlineto{\pgfqpoint{0.974033in}{0.630789in}}%
\pgfpathlineto{\pgfqpoint{0.971825in}{0.632484in}}%
\pgfpathlineto{\pgfqpoint{0.973534in}{0.634639in}}%
\pgfpathlineto{\pgfqpoint{0.971445in}{0.636313in}}%
\pgfpathlineto{\pgfqpoint{0.976297in}{0.641465in}}%
\pgfpathlineto{\pgfqpoint{0.997892in}{0.624739in}}%
\pgfpathlineto{\pgfqpoint{1.026608in}{0.603847in}}%
\pgfpathlineto{\pgfqpoint{1.052163in}{0.586365in}}%
\pgfpathlineto{\pgfqpoint{1.056440in}{0.584227in}}%
\pgfpathlineto{\pgfqpoint{1.066205in}{0.578011in}}%
\pgfpathlineto{\pgfqpoint{1.074192in}{0.589904in}}%
\pgfpathlineto{\pgfqpoint{1.083089in}{0.584700in}}%
\pgfpathlineto{\pgfqpoint{1.110388in}{0.567904in}}%
\pgfpathlineto{\pgfqpoint{1.124077in}{0.559590in}}%
\pgfpathlineto{\pgfqpoint{1.114317in}{0.541984in}}%
\pgfpathlineto{\pgfqpoint{1.101033in}{0.518021in}}%
\pgfpathlineto{\pgfqpoint{1.087919in}{0.494364in}}%
\pgfpathlineto{\pgfqpoint{1.075939in}{0.472754in}}%
\pgfpathlineto{\pgfqpoint{1.063110in}{0.449610in}}%
\pgfpathlineto{\pgfqpoint{1.050408in}{0.426699in}}%
\pgfpathlineto{\pgfqpoint{1.037825in}{0.403999in}}%
\pgfpathlineto{\pgfqpoint{1.027182in}{0.384801in}}%
\pgfpathclose%
\pgfusepath{fill}%
\end{pgfscope}%
\begin{pgfscope}%
\pgfpathrectangle{\pgfqpoint{0.100000in}{0.100000in}}{\pgfqpoint{3.608454in}{2.310000in}}%
\pgfusepath{clip}%
\pgfsetbuttcap%
\pgfsetmiterjoin%
\definecolor{currentfill}{rgb}{0.000000,0.580392,0.709804}%
\pgfsetfillcolor{currentfill}%
\pgfsetlinewidth{0.000000pt}%
\definecolor{currentstroke}{rgb}{0.000000,0.000000,0.000000}%
\pgfsetstrokecolor{currentstroke}%
\pgfsetstrokeopacity{0.000000}%
\pgfsetdash{}{0pt}%
\pgfpathmoveto{\pgfqpoint{2.289673in}{1.168159in}}%
\pgfpathlineto{\pgfqpoint{2.248336in}{1.167927in}}%
\pgfpathlineto{\pgfqpoint{2.211773in}{1.168036in}}%
\pgfpathlineto{\pgfqpoint{2.211810in}{1.181895in}}%
\pgfpathlineto{\pgfqpoint{2.194584in}{1.181869in}}%
\pgfpathlineto{\pgfqpoint{2.194554in}{1.204465in}}%
\pgfpathlineto{\pgfqpoint{2.193249in}{1.204475in}}%
\pgfpathlineto{\pgfqpoint{2.193719in}{1.240848in}}%
\pgfpathlineto{\pgfqpoint{2.192435in}{1.247763in}}%
\pgfpathlineto{\pgfqpoint{2.196058in}{1.254577in}}%
\pgfpathlineto{\pgfqpoint{2.199587in}{1.254527in}}%
\pgfpathlineto{\pgfqpoint{2.199885in}{1.267142in}}%
\pgfpathlineto{\pgfqpoint{2.227204in}{1.266822in}}%
\pgfpathlineto{\pgfqpoint{2.227021in}{1.254178in}}%
\pgfpathlineto{\pgfqpoint{2.255244in}{1.254033in}}%
\pgfpathlineto{\pgfqpoint{2.260153in}{1.246809in}}%
\pgfpathlineto{\pgfqpoint{2.268476in}{1.251773in}}%
\pgfpathlineto{\pgfqpoint{2.268526in}{1.240287in}}%
\pgfpathlineto{\pgfqpoint{2.278985in}{1.235414in}}%
\pgfpathlineto{\pgfqpoint{2.279053in}{1.231943in}}%
\pgfpathlineto{\pgfqpoint{2.280237in}{1.188871in}}%
\pgfpathlineto{\pgfqpoint{2.290504in}{1.188921in}}%
\pgfpathlineto{\pgfqpoint{2.289673in}{1.168159in}}%
\pgfpathclose%
\pgfusepath{fill}%
\end{pgfscope}%
\begin{pgfscope}%
\pgfpathrectangle{\pgfqpoint{0.100000in}{0.100000in}}{\pgfqpoint{3.608454in}{2.310000in}}%
\pgfusepath{clip}%
\pgfsetbuttcap%
\pgfsetmiterjoin%
\definecolor{currentfill}{rgb}{0.000000,0.800000,0.600000}%
\pgfsetfillcolor{currentfill}%
\pgfsetlinewidth{0.000000pt}%
\definecolor{currentstroke}{rgb}{0.000000,0.000000,0.000000}%
\pgfsetstrokecolor{currentstroke}%
\pgfsetstrokeopacity{0.000000}%
\pgfsetdash{}{0pt}%
\pgfpathmoveto{\pgfqpoint{2.175503in}{0.924555in}}%
\pgfpathlineto{\pgfqpoint{2.170065in}{0.926960in}}%
\pgfpathlineto{\pgfqpoint{2.157026in}{0.928424in}}%
\pgfpathlineto{\pgfqpoint{2.147313in}{0.939341in}}%
\pgfpathlineto{\pgfqpoint{2.141202in}{0.939334in}}%
\pgfpathlineto{\pgfqpoint{2.140773in}{0.915277in}}%
\pgfpathlineto{\pgfqpoint{2.123559in}{0.920377in}}%
\pgfpathlineto{\pgfqpoint{2.115386in}{0.923776in}}%
\pgfpathlineto{\pgfqpoint{2.109858in}{0.929360in}}%
\pgfpathlineto{\pgfqpoint{2.092356in}{0.940891in}}%
\pgfpathlineto{\pgfqpoint{2.087358in}{0.933839in}}%
\pgfpathlineto{\pgfqpoint{2.079921in}{0.933374in}}%
\pgfpathlineto{\pgfqpoint{2.070424in}{0.935180in}}%
\pgfpathlineto{\pgfqpoint{2.068217in}{0.938848in}}%
\pgfpathlineto{\pgfqpoint{2.056583in}{0.934051in}}%
\pgfpathlineto{\pgfqpoint{2.051407in}{0.934194in}}%
\pgfpathlineto{\pgfqpoint{2.045055in}{0.940136in}}%
\pgfpathlineto{\pgfqpoint{2.045185in}{0.954206in}}%
\pgfpathlineto{\pgfqpoint{2.041706in}{0.956720in}}%
\pgfpathlineto{\pgfqpoint{2.055634in}{0.956575in}}%
\pgfpathlineto{\pgfqpoint{2.055870in}{0.984320in}}%
\pgfpathlineto{\pgfqpoint{2.062908in}{0.984282in}}%
\pgfpathlineto{\pgfqpoint{2.062941in}{0.991185in}}%
\pgfpathlineto{\pgfqpoint{2.073230in}{0.991141in}}%
\pgfpathlineto{\pgfqpoint{2.073255in}{0.998037in}}%
\pgfpathlineto{\pgfqpoint{2.110850in}{0.997922in}}%
\pgfpathlineto{\pgfqpoint{2.110869in}{0.984129in}}%
\pgfpathlineto{\pgfqpoint{2.141998in}{0.984277in}}%
\pgfpathlineto{\pgfqpoint{2.142324in}{1.001839in}}%
\pgfpathlineto{\pgfqpoint{2.151260in}{0.999398in}}%
\pgfpathlineto{\pgfqpoint{2.170781in}{0.999186in}}%
\pgfpathlineto{\pgfqpoint{2.176519in}{0.997121in}}%
\pgfpathlineto{\pgfqpoint{2.176495in}{0.972103in}}%
\pgfpathlineto{\pgfqpoint{2.155587in}{0.972321in}}%
\pgfpathlineto{\pgfqpoint{2.156410in}{0.959401in}}%
\pgfpathlineto{\pgfqpoint{2.165791in}{0.959401in}}%
\pgfpathlineto{\pgfqpoint{2.166804in}{0.946907in}}%
\pgfpathlineto{\pgfqpoint{2.169177in}{0.935220in}}%
\pgfpathlineto{\pgfqpoint{2.174128in}{0.929756in}}%
\pgfpathlineto{\pgfqpoint{2.175503in}{0.924555in}}%
\pgfpathclose%
\pgfusepath{fill}%
\end{pgfscope}%
\begin{pgfscope}%
\pgfpathrectangle{\pgfqpoint{0.100000in}{0.100000in}}{\pgfqpoint{3.608454in}{2.310000in}}%
\pgfusepath{clip}%
\pgfsetbuttcap%
\pgfsetmiterjoin%
\definecolor{currentfill}{rgb}{0.000000,0.282353,0.858824}%
\pgfsetfillcolor{currentfill}%
\pgfsetlinewidth{0.000000pt}%
\definecolor{currentstroke}{rgb}{0.000000,0.000000,0.000000}%
\pgfsetstrokecolor{currentstroke}%
\pgfsetstrokeopacity{0.000000}%
\pgfsetdash{}{0pt}%
\pgfpathmoveto{\pgfqpoint{2.138316in}{1.510307in}}%
\pgfpathlineto{\pgfqpoint{2.124622in}{1.510289in}}%
\pgfpathlineto{\pgfqpoint{2.124501in}{1.545418in}}%
\pgfpathlineto{\pgfqpoint{2.121931in}{1.545408in}}%
\pgfpathlineto{\pgfqpoint{2.121832in}{1.566001in}}%
\pgfpathlineto{\pgfqpoint{2.156052in}{1.566272in}}%
\pgfpathlineto{\pgfqpoint{2.155774in}{1.593709in}}%
\pgfpathlineto{\pgfqpoint{2.196902in}{1.594193in}}%
\pgfpathlineto{\pgfqpoint{2.210467in}{1.594411in}}%
\pgfpathlineto{\pgfqpoint{2.210987in}{1.567017in}}%
\pgfpathlineto{\pgfqpoint{2.204148in}{1.566920in}}%
\pgfpathlineto{\pgfqpoint{2.204489in}{1.546166in}}%
\pgfpathlineto{\pgfqpoint{2.205803in}{1.538835in}}%
\pgfpathlineto{\pgfqpoint{2.206358in}{1.511376in}}%
\pgfpathlineto{\pgfqpoint{2.192689in}{1.511198in}}%
\pgfpathlineto{\pgfqpoint{2.172375in}{1.510519in}}%
\pgfpathlineto{\pgfqpoint{2.138316in}{1.510307in}}%
\pgfpathclose%
\pgfusepath{fill}%
\end{pgfscope}%
\begin{pgfscope}%
\pgfpathrectangle{\pgfqpoint{0.100000in}{0.100000in}}{\pgfqpoint{3.608454in}{2.310000in}}%
\pgfusepath{clip}%
\pgfsetbuttcap%
\pgfsetmiterjoin%
\definecolor{currentfill}{rgb}{0.000000,0.505882,0.747059}%
\pgfsetfillcolor{currentfill}%
\pgfsetlinewidth{0.000000pt}%
\definecolor{currentstroke}{rgb}{0.000000,0.000000,0.000000}%
\pgfsetstrokecolor{currentstroke}%
\pgfsetstrokeopacity{0.000000}%
\pgfsetdash{}{0pt}%
\pgfpathmoveto{\pgfqpoint{1.782538in}{1.034546in}}%
\pgfpathlineto{\pgfqpoint{1.781300in}{1.012079in}}%
\pgfpathlineto{\pgfqpoint{1.754289in}{1.013713in}}%
\pgfpathlineto{\pgfqpoint{1.746176in}{1.014219in}}%
\pgfpathlineto{\pgfqpoint{1.748391in}{1.048651in}}%
\pgfpathlineto{\pgfqpoint{1.750416in}{1.083136in}}%
\pgfpathlineto{\pgfqpoint{1.785102in}{1.081107in}}%
\pgfpathlineto{\pgfqpoint{1.782538in}{1.034546in}}%
\pgfpathclose%
\pgfusepath{fill}%
\end{pgfscope}%
\begin{pgfscope}%
\pgfpathrectangle{\pgfqpoint{0.100000in}{0.100000in}}{\pgfqpoint{3.608454in}{2.310000in}}%
\pgfusepath{clip}%
\pgfsetbuttcap%
\pgfsetmiterjoin%
\definecolor{currentfill}{rgb}{0.000000,0.486275,0.756863}%
\pgfsetfillcolor{currentfill}%
\pgfsetlinewidth{0.000000pt}%
\definecolor{currentstroke}{rgb}{0.000000,0.000000,0.000000}%
\pgfsetstrokecolor{currentstroke}%
\pgfsetstrokeopacity{0.000000}%
\pgfsetdash{}{0pt}%
\pgfpathmoveto{\pgfqpoint{1.893386in}{1.070793in}}%
\pgfpathlineto{\pgfqpoint{1.873435in}{1.071561in}}%
\pgfpathlineto{\pgfqpoint{1.873684in}{1.092215in}}%
\pgfpathlineto{\pgfqpoint{1.825972in}{1.094304in}}%
\pgfpathlineto{\pgfqpoint{1.826418in}{1.110191in}}%
\pgfpathlineto{\pgfqpoint{1.826923in}{1.122162in}}%
\pgfpathlineto{\pgfqpoint{1.854121in}{1.120697in}}%
\pgfpathlineto{\pgfqpoint{1.855366in}{1.155288in}}%
\pgfpathlineto{\pgfqpoint{1.844616in}{1.170342in}}%
\pgfpathlineto{\pgfqpoint{1.837450in}{1.170858in}}%
\pgfpathlineto{\pgfqpoint{1.830118in}{1.178718in}}%
\pgfpathlineto{\pgfqpoint{1.825437in}{1.188388in}}%
\pgfpathlineto{\pgfqpoint{1.854168in}{1.187060in}}%
\pgfpathlineto{\pgfqpoint{1.895416in}{1.185262in}}%
\pgfpathlineto{\pgfqpoint{1.951284in}{1.183534in}}%
\pgfpathlineto{\pgfqpoint{1.950456in}{1.151489in}}%
\pgfpathlineto{\pgfqpoint{1.949595in}{1.117555in}}%
\pgfpathlineto{\pgfqpoint{1.901714in}{1.119083in}}%
\pgfpathlineto{\pgfqpoint{1.900701in}{1.084287in}}%
\pgfpathlineto{\pgfqpoint{1.893880in}{1.084537in}}%
\pgfpathlineto{\pgfqpoint{1.893386in}{1.070793in}}%
\pgfpathclose%
\pgfusepath{fill}%
\end{pgfscope}%
\begin{pgfscope}%
\pgfpathrectangle{\pgfqpoint{0.100000in}{0.100000in}}{\pgfqpoint{3.608454in}{2.310000in}}%
\pgfusepath{clip}%
\pgfsetbuttcap%
\pgfsetmiterjoin%
\definecolor{currentfill}{rgb}{0.000000,0.678431,0.660784}%
\pgfsetfillcolor{currentfill}%
\pgfsetlinewidth{0.000000pt}%
\definecolor{currentstroke}{rgb}{0.000000,0.000000,0.000000}%
\pgfsetstrokecolor{currentstroke}%
\pgfsetstrokeopacity{0.000000}%
\pgfsetdash{}{0pt}%
\pgfpathmoveto{\pgfqpoint{1.068484in}{1.440713in}}%
\pgfpathlineto{\pgfqpoint{1.064107in}{1.434490in}}%
\pgfpathlineto{\pgfqpoint{1.060054in}{1.435184in}}%
\pgfpathlineto{\pgfqpoint{1.053099in}{1.429145in}}%
\pgfpathlineto{\pgfqpoint{1.051426in}{1.418402in}}%
\pgfpathlineto{\pgfqpoint{1.045136in}{1.415471in}}%
\pgfpathlineto{\pgfqpoint{1.036855in}{1.417471in}}%
\pgfpathlineto{\pgfqpoint{1.031123in}{1.409823in}}%
\pgfpathlineto{\pgfqpoint{1.000345in}{1.416064in}}%
\pgfpathlineto{\pgfqpoint{0.938194in}{1.428823in}}%
\pgfpathlineto{\pgfqpoint{0.939931in}{1.436894in}}%
\pgfpathlineto{\pgfqpoint{0.954290in}{1.503771in}}%
\pgfpathlineto{\pgfqpoint{0.960301in}{1.531851in}}%
\pgfpathlineto{\pgfqpoint{1.034454in}{1.516384in}}%
\pgfpathlineto{\pgfqpoint{1.061605in}{1.511149in}}%
\pgfpathlineto{\pgfqpoint{1.072847in}{1.517633in}}%
\pgfpathlineto{\pgfqpoint{1.077321in}{1.529704in}}%
\pgfpathlineto{\pgfqpoint{1.075863in}{1.537379in}}%
\pgfpathlineto{\pgfqpoint{1.077577in}{1.552756in}}%
\pgfpathlineto{\pgfqpoint{1.087943in}{1.551215in}}%
\pgfpathlineto{\pgfqpoint{1.090523in}{1.546657in}}%
\pgfpathlineto{\pgfqpoint{1.105133in}{1.552914in}}%
\pgfpathlineto{\pgfqpoint{1.116941in}{1.552445in}}%
\pgfpathlineto{\pgfqpoint{1.112401in}{1.545424in}}%
\pgfpathlineto{\pgfqpoint{1.120187in}{1.531417in}}%
\pgfpathlineto{\pgfqpoint{1.129198in}{1.527981in}}%
\pgfpathlineto{\pgfqpoint{1.131819in}{1.519099in}}%
\pgfpathlineto{\pgfqpoint{1.128853in}{1.510727in}}%
\pgfpathlineto{\pgfqpoint{1.137161in}{1.499935in}}%
\pgfpathlineto{\pgfqpoint{1.136590in}{1.496867in}}%
\pgfpathlineto{\pgfqpoint{1.148005in}{1.494799in}}%
\pgfpathlineto{\pgfqpoint{1.148850in}{1.487724in}}%
\pgfpathlineto{\pgfqpoint{1.101981in}{1.496172in}}%
\pgfpathlineto{\pgfqpoint{1.104984in}{1.493265in}}%
\pgfpathlineto{\pgfqpoint{1.095705in}{1.487818in}}%
\pgfpathlineto{\pgfqpoint{1.092637in}{1.469533in}}%
\pgfpathlineto{\pgfqpoint{1.089487in}{1.463907in}}%
\pgfpathlineto{\pgfqpoint{1.078936in}{1.465774in}}%
\pgfpathlineto{\pgfqpoint{1.072686in}{1.463362in}}%
\pgfpathlineto{\pgfqpoint{1.068484in}{1.440713in}}%
\pgfpathclose%
\pgfusepath{fill}%
\end{pgfscope}%
\begin{pgfscope}%
\pgfpathrectangle{\pgfqpoint{0.100000in}{0.100000in}}{\pgfqpoint{3.608454in}{2.310000in}}%
\pgfusepath{clip}%
\pgfsetbuttcap%
\pgfsetmiterjoin%
\definecolor{currentfill}{rgb}{0.000000,0.560784,0.719608}%
\pgfsetfillcolor{currentfill}%
\pgfsetlinewidth{0.000000pt}%
\definecolor{currentstroke}{rgb}{0.000000,0.000000,0.000000}%
\pgfsetstrokecolor{currentstroke}%
\pgfsetstrokeopacity{0.000000}%
\pgfsetdash{}{0pt}%
\pgfpathmoveto{\pgfqpoint{2.831040in}{1.546066in}}%
\pgfpathlineto{\pgfqpoint{2.825402in}{1.579904in}}%
\pgfpathlineto{\pgfqpoint{2.818905in}{1.581252in}}%
\pgfpathlineto{\pgfqpoint{2.833971in}{1.586462in}}%
\pgfpathlineto{\pgfqpoint{2.840470in}{1.580694in}}%
\pgfpathlineto{\pgfqpoint{2.847896in}{1.579132in}}%
\pgfpathlineto{\pgfqpoint{2.860187in}{1.585033in}}%
\pgfpathlineto{\pgfqpoint{2.873522in}{1.593422in}}%
\pgfpathlineto{\pgfqpoint{2.876219in}{1.593055in}}%
\pgfpathlineto{\pgfqpoint{2.877765in}{1.580902in}}%
\pgfpathlineto{\pgfqpoint{2.883315in}{1.581642in}}%
\pgfpathlineto{\pgfqpoint{2.884117in}{1.575727in}}%
\pgfpathlineto{\pgfqpoint{2.878552in}{1.574944in}}%
\pgfpathlineto{\pgfqpoint{2.879290in}{1.569013in}}%
\pgfpathlineto{\pgfqpoint{2.873516in}{1.568184in}}%
\pgfpathlineto{\pgfqpoint{2.868477in}{1.562448in}}%
\pgfpathlineto{\pgfqpoint{2.869199in}{1.556678in}}%
\pgfpathlineto{\pgfqpoint{2.853507in}{1.554639in}}%
\pgfpathlineto{\pgfqpoint{2.854535in}{1.548999in}}%
\pgfpathlineto{\pgfqpoint{2.831040in}{1.546066in}}%
\pgfpathclose%
\pgfusepath{fill}%
\end{pgfscope}%
\begin{pgfscope}%
\pgfpathrectangle{\pgfqpoint{0.100000in}{0.100000in}}{\pgfqpoint{3.608454in}{2.310000in}}%
\pgfusepath{clip}%
\pgfsetbuttcap%
\pgfsetmiterjoin%
\definecolor{currentfill}{rgb}{0.000000,0.611765,0.694118}%
\pgfsetfillcolor{currentfill}%
\pgfsetlinewidth{0.000000pt}%
\definecolor{currentstroke}{rgb}{0.000000,0.000000,0.000000}%
\pgfsetstrokecolor{currentstroke}%
\pgfsetstrokeopacity{0.000000}%
\pgfsetdash{}{0pt}%
\pgfpathmoveto{\pgfqpoint{1.611520in}{1.962062in}}%
\pgfpathlineto{\pgfqpoint{1.606786in}{1.914838in}}%
\pgfpathlineto{\pgfqpoint{1.602057in}{1.917205in}}%
\pgfpathlineto{\pgfqpoint{1.587963in}{1.918755in}}%
\pgfpathlineto{\pgfqpoint{1.588691in}{1.925458in}}%
\pgfpathlineto{\pgfqpoint{1.581763in}{1.926153in}}%
\pgfpathlineto{\pgfqpoint{1.582466in}{1.932625in}}%
\pgfpathlineto{\pgfqpoint{1.562162in}{1.934829in}}%
\pgfpathlineto{\pgfqpoint{1.564057in}{1.962319in}}%
\pgfpathlineto{\pgfqpoint{1.573489in}{1.961301in}}%
\pgfpathlineto{\pgfqpoint{1.574985in}{1.975097in}}%
\pgfpathlineto{\pgfqpoint{1.582076in}{1.976639in}}%
\pgfpathlineto{\pgfqpoint{1.592373in}{1.975555in}}%
\pgfpathlineto{\pgfqpoint{1.595495in}{1.971747in}}%
\pgfpathlineto{\pgfqpoint{1.612308in}{1.970040in}}%
\pgfpathlineto{\pgfqpoint{1.611520in}{1.962062in}}%
\pgfpathclose%
\pgfusepath{fill}%
\end{pgfscope}%
\begin{pgfscope}%
\pgfpathrectangle{\pgfqpoint{0.100000in}{0.100000in}}{\pgfqpoint{3.608454in}{2.310000in}}%
\pgfusepath{clip}%
\pgfsetbuttcap%
\pgfsetmiterjoin%
\definecolor{currentfill}{rgb}{0.000000,0.560784,0.719608}%
\pgfsetfillcolor{currentfill}%
\pgfsetlinewidth{0.000000pt}%
\definecolor{currentstroke}{rgb}{0.000000,0.000000,0.000000}%
\pgfsetstrokecolor{currentstroke}%
\pgfsetstrokeopacity{0.000000}%
\pgfsetdash{}{0pt}%
\pgfpathmoveto{\pgfqpoint{3.298349in}{1.383037in}}%
\pgfpathlineto{\pgfqpoint{3.291222in}{1.379099in}}%
\pgfpathlineto{\pgfqpoint{3.288344in}{1.383217in}}%
\pgfpathlineto{\pgfqpoint{3.291045in}{1.389057in}}%
\pgfpathlineto{\pgfqpoint{3.289366in}{1.396494in}}%
\pgfpathlineto{\pgfqpoint{3.281967in}{1.394943in}}%
\pgfpathlineto{\pgfqpoint{3.282347in}{1.403895in}}%
\pgfpathlineto{\pgfqpoint{3.281272in}{1.408524in}}%
\pgfpathlineto{\pgfqpoint{3.284538in}{1.423341in}}%
\pgfpathlineto{\pgfqpoint{3.290599in}{1.431089in}}%
\pgfpathlineto{\pgfqpoint{3.274847in}{1.487194in}}%
\pgfpathlineto{\pgfqpoint{3.281434in}{1.488221in}}%
\pgfpathlineto{\pgfqpoint{3.288460in}{1.495799in}}%
\pgfpathlineto{\pgfqpoint{3.293773in}{1.492637in}}%
\pgfpathlineto{\pgfqpoint{3.298424in}{1.484780in}}%
\pgfpathlineto{\pgfqpoint{3.300159in}{1.474127in}}%
\pgfpathlineto{\pgfqpoint{3.304555in}{1.471386in}}%
\pgfpathlineto{\pgfqpoint{3.308496in}{1.463629in}}%
\pgfpathlineto{\pgfqpoint{3.317538in}{1.456649in}}%
\pgfpathlineto{\pgfqpoint{3.324360in}{1.455596in}}%
\pgfpathlineto{\pgfqpoint{3.331882in}{1.431286in}}%
\pgfpathlineto{\pgfqpoint{3.329200in}{1.401923in}}%
\pgfpathlineto{\pgfqpoint{3.327417in}{1.395926in}}%
\pgfpathlineto{\pgfqpoint{3.304739in}{1.388327in}}%
\pgfpathlineto{\pgfqpoint{3.298349in}{1.383037in}}%
\pgfpathclose%
\pgfusepath{fill}%
\end{pgfscope}%
\begin{pgfscope}%
\pgfpathrectangle{\pgfqpoint{0.100000in}{0.100000in}}{\pgfqpoint{3.608454in}{2.310000in}}%
\pgfusepath{clip}%
\pgfsetbuttcap%
\pgfsetmiterjoin%
\definecolor{currentfill}{rgb}{0.000000,0.443137,0.778431}%
\pgfsetfillcolor{currentfill}%
\pgfsetlinewidth{0.000000pt}%
\definecolor{currentstroke}{rgb}{0.000000,0.000000,0.000000}%
\pgfsetstrokecolor{currentstroke}%
\pgfsetstrokeopacity{0.000000}%
\pgfsetdash{}{0pt}%
\pgfpathmoveto{\pgfqpoint{2.115090in}{1.593467in}}%
\pgfpathlineto{\pgfqpoint{2.067449in}{1.593735in}}%
\pgfpathlineto{\pgfqpoint{2.067679in}{1.614586in}}%
\pgfpathlineto{\pgfqpoint{2.063540in}{1.621479in}}%
\pgfpathlineto{\pgfqpoint{2.056624in}{1.621513in}}%
\pgfpathlineto{\pgfqpoint{2.056914in}{1.649093in}}%
\pgfpathlineto{\pgfqpoint{2.084271in}{1.648958in}}%
\pgfpathlineto{\pgfqpoint{2.111776in}{1.648912in}}%
\pgfpathlineto{\pgfqpoint{2.111772in}{1.614418in}}%
\pgfpathlineto{\pgfqpoint{2.115152in}{1.614399in}}%
\pgfpathlineto{\pgfqpoint{2.115090in}{1.593467in}}%
\pgfpathclose%
\pgfusepath{fill}%
\end{pgfscope}%
\begin{pgfscope}%
\pgfpathrectangle{\pgfqpoint{0.100000in}{0.100000in}}{\pgfqpoint{3.608454in}{2.310000in}}%
\pgfusepath{clip}%
\pgfsetbuttcap%
\pgfsetmiterjoin%
\definecolor{currentfill}{rgb}{0.000000,0.698039,0.650980}%
\pgfsetfillcolor{currentfill}%
\pgfsetlinewidth{0.000000pt}%
\definecolor{currentstroke}{rgb}{0.000000,0.000000,0.000000}%
\pgfsetstrokecolor{currentstroke}%
\pgfsetstrokeopacity{0.000000}%
\pgfsetdash{}{0pt}%
\pgfpathmoveto{\pgfqpoint{1.194163in}{1.451751in}}%
\pgfpathlineto{\pgfqpoint{1.249416in}{1.442841in}}%
\pgfpathlineto{\pgfqpoint{1.253203in}{1.445105in}}%
\pgfpathlineto{\pgfqpoint{1.251645in}{1.434868in}}%
\pgfpathlineto{\pgfqpoint{1.240533in}{1.367269in}}%
\pgfpathlineto{\pgfqpoint{1.181575in}{1.376785in}}%
\pgfpathlineto{\pgfqpoint{1.183363in}{1.379999in}}%
\pgfpathlineto{\pgfqpoint{1.179215in}{1.395609in}}%
\pgfpathlineto{\pgfqpoint{1.181137in}{1.398548in}}%
\pgfpathlineto{\pgfqpoint{1.177987in}{1.406555in}}%
\pgfpathlineto{\pgfqpoint{1.182094in}{1.423685in}}%
\pgfpathlineto{\pgfqpoint{1.189342in}{1.433929in}}%
\pgfpathlineto{\pgfqpoint{1.188609in}{1.438112in}}%
\pgfpathlineto{\pgfqpoint{1.192855in}{1.443180in}}%
\pgfpathlineto{\pgfqpoint{1.194163in}{1.451751in}}%
\pgfpathclose%
\pgfusepath{fill}%
\end{pgfscope}%
\begin{pgfscope}%
\pgfpathrectangle{\pgfqpoint{0.100000in}{0.100000in}}{\pgfqpoint{3.608454in}{2.310000in}}%
\pgfusepath{clip}%
\pgfsetbuttcap%
\pgfsetmiterjoin%
\definecolor{currentfill}{rgb}{0.000000,0.564706,0.717647}%
\pgfsetfillcolor{currentfill}%
\pgfsetlinewidth{0.000000pt}%
\definecolor{currentstroke}{rgb}{0.000000,0.000000,0.000000}%
\pgfsetstrokecolor{currentstroke}%
\pgfsetstrokeopacity{0.000000}%
\pgfsetdash{}{0pt}%
\pgfpathmoveto{\pgfqpoint{2.780892in}{1.025876in}}%
\pgfpathlineto{\pgfqpoint{2.779095in}{1.040030in}}%
\pgfpathlineto{\pgfqpoint{2.782890in}{1.048115in}}%
\pgfpathlineto{\pgfqpoint{2.780160in}{1.050381in}}%
\pgfpathlineto{\pgfqpoint{2.779057in}{1.060719in}}%
\pgfpathlineto{\pgfqpoint{2.798375in}{1.062974in}}%
\pgfpathlineto{\pgfqpoint{2.808303in}{1.061863in}}%
\pgfpathlineto{\pgfqpoint{2.814885in}{1.050035in}}%
\pgfpathlineto{\pgfqpoint{2.812171in}{1.037511in}}%
\pgfpathlineto{\pgfqpoint{2.810594in}{1.033632in}}%
\pgfpathlineto{\pgfqpoint{2.801686in}{1.031063in}}%
\pgfpathlineto{\pgfqpoint{2.799254in}{1.027968in}}%
\pgfpathlineto{\pgfqpoint{2.780892in}{1.025876in}}%
\pgfpathclose%
\pgfusepath{fill}%
\end{pgfscope}%
\begin{pgfscope}%
\pgfpathrectangle{\pgfqpoint{0.100000in}{0.100000in}}{\pgfqpoint{3.608454in}{2.310000in}}%
\pgfusepath{clip}%
\pgfsetbuttcap%
\pgfsetmiterjoin%
\definecolor{currentfill}{rgb}{0.000000,0.917647,0.541176}%
\pgfsetfillcolor{currentfill}%
\pgfsetlinewidth{0.000000pt}%
\definecolor{currentstroke}{rgb}{0.000000,0.000000,0.000000}%
\pgfsetstrokecolor{currentstroke}%
\pgfsetstrokeopacity{0.000000}%
\pgfsetdash{}{0pt}%
\pgfpathmoveto{\pgfqpoint{0.427334in}{1.927274in}}%
\pgfpathlineto{\pgfqpoint{0.440684in}{1.949326in}}%
\pgfpathlineto{\pgfqpoint{0.456163in}{1.968351in}}%
\pgfpathlineto{\pgfqpoint{0.467729in}{1.989714in}}%
\pgfpathlineto{\pgfqpoint{0.479271in}{1.985850in}}%
\pgfpathlineto{\pgfqpoint{0.487714in}{1.989945in}}%
\pgfpathlineto{\pgfqpoint{0.498412in}{1.984367in}}%
\pgfpathlineto{\pgfqpoint{0.501091in}{1.975934in}}%
\pgfpathlineto{\pgfqpoint{0.509770in}{1.969016in}}%
\pgfpathlineto{\pgfqpoint{0.521206in}{1.965301in}}%
\pgfpathlineto{\pgfqpoint{0.517042in}{1.952171in}}%
\pgfpathlineto{\pgfqpoint{0.517068in}{1.946672in}}%
\pgfpathlineto{\pgfqpoint{0.537209in}{1.940659in}}%
\pgfpathlineto{\pgfqpoint{0.534705in}{1.932520in}}%
\pgfpathlineto{\pgfqpoint{0.568235in}{1.922352in}}%
\pgfpathlineto{\pgfqpoint{0.572489in}{1.914545in}}%
\pgfpathlineto{\pgfqpoint{0.572519in}{1.906519in}}%
\pgfpathlineto{\pgfqpoint{0.565189in}{1.899442in}}%
\pgfpathlineto{\pgfqpoint{0.562002in}{1.893971in}}%
\pgfpathlineto{\pgfqpoint{0.551234in}{1.896670in}}%
\pgfpathlineto{\pgfqpoint{0.549594in}{1.891292in}}%
\pgfpathlineto{\pgfqpoint{0.543027in}{1.893318in}}%
\pgfpathlineto{\pgfqpoint{0.538396in}{1.890064in}}%
\pgfpathlineto{\pgfqpoint{0.532832in}{1.891746in}}%
\pgfpathlineto{\pgfqpoint{0.516046in}{1.883757in}}%
\pgfpathlineto{\pgfqpoint{0.503506in}{1.885466in}}%
\pgfpathlineto{\pgfqpoint{0.495724in}{1.888283in}}%
\pgfpathlineto{\pgfqpoint{0.491499in}{1.886385in}}%
\pgfpathlineto{\pgfqpoint{0.483261in}{1.890283in}}%
\pgfpathlineto{\pgfqpoint{0.478228in}{1.889504in}}%
\pgfpathlineto{\pgfqpoint{0.465159in}{1.900474in}}%
\pgfpathlineto{\pgfqpoint{0.456477in}{1.902564in}}%
\pgfpathlineto{\pgfqpoint{0.447212in}{1.898187in}}%
\pgfpathlineto{\pgfqpoint{0.438998in}{1.899672in}}%
\pgfpathlineto{\pgfqpoint{0.443285in}{1.914029in}}%
\pgfpathlineto{\pgfqpoint{0.439586in}{1.923181in}}%
\pgfpathlineto{\pgfqpoint{0.427334in}{1.927274in}}%
\pgfpathclose%
\pgfusepath{fill}%
\end{pgfscope}%
\begin{pgfscope}%
\pgfpathrectangle{\pgfqpoint{0.100000in}{0.100000in}}{\pgfqpoint{3.608454in}{2.310000in}}%
\pgfusepath{clip}%
\pgfsetbuttcap%
\pgfsetmiterjoin%
\definecolor{currentfill}{rgb}{0.000000,0.666667,0.666667}%
\pgfsetfillcolor{currentfill}%
\pgfsetlinewidth{0.000000pt}%
\definecolor{currentstroke}{rgb}{0.000000,0.000000,0.000000}%
\pgfsetstrokecolor{currentstroke}%
\pgfsetstrokeopacity{0.000000}%
\pgfsetdash{}{0pt}%
\pgfpathmoveto{\pgfqpoint{3.106445in}{0.607053in}}%
\pgfpathlineto{\pgfqpoint{3.090036in}{0.604447in}}%
\pgfpathlineto{\pgfqpoint{3.090875in}{0.590177in}}%
\pgfpathlineto{\pgfqpoint{3.080990in}{0.605844in}}%
\pgfpathlineto{\pgfqpoint{3.078312in}{0.602797in}}%
\pgfpathlineto{\pgfqpoint{3.071577in}{0.605034in}}%
\pgfpathlineto{\pgfqpoint{3.065010in}{0.604224in}}%
\pgfpathlineto{\pgfqpoint{3.060692in}{0.615927in}}%
\pgfpathlineto{\pgfqpoint{3.049410in}{0.623089in}}%
\pgfpathlineto{\pgfqpoint{3.046213in}{0.629170in}}%
\pgfpathlineto{\pgfqpoint{3.040376in}{0.630406in}}%
\pgfpathlineto{\pgfqpoint{3.037437in}{0.635958in}}%
\pgfpathlineto{\pgfqpoint{3.029457in}{0.642400in}}%
\pgfpathlineto{\pgfqpoint{3.025702in}{0.651277in}}%
\pgfpathlineto{\pgfqpoint{3.020704in}{0.653264in}}%
\pgfpathlineto{\pgfqpoint{3.006579in}{0.647225in}}%
\pgfpathlineto{\pgfqpoint{3.004229in}{0.666876in}}%
\pgfpathlineto{\pgfqpoint{3.011416in}{0.670241in}}%
\pgfpathlineto{\pgfqpoint{3.019158in}{0.678567in}}%
\pgfpathlineto{\pgfqpoint{3.035034in}{0.681162in}}%
\pgfpathlineto{\pgfqpoint{3.039796in}{0.675377in}}%
\pgfpathlineto{\pgfqpoint{3.041523in}{0.664546in}}%
\pgfpathlineto{\pgfqpoint{3.055227in}{0.666862in}}%
\pgfpathlineto{\pgfqpoint{3.062330in}{0.671592in}}%
\pgfpathlineto{\pgfqpoint{3.078208in}{0.644633in}}%
\pgfpathlineto{\pgfqpoint{3.086273in}{0.632061in}}%
\pgfpathlineto{\pgfqpoint{3.106445in}{0.607053in}}%
\pgfpathclose%
\pgfusepath{fill}%
\end{pgfscope}%
\begin{pgfscope}%
\pgfpathrectangle{\pgfqpoint{0.100000in}{0.100000in}}{\pgfqpoint{3.608454in}{2.310000in}}%
\pgfusepath{clip}%
\pgfsetbuttcap%
\pgfsetmiterjoin%
\definecolor{currentfill}{rgb}{0.000000,0.549020,0.725490}%
\pgfsetfillcolor{currentfill}%
\pgfsetlinewidth{0.000000pt}%
\definecolor{currentstroke}{rgb}{0.000000,0.000000,0.000000}%
\pgfsetstrokecolor{currentstroke}%
\pgfsetstrokeopacity{0.000000}%
\pgfsetdash{}{0pt}%
\pgfpathmoveto{\pgfqpoint{0.914070in}{1.089859in}}%
\pgfpathlineto{\pgfqpoint{0.900636in}{1.091652in}}%
\pgfpathlineto{\pgfqpoint{0.895182in}{1.086583in}}%
\pgfpathlineto{\pgfqpoint{0.868397in}{1.094701in}}%
\pgfpathlineto{\pgfqpoint{0.862231in}{1.099514in}}%
\pgfpathlineto{\pgfqpoint{0.851945in}{1.113704in}}%
\pgfpathlineto{\pgfqpoint{0.848041in}{1.126325in}}%
\pgfpathlineto{\pgfqpoint{0.847819in}{1.135796in}}%
\pgfpathlineto{\pgfqpoint{0.843543in}{1.141162in}}%
\pgfpathlineto{\pgfqpoint{0.840274in}{1.151795in}}%
\pgfpathlineto{\pgfqpoint{0.842429in}{1.160488in}}%
\pgfpathlineto{\pgfqpoint{0.792611in}{1.237224in}}%
\pgfpathlineto{\pgfqpoint{0.780428in}{1.255922in}}%
\pgfpathlineto{\pgfqpoint{0.720005in}{1.349216in}}%
\pgfpathlineto{\pgfqpoint{0.688870in}{1.397366in}}%
\pgfpathlineto{\pgfqpoint{0.661400in}{1.439746in}}%
\pgfpathlineto{\pgfqpoint{0.665974in}{1.438337in}}%
\pgfpathlineto{\pgfqpoint{0.717194in}{1.472445in}}%
\pgfpathlineto{\pgfqpoint{0.695807in}{1.514308in}}%
\pgfpathlineto{\pgfqpoint{0.697419in}{1.520493in}}%
\pgfpathlineto{\pgfqpoint{0.718561in}{1.521037in}}%
\pgfpathlineto{\pgfqpoint{0.724276in}{1.521184in}}%
\pgfpathlineto{\pgfqpoint{0.752002in}{1.519867in}}%
\pgfpathlineto{\pgfqpoint{0.795453in}{1.508995in}}%
\pgfpathlineto{\pgfqpoint{0.836825in}{1.499170in}}%
\pgfpathlineto{\pgfqpoint{0.882599in}{1.449402in}}%
\pgfpathlineto{\pgfqpoint{0.939931in}{1.436894in}}%
\pgfpathlineto{\pgfqpoint{0.938194in}{1.428823in}}%
\pgfpathlineto{\pgfqpoint{0.925741in}{1.369767in}}%
\pgfpathlineto{\pgfqpoint{0.912295in}{1.307222in}}%
\pgfpathlineto{\pgfqpoint{0.983512in}{1.292540in}}%
\pgfpathlineto{\pgfqpoint{1.005762in}{1.288169in}}%
\pgfpathlineto{\pgfqpoint{0.994707in}{1.264046in}}%
\pgfpathlineto{\pgfqpoint{0.991172in}{1.251855in}}%
\pgfpathlineto{\pgfqpoint{0.991491in}{1.243095in}}%
\pgfpathlineto{\pgfqpoint{0.986930in}{1.238736in}}%
\pgfpathlineto{\pgfqpoint{0.978736in}{1.235692in}}%
\pgfpathlineto{\pgfqpoint{0.961661in}{1.233095in}}%
\pgfpathlineto{\pgfqpoint{0.951426in}{1.230336in}}%
\pgfpathlineto{\pgfqpoint{0.949023in}{1.226105in}}%
\pgfpathlineto{\pgfqpoint{0.943617in}{1.227923in}}%
\pgfpathlineto{\pgfqpoint{0.940117in}{1.223821in}}%
\pgfpathlineto{\pgfqpoint{0.941162in}{1.217180in}}%
\pgfpathlineto{\pgfqpoint{0.937616in}{1.205156in}}%
\pgfpathlineto{\pgfqpoint{0.933245in}{1.183825in}}%
\pgfpathlineto{\pgfqpoint{0.914070in}{1.089859in}}%
\pgfpathclose%
\pgfusepath{fill}%
\end{pgfscope}%
\begin{pgfscope}%
\pgfpathrectangle{\pgfqpoint{0.100000in}{0.100000in}}{\pgfqpoint{3.608454in}{2.310000in}}%
\pgfusepath{clip}%
\pgfsetbuttcap%
\pgfsetmiterjoin%
\definecolor{currentfill}{rgb}{0.000000,0.552941,0.723529}%
\pgfsetfillcolor{currentfill}%
\pgfsetlinewidth{0.000000pt}%
\definecolor{currentstroke}{rgb}{0.000000,0.000000,0.000000}%
\pgfsetstrokecolor{currentstroke}%
\pgfsetstrokeopacity{0.000000}%
\pgfsetdash{}{0pt}%
\pgfpathmoveto{\pgfqpoint{2.956251in}{1.497406in}}%
\pgfpathlineto{\pgfqpoint{2.935644in}{1.495065in}}%
\pgfpathlineto{\pgfqpoint{2.928425in}{1.497499in}}%
\pgfpathlineto{\pgfqpoint{2.908656in}{1.495198in}}%
\pgfpathlineto{\pgfqpoint{2.909284in}{1.489578in}}%
\pgfpathlineto{\pgfqpoint{2.878220in}{1.486351in}}%
\pgfpathlineto{\pgfqpoint{2.876912in}{1.491928in}}%
\pgfpathlineto{\pgfqpoint{2.874225in}{1.518225in}}%
\pgfpathlineto{\pgfqpoint{2.871819in}{1.517489in}}%
\pgfpathlineto{\pgfqpoint{2.870685in}{1.525264in}}%
\pgfpathlineto{\pgfqpoint{2.876286in}{1.526111in}}%
\pgfpathlineto{\pgfqpoint{2.872479in}{1.551436in}}%
\pgfpathlineto{\pgfqpoint{2.901003in}{1.555251in}}%
\pgfpathlineto{\pgfqpoint{2.901826in}{1.549489in}}%
\pgfpathlineto{\pgfqpoint{2.915375in}{1.550896in}}%
\pgfpathlineto{\pgfqpoint{2.914636in}{1.557327in}}%
\pgfpathlineto{\pgfqpoint{2.934085in}{1.560268in}}%
\pgfpathlineto{\pgfqpoint{2.937196in}{1.539921in}}%
\pgfpathlineto{\pgfqpoint{2.947280in}{1.541412in}}%
\pgfpathlineto{\pgfqpoint{2.948242in}{1.534882in}}%
\pgfpathlineto{\pgfqpoint{2.952109in}{1.531961in}}%
\pgfpathlineto{\pgfqpoint{2.949140in}{1.528048in}}%
\pgfpathlineto{\pgfqpoint{2.949522in}{1.517575in}}%
\pgfpathlineto{\pgfqpoint{2.954062in}{1.518144in}}%
\pgfpathlineto{\pgfqpoint{2.956251in}{1.497406in}}%
\pgfpathclose%
\pgfusepath{fill}%
\end{pgfscope}%
\begin{pgfscope}%
\pgfpathrectangle{\pgfqpoint{0.100000in}{0.100000in}}{\pgfqpoint{3.608454in}{2.310000in}}%
\pgfusepath{clip}%
\pgfsetbuttcap%
\pgfsetmiterjoin%
\definecolor{currentfill}{rgb}{0.000000,0.658824,0.670588}%
\pgfsetfillcolor{currentfill}%
\pgfsetlinewidth{0.000000pt}%
\definecolor{currentstroke}{rgb}{0.000000,0.000000,0.000000}%
\pgfsetstrokecolor{currentstroke}%
\pgfsetstrokeopacity{0.000000}%
\pgfsetdash{}{0pt}%
\pgfpathmoveto{\pgfqpoint{2.996900in}{1.203053in}}%
\pgfpathlineto{\pgfqpoint{3.004020in}{1.198138in}}%
\pgfpathlineto{\pgfqpoint{3.006498in}{1.176840in}}%
\pgfpathlineto{\pgfqpoint{2.993174in}{1.172573in}}%
\pgfpathlineto{\pgfqpoint{2.987201in}{1.172347in}}%
\pgfpathlineto{\pgfqpoint{2.978982in}{1.167861in}}%
\pgfpathlineto{\pgfqpoint{2.972912in}{1.172989in}}%
\pgfpathlineto{\pgfqpoint{2.963826in}{1.175254in}}%
\pgfpathlineto{\pgfqpoint{2.968126in}{1.182900in}}%
\pgfpathlineto{\pgfqpoint{2.955269in}{1.192553in}}%
\pgfpathlineto{\pgfqpoint{2.948361in}{1.195220in}}%
\pgfpathlineto{\pgfqpoint{2.950303in}{1.201290in}}%
\pgfpathlineto{\pgfqpoint{2.950033in}{1.210864in}}%
\pgfpathlineto{\pgfqpoint{2.970559in}{1.212860in}}%
\pgfpathlineto{\pgfqpoint{2.970773in}{1.210739in}}%
\pgfpathlineto{\pgfqpoint{2.980945in}{1.198576in}}%
\pgfpathlineto{\pgfqpoint{2.987922in}{1.204284in}}%
\pgfpathlineto{\pgfqpoint{2.996900in}{1.203053in}}%
\pgfpathclose%
\pgfusepath{fill}%
\end{pgfscope}%
\begin{pgfscope}%
\pgfpathrectangle{\pgfqpoint{0.100000in}{0.100000in}}{\pgfqpoint{3.608454in}{2.310000in}}%
\pgfusepath{clip}%
\pgfsetbuttcap%
\pgfsetmiterjoin%
\definecolor{currentfill}{rgb}{0.000000,0.364706,0.817647}%
\pgfsetfillcolor{currentfill}%
\pgfsetlinewidth{0.000000pt}%
\definecolor{currentstroke}{rgb}{0.000000,0.000000,0.000000}%
\pgfsetstrokecolor{currentstroke}%
\pgfsetstrokeopacity{0.000000}%
\pgfsetdash{}{0pt}%
\pgfpathmoveto{\pgfqpoint{1.655822in}{1.596346in}}%
\pgfpathlineto{\pgfqpoint{1.691656in}{1.593744in}}%
\pgfpathlineto{\pgfqpoint{1.689939in}{1.572676in}}%
\pgfpathlineto{\pgfqpoint{1.694791in}{1.572316in}}%
\pgfpathlineto{\pgfqpoint{1.692731in}{1.544894in}}%
\pgfpathlineto{\pgfqpoint{1.688556in}{1.545253in}}%
\pgfpathlineto{\pgfqpoint{1.687474in}{1.531563in}}%
\pgfpathlineto{\pgfqpoint{1.654575in}{1.534236in}}%
\pgfpathlineto{\pgfqpoint{1.655781in}{1.547900in}}%
\pgfpathlineto{\pgfqpoint{1.654457in}{1.551499in}}%
\pgfpathlineto{\pgfqpoint{1.656181in}{1.573226in}}%
\pgfpathlineto{\pgfqpoint{1.654522in}{1.581965in}}%
\pgfpathlineto{\pgfqpoint{1.655822in}{1.596346in}}%
\pgfpathclose%
\pgfusepath{fill}%
\end{pgfscope}%
\begin{pgfscope}%
\pgfpathrectangle{\pgfqpoint{0.100000in}{0.100000in}}{\pgfqpoint{3.608454in}{2.310000in}}%
\pgfusepath{clip}%
\pgfsetbuttcap%
\pgfsetmiterjoin%
\definecolor{currentfill}{rgb}{0.000000,0.615686,0.692157}%
\pgfsetfillcolor{currentfill}%
\pgfsetlinewidth{0.000000pt}%
\definecolor{currentstroke}{rgb}{0.000000,0.000000,0.000000}%
\pgfsetstrokecolor{currentstroke}%
\pgfsetstrokeopacity{0.000000}%
\pgfsetdash{}{0pt}%
\pgfpathmoveto{\pgfqpoint{2.722959in}{1.184982in}}%
\pgfpathlineto{\pgfqpoint{2.722324in}{1.177007in}}%
\pgfpathlineto{\pgfqpoint{2.713741in}{1.174022in}}%
\pgfpathlineto{\pgfqpoint{2.710970in}{1.165941in}}%
\pgfpathlineto{\pgfqpoint{2.705300in}{1.170782in}}%
\pgfpathlineto{\pgfqpoint{2.696790in}{1.173951in}}%
\pgfpathlineto{\pgfqpoint{2.690027in}{1.171438in}}%
\pgfpathlineto{\pgfqpoint{2.690217in}{1.180528in}}%
\pgfpathlineto{\pgfqpoint{2.678723in}{1.180579in}}%
\pgfpathlineto{\pgfqpoint{2.677443in}{1.189404in}}%
\pgfpathlineto{\pgfqpoint{2.664358in}{1.203424in}}%
\pgfpathlineto{\pgfqpoint{2.669252in}{1.213390in}}%
\pgfpathlineto{\pgfqpoint{2.669571in}{1.226318in}}%
\pgfpathlineto{\pgfqpoint{2.661850in}{1.235103in}}%
\pgfpathlineto{\pgfqpoint{2.665870in}{1.236725in}}%
\pgfpathlineto{\pgfqpoint{2.667849in}{1.244767in}}%
\pgfpathlineto{\pgfqpoint{2.692446in}{1.245040in}}%
\pgfpathlineto{\pgfqpoint{2.690908in}{1.235308in}}%
\pgfpathlineto{\pgfqpoint{2.692552in}{1.225974in}}%
\pgfpathlineto{\pgfqpoint{2.696385in}{1.221833in}}%
\pgfpathlineto{\pgfqpoint{2.703168in}{1.221287in}}%
\pgfpathlineto{\pgfqpoint{2.705489in}{1.213403in}}%
\pgfpathlineto{\pgfqpoint{2.709289in}{1.208321in}}%
\pgfpathlineto{\pgfqpoint{2.723224in}{1.208766in}}%
\pgfpathlineto{\pgfqpoint{2.724860in}{1.203323in}}%
\pgfpathlineto{\pgfqpoint{2.722959in}{1.184982in}}%
\pgfpathclose%
\pgfusepath{fill}%
\end{pgfscope}%
\begin{pgfscope}%
\pgfpathrectangle{\pgfqpoint{0.100000in}{0.100000in}}{\pgfqpoint{3.608454in}{2.310000in}}%
\pgfusepath{clip}%
\pgfsetbuttcap%
\pgfsetmiterjoin%
\definecolor{currentfill}{rgb}{0.000000,0.505882,0.747059}%
\pgfsetfillcolor{currentfill}%
\pgfsetlinewidth{0.000000pt}%
\definecolor{currentstroke}{rgb}{0.000000,0.000000,0.000000}%
\pgfsetstrokecolor{currentstroke}%
\pgfsetstrokeopacity{0.000000}%
\pgfsetdash{}{0pt}%
\pgfpathmoveto{\pgfqpoint{3.206484in}{1.172807in}}%
\pgfpathlineto{\pgfqpoint{3.191049in}{1.170900in}}%
\pgfpathlineto{\pgfqpoint{3.186933in}{1.171546in}}%
\pgfpathlineto{\pgfqpoint{3.181671in}{1.180528in}}%
\pgfpathlineto{\pgfqpoint{3.175634in}{1.186715in}}%
\pgfpathlineto{\pgfqpoint{3.185638in}{1.219606in}}%
\pgfpathlineto{\pgfqpoint{3.195538in}{1.217166in}}%
\pgfpathlineto{\pgfqpoint{3.206943in}{1.219779in}}%
\pgfpathlineto{\pgfqpoint{3.216999in}{1.216602in}}%
\pgfpathlineto{\pgfqpoint{3.218220in}{1.212547in}}%
\pgfpathlineto{\pgfqpoint{3.226514in}{1.211542in}}%
\pgfpathlineto{\pgfqpoint{3.231820in}{1.204552in}}%
\pgfpathlineto{\pgfqpoint{3.232624in}{1.197760in}}%
\pgfpathlineto{\pgfqpoint{3.229923in}{1.198333in}}%
\pgfpathlineto{\pgfqpoint{3.206484in}{1.172807in}}%
\pgfpathclose%
\pgfusepath{fill}%
\end{pgfscope}%
\begin{pgfscope}%
\pgfpathrectangle{\pgfqpoint{0.100000in}{0.100000in}}{\pgfqpoint{3.608454in}{2.310000in}}%
\pgfusepath{clip}%
\pgfsetbuttcap%
\pgfsetmiterjoin%
\definecolor{currentfill}{rgb}{0.000000,0.568627,0.715686}%
\pgfsetfillcolor{currentfill}%
\pgfsetlinewidth{0.000000pt}%
\definecolor{currentstroke}{rgb}{0.000000,0.000000,0.000000}%
\pgfsetstrokecolor{currentstroke}%
\pgfsetstrokeopacity{0.000000}%
\pgfsetdash{}{0pt}%
\pgfpathmoveto{\pgfqpoint{1.051274in}{1.732636in}}%
\pgfpathlineto{\pgfqpoint{1.054206in}{1.729137in}}%
\pgfpathlineto{\pgfqpoint{1.064066in}{1.727041in}}%
\pgfpathlineto{\pgfqpoint{1.060120in}{1.706792in}}%
\pgfpathlineto{\pgfqpoint{1.054992in}{1.681239in}}%
\pgfpathlineto{\pgfqpoint{1.004416in}{1.690817in}}%
\pgfpathlineto{\pgfqpoint{0.994964in}{1.693281in}}%
\pgfpathlineto{\pgfqpoint{0.981161in}{1.696275in}}%
\pgfpathlineto{\pgfqpoint{0.988011in}{1.729060in}}%
\pgfpathlineto{\pgfqpoint{1.000366in}{1.726400in}}%
\pgfpathlineto{\pgfqpoint{1.002092in}{1.733915in}}%
\pgfpathlineto{\pgfqpoint{1.010156in}{1.733884in}}%
\pgfpathlineto{\pgfqpoint{1.013857in}{1.751700in}}%
\pgfpathlineto{\pgfqpoint{1.023421in}{1.749624in}}%
\pgfpathlineto{\pgfqpoint{1.027689in}{1.755650in}}%
\pgfpathlineto{\pgfqpoint{1.033321in}{1.782826in}}%
\pgfpathlineto{\pgfqpoint{1.050343in}{1.779280in}}%
\pgfpathlineto{\pgfqpoint{1.044792in}{1.752067in}}%
\pgfpathlineto{\pgfqpoint{1.041428in}{1.752757in}}%
\pgfpathlineto{\pgfqpoint{1.038549in}{1.738796in}}%
\pgfpathlineto{\pgfqpoint{1.044474in}{1.739072in}}%
\pgfpathlineto{\pgfqpoint{1.051274in}{1.732636in}}%
\pgfpathclose%
\pgfusepath{fill}%
\end{pgfscope}%
\begin{pgfscope}%
\pgfpathrectangle{\pgfqpoint{0.100000in}{0.100000in}}{\pgfqpoint{3.608454in}{2.310000in}}%
\pgfusepath{clip}%
\pgfsetbuttcap%
\pgfsetmiterjoin%
\definecolor{currentfill}{rgb}{0.000000,0.443137,0.778431}%
\pgfsetfillcolor{currentfill}%
\pgfsetlinewidth{0.000000pt}%
\definecolor{currentstroke}{rgb}{0.000000,0.000000,0.000000}%
\pgfsetstrokecolor{currentstroke}%
\pgfsetstrokeopacity{0.000000}%
\pgfsetdash{}{0pt}%
\pgfpathmoveto{\pgfqpoint{2.723904in}{1.022807in}}%
\pgfpathlineto{\pgfqpoint{2.715608in}{1.053498in}}%
\pgfpathlineto{\pgfqpoint{2.698781in}{1.052062in}}%
\pgfpathlineto{\pgfqpoint{2.696968in}{1.070359in}}%
\pgfpathlineto{\pgfqpoint{2.693282in}{1.075443in}}%
\pgfpathlineto{\pgfqpoint{2.694900in}{1.080379in}}%
\pgfpathlineto{\pgfqpoint{2.693772in}{1.094220in}}%
\pgfpathlineto{\pgfqpoint{2.697680in}{1.093767in}}%
\pgfpathlineto{\pgfqpoint{2.714180in}{1.097018in}}%
\pgfpathlineto{\pgfqpoint{2.722462in}{1.100629in}}%
\pgfpathlineto{\pgfqpoint{2.724406in}{1.094850in}}%
\pgfpathlineto{\pgfqpoint{2.736951in}{1.085202in}}%
\pgfpathlineto{\pgfqpoint{2.739787in}{1.093594in}}%
\pgfpathlineto{\pgfqpoint{2.749929in}{1.091001in}}%
\pgfpathlineto{\pgfqpoint{2.755774in}{1.081517in}}%
\pgfpathlineto{\pgfqpoint{2.751742in}{1.068739in}}%
\pgfpathlineto{\pgfqpoint{2.756131in}{1.058082in}}%
\pgfpathlineto{\pgfqpoint{2.752122in}{1.044231in}}%
\pgfpathlineto{\pgfqpoint{2.747587in}{1.043751in}}%
\pgfpathlineto{\pgfqpoint{2.746037in}{1.035657in}}%
\pgfpathlineto{\pgfqpoint{2.753682in}{1.036469in}}%
\pgfpathlineto{\pgfqpoint{2.754206in}{1.025936in}}%
\pgfpathlineto{\pgfqpoint{2.723904in}{1.022807in}}%
\pgfpathclose%
\pgfusepath{fill}%
\end{pgfscope}%
\begin{pgfscope}%
\pgfpathrectangle{\pgfqpoint{0.100000in}{0.100000in}}{\pgfqpoint{3.608454in}{2.310000in}}%
\pgfusepath{clip}%
\pgfsetbuttcap%
\pgfsetmiterjoin%
\definecolor{currentfill}{rgb}{0.000000,0.466667,0.766667}%
\pgfsetfillcolor{currentfill}%
\pgfsetlinewidth{0.000000pt}%
\definecolor{currentstroke}{rgb}{0.000000,0.000000,0.000000}%
\pgfsetstrokecolor{currentstroke}%
\pgfsetstrokeopacity{0.000000}%
\pgfsetdash{}{0pt}%
\pgfpathmoveto{\pgfqpoint{2.667350in}{1.499209in}}%
\pgfpathlineto{\pgfqpoint{2.653860in}{1.497742in}}%
\pgfpathlineto{\pgfqpoint{2.649395in}{1.497238in}}%
\pgfpathlineto{\pgfqpoint{2.646259in}{1.524584in}}%
\pgfpathlineto{\pgfqpoint{2.633101in}{1.523054in}}%
\pgfpathlineto{\pgfqpoint{2.633720in}{1.516230in}}%
\pgfpathlineto{\pgfqpoint{2.615905in}{1.514539in}}%
\pgfpathlineto{\pgfqpoint{2.614051in}{1.535158in}}%
\pgfpathlineto{\pgfqpoint{2.638514in}{1.537637in}}%
\pgfpathlineto{\pgfqpoint{2.636183in}{1.558287in}}%
\pgfpathlineto{\pgfqpoint{2.660092in}{1.560783in}}%
\pgfpathlineto{\pgfqpoint{2.661344in}{1.549614in}}%
\pgfpathlineto{\pgfqpoint{2.660236in}{1.540245in}}%
\pgfpathlineto{\pgfqpoint{2.661863in}{1.526375in}}%
\pgfpathlineto{\pgfqpoint{2.664197in}{1.526649in}}%
\pgfpathlineto{\pgfqpoint{2.667350in}{1.499209in}}%
\pgfpathclose%
\pgfusepath{fill}%
\end{pgfscope}%
\begin{pgfscope}%
\pgfpathrectangle{\pgfqpoint{0.100000in}{0.100000in}}{\pgfqpoint{3.608454in}{2.310000in}}%
\pgfusepath{clip}%
\pgfsetbuttcap%
\pgfsetmiterjoin%
\definecolor{currentfill}{rgb}{0.000000,0.690196,0.654902}%
\pgfsetfillcolor{currentfill}%
\pgfsetlinewidth{0.000000pt}%
\definecolor{currentstroke}{rgb}{0.000000,0.000000,0.000000}%
\pgfsetstrokecolor{currentstroke}%
\pgfsetstrokeopacity{0.000000}%
\pgfsetdash{}{0pt}%
\pgfpathmoveto{\pgfqpoint{0.429033in}{1.527264in}}%
\pgfpathlineto{\pgfqpoint{0.422980in}{1.525827in}}%
\pgfpathlineto{\pgfqpoint{0.419978in}{1.533247in}}%
\pgfpathlineto{\pgfqpoint{0.410415in}{1.537899in}}%
\pgfpathlineto{\pgfqpoint{0.403577in}{1.549156in}}%
\pgfpathlineto{\pgfqpoint{0.396978in}{1.549508in}}%
\pgfpathlineto{\pgfqpoint{0.393248in}{1.563598in}}%
\pgfpathlineto{\pgfqpoint{0.387572in}{1.570213in}}%
\pgfpathlineto{\pgfqpoint{0.382759in}{1.579907in}}%
\pgfpathlineto{\pgfqpoint{0.380186in}{1.589872in}}%
\pgfpathlineto{\pgfqpoint{0.372415in}{1.602800in}}%
\pgfpathlineto{\pgfqpoint{0.369555in}{1.609925in}}%
\pgfpathlineto{\pgfqpoint{0.374700in}{1.618945in}}%
\pgfpathlineto{\pgfqpoint{0.373785in}{1.631630in}}%
\pgfpathlineto{\pgfqpoint{0.374411in}{1.644351in}}%
\pgfpathlineto{\pgfqpoint{0.377409in}{1.651582in}}%
\pgfpathlineto{\pgfqpoint{0.382897in}{1.658786in}}%
\pgfpathlineto{\pgfqpoint{0.384563in}{1.669377in}}%
\pgfpathlineto{\pgfqpoint{0.384726in}{1.681179in}}%
\pgfpathlineto{\pgfqpoint{0.379033in}{1.697172in}}%
\pgfpathlineto{\pgfqpoint{0.406652in}{1.688065in}}%
\pgfpathlineto{\pgfqpoint{0.406038in}{1.686194in}}%
\pgfpathlineto{\pgfqpoint{0.441276in}{1.674970in}}%
\pgfpathlineto{\pgfqpoint{0.436761in}{1.661477in}}%
\pgfpathlineto{\pgfqpoint{0.437233in}{1.653925in}}%
\pgfpathlineto{\pgfqpoint{0.434583in}{1.644086in}}%
\pgfpathlineto{\pgfqpoint{0.443292in}{1.641381in}}%
\pgfpathlineto{\pgfqpoint{0.440384in}{1.631531in}}%
\pgfpathlineto{\pgfqpoint{0.435002in}{1.619433in}}%
\pgfpathlineto{\pgfqpoint{0.440654in}{1.612562in}}%
\pgfpathlineto{\pgfqpoint{0.447344in}{1.609339in}}%
\pgfpathlineto{\pgfqpoint{0.445039in}{1.597163in}}%
\pgfpathlineto{\pgfqpoint{0.448894in}{1.592474in}}%
\pgfpathlineto{\pgfqpoint{0.450911in}{1.584637in}}%
\pgfpathlineto{\pgfqpoint{0.446281in}{1.581110in}}%
\pgfpathlineto{\pgfqpoint{0.446440in}{1.577149in}}%
\pgfpathlineto{\pgfqpoint{0.438495in}{1.570367in}}%
\pgfpathlineto{\pgfqpoint{0.427977in}{1.570534in}}%
\pgfpathlineto{\pgfqpoint{0.425226in}{1.565689in}}%
\pgfpathlineto{\pgfqpoint{0.431374in}{1.550880in}}%
\pgfpathlineto{\pgfqpoint{0.433169in}{1.538861in}}%
\pgfpathlineto{\pgfqpoint{0.432587in}{1.529018in}}%
\pgfpathlineto{\pgfqpoint{0.429033in}{1.527264in}}%
\pgfpathclose%
\pgfusepath{fill}%
\end{pgfscope}%
\begin{pgfscope}%
\pgfpathrectangle{\pgfqpoint{0.100000in}{0.100000in}}{\pgfqpoint{3.608454in}{2.310000in}}%
\pgfusepath{clip}%
\pgfsetbuttcap%
\pgfsetmiterjoin%
\definecolor{currentfill}{rgb}{0.000000,0.686275,0.656863}%
\pgfsetfillcolor{currentfill}%
\pgfsetlinewidth{0.000000pt}%
\definecolor{currentstroke}{rgb}{0.000000,0.000000,0.000000}%
\pgfsetstrokecolor{currentstroke}%
\pgfsetstrokeopacity{0.000000}%
\pgfsetdash{}{0pt}%
\pgfpathmoveto{\pgfqpoint{2.929879in}{1.293625in}}%
\pgfpathlineto{\pgfqpoint{2.916184in}{1.292250in}}%
\pgfpathlineto{\pgfqpoint{2.909448in}{1.297631in}}%
\pgfpathlineto{\pgfqpoint{2.908038in}{1.306264in}}%
\pgfpathlineto{\pgfqpoint{2.902369in}{1.314765in}}%
\pgfpathlineto{\pgfqpoint{2.895571in}{1.311319in}}%
\pgfpathlineto{\pgfqpoint{2.900793in}{1.318863in}}%
\pgfpathlineto{\pgfqpoint{2.894629in}{1.327193in}}%
\pgfpathlineto{\pgfqpoint{2.894444in}{1.332079in}}%
\pgfpathlineto{\pgfqpoint{2.898128in}{1.341949in}}%
\pgfpathlineto{\pgfqpoint{2.906738in}{1.347296in}}%
\pgfpathlineto{\pgfqpoint{2.905122in}{1.355064in}}%
\pgfpathlineto{\pgfqpoint{2.903048in}{1.365422in}}%
\pgfpathlineto{\pgfqpoint{2.909141in}{1.370242in}}%
\pgfpathlineto{\pgfqpoint{2.920044in}{1.373705in}}%
\pgfpathlineto{\pgfqpoint{2.930069in}{1.365085in}}%
\pgfpathlineto{\pgfqpoint{2.936140in}{1.370645in}}%
\pgfpathlineto{\pgfqpoint{2.940346in}{1.366252in}}%
\pgfpathlineto{\pgfqpoint{2.957191in}{1.367025in}}%
\pgfpathlineto{\pgfqpoint{2.965422in}{1.379505in}}%
\pgfpathlineto{\pgfqpoint{2.972924in}{1.373592in}}%
\pgfpathlineto{\pgfqpoint{2.976752in}{1.368166in}}%
\pgfpathlineto{\pgfqpoint{2.975847in}{1.360884in}}%
\pgfpathlineto{\pgfqpoint{2.958041in}{1.346022in}}%
\pgfpathlineto{\pgfqpoint{2.953019in}{1.337104in}}%
\pgfpathlineto{\pgfqpoint{2.952368in}{1.321633in}}%
\pgfpathlineto{\pgfqpoint{2.944352in}{1.321971in}}%
\pgfpathlineto{\pgfqpoint{2.941096in}{1.316634in}}%
\pgfpathlineto{\pgfqpoint{2.946226in}{1.306488in}}%
\pgfpathlineto{\pgfqpoint{2.931210in}{1.301107in}}%
\pgfpathlineto{\pgfqpoint{2.935674in}{1.297744in}}%
\pgfpathlineto{\pgfqpoint{2.929879in}{1.293625in}}%
\pgfpathclose%
\pgfusepath{fill}%
\end{pgfscope}%
\begin{pgfscope}%
\pgfpathrectangle{\pgfqpoint{0.100000in}{0.100000in}}{\pgfqpoint{3.608454in}{2.310000in}}%
\pgfusepath{clip}%
\pgfsetbuttcap%
\pgfsetmiterjoin%
\definecolor{currentfill}{rgb}{0.000000,0.490196,0.754902}%
\pgfsetfillcolor{currentfill}%
\pgfsetlinewidth{0.000000pt}%
\definecolor{currentstroke}{rgb}{0.000000,0.000000,0.000000}%
\pgfsetstrokecolor{currentstroke}%
\pgfsetstrokeopacity{0.000000}%
\pgfsetdash{}{0pt}%
\pgfpathmoveto{\pgfqpoint{1.538610in}{1.726627in}}%
\pgfpathlineto{\pgfqpoint{1.532653in}{1.671867in}}%
\pgfpathlineto{\pgfqpoint{1.531372in}{1.656425in}}%
\pgfpathlineto{\pgfqpoint{1.508517in}{1.658724in}}%
\pgfpathlineto{\pgfqpoint{1.507108in}{1.644984in}}%
\pgfpathlineto{\pgfqpoint{1.504277in}{1.645283in}}%
\pgfpathlineto{\pgfqpoint{1.500514in}{1.634480in}}%
\pgfpathlineto{\pgfqpoint{1.486783in}{1.637214in}}%
\pgfpathlineto{\pgfqpoint{1.487410in}{1.642958in}}%
\pgfpathlineto{\pgfqpoint{1.491303in}{1.646739in}}%
\pgfpathlineto{\pgfqpoint{1.461309in}{1.650537in}}%
\pgfpathlineto{\pgfqpoint{1.429302in}{1.654336in}}%
\pgfpathlineto{\pgfqpoint{1.377332in}{1.661590in}}%
\pgfpathlineto{\pgfqpoint{1.376226in}{1.661744in}}%
\pgfpathlineto{\pgfqpoint{1.379922in}{1.688990in}}%
\pgfpathlineto{\pgfqpoint{1.382345in}{1.688646in}}%
\pgfpathlineto{\pgfqpoint{1.386059in}{1.715853in}}%
\pgfpathlineto{\pgfqpoint{1.388966in}{1.743023in}}%
\pgfpathlineto{\pgfqpoint{1.388262in}{1.745418in}}%
\pgfpathlineto{\pgfqpoint{1.412417in}{1.742036in}}%
\pgfpathlineto{\pgfqpoint{1.441739in}{1.737485in}}%
\pgfpathlineto{\pgfqpoint{1.474731in}{1.733639in}}%
\pgfpathlineto{\pgfqpoint{1.538610in}{1.726627in}}%
\pgfpathclose%
\pgfusepath{fill}%
\end{pgfscope}%
\begin{pgfscope}%
\pgfpathrectangle{\pgfqpoint{0.100000in}{0.100000in}}{\pgfqpoint{3.608454in}{2.310000in}}%
\pgfusepath{clip}%
\pgfsetbuttcap%
\pgfsetmiterjoin%
\definecolor{currentfill}{rgb}{0.000000,0.372549,0.813725}%
\pgfsetfillcolor{currentfill}%
\pgfsetlinewidth{0.000000pt}%
\definecolor{currentstroke}{rgb}{0.000000,0.000000,0.000000}%
\pgfsetstrokecolor{currentstroke}%
\pgfsetstrokeopacity{0.000000}%
\pgfsetdash{}{0pt}%
\pgfpathmoveto{\pgfqpoint{2.088125in}{1.418722in}}%
\pgfpathlineto{\pgfqpoint{2.086211in}{1.418729in}}%
\pgfpathlineto{\pgfqpoint{2.031759in}{1.419257in}}%
\pgfpathlineto{\pgfqpoint{2.018170in}{1.419481in}}%
\pgfpathlineto{\pgfqpoint{2.019155in}{1.481378in}}%
\pgfpathlineto{\pgfqpoint{2.056854in}{1.480886in}}%
\pgfpathlineto{\pgfqpoint{2.053550in}{1.477162in}}%
\pgfpathlineto{\pgfqpoint{2.059082in}{1.471267in}}%
\pgfpathlineto{\pgfqpoint{2.060821in}{1.465149in}}%
\pgfpathlineto{\pgfqpoint{2.069544in}{1.443619in}}%
\pgfpathlineto{\pgfqpoint{2.077207in}{1.438452in}}%
\pgfpathlineto{\pgfqpoint{2.077853in}{1.433443in}}%
\pgfpathlineto{\pgfqpoint{2.082351in}{1.428944in}}%
\pgfpathlineto{\pgfqpoint{2.081282in}{1.423410in}}%
\pgfpathlineto{\pgfqpoint{2.088125in}{1.418722in}}%
\pgfpathclose%
\pgfusepath{fill}%
\end{pgfscope}%
\begin{pgfscope}%
\pgfpathrectangle{\pgfqpoint{0.100000in}{0.100000in}}{\pgfqpoint{3.608454in}{2.310000in}}%
\pgfusepath{clip}%
\pgfsetbuttcap%
\pgfsetmiterjoin%
\definecolor{currentfill}{rgb}{0.000000,0.619608,0.690196}%
\pgfsetfillcolor{currentfill}%
\pgfsetlinewidth{0.000000pt}%
\definecolor{currentstroke}{rgb}{0.000000,0.000000,0.000000}%
\pgfsetstrokecolor{currentstroke}%
\pgfsetstrokeopacity{0.000000}%
\pgfsetdash{}{0pt}%
\pgfpathmoveto{\pgfqpoint{1.624223in}{0.845743in}}%
\pgfpathlineto{\pgfqpoint{1.566607in}{0.850506in}}%
\pgfpathlineto{\pgfqpoint{1.563559in}{0.816012in}}%
\pgfpathlineto{\pgfqpoint{1.562944in}{0.809149in}}%
\pgfpathlineto{\pgfqpoint{1.518670in}{0.813225in}}%
\pgfpathlineto{\pgfqpoint{1.522650in}{0.854609in}}%
\pgfpathlineto{\pgfqpoint{1.516573in}{0.855187in}}%
\pgfpathlineto{\pgfqpoint{1.519940in}{0.890229in}}%
\pgfpathlineto{\pgfqpoint{1.523143in}{0.889921in}}%
\pgfpathlineto{\pgfqpoint{1.526429in}{0.923948in}}%
\pgfpathlineto{\pgfqpoint{1.529406in}{0.923693in}}%
\pgfpathlineto{\pgfqpoint{1.530771in}{0.937426in}}%
\pgfpathlineto{\pgfqpoint{1.544636in}{0.936153in}}%
\pgfpathlineto{\pgfqpoint{1.574789in}{0.933425in}}%
\pgfpathlineto{\pgfqpoint{1.573241in}{0.919040in}}%
\pgfpathlineto{\pgfqpoint{1.638056in}{0.913724in}}%
\pgfpathlineto{\pgfqpoint{1.635380in}{0.879674in}}%
\pgfpathlineto{\pgfqpoint{1.626580in}{0.880308in}}%
\pgfpathlineto{\pgfqpoint{1.624223in}{0.845743in}}%
\pgfpathclose%
\pgfusepath{fill}%
\end{pgfscope}%
\begin{pgfscope}%
\pgfpathrectangle{\pgfqpoint{0.100000in}{0.100000in}}{\pgfqpoint{3.608454in}{2.310000in}}%
\pgfusepath{clip}%
\pgfsetbuttcap%
\pgfsetmiterjoin%
\definecolor{currentfill}{rgb}{0.000000,0.388235,0.805882}%
\pgfsetfillcolor{currentfill}%
\pgfsetlinewidth{0.000000pt}%
\definecolor{currentstroke}{rgb}{0.000000,0.000000,0.000000}%
\pgfsetstrokecolor{currentstroke}%
\pgfsetstrokeopacity{0.000000}%
\pgfsetdash{}{0pt}%
\pgfpathmoveto{\pgfqpoint{2.547881in}{1.571936in}}%
\pgfpathlineto{\pgfqpoint{2.550384in}{1.539587in}}%
\pgfpathlineto{\pgfqpoint{2.521564in}{1.536820in}}%
\pgfpathlineto{\pgfqpoint{2.522216in}{1.529995in}}%
\pgfpathlineto{\pgfqpoint{2.508442in}{1.528613in}}%
\pgfpathlineto{\pgfqpoint{2.508522in}{1.521688in}}%
\pgfpathlineto{\pgfqpoint{2.488674in}{1.519766in}}%
\pgfpathlineto{\pgfqpoint{2.484797in}{1.561007in}}%
\pgfpathlineto{\pgfqpoint{2.482413in}{1.595355in}}%
\pgfpathlineto{\pgfqpoint{2.482705in}{1.602290in}}%
\pgfpathlineto{\pgfqpoint{2.475849in}{1.601806in}}%
\pgfpathlineto{\pgfqpoint{2.473885in}{1.628654in}}%
\pgfpathlineto{\pgfqpoint{2.526558in}{1.632431in}}%
\pgfpathlineto{\pgfqpoint{2.525980in}{1.616699in}}%
\pgfpathlineto{\pgfqpoint{2.528499in}{1.610045in}}%
\pgfpathlineto{\pgfqpoint{2.536221in}{1.600158in}}%
\pgfpathlineto{\pgfqpoint{2.541972in}{1.582322in}}%
\pgfpathlineto{\pgfqpoint{2.547881in}{1.571936in}}%
\pgfpathclose%
\pgfusepath{fill}%
\end{pgfscope}%
\begin{pgfscope}%
\pgfpathrectangle{\pgfqpoint{0.100000in}{0.100000in}}{\pgfqpoint{3.608454in}{2.310000in}}%
\pgfusepath{clip}%
\pgfsetbuttcap%
\pgfsetmiterjoin%
\definecolor{currentfill}{rgb}{0.000000,0.490196,0.754902}%
\pgfsetfillcolor{currentfill}%
\pgfsetlinewidth{0.000000pt}%
\definecolor{currentstroke}{rgb}{0.000000,0.000000,0.000000}%
\pgfsetstrokecolor{currentstroke}%
\pgfsetstrokeopacity{0.000000}%
\pgfsetdash{}{0pt}%
\pgfpathmoveto{\pgfqpoint{1.962491in}{1.386340in}}%
\pgfpathlineto{\pgfqpoint{1.961886in}{1.365738in}}%
\pgfpathlineto{\pgfqpoint{1.927666in}{1.366756in}}%
\pgfpathlineto{\pgfqpoint{1.928379in}{1.387333in}}%
\pgfpathlineto{\pgfqpoint{1.893270in}{1.388628in}}%
\pgfpathlineto{\pgfqpoint{1.894644in}{1.422961in}}%
\pgfpathlineto{\pgfqpoint{1.908588in}{1.422459in}}%
\pgfpathlineto{\pgfqpoint{1.963356in}{1.420723in}}%
\pgfpathlineto{\pgfqpoint{1.962491in}{1.386340in}}%
\pgfpathclose%
\pgfusepath{fill}%
\end{pgfscope}%
\begin{pgfscope}%
\pgfpathrectangle{\pgfqpoint{0.100000in}{0.100000in}}{\pgfqpoint{3.608454in}{2.310000in}}%
\pgfusepath{clip}%
\pgfsetbuttcap%
\pgfsetmiterjoin%
\definecolor{currentfill}{rgb}{0.000000,0.698039,0.650980}%
\pgfsetfillcolor{currentfill}%
\pgfsetlinewidth{0.000000pt}%
\definecolor{currentstroke}{rgb}{0.000000,0.000000,0.000000}%
\pgfsetstrokecolor{currentstroke}%
\pgfsetstrokeopacity{0.000000}%
\pgfsetdash{}{0pt}%
\pgfpathmoveto{\pgfqpoint{1.691656in}{1.593744in}}%
\pgfpathlineto{\pgfqpoint{1.655822in}{1.596346in}}%
\pgfpathlineto{\pgfqpoint{1.652617in}{1.603533in}}%
\pgfpathlineto{\pgfqpoint{1.654543in}{1.630993in}}%
\pgfpathlineto{\pgfqpoint{1.653148in}{1.631106in}}%
\pgfpathlineto{\pgfqpoint{1.655483in}{1.658354in}}%
\pgfpathlineto{\pgfqpoint{1.655854in}{1.675393in}}%
\pgfpathlineto{\pgfqpoint{1.643808in}{1.676503in}}%
\pgfpathlineto{\pgfqpoint{1.648016in}{1.724287in}}%
\pgfpathlineto{\pgfqpoint{1.654175in}{1.729234in}}%
\pgfpathlineto{\pgfqpoint{1.659487in}{1.729827in}}%
\pgfpathlineto{\pgfqpoint{1.695616in}{1.726789in}}%
\pgfpathlineto{\pgfqpoint{1.697918in}{1.727667in}}%
\pgfpathlineto{\pgfqpoint{1.696530in}{1.709950in}}%
\pgfpathlineto{\pgfqpoint{1.697612in}{1.702965in}}%
\pgfpathlineto{\pgfqpoint{1.748309in}{1.699187in}}%
\pgfpathlineto{\pgfqpoint{1.746207in}{1.668241in}}%
\pgfpathlineto{\pgfqpoint{1.696825in}{1.671982in}}%
\pgfpathlineto{\pgfqpoint{1.696414in}{1.655020in}}%
\pgfpathlineto{\pgfqpoint{1.694322in}{1.628015in}}%
\pgfpathlineto{\pgfqpoint{1.695984in}{1.627874in}}%
\pgfpathlineto{\pgfqpoint{1.693738in}{1.600532in}}%
\pgfpathlineto{\pgfqpoint{1.691656in}{1.593744in}}%
\pgfpathclose%
\pgfusepath{fill}%
\end{pgfscope}%
\begin{pgfscope}%
\pgfpathrectangle{\pgfqpoint{0.100000in}{0.100000in}}{\pgfqpoint{3.608454in}{2.310000in}}%
\pgfusepath{clip}%
\pgfsetbuttcap%
\pgfsetmiterjoin%
\definecolor{currentfill}{rgb}{0.000000,0.721569,0.639216}%
\pgfsetfillcolor{currentfill}%
\pgfsetlinewidth{0.000000pt}%
\definecolor{currentstroke}{rgb}{0.000000,0.000000,0.000000}%
\pgfsetstrokecolor{currentstroke}%
\pgfsetstrokeopacity{0.000000}%
\pgfsetdash{}{0pt}%
\pgfpathmoveto{\pgfqpoint{2.136953in}{2.110228in}}%
\pgfpathlineto{\pgfqpoint{2.145780in}{2.110086in}}%
\pgfpathlineto{\pgfqpoint{2.146826in}{2.106493in}}%
\pgfpathlineto{\pgfqpoint{2.168582in}{2.104438in}}%
\pgfpathlineto{\pgfqpoint{2.170811in}{2.095732in}}%
\pgfpathlineto{\pgfqpoint{2.188133in}{2.098370in}}%
\pgfpathlineto{\pgfqpoint{2.188169in}{2.101766in}}%
\pgfpathlineto{\pgfqpoint{2.201785in}{2.106387in}}%
\pgfpathlineto{\pgfqpoint{2.208113in}{2.105322in}}%
\pgfpathlineto{\pgfqpoint{2.208627in}{2.060466in}}%
\pgfpathlineto{\pgfqpoint{2.209745in}{2.046401in}}%
\pgfpathlineto{\pgfqpoint{2.172471in}{2.046322in}}%
\pgfpathlineto{\pgfqpoint{2.172503in}{2.042117in}}%
\pgfpathlineto{\pgfqpoint{2.138015in}{2.041726in}}%
\pgfpathlineto{\pgfqpoint{2.137206in}{2.083483in}}%
\pgfpathlineto{\pgfqpoint{2.136953in}{2.110228in}}%
\pgfpathclose%
\pgfusepath{fill}%
\end{pgfscope}%
\begin{pgfscope}%
\pgfpathrectangle{\pgfqpoint{0.100000in}{0.100000in}}{\pgfqpoint{3.608454in}{2.310000in}}%
\pgfusepath{clip}%
\pgfsetbuttcap%
\pgfsetmiterjoin%
\definecolor{currentfill}{rgb}{0.000000,0.545098,0.727451}%
\pgfsetfillcolor{currentfill}%
\pgfsetlinewidth{0.000000pt}%
\definecolor{currentstroke}{rgb}{0.000000,0.000000,0.000000}%
\pgfsetstrokecolor{currentstroke}%
\pgfsetstrokeopacity{0.000000}%
\pgfsetdash{}{0pt}%
\pgfpathmoveto{\pgfqpoint{2.554067in}{1.495355in}}%
\pgfpathlineto{\pgfqpoint{2.555646in}{1.476018in}}%
\pgfpathlineto{\pgfqpoint{2.531157in}{1.473630in}}%
\pgfpathlineto{\pgfqpoint{2.520205in}{1.472936in}}%
\pgfpathlineto{\pgfqpoint{2.516336in}{1.513050in}}%
\pgfpathlineto{\pgfqpoint{2.509486in}{1.512307in}}%
\pgfpathlineto{\pgfqpoint{2.508522in}{1.521688in}}%
\pgfpathlineto{\pgfqpoint{2.508442in}{1.528613in}}%
\pgfpathlineto{\pgfqpoint{2.522216in}{1.529995in}}%
\pgfpathlineto{\pgfqpoint{2.521564in}{1.536820in}}%
\pgfpathlineto{\pgfqpoint{2.550384in}{1.539587in}}%
\pgfpathlineto{\pgfqpoint{2.554067in}{1.495355in}}%
\pgfpathclose%
\pgfusepath{fill}%
\end{pgfscope}%
\begin{pgfscope}%
\pgfpathrectangle{\pgfqpoint{0.100000in}{0.100000in}}{\pgfqpoint{3.608454in}{2.310000in}}%
\pgfusepath{clip}%
\pgfsetbuttcap%
\pgfsetmiterjoin%
\definecolor{currentfill}{rgb}{0.000000,0.537255,0.731373}%
\pgfsetfillcolor{currentfill}%
\pgfsetlinewidth{0.000000pt}%
\definecolor{currentstroke}{rgb}{0.000000,0.000000,0.000000}%
\pgfsetstrokecolor{currentstroke}%
\pgfsetstrokeopacity{0.000000}%
\pgfsetdash{}{0pt}%
\pgfpathmoveto{\pgfqpoint{2.452854in}{0.589739in}}%
\pgfpathlineto{\pgfqpoint{2.445754in}{0.592557in}}%
\pgfpathlineto{\pgfqpoint{2.442006in}{0.599263in}}%
\pgfpathlineto{\pgfqpoint{2.443757in}{0.607123in}}%
\pgfpathlineto{\pgfqpoint{2.440322in}{0.609463in}}%
\pgfpathlineto{\pgfqpoint{2.430825in}{0.609081in}}%
\pgfpathlineto{\pgfqpoint{2.428180in}{0.614445in}}%
\pgfpathlineto{\pgfqpoint{2.420890in}{0.618181in}}%
\pgfpathlineto{\pgfqpoint{2.416070in}{0.624294in}}%
\pgfpathlineto{\pgfqpoint{2.392099in}{0.624223in}}%
\pgfpathlineto{\pgfqpoint{2.386695in}{0.629800in}}%
\pgfpathlineto{\pgfqpoint{2.386115in}{0.636858in}}%
\pgfpathlineto{\pgfqpoint{2.389280in}{0.640751in}}%
\pgfpathlineto{\pgfqpoint{2.407987in}{0.645878in}}%
\pgfpathlineto{\pgfqpoint{2.415540in}{0.649843in}}%
\pgfpathlineto{\pgfqpoint{2.419484in}{0.654679in}}%
\pgfpathlineto{\pgfqpoint{2.431222in}{0.657222in}}%
\pgfpathlineto{\pgfqpoint{2.435277in}{0.651886in}}%
\pgfpathlineto{\pgfqpoint{2.432451in}{0.691509in}}%
\pgfpathlineto{\pgfqpoint{2.424922in}{0.713419in}}%
\pgfpathlineto{\pgfqpoint{2.467199in}{0.715885in}}%
\pgfpathlineto{\pgfqpoint{2.465118in}{0.704833in}}%
\pgfpathlineto{\pgfqpoint{2.461305in}{0.698015in}}%
\pgfpathlineto{\pgfqpoint{2.460731in}{0.688378in}}%
\pgfpathlineto{\pgfqpoint{2.466743in}{0.676753in}}%
\pgfpathlineto{\pgfqpoint{2.472822in}{0.672176in}}%
\pgfpathlineto{\pgfqpoint{2.478549in}{0.654208in}}%
\pgfpathlineto{\pgfqpoint{2.485040in}{0.651337in}}%
\pgfpathlineto{\pgfqpoint{2.478638in}{0.648879in}}%
\pgfpathlineto{\pgfqpoint{2.472203in}{0.638511in}}%
\pgfpathlineto{\pgfqpoint{2.465852in}{0.639118in}}%
\pgfpathlineto{\pgfqpoint{2.464257in}{0.631545in}}%
\pgfpathlineto{\pgfqpoint{2.470761in}{0.631547in}}%
\pgfpathlineto{\pgfqpoint{2.473516in}{0.626309in}}%
\pgfpathlineto{\pgfqpoint{2.481391in}{0.627004in}}%
\pgfpathlineto{\pgfqpoint{2.482013in}{0.636124in}}%
\pgfpathlineto{\pgfqpoint{2.487826in}{0.640181in}}%
\pgfpathlineto{\pgfqpoint{2.500761in}{0.631538in}}%
\pgfpathlineto{\pgfqpoint{2.493847in}{0.616542in}}%
\pgfpathlineto{\pgfqpoint{2.478334in}{0.616245in}}%
\pgfpathlineto{\pgfqpoint{2.480862in}{0.608985in}}%
\pgfpathlineto{\pgfqpoint{2.477552in}{0.599501in}}%
\pgfpathlineto{\pgfqpoint{2.485322in}{0.596196in}}%
\pgfpathlineto{\pgfqpoint{2.492517in}{0.589207in}}%
\pgfpathlineto{\pgfqpoint{2.511782in}{0.586087in}}%
\pgfpathlineto{\pgfqpoint{2.520554in}{0.574448in}}%
\pgfpathlineto{\pgfqpoint{2.526501in}{0.573415in}}%
\pgfpathlineto{\pgfqpoint{2.522992in}{0.564279in}}%
\pgfpathlineto{\pgfqpoint{2.516518in}{0.560200in}}%
\pgfpathlineto{\pgfqpoint{2.511813in}{0.562316in}}%
\pgfpathlineto{\pgfqpoint{2.503209in}{0.561121in}}%
\pgfpathlineto{\pgfqpoint{2.491659in}{0.574580in}}%
\pgfpathlineto{\pgfqpoint{2.475551in}{0.580137in}}%
\pgfpathlineto{\pgfqpoint{2.466715in}{0.580608in}}%
\pgfpathlineto{\pgfqpoint{2.468264in}{0.585589in}}%
\pgfpathlineto{\pgfqpoint{2.466744in}{0.592994in}}%
\pgfpathlineto{\pgfqpoint{2.452854in}{0.589739in}}%
\pgfpathclose%
\pgfusepath{fill}%
\end{pgfscope}%
\begin{pgfscope}%
\pgfpathrectangle{\pgfqpoint{0.100000in}{0.100000in}}{\pgfqpoint{3.608454in}{2.310000in}}%
\pgfusepath{clip}%
\pgfsetbuttcap%
\pgfsetmiterjoin%
\definecolor{currentfill}{rgb}{0.000000,0.607843,0.696078}%
\pgfsetfillcolor{currentfill}%
\pgfsetlinewidth{0.000000pt}%
\definecolor{currentstroke}{rgb}{0.000000,0.000000,0.000000}%
\pgfsetstrokecolor{currentstroke}%
\pgfsetstrokeopacity{0.000000}%
\pgfsetdash{}{0pt}%
\pgfpathmoveto{\pgfqpoint{2.306014in}{1.604048in}}%
\pgfpathlineto{\pgfqpoint{2.264937in}{1.602761in}}%
\pgfpathlineto{\pgfqpoint{2.237469in}{1.601952in}}%
\pgfpathlineto{\pgfqpoint{2.237674in}{1.595106in}}%
\pgfpathlineto{\pgfqpoint{2.223956in}{1.594717in}}%
\pgfpathlineto{\pgfqpoint{2.223427in}{1.615413in}}%
\pgfpathlineto{\pgfqpoint{2.222007in}{1.615392in}}%
\pgfpathlineto{\pgfqpoint{2.221382in}{1.649990in}}%
\pgfpathlineto{\pgfqpoint{2.276089in}{1.651404in}}%
\pgfpathlineto{\pgfqpoint{2.275677in}{1.665325in}}%
\pgfpathlineto{\pgfqpoint{2.303199in}{1.666191in}}%
\pgfpathlineto{\pgfqpoint{2.304826in}{1.617701in}}%
\pgfpathlineto{\pgfqpoint{2.306014in}{1.604048in}}%
\pgfpathclose%
\pgfusepath{fill}%
\end{pgfscope}%
\begin{pgfscope}%
\pgfpathrectangle{\pgfqpoint{0.100000in}{0.100000in}}{\pgfqpoint{3.608454in}{2.310000in}}%
\pgfusepath{clip}%
\pgfsetbuttcap%
\pgfsetmiterjoin%
\definecolor{currentfill}{rgb}{0.000000,0.333333,0.833333}%
\pgfsetfillcolor{currentfill}%
\pgfsetlinewidth{0.000000pt}%
\definecolor{currentstroke}{rgb}{0.000000,0.000000,0.000000}%
\pgfsetstrokecolor{currentstroke}%
\pgfsetstrokeopacity{0.000000}%
\pgfsetdash{}{0pt}%
\pgfpathmoveto{\pgfqpoint{2.866524in}{0.926111in}}%
\pgfpathlineto{\pgfqpoint{2.849065in}{0.920723in}}%
\pgfpathlineto{\pgfqpoint{2.848183in}{0.924494in}}%
\pgfpathlineto{\pgfqpoint{2.848096in}{0.929927in}}%
\pgfpathlineto{\pgfqpoint{2.843454in}{0.939687in}}%
\pgfpathlineto{\pgfqpoint{2.838874in}{0.945035in}}%
\pgfpathlineto{\pgfqpoint{2.830443in}{0.932055in}}%
\pgfpathlineto{\pgfqpoint{2.818507in}{0.934685in}}%
\pgfpathlineto{\pgfqpoint{2.809234in}{0.933588in}}%
\pgfpathlineto{\pgfqpoint{2.807171in}{0.925663in}}%
\pgfpathlineto{\pgfqpoint{2.802892in}{0.925244in}}%
\pgfpathlineto{\pgfqpoint{2.796122in}{0.932528in}}%
\pgfpathlineto{\pgfqpoint{2.791999in}{0.944257in}}%
\pgfpathlineto{\pgfqpoint{2.778830in}{0.942946in}}%
\pgfpathlineto{\pgfqpoint{2.773147in}{0.947109in}}%
\pgfpathlineto{\pgfqpoint{2.771603in}{0.963496in}}%
\pgfpathlineto{\pgfqpoint{2.762398in}{0.964970in}}%
\pgfpathlineto{\pgfqpoint{2.760784in}{0.972182in}}%
\pgfpathlineto{\pgfqpoint{2.765064in}{0.976399in}}%
\pgfpathlineto{\pgfqpoint{2.767572in}{0.987144in}}%
\pgfpathlineto{\pgfqpoint{2.784817in}{0.988730in}}%
\pgfpathlineto{\pgfqpoint{2.780892in}{1.025876in}}%
\pgfpathlineto{\pgfqpoint{2.799254in}{1.027968in}}%
\pgfpathlineto{\pgfqpoint{2.801686in}{1.031063in}}%
\pgfpathlineto{\pgfqpoint{2.810594in}{1.033632in}}%
\pgfpathlineto{\pgfqpoint{2.810860in}{1.028741in}}%
\pgfpathlineto{\pgfqpoint{2.818142in}{1.023149in}}%
\pgfpathlineto{\pgfqpoint{2.825899in}{1.020792in}}%
\pgfpathlineto{\pgfqpoint{2.830646in}{1.010243in}}%
\pgfpathlineto{\pgfqpoint{2.827384in}{1.003120in}}%
\pgfpathlineto{\pgfqpoint{2.836248in}{0.996454in}}%
\pgfpathlineto{\pgfqpoint{2.839244in}{0.999162in}}%
\pgfpathlineto{\pgfqpoint{2.837086in}{0.989053in}}%
\pgfpathlineto{\pgfqpoint{2.844938in}{0.981568in}}%
\pgfpathlineto{\pgfqpoint{2.850509in}{0.983786in}}%
\pgfpathlineto{\pgfqpoint{2.864261in}{0.976571in}}%
\pgfpathlineto{\pgfqpoint{2.870396in}{0.968762in}}%
\pgfpathlineto{\pgfqpoint{2.880855in}{0.952903in}}%
\pgfpathlineto{\pgfqpoint{2.864668in}{0.946906in}}%
\pgfpathlineto{\pgfqpoint{2.866524in}{0.926111in}}%
\pgfpathclose%
\pgfusepath{fill}%
\end{pgfscope}%
\begin{pgfscope}%
\pgfpathrectangle{\pgfqpoint{0.100000in}{0.100000in}}{\pgfqpoint{3.608454in}{2.310000in}}%
\pgfusepath{clip}%
\pgfsetbuttcap%
\pgfsetmiterjoin%
\definecolor{currentfill}{rgb}{0.000000,0.803922,0.598039}%
\pgfsetfillcolor{currentfill}%
\pgfsetlinewidth{0.000000pt}%
\definecolor{currentstroke}{rgb}{0.000000,0.000000,0.000000}%
\pgfsetstrokecolor{currentstroke}%
\pgfsetstrokeopacity{0.000000}%
\pgfsetdash{}{0pt}%
\pgfpathmoveto{\pgfqpoint{2.882548in}{1.076650in}}%
\pgfpathlineto{\pgfqpoint{2.876403in}{1.073700in}}%
\pgfpathlineto{\pgfqpoint{2.872545in}{1.080983in}}%
\pgfpathlineto{\pgfqpoint{2.867388in}{1.085037in}}%
\pgfpathlineto{\pgfqpoint{2.858178in}{1.097707in}}%
\pgfpathlineto{\pgfqpoint{2.851025in}{1.094550in}}%
\pgfpathlineto{\pgfqpoint{2.837326in}{1.091183in}}%
\pgfpathlineto{\pgfqpoint{2.835727in}{1.088275in}}%
\pgfpathlineto{\pgfqpoint{2.827437in}{1.087830in}}%
\pgfpathlineto{\pgfqpoint{2.819581in}{1.083755in}}%
\pgfpathlineto{\pgfqpoint{2.814347in}{1.089123in}}%
\pgfpathlineto{\pgfqpoint{2.813980in}{1.097663in}}%
\pgfpathlineto{\pgfqpoint{2.817068in}{1.103142in}}%
\pgfpathlineto{\pgfqpoint{2.828227in}{1.112365in}}%
\pgfpathlineto{\pgfqpoint{2.835111in}{1.113784in}}%
\pgfpathlineto{\pgfqpoint{2.845692in}{1.114629in}}%
\pgfpathlineto{\pgfqpoint{2.859932in}{1.127064in}}%
\pgfpathlineto{\pgfqpoint{2.864797in}{1.125728in}}%
\pgfpathlineto{\pgfqpoint{2.866841in}{1.114054in}}%
\pgfpathlineto{\pgfqpoint{2.884882in}{1.098977in}}%
\pgfpathlineto{\pgfqpoint{2.885380in}{1.093119in}}%
\pgfpathlineto{\pgfqpoint{2.879644in}{1.078954in}}%
\pgfpathlineto{\pgfqpoint{2.882548in}{1.076650in}}%
\pgfpathclose%
\pgfusepath{fill}%
\end{pgfscope}%
\begin{pgfscope}%
\pgfpathrectangle{\pgfqpoint{0.100000in}{0.100000in}}{\pgfqpoint{3.608454in}{2.310000in}}%
\pgfusepath{clip}%
\pgfsetbuttcap%
\pgfsetmiterjoin%
\definecolor{currentfill}{rgb}{0.000000,0.419608,0.790196}%
\pgfsetfillcolor{currentfill}%
\pgfsetlinewidth{0.000000pt}%
\definecolor{currentstroke}{rgb}{0.000000,0.000000,0.000000}%
\pgfsetstrokecolor{currentstroke}%
\pgfsetstrokeopacity{0.000000}%
\pgfsetdash{}{0pt}%
\pgfpathmoveto{\pgfqpoint{2.433615in}{1.653903in}}%
\pgfpathlineto{\pgfqpoint{2.435480in}{1.626662in}}%
\pgfpathlineto{\pgfqpoint{2.402794in}{1.625171in}}%
\pgfpathlineto{\pgfqpoint{2.361136in}{1.623207in}}%
\pgfpathlineto{\pgfqpoint{2.356020in}{1.633465in}}%
\pgfpathlineto{\pgfqpoint{2.342986in}{1.636283in}}%
\pgfpathlineto{\pgfqpoint{2.336274in}{1.640299in}}%
\pgfpathlineto{\pgfqpoint{2.333060in}{1.651644in}}%
\pgfpathlineto{\pgfqpoint{2.330505in}{1.653247in}}%
\pgfpathlineto{\pgfqpoint{2.327890in}{1.671376in}}%
\pgfpathlineto{\pgfqpoint{2.334254in}{1.681222in}}%
\pgfpathlineto{\pgfqpoint{2.325644in}{1.688292in}}%
\pgfpathlineto{\pgfqpoint{2.325187in}{1.694132in}}%
\pgfpathlineto{\pgfqpoint{2.356072in}{1.695517in}}%
\pgfpathlineto{\pgfqpoint{2.357123in}{1.675633in}}%
\pgfpathlineto{\pgfqpoint{2.363976in}{1.679112in}}%
\pgfpathlineto{\pgfqpoint{2.370685in}{1.678611in}}%
\pgfpathlineto{\pgfqpoint{2.372395in}{1.647911in}}%
\pgfpathlineto{\pgfqpoint{2.406543in}{1.649815in}}%
\pgfpathlineto{\pgfqpoint{2.406360in}{1.653264in}}%
\pgfpathlineto{\pgfqpoint{2.433615in}{1.653903in}}%
\pgfpathclose%
\pgfusepath{fill}%
\end{pgfscope}%
\begin{pgfscope}%
\pgfpathrectangle{\pgfqpoint{0.100000in}{0.100000in}}{\pgfqpoint{3.608454in}{2.310000in}}%
\pgfusepath{clip}%
\pgfsetbuttcap%
\pgfsetmiterjoin%
\definecolor{currentfill}{rgb}{0.000000,0.807843,0.596078}%
\pgfsetfillcolor{currentfill}%
\pgfsetlinewidth{0.000000pt}%
\definecolor{currentstroke}{rgb}{0.000000,0.000000,0.000000}%
\pgfsetstrokecolor{currentstroke}%
\pgfsetstrokeopacity{0.000000}%
\pgfsetdash{}{0pt}%
\pgfpathmoveto{\pgfqpoint{2.288166in}{1.125922in}}%
\pgfpathlineto{\pgfqpoint{2.288004in}{1.144708in}}%
\pgfpathlineto{\pgfqpoint{2.289909in}{1.144788in}}%
\pgfpathlineto{\pgfqpoint{2.289673in}{1.168159in}}%
\pgfpathlineto{\pgfqpoint{2.290504in}{1.188921in}}%
\pgfpathlineto{\pgfqpoint{2.280237in}{1.188871in}}%
\pgfpathlineto{\pgfqpoint{2.279053in}{1.231943in}}%
\pgfpathlineto{\pgfqpoint{2.310037in}{1.232515in}}%
\pgfpathlineto{\pgfqpoint{2.310515in}{1.218707in}}%
\pgfpathlineto{\pgfqpoint{2.346916in}{1.219387in}}%
\pgfpathlineto{\pgfqpoint{2.347208in}{1.211310in}}%
\pgfpathlineto{\pgfqpoint{2.350740in}{1.202506in}}%
\pgfpathlineto{\pgfqpoint{2.357801in}{1.198383in}}%
\pgfpathlineto{\pgfqpoint{2.357912in}{1.194600in}}%
\pgfpathlineto{\pgfqpoint{2.345378in}{1.193313in}}%
\pgfpathlineto{\pgfqpoint{2.345750in}{1.172705in}}%
\pgfpathlineto{\pgfqpoint{2.352645in}{1.172837in}}%
\pgfpathlineto{\pgfqpoint{2.353073in}{1.147045in}}%
\pgfpathlineto{\pgfqpoint{2.335244in}{1.146265in}}%
\pgfpathlineto{\pgfqpoint{2.332518in}{1.146200in}}%
\pgfpathlineto{\pgfqpoint{2.332736in}{1.133366in}}%
\pgfpathlineto{\pgfqpoint{2.325633in}{1.133246in}}%
\pgfpathlineto{\pgfqpoint{2.325882in}{1.126330in}}%
\pgfpathlineto{\pgfqpoint{2.288166in}{1.125922in}}%
\pgfpathclose%
\pgfusepath{fill}%
\end{pgfscope}%
\begin{pgfscope}%
\pgfpathrectangle{\pgfqpoint{0.100000in}{0.100000in}}{\pgfqpoint{3.608454in}{2.310000in}}%
\pgfusepath{clip}%
\pgfsetbuttcap%
\pgfsetmiterjoin%
\definecolor{currentfill}{rgb}{0.000000,0.509804,0.745098}%
\pgfsetfillcolor{currentfill}%
\pgfsetlinewidth{0.000000pt}%
\definecolor{currentstroke}{rgb}{0.000000,0.000000,0.000000}%
\pgfsetstrokecolor{currentstroke}%
\pgfsetstrokeopacity{0.000000}%
\pgfsetdash{}{0pt}%
\pgfpathmoveto{\pgfqpoint{1.202252in}{1.853038in}}%
\pgfpathlineto{\pgfqpoint{1.204580in}{1.868127in}}%
\pgfpathlineto{\pgfqpoint{1.226134in}{1.864328in}}%
\pgfpathlineto{\pgfqpoint{1.224995in}{1.857794in}}%
\pgfpathlineto{\pgfqpoint{1.241305in}{1.854999in}}%
\pgfpathlineto{\pgfqpoint{1.245880in}{1.848777in}}%
\pgfpathlineto{\pgfqpoint{1.243670in}{1.842268in}}%
\pgfpathlineto{\pgfqpoint{1.249318in}{1.830814in}}%
\pgfpathlineto{\pgfqpoint{1.250644in}{1.821729in}}%
\pgfpathlineto{\pgfqpoint{1.253366in}{1.816700in}}%
\pgfpathlineto{\pgfqpoint{1.251746in}{1.806946in}}%
\pgfpathlineto{\pgfqpoint{1.244733in}{1.764439in}}%
\pgfpathlineto{\pgfqpoint{1.243269in}{1.757690in}}%
\pgfpathlineto{\pgfqpoint{1.226972in}{1.760454in}}%
\pgfpathlineto{\pgfqpoint{1.225832in}{1.753673in}}%
\pgfpathlineto{\pgfqpoint{1.212511in}{1.756000in}}%
\pgfpathlineto{\pgfqpoint{1.211732in}{1.751597in}}%
\pgfpathlineto{\pgfqpoint{1.198346in}{1.753929in}}%
\pgfpathlineto{\pgfqpoint{1.199226in}{1.758954in}}%
\pgfpathlineto{\pgfqpoint{1.186356in}{1.762535in}}%
\pgfpathlineto{\pgfqpoint{1.188888in}{1.777013in}}%
\pgfpathlineto{\pgfqpoint{1.180737in}{1.781578in}}%
\pgfpathlineto{\pgfqpoint{1.179359in}{1.788436in}}%
\pgfpathlineto{\pgfqpoint{1.170551in}{1.790061in}}%
\pgfpathlineto{\pgfqpoint{1.174897in}{1.813546in}}%
\pgfpathlineto{\pgfqpoint{1.186131in}{1.812373in}}%
\pgfpathlineto{\pgfqpoint{1.195518in}{1.814680in}}%
\pgfpathlineto{\pgfqpoint{1.202252in}{1.853038in}}%
\pgfpathclose%
\pgfusepath{fill}%
\end{pgfscope}%
\begin{pgfscope}%
\pgfpathrectangle{\pgfqpoint{0.100000in}{0.100000in}}{\pgfqpoint{3.608454in}{2.310000in}}%
\pgfusepath{clip}%
\pgfsetbuttcap%
\pgfsetmiterjoin%
\definecolor{currentfill}{rgb}{0.000000,0.572549,0.713725}%
\pgfsetfillcolor{currentfill}%
\pgfsetlinewidth{0.000000pt}%
\definecolor{currentstroke}{rgb}{0.000000,0.000000,0.000000}%
\pgfsetstrokecolor{currentstroke}%
\pgfsetstrokeopacity{0.000000}%
\pgfsetdash{}{0pt}%
\pgfpathmoveto{\pgfqpoint{1.873743in}{0.830250in}}%
\pgfpathlineto{\pgfqpoint{1.873122in}{0.813238in}}%
\pgfpathlineto{\pgfqpoint{1.842140in}{0.796889in}}%
\pgfpathlineto{\pgfqpoint{1.788954in}{0.799743in}}%
\pgfpathlineto{\pgfqpoint{1.759459in}{0.801389in}}%
\pgfpathlineto{\pgfqpoint{1.763926in}{0.871040in}}%
\pgfpathlineto{\pgfqpoint{1.808729in}{0.868432in}}%
\pgfpathlineto{\pgfqpoint{1.868391in}{0.865576in}}%
\pgfpathlineto{\pgfqpoint{1.866975in}{0.830682in}}%
\pgfpathlineto{\pgfqpoint{1.873743in}{0.830250in}}%
\pgfpathclose%
\pgfusepath{fill}%
\end{pgfscope}%
\begin{pgfscope}%
\pgfpathrectangle{\pgfqpoint{0.100000in}{0.100000in}}{\pgfqpoint{3.608454in}{2.310000in}}%
\pgfusepath{clip}%
\pgfsetbuttcap%
\pgfsetmiterjoin%
\definecolor{currentfill}{rgb}{0.000000,0.666667,0.666667}%
\pgfsetfillcolor{currentfill}%
\pgfsetlinewidth{0.000000pt}%
\definecolor{currentstroke}{rgb}{0.000000,0.000000,0.000000}%
\pgfsetstrokecolor{currentstroke}%
\pgfsetstrokeopacity{0.000000}%
\pgfsetdash{}{0pt}%
\pgfpathmoveto{\pgfqpoint{1.103689in}{1.391163in}}%
\pgfpathlineto{\pgfqpoint{1.103543in}{1.390370in}}%
\pgfpathlineto{\pgfqpoint{1.141855in}{1.383558in}}%
\pgfpathlineto{\pgfqpoint{1.181575in}{1.376785in}}%
\pgfpathlineto{\pgfqpoint{1.183516in}{1.359664in}}%
\pgfpathlineto{\pgfqpoint{1.186021in}{1.352257in}}%
\pgfpathlineto{\pgfqpoint{1.183026in}{1.348649in}}%
\pgfpathlineto{\pgfqpoint{1.136475in}{1.356525in}}%
\pgfpathlineto{\pgfqpoint{1.065800in}{1.369302in}}%
\pgfpathlineto{\pgfqpoint{1.069417in}{1.390629in}}%
\pgfpathlineto{\pgfqpoint{1.076288in}{1.396187in}}%
\pgfpathlineto{\pgfqpoint{1.103689in}{1.391163in}}%
\pgfpathclose%
\pgfusepath{fill}%
\end{pgfscope}%
\begin{pgfscope}%
\pgfpathrectangle{\pgfqpoint{0.100000in}{0.100000in}}{\pgfqpoint{3.608454in}{2.310000in}}%
\pgfusepath{clip}%
\pgfsetbuttcap%
\pgfsetmiterjoin%
\definecolor{currentfill}{rgb}{0.000000,0.643137,0.678431}%
\pgfsetfillcolor{currentfill}%
\pgfsetlinewidth{0.000000pt}%
\definecolor{currentstroke}{rgb}{0.000000,0.000000,0.000000}%
\pgfsetstrokecolor{currentstroke}%
\pgfsetstrokeopacity{0.000000}%
\pgfsetdash{}{0pt}%
\pgfpathmoveto{\pgfqpoint{2.123559in}{0.920377in}}%
\pgfpathlineto{\pgfqpoint{2.140773in}{0.915277in}}%
\pgfpathlineto{\pgfqpoint{2.141202in}{0.939334in}}%
\pgfpathlineto{\pgfqpoint{2.147313in}{0.939341in}}%
\pgfpathlineto{\pgfqpoint{2.157026in}{0.928424in}}%
\pgfpathlineto{\pgfqpoint{2.170065in}{0.926960in}}%
\pgfpathlineto{\pgfqpoint{2.175503in}{0.924555in}}%
\pgfpathlineto{\pgfqpoint{2.175110in}{0.918121in}}%
\pgfpathlineto{\pgfqpoint{2.183884in}{0.913246in}}%
\pgfpathlineto{\pgfqpoint{2.191374in}{0.903396in}}%
\pgfpathlineto{\pgfqpoint{2.190887in}{0.898874in}}%
\pgfpathlineto{\pgfqpoint{2.195345in}{0.891554in}}%
\pgfpathlineto{\pgfqpoint{2.189397in}{0.878197in}}%
\pgfpathlineto{\pgfqpoint{2.183288in}{0.876547in}}%
\pgfpathlineto{\pgfqpoint{2.182676in}{0.872245in}}%
\pgfpathlineto{\pgfqpoint{2.186423in}{0.866575in}}%
\pgfpathlineto{\pgfqpoint{2.170562in}{0.866376in}}%
\pgfpathlineto{\pgfqpoint{2.170684in}{0.855388in}}%
\pgfpathlineto{\pgfqpoint{2.126427in}{0.854941in}}%
\pgfpathlineto{\pgfqpoint{2.125475in}{0.856951in}}%
\pgfpathlineto{\pgfqpoint{2.096641in}{0.856769in}}%
\pgfpathlineto{\pgfqpoint{2.095246in}{0.861494in}}%
\pgfpathlineto{\pgfqpoint{2.086189in}{0.861574in}}%
\pgfpathlineto{\pgfqpoint{2.086328in}{0.894484in}}%
\pgfpathlineto{\pgfqpoint{2.100603in}{0.896104in}}%
\pgfpathlineto{\pgfqpoint{2.115003in}{0.895858in}}%
\pgfpathlineto{\pgfqpoint{2.123587in}{0.890672in}}%
\pgfpathlineto{\pgfqpoint{2.123559in}{0.920377in}}%
\pgfpathclose%
\pgfusepath{fill}%
\end{pgfscope}%
\begin{pgfscope}%
\pgfpathrectangle{\pgfqpoint{0.100000in}{0.100000in}}{\pgfqpoint{3.608454in}{2.310000in}}%
\pgfusepath{clip}%
\pgfsetbuttcap%
\pgfsetmiterjoin%
\definecolor{currentfill}{rgb}{0.000000,0.890196,0.554902}%
\pgfsetfillcolor{currentfill}%
\pgfsetlinewidth{0.000000pt}%
\definecolor{currentstroke}{rgb}{0.000000,0.000000,0.000000}%
\pgfsetstrokecolor{currentstroke}%
\pgfsetstrokeopacity{0.000000}%
\pgfsetdash{}{0pt}%
\pgfpathmoveto{\pgfqpoint{3.001283in}{0.581332in}}%
\pgfpathlineto{\pgfqpoint{2.990145in}{0.581978in}}%
\pgfpathlineto{\pgfqpoint{2.974786in}{0.579836in}}%
\pgfpathlineto{\pgfqpoint{2.972832in}{0.594908in}}%
\pgfpathlineto{\pgfqpoint{2.963258in}{0.602568in}}%
\pgfpathlineto{\pgfqpoint{2.965965in}{0.605834in}}%
\pgfpathlineto{\pgfqpoint{2.973159in}{0.605106in}}%
\pgfpathlineto{\pgfqpoint{2.978125in}{0.608670in}}%
\pgfpathlineto{\pgfqpoint{2.976237in}{0.622163in}}%
\pgfpathlineto{\pgfqpoint{2.985432in}{0.623511in}}%
\pgfpathlineto{\pgfqpoint{2.982209in}{0.644864in}}%
\pgfpathlineto{\pgfqpoint{2.995671in}{0.646679in}}%
\pgfpathlineto{\pgfqpoint{2.996330in}{0.642369in}}%
\pgfpathlineto{\pgfqpoint{3.006579in}{0.647225in}}%
\pgfpathlineto{\pgfqpoint{3.020704in}{0.653264in}}%
\pgfpathlineto{\pgfqpoint{3.025702in}{0.651277in}}%
\pgfpathlineto{\pgfqpoint{3.029457in}{0.642400in}}%
\pgfpathlineto{\pgfqpoint{3.037437in}{0.635958in}}%
\pgfpathlineto{\pgfqpoint{3.040352in}{0.617807in}}%
\pgfpathlineto{\pgfqpoint{3.039989in}{0.610672in}}%
\pgfpathlineto{\pgfqpoint{2.994626in}{0.604123in}}%
\pgfpathlineto{\pgfqpoint{3.006411in}{0.592107in}}%
\pgfpathlineto{\pgfqpoint{3.001283in}{0.581332in}}%
\pgfpathclose%
\pgfusepath{fill}%
\end{pgfscope}%
\begin{pgfscope}%
\pgfpathrectangle{\pgfqpoint{0.100000in}{0.100000in}}{\pgfqpoint{3.608454in}{2.310000in}}%
\pgfusepath{clip}%
\pgfsetbuttcap%
\pgfsetmiterjoin%
\definecolor{currentfill}{rgb}{0.000000,0.615686,0.692157}%
\pgfsetfillcolor{currentfill}%
\pgfsetlinewidth{0.000000pt}%
\definecolor{currentstroke}{rgb}{0.000000,0.000000,0.000000}%
\pgfsetstrokecolor{currentstroke}%
\pgfsetstrokeopacity{0.000000}%
\pgfsetdash{}{0pt}%
\pgfpathmoveto{\pgfqpoint{2.002513in}{1.074759in}}%
\pgfpathlineto{\pgfqpoint{2.030056in}{1.074337in}}%
\pgfpathlineto{\pgfqpoint{2.030288in}{1.091590in}}%
\pgfpathlineto{\pgfqpoint{2.057637in}{1.091246in}}%
\pgfpathlineto{\pgfqpoint{2.057547in}{1.080899in}}%
\pgfpathlineto{\pgfqpoint{2.060962in}{1.080885in}}%
\pgfpathlineto{\pgfqpoint{2.060863in}{1.067101in}}%
\pgfpathlineto{\pgfqpoint{2.053817in}{1.067155in}}%
\pgfpathlineto{\pgfqpoint{2.053684in}{1.060206in}}%
\pgfpathlineto{\pgfqpoint{2.050250in}{1.053326in}}%
\pgfpathlineto{\pgfqpoint{2.043425in}{1.053406in}}%
\pgfpathlineto{\pgfqpoint{2.043347in}{1.046522in}}%
\pgfpathlineto{\pgfqpoint{2.013608in}{1.046935in}}%
\pgfpathlineto{\pgfqpoint{2.010159in}{1.057965in}}%
\pgfpathlineto{\pgfqpoint{2.001980in}{1.055914in}}%
\pgfpathlineto{\pgfqpoint{2.002513in}{1.074759in}}%
\pgfpathclose%
\pgfusepath{fill}%
\end{pgfscope}%
\begin{pgfscope}%
\pgfpathrectangle{\pgfqpoint{0.100000in}{0.100000in}}{\pgfqpoint{3.608454in}{2.310000in}}%
\pgfusepath{clip}%
\pgfsetbuttcap%
\pgfsetmiterjoin%
\definecolor{currentfill}{rgb}{0.000000,0.721569,0.639216}%
\pgfsetfillcolor{currentfill}%
\pgfsetlinewidth{0.000000pt}%
\definecolor{currentstroke}{rgb}{0.000000,0.000000,0.000000}%
\pgfsetstrokecolor{currentstroke}%
\pgfsetstrokeopacity{0.000000}%
\pgfsetdash{}{0pt}%
\pgfpathmoveto{\pgfqpoint{2.949055in}{1.160645in}}%
\pgfpathlineto{\pgfqpoint{2.942760in}{1.162773in}}%
\pgfpathlineto{\pgfqpoint{2.938556in}{1.155078in}}%
\pgfpathlineto{\pgfqpoint{2.934882in}{1.156995in}}%
\pgfpathlineto{\pgfqpoint{2.930144in}{1.169183in}}%
\pgfpathlineto{\pgfqpoint{2.933147in}{1.171253in}}%
\pgfpathlineto{\pgfqpoint{2.936507in}{1.182790in}}%
\pgfpathlineto{\pgfqpoint{2.929794in}{1.191773in}}%
\pgfpathlineto{\pgfqpoint{2.933739in}{1.199536in}}%
\pgfpathlineto{\pgfqpoint{2.934516in}{1.206979in}}%
\pgfpathlineto{\pgfqpoint{2.940357in}{1.211530in}}%
\pgfpathlineto{\pgfqpoint{2.951682in}{1.213011in}}%
\pgfpathlineto{\pgfqpoint{2.950033in}{1.210864in}}%
\pgfpathlineto{\pgfqpoint{2.950303in}{1.201290in}}%
\pgfpathlineto{\pgfqpoint{2.948361in}{1.195220in}}%
\pgfpathlineto{\pgfqpoint{2.955269in}{1.192553in}}%
\pgfpathlineto{\pgfqpoint{2.968126in}{1.182900in}}%
\pgfpathlineto{\pgfqpoint{2.963826in}{1.175254in}}%
\pgfpathlineto{\pgfqpoint{2.955589in}{1.174438in}}%
\pgfpathlineto{\pgfqpoint{2.949055in}{1.160645in}}%
\pgfpathclose%
\pgfusepath{fill}%
\end{pgfscope}%
\begin{pgfscope}%
\pgfpathrectangle{\pgfqpoint{0.100000in}{0.100000in}}{\pgfqpoint{3.608454in}{2.310000in}}%
\pgfusepath{clip}%
\pgfsetbuttcap%
\pgfsetmiterjoin%
\definecolor{currentfill}{rgb}{0.000000,0.419608,0.790196}%
\pgfsetfillcolor{currentfill}%
\pgfsetlinewidth{0.000000pt}%
\definecolor{currentstroke}{rgb}{0.000000,0.000000,0.000000}%
\pgfsetstrokecolor{currentstroke}%
\pgfsetstrokeopacity{0.000000}%
\pgfsetdash{}{0pt}%
\pgfpathmoveto{\pgfqpoint{1.812489in}{1.842876in}}%
\pgfpathlineto{\pgfqpoint{1.843275in}{1.841097in}}%
\pgfpathlineto{\pgfqpoint{1.850854in}{1.840684in}}%
\pgfpathlineto{\pgfqpoint{1.849510in}{1.813058in}}%
\pgfpathlineto{\pgfqpoint{1.843805in}{1.813366in}}%
\pgfpathlineto{\pgfqpoint{1.802764in}{1.815650in}}%
\pgfpathlineto{\pgfqpoint{1.802162in}{1.823198in}}%
\pgfpathlineto{\pgfqpoint{1.809782in}{1.824532in}}%
\pgfpathlineto{\pgfqpoint{1.808240in}{1.831561in}}%
\pgfpathlineto{\pgfqpoint{1.812489in}{1.842876in}}%
\pgfpathclose%
\pgfusepath{fill}%
\end{pgfscope}%
\begin{pgfscope}%
\pgfpathrectangle{\pgfqpoint{0.100000in}{0.100000in}}{\pgfqpoint{3.608454in}{2.310000in}}%
\pgfusepath{clip}%
\pgfsetbuttcap%
\pgfsetmiterjoin%
\definecolor{currentfill}{rgb}{0.000000,0.666667,0.666667}%
\pgfsetfillcolor{currentfill}%
\pgfsetlinewidth{0.000000pt}%
\definecolor{currentstroke}{rgb}{0.000000,0.000000,0.000000}%
\pgfsetstrokecolor{currentstroke}%
\pgfsetstrokeopacity{0.000000}%
\pgfsetdash{}{0pt}%
\pgfpathmoveto{\pgfqpoint{1.498395in}{0.815195in}}%
\pgfpathlineto{\pgfqpoint{1.518670in}{0.813225in}}%
\pgfpathlineto{\pgfqpoint{1.562944in}{0.809149in}}%
\pgfpathlineto{\pgfqpoint{1.563559in}{0.816012in}}%
\pgfpathlineto{\pgfqpoint{1.581375in}{0.814353in}}%
\pgfpathlineto{\pgfqpoint{1.578420in}{0.779909in}}%
\pgfpathlineto{\pgfqpoint{1.580562in}{0.779720in}}%
\pgfpathlineto{\pgfqpoint{1.578119in}{0.751288in}}%
\pgfpathlineto{\pgfqpoint{1.574315in}{0.749565in}}%
\pgfpathlineto{\pgfqpoint{1.567213in}{0.756734in}}%
\pgfpathlineto{\pgfqpoint{1.562114in}{0.758888in}}%
\pgfpathlineto{\pgfqpoint{1.518651in}{0.714404in}}%
\pgfpathlineto{\pgfqpoint{1.486032in}{0.744751in}}%
\pgfpathlineto{\pgfqpoint{1.498395in}{0.815195in}}%
\pgfpathclose%
\pgfusepath{fill}%
\end{pgfscope}%
\begin{pgfscope}%
\pgfpathrectangle{\pgfqpoint{0.100000in}{0.100000in}}{\pgfqpoint{3.608454in}{2.310000in}}%
\pgfusepath{clip}%
\pgfsetbuttcap%
\pgfsetmiterjoin%
\definecolor{currentfill}{rgb}{0.000000,0.345098,0.827451}%
\pgfsetfillcolor{currentfill}%
\pgfsetlinewidth{0.000000pt}%
\definecolor{currentstroke}{rgb}{0.000000,0.000000,0.000000}%
\pgfsetstrokecolor{currentstroke}%
\pgfsetstrokeopacity{0.000000}%
\pgfsetdash{}{0pt}%
\pgfpathmoveto{\pgfqpoint{3.271268in}{1.222696in}}%
\pgfpathlineto{\pgfqpoint{3.265839in}{1.231923in}}%
\pgfpathlineto{\pgfqpoint{3.266234in}{1.243000in}}%
\pgfpathlineto{\pgfqpoint{3.259748in}{1.247063in}}%
\pgfpathlineto{\pgfqpoint{3.249894in}{1.248153in}}%
\pgfpathlineto{\pgfqpoint{3.248646in}{1.259422in}}%
\pgfpathlineto{\pgfqpoint{3.248223in}{1.267423in}}%
\pgfpathlineto{\pgfqpoint{3.266535in}{1.293082in}}%
\pgfpathlineto{\pgfqpoint{3.269493in}{1.294748in}}%
\pgfpathlineto{\pgfqpoint{3.275926in}{1.293375in}}%
\pgfpathlineto{\pgfqpoint{3.281080in}{1.300393in}}%
\pgfpathlineto{\pgfqpoint{3.293773in}{1.298157in}}%
\pgfpathlineto{\pgfqpoint{3.299684in}{1.300538in}}%
\pgfpathlineto{\pgfqpoint{3.305220in}{1.288703in}}%
\pgfpathlineto{\pgfqpoint{3.312824in}{1.275983in}}%
\pgfpathlineto{\pgfqpoint{3.324933in}{1.250084in}}%
\pgfpathlineto{\pgfqpoint{3.324107in}{1.249673in}}%
\pgfpathlineto{\pgfqpoint{3.316533in}{1.263940in}}%
\pgfpathlineto{\pgfqpoint{3.313407in}{1.258882in}}%
\pgfpathlineto{\pgfqpoint{3.326492in}{1.238123in}}%
\pgfpathlineto{\pgfqpoint{3.319080in}{1.242583in}}%
\pgfpathlineto{\pgfqpoint{3.311911in}{1.242756in}}%
\pgfpathlineto{\pgfqpoint{3.299704in}{1.233838in}}%
\pgfpathlineto{\pgfqpoint{3.289367in}{1.230321in}}%
\pgfpathlineto{\pgfqpoint{3.283250in}{1.223621in}}%
\pgfpathlineto{\pgfqpoint{3.271268in}{1.222696in}}%
\pgfpathclose%
\pgfusepath{fill}%
\end{pgfscope}%
\begin{pgfscope}%
\pgfpathrectangle{\pgfqpoint{0.100000in}{0.100000in}}{\pgfqpoint{3.608454in}{2.310000in}}%
\pgfusepath{clip}%
\pgfsetbuttcap%
\pgfsetmiterjoin%
\definecolor{currentfill}{rgb}{0.000000,0.776471,0.611765}%
\pgfsetfillcolor{currentfill}%
\pgfsetlinewidth{0.000000pt}%
\definecolor{currentstroke}{rgb}{0.000000,0.000000,0.000000}%
\pgfsetstrokecolor{currentstroke}%
\pgfsetstrokeopacity{0.000000}%
\pgfsetdash{}{0pt}%
\pgfpathmoveto{\pgfqpoint{2.002912in}{1.098705in}}%
\pgfpathlineto{\pgfqpoint{1.969632in}{1.099419in}}%
\pgfpathlineto{\pgfqpoint{1.956052in}{1.103169in}}%
\pgfpathlineto{\pgfqpoint{1.956419in}{1.117360in}}%
\pgfpathlineto{\pgfqpoint{1.949595in}{1.117555in}}%
\pgfpathlineto{\pgfqpoint{1.950456in}{1.151489in}}%
\pgfpathlineto{\pgfqpoint{1.984832in}{1.151230in}}%
\pgfpathlineto{\pgfqpoint{1.985971in}{1.147335in}}%
\pgfpathlineto{\pgfqpoint{1.977572in}{1.143978in}}%
\pgfpathlineto{\pgfqpoint{1.977272in}{1.130240in}}%
\pgfpathlineto{\pgfqpoint{1.984100in}{1.130082in}}%
\pgfpathlineto{\pgfqpoint{1.983944in}{1.123221in}}%
\pgfpathlineto{\pgfqpoint{1.990717in}{1.123080in}}%
\pgfpathlineto{\pgfqpoint{1.990571in}{1.116149in}}%
\pgfpathlineto{\pgfqpoint{2.003198in}{1.115990in}}%
\pgfpathlineto{\pgfqpoint{2.002912in}{1.098705in}}%
\pgfpathclose%
\pgfusepath{fill}%
\end{pgfscope}%
\begin{pgfscope}%
\pgfpathrectangle{\pgfqpoint{0.100000in}{0.100000in}}{\pgfqpoint{3.608454in}{2.310000in}}%
\pgfusepath{clip}%
\pgfsetbuttcap%
\pgfsetmiterjoin%
\definecolor{currentfill}{rgb}{0.000000,0.439216,0.780392}%
\pgfsetfillcolor{currentfill}%
\pgfsetlinewidth{0.000000pt}%
\definecolor{currentstroke}{rgb}{0.000000,0.000000,0.000000}%
\pgfsetstrokecolor{currentstroke}%
\pgfsetstrokeopacity{0.000000}%
\pgfsetdash{}{0pt}%
\pgfpathmoveto{\pgfqpoint{1.857457in}{1.534695in}}%
\pgfpathlineto{\pgfqpoint{1.855925in}{1.507277in}}%
\pgfpathlineto{\pgfqpoint{1.842917in}{1.507856in}}%
\pgfpathlineto{\pgfqpoint{1.795374in}{1.510393in}}%
\pgfpathlineto{\pgfqpoint{1.796934in}{1.565224in}}%
\pgfpathlineto{\pgfqpoint{1.762096in}{1.567407in}}%
\pgfpathlineto{\pgfqpoint{1.763649in}{1.594885in}}%
\pgfpathlineto{\pgfqpoint{1.803419in}{1.592288in}}%
\pgfpathlineto{\pgfqpoint{1.831612in}{1.590776in}}%
\pgfpathlineto{\pgfqpoint{1.830208in}{1.563450in}}%
\pgfpathlineto{\pgfqpoint{1.858208in}{1.562101in}}%
\pgfpathlineto{\pgfqpoint{1.857457in}{1.534695in}}%
\pgfpathclose%
\pgfusepath{fill}%
\end{pgfscope}%
\begin{pgfscope}%
\pgfpathrectangle{\pgfqpoint{0.100000in}{0.100000in}}{\pgfqpoint{3.608454in}{2.310000in}}%
\pgfusepath{clip}%
\pgfsetbuttcap%
\pgfsetmiterjoin%
\definecolor{currentfill}{rgb}{0.000000,0.576471,0.711765}%
\pgfsetfillcolor{currentfill}%
\pgfsetlinewidth{0.000000pt}%
\definecolor{currentstroke}{rgb}{0.000000,0.000000,0.000000}%
\pgfsetstrokecolor{currentstroke}%
\pgfsetstrokeopacity{0.000000}%
\pgfsetdash{}{0pt}%
\pgfpathmoveto{\pgfqpoint{2.410748in}{1.065840in}}%
\pgfpathlineto{\pgfqpoint{2.389852in}{1.065601in}}%
\pgfpathlineto{\pgfqpoint{2.362249in}{1.063873in}}%
\pgfpathlineto{\pgfqpoint{2.361164in}{1.098864in}}%
\pgfpathlineto{\pgfqpoint{2.350507in}{1.098801in}}%
\pgfpathlineto{\pgfqpoint{2.340339in}{1.098431in}}%
\pgfpathlineto{\pgfqpoint{2.339731in}{1.125367in}}%
\pgfpathlineto{\pgfqpoint{2.345455in}{1.126590in}}%
\pgfpathlineto{\pgfqpoint{2.343960in}{1.136912in}}%
\pgfpathlineto{\pgfqpoint{2.335244in}{1.146265in}}%
\pgfpathlineto{\pgfqpoint{2.353073in}{1.147045in}}%
\pgfpathlineto{\pgfqpoint{2.410619in}{1.149798in}}%
\pgfpathlineto{\pgfqpoint{2.421101in}{1.141357in}}%
\pgfpathlineto{\pgfqpoint{2.420239in}{1.132410in}}%
\pgfpathlineto{\pgfqpoint{2.411938in}{1.125058in}}%
\pgfpathlineto{\pgfqpoint{2.406026in}{1.117238in}}%
\pgfpathlineto{\pgfqpoint{2.408349in}{1.109943in}}%
\pgfpathlineto{\pgfqpoint{2.410748in}{1.065840in}}%
\pgfpathclose%
\pgfusepath{fill}%
\end{pgfscope}%
\begin{pgfscope}%
\pgfpathrectangle{\pgfqpoint{0.100000in}{0.100000in}}{\pgfqpoint{3.608454in}{2.310000in}}%
\pgfusepath{clip}%
\pgfsetbuttcap%
\pgfsetmiterjoin%
\definecolor{currentfill}{rgb}{0.000000,0.650980,0.674510}%
\pgfsetfillcolor{currentfill}%
\pgfsetlinewidth{0.000000pt}%
\definecolor{currentstroke}{rgb}{0.000000,0.000000,0.000000}%
\pgfsetstrokecolor{currentstroke}%
\pgfsetstrokeopacity{0.000000}%
\pgfsetdash{}{0pt}%
\pgfpathmoveto{\pgfqpoint{1.383780in}{1.226748in}}%
\pgfpathlineto{\pgfqpoint{1.373495in}{1.246788in}}%
\pgfpathlineto{\pgfqpoint{1.375196in}{1.259964in}}%
\pgfpathlineto{\pgfqpoint{1.347103in}{1.263333in}}%
\pgfpathlineto{\pgfqpoint{1.347428in}{1.265725in}}%
\pgfpathlineto{\pgfqpoint{1.349065in}{1.285654in}}%
\pgfpathlineto{\pgfqpoint{1.352269in}{1.306231in}}%
\pgfpathlineto{\pgfqpoint{1.360950in}{1.306443in}}%
\pgfpathlineto{\pgfqpoint{1.365837in}{1.343240in}}%
\pgfpathlineto{\pgfqpoint{1.412097in}{1.337063in}}%
\pgfpathlineto{\pgfqpoint{1.422616in}{1.335846in}}%
\pgfpathlineto{\pgfqpoint{1.426840in}{1.337144in}}%
\pgfpathlineto{\pgfqpoint{1.444613in}{1.311073in}}%
\pgfpathlineto{\pgfqpoint{1.448675in}{1.296375in}}%
\pgfpathlineto{\pgfqpoint{1.454881in}{1.289984in}}%
\pgfpathlineto{\pgfqpoint{1.457227in}{1.286192in}}%
\pgfpathlineto{\pgfqpoint{1.450742in}{1.268060in}}%
\pgfpathlineto{\pgfqpoint{1.463748in}{1.269672in}}%
\pgfpathlineto{\pgfqpoint{1.469792in}{1.261228in}}%
\pgfpathlineto{\pgfqpoint{1.471264in}{1.248705in}}%
\pgfpathlineto{\pgfqpoint{1.469341in}{1.240362in}}%
\pgfpathlineto{\pgfqpoint{1.466660in}{1.216957in}}%
\pgfpathlineto{\pgfqpoint{1.462544in}{1.217413in}}%
\pgfpathlineto{\pgfqpoint{1.383780in}{1.226748in}}%
\pgfpathclose%
\pgfusepath{fill}%
\end{pgfscope}%
\begin{pgfscope}%
\pgfpathrectangle{\pgfqpoint{0.100000in}{0.100000in}}{\pgfqpoint{3.608454in}{2.310000in}}%
\pgfusepath{clip}%
\pgfsetbuttcap%
\pgfsetmiterjoin%
\definecolor{currentfill}{rgb}{0.000000,0.427451,0.786275}%
\pgfsetfillcolor{currentfill}%
\pgfsetlinewidth{0.000000pt}%
\definecolor{currentstroke}{rgb}{0.000000,0.000000,0.000000}%
\pgfsetstrokecolor{currentstroke}%
\pgfsetstrokeopacity{0.000000}%
\pgfsetdash{}{0pt}%
\pgfpathmoveto{\pgfqpoint{2.204405in}{1.465435in}}%
\pgfpathlineto{\pgfqpoint{2.193447in}{1.465250in}}%
\pgfpathlineto{\pgfqpoint{2.192689in}{1.511198in}}%
\pgfpathlineto{\pgfqpoint{2.206358in}{1.511376in}}%
\pgfpathlineto{\pgfqpoint{2.219965in}{1.511654in}}%
\pgfpathlineto{\pgfqpoint{2.220496in}{1.490915in}}%
\pgfpathlineto{\pgfqpoint{2.247888in}{1.491592in}}%
\pgfpathlineto{\pgfqpoint{2.248598in}{1.467244in}}%
\pgfpathlineto{\pgfqpoint{2.244011in}{1.467014in}}%
\pgfpathlineto{\pgfqpoint{2.204405in}{1.465435in}}%
\pgfpathclose%
\pgfusepath{fill}%
\end{pgfscope}%
\begin{pgfscope}%
\pgfpathrectangle{\pgfqpoint{0.100000in}{0.100000in}}{\pgfqpoint{3.608454in}{2.310000in}}%
\pgfusepath{clip}%
\pgfsetbuttcap%
\pgfsetmiterjoin%
\definecolor{currentfill}{rgb}{0.000000,0.180392,0.909804}%
\pgfsetfillcolor{currentfill}%
\pgfsetlinewidth{0.000000pt}%
\definecolor{currentstroke}{rgb}{0.000000,0.000000,0.000000}%
\pgfsetstrokecolor{currentstroke}%
\pgfsetstrokeopacity{0.000000}%
\pgfsetdash{}{0pt}%
\pgfpathmoveto{\pgfqpoint{1.945778in}{0.610158in}}%
\pgfpathlineto{\pgfqpoint{1.925990in}{0.598363in}}%
\pgfpathlineto{\pgfqpoint{1.923573in}{0.599973in}}%
\pgfpathlineto{\pgfqpoint{1.918807in}{0.601623in}}%
\pgfpathlineto{\pgfqpoint{1.907278in}{0.617155in}}%
\pgfpathlineto{\pgfqpoint{1.898375in}{0.608966in}}%
\pgfpathlineto{\pgfqpoint{1.896461in}{0.616726in}}%
\pgfpathlineto{\pgfqpoint{1.878483in}{0.632521in}}%
\pgfpathlineto{\pgfqpoint{1.870125in}{0.624788in}}%
\pgfpathlineto{\pgfqpoint{1.858713in}{0.641352in}}%
\pgfpathlineto{\pgfqpoint{1.859593in}{0.670192in}}%
\pgfpathlineto{\pgfqpoint{1.871994in}{0.669892in}}%
\pgfpathlineto{\pgfqpoint{1.891457in}{0.663121in}}%
\pgfpathlineto{\pgfqpoint{1.892292in}{0.669010in}}%
\pgfpathlineto{\pgfqpoint{1.903598in}{0.691454in}}%
\pgfpathlineto{\pgfqpoint{1.913119in}{0.701138in}}%
\pgfpathlineto{\pgfqpoint{1.926963in}{0.697491in}}%
\pgfpathlineto{\pgfqpoint{1.947915in}{0.687499in}}%
\pgfpathlineto{\pgfqpoint{1.952059in}{0.698355in}}%
\pgfpathlineto{\pgfqpoint{1.965169in}{0.705757in}}%
\pgfpathlineto{\pgfqpoint{1.981851in}{0.714949in}}%
\pgfpathlineto{\pgfqpoint{1.984720in}{0.711352in}}%
\pgfpathlineto{\pgfqpoint{1.986509in}{0.700655in}}%
\pgfpathlineto{\pgfqpoint{1.993815in}{0.692311in}}%
\pgfpathlineto{\pgfqpoint{1.995500in}{0.684736in}}%
\pgfpathlineto{\pgfqpoint{1.971709in}{0.671338in}}%
\pgfpathlineto{\pgfqpoint{1.975419in}{0.668492in}}%
\pgfpathlineto{\pgfqpoint{1.978079in}{0.661346in}}%
\pgfpathlineto{\pgfqpoint{1.986455in}{0.652084in}}%
\pgfpathlineto{\pgfqpoint{1.993561in}{0.650091in}}%
\pgfpathlineto{\pgfqpoint{1.987116in}{0.645931in}}%
\pgfpathlineto{\pgfqpoint{1.982721in}{0.639405in}}%
\pgfpathlineto{\pgfqpoint{1.979102in}{0.635917in}}%
\pgfpathlineto{\pgfqpoint{1.966633in}{0.631014in}}%
\pgfpathlineto{\pgfqpoint{1.945778in}{0.610158in}}%
\pgfpathclose%
\pgfusepath{fill}%
\end{pgfscope}%
\begin{pgfscope}%
\pgfpathrectangle{\pgfqpoint{0.100000in}{0.100000in}}{\pgfqpoint{3.608454in}{2.310000in}}%
\pgfusepath{clip}%
\pgfsetbuttcap%
\pgfsetmiterjoin%
\definecolor{currentfill}{rgb}{0.000000,0.690196,0.654902}%
\pgfsetfillcolor{currentfill}%
\pgfsetlinewidth{0.000000pt}%
\definecolor{currentstroke}{rgb}{0.000000,0.000000,0.000000}%
\pgfsetstrokecolor{currentstroke}%
\pgfsetstrokeopacity{0.000000}%
\pgfsetdash{}{0pt}%
\pgfpathmoveto{\pgfqpoint{2.288408in}{0.781863in}}%
\pgfpathlineto{\pgfqpoint{2.287535in}{0.813168in}}%
\pgfpathlineto{\pgfqpoint{2.280375in}{0.819890in}}%
\pgfpathlineto{\pgfqpoint{2.279806in}{0.840743in}}%
\pgfpathlineto{\pgfqpoint{2.272892in}{0.840599in}}%
\pgfpathlineto{\pgfqpoint{2.267654in}{0.847225in}}%
\pgfpathlineto{\pgfqpoint{2.258858in}{0.847301in}}%
\pgfpathlineto{\pgfqpoint{2.258381in}{0.867551in}}%
\pgfpathlineto{\pgfqpoint{2.302086in}{0.868312in}}%
\pgfpathlineto{\pgfqpoint{2.342635in}{0.869578in}}%
\pgfpathlineto{\pgfqpoint{2.344274in}{0.869656in}}%
\pgfpathlineto{\pgfqpoint{2.340706in}{0.860346in}}%
\pgfpathlineto{\pgfqpoint{2.337645in}{0.859095in}}%
\pgfpathlineto{\pgfqpoint{2.331640in}{0.845019in}}%
\pgfpathlineto{\pgfqpoint{2.334864in}{0.835567in}}%
\pgfpathlineto{\pgfqpoint{2.344959in}{0.835935in}}%
\pgfpathlineto{\pgfqpoint{2.344269in}{0.832393in}}%
\pgfpathlineto{\pgfqpoint{2.342569in}{0.828917in}}%
\pgfpathlineto{\pgfqpoint{2.344014in}{0.814843in}}%
\pgfpathlineto{\pgfqpoint{2.340198in}{0.812791in}}%
\pgfpathlineto{\pgfqpoint{2.342946in}{0.805709in}}%
\pgfpathlineto{\pgfqpoint{2.343388in}{0.795089in}}%
\pgfpathlineto{\pgfqpoint{2.338322in}{0.779971in}}%
\pgfpathlineto{\pgfqpoint{2.336629in}{0.786857in}}%
\pgfpathlineto{\pgfqpoint{2.324523in}{0.778629in}}%
\pgfpathlineto{\pgfqpoint{2.319400in}{0.785321in}}%
\pgfpathlineto{\pgfqpoint{2.315987in}{0.782756in}}%
\pgfpathlineto{\pgfqpoint{2.288408in}{0.781863in}}%
\pgfpathclose%
\pgfusepath{fill}%
\end{pgfscope}%
\begin{pgfscope}%
\pgfpathrectangle{\pgfqpoint{0.100000in}{0.100000in}}{\pgfqpoint{3.608454in}{2.310000in}}%
\pgfusepath{clip}%
\pgfsetbuttcap%
\pgfsetmiterjoin%
\definecolor{currentfill}{rgb}{0.000000,0.776471,0.611765}%
\pgfsetfillcolor{currentfill}%
\pgfsetlinewidth{0.000000pt}%
\definecolor{currentstroke}{rgb}{0.000000,0.000000,0.000000}%
\pgfsetstrokecolor{currentstroke}%
\pgfsetstrokeopacity{0.000000}%
\pgfsetdash{}{0pt}%
\pgfpathmoveto{\pgfqpoint{2.611522in}{1.917144in}}%
\pgfpathlineto{\pgfqpoint{2.589883in}{1.913322in}}%
\pgfpathlineto{\pgfqpoint{2.586900in}{1.909857in}}%
\pgfpathlineto{\pgfqpoint{2.586216in}{1.900993in}}%
\pgfpathlineto{\pgfqpoint{2.576778in}{1.896148in}}%
\pgfpathlineto{\pgfqpoint{2.572058in}{1.888198in}}%
\pgfpathlineto{\pgfqpoint{2.566665in}{1.889771in}}%
\pgfpathlineto{\pgfqpoint{2.572918in}{1.898549in}}%
\pgfpathlineto{\pgfqpoint{2.575837in}{1.905644in}}%
\pgfpathlineto{\pgfqpoint{2.562752in}{1.899838in}}%
\pgfpathlineto{\pgfqpoint{2.559587in}{1.892271in}}%
\pgfpathlineto{\pgfqpoint{2.552985in}{1.887697in}}%
\pgfpathlineto{\pgfqpoint{2.547425in}{1.892494in}}%
\pgfpathlineto{\pgfqpoint{2.541575in}{1.885933in}}%
\pgfpathlineto{\pgfqpoint{2.537283in}{1.876850in}}%
\pgfpathlineto{\pgfqpoint{2.533757in}{1.876549in}}%
\pgfpathlineto{\pgfqpoint{2.531516in}{1.904166in}}%
\pgfpathlineto{\pgfqpoint{2.528620in}{1.910846in}}%
\pgfpathlineto{\pgfqpoint{2.514867in}{1.909757in}}%
\pgfpathlineto{\pgfqpoint{2.513243in}{1.930478in}}%
\pgfpathlineto{\pgfqpoint{2.485703in}{1.928298in}}%
\pgfpathlineto{\pgfqpoint{2.484672in}{1.942048in}}%
\pgfpathlineto{\pgfqpoint{2.483177in}{1.962621in}}%
\pgfpathlineto{\pgfqpoint{2.489953in}{1.963218in}}%
\pgfpathlineto{\pgfqpoint{2.486521in}{1.969842in}}%
\pgfpathlineto{\pgfqpoint{2.485656in}{1.981514in}}%
\pgfpathlineto{\pgfqpoint{2.493445in}{1.982175in}}%
\pgfpathlineto{\pgfqpoint{2.508763in}{1.975427in}}%
\pgfpathlineto{\pgfqpoint{2.511603in}{1.969882in}}%
\pgfpathlineto{\pgfqpoint{2.526869in}{1.951632in}}%
\pgfpathlineto{\pgfqpoint{2.538957in}{1.952508in}}%
\pgfpathlineto{\pgfqpoint{2.544000in}{1.955889in}}%
\pgfpathlineto{\pgfqpoint{2.552498in}{1.948973in}}%
\pgfpathlineto{\pgfqpoint{2.564052in}{1.959757in}}%
\pgfpathlineto{\pgfqpoint{2.568367in}{1.953169in}}%
\pgfpathlineto{\pgfqpoint{2.574148in}{1.960469in}}%
\pgfpathlineto{\pgfqpoint{2.592197in}{1.970111in}}%
\pgfpathlineto{\pgfqpoint{2.605875in}{1.974408in}}%
\pgfpathlineto{\pgfqpoint{2.611522in}{1.917144in}}%
\pgfpathclose%
\pgfusepath{fill}%
\end{pgfscope}%
\begin{pgfscope}%
\pgfpathrectangle{\pgfqpoint{0.100000in}{0.100000in}}{\pgfqpoint{3.608454in}{2.310000in}}%
\pgfusepath{clip}%
\pgfsetbuttcap%
\pgfsetmiterjoin%
\definecolor{currentfill}{rgb}{0.000000,0.537255,0.731373}%
\pgfsetfillcolor{currentfill}%
\pgfsetlinewidth{0.000000pt}%
\definecolor{currentstroke}{rgb}{0.000000,0.000000,0.000000}%
\pgfsetstrokecolor{currentstroke}%
\pgfsetstrokeopacity{0.000000}%
\pgfsetdash{}{0pt}%
\pgfpathmoveto{\pgfqpoint{3.284382in}{1.624557in}}%
\pgfpathlineto{\pgfqpoint{3.276398in}{1.614345in}}%
\pgfpathlineto{\pgfqpoint{3.266342in}{1.608149in}}%
\pgfpathlineto{\pgfqpoint{3.258629in}{1.617098in}}%
\pgfpathlineto{\pgfqpoint{3.262436in}{1.621007in}}%
\pgfpathlineto{\pgfqpoint{3.252558in}{1.629161in}}%
\pgfpathlineto{\pgfqpoint{3.248335in}{1.629894in}}%
\pgfpathlineto{\pgfqpoint{3.244631in}{1.624617in}}%
\pgfpathlineto{\pgfqpoint{3.248038in}{1.619832in}}%
\pgfpathlineto{\pgfqpoint{3.234122in}{1.609162in}}%
\pgfpathlineto{\pgfqpoint{3.221218in}{1.609411in}}%
\pgfpathlineto{\pgfqpoint{3.217800in}{1.603350in}}%
\pgfpathlineto{\pgfqpoint{3.217800in}{1.596744in}}%
\pgfpathlineto{\pgfqpoint{3.213947in}{1.593817in}}%
\pgfpathlineto{\pgfqpoint{3.211501in}{1.598059in}}%
\pgfpathlineto{\pgfqpoint{3.206320in}{1.596141in}}%
\pgfpathlineto{\pgfqpoint{3.203466in}{1.605423in}}%
\pgfpathlineto{\pgfqpoint{3.195360in}{1.613257in}}%
\pgfpathlineto{\pgfqpoint{3.192630in}{1.620121in}}%
\pgfpathlineto{\pgfqpoint{3.195407in}{1.620855in}}%
\pgfpathlineto{\pgfqpoint{3.203745in}{1.634729in}}%
\pgfpathlineto{\pgfqpoint{3.209405in}{1.635945in}}%
\pgfpathlineto{\pgfqpoint{3.210875in}{1.652502in}}%
\pgfpathlineto{\pgfqpoint{3.210379in}{1.663470in}}%
\pgfpathlineto{\pgfqpoint{3.215154in}{1.664914in}}%
\pgfpathlineto{\pgfqpoint{3.253079in}{1.672195in}}%
\pgfpathlineto{\pgfqpoint{3.245833in}{1.699582in}}%
\pgfpathlineto{\pgfqpoint{3.252945in}{1.701154in}}%
\pgfpathlineto{\pgfqpoint{3.258600in}{1.697493in}}%
\pgfpathlineto{\pgfqpoint{3.260640in}{1.692420in}}%
\pgfpathlineto{\pgfqpoint{3.267733in}{1.692380in}}%
\pgfpathlineto{\pgfqpoint{3.274806in}{1.685909in}}%
\pgfpathlineto{\pgfqpoint{3.277583in}{1.676067in}}%
\pgfpathlineto{\pgfqpoint{3.282783in}{1.667998in}}%
\pgfpathlineto{\pgfqpoint{3.292838in}{1.664005in}}%
\pgfpathlineto{\pgfqpoint{3.298905in}{1.664653in}}%
\pgfpathlineto{\pgfqpoint{3.302353in}{1.660141in}}%
\pgfpathlineto{\pgfqpoint{3.295750in}{1.652950in}}%
\pgfpathlineto{\pgfqpoint{3.294577in}{1.644062in}}%
\pgfpathlineto{\pgfqpoint{3.284382in}{1.624557in}}%
\pgfpathclose%
\pgfusepath{fill}%
\end{pgfscope}%
\begin{pgfscope}%
\pgfpathrectangle{\pgfqpoint{0.100000in}{0.100000in}}{\pgfqpoint{3.608454in}{2.310000in}}%
\pgfusepath{clip}%
\pgfsetbuttcap%
\pgfsetmiterjoin%
\definecolor{currentfill}{rgb}{0.000000,0.364706,0.817647}%
\pgfsetfillcolor{currentfill}%
\pgfsetlinewidth{0.000000pt}%
\definecolor{currentstroke}{rgb}{0.000000,0.000000,0.000000}%
\pgfsetstrokecolor{currentstroke}%
\pgfsetstrokeopacity{0.000000}%
\pgfsetdash{}{0pt}%
\pgfpathmoveto{\pgfqpoint{1.858721in}{1.355685in}}%
\pgfpathlineto{\pgfqpoint{1.892889in}{1.354245in}}%
\pgfpathlineto{\pgfqpoint{1.892401in}{1.333545in}}%
\pgfpathlineto{\pgfqpoint{1.891797in}{1.319820in}}%
\pgfpathlineto{\pgfqpoint{1.824072in}{1.322793in}}%
\pgfpathlineto{\pgfqpoint{1.789033in}{1.324734in}}%
\pgfpathlineto{\pgfqpoint{1.791350in}{1.359031in}}%
\pgfpathlineto{\pgfqpoint{1.858721in}{1.355685in}}%
\pgfpathclose%
\pgfusepath{fill}%
\end{pgfscope}%
\begin{pgfscope}%
\pgfpathrectangle{\pgfqpoint{0.100000in}{0.100000in}}{\pgfqpoint{3.608454in}{2.310000in}}%
\pgfusepath{clip}%
\pgfsetbuttcap%
\pgfsetmiterjoin%
\definecolor{currentfill}{rgb}{0.000000,0.827451,0.586275}%
\pgfsetfillcolor{currentfill}%
\pgfsetlinewidth{0.000000pt}%
\definecolor{currentstroke}{rgb}{0.000000,0.000000,0.000000}%
\pgfsetstrokecolor{currentstroke}%
\pgfsetstrokeopacity{0.000000}%
\pgfsetdash{}{0pt}%
\pgfpathmoveto{\pgfqpoint{0.592949in}{0.431644in}}%
\pgfpathlineto{\pgfqpoint{0.591752in}{0.435712in}}%
\pgfpathlineto{\pgfqpoint{0.594116in}{0.436502in}}%
\pgfpathlineto{\pgfqpoint{0.596254in}{0.435454in}}%
\pgfpathlineto{\pgfqpoint{0.596813in}{0.431481in}}%
\pgfpathlineto{\pgfqpoint{0.595244in}{0.430874in}}%
\pgfpathlineto{\pgfqpoint{0.592949in}{0.431644in}}%
\pgfpathclose%
\pgfusepath{fill}%
\end{pgfscope}%
\begin{pgfscope}%
\pgfpathrectangle{\pgfqpoint{0.100000in}{0.100000in}}{\pgfqpoint{3.608454in}{2.310000in}}%
\pgfusepath{clip}%
\pgfsetbuttcap%
\pgfsetmiterjoin%
\definecolor{currentfill}{rgb}{0.000000,0.827451,0.586275}%
\pgfsetfillcolor{currentfill}%
\pgfsetlinewidth{0.000000pt}%
\definecolor{currentstroke}{rgb}{0.000000,0.000000,0.000000}%
\pgfsetstrokecolor{currentstroke}%
\pgfsetstrokeopacity{0.000000}%
\pgfsetdash{}{0pt}%
\pgfpathmoveto{\pgfqpoint{0.598016in}{0.429456in}}%
\pgfpathlineto{\pgfqpoint{0.599382in}{0.432202in}}%
\pgfpathlineto{\pgfqpoint{0.600924in}{0.431678in}}%
\pgfpathlineto{\pgfqpoint{0.601971in}{0.429670in}}%
\pgfpathlineto{\pgfqpoint{0.598016in}{0.429456in}}%
\pgfpathclose%
\pgfusepath{fill}%
\end{pgfscope}%
\begin{pgfscope}%
\pgfpathrectangle{\pgfqpoint{0.100000in}{0.100000in}}{\pgfqpoint{3.608454in}{2.310000in}}%
\pgfusepath{clip}%
\pgfsetbuttcap%
\pgfsetmiterjoin%
\definecolor{currentfill}{rgb}{0.000000,0.827451,0.586275}%
\pgfsetfillcolor{currentfill}%
\pgfsetlinewidth{0.000000pt}%
\definecolor{currentstroke}{rgb}{0.000000,0.000000,0.000000}%
\pgfsetstrokecolor{currentstroke}%
\pgfsetstrokeopacity{0.000000}%
\pgfsetdash{}{0pt}%
\pgfpathmoveto{\pgfqpoint{0.679037in}{0.397130in}}%
\pgfpathlineto{\pgfqpoint{0.677306in}{0.394866in}}%
\pgfpathlineto{\pgfqpoint{0.676295in}{0.398443in}}%
\pgfpathlineto{\pgfqpoint{0.673518in}{0.397756in}}%
\pgfpathlineto{\pgfqpoint{0.676763in}{0.402425in}}%
\pgfpathlineto{\pgfqpoint{0.680361in}{0.402575in}}%
\pgfpathlineto{\pgfqpoint{0.679428in}{0.400744in}}%
\pgfpathlineto{\pgfqpoint{0.680747in}{0.399061in}}%
\pgfpathlineto{\pgfqpoint{0.683725in}{0.397976in}}%
\pgfpathlineto{\pgfqpoint{0.681665in}{0.395162in}}%
\pgfpathlineto{\pgfqpoint{0.680358in}{0.397560in}}%
\pgfpathlineto{\pgfqpoint{0.679037in}{0.397130in}}%
\pgfpathclose%
\pgfusepath{fill}%
\end{pgfscope}%
\begin{pgfscope}%
\pgfpathrectangle{\pgfqpoint{0.100000in}{0.100000in}}{\pgfqpoint{3.608454in}{2.310000in}}%
\pgfusepath{clip}%
\pgfsetbuttcap%
\pgfsetmiterjoin%
\definecolor{currentfill}{rgb}{0.000000,0.827451,0.586275}%
\pgfsetfillcolor{currentfill}%
\pgfsetlinewidth{0.000000pt}%
\definecolor{currentstroke}{rgb}{0.000000,0.000000,0.000000}%
\pgfsetstrokecolor{currentstroke}%
\pgfsetstrokeopacity{0.000000}%
\pgfsetdash{}{0pt}%
\pgfpathmoveto{\pgfqpoint{0.650134in}{0.407855in}}%
\pgfpathlineto{\pgfqpoint{0.650363in}{0.410897in}}%
\pgfpathlineto{\pgfqpoint{0.653136in}{0.411103in}}%
\pgfpathlineto{\pgfqpoint{0.652771in}{0.408615in}}%
\pgfpathlineto{\pgfqpoint{0.650134in}{0.407855in}}%
\pgfpathclose%
\pgfusepath{fill}%
\end{pgfscope}%
\begin{pgfscope}%
\pgfpathrectangle{\pgfqpoint{0.100000in}{0.100000in}}{\pgfqpoint{3.608454in}{2.310000in}}%
\pgfusepath{clip}%
\pgfsetbuttcap%
\pgfsetmiterjoin%
\definecolor{currentfill}{rgb}{0.000000,0.827451,0.586275}%
\pgfsetfillcolor{currentfill}%
\pgfsetlinewidth{0.000000pt}%
\definecolor{currentstroke}{rgb}{0.000000,0.000000,0.000000}%
\pgfsetstrokecolor{currentstroke}%
\pgfsetstrokeopacity{0.000000}%
\pgfsetdash{}{0pt}%
\pgfpathmoveto{\pgfqpoint{0.676688in}{0.386431in}}%
\pgfpathlineto{\pgfqpoint{0.683893in}{0.390538in}}%
\pgfpathlineto{\pgfqpoint{0.685712in}{0.388279in}}%
\pgfpathlineto{\pgfqpoint{0.683374in}{0.387317in}}%
\pgfpathlineto{\pgfqpoint{0.676688in}{0.386431in}}%
\pgfpathclose%
\pgfusepath{fill}%
\end{pgfscope}%
\begin{pgfscope}%
\pgfpathrectangle{\pgfqpoint{0.100000in}{0.100000in}}{\pgfqpoint{3.608454in}{2.310000in}}%
\pgfusepath{clip}%
\pgfsetbuttcap%
\pgfsetmiterjoin%
\definecolor{currentfill}{rgb}{0.000000,0.827451,0.586275}%
\pgfsetfillcolor{currentfill}%
\pgfsetlinewidth{0.000000pt}%
\definecolor{currentstroke}{rgb}{0.000000,0.000000,0.000000}%
\pgfsetstrokecolor{currentstroke}%
\pgfsetstrokeopacity{0.000000}%
\pgfsetdash{}{0pt}%
\pgfpathmoveto{\pgfqpoint{0.698256in}{0.395576in}}%
\pgfpathlineto{\pgfqpoint{0.696998in}{0.397305in}}%
\pgfpathlineto{\pgfqpoint{0.701060in}{0.399308in}}%
\pgfpathlineto{\pgfqpoint{0.699966in}{0.402043in}}%
\pgfpathlineto{\pgfqpoint{0.697786in}{0.402037in}}%
\pgfpathlineto{\pgfqpoint{0.696446in}{0.403146in}}%
\pgfpathlineto{\pgfqpoint{0.691042in}{0.403748in}}%
\pgfpathlineto{\pgfqpoint{0.686539in}{0.403836in}}%
\pgfpathlineto{\pgfqpoint{0.684692in}{0.405317in}}%
\pgfpathlineto{\pgfqpoint{0.682874in}{0.404197in}}%
\pgfpathlineto{\pgfqpoint{0.680377in}{0.404275in}}%
\pgfpathlineto{\pgfqpoint{0.680706in}{0.406672in}}%
\pgfpathlineto{\pgfqpoint{0.675294in}{0.406801in}}%
\pgfpathlineto{\pgfqpoint{0.672429in}{0.407601in}}%
\pgfpathlineto{\pgfqpoint{0.669990in}{0.409978in}}%
\pgfpathlineto{\pgfqpoint{0.672102in}{0.411831in}}%
\pgfpathlineto{\pgfqpoint{0.676413in}{0.414087in}}%
\pgfpathlineto{\pgfqpoint{0.674276in}{0.416221in}}%
\pgfpathlineto{\pgfqpoint{0.672028in}{0.416339in}}%
\pgfpathlineto{\pgfqpoint{0.667446in}{0.412698in}}%
\pgfpathlineto{\pgfqpoint{0.664420in}{0.412115in}}%
\pgfpathlineto{\pgfqpoint{0.661755in}{0.412674in}}%
\pgfpathlineto{\pgfqpoint{0.660202in}{0.409906in}}%
\pgfpathlineto{\pgfqpoint{0.657982in}{0.409975in}}%
\pgfpathlineto{\pgfqpoint{0.652958in}{0.413217in}}%
\pgfpathlineto{\pgfqpoint{0.652756in}{0.415126in}}%
\pgfpathlineto{\pgfqpoint{0.654783in}{0.415315in}}%
\pgfpathlineto{\pgfqpoint{0.655313in}{0.417608in}}%
\pgfpathlineto{\pgfqpoint{0.654939in}{0.420817in}}%
\pgfpathlineto{\pgfqpoint{0.650913in}{0.416098in}}%
\pgfpathlineto{\pgfqpoint{0.650688in}{0.414349in}}%
\pgfpathlineto{\pgfqpoint{0.647513in}{0.414740in}}%
\pgfpathlineto{\pgfqpoint{0.645402in}{0.416502in}}%
\pgfpathlineto{\pgfqpoint{0.646738in}{0.419922in}}%
\pgfpathlineto{\pgfqpoint{0.645503in}{0.423049in}}%
\pgfpathlineto{\pgfqpoint{0.644228in}{0.422685in}}%
\pgfpathlineto{\pgfqpoint{0.644008in}{0.418053in}}%
\pgfpathlineto{\pgfqpoint{0.636879in}{0.418407in}}%
\pgfpathlineto{\pgfqpoint{0.638459in}{0.416408in}}%
\pgfpathlineto{\pgfqpoint{0.636161in}{0.415808in}}%
\pgfpathlineto{\pgfqpoint{0.635340in}{0.417833in}}%
\pgfpathlineto{\pgfqpoint{0.633246in}{0.416909in}}%
\pgfpathlineto{\pgfqpoint{0.629966in}{0.419033in}}%
\pgfpathlineto{\pgfqpoint{0.625891in}{0.422864in}}%
\pgfpathlineto{\pgfqpoint{0.621156in}{0.424136in}}%
\pgfpathlineto{\pgfqpoint{0.617033in}{0.423185in}}%
\pgfpathlineto{\pgfqpoint{0.614359in}{0.424662in}}%
\pgfpathlineto{\pgfqpoint{0.613000in}{0.429251in}}%
\pgfpathlineto{\pgfqpoint{0.614183in}{0.431739in}}%
\pgfpathlineto{\pgfqpoint{0.619670in}{0.431167in}}%
\pgfpathlineto{\pgfqpoint{0.626353in}{0.433684in}}%
\pgfpathlineto{\pgfqpoint{0.627719in}{0.431363in}}%
\pgfpathlineto{\pgfqpoint{0.630050in}{0.430192in}}%
\pgfpathlineto{\pgfqpoint{0.631614in}{0.430608in}}%
\pgfpathlineto{\pgfqpoint{0.636883in}{0.428698in}}%
\pgfpathlineto{\pgfqpoint{0.640000in}{0.425903in}}%
\pgfpathlineto{\pgfqpoint{0.638907in}{0.421869in}}%
\pgfpathlineto{\pgfqpoint{0.640826in}{0.421267in}}%
\pgfpathlineto{\pgfqpoint{0.643623in}{0.424929in}}%
\pgfpathlineto{\pgfqpoint{0.647952in}{0.423982in}}%
\pgfpathlineto{\pgfqpoint{0.649233in}{0.421491in}}%
\pgfpathlineto{\pgfqpoint{0.651625in}{0.421666in}}%
\pgfpathlineto{\pgfqpoint{0.667655in}{0.424721in}}%
\pgfpathlineto{\pgfqpoint{0.673676in}{0.424199in}}%
\pgfpathlineto{\pgfqpoint{0.676435in}{0.424381in}}%
\pgfpathlineto{\pgfqpoint{0.682895in}{0.421450in}}%
\pgfpathlineto{\pgfqpoint{0.686743in}{0.417731in}}%
\pgfpathlineto{\pgfqpoint{0.685490in}{0.416186in}}%
\pgfpathlineto{\pgfqpoint{0.685449in}{0.411697in}}%
\pgfpathlineto{\pgfqpoint{0.687707in}{0.412299in}}%
\pgfpathlineto{\pgfqpoint{0.691151in}{0.410326in}}%
\pgfpathlineto{\pgfqpoint{0.690257in}{0.408099in}}%
\pgfpathlineto{\pgfqpoint{0.691712in}{0.406391in}}%
\pgfpathlineto{\pgfqpoint{0.693589in}{0.407652in}}%
\pgfpathlineto{\pgfqpoint{0.692837in}{0.410463in}}%
\pgfpathlineto{\pgfqpoint{0.691256in}{0.412047in}}%
\pgfpathlineto{\pgfqpoint{0.692624in}{0.413404in}}%
\pgfpathlineto{\pgfqpoint{0.696548in}{0.415075in}}%
\pgfpathlineto{\pgfqpoint{0.699114in}{0.416794in}}%
\pgfpathlineto{\pgfqpoint{0.705119in}{0.417836in}}%
\pgfpathlineto{\pgfqpoint{0.708125in}{0.417265in}}%
\pgfpathlineto{\pgfqpoint{0.711215in}{0.417647in}}%
\pgfpathlineto{\pgfqpoint{0.714242in}{0.417166in}}%
\pgfpathlineto{\pgfqpoint{0.721904in}{0.414573in}}%
\pgfpathlineto{\pgfqpoint{0.721574in}{0.415801in}}%
\pgfpathlineto{\pgfqpoint{0.725765in}{0.415354in}}%
\pgfpathlineto{\pgfqpoint{0.727666in}{0.413737in}}%
\pgfpathlineto{\pgfqpoint{0.725147in}{0.414065in}}%
\pgfpathlineto{\pgfqpoint{0.723593in}{0.412237in}}%
\pgfpathlineto{\pgfqpoint{0.720711in}{0.414781in}}%
\pgfpathlineto{\pgfqpoint{0.719039in}{0.412825in}}%
\pgfpathlineto{\pgfqpoint{0.715284in}{0.416046in}}%
\pgfpathlineto{\pgfqpoint{0.713680in}{0.414200in}}%
\pgfpathlineto{\pgfqpoint{0.710015in}{0.417271in}}%
\pgfpathlineto{\pgfqpoint{0.706826in}{0.413694in}}%
\pgfpathlineto{\pgfqpoint{0.707923in}{0.412710in}}%
\pgfpathlineto{\pgfqpoint{0.701711in}{0.405421in}}%
\pgfpathlineto{\pgfqpoint{0.699223in}{0.404064in}}%
\pgfpathlineto{\pgfqpoint{0.702855in}{0.400929in}}%
\pgfpathlineto{\pgfqpoint{0.698256in}{0.395576in}}%
\pgfpathclose%
\pgfusepath{fill}%
\end{pgfscope}%
\begin{pgfscope}%
\pgfpathrectangle{\pgfqpoint{0.100000in}{0.100000in}}{\pgfqpoint{3.608454in}{2.310000in}}%
\pgfusepath{clip}%
\pgfsetbuttcap%
\pgfsetmiterjoin%
\definecolor{currentfill}{rgb}{0.000000,0.478431,0.760784}%
\pgfsetfillcolor{currentfill}%
\pgfsetlinewidth{0.000000pt}%
\definecolor{currentstroke}{rgb}{0.000000,0.000000,0.000000}%
\pgfsetstrokecolor{currentstroke}%
\pgfsetstrokeopacity{0.000000}%
\pgfsetdash{}{0pt}%
\pgfpathmoveto{\pgfqpoint{1.886398in}{1.921692in}}%
\pgfpathlineto{\pgfqpoint{1.885060in}{1.894397in}}%
\pgfpathlineto{\pgfqpoint{1.845659in}{1.896366in}}%
\pgfpathlineto{\pgfqpoint{1.802493in}{1.899008in}}%
\pgfpathlineto{\pgfqpoint{1.796957in}{1.908802in}}%
\pgfpathlineto{\pgfqpoint{1.796911in}{1.916921in}}%
\pgfpathlineto{\pgfqpoint{1.800710in}{1.922898in}}%
\pgfpathlineto{\pgfqpoint{1.801655in}{1.932573in}}%
\pgfpathlineto{\pgfqpoint{1.799830in}{1.937828in}}%
\pgfpathlineto{\pgfqpoint{1.802847in}{1.946100in}}%
\pgfpathlineto{\pgfqpoint{1.801850in}{1.950975in}}%
\pgfpathlineto{\pgfqpoint{1.796923in}{1.954510in}}%
\pgfpathlineto{\pgfqpoint{1.828664in}{1.952547in}}%
\pgfpathlineto{\pgfqpoint{1.863197in}{1.950599in}}%
\pgfpathlineto{\pgfqpoint{1.885796in}{1.949443in}}%
\pgfpathlineto{\pgfqpoint{1.884519in}{1.921788in}}%
\pgfpathlineto{\pgfqpoint{1.886398in}{1.921692in}}%
\pgfpathclose%
\pgfusepath{fill}%
\end{pgfscope}%
\begin{pgfscope}%
\pgfpathrectangle{\pgfqpoint{0.100000in}{0.100000in}}{\pgfqpoint{3.608454in}{2.310000in}}%
\pgfusepath{clip}%
\pgfsetbuttcap%
\pgfsetmiterjoin%
\definecolor{currentfill}{rgb}{0.000000,0.356863,0.821569}%
\pgfsetfillcolor{currentfill}%
\pgfsetlinewidth{0.000000pt}%
\definecolor{currentstroke}{rgb}{0.000000,0.000000,0.000000}%
\pgfsetstrokecolor{currentstroke}%
\pgfsetstrokeopacity{0.000000}%
\pgfsetdash{}{0pt}%
\pgfpathmoveto{\pgfqpoint{1.790983in}{1.190359in}}%
\pgfpathlineto{\pgfqpoint{1.788727in}{1.150707in}}%
\pgfpathlineto{\pgfqpoint{1.728388in}{1.154348in}}%
\pgfpathlineto{\pgfqpoint{1.731511in}{1.193664in}}%
\pgfpathlineto{\pgfqpoint{1.723873in}{1.194151in}}%
\pgfpathlineto{\pgfqpoint{1.725856in}{1.224921in}}%
\pgfpathlineto{\pgfqpoint{1.751890in}{1.223219in}}%
\pgfpathlineto{\pgfqpoint{1.752303in}{1.230115in}}%
\pgfpathlineto{\pgfqpoint{1.786488in}{1.228084in}}%
\pgfpathlineto{\pgfqpoint{1.787082in}{1.221084in}}%
\pgfpathlineto{\pgfqpoint{1.785504in}{1.190712in}}%
\pgfpathlineto{\pgfqpoint{1.790983in}{1.190359in}}%
\pgfpathclose%
\pgfusepath{fill}%
\end{pgfscope}%
\begin{pgfscope}%
\pgfpathrectangle{\pgfqpoint{0.100000in}{0.100000in}}{\pgfqpoint{3.608454in}{2.310000in}}%
\pgfusepath{clip}%
\pgfsetbuttcap%
\pgfsetmiterjoin%
\definecolor{currentfill}{rgb}{0.000000,0.525490,0.737255}%
\pgfsetfillcolor{currentfill}%
\pgfsetlinewidth{0.000000pt}%
\definecolor{currentstroke}{rgb}{0.000000,0.000000,0.000000}%
\pgfsetstrokecolor{currentstroke}%
\pgfsetstrokeopacity{0.000000}%
\pgfsetdash{}{0pt}%
\pgfpathmoveto{\pgfqpoint{2.084392in}{1.676317in}}%
\pgfpathlineto{\pgfqpoint{2.057034in}{1.676704in}}%
\pgfpathlineto{\pgfqpoint{2.057273in}{1.695912in}}%
\pgfpathlineto{\pgfqpoint{2.046244in}{1.696054in}}%
\pgfpathlineto{\pgfqpoint{2.046612in}{1.723715in}}%
\pgfpathlineto{\pgfqpoint{2.046268in}{1.751283in}}%
\pgfpathlineto{\pgfqpoint{2.073025in}{1.750974in}}%
\pgfpathlineto{\pgfqpoint{2.100603in}{1.750815in}}%
\pgfpathlineto{\pgfqpoint{2.101083in}{1.743915in}}%
\pgfpathlineto{\pgfqpoint{2.114776in}{1.743905in}}%
\pgfpathlineto{\pgfqpoint{2.115120in}{1.723283in}}%
\pgfpathlineto{\pgfqpoint{2.115164in}{1.695734in}}%
\pgfpathlineto{\pgfqpoint{2.111687in}{1.695719in}}%
\pgfpathlineto{\pgfqpoint{2.084490in}{1.695746in}}%
\pgfpathlineto{\pgfqpoint{2.084392in}{1.676317in}}%
\pgfpathclose%
\pgfusepath{fill}%
\end{pgfscope}%
\begin{pgfscope}%
\pgfpathrectangle{\pgfqpoint{0.100000in}{0.100000in}}{\pgfqpoint{3.608454in}{2.310000in}}%
\pgfusepath{clip}%
\pgfsetbuttcap%
\pgfsetmiterjoin%
\definecolor{currentfill}{rgb}{0.000000,0.427451,0.786275}%
\pgfsetfillcolor{currentfill}%
\pgfsetlinewidth{0.000000pt}%
\definecolor{currentstroke}{rgb}{0.000000,0.000000,0.000000}%
\pgfsetstrokecolor{currentstroke}%
\pgfsetstrokeopacity{0.000000}%
\pgfsetdash{}{0pt}%
\pgfpathmoveto{\pgfqpoint{2.138550in}{1.463912in}}%
\pgfpathlineto{\pgfqpoint{2.128892in}{1.463893in}}%
\pgfpathlineto{\pgfqpoint{2.060821in}{1.465149in}}%
\pgfpathlineto{\pgfqpoint{2.059082in}{1.471267in}}%
\pgfpathlineto{\pgfqpoint{2.053550in}{1.477162in}}%
\pgfpathlineto{\pgfqpoint{2.056854in}{1.480886in}}%
\pgfpathlineto{\pgfqpoint{2.058174in}{1.490199in}}%
\pgfpathlineto{\pgfqpoint{2.083784in}{1.490019in}}%
\pgfpathlineto{\pgfqpoint{2.083971in}{1.510379in}}%
\pgfpathlineto{\pgfqpoint{2.097514in}{1.510367in}}%
\pgfpathlineto{\pgfqpoint{2.124622in}{1.510289in}}%
\pgfpathlineto{\pgfqpoint{2.138316in}{1.510307in}}%
\pgfpathlineto{\pgfqpoint{2.138550in}{1.463912in}}%
\pgfpathclose%
\pgfusepath{fill}%
\end{pgfscope}%
\begin{pgfscope}%
\pgfpathrectangle{\pgfqpoint{0.100000in}{0.100000in}}{\pgfqpoint{3.608454in}{2.310000in}}%
\pgfusepath{clip}%
\pgfsetbuttcap%
\pgfsetmiterjoin%
\definecolor{currentfill}{rgb}{0.000000,0.556863,0.721569}%
\pgfsetfillcolor{currentfill}%
\pgfsetlinewidth{0.000000pt}%
\definecolor{currentstroke}{rgb}{0.000000,0.000000,0.000000}%
\pgfsetstrokecolor{currentstroke}%
\pgfsetstrokeopacity{0.000000}%
\pgfsetdash{}{0pt}%
\pgfpathmoveto{\pgfqpoint{1.941489in}{0.354044in}}%
\pgfpathlineto{\pgfqpoint{1.939142in}{0.354101in}}%
\pgfpathlineto{\pgfqpoint{1.934663in}{0.372389in}}%
\pgfpathlineto{\pgfqpoint{1.935228in}{0.399177in}}%
\pgfpathlineto{\pgfqpoint{1.938012in}{0.416549in}}%
\pgfpathlineto{\pgfqpoint{1.956579in}{0.451354in}}%
\pgfpathlineto{\pgfqpoint{1.961353in}{0.455090in}}%
\pgfpathlineto{\pgfqpoint{1.966578in}{0.467186in}}%
\pgfpathlineto{\pgfqpoint{1.973872in}{0.476940in}}%
\pgfpathlineto{\pgfqpoint{1.976992in}{0.475338in}}%
\pgfpathlineto{\pgfqpoint{1.975277in}{0.471174in}}%
\pgfpathlineto{\pgfqpoint{1.962738in}{0.456890in}}%
\pgfpathlineto{\pgfqpoint{1.952264in}{0.440704in}}%
\pgfpathlineto{\pgfqpoint{1.945455in}{0.427440in}}%
\pgfpathlineto{\pgfqpoint{1.940167in}{0.414451in}}%
\pgfpathlineto{\pgfqpoint{1.937055in}{0.402500in}}%
\pgfpathlineto{\pgfqpoint{1.935766in}{0.383962in}}%
\pgfpathlineto{\pgfqpoint{1.938010in}{0.367087in}}%
\pgfpathlineto{\pgfqpoint{1.941489in}{0.354044in}}%
\pgfpathclose%
\pgfusepath{fill}%
\end{pgfscope}%
\begin{pgfscope}%
\pgfpathrectangle{\pgfqpoint{0.100000in}{0.100000in}}{\pgfqpoint{3.608454in}{2.310000in}}%
\pgfusepath{clip}%
\pgfsetbuttcap%
\pgfsetmiterjoin%
\definecolor{currentfill}{rgb}{0.000000,0.556863,0.721569}%
\pgfsetfillcolor{currentfill}%
\pgfsetlinewidth{0.000000pt}%
\definecolor{currentstroke}{rgb}{0.000000,0.000000,0.000000}%
\pgfsetstrokecolor{currentstroke}%
\pgfsetstrokeopacity{0.000000}%
\pgfsetdash{}{0pt}%
\pgfpathmoveto{\pgfqpoint{1.930477in}{0.354309in}}%
\pgfpathlineto{\pgfqpoint{1.899944in}{0.355045in}}%
\pgfpathlineto{\pgfqpoint{1.891060in}{0.356772in}}%
\pgfpathlineto{\pgfqpoint{1.891484in}{0.370110in}}%
\pgfpathlineto{\pgfqpoint{1.859982in}{0.371433in}}%
\pgfpathlineto{\pgfqpoint{1.861148in}{0.393352in}}%
\pgfpathlineto{\pgfqpoint{1.857651in}{0.393438in}}%
\pgfpathlineto{\pgfqpoint{1.856313in}{0.408047in}}%
\pgfpathlineto{\pgfqpoint{1.852427in}{0.418217in}}%
\pgfpathlineto{\pgfqpoint{1.834828in}{0.418520in}}%
\pgfpathlineto{\pgfqpoint{1.836840in}{0.475148in}}%
\pgfpathlineto{\pgfqpoint{1.902126in}{0.472699in}}%
\pgfpathlineto{\pgfqpoint{1.900788in}{0.477356in}}%
\pgfpathlineto{\pgfqpoint{1.907060in}{0.482191in}}%
\pgfpathlineto{\pgfqpoint{1.924683in}{0.477880in}}%
\pgfpathlineto{\pgfqpoint{1.938836in}{0.498314in}}%
\pgfpathlineto{\pgfqpoint{1.954355in}{0.511185in}}%
\pgfpathlineto{\pgfqpoint{1.962232in}{0.510712in}}%
\pgfpathlineto{\pgfqpoint{1.966594in}{0.506597in}}%
\pgfpathlineto{\pgfqpoint{1.973994in}{0.506774in}}%
\pgfpathlineto{\pgfqpoint{1.981043in}{0.498344in}}%
\pgfpathlineto{\pgfqpoint{1.979461in}{0.488391in}}%
\pgfpathlineto{\pgfqpoint{1.978605in}{0.483866in}}%
\pgfpathlineto{\pgfqpoint{1.969758in}{0.476284in}}%
\pgfpathlineto{\pgfqpoint{1.965121in}{0.476210in}}%
\pgfpathlineto{\pgfqpoint{1.961299in}{0.468390in}}%
\pgfpathlineto{\pgfqpoint{1.951003in}{0.452565in}}%
\pgfpathlineto{\pgfqpoint{1.945104in}{0.457357in}}%
\pgfpathlineto{\pgfqpoint{1.938160in}{0.454348in}}%
\pgfpathlineto{\pgfqpoint{1.938434in}{0.445675in}}%
\pgfpathlineto{\pgfqpoint{1.946055in}{0.442493in}}%
\pgfpathlineto{\pgfqpoint{1.941015in}{0.432395in}}%
\pgfpathlineto{\pgfqpoint{1.934658in}{0.413330in}}%
\pgfpathlineto{\pgfqpoint{1.931292in}{0.396174in}}%
\pgfpathlineto{\pgfqpoint{1.923085in}{0.385830in}}%
\pgfpathlineto{\pgfqpoint{1.922580in}{0.373704in}}%
\pgfpathlineto{\pgfqpoint{1.928475in}{0.367207in}}%
\pgfpathlineto{\pgfqpoint{1.930477in}{0.354309in}}%
\pgfpathclose%
\pgfusepath{fill}%
\end{pgfscope}%
\begin{pgfscope}%
\pgfpathrectangle{\pgfqpoint{0.100000in}{0.100000in}}{\pgfqpoint{3.608454in}{2.310000in}}%
\pgfusepath{clip}%
\pgfsetbuttcap%
\pgfsetmiterjoin%
\definecolor{currentfill}{rgb}{0.000000,0.470588,0.764706}%
\pgfsetfillcolor{currentfill}%
\pgfsetlinewidth{0.000000pt}%
\definecolor{currentstroke}{rgb}{0.000000,0.000000,0.000000}%
\pgfsetstrokecolor{currentstroke}%
\pgfsetstrokeopacity{0.000000}%
\pgfsetdash{}{0pt}%
\pgfpathmoveto{\pgfqpoint{2.023670in}{1.724067in}}%
\pgfpathlineto{\pgfqpoint{1.998767in}{1.724416in}}%
\pgfpathlineto{\pgfqpoint{1.999302in}{1.751993in}}%
\pgfpathlineto{\pgfqpoint{1.944686in}{1.753364in}}%
\pgfpathlineto{\pgfqpoint{1.944449in}{1.753368in}}%
\pgfpathlineto{\pgfqpoint{1.945343in}{1.781068in}}%
\pgfpathlineto{\pgfqpoint{1.986361in}{1.779846in}}%
\pgfpathlineto{\pgfqpoint{2.024621in}{1.779133in}}%
\pgfpathlineto{\pgfqpoint{2.023670in}{1.724067in}}%
\pgfpathclose%
\pgfusepath{fill}%
\end{pgfscope}%
\begin{pgfscope}%
\pgfpathrectangle{\pgfqpoint{0.100000in}{0.100000in}}{\pgfqpoint{3.608454in}{2.310000in}}%
\pgfusepath{clip}%
\pgfsetbuttcap%
\pgfsetmiterjoin%
\definecolor{currentfill}{rgb}{0.000000,0.862745,0.568627}%
\pgfsetfillcolor{currentfill}%
\pgfsetlinewidth{0.000000pt}%
\definecolor{currentstroke}{rgb}{0.000000,0.000000,0.000000}%
\pgfsetstrokecolor{currentstroke}%
\pgfsetstrokeopacity{0.000000}%
\pgfsetdash{}{0pt}%
\pgfpathmoveto{\pgfqpoint{2.313347in}{1.069167in}}%
\pgfpathlineto{\pgfqpoint{2.313444in}{1.062948in}}%
\pgfpathlineto{\pgfqpoint{2.310094in}{1.055119in}}%
\pgfpathlineto{\pgfqpoint{2.293330in}{1.054914in}}%
\pgfpathlineto{\pgfqpoint{2.269431in}{1.054652in}}%
\pgfpathlineto{\pgfqpoint{2.269359in}{1.061552in}}%
\pgfpathlineto{\pgfqpoint{2.245419in}{1.061527in}}%
\pgfpathlineto{\pgfqpoint{2.247654in}{1.068396in}}%
\pgfpathlineto{\pgfqpoint{2.247558in}{1.082128in}}%
\pgfpathlineto{\pgfqpoint{2.259040in}{1.082097in}}%
\pgfpathlineto{\pgfqpoint{2.258979in}{1.087834in}}%
\pgfpathlineto{\pgfqpoint{2.272664in}{1.088030in}}%
\pgfpathlineto{\pgfqpoint{2.272396in}{1.102853in}}%
\pgfpathlineto{\pgfqpoint{2.279258in}{1.102907in}}%
\pgfpathlineto{\pgfqpoint{2.279199in}{1.109799in}}%
\pgfpathlineto{\pgfqpoint{2.286066in}{1.109865in}}%
\pgfpathlineto{\pgfqpoint{2.286012in}{1.115699in}}%
\pgfpathlineto{\pgfqpoint{2.295109in}{1.108191in}}%
\pgfpathlineto{\pgfqpoint{2.293303in}{1.102454in}}%
\pgfpathlineto{\pgfqpoint{2.302022in}{1.100303in}}%
\pgfpathlineto{\pgfqpoint{2.308868in}{1.093507in}}%
\pgfpathlineto{\pgfqpoint{2.307635in}{1.090812in}}%
\pgfpathlineto{\pgfqpoint{2.313117in}{1.083232in}}%
\pgfpathlineto{\pgfqpoint{2.313347in}{1.069167in}}%
\pgfpathclose%
\pgfusepath{fill}%
\end{pgfscope}%
\begin{pgfscope}%
\pgfpathrectangle{\pgfqpoint{0.100000in}{0.100000in}}{\pgfqpoint{3.608454in}{2.310000in}}%
\pgfusepath{clip}%
\pgfsetbuttcap%
\pgfsetmiterjoin%
\definecolor{currentfill}{rgb}{0.000000,0.568627,0.715686}%
\pgfsetfillcolor{currentfill}%
\pgfsetlinewidth{0.000000pt}%
\definecolor{currentstroke}{rgb}{0.000000,0.000000,0.000000}%
\pgfsetstrokecolor{currentstroke}%
\pgfsetstrokeopacity{0.000000}%
\pgfsetdash{}{0pt}%
\pgfpathmoveto{\pgfqpoint{2.958909in}{1.001900in}}%
\pgfpathlineto{\pgfqpoint{2.949644in}{0.997974in}}%
\pgfpathlineto{\pgfqpoint{2.953478in}{0.990526in}}%
\pgfpathlineto{\pgfqpoint{2.950161in}{0.981612in}}%
\pgfpathlineto{\pgfqpoint{2.956878in}{0.975335in}}%
\pgfpathlineto{\pgfqpoint{2.954883in}{0.971559in}}%
\pgfpathlineto{\pgfqpoint{2.948243in}{0.978295in}}%
\pgfpathlineto{\pgfqpoint{2.941645in}{0.986529in}}%
\pgfpathlineto{\pgfqpoint{2.922696in}{0.996327in}}%
\pgfpathlineto{\pgfqpoint{2.919954in}{1.001789in}}%
\pgfpathlineto{\pgfqpoint{2.910642in}{1.010268in}}%
\pgfpathlineto{\pgfqpoint{2.907921in}{1.017944in}}%
\pgfpathlineto{\pgfqpoint{2.895066in}{1.035008in}}%
\pgfpathlineto{\pgfqpoint{2.888625in}{1.033073in}}%
\pgfpathlineto{\pgfqpoint{2.884093in}{1.034847in}}%
\pgfpathlineto{\pgfqpoint{2.876324in}{1.042276in}}%
\pgfpathlineto{\pgfqpoint{2.872404in}{1.042068in}}%
\pgfpathlineto{\pgfqpoint{2.864794in}{1.047108in}}%
\pgfpathlineto{\pgfqpoint{2.863588in}{1.049271in}}%
\pgfpathlineto{\pgfqpoint{2.864683in}{1.055271in}}%
\pgfpathlineto{\pgfqpoint{2.869918in}{1.063176in}}%
\pgfpathlineto{\pgfqpoint{2.876200in}{1.069013in}}%
\pgfpathlineto{\pgfqpoint{2.876403in}{1.073700in}}%
\pgfpathlineto{\pgfqpoint{2.882548in}{1.076650in}}%
\pgfpathlineto{\pgfqpoint{2.903153in}{1.086612in}}%
\pgfpathlineto{\pgfqpoint{2.922867in}{1.095204in}}%
\pgfpathlineto{\pgfqpoint{2.931616in}{1.096854in}}%
\pgfpathlineto{\pgfqpoint{2.934795in}{1.069673in}}%
\pgfpathlineto{\pgfqpoint{2.951248in}{1.056924in}}%
\pgfpathlineto{\pgfqpoint{2.957922in}{1.053177in}}%
\pgfpathlineto{\pgfqpoint{2.975999in}{1.050234in}}%
\pgfpathlineto{\pgfqpoint{2.968957in}{1.035717in}}%
\pgfpathlineto{\pgfqpoint{2.962380in}{1.029555in}}%
\pgfpathlineto{\pgfqpoint{2.961263in}{1.022919in}}%
\pgfpathlineto{\pgfqpoint{2.963991in}{1.017466in}}%
\pgfpathlineto{\pgfqpoint{2.958909in}{1.001900in}}%
\pgfpathclose%
\pgfusepath{fill}%
\end{pgfscope}%
\begin{pgfscope}%
\pgfpathrectangle{\pgfqpoint{0.100000in}{0.100000in}}{\pgfqpoint{3.608454in}{2.310000in}}%
\pgfusepath{clip}%
\pgfsetbuttcap%
\pgfsetmiterjoin%
\definecolor{currentfill}{rgb}{0.000000,0.407843,0.796078}%
\pgfsetfillcolor{currentfill}%
\pgfsetlinewidth{0.000000pt}%
\definecolor{currentstroke}{rgb}{0.000000,0.000000,0.000000}%
\pgfsetstrokecolor{currentstroke}%
\pgfsetstrokeopacity{0.000000}%
\pgfsetdash{}{0pt}%
\pgfpathmoveto{\pgfqpoint{2.684273in}{1.275179in}}%
\pgfpathlineto{\pgfqpoint{2.678952in}{1.278333in}}%
\pgfpathlineto{\pgfqpoint{2.675593in}{1.285078in}}%
\pgfpathlineto{\pgfqpoint{2.669652in}{1.289191in}}%
\pgfpathlineto{\pgfqpoint{2.655765in}{1.288068in}}%
\pgfpathlineto{\pgfqpoint{2.649917in}{1.290900in}}%
\pgfpathlineto{\pgfqpoint{2.648711in}{1.297118in}}%
\pgfpathlineto{\pgfqpoint{2.643610in}{1.301513in}}%
\pgfpathlineto{\pgfqpoint{2.640611in}{1.294225in}}%
\pgfpathlineto{\pgfqpoint{2.636960in}{1.295013in}}%
\pgfpathlineto{\pgfqpoint{2.636434in}{1.301852in}}%
\pgfpathlineto{\pgfqpoint{2.629550in}{1.301264in}}%
\pgfpathlineto{\pgfqpoint{2.629213in}{1.305901in}}%
\pgfpathlineto{\pgfqpoint{2.622478in}{1.305021in}}%
\pgfpathlineto{\pgfqpoint{2.621408in}{1.315375in}}%
\pgfpathlineto{\pgfqpoint{2.644392in}{1.317477in}}%
\pgfpathlineto{\pgfqpoint{2.641760in}{1.344149in}}%
\pgfpathlineto{\pgfqpoint{2.643662in}{1.346766in}}%
\pgfpathlineto{\pgfqpoint{2.662944in}{1.348688in}}%
\pgfpathlineto{\pgfqpoint{2.667619in}{1.346789in}}%
\pgfpathlineto{\pgfqpoint{2.669050in}{1.334303in}}%
\pgfpathlineto{\pgfqpoint{2.674475in}{1.337189in}}%
\pgfpathlineto{\pgfqpoint{2.688201in}{1.338715in}}%
\pgfpathlineto{\pgfqpoint{2.697081in}{1.338065in}}%
\pgfpathlineto{\pgfqpoint{2.697292in}{1.333124in}}%
\pgfpathlineto{\pgfqpoint{2.707002in}{1.331992in}}%
\pgfpathlineto{\pgfqpoint{2.713052in}{1.339727in}}%
\pgfpathlineto{\pgfqpoint{2.718732in}{1.341235in}}%
\pgfpathlineto{\pgfqpoint{2.722897in}{1.338929in}}%
\pgfpathlineto{\pgfqpoint{2.724859in}{1.333241in}}%
\pgfpathlineto{\pgfqpoint{2.730992in}{1.324932in}}%
\pgfpathlineto{\pgfqpoint{2.726160in}{1.316456in}}%
\pgfpathlineto{\pgfqpoint{2.726400in}{1.307715in}}%
\pgfpathlineto{\pgfqpoint{2.724571in}{1.300390in}}%
\pgfpathlineto{\pgfqpoint{2.718323in}{1.291481in}}%
\pgfpathlineto{\pgfqpoint{2.704197in}{1.287428in}}%
\pgfpathlineto{\pgfqpoint{2.698255in}{1.290856in}}%
\pgfpathlineto{\pgfqpoint{2.690035in}{1.277558in}}%
\pgfpathlineto{\pgfqpoint{2.684273in}{1.275179in}}%
\pgfpathclose%
\pgfusepath{fill}%
\end{pgfscope}%
\begin{pgfscope}%
\pgfpathrectangle{\pgfqpoint{0.100000in}{0.100000in}}{\pgfqpoint{3.608454in}{2.310000in}}%
\pgfusepath{clip}%
\pgfsetbuttcap%
\pgfsetmiterjoin%
\definecolor{currentfill}{rgb}{0.000000,0.368627,0.815686}%
\pgfsetfillcolor{currentfill}%
\pgfsetlinewidth{0.000000pt}%
\definecolor{currentstroke}{rgb}{0.000000,0.000000,0.000000}%
\pgfsetstrokecolor{currentstroke}%
\pgfsetstrokeopacity{0.000000}%
\pgfsetdash{}{0pt}%
\pgfpathmoveto{\pgfqpoint{1.912766in}{1.573654in}}%
\pgfpathlineto{\pgfqpoint{1.912258in}{1.559919in}}%
\pgfpathlineto{\pgfqpoint{1.884919in}{1.560911in}}%
\pgfpathlineto{\pgfqpoint{1.885929in}{1.588436in}}%
\pgfpathlineto{\pgfqpoint{1.858155in}{1.589614in}}%
\pgfpathlineto{\pgfqpoint{1.860373in}{1.661671in}}%
\pgfpathlineto{\pgfqpoint{1.904126in}{1.659838in}}%
\pgfpathlineto{\pgfqpoint{1.905777in}{1.655731in}}%
\pgfpathlineto{\pgfqpoint{1.914725in}{1.650247in}}%
\pgfpathlineto{\pgfqpoint{1.914002in}{1.614954in}}%
\pgfpathlineto{\pgfqpoint{1.912766in}{1.573654in}}%
\pgfpathclose%
\pgfusepath{fill}%
\end{pgfscope}%
\begin{pgfscope}%
\pgfpathrectangle{\pgfqpoint{0.100000in}{0.100000in}}{\pgfqpoint{3.608454in}{2.310000in}}%
\pgfusepath{clip}%
\pgfsetbuttcap%
\pgfsetmiterjoin%
\definecolor{currentfill}{rgb}{0.000000,0.458824,0.770588}%
\pgfsetfillcolor{currentfill}%
\pgfsetlinewidth{0.000000pt}%
\definecolor{currentstroke}{rgb}{0.000000,0.000000,0.000000}%
\pgfsetstrokecolor{currentstroke}%
\pgfsetstrokeopacity{0.000000}%
\pgfsetdash{}{0pt}%
\pgfpathmoveto{\pgfqpoint{2.719775in}{1.481004in}}%
\pgfpathlineto{\pgfqpoint{2.720165in}{1.477649in}}%
\pgfpathlineto{\pgfqpoint{2.695336in}{1.474638in}}%
\pgfpathlineto{\pgfqpoint{2.694173in}{1.495218in}}%
\pgfpathlineto{\pgfqpoint{2.679497in}{1.493619in}}%
\pgfpathlineto{\pgfqpoint{2.678668in}{1.500441in}}%
\pgfpathlineto{\pgfqpoint{2.667350in}{1.499209in}}%
\pgfpathlineto{\pgfqpoint{2.664197in}{1.526649in}}%
\pgfpathlineto{\pgfqpoint{2.661863in}{1.526375in}}%
\pgfpathlineto{\pgfqpoint{2.660236in}{1.540245in}}%
\pgfpathlineto{\pgfqpoint{2.661344in}{1.549614in}}%
\pgfpathlineto{\pgfqpoint{2.668362in}{1.548067in}}%
\pgfpathlineto{\pgfqpoint{2.688778in}{1.550091in}}%
\pgfpathlineto{\pgfqpoint{2.684290in}{1.589030in}}%
\pgfpathlineto{\pgfqpoint{2.706079in}{1.591455in}}%
\pgfpathlineto{\pgfqpoint{2.707766in}{1.586556in}}%
\pgfpathlineto{\pgfqpoint{2.714189in}{1.531011in}}%
\pgfpathlineto{\pgfqpoint{2.719775in}{1.481004in}}%
\pgfpathclose%
\pgfusepath{fill}%
\end{pgfscope}%
\begin{pgfscope}%
\pgfpathrectangle{\pgfqpoint{0.100000in}{0.100000in}}{\pgfqpoint{3.608454in}{2.310000in}}%
\pgfusepath{clip}%
\pgfsetbuttcap%
\pgfsetmiterjoin%
\definecolor{currentfill}{rgb}{0.000000,1.000000,0.500000}%
\pgfsetfillcolor{currentfill}%
\pgfsetlinewidth{0.000000pt}%
\definecolor{currentstroke}{rgb}{0.000000,0.000000,0.000000}%
\pgfsetstrokecolor{currentstroke}%
\pgfsetstrokeopacity{0.000000}%
\pgfsetdash{}{0pt}%
\pgfpathmoveto{\pgfqpoint{2.837048in}{1.201444in}}%
\pgfpathlineto{\pgfqpoint{2.833845in}{1.206501in}}%
\pgfpathlineto{\pgfqpoint{2.833882in}{1.215723in}}%
\pgfpathlineto{\pgfqpoint{2.831285in}{1.222519in}}%
\pgfpathlineto{\pgfqpoint{2.826345in}{1.240166in}}%
\pgfpathlineto{\pgfqpoint{2.826981in}{1.247494in}}%
\pgfpathlineto{\pgfqpoint{2.825042in}{1.252479in}}%
\pgfpathlineto{\pgfqpoint{2.830889in}{1.256941in}}%
\pgfpathlineto{\pgfqpoint{2.833788in}{1.254755in}}%
\pgfpathlineto{\pgfqpoint{2.841576in}{1.260538in}}%
\pgfpathlineto{\pgfqpoint{2.850587in}{1.262241in}}%
\pgfpathlineto{\pgfqpoint{2.851560in}{1.271717in}}%
\pgfpathlineto{\pgfqpoint{2.860524in}{1.271364in}}%
\pgfpathlineto{\pgfqpoint{2.872542in}{1.266952in}}%
\pgfpathlineto{\pgfqpoint{2.874724in}{1.257922in}}%
\pgfpathlineto{\pgfqpoint{2.880671in}{1.252069in}}%
\pgfpathlineto{\pgfqpoint{2.887645in}{1.250541in}}%
\pgfpathlineto{\pgfqpoint{2.878670in}{1.243251in}}%
\pgfpathlineto{\pgfqpoint{2.878218in}{1.237030in}}%
\pgfpathlineto{\pgfqpoint{2.870950in}{1.230065in}}%
\pgfpathlineto{\pgfqpoint{2.871264in}{1.223809in}}%
\pgfpathlineto{\pgfqpoint{2.859365in}{1.219154in}}%
\pgfpathlineto{\pgfqpoint{2.856686in}{1.210206in}}%
\pgfpathlineto{\pgfqpoint{2.837048in}{1.201444in}}%
\pgfpathclose%
\pgfusepath{fill}%
\end{pgfscope}%
\begin{pgfscope}%
\pgfpathrectangle{\pgfqpoint{0.100000in}{0.100000in}}{\pgfqpoint{3.608454in}{2.310000in}}%
\pgfusepath{clip}%
\pgfsetbuttcap%
\pgfsetmiterjoin%
\definecolor{currentfill}{rgb}{0.000000,0.564706,0.717647}%
\pgfsetfillcolor{currentfill}%
\pgfsetlinewidth{0.000000pt}%
\definecolor{currentstroke}{rgb}{0.000000,0.000000,0.000000}%
\pgfsetstrokecolor{currentstroke}%
\pgfsetstrokeopacity{0.000000}%
\pgfsetdash{}{0pt}%
\pgfpathmoveto{\pgfqpoint{2.989806in}{1.055375in}}%
\pgfpathlineto{\pgfqpoint{2.990720in}{1.049995in}}%
\pgfpathlineto{\pgfqpoint{2.982517in}{1.046654in}}%
\pgfpathlineto{\pgfqpoint{2.975999in}{1.050234in}}%
\pgfpathlineto{\pgfqpoint{2.957922in}{1.053177in}}%
\pgfpathlineto{\pgfqpoint{2.951248in}{1.056924in}}%
\pgfpathlineto{\pgfqpoint{2.934795in}{1.069673in}}%
\pgfpathlineto{\pgfqpoint{2.931616in}{1.096854in}}%
\pgfpathlineto{\pgfqpoint{2.922867in}{1.095204in}}%
\pgfpathlineto{\pgfqpoint{2.922309in}{1.102781in}}%
\pgfpathlineto{\pgfqpoint{2.926537in}{1.111891in}}%
\pgfpathlineto{\pgfqpoint{2.932868in}{1.114078in}}%
\pgfpathlineto{\pgfqpoint{2.946773in}{1.103937in}}%
\pgfpathlineto{\pgfqpoint{2.947537in}{1.098430in}}%
\pgfpathlineto{\pgfqpoint{2.986399in}{1.102377in}}%
\pgfpathlineto{\pgfqpoint{2.985051in}{1.092680in}}%
\pgfpathlineto{\pgfqpoint{2.980220in}{1.091029in}}%
\pgfpathlineto{\pgfqpoint{2.984590in}{1.076632in}}%
\pgfpathlineto{\pgfqpoint{2.986559in}{1.063157in}}%
\pgfpathlineto{\pgfqpoint{2.989806in}{1.055375in}}%
\pgfpathclose%
\pgfusepath{fill}%
\end{pgfscope}%
\begin{pgfscope}%
\pgfpathrectangle{\pgfqpoint{0.100000in}{0.100000in}}{\pgfqpoint{3.608454in}{2.310000in}}%
\pgfusepath{clip}%
\pgfsetbuttcap%
\pgfsetmiterjoin%
\definecolor{currentfill}{rgb}{0.000000,0.827451,0.586275}%
\pgfsetfillcolor{currentfill}%
\pgfsetlinewidth{0.000000pt}%
\definecolor{currentstroke}{rgb}{0.000000,0.000000,0.000000}%
\pgfsetstrokecolor{currentstroke}%
\pgfsetstrokeopacity{0.000000}%
\pgfsetdash{}{0pt}%
\pgfpathmoveto{\pgfqpoint{1.037345in}{0.183729in}}%
\pgfpathlineto{\pgfqpoint{1.037725in}{0.186685in}}%
\pgfpathlineto{\pgfqpoint{1.039756in}{0.184678in}}%
\pgfpathlineto{\pgfqpoint{1.037345in}{0.183729in}}%
\pgfpathclose%
\pgfusepath{fill}%
\end{pgfscope}%
\begin{pgfscope}%
\pgfpathrectangle{\pgfqpoint{0.100000in}{0.100000in}}{\pgfqpoint{3.608454in}{2.310000in}}%
\pgfusepath{clip}%
\pgfsetbuttcap%
\pgfsetmiterjoin%
\definecolor{currentfill}{rgb}{0.000000,0.827451,0.586275}%
\pgfsetfillcolor{currentfill}%
\pgfsetlinewidth{0.000000pt}%
\definecolor{currentstroke}{rgb}{0.000000,0.000000,0.000000}%
\pgfsetstrokecolor{currentstroke}%
\pgfsetstrokeopacity{0.000000}%
\pgfsetdash{}{0pt}%
\pgfpathmoveto{\pgfqpoint{1.035979in}{0.195213in}}%
\pgfpathlineto{\pgfqpoint{1.035168in}{0.197283in}}%
\pgfpathlineto{\pgfqpoint{1.034662in}{0.206903in}}%
\pgfpathlineto{\pgfqpoint{1.035313in}{0.207991in}}%
\pgfpathlineto{\pgfqpoint{1.034954in}{0.212110in}}%
\pgfpathlineto{\pgfqpoint{1.035841in}{0.213389in}}%
\pgfpathlineto{\pgfqpoint{1.034185in}{0.215042in}}%
\pgfpathlineto{\pgfqpoint{1.034010in}{0.217336in}}%
\pgfpathlineto{\pgfqpoint{1.037398in}{0.218597in}}%
\pgfpathlineto{\pgfqpoint{1.038802in}{0.221050in}}%
\pgfpathlineto{\pgfqpoint{1.037009in}{0.222070in}}%
\pgfpathlineto{\pgfqpoint{1.036796in}{0.223966in}}%
\pgfpathlineto{\pgfqpoint{1.038521in}{0.226447in}}%
\pgfpathlineto{\pgfqpoint{1.036445in}{0.231584in}}%
\pgfpathlineto{\pgfqpoint{1.041639in}{0.235717in}}%
\pgfpathlineto{\pgfqpoint{1.043821in}{0.231654in}}%
\pgfpathlineto{\pgfqpoint{1.048322in}{0.228489in}}%
\pgfpathlineto{\pgfqpoint{1.045161in}{0.227298in}}%
\pgfpathlineto{\pgfqpoint{1.046187in}{0.224668in}}%
\pgfpathlineto{\pgfqpoint{1.045919in}{0.221757in}}%
\pgfpathlineto{\pgfqpoint{1.044595in}{0.217845in}}%
\pgfpathlineto{\pgfqpoint{1.044114in}{0.214196in}}%
\pgfpathlineto{\pgfqpoint{1.042540in}{0.210865in}}%
\pgfpathlineto{\pgfqpoint{1.043020in}{0.209624in}}%
\pgfpathlineto{\pgfqpoint{1.041842in}{0.206198in}}%
\pgfpathlineto{\pgfqpoint{1.039406in}{0.205292in}}%
\pgfpathlineto{\pgfqpoint{1.040077in}{0.203620in}}%
\pgfpathlineto{\pgfqpoint{1.037635in}{0.200811in}}%
\pgfpathlineto{\pgfqpoint{1.035979in}{0.195213in}}%
\pgfpathclose%
\pgfusepath{fill}%
\end{pgfscope}%
\begin{pgfscope}%
\pgfpathrectangle{\pgfqpoint{0.100000in}{0.100000in}}{\pgfqpoint{3.608454in}{2.310000in}}%
\pgfusepath{clip}%
\pgfsetbuttcap%
\pgfsetmiterjoin%
\definecolor{currentfill}{rgb}{0.000000,0.827451,0.586275}%
\pgfsetfillcolor{currentfill}%
\pgfsetlinewidth{0.000000pt}%
\definecolor{currentstroke}{rgb}{0.000000,0.000000,0.000000}%
\pgfsetstrokecolor{currentstroke}%
\pgfsetstrokeopacity{0.000000}%
\pgfsetdash{}{0pt}%
\pgfpathmoveto{\pgfqpoint{1.029215in}{0.224020in}}%
\pgfpathlineto{\pgfqpoint{1.032249in}{0.226756in}}%
\pgfpathlineto{\pgfqpoint{1.032356in}{0.230382in}}%
\pgfpathlineto{\pgfqpoint{1.033329in}{0.232191in}}%
\pgfpathlineto{\pgfqpoint{1.035308in}{0.231235in}}%
\pgfpathlineto{\pgfqpoint{1.034682in}{0.226435in}}%
\pgfpathlineto{\pgfqpoint{1.033872in}{0.224494in}}%
\pgfpathlineto{\pgfqpoint{1.031932in}{0.223434in}}%
\pgfpathlineto{\pgfqpoint{1.029215in}{0.224020in}}%
\pgfpathclose%
\pgfusepath{fill}%
\end{pgfscope}%
\begin{pgfscope}%
\pgfpathrectangle{\pgfqpoint{0.100000in}{0.100000in}}{\pgfqpoint{3.608454in}{2.310000in}}%
\pgfusepath{clip}%
\pgfsetbuttcap%
\pgfsetmiterjoin%
\definecolor{currentfill}{rgb}{0.000000,0.827451,0.586275}%
\pgfsetfillcolor{currentfill}%
\pgfsetlinewidth{0.000000pt}%
\definecolor{currentstroke}{rgb}{0.000000,0.000000,0.000000}%
\pgfsetstrokecolor{currentstroke}%
\pgfsetstrokeopacity{0.000000}%
\pgfsetdash{}{0pt}%
\pgfpathmoveto{\pgfqpoint{1.031898in}{0.249479in}}%
\pgfpathlineto{\pgfqpoint{1.037351in}{0.246961in}}%
\pgfpathlineto{\pgfqpoint{1.041099in}{0.250574in}}%
\pgfpathlineto{\pgfqpoint{1.042604in}{0.247304in}}%
\pgfpathlineto{\pgfqpoint{1.044554in}{0.244818in}}%
\pgfpathlineto{\pgfqpoint{1.047084in}{0.239857in}}%
\pgfpathlineto{\pgfqpoint{1.050384in}{0.238572in}}%
\pgfpathlineto{\pgfqpoint{1.051184in}{0.237535in}}%
\pgfpathlineto{\pgfqpoint{1.049176in}{0.229939in}}%
\pgfpathlineto{\pgfqpoint{1.046352in}{0.230420in}}%
\pgfpathlineto{\pgfqpoint{1.040103in}{0.241756in}}%
\pgfpathlineto{\pgfqpoint{1.039585in}{0.239633in}}%
\pgfpathlineto{\pgfqpoint{1.040523in}{0.237630in}}%
\pgfpathlineto{\pgfqpoint{1.039041in}{0.234732in}}%
\pgfpathlineto{\pgfqpoint{1.036002in}{0.232709in}}%
\pgfpathlineto{\pgfqpoint{1.034369in}{0.234138in}}%
\pgfpathlineto{\pgfqpoint{1.033903in}{0.237752in}}%
\pgfpathlineto{\pgfqpoint{1.034739in}{0.241664in}}%
\pgfpathlineto{\pgfqpoint{1.031898in}{0.249479in}}%
\pgfpathclose%
\pgfusepath{fill}%
\end{pgfscope}%
\begin{pgfscope}%
\pgfpathrectangle{\pgfqpoint{0.100000in}{0.100000in}}{\pgfqpoint{3.608454in}{2.310000in}}%
\pgfusepath{clip}%
\pgfsetbuttcap%
\pgfsetmiterjoin%
\definecolor{currentfill}{rgb}{0.000000,0.827451,0.586275}%
\pgfsetfillcolor{currentfill}%
\pgfsetlinewidth{0.000000pt}%
\definecolor{currentstroke}{rgb}{0.000000,0.000000,0.000000}%
\pgfsetstrokecolor{currentstroke}%
\pgfsetstrokeopacity{0.000000}%
\pgfsetdash{}{0pt}%
\pgfpathmoveto{\pgfqpoint{1.062273in}{0.185108in}}%
\pgfpathlineto{\pgfqpoint{1.060831in}{0.189361in}}%
\pgfpathlineto{\pgfqpoint{1.064048in}{0.191263in}}%
\pgfpathlineto{\pgfqpoint{1.066461in}{0.190126in}}%
\pgfpathlineto{\pgfqpoint{1.067412in}{0.187983in}}%
\pgfpathlineto{\pgfqpoint{1.065144in}{0.184845in}}%
\pgfpathlineto{\pgfqpoint{1.062273in}{0.185108in}}%
\pgfpathclose%
\pgfusepath{fill}%
\end{pgfscope}%
\begin{pgfscope}%
\pgfpathrectangle{\pgfqpoint{0.100000in}{0.100000in}}{\pgfqpoint{3.608454in}{2.310000in}}%
\pgfusepath{clip}%
\pgfsetbuttcap%
\pgfsetmiterjoin%
\definecolor{currentfill}{rgb}{0.000000,0.827451,0.586275}%
\pgfsetfillcolor{currentfill}%
\pgfsetlinewidth{0.000000pt}%
\definecolor{currentstroke}{rgb}{0.000000,0.000000,0.000000}%
\pgfsetstrokecolor{currentstroke}%
\pgfsetstrokeopacity{0.000000}%
\pgfsetdash{}{0pt}%
\pgfpathmoveto{\pgfqpoint{1.053060in}{0.199716in}}%
\pgfpathlineto{\pgfqpoint{1.049731in}{0.199089in}}%
\pgfpathlineto{\pgfqpoint{1.049501in}{0.194791in}}%
\pgfpathlineto{\pgfqpoint{1.047203in}{0.194053in}}%
\pgfpathlineto{\pgfqpoint{1.047508in}{0.191183in}}%
\pgfpathlineto{\pgfqpoint{1.045251in}{0.188644in}}%
\pgfpathlineto{\pgfqpoint{1.044390in}{0.191995in}}%
\pgfpathlineto{\pgfqpoint{1.042486in}{0.190633in}}%
\pgfpathlineto{\pgfqpoint{1.041396in}{0.191487in}}%
\pgfpathlineto{\pgfqpoint{1.042996in}{0.194874in}}%
\pgfpathlineto{\pgfqpoint{1.045532in}{0.196881in}}%
\pgfpathlineto{\pgfqpoint{1.047284in}{0.196813in}}%
\pgfpathlineto{\pgfqpoint{1.048561in}{0.199863in}}%
\pgfpathlineto{\pgfqpoint{1.047065in}{0.200403in}}%
\pgfpathlineto{\pgfqpoint{1.046014in}{0.202505in}}%
\pgfpathlineto{\pgfqpoint{1.046696in}{0.204982in}}%
\pgfpathlineto{\pgfqpoint{1.047059in}{0.209727in}}%
\pgfpathlineto{\pgfqpoint{1.050063in}{0.211128in}}%
\pgfpathlineto{\pgfqpoint{1.054012in}{0.204553in}}%
\pgfpathlineto{\pgfqpoint{1.054925in}{0.201896in}}%
\pgfpathlineto{\pgfqpoint{1.053060in}{0.199716in}}%
\pgfpathclose%
\pgfusepath{fill}%
\end{pgfscope}%
\begin{pgfscope}%
\pgfpathrectangle{\pgfqpoint{0.100000in}{0.100000in}}{\pgfqpoint{3.608454in}{2.310000in}}%
\pgfusepath{clip}%
\pgfsetbuttcap%
\pgfsetmiterjoin%
\definecolor{currentfill}{rgb}{0.000000,0.827451,0.586275}%
\pgfsetfillcolor{currentfill}%
\pgfsetlinewidth{0.000000pt}%
\definecolor{currentstroke}{rgb}{0.000000,0.000000,0.000000}%
\pgfsetstrokecolor{currentstroke}%
\pgfsetstrokeopacity{0.000000}%
\pgfsetdash{}{0pt}%
\pgfpathmoveto{\pgfqpoint{1.064056in}{0.213557in}}%
\pgfpathlineto{\pgfqpoint{1.067915in}{0.213771in}}%
\pgfpathlineto{\pgfqpoint{1.069651in}{0.211374in}}%
\pgfpathlineto{\pgfqpoint{1.071517in}{0.212096in}}%
\pgfpathlineto{\pgfqpoint{1.073207in}{0.214262in}}%
\pgfpathlineto{\pgfqpoint{1.079012in}{0.214678in}}%
\pgfpathlineto{\pgfqpoint{1.080487in}{0.211762in}}%
\pgfpathlineto{\pgfqpoint{1.082084in}{0.210879in}}%
\pgfpathlineto{\pgfqpoint{1.082183in}{0.207755in}}%
\pgfpathlineto{\pgfqpoint{1.079181in}{0.204297in}}%
\pgfpathlineto{\pgfqpoint{1.082604in}{0.201601in}}%
\pgfpathlineto{\pgfqpoint{1.080183in}{0.196814in}}%
\pgfpathlineto{\pgfqpoint{1.082403in}{0.194305in}}%
\pgfpathlineto{\pgfqpoint{1.081247in}{0.188005in}}%
\pgfpathlineto{\pgfqpoint{1.083332in}{0.187330in}}%
\pgfpathlineto{\pgfqpoint{1.085981in}{0.184345in}}%
\pgfpathlineto{\pgfqpoint{1.090049in}{0.178164in}}%
\pgfpathlineto{\pgfqpoint{1.089922in}{0.175542in}}%
\pgfpathlineto{\pgfqpoint{1.087857in}{0.173091in}}%
\pgfpathlineto{\pgfqpoint{1.085139in}{0.174080in}}%
\pgfpathlineto{\pgfqpoint{1.081805in}{0.173502in}}%
\pgfpathlineto{\pgfqpoint{1.082038in}{0.171601in}}%
\pgfpathlineto{\pgfqpoint{1.079838in}{0.170598in}}%
\pgfpathlineto{\pgfqpoint{1.078315in}{0.173884in}}%
\pgfpathlineto{\pgfqpoint{1.075481in}{0.173524in}}%
\pgfpathlineto{\pgfqpoint{1.073238in}{0.172315in}}%
\pgfpathlineto{\pgfqpoint{1.072944in}{0.170978in}}%
\pgfpathlineto{\pgfqpoint{1.070260in}{0.170374in}}%
\pgfpathlineto{\pgfqpoint{1.070834in}{0.168040in}}%
\pgfpathlineto{\pgfqpoint{1.069320in}{0.165633in}}%
\pgfpathlineto{\pgfqpoint{1.067405in}{0.165345in}}%
\pgfpathlineto{\pgfqpoint{1.065482in}{0.169372in}}%
\pgfpathlineto{\pgfqpoint{1.069168in}{0.169743in}}%
\pgfpathlineto{\pgfqpoint{1.070313in}{0.171536in}}%
\pgfpathlineto{\pgfqpoint{1.072838in}{0.172914in}}%
\pgfpathlineto{\pgfqpoint{1.074001in}{0.176696in}}%
\pgfpathlineto{\pgfqpoint{1.075966in}{0.177890in}}%
\pgfpathlineto{\pgfqpoint{1.073883in}{0.179967in}}%
\pgfpathlineto{\pgfqpoint{1.073154in}{0.177269in}}%
\pgfpathlineto{\pgfqpoint{1.069781in}{0.176867in}}%
\pgfpathlineto{\pgfqpoint{1.068604in}{0.173680in}}%
\pgfpathlineto{\pgfqpoint{1.065898in}{0.174566in}}%
\pgfpathlineto{\pgfqpoint{1.065624in}{0.179594in}}%
\pgfpathlineto{\pgfqpoint{1.063296in}{0.179530in}}%
\pgfpathlineto{\pgfqpoint{1.064425in}{0.183198in}}%
\pgfpathlineto{\pgfqpoint{1.069941in}{0.185345in}}%
\pgfpathlineto{\pgfqpoint{1.069443in}{0.182567in}}%
\pgfpathlineto{\pgfqpoint{1.070640in}{0.180586in}}%
\pgfpathlineto{\pgfqpoint{1.071934in}{0.187975in}}%
\pgfpathlineto{\pgfqpoint{1.073144in}{0.190094in}}%
\pgfpathlineto{\pgfqpoint{1.076606in}{0.193050in}}%
\pgfpathlineto{\pgfqpoint{1.072321in}{0.195142in}}%
\pgfpathlineto{\pgfqpoint{1.072458in}{0.197025in}}%
\pgfpathlineto{\pgfqpoint{1.070874in}{0.199642in}}%
\pgfpathlineto{\pgfqpoint{1.070121in}{0.204047in}}%
\pgfpathlineto{\pgfqpoint{1.071795in}{0.204349in}}%
\pgfpathlineto{\pgfqpoint{1.071791in}{0.206742in}}%
\pgfpathlineto{\pgfqpoint{1.069524in}{0.207403in}}%
\pgfpathlineto{\pgfqpoint{1.067853in}{0.209464in}}%
\pgfpathlineto{\pgfqpoint{1.067534in}{0.211631in}}%
\pgfpathlineto{\pgfqpoint{1.065578in}{0.211187in}}%
\pgfpathlineto{\pgfqpoint{1.064056in}{0.213557in}}%
\pgfpathclose%
\pgfusepath{fill}%
\end{pgfscope}%
\begin{pgfscope}%
\pgfpathrectangle{\pgfqpoint{0.100000in}{0.100000in}}{\pgfqpoint{3.608454in}{2.310000in}}%
\pgfusepath{clip}%
\pgfsetbuttcap%
\pgfsetmiterjoin%
\definecolor{currentfill}{rgb}{0.000000,0.827451,0.586275}%
\pgfsetfillcolor{currentfill}%
\pgfsetlinewidth{0.000000pt}%
\definecolor{currentstroke}{rgb}{0.000000,0.000000,0.000000}%
\pgfsetstrokecolor{currentstroke}%
\pgfsetstrokeopacity{0.000000}%
\pgfsetdash{}{0pt}%
\pgfpathmoveto{\pgfqpoint{1.053511in}{0.195656in}}%
\pgfpathlineto{\pgfqpoint{1.056239in}{0.203313in}}%
\pgfpathlineto{\pgfqpoint{1.055808in}{0.208025in}}%
\pgfpathlineto{\pgfqpoint{1.056404in}{0.208822in}}%
\pgfpathlineto{\pgfqpoint{1.055449in}{0.213304in}}%
\pgfpathlineto{\pgfqpoint{1.057580in}{0.213988in}}%
\pgfpathlineto{\pgfqpoint{1.064166in}{0.208247in}}%
\pgfpathlineto{\pgfqpoint{1.067947in}{0.205632in}}%
\pgfpathlineto{\pgfqpoint{1.068462in}{0.202960in}}%
\pgfpathlineto{\pgfqpoint{1.067374in}{0.200817in}}%
\pgfpathlineto{\pgfqpoint{1.069430in}{0.198670in}}%
\pgfpathlineto{\pgfqpoint{1.071014in}{0.192172in}}%
\pgfpathlineto{\pgfqpoint{1.066528in}{0.191514in}}%
\pgfpathlineto{\pgfqpoint{1.064349in}{0.192706in}}%
\pgfpathlineto{\pgfqpoint{1.066503in}{0.196582in}}%
\pgfpathlineto{\pgfqpoint{1.064154in}{0.196268in}}%
\pgfpathlineto{\pgfqpoint{1.062590in}{0.194222in}}%
\pgfpathlineto{\pgfqpoint{1.059615in}{0.193121in}}%
\pgfpathlineto{\pgfqpoint{1.057190in}{0.195536in}}%
\pgfpathlineto{\pgfqpoint{1.055925in}{0.194675in}}%
\pgfpathlineto{\pgfqpoint{1.053511in}{0.195656in}}%
\pgfpathclose%
\pgfusepath{fill}%
\end{pgfscope}%
\begin{pgfscope}%
\pgfpathrectangle{\pgfqpoint{0.100000in}{0.100000in}}{\pgfqpoint{3.608454in}{2.310000in}}%
\pgfusepath{clip}%
\pgfsetbuttcap%
\pgfsetmiterjoin%
\definecolor{currentfill}{rgb}{0.000000,0.694118,0.652941}%
\pgfsetfillcolor{currentfill}%
\pgfsetlinewidth{0.000000pt}%
\definecolor{currentstroke}{rgb}{0.000000,0.000000,0.000000}%
\pgfsetstrokecolor{currentstroke}%
\pgfsetstrokeopacity{0.000000}%
\pgfsetdash{}{0pt}%
\pgfpathmoveto{\pgfqpoint{1.157736in}{1.964285in}}%
\pgfpathlineto{\pgfqpoint{1.140024in}{1.967674in}}%
\pgfpathlineto{\pgfqpoint{1.137125in}{1.963079in}}%
\pgfpathlineto{\pgfqpoint{1.128763in}{1.959930in}}%
\pgfpathlineto{\pgfqpoint{1.126712in}{1.970229in}}%
\pgfpathlineto{\pgfqpoint{1.113810in}{1.978599in}}%
\pgfpathlineto{\pgfqpoint{1.110227in}{1.984421in}}%
\pgfpathlineto{\pgfqpoint{1.103252in}{1.984857in}}%
\pgfpathlineto{\pgfqpoint{1.092311in}{1.978823in}}%
\pgfpathlineto{\pgfqpoint{1.088551in}{1.993499in}}%
\pgfpathlineto{\pgfqpoint{1.081829in}{1.997042in}}%
\pgfpathlineto{\pgfqpoint{1.080846in}{2.002972in}}%
\pgfpathlineto{\pgfqpoint{1.076031in}{2.007092in}}%
\pgfpathlineto{\pgfqpoint{1.076933in}{2.013289in}}%
\pgfpathlineto{\pgfqpoint{1.081296in}{2.021948in}}%
\pgfpathlineto{\pgfqpoint{1.081054in}{2.029697in}}%
\pgfpathlineto{\pgfqpoint{1.077963in}{2.034819in}}%
\pgfpathlineto{\pgfqpoint{1.078479in}{2.042039in}}%
\pgfpathlineto{\pgfqpoint{1.085312in}{2.052758in}}%
\pgfpathlineto{\pgfqpoint{1.092070in}{2.051353in}}%
\pgfpathlineto{\pgfqpoint{1.092981in}{2.055856in}}%
\pgfpathlineto{\pgfqpoint{1.103510in}{2.056028in}}%
\pgfpathlineto{\pgfqpoint{1.108312in}{2.062127in}}%
\pgfpathlineto{\pgfqpoint{1.114370in}{2.060903in}}%
\pgfpathlineto{\pgfqpoint{1.119922in}{2.088067in}}%
\pgfpathlineto{\pgfqpoint{1.111233in}{2.089809in}}%
\pgfpathlineto{\pgfqpoint{1.118004in}{2.122768in}}%
\pgfpathlineto{\pgfqpoint{1.134932in}{2.119015in}}%
\pgfpathlineto{\pgfqpoint{1.137742in}{2.109618in}}%
\pgfpathlineto{\pgfqpoint{1.132903in}{2.085429in}}%
\pgfpathlineto{\pgfqpoint{1.146831in}{2.082471in}}%
\pgfpathlineto{\pgfqpoint{1.141533in}{2.055406in}}%
\pgfpathlineto{\pgfqpoint{1.155484in}{2.052872in}}%
\pgfpathlineto{\pgfqpoint{1.153421in}{2.042231in}}%
\pgfpathlineto{\pgfqpoint{1.160126in}{2.041022in}}%
\pgfpathlineto{\pgfqpoint{1.164980in}{2.032458in}}%
\pgfpathlineto{\pgfqpoint{1.162054in}{2.020418in}}%
\pgfpathlineto{\pgfqpoint{1.158552in}{2.015787in}}%
\pgfpathlineto{\pgfqpoint{1.148015in}{2.013748in}}%
\pgfpathlineto{\pgfqpoint{1.145294in}{2.011278in}}%
\pgfpathlineto{\pgfqpoint{1.143228in}{2.002013in}}%
\pgfpathlineto{\pgfqpoint{1.147664in}{1.994687in}}%
\pgfpathlineto{\pgfqpoint{1.146770in}{1.973621in}}%
\pgfpathlineto{\pgfqpoint{1.157736in}{1.964285in}}%
\pgfpathclose%
\pgfusepath{fill}%
\end{pgfscope}%
\begin{pgfscope}%
\pgfpathrectangle{\pgfqpoint{0.100000in}{0.100000in}}{\pgfqpoint{3.608454in}{2.310000in}}%
\pgfusepath{clip}%
\pgfsetbuttcap%
\pgfsetmiterjoin%
\definecolor{currentfill}{rgb}{0.000000,0.435294,0.782353}%
\pgfsetfillcolor{currentfill}%
\pgfsetlinewidth{0.000000pt}%
\definecolor{currentstroke}{rgb}{0.000000,0.000000,0.000000}%
\pgfsetstrokecolor{currentstroke}%
\pgfsetstrokeopacity{0.000000}%
\pgfsetdash{}{0pt}%
\pgfpathmoveto{\pgfqpoint{1.937428in}{1.476563in}}%
\pgfpathlineto{\pgfqpoint{1.883772in}{1.478524in}}%
\pgfpathlineto{\pgfqpoint{1.884864in}{1.505981in}}%
\pgfpathlineto{\pgfqpoint{1.883282in}{1.506053in}}%
\pgfpathlineto{\pgfqpoint{1.884664in}{1.533508in}}%
\pgfpathlineto{\pgfqpoint{1.911513in}{1.532437in}}%
\pgfpathlineto{\pgfqpoint{1.912100in}{1.522170in}}%
\pgfpathlineto{\pgfqpoint{1.922421in}{1.521780in}}%
\pgfpathlineto{\pgfqpoint{1.925386in}{1.523978in}}%
\pgfpathlineto{\pgfqpoint{1.946070in}{1.524600in}}%
\pgfpathlineto{\pgfqpoint{1.946376in}{1.531379in}}%
\pgfpathlineto{\pgfqpoint{1.952621in}{1.531206in}}%
\pgfpathlineto{\pgfqpoint{1.952480in}{1.526297in}}%
\pgfpathlineto{\pgfqpoint{1.942299in}{1.518835in}}%
\pgfpathlineto{\pgfqpoint{1.938437in}{1.514106in}}%
\pgfpathlineto{\pgfqpoint{1.937428in}{1.476563in}}%
\pgfpathclose%
\pgfusepath{fill}%
\end{pgfscope}%
\begin{pgfscope}%
\pgfpathrectangle{\pgfqpoint{0.100000in}{0.100000in}}{\pgfqpoint{3.608454in}{2.310000in}}%
\pgfusepath{clip}%
\pgfsetbuttcap%
\pgfsetmiterjoin%
\definecolor{currentfill}{rgb}{0.000000,0.678431,0.660784}%
\pgfsetfillcolor{currentfill}%
\pgfsetlinewidth{0.000000pt}%
\definecolor{currentstroke}{rgb}{0.000000,0.000000,0.000000}%
\pgfsetstrokecolor{currentstroke}%
\pgfsetstrokeopacity{0.000000}%
\pgfsetdash{}{0pt}%
\pgfpathmoveto{\pgfqpoint{0.799750in}{1.984499in}}%
\pgfpathlineto{\pgfqpoint{0.808158in}{1.981911in}}%
\pgfpathlineto{\pgfqpoint{0.814622in}{1.975723in}}%
\pgfpathlineto{\pgfqpoint{0.813336in}{1.971025in}}%
\pgfpathlineto{\pgfqpoint{0.807404in}{1.969358in}}%
\pgfpathlineto{\pgfqpoint{0.807519in}{1.962306in}}%
\pgfpathlineto{\pgfqpoint{0.793910in}{1.961931in}}%
\pgfpathlineto{\pgfqpoint{0.793832in}{1.958919in}}%
\pgfpathlineto{\pgfqpoint{0.801936in}{1.953999in}}%
\pgfpathlineto{\pgfqpoint{0.803077in}{1.950037in}}%
\pgfpathlineto{\pgfqpoint{0.793934in}{1.941045in}}%
\pgfpathlineto{\pgfqpoint{0.792882in}{1.935165in}}%
\pgfpathlineto{\pgfqpoint{0.785890in}{1.927298in}}%
\pgfpathlineto{\pgfqpoint{0.800486in}{1.923543in}}%
\pgfpathlineto{\pgfqpoint{0.796408in}{1.906884in}}%
\pgfpathlineto{\pgfqpoint{0.764102in}{1.916069in}}%
\pgfpathlineto{\pgfqpoint{0.762277in}{1.909354in}}%
\pgfpathlineto{\pgfqpoint{0.715935in}{1.921940in}}%
\pgfpathlineto{\pgfqpoint{0.723444in}{1.948508in}}%
\pgfpathlineto{\pgfqpoint{0.734618in}{1.987950in}}%
\pgfpathlineto{\pgfqpoint{0.737189in}{2.001359in}}%
\pgfpathlineto{\pgfqpoint{0.799750in}{1.984499in}}%
\pgfpathclose%
\pgfusepath{fill}%
\end{pgfscope}%
\begin{pgfscope}%
\pgfpathrectangle{\pgfqpoint{0.100000in}{0.100000in}}{\pgfqpoint{3.608454in}{2.310000in}}%
\pgfusepath{clip}%
\pgfsetbuttcap%
\pgfsetmiterjoin%
\definecolor{currentfill}{rgb}{0.000000,0.698039,0.650980}%
\pgfsetfillcolor{currentfill}%
\pgfsetlinewidth{0.000000pt}%
\definecolor{currentstroke}{rgb}{0.000000,0.000000,0.000000}%
\pgfsetstrokecolor{currentstroke}%
\pgfsetstrokeopacity{0.000000}%
\pgfsetdash{}{0pt}%
\pgfpathmoveto{\pgfqpoint{1.267460in}{1.535366in}}%
\pgfpathlineto{\pgfqpoint{1.255196in}{1.457818in}}%
\pgfpathlineto{\pgfqpoint{1.253203in}{1.445105in}}%
\pgfpathlineto{\pgfqpoint{1.249416in}{1.442841in}}%
\pgfpathlineto{\pgfqpoint{1.194163in}{1.451751in}}%
\pgfpathlineto{\pgfqpoint{1.194698in}{1.460074in}}%
\pgfpathlineto{\pgfqpoint{1.199187in}{1.467135in}}%
\pgfpathlineto{\pgfqpoint{1.199560in}{1.472867in}}%
\pgfpathlineto{\pgfqpoint{1.206990in}{1.477188in}}%
\pgfpathlineto{\pgfqpoint{1.169072in}{1.483661in}}%
\pgfpathlineto{\pgfqpoint{1.148850in}{1.487724in}}%
\pgfpathlineto{\pgfqpoint{1.148005in}{1.494799in}}%
\pgfpathlineto{\pgfqpoint{1.158113in}{1.555519in}}%
\pgfpathlineto{\pgfqpoint{1.167943in}{1.559368in}}%
\pgfpathlineto{\pgfqpoint{1.173397in}{1.557849in}}%
\pgfpathlineto{\pgfqpoint{1.190708in}{1.558686in}}%
\pgfpathlineto{\pgfqpoint{1.198089in}{1.561222in}}%
\pgfpathlineto{\pgfqpoint{1.203263in}{1.558508in}}%
\pgfpathlineto{\pgfqpoint{1.213136in}{1.556950in}}%
\pgfpathlineto{\pgfqpoint{1.220332in}{1.551758in}}%
\pgfpathlineto{\pgfqpoint{1.241154in}{1.553467in}}%
\pgfpathlineto{\pgfqpoint{1.242723in}{1.546157in}}%
\pgfpathlineto{\pgfqpoint{1.248664in}{1.549321in}}%
\pgfpathlineto{\pgfqpoint{1.249496in}{1.554569in}}%
\pgfpathlineto{\pgfqpoint{1.260798in}{1.552822in}}%
\pgfpathlineto{\pgfqpoint{1.262249in}{1.546769in}}%
\pgfpathlineto{\pgfqpoint{1.261144in}{1.538773in}}%
\pgfpathlineto{\pgfqpoint{1.267460in}{1.535366in}}%
\pgfpathclose%
\pgfusepath{fill}%
\end{pgfscope}%
\begin{pgfscope}%
\pgfpathrectangle{\pgfqpoint{0.100000in}{0.100000in}}{\pgfqpoint{3.608454in}{2.310000in}}%
\pgfusepath{clip}%
\pgfsetbuttcap%
\pgfsetmiterjoin%
\definecolor{currentfill}{rgb}{0.000000,0.631373,0.684314}%
\pgfsetfillcolor{currentfill}%
\pgfsetlinewidth{0.000000pt}%
\definecolor{currentstroke}{rgb}{0.000000,0.000000,0.000000}%
\pgfsetstrokecolor{currentstroke}%
\pgfsetstrokeopacity{0.000000}%
\pgfsetdash{}{0pt}%
\pgfpathmoveto{\pgfqpoint{2.106820in}{1.155478in}}%
\pgfpathlineto{\pgfqpoint{2.106837in}{1.177041in}}%
\pgfpathlineto{\pgfqpoint{2.102150in}{1.181491in}}%
\pgfpathlineto{\pgfqpoint{2.102082in}{1.211883in}}%
\pgfpathlineto{\pgfqpoint{2.101288in}{1.239463in}}%
\pgfpathlineto{\pgfqpoint{2.101955in}{1.263584in}}%
\pgfpathlineto{\pgfqpoint{2.130827in}{1.263573in}}%
\pgfpathlineto{\pgfqpoint{2.130830in}{1.210394in}}%
\pgfpathlineto{\pgfqpoint{2.164491in}{1.209439in}}%
\pgfpathlineto{\pgfqpoint{2.166408in}{1.204778in}}%
\pgfpathlineto{\pgfqpoint{2.165811in}{1.161887in}}%
\pgfpathlineto{\pgfqpoint{2.130979in}{1.163114in}}%
\pgfpathlineto{\pgfqpoint{2.131026in}{1.155307in}}%
\pgfpathlineto{\pgfqpoint{2.106820in}{1.155478in}}%
\pgfpathclose%
\pgfusepath{fill}%
\end{pgfscope}%
\begin{pgfscope}%
\pgfpathrectangle{\pgfqpoint{0.100000in}{0.100000in}}{\pgfqpoint{3.608454in}{2.310000in}}%
\pgfusepath{clip}%
\pgfsetbuttcap%
\pgfsetmiterjoin%
\definecolor{currentfill}{rgb}{0.000000,0.333333,0.833333}%
\pgfsetfillcolor{currentfill}%
\pgfsetlinewidth{0.000000pt}%
\definecolor{currentstroke}{rgb}{0.000000,0.000000,0.000000}%
\pgfsetstrokecolor{currentstroke}%
\pgfsetstrokeopacity{0.000000}%
\pgfsetdash{}{0pt}%
\pgfpathmoveto{\pgfqpoint{3.080944in}{0.255920in}}%
\pgfpathlineto{\pgfqpoint{3.081905in}{0.258847in}}%
\pgfpathlineto{\pgfqpoint{3.093100in}{0.264126in}}%
\pgfpathlineto{\pgfqpoint{3.097421in}{0.270578in}}%
\pgfpathlineto{\pgfqpoint{3.104568in}{0.275364in}}%
\pgfpathlineto{\pgfqpoint{3.114895in}{0.274075in}}%
\pgfpathlineto{\pgfqpoint{3.119568in}{0.271124in}}%
\pgfpathlineto{\pgfqpoint{3.110703in}{0.266525in}}%
\pgfpathlineto{\pgfqpoint{3.103147in}{0.268644in}}%
\pgfpathlineto{\pgfqpoint{3.101531in}{0.264974in}}%
\pgfpathlineto{\pgfqpoint{3.089987in}{0.258378in}}%
\pgfpathlineto{\pgfqpoint{3.080944in}{0.255920in}}%
\pgfpathclose%
\pgfusepath{fill}%
\end{pgfscope}%
\begin{pgfscope}%
\pgfpathrectangle{\pgfqpoint{0.100000in}{0.100000in}}{\pgfqpoint{3.608454in}{2.310000in}}%
\pgfusepath{clip}%
\pgfsetbuttcap%
\pgfsetmiterjoin%
\definecolor{currentfill}{rgb}{0.000000,0.333333,0.833333}%
\pgfsetfillcolor{currentfill}%
\pgfsetlinewidth{0.000000pt}%
\definecolor{currentstroke}{rgb}{0.000000,0.000000,0.000000}%
\pgfsetstrokecolor{currentstroke}%
\pgfsetstrokeopacity{0.000000}%
\pgfsetdash{}{0pt}%
\pgfpathmoveto{\pgfqpoint{3.099940in}{0.362168in}}%
\pgfpathlineto{\pgfqpoint{3.133900in}{0.367620in}}%
\pgfpathlineto{\pgfqpoint{3.127726in}{0.403873in}}%
\pgfpathlineto{\pgfqpoint{3.126693in}{0.409803in}}%
\pgfpathlineto{\pgfqpoint{3.173551in}{0.417385in}}%
\pgfpathlineto{\pgfqpoint{3.184616in}{0.418072in}}%
\pgfpathlineto{\pgfqpoint{3.185207in}{0.398995in}}%
\pgfpathlineto{\pgfqpoint{3.187948in}{0.377138in}}%
\pgfpathlineto{\pgfqpoint{3.187162in}{0.370691in}}%
\pgfpathlineto{\pgfqpoint{3.180144in}{0.367508in}}%
\pgfpathlineto{\pgfqpoint{3.176678in}{0.346943in}}%
\pgfpathlineto{\pgfqpoint{3.178542in}{0.336148in}}%
\pgfpathlineto{\pgfqpoint{3.168359in}{0.323529in}}%
\pgfpathlineto{\pgfqpoint{3.163072in}{0.325621in}}%
\pgfpathlineto{\pgfqpoint{3.160642in}{0.320071in}}%
\pgfpathlineto{\pgfqpoint{3.154594in}{0.316205in}}%
\pgfpathlineto{\pgfqpoint{3.143564in}{0.317485in}}%
\pgfpathlineto{\pgfqpoint{3.140330in}{0.313854in}}%
\pgfpathlineto{\pgfqpoint{3.126999in}{0.309857in}}%
\pgfpathlineto{\pgfqpoint{3.119526in}{0.317390in}}%
\pgfpathlineto{\pgfqpoint{3.119713in}{0.326025in}}%
\pgfpathlineto{\pgfqpoint{3.116825in}{0.336890in}}%
\pgfpathlineto{\pgfqpoint{3.113204in}{0.339888in}}%
\pgfpathlineto{\pgfqpoint{3.109115in}{0.349472in}}%
\pgfpathlineto{\pgfqpoint{3.101613in}{0.355282in}}%
\pgfpathlineto{\pgfqpoint{3.099940in}{0.362168in}}%
\pgfpathclose%
\pgfusepath{fill}%
\end{pgfscope}%
\begin{pgfscope}%
\pgfpathrectangle{\pgfqpoint{0.100000in}{0.100000in}}{\pgfqpoint{3.608454in}{2.310000in}}%
\pgfusepath{clip}%
\pgfsetbuttcap%
\pgfsetmiterjoin%
\definecolor{currentfill}{rgb}{0.000000,0.698039,0.650980}%
\pgfsetfillcolor{currentfill}%
\pgfsetlinewidth{0.000000pt}%
\definecolor{currentstroke}{rgb}{0.000000,0.000000,0.000000}%
\pgfsetstrokecolor{currentstroke}%
\pgfsetstrokeopacity{0.000000}%
\pgfsetdash{}{0pt}%
\pgfpathmoveto{\pgfqpoint{2.547754in}{0.792351in}}%
\pgfpathlineto{\pgfqpoint{2.536679in}{0.788374in}}%
\pgfpathlineto{\pgfqpoint{2.518619in}{0.784958in}}%
\pgfpathlineto{\pgfqpoint{2.516272in}{0.816527in}}%
\pgfpathlineto{\pgfqpoint{2.488805in}{0.814690in}}%
\pgfpathlineto{\pgfqpoint{2.485523in}{0.870923in}}%
\pgfpathlineto{\pgfqpoint{2.519263in}{0.872679in}}%
\pgfpathlineto{\pgfqpoint{2.550184in}{0.875112in}}%
\pgfpathlineto{\pgfqpoint{2.549490in}{0.847081in}}%
\pgfpathlineto{\pgfqpoint{2.547754in}{0.792351in}}%
\pgfpathclose%
\pgfusepath{fill}%
\end{pgfscope}%
\begin{pgfscope}%
\pgfpathrectangle{\pgfqpoint{0.100000in}{0.100000in}}{\pgfqpoint{3.608454in}{2.310000in}}%
\pgfusepath{clip}%
\pgfsetbuttcap%
\pgfsetmiterjoin%
\definecolor{currentfill}{rgb}{0.000000,0.490196,0.754902}%
\pgfsetfillcolor{currentfill}%
\pgfsetlinewidth{0.000000pt}%
\definecolor{currentstroke}{rgb}{0.000000,0.000000,0.000000}%
\pgfsetstrokecolor{currentstroke}%
\pgfsetstrokeopacity{0.000000}%
\pgfsetdash{}{0pt}%
\pgfpathmoveto{\pgfqpoint{1.244902in}{0.674701in}}%
\pgfpathlineto{\pgfqpoint{1.249588in}{0.682742in}}%
\pgfpathlineto{\pgfqpoint{1.263659in}{0.679913in}}%
\pgfpathlineto{\pgfqpoint{1.258636in}{0.675464in}}%
\pgfpathlineto{\pgfqpoint{1.252077in}{0.677773in}}%
\pgfpathlineto{\pgfqpoint{1.244902in}{0.674701in}}%
\pgfpathclose%
\pgfusepath{fill}%
\end{pgfscope}%
\begin{pgfscope}%
\pgfpathrectangle{\pgfqpoint{0.100000in}{0.100000in}}{\pgfqpoint{3.608454in}{2.310000in}}%
\pgfusepath{clip}%
\pgfsetbuttcap%
\pgfsetmiterjoin%
\definecolor{currentfill}{rgb}{0.000000,0.490196,0.754902}%
\pgfsetfillcolor{currentfill}%
\pgfsetlinewidth{0.000000pt}%
\definecolor{currentstroke}{rgb}{0.000000,0.000000,0.000000}%
\pgfsetstrokecolor{currentstroke}%
\pgfsetstrokeopacity{0.000000}%
\pgfsetdash{}{0pt}%
\pgfpathmoveto{\pgfqpoint{1.293790in}{0.642427in}}%
\pgfpathlineto{\pgfqpoint{1.285285in}{0.654449in}}%
\pgfpathlineto{\pgfqpoint{1.281587in}{0.664883in}}%
\pgfpathlineto{\pgfqpoint{1.282478in}{0.669549in}}%
\pgfpathlineto{\pgfqpoint{1.292812in}{0.673886in}}%
\pgfpathlineto{\pgfqpoint{1.304918in}{0.671461in}}%
\pgfpathlineto{\pgfqpoint{1.308962in}{0.666408in}}%
\pgfpathlineto{\pgfqpoint{1.315946in}{0.662668in}}%
\pgfpathlineto{\pgfqpoint{1.318323in}{0.655591in}}%
\pgfpathlineto{\pgfqpoint{1.315484in}{0.650135in}}%
\pgfpathlineto{\pgfqpoint{1.309751in}{0.648208in}}%
\pgfpathlineto{\pgfqpoint{1.305345in}{0.642514in}}%
\pgfpathlineto{\pgfqpoint{1.293790in}{0.642427in}}%
\pgfpathclose%
\pgfusepath{fill}%
\end{pgfscope}%
\begin{pgfscope}%
\pgfpathrectangle{\pgfqpoint{0.100000in}{0.100000in}}{\pgfqpoint{3.608454in}{2.310000in}}%
\pgfusepath{clip}%
\pgfsetbuttcap%
\pgfsetmiterjoin%
\definecolor{currentfill}{rgb}{0.000000,0.478431,0.760784}%
\pgfsetfillcolor{currentfill}%
\pgfsetlinewidth{0.000000pt}%
\definecolor{currentstroke}{rgb}{0.000000,0.000000,0.000000}%
\pgfsetstrokecolor{currentstroke}%
\pgfsetstrokeopacity{0.000000}%
\pgfsetdash{}{0pt}%
\pgfpathmoveto{\pgfqpoint{1.946085in}{2.030374in}}%
\pgfpathlineto{\pgfqpoint{1.945081in}{2.002729in}}%
\pgfpathlineto{\pgfqpoint{1.946235in}{1.995849in}}%
\pgfpathlineto{\pgfqpoint{1.918791in}{1.996815in}}%
\pgfpathlineto{\pgfqpoint{1.917334in}{2.003750in}}%
\pgfpathlineto{\pgfqpoint{1.918355in}{2.031373in}}%
\pgfpathlineto{\pgfqpoint{1.946085in}{2.030374in}}%
\pgfpathclose%
\pgfusepath{fill}%
\end{pgfscope}%
\begin{pgfscope}%
\pgfpathrectangle{\pgfqpoint{0.100000in}{0.100000in}}{\pgfqpoint{3.608454in}{2.310000in}}%
\pgfusepath{clip}%
\pgfsetbuttcap%
\pgfsetmiterjoin%
\definecolor{currentfill}{rgb}{0.000000,0.439216,0.780392}%
\pgfsetfillcolor{currentfill}%
\pgfsetlinewidth{0.000000pt}%
\definecolor{currentstroke}{rgb}{0.000000,0.000000,0.000000}%
\pgfsetstrokecolor{currentstroke}%
\pgfsetstrokeopacity{0.000000}%
\pgfsetdash{}{0pt}%
\pgfpathmoveto{\pgfqpoint{1.474731in}{1.733639in}}%
\pgfpathlineto{\pgfqpoint{1.441739in}{1.737485in}}%
\pgfpathlineto{\pgfqpoint{1.412417in}{1.742036in}}%
\pgfpathlineto{\pgfqpoint{1.417395in}{1.794618in}}%
\pgfpathlineto{\pgfqpoint{1.419171in}{1.807060in}}%
\pgfpathlineto{\pgfqpoint{1.417230in}{1.817131in}}%
\pgfpathlineto{\pgfqpoint{1.412375in}{1.821775in}}%
\pgfpathlineto{\pgfqpoint{1.406562in}{1.831802in}}%
\pgfpathlineto{\pgfqpoint{1.396824in}{1.840362in}}%
\pgfpathlineto{\pgfqpoint{1.389944in}{1.843550in}}%
\pgfpathlineto{\pgfqpoint{1.384830in}{1.855882in}}%
\pgfpathlineto{\pgfqpoint{1.383661in}{1.866297in}}%
\pgfpathlineto{\pgfqpoint{1.364865in}{1.868873in}}%
\pgfpathlineto{\pgfqpoint{1.372850in}{1.883266in}}%
\pgfpathlineto{\pgfqpoint{1.375793in}{1.885029in}}%
\pgfpathlineto{\pgfqpoint{1.342451in}{1.889782in}}%
\pgfpathlineto{\pgfqpoint{1.345983in}{1.906551in}}%
\pgfpathlineto{\pgfqpoint{1.348402in}{1.908679in}}%
\pgfpathlineto{\pgfqpoint{1.366491in}{1.905688in}}%
\pgfpathlineto{\pgfqpoint{1.380767in}{1.908242in}}%
\pgfpathlineto{\pgfqpoint{1.383830in}{1.929135in}}%
\pgfpathlineto{\pgfqpoint{1.386336in}{1.938080in}}%
\pgfpathlineto{\pgfqpoint{1.395427in}{1.936746in}}%
\pgfpathlineto{\pgfqpoint{1.400977in}{1.942930in}}%
\pgfpathlineto{\pgfqpoint{1.407750in}{1.941975in}}%
\pgfpathlineto{\pgfqpoint{1.408401in}{1.946543in}}%
\pgfpathlineto{\pgfqpoint{1.417308in}{1.945271in}}%
\pgfpathlineto{\pgfqpoint{1.422053in}{1.944628in}}%
\pgfpathlineto{\pgfqpoint{1.421111in}{1.937794in}}%
\pgfpathlineto{\pgfqpoint{1.434747in}{1.935937in}}%
\pgfpathlineto{\pgfqpoint{1.446859in}{1.927356in}}%
\pgfpathlineto{\pgfqpoint{1.446358in}{1.912489in}}%
\pgfpathlineto{\pgfqpoint{1.456350in}{1.910291in}}%
\pgfpathlineto{\pgfqpoint{1.453065in}{1.884941in}}%
\pgfpathlineto{\pgfqpoint{1.449171in}{1.871656in}}%
\pgfpathlineto{\pgfqpoint{1.477206in}{1.868092in}}%
\pgfpathlineto{\pgfqpoint{1.475384in}{1.853420in}}%
\pgfpathlineto{\pgfqpoint{1.488701in}{1.851762in}}%
\pgfpathlineto{\pgfqpoint{1.486605in}{1.827974in}}%
\pgfpathlineto{\pgfqpoint{1.481253in}{1.786760in}}%
\pgfpathlineto{\pgfqpoint{1.474731in}{1.733639in}}%
\pgfpathclose%
\pgfusepath{fill}%
\end{pgfscope}%
\begin{pgfscope}%
\pgfpathrectangle{\pgfqpoint{0.100000in}{0.100000in}}{\pgfqpoint{3.608454in}{2.310000in}}%
\pgfusepath{clip}%
\pgfsetbuttcap%
\pgfsetmiterjoin%
\definecolor{currentfill}{rgb}{0.000000,0.603922,0.698039}%
\pgfsetfillcolor{currentfill}%
\pgfsetlinewidth{0.000000pt}%
\definecolor{currentstroke}{rgb}{0.000000,0.000000,0.000000}%
\pgfsetstrokecolor{currentstroke}%
\pgfsetstrokeopacity{0.000000}%
\pgfsetdash{}{0pt}%
\pgfpathmoveto{\pgfqpoint{2.184096in}{1.723863in}}%
\pgfpathlineto{\pgfqpoint{2.177278in}{1.723751in}}%
\pgfpathlineto{\pgfqpoint{2.176891in}{1.751349in}}%
\pgfpathlineto{\pgfqpoint{2.190701in}{1.751566in}}%
\pgfpathlineto{\pgfqpoint{2.190305in}{1.779149in}}%
\pgfpathlineto{\pgfqpoint{2.204050in}{1.779422in}}%
\pgfpathlineto{\pgfqpoint{2.204133in}{1.773697in}}%
\pgfpathlineto{\pgfqpoint{2.217878in}{1.773970in}}%
\pgfpathlineto{\pgfqpoint{2.218568in}{1.724511in}}%
\pgfpathlineto{\pgfqpoint{2.184096in}{1.723863in}}%
\pgfpathclose%
\pgfusepath{fill}%
\end{pgfscope}%
\begin{pgfscope}%
\pgfpathrectangle{\pgfqpoint{0.100000in}{0.100000in}}{\pgfqpoint{3.608454in}{2.310000in}}%
\pgfusepath{clip}%
\pgfsetbuttcap%
\pgfsetmiterjoin%
\definecolor{currentfill}{rgb}{0.000000,0.364706,0.817647}%
\pgfsetfillcolor{currentfill}%
\pgfsetlinewidth{0.000000pt}%
\definecolor{currentstroke}{rgb}{0.000000,0.000000,0.000000}%
\pgfsetstrokecolor{currentstroke}%
\pgfsetstrokeopacity{0.000000}%
\pgfsetdash{}{0pt}%
\pgfpathmoveto{\pgfqpoint{1.789033in}{1.324734in}}%
\pgfpathlineto{\pgfqpoint{1.824072in}{1.322793in}}%
\pgfpathlineto{\pgfqpoint{1.822395in}{1.288502in}}%
\pgfpathlineto{\pgfqpoint{1.782549in}{1.290657in}}%
\pgfpathlineto{\pgfqpoint{1.754213in}{1.292521in}}%
\pgfpathlineto{\pgfqpoint{1.756162in}{1.326912in}}%
\pgfpathlineto{\pgfqpoint{1.789033in}{1.324734in}}%
\pgfpathclose%
\pgfusepath{fill}%
\end{pgfscope}%
\begin{pgfscope}%
\pgfpathrectangle{\pgfqpoint{0.100000in}{0.100000in}}{\pgfqpoint{3.608454in}{2.310000in}}%
\pgfusepath{clip}%
\pgfsetbuttcap%
\pgfsetmiterjoin%
\definecolor{currentfill}{rgb}{0.000000,0.619608,0.690196}%
\pgfsetfillcolor{currentfill}%
\pgfsetlinewidth{0.000000pt}%
\definecolor{currentstroke}{rgb}{0.000000,0.000000,0.000000}%
\pgfsetstrokecolor{currentstroke}%
\pgfsetstrokeopacity{0.000000}%
\pgfsetdash{}{0pt}%
\pgfpathmoveto{\pgfqpoint{1.720336in}{1.913435in}}%
\pgfpathlineto{\pgfqpoint{1.726123in}{1.914246in}}%
\pgfpathlineto{\pgfqpoint{1.730862in}{1.910799in}}%
\pgfpathlineto{\pgfqpoint{1.744713in}{1.908214in}}%
\pgfpathlineto{\pgfqpoint{1.756437in}{1.912852in}}%
\pgfpathlineto{\pgfqpoint{1.759368in}{1.916843in}}%
\pgfpathlineto{\pgfqpoint{1.775193in}{1.927930in}}%
\pgfpathlineto{\pgfqpoint{1.783790in}{1.937380in}}%
\pgfpathlineto{\pgfqpoint{1.787591in}{1.935130in}}%
\pgfpathlineto{\pgfqpoint{1.799830in}{1.937828in}}%
\pgfpathlineto{\pgfqpoint{1.801655in}{1.932573in}}%
\pgfpathlineto{\pgfqpoint{1.800710in}{1.922898in}}%
\pgfpathlineto{\pgfqpoint{1.796911in}{1.916921in}}%
\pgfpathlineto{\pgfqpoint{1.796957in}{1.908802in}}%
\pgfpathlineto{\pgfqpoint{1.802493in}{1.899008in}}%
\pgfpathlineto{\pgfqpoint{1.808345in}{1.892715in}}%
\pgfpathlineto{\pgfqpoint{1.811895in}{1.879653in}}%
\pgfpathlineto{\pgfqpoint{1.807298in}{1.876074in}}%
\pgfpathlineto{\pgfqpoint{1.801983in}{1.865242in}}%
\pgfpathlineto{\pgfqpoint{1.808999in}{1.861084in}}%
\pgfpathlineto{\pgfqpoint{1.773598in}{1.863245in}}%
\pgfpathlineto{\pgfqpoint{1.716695in}{1.867335in}}%
\pgfpathlineto{\pgfqpoint{1.720336in}{1.913435in}}%
\pgfpathclose%
\pgfusepath{fill}%
\end{pgfscope}%
\begin{pgfscope}%
\pgfpathrectangle{\pgfqpoint{0.100000in}{0.100000in}}{\pgfqpoint{3.608454in}{2.310000in}}%
\pgfusepath{clip}%
\pgfsetbuttcap%
\pgfsetmiterjoin%
\definecolor{currentfill}{rgb}{0.000000,0.407843,0.796078}%
\pgfsetfillcolor{currentfill}%
\pgfsetlinewidth{0.000000pt}%
\definecolor{currentstroke}{rgb}{0.000000,0.000000,0.000000}%
\pgfsetstrokecolor{currentstroke}%
\pgfsetstrokeopacity{0.000000}%
\pgfsetdash{}{0pt}%
\pgfpathmoveto{\pgfqpoint{1.347428in}{1.265725in}}%
\pgfpathlineto{\pgfqpoint{1.325433in}{1.268700in}}%
\pgfpathlineto{\pgfqpoint{1.329166in}{1.295757in}}%
\pgfpathlineto{\pgfqpoint{1.327961in}{1.309954in}}%
\pgfpathlineto{\pgfqpoint{1.325950in}{1.311919in}}%
\pgfpathlineto{\pgfqpoint{1.328108in}{1.320759in}}%
\pgfpathlineto{\pgfqpoint{1.328146in}{1.332828in}}%
\pgfpathlineto{\pgfqpoint{1.325556in}{1.338863in}}%
\pgfpathlineto{\pgfqpoint{1.333852in}{1.337711in}}%
\pgfpathlineto{\pgfqpoint{1.343717in}{1.409436in}}%
\pgfpathlineto{\pgfqpoint{1.350593in}{1.411552in}}%
\pgfpathlineto{\pgfqpoint{1.352191in}{1.404691in}}%
\pgfpathlineto{\pgfqpoint{1.355918in}{1.399849in}}%
\pgfpathlineto{\pgfqpoint{1.369057in}{1.398085in}}%
\pgfpathlineto{\pgfqpoint{1.377365in}{1.386918in}}%
\pgfpathlineto{\pgfqpoint{1.383841in}{1.385038in}}%
\pgfpathlineto{\pgfqpoint{1.387549in}{1.389567in}}%
\pgfpathlineto{\pgfqpoint{1.393282in}{1.388976in}}%
\pgfpathlineto{\pgfqpoint{1.396241in}{1.384895in}}%
\pgfpathlineto{\pgfqpoint{1.403546in}{1.376980in}}%
\pgfpathlineto{\pgfqpoint{1.412208in}{1.374642in}}%
\pgfpathlineto{\pgfqpoint{1.405672in}{1.369838in}}%
\pgfpathlineto{\pgfqpoint{1.404621in}{1.362130in}}%
\pgfpathlineto{\pgfqpoint{1.406993in}{1.344082in}}%
\pgfpathlineto{\pgfqpoint{1.412097in}{1.337063in}}%
\pgfpathlineto{\pgfqpoint{1.365837in}{1.343240in}}%
\pgfpathlineto{\pgfqpoint{1.360950in}{1.306443in}}%
\pgfpathlineto{\pgfqpoint{1.352269in}{1.306231in}}%
\pgfpathlineto{\pgfqpoint{1.349065in}{1.285654in}}%
\pgfpathlineto{\pgfqpoint{1.347428in}{1.265725in}}%
\pgfpathclose%
\pgfusepath{fill}%
\end{pgfscope}%
\begin{pgfscope}%
\pgfpathrectangle{\pgfqpoint{0.100000in}{0.100000in}}{\pgfqpoint{3.608454in}{2.310000in}}%
\pgfusepath{clip}%
\pgfsetbuttcap%
\pgfsetmiterjoin%
\definecolor{currentfill}{rgb}{0.000000,0.796078,0.601961}%
\pgfsetfillcolor{currentfill}%
\pgfsetlinewidth{0.000000pt}%
\definecolor{currentstroke}{rgb}{0.000000,0.000000,0.000000}%
\pgfsetstrokecolor{currentstroke}%
\pgfsetstrokeopacity{0.000000}%
\pgfsetdash{}{0pt}%
\pgfpathmoveto{\pgfqpoint{1.716695in}{1.867335in}}%
\pgfpathlineto{\pgfqpoint{1.773598in}{1.863245in}}%
\pgfpathlineto{\pgfqpoint{1.808999in}{1.861084in}}%
\pgfpathlineto{\pgfqpoint{1.812344in}{1.853586in}}%
\pgfpathlineto{\pgfqpoint{1.809666in}{1.848703in}}%
\pgfpathlineto{\pgfqpoint{1.812489in}{1.842876in}}%
\pgfpathlineto{\pgfqpoint{1.808240in}{1.831561in}}%
\pgfpathlineto{\pgfqpoint{1.809782in}{1.824532in}}%
\pgfpathlineto{\pgfqpoint{1.802162in}{1.823198in}}%
\pgfpathlineto{\pgfqpoint{1.802764in}{1.815650in}}%
\pgfpathlineto{\pgfqpoint{1.803220in}{1.813204in}}%
\pgfpathlineto{\pgfqpoint{1.795074in}{1.805632in}}%
\pgfpathlineto{\pgfqpoint{1.787641in}{1.810142in}}%
\pgfpathlineto{\pgfqpoint{1.781507in}{1.805854in}}%
\pgfpathlineto{\pgfqpoint{1.778292in}{1.808049in}}%
\pgfpathlineto{\pgfqpoint{1.767130in}{1.804067in}}%
\pgfpathlineto{\pgfqpoint{1.760814in}{1.806325in}}%
\pgfpathlineto{\pgfqpoint{1.754993in}{1.802225in}}%
\pgfpathlineto{\pgfqpoint{1.747956in}{1.802589in}}%
\pgfpathlineto{\pgfqpoint{1.737625in}{1.793677in}}%
\pgfpathlineto{\pgfqpoint{1.729800in}{1.795494in}}%
\pgfpathlineto{\pgfqpoint{1.723276in}{1.792199in}}%
\pgfpathlineto{\pgfqpoint{1.710799in}{1.791190in}}%
\pgfpathlineto{\pgfqpoint{1.714041in}{1.832923in}}%
\pgfpathlineto{\pgfqpoint{1.716695in}{1.867335in}}%
\pgfpathclose%
\pgfusepath{fill}%
\end{pgfscope}%
\begin{pgfscope}%
\pgfpathrectangle{\pgfqpoint{0.100000in}{0.100000in}}{\pgfqpoint{3.608454in}{2.310000in}}%
\pgfusepath{clip}%
\pgfsetbuttcap%
\pgfsetmiterjoin%
\definecolor{currentfill}{rgb}{0.000000,0.490196,0.754902}%
\pgfsetfillcolor{currentfill}%
\pgfsetlinewidth{0.000000pt}%
\definecolor{currentstroke}{rgb}{0.000000,0.000000,0.000000}%
\pgfsetstrokecolor{currentstroke}%
\pgfsetstrokeopacity{0.000000}%
\pgfsetdash{}{0pt}%
\pgfpathmoveto{\pgfqpoint{2.835111in}{1.113784in}}%
\pgfpathlineto{\pgfqpoint{2.828227in}{1.112365in}}%
\pgfpathlineto{\pgfqpoint{2.817068in}{1.103142in}}%
\pgfpathlineto{\pgfqpoint{2.805797in}{1.111156in}}%
\pgfpathlineto{\pgfqpoint{2.790002in}{1.116268in}}%
\pgfpathlineto{\pgfqpoint{2.775615in}{1.111731in}}%
\pgfpathlineto{\pgfqpoint{2.772428in}{1.117636in}}%
\pgfpathlineto{\pgfqpoint{2.765621in}{1.120136in}}%
\pgfpathlineto{\pgfqpoint{2.761095in}{1.125193in}}%
\pgfpathlineto{\pgfqpoint{2.766985in}{1.132686in}}%
\pgfpathlineto{\pgfqpoint{2.763659in}{1.139195in}}%
\pgfpathlineto{\pgfqpoint{2.750357in}{1.150558in}}%
\pgfpathlineto{\pgfqpoint{2.751113in}{1.160516in}}%
\pgfpathlineto{\pgfqpoint{2.761563in}{1.168904in}}%
\pgfpathlineto{\pgfqpoint{2.771140in}{1.159342in}}%
\pgfpathlineto{\pgfqpoint{2.781495in}{1.155039in}}%
\pgfpathlineto{\pgfqpoint{2.785832in}{1.164847in}}%
\pgfpathlineto{\pgfqpoint{2.783231in}{1.181649in}}%
\pgfpathlineto{\pgfqpoint{2.788196in}{1.183891in}}%
\pgfpathlineto{\pgfqpoint{2.789452in}{1.189777in}}%
\pgfpathlineto{\pgfqpoint{2.804581in}{1.191394in}}%
\pgfpathlineto{\pgfqpoint{2.810250in}{1.179188in}}%
\pgfpathlineto{\pgfqpoint{2.814063in}{1.180206in}}%
\pgfpathlineto{\pgfqpoint{2.827195in}{1.174638in}}%
\pgfpathlineto{\pgfqpoint{2.824792in}{1.161761in}}%
\pgfpathlineto{\pgfqpoint{2.831674in}{1.145493in}}%
\pgfpathlineto{\pgfqpoint{2.823619in}{1.137722in}}%
\pgfpathlineto{\pgfqpoint{2.829784in}{1.127369in}}%
\pgfpathlineto{\pgfqpoint{2.833923in}{1.123330in}}%
\pgfpathlineto{\pgfqpoint{2.835111in}{1.113784in}}%
\pgfpathclose%
\pgfusepath{fill}%
\end{pgfscope}%
\begin{pgfscope}%
\pgfpathrectangle{\pgfqpoint{0.100000in}{0.100000in}}{\pgfqpoint{3.608454in}{2.310000in}}%
\pgfusepath{clip}%
\pgfsetbuttcap%
\pgfsetmiterjoin%
\definecolor{currentfill}{rgb}{0.000000,0.364706,0.817647}%
\pgfsetfillcolor{currentfill}%
\pgfsetlinewidth{0.000000pt}%
\definecolor{currentstroke}{rgb}{0.000000,0.000000,0.000000}%
\pgfsetstrokecolor{currentstroke}%
\pgfsetstrokeopacity{0.000000}%
\pgfsetdash{}{0pt}%
\pgfpathmoveto{\pgfqpoint{1.793434in}{1.427744in}}%
\pgfpathlineto{\pgfqpoint{1.791324in}{1.393448in}}%
\pgfpathlineto{\pgfqpoint{1.717833in}{1.398242in}}%
\pgfpathlineto{\pgfqpoint{1.716323in}{1.398352in}}%
\pgfpathlineto{\pgfqpoint{1.718916in}{1.432564in}}%
\pgfpathlineto{\pgfqpoint{1.724173in}{1.432221in}}%
\pgfpathlineto{\pgfqpoint{1.792481in}{1.427797in}}%
\pgfpathlineto{\pgfqpoint{1.793434in}{1.427744in}}%
\pgfpathclose%
\pgfusepath{fill}%
\end{pgfscope}%
\begin{pgfscope}%
\pgfpathrectangle{\pgfqpoint{0.100000in}{0.100000in}}{\pgfqpoint{3.608454in}{2.310000in}}%
\pgfusepath{clip}%
\pgfsetbuttcap%
\pgfsetmiterjoin%
\definecolor{currentfill}{rgb}{0.000000,0.607843,0.696078}%
\pgfsetfillcolor{currentfill}%
\pgfsetlinewidth{0.000000pt}%
\definecolor{currentstroke}{rgb}{0.000000,0.000000,0.000000}%
\pgfsetstrokecolor{currentstroke}%
\pgfsetstrokeopacity{0.000000}%
\pgfsetdash{}{0pt}%
\pgfpathmoveto{\pgfqpoint{3.083124in}{1.089029in}}%
\pgfpathlineto{\pgfqpoint{3.099339in}{1.091621in}}%
\pgfpathlineto{\pgfqpoint{3.116266in}{1.079750in}}%
\pgfpathlineto{\pgfqpoint{3.108787in}{1.052476in}}%
\pgfpathlineto{\pgfqpoint{3.095167in}{1.065172in}}%
\pgfpathlineto{\pgfqpoint{3.079593in}{1.063190in}}%
\pgfpathlineto{\pgfqpoint{3.065576in}{1.050744in}}%
\pgfpathlineto{\pgfqpoint{3.060908in}{1.061022in}}%
\pgfpathlineto{\pgfqpoint{3.054685in}{1.068940in}}%
\pgfpathlineto{\pgfqpoint{3.042269in}{1.083215in}}%
\pgfpathlineto{\pgfqpoint{3.083124in}{1.089029in}}%
\pgfpathclose%
\pgfusepath{fill}%
\end{pgfscope}%
\begin{pgfscope}%
\pgfpathrectangle{\pgfqpoint{0.100000in}{0.100000in}}{\pgfqpoint{3.608454in}{2.310000in}}%
\pgfusepath{clip}%
\pgfsetbuttcap%
\pgfsetmiterjoin%
\definecolor{currentfill}{rgb}{0.000000,0.713725,0.643137}%
\pgfsetfillcolor{currentfill}%
\pgfsetlinewidth{0.000000pt}%
\definecolor{currentstroke}{rgb}{0.000000,0.000000,0.000000}%
\pgfsetstrokecolor{currentstroke}%
\pgfsetstrokeopacity{0.000000}%
\pgfsetdash{}{0pt}%
\pgfpathmoveto{\pgfqpoint{1.748728in}{0.562219in}}%
\pgfpathlineto{\pgfqpoint{1.709808in}{0.564368in}}%
\pgfpathlineto{\pgfqpoint{1.702681in}{0.571973in}}%
\pgfpathlineto{\pgfqpoint{1.701376in}{0.578127in}}%
\pgfpathlineto{\pgfqpoint{1.687298in}{0.588703in}}%
\pgfpathlineto{\pgfqpoint{1.682525in}{0.597073in}}%
\pgfpathlineto{\pgfqpoint{1.678141in}{0.597930in}}%
\pgfpathlineto{\pgfqpoint{1.672342in}{0.605442in}}%
\pgfpathlineto{\pgfqpoint{1.667527in}{0.607537in}}%
\pgfpathlineto{\pgfqpoint{1.661236in}{0.620787in}}%
\pgfpathlineto{\pgfqpoint{1.653250in}{0.625462in}}%
\pgfpathlineto{\pgfqpoint{1.648358in}{0.623948in}}%
\pgfpathlineto{\pgfqpoint{1.623671in}{0.627789in}}%
\pgfpathlineto{\pgfqpoint{1.617429in}{0.627185in}}%
\pgfpathlineto{\pgfqpoint{1.599353in}{0.636098in}}%
\pgfpathlineto{\pgfqpoint{1.583492in}{0.651351in}}%
\pgfpathlineto{\pgfqpoint{1.585018in}{0.669731in}}%
\pgfpathlineto{\pgfqpoint{1.600425in}{0.668531in}}%
\pgfpathlineto{\pgfqpoint{1.602461in}{0.693594in}}%
\pgfpathlineto{\pgfqpoint{1.616475in}{0.692376in}}%
\pgfpathlineto{\pgfqpoint{1.616885in}{0.697027in}}%
\pgfpathlineto{\pgfqpoint{1.644365in}{0.694777in}}%
\pgfpathlineto{\pgfqpoint{1.648797in}{0.693179in}}%
\pgfpathlineto{\pgfqpoint{1.649534in}{0.685985in}}%
\pgfpathlineto{\pgfqpoint{1.642706in}{0.676280in}}%
\pgfpathlineto{\pgfqpoint{1.647908in}{0.670831in}}%
\pgfpathlineto{\pgfqpoint{1.640690in}{0.665716in}}%
\pgfpathlineto{\pgfqpoint{1.695622in}{0.661844in}}%
\pgfpathlineto{\pgfqpoint{1.713553in}{0.660732in}}%
\pgfpathlineto{\pgfqpoint{1.710253in}{0.607672in}}%
\pgfpathlineto{\pgfqpoint{1.751098in}{0.605189in}}%
\pgfpathlineto{\pgfqpoint{1.748728in}{0.562219in}}%
\pgfpathclose%
\pgfusepath{fill}%
\end{pgfscope}%
\begin{pgfscope}%
\pgfpathrectangle{\pgfqpoint{0.100000in}{0.100000in}}{\pgfqpoint{3.608454in}{2.310000in}}%
\pgfusepath{clip}%
\pgfsetbuttcap%
\pgfsetmiterjoin%
\definecolor{currentfill}{rgb}{0.000000,0.654902,0.672549}%
\pgfsetfillcolor{currentfill}%
\pgfsetlinewidth{0.000000pt}%
\definecolor{currentstroke}{rgb}{0.000000,0.000000,0.000000}%
\pgfsetstrokecolor{currentstroke}%
\pgfsetstrokeopacity{0.000000}%
\pgfsetdash{}{0pt}%
\pgfpathmoveto{\pgfqpoint{3.051009in}{1.477958in}}%
\pgfpathlineto{\pgfqpoint{3.051273in}{1.485944in}}%
\pgfpathlineto{\pgfqpoint{3.048216in}{1.491838in}}%
\pgfpathlineto{\pgfqpoint{3.054176in}{1.507940in}}%
\pgfpathlineto{\pgfqpoint{3.061504in}{1.519627in}}%
\pgfpathlineto{\pgfqpoint{3.068856in}{1.543568in}}%
\pgfpathlineto{\pgfqpoint{3.072318in}{1.562246in}}%
\pgfpathlineto{\pgfqpoint{3.098915in}{1.567130in}}%
\pgfpathlineto{\pgfqpoint{3.107213in}{1.564628in}}%
\pgfpathlineto{\pgfqpoint{3.111267in}{1.560885in}}%
\pgfpathlineto{\pgfqpoint{3.110368in}{1.555879in}}%
\pgfpathlineto{\pgfqpoint{3.115439in}{1.550679in}}%
\pgfpathlineto{\pgfqpoint{3.110584in}{1.534960in}}%
\pgfpathlineto{\pgfqpoint{3.112902in}{1.529498in}}%
\pgfpathlineto{\pgfqpoint{3.119792in}{1.526024in}}%
\pgfpathlineto{\pgfqpoint{3.116662in}{1.519085in}}%
\pgfpathlineto{\pgfqpoint{3.117244in}{1.511717in}}%
\pgfpathlineto{\pgfqpoint{3.114952in}{1.497798in}}%
\pgfpathlineto{\pgfqpoint{3.111561in}{1.488897in}}%
\pgfpathlineto{\pgfqpoint{3.051009in}{1.477958in}}%
\pgfpathclose%
\pgfusepath{fill}%
\end{pgfscope}%
\begin{pgfscope}%
\pgfpathrectangle{\pgfqpoint{0.100000in}{0.100000in}}{\pgfqpoint{3.608454in}{2.310000in}}%
\pgfusepath{clip}%
\pgfsetbuttcap%
\pgfsetmiterjoin%
\definecolor{currentfill}{rgb}{0.000000,0.400000,0.800000}%
\pgfsetfillcolor{currentfill}%
\pgfsetlinewidth{0.000000pt}%
\definecolor{currentstroke}{rgb}{0.000000,0.000000,0.000000}%
\pgfsetstrokecolor{currentstroke}%
\pgfsetstrokeopacity{0.000000}%
\pgfsetdash{}{0pt}%
\pgfpathmoveto{\pgfqpoint{2.335984in}{1.355415in}}%
\pgfpathlineto{\pgfqpoint{2.336652in}{1.332026in}}%
\pgfpathlineto{\pgfqpoint{2.327371in}{1.331768in}}%
\pgfpathlineto{\pgfqpoint{2.327621in}{1.321058in}}%
\pgfpathlineto{\pgfqpoint{2.319587in}{1.318096in}}%
\pgfpathlineto{\pgfqpoint{2.313973in}{1.320063in}}%
\pgfpathlineto{\pgfqpoint{2.313325in}{1.348137in}}%
\pgfpathlineto{\pgfqpoint{2.284099in}{1.347536in}}%
\pgfpathlineto{\pgfqpoint{2.284008in}{1.361432in}}%
\pgfpathlineto{\pgfqpoint{2.271150in}{1.361570in}}%
\pgfpathlineto{\pgfqpoint{2.271164in}{1.369568in}}%
\pgfpathlineto{\pgfqpoint{2.307376in}{1.370072in}}%
\pgfpathlineto{\pgfqpoint{2.324527in}{1.369104in}}%
\pgfpathlineto{\pgfqpoint{2.326861in}{1.355261in}}%
\pgfpathlineto{\pgfqpoint{2.335984in}{1.355415in}}%
\pgfpathclose%
\pgfusepath{fill}%
\end{pgfscope}%
\begin{pgfscope}%
\pgfpathrectangle{\pgfqpoint{0.100000in}{0.100000in}}{\pgfqpoint{3.608454in}{2.310000in}}%
\pgfusepath{clip}%
\pgfsetbuttcap%
\pgfsetmiterjoin%
\definecolor{currentfill}{rgb}{0.000000,0.537255,0.731373}%
\pgfsetfillcolor{currentfill}%
\pgfsetlinewidth{0.000000pt}%
\definecolor{currentstroke}{rgb}{0.000000,0.000000,0.000000}%
\pgfsetstrokecolor{currentstroke}%
\pgfsetstrokeopacity{0.000000}%
\pgfsetdash{}{0pt}%
\pgfpathmoveto{\pgfqpoint{2.182439in}{1.381621in}}%
\pgfpathlineto{\pgfqpoint{2.155044in}{1.381527in}}%
\pgfpathlineto{\pgfqpoint{2.154965in}{1.375795in}}%
\pgfpathlineto{\pgfqpoint{2.131152in}{1.375734in}}%
\pgfpathlineto{\pgfqpoint{2.131100in}{1.381764in}}%
\pgfpathlineto{\pgfqpoint{2.100521in}{1.381808in}}%
\pgfpathlineto{\pgfqpoint{2.103446in}{1.388933in}}%
\pgfpathlineto{\pgfqpoint{2.099783in}{1.391231in}}%
\pgfpathlineto{\pgfqpoint{2.086114in}{1.391298in}}%
\pgfpathlineto{\pgfqpoint{2.086211in}{1.418729in}}%
\pgfpathlineto{\pgfqpoint{2.088125in}{1.418722in}}%
\pgfpathlineto{\pgfqpoint{2.094548in}{1.411120in}}%
\pgfpathlineto{\pgfqpoint{2.101635in}{1.407776in}}%
\pgfpathlineto{\pgfqpoint{2.107194in}{1.411974in}}%
\pgfpathlineto{\pgfqpoint{2.104194in}{1.428993in}}%
\pgfpathlineto{\pgfqpoint{2.130612in}{1.428716in}}%
\pgfpathlineto{\pgfqpoint{2.130676in}{1.421843in}}%
\pgfpathlineto{\pgfqpoint{2.154074in}{1.421652in}}%
\pgfpathlineto{\pgfqpoint{2.154147in}{1.429674in}}%
\pgfpathlineto{\pgfqpoint{2.181491in}{1.429705in}}%
\pgfpathlineto{\pgfqpoint{2.181819in}{1.416001in}}%
\pgfpathlineto{\pgfqpoint{2.182439in}{1.381621in}}%
\pgfpathclose%
\pgfusepath{fill}%
\end{pgfscope}%
\begin{pgfscope}%
\pgfpathrectangle{\pgfqpoint{0.100000in}{0.100000in}}{\pgfqpoint{3.608454in}{2.310000in}}%
\pgfusepath{clip}%
\pgfsetbuttcap%
\pgfsetmiterjoin%
\definecolor{currentfill}{rgb}{0.000000,0.521569,0.739216}%
\pgfsetfillcolor{currentfill}%
\pgfsetlinewidth{0.000000pt}%
\definecolor{currentstroke}{rgb}{0.000000,0.000000,0.000000}%
\pgfsetstrokecolor{currentstroke}%
\pgfsetstrokeopacity{0.000000}%
\pgfsetdash{}{0pt}%
\pgfpathmoveto{\pgfqpoint{2.461626in}{1.152803in}}%
\pgfpathlineto{\pgfqpoint{2.457328in}{1.152433in}}%
\pgfpathlineto{\pgfqpoint{2.453898in}{1.152266in}}%
\pgfpathlineto{\pgfqpoint{2.454276in}{1.139497in}}%
\pgfpathlineto{\pgfqpoint{2.450987in}{1.144593in}}%
\pgfpathlineto{\pgfqpoint{2.440887in}{1.142638in}}%
\pgfpathlineto{\pgfqpoint{2.426853in}{1.142086in}}%
\pgfpathlineto{\pgfqpoint{2.426666in}{1.161074in}}%
\pgfpathlineto{\pgfqpoint{2.414649in}{1.160533in}}%
\pgfpathlineto{\pgfqpoint{2.413654in}{1.167791in}}%
\pgfpathlineto{\pgfqpoint{2.406457in}{1.183183in}}%
\pgfpathlineto{\pgfqpoint{2.415125in}{1.196618in}}%
\pgfpathlineto{\pgfqpoint{2.408011in}{1.196285in}}%
\pgfpathlineto{\pgfqpoint{2.407413in}{1.214350in}}%
\pgfpathlineto{\pgfqpoint{2.411944in}{1.214363in}}%
\pgfpathlineto{\pgfqpoint{2.410673in}{1.236890in}}%
\pgfpathlineto{\pgfqpoint{2.444353in}{1.238826in}}%
\pgfpathlineto{\pgfqpoint{2.449905in}{1.236650in}}%
\pgfpathlineto{\pgfqpoint{2.457038in}{1.223290in}}%
\pgfpathlineto{\pgfqpoint{2.451445in}{1.214650in}}%
\pgfpathlineto{\pgfqpoint{2.460932in}{1.196917in}}%
\pgfpathlineto{\pgfqpoint{2.467123in}{1.192232in}}%
\pgfpathlineto{\pgfqpoint{2.472760in}{1.194489in}}%
\pgfpathlineto{\pgfqpoint{2.477226in}{1.192084in}}%
\pgfpathlineto{\pgfqpoint{2.479456in}{1.190579in}}%
\pgfpathlineto{\pgfqpoint{2.477580in}{1.181419in}}%
\pgfpathlineto{\pgfqpoint{2.478809in}{1.176580in}}%
\pgfpathlineto{\pgfqpoint{2.474357in}{1.170803in}}%
\pgfpathlineto{\pgfqpoint{2.477166in}{1.167044in}}%
\pgfpathlineto{\pgfqpoint{2.473332in}{1.159111in}}%
\pgfpathlineto{\pgfqpoint{2.464384in}{1.163029in}}%
\pgfpathlineto{\pgfqpoint{2.461626in}{1.152803in}}%
\pgfpathclose%
\pgfusepath{fill}%
\end{pgfscope}%
\begin{pgfscope}%
\pgfpathrectangle{\pgfqpoint{0.100000in}{0.100000in}}{\pgfqpoint{3.608454in}{2.310000in}}%
\pgfusepath{clip}%
\pgfsetbuttcap%
\pgfsetmiterjoin%
\definecolor{currentfill}{rgb}{0.000000,0.686275,0.656863}%
\pgfsetfillcolor{currentfill}%
\pgfsetlinewidth{0.000000pt}%
\definecolor{currentstroke}{rgb}{0.000000,0.000000,0.000000}%
\pgfsetstrokecolor{currentstroke}%
\pgfsetstrokeopacity{0.000000}%
\pgfsetdash{}{0pt}%
\pgfpathmoveto{\pgfqpoint{2.547754in}{0.792351in}}%
\pgfpathlineto{\pgfqpoint{2.549469in}{0.776821in}}%
\pgfpathlineto{\pgfqpoint{2.551986in}{0.756063in}}%
\pgfpathlineto{\pgfqpoint{2.525381in}{0.754028in}}%
\pgfpathlineto{\pgfqpoint{2.487539in}{0.751603in}}%
\pgfpathlineto{\pgfqpoint{2.485619in}{0.780455in}}%
\pgfpathlineto{\pgfqpoint{2.468672in}{0.778139in}}%
\pgfpathlineto{\pgfqpoint{2.467492in}{0.796662in}}%
\pgfpathlineto{\pgfqpoint{2.462239in}{0.799019in}}%
\pgfpathlineto{\pgfqpoint{2.461411in}{0.812905in}}%
\pgfpathlineto{\pgfqpoint{2.488805in}{0.814690in}}%
\pgfpathlineto{\pgfqpoint{2.516272in}{0.816527in}}%
\pgfpathlineto{\pgfqpoint{2.518619in}{0.784958in}}%
\pgfpathlineto{\pgfqpoint{2.536679in}{0.788374in}}%
\pgfpathlineto{\pgfqpoint{2.547754in}{0.792351in}}%
\pgfpathclose%
\pgfusepath{fill}%
\end{pgfscope}%
\begin{pgfscope}%
\pgfpathrectangle{\pgfqpoint{0.100000in}{0.100000in}}{\pgfqpoint{3.608454in}{2.310000in}}%
\pgfusepath{clip}%
\pgfsetbuttcap%
\pgfsetmiterjoin%
\definecolor{currentfill}{rgb}{0.000000,0.611765,0.694118}%
\pgfsetfillcolor{currentfill}%
\pgfsetlinewidth{0.000000pt}%
\definecolor{currentstroke}{rgb}{0.000000,0.000000,0.000000}%
\pgfsetstrokecolor{currentstroke}%
\pgfsetstrokeopacity{0.000000}%
\pgfsetdash{}{0pt}%
\pgfpathmoveto{\pgfqpoint{1.557865in}{2.068157in}}%
\pgfpathlineto{\pgfqpoint{1.551007in}{2.068911in}}%
\pgfpathlineto{\pgfqpoint{1.548607in}{2.048174in}}%
\pgfpathlineto{\pgfqpoint{1.546296in}{2.048425in}}%
\pgfpathlineto{\pgfqpoint{1.543137in}{2.020826in}}%
\pgfpathlineto{\pgfqpoint{1.520131in}{2.023482in}}%
\pgfpathlineto{\pgfqpoint{1.519298in}{2.016515in}}%
\pgfpathlineto{\pgfqpoint{1.512427in}{2.017326in}}%
\pgfpathlineto{\pgfqpoint{1.513276in}{2.024315in}}%
\pgfpathlineto{\pgfqpoint{1.506437in}{2.025137in}}%
\pgfpathlineto{\pgfqpoint{1.501940in}{2.025689in}}%
\pgfpathlineto{\pgfqpoint{1.503629in}{2.039468in}}%
\pgfpathlineto{\pgfqpoint{1.496777in}{2.040294in}}%
\pgfpathlineto{\pgfqpoint{1.500521in}{2.053891in}}%
\pgfpathlineto{\pgfqpoint{1.503848in}{2.080824in}}%
\pgfpathlineto{\pgfqpoint{1.498672in}{2.081480in}}%
\pgfpathlineto{\pgfqpoint{1.496354in}{2.088951in}}%
\pgfpathlineto{\pgfqpoint{1.498726in}{2.097310in}}%
\pgfpathlineto{\pgfqpoint{1.507101in}{2.093171in}}%
\pgfpathlineto{\pgfqpoint{1.527435in}{2.089391in}}%
\pgfpathlineto{\pgfqpoint{1.529442in}{2.105984in}}%
\pgfpathlineto{\pgfqpoint{1.531517in}{2.105747in}}%
\pgfpathlineto{\pgfqpoint{1.534783in}{2.133134in}}%
\pgfpathlineto{\pgfqpoint{1.578597in}{2.128096in}}%
\pgfpathlineto{\pgfqpoint{1.589978in}{2.126848in}}%
\pgfpathlineto{\pgfqpoint{1.589220in}{2.119949in}}%
\pgfpathlineto{\pgfqpoint{1.596092in}{2.119196in}}%
\pgfpathlineto{\pgfqpoint{1.595340in}{2.112255in}}%
\pgfpathlineto{\pgfqpoint{1.626106in}{2.109098in}}%
\pgfpathlineto{\pgfqpoint{1.623124in}{2.077735in}}%
\pgfpathlineto{\pgfqpoint{1.619223in}{2.081466in}}%
\pgfpathlineto{\pgfqpoint{1.607083in}{2.083595in}}%
\pgfpathlineto{\pgfqpoint{1.601867in}{2.089959in}}%
\pgfpathlineto{\pgfqpoint{1.590181in}{2.089744in}}%
\pgfpathlineto{\pgfqpoint{1.582871in}{2.091842in}}%
\pgfpathlineto{\pgfqpoint{1.569948in}{2.088334in}}%
\pgfpathlineto{\pgfqpoint{1.562458in}{2.089780in}}%
\pgfpathlineto{\pgfqpoint{1.560678in}{2.074073in}}%
\pgfpathlineto{\pgfqpoint{1.557865in}{2.068157in}}%
\pgfpathclose%
\pgfusepath{fill}%
\end{pgfscope}%
\begin{pgfscope}%
\pgfpathrectangle{\pgfqpoint{0.100000in}{0.100000in}}{\pgfqpoint{3.608454in}{2.310000in}}%
\pgfusepath{clip}%
\pgfsetbuttcap%
\pgfsetmiterjoin%
\definecolor{currentfill}{rgb}{0.000000,0.854902,0.572549}%
\pgfsetfillcolor{currentfill}%
\pgfsetlinewidth{0.000000pt}%
\definecolor{currentstroke}{rgb}{0.000000,0.000000,0.000000}%
\pgfsetstrokecolor{currentstroke}%
\pgfsetstrokeopacity{0.000000}%
\pgfsetdash{}{0pt}%
\pgfpathmoveto{\pgfqpoint{1.795506in}{0.488863in}}%
\pgfpathlineto{\pgfqpoint{1.744553in}{0.490983in}}%
\pgfpathlineto{\pgfqpoint{1.748728in}{0.562219in}}%
\pgfpathlineto{\pgfqpoint{1.797569in}{0.560061in}}%
\pgfpathlineto{\pgfqpoint{1.796154in}{0.523908in}}%
\pgfpathlineto{\pgfqpoint{1.797091in}{0.523880in}}%
\pgfpathlineto{\pgfqpoint{1.795506in}{0.488863in}}%
\pgfpathclose%
\pgfusepath{fill}%
\end{pgfscope}%
\begin{pgfscope}%
\pgfpathrectangle{\pgfqpoint{0.100000in}{0.100000in}}{\pgfqpoint{3.608454in}{2.310000in}}%
\pgfusepath{clip}%
\pgfsetbuttcap%
\pgfsetmiterjoin%
\definecolor{currentfill}{rgb}{0.000000,0.666667,0.666667}%
\pgfsetfillcolor{currentfill}%
\pgfsetlinewidth{0.000000pt}%
\definecolor{currentstroke}{rgb}{0.000000,0.000000,0.000000}%
\pgfsetstrokecolor{currentstroke}%
\pgfsetstrokeopacity{0.000000}%
\pgfsetdash{}{0pt}%
\pgfpathmoveto{\pgfqpoint{1.443108in}{0.820998in}}%
\pgfpathlineto{\pgfqpoint{1.498395in}{0.815195in}}%
\pgfpathlineto{\pgfqpoint{1.486032in}{0.744751in}}%
\pgfpathlineto{\pgfqpoint{1.426837in}{0.715528in}}%
\pgfpathlineto{\pgfqpoint{1.429579in}{0.734241in}}%
\pgfpathlineto{\pgfqpoint{1.435480in}{0.791748in}}%
\pgfpathlineto{\pgfqpoint{1.438374in}{0.821513in}}%
\pgfpathlineto{\pgfqpoint{1.443108in}{0.820998in}}%
\pgfpathclose%
\pgfusepath{fill}%
\end{pgfscope}%
\begin{pgfscope}%
\pgfpathrectangle{\pgfqpoint{0.100000in}{0.100000in}}{\pgfqpoint{3.608454in}{2.310000in}}%
\pgfusepath{clip}%
\pgfsetbuttcap%
\pgfsetmiterjoin%
\definecolor{currentfill}{rgb}{0.000000,0.721569,0.639216}%
\pgfsetfillcolor{currentfill}%
\pgfsetlinewidth{0.000000pt}%
\definecolor{currentstroke}{rgb}{0.000000,0.000000,0.000000}%
\pgfsetstrokecolor{currentstroke}%
\pgfsetstrokeopacity{0.000000}%
\pgfsetdash{}{0pt}%
\pgfpathmoveto{\pgfqpoint{1.596724in}{1.204258in}}%
\pgfpathlineto{\pgfqpoint{1.602002in}{1.203777in}}%
\pgfpathlineto{\pgfqpoint{1.598511in}{1.164402in}}%
\pgfpathlineto{\pgfqpoint{1.596009in}{1.164627in}}%
\pgfpathlineto{\pgfqpoint{1.590719in}{1.104616in}}%
\pgfpathlineto{\pgfqpoint{1.569376in}{1.106608in}}%
\pgfpathlineto{\pgfqpoint{1.572537in}{1.133851in}}%
\pgfpathlineto{\pgfqpoint{1.545193in}{1.136393in}}%
\pgfpathlineto{\pgfqpoint{1.545891in}{1.143339in}}%
\pgfpathlineto{\pgfqpoint{1.532225in}{1.144723in}}%
\pgfpathlineto{\pgfqpoint{1.532570in}{1.148149in}}%
\pgfpathlineto{\pgfqpoint{1.538731in}{1.209408in}}%
\pgfpathlineto{\pgfqpoint{1.596724in}{1.204258in}}%
\pgfpathclose%
\pgfusepath{fill}%
\end{pgfscope}%
\begin{pgfscope}%
\pgfpathrectangle{\pgfqpoint{0.100000in}{0.100000in}}{\pgfqpoint{3.608454in}{2.310000in}}%
\pgfusepath{clip}%
\pgfsetbuttcap%
\pgfsetmiterjoin%
\definecolor{currentfill}{rgb}{0.000000,0.827451,0.586275}%
\pgfsetfillcolor{currentfill}%
\pgfsetlinewidth{0.000000pt}%
\definecolor{currentstroke}{rgb}{0.000000,0.000000,0.000000}%
\pgfsetstrokecolor{currentstroke}%
\pgfsetstrokeopacity{0.000000}%
\pgfsetdash{}{0pt}%
\pgfpathmoveto{\pgfqpoint{0.655369in}{0.626110in}}%
\pgfpathlineto{\pgfqpoint{0.652280in}{0.629547in}}%
\pgfpathlineto{\pgfqpoint{0.652480in}{0.635784in}}%
\pgfpathlineto{\pgfqpoint{0.651564in}{0.638676in}}%
\pgfpathlineto{\pgfqpoint{0.654993in}{0.639490in}}%
\pgfpathlineto{\pgfqpoint{0.653508in}{0.635458in}}%
\pgfpathlineto{\pgfqpoint{0.653829in}{0.632282in}}%
\pgfpathlineto{\pgfqpoint{0.656074in}{0.628020in}}%
\pgfpathlineto{\pgfqpoint{0.655369in}{0.626110in}}%
\pgfpathclose%
\pgfusepath{fill}%
\end{pgfscope}%
\begin{pgfscope}%
\pgfpathrectangle{\pgfqpoint{0.100000in}{0.100000in}}{\pgfqpoint{3.608454in}{2.310000in}}%
\pgfusepath{clip}%
\pgfsetbuttcap%
\pgfsetmiterjoin%
\definecolor{currentfill}{rgb}{0.000000,0.827451,0.586275}%
\pgfsetfillcolor{currentfill}%
\pgfsetlinewidth{0.000000pt}%
\definecolor{currentstroke}{rgb}{0.000000,0.000000,0.000000}%
\pgfsetstrokecolor{currentstroke}%
\pgfsetstrokeopacity{0.000000}%
\pgfsetdash{}{0pt}%
\pgfpathmoveto{\pgfqpoint{0.705347in}{0.551735in}}%
\pgfpathlineto{\pgfqpoint{0.705017in}{0.553533in}}%
\pgfpathlineto{\pgfqpoint{0.702425in}{0.555599in}}%
\pgfpathlineto{\pgfqpoint{0.701422in}{0.558633in}}%
\pgfpathlineto{\pgfqpoint{0.701257in}{0.561453in}}%
\pgfpathlineto{\pgfqpoint{0.699278in}{0.565095in}}%
\pgfpathlineto{\pgfqpoint{0.699632in}{0.569687in}}%
\pgfpathlineto{\pgfqpoint{0.700852in}{0.572122in}}%
\pgfpathlineto{\pgfqpoint{0.702514in}{0.571731in}}%
\pgfpathlineto{\pgfqpoint{0.706248in}{0.566407in}}%
\pgfpathlineto{\pgfqpoint{0.710913in}{0.566989in}}%
\pgfpathlineto{\pgfqpoint{0.713889in}{0.566223in}}%
\pgfpathlineto{\pgfqpoint{0.714374in}{0.564683in}}%
\pgfpathlineto{\pgfqpoint{0.716943in}{0.563500in}}%
\pgfpathlineto{\pgfqpoint{0.716755in}{0.561123in}}%
\pgfpathlineto{\pgfqpoint{0.719375in}{0.558723in}}%
\pgfpathlineto{\pgfqpoint{0.719272in}{0.555594in}}%
\pgfpathlineto{\pgfqpoint{0.717154in}{0.553156in}}%
\pgfpathlineto{\pgfqpoint{0.715054in}{0.551897in}}%
\pgfpathlineto{\pgfqpoint{0.714314in}{0.546900in}}%
\pgfpathlineto{\pgfqpoint{0.711935in}{0.547634in}}%
\pgfpathlineto{\pgfqpoint{0.708937in}{0.550219in}}%
\pgfpathlineto{\pgfqpoint{0.705347in}{0.551735in}}%
\pgfpathclose%
\pgfusepath{fill}%
\end{pgfscope}%
\begin{pgfscope}%
\pgfpathrectangle{\pgfqpoint{0.100000in}{0.100000in}}{\pgfqpoint{3.608454in}{2.310000in}}%
\pgfusepath{clip}%
\pgfsetbuttcap%
\pgfsetmiterjoin%
\definecolor{currentfill}{rgb}{0.000000,0.827451,0.586275}%
\pgfsetfillcolor{currentfill}%
\pgfsetlinewidth{0.000000pt}%
\definecolor{currentstroke}{rgb}{0.000000,0.000000,0.000000}%
\pgfsetstrokecolor{currentstroke}%
\pgfsetstrokeopacity{0.000000}%
\pgfsetdash{}{0pt}%
\pgfpathmoveto{\pgfqpoint{0.894650in}{0.485392in}}%
\pgfpathlineto{\pgfqpoint{0.878747in}{0.464023in}}%
\pgfpathlineto{\pgfqpoint{0.878374in}{0.466226in}}%
\pgfpathlineto{\pgfqpoint{0.874246in}{0.469508in}}%
\pgfpathlineto{\pgfqpoint{0.870359in}{0.464352in}}%
\pgfpathlineto{\pgfqpoint{0.869217in}{0.461424in}}%
\pgfpathlineto{\pgfqpoint{0.864761in}{0.455531in}}%
\pgfpathlineto{\pgfqpoint{0.853555in}{0.463966in}}%
\pgfpathlineto{\pgfqpoint{0.836300in}{0.477446in}}%
\pgfpathlineto{\pgfqpoint{0.822897in}{0.488128in}}%
\pgfpathlineto{\pgfqpoint{0.824629in}{0.490280in}}%
\pgfpathlineto{\pgfqpoint{0.820776in}{0.493434in}}%
\pgfpathlineto{\pgfqpoint{0.819027in}{0.491316in}}%
\pgfpathlineto{\pgfqpoint{0.816987in}{0.492818in}}%
\pgfpathlineto{\pgfqpoint{0.815390in}{0.490823in}}%
\pgfpathlineto{\pgfqpoint{0.813395in}{0.492489in}}%
\pgfpathlineto{\pgfqpoint{0.814957in}{0.494457in}}%
\pgfpathlineto{\pgfqpoint{0.804074in}{0.503387in}}%
\pgfpathlineto{\pgfqpoint{0.802432in}{0.501473in}}%
\pgfpathlineto{\pgfqpoint{0.800941in}{0.502662in}}%
\pgfpathlineto{\pgfqpoint{0.799211in}{0.500636in}}%
\pgfpathlineto{\pgfqpoint{0.797356in}{0.502139in}}%
\pgfpathlineto{\pgfqpoint{0.795190in}{0.500666in}}%
\pgfpathlineto{\pgfqpoint{0.791902in}{0.496847in}}%
\pgfpathlineto{\pgfqpoint{0.789763in}{0.498705in}}%
\pgfpathlineto{\pgfqpoint{0.788112in}{0.496794in}}%
\pgfpathlineto{\pgfqpoint{0.786206in}{0.498438in}}%
\pgfpathlineto{\pgfqpoint{0.782945in}{0.494696in}}%
\pgfpathlineto{\pgfqpoint{0.781042in}{0.496290in}}%
\pgfpathlineto{\pgfqpoint{0.778697in}{0.494994in}}%
\pgfpathlineto{\pgfqpoint{0.775519in}{0.491429in}}%
\pgfpathlineto{\pgfqpoint{0.773696in}{0.492989in}}%
\pgfpathlineto{\pgfqpoint{0.770402in}{0.489224in}}%
\pgfpathlineto{\pgfqpoint{0.767845in}{0.491365in}}%
\pgfpathlineto{\pgfqpoint{0.764542in}{0.487613in}}%
\pgfpathlineto{\pgfqpoint{0.762636in}{0.489281in}}%
\pgfpathlineto{\pgfqpoint{0.759307in}{0.485494in}}%
\pgfpathlineto{\pgfqpoint{0.756995in}{0.484116in}}%
\pgfpathlineto{\pgfqpoint{0.755169in}{0.485784in}}%
\pgfpathlineto{\pgfqpoint{0.752873in}{0.484455in}}%
\pgfpathlineto{\pgfqpoint{0.749547in}{0.480704in}}%
\pgfpathlineto{\pgfqpoint{0.747682in}{0.482304in}}%
\pgfpathlineto{\pgfqpoint{0.746173in}{0.480590in}}%
\pgfpathlineto{\pgfqpoint{0.744357in}{0.482168in}}%
\pgfpathlineto{\pgfqpoint{0.740722in}{0.478712in}}%
\pgfpathlineto{\pgfqpoint{0.740538in}{0.478732in}}%
\pgfpathlineto{\pgfqpoint{0.737273in}{0.481661in}}%
\pgfpathlineto{\pgfqpoint{0.735963in}{0.479492in}}%
\pgfpathlineto{\pgfqpoint{0.734892in}{0.478188in}}%
\pgfpathlineto{\pgfqpoint{0.727742in}{0.479830in}}%
\pgfpathlineto{\pgfqpoint{0.728503in}{0.483239in}}%
\pgfpathlineto{\pgfqpoint{0.731793in}{0.484480in}}%
\pgfpathlineto{\pgfqpoint{0.735386in}{0.488777in}}%
\pgfpathlineto{\pgfqpoint{0.736702in}{0.491209in}}%
\pgfpathlineto{\pgfqpoint{0.736725in}{0.494624in}}%
\pgfpathlineto{\pgfqpoint{0.737998in}{0.497489in}}%
\pgfpathlineto{\pgfqpoint{0.739339in}{0.496648in}}%
\pgfpathlineto{\pgfqpoint{0.741092in}{0.498876in}}%
\pgfpathlineto{\pgfqpoint{0.745616in}{0.498996in}}%
\pgfpathlineto{\pgfqpoint{0.746961in}{0.501426in}}%
\pgfpathlineto{\pgfqpoint{0.748923in}{0.507053in}}%
\pgfpathlineto{\pgfqpoint{0.749487in}{0.509973in}}%
\pgfpathlineto{\pgfqpoint{0.750888in}{0.512260in}}%
\pgfpathlineto{\pgfqpoint{0.751044in}{0.515415in}}%
\pgfpathlineto{\pgfqpoint{0.754037in}{0.517140in}}%
\pgfpathlineto{\pgfqpoint{0.753941in}{0.521042in}}%
\pgfpathlineto{\pgfqpoint{0.752010in}{0.522096in}}%
\pgfpathlineto{\pgfqpoint{0.748553in}{0.518499in}}%
\pgfpathlineto{\pgfqpoint{0.745838in}{0.518264in}}%
\pgfpathlineto{\pgfqpoint{0.744687in}{0.519881in}}%
\pgfpathlineto{\pgfqpoint{0.738995in}{0.520517in}}%
\pgfpathlineto{\pgfqpoint{0.733381in}{0.523327in}}%
\pgfpathlineto{\pgfqpoint{0.730861in}{0.525258in}}%
\pgfpathlineto{\pgfqpoint{0.728015in}{0.528863in}}%
\pgfpathlineto{\pgfqpoint{0.727312in}{0.530894in}}%
\pgfpathlineto{\pgfqpoint{0.728494in}{0.533400in}}%
\pgfpathlineto{\pgfqpoint{0.730236in}{0.534212in}}%
\pgfpathlineto{\pgfqpoint{0.729733in}{0.537152in}}%
\pgfpathlineto{\pgfqpoint{0.730007in}{0.540442in}}%
\pgfpathlineto{\pgfqpoint{0.731612in}{0.540941in}}%
\pgfpathlineto{\pgfqpoint{0.730213in}{0.545263in}}%
\pgfpathlineto{\pgfqpoint{0.731563in}{0.546123in}}%
\pgfpathlineto{\pgfqpoint{0.728919in}{0.548253in}}%
\pgfpathlineto{\pgfqpoint{0.728522in}{0.551246in}}%
\pgfpathlineto{\pgfqpoint{0.729262in}{0.553763in}}%
\pgfpathlineto{\pgfqpoint{0.731148in}{0.553340in}}%
\pgfpathlineto{\pgfqpoint{0.731806in}{0.555149in}}%
\pgfpathlineto{\pgfqpoint{0.729059in}{0.555932in}}%
\pgfpathlineto{\pgfqpoint{0.728087in}{0.558977in}}%
\pgfpathlineto{\pgfqpoint{0.731970in}{0.558050in}}%
\pgfpathlineto{\pgfqpoint{0.736704in}{0.560032in}}%
\pgfpathlineto{\pgfqpoint{0.739886in}{0.560095in}}%
\pgfpathlineto{\pgfqpoint{0.738328in}{0.563808in}}%
\pgfpathlineto{\pgfqpoint{0.739124in}{0.565657in}}%
\pgfpathlineto{\pgfqpoint{0.741979in}{0.566643in}}%
\pgfpathlineto{\pgfqpoint{0.742241in}{0.570962in}}%
\pgfpathlineto{\pgfqpoint{0.739760in}{0.569380in}}%
\pgfpathlineto{\pgfqpoint{0.738485in}{0.573179in}}%
\pgfpathlineto{\pgfqpoint{0.741686in}{0.576289in}}%
\pgfpathlineto{\pgfqpoint{0.739656in}{0.579992in}}%
\pgfpathlineto{\pgfqpoint{0.741135in}{0.581696in}}%
\pgfpathlineto{\pgfqpoint{0.743178in}{0.581393in}}%
\pgfpathlineto{\pgfqpoint{0.744280in}{0.583186in}}%
\pgfpathlineto{\pgfqpoint{0.743630in}{0.585155in}}%
\pgfpathlineto{\pgfqpoint{0.740303in}{0.585479in}}%
\pgfpathlineto{\pgfqpoint{0.741843in}{0.588299in}}%
\pgfpathlineto{\pgfqpoint{0.746374in}{0.589014in}}%
\pgfpathlineto{\pgfqpoint{0.747420in}{0.592192in}}%
\pgfpathlineto{\pgfqpoint{0.750741in}{0.589193in}}%
\pgfpathlineto{\pgfqpoint{0.753431in}{0.589913in}}%
\pgfpathlineto{\pgfqpoint{0.754707in}{0.592801in}}%
\pgfpathlineto{\pgfqpoint{0.757034in}{0.594176in}}%
\pgfpathlineto{\pgfqpoint{0.767817in}{0.596834in}}%
\pgfpathlineto{\pgfqpoint{0.772792in}{0.597266in}}%
\pgfpathlineto{\pgfqpoint{0.774461in}{0.595375in}}%
\pgfpathlineto{\pgfqpoint{0.777267in}{0.596507in}}%
\pgfpathlineto{\pgfqpoint{0.778048in}{0.599219in}}%
\pgfpathlineto{\pgfqpoint{0.783684in}{0.603781in}}%
\pgfpathlineto{\pgfqpoint{0.789079in}{0.605814in}}%
\pgfpathlineto{\pgfqpoint{0.794866in}{0.605498in}}%
\pgfpathlineto{\pgfqpoint{0.798355in}{0.602512in}}%
\pgfpathlineto{\pgfqpoint{0.799353in}{0.597971in}}%
\pgfpathlineto{\pgfqpoint{0.799751in}{0.593446in}}%
\pgfpathlineto{\pgfqpoint{0.801076in}{0.590791in}}%
\pgfpathlineto{\pgfqpoint{0.803852in}{0.589451in}}%
\pgfpathlineto{\pgfqpoint{0.808679in}{0.590732in}}%
\pgfpathlineto{\pgfqpoint{0.812082in}{0.590252in}}%
\pgfpathlineto{\pgfqpoint{0.814106in}{0.588275in}}%
\pgfpathlineto{\pgfqpoint{0.812396in}{0.586335in}}%
\pgfpathlineto{\pgfqpoint{0.814282in}{0.584612in}}%
\pgfpathlineto{\pgfqpoint{0.812542in}{0.582561in}}%
\pgfpathlineto{\pgfqpoint{0.814280in}{0.580959in}}%
\pgfpathlineto{\pgfqpoint{0.812726in}{0.579011in}}%
\pgfpathlineto{\pgfqpoint{0.824219in}{0.568707in}}%
\pgfpathlineto{\pgfqpoint{0.823025in}{0.569214in}}%
\pgfpathlineto{\pgfqpoint{0.820507in}{0.567425in}}%
\pgfpathlineto{\pgfqpoint{0.813592in}{0.559787in}}%
\pgfpathlineto{\pgfqpoint{0.813107in}{0.560166in}}%
\pgfpathlineto{\pgfqpoint{0.806268in}{0.552525in}}%
\pgfpathlineto{\pgfqpoint{0.807682in}{0.551257in}}%
\pgfpathlineto{\pgfqpoint{0.802699in}{0.545595in}}%
\pgfpathlineto{\pgfqpoint{0.812784in}{0.536599in}}%
\pgfpathlineto{\pgfqpoint{0.819842in}{0.530526in}}%
\pgfpathlineto{\pgfqpoint{0.821946in}{0.532976in}}%
\pgfpathlineto{\pgfqpoint{0.830011in}{0.526141in}}%
\pgfpathlineto{\pgfqpoint{0.831654in}{0.528086in}}%
\pgfpathlineto{\pgfqpoint{0.846928in}{0.515267in}}%
\pgfpathlineto{\pgfqpoint{0.845087in}{0.513012in}}%
\pgfpathlineto{\pgfqpoint{0.854046in}{0.505533in}}%
\pgfpathlineto{\pgfqpoint{0.867623in}{0.494883in}}%
\pgfpathlineto{\pgfqpoint{0.878680in}{0.486403in}}%
\pgfpathlineto{\pgfqpoint{0.880946in}{0.484915in}}%
\pgfpathlineto{\pgfqpoint{0.883274in}{0.487912in}}%
\pgfpathlineto{\pgfqpoint{0.887256in}{0.484755in}}%
\pgfpathlineto{\pgfqpoint{0.888822in}{0.486756in}}%
\pgfpathlineto{\pgfqpoint{0.892390in}{0.483966in}}%
\pgfpathlineto{\pgfqpoint{0.894650in}{0.485392in}}%
\pgfpathclose%
\pgfusepath{fill}%
\end{pgfscope}%
\begin{pgfscope}%
\pgfpathrectangle{\pgfqpoint{0.100000in}{0.100000in}}{\pgfqpoint{3.608454in}{2.310000in}}%
\pgfusepath{clip}%
\pgfsetbuttcap%
\pgfsetmiterjoin%
\definecolor{currentfill}{rgb}{0.000000,0.713725,0.643137}%
\pgfsetfillcolor{currentfill}%
\pgfsetlinewidth{0.000000pt}%
\definecolor{currentstroke}{rgb}{0.000000,0.000000,0.000000}%
\pgfsetstrokecolor{currentstroke}%
\pgfsetstrokeopacity{0.000000}%
\pgfsetdash{}{0pt}%
\pgfpathmoveto{\pgfqpoint{1.012082in}{1.949489in}}%
\pgfpathlineto{\pgfqpoint{1.009235in}{1.946422in}}%
\pgfpathlineto{\pgfqpoint{1.004445in}{1.925280in}}%
\pgfpathlineto{\pgfqpoint{0.989615in}{1.915702in}}%
\pgfpathlineto{\pgfqpoint{0.981235in}{1.920433in}}%
\pgfpathlineto{\pgfqpoint{0.976747in}{1.912867in}}%
\pgfpathlineto{\pgfqpoint{0.969772in}{1.911307in}}%
\pgfpathlineto{\pgfqpoint{0.967552in}{1.905305in}}%
\pgfpathlineto{\pgfqpoint{0.970494in}{1.901193in}}%
\pgfpathlineto{\pgfqpoint{0.969006in}{1.895451in}}%
\pgfpathlineto{\pgfqpoint{0.964170in}{1.890384in}}%
\pgfpathlineto{\pgfqpoint{0.953430in}{1.884971in}}%
\pgfpathlineto{\pgfqpoint{0.949827in}{1.885175in}}%
\pgfpathlineto{\pgfqpoint{0.917584in}{1.892808in}}%
\pgfpathlineto{\pgfqpoint{0.915455in}{1.886073in}}%
\pgfpathlineto{\pgfqpoint{0.909972in}{1.887658in}}%
\pgfpathlineto{\pgfqpoint{0.913150in}{1.901010in}}%
\pgfpathlineto{\pgfqpoint{0.916455in}{1.900234in}}%
\pgfpathlineto{\pgfqpoint{0.919645in}{1.913720in}}%
\pgfpathlineto{\pgfqpoint{0.910516in}{1.911417in}}%
\pgfpathlineto{\pgfqpoint{0.904574in}{1.912817in}}%
\pgfpathlineto{\pgfqpoint{0.898851in}{1.918021in}}%
\pgfpathlineto{\pgfqpoint{0.901098in}{1.927625in}}%
\pgfpathlineto{\pgfqpoint{0.896730in}{1.932620in}}%
\pgfpathlineto{\pgfqpoint{0.899991in}{1.946190in}}%
\pgfpathlineto{\pgfqpoint{0.884981in}{1.950005in}}%
\pgfpathlineto{\pgfqpoint{0.890375in}{1.956866in}}%
\pgfpathlineto{\pgfqpoint{0.891257in}{1.963356in}}%
\pgfpathlineto{\pgfqpoint{0.895815in}{1.966858in}}%
\pgfpathlineto{\pgfqpoint{0.899963in}{1.971006in}}%
\pgfpathlineto{\pgfqpoint{0.904646in}{1.980156in}}%
\pgfpathlineto{\pgfqpoint{0.923373in}{1.975654in}}%
\pgfpathlineto{\pgfqpoint{0.923969in}{1.964164in}}%
\pgfpathlineto{\pgfqpoint{0.931228in}{1.960671in}}%
\pgfpathlineto{\pgfqpoint{0.942731in}{1.965121in}}%
\pgfpathlineto{\pgfqpoint{1.012082in}{1.949489in}}%
\pgfpathclose%
\pgfusepath{fill}%
\end{pgfscope}%
\begin{pgfscope}%
\pgfpathrectangle{\pgfqpoint{0.100000in}{0.100000in}}{\pgfqpoint{3.608454in}{2.310000in}}%
\pgfusepath{clip}%
\pgfsetbuttcap%
\pgfsetmiterjoin%
\definecolor{currentfill}{rgb}{0.000000,0.709804,0.645098}%
\pgfsetfillcolor{currentfill}%
\pgfsetlinewidth{0.000000pt}%
\definecolor{currentstroke}{rgb}{0.000000,0.000000,0.000000}%
\pgfsetstrokecolor{currentstroke}%
\pgfsetstrokeopacity{0.000000}%
\pgfsetdash{}{0pt}%
\pgfpathmoveto{\pgfqpoint{2.477226in}{1.192084in}}%
\pgfpathlineto{\pgfqpoint{2.472760in}{1.194489in}}%
\pgfpathlineto{\pgfqpoint{2.467123in}{1.192232in}}%
\pgfpathlineto{\pgfqpoint{2.460932in}{1.196917in}}%
\pgfpathlineto{\pgfqpoint{2.451445in}{1.214650in}}%
\pgfpathlineto{\pgfqpoint{2.457038in}{1.223290in}}%
\pgfpathlineto{\pgfqpoint{2.449905in}{1.236650in}}%
\pgfpathlineto{\pgfqpoint{2.452467in}{1.239392in}}%
\pgfpathlineto{\pgfqpoint{2.449790in}{1.246442in}}%
\pgfpathlineto{\pgfqpoint{2.445368in}{1.248025in}}%
\pgfpathlineto{\pgfqpoint{2.434982in}{1.258000in}}%
\pgfpathlineto{\pgfqpoint{2.428315in}{1.262231in}}%
\pgfpathlineto{\pgfqpoint{2.421525in}{1.260176in}}%
\pgfpathlineto{\pgfqpoint{2.418250in}{1.266369in}}%
\pgfpathlineto{\pgfqpoint{2.409704in}{1.273645in}}%
\pgfpathlineto{\pgfqpoint{2.404880in}{1.275417in}}%
\pgfpathlineto{\pgfqpoint{2.415279in}{1.279749in}}%
\pgfpathlineto{\pgfqpoint{2.414877in}{1.286654in}}%
\pgfpathlineto{\pgfqpoint{2.435549in}{1.287521in}}%
\pgfpathlineto{\pgfqpoint{2.470034in}{1.289131in}}%
\pgfpathlineto{\pgfqpoint{2.470338in}{1.282183in}}%
\pgfpathlineto{\pgfqpoint{2.497937in}{1.284068in}}%
\pgfpathlineto{\pgfqpoint{2.499015in}{1.266823in}}%
\pgfpathlineto{\pgfqpoint{2.500587in}{1.242556in}}%
\pgfpathlineto{\pgfqpoint{2.501888in}{1.221865in}}%
\pgfpathlineto{\pgfqpoint{2.489480in}{1.220900in}}%
\pgfpathlineto{\pgfqpoint{2.488838in}{1.212212in}}%
\pgfpathlineto{\pgfqpoint{2.482512in}{1.210574in}}%
\pgfpathlineto{\pgfqpoint{2.473800in}{1.196498in}}%
\pgfpathlineto{\pgfqpoint{2.477226in}{1.192084in}}%
\pgfpathclose%
\pgfusepath{fill}%
\end{pgfscope}%
\begin{pgfscope}%
\pgfpathrectangle{\pgfqpoint{0.100000in}{0.100000in}}{\pgfqpoint{3.608454in}{2.310000in}}%
\pgfusepath{clip}%
\pgfsetbuttcap%
\pgfsetmiterjoin%
\definecolor{currentfill}{rgb}{0.000000,0.454902,0.772549}%
\pgfsetfillcolor{currentfill}%
\pgfsetlinewidth{0.000000pt}%
\definecolor{currentstroke}{rgb}{0.000000,0.000000,0.000000}%
\pgfsetstrokecolor{currentstroke}%
\pgfsetstrokeopacity{0.000000}%
\pgfsetdash{}{0pt}%
\pgfpathmoveto{\pgfqpoint{3.245459in}{1.515286in}}%
\pgfpathlineto{\pgfqpoint{3.239646in}{1.514041in}}%
\pgfpathlineto{\pgfqpoint{3.239245in}{1.513965in}}%
\pgfpathlineto{\pgfqpoint{3.227209in}{1.526852in}}%
\pgfpathlineto{\pgfqpoint{3.221777in}{1.528537in}}%
\pgfpathlineto{\pgfqpoint{3.216418in}{1.535851in}}%
\pgfpathlineto{\pgfqpoint{3.208552in}{1.535223in}}%
\pgfpathlineto{\pgfqpoint{3.204878in}{1.540940in}}%
\pgfpathlineto{\pgfqpoint{3.212062in}{1.546662in}}%
\pgfpathlineto{\pgfqpoint{3.201240in}{1.566916in}}%
\pgfpathlineto{\pgfqpoint{3.208359in}{1.574869in}}%
\pgfpathlineto{\pgfqpoint{3.196959in}{1.580835in}}%
\pgfpathlineto{\pgfqpoint{3.213947in}{1.593817in}}%
\pgfpathlineto{\pgfqpoint{3.217800in}{1.596744in}}%
\pgfpathlineto{\pgfqpoint{3.217800in}{1.603350in}}%
\pgfpathlineto{\pgfqpoint{3.221218in}{1.609411in}}%
\pgfpathlineto{\pgfqpoint{3.234122in}{1.609162in}}%
\pgfpathlineto{\pgfqpoint{3.251074in}{1.598426in}}%
\pgfpathlineto{\pgfqpoint{3.244201in}{1.592250in}}%
\pgfpathlineto{\pgfqpoint{3.269282in}{1.579037in}}%
\pgfpathlineto{\pgfqpoint{3.262873in}{1.561037in}}%
\pgfpathlineto{\pgfqpoint{3.250514in}{1.547606in}}%
\pgfpathlineto{\pgfqpoint{3.252158in}{1.542409in}}%
\pgfpathlineto{\pgfqpoint{3.250059in}{1.534993in}}%
\pgfpathlineto{\pgfqpoint{3.250934in}{1.527034in}}%
\pgfpathlineto{\pgfqpoint{3.248921in}{1.519116in}}%
\pgfpathlineto{\pgfqpoint{3.245459in}{1.515286in}}%
\pgfpathclose%
\pgfusepath{fill}%
\end{pgfscope}%
\begin{pgfscope}%
\pgfpathrectangle{\pgfqpoint{0.100000in}{0.100000in}}{\pgfqpoint{3.608454in}{2.310000in}}%
\pgfusepath{clip}%
\pgfsetbuttcap%
\pgfsetmiterjoin%
\definecolor{currentfill}{rgb}{0.000000,0.592157,0.703922}%
\pgfsetfillcolor{currentfill}%
\pgfsetlinewidth{0.000000pt}%
\definecolor{currentstroke}{rgb}{0.000000,0.000000,0.000000}%
\pgfsetstrokecolor{currentstroke}%
\pgfsetstrokeopacity{0.000000}%
\pgfsetdash{}{0pt}%
\pgfpathmoveto{\pgfqpoint{2.887245in}{0.726366in}}%
\pgfpathlineto{\pgfqpoint{2.878161in}{0.725802in}}%
\pgfpathlineto{\pgfqpoint{2.874929in}{0.755916in}}%
\pgfpathlineto{\pgfqpoint{2.856733in}{0.753994in}}%
\pgfpathlineto{\pgfqpoint{2.856403in}{0.756814in}}%
\pgfpathlineto{\pgfqpoint{2.854226in}{0.777256in}}%
\pgfpathlineto{\pgfqpoint{2.877599in}{0.779820in}}%
\pgfpathlineto{\pgfqpoint{2.875554in}{0.798633in}}%
\pgfpathlineto{\pgfqpoint{2.870385in}{0.802308in}}%
\pgfpathlineto{\pgfqpoint{2.865088in}{0.801722in}}%
\pgfpathlineto{\pgfqpoint{2.862925in}{0.815951in}}%
\pgfpathlineto{\pgfqpoint{2.875686in}{0.817596in}}%
\pgfpathlineto{\pgfqpoint{2.875215in}{0.821548in}}%
\pgfpathlineto{\pgfqpoint{2.884148in}{0.822135in}}%
\pgfpathlineto{\pgfqpoint{2.887379in}{0.811341in}}%
\pgfpathlineto{\pgfqpoint{2.884962in}{0.810094in}}%
\pgfpathlineto{\pgfqpoint{2.885535in}{0.800050in}}%
\pgfpathlineto{\pgfqpoint{2.891131in}{0.799434in}}%
\pgfpathlineto{\pgfqpoint{2.894796in}{0.794134in}}%
\pgfpathlineto{\pgfqpoint{2.910462in}{0.795395in}}%
\pgfpathlineto{\pgfqpoint{2.911798in}{0.789691in}}%
\pgfpathlineto{\pgfqpoint{2.920094in}{0.780812in}}%
\pgfpathlineto{\pgfqpoint{2.920160in}{0.773477in}}%
\pgfpathlineto{\pgfqpoint{2.925268in}{0.774204in}}%
\pgfpathlineto{\pgfqpoint{2.928578in}{0.749275in}}%
\pgfpathlineto{\pgfqpoint{2.937427in}{0.747697in}}%
\pgfpathlineto{\pgfqpoint{2.943923in}{0.740325in}}%
\pgfpathlineto{\pgfqpoint{2.955222in}{0.740323in}}%
\pgfpathlineto{\pgfqpoint{2.957900in}{0.730883in}}%
\pgfpathlineto{\pgfqpoint{2.887245in}{0.726366in}}%
\pgfpathclose%
\pgfusepath{fill}%
\end{pgfscope}%
\begin{pgfscope}%
\pgfpathrectangle{\pgfqpoint{0.100000in}{0.100000in}}{\pgfqpoint{3.608454in}{2.310000in}}%
\pgfusepath{clip}%
\pgfsetbuttcap%
\pgfsetmiterjoin%
\definecolor{currentfill}{rgb}{0.000000,0.827451,0.586275}%
\pgfsetfillcolor{currentfill}%
\pgfsetlinewidth{0.000000pt}%
\definecolor{currentstroke}{rgb}{0.000000,0.000000,0.000000}%
\pgfsetstrokecolor{currentstroke}%
\pgfsetstrokeopacity{0.000000}%
\pgfsetdash{}{0pt}%
\pgfpathmoveto{\pgfqpoint{2.416578in}{0.896534in}}%
\pgfpathlineto{\pgfqpoint{2.408616in}{0.893304in}}%
\pgfpathlineto{\pgfqpoint{2.408409in}{0.898240in}}%
\pgfpathlineto{\pgfqpoint{2.406598in}{0.936460in}}%
\pgfpathlineto{\pgfqpoint{2.424326in}{0.937210in}}%
\pgfpathlineto{\pgfqpoint{2.416509in}{0.931498in}}%
\pgfpathlineto{\pgfqpoint{2.427713in}{0.930421in}}%
\pgfpathlineto{\pgfqpoint{2.427361in}{0.937371in}}%
\pgfpathlineto{\pgfqpoint{2.441137in}{0.939280in}}%
\pgfpathlineto{\pgfqpoint{2.440822in}{0.945153in}}%
\pgfpathlineto{\pgfqpoint{2.451281in}{0.943313in}}%
\pgfpathlineto{\pgfqpoint{2.468504in}{0.944341in}}%
\pgfpathlineto{\pgfqpoint{2.469213in}{0.932802in}}%
\pgfpathlineto{\pgfqpoint{2.470515in}{0.912039in}}%
\pgfpathlineto{\pgfqpoint{2.474111in}{0.910147in}}%
\pgfpathlineto{\pgfqpoint{2.474841in}{0.898468in}}%
\pgfpathlineto{\pgfqpoint{2.458419in}{0.897453in}}%
\pgfpathlineto{\pgfqpoint{2.455570in}{0.891636in}}%
\pgfpathlineto{\pgfqpoint{2.426856in}{0.899501in}}%
\pgfpathlineto{\pgfqpoint{2.419871in}{0.899622in}}%
\pgfpathlineto{\pgfqpoint{2.416578in}{0.896534in}}%
\pgfpathclose%
\pgfusepath{fill}%
\end{pgfscope}%
\begin{pgfscope}%
\pgfpathrectangle{\pgfqpoint{0.100000in}{0.100000in}}{\pgfqpoint{3.608454in}{2.310000in}}%
\pgfusepath{clip}%
\pgfsetbuttcap%
\pgfsetmiterjoin%
\definecolor{currentfill}{rgb}{0.000000,0.333333,0.833333}%
\pgfsetfillcolor{currentfill}%
\pgfsetlinewidth{0.000000pt}%
\definecolor{currentstroke}{rgb}{0.000000,0.000000,0.000000}%
\pgfsetstrokecolor{currentstroke}%
\pgfsetstrokeopacity{0.000000}%
\pgfsetdash{}{0pt}%
\pgfpathmoveto{\pgfqpoint{1.913591in}{0.794961in}}%
\pgfpathlineto{\pgfqpoint{1.908521in}{0.806762in}}%
\pgfpathlineto{\pgfqpoint{1.900998in}{0.829156in}}%
\pgfpathlineto{\pgfqpoint{1.873743in}{0.830250in}}%
\pgfpathlineto{\pgfqpoint{1.866975in}{0.830682in}}%
\pgfpathlineto{\pgfqpoint{1.868391in}{0.865576in}}%
\pgfpathlineto{\pgfqpoint{1.878275in}{0.865031in}}%
\pgfpathlineto{\pgfqpoint{1.878486in}{0.869447in}}%
\pgfpathlineto{\pgfqpoint{1.912098in}{0.867745in}}%
\pgfpathlineto{\pgfqpoint{1.913431in}{0.902088in}}%
\pgfpathlineto{\pgfqpoint{1.909495in}{0.904858in}}%
\pgfpathlineto{\pgfqpoint{1.910664in}{0.938402in}}%
\pgfpathlineto{\pgfqpoint{1.917943in}{0.934949in}}%
\pgfpathlineto{\pgfqpoint{1.931087in}{0.945835in}}%
\pgfpathlineto{\pgfqpoint{1.938143in}{0.938336in}}%
\pgfpathlineto{\pgfqpoint{1.943227in}{0.939444in}}%
\pgfpathlineto{\pgfqpoint{1.941990in}{0.901235in}}%
\pgfpathlineto{\pgfqpoint{1.948881in}{0.900975in}}%
\pgfpathlineto{\pgfqpoint{1.946927in}{0.865843in}}%
\pgfpathlineto{\pgfqpoint{1.971315in}{0.864806in}}%
\pgfpathlineto{\pgfqpoint{1.970138in}{0.830158in}}%
\pgfpathlineto{\pgfqpoint{1.966891in}{0.830295in}}%
\pgfpathlineto{\pgfqpoint{1.966416in}{0.807704in}}%
\pgfpathlineto{\pgfqpoint{1.940012in}{0.801698in}}%
\pgfpathlineto{\pgfqpoint{1.934684in}{0.799256in}}%
\pgfpathlineto{\pgfqpoint{1.929646in}{0.804003in}}%
\pgfpathlineto{\pgfqpoint{1.913591in}{0.794961in}}%
\pgfpathclose%
\pgfusepath{fill}%
\end{pgfscope}%
\begin{pgfscope}%
\pgfpathrectangle{\pgfqpoint{0.100000in}{0.100000in}}{\pgfqpoint{3.608454in}{2.310000in}}%
\pgfusepath{clip}%
\pgfsetbuttcap%
\pgfsetmiterjoin%
\definecolor{currentfill}{rgb}{0.000000,0.454902,0.772549}%
\pgfsetfillcolor{currentfill}%
\pgfsetlinewidth{0.000000pt}%
\definecolor{currentstroke}{rgb}{0.000000,0.000000,0.000000}%
\pgfsetstrokecolor{currentstroke}%
\pgfsetstrokeopacity{0.000000}%
\pgfsetdash{}{0pt}%
\pgfpathmoveto{\pgfqpoint{2.698781in}{1.052062in}}%
\pgfpathlineto{\pgfqpoint{2.669947in}{1.049451in}}%
\pgfpathlineto{\pgfqpoint{2.668473in}{1.060141in}}%
\pgfpathlineto{\pgfqpoint{2.665061in}{1.060409in}}%
\pgfpathlineto{\pgfqpoint{2.662888in}{1.068219in}}%
\pgfpathlineto{\pgfqpoint{2.648588in}{1.077186in}}%
\pgfpathlineto{\pgfqpoint{2.644297in}{1.082383in}}%
\pgfpathlineto{\pgfqpoint{2.643683in}{1.102257in}}%
\pgfpathlineto{\pgfqpoint{2.651600in}{1.103191in}}%
\pgfpathlineto{\pgfqpoint{2.654621in}{1.100124in}}%
\pgfpathlineto{\pgfqpoint{2.668887in}{1.104876in}}%
\pgfpathlineto{\pgfqpoint{2.681536in}{1.102539in}}%
\pgfpathlineto{\pgfqpoint{2.685775in}{1.104241in}}%
\pgfpathlineto{\pgfqpoint{2.688788in}{1.097666in}}%
\pgfpathlineto{\pgfqpoint{2.693772in}{1.094220in}}%
\pgfpathlineto{\pgfqpoint{2.694900in}{1.080379in}}%
\pgfpathlineto{\pgfqpoint{2.693282in}{1.075443in}}%
\pgfpathlineto{\pgfqpoint{2.696968in}{1.070359in}}%
\pgfpathlineto{\pgfqpoint{2.698781in}{1.052062in}}%
\pgfpathclose%
\pgfusepath{fill}%
\end{pgfscope}%
\begin{pgfscope}%
\pgfpathrectangle{\pgfqpoint{0.100000in}{0.100000in}}{\pgfqpoint{3.608454in}{2.310000in}}%
\pgfusepath{clip}%
\pgfsetbuttcap%
\pgfsetmiterjoin%
\definecolor{currentfill}{rgb}{0.000000,0.647059,0.676471}%
\pgfsetfillcolor{currentfill}%
\pgfsetlinewidth{0.000000pt}%
\definecolor{currentstroke}{rgb}{0.000000,0.000000,0.000000}%
\pgfsetstrokecolor{currentstroke}%
\pgfsetstrokeopacity{0.000000}%
\pgfsetdash{}{0pt}%
\pgfpathmoveto{\pgfqpoint{2.550184in}{0.875112in}}%
\pgfpathlineto{\pgfqpoint{2.519263in}{0.872679in}}%
\pgfpathlineto{\pgfqpoint{2.517297in}{0.901251in}}%
\pgfpathlineto{\pgfqpoint{2.526832in}{0.901949in}}%
\pgfpathlineto{\pgfqpoint{2.525476in}{0.919309in}}%
\pgfpathlineto{\pgfqpoint{2.534527in}{0.922423in}}%
\pgfpathlineto{\pgfqpoint{2.537884in}{0.927921in}}%
\pgfpathlineto{\pgfqpoint{2.535053in}{0.931443in}}%
\pgfpathlineto{\pgfqpoint{2.544297in}{0.934956in}}%
\pgfpathlineto{\pgfqpoint{2.546945in}{0.939900in}}%
\pgfpathlineto{\pgfqpoint{2.551959in}{0.940228in}}%
\pgfpathlineto{\pgfqpoint{2.551506in}{0.923399in}}%
\pgfpathlineto{\pgfqpoint{2.580198in}{0.924865in}}%
\pgfpathlineto{\pgfqpoint{2.582729in}{0.895488in}}%
\pgfpathlineto{\pgfqpoint{2.575754in}{0.891315in}}%
\pgfpathlineto{\pgfqpoint{2.570549in}{0.885685in}}%
\pgfpathlineto{\pgfqpoint{2.564922in}{0.885225in}}%
\pgfpathlineto{\pgfqpoint{2.561497in}{0.881275in}}%
\pgfpathlineto{\pgfqpoint{2.550318in}{0.880073in}}%
\pgfpathlineto{\pgfqpoint{2.550184in}{0.875112in}}%
\pgfpathclose%
\pgfusepath{fill}%
\end{pgfscope}%
\begin{pgfscope}%
\pgfpathrectangle{\pgfqpoint{0.100000in}{0.100000in}}{\pgfqpoint{3.608454in}{2.310000in}}%
\pgfusepath{clip}%
\pgfsetbuttcap%
\pgfsetmiterjoin%
\definecolor{currentfill}{rgb}{0.000000,0.725490,0.637255}%
\pgfsetfillcolor{currentfill}%
\pgfsetlinewidth{0.000000pt}%
\definecolor{currentstroke}{rgb}{0.000000,0.000000,0.000000}%
\pgfsetstrokecolor{currentstroke}%
\pgfsetstrokeopacity{0.000000}%
\pgfsetdash{}{0pt}%
\pgfpathmoveto{\pgfqpoint{3.215154in}{1.664914in}}%
\pgfpathlineto{\pgfqpoint{3.207689in}{1.691369in}}%
\pgfpathlineto{\pgfqpoint{3.183986in}{1.686549in}}%
\pgfpathlineto{\pgfqpoint{3.180627in}{1.708614in}}%
\pgfpathlineto{\pgfqpoint{3.187096in}{1.712911in}}%
\pgfpathlineto{\pgfqpoint{3.196916in}{1.713203in}}%
\pgfpathlineto{\pgfqpoint{3.194930in}{1.721757in}}%
\pgfpathlineto{\pgfqpoint{3.217063in}{1.727096in}}%
\pgfpathlineto{\pgfqpoint{3.220803in}{1.715267in}}%
\pgfpathlineto{\pgfqpoint{3.232758in}{1.716945in}}%
\pgfpathlineto{\pgfqpoint{3.233713in}{1.712856in}}%
\pgfpathlineto{\pgfqpoint{3.246255in}{1.715536in}}%
\pgfpathlineto{\pgfqpoint{3.248659in}{1.703631in}}%
\pgfpathlineto{\pgfqpoint{3.252945in}{1.701154in}}%
\pgfpathlineto{\pgfqpoint{3.245833in}{1.699582in}}%
\pgfpathlineto{\pgfqpoint{3.253079in}{1.672195in}}%
\pgfpathlineto{\pgfqpoint{3.215154in}{1.664914in}}%
\pgfpathclose%
\pgfusepath{fill}%
\end{pgfscope}%
\begin{pgfscope}%
\pgfpathrectangle{\pgfqpoint{0.100000in}{0.100000in}}{\pgfqpoint{3.608454in}{2.310000in}}%
\pgfusepath{clip}%
\pgfsetbuttcap%
\pgfsetmiterjoin%
\definecolor{currentfill}{rgb}{0.000000,0.521569,0.739216}%
\pgfsetfillcolor{currentfill}%
\pgfsetlinewidth{0.000000pt}%
\definecolor{currentstroke}{rgb}{0.000000,0.000000,0.000000}%
\pgfsetstrokecolor{currentstroke}%
\pgfsetstrokeopacity{0.000000}%
\pgfsetdash{}{0pt}%
\pgfpathmoveto{\pgfqpoint{2.020851in}{0.893088in}}%
\pgfpathlineto{\pgfqpoint{2.014996in}{0.896820in}}%
\pgfpathlineto{\pgfqpoint{1.984737in}{0.897974in}}%
\pgfpathlineto{\pgfqpoint{1.977949in}{0.898974in}}%
\pgfpathlineto{\pgfqpoint{1.978804in}{0.941263in}}%
\pgfpathlineto{\pgfqpoint{1.982724in}{0.934735in}}%
\pgfpathlineto{\pgfqpoint{1.993908in}{0.931626in}}%
\pgfpathlineto{\pgfqpoint{2.000025in}{0.945069in}}%
\pgfpathlineto{\pgfqpoint{2.014398in}{0.957120in}}%
\pgfpathlineto{\pgfqpoint{2.041706in}{0.956720in}}%
\pgfpathlineto{\pgfqpoint{2.045185in}{0.954206in}}%
\pgfpathlineto{\pgfqpoint{2.045055in}{0.940136in}}%
\pgfpathlineto{\pgfqpoint{2.051407in}{0.934194in}}%
\pgfpathlineto{\pgfqpoint{2.056583in}{0.934051in}}%
\pgfpathlineto{\pgfqpoint{2.050404in}{0.930484in}}%
\pgfpathlineto{\pgfqpoint{2.049865in}{0.897339in}}%
\pgfpathlineto{\pgfqpoint{2.020851in}{0.893088in}}%
\pgfpathclose%
\pgfusepath{fill}%
\end{pgfscope}%
\begin{pgfscope}%
\pgfpathrectangle{\pgfqpoint{0.100000in}{0.100000in}}{\pgfqpoint{3.608454in}{2.310000in}}%
\pgfusepath{clip}%
\pgfsetbuttcap%
\pgfsetmiterjoin%
\definecolor{currentfill}{rgb}{0.000000,0.564706,0.717647}%
\pgfsetfillcolor{currentfill}%
\pgfsetlinewidth{0.000000pt}%
\definecolor{currentstroke}{rgb}{0.000000,0.000000,0.000000}%
\pgfsetstrokecolor{currentstroke}%
\pgfsetstrokeopacity{0.000000}%
\pgfsetdash{}{0pt}%
\pgfpathmoveto{\pgfqpoint{2.561760in}{0.672494in}}%
\pgfpathlineto{\pgfqpoint{2.556138in}{0.721372in}}%
\pgfpathlineto{\pgfqpoint{2.551986in}{0.756063in}}%
\pgfpathlineto{\pgfqpoint{2.549469in}{0.776821in}}%
\pgfpathlineto{\pgfqpoint{2.574889in}{0.778819in}}%
\pgfpathlineto{\pgfqpoint{2.573394in}{0.783619in}}%
\pgfpathlineto{\pgfqpoint{2.567924in}{0.787803in}}%
\pgfpathlineto{\pgfqpoint{2.574126in}{0.801999in}}%
\pgfpathlineto{\pgfqpoint{2.601434in}{0.804239in}}%
\pgfpathlineto{\pgfqpoint{2.604373in}{0.805606in}}%
\pgfpathlineto{\pgfqpoint{2.603471in}{0.815667in}}%
\pgfpathlineto{\pgfqpoint{2.610325in}{0.816266in}}%
\pgfpathlineto{\pgfqpoint{2.609428in}{0.825937in}}%
\pgfpathlineto{\pgfqpoint{2.612796in}{0.827022in}}%
\pgfpathlineto{\pgfqpoint{2.622706in}{0.822514in}}%
\pgfpathlineto{\pgfqpoint{2.634007in}{0.811449in}}%
\pgfpathlineto{\pgfqpoint{2.652230in}{0.813085in}}%
\pgfpathlineto{\pgfqpoint{2.653773in}{0.795845in}}%
\pgfpathlineto{\pgfqpoint{2.613728in}{0.792291in}}%
\pgfpathlineto{\pgfqpoint{2.613565in}{0.781771in}}%
\pgfpathlineto{\pgfqpoint{2.610143in}{0.781433in}}%
\pgfpathlineto{\pgfqpoint{2.611445in}{0.760754in}}%
\pgfpathlineto{\pgfqpoint{2.605232in}{0.759125in}}%
\pgfpathlineto{\pgfqpoint{2.601460in}{0.754867in}}%
\pgfpathlineto{\pgfqpoint{2.602701in}{0.749237in}}%
\pgfpathlineto{\pgfqpoint{2.609757in}{0.745194in}}%
\pgfpathlineto{\pgfqpoint{2.612513in}{0.725646in}}%
\pgfpathlineto{\pgfqpoint{2.610908in}{0.714979in}}%
\pgfpathlineto{\pgfqpoint{2.618895in}{0.705619in}}%
\pgfpathlineto{\pgfqpoint{2.627748in}{0.701087in}}%
\pgfpathlineto{\pgfqpoint{2.628964in}{0.696389in}}%
\pgfpathlineto{\pgfqpoint{2.626017in}{0.689071in}}%
\pgfpathlineto{\pgfqpoint{2.632066in}{0.682154in}}%
\pgfpathlineto{\pgfqpoint{2.627322in}{0.678308in}}%
\pgfpathlineto{\pgfqpoint{2.622714in}{0.669017in}}%
\pgfpathlineto{\pgfqpoint{2.613254in}{0.665766in}}%
\pgfpathlineto{\pgfqpoint{2.606414in}{0.667773in}}%
\pgfpathlineto{\pgfqpoint{2.595006in}{0.677521in}}%
\pgfpathlineto{\pgfqpoint{2.592621in}{0.682803in}}%
\pgfpathlineto{\pgfqpoint{2.594483in}{0.687365in}}%
\pgfpathlineto{\pgfqpoint{2.592065in}{0.696291in}}%
\pgfpathlineto{\pgfqpoint{2.588127in}{0.698733in}}%
\pgfpathlineto{\pgfqpoint{2.583318in}{0.695147in}}%
\pgfpathlineto{\pgfqpoint{2.581260in}{0.684211in}}%
\pgfpathlineto{\pgfqpoint{2.581606in}{0.673454in}}%
\pgfpathlineto{\pgfqpoint{2.580066in}{0.669312in}}%
\pgfpathlineto{\pgfqpoint{2.564924in}{0.674357in}}%
\pgfpathlineto{\pgfqpoint{2.561760in}{0.672494in}}%
\pgfpathclose%
\pgfusepath{fill}%
\end{pgfscope}%
\begin{pgfscope}%
\pgfpathrectangle{\pgfqpoint{0.100000in}{0.100000in}}{\pgfqpoint{3.608454in}{2.310000in}}%
\pgfusepath{clip}%
\pgfsetbuttcap%
\pgfsetmiterjoin%
\definecolor{currentfill}{rgb}{0.000000,0.545098,0.727451}%
\pgfsetfillcolor{currentfill}%
\pgfsetlinewidth{0.000000pt}%
\definecolor{currentstroke}{rgb}{0.000000,0.000000,0.000000}%
\pgfsetstrokecolor{currentstroke}%
\pgfsetstrokeopacity{0.000000}%
\pgfsetdash{}{0pt}%
\pgfpathmoveto{\pgfqpoint{1.440664in}{1.064305in}}%
\pgfpathlineto{\pgfqpoint{1.436822in}{1.030070in}}%
\pgfpathlineto{\pgfqpoint{1.435358in}{1.030206in}}%
\pgfpathlineto{\pgfqpoint{1.432998in}{1.009699in}}%
\pgfpathlineto{\pgfqpoint{1.432205in}{1.002872in}}%
\pgfpathlineto{\pgfqpoint{1.392442in}{1.007538in}}%
\pgfpathlineto{\pgfqpoint{1.360437in}{1.011466in}}%
\pgfpathlineto{\pgfqpoint{1.362154in}{1.025644in}}%
\pgfpathlineto{\pgfqpoint{1.340304in}{1.035758in}}%
\pgfpathlineto{\pgfqpoint{1.333401in}{1.040515in}}%
\pgfpathlineto{\pgfqpoint{1.278910in}{1.047826in}}%
\pgfpathlineto{\pgfqpoint{1.236876in}{1.053946in}}%
\pgfpathlineto{\pgfqpoint{1.193614in}{1.060581in}}%
\pgfpathlineto{\pgfqpoint{1.198276in}{1.090371in}}%
\pgfpathlineto{\pgfqpoint{1.235282in}{1.084666in}}%
\pgfpathlineto{\pgfqpoint{1.239395in}{1.111911in}}%
\pgfpathlineto{\pgfqpoint{1.313512in}{1.101203in}}%
\pgfpathlineto{\pgfqpoint{1.321043in}{1.155597in}}%
\pgfpathlineto{\pgfqpoint{1.300870in}{1.158512in}}%
\pgfpathlineto{\pgfqpoint{1.303419in}{1.175695in}}%
\pgfpathlineto{\pgfqpoint{1.350154in}{1.169097in}}%
\pgfpathlineto{\pgfqpoint{1.347890in}{1.152024in}}%
\pgfpathlineto{\pgfqpoint{1.388440in}{1.147038in}}%
\pgfpathlineto{\pgfqpoint{1.378347in}{1.144995in}}%
\pgfpathlineto{\pgfqpoint{1.377042in}{1.134592in}}%
\pgfpathlineto{\pgfqpoint{1.385971in}{1.127718in}}%
\pgfpathlineto{\pgfqpoint{1.379252in}{1.071453in}}%
\pgfpathlineto{\pgfqpoint{1.413364in}{1.067432in}}%
\pgfpathlineto{\pgfqpoint{1.440664in}{1.064305in}}%
\pgfpathclose%
\pgfusepath{fill}%
\end{pgfscope}%
\begin{pgfscope}%
\pgfpathrectangle{\pgfqpoint{0.100000in}{0.100000in}}{\pgfqpoint{3.608454in}{2.310000in}}%
\pgfusepath{clip}%
\pgfsetbuttcap%
\pgfsetmiterjoin%
\definecolor{currentfill}{rgb}{0.000000,0.627451,0.686275}%
\pgfsetfillcolor{currentfill}%
\pgfsetlinewidth{0.000000pt}%
\definecolor{currentstroke}{rgb}{0.000000,0.000000,0.000000}%
\pgfsetstrokecolor{currentstroke}%
\pgfsetstrokeopacity{0.000000}%
\pgfsetdash{}{0pt}%
\pgfpathmoveto{\pgfqpoint{1.993561in}{0.650091in}}%
\pgfpathlineto{\pgfqpoint{1.986455in}{0.652084in}}%
\pgfpathlineto{\pgfqpoint{1.978079in}{0.661346in}}%
\pgfpathlineto{\pgfqpoint{1.975419in}{0.668492in}}%
\pgfpathlineto{\pgfqpoint{1.971709in}{0.671338in}}%
\pgfpathlineto{\pgfqpoint{1.995500in}{0.684736in}}%
\pgfpathlineto{\pgfqpoint{1.993815in}{0.692311in}}%
\pgfpathlineto{\pgfqpoint{1.986509in}{0.700655in}}%
\pgfpathlineto{\pgfqpoint{1.984720in}{0.711352in}}%
\pgfpathlineto{\pgfqpoint{1.981851in}{0.714949in}}%
\pgfpathlineto{\pgfqpoint{1.997875in}{0.723825in}}%
\pgfpathlineto{\pgfqpoint{2.016972in}{0.734385in}}%
\pgfpathlineto{\pgfqpoint{2.016298in}{0.728776in}}%
\pgfpathlineto{\pgfqpoint{2.021875in}{0.703710in}}%
\pgfpathlineto{\pgfqpoint{2.026699in}{0.691630in}}%
\pgfpathlineto{\pgfqpoint{2.047653in}{0.694679in}}%
\pgfpathlineto{\pgfqpoint{2.049731in}{0.675985in}}%
\pgfpathlineto{\pgfqpoint{2.051242in}{0.645234in}}%
\pgfpathlineto{\pgfqpoint{2.031567in}{0.644083in}}%
\pgfpathlineto{\pgfqpoint{2.031314in}{0.647827in}}%
\pgfpathlineto{\pgfqpoint{2.025563in}{0.657506in}}%
\pgfpathlineto{\pgfqpoint{2.016277in}{0.656724in}}%
\pgfpathlineto{\pgfqpoint{2.008653in}{0.653539in}}%
\pgfpathlineto{\pgfqpoint{1.993561in}{0.650091in}}%
\pgfpathclose%
\pgfusepath{fill}%
\end{pgfscope}%
\begin{pgfscope}%
\pgfpathrectangle{\pgfqpoint{0.100000in}{0.100000in}}{\pgfqpoint{3.608454in}{2.310000in}}%
\pgfusepath{clip}%
\pgfsetbuttcap%
\pgfsetmiterjoin%
\definecolor{currentfill}{rgb}{0.000000,0.850980,0.574510}%
\pgfsetfillcolor{currentfill}%
\pgfsetlinewidth{0.000000pt}%
\definecolor{currentstroke}{rgb}{0.000000,0.000000,0.000000}%
\pgfsetstrokecolor{currentstroke}%
\pgfsetstrokeopacity{0.000000}%
\pgfsetdash{}{0pt}%
\pgfpathmoveto{\pgfqpoint{2.991086in}{1.284432in}}%
\pgfpathlineto{\pgfqpoint{2.974604in}{1.294585in}}%
\pgfpathlineto{\pgfqpoint{2.968540in}{1.292624in}}%
\pgfpathlineto{\pgfqpoint{2.967623in}{1.287269in}}%
\pgfpathlineto{\pgfqpoint{2.961369in}{1.291611in}}%
\pgfpathlineto{\pgfqpoint{2.952046in}{1.304075in}}%
\pgfpathlineto{\pgfqpoint{2.946226in}{1.306488in}}%
\pgfpathlineto{\pgfqpoint{2.941096in}{1.316634in}}%
\pgfpathlineto{\pgfqpoint{2.944352in}{1.321971in}}%
\pgfpathlineto{\pgfqpoint{2.952368in}{1.321633in}}%
\pgfpathlineto{\pgfqpoint{2.953019in}{1.337104in}}%
\pgfpathlineto{\pgfqpoint{2.958041in}{1.346022in}}%
\pgfpathlineto{\pgfqpoint{2.968495in}{1.342247in}}%
\pgfpathlineto{\pgfqpoint{2.973652in}{1.345179in}}%
\pgfpathlineto{\pgfqpoint{2.981470in}{1.336849in}}%
\pgfpathlineto{\pgfqpoint{2.992200in}{1.336598in}}%
\pgfpathlineto{\pgfqpoint{3.006750in}{1.353994in}}%
\pgfpathlineto{\pgfqpoint{3.011961in}{1.351516in}}%
\pgfpathlineto{\pgfqpoint{3.013175in}{1.342782in}}%
\pgfpathlineto{\pgfqpoint{3.020093in}{1.338488in}}%
\pgfpathlineto{\pgfqpoint{3.026105in}{1.338632in}}%
\pgfpathlineto{\pgfqpoint{3.038592in}{1.342913in}}%
\pgfpathlineto{\pgfqpoint{3.037059in}{1.337144in}}%
\pgfpathlineto{\pgfqpoint{3.025954in}{1.320474in}}%
\pgfpathlineto{\pgfqpoint{3.022678in}{1.310484in}}%
\pgfpathlineto{\pgfqpoint{3.013170in}{1.308307in}}%
\pgfpathlineto{\pgfqpoint{3.000083in}{1.309688in}}%
\pgfpathlineto{\pgfqpoint{2.991086in}{1.284432in}}%
\pgfpathclose%
\pgfusepath{fill}%
\end{pgfscope}%
\begin{pgfscope}%
\pgfpathrectangle{\pgfqpoint{0.100000in}{0.100000in}}{\pgfqpoint{3.608454in}{2.310000in}}%
\pgfusepath{clip}%
\pgfsetbuttcap%
\pgfsetmiterjoin%
\definecolor{currentfill}{rgb}{0.000000,0.592157,0.703922}%
\pgfsetfillcolor{currentfill}%
\pgfsetlinewidth{0.000000pt}%
\definecolor{currentstroke}{rgb}{0.000000,0.000000,0.000000}%
\pgfsetstrokecolor{currentstroke}%
\pgfsetstrokeopacity{0.000000}%
\pgfsetdash{}{0pt}%
\pgfpathmoveto{\pgfqpoint{3.302028in}{1.725960in}}%
\pgfpathlineto{\pgfqpoint{3.267733in}{1.692380in}}%
\pgfpathlineto{\pgfqpoint{3.260640in}{1.692420in}}%
\pgfpathlineto{\pgfqpoint{3.258600in}{1.697493in}}%
\pgfpathlineto{\pgfqpoint{3.252945in}{1.701154in}}%
\pgfpathlineto{\pgfqpoint{3.248659in}{1.703631in}}%
\pgfpathlineto{\pgfqpoint{3.246255in}{1.715536in}}%
\pgfpathlineto{\pgfqpoint{3.233713in}{1.712856in}}%
\pgfpathlineto{\pgfqpoint{3.232758in}{1.716945in}}%
\pgfpathlineto{\pgfqpoint{3.220803in}{1.715267in}}%
\pgfpathlineto{\pgfqpoint{3.217063in}{1.727096in}}%
\pgfpathlineto{\pgfqpoint{3.210494in}{1.750645in}}%
\pgfpathlineto{\pgfqpoint{3.243936in}{1.759541in}}%
\pgfpathlineto{\pgfqpoint{3.246838in}{1.769301in}}%
\pgfpathlineto{\pgfqpoint{3.252161in}{1.774686in}}%
\pgfpathlineto{\pgfqpoint{3.264610in}{1.770699in}}%
\pgfpathlineto{\pgfqpoint{3.264917in}{1.776723in}}%
\pgfpathlineto{\pgfqpoint{3.272069in}{1.775491in}}%
\pgfpathlineto{\pgfqpoint{3.279173in}{1.774369in}}%
\pgfpathlineto{\pgfqpoint{3.279470in}{1.768032in}}%
\pgfpathlineto{\pgfqpoint{3.283745in}{1.758930in}}%
\pgfpathlineto{\pgfqpoint{3.281055in}{1.749476in}}%
\pgfpathlineto{\pgfqpoint{3.287975in}{1.743465in}}%
\pgfpathlineto{\pgfqpoint{3.300176in}{1.740144in}}%
\pgfpathlineto{\pgfqpoint{3.296564in}{1.727321in}}%
\pgfpathlineto{\pgfqpoint{3.302028in}{1.725960in}}%
\pgfpathclose%
\pgfusepath{fill}%
\end{pgfscope}%
\begin{pgfscope}%
\pgfpathrectangle{\pgfqpoint{0.100000in}{0.100000in}}{\pgfqpoint{3.608454in}{2.310000in}}%
\pgfusepath{clip}%
\pgfsetbuttcap%
\pgfsetmiterjoin%
\definecolor{currentfill}{rgb}{0.000000,0.478431,0.760784}%
\pgfsetfillcolor{currentfill}%
\pgfsetlinewidth{0.000000pt}%
\definecolor{currentstroke}{rgb}{0.000000,0.000000,0.000000}%
\pgfsetstrokecolor{currentstroke}%
\pgfsetstrokeopacity{0.000000}%
\pgfsetdash{}{0pt}%
\pgfpathmoveto{\pgfqpoint{1.917334in}{2.003750in}}%
\pgfpathlineto{\pgfqpoint{1.875835in}{2.005576in}}%
\pgfpathlineto{\pgfqpoint{1.877148in}{2.033168in}}%
\pgfpathlineto{\pgfqpoint{1.876114in}{2.047161in}}%
\pgfpathlineto{\pgfqpoint{1.917529in}{2.045335in}}%
\pgfpathlineto{\pgfqpoint{1.918355in}{2.031373in}}%
\pgfpathlineto{\pgfqpoint{1.917334in}{2.003750in}}%
\pgfpathclose%
\pgfusepath{fill}%
\end{pgfscope}%
\begin{pgfscope}%
\pgfpathrectangle{\pgfqpoint{0.100000in}{0.100000in}}{\pgfqpoint{3.608454in}{2.310000in}}%
\pgfusepath{clip}%
\pgfsetbuttcap%
\pgfsetmiterjoin%
\definecolor{currentfill}{rgb}{0.000000,0.576471,0.711765}%
\pgfsetfillcolor{currentfill}%
\pgfsetlinewidth{0.000000pt}%
\definecolor{currentstroke}{rgb}{0.000000,0.000000,0.000000}%
\pgfsetstrokecolor{currentstroke}%
\pgfsetstrokeopacity{0.000000}%
\pgfsetdash{}{0pt}%
\pgfpathmoveto{\pgfqpoint{1.933314in}{1.014281in}}%
\pgfpathlineto{\pgfqpoint{1.932881in}{1.000512in}}%
\pgfpathlineto{\pgfqpoint{1.905475in}{1.001362in}}%
\pgfpathlineto{\pgfqpoint{1.905912in}{1.015133in}}%
\pgfpathlineto{\pgfqpoint{1.871452in}{1.016408in}}%
\pgfpathlineto{\pgfqpoint{1.872137in}{1.035604in}}%
\pgfpathlineto{\pgfqpoint{1.873435in}{1.071561in}}%
\pgfpathlineto{\pgfqpoint{1.893386in}{1.070793in}}%
\pgfpathlineto{\pgfqpoint{1.893326in}{1.056979in}}%
\pgfpathlineto{\pgfqpoint{1.911760in}{1.056365in}}%
\pgfpathlineto{\pgfqpoint{1.923203in}{1.052514in}}%
\pgfpathlineto{\pgfqpoint{1.934200in}{1.052365in}}%
\pgfpathlineto{\pgfqpoint{1.933314in}{1.014281in}}%
\pgfpathclose%
\pgfusepath{fill}%
\end{pgfscope}%
\begin{pgfscope}%
\pgfpathrectangle{\pgfqpoint{0.100000in}{0.100000in}}{\pgfqpoint{3.608454in}{2.310000in}}%
\pgfusepath{clip}%
\pgfsetbuttcap%
\pgfsetmiterjoin%
\definecolor{currentfill}{rgb}{0.000000,0.458824,0.770588}%
\pgfsetfillcolor{currentfill}%
\pgfsetlinewidth{0.000000pt}%
\definecolor{currentstroke}{rgb}{0.000000,0.000000,0.000000}%
\pgfsetstrokecolor{currentstroke}%
\pgfsetstrokeopacity{0.000000}%
\pgfsetdash{}{0pt}%
\pgfpathmoveto{\pgfqpoint{2.165979in}{1.695985in}}%
\pgfpathlineto{\pgfqpoint{2.166463in}{1.649078in}}%
\pgfpathlineto{\pgfqpoint{2.111776in}{1.648912in}}%
\pgfpathlineto{\pgfqpoint{2.084271in}{1.648958in}}%
\pgfpathlineto{\pgfqpoint{2.084392in}{1.676317in}}%
\pgfpathlineto{\pgfqpoint{2.084490in}{1.695746in}}%
\pgfpathlineto{\pgfqpoint{2.111687in}{1.695719in}}%
\pgfpathlineto{\pgfqpoint{2.111765in}{1.676267in}}%
\pgfpathlineto{\pgfqpoint{2.138939in}{1.676359in}}%
\pgfpathlineto{\pgfqpoint{2.138833in}{1.695825in}}%
\pgfpathlineto{\pgfqpoint{2.165979in}{1.695985in}}%
\pgfpathclose%
\pgfusepath{fill}%
\end{pgfscope}%
\begin{pgfscope}%
\pgfpathrectangle{\pgfqpoint{0.100000in}{0.100000in}}{\pgfqpoint{3.608454in}{2.310000in}}%
\pgfusepath{clip}%
\pgfsetbuttcap%
\pgfsetmiterjoin%
\definecolor{currentfill}{rgb}{0.000000,0.713725,0.643137}%
\pgfsetfillcolor{currentfill}%
\pgfsetlinewidth{0.000000pt}%
\definecolor{currentstroke}{rgb}{0.000000,0.000000,0.000000}%
\pgfsetstrokecolor{currentstroke}%
\pgfsetstrokeopacity{0.000000}%
\pgfsetdash{}{0pt}%
\pgfpathmoveto{\pgfqpoint{2.824999in}{1.309058in}}%
\pgfpathlineto{\pgfqpoint{2.825277in}{1.301789in}}%
\pgfpathlineto{\pgfqpoint{2.820760in}{1.291911in}}%
\pgfpathlineto{\pgfqpoint{2.814875in}{1.291146in}}%
\pgfpathlineto{\pgfqpoint{2.806402in}{1.296003in}}%
\pgfpathlineto{\pgfqpoint{2.793467in}{1.297141in}}%
\pgfpathlineto{\pgfqpoint{2.784277in}{1.310449in}}%
\pgfpathlineto{\pgfqpoint{2.796848in}{1.326453in}}%
\pgfpathlineto{\pgfqpoint{2.804061in}{1.326162in}}%
\pgfpathlineto{\pgfqpoint{2.811045in}{1.319515in}}%
\pgfpathlineto{\pgfqpoint{2.824999in}{1.309058in}}%
\pgfpathclose%
\pgfusepath{fill}%
\end{pgfscope}%
\begin{pgfscope}%
\pgfpathrectangle{\pgfqpoint{0.100000in}{0.100000in}}{\pgfqpoint{3.608454in}{2.310000in}}%
\pgfusepath{clip}%
\pgfsetbuttcap%
\pgfsetmiterjoin%
\definecolor{currentfill}{rgb}{0.000000,0.607843,0.696078}%
\pgfsetfillcolor{currentfill}%
\pgfsetlinewidth{0.000000pt}%
\definecolor{currentstroke}{rgb}{0.000000,0.000000,0.000000}%
\pgfsetstrokecolor{currentstroke}%
\pgfsetstrokeopacity{0.000000}%
\pgfsetdash{}{0pt}%
\pgfpathmoveto{\pgfqpoint{2.586580in}{1.532701in}}%
\pgfpathlineto{\pgfqpoint{2.588380in}{1.512229in}}%
\pgfpathlineto{\pgfqpoint{2.585581in}{1.506060in}}%
\pgfpathlineto{\pgfqpoint{2.578828in}{1.505444in}}%
\pgfpathlineto{\pgfqpoint{2.579594in}{1.497487in}}%
\pgfpathlineto{\pgfqpoint{2.554067in}{1.495355in}}%
\pgfpathlineto{\pgfqpoint{2.550384in}{1.539587in}}%
\pgfpathlineto{\pgfqpoint{2.547881in}{1.571936in}}%
\pgfpathlineto{\pgfqpoint{2.554797in}{1.567328in}}%
\pgfpathlineto{\pgfqpoint{2.566678in}{1.566887in}}%
\pgfpathlineto{\pgfqpoint{2.582676in}{1.575047in}}%
\pgfpathlineto{\pgfqpoint{2.586580in}{1.532701in}}%
\pgfpathclose%
\pgfusepath{fill}%
\end{pgfscope}%
\begin{pgfscope}%
\pgfpathrectangle{\pgfqpoint{0.100000in}{0.100000in}}{\pgfqpoint{3.608454in}{2.310000in}}%
\pgfusepath{clip}%
\pgfsetbuttcap%
\pgfsetmiterjoin%
\definecolor{currentfill}{rgb}{0.000000,0.478431,0.760784}%
\pgfsetfillcolor{currentfill}%
\pgfsetlinewidth{0.000000pt}%
\definecolor{currentstroke}{rgb}{0.000000,0.000000,0.000000}%
\pgfsetstrokecolor{currentstroke}%
\pgfsetstrokeopacity{0.000000}%
\pgfsetdash{}{0pt}%
\pgfpathmoveto{\pgfqpoint{1.844714in}{0.901800in}}%
\pgfpathlineto{\pgfqpoint{1.880039in}{0.900259in}}%
\pgfpathlineto{\pgfqpoint{1.880200in}{0.905864in}}%
\pgfpathlineto{\pgfqpoint{1.909495in}{0.904858in}}%
\pgfpathlineto{\pgfqpoint{1.913431in}{0.902088in}}%
\pgfpathlineto{\pgfqpoint{1.912098in}{0.867745in}}%
\pgfpathlineto{\pgfqpoint{1.878486in}{0.869447in}}%
\pgfpathlineto{\pgfqpoint{1.878275in}{0.865031in}}%
\pgfpathlineto{\pgfqpoint{1.868391in}{0.865576in}}%
\pgfpathlineto{\pgfqpoint{1.808729in}{0.868432in}}%
\pgfpathlineto{\pgfqpoint{1.810377in}{0.903512in}}%
\pgfpathlineto{\pgfqpoint{1.844714in}{0.901800in}}%
\pgfpathclose%
\pgfusepath{fill}%
\end{pgfscope}%
\begin{pgfscope}%
\pgfpathrectangle{\pgfqpoint{0.100000in}{0.100000in}}{\pgfqpoint{3.608454in}{2.310000in}}%
\pgfusepath{clip}%
\pgfsetbuttcap%
\pgfsetmiterjoin%
\definecolor{currentfill}{rgb}{0.000000,0.572549,0.713725}%
\pgfsetfillcolor{currentfill}%
\pgfsetlinewidth{0.000000pt}%
\definecolor{currentstroke}{rgb}{0.000000,0.000000,0.000000}%
\pgfsetstrokecolor{currentstroke}%
\pgfsetstrokeopacity{0.000000}%
\pgfsetdash{}{0pt}%
\pgfpathmoveto{\pgfqpoint{2.307376in}{1.370072in}}%
\pgfpathlineto{\pgfqpoint{2.271164in}{1.369568in}}%
\pgfpathlineto{\pgfqpoint{2.271150in}{1.361570in}}%
\pgfpathlineto{\pgfqpoint{2.264061in}{1.361534in}}%
\pgfpathlineto{\pgfqpoint{2.246959in}{1.366834in}}%
\pgfpathlineto{\pgfqpoint{2.247305in}{1.396610in}}%
\pgfpathlineto{\pgfqpoint{2.237052in}{1.396510in}}%
\pgfpathlineto{\pgfqpoint{2.237055in}{1.423181in}}%
\pgfpathlineto{\pgfqpoint{2.267438in}{1.423982in}}%
\pgfpathlineto{\pgfqpoint{2.267659in}{1.417346in}}%
\pgfpathlineto{\pgfqpoint{2.291417in}{1.417786in}}%
\pgfpathlineto{\pgfqpoint{2.298225in}{1.417924in}}%
\pgfpathlineto{\pgfqpoint{2.298632in}{1.395025in}}%
\pgfpathlineto{\pgfqpoint{2.306650in}{1.395239in}}%
\pgfpathlineto{\pgfqpoint{2.307376in}{1.370072in}}%
\pgfpathclose%
\pgfusepath{fill}%
\end{pgfscope}%
\begin{pgfscope}%
\pgfpathrectangle{\pgfqpoint{0.100000in}{0.100000in}}{\pgfqpoint{3.608454in}{2.310000in}}%
\pgfusepath{clip}%
\pgfsetbuttcap%
\pgfsetmiterjoin%
\definecolor{currentfill}{rgb}{0.000000,0.666667,0.666667}%
\pgfsetfillcolor{currentfill}%
\pgfsetlinewidth{0.000000pt}%
\definecolor{currentstroke}{rgb}{0.000000,0.000000,0.000000}%
\pgfsetstrokecolor{currentstroke}%
\pgfsetstrokeopacity{0.000000}%
\pgfsetdash{}{0pt}%
\pgfpathmoveto{\pgfqpoint{2.043077in}{1.035565in}}%
\pgfpathlineto{\pgfqpoint{2.043347in}{1.046522in}}%
\pgfpathlineto{\pgfqpoint{2.043425in}{1.053406in}}%
\pgfpathlineto{\pgfqpoint{2.050250in}{1.053326in}}%
\pgfpathlineto{\pgfqpoint{2.053684in}{1.060206in}}%
\pgfpathlineto{\pgfqpoint{2.053817in}{1.067155in}}%
\pgfpathlineto{\pgfqpoint{2.060863in}{1.067101in}}%
\pgfpathlineto{\pgfqpoint{2.060962in}{1.080885in}}%
\pgfpathlineto{\pgfqpoint{2.057547in}{1.080899in}}%
\pgfpathlineto{\pgfqpoint{2.057637in}{1.091246in}}%
\pgfpathlineto{\pgfqpoint{2.065063in}{1.091216in}}%
\pgfpathlineto{\pgfqpoint{2.068981in}{1.083880in}}%
\pgfpathlineto{\pgfqpoint{2.076132in}{1.086986in}}%
\pgfpathlineto{\pgfqpoint{2.089737in}{1.087648in}}%
\pgfpathlineto{\pgfqpoint{2.088937in}{1.099591in}}%
\pgfpathlineto{\pgfqpoint{2.093490in}{1.108334in}}%
\pgfpathlineto{\pgfqpoint{2.099232in}{1.108302in}}%
\pgfpathlineto{\pgfqpoint{2.099224in}{1.115204in}}%
\pgfpathlineto{\pgfqpoint{2.106071in}{1.115212in}}%
\pgfpathlineto{\pgfqpoint{2.134726in}{1.115276in}}%
\pgfpathlineto{\pgfqpoint{2.139356in}{1.083433in}}%
\pgfpathlineto{\pgfqpoint{2.140727in}{1.073891in}}%
\pgfpathlineto{\pgfqpoint{2.098590in}{1.073824in}}%
\pgfpathlineto{\pgfqpoint{2.098258in}{1.064900in}}%
\pgfpathlineto{\pgfqpoint{2.101602in}{1.057429in}}%
\pgfpathlineto{\pgfqpoint{2.096410in}{1.052482in}}%
\pgfpathlineto{\pgfqpoint{2.091976in}{1.044393in}}%
\pgfpathlineto{\pgfqpoint{2.081865in}{1.049099in}}%
\pgfpathlineto{\pgfqpoint{2.069896in}{1.042778in}}%
\pgfpathlineto{\pgfqpoint{2.058740in}{1.038661in}}%
\pgfpathlineto{\pgfqpoint{2.050477in}{1.038266in}}%
\pgfpathlineto{\pgfqpoint{2.043077in}{1.035565in}}%
\pgfpathclose%
\pgfusepath{fill}%
\end{pgfscope}%
\begin{pgfscope}%
\pgfpathrectangle{\pgfqpoint{0.100000in}{0.100000in}}{\pgfqpoint{3.608454in}{2.310000in}}%
\pgfusepath{clip}%
\pgfsetbuttcap%
\pgfsetmiterjoin%
\definecolor{currentfill}{rgb}{0.000000,0.627451,0.686275}%
\pgfsetfillcolor{currentfill}%
\pgfsetlinewidth{0.000000pt}%
\definecolor{currentstroke}{rgb}{0.000000,0.000000,0.000000}%
\pgfsetstrokecolor{currentstroke}%
\pgfsetstrokeopacity{0.000000}%
\pgfsetdash{}{0pt}%
\pgfpathmoveto{\pgfqpoint{2.088125in}{1.418722in}}%
\pgfpathlineto{\pgfqpoint{2.081282in}{1.423410in}}%
\pgfpathlineto{\pgfqpoint{2.082351in}{1.428944in}}%
\pgfpathlineto{\pgfqpoint{2.077853in}{1.433443in}}%
\pgfpathlineto{\pgfqpoint{2.077207in}{1.438452in}}%
\pgfpathlineto{\pgfqpoint{2.069544in}{1.443619in}}%
\pgfpathlineto{\pgfqpoint{2.060821in}{1.465149in}}%
\pgfpathlineto{\pgfqpoint{2.128892in}{1.463893in}}%
\pgfpathlineto{\pgfqpoint{2.130971in}{1.449316in}}%
\pgfpathlineto{\pgfqpoint{2.130612in}{1.428716in}}%
\pgfpathlineto{\pgfqpoint{2.104194in}{1.428993in}}%
\pgfpathlineto{\pgfqpoint{2.107194in}{1.411974in}}%
\pgfpathlineto{\pgfqpoint{2.101635in}{1.407776in}}%
\pgfpathlineto{\pgfqpoint{2.094548in}{1.411120in}}%
\pgfpathlineto{\pgfqpoint{2.088125in}{1.418722in}}%
\pgfpathclose%
\pgfusepath{fill}%
\end{pgfscope}%
\begin{pgfscope}%
\pgfpathrectangle{\pgfqpoint{0.100000in}{0.100000in}}{\pgfqpoint{3.608454in}{2.310000in}}%
\pgfusepath{clip}%
\pgfsetbuttcap%
\pgfsetmiterjoin%
\definecolor{currentfill}{rgb}{0.000000,0.733333,0.633333}%
\pgfsetfillcolor{currentfill}%
\pgfsetlinewidth{0.000000pt}%
\definecolor{currentstroke}{rgb}{0.000000,0.000000,0.000000}%
\pgfsetstrokecolor{currentstroke}%
\pgfsetstrokeopacity{0.000000}%
\pgfsetdash{}{0pt}%
\pgfpathmoveto{\pgfqpoint{2.722462in}{1.100629in}}%
\pgfpathlineto{\pgfqpoint{2.714180in}{1.097018in}}%
\pgfpathlineto{\pgfqpoint{2.697680in}{1.093767in}}%
\pgfpathlineto{\pgfqpoint{2.693772in}{1.094220in}}%
\pgfpathlineto{\pgfqpoint{2.688788in}{1.097666in}}%
\pgfpathlineto{\pgfqpoint{2.685775in}{1.104241in}}%
\pgfpathlineto{\pgfqpoint{2.685202in}{1.110108in}}%
\pgfpathlineto{\pgfqpoint{2.689025in}{1.121466in}}%
\pgfpathlineto{\pgfqpoint{2.681516in}{1.127762in}}%
\pgfpathlineto{\pgfqpoint{2.678100in}{1.135237in}}%
\pgfpathlineto{\pgfqpoint{2.683825in}{1.139915in}}%
\pgfpathlineto{\pgfqpoint{2.690045in}{1.139163in}}%
\pgfpathlineto{\pgfqpoint{2.693376in}{1.142517in}}%
\pgfpathlineto{\pgfqpoint{2.695803in}{1.138527in}}%
\pgfpathlineto{\pgfqpoint{2.701451in}{1.138405in}}%
\pgfpathlineto{\pgfqpoint{2.703971in}{1.134522in}}%
\pgfpathlineto{\pgfqpoint{2.712831in}{1.140583in}}%
\pgfpathlineto{\pgfqpoint{2.722754in}{1.138754in}}%
\pgfpathlineto{\pgfqpoint{2.729200in}{1.134130in}}%
\pgfpathlineto{\pgfqpoint{2.732727in}{1.126386in}}%
\pgfpathlineto{\pgfqpoint{2.731314in}{1.113832in}}%
\pgfpathlineto{\pgfqpoint{2.722462in}{1.100629in}}%
\pgfpathclose%
\pgfusepath{fill}%
\end{pgfscope}%
\begin{pgfscope}%
\pgfpathrectangle{\pgfqpoint{0.100000in}{0.100000in}}{\pgfqpoint{3.608454in}{2.310000in}}%
\pgfusepath{clip}%
\pgfsetbuttcap%
\pgfsetmiterjoin%
\definecolor{currentfill}{rgb}{0.000000,0.627451,0.686275}%
\pgfsetfillcolor{currentfill}%
\pgfsetlinewidth{0.000000pt}%
\definecolor{currentstroke}{rgb}{0.000000,0.000000,0.000000}%
\pgfsetstrokecolor{currentstroke}%
\pgfsetstrokeopacity{0.000000}%
\pgfsetdash{}{0pt}%
\pgfpathmoveto{\pgfqpoint{1.411936in}{2.184794in}}%
\pgfpathlineto{\pgfqpoint{1.473904in}{2.176145in}}%
\pgfpathlineto{\pgfqpoint{1.522840in}{2.169802in}}%
\pgfpathlineto{\pgfqpoint{1.519036in}{2.156103in}}%
\pgfpathlineto{\pgfqpoint{1.525884in}{2.155265in}}%
\pgfpathlineto{\pgfqpoint{1.523357in}{2.134506in}}%
\pgfpathlineto{\pgfqpoint{1.534783in}{2.133134in}}%
\pgfpathlineto{\pgfqpoint{1.531517in}{2.105747in}}%
\pgfpathlineto{\pgfqpoint{1.529442in}{2.105984in}}%
\pgfpathlineto{\pgfqpoint{1.527435in}{2.089391in}}%
\pgfpathlineto{\pgfqpoint{1.507101in}{2.093171in}}%
\pgfpathlineto{\pgfqpoint{1.498726in}{2.097310in}}%
\pgfpathlineto{\pgfqpoint{1.496354in}{2.088951in}}%
\pgfpathlineto{\pgfqpoint{1.492326in}{2.087153in}}%
\pgfpathlineto{\pgfqpoint{1.487031in}{2.078600in}}%
\pgfpathlineto{\pgfqpoint{1.476331in}{2.073148in}}%
\pgfpathlineto{\pgfqpoint{1.470183in}{2.074395in}}%
\pgfpathlineto{\pgfqpoint{1.468292in}{2.069917in}}%
\pgfpathlineto{\pgfqpoint{1.449108in}{2.071253in}}%
\pgfpathlineto{\pgfqpoint{1.440516in}{2.074870in}}%
\pgfpathlineto{\pgfqpoint{1.437875in}{2.069615in}}%
\pgfpathlineto{\pgfqpoint{1.428142in}{2.073167in}}%
\pgfpathlineto{\pgfqpoint{1.422183in}{2.067656in}}%
\pgfpathlineto{\pgfqpoint{1.414702in}{2.064290in}}%
\pgfpathlineto{\pgfqpoint{1.411450in}{2.059486in}}%
\pgfpathlineto{\pgfqpoint{1.411723in}{2.068893in}}%
\pgfpathlineto{\pgfqpoint{1.404841in}{2.072375in}}%
\pgfpathlineto{\pgfqpoint{1.389242in}{2.072875in}}%
\pgfpathlineto{\pgfqpoint{1.385004in}{2.076421in}}%
\pgfpathlineto{\pgfqpoint{1.377859in}{2.077250in}}%
\pgfpathlineto{\pgfqpoint{1.367934in}{2.081540in}}%
\pgfpathlineto{\pgfqpoint{1.362402in}{2.089710in}}%
\pgfpathlineto{\pgfqpoint{1.364847in}{2.104580in}}%
\pgfpathlineto{\pgfqpoint{1.378739in}{2.102305in}}%
\pgfpathlineto{\pgfqpoint{1.380515in}{2.107723in}}%
\pgfpathlineto{\pgfqpoint{1.389566in}{2.105191in}}%
\pgfpathlineto{\pgfqpoint{1.392437in}{2.124428in}}%
\pgfpathlineto{\pgfqpoint{1.396285in}{2.142117in}}%
\pgfpathlineto{\pgfqpoint{1.399456in}{2.139818in}}%
\pgfpathlineto{\pgfqpoint{1.402592in}{2.151198in}}%
\pgfpathlineto{\pgfqpoint{1.404613in}{2.164996in}}%
\pgfpathlineto{\pgfqpoint{1.407741in}{2.164527in}}%
\pgfpathlineto{\pgfqpoint{1.411936in}{2.184794in}}%
\pgfpathclose%
\pgfusepath{fill}%
\end{pgfscope}%
\begin{pgfscope}%
\pgfpathrectangle{\pgfqpoint{0.100000in}{0.100000in}}{\pgfqpoint{3.608454in}{2.310000in}}%
\pgfusepath{clip}%
\pgfsetbuttcap%
\pgfsetmiterjoin%
\definecolor{currentfill}{rgb}{0.000000,0.572549,0.713725}%
\pgfsetfillcolor{currentfill}%
\pgfsetlinewidth{0.000000pt}%
\definecolor{currentstroke}{rgb}{0.000000,0.000000,0.000000}%
\pgfsetstrokecolor{currentstroke}%
\pgfsetstrokeopacity{0.000000}%
\pgfsetdash{}{0pt}%
\pgfpathmoveto{\pgfqpoint{3.150978in}{0.513727in}}%
\pgfpathlineto{\pgfqpoint{3.118731in}{0.508688in}}%
\pgfpathlineto{\pgfqpoint{3.117607in}{0.515401in}}%
\pgfpathlineto{\pgfqpoint{3.110913in}{0.514275in}}%
\pgfpathlineto{\pgfqpoint{3.102843in}{0.570453in}}%
\pgfpathlineto{\pgfqpoint{3.099105in}{0.576837in}}%
\pgfpathlineto{\pgfqpoint{3.099275in}{0.583081in}}%
\pgfpathlineto{\pgfqpoint{3.090875in}{0.590177in}}%
\pgfpathlineto{\pgfqpoint{3.090036in}{0.604447in}}%
\pgfpathlineto{\pgfqpoint{3.106445in}{0.607053in}}%
\pgfpathlineto{\pgfqpoint{3.119882in}{0.592598in}}%
\pgfpathlineto{\pgfqpoint{3.125061in}{0.582905in}}%
\pgfpathlineto{\pgfqpoint{3.120768in}{0.574633in}}%
\pgfpathlineto{\pgfqpoint{3.121728in}{0.567411in}}%
\pgfpathlineto{\pgfqpoint{3.127445in}{0.552410in}}%
\pgfpathlineto{\pgfqpoint{3.144148in}{0.527548in}}%
\pgfpathlineto{\pgfqpoint{3.150978in}{0.513727in}}%
\pgfpathclose%
\pgfusepath{fill}%
\end{pgfscope}%
\begin{pgfscope}%
\pgfpathrectangle{\pgfqpoint{0.100000in}{0.100000in}}{\pgfqpoint{3.608454in}{2.310000in}}%
\pgfusepath{clip}%
\pgfsetbuttcap%
\pgfsetmiterjoin%
\definecolor{currentfill}{rgb}{0.000000,0.674510,0.662745}%
\pgfsetfillcolor{currentfill}%
\pgfsetlinewidth{0.000000pt}%
\definecolor{currentstroke}{rgb}{0.000000,0.000000,0.000000}%
\pgfsetstrokecolor{currentstroke}%
\pgfsetstrokeopacity{0.000000}%
\pgfsetdash{}{0pt}%
\pgfpathmoveto{\pgfqpoint{2.218257in}{1.006289in}}%
\pgfpathlineto{\pgfqpoint{2.211366in}{1.003801in}}%
\pgfpathlineto{\pgfqpoint{2.190786in}{1.003624in}}%
\pgfpathlineto{\pgfqpoint{2.190828in}{1.025374in}}%
\pgfpathlineto{\pgfqpoint{2.166850in}{1.025625in}}%
\pgfpathlineto{\pgfqpoint{2.162454in}{1.027888in}}%
\pgfpathlineto{\pgfqpoint{2.162548in}{1.034836in}}%
\pgfpathlineto{\pgfqpoint{2.166153in}{1.040540in}}%
\pgfpathlineto{\pgfqpoint{2.166477in}{1.058801in}}%
\pgfpathlineto{\pgfqpoint{2.169375in}{1.063715in}}%
\pgfpathlineto{\pgfqpoint{2.166287in}{1.069101in}}%
\pgfpathlineto{\pgfqpoint{2.166395in}{1.075969in}}%
\pgfpathlineto{\pgfqpoint{2.172137in}{1.075918in}}%
\pgfpathlineto{\pgfqpoint{2.176762in}{1.083808in}}%
\pgfpathlineto{\pgfqpoint{2.201915in}{1.084462in}}%
\pgfpathlineto{\pgfqpoint{2.201928in}{1.082158in}}%
\pgfpathlineto{\pgfqpoint{2.247558in}{1.082128in}}%
\pgfpathlineto{\pgfqpoint{2.247654in}{1.068396in}}%
\pgfpathlineto{\pgfqpoint{2.245419in}{1.061527in}}%
\pgfpathlineto{\pgfqpoint{2.245679in}{1.038542in}}%
\pgfpathlineto{\pgfqpoint{2.241035in}{1.037964in}}%
\pgfpathlineto{\pgfqpoint{2.226237in}{1.022520in}}%
\pgfpathlineto{\pgfqpoint{2.216940in}{1.014382in}}%
\pgfpathlineto{\pgfqpoint{2.218257in}{1.006289in}}%
\pgfpathclose%
\pgfusepath{fill}%
\end{pgfscope}%
\begin{pgfscope}%
\pgfpathrectangle{\pgfqpoint{0.100000in}{0.100000in}}{\pgfqpoint{3.608454in}{2.310000in}}%
\pgfusepath{clip}%
\pgfsetbuttcap%
\pgfsetmiterjoin%
\definecolor{currentfill}{rgb}{0.000000,0.411765,0.794118}%
\pgfsetfillcolor{currentfill}%
\pgfsetlinewidth{0.000000pt}%
\definecolor{currentstroke}{rgb}{0.000000,0.000000,0.000000}%
\pgfsetstrokecolor{currentstroke}%
\pgfsetstrokeopacity{0.000000}%
\pgfsetdash{}{0pt}%
\pgfpathmoveto{\pgfqpoint{1.075355in}{0.913752in}}%
\pgfpathlineto{\pgfqpoint{1.001421in}{0.926345in}}%
\pgfpathlineto{\pgfqpoint{0.934927in}{0.938858in}}%
\pgfpathlineto{\pgfqpoint{0.885192in}{0.948795in}}%
\pgfpathlineto{\pgfqpoint{0.899087in}{1.016739in}}%
\pgfpathlineto{\pgfqpoint{0.908989in}{1.065097in}}%
\pgfpathlineto{\pgfqpoint{0.947034in}{1.057499in}}%
\pgfpathlineto{\pgfqpoint{0.975479in}{1.042443in}}%
\pgfpathlineto{\pgfqpoint{0.985165in}{1.053983in}}%
\pgfpathlineto{\pgfqpoint{1.012835in}{1.044864in}}%
\pgfpathlineto{\pgfqpoint{1.027746in}{1.042105in}}%
\pgfpathlineto{\pgfqpoint{1.031005in}{1.053711in}}%
\pgfpathlineto{\pgfqpoint{1.015879in}{1.056517in}}%
\pgfpathlineto{\pgfqpoint{1.022683in}{1.074223in}}%
\pgfpathlineto{\pgfqpoint{1.031414in}{1.079449in}}%
\pgfpathlineto{\pgfqpoint{1.039948in}{1.073968in}}%
\pgfpathlineto{\pgfqpoint{1.049066in}{1.073939in}}%
\pgfpathlineto{\pgfqpoint{1.055533in}{1.068101in}}%
\pgfpathlineto{\pgfqpoint{1.063896in}{1.064728in}}%
\pgfpathlineto{\pgfqpoint{1.073305in}{1.055337in}}%
\pgfpathlineto{\pgfqpoint{1.079528in}{1.054057in}}%
\pgfpathlineto{\pgfqpoint{1.075926in}{1.033485in}}%
\pgfpathlineto{\pgfqpoint{1.124557in}{1.025178in}}%
\pgfpathlineto{\pgfqpoint{1.119039in}{0.992055in}}%
\pgfpathlineto{\pgfqpoint{1.117590in}{0.983356in}}%
\pgfpathlineto{\pgfqpoint{1.106555in}{0.985188in}}%
\pgfpathlineto{\pgfqpoint{1.096233in}{0.984306in}}%
\pgfpathlineto{\pgfqpoint{1.087666in}{0.981611in}}%
\pgfpathlineto{\pgfqpoint{1.084657in}{0.967062in}}%
\pgfpathlineto{\pgfqpoint{1.075355in}{0.913752in}}%
\pgfpathclose%
\pgfusepath{fill}%
\end{pgfscope}%
\begin{pgfscope}%
\pgfpathrectangle{\pgfqpoint{0.100000in}{0.100000in}}{\pgfqpoint{3.608454in}{2.310000in}}%
\pgfusepath{clip}%
\pgfsetbuttcap%
\pgfsetmiterjoin%
\definecolor{currentfill}{rgb}{0.000000,0.447059,0.776471}%
\pgfsetfillcolor{currentfill}%
\pgfsetlinewidth{0.000000pt}%
\definecolor{currentstroke}{rgb}{0.000000,0.000000,0.000000}%
\pgfsetstrokecolor{currentstroke}%
\pgfsetstrokeopacity{0.000000}%
\pgfsetdash{}{0pt}%
\pgfpathmoveto{\pgfqpoint{2.032756in}{2.134709in}}%
\pgfpathlineto{\pgfqpoint{2.098692in}{2.134056in}}%
\pgfpathlineto{\pgfqpoint{2.098766in}{2.165029in}}%
\pgfpathlineto{\pgfqpoint{2.103732in}{2.162515in}}%
\pgfpathlineto{\pgfqpoint{2.109017in}{2.163883in}}%
\pgfpathlineto{\pgfqpoint{2.115955in}{2.157789in}}%
\pgfpathlineto{\pgfqpoint{2.123516in}{2.124861in}}%
\pgfpathlineto{\pgfqpoint{2.123134in}{2.116351in}}%
\pgfpathlineto{\pgfqpoint{2.128642in}{2.111522in}}%
\pgfpathlineto{\pgfqpoint{2.136953in}{2.110228in}}%
\pgfpathlineto{\pgfqpoint{2.137206in}{2.083483in}}%
\pgfpathlineto{\pgfqpoint{2.095560in}{2.083291in}}%
\pgfpathlineto{\pgfqpoint{2.095537in}{2.097237in}}%
\pgfpathlineto{\pgfqpoint{2.074816in}{2.097239in}}%
\pgfpathlineto{\pgfqpoint{2.033124in}{2.098129in}}%
\pgfpathlineto{\pgfqpoint{2.032428in}{2.112015in}}%
\pgfpathlineto{\pgfqpoint{2.032756in}{2.134709in}}%
\pgfpathclose%
\pgfusepath{fill}%
\end{pgfscope}%
\begin{pgfscope}%
\pgfpathrectangle{\pgfqpoint{0.100000in}{0.100000in}}{\pgfqpoint{3.608454in}{2.310000in}}%
\pgfusepath{clip}%
\pgfsetbuttcap%
\pgfsetmiterjoin%
\definecolor{currentfill}{rgb}{0.000000,0.576471,0.711765}%
\pgfsetfillcolor{currentfill}%
\pgfsetlinewidth{0.000000pt}%
\definecolor{currentstroke}{rgb}{0.000000,0.000000,0.000000}%
\pgfsetstrokecolor{currentstroke}%
\pgfsetstrokeopacity{0.000000}%
\pgfsetdash{}{0pt}%
\pgfpathmoveto{\pgfqpoint{1.001953in}{2.138275in}}%
\pgfpathlineto{\pgfqpoint{0.996502in}{2.145181in}}%
\pgfpathlineto{\pgfqpoint{1.000929in}{2.147949in}}%
\pgfpathlineto{\pgfqpoint{1.000904in}{2.156205in}}%
\pgfpathlineto{\pgfqpoint{0.996594in}{2.161889in}}%
\pgfpathlineto{\pgfqpoint{0.988985in}{2.182170in}}%
\pgfpathlineto{\pgfqpoint{0.993310in}{2.200722in}}%
\pgfpathlineto{\pgfqpoint{0.994462in}{2.195826in}}%
\pgfpathlineto{\pgfqpoint{1.011635in}{2.199142in}}%
\pgfpathlineto{\pgfqpoint{1.012094in}{2.185135in}}%
\pgfpathlineto{\pgfqpoint{1.015735in}{2.176278in}}%
\pgfpathlineto{\pgfqpoint{1.015079in}{2.170487in}}%
\pgfpathlineto{\pgfqpoint{1.027896in}{2.166208in}}%
\pgfpathlineto{\pgfqpoint{1.033549in}{2.168208in}}%
\pgfpathlineto{\pgfqpoint{1.036396in}{2.174561in}}%
\pgfpathlineto{\pgfqpoint{1.043899in}{2.172864in}}%
\pgfpathlineto{\pgfqpoint{1.047061in}{2.189256in}}%
\pgfpathlineto{\pgfqpoint{1.056030in}{2.187273in}}%
\pgfpathlineto{\pgfqpoint{1.061972in}{2.214474in}}%
\pgfpathlineto{\pgfqpoint{1.061432in}{2.221573in}}%
\pgfpathlineto{\pgfqpoint{1.072055in}{2.221371in}}%
\pgfpathlineto{\pgfqpoint{1.075971in}{2.228263in}}%
\pgfpathlineto{\pgfqpoint{1.074794in}{2.233716in}}%
\pgfpathlineto{\pgfqpoint{1.075629in}{2.246526in}}%
\pgfpathlineto{\pgfqpoint{1.109601in}{2.239048in}}%
\pgfpathlineto{\pgfqpoint{1.109470in}{2.229400in}}%
\pgfpathlineto{\pgfqpoint{1.112359in}{2.223590in}}%
\pgfpathlineto{\pgfqpoint{1.118945in}{2.225334in}}%
\pgfpathlineto{\pgfqpoint{1.123862in}{2.213141in}}%
\pgfpathlineto{\pgfqpoint{1.119541in}{2.205769in}}%
\pgfpathlineto{\pgfqpoint{1.133270in}{2.197089in}}%
\pgfpathlineto{\pgfqpoint{1.131142in}{2.190334in}}%
\pgfpathlineto{\pgfqpoint{1.136982in}{2.181768in}}%
\pgfpathlineto{\pgfqpoint{1.134122in}{2.180102in}}%
\pgfpathlineto{\pgfqpoint{1.140465in}{2.171053in}}%
\pgfpathlineto{\pgfqpoint{1.139517in}{2.164977in}}%
\pgfpathlineto{\pgfqpoint{1.150149in}{2.159708in}}%
\pgfpathlineto{\pgfqpoint{1.154266in}{2.150464in}}%
\pgfpathlineto{\pgfqpoint{1.149026in}{2.145360in}}%
\pgfpathlineto{\pgfqpoint{1.143835in}{2.140807in}}%
\pgfpathlineto{\pgfqpoint{1.140741in}{2.130368in}}%
\pgfpathlineto{\pgfqpoint{1.135799in}{2.128987in}}%
\pgfpathlineto{\pgfqpoint{1.134932in}{2.119015in}}%
\pgfpathlineto{\pgfqpoint{1.118004in}{2.122768in}}%
\pgfpathlineto{\pgfqpoint{1.093543in}{2.127613in}}%
\pgfpathlineto{\pgfqpoint{1.090372in}{2.118413in}}%
\pgfpathlineto{\pgfqpoint{1.093234in}{2.110956in}}%
\pgfpathlineto{\pgfqpoint{1.091464in}{2.102885in}}%
\pgfpathlineto{\pgfqpoint{1.092364in}{2.093954in}}%
\pgfpathlineto{\pgfqpoint{1.085713in}{2.091817in}}%
\pgfpathlineto{\pgfqpoint{1.072235in}{2.094680in}}%
\pgfpathlineto{\pgfqpoint{1.068378in}{2.094075in}}%
\pgfpathlineto{\pgfqpoint{1.060904in}{2.103004in}}%
\pgfpathlineto{\pgfqpoint{1.048078in}{2.111068in}}%
\pgfpathlineto{\pgfqpoint{1.039741in}{2.112438in}}%
\pgfpathlineto{\pgfqpoint{1.033798in}{2.117353in}}%
\pgfpathlineto{\pgfqpoint{1.034696in}{2.123940in}}%
\pgfpathlineto{\pgfqpoint{1.020317in}{2.134665in}}%
\pgfpathlineto{\pgfqpoint{1.011451in}{2.134988in}}%
\pgfpathlineto{\pgfqpoint{1.001953in}{2.138275in}}%
\pgfpathclose%
\pgfusepath{fill}%
\end{pgfscope}%
\begin{pgfscope}%
\pgfpathrectangle{\pgfqpoint{0.100000in}{0.100000in}}{\pgfqpoint{3.608454in}{2.310000in}}%
\pgfusepath{clip}%
\pgfsetbuttcap%
\pgfsetmiterjoin%
\definecolor{currentfill}{rgb}{0.000000,0.666667,0.666667}%
\pgfsetfillcolor{currentfill}%
\pgfsetlinewidth{0.000000pt}%
\definecolor{currentstroke}{rgb}{0.000000,0.000000,0.000000}%
\pgfsetstrokecolor{currentstroke}%
\pgfsetstrokeopacity{0.000000}%
\pgfsetdash{}{0pt}%
\pgfpathmoveto{\pgfqpoint{1.583492in}{0.651351in}}%
\pgfpathlineto{\pgfqpoint{1.599353in}{0.636098in}}%
\pgfpathlineto{\pgfqpoint{1.593544in}{0.627739in}}%
\pgfpathlineto{\pgfqpoint{1.573827in}{0.627230in}}%
\pgfpathlineto{\pgfqpoint{1.567953in}{0.615750in}}%
\pgfpathlineto{\pgfqpoint{1.563163in}{0.610468in}}%
\pgfpathlineto{\pgfqpoint{1.559813in}{0.597646in}}%
\pgfpathlineto{\pgfqpoint{1.546004in}{0.580048in}}%
\pgfpathlineto{\pgfqpoint{1.539619in}{0.575381in}}%
\pgfpathlineto{\pgfqpoint{1.538032in}{0.569527in}}%
\pgfpathlineto{\pgfqpoint{1.529354in}{0.569946in}}%
\pgfpathlineto{\pgfqpoint{1.514568in}{0.578283in}}%
\pgfpathlineto{\pgfqpoint{1.508590in}{0.586063in}}%
\pgfpathlineto{\pgfqpoint{1.497563in}{0.588987in}}%
\pgfpathlineto{\pgfqpoint{1.493508in}{0.596165in}}%
\pgfpathlineto{\pgfqpoint{1.480379in}{0.599940in}}%
\pgfpathlineto{\pgfqpoint{1.469478in}{0.607802in}}%
\pgfpathlineto{\pgfqpoint{1.463905in}{0.617758in}}%
\pgfpathlineto{\pgfqpoint{1.457236in}{0.620260in}}%
\pgfpathlineto{\pgfqpoint{1.445703in}{0.631256in}}%
\pgfpathlineto{\pgfqpoint{1.443059in}{0.642585in}}%
\pgfpathlineto{\pgfqpoint{1.437063in}{0.653883in}}%
\pgfpathlineto{\pgfqpoint{1.436330in}{0.666295in}}%
\pgfpathlineto{\pgfqpoint{1.437587in}{0.679889in}}%
\pgfpathlineto{\pgfqpoint{1.428449in}{0.693343in}}%
\pgfpathlineto{\pgfqpoint{1.428817in}{0.701695in}}%
\pgfpathlineto{\pgfqpoint{1.425660in}{0.710940in}}%
\pgfpathlineto{\pgfqpoint{1.422243in}{0.713258in}}%
\pgfpathlineto{\pgfqpoint{1.426837in}{0.715528in}}%
\pgfpathlineto{\pgfqpoint{1.486032in}{0.744751in}}%
\pgfpathlineto{\pgfqpoint{1.518651in}{0.714404in}}%
\pgfpathlineto{\pgfqpoint{1.583492in}{0.651351in}}%
\pgfpathclose%
\pgfusepath{fill}%
\end{pgfscope}%
\begin{pgfscope}%
\pgfpathrectangle{\pgfqpoint{0.100000in}{0.100000in}}{\pgfqpoint{3.608454in}{2.310000in}}%
\pgfusepath{clip}%
\pgfsetbuttcap%
\pgfsetmiterjoin%
\definecolor{currentfill}{rgb}{0.000000,0.639216,0.680392}%
\pgfsetfillcolor{currentfill}%
\pgfsetlinewidth{0.000000pt}%
\definecolor{currentstroke}{rgb}{0.000000,0.000000,0.000000}%
\pgfsetstrokecolor{currentstroke}%
\pgfsetstrokeopacity{0.000000}%
\pgfsetdash{}{0pt}%
\pgfpathmoveto{\pgfqpoint{2.715165in}{0.972447in}}%
\pgfpathlineto{\pgfqpoint{2.705960in}{0.970065in}}%
\pgfpathlineto{\pgfqpoint{2.694783in}{0.960330in}}%
\pgfpathlineto{\pgfqpoint{2.689264in}{0.964347in}}%
\pgfpathlineto{\pgfqpoint{2.684888in}{0.971060in}}%
\pgfpathlineto{\pgfqpoint{2.678957in}{0.970072in}}%
\pgfpathlineto{\pgfqpoint{2.675715in}{0.974764in}}%
\pgfpathlineto{\pgfqpoint{2.677206in}{0.979100in}}%
\pgfpathlineto{\pgfqpoint{2.664313in}{0.994103in}}%
\pgfpathlineto{\pgfqpoint{2.657488in}{0.993619in}}%
\pgfpathlineto{\pgfqpoint{2.659190in}{1.009555in}}%
\pgfpathlineto{\pgfqpoint{2.666467in}{1.008380in}}%
\pgfpathlineto{\pgfqpoint{2.673359in}{1.013797in}}%
\pgfpathlineto{\pgfqpoint{2.669558in}{1.020872in}}%
\pgfpathlineto{\pgfqpoint{2.670176in}{1.036633in}}%
\pgfpathlineto{\pgfqpoint{2.668245in}{1.039940in}}%
\pgfpathlineto{\pgfqpoint{2.669947in}{1.049451in}}%
\pgfpathlineto{\pgfqpoint{2.698781in}{1.052062in}}%
\pgfpathlineto{\pgfqpoint{2.715608in}{1.053498in}}%
\pgfpathlineto{\pgfqpoint{2.723904in}{1.022807in}}%
\pgfpathlineto{\pgfqpoint{2.725316in}{1.017812in}}%
\pgfpathlineto{\pgfqpoint{2.721545in}{1.014250in}}%
\pgfpathlineto{\pgfqpoint{2.718615in}{1.004553in}}%
\pgfpathlineto{\pgfqpoint{2.710524in}{0.995150in}}%
\pgfpathlineto{\pgfqpoint{2.706101in}{0.993510in}}%
\pgfpathlineto{\pgfqpoint{2.715165in}{0.972447in}}%
\pgfpathclose%
\pgfusepath{fill}%
\end{pgfscope}%
\begin{pgfscope}%
\pgfpathrectangle{\pgfqpoint{0.100000in}{0.100000in}}{\pgfqpoint{3.608454in}{2.310000in}}%
\pgfusepath{clip}%
\pgfsetbuttcap%
\pgfsetmiterjoin%
\definecolor{currentfill}{rgb}{0.000000,0.945098,0.527451}%
\pgfsetfillcolor{currentfill}%
\pgfsetlinewidth{0.000000pt}%
\definecolor{currentstroke}{rgb}{0.000000,0.000000,0.000000}%
\pgfsetstrokecolor{currentstroke}%
\pgfsetstrokeopacity{0.000000}%
\pgfsetdash{}{0pt}%
\pgfpathmoveto{\pgfqpoint{0.605938in}{1.604509in}}%
\pgfpathlineto{\pgfqpoint{0.600479in}{1.609616in}}%
\pgfpathlineto{\pgfqpoint{0.604334in}{1.622797in}}%
\pgfpathlineto{\pgfqpoint{0.600541in}{1.629953in}}%
\pgfpathlineto{\pgfqpoint{0.601624in}{1.635606in}}%
\pgfpathlineto{\pgfqpoint{0.594599in}{1.640016in}}%
\pgfpathlineto{\pgfqpoint{0.584658in}{1.656876in}}%
\pgfpathlineto{\pgfqpoint{0.574470in}{1.662350in}}%
\pgfpathlineto{\pgfqpoint{0.566606in}{1.658522in}}%
\pgfpathlineto{\pgfqpoint{0.558505in}{1.659196in}}%
\pgfpathlineto{\pgfqpoint{0.555953in}{1.662869in}}%
\pgfpathlineto{\pgfqpoint{0.560209in}{1.677284in}}%
\pgfpathlineto{\pgfqpoint{0.544816in}{1.681752in}}%
\pgfpathlineto{\pgfqpoint{0.555668in}{1.716515in}}%
\pgfpathlineto{\pgfqpoint{0.561272in}{1.737785in}}%
\pgfpathlineto{\pgfqpoint{0.494828in}{1.757961in}}%
\pgfpathlineto{\pgfqpoint{0.498294in}{1.770162in}}%
\pgfpathlineto{\pgfqpoint{0.494062in}{1.772814in}}%
\pgfpathlineto{\pgfqpoint{0.477526in}{1.765034in}}%
\pgfpathlineto{\pgfqpoint{0.469737in}{1.765482in}}%
\pgfpathlineto{\pgfqpoint{0.466061in}{1.760966in}}%
\pgfpathlineto{\pgfqpoint{0.468505in}{1.753515in}}%
\pgfpathlineto{\pgfqpoint{0.459959in}{1.754157in}}%
\pgfpathlineto{\pgfqpoint{0.457583in}{1.760831in}}%
\pgfpathlineto{\pgfqpoint{0.449915in}{1.763500in}}%
\pgfpathlineto{\pgfqpoint{0.449403in}{1.767395in}}%
\pgfpathlineto{\pgfqpoint{0.443568in}{1.772460in}}%
\pgfpathlineto{\pgfqpoint{0.442827in}{1.791103in}}%
\pgfpathlineto{\pgfqpoint{0.433774in}{1.794008in}}%
\pgfpathlineto{\pgfqpoint{0.437233in}{1.797086in}}%
\pgfpathlineto{\pgfqpoint{0.438276in}{1.805265in}}%
\pgfpathlineto{\pgfqpoint{0.436698in}{1.812439in}}%
\pgfpathlineto{\pgfqpoint{0.441986in}{1.818955in}}%
\pgfpathlineto{\pgfqpoint{0.442382in}{1.828434in}}%
\pgfpathlineto{\pgfqpoint{0.452336in}{1.831556in}}%
\pgfpathlineto{\pgfqpoint{0.456983in}{1.837942in}}%
\pgfpathlineto{\pgfqpoint{0.526008in}{1.816628in}}%
\pgfpathlineto{\pgfqpoint{0.549594in}{1.891292in}}%
\pgfpathlineto{\pgfqpoint{0.551234in}{1.896670in}}%
\pgfpathlineto{\pgfqpoint{0.562002in}{1.893971in}}%
\pgfpathlineto{\pgfqpoint{0.565189in}{1.899442in}}%
\pgfpathlineto{\pgfqpoint{0.572519in}{1.906519in}}%
\pgfpathlineto{\pgfqpoint{0.572489in}{1.914545in}}%
\pgfpathlineto{\pgfqpoint{0.568235in}{1.922352in}}%
\pgfpathlineto{\pgfqpoint{0.571017in}{1.931177in}}%
\pgfpathlineto{\pgfqpoint{0.579411in}{1.933417in}}%
\pgfpathlineto{\pgfqpoint{0.616244in}{1.922404in}}%
\pgfpathlineto{\pgfqpoint{0.609494in}{1.902912in}}%
\pgfpathlineto{\pgfqpoint{0.595602in}{1.856614in}}%
\pgfpathlineto{\pgfqpoint{0.621562in}{1.848713in}}%
\pgfpathlineto{\pgfqpoint{0.605093in}{1.791654in}}%
\pgfpathlineto{\pgfqpoint{0.654872in}{1.777406in}}%
\pgfpathlineto{\pgfqpoint{0.637413in}{1.715698in}}%
\pgfpathlineto{\pgfqpoint{0.622947in}{1.663573in}}%
\pgfpathlineto{\pgfqpoint{0.605938in}{1.604509in}}%
\pgfpathclose%
\pgfusepath{fill}%
\end{pgfscope}%
\begin{pgfscope}%
\pgfpathrectangle{\pgfqpoint{0.100000in}{0.100000in}}{\pgfqpoint{3.608454in}{2.310000in}}%
\pgfusepath{clip}%
\pgfsetbuttcap%
\pgfsetmiterjoin%
\definecolor{currentfill}{rgb}{0.000000,0.580392,0.709804}%
\pgfsetfillcolor{currentfill}%
\pgfsetlinewidth{0.000000pt}%
\definecolor{currentstroke}{rgb}{0.000000,0.000000,0.000000}%
\pgfsetstrokecolor{currentstroke}%
\pgfsetstrokeopacity{0.000000}%
\pgfsetdash{}{0pt}%
\pgfpathmoveto{\pgfqpoint{2.636183in}{1.558287in}}%
\pgfpathlineto{\pgfqpoint{2.638514in}{1.537637in}}%
\pgfpathlineto{\pgfqpoint{2.614051in}{1.535158in}}%
\pgfpathlineto{\pgfqpoint{2.586580in}{1.532701in}}%
\pgfpathlineto{\pgfqpoint{2.582676in}{1.575047in}}%
\pgfpathlineto{\pgfqpoint{2.596356in}{1.586648in}}%
\pgfpathlineto{\pgfqpoint{2.601592in}{1.595258in}}%
\pgfpathlineto{\pgfqpoint{2.605772in}{1.609448in}}%
\pgfpathlineto{\pgfqpoint{2.612010in}{1.620134in}}%
\pgfpathlineto{\pgfqpoint{2.620303in}{1.620947in}}%
\pgfpathlineto{\pgfqpoint{2.621651in}{1.607409in}}%
\pgfpathlineto{\pgfqpoint{2.648568in}{1.609914in}}%
\pgfpathlineto{\pgfqpoint{2.650897in}{1.588647in}}%
\pgfpathlineto{\pgfqpoint{2.649386in}{1.585311in}}%
\pgfpathlineto{\pgfqpoint{2.633441in}{1.583836in}}%
\pgfpathlineto{\pgfqpoint{2.636183in}{1.558287in}}%
\pgfpathclose%
\pgfusepath{fill}%
\end{pgfscope}%
\begin{pgfscope}%
\pgfpathrectangle{\pgfqpoint{0.100000in}{0.100000in}}{\pgfqpoint{3.608454in}{2.310000in}}%
\pgfusepath{clip}%
\pgfsetbuttcap%
\pgfsetmiterjoin%
\definecolor{currentfill}{rgb}{0.000000,0.525490,0.737255}%
\pgfsetfillcolor{currentfill}%
\pgfsetlinewidth{0.000000pt}%
\definecolor{currentstroke}{rgb}{0.000000,0.000000,0.000000}%
\pgfsetstrokecolor{currentstroke}%
\pgfsetstrokeopacity{0.000000}%
\pgfsetdash{}{0pt}%
\pgfpathmoveto{\pgfqpoint{2.395693in}{0.767722in}}%
\pgfpathlineto{\pgfqpoint{2.396163in}{0.790004in}}%
\pgfpathlineto{\pgfqpoint{2.394278in}{0.809641in}}%
\pgfpathlineto{\pgfqpoint{2.395425in}{0.819431in}}%
\pgfpathlineto{\pgfqpoint{2.399936in}{0.821863in}}%
\pgfpathlineto{\pgfqpoint{2.401669in}{0.830402in}}%
\pgfpathlineto{\pgfqpoint{2.404935in}{0.832654in}}%
\pgfpathlineto{\pgfqpoint{2.412628in}{0.839279in}}%
\pgfpathlineto{\pgfqpoint{2.428148in}{0.846306in}}%
\pgfpathlineto{\pgfqpoint{2.438751in}{0.854506in}}%
\pgfpathlineto{\pgfqpoint{2.442684in}{0.864247in}}%
\pgfpathlineto{\pgfqpoint{2.458381in}{0.865552in}}%
\pgfpathlineto{\pgfqpoint{2.458164in}{0.869135in}}%
\pgfpathlineto{\pgfqpoint{2.485523in}{0.870923in}}%
\pgfpathlineto{\pgfqpoint{2.488805in}{0.814690in}}%
\pgfpathlineto{\pgfqpoint{2.461411in}{0.812905in}}%
\pgfpathlineto{\pgfqpoint{2.462239in}{0.799019in}}%
\pgfpathlineto{\pgfqpoint{2.467492in}{0.796662in}}%
\pgfpathlineto{\pgfqpoint{2.468672in}{0.778139in}}%
\pgfpathlineto{\pgfqpoint{2.447021in}{0.775387in}}%
\pgfpathlineto{\pgfqpoint{2.436793in}{0.774096in}}%
\pgfpathlineto{\pgfqpoint{2.426622in}{0.769362in}}%
\pgfpathlineto{\pgfqpoint{2.395693in}{0.767722in}}%
\pgfpathclose%
\pgfusepath{fill}%
\end{pgfscope}%
\begin{pgfscope}%
\pgfpathrectangle{\pgfqpoint{0.100000in}{0.100000in}}{\pgfqpoint{3.608454in}{2.310000in}}%
\pgfusepath{clip}%
\pgfsetbuttcap%
\pgfsetmiterjoin%
\definecolor{currentfill}{rgb}{0.000000,0.666667,0.666667}%
\pgfsetfillcolor{currentfill}%
\pgfsetlinewidth{0.000000pt}%
\definecolor{currentstroke}{rgb}{0.000000,0.000000,0.000000}%
\pgfsetstrokecolor{currentstroke}%
\pgfsetstrokeopacity{0.000000}%
\pgfsetdash{}{0pt}%
\pgfpathmoveto{\pgfqpoint{2.244140in}{0.763307in}}%
\pgfpathlineto{\pgfqpoint{2.248851in}{0.756615in}}%
\pgfpathlineto{\pgfqpoint{2.256019in}{0.754662in}}%
\pgfpathlineto{\pgfqpoint{2.261771in}{0.748472in}}%
\pgfpathlineto{\pgfqpoint{2.248864in}{0.736251in}}%
\pgfpathlineto{\pgfqpoint{2.244244in}{0.734424in}}%
\pgfpathlineto{\pgfqpoint{2.240752in}{0.735755in}}%
\pgfpathlineto{\pgfqpoint{2.226879in}{0.735487in}}%
\pgfpathlineto{\pgfqpoint{2.226572in}{0.749343in}}%
\pgfpathlineto{\pgfqpoint{2.219726in}{0.749195in}}%
\pgfpathlineto{\pgfqpoint{2.219367in}{0.763229in}}%
\pgfpathlineto{\pgfqpoint{2.212402in}{0.763123in}}%
\pgfpathlineto{\pgfqpoint{2.212262in}{0.773539in}}%
\pgfpathlineto{\pgfqpoint{2.182637in}{0.773027in}}%
\pgfpathlineto{\pgfqpoint{2.173280in}{0.783716in}}%
\pgfpathlineto{\pgfqpoint{2.171444in}{0.784825in}}%
\pgfpathlineto{\pgfqpoint{2.170684in}{0.855388in}}%
\pgfpathlineto{\pgfqpoint{2.170562in}{0.866376in}}%
\pgfpathlineto{\pgfqpoint{2.186423in}{0.866575in}}%
\pgfpathlineto{\pgfqpoint{2.240807in}{0.867381in}}%
\pgfpathlineto{\pgfqpoint{2.258381in}{0.867551in}}%
\pgfpathlineto{\pgfqpoint{2.258858in}{0.847301in}}%
\pgfpathlineto{\pgfqpoint{2.251967in}{0.847074in}}%
\pgfpathlineto{\pgfqpoint{2.248708in}{0.834367in}}%
\pgfpathlineto{\pgfqpoint{2.249143in}{0.822767in}}%
\pgfpathlineto{\pgfqpoint{2.255994in}{0.822878in}}%
\pgfpathlineto{\pgfqpoint{2.256664in}{0.805503in}}%
\pgfpathlineto{\pgfqpoint{2.254067in}{0.798428in}}%
\pgfpathlineto{\pgfqpoint{2.245714in}{0.798315in}}%
\pgfpathlineto{\pgfqpoint{2.249736in}{0.781810in}}%
\pgfpathlineto{\pgfqpoint{2.246031in}{0.774913in}}%
\pgfpathlineto{\pgfqpoint{2.244140in}{0.763307in}}%
\pgfpathclose%
\pgfusepath{fill}%
\end{pgfscope}%
\begin{pgfscope}%
\pgfpathrectangle{\pgfqpoint{0.100000in}{0.100000in}}{\pgfqpoint{3.608454in}{2.310000in}}%
\pgfusepath{clip}%
\pgfsetbuttcap%
\pgfsetmiterjoin%
\definecolor{currentfill}{rgb}{0.000000,0.541176,0.729412}%
\pgfsetfillcolor{currentfill}%
\pgfsetlinewidth{0.000000pt}%
\definecolor{currentstroke}{rgb}{0.000000,0.000000,0.000000}%
\pgfsetstrokecolor{currentstroke}%
\pgfsetstrokeopacity{0.000000}%
\pgfsetdash{}{0pt}%
\pgfpathmoveto{\pgfqpoint{2.859932in}{1.127064in}}%
\pgfpathlineto{\pgfqpoint{2.845692in}{1.114629in}}%
\pgfpathlineto{\pgfqpoint{2.835111in}{1.113784in}}%
\pgfpathlineto{\pgfqpoint{2.833923in}{1.123330in}}%
\pgfpathlineto{\pgfqpoint{2.829784in}{1.127369in}}%
\pgfpathlineto{\pgfqpoint{2.823619in}{1.137722in}}%
\pgfpathlineto{\pgfqpoint{2.831674in}{1.145493in}}%
\pgfpathlineto{\pgfqpoint{2.824792in}{1.161761in}}%
\pgfpathlineto{\pgfqpoint{2.827195in}{1.174638in}}%
\pgfpathlineto{\pgfqpoint{2.836596in}{1.177507in}}%
\pgfpathlineto{\pgfqpoint{2.846368in}{1.183090in}}%
\pgfpathlineto{\pgfqpoint{2.851139in}{1.181704in}}%
\pgfpathlineto{\pgfqpoint{2.853813in}{1.173625in}}%
\pgfpathlineto{\pgfqpoint{2.859396in}{1.177849in}}%
\pgfpathlineto{\pgfqpoint{2.865237in}{1.171385in}}%
\pgfpathlineto{\pgfqpoint{2.860386in}{1.164003in}}%
\pgfpathlineto{\pgfqpoint{2.868331in}{1.159628in}}%
\pgfpathlineto{\pgfqpoint{2.879988in}{1.149617in}}%
\pgfpathlineto{\pgfqpoint{2.877093in}{1.136816in}}%
\pgfpathlineto{\pgfqpoint{2.865280in}{1.133278in}}%
\pgfpathlineto{\pgfqpoint{2.859932in}{1.127064in}}%
\pgfpathclose%
\pgfusepath{fill}%
\end{pgfscope}%
\begin{pgfscope}%
\pgfpathrectangle{\pgfqpoint{0.100000in}{0.100000in}}{\pgfqpoint{3.608454in}{2.310000in}}%
\pgfusepath{clip}%
\pgfsetbuttcap%
\pgfsetmiterjoin%
\definecolor{currentfill}{rgb}{0.000000,0.439216,0.780392}%
\pgfsetfillcolor{currentfill}%
\pgfsetlinewidth{0.000000pt}%
\definecolor{currentstroke}{rgb}{0.000000,0.000000,0.000000}%
\pgfsetstrokecolor{currentstroke}%
\pgfsetstrokeopacity{0.000000}%
\pgfsetdash{}{0pt}%
\pgfpathmoveto{\pgfqpoint{1.291011in}{1.784895in}}%
\pgfpathlineto{\pgfqpoint{1.303529in}{1.783157in}}%
\pgfpathlineto{\pgfqpoint{1.304569in}{1.790081in}}%
\pgfpathlineto{\pgfqpoint{1.314743in}{1.788523in}}%
\pgfpathlineto{\pgfqpoint{1.315604in}{1.795301in}}%
\pgfpathlineto{\pgfqpoint{1.322614in}{1.794241in}}%
\pgfpathlineto{\pgfqpoint{1.323648in}{1.801191in}}%
\pgfpathlineto{\pgfqpoint{1.337257in}{1.799111in}}%
\pgfpathlineto{\pgfqpoint{1.338318in}{1.806097in}}%
\pgfpathlineto{\pgfqpoint{1.417395in}{1.794618in}}%
\pgfpathlineto{\pgfqpoint{1.412417in}{1.742036in}}%
\pgfpathlineto{\pgfqpoint{1.388262in}{1.745418in}}%
\pgfpathlineto{\pgfqpoint{1.384511in}{1.743611in}}%
\pgfpathlineto{\pgfqpoint{1.352534in}{1.748041in}}%
\pgfpathlineto{\pgfqpoint{1.342072in}{1.748707in}}%
\pgfpathlineto{\pgfqpoint{1.322608in}{1.757434in}}%
\pgfpathlineto{\pgfqpoint{1.323149in}{1.760841in}}%
\pgfpathlineto{\pgfqpoint{1.310078in}{1.766325in}}%
\pgfpathlineto{\pgfqpoint{1.296615in}{1.768471in}}%
\pgfpathlineto{\pgfqpoint{1.297605in}{1.774698in}}%
\pgfpathlineto{\pgfqpoint{1.291011in}{1.784895in}}%
\pgfpathclose%
\pgfusepath{fill}%
\end{pgfscope}%
\begin{pgfscope}%
\pgfpathrectangle{\pgfqpoint{0.100000in}{0.100000in}}{\pgfqpoint{3.608454in}{2.310000in}}%
\pgfusepath{clip}%
\pgfsetbuttcap%
\pgfsetmiterjoin%
\definecolor{currentfill}{rgb}{0.000000,0.411765,0.794118}%
\pgfsetfillcolor{currentfill}%
\pgfsetlinewidth{0.000000pt}%
\definecolor{currentstroke}{rgb}{0.000000,0.000000,0.000000}%
\pgfsetstrokecolor{currentstroke}%
\pgfsetstrokeopacity{0.000000}%
\pgfsetdash{}{0pt}%
\pgfpathmoveto{\pgfqpoint{2.522980in}{1.660007in}}%
\pgfpathlineto{\pgfqpoint{2.467930in}{1.655975in}}%
\pgfpathlineto{\pgfqpoint{2.454210in}{1.655405in}}%
\pgfpathlineto{\pgfqpoint{2.450492in}{1.717550in}}%
\pgfpathlineto{\pgfqpoint{2.485186in}{1.719799in}}%
\pgfpathlineto{\pgfqpoint{2.485729in}{1.712888in}}%
\pgfpathlineto{\pgfqpoint{2.499443in}{1.713874in}}%
\pgfpathlineto{\pgfqpoint{2.520499in}{1.715478in}}%
\pgfpathlineto{\pgfqpoint{2.520327in}{1.709190in}}%
\pgfpathlineto{\pgfqpoint{2.517313in}{1.703277in}}%
\pgfpathlineto{\pgfqpoint{2.515606in}{1.691180in}}%
\pgfpathlineto{\pgfqpoint{2.522980in}{1.660007in}}%
\pgfpathclose%
\pgfusepath{fill}%
\end{pgfscope}%
\begin{pgfscope}%
\pgfpathrectangle{\pgfqpoint{0.100000in}{0.100000in}}{\pgfqpoint{3.608454in}{2.310000in}}%
\pgfusepath{clip}%
\pgfsetbuttcap%
\pgfsetmiterjoin%
\definecolor{currentfill}{rgb}{0.000000,0.419608,0.790196}%
\pgfsetfillcolor{currentfill}%
\pgfsetlinewidth{0.000000pt}%
\definecolor{currentstroke}{rgb}{0.000000,0.000000,0.000000}%
\pgfsetstrokecolor{currentstroke}%
\pgfsetstrokeopacity{0.000000}%
\pgfsetdash{}{0pt}%
\pgfpathmoveto{\pgfqpoint{0.721788in}{1.059750in}}%
\pgfpathlineto{\pgfqpoint{0.715639in}{1.032543in}}%
\pgfpathlineto{\pgfqpoint{0.714200in}{1.032869in}}%
\pgfpathlineto{\pgfqpoint{0.705665in}{0.997629in}}%
\pgfpathlineto{\pgfqpoint{0.637733in}{1.007252in}}%
\pgfpathlineto{\pgfqpoint{0.638583in}{1.013548in}}%
\pgfpathlineto{\pgfqpoint{0.632350in}{1.019382in}}%
\pgfpathlineto{\pgfqpoint{0.635822in}{1.034911in}}%
\pgfpathlineto{\pgfqpoint{0.636748in}{1.045662in}}%
\pgfpathlineto{\pgfqpoint{0.634791in}{1.058910in}}%
\pgfpathlineto{\pgfqpoint{0.628784in}{1.074472in}}%
\pgfpathlineto{\pgfqpoint{0.623749in}{1.080645in}}%
\pgfpathlineto{\pgfqpoint{0.626202in}{1.085453in}}%
\pgfpathlineto{\pgfqpoint{0.631612in}{1.088268in}}%
\pgfpathlineto{\pgfqpoint{0.640257in}{1.084919in}}%
\pgfpathlineto{\pgfqpoint{0.647432in}{1.078308in}}%
\pgfpathlineto{\pgfqpoint{0.660882in}{1.074556in}}%
\pgfpathlineto{\pgfqpoint{0.721788in}{1.059750in}}%
\pgfpathclose%
\pgfusepath{fill}%
\end{pgfscope}%
\begin{pgfscope}%
\pgfpathrectangle{\pgfqpoint{0.100000in}{0.100000in}}{\pgfqpoint{3.608454in}{2.310000in}}%
\pgfusepath{clip}%
\pgfsetbuttcap%
\pgfsetmiterjoin%
\definecolor{currentfill}{rgb}{0.000000,0.713725,0.643137}%
\pgfsetfillcolor{currentfill}%
\pgfsetlinewidth{0.000000pt}%
\definecolor{currentstroke}{rgb}{0.000000,0.000000,0.000000}%
\pgfsetstrokecolor{currentstroke}%
\pgfsetstrokeopacity{0.000000}%
\pgfsetdash{}{0pt}%
\pgfpathmoveto{\pgfqpoint{2.380907in}{1.900334in}}%
\pgfpathlineto{\pgfqpoint{2.366647in}{1.899528in}}%
\pgfpathlineto{\pgfqpoint{2.365878in}{1.913327in}}%
\pgfpathlineto{\pgfqpoint{2.359091in}{1.912916in}}%
\pgfpathlineto{\pgfqpoint{2.358761in}{1.919792in}}%
\pgfpathlineto{\pgfqpoint{2.351813in}{1.919479in}}%
\pgfpathlineto{\pgfqpoint{2.350608in}{1.946791in}}%
\pgfpathlineto{\pgfqpoint{2.356731in}{1.945339in}}%
\pgfpathlineto{\pgfqpoint{2.361222in}{1.948386in}}%
\pgfpathlineto{\pgfqpoint{2.378630in}{1.955337in}}%
\pgfpathlineto{\pgfqpoint{2.391147in}{1.967641in}}%
\pgfpathlineto{\pgfqpoint{2.411560in}{1.970914in}}%
\pgfpathlineto{\pgfqpoint{2.420196in}{1.976038in}}%
\pgfpathlineto{\pgfqpoint{2.425368in}{1.983108in}}%
\pgfpathlineto{\pgfqpoint{2.433892in}{1.984602in}}%
\pgfpathlineto{\pgfqpoint{2.436662in}{1.987686in}}%
\pgfpathlineto{\pgfqpoint{2.438144in}{1.966329in}}%
\pgfpathlineto{\pgfqpoint{2.442299in}{1.959772in}}%
\pgfpathlineto{\pgfqpoint{2.435480in}{1.959378in}}%
\pgfpathlineto{\pgfqpoint{2.436761in}{1.938649in}}%
\pgfpathlineto{\pgfqpoint{2.438472in}{1.913015in}}%
\pgfpathlineto{\pgfqpoint{2.432701in}{1.915926in}}%
\pgfpathlineto{\pgfqpoint{2.385839in}{1.925963in}}%
\pgfpathlineto{\pgfqpoint{2.387268in}{1.900687in}}%
\pgfpathlineto{\pgfqpoint{2.380907in}{1.900334in}}%
\pgfpathclose%
\pgfusepath{fill}%
\end{pgfscope}%
\begin{pgfscope}%
\pgfpathrectangle{\pgfqpoint{0.100000in}{0.100000in}}{\pgfqpoint{3.608454in}{2.310000in}}%
\pgfusepath{clip}%
\pgfsetbuttcap%
\pgfsetmiterjoin%
\definecolor{currentfill}{rgb}{0.000000,0.403922,0.798039}%
\pgfsetfillcolor{currentfill}%
\pgfsetlinewidth{0.000000pt}%
\definecolor{currentstroke}{rgb}{0.000000,0.000000,0.000000}%
\pgfsetstrokecolor{currentstroke}%
\pgfsetstrokeopacity{0.000000}%
\pgfsetdash{}{0pt}%
\pgfpathmoveto{\pgfqpoint{1.577074in}{1.409804in}}%
\pgfpathlineto{\pgfqpoint{1.610603in}{1.406638in}}%
\pgfpathlineto{\pgfqpoint{1.606968in}{1.372189in}}%
\pgfpathlineto{\pgfqpoint{1.603799in}{1.337764in}}%
\pgfpathlineto{\pgfqpoint{1.602094in}{1.324825in}}%
\pgfpathlineto{\pgfqpoint{1.581597in}{1.326016in}}%
\pgfpathlineto{\pgfqpoint{1.547834in}{1.329763in}}%
\pgfpathlineto{\pgfqpoint{1.550467in}{1.357020in}}%
\pgfpathlineto{\pgfqpoint{1.571014in}{1.354845in}}%
\pgfpathlineto{\pgfqpoint{1.577074in}{1.409804in}}%
\pgfpathclose%
\pgfusepath{fill}%
\end{pgfscope}%
\begin{pgfscope}%
\pgfpathrectangle{\pgfqpoint{0.100000in}{0.100000in}}{\pgfqpoint{3.608454in}{2.310000in}}%
\pgfusepath{clip}%
\pgfsetbuttcap%
\pgfsetmiterjoin%
\definecolor{currentfill}{rgb}{0.000000,0.537255,0.731373}%
\pgfsetfillcolor{currentfill}%
\pgfsetlinewidth{0.000000pt}%
\definecolor{currentstroke}{rgb}{0.000000,0.000000,0.000000}%
\pgfsetstrokecolor{currentstroke}%
\pgfsetstrokeopacity{0.000000}%
\pgfsetdash{}{0pt}%
\pgfpathmoveto{\pgfqpoint{2.101955in}{1.263584in}}%
\pgfpathlineto{\pgfqpoint{2.101288in}{1.239463in}}%
\pgfpathlineto{\pgfqpoint{2.046716in}{1.239913in}}%
\pgfpathlineto{\pgfqpoint{2.047067in}{1.264108in}}%
\pgfpathlineto{\pgfqpoint{2.074465in}{1.263717in}}%
\pgfpathlineto{\pgfqpoint{2.075326in}{1.295024in}}%
\pgfpathlineto{\pgfqpoint{2.075903in}{1.319066in}}%
\pgfpathlineto{\pgfqpoint{2.103312in}{1.318980in}}%
\pgfpathlineto{\pgfqpoint{2.102737in}{1.315533in}}%
\pgfpathlineto{\pgfqpoint{2.102752in}{1.281057in}}%
\pgfpathlineto{\pgfqpoint{2.101955in}{1.263584in}}%
\pgfpathclose%
\pgfusepath{fill}%
\end{pgfscope}%
\begin{pgfscope}%
\pgfpathrectangle{\pgfqpoint{0.100000in}{0.100000in}}{\pgfqpoint{3.608454in}{2.310000in}}%
\pgfusepath{clip}%
\pgfsetbuttcap%
\pgfsetmiterjoin%
\definecolor{currentfill}{rgb}{0.000000,0.517647,0.741176}%
\pgfsetfillcolor{currentfill}%
\pgfsetlinewidth{0.000000pt}%
\definecolor{currentstroke}{rgb}{0.000000,0.000000,0.000000}%
\pgfsetstrokecolor{currentstroke}%
\pgfsetstrokeopacity{0.000000}%
\pgfsetdash{}{0pt}%
\pgfpathmoveto{\pgfqpoint{2.356072in}{1.695517in}}%
\pgfpathlineto{\pgfqpoint{2.325187in}{1.694132in}}%
\pgfpathlineto{\pgfqpoint{2.324231in}{1.700249in}}%
\pgfpathlineto{\pgfqpoint{2.301639in}{1.699356in}}%
\pgfpathlineto{\pgfqpoint{2.294773in}{1.699116in}}%
\pgfpathlineto{\pgfqpoint{2.293781in}{1.726557in}}%
\pgfpathlineto{\pgfqpoint{2.319291in}{1.727547in}}%
\pgfpathlineto{\pgfqpoint{2.310801in}{1.738104in}}%
\pgfpathlineto{\pgfqpoint{2.316678in}{1.738350in}}%
\pgfpathlineto{\pgfqpoint{2.317255in}{1.744761in}}%
\pgfpathlineto{\pgfqpoint{2.336281in}{1.746036in}}%
\pgfpathlineto{\pgfqpoint{2.339782in}{1.753146in}}%
\pgfpathlineto{\pgfqpoint{2.373549in}{1.754537in}}%
\pgfpathlineto{\pgfqpoint{2.373196in}{1.761948in}}%
\pgfpathlineto{\pgfqpoint{2.396439in}{1.763257in}}%
\pgfpathlineto{\pgfqpoint{2.389968in}{1.750300in}}%
\pgfpathlineto{\pgfqpoint{2.394954in}{1.739243in}}%
\pgfpathlineto{\pgfqpoint{2.395001in}{1.732218in}}%
\pgfpathlineto{\pgfqpoint{2.400634in}{1.726438in}}%
\pgfpathlineto{\pgfqpoint{2.405834in}{1.715444in}}%
\pgfpathlineto{\pgfqpoint{2.375661in}{1.713794in}}%
\pgfpathlineto{\pgfqpoint{2.376100in}{1.706906in}}%
\pgfpathlineto{\pgfqpoint{2.355403in}{1.705801in}}%
\pgfpathlineto{\pgfqpoint{2.356072in}{1.695517in}}%
\pgfpathclose%
\pgfusepath{fill}%
\end{pgfscope}%
\begin{pgfscope}%
\pgfpathrectangle{\pgfqpoint{0.100000in}{0.100000in}}{\pgfqpoint{3.608454in}{2.310000in}}%
\pgfusepath{clip}%
\pgfsetbuttcap%
\pgfsetmiterjoin%
\definecolor{currentfill}{rgb}{0.000000,0.603922,0.698039}%
\pgfsetfillcolor{currentfill}%
\pgfsetlinewidth{0.000000pt}%
\definecolor{currentstroke}{rgb}{0.000000,0.000000,0.000000}%
\pgfsetstrokecolor{currentstroke}%
\pgfsetstrokeopacity{0.000000}%
\pgfsetdash{}{0pt}%
\pgfpathmoveto{\pgfqpoint{1.234540in}{2.143503in}}%
\pgfpathlineto{\pgfqpoint{1.232482in}{2.132070in}}%
\pgfpathlineto{\pgfqpoint{1.202148in}{2.137522in}}%
\pgfpathlineto{\pgfqpoint{1.200330in}{2.146187in}}%
\pgfpathlineto{\pgfqpoint{1.194140in}{2.150958in}}%
\pgfpathlineto{\pgfqpoint{1.150149in}{2.159708in}}%
\pgfpathlineto{\pgfqpoint{1.139517in}{2.164977in}}%
\pgfpathlineto{\pgfqpoint{1.140465in}{2.171053in}}%
\pgfpathlineto{\pgfqpoint{1.134122in}{2.180102in}}%
\pgfpathlineto{\pgfqpoint{1.136982in}{2.181768in}}%
\pgfpathlineto{\pgfqpoint{1.131142in}{2.190334in}}%
\pgfpathlineto{\pgfqpoint{1.133270in}{2.197089in}}%
\pgfpathlineto{\pgfqpoint{1.119541in}{2.205769in}}%
\pgfpathlineto{\pgfqpoint{1.123862in}{2.213141in}}%
\pgfpathlineto{\pgfqpoint{1.118945in}{2.225334in}}%
\pgfpathlineto{\pgfqpoint{1.112359in}{2.223590in}}%
\pgfpathlineto{\pgfqpoint{1.109470in}{2.229400in}}%
\pgfpathlineto{\pgfqpoint{1.109601in}{2.239048in}}%
\pgfpathlineto{\pgfqpoint{1.157261in}{2.229106in}}%
\pgfpathlineto{\pgfqpoint{1.193179in}{2.222014in}}%
\pgfpathlineto{\pgfqpoint{1.254169in}{2.210455in}}%
\pgfpathlineto{\pgfqpoint{1.242847in}{2.149099in}}%
\pgfpathlineto{\pgfqpoint{1.235802in}{2.150383in}}%
\pgfpathlineto{\pgfqpoint{1.234540in}{2.143503in}}%
\pgfpathclose%
\pgfusepath{fill}%
\end{pgfscope}%
\begin{pgfscope}%
\pgfpathrectangle{\pgfqpoint{0.100000in}{0.100000in}}{\pgfqpoint{3.608454in}{2.310000in}}%
\pgfusepath{clip}%
\pgfsetbuttcap%
\pgfsetmiterjoin%
\definecolor{currentfill}{rgb}{0.000000,0.380392,0.809804}%
\pgfsetfillcolor{currentfill}%
\pgfsetlinewidth{0.000000pt}%
\definecolor{currentstroke}{rgb}{0.000000,0.000000,0.000000}%
\pgfsetstrokecolor{currentstroke}%
\pgfsetstrokeopacity{0.000000}%
\pgfsetdash{}{0pt}%
\pgfpathmoveto{\pgfqpoint{1.807069in}{2.050968in}}%
\pgfpathlineto{\pgfqpoint{1.738152in}{2.055718in}}%
\pgfpathlineto{\pgfqpoint{1.739211in}{2.069716in}}%
\pgfpathlineto{\pgfqpoint{1.736501in}{2.069931in}}%
\pgfpathlineto{\pgfqpoint{1.738640in}{2.097811in}}%
\pgfpathlineto{\pgfqpoint{1.735922in}{2.098000in}}%
\pgfpathlineto{\pgfqpoint{1.737002in}{2.111989in}}%
\pgfpathlineto{\pgfqpoint{1.723190in}{2.112972in}}%
\pgfpathlineto{\pgfqpoint{1.724331in}{2.126897in}}%
\pgfpathlineto{\pgfqpoint{1.728606in}{2.126565in}}%
\pgfpathlineto{\pgfqpoint{1.729180in}{2.133512in}}%
\pgfpathlineto{\pgfqpoint{1.736050in}{2.132965in}}%
\pgfpathlineto{\pgfqpoint{1.737269in}{2.148337in}}%
\pgfpathlineto{\pgfqpoint{1.790039in}{2.144561in}}%
\pgfpathlineto{\pgfqpoint{1.833904in}{2.141855in}}%
\pgfpathlineto{\pgfqpoint{1.832606in}{2.119399in}}%
\pgfpathlineto{\pgfqpoint{1.833772in}{2.105376in}}%
\pgfpathlineto{\pgfqpoint{1.826839in}{2.105738in}}%
\pgfpathlineto{\pgfqpoint{1.828130in}{2.091731in}}%
\pgfpathlineto{\pgfqpoint{1.826539in}{2.063794in}}%
\pgfpathlineto{\pgfqpoint{1.827828in}{2.049744in}}%
\pgfpathlineto{\pgfqpoint{1.807069in}{2.050968in}}%
\pgfpathclose%
\pgfusepath{fill}%
\end{pgfscope}%
\begin{pgfscope}%
\pgfpathrectangle{\pgfqpoint{0.100000in}{0.100000in}}{\pgfqpoint{3.608454in}{2.310000in}}%
\pgfusepath{clip}%
\pgfsetbuttcap%
\pgfsetmiterjoin%
\definecolor{currentfill}{rgb}{0.000000,0.639216,0.680392}%
\pgfsetfillcolor{currentfill}%
\pgfsetlinewidth{0.000000pt}%
\definecolor{currentstroke}{rgb}{0.000000,0.000000,0.000000}%
\pgfsetstrokecolor{currentstroke}%
\pgfsetstrokeopacity{0.000000}%
\pgfsetdash{}{0pt}%
\pgfpathmoveto{\pgfqpoint{1.889333in}{1.734748in}}%
\pgfpathlineto{\pgfqpoint{1.888636in}{1.700186in}}%
\pgfpathlineto{\pgfqpoint{1.859723in}{1.701494in}}%
\pgfpathlineto{\pgfqpoint{1.841021in}{1.702423in}}%
\pgfpathlineto{\pgfqpoint{1.842006in}{1.721715in}}%
\pgfpathlineto{\pgfqpoint{1.829398in}{1.718027in}}%
\pgfpathlineto{\pgfqpoint{1.817373in}{1.718961in}}%
\pgfpathlineto{\pgfqpoint{1.807130in}{1.721260in}}%
\pgfpathlineto{\pgfqpoint{1.800957in}{1.721865in}}%
\pgfpathlineto{\pgfqpoint{1.800009in}{1.732253in}}%
\pgfpathlineto{\pgfqpoint{1.801456in}{1.757724in}}%
\pgfpathlineto{\pgfqpoint{1.801576in}{1.759791in}}%
\pgfpathlineto{\pgfqpoint{1.825951in}{1.758455in}}%
\pgfpathlineto{\pgfqpoint{1.830770in}{1.752552in}}%
\pgfpathlineto{\pgfqpoint{1.835653in}{1.753170in}}%
\pgfpathlineto{\pgfqpoint{1.843095in}{1.749824in}}%
\pgfpathlineto{\pgfqpoint{1.841646in}{1.759396in}}%
\pgfpathlineto{\pgfqpoint{1.846482in}{1.757163in}}%
\pgfpathlineto{\pgfqpoint{1.883458in}{1.755759in}}%
\pgfpathlineto{\pgfqpoint{1.882514in}{1.735032in}}%
\pgfpathlineto{\pgfqpoint{1.889333in}{1.734748in}}%
\pgfpathclose%
\pgfusepath{fill}%
\end{pgfscope}%
\begin{pgfscope}%
\pgfpathrectangle{\pgfqpoint{0.100000in}{0.100000in}}{\pgfqpoint{3.608454in}{2.310000in}}%
\pgfusepath{clip}%
\pgfsetbuttcap%
\pgfsetmiterjoin%
\definecolor{currentfill}{rgb}{0.000000,0.925490,0.537255}%
\pgfsetfillcolor{currentfill}%
\pgfsetlinewidth{0.000000pt}%
\definecolor{currentstroke}{rgb}{0.000000,0.000000,0.000000}%
\pgfsetstrokecolor{currentstroke}%
\pgfsetstrokeopacity{0.000000}%
\pgfsetdash{}{0pt}%
\pgfpathmoveto{\pgfqpoint{2.895571in}{1.311319in}}%
\pgfpathlineto{\pgfqpoint{2.902369in}{1.314765in}}%
\pgfpathlineto{\pgfqpoint{2.908038in}{1.306264in}}%
\pgfpathlineto{\pgfqpoint{2.909448in}{1.297631in}}%
\pgfpathlineto{\pgfqpoint{2.916184in}{1.292250in}}%
\pgfpathlineto{\pgfqpoint{2.929879in}{1.293625in}}%
\pgfpathlineto{\pgfqpoint{2.927867in}{1.284414in}}%
\pgfpathlineto{\pgfqpoint{2.923822in}{1.280887in}}%
\pgfpathlineto{\pgfqpoint{2.921023in}{1.282534in}}%
\pgfpathlineto{\pgfqpoint{2.902185in}{1.260526in}}%
\pgfpathlineto{\pgfqpoint{2.887645in}{1.250541in}}%
\pgfpathlineto{\pgfqpoint{2.880671in}{1.252069in}}%
\pgfpathlineto{\pgfqpoint{2.874724in}{1.257922in}}%
\pgfpathlineto{\pgfqpoint{2.872542in}{1.266952in}}%
\pgfpathlineto{\pgfqpoint{2.860524in}{1.271364in}}%
\pgfpathlineto{\pgfqpoint{2.849787in}{1.279563in}}%
\pgfpathlineto{\pgfqpoint{2.839446in}{1.282704in}}%
\pgfpathlineto{\pgfqpoint{2.837862in}{1.289352in}}%
\pgfpathlineto{\pgfqpoint{2.847809in}{1.300549in}}%
\pgfpathlineto{\pgfqpoint{2.855369in}{1.302588in}}%
\pgfpathlineto{\pgfqpoint{2.850698in}{1.310252in}}%
\pgfpathlineto{\pgfqpoint{2.857831in}{1.319548in}}%
\pgfpathlineto{\pgfqpoint{2.855274in}{1.324556in}}%
\pgfpathlineto{\pgfqpoint{2.861780in}{1.330306in}}%
\pgfpathlineto{\pgfqpoint{2.874015in}{1.332782in}}%
\pgfpathlineto{\pgfqpoint{2.873043in}{1.323790in}}%
\pgfpathlineto{\pgfqpoint{2.885378in}{1.313095in}}%
\pgfpathlineto{\pgfqpoint{2.884060in}{1.308388in}}%
\pgfpathlineto{\pgfqpoint{2.890443in}{1.304155in}}%
\pgfpathlineto{\pgfqpoint{2.895571in}{1.311319in}}%
\pgfpathclose%
\pgfusepath{fill}%
\end{pgfscope}%
\begin{pgfscope}%
\pgfpathrectangle{\pgfqpoint{0.100000in}{0.100000in}}{\pgfqpoint{3.608454in}{2.310000in}}%
\pgfusepath{clip}%
\pgfsetbuttcap%
\pgfsetmiterjoin%
\definecolor{currentfill}{rgb}{0.000000,0.462745,0.768627}%
\pgfsetfillcolor{currentfill}%
\pgfsetlinewidth{0.000000pt}%
\definecolor{currentstroke}{rgb}{0.000000,0.000000,0.000000}%
\pgfsetstrokecolor{currentstroke}%
\pgfsetstrokeopacity{0.000000}%
\pgfsetdash{}{0pt}%
\pgfpathmoveto{\pgfqpoint{2.193901in}{1.649520in}}%
\pgfpathlineto{\pgfqpoint{2.166463in}{1.649078in}}%
\pgfpathlineto{\pgfqpoint{2.165979in}{1.695985in}}%
\pgfpathlineto{\pgfqpoint{2.184501in}{1.696223in}}%
\pgfpathlineto{\pgfqpoint{2.220391in}{1.696881in}}%
\pgfpathlineto{\pgfqpoint{2.221382in}{1.649990in}}%
\pgfpathlineto{\pgfqpoint{2.193901in}{1.649520in}}%
\pgfpathclose%
\pgfusepath{fill}%
\end{pgfscope}%
\begin{pgfscope}%
\pgfpathrectangle{\pgfqpoint{0.100000in}{0.100000in}}{\pgfqpoint{3.608454in}{2.310000in}}%
\pgfusepath{clip}%
\pgfsetbuttcap%
\pgfsetmiterjoin%
\definecolor{currentfill}{rgb}{0.000000,0.768627,0.615686}%
\pgfsetfillcolor{currentfill}%
\pgfsetlinewidth{0.000000pt}%
\definecolor{currentstroke}{rgb}{0.000000,0.000000,0.000000}%
\pgfsetstrokecolor{currentstroke}%
\pgfsetstrokeopacity{0.000000}%
\pgfsetdash{}{0pt}%
\pgfpathmoveto{\pgfqpoint{2.936140in}{1.370645in}}%
\pgfpathlineto{\pgfqpoint{2.930069in}{1.365085in}}%
\pgfpathlineto{\pgfqpoint{2.920044in}{1.373705in}}%
\pgfpathlineto{\pgfqpoint{2.910035in}{1.386821in}}%
\pgfpathlineto{\pgfqpoint{2.916733in}{1.399937in}}%
\pgfpathlineto{\pgfqpoint{2.915344in}{1.412906in}}%
\pgfpathlineto{\pgfqpoint{2.916394in}{1.420191in}}%
\pgfpathlineto{\pgfqpoint{2.907865in}{1.422811in}}%
\pgfpathlineto{\pgfqpoint{2.906878in}{1.433220in}}%
\pgfpathlineto{\pgfqpoint{2.907685in}{1.436860in}}%
\pgfpathlineto{\pgfqpoint{2.914756in}{1.436835in}}%
\pgfpathlineto{\pgfqpoint{2.914676in}{1.445590in}}%
\pgfpathlineto{\pgfqpoint{2.927738in}{1.446958in}}%
\pgfpathlineto{\pgfqpoint{2.938939in}{1.450447in}}%
\pgfpathlineto{\pgfqpoint{2.941441in}{1.448688in}}%
\pgfpathlineto{\pgfqpoint{2.954171in}{1.450091in}}%
\pgfpathlineto{\pgfqpoint{2.951761in}{1.443677in}}%
\pgfpathlineto{\pgfqpoint{2.962415in}{1.436530in}}%
\pgfpathlineto{\pgfqpoint{2.966480in}{1.429783in}}%
\pgfpathlineto{\pgfqpoint{2.965224in}{1.427879in}}%
\pgfpathlineto{\pgfqpoint{2.973331in}{1.416066in}}%
\pgfpathlineto{\pgfqpoint{2.967704in}{1.409901in}}%
\pgfpathlineto{\pgfqpoint{2.961168in}{1.406181in}}%
\pgfpathlineto{\pgfqpoint{2.954499in}{1.407993in}}%
\pgfpathlineto{\pgfqpoint{2.943680in}{1.394391in}}%
\pgfpathlineto{\pgfqpoint{2.936799in}{1.397031in}}%
\pgfpathlineto{\pgfqpoint{2.932362in}{1.392633in}}%
\pgfpathlineto{\pgfqpoint{2.933951in}{1.375085in}}%
\pgfpathlineto{\pgfqpoint{2.936140in}{1.370645in}}%
\pgfpathclose%
\pgfusepath{fill}%
\end{pgfscope}%
\begin{pgfscope}%
\pgfpathrectangle{\pgfqpoint{0.100000in}{0.100000in}}{\pgfqpoint{3.608454in}{2.310000in}}%
\pgfusepath{clip}%
\pgfsetbuttcap%
\pgfsetmiterjoin%
\definecolor{currentfill}{rgb}{0.000000,0.431373,0.784314}%
\pgfsetfillcolor{currentfill}%
\pgfsetlinewidth{0.000000pt}%
\definecolor{currentstroke}{rgb}{0.000000,0.000000,0.000000}%
\pgfsetstrokecolor{currentstroke}%
\pgfsetstrokeopacity{0.000000}%
\pgfsetdash{}{0pt}%
\pgfpathmoveto{\pgfqpoint{2.258667in}{1.780519in}}%
\pgfpathlineto{\pgfqpoint{2.268903in}{1.780718in}}%
\pgfpathlineto{\pgfqpoint{2.267401in}{1.826914in}}%
\pgfpathlineto{\pgfqpoint{2.266038in}{1.833884in}}%
\pgfpathlineto{\pgfqpoint{2.300433in}{1.834842in}}%
\pgfpathlineto{\pgfqpoint{2.300225in}{1.841636in}}%
\pgfpathlineto{\pgfqpoint{2.334657in}{1.843071in}}%
\pgfpathlineto{\pgfqpoint{2.337567in}{1.780978in}}%
\pgfpathlineto{\pgfqpoint{2.344426in}{1.781264in}}%
\pgfpathlineto{\pgfqpoint{2.344731in}{1.774369in}}%
\pgfpathlineto{\pgfqpoint{2.372249in}{1.775882in}}%
\pgfpathlineto{\pgfqpoint{2.373196in}{1.761948in}}%
\pgfpathlineto{\pgfqpoint{2.373549in}{1.754537in}}%
\pgfpathlineto{\pgfqpoint{2.339782in}{1.753146in}}%
\pgfpathlineto{\pgfqpoint{2.336281in}{1.746036in}}%
\pgfpathlineto{\pgfqpoint{2.317255in}{1.744761in}}%
\pgfpathlineto{\pgfqpoint{2.316678in}{1.738350in}}%
\pgfpathlineto{\pgfqpoint{2.310801in}{1.738104in}}%
\pgfpathlineto{\pgfqpoint{2.319291in}{1.727547in}}%
\pgfpathlineto{\pgfqpoint{2.293781in}{1.726557in}}%
\pgfpathlineto{\pgfqpoint{2.273836in}{1.725888in}}%
\pgfpathlineto{\pgfqpoint{2.272950in}{1.753323in}}%
\pgfpathlineto{\pgfqpoint{2.285525in}{1.753739in}}%
\pgfpathlineto{\pgfqpoint{2.281318in}{1.764757in}}%
\pgfpathlineto{\pgfqpoint{2.273713in}{1.770000in}}%
\pgfpathlineto{\pgfqpoint{2.264630in}{1.772555in}}%
\pgfpathlineto{\pgfqpoint{2.258667in}{1.780519in}}%
\pgfpathclose%
\pgfusepath{fill}%
\end{pgfscope}%
\begin{pgfscope}%
\pgfpathrectangle{\pgfqpoint{0.100000in}{0.100000in}}{\pgfqpoint{3.608454in}{2.310000in}}%
\pgfusepath{clip}%
\pgfsetbuttcap%
\pgfsetmiterjoin%
\definecolor{currentfill}{rgb}{0.000000,0.678431,0.660784}%
\pgfsetfillcolor{currentfill}%
\pgfsetlinewidth{0.000000pt}%
\definecolor{currentstroke}{rgb}{0.000000,0.000000,0.000000}%
\pgfsetstrokecolor{currentstroke}%
\pgfsetstrokeopacity{0.000000}%
\pgfsetdash{}{0pt}%
\pgfpathmoveto{\pgfqpoint{1.938836in}{0.498314in}}%
\pgfpathlineto{\pgfqpoint{1.924683in}{0.477880in}}%
\pgfpathlineto{\pgfqpoint{1.907060in}{0.482191in}}%
\pgfpathlineto{\pgfqpoint{1.907719in}{0.482706in}}%
\pgfpathlineto{\pgfqpoint{1.889121in}{0.521871in}}%
\pgfpathlineto{\pgfqpoint{1.895145in}{0.523860in}}%
\pgfpathlineto{\pgfqpoint{1.882610in}{0.539717in}}%
\pgfpathlineto{\pgfqpoint{1.915913in}{0.565873in}}%
\pgfpathlineto{\pgfqpoint{1.923720in}{0.556660in}}%
\pgfpathlineto{\pgfqpoint{1.913527in}{0.548723in}}%
\pgfpathlineto{\pgfqpoint{1.925742in}{0.532805in}}%
\pgfpathlineto{\pgfqpoint{1.911040in}{0.521565in}}%
\pgfpathlineto{\pgfqpoint{1.916523in}{0.511256in}}%
\pgfpathlineto{\pgfqpoint{1.926187in}{0.508771in}}%
\pgfpathlineto{\pgfqpoint{1.926522in}{0.503678in}}%
\pgfpathlineto{\pgfqpoint{1.938836in}{0.498314in}}%
\pgfpathclose%
\pgfusepath{fill}%
\end{pgfscope}%
\begin{pgfscope}%
\pgfpathrectangle{\pgfqpoint{0.100000in}{0.100000in}}{\pgfqpoint{3.608454in}{2.310000in}}%
\pgfusepath{clip}%
\pgfsetbuttcap%
\pgfsetmiterjoin%
\definecolor{currentfill}{rgb}{0.000000,0.596078,0.701961}%
\pgfsetfillcolor{currentfill}%
\pgfsetlinewidth{0.000000pt}%
\definecolor{currentstroke}{rgb}{0.000000,0.000000,0.000000}%
\pgfsetstrokecolor{currentstroke}%
\pgfsetstrokeopacity{0.000000}%
\pgfsetdash{}{0pt}%
\pgfpathmoveto{\pgfqpoint{2.999992in}{0.934399in}}%
\pgfpathlineto{\pgfqpoint{2.986683in}{0.921479in}}%
\pgfpathlineto{\pgfqpoint{2.980224in}{0.924118in}}%
\pgfpathlineto{\pgfqpoint{2.965871in}{0.919067in}}%
\pgfpathlineto{\pgfqpoint{2.962570in}{0.910798in}}%
\pgfpathlineto{\pgfqpoint{2.959107in}{0.909040in}}%
\pgfpathlineto{\pgfqpoint{2.950977in}{0.910809in}}%
\pgfpathlineto{\pgfqpoint{2.944266in}{0.903605in}}%
\pgfpathlineto{\pgfqpoint{2.937410in}{0.908377in}}%
\pgfpathlineto{\pgfqpoint{2.937493in}{0.915146in}}%
\pgfpathlineto{\pgfqpoint{2.933286in}{0.922296in}}%
\pgfpathlineto{\pgfqpoint{2.925307in}{0.930408in}}%
\pgfpathlineto{\pgfqpoint{2.919634in}{0.933284in}}%
\pgfpathlineto{\pgfqpoint{2.918368in}{0.938239in}}%
\pgfpathlineto{\pgfqpoint{2.909363in}{0.953452in}}%
\pgfpathlineto{\pgfqpoint{2.910547in}{0.958327in}}%
\pgfpathlineto{\pgfqpoint{2.915431in}{0.959683in}}%
\pgfpathlineto{\pgfqpoint{2.918991in}{0.967320in}}%
\pgfpathlineto{\pgfqpoint{2.922909in}{0.970556in}}%
\pgfpathlineto{\pgfqpoint{2.932420in}{0.972207in}}%
\pgfpathlineto{\pgfqpoint{2.942345in}{0.978613in}}%
\pgfpathlineto{\pgfqpoint{2.948243in}{0.978295in}}%
\pgfpathlineto{\pgfqpoint{2.954883in}{0.971559in}}%
\pgfpathlineto{\pgfqpoint{2.956878in}{0.975335in}}%
\pgfpathlineto{\pgfqpoint{2.950161in}{0.981612in}}%
\pgfpathlineto{\pgfqpoint{2.953478in}{0.990526in}}%
\pgfpathlineto{\pgfqpoint{2.949644in}{0.997974in}}%
\pgfpathlineto{\pgfqpoint{2.958909in}{1.001900in}}%
\pgfpathlineto{\pgfqpoint{2.966134in}{1.004041in}}%
\pgfpathlineto{\pgfqpoint{2.971145in}{0.996030in}}%
\pgfpathlineto{\pgfqpoint{2.983807in}{0.993720in}}%
\pgfpathlineto{\pgfqpoint{2.988314in}{0.999433in}}%
\pgfpathlineto{\pgfqpoint{2.999338in}{0.990301in}}%
\pgfpathlineto{\pgfqpoint{3.016045in}{0.985596in}}%
\pgfpathlineto{\pgfqpoint{3.005840in}{0.971004in}}%
\pgfpathlineto{\pgfqpoint{2.984051in}{0.944289in}}%
\pgfpathlineto{\pgfqpoint{2.988444in}{0.938374in}}%
\pgfpathlineto{\pgfqpoint{2.999992in}{0.934399in}}%
\pgfpathclose%
\pgfusepath{fill}%
\end{pgfscope}%
\begin{pgfscope}%
\pgfpathrectangle{\pgfqpoint{0.100000in}{0.100000in}}{\pgfqpoint{3.608454in}{2.310000in}}%
\pgfusepath{clip}%
\pgfsetbuttcap%
\pgfsetmiterjoin%
\definecolor{currentfill}{rgb}{0.000000,0.466667,0.766667}%
\pgfsetfillcolor{currentfill}%
\pgfsetlinewidth{0.000000pt}%
\definecolor{currentstroke}{rgb}{0.000000,0.000000,0.000000}%
\pgfsetstrokecolor{currentstroke}%
\pgfsetstrokeopacity{0.000000}%
\pgfsetdash{}{0pt}%
\pgfpathmoveto{\pgfqpoint{3.243003in}{1.088398in}}%
\pgfpathlineto{\pgfqpoint{3.237815in}{1.092298in}}%
\pgfpathlineto{\pgfqpoint{3.228319in}{1.108263in}}%
\pgfpathlineto{\pgfqpoint{3.227150in}{1.124441in}}%
\pgfpathlineto{\pgfqpoint{3.219619in}{1.130949in}}%
\pgfpathlineto{\pgfqpoint{3.216293in}{1.140248in}}%
\pgfpathlineto{\pgfqpoint{3.211683in}{1.141958in}}%
\pgfpathlineto{\pgfqpoint{3.206484in}{1.172807in}}%
\pgfpathlineto{\pgfqpoint{3.229923in}{1.198333in}}%
\pgfpathlineto{\pgfqpoint{3.232624in}{1.197760in}}%
\pgfpathlineto{\pgfqpoint{3.244554in}{1.193790in}}%
\pgfpathlineto{\pgfqpoint{3.246587in}{1.185703in}}%
\pgfpathlineto{\pgfqpoint{3.250804in}{1.179929in}}%
\pgfpathlineto{\pgfqpoint{3.248243in}{1.176631in}}%
\pgfpathlineto{\pgfqpoint{3.248882in}{1.168666in}}%
\pgfpathlineto{\pgfqpoint{3.265343in}{1.162433in}}%
\pgfpathlineto{\pgfqpoint{3.273900in}{1.157290in}}%
\pgfpathlineto{\pgfqpoint{3.287098in}{1.161529in}}%
\pgfpathlineto{\pgfqpoint{3.286885in}{1.169498in}}%
\pgfpathlineto{\pgfqpoint{3.295207in}{1.169584in}}%
\pgfpathlineto{\pgfqpoint{3.297149in}{1.166013in}}%
\pgfpathlineto{\pgfqpoint{3.293259in}{1.151590in}}%
\pgfpathlineto{\pgfqpoint{3.288603in}{1.146209in}}%
\pgfpathlineto{\pgfqpoint{3.292278in}{1.142409in}}%
\pgfpathlineto{\pgfqpoint{3.303080in}{1.146473in}}%
\pgfpathlineto{\pgfqpoint{3.308404in}{1.145005in}}%
\pgfpathlineto{\pgfqpoint{3.312700in}{1.148475in}}%
\pgfpathlineto{\pgfqpoint{3.320379in}{1.148245in}}%
\pgfpathlineto{\pgfqpoint{3.305824in}{1.120574in}}%
\pgfpathlineto{\pgfqpoint{3.300774in}{1.117204in}}%
\pgfpathlineto{\pgfqpoint{3.293113in}{1.118696in}}%
\pgfpathlineto{\pgfqpoint{3.285161in}{1.117485in}}%
\pgfpathlineto{\pgfqpoint{3.263408in}{1.107685in}}%
\pgfpathlineto{\pgfqpoint{3.243003in}{1.088398in}}%
\pgfpathclose%
\pgfusepath{fill}%
\end{pgfscope}%
\begin{pgfscope}%
\pgfpathrectangle{\pgfqpoint{0.100000in}{0.100000in}}{\pgfqpoint{3.608454in}{2.310000in}}%
\pgfusepath{clip}%
\pgfsetbuttcap%
\pgfsetmiterjoin%
\definecolor{currentfill}{rgb}{0.000000,0.509804,0.745098}%
\pgfsetfillcolor{currentfill}%
\pgfsetlinewidth{0.000000pt}%
\definecolor{currentstroke}{rgb}{0.000000,0.000000,0.000000}%
\pgfsetstrokecolor{currentstroke}%
\pgfsetstrokeopacity{0.000000}%
\pgfsetdash{}{0pt}%
\pgfpathmoveto{\pgfqpoint{1.757470in}{0.722100in}}%
\pgfpathlineto{\pgfqpoint{1.755704in}{0.692007in}}%
\pgfpathlineto{\pgfqpoint{1.697817in}{0.695195in}}%
\pgfpathlineto{\pgfqpoint{1.699678in}{0.725186in}}%
\pgfpathlineto{\pgfqpoint{1.678348in}{0.726376in}}%
\pgfpathlineto{\pgfqpoint{1.681284in}{0.762092in}}%
\pgfpathlineto{\pgfqpoint{1.684601in}{0.806456in}}%
\pgfpathlineto{\pgfqpoint{1.690000in}{0.806092in}}%
\pgfpathlineto{\pgfqpoint{1.724900in}{0.803670in}}%
\pgfpathlineto{\pgfqpoint{1.753852in}{0.801689in}}%
\pgfpathlineto{\pgfqpoint{1.751590in}{0.761847in}}%
\pgfpathlineto{\pgfqpoint{1.759948in}{0.761279in}}%
\pgfpathlineto{\pgfqpoint{1.757470in}{0.722100in}}%
\pgfpathclose%
\pgfusepath{fill}%
\end{pgfscope}%
\begin{pgfscope}%
\pgfpathrectangle{\pgfqpoint{0.100000in}{0.100000in}}{\pgfqpoint{3.608454in}{2.310000in}}%
\pgfusepath{clip}%
\pgfsetbuttcap%
\pgfsetmiterjoin%
\definecolor{currentfill}{rgb}{0.000000,0.505882,0.747059}%
\pgfsetfillcolor{currentfill}%
\pgfsetlinewidth{0.000000pt}%
\definecolor{currentstroke}{rgb}{0.000000,0.000000,0.000000}%
\pgfsetstrokecolor{currentstroke}%
\pgfsetstrokeopacity{0.000000}%
\pgfsetdash{}{0pt}%
\pgfpathmoveto{\pgfqpoint{2.839244in}{0.999162in}}%
\pgfpathlineto{\pgfqpoint{2.836248in}{0.996454in}}%
\pgfpathlineto{\pgfqpoint{2.827384in}{1.003120in}}%
\pgfpathlineto{\pgfqpoint{2.830646in}{1.010243in}}%
\pgfpathlineto{\pgfqpoint{2.825899in}{1.020792in}}%
\pgfpathlineto{\pgfqpoint{2.818142in}{1.023149in}}%
\pgfpathlineto{\pgfqpoint{2.810860in}{1.028741in}}%
\pgfpathlineto{\pgfqpoint{2.810594in}{1.033632in}}%
\pgfpathlineto{\pgfqpoint{2.812171in}{1.037511in}}%
\pgfpathlineto{\pgfqpoint{2.820073in}{1.037913in}}%
\pgfpathlineto{\pgfqpoint{2.825426in}{1.046467in}}%
\pgfpathlineto{\pgfqpoint{2.830948in}{1.045651in}}%
\pgfpathlineto{\pgfqpoint{2.833248in}{1.051039in}}%
\pgfpathlineto{\pgfqpoint{2.843160in}{1.055221in}}%
\pgfpathlineto{\pgfqpoint{2.847354in}{1.055558in}}%
\pgfpathlineto{\pgfqpoint{2.850394in}{1.050302in}}%
\pgfpathlineto{\pgfqpoint{2.863588in}{1.049271in}}%
\pgfpathlineto{\pgfqpoint{2.864794in}{1.047108in}}%
\pgfpathlineto{\pgfqpoint{2.863172in}{1.045964in}}%
\pgfpathlineto{\pgfqpoint{2.859003in}{1.029918in}}%
\pgfpathlineto{\pgfqpoint{2.864516in}{1.026478in}}%
\pgfpathlineto{\pgfqpoint{2.864861in}{1.018162in}}%
\pgfpathlineto{\pgfqpoint{2.868390in}{1.010555in}}%
\pgfpathlineto{\pgfqpoint{2.865578in}{1.008074in}}%
\pgfpathlineto{\pgfqpoint{2.869364in}{1.002057in}}%
\pgfpathlineto{\pgfqpoint{2.868081in}{0.996615in}}%
\pgfpathlineto{\pgfqpoint{2.859121in}{0.988728in}}%
\pgfpathlineto{\pgfqpoint{2.856780in}{0.993705in}}%
\pgfpathlineto{\pgfqpoint{2.843269in}{0.994686in}}%
\pgfpathlineto{\pgfqpoint{2.839244in}{0.999162in}}%
\pgfpathclose%
\pgfusepath{fill}%
\end{pgfscope}%
\begin{pgfscope}%
\pgfpathrectangle{\pgfqpoint{0.100000in}{0.100000in}}{\pgfqpoint{3.608454in}{2.310000in}}%
\pgfusepath{clip}%
\pgfsetbuttcap%
\pgfsetmiterjoin%
\definecolor{currentfill}{rgb}{0.000000,0.545098,0.727451}%
\pgfsetfillcolor{currentfill}%
\pgfsetlinewidth{0.000000pt}%
\definecolor{currentstroke}{rgb}{0.000000,0.000000,0.000000}%
\pgfsetstrokecolor{currentstroke}%
\pgfsetstrokeopacity{0.000000}%
\pgfsetdash{}{0pt}%
\pgfpathmoveto{\pgfqpoint{2.075326in}{1.295024in}}%
\pgfpathlineto{\pgfqpoint{2.047953in}{1.295247in}}%
\pgfpathlineto{\pgfqpoint{2.048478in}{1.319317in}}%
\pgfpathlineto{\pgfqpoint{2.023407in}{1.319659in}}%
\pgfpathlineto{\pgfqpoint{2.021173in}{1.326536in}}%
\pgfpathlineto{\pgfqpoint{2.014334in}{1.326679in}}%
\pgfpathlineto{\pgfqpoint{2.014646in}{1.343860in}}%
\pgfpathlineto{\pgfqpoint{2.021423in}{1.343723in}}%
\pgfpathlineto{\pgfqpoint{2.021694in}{1.353951in}}%
\pgfpathlineto{\pgfqpoint{2.030596in}{1.356610in}}%
\pgfpathlineto{\pgfqpoint{2.034708in}{1.355023in}}%
\pgfpathlineto{\pgfqpoint{2.043381in}{1.357123in}}%
\pgfpathlineto{\pgfqpoint{2.043672in}{1.384754in}}%
\pgfpathlineto{\pgfqpoint{2.058697in}{1.384598in}}%
\pgfpathlineto{\pgfqpoint{2.058778in}{1.391470in}}%
\pgfpathlineto{\pgfqpoint{2.072459in}{1.391367in}}%
\pgfpathlineto{\pgfqpoint{2.071958in}{1.372885in}}%
\pgfpathlineto{\pgfqpoint{2.095818in}{1.372775in}}%
\pgfpathlineto{\pgfqpoint{2.095756in}{1.349849in}}%
\pgfpathlineto{\pgfqpoint{2.096838in}{1.338167in}}%
\pgfpathlineto{\pgfqpoint{2.103336in}{1.338237in}}%
\pgfpathlineto{\pgfqpoint{2.103312in}{1.318980in}}%
\pgfpathlineto{\pgfqpoint{2.075903in}{1.319066in}}%
\pgfpathlineto{\pgfqpoint{2.075326in}{1.295024in}}%
\pgfpathclose%
\pgfusepath{fill}%
\end{pgfscope}%
\begin{pgfscope}%
\pgfpathrectangle{\pgfqpoint{0.100000in}{0.100000in}}{\pgfqpoint{3.608454in}{2.310000in}}%
\pgfusepath{clip}%
\pgfsetbuttcap%
\pgfsetmiterjoin%
\definecolor{currentfill}{rgb}{0.000000,0.478431,0.760784}%
\pgfsetfillcolor{currentfill}%
\pgfsetlinewidth{0.000000pt}%
\definecolor{currentstroke}{rgb}{0.000000,0.000000,0.000000}%
\pgfsetstrokecolor{currentstroke}%
\pgfsetstrokeopacity{0.000000}%
\pgfsetdash{}{0pt}%
\pgfpathmoveto{\pgfqpoint{2.024967in}{0.596552in}}%
\pgfpathlineto{\pgfqpoint{2.015636in}{0.587127in}}%
\pgfpathlineto{\pgfqpoint{2.012654in}{0.578657in}}%
\pgfpathlineto{\pgfqpoint{1.992065in}{0.566132in}}%
\pgfpathlineto{\pgfqpoint{1.997763in}{0.573057in}}%
\pgfpathlineto{\pgfqpoint{1.981908in}{0.592727in}}%
\pgfpathlineto{\pgfqpoint{1.976342in}{0.597283in}}%
\pgfpathlineto{\pgfqpoint{1.957611in}{0.597327in}}%
\pgfpathlineto{\pgfqpoint{1.945778in}{0.610158in}}%
\pgfpathlineto{\pgfqpoint{1.966633in}{0.631014in}}%
\pgfpathlineto{\pgfqpoint{1.979102in}{0.635917in}}%
\pgfpathlineto{\pgfqpoint{1.982721in}{0.639405in}}%
\pgfpathlineto{\pgfqpoint{1.990391in}{0.638498in}}%
\pgfpathlineto{\pgfqpoint{2.000273in}{0.618823in}}%
\pgfpathlineto{\pgfqpoint{2.008648in}{0.612091in}}%
\pgfpathlineto{\pgfqpoint{2.014382in}{0.611205in}}%
\pgfpathlineto{\pgfqpoint{2.019290in}{0.599307in}}%
\pgfpathlineto{\pgfqpoint{2.024967in}{0.596552in}}%
\pgfpathclose%
\pgfusepath{fill}%
\end{pgfscope}%
\begin{pgfscope}%
\pgfpathrectangle{\pgfqpoint{0.100000in}{0.100000in}}{\pgfqpoint{3.608454in}{2.310000in}}%
\pgfusepath{clip}%
\pgfsetbuttcap%
\pgfsetmiterjoin%
\definecolor{currentfill}{rgb}{0.000000,0.647059,0.676471}%
\pgfsetfillcolor{currentfill}%
\pgfsetlinewidth{0.000000pt}%
\definecolor{currentstroke}{rgb}{0.000000,0.000000,0.000000}%
\pgfsetstrokecolor{currentstroke}%
\pgfsetstrokeopacity{0.000000}%
\pgfsetdash{}{0pt}%
\pgfpathmoveto{\pgfqpoint{0.755783in}{1.587192in}}%
\pgfpathlineto{\pgfqpoint{0.693366in}{1.603165in}}%
\pgfpathlineto{\pgfqpoint{0.650489in}{1.614771in}}%
\pgfpathlineto{\pgfqpoint{0.661398in}{1.654341in}}%
\pgfpathlineto{\pgfqpoint{0.662884in}{1.653976in}}%
\pgfpathlineto{\pgfqpoint{0.672323in}{1.687633in}}%
\pgfpathlineto{\pgfqpoint{0.671164in}{1.687956in}}%
\pgfpathlineto{\pgfqpoint{0.681032in}{1.722500in}}%
\pgfpathlineto{\pgfqpoint{0.693113in}{1.766738in}}%
\pgfpathlineto{\pgfqpoint{0.728707in}{1.756911in}}%
\pgfpathlineto{\pgfqpoint{0.757203in}{1.749818in}}%
\pgfpathlineto{\pgfqpoint{0.824016in}{1.733109in}}%
\pgfpathlineto{\pgfqpoint{0.824452in}{1.732928in}}%
\pgfpathlineto{\pgfqpoint{0.805480in}{1.656197in}}%
\pgfpathlineto{\pgfqpoint{0.798688in}{1.628778in}}%
\pgfpathlineto{\pgfqpoint{0.785394in}{1.632033in}}%
\pgfpathlineto{\pgfqpoint{0.755783in}{1.587192in}}%
\pgfpathclose%
\pgfusepath{fill}%
\end{pgfscope}%
\begin{pgfscope}%
\pgfpathrectangle{\pgfqpoint{0.100000in}{0.100000in}}{\pgfqpoint{3.608454in}{2.310000in}}%
\pgfusepath{clip}%
\pgfsetbuttcap%
\pgfsetmiterjoin%
\definecolor{currentfill}{rgb}{0.000000,0.329412,0.835294}%
\pgfsetfillcolor{currentfill}%
\pgfsetlinewidth{0.000000pt}%
\definecolor{currentstroke}{rgb}{0.000000,0.000000,0.000000}%
\pgfsetstrokecolor{currentstroke}%
\pgfsetstrokeopacity{0.000000}%
\pgfsetdash{}{0pt}%
\pgfpathmoveto{\pgfqpoint{1.989587in}{1.638444in}}%
\pgfpathlineto{\pgfqpoint{1.988757in}{1.605977in}}%
\pgfpathlineto{\pgfqpoint{1.968280in}{1.606524in}}%
\pgfpathlineto{\pgfqpoint{1.968471in}{1.613417in}}%
\pgfpathlineto{\pgfqpoint{1.914002in}{1.614954in}}%
\pgfpathlineto{\pgfqpoint{1.914725in}{1.650247in}}%
\pgfpathlineto{\pgfqpoint{1.922640in}{1.646360in}}%
\pgfpathlineto{\pgfqpoint{1.925767in}{1.649254in}}%
\pgfpathlineto{\pgfqpoint{1.927556in}{1.657034in}}%
\pgfpathlineto{\pgfqpoint{1.926808in}{1.664949in}}%
\pgfpathlineto{\pgfqpoint{1.928813in}{1.672403in}}%
\pgfpathlineto{\pgfqpoint{1.968146in}{1.671305in}}%
\pgfpathlineto{\pgfqpoint{1.981953in}{1.671006in}}%
\pgfpathlineto{\pgfqpoint{1.981253in}{1.641675in}}%
\pgfpathlineto{\pgfqpoint{1.989587in}{1.638444in}}%
\pgfpathclose%
\pgfusepath{fill}%
\end{pgfscope}%
\begin{pgfscope}%
\pgfpathrectangle{\pgfqpoint{0.100000in}{0.100000in}}{\pgfqpoint{3.608454in}{2.310000in}}%
\pgfusepath{clip}%
\pgfsetbuttcap%
\pgfsetmiterjoin%
\definecolor{currentfill}{rgb}{0.000000,0.545098,0.727451}%
\pgfsetfillcolor{currentfill}%
\pgfsetlinewidth{0.000000pt}%
\definecolor{currentstroke}{rgb}{0.000000,0.000000,0.000000}%
\pgfsetstrokecolor{currentstroke}%
\pgfsetstrokeopacity{0.000000}%
\pgfsetdash{}{0pt}%
\pgfpathmoveto{\pgfqpoint{3.191049in}{1.170900in}}%
\pgfpathlineto{\pgfqpoint{3.206484in}{1.172807in}}%
\pgfpathlineto{\pgfqpoint{3.211683in}{1.141958in}}%
\pgfpathlineto{\pgfqpoint{3.216293in}{1.140248in}}%
\pgfpathlineto{\pgfqpoint{3.219619in}{1.130949in}}%
\pgfpathlineto{\pgfqpoint{3.227150in}{1.124441in}}%
\pgfpathlineto{\pgfqpoint{3.228319in}{1.108263in}}%
\pgfpathlineto{\pgfqpoint{3.200611in}{1.103096in}}%
\pgfpathlineto{\pgfqpoint{3.194034in}{1.089009in}}%
\pgfpathlineto{\pgfqpoint{3.182406in}{1.101945in}}%
\pgfpathlineto{\pgfqpoint{3.174249in}{1.109020in}}%
\pgfpathlineto{\pgfqpoint{3.162834in}{1.117655in}}%
\pgfpathlineto{\pgfqpoint{3.160679in}{1.126465in}}%
\pgfpathlineto{\pgfqpoint{3.160885in}{1.137946in}}%
\pgfpathlineto{\pgfqpoint{3.164716in}{1.144245in}}%
\pgfpathlineto{\pgfqpoint{3.168634in}{1.140787in}}%
\pgfpathlineto{\pgfqpoint{3.173892in}{1.141190in}}%
\pgfpathlineto{\pgfqpoint{3.180850in}{1.147206in}}%
\pgfpathlineto{\pgfqpoint{3.188860in}{1.151551in}}%
\pgfpathlineto{\pgfqpoint{3.187783in}{1.156766in}}%
\pgfpathlineto{\pgfqpoint{3.191049in}{1.170900in}}%
\pgfpathclose%
\pgfusepath{fill}%
\end{pgfscope}%
\begin{pgfscope}%
\pgfpathrectangle{\pgfqpoint{0.100000in}{0.100000in}}{\pgfqpoint{3.608454in}{2.310000in}}%
\pgfusepath{clip}%
\pgfsetbuttcap%
\pgfsetmiterjoin%
\definecolor{currentfill}{rgb}{0.000000,0.576471,0.711765}%
\pgfsetfillcolor{currentfill}%
\pgfsetlinewidth{0.000000pt}%
\definecolor{currentstroke}{rgb}{0.000000,0.000000,0.000000}%
\pgfsetstrokecolor{currentstroke}%
\pgfsetstrokeopacity{0.000000}%
\pgfsetdash{}{0pt}%
\pgfpathmoveto{\pgfqpoint{2.786852in}{0.748618in}}%
\pgfpathlineto{\pgfqpoint{2.789288in}{0.742878in}}%
\pgfpathlineto{\pgfqpoint{2.719091in}{0.734938in}}%
\pgfpathlineto{\pgfqpoint{2.708723in}{0.733977in}}%
\pgfpathlineto{\pgfqpoint{2.704075in}{0.776471in}}%
\pgfpathlineto{\pgfqpoint{2.706700in}{0.783758in}}%
\pgfpathlineto{\pgfqpoint{2.733549in}{0.786465in}}%
\pgfpathlineto{\pgfqpoint{2.755934in}{0.788907in}}%
\pgfpathlineto{\pgfqpoint{2.755229in}{0.795820in}}%
\pgfpathlineto{\pgfqpoint{2.774910in}{0.796997in}}%
\pgfpathlineto{\pgfqpoint{2.780161in}{0.791452in}}%
\pgfpathlineto{\pgfqpoint{2.781911in}{0.785540in}}%
\pgfpathlineto{\pgfqpoint{2.779513in}{0.764362in}}%
\pgfpathlineto{\pgfqpoint{2.781206in}{0.755220in}}%
\pgfpathlineto{\pgfqpoint{2.786852in}{0.748618in}}%
\pgfpathclose%
\pgfusepath{fill}%
\end{pgfscope}%
\begin{pgfscope}%
\pgfpathrectangle{\pgfqpoint{0.100000in}{0.100000in}}{\pgfqpoint{3.608454in}{2.310000in}}%
\pgfusepath{clip}%
\pgfsetbuttcap%
\pgfsetmiterjoin%
\definecolor{currentfill}{rgb}{0.000000,0.654902,0.672549}%
\pgfsetfillcolor{currentfill}%
\pgfsetlinewidth{0.000000pt}%
\definecolor{currentstroke}{rgb}{0.000000,0.000000,0.000000}%
\pgfsetstrokecolor{currentstroke}%
\pgfsetstrokeopacity{0.000000}%
\pgfsetdash{}{0pt}%
\pgfpathmoveto{\pgfqpoint{0.875234in}{2.151233in}}%
\pgfpathlineto{\pgfqpoint{0.821826in}{2.165426in}}%
\pgfpathlineto{\pgfqpoint{0.836363in}{2.217995in}}%
\pgfpathlineto{\pgfqpoint{0.843741in}{2.216808in}}%
\pgfpathlineto{\pgfqpoint{0.857008in}{2.211536in}}%
\pgfpathlineto{\pgfqpoint{0.862221in}{2.202501in}}%
\pgfpathlineto{\pgfqpoint{0.873890in}{2.207982in}}%
\pgfpathlineto{\pgfqpoint{0.878250in}{2.201957in}}%
\pgfpathlineto{\pgfqpoint{0.879920in}{2.194564in}}%
\pgfpathlineto{\pgfqpoint{0.891121in}{2.195618in}}%
\pgfpathlineto{\pgfqpoint{0.894472in}{2.191910in}}%
\pgfpathlineto{\pgfqpoint{0.902377in}{2.193011in}}%
\pgfpathlineto{\pgfqpoint{0.908021in}{2.186968in}}%
\pgfpathlineto{\pgfqpoint{0.913019in}{2.206570in}}%
\pgfpathlineto{\pgfqpoint{0.916424in}{2.212894in}}%
\pgfpathlineto{\pgfqpoint{0.926800in}{2.253370in}}%
\pgfpathlineto{\pgfqpoint{0.920139in}{2.255027in}}%
\pgfpathlineto{\pgfqpoint{0.925222in}{2.260906in}}%
\pgfpathlineto{\pgfqpoint{0.927075in}{2.268237in}}%
\pgfpathlineto{\pgfqpoint{0.937150in}{2.279229in}}%
\pgfpathlineto{\pgfqpoint{0.957430in}{2.274082in}}%
\pgfpathlineto{\pgfqpoint{0.954484in}{2.262189in}}%
\pgfpathlineto{\pgfqpoint{0.967184in}{2.259137in}}%
\pgfpathlineto{\pgfqpoint{0.960593in}{2.232110in}}%
\pgfpathlineto{\pgfqpoint{0.998620in}{2.223097in}}%
\pgfpathlineto{\pgfqpoint{0.993310in}{2.200722in}}%
\pgfpathlineto{\pgfqpoint{0.988985in}{2.182170in}}%
\pgfpathlineto{\pgfqpoint{0.996594in}{2.161889in}}%
\pgfpathlineto{\pgfqpoint{1.000904in}{2.156205in}}%
\pgfpathlineto{\pgfqpoint{1.000929in}{2.147949in}}%
\pgfpathlineto{\pgfqpoint{0.996502in}{2.145181in}}%
\pgfpathlineto{\pgfqpoint{1.001953in}{2.138275in}}%
\pgfpathlineto{\pgfqpoint{0.994972in}{2.136779in}}%
\pgfpathlineto{\pgfqpoint{1.003713in}{2.127225in}}%
\pgfpathlineto{\pgfqpoint{1.003956in}{2.123021in}}%
\pgfpathlineto{\pgfqpoint{1.014258in}{2.117180in}}%
\pgfpathlineto{\pgfqpoint{1.023030in}{2.093793in}}%
\pgfpathlineto{\pgfqpoint{1.027701in}{2.087646in}}%
\pgfpathlineto{\pgfqpoint{0.993104in}{2.095521in}}%
\pgfpathlineto{\pgfqpoint{0.954967in}{2.104591in}}%
\pgfpathlineto{\pgfqpoint{0.956600in}{2.111418in}}%
\pgfpathlineto{\pgfqpoint{0.950036in}{2.114158in}}%
\pgfpathlineto{\pgfqpoint{0.933521in}{2.118231in}}%
\pgfpathlineto{\pgfqpoint{0.921056in}{2.128678in}}%
\pgfpathlineto{\pgfqpoint{0.923573in}{2.138917in}}%
\pgfpathlineto{\pgfqpoint{0.875234in}{2.151233in}}%
\pgfpathclose%
\pgfusepath{fill}%
\end{pgfscope}%
\begin{pgfscope}%
\pgfpathrectangle{\pgfqpoint{0.100000in}{0.100000in}}{\pgfqpoint{3.608454in}{2.310000in}}%
\pgfusepath{clip}%
\pgfsetbuttcap%
\pgfsetmiterjoin%
\definecolor{currentfill}{rgb}{0.000000,0.682353,0.658824}%
\pgfsetfillcolor{currentfill}%
\pgfsetlinewidth{0.000000pt}%
\definecolor{currentstroke}{rgb}{0.000000,0.000000,0.000000}%
\pgfsetstrokecolor{currentstroke}%
\pgfsetstrokeopacity{0.000000}%
\pgfsetdash{}{0pt}%
\pgfpathmoveto{\pgfqpoint{2.187595in}{0.601129in}}%
\pgfpathlineto{\pgfqpoint{2.169818in}{0.598417in}}%
\pgfpathlineto{\pgfqpoint{2.151781in}{0.590398in}}%
\pgfpathlineto{\pgfqpoint{2.151337in}{0.616492in}}%
\pgfpathlineto{\pgfqpoint{2.145490in}{0.616619in}}%
\pgfpathlineto{\pgfqpoint{2.145181in}{0.634496in}}%
\pgfpathlineto{\pgfqpoint{2.134719in}{0.634369in}}%
\pgfpathlineto{\pgfqpoint{2.125194in}{0.664549in}}%
\pgfpathlineto{\pgfqpoint{2.109149in}{0.671435in}}%
\pgfpathlineto{\pgfqpoint{2.105170in}{0.678118in}}%
\pgfpathlineto{\pgfqpoint{2.098891in}{0.678184in}}%
\pgfpathlineto{\pgfqpoint{2.097748in}{0.687767in}}%
\pgfpathlineto{\pgfqpoint{2.093068in}{0.691255in}}%
\pgfpathlineto{\pgfqpoint{2.110954in}{0.708601in}}%
\pgfpathlineto{\pgfqpoint{2.117525in}{0.716961in}}%
\pgfpathlineto{\pgfqpoint{2.123775in}{0.713712in}}%
\pgfpathlineto{\pgfqpoint{2.132588in}{0.712274in}}%
\pgfpathlineto{\pgfqpoint{2.143868in}{0.708025in}}%
\pgfpathlineto{\pgfqpoint{2.166253in}{0.713450in}}%
\pgfpathlineto{\pgfqpoint{2.181123in}{0.718301in}}%
\pgfpathlineto{\pgfqpoint{2.202317in}{0.720035in}}%
\pgfpathlineto{\pgfqpoint{2.205558in}{0.720837in}}%
\pgfpathlineto{\pgfqpoint{2.204984in}{0.705199in}}%
\pgfpathlineto{\pgfqpoint{2.207696in}{0.700477in}}%
\pgfpathlineto{\pgfqpoint{2.204721in}{0.696052in}}%
\pgfpathlineto{\pgfqpoint{2.200790in}{0.680240in}}%
\pgfpathlineto{\pgfqpoint{2.197286in}{0.677217in}}%
\pgfpathlineto{\pgfqpoint{2.193927in}{0.668423in}}%
\pgfpathlineto{\pgfqpoint{2.196505in}{0.661211in}}%
\pgfpathlineto{\pgfqpoint{2.191936in}{0.652528in}}%
\pgfpathlineto{\pgfqpoint{2.196201in}{0.649140in}}%
\pgfpathlineto{\pgfqpoint{2.196138in}{0.630275in}}%
\pgfpathlineto{\pgfqpoint{2.190675in}{0.624923in}}%
\pgfpathlineto{\pgfqpoint{2.188702in}{0.619056in}}%
\pgfpathlineto{\pgfqpoint{2.181395in}{0.610955in}}%
\pgfpathlineto{\pgfqpoint{2.187595in}{0.601129in}}%
\pgfpathclose%
\pgfusepath{fill}%
\end{pgfscope}%
\begin{pgfscope}%
\pgfpathrectangle{\pgfqpoint{0.100000in}{0.100000in}}{\pgfqpoint{3.608454in}{2.310000in}}%
\pgfusepath{clip}%
\pgfsetbuttcap%
\pgfsetmiterjoin%
\definecolor{currentfill}{rgb}{0.000000,0.301961,0.849020}%
\pgfsetfillcolor{currentfill}%
\pgfsetlinewidth{0.000000pt}%
\definecolor{currentstroke}{rgb}{0.000000,0.000000,0.000000}%
\pgfsetstrokecolor{currentstroke}%
\pgfsetstrokeopacity{0.000000}%
\pgfsetdash{}{0pt}%
\pgfpathmoveto{\pgfqpoint{3.217018in}{1.367687in}}%
\pgfpathlineto{\pgfqpoint{3.210448in}{1.365277in}}%
\pgfpathlineto{\pgfqpoint{3.202566in}{1.350706in}}%
\pgfpathlineto{\pgfqpoint{3.197288in}{1.352635in}}%
\pgfpathlineto{\pgfqpoint{3.195177in}{1.357724in}}%
\pgfpathlineto{\pgfqpoint{3.178627in}{1.363629in}}%
\pgfpathlineto{\pgfqpoint{3.168135in}{1.369114in}}%
\pgfpathlineto{\pgfqpoint{3.161085in}{1.369292in}}%
\pgfpathlineto{\pgfqpoint{3.175771in}{1.395430in}}%
\pgfpathlineto{\pgfqpoint{3.175115in}{1.399614in}}%
\pgfpathlineto{\pgfqpoint{3.182634in}{1.411460in}}%
\pgfpathlineto{\pgfqpoint{3.194258in}{1.405953in}}%
\pgfpathlineto{\pgfqpoint{3.196379in}{1.396687in}}%
\pgfpathlineto{\pgfqpoint{3.199440in}{1.393885in}}%
\pgfpathlineto{\pgfqpoint{3.209863in}{1.398797in}}%
\pgfpathlineto{\pgfqpoint{3.214679in}{1.396873in}}%
\pgfpathlineto{\pgfqpoint{3.216571in}{1.392855in}}%
\pgfpathlineto{\pgfqpoint{3.214965in}{1.384823in}}%
\pgfpathlineto{\pgfqpoint{3.209188in}{1.375295in}}%
\pgfpathlineto{\pgfqpoint{3.217018in}{1.367687in}}%
\pgfpathclose%
\pgfusepath{fill}%
\end{pgfscope}%
\begin{pgfscope}%
\pgfpathrectangle{\pgfqpoint{0.100000in}{0.100000in}}{\pgfqpoint{3.608454in}{2.310000in}}%
\pgfusepath{clip}%
\pgfsetbuttcap%
\pgfsetmiterjoin%
\definecolor{currentfill}{rgb}{0.000000,0.333333,0.833333}%
\pgfsetfillcolor{currentfill}%
\pgfsetlinewidth{0.000000pt}%
\definecolor{currentstroke}{rgb}{0.000000,0.000000,0.000000}%
\pgfsetstrokecolor{currentstroke}%
\pgfsetstrokeopacity{0.000000}%
\pgfsetdash{}{0pt}%
\pgfpathmoveto{\pgfqpoint{1.821457in}{1.218805in}}%
\pgfpathlineto{\pgfqpoint{1.820080in}{1.188664in}}%
\pgfpathlineto{\pgfqpoint{1.790983in}{1.190359in}}%
\pgfpathlineto{\pgfqpoint{1.785504in}{1.190712in}}%
\pgfpathlineto{\pgfqpoint{1.787082in}{1.221084in}}%
\pgfpathlineto{\pgfqpoint{1.786488in}{1.228084in}}%
\pgfpathlineto{\pgfqpoint{1.752303in}{1.230115in}}%
\pgfpathlineto{\pgfqpoint{1.754126in}{1.258033in}}%
\pgfpathlineto{\pgfqpoint{1.754193in}{1.271800in}}%
\pgfpathlineto{\pgfqpoint{1.781396in}{1.270043in}}%
\pgfpathlineto{\pgfqpoint{1.782549in}{1.290657in}}%
\pgfpathlineto{\pgfqpoint{1.822395in}{1.288502in}}%
\pgfpathlineto{\pgfqpoint{1.823241in}{1.288458in}}%
\pgfpathlineto{\pgfqpoint{1.821521in}{1.253978in}}%
\pgfpathlineto{\pgfqpoint{1.821831in}{1.246792in}}%
\pgfpathlineto{\pgfqpoint{1.821457in}{1.218805in}}%
\pgfpathclose%
\pgfusepath{fill}%
\end{pgfscope}%
\begin{pgfscope}%
\pgfpathrectangle{\pgfqpoint{0.100000in}{0.100000in}}{\pgfqpoint{3.608454in}{2.310000in}}%
\pgfusepath{clip}%
\pgfsetbuttcap%
\pgfsetmiterjoin%
\definecolor{currentfill}{rgb}{0.000000,0.674510,0.662745}%
\pgfsetfillcolor{currentfill}%
\pgfsetlinewidth{0.000000pt}%
\definecolor{currentstroke}{rgb}{0.000000,0.000000,0.000000}%
\pgfsetstrokecolor{currentstroke}%
\pgfsetstrokeopacity{0.000000}%
\pgfsetdash{}{0pt}%
\pgfpathmoveto{\pgfqpoint{1.759459in}{0.801389in}}%
\pgfpathlineto{\pgfqpoint{1.753852in}{0.801689in}}%
\pgfpathlineto{\pgfqpoint{1.724900in}{0.803670in}}%
\pgfpathlineto{\pgfqpoint{1.729883in}{0.873343in}}%
\pgfpathlineto{\pgfqpoint{1.738977in}{0.872740in}}%
\pgfpathlineto{\pgfqpoint{1.763926in}{0.871040in}}%
\pgfpathlineto{\pgfqpoint{1.759459in}{0.801389in}}%
\pgfpathclose%
\pgfusepath{fill}%
\end{pgfscope}%
\begin{pgfscope}%
\pgfpathrectangle{\pgfqpoint{0.100000in}{0.100000in}}{\pgfqpoint{3.608454in}{2.310000in}}%
\pgfusepath{clip}%
\pgfsetbuttcap%
\pgfsetmiterjoin%
\definecolor{currentfill}{rgb}{0.000000,0.333333,0.833333}%
\pgfsetfillcolor{currentfill}%
\pgfsetlinewidth{0.000000pt}%
\definecolor{currentstroke}{rgb}{0.000000,0.000000,0.000000}%
\pgfsetstrokecolor{currentstroke}%
\pgfsetstrokeopacity{0.000000}%
\pgfsetdash{}{0pt}%
\pgfpathmoveto{\pgfqpoint{1.752303in}{1.230115in}}%
\pgfpathlineto{\pgfqpoint{1.751890in}{1.223219in}}%
\pgfpathlineto{\pgfqpoint{1.725856in}{1.224921in}}%
\pgfpathlineto{\pgfqpoint{1.724493in}{1.225005in}}%
\pgfpathlineto{\pgfqpoint{1.726337in}{1.252520in}}%
\pgfpathlineto{\pgfqpoint{1.666954in}{1.256981in}}%
\pgfpathlineto{\pgfqpoint{1.670021in}{1.298287in}}%
\pgfpathlineto{\pgfqpoint{1.699539in}{1.296124in}}%
\pgfpathlineto{\pgfqpoint{1.754213in}{1.292521in}}%
\pgfpathlineto{\pgfqpoint{1.782549in}{1.290657in}}%
\pgfpathlineto{\pgfqpoint{1.781396in}{1.270043in}}%
\pgfpathlineto{\pgfqpoint{1.754193in}{1.271800in}}%
\pgfpathlineto{\pgfqpoint{1.754126in}{1.258033in}}%
\pgfpathlineto{\pgfqpoint{1.752303in}{1.230115in}}%
\pgfpathclose%
\pgfusepath{fill}%
\end{pgfscope}%
\begin{pgfscope}%
\pgfpathrectangle{\pgfqpoint{0.100000in}{0.100000in}}{\pgfqpoint{3.608454in}{2.310000in}}%
\pgfusepath{clip}%
\pgfsetbuttcap%
\pgfsetmiterjoin%
\definecolor{currentfill}{rgb}{0.000000,0.521569,0.739216}%
\pgfsetfillcolor{currentfill}%
\pgfsetlinewidth{0.000000pt}%
\definecolor{currentstroke}{rgb}{0.000000,0.000000,0.000000}%
\pgfsetstrokecolor{currentstroke}%
\pgfsetstrokeopacity{0.000000}%
\pgfsetdash{}{0pt}%
\pgfpathmoveto{\pgfqpoint{2.280744in}{1.286387in}}%
\pgfpathlineto{\pgfqpoint{2.314915in}{1.287249in}}%
\pgfpathlineto{\pgfqpoint{2.315298in}{1.276599in}}%
\pgfpathlineto{\pgfqpoint{2.322748in}{1.276780in}}%
\pgfpathlineto{\pgfqpoint{2.323611in}{1.248060in}}%
\pgfpathlineto{\pgfqpoint{2.337381in}{1.248460in}}%
\pgfpathlineto{\pgfqpoint{2.337514in}{1.241571in}}%
\pgfpathlineto{\pgfqpoint{2.347355in}{1.241765in}}%
\pgfpathlineto{\pgfqpoint{2.347612in}{1.233131in}}%
\pgfpathlineto{\pgfqpoint{2.358074in}{1.234831in}}%
\pgfpathlineto{\pgfqpoint{2.371740in}{1.235384in}}%
\pgfpathlineto{\pgfqpoint{2.372317in}{1.217256in}}%
\pgfpathlineto{\pgfqpoint{2.375052in}{1.209261in}}%
\pgfpathlineto{\pgfqpoint{2.375334in}{1.200990in}}%
\pgfpathlineto{\pgfqpoint{2.373090in}{1.191695in}}%
\pgfpathlineto{\pgfqpoint{2.361419in}{1.191422in}}%
\pgfpathlineto{\pgfqpoint{2.357912in}{1.194600in}}%
\pgfpathlineto{\pgfqpoint{2.357801in}{1.198383in}}%
\pgfpathlineto{\pgfqpoint{2.350740in}{1.202506in}}%
\pgfpathlineto{\pgfqpoint{2.347208in}{1.211310in}}%
\pgfpathlineto{\pgfqpoint{2.346916in}{1.219387in}}%
\pgfpathlineto{\pgfqpoint{2.310515in}{1.218707in}}%
\pgfpathlineto{\pgfqpoint{2.310037in}{1.232515in}}%
\pgfpathlineto{\pgfqpoint{2.279053in}{1.231943in}}%
\pgfpathlineto{\pgfqpoint{2.278985in}{1.235414in}}%
\pgfpathlineto{\pgfqpoint{2.268526in}{1.240287in}}%
\pgfpathlineto{\pgfqpoint{2.268476in}{1.251773in}}%
\pgfpathlineto{\pgfqpoint{2.268296in}{1.264580in}}%
\pgfpathlineto{\pgfqpoint{2.282058in}{1.264607in}}%
\pgfpathlineto{\pgfqpoint{2.280744in}{1.286387in}}%
\pgfpathclose%
\pgfusepath{fill}%
\end{pgfscope}%
\begin{pgfscope}%
\pgfpathrectangle{\pgfqpoint{0.100000in}{0.100000in}}{\pgfqpoint{3.608454in}{2.310000in}}%
\pgfusepath{clip}%
\pgfsetbuttcap%
\pgfsetmiterjoin%
\definecolor{currentfill}{rgb}{0.000000,0.682353,0.658824}%
\pgfsetfillcolor{currentfill}%
\pgfsetlinewidth{0.000000pt}%
\definecolor{currentstroke}{rgb}{0.000000,0.000000,0.000000}%
\pgfsetstrokecolor{currentstroke}%
\pgfsetstrokeopacity{0.000000}%
\pgfsetdash{}{0pt}%
\pgfpathmoveto{\pgfqpoint{2.831112in}{0.753917in}}%
\pgfpathlineto{\pgfqpoint{2.834431in}{0.723057in}}%
\pgfpathlineto{\pgfqpoint{2.801217in}{0.720980in}}%
\pgfpathlineto{\pgfqpoint{2.795289in}{0.729104in}}%
\pgfpathlineto{\pgfqpoint{2.794346in}{0.734429in}}%
\pgfpathlineto{\pgfqpoint{2.789288in}{0.742878in}}%
\pgfpathlineto{\pgfqpoint{2.786852in}{0.748618in}}%
\pgfpathlineto{\pgfqpoint{2.794089in}{0.749184in}}%
\pgfpathlineto{\pgfqpoint{2.792796in}{0.763762in}}%
\pgfpathlineto{\pgfqpoint{2.811462in}{0.766109in}}%
\pgfpathlineto{\pgfqpoint{2.818573in}{0.766684in}}%
\pgfpathlineto{\pgfqpoint{2.819817in}{0.752621in}}%
\pgfpathlineto{\pgfqpoint{2.831112in}{0.753917in}}%
\pgfpathclose%
\pgfusepath{fill}%
\end{pgfscope}%
\begin{pgfscope}%
\pgfpathrectangle{\pgfqpoint{0.100000in}{0.100000in}}{\pgfqpoint{3.608454in}{2.310000in}}%
\pgfusepath{clip}%
\pgfsetbuttcap%
\pgfsetmiterjoin%
\definecolor{currentfill}{rgb}{0.000000,0.431373,0.784314}%
\pgfsetfillcolor{currentfill}%
\pgfsetlinewidth{0.000000pt}%
\definecolor{currentstroke}{rgb}{0.000000,0.000000,0.000000}%
\pgfsetstrokecolor{currentstroke}%
\pgfsetstrokeopacity{0.000000}%
\pgfsetdash{}{0pt}%
\pgfpathmoveto{\pgfqpoint{2.024967in}{0.596552in}}%
\pgfpathlineto{\pgfqpoint{2.019290in}{0.599307in}}%
\pgfpathlineto{\pgfqpoint{2.014382in}{0.611205in}}%
\pgfpathlineto{\pgfqpoint{2.008648in}{0.612091in}}%
\pgfpathlineto{\pgfqpoint{2.000273in}{0.618823in}}%
\pgfpathlineto{\pgfqpoint{1.990391in}{0.638498in}}%
\pgfpathlineto{\pgfqpoint{1.982721in}{0.639405in}}%
\pgfpathlineto{\pgfqpoint{1.987116in}{0.645931in}}%
\pgfpathlineto{\pgfqpoint{1.993561in}{0.650091in}}%
\pgfpathlineto{\pgfqpoint{2.008653in}{0.653539in}}%
\pgfpathlineto{\pgfqpoint{2.016277in}{0.656724in}}%
\pgfpathlineto{\pgfqpoint{2.025563in}{0.657506in}}%
\pgfpathlineto{\pgfqpoint{2.031314in}{0.647827in}}%
\pgfpathlineto{\pgfqpoint{2.031567in}{0.644083in}}%
\pgfpathlineto{\pgfqpoint{2.051242in}{0.645234in}}%
\pgfpathlineto{\pgfqpoint{2.049731in}{0.675985in}}%
\pgfpathlineto{\pgfqpoint{2.065563in}{0.666170in}}%
\pgfpathlineto{\pgfqpoint{2.081886in}{0.666080in}}%
\pgfpathlineto{\pgfqpoint{2.084351in}{0.694073in}}%
\pgfpathlineto{\pgfqpoint{2.089584in}{0.697805in}}%
\pgfpathlineto{\pgfqpoint{2.093068in}{0.691255in}}%
\pgfpathlineto{\pgfqpoint{2.097748in}{0.687767in}}%
\pgfpathlineto{\pgfqpoint{2.098891in}{0.678184in}}%
\pgfpathlineto{\pgfqpoint{2.105170in}{0.678118in}}%
\pgfpathlineto{\pgfqpoint{2.109149in}{0.671435in}}%
\pgfpathlineto{\pgfqpoint{2.125194in}{0.664549in}}%
\pgfpathlineto{\pgfqpoint{2.134719in}{0.634369in}}%
\pgfpathlineto{\pgfqpoint{2.145181in}{0.634496in}}%
\pgfpathlineto{\pgfqpoint{2.145490in}{0.616619in}}%
\pgfpathlineto{\pgfqpoint{2.151337in}{0.616492in}}%
\pgfpathlineto{\pgfqpoint{2.151781in}{0.590398in}}%
\pgfpathlineto{\pgfqpoint{2.133973in}{0.582254in}}%
\pgfpathlineto{\pgfqpoint{2.137484in}{0.591301in}}%
\pgfpathlineto{\pgfqpoint{2.124852in}{0.587349in}}%
\pgfpathlineto{\pgfqpoint{2.128361in}{0.601241in}}%
\pgfpathlineto{\pgfqpoint{2.125042in}{0.608247in}}%
\pgfpathlineto{\pgfqpoint{2.119618in}{0.605904in}}%
\pgfpathlineto{\pgfqpoint{2.115649in}{0.599011in}}%
\pgfpathlineto{\pgfqpoint{2.109756in}{0.601205in}}%
\pgfpathlineto{\pgfqpoint{2.105684in}{0.596360in}}%
\pgfpathlineto{\pgfqpoint{2.105660in}{0.588738in}}%
\pgfpathlineto{\pgfqpoint{2.113115in}{0.585047in}}%
\pgfpathlineto{\pgfqpoint{2.114156in}{0.572287in}}%
\pgfpathlineto{\pgfqpoint{2.125821in}{0.572042in}}%
\pgfpathlineto{\pgfqpoint{2.098432in}{0.551471in}}%
\pgfpathlineto{\pgfqpoint{2.102863in}{0.561208in}}%
\pgfpathlineto{\pgfqpoint{2.090532in}{0.582574in}}%
\pgfpathlineto{\pgfqpoint{2.091703in}{0.589805in}}%
\pgfpathlineto{\pgfqpoint{2.088573in}{0.592795in}}%
\pgfpathlineto{\pgfqpoint{2.077251in}{0.591805in}}%
\pgfpathlineto{\pgfqpoint{2.074520in}{0.580523in}}%
\pgfpathlineto{\pgfqpoint{2.068450in}{0.580491in}}%
\pgfpathlineto{\pgfqpoint{2.066381in}{0.572643in}}%
\pgfpathlineto{\pgfqpoint{2.061080in}{0.568664in}}%
\pgfpathlineto{\pgfqpoint{2.054185in}{0.571463in}}%
\pgfpathlineto{\pgfqpoint{2.047518in}{0.566563in}}%
\pgfpathlineto{\pgfqpoint{2.040450in}{0.572451in}}%
\pgfpathlineto{\pgfqpoint{2.037581in}{0.580386in}}%
\pgfpathlineto{\pgfqpoint{2.032880in}{0.583631in}}%
\pgfpathlineto{\pgfqpoint{2.032870in}{0.593014in}}%
\pgfpathlineto{\pgfqpoint{2.024967in}{0.596552in}}%
\pgfpathclose%
\pgfusepath{fill}%
\end{pgfscope}%
\begin{pgfscope}%
\pgfpathrectangle{\pgfqpoint{0.100000in}{0.100000in}}{\pgfqpoint{3.608454in}{2.310000in}}%
\pgfusepath{clip}%
\pgfsetbuttcap%
\pgfsetmiterjoin%
\definecolor{currentfill}{rgb}{0.000000,0.439216,0.780392}%
\pgfsetfillcolor{currentfill}%
\pgfsetlinewidth{0.000000pt}%
\definecolor{currentstroke}{rgb}{0.000000,0.000000,0.000000}%
\pgfsetstrokecolor{currentstroke}%
\pgfsetstrokeopacity{0.000000}%
\pgfsetdash{}{0pt}%
\pgfpathmoveto{\pgfqpoint{1.938143in}{0.938336in}}%
\pgfpathlineto{\pgfqpoint{1.931087in}{0.945835in}}%
\pgfpathlineto{\pgfqpoint{1.917943in}{0.934949in}}%
\pgfpathlineto{\pgfqpoint{1.910664in}{0.938402in}}%
\pgfpathlineto{\pgfqpoint{1.913020in}{0.946294in}}%
\pgfpathlineto{\pgfqpoint{1.906317in}{0.946847in}}%
\pgfpathlineto{\pgfqpoint{1.898581in}{0.956614in}}%
\pgfpathlineto{\pgfqpoint{1.885921in}{0.960469in}}%
\pgfpathlineto{\pgfqpoint{1.883512in}{0.956032in}}%
\pgfpathlineto{\pgfqpoint{1.877716in}{0.953284in}}%
\pgfpathlineto{\pgfqpoint{1.869882in}{0.961093in}}%
\pgfpathlineto{\pgfqpoint{1.870481in}{0.975087in}}%
\pgfpathlineto{\pgfqpoint{1.867064in}{0.975207in}}%
\pgfpathlineto{\pgfqpoint{1.867617in}{0.988984in}}%
\pgfpathlineto{\pgfqpoint{1.856890in}{0.989419in}}%
\pgfpathlineto{\pgfqpoint{1.857172in}{0.996308in}}%
\pgfpathlineto{\pgfqpoint{1.858037in}{1.016994in}}%
\pgfpathlineto{\pgfqpoint{1.871452in}{1.016408in}}%
\pgfpathlineto{\pgfqpoint{1.905912in}{1.015133in}}%
\pgfpathlineto{\pgfqpoint{1.905475in}{1.001362in}}%
\pgfpathlineto{\pgfqpoint{1.932881in}{1.000512in}}%
\pgfpathlineto{\pgfqpoint{1.939766in}{1.000299in}}%
\pgfpathlineto{\pgfqpoint{1.938143in}{0.938336in}}%
\pgfpathclose%
\pgfusepath{fill}%
\end{pgfscope}%
\begin{pgfscope}%
\pgfpathrectangle{\pgfqpoint{0.100000in}{0.100000in}}{\pgfqpoint{3.608454in}{2.310000in}}%
\pgfusepath{clip}%
\pgfsetbuttcap%
\pgfsetmiterjoin%
\definecolor{currentfill}{rgb}{0.000000,0.498039,0.750980}%
\pgfsetfillcolor{currentfill}%
\pgfsetlinewidth{0.000000pt}%
\definecolor{currentstroke}{rgb}{0.000000,0.000000,0.000000}%
\pgfsetstrokecolor{currentstroke}%
\pgfsetstrokeopacity{0.000000}%
\pgfsetdash{}{0pt}%
\pgfpathmoveto{\pgfqpoint{3.358769in}{1.729605in}}%
\pgfpathlineto{\pgfqpoint{3.358373in}{1.723673in}}%
\pgfpathlineto{\pgfqpoint{3.347348in}{1.723269in}}%
\pgfpathlineto{\pgfqpoint{3.335015in}{1.726252in}}%
\pgfpathlineto{\pgfqpoint{3.333233in}{1.729959in}}%
\pgfpathlineto{\pgfqpoint{3.325396in}{1.731486in}}%
\pgfpathlineto{\pgfqpoint{3.324878in}{1.725364in}}%
\pgfpathlineto{\pgfqpoint{3.311248in}{1.723631in}}%
\pgfpathlineto{\pgfqpoint{3.302028in}{1.725960in}}%
\pgfpathlineto{\pgfqpoint{3.296564in}{1.727321in}}%
\pgfpathlineto{\pgfqpoint{3.300176in}{1.740144in}}%
\pgfpathlineto{\pgfqpoint{3.287975in}{1.743465in}}%
\pgfpathlineto{\pgfqpoint{3.281055in}{1.749476in}}%
\pgfpathlineto{\pgfqpoint{3.283745in}{1.758930in}}%
\pgfpathlineto{\pgfqpoint{3.279470in}{1.768032in}}%
\pgfpathlineto{\pgfqpoint{3.279173in}{1.774369in}}%
\pgfpathlineto{\pgfqpoint{3.291128in}{1.772508in}}%
\pgfpathlineto{\pgfqpoint{3.300308in}{1.775508in}}%
\pgfpathlineto{\pgfqpoint{3.309890in}{1.786988in}}%
\pgfpathlineto{\pgfqpoint{3.287566in}{1.849431in}}%
\pgfpathlineto{\pgfqpoint{3.296075in}{1.852707in}}%
\pgfpathlineto{\pgfqpoint{3.332649in}{1.866641in}}%
\pgfpathlineto{\pgfqpoint{3.333464in}{1.857292in}}%
\pgfpathlineto{\pgfqpoint{3.344700in}{1.848799in}}%
\pgfpathlineto{\pgfqpoint{3.348859in}{1.830296in}}%
\pgfpathlineto{\pgfqpoint{3.356483in}{1.793242in}}%
\pgfpathlineto{\pgfqpoint{3.358955in}{1.786375in}}%
\pgfpathlineto{\pgfqpoint{3.357447in}{1.732265in}}%
\pgfpathlineto{\pgfqpoint{3.358769in}{1.729605in}}%
\pgfpathclose%
\pgfusepath{fill}%
\end{pgfscope}%
\begin{pgfscope}%
\pgfpathrectangle{\pgfqpoint{0.100000in}{0.100000in}}{\pgfqpoint{3.608454in}{2.310000in}}%
\pgfusepath{clip}%
\pgfsetbuttcap%
\pgfsetmiterjoin%
\definecolor{currentfill}{rgb}{0.000000,0.466667,0.766667}%
\pgfsetfillcolor{currentfill}%
\pgfsetlinewidth{0.000000pt}%
\definecolor{currentstroke}{rgb}{0.000000,0.000000,0.000000}%
\pgfsetstrokecolor{currentstroke}%
\pgfsetstrokeopacity{0.000000}%
\pgfsetdash{}{0pt}%
\pgfpathmoveto{\pgfqpoint{1.364865in}{1.868873in}}%
\pgfpathlineto{\pgfqpoint{1.318547in}{1.875904in}}%
\pgfpathlineto{\pgfqpoint{1.317490in}{1.876575in}}%
\pgfpathlineto{\pgfqpoint{1.278870in}{1.882470in}}%
\pgfpathlineto{\pgfqpoint{1.281086in}{1.895779in}}%
\pgfpathlineto{\pgfqpoint{1.266400in}{1.898228in}}%
\pgfpathlineto{\pgfqpoint{1.268744in}{1.912078in}}%
\pgfpathlineto{\pgfqpoint{1.275970in}{1.910859in}}%
\pgfpathlineto{\pgfqpoint{1.278245in}{1.924515in}}%
\pgfpathlineto{\pgfqpoint{1.292921in}{1.929061in}}%
\pgfpathlineto{\pgfqpoint{1.299700in}{1.927978in}}%
\pgfpathlineto{\pgfqpoint{1.301786in}{1.941660in}}%
\pgfpathlineto{\pgfqpoint{1.307165in}{1.954995in}}%
\pgfpathlineto{\pgfqpoint{1.311739in}{1.954347in}}%
\pgfpathlineto{\pgfqpoint{1.313126in}{1.960922in}}%
\pgfpathlineto{\pgfqpoint{1.302894in}{1.962642in}}%
\pgfpathlineto{\pgfqpoint{1.301322in}{1.969905in}}%
\pgfpathlineto{\pgfqpoint{1.302428in}{1.976762in}}%
\pgfpathlineto{\pgfqpoint{1.316114in}{1.974564in}}%
\pgfpathlineto{\pgfqpoint{1.322819in}{2.011773in}}%
\pgfpathlineto{\pgfqpoint{1.323552in}{2.016288in}}%
\pgfpathlineto{\pgfqpoint{1.364475in}{2.009800in}}%
\pgfpathlineto{\pgfqpoint{1.407997in}{2.003940in}}%
\pgfpathlineto{\pgfqpoint{1.409028in}{1.997695in}}%
\pgfpathlineto{\pgfqpoint{1.406757in}{1.983167in}}%
\pgfpathlineto{\pgfqpoint{1.410750in}{1.980307in}}%
\pgfpathlineto{\pgfqpoint{1.414681in}{1.966931in}}%
\pgfpathlineto{\pgfqpoint{1.421242in}{1.955525in}}%
\pgfpathlineto{\pgfqpoint{1.417308in}{1.945271in}}%
\pgfpathlineto{\pgfqpoint{1.408401in}{1.946543in}}%
\pgfpathlineto{\pgfqpoint{1.407750in}{1.941975in}}%
\pgfpathlineto{\pgfqpoint{1.400977in}{1.942930in}}%
\pgfpathlineto{\pgfqpoint{1.395427in}{1.936746in}}%
\pgfpathlineto{\pgfqpoint{1.386336in}{1.938080in}}%
\pgfpathlineto{\pgfqpoint{1.383830in}{1.929135in}}%
\pgfpathlineto{\pgfqpoint{1.380767in}{1.908242in}}%
\pgfpathlineto{\pgfqpoint{1.366491in}{1.905688in}}%
\pgfpathlineto{\pgfqpoint{1.348402in}{1.908679in}}%
\pgfpathlineto{\pgfqpoint{1.345983in}{1.906551in}}%
\pgfpathlineto{\pgfqpoint{1.342451in}{1.889782in}}%
\pgfpathlineto{\pgfqpoint{1.375793in}{1.885029in}}%
\pgfpathlineto{\pgfqpoint{1.372850in}{1.883266in}}%
\pgfpathlineto{\pgfqpoint{1.364865in}{1.868873in}}%
\pgfpathclose%
\pgfusepath{fill}%
\end{pgfscope}%
\begin{pgfscope}%
\pgfpathrectangle{\pgfqpoint{0.100000in}{0.100000in}}{\pgfqpoint{3.608454in}{2.310000in}}%
\pgfusepath{clip}%
\pgfsetbuttcap%
\pgfsetmiterjoin%
\definecolor{currentfill}{rgb}{0.000000,0.647059,0.676471}%
\pgfsetfillcolor{currentfill}%
\pgfsetlinewidth{0.000000pt}%
\definecolor{currentstroke}{rgb}{0.000000,0.000000,0.000000}%
\pgfsetstrokecolor{currentstroke}%
\pgfsetstrokeopacity{0.000000}%
\pgfsetdash{}{0pt}%
\pgfpathmoveto{\pgfqpoint{2.893588in}{0.882495in}}%
\pgfpathlineto{\pgfqpoint{2.878409in}{0.905708in}}%
\pgfpathlineto{\pgfqpoint{2.881402in}{0.908328in}}%
\pgfpathlineto{\pgfqpoint{2.874063in}{0.928161in}}%
\pgfpathlineto{\pgfqpoint{2.866524in}{0.926111in}}%
\pgfpathlineto{\pgfqpoint{2.864668in}{0.946906in}}%
\pgfpathlineto{\pgfqpoint{2.880855in}{0.952903in}}%
\pgfpathlineto{\pgfqpoint{2.889760in}{0.943778in}}%
\pgfpathlineto{\pgfqpoint{2.898524in}{0.954072in}}%
\pgfpathlineto{\pgfqpoint{2.909363in}{0.953452in}}%
\pgfpathlineto{\pgfqpoint{2.918368in}{0.938239in}}%
\pgfpathlineto{\pgfqpoint{2.911845in}{0.934034in}}%
\pgfpathlineto{\pgfqpoint{2.910167in}{0.929143in}}%
\pgfpathlineto{\pgfqpoint{2.905128in}{0.926113in}}%
\pgfpathlineto{\pgfqpoint{2.899562in}{0.918887in}}%
\pgfpathlineto{\pgfqpoint{2.899980in}{0.912387in}}%
\pgfpathlineto{\pgfqpoint{2.903959in}{0.905713in}}%
\pgfpathlineto{\pgfqpoint{2.909397in}{0.902008in}}%
\pgfpathlineto{\pgfqpoint{2.910191in}{0.894634in}}%
\pgfpathlineto{\pgfqpoint{2.893588in}{0.882495in}}%
\pgfpathclose%
\pgfusepath{fill}%
\end{pgfscope}%
\begin{pgfscope}%
\pgfpathrectangle{\pgfqpoint{0.100000in}{0.100000in}}{\pgfqpoint{3.608454in}{2.310000in}}%
\pgfusepath{clip}%
\pgfsetbuttcap%
\pgfsetmiterjoin%
\definecolor{currentfill}{rgb}{0.000000,0.611765,0.694118}%
\pgfsetfillcolor{currentfill}%
\pgfsetlinewidth{0.000000pt}%
\definecolor{currentstroke}{rgb}{0.000000,0.000000,0.000000}%
\pgfsetstrokecolor{currentstroke}%
\pgfsetstrokeopacity{0.000000}%
\pgfsetdash{}{0pt}%
\pgfpathmoveto{\pgfqpoint{1.496354in}{2.088951in}}%
\pgfpathlineto{\pgfqpoint{1.498672in}{2.081480in}}%
\pgfpathlineto{\pgfqpoint{1.503848in}{2.080824in}}%
\pgfpathlineto{\pgfqpoint{1.500521in}{2.053891in}}%
\pgfpathlineto{\pgfqpoint{1.496777in}{2.040294in}}%
\pgfpathlineto{\pgfqpoint{1.503629in}{2.039468in}}%
\pgfpathlineto{\pgfqpoint{1.501940in}{2.025689in}}%
\pgfpathlineto{\pgfqpoint{1.506437in}{2.025137in}}%
\pgfpathlineto{\pgfqpoint{1.503208in}{1.998589in}}%
\pgfpathlineto{\pgfqpoint{1.468884in}{2.002862in}}%
\pgfpathlineto{\pgfqpoint{1.469044in}{2.003996in}}%
\pgfpathlineto{\pgfqpoint{1.405583in}{2.011993in}}%
\pgfpathlineto{\pgfqpoint{1.403304in}{2.014885in}}%
\pgfpathlineto{\pgfqpoint{1.407470in}{2.024546in}}%
\pgfpathlineto{\pgfqpoint{1.405800in}{2.033757in}}%
\pgfpathlineto{\pgfqpoint{1.406013in}{2.044122in}}%
\pgfpathlineto{\pgfqpoint{1.411450in}{2.059486in}}%
\pgfpathlineto{\pgfqpoint{1.414702in}{2.064290in}}%
\pgfpathlineto{\pgfqpoint{1.422183in}{2.067656in}}%
\pgfpathlineto{\pgfqpoint{1.428142in}{2.073167in}}%
\pgfpathlineto{\pgfqpoint{1.437875in}{2.069615in}}%
\pgfpathlineto{\pgfqpoint{1.440516in}{2.074870in}}%
\pgfpathlineto{\pgfqpoint{1.449108in}{2.071253in}}%
\pgfpathlineto{\pgfqpoint{1.468292in}{2.069917in}}%
\pgfpathlineto{\pgfqpoint{1.470183in}{2.074395in}}%
\pgfpathlineto{\pgfqpoint{1.476331in}{2.073148in}}%
\pgfpathlineto{\pgfqpoint{1.487031in}{2.078600in}}%
\pgfpathlineto{\pgfqpoint{1.492326in}{2.087153in}}%
\pgfpathlineto{\pgfqpoint{1.496354in}{2.088951in}}%
\pgfpathclose%
\pgfusepath{fill}%
\end{pgfscope}%
\begin{pgfscope}%
\pgfpathrectangle{\pgfqpoint{0.100000in}{0.100000in}}{\pgfqpoint{3.608454in}{2.310000in}}%
\pgfusepath{clip}%
\pgfsetbuttcap%
\pgfsetmiterjoin%
\definecolor{currentfill}{rgb}{0.000000,0.423529,0.788235}%
\pgfsetfillcolor{currentfill}%
\pgfsetlinewidth{0.000000pt}%
\definecolor{currentstroke}{rgb}{0.000000,0.000000,0.000000}%
\pgfsetstrokecolor{currentstroke}%
\pgfsetstrokeopacity{0.000000}%
\pgfsetdash{}{0pt}%
\pgfpathmoveto{\pgfqpoint{3.153317in}{1.449269in}}%
\pgfpathlineto{\pgfqpoint{3.117245in}{1.469315in}}%
\pgfpathlineto{\pgfqpoint{3.109226in}{1.471916in}}%
\pgfpathlineto{\pgfqpoint{3.110506in}{1.479451in}}%
\pgfpathlineto{\pgfqpoint{3.116008in}{1.487243in}}%
\pgfpathlineto{\pgfqpoint{3.111561in}{1.488897in}}%
\pgfpathlineto{\pgfqpoint{3.114952in}{1.497798in}}%
\pgfpathlineto{\pgfqpoint{3.117244in}{1.511717in}}%
\pgfpathlineto{\pgfqpoint{3.116662in}{1.519085in}}%
\pgfpathlineto{\pgfqpoint{3.119792in}{1.526024in}}%
\pgfpathlineto{\pgfqpoint{3.133137in}{1.523448in}}%
\pgfpathlineto{\pgfqpoint{3.137378in}{1.521010in}}%
\pgfpathlineto{\pgfqpoint{3.143924in}{1.538541in}}%
\pgfpathlineto{\pgfqpoint{3.146592in}{1.539583in}}%
\pgfpathlineto{\pgfqpoint{3.162472in}{1.516430in}}%
\pgfpathlineto{\pgfqpoint{3.166603in}{1.499150in}}%
\pgfpathlineto{\pgfqpoint{3.161516in}{1.490087in}}%
\pgfpathlineto{\pgfqpoint{3.159520in}{1.465911in}}%
\pgfpathlineto{\pgfqpoint{3.156244in}{1.464323in}}%
\pgfpathlineto{\pgfqpoint{3.153317in}{1.449269in}}%
\pgfpathclose%
\pgfusepath{fill}%
\end{pgfscope}%
\begin{pgfscope}%
\pgfpathrectangle{\pgfqpoint{0.100000in}{0.100000in}}{\pgfqpoint{3.608454in}{2.310000in}}%
\pgfusepath{clip}%
\pgfsetbuttcap%
\pgfsetmiterjoin%
\definecolor{currentfill}{rgb}{0.000000,0.827451,0.586275}%
\pgfsetfillcolor{currentfill}%
\pgfsetlinewidth{0.000000pt}%
\definecolor{currentstroke}{rgb}{0.000000,0.000000,0.000000}%
\pgfsetstrokecolor{currentstroke}%
\pgfsetstrokeopacity{0.000000}%
\pgfsetdash{}{0pt}%
\pgfpathmoveto{\pgfqpoint{0.735331in}{0.472883in}}%
\pgfpathlineto{\pgfqpoint{0.738160in}{0.475971in}}%
\pgfpathlineto{\pgfqpoint{0.743525in}{0.474782in}}%
\pgfpathlineto{\pgfqpoint{0.740586in}{0.472887in}}%
\pgfpathlineto{\pgfqpoint{0.735331in}{0.472883in}}%
\pgfpathclose%
\pgfusepath{fill}%
\end{pgfscope}%
\begin{pgfscope}%
\pgfpathrectangle{\pgfqpoint{0.100000in}{0.100000in}}{\pgfqpoint{3.608454in}{2.310000in}}%
\pgfusepath{clip}%
\pgfsetbuttcap%
\pgfsetmiterjoin%
\definecolor{currentfill}{rgb}{0.000000,0.827451,0.586275}%
\pgfsetfillcolor{currentfill}%
\pgfsetlinewidth{0.000000pt}%
\definecolor{currentstroke}{rgb}{0.000000,0.000000,0.000000}%
\pgfsetstrokecolor{currentstroke}%
\pgfsetstrokeopacity{0.000000}%
\pgfsetdash{}{0pt}%
\pgfpathmoveto{\pgfqpoint{0.864761in}{0.455531in}}%
\pgfpathlineto{\pgfqpoint{0.863736in}{0.453184in}}%
\pgfpathlineto{\pgfqpoint{0.857604in}{0.444984in}}%
\pgfpathlineto{\pgfqpoint{0.856061in}{0.446120in}}%
\pgfpathlineto{\pgfqpoint{0.849915in}{0.438009in}}%
\pgfpathlineto{\pgfqpoint{0.848446in}{0.439078in}}%
\pgfpathlineto{\pgfqpoint{0.842589in}{0.431260in}}%
\pgfpathlineto{\pgfqpoint{0.841100in}{0.432388in}}%
\pgfpathlineto{\pgfqpoint{0.839547in}{0.430349in}}%
\pgfpathlineto{\pgfqpoint{0.837531in}{0.431857in}}%
\pgfpathlineto{\pgfqpoint{0.834469in}{0.427785in}}%
\pgfpathlineto{\pgfqpoint{0.832475in}{0.429280in}}%
\pgfpathlineto{\pgfqpoint{0.830966in}{0.427222in}}%
\pgfpathlineto{\pgfqpoint{0.827494in}{0.429779in}}%
\pgfpathlineto{\pgfqpoint{0.824498in}{0.425742in}}%
\pgfpathlineto{\pgfqpoint{0.820512in}{0.428750in}}%
\pgfpathlineto{\pgfqpoint{0.817435in}{0.424709in}}%
\pgfpathlineto{\pgfqpoint{0.818042in}{0.424203in}}%
\pgfpathlineto{\pgfqpoint{0.812034in}{0.416427in}}%
\pgfpathlineto{\pgfqpoint{0.811149in}{0.413923in}}%
\pgfpathlineto{\pgfqpoint{0.815079in}{0.410909in}}%
\pgfpathlineto{\pgfqpoint{0.811859in}{0.406633in}}%
\pgfpathlineto{\pgfqpoint{0.809601in}{0.406866in}}%
\pgfpathlineto{\pgfqpoint{0.808196in}{0.405860in}}%
\pgfpathlineto{\pgfqpoint{0.804050in}{0.407001in}}%
\pgfpathlineto{\pgfqpoint{0.800501in}{0.409249in}}%
\pgfpathlineto{\pgfqpoint{0.798949in}{0.408471in}}%
\pgfpathlineto{\pgfqpoint{0.797005in}{0.409992in}}%
\pgfpathlineto{\pgfqpoint{0.794932in}{0.410045in}}%
\pgfpathlineto{\pgfqpoint{0.789736in}{0.403394in}}%
\pgfpathlineto{\pgfqpoint{0.786031in}{0.403802in}}%
\pgfpathlineto{\pgfqpoint{0.784195in}{0.405252in}}%
\pgfpathlineto{\pgfqpoint{0.781062in}{0.403097in}}%
\pgfpathlineto{\pgfqpoint{0.777277in}{0.402089in}}%
\pgfpathlineto{\pgfqpoint{0.777113in}{0.402156in}}%
\pgfpathlineto{\pgfqpoint{0.777652in}{0.403284in}}%
\pgfpathlineto{\pgfqpoint{0.773253in}{0.404050in}}%
\pgfpathlineto{\pgfqpoint{0.771380in}{0.405564in}}%
\pgfpathlineto{\pgfqpoint{0.768215in}{0.404903in}}%
\pgfpathlineto{\pgfqpoint{0.766993in}{0.403276in}}%
\pgfpathlineto{\pgfqpoint{0.765003in}{0.403792in}}%
\pgfpathlineto{\pgfqpoint{0.761612in}{0.402864in}}%
\pgfpathlineto{\pgfqpoint{0.760327in}{0.401259in}}%
\pgfpathlineto{\pgfqpoint{0.764308in}{0.397983in}}%
\pgfpathlineto{\pgfqpoint{0.759811in}{0.395476in}}%
\pgfpathlineto{\pgfqpoint{0.758622in}{0.397717in}}%
\pgfpathlineto{\pgfqpoint{0.754992in}{0.396142in}}%
\pgfpathlineto{\pgfqpoint{0.753722in}{0.397568in}}%
\pgfpathlineto{\pgfqpoint{0.743096in}{0.399381in}}%
\pgfpathlineto{\pgfqpoint{0.741766in}{0.396518in}}%
\pgfpathlineto{\pgfqpoint{0.739332in}{0.398261in}}%
\pgfpathlineto{\pgfqpoint{0.739251in}{0.400641in}}%
\pgfpathlineto{\pgfqpoint{0.736510in}{0.401202in}}%
\pgfpathlineto{\pgfqpoint{0.736143in}{0.398969in}}%
\pgfpathlineto{\pgfqpoint{0.734429in}{0.397349in}}%
\pgfpathlineto{\pgfqpoint{0.730331in}{0.398641in}}%
\pgfpathlineto{\pgfqpoint{0.727510in}{0.401229in}}%
\pgfpathlineto{\pgfqpoint{0.723876in}{0.399715in}}%
\pgfpathlineto{\pgfqpoint{0.726580in}{0.395915in}}%
\pgfpathlineto{\pgfqpoint{0.720792in}{0.394591in}}%
\pgfpathlineto{\pgfqpoint{0.718356in}{0.391937in}}%
\pgfpathlineto{\pgfqpoint{0.717498in}{0.394606in}}%
\pgfpathlineto{\pgfqpoint{0.714729in}{0.394241in}}%
\pgfpathlineto{\pgfqpoint{0.712644in}{0.395735in}}%
\pgfpathlineto{\pgfqpoint{0.710848in}{0.395775in}}%
\pgfpathlineto{\pgfqpoint{0.705221in}{0.398449in}}%
\pgfpathlineto{\pgfqpoint{0.702788in}{0.397277in}}%
\pgfpathlineto{\pgfqpoint{0.700932in}{0.397327in}}%
\pgfpathlineto{\pgfqpoint{0.698256in}{0.395576in}}%
\pgfpathlineto{\pgfqpoint{0.702855in}{0.400929in}}%
\pgfpathlineto{\pgfqpoint{0.699223in}{0.404064in}}%
\pgfpathlineto{\pgfqpoint{0.701711in}{0.405421in}}%
\pgfpathlineto{\pgfqpoint{0.707923in}{0.412710in}}%
\pgfpathlineto{\pgfqpoint{0.706826in}{0.413694in}}%
\pgfpathlineto{\pgfqpoint{0.710015in}{0.417271in}}%
\pgfpathlineto{\pgfqpoint{0.713680in}{0.414200in}}%
\pgfpathlineto{\pgfqpoint{0.715284in}{0.416046in}}%
\pgfpathlineto{\pgfqpoint{0.719039in}{0.412825in}}%
\pgfpathlineto{\pgfqpoint{0.720711in}{0.414781in}}%
\pgfpathlineto{\pgfqpoint{0.723593in}{0.412237in}}%
\pgfpathlineto{\pgfqpoint{0.725147in}{0.414065in}}%
\pgfpathlineto{\pgfqpoint{0.727666in}{0.413737in}}%
\pgfpathlineto{\pgfqpoint{0.730412in}{0.411143in}}%
\pgfpathlineto{\pgfqpoint{0.733440in}{0.415672in}}%
\pgfpathlineto{\pgfqpoint{0.739202in}{0.417909in}}%
\pgfpathlineto{\pgfqpoint{0.745161in}{0.418898in}}%
\pgfpathlineto{\pgfqpoint{0.748345in}{0.417706in}}%
\pgfpathlineto{\pgfqpoint{0.751670in}{0.419346in}}%
\pgfpathlineto{\pgfqpoint{0.756304in}{0.418993in}}%
\pgfpathlineto{\pgfqpoint{0.757747in}{0.421251in}}%
\pgfpathlineto{\pgfqpoint{0.768447in}{0.430721in}}%
\pgfpathlineto{\pgfqpoint{0.770758in}{0.430833in}}%
\pgfpathlineto{\pgfqpoint{0.770614in}{0.432762in}}%
\pgfpathlineto{\pgfqpoint{0.774098in}{0.436410in}}%
\pgfpathlineto{\pgfqpoint{0.779568in}{0.437653in}}%
\pgfpathlineto{\pgfqpoint{0.790893in}{0.428785in}}%
\pgfpathlineto{\pgfqpoint{0.796011in}{0.435184in}}%
\pgfpathlineto{\pgfqpoint{0.787216in}{0.442053in}}%
\pgfpathlineto{\pgfqpoint{0.772048in}{0.445375in}}%
\pgfpathlineto{\pgfqpoint{0.769184in}{0.446551in}}%
\pgfpathlineto{\pgfqpoint{0.768571in}{0.448855in}}%
\pgfpathlineto{\pgfqpoint{0.769448in}{0.452003in}}%
\pgfpathlineto{\pgfqpoint{0.768194in}{0.453396in}}%
\pgfpathlineto{\pgfqpoint{0.771817in}{0.456869in}}%
\pgfpathlineto{\pgfqpoint{0.770582in}{0.457460in}}%
\pgfpathlineto{\pgfqpoint{0.763393in}{0.455439in}}%
\pgfpathlineto{\pgfqpoint{0.761052in}{0.449862in}}%
\pgfpathlineto{\pgfqpoint{0.759380in}{0.447981in}}%
\pgfpathlineto{\pgfqpoint{0.757127in}{0.448341in}}%
\pgfpathlineto{\pgfqpoint{0.755962in}{0.451093in}}%
\pgfpathlineto{\pgfqpoint{0.756950in}{0.451960in}}%
\pgfpathlineto{\pgfqpoint{0.756995in}{0.455328in}}%
\pgfpathlineto{\pgfqpoint{0.758416in}{0.463072in}}%
\pgfpathlineto{\pgfqpoint{0.757580in}{0.464630in}}%
\pgfpathlineto{\pgfqpoint{0.759553in}{0.467203in}}%
\pgfpathlineto{\pgfqpoint{0.757448in}{0.468248in}}%
\pgfpathlineto{\pgfqpoint{0.756266in}{0.467141in}}%
\pgfpathlineto{\pgfqpoint{0.752660in}{0.466199in}}%
\pgfpathlineto{\pgfqpoint{0.752967in}{0.468537in}}%
\pgfpathlineto{\pgfqpoint{0.751538in}{0.473475in}}%
\pgfpathlineto{\pgfqpoint{0.753567in}{0.476503in}}%
\pgfpathlineto{\pgfqpoint{0.750953in}{0.477091in}}%
\pgfpathlineto{\pgfqpoint{0.745672in}{0.477306in}}%
\pgfpathlineto{\pgfqpoint{0.740722in}{0.478712in}}%
\pgfpathlineto{\pgfqpoint{0.744357in}{0.482168in}}%
\pgfpathlineto{\pgfqpoint{0.746173in}{0.480590in}}%
\pgfpathlineto{\pgfqpoint{0.747682in}{0.482304in}}%
\pgfpathlineto{\pgfqpoint{0.749547in}{0.480704in}}%
\pgfpathlineto{\pgfqpoint{0.752873in}{0.484455in}}%
\pgfpathlineto{\pgfqpoint{0.755169in}{0.485784in}}%
\pgfpathlineto{\pgfqpoint{0.756995in}{0.484116in}}%
\pgfpathlineto{\pgfqpoint{0.759307in}{0.485494in}}%
\pgfpathlineto{\pgfqpoint{0.762636in}{0.489281in}}%
\pgfpathlineto{\pgfqpoint{0.764542in}{0.487613in}}%
\pgfpathlineto{\pgfqpoint{0.767845in}{0.491365in}}%
\pgfpathlineto{\pgfqpoint{0.770402in}{0.489224in}}%
\pgfpathlineto{\pgfqpoint{0.773696in}{0.492989in}}%
\pgfpathlineto{\pgfqpoint{0.775519in}{0.491429in}}%
\pgfpathlineto{\pgfqpoint{0.778697in}{0.494994in}}%
\pgfpathlineto{\pgfqpoint{0.781042in}{0.496290in}}%
\pgfpathlineto{\pgfqpoint{0.782945in}{0.494696in}}%
\pgfpathlineto{\pgfqpoint{0.786206in}{0.498438in}}%
\pgfpathlineto{\pgfqpoint{0.788112in}{0.496794in}}%
\pgfpathlineto{\pgfqpoint{0.789763in}{0.498705in}}%
\pgfpathlineto{\pgfqpoint{0.791902in}{0.496847in}}%
\pgfpathlineto{\pgfqpoint{0.795190in}{0.500666in}}%
\pgfpathlineto{\pgfqpoint{0.797356in}{0.502139in}}%
\pgfpathlineto{\pgfqpoint{0.799211in}{0.500636in}}%
\pgfpathlineto{\pgfqpoint{0.800941in}{0.502662in}}%
\pgfpathlineto{\pgfqpoint{0.802432in}{0.501473in}}%
\pgfpathlineto{\pgfqpoint{0.804074in}{0.503387in}}%
\pgfpathlineto{\pgfqpoint{0.814957in}{0.494457in}}%
\pgfpathlineto{\pgfqpoint{0.813395in}{0.492489in}}%
\pgfpathlineto{\pgfqpoint{0.815390in}{0.490823in}}%
\pgfpathlineto{\pgfqpoint{0.816987in}{0.492818in}}%
\pgfpathlineto{\pgfqpoint{0.819027in}{0.491316in}}%
\pgfpathlineto{\pgfqpoint{0.820776in}{0.493434in}}%
\pgfpathlineto{\pgfqpoint{0.824629in}{0.490280in}}%
\pgfpathlineto{\pgfqpoint{0.822897in}{0.488128in}}%
\pgfpathlineto{\pgfqpoint{0.836300in}{0.477446in}}%
\pgfpathlineto{\pgfqpoint{0.853555in}{0.463966in}}%
\pgfpathlineto{\pgfqpoint{0.864761in}{0.455531in}}%
\pgfpathclose%
\pgfusepath{fill}%
\end{pgfscope}%
\begin{pgfscope}%
\pgfpathrectangle{\pgfqpoint{0.100000in}{0.100000in}}{\pgfqpoint{3.608454in}{2.310000in}}%
\pgfusepath{clip}%
\pgfsetbuttcap%
\pgfsetmiterjoin%
\definecolor{currentfill}{rgb}{0.000000,0.827451,0.586275}%
\pgfsetfillcolor{currentfill}%
\pgfsetlinewidth{0.000000pt}%
\definecolor{currentstroke}{rgb}{0.000000,0.000000,0.000000}%
\pgfsetstrokecolor{currentstroke}%
\pgfsetstrokeopacity{0.000000}%
\pgfsetdash{}{0pt}%
\pgfpathmoveto{\pgfqpoint{0.735963in}{0.479492in}}%
\pgfpathlineto{\pgfqpoint{0.737273in}{0.481661in}}%
\pgfpathlineto{\pgfqpoint{0.740538in}{0.478732in}}%
\pgfpathlineto{\pgfqpoint{0.735963in}{0.479492in}}%
\pgfpathclose%
\pgfusepath{fill}%
\end{pgfscope}%
\begin{pgfscope}%
\pgfpathrectangle{\pgfqpoint{0.100000in}{0.100000in}}{\pgfqpoint{3.608454in}{2.310000in}}%
\pgfusepath{clip}%
\pgfsetbuttcap%
\pgfsetmiterjoin%
\definecolor{currentfill}{rgb}{0.000000,0.694118,0.652941}%
\pgfsetfillcolor{currentfill}%
\pgfsetlinewidth{0.000000pt}%
\definecolor{currentstroke}{rgb}{0.000000,0.000000,0.000000}%
\pgfsetstrokecolor{currentstroke}%
\pgfsetstrokeopacity{0.000000}%
\pgfsetdash{}{0pt}%
\pgfpathmoveto{\pgfqpoint{1.128763in}{1.959930in}}%
\pgfpathlineto{\pgfqpoint{1.126427in}{1.948422in}}%
\pgfpathlineto{\pgfqpoint{1.131774in}{1.943515in}}%
\pgfpathlineto{\pgfqpoint{1.135886in}{1.943764in}}%
\pgfpathlineto{\pgfqpoint{1.137082in}{1.935988in}}%
\pgfpathlineto{\pgfqpoint{1.133161in}{1.915579in}}%
\pgfpathlineto{\pgfqpoint{1.139942in}{1.914263in}}%
\pgfpathlineto{\pgfqpoint{1.138589in}{1.907407in}}%
\pgfpathlineto{\pgfqpoint{1.146489in}{1.905873in}}%
\pgfpathlineto{\pgfqpoint{1.145219in}{1.892062in}}%
\pgfpathlineto{\pgfqpoint{1.150185in}{1.887641in}}%
\pgfpathlineto{\pgfqpoint{1.165886in}{1.884470in}}%
\pgfpathlineto{\pgfqpoint{1.164838in}{1.878825in}}%
\pgfpathlineto{\pgfqpoint{1.181983in}{1.875682in}}%
\pgfpathlineto{\pgfqpoint{1.177604in}{1.866287in}}%
\pgfpathlineto{\pgfqpoint{1.166980in}{1.866417in}}%
\pgfpathlineto{\pgfqpoint{1.159714in}{1.863660in}}%
\pgfpathlineto{\pgfqpoint{1.157878in}{1.868529in}}%
\pgfpathlineto{\pgfqpoint{1.144630in}{1.867633in}}%
\pgfpathlineto{\pgfqpoint{1.134588in}{1.873151in}}%
\pgfpathlineto{\pgfqpoint{1.130114in}{1.870874in}}%
\pgfpathlineto{\pgfqpoint{1.127333in}{1.864844in}}%
\pgfpathlineto{\pgfqpoint{1.123023in}{1.868301in}}%
\pgfpathlineto{\pgfqpoint{1.105971in}{1.872044in}}%
\pgfpathlineto{\pgfqpoint{1.102494in}{1.863989in}}%
\pgfpathlineto{\pgfqpoint{1.100477in}{1.862882in}}%
\pgfpathlineto{\pgfqpoint{1.093092in}{1.871853in}}%
\pgfpathlineto{\pgfqpoint{1.094071in}{1.877735in}}%
\pgfpathlineto{\pgfqpoint{1.091986in}{1.894613in}}%
\pgfpathlineto{\pgfqpoint{1.088102in}{1.901286in}}%
\pgfpathlineto{\pgfqpoint{1.079391in}{1.904532in}}%
\pgfpathlineto{\pgfqpoint{1.074674in}{1.909170in}}%
\pgfpathlineto{\pgfqpoint{1.077934in}{1.924256in}}%
\pgfpathlineto{\pgfqpoint{1.073148in}{1.929599in}}%
\pgfpathlineto{\pgfqpoint{1.066629in}{1.948325in}}%
\pgfpathlineto{\pgfqpoint{1.067866in}{1.953218in}}%
\pgfpathlineto{\pgfqpoint{1.064456in}{1.964544in}}%
\pgfpathlineto{\pgfqpoint{1.067306in}{1.970565in}}%
\pgfpathlineto{\pgfqpoint{1.061696in}{1.979359in}}%
\pgfpathlineto{\pgfqpoint{1.069612in}{1.982716in}}%
\pgfpathlineto{\pgfqpoint{1.076963in}{1.988148in}}%
\pgfpathlineto{\pgfqpoint{1.086269in}{1.988945in}}%
\pgfpathlineto{\pgfqpoint{1.088551in}{1.993499in}}%
\pgfpathlineto{\pgfqpoint{1.092311in}{1.978823in}}%
\pgfpathlineto{\pgfqpoint{1.103252in}{1.984857in}}%
\pgfpathlineto{\pgfqpoint{1.110227in}{1.984421in}}%
\pgfpathlineto{\pgfqpoint{1.113810in}{1.978599in}}%
\pgfpathlineto{\pgfqpoint{1.126712in}{1.970229in}}%
\pgfpathlineto{\pgfqpoint{1.128763in}{1.959930in}}%
\pgfpathclose%
\pgfusepath{fill}%
\end{pgfscope}%
\begin{pgfscope}%
\pgfpathrectangle{\pgfqpoint{0.100000in}{0.100000in}}{\pgfqpoint{3.608454in}{2.310000in}}%
\pgfusepath{clip}%
\pgfsetbuttcap%
\pgfsetmiterjoin%
\definecolor{currentfill}{rgb}{0.000000,0.607843,0.696078}%
\pgfsetfillcolor{currentfill}%
\pgfsetlinewidth{0.000000pt}%
\definecolor{currentstroke}{rgb}{0.000000,0.000000,0.000000}%
\pgfsetstrokecolor{currentstroke}%
\pgfsetstrokeopacity{0.000000}%
\pgfsetdash{}{0pt}%
\pgfpathmoveto{\pgfqpoint{0.869214in}{2.048910in}}%
\pgfpathlineto{\pgfqpoint{0.906027in}{2.039422in}}%
\pgfpathlineto{\pgfqpoint{0.909977in}{2.024514in}}%
\pgfpathlineto{\pgfqpoint{0.914485in}{2.022985in}}%
\pgfpathlineto{\pgfqpoint{0.921587in}{2.014227in}}%
\pgfpathlineto{\pgfqpoint{0.923281in}{2.004122in}}%
\pgfpathlineto{\pgfqpoint{0.916109in}{1.996717in}}%
\pgfpathlineto{\pgfqpoint{0.904646in}{1.980156in}}%
\pgfpathlineto{\pgfqpoint{0.899963in}{1.971006in}}%
\pgfpathlineto{\pgfqpoint{0.895815in}{1.966858in}}%
\pgfpathlineto{\pgfqpoint{0.869552in}{1.973496in}}%
\pgfpathlineto{\pgfqpoint{0.871160in}{1.980064in}}%
\pgfpathlineto{\pgfqpoint{0.859744in}{1.982888in}}%
\pgfpathlineto{\pgfqpoint{0.859301in}{1.991402in}}%
\pgfpathlineto{\pgfqpoint{0.854053in}{1.998430in}}%
\pgfpathlineto{\pgfqpoint{0.853023in}{2.013264in}}%
\pgfpathlineto{\pgfqpoint{0.855049in}{2.021169in}}%
\pgfpathlineto{\pgfqpoint{0.853215in}{2.027577in}}%
\pgfpathlineto{\pgfqpoint{0.858767in}{2.040242in}}%
\pgfpathlineto{\pgfqpoint{0.846645in}{2.043323in}}%
\pgfpathlineto{\pgfqpoint{0.849176in}{2.054195in}}%
\pgfpathlineto{\pgfqpoint{0.869214in}{2.048910in}}%
\pgfpathclose%
\pgfusepath{fill}%
\end{pgfscope}%
\begin{pgfscope}%
\pgfpathrectangle{\pgfqpoint{0.100000in}{0.100000in}}{\pgfqpoint{3.608454in}{2.310000in}}%
\pgfusepath{clip}%
\pgfsetbuttcap%
\pgfsetmiterjoin%
\definecolor{currentfill}{rgb}{0.000000,0.364706,0.817647}%
\pgfsetfillcolor{currentfill}%
\pgfsetlinewidth{0.000000pt}%
\definecolor{currentstroke}{rgb}{0.000000,0.000000,0.000000}%
\pgfsetstrokecolor{currentstroke}%
\pgfsetstrokeopacity{0.000000}%
\pgfsetdash{}{0pt}%
\pgfpathmoveto{\pgfqpoint{1.707133in}{1.330203in}}%
\pgfpathlineto{\pgfqpoint{1.672634in}{1.332579in}}%
\pgfpathlineto{\pgfqpoint{1.677687in}{1.401208in}}%
\pgfpathlineto{\pgfqpoint{1.716323in}{1.398352in}}%
\pgfpathlineto{\pgfqpoint{1.717833in}{1.398242in}}%
\pgfpathlineto{\pgfqpoint{1.715265in}{1.364101in}}%
\pgfpathlineto{\pgfqpoint{1.709976in}{1.364447in}}%
\pgfpathlineto{\pgfqpoint{1.707133in}{1.330203in}}%
\pgfpathclose%
\pgfusepath{fill}%
\end{pgfscope}%
\begin{pgfscope}%
\pgfpathrectangle{\pgfqpoint{0.100000in}{0.100000in}}{\pgfqpoint{3.608454in}{2.310000in}}%
\pgfusepath{clip}%
\pgfsetbuttcap%
\pgfsetmiterjoin%
\definecolor{currentfill}{rgb}{0.000000,0.243137,0.878431}%
\pgfsetfillcolor{currentfill}%
\pgfsetlinewidth{0.000000pt}%
\definecolor{currentstroke}{rgb}{0.000000,0.000000,0.000000}%
\pgfsetstrokecolor{currentstroke}%
\pgfsetstrokeopacity{0.000000}%
\pgfsetdash{}{0pt}%
\pgfpathmoveto{\pgfqpoint{2.034871in}{0.852102in}}%
\pgfpathlineto{\pgfqpoint{2.034479in}{0.813847in}}%
\pgfpathlineto{\pgfqpoint{2.013777in}{0.811827in}}%
\pgfpathlineto{\pgfqpoint{1.978879in}{0.792186in}}%
\pgfpathlineto{\pgfqpoint{1.975845in}{0.790506in}}%
\pgfpathlineto{\pgfqpoint{1.966416in}{0.807704in}}%
\pgfpathlineto{\pgfqpoint{1.966891in}{0.830295in}}%
\pgfpathlineto{\pgfqpoint{1.970138in}{0.830158in}}%
\pgfpathlineto{\pgfqpoint{1.971315in}{0.864806in}}%
\pgfpathlineto{\pgfqpoint{1.946927in}{0.865843in}}%
\pgfpathlineto{\pgfqpoint{1.948881in}{0.900975in}}%
\pgfpathlineto{\pgfqpoint{1.941990in}{0.901235in}}%
\pgfpathlineto{\pgfqpoint{1.943227in}{0.939444in}}%
\pgfpathlineto{\pgfqpoint{1.945662in}{0.932105in}}%
\pgfpathlineto{\pgfqpoint{1.950349in}{0.931638in}}%
\pgfpathlineto{\pgfqpoint{1.958950in}{0.938059in}}%
\pgfpathlineto{\pgfqpoint{1.963046in}{0.937524in}}%
\pgfpathlineto{\pgfqpoint{1.961356in}{0.929934in}}%
\pgfpathlineto{\pgfqpoint{1.964587in}{0.923699in}}%
\pgfpathlineto{\pgfqpoint{1.968361in}{0.923919in}}%
\pgfpathlineto{\pgfqpoint{1.978804in}{0.941263in}}%
\pgfpathlineto{\pgfqpoint{1.977949in}{0.898974in}}%
\pgfpathlineto{\pgfqpoint{1.984737in}{0.897974in}}%
\pgfpathlineto{\pgfqpoint{2.014996in}{0.896820in}}%
\pgfpathlineto{\pgfqpoint{2.020851in}{0.893088in}}%
\pgfpathlineto{\pgfqpoint{2.020147in}{0.852551in}}%
\pgfpathlineto{\pgfqpoint{2.034871in}{0.852102in}}%
\pgfpathclose%
\pgfusepath{fill}%
\end{pgfscope}%
\begin{pgfscope}%
\pgfpathrectangle{\pgfqpoint{0.100000in}{0.100000in}}{\pgfqpoint{3.608454in}{2.310000in}}%
\pgfusepath{clip}%
\pgfsetbuttcap%
\pgfsetmiterjoin%
\definecolor{currentfill}{rgb}{0.000000,0.545098,0.727451}%
\pgfsetfillcolor{currentfill}%
\pgfsetlinewidth{0.000000pt}%
\definecolor{currentstroke}{rgb}{0.000000,0.000000,0.000000}%
\pgfsetstrokecolor{currentstroke}%
\pgfsetstrokeopacity{0.000000}%
\pgfsetdash{}{0pt}%
\pgfpathmoveto{\pgfqpoint{1.978804in}{0.941263in}}%
\pgfpathlineto{\pgfqpoint{1.968361in}{0.923919in}}%
\pgfpathlineto{\pgfqpoint{1.964587in}{0.923699in}}%
\pgfpathlineto{\pgfqpoint{1.961356in}{0.929934in}}%
\pgfpathlineto{\pgfqpoint{1.963046in}{0.937524in}}%
\pgfpathlineto{\pgfqpoint{1.958950in}{0.938059in}}%
\pgfpathlineto{\pgfqpoint{1.950349in}{0.931638in}}%
\pgfpathlineto{\pgfqpoint{1.945662in}{0.932105in}}%
\pgfpathlineto{\pgfqpoint{1.943227in}{0.939444in}}%
\pgfpathlineto{\pgfqpoint{1.938143in}{0.938336in}}%
\pgfpathlineto{\pgfqpoint{1.939766in}{1.000299in}}%
\pgfpathlineto{\pgfqpoint{1.932881in}{1.000512in}}%
\pgfpathlineto{\pgfqpoint{1.933314in}{1.014281in}}%
\pgfpathlineto{\pgfqpoint{1.981120in}{1.013018in}}%
\pgfpathlineto{\pgfqpoint{1.980692in}{0.992304in}}%
\pgfpathlineto{\pgfqpoint{1.987552in}{0.992168in}}%
\pgfpathlineto{\pgfqpoint{1.987413in}{0.985257in}}%
\pgfpathlineto{\pgfqpoint{2.007887in}{0.984833in}}%
\pgfpathlineto{\pgfqpoint{2.007778in}{0.978034in}}%
\pgfpathlineto{\pgfqpoint{2.014740in}{0.977835in}}%
\pgfpathlineto{\pgfqpoint{2.014398in}{0.957120in}}%
\pgfpathlineto{\pgfqpoint{2.000025in}{0.945069in}}%
\pgfpathlineto{\pgfqpoint{1.993908in}{0.931626in}}%
\pgfpathlineto{\pgfqpoint{1.982724in}{0.934735in}}%
\pgfpathlineto{\pgfqpoint{1.978804in}{0.941263in}}%
\pgfpathclose%
\pgfusepath{fill}%
\end{pgfscope}%
\begin{pgfscope}%
\pgfpathrectangle{\pgfqpoint{0.100000in}{0.100000in}}{\pgfqpoint{3.608454in}{2.310000in}}%
\pgfusepath{clip}%
\pgfsetbuttcap%
\pgfsetmiterjoin%
\definecolor{currentfill}{rgb}{0.000000,0.674510,0.662745}%
\pgfsetfillcolor{currentfill}%
\pgfsetlinewidth{0.000000pt}%
\definecolor{currentstroke}{rgb}{0.000000,0.000000,0.000000}%
\pgfsetstrokecolor{currentstroke}%
\pgfsetstrokeopacity{0.000000}%
\pgfsetdash{}{0pt}%
\pgfpathmoveto{\pgfqpoint{1.355496in}{0.949729in}}%
\pgfpathlineto{\pgfqpoint{1.356403in}{0.942467in}}%
\pgfpathlineto{\pgfqpoint{1.353414in}{0.915777in}}%
\pgfpathlineto{\pgfqpoint{1.314868in}{0.902303in}}%
\pgfpathlineto{\pgfqpoint{1.314425in}{0.898871in}}%
\pgfpathlineto{\pgfqpoint{1.287163in}{0.902597in}}%
\pgfpathlineto{\pgfqpoint{1.285282in}{0.888829in}}%
\pgfpathlineto{\pgfqpoint{1.264784in}{0.891644in}}%
\pgfpathlineto{\pgfqpoint{1.256509in}{0.894758in}}%
\pgfpathlineto{\pgfqpoint{1.259177in}{0.906354in}}%
\pgfpathlineto{\pgfqpoint{1.256598in}{0.912844in}}%
\pgfpathlineto{\pgfqpoint{1.257767in}{0.919573in}}%
\pgfpathlineto{\pgfqpoint{1.253634in}{0.922827in}}%
\pgfpathlineto{\pgfqpoint{1.255093in}{0.930635in}}%
\pgfpathlineto{\pgfqpoint{1.253754in}{0.941144in}}%
\pgfpathlineto{\pgfqpoint{1.245517in}{0.942241in}}%
\pgfpathlineto{\pgfqpoint{1.248648in}{0.963976in}}%
\pgfpathlineto{\pgfqpoint{1.252648in}{0.963109in}}%
\pgfpathlineto{\pgfqpoint{1.318032in}{0.954222in}}%
\pgfpathlineto{\pgfqpoint{1.355496in}{0.949729in}}%
\pgfpathclose%
\pgfusepath{fill}%
\end{pgfscope}%
\begin{pgfscope}%
\pgfpathrectangle{\pgfqpoint{0.100000in}{0.100000in}}{\pgfqpoint{3.608454in}{2.310000in}}%
\pgfusepath{clip}%
\pgfsetbuttcap%
\pgfsetmiterjoin%
\definecolor{currentfill}{rgb}{0.000000,0.329412,0.835294}%
\pgfsetfillcolor{currentfill}%
\pgfsetlinewidth{0.000000pt}%
\definecolor{currentstroke}{rgb}{0.000000,0.000000,0.000000}%
\pgfsetstrokecolor{currentstroke}%
\pgfsetstrokeopacity{0.000000}%
\pgfsetdash{}{0pt}%
\pgfpathmoveto{\pgfqpoint{2.042559in}{1.951936in}}%
\pgfpathlineto{\pgfqpoint{2.042594in}{1.944974in}}%
\pgfpathlineto{\pgfqpoint{2.036642in}{1.945064in}}%
\pgfpathlineto{\pgfqpoint{2.036301in}{1.917281in}}%
\pgfpathlineto{\pgfqpoint{2.036803in}{1.896501in}}%
\pgfpathlineto{\pgfqpoint{2.019610in}{1.896790in}}%
\pgfpathlineto{\pgfqpoint{2.020229in}{1.889881in}}%
\pgfpathlineto{\pgfqpoint{1.983409in}{1.890641in}}%
\pgfpathlineto{\pgfqpoint{1.981775in}{1.895615in}}%
\pgfpathlineto{\pgfqpoint{1.981211in}{1.918353in}}%
\pgfpathlineto{\pgfqpoint{1.981936in}{1.946008in}}%
\pgfpathlineto{\pgfqpoint{1.959926in}{1.946668in}}%
\pgfpathlineto{\pgfqpoint{1.960762in}{1.974477in}}%
\pgfpathlineto{\pgfqpoint{1.960057in}{1.995378in}}%
\pgfpathlineto{\pgfqpoint{2.007393in}{1.994001in}}%
\pgfpathlineto{\pgfqpoint{2.006954in}{2.014902in}}%
\pgfpathlineto{\pgfqpoint{2.049145in}{2.014243in}}%
\pgfpathlineto{\pgfqpoint{2.048844in}{1.986478in}}%
\pgfpathlineto{\pgfqpoint{2.041913in}{1.986517in}}%
\pgfpathlineto{\pgfqpoint{2.042858in}{1.972645in}}%
\pgfpathlineto{\pgfqpoint{2.042559in}{1.951936in}}%
\pgfpathclose%
\pgfusepath{fill}%
\end{pgfscope}%
\begin{pgfscope}%
\pgfpathrectangle{\pgfqpoint{0.100000in}{0.100000in}}{\pgfqpoint{3.608454in}{2.310000in}}%
\pgfusepath{clip}%
\pgfsetbuttcap%
\pgfsetmiterjoin%
\definecolor{currentfill}{rgb}{0.000000,0.854902,0.572549}%
\pgfsetfillcolor{currentfill}%
\pgfsetlinewidth{0.000000pt}%
\definecolor{currentstroke}{rgb}{0.000000,0.000000,0.000000}%
\pgfsetstrokecolor{currentstroke}%
\pgfsetstrokeopacity{0.000000}%
\pgfsetdash{}{0pt}%
\pgfpathmoveto{\pgfqpoint{2.738013in}{1.786827in}}%
\pgfpathlineto{\pgfqpoint{2.697115in}{1.781860in}}%
\pgfpathlineto{\pgfqpoint{2.683319in}{1.780357in}}%
\pgfpathlineto{\pgfqpoint{2.680280in}{1.807943in}}%
\pgfpathlineto{\pgfqpoint{2.677389in}{1.835395in}}%
\pgfpathlineto{\pgfqpoint{2.674595in}{1.855705in}}%
\pgfpathlineto{\pgfqpoint{2.681411in}{1.856342in}}%
\pgfpathlineto{\pgfqpoint{2.680662in}{1.863242in}}%
\pgfpathlineto{\pgfqpoint{2.707810in}{1.866142in}}%
\pgfpathlineto{\pgfqpoint{2.728097in}{1.868854in}}%
\pgfpathlineto{\pgfqpoint{2.729298in}{1.855252in}}%
\pgfpathlineto{\pgfqpoint{2.738013in}{1.786827in}}%
\pgfpathclose%
\pgfusepath{fill}%
\end{pgfscope}%
\begin{pgfscope}%
\pgfpathrectangle{\pgfqpoint{0.100000in}{0.100000in}}{\pgfqpoint{3.608454in}{2.310000in}}%
\pgfusepath{clip}%
\pgfsetbuttcap%
\pgfsetmiterjoin%
\definecolor{currentfill}{rgb}{0.000000,0.815686,0.592157}%
\pgfsetfillcolor{currentfill}%
\pgfsetlinewidth{0.000000pt}%
\definecolor{currentstroke}{rgb}{0.000000,0.000000,0.000000}%
\pgfsetstrokecolor{currentstroke}%
\pgfsetstrokeopacity{0.000000}%
\pgfsetdash{}{0pt}%
\pgfpathmoveto{\pgfqpoint{3.069493in}{0.458265in}}%
\pgfpathlineto{\pgfqpoint{3.072400in}{0.437158in}}%
\pgfpathlineto{\pgfqpoint{3.037010in}{0.432074in}}%
\pgfpathlineto{\pgfqpoint{3.036461in}{0.440065in}}%
\pgfpathlineto{\pgfqpoint{3.030688in}{0.445529in}}%
\pgfpathlineto{\pgfqpoint{3.027406in}{0.441545in}}%
\pgfpathlineto{\pgfqpoint{3.030649in}{0.432623in}}%
\pgfpathlineto{\pgfqpoint{3.024312in}{0.431402in}}%
\pgfpathlineto{\pgfqpoint{3.021799in}{0.431432in}}%
\pgfpathlineto{\pgfqpoint{3.006353in}{0.451398in}}%
\pgfpathlineto{\pgfqpoint{2.995081in}{0.469136in}}%
\pgfpathlineto{\pgfqpoint{2.984705in}{0.479666in}}%
\pgfpathlineto{\pgfqpoint{2.986857in}{0.487228in}}%
\pgfpathlineto{\pgfqpoint{2.992126in}{0.496993in}}%
\pgfpathlineto{\pgfqpoint{3.027503in}{0.502091in}}%
\pgfpathlineto{\pgfqpoint{3.030929in}{0.477492in}}%
\pgfpathlineto{\pgfqpoint{3.065776in}{0.482786in}}%
\pgfpathlineto{\pgfqpoint{3.069493in}{0.458265in}}%
\pgfpathclose%
\pgfusepath{fill}%
\end{pgfscope}%
\begin{pgfscope}%
\pgfpathrectangle{\pgfqpoint{0.100000in}{0.100000in}}{\pgfqpoint{3.608454in}{2.310000in}}%
\pgfusepath{clip}%
\pgfsetbuttcap%
\pgfsetmiterjoin%
\definecolor{currentfill}{rgb}{0.000000,0.662745,0.668627}%
\pgfsetfillcolor{currentfill}%
\pgfsetlinewidth{0.000000pt}%
\definecolor{currentstroke}{rgb}{0.000000,0.000000,0.000000}%
\pgfsetstrokecolor{currentstroke}%
\pgfsetstrokeopacity{0.000000}%
\pgfsetdash{}{0pt}%
\pgfpathmoveto{\pgfqpoint{3.046524in}{1.607153in}}%
\pgfpathlineto{\pgfqpoint{3.045476in}{1.613591in}}%
\pgfpathlineto{\pgfqpoint{3.051910in}{1.617021in}}%
\pgfpathlineto{\pgfqpoint{3.050721in}{1.630639in}}%
\pgfpathlineto{\pgfqpoint{3.078140in}{1.636089in}}%
\pgfpathlineto{\pgfqpoint{3.094591in}{1.638194in}}%
\pgfpathlineto{\pgfqpoint{3.105576in}{1.628770in}}%
\pgfpathlineto{\pgfqpoint{3.109153in}{1.629395in}}%
\pgfpathlineto{\pgfqpoint{3.110757in}{1.621055in}}%
\pgfpathlineto{\pgfqpoint{3.106858in}{1.608202in}}%
\pgfpathlineto{\pgfqpoint{3.068655in}{1.604850in}}%
\pgfpathlineto{\pgfqpoint{3.054170in}{1.610224in}}%
\pgfpathlineto{\pgfqpoint{3.046524in}{1.607153in}}%
\pgfpathclose%
\pgfusepath{fill}%
\end{pgfscope}%
\begin{pgfscope}%
\pgfpathrectangle{\pgfqpoint{0.100000in}{0.100000in}}{\pgfqpoint{3.608454in}{2.310000in}}%
\pgfusepath{clip}%
\pgfsetbuttcap%
\pgfsetmiterjoin%
\definecolor{currentfill}{rgb}{0.000000,0.603922,0.698039}%
\pgfsetfillcolor{currentfill}%
\pgfsetlinewidth{0.000000pt}%
\definecolor{currentstroke}{rgb}{0.000000,0.000000,0.000000}%
\pgfsetstrokecolor{currentstroke}%
\pgfsetstrokeopacity{0.000000}%
\pgfsetdash{}{0pt}%
\pgfpathmoveto{\pgfqpoint{2.583434in}{1.352723in}}%
\pgfpathlineto{\pgfqpoint{2.566176in}{1.350839in}}%
\pgfpathlineto{\pgfqpoint{2.562407in}{1.357174in}}%
\pgfpathlineto{\pgfqpoint{2.562133in}{1.362791in}}%
\pgfpathlineto{\pgfqpoint{2.558058in}{1.370400in}}%
\pgfpathlineto{\pgfqpoint{2.560407in}{1.373999in}}%
\pgfpathlineto{\pgfqpoint{2.558464in}{1.382618in}}%
\pgfpathlineto{\pgfqpoint{2.562752in}{1.385889in}}%
\pgfpathlineto{\pgfqpoint{2.559183in}{1.428045in}}%
\pgfpathlineto{\pgfqpoint{2.557572in}{1.448924in}}%
\pgfpathlineto{\pgfqpoint{2.565221in}{1.447932in}}%
\pgfpathlineto{\pgfqpoint{2.565557in}{1.434073in}}%
\pgfpathlineto{\pgfqpoint{2.585430in}{1.435815in}}%
\pgfpathlineto{\pgfqpoint{2.583559in}{1.456426in}}%
\pgfpathlineto{\pgfqpoint{2.607399in}{1.458524in}}%
\pgfpathlineto{\pgfqpoint{2.607655in}{1.455712in}}%
\pgfpathlineto{\pgfqpoint{2.612117in}{1.412594in}}%
\pgfpathlineto{\pgfqpoint{2.614715in}{1.406613in}}%
\pgfpathlineto{\pgfqpoint{2.613308in}{1.399991in}}%
\pgfpathlineto{\pgfqpoint{2.597905in}{1.398852in}}%
\pgfpathlineto{\pgfqpoint{2.598672in}{1.388501in}}%
\pgfpathlineto{\pgfqpoint{2.591897in}{1.387949in}}%
\pgfpathlineto{\pgfqpoint{2.593047in}{1.374201in}}%
\pgfpathlineto{\pgfqpoint{2.581657in}{1.373507in}}%
\pgfpathlineto{\pgfqpoint{2.583434in}{1.352723in}}%
\pgfpathclose%
\pgfusepath{fill}%
\end{pgfscope}%
\begin{pgfscope}%
\pgfpathrectangle{\pgfqpoint{0.100000in}{0.100000in}}{\pgfqpoint{3.608454in}{2.310000in}}%
\pgfusepath{clip}%
\pgfsetbuttcap%
\pgfsetmiterjoin%
\definecolor{currentfill}{rgb}{0.000000,0.341176,0.829412}%
\pgfsetfillcolor{currentfill}%
\pgfsetlinewidth{0.000000pt}%
\definecolor{currentstroke}{rgb}{0.000000,0.000000,0.000000}%
\pgfsetstrokecolor{currentstroke}%
\pgfsetstrokeopacity{0.000000}%
\pgfsetdash{}{0pt}%
\pgfpathmoveto{\pgfqpoint{1.580064in}{1.651542in}}%
\pgfpathlineto{\pgfqpoint{1.571835in}{1.569055in}}%
\pgfpathlineto{\pgfqpoint{1.536486in}{1.572765in}}%
\pgfpathlineto{\pgfqpoint{1.537205in}{1.579689in}}%
\pgfpathlineto{\pgfqpoint{1.500517in}{1.584019in}}%
\pgfpathlineto{\pgfqpoint{1.503839in}{1.609878in}}%
\pgfpathlineto{\pgfqpoint{1.507108in}{1.644984in}}%
\pgfpathlineto{\pgfqpoint{1.508517in}{1.658724in}}%
\pgfpathlineto{\pgfqpoint{1.531372in}{1.656425in}}%
\pgfpathlineto{\pgfqpoint{1.580064in}{1.651542in}}%
\pgfpathclose%
\pgfusepath{fill}%
\end{pgfscope}%
\begin{pgfscope}%
\pgfpathrectangle{\pgfqpoint{0.100000in}{0.100000in}}{\pgfqpoint{3.608454in}{2.310000in}}%
\pgfusepath{clip}%
\pgfsetbuttcap%
\pgfsetmiterjoin%
\definecolor{currentfill}{rgb}{0.000000,0.615686,0.692157}%
\pgfsetfillcolor{currentfill}%
\pgfsetlinewidth{0.000000pt}%
\definecolor{currentstroke}{rgb}{0.000000,0.000000,0.000000}%
\pgfsetstrokecolor{currentstroke}%
\pgfsetstrokeopacity{0.000000}%
\pgfsetdash{}{0pt}%
\pgfpathmoveto{\pgfqpoint{1.651469in}{1.020799in}}%
\pgfpathlineto{\pgfqpoint{1.685686in}{1.018201in}}%
\pgfpathlineto{\pgfqpoint{1.683230in}{0.983785in}}%
\pgfpathlineto{\pgfqpoint{1.711338in}{0.981859in}}%
\pgfpathlineto{\pgfqpoint{1.708835in}{0.943953in}}%
\pgfpathlineto{\pgfqpoint{1.640100in}{0.948213in}}%
\pgfpathlineto{\pgfqpoint{1.642812in}{0.986873in}}%
\pgfpathlineto{\pgfqpoint{1.648845in}{0.986397in}}%
\pgfpathlineto{\pgfqpoint{1.651469in}{1.020799in}}%
\pgfpathclose%
\pgfusepath{fill}%
\end{pgfscope}%
\begin{pgfscope}%
\pgfpathrectangle{\pgfqpoint{0.100000in}{0.100000in}}{\pgfqpoint{3.608454in}{2.310000in}}%
\pgfusepath{clip}%
\pgfsetbuttcap%
\pgfsetmiterjoin%
\definecolor{currentfill}{rgb}{0.000000,0.156863,0.921569}%
\pgfsetfillcolor{currentfill}%
\pgfsetlinewidth{0.000000pt}%
\definecolor{currentstroke}{rgb}{0.000000,0.000000,0.000000}%
\pgfsetstrokecolor{currentstroke}%
\pgfsetstrokeopacity{0.000000}%
\pgfsetdash{}{0pt}%
\pgfpathmoveto{\pgfqpoint{1.396241in}{1.384895in}}%
\pgfpathlineto{\pgfqpoint{1.393282in}{1.388976in}}%
\pgfpathlineto{\pgfqpoint{1.387549in}{1.389567in}}%
\pgfpathlineto{\pgfqpoint{1.383841in}{1.385038in}}%
\pgfpathlineto{\pgfqpoint{1.377365in}{1.386918in}}%
\pgfpathlineto{\pgfqpoint{1.369057in}{1.398085in}}%
\pgfpathlineto{\pgfqpoint{1.355918in}{1.399849in}}%
\pgfpathlineto{\pgfqpoint{1.352191in}{1.404691in}}%
\pgfpathlineto{\pgfqpoint{1.350593in}{1.411552in}}%
\pgfpathlineto{\pgfqpoint{1.346932in}{1.416687in}}%
\pgfpathlineto{\pgfqpoint{1.349566in}{1.420460in}}%
\pgfpathlineto{\pgfqpoint{1.290367in}{1.428918in}}%
\pgfpathlineto{\pgfqpoint{1.251645in}{1.434868in}}%
\pgfpathlineto{\pgfqpoint{1.253203in}{1.445105in}}%
\pgfpathlineto{\pgfqpoint{1.255196in}{1.457818in}}%
\pgfpathlineto{\pgfqpoint{1.284941in}{1.452441in}}%
\pgfpathlineto{\pgfqpoint{1.285446in}{1.455840in}}%
\pgfpathlineto{\pgfqpoint{1.322533in}{1.450461in}}%
\pgfpathlineto{\pgfqpoint{1.324000in}{1.460674in}}%
\pgfpathlineto{\pgfqpoint{1.354499in}{1.456659in}}%
\pgfpathlineto{\pgfqpoint{1.355468in}{1.463655in}}%
\pgfpathlineto{\pgfqpoint{1.362220in}{1.462510in}}%
\pgfpathlineto{\pgfqpoint{1.364204in}{1.476264in}}%
\pgfpathlineto{\pgfqpoint{1.380971in}{1.474066in}}%
\pgfpathlineto{\pgfqpoint{1.379506in}{1.460495in}}%
\pgfpathlineto{\pgfqpoint{1.403922in}{1.457766in}}%
\pgfpathlineto{\pgfqpoint{1.426636in}{1.453930in}}%
\pgfpathlineto{\pgfqpoint{1.435412in}{1.444400in}}%
\pgfpathlineto{\pgfqpoint{1.437752in}{1.434786in}}%
\pgfpathlineto{\pgfqpoint{1.444005in}{1.434780in}}%
\pgfpathlineto{\pgfqpoint{1.451897in}{1.428337in}}%
\pgfpathlineto{\pgfqpoint{1.439656in}{1.412644in}}%
\pgfpathlineto{\pgfqpoint{1.424568in}{1.406130in}}%
\pgfpathlineto{\pgfqpoint{1.423585in}{1.392666in}}%
\pgfpathlineto{\pgfqpoint{1.421683in}{1.386315in}}%
\pgfpathlineto{\pgfqpoint{1.398138in}{1.389426in}}%
\pgfpathlineto{\pgfqpoint{1.396241in}{1.384895in}}%
\pgfpathclose%
\pgfusepath{fill}%
\end{pgfscope}%
\begin{pgfscope}%
\pgfpathrectangle{\pgfqpoint{0.100000in}{0.100000in}}{\pgfqpoint{3.608454in}{2.310000in}}%
\pgfusepath{clip}%
\pgfsetbuttcap%
\pgfsetmiterjoin%
\definecolor{currentfill}{rgb}{0.000000,0.603922,0.698039}%
\pgfsetfillcolor{currentfill}%
\pgfsetlinewidth{0.000000pt}%
\definecolor{currentstroke}{rgb}{0.000000,0.000000,0.000000}%
\pgfsetstrokecolor{currentstroke}%
\pgfsetstrokeopacity{0.000000}%
\pgfsetdash{}{0pt}%
\pgfpathmoveto{\pgfqpoint{3.091930in}{0.511405in}}%
\pgfpathlineto{\pgfqpoint{3.088462in}{0.501164in}}%
\pgfpathlineto{\pgfqpoint{3.097174in}{0.491133in}}%
\pgfpathlineto{\pgfqpoint{3.100740in}{0.491314in}}%
\pgfpathlineto{\pgfqpoint{3.109319in}{0.479362in}}%
\pgfpathlineto{\pgfqpoint{3.095343in}{0.476565in}}%
\pgfpathlineto{\pgfqpoint{3.096465in}{0.469557in}}%
\pgfpathlineto{\pgfqpoint{3.089414in}{0.468427in}}%
\pgfpathlineto{\pgfqpoint{3.090529in}{0.461360in}}%
\pgfpathlineto{\pgfqpoint{3.069493in}{0.458265in}}%
\pgfpathlineto{\pgfqpoint{3.065776in}{0.482786in}}%
\pgfpathlineto{\pgfqpoint{3.030929in}{0.477492in}}%
\pgfpathlineto{\pgfqpoint{3.027503in}{0.502091in}}%
\pgfpathlineto{\pgfqpoint{3.021367in}{0.543921in}}%
\pgfpathlineto{\pgfqpoint{3.016866in}{0.550384in}}%
\pgfpathlineto{\pgfqpoint{3.019784in}{0.555093in}}%
\pgfpathlineto{\pgfqpoint{3.026291in}{0.558732in}}%
\pgfpathlineto{\pgfqpoint{3.047294in}{0.561961in}}%
\pgfpathlineto{\pgfqpoint{3.048333in}{0.554969in}}%
\pgfpathlineto{\pgfqpoint{3.055318in}{0.556018in}}%
\pgfpathlineto{\pgfqpoint{3.059031in}{0.547094in}}%
\pgfpathlineto{\pgfqpoint{3.070237in}{0.538537in}}%
\pgfpathlineto{\pgfqpoint{3.076539in}{0.531713in}}%
\pgfpathlineto{\pgfqpoint{3.077818in}{0.527142in}}%
\pgfpathlineto{\pgfqpoint{3.084501in}{0.525466in}}%
\pgfpathlineto{\pgfqpoint{3.090421in}{0.517236in}}%
\pgfpathlineto{\pgfqpoint{3.091930in}{0.511405in}}%
\pgfpathclose%
\pgfusepath{fill}%
\end{pgfscope}%
\begin{pgfscope}%
\pgfpathrectangle{\pgfqpoint{0.100000in}{0.100000in}}{\pgfqpoint{3.608454in}{2.310000in}}%
\pgfusepath{clip}%
\pgfsetbuttcap%
\pgfsetmiterjoin%
\definecolor{currentfill}{rgb}{0.000000,0.835294,0.582353}%
\pgfsetfillcolor{currentfill}%
\pgfsetlinewidth{0.000000pt}%
\definecolor{currentstroke}{rgb}{0.000000,0.000000,0.000000}%
\pgfsetstrokecolor{currentstroke}%
\pgfsetstrokeopacity{0.000000}%
\pgfsetdash{}{0pt}%
\pgfpathmoveto{\pgfqpoint{0.509532in}{1.308555in}}%
\pgfpathlineto{\pgfqpoint{0.550101in}{1.297200in}}%
\pgfpathlineto{\pgfqpoint{0.606581in}{1.281871in}}%
\pgfpathlineto{\pgfqpoint{0.645007in}{1.271759in}}%
\pgfpathlineto{\pgfqpoint{0.668493in}{1.266395in}}%
\pgfpathlineto{\pgfqpoint{0.649520in}{1.191532in}}%
\pgfpathlineto{\pgfqpoint{0.598059in}{1.204495in}}%
\pgfpathlineto{\pgfqpoint{0.547967in}{1.217656in}}%
\pgfpathlineto{\pgfqpoint{0.547247in}{1.223234in}}%
\pgfpathlineto{\pgfqpoint{0.537333in}{1.227732in}}%
\pgfpathlineto{\pgfqpoint{0.539132in}{1.241671in}}%
\pgfpathlineto{\pgfqpoint{0.533846in}{1.244020in}}%
\pgfpathlineto{\pgfqpoint{0.536255in}{1.250941in}}%
\pgfpathlineto{\pgfqpoint{0.529070in}{1.252510in}}%
\pgfpathlineto{\pgfqpoint{0.530958in}{1.259199in}}%
\pgfpathlineto{\pgfqpoint{0.522085in}{1.261761in}}%
\pgfpathlineto{\pgfqpoint{0.523955in}{1.268417in}}%
\pgfpathlineto{\pgfqpoint{0.519557in}{1.269679in}}%
\pgfpathlineto{\pgfqpoint{0.521448in}{1.276381in}}%
\pgfpathlineto{\pgfqpoint{0.514797in}{1.280686in}}%
\pgfpathlineto{\pgfqpoint{0.510504in}{1.286661in}}%
\pgfpathlineto{\pgfqpoint{0.512448in}{1.293350in}}%
\pgfpathlineto{\pgfqpoint{0.505758in}{1.295236in}}%
\pgfpathlineto{\pgfqpoint{0.509532in}{1.308555in}}%
\pgfpathclose%
\pgfusepath{fill}%
\end{pgfscope}%
\begin{pgfscope}%
\pgfpathrectangle{\pgfqpoint{0.100000in}{0.100000in}}{\pgfqpoint{3.608454in}{2.310000in}}%
\pgfusepath{clip}%
\pgfsetbuttcap%
\pgfsetmiterjoin%
\definecolor{currentfill}{rgb}{0.000000,0.427451,0.786275}%
\pgfsetfillcolor{currentfill}%
\pgfsetlinewidth{0.000000pt}%
\definecolor{currentstroke}{rgb}{0.000000,0.000000,0.000000}%
\pgfsetstrokecolor{currentstroke}%
\pgfsetstrokeopacity{0.000000}%
\pgfsetdash{}{0pt}%
\pgfpathmoveto{\pgfqpoint{2.193447in}{1.465250in}}%
\pgfpathlineto{\pgfqpoint{2.180394in}{1.464846in}}%
\pgfpathlineto{\pgfqpoint{2.138550in}{1.463912in}}%
\pgfpathlineto{\pgfqpoint{2.138316in}{1.510307in}}%
\pgfpathlineto{\pgfqpoint{2.172375in}{1.510519in}}%
\pgfpathlineto{\pgfqpoint{2.192689in}{1.511198in}}%
\pgfpathlineto{\pgfqpoint{2.193447in}{1.465250in}}%
\pgfpathclose%
\pgfusepath{fill}%
\end{pgfscope}%
\begin{pgfscope}%
\pgfpathrectangle{\pgfqpoint{0.100000in}{0.100000in}}{\pgfqpoint{3.608454in}{2.310000in}}%
\pgfusepath{clip}%
\pgfsetbuttcap%
\pgfsetmiterjoin%
\definecolor{currentfill}{rgb}{0.000000,0.709804,0.645098}%
\pgfsetfillcolor{currentfill}%
\pgfsetlinewidth{0.000000pt}%
\definecolor{currentstroke}{rgb}{0.000000,0.000000,0.000000}%
\pgfsetstrokecolor{currentstroke}%
\pgfsetstrokeopacity{0.000000}%
\pgfsetdash{}{0pt}%
\pgfpathmoveto{\pgfqpoint{2.724860in}{1.203323in}}%
\pgfpathlineto{\pgfqpoint{2.723224in}{1.208766in}}%
\pgfpathlineto{\pgfqpoint{2.709289in}{1.208321in}}%
\pgfpathlineto{\pgfqpoint{2.705489in}{1.213403in}}%
\pgfpathlineto{\pgfqpoint{2.703168in}{1.221287in}}%
\pgfpathlineto{\pgfqpoint{2.696385in}{1.221833in}}%
\pgfpathlineto{\pgfqpoint{2.692552in}{1.225974in}}%
\pgfpathlineto{\pgfqpoint{2.690908in}{1.235308in}}%
\pgfpathlineto{\pgfqpoint{2.692446in}{1.245040in}}%
\pgfpathlineto{\pgfqpoint{2.693473in}{1.249077in}}%
\pgfpathlineto{\pgfqpoint{2.704025in}{1.249697in}}%
\pgfpathlineto{\pgfqpoint{2.711292in}{1.250974in}}%
\pgfpathlineto{\pgfqpoint{2.718150in}{1.247622in}}%
\pgfpathlineto{\pgfqpoint{2.722412in}{1.251628in}}%
\pgfpathlineto{\pgfqpoint{2.730780in}{1.248329in}}%
\pgfpathlineto{\pgfqpoint{2.729948in}{1.258824in}}%
\pgfpathlineto{\pgfqpoint{2.741472in}{1.258776in}}%
\pgfpathlineto{\pgfqpoint{2.742755in}{1.250848in}}%
\pgfpathlineto{\pgfqpoint{2.751350in}{1.246500in}}%
\pgfpathlineto{\pgfqpoint{2.752556in}{1.236924in}}%
\pgfpathlineto{\pgfqpoint{2.742337in}{1.226077in}}%
\pgfpathlineto{\pgfqpoint{2.742647in}{1.220616in}}%
\pgfpathlineto{\pgfqpoint{2.747524in}{1.217198in}}%
\pgfpathlineto{\pgfqpoint{2.731198in}{1.201857in}}%
\pgfpathlineto{\pgfqpoint{2.724860in}{1.203323in}}%
\pgfpathclose%
\pgfusepath{fill}%
\end{pgfscope}%
\begin{pgfscope}%
\pgfpathrectangle{\pgfqpoint{0.100000in}{0.100000in}}{\pgfqpoint{3.608454in}{2.310000in}}%
\pgfusepath{clip}%
\pgfsetbuttcap%
\pgfsetmiterjoin%
\definecolor{currentfill}{rgb}{0.000000,0.505882,0.747059}%
\pgfsetfillcolor{currentfill}%
\pgfsetlinewidth{0.000000pt}%
\definecolor{currentstroke}{rgb}{0.000000,0.000000,0.000000}%
\pgfsetstrokecolor{currentstroke}%
\pgfsetstrokeopacity{0.000000}%
\pgfsetdash{}{0pt}%
\pgfpathmoveto{\pgfqpoint{1.711338in}{0.981859in}}%
\pgfpathlineto{\pgfqpoint{1.683230in}{0.983785in}}%
\pgfpathlineto{\pgfqpoint{1.685686in}{1.018201in}}%
\pgfpathlineto{\pgfqpoint{1.710432in}{1.016551in}}%
\pgfpathlineto{\pgfqpoint{1.713023in}{1.050837in}}%
\pgfpathlineto{\pgfqpoint{1.748391in}{1.048651in}}%
\pgfpathlineto{\pgfqpoint{1.746176in}{1.014219in}}%
\pgfpathlineto{\pgfqpoint{1.754289in}{1.013713in}}%
\pgfpathlineto{\pgfqpoint{1.752117in}{0.979371in}}%
\pgfpathlineto{\pgfqpoint{1.745610in}{0.979809in}}%
\pgfpathlineto{\pgfqpoint{1.711338in}{0.981859in}}%
\pgfpathclose%
\pgfusepath{fill}%
\end{pgfscope}%
\begin{pgfscope}%
\pgfpathrectangle{\pgfqpoint{0.100000in}{0.100000in}}{\pgfqpoint{3.608454in}{2.310000in}}%
\pgfusepath{clip}%
\pgfsetbuttcap%
\pgfsetmiterjoin%
\definecolor{currentfill}{rgb}{0.000000,0.650980,0.674510}%
\pgfsetfillcolor{currentfill}%
\pgfsetlinewidth{0.000000pt}%
\definecolor{currentstroke}{rgb}{0.000000,0.000000,0.000000}%
\pgfsetstrokecolor{currentstroke}%
\pgfsetstrokeopacity{0.000000}%
\pgfsetdash{}{0pt}%
\pgfpathmoveto{\pgfqpoint{1.666412in}{1.249548in}}%
\pgfpathlineto{\pgfqpoint{1.664847in}{1.229439in}}%
\pgfpathlineto{\pgfqpoint{1.662385in}{1.198210in}}%
\pgfpathlineto{\pgfqpoint{1.621102in}{1.201730in}}%
\pgfpathlineto{\pgfqpoint{1.602002in}{1.203777in}}%
\pgfpathlineto{\pgfqpoint{1.596724in}{1.204258in}}%
\pgfpathlineto{\pgfqpoint{1.601877in}{1.254860in}}%
\pgfpathlineto{\pgfqpoint{1.666412in}{1.249548in}}%
\pgfpathclose%
\pgfusepath{fill}%
\end{pgfscope}%
\begin{pgfscope}%
\pgfpathrectangle{\pgfqpoint{0.100000in}{0.100000in}}{\pgfqpoint{3.608454in}{2.310000in}}%
\pgfusepath{clip}%
\pgfsetbuttcap%
\pgfsetmiterjoin%
\definecolor{currentfill}{rgb}{0.000000,0.309804,0.845098}%
\pgfsetfillcolor{currentfill}%
\pgfsetlinewidth{0.000000pt}%
\definecolor{currentstroke}{rgb}{0.000000,0.000000,0.000000}%
\pgfsetstrokecolor{currentstroke}%
\pgfsetstrokeopacity{0.000000}%
\pgfsetdash{}{0pt}%
\pgfpathmoveto{\pgfqpoint{2.685775in}{1.104241in}}%
\pgfpathlineto{\pgfqpoint{2.681536in}{1.102539in}}%
\pgfpathlineto{\pgfqpoint{2.668887in}{1.104876in}}%
\pgfpathlineto{\pgfqpoint{2.654621in}{1.100124in}}%
\pgfpathlineto{\pgfqpoint{2.651600in}{1.103191in}}%
\pgfpathlineto{\pgfqpoint{2.643683in}{1.102257in}}%
\pgfpathlineto{\pgfqpoint{2.632392in}{1.103973in}}%
\pgfpathlineto{\pgfqpoint{2.605693in}{1.111968in}}%
\pgfpathlineto{\pgfqpoint{2.605651in}{1.120589in}}%
\pgfpathlineto{\pgfqpoint{2.606773in}{1.136211in}}%
\pgfpathlineto{\pgfqpoint{2.604989in}{1.140525in}}%
\pgfpathlineto{\pgfqpoint{2.606315in}{1.148332in}}%
\pgfpathlineto{\pgfqpoint{2.597965in}{1.148745in}}%
\pgfpathlineto{\pgfqpoint{2.599908in}{1.153357in}}%
\pgfpathlineto{\pgfqpoint{2.607032in}{1.160645in}}%
\pgfpathlineto{\pgfqpoint{2.606663in}{1.174945in}}%
\pgfpathlineto{\pgfqpoint{2.610086in}{1.175304in}}%
\pgfpathlineto{\pgfqpoint{2.639785in}{1.178722in}}%
\pgfpathlineto{\pgfqpoint{2.656122in}{1.179965in}}%
\pgfpathlineto{\pgfqpoint{2.678723in}{1.180579in}}%
\pgfpathlineto{\pgfqpoint{2.690217in}{1.180528in}}%
\pgfpathlineto{\pgfqpoint{2.690027in}{1.171438in}}%
\pgfpathlineto{\pgfqpoint{2.689646in}{1.154594in}}%
\pgfpathlineto{\pgfqpoint{2.693285in}{1.148452in}}%
\pgfpathlineto{\pgfqpoint{2.693376in}{1.142517in}}%
\pgfpathlineto{\pgfqpoint{2.690045in}{1.139163in}}%
\pgfpathlineto{\pgfqpoint{2.683825in}{1.139915in}}%
\pgfpathlineto{\pgfqpoint{2.678100in}{1.135237in}}%
\pgfpathlineto{\pgfqpoint{2.681516in}{1.127762in}}%
\pgfpathlineto{\pgfqpoint{2.689025in}{1.121466in}}%
\pgfpathlineto{\pgfqpoint{2.685202in}{1.110108in}}%
\pgfpathlineto{\pgfqpoint{2.685775in}{1.104241in}}%
\pgfpathclose%
\pgfusepath{fill}%
\end{pgfscope}%
\begin{pgfscope}%
\pgfpathrectangle{\pgfqpoint{0.100000in}{0.100000in}}{\pgfqpoint{3.608454in}{2.310000in}}%
\pgfusepath{clip}%
\pgfsetbuttcap%
\pgfsetmiterjoin%
\definecolor{currentfill}{rgb}{0.000000,0.752941,0.623529}%
\pgfsetfillcolor{currentfill}%
\pgfsetlinewidth{0.000000pt}%
\definecolor{currentstroke}{rgb}{0.000000,0.000000,0.000000}%
\pgfsetstrokecolor{currentstroke}%
\pgfsetstrokeopacity{0.000000}%
\pgfsetdash{}{0pt}%
\pgfpathmoveto{\pgfqpoint{0.486008in}{1.393700in}}%
\pgfpathlineto{\pgfqpoint{0.474622in}{1.404992in}}%
\pgfpathlineto{\pgfqpoint{0.471004in}{1.413448in}}%
\pgfpathlineto{\pgfqpoint{0.475697in}{1.423778in}}%
\pgfpathlineto{\pgfqpoint{0.476039in}{1.431016in}}%
\pgfpathlineto{\pgfqpoint{0.465842in}{1.434190in}}%
\pgfpathlineto{\pgfqpoint{0.465818in}{1.444533in}}%
\pgfpathlineto{\pgfqpoint{0.470704in}{1.451769in}}%
\pgfpathlineto{\pgfqpoint{0.469193in}{1.459876in}}%
\pgfpathlineto{\pgfqpoint{0.469229in}{1.459834in}}%
\pgfpathlineto{\pgfqpoint{0.489989in}{1.471204in}}%
\pgfpathlineto{\pgfqpoint{0.504262in}{1.472446in}}%
\pgfpathlineto{\pgfqpoint{0.508101in}{1.469498in}}%
\pgfpathlineto{\pgfqpoint{0.515295in}{1.495292in}}%
\pgfpathlineto{\pgfqpoint{0.526181in}{1.471930in}}%
\pgfpathlineto{\pgfqpoint{0.537164in}{1.478720in}}%
\pgfpathlineto{\pgfqpoint{0.550096in}{1.491227in}}%
\pgfpathlineto{\pgfqpoint{0.566537in}{1.504878in}}%
\pgfpathlineto{\pgfqpoint{0.577147in}{1.506768in}}%
\pgfpathlineto{\pgfqpoint{0.583557in}{1.498623in}}%
\pgfpathlineto{\pgfqpoint{0.592836in}{1.501027in}}%
\pgfpathlineto{\pgfqpoint{0.597596in}{1.492295in}}%
\pgfpathlineto{\pgfqpoint{0.596004in}{1.489421in}}%
\pgfpathlineto{\pgfqpoint{0.597769in}{1.478364in}}%
\pgfpathlineto{\pgfqpoint{0.610025in}{1.468870in}}%
\pgfpathlineto{\pgfqpoint{0.609428in}{1.457861in}}%
\pgfpathlineto{\pgfqpoint{0.613111in}{1.453862in}}%
\pgfpathlineto{\pgfqpoint{0.612588in}{1.444548in}}%
\pgfpathlineto{\pgfqpoint{0.607467in}{1.441382in}}%
\pgfpathlineto{\pgfqpoint{0.605864in}{1.444998in}}%
\pgfpathlineto{\pgfqpoint{0.584724in}{1.432975in}}%
\pgfpathlineto{\pgfqpoint{0.583298in}{1.427984in}}%
\pgfpathlineto{\pgfqpoint{0.577622in}{1.423252in}}%
\pgfpathlineto{\pgfqpoint{0.570915in}{1.425063in}}%
\pgfpathlineto{\pgfqpoint{0.548331in}{1.412190in}}%
\pgfpathlineto{\pgfqpoint{0.533257in}{1.413969in}}%
\pgfpathlineto{\pgfqpoint{0.520707in}{1.412942in}}%
\pgfpathlineto{\pgfqpoint{0.486008in}{1.393700in}}%
\pgfpathclose%
\pgfusepath{fill}%
\end{pgfscope}%
\begin{pgfscope}%
\pgfpathrectangle{\pgfqpoint{0.100000in}{0.100000in}}{\pgfqpoint{3.608454in}{2.310000in}}%
\pgfusepath{clip}%
\pgfsetbuttcap%
\pgfsetmiterjoin%
\definecolor{currentfill}{rgb}{0.000000,0.431373,0.784314}%
\pgfsetfillcolor{currentfill}%
\pgfsetlinewidth{0.000000pt}%
\definecolor{currentstroke}{rgb}{0.000000,0.000000,0.000000}%
\pgfsetstrokecolor{currentstroke}%
\pgfsetstrokeopacity{0.000000}%
\pgfsetdash{}{0pt}%
\pgfpathmoveto{\pgfqpoint{1.664670in}{1.909384in}}%
\pgfpathlineto{\pgfqpoint{1.606786in}{1.914838in}}%
\pgfpathlineto{\pgfqpoint{1.611520in}{1.962062in}}%
\pgfpathlineto{\pgfqpoint{1.624849in}{1.960734in}}%
\pgfpathlineto{\pgfqpoint{1.625532in}{1.967784in}}%
\pgfpathlineto{\pgfqpoint{1.635984in}{1.966799in}}%
\pgfpathlineto{\pgfqpoint{1.673140in}{1.963430in}}%
\pgfpathlineto{\pgfqpoint{1.670909in}{1.935733in}}%
\pgfpathlineto{\pgfqpoint{1.667012in}{1.936010in}}%
\pgfpathlineto{\pgfqpoint{1.664670in}{1.909384in}}%
\pgfpathclose%
\pgfusepath{fill}%
\end{pgfscope}%
\begin{pgfscope}%
\pgfpathrectangle{\pgfqpoint{0.100000in}{0.100000in}}{\pgfqpoint{3.608454in}{2.310000in}}%
\pgfusepath{clip}%
\pgfsetbuttcap%
\pgfsetmiterjoin%
\definecolor{currentfill}{rgb}{0.000000,0.521569,0.739216}%
\pgfsetfillcolor{currentfill}%
\pgfsetlinewidth{0.000000pt}%
\definecolor{currentstroke}{rgb}{0.000000,0.000000,0.000000}%
\pgfsetstrokecolor{currentstroke}%
\pgfsetstrokeopacity{0.000000}%
\pgfsetdash{}{0pt}%
\pgfpathmoveto{\pgfqpoint{3.282347in}{1.403895in}}%
\pgfpathlineto{\pgfqpoint{3.278306in}{1.404567in}}%
\pgfpathlineto{\pgfqpoint{3.267145in}{1.400703in}}%
\pgfpathlineto{\pgfqpoint{3.253828in}{1.416317in}}%
\pgfpathlineto{\pgfqpoint{3.255954in}{1.419237in}}%
\pgfpathlineto{\pgfqpoint{3.254537in}{1.428747in}}%
\pgfpathlineto{\pgfqpoint{3.261997in}{1.431811in}}%
\pgfpathlineto{\pgfqpoint{3.248879in}{1.439534in}}%
\pgfpathlineto{\pgfqpoint{3.254321in}{1.452877in}}%
\pgfpathlineto{\pgfqpoint{3.250356in}{1.457926in}}%
\pgfpathlineto{\pgfqpoint{3.250047in}{1.464723in}}%
\pgfpathlineto{\pgfqpoint{3.246140in}{1.469571in}}%
\pgfpathlineto{\pgfqpoint{3.249034in}{1.483689in}}%
\pgfpathlineto{\pgfqpoint{3.256327in}{1.489994in}}%
\pgfpathlineto{\pgfqpoint{3.266239in}{1.490550in}}%
\pgfpathlineto{\pgfqpoint{3.273108in}{1.493337in}}%
\pgfpathlineto{\pgfqpoint{3.274847in}{1.487194in}}%
\pgfpathlineto{\pgfqpoint{3.290599in}{1.431089in}}%
\pgfpathlineto{\pgfqpoint{3.284538in}{1.423341in}}%
\pgfpathlineto{\pgfqpoint{3.281272in}{1.408524in}}%
\pgfpathlineto{\pgfqpoint{3.282347in}{1.403895in}}%
\pgfpathclose%
\pgfusepath{fill}%
\end{pgfscope}%
\begin{pgfscope}%
\pgfpathrectangle{\pgfqpoint{0.100000in}{0.100000in}}{\pgfqpoint{3.608454in}{2.310000in}}%
\pgfusepath{clip}%
\pgfsetbuttcap%
\pgfsetmiterjoin%
\definecolor{currentfill}{rgb}{0.000000,0.443137,0.778431}%
\pgfsetfillcolor{currentfill}%
\pgfsetlinewidth{0.000000pt}%
\definecolor{currentstroke}{rgb}{0.000000,0.000000,0.000000}%
\pgfsetstrokecolor{currentstroke}%
\pgfsetstrokeopacity{0.000000}%
\pgfsetdash{}{0pt}%
\pgfpathmoveto{\pgfqpoint{2.529824in}{1.778725in}}%
\pgfpathlineto{\pgfqpoint{2.510243in}{1.777129in}}%
\pgfpathlineto{\pgfqpoint{2.510762in}{1.770224in}}%
\pgfpathlineto{\pgfqpoint{2.501961in}{1.769586in}}%
\pgfpathlineto{\pgfqpoint{2.493489in}{1.769026in}}%
\pgfpathlineto{\pgfqpoint{2.491553in}{1.796332in}}%
\pgfpathlineto{\pgfqpoint{2.488486in}{1.795943in}}%
\pgfpathlineto{\pgfqpoint{2.486722in}{1.817142in}}%
\pgfpathlineto{\pgfqpoint{2.473187in}{1.816374in}}%
\pgfpathlineto{\pgfqpoint{2.472001in}{1.837128in}}%
\pgfpathlineto{\pgfqpoint{2.463191in}{1.836540in}}%
\pgfpathlineto{\pgfqpoint{2.460529in}{1.843339in}}%
\pgfpathlineto{\pgfqpoint{2.459694in}{1.857127in}}%
\pgfpathlineto{\pgfqpoint{2.473620in}{1.857983in}}%
\pgfpathlineto{\pgfqpoint{2.480448in}{1.858288in}}%
\pgfpathlineto{\pgfqpoint{2.481279in}{1.844556in}}%
\pgfpathlineto{\pgfqpoint{2.488144in}{1.844796in}}%
\pgfpathlineto{\pgfqpoint{2.492447in}{1.838114in}}%
\pgfpathlineto{\pgfqpoint{2.492843in}{1.831235in}}%
\pgfpathlineto{\pgfqpoint{2.513293in}{1.828191in}}%
\pgfpathlineto{\pgfqpoint{2.502443in}{1.807863in}}%
\pgfpathlineto{\pgfqpoint{2.500063in}{1.795356in}}%
\pgfpathlineto{\pgfqpoint{2.505725in}{1.792829in}}%
\pgfpathlineto{\pgfqpoint{2.509822in}{1.800026in}}%
\pgfpathlineto{\pgfqpoint{2.514631in}{1.802211in}}%
\pgfpathlineto{\pgfqpoint{2.522578in}{1.818649in}}%
\pgfpathlineto{\pgfqpoint{2.529869in}{1.821187in}}%
\pgfpathlineto{\pgfqpoint{2.533595in}{1.825201in}}%
\pgfpathlineto{\pgfqpoint{2.540562in}{1.840166in}}%
\pgfpathlineto{\pgfqpoint{2.541258in}{1.846577in}}%
\pgfpathlineto{\pgfqpoint{2.552435in}{1.851077in}}%
\pgfpathlineto{\pgfqpoint{2.552379in}{1.842511in}}%
\pgfpathlineto{\pgfqpoint{2.545403in}{1.830966in}}%
\pgfpathlineto{\pgfqpoint{2.545230in}{1.823543in}}%
\pgfpathlineto{\pgfqpoint{2.541717in}{1.820773in}}%
\pgfpathlineto{\pgfqpoint{2.533894in}{1.801143in}}%
\pgfpathlineto{\pgfqpoint{2.529824in}{1.778725in}}%
\pgfpathclose%
\pgfusepath{fill}%
\end{pgfscope}%
\begin{pgfscope}%
\pgfpathrectangle{\pgfqpoint{0.100000in}{0.100000in}}{\pgfqpoint{3.608454in}{2.310000in}}%
\pgfusepath{clip}%
\pgfsetbuttcap%
\pgfsetmiterjoin%
\definecolor{currentfill}{rgb}{0.000000,0.654902,0.672549}%
\pgfsetfillcolor{currentfill}%
\pgfsetlinewidth{0.000000pt}%
\definecolor{currentstroke}{rgb}{0.000000,0.000000,0.000000}%
\pgfsetstrokecolor{currentstroke}%
\pgfsetstrokeopacity{0.000000}%
\pgfsetdash{}{0pt}%
\pgfpathmoveto{\pgfqpoint{3.143924in}{1.538541in}}%
\pgfpathlineto{\pgfqpoint{3.137378in}{1.521010in}}%
\pgfpathlineto{\pgfqpoint{3.133137in}{1.523448in}}%
\pgfpathlineto{\pgfqpoint{3.119792in}{1.526024in}}%
\pgfpathlineto{\pgfqpoint{3.112902in}{1.529498in}}%
\pgfpathlineto{\pgfqpoint{3.110584in}{1.534960in}}%
\pgfpathlineto{\pgfqpoint{3.115439in}{1.550679in}}%
\pgfpathlineto{\pgfqpoint{3.110368in}{1.555879in}}%
\pgfpathlineto{\pgfqpoint{3.111267in}{1.560885in}}%
\pgfpathlineto{\pgfqpoint{3.107213in}{1.564628in}}%
\pgfpathlineto{\pgfqpoint{3.098915in}{1.567130in}}%
\pgfpathlineto{\pgfqpoint{3.072318in}{1.562246in}}%
\pgfpathlineto{\pgfqpoint{3.069571in}{1.576329in}}%
\pgfpathlineto{\pgfqpoint{3.045363in}{1.572457in}}%
\pgfpathlineto{\pgfqpoint{3.043474in}{1.583306in}}%
\pgfpathlineto{\pgfqpoint{3.039983in}{1.605278in}}%
\pgfpathlineto{\pgfqpoint{3.046524in}{1.607153in}}%
\pgfpathlineto{\pgfqpoint{3.054170in}{1.610224in}}%
\pgfpathlineto{\pgfqpoint{3.068655in}{1.604850in}}%
\pgfpathlineto{\pgfqpoint{3.106858in}{1.608202in}}%
\pgfpathlineto{\pgfqpoint{3.111013in}{1.603891in}}%
\pgfpathlineto{\pgfqpoint{3.125203in}{1.608488in}}%
\pgfpathlineto{\pgfqpoint{3.128786in}{1.603472in}}%
\pgfpathlineto{\pgfqpoint{3.133554in}{1.604605in}}%
\pgfpathlineto{\pgfqpoint{3.144862in}{1.595009in}}%
\pgfpathlineto{\pgfqpoint{3.154199in}{1.598846in}}%
\pgfpathlineto{\pgfqpoint{3.165006in}{1.605498in}}%
\pgfpathlineto{\pgfqpoint{3.155139in}{1.587716in}}%
\pgfpathlineto{\pgfqpoint{3.139700in}{1.576466in}}%
\pgfpathlineto{\pgfqpoint{3.131483in}{1.559276in}}%
\pgfpathlineto{\pgfqpoint{3.133732in}{1.555574in}}%
\pgfpathlineto{\pgfqpoint{3.130234in}{1.548464in}}%
\pgfpathlineto{\pgfqpoint{3.133310in}{1.544667in}}%
\pgfpathlineto{\pgfqpoint{3.138889in}{1.546016in}}%
\pgfpathlineto{\pgfqpoint{3.143924in}{1.538541in}}%
\pgfpathclose%
\pgfusepath{fill}%
\end{pgfscope}%
\begin{pgfscope}%
\pgfpathrectangle{\pgfqpoint{0.100000in}{0.100000in}}{\pgfqpoint{3.608454in}{2.310000in}}%
\pgfusepath{clip}%
\pgfsetbuttcap%
\pgfsetmiterjoin%
\definecolor{currentfill}{rgb}{0.000000,0.345098,0.827451}%
\pgfsetfillcolor{currentfill}%
\pgfsetlinewidth{0.000000pt}%
\definecolor{currentstroke}{rgb}{0.000000,0.000000,0.000000}%
\pgfsetstrokecolor{currentstroke}%
\pgfsetstrokeopacity{0.000000}%
\pgfsetdash{}{0pt}%
\pgfpathmoveto{\pgfqpoint{2.196902in}{1.594193in}}%
\pgfpathlineto{\pgfqpoint{2.155774in}{1.593709in}}%
\pgfpathlineto{\pgfqpoint{2.115090in}{1.593467in}}%
\pgfpathlineto{\pgfqpoint{2.115152in}{1.614399in}}%
\pgfpathlineto{\pgfqpoint{2.111772in}{1.614418in}}%
\pgfpathlineto{\pgfqpoint{2.111776in}{1.648912in}}%
\pgfpathlineto{\pgfqpoint{2.166463in}{1.649078in}}%
\pgfpathlineto{\pgfqpoint{2.193901in}{1.649520in}}%
\pgfpathlineto{\pgfqpoint{2.194446in}{1.614873in}}%
\pgfpathlineto{\pgfqpoint{2.196575in}{1.614898in}}%
\pgfpathlineto{\pgfqpoint{2.196902in}{1.594193in}}%
\pgfpathclose%
\pgfusepath{fill}%
\end{pgfscope}%
\begin{pgfscope}%
\pgfpathrectangle{\pgfqpoint{0.100000in}{0.100000in}}{\pgfqpoint{3.608454in}{2.310000in}}%
\pgfusepath{clip}%
\pgfsetbuttcap%
\pgfsetmiterjoin%
\definecolor{currentfill}{rgb}{0.000000,0.447059,0.776471}%
\pgfsetfillcolor{currentfill}%
\pgfsetlinewidth{0.000000pt}%
\definecolor{currentstroke}{rgb}{0.000000,0.000000,0.000000}%
\pgfsetstrokecolor{currentstroke}%
\pgfsetstrokeopacity{0.000000}%
\pgfsetdash{}{0pt}%
\pgfpathmoveto{\pgfqpoint{3.466647in}{1.697217in}}%
\pgfpathlineto{\pgfqpoint{3.461826in}{1.697323in}}%
\pgfpathlineto{\pgfqpoint{3.452537in}{1.691767in}}%
\pgfpathlineto{\pgfqpoint{3.447030in}{1.691452in}}%
\pgfpathlineto{\pgfqpoint{3.437379in}{1.684748in}}%
\pgfpathlineto{\pgfqpoint{3.433364in}{1.685208in}}%
\pgfpathlineto{\pgfqpoint{3.406113in}{1.677314in}}%
\pgfpathlineto{\pgfqpoint{3.369189in}{1.645562in}}%
\pgfpathlineto{\pgfqpoint{3.362988in}{1.653409in}}%
\pgfpathlineto{\pgfqpoint{3.375111in}{1.665432in}}%
\pgfpathlineto{\pgfqpoint{3.369634in}{1.670838in}}%
\pgfpathlineto{\pgfqpoint{3.366550in}{1.688953in}}%
\pgfpathlineto{\pgfqpoint{3.359316in}{1.729735in}}%
\pgfpathlineto{\pgfqpoint{3.384285in}{1.735025in}}%
\pgfpathlineto{\pgfqpoint{3.436777in}{1.747450in}}%
\pgfpathlineto{\pgfqpoint{3.456367in}{1.750710in}}%
\pgfpathlineto{\pgfqpoint{3.464103in}{1.723936in}}%
\pgfpathlineto{\pgfqpoint{3.468281in}{1.705449in}}%
\pgfpathlineto{\pgfqpoint{3.466647in}{1.697217in}}%
\pgfpathclose%
\pgfusepath{fill}%
\end{pgfscope}%
\begin{pgfscope}%
\pgfpathrectangle{\pgfqpoint{0.100000in}{0.100000in}}{\pgfqpoint{3.608454in}{2.310000in}}%
\pgfusepath{clip}%
\pgfsetbuttcap%
\pgfsetmiterjoin%
\definecolor{currentfill}{rgb}{0.000000,0.580392,0.709804}%
\pgfsetfillcolor{currentfill}%
\pgfsetlinewidth{0.000000pt}%
\definecolor{currentstroke}{rgb}{0.000000,0.000000,0.000000}%
\pgfsetstrokecolor{currentstroke}%
\pgfsetstrokeopacity{0.000000}%
\pgfsetdash{}{0pt}%
\pgfpathmoveto{\pgfqpoint{2.353369in}{1.446719in}}%
\pgfpathlineto{\pgfqpoint{2.353565in}{1.439495in}}%
\pgfpathlineto{\pgfqpoint{2.317746in}{1.438593in}}%
\pgfpathlineto{\pgfqpoint{2.318087in}{1.442426in}}%
\pgfpathlineto{\pgfqpoint{2.290718in}{1.442111in}}%
\pgfpathlineto{\pgfqpoint{2.290253in}{1.469670in}}%
\pgfpathlineto{\pgfqpoint{2.303069in}{1.470734in}}%
\pgfpathlineto{\pgfqpoint{2.302293in}{1.514165in}}%
\pgfpathlineto{\pgfqpoint{2.322856in}{1.514824in}}%
\pgfpathlineto{\pgfqpoint{2.323109in}{1.507938in}}%
\pgfpathlineto{\pgfqpoint{2.348213in}{1.508885in}}%
\pgfpathlineto{\pgfqpoint{2.358001in}{1.509016in}}%
\pgfpathlineto{\pgfqpoint{2.359331in}{1.474799in}}%
\pgfpathlineto{\pgfqpoint{2.352440in}{1.474762in}}%
\pgfpathlineto{\pgfqpoint{2.353369in}{1.446719in}}%
\pgfpathclose%
\pgfusepath{fill}%
\end{pgfscope}%
\begin{pgfscope}%
\pgfpathrectangle{\pgfqpoint{0.100000in}{0.100000in}}{\pgfqpoint{3.608454in}{2.310000in}}%
\pgfusepath{clip}%
\pgfsetbuttcap%
\pgfsetmiterjoin%
\definecolor{currentfill}{rgb}{0.000000,0.635294,0.682353}%
\pgfsetfillcolor{currentfill}%
\pgfsetlinewidth{0.000000pt}%
\definecolor{currentstroke}{rgb}{0.000000,0.000000,0.000000}%
\pgfsetstrokecolor{currentstroke}%
\pgfsetstrokeopacity{0.000000}%
\pgfsetdash{}{0pt}%
\pgfpathmoveto{\pgfqpoint{2.798375in}{1.062974in}}%
\pgfpathlineto{\pgfqpoint{2.779057in}{1.060719in}}%
\pgfpathlineto{\pgfqpoint{2.756131in}{1.058082in}}%
\pgfpathlineto{\pgfqpoint{2.751742in}{1.068739in}}%
\pgfpathlineto{\pgfqpoint{2.755774in}{1.081517in}}%
\pgfpathlineto{\pgfqpoint{2.749929in}{1.091001in}}%
\pgfpathlineto{\pgfqpoint{2.755630in}{1.093871in}}%
\pgfpathlineto{\pgfqpoint{2.763588in}{1.107918in}}%
\pgfpathlineto{\pgfqpoint{2.762622in}{1.113861in}}%
\pgfpathlineto{\pgfqpoint{2.765621in}{1.120136in}}%
\pgfpathlineto{\pgfqpoint{2.772428in}{1.117636in}}%
\pgfpathlineto{\pgfqpoint{2.775615in}{1.111731in}}%
\pgfpathlineto{\pgfqpoint{2.790002in}{1.116268in}}%
\pgfpathlineto{\pgfqpoint{2.805797in}{1.111156in}}%
\pgfpathlineto{\pgfqpoint{2.817068in}{1.103142in}}%
\pgfpathlineto{\pgfqpoint{2.813980in}{1.097663in}}%
\pgfpathlineto{\pgfqpoint{2.814347in}{1.089123in}}%
\pgfpathlineto{\pgfqpoint{2.807766in}{1.084602in}}%
\pgfpathlineto{\pgfqpoint{2.802072in}{1.085804in}}%
\pgfpathlineto{\pgfqpoint{2.798205in}{1.081874in}}%
\pgfpathlineto{\pgfqpoint{2.798375in}{1.062974in}}%
\pgfpathclose%
\pgfusepath{fill}%
\end{pgfscope}%
\begin{pgfscope}%
\pgfpathrectangle{\pgfqpoint{0.100000in}{0.100000in}}{\pgfqpoint{3.608454in}{2.310000in}}%
\pgfusepath{clip}%
\pgfsetbuttcap%
\pgfsetmiterjoin%
\definecolor{currentfill}{rgb}{0.000000,0.682353,0.658824}%
\pgfsetfillcolor{currentfill}%
\pgfsetlinewidth{0.000000pt}%
\definecolor{currentstroke}{rgb}{0.000000,0.000000,0.000000}%
\pgfsetstrokecolor{currentstroke}%
\pgfsetstrokeopacity{0.000000}%
\pgfsetdash{}{0pt}%
\pgfpathmoveto{\pgfqpoint{1.516667in}{1.064131in}}%
\pgfpathlineto{\pgfqpoint{1.517207in}{1.069928in}}%
\pgfpathlineto{\pgfqpoint{1.442195in}{1.078025in}}%
\pgfpathlineto{\pgfqpoint{1.440664in}{1.064305in}}%
\pgfpathlineto{\pgfqpoint{1.413364in}{1.067432in}}%
\pgfpathlineto{\pgfqpoint{1.421911in}{1.141159in}}%
\pgfpathlineto{\pgfqpoint{1.431076in}{1.139578in}}%
\pgfpathlineto{\pgfqpoint{1.442215in}{1.150974in}}%
\pgfpathlineto{\pgfqpoint{1.445381in}{1.160260in}}%
\pgfpathlineto{\pgfqpoint{1.448410in}{1.160662in}}%
\pgfpathlineto{\pgfqpoint{1.450548in}{1.172051in}}%
\pgfpathlineto{\pgfqpoint{1.450272in}{1.192914in}}%
\pgfpathlineto{\pgfqpoint{1.458207in}{1.195789in}}%
\pgfpathlineto{\pgfqpoint{1.462472in}{1.211222in}}%
\pgfpathlineto{\pgfqpoint{1.462544in}{1.217413in}}%
\pgfpathlineto{\pgfqpoint{1.466660in}{1.216957in}}%
\pgfpathlineto{\pgfqpoint{1.513551in}{1.211748in}}%
\pgfpathlineto{\pgfqpoint{1.538731in}{1.209408in}}%
\pgfpathlineto{\pgfqpoint{1.532570in}{1.148149in}}%
\pgfpathlineto{\pgfqpoint{1.505428in}{1.150972in}}%
\pgfpathlineto{\pgfqpoint{1.508502in}{1.136752in}}%
\pgfpathlineto{\pgfqpoint{1.508110in}{1.125109in}}%
\pgfpathlineto{\pgfqpoint{1.506250in}{1.116084in}}%
\pgfpathlineto{\pgfqpoint{1.528359in}{1.113844in}}%
\pgfpathlineto{\pgfqpoint{1.531357in}{1.115154in}}%
\pgfpathlineto{\pgfqpoint{1.544598in}{1.083669in}}%
\pgfpathlineto{\pgfqpoint{1.550211in}{1.083114in}}%
\pgfpathlineto{\pgfqpoint{1.548858in}{1.068831in}}%
\pgfpathlineto{\pgfqpoint{1.534671in}{1.070306in}}%
\pgfpathlineto{\pgfqpoint{1.516667in}{1.064131in}}%
\pgfpathclose%
\pgfusepath{fill}%
\end{pgfscope}%
\begin{pgfscope}%
\pgfpathrectangle{\pgfqpoint{0.100000in}{0.100000in}}{\pgfqpoint{3.608454in}{2.310000in}}%
\pgfusepath{clip}%
\pgfsetbuttcap%
\pgfsetmiterjoin%
\definecolor{currentfill}{rgb}{0.000000,0.411765,0.794118}%
\pgfsetfillcolor{currentfill}%
\pgfsetlinewidth{0.000000pt}%
\definecolor{currentstroke}{rgb}{0.000000,0.000000,0.000000}%
\pgfsetstrokecolor{currentstroke}%
\pgfsetstrokeopacity{0.000000}%
\pgfsetdash{}{0pt}%
\pgfpathmoveto{\pgfqpoint{2.266869in}{1.540448in}}%
\pgfpathlineto{\pgfqpoint{2.205803in}{1.538835in}}%
\pgfpathlineto{\pgfqpoint{2.204489in}{1.546166in}}%
\pgfpathlineto{\pgfqpoint{2.204148in}{1.566920in}}%
\pgfpathlineto{\pgfqpoint{2.210987in}{1.567017in}}%
\pgfpathlineto{\pgfqpoint{2.210467in}{1.594411in}}%
\pgfpathlineto{\pgfqpoint{2.223956in}{1.594717in}}%
\pgfpathlineto{\pgfqpoint{2.237674in}{1.595106in}}%
\pgfpathlineto{\pgfqpoint{2.237469in}{1.601952in}}%
\pgfpathlineto{\pgfqpoint{2.264937in}{1.602761in}}%
\pgfpathlineto{\pgfqpoint{2.266869in}{1.540448in}}%
\pgfpathclose%
\pgfusepath{fill}%
\end{pgfscope}%
\begin{pgfscope}%
\pgfpathrectangle{\pgfqpoint{0.100000in}{0.100000in}}{\pgfqpoint{3.608454in}{2.310000in}}%
\pgfusepath{clip}%
\pgfsetbuttcap%
\pgfsetmiterjoin%
\definecolor{currentfill}{rgb}{0.000000,0.647059,0.676471}%
\pgfsetfillcolor{currentfill}%
\pgfsetlinewidth{0.000000pt}%
\definecolor{currentstroke}{rgb}{0.000000,0.000000,0.000000}%
\pgfsetstrokecolor{currentstroke}%
\pgfsetstrokeopacity{0.000000}%
\pgfsetdash{}{0pt}%
\pgfpathmoveto{\pgfqpoint{0.718561in}{1.521037in}}%
\pgfpathlineto{\pgfqpoint{0.697419in}{1.520493in}}%
\pgfpathlineto{\pgfqpoint{0.695807in}{1.514308in}}%
\pgfpathlineto{\pgfqpoint{0.717194in}{1.472445in}}%
\pgfpathlineto{\pgfqpoint{0.665974in}{1.438337in}}%
\pgfpathlineto{\pgfqpoint{0.661400in}{1.439746in}}%
\pgfpathlineto{\pgfqpoint{0.628251in}{1.491065in}}%
\pgfpathlineto{\pgfqpoint{0.643201in}{1.487011in}}%
\pgfpathlineto{\pgfqpoint{0.652474in}{1.520325in}}%
\pgfpathlineto{\pgfqpoint{0.645777in}{1.522062in}}%
\pgfpathlineto{\pgfqpoint{0.647659in}{1.529088in}}%
\pgfpathlineto{\pgfqpoint{0.655392in}{1.537674in}}%
\pgfpathlineto{\pgfqpoint{0.718561in}{1.521037in}}%
\pgfpathclose%
\pgfusepath{fill}%
\end{pgfscope}%
\begin{pgfscope}%
\pgfpathrectangle{\pgfqpoint{0.100000in}{0.100000in}}{\pgfqpoint{3.608454in}{2.310000in}}%
\pgfusepath{clip}%
\pgfsetbuttcap%
\pgfsetmiterjoin%
\definecolor{currentfill}{rgb}{0.000000,0.737255,0.631373}%
\pgfsetfillcolor{currentfill}%
\pgfsetlinewidth{0.000000pt}%
\definecolor{currentstroke}{rgb}{0.000000,0.000000,0.000000}%
\pgfsetstrokecolor{currentstroke}%
\pgfsetstrokeopacity{0.000000}%
\pgfsetdash{}{0pt}%
\pgfpathmoveto{\pgfqpoint{2.996900in}{1.203053in}}%
\pgfpathlineto{\pgfqpoint{2.987922in}{1.204284in}}%
\pgfpathlineto{\pgfqpoint{2.980945in}{1.198576in}}%
\pgfpathlineto{\pgfqpoint{2.970773in}{1.210739in}}%
\pgfpathlineto{\pgfqpoint{2.970559in}{1.212860in}}%
\pgfpathlineto{\pgfqpoint{2.950033in}{1.210864in}}%
\pgfpathlineto{\pgfqpoint{2.951682in}{1.213011in}}%
\pgfpathlineto{\pgfqpoint{2.957411in}{1.217461in}}%
\pgfpathlineto{\pgfqpoint{2.958209in}{1.221651in}}%
\pgfpathlineto{\pgfqpoint{2.974075in}{1.228469in}}%
\pgfpathlineto{\pgfqpoint{2.984220in}{1.230496in}}%
\pgfpathlineto{\pgfqpoint{2.986436in}{1.233543in}}%
\pgfpathlineto{\pgfqpoint{3.005275in}{1.242589in}}%
\pgfpathlineto{\pgfqpoint{3.011272in}{1.247768in}}%
\pgfpathlineto{\pgfqpoint{3.024683in}{1.232048in}}%
\pgfpathlineto{\pgfqpoint{3.025017in}{1.229246in}}%
\pgfpathlineto{\pgfqpoint{3.015913in}{1.223529in}}%
\pgfpathlineto{\pgfqpoint{3.017299in}{1.218774in}}%
\pgfpathlineto{\pgfqpoint{2.999075in}{1.216208in}}%
\pgfpathlineto{\pgfqpoint{2.995823in}{1.206373in}}%
\pgfpathlineto{\pgfqpoint{2.996900in}{1.203053in}}%
\pgfpathclose%
\pgfusepath{fill}%
\end{pgfscope}%
\begin{pgfscope}%
\pgfpathrectangle{\pgfqpoint{0.100000in}{0.100000in}}{\pgfqpoint{3.608454in}{2.310000in}}%
\pgfusepath{clip}%
\pgfsetbuttcap%
\pgfsetmiterjoin%
\definecolor{currentfill}{rgb}{0.000000,0.678431,0.660784}%
\pgfsetfillcolor{currentfill}%
\pgfsetlinewidth{0.000000pt}%
\definecolor{currentstroke}{rgb}{0.000000,0.000000,0.000000}%
\pgfsetstrokecolor{currentstroke}%
\pgfsetstrokeopacity{0.000000}%
\pgfsetdash{}{0pt}%
\pgfpathmoveto{\pgfqpoint{2.869199in}{1.556678in}}%
\pgfpathlineto{\pgfqpoint{2.872479in}{1.551436in}}%
\pgfpathlineto{\pgfqpoint{2.876286in}{1.526111in}}%
\pgfpathlineto{\pgfqpoint{2.870685in}{1.525264in}}%
\pgfpathlineto{\pgfqpoint{2.871819in}{1.517489in}}%
\pgfpathlineto{\pgfqpoint{2.862804in}{1.514836in}}%
\pgfpathlineto{\pgfqpoint{2.846858in}{1.512561in}}%
\pgfpathlineto{\pgfqpoint{2.848845in}{1.496480in}}%
\pgfpathlineto{\pgfqpoint{2.831782in}{1.495222in}}%
\pgfpathlineto{\pgfqpoint{2.831500in}{1.499705in}}%
\pgfpathlineto{\pgfqpoint{2.811880in}{1.499432in}}%
\pgfpathlineto{\pgfqpoint{2.811270in}{1.504385in}}%
\pgfpathlineto{\pgfqpoint{2.801132in}{1.502933in}}%
\pgfpathlineto{\pgfqpoint{2.799219in}{1.517163in}}%
\pgfpathlineto{\pgfqpoint{2.817395in}{1.520810in}}%
\pgfpathlineto{\pgfqpoint{2.814322in}{1.543583in}}%
\pgfpathlineto{\pgfqpoint{2.831040in}{1.546066in}}%
\pgfpathlineto{\pgfqpoint{2.854535in}{1.548999in}}%
\pgfpathlineto{\pgfqpoint{2.853507in}{1.554639in}}%
\pgfpathlineto{\pgfqpoint{2.869199in}{1.556678in}}%
\pgfpathclose%
\pgfusepath{fill}%
\end{pgfscope}%
\begin{pgfscope}%
\pgfpathrectangle{\pgfqpoint{0.100000in}{0.100000in}}{\pgfqpoint{3.608454in}{2.310000in}}%
\pgfusepath{clip}%
\pgfsetbuttcap%
\pgfsetmiterjoin%
\definecolor{currentfill}{rgb}{0.000000,0.878431,0.560784}%
\pgfsetfillcolor{currentfill}%
\pgfsetlinewidth{0.000000pt}%
\definecolor{currentstroke}{rgb}{0.000000,0.000000,0.000000}%
\pgfsetstrokecolor{currentstroke}%
\pgfsetstrokeopacity{0.000000}%
\pgfsetdash{}{0pt}%
\pgfpathmoveto{\pgfqpoint{2.722462in}{1.100629in}}%
\pgfpathlineto{\pgfqpoint{2.731314in}{1.113832in}}%
\pgfpathlineto{\pgfqpoint{2.732727in}{1.126386in}}%
\pgfpathlineto{\pgfqpoint{2.729200in}{1.134130in}}%
\pgfpathlineto{\pgfqpoint{2.729930in}{1.144288in}}%
\pgfpathlineto{\pgfqpoint{2.735267in}{1.144371in}}%
\pgfpathlineto{\pgfqpoint{2.738900in}{1.152461in}}%
\pgfpathlineto{\pgfqpoint{2.735445in}{1.160714in}}%
\pgfpathlineto{\pgfqpoint{2.734679in}{1.168795in}}%
\pgfpathlineto{\pgfqpoint{2.738211in}{1.181250in}}%
\pgfpathlineto{\pgfqpoint{2.744380in}{1.183721in}}%
\pgfpathlineto{\pgfqpoint{2.758225in}{1.180770in}}%
\pgfpathlineto{\pgfqpoint{2.763981in}{1.171120in}}%
\pgfpathlineto{\pgfqpoint{2.761563in}{1.168904in}}%
\pgfpathlineto{\pgfqpoint{2.751113in}{1.160516in}}%
\pgfpathlineto{\pgfqpoint{2.750357in}{1.150558in}}%
\pgfpathlineto{\pgfqpoint{2.763659in}{1.139195in}}%
\pgfpathlineto{\pgfqpoint{2.766985in}{1.132686in}}%
\pgfpathlineto{\pgfqpoint{2.761095in}{1.125193in}}%
\pgfpathlineto{\pgfqpoint{2.765621in}{1.120136in}}%
\pgfpathlineto{\pgfqpoint{2.762622in}{1.113861in}}%
\pgfpathlineto{\pgfqpoint{2.763588in}{1.107918in}}%
\pgfpathlineto{\pgfqpoint{2.755630in}{1.093871in}}%
\pgfpathlineto{\pgfqpoint{2.749929in}{1.091001in}}%
\pgfpathlineto{\pgfqpoint{2.739787in}{1.093594in}}%
\pgfpathlineto{\pgfqpoint{2.736951in}{1.085202in}}%
\pgfpathlineto{\pgfqpoint{2.724406in}{1.094850in}}%
\pgfpathlineto{\pgfqpoint{2.722462in}{1.100629in}}%
\pgfpathclose%
\pgfusepath{fill}%
\end{pgfscope}%
\begin{pgfscope}%
\pgfpathrectangle{\pgfqpoint{0.100000in}{0.100000in}}{\pgfqpoint{3.608454in}{2.310000in}}%
\pgfusepath{clip}%
\pgfsetbuttcap%
\pgfsetmiterjoin%
\definecolor{currentfill}{rgb}{0.000000,0.690196,0.654902}%
\pgfsetfillcolor{currentfill}%
\pgfsetlinewidth{0.000000pt}%
\definecolor{currentstroke}{rgb}{0.000000,0.000000,0.000000}%
\pgfsetstrokecolor{currentstroke}%
\pgfsetstrokeopacity{0.000000}%
\pgfsetdash{}{0pt}%
\pgfpathmoveto{\pgfqpoint{3.271268in}{1.222696in}}%
\pgfpathlineto{\pgfqpoint{3.272014in}{1.215831in}}%
\pgfpathlineto{\pgfqpoint{3.269662in}{1.209194in}}%
\pgfpathlineto{\pgfqpoint{3.265874in}{1.209619in}}%
\pgfpathlineto{\pgfqpoint{3.262155in}{1.203038in}}%
\pgfpathlineto{\pgfqpoint{3.258356in}{1.209245in}}%
\pgfpathlineto{\pgfqpoint{3.253779in}{1.205127in}}%
\pgfpathlineto{\pgfqpoint{3.247206in}{1.209157in}}%
\pgfpathlineto{\pgfqpoint{3.239989in}{1.209565in}}%
\pgfpathlineto{\pgfqpoint{3.238791in}{1.214139in}}%
\pgfpathlineto{\pgfqpoint{3.230833in}{1.216563in}}%
\pgfpathlineto{\pgfqpoint{3.226514in}{1.211542in}}%
\pgfpathlineto{\pgfqpoint{3.218220in}{1.212547in}}%
\pgfpathlineto{\pgfqpoint{3.216999in}{1.216602in}}%
\pgfpathlineto{\pgfqpoint{3.206943in}{1.219779in}}%
\pgfpathlineto{\pgfqpoint{3.195538in}{1.217166in}}%
\pgfpathlineto{\pgfqpoint{3.185638in}{1.219606in}}%
\pgfpathlineto{\pgfqpoint{3.188912in}{1.234693in}}%
\pgfpathlineto{\pgfqpoint{3.187275in}{1.247460in}}%
\pgfpathlineto{\pgfqpoint{3.195545in}{1.249102in}}%
\pgfpathlineto{\pgfqpoint{3.199968in}{1.263498in}}%
\pgfpathlineto{\pgfqpoint{3.197116in}{1.277528in}}%
\pgfpathlineto{\pgfqpoint{3.202830in}{1.275174in}}%
\pgfpathlineto{\pgfqpoint{3.209973in}{1.277484in}}%
\pgfpathlineto{\pgfqpoint{3.210178in}{1.266083in}}%
\pgfpathlineto{\pgfqpoint{3.214106in}{1.265774in}}%
\pgfpathlineto{\pgfqpoint{3.221614in}{1.254572in}}%
\pgfpathlineto{\pgfqpoint{3.248646in}{1.259422in}}%
\pgfpathlineto{\pgfqpoint{3.249894in}{1.248153in}}%
\pgfpathlineto{\pgfqpoint{3.259748in}{1.247063in}}%
\pgfpathlineto{\pgfqpoint{3.266234in}{1.243000in}}%
\pgfpathlineto{\pgfqpoint{3.265839in}{1.231923in}}%
\pgfpathlineto{\pgfqpoint{3.271268in}{1.222696in}}%
\pgfpathclose%
\pgfusepath{fill}%
\end{pgfscope}%
\begin{pgfscope}%
\pgfpathrectangle{\pgfqpoint{0.100000in}{0.100000in}}{\pgfqpoint{3.608454in}{2.310000in}}%
\pgfusepath{clip}%
\pgfsetbuttcap%
\pgfsetmiterjoin%
\definecolor{currentfill}{rgb}{0.000000,0.486275,0.756863}%
\pgfsetfillcolor{currentfill}%
\pgfsetlinewidth{0.000000pt}%
\definecolor{currentstroke}{rgb}{0.000000,0.000000,0.000000}%
\pgfsetstrokecolor{currentstroke}%
\pgfsetstrokeopacity{0.000000}%
\pgfsetdash{}{0pt}%
\pgfpathmoveto{\pgfqpoint{1.826418in}{1.110191in}}%
\pgfpathlineto{\pgfqpoint{1.818722in}{1.107444in}}%
\pgfpathlineto{\pgfqpoint{1.814486in}{1.099133in}}%
\pgfpathlineto{\pgfqpoint{1.804299in}{1.099161in}}%
\pgfpathlineto{\pgfqpoint{1.798623in}{1.104938in}}%
\pgfpathlineto{\pgfqpoint{1.792285in}{1.107896in}}%
\pgfpathlineto{\pgfqpoint{1.786242in}{1.101806in}}%
\pgfpathlineto{\pgfqpoint{1.787003in}{1.115616in}}%
\pgfpathlineto{\pgfqpoint{1.788727in}{1.150707in}}%
\pgfpathlineto{\pgfqpoint{1.790983in}{1.190359in}}%
\pgfpathlineto{\pgfqpoint{1.820080in}{1.188664in}}%
\pgfpathlineto{\pgfqpoint{1.825437in}{1.188388in}}%
\pgfpathlineto{\pgfqpoint{1.830118in}{1.178718in}}%
\pgfpathlineto{\pgfqpoint{1.837450in}{1.170858in}}%
\pgfpathlineto{\pgfqpoint{1.844616in}{1.170342in}}%
\pgfpathlineto{\pgfqpoint{1.855366in}{1.155288in}}%
\pgfpathlineto{\pgfqpoint{1.854121in}{1.120697in}}%
\pgfpathlineto{\pgfqpoint{1.826923in}{1.122162in}}%
\pgfpathlineto{\pgfqpoint{1.826418in}{1.110191in}}%
\pgfpathclose%
\pgfusepath{fill}%
\end{pgfscope}%
\begin{pgfscope}%
\pgfpathrectangle{\pgfqpoint{0.100000in}{0.100000in}}{\pgfqpoint{3.608454in}{2.310000in}}%
\pgfusepath{clip}%
\pgfsetbuttcap%
\pgfsetmiterjoin%
\definecolor{currentfill}{rgb}{0.000000,0.313725,0.843137}%
\pgfsetfillcolor{currentfill}%
\pgfsetlinewidth{0.000000pt}%
\definecolor{currentstroke}{rgb}{0.000000,0.000000,0.000000}%
\pgfsetstrokecolor{currentstroke}%
\pgfsetstrokeopacity{0.000000}%
\pgfsetdash{}{0pt}%
\pgfpathmoveto{\pgfqpoint{3.083124in}{1.089029in}}%
\pgfpathlineto{\pgfqpoint{3.042269in}{1.083215in}}%
\pgfpathlineto{\pgfqpoint{3.054685in}{1.068940in}}%
\pgfpathlineto{\pgfqpoint{3.046213in}{1.063185in}}%
\pgfpathlineto{\pgfqpoint{3.047179in}{1.057968in}}%
\pgfpathlineto{\pgfqpoint{3.040979in}{1.056871in}}%
\pgfpathlineto{\pgfqpoint{3.034032in}{1.059624in}}%
\pgfpathlineto{\pgfqpoint{3.026275in}{1.051810in}}%
\pgfpathlineto{\pgfqpoint{3.025339in}{1.058515in}}%
\pgfpathlineto{\pgfqpoint{2.989806in}{1.055375in}}%
\pgfpathlineto{\pgfqpoint{2.986559in}{1.063157in}}%
\pgfpathlineto{\pgfqpoint{2.984590in}{1.076632in}}%
\pgfpathlineto{\pgfqpoint{2.980220in}{1.091029in}}%
\pgfpathlineto{\pgfqpoint{2.985051in}{1.092680in}}%
\pgfpathlineto{\pgfqpoint{2.986399in}{1.102377in}}%
\pgfpathlineto{\pgfqpoint{3.008536in}{1.104446in}}%
\pgfpathlineto{\pgfqpoint{3.007642in}{1.113989in}}%
\pgfpathlineto{\pgfqpoint{3.009870in}{1.122844in}}%
\pgfpathlineto{\pgfqpoint{3.008399in}{1.134076in}}%
\pgfpathlineto{\pgfqpoint{3.022456in}{1.135384in}}%
\pgfpathlineto{\pgfqpoint{3.018191in}{1.148162in}}%
\pgfpathlineto{\pgfqpoint{3.020036in}{1.162708in}}%
\pgfpathlineto{\pgfqpoint{3.024951in}{1.162919in}}%
\pgfpathlineto{\pgfqpoint{3.037191in}{1.156638in}}%
\pgfpathlineto{\pgfqpoint{3.046178in}{1.152956in}}%
\pgfpathlineto{\pgfqpoint{3.055178in}{1.146538in}}%
\pgfpathlineto{\pgfqpoint{3.057775in}{1.140860in}}%
\pgfpathlineto{\pgfqpoint{3.062964in}{1.138236in}}%
\pgfpathlineto{\pgfqpoint{3.067859in}{1.131866in}}%
\pgfpathlineto{\pgfqpoint{3.066320in}{1.122502in}}%
\pgfpathlineto{\pgfqpoint{3.070229in}{1.116768in}}%
\pgfpathlineto{\pgfqpoint{3.070169in}{1.111400in}}%
\pgfpathlineto{\pgfqpoint{3.078110in}{1.113287in}}%
\pgfpathlineto{\pgfqpoint{3.083344in}{1.107498in}}%
\pgfpathlineto{\pgfqpoint{3.086909in}{1.097540in}}%
\pgfpathlineto{\pgfqpoint{3.083124in}{1.089029in}}%
\pgfpathclose%
\pgfusepath{fill}%
\end{pgfscope}%
\begin{pgfscope}%
\pgfpathrectangle{\pgfqpoint{0.100000in}{0.100000in}}{\pgfqpoint{3.608454in}{2.310000in}}%
\pgfusepath{clip}%
\pgfsetbuttcap%
\pgfsetmiterjoin%
\definecolor{currentfill}{rgb}{0.000000,0.439216,0.780392}%
\pgfsetfillcolor{currentfill}%
\pgfsetlinewidth{0.000000pt}%
\definecolor{currentstroke}{rgb}{0.000000,0.000000,0.000000}%
\pgfsetstrokecolor{currentstroke}%
\pgfsetstrokeopacity{0.000000}%
\pgfsetdash{}{0pt}%
\pgfpathmoveto{\pgfqpoint{2.034871in}{0.852102in}}%
\pgfpathlineto{\pgfqpoint{2.020147in}{0.852551in}}%
\pgfpathlineto{\pgfqpoint{2.020851in}{0.893088in}}%
\pgfpathlineto{\pgfqpoint{2.049865in}{0.897339in}}%
\pgfpathlineto{\pgfqpoint{2.050404in}{0.930484in}}%
\pgfpathlineto{\pgfqpoint{2.056583in}{0.934051in}}%
\pgfpathlineto{\pgfqpoint{2.068217in}{0.938848in}}%
\pgfpathlineto{\pgfqpoint{2.070424in}{0.935180in}}%
\pgfpathlineto{\pgfqpoint{2.079921in}{0.933374in}}%
\pgfpathlineto{\pgfqpoint{2.087358in}{0.933839in}}%
\pgfpathlineto{\pgfqpoint{2.092356in}{0.940891in}}%
\pgfpathlineto{\pgfqpoint{2.109858in}{0.929360in}}%
\pgfpathlineto{\pgfqpoint{2.115386in}{0.923776in}}%
\pgfpathlineto{\pgfqpoint{2.123559in}{0.920377in}}%
\pgfpathlineto{\pgfqpoint{2.123587in}{0.890672in}}%
\pgfpathlineto{\pgfqpoint{2.115003in}{0.895858in}}%
\pgfpathlineto{\pgfqpoint{2.100603in}{0.896104in}}%
\pgfpathlineto{\pgfqpoint{2.086328in}{0.894484in}}%
\pgfpathlineto{\pgfqpoint{2.086189in}{0.861574in}}%
\pgfpathlineto{\pgfqpoint{2.062420in}{0.861532in}}%
\pgfpathlineto{\pgfqpoint{2.064298in}{0.842428in}}%
\pgfpathlineto{\pgfqpoint{2.053670in}{0.846942in}}%
\pgfpathlineto{\pgfqpoint{2.050385in}{0.846426in}}%
\pgfpathlineto{\pgfqpoint{2.044379in}{0.851895in}}%
\pgfpathlineto{\pgfqpoint{2.034871in}{0.852102in}}%
\pgfpathclose%
\pgfusepath{fill}%
\end{pgfscope}%
\begin{pgfscope}%
\pgfpathrectangle{\pgfqpoint{0.100000in}{0.100000in}}{\pgfqpoint{3.608454in}{2.310000in}}%
\pgfusepath{clip}%
\pgfsetbuttcap%
\pgfsetmiterjoin%
\definecolor{currentfill}{rgb}{0.000000,0.956863,0.521569}%
\pgfsetfillcolor{currentfill}%
\pgfsetlinewidth{0.000000pt}%
\definecolor{currentstroke}{rgb}{0.000000,0.000000,0.000000}%
\pgfsetstrokecolor{currentstroke}%
\pgfsetstrokeopacity{0.000000}%
\pgfsetdash{}{0pt}%
\pgfpathmoveto{\pgfqpoint{2.961168in}{1.406181in}}%
\pgfpathlineto{\pgfqpoint{2.965023in}{1.393727in}}%
\pgfpathlineto{\pgfqpoint{2.962379in}{1.389071in}}%
\pgfpathlineto{\pgfqpoint{2.967663in}{1.383858in}}%
\pgfpathlineto{\pgfqpoint{2.965422in}{1.379505in}}%
\pgfpathlineto{\pgfqpoint{2.957191in}{1.367025in}}%
\pgfpathlineto{\pgfqpoint{2.940346in}{1.366252in}}%
\pgfpathlineto{\pgfqpoint{2.936140in}{1.370645in}}%
\pgfpathlineto{\pgfqpoint{2.933951in}{1.375085in}}%
\pgfpathlineto{\pgfqpoint{2.932362in}{1.392633in}}%
\pgfpathlineto{\pgfqpoint{2.936799in}{1.397031in}}%
\pgfpathlineto{\pgfqpoint{2.943680in}{1.394391in}}%
\pgfpathlineto{\pgfqpoint{2.954499in}{1.407993in}}%
\pgfpathlineto{\pgfqpoint{2.961168in}{1.406181in}}%
\pgfpathclose%
\pgfusepath{fill}%
\end{pgfscope}%
\begin{pgfscope}%
\pgfpathrectangle{\pgfqpoint{0.100000in}{0.100000in}}{\pgfqpoint{3.608454in}{2.310000in}}%
\pgfusepath{clip}%
\pgfsetbuttcap%
\pgfsetmiterjoin%
\definecolor{currentfill}{rgb}{0.000000,0.427451,0.786275}%
\pgfsetfillcolor{currentfill}%
\pgfsetlinewidth{0.000000pt}%
\definecolor{currentstroke}{rgb}{0.000000,0.000000,0.000000}%
\pgfsetstrokecolor{currentstroke}%
\pgfsetstrokeopacity{0.000000}%
\pgfsetdash{}{0pt}%
\pgfpathmoveto{\pgfqpoint{1.623124in}{2.077735in}}%
\pgfpathlineto{\pgfqpoint{1.626106in}{2.109098in}}%
\pgfpathlineto{\pgfqpoint{1.630877in}{2.157836in}}%
\pgfpathlineto{\pgfqpoint{1.689092in}{2.152335in}}%
\pgfpathlineto{\pgfqpoint{1.737269in}{2.148337in}}%
\pgfpathlineto{\pgfqpoint{1.736050in}{2.132965in}}%
\pgfpathlineto{\pgfqpoint{1.729180in}{2.133512in}}%
\pgfpathlineto{\pgfqpoint{1.728606in}{2.126565in}}%
\pgfpathlineto{\pgfqpoint{1.724331in}{2.126897in}}%
\pgfpathlineto{\pgfqpoint{1.723190in}{2.112972in}}%
\pgfpathlineto{\pgfqpoint{1.688717in}{2.115878in}}%
\pgfpathlineto{\pgfqpoint{1.687537in}{2.102006in}}%
\pgfpathlineto{\pgfqpoint{1.690580in}{2.101766in}}%
\pgfpathlineto{\pgfqpoint{1.686894in}{2.083116in}}%
\pgfpathlineto{\pgfqpoint{1.669040in}{2.084553in}}%
\pgfpathlineto{\pgfqpoint{1.664466in}{2.078762in}}%
\pgfpathlineto{\pgfqpoint{1.653010in}{2.075697in}}%
\pgfpathlineto{\pgfqpoint{1.648734in}{2.078432in}}%
\pgfpathlineto{\pgfqpoint{1.647283in}{2.084906in}}%
\pgfpathlineto{\pgfqpoint{1.643336in}{2.085862in}}%
\pgfpathlineto{\pgfqpoint{1.632154in}{2.074810in}}%
\pgfpathlineto{\pgfqpoint{1.623124in}{2.077735in}}%
\pgfpathclose%
\pgfusepath{fill}%
\end{pgfscope}%
\begin{pgfscope}%
\pgfpathrectangle{\pgfqpoint{0.100000in}{0.100000in}}{\pgfqpoint{3.608454in}{2.310000in}}%
\pgfusepath{clip}%
\pgfsetbuttcap%
\pgfsetmiterjoin%
\definecolor{currentfill}{rgb}{0.000000,0.505882,0.747059}%
\pgfsetfillcolor{currentfill}%
\pgfsetlinewidth{0.000000pt}%
\definecolor{currentstroke}{rgb}{0.000000,0.000000,0.000000}%
\pgfsetstrokecolor{currentstroke}%
\pgfsetstrokeopacity{0.000000}%
\pgfsetdash{}{0pt}%
\pgfpathmoveto{\pgfqpoint{1.576920in}{0.953485in}}%
\pgfpathlineto{\pgfqpoint{1.640100in}{0.948213in}}%
\pgfpathlineto{\pgfqpoint{1.708835in}{0.943953in}}%
\pgfpathlineto{\pgfqpoint{1.706689in}{0.909371in}}%
\pgfpathlineto{\pgfqpoint{1.704446in}{0.875520in}}%
\pgfpathlineto{\pgfqpoint{1.695447in}{0.875598in}}%
\pgfpathlineto{\pgfqpoint{1.635380in}{0.879674in}}%
\pgfpathlineto{\pgfqpoint{1.638056in}{0.913724in}}%
\pgfpathlineto{\pgfqpoint{1.573241in}{0.919040in}}%
\pgfpathlineto{\pgfqpoint{1.574789in}{0.933425in}}%
\pgfpathlineto{\pgfqpoint{1.576920in}{0.953485in}}%
\pgfpathclose%
\pgfusepath{fill}%
\end{pgfscope}%
\begin{pgfscope}%
\pgfpathrectangle{\pgfqpoint{0.100000in}{0.100000in}}{\pgfqpoint{3.608454in}{2.310000in}}%
\pgfusepath{clip}%
\pgfsetbuttcap%
\pgfsetmiterjoin%
\definecolor{currentfill}{rgb}{0.000000,0.584314,0.707843}%
\pgfsetfillcolor{currentfill}%
\pgfsetlinewidth{0.000000pt}%
\definecolor{currentstroke}{rgb}{0.000000,0.000000,0.000000}%
\pgfsetstrokecolor{currentstroke}%
\pgfsetstrokeopacity{0.000000}%
\pgfsetdash{}{0pt}%
\pgfpathmoveto{\pgfqpoint{3.145061in}{1.058883in}}%
\pgfpathlineto{\pgfqpoint{3.116266in}{1.079750in}}%
\pgfpathlineto{\pgfqpoint{3.099339in}{1.091621in}}%
\pgfpathlineto{\pgfqpoint{3.083124in}{1.089029in}}%
\pgfpathlineto{\pgfqpoint{3.086909in}{1.097540in}}%
\pgfpathlineto{\pgfqpoint{3.083344in}{1.107498in}}%
\pgfpathlineto{\pgfqpoint{3.078110in}{1.113287in}}%
\pgfpathlineto{\pgfqpoint{3.070169in}{1.111400in}}%
\pgfpathlineto{\pgfqpoint{3.070229in}{1.116768in}}%
\pgfpathlineto{\pgfqpoint{3.066320in}{1.122502in}}%
\pgfpathlineto{\pgfqpoint{3.067859in}{1.131866in}}%
\pgfpathlineto{\pgfqpoint{3.062964in}{1.138236in}}%
\pgfpathlineto{\pgfqpoint{3.057775in}{1.140860in}}%
\pgfpathlineto{\pgfqpoint{3.097477in}{1.148400in}}%
\pgfpathlineto{\pgfqpoint{3.110510in}{1.150879in}}%
\pgfpathlineto{\pgfqpoint{3.110689in}{1.147372in}}%
\pgfpathlineto{\pgfqpoint{3.121994in}{1.132796in}}%
\pgfpathlineto{\pgfqpoint{3.131128in}{1.128295in}}%
\pgfpathlineto{\pgfqpoint{3.149312in}{1.137485in}}%
\pgfpathlineto{\pgfqpoint{3.160885in}{1.137946in}}%
\pgfpathlineto{\pgfqpoint{3.160679in}{1.126465in}}%
\pgfpathlineto{\pgfqpoint{3.162834in}{1.117655in}}%
\pgfpathlineto{\pgfqpoint{3.174249in}{1.109020in}}%
\pgfpathlineto{\pgfqpoint{3.154787in}{1.104979in}}%
\pgfpathlineto{\pgfqpoint{3.148549in}{1.102631in}}%
\pgfpathlineto{\pgfqpoint{3.156648in}{1.092471in}}%
\pgfpathlineto{\pgfqpoint{3.156843in}{1.077749in}}%
\pgfpathlineto{\pgfqpoint{3.150261in}{1.071349in}}%
\pgfpathlineto{\pgfqpoint{3.145061in}{1.058883in}}%
\pgfpathclose%
\pgfusepath{fill}%
\end{pgfscope}%
\begin{pgfscope}%
\pgfpathrectangle{\pgfqpoint{0.100000in}{0.100000in}}{\pgfqpoint{3.608454in}{2.310000in}}%
\pgfusepath{clip}%
\pgfsetbuttcap%
\pgfsetmiterjoin%
\definecolor{currentfill}{rgb}{0.000000,0.435294,0.782353}%
\pgfsetfillcolor{currentfill}%
\pgfsetlinewidth{0.000000pt}%
\definecolor{currentstroke}{rgb}{0.000000,0.000000,0.000000}%
\pgfsetstrokecolor{currentstroke}%
\pgfsetstrokeopacity{0.000000}%
\pgfsetdash{}{0pt}%
\pgfpathmoveto{\pgfqpoint{1.964117in}{1.448246in}}%
\pgfpathlineto{\pgfqpoint{1.963356in}{1.420723in}}%
\pgfpathlineto{\pgfqpoint{1.908588in}{1.422459in}}%
\pgfpathlineto{\pgfqpoint{1.909610in}{1.449945in}}%
\pgfpathlineto{\pgfqpoint{1.936670in}{1.449037in}}%
\pgfpathlineto{\pgfqpoint{1.964117in}{1.448246in}}%
\pgfpathclose%
\pgfusepath{fill}%
\end{pgfscope}%
\begin{pgfscope}%
\pgfpathrectangle{\pgfqpoint{0.100000in}{0.100000in}}{\pgfqpoint{3.608454in}{2.310000in}}%
\pgfusepath{clip}%
\pgfsetbuttcap%
\pgfsetmiterjoin%
\definecolor{currentfill}{rgb}{0.000000,0.717647,0.641176}%
\pgfsetfillcolor{currentfill}%
\pgfsetlinewidth{0.000000pt}%
\definecolor{currentstroke}{rgb}{0.000000,0.000000,0.000000}%
\pgfsetstrokecolor{currentstroke}%
\pgfsetstrokeopacity{0.000000}%
\pgfsetdash{}{0pt}%
\pgfpathmoveto{\pgfqpoint{2.510006in}{1.121373in}}%
\pgfpathlineto{\pgfqpoint{2.498683in}{1.126546in}}%
\pgfpathlineto{\pgfqpoint{2.492139in}{1.131174in}}%
\pgfpathlineto{\pgfqpoint{2.479806in}{1.130609in}}%
\pgfpathlineto{\pgfqpoint{2.458843in}{1.129936in}}%
\pgfpathlineto{\pgfqpoint{2.462058in}{1.133851in}}%
\pgfpathlineto{\pgfqpoint{2.465696in}{1.145939in}}%
\pgfpathlineto{\pgfqpoint{2.461626in}{1.152803in}}%
\pgfpathlineto{\pgfqpoint{2.464384in}{1.163029in}}%
\pgfpathlineto{\pgfqpoint{2.473332in}{1.159111in}}%
\pgfpathlineto{\pgfqpoint{2.477166in}{1.167044in}}%
\pgfpathlineto{\pgfqpoint{2.474357in}{1.170803in}}%
\pgfpathlineto{\pgfqpoint{2.478809in}{1.176580in}}%
\pgfpathlineto{\pgfqpoint{2.498496in}{1.176925in}}%
\pgfpathlineto{\pgfqpoint{2.497579in}{1.190610in}}%
\pgfpathlineto{\pgfqpoint{2.518313in}{1.191718in}}%
\pgfpathlineto{\pgfqpoint{2.519321in}{1.176572in}}%
\pgfpathlineto{\pgfqpoint{2.520509in}{1.156888in}}%
\pgfpathlineto{\pgfqpoint{2.518781in}{1.156793in}}%
\pgfpathlineto{\pgfqpoint{2.519951in}{1.126253in}}%
\pgfpathlineto{\pgfqpoint{2.512080in}{1.125849in}}%
\pgfpathlineto{\pgfqpoint{2.510006in}{1.121373in}}%
\pgfpathclose%
\pgfusepath{fill}%
\end{pgfscope}%
\begin{pgfscope}%
\pgfpathrectangle{\pgfqpoint{0.100000in}{0.100000in}}{\pgfqpoint{3.608454in}{2.310000in}}%
\pgfusepath{clip}%
\pgfsetbuttcap%
\pgfsetmiterjoin%
\definecolor{currentfill}{rgb}{0.000000,0.474510,0.762745}%
\pgfsetfillcolor{currentfill}%
\pgfsetlinewidth{0.000000pt}%
\definecolor{currentstroke}{rgb}{0.000000,0.000000,0.000000}%
\pgfsetstrokecolor{currentstroke}%
\pgfsetstrokeopacity{0.000000}%
\pgfsetdash{}{0pt}%
\pgfpathmoveto{\pgfqpoint{1.401111in}{1.544070in}}%
\pgfpathlineto{\pgfqpoint{1.432841in}{1.539719in}}%
\pgfpathlineto{\pgfqpoint{1.436631in}{1.570812in}}%
\pgfpathlineto{\pgfqpoint{1.451617in}{1.568947in}}%
\pgfpathlineto{\pgfqpoint{1.457886in}{1.623314in}}%
\pgfpathlineto{\pgfqpoint{1.461309in}{1.650537in}}%
\pgfpathlineto{\pgfqpoint{1.491303in}{1.646739in}}%
\pgfpathlineto{\pgfqpoint{1.487410in}{1.642958in}}%
\pgfpathlineto{\pgfqpoint{1.486783in}{1.637214in}}%
\pgfpathlineto{\pgfqpoint{1.500514in}{1.634480in}}%
\pgfpathlineto{\pgfqpoint{1.504277in}{1.645283in}}%
\pgfpathlineto{\pgfqpoint{1.507108in}{1.644984in}}%
\pgfpathlineto{\pgfqpoint{1.503839in}{1.609878in}}%
\pgfpathlineto{\pgfqpoint{1.500517in}{1.584019in}}%
\pgfpathlineto{\pgfqpoint{1.494704in}{1.532257in}}%
\pgfpathlineto{\pgfqpoint{1.440561in}{1.538635in}}%
\pgfpathlineto{\pgfqpoint{1.440255in}{1.533604in}}%
\pgfpathlineto{\pgfqpoint{1.446700in}{1.523501in}}%
\pgfpathlineto{\pgfqpoint{1.455629in}{1.496066in}}%
\pgfpathlineto{\pgfqpoint{1.447426in}{1.486216in}}%
\pgfpathlineto{\pgfqpoint{1.444777in}{1.488383in}}%
\pgfpathlineto{\pgfqpoint{1.436560in}{1.486098in}}%
\pgfpathlineto{\pgfqpoint{1.426803in}{1.487168in}}%
\pgfpathlineto{\pgfqpoint{1.410993in}{1.493320in}}%
\pgfpathlineto{\pgfqpoint{1.407647in}{1.498770in}}%
\pgfpathlineto{\pgfqpoint{1.405998in}{1.509570in}}%
\pgfpathlineto{\pgfqpoint{1.412111in}{1.525836in}}%
\pgfpathlineto{\pgfqpoint{1.411585in}{1.530462in}}%
\pgfpathlineto{\pgfqpoint{1.400660in}{1.537925in}}%
\pgfpathlineto{\pgfqpoint{1.401111in}{1.544070in}}%
\pgfpathclose%
\pgfusepath{fill}%
\end{pgfscope}%
\begin{pgfscope}%
\pgfpathrectangle{\pgfqpoint{0.100000in}{0.100000in}}{\pgfqpoint{3.608454in}{2.310000in}}%
\pgfusepath{clip}%
\pgfsetbuttcap%
\pgfsetmiterjoin%
\definecolor{currentfill}{rgb}{0.000000,0.749020,0.625490}%
\pgfsetfillcolor{currentfill}%
\pgfsetlinewidth{0.000000pt}%
\definecolor{currentstroke}{rgb}{0.000000,0.000000,0.000000}%
\pgfsetstrokecolor{currentstroke}%
\pgfsetstrokeopacity{0.000000}%
\pgfsetdash{}{0pt}%
\pgfpathmoveto{\pgfqpoint{2.815217in}{1.759142in}}%
\pgfpathlineto{\pgfqpoint{2.786333in}{1.754076in}}%
\pgfpathlineto{\pgfqpoint{2.790730in}{1.726819in}}%
\pgfpathlineto{\pgfqpoint{2.776445in}{1.724555in}}%
\pgfpathlineto{\pgfqpoint{2.777478in}{1.717721in}}%
\pgfpathlineto{\pgfqpoint{2.771368in}{1.715974in}}%
\pgfpathlineto{\pgfqpoint{2.744382in}{1.712367in}}%
\pgfpathlineto{\pgfqpoint{2.745404in}{1.705512in}}%
\pgfpathlineto{\pgfqpoint{2.720255in}{1.702142in}}%
\pgfpathlineto{\pgfqpoint{2.717010in}{1.728719in}}%
\pgfpathlineto{\pgfqpoint{2.703428in}{1.727130in}}%
\pgfpathlineto{\pgfqpoint{2.697115in}{1.781860in}}%
\pgfpathlineto{\pgfqpoint{2.738013in}{1.786827in}}%
\pgfpathlineto{\pgfqpoint{2.756041in}{1.789197in}}%
\pgfpathlineto{\pgfqpoint{2.756426in}{1.784651in}}%
\pgfpathlineto{\pgfqpoint{2.746228in}{1.773703in}}%
\pgfpathlineto{\pgfqpoint{2.742788in}{1.773732in}}%
\pgfpathlineto{\pgfqpoint{2.738774in}{1.765761in}}%
\pgfpathlineto{\pgfqpoint{2.738244in}{1.753504in}}%
\pgfpathlineto{\pgfqpoint{2.741189in}{1.748238in}}%
\pgfpathlineto{\pgfqpoint{2.756060in}{1.743595in}}%
\pgfpathlineto{\pgfqpoint{2.775084in}{1.776539in}}%
\pgfpathlineto{\pgfqpoint{2.786967in}{1.781353in}}%
\pgfpathlineto{\pgfqpoint{2.792933in}{1.786497in}}%
\pgfpathlineto{\pgfqpoint{2.804037in}{1.782140in}}%
\pgfpathlineto{\pgfqpoint{2.812223in}{1.770016in}}%
\pgfpathlineto{\pgfqpoint{2.815217in}{1.759142in}}%
\pgfpathclose%
\pgfusepath{fill}%
\end{pgfscope}%
\begin{pgfscope}%
\pgfpathrectangle{\pgfqpoint{0.100000in}{0.100000in}}{\pgfqpoint{3.608454in}{2.310000in}}%
\pgfusepath{clip}%
\pgfsetbuttcap%
\pgfsetmiterjoin%
\definecolor{currentfill}{rgb}{0.000000,0.701961,0.649020}%
\pgfsetfillcolor{currentfill}%
\pgfsetlinewidth{0.000000pt}%
\definecolor{currentstroke}{rgb}{0.000000,0.000000,0.000000}%
\pgfsetstrokecolor{currentstroke}%
\pgfsetstrokeopacity{0.000000}%
\pgfsetdash{}{0pt}%
\pgfpathmoveto{\pgfqpoint{3.050650in}{1.387261in}}%
\pgfpathlineto{\pgfqpoint{3.050299in}{1.374365in}}%
\pgfpathlineto{\pgfqpoint{3.044609in}{1.363268in}}%
\pgfpathlineto{\pgfqpoint{3.046369in}{1.360627in}}%
\pgfpathlineto{\pgfqpoint{3.039437in}{1.352438in}}%
\pgfpathlineto{\pgfqpoint{3.038592in}{1.342913in}}%
\pgfpathlineto{\pgfqpoint{3.026105in}{1.338632in}}%
\pgfpathlineto{\pgfqpoint{3.020093in}{1.338488in}}%
\pgfpathlineto{\pgfqpoint{3.013175in}{1.342782in}}%
\pgfpathlineto{\pgfqpoint{3.011961in}{1.351516in}}%
\pgfpathlineto{\pgfqpoint{3.012394in}{1.360390in}}%
\pgfpathlineto{\pgfqpoint{3.016902in}{1.365346in}}%
\pgfpathlineto{\pgfqpoint{3.018910in}{1.376644in}}%
\pgfpathlineto{\pgfqpoint{3.010884in}{1.388904in}}%
\pgfpathlineto{\pgfqpoint{3.012891in}{1.393149in}}%
\pgfpathlineto{\pgfqpoint{3.019598in}{1.394420in}}%
\pgfpathlineto{\pgfqpoint{3.018667in}{1.406375in}}%
\pgfpathlineto{\pgfqpoint{3.021096in}{1.413057in}}%
\pgfpathlineto{\pgfqpoint{3.008758in}{1.422029in}}%
\pgfpathlineto{\pgfqpoint{3.010773in}{1.432733in}}%
\pgfpathlineto{\pgfqpoint{3.019335in}{1.434599in}}%
\pgfpathlineto{\pgfqpoint{3.025784in}{1.440073in}}%
\pgfpathlineto{\pgfqpoint{3.032463in}{1.435406in}}%
\pgfpathlineto{\pgfqpoint{3.039311in}{1.439843in}}%
\pgfpathlineto{\pgfqpoint{3.052407in}{1.435951in}}%
\pgfpathlineto{\pgfqpoint{3.060760in}{1.438938in}}%
\pgfpathlineto{\pgfqpoint{3.063986in}{1.437446in}}%
\pgfpathlineto{\pgfqpoint{3.062165in}{1.429526in}}%
\pgfpathlineto{\pgfqpoint{3.065468in}{1.428454in}}%
\pgfpathlineto{\pgfqpoint{3.063909in}{1.418899in}}%
\pgfpathlineto{\pgfqpoint{3.055324in}{1.410700in}}%
\pgfpathlineto{\pgfqpoint{3.056652in}{1.404504in}}%
\pgfpathlineto{\pgfqpoint{3.050650in}{1.387261in}}%
\pgfpathclose%
\pgfusepath{fill}%
\end{pgfscope}%
\begin{pgfscope}%
\pgfpathrectangle{\pgfqpoint{0.100000in}{0.100000in}}{\pgfqpoint{3.608454in}{2.310000in}}%
\pgfusepath{clip}%
\pgfsetbuttcap%
\pgfsetmiterjoin%
\definecolor{currentfill}{rgb}{0.000000,0.600000,0.700000}%
\pgfsetfillcolor{currentfill}%
\pgfsetlinewidth{0.000000pt}%
\definecolor{currentstroke}{rgb}{0.000000,0.000000,0.000000}%
\pgfsetstrokecolor{currentstroke}%
\pgfsetstrokeopacity{0.000000}%
\pgfsetdash{}{0pt}%
\pgfpathmoveto{\pgfqpoint{3.217063in}{1.727096in}}%
\pgfpathlineto{\pgfqpoint{3.194930in}{1.721757in}}%
\pgfpathlineto{\pgfqpoint{3.190729in}{1.738354in}}%
\pgfpathlineto{\pgfqpoint{3.167852in}{1.733612in}}%
\pgfpathlineto{\pgfqpoint{3.162874in}{1.739825in}}%
\pgfpathlineto{\pgfqpoint{3.158361in}{1.759713in}}%
\pgfpathlineto{\pgfqpoint{3.158574in}{1.768645in}}%
\pgfpathlineto{\pgfqpoint{3.153298in}{1.788862in}}%
\pgfpathlineto{\pgfqpoint{3.157336in}{1.795352in}}%
\pgfpathlineto{\pgfqpoint{3.167511in}{1.806116in}}%
\pgfpathlineto{\pgfqpoint{3.177989in}{1.809166in}}%
\pgfpathlineto{\pgfqpoint{3.177265in}{1.821081in}}%
\pgfpathlineto{\pgfqpoint{3.186645in}{1.825233in}}%
\pgfpathlineto{\pgfqpoint{3.200994in}{1.826823in}}%
\pgfpathlineto{\pgfqpoint{3.205664in}{1.810106in}}%
\pgfpathlineto{\pgfqpoint{3.219039in}{1.808893in}}%
\pgfpathlineto{\pgfqpoint{3.239480in}{1.829258in}}%
\pgfpathlineto{\pgfqpoint{3.227847in}{1.865817in}}%
\pgfpathlineto{\pgfqpoint{3.234624in}{1.863539in}}%
\pgfpathlineto{\pgfqpoint{3.245888in}{1.867665in}}%
\pgfpathlineto{\pgfqpoint{3.260468in}{1.823521in}}%
\pgfpathlineto{\pgfqpoint{3.257858in}{1.810998in}}%
\pgfpathlineto{\pgfqpoint{3.267521in}{1.808820in}}%
\pgfpathlineto{\pgfqpoint{3.270444in}{1.800501in}}%
\pgfpathlineto{\pgfqpoint{3.267628in}{1.792368in}}%
\pgfpathlineto{\pgfqpoint{3.271017in}{1.786225in}}%
\pgfpathlineto{\pgfqpoint{3.272069in}{1.775491in}}%
\pgfpathlineto{\pgfqpoint{3.264917in}{1.776723in}}%
\pgfpathlineto{\pgfqpoint{3.264610in}{1.770699in}}%
\pgfpathlineto{\pgfqpoint{3.252161in}{1.774686in}}%
\pgfpathlineto{\pgfqpoint{3.246838in}{1.769301in}}%
\pgfpathlineto{\pgfqpoint{3.243936in}{1.759541in}}%
\pgfpathlineto{\pgfqpoint{3.210494in}{1.750645in}}%
\pgfpathlineto{\pgfqpoint{3.217063in}{1.727096in}}%
\pgfpathclose%
\pgfusepath{fill}%
\end{pgfscope}%
\begin{pgfscope}%
\pgfpathrectangle{\pgfqpoint{0.100000in}{0.100000in}}{\pgfqpoint{3.608454in}{2.310000in}}%
\pgfusepath{clip}%
\pgfsetbuttcap%
\pgfsetmiterjoin%
\definecolor{currentfill}{rgb}{0.000000,0.443137,0.778431}%
\pgfsetfillcolor{currentfill}%
\pgfsetlinewidth{0.000000pt}%
\definecolor{currentstroke}{rgb}{0.000000,0.000000,0.000000}%
\pgfsetstrokecolor{currentstroke}%
\pgfsetstrokeopacity{0.000000}%
\pgfsetdash{}{0pt}%
\pgfpathmoveto{\pgfqpoint{2.115090in}{1.593467in}}%
\pgfpathlineto{\pgfqpoint{2.155774in}{1.593709in}}%
\pgfpathlineto{\pgfqpoint{2.156052in}{1.566272in}}%
\pgfpathlineto{\pgfqpoint{2.121832in}{1.566001in}}%
\pgfpathlineto{\pgfqpoint{2.073952in}{1.566162in}}%
\pgfpathlineto{\pgfqpoint{2.067148in}{1.566188in}}%
\pgfpathlineto{\pgfqpoint{2.067449in}{1.593735in}}%
\pgfpathlineto{\pgfqpoint{2.115090in}{1.593467in}}%
\pgfpathclose%
\pgfusepath{fill}%
\end{pgfscope}%
\begin{pgfscope}%
\pgfpathrectangle{\pgfqpoint{0.100000in}{0.100000in}}{\pgfqpoint{3.608454in}{2.310000in}}%
\pgfusepath{clip}%
\pgfsetbuttcap%
\pgfsetmiterjoin%
\definecolor{currentfill}{rgb}{0.000000,0.603922,0.698039}%
\pgfsetfillcolor{currentfill}%
\pgfsetlinewidth{0.000000pt}%
\definecolor{currentstroke}{rgb}{0.000000,0.000000,0.000000}%
\pgfsetstrokecolor{currentstroke}%
\pgfsetstrokeopacity{0.000000}%
\pgfsetdash{}{0pt}%
\pgfpathmoveto{\pgfqpoint{2.647203in}{0.861864in}}%
\pgfpathlineto{\pgfqpoint{2.648743in}{0.852393in}}%
\pgfpathlineto{\pgfqpoint{2.656320in}{0.836856in}}%
\pgfpathlineto{\pgfqpoint{2.650875in}{0.827115in}}%
\pgfpathlineto{\pgfqpoint{2.652230in}{0.813085in}}%
\pgfpathlineto{\pgfqpoint{2.634007in}{0.811449in}}%
\pgfpathlineto{\pgfqpoint{2.622706in}{0.822514in}}%
\pgfpathlineto{\pgfqpoint{2.612796in}{0.827022in}}%
\pgfpathlineto{\pgfqpoint{2.609026in}{0.830093in}}%
\pgfpathlineto{\pgfqpoint{2.607870in}{0.843949in}}%
\pgfpathlineto{\pgfqpoint{2.594215in}{0.842750in}}%
\pgfpathlineto{\pgfqpoint{2.588335in}{0.845764in}}%
\pgfpathlineto{\pgfqpoint{2.589627in}{0.852306in}}%
\pgfpathlineto{\pgfqpoint{2.586111in}{0.864841in}}%
\pgfpathlineto{\pgfqpoint{2.586494in}{0.875360in}}%
\pgfpathlineto{\pgfqpoint{2.591160in}{0.879866in}}%
\pgfpathlineto{\pgfqpoint{2.591982in}{0.885391in}}%
\pgfpathlineto{\pgfqpoint{2.583933in}{0.884712in}}%
\pgfpathlineto{\pgfqpoint{2.582729in}{0.895488in}}%
\pgfpathlineto{\pgfqpoint{2.580198in}{0.924865in}}%
\pgfpathlineto{\pgfqpoint{2.591702in}{0.925524in}}%
\pgfpathlineto{\pgfqpoint{2.593439in}{0.932696in}}%
\pgfpathlineto{\pgfqpoint{2.610666in}{0.933472in}}%
\pgfpathlineto{\pgfqpoint{2.618130in}{0.927031in}}%
\pgfpathlineto{\pgfqpoint{2.617467in}{0.921208in}}%
\pgfpathlineto{\pgfqpoint{2.625197in}{0.912348in}}%
\pgfpathlineto{\pgfqpoint{2.635819in}{0.907320in}}%
\pgfpathlineto{\pgfqpoint{2.636464in}{0.900943in}}%
\pgfpathlineto{\pgfqpoint{2.640666in}{0.895938in}}%
\pgfpathlineto{\pgfqpoint{2.647138in}{0.892648in}}%
\pgfpathlineto{\pgfqpoint{2.648704in}{0.875737in}}%
\pgfpathlineto{\pgfqpoint{2.639182in}{0.874953in}}%
\pgfpathlineto{\pgfqpoint{2.640493in}{0.861198in}}%
\pgfpathlineto{\pgfqpoint{2.647203in}{0.861864in}}%
\pgfpathclose%
\pgfusepath{fill}%
\end{pgfscope}%
\begin{pgfscope}%
\pgfpathrectangle{\pgfqpoint{0.100000in}{0.100000in}}{\pgfqpoint{3.608454in}{2.310000in}}%
\pgfusepath{clip}%
\pgfsetbuttcap%
\pgfsetmiterjoin%
\definecolor{currentfill}{rgb}{0.000000,0.321569,0.839216}%
\pgfsetfillcolor{currentfill}%
\pgfsetlinewidth{0.000000pt}%
\definecolor{currentstroke}{rgb}{0.000000,0.000000,0.000000}%
\pgfsetstrokecolor{currentstroke}%
\pgfsetstrokeopacity{0.000000}%
\pgfsetdash{}{0pt}%
\pgfpathmoveto{\pgfqpoint{2.223956in}{1.594717in}}%
\pgfpathlineto{\pgfqpoint{2.210467in}{1.594411in}}%
\pgfpathlineto{\pgfqpoint{2.196902in}{1.594193in}}%
\pgfpathlineto{\pgfqpoint{2.196575in}{1.614898in}}%
\pgfpathlineto{\pgfqpoint{2.194446in}{1.614873in}}%
\pgfpathlineto{\pgfqpoint{2.193901in}{1.649520in}}%
\pgfpathlineto{\pgfqpoint{2.221382in}{1.649990in}}%
\pgfpathlineto{\pgfqpoint{2.222007in}{1.615392in}}%
\pgfpathlineto{\pgfqpoint{2.223427in}{1.615413in}}%
\pgfpathlineto{\pgfqpoint{2.223956in}{1.594717in}}%
\pgfpathclose%
\pgfusepath{fill}%
\end{pgfscope}%
\begin{pgfscope}%
\pgfpathrectangle{\pgfqpoint{0.100000in}{0.100000in}}{\pgfqpoint{3.608454in}{2.310000in}}%
\pgfusepath{clip}%
\pgfsetbuttcap%
\pgfsetmiterjoin%
\definecolor{currentfill}{rgb}{0.000000,0.694118,0.652941}%
\pgfsetfillcolor{currentfill}%
\pgfsetlinewidth{0.000000pt}%
\definecolor{currentstroke}{rgb}{0.000000,0.000000,0.000000}%
\pgfsetstrokecolor{currentstroke}%
\pgfsetstrokeopacity{0.000000}%
\pgfsetdash{}{0pt}%
\pgfpathmoveto{\pgfqpoint{2.720255in}{1.702142in}}%
\pgfpathlineto{\pgfqpoint{2.720350in}{1.701323in}}%
\pgfpathlineto{\pgfqpoint{2.693396in}{1.698299in}}%
\pgfpathlineto{\pgfqpoint{2.689796in}{1.725591in}}%
\pgfpathlineto{\pgfqpoint{2.676133in}{1.724089in}}%
\pgfpathlineto{\pgfqpoint{2.669984in}{1.779116in}}%
\pgfpathlineto{\pgfqpoint{2.683319in}{1.780357in}}%
\pgfpathlineto{\pgfqpoint{2.697115in}{1.781860in}}%
\pgfpathlineto{\pgfqpoint{2.703428in}{1.727130in}}%
\pgfpathlineto{\pgfqpoint{2.717010in}{1.728719in}}%
\pgfpathlineto{\pgfqpoint{2.720255in}{1.702142in}}%
\pgfpathclose%
\pgfusepath{fill}%
\end{pgfscope}%
\begin{pgfscope}%
\pgfpathrectangle{\pgfqpoint{0.100000in}{0.100000in}}{\pgfqpoint{3.608454in}{2.310000in}}%
\pgfusepath{clip}%
\pgfsetbuttcap%
\pgfsetmiterjoin%
\definecolor{currentfill}{rgb}{0.000000,0.635294,0.682353}%
\pgfsetfillcolor{currentfill}%
\pgfsetlinewidth{0.000000pt}%
\definecolor{currentstroke}{rgb}{0.000000,0.000000,0.000000}%
\pgfsetstrokecolor{currentstroke}%
\pgfsetstrokeopacity{0.000000}%
\pgfsetdash{}{0pt}%
\pgfpathmoveto{\pgfqpoint{2.882548in}{1.076650in}}%
\pgfpathlineto{\pgfqpoint{2.879644in}{1.078954in}}%
\pgfpathlineto{\pgfqpoint{2.885380in}{1.093119in}}%
\pgfpathlineto{\pgfqpoint{2.884882in}{1.098977in}}%
\pgfpathlineto{\pgfqpoint{2.866841in}{1.114054in}}%
\pgfpathlineto{\pgfqpoint{2.864797in}{1.125728in}}%
\pgfpathlineto{\pgfqpoint{2.859932in}{1.127064in}}%
\pgfpathlineto{\pgfqpoint{2.865280in}{1.133278in}}%
\pgfpathlineto{\pgfqpoint{2.877093in}{1.136816in}}%
\pgfpathlineto{\pgfqpoint{2.879988in}{1.149617in}}%
\pgfpathlineto{\pgfqpoint{2.895600in}{1.161365in}}%
\pgfpathlineto{\pgfqpoint{2.898012in}{1.153842in}}%
\pgfpathlineto{\pgfqpoint{2.904596in}{1.155570in}}%
\pgfpathlineto{\pgfqpoint{2.905552in}{1.151887in}}%
\pgfpathlineto{\pgfqpoint{2.916363in}{1.143909in}}%
\pgfpathlineto{\pgfqpoint{2.925369in}{1.125057in}}%
\pgfpathlineto{\pgfqpoint{2.930944in}{1.123301in}}%
\pgfpathlineto{\pgfqpoint{2.926944in}{1.122067in}}%
\pgfpathlineto{\pgfqpoint{2.924776in}{1.115579in}}%
\pgfpathlineto{\pgfqpoint{2.926537in}{1.111891in}}%
\pgfpathlineto{\pgfqpoint{2.922309in}{1.102781in}}%
\pgfpathlineto{\pgfqpoint{2.922867in}{1.095204in}}%
\pgfpathlineto{\pgfqpoint{2.903153in}{1.086612in}}%
\pgfpathlineto{\pgfqpoint{2.882548in}{1.076650in}}%
\pgfpathclose%
\pgfusepath{fill}%
\end{pgfscope}%
\begin{pgfscope}%
\pgfpathrectangle{\pgfqpoint{0.100000in}{0.100000in}}{\pgfqpoint{3.608454in}{2.310000in}}%
\pgfusepath{clip}%
\pgfsetbuttcap%
\pgfsetmiterjoin%
\definecolor{currentfill}{rgb}{0.000000,0.764706,0.617647}%
\pgfsetfillcolor{currentfill}%
\pgfsetlinewidth{0.000000pt}%
\definecolor{currentstroke}{rgb}{0.000000,0.000000,0.000000}%
\pgfsetstrokecolor{currentstroke}%
\pgfsetstrokeopacity{0.000000}%
\pgfsetdash{}{0pt}%
\pgfpathmoveto{\pgfqpoint{2.272396in}{1.102853in}}%
\pgfpathlineto{\pgfqpoint{2.272664in}{1.088030in}}%
\pgfpathlineto{\pgfqpoint{2.258979in}{1.087834in}}%
\pgfpathlineto{\pgfqpoint{2.259040in}{1.082097in}}%
\pgfpathlineto{\pgfqpoint{2.247558in}{1.082128in}}%
\pgfpathlineto{\pgfqpoint{2.201928in}{1.082158in}}%
\pgfpathlineto{\pgfqpoint{2.201915in}{1.084462in}}%
\pgfpathlineto{\pgfqpoint{2.201478in}{1.093804in}}%
\pgfpathlineto{\pgfqpoint{2.206038in}{1.100680in}}%
\pgfpathlineto{\pgfqpoint{2.206212in}{1.109649in}}%
\pgfpathlineto{\pgfqpoint{2.199417in}{1.113130in}}%
\pgfpathlineto{\pgfqpoint{2.194911in}{1.119004in}}%
\pgfpathlineto{\pgfqpoint{2.192391in}{1.127027in}}%
\pgfpathlineto{\pgfqpoint{2.178906in}{1.127188in}}%
\pgfpathlineto{\pgfqpoint{2.178864in}{1.142325in}}%
\pgfpathlineto{\pgfqpoint{2.243255in}{1.143432in}}%
\pgfpathlineto{\pgfqpoint{2.241311in}{1.142821in}}%
\pgfpathlineto{\pgfqpoint{2.241638in}{1.113006in}}%
\pgfpathlineto{\pgfqpoint{2.245103in}{1.109604in}}%
\pgfpathlineto{\pgfqpoint{2.272381in}{1.109733in}}%
\pgfpathlineto{\pgfqpoint{2.272396in}{1.102853in}}%
\pgfpathclose%
\pgfusepath{fill}%
\end{pgfscope}%
\begin{pgfscope}%
\pgfpathrectangle{\pgfqpoint{0.100000in}{0.100000in}}{\pgfqpoint{3.608454in}{2.310000in}}%
\pgfusepath{clip}%
\pgfsetbuttcap%
\pgfsetmiterjoin%
\definecolor{currentfill}{rgb}{0.000000,0.447059,0.776471}%
\pgfsetfillcolor{currentfill}%
\pgfsetlinewidth{0.000000pt}%
\definecolor{currentstroke}{rgb}{0.000000,0.000000,0.000000}%
\pgfsetstrokecolor{currentstroke}%
\pgfsetstrokeopacity{0.000000}%
\pgfsetdash{}{0pt}%
\pgfpathmoveto{\pgfqpoint{2.526558in}{1.632431in}}%
\pgfpathlineto{\pgfqpoint{2.473885in}{1.628654in}}%
\pgfpathlineto{\pgfqpoint{2.435480in}{1.626662in}}%
\pgfpathlineto{\pgfqpoint{2.433615in}{1.653903in}}%
\pgfpathlineto{\pgfqpoint{2.454210in}{1.655405in}}%
\pgfpathlineto{\pgfqpoint{2.467930in}{1.655975in}}%
\pgfpathlineto{\pgfqpoint{2.522980in}{1.660007in}}%
\pgfpathlineto{\pgfqpoint{2.527343in}{1.655339in}}%
\pgfpathlineto{\pgfqpoint{2.524832in}{1.642175in}}%
\pgfpathlineto{\pgfqpoint{2.526558in}{1.632431in}}%
\pgfpathclose%
\pgfusepath{fill}%
\end{pgfscope}%
\begin{pgfscope}%
\pgfpathrectangle{\pgfqpoint{0.100000in}{0.100000in}}{\pgfqpoint{3.608454in}{2.310000in}}%
\pgfusepath{clip}%
\pgfsetbuttcap%
\pgfsetmiterjoin%
\definecolor{currentfill}{rgb}{0.000000,0.501961,0.749020}%
\pgfsetfillcolor{currentfill}%
\pgfsetlinewidth{0.000000pt}%
\definecolor{currentstroke}{rgb}{0.000000,0.000000,0.000000}%
\pgfsetstrokecolor{currentstroke}%
\pgfsetstrokeopacity{0.000000}%
\pgfsetdash{}{0pt}%
\pgfpathmoveto{\pgfqpoint{1.774163in}{0.940149in}}%
\pgfpathlineto{\pgfqpoint{1.743217in}{0.941925in}}%
\pgfpathlineto{\pgfqpoint{1.745610in}{0.979809in}}%
\pgfpathlineto{\pgfqpoint{1.752117in}{0.979371in}}%
\pgfpathlineto{\pgfqpoint{1.754289in}{1.013713in}}%
\pgfpathlineto{\pgfqpoint{1.781300in}{1.012079in}}%
\pgfpathlineto{\pgfqpoint{1.780677in}{0.997355in}}%
\pgfpathlineto{\pgfqpoint{1.779205in}{0.970733in}}%
\pgfpathlineto{\pgfqpoint{1.775993in}{0.971342in}}%
\pgfpathlineto{\pgfqpoint{1.774163in}{0.940149in}}%
\pgfpathclose%
\pgfusepath{fill}%
\end{pgfscope}%
\begin{pgfscope}%
\pgfpathrectangle{\pgfqpoint{0.100000in}{0.100000in}}{\pgfqpoint{3.608454in}{2.310000in}}%
\pgfusepath{clip}%
\pgfsetbuttcap%
\pgfsetmiterjoin%
\definecolor{currentfill}{rgb}{0.000000,0.811765,0.594118}%
\pgfsetfillcolor{currentfill}%
\pgfsetlinewidth{0.000000pt}%
\definecolor{currentstroke}{rgb}{0.000000,0.000000,0.000000}%
\pgfsetstrokecolor{currentstroke}%
\pgfsetstrokeopacity{0.000000}%
\pgfsetdash{}{0pt}%
\pgfpathmoveto{\pgfqpoint{2.338322in}{0.779971in}}%
\pgfpathlineto{\pgfqpoint{2.338396in}{0.773253in}}%
\pgfpathlineto{\pgfqpoint{2.334485in}{0.767523in}}%
\pgfpathlineto{\pgfqpoint{2.330853in}{0.768113in}}%
\pgfpathlineto{\pgfqpoint{2.325362in}{0.751678in}}%
\pgfpathlineto{\pgfqpoint{2.319974in}{0.746106in}}%
\pgfpathlineto{\pgfqpoint{2.322557in}{0.730252in}}%
\pgfpathlineto{\pgfqpoint{2.311015in}{0.733955in}}%
\pgfpathlineto{\pgfqpoint{2.300756in}{0.740640in}}%
\pgfpathlineto{\pgfqpoint{2.299838in}{0.745415in}}%
\pgfpathlineto{\pgfqpoint{2.291967in}{0.750029in}}%
\pgfpathlineto{\pgfqpoint{2.286412in}{0.757594in}}%
\pgfpathlineto{\pgfqpoint{2.285332in}{0.771355in}}%
\pgfpathlineto{\pgfqpoint{2.288408in}{0.781863in}}%
\pgfpathlineto{\pgfqpoint{2.315987in}{0.782756in}}%
\pgfpathlineto{\pgfqpoint{2.319400in}{0.785321in}}%
\pgfpathlineto{\pgfqpoint{2.324523in}{0.778629in}}%
\pgfpathlineto{\pgfqpoint{2.336629in}{0.786857in}}%
\pgfpathlineto{\pgfqpoint{2.338322in}{0.779971in}}%
\pgfpathclose%
\pgfusepath{fill}%
\end{pgfscope}%
\begin{pgfscope}%
\pgfpathrectangle{\pgfqpoint{0.100000in}{0.100000in}}{\pgfqpoint{3.608454in}{2.310000in}}%
\pgfusepath{clip}%
\pgfsetbuttcap%
\pgfsetmiterjoin%
\definecolor{currentfill}{rgb}{0.000000,0.654902,0.672549}%
\pgfsetfillcolor{currentfill}%
\pgfsetlinewidth{0.000000pt}%
\definecolor{currentstroke}{rgb}{0.000000,0.000000,0.000000}%
\pgfsetstrokecolor{currentstroke}%
\pgfsetstrokeopacity{0.000000}%
\pgfsetdash{}{0pt}%
\pgfpathmoveto{\pgfqpoint{2.618130in}{0.927031in}}%
\pgfpathlineto{\pgfqpoint{2.610666in}{0.933472in}}%
\pgfpathlineto{\pgfqpoint{2.593439in}{0.932696in}}%
\pgfpathlineto{\pgfqpoint{2.591702in}{0.925524in}}%
\pgfpathlineto{\pgfqpoint{2.580198in}{0.924865in}}%
\pgfpathlineto{\pgfqpoint{2.551506in}{0.923399in}}%
\pgfpathlineto{\pgfqpoint{2.551959in}{0.940228in}}%
\pgfpathlineto{\pgfqpoint{2.553484in}{0.986177in}}%
\pgfpathlineto{\pgfqpoint{2.595636in}{0.988190in}}%
\pgfpathlineto{\pgfqpoint{2.623076in}{0.990093in}}%
\pgfpathlineto{\pgfqpoint{2.625447in}{0.963052in}}%
\pgfpathlineto{\pgfqpoint{2.627618in}{0.959226in}}%
\pgfpathlineto{\pgfqpoint{2.635731in}{0.956087in}}%
\pgfpathlineto{\pgfqpoint{2.632392in}{0.948970in}}%
\pgfpathlineto{\pgfqpoint{2.631914in}{0.941959in}}%
\pgfpathlineto{\pgfqpoint{2.628827in}{0.937113in}}%
\pgfpathlineto{\pgfqpoint{2.623213in}{0.935505in}}%
\pgfpathlineto{\pgfqpoint{2.618130in}{0.927031in}}%
\pgfpathclose%
\pgfusepath{fill}%
\end{pgfscope}%
\begin{pgfscope}%
\pgfpathrectangle{\pgfqpoint{0.100000in}{0.100000in}}{\pgfqpoint{3.608454in}{2.310000in}}%
\pgfusepath{clip}%
\pgfsetbuttcap%
\pgfsetmiterjoin%
\definecolor{currentfill}{rgb}{0.000000,0.525490,0.737255}%
\pgfsetfillcolor{currentfill}%
\pgfsetlinewidth{0.000000pt}%
\definecolor{currentstroke}{rgb}{0.000000,0.000000,0.000000}%
\pgfsetstrokecolor{currentstroke}%
\pgfsetstrokeopacity{0.000000}%
\pgfsetdash{}{0pt}%
\pgfpathmoveto{\pgfqpoint{2.949055in}{1.160645in}}%
\pgfpathlineto{\pgfqpoint{2.956204in}{1.154739in}}%
\pgfpathlineto{\pgfqpoint{2.967129in}{1.148531in}}%
\pgfpathlineto{\pgfqpoint{2.973998in}{1.150189in}}%
\pgfpathlineto{\pgfqpoint{2.978683in}{1.148660in}}%
\pgfpathlineto{\pgfqpoint{2.970938in}{1.132406in}}%
\pgfpathlineto{\pgfqpoint{2.963707in}{1.130755in}}%
\pgfpathlineto{\pgfqpoint{2.956399in}{1.133737in}}%
\pgfpathlineto{\pgfqpoint{2.951598in}{1.127420in}}%
\pgfpathlineto{\pgfqpoint{2.943450in}{1.124830in}}%
\pgfpathlineto{\pgfqpoint{2.930944in}{1.123301in}}%
\pgfpathlineto{\pgfqpoint{2.925369in}{1.125057in}}%
\pgfpathlineto{\pgfqpoint{2.916363in}{1.143909in}}%
\pgfpathlineto{\pgfqpoint{2.905552in}{1.151887in}}%
\pgfpathlineto{\pgfqpoint{2.904596in}{1.155570in}}%
\pgfpathlineto{\pgfqpoint{2.909797in}{1.165077in}}%
\pgfpathlineto{\pgfqpoint{2.920582in}{1.172131in}}%
\pgfpathlineto{\pgfqpoint{2.930144in}{1.169183in}}%
\pgfpathlineto{\pgfqpoint{2.934882in}{1.156995in}}%
\pgfpathlineto{\pgfqpoint{2.938556in}{1.155078in}}%
\pgfpathlineto{\pgfqpoint{2.942760in}{1.162773in}}%
\pgfpathlineto{\pgfqpoint{2.949055in}{1.160645in}}%
\pgfpathclose%
\pgfusepath{fill}%
\end{pgfscope}%
\begin{pgfscope}%
\pgfpathrectangle{\pgfqpoint{0.100000in}{0.100000in}}{\pgfqpoint{3.608454in}{2.310000in}}%
\pgfusepath{clip}%
\pgfsetbuttcap%
\pgfsetmiterjoin%
\definecolor{currentfill}{rgb}{0.000000,0.603922,0.698039}%
\pgfsetfillcolor{currentfill}%
\pgfsetlinewidth{0.000000pt}%
\definecolor{currentstroke}{rgb}{0.000000,0.000000,0.000000}%
\pgfsetstrokecolor{currentstroke}%
\pgfsetstrokeopacity{0.000000}%
\pgfsetdash{}{0pt}%
\pgfpathmoveto{\pgfqpoint{0.650466in}{2.312218in}}%
\pgfpathlineto{\pgfqpoint{0.640068in}{2.318805in}}%
\pgfpathlineto{\pgfqpoint{0.635270in}{2.326403in}}%
\pgfpathlineto{\pgfqpoint{0.636669in}{2.333571in}}%
\pgfpathlineto{\pgfqpoint{0.643160in}{2.329044in}}%
\pgfpathlineto{\pgfqpoint{0.650928in}{2.336446in}}%
\pgfpathlineto{\pgfqpoint{0.659748in}{2.329464in}}%
\pgfpathlineto{\pgfqpoint{0.650466in}{2.312218in}}%
\pgfpathclose%
\pgfusepath{fill}%
\end{pgfscope}%
\begin{pgfscope}%
\pgfpathrectangle{\pgfqpoint{0.100000in}{0.100000in}}{\pgfqpoint{3.608454in}{2.310000in}}%
\pgfusepath{clip}%
\pgfsetbuttcap%
\pgfsetmiterjoin%
\definecolor{currentfill}{rgb}{0.000000,0.556863,0.721569}%
\pgfsetfillcolor{currentfill}%
\pgfsetlinewidth{0.000000pt}%
\definecolor{currentstroke}{rgb}{0.000000,0.000000,0.000000}%
\pgfsetstrokecolor{currentstroke}%
\pgfsetstrokeopacity{0.000000}%
\pgfsetdash{}{0pt}%
\pgfpathmoveto{\pgfqpoint{1.644176in}{0.728867in}}%
\pgfpathlineto{\pgfqpoint{1.678348in}{0.726376in}}%
\pgfpathlineto{\pgfqpoint{1.699678in}{0.725186in}}%
\pgfpathlineto{\pgfqpoint{1.697817in}{0.695195in}}%
\pgfpathlineto{\pgfqpoint{1.695622in}{0.661844in}}%
\pgfpathlineto{\pgfqpoint{1.640690in}{0.665716in}}%
\pgfpathlineto{\pgfqpoint{1.647908in}{0.670831in}}%
\pgfpathlineto{\pgfqpoint{1.642706in}{0.676280in}}%
\pgfpathlineto{\pgfqpoint{1.649534in}{0.685985in}}%
\pgfpathlineto{\pgfqpoint{1.648797in}{0.693179in}}%
\pgfpathlineto{\pgfqpoint{1.644365in}{0.694777in}}%
\pgfpathlineto{\pgfqpoint{1.616885in}{0.697027in}}%
\pgfpathlineto{\pgfqpoint{1.616475in}{0.692376in}}%
\pgfpathlineto{\pgfqpoint{1.602461in}{0.693594in}}%
\pgfpathlineto{\pgfqpoint{1.600425in}{0.668531in}}%
\pgfpathlineto{\pgfqpoint{1.585018in}{0.669731in}}%
\pgfpathlineto{\pgfqpoint{1.583492in}{0.651351in}}%
\pgfpathlineto{\pgfqpoint{1.518651in}{0.714404in}}%
\pgfpathlineto{\pgfqpoint{1.562114in}{0.758888in}}%
\pgfpathlineto{\pgfqpoint{1.567213in}{0.756734in}}%
\pgfpathlineto{\pgfqpoint{1.574315in}{0.749565in}}%
\pgfpathlineto{\pgfqpoint{1.578119in}{0.751288in}}%
\pgfpathlineto{\pgfqpoint{1.584520in}{0.753390in}}%
\pgfpathlineto{\pgfqpoint{1.599669in}{0.742990in}}%
\pgfpathlineto{\pgfqpoint{1.601306in}{0.732857in}}%
\pgfpathlineto{\pgfqpoint{1.644176in}{0.728867in}}%
\pgfpathclose%
\pgfusepath{fill}%
\end{pgfscope}%
\begin{pgfscope}%
\pgfpathrectangle{\pgfqpoint{0.100000in}{0.100000in}}{\pgfqpoint{3.608454in}{2.310000in}}%
\pgfusepath{clip}%
\pgfsetbuttcap%
\pgfsetmiterjoin%
\definecolor{currentfill}{rgb}{0.000000,0.745098,0.627451}%
\pgfsetfillcolor{currentfill}%
\pgfsetlinewidth{0.000000pt}%
\definecolor{currentstroke}{rgb}{0.000000,0.000000,0.000000}%
\pgfsetstrokecolor{currentstroke}%
\pgfsetstrokeopacity{0.000000}%
\pgfsetdash{}{0pt}%
\pgfpathmoveto{\pgfqpoint{2.297141in}{0.880461in}}%
\pgfpathlineto{\pgfqpoint{2.299404in}{0.870120in}}%
\pgfpathlineto{\pgfqpoint{2.302086in}{0.868312in}}%
\pgfpathlineto{\pgfqpoint{2.258381in}{0.867551in}}%
\pgfpathlineto{\pgfqpoint{2.240807in}{0.867381in}}%
\pgfpathlineto{\pgfqpoint{2.240853in}{0.895980in}}%
\pgfpathlineto{\pgfqpoint{2.231529in}{0.896018in}}%
\pgfpathlineto{\pgfqpoint{2.231585in}{0.901788in}}%
\pgfpathlineto{\pgfqpoint{2.231849in}{0.927551in}}%
\pgfpathlineto{\pgfqpoint{2.241965in}{0.929323in}}%
\pgfpathlineto{\pgfqpoint{2.245189in}{0.934157in}}%
\pgfpathlineto{\pgfqpoint{2.245528in}{0.957904in}}%
\pgfpathlineto{\pgfqpoint{2.259184in}{0.957741in}}%
\pgfpathlineto{\pgfqpoint{2.271324in}{0.957636in}}%
\pgfpathlineto{\pgfqpoint{2.281822in}{0.951221in}}%
\pgfpathlineto{\pgfqpoint{2.272954in}{0.950762in}}%
\pgfpathlineto{\pgfqpoint{2.274161in}{0.944083in}}%
\pgfpathlineto{\pgfqpoint{2.282183in}{0.933447in}}%
\pgfpathlineto{\pgfqpoint{2.284812in}{0.906848in}}%
\pgfpathlineto{\pgfqpoint{2.280766in}{0.896956in}}%
\pgfpathlineto{\pgfqpoint{2.282169in}{0.889435in}}%
\pgfpathlineto{\pgfqpoint{2.286293in}{0.889498in}}%
\pgfpathlineto{\pgfqpoint{2.297141in}{0.880461in}}%
\pgfpathclose%
\pgfusepath{fill}%
\end{pgfscope}%
\begin{pgfscope}%
\pgfpathrectangle{\pgfqpoint{0.100000in}{0.100000in}}{\pgfqpoint{3.608454in}{2.310000in}}%
\pgfusepath{clip}%
\pgfsetbuttcap%
\pgfsetmiterjoin%
\definecolor{currentfill}{rgb}{0.000000,0.521569,0.739216}%
\pgfsetfillcolor{currentfill}%
\pgfsetlinewidth{0.000000pt}%
\definecolor{currentstroke}{rgb}{0.000000,0.000000,0.000000}%
\pgfsetstrokecolor{currentstroke}%
\pgfsetstrokeopacity{0.000000}%
\pgfsetdash{}{0pt}%
\pgfpathmoveto{\pgfqpoint{1.795374in}{1.510393in}}%
\pgfpathlineto{\pgfqpoint{1.793857in}{1.483052in}}%
\pgfpathlineto{\pgfqpoint{1.760676in}{1.485050in}}%
\pgfpathlineto{\pgfqpoint{1.732544in}{1.486727in}}%
\pgfpathlineto{\pgfqpoint{1.734320in}{1.514317in}}%
\pgfpathlineto{\pgfqpoint{1.733109in}{1.514401in}}%
\pgfpathlineto{\pgfqpoint{1.735045in}{1.541820in}}%
\pgfpathlineto{\pgfqpoint{1.726904in}{1.542427in}}%
\pgfpathlineto{\pgfqpoint{1.728889in}{1.569871in}}%
\pgfpathlineto{\pgfqpoint{1.762096in}{1.567407in}}%
\pgfpathlineto{\pgfqpoint{1.796934in}{1.565224in}}%
\pgfpathlineto{\pgfqpoint{1.795374in}{1.510393in}}%
\pgfpathclose%
\pgfusepath{fill}%
\end{pgfscope}%
\begin{pgfscope}%
\pgfpathrectangle{\pgfqpoint{0.100000in}{0.100000in}}{\pgfqpoint{3.608454in}{2.310000in}}%
\pgfusepath{clip}%
\pgfsetbuttcap%
\pgfsetmiterjoin%
\definecolor{currentfill}{rgb}{0.000000,0.513725,0.743137}%
\pgfsetfillcolor{currentfill}%
\pgfsetlinewidth{0.000000pt}%
\definecolor{currentstroke}{rgb}{0.000000,0.000000,0.000000}%
\pgfsetstrokecolor{currentstroke}%
\pgfsetstrokeopacity{0.000000}%
\pgfsetdash{}{0pt}%
\pgfpathmoveto{\pgfqpoint{2.794880in}{1.439319in}}%
\pgfpathlineto{\pgfqpoint{2.795500in}{1.426123in}}%
\pgfpathlineto{\pgfqpoint{2.776762in}{1.425313in}}%
\pgfpathlineto{\pgfqpoint{2.757924in}{1.424641in}}%
\pgfpathlineto{\pgfqpoint{2.746401in}{1.421668in}}%
\pgfpathlineto{\pgfqpoint{2.726071in}{1.419220in}}%
\pgfpathlineto{\pgfqpoint{2.722450in}{1.453627in}}%
\pgfpathlineto{\pgfqpoint{2.720165in}{1.477649in}}%
\pgfpathlineto{\pgfqpoint{2.719775in}{1.481004in}}%
\pgfpathlineto{\pgfqpoint{2.741875in}{1.483652in}}%
\pgfpathlineto{\pgfqpoint{2.747392in}{1.486269in}}%
\pgfpathlineto{\pgfqpoint{2.746418in}{1.494296in}}%
\pgfpathlineto{\pgfqpoint{2.767029in}{1.496892in}}%
\pgfpathlineto{\pgfqpoint{2.766536in}{1.500999in}}%
\pgfpathlineto{\pgfqpoint{2.773279in}{1.502104in}}%
\pgfpathlineto{\pgfqpoint{2.795078in}{1.502065in}}%
\pgfpathlineto{\pgfqpoint{2.795940in}{1.480285in}}%
\pgfpathlineto{\pgfqpoint{2.799407in}{1.480427in}}%
\pgfpathlineto{\pgfqpoint{2.800713in}{1.463456in}}%
\pgfpathlineto{\pgfqpoint{2.798306in}{1.443885in}}%
\pgfpathlineto{\pgfqpoint{2.794880in}{1.439319in}}%
\pgfpathclose%
\pgfusepath{fill}%
\end{pgfscope}%
\begin{pgfscope}%
\pgfpathrectangle{\pgfqpoint{0.100000in}{0.100000in}}{\pgfqpoint{3.608454in}{2.310000in}}%
\pgfusepath{clip}%
\pgfsetbuttcap%
\pgfsetmiterjoin%
\definecolor{currentfill}{rgb}{0.000000,0.584314,0.707843}%
\pgfsetfillcolor{currentfill}%
\pgfsetlinewidth{0.000000pt}%
\definecolor{currentstroke}{rgb}{0.000000,0.000000,0.000000}%
\pgfsetstrokecolor{currentstroke}%
\pgfsetstrokeopacity{0.000000}%
\pgfsetdash{}{0pt}%
\pgfpathmoveto{\pgfqpoint{2.290718in}{1.442111in}}%
\pgfpathlineto{\pgfqpoint{2.291417in}{1.417786in}}%
\pgfpathlineto{\pgfqpoint{2.267659in}{1.417346in}}%
\pgfpathlineto{\pgfqpoint{2.267438in}{1.423982in}}%
\pgfpathlineto{\pgfqpoint{2.237055in}{1.423181in}}%
\pgfpathlineto{\pgfqpoint{2.236527in}{1.423164in}}%
\pgfpathlineto{\pgfqpoint{2.235992in}{1.447300in}}%
\pgfpathlineto{\pgfqpoint{2.246319in}{1.447619in}}%
\pgfpathlineto{\pgfqpoint{2.246988in}{1.453136in}}%
\pgfpathlineto{\pgfqpoint{2.244011in}{1.467014in}}%
\pgfpathlineto{\pgfqpoint{2.248598in}{1.467244in}}%
\pgfpathlineto{\pgfqpoint{2.290253in}{1.469670in}}%
\pgfpathlineto{\pgfqpoint{2.290718in}{1.442111in}}%
\pgfpathclose%
\pgfusepath{fill}%
\end{pgfscope}%
\begin{pgfscope}%
\pgfpathrectangle{\pgfqpoint{0.100000in}{0.100000in}}{\pgfqpoint{3.608454in}{2.310000in}}%
\pgfusepath{clip}%
\pgfsetbuttcap%
\pgfsetmiterjoin%
\definecolor{currentfill}{rgb}{0.000000,0.709804,0.645098}%
\pgfsetfillcolor{currentfill}%
\pgfsetlinewidth{0.000000pt}%
\definecolor{currentstroke}{rgb}{0.000000,0.000000,0.000000}%
\pgfsetstrokecolor{currentstroke}%
\pgfsetstrokeopacity{0.000000}%
\pgfsetdash{}{0pt}%
\pgfpathmoveto{\pgfqpoint{2.190786in}{1.003624in}}%
\pgfpathlineto{\pgfqpoint{2.180404in}{1.000901in}}%
\pgfpathlineto{\pgfqpoint{2.176519in}{0.997121in}}%
\pgfpathlineto{\pgfqpoint{2.170781in}{0.999186in}}%
\pgfpathlineto{\pgfqpoint{2.151260in}{0.999398in}}%
\pgfpathlineto{\pgfqpoint{2.142324in}{1.001839in}}%
\pgfpathlineto{\pgfqpoint{2.141998in}{0.984277in}}%
\pgfpathlineto{\pgfqpoint{2.110869in}{0.984129in}}%
\pgfpathlineto{\pgfqpoint{2.110850in}{0.997922in}}%
\pgfpathlineto{\pgfqpoint{2.073255in}{0.998037in}}%
\pgfpathlineto{\pgfqpoint{2.073230in}{0.991141in}}%
\pgfpathlineto{\pgfqpoint{2.062941in}{0.991185in}}%
\pgfpathlineto{\pgfqpoint{2.049224in}{0.991276in}}%
\pgfpathlineto{\pgfqpoint{2.049296in}{0.998169in}}%
\pgfpathlineto{\pgfqpoint{2.035603in}{0.998326in}}%
\pgfpathlineto{\pgfqpoint{2.036171in}{1.027587in}}%
\pgfpathlineto{\pgfqpoint{2.043077in}{1.035565in}}%
\pgfpathlineto{\pgfqpoint{2.050477in}{1.038266in}}%
\pgfpathlineto{\pgfqpoint{2.058740in}{1.038661in}}%
\pgfpathlineto{\pgfqpoint{2.069896in}{1.042778in}}%
\pgfpathlineto{\pgfqpoint{2.081865in}{1.049099in}}%
\pgfpathlineto{\pgfqpoint{2.091976in}{1.044393in}}%
\pgfpathlineto{\pgfqpoint{2.096410in}{1.052482in}}%
\pgfpathlineto{\pgfqpoint{2.101602in}{1.057429in}}%
\pgfpathlineto{\pgfqpoint{2.098258in}{1.064900in}}%
\pgfpathlineto{\pgfqpoint{2.098590in}{1.073824in}}%
\pgfpathlineto{\pgfqpoint{2.140727in}{1.073891in}}%
\pgfpathlineto{\pgfqpoint{2.139356in}{1.083433in}}%
\pgfpathlineto{\pgfqpoint{2.163020in}{1.082921in}}%
\pgfpathlineto{\pgfqpoint{2.176762in}{1.083808in}}%
\pgfpathlineto{\pgfqpoint{2.172137in}{1.075918in}}%
\pgfpathlineto{\pgfqpoint{2.166395in}{1.075969in}}%
\pgfpathlineto{\pgfqpoint{2.166287in}{1.069101in}}%
\pgfpathlineto{\pgfqpoint{2.169375in}{1.063715in}}%
\pgfpathlineto{\pgfqpoint{2.166477in}{1.058801in}}%
\pgfpathlineto{\pgfqpoint{2.166153in}{1.040540in}}%
\pgfpathlineto{\pgfqpoint{2.162548in}{1.034836in}}%
\pgfpathlineto{\pgfqpoint{2.162454in}{1.027888in}}%
\pgfpathlineto{\pgfqpoint{2.166850in}{1.025625in}}%
\pgfpathlineto{\pgfqpoint{2.190828in}{1.025374in}}%
\pgfpathlineto{\pgfqpoint{2.190786in}{1.003624in}}%
\pgfpathclose%
\pgfusepath{fill}%
\end{pgfscope}%
\begin{pgfscope}%
\pgfpathrectangle{\pgfqpoint{0.100000in}{0.100000in}}{\pgfqpoint{3.608454in}{2.310000in}}%
\pgfusepath{clip}%
\pgfsetbuttcap%
\pgfsetmiterjoin%
\definecolor{currentfill}{rgb}{0.000000,0.623529,0.688235}%
\pgfsetfillcolor{currentfill}%
\pgfsetlinewidth{0.000000pt}%
\definecolor{currentstroke}{rgb}{0.000000,0.000000,0.000000}%
\pgfsetstrokecolor{currentstroke}%
\pgfsetstrokeopacity{0.000000}%
\pgfsetdash{}{0pt}%
\pgfpathmoveto{\pgfqpoint{3.573009in}{1.957078in}}%
\pgfpathlineto{\pgfqpoint{3.580481in}{1.956181in}}%
\pgfpathlineto{\pgfqpoint{3.577419in}{1.950172in}}%
\pgfpathlineto{\pgfqpoint{3.573009in}{1.957078in}}%
\pgfpathclose%
\pgfusepath{fill}%
\end{pgfscope}%
\begin{pgfscope}%
\pgfpathrectangle{\pgfqpoint{0.100000in}{0.100000in}}{\pgfqpoint{3.608454in}{2.310000in}}%
\pgfusepath{clip}%
\pgfsetbuttcap%
\pgfsetmiterjoin%
\definecolor{currentfill}{rgb}{0.000000,0.623529,0.688235}%
\pgfsetfillcolor{currentfill}%
\pgfsetlinewidth{0.000000pt}%
\definecolor{currentstroke}{rgb}{0.000000,0.000000,0.000000}%
\pgfsetstrokecolor{currentstroke}%
\pgfsetstrokeopacity{0.000000}%
\pgfsetdash{}{0pt}%
\pgfpathmoveto{\pgfqpoint{3.550714in}{1.937007in}}%
\pgfpathlineto{\pgfqpoint{3.553924in}{1.946267in}}%
\pgfpathlineto{\pgfqpoint{3.547311in}{1.956692in}}%
\pgfpathlineto{\pgfqpoint{3.543555in}{1.955713in}}%
\pgfpathlineto{\pgfqpoint{3.538805in}{1.962876in}}%
\pgfpathlineto{\pgfqpoint{3.535140in}{1.964201in}}%
\pgfpathlineto{\pgfqpoint{3.535820in}{1.974394in}}%
\pgfpathlineto{\pgfqpoint{3.534775in}{1.988683in}}%
\pgfpathlineto{\pgfqpoint{3.532167in}{1.996604in}}%
\pgfpathlineto{\pgfqpoint{3.539818in}{1.996897in}}%
\pgfpathlineto{\pgfqpoint{3.527215in}{2.022326in}}%
\pgfpathlineto{\pgfqpoint{3.514278in}{2.013412in}}%
\pgfpathlineto{\pgfqpoint{3.494071in}{2.051630in}}%
\pgfpathlineto{\pgfqpoint{3.496496in}{2.059377in}}%
\pgfpathlineto{\pgfqpoint{3.490714in}{2.062918in}}%
\pgfpathlineto{\pgfqpoint{3.491275in}{2.076574in}}%
\pgfpathlineto{\pgfqpoint{3.487328in}{2.085547in}}%
\pgfpathlineto{\pgfqpoint{3.478067in}{2.117502in}}%
\pgfpathlineto{\pgfqpoint{3.474712in}{2.131510in}}%
\pgfpathlineto{\pgfqpoint{3.522068in}{2.145170in}}%
\pgfpathlineto{\pgfqpoint{3.526210in}{2.131697in}}%
\pgfpathlineto{\pgfqpoint{3.546699in}{2.136666in}}%
\pgfpathlineto{\pgfqpoint{3.557203in}{2.103454in}}%
\pgfpathlineto{\pgfqpoint{3.565274in}{2.075679in}}%
\pgfpathlineto{\pgfqpoint{3.566650in}{2.075414in}}%
\pgfpathlineto{\pgfqpoint{3.584395in}{2.086358in}}%
\pgfpathlineto{\pgfqpoint{3.598893in}{2.059799in}}%
\pgfpathlineto{\pgfqpoint{3.592928in}{2.056963in}}%
\pgfpathlineto{\pgfqpoint{3.603808in}{2.035391in}}%
\pgfpathlineto{\pgfqpoint{3.597567in}{2.032110in}}%
\pgfpathlineto{\pgfqpoint{3.610482in}{2.006964in}}%
\pgfpathlineto{\pgfqpoint{3.618035in}{1.994535in}}%
\pgfpathlineto{\pgfqpoint{3.613315in}{1.987169in}}%
\pgfpathlineto{\pgfqpoint{3.607145in}{1.996865in}}%
\pgfpathlineto{\pgfqpoint{3.599803in}{1.992874in}}%
\pgfpathlineto{\pgfqpoint{3.608045in}{1.984589in}}%
\pgfpathlineto{\pgfqpoint{3.601528in}{1.973738in}}%
\pgfpathlineto{\pgfqpoint{3.594431in}{1.979179in}}%
\pgfpathlineto{\pgfqpoint{3.594740in}{1.984518in}}%
\pgfpathlineto{\pgfqpoint{3.589736in}{1.989331in}}%
\pgfpathlineto{\pgfqpoint{3.587038in}{1.983926in}}%
\pgfpathlineto{\pgfqpoint{3.586028in}{1.973437in}}%
\pgfpathlineto{\pgfqpoint{3.576059in}{1.975363in}}%
\pgfpathlineto{\pgfqpoint{3.564314in}{1.981978in}}%
\pgfpathlineto{\pgfqpoint{3.561572in}{1.978778in}}%
\pgfpathlineto{\pgfqpoint{3.565543in}{1.970193in}}%
\pgfpathlineto{\pgfqpoint{3.562497in}{1.956796in}}%
\pgfpathlineto{\pgfqpoint{3.565408in}{1.948473in}}%
\pgfpathlineto{\pgfqpoint{3.560888in}{1.939136in}}%
\pgfpathlineto{\pgfqpoint{3.550714in}{1.937007in}}%
\pgfpathclose%
\pgfusepath{fill}%
\end{pgfscope}%
\begin{pgfscope}%
\pgfpathrectangle{\pgfqpoint{0.100000in}{0.100000in}}{\pgfqpoint{3.608454in}{2.310000in}}%
\pgfusepath{clip}%
\pgfsetbuttcap%
\pgfsetmiterjoin%
\definecolor{currentfill}{rgb}{0.000000,0.564706,0.717647}%
\pgfsetfillcolor{currentfill}%
\pgfsetlinewidth{0.000000pt}%
\definecolor{currentstroke}{rgb}{0.000000,0.000000,0.000000}%
\pgfsetstrokecolor{currentstroke}%
\pgfsetstrokeopacity{0.000000}%
\pgfsetdash{}{0pt}%
\pgfpathmoveto{\pgfqpoint{1.859593in}{0.670192in}}%
\pgfpathlineto{\pgfqpoint{1.833962in}{0.671147in}}%
\pgfpathlineto{\pgfqpoint{1.835408in}{0.704889in}}%
\pgfpathlineto{\pgfqpoint{1.870905in}{0.703446in}}%
\pgfpathlineto{\pgfqpoint{1.871649in}{0.712052in}}%
\pgfpathlineto{\pgfqpoint{1.907744in}{0.711220in}}%
\pgfpathlineto{\pgfqpoint{1.913119in}{0.701138in}}%
\pgfpathlineto{\pgfqpoint{1.903598in}{0.691454in}}%
\pgfpathlineto{\pgfqpoint{1.892292in}{0.669010in}}%
\pgfpathlineto{\pgfqpoint{1.891457in}{0.663121in}}%
\pgfpathlineto{\pgfqpoint{1.871994in}{0.669892in}}%
\pgfpathlineto{\pgfqpoint{1.859593in}{0.670192in}}%
\pgfpathclose%
\pgfusepath{fill}%
\end{pgfscope}%
\begin{pgfscope}%
\pgfpathrectangle{\pgfqpoint{0.100000in}{0.100000in}}{\pgfqpoint{3.608454in}{2.310000in}}%
\pgfusepath{clip}%
\pgfsetbuttcap%
\pgfsetmiterjoin%
\definecolor{currentfill}{rgb}{0.000000,0.803922,0.598039}%
\pgfsetfillcolor{currentfill}%
\pgfsetlinewidth{0.000000pt}%
\definecolor{currentstroke}{rgb}{0.000000,0.000000,0.000000}%
\pgfsetstrokecolor{currentstroke}%
\pgfsetstrokeopacity{0.000000}%
\pgfsetdash{}{0pt}%
\pgfpathmoveto{\pgfqpoint{2.362240in}{0.870230in}}%
\pgfpathlineto{\pgfqpoint{2.367614in}{0.868244in}}%
\pgfpathlineto{\pgfqpoint{2.368994in}{0.861333in}}%
\pgfpathlineto{\pgfqpoint{2.363366in}{0.855813in}}%
\pgfpathlineto{\pgfqpoint{2.363126in}{0.851002in}}%
\pgfpathlineto{\pgfqpoint{2.370206in}{0.848561in}}%
\pgfpathlineto{\pgfqpoint{2.364353in}{0.841406in}}%
\pgfpathlineto{\pgfqpoint{2.366196in}{0.837164in}}%
\pgfpathlineto{\pgfqpoint{2.371879in}{0.836595in}}%
\pgfpathlineto{\pgfqpoint{2.365973in}{0.833290in}}%
\pgfpathlineto{\pgfqpoint{2.344269in}{0.832393in}}%
\pgfpathlineto{\pgfqpoint{2.344959in}{0.835935in}}%
\pgfpathlineto{\pgfqpoint{2.334864in}{0.835567in}}%
\pgfpathlineto{\pgfqpoint{2.331640in}{0.845019in}}%
\pgfpathlineto{\pgfqpoint{2.337645in}{0.859095in}}%
\pgfpathlineto{\pgfqpoint{2.340706in}{0.860346in}}%
\pgfpathlineto{\pgfqpoint{2.344274in}{0.869656in}}%
\pgfpathlineto{\pgfqpoint{2.362240in}{0.870230in}}%
\pgfpathclose%
\pgfusepath{fill}%
\end{pgfscope}%
\begin{pgfscope}%
\pgfpathrectangle{\pgfqpoint{0.100000in}{0.100000in}}{\pgfqpoint{3.608454in}{2.310000in}}%
\pgfusepath{clip}%
\pgfsetbuttcap%
\pgfsetmiterjoin%
\definecolor{currentfill}{rgb}{0.000000,0.749020,0.625490}%
\pgfsetfillcolor{currentfill}%
\pgfsetlinewidth{0.000000pt}%
\definecolor{currentstroke}{rgb}{0.000000,0.000000,0.000000}%
\pgfsetstrokecolor{currentstroke}%
\pgfsetstrokeopacity{0.000000}%
\pgfsetdash{}{0pt}%
\pgfpathmoveto{\pgfqpoint{2.093068in}{0.691255in}}%
\pgfpathlineto{\pgfqpoint{2.089584in}{0.697805in}}%
\pgfpathlineto{\pgfqpoint{2.084351in}{0.694073in}}%
\pgfpathlineto{\pgfqpoint{2.081886in}{0.666080in}}%
\pgfpathlineto{\pgfqpoint{2.065563in}{0.666170in}}%
\pgfpathlineto{\pgfqpoint{2.049731in}{0.675985in}}%
\pgfpathlineto{\pgfqpoint{2.047653in}{0.694679in}}%
\pgfpathlineto{\pgfqpoint{2.026699in}{0.691630in}}%
\pgfpathlineto{\pgfqpoint{2.021875in}{0.703710in}}%
\pgfpathlineto{\pgfqpoint{2.040106in}{0.712927in}}%
\pgfpathlineto{\pgfqpoint{2.054567in}{0.712981in}}%
\pgfpathlineto{\pgfqpoint{2.053852in}{0.716384in}}%
\pgfpathlineto{\pgfqpoint{2.057335in}{0.727032in}}%
\pgfpathlineto{\pgfqpoint{2.061056in}{0.729018in}}%
\pgfpathlineto{\pgfqpoint{2.056507in}{0.745639in}}%
\pgfpathlineto{\pgfqpoint{2.062519in}{0.748614in}}%
\pgfpathlineto{\pgfqpoint{2.088232in}{0.752553in}}%
\pgfpathlineto{\pgfqpoint{2.096621in}{0.750671in}}%
\pgfpathlineto{\pgfqpoint{2.099586in}{0.743769in}}%
\pgfpathlineto{\pgfqpoint{2.106573in}{0.739232in}}%
\pgfpathlineto{\pgfqpoint{2.115910in}{0.747279in}}%
\pgfpathlineto{\pgfqpoint{2.125247in}{0.741762in}}%
\pgfpathlineto{\pgfqpoint{2.140576in}{0.738107in}}%
\pgfpathlineto{\pgfqpoint{2.145077in}{0.730301in}}%
\pgfpathlineto{\pgfqpoint{2.152123in}{0.723725in}}%
\pgfpathlineto{\pgfqpoint{2.166253in}{0.713450in}}%
\pgfpathlineto{\pgfqpoint{2.143868in}{0.708025in}}%
\pgfpathlineto{\pgfqpoint{2.132588in}{0.712274in}}%
\pgfpathlineto{\pgfqpoint{2.123775in}{0.713712in}}%
\pgfpathlineto{\pgfqpoint{2.117525in}{0.716961in}}%
\pgfpathlineto{\pgfqpoint{2.110954in}{0.708601in}}%
\pgfpathlineto{\pgfqpoint{2.093068in}{0.691255in}}%
\pgfpathclose%
\pgfusepath{fill}%
\end{pgfscope}%
\begin{pgfscope}%
\pgfpathrectangle{\pgfqpoint{0.100000in}{0.100000in}}{\pgfqpoint{3.608454in}{2.310000in}}%
\pgfusepath{clip}%
\pgfsetbuttcap%
\pgfsetmiterjoin%
\definecolor{currentfill}{rgb}{0.000000,0.654902,0.672549}%
\pgfsetfillcolor{currentfill}%
\pgfsetlinewidth{0.000000pt}%
\definecolor{currentstroke}{rgb}{0.000000,0.000000,0.000000}%
\pgfsetstrokecolor{currentstroke}%
\pgfsetstrokeopacity{0.000000}%
\pgfsetdash{}{0pt}%
\pgfpathmoveto{\pgfqpoint{2.904596in}{1.155570in}}%
\pgfpathlineto{\pgfqpoint{2.898012in}{1.153842in}}%
\pgfpathlineto{\pgfqpoint{2.895600in}{1.161365in}}%
\pgfpathlineto{\pgfqpoint{2.879988in}{1.149617in}}%
\pgfpathlineto{\pgfqpoint{2.868331in}{1.159628in}}%
\pgfpathlineto{\pgfqpoint{2.860386in}{1.164003in}}%
\pgfpathlineto{\pgfqpoint{2.865237in}{1.171385in}}%
\pgfpathlineto{\pgfqpoint{2.859396in}{1.177849in}}%
\pgfpathlineto{\pgfqpoint{2.853813in}{1.173625in}}%
\pgfpathlineto{\pgfqpoint{2.851139in}{1.181704in}}%
\pgfpathlineto{\pgfqpoint{2.856311in}{1.184739in}}%
\pgfpathlineto{\pgfqpoint{2.859737in}{1.192156in}}%
\pgfpathlineto{\pgfqpoint{2.867721in}{1.199764in}}%
\pgfpathlineto{\pgfqpoint{2.869228in}{1.206518in}}%
\pgfpathlineto{\pgfqpoint{2.876442in}{1.212071in}}%
\pgfpathlineto{\pgfqpoint{2.880082in}{1.220503in}}%
\pgfpathlineto{\pgfqpoint{2.886736in}{1.225061in}}%
\pgfpathlineto{\pgfqpoint{2.901040in}{1.226866in}}%
\pgfpathlineto{\pgfqpoint{2.907547in}{1.214437in}}%
\pgfpathlineto{\pgfqpoint{2.920870in}{1.223465in}}%
\pgfpathlineto{\pgfqpoint{2.921871in}{1.228104in}}%
\pgfpathlineto{\pgfqpoint{2.933591in}{1.232567in}}%
\pgfpathlineto{\pgfqpoint{2.937030in}{1.237689in}}%
\pgfpathlineto{\pgfqpoint{2.943559in}{1.240996in}}%
\pgfpathlineto{\pgfqpoint{2.960170in}{1.246122in}}%
\pgfpathlineto{\pgfqpoint{2.968379in}{1.238517in}}%
\pgfpathlineto{\pgfqpoint{2.974075in}{1.228469in}}%
\pgfpathlineto{\pgfqpoint{2.958209in}{1.221651in}}%
\pgfpathlineto{\pgfqpoint{2.957411in}{1.217461in}}%
\pgfpathlineto{\pgfqpoint{2.951682in}{1.213011in}}%
\pgfpathlineto{\pgfqpoint{2.940357in}{1.211530in}}%
\pgfpathlineto{\pgfqpoint{2.934516in}{1.206979in}}%
\pgfpathlineto{\pgfqpoint{2.933739in}{1.199536in}}%
\pgfpathlineto{\pgfqpoint{2.929794in}{1.191773in}}%
\pgfpathlineto{\pgfqpoint{2.936507in}{1.182790in}}%
\pgfpathlineto{\pgfqpoint{2.933147in}{1.171253in}}%
\pgfpathlineto{\pgfqpoint{2.930144in}{1.169183in}}%
\pgfpathlineto{\pgfqpoint{2.920582in}{1.172131in}}%
\pgfpathlineto{\pgfqpoint{2.909797in}{1.165077in}}%
\pgfpathlineto{\pgfqpoint{2.904596in}{1.155570in}}%
\pgfpathclose%
\pgfusepath{fill}%
\end{pgfscope}%
\begin{pgfscope}%
\pgfpathrectangle{\pgfqpoint{0.100000in}{0.100000in}}{\pgfqpoint{3.608454in}{2.310000in}}%
\pgfusepath{clip}%
\pgfsetbuttcap%
\pgfsetmiterjoin%
\definecolor{currentfill}{rgb}{0.000000,0.701961,0.649020}%
\pgfsetfillcolor{currentfill}%
\pgfsetlinewidth{0.000000pt}%
\definecolor{currentstroke}{rgb}{0.000000,0.000000,0.000000}%
\pgfsetstrokecolor{currentstroke}%
\pgfsetstrokeopacity{0.000000}%
\pgfsetdash{}{0pt}%
\pgfpathmoveto{\pgfqpoint{2.537283in}{1.876850in}}%
\pgfpathlineto{\pgfqpoint{2.528827in}{1.854971in}}%
\pgfpathlineto{\pgfqpoint{2.521970in}{1.844929in}}%
\pgfpathlineto{\pgfqpoint{2.520382in}{1.829615in}}%
\pgfpathlineto{\pgfqpoint{2.513293in}{1.828191in}}%
\pgfpathlineto{\pgfqpoint{2.492843in}{1.831235in}}%
\pgfpathlineto{\pgfqpoint{2.492447in}{1.838114in}}%
\pgfpathlineto{\pgfqpoint{2.488144in}{1.844796in}}%
\pgfpathlineto{\pgfqpoint{2.481279in}{1.844556in}}%
\pgfpathlineto{\pgfqpoint{2.480448in}{1.858288in}}%
\pgfpathlineto{\pgfqpoint{2.473620in}{1.857983in}}%
\pgfpathlineto{\pgfqpoint{2.471787in}{1.885405in}}%
\pgfpathlineto{\pgfqpoint{2.457894in}{1.884450in}}%
\pgfpathlineto{\pgfqpoint{2.456551in}{1.904950in}}%
\pgfpathlineto{\pgfqpoint{2.438472in}{1.913015in}}%
\pgfpathlineto{\pgfqpoint{2.436761in}{1.938649in}}%
\pgfpathlineto{\pgfqpoint{2.484672in}{1.942048in}}%
\pgfpathlineto{\pgfqpoint{2.485703in}{1.928298in}}%
\pgfpathlineto{\pgfqpoint{2.513243in}{1.930478in}}%
\pgfpathlineto{\pgfqpoint{2.514867in}{1.909757in}}%
\pgfpathlineto{\pgfqpoint{2.528620in}{1.910846in}}%
\pgfpathlineto{\pgfqpoint{2.531516in}{1.904166in}}%
\pgfpathlineto{\pgfqpoint{2.533757in}{1.876549in}}%
\pgfpathlineto{\pgfqpoint{2.537283in}{1.876850in}}%
\pgfpathclose%
\pgfusepath{fill}%
\end{pgfscope}%
\begin{pgfscope}%
\pgfpathrectangle{\pgfqpoint{0.100000in}{0.100000in}}{\pgfqpoint{3.608454in}{2.310000in}}%
\pgfusepath{clip}%
\pgfsetbuttcap%
\pgfsetmiterjoin%
\definecolor{currentfill}{rgb}{0.000000,0.545098,0.727451}%
\pgfsetfillcolor{currentfill}%
\pgfsetlinewidth{0.000000pt}%
\definecolor{currentstroke}{rgb}{0.000000,0.000000,0.000000}%
\pgfsetstrokecolor{currentstroke}%
\pgfsetstrokeopacity{0.000000}%
\pgfsetdash{}{0pt}%
\pgfpathmoveto{\pgfqpoint{2.507414in}{1.036455in}}%
\pgfpathlineto{\pgfqpoint{2.483127in}{1.034794in}}%
\pgfpathlineto{\pgfqpoint{2.482016in}{1.066685in}}%
\pgfpathlineto{\pgfqpoint{2.488608in}{1.069834in}}%
\pgfpathlineto{\pgfqpoint{2.487860in}{1.090523in}}%
\pgfpathlineto{\pgfqpoint{2.469563in}{1.097602in}}%
\pgfpathlineto{\pgfqpoint{2.480144in}{1.116463in}}%
\pgfpathlineto{\pgfqpoint{2.479806in}{1.130609in}}%
\pgfpathlineto{\pgfqpoint{2.492139in}{1.131174in}}%
\pgfpathlineto{\pgfqpoint{2.498683in}{1.126546in}}%
\pgfpathlineto{\pgfqpoint{2.510006in}{1.121373in}}%
\pgfpathlineto{\pgfqpoint{2.510594in}{1.099857in}}%
\pgfpathlineto{\pgfqpoint{2.517006in}{1.100169in}}%
\pgfpathlineto{\pgfqpoint{2.517692in}{1.084251in}}%
\pgfpathlineto{\pgfqpoint{2.525076in}{1.078728in}}%
\pgfpathlineto{\pgfqpoint{2.531082in}{1.076803in}}%
\pgfpathlineto{\pgfqpoint{2.534914in}{1.072050in}}%
\pgfpathlineto{\pgfqpoint{2.535012in}{1.069064in}}%
\pgfpathlineto{\pgfqpoint{2.521534in}{1.068158in}}%
\pgfpathlineto{\pgfqpoint{2.517201in}{1.066138in}}%
\pgfpathlineto{\pgfqpoint{2.512736in}{1.056859in}}%
\pgfpathlineto{\pgfqpoint{2.508734in}{1.056586in}}%
\pgfpathlineto{\pgfqpoint{2.509787in}{1.036620in}}%
\pgfpathlineto{\pgfqpoint{2.507414in}{1.036455in}}%
\pgfpathclose%
\pgfusepath{fill}%
\end{pgfscope}%
\begin{pgfscope}%
\pgfpathrectangle{\pgfqpoint{0.100000in}{0.100000in}}{\pgfqpoint{3.608454in}{2.310000in}}%
\pgfusepath{clip}%
\pgfsetbuttcap%
\pgfsetmiterjoin%
\definecolor{currentfill}{rgb}{0.000000,0.490196,0.754902}%
\pgfsetfillcolor{currentfill}%
\pgfsetlinewidth{0.000000pt}%
\definecolor{currentstroke}{rgb}{0.000000,0.000000,0.000000}%
\pgfsetstrokecolor{currentstroke}%
\pgfsetstrokeopacity{0.000000}%
\pgfsetdash{}{0pt}%
\pgfpathmoveto{\pgfqpoint{1.960306in}{1.310682in}}%
\pgfpathlineto{\pgfqpoint{1.926190in}{1.311718in}}%
\pgfpathlineto{\pgfqpoint{1.926584in}{1.332368in}}%
\pgfpathlineto{\pgfqpoint{1.892401in}{1.333545in}}%
\pgfpathlineto{\pgfqpoint{1.892889in}{1.354245in}}%
\pgfpathlineto{\pgfqpoint{1.893147in}{1.361110in}}%
\pgfpathlineto{\pgfqpoint{1.927400in}{1.359864in}}%
\pgfpathlineto{\pgfqpoint{1.927666in}{1.366756in}}%
\pgfpathlineto{\pgfqpoint{1.961886in}{1.365738in}}%
\pgfpathlineto{\pgfqpoint{1.960306in}{1.310682in}}%
\pgfpathclose%
\pgfusepath{fill}%
\end{pgfscope}%
\begin{pgfscope}%
\pgfpathrectangle{\pgfqpoint{0.100000in}{0.100000in}}{\pgfqpoint{3.608454in}{2.310000in}}%
\pgfusepath{clip}%
\pgfsetbuttcap%
\pgfsetmiterjoin%
\definecolor{currentfill}{rgb}{0.000000,0.494118,0.752941}%
\pgfsetfillcolor{currentfill}%
\pgfsetlinewidth{0.000000pt}%
\definecolor{currentstroke}{rgb}{0.000000,0.000000,0.000000}%
\pgfsetstrokecolor{currentstroke}%
\pgfsetstrokeopacity{0.000000}%
\pgfsetdash{}{0pt}%
\pgfpathmoveto{\pgfqpoint{2.548186in}{1.158629in}}%
\pgfpathlineto{\pgfqpoint{2.549236in}{1.162164in}}%
\pgfpathlineto{\pgfqpoint{2.543041in}{1.178054in}}%
\pgfpathlineto{\pgfqpoint{2.539380in}{1.187318in}}%
\pgfpathlineto{\pgfqpoint{2.555635in}{1.195943in}}%
\pgfpathlineto{\pgfqpoint{2.563459in}{1.197714in}}%
\pgfpathlineto{\pgfqpoint{2.566571in}{1.201683in}}%
\pgfpathlineto{\pgfqpoint{2.567654in}{1.211922in}}%
\pgfpathlineto{\pgfqpoint{2.577947in}{1.209252in}}%
\pgfpathlineto{\pgfqpoint{2.589411in}{1.214310in}}%
\pgfpathlineto{\pgfqpoint{2.595443in}{1.205574in}}%
\pgfpathlineto{\pgfqpoint{2.607679in}{1.208265in}}%
\pgfpathlineto{\pgfqpoint{2.610086in}{1.175304in}}%
\pgfpathlineto{\pgfqpoint{2.606663in}{1.174945in}}%
\pgfpathlineto{\pgfqpoint{2.607032in}{1.160645in}}%
\pgfpathlineto{\pgfqpoint{2.599908in}{1.153357in}}%
\pgfpathlineto{\pgfqpoint{2.597965in}{1.148745in}}%
\pgfpathlineto{\pgfqpoint{2.583459in}{1.148544in}}%
\pgfpathlineto{\pgfqpoint{2.583376in}{1.141723in}}%
\pgfpathlineto{\pgfqpoint{2.579740in}{1.137359in}}%
\pgfpathlineto{\pgfqpoint{2.570262in}{1.139592in}}%
\pgfpathlineto{\pgfqpoint{2.556313in}{1.139100in}}%
\pgfpathlineto{\pgfqpoint{2.555391in}{1.144733in}}%
\pgfpathlineto{\pgfqpoint{2.548297in}{1.153932in}}%
\pgfpathlineto{\pgfqpoint{2.548186in}{1.158629in}}%
\pgfpathclose%
\pgfusepath{fill}%
\end{pgfscope}%
\begin{pgfscope}%
\pgfpathrectangle{\pgfqpoint{0.100000in}{0.100000in}}{\pgfqpoint{3.608454in}{2.310000in}}%
\pgfusepath{clip}%
\pgfsetbuttcap%
\pgfsetmiterjoin%
\definecolor{currentfill}{rgb}{0.000000,0.317647,0.841176}%
\pgfsetfillcolor{currentfill}%
\pgfsetlinewidth{0.000000pt}%
\definecolor{currentstroke}{rgb}{0.000000,0.000000,0.000000}%
\pgfsetstrokecolor{currentstroke}%
\pgfsetstrokeopacity{0.000000}%
\pgfsetdash{}{0pt}%
\pgfpathmoveto{\pgfqpoint{3.245459in}{1.515286in}}%
\pgfpathlineto{\pgfqpoint{3.266966in}{1.519876in}}%
\pgfpathlineto{\pgfqpoint{3.271013in}{1.528198in}}%
\pgfpathlineto{\pgfqpoint{3.275612in}{1.531070in}}%
\pgfpathlineto{\pgfqpoint{3.284318in}{1.531602in}}%
\pgfpathlineto{\pgfqpoint{3.284745in}{1.526913in}}%
\pgfpathlineto{\pgfqpoint{3.281349in}{1.515956in}}%
\pgfpathlineto{\pgfqpoint{3.285117in}{1.511332in}}%
\pgfpathlineto{\pgfqpoint{3.285323in}{1.505755in}}%
\pgfpathlineto{\pgfqpoint{3.282148in}{1.502290in}}%
\pgfpathlineto{\pgfqpoint{3.288460in}{1.495799in}}%
\pgfpathlineto{\pgfqpoint{3.281434in}{1.488221in}}%
\pgfpathlineto{\pgfqpoint{3.274847in}{1.487194in}}%
\pgfpathlineto{\pgfqpoint{3.273108in}{1.493337in}}%
\pgfpathlineto{\pgfqpoint{3.266239in}{1.490550in}}%
\pgfpathlineto{\pgfqpoint{3.256327in}{1.489994in}}%
\pgfpathlineto{\pgfqpoint{3.258440in}{1.494561in}}%
\pgfpathlineto{\pgfqpoint{3.257846in}{1.502321in}}%
\pgfpathlineto{\pgfqpoint{3.251965in}{1.502268in}}%
\pgfpathlineto{\pgfqpoint{3.239646in}{1.514041in}}%
\pgfpathlineto{\pgfqpoint{3.245459in}{1.515286in}}%
\pgfpathclose%
\pgfusepath{fill}%
\end{pgfscope}%
\begin{pgfscope}%
\pgfpathrectangle{\pgfqpoint{0.100000in}{0.100000in}}{\pgfqpoint{3.608454in}{2.310000in}}%
\pgfusepath{clip}%
\pgfsetbuttcap%
\pgfsetmiterjoin%
\definecolor{currentfill}{rgb}{0.000000,0.364706,0.817647}%
\pgfsetfillcolor{currentfill}%
\pgfsetlinewidth{0.000000pt}%
\definecolor{currentstroke}{rgb}{0.000000,0.000000,0.000000}%
\pgfsetstrokecolor{currentstroke}%
\pgfsetstrokeopacity{0.000000}%
\pgfsetdash{}{0pt}%
\pgfpathmoveto{\pgfqpoint{1.791350in}{1.359031in}}%
\pgfpathlineto{\pgfqpoint{1.790342in}{1.359067in}}%
\pgfpathlineto{\pgfqpoint{1.792462in}{1.393414in}}%
\pgfpathlineto{\pgfqpoint{1.824934in}{1.391676in}}%
\pgfpathlineto{\pgfqpoint{1.860470in}{1.390012in}}%
\pgfpathlineto{\pgfqpoint{1.858721in}{1.355685in}}%
\pgfpathlineto{\pgfqpoint{1.791350in}{1.359031in}}%
\pgfpathclose%
\pgfusepath{fill}%
\end{pgfscope}%
\begin{pgfscope}%
\pgfpathrectangle{\pgfqpoint{0.100000in}{0.100000in}}{\pgfqpoint{3.608454in}{2.310000in}}%
\pgfusepath{clip}%
\pgfsetbuttcap%
\pgfsetmiterjoin%
\definecolor{currentfill}{rgb}{0.000000,0.478431,0.760784}%
\pgfsetfillcolor{currentfill}%
\pgfsetlinewidth{0.000000pt}%
\definecolor{currentstroke}{rgb}{0.000000,0.000000,0.000000}%
\pgfsetstrokecolor{currentstroke}%
\pgfsetstrokeopacity{0.000000}%
\pgfsetdash{}{0pt}%
\pgfpathmoveto{\pgfqpoint{1.895416in}{1.185262in}}%
\pgfpathlineto{\pgfqpoint{1.896376in}{1.215785in}}%
\pgfpathlineto{\pgfqpoint{1.889149in}{1.216054in}}%
\pgfpathlineto{\pgfqpoint{1.890205in}{1.243605in}}%
\pgfpathlineto{\pgfqpoint{1.938084in}{1.242176in}}%
\pgfpathlineto{\pgfqpoint{1.938292in}{1.256226in}}%
\pgfpathlineto{\pgfqpoint{1.972560in}{1.255285in}}%
\pgfpathlineto{\pgfqpoint{1.972846in}{1.269115in}}%
\pgfpathlineto{\pgfqpoint{1.992256in}{1.268523in}}%
\pgfpathlineto{\pgfqpoint{2.012052in}{1.268224in}}%
\pgfpathlineto{\pgfqpoint{2.011063in}{1.220009in}}%
\pgfpathlineto{\pgfqpoint{1.971690in}{1.220737in}}%
\pgfpathlineto{\pgfqpoint{1.971148in}{1.183063in}}%
\pgfpathlineto{\pgfqpoint{1.951284in}{1.183534in}}%
\pgfpathlineto{\pgfqpoint{1.895416in}{1.185262in}}%
\pgfpathclose%
\pgfusepath{fill}%
\end{pgfscope}%
\begin{pgfscope}%
\pgfpathrectangle{\pgfqpoint{0.100000in}{0.100000in}}{\pgfqpoint{3.608454in}{2.310000in}}%
\pgfusepath{clip}%
\pgfsetbuttcap%
\pgfsetmiterjoin%
\definecolor{currentfill}{rgb}{0.000000,0.470588,0.764706}%
\pgfsetfillcolor{currentfill}%
\pgfsetlinewidth{0.000000pt}%
\definecolor{currentstroke}{rgb}{0.000000,0.000000,0.000000}%
\pgfsetstrokecolor{currentstroke}%
\pgfsetstrokeopacity{0.000000}%
\pgfsetdash{}{0pt}%
\pgfpathmoveto{\pgfqpoint{1.811548in}{0.930044in}}%
\pgfpathlineto{\pgfqpoint{1.810392in}{0.932854in}}%
\pgfpathlineto{\pgfqpoint{1.797246in}{0.938094in}}%
\pgfpathlineto{\pgfqpoint{1.774163in}{0.940149in}}%
\pgfpathlineto{\pgfqpoint{1.775993in}{0.971342in}}%
\pgfpathlineto{\pgfqpoint{1.779205in}{0.970733in}}%
\pgfpathlineto{\pgfqpoint{1.780677in}{0.997355in}}%
\pgfpathlineto{\pgfqpoint{1.785322in}{0.998334in}}%
\pgfpathlineto{\pgfqpoint{1.799655in}{0.981871in}}%
\pgfpathlineto{\pgfqpoint{1.805846in}{0.981264in}}%
\pgfpathlineto{\pgfqpoint{1.811536in}{0.984126in}}%
\pgfpathlineto{\pgfqpoint{1.818337in}{0.980492in}}%
\pgfpathlineto{\pgfqpoint{1.820478in}{0.987070in}}%
\pgfpathlineto{\pgfqpoint{1.831143in}{0.977031in}}%
\pgfpathlineto{\pgfqpoint{1.832097in}{0.967259in}}%
\pgfpathlineto{\pgfqpoint{1.847621in}{0.966431in}}%
\pgfpathlineto{\pgfqpoint{1.846276in}{0.936427in}}%
\pgfpathlineto{\pgfqpoint{1.811868in}{0.938019in}}%
\pgfpathlineto{\pgfqpoint{1.811548in}{0.930044in}}%
\pgfpathclose%
\pgfusepath{fill}%
\end{pgfscope}%
\begin{pgfscope}%
\pgfpathrectangle{\pgfqpoint{0.100000in}{0.100000in}}{\pgfqpoint{3.608454in}{2.310000in}}%
\pgfusepath{clip}%
\pgfsetbuttcap%
\pgfsetmiterjoin%
\definecolor{currentfill}{rgb}{0.000000,0.537255,0.731373}%
\pgfsetfillcolor{currentfill}%
\pgfsetlinewidth{0.000000pt}%
\definecolor{currentstroke}{rgb}{0.000000,0.000000,0.000000}%
\pgfsetstrokecolor{currentstroke}%
\pgfsetstrokeopacity{0.000000}%
\pgfsetdash{}{0pt}%
\pgfpathmoveto{\pgfqpoint{0.906140in}{0.393309in}}%
\pgfpathlineto{\pgfqpoint{0.902140in}{0.395283in}}%
\pgfpathlineto{\pgfqpoint{0.900276in}{0.397748in}}%
\pgfpathlineto{\pgfqpoint{0.897685in}{0.398394in}}%
\pgfpathlineto{\pgfqpoint{0.898240in}{0.401261in}}%
\pgfpathlineto{\pgfqpoint{0.897317in}{0.402234in}}%
\pgfpathlineto{\pgfqpoint{0.894866in}{0.400086in}}%
\pgfpathlineto{\pgfqpoint{0.892669in}{0.399270in}}%
\pgfpathlineto{\pgfqpoint{0.891166in}{0.400988in}}%
\pgfpathlineto{\pgfqpoint{0.889388in}{0.400332in}}%
\pgfpathlineto{\pgfqpoint{0.888087in}{0.396647in}}%
\pgfpathlineto{\pgfqpoint{0.885872in}{0.401723in}}%
\pgfpathlineto{\pgfqpoint{0.885272in}{0.397710in}}%
\pgfpathlineto{\pgfqpoint{0.881126in}{0.397511in}}%
\pgfpathlineto{\pgfqpoint{0.880525in}{0.396050in}}%
\pgfpathlineto{\pgfqpoint{0.876957in}{0.396163in}}%
\pgfpathlineto{\pgfqpoint{0.872853in}{0.397496in}}%
\pgfpathlineto{\pgfqpoint{0.866722in}{0.396548in}}%
\pgfpathlineto{\pgfqpoint{0.864048in}{0.397682in}}%
\pgfpathlineto{\pgfqpoint{0.863015in}{0.395943in}}%
\pgfpathlineto{\pgfqpoint{0.860626in}{0.397438in}}%
\pgfpathlineto{\pgfqpoint{0.860500in}{0.399959in}}%
\pgfpathlineto{\pgfqpoint{0.857986in}{0.398830in}}%
\pgfpathlineto{\pgfqpoint{0.854460in}{0.399934in}}%
\pgfpathlineto{\pgfqpoint{0.854188in}{0.405776in}}%
\pgfpathlineto{\pgfqpoint{0.857163in}{0.407525in}}%
\pgfpathlineto{\pgfqpoint{0.861136in}{0.406969in}}%
\pgfpathlineto{\pgfqpoint{0.862984in}{0.405050in}}%
\pgfpathlineto{\pgfqpoint{0.865125in}{0.405991in}}%
\pgfpathlineto{\pgfqpoint{0.867877in}{0.404847in}}%
\pgfpathlineto{\pgfqpoint{0.868825in}{0.406515in}}%
\pgfpathlineto{\pgfqpoint{0.874638in}{0.407994in}}%
\pgfpathlineto{\pgfqpoint{0.872011in}{0.409155in}}%
\pgfpathlineto{\pgfqpoint{0.868887in}{0.408859in}}%
\pgfpathlineto{\pgfqpoint{0.865823in}{0.407518in}}%
\pgfpathlineto{\pgfqpoint{0.863498in}{0.411733in}}%
\pgfpathlineto{\pgfqpoint{0.863048in}{0.414812in}}%
\pgfpathlineto{\pgfqpoint{0.869799in}{0.420224in}}%
\pgfpathlineto{\pgfqpoint{0.874638in}{0.421627in}}%
\pgfpathlineto{\pgfqpoint{0.880313in}{0.424977in}}%
\pgfpathlineto{\pgfqpoint{0.883440in}{0.427931in}}%
\pgfpathlineto{\pgfqpoint{0.884480in}{0.433448in}}%
\pgfpathlineto{\pgfqpoint{0.887231in}{0.433726in}}%
\pgfpathlineto{\pgfqpoint{0.889978in}{0.432755in}}%
\pgfpathlineto{\pgfqpoint{0.896244in}{0.433381in}}%
\pgfpathlineto{\pgfqpoint{0.902329in}{0.432923in}}%
\pgfpathlineto{\pgfqpoint{0.902058in}{0.429768in}}%
\pgfpathlineto{\pgfqpoint{0.903873in}{0.427224in}}%
\pgfpathlineto{\pgfqpoint{0.909287in}{0.427185in}}%
\pgfpathlineto{\pgfqpoint{0.907671in}{0.433445in}}%
\pgfpathlineto{\pgfqpoint{0.909967in}{0.434891in}}%
\pgfpathlineto{\pgfqpoint{0.905039in}{0.438552in}}%
\pgfpathlineto{\pgfqpoint{0.903341in}{0.440916in}}%
\pgfpathlineto{\pgfqpoint{0.897892in}{0.441660in}}%
\pgfpathlineto{\pgfqpoint{0.894137in}{0.440185in}}%
\pgfpathlineto{\pgfqpoint{0.888440in}{0.441543in}}%
\pgfpathlineto{\pgfqpoint{0.882939in}{0.440631in}}%
\pgfpathlineto{\pgfqpoint{0.881028in}{0.436394in}}%
\pgfpathlineto{\pgfqpoint{0.879917in}{0.438011in}}%
\pgfpathlineto{\pgfqpoint{0.876955in}{0.437975in}}%
\pgfpathlineto{\pgfqpoint{0.871270in}{0.436691in}}%
\pgfpathlineto{\pgfqpoint{0.868136in}{0.433288in}}%
\pgfpathlineto{\pgfqpoint{0.865686in}{0.432241in}}%
\pgfpathlineto{\pgfqpoint{0.862763in}{0.432076in}}%
\pgfpathlineto{\pgfqpoint{0.860841in}{0.433543in}}%
\pgfpathlineto{\pgfqpoint{0.860190in}{0.428478in}}%
\pgfpathlineto{\pgfqpoint{0.855724in}{0.426050in}}%
\pgfpathlineto{\pgfqpoint{0.853611in}{0.426386in}}%
\pgfpathlineto{\pgfqpoint{0.851971in}{0.427933in}}%
\pgfpathlineto{\pgfqpoint{0.849085in}{0.424656in}}%
\pgfpathlineto{\pgfqpoint{0.846580in}{0.424147in}}%
\pgfpathlineto{\pgfqpoint{0.840273in}{0.426897in}}%
\pgfpathlineto{\pgfqpoint{0.838838in}{0.425763in}}%
\pgfpathlineto{\pgfqpoint{0.836699in}{0.425925in}}%
\pgfpathlineto{\pgfqpoint{0.835510in}{0.423824in}}%
\pgfpathlineto{\pgfqpoint{0.831134in}{0.425525in}}%
\pgfpathlineto{\pgfqpoint{0.826995in}{0.422232in}}%
\pgfpathlineto{\pgfqpoint{0.824051in}{0.421527in}}%
\pgfpathlineto{\pgfqpoint{0.825544in}{0.418315in}}%
\pgfpathlineto{\pgfqpoint{0.828849in}{0.415327in}}%
\pgfpathlineto{\pgfqpoint{0.831233in}{0.409187in}}%
\pgfpathlineto{\pgfqpoint{0.831242in}{0.406233in}}%
\pgfpathlineto{\pgfqpoint{0.828822in}{0.408037in}}%
\pgfpathlineto{\pgfqpoint{0.826771in}{0.405229in}}%
\pgfpathlineto{\pgfqpoint{0.822419in}{0.408486in}}%
\pgfpathlineto{\pgfqpoint{0.820929in}{0.406477in}}%
\pgfpathlineto{\pgfqpoint{0.815079in}{0.410909in}}%
\pgfpathlineto{\pgfqpoint{0.811149in}{0.413923in}}%
\pgfpathlineto{\pgfqpoint{0.812034in}{0.416427in}}%
\pgfpathlineto{\pgfqpoint{0.818042in}{0.424203in}}%
\pgfpathlineto{\pgfqpoint{0.817435in}{0.424709in}}%
\pgfpathlineto{\pgfqpoint{0.820512in}{0.428750in}}%
\pgfpathlineto{\pgfqpoint{0.824498in}{0.425742in}}%
\pgfpathlineto{\pgfqpoint{0.827494in}{0.429779in}}%
\pgfpathlineto{\pgfqpoint{0.830966in}{0.427222in}}%
\pgfpathlineto{\pgfqpoint{0.832475in}{0.429280in}}%
\pgfpathlineto{\pgfqpoint{0.834469in}{0.427785in}}%
\pgfpathlineto{\pgfqpoint{0.837531in}{0.431857in}}%
\pgfpathlineto{\pgfqpoint{0.839547in}{0.430349in}}%
\pgfpathlineto{\pgfqpoint{0.841100in}{0.432388in}}%
\pgfpathlineto{\pgfqpoint{0.842589in}{0.431260in}}%
\pgfpathlineto{\pgfqpoint{0.848446in}{0.439078in}}%
\pgfpathlineto{\pgfqpoint{0.849915in}{0.438009in}}%
\pgfpathlineto{\pgfqpoint{0.856061in}{0.446120in}}%
\pgfpathlineto{\pgfqpoint{0.857604in}{0.444984in}}%
\pgfpathlineto{\pgfqpoint{0.863736in}{0.453184in}}%
\pgfpathlineto{\pgfqpoint{0.864761in}{0.455531in}}%
\pgfpathlineto{\pgfqpoint{0.869217in}{0.461424in}}%
\pgfpathlineto{\pgfqpoint{0.870359in}{0.464352in}}%
\pgfpathlineto{\pgfqpoint{0.874246in}{0.469508in}}%
\pgfpathlineto{\pgfqpoint{0.878374in}{0.466226in}}%
\pgfpathlineto{\pgfqpoint{0.878747in}{0.464023in}}%
\pgfpathlineto{\pgfqpoint{0.894650in}{0.485392in}}%
\pgfpathlineto{\pgfqpoint{0.902205in}{0.495355in}}%
\pgfpathlineto{\pgfqpoint{0.914054in}{0.486250in}}%
\pgfpathlineto{\pgfqpoint{0.915352in}{0.487908in}}%
\pgfpathlineto{\pgfqpoint{0.940198in}{0.483801in}}%
\pgfpathlineto{\pgfqpoint{0.950166in}{0.482212in}}%
\pgfpathlineto{\pgfqpoint{0.966684in}{0.470967in}}%
\pgfpathlineto{\pgfqpoint{0.968975in}{0.474416in}}%
\pgfpathlineto{\pgfqpoint{0.972860in}{0.471797in}}%
\pgfpathlineto{\pgfqpoint{0.986452in}{0.463301in}}%
\pgfpathlineto{\pgfqpoint{0.979580in}{0.452265in}}%
\pgfpathlineto{\pgfqpoint{0.977356in}{0.447349in}}%
\pgfpathlineto{\pgfqpoint{0.967777in}{0.431965in}}%
\pgfpathlineto{\pgfqpoint{0.961359in}{0.436162in}}%
\pgfpathlineto{\pgfqpoint{0.960218in}{0.433861in}}%
\pgfpathlineto{\pgfqpoint{0.949551in}{0.416794in}}%
\pgfpathlineto{\pgfqpoint{0.946494in}{0.418805in}}%
\pgfpathlineto{\pgfqpoint{0.945694in}{0.417578in}}%
\pgfpathlineto{\pgfqpoint{0.930988in}{0.427301in}}%
\pgfpathlineto{\pgfqpoint{0.926490in}{0.421058in}}%
\pgfpathlineto{\pgfqpoint{0.921451in}{0.413351in}}%
\pgfpathlineto{\pgfqpoint{0.918151in}{0.415401in}}%
\pgfpathlineto{\pgfqpoint{0.916327in}{0.412673in}}%
\pgfpathlineto{\pgfqpoint{0.917407in}{0.411973in}}%
\pgfpathlineto{\pgfqpoint{0.912214in}{0.404513in}}%
\pgfpathlineto{\pgfqpoint{0.913452in}{0.403668in}}%
\pgfpathlineto{\pgfqpoint{0.908663in}{0.396558in}}%
\pgfpathlineto{\pgfqpoint{0.906140in}{0.393309in}}%
\pgfpathclose%
\pgfusepath{fill}%
\end{pgfscope}%
\begin{pgfscope}%
\pgfpathrectangle{\pgfqpoint{0.100000in}{0.100000in}}{\pgfqpoint{3.608454in}{2.310000in}}%
\pgfusepath{clip}%
\pgfsetbuttcap%
\pgfsetmiterjoin%
\definecolor{currentfill}{rgb}{0.000000,0.874510,0.562745}%
\pgfsetfillcolor{currentfill}%
\pgfsetlinewidth{0.000000pt}%
\definecolor{currentstroke}{rgb}{0.000000,0.000000,0.000000}%
\pgfsetstrokecolor{currentstroke}%
\pgfsetstrokeopacity{0.000000}%
\pgfsetdash{}{0pt}%
\pgfpathmoveto{\pgfqpoint{2.901040in}{1.226866in}}%
\pgfpathlineto{\pgfqpoint{2.907489in}{1.230091in}}%
\pgfpathlineto{\pgfqpoint{2.904928in}{1.234962in}}%
\pgfpathlineto{\pgfqpoint{2.916003in}{1.242873in}}%
\pgfpathlineto{\pgfqpoint{2.908022in}{1.254600in}}%
\pgfpathlineto{\pgfqpoint{2.902185in}{1.260526in}}%
\pgfpathlineto{\pgfqpoint{2.921023in}{1.282534in}}%
\pgfpathlineto{\pgfqpoint{2.923822in}{1.280887in}}%
\pgfpathlineto{\pgfqpoint{2.923526in}{1.275100in}}%
\pgfpathlineto{\pgfqpoint{2.926305in}{1.268226in}}%
\pgfpathlineto{\pgfqpoint{2.935481in}{1.264222in}}%
\pgfpathlineto{\pgfqpoint{2.942904in}{1.258852in}}%
\pgfpathlineto{\pgfqpoint{2.950534in}{1.260497in}}%
\pgfpathlineto{\pgfqpoint{2.960443in}{1.270838in}}%
\pgfpathlineto{\pgfqpoint{2.965505in}{1.286020in}}%
\pgfpathlineto{\pgfqpoint{2.967623in}{1.287269in}}%
\pgfpathlineto{\pgfqpoint{2.968540in}{1.292624in}}%
\pgfpathlineto{\pgfqpoint{2.974604in}{1.294585in}}%
\pgfpathlineto{\pgfqpoint{2.991086in}{1.284432in}}%
\pgfpathlineto{\pgfqpoint{2.992726in}{1.278073in}}%
\pgfpathlineto{\pgfqpoint{2.983343in}{1.270885in}}%
\pgfpathlineto{\pgfqpoint{2.994849in}{1.262443in}}%
\pgfpathlineto{\pgfqpoint{2.990512in}{1.259171in}}%
\pgfpathlineto{\pgfqpoint{2.993913in}{1.252683in}}%
\pgfpathlineto{\pgfqpoint{3.005275in}{1.242589in}}%
\pgfpathlineto{\pgfqpoint{2.986436in}{1.233543in}}%
\pgfpathlineto{\pgfqpoint{2.984220in}{1.230496in}}%
\pgfpathlineto{\pgfqpoint{2.974075in}{1.228469in}}%
\pgfpathlineto{\pgfqpoint{2.968379in}{1.238517in}}%
\pgfpathlineto{\pgfqpoint{2.960170in}{1.246122in}}%
\pgfpathlineto{\pgfqpoint{2.943559in}{1.240996in}}%
\pgfpathlineto{\pgfqpoint{2.937030in}{1.237689in}}%
\pgfpathlineto{\pgfqpoint{2.933591in}{1.232567in}}%
\pgfpathlineto{\pgfqpoint{2.921871in}{1.228104in}}%
\pgfpathlineto{\pgfqpoint{2.920870in}{1.223465in}}%
\pgfpathlineto{\pgfqpoint{2.907547in}{1.214437in}}%
\pgfpathlineto{\pgfqpoint{2.901040in}{1.226866in}}%
\pgfpathclose%
\pgfusepath{fill}%
\end{pgfscope}%
\begin{pgfscope}%
\pgfpathrectangle{\pgfqpoint{0.100000in}{0.100000in}}{\pgfqpoint{3.608454in}{2.310000in}}%
\pgfusepath{clip}%
\pgfsetbuttcap%
\pgfsetmiterjoin%
\definecolor{currentfill}{rgb}{0.000000,0.478431,0.760784}%
\pgfsetfillcolor{currentfill}%
\pgfsetlinewidth{0.000000pt}%
\definecolor{currentstroke}{rgb}{0.000000,0.000000,0.000000}%
\pgfsetstrokecolor{currentstroke}%
\pgfsetstrokeopacity{0.000000}%
\pgfsetdash{}{0pt}%
\pgfpathmoveto{\pgfqpoint{1.864232in}{2.006135in}}%
\pgfpathlineto{\pgfqpoint{1.862851in}{1.978498in}}%
\pgfpathlineto{\pgfqpoint{1.864737in}{1.978421in}}%
\pgfpathlineto{\pgfqpoint{1.863197in}{1.950599in}}%
\pgfpathlineto{\pgfqpoint{1.828664in}{1.952547in}}%
\pgfpathlineto{\pgfqpoint{1.830555in}{1.980267in}}%
\pgfpathlineto{\pgfqpoint{1.828413in}{1.980392in}}%
\pgfpathlineto{\pgfqpoint{1.830021in}{2.007934in}}%
\pgfpathlineto{\pgfqpoint{1.864232in}{2.006135in}}%
\pgfpathclose%
\pgfusepath{fill}%
\end{pgfscope}%
\begin{pgfscope}%
\pgfpathrectangle{\pgfqpoint{0.100000in}{0.100000in}}{\pgfqpoint{3.608454in}{2.310000in}}%
\pgfusepath{clip}%
\pgfsetbuttcap%
\pgfsetmiterjoin%
\definecolor{currentfill}{rgb}{0.000000,0.505882,0.747059}%
\pgfsetfillcolor{currentfill}%
\pgfsetlinewidth{0.000000pt}%
\definecolor{currentstroke}{rgb}{0.000000,0.000000,0.000000}%
\pgfsetstrokecolor{currentstroke}%
\pgfsetstrokeopacity{0.000000}%
\pgfsetdash{}{0pt}%
\pgfpathmoveto{\pgfqpoint{1.728388in}{1.154348in}}%
\pgfpathlineto{\pgfqpoint{1.788727in}{1.150707in}}%
\pgfpathlineto{\pgfqpoint{1.787003in}{1.115616in}}%
\pgfpathlineto{\pgfqpoint{1.752524in}{1.117682in}}%
\pgfpathlineto{\pgfqpoint{1.683362in}{1.122232in}}%
\pgfpathlineto{\pgfqpoint{1.685892in}{1.157286in}}%
\pgfpathlineto{\pgfqpoint{1.728388in}{1.154348in}}%
\pgfpathclose%
\pgfusepath{fill}%
\end{pgfscope}%
\begin{pgfscope}%
\pgfpathrectangle{\pgfqpoint{0.100000in}{0.100000in}}{\pgfqpoint{3.608454in}{2.310000in}}%
\pgfusepath{clip}%
\pgfsetbuttcap%
\pgfsetmiterjoin%
\definecolor{currentfill}{rgb}{0.000000,0.482353,0.758824}%
\pgfsetfillcolor{currentfill}%
\pgfsetlinewidth{0.000000pt}%
\definecolor{currentstroke}{rgb}{0.000000,0.000000,0.000000}%
\pgfsetstrokecolor{currentstroke}%
\pgfsetstrokeopacity{0.000000}%
\pgfsetdash{}{0pt}%
\pgfpathmoveto{\pgfqpoint{2.615996in}{1.634560in}}%
\pgfpathlineto{\pgfqpoint{2.617722in}{1.657792in}}%
\pgfpathlineto{\pgfqpoint{2.615667in}{1.673618in}}%
\pgfpathlineto{\pgfqpoint{2.611600in}{1.688088in}}%
\pgfpathlineto{\pgfqpoint{2.600135in}{1.708748in}}%
\pgfpathlineto{\pgfqpoint{2.592486in}{1.725771in}}%
\pgfpathlineto{\pgfqpoint{2.591900in}{1.732408in}}%
\pgfpathlineto{\pgfqpoint{2.596450in}{1.744135in}}%
\pgfpathlineto{\pgfqpoint{2.618822in}{1.745974in}}%
\pgfpathlineto{\pgfqpoint{2.645951in}{1.748717in}}%
\pgfpathlineto{\pgfqpoint{2.648810in}{1.721346in}}%
\pgfpathlineto{\pgfqpoint{2.676133in}{1.724089in}}%
\pgfpathlineto{\pgfqpoint{2.689796in}{1.725591in}}%
\pgfpathlineto{\pgfqpoint{2.693396in}{1.698299in}}%
\pgfpathlineto{\pgfqpoint{2.696454in}{1.670883in}}%
\pgfpathlineto{\pgfqpoint{2.682759in}{1.669371in}}%
\pgfpathlineto{\pgfqpoint{2.655531in}{1.666251in}}%
\pgfpathlineto{\pgfqpoint{2.658513in}{1.638934in}}%
\pgfpathlineto{\pgfqpoint{2.615996in}{1.634560in}}%
\pgfpathclose%
\pgfusepath{fill}%
\end{pgfscope}%
\begin{pgfscope}%
\pgfpathrectangle{\pgfqpoint{0.100000in}{0.100000in}}{\pgfqpoint{3.608454in}{2.310000in}}%
\pgfusepath{clip}%
\pgfsetbuttcap%
\pgfsetmiterjoin%
\definecolor{currentfill}{rgb}{0.000000,0.372549,0.813725}%
\pgfsetfillcolor{currentfill}%
\pgfsetlinewidth{0.000000pt}%
\definecolor{currentstroke}{rgb}{0.000000,0.000000,0.000000}%
\pgfsetstrokecolor{currentstroke}%
\pgfsetstrokeopacity{0.000000}%
\pgfsetdash{}{0pt}%
\pgfpathmoveto{\pgfqpoint{1.148005in}{1.494799in}}%
\pgfpathlineto{\pgfqpoint{1.136590in}{1.496867in}}%
\pgfpathlineto{\pgfqpoint{1.137161in}{1.499935in}}%
\pgfpathlineto{\pgfqpoint{1.128853in}{1.510727in}}%
\pgfpathlineto{\pgfqpoint{1.131819in}{1.519099in}}%
\pgfpathlineto{\pgfqpoint{1.129198in}{1.527981in}}%
\pgfpathlineto{\pgfqpoint{1.120187in}{1.531417in}}%
\pgfpathlineto{\pgfqpoint{1.112401in}{1.545424in}}%
\pgfpathlineto{\pgfqpoint{1.116941in}{1.552445in}}%
\pgfpathlineto{\pgfqpoint{1.105133in}{1.552914in}}%
\pgfpathlineto{\pgfqpoint{1.090523in}{1.546657in}}%
\pgfpathlineto{\pgfqpoint{1.087943in}{1.551215in}}%
\pgfpathlineto{\pgfqpoint{1.077577in}{1.552756in}}%
\pgfpathlineto{\pgfqpoint{1.075863in}{1.537379in}}%
\pgfpathlineto{\pgfqpoint{1.077321in}{1.529704in}}%
\pgfpathlineto{\pgfqpoint{1.072847in}{1.517633in}}%
\pgfpathlineto{\pgfqpoint{1.061605in}{1.511149in}}%
\pgfpathlineto{\pgfqpoint{1.034454in}{1.516384in}}%
\pgfpathlineto{\pgfqpoint{0.960301in}{1.531851in}}%
\pgfpathlineto{\pgfqpoint{0.978586in}{1.616384in}}%
\pgfpathlineto{\pgfqpoint{1.051249in}{1.601475in}}%
\pgfpathlineto{\pgfqpoint{1.070321in}{1.603891in}}%
\pgfpathlineto{\pgfqpoint{1.089153in}{1.621193in}}%
\pgfpathlineto{\pgfqpoint{1.101371in}{1.618838in}}%
\pgfpathlineto{\pgfqpoint{1.110470in}{1.623700in}}%
\pgfpathlineto{\pgfqpoint{1.117387in}{1.618454in}}%
\pgfpathlineto{\pgfqpoint{1.121923in}{1.621467in}}%
\pgfpathlineto{\pgfqpoint{1.132825in}{1.619941in}}%
\pgfpathlineto{\pgfqpoint{1.136553in}{1.614347in}}%
\pgfpathlineto{\pgfqpoint{1.144102in}{1.610430in}}%
\pgfpathlineto{\pgfqpoint{1.143956in}{1.603549in}}%
\pgfpathlineto{\pgfqpoint{1.146992in}{1.596437in}}%
\pgfpathlineto{\pgfqpoint{1.157543in}{1.601688in}}%
\pgfpathlineto{\pgfqpoint{1.154007in}{1.581932in}}%
\pgfpathlineto{\pgfqpoint{1.186062in}{1.576054in}}%
\pgfpathlineto{\pgfqpoint{1.215522in}{1.571272in}}%
\pgfpathlineto{\pgfqpoint{1.213136in}{1.556950in}}%
\pgfpathlineto{\pgfqpoint{1.203263in}{1.558508in}}%
\pgfpathlineto{\pgfqpoint{1.198089in}{1.561222in}}%
\pgfpathlineto{\pgfqpoint{1.190708in}{1.558686in}}%
\pgfpathlineto{\pgfqpoint{1.173397in}{1.557849in}}%
\pgfpathlineto{\pgfqpoint{1.167943in}{1.559368in}}%
\pgfpathlineto{\pgfqpoint{1.158113in}{1.555519in}}%
\pgfpathlineto{\pgfqpoint{1.148005in}{1.494799in}}%
\pgfpathclose%
\pgfusepath{fill}%
\end{pgfscope}%
\begin{pgfscope}%
\pgfpathrectangle{\pgfqpoint{0.100000in}{0.100000in}}{\pgfqpoint{3.608454in}{2.310000in}}%
\pgfusepath{clip}%
\pgfsetbuttcap%
\pgfsetmiterjoin%
\definecolor{currentfill}{rgb}{0.000000,0.286275,0.856863}%
\pgfsetfillcolor{currentfill}%
\pgfsetlinewidth{0.000000pt}%
\definecolor{currentstroke}{rgb}{0.000000,0.000000,0.000000}%
\pgfsetstrokecolor{currentstroke}%
\pgfsetstrokeopacity{0.000000}%
\pgfsetdash{}{0pt}%
\pgfpathmoveto{\pgfqpoint{2.056854in}{1.480886in}}%
\pgfpathlineto{\pgfqpoint{2.019155in}{1.481378in}}%
\pgfpathlineto{\pgfqpoint{2.019500in}{1.502021in}}%
\pgfpathlineto{\pgfqpoint{1.992970in}{1.502588in}}%
\pgfpathlineto{\pgfqpoint{1.994270in}{1.557682in}}%
\pgfpathlineto{\pgfqpoint{2.021531in}{1.557101in}}%
\pgfpathlineto{\pgfqpoint{2.021430in}{1.552509in}}%
\pgfpathlineto{\pgfqpoint{2.040513in}{1.552171in}}%
\pgfpathlineto{\pgfqpoint{2.043988in}{1.561009in}}%
\pgfpathlineto{\pgfqpoint{2.039668in}{1.566696in}}%
\pgfpathlineto{\pgfqpoint{2.067148in}{1.566188in}}%
\pgfpathlineto{\pgfqpoint{2.073952in}{1.566162in}}%
\pgfpathlineto{\pgfqpoint{2.073777in}{1.545495in}}%
\pgfpathlineto{\pgfqpoint{2.077274in}{1.537859in}}%
\pgfpathlineto{\pgfqpoint{2.097619in}{1.537721in}}%
\pgfpathlineto{\pgfqpoint{2.097514in}{1.510367in}}%
\pgfpathlineto{\pgfqpoint{2.083971in}{1.510379in}}%
\pgfpathlineto{\pgfqpoint{2.083784in}{1.490019in}}%
\pgfpathlineto{\pgfqpoint{2.058174in}{1.490199in}}%
\pgfpathlineto{\pgfqpoint{2.056854in}{1.480886in}}%
\pgfpathclose%
\pgfusepath{fill}%
\end{pgfscope}%
\begin{pgfscope}%
\pgfpathrectangle{\pgfqpoint{0.100000in}{0.100000in}}{\pgfqpoint{3.608454in}{2.310000in}}%
\pgfusepath{clip}%
\pgfsetbuttcap%
\pgfsetmiterjoin%
\definecolor{currentfill}{rgb}{0.000000,0.490196,0.754902}%
\pgfsetfillcolor{currentfill}%
\pgfsetlinewidth{0.000000pt}%
\definecolor{currentstroke}{rgb}{0.000000,0.000000,0.000000}%
\pgfsetstrokecolor{currentstroke}%
\pgfsetstrokeopacity{0.000000}%
\pgfsetdash{}{0pt}%
\pgfpathmoveto{\pgfqpoint{2.293781in}{1.726557in}}%
\pgfpathlineto{\pgfqpoint{2.294773in}{1.699116in}}%
\pgfpathlineto{\pgfqpoint{2.253472in}{1.697776in}}%
\pgfpathlineto{\pgfqpoint{2.252724in}{1.724220in}}%
\pgfpathlineto{\pgfqpoint{2.238913in}{1.724992in}}%
\pgfpathlineto{\pgfqpoint{2.218568in}{1.724511in}}%
\pgfpathlineto{\pgfqpoint{2.217878in}{1.773970in}}%
\pgfpathlineto{\pgfqpoint{2.217797in}{1.777469in}}%
\pgfpathlineto{\pgfqpoint{2.231724in}{1.779996in}}%
\pgfpathlineto{\pgfqpoint{2.231530in}{1.786839in}}%
\pgfpathlineto{\pgfqpoint{2.241126in}{1.786250in}}%
\pgfpathlineto{\pgfqpoint{2.247155in}{1.782906in}}%
\pgfpathlineto{\pgfqpoint{2.258667in}{1.780519in}}%
\pgfpathlineto{\pgfqpoint{2.264630in}{1.772555in}}%
\pgfpathlineto{\pgfqpoint{2.273713in}{1.770000in}}%
\pgfpathlineto{\pgfqpoint{2.281318in}{1.764757in}}%
\pgfpathlineto{\pgfqpoint{2.285525in}{1.753739in}}%
\pgfpathlineto{\pgfqpoint{2.272950in}{1.753323in}}%
\pgfpathlineto{\pgfqpoint{2.273836in}{1.725888in}}%
\pgfpathlineto{\pgfqpoint{2.293781in}{1.726557in}}%
\pgfpathclose%
\pgfusepath{fill}%
\end{pgfscope}%
\begin{pgfscope}%
\pgfpathrectangle{\pgfqpoint{0.100000in}{0.100000in}}{\pgfqpoint{3.608454in}{2.310000in}}%
\pgfusepath{clip}%
\pgfsetbuttcap%
\pgfsetmiterjoin%
\definecolor{currentfill}{rgb}{0.000000,0.470588,0.764706}%
\pgfsetfillcolor{currentfill}%
\pgfsetlinewidth{0.000000pt}%
\definecolor{currentstroke}{rgb}{0.000000,0.000000,0.000000}%
\pgfsetstrokecolor{currentstroke}%
\pgfsetstrokeopacity{0.000000}%
\pgfsetdash{}{0pt}%
\pgfpathmoveto{\pgfqpoint{1.944686in}{1.753364in}}%
\pgfpathlineto{\pgfqpoint{1.999302in}{1.751993in}}%
\pgfpathlineto{\pgfqpoint{1.998767in}{1.724416in}}%
\pgfpathlineto{\pgfqpoint{1.957610in}{1.725492in}}%
\pgfpathlineto{\pgfqpoint{1.943727in}{1.725985in}}%
\pgfpathlineto{\pgfqpoint{1.944686in}{1.753364in}}%
\pgfpathclose%
\pgfusepath{fill}%
\end{pgfscope}%
\begin{pgfscope}%
\pgfpathrectangle{\pgfqpoint{0.100000in}{0.100000in}}{\pgfqpoint{3.608454in}{2.310000in}}%
\pgfusepath{clip}%
\pgfsetbuttcap%
\pgfsetmiterjoin%
\definecolor{currentfill}{rgb}{0.000000,0.501961,0.749020}%
\pgfsetfillcolor{currentfill}%
\pgfsetlinewidth{0.000000pt}%
\definecolor{currentstroke}{rgb}{0.000000,0.000000,0.000000}%
\pgfsetstrokecolor{currentstroke}%
\pgfsetstrokeopacity{0.000000}%
\pgfsetdash{}{0pt}%
\pgfpathmoveto{\pgfqpoint{2.086211in}{1.418729in}}%
\pgfpathlineto{\pgfqpoint{2.086114in}{1.391298in}}%
\pgfpathlineto{\pgfqpoint{2.072459in}{1.391367in}}%
\pgfpathlineto{\pgfqpoint{2.058778in}{1.391470in}}%
\pgfpathlineto{\pgfqpoint{2.058697in}{1.384598in}}%
\pgfpathlineto{\pgfqpoint{2.043672in}{1.384754in}}%
\pgfpathlineto{\pgfqpoint{2.031304in}{1.384920in}}%
\pgfpathlineto{\pgfqpoint{2.031759in}{1.419257in}}%
\pgfpathlineto{\pgfqpoint{2.086211in}{1.418729in}}%
\pgfpathclose%
\pgfusepath{fill}%
\end{pgfscope}%
\begin{pgfscope}%
\pgfpathrectangle{\pgfqpoint{0.100000in}{0.100000in}}{\pgfqpoint{3.608454in}{2.310000in}}%
\pgfusepath{clip}%
\pgfsetbuttcap%
\pgfsetmiterjoin%
\definecolor{currentfill}{rgb}{0.000000,0.458824,0.770588}%
\pgfsetfillcolor{currentfill}%
\pgfsetlinewidth{0.000000pt}%
\definecolor{currentstroke}{rgb}{0.000000,0.000000,0.000000}%
\pgfsetstrokecolor{currentstroke}%
\pgfsetstrokeopacity{0.000000}%
\pgfsetdash{}{0pt}%
\pgfpathmoveto{\pgfqpoint{2.414986in}{1.743088in}}%
\pgfpathlineto{\pgfqpoint{2.416473in}{1.716189in}}%
\pgfpathlineto{\pgfqpoint{2.405834in}{1.715444in}}%
\pgfpathlineto{\pgfqpoint{2.400634in}{1.726438in}}%
\pgfpathlineto{\pgfqpoint{2.395001in}{1.732218in}}%
\pgfpathlineto{\pgfqpoint{2.394954in}{1.739243in}}%
\pgfpathlineto{\pgfqpoint{2.389968in}{1.750300in}}%
\pgfpathlineto{\pgfqpoint{2.396439in}{1.763257in}}%
\pgfpathlineto{\pgfqpoint{2.373196in}{1.761948in}}%
\pgfpathlineto{\pgfqpoint{2.372249in}{1.775882in}}%
\pgfpathlineto{\pgfqpoint{2.344731in}{1.774369in}}%
\pgfpathlineto{\pgfqpoint{2.344426in}{1.781264in}}%
\pgfpathlineto{\pgfqpoint{2.337567in}{1.780978in}}%
\pgfpathlineto{\pgfqpoint{2.334657in}{1.843071in}}%
\pgfpathlineto{\pgfqpoint{2.334351in}{1.850010in}}%
\pgfpathlineto{\pgfqpoint{2.348110in}{1.850512in}}%
\pgfpathlineto{\pgfqpoint{2.383604in}{1.852652in}}%
\pgfpathlineto{\pgfqpoint{2.382793in}{1.866283in}}%
\pgfpathlineto{\pgfqpoint{2.417000in}{1.868413in}}%
\pgfpathlineto{\pgfqpoint{2.417621in}{1.861430in}}%
\pgfpathlineto{\pgfqpoint{2.419246in}{1.833798in}}%
\pgfpathlineto{\pgfqpoint{2.430551in}{1.834455in}}%
\pgfpathlineto{\pgfqpoint{2.432784in}{1.799789in}}%
\pgfpathlineto{\pgfqpoint{2.434927in}{1.765090in}}%
\pgfpathlineto{\pgfqpoint{2.413744in}{1.763959in}}%
\pgfpathlineto{\pgfqpoint{2.414986in}{1.743088in}}%
\pgfpathclose%
\pgfusepath{fill}%
\end{pgfscope}%
\begin{pgfscope}%
\pgfpathrectangle{\pgfqpoint{0.100000in}{0.100000in}}{\pgfqpoint{3.608454in}{2.310000in}}%
\pgfusepath{clip}%
\pgfsetbuttcap%
\pgfsetmiterjoin%
\definecolor{currentfill}{rgb}{0.000000,0.894118,0.552941}%
\pgfsetfillcolor{currentfill}%
\pgfsetlinewidth{0.000000pt}%
\definecolor{currentstroke}{rgb}{0.000000,0.000000,0.000000}%
\pgfsetstrokecolor{currentstroke}%
\pgfsetstrokeopacity{0.000000}%
\pgfsetdash{}{0pt}%
\pgfpathmoveto{\pgfqpoint{0.531886in}{1.661402in}}%
\pgfpathlineto{\pgfqpoint{0.522050in}{1.660114in}}%
\pgfpathlineto{\pgfqpoint{0.515901in}{1.652222in}}%
\pgfpathlineto{\pgfqpoint{0.512624in}{1.653371in}}%
\pgfpathlineto{\pgfqpoint{0.504486in}{1.647574in}}%
\pgfpathlineto{\pgfqpoint{0.490498in}{1.651799in}}%
\pgfpathlineto{\pgfqpoint{0.488413in}{1.645315in}}%
\pgfpathlineto{\pgfqpoint{0.436761in}{1.661477in}}%
\pgfpathlineto{\pgfqpoint{0.441276in}{1.674970in}}%
\pgfpathlineto{\pgfqpoint{0.442069in}{1.688570in}}%
\pgfpathlineto{\pgfqpoint{0.444780in}{1.696306in}}%
\pgfpathlineto{\pgfqpoint{0.441132in}{1.699873in}}%
\pgfpathlineto{\pgfqpoint{0.442506in}{1.705022in}}%
\pgfpathlineto{\pgfqpoint{0.448092in}{1.709303in}}%
\pgfpathlineto{\pgfqpoint{0.454296in}{1.709584in}}%
\pgfpathlineto{\pgfqpoint{0.458930in}{1.713085in}}%
\pgfpathlineto{\pgfqpoint{0.469494in}{1.715995in}}%
\pgfpathlineto{\pgfqpoint{0.468741in}{1.723271in}}%
\pgfpathlineto{\pgfqpoint{0.475334in}{1.730099in}}%
\pgfpathlineto{\pgfqpoint{0.485004in}{1.744684in}}%
\pgfpathlineto{\pgfqpoint{0.489189in}{1.745779in}}%
\pgfpathlineto{\pgfqpoint{0.494828in}{1.757961in}}%
\pgfpathlineto{\pgfqpoint{0.561272in}{1.737785in}}%
\pgfpathlineto{\pgfqpoint{0.555668in}{1.716515in}}%
\pgfpathlineto{\pgfqpoint{0.544816in}{1.681752in}}%
\pgfpathlineto{\pgfqpoint{0.535033in}{1.684697in}}%
\pgfpathlineto{\pgfqpoint{0.534540in}{1.676936in}}%
\pgfpathlineto{\pgfqpoint{0.540383in}{1.672372in}}%
\pgfpathlineto{\pgfqpoint{0.537339in}{1.664898in}}%
\pgfpathlineto{\pgfqpoint{0.531886in}{1.661402in}}%
\pgfpathclose%
\pgfusepath{fill}%
\end{pgfscope}%
\begin{pgfscope}%
\pgfpathrectangle{\pgfqpoint{0.100000in}{0.100000in}}{\pgfqpoint{3.608454in}{2.310000in}}%
\pgfusepath{clip}%
\pgfsetbuttcap%
\pgfsetmiterjoin%
\definecolor{currentfill}{rgb}{0.000000,0.415686,0.792157}%
\pgfsetfillcolor{currentfill}%
\pgfsetlinewidth{0.000000pt}%
\definecolor{currentstroke}{rgb}{0.000000,0.000000,0.000000}%
\pgfsetstrokecolor{currentstroke}%
\pgfsetstrokeopacity{0.000000}%
\pgfsetdash{}{0pt}%
\pgfpathmoveto{\pgfqpoint{2.688201in}{1.338715in}}%
\pgfpathlineto{\pgfqpoint{2.687551in}{1.344405in}}%
\pgfpathlineto{\pgfqpoint{2.680182in}{1.348259in}}%
\pgfpathlineto{\pgfqpoint{2.679482in}{1.355109in}}%
\pgfpathlineto{\pgfqpoint{2.683930in}{1.355595in}}%
\pgfpathlineto{\pgfqpoint{2.693342in}{1.363630in}}%
\pgfpathlineto{\pgfqpoint{2.708117in}{1.365276in}}%
\pgfpathlineto{\pgfqpoint{2.710112in}{1.347801in}}%
\pgfpathlineto{\pgfqpoint{2.720376in}{1.354526in}}%
\pgfpathlineto{\pgfqpoint{2.726746in}{1.347306in}}%
\pgfpathlineto{\pgfqpoint{2.718732in}{1.341235in}}%
\pgfpathlineto{\pgfqpoint{2.713052in}{1.339727in}}%
\pgfpathlineto{\pgfqpoint{2.707002in}{1.331992in}}%
\pgfpathlineto{\pgfqpoint{2.697292in}{1.333124in}}%
\pgfpathlineto{\pgfqpoint{2.697081in}{1.338065in}}%
\pgfpathlineto{\pgfqpoint{2.688201in}{1.338715in}}%
\pgfpathclose%
\pgfusepath{fill}%
\end{pgfscope}%
\begin{pgfscope}%
\pgfpathrectangle{\pgfqpoint{0.100000in}{0.100000in}}{\pgfqpoint{3.608454in}{2.310000in}}%
\pgfusepath{clip}%
\pgfsetbuttcap%
\pgfsetmiterjoin%
\definecolor{currentfill}{rgb}{0.000000,0.435294,0.782353}%
\pgfsetfillcolor{currentfill}%
\pgfsetlinewidth{0.000000pt}%
\definecolor{currentstroke}{rgb}{0.000000,0.000000,0.000000}%
\pgfsetstrokecolor{currentstroke}%
\pgfsetstrokeopacity{0.000000}%
\pgfsetdash{}{0pt}%
\pgfpathmoveto{\pgfqpoint{1.912258in}{1.559919in}}%
\pgfpathlineto{\pgfqpoint{1.911513in}{1.532437in}}%
\pgfpathlineto{\pgfqpoint{1.884664in}{1.533508in}}%
\pgfpathlineto{\pgfqpoint{1.857457in}{1.534695in}}%
\pgfpathlineto{\pgfqpoint{1.858208in}{1.562101in}}%
\pgfpathlineto{\pgfqpoint{1.830208in}{1.563450in}}%
\pgfpathlineto{\pgfqpoint{1.831612in}{1.590776in}}%
\pgfpathlineto{\pgfqpoint{1.858155in}{1.589614in}}%
\pgfpathlineto{\pgfqpoint{1.885929in}{1.588436in}}%
\pgfpathlineto{\pgfqpoint{1.884919in}{1.560911in}}%
\pgfpathlineto{\pgfqpoint{1.912258in}{1.559919in}}%
\pgfpathclose%
\pgfusepath{fill}%
\end{pgfscope}%
\begin{pgfscope}%
\pgfpathrectangle{\pgfqpoint{0.100000in}{0.100000in}}{\pgfqpoint{3.608454in}{2.310000in}}%
\pgfusepath{clip}%
\pgfsetbuttcap%
\pgfsetmiterjoin%
\definecolor{currentfill}{rgb}{0.000000,0.313725,0.843137}%
\pgfsetfillcolor{currentfill}%
\pgfsetlinewidth{0.000000pt}%
\definecolor{currentstroke}{rgb}{0.000000,0.000000,0.000000}%
\pgfsetstrokecolor{currentstroke}%
\pgfsetstrokeopacity{0.000000}%
\pgfsetdash{}{0pt}%
\pgfpathmoveto{\pgfqpoint{1.968280in}{1.606524in}}%
\pgfpathlineto{\pgfqpoint{1.988757in}{1.605977in}}%
\pgfpathlineto{\pgfqpoint{1.989587in}{1.638444in}}%
\pgfpathlineto{\pgfqpoint{2.001679in}{1.633660in}}%
\pgfpathlineto{\pgfqpoint{2.002309in}{1.663696in}}%
\pgfpathlineto{\pgfqpoint{2.022652in}{1.663329in}}%
\pgfpathlineto{\pgfqpoint{2.019320in}{1.660716in}}%
\pgfpathlineto{\pgfqpoint{2.019889in}{1.653420in}}%
\pgfpathlineto{\pgfqpoint{2.017541in}{1.649592in}}%
\pgfpathlineto{\pgfqpoint{2.056914in}{1.649093in}}%
\pgfpathlineto{\pgfqpoint{2.056624in}{1.621513in}}%
\pgfpathlineto{\pgfqpoint{2.063540in}{1.621479in}}%
\pgfpathlineto{\pgfqpoint{2.067679in}{1.614586in}}%
\pgfpathlineto{\pgfqpoint{2.067449in}{1.593735in}}%
\pgfpathlineto{\pgfqpoint{2.067148in}{1.566188in}}%
\pgfpathlineto{\pgfqpoint{2.039668in}{1.566696in}}%
\pgfpathlineto{\pgfqpoint{2.040288in}{1.575134in}}%
\pgfpathlineto{\pgfqpoint{2.036665in}{1.576098in}}%
\pgfpathlineto{\pgfqpoint{2.027934in}{1.590280in}}%
\pgfpathlineto{\pgfqpoint{2.027350in}{1.599345in}}%
\pgfpathlineto{\pgfqpoint{2.005666in}{1.599833in}}%
\pgfpathlineto{\pgfqpoint{2.000030in}{1.598859in}}%
\pgfpathlineto{\pgfqpoint{1.999669in}{1.585100in}}%
\pgfpathlineto{\pgfqpoint{1.988196in}{1.585344in}}%
\pgfpathlineto{\pgfqpoint{1.967706in}{1.585865in}}%
\pgfpathlineto{\pgfqpoint{1.968280in}{1.606524in}}%
\pgfpathclose%
\pgfusepath{fill}%
\end{pgfscope}%
\begin{pgfscope}%
\pgfpathrectangle{\pgfqpoint{0.100000in}{0.100000in}}{\pgfqpoint{3.608454in}{2.310000in}}%
\pgfusepath{clip}%
\pgfsetbuttcap%
\pgfsetmiterjoin%
\definecolor{currentfill}{rgb}{0.000000,0.650980,0.674510}%
\pgfsetfillcolor{currentfill}%
\pgfsetlinewidth{0.000000pt}%
\definecolor{currentstroke}{rgb}{0.000000,0.000000,0.000000}%
\pgfsetstrokecolor{currentstroke}%
\pgfsetstrokeopacity{0.000000}%
\pgfsetdash{}{0pt}%
\pgfpathmoveto{\pgfqpoint{3.051321in}{0.914811in}}%
\pgfpathlineto{\pgfqpoint{3.047304in}{0.907519in}}%
\pgfpathlineto{\pgfqpoint{3.038277in}{0.916697in}}%
\pgfpathlineto{\pgfqpoint{3.022585in}{0.898037in}}%
\pgfpathlineto{\pgfqpoint{3.015378in}{0.900512in}}%
\pgfpathlineto{\pgfqpoint{3.011589in}{0.908504in}}%
\pgfpathlineto{\pgfqpoint{3.010014in}{0.919433in}}%
\pgfpathlineto{\pgfqpoint{3.003885in}{0.926116in}}%
\pgfpathlineto{\pgfqpoint{3.003733in}{0.931980in}}%
\pgfpathlineto{\pgfqpoint{2.999992in}{0.934399in}}%
\pgfpathlineto{\pgfqpoint{2.988444in}{0.938374in}}%
\pgfpathlineto{\pgfqpoint{2.984051in}{0.944289in}}%
\pgfpathlineto{\pgfqpoint{3.005840in}{0.971004in}}%
\pgfpathlineto{\pgfqpoint{3.010939in}{0.970937in}}%
\pgfpathlineto{\pgfqpoint{3.018836in}{0.966601in}}%
\pgfpathlineto{\pgfqpoint{3.034445in}{0.961587in}}%
\pgfpathlineto{\pgfqpoint{3.047350in}{0.952216in}}%
\pgfpathlineto{\pgfqpoint{3.030466in}{0.937480in}}%
\pgfpathlineto{\pgfqpoint{3.038279in}{0.928637in}}%
\pgfpathlineto{\pgfqpoint{3.044811in}{0.924182in}}%
\pgfpathlineto{\pgfqpoint{3.051321in}{0.914811in}}%
\pgfpathclose%
\pgfusepath{fill}%
\end{pgfscope}%
\begin{pgfscope}%
\pgfpathrectangle{\pgfqpoint{0.100000in}{0.100000in}}{\pgfqpoint{3.608454in}{2.310000in}}%
\pgfusepath{clip}%
\pgfsetbuttcap%
\pgfsetmiterjoin%
\definecolor{currentfill}{rgb}{0.000000,0.478431,0.760784}%
\pgfsetfillcolor{currentfill}%
\pgfsetlinewidth{0.000000pt}%
\definecolor{currentstroke}{rgb}{0.000000,0.000000,0.000000}%
\pgfsetstrokecolor{currentstroke}%
\pgfsetstrokeopacity{0.000000}%
\pgfsetdash{}{0pt}%
\pgfpathmoveto{\pgfqpoint{1.046091in}{0.176022in}}%
\pgfpathlineto{\pgfqpoint{1.046239in}{0.179505in}}%
\pgfpathlineto{\pgfqpoint{1.048650in}{0.177612in}}%
\pgfpathlineto{\pgfqpoint{1.046091in}{0.176022in}}%
\pgfpathclose%
\pgfusepath{fill}%
\end{pgfscope}%
\begin{pgfscope}%
\pgfpathrectangle{\pgfqpoint{0.100000in}{0.100000in}}{\pgfqpoint{3.608454in}{2.310000in}}%
\pgfusepath{clip}%
\pgfsetbuttcap%
\pgfsetmiterjoin%
\definecolor{currentfill}{rgb}{0.000000,0.478431,0.760784}%
\pgfsetfillcolor{currentfill}%
\pgfsetlinewidth{0.000000pt}%
\definecolor{currentstroke}{rgb}{0.000000,0.000000,0.000000}%
\pgfsetstrokecolor{currentstroke}%
\pgfsetstrokeopacity{0.000000}%
\pgfsetdash{}{0pt}%
\pgfpathmoveto{\pgfqpoint{1.042447in}{0.162554in}}%
\pgfpathlineto{\pgfqpoint{1.043313in}{0.165192in}}%
\pgfpathlineto{\pgfqpoint{1.044966in}{0.165456in}}%
\pgfpathlineto{\pgfqpoint{1.044338in}{0.161816in}}%
\pgfpathlineto{\pgfqpoint{1.042447in}{0.162554in}}%
\pgfpathclose%
\pgfusepath{fill}%
\end{pgfscope}%
\begin{pgfscope}%
\pgfpathrectangle{\pgfqpoint{0.100000in}{0.100000in}}{\pgfqpoint{3.608454in}{2.310000in}}%
\pgfusepath{clip}%
\pgfsetbuttcap%
\pgfsetmiterjoin%
\definecolor{currentfill}{rgb}{0.000000,0.478431,0.760784}%
\pgfsetfillcolor{currentfill}%
\pgfsetlinewidth{0.000000pt}%
\definecolor{currentstroke}{rgb}{0.000000,0.000000,0.000000}%
\pgfsetstrokecolor{currentstroke}%
\pgfsetstrokeopacity{0.000000}%
\pgfsetdash{}{0pt}%
\pgfpathmoveto{\pgfqpoint{1.045943in}{0.145022in}}%
\pgfpathlineto{\pgfqpoint{1.044467in}{0.150857in}}%
\pgfpathlineto{\pgfqpoint{1.045363in}{0.152527in}}%
\pgfpathlineto{\pgfqpoint{1.043425in}{0.154046in}}%
\pgfpathlineto{\pgfqpoint{1.045163in}{0.157939in}}%
\pgfpathlineto{\pgfqpoint{1.045050in}{0.160374in}}%
\pgfpathlineto{\pgfqpoint{1.047193in}{0.161279in}}%
\pgfpathlineto{\pgfqpoint{1.047667in}{0.157760in}}%
\pgfpathlineto{\pgfqpoint{1.045744in}{0.156113in}}%
\pgfpathlineto{\pgfqpoint{1.046983in}{0.154946in}}%
\pgfpathlineto{\pgfqpoint{1.047011in}{0.149220in}}%
\pgfpathlineto{\pgfqpoint{1.049976in}{0.148483in}}%
\pgfpathlineto{\pgfqpoint{1.045943in}{0.145022in}}%
\pgfpathclose%
\pgfusepath{fill}%
\end{pgfscope}%
\begin{pgfscope}%
\pgfpathrectangle{\pgfqpoint{0.100000in}{0.100000in}}{\pgfqpoint{3.608454in}{2.310000in}}%
\pgfusepath{clip}%
\pgfsetbuttcap%
\pgfsetmiterjoin%
\definecolor{currentfill}{rgb}{0.000000,0.478431,0.760784}%
\pgfsetfillcolor{currentfill}%
\pgfsetlinewidth{0.000000pt}%
\definecolor{currentstroke}{rgb}{0.000000,0.000000,0.000000}%
\pgfsetstrokecolor{currentstroke}%
\pgfsetstrokeopacity{0.000000}%
\pgfsetdash{}{0pt}%
\pgfpathmoveto{\pgfqpoint{1.071810in}{0.150553in}}%
\pgfpathlineto{\pgfqpoint{1.072025in}{0.147586in}}%
\pgfpathlineto{\pgfqpoint{1.070746in}{0.144824in}}%
\pgfpathlineto{\pgfqpoint{1.069328in}{0.144797in}}%
\pgfpathlineto{\pgfqpoint{1.068043in}{0.147923in}}%
\pgfpathlineto{\pgfqpoint{1.069455in}{0.148655in}}%
\pgfpathlineto{\pgfqpoint{1.069622in}{0.151729in}}%
\pgfpathlineto{\pgfqpoint{1.071810in}{0.150553in}}%
\pgfpathclose%
\pgfusepath{fill}%
\end{pgfscope}%
\begin{pgfscope}%
\pgfpathrectangle{\pgfqpoint{0.100000in}{0.100000in}}{\pgfqpoint{3.608454in}{2.310000in}}%
\pgfusepath{clip}%
\pgfsetbuttcap%
\pgfsetmiterjoin%
\definecolor{currentfill}{rgb}{0.000000,0.478431,0.760784}%
\pgfsetfillcolor{currentfill}%
\pgfsetlinewidth{0.000000pt}%
\definecolor{currentstroke}{rgb}{0.000000,0.000000,0.000000}%
\pgfsetstrokecolor{currentstroke}%
\pgfsetstrokeopacity{0.000000}%
\pgfsetdash{}{0pt}%
\pgfpathmoveto{\pgfqpoint{1.065482in}{0.169372in}}%
\pgfpathlineto{\pgfqpoint{1.067405in}{0.165345in}}%
\pgfpathlineto{\pgfqpoint{1.069320in}{0.165633in}}%
\pgfpathlineto{\pgfqpoint{1.070834in}{0.168040in}}%
\pgfpathlineto{\pgfqpoint{1.070260in}{0.170374in}}%
\pgfpathlineto{\pgfqpoint{1.072944in}{0.170978in}}%
\pgfpathlineto{\pgfqpoint{1.073238in}{0.172315in}}%
\pgfpathlineto{\pgfqpoint{1.075481in}{0.173524in}}%
\pgfpathlineto{\pgfqpoint{1.078315in}{0.173884in}}%
\pgfpathlineto{\pgfqpoint{1.079838in}{0.170598in}}%
\pgfpathlineto{\pgfqpoint{1.082038in}{0.171601in}}%
\pgfpathlineto{\pgfqpoint{1.081805in}{0.173502in}}%
\pgfpathlineto{\pgfqpoint{1.085139in}{0.174080in}}%
\pgfpathlineto{\pgfqpoint{1.087857in}{0.173091in}}%
\pgfpathlineto{\pgfqpoint{1.089922in}{0.175542in}}%
\pgfpathlineto{\pgfqpoint{1.090049in}{0.178164in}}%
\pgfpathlineto{\pgfqpoint{1.093019in}{0.176002in}}%
\pgfpathlineto{\pgfqpoint{1.096013in}{0.170850in}}%
\pgfpathlineto{\pgfqpoint{1.096478in}{0.167420in}}%
\pgfpathlineto{\pgfqpoint{1.098614in}{0.165154in}}%
\pgfpathlineto{\pgfqpoint{1.100805in}{0.164982in}}%
\pgfpathlineto{\pgfqpoint{1.101142in}{0.162059in}}%
\pgfpathlineto{\pgfqpoint{1.100030in}{0.159422in}}%
\pgfpathlineto{\pgfqpoint{1.096111in}{0.156382in}}%
\pgfpathlineto{\pgfqpoint{1.095842in}{0.153748in}}%
\pgfpathlineto{\pgfqpoint{1.094580in}{0.151215in}}%
\pgfpathlineto{\pgfqpoint{1.094242in}{0.146717in}}%
\pgfpathlineto{\pgfqpoint{1.093131in}{0.142662in}}%
\pgfpathlineto{\pgfqpoint{1.090632in}{0.141147in}}%
\pgfpathlineto{\pgfqpoint{1.084308in}{0.135553in}}%
\pgfpathlineto{\pgfqpoint{1.081972in}{0.136078in}}%
\pgfpathlineto{\pgfqpoint{1.079389in}{0.134752in}}%
\pgfpathlineto{\pgfqpoint{1.077089in}{0.136527in}}%
\pgfpathlineto{\pgfqpoint{1.074765in}{0.136279in}}%
\pgfpathlineto{\pgfqpoint{1.076502in}{0.144158in}}%
\pgfpathlineto{\pgfqpoint{1.076368in}{0.146457in}}%
\pgfpathlineto{\pgfqpoint{1.079906in}{0.148920in}}%
\pgfpathlineto{\pgfqpoint{1.081825in}{0.148421in}}%
\pgfpathlineto{\pgfqpoint{1.081856in}{0.151528in}}%
\pgfpathlineto{\pgfqpoint{1.083827in}{0.155130in}}%
\pgfpathlineto{\pgfqpoint{1.085699in}{0.162221in}}%
\pgfpathlineto{\pgfqpoint{1.083172in}{0.161819in}}%
\pgfpathlineto{\pgfqpoint{1.083697in}{0.159212in}}%
\pgfpathlineto{\pgfqpoint{1.082643in}{0.155646in}}%
\pgfpathlineto{\pgfqpoint{1.078109in}{0.149553in}}%
\pgfpathlineto{\pgfqpoint{1.074518in}{0.148168in}}%
\pgfpathlineto{\pgfqpoint{1.073929in}{0.150391in}}%
\pgfpathlineto{\pgfqpoint{1.070564in}{0.153361in}}%
\pgfpathlineto{\pgfqpoint{1.067603in}{0.152672in}}%
\pgfpathlineto{\pgfqpoint{1.065718in}{0.151421in}}%
\pgfpathlineto{\pgfqpoint{1.067029in}{0.156508in}}%
\pgfpathlineto{\pgfqpoint{1.068970in}{0.159001in}}%
\pgfpathlineto{\pgfqpoint{1.070918in}{0.159489in}}%
\pgfpathlineto{\pgfqpoint{1.072200in}{0.162116in}}%
\pgfpathlineto{\pgfqpoint{1.074427in}{0.163792in}}%
\pgfpathlineto{\pgfqpoint{1.074481in}{0.165957in}}%
\pgfpathlineto{\pgfqpoint{1.077573in}{0.166662in}}%
\pgfpathlineto{\pgfqpoint{1.077538in}{0.168796in}}%
\pgfpathlineto{\pgfqpoint{1.074777in}{0.170290in}}%
\pgfpathlineto{\pgfqpoint{1.071926in}{0.169175in}}%
\pgfpathlineto{\pgfqpoint{1.073508in}{0.167354in}}%
\pgfpathlineto{\pgfqpoint{1.070989in}{0.164095in}}%
\pgfpathlineto{\pgfqpoint{1.067088in}{0.160807in}}%
\pgfpathlineto{\pgfqpoint{1.065060in}{0.165053in}}%
\pgfpathlineto{\pgfqpoint{1.065482in}{0.169372in}}%
\pgfpathclose%
\pgfusepath{fill}%
\end{pgfscope}%
\begin{pgfscope}%
\pgfpathrectangle{\pgfqpoint{0.100000in}{0.100000in}}{\pgfqpoint{3.608454in}{2.310000in}}%
\pgfusepath{clip}%
\pgfsetbuttcap%
\pgfsetmiterjoin%
\definecolor{currentfill}{rgb}{0.000000,0.478431,0.760784}%
\pgfsetfillcolor{currentfill}%
\pgfsetlinewidth{0.000000pt}%
\definecolor{currentstroke}{rgb}{0.000000,0.000000,0.000000}%
\pgfsetstrokecolor{currentstroke}%
\pgfsetstrokeopacity{0.000000}%
\pgfsetdash{}{0pt}%
\pgfpathmoveto{\pgfqpoint{1.041943in}{0.169830in}}%
\pgfpathlineto{\pgfqpoint{1.039521in}{0.170855in}}%
\pgfpathlineto{\pgfqpoint{1.042951in}{0.172853in}}%
\pgfpathlineto{\pgfqpoint{1.043959in}{0.170752in}}%
\pgfpathlineto{\pgfqpoint{1.047379in}{0.171633in}}%
\pgfpathlineto{\pgfqpoint{1.048310in}{0.169620in}}%
\pgfpathlineto{\pgfqpoint{1.045621in}{0.169580in}}%
\pgfpathlineto{\pgfqpoint{1.044188in}{0.167269in}}%
\pgfpathlineto{\pgfqpoint{1.041932in}{0.166429in}}%
\pgfpathlineto{\pgfqpoint{1.040306in}{0.167304in}}%
\pgfpathlineto{\pgfqpoint{1.041943in}{0.169830in}}%
\pgfpathclose%
\pgfusepath{fill}%
\end{pgfscope}%
\begin{pgfscope}%
\pgfpathrectangle{\pgfqpoint{0.100000in}{0.100000in}}{\pgfqpoint{3.608454in}{2.310000in}}%
\pgfusepath{clip}%
\pgfsetbuttcap%
\pgfsetmiterjoin%
\definecolor{currentfill}{rgb}{0.000000,0.478431,0.760784}%
\pgfsetfillcolor{currentfill}%
\pgfsetlinewidth{0.000000pt}%
\definecolor{currentstroke}{rgb}{0.000000,0.000000,0.000000}%
\pgfsetstrokecolor{currentstroke}%
\pgfsetstrokeopacity{0.000000}%
\pgfsetdash{}{0pt}%
\pgfpathmoveto{\pgfqpoint{1.051425in}{0.157660in}}%
\pgfpathlineto{\pgfqpoint{1.051517in}{0.160010in}}%
\pgfpathlineto{\pgfqpoint{1.046502in}{0.163773in}}%
\pgfpathlineto{\pgfqpoint{1.050660in}{0.165847in}}%
\pgfpathlineto{\pgfqpoint{1.049847in}{0.167856in}}%
\pgfpathlineto{\pgfqpoint{1.050846in}{0.170035in}}%
\pgfpathlineto{\pgfqpoint{1.048993in}{0.170836in}}%
\pgfpathlineto{\pgfqpoint{1.047832in}{0.173344in}}%
\pgfpathlineto{\pgfqpoint{1.051430in}{0.175827in}}%
\pgfpathlineto{\pgfqpoint{1.051845in}{0.178802in}}%
\pgfpathlineto{\pgfqpoint{1.053869in}{0.179753in}}%
\pgfpathlineto{\pgfqpoint{1.053809in}{0.182723in}}%
\pgfpathlineto{\pgfqpoint{1.051267in}{0.183712in}}%
\pgfpathlineto{\pgfqpoint{1.045462in}{0.182905in}}%
\pgfpathlineto{\pgfqpoint{1.047279in}{0.185118in}}%
\pgfpathlineto{\pgfqpoint{1.049511in}{0.186205in}}%
\pgfpathlineto{\pgfqpoint{1.052044in}{0.186328in}}%
\pgfpathlineto{\pgfqpoint{1.051609in}{0.190204in}}%
\pgfpathlineto{\pgfqpoint{1.052862in}{0.192676in}}%
\pgfpathlineto{\pgfqpoint{1.054788in}{0.191850in}}%
\pgfpathlineto{\pgfqpoint{1.059113in}{0.188805in}}%
\pgfpathlineto{\pgfqpoint{1.059157in}{0.184625in}}%
\pgfpathlineto{\pgfqpoint{1.057111in}{0.183014in}}%
\pgfpathlineto{\pgfqpoint{1.057987in}{0.179356in}}%
\pgfpathlineto{\pgfqpoint{1.059495in}{0.179632in}}%
\pgfpathlineto{\pgfqpoint{1.061360in}{0.177220in}}%
\pgfpathlineto{\pgfqpoint{1.062648in}{0.172710in}}%
\pgfpathlineto{\pgfqpoint{1.062360in}{0.170277in}}%
\pgfpathlineto{\pgfqpoint{1.060929in}{0.168990in}}%
\pgfpathlineto{\pgfqpoint{1.062884in}{0.167409in}}%
\pgfpathlineto{\pgfqpoint{1.062703in}{0.164914in}}%
\pgfpathlineto{\pgfqpoint{1.060871in}{0.164522in}}%
\pgfpathlineto{\pgfqpoint{1.059752in}{0.166916in}}%
\pgfpathlineto{\pgfqpoint{1.058308in}{0.164637in}}%
\pgfpathlineto{\pgfqpoint{1.061166in}{0.162445in}}%
\pgfpathlineto{\pgfqpoint{1.061172in}{0.160447in}}%
\pgfpathlineto{\pgfqpoint{1.062512in}{0.157360in}}%
\pgfpathlineto{\pgfqpoint{1.060934in}{0.153696in}}%
\pgfpathlineto{\pgfqpoint{1.064018in}{0.154722in}}%
\pgfpathlineto{\pgfqpoint{1.063461in}{0.152526in}}%
\pgfpathlineto{\pgfqpoint{1.061554in}{0.151190in}}%
\pgfpathlineto{\pgfqpoint{1.060026in}{0.151947in}}%
\pgfpathlineto{\pgfqpoint{1.058945in}{0.148455in}}%
\pgfpathlineto{\pgfqpoint{1.061641in}{0.148773in}}%
\pgfpathlineto{\pgfqpoint{1.059952in}{0.145277in}}%
\pgfpathlineto{\pgfqpoint{1.059401in}{0.142596in}}%
\pgfpathlineto{\pgfqpoint{1.057983in}{0.140914in}}%
\pgfpathlineto{\pgfqpoint{1.055057in}{0.141929in}}%
\pgfpathlineto{\pgfqpoint{1.054548in}{0.147234in}}%
\pgfpathlineto{\pgfqpoint{1.052684in}{0.150955in}}%
\pgfpathlineto{\pgfqpoint{1.052499in}{0.153302in}}%
\pgfpathlineto{\pgfqpoint{1.054358in}{0.157385in}}%
\pgfpathlineto{\pgfqpoint{1.051425in}{0.157660in}}%
\pgfpathclose%
\pgfusepath{fill}%
\end{pgfscope}%
\begin{pgfscope}%
\pgfpathrectangle{\pgfqpoint{0.100000in}{0.100000in}}{\pgfqpoint{3.608454in}{2.310000in}}%
\pgfusepath{clip}%
\pgfsetbuttcap%
\pgfsetmiterjoin%
\definecolor{currentfill}{rgb}{0.000000,0.682353,0.658824}%
\pgfsetfillcolor{currentfill}%
\pgfsetlinewidth{0.000000pt}%
\definecolor{currentstroke}{rgb}{0.000000,0.000000,0.000000}%
\pgfsetstrokecolor{currentstroke}%
\pgfsetstrokeopacity{0.000000}%
\pgfsetdash{}{0pt}%
\pgfpathmoveto{\pgfqpoint{0.875234in}{2.151233in}}%
\pgfpathlineto{\pgfqpoint{0.868405in}{2.124498in}}%
\pgfpathlineto{\pgfqpoint{0.863100in}{2.115696in}}%
\pgfpathlineto{\pgfqpoint{0.851613in}{2.114354in}}%
\pgfpathlineto{\pgfqpoint{0.823108in}{2.121596in}}%
\pgfpathlineto{\pgfqpoint{0.790335in}{2.130635in}}%
\pgfpathlineto{\pgfqpoint{0.784697in}{2.127385in}}%
\pgfpathlineto{\pgfqpoint{0.782427in}{2.131997in}}%
\pgfpathlineto{\pgfqpoint{0.774845in}{2.127276in}}%
\pgfpathlineto{\pgfqpoint{0.761418in}{2.129669in}}%
\pgfpathlineto{\pgfqpoint{0.758684in}{2.134138in}}%
\pgfpathlineto{\pgfqpoint{0.762232in}{2.143437in}}%
\pgfpathlineto{\pgfqpoint{0.763393in}{2.155197in}}%
\pgfpathlineto{\pgfqpoint{0.762473in}{2.166902in}}%
\pgfpathlineto{\pgfqpoint{0.767233in}{2.177075in}}%
\pgfpathlineto{\pgfqpoint{0.769524in}{2.184730in}}%
\pgfpathlineto{\pgfqpoint{0.774962in}{2.186885in}}%
\pgfpathlineto{\pgfqpoint{0.778779in}{2.191862in}}%
\pgfpathlineto{\pgfqpoint{0.796683in}{2.187200in}}%
\pgfpathlineto{\pgfqpoint{0.798560in}{2.193839in}}%
\pgfpathlineto{\pgfqpoint{0.807006in}{2.198546in}}%
\pgfpathlineto{\pgfqpoint{0.810303in}{2.197629in}}%
\pgfpathlineto{\pgfqpoint{0.814986in}{2.205959in}}%
\pgfpathlineto{\pgfqpoint{0.820986in}{2.210320in}}%
\pgfpathlineto{\pgfqpoint{0.822836in}{2.217028in}}%
\pgfpathlineto{\pgfqpoint{0.829989in}{2.221231in}}%
\pgfpathlineto{\pgfqpoint{0.836288in}{2.219545in}}%
\pgfpathlineto{\pgfqpoint{0.836363in}{2.217995in}}%
\pgfpathlineto{\pgfqpoint{0.821826in}{2.165426in}}%
\pgfpathlineto{\pgfqpoint{0.875234in}{2.151233in}}%
\pgfpathclose%
\pgfusepath{fill}%
\end{pgfscope}%
\begin{pgfscope}%
\pgfpathrectangle{\pgfqpoint{0.100000in}{0.100000in}}{\pgfqpoint{3.608454in}{2.310000in}}%
\pgfusepath{clip}%
\pgfsetbuttcap%
\pgfsetmiterjoin%
\definecolor{currentfill}{rgb}{0.000000,0.709804,0.645098}%
\pgfsetfillcolor{currentfill}%
\pgfsetlinewidth{0.000000pt}%
\definecolor{currentstroke}{rgb}{0.000000,0.000000,0.000000}%
\pgfsetstrokecolor{currentstroke}%
\pgfsetstrokeopacity{0.000000}%
\pgfsetdash{}{0pt}%
\pgfpathmoveto{\pgfqpoint{2.966480in}{1.429783in}}%
\pgfpathlineto{\pgfqpoint{2.962415in}{1.436530in}}%
\pgfpathlineto{\pgfqpoint{2.951761in}{1.443677in}}%
\pgfpathlineto{\pgfqpoint{2.954171in}{1.450091in}}%
\pgfpathlineto{\pgfqpoint{2.941441in}{1.448688in}}%
\pgfpathlineto{\pgfqpoint{2.938939in}{1.450447in}}%
\pgfpathlineto{\pgfqpoint{2.935564in}{1.458054in}}%
\pgfpathlineto{\pgfqpoint{2.933975in}{1.470774in}}%
\pgfpathlineto{\pgfqpoint{2.938569in}{1.471342in}}%
\pgfpathlineto{\pgfqpoint{2.935644in}{1.495065in}}%
\pgfpathlineto{\pgfqpoint{2.956251in}{1.497406in}}%
\pgfpathlineto{\pgfqpoint{2.977955in}{1.500863in}}%
\pgfpathlineto{\pgfqpoint{2.983438in}{1.466654in}}%
\pgfpathlineto{\pgfqpoint{2.989320in}{1.467586in}}%
\pgfpathlineto{\pgfqpoint{2.992195in}{1.461918in}}%
\pgfpathlineto{\pgfqpoint{2.986701in}{1.454490in}}%
\pgfpathlineto{\pgfqpoint{2.989302in}{1.450132in}}%
\pgfpathlineto{\pgfqpoint{2.987075in}{1.444310in}}%
\pgfpathlineto{\pgfqpoint{2.980855in}{1.444545in}}%
\pgfpathlineto{\pgfqpoint{2.966480in}{1.429783in}}%
\pgfpathclose%
\pgfusepath{fill}%
\end{pgfscope}%
\begin{pgfscope}%
\pgfpathrectangle{\pgfqpoint{0.100000in}{0.100000in}}{\pgfqpoint{3.608454in}{2.310000in}}%
\pgfusepath{clip}%
\pgfsetbuttcap%
\pgfsetmiterjoin%
\definecolor{currentfill}{rgb}{0.000000,0.411765,0.794118}%
\pgfsetfillcolor{currentfill}%
\pgfsetlinewidth{0.000000pt}%
\definecolor{currentstroke}{rgb}{0.000000,0.000000,0.000000}%
\pgfsetstrokecolor{currentstroke}%
\pgfsetstrokeopacity{0.000000}%
\pgfsetdash{}{0pt}%
\pgfpathmoveto{\pgfqpoint{3.466647in}{1.697217in}}%
\pgfpathlineto{\pgfqpoint{3.468281in}{1.705449in}}%
\pgfpathlineto{\pgfqpoint{3.464103in}{1.723936in}}%
\pgfpathlineto{\pgfqpoint{3.456367in}{1.750710in}}%
\pgfpathlineto{\pgfqpoint{3.479909in}{1.757791in}}%
\pgfpathlineto{\pgfqpoint{3.481572in}{1.755490in}}%
\pgfpathlineto{\pgfqpoint{3.495434in}{1.768247in}}%
\pgfpathlineto{\pgfqpoint{3.499874in}{1.758602in}}%
\pgfpathlineto{\pgfqpoint{3.513375in}{1.745585in}}%
\pgfpathlineto{\pgfqpoint{3.518613in}{1.736187in}}%
\pgfpathlineto{\pgfqpoint{3.513987in}{1.733317in}}%
\pgfpathlineto{\pgfqpoint{3.514812in}{1.727935in}}%
\pgfpathlineto{\pgfqpoint{3.510151in}{1.723018in}}%
\pgfpathlineto{\pgfqpoint{3.494576in}{1.717460in}}%
\pgfpathlineto{\pgfqpoint{3.494096in}{1.728843in}}%
\pgfpathlineto{\pgfqpoint{3.490331in}{1.736004in}}%
\pgfpathlineto{\pgfqpoint{3.485585in}{1.730146in}}%
\pgfpathlineto{\pgfqpoint{3.485574in}{1.722330in}}%
\pgfpathlineto{\pgfqpoint{3.488922in}{1.715480in}}%
\pgfpathlineto{\pgfqpoint{3.487219in}{1.706707in}}%
\pgfpathlineto{\pgfqpoint{3.482998in}{1.705766in}}%
\pgfpathlineto{\pgfqpoint{3.466647in}{1.697217in}}%
\pgfpathclose%
\pgfusepath{fill}%
\end{pgfscope}%
\begin{pgfscope}%
\pgfpathrectangle{\pgfqpoint{0.100000in}{0.100000in}}{\pgfqpoint{3.608454in}{2.310000in}}%
\pgfusepath{clip}%
\pgfsetbuttcap%
\pgfsetmiterjoin%
\definecolor{currentfill}{rgb}{0.000000,0.458824,0.770588}%
\pgfsetfillcolor{currentfill}%
\pgfsetlinewidth{0.000000pt}%
\definecolor{currentstroke}{rgb}{0.000000,0.000000,0.000000}%
\pgfsetstrokecolor{currentstroke}%
\pgfsetstrokeopacity{0.000000}%
\pgfsetdash{}{0pt}%
\pgfpathmoveto{\pgfqpoint{2.148148in}{1.837483in}}%
\pgfpathlineto{\pgfqpoint{2.141258in}{1.840834in}}%
\pgfpathlineto{\pgfqpoint{2.120021in}{1.840682in}}%
\pgfpathlineto{\pgfqpoint{2.120016in}{1.847577in}}%
\pgfpathlineto{\pgfqpoint{2.099439in}{1.847533in}}%
\pgfpathlineto{\pgfqpoint{2.099037in}{1.876260in}}%
\pgfpathlineto{\pgfqpoint{2.147223in}{1.876558in}}%
\pgfpathlineto{\pgfqpoint{2.142690in}{1.880249in}}%
\pgfpathlineto{\pgfqpoint{2.175363in}{1.880599in}}%
\pgfpathlineto{\pgfqpoint{2.174916in}{1.893366in}}%
\pgfpathlineto{\pgfqpoint{2.172623in}{1.893374in}}%
\pgfpathlineto{\pgfqpoint{2.172440in}{1.907196in}}%
\pgfpathlineto{\pgfqpoint{2.172320in}{1.914091in}}%
\pgfpathlineto{\pgfqpoint{2.193274in}{1.914649in}}%
\pgfpathlineto{\pgfqpoint{2.193329in}{1.907307in}}%
\pgfpathlineto{\pgfqpoint{2.193529in}{1.893561in}}%
\pgfpathlineto{\pgfqpoint{2.188794in}{1.893513in}}%
\pgfpathlineto{\pgfqpoint{2.189353in}{1.873824in}}%
\pgfpathlineto{\pgfqpoint{2.190082in}{1.834950in}}%
\pgfpathlineto{\pgfqpoint{2.186149in}{1.838564in}}%
\pgfpathlineto{\pgfqpoint{2.176492in}{1.838639in}}%
\pgfpathlineto{\pgfqpoint{2.160005in}{1.848588in}}%
\pgfpathlineto{\pgfqpoint{2.156256in}{1.841131in}}%
\pgfpathlineto{\pgfqpoint{2.148148in}{1.837483in}}%
\pgfpathclose%
\pgfusepath{fill}%
\end{pgfscope}%
\begin{pgfscope}%
\pgfpathrectangle{\pgfqpoint{0.100000in}{0.100000in}}{\pgfqpoint{3.608454in}{2.310000in}}%
\pgfusepath{clip}%
\pgfsetbuttcap%
\pgfsetmiterjoin%
\definecolor{currentfill}{rgb}{0.000000,0.678431,0.660784}%
\pgfsetfillcolor{currentfill}%
\pgfsetlinewidth{0.000000pt}%
\definecolor{currentstroke}{rgb}{0.000000,0.000000,0.000000}%
\pgfsetstrokecolor{currentstroke}%
\pgfsetstrokeopacity{0.000000}%
\pgfsetdash{}{0pt}%
\pgfpathmoveto{\pgfqpoint{1.141273in}{1.664762in}}%
\pgfpathlineto{\pgfqpoint{1.141528in}{1.670743in}}%
\pgfpathlineto{\pgfqpoint{1.137582in}{1.673009in}}%
\pgfpathlineto{\pgfqpoint{1.137407in}{1.682303in}}%
\pgfpathlineto{\pgfqpoint{1.141935in}{1.686685in}}%
\pgfpathlineto{\pgfqpoint{1.143732in}{1.692681in}}%
\pgfpathlineto{\pgfqpoint{1.141976in}{1.698141in}}%
\pgfpathlineto{\pgfqpoint{1.126068in}{1.701146in}}%
\pgfpathlineto{\pgfqpoint{1.126636in}{1.717223in}}%
\pgfpathlineto{\pgfqpoint{1.125111in}{1.725638in}}%
\pgfpathlineto{\pgfqpoint{1.117055in}{1.725647in}}%
\pgfpathlineto{\pgfqpoint{1.116145in}{1.733195in}}%
\pgfpathlineto{\pgfqpoint{1.117477in}{1.741580in}}%
\pgfpathlineto{\pgfqpoint{1.124259in}{1.750336in}}%
\pgfpathlineto{\pgfqpoint{1.182268in}{1.739464in}}%
\pgfpathlineto{\pgfqpoint{1.167956in}{1.660095in}}%
\pgfpathlineto{\pgfqpoint{1.141273in}{1.664762in}}%
\pgfpathclose%
\pgfusepath{fill}%
\end{pgfscope}%
\begin{pgfscope}%
\pgfpathrectangle{\pgfqpoint{0.100000in}{0.100000in}}{\pgfqpoint{3.608454in}{2.310000in}}%
\pgfusepath{clip}%
\pgfsetbuttcap%
\pgfsetmiterjoin%
\definecolor{currentfill}{rgb}{0.000000,0.713725,0.643137}%
\pgfsetfillcolor{currentfill}%
\pgfsetlinewidth{0.000000pt}%
\definecolor{currentstroke}{rgb}{0.000000,0.000000,0.000000}%
\pgfsetstrokecolor{currentstroke}%
\pgfsetstrokeopacity{0.000000}%
\pgfsetdash{}{0pt}%
\pgfpathmoveto{\pgfqpoint{2.193249in}{1.204475in}}%
\pgfpathlineto{\pgfqpoint{2.166408in}{1.204778in}}%
\pgfpathlineto{\pgfqpoint{2.164491in}{1.209439in}}%
\pgfpathlineto{\pgfqpoint{2.130830in}{1.210394in}}%
\pgfpathlineto{\pgfqpoint{2.130827in}{1.263573in}}%
\pgfpathlineto{\pgfqpoint{2.130815in}{1.265402in}}%
\pgfpathlineto{\pgfqpoint{2.154102in}{1.264813in}}%
\pgfpathlineto{\pgfqpoint{2.165424in}{1.263791in}}%
\pgfpathlineto{\pgfqpoint{2.165075in}{1.253073in}}%
\pgfpathlineto{\pgfqpoint{2.181129in}{1.252610in}}%
\pgfpathlineto{\pgfqpoint{2.180991in}{1.248016in}}%
\pgfpathlineto{\pgfqpoint{2.192435in}{1.247763in}}%
\pgfpathlineto{\pgfqpoint{2.193719in}{1.240848in}}%
\pgfpathlineto{\pgfqpoint{2.193249in}{1.204475in}}%
\pgfpathclose%
\pgfusepath{fill}%
\end{pgfscope}%
\begin{pgfscope}%
\pgfpathrectangle{\pgfqpoint{0.100000in}{0.100000in}}{\pgfqpoint{3.608454in}{2.310000in}}%
\pgfusepath{clip}%
\pgfsetbuttcap%
\pgfsetmiterjoin%
\definecolor{currentfill}{rgb}{0.000000,0.552941,0.723529}%
\pgfsetfillcolor{currentfill}%
\pgfsetlinewidth{0.000000pt}%
\definecolor{currentstroke}{rgb}{0.000000,0.000000,0.000000}%
\pgfsetstrokecolor{currentstroke}%
\pgfsetstrokeopacity{0.000000}%
\pgfsetdash{}{0pt}%
\pgfpathmoveto{\pgfqpoint{3.097477in}{1.148400in}}%
\pgfpathlineto{\pgfqpoint{3.057775in}{1.140860in}}%
\pgfpathlineto{\pgfqpoint{3.055178in}{1.146538in}}%
\pgfpathlineto{\pgfqpoint{3.046178in}{1.152956in}}%
\pgfpathlineto{\pgfqpoint{3.037191in}{1.156638in}}%
\pgfpathlineto{\pgfqpoint{3.035098in}{1.163147in}}%
\pgfpathlineto{\pgfqpoint{3.039762in}{1.164964in}}%
\pgfpathlineto{\pgfqpoint{3.039491in}{1.174669in}}%
\pgfpathlineto{\pgfqpoint{3.049089in}{1.181346in}}%
\pgfpathlineto{\pgfqpoint{3.060084in}{1.181854in}}%
\pgfpathlineto{\pgfqpoint{3.054182in}{1.223554in}}%
\pgfpathlineto{\pgfqpoint{3.073773in}{1.226802in}}%
\pgfpathlineto{\pgfqpoint{3.074241in}{1.252094in}}%
\pgfpathlineto{\pgfqpoint{3.074664in}{1.267099in}}%
\pgfpathlineto{\pgfqpoint{3.081510in}{1.270060in}}%
\pgfpathlineto{\pgfqpoint{3.089238in}{1.276542in}}%
\pgfpathlineto{\pgfqpoint{3.095309in}{1.277259in}}%
\pgfpathlineto{\pgfqpoint{3.099433in}{1.272390in}}%
\pgfpathlineto{\pgfqpoint{3.105607in}{1.274029in}}%
\pgfpathlineto{\pgfqpoint{3.104858in}{1.232143in}}%
\pgfpathlineto{\pgfqpoint{3.109908in}{1.233039in}}%
\pgfpathlineto{\pgfqpoint{3.113032in}{1.209517in}}%
\pgfpathlineto{\pgfqpoint{3.106519in}{1.208514in}}%
\pgfpathlineto{\pgfqpoint{3.110627in}{1.182204in}}%
\pgfpathlineto{\pgfqpoint{3.113209in}{1.177564in}}%
\pgfpathlineto{\pgfqpoint{3.093928in}{1.174159in}}%
\pgfpathlineto{\pgfqpoint{3.097477in}{1.148400in}}%
\pgfpathclose%
\pgfusepath{fill}%
\end{pgfscope}%
\begin{pgfscope}%
\pgfpathrectangle{\pgfqpoint{0.100000in}{0.100000in}}{\pgfqpoint{3.608454in}{2.310000in}}%
\pgfusepath{clip}%
\pgfsetbuttcap%
\pgfsetmiterjoin%
\definecolor{currentfill}{rgb}{0.000000,0.490196,0.754902}%
\pgfsetfillcolor{currentfill}%
\pgfsetlinewidth{0.000000pt}%
\definecolor{currentstroke}{rgb}{0.000000,0.000000,0.000000}%
\pgfsetstrokecolor{currentstroke}%
\pgfsetstrokeopacity{0.000000}%
\pgfsetdash{}{0pt}%
\pgfpathmoveto{\pgfqpoint{1.380303in}{0.447788in}}%
\pgfpathlineto{\pgfqpoint{1.383896in}{0.457637in}}%
\pgfpathlineto{\pgfqpoint{1.384080in}{0.463783in}}%
\pgfpathlineto{\pgfqpoint{1.393078in}{0.457489in}}%
\pgfpathlineto{\pgfqpoint{1.394896in}{0.449171in}}%
\pgfpathlineto{\pgfqpoint{1.389591in}{0.443421in}}%
\pgfpathlineto{\pgfqpoint{1.380303in}{0.447788in}}%
\pgfpathclose%
\pgfusepath{fill}%
\end{pgfscope}%
\begin{pgfscope}%
\pgfpathrectangle{\pgfqpoint{0.100000in}{0.100000in}}{\pgfqpoint{3.608454in}{2.310000in}}%
\pgfusepath{clip}%
\pgfsetbuttcap%
\pgfsetmiterjoin%
\definecolor{currentfill}{rgb}{0.000000,0.490196,0.754902}%
\pgfsetfillcolor{currentfill}%
\pgfsetlinewidth{0.000000pt}%
\definecolor{currentstroke}{rgb}{0.000000,0.000000,0.000000}%
\pgfsetstrokecolor{currentstroke}%
\pgfsetstrokeopacity{0.000000}%
\pgfsetdash{}{0pt}%
\pgfpathmoveto{\pgfqpoint{1.385807in}{0.418653in}}%
\pgfpathlineto{\pgfqpoint{1.386519in}{0.422119in}}%
\pgfpathlineto{\pgfqpoint{1.397550in}{0.420104in}}%
\pgfpathlineto{\pgfqpoint{1.394836in}{0.415075in}}%
\pgfpathlineto{\pgfqpoint{1.385807in}{0.418653in}}%
\pgfpathclose%
\pgfusepath{fill}%
\end{pgfscope}%
\begin{pgfscope}%
\pgfpathrectangle{\pgfqpoint{0.100000in}{0.100000in}}{\pgfqpoint{3.608454in}{2.310000in}}%
\pgfusepath{clip}%
\pgfsetbuttcap%
\pgfsetmiterjoin%
\definecolor{currentfill}{rgb}{0.000000,0.490196,0.754902}%
\pgfsetfillcolor{currentfill}%
\pgfsetlinewidth{0.000000pt}%
\definecolor{currentstroke}{rgb}{0.000000,0.000000,0.000000}%
\pgfsetstrokecolor{currentstroke}%
\pgfsetstrokeopacity{0.000000}%
\pgfsetdash{}{0pt}%
\pgfpathmoveto{\pgfqpoint{1.407961in}{0.408746in}}%
\pgfpathlineto{\pgfqpoint{1.405253in}{0.414109in}}%
\pgfpathlineto{\pgfqpoint{1.413113in}{0.426464in}}%
\pgfpathlineto{\pgfqpoint{1.404816in}{0.436091in}}%
\pgfpathlineto{\pgfqpoint{1.404807in}{0.443981in}}%
\pgfpathlineto{\pgfqpoint{1.412629in}{0.450646in}}%
\pgfpathlineto{\pgfqpoint{1.418487in}{0.448805in}}%
\pgfpathlineto{\pgfqpoint{1.419893in}{0.442848in}}%
\pgfpathlineto{\pgfqpoint{1.418178in}{0.434960in}}%
\pgfpathlineto{\pgfqpoint{1.432543in}{0.428456in}}%
\pgfpathlineto{\pgfqpoint{1.436516in}{0.412431in}}%
\pgfpathlineto{\pgfqpoint{1.441689in}{0.404958in}}%
\pgfpathlineto{\pgfqpoint{1.438189in}{0.398895in}}%
\pgfpathlineto{\pgfqpoint{1.431764in}{0.398037in}}%
\pgfpathlineto{\pgfqpoint{1.421058in}{0.403867in}}%
\pgfpathlineto{\pgfqpoint{1.407961in}{0.408746in}}%
\pgfpathclose%
\pgfusepath{fill}%
\end{pgfscope}%
\begin{pgfscope}%
\pgfpathrectangle{\pgfqpoint{0.100000in}{0.100000in}}{\pgfqpoint{3.608454in}{2.310000in}}%
\pgfusepath{clip}%
\pgfsetbuttcap%
\pgfsetmiterjoin%
\definecolor{currentfill}{rgb}{0.000000,0.490196,0.754902}%
\pgfsetfillcolor{currentfill}%
\pgfsetlinewidth{0.000000pt}%
\definecolor{currentstroke}{rgb}{0.000000,0.000000,0.000000}%
\pgfsetstrokecolor{currentstroke}%
\pgfsetstrokeopacity{0.000000}%
\pgfsetdash{}{0pt}%
\pgfpathmoveto{\pgfqpoint{1.406047in}{0.473337in}}%
\pgfpathlineto{\pgfqpoint{1.416814in}{0.465266in}}%
\pgfpathlineto{\pgfqpoint{1.412882in}{0.461231in}}%
\pgfpathlineto{\pgfqpoint{1.404816in}{0.462303in}}%
\pgfpathlineto{\pgfqpoint{1.398464in}{0.468448in}}%
\pgfpathlineto{\pgfqpoint{1.393702in}{0.476372in}}%
\pgfpathlineto{\pgfqpoint{1.378874in}{0.487793in}}%
\pgfpathlineto{\pgfqpoint{1.389247in}{0.492428in}}%
\pgfpathlineto{\pgfqpoint{1.406047in}{0.473337in}}%
\pgfpathclose%
\pgfusepath{fill}%
\end{pgfscope}%
\begin{pgfscope}%
\pgfpathrectangle{\pgfqpoint{0.100000in}{0.100000in}}{\pgfqpoint{3.608454in}{2.310000in}}%
\pgfusepath{clip}%
\pgfsetbuttcap%
\pgfsetmiterjoin%
\definecolor{currentfill}{rgb}{0.000000,0.694118,0.652941}%
\pgfsetfillcolor{currentfill}%
\pgfsetlinewidth{0.000000pt}%
\definecolor{currentstroke}{rgb}{0.000000,0.000000,0.000000}%
\pgfsetstrokecolor{currentstroke}%
\pgfsetstrokeopacity{0.000000}%
\pgfsetdash{}{0pt}%
\pgfpathmoveto{\pgfqpoint{1.576920in}{0.953485in}}%
\pgfpathlineto{\pgfqpoint{1.574789in}{0.933425in}}%
\pgfpathlineto{\pgfqpoint{1.544636in}{0.936153in}}%
\pgfpathlineto{\pgfqpoint{1.545317in}{0.943021in}}%
\pgfpathlineto{\pgfqpoint{1.531592in}{0.944243in}}%
\pgfpathlineto{\pgfqpoint{1.532929in}{0.957126in}}%
\pgfpathlineto{\pgfqpoint{1.524717in}{0.957918in}}%
\pgfpathlineto{\pgfqpoint{1.526759in}{0.978642in}}%
\pgfpathlineto{\pgfqpoint{1.519945in}{0.979398in}}%
\pgfpathlineto{\pgfqpoint{1.523829in}{1.020634in}}%
\pgfpathlineto{\pgfqpoint{1.539569in}{1.019066in}}%
\pgfpathlineto{\pgfqpoint{1.540290in}{1.025994in}}%
\pgfpathlineto{\pgfqpoint{1.553889in}{1.024630in}}%
\pgfpathlineto{\pgfqpoint{1.554565in}{1.031506in}}%
\pgfpathlineto{\pgfqpoint{1.561377in}{1.030839in}}%
\pgfpathlineto{\pgfqpoint{1.562067in}{1.037697in}}%
\pgfpathlineto{\pgfqpoint{1.568889in}{1.037039in}}%
\pgfpathlineto{\pgfqpoint{1.569604in}{1.044093in}}%
\pgfpathlineto{\pgfqpoint{1.585141in}{1.042675in}}%
\pgfpathlineto{\pgfqpoint{1.583688in}{1.026354in}}%
\pgfpathlineto{\pgfqpoint{1.617270in}{1.023437in}}%
\pgfpathlineto{\pgfqpoint{1.614395in}{0.989155in}}%
\pgfpathlineto{\pgfqpoint{1.580563in}{0.992038in}}%
\pgfpathlineto{\pgfqpoint{1.576920in}{0.953485in}}%
\pgfpathclose%
\pgfusepath{fill}%
\end{pgfscope}%
\begin{pgfscope}%
\pgfpathrectangle{\pgfqpoint{0.100000in}{0.100000in}}{\pgfqpoint{3.608454in}{2.310000in}}%
\pgfusepath{clip}%
\pgfsetbuttcap%
\pgfsetmiterjoin%
\definecolor{currentfill}{rgb}{0.000000,0.611765,0.694118}%
\pgfsetfillcolor{currentfill}%
\pgfsetlinewidth{0.000000pt}%
\definecolor{currentstroke}{rgb}{0.000000,0.000000,0.000000}%
\pgfsetstrokecolor{currentstroke}%
\pgfsetstrokeopacity{0.000000}%
\pgfsetdash{}{0pt}%
\pgfpathmoveto{\pgfqpoint{3.088906in}{0.984668in}}%
\pgfpathlineto{\pgfqpoint{3.081713in}{0.979204in}}%
\pgfpathlineto{\pgfqpoint{3.073391in}{0.976759in}}%
\pgfpathlineto{\pgfqpoint{3.065412in}{0.981312in}}%
\pgfpathlineto{\pgfqpoint{3.059264in}{0.991086in}}%
\pgfpathlineto{\pgfqpoint{3.051594in}{0.997455in}}%
\pgfpathlineto{\pgfqpoint{3.052596in}{1.003504in}}%
\pgfpathlineto{\pgfqpoint{3.049719in}{1.006843in}}%
\pgfpathlineto{\pgfqpoint{3.049502in}{1.021849in}}%
\pgfpathlineto{\pgfqpoint{3.046504in}{1.026091in}}%
\pgfpathlineto{\pgfqpoint{3.041428in}{1.022979in}}%
\pgfpathlineto{\pgfqpoint{3.030270in}{1.030329in}}%
\pgfpathlineto{\pgfqpoint{3.034331in}{1.046602in}}%
\pgfpathlineto{\pgfqpoint{3.026275in}{1.051810in}}%
\pgfpathlineto{\pgfqpoint{3.034032in}{1.059624in}}%
\pgfpathlineto{\pgfqpoint{3.040979in}{1.056871in}}%
\pgfpathlineto{\pgfqpoint{3.047179in}{1.057968in}}%
\pgfpathlineto{\pgfqpoint{3.046213in}{1.063185in}}%
\pgfpathlineto{\pgfqpoint{3.054685in}{1.068940in}}%
\pgfpathlineto{\pgfqpoint{3.060908in}{1.061022in}}%
\pgfpathlineto{\pgfqpoint{3.065576in}{1.050744in}}%
\pgfpathlineto{\pgfqpoint{3.073425in}{1.048823in}}%
\pgfpathlineto{\pgfqpoint{3.078779in}{1.041854in}}%
\pgfpathlineto{\pgfqpoint{3.076612in}{1.036110in}}%
\pgfpathlineto{\pgfqpoint{3.096046in}{1.025469in}}%
\pgfpathlineto{\pgfqpoint{3.091334in}{1.021253in}}%
\pgfpathlineto{\pgfqpoint{3.096860in}{1.011889in}}%
\pgfpathlineto{\pgfqpoint{3.092298in}{1.007042in}}%
\pgfpathlineto{\pgfqpoint{3.094287in}{1.003646in}}%
\pgfpathlineto{\pgfqpoint{3.088906in}{0.984668in}}%
\pgfpathclose%
\pgfusepath{fill}%
\end{pgfscope}%
\begin{pgfscope}%
\pgfpathrectangle{\pgfqpoint{0.100000in}{0.100000in}}{\pgfqpoint{3.608454in}{2.310000in}}%
\pgfusepath{clip}%
\pgfsetbuttcap%
\pgfsetmiterjoin%
\definecolor{currentfill}{rgb}{0.000000,0.611765,0.694118}%
\pgfsetfillcolor{currentfill}%
\pgfsetlinewidth{0.000000pt}%
\definecolor{currentstroke}{rgb}{0.000000,0.000000,0.000000}%
\pgfsetstrokecolor{currentstroke}%
\pgfsetstrokeopacity{0.000000}%
\pgfsetdash{}{0pt}%
\pgfpathmoveto{\pgfqpoint{1.562162in}{1.934829in}}%
\pgfpathlineto{\pgfqpoint{1.559222in}{1.935142in}}%
\pgfpathlineto{\pgfqpoint{1.556192in}{1.907466in}}%
\pgfpathlineto{\pgfqpoint{1.553831in}{1.907730in}}%
\pgfpathlineto{\pgfqpoint{1.487149in}{1.915660in}}%
\pgfpathlineto{\pgfqpoint{1.490580in}{1.943193in}}%
\pgfpathlineto{\pgfqpoint{1.492673in}{1.942949in}}%
\pgfpathlineto{\pgfqpoint{1.495943in}{1.970351in}}%
\pgfpathlineto{\pgfqpoint{1.497843in}{1.970127in}}%
\pgfpathlineto{\pgfqpoint{1.501201in}{1.997658in}}%
\pgfpathlineto{\pgfqpoint{1.503208in}{1.998589in}}%
\pgfpathlineto{\pgfqpoint{1.506437in}{2.025137in}}%
\pgfpathlineto{\pgfqpoint{1.513276in}{2.024315in}}%
\pgfpathlineto{\pgfqpoint{1.512427in}{2.017326in}}%
\pgfpathlineto{\pgfqpoint{1.519298in}{2.016515in}}%
\pgfpathlineto{\pgfqpoint{1.520131in}{2.023482in}}%
\pgfpathlineto{\pgfqpoint{1.543137in}{2.020826in}}%
\pgfpathlineto{\pgfqpoint{1.547528in}{2.020313in}}%
\pgfpathlineto{\pgfqpoint{1.545674in}{2.004107in}}%
\pgfpathlineto{\pgfqpoint{1.552609in}{2.003322in}}%
\pgfpathlineto{\pgfqpoint{1.552079in}{1.998723in}}%
\pgfpathlineto{\pgfqpoint{1.558985in}{1.997946in}}%
\pgfpathlineto{\pgfqpoint{1.560751in}{1.993122in}}%
\pgfpathlineto{\pgfqpoint{1.583733in}{1.990580in}}%
\pgfpathlineto{\pgfqpoint{1.582076in}{1.976639in}}%
\pgfpathlineto{\pgfqpoint{1.574985in}{1.975097in}}%
\pgfpathlineto{\pgfqpoint{1.573489in}{1.961301in}}%
\pgfpathlineto{\pgfqpoint{1.564057in}{1.962319in}}%
\pgfpathlineto{\pgfqpoint{1.562162in}{1.934829in}}%
\pgfpathclose%
\pgfusepath{fill}%
\end{pgfscope}%
\begin{pgfscope}%
\pgfpathrectangle{\pgfqpoint{0.100000in}{0.100000in}}{\pgfqpoint{3.608454in}{2.310000in}}%
\pgfusepath{clip}%
\pgfsetbuttcap%
\pgfsetmiterjoin%
\definecolor{currentfill}{rgb}{0.000000,0.431373,0.784314}%
\pgfsetfillcolor{currentfill}%
\pgfsetlinewidth{0.000000pt}%
\definecolor{currentstroke}{rgb}{0.000000,0.000000,0.000000}%
\pgfsetstrokecolor{currentstroke}%
\pgfsetstrokeopacity{0.000000}%
\pgfsetdash{}{0pt}%
\pgfpathmoveto{\pgfqpoint{2.013559in}{0.517174in}}%
\pgfpathlineto{\pgfqpoint{2.014894in}{0.542939in}}%
\pgfpathlineto{\pgfqpoint{1.992065in}{0.566132in}}%
\pgfpathlineto{\pgfqpoint{2.012654in}{0.578657in}}%
\pgfpathlineto{\pgfqpoint{2.015636in}{0.587127in}}%
\pgfpathlineto{\pgfqpoint{2.024967in}{0.596552in}}%
\pgfpathlineto{\pgfqpoint{2.032870in}{0.593014in}}%
\pgfpathlineto{\pgfqpoint{2.032880in}{0.583631in}}%
\pgfpathlineto{\pgfqpoint{2.037581in}{0.580386in}}%
\pgfpathlineto{\pgfqpoint{2.040450in}{0.572451in}}%
\pgfpathlineto{\pgfqpoint{2.047518in}{0.566563in}}%
\pgfpathlineto{\pgfqpoint{2.054185in}{0.571463in}}%
\pgfpathlineto{\pgfqpoint{2.061080in}{0.568664in}}%
\pgfpathlineto{\pgfqpoint{2.066381in}{0.572643in}}%
\pgfpathlineto{\pgfqpoint{2.068450in}{0.580491in}}%
\pgfpathlineto{\pgfqpoint{2.074520in}{0.580523in}}%
\pgfpathlineto{\pgfqpoint{2.077251in}{0.591805in}}%
\pgfpathlineto{\pgfqpoint{2.088573in}{0.592795in}}%
\pgfpathlineto{\pgfqpoint{2.091703in}{0.589805in}}%
\pgfpathlineto{\pgfqpoint{2.090532in}{0.582574in}}%
\pgfpathlineto{\pgfqpoint{2.102863in}{0.561208in}}%
\pgfpathlineto{\pgfqpoint{2.098432in}{0.551471in}}%
\pgfpathlineto{\pgfqpoint{2.079822in}{0.534485in}}%
\pgfpathlineto{\pgfqpoint{2.074339in}{0.533228in}}%
\pgfpathlineto{\pgfqpoint{2.049544in}{0.518585in}}%
\pgfpathlineto{\pgfqpoint{2.036951in}{0.514000in}}%
\pgfpathlineto{\pgfqpoint{2.033736in}{0.517714in}}%
\pgfpathlineto{\pgfqpoint{2.020863in}{0.511930in}}%
\pgfpathlineto{\pgfqpoint{2.021021in}{0.521517in}}%
\pgfpathlineto{\pgfqpoint{2.013559in}{0.517174in}}%
\pgfpathclose%
\pgfusepath{fill}%
\end{pgfscope}%
\begin{pgfscope}%
\pgfpathrectangle{\pgfqpoint{0.100000in}{0.100000in}}{\pgfqpoint{3.608454in}{2.310000in}}%
\pgfusepath{clip}%
\pgfsetbuttcap%
\pgfsetmiterjoin%
\definecolor{currentfill}{rgb}{0.000000,0.686275,0.656863}%
\pgfsetfillcolor{currentfill}%
\pgfsetlinewidth{0.000000pt}%
\definecolor{currentstroke}{rgb}{0.000000,0.000000,0.000000}%
\pgfsetstrokecolor{currentstroke}%
\pgfsetstrokeopacity{0.000000}%
\pgfsetdash{}{0pt}%
\pgfpathmoveto{\pgfqpoint{3.111250in}{1.435310in}}%
\pgfpathlineto{\pgfqpoint{3.103999in}{1.422556in}}%
\pgfpathlineto{\pgfqpoint{3.099019in}{1.419473in}}%
\pgfpathlineto{\pgfqpoint{3.095711in}{1.408904in}}%
\pgfpathlineto{\pgfqpoint{3.086938in}{1.414331in}}%
\pgfpathlineto{\pgfqpoint{3.084328in}{1.406748in}}%
\pgfpathlineto{\pgfqpoint{3.063909in}{1.418899in}}%
\pgfpathlineto{\pgfqpoint{3.065468in}{1.428454in}}%
\pgfpathlineto{\pgfqpoint{3.062165in}{1.429526in}}%
\pgfpathlineto{\pgfqpoint{3.063986in}{1.437446in}}%
\pgfpathlineto{\pgfqpoint{3.060760in}{1.438938in}}%
\pgfpathlineto{\pgfqpoint{3.052407in}{1.435951in}}%
\pgfpathlineto{\pgfqpoint{3.045971in}{1.477046in}}%
\pgfpathlineto{\pgfqpoint{3.051009in}{1.477958in}}%
\pgfpathlineto{\pgfqpoint{3.111561in}{1.488897in}}%
\pgfpathlineto{\pgfqpoint{3.116008in}{1.487243in}}%
\pgfpathlineto{\pgfqpoint{3.110506in}{1.479451in}}%
\pgfpathlineto{\pgfqpoint{3.109226in}{1.471916in}}%
\pgfpathlineto{\pgfqpoint{3.097628in}{1.471273in}}%
\pgfpathlineto{\pgfqpoint{3.091266in}{1.461236in}}%
\pgfpathlineto{\pgfqpoint{3.088891in}{1.453283in}}%
\pgfpathlineto{\pgfqpoint{3.082421in}{1.444639in}}%
\pgfpathlineto{\pgfqpoint{3.087837in}{1.442494in}}%
\pgfpathlineto{\pgfqpoint{3.105576in}{1.439244in}}%
\pgfpathlineto{\pgfqpoint{3.111250in}{1.435310in}}%
\pgfpathclose%
\pgfusepath{fill}%
\end{pgfscope}%
\begin{pgfscope}%
\pgfpathrectangle{\pgfqpoint{0.100000in}{0.100000in}}{\pgfqpoint{3.608454in}{2.310000in}}%
\pgfusepath{clip}%
\pgfsetbuttcap%
\pgfsetmiterjoin%
\definecolor{currentfill}{rgb}{0.000000,0.611765,0.694118}%
\pgfsetfillcolor{currentfill}%
\pgfsetlinewidth{0.000000pt}%
\definecolor{currentstroke}{rgb}{0.000000,0.000000,0.000000}%
\pgfsetstrokecolor{currentstroke}%
\pgfsetstrokeopacity{0.000000}%
\pgfsetdash{}{0pt}%
\pgfpathmoveto{\pgfqpoint{3.279173in}{1.774369in}}%
\pgfpathlineto{\pgfqpoint{3.272069in}{1.775491in}}%
\pgfpathlineto{\pgfqpoint{3.271017in}{1.786225in}}%
\pgfpathlineto{\pgfqpoint{3.267628in}{1.792368in}}%
\pgfpathlineto{\pgfqpoint{3.270444in}{1.800501in}}%
\pgfpathlineto{\pgfqpoint{3.267521in}{1.808820in}}%
\pgfpathlineto{\pgfqpoint{3.257858in}{1.810998in}}%
\pgfpathlineto{\pgfqpoint{3.260468in}{1.823521in}}%
\pgfpathlineto{\pgfqpoint{3.245888in}{1.867665in}}%
\pgfpathlineto{\pgfqpoint{3.263105in}{1.873951in}}%
\pgfpathlineto{\pgfqpoint{3.276788in}{1.878840in}}%
\pgfpathlineto{\pgfqpoint{3.280930in}{1.867522in}}%
\pgfpathlineto{\pgfqpoint{3.277231in}{1.863034in}}%
\pgfpathlineto{\pgfqpoint{3.286142in}{1.855787in}}%
\pgfpathlineto{\pgfqpoint{3.295690in}{1.856839in}}%
\pgfpathlineto{\pgfqpoint{3.296075in}{1.852707in}}%
\pgfpathlineto{\pgfqpoint{3.287566in}{1.849431in}}%
\pgfpathlineto{\pgfqpoint{3.309890in}{1.786988in}}%
\pgfpathlineto{\pgfqpoint{3.300308in}{1.775508in}}%
\pgfpathlineto{\pgfqpoint{3.291128in}{1.772508in}}%
\pgfpathlineto{\pgfqpoint{3.279173in}{1.774369in}}%
\pgfpathclose%
\pgfusepath{fill}%
\end{pgfscope}%
\begin{pgfscope}%
\pgfpathrectangle{\pgfqpoint{0.100000in}{0.100000in}}{\pgfqpoint{3.608454in}{2.310000in}}%
\pgfusepath{clip}%
\pgfsetbuttcap%
\pgfsetmiterjoin%
\definecolor{currentfill}{rgb}{0.000000,0.517647,0.741176}%
\pgfsetfillcolor{currentfill}%
\pgfsetlinewidth{0.000000pt}%
\definecolor{currentstroke}{rgb}{0.000000,0.000000,0.000000}%
\pgfsetstrokecolor{currentstroke}%
\pgfsetstrokeopacity{0.000000}%
\pgfsetdash{}{0pt}%
\pgfpathmoveto{\pgfqpoint{2.558058in}{1.370400in}}%
\pgfpathlineto{\pgfqpoint{2.550764in}{1.371543in}}%
\pgfpathlineto{\pgfqpoint{2.506514in}{1.367545in}}%
\pgfpathlineto{\pgfqpoint{2.506265in}{1.370984in}}%
\pgfpathlineto{\pgfqpoint{2.503724in}{1.405402in}}%
\pgfpathlineto{\pgfqpoint{2.528514in}{1.407327in}}%
\pgfpathlineto{\pgfqpoint{2.534185in}{1.410410in}}%
\pgfpathlineto{\pgfqpoint{2.534753in}{1.425828in}}%
\pgfpathlineto{\pgfqpoint{2.559183in}{1.428045in}}%
\pgfpathlineto{\pgfqpoint{2.562752in}{1.385889in}}%
\pgfpathlineto{\pgfqpoint{2.558464in}{1.382618in}}%
\pgfpathlineto{\pgfqpoint{2.560407in}{1.373999in}}%
\pgfpathlineto{\pgfqpoint{2.558058in}{1.370400in}}%
\pgfpathclose%
\pgfusepath{fill}%
\end{pgfscope}%
\begin{pgfscope}%
\pgfpathrectangle{\pgfqpoint{0.100000in}{0.100000in}}{\pgfqpoint{3.608454in}{2.310000in}}%
\pgfusepath{clip}%
\pgfsetbuttcap%
\pgfsetmiterjoin%
\definecolor{currentfill}{rgb}{0.000000,0.419608,0.790196}%
\pgfsetfillcolor{currentfill}%
\pgfsetlinewidth{0.000000pt}%
\definecolor{currentstroke}{rgb}{0.000000,0.000000,0.000000}%
\pgfsetstrokecolor{currentstroke}%
\pgfsetstrokeopacity{0.000000}%
\pgfsetdash{}{0pt}%
\pgfpathmoveto{\pgfqpoint{1.843275in}{1.841097in}}%
\pgfpathlineto{\pgfqpoint{1.812489in}{1.842876in}}%
\pgfpathlineto{\pgfqpoint{1.809666in}{1.848703in}}%
\pgfpathlineto{\pgfqpoint{1.812344in}{1.853586in}}%
\pgfpathlineto{\pgfqpoint{1.808999in}{1.861084in}}%
\pgfpathlineto{\pgfqpoint{1.801983in}{1.865242in}}%
\pgfpathlineto{\pgfqpoint{1.807298in}{1.876074in}}%
\pgfpathlineto{\pgfqpoint{1.811895in}{1.879653in}}%
\pgfpathlineto{\pgfqpoint{1.808345in}{1.892715in}}%
\pgfpathlineto{\pgfqpoint{1.802493in}{1.899008in}}%
\pgfpathlineto{\pgfqpoint{1.845659in}{1.896366in}}%
\pgfpathlineto{\pgfqpoint{1.843275in}{1.841097in}}%
\pgfpathclose%
\pgfusepath{fill}%
\end{pgfscope}%
\begin{pgfscope}%
\pgfpathrectangle{\pgfqpoint{0.100000in}{0.100000in}}{\pgfqpoint{3.608454in}{2.310000in}}%
\pgfusepath{clip}%
\pgfsetbuttcap%
\pgfsetmiterjoin%
\definecolor{currentfill}{rgb}{0.000000,0.827451,0.586275}%
\pgfsetfillcolor{currentfill}%
\pgfsetlinewidth{0.000000pt}%
\definecolor{currentstroke}{rgb}{0.000000,0.000000,0.000000}%
\pgfsetstrokecolor{currentstroke}%
\pgfsetstrokeopacity{0.000000}%
\pgfsetdash{}{0pt}%
\pgfpathmoveto{\pgfqpoint{0.787216in}{0.442053in}}%
\pgfpathlineto{\pgfqpoint{0.796011in}{0.435184in}}%
\pgfpathlineto{\pgfqpoint{0.790893in}{0.428785in}}%
\pgfpathlineto{\pgfqpoint{0.779568in}{0.437653in}}%
\pgfpathlineto{\pgfqpoint{0.783683in}{0.438305in}}%
\pgfpathlineto{\pgfqpoint{0.787048in}{0.440340in}}%
\pgfpathlineto{\pgfqpoint{0.787216in}{0.442053in}}%
\pgfpathclose%
\pgfusepath{fill}%
\end{pgfscope}%
\begin{pgfscope}%
\pgfpathrectangle{\pgfqpoint{0.100000in}{0.100000in}}{\pgfqpoint{3.608454in}{2.310000in}}%
\pgfusepath{clip}%
\pgfsetbuttcap%
\pgfsetmiterjoin%
\definecolor{currentfill}{rgb}{0.000000,0.796078,0.601961}%
\pgfsetfillcolor{currentfill}%
\pgfsetlinewidth{0.000000pt}%
\definecolor{currentstroke}{rgb}{0.000000,0.000000,0.000000}%
\pgfsetstrokecolor{currentstroke}%
\pgfsetstrokeopacity{0.000000}%
\pgfsetdash{}{0pt}%
\pgfpathmoveto{\pgfqpoint{1.760148in}{1.734344in}}%
\pgfpathlineto{\pgfqpoint{1.750543in}{1.731001in}}%
\pgfpathlineto{\pgfqpoint{1.748309in}{1.699187in}}%
\pgfpathlineto{\pgfqpoint{1.697612in}{1.702965in}}%
\pgfpathlineto{\pgfqpoint{1.696530in}{1.709950in}}%
\pgfpathlineto{\pgfqpoint{1.697918in}{1.727667in}}%
\pgfpathlineto{\pgfqpoint{1.704828in}{1.727690in}}%
\pgfpathlineto{\pgfqpoint{1.707249in}{1.750308in}}%
\pgfpathlineto{\pgfqpoint{1.710799in}{1.791190in}}%
\pgfpathlineto{\pgfqpoint{1.723276in}{1.792199in}}%
\pgfpathlineto{\pgfqpoint{1.729800in}{1.795494in}}%
\pgfpathlineto{\pgfqpoint{1.737625in}{1.793677in}}%
\pgfpathlineto{\pgfqpoint{1.747956in}{1.802589in}}%
\pgfpathlineto{\pgfqpoint{1.754993in}{1.802225in}}%
\pgfpathlineto{\pgfqpoint{1.760814in}{1.806325in}}%
\pgfpathlineto{\pgfqpoint{1.759531in}{1.801865in}}%
\pgfpathlineto{\pgfqpoint{1.757629in}{1.774393in}}%
\pgfpathlineto{\pgfqpoint{1.756032in}{1.760574in}}%
\pgfpathlineto{\pgfqpoint{1.762841in}{1.760148in}}%
\pgfpathlineto{\pgfqpoint{1.760148in}{1.734344in}}%
\pgfpathclose%
\pgfusepath{fill}%
\end{pgfscope}%
\begin{pgfscope}%
\pgfpathrectangle{\pgfqpoint{0.100000in}{0.100000in}}{\pgfqpoint{3.608454in}{2.310000in}}%
\pgfusepath{clip}%
\pgfsetbuttcap%
\pgfsetmiterjoin%
\definecolor{currentfill}{rgb}{0.000000,0.541176,0.729412}%
\pgfsetfillcolor{currentfill}%
\pgfsetlinewidth{0.000000pt}%
\definecolor{currentstroke}{rgb}{0.000000,0.000000,0.000000}%
\pgfsetstrokecolor{currentstroke}%
\pgfsetstrokeopacity{0.000000}%
\pgfsetdash{}{0pt}%
\pgfpathmoveto{\pgfqpoint{2.406029in}{1.504012in}}%
\pgfpathlineto{\pgfqpoint{2.405330in}{1.517808in}}%
\pgfpathlineto{\pgfqpoint{2.378376in}{1.516534in}}%
\pgfpathlineto{\pgfqpoint{2.378619in}{1.509638in}}%
\pgfpathlineto{\pgfqpoint{2.358001in}{1.509016in}}%
\pgfpathlineto{\pgfqpoint{2.348213in}{1.508885in}}%
\pgfpathlineto{\pgfqpoint{2.345525in}{1.515337in}}%
\pgfpathlineto{\pgfqpoint{2.337899in}{1.521676in}}%
\pgfpathlineto{\pgfqpoint{2.341290in}{1.535608in}}%
\pgfpathlineto{\pgfqpoint{2.356534in}{1.539366in}}%
\pgfpathlineto{\pgfqpoint{2.356154in}{1.550723in}}%
\pgfpathlineto{\pgfqpoint{2.349290in}{1.550458in}}%
\pgfpathlineto{\pgfqpoint{2.348758in}{1.564236in}}%
\pgfpathlineto{\pgfqpoint{2.376029in}{1.564684in}}%
\pgfpathlineto{\pgfqpoint{2.380372in}{1.562314in}}%
\pgfpathlineto{\pgfqpoint{2.387375in}{1.567036in}}%
\pgfpathlineto{\pgfqpoint{2.391551in}{1.551544in}}%
\pgfpathlineto{\pgfqpoint{2.410650in}{1.552521in}}%
\pgfpathlineto{\pgfqpoint{2.412483in}{1.524944in}}%
\pgfpathlineto{\pgfqpoint{2.425489in}{1.525655in}}%
\pgfpathlineto{\pgfqpoint{2.426710in}{1.505121in}}%
\pgfpathlineto{\pgfqpoint{2.406029in}{1.504012in}}%
\pgfpathclose%
\pgfusepath{fill}%
\end{pgfscope}%
\begin{pgfscope}%
\pgfpathrectangle{\pgfqpoint{0.100000in}{0.100000in}}{\pgfqpoint{3.608454in}{2.310000in}}%
\pgfusepath{clip}%
\pgfsetbuttcap%
\pgfsetmiterjoin%
\definecolor{currentfill}{rgb}{0.000000,0.494118,0.752941}%
\pgfsetfillcolor{currentfill}%
\pgfsetlinewidth{0.000000pt}%
\definecolor{currentstroke}{rgb}{0.000000,0.000000,0.000000}%
\pgfsetstrokecolor{currentstroke}%
\pgfsetstrokeopacity{0.000000}%
\pgfsetdash{}{0pt}%
\pgfpathmoveto{\pgfqpoint{2.708723in}{0.733977in}}%
\pgfpathlineto{\pgfqpoint{2.719091in}{0.734938in}}%
\pgfpathlineto{\pgfqpoint{2.721186in}{0.711980in}}%
\pgfpathlineto{\pgfqpoint{2.733506in}{0.712730in}}%
\pgfpathlineto{\pgfqpoint{2.730084in}{0.707397in}}%
\pgfpathlineto{\pgfqpoint{2.735056in}{0.696150in}}%
\pgfpathlineto{\pgfqpoint{2.726751in}{0.687235in}}%
\pgfpathlineto{\pgfqpoint{2.727447in}{0.677658in}}%
\pgfpathlineto{\pgfqpoint{2.710794in}{0.682159in}}%
\pgfpathlineto{\pgfqpoint{2.699063in}{0.683673in}}%
\pgfpathlineto{\pgfqpoint{2.682757in}{0.683563in}}%
\pgfpathlineto{\pgfqpoint{2.666620in}{0.680661in}}%
\pgfpathlineto{\pgfqpoint{2.645613in}{0.674473in}}%
\pgfpathlineto{\pgfqpoint{2.635436in}{0.672947in}}%
\pgfpathlineto{\pgfqpoint{2.622714in}{0.669017in}}%
\pgfpathlineto{\pgfqpoint{2.627322in}{0.678308in}}%
\pgfpathlineto{\pgfqpoint{2.632066in}{0.682154in}}%
\pgfpathlineto{\pgfqpoint{2.626017in}{0.689071in}}%
\pgfpathlineto{\pgfqpoint{2.628964in}{0.696389in}}%
\pgfpathlineto{\pgfqpoint{2.627748in}{0.701087in}}%
\pgfpathlineto{\pgfqpoint{2.618895in}{0.705619in}}%
\pgfpathlineto{\pgfqpoint{2.610908in}{0.714979in}}%
\pgfpathlineto{\pgfqpoint{2.612513in}{0.725646in}}%
\pgfpathlineto{\pgfqpoint{2.674593in}{0.730790in}}%
\pgfpathlineto{\pgfqpoint{2.708723in}{0.733977in}}%
\pgfpathclose%
\pgfusepath{fill}%
\end{pgfscope}%
\begin{pgfscope}%
\pgfpathrectangle{\pgfqpoint{0.100000in}{0.100000in}}{\pgfqpoint{3.608454in}{2.310000in}}%
\pgfusepath{clip}%
\pgfsetbuttcap%
\pgfsetmiterjoin%
\definecolor{currentfill}{rgb}{0.000000,0.364706,0.817647}%
\pgfsetfillcolor{currentfill}%
\pgfsetlinewidth{0.000000pt}%
\definecolor{currentstroke}{rgb}{0.000000,0.000000,0.000000}%
\pgfsetstrokecolor{currentstroke}%
\pgfsetstrokeopacity{0.000000}%
\pgfsetdash{}{0pt}%
\pgfpathmoveto{\pgfqpoint{1.826801in}{1.425965in}}%
\pgfpathlineto{\pgfqpoint{1.824934in}{1.391676in}}%
\pgfpathlineto{\pgfqpoint{1.792462in}{1.393414in}}%
\pgfpathlineto{\pgfqpoint{1.791324in}{1.393448in}}%
\pgfpathlineto{\pgfqpoint{1.793434in}{1.427744in}}%
\pgfpathlineto{\pgfqpoint{1.826801in}{1.425965in}}%
\pgfpathclose%
\pgfusepath{fill}%
\end{pgfscope}%
\begin{pgfscope}%
\pgfpathrectangle{\pgfqpoint{0.100000in}{0.100000in}}{\pgfqpoint{3.608454in}{2.310000in}}%
\pgfusepath{clip}%
\pgfsetbuttcap%
\pgfsetmiterjoin%
\definecolor{currentfill}{rgb}{0.000000,0.815686,0.592157}%
\pgfsetfillcolor{currentfill}%
\pgfsetlinewidth{0.000000pt}%
\definecolor{currentstroke}{rgb}{0.000000,0.000000,0.000000}%
\pgfsetstrokecolor{currentstroke}%
\pgfsetstrokeopacity{0.000000}%
\pgfsetdash{}{0pt}%
\pgfpathmoveto{\pgfqpoint{2.350507in}{1.098801in}}%
\pgfpathlineto{\pgfqpoint{2.343663in}{1.088662in}}%
\pgfpathlineto{\pgfqpoint{2.340391in}{1.078576in}}%
\pgfpathlineto{\pgfqpoint{2.340700in}{1.069582in}}%
\pgfpathlineto{\pgfqpoint{2.313347in}{1.069167in}}%
\pgfpathlineto{\pgfqpoint{2.313117in}{1.083232in}}%
\pgfpathlineto{\pgfqpoint{2.307635in}{1.090812in}}%
\pgfpathlineto{\pgfqpoint{2.308868in}{1.093507in}}%
\pgfpathlineto{\pgfqpoint{2.302022in}{1.100303in}}%
\pgfpathlineto{\pgfqpoint{2.293303in}{1.102454in}}%
\pgfpathlineto{\pgfqpoint{2.295109in}{1.108191in}}%
\pgfpathlineto{\pgfqpoint{2.286012in}{1.115699in}}%
\pgfpathlineto{\pgfqpoint{2.288166in}{1.125922in}}%
\pgfpathlineto{\pgfqpoint{2.325882in}{1.126330in}}%
\pgfpathlineto{\pgfqpoint{2.325633in}{1.133246in}}%
\pgfpathlineto{\pgfqpoint{2.332736in}{1.133366in}}%
\pgfpathlineto{\pgfqpoint{2.332518in}{1.146200in}}%
\pgfpathlineto{\pgfqpoint{2.335244in}{1.146265in}}%
\pgfpathlineto{\pgfqpoint{2.343960in}{1.136912in}}%
\pgfpathlineto{\pgfqpoint{2.345455in}{1.126590in}}%
\pgfpathlineto{\pgfqpoint{2.339731in}{1.125367in}}%
\pgfpathlineto{\pgfqpoint{2.340339in}{1.098431in}}%
\pgfpathlineto{\pgfqpoint{2.350507in}{1.098801in}}%
\pgfpathclose%
\pgfusepath{fill}%
\end{pgfscope}%
\begin{pgfscope}%
\pgfpathrectangle{\pgfqpoint{0.100000in}{0.100000in}}{\pgfqpoint{3.608454in}{2.310000in}}%
\pgfusepath{clip}%
\pgfsetbuttcap%
\pgfsetmiterjoin%
\definecolor{currentfill}{rgb}{0.000000,0.411765,0.794118}%
\pgfsetfillcolor{currentfill}%
\pgfsetlinewidth{0.000000pt}%
\definecolor{currentstroke}{rgb}{0.000000,0.000000,0.000000}%
\pgfsetstrokecolor{currentstroke}%
\pgfsetstrokeopacity{0.000000}%
\pgfsetdash{}{0pt}%
\pgfpathmoveto{\pgfqpoint{2.313973in}{1.320063in}}%
\pgfpathlineto{\pgfqpoint{2.314915in}{1.287249in}}%
\pgfpathlineto{\pgfqpoint{2.280744in}{1.286387in}}%
\pgfpathlineto{\pgfqpoint{2.274008in}{1.289693in}}%
\pgfpathlineto{\pgfqpoint{2.267441in}{1.289875in}}%
\pgfpathlineto{\pgfqpoint{2.267557in}{1.296520in}}%
\pgfpathlineto{\pgfqpoint{2.253822in}{1.296576in}}%
\pgfpathlineto{\pgfqpoint{2.240098in}{1.309360in}}%
\pgfpathlineto{\pgfqpoint{2.240062in}{1.316251in}}%
\pgfpathlineto{\pgfqpoint{2.226458in}{1.316665in}}%
\pgfpathlineto{\pgfqpoint{2.226654in}{1.335261in}}%
\pgfpathlineto{\pgfqpoint{2.226701in}{1.338754in}}%
\pgfpathlineto{\pgfqpoint{2.233532in}{1.346189in}}%
\pgfpathlineto{\pgfqpoint{2.232957in}{1.349258in}}%
\pgfpathlineto{\pgfqpoint{2.238456in}{1.359055in}}%
\pgfpathlineto{\pgfqpoint{2.241843in}{1.368445in}}%
\pgfpathlineto{\pgfqpoint{2.246959in}{1.366834in}}%
\pgfpathlineto{\pgfqpoint{2.264061in}{1.361534in}}%
\pgfpathlineto{\pgfqpoint{2.271150in}{1.361570in}}%
\pgfpathlineto{\pgfqpoint{2.284008in}{1.361432in}}%
\pgfpathlineto{\pgfqpoint{2.284099in}{1.347536in}}%
\pgfpathlineto{\pgfqpoint{2.313325in}{1.348137in}}%
\pgfpathlineto{\pgfqpoint{2.313973in}{1.320063in}}%
\pgfpathclose%
\pgfusepath{fill}%
\end{pgfscope}%
\begin{pgfscope}%
\pgfpathrectangle{\pgfqpoint{0.100000in}{0.100000in}}{\pgfqpoint{3.608454in}{2.310000in}}%
\pgfusepath{clip}%
\pgfsetbuttcap%
\pgfsetmiterjoin%
\definecolor{currentfill}{rgb}{0.000000,0.521569,0.739216}%
\pgfsetfillcolor{currentfill}%
\pgfsetlinewidth{0.000000pt}%
\definecolor{currentstroke}{rgb}{0.000000,0.000000,0.000000}%
\pgfsetstrokecolor{currentstroke}%
\pgfsetstrokeopacity{0.000000}%
\pgfsetdash{}{0pt}%
\pgfpathmoveto{\pgfqpoint{2.986023in}{0.797823in}}%
\pgfpathlineto{\pgfqpoint{2.983849in}{0.804589in}}%
\pgfpathlineto{\pgfqpoint{2.978722in}{0.804662in}}%
\pgfpathlineto{\pgfqpoint{2.975294in}{0.828495in}}%
\pgfpathlineto{\pgfqpoint{2.979065in}{0.834250in}}%
\pgfpathlineto{\pgfqpoint{2.975754in}{0.837437in}}%
\pgfpathlineto{\pgfqpoint{2.967579in}{0.838638in}}%
\pgfpathlineto{\pgfqpoint{2.967645in}{0.858971in}}%
\pgfpathlineto{\pgfqpoint{2.962520in}{0.870538in}}%
\pgfpathlineto{\pgfqpoint{2.991213in}{0.868134in}}%
\pgfpathlineto{\pgfqpoint{2.994468in}{0.861760in}}%
\pgfpathlineto{\pgfqpoint{2.996708in}{0.858258in}}%
\pgfpathlineto{\pgfqpoint{3.005118in}{0.855085in}}%
\pgfpathlineto{\pgfqpoint{3.011213in}{0.848731in}}%
\pgfpathlineto{\pgfqpoint{3.020884in}{0.848437in}}%
\pgfpathlineto{\pgfqpoint{3.025040in}{0.841844in}}%
\pgfpathlineto{\pgfqpoint{3.033775in}{0.835716in}}%
\pgfpathlineto{\pgfqpoint{3.039659in}{0.833647in}}%
\pgfpathlineto{\pgfqpoint{3.044197in}{0.827018in}}%
\pgfpathlineto{\pgfqpoint{3.040729in}{0.821509in}}%
\pgfpathlineto{\pgfqpoint{3.035156in}{0.827393in}}%
\pgfpathlineto{\pgfqpoint{3.023851in}{0.824906in}}%
\pgfpathlineto{\pgfqpoint{3.019148in}{0.828947in}}%
\pgfpathlineto{\pgfqpoint{3.015577in}{0.818559in}}%
\pgfpathlineto{\pgfqpoint{3.009224in}{0.814410in}}%
\pgfpathlineto{\pgfqpoint{3.013726in}{0.809687in}}%
\pgfpathlineto{\pgfqpoint{3.007825in}{0.797264in}}%
\pgfpathlineto{\pgfqpoint{3.002283in}{0.799408in}}%
\pgfpathlineto{\pgfqpoint{2.994265in}{0.796799in}}%
\pgfpathlineto{\pgfqpoint{2.986023in}{0.797823in}}%
\pgfpathclose%
\pgfusepath{fill}%
\end{pgfscope}%
\begin{pgfscope}%
\pgfpathrectangle{\pgfqpoint{0.100000in}{0.100000in}}{\pgfqpoint{3.608454in}{2.310000in}}%
\pgfusepath{clip}%
\pgfsetbuttcap%
\pgfsetmiterjoin%
\definecolor{currentfill}{rgb}{0.000000,0.419608,0.790196}%
\pgfsetfillcolor{currentfill}%
\pgfsetlinewidth{0.000000pt}%
\definecolor{currentstroke}{rgb}{0.000000,0.000000,0.000000}%
\pgfsetstrokecolor{currentstroke}%
\pgfsetstrokeopacity{0.000000}%
\pgfsetdash{}{0pt}%
\pgfpathmoveto{\pgfqpoint{2.025168in}{1.813601in}}%
\pgfpathlineto{\pgfqpoint{2.024747in}{1.786093in}}%
\pgfpathlineto{\pgfqpoint{2.024621in}{1.779133in}}%
\pgfpathlineto{\pgfqpoint{1.986361in}{1.779846in}}%
\pgfpathlineto{\pgfqpoint{1.945343in}{1.781068in}}%
\pgfpathlineto{\pgfqpoint{1.945390in}{1.787969in}}%
\pgfpathlineto{\pgfqpoint{1.938531in}{1.788233in}}%
\pgfpathlineto{\pgfqpoint{1.939662in}{1.829664in}}%
\pgfpathlineto{\pgfqpoint{1.981983in}{1.828333in}}%
\pgfpathlineto{\pgfqpoint{2.001203in}{1.827773in}}%
\pgfpathlineto{\pgfqpoint{2.000983in}{1.813993in}}%
\pgfpathlineto{\pgfqpoint{2.025168in}{1.813601in}}%
\pgfpathclose%
\pgfusepath{fill}%
\end{pgfscope}%
\begin{pgfscope}%
\pgfpathrectangle{\pgfqpoint{0.100000in}{0.100000in}}{\pgfqpoint{3.608454in}{2.310000in}}%
\pgfusepath{clip}%
\pgfsetbuttcap%
\pgfsetmiterjoin%
\definecolor{currentfill}{rgb}{0.000000,0.678431,0.660784}%
\pgfsetfillcolor{currentfill}%
\pgfsetlinewidth{0.000000pt}%
\definecolor{currentstroke}{rgb}{0.000000,0.000000,0.000000}%
\pgfsetstrokecolor{currentstroke}%
\pgfsetstrokeopacity{0.000000}%
\pgfsetdash{}{0pt}%
\pgfpathmoveto{\pgfqpoint{1.919219in}{0.761725in}}%
\pgfpathlineto{\pgfqpoint{1.884281in}{0.742465in}}%
\pgfpathlineto{\pgfqpoint{1.870147in}{0.767494in}}%
\pgfpathlineto{\pgfqpoint{1.858178in}{0.766124in}}%
\pgfpathlineto{\pgfqpoint{1.842140in}{0.796889in}}%
\pgfpathlineto{\pgfqpoint{1.873122in}{0.813238in}}%
\pgfpathlineto{\pgfqpoint{1.873743in}{0.830250in}}%
\pgfpathlineto{\pgfqpoint{1.900998in}{0.829156in}}%
\pgfpathlineto{\pgfqpoint{1.908521in}{0.806762in}}%
\pgfpathlineto{\pgfqpoint{1.913591in}{0.794961in}}%
\pgfpathlineto{\pgfqpoint{1.903937in}{0.789730in}}%
\pgfpathlineto{\pgfqpoint{1.919219in}{0.761725in}}%
\pgfpathclose%
\pgfusepath{fill}%
\end{pgfscope}%
\begin{pgfscope}%
\pgfpathrectangle{\pgfqpoint{0.100000in}{0.100000in}}{\pgfqpoint{3.608454in}{2.310000in}}%
\pgfusepath{clip}%
\pgfsetbuttcap%
\pgfsetmiterjoin%
\definecolor{currentfill}{rgb}{0.000000,0.580392,0.709804}%
\pgfsetfillcolor{currentfill}%
\pgfsetlinewidth{0.000000pt}%
\definecolor{currentstroke}{rgb}{0.000000,0.000000,0.000000}%
\pgfsetstrokecolor{currentstroke}%
\pgfsetstrokeopacity{0.000000}%
\pgfsetdash{}{0pt}%
\pgfpathmoveto{\pgfqpoint{3.111250in}{1.435310in}}%
\pgfpathlineto{\pgfqpoint{3.117131in}{1.434042in}}%
\pgfpathlineto{\pgfqpoint{3.119217in}{1.439782in}}%
\pgfpathlineto{\pgfqpoint{3.125843in}{1.434080in}}%
\pgfpathlineto{\pgfqpoint{3.127584in}{1.429231in}}%
\pgfpathlineto{\pgfqpoint{3.124006in}{1.419453in}}%
\pgfpathlineto{\pgfqpoint{3.131195in}{1.415085in}}%
\pgfpathlineto{\pgfqpoint{3.129858in}{1.404228in}}%
\pgfpathlineto{\pgfqpoint{3.125203in}{1.394430in}}%
\pgfpathlineto{\pgfqpoint{3.123623in}{1.385333in}}%
\pgfpathlineto{\pgfqpoint{3.115799in}{1.372179in}}%
\pgfpathlineto{\pgfqpoint{3.110807in}{1.366936in}}%
\pgfpathlineto{\pgfqpoint{3.078058in}{1.382882in}}%
\pgfpathlineto{\pgfqpoint{3.073594in}{1.376808in}}%
\pgfpathlineto{\pgfqpoint{3.062954in}{1.378601in}}%
\pgfpathlineto{\pgfqpoint{3.058026in}{1.385268in}}%
\pgfpathlineto{\pgfqpoint{3.050650in}{1.387261in}}%
\pgfpathlineto{\pgfqpoint{3.056652in}{1.404504in}}%
\pgfpathlineto{\pgfqpoint{3.055324in}{1.410700in}}%
\pgfpathlineto{\pgfqpoint{3.063909in}{1.418899in}}%
\pgfpathlineto{\pgfqpoint{3.084328in}{1.406748in}}%
\pgfpathlineto{\pgfqpoint{3.086938in}{1.414331in}}%
\pgfpathlineto{\pgfqpoint{3.095711in}{1.408904in}}%
\pgfpathlineto{\pgfqpoint{3.099019in}{1.419473in}}%
\pgfpathlineto{\pgfqpoint{3.103999in}{1.422556in}}%
\pgfpathlineto{\pgfqpoint{3.111250in}{1.435310in}}%
\pgfpathclose%
\pgfusepath{fill}%
\end{pgfscope}%
\begin{pgfscope}%
\pgfpathrectangle{\pgfqpoint{0.100000in}{0.100000in}}{\pgfqpoint{3.608454in}{2.310000in}}%
\pgfusepath{clip}%
\pgfsetbuttcap%
\pgfsetmiterjoin%
\definecolor{currentfill}{rgb}{0.000000,0.741176,0.629412}%
\pgfsetfillcolor{currentfill}%
\pgfsetlinewidth{0.000000pt}%
\definecolor{currentstroke}{rgb}{0.000000,0.000000,0.000000}%
\pgfsetstrokecolor{currentstroke}%
\pgfsetstrokeopacity{0.000000}%
\pgfsetdash{}{0pt}%
\pgfpathmoveto{\pgfqpoint{2.414649in}{1.160533in}}%
\pgfpathlineto{\pgfqpoint{2.410619in}{1.149798in}}%
\pgfpathlineto{\pgfqpoint{2.353073in}{1.147045in}}%
\pgfpathlineto{\pgfqpoint{2.352645in}{1.172837in}}%
\pgfpathlineto{\pgfqpoint{2.345750in}{1.172705in}}%
\pgfpathlineto{\pgfqpoint{2.345378in}{1.193313in}}%
\pgfpathlineto{\pgfqpoint{2.357912in}{1.194600in}}%
\pgfpathlineto{\pgfqpoint{2.361419in}{1.191422in}}%
\pgfpathlineto{\pgfqpoint{2.373090in}{1.191695in}}%
\pgfpathlineto{\pgfqpoint{2.375334in}{1.200990in}}%
\pgfpathlineto{\pgfqpoint{2.375052in}{1.209261in}}%
\pgfpathlineto{\pgfqpoint{2.386664in}{1.209942in}}%
\pgfpathlineto{\pgfqpoint{2.386546in}{1.213497in}}%
\pgfpathlineto{\pgfqpoint{2.407413in}{1.214350in}}%
\pgfpathlineto{\pgfqpoint{2.408011in}{1.196285in}}%
\pgfpathlineto{\pgfqpoint{2.415125in}{1.196618in}}%
\pgfpathlineto{\pgfqpoint{2.406457in}{1.183183in}}%
\pgfpathlineto{\pgfqpoint{2.413654in}{1.167791in}}%
\pgfpathlineto{\pgfqpoint{2.414649in}{1.160533in}}%
\pgfpathclose%
\pgfusepath{fill}%
\end{pgfscope}%
\begin{pgfscope}%
\pgfpathrectangle{\pgfqpoint{0.100000in}{0.100000in}}{\pgfqpoint{3.608454in}{2.310000in}}%
\pgfusepath{clip}%
\pgfsetbuttcap%
\pgfsetmiterjoin%
\definecolor{currentfill}{rgb}{0.000000,0.611765,0.694118}%
\pgfsetfillcolor{currentfill}%
\pgfsetlinewidth{0.000000pt}%
\definecolor{currentstroke}{rgb}{0.000000,0.000000,0.000000}%
\pgfsetstrokecolor{currentstroke}%
\pgfsetstrokeopacity{0.000000}%
\pgfsetdash{}{0pt}%
\pgfpathmoveto{\pgfqpoint{1.578022in}{2.163356in}}%
\pgfpathlineto{\pgfqpoint{1.630877in}{2.157836in}}%
\pgfpathlineto{\pgfqpoint{1.626106in}{2.109098in}}%
\pgfpathlineto{\pgfqpoint{1.595340in}{2.112255in}}%
\pgfpathlineto{\pgfqpoint{1.596092in}{2.119196in}}%
\pgfpathlineto{\pgfqpoint{1.589220in}{2.119949in}}%
\pgfpathlineto{\pgfqpoint{1.589978in}{2.126848in}}%
\pgfpathlineto{\pgfqpoint{1.578597in}{2.128096in}}%
\pgfpathlineto{\pgfqpoint{1.575926in}{2.135412in}}%
\pgfpathlineto{\pgfqpoint{1.578243in}{2.156172in}}%
\pgfpathlineto{\pgfqpoint{1.578022in}{2.163356in}}%
\pgfpathclose%
\pgfusepath{fill}%
\end{pgfscope}%
\begin{pgfscope}%
\pgfpathrectangle{\pgfqpoint{0.100000in}{0.100000in}}{\pgfqpoint{3.608454in}{2.310000in}}%
\pgfusepath{clip}%
\pgfsetbuttcap%
\pgfsetmiterjoin%
\definecolor{currentfill}{rgb}{0.000000,0.584314,0.707843}%
\pgfsetfillcolor{currentfill}%
\pgfsetlinewidth{0.000000pt}%
\definecolor{currentstroke}{rgb}{0.000000,0.000000,0.000000}%
\pgfsetstrokecolor{currentstroke}%
\pgfsetstrokeopacity{0.000000}%
\pgfsetdash{}{0pt}%
\pgfpathmoveto{\pgfqpoint{2.118427in}{1.958403in}}%
\pgfpathlineto{\pgfqpoint{2.097902in}{1.958459in}}%
\pgfpathlineto{\pgfqpoint{2.097851in}{1.951504in}}%
\pgfpathlineto{\pgfqpoint{2.042559in}{1.951936in}}%
\pgfpathlineto{\pgfqpoint{2.042858in}{1.972645in}}%
\pgfpathlineto{\pgfqpoint{2.041913in}{1.986517in}}%
\pgfpathlineto{\pgfqpoint{2.048844in}{1.986478in}}%
\pgfpathlineto{\pgfqpoint{2.049145in}{2.014243in}}%
\pgfpathlineto{\pgfqpoint{2.076864in}{2.014087in}}%
\pgfpathlineto{\pgfqpoint{2.075622in}{2.027945in}}%
\pgfpathlineto{\pgfqpoint{2.075577in}{2.048780in}}%
\pgfpathlineto{\pgfqpoint{2.075068in}{2.055715in}}%
\pgfpathlineto{\pgfqpoint{2.074816in}{2.097239in}}%
\pgfpathlineto{\pgfqpoint{2.095537in}{2.097237in}}%
\pgfpathlineto{\pgfqpoint{2.095560in}{2.083291in}}%
\pgfpathlineto{\pgfqpoint{2.137206in}{2.083483in}}%
\pgfpathlineto{\pgfqpoint{2.138015in}{2.041726in}}%
\pgfpathlineto{\pgfqpoint{2.138354in}{2.007024in}}%
\pgfpathlineto{\pgfqpoint{2.124628in}{2.006820in}}%
\pgfpathlineto{\pgfqpoint{2.125367in}{1.958378in}}%
\pgfpathlineto{\pgfqpoint{2.118427in}{1.958403in}}%
\pgfpathclose%
\pgfusepath{fill}%
\end{pgfscope}%
\begin{pgfscope}%
\pgfpathrectangle{\pgfqpoint{0.100000in}{0.100000in}}{\pgfqpoint{3.608454in}{2.310000in}}%
\pgfusepath{clip}%
\pgfsetbuttcap%
\pgfsetmiterjoin%
\definecolor{currentfill}{rgb}{0.000000,0.666667,0.666667}%
\pgfsetfillcolor{currentfill}%
\pgfsetlinewidth{0.000000pt}%
\definecolor{currentstroke}{rgb}{0.000000,0.000000,0.000000}%
\pgfsetstrokecolor{currentstroke}%
\pgfsetstrokeopacity{0.000000}%
\pgfsetdash{}{0pt}%
\pgfpathmoveto{\pgfqpoint{2.272778in}{0.593877in}}%
\pgfpathlineto{\pgfqpoint{2.271892in}{0.630559in}}%
\pgfpathlineto{\pgfqpoint{2.269704in}{0.666105in}}%
\pgfpathlineto{\pgfqpoint{2.272007in}{0.674022in}}%
\pgfpathlineto{\pgfqpoint{2.271317in}{0.699125in}}%
\pgfpathlineto{\pgfqpoint{2.278789in}{0.700272in}}%
\pgfpathlineto{\pgfqpoint{2.285347in}{0.707925in}}%
\pgfpathlineto{\pgfqpoint{2.292928in}{0.702917in}}%
\pgfpathlineto{\pgfqpoint{2.296084in}{0.696005in}}%
\pgfpathlineto{\pgfqpoint{2.325022in}{0.696944in}}%
\pgfpathlineto{\pgfqpoint{2.331030in}{0.685463in}}%
\pgfpathlineto{\pgfqpoint{2.329801in}{0.674430in}}%
\pgfpathlineto{\pgfqpoint{2.331431in}{0.670010in}}%
\pgfpathlineto{\pgfqpoint{2.338216in}{0.665019in}}%
\pgfpathlineto{\pgfqpoint{2.339922in}{0.654462in}}%
\pgfpathlineto{\pgfqpoint{2.344148in}{0.649181in}}%
\pgfpathlineto{\pgfqpoint{2.350223in}{0.648417in}}%
\pgfpathlineto{\pgfqpoint{2.351410in}{0.638299in}}%
\pgfpathlineto{\pgfqpoint{2.368222in}{0.632799in}}%
\pgfpathlineto{\pgfqpoint{2.365882in}{0.630690in}}%
\pgfpathlineto{\pgfqpoint{2.371581in}{0.618635in}}%
\pgfpathlineto{\pgfqpoint{2.376744in}{0.617237in}}%
\pgfpathlineto{\pgfqpoint{2.378486in}{0.611623in}}%
\pgfpathlineto{\pgfqpoint{2.372197in}{0.609200in}}%
\pgfpathlineto{\pgfqpoint{2.357771in}{0.613019in}}%
\pgfpathlineto{\pgfqpoint{2.358501in}{0.617112in}}%
\pgfpathlineto{\pgfqpoint{2.351303in}{0.626566in}}%
\pgfpathlineto{\pgfqpoint{2.340753in}{0.625307in}}%
\pgfpathlineto{\pgfqpoint{2.335052in}{0.617153in}}%
\pgfpathlineto{\pgfqpoint{2.323648in}{0.606472in}}%
\pgfpathlineto{\pgfqpoint{2.323245in}{0.615819in}}%
\pgfpathlineto{\pgfqpoint{2.312586in}{0.611195in}}%
\pgfpathlineto{\pgfqpoint{2.305299in}{0.605725in}}%
\pgfpathlineto{\pgfqpoint{2.307552in}{0.598624in}}%
\pgfpathlineto{\pgfqpoint{2.315848in}{0.597697in}}%
\pgfpathlineto{\pgfqpoint{2.326725in}{0.599490in}}%
\pgfpathlineto{\pgfqpoint{2.336035in}{0.593898in}}%
\pgfpathlineto{\pgfqpoint{2.328591in}{0.587197in}}%
\pgfpathlineto{\pgfqpoint{2.311674in}{0.595112in}}%
\pgfpathlineto{\pgfqpoint{2.304943in}{0.595012in}}%
\pgfpathlineto{\pgfqpoint{2.294524in}{0.590708in}}%
\pgfpathlineto{\pgfqpoint{2.272778in}{0.593877in}}%
\pgfpathclose%
\pgfusepath{fill}%
\end{pgfscope}%
\begin{pgfscope}%
\pgfpathrectangle{\pgfqpoint{0.100000in}{0.100000in}}{\pgfqpoint{3.608454in}{2.310000in}}%
\pgfusepath{clip}%
\pgfsetbuttcap%
\pgfsetmiterjoin%
\definecolor{currentfill}{rgb}{0.000000,0.682353,0.658824}%
\pgfsetfillcolor{currentfill}%
\pgfsetlinewidth{0.000000pt}%
\definecolor{currentstroke}{rgb}{0.000000,0.000000,0.000000}%
\pgfsetstrokecolor{currentstroke}%
\pgfsetstrokeopacity{0.000000}%
\pgfsetdash{}{0pt}%
\pgfpathmoveto{\pgfqpoint{1.836840in}{0.475148in}}%
\pgfpathlineto{\pgfqpoint{1.795200in}{0.474847in}}%
\pgfpathlineto{\pgfqpoint{1.795506in}{0.488863in}}%
\pgfpathlineto{\pgfqpoint{1.797091in}{0.523880in}}%
\pgfpathlineto{\pgfqpoint{1.796154in}{0.523908in}}%
\pgfpathlineto{\pgfqpoint{1.797569in}{0.560061in}}%
\pgfpathlineto{\pgfqpoint{1.840202in}{0.558059in}}%
\pgfpathlineto{\pgfqpoint{1.839000in}{0.522528in}}%
\pgfpathlineto{\pgfqpoint{1.836840in}{0.475148in}}%
\pgfpathclose%
\pgfusepath{fill}%
\end{pgfscope}%
\begin{pgfscope}%
\pgfpathrectangle{\pgfqpoint{0.100000in}{0.100000in}}{\pgfqpoint{3.608454in}{2.310000in}}%
\pgfusepath{clip}%
\pgfsetbuttcap%
\pgfsetmiterjoin%
\definecolor{currentfill}{rgb}{0.000000,0.588235,0.705882}%
\pgfsetfillcolor{currentfill}%
\pgfsetlinewidth{0.000000pt}%
\definecolor{currentstroke}{rgb}{0.000000,0.000000,0.000000}%
\pgfsetstrokecolor{currentstroke}%
\pgfsetstrokeopacity{0.000000}%
\pgfsetdash{}{0pt}%
\pgfpathmoveto{\pgfqpoint{2.966459in}{0.731456in}}%
\pgfpathlineto{\pgfqpoint{2.957900in}{0.730883in}}%
\pgfpathlineto{\pgfqpoint{2.955222in}{0.740323in}}%
\pgfpathlineto{\pgfqpoint{2.943923in}{0.740325in}}%
\pgfpathlineto{\pgfqpoint{2.937427in}{0.747697in}}%
\pgfpathlineto{\pgfqpoint{2.928578in}{0.749275in}}%
\pgfpathlineto{\pgfqpoint{2.925268in}{0.774204in}}%
\pgfpathlineto{\pgfqpoint{2.920160in}{0.773477in}}%
\pgfpathlineto{\pgfqpoint{2.920094in}{0.780812in}}%
\pgfpathlineto{\pgfqpoint{2.911798in}{0.789691in}}%
\pgfpathlineto{\pgfqpoint{2.910462in}{0.795395in}}%
\pgfpathlineto{\pgfqpoint{2.912735in}{0.796174in}}%
\pgfpathlineto{\pgfqpoint{2.914876in}{0.807165in}}%
\pgfpathlineto{\pgfqpoint{2.918401in}{0.811582in}}%
\pgfpathlineto{\pgfqpoint{2.917347in}{0.821092in}}%
\pgfpathlineto{\pgfqpoint{2.926243in}{0.822499in}}%
\pgfpathlineto{\pgfqpoint{2.929182in}{0.813904in}}%
\pgfpathlineto{\pgfqpoint{2.953500in}{0.817959in}}%
\pgfpathlineto{\pgfqpoint{2.970476in}{0.810318in}}%
\pgfpathlineto{\pgfqpoint{2.978722in}{0.804662in}}%
\pgfpathlineto{\pgfqpoint{2.983849in}{0.804589in}}%
\pgfpathlineto{\pgfqpoint{2.986023in}{0.797823in}}%
\pgfpathlineto{\pgfqpoint{2.990030in}{0.793119in}}%
\pgfpathlineto{\pgfqpoint{2.983748in}{0.789742in}}%
\pgfpathlineto{\pgfqpoint{2.977554in}{0.783183in}}%
\pgfpathlineto{\pgfqpoint{2.977965in}{0.773429in}}%
\pgfpathlineto{\pgfqpoint{2.983960in}{0.768316in}}%
\pgfpathlineto{\pgfqpoint{2.964698in}{0.765826in}}%
\pgfpathlineto{\pgfqpoint{2.966799in}{0.748504in}}%
\pgfpathlineto{\pgfqpoint{2.985291in}{0.750228in}}%
\pgfpathlineto{\pgfqpoint{2.983300in}{0.732552in}}%
\pgfpathlineto{\pgfqpoint{2.966459in}{0.731456in}}%
\pgfpathclose%
\pgfusepath{fill}%
\end{pgfscope}%
\begin{pgfscope}%
\pgfpathrectangle{\pgfqpoint{0.100000in}{0.100000in}}{\pgfqpoint{3.608454in}{2.310000in}}%
\pgfusepath{clip}%
\pgfsetbuttcap%
\pgfsetmiterjoin%
\definecolor{currentfill}{rgb}{0.000000,0.670588,0.664706}%
\pgfsetfillcolor{currentfill}%
\pgfsetlinewidth{0.000000pt}%
\definecolor{currentstroke}{rgb}{0.000000,0.000000,0.000000}%
\pgfsetstrokecolor{currentstroke}%
\pgfsetstrokeopacity{0.000000}%
\pgfsetdash{}{0pt}%
\pgfpathmoveto{\pgfqpoint{0.579411in}{1.933417in}}%
\pgfpathlineto{\pgfqpoint{0.571017in}{1.931177in}}%
\pgfpathlineto{\pgfqpoint{0.568235in}{1.922352in}}%
\pgfpathlineto{\pgfqpoint{0.534705in}{1.932520in}}%
\pgfpathlineto{\pgfqpoint{0.537209in}{1.940659in}}%
\pgfpathlineto{\pgfqpoint{0.517068in}{1.946672in}}%
\pgfpathlineto{\pgfqpoint{0.517042in}{1.952171in}}%
\pgfpathlineto{\pgfqpoint{0.521206in}{1.965301in}}%
\pgfpathlineto{\pgfqpoint{0.509770in}{1.969016in}}%
\pgfpathlineto{\pgfqpoint{0.501091in}{1.975934in}}%
\pgfpathlineto{\pgfqpoint{0.498412in}{1.984367in}}%
\pgfpathlineto{\pgfqpoint{0.487714in}{1.989945in}}%
\pgfpathlineto{\pgfqpoint{0.479271in}{1.985850in}}%
\pgfpathlineto{\pgfqpoint{0.467729in}{1.989714in}}%
\pgfpathlineto{\pgfqpoint{0.475658in}{2.007052in}}%
\pgfpathlineto{\pgfqpoint{0.480357in}{2.020001in}}%
\pgfpathlineto{\pgfqpoint{0.498912in}{2.014543in}}%
\pgfpathlineto{\pgfqpoint{0.504063in}{2.019289in}}%
\pgfpathlineto{\pgfqpoint{0.505570in}{2.024850in}}%
\pgfpathlineto{\pgfqpoint{0.512206in}{2.022672in}}%
\pgfpathlineto{\pgfqpoint{0.519006in}{2.044536in}}%
\pgfpathlineto{\pgfqpoint{0.512881in}{2.048028in}}%
\pgfpathlineto{\pgfqpoint{0.521244in}{2.073470in}}%
\pgfpathlineto{\pgfqpoint{0.518018in}{2.074545in}}%
\pgfpathlineto{\pgfqpoint{0.521499in}{2.085093in}}%
\pgfpathlineto{\pgfqpoint{0.538605in}{2.079539in}}%
\pgfpathlineto{\pgfqpoint{0.543848in}{2.096029in}}%
\pgfpathlineto{\pgfqpoint{0.561348in}{2.090351in}}%
\pgfpathlineto{\pgfqpoint{0.566154in}{2.086439in}}%
\pgfpathlineto{\pgfqpoint{0.568024in}{2.080965in}}%
\pgfpathlineto{\pgfqpoint{0.576179in}{2.072349in}}%
\pgfpathlineto{\pgfqpoint{0.575705in}{2.064800in}}%
\pgfpathlineto{\pgfqpoint{0.572541in}{2.061484in}}%
\pgfpathlineto{\pgfqpoint{0.575268in}{2.054730in}}%
\pgfpathlineto{\pgfqpoint{0.580214in}{2.049876in}}%
\pgfpathlineto{\pgfqpoint{0.581481in}{2.042623in}}%
\pgfpathlineto{\pgfqpoint{0.587828in}{2.036379in}}%
\pgfpathlineto{\pgfqpoint{0.623427in}{2.025636in}}%
\pgfpathlineto{\pgfqpoint{0.620648in}{2.021281in}}%
\pgfpathlineto{\pgfqpoint{0.616042in}{2.014692in}}%
\pgfpathlineto{\pgfqpoint{0.613506in}{2.004223in}}%
\pgfpathlineto{\pgfqpoint{0.608267in}{1.997300in}}%
\pgfpathlineto{\pgfqpoint{0.605387in}{1.987819in}}%
\pgfpathlineto{\pgfqpoint{0.605729in}{1.977984in}}%
\pgfpathlineto{\pgfqpoint{0.603007in}{1.965607in}}%
\pgfpathlineto{\pgfqpoint{0.595256in}{1.957298in}}%
\pgfpathlineto{\pgfqpoint{0.586510in}{1.951287in}}%
\pgfpathlineto{\pgfqpoint{0.585151in}{1.943983in}}%
\pgfpathlineto{\pgfqpoint{0.579411in}{1.933417in}}%
\pgfpathclose%
\pgfusepath{fill}%
\end{pgfscope}%
\begin{pgfscope}%
\pgfpathrectangle{\pgfqpoint{0.100000in}{0.100000in}}{\pgfqpoint{3.608454in}{2.310000in}}%
\pgfusepath{clip}%
\pgfsetbuttcap%
\pgfsetmiterjoin%
\definecolor{currentfill}{rgb}{0.000000,0.545098,0.727451}%
\pgfsetfillcolor{currentfill}%
\pgfsetlinewidth{0.000000pt}%
\definecolor{currentstroke}{rgb}{0.000000,0.000000,0.000000}%
\pgfsetstrokecolor{currentstroke}%
\pgfsetstrokeopacity{0.000000}%
\pgfsetdash{}{0pt}%
\pgfpathmoveto{\pgfqpoint{2.765201in}{0.880136in}}%
\pgfpathlineto{\pgfqpoint{2.755362in}{0.877808in}}%
\pgfpathlineto{\pgfqpoint{2.734886in}{0.875470in}}%
\pgfpathlineto{\pgfqpoint{2.731773in}{0.905420in}}%
\pgfpathlineto{\pgfqpoint{2.727775in}{0.904950in}}%
\pgfpathlineto{\pgfqpoint{2.725270in}{0.935776in}}%
\pgfpathlineto{\pgfqpoint{2.717216in}{0.935165in}}%
\pgfpathlineto{\pgfqpoint{2.714599in}{0.941836in}}%
\pgfpathlineto{\pgfqpoint{2.701405in}{0.940587in}}%
\pgfpathlineto{\pgfqpoint{2.697028in}{0.947184in}}%
\pgfpathlineto{\pgfqpoint{2.690775in}{0.946928in}}%
\pgfpathlineto{\pgfqpoint{2.696367in}{0.955422in}}%
\pgfpathlineto{\pgfqpoint{2.694783in}{0.960330in}}%
\pgfpathlineto{\pgfqpoint{2.705960in}{0.970065in}}%
\pgfpathlineto{\pgfqpoint{2.715165in}{0.972447in}}%
\pgfpathlineto{\pgfqpoint{2.728983in}{0.972881in}}%
\pgfpathlineto{\pgfqpoint{2.737463in}{0.974424in}}%
\pgfpathlineto{\pgfqpoint{2.738788in}{0.969584in}}%
\pgfpathlineto{\pgfqpoint{2.760784in}{0.972182in}}%
\pgfpathlineto{\pgfqpoint{2.762398in}{0.964970in}}%
\pgfpathlineto{\pgfqpoint{2.771603in}{0.963496in}}%
\pgfpathlineto{\pgfqpoint{2.773147in}{0.947109in}}%
\pgfpathlineto{\pgfqpoint{2.778830in}{0.942946in}}%
\pgfpathlineto{\pgfqpoint{2.791999in}{0.944257in}}%
\pgfpathlineto{\pgfqpoint{2.796122in}{0.932528in}}%
\pgfpathlineto{\pgfqpoint{2.802892in}{0.925244in}}%
\pgfpathlineto{\pgfqpoint{2.802911in}{0.922322in}}%
\pgfpathlineto{\pgfqpoint{2.779766in}{0.919893in}}%
\pgfpathlineto{\pgfqpoint{2.782280in}{0.891779in}}%
\pgfpathlineto{\pgfqpoint{2.760875in}{0.889582in}}%
\pgfpathlineto{\pgfqpoint{2.765201in}{0.880136in}}%
\pgfpathclose%
\pgfusepath{fill}%
\end{pgfscope}%
\begin{pgfscope}%
\pgfpathrectangle{\pgfqpoint{0.100000in}{0.100000in}}{\pgfqpoint{3.608454in}{2.310000in}}%
\pgfusepath{clip}%
\pgfsetbuttcap%
\pgfsetmiterjoin%
\definecolor{currentfill}{rgb}{0.000000,0.556863,0.721569}%
\pgfsetfillcolor{currentfill}%
\pgfsetlinewidth{0.000000pt}%
\definecolor{currentstroke}{rgb}{0.000000,0.000000,0.000000}%
\pgfsetstrokecolor{currentstroke}%
\pgfsetstrokeopacity{0.000000}%
\pgfsetdash{}{0pt}%
\pgfpathmoveto{\pgfqpoint{1.305818in}{1.299625in}}%
\pgfpathlineto{\pgfqpoint{1.298775in}{1.307121in}}%
\pgfpathlineto{\pgfqpoint{1.284736in}{1.306313in}}%
\pgfpathlineto{\pgfqpoint{1.282717in}{1.312481in}}%
\pgfpathlineto{\pgfqpoint{1.248158in}{1.317741in}}%
\pgfpathlineto{\pgfqpoint{1.234133in}{1.318758in}}%
\pgfpathlineto{\pgfqpoint{1.237566in}{1.340907in}}%
\pgfpathlineto{\pgfqpoint{1.237792in}{1.349736in}}%
\pgfpathlineto{\pgfqpoint{1.240533in}{1.367269in}}%
\pgfpathlineto{\pgfqpoint{1.251645in}{1.434868in}}%
\pgfpathlineto{\pgfqpoint{1.290367in}{1.428918in}}%
\pgfpathlineto{\pgfqpoint{1.349566in}{1.420460in}}%
\pgfpathlineto{\pgfqpoint{1.346932in}{1.416687in}}%
\pgfpathlineto{\pgfqpoint{1.350593in}{1.411552in}}%
\pgfpathlineto{\pgfqpoint{1.343717in}{1.409436in}}%
\pgfpathlineto{\pgfqpoint{1.333852in}{1.337711in}}%
\pgfpathlineto{\pgfqpoint{1.325556in}{1.338863in}}%
\pgfpathlineto{\pgfqpoint{1.328146in}{1.332828in}}%
\pgfpathlineto{\pgfqpoint{1.328108in}{1.320759in}}%
\pgfpathlineto{\pgfqpoint{1.325950in}{1.311919in}}%
\pgfpathlineto{\pgfqpoint{1.314836in}{1.308720in}}%
\pgfpathlineto{\pgfqpoint{1.305818in}{1.299625in}}%
\pgfpathclose%
\pgfusepath{fill}%
\end{pgfscope}%
\begin{pgfscope}%
\pgfpathrectangle{\pgfqpoint{0.100000in}{0.100000in}}{\pgfqpoint{3.608454in}{2.310000in}}%
\pgfusepath{clip}%
\pgfsetbuttcap%
\pgfsetmiterjoin%
\definecolor{currentfill}{rgb}{0.000000,0.827451,0.586275}%
\pgfsetfillcolor{currentfill}%
\pgfsetlinewidth{0.000000pt}%
\definecolor{currentstroke}{rgb}{0.000000,0.000000,0.000000}%
\pgfsetstrokecolor{currentstroke}%
\pgfsetstrokeopacity{0.000000}%
\pgfsetdash{}{0pt}%
\pgfpathmoveto{\pgfqpoint{1.124077in}{0.559590in}}%
\pgfpathlineto{\pgfqpoint{1.110388in}{0.567904in}}%
\pgfpathlineto{\pgfqpoint{1.083089in}{0.584700in}}%
\pgfpathlineto{\pgfqpoint{1.074192in}{0.589904in}}%
\pgfpathlineto{\pgfqpoint{1.066205in}{0.578011in}}%
\pgfpathlineto{\pgfqpoint{1.056440in}{0.584227in}}%
\pgfpathlineto{\pgfqpoint{1.052163in}{0.586365in}}%
\pgfpathlineto{\pgfqpoint{1.026608in}{0.603847in}}%
\pgfpathlineto{\pgfqpoint{0.997892in}{0.624739in}}%
\pgfpathlineto{\pgfqpoint{0.976297in}{0.641465in}}%
\pgfpathlineto{\pgfqpoint{0.976829in}{0.642165in}}%
\pgfpathlineto{\pgfqpoint{0.968183in}{0.648986in}}%
\pgfpathlineto{\pgfqpoint{0.969832in}{0.651052in}}%
\pgfpathlineto{\pgfqpoint{0.961484in}{0.657889in}}%
\pgfpathlineto{\pgfqpoint{0.963280in}{0.660093in}}%
\pgfpathlineto{\pgfqpoint{0.954477in}{0.667338in}}%
\pgfpathlineto{\pgfqpoint{0.940158in}{0.679506in}}%
\pgfpathlineto{\pgfqpoint{0.941559in}{0.681130in}}%
\pgfpathlineto{\pgfqpoint{0.937775in}{0.684411in}}%
\pgfpathlineto{\pgfqpoint{0.936363in}{0.682792in}}%
\pgfpathlineto{\pgfqpoint{0.927789in}{0.690351in}}%
\pgfpathlineto{\pgfqpoint{0.929228in}{0.691967in}}%
\pgfpathlineto{\pgfqpoint{0.925549in}{0.695258in}}%
\pgfpathlineto{\pgfqpoint{0.924102in}{0.693649in}}%
\pgfpathlineto{\pgfqpoint{0.919895in}{0.697448in}}%
\pgfpathlineto{\pgfqpoint{0.921351in}{0.699050in}}%
\pgfpathlineto{\pgfqpoint{0.915131in}{0.704745in}}%
\pgfpathlineto{\pgfqpoint{0.913662in}{0.703154in}}%
\pgfpathlineto{\pgfqpoint{0.898458in}{0.717465in}}%
\pgfpathlineto{\pgfqpoint{0.894146in}{0.712972in}}%
\pgfpathlineto{\pgfqpoint{0.886741in}{0.720168in}}%
\pgfpathlineto{\pgfqpoint{0.883880in}{0.727259in}}%
\pgfpathlineto{\pgfqpoint{0.884750in}{0.730564in}}%
\pgfpathlineto{\pgfqpoint{0.884173in}{0.733970in}}%
\pgfpathlineto{\pgfqpoint{0.881741in}{0.738770in}}%
\pgfpathlineto{\pgfqpoint{0.884973in}{0.737504in}}%
\pgfpathlineto{\pgfqpoint{0.888394in}{0.737302in}}%
\pgfpathlineto{\pgfqpoint{0.891146in}{0.738897in}}%
\pgfpathlineto{\pgfqpoint{0.897248in}{0.744755in}}%
\pgfpathlineto{\pgfqpoint{0.899391in}{0.744053in}}%
\pgfpathlineto{\pgfqpoint{0.904078in}{0.738998in}}%
\pgfpathlineto{\pgfqpoint{0.914603in}{0.732006in}}%
\pgfpathlineto{\pgfqpoint{0.917793in}{0.730737in}}%
\pgfpathlineto{\pgfqpoint{0.924697in}{0.730323in}}%
\pgfpathlineto{\pgfqpoint{0.934212in}{0.732247in}}%
\pgfpathlineto{\pgfqpoint{0.938989in}{0.735849in}}%
\pgfpathlineto{\pgfqpoint{0.940161in}{0.737749in}}%
\pgfpathlineto{\pgfqpoint{0.943533in}{0.740578in}}%
\pgfpathlineto{\pgfqpoint{0.945793in}{0.741496in}}%
\pgfpathlineto{\pgfqpoint{0.952899in}{0.741432in}}%
\pgfpathlineto{\pgfqpoint{0.955066in}{0.743565in}}%
\pgfpathlineto{\pgfqpoint{0.959761in}{0.743939in}}%
\pgfpathlineto{\pgfqpoint{0.965368in}{0.745161in}}%
\pgfpathlineto{\pgfqpoint{0.968573in}{0.740621in}}%
\pgfpathlineto{\pgfqpoint{0.974554in}{0.739226in}}%
\pgfpathlineto{\pgfqpoint{0.985159in}{0.739890in}}%
\pgfpathlineto{\pgfqpoint{0.995988in}{0.741765in}}%
\pgfpathlineto{\pgfqpoint{0.999534in}{0.740148in}}%
\pgfpathlineto{\pgfqpoint{0.999964in}{0.737676in}}%
\pgfpathlineto{\pgfqpoint{1.004192in}{0.734276in}}%
\pgfpathlineto{\pgfqpoint{1.008232in}{0.732003in}}%
\pgfpathlineto{\pgfqpoint{1.011631in}{0.730934in}}%
\pgfpathlineto{\pgfqpoint{1.018774in}{0.731477in}}%
\pgfpathlineto{\pgfqpoint{1.028230in}{0.734980in}}%
\pgfpathlineto{\pgfqpoint{1.031476in}{0.735474in}}%
\pgfpathlineto{\pgfqpoint{1.033074in}{0.730543in}}%
\pgfpathlineto{\pgfqpoint{1.032211in}{0.727412in}}%
\pgfpathlineto{\pgfqpoint{1.034678in}{0.727466in}}%
\pgfpathlineto{\pgfqpoint{1.036005in}{0.725068in}}%
\pgfpathlineto{\pgfqpoint{1.036019in}{0.722231in}}%
\pgfpathlineto{\pgfqpoint{1.030729in}{0.720182in}}%
\pgfpathlineto{\pgfqpoint{1.026664in}{0.721652in}}%
\pgfpathlineto{\pgfqpoint{1.027301in}{0.717005in}}%
\pgfpathlineto{\pgfqpoint{1.031426in}{0.715395in}}%
\pgfpathlineto{\pgfqpoint{1.033829in}{0.717336in}}%
\pgfpathlineto{\pgfqpoint{1.039293in}{0.717228in}}%
\pgfpathlineto{\pgfqpoint{1.042163in}{0.713874in}}%
\pgfpathlineto{\pgfqpoint{1.038932in}{0.710194in}}%
\pgfpathlineto{\pgfqpoint{1.038327in}{0.708052in}}%
\pgfpathlineto{\pgfqpoint{1.040735in}{0.706277in}}%
\pgfpathlineto{\pgfqpoint{1.041143in}{0.703816in}}%
\pgfpathlineto{\pgfqpoint{1.044548in}{0.704736in}}%
\pgfpathlineto{\pgfqpoint{1.049310in}{0.702353in}}%
\pgfpathlineto{\pgfqpoint{1.052657in}{0.701365in}}%
\pgfpathlineto{\pgfqpoint{1.054785in}{0.696074in}}%
\pgfpathlineto{\pgfqpoint{1.056904in}{0.696328in}}%
\pgfpathlineto{\pgfqpoint{1.058734in}{0.693326in}}%
\pgfpathlineto{\pgfqpoint{1.052615in}{0.689071in}}%
\pgfpathlineto{\pgfqpoint{1.050791in}{0.686844in}}%
\pgfpathlineto{\pgfqpoint{1.057270in}{0.682397in}}%
\pgfpathlineto{\pgfqpoint{1.056506in}{0.680783in}}%
\pgfpathlineto{\pgfqpoint{1.053335in}{0.680298in}}%
\pgfpathlineto{\pgfqpoint{1.058156in}{0.676641in}}%
\pgfpathlineto{\pgfqpoint{1.058655in}{0.673622in}}%
\pgfpathlineto{\pgfqpoint{1.061740in}{0.674956in}}%
\pgfpathlineto{\pgfqpoint{1.066675in}{0.673514in}}%
\pgfpathlineto{\pgfqpoint{1.067308in}{0.671656in}}%
\pgfpathlineto{\pgfqpoint{1.064920in}{0.669776in}}%
\pgfpathlineto{\pgfqpoint{1.069974in}{0.668359in}}%
\pgfpathlineto{\pgfqpoint{1.073388in}{0.668942in}}%
\pgfpathlineto{\pgfqpoint{1.077468in}{0.664660in}}%
\pgfpathlineto{\pgfqpoint{1.078685in}{0.664612in}}%
\pgfpathlineto{\pgfqpoint{1.079652in}{0.659570in}}%
\pgfpathlineto{\pgfqpoint{1.082792in}{0.657912in}}%
\pgfpathlineto{\pgfqpoint{1.082915in}{0.655259in}}%
\pgfpathlineto{\pgfqpoint{1.085209in}{0.653941in}}%
\pgfpathlineto{\pgfqpoint{1.085707in}{0.649109in}}%
\pgfpathlineto{\pgfqpoint{1.092374in}{0.641077in}}%
\pgfpathlineto{\pgfqpoint{1.095064in}{0.640992in}}%
\pgfpathlineto{\pgfqpoint{1.098844in}{0.638529in}}%
\pgfpathlineto{\pgfqpoint{1.101206in}{0.635275in}}%
\pgfpathlineto{\pgfqpoint{1.104111in}{0.633554in}}%
\pgfpathlineto{\pgfqpoint{1.106568in}{0.626099in}}%
\pgfpathlineto{\pgfqpoint{1.108579in}{0.623361in}}%
\pgfpathlineto{\pgfqpoint{1.113936in}{0.621999in}}%
\pgfpathlineto{\pgfqpoint{1.120984in}{0.619591in}}%
\pgfpathlineto{\pgfqpoint{1.123831in}{0.619357in}}%
\pgfpathlineto{\pgfqpoint{1.127068in}{0.617346in}}%
\pgfpathlineto{\pgfqpoint{1.130555in}{0.612352in}}%
\pgfpathlineto{\pgfqpoint{1.132716in}{0.606694in}}%
\pgfpathlineto{\pgfqpoint{1.132681in}{0.605047in}}%
\pgfpathlineto{\pgfqpoint{1.134831in}{0.600830in}}%
\pgfpathlineto{\pgfqpoint{1.137136in}{0.597981in}}%
\pgfpathlineto{\pgfqpoint{1.137981in}{0.595190in}}%
\pgfpathlineto{\pgfqpoint{1.141010in}{0.591617in}}%
\pgfpathlineto{\pgfqpoint{1.139745in}{0.587853in}}%
\pgfpathlineto{\pgfqpoint{1.124077in}{0.559590in}}%
\pgfpathclose%
\pgfusepath{fill}%
\end{pgfscope}%
\begin{pgfscope}%
\pgfpathrectangle{\pgfqpoint{0.100000in}{0.100000in}}{\pgfqpoint{3.608454in}{2.310000in}}%
\pgfusepath{clip}%
\pgfsetbuttcap%
\pgfsetmiterjoin%
\definecolor{currentfill}{rgb}{0.000000,0.776471,0.611765}%
\pgfsetfillcolor{currentfill}%
\pgfsetlinewidth{0.000000pt}%
\definecolor{currentstroke}{rgb}{0.000000,0.000000,0.000000}%
\pgfsetstrokecolor{currentstroke}%
\pgfsetstrokeopacity{0.000000}%
\pgfsetdash{}{0pt}%
\pgfpathmoveto{\pgfqpoint{2.523585in}{1.228784in}}%
\pgfpathlineto{\pgfqpoint{2.519923in}{1.230014in}}%
\pgfpathlineto{\pgfqpoint{2.519105in}{1.243921in}}%
\pgfpathlineto{\pgfqpoint{2.500587in}{1.242556in}}%
\pgfpathlineto{\pgfqpoint{2.499015in}{1.266823in}}%
\pgfpathlineto{\pgfqpoint{2.519677in}{1.268419in}}%
\pgfpathlineto{\pgfqpoint{2.534479in}{1.269453in}}%
\pgfpathlineto{\pgfqpoint{2.537345in}{1.268351in}}%
\pgfpathlineto{\pgfqpoint{2.541890in}{1.261426in}}%
\pgfpathlineto{\pgfqpoint{2.534558in}{1.250105in}}%
\pgfpathlineto{\pgfqpoint{2.540147in}{1.235271in}}%
\pgfpathlineto{\pgfqpoint{2.528115in}{1.232902in}}%
\pgfpathlineto{\pgfqpoint{2.523585in}{1.228784in}}%
\pgfpathclose%
\pgfusepath{fill}%
\end{pgfscope}%
\begin{pgfscope}%
\pgfpathrectangle{\pgfqpoint{0.100000in}{0.100000in}}{\pgfqpoint{3.608454in}{2.310000in}}%
\pgfusepath{clip}%
\pgfsetbuttcap%
\pgfsetmiterjoin%
\definecolor{currentfill}{rgb}{0.000000,0.643137,0.678431}%
\pgfsetfillcolor{currentfill}%
\pgfsetlinewidth{0.000000pt}%
\definecolor{currentstroke}{rgb}{0.000000,0.000000,0.000000}%
\pgfsetstrokecolor{currentstroke}%
\pgfsetstrokeopacity{0.000000}%
\pgfsetdash{}{0pt}%
\pgfpathmoveto{\pgfqpoint{2.468504in}{0.944341in}}%
\pgfpathlineto{\pgfqpoint{2.451281in}{0.943313in}}%
\pgfpathlineto{\pgfqpoint{2.440822in}{0.945153in}}%
\pgfpathlineto{\pgfqpoint{2.441137in}{0.939280in}}%
\pgfpathlineto{\pgfqpoint{2.427361in}{0.937371in}}%
\pgfpathlineto{\pgfqpoint{2.427713in}{0.930421in}}%
\pgfpathlineto{\pgfqpoint{2.416509in}{0.931498in}}%
\pgfpathlineto{\pgfqpoint{2.424326in}{0.937210in}}%
\pgfpathlineto{\pgfqpoint{2.406598in}{0.936460in}}%
\pgfpathlineto{\pgfqpoint{2.405968in}{0.950277in}}%
\pgfpathlineto{\pgfqpoint{2.392184in}{0.949754in}}%
\pgfpathlineto{\pgfqpoint{2.391745in}{0.960160in}}%
\pgfpathlineto{\pgfqpoint{2.372289in}{0.959362in}}%
\pgfpathlineto{\pgfqpoint{2.384214in}{0.972642in}}%
\pgfpathlineto{\pgfqpoint{2.384480in}{0.979541in}}%
\pgfpathlineto{\pgfqpoint{2.395789in}{0.983518in}}%
\pgfpathlineto{\pgfqpoint{2.396088in}{0.992607in}}%
\pgfpathlineto{\pgfqpoint{2.407574in}{0.985155in}}%
\pgfpathlineto{\pgfqpoint{2.413949in}{0.985540in}}%
\pgfpathlineto{\pgfqpoint{2.413440in}{0.992269in}}%
\pgfpathlineto{\pgfqpoint{2.420148in}{0.996176in}}%
\pgfpathlineto{\pgfqpoint{2.467215in}{0.998833in}}%
\pgfpathlineto{\pgfqpoint{2.482530in}{0.995135in}}%
\pgfpathlineto{\pgfqpoint{2.483100in}{0.985946in}}%
\pgfpathlineto{\pgfqpoint{2.484251in}{0.968666in}}%
\pgfpathlineto{\pgfqpoint{2.466935in}{0.967664in}}%
\pgfpathlineto{\pgfqpoint{2.468504in}{0.944341in}}%
\pgfpathclose%
\pgfusepath{fill}%
\end{pgfscope}%
\begin{pgfscope}%
\pgfpathrectangle{\pgfqpoint{0.100000in}{0.100000in}}{\pgfqpoint{3.608454in}{2.310000in}}%
\pgfusepath{clip}%
\pgfsetbuttcap%
\pgfsetmiterjoin%
\definecolor{currentfill}{rgb}{0.000000,0.427451,0.786275}%
\pgfsetfillcolor{currentfill}%
\pgfsetlinewidth{0.000000pt}%
\definecolor{currentstroke}{rgb}{0.000000,0.000000,0.000000}%
\pgfsetstrokecolor{currentstroke}%
\pgfsetstrokeopacity{0.000000}%
\pgfsetdash{}{0pt}%
\pgfpathmoveto{\pgfqpoint{3.506058in}{1.889401in}}%
\pgfpathlineto{\pgfqpoint{3.496870in}{1.892724in}}%
\pgfpathlineto{\pgfqpoint{3.498018in}{1.895957in}}%
\pgfpathlineto{\pgfqpoint{3.491360in}{1.900167in}}%
\pgfpathlineto{\pgfqpoint{3.485597in}{1.897568in}}%
\pgfpathlineto{\pgfqpoint{3.484176in}{1.904080in}}%
\pgfpathlineto{\pgfqpoint{3.474707in}{1.903468in}}%
\pgfpathlineto{\pgfqpoint{3.465497in}{1.899405in}}%
\pgfpathlineto{\pgfqpoint{3.454146in}{1.936802in}}%
\pgfpathlineto{\pgfqpoint{3.442334in}{1.974192in}}%
\pgfpathlineto{\pgfqpoint{3.428979in}{2.013989in}}%
\pgfpathlineto{\pgfqpoint{3.435281in}{2.018401in}}%
\pgfpathlineto{\pgfqpoint{3.440288in}{2.012007in}}%
\pgfpathlineto{\pgfqpoint{3.443602in}{2.017361in}}%
\pgfpathlineto{\pgfqpoint{3.442691in}{2.028080in}}%
\pgfpathlineto{\pgfqpoint{3.447268in}{2.025988in}}%
\pgfpathlineto{\pgfqpoint{3.451169in}{2.029587in}}%
\pgfpathlineto{\pgfqpoint{3.444152in}{2.035398in}}%
\pgfpathlineto{\pgfqpoint{3.446349in}{2.043593in}}%
\pgfpathlineto{\pgfqpoint{3.457409in}{2.056674in}}%
\pgfpathlineto{\pgfqpoint{3.455828in}{2.064013in}}%
\pgfpathlineto{\pgfqpoint{3.460024in}{2.069258in}}%
\pgfpathlineto{\pgfqpoint{3.460838in}{2.075413in}}%
\pgfpathlineto{\pgfqpoint{3.455091in}{2.080073in}}%
\pgfpathlineto{\pgfqpoint{3.456695in}{2.090917in}}%
\pgfpathlineto{\pgfqpoint{3.452786in}{2.093614in}}%
\pgfpathlineto{\pgfqpoint{3.454249in}{2.105442in}}%
\pgfpathlineto{\pgfqpoint{3.460446in}{2.114346in}}%
\pgfpathlineto{\pgfqpoint{3.458788in}{2.127000in}}%
\pgfpathlineto{\pgfqpoint{3.474712in}{2.131510in}}%
\pgfpathlineto{\pgfqpoint{3.478067in}{2.117502in}}%
\pgfpathlineto{\pgfqpoint{3.487328in}{2.085547in}}%
\pgfpathlineto{\pgfqpoint{3.491275in}{2.076574in}}%
\pgfpathlineto{\pgfqpoint{3.490714in}{2.062918in}}%
\pgfpathlineto{\pgfqpoint{3.496496in}{2.059377in}}%
\pgfpathlineto{\pgfqpoint{3.494071in}{2.051630in}}%
\pgfpathlineto{\pgfqpoint{3.514278in}{2.013412in}}%
\pgfpathlineto{\pgfqpoint{3.527215in}{2.022326in}}%
\pgfpathlineto{\pgfqpoint{3.539818in}{1.996897in}}%
\pgfpathlineto{\pgfqpoint{3.532167in}{1.996604in}}%
\pgfpathlineto{\pgfqpoint{3.534775in}{1.988683in}}%
\pgfpathlineto{\pgfqpoint{3.535820in}{1.974394in}}%
\pgfpathlineto{\pgfqpoint{3.535140in}{1.964201in}}%
\pgfpathlineto{\pgfqpoint{3.538805in}{1.962876in}}%
\pgfpathlineto{\pgfqpoint{3.543555in}{1.955713in}}%
\pgfpathlineto{\pgfqpoint{3.547311in}{1.956692in}}%
\pgfpathlineto{\pgfqpoint{3.553924in}{1.946267in}}%
\pgfpathlineto{\pgfqpoint{3.550714in}{1.937007in}}%
\pgfpathlineto{\pgfqpoint{3.547033in}{1.936458in}}%
\pgfpathlineto{\pgfqpoint{3.546482in}{1.926493in}}%
\pgfpathlineto{\pgfqpoint{3.535703in}{1.920334in}}%
\pgfpathlineto{\pgfqpoint{3.534745in}{1.914657in}}%
\pgfpathlineto{\pgfqpoint{3.529959in}{1.911075in}}%
\pgfpathlineto{\pgfqpoint{3.516703in}{1.917002in}}%
\pgfpathlineto{\pgfqpoint{3.510197in}{1.910231in}}%
\pgfpathlineto{\pgfqpoint{3.508494in}{1.902315in}}%
\pgfpathlineto{\pgfqpoint{3.514068in}{1.894022in}}%
\pgfpathlineto{\pgfqpoint{3.506058in}{1.889401in}}%
\pgfpathclose%
\pgfusepath{fill}%
\end{pgfscope}%
\begin{pgfscope}%
\pgfpathrectangle{\pgfqpoint{0.100000in}{0.100000in}}{\pgfqpoint{3.608454in}{2.310000in}}%
\pgfusepath{clip}%
\pgfsetbuttcap%
\pgfsetmiterjoin%
\definecolor{currentfill}{rgb}{0.000000,0.615686,0.692157}%
\pgfsetfillcolor{currentfill}%
\pgfsetlinewidth{0.000000pt}%
\definecolor{currentstroke}{rgb}{0.000000,0.000000,0.000000}%
\pgfsetstrokecolor{currentstroke}%
\pgfsetstrokeopacity{0.000000}%
\pgfsetdash{}{0pt}%
\pgfpathmoveto{\pgfqpoint{2.519263in}{0.872679in}}%
\pgfpathlineto{\pgfqpoint{2.485523in}{0.870923in}}%
\pgfpathlineto{\pgfqpoint{2.483949in}{0.899006in}}%
\pgfpathlineto{\pgfqpoint{2.474841in}{0.898468in}}%
\pgfpathlineto{\pgfqpoint{2.474111in}{0.910147in}}%
\pgfpathlineto{\pgfqpoint{2.470515in}{0.912039in}}%
\pgfpathlineto{\pgfqpoint{2.469213in}{0.932802in}}%
\pgfpathlineto{\pgfqpoint{2.489985in}{0.934122in}}%
\pgfpathlineto{\pgfqpoint{2.506964in}{0.937791in}}%
\pgfpathlineto{\pgfqpoint{2.506684in}{0.942391in}}%
\pgfpathlineto{\pgfqpoint{2.520583in}{0.943229in}}%
\pgfpathlineto{\pgfqpoint{2.521327in}{0.932876in}}%
\pgfpathlineto{\pgfqpoint{2.535053in}{0.931443in}}%
\pgfpathlineto{\pgfqpoint{2.537884in}{0.927921in}}%
\pgfpathlineto{\pgfqpoint{2.534527in}{0.922423in}}%
\pgfpathlineto{\pgfqpoint{2.525476in}{0.919309in}}%
\pgfpathlineto{\pgfqpoint{2.526832in}{0.901949in}}%
\pgfpathlineto{\pgfqpoint{2.517297in}{0.901251in}}%
\pgfpathlineto{\pgfqpoint{2.519263in}{0.872679in}}%
\pgfpathclose%
\pgfusepath{fill}%
\end{pgfscope}%
\begin{pgfscope}%
\pgfpathrectangle{\pgfqpoint{0.100000in}{0.100000in}}{\pgfqpoint{3.608454in}{2.310000in}}%
\pgfusepath{clip}%
\pgfsetbuttcap%
\pgfsetmiterjoin%
\definecolor{currentfill}{rgb}{0.000000,0.403922,0.798039}%
\pgfsetfillcolor{currentfill}%
\pgfsetlinewidth{0.000000pt}%
\definecolor{currentstroke}{rgb}{0.000000,0.000000,0.000000}%
\pgfsetstrokecolor{currentstroke}%
\pgfsetstrokeopacity{0.000000}%
\pgfsetdash{}{0pt}%
\pgfpathmoveto{\pgfqpoint{0.623427in}{2.025636in}}%
\pgfpathlineto{\pgfqpoint{0.587828in}{2.036379in}}%
\pgfpathlineto{\pgfqpoint{0.581481in}{2.042623in}}%
\pgfpathlineto{\pgfqpoint{0.580214in}{2.049876in}}%
\pgfpathlineto{\pgfqpoint{0.575268in}{2.054730in}}%
\pgfpathlineto{\pgfqpoint{0.572541in}{2.061484in}}%
\pgfpathlineto{\pgfqpoint{0.575705in}{2.064800in}}%
\pgfpathlineto{\pgfqpoint{0.576179in}{2.072349in}}%
\pgfpathlineto{\pgfqpoint{0.568024in}{2.080965in}}%
\pgfpathlineto{\pgfqpoint{0.566154in}{2.086439in}}%
\pgfpathlineto{\pgfqpoint{0.561348in}{2.090351in}}%
\pgfpathlineto{\pgfqpoint{0.543848in}{2.096029in}}%
\pgfpathlineto{\pgfqpoint{0.550017in}{2.103751in}}%
\pgfpathlineto{\pgfqpoint{0.551015in}{2.110628in}}%
\pgfpathlineto{\pgfqpoint{0.556089in}{2.115055in}}%
\pgfpathlineto{\pgfqpoint{0.557819in}{2.120432in}}%
\pgfpathlineto{\pgfqpoint{0.566644in}{2.148266in}}%
\pgfpathlineto{\pgfqpoint{0.578043in}{2.148170in}}%
\pgfpathlineto{\pgfqpoint{0.589293in}{2.135786in}}%
\pgfpathlineto{\pgfqpoint{0.593066in}{2.119148in}}%
\pgfpathlineto{\pgfqpoint{0.609486in}{2.121452in}}%
\pgfpathlineto{\pgfqpoint{0.615173in}{2.117075in}}%
\pgfpathlineto{\pgfqpoint{0.623310in}{2.122649in}}%
\pgfpathlineto{\pgfqpoint{0.631314in}{2.147810in}}%
\pgfpathlineto{\pgfqpoint{0.669147in}{2.136525in}}%
\pgfpathlineto{\pgfqpoint{0.661241in}{2.110265in}}%
\pgfpathlineto{\pgfqpoint{0.656547in}{2.111660in}}%
\pgfpathlineto{\pgfqpoint{0.650732in}{2.091686in}}%
\pgfpathlineto{\pgfqpoint{0.653479in}{2.086201in}}%
\pgfpathlineto{\pgfqpoint{0.641984in}{2.086967in}}%
\pgfpathlineto{\pgfqpoint{0.634943in}{2.089043in}}%
\pgfpathlineto{\pgfqpoint{0.630988in}{2.086576in}}%
\pgfpathlineto{\pgfqpoint{0.628826in}{2.076535in}}%
\pgfpathlineto{\pgfqpoint{0.636428in}{2.062774in}}%
\pgfpathlineto{\pgfqpoint{0.634662in}{2.052316in}}%
\pgfpathlineto{\pgfqpoint{0.629691in}{2.048489in}}%
\pgfpathlineto{\pgfqpoint{0.630035in}{2.038493in}}%
\pgfpathlineto{\pgfqpoint{0.622394in}{2.035924in}}%
\pgfpathlineto{\pgfqpoint{0.623427in}{2.025636in}}%
\pgfpathclose%
\pgfusepath{fill}%
\end{pgfscope}%
\begin{pgfscope}%
\pgfpathrectangle{\pgfqpoint{0.100000in}{0.100000in}}{\pgfqpoint{3.608454in}{2.310000in}}%
\pgfusepath{clip}%
\pgfsetbuttcap%
\pgfsetmiterjoin%
\definecolor{currentfill}{rgb}{0.000000,0.466667,0.766667}%
\pgfsetfillcolor{currentfill}%
\pgfsetlinewidth{0.000000pt}%
\definecolor{currentstroke}{rgb}{0.000000,0.000000,0.000000}%
\pgfsetstrokecolor{currentstroke}%
\pgfsetstrokeopacity{0.000000}%
\pgfsetdash{}{0pt}%
\pgfpathmoveto{\pgfqpoint{3.027503in}{0.502091in}}%
\pgfpathlineto{\pgfqpoint{2.992126in}{0.496993in}}%
\pgfpathlineto{\pgfqpoint{2.998004in}{0.510943in}}%
\pgfpathlineto{\pgfqpoint{2.992287in}{0.512337in}}%
\pgfpathlineto{\pgfqpoint{2.985816in}{0.507630in}}%
\pgfpathlineto{\pgfqpoint{2.984712in}{0.500436in}}%
\pgfpathlineto{\pgfqpoint{2.980383in}{0.498839in}}%
\pgfpathlineto{\pgfqpoint{2.968704in}{0.512418in}}%
\pgfpathlineto{\pgfqpoint{2.971482in}{0.527436in}}%
\pgfpathlineto{\pgfqpoint{2.969072in}{0.537771in}}%
\pgfpathlineto{\pgfqpoint{2.971652in}{0.541641in}}%
\pgfpathlineto{\pgfqpoint{2.975315in}{0.570540in}}%
\pgfpathlineto{\pgfqpoint{2.974786in}{0.579836in}}%
\pgfpathlineto{\pgfqpoint{2.990145in}{0.581978in}}%
\pgfpathlineto{\pgfqpoint{3.001283in}{0.581332in}}%
\pgfpathlineto{\pgfqpoint{3.006224in}{0.574274in}}%
\pgfpathlineto{\pgfqpoint{3.017489in}{0.571785in}}%
\pgfpathlineto{\pgfqpoint{3.019784in}{0.555093in}}%
\pgfpathlineto{\pgfqpoint{3.016866in}{0.550384in}}%
\pgfpathlineto{\pgfqpoint{3.021367in}{0.543921in}}%
\pgfpathlineto{\pgfqpoint{3.027503in}{0.502091in}}%
\pgfpathclose%
\pgfusepath{fill}%
\end{pgfscope}%
\begin{pgfscope}%
\pgfpathrectangle{\pgfqpoint{0.100000in}{0.100000in}}{\pgfqpoint{3.608454in}{2.310000in}}%
\pgfusepath{clip}%
\pgfsetbuttcap%
\pgfsetmiterjoin%
\definecolor{currentfill}{rgb}{0.000000,0.290196,0.854902}%
\pgfsetfillcolor{currentfill}%
\pgfsetlinewidth{0.000000pt}%
\definecolor{currentstroke}{rgb}{0.000000,0.000000,0.000000}%
\pgfsetstrokecolor{currentstroke}%
\pgfsetstrokeopacity{0.000000}%
\pgfsetdash{}{0pt}%
\pgfpathmoveto{\pgfqpoint{3.420582in}{1.833115in}}%
\pgfpathlineto{\pgfqpoint{3.421042in}{1.838069in}}%
\pgfpathlineto{\pgfqpoint{3.413847in}{1.847236in}}%
\pgfpathlineto{\pgfqpoint{3.415970in}{1.857152in}}%
\pgfpathlineto{\pgfqpoint{3.421501in}{1.872886in}}%
\pgfpathlineto{\pgfqpoint{3.424616in}{1.868740in}}%
\pgfpathlineto{\pgfqpoint{3.432335in}{1.876511in}}%
\pgfpathlineto{\pgfqpoint{3.431173in}{1.881530in}}%
\pgfpathlineto{\pgfqpoint{3.437941in}{1.884217in}}%
\pgfpathlineto{\pgfqpoint{3.431626in}{1.898622in}}%
\pgfpathlineto{\pgfqpoint{3.442370in}{1.903310in}}%
\pgfpathlineto{\pgfqpoint{3.438972in}{1.915003in}}%
\pgfpathlineto{\pgfqpoint{3.433355in}{1.925260in}}%
\pgfpathlineto{\pgfqpoint{3.440922in}{1.921126in}}%
\pgfpathlineto{\pgfqpoint{3.441678in}{1.930119in}}%
\pgfpathlineto{\pgfqpoint{3.453336in}{1.932723in}}%
\pgfpathlineto{\pgfqpoint{3.454146in}{1.936802in}}%
\pgfpathlineto{\pgfqpoint{3.465497in}{1.899405in}}%
\pgfpathlineto{\pgfqpoint{3.474707in}{1.903468in}}%
\pgfpathlineto{\pgfqpoint{3.484176in}{1.904080in}}%
\pgfpathlineto{\pgfqpoint{3.485597in}{1.897568in}}%
\pgfpathlineto{\pgfqpoint{3.491360in}{1.900167in}}%
\pgfpathlineto{\pgfqpoint{3.498018in}{1.895957in}}%
\pgfpathlineto{\pgfqpoint{3.496870in}{1.892724in}}%
\pgfpathlineto{\pgfqpoint{3.506058in}{1.889401in}}%
\pgfpathlineto{\pgfqpoint{3.507318in}{1.881477in}}%
\pgfpathlineto{\pgfqpoint{3.505212in}{1.874640in}}%
\pgfpathlineto{\pgfqpoint{3.499936in}{1.871265in}}%
\pgfpathlineto{\pgfqpoint{3.499231in}{1.853431in}}%
\pgfpathlineto{\pgfqpoint{3.493789in}{1.834815in}}%
\pgfpathlineto{\pgfqpoint{3.494171in}{1.831598in}}%
\pgfpathlineto{\pgfqpoint{3.487553in}{1.830826in}}%
\pgfpathlineto{\pgfqpoint{3.477550in}{1.822939in}}%
\pgfpathlineto{\pgfqpoint{3.471016in}{1.810988in}}%
\pgfpathlineto{\pgfqpoint{3.434992in}{1.802755in}}%
\pgfpathlineto{\pgfqpoint{3.430894in}{1.808006in}}%
\pgfpathlineto{\pgfqpoint{3.426046in}{1.819395in}}%
\pgfpathlineto{\pgfqpoint{3.422865in}{1.818650in}}%
\pgfpathlineto{\pgfqpoint{3.420582in}{1.833115in}}%
\pgfpathclose%
\pgfusepath{fill}%
\end{pgfscope}%
\begin{pgfscope}%
\pgfpathrectangle{\pgfqpoint{0.100000in}{0.100000in}}{\pgfqpoint{3.608454in}{2.310000in}}%
\pgfusepath{clip}%
\pgfsetbuttcap%
\pgfsetmiterjoin%
\definecolor{currentfill}{rgb}{0.000000,0.698039,0.650980}%
\pgfsetfillcolor{currentfill}%
\pgfsetlinewidth{0.000000pt}%
\definecolor{currentstroke}{rgb}{0.000000,0.000000,0.000000}%
\pgfsetstrokecolor{currentstroke}%
\pgfsetstrokeopacity{0.000000}%
\pgfsetdash{}{0pt}%
\pgfpathmoveto{\pgfqpoint{2.478809in}{1.176580in}}%
\pgfpathlineto{\pgfqpoint{2.477580in}{1.181419in}}%
\pgfpathlineto{\pgfqpoint{2.479456in}{1.190579in}}%
\pgfpathlineto{\pgfqpoint{2.477226in}{1.192084in}}%
\pgfpathlineto{\pgfqpoint{2.473800in}{1.196498in}}%
\pgfpathlineto{\pgfqpoint{2.482512in}{1.210574in}}%
\pgfpathlineto{\pgfqpoint{2.488838in}{1.212212in}}%
\pgfpathlineto{\pgfqpoint{2.489480in}{1.220900in}}%
\pgfpathlineto{\pgfqpoint{2.501888in}{1.221865in}}%
\pgfpathlineto{\pgfqpoint{2.500587in}{1.242556in}}%
\pgfpathlineto{\pgfqpoint{2.519105in}{1.243921in}}%
\pgfpathlineto{\pgfqpoint{2.519923in}{1.230014in}}%
\pgfpathlineto{\pgfqpoint{2.523585in}{1.228784in}}%
\pgfpathlineto{\pgfqpoint{2.533129in}{1.219044in}}%
\pgfpathlineto{\pgfqpoint{2.535791in}{1.209919in}}%
\pgfpathlineto{\pgfqpoint{2.533377in}{1.204248in}}%
\pgfpathlineto{\pgfqpoint{2.539380in}{1.187318in}}%
\pgfpathlineto{\pgfqpoint{2.543041in}{1.178054in}}%
\pgfpathlineto{\pgfqpoint{2.519321in}{1.176572in}}%
\pgfpathlineto{\pgfqpoint{2.518313in}{1.191718in}}%
\pgfpathlineto{\pgfqpoint{2.497579in}{1.190610in}}%
\pgfpathlineto{\pgfqpoint{2.498496in}{1.176925in}}%
\pgfpathlineto{\pgfqpoint{2.478809in}{1.176580in}}%
\pgfpathclose%
\pgfusepath{fill}%
\end{pgfscope}%
\begin{pgfscope}%
\pgfpathrectangle{\pgfqpoint{0.100000in}{0.100000in}}{\pgfqpoint{3.608454in}{2.310000in}}%
\pgfusepath{clip}%
\pgfsetbuttcap%
\pgfsetmiterjoin%
\definecolor{currentfill}{rgb}{0.000000,0.505882,0.747059}%
\pgfsetfillcolor{currentfill}%
\pgfsetlinewidth{0.000000pt}%
\definecolor{currentstroke}{rgb}{0.000000,0.000000,0.000000}%
\pgfsetstrokecolor{currentstroke}%
\pgfsetstrokeopacity{0.000000}%
\pgfsetdash{}{0pt}%
\pgfpathmoveto{\pgfqpoint{3.165006in}{1.605498in}}%
\pgfpathlineto{\pgfqpoint{3.154199in}{1.598846in}}%
\pgfpathlineto{\pgfqpoint{3.144862in}{1.595009in}}%
\pgfpathlineto{\pgfqpoint{3.133554in}{1.604605in}}%
\pgfpathlineto{\pgfqpoint{3.128786in}{1.603472in}}%
\pgfpathlineto{\pgfqpoint{3.125203in}{1.608488in}}%
\pgfpathlineto{\pgfqpoint{3.111013in}{1.603891in}}%
\pgfpathlineto{\pgfqpoint{3.106858in}{1.608202in}}%
\pgfpathlineto{\pgfqpoint{3.110757in}{1.621055in}}%
\pgfpathlineto{\pgfqpoint{3.109153in}{1.629395in}}%
\pgfpathlineto{\pgfqpoint{3.131860in}{1.634049in}}%
\pgfpathlineto{\pgfqpoint{3.130816in}{1.638979in}}%
\pgfpathlineto{\pgfqpoint{3.167806in}{1.647018in}}%
\pgfpathlineto{\pgfqpoint{3.172073in}{1.651526in}}%
\pgfpathlineto{\pgfqpoint{3.162699in}{1.682336in}}%
\pgfpathlineto{\pgfqpoint{3.183986in}{1.686549in}}%
\pgfpathlineto{\pgfqpoint{3.207689in}{1.691369in}}%
\pgfpathlineto{\pgfqpoint{3.215154in}{1.664914in}}%
\pgfpathlineto{\pgfqpoint{3.210379in}{1.663470in}}%
\pgfpathlineto{\pgfqpoint{3.210875in}{1.652502in}}%
\pgfpathlineto{\pgfqpoint{3.209405in}{1.635945in}}%
\pgfpathlineto{\pgfqpoint{3.203745in}{1.634729in}}%
\pgfpathlineto{\pgfqpoint{3.195407in}{1.620855in}}%
\pgfpathlineto{\pgfqpoint{3.192630in}{1.620121in}}%
\pgfpathlineto{\pgfqpoint{3.183685in}{1.619914in}}%
\pgfpathlineto{\pgfqpoint{3.174106in}{1.615801in}}%
\pgfpathlineto{\pgfqpoint{3.173979in}{1.610853in}}%
\pgfpathlineto{\pgfqpoint{3.165006in}{1.605498in}}%
\pgfpathclose%
\pgfusepath{fill}%
\end{pgfscope}%
\begin{pgfscope}%
\pgfpathrectangle{\pgfqpoint{0.100000in}{0.100000in}}{\pgfqpoint{3.608454in}{2.310000in}}%
\pgfusepath{clip}%
\pgfsetbuttcap%
\pgfsetmiterjoin%
\definecolor{currentfill}{rgb}{0.000000,0.823529,0.588235}%
\pgfsetfillcolor{currentfill}%
\pgfsetlinewidth{0.000000pt}%
\definecolor{currentstroke}{rgb}{0.000000,0.000000,0.000000}%
\pgfsetstrokecolor{currentstroke}%
\pgfsetstrokeopacity{0.000000}%
\pgfsetdash{}{0pt}%
\pgfpathmoveto{\pgfqpoint{2.674593in}{0.730790in}}%
\pgfpathlineto{\pgfqpoint{2.612513in}{0.725646in}}%
\pgfpathlineto{\pgfqpoint{2.609757in}{0.745194in}}%
\pgfpathlineto{\pgfqpoint{2.602701in}{0.749237in}}%
\pgfpathlineto{\pgfqpoint{2.601460in}{0.754867in}}%
\pgfpathlineto{\pgfqpoint{2.605232in}{0.759125in}}%
\pgfpathlineto{\pgfqpoint{2.611445in}{0.760754in}}%
\pgfpathlineto{\pgfqpoint{2.610143in}{0.781433in}}%
\pgfpathlineto{\pgfqpoint{2.613565in}{0.781771in}}%
\pgfpathlineto{\pgfqpoint{2.613728in}{0.792291in}}%
\pgfpathlineto{\pgfqpoint{2.653773in}{0.795845in}}%
\pgfpathlineto{\pgfqpoint{2.655184in}{0.780136in}}%
\pgfpathlineto{\pgfqpoint{2.660506in}{0.772013in}}%
\pgfpathlineto{\pgfqpoint{2.669844in}{0.772749in}}%
\pgfpathlineto{\pgfqpoint{2.673739in}{0.731186in}}%
\pgfpathlineto{\pgfqpoint{2.674593in}{0.730790in}}%
\pgfpathclose%
\pgfusepath{fill}%
\end{pgfscope}%
\begin{pgfscope}%
\pgfpathrectangle{\pgfqpoint{0.100000in}{0.100000in}}{\pgfqpoint{3.608454in}{2.310000in}}%
\pgfusepath{clip}%
\pgfsetbuttcap%
\pgfsetmiterjoin%
\definecolor{currentfill}{rgb}{0.000000,0.674510,0.662745}%
\pgfsetfillcolor{currentfill}%
\pgfsetlinewidth{0.000000pt}%
\definecolor{currentstroke}{rgb}{0.000000,0.000000,0.000000}%
\pgfsetstrokecolor{currentstroke}%
\pgfsetstrokeopacity{0.000000}%
\pgfsetdash{}{0pt}%
\pgfpathmoveto{\pgfqpoint{1.245517in}{0.942241in}}%
\pgfpathlineto{\pgfqpoint{1.176800in}{0.953138in}}%
\pgfpathlineto{\pgfqpoint{1.183780in}{0.997802in}}%
\pgfpathlineto{\pgfqpoint{1.193614in}{1.060581in}}%
\pgfpathlineto{\pgfqpoint{1.236876in}{1.053946in}}%
\pgfpathlineto{\pgfqpoint{1.278910in}{1.047826in}}%
\pgfpathlineto{\pgfqpoint{1.333401in}{1.040515in}}%
\pgfpathlineto{\pgfqpoint{1.340304in}{1.035758in}}%
\pgfpathlineto{\pgfqpoint{1.362154in}{1.025644in}}%
\pgfpathlineto{\pgfqpoint{1.360437in}{1.011466in}}%
\pgfpathlineto{\pgfqpoint{1.392442in}{1.007538in}}%
\pgfpathlineto{\pgfqpoint{1.388260in}{0.973319in}}%
\pgfpathlineto{\pgfqpoint{1.379878in}{0.974364in}}%
\pgfpathlineto{\pgfqpoint{1.378188in}{0.960531in}}%
\pgfpathlineto{\pgfqpoint{1.357228in}{0.963939in}}%
\pgfpathlineto{\pgfqpoint{1.355496in}{0.949729in}}%
\pgfpathlineto{\pgfqpoint{1.318032in}{0.954222in}}%
\pgfpathlineto{\pgfqpoint{1.252648in}{0.963109in}}%
\pgfpathlineto{\pgfqpoint{1.248648in}{0.963976in}}%
\pgfpathlineto{\pgfqpoint{1.245517in}{0.942241in}}%
\pgfpathclose%
\pgfusepath{fill}%
\end{pgfscope}%
\begin{pgfscope}%
\pgfpathrectangle{\pgfqpoint{0.100000in}{0.100000in}}{\pgfqpoint{3.608454in}{2.310000in}}%
\pgfusepath{clip}%
\pgfsetbuttcap%
\pgfsetmiterjoin%
\definecolor{currentfill}{rgb}{0.000000,0.670588,0.664706}%
\pgfsetfillcolor{currentfill}%
\pgfsetlinewidth{0.000000pt}%
\definecolor{currentstroke}{rgb}{0.000000,0.000000,0.000000}%
\pgfsetstrokecolor{currentstroke}%
\pgfsetstrokeopacity{0.000000}%
\pgfsetdash{}{0pt}%
\pgfpathmoveto{\pgfqpoint{2.676133in}{1.724089in}}%
\pgfpathlineto{\pgfqpoint{2.648810in}{1.721346in}}%
\pgfpathlineto{\pgfqpoint{2.645951in}{1.748717in}}%
\pgfpathlineto{\pgfqpoint{2.618822in}{1.745974in}}%
\pgfpathlineto{\pgfqpoint{2.615872in}{1.773777in}}%
\pgfpathlineto{\pgfqpoint{2.628418in}{1.774736in}}%
\pgfpathlineto{\pgfqpoint{2.625838in}{1.802295in}}%
\pgfpathlineto{\pgfqpoint{2.680280in}{1.807943in}}%
\pgfpathlineto{\pgfqpoint{2.683319in}{1.780357in}}%
\pgfpathlineto{\pgfqpoint{2.669984in}{1.779116in}}%
\pgfpathlineto{\pgfqpoint{2.676133in}{1.724089in}}%
\pgfpathclose%
\pgfusepath{fill}%
\end{pgfscope}%
\begin{pgfscope}%
\pgfpathrectangle{\pgfqpoint{0.100000in}{0.100000in}}{\pgfqpoint{3.608454in}{2.310000in}}%
\pgfusepath{clip}%
\pgfsetbuttcap%
\pgfsetmiterjoin%
\definecolor{currentfill}{rgb}{0.000000,0.411765,0.794118}%
\pgfsetfillcolor{currentfill}%
\pgfsetlinewidth{0.000000pt}%
\definecolor{currentstroke}{rgb}{0.000000,0.000000,0.000000}%
\pgfsetstrokecolor{currentstroke}%
\pgfsetstrokeopacity{0.000000}%
\pgfsetdash{}{0pt}%
\pgfpathmoveto{\pgfqpoint{2.575892in}{1.080309in}}%
\pgfpathlineto{\pgfqpoint{2.579650in}{1.089158in}}%
\pgfpathlineto{\pgfqpoint{2.573068in}{1.094430in}}%
\pgfpathlineto{\pgfqpoint{2.573691in}{1.108364in}}%
\pgfpathlineto{\pgfqpoint{2.573891in}{1.113688in}}%
\pgfpathlineto{\pgfqpoint{2.584331in}{1.121514in}}%
\pgfpathlineto{\pgfqpoint{2.605651in}{1.120589in}}%
\pgfpathlineto{\pgfqpoint{2.605693in}{1.111968in}}%
\pgfpathlineto{\pgfqpoint{2.632392in}{1.103973in}}%
\pgfpathlineto{\pgfqpoint{2.643683in}{1.102257in}}%
\pgfpathlineto{\pgfqpoint{2.644297in}{1.082383in}}%
\pgfpathlineto{\pgfqpoint{2.648588in}{1.077186in}}%
\pgfpathlineto{\pgfqpoint{2.648663in}{1.074175in}}%
\pgfpathlineto{\pgfqpoint{2.639911in}{1.067474in}}%
\pgfpathlineto{\pgfqpoint{2.634590in}{1.067850in}}%
\pgfpathlineto{\pgfqpoint{2.636003in}{1.046348in}}%
\pgfpathlineto{\pgfqpoint{2.611745in}{1.044790in}}%
\pgfpathlineto{\pgfqpoint{2.610505in}{1.048649in}}%
\pgfpathlineto{\pgfqpoint{2.608880in}{1.074941in}}%
\pgfpathlineto{\pgfqpoint{2.607711in}{1.078660in}}%
\pgfpathlineto{\pgfqpoint{2.601253in}{1.080261in}}%
\pgfpathlineto{\pgfqpoint{2.585581in}{1.074349in}}%
\pgfpathlineto{\pgfqpoint{2.580566in}{1.079844in}}%
\pgfpathlineto{\pgfqpoint{2.575892in}{1.080309in}}%
\pgfpathclose%
\pgfusepath{fill}%
\end{pgfscope}%
\begin{pgfscope}%
\pgfpathrectangle{\pgfqpoint{0.100000in}{0.100000in}}{\pgfqpoint{3.608454in}{2.310000in}}%
\pgfusepath{clip}%
\pgfsetbuttcap%
\pgfsetmiterjoin%
\definecolor{currentfill}{rgb}{0.000000,0.949020,0.525490}%
\pgfsetfillcolor{currentfill}%
\pgfsetlinewidth{0.000000pt}%
\definecolor{currentstroke}{rgb}{0.000000,0.000000,0.000000}%
\pgfsetstrokecolor{currentstroke}%
\pgfsetstrokeopacity{0.000000}%
\pgfsetdash{}{0pt}%
\pgfpathmoveto{\pgfqpoint{3.010884in}{1.388904in}}%
\pgfpathlineto{\pgfqpoint{3.018910in}{1.376644in}}%
\pgfpathlineto{\pgfqpoint{3.016902in}{1.365346in}}%
\pgfpathlineto{\pgfqpoint{3.012394in}{1.360390in}}%
\pgfpathlineto{\pgfqpoint{3.011961in}{1.351516in}}%
\pgfpathlineto{\pgfqpoint{3.006750in}{1.353994in}}%
\pgfpathlineto{\pgfqpoint{2.992200in}{1.336598in}}%
\pgfpathlineto{\pgfqpoint{2.981470in}{1.336849in}}%
\pgfpathlineto{\pgfqpoint{2.973652in}{1.345179in}}%
\pgfpathlineto{\pgfqpoint{2.968495in}{1.342247in}}%
\pgfpathlineto{\pgfqpoint{2.958041in}{1.346022in}}%
\pgfpathlineto{\pgfqpoint{2.975847in}{1.360884in}}%
\pgfpathlineto{\pgfqpoint{2.976752in}{1.368166in}}%
\pgfpathlineto{\pgfqpoint{2.972924in}{1.373592in}}%
\pgfpathlineto{\pgfqpoint{2.965422in}{1.379505in}}%
\pgfpathlineto{\pgfqpoint{2.967663in}{1.383858in}}%
\pgfpathlineto{\pgfqpoint{2.962379in}{1.389071in}}%
\pgfpathlineto{\pgfqpoint{2.965023in}{1.393727in}}%
\pgfpathlineto{\pgfqpoint{2.961168in}{1.406181in}}%
\pgfpathlineto{\pgfqpoint{2.967704in}{1.409901in}}%
\pgfpathlineto{\pgfqpoint{2.973331in}{1.416066in}}%
\pgfpathlineto{\pgfqpoint{2.978628in}{1.415804in}}%
\pgfpathlineto{\pgfqpoint{2.980404in}{1.409102in}}%
\pgfpathlineto{\pgfqpoint{2.989078in}{1.401477in}}%
\pgfpathlineto{\pgfqpoint{2.997346in}{1.397787in}}%
\pgfpathlineto{\pgfqpoint{3.001117in}{1.390072in}}%
\pgfpathlineto{\pgfqpoint{3.010884in}{1.388904in}}%
\pgfpathclose%
\pgfusepath{fill}%
\end{pgfscope}%
\begin{pgfscope}%
\pgfpathrectangle{\pgfqpoint{0.100000in}{0.100000in}}{\pgfqpoint{3.608454in}{2.310000in}}%
\pgfusepath{clip}%
\pgfsetbuttcap%
\pgfsetmiterjoin%
\definecolor{currentfill}{rgb}{0.000000,0.784314,0.607843}%
\pgfsetfillcolor{currentfill}%
\pgfsetlinewidth{0.000000pt}%
\definecolor{currentstroke}{rgb}{0.000000,0.000000,0.000000}%
\pgfsetstrokecolor{currentstroke}%
\pgfsetstrokeopacity{0.000000}%
\pgfsetdash{}{0pt}%
\pgfpathmoveto{\pgfqpoint{0.983512in}{1.292540in}}%
\pgfpathlineto{\pgfqpoint{0.912295in}{1.307222in}}%
\pgfpathlineto{\pgfqpoint{0.925741in}{1.369767in}}%
\pgfpathlineto{\pgfqpoint{0.938194in}{1.428823in}}%
\pgfpathlineto{\pgfqpoint{1.000345in}{1.416064in}}%
\pgfpathlineto{\pgfqpoint{1.031123in}{1.409823in}}%
\pgfpathlineto{\pgfqpoint{1.030622in}{1.400197in}}%
\pgfpathlineto{\pgfqpoint{1.034771in}{1.396805in}}%
\pgfpathlineto{\pgfqpoint{1.038444in}{1.389127in}}%
\pgfpathlineto{\pgfqpoint{1.032898in}{1.379983in}}%
\pgfpathlineto{\pgfqpoint{1.026917in}{1.376417in}}%
\pgfpathlineto{\pgfqpoint{1.023679in}{1.356359in}}%
\pgfpathlineto{\pgfqpoint{1.016803in}{1.357032in}}%
\pgfpathlineto{\pgfqpoint{1.015614in}{1.351012in}}%
\pgfpathlineto{\pgfqpoint{1.008850in}{1.352444in}}%
\pgfpathlineto{\pgfqpoint{1.005205in}{1.331999in}}%
\pgfpathlineto{\pgfqpoint{0.991804in}{1.334673in}}%
\pgfpathlineto{\pgfqpoint{0.983512in}{1.292540in}}%
\pgfpathclose%
\pgfusepath{fill}%
\end{pgfscope}%
\begin{pgfscope}%
\pgfpathrectangle{\pgfqpoint{0.100000in}{0.100000in}}{\pgfqpoint{3.608454in}{2.310000in}}%
\pgfusepath{clip}%
\pgfsetbuttcap%
\pgfsetmiterjoin%
\definecolor{currentfill}{rgb}{0.000000,0.490196,0.754902}%
\pgfsetfillcolor{currentfill}%
\pgfsetlinewidth{0.000000pt}%
\definecolor{currentstroke}{rgb}{0.000000,0.000000,0.000000}%
\pgfsetstrokecolor{currentstroke}%
\pgfsetstrokeopacity{0.000000}%
\pgfsetdash{}{0pt}%
\pgfpathmoveto{\pgfqpoint{2.684273in}{1.275179in}}%
\pgfpathlineto{\pgfqpoint{2.691418in}{1.266749in}}%
\pgfpathlineto{\pgfqpoint{2.694291in}{1.266816in}}%
\pgfpathlineto{\pgfqpoint{2.694535in}{1.255275in}}%
\pgfpathlineto{\pgfqpoint{2.701382in}{1.255633in}}%
\pgfpathlineto{\pgfqpoint{2.704025in}{1.249697in}}%
\pgfpathlineto{\pgfqpoint{2.693473in}{1.249077in}}%
\pgfpathlineto{\pgfqpoint{2.692446in}{1.245040in}}%
\pgfpathlineto{\pgfqpoint{2.667849in}{1.244767in}}%
\pgfpathlineto{\pgfqpoint{2.665870in}{1.236725in}}%
\pgfpathlineto{\pgfqpoint{2.661850in}{1.235103in}}%
\pgfpathlineto{\pgfqpoint{2.642473in}{1.232080in}}%
\pgfpathlineto{\pgfqpoint{2.641029in}{1.235231in}}%
\pgfpathlineto{\pgfqpoint{2.632971in}{1.237093in}}%
\pgfpathlineto{\pgfqpoint{2.628815in}{1.247944in}}%
\pgfpathlineto{\pgfqpoint{2.630443in}{1.257485in}}%
\pgfpathlineto{\pgfqpoint{2.625972in}{1.268011in}}%
\pgfpathlineto{\pgfqpoint{2.627023in}{1.272051in}}%
\pgfpathlineto{\pgfqpoint{2.630595in}{1.278625in}}%
\pgfpathlineto{\pgfqpoint{2.635552in}{1.279697in}}%
\pgfpathlineto{\pgfqpoint{2.634127in}{1.288599in}}%
\pgfpathlineto{\pgfqpoint{2.636960in}{1.295013in}}%
\pgfpathlineto{\pgfqpoint{2.640611in}{1.294225in}}%
\pgfpathlineto{\pgfqpoint{2.643610in}{1.301513in}}%
\pgfpathlineto{\pgfqpoint{2.648711in}{1.297118in}}%
\pgfpathlineto{\pgfqpoint{2.649917in}{1.290900in}}%
\pgfpathlineto{\pgfqpoint{2.655765in}{1.288068in}}%
\pgfpathlineto{\pgfqpoint{2.669652in}{1.289191in}}%
\pgfpathlineto{\pgfqpoint{2.675593in}{1.285078in}}%
\pgfpathlineto{\pgfqpoint{2.678952in}{1.278333in}}%
\pgfpathlineto{\pgfqpoint{2.684273in}{1.275179in}}%
\pgfpathclose%
\pgfusepath{fill}%
\end{pgfscope}%
\begin{pgfscope}%
\pgfpathrectangle{\pgfqpoint{0.100000in}{0.100000in}}{\pgfqpoint{3.608454in}{2.310000in}}%
\pgfusepath{clip}%
\pgfsetbuttcap%
\pgfsetmiterjoin%
\definecolor{currentfill}{rgb}{0.000000,0.478431,0.760784}%
\pgfsetfillcolor{currentfill}%
\pgfsetlinewidth{0.000000pt}%
\definecolor{currentstroke}{rgb}{0.000000,0.000000,0.000000}%
\pgfsetstrokecolor{currentstroke}%
\pgfsetstrokeopacity{0.000000}%
\pgfsetdash{}{0pt}%
\pgfpathmoveto{\pgfqpoint{1.896146in}{2.138772in}}%
\pgfpathlineto{\pgfqpoint{1.895146in}{2.116164in}}%
\pgfpathlineto{\pgfqpoint{1.896095in}{2.102164in}}%
\pgfpathlineto{\pgfqpoint{1.895459in}{2.088252in}}%
\pgfpathlineto{\pgfqpoint{1.867703in}{2.089585in}}%
\pgfpathlineto{\pgfqpoint{1.868375in}{2.103508in}}%
\pgfpathlineto{\pgfqpoint{1.833772in}{2.105376in}}%
\pgfpathlineto{\pgfqpoint{1.832606in}{2.119399in}}%
\pgfpathlineto{\pgfqpoint{1.833904in}{2.141855in}}%
\pgfpathlineto{\pgfqpoint{1.896146in}{2.138772in}}%
\pgfpathclose%
\pgfusepath{fill}%
\end{pgfscope}%
\begin{pgfscope}%
\pgfpathrectangle{\pgfqpoint{0.100000in}{0.100000in}}{\pgfqpoint{3.608454in}{2.310000in}}%
\pgfusepath{clip}%
\pgfsetbuttcap%
\pgfsetmiterjoin%
\definecolor{currentfill}{rgb}{0.000000,0.490196,0.754902}%
\pgfsetfillcolor{currentfill}%
\pgfsetlinewidth{0.000000pt}%
\definecolor{currentstroke}{rgb}{0.000000,0.000000,0.000000}%
\pgfsetstrokecolor{currentstroke}%
\pgfsetstrokeopacity{0.000000}%
\pgfsetdash{}{0pt}%
\pgfpathmoveto{\pgfqpoint{2.013608in}{1.046935in}}%
\pgfpathlineto{\pgfqpoint{2.013392in}{1.033129in}}%
\pgfpathlineto{\pgfqpoint{2.010212in}{1.033190in}}%
\pgfpathlineto{\pgfqpoint{2.009922in}{1.016922in}}%
\pgfpathlineto{\pgfqpoint{1.999760in}{1.016432in}}%
\pgfpathlineto{\pgfqpoint{1.994760in}{1.012926in}}%
\pgfpathlineto{\pgfqpoint{1.988856in}{1.020093in}}%
\pgfpathlineto{\pgfqpoint{1.984031in}{1.018827in}}%
\pgfpathlineto{\pgfqpoint{1.981120in}{1.013018in}}%
\pgfpathlineto{\pgfqpoint{1.933314in}{1.014281in}}%
\pgfpathlineto{\pgfqpoint{1.934200in}{1.052365in}}%
\pgfpathlineto{\pgfqpoint{1.923203in}{1.052514in}}%
\pgfpathlineto{\pgfqpoint{1.911760in}{1.056365in}}%
\pgfpathlineto{\pgfqpoint{1.893326in}{1.056979in}}%
\pgfpathlineto{\pgfqpoint{1.893386in}{1.070793in}}%
\pgfpathlineto{\pgfqpoint{1.893880in}{1.084537in}}%
\pgfpathlineto{\pgfqpoint{1.900701in}{1.084287in}}%
\pgfpathlineto{\pgfqpoint{1.901714in}{1.119083in}}%
\pgfpathlineto{\pgfqpoint{1.949595in}{1.117555in}}%
\pgfpathlineto{\pgfqpoint{1.956419in}{1.117360in}}%
\pgfpathlineto{\pgfqpoint{1.956052in}{1.103169in}}%
\pgfpathlineto{\pgfqpoint{1.969632in}{1.099419in}}%
\pgfpathlineto{\pgfqpoint{2.002912in}{1.098705in}}%
\pgfpathlineto{\pgfqpoint{2.002513in}{1.074759in}}%
\pgfpathlineto{\pgfqpoint{2.001980in}{1.055914in}}%
\pgfpathlineto{\pgfqpoint{2.010159in}{1.057965in}}%
\pgfpathlineto{\pgfqpoint{2.013608in}{1.046935in}}%
\pgfpathclose%
\pgfusepath{fill}%
\end{pgfscope}%
\begin{pgfscope}%
\pgfpathrectangle{\pgfqpoint{0.100000in}{0.100000in}}{\pgfqpoint{3.608454in}{2.310000in}}%
\pgfusepath{clip}%
\pgfsetbuttcap%
\pgfsetmiterjoin%
\definecolor{currentfill}{rgb}{0.000000,0.482353,0.758824}%
\pgfsetfillcolor{currentfill}%
\pgfsetlinewidth{0.000000pt}%
\definecolor{currentstroke}{rgb}{0.000000,0.000000,0.000000}%
\pgfsetstrokecolor{currentstroke}%
\pgfsetstrokeopacity{0.000000}%
\pgfsetdash{}{0pt}%
\pgfpathmoveto{\pgfqpoint{2.111687in}{1.695719in}}%
\pgfpathlineto{\pgfqpoint{2.115164in}{1.695734in}}%
\pgfpathlineto{\pgfqpoint{2.115120in}{1.723283in}}%
\pgfpathlineto{\pgfqpoint{2.177278in}{1.723751in}}%
\pgfpathlineto{\pgfqpoint{2.184096in}{1.723863in}}%
\pgfpathlineto{\pgfqpoint{2.184501in}{1.696223in}}%
\pgfpathlineto{\pgfqpoint{2.165979in}{1.695985in}}%
\pgfpathlineto{\pgfqpoint{2.138833in}{1.695825in}}%
\pgfpathlineto{\pgfqpoint{2.138939in}{1.676359in}}%
\pgfpathlineto{\pgfqpoint{2.111765in}{1.676267in}}%
\pgfpathlineto{\pgfqpoint{2.111687in}{1.695719in}}%
\pgfpathclose%
\pgfusepath{fill}%
\end{pgfscope}%
\begin{pgfscope}%
\pgfpathrectangle{\pgfqpoint{0.100000in}{0.100000in}}{\pgfqpoint{3.608454in}{2.310000in}}%
\pgfusepath{clip}%
\pgfsetbuttcap%
\pgfsetmiterjoin%
\definecolor{currentfill}{rgb}{0.000000,0.823529,0.588235}%
\pgfsetfillcolor{currentfill}%
\pgfsetlinewidth{0.000000pt}%
\definecolor{currentstroke}{rgb}{0.000000,0.000000,0.000000}%
\pgfsetstrokecolor{currentstroke}%
\pgfsetstrokeopacity{0.000000}%
\pgfsetdash{}{0pt}%
\pgfpathmoveto{\pgfqpoint{2.804581in}{1.191394in}}%
\pgfpathlineto{\pgfqpoint{2.789452in}{1.189777in}}%
\pgfpathlineto{\pgfqpoint{2.783865in}{1.195392in}}%
\pgfpathlineto{\pgfqpoint{2.779295in}{1.206052in}}%
\pgfpathlineto{\pgfqpoint{2.781636in}{1.214245in}}%
\pgfpathlineto{\pgfqpoint{2.778898in}{1.226398in}}%
\pgfpathlineto{\pgfqpoint{2.780361in}{1.233265in}}%
\pgfpathlineto{\pgfqpoint{2.781851in}{1.240501in}}%
\pgfpathlineto{\pgfqpoint{2.788710in}{1.248796in}}%
\pgfpathlineto{\pgfqpoint{2.797121in}{1.243378in}}%
\pgfpathlineto{\pgfqpoint{2.801049in}{1.243631in}}%
\pgfpathlineto{\pgfqpoint{2.809988in}{1.252552in}}%
\pgfpathlineto{\pgfqpoint{2.816644in}{1.252449in}}%
\pgfpathlineto{\pgfqpoint{2.826981in}{1.247494in}}%
\pgfpathlineto{\pgfqpoint{2.826345in}{1.240166in}}%
\pgfpathlineto{\pgfqpoint{2.831285in}{1.222519in}}%
\pgfpathlineto{\pgfqpoint{2.815210in}{1.209500in}}%
\pgfpathlineto{\pgfqpoint{2.812724in}{1.202234in}}%
\pgfpathlineto{\pgfqpoint{2.806029in}{1.196000in}}%
\pgfpathlineto{\pgfqpoint{2.804581in}{1.191394in}}%
\pgfpathclose%
\pgfusepath{fill}%
\end{pgfscope}%
\begin{pgfscope}%
\pgfpathrectangle{\pgfqpoint{0.100000in}{0.100000in}}{\pgfqpoint{3.608454in}{2.310000in}}%
\pgfusepath{clip}%
\pgfsetbuttcap%
\pgfsetmiterjoin%
\definecolor{currentfill}{rgb}{0.000000,0.615686,0.692157}%
\pgfsetfillcolor{currentfill}%
\pgfsetlinewidth{0.000000pt}%
\definecolor{currentstroke}{rgb}{0.000000,0.000000,0.000000}%
\pgfsetstrokecolor{currentstroke}%
\pgfsetstrokeopacity{0.000000}%
\pgfsetdash{}{0pt}%
\pgfpathmoveto{\pgfqpoint{2.765354in}{1.245580in}}%
\pgfpathlineto{\pgfqpoint{2.761764in}{1.247341in}}%
\pgfpathlineto{\pgfqpoint{2.752556in}{1.236924in}}%
\pgfpathlineto{\pgfqpoint{2.751350in}{1.246500in}}%
\pgfpathlineto{\pgfqpoint{2.742755in}{1.250848in}}%
\pgfpathlineto{\pgfqpoint{2.741472in}{1.258776in}}%
\pgfpathlineto{\pgfqpoint{2.729948in}{1.258824in}}%
\pgfpathlineto{\pgfqpoint{2.729699in}{1.283399in}}%
\pgfpathlineto{\pgfqpoint{2.727518in}{1.286122in}}%
\pgfpathlineto{\pgfqpoint{2.733018in}{1.291688in}}%
\pgfpathlineto{\pgfqpoint{2.740147in}{1.292411in}}%
\pgfpathlineto{\pgfqpoint{2.757404in}{1.276520in}}%
\pgfpathlineto{\pgfqpoint{2.759930in}{1.280063in}}%
\pgfpathlineto{\pgfqpoint{2.773095in}{1.263245in}}%
\pgfpathlineto{\pgfqpoint{2.767417in}{1.258344in}}%
\pgfpathlineto{\pgfqpoint{2.765354in}{1.245580in}}%
\pgfpathclose%
\pgfusepath{fill}%
\end{pgfscope}%
\begin{pgfscope}%
\pgfpathrectangle{\pgfqpoint{0.100000in}{0.100000in}}{\pgfqpoint{3.608454in}{2.310000in}}%
\pgfusepath{clip}%
\pgfsetbuttcap%
\pgfsetmiterjoin%
\definecolor{currentfill}{rgb}{0.000000,0.403922,0.798039}%
\pgfsetfillcolor{currentfill}%
\pgfsetlinewidth{0.000000pt}%
\definecolor{currentstroke}{rgb}{0.000000,0.000000,0.000000}%
\pgfsetstrokecolor{currentstroke}%
\pgfsetstrokeopacity{0.000000}%
\pgfsetdash{}{0pt}%
\pgfpathmoveto{\pgfqpoint{1.682362in}{1.462771in}}%
\pgfpathlineto{\pgfqpoint{1.677720in}{1.401672in}}%
\pgfpathlineto{\pgfqpoint{1.610603in}{1.406638in}}%
\pgfpathlineto{\pgfqpoint{1.577074in}{1.409804in}}%
\pgfpathlineto{\pgfqpoint{1.580447in}{1.444028in}}%
\pgfpathlineto{\pgfqpoint{1.594536in}{1.442692in}}%
\pgfpathlineto{\pgfqpoint{1.598652in}{1.483744in}}%
\pgfpathlineto{\pgfqpoint{1.591803in}{1.484401in}}%
\pgfpathlineto{\pgfqpoint{1.595804in}{1.522015in}}%
\pgfpathlineto{\pgfqpoint{1.650561in}{1.517162in}}%
\pgfpathlineto{\pgfqpoint{1.649013in}{1.497206in}}%
\pgfpathlineto{\pgfqpoint{1.684833in}{1.494331in}}%
\pgfpathlineto{\pgfqpoint{1.684512in}{1.490228in}}%
\pgfpathlineto{\pgfqpoint{1.682362in}{1.462771in}}%
\pgfpathclose%
\pgfusepath{fill}%
\end{pgfscope}%
\begin{pgfscope}%
\pgfpathrectangle{\pgfqpoint{0.100000in}{0.100000in}}{\pgfqpoint{3.608454in}{2.310000in}}%
\pgfusepath{clip}%
\pgfsetbuttcap%
\pgfsetmiterjoin%
\definecolor{currentfill}{rgb}{0.000000,0.470588,0.764706}%
\pgfsetfillcolor{currentfill}%
\pgfsetlinewidth{0.000000pt}%
\definecolor{currentstroke}{rgb}{0.000000,0.000000,0.000000}%
\pgfsetstrokecolor{currentstroke}%
\pgfsetstrokeopacity{0.000000}%
\pgfsetdash{}{0pt}%
\pgfpathmoveto{\pgfqpoint{1.846276in}{0.936427in}}%
\pgfpathlineto{\pgfqpoint{1.844714in}{0.901800in}}%
\pgfpathlineto{\pgfqpoint{1.810377in}{0.903512in}}%
\pgfpathlineto{\pgfqpoint{1.811548in}{0.930044in}}%
\pgfpathlineto{\pgfqpoint{1.811868in}{0.938019in}}%
\pgfpathlineto{\pgfqpoint{1.846276in}{0.936427in}}%
\pgfpathclose%
\pgfusepath{fill}%
\end{pgfscope}%
\begin{pgfscope}%
\pgfpathrectangle{\pgfqpoint{0.100000in}{0.100000in}}{\pgfqpoint{3.608454in}{2.310000in}}%
\pgfusepath{clip}%
\pgfsetbuttcap%
\pgfsetmiterjoin%
\definecolor{currentfill}{rgb}{0.000000,0.588235,0.705882}%
\pgfsetfillcolor{currentfill}%
\pgfsetlinewidth{0.000000pt}%
\definecolor{currentstroke}{rgb}{0.000000,0.000000,0.000000}%
\pgfsetstrokecolor{currentstroke}%
\pgfsetstrokeopacity{0.000000}%
\pgfsetdash{}{0pt}%
\pgfpathmoveto{\pgfqpoint{2.238456in}{1.359055in}}%
\pgfpathlineto{\pgfqpoint{2.232957in}{1.349258in}}%
\pgfpathlineto{\pgfqpoint{2.233532in}{1.346189in}}%
\pgfpathlineto{\pgfqpoint{2.226701in}{1.338754in}}%
\pgfpathlineto{\pgfqpoint{2.226654in}{1.335261in}}%
\pgfpathlineto{\pgfqpoint{2.199219in}{1.335898in}}%
\pgfpathlineto{\pgfqpoint{2.198627in}{1.357636in}}%
\pgfpathlineto{\pgfqpoint{2.188988in}{1.359699in}}%
\pgfpathlineto{\pgfqpoint{2.182827in}{1.356530in}}%
\pgfpathlineto{\pgfqpoint{2.182439in}{1.381621in}}%
\pgfpathlineto{\pgfqpoint{2.181819in}{1.416001in}}%
\pgfpathlineto{\pgfqpoint{2.206018in}{1.417015in}}%
\pgfpathlineto{\pgfqpoint{2.206226in}{1.396171in}}%
\pgfpathlineto{\pgfqpoint{2.211089in}{1.394791in}}%
\pgfpathlineto{\pgfqpoint{2.212943in}{1.384548in}}%
\pgfpathlineto{\pgfqpoint{2.211917in}{1.379066in}}%
\pgfpathlineto{\pgfqpoint{2.215503in}{1.374729in}}%
\pgfpathlineto{\pgfqpoint{2.221539in}{1.373866in}}%
\pgfpathlineto{\pgfqpoint{2.224430in}{1.367560in}}%
\pgfpathlineto{\pgfqpoint{2.230926in}{1.365957in}}%
\pgfpathlineto{\pgfqpoint{2.233381in}{1.359171in}}%
\pgfpathlineto{\pgfqpoint{2.238456in}{1.359055in}}%
\pgfpathclose%
\pgfusepath{fill}%
\end{pgfscope}%
\begin{pgfscope}%
\pgfpathrectangle{\pgfqpoint{0.100000in}{0.100000in}}{\pgfqpoint{3.608454in}{2.310000in}}%
\pgfusepath{clip}%
\pgfsetbuttcap%
\pgfsetmiterjoin%
\definecolor{currentfill}{rgb}{0.000000,0.356863,0.821569}%
\pgfsetfillcolor{currentfill}%
\pgfsetlinewidth{0.000000pt}%
\definecolor{currentstroke}{rgb}{0.000000,0.000000,0.000000}%
\pgfsetstrokecolor{currentstroke}%
\pgfsetstrokeopacity{0.000000}%
\pgfsetdash{}{0pt}%
\pgfpathmoveto{\pgfqpoint{2.886708in}{1.438806in}}%
\pgfpathlineto{\pgfqpoint{2.887216in}{1.431682in}}%
\pgfpathlineto{\pgfqpoint{2.880330in}{1.431219in}}%
\pgfpathlineto{\pgfqpoint{2.880748in}{1.424222in}}%
\pgfpathlineto{\pgfqpoint{2.867003in}{1.423229in}}%
\pgfpathlineto{\pgfqpoint{2.867100in}{1.421144in}}%
\pgfpathlineto{\pgfqpoint{2.853130in}{1.419128in}}%
\pgfpathlineto{\pgfqpoint{2.852553in}{1.427030in}}%
\pgfpathlineto{\pgfqpoint{2.837395in}{1.428067in}}%
\pgfpathlineto{\pgfqpoint{2.820238in}{1.426570in}}%
\pgfpathlineto{\pgfqpoint{2.819277in}{1.440727in}}%
\pgfpathlineto{\pgfqpoint{2.794880in}{1.439319in}}%
\pgfpathlineto{\pgfqpoint{2.798306in}{1.443885in}}%
\pgfpathlineto{\pgfqpoint{2.800713in}{1.463456in}}%
\pgfpathlineto{\pgfqpoint{2.799407in}{1.480427in}}%
\pgfpathlineto{\pgfqpoint{2.795940in}{1.480285in}}%
\pgfpathlineto{\pgfqpoint{2.795078in}{1.502065in}}%
\pgfpathlineto{\pgfqpoint{2.801132in}{1.502933in}}%
\pgfpathlineto{\pgfqpoint{2.811270in}{1.504385in}}%
\pgfpathlineto{\pgfqpoint{2.811880in}{1.499432in}}%
\pgfpathlineto{\pgfqpoint{2.831500in}{1.499705in}}%
\pgfpathlineto{\pgfqpoint{2.831782in}{1.495222in}}%
\pgfpathlineto{\pgfqpoint{2.848845in}{1.496480in}}%
\pgfpathlineto{\pgfqpoint{2.846858in}{1.512561in}}%
\pgfpathlineto{\pgfqpoint{2.862804in}{1.514836in}}%
\pgfpathlineto{\pgfqpoint{2.871819in}{1.517489in}}%
\pgfpathlineto{\pgfqpoint{2.874225in}{1.518225in}}%
\pgfpathlineto{\pgfqpoint{2.876912in}{1.491928in}}%
\pgfpathlineto{\pgfqpoint{2.878220in}{1.486351in}}%
\pgfpathlineto{\pgfqpoint{2.879950in}{1.469292in}}%
\pgfpathlineto{\pgfqpoint{2.878244in}{1.466104in}}%
\pgfpathlineto{\pgfqpoint{2.864318in}{1.465519in}}%
\pgfpathlineto{\pgfqpoint{2.864747in}{1.458144in}}%
\pgfpathlineto{\pgfqpoint{2.869385in}{1.458446in}}%
\pgfpathlineto{\pgfqpoint{2.872185in}{1.451622in}}%
\pgfpathlineto{\pgfqpoint{2.872947in}{1.440094in}}%
\pgfpathlineto{\pgfqpoint{2.886708in}{1.438806in}}%
\pgfpathclose%
\pgfusepath{fill}%
\end{pgfscope}%
\begin{pgfscope}%
\pgfpathrectangle{\pgfqpoint{0.100000in}{0.100000in}}{\pgfqpoint{3.608454in}{2.310000in}}%
\pgfusepath{clip}%
\pgfsetbuttcap%
\pgfsetmiterjoin%
\definecolor{currentfill}{rgb}{0.000000,0.627451,0.686275}%
\pgfsetfillcolor{currentfill}%
\pgfsetlinewidth{0.000000pt}%
\definecolor{currentstroke}{rgb}{0.000000,0.000000,0.000000}%
\pgfsetstrokecolor{currentstroke}%
\pgfsetstrokeopacity{0.000000}%
\pgfsetdash{}{0pt}%
\pgfpathmoveto{\pgfqpoint{2.130612in}{1.428716in}}%
\pgfpathlineto{\pgfqpoint{2.130971in}{1.449316in}}%
\pgfpathlineto{\pgfqpoint{2.128892in}{1.463893in}}%
\pgfpathlineto{\pgfqpoint{2.138550in}{1.463912in}}%
\pgfpathlineto{\pgfqpoint{2.180394in}{1.464846in}}%
\pgfpathlineto{\pgfqpoint{2.181491in}{1.429705in}}%
\pgfpathlineto{\pgfqpoint{2.154147in}{1.429674in}}%
\pgfpathlineto{\pgfqpoint{2.154074in}{1.421652in}}%
\pgfpathlineto{\pgfqpoint{2.130676in}{1.421843in}}%
\pgfpathlineto{\pgfqpoint{2.130612in}{1.428716in}}%
\pgfpathclose%
\pgfusepath{fill}%
\end{pgfscope}%
\begin{pgfscope}%
\pgfpathrectangle{\pgfqpoint{0.100000in}{0.100000in}}{\pgfqpoint{3.608454in}{2.310000in}}%
\pgfusepath{clip}%
\pgfsetbuttcap%
\pgfsetmiterjoin%
\definecolor{currentfill}{rgb}{0.000000,0.427451,0.786275}%
\pgfsetfillcolor{currentfill}%
\pgfsetlinewidth{0.000000pt}%
\definecolor{currentstroke}{rgb}{0.000000,0.000000,0.000000}%
\pgfsetstrokecolor{currentstroke}%
\pgfsetstrokeopacity{0.000000}%
\pgfsetdash{}{0pt}%
\pgfpathmoveto{\pgfqpoint{2.073124in}{1.778481in}}%
\pgfpathlineto{\pgfqpoint{2.072659in}{1.785449in}}%
\pgfpathlineto{\pgfqpoint{2.024747in}{1.786093in}}%
\pgfpathlineto{\pgfqpoint{2.025168in}{1.813601in}}%
\pgfpathlineto{\pgfqpoint{2.025490in}{1.836778in}}%
\pgfpathlineto{\pgfqpoint{2.034935in}{1.834889in}}%
\pgfpathlineto{\pgfqpoint{2.041094in}{1.830482in}}%
\pgfpathlineto{\pgfqpoint{2.044201in}{1.834191in}}%
\pgfpathlineto{\pgfqpoint{2.044431in}{1.847876in}}%
\pgfpathlineto{\pgfqpoint{2.099439in}{1.847533in}}%
\pgfpathlineto{\pgfqpoint{2.120016in}{1.847577in}}%
\pgfpathlineto{\pgfqpoint{2.120021in}{1.840682in}}%
\pgfpathlineto{\pgfqpoint{2.120405in}{1.806180in}}%
\pgfpathlineto{\pgfqpoint{2.079580in}{1.806133in}}%
\pgfpathlineto{\pgfqpoint{2.079554in}{1.795195in}}%
\pgfpathlineto{\pgfqpoint{2.086458in}{1.790823in}}%
\pgfpathlineto{\pgfqpoint{2.086396in}{1.778361in}}%
\pgfpathlineto{\pgfqpoint{2.073124in}{1.778481in}}%
\pgfpathclose%
\pgfusepath{fill}%
\end{pgfscope}%
\begin{pgfscope}%
\pgfpathrectangle{\pgfqpoint{0.100000in}{0.100000in}}{\pgfqpoint{3.608454in}{2.310000in}}%
\pgfusepath{clip}%
\pgfsetbuttcap%
\pgfsetmiterjoin%
\definecolor{currentfill}{rgb}{0.000000,0.705882,0.647059}%
\pgfsetfillcolor{currentfill}%
\pgfsetlinewidth{0.000000pt}%
\definecolor{currentstroke}{rgb}{0.000000,0.000000,0.000000}%
\pgfsetstrokecolor{currentstroke}%
\pgfsetstrokeopacity{0.000000}%
\pgfsetdash{}{0pt}%
\pgfpathmoveto{\pgfqpoint{2.248336in}{1.167927in}}%
\pgfpathlineto{\pgfqpoint{2.248459in}{1.143536in}}%
\pgfpathlineto{\pgfqpoint{2.243255in}{1.143432in}}%
\pgfpathlineto{\pgfqpoint{2.178864in}{1.142325in}}%
\pgfpathlineto{\pgfqpoint{2.165488in}{1.142173in}}%
\pgfpathlineto{\pgfqpoint{2.165811in}{1.161887in}}%
\pgfpathlineto{\pgfqpoint{2.166408in}{1.204778in}}%
\pgfpathlineto{\pgfqpoint{2.193249in}{1.204475in}}%
\pgfpathlineto{\pgfqpoint{2.194554in}{1.204465in}}%
\pgfpathlineto{\pgfqpoint{2.194584in}{1.181869in}}%
\pgfpathlineto{\pgfqpoint{2.211810in}{1.181895in}}%
\pgfpathlineto{\pgfqpoint{2.211773in}{1.168036in}}%
\pgfpathlineto{\pgfqpoint{2.248336in}{1.167927in}}%
\pgfpathclose%
\pgfusepath{fill}%
\end{pgfscope}%
\begin{pgfscope}%
\pgfpathrectangle{\pgfqpoint{0.100000in}{0.100000in}}{\pgfqpoint{3.608454in}{2.310000in}}%
\pgfusepath{clip}%
\pgfsetbuttcap%
\pgfsetmiterjoin%
\definecolor{currentfill}{rgb}{0.000000,0.517647,0.741176}%
\pgfsetfillcolor{currentfill}%
\pgfsetlinewidth{0.000000pt}%
\definecolor{currentstroke}{rgb}{0.000000,0.000000,0.000000}%
\pgfsetstrokecolor{currentstroke}%
\pgfsetstrokeopacity{0.000000}%
\pgfsetdash{}{0pt}%
\pgfpathmoveto{\pgfqpoint{2.722959in}{1.184982in}}%
\pgfpathlineto{\pgfqpoint{2.754147in}{1.186574in}}%
\pgfpathlineto{\pgfqpoint{2.758225in}{1.180770in}}%
\pgfpathlineto{\pgfqpoint{2.744380in}{1.183721in}}%
\pgfpathlineto{\pgfqpoint{2.738211in}{1.181250in}}%
\pgfpathlineto{\pgfqpoint{2.734679in}{1.168795in}}%
\pgfpathlineto{\pgfqpoint{2.735445in}{1.160714in}}%
\pgfpathlineto{\pgfqpoint{2.738900in}{1.152461in}}%
\pgfpathlineto{\pgfqpoint{2.735267in}{1.144371in}}%
\pgfpathlineto{\pgfqpoint{2.729930in}{1.144288in}}%
\pgfpathlineto{\pgfqpoint{2.729200in}{1.134130in}}%
\pgfpathlineto{\pgfqpoint{2.722754in}{1.138754in}}%
\pgfpathlineto{\pgfqpoint{2.712831in}{1.140583in}}%
\pgfpathlineto{\pgfqpoint{2.703971in}{1.134522in}}%
\pgfpathlineto{\pgfqpoint{2.701451in}{1.138405in}}%
\pgfpathlineto{\pgfqpoint{2.695803in}{1.138527in}}%
\pgfpathlineto{\pgfqpoint{2.693376in}{1.142517in}}%
\pgfpathlineto{\pgfqpoint{2.693285in}{1.148452in}}%
\pgfpathlineto{\pgfqpoint{2.689646in}{1.154594in}}%
\pgfpathlineto{\pgfqpoint{2.690027in}{1.171438in}}%
\pgfpathlineto{\pgfqpoint{2.696790in}{1.173951in}}%
\pgfpathlineto{\pgfqpoint{2.705300in}{1.170782in}}%
\pgfpathlineto{\pgfqpoint{2.710970in}{1.165941in}}%
\pgfpathlineto{\pgfqpoint{2.713741in}{1.174022in}}%
\pgfpathlineto{\pgfqpoint{2.722324in}{1.177007in}}%
\pgfpathlineto{\pgfqpoint{2.722959in}{1.184982in}}%
\pgfpathclose%
\pgfusepath{fill}%
\end{pgfscope}%
\begin{pgfscope}%
\pgfpathrectangle{\pgfqpoint{0.100000in}{0.100000in}}{\pgfqpoint{3.608454in}{2.310000in}}%
\pgfusepath{clip}%
\pgfsetbuttcap%
\pgfsetmiterjoin%
\definecolor{currentfill}{rgb}{0.000000,0.341176,0.829412}%
\pgfsetfillcolor{currentfill}%
\pgfsetlinewidth{0.000000pt}%
\definecolor{currentstroke}{rgb}{0.000000,0.000000,0.000000}%
\pgfsetstrokecolor{currentstroke}%
\pgfsetstrokeopacity{0.000000}%
\pgfsetdash{}{0pt}%
\pgfpathmoveto{\pgfqpoint{1.586973in}{1.721839in}}%
\pgfpathlineto{\pgfqpoint{1.538610in}{1.726627in}}%
\pgfpathlineto{\pgfqpoint{1.474731in}{1.733639in}}%
\pgfpathlineto{\pgfqpoint{1.481253in}{1.786760in}}%
\pgfpathlineto{\pgfqpoint{1.486605in}{1.827974in}}%
\pgfpathlineto{\pgfqpoint{1.488701in}{1.851762in}}%
\pgfpathlineto{\pgfqpoint{1.494987in}{1.851589in}}%
\pgfpathlineto{\pgfqpoint{1.543825in}{1.845887in}}%
\pgfpathlineto{\pgfqpoint{1.598608in}{1.839877in}}%
\pgfpathlineto{\pgfqpoint{1.592333in}{1.775294in}}%
\pgfpathlineto{\pgfqpoint{1.586973in}{1.721839in}}%
\pgfpathclose%
\pgfusepath{fill}%
\end{pgfscope}%
\begin{pgfscope}%
\pgfpathrectangle{\pgfqpoint{0.100000in}{0.100000in}}{\pgfqpoint{3.608454in}{2.310000in}}%
\pgfusepath{clip}%
\pgfsetbuttcap%
\pgfsetmiterjoin%
\definecolor{currentfill}{rgb}{0.000000,0.776471,0.611765}%
\pgfsetfillcolor{currentfill}%
\pgfsetlinewidth{0.000000pt}%
\definecolor{currentstroke}{rgb}{0.000000,0.000000,0.000000}%
\pgfsetstrokecolor{currentstroke}%
\pgfsetstrokeopacity{0.000000}%
\pgfsetdash{}{0pt}%
\pgfpathmoveto{\pgfqpoint{1.365564in}{0.266179in}}%
\pgfpathlineto{\pgfqpoint{1.364279in}{0.272148in}}%
\pgfpathlineto{\pgfqpoint{1.358083in}{0.284462in}}%
\pgfpathlineto{\pgfqpoint{1.360076in}{0.291758in}}%
\pgfpathlineto{\pgfqpoint{1.373325in}{0.304088in}}%
\pgfpathlineto{\pgfqpoint{1.377160in}{0.317108in}}%
\pgfpathlineto{\pgfqpoint{1.381805in}{0.326247in}}%
\pgfpathlineto{\pgfqpoint{1.380932in}{0.332508in}}%
\pgfpathlineto{\pgfqpoint{1.385909in}{0.340951in}}%
\pgfpathlineto{\pgfqpoint{1.410377in}{0.343837in}}%
\pgfpathlineto{\pgfqpoint{1.412863in}{0.346234in}}%
\pgfpathlineto{\pgfqpoint{1.413879in}{0.356269in}}%
\pgfpathlineto{\pgfqpoint{1.419360in}{0.363772in}}%
\pgfpathlineto{\pgfqpoint{1.424674in}{0.362812in}}%
\pgfpathlineto{\pgfqpoint{1.428368in}{0.354664in}}%
\pgfpathlineto{\pgfqpoint{1.431876in}{0.341640in}}%
\pgfpathlineto{\pgfqpoint{1.439179in}{0.333553in}}%
\pgfpathlineto{\pgfqpoint{1.446187in}{0.319277in}}%
\pgfpathlineto{\pgfqpoint{1.449694in}{0.299992in}}%
\pgfpathlineto{\pgfqpoint{1.442998in}{0.291249in}}%
\pgfpathlineto{\pgfqpoint{1.448609in}{0.287822in}}%
\pgfpathlineto{\pgfqpoint{1.445177in}{0.280721in}}%
\pgfpathlineto{\pgfqpoint{1.446258in}{0.271967in}}%
\pgfpathlineto{\pgfqpoint{1.449844in}{0.264785in}}%
\pgfpathlineto{\pgfqpoint{1.446708in}{0.261925in}}%
\pgfpathlineto{\pgfqpoint{1.430233in}{0.260876in}}%
\pgfpathlineto{\pgfqpoint{1.416961in}{0.263446in}}%
\pgfpathlineto{\pgfqpoint{1.406310in}{0.270716in}}%
\pgfpathlineto{\pgfqpoint{1.400271in}{0.270027in}}%
\pgfpathlineto{\pgfqpoint{1.387382in}{0.272320in}}%
\pgfpathlineto{\pgfqpoint{1.365564in}{0.266179in}}%
\pgfpathclose%
\pgfusepath{fill}%
\end{pgfscope}%
\begin{pgfscope}%
\pgfpathrectangle{\pgfqpoint{0.100000in}{0.100000in}}{\pgfqpoint{3.608454in}{2.310000in}}%
\pgfusepath{clip}%
\pgfsetbuttcap%
\pgfsetmiterjoin%
\definecolor{currentfill}{rgb}{0.000000,0.368627,0.815686}%
\pgfsetfillcolor{currentfill}%
\pgfsetlinewidth{0.000000pt}%
\definecolor{currentstroke}{rgb}{0.000000,0.000000,0.000000}%
\pgfsetstrokecolor{currentstroke}%
\pgfsetstrokeopacity{0.000000}%
\pgfsetdash{}{0pt}%
\pgfpathmoveto{\pgfqpoint{2.168433in}{1.778817in}}%
\pgfpathlineto{\pgfqpoint{2.167464in}{1.771934in}}%
\pgfpathlineto{\pgfqpoint{2.128088in}{1.771585in}}%
\pgfpathlineto{\pgfqpoint{2.127691in}{1.792284in}}%
\pgfpathlineto{\pgfqpoint{2.135124in}{1.792351in}}%
\pgfpathlineto{\pgfqpoint{2.135067in}{1.806256in}}%
\pgfpathlineto{\pgfqpoint{2.120405in}{1.806180in}}%
\pgfpathlineto{\pgfqpoint{2.120021in}{1.840682in}}%
\pgfpathlineto{\pgfqpoint{2.141258in}{1.840834in}}%
\pgfpathlineto{\pgfqpoint{2.148148in}{1.837483in}}%
\pgfpathlineto{\pgfqpoint{2.148622in}{1.813278in}}%
\pgfpathlineto{\pgfqpoint{2.162319in}{1.813349in}}%
\pgfpathlineto{\pgfqpoint{2.162655in}{1.792587in}}%
\pgfpathlineto{\pgfqpoint{2.169520in}{1.792650in}}%
\pgfpathlineto{\pgfqpoint{2.168433in}{1.778817in}}%
\pgfpathclose%
\pgfusepath{fill}%
\end{pgfscope}%
\begin{pgfscope}%
\pgfpathrectangle{\pgfqpoint{0.100000in}{0.100000in}}{\pgfqpoint{3.608454in}{2.310000in}}%
\pgfusepath{clip}%
\pgfsetbuttcap%
\pgfsetmiterjoin%
\definecolor{currentfill}{rgb}{0.000000,0.454902,0.772549}%
\pgfsetfillcolor{currentfill}%
\pgfsetlinewidth{0.000000pt}%
\definecolor{currentstroke}{rgb}{0.000000,0.000000,0.000000}%
\pgfsetstrokecolor{currentstroke}%
\pgfsetstrokeopacity{0.000000}%
\pgfsetdash{}{0pt}%
\pgfpathmoveto{\pgfqpoint{2.636960in}{1.295013in}}%
\pgfpathlineto{\pgfqpoint{2.634127in}{1.288599in}}%
\pgfpathlineto{\pgfqpoint{2.635552in}{1.279697in}}%
\pgfpathlineto{\pgfqpoint{2.630595in}{1.278625in}}%
\pgfpathlineto{\pgfqpoint{2.627023in}{1.272051in}}%
\pgfpathlineto{\pgfqpoint{2.627054in}{1.277303in}}%
\pgfpathlineto{\pgfqpoint{2.620883in}{1.276667in}}%
\pgfpathlineto{\pgfqpoint{2.617290in}{1.282808in}}%
\pgfpathlineto{\pgfqpoint{2.602387in}{1.273595in}}%
\pgfpathlineto{\pgfqpoint{2.601681in}{1.267043in}}%
\pgfpathlineto{\pgfqpoint{2.586412in}{1.272897in}}%
\pgfpathlineto{\pgfqpoint{2.578231in}{1.275886in}}%
\pgfpathlineto{\pgfqpoint{2.570793in}{1.273710in}}%
\pgfpathlineto{\pgfqpoint{2.567447in}{1.266096in}}%
\pgfpathlineto{\pgfqpoint{2.562848in}{1.271335in}}%
\pgfpathlineto{\pgfqpoint{2.553706in}{1.268477in}}%
\pgfpathlineto{\pgfqpoint{2.546912in}{1.269077in}}%
\pgfpathlineto{\pgfqpoint{2.549388in}{1.262674in}}%
\pgfpathlineto{\pgfqpoint{2.541890in}{1.261426in}}%
\pgfpathlineto{\pgfqpoint{2.537345in}{1.268351in}}%
\pgfpathlineto{\pgfqpoint{2.545025in}{1.287628in}}%
\pgfpathlineto{\pgfqpoint{2.541416in}{1.297901in}}%
\pgfpathlineto{\pgfqpoint{2.543804in}{1.301922in}}%
\pgfpathlineto{\pgfqpoint{2.541523in}{1.307274in}}%
\pgfpathlineto{\pgfqpoint{2.542054in}{1.316771in}}%
\pgfpathlineto{\pgfqpoint{2.544363in}{1.322774in}}%
\pgfpathlineto{\pgfqpoint{2.560427in}{1.323864in}}%
\pgfpathlineto{\pgfqpoint{2.565371in}{1.317560in}}%
\pgfpathlineto{\pgfqpoint{2.572272in}{1.322088in}}%
\pgfpathlineto{\pgfqpoint{2.585800in}{1.324135in}}%
\pgfpathlineto{\pgfqpoint{2.594128in}{1.324280in}}%
\pgfpathlineto{\pgfqpoint{2.608516in}{1.322156in}}%
\pgfpathlineto{\pgfqpoint{2.610839in}{1.324868in}}%
\pgfpathlineto{\pgfqpoint{2.620401in}{1.325720in}}%
\pgfpathlineto{\pgfqpoint{2.621408in}{1.315375in}}%
\pgfpathlineto{\pgfqpoint{2.622478in}{1.305021in}}%
\pgfpathlineto{\pgfqpoint{2.629213in}{1.305901in}}%
\pgfpathlineto{\pgfqpoint{2.629550in}{1.301264in}}%
\pgfpathlineto{\pgfqpoint{2.636434in}{1.301852in}}%
\pgfpathlineto{\pgfqpoint{2.636960in}{1.295013in}}%
\pgfpathclose%
\pgfusepath{fill}%
\end{pgfscope}%
\begin{pgfscope}%
\pgfpathrectangle{\pgfqpoint{0.100000in}{0.100000in}}{\pgfqpoint{3.608454in}{2.310000in}}%
\pgfusepath{clip}%
\pgfsetbuttcap%
\pgfsetmiterjoin%
\definecolor{currentfill}{rgb}{0.000000,0.721569,0.639216}%
\pgfsetfillcolor{currentfill}%
\pgfsetlinewidth{0.000000pt}%
\definecolor{currentstroke}{rgb}{0.000000,0.000000,0.000000}%
\pgfsetstrokecolor{currentstroke}%
\pgfsetstrokeopacity{0.000000}%
\pgfsetdash{}{0pt}%
\pgfpathmoveto{\pgfqpoint{2.295263in}{1.958166in}}%
\pgfpathlineto{\pgfqpoint{2.297058in}{1.910416in}}%
\pgfpathlineto{\pgfqpoint{2.269571in}{1.909501in}}%
\pgfpathlineto{\pgfqpoint{2.256111in}{1.909051in}}%
\pgfpathlineto{\pgfqpoint{2.255541in}{1.929754in}}%
\pgfpathlineto{\pgfqpoint{2.213676in}{1.928846in}}%
\pgfpathlineto{\pgfqpoint{2.212805in}{1.945663in}}%
\pgfpathlineto{\pgfqpoint{2.212580in}{1.977269in}}%
\pgfpathlineto{\pgfqpoint{2.211215in}{2.006566in}}%
\pgfpathlineto{\pgfqpoint{2.211106in}{2.032691in}}%
\pgfpathlineto{\pgfqpoint{2.209745in}{2.046401in}}%
\pgfpathlineto{\pgfqpoint{2.208627in}{2.060466in}}%
\pgfpathlineto{\pgfqpoint{2.208113in}{2.105322in}}%
\pgfpathlineto{\pgfqpoint{2.213604in}{2.105132in}}%
\pgfpathlineto{\pgfqpoint{2.227390in}{2.098684in}}%
\pgfpathlineto{\pgfqpoint{2.232324in}{2.099097in}}%
\pgfpathlineto{\pgfqpoint{2.231425in}{2.090560in}}%
\pgfpathlineto{\pgfqpoint{2.239307in}{2.091665in}}%
\pgfpathlineto{\pgfqpoint{2.247149in}{2.073655in}}%
\pgfpathlineto{\pgfqpoint{2.251836in}{2.075702in}}%
\pgfpathlineto{\pgfqpoint{2.251160in}{2.083595in}}%
\pgfpathlineto{\pgfqpoint{2.263561in}{2.085282in}}%
\pgfpathlineto{\pgfqpoint{2.266399in}{2.077856in}}%
\pgfpathlineto{\pgfqpoint{2.274138in}{2.073437in}}%
\pgfpathlineto{\pgfqpoint{2.282131in}{2.073030in}}%
\pgfpathlineto{\pgfqpoint{2.282543in}{2.066327in}}%
\pgfpathlineto{\pgfqpoint{2.297364in}{2.061677in}}%
\pgfpathlineto{\pgfqpoint{2.307312in}{2.064841in}}%
\pgfpathlineto{\pgfqpoint{2.318616in}{2.073687in}}%
\pgfpathlineto{\pgfqpoint{2.320258in}{2.049681in}}%
\pgfpathlineto{\pgfqpoint{2.321610in}{2.015873in}}%
\pgfpathlineto{\pgfqpoint{2.309782in}{2.000701in}}%
\pgfpathlineto{\pgfqpoint{2.298141in}{1.987730in}}%
\pgfpathlineto{\pgfqpoint{2.293110in}{1.984555in}}%
\pgfpathlineto{\pgfqpoint{2.282209in}{1.972690in}}%
\pgfpathlineto{\pgfqpoint{2.265774in}{1.959901in}}%
\pgfpathlineto{\pgfqpoint{2.269412in}{1.953632in}}%
\pgfpathlineto{\pgfqpoint{2.274053in}{1.951302in}}%
\pgfpathlineto{\pgfqpoint{2.282372in}{1.952765in}}%
\pgfpathlineto{\pgfqpoint{2.295263in}{1.958166in}}%
\pgfpathclose%
\pgfusepath{fill}%
\end{pgfscope}%
\begin{pgfscope}%
\pgfpathrectangle{\pgfqpoint{0.100000in}{0.100000in}}{\pgfqpoint{3.608454in}{2.310000in}}%
\pgfusepath{clip}%
\pgfsetbuttcap%
\pgfsetmiterjoin%
\definecolor{currentfill}{rgb}{0.000000,0.372549,0.813725}%
\pgfsetfillcolor{currentfill}%
\pgfsetlinewidth{0.000000pt}%
\definecolor{currentstroke}{rgb}{0.000000,0.000000,0.000000}%
\pgfsetstrokecolor{currentstroke}%
\pgfsetstrokeopacity{0.000000}%
\pgfsetdash{}{0pt}%
\pgfpathmoveto{\pgfqpoint{2.287673in}{1.541193in}}%
\pgfpathlineto{\pgfqpoint{2.266869in}{1.540448in}}%
\pgfpathlineto{\pgfqpoint{2.264937in}{1.602761in}}%
\pgfpathlineto{\pgfqpoint{2.306014in}{1.604048in}}%
\pgfpathlineto{\pgfqpoint{2.346907in}{1.605656in}}%
\pgfpathlineto{\pgfqpoint{2.348123in}{1.578046in}}%
\pgfpathlineto{\pgfqpoint{2.320640in}{1.576968in}}%
\pgfpathlineto{\pgfqpoint{2.320902in}{1.570069in}}%
\pgfpathlineto{\pgfqpoint{2.293482in}{1.569183in}}%
\pgfpathlineto{\pgfqpoint{2.294578in}{1.541485in}}%
\pgfpathlineto{\pgfqpoint{2.287673in}{1.541193in}}%
\pgfpathclose%
\pgfusepath{fill}%
\end{pgfscope}%
\begin{pgfscope}%
\pgfpathrectangle{\pgfqpoint{0.100000in}{0.100000in}}{\pgfqpoint{3.608454in}{2.310000in}}%
\pgfusepath{clip}%
\pgfsetbuttcap%
\pgfsetmiterjoin%
\definecolor{currentfill}{rgb}{0.000000,0.549020,0.725490}%
\pgfsetfillcolor{currentfill}%
\pgfsetlinewidth{0.000000pt}%
\definecolor{currentstroke}{rgb}{0.000000,0.000000,0.000000}%
\pgfsetstrokecolor{currentstroke}%
\pgfsetstrokeopacity{0.000000}%
\pgfsetdash{}{0pt}%
\pgfpathmoveto{\pgfqpoint{3.039659in}{0.833647in}}%
\pgfpathlineto{\pgfqpoint{3.033775in}{0.835716in}}%
\pgfpathlineto{\pgfqpoint{3.025040in}{0.841844in}}%
\pgfpathlineto{\pgfqpoint{3.020884in}{0.848437in}}%
\pgfpathlineto{\pgfqpoint{3.011213in}{0.848731in}}%
\pgfpathlineto{\pgfqpoint{3.005118in}{0.855085in}}%
\pgfpathlineto{\pgfqpoint{2.996708in}{0.858258in}}%
\pgfpathlineto{\pgfqpoint{2.994468in}{0.861760in}}%
\pgfpathlineto{\pgfqpoint{3.016488in}{0.872188in}}%
\pgfpathlineto{\pgfqpoint{3.014555in}{0.877148in}}%
\pgfpathlineto{\pgfqpoint{3.008209in}{0.883079in}}%
\pgfpathlineto{\pgfqpoint{3.006499in}{0.892231in}}%
\pgfpathlineto{\pgfqpoint{3.015378in}{0.900512in}}%
\pgfpathlineto{\pgfqpoint{3.022585in}{0.898037in}}%
\pgfpathlineto{\pgfqpoint{3.038277in}{0.916697in}}%
\pgfpathlineto{\pgfqpoint{3.047304in}{0.907519in}}%
\pgfpathlineto{\pgfqpoint{3.051321in}{0.914811in}}%
\pgfpathlineto{\pgfqpoint{3.061084in}{0.913825in}}%
\pgfpathlineto{\pgfqpoint{3.066440in}{0.907291in}}%
\pgfpathlineto{\pgfqpoint{3.071225in}{0.906814in}}%
\pgfpathlineto{\pgfqpoint{3.077208in}{0.901258in}}%
\pgfpathlineto{\pgfqpoint{3.081530in}{0.894395in}}%
\pgfpathlineto{\pgfqpoint{3.080811in}{0.888841in}}%
\pgfpathlineto{\pgfqpoint{3.069735in}{0.881587in}}%
\pgfpathlineto{\pgfqpoint{3.064847in}{0.872766in}}%
\pgfpathlineto{\pgfqpoint{3.055250in}{0.864816in}}%
\pgfpathlineto{\pgfqpoint{3.057975in}{0.856666in}}%
\pgfpathlineto{\pgfqpoint{3.043134in}{0.834472in}}%
\pgfpathlineto{\pgfqpoint{3.039659in}{0.833647in}}%
\pgfpathclose%
\pgfusepath{fill}%
\end{pgfscope}%
\begin{pgfscope}%
\pgfpathrectangle{\pgfqpoint{0.100000in}{0.100000in}}{\pgfqpoint{3.608454in}{2.310000in}}%
\pgfusepath{clip}%
\pgfsetbuttcap%
\pgfsetmiterjoin%
\definecolor{currentfill}{rgb}{0.000000,0.909804,0.545098}%
\pgfsetfillcolor{currentfill}%
\pgfsetlinewidth{0.000000pt}%
\definecolor{currentstroke}{rgb}{0.000000,0.000000,0.000000}%
\pgfsetstrokecolor{currentstroke}%
\pgfsetstrokeopacity{0.000000}%
\pgfsetdash{}{0pt}%
\pgfpathmoveto{\pgfqpoint{0.715935in}{1.921940in}}%
\pgfpathlineto{\pgfqpoint{0.762277in}{1.909354in}}%
\pgfpathlineto{\pgfqpoint{0.764102in}{1.916069in}}%
\pgfpathlineto{\pgfqpoint{0.796408in}{1.906884in}}%
\pgfpathlineto{\pgfqpoint{0.790924in}{1.886917in}}%
\pgfpathlineto{\pgfqpoint{0.773761in}{1.820585in}}%
\pgfpathlineto{\pgfqpoint{0.774453in}{1.820402in}}%
\pgfpathlineto{\pgfqpoint{0.761786in}{1.771448in}}%
\pgfpathlineto{\pgfqpoint{0.757203in}{1.749818in}}%
\pgfpathlineto{\pgfqpoint{0.728707in}{1.756911in}}%
\pgfpathlineto{\pgfqpoint{0.693113in}{1.766738in}}%
\pgfpathlineto{\pgfqpoint{0.691075in}{1.767312in}}%
\pgfpathlineto{\pgfqpoint{0.706614in}{1.825006in}}%
\pgfpathlineto{\pgfqpoint{0.674134in}{1.833816in}}%
\pgfpathlineto{\pgfqpoint{0.684062in}{1.866632in}}%
\pgfpathlineto{\pgfqpoint{0.686047in}{1.866076in}}%
\pgfpathlineto{\pgfqpoint{0.695319in}{1.898980in}}%
\pgfpathlineto{\pgfqpoint{0.697108in}{1.905730in}}%
\pgfpathlineto{\pgfqpoint{0.703763in}{1.903845in}}%
\pgfpathlineto{\pgfqpoint{0.709441in}{1.923753in}}%
\pgfpathlineto{\pgfqpoint{0.715935in}{1.921940in}}%
\pgfpathclose%
\pgfusepath{fill}%
\end{pgfscope}%
\begin{pgfscope}%
\pgfpathrectangle{\pgfqpoint{0.100000in}{0.100000in}}{\pgfqpoint{3.608454in}{2.310000in}}%
\pgfusepath{clip}%
\pgfsetbuttcap%
\pgfsetmiterjoin%
\definecolor{currentfill}{rgb}{0.000000,0.568627,0.715686}%
\pgfsetfillcolor{currentfill}%
\pgfsetlinewidth{0.000000pt}%
\definecolor{currentstroke}{rgb}{0.000000,0.000000,0.000000}%
\pgfsetstrokecolor{currentstroke}%
\pgfsetstrokeopacity{0.000000}%
\pgfsetdash{}{0pt}%
\pgfpathmoveto{\pgfqpoint{3.127726in}{0.403873in}}%
\pgfpathlineto{\pgfqpoint{3.099914in}{0.399053in}}%
\pgfpathlineto{\pgfqpoint{3.096468in}{0.420138in}}%
\pgfpathlineto{\pgfqpoint{3.075618in}{0.416692in}}%
\pgfpathlineto{\pgfqpoint{3.072400in}{0.437158in}}%
\pgfpathlineto{\pgfqpoint{3.069493in}{0.458265in}}%
\pgfpathlineto{\pgfqpoint{3.090529in}{0.461360in}}%
\pgfpathlineto{\pgfqpoint{3.089414in}{0.468427in}}%
\pgfpathlineto{\pgfqpoint{3.096465in}{0.469557in}}%
\pgfpathlineto{\pgfqpoint{3.095343in}{0.476565in}}%
\pgfpathlineto{\pgfqpoint{3.109319in}{0.479362in}}%
\pgfpathlineto{\pgfqpoint{3.100740in}{0.491314in}}%
\pgfpathlineto{\pgfqpoint{3.097174in}{0.491133in}}%
\pgfpathlineto{\pgfqpoint{3.088462in}{0.501164in}}%
\pgfpathlineto{\pgfqpoint{3.091930in}{0.511405in}}%
\pgfpathlineto{\pgfqpoint{3.110913in}{0.514275in}}%
\pgfpathlineto{\pgfqpoint{3.117607in}{0.515401in}}%
\pgfpathlineto{\pgfqpoint{3.118731in}{0.508688in}}%
\pgfpathlineto{\pgfqpoint{3.150978in}{0.513727in}}%
\pgfpathlineto{\pgfqpoint{3.156513in}{0.502751in}}%
\pgfpathlineto{\pgfqpoint{3.167005in}{0.486501in}}%
\pgfpathlineto{\pgfqpoint{3.173443in}{0.474576in}}%
\pgfpathlineto{\pgfqpoint{3.181412in}{0.456522in}}%
\pgfpathlineto{\pgfqpoint{3.183733in}{0.440357in}}%
\pgfpathlineto{\pgfqpoint{3.184616in}{0.418072in}}%
\pgfpathlineto{\pgfqpoint{3.173551in}{0.417385in}}%
\pgfpathlineto{\pgfqpoint{3.126693in}{0.409803in}}%
\pgfpathlineto{\pgfqpoint{3.127726in}{0.403873in}}%
\pgfpathclose%
\pgfusepath{fill}%
\end{pgfscope}%
\begin{pgfscope}%
\pgfpathrectangle{\pgfqpoint{0.100000in}{0.100000in}}{\pgfqpoint{3.608454in}{2.310000in}}%
\pgfusepath{clip}%
\pgfsetbuttcap%
\pgfsetmiterjoin%
\definecolor{currentfill}{rgb}{0.000000,0.521569,0.739216}%
\pgfsetfillcolor{currentfill}%
\pgfsetlinewidth{0.000000pt}%
\definecolor{currentstroke}{rgb}{0.000000,0.000000,0.000000}%
\pgfsetstrokecolor{currentstroke}%
\pgfsetstrokeopacity{0.000000}%
\pgfsetdash{}{0pt}%
\pgfpathmoveto{\pgfqpoint{2.621408in}{1.315375in}}%
\pgfpathlineto{\pgfqpoint{2.620401in}{1.325720in}}%
\pgfpathlineto{\pgfqpoint{2.610839in}{1.324868in}}%
\pgfpathlineto{\pgfqpoint{2.608516in}{1.322156in}}%
\pgfpathlineto{\pgfqpoint{2.594128in}{1.324280in}}%
\pgfpathlineto{\pgfqpoint{2.585800in}{1.324135in}}%
\pgfpathlineto{\pgfqpoint{2.582510in}{1.335565in}}%
\pgfpathlineto{\pgfqpoint{2.584141in}{1.345785in}}%
\pgfpathlineto{\pgfqpoint{2.590013in}{1.348185in}}%
\pgfpathlineto{\pgfqpoint{2.590232in}{1.353266in}}%
\pgfpathlineto{\pgfqpoint{2.583434in}{1.352723in}}%
\pgfpathlineto{\pgfqpoint{2.581657in}{1.373507in}}%
\pgfpathlineto{\pgfqpoint{2.593047in}{1.374201in}}%
\pgfpathlineto{\pgfqpoint{2.591897in}{1.387949in}}%
\pgfpathlineto{\pgfqpoint{2.598672in}{1.388501in}}%
\pgfpathlineto{\pgfqpoint{2.597905in}{1.398852in}}%
\pgfpathlineto{\pgfqpoint{2.613308in}{1.399991in}}%
\pgfpathlineto{\pgfqpoint{2.616632in}{1.400311in}}%
\pgfpathlineto{\pgfqpoint{2.617652in}{1.390119in}}%
\pgfpathlineto{\pgfqpoint{2.632842in}{1.391462in}}%
\pgfpathlineto{\pgfqpoint{2.635715in}{1.368602in}}%
\pgfpathlineto{\pgfqpoint{2.638885in}{1.368939in}}%
\pgfpathlineto{\pgfqpoint{2.641627in}{1.364614in}}%
\pgfpathlineto{\pgfqpoint{2.643662in}{1.346766in}}%
\pgfpathlineto{\pgfqpoint{2.641760in}{1.344149in}}%
\pgfpathlineto{\pgfqpoint{2.644392in}{1.317477in}}%
\pgfpathlineto{\pgfqpoint{2.621408in}{1.315375in}}%
\pgfpathclose%
\pgfusepath{fill}%
\end{pgfscope}%
\begin{pgfscope}%
\pgfpathrectangle{\pgfqpoint{0.100000in}{0.100000in}}{\pgfqpoint{3.608454in}{2.310000in}}%
\pgfusepath{clip}%
\pgfsetbuttcap%
\pgfsetmiterjoin%
\definecolor{currentfill}{rgb}{0.000000,0.905882,0.547059}%
\pgfsetfillcolor{currentfill}%
\pgfsetlinewidth{0.000000pt}%
\definecolor{currentstroke}{rgb}{0.000000,0.000000,0.000000}%
\pgfsetstrokecolor{currentstroke}%
\pgfsetstrokeopacity{0.000000}%
\pgfsetdash{}{0pt}%
\pgfpathmoveto{\pgfqpoint{2.258858in}{0.847301in}}%
\pgfpathlineto{\pgfqpoint{2.267654in}{0.847225in}}%
\pgfpathlineto{\pgfqpoint{2.272892in}{0.840599in}}%
\pgfpathlineto{\pgfqpoint{2.279806in}{0.840743in}}%
\pgfpathlineto{\pgfqpoint{2.280375in}{0.819890in}}%
\pgfpathlineto{\pgfqpoint{2.287535in}{0.813168in}}%
\pgfpathlineto{\pgfqpoint{2.288408in}{0.781863in}}%
\pgfpathlineto{\pgfqpoint{2.285332in}{0.771355in}}%
\pgfpathlineto{\pgfqpoint{2.267964in}{0.770915in}}%
\pgfpathlineto{\pgfqpoint{2.268095in}{0.763939in}}%
\pgfpathlineto{\pgfqpoint{2.244140in}{0.763307in}}%
\pgfpathlineto{\pgfqpoint{2.246031in}{0.774913in}}%
\pgfpathlineto{\pgfqpoint{2.249736in}{0.781810in}}%
\pgfpathlineto{\pgfqpoint{2.245714in}{0.798315in}}%
\pgfpathlineto{\pgfqpoint{2.254067in}{0.798428in}}%
\pgfpathlineto{\pgfqpoint{2.256664in}{0.805503in}}%
\pgfpathlineto{\pgfqpoint{2.255994in}{0.822878in}}%
\pgfpathlineto{\pgfqpoint{2.249143in}{0.822767in}}%
\pgfpathlineto{\pgfqpoint{2.248708in}{0.834367in}}%
\pgfpathlineto{\pgfqpoint{2.251967in}{0.847074in}}%
\pgfpathlineto{\pgfqpoint{2.258858in}{0.847301in}}%
\pgfpathclose%
\pgfusepath{fill}%
\end{pgfscope}%
\begin{pgfscope}%
\pgfpathrectangle{\pgfqpoint{0.100000in}{0.100000in}}{\pgfqpoint{3.608454in}{2.310000in}}%
\pgfusepath{clip}%
\pgfsetbuttcap%
\pgfsetmiterjoin%
\definecolor{currentfill}{rgb}{0.000000,0.603922,0.698039}%
\pgfsetfillcolor{currentfill}%
\pgfsetlinewidth{0.000000pt}%
\definecolor{currentstroke}{rgb}{0.000000,0.000000,0.000000}%
\pgfsetstrokecolor{currentstroke}%
\pgfsetstrokeopacity{0.000000}%
\pgfsetdash{}{0pt}%
\pgfpathmoveto{\pgfqpoint{1.291011in}{1.784895in}}%
\pgfpathlineto{\pgfqpoint{1.297605in}{1.774698in}}%
\pgfpathlineto{\pgfqpoint{1.296615in}{1.768471in}}%
\pgfpathlineto{\pgfqpoint{1.310078in}{1.766325in}}%
\pgfpathlineto{\pgfqpoint{1.323149in}{1.760841in}}%
\pgfpathlineto{\pgfqpoint{1.322608in}{1.757434in}}%
\pgfpathlineto{\pgfqpoint{1.342072in}{1.748707in}}%
\pgfpathlineto{\pgfqpoint{1.352534in}{1.748041in}}%
\pgfpathlineto{\pgfqpoint{1.384511in}{1.743611in}}%
\pgfpathlineto{\pgfqpoint{1.388262in}{1.745418in}}%
\pgfpathlineto{\pgfqpoint{1.388966in}{1.743023in}}%
\pgfpathlineto{\pgfqpoint{1.386059in}{1.715853in}}%
\pgfpathlineto{\pgfqpoint{1.382345in}{1.688646in}}%
\pgfpathlineto{\pgfqpoint{1.379922in}{1.688990in}}%
\pgfpathlineto{\pgfqpoint{1.376226in}{1.661744in}}%
\pgfpathlineto{\pgfqpoint{1.377332in}{1.661590in}}%
\pgfpathlineto{\pgfqpoint{1.375467in}{1.648053in}}%
\pgfpathlineto{\pgfqpoint{1.347746in}{1.651996in}}%
\pgfpathlineto{\pgfqpoint{1.287325in}{1.661183in}}%
\pgfpathlineto{\pgfqpoint{1.287915in}{1.674697in}}%
\pgfpathlineto{\pgfqpoint{1.290897in}{1.695072in}}%
\pgfpathlineto{\pgfqpoint{1.282465in}{1.703069in}}%
\pgfpathlineto{\pgfqpoint{1.275945in}{1.717107in}}%
\pgfpathlineto{\pgfqpoint{1.270191in}{1.723778in}}%
\pgfpathlineto{\pgfqpoint{1.264849in}{1.733953in}}%
\pgfpathlineto{\pgfqpoint{1.262997in}{1.742086in}}%
\pgfpathlineto{\pgfqpoint{1.263755in}{1.753309in}}%
\pgfpathlineto{\pgfqpoint{1.261563in}{1.761526in}}%
\pgfpathlineto{\pgfqpoint{1.244733in}{1.764439in}}%
\pgfpathlineto{\pgfqpoint{1.251746in}{1.806946in}}%
\pgfpathlineto{\pgfqpoint{1.254530in}{1.801740in}}%
\pgfpathlineto{\pgfqpoint{1.260477in}{1.800714in}}%
\pgfpathlineto{\pgfqpoint{1.264050in}{1.788274in}}%
\pgfpathlineto{\pgfqpoint{1.271394in}{1.792453in}}%
\pgfpathlineto{\pgfqpoint{1.273821in}{1.796453in}}%
\pgfpathlineto{\pgfqpoint{1.279919in}{1.798565in}}%
\pgfpathlineto{\pgfqpoint{1.285167in}{1.795237in}}%
\pgfpathlineto{\pgfqpoint{1.291011in}{1.784895in}}%
\pgfpathclose%
\pgfusepath{fill}%
\end{pgfscope}%
\begin{pgfscope}%
\pgfpathrectangle{\pgfqpoint{0.100000in}{0.100000in}}{\pgfqpoint{3.608454in}{2.310000in}}%
\pgfusepath{clip}%
\pgfsetbuttcap%
\pgfsetmiterjoin%
\definecolor{currentfill}{rgb}{0.000000,0.588235,0.705882}%
\pgfsetfillcolor{currentfill}%
\pgfsetlinewidth{0.000000pt}%
\definecolor{currentstroke}{rgb}{0.000000,0.000000,0.000000}%
\pgfsetstrokecolor{currentstroke}%
\pgfsetstrokeopacity{0.000000}%
\pgfsetdash{}{0pt}%
\pgfpathmoveto{\pgfqpoint{2.417621in}{1.861430in}}%
\pgfpathlineto{\pgfqpoint{2.417000in}{1.868413in}}%
\pgfpathlineto{\pgfqpoint{2.382793in}{1.866283in}}%
\pgfpathlineto{\pgfqpoint{2.380907in}{1.900334in}}%
\pgfpathlineto{\pgfqpoint{2.387268in}{1.900687in}}%
\pgfpathlineto{\pgfqpoint{2.385839in}{1.925963in}}%
\pgfpathlineto{\pgfqpoint{2.432701in}{1.915926in}}%
\pgfpathlineto{\pgfqpoint{2.438472in}{1.913015in}}%
\pgfpathlineto{\pgfqpoint{2.456551in}{1.904950in}}%
\pgfpathlineto{\pgfqpoint{2.457894in}{1.884450in}}%
\pgfpathlineto{\pgfqpoint{2.471787in}{1.885405in}}%
\pgfpathlineto{\pgfqpoint{2.473620in}{1.857983in}}%
\pgfpathlineto{\pgfqpoint{2.459694in}{1.857127in}}%
\pgfpathlineto{\pgfqpoint{2.445861in}{1.856178in}}%
\pgfpathlineto{\pgfqpoint{2.445567in}{1.863054in}}%
\pgfpathlineto{\pgfqpoint{2.417621in}{1.861430in}}%
\pgfpathclose%
\pgfusepath{fill}%
\end{pgfscope}%
\begin{pgfscope}%
\pgfpathrectangle{\pgfqpoint{0.100000in}{0.100000in}}{\pgfqpoint{3.608454in}{2.310000in}}%
\pgfusepath{clip}%
\pgfsetbuttcap%
\pgfsetmiterjoin%
\definecolor{currentfill}{rgb}{0.000000,0.701961,0.649020}%
\pgfsetfillcolor{currentfill}%
\pgfsetlinewidth{0.000000pt}%
\definecolor{currentstroke}{rgb}{0.000000,0.000000,0.000000}%
\pgfsetstrokecolor{currentstroke}%
\pgfsetstrokeopacity{0.000000}%
\pgfsetdash{}{0pt}%
\pgfpathmoveto{\pgfqpoint{1.126068in}{1.701146in}}%
\pgfpathlineto{\pgfqpoint{1.122487in}{1.688955in}}%
\pgfpathlineto{\pgfqpoint{1.117899in}{1.692151in}}%
\pgfpathlineto{\pgfqpoint{1.109727in}{1.693683in}}%
\pgfpathlineto{\pgfqpoint{1.101946in}{1.704218in}}%
\pgfpathlineto{\pgfqpoint{1.098147in}{1.713450in}}%
\pgfpathlineto{\pgfqpoint{1.091640in}{1.714726in}}%
\pgfpathlineto{\pgfqpoint{1.090342in}{1.707956in}}%
\pgfpathlineto{\pgfqpoint{1.081382in}{1.709710in}}%
\pgfpathlineto{\pgfqpoint{1.080051in}{1.702943in}}%
\pgfpathlineto{\pgfqpoint{1.060120in}{1.706792in}}%
\pgfpathlineto{\pgfqpoint{1.064066in}{1.727041in}}%
\pgfpathlineto{\pgfqpoint{1.054206in}{1.729137in}}%
\pgfpathlineto{\pgfqpoint{1.051274in}{1.732636in}}%
\pgfpathlineto{\pgfqpoint{1.055372in}{1.735866in}}%
\pgfpathlineto{\pgfqpoint{1.053439in}{1.743275in}}%
\pgfpathlineto{\pgfqpoint{1.058573in}{1.770425in}}%
\pgfpathlineto{\pgfqpoint{1.071945in}{1.767715in}}%
\pgfpathlineto{\pgfqpoint{1.074676in}{1.781213in}}%
\pgfpathlineto{\pgfqpoint{1.054623in}{1.785268in}}%
\pgfpathlineto{\pgfqpoint{1.055937in}{1.791680in}}%
\pgfpathlineto{\pgfqpoint{1.040506in}{1.794892in}}%
\pgfpathlineto{\pgfqpoint{1.041894in}{1.801489in}}%
\pgfpathlineto{\pgfqpoint{1.038897in}{1.813232in}}%
\pgfpathlineto{\pgfqpoint{1.034613in}{1.812423in}}%
\pgfpathlineto{\pgfqpoint{1.037674in}{1.816193in}}%
\pgfpathlineto{\pgfqpoint{1.049281in}{1.819620in}}%
\pgfpathlineto{\pgfqpoint{1.057180in}{1.823892in}}%
\pgfpathlineto{\pgfqpoint{1.061051in}{1.840480in}}%
\pgfpathlineto{\pgfqpoint{1.063909in}{1.846064in}}%
\pgfpathlineto{\pgfqpoint{1.066665in}{1.859594in}}%
\pgfpathlineto{\pgfqpoint{1.100225in}{1.852591in}}%
\pgfpathlineto{\pgfqpoint{1.102494in}{1.863989in}}%
\pgfpathlineto{\pgfqpoint{1.105971in}{1.872044in}}%
\pgfpathlineto{\pgfqpoint{1.123023in}{1.868301in}}%
\pgfpathlineto{\pgfqpoint{1.127333in}{1.864844in}}%
\pgfpathlineto{\pgfqpoint{1.130114in}{1.870874in}}%
\pgfpathlineto{\pgfqpoint{1.134588in}{1.873151in}}%
\pgfpathlineto{\pgfqpoint{1.144630in}{1.867633in}}%
\pgfpathlineto{\pgfqpoint{1.157878in}{1.868529in}}%
\pgfpathlineto{\pgfqpoint{1.159714in}{1.863660in}}%
\pgfpathlineto{\pgfqpoint{1.166980in}{1.866417in}}%
\pgfpathlineto{\pgfqpoint{1.177604in}{1.866287in}}%
\pgfpathlineto{\pgfqpoint{1.181983in}{1.875682in}}%
\pgfpathlineto{\pgfqpoint{1.187877in}{1.878039in}}%
\pgfpathlineto{\pgfqpoint{1.191695in}{1.873351in}}%
\pgfpathlineto{\pgfqpoint{1.198018in}{1.855881in}}%
\pgfpathlineto{\pgfqpoint{1.202252in}{1.853038in}}%
\pgfpathlineto{\pgfqpoint{1.195518in}{1.814680in}}%
\pgfpathlineto{\pgfqpoint{1.186131in}{1.812373in}}%
\pgfpathlineto{\pgfqpoint{1.174897in}{1.813546in}}%
\pgfpathlineto{\pgfqpoint{1.170551in}{1.790061in}}%
\pgfpathlineto{\pgfqpoint{1.179359in}{1.788436in}}%
\pgfpathlineto{\pgfqpoint{1.180737in}{1.781578in}}%
\pgfpathlineto{\pgfqpoint{1.188888in}{1.777013in}}%
\pgfpathlineto{\pgfqpoint{1.186356in}{1.762535in}}%
\pgfpathlineto{\pgfqpoint{1.182268in}{1.739464in}}%
\pgfpathlineto{\pgfqpoint{1.124259in}{1.750336in}}%
\pgfpathlineto{\pgfqpoint{1.117477in}{1.741580in}}%
\pgfpathlineto{\pgfqpoint{1.116145in}{1.733195in}}%
\pgfpathlineto{\pgfqpoint{1.117055in}{1.725647in}}%
\pgfpathlineto{\pgfqpoint{1.125111in}{1.725638in}}%
\pgfpathlineto{\pgfqpoint{1.126636in}{1.717223in}}%
\pgfpathlineto{\pgfqpoint{1.126068in}{1.701146in}}%
\pgfpathclose%
\pgfusepath{fill}%
\end{pgfscope}%
\begin{pgfscope}%
\pgfpathrectangle{\pgfqpoint{0.100000in}{0.100000in}}{\pgfqpoint{3.608454in}{2.310000in}}%
\pgfusepath{clip}%
\pgfsetbuttcap%
\pgfsetmiterjoin%
\definecolor{currentfill}{rgb}{0.000000,0.701961,0.649020}%
\pgfsetfillcolor{currentfill}%
\pgfsetlinewidth{0.000000pt}%
\definecolor{currentstroke}{rgb}{0.000000,0.000000,0.000000}%
\pgfsetstrokecolor{currentstroke}%
\pgfsetstrokeopacity{0.000000}%
\pgfsetdash{}{0pt}%
\pgfpathmoveto{\pgfqpoint{0.718104in}{2.222638in}}%
\pgfpathlineto{\pgfqpoint{0.722390in}{2.214825in}}%
\pgfpathlineto{\pgfqpoint{0.724213in}{2.207197in}}%
\pgfpathlineto{\pgfqpoint{0.733242in}{2.201231in}}%
\pgfpathlineto{\pgfqpoint{0.735660in}{2.196356in}}%
\pgfpathlineto{\pgfqpoint{0.742925in}{2.192347in}}%
\pgfpathlineto{\pgfqpoint{0.748201in}{2.185881in}}%
\pgfpathlineto{\pgfqpoint{0.763587in}{2.181602in}}%
\pgfpathlineto{\pgfqpoint{0.767233in}{2.177075in}}%
\pgfpathlineto{\pgfqpoint{0.762473in}{2.166902in}}%
\pgfpathlineto{\pgfqpoint{0.763393in}{2.155197in}}%
\pgfpathlineto{\pgfqpoint{0.762232in}{2.143437in}}%
\pgfpathlineto{\pgfqpoint{0.758684in}{2.134138in}}%
\pgfpathlineto{\pgfqpoint{0.761418in}{2.129669in}}%
\pgfpathlineto{\pgfqpoint{0.749167in}{2.084439in}}%
\pgfpathlineto{\pgfqpoint{0.719253in}{2.092888in}}%
\pgfpathlineto{\pgfqpoint{0.661241in}{2.110265in}}%
\pgfpathlineto{\pgfqpoint{0.669147in}{2.136525in}}%
\pgfpathlineto{\pgfqpoint{0.675514in}{2.134601in}}%
\pgfpathlineto{\pgfqpoint{0.675711in}{2.144078in}}%
\pgfpathlineto{\pgfqpoint{0.682016in}{2.153295in}}%
\pgfpathlineto{\pgfqpoint{0.682663in}{2.166183in}}%
\pgfpathlineto{\pgfqpoint{0.680298in}{2.173569in}}%
\pgfpathlineto{\pgfqpoint{0.697139in}{2.190297in}}%
\pgfpathlineto{\pgfqpoint{0.699138in}{2.201963in}}%
\pgfpathlineto{\pgfqpoint{0.694658in}{2.204301in}}%
\pgfpathlineto{\pgfqpoint{0.694418in}{2.209381in}}%
\pgfpathlineto{\pgfqpoint{0.701083in}{2.215781in}}%
\pgfpathlineto{\pgfqpoint{0.712161in}{2.222097in}}%
\pgfpathlineto{\pgfqpoint{0.718104in}{2.222638in}}%
\pgfpathclose%
\pgfusepath{fill}%
\end{pgfscope}%
\begin{pgfscope}%
\pgfpathrectangle{\pgfqpoint{0.100000in}{0.100000in}}{\pgfqpoint{3.608454in}{2.310000in}}%
\pgfusepath{clip}%
\pgfsetbuttcap%
\pgfsetmiterjoin%
\definecolor{currentfill}{rgb}{0.000000,0.776471,0.611765}%
\pgfsetfillcolor{currentfill}%
\pgfsetlinewidth{0.000000pt}%
\definecolor{currentstroke}{rgb}{0.000000,0.000000,0.000000}%
\pgfsetstrokecolor{currentstroke}%
\pgfsetstrokeopacity{0.000000}%
\pgfsetdash{}{0pt}%
\pgfpathmoveto{\pgfqpoint{0.538914in}{1.596597in}}%
\pgfpathlineto{\pgfqpoint{0.532464in}{1.597564in}}%
\pgfpathlineto{\pgfqpoint{0.520214in}{1.588877in}}%
\pgfpathlineto{\pgfqpoint{0.514461in}{1.575916in}}%
\pgfpathlineto{\pgfqpoint{0.507112in}{1.573994in}}%
\pgfpathlineto{\pgfqpoint{0.502169in}{1.569148in}}%
\pgfpathlineto{\pgfqpoint{0.498193in}{1.555922in}}%
\pgfpathlineto{\pgfqpoint{0.489967in}{1.557121in}}%
\pgfpathlineto{\pgfqpoint{0.484414in}{1.564406in}}%
\pgfpathlineto{\pgfqpoint{0.485256in}{1.568714in}}%
\pgfpathlineto{\pgfqpoint{0.480562in}{1.575547in}}%
\pgfpathlineto{\pgfqpoint{0.450911in}{1.584637in}}%
\pgfpathlineto{\pgfqpoint{0.448894in}{1.592474in}}%
\pgfpathlineto{\pgfqpoint{0.445039in}{1.597163in}}%
\pgfpathlineto{\pgfqpoint{0.447344in}{1.609339in}}%
\pgfpathlineto{\pgfqpoint{0.440654in}{1.612562in}}%
\pgfpathlineto{\pgfqpoint{0.435002in}{1.619433in}}%
\pgfpathlineto{\pgfqpoint{0.440384in}{1.631531in}}%
\pgfpathlineto{\pgfqpoint{0.443292in}{1.641381in}}%
\pgfpathlineto{\pgfqpoint{0.434583in}{1.644086in}}%
\pgfpathlineto{\pgfqpoint{0.437233in}{1.653925in}}%
\pgfpathlineto{\pgfqpoint{0.436761in}{1.661477in}}%
\pgfpathlineto{\pgfqpoint{0.488413in}{1.645315in}}%
\pgfpathlineto{\pgfqpoint{0.490498in}{1.651799in}}%
\pgfpathlineto{\pgfqpoint{0.504486in}{1.647574in}}%
\pgfpathlineto{\pgfqpoint{0.512624in}{1.653371in}}%
\pgfpathlineto{\pgfqpoint{0.515901in}{1.652222in}}%
\pgfpathlineto{\pgfqpoint{0.522050in}{1.660114in}}%
\pgfpathlineto{\pgfqpoint{0.531886in}{1.661402in}}%
\pgfpathlineto{\pgfqpoint{0.534394in}{1.657082in}}%
\pgfpathlineto{\pgfqpoint{0.529796in}{1.650793in}}%
\pgfpathlineto{\pgfqpoint{0.526964in}{1.639730in}}%
\pgfpathlineto{\pgfqpoint{0.529553in}{1.637642in}}%
\pgfpathlineto{\pgfqpoint{0.540950in}{1.606159in}}%
\pgfpathlineto{\pgfqpoint{0.538914in}{1.596597in}}%
\pgfpathclose%
\pgfusepath{fill}%
\end{pgfscope}%
\begin{pgfscope}%
\pgfpathrectangle{\pgfqpoint{0.100000in}{0.100000in}}{\pgfqpoint{3.608454in}{2.310000in}}%
\pgfusepath{clip}%
\pgfsetbuttcap%
\pgfsetmiterjoin%
\definecolor{currentfill}{rgb}{0.000000,0.596078,0.701961}%
\pgfsetfillcolor{currentfill}%
\pgfsetlinewidth{0.000000pt}%
\definecolor{currentstroke}{rgb}{0.000000,0.000000,0.000000}%
\pgfsetstrokecolor{currentstroke}%
\pgfsetstrokeopacity{0.000000}%
\pgfsetdash{}{0pt}%
\pgfpathmoveto{\pgfqpoint{2.075326in}{1.295024in}}%
\pgfpathlineto{\pgfqpoint{2.074465in}{1.263717in}}%
\pgfpathlineto{\pgfqpoint{2.047067in}{1.264108in}}%
\pgfpathlineto{\pgfqpoint{2.047190in}{1.274443in}}%
\pgfpathlineto{\pgfqpoint{2.022450in}{1.274884in}}%
\pgfpathlineto{\pgfqpoint{2.022297in}{1.268017in}}%
\pgfpathlineto{\pgfqpoint{2.012052in}{1.268224in}}%
\pgfpathlineto{\pgfqpoint{1.992256in}{1.268523in}}%
\pgfpathlineto{\pgfqpoint{1.992866in}{1.289314in}}%
\pgfpathlineto{\pgfqpoint{1.994270in}{1.303029in}}%
\pgfpathlineto{\pgfqpoint{2.023077in}{1.302462in}}%
\pgfpathlineto{\pgfqpoint{2.023407in}{1.319659in}}%
\pgfpathlineto{\pgfqpoint{2.048478in}{1.319317in}}%
\pgfpathlineto{\pgfqpoint{2.047953in}{1.295247in}}%
\pgfpathlineto{\pgfqpoint{2.075326in}{1.295024in}}%
\pgfpathclose%
\pgfusepath{fill}%
\end{pgfscope}%
\begin{pgfscope}%
\pgfpathrectangle{\pgfqpoint{0.100000in}{0.100000in}}{\pgfqpoint{3.608454in}{2.310000in}}%
\pgfusepath{clip}%
\pgfsetbuttcap%
\pgfsetmiterjoin%
\definecolor{currentfill}{rgb}{0.000000,0.462745,0.768627}%
\pgfsetfillcolor{currentfill}%
\pgfsetlinewidth{0.000000pt}%
\definecolor{currentstroke}{rgb}{0.000000,0.000000,0.000000}%
\pgfsetstrokecolor{currentstroke}%
\pgfsetstrokeopacity{0.000000}%
\pgfsetdash{}{0pt}%
\pgfpathmoveto{\pgfqpoint{1.303565in}{2.014897in}}%
\pgfpathlineto{\pgfqpoint{1.299591in}{2.015508in}}%
\pgfpathlineto{\pgfqpoint{1.292244in}{2.021526in}}%
\pgfpathlineto{\pgfqpoint{1.264971in}{2.020151in}}%
\pgfpathlineto{\pgfqpoint{1.259993in}{2.025470in}}%
\pgfpathlineto{\pgfqpoint{1.251605in}{2.040078in}}%
\pgfpathlineto{\pgfqpoint{1.244581in}{2.048177in}}%
\pgfpathlineto{\pgfqpoint{1.231866in}{2.052930in}}%
\pgfpathlineto{\pgfqpoint{1.211322in}{2.055765in}}%
\pgfpathlineto{\pgfqpoint{1.210055in}{2.048940in}}%
\pgfpathlineto{\pgfqpoint{1.203686in}{2.050112in}}%
\pgfpathlineto{\pgfqpoint{1.196622in}{2.051472in}}%
\pgfpathlineto{\pgfqpoint{1.199816in}{2.068342in}}%
\pgfpathlineto{\pgfqpoint{1.192015in}{2.070842in}}%
\pgfpathlineto{\pgfqpoint{1.187119in}{2.075916in}}%
\pgfpathlineto{\pgfqpoint{1.191810in}{2.101317in}}%
\pgfpathlineto{\pgfqpoint{1.179403in}{2.102869in}}%
\pgfpathlineto{\pgfqpoint{1.173331in}{2.107985in}}%
\pgfpathlineto{\pgfqpoint{1.173281in}{2.111946in}}%
\pgfpathlineto{\pgfqpoint{1.165476in}{2.115944in}}%
\pgfpathlineto{\pgfqpoint{1.152365in}{2.116451in}}%
\pgfpathlineto{\pgfqpoint{1.148672in}{2.124295in}}%
\pgfpathlineto{\pgfqpoint{1.150853in}{2.134585in}}%
\pgfpathlineto{\pgfqpoint{1.149026in}{2.145360in}}%
\pgfpathlineto{\pgfqpoint{1.154266in}{2.150464in}}%
\pgfpathlineto{\pgfqpoint{1.150149in}{2.159708in}}%
\pgfpathlineto{\pgfqpoint{1.194140in}{2.150958in}}%
\pgfpathlineto{\pgfqpoint{1.200330in}{2.146187in}}%
\pgfpathlineto{\pgfqpoint{1.202148in}{2.137522in}}%
\pgfpathlineto{\pgfqpoint{1.232482in}{2.132070in}}%
\pgfpathlineto{\pgfqpoint{1.234540in}{2.143503in}}%
\pgfpathlineto{\pgfqpoint{1.265169in}{2.138097in}}%
\pgfpathlineto{\pgfqpoint{1.266372in}{2.144919in}}%
\pgfpathlineto{\pgfqpoint{1.276965in}{2.143064in}}%
\pgfpathlineto{\pgfqpoint{1.278167in}{2.149917in}}%
\pgfpathlineto{\pgfqpoint{1.319123in}{2.142944in}}%
\pgfpathlineto{\pgfqpoint{1.318309in}{2.136230in}}%
\pgfpathlineto{\pgfqpoint{1.324278in}{2.135234in}}%
\pgfpathlineto{\pgfqpoint{1.323124in}{2.128216in}}%
\pgfpathlineto{\pgfqpoint{1.333325in}{2.126759in}}%
\pgfpathlineto{\pgfqpoint{1.328054in}{2.095490in}}%
\pgfpathlineto{\pgfqpoint{1.312471in}{2.096193in}}%
\pgfpathlineto{\pgfqpoint{1.309109in}{2.082641in}}%
\pgfpathlineto{\pgfqpoint{1.293100in}{2.076361in}}%
\pgfpathlineto{\pgfqpoint{1.290585in}{2.063837in}}%
\pgfpathlineto{\pgfqpoint{1.296843in}{2.059295in}}%
\pgfpathlineto{\pgfqpoint{1.303684in}{2.058169in}}%
\pgfpathlineto{\pgfqpoint{1.309306in}{2.053731in}}%
\pgfpathlineto{\pgfqpoint{1.305651in}{2.030921in}}%
\pgfpathlineto{\pgfqpoint{1.303017in}{2.028995in}}%
\pgfpathlineto{\pgfqpoint{1.303565in}{2.014897in}}%
\pgfpathclose%
\pgfusepath{fill}%
\end{pgfscope}%
\begin{pgfscope}%
\pgfpathrectangle{\pgfqpoint{0.100000in}{0.100000in}}{\pgfqpoint{3.608454in}{2.310000in}}%
\pgfusepath{clip}%
\pgfsetbuttcap%
\pgfsetmiterjoin%
\definecolor{currentfill}{rgb}{0.000000,0.462745,0.768627}%
\pgfsetfillcolor{currentfill}%
\pgfsetlinewidth{0.000000pt}%
\definecolor{currentstroke}{rgb}{0.000000,0.000000,0.000000}%
\pgfsetstrokecolor{currentstroke}%
\pgfsetstrokeopacity{0.000000}%
\pgfsetdash{}{0pt}%
\pgfpathmoveto{\pgfqpoint{1.807069in}{2.050968in}}%
\pgfpathlineto{\pgfqpoint{1.827828in}{2.049744in}}%
\pgfpathlineto{\pgfqpoint{1.826539in}{2.063794in}}%
\pgfpathlineto{\pgfqpoint{1.828130in}{2.091731in}}%
\pgfpathlineto{\pgfqpoint{1.826839in}{2.105738in}}%
\pgfpathlineto{\pgfqpoint{1.833772in}{2.105376in}}%
\pgfpathlineto{\pgfqpoint{1.868375in}{2.103508in}}%
\pgfpathlineto{\pgfqpoint{1.867703in}{2.089585in}}%
\pgfpathlineto{\pgfqpoint{1.848899in}{2.090580in}}%
\pgfpathlineto{\pgfqpoint{1.847326in}{2.062601in}}%
\pgfpathlineto{\pgfqpoint{1.848523in}{2.048601in}}%
\pgfpathlineto{\pgfqpoint{1.876114in}{2.047161in}}%
\pgfpathlineto{\pgfqpoint{1.877148in}{2.033168in}}%
\pgfpathlineto{\pgfqpoint{1.875835in}{2.005576in}}%
\pgfpathlineto{\pgfqpoint{1.864232in}{2.006135in}}%
\pgfpathlineto{\pgfqpoint{1.830021in}{2.007934in}}%
\pgfpathlineto{\pgfqpoint{1.799746in}{2.009759in}}%
\pgfpathlineto{\pgfqpoint{1.801529in}{2.037408in}}%
\pgfpathlineto{\pgfqpoint{1.806219in}{2.037109in}}%
\pgfpathlineto{\pgfqpoint{1.807069in}{2.050968in}}%
\pgfpathclose%
\pgfusepath{fill}%
\end{pgfscope}%
\begin{pgfscope}%
\pgfpathrectangle{\pgfqpoint{0.100000in}{0.100000in}}{\pgfqpoint{3.608454in}{2.310000in}}%
\pgfusepath{clip}%
\pgfsetbuttcap%
\pgfsetmiterjoin%
\definecolor{currentfill}{rgb}{0.000000,0.580392,0.709804}%
\pgfsetfillcolor{currentfill}%
\pgfsetlinewidth{0.000000pt}%
\definecolor{currentstroke}{rgb}{0.000000,0.000000,0.000000}%
\pgfsetstrokecolor{currentstroke}%
\pgfsetstrokeopacity{0.000000}%
\pgfsetdash{}{0pt}%
\pgfpathmoveto{\pgfqpoint{2.409257in}{1.435587in}}%
\pgfpathlineto{\pgfqpoint{2.397384in}{1.430530in}}%
\pgfpathlineto{\pgfqpoint{2.391369in}{1.431061in}}%
\pgfpathlineto{\pgfqpoint{2.389932in}{1.435148in}}%
\pgfpathlineto{\pgfqpoint{2.396575in}{1.440884in}}%
\pgfpathlineto{\pgfqpoint{2.401121in}{1.445010in}}%
\pgfpathlineto{\pgfqpoint{2.405497in}{1.456494in}}%
\pgfpathlineto{\pgfqpoint{2.412968in}{1.462469in}}%
\pgfpathlineto{\pgfqpoint{2.414833in}{1.467947in}}%
\pgfpathlineto{\pgfqpoint{2.414274in}{1.476742in}}%
\pgfpathlineto{\pgfqpoint{2.407342in}{1.476448in}}%
\pgfpathlineto{\pgfqpoint{2.406029in}{1.504012in}}%
\pgfpathlineto{\pgfqpoint{2.426710in}{1.505121in}}%
\pgfpathlineto{\pgfqpoint{2.425489in}{1.525655in}}%
\pgfpathlineto{\pgfqpoint{2.412483in}{1.524944in}}%
\pgfpathlineto{\pgfqpoint{2.410650in}{1.552521in}}%
\pgfpathlineto{\pgfqpoint{2.451750in}{1.555104in}}%
\pgfpathlineto{\pgfqpoint{2.451488in}{1.558550in}}%
\pgfpathlineto{\pgfqpoint{2.484797in}{1.561007in}}%
\pgfpathlineto{\pgfqpoint{2.488674in}{1.519766in}}%
\pgfpathlineto{\pgfqpoint{2.468185in}{1.518191in}}%
\pgfpathlineto{\pgfqpoint{2.470141in}{1.490384in}}%
\pgfpathlineto{\pgfqpoint{2.467299in}{1.483195in}}%
\pgfpathlineto{\pgfqpoint{2.458726in}{1.477228in}}%
\pgfpathlineto{\pgfqpoint{2.450616in}{1.476514in}}%
\pgfpathlineto{\pgfqpoint{2.452287in}{1.455302in}}%
\pgfpathlineto{\pgfqpoint{2.431914in}{1.453643in}}%
\pgfpathlineto{\pgfqpoint{2.432920in}{1.438056in}}%
\pgfpathlineto{\pgfqpoint{2.427687in}{1.440218in}}%
\pgfpathlineto{\pgfqpoint{2.418888in}{1.437396in}}%
\pgfpathlineto{\pgfqpoint{2.410568in}{1.438329in}}%
\pgfpathlineto{\pgfqpoint{2.409257in}{1.435587in}}%
\pgfpathclose%
\pgfusepath{fill}%
\end{pgfscope}%
\begin{pgfscope}%
\pgfpathrectangle{\pgfqpoint{0.100000in}{0.100000in}}{\pgfqpoint{3.608454in}{2.310000in}}%
\pgfusepath{clip}%
\pgfsetbuttcap%
\pgfsetmiterjoin%
\definecolor{currentfill}{rgb}{0.000000,0.560784,0.719608}%
\pgfsetfillcolor{currentfill}%
\pgfsetlinewidth{0.000000pt}%
\definecolor{currentstroke}{rgb}{0.000000,0.000000,0.000000}%
\pgfsetstrokecolor{currentstroke}%
\pgfsetstrokeopacity{0.000000}%
\pgfsetdash{}{0pt}%
\pgfpathmoveto{\pgfqpoint{1.843805in}{1.813366in}}%
\pgfpathlineto{\pgfqpoint{1.849510in}{1.813058in}}%
\pgfpathlineto{\pgfqpoint{1.898285in}{1.810809in}}%
\pgfpathlineto{\pgfqpoint{1.897415in}{1.789939in}}%
\pgfpathlineto{\pgfqpoint{1.896283in}{1.755231in}}%
\pgfpathlineto{\pgfqpoint{1.883458in}{1.755759in}}%
\pgfpathlineto{\pgfqpoint{1.846482in}{1.757163in}}%
\pgfpathlineto{\pgfqpoint{1.841646in}{1.759396in}}%
\pgfpathlineto{\pgfqpoint{1.842265in}{1.785638in}}%
\pgfpathlineto{\pgfqpoint{1.843805in}{1.813366in}}%
\pgfpathclose%
\pgfusepath{fill}%
\end{pgfscope}%
\begin{pgfscope}%
\pgfpathrectangle{\pgfqpoint{0.100000in}{0.100000in}}{\pgfqpoint{3.608454in}{2.310000in}}%
\pgfusepath{clip}%
\pgfsetbuttcap%
\pgfsetmiterjoin%
\definecolor{currentfill}{rgb}{0.000000,0.796078,0.601961}%
\pgfsetfillcolor{currentfill}%
\pgfsetlinewidth{0.000000pt}%
\definecolor{currentstroke}{rgb}{0.000000,0.000000,0.000000}%
\pgfsetstrokecolor{currentstroke}%
\pgfsetstrokeopacity{0.000000}%
\pgfsetdash{}{0pt}%
\pgfpathmoveto{\pgfqpoint{2.336789in}{0.709454in}}%
\pgfpathlineto{\pgfqpoint{2.341562in}{0.715016in}}%
\pgfpathlineto{\pgfqpoint{2.337261in}{0.720197in}}%
\pgfpathlineto{\pgfqpoint{2.341299in}{0.728481in}}%
\pgfpathlineto{\pgfqpoint{2.346226in}{0.727949in}}%
\pgfpathlineto{\pgfqpoint{2.347315in}{0.735413in}}%
\pgfpathlineto{\pgfqpoint{2.362838in}{0.740120in}}%
\pgfpathlineto{\pgfqpoint{2.371609in}{0.736436in}}%
\pgfpathlineto{\pgfqpoint{2.377876in}{0.738907in}}%
\pgfpathlineto{\pgfqpoint{2.404037in}{0.740243in}}%
\pgfpathlineto{\pgfqpoint{2.439716in}{0.742196in}}%
\pgfpathlineto{\pgfqpoint{2.444160in}{0.745936in}}%
\pgfpathlineto{\pgfqpoint{2.449720in}{0.745908in}}%
\pgfpathlineto{\pgfqpoint{2.458936in}{0.746389in}}%
\pgfpathlineto{\pgfqpoint{2.458660in}{0.749893in}}%
\pgfpathlineto{\pgfqpoint{2.470222in}{0.750540in}}%
\pgfpathlineto{\pgfqpoint{2.472257in}{0.716197in}}%
\pgfpathlineto{\pgfqpoint{2.467199in}{0.715885in}}%
\pgfpathlineto{\pgfqpoint{2.424922in}{0.713419in}}%
\pgfpathlineto{\pgfqpoint{2.392212in}{0.711750in}}%
\pgfpathlineto{\pgfqpoint{2.336789in}{0.709454in}}%
\pgfpathclose%
\pgfusepath{fill}%
\end{pgfscope}%
\begin{pgfscope}%
\pgfpathrectangle{\pgfqpoint{0.100000in}{0.100000in}}{\pgfqpoint{3.608454in}{2.310000in}}%
\pgfusepath{clip}%
\pgfsetbuttcap%
\pgfsetmiterjoin%
\definecolor{currentfill}{rgb}{0.000000,0.439216,0.780392}%
\pgfsetfillcolor{currentfill}%
\pgfsetlinewidth{0.000000pt}%
\definecolor{currentstroke}{rgb}{0.000000,0.000000,0.000000}%
\pgfsetstrokecolor{currentstroke}%
\pgfsetstrokeopacity{0.000000}%
\pgfsetdash{}{0pt}%
\pgfpathmoveto{\pgfqpoint{2.221382in}{1.649990in}}%
\pgfpathlineto{\pgfqpoint{2.220391in}{1.696881in}}%
\pgfpathlineto{\pgfqpoint{2.253472in}{1.697776in}}%
\pgfpathlineto{\pgfqpoint{2.294773in}{1.699116in}}%
\pgfpathlineto{\pgfqpoint{2.301639in}{1.699356in}}%
\pgfpathlineto{\pgfqpoint{2.303199in}{1.666191in}}%
\pgfpathlineto{\pgfqpoint{2.275677in}{1.665325in}}%
\pgfpathlineto{\pgfqpoint{2.276089in}{1.651404in}}%
\pgfpathlineto{\pgfqpoint{2.221382in}{1.649990in}}%
\pgfpathclose%
\pgfusepath{fill}%
\end{pgfscope}%
\begin{pgfscope}%
\pgfpathrectangle{\pgfqpoint{0.100000in}{0.100000in}}{\pgfqpoint{3.608454in}{2.310000in}}%
\pgfusepath{clip}%
\pgfsetbuttcap%
\pgfsetmiterjoin%
\definecolor{currentfill}{rgb}{0.000000,0.529412,0.735294}%
\pgfsetfillcolor{currentfill}%
\pgfsetlinewidth{0.000000pt}%
\definecolor{currentstroke}{rgb}{0.000000,0.000000,0.000000}%
\pgfsetstrokecolor{currentstroke}%
\pgfsetstrokeopacity{0.000000}%
\pgfsetdash{}{0pt}%
\pgfpathmoveto{\pgfqpoint{3.109751in}{1.709791in}}%
\pgfpathlineto{\pgfqpoint{3.102306in}{1.712116in}}%
\pgfpathlineto{\pgfqpoint{3.066495in}{1.705513in}}%
\pgfpathlineto{\pgfqpoint{3.064906in}{1.706470in}}%
\pgfpathlineto{\pgfqpoint{3.049340in}{1.697252in}}%
\pgfpathlineto{\pgfqpoint{3.042918in}{1.694827in}}%
\pgfpathlineto{\pgfqpoint{3.036017in}{1.700877in}}%
\pgfpathlineto{\pgfqpoint{3.032000in}{1.700788in}}%
\pgfpathlineto{\pgfqpoint{3.027192in}{1.702502in}}%
\pgfpathlineto{\pgfqpoint{3.030633in}{1.712923in}}%
\pgfpathlineto{\pgfqpoint{3.037383in}{1.717726in}}%
\pgfpathlineto{\pgfqpoint{3.040436in}{1.722640in}}%
\pgfpathlineto{\pgfqpoint{3.033554in}{1.734692in}}%
\pgfpathlineto{\pgfqpoint{3.028625in}{1.736179in}}%
\pgfpathlineto{\pgfqpoint{3.027760in}{1.742504in}}%
\pgfpathlineto{\pgfqpoint{3.022366in}{1.756709in}}%
\pgfpathlineto{\pgfqpoint{3.033985in}{1.763896in}}%
\pgfpathlineto{\pgfqpoint{3.044744in}{1.769031in}}%
\pgfpathlineto{\pgfqpoint{3.058321in}{1.773021in}}%
\pgfpathlineto{\pgfqpoint{3.081941in}{1.776783in}}%
\pgfpathlineto{\pgfqpoint{3.094629in}{1.776945in}}%
\pgfpathlineto{\pgfqpoint{3.106459in}{1.771415in}}%
\pgfpathlineto{\pgfqpoint{3.117744in}{1.776426in}}%
\pgfpathlineto{\pgfqpoint{3.141833in}{1.780817in}}%
\pgfpathlineto{\pgfqpoint{3.153298in}{1.788862in}}%
\pgfpathlineto{\pgfqpoint{3.158574in}{1.768645in}}%
\pgfpathlineto{\pgfqpoint{3.158361in}{1.759713in}}%
\pgfpathlineto{\pgfqpoint{3.162874in}{1.739825in}}%
\pgfpathlineto{\pgfqpoint{3.167852in}{1.733612in}}%
\pgfpathlineto{\pgfqpoint{3.173622in}{1.728849in}}%
\pgfpathlineto{\pgfqpoint{3.155225in}{1.724423in}}%
\pgfpathlineto{\pgfqpoint{3.156855in}{1.718354in}}%
\pgfpathlineto{\pgfqpoint{3.142173in}{1.718464in}}%
\pgfpathlineto{\pgfqpoint{3.141312in}{1.724493in}}%
\pgfpathlineto{\pgfqpoint{3.111728in}{1.718970in}}%
\pgfpathlineto{\pgfqpoint{3.109751in}{1.709791in}}%
\pgfpathclose%
\pgfusepath{fill}%
\end{pgfscope}%
\begin{pgfscope}%
\pgfpathrectangle{\pgfqpoint{0.100000in}{0.100000in}}{\pgfqpoint{3.608454in}{2.310000in}}%
\pgfusepath{clip}%
\pgfsetbuttcap%
\pgfsetmiterjoin%
\definecolor{currentfill}{rgb}{0.000000,0.466667,0.766667}%
\pgfsetfillcolor{currentfill}%
\pgfsetlinewidth{0.000000pt}%
\definecolor{currentstroke}{rgb}{0.000000,0.000000,0.000000}%
\pgfsetstrokecolor{currentstroke}%
\pgfsetstrokeopacity{0.000000}%
\pgfsetdash{}{0pt}%
\pgfpathmoveto{\pgfqpoint{2.989320in}{1.467586in}}%
\pgfpathlineto{\pgfqpoint{2.983438in}{1.466654in}}%
\pgfpathlineto{\pgfqpoint{2.977955in}{1.500863in}}%
\pgfpathlineto{\pgfqpoint{2.971951in}{1.538265in}}%
\pgfpathlineto{\pgfqpoint{2.969282in}{1.554858in}}%
\pgfpathlineto{\pgfqpoint{2.990516in}{1.558619in}}%
\pgfpathlineto{\pgfqpoint{2.988249in}{1.569893in}}%
\pgfpathlineto{\pgfqpoint{2.995768in}{1.584878in}}%
\pgfpathlineto{\pgfqpoint{3.013646in}{1.587974in}}%
\pgfpathlineto{\pgfqpoint{3.020349in}{1.574020in}}%
\pgfpathlineto{\pgfqpoint{3.028102in}{1.574866in}}%
\pgfpathlineto{\pgfqpoint{3.040035in}{1.579416in}}%
\pgfpathlineto{\pgfqpoint{3.043474in}{1.583306in}}%
\pgfpathlineto{\pgfqpoint{3.045363in}{1.572457in}}%
\pgfpathlineto{\pgfqpoint{3.069571in}{1.576329in}}%
\pgfpathlineto{\pgfqpoint{3.072318in}{1.562246in}}%
\pgfpathlineto{\pgfqpoint{3.068856in}{1.543568in}}%
\pgfpathlineto{\pgfqpoint{3.061504in}{1.519627in}}%
\pgfpathlineto{\pgfqpoint{3.054176in}{1.507940in}}%
\pgfpathlineto{\pgfqpoint{3.048216in}{1.491838in}}%
\pgfpathlineto{\pgfqpoint{3.051273in}{1.485944in}}%
\pgfpathlineto{\pgfqpoint{3.051009in}{1.477958in}}%
\pgfpathlineto{\pgfqpoint{3.045971in}{1.477046in}}%
\pgfpathlineto{\pgfqpoint{2.989320in}{1.467586in}}%
\pgfpathclose%
\pgfusepath{fill}%
\end{pgfscope}%
\begin{pgfscope}%
\pgfpathrectangle{\pgfqpoint{0.100000in}{0.100000in}}{\pgfqpoint{3.608454in}{2.310000in}}%
\pgfusepath{clip}%
\pgfsetbuttcap%
\pgfsetmiterjoin%
\definecolor{currentfill}{rgb}{0.000000,0.596078,0.701961}%
\pgfsetfillcolor{currentfill}%
\pgfsetlinewidth{0.000000pt}%
\definecolor{currentstroke}{rgb}{0.000000,0.000000,0.000000}%
\pgfsetstrokecolor{currentstroke}%
\pgfsetstrokeopacity{0.000000}%
\pgfsetdash{}{0pt}%
\pgfpathmoveto{\pgfqpoint{1.787654in}{0.626021in}}%
\pgfpathlineto{\pgfqpoint{1.781527in}{0.626427in}}%
\pgfpathlineto{\pgfqpoint{1.782309in}{0.639635in}}%
\pgfpathlineto{\pgfqpoint{1.777624in}{0.639888in}}%
\pgfpathlineto{\pgfqpoint{1.778772in}{0.657165in}}%
\pgfpathlineto{\pgfqpoint{1.809932in}{0.655294in}}%
\pgfpathlineto{\pgfqpoint{1.810600in}{0.672318in}}%
\pgfpathlineto{\pgfqpoint{1.833962in}{0.671147in}}%
\pgfpathlineto{\pgfqpoint{1.859593in}{0.670192in}}%
\pgfpathlineto{\pgfqpoint{1.858713in}{0.641352in}}%
\pgfpathlineto{\pgfqpoint{1.835751in}{0.642247in}}%
\pgfpathlineto{\pgfqpoint{1.834695in}{0.613722in}}%
\pgfpathlineto{\pgfqpoint{1.817099in}{0.623588in}}%
\pgfpathlineto{\pgfqpoint{1.787654in}{0.626021in}}%
\pgfpathclose%
\pgfusepath{fill}%
\end{pgfscope}%
\begin{pgfscope}%
\pgfpathrectangle{\pgfqpoint{0.100000in}{0.100000in}}{\pgfqpoint{3.608454in}{2.310000in}}%
\pgfusepath{clip}%
\pgfsetbuttcap%
\pgfsetmiterjoin%
\definecolor{currentfill}{rgb}{0.000000,0.619608,0.690196}%
\pgfsetfillcolor{currentfill}%
\pgfsetlinewidth{0.000000pt}%
\definecolor{currentstroke}{rgb}{0.000000,0.000000,0.000000}%
\pgfsetstrokecolor{currentstroke}%
\pgfsetstrokeopacity{0.000000}%
\pgfsetdash{}{0pt}%
\pgfpathmoveto{\pgfqpoint{2.870396in}{0.968762in}}%
\pgfpathlineto{\pgfqpoint{2.877953in}{0.974806in}}%
\pgfpathlineto{\pgfqpoint{2.889208in}{0.971223in}}%
\pgfpathlineto{\pgfqpoint{2.896071in}{0.971539in}}%
\pgfpathlineto{\pgfqpoint{2.896953in}{0.978821in}}%
\pgfpathlineto{\pgfqpoint{2.902891in}{0.990956in}}%
\pgfpathlineto{\pgfqpoint{2.908530in}{0.995600in}}%
\pgfpathlineto{\pgfqpoint{2.916400in}{0.998276in}}%
\pgfpathlineto{\pgfqpoint{2.922696in}{0.996327in}}%
\pgfpathlineto{\pgfqpoint{2.941645in}{0.986529in}}%
\pgfpathlineto{\pgfqpoint{2.948243in}{0.978295in}}%
\pgfpathlineto{\pgfqpoint{2.942345in}{0.978613in}}%
\pgfpathlineto{\pgfqpoint{2.932420in}{0.972207in}}%
\pgfpathlineto{\pgfqpoint{2.922909in}{0.970556in}}%
\pgfpathlineto{\pgfqpoint{2.918991in}{0.967320in}}%
\pgfpathlineto{\pgfqpoint{2.915431in}{0.959683in}}%
\pgfpathlineto{\pgfqpoint{2.910547in}{0.958327in}}%
\pgfpathlineto{\pgfqpoint{2.909363in}{0.953452in}}%
\pgfpathlineto{\pgfqpoint{2.898524in}{0.954072in}}%
\pgfpathlineto{\pgfqpoint{2.889760in}{0.943778in}}%
\pgfpathlineto{\pgfqpoint{2.880855in}{0.952903in}}%
\pgfpathlineto{\pgfqpoint{2.870396in}{0.968762in}}%
\pgfpathclose%
\pgfusepath{fill}%
\end{pgfscope}%
\begin{pgfscope}%
\pgfpathrectangle{\pgfqpoint{0.100000in}{0.100000in}}{\pgfqpoint{3.608454in}{2.310000in}}%
\pgfusepath{clip}%
\pgfsetbuttcap%
\pgfsetmiterjoin%
\definecolor{currentfill}{rgb}{0.000000,0.627451,0.686275}%
\pgfsetfillcolor{currentfill}%
\pgfsetlinewidth{0.000000pt}%
\definecolor{currentstroke}{rgb}{0.000000,0.000000,0.000000}%
\pgfsetstrokecolor{currentstroke}%
\pgfsetstrokeopacity{0.000000}%
\pgfsetdash{}{0pt}%
\pgfpathmoveto{\pgfqpoint{2.205934in}{1.422201in}}%
\pgfpathlineto{\pgfqpoint{2.206018in}{1.417015in}}%
\pgfpathlineto{\pgfqpoint{2.181819in}{1.416001in}}%
\pgfpathlineto{\pgfqpoint{2.181491in}{1.429705in}}%
\pgfpathlineto{\pgfqpoint{2.180394in}{1.464846in}}%
\pgfpathlineto{\pgfqpoint{2.193447in}{1.465250in}}%
\pgfpathlineto{\pgfqpoint{2.204405in}{1.465435in}}%
\pgfpathlineto{\pgfqpoint{2.205934in}{1.422201in}}%
\pgfpathclose%
\pgfusepath{fill}%
\end{pgfscope}%
\begin{pgfscope}%
\pgfpathrectangle{\pgfqpoint{0.100000in}{0.100000in}}{\pgfqpoint{3.608454in}{2.310000in}}%
\pgfusepath{clip}%
\pgfsetbuttcap%
\pgfsetmiterjoin%
\definecolor{currentfill}{rgb}{0.000000,0.615686,0.692157}%
\pgfsetfillcolor{currentfill}%
\pgfsetlinewidth{0.000000pt}%
\definecolor{currentstroke}{rgb}{0.000000,0.000000,0.000000}%
\pgfsetstrokecolor{currentstroke}%
\pgfsetstrokeopacity{0.000000}%
\pgfsetdash{}{0pt}%
\pgfpathmoveto{\pgfqpoint{1.826743in}{0.706842in}}%
\pgfpathlineto{\pgfqpoint{1.799907in}{0.708077in}}%
\pgfpathlineto{\pgfqpoint{1.791780in}{0.708494in}}%
\pgfpathlineto{\pgfqpoint{1.792400in}{0.720174in}}%
\pgfpathlineto{\pgfqpoint{1.757470in}{0.722100in}}%
\pgfpathlineto{\pgfqpoint{1.759948in}{0.761279in}}%
\pgfpathlineto{\pgfqpoint{1.751590in}{0.761847in}}%
\pgfpathlineto{\pgfqpoint{1.753852in}{0.801689in}}%
\pgfpathlineto{\pgfqpoint{1.759459in}{0.801389in}}%
\pgfpathlineto{\pgfqpoint{1.788954in}{0.799743in}}%
\pgfpathlineto{\pgfqpoint{1.786260in}{0.758457in}}%
\pgfpathlineto{\pgfqpoint{1.792433in}{0.750538in}}%
\pgfpathlineto{\pgfqpoint{1.804967in}{0.750514in}}%
\pgfpathlineto{\pgfqpoint{1.813328in}{0.744891in}}%
\pgfpathlineto{\pgfqpoint{1.818920in}{0.750454in}}%
\pgfpathlineto{\pgfqpoint{1.828725in}{0.748261in}}%
\pgfpathlineto{\pgfqpoint{1.826743in}{0.706842in}}%
\pgfpathclose%
\pgfusepath{fill}%
\end{pgfscope}%
\begin{pgfscope}%
\pgfpathrectangle{\pgfqpoint{0.100000in}{0.100000in}}{\pgfqpoint{3.608454in}{2.310000in}}%
\pgfusepath{clip}%
\pgfsetbuttcap%
\pgfsetmiterjoin%
\definecolor{currentfill}{rgb}{0.000000,0.639216,0.680392}%
\pgfsetfillcolor{currentfill}%
\pgfsetlinewidth{0.000000pt}%
\definecolor{currentstroke}{rgb}{0.000000,0.000000,0.000000}%
\pgfsetstrokecolor{currentstroke}%
\pgfsetstrokeopacity{0.000000}%
\pgfsetdash{}{0pt}%
\pgfpathmoveto{\pgfqpoint{2.472257in}{0.716197in}}%
\pgfpathlineto{\pgfqpoint{2.470222in}{0.750540in}}%
\pgfpathlineto{\pgfqpoint{2.458660in}{0.749893in}}%
\pgfpathlineto{\pgfqpoint{2.458936in}{0.746389in}}%
\pgfpathlineto{\pgfqpoint{2.449720in}{0.745908in}}%
\pgfpathlineto{\pgfqpoint{2.447975in}{0.756350in}}%
\pgfpathlineto{\pgfqpoint{2.447021in}{0.775387in}}%
\pgfpathlineto{\pgfqpoint{2.468672in}{0.778139in}}%
\pgfpathlineto{\pgfqpoint{2.485619in}{0.780455in}}%
\pgfpathlineto{\pgfqpoint{2.487539in}{0.751603in}}%
\pgfpathlineto{\pgfqpoint{2.525381in}{0.754028in}}%
\pgfpathlineto{\pgfqpoint{2.528702in}{0.712404in}}%
\pgfpathlineto{\pgfqpoint{2.494101in}{0.710117in}}%
\pgfpathlineto{\pgfqpoint{2.493658in}{0.717377in}}%
\pgfpathlineto{\pgfqpoint{2.482623in}{0.717728in}}%
\pgfpathlineto{\pgfqpoint{2.472257in}{0.716197in}}%
\pgfpathclose%
\pgfusepath{fill}%
\end{pgfscope}%
\begin{pgfscope}%
\pgfpathrectangle{\pgfqpoint{0.100000in}{0.100000in}}{\pgfqpoint{3.608454in}{2.310000in}}%
\pgfusepath{clip}%
\pgfsetbuttcap%
\pgfsetmiterjoin%
\definecolor{currentfill}{rgb}{0.000000,0.196078,0.901961}%
\pgfsetfillcolor{currentfill}%
\pgfsetlinewidth{0.000000pt}%
\definecolor{currentstroke}{rgb}{0.000000,0.000000,0.000000}%
\pgfsetstrokecolor{currentstroke}%
\pgfsetstrokeopacity{0.000000}%
\pgfsetdash{}{0pt}%
\pgfpathmoveto{\pgfqpoint{3.129858in}{1.404228in}}%
\pgfpathlineto{\pgfqpoint{3.131195in}{1.415085in}}%
\pgfpathlineto{\pgfqpoint{3.124006in}{1.419453in}}%
\pgfpathlineto{\pgfqpoint{3.127584in}{1.429231in}}%
\pgfpathlineto{\pgfqpoint{3.125843in}{1.434080in}}%
\pgfpathlineto{\pgfqpoint{3.141078in}{1.437828in}}%
\pgfpathlineto{\pgfqpoint{3.144962in}{1.435381in}}%
\pgfpathlineto{\pgfqpoint{3.152017in}{1.444895in}}%
\pgfpathlineto{\pgfqpoint{3.153317in}{1.449269in}}%
\pgfpathlineto{\pgfqpoint{3.156244in}{1.464323in}}%
\pgfpathlineto{\pgfqpoint{3.159520in}{1.465911in}}%
\pgfpathlineto{\pgfqpoint{3.161516in}{1.490087in}}%
\pgfpathlineto{\pgfqpoint{3.166603in}{1.499150in}}%
\pgfpathlineto{\pgfqpoint{3.181076in}{1.501996in}}%
\pgfpathlineto{\pgfqpoint{3.177742in}{1.492668in}}%
\pgfpathlineto{\pgfqpoint{3.186509in}{1.489666in}}%
\pgfpathlineto{\pgfqpoint{3.191082in}{1.485805in}}%
\pgfpathlineto{\pgfqpoint{3.188742in}{1.472751in}}%
\pgfpathlineto{\pgfqpoint{3.192873in}{1.468222in}}%
\pgfpathlineto{\pgfqpoint{3.201406in}{1.464696in}}%
\pgfpathlineto{\pgfqpoint{3.204352in}{1.460668in}}%
\pgfpathlineto{\pgfqpoint{3.223628in}{1.451194in}}%
\pgfpathlineto{\pgfqpoint{3.226611in}{1.445349in}}%
\pgfpathlineto{\pgfqpoint{3.225469in}{1.437928in}}%
\pgfpathlineto{\pgfqpoint{3.227384in}{1.433116in}}%
\pgfpathlineto{\pgfqpoint{3.238143in}{1.432326in}}%
\pgfpathlineto{\pgfqpoint{3.241599in}{1.419010in}}%
\pgfpathlineto{\pgfqpoint{3.251419in}{1.409012in}}%
\pgfpathlineto{\pgfqpoint{3.254865in}{1.395654in}}%
\pgfpathlineto{\pgfqpoint{3.260051in}{1.390426in}}%
\pgfpathlineto{\pgfqpoint{3.258704in}{1.383890in}}%
\pgfpathlineto{\pgfqpoint{3.252557in}{1.388608in}}%
\pgfpathlineto{\pgfqpoint{3.237222in}{1.393476in}}%
\pgfpathlineto{\pgfqpoint{3.230597in}{1.391657in}}%
\pgfpathlineto{\pgfqpoint{3.214679in}{1.396873in}}%
\pgfpathlineto{\pgfqpoint{3.209863in}{1.398797in}}%
\pgfpathlineto{\pgfqpoint{3.199440in}{1.393885in}}%
\pgfpathlineto{\pgfqpoint{3.196379in}{1.396687in}}%
\pgfpathlineto{\pgfqpoint{3.194258in}{1.405953in}}%
\pgfpathlineto{\pgfqpoint{3.182634in}{1.411460in}}%
\pgfpathlineto{\pgfqpoint{3.175115in}{1.399614in}}%
\pgfpathlineto{\pgfqpoint{3.175771in}{1.395430in}}%
\pgfpathlineto{\pgfqpoint{3.169602in}{1.394541in}}%
\pgfpathlineto{\pgfqpoint{3.158541in}{1.406177in}}%
\pgfpathlineto{\pgfqpoint{3.153282in}{1.414127in}}%
\pgfpathlineto{\pgfqpoint{3.141594in}{1.398020in}}%
\pgfpathlineto{\pgfqpoint{3.137714in}{1.398047in}}%
\pgfpathlineto{\pgfqpoint{3.129858in}{1.404228in}}%
\pgfpathclose%
\pgfusepath{fill}%
\end{pgfscope}%
\begin{pgfscope}%
\pgfpathrectangle{\pgfqpoint{0.100000in}{0.100000in}}{\pgfqpoint{3.608454in}{2.310000in}}%
\pgfusepath{clip}%
\pgfsetbuttcap%
\pgfsetmiterjoin%
\definecolor{currentfill}{rgb}{0.000000,0.952941,0.523529}%
\pgfsetfillcolor{currentfill}%
\pgfsetlinewidth{0.000000pt}%
\definecolor{currentstroke}{rgb}{0.000000,0.000000,0.000000}%
\pgfsetstrokecolor{currentstroke}%
\pgfsetstrokeopacity{0.000000}%
\pgfsetdash{}{0pt}%
\pgfpathmoveto{\pgfqpoint{0.605938in}{1.604509in}}%
\pgfpathlineto{\pgfqpoint{0.599878in}{1.583444in}}%
\pgfpathlineto{\pgfqpoint{0.570419in}{1.591928in}}%
\pgfpathlineto{\pgfqpoint{0.567874in}{1.599010in}}%
\pgfpathlineto{\pgfqpoint{0.563417in}{1.600581in}}%
\pgfpathlineto{\pgfqpoint{0.556871in}{1.597127in}}%
\pgfpathlineto{\pgfqpoint{0.538914in}{1.596597in}}%
\pgfpathlineto{\pgfqpoint{0.540950in}{1.606159in}}%
\pgfpathlineto{\pgfqpoint{0.529553in}{1.637642in}}%
\pgfpathlineto{\pgfqpoint{0.526964in}{1.639730in}}%
\pgfpathlineto{\pgfqpoint{0.529796in}{1.650793in}}%
\pgfpathlineto{\pgfqpoint{0.534394in}{1.657082in}}%
\pgfpathlineto{\pgfqpoint{0.531886in}{1.661402in}}%
\pgfpathlineto{\pgfqpoint{0.537339in}{1.664898in}}%
\pgfpathlineto{\pgfqpoint{0.540383in}{1.672372in}}%
\pgfpathlineto{\pgfqpoint{0.534540in}{1.676936in}}%
\pgfpathlineto{\pgfqpoint{0.535033in}{1.684697in}}%
\pgfpathlineto{\pgfqpoint{0.544816in}{1.681752in}}%
\pgfpathlineto{\pgfqpoint{0.560209in}{1.677284in}}%
\pgfpathlineto{\pgfqpoint{0.555953in}{1.662869in}}%
\pgfpathlineto{\pgfqpoint{0.558505in}{1.659196in}}%
\pgfpathlineto{\pgfqpoint{0.566606in}{1.658522in}}%
\pgfpathlineto{\pgfqpoint{0.574470in}{1.662350in}}%
\pgfpathlineto{\pgfqpoint{0.584658in}{1.656876in}}%
\pgfpathlineto{\pgfqpoint{0.594599in}{1.640016in}}%
\pgfpathlineto{\pgfqpoint{0.601624in}{1.635606in}}%
\pgfpathlineto{\pgfqpoint{0.600541in}{1.629953in}}%
\pgfpathlineto{\pgfqpoint{0.604334in}{1.622797in}}%
\pgfpathlineto{\pgfqpoint{0.600479in}{1.609616in}}%
\pgfpathlineto{\pgfqpoint{0.605938in}{1.604509in}}%
\pgfpathclose%
\pgfusepath{fill}%
\end{pgfscope}%
\begin{pgfscope}%
\pgfpathrectangle{\pgfqpoint{0.100000in}{0.100000in}}{\pgfqpoint{3.608454in}{2.310000in}}%
\pgfusepath{clip}%
\pgfsetbuttcap%
\pgfsetmiterjoin%
\definecolor{currentfill}{rgb}{0.000000,0.470588,0.764706}%
\pgfsetfillcolor{currentfill}%
\pgfsetlinewidth{0.000000pt}%
\definecolor{currentstroke}{rgb}{0.000000,0.000000,0.000000}%
\pgfsetstrokecolor{currentstroke}%
\pgfsetstrokeopacity{0.000000}%
\pgfsetdash{}{0pt}%
\pgfpathmoveto{\pgfqpoint{1.888636in}{1.700186in}}%
\pgfpathlineto{\pgfqpoint{1.889333in}{1.734748in}}%
\pgfpathlineto{\pgfqpoint{1.916523in}{1.733891in}}%
\pgfpathlineto{\pgfqpoint{1.917260in}{1.754415in}}%
\pgfpathlineto{\pgfqpoint{1.944449in}{1.753368in}}%
\pgfpathlineto{\pgfqpoint{1.944686in}{1.753364in}}%
\pgfpathlineto{\pgfqpoint{1.943727in}{1.725985in}}%
\pgfpathlineto{\pgfqpoint{1.957610in}{1.725492in}}%
\pgfpathlineto{\pgfqpoint{1.956891in}{1.697804in}}%
\pgfpathlineto{\pgfqpoint{1.888636in}{1.700186in}}%
\pgfpathclose%
\pgfusepath{fill}%
\end{pgfscope}%
\begin{pgfscope}%
\pgfpathrectangle{\pgfqpoint{0.100000in}{0.100000in}}{\pgfqpoint{3.608454in}{2.310000in}}%
\pgfusepath{clip}%
\pgfsetbuttcap%
\pgfsetmiterjoin%
\definecolor{currentfill}{rgb}{0.000000,0.909804,0.545098}%
\pgfsetfillcolor{currentfill}%
\pgfsetlinewidth{0.000000pt}%
\definecolor{currentstroke}{rgb}{0.000000,0.000000,0.000000}%
\pgfsetstrokecolor{currentstroke}%
\pgfsetstrokeopacity{0.000000}%
\pgfsetdash{}{0pt}%
\pgfpathmoveto{\pgfqpoint{0.937150in}{2.279229in}}%
\pgfpathlineto{\pgfqpoint{0.927075in}{2.268237in}}%
\pgfpathlineto{\pgfqpoint{0.925222in}{2.260906in}}%
\pgfpathlineto{\pgfqpoint{0.920139in}{2.255027in}}%
\pgfpathlineto{\pgfqpoint{0.926800in}{2.253370in}}%
\pgfpathlineto{\pgfqpoint{0.916424in}{2.212894in}}%
\pgfpathlineto{\pgfqpoint{0.913019in}{2.206570in}}%
\pgfpathlineto{\pgfqpoint{0.908021in}{2.186968in}}%
\pgfpathlineto{\pgfqpoint{0.902377in}{2.193011in}}%
\pgfpathlineto{\pgfqpoint{0.894472in}{2.191910in}}%
\pgfpathlineto{\pgfqpoint{0.891121in}{2.195618in}}%
\pgfpathlineto{\pgfqpoint{0.879920in}{2.194564in}}%
\pgfpathlineto{\pgfqpoint{0.878250in}{2.201957in}}%
\pgfpathlineto{\pgfqpoint{0.873890in}{2.207982in}}%
\pgfpathlineto{\pgfqpoint{0.862221in}{2.202501in}}%
\pgfpathlineto{\pgfqpoint{0.857008in}{2.211536in}}%
\pgfpathlineto{\pgfqpoint{0.843741in}{2.216808in}}%
\pgfpathlineto{\pgfqpoint{0.836363in}{2.217995in}}%
\pgfpathlineto{\pgfqpoint{0.836288in}{2.219545in}}%
\pgfpathlineto{\pgfqpoint{0.836249in}{2.233779in}}%
\pgfpathlineto{\pgfqpoint{0.832614in}{2.236217in}}%
\pgfpathlineto{\pgfqpoint{0.819597in}{2.231350in}}%
\pgfpathlineto{\pgfqpoint{0.815034in}{2.234984in}}%
\pgfpathlineto{\pgfqpoint{0.802328in}{2.232125in}}%
\pgfpathlineto{\pgfqpoint{0.800018in}{2.241064in}}%
\pgfpathlineto{\pgfqpoint{0.791522in}{2.239767in}}%
\pgfpathlineto{\pgfqpoint{0.790372in}{2.232053in}}%
\pgfpathlineto{\pgfqpoint{0.780931in}{2.234819in}}%
\pgfpathlineto{\pgfqpoint{0.780160in}{2.239854in}}%
\pgfpathlineto{\pgfqpoint{0.769468in}{2.254538in}}%
\pgfpathlineto{\pgfqpoint{0.771087in}{2.257398in}}%
\pgfpathlineto{\pgfqpoint{0.765081in}{2.268518in}}%
\pgfpathlineto{\pgfqpoint{0.760184in}{2.281456in}}%
\pgfpathlineto{\pgfqpoint{0.760304in}{2.288239in}}%
\pgfpathlineto{\pgfqpoint{0.757956in}{2.296770in}}%
\pgfpathlineto{\pgfqpoint{0.760555in}{2.298682in}}%
\pgfpathlineto{\pgfqpoint{0.765166in}{2.300496in}}%
\pgfpathlineto{\pgfqpoint{0.763231in}{2.326563in}}%
\pgfpathlineto{\pgfqpoint{0.801462in}{2.315507in}}%
\pgfpathlineto{\pgfqpoint{0.847920in}{2.302634in}}%
\pgfpathlineto{\pgfqpoint{0.882493in}{2.293352in}}%
\pgfpathlineto{\pgfqpoint{0.937150in}{2.279229in}}%
\pgfpathclose%
\pgfusepath{fill}%
\end{pgfscope}%
\begin{pgfscope}%
\pgfpathrectangle{\pgfqpoint{0.100000in}{0.100000in}}{\pgfqpoint{3.608454in}{2.310000in}}%
\pgfusepath{clip}%
\pgfsetbuttcap%
\pgfsetmiterjoin%
\definecolor{currentfill}{rgb}{0.000000,0.580392,0.709804}%
\pgfsetfillcolor{currentfill}%
\pgfsetlinewidth{0.000000pt}%
\definecolor{currentstroke}{rgb}{0.000000,0.000000,0.000000}%
\pgfsetstrokecolor{currentstroke}%
\pgfsetstrokeopacity{0.000000}%
\pgfsetdash{}{0pt}%
\pgfpathmoveto{\pgfqpoint{2.821573in}{0.886623in}}%
\pgfpathlineto{\pgfqpoint{2.814267in}{0.888921in}}%
\pgfpathlineto{\pgfqpoint{2.805495in}{0.895708in}}%
\pgfpathlineto{\pgfqpoint{2.802438in}{0.905479in}}%
\pgfpathlineto{\pgfqpoint{2.802911in}{0.922322in}}%
\pgfpathlineto{\pgfqpoint{2.802892in}{0.925244in}}%
\pgfpathlineto{\pgfqpoint{2.807171in}{0.925663in}}%
\pgfpathlineto{\pgfqpoint{2.809234in}{0.933588in}}%
\pgfpathlineto{\pgfqpoint{2.818507in}{0.934685in}}%
\pgfpathlineto{\pgfqpoint{2.830443in}{0.932055in}}%
\pgfpathlineto{\pgfqpoint{2.838874in}{0.945035in}}%
\pgfpathlineto{\pgfqpoint{2.843454in}{0.939687in}}%
\pgfpathlineto{\pgfqpoint{2.848096in}{0.929927in}}%
\pgfpathlineto{\pgfqpoint{2.848183in}{0.924494in}}%
\pgfpathlineto{\pgfqpoint{2.844937in}{0.925826in}}%
\pgfpathlineto{\pgfqpoint{2.833490in}{0.924483in}}%
\pgfpathlineto{\pgfqpoint{2.835752in}{0.904367in}}%
\pgfpathlineto{\pgfqpoint{2.830655in}{0.902493in}}%
\pgfpathlineto{\pgfqpoint{2.830174in}{0.891948in}}%
\pgfpathlineto{\pgfqpoint{2.827685in}{0.882746in}}%
\pgfpathlineto{\pgfqpoint{2.821573in}{0.886623in}}%
\pgfpathclose%
\pgfusepath{fill}%
\end{pgfscope}%
\begin{pgfscope}%
\pgfpathrectangle{\pgfqpoint{0.100000in}{0.100000in}}{\pgfqpoint{3.608454in}{2.310000in}}%
\pgfusepath{clip}%
\pgfsetbuttcap%
\pgfsetmiterjoin%
\definecolor{currentfill}{rgb}{0.000000,0.286275,0.856863}%
\pgfsetfillcolor{currentfill}%
\pgfsetlinewidth{0.000000pt}%
\definecolor{currentstroke}{rgb}{0.000000,0.000000,0.000000}%
\pgfsetstrokecolor{currentstroke}%
\pgfsetstrokeopacity{0.000000}%
\pgfsetdash{}{0pt}%
\pgfpathmoveto{\pgfqpoint{3.238143in}{1.432326in}}%
\pgfpathlineto{\pgfqpoint{3.227384in}{1.433116in}}%
\pgfpathlineto{\pgfqpoint{3.225469in}{1.437928in}}%
\pgfpathlineto{\pgfqpoint{3.226611in}{1.445349in}}%
\pgfpathlineto{\pgfqpoint{3.223628in}{1.451194in}}%
\pgfpathlineto{\pgfqpoint{3.204352in}{1.460668in}}%
\pgfpathlineto{\pgfqpoint{3.201406in}{1.464696in}}%
\pgfpathlineto{\pgfqpoint{3.192873in}{1.468222in}}%
\pgfpathlineto{\pgfqpoint{3.188742in}{1.472751in}}%
\pgfpathlineto{\pgfqpoint{3.191082in}{1.485805in}}%
\pgfpathlineto{\pgfqpoint{3.186509in}{1.489666in}}%
\pgfpathlineto{\pgfqpoint{3.177742in}{1.492668in}}%
\pgfpathlineto{\pgfqpoint{3.181076in}{1.501996in}}%
\pgfpathlineto{\pgfqpoint{3.239245in}{1.513965in}}%
\pgfpathlineto{\pgfqpoint{3.239646in}{1.514041in}}%
\pgfpathlineto{\pgfqpoint{3.251965in}{1.502268in}}%
\pgfpathlineto{\pgfqpoint{3.252119in}{1.493455in}}%
\pgfpathlineto{\pgfqpoint{3.248435in}{1.488117in}}%
\pgfpathlineto{\pgfqpoint{3.238301in}{1.480374in}}%
\pgfpathlineto{\pgfqpoint{3.235613in}{1.470733in}}%
\pgfpathlineto{\pgfqpoint{3.238603in}{1.459477in}}%
\pgfpathlineto{\pgfqpoint{3.239193in}{1.448194in}}%
\pgfpathlineto{\pgfqpoint{3.235959in}{1.434274in}}%
\pgfpathlineto{\pgfqpoint{3.238143in}{1.432326in}}%
\pgfpathclose%
\pgfusepath{fill}%
\end{pgfscope}%
\begin{pgfscope}%
\pgfpathrectangle{\pgfqpoint{0.100000in}{0.100000in}}{\pgfqpoint{3.608454in}{2.310000in}}%
\pgfusepath{clip}%
\pgfsetbuttcap%
\pgfsetmiterjoin%
\definecolor{currentfill}{rgb}{0.000000,0.592157,0.703922}%
\pgfsetfillcolor{currentfill}%
\pgfsetlinewidth{0.000000pt}%
\definecolor{currentstroke}{rgb}{0.000000,0.000000,0.000000}%
\pgfsetstrokecolor{currentstroke}%
\pgfsetstrokeopacity{0.000000}%
\pgfsetdash{}{0pt}%
\pgfpathmoveto{\pgfqpoint{1.821831in}{1.246792in}}%
\pgfpathlineto{\pgfqpoint{1.821521in}{1.253978in}}%
\pgfpathlineto{\pgfqpoint{1.823241in}{1.288458in}}%
\pgfpathlineto{\pgfqpoint{1.822395in}{1.288502in}}%
\pgfpathlineto{\pgfqpoint{1.824072in}{1.322793in}}%
\pgfpathlineto{\pgfqpoint{1.891797in}{1.319820in}}%
\pgfpathlineto{\pgfqpoint{1.890012in}{1.250882in}}%
\pgfpathlineto{\pgfqpoint{1.856262in}{1.252307in}}%
\pgfpathlineto{\pgfqpoint{1.855927in}{1.245032in}}%
\pgfpathlineto{\pgfqpoint{1.821831in}{1.246792in}}%
\pgfpathclose%
\pgfusepath{fill}%
\end{pgfscope}%
\begin{pgfscope}%
\pgfpathrectangle{\pgfqpoint{0.100000in}{0.100000in}}{\pgfqpoint{3.608454in}{2.310000in}}%
\pgfusepath{clip}%
\pgfsetbuttcap%
\pgfsetmiterjoin%
\definecolor{currentfill}{rgb}{0.000000,0.678431,0.660784}%
\pgfsetfillcolor{currentfill}%
\pgfsetlinewidth{0.000000pt}%
\definecolor{currentstroke}{rgb}{0.000000,0.000000,0.000000}%
\pgfsetstrokecolor{currentstroke}%
\pgfsetstrokeopacity{0.000000}%
\pgfsetdash{}{0pt}%
\pgfpathmoveto{\pgfqpoint{2.625838in}{1.802295in}}%
\pgfpathlineto{\pgfqpoint{2.602311in}{1.800403in}}%
\pgfpathlineto{\pgfqpoint{2.600324in}{1.814817in}}%
\pgfpathlineto{\pgfqpoint{2.608715in}{1.818757in}}%
\pgfpathlineto{\pgfqpoint{2.608533in}{1.830495in}}%
\pgfpathlineto{\pgfqpoint{2.613100in}{1.831951in}}%
\pgfpathlineto{\pgfqpoint{2.615491in}{1.837204in}}%
\pgfpathlineto{\pgfqpoint{2.623020in}{1.837013in}}%
\pgfpathlineto{\pgfqpoint{2.625532in}{1.845348in}}%
\pgfpathlineto{\pgfqpoint{2.632241in}{1.851902in}}%
\pgfpathlineto{\pgfqpoint{2.634383in}{1.847792in}}%
\pgfpathlineto{\pgfqpoint{2.634627in}{1.836092in}}%
\pgfpathlineto{\pgfqpoint{2.632415in}{1.832536in}}%
\pgfpathlineto{\pgfqpoint{2.633665in}{1.824033in}}%
\pgfpathlineto{\pgfqpoint{2.640260in}{1.822509in}}%
\pgfpathlineto{\pgfqpoint{2.645608in}{1.835898in}}%
\pgfpathlineto{\pgfqpoint{2.646411in}{1.852152in}}%
\pgfpathlineto{\pgfqpoint{2.644203in}{1.859587in}}%
\pgfpathlineto{\pgfqpoint{2.653419in}{1.860520in}}%
\pgfpathlineto{\pgfqpoint{2.654113in}{1.853564in}}%
\pgfpathlineto{\pgfqpoint{2.674595in}{1.855705in}}%
\pgfpathlineto{\pgfqpoint{2.677389in}{1.835395in}}%
\pgfpathlineto{\pgfqpoint{2.680280in}{1.807943in}}%
\pgfpathlineto{\pgfqpoint{2.625838in}{1.802295in}}%
\pgfpathclose%
\pgfusepath{fill}%
\end{pgfscope}%
\begin{pgfscope}%
\pgfpathrectangle{\pgfqpoint{0.100000in}{0.100000in}}{\pgfqpoint{3.608454in}{2.310000in}}%
\pgfusepath{clip}%
\pgfsetbuttcap%
\pgfsetmiterjoin%
\definecolor{currentfill}{rgb}{0.000000,0.305882,0.847059}%
\pgfsetfillcolor{currentfill}%
\pgfsetlinewidth{0.000000pt}%
\definecolor{currentstroke}{rgb}{0.000000,0.000000,0.000000}%
\pgfsetstrokecolor{currentstroke}%
\pgfsetstrokeopacity{0.000000}%
\pgfsetdash{}{0pt}%
\pgfpathmoveto{\pgfqpoint{1.830021in}{2.007934in}}%
\pgfpathlineto{\pgfqpoint{1.828413in}{1.980392in}}%
\pgfpathlineto{\pgfqpoint{1.830555in}{1.980267in}}%
\pgfpathlineto{\pgfqpoint{1.828664in}{1.952547in}}%
\pgfpathlineto{\pgfqpoint{1.796923in}{1.954510in}}%
\pgfpathlineto{\pgfqpoint{1.801850in}{1.950975in}}%
\pgfpathlineto{\pgfqpoint{1.802847in}{1.946100in}}%
\pgfpathlineto{\pgfqpoint{1.799830in}{1.937828in}}%
\pgfpathlineto{\pgfqpoint{1.787591in}{1.935130in}}%
\pgfpathlineto{\pgfqpoint{1.783790in}{1.937380in}}%
\pgfpathlineto{\pgfqpoint{1.775193in}{1.927930in}}%
\pgfpathlineto{\pgfqpoint{1.774399in}{1.934946in}}%
\pgfpathlineto{\pgfqpoint{1.760656in}{1.935885in}}%
\pgfpathlineto{\pgfqpoint{1.762088in}{1.956478in}}%
\pgfpathlineto{\pgfqpoint{1.739184in}{1.958172in}}%
\pgfpathlineto{\pgfqpoint{1.739705in}{1.965094in}}%
\pgfpathlineto{\pgfqpoint{1.719085in}{1.966639in}}%
\pgfpathlineto{\pgfqpoint{1.720698in}{1.987602in}}%
\pgfpathlineto{\pgfqpoint{1.718098in}{1.987796in}}%
\pgfpathlineto{\pgfqpoint{1.720286in}{2.015429in}}%
\pgfpathlineto{\pgfqpoint{1.716982in}{2.015681in}}%
\pgfpathlineto{\pgfqpoint{1.718555in}{2.035319in}}%
\pgfpathlineto{\pgfqpoint{1.715648in}{2.039417in}}%
\pgfpathlineto{\pgfqpoint{1.716886in}{2.052052in}}%
\pgfpathlineto{\pgfqpoint{1.710047in}{2.050726in}}%
\pgfpathlineto{\pgfqpoint{1.710634in}{2.057876in}}%
\pgfpathlineto{\pgfqpoint{1.738152in}{2.055718in}}%
\pgfpathlineto{\pgfqpoint{1.807069in}{2.050968in}}%
\pgfpathlineto{\pgfqpoint{1.806219in}{2.037109in}}%
\pgfpathlineto{\pgfqpoint{1.801529in}{2.037408in}}%
\pgfpathlineto{\pgfqpoint{1.799746in}{2.009759in}}%
\pgfpathlineto{\pgfqpoint{1.830021in}{2.007934in}}%
\pgfpathclose%
\pgfusepath{fill}%
\end{pgfscope}%
\begin{pgfscope}%
\pgfpathrectangle{\pgfqpoint{0.100000in}{0.100000in}}{\pgfqpoint{3.608454in}{2.310000in}}%
\pgfusepath{clip}%
\pgfsetbuttcap%
\pgfsetmiterjoin%
\definecolor{currentfill}{rgb}{0.000000,0.423529,0.788235}%
\pgfsetfillcolor{currentfill}%
\pgfsetlinewidth{0.000000pt}%
\definecolor{currentstroke}{rgb}{0.000000,0.000000,0.000000}%
\pgfsetstrokecolor{currentstroke}%
\pgfsetstrokeopacity{0.000000}%
\pgfsetdash{}{0pt}%
\pgfpathmoveto{\pgfqpoint{3.285323in}{1.505755in}}%
\pgfpathlineto{\pgfqpoint{3.285117in}{1.511332in}}%
\pgfpathlineto{\pgfqpoint{3.281349in}{1.515956in}}%
\pgfpathlineto{\pgfqpoint{3.284745in}{1.526913in}}%
\pgfpathlineto{\pgfqpoint{3.284318in}{1.531602in}}%
\pgfpathlineto{\pgfqpoint{3.275612in}{1.531070in}}%
\pgfpathlineto{\pgfqpoint{3.271013in}{1.528198in}}%
\pgfpathlineto{\pgfqpoint{3.266966in}{1.519876in}}%
\pgfpathlineto{\pgfqpoint{3.245459in}{1.515286in}}%
\pgfpathlineto{\pgfqpoint{3.248921in}{1.519116in}}%
\pgfpathlineto{\pgfqpoint{3.250934in}{1.527034in}}%
\pgfpathlineto{\pgfqpoint{3.250059in}{1.534993in}}%
\pgfpathlineto{\pgfqpoint{3.252158in}{1.542409in}}%
\pgfpathlineto{\pgfqpoint{3.250514in}{1.547606in}}%
\pgfpathlineto{\pgfqpoint{3.262873in}{1.561037in}}%
\pgfpathlineto{\pgfqpoint{3.269282in}{1.579037in}}%
\pgfpathlineto{\pgfqpoint{3.275573in}{1.583652in}}%
\pgfpathlineto{\pgfqpoint{3.286080in}{1.595797in}}%
\pgfpathlineto{\pgfqpoint{3.294732in}{1.592344in}}%
\pgfpathlineto{\pgfqpoint{3.297442in}{1.582891in}}%
\pgfpathlineto{\pgfqpoint{3.303159in}{1.582589in}}%
\pgfpathlineto{\pgfqpoint{3.313073in}{1.572820in}}%
\pgfpathlineto{\pgfqpoint{3.330022in}{1.567556in}}%
\pgfpathlineto{\pgfqpoint{3.348171in}{1.542140in}}%
\pgfpathlineto{\pgfqpoint{3.353587in}{1.521095in}}%
\pgfpathlineto{\pgfqpoint{3.358684in}{1.520578in}}%
\pgfpathlineto{\pgfqpoint{3.354142in}{1.510014in}}%
\pgfpathlineto{\pgfqpoint{3.344357in}{1.498736in}}%
\pgfpathlineto{\pgfqpoint{3.337885in}{1.476407in}}%
\pgfpathlineto{\pgfqpoint{3.334527in}{1.471502in}}%
\pgfpathlineto{\pgfqpoint{3.328098in}{1.470069in}}%
\pgfpathlineto{\pgfqpoint{3.329144in}{1.488418in}}%
\pgfpathlineto{\pgfqpoint{3.310575in}{1.490636in}}%
\pgfpathlineto{\pgfqpoint{3.301327in}{1.493916in}}%
\pgfpathlineto{\pgfqpoint{3.288516in}{1.501805in}}%
\pgfpathlineto{\pgfqpoint{3.285323in}{1.505755in}}%
\pgfpathclose%
\pgfusepath{fill}%
\end{pgfscope}%
\begin{pgfscope}%
\pgfpathrectangle{\pgfqpoint{0.100000in}{0.100000in}}{\pgfqpoint{3.608454in}{2.310000in}}%
\pgfusepath{clip}%
\pgfsetbuttcap%
\pgfsetmiterjoin%
\definecolor{currentfill}{rgb}{0.000000,0.333333,0.833333}%
\pgfsetfillcolor{currentfill}%
\pgfsetlinewidth{0.000000pt}%
\definecolor{currentstroke}{rgb}{0.000000,0.000000,0.000000}%
\pgfsetstrokecolor{currentstroke}%
\pgfsetstrokeopacity{0.000000}%
\pgfsetdash{}{0pt}%
\pgfpathmoveto{\pgfqpoint{1.725856in}{1.224921in}}%
\pgfpathlineto{\pgfqpoint{1.723873in}{1.194151in}}%
\pgfpathlineto{\pgfqpoint{1.693075in}{1.196095in}}%
\pgfpathlineto{\pgfqpoint{1.695230in}{1.227106in}}%
\pgfpathlineto{\pgfqpoint{1.664847in}{1.229439in}}%
\pgfpathlineto{\pgfqpoint{1.666412in}{1.249548in}}%
\pgfpathlineto{\pgfqpoint{1.666954in}{1.256981in}}%
\pgfpathlineto{\pgfqpoint{1.726337in}{1.252520in}}%
\pgfpathlineto{\pgfqpoint{1.724493in}{1.225005in}}%
\pgfpathlineto{\pgfqpoint{1.725856in}{1.224921in}}%
\pgfpathclose%
\pgfusepath{fill}%
\end{pgfscope}%
\begin{pgfscope}%
\pgfpathrectangle{\pgfqpoint{0.100000in}{0.100000in}}{\pgfqpoint{3.608454in}{2.310000in}}%
\pgfusepath{clip}%
\pgfsetbuttcap%
\pgfsetmiterjoin%
\definecolor{currentfill}{rgb}{0.000000,0.596078,0.701961}%
\pgfsetfillcolor{currentfill}%
\pgfsetlinewidth{0.000000pt}%
\definecolor{currentstroke}{rgb}{0.000000,0.000000,0.000000}%
\pgfsetstrokecolor{currentstroke}%
\pgfsetstrokeopacity{0.000000}%
\pgfsetdash{}{0pt}%
\pgfpathmoveto{\pgfqpoint{2.856403in}{0.756814in}}%
\pgfpathlineto{\pgfqpoint{2.856733in}{0.753994in}}%
\pgfpathlineto{\pgfqpoint{2.874929in}{0.755916in}}%
\pgfpathlineto{\pgfqpoint{2.878161in}{0.725802in}}%
\pgfpathlineto{\pgfqpoint{2.834431in}{0.723057in}}%
\pgfpathlineto{\pgfqpoint{2.831112in}{0.753917in}}%
\pgfpathlineto{\pgfqpoint{2.856403in}{0.756814in}}%
\pgfpathclose%
\pgfusepath{fill}%
\end{pgfscope}%
\begin{pgfscope}%
\pgfpathrectangle{\pgfqpoint{0.100000in}{0.100000in}}{\pgfqpoint{3.608454in}{2.310000in}}%
\pgfusepath{clip}%
\pgfsetbuttcap%
\pgfsetmiterjoin%
\definecolor{currentfill}{rgb}{0.000000,0.552941,0.723529}%
\pgfsetfillcolor{currentfill}%
\pgfsetlinewidth{0.000000pt}%
\definecolor{currentstroke}{rgb}{0.000000,0.000000,0.000000}%
\pgfsetstrokecolor{currentstroke}%
\pgfsetstrokeopacity{0.000000}%
\pgfsetdash{}{0pt}%
\pgfpathmoveto{\pgfqpoint{2.519677in}{1.268419in}}%
\pgfpathlineto{\pgfqpoint{2.499015in}{1.266823in}}%
\pgfpathlineto{\pgfqpoint{2.497937in}{1.284068in}}%
\pgfpathlineto{\pgfqpoint{2.470338in}{1.282183in}}%
\pgfpathlineto{\pgfqpoint{2.470034in}{1.289131in}}%
\pgfpathlineto{\pgfqpoint{2.435549in}{1.287521in}}%
\pgfpathlineto{\pgfqpoint{2.433309in}{1.321900in}}%
\pgfpathlineto{\pgfqpoint{2.440188in}{1.322366in}}%
\pgfpathlineto{\pgfqpoint{2.439649in}{1.329250in}}%
\pgfpathlineto{\pgfqpoint{2.460771in}{1.330448in}}%
\pgfpathlineto{\pgfqpoint{2.467926in}{1.330458in}}%
\pgfpathlineto{\pgfqpoint{2.467492in}{1.337391in}}%
\pgfpathlineto{\pgfqpoint{2.494687in}{1.339376in}}%
\pgfpathlineto{\pgfqpoint{2.495702in}{1.322010in}}%
\pgfpathlineto{\pgfqpoint{2.497373in}{1.294428in}}%
\pgfpathlineto{\pgfqpoint{2.517928in}{1.295771in}}%
\pgfpathlineto{\pgfqpoint{2.519677in}{1.268419in}}%
\pgfpathclose%
\pgfusepath{fill}%
\end{pgfscope}%
\begin{pgfscope}%
\pgfpathrectangle{\pgfqpoint{0.100000in}{0.100000in}}{\pgfqpoint{3.608454in}{2.310000in}}%
\pgfusepath{clip}%
\pgfsetbuttcap%
\pgfsetmiterjoin%
\definecolor{currentfill}{rgb}{0.000000,0.647059,0.676471}%
\pgfsetfillcolor{currentfill}%
\pgfsetlinewidth{0.000000pt}%
\definecolor{currentstroke}{rgb}{0.000000,0.000000,0.000000}%
\pgfsetstrokecolor{currentstroke}%
\pgfsetstrokeopacity{0.000000}%
\pgfsetdash{}{0pt}%
\pgfpathmoveto{\pgfqpoint{2.669844in}{0.772749in}}%
\pgfpathlineto{\pgfqpoint{2.660506in}{0.772013in}}%
\pgfpathlineto{\pgfqpoint{2.655184in}{0.780136in}}%
\pgfpathlineto{\pgfqpoint{2.653773in}{0.795845in}}%
\pgfpathlineto{\pgfqpoint{2.652230in}{0.813085in}}%
\pgfpathlineto{\pgfqpoint{2.650875in}{0.827115in}}%
\pgfpathlineto{\pgfqpoint{2.656320in}{0.836856in}}%
\pgfpathlineto{\pgfqpoint{2.648743in}{0.852393in}}%
\pgfpathlineto{\pgfqpoint{2.647203in}{0.861864in}}%
\pgfpathlineto{\pgfqpoint{2.660793in}{0.862897in}}%
\pgfpathlineto{\pgfqpoint{2.660419in}{0.866374in}}%
\pgfpathlineto{\pgfqpoint{2.680466in}{0.868375in}}%
\pgfpathlineto{\pgfqpoint{2.682505in}{0.872318in}}%
\pgfpathlineto{\pgfqpoint{2.715600in}{0.875575in}}%
\pgfpathlineto{\pgfqpoint{2.717264in}{0.854842in}}%
\pgfpathlineto{\pgfqpoint{2.708963in}{0.847375in}}%
\pgfpathlineto{\pgfqpoint{2.711685in}{0.841617in}}%
\pgfpathlineto{\pgfqpoint{2.713352in}{0.819097in}}%
\pgfpathlineto{\pgfqpoint{2.714111in}{0.812491in}}%
\pgfpathlineto{\pgfqpoint{2.693497in}{0.810278in}}%
\pgfpathlineto{\pgfqpoint{2.692745in}{0.817134in}}%
\pgfpathlineto{\pgfqpoint{2.685856in}{0.816401in}}%
\pgfpathlineto{\pgfqpoint{2.683683in}{0.809309in}}%
\pgfpathlineto{\pgfqpoint{2.685966in}{0.784799in}}%
\pgfpathlineto{\pgfqpoint{2.682532in}{0.784446in}}%
\pgfpathlineto{\pgfqpoint{2.683521in}{0.774139in}}%
\pgfpathlineto{\pgfqpoint{2.669844in}{0.772749in}}%
\pgfpathclose%
\pgfusepath{fill}%
\end{pgfscope}%
\begin{pgfscope}%
\pgfpathrectangle{\pgfqpoint{0.100000in}{0.100000in}}{\pgfqpoint{3.608454in}{2.310000in}}%
\pgfusepath{clip}%
\pgfsetbuttcap%
\pgfsetmiterjoin%
\definecolor{currentfill}{rgb}{0.000000,0.368627,0.815686}%
\pgfsetfillcolor{currentfill}%
\pgfsetlinewidth{0.000000pt}%
\definecolor{currentstroke}{rgb}{0.000000,0.000000,0.000000}%
\pgfsetstrokecolor{currentstroke}%
\pgfsetstrokeopacity{0.000000}%
\pgfsetdash{}{0pt}%
\pgfpathmoveto{\pgfqpoint{0.658484in}{2.308460in}}%
\pgfpathlineto{\pgfqpoint{0.663035in}{2.299847in}}%
\pgfpathlineto{\pgfqpoint{0.657090in}{2.299503in}}%
\pgfpathlineto{\pgfqpoint{0.654017in}{2.294863in}}%
\pgfpathlineto{\pgfqpoint{0.658915in}{2.282360in}}%
\pgfpathlineto{\pgfqpoint{0.663385in}{2.275790in}}%
\pgfpathlineto{\pgfqpoint{0.662941in}{2.269541in}}%
\pgfpathlineto{\pgfqpoint{0.657685in}{2.267521in}}%
\pgfpathlineto{\pgfqpoint{0.656759in}{2.274209in}}%
\pgfpathlineto{\pgfqpoint{0.651475in}{2.279265in}}%
\pgfpathlineto{\pgfqpoint{0.653699in}{2.284940in}}%
\pgfpathlineto{\pgfqpoint{0.647875in}{2.296843in}}%
\pgfpathlineto{\pgfqpoint{0.658484in}{2.308460in}}%
\pgfpathclose%
\pgfusepath{fill}%
\end{pgfscope}%
\begin{pgfscope}%
\pgfpathrectangle{\pgfqpoint{0.100000in}{0.100000in}}{\pgfqpoint{3.608454in}{2.310000in}}%
\pgfusepath{clip}%
\pgfsetbuttcap%
\pgfsetmiterjoin%
\definecolor{currentfill}{rgb}{0.000000,0.368627,0.815686}%
\pgfsetfillcolor{currentfill}%
\pgfsetlinewidth{0.000000pt}%
\definecolor{currentstroke}{rgb}{0.000000,0.000000,0.000000}%
\pgfsetstrokecolor{currentstroke}%
\pgfsetstrokeopacity{0.000000}%
\pgfsetdash{}{0pt}%
\pgfpathmoveto{\pgfqpoint{0.669147in}{2.136525in}}%
\pgfpathlineto{\pgfqpoint{0.631314in}{2.147810in}}%
\pgfpathlineto{\pgfqpoint{0.572748in}{2.166216in}}%
\pgfpathlineto{\pgfqpoint{0.582091in}{2.197407in}}%
\pgfpathlineto{\pgfqpoint{0.593046in}{2.193971in}}%
\pgfpathlineto{\pgfqpoint{0.598094in}{2.209351in}}%
\pgfpathlineto{\pgfqpoint{0.597993in}{2.216816in}}%
\pgfpathlineto{\pgfqpoint{0.583086in}{2.221440in}}%
\pgfpathlineto{\pgfqpoint{0.586531in}{2.235034in}}%
\pgfpathlineto{\pgfqpoint{0.593017in}{2.254746in}}%
\pgfpathlineto{\pgfqpoint{0.595222in}{2.261545in}}%
\pgfpathlineto{\pgfqpoint{0.623636in}{2.252362in}}%
\pgfpathlineto{\pgfqpoint{0.635231in}{2.255252in}}%
\pgfpathlineto{\pgfqpoint{0.639739in}{2.262486in}}%
\pgfpathlineto{\pgfqpoint{0.648220in}{2.267247in}}%
\pgfpathlineto{\pgfqpoint{0.647635in}{2.272305in}}%
\pgfpathlineto{\pgfqpoint{0.652787in}{2.268601in}}%
\pgfpathlineto{\pgfqpoint{0.651158in}{2.248072in}}%
\pgfpathlineto{\pgfqpoint{0.656337in}{2.252428in}}%
\pgfpathlineto{\pgfqpoint{0.657168in}{2.258990in}}%
\pgfpathlineto{\pgfqpoint{0.664956in}{2.268322in}}%
\pgfpathlineto{\pgfqpoint{0.670631in}{2.272725in}}%
\pgfpathlineto{\pgfqpoint{0.666622in}{2.280032in}}%
\pgfpathlineto{\pgfqpoint{0.667712in}{2.286805in}}%
\pgfpathlineto{\pgfqpoint{0.660569in}{2.294722in}}%
\pgfpathlineto{\pgfqpoint{0.669465in}{2.296315in}}%
\pgfpathlineto{\pgfqpoint{0.662370in}{2.309255in}}%
\pgfpathlineto{\pgfqpoint{0.674084in}{2.320271in}}%
\pgfpathlineto{\pgfqpoint{0.672017in}{2.324335in}}%
\pgfpathlineto{\pgfqpoint{0.728782in}{2.306722in}}%
\pgfpathlineto{\pgfqpoint{0.752119in}{2.299760in}}%
\pgfpathlineto{\pgfqpoint{0.760555in}{2.298682in}}%
\pgfpathlineto{\pgfqpoint{0.757956in}{2.296770in}}%
\pgfpathlineto{\pgfqpoint{0.760304in}{2.288239in}}%
\pgfpathlineto{\pgfqpoint{0.752759in}{2.291869in}}%
\pgfpathlineto{\pgfqpoint{0.740255in}{2.288121in}}%
\pgfpathlineto{\pgfqpoint{0.736966in}{2.278404in}}%
\pgfpathlineto{\pgfqpoint{0.741543in}{2.265250in}}%
\pgfpathlineto{\pgfqpoint{0.738201in}{2.260046in}}%
\pgfpathlineto{\pgfqpoint{0.727318in}{2.257197in}}%
\pgfpathlineto{\pgfqpoint{0.721489in}{2.246797in}}%
\pgfpathlineto{\pgfqpoint{0.722977in}{2.230297in}}%
\pgfpathlineto{\pgfqpoint{0.718104in}{2.222638in}}%
\pgfpathlineto{\pgfqpoint{0.712161in}{2.222097in}}%
\pgfpathlineto{\pgfqpoint{0.701083in}{2.215781in}}%
\pgfpathlineto{\pgfqpoint{0.694418in}{2.209381in}}%
\pgfpathlineto{\pgfqpoint{0.694658in}{2.204301in}}%
\pgfpathlineto{\pgfqpoint{0.699138in}{2.201963in}}%
\pgfpathlineto{\pgfqpoint{0.697139in}{2.190297in}}%
\pgfpathlineto{\pgfqpoint{0.680298in}{2.173569in}}%
\pgfpathlineto{\pgfqpoint{0.682663in}{2.166183in}}%
\pgfpathlineto{\pgfqpoint{0.682016in}{2.153295in}}%
\pgfpathlineto{\pgfqpoint{0.675711in}{2.144078in}}%
\pgfpathlineto{\pgfqpoint{0.675514in}{2.134601in}}%
\pgfpathlineto{\pgfqpoint{0.669147in}{2.136525in}}%
\pgfpathclose%
\pgfusepath{fill}%
\end{pgfscope}%
\begin{pgfscope}%
\pgfpathrectangle{\pgfqpoint{0.100000in}{0.100000in}}{\pgfqpoint{3.608454in}{2.310000in}}%
\pgfusepath{clip}%
\pgfsetbuttcap%
\pgfsetmiterjoin%
\definecolor{currentfill}{rgb}{0.000000,0.541176,0.729412}%
\pgfsetfillcolor{currentfill}%
\pgfsetlinewidth{0.000000pt}%
\definecolor{currentstroke}{rgb}{0.000000,0.000000,0.000000}%
\pgfsetstrokecolor{currentstroke}%
\pgfsetstrokeopacity{0.000000}%
\pgfsetdash{}{0pt}%
\pgfpathmoveto{\pgfqpoint{2.615996in}{1.634560in}}%
\pgfpathlineto{\pgfqpoint{2.658513in}{1.638934in}}%
\pgfpathlineto{\pgfqpoint{2.655531in}{1.666251in}}%
\pgfpathlineto{\pgfqpoint{2.682759in}{1.669371in}}%
\pgfpathlineto{\pgfqpoint{2.685918in}{1.641892in}}%
\pgfpathlineto{\pgfqpoint{2.706426in}{1.644227in}}%
\pgfpathlineto{\pgfqpoint{2.710082in}{1.616628in}}%
\pgfpathlineto{\pgfqpoint{2.703247in}{1.616030in}}%
\pgfpathlineto{\pgfqpoint{2.648568in}{1.609914in}}%
\pgfpathlineto{\pgfqpoint{2.621651in}{1.607409in}}%
\pgfpathlineto{\pgfqpoint{2.620303in}{1.620947in}}%
\pgfpathlineto{\pgfqpoint{2.612010in}{1.620134in}}%
\pgfpathlineto{\pgfqpoint{2.615996in}{1.634560in}}%
\pgfpathclose%
\pgfusepath{fill}%
\end{pgfscope}%
\begin{pgfscope}%
\pgfpathrectangle{\pgfqpoint{0.100000in}{0.100000in}}{\pgfqpoint{3.608454in}{2.310000in}}%
\pgfusepath{clip}%
\pgfsetbuttcap%
\pgfsetmiterjoin%
\definecolor{currentfill}{rgb}{0.000000,0.654902,0.672549}%
\pgfsetfillcolor{currentfill}%
\pgfsetlinewidth{0.000000pt}%
\definecolor{currentstroke}{rgb}{0.000000,0.000000,0.000000}%
\pgfsetstrokecolor{currentstroke}%
\pgfsetstrokeopacity{0.000000}%
\pgfsetdash{}{0pt}%
\pgfpathmoveto{\pgfqpoint{1.303565in}{2.014897in}}%
\pgfpathlineto{\pgfqpoint{1.322819in}{2.011773in}}%
\pgfpathlineto{\pgfqpoint{1.316114in}{1.974564in}}%
\pgfpathlineto{\pgfqpoint{1.302428in}{1.976762in}}%
\pgfpathlineto{\pgfqpoint{1.268315in}{1.982523in}}%
\pgfpathlineto{\pgfqpoint{1.261011in}{1.981452in}}%
\pgfpathlineto{\pgfqpoint{1.225529in}{1.987742in}}%
\pgfpathlineto{\pgfqpoint{1.224264in}{1.998009in}}%
\pgfpathlineto{\pgfqpoint{1.228711in}{2.003790in}}%
\pgfpathlineto{\pgfqpoint{1.224493in}{2.008821in}}%
\pgfpathlineto{\pgfqpoint{1.219896in}{2.007840in}}%
\pgfpathlineto{\pgfqpoint{1.216917in}{2.019801in}}%
\pgfpathlineto{\pgfqpoint{1.213019in}{2.026776in}}%
\pgfpathlineto{\pgfqpoint{1.210034in}{2.040419in}}%
\pgfpathlineto{\pgfqpoint{1.203339in}{2.044520in}}%
\pgfpathlineto{\pgfqpoint{1.203686in}{2.050112in}}%
\pgfpathlineto{\pgfqpoint{1.210055in}{2.048940in}}%
\pgfpathlineto{\pgfqpoint{1.211322in}{2.055765in}}%
\pgfpathlineto{\pgfqpoint{1.231866in}{2.052930in}}%
\pgfpathlineto{\pgfqpoint{1.244581in}{2.048177in}}%
\pgfpathlineto{\pgfqpoint{1.251605in}{2.040078in}}%
\pgfpathlineto{\pgfqpoint{1.259993in}{2.025470in}}%
\pgfpathlineto{\pgfqpoint{1.264971in}{2.020151in}}%
\pgfpathlineto{\pgfqpoint{1.292244in}{2.021526in}}%
\pgfpathlineto{\pgfqpoint{1.299591in}{2.015508in}}%
\pgfpathlineto{\pgfqpoint{1.303565in}{2.014897in}}%
\pgfpathclose%
\pgfusepath{fill}%
\end{pgfscope}%
\begin{pgfscope}%
\pgfpathrectangle{\pgfqpoint{0.100000in}{0.100000in}}{\pgfqpoint{3.608454in}{2.310000in}}%
\pgfusepath{clip}%
\pgfsetbuttcap%
\pgfsetmiterjoin%
\definecolor{currentfill}{rgb}{0.000000,0.380392,0.809804}%
\pgfsetfillcolor{currentfill}%
\pgfsetlinewidth{0.000000pt}%
\definecolor{currentstroke}{rgb}{0.000000,0.000000,0.000000}%
\pgfsetstrokecolor{currentstroke}%
\pgfsetstrokeopacity{0.000000}%
\pgfsetdash{}{0pt}%
\pgfpathmoveto{\pgfqpoint{3.343167in}{1.599400in}}%
\pgfpathlineto{\pgfqpoint{3.345889in}{1.596696in}}%
\pgfpathlineto{\pgfqpoint{3.339300in}{1.581420in}}%
\pgfpathlineto{\pgfqpoint{3.327731in}{1.570641in}}%
\pgfpathlineto{\pgfqpoint{3.330022in}{1.567556in}}%
\pgfpathlineto{\pgfqpoint{3.313073in}{1.572820in}}%
\pgfpathlineto{\pgfqpoint{3.303159in}{1.582589in}}%
\pgfpathlineto{\pgfqpoint{3.297442in}{1.582891in}}%
\pgfpathlineto{\pgfqpoint{3.294732in}{1.592344in}}%
\pgfpathlineto{\pgfqpoint{3.286080in}{1.595797in}}%
\pgfpathlineto{\pgfqpoint{3.284662in}{1.609183in}}%
\pgfpathlineto{\pgfqpoint{3.288120in}{1.611032in}}%
\pgfpathlineto{\pgfqpoint{3.289903in}{1.618119in}}%
\pgfpathlineto{\pgfqpoint{3.284382in}{1.624557in}}%
\pgfpathlineto{\pgfqpoint{3.294577in}{1.644062in}}%
\pgfpathlineto{\pgfqpoint{3.295750in}{1.652950in}}%
\pgfpathlineto{\pgfqpoint{3.302353in}{1.660141in}}%
\pgfpathlineto{\pgfqpoint{3.332839in}{1.649737in}}%
\pgfpathlineto{\pgfqpoint{3.344220in}{1.667169in}}%
\pgfpathlineto{\pgfqpoint{3.346226in}{1.661401in}}%
\pgfpathlineto{\pgfqpoint{3.353043in}{1.654103in}}%
\pgfpathlineto{\pgfqpoint{3.355371in}{1.641928in}}%
\pgfpathlineto{\pgfqpoint{3.352075in}{1.614317in}}%
\pgfpathlineto{\pgfqpoint{3.344569in}{1.612129in}}%
\pgfpathlineto{\pgfqpoint{3.343167in}{1.599400in}}%
\pgfpathclose%
\pgfusepath{fill}%
\end{pgfscope}%
\begin{pgfscope}%
\pgfpathrectangle{\pgfqpoint{0.100000in}{0.100000in}}{\pgfqpoint{3.608454in}{2.310000in}}%
\pgfusepath{clip}%
\pgfsetbuttcap%
\pgfsetmiterjoin%
\definecolor{currentfill}{rgb}{0.000000,0.670588,0.664706}%
\pgfsetfillcolor{currentfill}%
\pgfsetlinewidth{0.000000pt}%
\definecolor{currentstroke}{rgb}{0.000000,0.000000,0.000000}%
\pgfsetstrokecolor{currentstroke}%
\pgfsetstrokeopacity{0.000000}%
\pgfsetdash{}{0pt}%
\pgfpathmoveto{\pgfqpoint{2.893588in}{0.882495in}}%
\pgfpathlineto{\pgfqpoint{2.910191in}{0.894634in}}%
\pgfpathlineto{\pgfqpoint{2.909397in}{0.902008in}}%
\pgfpathlineto{\pgfqpoint{2.903959in}{0.905713in}}%
\pgfpathlineto{\pgfqpoint{2.899980in}{0.912387in}}%
\pgfpathlineto{\pgfqpoint{2.899562in}{0.918887in}}%
\pgfpathlineto{\pgfqpoint{2.905128in}{0.926113in}}%
\pgfpathlineto{\pgfqpoint{2.910167in}{0.929143in}}%
\pgfpathlineto{\pgfqpoint{2.911845in}{0.934034in}}%
\pgfpathlineto{\pgfqpoint{2.918368in}{0.938239in}}%
\pgfpathlineto{\pgfqpoint{2.919634in}{0.933284in}}%
\pgfpathlineto{\pgfqpoint{2.925307in}{0.930408in}}%
\pgfpathlineto{\pgfqpoint{2.933286in}{0.922296in}}%
\pgfpathlineto{\pgfqpoint{2.937493in}{0.915146in}}%
\pgfpathlineto{\pgfqpoint{2.937410in}{0.908377in}}%
\pgfpathlineto{\pgfqpoint{2.944266in}{0.903605in}}%
\pgfpathlineto{\pgfqpoint{2.940759in}{0.897010in}}%
\pgfpathlineto{\pgfqpoint{2.943276in}{0.894232in}}%
\pgfpathlineto{\pgfqpoint{2.935640in}{0.887719in}}%
\pgfpathlineto{\pgfqpoint{2.932829in}{0.881990in}}%
\pgfpathlineto{\pgfqpoint{2.928863in}{0.867621in}}%
\pgfpathlineto{\pgfqpoint{2.930056in}{0.865279in}}%
\pgfpathlineto{\pgfqpoint{2.922187in}{0.854988in}}%
\pgfpathlineto{\pgfqpoint{2.913824in}{0.850086in}}%
\pgfpathlineto{\pgfqpoint{2.901077in}{0.870542in}}%
\pgfpathlineto{\pgfqpoint{2.893588in}{0.882495in}}%
\pgfpathclose%
\pgfusepath{fill}%
\end{pgfscope}%
\begin{pgfscope}%
\pgfpathrectangle{\pgfqpoint{0.100000in}{0.100000in}}{\pgfqpoint{3.608454in}{2.310000in}}%
\pgfusepath{clip}%
\pgfsetbuttcap%
\pgfsetmiterjoin%
\definecolor{currentfill}{rgb}{0.000000,0.509804,0.745098}%
\pgfsetfillcolor{currentfill}%
\pgfsetlinewidth{0.000000pt}%
\definecolor{currentstroke}{rgb}{0.000000,0.000000,0.000000}%
\pgfsetstrokecolor{currentstroke}%
\pgfsetstrokeopacity{0.000000}%
\pgfsetdash{}{0pt}%
\pgfpathmoveto{\pgfqpoint{2.515690in}{0.998385in}}%
\pgfpathlineto{\pgfqpoint{2.512313in}{0.997036in}}%
\pgfpathlineto{\pgfqpoint{2.512839in}{0.989031in}}%
\pgfpathlineto{\pgfqpoint{2.503832in}{0.987282in}}%
\pgfpathlineto{\pgfqpoint{2.483100in}{0.985946in}}%
\pgfpathlineto{\pgfqpoint{2.482530in}{0.995135in}}%
\pgfpathlineto{\pgfqpoint{2.478692in}{1.001827in}}%
\pgfpathlineto{\pgfqpoint{2.477335in}{1.022510in}}%
\pgfpathlineto{\pgfqpoint{2.473863in}{1.022295in}}%
\pgfpathlineto{\pgfqpoint{2.473129in}{1.034113in}}%
\pgfpathlineto{\pgfqpoint{2.483127in}{1.034794in}}%
\pgfpathlineto{\pgfqpoint{2.507414in}{1.036455in}}%
\pgfpathlineto{\pgfqpoint{2.508135in}{1.025655in}}%
\pgfpathlineto{\pgfqpoint{2.512709in}{1.025980in}}%
\pgfpathlineto{\pgfqpoint{2.515503in}{1.018069in}}%
\pgfpathlineto{\pgfqpoint{2.515690in}{0.998385in}}%
\pgfpathclose%
\pgfusepath{fill}%
\end{pgfscope}%
\begin{pgfscope}%
\pgfpathrectangle{\pgfqpoint{0.100000in}{0.100000in}}{\pgfqpoint{3.608454in}{2.310000in}}%
\pgfusepath{clip}%
\pgfsetbuttcap%
\pgfsetmiterjoin%
\definecolor{currentfill}{rgb}{0.000000,0.564706,0.717647}%
\pgfsetfillcolor{currentfill}%
\pgfsetlinewidth{0.000000pt}%
\definecolor{currentstroke}{rgb}{0.000000,0.000000,0.000000}%
\pgfsetstrokecolor{currentstroke}%
\pgfsetstrokeopacity{0.000000}%
\pgfsetdash{}{0pt}%
\pgfpathmoveto{\pgfqpoint{2.595636in}{0.988190in}}%
\pgfpathlineto{\pgfqpoint{2.553484in}{0.986177in}}%
\pgfpathlineto{\pgfqpoint{2.553769in}{0.997504in}}%
\pgfpathlineto{\pgfqpoint{2.554995in}{1.031701in}}%
\pgfpathlineto{\pgfqpoint{2.547499in}{1.040339in}}%
\pgfpathlineto{\pgfqpoint{2.561622in}{1.041248in}}%
\pgfpathlineto{\pgfqpoint{2.560066in}{1.064151in}}%
\pgfpathlineto{\pgfqpoint{2.556697in}{1.071568in}}%
\pgfpathlineto{\pgfqpoint{2.559562in}{1.077181in}}%
\pgfpathlineto{\pgfqpoint{2.575892in}{1.080309in}}%
\pgfpathlineto{\pgfqpoint{2.580566in}{1.079844in}}%
\pgfpathlineto{\pgfqpoint{2.585581in}{1.074349in}}%
\pgfpathlineto{\pgfqpoint{2.601253in}{1.080261in}}%
\pgfpathlineto{\pgfqpoint{2.607711in}{1.078660in}}%
\pgfpathlineto{\pgfqpoint{2.608880in}{1.074941in}}%
\pgfpathlineto{\pgfqpoint{2.610505in}{1.048649in}}%
\pgfpathlineto{\pgfqpoint{2.611745in}{1.044790in}}%
\pgfpathlineto{\pgfqpoint{2.612717in}{1.030317in}}%
\pgfpathlineto{\pgfqpoint{2.607222in}{1.025513in}}%
\pgfpathlineto{\pgfqpoint{2.599104in}{1.027892in}}%
\pgfpathlineto{\pgfqpoint{2.599703in}{1.020694in}}%
\pgfpathlineto{\pgfqpoint{2.593933in}{1.008935in}}%
\pgfpathlineto{\pgfqpoint{2.595636in}{0.988190in}}%
\pgfpathclose%
\pgfusepath{fill}%
\end{pgfscope}%
\begin{pgfscope}%
\pgfpathrectangle{\pgfqpoint{0.100000in}{0.100000in}}{\pgfqpoint{3.608454in}{2.310000in}}%
\pgfusepath{clip}%
\pgfsetbuttcap%
\pgfsetmiterjoin%
\definecolor{currentfill}{rgb}{0.000000,0.549020,0.725490}%
\pgfsetfillcolor{currentfill}%
\pgfsetlinewidth{0.000000pt}%
\definecolor{currentstroke}{rgb}{0.000000,0.000000,0.000000}%
\pgfsetstrokecolor{currentstroke}%
\pgfsetstrokeopacity{0.000000}%
\pgfsetdash{}{0pt}%
\pgfpathmoveto{\pgfqpoint{1.975277in}{0.471174in}}%
\pgfpathlineto{\pgfqpoint{1.976992in}{0.475338in}}%
\pgfpathlineto{\pgfqpoint{1.978118in}{0.480166in}}%
\pgfpathlineto{\pgfqpoint{1.984185in}{0.481426in}}%
\pgfpathlineto{\pgfqpoint{1.994367in}{0.489945in}}%
\pgfpathlineto{\pgfqpoint{1.996428in}{0.487179in}}%
\pgfpathlineto{\pgfqpoint{1.984727in}{0.479179in}}%
\pgfpathlineto{\pgfqpoint{1.975277in}{0.471174in}}%
\pgfpathclose%
\pgfusepath{fill}%
\end{pgfscope}%
\begin{pgfscope}%
\pgfpathrectangle{\pgfqpoint{0.100000in}{0.100000in}}{\pgfqpoint{3.608454in}{2.310000in}}%
\pgfusepath{clip}%
\pgfsetbuttcap%
\pgfsetmiterjoin%
\definecolor{currentfill}{rgb}{0.000000,0.549020,0.725490}%
\pgfsetfillcolor{currentfill}%
\pgfsetlinewidth{0.000000pt}%
\definecolor{currentstroke}{rgb}{0.000000,0.000000,0.000000}%
\pgfsetstrokecolor{currentstroke}%
\pgfsetstrokeopacity{0.000000}%
\pgfsetdash{}{0pt}%
\pgfpathmoveto{\pgfqpoint{2.013559in}{0.517174in}}%
\pgfpathlineto{\pgfqpoint{2.002833in}{0.511803in}}%
\pgfpathlineto{\pgfqpoint{1.996542in}{0.517508in}}%
\pgfpathlineto{\pgfqpoint{1.995156in}{0.524126in}}%
\pgfpathlineto{\pgfqpoint{1.989574in}{0.521754in}}%
\pgfpathlineto{\pgfqpoint{1.994165in}{0.512797in}}%
\pgfpathlineto{\pgfqpoint{2.008150in}{0.500988in}}%
\pgfpathlineto{\pgfqpoint{1.995428in}{0.494588in}}%
\pgfpathlineto{\pgfqpoint{1.992059in}{0.491469in}}%
\pgfpathlineto{\pgfqpoint{1.981043in}{0.498344in}}%
\pgfpathlineto{\pgfqpoint{1.973994in}{0.506774in}}%
\pgfpathlineto{\pgfqpoint{1.966594in}{0.506597in}}%
\pgfpathlineto{\pgfqpoint{1.962232in}{0.510712in}}%
\pgfpathlineto{\pgfqpoint{1.954355in}{0.511185in}}%
\pgfpathlineto{\pgfqpoint{1.938836in}{0.498314in}}%
\pgfpathlineto{\pgfqpoint{1.926522in}{0.503678in}}%
\pgfpathlineto{\pgfqpoint{1.926187in}{0.508771in}}%
\pgfpathlineto{\pgfqpoint{1.916523in}{0.511256in}}%
\pgfpathlineto{\pgfqpoint{1.911040in}{0.521565in}}%
\pgfpathlineto{\pgfqpoint{1.925742in}{0.532805in}}%
\pgfpathlineto{\pgfqpoint{1.913527in}{0.548723in}}%
\pgfpathlineto{\pgfqpoint{1.923720in}{0.556660in}}%
\pgfpathlineto{\pgfqpoint{1.950391in}{0.577980in}}%
\pgfpathlineto{\pgfqpoint{1.953286in}{0.593627in}}%
\pgfpathlineto{\pgfqpoint{1.957611in}{0.597327in}}%
\pgfpathlineto{\pgfqpoint{1.976342in}{0.597283in}}%
\pgfpathlineto{\pgfqpoint{1.981908in}{0.592727in}}%
\pgfpathlineto{\pgfqpoint{1.997763in}{0.573057in}}%
\pgfpathlineto{\pgfqpoint{1.992065in}{0.566132in}}%
\pgfpathlineto{\pgfqpoint{2.014894in}{0.542939in}}%
\pgfpathlineto{\pgfqpoint{2.013559in}{0.517174in}}%
\pgfpathclose%
\pgfusepath{fill}%
\end{pgfscope}%
\begin{pgfscope}%
\pgfpathrectangle{\pgfqpoint{0.100000in}{0.100000in}}{\pgfqpoint{3.608454in}{2.310000in}}%
\pgfusepath{clip}%
\pgfsetbuttcap%
\pgfsetmiterjoin%
\definecolor{currentfill}{rgb}{0.000000,0.882353,0.558824}%
\pgfsetfillcolor{currentfill}%
\pgfsetlinewidth{0.000000pt}%
\definecolor{currentstroke}{rgb}{0.000000,0.000000,0.000000}%
\pgfsetstrokecolor{currentstroke}%
\pgfsetstrokeopacity{0.000000}%
\pgfsetdash{}{0pt}%
\pgfpathmoveto{\pgfqpoint{2.362240in}{0.870230in}}%
\pgfpathlineto{\pgfqpoint{2.344274in}{0.869656in}}%
\pgfpathlineto{\pgfqpoint{2.342635in}{0.869578in}}%
\pgfpathlineto{\pgfqpoint{2.341651in}{0.899996in}}%
\pgfpathlineto{\pgfqpoint{2.306319in}{0.899456in}}%
\pgfpathlineto{\pgfqpoint{2.307139in}{0.891576in}}%
\pgfpathlineto{\pgfqpoint{2.297645in}{0.883786in}}%
\pgfpathlineto{\pgfqpoint{2.297141in}{0.880461in}}%
\pgfpathlineto{\pgfqpoint{2.286293in}{0.889498in}}%
\pgfpathlineto{\pgfqpoint{2.282169in}{0.889435in}}%
\pgfpathlineto{\pgfqpoint{2.280766in}{0.896956in}}%
\pgfpathlineto{\pgfqpoint{2.284812in}{0.906848in}}%
\pgfpathlineto{\pgfqpoint{2.282183in}{0.933447in}}%
\pgfpathlineto{\pgfqpoint{2.274161in}{0.944083in}}%
\pgfpathlineto{\pgfqpoint{2.272954in}{0.950762in}}%
\pgfpathlineto{\pgfqpoint{2.281822in}{0.951221in}}%
\pgfpathlineto{\pgfqpoint{2.288597in}{0.951635in}}%
\pgfpathlineto{\pgfqpoint{2.289533in}{0.964998in}}%
\pgfpathlineto{\pgfqpoint{2.289234in}{0.985811in}}%
\pgfpathlineto{\pgfqpoint{2.322083in}{0.986111in}}%
\pgfpathlineto{\pgfqpoint{2.322365in}{0.966382in}}%
\pgfpathlineto{\pgfqpoint{2.333604in}{0.966070in}}%
\pgfpathlineto{\pgfqpoint{2.335508in}{0.959654in}}%
\pgfpathlineto{\pgfqpoint{2.339665in}{0.959110in}}%
\pgfpathlineto{\pgfqpoint{2.341075in}{0.950676in}}%
\pgfpathlineto{\pgfqpoint{2.349626in}{0.945687in}}%
\pgfpathlineto{\pgfqpoint{2.361253in}{0.945877in}}%
\pgfpathlineto{\pgfqpoint{2.356734in}{0.953261in}}%
\pgfpathlineto{\pgfqpoint{2.361668in}{0.958818in}}%
\pgfpathlineto{\pgfqpoint{2.372346in}{0.959262in}}%
\pgfpathlineto{\pgfqpoint{2.377959in}{0.956142in}}%
\pgfpathlineto{\pgfqpoint{2.376340in}{0.951894in}}%
\pgfpathlineto{\pgfqpoint{2.364126in}{0.947506in}}%
\pgfpathlineto{\pgfqpoint{2.369367in}{0.943780in}}%
\pgfpathlineto{\pgfqpoint{2.365618in}{0.938943in}}%
\pgfpathlineto{\pgfqpoint{2.370618in}{0.931411in}}%
\pgfpathlineto{\pgfqpoint{2.361535in}{0.931873in}}%
\pgfpathlineto{\pgfqpoint{2.356747in}{0.922152in}}%
\pgfpathlineto{\pgfqpoint{2.362379in}{0.917175in}}%
\pgfpathlineto{\pgfqpoint{2.356517in}{0.914773in}}%
\pgfpathlineto{\pgfqpoint{2.362531in}{0.900120in}}%
\pgfpathlineto{\pgfqpoint{2.362887in}{0.893699in}}%
\pgfpathlineto{\pgfqpoint{2.366452in}{0.887645in}}%
\pgfpathlineto{\pgfqpoint{2.366771in}{0.881093in}}%
\pgfpathlineto{\pgfqpoint{2.362240in}{0.870230in}}%
\pgfpathclose%
\pgfusepath{fill}%
\end{pgfscope}%
\begin{pgfscope}%
\pgfpathrectangle{\pgfqpoint{0.100000in}{0.100000in}}{\pgfqpoint{3.608454in}{2.310000in}}%
\pgfusepath{clip}%
\pgfsetbuttcap%
\pgfsetmiterjoin%
\definecolor{currentfill}{rgb}{0.000000,0.686275,0.656863}%
\pgfsetfillcolor{currentfill}%
\pgfsetlinewidth{0.000000pt}%
\definecolor{currentstroke}{rgb}{0.000000,0.000000,0.000000}%
\pgfsetstrokecolor{currentstroke}%
\pgfsetstrokeopacity{0.000000}%
\pgfsetdash{}{0pt}%
\pgfpathmoveto{\pgfqpoint{3.332649in}{1.866641in}}%
\pgfpathlineto{\pgfqpoint{3.296075in}{1.852707in}}%
\pgfpathlineto{\pgfqpoint{3.295690in}{1.856839in}}%
\pgfpathlineto{\pgfqpoint{3.286142in}{1.855787in}}%
\pgfpathlineto{\pgfqpoint{3.277231in}{1.863034in}}%
\pgfpathlineto{\pgfqpoint{3.280930in}{1.867522in}}%
\pgfpathlineto{\pgfqpoint{3.276788in}{1.878840in}}%
\pgfpathlineto{\pgfqpoint{3.263105in}{1.873951in}}%
\pgfpathlineto{\pgfqpoint{3.262366in}{1.879593in}}%
\pgfpathlineto{\pgfqpoint{3.242124in}{1.938669in}}%
\pgfpathlineto{\pgfqpoint{3.236756in}{1.940890in}}%
\pgfpathlineto{\pgfqpoint{3.268366in}{1.947882in}}%
\pgfpathlineto{\pgfqpoint{3.312126in}{1.959898in}}%
\pgfpathlineto{\pgfqpoint{3.314088in}{1.952894in}}%
\pgfpathlineto{\pgfqpoint{3.313379in}{1.946070in}}%
\pgfpathlineto{\pgfqpoint{3.316446in}{1.944043in}}%
\pgfpathlineto{\pgfqpoint{3.316899in}{1.928969in}}%
\pgfpathlineto{\pgfqpoint{3.323703in}{1.920977in}}%
\pgfpathlineto{\pgfqpoint{3.324626in}{1.910739in}}%
\pgfpathlineto{\pgfqpoint{3.327709in}{1.903478in}}%
\pgfpathlineto{\pgfqpoint{3.324873in}{1.895982in}}%
\pgfpathlineto{\pgfqpoint{3.324985in}{1.884098in}}%
\pgfpathlineto{\pgfqpoint{3.332649in}{1.866641in}}%
\pgfpathclose%
\pgfusepath{fill}%
\end{pgfscope}%
\begin{pgfscope}%
\pgfpathrectangle{\pgfqpoint{0.100000in}{0.100000in}}{\pgfqpoint{3.608454in}{2.310000in}}%
\pgfusepath{clip}%
\pgfsetbuttcap%
\pgfsetmiterjoin%
\definecolor{currentfill}{rgb}{0.000000,0.556863,0.721569}%
\pgfsetfillcolor{currentfill}%
\pgfsetlinewidth{0.000000pt}%
\definecolor{currentstroke}{rgb}{0.000000,0.000000,0.000000}%
\pgfsetstrokecolor{currentstroke}%
\pgfsetstrokeopacity{0.000000}%
\pgfsetdash{}{0pt}%
\pgfpathmoveto{\pgfqpoint{1.678348in}{0.726376in}}%
\pgfpathlineto{\pgfqpoint{1.644176in}{0.728867in}}%
\pgfpathlineto{\pgfqpoint{1.650167in}{0.808902in}}%
\pgfpathlineto{\pgfqpoint{1.655619in}{0.808558in}}%
\pgfpathlineto{\pgfqpoint{1.658603in}{0.843265in}}%
\pgfpathlineto{\pgfqpoint{1.624223in}{0.845743in}}%
\pgfpathlineto{\pgfqpoint{1.626580in}{0.880308in}}%
\pgfpathlineto{\pgfqpoint{1.635380in}{0.879674in}}%
\pgfpathlineto{\pgfqpoint{1.695447in}{0.875598in}}%
\pgfpathlineto{\pgfqpoint{1.692995in}{0.840738in}}%
\pgfpathlineto{\pgfqpoint{1.690000in}{0.806092in}}%
\pgfpathlineto{\pgfqpoint{1.684601in}{0.806456in}}%
\pgfpathlineto{\pgfqpoint{1.681284in}{0.762092in}}%
\pgfpathlineto{\pgfqpoint{1.678348in}{0.726376in}}%
\pgfpathclose%
\pgfusepath{fill}%
\end{pgfscope}%
\begin{pgfscope}%
\pgfpathrectangle{\pgfqpoint{0.100000in}{0.100000in}}{\pgfqpoint{3.608454in}{2.310000in}}%
\pgfusepath{clip}%
\pgfsetbuttcap%
\pgfsetmiterjoin%
\definecolor{currentfill}{rgb}{0.000000,0.827451,0.586275}%
\pgfsetfillcolor{currentfill}%
\pgfsetlinewidth{0.000000pt}%
\definecolor{currentstroke}{rgb}{0.000000,0.000000,0.000000}%
\pgfsetstrokecolor{currentstroke}%
\pgfsetstrokeopacity{0.000000}%
\pgfsetdash{}{0pt}%
\pgfpathmoveto{\pgfqpoint{0.779307in}{0.362404in}}%
\pgfpathlineto{\pgfqpoint{0.779048in}{0.364878in}}%
\pgfpathlineto{\pgfqpoint{0.781778in}{0.364420in}}%
\pgfpathlineto{\pgfqpoint{0.784239in}{0.359816in}}%
\pgfpathlineto{\pgfqpoint{0.779307in}{0.362404in}}%
\pgfpathclose%
\pgfusepath{fill}%
\end{pgfscope}%
\begin{pgfscope}%
\pgfpathrectangle{\pgfqpoint{0.100000in}{0.100000in}}{\pgfqpoint{3.608454in}{2.310000in}}%
\pgfusepath{clip}%
\pgfsetbuttcap%
\pgfsetmiterjoin%
\definecolor{currentfill}{rgb}{0.000000,0.827451,0.586275}%
\pgfsetfillcolor{currentfill}%
\pgfsetlinewidth{0.000000pt}%
\definecolor{currentstroke}{rgb}{0.000000,0.000000,0.000000}%
\pgfsetstrokecolor{currentstroke}%
\pgfsetstrokeopacity{0.000000}%
\pgfsetdash{}{0pt}%
\pgfpathmoveto{\pgfqpoint{0.771381in}{0.365345in}}%
\pgfpathlineto{\pgfqpoint{0.773191in}{0.367159in}}%
\pgfpathlineto{\pgfqpoint{0.777403in}{0.367463in}}%
\pgfpathlineto{\pgfqpoint{0.776562in}{0.365684in}}%
\pgfpathlineto{\pgfqpoint{0.771381in}{0.365345in}}%
\pgfpathclose%
\pgfusepath{fill}%
\end{pgfscope}%
\begin{pgfscope}%
\pgfpathrectangle{\pgfqpoint{0.100000in}{0.100000in}}{\pgfqpoint{3.608454in}{2.310000in}}%
\pgfusepath{clip}%
\pgfsetbuttcap%
\pgfsetmiterjoin%
\definecolor{currentfill}{rgb}{0.000000,0.827451,0.586275}%
\pgfsetfillcolor{currentfill}%
\pgfsetlinewidth{0.000000pt}%
\definecolor{currentstroke}{rgb}{0.000000,0.000000,0.000000}%
\pgfsetstrokecolor{currentstroke}%
\pgfsetstrokeopacity{0.000000}%
\pgfsetdash{}{0pt}%
\pgfpathmoveto{\pgfqpoint{0.764308in}{0.397983in}}%
\pgfpathlineto{\pgfqpoint{0.760327in}{0.401259in}}%
\pgfpathlineto{\pgfqpoint{0.761612in}{0.402864in}}%
\pgfpathlineto{\pgfqpoint{0.765003in}{0.403792in}}%
\pgfpathlineto{\pgfqpoint{0.766993in}{0.403276in}}%
\pgfpathlineto{\pgfqpoint{0.768215in}{0.404903in}}%
\pgfpathlineto{\pgfqpoint{0.771380in}{0.405564in}}%
\pgfpathlineto{\pgfqpoint{0.773253in}{0.404050in}}%
\pgfpathlineto{\pgfqpoint{0.777652in}{0.403284in}}%
\pgfpathlineto{\pgfqpoint{0.777113in}{0.402156in}}%
\pgfpathlineto{\pgfqpoint{0.772733in}{0.402967in}}%
\pgfpathlineto{\pgfqpoint{0.767839in}{0.402629in}}%
\pgfpathlineto{\pgfqpoint{0.767479in}{0.399727in}}%
\pgfpathlineto{\pgfqpoint{0.764308in}{0.397983in}}%
\pgfpathclose%
\pgfusepath{fill}%
\end{pgfscope}%
\begin{pgfscope}%
\pgfpathrectangle{\pgfqpoint{0.100000in}{0.100000in}}{\pgfqpoint{3.608454in}{2.310000in}}%
\pgfusepath{clip}%
\pgfsetbuttcap%
\pgfsetmiterjoin%
\definecolor{currentfill}{rgb}{0.000000,0.827451,0.586275}%
\pgfsetfillcolor{currentfill}%
\pgfsetlinewidth{0.000000pt}%
\definecolor{currentstroke}{rgb}{0.000000,0.000000,0.000000}%
\pgfsetstrokecolor{currentstroke}%
\pgfsetstrokeopacity{0.000000}%
\pgfsetdash{}{0pt}%
\pgfpathmoveto{\pgfqpoint{0.833159in}{0.390083in}}%
\pgfpathlineto{\pgfqpoint{0.831710in}{0.387820in}}%
\pgfpathlineto{\pgfqpoint{0.834237in}{0.387444in}}%
\pgfpathlineto{\pgfqpoint{0.834627in}{0.385754in}}%
\pgfpathlineto{\pgfqpoint{0.836745in}{0.385201in}}%
\pgfpathlineto{\pgfqpoint{0.837269in}{0.381931in}}%
\pgfpathlineto{\pgfqpoint{0.835377in}{0.380107in}}%
\pgfpathlineto{\pgfqpoint{0.832974in}{0.379753in}}%
\pgfpathlineto{\pgfqpoint{0.827040in}{0.383181in}}%
\pgfpathlineto{\pgfqpoint{0.822060in}{0.382223in}}%
\pgfpathlineto{\pgfqpoint{0.818304in}{0.385966in}}%
\pgfpathlineto{\pgfqpoint{0.817822in}{0.388703in}}%
\pgfpathlineto{\pgfqpoint{0.819792in}{0.390087in}}%
\pgfpathlineto{\pgfqpoint{0.823330in}{0.389436in}}%
\pgfpathlineto{\pgfqpoint{0.823863in}{0.391532in}}%
\pgfpathlineto{\pgfqpoint{0.826549in}{0.387455in}}%
\pgfpathlineto{\pgfqpoint{0.828702in}{0.391954in}}%
\pgfpathlineto{\pgfqpoint{0.832601in}{0.391664in}}%
\pgfpathlineto{\pgfqpoint{0.834521in}{0.394379in}}%
\pgfpathlineto{\pgfqpoint{0.835969in}{0.392848in}}%
\pgfpathlineto{\pgfqpoint{0.836007in}{0.391059in}}%
\pgfpathlineto{\pgfqpoint{0.833159in}{0.390083in}}%
\pgfpathclose%
\pgfusepath{fill}%
\end{pgfscope}%
\begin{pgfscope}%
\pgfpathrectangle{\pgfqpoint{0.100000in}{0.100000in}}{\pgfqpoint{3.608454in}{2.310000in}}%
\pgfusepath{clip}%
\pgfsetbuttcap%
\pgfsetmiterjoin%
\definecolor{currentfill}{rgb}{0.000000,0.827451,0.586275}%
\pgfsetfillcolor{currentfill}%
\pgfsetlinewidth{0.000000pt}%
\definecolor{currentstroke}{rgb}{0.000000,0.000000,0.000000}%
\pgfsetstrokecolor{currentstroke}%
\pgfsetstrokeopacity{0.000000}%
\pgfsetdash{}{0pt}%
\pgfpathmoveto{\pgfqpoint{0.824220in}{0.377345in}}%
\pgfpathlineto{\pgfqpoint{0.824880in}{0.374202in}}%
\pgfpathlineto{\pgfqpoint{0.822730in}{0.374227in}}%
\pgfpathlineto{\pgfqpoint{0.819725in}{0.370152in}}%
\pgfpathlineto{\pgfqpoint{0.823688in}{0.368053in}}%
\pgfpathlineto{\pgfqpoint{0.818723in}{0.365758in}}%
\pgfpathlineto{\pgfqpoint{0.816629in}{0.366618in}}%
\pgfpathlineto{\pgfqpoint{0.815348in}{0.369220in}}%
\pgfpathlineto{\pgfqpoint{0.813787in}{0.368393in}}%
\pgfpathlineto{\pgfqpoint{0.813447in}{0.366368in}}%
\pgfpathlineto{\pgfqpoint{0.811430in}{0.365767in}}%
\pgfpathlineto{\pgfqpoint{0.803663in}{0.368137in}}%
\pgfpathlineto{\pgfqpoint{0.804036in}{0.365808in}}%
\pgfpathlineto{\pgfqpoint{0.801470in}{0.365130in}}%
\pgfpathlineto{\pgfqpoint{0.797979in}{0.368165in}}%
\pgfpathlineto{\pgfqpoint{0.796078in}{0.368564in}}%
\pgfpathlineto{\pgfqpoint{0.791325in}{0.365921in}}%
\pgfpathlineto{\pgfqpoint{0.786527in}{0.365586in}}%
\pgfpathlineto{\pgfqpoint{0.787213in}{0.367927in}}%
\pgfpathlineto{\pgfqpoint{0.792491in}{0.369127in}}%
\pgfpathlineto{\pgfqpoint{0.792763in}{0.372545in}}%
\pgfpathlineto{\pgfqpoint{0.789745in}{0.372900in}}%
\pgfpathlineto{\pgfqpoint{0.786003in}{0.370933in}}%
\pgfpathlineto{\pgfqpoint{0.784989in}{0.376973in}}%
\pgfpathlineto{\pgfqpoint{0.787640in}{0.381629in}}%
\pgfpathlineto{\pgfqpoint{0.786375in}{0.384997in}}%
\pgfpathlineto{\pgfqpoint{0.792271in}{0.388607in}}%
\pgfpathlineto{\pgfqpoint{0.796911in}{0.388654in}}%
\pgfpathlineto{\pgfqpoint{0.799734in}{0.387836in}}%
\pgfpathlineto{\pgfqpoint{0.801573in}{0.385770in}}%
\pgfpathlineto{\pgfqpoint{0.799429in}{0.380434in}}%
\pgfpathlineto{\pgfqpoint{0.803009in}{0.382399in}}%
\pgfpathlineto{\pgfqpoint{0.804618in}{0.387969in}}%
\pgfpathlineto{\pgfqpoint{0.807031in}{0.389222in}}%
\pgfpathlineto{\pgfqpoint{0.809285in}{0.388827in}}%
\pgfpathlineto{\pgfqpoint{0.809591in}{0.384951in}}%
\pgfpathlineto{\pgfqpoint{0.811916in}{0.384419in}}%
\pgfpathlineto{\pgfqpoint{0.812085in}{0.387811in}}%
\pgfpathlineto{\pgfqpoint{0.816863in}{0.385495in}}%
\pgfpathlineto{\pgfqpoint{0.818200in}{0.383011in}}%
\pgfpathlineto{\pgfqpoint{0.820527in}{0.380666in}}%
\pgfpathlineto{\pgfqpoint{0.818276in}{0.378662in}}%
\pgfpathlineto{\pgfqpoint{0.824441in}{0.378263in}}%
\pgfpathlineto{\pgfqpoint{0.824220in}{0.377345in}}%
\pgfpathclose%
\pgfusepath{fill}%
\end{pgfscope}%
\begin{pgfscope}%
\pgfpathrectangle{\pgfqpoint{0.100000in}{0.100000in}}{\pgfqpoint{3.608454in}{2.310000in}}%
\pgfusepath{clip}%
\pgfsetbuttcap%
\pgfsetmiterjoin%
\definecolor{currentfill}{rgb}{0.000000,0.827451,0.586275}%
\pgfsetfillcolor{currentfill}%
\pgfsetlinewidth{0.000000pt}%
\definecolor{currentstroke}{rgb}{0.000000,0.000000,0.000000}%
\pgfsetstrokecolor{currentstroke}%
\pgfsetstrokeopacity{0.000000}%
\pgfsetdash{}{0pt}%
\pgfpathmoveto{\pgfqpoint{0.831242in}{0.406233in}}%
\pgfpathlineto{\pgfqpoint{0.829444in}{0.406598in}}%
\pgfpathlineto{\pgfqpoint{0.826973in}{0.404781in}}%
\pgfpathlineto{\pgfqpoint{0.821584in}{0.404692in}}%
\pgfpathlineto{\pgfqpoint{0.820126in}{0.405792in}}%
\pgfpathlineto{\pgfqpoint{0.817124in}{0.404505in}}%
\pgfpathlineto{\pgfqpoint{0.814345in}{0.404191in}}%
\pgfpathlineto{\pgfqpoint{0.812992in}{0.402013in}}%
\pgfpathlineto{\pgfqpoint{0.811166in}{0.403256in}}%
\pgfpathlineto{\pgfqpoint{0.810441in}{0.400723in}}%
\pgfpathlineto{\pgfqpoint{0.803324in}{0.399430in}}%
\pgfpathlineto{\pgfqpoint{0.800398in}{0.399952in}}%
\pgfpathlineto{\pgfqpoint{0.797438in}{0.402092in}}%
\pgfpathlineto{\pgfqpoint{0.794014in}{0.402817in}}%
\pgfpathlineto{\pgfqpoint{0.793204in}{0.401308in}}%
\pgfpathlineto{\pgfqpoint{0.788959in}{0.401438in}}%
\pgfpathlineto{\pgfqpoint{0.786334in}{0.400255in}}%
\pgfpathlineto{\pgfqpoint{0.785696in}{0.403237in}}%
\pgfpathlineto{\pgfqpoint{0.783361in}{0.401415in}}%
\pgfpathlineto{\pgfqpoint{0.780693in}{0.402239in}}%
\pgfpathlineto{\pgfqpoint{0.779576in}{0.400441in}}%
\pgfpathlineto{\pgfqpoint{0.777277in}{0.402089in}}%
\pgfpathlineto{\pgfqpoint{0.781062in}{0.403097in}}%
\pgfpathlineto{\pgfqpoint{0.784195in}{0.405252in}}%
\pgfpathlineto{\pgfqpoint{0.786031in}{0.403802in}}%
\pgfpathlineto{\pgfqpoint{0.789736in}{0.403394in}}%
\pgfpathlineto{\pgfqpoint{0.794932in}{0.410045in}}%
\pgfpathlineto{\pgfqpoint{0.797005in}{0.409992in}}%
\pgfpathlineto{\pgfqpoint{0.798949in}{0.408471in}}%
\pgfpathlineto{\pgfqpoint{0.800501in}{0.409249in}}%
\pgfpathlineto{\pgfqpoint{0.804050in}{0.407001in}}%
\pgfpathlineto{\pgfqpoint{0.808196in}{0.405860in}}%
\pgfpathlineto{\pgfqpoint{0.809601in}{0.406866in}}%
\pgfpathlineto{\pgfqpoint{0.811859in}{0.406633in}}%
\pgfpathlineto{\pgfqpoint{0.815079in}{0.410909in}}%
\pgfpathlineto{\pgfqpoint{0.820929in}{0.406477in}}%
\pgfpathlineto{\pgfqpoint{0.822419in}{0.408486in}}%
\pgfpathlineto{\pgfqpoint{0.826771in}{0.405229in}}%
\pgfpathlineto{\pgfqpoint{0.828822in}{0.408037in}}%
\pgfpathlineto{\pgfqpoint{0.831242in}{0.406233in}}%
\pgfpathclose%
\pgfusepath{fill}%
\end{pgfscope}%
\begin{pgfscope}%
\pgfpathrectangle{\pgfqpoint{0.100000in}{0.100000in}}{\pgfqpoint{3.608454in}{2.310000in}}%
\pgfusepath{clip}%
\pgfsetbuttcap%
\pgfsetmiterjoin%
\definecolor{currentfill}{rgb}{0.000000,0.454902,0.772549}%
\pgfsetfillcolor{currentfill}%
\pgfsetlinewidth{0.000000pt}%
\definecolor{currentstroke}{rgb}{0.000000,0.000000,0.000000}%
\pgfsetstrokecolor{currentstroke}%
\pgfsetstrokeopacity{0.000000}%
\pgfsetdash{}{0pt}%
\pgfpathmoveto{\pgfqpoint{1.204580in}{1.868127in}}%
\pgfpathlineto{\pgfqpoint{1.202252in}{1.853038in}}%
\pgfpathlineto{\pgfqpoint{1.198018in}{1.855881in}}%
\pgfpathlineto{\pgfqpoint{1.191695in}{1.873351in}}%
\pgfpathlineto{\pgfqpoint{1.187877in}{1.878039in}}%
\pgfpathlineto{\pgfqpoint{1.181983in}{1.875682in}}%
\pgfpathlineto{\pgfqpoint{1.164838in}{1.878825in}}%
\pgfpathlineto{\pgfqpoint{1.165886in}{1.884470in}}%
\pgfpathlineto{\pgfqpoint{1.150185in}{1.887641in}}%
\pgfpathlineto{\pgfqpoint{1.145219in}{1.892062in}}%
\pgfpathlineto{\pgfqpoint{1.146489in}{1.905873in}}%
\pgfpathlineto{\pgfqpoint{1.138589in}{1.907407in}}%
\pgfpathlineto{\pgfqpoint{1.139942in}{1.914263in}}%
\pgfpathlineto{\pgfqpoint{1.133161in}{1.915579in}}%
\pgfpathlineto{\pgfqpoint{1.137082in}{1.935988in}}%
\pgfpathlineto{\pgfqpoint{1.135886in}{1.943764in}}%
\pgfpathlineto{\pgfqpoint{1.131774in}{1.943515in}}%
\pgfpathlineto{\pgfqpoint{1.126427in}{1.948422in}}%
\pgfpathlineto{\pgfqpoint{1.128763in}{1.959930in}}%
\pgfpathlineto{\pgfqpoint{1.137125in}{1.963079in}}%
\pgfpathlineto{\pgfqpoint{1.140024in}{1.967674in}}%
\pgfpathlineto{\pgfqpoint{1.157736in}{1.964285in}}%
\pgfpathlineto{\pgfqpoint{1.165857in}{1.971182in}}%
\pgfpathlineto{\pgfqpoint{1.173618in}{1.970061in}}%
\pgfpathlineto{\pgfqpoint{1.176413in}{1.966592in}}%
\pgfpathlineto{\pgfqpoint{1.185357in}{1.963304in}}%
\pgfpathlineto{\pgfqpoint{1.192609in}{1.969852in}}%
\pgfpathlineto{\pgfqpoint{1.199884in}{1.972842in}}%
\pgfpathlineto{\pgfqpoint{1.203995in}{1.980934in}}%
\pgfpathlineto{\pgfqpoint{1.208958in}{1.985065in}}%
\pgfpathlineto{\pgfqpoint{1.210052in}{1.990143in}}%
\pgfpathlineto{\pgfqpoint{1.225529in}{1.987742in}}%
\pgfpathlineto{\pgfqpoint{1.261011in}{1.981452in}}%
\pgfpathlineto{\pgfqpoint{1.268315in}{1.982523in}}%
\pgfpathlineto{\pgfqpoint{1.302428in}{1.976762in}}%
\pgfpathlineto{\pgfqpoint{1.301322in}{1.969905in}}%
\pgfpathlineto{\pgfqpoint{1.302894in}{1.962642in}}%
\pgfpathlineto{\pgfqpoint{1.313126in}{1.960922in}}%
\pgfpathlineto{\pgfqpoint{1.311739in}{1.954347in}}%
\pgfpathlineto{\pgfqpoint{1.307165in}{1.954995in}}%
\pgfpathlineto{\pgfqpoint{1.301786in}{1.941660in}}%
\pgfpathlineto{\pgfqpoint{1.299700in}{1.927978in}}%
\pgfpathlineto{\pgfqpoint{1.292921in}{1.929061in}}%
\pgfpathlineto{\pgfqpoint{1.278245in}{1.924515in}}%
\pgfpathlineto{\pgfqpoint{1.275970in}{1.910859in}}%
\pgfpathlineto{\pgfqpoint{1.268744in}{1.912078in}}%
\pgfpathlineto{\pgfqpoint{1.266400in}{1.898228in}}%
\pgfpathlineto{\pgfqpoint{1.281086in}{1.895779in}}%
\pgfpathlineto{\pgfqpoint{1.278870in}{1.882470in}}%
\pgfpathlineto{\pgfqpoint{1.228507in}{1.890208in}}%
\pgfpathlineto{\pgfqpoint{1.209266in}{1.894331in}}%
\pgfpathlineto{\pgfqpoint{1.204580in}{1.868127in}}%
\pgfpathclose%
\pgfusepath{fill}%
\end{pgfscope}%
\begin{pgfscope}%
\pgfpathrectangle{\pgfqpoint{0.100000in}{0.100000in}}{\pgfqpoint{3.608454in}{2.310000in}}%
\pgfusepath{clip}%
\pgfsetbuttcap%
\pgfsetmiterjoin%
\definecolor{currentfill}{rgb}{0.000000,0.509804,0.745098}%
\pgfsetfillcolor{currentfill}%
\pgfsetlinewidth{0.000000pt}%
\definecolor{currentstroke}{rgb}{0.000000,0.000000,0.000000}%
\pgfsetstrokecolor{currentstroke}%
\pgfsetstrokeopacity{0.000000}%
\pgfsetdash{}{0pt}%
\pgfpathmoveto{\pgfqpoint{1.908588in}{1.422459in}}%
\pgfpathlineto{\pgfqpoint{1.894644in}{1.422961in}}%
\pgfpathlineto{\pgfqpoint{1.881222in}{1.423490in}}%
\pgfpathlineto{\pgfqpoint{1.883772in}{1.478524in}}%
\pgfpathlineto{\pgfqpoint{1.937428in}{1.476563in}}%
\pgfpathlineto{\pgfqpoint{1.936670in}{1.449037in}}%
\pgfpathlineto{\pgfqpoint{1.909610in}{1.449945in}}%
\pgfpathlineto{\pgfqpoint{1.908588in}{1.422459in}}%
\pgfpathclose%
\pgfusepath{fill}%
\end{pgfscope}%
\begin{pgfscope}%
\pgfpathrectangle{\pgfqpoint{0.100000in}{0.100000in}}{\pgfqpoint{3.608454in}{2.310000in}}%
\pgfusepath{clip}%
\pgfsetbuttcap%
\pgfsetmiterjoin%
\definecolor{currentfill}{rgb}{0.000000,0.576471,0.711765}%
\pgfsetfillcolor{currentfill}%
\pgfsetlinewidth{0.000000pt}%
\definecolor{currentstroke}{rgb}{0.000000,0.000000,0.000000}%
\pgfsetstrokecolor{currentstroke}%
\pgfsetstrokeopacity{0.000000}%
\pgfsetdash{}{0pt}%
\pgfpathmoveto{\pgfqpoint{0.695319in}{1.898980in}}%
\pgfpathlineto{\pgfqpoint{0.668725in}{1.906595in}}%
\pgfpathlineto{\pgfqpoint{0.616244in}{1.922404in}}%
\pgfpathlineto{\pgfqpoint{0.579411in}{1.933417in}}%
\pgfpathlineto{\pgfqpoint{0.585151in}{1.943983in}}%
\pgfpathlineto{\pgfqpoint{0.586510in}{1.951287in}}%
\pgfpathlineto{\pgfqpoint{0.595256in}{1.957298in}}%
\pgfpathlineto{\pgfqpoint{0.603007in}{1.965607in}}%
\pgfpathlineto{\pgfqpoint{0.605729in}{1.977984in}}%
\pgfpathlineto{\pgfqpoint{0.605387in}{1.987819in}}%
\pgfpathlineto{\pgfqpoint{0.608267in}{1.997300in}}%
\pgfpathlineto{\pgfqpoint{0.613506in}{2.004223in}}%
\pgfpathlineto{\pgfqpoint{0.616042in}{2.014692in}}%
\pgfpathlineto{\pgfqpoint{0.620648in}{2.021281in}}%
\pgfpathlineto{\pgfqpoint{0.656754in}{2.010285in}}%
\pgfpathlineto{\pgfqpoint{0.695564in}{1.998853in}}%
\pgfpathlineto{\pgfqpoint{0.693514in}{1.994463in}}%
\pgfpathlineto{\pgfqpoint{0.686387in}{1.969750in}}%
\pgfpathlineto{\pgfqpoint{0.706253in}{1.964177in}}%
\pgfpathlineto{\pgfqpoint{0.711901in}{1.958542in}}%
\pgfpathlineto{\pgfqpoint{0.710132in}{1.952205in}}%
\pgfpathlineto{\pgfqpoint{0.723444in}{1.948508in}}%
\pgfpathlineto{\pgfqpoint{0.715935in}{1.921940in}}%
\pgfpathlineto{\pgfqpoint{0.709441in}{1.923753in}}%
\pgfpathlineto{\pgfqpoint{0.703763in}{1.903845in}}%
\pgfpathlineto{\pgfqpoint{0.697108in}{1.905730in}}%
\pgfpathlineto{\pgfqpoint{0.695319in}{1.898980in}}%
\pgfpathclose%
\pgfusepath{fill}%
\end{pgfscope}%
\begin{pgfscope}%
\pgfpathrectangle{\pgfqpoint{0.100000in}{0.100000in}}{\pgfqpoint{3.608454in}{2.310000in}}%
\pgfusepath{clip}%
\pgfsetbuttcap%
\pgfsetmiterjoin%
\definecolor{currentfill}{rgb}{0.000000,0.423529,0.788235}%
\pgfsetfillcolor{currentfill}%
\pgfsetlinewidth{0.000000pt}%
\definecolor{currentstroke}{rgb}{0.000000,0.000000,0.000000}%
\pgfsetstrokecolor{currentstroke}%
\pgfsetstrokeopacity{0.000000}%
\pgfsetdash{}{0pt}%
\pgfpathmoveto{\pgfqpoint{1.716323in}{1.398352in}}%
\pgfpathlineto{\pgfqpoint{1.677687in}{1.401208in}}%
\pgfpathlineto{\pgfqpoint{1.677720in}{1.401672in}}%
\pgfpathlineto{\pgfqpoint{1.682362in}{1.462771in}}%
\pgfpathlineto{\pgfqpoint{1.726124in}{1.459646in}}%
\pgfpathlineto{\pgfqpoint{1.724173in}{1.432221in}}%
\pgfpathlineto{\pgfqpoint{1.718916in}{1.432564in}}%
\pgfpathlineto{\pgfqpoint{1.716323in}{1.398352in}}%
\pgfpathclose%
\pgfusepath{fill}%
\end{pgfscope}%
\begin{pgfscope}%
\pgfpathrectangle{\pgfqpoint{0.100000in}{0.100000in}}{\pgfqpoint{3.608454in}{2.310000in}}%
\pgfusepath{clip}%
\pgfsetbuttcap%
\pgfsetmiterjoin%
\definecolor{currentfill}{rgb}{0.000000,0.815686,0.592157}%
\pgfsetfillcolor{currentfill}%
\pgfsetlinewidth{0.000000pt}%
\definecolor{currentstroke}{rgb}{0.000000,0.000000,0.000000}%
\pgfsetstrokecolor{currentstroke}%
\pgfsetstrokeopacity{0.000000}%
\pgfsetdash{}{0pt}%
\pgfpathmoveto{\pgfqpoint{0.908989in}{1.065097in}}%
\pgfpathlineto{\pgfqpoint{0.914070in}{1.089859in}}%
\pgfpathlineto{\pgfqpoint{0.933245in}{1.183825in}}%
\pgfpathlineto{\pgfqpoint{0.937616in}{1.205156in}}%
\pgfpathlineto{\pgfqpoint{0.941162in}{1.217180in}}%
\pgfpathlineto{\pgfqpoint{0.940117in}{1.223821in}}%
\pgfpathlineto{\pgfqpoint{0.943617in}{1.227923in}}%
\pgfpathlineto{\pgfqpoint{0.949023in}{1.226105in}}%
\pgfpathlineto{\pgfqpoint{0.951426in}{1.230336in}}%
\pgfpathlineto{\pgfqpoint{0.961661in}{1.233095in}}%
\pgfpathlineto{\pgfqpoint{0.978736in}{1.235692in}}%
\pgfpathlineto{\pgfqpoint{0.986930in}{1.238736in}}%
\pgfpathlineto{\pgfqpoint{0.991491in}{1.243095in}}%
\pgfpathlineto{\pgfqpoint{0.991172in}{1.251855in}}%
\pgfpathlineto{\pgfqpoint{0.994707in}{1.264046in}}%
\pgfpathlineto{\pgfqpoint{1.005762in}{1.288169in}}%
\pgfpathlineto{\pgfqpoint{0.983512in}{1.292540in}}%
\pgfpathlineto{\pgfqpoint{0.991804in}{1.334673in}}%
\pgfpathlineto{\pgfqpoint{1.005205in}{1.331999in}}%
\pgfpathlineto{\pgfqpoint{1.024916in}{1.327610in}}%
\pgfpathlineto{\pgfqpoint{1.104844in}{1.313028in}}%
\pgfpathlineto{\pgfqpoint{1.130720in}{1.308550in}}%
\pgfpathlineto{\pgfqpoint{1.122123in}{1.294263in}}%
\pgfpathlineto{\pgfqpoint{1.113627in}{1.291412in}}%
\pgfpathlineto{\pgfqpoint{1.109948in}{1.282813in}}%
\pgfpathlineto{\pgfqpoint{1.099188in}{1.278642in}}%
\pgfpathlineto{\pgfqpoint{1.091741in}{1.280236in}}%
\pgfpathlineto{\pgfqpoint{1.087638in}{1.275829in}}%
\pgfpathlineto{\pgfqpoint{1.078577in}{1.274542in}}%
\pgfpathlineto{\pgfqpoint{1.116943in}{1.267777in}}%
\pgfpathlineto{\pgfqpoint{1.079528in}{1.054057in}}%
\pgfpathlineto{\pgfqpoint{1.073305in}{1.055337in}}%
\pgfpathlineto{\pgfqpoint{1.063896in}{1.064728in}}%
\pgfpathlineto{\pgfqpoint{1.055533in}{1.068101in}}%
\pgfpathlineto{\pgfqpoint{1.049066in}{1.073939in}}%
\pgfpathlineto{\pgfqpoint{1.039948in}{1.073968in}}%
\pgfpathlineto{\pgfqpoint{1.031414in}{1.079449in}}%
\pgfpathlineto{\pgfqpoint{1.022683in}{1.074223in}}%
\pgfpathlineto{\pgfqpoint{1.015879in}{1.056517in}}%
\pgfpathlineto{\pgfqpoint{1.031005in}{1.053711in}}%
\pgfpathlineto{\pgfqpoint{1.027746in}{1.042105in}}%
\pgfpathlineto{\pgfqpoint{1.012835in}{1.044864in}}%
\pgfpathlineto{\pgfqpoint{0.985165in}{1.053983in}}%
\pgfpathlineto{\pgfqpoint{0.975479in}{1.042443in}}%
\pgfpathlineto{\pgfqpoint{0.947034in}{1.057499in}}%
\pgfpathlineto{\pgfqpoint{0.908989in}{1.065097in}}%
\pgfpathclose%
\pgfusepath{fill}%
\end{pgfscope}%
\begin{pgfscope}%
\pgfpathrectangle{\pgfqpoint{0.100000in}{0.100000in}}{\pgfqpoint{3.608454in}{2.310000in}}%
\pgfusepath{clip}%
\pgfsetbuttcap%
\pgfsetmiterjoin%
\definecolor{currentfill}{rgb}{0.000000,0.219608,0.890196}%
\pgfsetfillcolor{currentfill}%
\pgfsetlinewidth{0.000000pt}%
\definecolor{currentstroke}{rgb}{0.000000,0.000000,0.000000}%
\pgfsetstrokecolor{currentstroke}%
\pgfsetstrokeopacity{0.000000}%
\pgfsetdash{}{0pt}%
\pgfpathmoveto{\pgfqpoint{1.577074in}{1.409804in}}%
\pgfpathlineto{\pgfqpoint{1.571014in}{1.354845in}}%
\pgfpathlineto{\pgfqpoint{1.550467in}{1.357020in}}%
\pgfpathlineto{\pgfqpoint{1.552744in}{1.377402in}}%
\pgfpathlineto{\pgfqpoint{1.474789in}{1.385857in}}%
\pgfpathlineto{\pgfqpoint{1.470992in}{1.351885in}}%
\pgfpathlineto{\pgfqpoint{1.431712in}{1.356218in}}%
\pgfpathlineto{\pgfqpoint{1.436653in}{1.364650in}}%
\pgfpathlineto{\pgfqpoint{1.436137in}{1.369904in}}%
\pgfpathlineto{\pgfqpoint{1.430521in}{1.376213in}}%
\pgfpathlineto{\pgfqpoint{1.421188in}{1.383353in}}%
\pgfpathlineto{\pgfqpoint{1.421683in}{1.386315in}}%
\pgfpathlineto{\pgfqpoint{1.423585in}{1.392666in}}%
\pgfpathlineto{\pgfqpoint{1.424568in}{1.406130in}}%
\pgfpathlineto{\pgfqpoint{1.439656in}{1.412644in}}%
\pgfpathlineto{\pgfqpoint{1.451897in}{1.428337in}}%
\pgfpathlineto{\pgfqpoint{1.444005in}{1.434780in}}%
\pgfpathlineto{\pgfqpoint{1.450957in}{1.440841in}}%
\pgfpathlineto{\pgfqpoint{1.459571in}{1.445102in}}%
\pgfpathlineto{\pgfqpoint{1.461324in}{1.456249in}}%
\pgfpathlineto{\pgfqpoint{1.464403in}{1.459400in}}%
\pgfpathlineto{\pgfqpoint{1.465579in}{1.476906in}}%
\pgfpathlineto{\pgfqpoint{1.501446in}{1.472932in}}%
\pgfpathlineto{\pgfqpoint{1.499141in}{1.452348in}}%
\pgfpathlineto{\pgfqpoint{1.553648in}{1.446628in}}%
\pgfpathlineto{\pgfqpoint{1.580447in}{1.444028in}}%
\pgfpathlineto{\pgfqpoint{1.577074in}{1.409804in}}%
\pgfpathclose%
\pgfusepath{fill}%
\end{pgfscope}%
\begin{pgfscope}%
\pgfpathrectangle{\pgfqpoint{0.100000in}{0.100000in}}{\pgfqpoint{3.608454in}{2.310000in}}%
\pgfusepath{clip}%
\pgfsetbuttcap%
\pgfsetmiterjoin%
\definecolor{currentfill}{rgb}{0.000000,0.701961,0.649020}%
\pgfsetfillcolor{currentfill}%
\pgfsetlinewidth{0.000000pt}%
\definecolor{currentstroke}{rgb}{0.000000,0.000000,0.000000}%
\pgfsetstrokecolor{currentstroke}%
\pgfsetstrokeopacity{0.000000}%
\pgfsetdash{}{0pt}%
\pgfpathmoveto{\pgfqpoint{1.797569in}{0.560061in}}%
\pgfpathlineto{\pgfqpoint{1.748728in}{0.562219in}}%
\pgfpathlineto{\pgfqpoint{1.751098in}{0.605189in}}%
\pgfpathlineto{\pgfqpoint{1.710253in}{0.607672in}}%
\pgfpathlineto{\pgfqpoint{1.713553in}{0.660732in}}%
\pgfpathlineto{\pgfqpoint{1.695622in}{0.661844in}}%
\pgfpathlineto{\pgfqpoint{1.697817in}{0.695195in}}%
\pgfpathlineto{\pgfqpoint{1.755704in}{0.692007in}}%
\pgfpathlineto{\pgfqpoint{1.753800in}{0.658495in}}%
\pgfpathlineto{\pgfqpoint{1.778772in}{0.657165in}}%
\pgfpathlineto{\pgfqpoint{1.777624in}{0.639888in}}%
\pgfpathlineto{\pgfqpoint{1.782309in}{0.639635in}}%
\pgfpathlineto{\pgfqpoint{1.781527in}{0.626427in}}%
\pgfpathlineto{\pgfqpoint{1.787654in}{0.626021in}}%
\pgfpathlineto{\pgfqpoint{1.786496in}{0.603609in}}%
\pgfpathlineto{\pgfqpoint{1.799798in}{0.602976in}}%
\pgfpathlineto{\pgfqpoint{1.797569in}{0.560061in}}%
\pgfpathclose%
\pgfusepath{fill}%
\end{pgfscope}%
\begin{pgfscope}%
\pgfpathrectangle{\pgfqpoint{0.100000in}{0.100000in}}{\pgfqpoint{3.608454in}{2.310000in}}%
\pgfusepath{clip}%
\pgfsetbuttcap%
\pgfsetmiterjoin%
\definecolor{currentfill}{rgb}{0.000000,0.674510,0.662745}%
\pgfsetfillcolor{currentfill}%
\pgfsetlinewidth{0.000000pt}%
\definecolor{currentstroke}{rgb}{0.000000,0.000000,0.000000}%
\pgfsetstrokecolor{currentstroke}%
\pgfsetstrokeopacity{0.000000}%
\pgfsetdash{}{0pt}%
\pgfpathmoveto{\pgfqpoint{1.167197in}{0.891666in}}%
\pgfpathlineto{\pgfqpoint{1.176800in}{0.953138in}}%
\pgfpathlineto{\pgfqpoint{1.245517in}{0.942241in}}%
\pgfpathlineto{\pgfqpoint{1.253754in}{0.941144in}}%
\pgfpathlineto{\pgfqpoint{1.255093in}{0.930635in}}%
\pgfpathlineto{\pgfqpoint{1.253634in}{0.922827in}}%
\pgfpathlineto{\pgfqpoint{1.257767in}{0.919573in}}%
\pgfpathlineto{\pgfqpoint{1.256598in}{0.912844in}}%
\pgfpathlineto{\pgfqpoint{1.259177in}{0.906354in}}%
\pgfpathlineto{\pgfqpoint{1.256509in}{0.894758in}}%
\pgfpathlineto{\pgfqpoint{1.264784in}{0.891644in}}%
\pgfpathlineto{\pgfqpoint{1.285282in}{0.888829in}}%
\pgfpathlineto{\pgfqpoint{1.276635in}{0.824008in}}%
\pgfpathlineto{\pgfqpoint{1.215509in}{0.832612in}}%
\pgfpathlineto{\pgfqpoint{1.210301in}{0.797108in}}%
\pgfpathlineto{\pgfqpoint{1.153619in}{0.805601in}}%
\pgfpathlineto{\pgfqpoint{1.167197in}{0.891666in}}%
\pgfpathclose%
\pgfusepath{fill}%
\end{pgfscope}%
\begin{pgfscope}%
\pgfpathrectangle{\pgfqpoint{0.100000in}{0.100000in}}{\pgfqpoint{3.608454in}{2.310000in}}%
\pgfusepath{clip}%
\pgfsetbuttcap%
\pgfsetmiterjoin%
\definecolor{currentfill}{rgb}{0.000000,0.478431,0.760784}%
\pgfsetfillcolor{currentfill}%
\pgfsetlinewidth{0.000000pt}%
\definecolor{currentstroke}{rgb}{0.000000,0.000000,0.000000}%
\pgfsetstrokecolor{currentstroke}%
\pgfsetstrokeopacity{0.000000}%
\pgfsetdash{}{0pt}%
\pgfpathmoveto{\pgfqpoint{1.983409in}{1.890641in}}%
\pgfpathlineto{\pgfqpoint{1.941869in}{1.891889in}}%
\pgfpathlineto{\pgfqpoint{1.885060in}{1.894397in}}%
\pgfpathlineto{\pgfqpoint{1.886398in}{1.921692in}}%
\pgfpathlineto{\pgfqpoint{1.939707in}{1.919594in}}%
\pgfpathlineto{\pgfqpoint{1.940671in}{1.947331in}}%
\pgfpathlineto{\pgfqpoint{1.959926in}{1.946668in}}%
\pgfpathlineto{\pgfqpoint{1.981936in}{1.946008in}}%
\pgfpathlineto{\pgfqpoint{1.981211in}{1.918353in}}%
\pgfpathlineto{\pgfqpoint{1.981775in}{1.895615in}}%
\pgfpathlineto{\pgfqpoint{1.983409in}{1.890641in}}%
\pgfpathclose%
\pgfusepath{fill}%
\end{pgfscope}%
\begin{pgfscope}%
\pgfpathrectangle{\pgfqpoint{0.100000in}{0.100000in}}{\pgfqpoint{3.608454in}{2.310000in}}%
\pgfusepath{clip}%
\pgfsetbuttcap%
\pgfsetmiterjoin%
\definecolor{currentfill}{rgb}{0.000000,0.556863,0.721569}%
\pgfsetfillcolor{currentfill}%
\pgfsetlinewidth{0.000000pt}%
\definecolor{currentstroke}{rgb}{0.000000,0.000000,0.000000}%
\pgfsetstrokecolor{currentstroke}%
\pgfsetstrokeopacity{0.000000}%
\pgfsetdash{}{0pt}%
\pgfpathmoveto{\pgfqpoint{3.371539in}{1.933332in}}%
\pgfpathlineto{\pgfqpoint{3.373495in}{1.940927in}}%
\pgfpathlineto{\pgfqpoint{3.366415in}{1.942507in}}%
\pgfpathlineto{\pgfqpoint{3.368238in}{1.950258in}}%
\pgfpathlineto{\pgfqpoint{3.358298in}{1.952554in}}%
\pgfpathlineto{\pgfqpoint{3.359777in}{1.957182in}}%
\pgfpathlineto{\pgfqpoint{3.355196in}{1.970349in}}%
\pgfpathlineto{\pgfqpoint{3.368542in}{1.973373in}}%
\pgfpathlineto{\pgfqpoint{3.412317in}{1.985521in}}%
\pgfpathlineto{\pgfqpoint{3.415090in}{2.005571in}}%
\pgfpathlineto{\pgfqpoint{3.427643in}{2.008700in}}%
\pgfpathlineto{\pgfqpoint{3.428979in}{2.013989in}}%
\pgfpathlineto{\pgfqpoint{3.442334in}{1.974192in}}%
\pgfpathlineto{\pgfqpoint{3.454146in}{1.936802in}}%
\pgfpathlineto{\pgfqpoint{3.453336in}{1.932723in}}%
\pgfpathlineto{\pgfqpoint{3.441678in}{1.930119in}}%
\pgfpathlineto{\pgfqpoint{3.440922in}{1.921126in}}%
\pgfpathlineto{\pgfqpoint{3.433355in}{1.925260in}}%
\pgfpathlineto{\pgfqpoint{3.431085in}{1.927713in}}%
\pgfpathlineto{\pgfqpoint{3.423697in}{1.926204in}}%
\pgfpathlineto{\pgfqpoint{3.422059in}{1.929884in}}%
\pgfpathlineto{\pgfqpoint{3.415486in}{1.930416in}}%
\pgfpathlineto{\pgfqpoint{3.410255in}{1.935145in}}%
\pgfpathlineto{\pgfqpoint{3.405777in}{1.928235in}}%
\pgfpathlineto{\pgfqpoint{3.399587in}{1.926596in}}%
\pgfpathlineto{\pgfqpoint{3.396256in}{1.920610in}}%
\pgfpathlineto{\pgfqpoint{3.398006in}{1.914180in}}%
\pgfpathlineto{\pgfqpoint{3.384973in}{1.910295in}}%
\pgfpathlineto{\pgfqpoint{3.381069in}{1.911013in}}%
\pgfpathlineto{\pgfqpoint{3.384688in}{1.930010in}}%
\pgfpathlineto{\pgfqpoint{3.371539in}{1.933332in}}%
\pgfpathclose%
\pgfusepath{fill}%
\end{pgfscope}%
\begin{pgfscope}%
\pgfpathrectangle{\pgfqpoint{0.100000in}{0.100000in}}{\pgfqpoint{3.608454in}{2.310000in}}%
\pgfusepath{clip}%
\pgfsetbuttcap%
\pgfsetmiterjoin%
\definecolor{currentfill}{rgb}{0.000000,0.325490,0.837255}%
\pgfsetfillcolor{currentfill}%
\pgfsetlinewidth{0.000000pt}%
\definecolor{currentstroke}{rgb}{0.000000,0.000000,0.000000}%
\pgfsetstrokecolor{currentstroke}%
\pgfsetstrokeopacity{0.000000}%
\pgfsetdash{}{0pt}%
\pgfpathmoveto{\pgfqpoint{1.912766in}{1.573654in}}%
\pgfpathlineto{\pgfqpoint{1.914002in}{1.614954in}}%
\pgfpathlineto{\pgfqpoint{1.968471in}{1.613417in}}%
\pgfpathlineto{\pgfqpoint{1.968280in}{1.606524in}}%
\pgfpathlineto{\pgfqpoint{1.967706in}{1.585865in}}%
\pgfpathlineto{\pgfqpoint{1.988196in}{1.585344in}}%
\pgfpathlineto{\pgfqpoint{1.987553in}{1.557852in}}%
\pgfpathlineto{\pgfqpoint{1.939698in}{1.559064in}}%
\pgfpathlineto{\pgfqpoint{1.939982in}{1.572847in}}%
\pgfpathlineto{\pgfqpoint{1.912766in}{1.573654in}}%
\pgfpathclose%
\pgfusepath{fill}%
\end{pgfscope}%
\begin{pgfscope}%
\pgfpathrectangle{\pgfqpoint{0.100000in}{0.100000in}}{\pgfqpoint{3.608454in}{2.310000in}}%
\pgfusepath{clip}%
\pgfsetbuttcap%
\pgfsetmiterjoin%
\definecolor{currentfill}{rgb}{0.000000,0.450980,0.774510}%
\pgfsetfillcolor{currentfill}%
\pgfsetlinewidth{0.000000pt}%
\definecolor{currentstroke}{rgb}{0.000000,0.000000,0.000000}%
\pgfsetstrokecolor{currentstroke}%
\pgfsetstrokeopacity{0.000000}%
\pgfsetdash{}{0pt}%
\pgfpathmoveto{\pgfqpoint{1.580447in}{1.444028in}}%
\pgfpathlineto{\pgfqpoint{1.553648in}{1.446628in}}%
\pgfpathlineto{\pgfqpoint{1.499141in}{1.452348in}}%
\pgfpathlineto{\pgfqpoint{1.501446in}{1.472932in}}%
\pgfpathlineto{\pgfqpoint{1.465579in}{1.476906in}}%
\pgfpathlineto{\pgfqpoint{1.459655in}{1.485514in}}%
\pgfpathlineto{\pgfqpoint{1.455629in}{1.496066in}}%
\pgfpathlineto{\pgfqpoint{1.446700in}{1.523501in}}%
\pgfpathlineto{\pgfqpoint{1.440255in}{1.533604in}}%
\pgfpathlineto{\pgfqpoint{1.440561in}{1.538635in}}%
\pgfpathlineto{\pgfqpoint{1.494704in}{1.532257in}}%
\pgfpathlineto{\pgfqpoint{1.533391in}{1.528263in}}%
\pgfpathlineto{\pgfqpoint{1.595804in}{1.522015in}}%
\pgfpathlineto{\pgfqpoint{1.591803in}{1.484401in}}%
\pgfpathlineto{\pgfqpoint{1.598652in}{1.483744in}}%
\pgfpathlineto{\pgfqpoint{1.594536in}{1.442692in}}%
\pgfpathlineto{\pgfqpoint{1.580447in}{1.444028in}}%
\pgfpathclose%
\pgfusepath{fill}%
\end{pgfscope}%
\begin{pgfscope}%
\pgfpathrectangle{\pgfqpoint{0.100000in}{0.100000in}}{\pgfqpoint{3.608454in}{2.310000in}}%
\pgfusepath{clip}%
\pgfsetbuttcap%
\pgfsetmiterjoin%
\definecolor{currentfill}{rgb}{0.000000,0.741176,0.629412}%
\pgfsetfillcolor{currentfill}%
\pgfsetlinewidth{0.000000pt}%
\definecolor{currentstroke}{rgb}{0.000000,0.000000,0.000000}%
\pgfsetstrokecolor{currentstroke}%
\pgfsetstrokeopacity{0.000000}%
\pgfsetdash{}{0pt}%
\pgfpathmoveto{\pgfqpoint{0.869214in}{2.048910in}}%
\pgfpathlineto{\pgfqpoint{0.875412in}{2.075250in}}%
\pgfpathlineto{\pgfqpoint{0.872020in}{2.076043in}}%
\pgfpathlineto{\pgfqpoint{0.871344in}{2.087217in}}%
\pgfpathlineto{\pgfqpoint{0.864637in}{2.088901in}}%
\pgfpathlineto{\pgfqpoint{0.868296in}{2.100626in}}%
\pgfpathlineto{\pgfqpoint{0.878570in}{2.104452in}}%
\pgfpathlineto{\pgfqpoint{0.882536in}{2.100932in}}%
\pgfpathlineto{\pgfqpoint{0.890708in}{2.100407in}}%
\pgfpathlineto{\pgfqpoint{0.892918in}{2.093767in}}%
\pgfpathlineto{\pgfqpoint{0.898511in}{2.086388in}}%
\pgfpathlineto{\pgfqpoint{0.899012in}{2.076012in}}%
\pgfpathlineto{\pgfqpoint{0.907603in}{2.074322in}}%
\pgfpathlineto{\pgfqpoint{0.909838in}{2.083326in}}%
\pgfpathlineto{\pgfqpoint{0.926998in}{2.079195in}}%
\pgfpathlineto{\pgfqpoint{0.933769in}{2.084665in}}%
\pgfpathlineto{\pgfqpoint{0.949294in}{2.080795in}}%
\pgfpathlineto{\pgfqpoint{0.954967in}{2.104591in}}%
\pgfpathlineto{\pgfqpoint{0.993104in}{2.095521in}}%
\pgfpathlineto{\pgfqpoint{1.027701in}{2.087646in}}%
\pgfpathlineto{\pgfqpoint{1.027942in}{2.078990in}}%
\pgfpathlineto{\pgfqpoint{1.041857in}{2.066086in}}%
\pgfpathlineto{\pgfqpoint{1.041035in}{2.060695in}}%
\pgfpathlineto{\pgfqpoint{1.055012in}{2.060043in}}%
\pgfpathlineto{\pgfqpoint{1.056865in}{2.058071in}}%
\pgfpathlineto{\pgfqpoint{1.049410in}{2.040224in}}%
\pgfpathlineto{\pgfqpoint{1.042825in}{2.030433in}}%
\pgfpathlineto{\pgfqpoint{1.042048in}{2.022019in}}%
\pgfpathlineto{\pgfqpoint{1.037550in}{2.013584in}}%
\pgfpathlineto{\pgfqpoint{1.040540in}{1.999733in}}%
\pgfpathlineto{\pgfqpoint{1.033186in}{1.997675in}}%
\pgfpathlineto{\pgfqpoint{1.028886in}{1.992775in}}%
\pgfpathlineto{\pgfqpoint{1.030719in}{1.983835in}}%
\pgfpathlineto{\pgfqpoint{1.025284in}{1.975956in}}%
\pgfpathlineto{\pgfqpoint{1.018436in}{1.970411in}}%
\pgfpathlineto{\pgfqpoint{1.011182in}{1.972459in}}%
\pgfpathlineto{\pgfqpoint{1.018689in}{1.956057in}}%
\pgfpathlineto{\pgfqpoint{1.012082in}{1.949489in}}%
\pgfpathlineto{\pgfqpoint{0.942731in}{1.965121in}}%
\pgfpathlineto{\pgfqpoint{0.931228in}{1.960671in}}%
\pgfpathlineto{\pgfqpoint{0.923969in}{1.964164in}}%
\pgfpathlineto{\pgfqpoint{0.923373in}{1.975654in}}%
\pgfpathlineto{\pgfqpoint{0.904646in}{1.980156in}}%
\pgfpathlineto{\pgfqpoint{0.916109in}{1.996717in}}%
\pgfpathlineto{\pgfqpoint{0.923281in}{2.004122in}}%
\pgfpathlineto{\pgfqpoint{0.921587in}{2.014227in}}%
\pgfpathlineto{\pgfqpoint{0.914485in}{2.022985in}}%
\pgfpathlineto{\pgfqpoint{0.909977in}{2.024514in}}%
\pgfpathlineto{\pgfqpoint{0.906027in}{2.039422in}}%
\pgfpathlineto{\pgfqpoint{0.869214in}{2.048910in}}%
\pgfpathclose%
\pgfusepath{fill}%
\end{pgfscope}%
\begin{pgfscope}%
\pgfpathrectangle{\pgfqpoint{0.100000in}{0.100000in}}{\pgfqpoint{3.608454in}{2.310000in}}%
\pgfusepath{clip}%
\pgfsetbuttcap%
\pgfsetmiterjoin%
\definecolor{currentfill}{rgb}{0.000000,0.698039,0.650980}%
\pgfsetfillcolor{currentfill}%
\pgfsetlinewidth{0.000000pt}%
\definecolor{currentstroke}{rgb}{0.000000,0.000000,0.000000}%
\pgfsetstrokecolor{currentstroke}%
\pgfsetstrokeopacity{0.000000}%
\pgfsetdash{}{0pt}%
\pgfpathmoveto{\pgfqpoint{2.971461in}{0.696012in}}%
\pgfpathlineto{\pgfqpoint{2.999393in}{0.700496in}}%
\pgfpathlineto{\pgfqpoint{3.004229in}{0.666876in}}%
\pgfpathlineto{\pgfqpoint{3.006579in}{0.647225in}}%
\pgfpathlineto{\pgfqpoint{2.996330in}{0.642369in}}%
\pgfpathlineto{\pgfqpoint{2.995671in}{0.646679in}}%
\pgfpathlineto{\pgfqpoint{2.982209in}{0.644864in}}%
\pgfpathlineto{\pgfqpoint{2.985432in}{0.623511in}}%
\pgfpathlineto{\pgfqpoint{2.976237in}{0.622163in}}%
\pgfpathlineto{\pgfqpoint{2.978125in}{0.608670in}}%
\pgfpathlineto{\pgfqpoint{2.973159in}{0.605106in}}%
\pgfpathlineto{\pgfqpoint{2.965965in}{0.605834in}}%
\pgfpathlineto{\pgfqpoint{2.963258in}{0.602568in}}%
\pgfpathlineto{\pgfqpoint{2.962394in}{0.607479in}}%
\pgfpathlineto{\pgfqpoint{2.957118in}{0.615111in}}%
\pgfpathlineto{\pgfqpoint{2.942926in}{0.613795in}}%
\pgfpathlineto{\pgfqpoint{2.937858in}{0.621000in}}%
\pgfpathlineto{\pgfqpoint{2.930039in}{0.626757in}}%
\pgfpathlineto{\pgfqpoint{2.926843in}{0.632121in}}%
\pgfpathlineto{\pgfqpoint{2.921124in}{0.633342in}}%
\pgfpathlineto{\pgfqpoint{2.913615in}{0.638189in}}%
\pgfpathlineto{\pgfqpoint{2.910647in}{0.649877in}}%
\pgfpathlineto{\pgfqpoint{2.913541in}{0.650279in}}%
\pgfpathlineto{\pgfqpoint{2.915694in}{0.663204in}}%
\pgfpathlineto{\pgfqpoint{2.944970in}{0.667447in}}%
\pgfpathlineto{\pgfqpoint{2.946140in}{0.674377in}}%
\pgfpathlineto{\pgfqpoint{2.951329in}{0.676818in}}%
\pgfpathlineto{\pgfqpoint{2.955684in}{0.671251in}}%
\pgfpathlineto{\pgfqpoint{2.962277in}{0.669805in}}%
\pgfpathlineto{\pgfqpoint{2.965236in}{0.672986in}}%
\pgfpathlineto{\pgfqpoint{2.965631in}{0.682186in}}%
\pgfpathlineto{\pgfqpoint{2.971461in}{0.696012in}}%
\pgfpathclose%
\pgfusepath{fill}%
\end{pgfscope}%
\begin{pgfscope}%
\pgfpathrectangle{\pgfqpoint{0.100000in}{0.100000in}}{\pgfqpoint{3.608454in}{2.310000in}}%
\pgfusepath{clip}%
\pgfsetbuttcap%
\pgfsetmiterjoin%
\definecolor{currentfill}{rgb}{0.000000,0.427451,0.786275}%
\pgfsetfillcolor{currentfill}%
\pgfsetlinewidth{0.000000pt}%
\definecolor{currentstroke}{rgb}{0.000000,0.000000,0.000000}%
\pgfsetstrokecolor{currentstroke}%
\pgfsetstrokeopacity{0.000000}%
\pgfsetdash{}{0pt}%
\pgfpathmoveto{\pgfqpoint{2.817692in}{1.389496in}}%
\pgfpathlineto{\pgfqpoint{2.825080in}{1.387318in}}%
\pgfpathlineto{\pgfqpoint{2.829317in}{1.356171in}}%
\pgfpathlineto{\pgfqpoint{2.812960in}{1.360831in}}%
\pgfpathlineto{\pgfqpoint{2.807405in}{1.358642in}}%
\pgfpathlineto{\pgfqpoint{2.804813in}{1.353816in}}%
\pgfpathlineto{\pgfqpoint{2.798224in}{1.355170in}}%
\pgfpathlineto{\pgfqpoint{2.790830in}{1.362794in}}%
\pgfpathlineto{\pgfqpoint{2.784131in}{1.363939in}}%
\pgfpathlineto{\pgfqpoint{2.777359in}{1.362093in}}%
\pgfpathlineto{\pgfqpoint{2.768060in}{1.364802in}}%
\pgfpathlineto{\pgfqpoint{2.772188in}{1.346181in}}%
\pgfpathlineto{\pgfqpoint{2.760419in}{1.343088in}}%
\pgfpathlineto{\pgfqpoint{2.750154in}{1.334910in}}%
\pgfpathlineto{\pgfqpoint{2.736943in}{1.344983in}}%
\pgfpathlineto{\pgfqpoint{2.735344in}{1.352880in}}%
\pgfpathlineto{\pgfqpoint{2.726746in}{1.347306in}}%
\pgfpathlineto{\pgfqpoint{2.720376in}{1.354526in}}%
\pgfpathlineto{\pgfqpoint{2.728448in}{1.358041in}}%
\pgfpathlineto{\pgfqpoint{2.733971in}{1.365444in}}%
\pgfpathlineto{\pgfqpoint{2.727957in}{1.367391in}}%
\pgfpathlineto{\pgfqpoint{2.729795in}{1.373635in}}%
\pgfpathlineto{\pgfqpoint{2.725607in}{1.378219in}}%
\pgfpathlineto{\pgfqpoint{2.729836in}{1.382914in}}%
\pgfpathlineto{\pgfqpoint{2.726448in}{1.415659in}}%
\pgfpathlineto{\pgfqpoint{2.726071in}{1.419220in}}%
\pgfpathlineto{\pgfqpoint{2.746401in}{1.421668in}}%
\pgfpathlineto{\pgfqpoint{2.757924in}{1.424641in}}%
\pgfpathlineto{\pgfqpoint{2.776762in}{1.425313in}}%
\pgfpathlineto{\pgfqpoint{2.777953in}{1.400497in}}%
\pgfpathlineto{\pgfqpoint{2.786603in}{1.400891in}}%
\pgfpathlineto{\pgfqpoint{2.788388in}{1.383110in}}%
\pgfpathlineto{\pgfqpoint{2.804293in}{1.384898in}}%
\pgfpathlineto{\pgfqpoint{2.817692in}{1.389496in}}%
\pgfpathclose%
\pgfusepath{fill}%
\end{pgfscope}%
\begin{pgfscope}%
\pgfpathrectangle{\pgfqpoint{0.100000in}{0.100000in}}{\pgfqpoint{3.608454in}{2.310000in}}%
\pgfusepath{clip}%
\pgfsetbuttcap%
\pgfsetmiterjoin%
\definecolor{currentfill}{rgb}{0.000000,0.611765,0.694118}%
\pgfsetfillcolor{currentfill}%
\pgfsetlinewidth{0.000000pt}%
\definecolor{currentstroke}{rgb}{0.000000,0.000000,0.000000}%
\pgfsetstrokecolor{currentstroke}%
\pgfsetstrokeopacity{0.000000}%
\pgfsetdash{}{0pt}%
\pgfpathmoveto{\pgfqpoint{2.906878in}{1.433220in}}%
\pgfpathlineto{\pgfqpoint{2.894009in}{1.432157in}}%
\pgfpathlineto{\pgfqpoint{2.893522in}{1.439324in}}%
\pgfpathlineto{\pgfqpoint{2.886708in}{1.438806in}}%
\pgfpathlineto{\pgfqpoint{2.872947in}{1.440094in}}%
\pgfpathlineto{\pgfqpoint{2.872185in}{1.451622in}}%
\pgfpathlineto{\pgfqpoint{2.869385in}{1.458446in}}%
\pgfpathlineto{\pgfqpoint{2.864747in}{1.458144in}}%
\pgfpathlineto{\pgfqpoint{2.864318in}{1.465519in}}%
\pgfpathlineto{\pgfqpoint{2.878244in}{1.466104in}}%
\pgfpathlineto{\pgfqpoint{2.879950in}{1.469292in}}%
\pgfpathlineto{\pgfqpoint{2.878220in}{1.486351in}}%
\pgfpathlineto{\pgfqpoint{2.909284in}{1.489578in}}%
\pgfpathlineto{\pgfqpoint{2.908656in}{1.495198in}}%
\pgfpathlineto{\pgfqpoint{2.928425in}{1.497499in}}%
\pgfpathlineto{\pgfqpoint{2.935644in}{1.495065in}}%
\pgfpathlineto{\pgfqpoint{2.938569in}{1.471342in}}%
\pgfpathlineto{\pgfqpoint{2.933975in}{1.470774in}}%
\pgfpathlineto{\pgfqpoint{2.935564in}{1.458054in}}%
\pgfpathlineto{\pgfqpoint{2.938939in}{1.450447in}}%
\pgfpathlineto{\pgfqpoint{2.927738in}{1.446958in}}%
\pgfpathlineto{\pgfqpoint{2.914676in}{1.445590in}}%
\pgfpathlineto{\pgfqpoint{2.914756in}{1.436835in}}%
\pgfpathlineto{\pgfqpoint{2.907685in}{1.436860in}}%
\pgfpathlineto{\pgfqpoint{2.906878in}{1.433220in}}%
\pgfpathclose%
\pgfusepath{fill}%
\end{pgfscope}%
\begin{pgfscope}%
\pgfpathrectangle{\pgfqpoint{0.100000in}{0.100000in}}{\pgfqpoint{3.608454in}{2.310000in}}%
\pgfusepath{clip}%
\pgfsetbuttcap%
\pgfsetmiterjoin%
\definecolor{currentfill}{rgb}{0.000000,0.458824,0.770588}%
\pgfsetfillcolor{currentfill}%
\pgfsetlinewidth{0.000000pt}%
\definecolor{currentstroke}{rgb}{0.000000,0.000000,0.000000}%
\pgfsetstrokecolor{currentstroke}%
\pgfsetstrokeopacity{0.000000}%
\pgfsetdash{}{0pt}%
\pgfpathmoveto{\pgfqpoint{2.025168in}{1.813601in}}%
\pgfpathlineto{\pgfqpoint{2.000983in}{1.813993in}}%
\pgfpathlineto{\pgfqpoint{2.001203in}{1.827773in}}%
\pgfpathlineto{\pgfqpoint{1.981983in}{1.828333in}}%
\pgfpathlineto{\pgfqpoint{1.982265in}{1.839933in}}%
\pgfpathlineto{\pgfqpoint{1.994509in}{1.839579in}}%
\pgfpathlineto{\pgfqpoint{1.995378in}{1.841947in}}%
\pgfpathlineto{\pgfqpoint{2.024583in}{1.841364in}}%
\pgfpathlineto{\pgfqpoint{2.021780in}{1.845315in}}%
\pgfpathlineto{\pgfqpoint{2.012969in}{1.848244in}}%
\pgfpathlineto{\pgfqpoint{2.003389in}{1.863983in}}%
\pgfpathlineto{\pgfqpoint{2.004519in}{1.867267in}}%
\pgfpathlineto{\pgfqpoint{2.015031in}{1.874970in}}%
\pgfpathlineto{\pgfqpoint{2.018973in}{1.880728in}}%
\pgfpathlineto{\pgfqpoint{2.020229in}{1.889881in}}%
\pgfpathlineto{\pgfqpoint{2.019610in}{1.896790in}}%
\pgfpathlineto{\pgfqpoint{2.036803in}{1.896501in}}%
\pgfpathlineto{\pgfqpoint{2.037372in}{1.889582in}}%
\pgfpathlineto{\pgfqpoint{2.037019in}{1.861775in}}%
\pgfpathlineto{\pgfqpoint{2.037488in}{1.848022in}}%
\pgfpathlineto{\pgfqpoint{2.044431in}{1.847876in}}%
\pgfpathlineto{\pgfqpoint{2.044201in}{1.834191in}}%
\pgfpathlineto{\pgfqpoint{2.041094in}{1.830482in}}%
\pgfpathlineto{\pgfqpoint{2.034935in}{1.834889in}}%
\pgfpathlineto{\pgfqpoint{2.025490in}{1.836778in}}%
\pgfpathlineto{\pgfqpoint{2.025168in}{1.813601in}}%
\pgfpathclose%
\pgfusepath{fill}%
\end{pgfscope}%
\begin{pgfscope}%
\pgfpathrectangle{\pgfqpoint{0.100000in}{0.100000in}}{\pgfqpoint{3.608454in}{2.310000in}}%
\pgfusepath{clip}%
\pgfsetbuttcap%
\pgfsetmiterjoin%
\definecolor{currentfill}{rgb}{0.000000,0.647059,0.676471}%
\pgfsetfillcolor{currentfill}%
\pgfsetlinewidth{0.000000pt}%
\definecolor{currentstroke}{rgb}{0.000000,0.000000,0.000000}%
\pgfsetstrokecolor{currentstroke}%
\pgfsetstrokeopacity{0.000000}%
\pgfsetdash{}{0pt}%
\pgfpathmoveto{\pgfqpoint{2.661344in}{1.549614in}}%
\pgfpathlineto{\pgfqpoint{2.660092in}{1.560783in}}%
\pgfpathlineto{\pgfqpoint{2.636183in}{1.558287in}}%
\pgfpathlineto{\pgfqpoint{2.633441in}{1.583836in}}%
\pgfpathlineto{\pgfqpoint{2.649386in}{1.585311in}}%
\pgfpathlineto{\pgfqpoint{2.650897in}{1.588647in}}%
\pgfpathlineto{\pgfqpoint{2.648568in}{1.609914in}}%
\pgfpathlineto{\pgfqpoint{2.703247in}{1.616030in}}%
\pgfpathlineto{\pgfqpoint{2.706079in}{1.591455in}}%
\pgfpathlineto{\pgfqpoint{2.684290in}{1.589030in}}%
\pgfpathlineto{\pgfqpoint{2.688778in}{1.550091in}}%
\pgfpathlineto{\pgfqpoint{2.668362in}{1.548067in}}%
\pgfpathlineto{\pgfqpoint{2.661344in}{1.549614in}}%
\pgfpathclose%
\pgfusepath{fill}%
\end{pgfscope}%
\begin{pgfscope}%
\pgfpathrectangle{\pgfqpoint{0.100000in}{0.100000in}}{\pgfqpoint{3.608454in}{2.310000in}}%
\pgfusepath{clip}%
\pgfsetbuttcap%
\pgfsetmiterjoin%
\definecolor{currentfill}{rgb}{0.000000,0.000000,1.000000}%
\pgfsetfillcolor{currentfill}%
\pgfsetlinewidth{0.000000pt}%
\definecolor{currentstroke}{rgb}{0.000000,0.000000,0.000000}%
\pgfsetstrokecolor{currentstroke}%
\pgfsetstrokeopacity{0.000000}%
\pgfsetdash{}{0pt}%
\pgfpathmoveto{\pgfqpoint{1.455629in}{1.496066in}}%
\pgfpathlineto{\pgfqpoint{1.459655in}{1.485514in}}%
\pgfpathlineto{\pgfqpoint{1.465579in}{1.476906in}}%
\pgfpathlineto{\pgfqpoint{1.464403in}{1.459400in}}%
\pgfpathlineto{\pgfqpoint{1.461324in}{1.456249in}}%
\pgfpathlineto{\pgfqpoint{1.459571in}{1.445102in}}%
\pgfpathlineto{\pgfqpoint{1.450957in}{1.440841in}}%
\pgfpathlineto{\pgfqpoint{1.444005in}{1.434780in}}%
\pgfpathlineto{\pgfqpoint{1.437752in}{1.434786in}}%
\pgfpathlineto{\pgfqpoint{1.435412in}{1.444400in}}%
\pgfpathlineto{\pgfqpoint{1.426636in}{1.453930in}}%
\pgfpathlineto{\pgfqpoint{1.403922in}{1.457766in}}%
\pgfpathlineto{\pgfqpoint{1.404073in}{1.463909in}}%
\pgfpathlineto{\pgfqpoint{1.407785in}{1.490488in}}%
\pgfpathlineto{\pgfqpoint{1.407647in}{1.498770in}}%
\pgfpathlineto{\pgfqpoint{1.410993in}{1.493320in}}%
\pgfpathlineto{\pgfqpoint{1.426803in}{1.487168in}}%
\pgfpathlineto{\pgfqpoint{1.436560in}{1.486098in}}%
\pgfpathlineto{\pgfqpoint{1.444777in}{1.488383in}}%
\pgfpathlineto{\pgfqpoint{1.447426in}{1.486216in}}%
\pgfpathlineto{\pgfqpoint{1.455629in}{1.496066in}}%
\pgfpathclose%
\pgfusepath{fill}%
\end{pgfscope}%
\begin{pgfscope}%
\pgfpathrectangle{\pgfqpoint{0.100000in}{0.100000in}}{\pgfqpoint{3.608454in}{2.310000in}}%
\pgfusepath{clip}%
\pgfsetbuttcap%
\pgfsetmiterjoin%
\definecolor{currentfill}{rgb}{0.000000,0.384314,0.807843}%
\pgfsetfillcolor{currentfill}%
\pgfsetlinewidth{0.000000pt}%
\definecolor{currentstroke}{rgb}{0.000000,0.000000,0.000000}%
\pgfsetstrokecolor{currentstroke}%
\pgfsetstrokeopacity{0.000000}%
\pgfsetdash{}{0pt}%
\pgfpathmoveto{\pgfqpoint{2.638885in}{1.368939in}}%
\pgfpathlineto{\pgfqpoint{2.635715in}{1.368602in}}%
\pgfpathlineto{\pgfqpoint{2.632842in}{1.391462in}}%
\pgfpathlineto{\pgfqpoint{2.617652in}{1.390119in}}%
\pgfpathlineto{\pgfqpoint{2.616632in}{1.400311in}}%
\pgfpathlineto{\pgfqpoint{2.613308in}{1.399991in}}%
\pgfpathlineto{\pgfqpoint{2.614715in}{1.406613in}}%
\pgfpathlineto{\pgfqpoint{2.612117in}{1.412594in}}%
\pgfpathlineto{\pgfqpoint{2.607655in}{1.455712in}}%
\pgfpathlineto{\pgfqpoint{2.634909in}{1.458446in}}%
\pgfpathlineto{\pgfqpoint{2.634617in}{1.461203in}}%
\pgfpathlineto{\pgfqpoint{2.657467in}{1.463709in}}%
\pgfpathlineto{\pgfqpoint{2.659619in}{1.442049in}}%
\pgfpathlineto{\pgfqpoint{2.676851in}{1.443992in}}%
\pgfpathlineto{\pgfqpoint{2.676915in}{1.431338in}}%
\pgfpathlineto{\pgfqpoint{2.694747in}{1.433296in}}%
\pgfpathlineto{\pgfqpoint{2.697158in}{1.412713in}}%
\pgfpathlineto{\pgfqpoint{2.697801in}{1.407041in}}%
\pgfpathlineto{\pgfqpoint{2.677647in}{1.404907in}}%
\pgfpathlineto{\pgfqpoint{2.675138in}{1.396451in}}%
\pgfpathlineto{\pgfqpoint{2.650805in}{1.393553in}}%
\pgfpathlineto{\pgfqpoint{2.653405in}{1.370465in}}%
\pgfpathlineto{\pgfqpoint{2.638885in}{1.368939in}}%
\pgfpathclose%
\pgfusepath{fill}%
\end{pgfscope}%
\begin{pgfscope}%
\pgfpathrectangle{\pgfqpoint{0.100000in}{0.100000in}}{\pgfqpoint{3.608454in}{2.310000in}}%
\pgfusepath{clip}%
\pgfsetbuttcap%
\pgfsetmiterjoin%
\definecolor{currentfill}{rgb}{0.000000,0.827451,0.586275}%
\pgfsetfillcolor{currentfill}%
\pgfsetlinewidth{0.000000pt}%
\definecolor{currentstroke}{rgb}{0.000000,0.000000,0.000000}%
\pgfsetstrokecolor{currentstroke}%
\pgfsetstrokeopacity{0.000000}%
\pgfsetdash{}{0pt}%
\pgfpathmoveto{\pgfqpoint{0.899087in}{1.016739in}}%
\pgfpathlineto{\pgfqpoint{0.885192in}{0.948795in}}%
\pgfpathlineto{\pgfqpoint{0.877772in}{0.912470in}}%
\pgfpathlineto{\pgfqpoint{0.787923in}{0.968647in}}%
\pgfpathlineto{\pgfqpoint{0.790528in}{0.978335in}}%
\pgfpathlineto{\pgfqpoint{0.797923in}{0.984723in}}%
\pgfpathlineto{\pgfqpoint{0.705665in}{0.997629in}}%
\pgfpathlineto{\pgfqpoint{0.714200in}{1.032869in}}%
\pgfpathlineto{\pgfqpoint{0.715639in}{1.032543in}}%
\pgfpathlineto{\pgfqpoint{0.721788in}{1.059750in}}%
\pgfpathlineto{\pgfqpoint{0.787832in}{1.044715in}}%
\pgfpathlineto{\pgfqpoint{0.809266in}{1.037990in}}%
\pgfpathlineto{\pgfqpoint{0.809954in}{1.027225in}}%
\pgfpathlineto{\pgfqpoint{0.805414in}{1.014538in}}%
\pgfpathlineto{\pgfqpoint{0.807337in}{1.008296in}}%
\pgfpathlineto{\pgfqpoint{0.832599in}{1.002476in}}%
\pgfpathlineto{\pgfqpoint{0.839911in}{1.036150in}}%
\pgfpathlineto{\pgfqpoint{0.859997in}{1.031950in}}%
\pgfpathlineto{\pgfqpoint{0.858565in}{1.025184in}}%
\pgfpathlineto{\pgfqpoint{0.899087in}{1.016739in}}%
\pgfpathclose%
\pgfusepath{fill}%
\end{pgfscope}%
\begin{pgfscope}%
\pgfpathrectangle{\pgfqpoint{3.319565in}{0.488889in}}{\pgfqpoint{0.360845in}{0.924000in}}%
\pgfusepath{clip}%
\pgfsetbuttcap%
\pgfsetmiterjoin%
\definecolor{currentfill}{rgb}{0.000000,0.854902,0.572549}%
\pgfsetfillcolor{currentfill}%
\pgfsetlinewidth{0.000000pt}%
\definecolor{currentstroke}{rgb}{0.000000,0.000000,0.000000}%
\pgfsetstrokecolor{currentstroke}%
\pgfsetstrokeopacity{0.000000}%
\pgfsetdash{}{0pt}%
\pgfpathmoveto{\pgfqpoint{3.581475in}{0.488889in}}%
\pgfpathlineto{\pgfqpoint{3.680411in}{0.488889in}}%
\pgfpathlineto{\pgfqpoint{3.680411in}{0.581289in}}%
\pgfpathlineto{\pgfqpoint{3.581475in}{0.581289in}}%
\pgfpathlineto{\pgfqpoint{3.581475in}{0.488889in}}%
\pgfpathclose%
\pgfusepath{fill}%
\end{pgfscope}%
\begin{pgfscope}%
\pgfpathrectangle{\pgfqpoint{3.319565in}{0.488889in}}{\pgfqpoint{0.360845in}{0.924000in}}%
\pgfusepath{clip}%
\pgfsetbuttcap%
\pgfsetmiterjoin%
\definecolor{currentfill}{rgb}{0.000000,0.815686,0.592157}%
\pgfsetfillcolor{currentfill}%
\pgfsetlinewidth{0.000000pt}%
\definecolor{currentstroke}{rgb}{0.000000,0.000000,0.000000}%
\pgfsetstrokecolor{currentstroke}%
\pgfsetstrokeopacity{0.000000}%
\pgfsetdash{}{0pt}%
\pgfpathmoveto{\pgfqpoint{3.578539in}{0.498129in}}%
\pgfpathlineto{\pgfqpoint{3.680411in}{0.498129in}}%
\pgfpathlineto{\pgfqpoint{3.680411in}{0.590529in}}%
\pgfpathlineto{\pgfqpoint{3.578539in}{0.590529in}}%
\pgfpathlineto{\pgfqpoint{3.578539in}{0.498129in}}%
\pgfpathclose%
\pgfusepath{fill}%
\end{pgfscope}%
\begin{pgfscope}%
\pgfpathrectangle{\pgfqpoint{3.319565in}{0.488889in}}{\pgfqpoint{0.360845in}{0.924000in}}%
\pgfusepath{clip}%
\pgfsetbuttcap%
\pgfsetmiterjoin%
\definecolor{currentfill}{rgb}{0.000000,0.788235,0.605882}%
\pgfsetfillcolor{currentfill}%
\pgfsetlinewidth{0.000000pt}%
\definecolor{currentstroke}{rgb}{0.000000,0.000000,0.000000}%
\pgfsetstrokecolor{currentstroke}%
\pgfsetstrokeopacity{0.000000}%
\pgfsetdash{}{0pt}%
\pgfpathmoveto{\pgfqpoint{3.576250in}{0.507369in}}%
\pgfpathlineto{\pgfqpoint{3.680411in}{0.507369in}}%
\pgfpathlineto{\pgfqpoint{3.680411in}{0.599769in}}%
\pgfpathlineto{\pgfqpoint{3.576250in}{0.599769in}}%
\pgfpathlineto{\pgfqpoint{3.576250in}{0.507369in}}%
\pgfpathclose%
\pgfusepath{fill}%
\end{pgfscope}%
\begin{pgfscope}%
\pgfpathrectangle{\pgfqpoint{3.319565in}{0.488889in}}{\pgfqpoint{0.360845in}{0.924000in}}%
\pgfusepath{clip}%
\pgfsetbuttcap%
\pgfsetmiterjoin%
\definecolor{currentfill}{rgb}{0.000000,0.760784,0.619608}%
\pgfsetfillcolor{currentfill}%
\pgfsetlinewidth{0.000000pt}%
\definecolor{currentstroke}{rgb}{0.000000,0.000000,0.000000}%
\pgfsetstrokecolor{currentstroke}%
\pgfsetstrokeopacity{0.000000}%
\pgfsetdash{}{0pt}%
\pgfpathmoveto{\pgfqpoint{3.574081in}{0.516609in}}%
\pgfpathlineto{\pgfqpoint{3.680411in}{0.516609in}}%
\pgfpathlineto{\pgfqpoint{3.680411in}{0.609009in}}%
\pgfpathlineto{\pgfqpoint{3.574081in}{0.609009in}}%
\pgfpathlineto{\pgfqpoint{3.574081in}{0.516609in}}%
\pgfpathclose%
\pgfusepath{fill}%
\end{pgfscope}%
\begin{pgfscope}%
\pgfpathrectangle{\pgfqpoint{3.319565in}{0.488889in}}{\pgfqpoint{0.360845in}{0.924000in}}%
\pgfusepath{clip}%
\pgfsetbuttcap%
\pgfsetmiterjoin%
\definecolor{currentfill}{rgb}{0.000000,0.749020,0.625490}%
\pgfsetfillcolor{currentfill}%
\pgfsetlinewidth{0.000000pt}%
\definecolor{currentstroke}{rgb}{0.000000,0.000000,0.000000}%
\pgfsetstrokecolor{currentstroke}%
\pgfsetstrokeopacity{0.000000}%
\pgfsetdash{}{0pt}%
\pgfpathmoveto{\pgfqpoint{3.573102in}{0.525849in}}%
\pgfpathlineto{\pgfqpoint{3.680411in}{0.525849in}}%
\pgfpathlineto{\pgfqpoint{3.680411in}{0.618249in}}%
\pgfpathlineto{\pgfqpoint{3.573102in}{0.618249in}}%
\pgfpathlineto{\pgfqpoint{3.573102in}{0.525849in}}%
\pgfpathclose%
\pgfusepath{fill}%
\end{pgfscope}%
\begin{pgfscope}%
\pgfpathrectangle{\pgfqpoint{3.319565in}{0.488889in}}{\pgfqpoint{0.360845in}{0.924000in}}%
\pgfusepath{clip}%
\pgfsetbuttcap%
\pgfsetmiterjoin%
\definecolor{currentfill}{rgb}{0.000000,0.713725,0.643137}%
\pgfsetfillcolor{currentfill}%
\pgfsetlinewidth{0.000000pt}%
\definecolor{currentstroke}{rgb}{0.000000,0.000000,0.000000}%
\pgfsetstrokecolor{currentstroke}%
\pgfsetstrokeopacity{0.000000}%
\pgfsetdash{}{0pt}%
\pgfpathmoveto{\pgfqpoint{3.570208in}{0.535089in}}%
\pgfpathlineto{\pgfqpoint{3.680411in}{0.535089in}}%
\pgfpathlineto{\pgfqpoint{3.680411in}{0.627489in}}%
\pgfpathlineto{\pgfqpoint{3.570208in}{0.627489in}}%
\pgfpathlineto{\pgfqpoint{3.570208in}{0.535089in}}%
\pgfpathclose%
\pgfusepath{fill}%
\end{pgfscope}%
\begin{pgfscope}%
\pgfpathrectangle{\pgfqpoint{3.319565in}{0.488889in}}{\pgfqpoint{0.360845in}{0.924000in}}%
\pgfusepath{clip}%
\pgfsetbuttcap%
\pgfsetmiterjoin%
\definecolor{currentfill}{rgb}{0.000000,0.701961,0.649020}%
\pgfsetfillcolor{currentfill}%
\pgfsetlinewidth{0.000000pt}%
\definecolor{currentstroke}{rgb}{0.000000,0.000000,0.000000}%
\pgfsetstrokecolor{currentstroke}%
\pgfsetstrokeopacity{0.000000}%
\pgfsetdash{}{0pt}%
\pgfpathmoveto{\pgfqpoint{3.569139in}{0.544329in}}%
\pgfpathlineto{\pgfqpoint{3.680411in}{0.544329in}}%
\pgfpathlineto{\pgfqpoint{3.680411in}{0.636729in}}%
\pgfpathlineto{\pgfqpoint{3.569139in}{0.636729in}}%
\pgfpathlineto{\pgfqpoint{3.569139in}{0.544329in}}%
\pgfpathclose%
\pgfusepath{fill}%
\end{pgfscope}%
\begin{pgfscope}%
\pgfpathrectangle{\pgfqpoint{3.319565in}{0.488889in}}{\pgfqpoint{0.360845in}{0.924000in}}%
\pgfusepath{clip}%
\pgfsetbuttcap%
\pgfsetmiterjoin%
\definecolor{currentfill}{rgb}{0.000000,0.690196,0.654902}%
\pgfsetfillcolor{currentfill}%
\pgfsetlinewidth{0.000000pt}%
\definecolor{currentstroke}{rgb}{0.000000,0.000000,0.000000}%
\pgfsetstrokecolor{currentstroke}%
\pgfsetstrokeopacity{0.000000}%
\pgfsetdash{}{0pt}%
\pgfpathmoveto{\pgfqpoint{3.568218in}{0.553569in}}%
\pgfpathlineto{\pgfqpoint{3.680411in}{0.553569in}}%
\pgfpathlineto{\pgfqpoint{3.680411in}{0.645969in}}%
\pgfpathlineto{\pgfqpoint{3.568218in}{0.645969in}}%
\pgfpathlineto{\pgfqpoint{3.568218in}{0.553569in}}%
\pgfpathclose%
\pgfusepath{fill}%
\end{pgfscope}%
\begin{pgfscope}%
\pgfpathrectangle{\pgfqpoint{3.319565in}{0.488889in}}{\pgfqpoint{0.360845in}{0.924000in}}%
\pgfusepath{clip}%
\pgfsetbuttcap%
\pgfsetmiterjoin%
\definecolor{currentfill}{rgb}{0.000000,0.678431,0.660784}%
\pgfsetfillcolor{currentfill}%
\pgfsetlinewidth{0.000000pt}%
\definecolor{currentstroke}{rgb}{0.000000,0.000000,0.000000}%
\pgfsetstrokecolor{currentstroke}%
\pgfsetstrokeopacity{0.000000}%
\pgfsetdash{}{0pt}%
\pgfpathmoveto{\pgfqpoint{3.567329in}{0.562809in}}%
\pgfpathlineto{\pgfqpoint{3.680411in}{0.562809in}}%
\pgfpathlineto{\pgfqpoint{3.680411in}{0.655209in}}%
\pgfpathlineto{\pgfqpoint{3.567329in}{0.655209in}}%
\pgfpathlineto{\pgfqpoint{3.567329in}{0.562809in}}%
\pgfpathclose%
\pgfusepath{fill}%
\end{pgfscope}%
\begin{pgfscope}%
\pgfpathrectangle{\pgfqpoint{3.319565in}{0.488889in}}{\pgfqpoint{0.360845in}{0.924000in}}%
\pgfusepath{clip}%
\pgfsetbuttcap%
\pgfsetmiterjoin%
\definecolor{currentfill}{rgb}{0.000000,0.670588,0.664706}%
\pgfsetfillcolor{currentfill}%
\pgfsetlinewidth{0.000000pt}%
\definecolor{currentstroke}{rgb}{0.000000,0.000000,0.000000}%
\pgfsetstrokecolor{currentstroke}%
\pgfsetstrokeopacity{0.000000}%
\pgfsetdash{}{0pt}%
\pgfpathmoveto{\pgfqpoint{3.566822in}{0.572049in}}%
\pgfpathlineto{\pgfqpoint{3.680411in}{0.572049in}}%
\pgfpathlineto{\pgfqpoint{3.680411in}{0.664449in}}%
\pgfpathlineto{\pgfqpoint{3.566822in}{0.664449in}}%
\pgfpathlineto{\pgfqpoint{3.566822in}{0.572049in}}%
\pgfpathclose%
\pgfusepath{fill}%
\end{pgfscope}%
\begin{pgfscope}%
\pgfpathrectangle{\pgfqpoint{3.319565in}{0.488889in}}{\pgfqpoint{0.360845in}{0.924000in}}%
\pgfusepath{clip}%
\pgfsetbuttcap%
\pgfsetmiterjoin%
\definecolor{currentfill}{rgb}{0.000000,0.666667,0.666667}%
\pgfsetfillcolor{currentfill}%
\pgfsetlinewidth{0.000000pt}%
\definecolor{currentstroke}{rgb}{0.000000,0.000000,0.000000}%
\pgfsetstrokecolor{currentstroke}%
\pgfsetstrokeopacity{0.000000}%
\pgfsetdash{}{0pt}%
\pgfpathmoveto{\pgfqpoint{3.566410in}{0.581289in}}%
\pgfpathlineto{\pgfqpoint{3.680411in}{0.581289in}}%
\pgfpathlineto{\pgfqpoint{3.680411in}{0.673689in}}%
\pgfpathlineto{\pgfqpoint{3.566410in}{0.673689in}}%
\pgfpathlineto{\pgfqpoint{3.566410in}{0.581289in}}%
\pgfpathclose%
\pgfusepath{fill}%
\end{pgfscope}%
\begin{pgfscope}%
\pgfpathrectangle{\pgfqpoint{3.319565in}{0.488889in}}{\pgfqpoint{0.360845in}{0.924000in}}%
\pgfusepath{clip}%
\pgfsetbuttcap%
\pgfsetmiterjoin%
\definecolor{currentfill}{rgb}{0.000000,0.654902,0.672549}%
\pgfsetfillcolor{currentfill}%
\pgfsetlinewidth{0.000000pt}%
\definecolor{currentstroke}{rgb}{0.000000,0.000000,0.000000}%
\pgfsetstrokecolor{currentstroke}%
\pgfsetstrokeopacity{0.000000}%
\pgfsetdash{}{0pt}%
\pgfpathmoveto{\pgfqpoint{3.565309in}{0.590529in}}%
\pgfpathlineto{\pgfqpoint{3.680411in}{0.590529in}}%
\pgfpathlineto{\pgfqpoint{3.680411in}{0.682929in}}%
\pgfpathlineto{\pgfqpoint{3.565309in}{0.682929in}}%
\pgfpathlineto{\pgfqpoint{3.565309in}{0.590529in}}%
\pgfpathclose%
\pgfusepath{fill}%
\end{pgfscope}%
\begin{pgfscope}%
\pgfpathrectangle{\pgfqpoint{3.319565in}{0.488889in}}{\pgfqpoint{0.360845in}{0.924000in}}%
\pgfusepath{clip}%
\pgfsetbuttcap%
\pgfsetmiterjoin%
\definecolor{currentfill}{rgb}{0.000000,0.643137,0.678431}%
\pgfsetfillcolor{currentfill}%
\pgfsetlinewidth{0.000000pt}%
\definecolor{currentstroke}{rgb}{0.000000,0.000000,0.000000}%
\pgfsetstrokecolor{currentstroke}%
\pgfsetstrokeopacity{0.000000}%
\pgfsetdash{}{0pt}%
\pgfpathmoveto{\pgfqpoint{3.564406in}{0.599769in}}%
\pgfpathlineto{\pgfqpoint{3.680411in}{0.599769in}}%
\pgfpathlineto{\pgfqpoint{3.680411in}{0.692169in}}%
\pgfpathlineto{\pgfqpoint{3.564406in}{0.692169in}}%
\pgfpathlineto{\pgfqpoint{3.564406in}{0.599769in}}%
\pgfpathclose%
\pgfusepath{fill}%
\end{pgfscope}%
\begin{pgfscope}%
\pgfpathrectangle{\pgfqpoint{3.319565in}{0.488889in}}{\pgfqpoint{0.360845in}{0.924000in}}%
\pgfusepath{clip}%
\pgfsetbuttcap%
\pgfsetmiterjoin%
\definecolor{currentfill}{rgb}{0.000000,0.631373,0.684314}%
\pgfsetfillcolor{currentfill}%
\pgfsetlinewidth{0.000000pt}%
\definecolor{currentstroke}{rgb}{0.000000,0.000000,0.000000}%
\pgfsetstrokecolor{currentstroke}%
\pgfsetstrokeopacity{0.000000}%
\pgfsetdash{}{0pt}%
\pgfpathmoveto{\pgfqpoint{3.563660in}{0.609009in}}%
\pgfpathlineto{\pgfqpoint{3.680411in}{0.609009in}}%
\pgfpathlineto{\pgfqpoint{3.680411in}{0.701409in}}%
\pgfpathlineto{\pgfqpoint{3.563660in}{0.701409in}}%
\pgfpathlineto{\pgfqpoint{3.563660in}{0.609009in}}%
\pgfpathclose%
\pgfusepath{fill}%
\end{pgfscope}%
\begin{pgfscope}%
\pgfpathrectangle{\pgfqpoint{3.319565in}{0.488889in}}{\pgfqpoint{0.360845in}{0.924000in}}%
\pgfusepath{clip}%
\pgfsetbuttcap%
\pgfsetmiterjoin%
\definecolor{currentfill}{rgb}{0.000000,0.627451,0.686275}%
\pgfsetfillcolor{currentfill}%
\pgfsetlinewidth{0.000000pt}%
\definecolor{currentstroke}{rgb}{0.000000,0.000000,0.000000}%
\pgfsetstrokecolor{currentstroke}%
\pgfsetstrokeopacity{0.000000}%
\pgfsetdash{}{0pt}%
\pgfpathmoveto{\pgfqpoint{3.563249in}{0.618249in}}%
\pgfpathlineto{\pgfqpoint{3.680411in}{0.618249in}}%
\pgfpathlineto{\pgfqpoint{3.680411in}{0.710649in}}%
\pgfpathlineto{\pgfqpoint{3.563249in}{0.710649in}}%
\pgfpathlineto{\pgfqpoint{3.563249in}{0.618249in}}%
\pgfpathclose%
\pgfusepath{fill}%
\end{pgfscope}%
\begin{pgfscope}%
\pgfpathrectangle{\pgfqpoint{3.319565in}{0.488889in}}{\pgfqpoint{0.360845in}{0.924000in}}%
\pgfusepath{clip}%
\pgfsetbuttcap%
\pgfsetmiterjoin%
\definecolor{currentfill}{rgb}{0.000000,0.607843,0.696078}%
\pgfsetfillcolor{currentfill}%
\pgfsetlinewidth{0.000000pt}%
\definecolor{currentstroke}{rgb}{0.000000,0.000000,0.000000}%
\pgfsetstrokecolor{currentstroke}%
\pgfsetstrokeopacity{0.000000}%
\pgfsetdash{}{0pt}%
\pgfpathmoveto{\pgfqpoint{3.561733in}{0.627489in}}%
\pgfpathlineto{\pgfqpoint{3.680411in}{0.627489in}}%
\pgfpathlineto{\pgfqpoint{3.680411in}{0.719889in}}%
\pgfpathlineto{\pgfqpoint{3.561733in}{0.719889in}}%
\pgfpathlineto{\pgfqpoint{3.561733in}{0.627489in}}%
\pgfpathclose%
\pgfusepath{fill}%
\end{pgfscope}%
\begin{pgfscope}%
\pgfpathrectangle{\pgfqpoint{3.319565in}{0.488889in}}{\pgfqpoint{0.360845in}{0.924000in}}%
\pgfusepath{clip}%
\pgfsetbuttcap%
\pgfsetmiterjoin%
\definecolor{currentfill}{rgb}{0.000000,0.603922,0.698039}%
\pgfsetfillcolor{currentfill}%
\pgfsetlinewidth{0.000000pt}%
\definecolor{currentstroke}{rgb}{0.000000,0.000000,0.000000}%
\pgfsetstrokecolor{currentstroke}%
\pgfsetstrokeopacity{0.000000}%
\pgfsetdash{}{0pt}%
\pgfpathmoveto{\pgfqpoint{3.561363in}{0.636729in}}%
\pgfpathlineto{\pgfqpoint{3.680411in}{0.636729in}}%
\pgfpathlineto{\pgfqpoint{3.680411in}{0.729129in}}%
\pgfpathlineto{\pgfqpoint{3.561363in}{0.729129in}}%
\pgfpathlineto{\pgfqpoint{3.561363in}{0.636729in}}%
\pgfpathclose%
\pgfusepath{fill}%
\end{pgfscope}%
\begin{pgfscope}%
\pgfpathrectangle{\pgfqpoint{3.319565in}{0.488889in}}{\pgfqpoint{0.360845in}{0.924000in}}%
\pgfusepath{clip}%
\pgfsetbuttcap%
\pgfsetmiterjoin%
\definecolor{currentfill}{rgb}{0.000000,0.600000,0.700000}%
\pgfsetfillcolor{currentfill}%
\pgfsetlinewidth{0.000000pt}%
\definecolor{currentstroke}{rgb}{0.000000,0.000000,0.000000}%
\pgfsetstrokecolor{currentstroke}%
\pgfsetstrokeopacity{0.000000}%
\pgfsetdash{}{0pt}%
\pgfpathmoveto{\pgfqpoint{3.560871in}{0.645969in}}%
\pgfpathlineto{\pgfqpoint{3.680411in}{0.645969in}}%
\pgfpathlineto{\pgfqpoint{3.680411in}{0.738369in}}%
\pgfpathlineto{\pgfqpoint{3.560871in}{0.738369in}}%
\pgfpathlineto{\pgfqpoint{3.560871in}{0.645969in}}%
\pgfpathclose%
\pgfusepath{fill}%
\end{pgfscope}%
\begin{pgfscope}%
\pgfpathrectangle{\pgfqpoint{3.319565in}{0.488889in}}{\pgfqpoint{0.360845in}{0.924000in}}%
\pgfusepath{clip}%
\pgfsetbuttcap%
\pgfsetmiterjoin%
\definecolor{currentfill}{rgb}{0.000000,0.596078,0.701961}%
\pgfsetfillcolor{currentfill}%
\pgfsetlinewidth{0.000000pt}%
\definecolor{currentstroke}{rgb}{0.000000,0.000000,0.000000}%
\pgfsetstrokecolor{currentstroke}%
\pgfsetstrokeopacity{0.000000}%
\pgfsetdash{}{0pt}%
\pgfpathmoveto{\pgfqpoint{3.560811in}{0.655209in}}%
\pgfpathlineto{\pgfqpoint{3.680411in}{0.655209in}}%
\pgfpathlineto{\pgfqpoint{3.680411in}{0.747609in}}%
\pgfpathlineto{\pgfqpoint{3.560811in}{0.747609in}}%
\pgfpathlineto{\pgfqpoint{3.560811in}{0.655209in}}%
\pgfpathclose%
\pgfusepath{fill}%
\end{pgfscope}%
\begin{pgfscope}%
\pgfpathrectangle{\pgfqpoint{3.319565in}{0.488889in}}{\pgfqpoint{0.360845in}{0.924000in}}%
\pgfusepath{clip}%
\pgfsetbuttcap%
\pgfsetmiterjoin%
\definecolor{currentfill}{rgb}{0.000000,0.584314,0.707843}%
\pgfsetfillcolor{currentfill}%
\pgfsetlinewidth{0.000000pt}%
\definecolor{currentstroke}{rgb}{0.000000,0.000000,0.000000}%
\pgfsetstrokecolor{currentstroke}%
\pgfsetstrokeopacity{0.000000}%
\pgfsetdash{}{0pt}%
\pgfpathmoveto{\pgfqpoint{3.559810in}{0.664449in}}%
\pgfpathlineto{\pgfqpoint{3.680411in}{0.664449in}}%
\pgfpathlineto{\pgfqpoint{3.680411in}{0.756849in}}%
\pgfpathlineto{\pgfqpoint{3.559810in}{0.756849in}}%
\pgfpathlineto{\pgfqpoint{3.559810in}{0.664449in}}%
\pgfpathclose%
\pgfusepath{fill}%
\end{pgfscope}%
\begin{pgfscope}%
\pgfpathrectangle{\pgfqpoint{3.319565in}{0.488889in}}{\pgfqpoint{0.360845in}{0.924000in}}%
\pgfusepath{clip}%
\pgfsetbuttcap%
\pgfsetmiterjoin%
\definecolor{currentfill}{rgb}{0.000000,0.576471,0.711765}%
\pgfsetfillcolor{currentfill}%
\pgfsetlinewidth{0.000000pt}%
\definecolor{currentstroke}{rgb}{0.000000,0.000000,0.000000}%
\pgfsetstrokecolor{currentstroke}%
\pgfsetstrokeopacity{0.000000}%
\pgfsetdash{}{0pt}%
\pgfpathmoveto{\pgfqpoint{3.559144in}{0.673689in}}%
\pgfpathlineto{\pgfqpoint{3.680411in}{0.673689in}}%
\pgfpathlineto{\pgfqpoint{3.680411in}{0.766089in}}%
\pgfpathlineto{\pgfqpoint{3.559144in}{0.766089in}}%
\pgfpathlineto{\pgfqpoint{3.559144in}{0.673689in}}%
\pgfpathclose%
\pgfusepath{fill}%
\end{pgfscope}%
\begin{pgfscope}%
\pgfpathrectangle{\pgfqpoint{3.319565in}{0.488889in}}{\pgfqpoint{0.360845in}{0.924000in}}%
\pgfusepath{clip}%
\pgfsetbuttcap%
\pgfsetmiterjoin%
\definecolor{currentfill}{rgb}{0.000000,0.568627,0.715686}%
\pgfsetfillcolor{currentfill}%
\pgfsetlinewidth{0.000000pt}%
\definecolor{currentstroke}{rgb}{0.000000,0.000000,0.000000}%
\pgfsetstrokecolor{currentstroke}%
\pgfsetstrokeopacity{0.000000}%
\pgfsetdash{}{0pt}%
\pgfpathmoveto{\pgfqpoint{3.558529in}{0.682929in}}%
\pgfpathlineto{\pgfqpoint{3.680411in}{0.682929in}}%
\pgfpathlineto{\pgfqpoint{3.680411in}{0.775329in}}%
\pgfpathlineto{\pgfqpoint{3.558529in}{0.775329in}}%
\pgfpathlineto{\pgfqpoint{3.558529in}{0.682929in}}%
\pgfpathclose%
\pgfusepath{fill}%
\end{pgfscope}%
\begin{pgfscope}%
\pgfpathrectangle{\pgfqpoint{3.319565in}{0.488889in}}{\pgfqpoint{0.360845in}{0.924000in}}%
\pgfusepath{clip}%
\pgfsetbuttcap%
\pgfsetmiterjoin%
\definecolor{currentfill}{rgb}{0.000000,0.564706,0.717647}%
\pgfsetfillcolor{currentfill}%
\pgfsetlinewidth{0.000000pt}%
\definecolor{currentstroke}{rgb}{0.000000,0.000000,0.000000}%
\pgfsetstrokecolor{currentstroke}%
\pgfsetstrokeopacity{0.000000}%
\pgfsetdash{}{0pt}%
\pgfpathmoveto{\pgfqpoint{3.558230in}{0.692169in}}%
\pgfpathlineto{\pgfqpoint{3.680411in}{0.692169in}}%
\pgfpathlineto{\pgfqpoint{3.680411in}{0.784569in}}%
\pgfpathlineto{\pgfqpoint{3.558230in}{0.784569in}}%
\pgfpathlineto{\pgfqpoint{3.558230in}{0.692169in}}%
\pgfpathclose%
\pgfusepath{fill}%
\end{pgfscope}%
\begin{pgfscope}%
\pgfpathrectangle{\pgfqpoint{3.319565in}{0.488889in}}{\pgfqpoint{0.360845in}{0.924000in}}%
\pgfusepath{clip}%
\pgfsetbuttcap%
\pgfsetmiterjoin%
\definecolor{currentfill}{rgb}{0.000000,0.560784,0.719608}%
\pgfsetfillcolor{currentfill}%
\pgfsetlinewidth{0.000000pt}%
\definecolor{currentstroke}{rgb}{0.000000,0.000000,0.000000}%
\pgfsetstrokecolor{currentstroke}%
\pgfsetstrokeopacity{0.000000}%
\pgfsetdash{}{0pt}%
\pgfpathmoveto{\pgfqpoint{3.557690in}{0.701409in}}%
\pgfpathlineto{\pgfqpoint{3.680411in}{0.701409in}}%
\pgfpathlineto{\pgfqpoint{3.680411in}{0.793809in}}%
\pgfpathlineto{\pgfqpoint{3.557690in}{0.793809in}}%
\pgfpathlineto{\pgfqpoint{3.557690in}{0.701409in}}%
\pgfpathclose%
\pgfusepath{fill}%
\end{pgfscope}%
\begin{pgfscope}%
\pgfpathrectangle{\pgfqpoint{3.319565in}{0.488889in}}{\pgfqpoint{0.360845in}{0.924000in}}%
\pgfusepath{clip}%
\pgfsetbuttcap%
\pgfsetmiterjoin%
\definecolor{currentfill}{rgb}{0.000000,0.552941,0.723529}%
\pgfsetfillcolor{currentfill}%
\pgfsetlinewidth{0.000000pt}%
\definecolor{currentstroke}{rgb}{0.000000,0.000000,0.000000}%
\pgfsetstrokecolor{currentstroke}%
\pgfsetstrokeopacity{0.000000}%
\pgfsetdash{}{0pt}%
\pgfpathmoveto{\pgfqpoint{3.557112in}{0.710649in}}%
\pgfpathlineto{\pgfqpoint{3.680411in}{0.710649in}}%
\pgfpathlineto{\pgfqpoint{3.680411in}{0.803049in}}%
\pgfpathlineto{\pgfqpoint{3.557112in}{0.803049in}}%
\pgfpathlineto{\pgfqpoint{3.557112in}{0.710649in}}%
\pgfpathclose%
\pgfusepath{fill}%
\end{pgfscope}%
\begin{pgfscope}%
\pgfpathrectangle{\pgfqpoint{3.319565in}{0.488889in}}{\pgfqpoint{0.360845in}{0.924000in}}%
\pgfusepath{clip}%
\pgfsetbuttcap%
\pgfsetmiterjoin%
\definecolor{currentfill}{rgb}{0.000000,0.549020,0.725490}%
\pgfsetfillcolor{currentfill}%
\pgfsetlinewidth{0.000000pt}%
\definecolor{currentstroke}{rgb}{0.000000,0.000000,0.000000}%
\pgfsetstrokecolor{currentstroke}%
\pgfsetstrokeopacity{0.000000}%
\pgfsetdash{}{0pt}%
\pgfpathmoveto{\pgfqpoint{3.556888in}{0.719889in}}%
\pgfpathlineto{\pgfqpoint{3.680411in}{0.719889in}}%
\pgfpathlineto{\pgfqpoint{3.680411in}{0.812289in}}%
\pgfpathlineto{\pgfqpoint{3.556888in}{0.812289in}}%
\pgfpathlineto{\pgfqpoint{3.556888in}{0.719889in}}%
\pgfpathclose%
\pgfusepath{fill}%
\end{pgfscope}%
\begin{pgfscope}%
\pgfpathrectangle{\pgfqpoint{3.319565in}{0.488889in}}{\pgfqpoint{0.360845in}{0.924000in}}%
\pgfusepath{clip}%
\pgfsetbuttcap%
\pgfsetmiterjoin%
\definecolor{currentfill}{rgb}{0.000000,0.545098,0.727451}%
\pgfsetfillcolor{currentfill}%
\pgfsetlinewidth{0.000000pt}%
\definecolor{currentstroke}{rgb}{0.000000,0.000000,0.000000}%
\pgfsetstrokecolor{currentstroke}%
\pgfsetstrokeopacity{0.000000}%
\pgfsetdash{}{0pt}%
\pgfpathmoveto{\pgfqpoint{3.556562in}{0.729129in}}%
\pgfpathlineto{\pgfqpoint{3.680411in}{0.729129in}}%
\pgfpathlineto{\pgfqpoint{3.680411in}{0.821529in}}%
\pgfpathlineto{\pgfqpoint{3.556562in}{0.821529in}}%
\pgfpathlineto{\pgfqpoint{3.556562in}{0.729129in}}%
\pgfpathclose%
\pgfusepath{fill}%
\end{pgfscope}%
\begin{pgfscope}%
\pgfpathrectangle{\pgfqpoint{3.319565in}{0.488889in}}{\pgfqpoint{0.360845in}{0.924000in}}%
\pgfusepath{clip}%
\pgfsetbuttcap%
\pgfsetmiterjoin%
\definecolor{currentfill}{rgb}{0.000000,0.537255,0.731373}%
\pgfsetfillcolor{currentfill}%
\pgfsetlinewidth{0.000000pt}%
\definecolor{currentstroke}{rgb}{0.000000,0.000000,0.000000}%
\pgfsetstrokecolor{currentstroke}%
\pgfsetstrokeopacity{0.000000}%
\pgfsetdash{}{0pt}%
\pgfpathmoveto{\pgfqpoint{3.556029in}{0.738369in}}%
\pgfpathlineto{\pgfqpoint{3.680411in}{0.738369in}}%
\pgfpathlineto{\pgfqpoint{3.680411in}{0.830769in}}%
\pgfpathlineto{\pgfqpoint{3.556029in}{0.830769in}}%
\pgfpathlineto{\pgfqpoint{3.556029in}{0.738369in}}%
\pgfpathclose%
\pgfusepath{fill}%
\end{pgfscope}%
\begin{pgfscope}%
\pgfpathrectangle{\pgfqpoint{3.319565in}{0.488889in}}{\pgfqpoint{0.360845in}{0.924000in}}%
\pgfusepath{clip}%
\pgfsetbuttcap%
\pgfsetmiterjoin%
\definecolor{currentfill}{rgb}{0.000000,0.537255,0.731373}%
\pgfsetfillcolor{currentfill}%
\pgfsetlinewidth{0.000000pt}%
\definecolor{currentstroke}{rgb}{0.000000,0.000000,0.000000}%
\pgfsetstrokecolor{currentstroke}%
\pgfsetstrokeopacity{0.000000}%
\pgfsetdash{}{0pt}%
\pgfpathmoveto{\pgfqpoint{3.555811in}{0.747609in}}%
\pgfpathlineto{\pgfqpoint{3.680411in}{0.747609in}}%
\pgfpathlineto{\pgfqpoint{3.680411in}{0.840009in}}%
\pgfpathlineto{\pgfqpoint{3.555811in}{0.840009in}}%
\pgfpathlineto{\pgfqpoint{3.555811in}{0.747609in}}%
\pgfpathclose%
\pgfusepath{fill}%
\end{pgfscope}%
\begin{pgfscope}%
\pgfpathrectangle{\pgfqpoint{3.319565in}{0.488889in}}{\pgfqpoint{0.360845in}{0.924000in}}%
\pgfusepath{clip}%
\pgfsetbuttcap%
\pgfsetmiterjoin%
\definecolor{currentfill}{rgb}{0.000000,0.537255,0.731373}%
\pgfsetfillcolor{currentfill}%
\pgfsetlinewidth{0.000000pt}%
\definecolor{currentstroke}{rgb}{0.000000,0.000000,0.000000}%
\pgfsetstrokecolor{currentstroke}%
\pgfsetstrokeopacity{0.000000}%
\pgfsetdash{}{0pt}%
\pgfpathmoveto{\pgfqpoint{3.555811in}{0.756849in}}%
\pgfpathlineto{\pgfqpoint{3.680411in}{0.756849in}}%
\pgfpathlineto{\pgfqpoint{3.680411in}{0.849249in}}%
\pgfpathlineto{\pgfqpoint{3.555811in}{0.849249in}}%
\pgfpathlineto{\pgfqpoint{3.555811in}{0.756849in}}%
\pgfpathclose%
\pgfusepath{fill}%
\end{pgfscope}%
\begin{pgfscope}%
\pgfpathrectangle{\pgfqpoint{3.319565in}{0.488889in}}{\pgfqpoint{0.360845in}{0.924000in}}%
\pgfusepath{clip}%
\pgfsetbuttcap%
\pgfsetmiterjoin%
\definecolor{currentfill}{rgb}{0.000000,0.529412,0.735294}%
\pgfsetfillcolor{currentfill}%
\pgfsetlinewidth{0.000000pt}%
\definecolor{currentstroke}{rgb}{0.000000,0.000000,0.000000}%
\pgfsetstrokecolor{currentstroke}%
\pgfsetstrokeopacity{0.000000}%
\pgfsetdash{}{0pt}%
\pgfpathmoveto{\pgfqpoint{3.555410in}{0.766089in}}%
\pgfpathlineto{\pgfqpoint{3.680411in}{0.766089in}}%
\pgfpathlineto{\pgfqpoint{3.680411in}{0.858489in}}%
\pgfpathlineto{\pgfqpoint{3.555410in}{0.858489in}}%
\pgfpathlineto{\pgfqpoint{3.555410in}{0.766089in}}%
\pgfpathclose%
\pgfusepath{fill}%
\end{pgfscope}%
\begin{pgfscope}%
\pgfpathrectangle{\pgfqpoint{3.319565in}{0.488889in}}{\pgfqpoint{0.360845in}{0.924000in}}%
\pgfusepath{clip}%
\pgfsetbuttcap%
\pgfsetmiterjoin%
\definecolor{currentfill}{rgb}{0.000000,0.525490,0.737255}%
\pgfsetfillcolor{currentfill}%
\pgfsetlinewidth{0.000000pt}%
\definecolor{currentstroke}{rgb}{0.000000,0.000000,0.000000}%
\pgfsetstrokecolor{currentstroke}%
\pgfsetstrokeopacity{0.000000}%
\pgfsetdash{}{0pt}%
\pgfpathmoveto{\pgfqpoint{3.555056in}{0.775329in}}%
\pgfpathlineto{\pgfqpoint{3.680411in}{0.775329in}}%
\pgfpathlineto{\pgfqpoint{3.680411in}{0.867729in}}%
\pgfpathlineto{\pgfqpoint{3.555056in}{0.867729in}}%
\pgfpathlineto{\pgfqpoint{3.555056in}{0.775329in}}%
\pgfpathclose%
\pgfusepath{fill}%
\end{pgfscope}%
\begin{pgfscope}%
\pgfpathrectangle{\pgfqpoint{3.319565in}{0.488889in}}{\pgfqpoint{0.360845in}{0.924000in}}%
\pgfusepath{clip}%
\pgfsetbuttcap%
\pgfsetmiterjoin%
\definecolor{currentfill}{rgb}{0.000000,0.521569,0.739216}%
\pgfsetfillcolor{currentfill}%
\pgfsetlinewidth{0.000000pt}%
\definecolor{currentstroke}{rgb}{0.000000,0.000000,0.000000}%
\pgfsetstrokecolor{currentstroke}%
\pgfsetstrokeopacity{0.000000}%
\pgfsetdash{}{0pt}%
\pgfpathmoveto{\pgfqpoint{3.554599in}{0.784569in}}%
\pgfpathlineto{\pgfqpoint{3.680411in}{0.784569in}}%
\pgfpathlineto{\pgfqpoint{3.680411in}{0.876969in}}%
\pgfpathlineto{\pgfqpoint{3.554599in}{0.876969in}}%
\pgfpathlineto{\pgfqpoint{3.554599in}{0.784569in}}%
\pgfpathclose%
\pgfusepath{fill}%
\end{pgfscope}%
\begin{pgfscope}%
\pgfpathrectangle{\pgfqpoint{3.319565in}{0.488889in}}{\pgfqpoint{0.360845in}{0.924000in}}%
\pgfusepath{clip}%
\pgfsetbuttcap%
\pgfsetmiterjoin%
\definecolor{currentfill}{rgb}{0.000000,0.521569,0.739216}%
\pgfsetfillcolor{currentfill}%
\pgfsetlinewidth{0.000000pt}%
\definecolor{currentstroke}{rgb}{0.000000,0.000000,0.000000}%
\pgfsetstrokecolor{currentstroke}%
\pgfsetstrokeopacity{0.000000}%
\pgfsetdash{}{0pt}%
\pgfpathmoveto{\pgfqpoint{3.554552in}{0.793809in}}%
\pgfpathlineto{\pgfqpoint{3.680411in}{0.793809in}}%
\pgfpathlineto{\pgfqpoint{3.680411in}{0.886209in}}%
\pgfpathlineto{\pgfqpoint{3.554552in}{0.886209in}}%
\pgfpathlineto{\pgfqpoint{3.554552in}{0.793809in}}%
\pgfpathclose%
\pgfusepath{fill}%
\end{pgfscope}%
\begin{pgfscope}%
\pgfpathrectangle{\pgfqpoint{3.319565in}{0.488889in}}{\pgfqpoint{0.360845in}{0.924000in}}%
\pgfusepath{clip}%
\pgfsetbuttcap%
\pgfsetmiterjoin%
\definecolor{currentfill}{rgb}{0.000000,0.521569,0.739216}%
\pgfsetfillcolor{currentfill}%
\pgfsetlinewidth{0.000000pt}%
\definecolor{currentstroke}{rgb}{0.000000,0.000000,0.000000}%
\pgfsetstrokecolor{currentstroke}%
\pgfsetstrokeopacity{0.000000}%
\pgfsetdash{}{0pt}%
\pgfpathmoveto{\pgfqpoint{3.554552in}{0.803049in}}%
\pgfpathlineto{\pgfqpoint{3.680411in}{0.803049in}}%
\pgfpathlineto{\pgfqpoint{3.680411in}{0.895449in}}%
\pgfpathlineto{\pgfqpoint{3.554552in}{0.895449in}}%
\pgfpathlineto{\pgfqpoint{3.554552in}{0.803049in}}%
\pgfpathclose%
\pgfusepath{fill}%
\end{pgfscope}%
\begin{pgfscope}%
\pgfpathrectangle{\pgfqpoint{3.319565in}{0.488889in}}{\pgfqpoint{0.360845in}{0.924000in}}%
\pgfusepath{clip}%
\pgfsetbuttcap%
\pgfsetmiterjoin%
\definecolor{currentfill}{rgb}{0.000000,0.521569,0.739216}%
\pgfsetfillcolor{currentfill}%
\pgfsetlinewidth{0.000000pt}%
\definecolor{currentstroke}{rgb}{0.000000,0.000000,0.000000}%
\pgfsetstrokecolor{currentstroke}%
\pgfsetstrokeopacity{0.000000}%
\pgfsetdash{}{0pt}%
\pgfpathmoveto{\pgfqpoint{3.554552in}{0.812289in}}%
\pgfpathlineto{\pgfqpoint{3.680411in}{0.812289in}}%
\pgfpathlineto{\pgfqpoint{3.680411in}{0.904689in}}%
\pgfpathlineto{\pgfqpoint{3.554552in}{0.904689in}}%
\pgfpathlineto{\pgfqpoint{3.554552in}{0.812289in}}%
\pgfpathclose%
\pgfusepath{fill}%
\end{pgfscope}%
\begin{pgfscope}%
\pgfpathrectangle{\pgfqpoint{3.319565in}{0.488889in}}{\pgfqpoint{0.360845in}{0.924000in}}%
\pgfusepath{clip}%
\pgfsetbuttcap%
\pgfsetmiterjoin%
\definecolor{currentfill}{rgb}{0.000000,0.521569,0.739216}%
\pgfsetfillcolor{currentfill}%
\pgfsetlinewidth{0.000000pt}%
\definecolor{currentstroke}{rgb}{0.000000,0.000000,0.000000}%
\pgfsetstrokecolor{currentstroke}%
\pgfsetstrokeopacity{0.000000}%
\pgfsetdash{}{0pt}%
\pgfpathmoveto{\pgfqpoint{3.554552in}{0.821529in}}%
\pgfpathlineto{\pgfqpoint{3.680411in}{0.821529in}}%
\pgfpathlineto{\pgfqpoint{3.680411in}{0.913929in}}%
\pgfpathlineto{\pgfqpoint{3.554552in}{0.913929in}}%
\pgfpathlineto{\pgfqpoint{3.554552in}{0.821529in}}%
\pgfpathclose%
\pgfusepath{fill}%
\end{pgfscope}%
\begin{pgfscope}%
\pgfpathrectangle{\pgfqpoint{3.319565in}{0.488889in}}{\pgfqpoint{0.360845in}{0.924000in}}%
\pgfusepath{clip}%
\pgfsetbuttcap%
\pgfsetmiterjoin%
\definecolor{currentfill}{rgb}{0.000000,0.521569,0.739216}%
\pgfsetfillcolor{currentfill}%
\pgfsetlinewidth{0.000000pt}%
\definecolor{currentstroke}{rgb}{0.000000,0.000000,0.000000}%
\pgfsetstrokecolor{currentstroke}%
\pgfsetstrokeopacity{0.000000}%
\pgfsetdash{}{0pt}%
\pgfpathmoveto{\pgfqpoint{3.554552in}{0.830769in}}%
\pgfpathlineto{\pgfqpoint{3.680411in}{0.830769in}}%
\pgfpathlineto{\pgfqpoint{3.680411in}{0.923169in}}%
\pgfpathlineto{\pgfqpoint{3.554552in}{0.923169in}}%
\pgfpathlineto{\pgfqpoint{3.554552in}{0.830769in}}%
\pgfpathclose%
\pgfusepath{fill}%
\end{pgfscope}%
\begin{pgfscope}%
\pgfpathrectangle{\pgfqpoint{3.319565in}{0.488889in}}{\pgfqpoint{0.360845in}{0.924000in}}%
\pgfusepath{clip}%
\pgfsetbuttcap%
\pgfsetmiterjoin%
\definecolor{currentfill}{rgb}{0.000000,0.517647,0.741176}%
\pgfsetfillcolor{currentfill}%
\pgfsetlinewidth{0.000000pt}%
\definecolor{currentstroke}{rgb}{0.000000,0.000000,0.000000}%
\pgfsetstrokecolor{currentstroke}%
\pgfsetstrokeopacity{0.000000}%
\pgfsetdash{}{0pt}%
\pgfpathmoveto{\pgfqpoint{3.554430in}{0.840009in}}%
\pgfpathlineto{\pgfqpoint{3.680411in}{0.840009in}}%
\pgfpathlineto{\pgfqpoint{3.680411in}{0.932409in}}%
\pgfpathlineto{\pgfqpoint{3.554430in}{0.932409in}}%
\pgfpathlineto{\pgfqpoint{3.554430in}{0.840009in}}%
\pgfpathclose%
\pgfusepath{fill}%
\end{pgfscope}%
\begin{pgfscope}%
\pgfpathrectangle{\pgfqpoint{3.319565in}{0.488889in}}{\pgfqpoint{0.360845in}{0.924000in}}%
\pgfusepath{clip}%
\pgfsetbuttcap%
\pgfsetmiterjoin%
\definecolor{currentfill}{rgb}{0.000000,0.505882,0.747059}%
\pgfsetfillcolor{currentfill}%
\pgfsetlinewidth{0.000000pt}%
\definecolor{currentstroke}{rgb}{0.000000,0.000000,0.000000}%
\pgfsetstrokecolor{currentstroke}%
\pgfsetstrokeopacity{0.000000}%
\pgfsetdash{}{0pt}%
\pgfpathmoveto{\pgfqpoint{3.553331in}{0.849249in}}%
\pgfpathlineto{\pgfqpoint{3.680411in}{0.849249in}}%
\pgfpathlineto{\pgfqpoint{3.680411in}{0.941649in}}%
\pgfpathlineto{\pgfqpoint{3.553331in}{0.941649in}}%
\pgfpathlineto{\pgfqpoint{3.553331in}{0.849249in}}%
\pgfpathclose%
\pgfusepath{fill}%
\end{pgfscope}%
\begin{pgfscope}%
\pgfpathrectangle{\pgfqpoint{3.319565in}{0.488889in}}{\pgfqpoint{0.360845in}{0.924000in}}%
\pgfusepath{clip}%
\pgfsetbuttcap%
\pgfsetmiterjoin%
\definecolor{currentfill}{rgb}{0.000000,0.494118,0.752941}%
\pgfsetfillcolor{currentfill}%
\pgfsetlinewidth{0.000000pt}%
\definecolor{currentstroke}{rgb}{0.000000,0.000000,0.000000}%
\pgfsetstrokecolor{currentstroke}%
\pgfsetstrokeopacity{0.000000}%
\pgfsetdash{}{0pt}%
\pgfpathmoveto{\pgfqpoint{3.552507in}{0.858489in}}%
\pgfpathlineto{\pgfqpoint{3.680411in}{0.858489in}}%
\pgfpathlineto{\pgfqpoint{3.680411in}{0.950889in}}%
\pgfpathlineto{\pgfqpoint{3.552507in}{0.950889in}}%
\pgfpathlineto{\pgfqpoint{3.552507in}{0.858489in}}%
\pgfpathclose%
\pgfusepath{fill}%
\end{pgfscope}%
\begin{pgfscope}%
\pgfpathrectangle{\pgfqpoint{3.319565in}{0.488889in}}{\pgfqpoint{0.360845in}{0.924000in}}%
\pgfusepath{clip}%
\pgfsetbuttcap%
\pgfsetmiterjoin%
\definecolor{currentfill}{rgb}{0.000000,0.490196,0.754902}%
\pgfsetfillcolor{currentfill}%
\pgfsetlinewidth{0.000000pt}%
\definecolor{currentstroke}{rgb}{0.000000,0.000000,0.000000}%
\pgfsetstrokecolor{currentstroke}%
\pgfsetstrokeopacity{0.000000}%
\pgfsetdash{}{0pt}%
\pgfpathmoveto{\pgfqpoint{3.552233in}{0.867729in}}%
\pgfpathlineto{\pgfqpoint{3.680411in}{0.867729in}}%
\pgfpathlineto{\pgfqpoint{3.680411in}{0.960129in}}%
\pgfpathlineto{\pgfqpoint{3.552233in}{0.960129in}}%
\pgfpathlineto{\pgfqpoint{3.552233in}{0.867729in}}%
\pgfpathclose%
\pgfusepath{fill}%
\end{pgfscope}%
\begin{pgfscope}%
\pgfpathrectangle{\pgfqpoint{3.319565in}{0.488889in}}{\pgfqpoint{0.360845in}{0.924000in}}%
\pgfusepath{clip}%
\pgfsetbuttcap%
\pgfsetmiterjoin%
\definecolor{currentfill}{rgb}{0.000000,0.482353,0.758824}%
\pgfsetfillcolor{currentfill}%
\pgfsetlinewidth{0.000000pt}%
\definecolor{currentstroke}{rgb}{0.000000,0.000000,0.000000}%
\pgfsetstrokecolor{currentstroke}%
\pgfsetstrokeopacity{0.000000}%
\pgfsetdash{}{0pt}%
\pgfpathmoveto{\pgfqpoint{3.551600in}{0.876969in}}%
\pgfpathlineto{\pgfqpoint{3.680411in}{0.876969in}}%
\pgfpathlineto{\pgfqpoint{3.680411in}{0.969369in}}%
\pgfpathlineto{\pgfqpoint{3.551600in}{0.969369in}}%
\pgfpathlineto{\pgfqpoint{3.551600in}{0.876969in}}%
\pgfpathclose%
\pgfusepath{fill}%
\end{pgfscope}%
\begin{pgfscope}%
\pgfpathrectangle{\pgfqpoint{3.319565in}{0.488889in}}{\pgfqpoint{0.360845in}{0.924000in}}%
\pgfusepath{clip}%
\pgfsetbuttcap%
\pgfsetmiterjoin%
\definecolor{currentfill}{rgb}{0.000000,0.482353,0.758824}%
\pgfsetfillcolor{currentfill}%
\pgfsetlinewidth{0.000000pt}%
\definecolor{currentstroke}{rgb}{0.000000,0.000000,0.000000}%
\pgfsetstrokecolor{currentstroke}%
\pgfsetstrokeopacity{0.000000}%
\pgfsetdash{}{0pt}%
\pgfpathmoveto{\pgfqpoint{3.551392in}{0.886209in}}%
\pgfpathlineto{\pgfqpoint{3.680411in}{0.886209in}}%
\pgfpathlineto{\pgfqpoint{3.680411in}{0.978609in}}%
\pgfpathlineto{\pgfqpoint{3.551392in}{0.978609in}}%
\pgfpathlineto{\pgfqpoint{3.551392in}{0.886209in}}%
\pgfpathclose%
\pgfusepath{fill}%
\end{pgfscope}%
\begin{pgfscope}%
\pgfpathrectangle{\pgfqpoint{3.319565in}{0.488889in}}{\pgfqpoint{0.360845in}{0.924000in}}%
\pgfusepath{clip}%
\pgfsetbuttcap%
\pgfsetmiterjoin%
\definecolor{currentfill}{rgb}{0.000000,0.482353,0.758824}%
\pgfsetfillcolor{currentfill}%
\pgfsetlinewidth{0.000000pt}%
\definecolor{currentstroke}{rgb}{0.000000,0.000000,0.000000}%
\pgfsetstrokecolor{currentstroke}%
\pgfsetstrokeopacity{0.000000}%
\pgfsetdash{}{0pt}%
\pgfpathmoveto{\pgfqpoint{3.551392in}{0.895449in}}%
\pgfpathlineto{\pgfqpoint{3.680411in}{0.895449in}}%
\pgfpathlineto{\pgfqpoint{3.680411in}{0.987849in}}%
\pgfpathlineto{\pgfqpoint{3.551392in}{0.987849in}}%
\pgfpathlineto{\pgfqpoint{3.551392in}{0.895449in}}%
\pgfpathclose%
\pgfusepath{fill}%
\end{pgfscope}%
\begin{pgfscope}%
\pgfpathrectangle{\pgfqpoint{3.319565in}{0.488889in}}{\pgfqpoint{0.360845in}{0.924000in}}%
\pgfusepath{clip}%
\pgfsetbuttcap%
\pgfsetmiterjoin%
\definecolor{currentfill}{rgb}{0.000000,0.482353,0.758824}%
\pgfsetfillcolor{currentfill}%
\pgfsetlinewidth{0.000000pt}%
\definecolor{currentstroke}{rgb}{0.000000,0.000000,0.000000}%
\pgfsetstrokecolor{currentstroke}%
\pgfsetstrokeopacity{0.000000}%
\pgfsetdash{}{0pt}%
\pgfpathmoveto{\pgfqpoint{3.551392in}{0.904689in}}%
\pgfpathlineto{\pgfqpoint{3.680411in}{0.904689in}}%
\pgfpathlineto{\pgfqpoint{3.680411in}{0.997089in}}%
\pgfpathlineto{\pgfqpoint{3.551392in}{0.997089in}}%
\pgfpathlineto{\pgfqpoint{3.551392in}{0.904689in}}%
\pgfpathclose%
\pgfusepath{fill}%
\end{pgfscope}%
\begin{pgfscope}%
\pgfpathrectangle{\pgfqpoint{3.319565in}{0.488889in}}{\pgfqpoint{0.360845in}{0.924000in}}%
\pgfusepath{clip}%
\pgfsetbuttcap%
\pgfsetmiterjoin%
\definecolor{currentfill}{rgb}{0.000000,0.482353,0.758824}%
\pgfsetfillcolor{currentfill}%
\pgfsetlinewidth{0.000000pt}%
\definecolor{currentstroke}{rgb}{0.000000,0.000000,0.000000}%
\pgfsetstrokecolor{currentstroke}%
\pgfsetstrokeopacity{0.000000}%
\pgfsetdash{}{0pt}%
\pgfpathmoveto{\pgfqpoint{3.551392in}{0.913929in}}%
\pgfpathlineto{\pgfqpoint{3.680411in}{0.913929in}}%
\pgfpathlineto{\pgfqpoint{3.680411in}{1.006329in}}%
\pgfpathlineto{\pgfqpoint{3.551392in}{1.006329in}}%
\pgfpathlineto{\pgfqpoint{3.551392in}{0.913929in}}%
\pgfpathclose%
\pgfusepath{fill}%
\end{pgfscope}%
\begin{pgfscope}%
\pgfpathrectangle{\pgfqpoint{3.319565in}{0.488889in}}{\pgfqpoint{0.360845in}{0.924000in}}%
\pgfusepath{clip}%
\pgfsetbuttcap%
\pgfsetmiterjoin%
\definecolor{currentfill}{rgb}{0.000000,0.478431,0.760784}%
\pgfsetfillcolor{currentfill}%
\pgfsetlinewidth{0.000000pt}%
\definecolor{currentstroke}{rgb}{0.000000,0.000000,0.000000}%
\pgfsetstrokecolor{currentstroke}%
\pgfsetstrokeopacity{0.000000}%
\pgfsetdash{}{0pt}%
\pgfpathmoveto{\pgfqpoint{3.551102in}{0.923169in}}%
\pgfpathlineto{\pgfqpoint{3.680411in}{0.923169in}}%
\pgfpathlineto{\pgfqpoint{3.680411in}{1.015569in}}%
\pgfpathlineto{\pgfqpoint{3.551102in}{1.015569in}}%
\pgfpathlineto{\pgfqpoint{3.551102in}{0.923169in}}%
\pgfpathclose%
\pgfusepath{fill}%
\end{pgfscope}%
\begin{pgfscope}%
\pgfpathrectangle{\pgfqpoint{3.319565in}{0.488889in}}{\pgfqpoint{0.360845in}{0.924000in}}%
\pgfusepath{clip}%
\pgfsetbuttcap%
\pgfsetmiterjoin%
\definecolor{currentfill}{rgb}{0.000000,0.474510,0.762745}%
\pgfsetfillcolor{currentfill}%
\pgfsetlinewidth{0.000000pt}%
\definecolor{currentstroke}{rgb}{0.000000,0.000000,0.000000}%
\pgfsetstrokecolor{currentstroke}%
\pgfsetstrokeopacity{0.000000}%
\pgfsetdash{}{0pt}%
\pgfpathmoveto{\pgfqpoint{3.550994in}{0.932409in}}%
\pgfpathlineto{\pgfqpoint{3.680411in}{0.932409in}}%
\pgfpathlineto{\pgfqpoint{3.680411in}{1.024809in}}%
\pgfpathlineto{\pgfqpoint{3.550994in}{1.024809in}}%
\pgfpathlineto{\pgfqpoint{3.550994in}{0.932409in}}%
\pgfpathclose%
\pgfusepath{fill}%
\end{pgfscope}%
\begin{pgfscope}%
\pgfpathrectangle{\pgfqpoint{3.319565in}{0.488889in}}{\pgfqpoint{0.360845in}{0.924000in}}%
\pgfusepath{clip}%
\pgfsetbuttcap%
\pgfsetmiterjoin%
\definecolor{currentfill}{rgb}{0.000000,0.466667,0.766667}%
\pgfsetfillcolor{currentfill}%
\pgfsetlinewidth{0.000000pt}%
\definecolor{currentstroke}{rgb}{0.000000,0.000000,0.000000}%
\pgfsetstrokecolor{currentstroke}%
\pgfsetstrokeopacity{0.000000}%
\pgfsetdash{}{0pt}%
\pgfpathmoveto{\pgfqpoint{3.550302in}{0.941649in}}%
\pgfpathlineto{\pgfqpoint{3.680411in}{0.941649in}}%
\pgfpathlineto{\pgfqpoint{3.680411in}{1.034049in}}%
\pgfpathlineto{\pgfqpoint{3.550302in}{1.034049in}}%
\pgfpathlineto{\pgfqpoint{3.550302in}{0.941649in}}%
\pgfpathclose%
\pgfusepath{fill}%
\end{pgfscope}%
\begin{pgfscope}%
\pgfpathrectangle{\pgfqpoint{3.319565in}{0.488889in}}{\pgfqpoint{0.360845in}{0.924000in}}%
\pgfusepath{clip}%
\pgfsetbuttcap%
\pgfsetmiterjoin%
\definecolor{currentfill}{rgb}{0.000000,0.466667,0.766667}%
\pgfsetfillcolor{currentfill}%
\pgfsetlinewidth{0.000000pt}%
\definecolor{currentstroke}{rgb}{0.000000,0.000000,0.000000}%
\pgfsetstrokecolor{currentstroke}%
\pgfsetstrokeopacity{0.000000}%
\pgfsetdash{}{0pt}%
\pgfpathmoveto{\pgfqpoint{3.550168in}{0.950889in}}%
\pgfpathlineto{\pgfqpoint{3.680411in}{0.950889in}}%
\pgfpathlineto{\pgfqpoint{3.680411in}{1.043289in}}%
\pgfpathlineto{\pgfqpoint{3.550168in}{1.043289in}}%
\pgfpathlineto{\pgfqpoint{3.550168in}{0.950889in}}%
\pgfpathclose%
\pgfusepath{fill}%
\end{pgfscope}%
\begin{pgfscope}%
\pgfpathrectangle{\pgfqpoint{3.319565in}{0.488889in}}{\pgfqpoint{0.360845in}{0.924000in}}%
\pgfusepath{clip}%
\pgfsetbuttcap%
\pgfsetmiterjoin%
\definecolor{currentfill}{rgb}{0.000000,0.462745,0.768627}%
\pgfsetfillcolor{currentfill}%
\pgfsetlinewidth{0.000000pt}%
\definecolor{currentstroke}{rgb}{0.000000,0.000000,0.000000}%
\pgfsetstrokecolor{currentstroke}%
\pgfsetstrokeopacity{0.000000}%
\pgfsetdash{}{0pt}%
\pgfpathmoveto{\pgfqpoint{3.549899in}{0.960129in}}%
\pgfpathlineto{\pgfqpoint{3.680411in}{0.960129in}}%
\pgfpathlineto{\pgfqpoint{3.680411in}{1.052529in}}%
\pgfpathlineto{\pgfqpoint{3.549899in}{1.052529in}}%
\pgfpathlineto{\pgfqpoint{3.549899in}{0.960129in}}%
\pgfpathclose%
\pgfusepath{fill}%
\end{pgfscope}%
\begin{pgfscope}%
\pgfpathrectangle{\pgfqpoint{3.319565in}{0.488889in}}{\pgfqpoint{0.360845in}{0.924000in}}%
\pgfusepath{clip}%
\pgfsetbuttcap%
\pgfsetmiterjoin%
\definecolor{currentfill}{rgb}{0.000000,0.454902,0.772549}%
\pgfsetfillcolor{currentfill}%
\pgfsetlinewidth{0.000000pt}%
\definecolor{currentstroke}{rgb}{0.000000,0.000000,0.000000}%
\pgfsetstrokecolor{currentstroke}%
\pgfsetstrokeopacity{0.000000}%
\pgfsetdash{}{0pt}%
\pgfpathmoveto{\pgfqpoint{3.549135in}{0.969369in}}%
\pgfpathlineto{\pgfqpoint{3.680411in}{0.969369in}}%
\pgfpathlineto{\pgfqpoint{3.680411in}{1.061769in}}%
\pgfpathlineto{\pgfqpoint{3.549135in}{1.061769in}}%
\pgfpathlineto{\pgfqpoint{3.549135in}{0.969369in}}%
\pgfpathclose%
\pgfusepath{fill}%
\end{pgfscope}%
\begin{pgfscope}%
\pgfpathrectangle{\pgfqpoint{3.319565in}{0.488889in}}{\pgfqpoint{0.360845in}{0.924000in}}%
\pgfusepath{clip}%
\pgfsetbuttcap%
\pgfsetmiterjoin%
\definecolor{currentfill}{rgb}{0.000000,0.447059,0.776471}%
\pgfsetfillcolor{currentfill}%
\pgfsetlinewidth{0.000000pt}%
\definecolor{currentstroke}{rgb}{0.000000,0.000000,0.000000}%
\pgfsetstrokecolor{currentstroke}%
\pgfsetstrokeopacity{0.000000}%
\pgfsetdash{}{0pt}%
\pgfpathmoveto{\pgfqpoint{3.548594in}{0.978609in}}%
\pgfpathlineto{\pgfqpoint{3.680411in}{0.978609in}}%
\pgfpathlineto{\pgfqpoint{3.680411in}{1.071009in}}%
\pgfpathlineto{\pgfqpoint{3.548594in}{1.071009in}}%
\pgfpathlineto{\pgfqpoint{3.548594in}{0.978609in}}%
\pgfpathclose%
\pgfusepath{fill}%
\end{pgfscope}%
\begin{pgfscope}%
\pgfpathrectangle{\pgfqpoint{3.319565in}{0.488889in}}{\pgfqpoint{0.360845in}{0.924000in}}%
\pgfusepath{clip}%
\pgfsetbuttcap%
\pgfsetmiterjoin%
\definecolor{currentfill}{rgb}{0.000000,0.447059,0.776471}%
\pgfsetfillcolor{currentfill}%
\pgfsetlinewidth{0.000000pt}%
\definecolor{currentstroke}{rgb}{0.000000,0.000000,0.000000}%
\pgfsetstrokecolor{currentstroke}%
\pgfsetstrokeopacity{0.000000}%
\pgfsetdash{}{0pt}%
\pgfpathmoveto{\pgfqpoint{3.548591in}{0.987849in}}%
\pgfpathlineto{\pgfqpoint{3.680411in}{0.987849in}}%
\pgfpathlineto{\pgfqpoint{3.680411in}{1.080249in}}%
\pgfpathlineto{\pgfqpoint{3.548591in}{1.080249in}}%
\pgfpathlineto{\pgfqpoint{3.548591in}{0.987849in}}%
\pgfpathclose%
\pgfusepath{fill}%
\end{pgfscope}%
\begin{pgfscope}%
\pgfpathrectangle{\pgfqpoint{3.319565in}{0.488889in}}{\pgfqpoint{0.360845in}{0.924000in}}%
\pgfusepath{clip}%
\pgfsetbuttcap%
\pgfsetmiterjoin%
\definecolor{currentfill}{rgb}{0.000000,0.443137,0.778431}%
\pgfsetfillcolor{currentfill}%
\pgfsetlinewidth{0.000000pt}%
\definecolor{currentstroke}{rgb}{0.000000,0.000000,0.000000}%
\pgfsetstrokecolor{currentstroke}%
\pgfsetstrokeopacity{0.000000}%
\pgfsetdash{}{0pt}%
\pgfpathmoveto{\pgfqpoint{3.548193in}{0.997089in}}%
\pgfpathlineto{\pgfqpoint{3.680411in}{0.997089in}}%
\pgfpathlineto{\pgfqpoint{3.680411in}{1.089489in}}%
\pgfpathlineto{\pgfqpoint{3.548193in}{1.089489in}}%
\pgfpathlineto{\pgfqpoint{3.548193in}{0.997089in}}%
\pgfpathclose%
\pgfusepath{fill}%
\end{pgfscope}%
\begin{pgfscope}%
\pgfpathrectangle{\pgfqpoint{3.319565in}{0.488889in}}{\pgfqpoint{0.360845in}{0.924000in}}%
\pgfusepath{clip}%
\pgfsetbuttcap%
\pgfsetmiterjoin%
\definecolor{currentfill}{rgb}{0.000000,0.439216,0.780392}%
\pgfsetfillcolor{currentfill}%
\pgfsetlinewidth{0.000000pt}%
\definecolor{currentstroke}{rgb}{0.000000,0.000000,0.000000}%
\pgfsetstrokecolor{currentstroke}%
\pgfsetstrokeopacity{0.000000}%
\pgfsetdash{}{0pt}%
\pgfpathmoveto{\pgfqpoint{3.548087in}{1.006329in}}%
\pgfpathlineto{\pgfqpoint{3.680411in}{1.006329in}}%
\pgfpathlineto{\pgfqpoint{3.680411in}{1.098729in}}%
\pgfpathlineto{\pgfqpoint{3.548087in}{1.098729in}}%
\pgfpathlineto{\pgfqpoint{3.548087in}{1.006329in}}%
\pgfpathclose%
\pgfusepath{fill}%
\end{pgfscope}%
\begin{pgfscope}%
\pgfpathrectangle{\pgfqpoint{3.319565in}{0.488889in}}{\pgfqpoint{0.360845in}{0.924000in}}%
\pgfusepath{clip}%
\pgfsetbuttcap%
\pgfsetmiterjoin%
\definecolor{currentfill}{rgb}{0.000000,0.431373,0.784314}%
\pgfsetfillcolor{currentfill}%
\pgfsetlinewidth{0.000000pt}%
\definecolor{currentstroke}{rgb}{0.000000,0.000000,0.000000}%
\pgfsetstrokecolor{currentstroke}%
\pgfsetstrokeopacity{0.000000}%
\pgfsetdash{}{0pt}%
\pgfpathmoveto{\pgfqpoint{3.547398in}{1.015569in}}%
\pgfpathlineto{\pgfqpoint{3.680411in}{1.015569in}}%
\pgfpathlineto{\pgfqpoint{3.680411in}{1.107969in}}%
\pgfpathlineto{\pgfqpoint{3.547398in}{1.107969in}}%
\pgfpathlineto{\pgfqpoint{3.547398in}{1.015569in}}%
\pgfpathclose%
\pgfusepath{fill}%
\end{pgfscope}%
\begin{pgfscope}%
\pgfpathrectangle{\pgfqpoint{3.319565in}{0.488889in}}{\pgfqpoint{0.360845in}{0.924000in}}%
\pgfusepath{clip}%
\pgfsetbuttcap%
\pgfsetmiterjoin%
\definecolor{currentfill}{rgb}{0.000000,0.431373,0.784314}%
\pgfsetfillcolor{currentfill}%
\pgfsetlinewidth{0.000000pt}%
\definecolor{currentstroke}{rgb}{0.000000,0.000000,0.000000}%
\pgfsetstrokecolor{currentstroke}%
\pgfsetstrokeopacity{0.000000}%
\pgfsetdash{}{0pt}%
\pgfpathmoveto{\pgfqpoint{3.547398in}{1.024809in}}%
\pgfpathlineto{\pgfqpoint{3.680411in}{1.024809in}}%
\pgfpathlineto{\pgfqpoint{3.680411in}{1.117209in}}%
\pgfpathlineto{\pgfqpoint{3.547398in}{1.117209in}}%
\pgfpathlineto{\pgfqpoint{3.547398in}{1.024809in}}%
\pgfpathclose%
\pgfusepath{fill}%
\end{pgfscope}%
\begin{pgfscope}%
\pgfpathrectangle{\pgfqpoint{3.319565in}{0.488889in}}{\pgfqpoint{0.360845in}{0.924000in}}%
\pgfusepath{clip}%
\pgfsetbuttcap%
\pgfsetmiterjoin%
\definecolor{currentfill}{rgb}{0.000000,0.431373,0.784314}%
\pgfsetfillcolor{currentfill}%
\pgfsetlinewidth{0.000000pt}%
\definecolor{currentstroke}{rgb}{0.000000,0.000000,0.000000}%
\pgfsetstrokecolor{currentstroke}%
\pgfsetstrokeopacity{0.000000}%
\pgfsetdash{}{0pt}%
\pgfpathmoveto{\pgfqpoint{3.547398in}{1.034049in}}%
\pgfpathlineto{\pgfqpoint{3.680411in}{1.034049in}}%
\pgfpathlineto{\pgfqpoint{3.680411in}{1.126449in}}%
\pgfpathlineto{\pgfqpoint{3.547398in}{1.126449in}}%
\pgfpathlineto{\pgfqpoint{3.547398in}{1.034049in}}%
\pgfpathclose%
\pgfusepath{fill}%
\end{pgfscope}%
\begin{pgfscope}%
\pgfpathrectangle{\pgfqpoint{3.319565in}{0.488889in}}{\pgfqpoint{0.360845in}{0.924000in}}%
\pgfusepath{clip}%
\pgfsetbuttcap%
\pgfsetmiterjoin%
\definecolor{currentfill}{rgb}{0.000000,0.427451,0.786275}%
\pgfsetfillcolor{currentfill}%
\pgfsetlinewidth{0.000000pt}%
\definecolor{currentstroke}{rgb}{0.000000,0.000000,0.000000}%
\pgfsetstrokecolor{currentstroke}%
\pgfsetstrokeopacity{0.000000}%
\pgfsetdash{}{0pt}%
\pgfpathmoveto{\pgfqpoint{3.547013in}{1.043289in}}%
\pgfpathlineto{\pgfqpoint{3.680411in}{1.043289in}}%
\pgfpathlineto{\pgfqpoint{3.680411in}{1.135689in}}%
\pgfpathlineto{\pgfqpoint{3.547013in}{1.135689in}}%
\pgfpathlineto{\pgfqpoint{3.547013in}{1.043289in}}%
\pgfpathclose%
\pgfusepath{fill}%
\end{pgfscope}%
\begin{pgfscope}%
\pgfpathrectangle{\pgfqpoint{3.319565in}{0.488889in}}{\pgfqpoint{0.360845in}{0.924000in}}%
\pgfusepath{clip}%
\pgfsetbuttcap%
\pgfsetmiterjoin%
\definecolor{currentfill}{rgb}{0.000000,0.427451,0.786275}%
\pgfsetfillcolor{currentfill}%
\pgfsetlinewidth{0.000000pt}%
\definecolor{currentstroke}{rgb}{0.000000,0.000000,0.000000}%
\pgfsetstrokecolor{currentstroke}%
\pgfsetstrokeopacity{0.000000}%
\pgfsetdash{}{0pt}%
\pgfpathmoveto{\pgfqpoint{3.546966in}{1.052529in}}%
\pgfpathlineto{\pgfqpoint{3.680411in}{1.052529in}}%
\pgfpathlineto{\pgfqpoint{3.680411in}{1.144929in}}%
\pgfpathlineto{\pgfqpoint{3.546966in}{1.144929in}}%
\pgfpathlineto{\pgfqpoint{3.546966in}{1.052529in}}%
\pgfpathclose%
\pgfusepath{fill}%
\end{pgfscope}%
\begin{pgfscope}%
\pgfpathrectangle{\pgfqpoint{3.319565in}{0.488889in}}{\pgfqpoint{0.360845in}{0.924000in}}%
\pgfusepath{clip}%
\pgfsetbuttcap%
\pgfsetmiterjoin%
\definecolor{currentfill}{rgb}{0.000000,0.423529,0.788235}%
\pgfsetfillcolor{currentfill}%
\pgfsetlinewidth{0.000000pt}%
\definecolor{currentstroke}{rgb}{0.000000,0.000000,0.000000}%
\pgfsetstrokecolor{currentstroke}%
\pgfsetstrokeopacity{0.000000}%
\pgfsetdash{}{0pt}%
\pgfpathmoveto{\pgfqpoint{3.546797in}{1.061769in}}%
\pgfpathlineto{\pgfqpoint{3.680411in}{1.061769in}}%
\pgfpathlineto{\pgfqpoint{3.680411in}{1.154169in}}%
\pgfpathlineto{\pgfqpoint{3.546797in}{1.154169in}}%
\pgfpathlineto{\pgfqpoint{3.546797in}{1.061769in}}%
\pgfpathclose%
\pgfusepath{fill}%
\end{pgfscope}%
\begin{pgfscope}%
\pgfpathrectangle{\pgfqpoint{3.319565in}{0.488889in}}{\pgfqpoint{0.360845in}{0.924000in}}%
\pgfusepath{clip}%
\pgfsetbuttcap%
\pgfsetmiterjoin%
\definecolor{currentfill}{rgb}{0.000000,0.423529,0.788235}%
\pgfsetfillcolor{currentfill}%
\pgfsetlinewidth{0.000000pt}%
\definecolor{currentstroke}{rgb}{0.000000,0.000000,0.000000}%
\pgfsetstrokecolor{currentstroke}%
\pgfsetstrokeopacity{0.000000}%
\pgfsetdash{}{0pt}%
\pgfpathmoveto{\pgfqpoint{3.546797in}{1.071009in}}%
\pgfpathlineto{\pgfqpoint{3.680411in}{1.071009in}}%
\pgfpathlineto{\pgfqpoint{3.680411in}{1.163409in}}%
\pgfpathlineto{\pgfqpoint{3.546797in}{1.163409in}}%
\pgfpathlineto{\pgfqpoint{3.546797in}{1.071009in}}%
\pgfpathclose%
\pgfusepath{fill}%
\end{pgfscope}%
\begin{pgfscope}%
\pgfpathrectangle{\pgfqpoint{3.319565in}{0.488889in}}{\pgfqpoint{0.360845in}{0.924000in}}%
\pgfusepath{clip}%
\pgfsetbuttcap%
\pgfsetmiterjoin%
\definecolor{currentfill}{rgb}{0.000000,0.419608,0.790196}%
\pgfsetfillcolor{currentfill}%
\pgfsetlinewidth{0.000000pt}%
\definecolor{currentstroke}{rgb}{0.000000,0.000000,0.000000}%
\pgfsetstrokecolor{currentstroke}%
\pgfsetstrokeopacity{0.000000}%
\pgfsetdash{}{0pt}%
\pgfpathmoveto{\pgfqpoint{3.546460in}{1.080249in}}%
\pgfpathlineto{\pgfqpoint{3.680411in}{1.080249in}}%
\pgfpathlineto{\pgfqpoint{3.680411in}{1.172649in}}%
\pgfpathlineto{\pgfqpoint{3.546460in}{1.172649in}}%
\pgfpathlineto{\pgfqpoint{3.546460in}{1.080249in}}%
\pgfpathclose%
\pgfusepath{fill}%
\end{pgfscope}%
\begin{pgfscope}%
\pgfpathrectangle{\pgfqpoint{3.319565in}{0.488889in}}{\pgfqpoint{0.360845in}{0.924000in}}%
\pgfusepath{clip}%
\pgfsetbuttcap%
\pgfsetmiterjoin%
\definecolor{currentfill}{rgb}{0.000000,0.419608,0.790196}%
\pgfsetfillcolor{currentfill}%
\pgfsetlinewidth{0.000000pt}%
\definecolor{currentstroke}{rgb}{0.000000,0.000000,0.000000}%
\pgfsetstrokecolor{currentstroke}%
\pgfsetstrokeopacity{0.000000}%
\pgfsetdash{}{0pt}%
\pgfpathmoveto{\pgfqpoint{3.546402in}{1.089489in}}%
\pgfpathlineto{\pgfqpoint{3.680411in}{1.089489in}}%
\pgfpathlineto{\pgfqpoint{3.680411in}{1.181889in}}%
\pgfpathlineto{\pgfqpoint{3.546402in}{1.181889in}}%
\pgfpathlineto{\pgfqpoint{3.546402in}{1.089489in}}%
\pgfpathclose%
\pgfusepath{fill}%
\end{pgfscope}%
\begin{pgfscope}%
\pgfpathrectangle{\pgfqpoint{3.319565in}{0.488889in}}{\pgfqpoint{0.360845in}{0.924000in}}%
\pgfusepath{clip}%
\pgfsetbuttcap%
\pgfsetmiterjoin%
\definecolor{currentfill}{rgb}{0.000000,0.411765,0.794118}%
\pgfsetfillcolor{currentfill}%
\pgfsetlinewidth{0.000000pt}%
\definecolor{currentstroke}{rgb}{0.000000,0.000000,0.000000}%
\pgfsetstrokecolor{currentstroke}%
\pgfsetstrokeopacity{0.000000}%
\pgfsetdash{}{0pt}%
\pgfpathmoveto{\pgfqpoint{3.545929in}{1.098729in}}%
\pgfpathlineto{\pgfqpoint{3.680411in}{1.098729in}}%
\pgfpathlineto{\pgfqpoint{3.680411in}{1.191129in}}%
\pgfpathlineto{\pgfqpoint{3.545929in}{1.191129in}}%
\pgfpathlineto{\pgfqpoint{3.545929in}{1.098729in}}%
\pgfpathclose%
\pgfusepath{fill}%
\end{pgfscope}%
\begin{pgfscope}%
\pgfpathrectangle{\pgfqpoint{3.319565in}{0.488889in}}{\pgfqpoint{0.360845in}{0.924000in}}%
\pgfusepath{clip}%
\pgfsetbuttcap%
\pgfsetmiterjoin%
\definecolor{currentfill}{rgb}{0.000000,0.411765,0.794118}%
\pgfsetfillcolor{currentfill}%
\pgfsetlinewidth{0.000000pt}%
\definecolor{currentstroke}{rgb}{0.000000,0.000000,0.000000}%
\pgfsetstrokecolor{currentstroke}%
\pgfsetstrokeopacity{0.000000}%
\pgfsetdash{}{0pt}%
\pgfpathmoveto{\pgfqpoint{3.545690in}{1.107969in}}%
\pgfpathlineto{\pgfqpoint{3.680411in}{1.107969in}}%
\pgfpathlineto{\pgfqpoint{3.680411in}{1.200369in}}%
\pgfpathlineto{\pgfqpoint{3.545690in}{1.200369in}}%
\pgfpathlineto{\pgfqpoint{3.545690in}{1.107969in}}%
\pgfpathclose%
\pgfusepath{fill}%
\end{pgfscope}%
\begin{pgfscope}%
\pgfpathrectangle{\pgfqpoint{3.319565in}{0.488889in}}{\pgfqpoint{0.360845in}{0.924000in}}%
\pgfusepath{clip}%
\pgfsetbuttcap%
\pgfsetmiterjoin%
\definecolor{currentfill}{rgb}{0.000000,0.411765,0.794118}%
\pgfsetfillcolor{currentfill}%
\pgfsetlinewidth{0.000000pt}%
\definecolor{currentstroke}{rgb}{0.000000,0.000000,0.000000}%
\pgfsetstrokecolor{currentstroke}%
\pgfsetstrokeopacity{0.000000}%
\pgfsetdash{}{0pt}%
\pgfpathmoveto{\pgfqpoint{3.545690in}{1.117209in}}%
\pgfpathlineto{\pgfqpoint{3.680411in}{1.117209in}}%
\pgfpathlineto{\pgfqpoint{3.680411in}{1.209609in}}%
\pgfpathlineto{\pgfqpoint{3.545690in}{1.209609in}}%
\pgfpathlineto{\pgfqpoint{3.545690in}{1.117209in}}%
\pgfpathclose%
\pgfusepath{fill}%
\end{pgfscope}%
\begin{pgfscope}%
\pgfpathrectangle{\pgfqpoint{3.319565in}{0.488889in}}{\pgfqpoint{0.360845in}{0.924000in}}%
\pgfusepath{clip}%
\pgfsetbuttcap%
\pgfsetmiterjoin%
\definecolor{currentfill}{rgb}{0.000000,0.403922,0.798039}%
\pgfsetfillcolor{currentfill}%
\pgfsetlinewidth{0.000000pt}%
\definecolor{currentstroke}{rgb}{0.000000,0.000000,0.000000}%
\pgfsetstrokecolor{currentstroke}%
\pgfsetstrokeopacity{0.000000}%
\pgfsetdash{}{0pt}%
\pgfpathmoveto{\pgfqpoint{3.545305in}{1.126449in}}%
\pgfpathlineto{\pgfqpoint{3.680411in}{1.126449in}}%
\pgfpathlineto{\pgfqpoint{3.680411in}{1.218849in}}%
\pgfpathlineto{\pgfqpoint{3.545305in}{1.218849in}}%
\pgfpathlineto{\pgfqpoint{3.545305in}{1.126449in}}%
\pgfpathclose%
\pgfusepath{fill}%
\end{pgfscope}%
\begin{pgfscope}%
\pgfpathrectangle{\pgfqpoint{3.319565in}{0.488889in}}{\pgfqpoint{0.360845in}{0.924000in}}%
\pgfusepath{clip}%
\pgfsetbuttcap%
\pgfsetmiterjoin%
\definecolor{currentfill}{rgb}{0.000000,0.392157,0.803922}%
\pgfsetfillcolor{currentfill}%
\pgfsetlinewidth{0.000000pt}%
\definecolor{currentstroke}{rgb}{0.000000,0.000000,0.000000}%
\pgfsetstrokecolor{currentstroke}%
\pgfsetstrokeopacity{0.000000}%
\pgfsetdash{}{0pt}%
\pgfpathmoveto{\pgfqpoint{3.544072in}{1.135689in}}%
\pgfpathlineto{\pgfqpoint{3.680411in}{1.135689in}}%
\pgfpathlineto{\pgfqpoint{3.680411in}{1.228089in}}%
\pgfpathlineto{\pgfqpoint{3.544072in}{1.228089in}}%
\pgfpathlineto{\pgfqpoint{3.544072in}{1.135689in}}%
\pgfpathclose%
\pgfusepath{fill}%
\end{pgfscope}%
\begin{pgfscope}%
\pgfpathrectangle{\pgfqpoint{3.319565in}{0.488889in}}{\pgfqpoint{0.360845in}{0.924000in}}%
\pgfusepath{clip}%
\pgfsetbuttcap%
\pgfsetmiterjoin%
\definecolor{currentfill}{rgb}{0.000000,0.392157,0.803922}%
\pgfsetfillcolor{currentfill}%
\pgfsetlinewidth{0.000000pt}%
\definecolor{currentstroke}{rgb}{0.000000,0.000000,0.000000}%
\pgfsetstrokecolor{currentstroke}%
\pgfsetstrokeopacity{0.000000}%
\pgfsetdash{}{0pt}%
\pgfpathmoveto{\pgfqpoint{3.544072in}{1.144929in}}%
\pgfpathlineto{\pgfqpoint{3.680411in}{1.144929in}}%
\pgfpathlineto{\pgfqpoint{3.680411in}{1.237329in}}%
\pgfpathlineto{\pgfqpoint{3.544072in}{1.237329in}}%
\pgfpathlineto{\pgfqpoint{3.544072in}{1.144929in}}%
\pgfpathclose%
\pgfusepath{fill}%
\end{pgfscope}%
\begin{pgfscope}%
\pgfpathrectangle{\pgfqpoint{3.319565in}{0.488889in}}{\pgfqpoint{0.360845in}{0.924000in}}%
\pgfusepath{clip}%
\pgfsetbuttcap%
\pgfsetmiterjoin%
\definecolor{currentfill}{rgb}{0.000000,0.388235,0.805882}%
\pgfsetfillcolor{currentfill}%
\pgfsetlinewidth{0.000000pt}%
\definecolor{currentstroke}{rgb}{0.000000,0.000000,0.000000}%
\pgfsetstrokecolor{currentstroke}%
\pgfsetstrokeopacity{0.000000}%
\pgfsetdash{}{0pt}%
\pgfpathmoveto{\pgfqpoint{3.543791in}{1.154169in}}%
\pgfpathlineto{\pgfqpoint{3.680411in}{1.154169in}}%
\pgfpathlineto{\pgfqpoint{3.680411in}{1.246569in}}%
\pgfpathlineto{\pgfqpoint{3.543791in}{1.246569in}}%
\pgfpathlineto{\pgfqpoint{3.543791in}{1.154169in}}%
\pgfpathclose%
\pgfusepath{fill}%
\end{pgfscope}%
\begin{pgfscope}%
\pgfpathrectangle{\pgfqpoint{3.319565in}{0.488889in}}{\pgfqpoint{0.360845in}{0.924000in}}%
\pgfusepath{clip}%
\pgfsetbuttcap%
\pgfsetmiterjoin%
\definecolor{currentfill}{rgb}{0.000000,0.388235,0.805882}%
\pgfsetfillcolor{currentfill}%
\pgfsetlinewidth{0.000000pt}%
\definecolor{currentstroke}{rgb}{0.000000,0.000000,0.000000}%
\pgfsetstrokecolor{currentstroke}%
\pgfsetstrokeopacity{0.000000}%
\pgfsetdash{}{0pt}%
\pgfpathmoveto{\pgfqpoint{3.543791in}{1.163409in}}%
\pgfpathlineto{\pgfqpoint{3.680411in}{1.163409in}}%
\pgfpathlineto{\pgfqpoint{3.680411in}{1.255809in}}%
\pgfpathlineto{\pgfqpoint{3.543791in}{1.255809in}}%
\pgfpathlineto{\pgfqpoint{3.543791in}{1.163409in}}%
\pgfpathclose%
\pgfusepath{fill}%
\end{pgfscope}%
\begin{pgfscope}%
\pgfpathrectangle{\pgfqpoint{3.319565in}{0.488889in}}{\pgfqpoint{0.360845in}{0.924000in}}%
\pgfusepath{clip}%
\pgfsetbuttcap%
\pgfsetmiterjoin%
\definecolor{currentfill}{rgb}{0.000000,0.388235,0.805882}%
\pgfsetfillcolor{currentfill}%
\pgfsetlinewidth{0.000000pt}%
\definecolor{currentstroke}{rgb}{0.000000,0.000000,0.000000}%
\pgfsetstrokecolor{currentstroke}%
\pgfsetstrokeopacity{0.000000}%
\pgfsetdash{}{0pt}%
\pgfpathmoveto{\pgfqpoint{3.543791in}{1.172649in}}%
\pgfpathlineto{\pgfqpoint{3.680411in}{1.172649in}}%
\pgfpathlineto{\pgfqpoint{3.680411in}{1.265049in}}%
\pgfpathlineto{\pgfqpoint{3.543791in}{1.265049in}}%
\pgfpathlineto{\pgfqpoint{3.543791in}{1.172649in}}%
\pgfpathclose%
\pgfusepath{fill}%
\end{pgfscope}%
\begin{pgfscope}%
\pgfpathrectangle{\pgfqpoint{3.319565in}{0.488889in}}{\pgfqpoint{0.360845in}{0.924000in}}%
\pgfusepath{clip}%
\pgfsetbuttcap%
\pgfsetmiterjoin%
\definecolor{currentfill}{rgb}{0.000000,0.380392,0.809804}%
\pgfsetfillcolor{currentfill}%
\pgfsetlinewidth{0.000000pt}%
\definecolor{currentstroke}{rgb}{0.000000,0.000000,0.000000}%
\pgfsetstrokecolor{currentstroke}%
\pgfsetstrokeopacity{0.000000}%
\pgfsetdash{}{0pt}%
\pgfpathmoveto{\pgfqpoint{3.543213in}{1.181889in}}%
\pgfpathlineto{\pgfqpoint{3.680411in}{1.181889in}}%
\pgfpathlineto{\pgfqpoint{3.680411in}{1.274289in}}%
\pgfpathlineto{\pgfqpoint{3.543213in}{1.274289in}}%
\pgfpathlineto{\pgfqpoint{3.543213in}{1.181889in}}%
\pgfpathclose%
\pgfusepath{fill}%
\end{pgfscope}%
\begin{pgfscope}%
\pgfpathrectangle{\pgfqpoint{3.319565in}{0.488889in}}{\pgfqpoint{0.360845in}{0.924000in}}%
\pgfusepath{clip}%
\pgfsetbuttcap%
\pgfsetmiterjoin%
\definecolor{currentfill}{rgb}{0.000000,0.380392,0.809804}%
\pgfsetfillcolor{currentfill}%
\pgfsetlinewidth{0.000000pt}%
\definecolor{currentstroke}{rgb}{0.000000,0.000000,0.000000}%
\pgfsetstrokecolor{currentstroke}%
\pgfsetstrokeopacity{0.000000}%
\pgfsetdash{}{0pt}%
\pgfpathmoveto{\pgfqpoint{3.543213in}{1.191129in}}%
\pgfpathlineto{\pgfqpoint{3.680411in}{1.191129in}}%
\pgfpathlineto{\pgfqpoint{3.680411in}{1.283529in}}%
\pgfpathlineto{\pgfqpoint{3.543213in}{1.283529in}}%
\pgfpathlineto{\pgfqpoint{3.543213in}{1.191129in}}%
\pgfpathclose%
\pgfusepath{fill}%
\end{pgfscope}%
\begin{pgfscope}%
\pgfpathrectangle{\pgfqpoint{3.319565in}{0.488889in}}{\pgfqpoint{0.360845in}{0.924000in}}%
\pgfusepath{clip}%
\pgfsetbuttcap%
\pgfsetmiterjoin%
\definecolor{currentfill}{rgb}{0.000000,0.380392,0.809804}%
\pgfsetfillcolor{currentfill}%
\pgfsetlinewidth{0.000000pt}%
\definecolor{currentstroke}{rgb}{0.000000,0.000000,0.000000}%
\pgfsetstrokecolor{currentstroke}%
\pgfsetstrokeopacity{0.000000}%
\pgfsetdash{}{0pt}%
\pgfpathmoveto{\pgfqpoint{3.543104in}{1.200369in}}%
\pgfpathlineto{\pgfqpoint{3.680411in}{1.200369in}}%
\pgfpathlineto{\pgfqpoint{3.680411in}{1.292769in}}%
\pgfpathlineto{\pgfqpoint{3.543104in}{1.292769in}}%
\pgfpathlineto{\pgfqpoint{3.543104in}{1.200369in}}%
\pgfpathclose%
\pgfusepath{fill}%
\end{pgfscope}%
\begin{pgfscope}%
\pgfpathrectangle{\pgfqpoint{3.319565in}{0.488889in}}{\pgfqpoint{0.360845in}{0.924000in}}%
\pgfusepath{clip}%
\pgfsetbuttcap%
\pgfsetmiterjoin%
\definecolor{currentfill}{rgb}{0.000000,0.372549,0.813725}%
\pgfsetfillcolor{currentfill}%
\pgfsetlinewidth{0.000000pt}%
\definecolor{currentstroke}{rgb}{0.000000,0.000000,0.000000}%
\pgfsetstrokecolor{currentstroke}%
\pgfsetstrokeopacity{0.000000}%
\pgfsetdash{}{0pt}%
\pgfpathmoveto{\pgfqpoint{3.542527in}{1.209609in}}%
\pgfpathlineto{\pgfqpoint{3.680411in}{1.209609in}}%
\pgfpathlineto{\pgfqpoint{3.680411in}{1.302009in}}%
\pgfpathlineto{\pgfqpoint{3.542527in}{1.302009in}}%
\pgfpathlineto{\pgfqpoint{3.542527in}{1.209609in}}%
\pgfpathclose%
\pgfusepath{fill}%
\end{pgfscope}%
\begin{pgfscope}%
\pgfpathrectangle{\pgfqpoint{3.319565in}{0.488889in}}{\pgfqpoint{0.360845in}{0.924000in}}%
\pgfusepath{clip}%
\pgfsetbuttcap%
\pgfsetmiterjoin%
\definecolor{currentfill}{rgb}{0.000000,0.368627,0.815686}%
\pgfsetfillcolor{currentfill}%
\pgfsetlinewidth{0.000000pt}%
\definecolor{currentstroke}{rgb}{0.000000,0.000000,0.000000}%
\pgfsetstrokecolor{currentstroke}%
\pgfsetstrokeopacity{0.000000}%
\pgfsetdash{}{0pt}%
\pgfpathmoveto{\pgfqpoint{3.542427in}{1.218849in}}%
\pgfpathlineto{\pgfqpoint{3.680411in}{1.218849in}}%
\pgfpathlineto{\pgfqpoint{3.680411in}{1.311249in}}%
\pgfpathlineto{\pgfqpoint{3.542427in}{1.311249in}}%
\pgfpathlineto{\pgfqpoint{3.542427in}{1.218849in}}%
\pgfpathclose%
\pgfusepath{fill}%
\end{pgfscope}%
\begin{pgfscope}%
\pgfpathrectangle{\pgfqpoint{3.319565in}{0.488889in}}{\pgfqpoint{0.360845in}{0.924000in}}%
\pgfusepath{clip}%
\pgfsetbuttcap%
\pgfsetmiterjoin%
\definecolor{currentfill}{rgb}{0.000000,0.364706,0.817647}%
\pgfsetfillcolor{currentfill}%
\pgfsetlinewidth{0.000000pt}%
\definecolor{currentstroke}{rgb}{0.000000,0.000000,0.000000}%
\pgfsetstrokecolor{currentstroke}%
\pgfsetstrokeopacity{0.000000}%
\pgfsetdash{}{0pt}%
\pgfpathmoveto{\pgfqpoint{3.541911in}{1.228089in}}%
\pgfpathlineto{\pgfqpoint{3.680411in}{1.228089in}}%
\pgfpathlineto{\pgfqpoint{3.680411in}{1.320489in}}%
\pgfpathlineto{\pgfqpoint{3.541911in}{1.320489in}}%
\pgfpathlineto{\pgfqpoint{3.541911in}{1.228089in}}%
\pgfpathclose%
\pgfusepath{fill}%
\end{pgfscope}%
\begin{pgfscope}%
\pgfpathrectangle{\pgfqpoint{3.319565in}{0.488889in}}{\pgfqpoint{0.360845in}{0.924000in}}%
\pgfusepath{clip}%
\pgfsetbuttcap%
\pgfsetmiterjoin%
\definecolor{currentfill}{rgb}{0.000000,0.356863,0.821569}%
\pgfsetfillcolor{currentfill}%
\pgfsetlinewidth{0.000000pt}%
\definecolor{currentstroke}{rgb}{0.000000,0.000000,0.000000}%
\pgfsetstrokecolor{currentstroke}%
\pgfsetstrokeopacity{0.000000}%
\pgfsetdash{}{0pt}%
\pgfpathmoveto{\pgfqpoint{3.541325in}{1.237329in}}%
\pgfpathlineto{\pgfqpoint{3.680411in}{1.237329in}}%
\pgfpathlineto{\pgfqpoint{3.680411in}{1.329729in}}%
\pgfpathlineto{\pgfqpoint{3.541325in}{1.329729in}}%
\pgfpathlineto{\pgfqpoint{3.541325in}{1.237329in}}%
\pgfpathclose%
\pgfusepath{fill}%
\end{pgfscope}%
\begin{pgfscope}%
\pgfpathrectangle{\pgfqpoint{3.319565in}{0.488889in}}{\pgfqpoint{0.360845in}{0.924000in}}%
\pgfusepath{clip}%
\pgfsetbuttcap%
\pgfsetmiterjoin%
\definecolor{currentfill}{rgb}{0.000000,0.352941,0.823529}%
\pgfsetfillcolor{currentfill}%
\pgfsetlinewidth{0.000000pt}%
\definecolor{currentstroke}{rgb}{0.000000,0.000000,0.000000}%
\pgfsetstrokecolor{currentstroke}%
\pgfsetstrokeopacity{0.000000}%
\pgfsetdash{}{0pt}%
\pgfpathmoveto{\pgfqpoint{3.540997in}{1.246569in}}%
\pgfpathlineto{\pgfqpoint{3.680411in}{1.246569in}}%
\pgfpathlineto{\pgfqpoint{3.680411in}{1.338969in}}%
\pgfpathlineto{\pgfqpoint{3.540997in}{1.338969in}}%
\pgfpathlineto{\pgfqpoint{3.540997in}{1.246569in}}%
\pgfpathclose%
\pgfusepath{fill}%
\end{pgfscope}%
\begin{pgfscope}%
\pgfpathrectangle{\pgfqpoint{3.319565in}{0.488889in}}{\pgfqpoint{0.360845in}{0.924000in}}%
\pgfusepath{clip}%
\pgfsetbuttcap%
\pgfsetmiterjoin%
\definecolor{currentfill}{rgb}{0.000000,0.352941,0.823529}%
\pgfsetfillcolor{currentfill}%
\pgfsetlinewidth{0.000000pt}%
\definecolor{currentstroke}{rgb}{0.000000,0.000000,0.000000}%
\pgfsetstrokecolor{currentstroke}%
\pgfsetstrokeopacity{0.000000}%
\pgfsetdash{}{0pt}%
\pgfpathmoveto{\pgfqpoint{3.540986in}{1.255809in}}%
\pgfpathlineto{\pgfqpoint{3.680411in}{1.255809in}}%
\pgfpathlineto{\pgfqpoint{3.680411in}{1.348209in}}%
\pgfpathlineto{\pgfqpoint{3.540986in}{1.348209in}}%
\pgfpathlineto{\pgfqpoint{3.540986in}{1.255809in}}%
\pgfpathclose%
\pgfusepath{fill}%
\end{pgfscope}%
\begin{pgfscope}%
\pgfpathrectangle{\pgfqpoint{3.319565in}{0.488889in}}{\pgfqpoint{0.360845in}{0.924000in}}%
\pgfusepath{clip}%
\pgfsetbuttcap%
\pgfsetmiterjoin%
\definecolor{currentfill}{rgb}{0.000000,0.352941,0.823529}%
\pgfsetfillcolor{currentfill}%
\pgfsetlinewidth{0.000000pt}%
\definecolor{currentstroke}{rgb}{0.000000,0.000000,0.000000}%
\pgfsetstrokecolor{currentstroke}%
\pgfsetstrokeopacity{0.000000}%
\pgfsetdash{}{0pt}%
\pgfpathmoveto{\pgfqpoint{3.540986in}{1.265049in}}%
\pgfpathlineto{\pgfqpoint{3.680411in}{1.265049in}}%
\pgfpathlineto{\pgfqpoint{3.680411in}{1.357449in}}%
\pgfpathlineto{\pgfqpoint{3.540986in}{1.357449in}}%
\pgfpathlineto{\pgfqpoint{3.540986in}{1.265049in}}%
\pgfpathclose%
\pgfusepath{fill}%
\end{pgfscope}%
\begin{pgfscope}%
\pgfpathrectangle{\pgfqpoint{3.319565in}{0.488889in}}{\pgfqpoint{0.360845in}{0.924000in}}%
\pgfusepath{clip}%
\pgfsetbuttcap%
\pgfsetmiterjoin%
\definecolor{currentfill}{rgb}{0.000000,0.341176,0.829412}%
\pgfsetfillcolor{currentfill}%
\pgfsetlinewidth{0.000000pt}%
\definecolor{currentstroke}{rgb}{0.000000,0.000000,0.000000}%
\pgfsetstrokecolor{currentstroke}%
\pgfsetstrokeopacity{0.000000}%
\pgfsetdash{}{0pt}%
\pgfpathmoveto{\pgfqpoint{3.540080in}{1.274289in}}%
\pgfpathlineto{\pgfqpoint{3.680411in}{1.274289in}}%
\pgfpathlineto{\pgfqpoint{3.680411in}{1.366689in}}%
\pgfpathlineto{\pgfqpoint{3.540080in}{1.366689in}}%
\pgfpathlineto{\pgfqpoint{3.540080in}{1.274289in}}%
\pgfpathclose%
\pgfusepath{fill}%
\end{pgfscope}%
\begin{pgfscope}%
\pgfpathrectangle{\pgfqpoint{3.319565in}{0.488889in}}{\pgfqpoint{0.360845in}{0.924000in}}%
\pgfusepath{clip}%
\pgfsetbuttcap%
\pgfsetmiterjoin%
\definecolor{currentfill}{rgb}{0.000000,0.333333,0.833333}%
\pgfsetfillcolor{currentfill}%
\pgfsetlinewidth{0.000000pt}%
\definecolor{currentstroke}{rgb}{0.000000,0.000000,0.000000}%
\pgfsetstrokecolor{currentstroke}%
\pgfsetstrokeopacity{0.000000}%
\pgfsetdash{}{0pt}%
\pgfpathmoveto{\pgfqpoint{3.539586in}{1.283529in}}%
\pgfpathlineto{\pgfqpoint{3.680411in}{1.283529in}}%
\pgfpathlineto{\pgfqpoint{3.680411in}{1.375929in}}%
\pgfpathlineto{\pgfqpoint{3.539586in}{1.375929in}}%
\pgfpathlineto{\pgfqpoint{3.539586in}{1.283529in}}%
\pgfpathclose%
\pgfusepath{fill}%
\end{pgfscope}%
\begin{pgfscope}%
\pgfpathrectangle{\pgfqpoint{3.319565in}{0.488889in}}{\pgfqpoint{0.360845in}{0.924000in}}%
\pgfusepath{clip}%
\pgfsetbuttcap%
\pgfsetmiterjoin%
\definecolor{currentfill}{rgb}{0.000000,0.333333,0.833333}%
\pgfsetfillcolor{currentfill}%
\pgfsetlinewidth{0.000000pt}%
\definecolor{currentstroke}{rgb}{0.000000,0.000000,0.000000}%
\pgfsetstrokecolor{currentstroke}%
\pgfsetstrokeopacity{0.000000}%
\pgfsetdash{}{0pt}%
\pgfpathmoveto{\pgfqpoint{3.539576in}{1.292769in}}%
\pgfpathlineto{\pgfqpoint{3.680411in}{1.292769in}}%
\pgfpathlineto{\pgfqpoint{3.680411in}{1.385169in}}%
\pgfpathlineto{\pgfqpoint{3.539576in}{1.385169in}}%
\pgfpathlineto{\pgfqpoint{3.539576in}{1.292769in}}%
\pgfpathclose%
\pgfusepath{fill}%
\end{pgfscope}%
\begin{pgfscope}%
\pgfpathrectangle{\pgfqpoint{3.319565in}{0.488889in}}{\pgfqpoint{0.360845in}{0.924000in}}%
\pgfusepath{clip}%
\pgfsetbuttcap%
\pgfsetmiterjoin%
\definecolor{currentfill}{rgb}{0.000000,0.333333,0.833333}%
\pgfsetfillcolor{currentfill}%
\pgfsetlinewidth{0.000000pt}%
\definecolor{currentstroke}{rgb}{0.000000,0.000000,0.000000}%
\pgfsetstrokecolor{currentstroke}%
\pgfsetstrokeopacity{0.000000}%
\pgfsetdash{}{0pt}%
\pgfpathmoveto{\pgfqpoint{3.539297in}{1.302009in}}%
\pgfpathlineto{\pgfqpoint{3.680411in}{1.302009in}}%
\pgfpathlineto{\pgfqpoint{3.680411in}{1.394409in}}%
\pgfpathlineto{\pgfqpoint{3.539297in}{1.394409in}}%
\pgfpathlineto{\pgfqpoint{3.539297in}{1.302009in}}%
\pgfpathclose%
\pgfusepath{fill}%
\end{pgfscope}%
\begin{pgfscope}%
\pgfpathrectangle{\pgfqpoint{3.319565in}{0.488889in}}{\pgfqpoint{0.360845in}{0.924000in}}%
\pgfusepath{clip}%
\pgfsetbuttcap%
\pgfsetmiterjoin%
\definecolor{currentfill}{rgb}{0.000000,0.333333,0.833333}%
\pgfsetfillcolor{currentfill}%
\pgfsetlinewidth{0.000000pt}%
\definecolor{currentstroke}{rgb}{0.000000,0.000000,0.000000}%
\pgfsetstrokecolor{currentstroke}%
\pgfsetstrokeopacity{0.000000}%
\pgfsetdash{}{0pt}%
\pgfpathmoveto{\pgfqpoint{3.539297in}{1.311249in}}%
\pgfpathlineto{\pgfqpoint{3.680411in}{1.311249in}}%
\pgfpathlineto{\pgfqpoint{3.680411in}{1.403649in}}%
\pgfpathlineto{\pgfqpoint{3.539297in}{1.403649in}}%
\pgfpathlineto{\pgfqpoint{3.539297in}{1.311249in}}%
\pgfpathclose%
\pgfusepath{fill}%
\end{pgfscope}%
\begin{pgfscope}%
\pgfpathrectangle{\pgfqpoint{3.319565in}{0.488889in}}{\pgfqpoint{0.360845in}{0.924000in}}%
\pgfusepath{clip}%
\pgfsetbuttcap%
\pgfsetmiterjoin%
\definecolor{currentfill}{rgb}{0.000000,0.313725,0.843137}%
\pgfsetfillcolor{currentfill}%
\pgfsetlinewidth{0.000000pt}%
\definecolor{currentstroke}{rgb}{0.000000,0.000000,0.000000}%
\pgfsetstrokecolor{currentstroke}%
\pgfsetstrokeopacity{0.000000}%
\pgfsetdash{}{0pt}%
\pgfpathmoveto{\pgfqpoint{3.537936in}{1.320489in}}%
\pgfpathlineto{\pgfqpoint{3.680411in}{1.320489in}}%
\pgfpathlineto{\pgfqpoint{3.680411in}{1.412889in}}%
\pgfpathlineto{\pgfqpoint{3.537936in}{1.412889in}}%
\pgfpathlineto{\pgfqpoint{3.537936in}{1.320489in}}%
\pgfpathclose%
\pgfusepath{fill}%
\end{pgfscope}%
\begin{pgfscope}%
\pgfpathrectangle{\pgfqpoint{3.319565in}{0.488889in}}{\pgfqpoint{0.360845in}{0.924000in}}%
\pgfusepath{clip}%
\pgfsetbuttcap%
\pgfsetmiterjoin%
\definecolor{currentfill}{rgb}{0.000000,0.301961,0.849020}%
\pgfsetfillcolor{currentfill}%
\pgfsetlinewidth{0.000000pt}%
\definecolor{currentstroke}{rgb}{0.000000,0.000000,0.000000}%
\pgfsetstrokecolor{currentstroke}%
\pgfsetstrokeopacity{0.000000}%
\pgfsetdash{}{0pt}%
\pgfpathmoveto{\pgfqpoint{3.536971in}{1.329729in}}%
\pgfpathlineto{\pgfqpoint{3.680411in}{1.329729in}}%
\pgfpathlineto{\pgfqpoint{3.680411in}{1.422129in}}%
\pgfpathlineto{\pgfqpoint{3.536971in}{1.422129in}}%
\pgfpathlineto{\pgfqpoint{3.536971in}{1.329729in}}%
\pgfpathclose%
\pgfusepath{fill}%
\end{pgfscope}%
\begin{pgfscope}%
\pgfpathrectangle{\pgfqpoint{3.319565in}{0.488889in}}{\pgfqpoint{0.360845in}{0.924000in}}%
\pgfusepath{clip}%
\pgfsetbuttcap%
\pgfsetmiterjoin%
\definecolor{currentfill}{rgb}{0.000000,0.290196,0.854902}%
\pgfsetfillcolor{currentfill}%
\pgfsetlinewidth{0.000000pt}%
\definecolor{currentstroke}{rgb}{0.000000,0.000000,0.000000}%
\pgfsetstrokecolor{currentstroke}%
\pgfsetstrokeopacity{0.000000}%
\pgfsetdash{}{0pt}%
\pgfpathmoveto{\pgfqpoint{3.535827in}{1.338969in}}%
\pgfpathlineto{\pgfqpoint{3.680411in}{1.338969in}}%
\pgfpathlineto{\pgfqpoint{3.680411in}{1.431369in}}%
\pgfpathlineto{\pgfqpoint{3.535827in}{1.431369in}}%
\pgfpathlineto{\pgfqpoint{3.535827in}{1.338969in}}%
\pgfpathclose%
\pgfusepath{fill}%
\end{pgfscope}%
\begin{pgfscope}%
\pgfpathrectangle{\pgfqpoint{3.319565in}{0.488889in}}{\pgfqpoint{0.360845in}{0.924000in}}%
\pgfusepath{clip}%
\pgfsetbuttcap%
\pgfsetmiterjoin%
\definecolor{currentfill}{rgb}{0.000000,0.286275,0.856863}%
\pgfsetfillcolor{currentfill}%
\pgfsetlinewidth{0.000000pt}%
\definecolor{currentstroke}{rgb}{0.000000,0.000000,0.000000}%
\pgfsetstrokecolor{currentstroke}%
\pgfsetstrokeopacity{0.000000}%
\pgfsetdash{}{0pt}%
\pgfpathmoveto{\pgfqpoint{3.535510in}{1.348209in}}%
\pgfpathlineto{\pgfqpoint{3.680411in}{1.348209in}}%
\pgfpathlineto{\pgfqpoint{3.680411in}{1.440609in}}%
\pgfpathlineto{\pgfqpoint{3.535510in}{1.440609in}}%
\pgfpathlineto{\pgfqpoint{3.535510in}{1.348209in}}%
\pgfpathclose%
\pgfusepath{fill}%
\end{pgfscope}%
\begin{pgfscope}%
\pgfpathrectangle{\pgfqpoint{3.319565in}{0.488889in}}{\pgfqpoint{0.360845in}{0.924000in}}%
\pgfusepath{clip}%
\pgfsetbuttcap%
\pgfsetmiterjoin%
\definecolor{currentfill}{rgb}{0.000000,0.243137,0.878431}%
\pgfsetfillcolor{currentfill}%
\pgfsetlinewidth{0.000000pt}%
\definecolor{currentstroke}{rgb}{0.000000,0.000000,0.000000}%
\pgfsetstrokecolor{currentstroke}%
\pgfsetstrokeopacity{0.000000}%
\pgfsetdash{}{0pt}%
\pgfpathmoveto{\pgfqpoint{3.532156in}{1.357449in}}%
\pgfpathlineto{\pgfqpoint{3.680411in}{1.357449in}}%
\pgfpathlineto{\pgfqpoint{3.680411in}{1.449849in}}%
\pgfpathlineto{\pgfqpoint{3.532156in}{1.449849in}}%
\pgfpathlineto{\pgfqpoint{3.532156in}{1.357449in}}%
\pgfpathclose%
\pgfusepath{fill}%
\end{pgfscope}%
\begin{pgfscope}%
\pgfpathrectangle{\pgfqpoint{3.319565in}{0.488889in}}{\pgfqpoint{0.360845in}{0.924000in}}%
\pgfusepath{clip}%
\pgfsetbuttcap%
\pgfsetmiterjoin%
\definecolor{currentfill}{rgb}{0.000000,0.243137,0.878431}%
\pgfsetfillcolor{currentfill}%
\pgfsetlinewidth{0.000000pt}%
\definecolor{currentstroke}{rgb}{0.000000,0.000000,0.000000}%
\pgfsetstrokecolor{currentstroke}%
\pgfsetstrokeopacity{0.000000}%
\pgfsetdash{}{0pt}%
\pgfpathmoveto{\pgfqpoint{3.532156in}{1.366689in}}%
\pgfpathlineto{\pgfqpoint{3.680411in}{1.366689in}}%
\pgfpathlineto{\pgfqpoint{3.680411in}{1.459089in}}%
\pgfpathlineto{\pgfqpoint{3.532156in}{1.459089in}}%
\pgfpathlineto{\pgfqpoint{3.532156in}{1.366689in}}%
\pgfpathclose%
\pgfusepath{fill}%
\end{pgfscope}%
\begin{pgfscope}%
\pgfpathrectangle{\pgfqpoint{3.319565in}{0.488889in}}{\pgfqpoint{0.360845in}{0.924000in}}%
\pgfusepath{clip}%
\pgfsetbuttcap%
\pgfsetmiterjoin%
\definecolor{currentfill}{rgb}{0.000000,0.219608,0.890196}%
\pgfsetfillcolor{currentfill}%
\pgfsetlinewidth{0.000000pt}%
\definecolor{currentstroke}{rgb}{0.000000,0.000000,0.000000}%
\pgfsetstrokecolor{currentstroke}%
\pgfsetstrokeopacity{0.000000}%
\pgfsetdash{}{0pt}%
\pgfpathmoveto{\pgfqpoint{3.530237in}{1.375929in}}%
\pgfpathlineto{\pgfqpoint{3.680411in}{1.375929in}}%
\pgfpathlineto{\pgfqpoint{3.680411in}{1.468329in}}%
\pgfpathlineto{\pgfqpoint{3.530237in}{1.468329in}}%
\pgfpathlineto{\pgfqpoint{3.530237in}{1.375929in}}%
\pgfpathclose%
\pgfusepath{fill}%
\end{pgfscope}%
\begin{pgfscope}%
\pgfpathrectangle{\pgfqpoint{3.319565in}{0.488889in}}{\pgfqpoint{0.360845in}{0.924000in}}%
\pgfusepath{clip}%
\pgfsetbuttcap%
\pgfsetmiterjoin%
\definecolor{currentfill}{rgb}{0.000000,0.196078,0.901961}%
\pgfsetfillcolor{currentfill}%
\pgfsetlinewidth{0.000000pt}%
\definecolor{currentstroke}{rgb}{0.000000,0.000000,0.000000}%
\pgfsetstrokecolor{currentstroke}%
\pgfsetstrokeopacity{0.000000}%
\pgfsetdash{}{0pt}%
\pgfpathmoveto{\pgfqpoint{3.528298in}{1.385169in}}%
\pgfpathlineto{\pgfqpoint{3.680411in}{1.385169in}}%
\pgfpathlineto{\pgfqpoint{3.680411in}{1.477569in}}%
\pgfpathlineto{\pgfqpoint{3.528298in}{1.477569in}}%
\pgfpathlineto{\pgfqpoint{3.528298in}{1.385169in}}%
\pgfpathclose%
\pgfusepath{fill}%
\end{pgfscope}%
\begin{pgfscope}%
\pgfpathrectangle{\pgfqpoint{3.319565in}{0.488889in}}{\pgfqpoint{0.360845in}{0.924000in}}%
\pgfusepath{clip}%
\pgfsetbuttcap%
\pgfsetmiterjoin%
\definecolor{currentfill}{rgb}{0.000000,0.196078,0.901961}%
\pgfsetfillcolor{currentfill}%
\pgfsetlinewidth{0.000000pt}%
\definecolor{currentstroke}{rgb}{0.000000,0.000000,0.000000}%
\pgfsetstrokecolor{currentstroke}%
\pgfsetstrokeopacity{0.000000}%
\pgfsetdash{}{0pt}%
\pgfpathmoveto{\pgfqpoint{3.528298in}{1.394409in}}%
\pgfpathlineto{\pgfqpoint{3.680411in}{1.394409in}}%
\pgfpathlineto{\pgfqpoint{3.680411in}{1.486809in}}%
\pgfpathlineto{\pgfqpoint{3.528298in}{1.486809in}}%
\pgfpathlineto{\pgfqpoint{3.528298in}{1.394409in}}%
\pgfpathclose%
\pgfusepath{fill}%
\end{pgfscope}%
\begin{pgfscope}%
\pgfpathrectangle{\pgfqpoint{3.319565in}{0.488889in}}{\pgfqpoint{0.360845in}{0.924000in}}%
\pgfusepath{clip}%
\pgfsetbuttcap%
\pgfsetmiterjoin%
\definecolor{currentfill}{rgb}{0.000000,0.000000,1.000000}%
\pgfsetfillcolor{currentfill}%
\pgfsetlinewidth{0.000000pt}%
\definecolor{currentstroke}{rgb}{0.000000,0.000000,0.000000}%
\pgfsetstrokecolor{currentstroke}%
\pgfsetstrokeopacity{0.000000}%
\pgfsetdash{}{0pt}%
\pgfpathmoveto{\pgfqpoint{3.512377in}{1.403649in}}%
\pgfpathlineto{\pgfqpoint{3.680411in}{1.403649in}}%
\pgfpathlineto{\pgfqpoint{3.680411in}{1.496049in}}%
\pgfpathlineto{\pgfqpoint{3.512377in}{1.496049in}}%
\pgfpathlineto{\pgfqpoint{3.512377in}{1.403649in}}%
\pgfpathclose%
\pgfusepath{fill}%
\end{pgfscope}%
\begin{pgfscope}%
\definecolor{textcolor}{rgb}{0.000000,0.000000,0.000000}%
\pgfsetstrokecolor{textcolor}%
\pgfsetfillcolor{textcolor}%
\pgftext[x=3.471121in,y=1.468329in,left,base]{\color{textcolor}\setmainfont{Lato}\rmfamily\fontsize{7.000000}{8.400000}\selectfont 46.6\%}%
\end{pgfscope}%
\begin{pgfscope}%
\definecolor{textcolor}{rgb}{0.000000,0.000000,0.000000}%
\pgfsetstrokecolor{textcolor}%
\pgfsetfillcolor{textcolor}%
\pgftext[x=3.471121in,y=0.359529in,left,base]{\color{textcolor}\setmainfont{Lato}\rmfamily\fontsize{7.000000}{8.400000}\selectfont 24.1\%}%
\end{pgfscope}%
\end{pgfpicture}%
\makeatother%
\endgroup%


\vspace{-4mm}
\footnotesize{Source: American Community Survey, Dorn, Author's Calculations}
\end{minipage} \hspace{3mm}
\begin{minipage}{0.24\textwidth}
\vspace{-1mm}

\footnotesize \input{text/acs_ftfy.txt}
\end{minipage}
\newpage
\vspace*{-6mm}
\begin{minipage}{0.76\textwidth}
\index{prices!consumer price index}
\small The prices of some items are more volatile than others. Food and energy prices, for example, are sometimes separated from the rest of the CPI basket, which is referred to as the \textit{core}, because swings in food and energy prices are larger and more frequent. 

Core inflation includes core goods, core services other than shelter, and shelter. Core goods inflation was barely existent from 2013 through the start of the pandemic. Core goods prices are disproportionately affected by import prices and by changes in the quality of goods, for example from technological improvement. In contrast, domestic wage growth affects the prices for core services more than the other categories. Shelter prices are affected by housing supply and construction. 

\input{text/cpi_decomp.txt}
\vspace{1mm}

\normalsize \textbf{CPI Decomposition}\\
\footnotesize{\textit{contribution to one-year change in CPI-U, percentage points}}
\vspace{4.5cm}

\hspace{3mm} \begin{tikzpicture}[overlay]
	\begin{axis}[\bbar{y}{0}, \tydateaxisticks ytick={-4, -2, 0, 2, 4, 6, 8}, 
		height=6.5cm, width=12.9cm, ymax=7.5, enlarge x limits={0.01}, 
		yticklabel style={text width=1.2em}, clip=false,
        xticklabel={`\short{\year}}, legend cell align={left},
		legend style={legend columns=2, at={(0.53, 0.98)}}]
	\rbar
	\ctsbar{blue!85!black}{date}{core_goods}{data/cpi_decomp.csv}{2.4pt}
	\ctsbar{green!60!black}{date}{core_services}{data/cpi_decomp.csv}{2.4pt}
	\ctsbar{cyan!50!white}{date}{shelter}{data/cpi_decomp.csv}{2.4pt}
	\ctsbar{orange!80!red}{date}{food_energy}{data/cpi_decomp.csv}{2.4pt}
	\legend{Core Goods, Core Services ex. Shelter, Shelter, Food \& Energy};
	\end{axis}
\end{tikzpicture}

\footnotesize{Source: Bureau of Labor Statistics, Author's Calculations} \hfill \tbllink{cpi_decomp.csv}
\vspace{2mm}

\subsubsection*{Relative Prices}
\small Some prices increase faster or slower than others. Additionally, the basket of goods used to calculate the CPI is based on average spending patterns across individuals. At a given point, individuals may dedicate a large share of spending to a certain categories or have no expenses at all in a category. For example, day care costs are paid generally only for a few years of a child's life and only some households contain day-care-age children. But within those households, day care is a large share of overall spending. 

One-year inflation rates for different categories of goods and services, including some smaller categories, are captured in the following section and table. The table also shows cumulative price changes since February 2020, the last month of data before the COVID-19 pandemic shutdown in the US. Additionally, the weight that a category has in the overall index--the category's share of the basket of goods and services used to calculate the CPI--is included as the last column in the table. This weight comes from each category's share of overall consumer spending during the most recent reference period, and is updated by changes in prices since the reference period. 
\end{minipage}
\newpage
\printindex
\end{document}
