% % % % % % % % % % % % % % 
%
%	U.S. Chartbook
%	Brian W. Dew (brianwdew@gmail.com)
%	Updated: December 15, 2019
%	GitHub repo contains to do list (issues)
%   https://github.com/bdecon/US-chartbook
%
% % % % % % % % % % % % % %
\PassOptionsToPackage{table}{xcolor}
\documentclass{report}

%
% % % % % % Packages % % % % % % % % % 
%
	
	\usepackage[letterpaper, margin=1.18in]{geometry}
	\usepackage{microtype}
	\usepackage[default]{lato}
	\usepackage{pgfplots, pgfplotstable}
	\usepackage[eulergreek]{sansmath}
	\usepackage{xcolor}
	\usepackage{array}
	\usepackage{fontawesome5}
	\usepackage{titlesec}
	\usepackage{fancyhdr}
	\usepackage[colorlinks, linkcolor=blue, filecolor=blue, 
		citecolor=blue, urlcolor=blue, linktoc=all, 
		pdfencoding=auto]{hyperref}
	\usetikzlibrary{pgfplots.dateplot, pgfplots.fillbetween, patterns}

%
% % % % % Document Settings % % % % % % % 
%

	% Paragraph spacing
	\usepackage{parskip}
	\setlength\parindent{0pt}
	\setlength{\parskip}{8pt}
	\makeatletter
		\newcommand{\@minipagerestore}{\setlength{\parskip}{8pt}}
	\makeatother
	
	% Section and Subsection Headings
	\titleformat{\section}
  		{\color{darkgray} \LARGE \seriffont \bfseries}
  		{\thesection}{1em}{}
	\titleformat{\subsection}
  		{\color{black!70} \seriffont \bfseries \large}
  		{\thesection}{1em}{}
	\titleformat{\subsubsection}
  		{\color{black!70} \seriffont \bfseries \normalsize}
  		{\thesection}{1em}{}		
%
% % % % % Graph Settings % % % % % % % 
%
	
	% Header and footer
	
	\pagestyle{fancy}
	\fancyhf{}
	\renewcommand{\headrulewidth}{0pt}
	\rfoot{\hyperlink{toc}{\faList}}
	\cfoot{\thepage}	
	
	
	% Color square
	\newcommand{\cbox}[1]{
		\begin{tikzpicture} \draw [#1, line width=6](0,0) -- (.2,0);  
		\end{tikzpicture}}
	\newcommand{\colorline}[2]{
		\begin{tikzpicture} \draw [#1, line width=1.8](0,0.2) -- +(0.6,0) node[right, black!80] {#2}; 
		\end{tikzpicture}}
		
	% Table link
	\newcommand{\tbllink}[1]{\href{https://raw.githubusercontent.com/bdecon/US-chartbook/master/chartbook/data/#1}{\faTable}}
	
	% Last two digits of year
	\makeatletter
	\newcommand*\short[1]{\expandafter\@gobbletwo\number\numexpr#1\relax}
	\makeatother	
	
	% Column width and alignment
	\newcolumntype{R}[1]{>{\raggedleft\let\newline\\\arraybackslash\hspace{0pt}}m{#1}}	
	
	% Style for date plots
	\pgfplotsset{compat=newest, 
		scaled y ticks=false,
		axis line style={black!20}, 
		xtick style={black!20}, ytick style={draw=none},
		every tick label/.style={black!50, font=\scriptsize,
			/pgf/number format/assume math mode=true},
		width=12.8cm, height=4.8cm, 
		xticklabel style={align=left}, 
		yticklabel style={text width=0.9em, align=right},       
		axis x line*=bottom, x axis line style={black!50},
	    axis y line=left, y axis line style={opacity=0},
	    ymajorgrids, grid style={very thin, black!10},	        
	    every node near coord/.style={/pgf/number format/fixed,
	    	font=\scriptsize, style={black!70}},
	    legend style={legend columns=-1, draw=none, fill=none,
	    	/tikz/every even column/.append style={column sep=0.3cm}}}
	    	
	
	% stacked diverging bar
	\newcommand{\sbar}[4]{
		\addplot[ybar stacked, bar width=2.45pt, draw opacity=0, fill=#1] 
			table [x=#2, y=#3, col sep=comma]{#4};}
			
	% thin stacked diverging bar
	\newcommand{\tsbar}[4]{
		\addplot[ybar stacked, bar width=2.2pt, draw opacity=0, fill=#1] 
			table [x=#2, y=#3, col sep=comma]{#4};}
			
	% custom width stacked diverging bar
	\newcommand{\ctsbar}[5]{
		\addplot[ybar stacked, bar width=#5, draw opacity=0, fill=#1] 
			table [x=#2, y=#3, col sep=comma]{#4};}
			
	% area plot segment
	\newcommand{\abar}[4]{
		\addplot[stack plots=y, area style, draw=none, fill=#1] 
			table [x=#2, y=#3, col sep=comma]{#4}\closedcycle;}
					
	% text node
	\newcommand{\stdnode}[3]{\node[below, align=left, shift=({#1,#2})]{#3};}	
	
	% text node located by data	 
	\newcommand{\absnode}[3]{\node[below right, align=left] at (axis cs: #1,#2) {#3};}        
		        
	% Date (X) Axis Tick Marks, one tick per year, every even year labeled
	\newcommand{\dateaxisticks}{
		date coordinates in=x, axis line style={draw=none},
		xmax={2022-02-28},
		max space between ticks=40,	    
		xtick={{1990-01-01}, {1992-01-01}, {1994-01-01}, 
			{1996-01-01}, {1998-01-01}, {2000-01-01}, 
			{2002-01-01}, {2004-01-01}, {2006-01-01},
			{2008-01-01}, {2010-01-01}, {2012-01-01}, {2014-01-01},
		    {2016-01-01}, {2018-01-01}, {2020-01-01}, {2022-01-01}, 
		    {2024-01-01}, {2026-01-01}},
		minor xtick={{1989-01-01}, {1991-01-01}, {1993-01-01},
			{1995-01-01}, {1997-01-01}, {1999-01-01}, 
			{2001-01-01}, {2003-01-01}, {2005-01-01}, {2007-01-01},
		    {2009-01-01}, {2011-01-01}, {2013-01-01}, {2015-01-01},
		    {2017-01-01}, {2019-01-01}, {2021-01-01}, {2023-01-01}, 
		    {2025-01-01}, {2027-01-01}},
		enlarge y limits={0.06}, enlarge x limits={0.01},
		}
		
	% Date (X) Axis Tick Marks, one tick per year, every even year labeled
	\newcommand{\shdateaxisticks}{
		date coordinates in=x, axis line style={draw=none},
		xmax={2022-02-28},
		max space between ticks=40,	    
		xtick={{1990-01-01}, {1995-01-01}, {2000-01-01}, 
			{2005-01-01}, {2010-01-01}, {2015-01-01}, {2020-01-01}},
		minor xtick={},
		enlarge y limits={0.06}, enlarge x limits={0.01},
		}
		
	% Date (X) Axis Tick Marks, one tick per year, every even year labeled
	\newcommand{\ltdateaxisticks}{
		date coordinates in=x, axis line style={draw=none},
		xmax={2022-02-28},
		max space between ticks=40,	    
		xtick={{2015-01-01}, {2016-01-01}, {2017-01-01}, {2018-01-01}, 
		    {2019-01-01}, {2020-01-01}, {2021-01-01}, {2022-01-01}},
		enlarge y limits={0.06}, enlarge x limits={0.01},
		}
		
	% Date (X) Axis Tick Marks, one tick per year, every even year labeled
	\newcommand{\lfdateaxisticks}{
		date coordinates in=x, axis line style={draw=none},
		xmin={2018-01-01}, xmax={2022-02-28},
		max space between ticks=40,	    
		xtick={{2018-01-01}, {2019-01-01}, {2020-01-01}, {2021-01-01}, {2022-01-01}},
		enlarge y limits={value=0.06, upper}, enlarge x limits={0.02}, ymin=0,
		height=6.2cm, width=6.7cm,
		}
		
	% Date (X) Axis Tick Marks, one tick per year, every even year labeled
	\newcommand{\tydateaxisticks}{
		date coordinates in=x, axis line style={draw=none},
		xmax={2022-02-28}, max space between ticks=40,	    
		xtick={{2011-01-01}, {2012-01-01}, {2013-01-01}, {2014-01-01}, {2015-01-01}, {2016-01-01}, 
			{2017-01-01}, {2018-01-01}, {2019-01-01}, {2020-01-01}, {2021-01-01}, {2022-01-01}},
		enlarge y limits={0.06}, enlarge x limits={0.01},
		}
		
	% Settings for y label text in horizontal bar charts
	\newcommand{\barylab}[2]{yticklabel style={text width=#1, align=right, 
		style={black!70}, text height=#2},}
	
	% Solid bars at significant  x or y values
	\newcommand{\bbar}[2]{extra #1 ticks = {{#2}}, extra #1 tick labels = ,
		extra #1 tick style = {grid=major, grid style={thick, black!25}},}
		
			% Solid bars at significant  x or y values
	\newcommand{\dbar}[2]{extra #1 ticks = {{#2}}, extra #1 tick labels = ,
		extra #1 tick style = {grid=major, grid style={dashed, thick, black!50}},}
		
	% Standard line
	\newcommand{\stdline}[4]{\addplot[very thick, no markers, color=#1] 
		table [x=#2, y=#3, col sep=comma] {#4};	}
		
	% Thin line
	\newcommand{\thinline}[4]{\addplot[no markers, color=#1] 
		table [x=#2, y=#3, col sep=comma] {#4};	}
		
	% Dashed line
	\newcommand{\dashline}[4]{\addplot[very thick, dashed, no markers, color=#1] 
		table [x=#2, y=#3, col sep=comma] {#4};	}
		
	% Thicker line
	\newcommand{\thickline}[4]{\addplot[ultra thick, no markers, color=#1] 
		table [x=#2, y=#3, col sep=comma] {#4};	}
		
	% Style for bar plots legend symbol		
		\pgfplotsset{/pgfplots/area legend/.style={/pgfplots/legend image code/.code={
            \fill[##1] (0cm, -0.1cm) rectangle (0.6cm, 0.1cm);}},}		
		
	% Additional bar plot settings
	\newcommand{\barplotnogrid}{xbar=0pt, axis line style={draw=none},
	    yticklabel style={align=left, anchor=east},
      		xmajorticks=false, ymajorgrids=false,   
	    ytick=data, tickwidth=0pt, area legend, reverse legend,
	    nodes near coords align={horizontal},}  
		
	% Recession bars		
	\newcommand{\rbars}{
		\fill[color=black!10] (axis cs:{1990-07-01},\pgfkeysvalueof{/pgfplots/ymin}) rectangle 
			(axis cs:{1991-03-01}, \pgfkeysvalueof{/pgfplots/ymax});
		\fill[color=black!10] (axis cs:{2007-12-01},\pgfkeysvalueof{/pgfplots/ymin}) rectangle 
			(axis cs:{2009-07-01}, \pgfkeysvalueof{/pgfplots/ymax});
		\fill[color=black!10] (axis cs:{2001-03-01},\pgfkeysvalueof{/pgfplots/ymin}) rectangle 
			(axis cs:{2001-11-01}, \pgfkeysvalueof{/pgfplots/ymax});
		\fill[color=black!10] (axis cs:{2020-02-01},\pgfkeysvalueof{/pgfplots/ymin}) rectangle 
			(axis cs:{2020-05-01}, \pgfkeysvalueof{/pgfplots/ymax});}
			
	\newcommand{\rebars}{
		\fill[color=black!10] (axis cs:{2007-12-01},\pgfkeysvalueof{/pgfplots/ymin}) rectangle 
			(axis cs:{2009-07-01}, \pgfkeysvalueof{/pgfplots/ymax});
		\fill[color=black!10] (axis cs:{2001-03-01},\pgfkeysvalueof{/pgfplots/ymin}) rectangle 
			(axis cs:{2001-11-01}, \pgfkeysvalueof{/pgfplots/ymax});
		\fill[color=black!10] (axis cs:{2020-02-01},\pgfkeysvalueof{/pgfplots/ymin}) rectangle 
			(axis cs:{2020-05-01}, \pgfkeysvalueof{/pgfplots/ymax});}
			
	\newcommand{\recbars}{
		\fill[color=black!10] (axis cs:{2007-12-01},\pgfkeysvalueof{/pgfplots/ymin}) rectangle 
			(axis cs:{2009-07-01}, \pgfkeysvalueof{/pgfplots/ymax});
		\fill[color=black!10] (axis cs:{2020-02-01},\pgfkeysvalueof{/pgfplots/ymin}) rectangle 
			(axis cs:{2020-05-01}, \pgfkeysvalueof{/pgfplots/ymax});}
			
	\newcommand{\rbar}{
		\fill[color=black!10] (axis cs:{2020-02-01},\pgfkeysvalueof{/pgfplots/ymin}) rectangle 
			(axis cs:{2020-05-01}, \pgfkeysvalueof{/pgfplots/ymax});}
	
	\newfontfamily\seriffont{RobotoSlab}	
	
	\pgfplotstableread[header=true, col sep=comma]{data/cpi_comp.csv}\cpi
	\pgfplotstableread[header=true, col sep=semicolon]{data/ip_comp.csv}\ip
	\pgfplotstableread[header=true, col sep=comma]{data/rs_comp.csv}\rs
	\pgfplotstableread[header=true, col sep=comma]{data/ahe_ind.csv}\ahe
	\pgfplotstableread[header=true, col sep=comma]{data/poor.csv}\poor
	\pgfplotstableread[header=true, col sep=comma]{data/poor2.csv}\pvrt
	\pgfplotstableread[header=true, col sep=comma]{data/spmtbl20.csv}\spm
	\pgfplotstableread[header=true, col sep=semicolon]{data/occs.csv}\occ
	\pgfplotstableread[header=true, col sep=comma]{data/empgroups.csv}\emp
	\pgfplotstableread[header=true, col sep=comma]{data/empgroups2.csv}\empt
	\pgfplotstableread[header=true, col sep=comma]{data/unempgroups.csv}\unemp
	\pgfplotstableread[header=true, col sep=comma]{data/unempgroups2.csv}\unempt
	\pgfplotstableread[header=true, col sep=comma]{data/unempgroups3.csv}\unemptt
	\pgfplotstableread[header=true, col sep=semicolon]{data/cps_educ.csv}\edsh
	\pgfplotstableread[header=true, col sep=comma]{data/cps_educ_tot.csv}\edtot
	\pgfplotstableread[header=true, col sep=comma]{data/cps_age.csv}\agesh
	\pgfplotstableread[header=true, col sep=semicolon]{data/union_ind.csv}\unmem
	\pgfplotstableread[header=true, col sep=semicolon]{data/quits_ind.csv}\quits
	\pgfplotstableread[header=true, col sep=semicolon]{data/state_pa_epop.csv}\paepop
	\pgfplotstableread[header=true, col sep=semicolon]{data/state_pa_epop2.csv}\paepopt
	\pgfplotstableread[header=true, col sep=semicolon]{data/state_pa_epop3.csv}\paepoptt
	\pgfplotstableread[header=true, col sep=semicolon]{data/openings_ind.csv}\opens
	\pgfplotstableread[header=true, col sep=comma]{data/nilf_comp.csv}\nilf
	\pgfplotstableread[header=true, col sep=comma]{data/pinc.csv}\pinc
	\pgfplotstableread[header=true, col sep=comma]{data/unemp_grp.csv}\ungrp
	\pgfplotstableread[header=true, col sep=comma]{data/unemp_grpsh.csv}\ungrpsh
	\pgfplotstableread[header=true, col sep=comma]{data/ce_age.csv}\ceage
	\pgfplotstableread[header=true, col sep=comma]{data/ce_inc.csv}\ceinc
	\pgfplotstableread[header=true, col sep=comma]{data/cpi_monthly.csv}\cpimo
	\pgfplotstableread[header=true, col sep=comma]{data/ccdebtbar.csv}\ccbar
	
	% Required for bar plots with individual bar colors for categories
	\pgfplotsset{discard if not/.style 2 args={
        x filter/.code={
            \edef\tempa{\thisrow{#1}}
            \edef\tempb{#2}
            \ifx\tempa\tempb
            \else
                \def\pgfmathresult{inf}
            \fi}}}	
	
% % % % % % % %
%
%  Begin Document
%
% % % % % % % %		
\begin{document}
\newpage
\begin{minipage}{0.76\textwidth}
\normalsize Manufacturing\\
\hspace*{-2mm} \begin{tikzpicture}
	\begin{axis}[\bbar{y}{0}, \ltdateaxisticks
		yticklabel style={text width=1.5em}, enlarge y limits={0.1},
		xticklabel={`\short{\year}}, width=4.6cm, height=3.4cm,
		clip=false]
	\rbar
	\thickline{blue!60!black}{date}{Man}{data/ces_ind_sh.csv}
	\end{axis}
\end{tikzpicture}\\
%\footnotesize{Source: Federal Reserve} \hfill \tbllink{msa_unemp_rate.csv}
\end{minipage}
\newpage
\begin{minipage}{0.76\textwidth}
\subsubsection*{Unemployment by Metro Area}
\vspace{-1mm}

\small The Bureau of Labor Statistics \href{https://www.bls.gov/lau/}{produce} local area estimates of unemployment, including the \textbf{unemployment rate for metro areas}. The following map shows changes since 2019 in metro area unemployment rates. An increase in the unemployment rate is shown by a blue circle and a decrease is shown by a light green circle; circle size is the magnitude of the change. 

\input{text/msa_unemp_ch_n.txt}
\end{minipage}
\vspace{1mm}

\begin{minipage}{0.82\textwidth}
\normalsize \textbf{Change in Unemployment Rate by Metro Area}\\
\footnotesize{\textit{from \input{text/unemp_map_date.txt}\unskip, percentage points}}
\vspace{-4mm}

\hspace*{-8mm} %% Creator: Matplotlib, PGF backend
%%
%% To include the figure in your LaTeX document, write
%%   \input{<filename>.pgf}
%%
%% Make sure the required packages are loaded in your preamble
%%   \usepackage{pgf}
%%
%% Figures using additional raster images can only be included by \input if
%% they are in the same directory as the main LaTeX file. For loading figures
%% from other directories you can use the `import` package
%%   \usepackage{import}
%%
%% and then include the figures with
%%   \import{<path to file>}{<filename>.pgf}
%%
%% Matplotlib used the following preamble
%%   \usepackage{fontspec}
%%   \setmainfont{DejaVuSerif.ttf}[Path=\detokenize{/home/brian/miniconda3/lib/python3.8/site-packages/matplotlib/mpl-data/fonts/ttf/}]
%%   \setsansfont{DejaVuSans.ttf}[Path=\detokenize{/home/brian/miniconda3/lib/python3.8/site-packages/matplotlib/mpl-data/fonts/ttf/}]
%%   \setmonofont{DejaVuSansMono.ttf}[Path=\detokenize{/home/brian/miniconda3/lib/python3.8/site-packages/matplotlib/mpl-data/fonts/ttf/}]
%%
\begingroup%
\makeatletter%
\begin{pgfpicture}%
\pgfpathrectangle{\pgfpointorigin}{\pgfqpoint{5.237500in}{5.940271in}}%
\pgfusepath{use as bounding box, clip}%
\begin{pgfscope}%
\pgfsetbuttcap%
\pgfsetmiterjoin%
\pgfsetlinewidth{0.000000pt}%
\definecolor{currentstroke}{rgb}{1.000000,1.000000,1.000000}%
\pgfsetstrokecolor{currentstroke}%
\pgfsetstrokeopacity{0.000000}%
\pgfsetdash{}{0pt}%
\pgfpathmoveto{\pgfqpoint{0.000000in}{0.000000in}}%
\pgfpathlineto{\pgfqpoint{5.237500in}{0.000000in}}%
\pgfpathlineto{\pgfqpoint{5.237500in}{5.940271in}}%
\pgfpathlineto{\pgfqpoint{0.000000in}{5.940271in}}%
\pgfpathclose%
\pgfusepath{}%
\end{pgfscope}%
\begin{pgfscope}%
\pgfpathrectangle{\pgfqpoint{0.100000in}{2.413063in}}{\pgfqpoint{5.037500in}{3.427208in}}%
\pgfusepath{clip}%
\pgfsetbuttcap%
\pgfsetmiterjoin%
\definecolor{currentfill}{rgb}{1.000000,1.000000,1.000000}%
\pgfsetfillcolor{currentfill}%
\pgfsetlinewidth{0.501875pt}%
\definecolor{currentstroke}{rgb}{0.827451,0.827451,0.827451}%
\pgfsetstrokecolor{currentstroke}%
\pgfsetdash{}{0pt}%
\pgfpathmoveto{\pgfqpoint{1.635246in}{3.264091in}}%
\pgfpathlineto{\pgfqpoint{1.620137in}{3.264339in}}%
\pgfpathlineto{\pgfqpoint{1.609854in}{3.275720in}}%
\pgfpathlineto{\pgfqpoint{1.602861in}{3.305947in}}%
\pgfpathlineto{\pgfqpoint{1.619112in}{3.309734in}}%
\pgfpathlineto{\pgfqpoint{1.635621in}{3.305661in}}%
\pgfpathlineto{\pgfqpoint{1.652422in}{3.293753in}}%
\pgfpathlineto{\pgfqpoint{1.652414in}{3.277285in}}%
\pgfpathclose%
\pgfusepath{stroke,fill}%
\end{pgfscope}%
\begin{pgfscope}%
\pgfpathrectangle{\pgfqpoint{0.100000in}{2.413063in}}{\pgfqpoint{5.037500in}{3.427208in}}%
\pgfusepath{clip}%
\pgfsetbuttcap%
\pgfsetmiterjoin%
\definecolor{currentfill}{rgb}{1.000000,1.000000,1.000000}%
\pgfsetfillcolor{currentfill}%
\pgfsetlinewidth{0.501875pt}%
\definecolor{currentstroke}{rgb}{0.827451,0.827451,0.827451}%
\pgfsetstrokecolor{currentstroke}%
\pgfsetdash{}{0pt}%
\pgfpathmoveto{\pgfqpoint{1.727210in}{3.084640in}}%
\pgfpathlineto{\pgfqpoint{1.710308in}{3.091076in}}%
\pgfpathlineto{\pgfqpoint{1.710230in}{3.103701in}}%
\pgfpathlineto{\pgfqpoint{1.689811in}{3.114378in}}%
\pgfpathlineto{\pgfqpoint{1.692298in}{3.141261in}}%
\pgfpathlineto{\pgfqpoint{1.698212in}{3.153947in}}%
\pgfpathlineto{\pgfqpoint{1.710698in}{3.142271in}}%
\pgfpathlineto{\pgfqpoint{1.731004in}{3.144222in}}%
\pgfpathlineto{\pgfqpoint{1.725876in}{3.107065in}}%
\pgfpathclose%
\pgfusepath{stroke,fill}%
\end{pgfscope}%
\begin{pgfscope}%
\pgfpathrectangle{\pgfqpoint{0.100000in}{2.413063in}}{\pgfqpoint{5.037500in}{3.427208in}}%
\pgfusepath{clip}%
\pgfsetbuttcap%
\pgfsetmiterjoin%
\definecolor{currentfill}{rgb}{1.000000,1.000000,1.000000}%
\pgfsetfillcolor{currentfill}%
\pgfsetlinewidth{0.501875pt}%
\definecolor{currentstroke}{rgb}{0.827451,0.827451,0.827451}%
\pgfsetstrokecolor{currentstroke}%
\pgfsetdash{}{0pt}%
\pgfpathmoveto{\pgfqpoint{1.798159in}{3.001762in}}%
\pgfpathlineto{\pgfqpoint{1.776708in}{3.005415in}}%
\pgfpathlineto{\pgfqpoint{1.764847in}{3.023779in}}%
\pgfpathlineto{\pgfqpoint{1.763172in}{3.038878in}}%
\pgfpathclose%
\pgfusepath{stroke,fill}%
\end{pgfscope}%
\begin{pgfscope}%
\pgfpathrectangle{\pgfqpoint{0.100000in}{2.413063in}}{\pgfqpoint{5.037500in}{3.427208in}}%
\pgfusepath{clip}%
\pgfsetbuttcap%
\pgfsetmiterjoin%
\definecolor{currentfill}{rgb}{1.000000,1.000000,1.000000}%
\pgfsetfillcolor{currentfill}%
\pgfsetlinewidth{0.501875pt}%
\definecolor{currentstroke}{rgb}{0.827451,0.827451,0.827451}%
\pgfsetstrokecolor{currentstroke}%
\pgfsetdash{}{0pt}%
\pgfpathmoveto{\pgfqpoint{1.754202in}{3.004965in}}%
\pgfpathlineto{\pgfqpoint{1.765448in}{2.995835in}}%
\pgfpathlineto{\pgfqpoint{1.767890in}{2.980139in}}%
\pgfpathlineto{\pgfqpoint{1.745251in}{2.981006in}}%
\pgfpathclose%
\pgfusepath{stroke,fill}%
\end{pgfscope}%
\begin{pgfscope}%
\pgfpathrectangle{\pgfqpoint{0.100000in}{2.413063in}}{\pgfqpoint{5.037500in}{3.427208in}}%
\pgfusepath{clip}%
\pgfsetbuttcap%
\pgfsetmiterjoin%
\definecolor{currentfill}{rgb}{1.000000,1.000000,1.000000}%
\pgfsetfillcolor{currentfill}%
\pgfsetlinewidth{0.501875pt}%
\definecolor{currentstroke}{rgb}{0.827451,0.827451,0.827451}%
\pgfsetstrokecolor{currentstroke}%
\pgfsetdash{}{0pt}%
\pgfpathmoveto{\pgfqpoint{1.805849in}{2.917287in}}%
\pgfpathlineto{\pgfqpoint{1.783257in}{2.925408in}}%
\pgfpathlineto{\pgfqpoint{1.789731in}{2.945097in}}%
\pgfpathlineto{\pgfqpoint{1.780262in}{2.965159in}}%
\pgfpathlineto{\pgfqpoint{1.783236in}{2.978951in}}%
\pgfpathlineto{\pgfqpoint{1.801416in}{2.984207in}}%
\pgfpathlineto{\pgfqpoint{1.801066in}{2.962112in}}%
\pgfpathlineto{\pgfqpoint{1.822569in}{2.949856in}}%
\pgfpathlineto{\pgfqpoint{1.834868in}{2.920225in}}%
\pgfpathlineto{\pgfqpoint{1.821162in}{2.908814in}}%
\pgfpathclose%
\pgfusepath{stroke,fill}%
\end{pgfscope}%
\begin{pgfscope}%
\pgfpathrectangle{\pgfqpoint{0.100000in}{2.413063in}}{\pgfqpoint{5.037500in}{3.427208in}}%
\pgfusepath{clip}%
\pgfsetbuttcap%
\pgfsetmiterjoin%
\definecolor{currentfill}{rgb}{1.000000,1.000000,1.000000}%
\pgfsetfillcolor{currentfill}%
\pgfsetlinewidth{0.501875pt}%
\definecolor{currentstroke}{rgb}{0.827451,0.827451,0.827451}%
\pgfsetstrokecolor{currentstroke}%
\pgfsetdash{}{0pt}%
\pgfpathmoveto{\pgfqpoint{1.724255in}{2.716939in}}%
\pgfpathlineto{\pgfqpoint{1.713491in}{2.739173in}}%
\pgfpathlineto{\pgfqpoint{1.718257in}{2.754128in}}%
\pgfpathlineto{\pgfqpoint{1.739369in}{2.774829in}}%
\pgfpathlineto{\pgfqpoint{1.741367in}{2.790698in}}%
\pgfpathlineto{\pgfqpoint{1.755183in}{2.825830in}}%
\pgfpathlineto{\pgfqpoint{1.791521in}{2.831833in}}%
\pgfpathlineto{\pgfqpoint{1.798476in}{2.852398in}}%
\pgfpathlineto{\pgfqpoint{1.814497in}{2.844687in}}%
\pgfpathlineto{\pgfqpoint{1.847051in}{2.786970in}}%
\pgfpathlineto{\pgfqpoint{1.847462in}{2.768126in}}%
\pgfpathlineto{\pgfqpoint{1.838260in}{2.751817in}}%
\pgfpathlineto{\pgfqpoint{1.849936in}{2.714370in}}%
\pgfpathlineto{\pgfqpoint{1.841778in}{2.708333in}}%
\pgfpathlineto{\pgfqpoint{1.813989in}{2.710088in}}%
\pgfpathlineto{\pgfqpoint{1.780572in}{2.721257in}}%
\pgfpathlineto{\pgfqpoint{1.752487in}{2.725868in}}%
\pgfpathlineto{\pgfqpoint{1.744986in}{2.719316in}}%
\pgfpathclose%
\pgfusepath{stroke,fill}%
\end{pgfscope}%
\begin{pgfscope}%
\pgfpathrectangle{\pgfqpoint{0.100000in}{2.413063in}}{\pgfqpoint{5.037500in}{3.427208in}}%
\pgfusepath{clip}%
\pgfsetbuttcap%
\pgfsetmiterjoin%
\definecolor{currentfill}{rgb}{1.000000,1.000000,1.000000}%
\pgfsetfillcolor{currentfill}%
\pgfsetlinewidth{0.501875pt}%
\definecolor{currentstroke}{rgb}{0.827451,0.827451,0.827451}%
\pgfsetstrokecolor{currentstroke}%
\pgfsetdash{}{0pt}%
\pgfpathmoveto{\pgfqpoint{0.951624in}{2.987308in}}%
\pgfpathlineto{\pgfqpoint{0.949465in}{2.985407in}}%
\pgfpathlineto{\pgfqpoint{0.939556in}{2.987940in}}%
\pgfpathlineto{\pgfqpoint{0.940296in}{2.992929in}}%
\pgfpathlineto{\pgfqpoint{0.945905in}{2.995212in}}%
\pgfpathlineto{\pgfqpoint{0.953902in}{3.006708in}}%
\pgfpathlineto{\pgfqpoint{0.955231in}{3.014478in}}%
\pgfpathlineto{\pgfqpoint{0.959726in}{3.016515in}}%
\pgfpathlineto{\pgfqpoint{0.967041in}{3.016522in}}%
\pgfpathlineto{\pgfqpoint{0.974687in}{3.041738in}}%
\pgfpathlineto{\pgfqpoint{0.977683in}{3.043812in}}%
\pgfpathlineto{\pgfqpoint{0.974996in}{3.048741in}}%
\pgfpathlineto{\pgfqpoint{0.968708in}{3.043755in}}%
\pgfpathlineto{\pgfqpoint{0.950231in}{3.050477in}}%
\pgfpathlineto{\pgfqpoint{0.939203in}{3.061802in}}%
\pgfpathlineto{\pgfqpoint{0.944011in}{3.066969in}}%
\pgfpathlineto{\pgfqpoint{0.941824in}{3.074257in}}%
\pgfpathlineto{\pgfqpoint{0.943172in}{3.082867in}}%
\pgfpathlineto{\pgfqpoint{0.940800in}{3.092932in}}%
\pgfpathlineto{\pgfqpoint{0.940421in}{3.103009in}}%
\pgfpathlineto{\pgfqpoint{0.951035in}{3.101323in}}%
\pgfpathlineto{\pgfqpoint{0.953137in}{3.104293in}}%
\pgfpathlineto{\pgfqpoint{0.958416in}{3.104005in}}%
\pgfpathlineto{\pgfqpoint{0.964496in}{3.093258in}}%
\pgfpathlineto{\pgfqpoint{0.970359in}{3.085389in}}%
\pgfpathlineto{\pgfqpoint{0.965663in}{3.080268in}}%
\pgfpathlineto{\pgfqpoint{0.973783in}{3.078303in}}%
\pgfpathlineto{\pgfqpoint{0.975632in}{3.080815in}}%
\pgfpathlineto{\pgfqpoint{0.971290in}{3.091008in}}%
\pgfpathlineto{\pgfqpoint{0.962148in}{3.099455in}}%
\pgfpathlineto{\pgfqpoint{0.963266in}{3.102995in}}%
\pgfpathlineto{\pgfqpoint{0.955970in}{3.111941in}}%
\pgfpathlineto{\pgfqpoint{0.963421in}{3.113757in}}%
\pgfpathlineto{\pgfqpoint{0.956965in}{3.132911in}}%
\pgfpathlineto{\pgfqpoint{0.962214in}{3.135566in}}%
\pgfpathlineto{\pgfqpoint{0.963494in}{3.140158in}}%
\pgfpathlineto{\pgfqpoint{0.960079in}{3.145613in}}%
\pgfpathlineto{\pgfqpoint{0.967127in}{3.146979in}}%
\pgfpathlineto{\pgfqpoint{0.968391in}{3.151105in}}%
\pgfpathlineto{\pgfqpoint{0.976797in}{3.144912in}}%
\pgfpathlineto{\pgfqpoint{0.978664in}{3.151627in}}%
\pgfpathlineto{\pgfqpoint{0.983081in}{3.154581in}}%
\pgfpathlineto{\pgfqpoint{1.004574in}{3.158691in}}%
\pgfpathlineto{\pgfqpoint{1.011172in}{3.157238in}}%
\pgfpathlineto{\pgfqpoint{1.016700in}{3.165394in}}%
\pgfpathlineto{\pgfqpoint{1.029815in}{3.170744in}}%
\pgfpathlineto{\pgfqpoint{1.039917in}{3.168715in}}%
\pgfpathlineto{\pgfqpoint{1.044526in}{3.161612in}}%
\pgfpathlineto{\pgfqpoint{1.044284in}{3.150327in}}%
\pgfpathlineto{\pgfqpoint{1.049454in}{3.147769in}}%
\pgfpathlineto{\pgfqpoint{1.059999in}{3.148126in}}%
\pgfpathlineto{\pgfqpoint{1.073506in}{3.151062in}}%
\pgfpathlineto{\pgfqpoint{1.073340in}{3.147453in}}%
\pgfpathlineto{\pgfqpoint{1.090413in}{3.137164in}}%
\pgfpathlineto{\pgfqpoint{1.102354in}{3.139923in}}%
\pgfpathlineto{\pgfqpoint{1.105870in}{3.143494in}}%
\pgfpathlineto{\pgfqpoint{1.109634in}{3.152639in}}%
\pgfpathlineto{\pgfqpoint{1.114303in}{3.158496in}}%
\pgfpathlineto{\pgfqpoint{1.114719in}{3.167272in}}%
\pgfpathlineto{\pgfqpoint{1.110445in}{3.170724in}}%
\pgfpathlineto{\pgfqpoint{1.116774in}{3.173843in}}%
\pgfpathlineto{\pgfqpoint{1.127410in}{3.169156in}}%
\pgfpathlineto{\pgfqpoint{1.130874in}{3.172050in}}%
\pgfpathlineto{\pgfqpoint{1.131483in}{3.184466in}}%
\pgfpathlineto{\pgfqpoint{1.124609in}{3.181871in}}%
\pgfpathlineto{\pgfqpoint{1.119841in}{3.186174in}}%
\pgfpathlineto{\pgfqpoint{1.112102in}{3.188855in}}%
\pgfpathlineto{\pgfqpoint{1.099816in}{3.188408in}}%
\pgfpathlineto{\pgfqpoint{1.088046in}{3.192487in}}%
\pgfpathlineto{\pgfqpoint{1.087455in}{3.202649in}}%
\pgfpathlineto{\pgfqpoint{1.076928in}{3.213444in}}%
\pgfpathlineto{\pgfqpoint{1.063579in}{3.217975in}}%
\pgfpathlineto{\pgfqpoint{1.051926in}{3.238872in}}%
\pgfpathlineto{\pgfqpoint{1.052882in}{3.246605in}}%
\pgfpathlineto{\pgfqpoint{1.059156in}{3.250131in}}%
\pgfpathlineto{\pgfqpoint{1.058232in}{3.257452in}}%
\pgfpathlineto{\pgfqpoint{1.059689in}{3.264734in}}%
\pgfpathlineto{\pgfqpoint{1.064654in}{3.259749in}}%
\pgfpathlineto{\pgfqpoint{1.071209in}{3.261829in}}%
\pgfpathlineto{\pgfqpoint{1.070881in}{3.268154in}}%
\pgfpathlineto{\pgfqpoint{1.061792in}{3.280854in}}%
\pgfpathlineto{\pgfqpoint{1.059132in}{3.293954in}}%
\pgfpathlineto{\pgfqpoint{1.066739in}{3.294953in}}%
\pgfpathlineto{\pgfqpoint{1.078126in}{3.294149in}}%
\pgfpathlineto{\pgfqpoint{1.081711in}{3.289767in}}%
\pgfpathlineto{\pgfqpoint{1.091918in}{3.291335in}}%
\pgfpathlineto{\pgfqpoint{1.101406in}{3.289277in}}%
\pgfpathlineto{\pgfqpoint{1.107819in}{3.285658in}}%
\pgfpathlineto{\pgfqpoint{1.112258in}{3.287312in}}%
\pgfpathlineto{\pgfqpoint{1.124045in}{3.283696in}}%
\pgfpathlineto{\pgfqpoint{1.124153in}{3.280493in}}%
\pgfpathlineto{\pgfqpoint{1.133129in}{3.281575in}}%
\pgfpathlineto{\pgfqpoint{1.145165in}{3.273762in}}%
\pgfpathlineto{\pgfqpoint{1.144698in}{3.268577in}}%
\pgfpathlineto{\pgfqpoint{1.138971in}{3.266182in}}%
\pgfpathlineto{\pgfqpoint{1.132884in}{3.260488in}}%
\pgfpathlineto{\pgfqpoint{1.132148in}{3.252669in}}%
\pgfpathlineto{\pgfqpoint{1.145260in}{3.242696in}}%
\pgfpathlineto{\pgfqpoint{1.143163in}{3.237918in}}%
\pgfpathlineto{\pgfqpoint{1.152493in}{3.234247in}}%
\pgfpathlineto{\pgfqpoint{1.154784in}{3.228253in}}%
\pgfpathlineto{\pgfqpoint{1.166624in}{3.233016in}}%
\pgfpathlineto{\pgfqpoint{1.171839in}{3.226774in}}%
\pgfpathlineto{\pgfqpoint{1.174245in}{3.230552in}}%
\pgfpathlineto{\pgfqpoint{1.167159in}{3.242850in}}%
\pgfpathlineto{\pgfqpoint{1.169677in}{3.246534in}}%
\pgfpathlineto{\pgfqpoint{1.170793in}{3.256248in}}%
\pgfpathlineto{\pgfqpoint{1.167971in}{3.260522in}}%
\pgfpathlineto{\pgfqpoint{1.170088in}{3.266265in}}%
\pgfpathlineto{\pgfqpoint{1.177026in}{3.265725in}}%
\pgfpathlineto{\pgfqpoint{1.175590in}{3.255953in}}%
\pgfpathlineto{\pgfqpoint{1.171161in}{3.252693in}}%
\pgfpathlineto{\pgfqpoint{1.171475in}{3.239546in}}%
\pgfpathlineto{\pgfqpoint{1.179422in}{3.237975in}}%
\pgfpathlineto{\pgfqpoint{1.179519in}{3.228201in}}%
\pgfpathlineto{\pgfqpoint{1.188198in}{3.221859in}}%
\pgfpathlineto{\pgfqpoint{1.192188in}{3.225858in}}%
\pgfpathlineto{\pgfqpoint{1.187827in}{3.238028in}}%
\pgfpathlineto{\pgfqpoint{1.183318in}{3.241107in}}%
\pgfpathlineto{\pgfqpoint{1.178882in}{3.239081in}}%
\pgfpathlineto{\pgfqpoint{1.174978in}{3.241667in}}%
\pgfpathlineto{\pgfqpoint{1.175659in}{3.252913in}}%
\pgfpathlineto{\pgfqpoint{1.182173in}{3.257512in}}%
\pgfpathlineto{\pgfqpoint{1.188584in}{3.257089in}}%
\pgfpathlineto{\pgfqpoint{1.185928in}{3.263452in}}%
\pgfpathlineto{\pgfqpoint{1.176318in}{3.268326in}}%
\pgfpathlineto{\pgfqpoint{1.163592in}{3.287746in}}%
\pgfpathlineto{\pgfqpoint{1.169943in}{3.297060in}}%
\pgfpathlineto{\pgfqpoint{1.173478in}{3.305832in}}%
\pgfpathlineto{\pgfqpoint{1.173756in}{3.323448in}}%
\pgfpathlineto{\pgfqpoint{1.172395in}{3.338309in}}%
\pgfpathlineto{\pgfqpoint{1.168512in}{3.348067in}}%
\pgfpathlineto{\pgfqpoint{1.169286in}{3.357824in}}%
\pgfpathlineto{\pgfqpoint{1.179218in}{3.366286in}}%
\pgfpathlineto{\pgfqpoint{1.189095in}{3.376341in}}%
\pgfpathlineto{\pgfqpoint{1.213857in}{3.355796in}}%
\pgfpathlineto{\pgfqpoint{1.230047in}{3.354282in}}%
\pgfpathlineto{\pgfqpoint{1.243410in}{3.359398in}}%
\pgfpathlineto{\pgfqpoint{1.255140in}{3.366135in}}%
\pgfpathlineto{\pgfqpoint{1.257633in}{3.369344in}}%
\pgfpathlineto{\pgfqpoint{1.286589in}{3.376664in}}%
\pgfpathlineto{\pgfqpoint{1.289104in}{3.371302in}}%
\pgfpathlineto{\pgfqpoint{1.298944in}{3.366914in}}%
\pgfpathlineto{\pgfqpoint{1.317722in}{3.368261in}}%
\pgfpathlineto{\pgfqpoint{1.321152in}{3.367062in}}%
\pgfpathlineto{\pgfqpoint{1.329592in}{3.370264in}}%
\pgfpathlineto{\pgfqpoint{1.332985in}{3.364727in}}%
\pgfpathlineto{\pgfqpoint{1.341089in}{3.359668in}}%
\pgfpathlineto{\pgfqpoint{1.345143in}{3.360240in}}%
\pgfpathlineto{\pgfqpoint{1.352990in}{3.354162in}}%
\pgfpathlineto{\pgfqpoint{1.366250in}{3.355279in}}%
\pgfpathlineto{\pgfqpoint{1.379495in}{3.360200in}}%
\pgfpathlineto{\pgfqpoint{1.390311in}{3.344464in}}%
\pgfpathlineto{\pgfqpoint{1.389472in}{3.340932in}}%
\pgfpathlineto{\pgfqpoint{1.377537in}{3.336874in}}%
\pgfpathlineto{\pgfqpoint{1.380726in}{3.330737in}}%
\pgfpathlineto{\pgfqpoint{1.391713in}{3.337196in}}%
\pgfpathlineto{\pgfqpoint{1.396199in}{3.335631in}}%
\pgfpathlineto{\pgfqpoint{1.398107in}{3.327905in}}%
\pgfpathlineto{\pgfqpoint{1.391548in}{3.324638in}}%
\pgfpathlineto{\pgfqpoint{1.396233in}{3.314872in}}%
\pgfpathlineto{\pgfqpoint{1.403035in}{3.316522in}}%
\pgfpathlineto{\pgfqpoint{1.412907in}{3.311482in}}%
\pgfpathlineto{\pgfqpoint{1.422347in}{3.297492in}}%
\pgfpathlineto{\pgfqpoint{1.414040in}{3.293608in}}%
\pgfpathlineto{\pgfqpoint{1.421379in}{3.283300in}}%
\pgfpathlineto{\pgfqpoint{1.415196in}{3.280891in}}%
\pgfpathlineto{\pgfqpoint{1.422888in}{3.271101in}}%
\pgfpathlineto{\pgfqpoint{1.432268in}{3.272144in}}%
\pgfpathlineto{\pgfqpoint{1.437208in}{3.267739in}}%
\pgfpathlineto{\pgfqpoint{1.449701in}{3.261244in}}%
\pgfpathlineto{\pgfqpoint{1.465817in}{3.238189in}}%
\pgfpathlineto{\pgfqpoint{1.465453in}{3.232675in}}%
\pgfpathlineto{\pgfqpoint{1.489622in}{3.215013in}}%
\pgfpathlineto{\pgfqpoint{1.490143in}{3.208766in}}%
\pgfpathlineto{\pgfqpoint{1.496584in}{3.199419in}}%
\pgfpathlineto{\pgfqpoint{1.516372in}{3.194747in}}%
\pgfpathlineto{\pgfqpoint{1.523710in}{3.191424in}}%
\pgfpathlineto{\pgfqpoint{1.530615in}{3.181120in}}%
\pgfpathlineto{\pgfqpoint{1.531504in}{3.172094in}}%
\pgfpathlineto{\pgfqpoint{1.537437in}{3.164738in}}%
\pgfpathlineto{\pgfqpoint{1.539220in}{3.156441in}}%
\pgfpathlineto{\pgfqpoint{1.544780in}{3.153411in}}%
\pgfpathlineto{\pgfqpoint{1.517233in}{3.103693in}}%
\pgfpathlineto{\pgfqpoint{1.460054in}{3.000474in}}%
\pgfpathlineto{\pgfqpoint{1.376162in}{2.849003in}}%
\pgfpathlineto{\pgfqpoint{1.343317in}{2.789686in}}%
\pgfpathlineto{\pgfqpoint{1.350313in}{2.781720in}}%
\pgfpathlineto{\pgfqpoint{1.353395in}{2.784262in}}%
\pgfpathlineto{\pgfqpoint{1.359643in}{2.774988in}}%
\pgfpathlineto{\pgfqpoint{1.368318in}{2.777678in}}%
\pgfpathlineto{\pgfqpoint{1.379832in}{2.772122in}}%
\pgfpathlineto{\pgfqpoint{1.372374in}{2.763439in}}%
\pgfpathlineto{\pgfqpoint{1.378134in}{2.751722in}}%
\pgfpathlineto{\pgfqpoint{1.376943in}{2.746017in}}%
\pgfpathlineto{\pgfqpoint{1.386129in}{2.715933in}}%
\pgfpathlineto{\pgfqpoint{1.382242in}{2.702252in}}%
\pgfpathlineto{\pgfqpoint{1.399793in}{2.705640in}}%
\pgfpathlineto{\pgfqpoint{1.404102in}{2.703420in}}%
\pgfpathlineto{\pgfqpoint{1.408780in}{2.707029in}}%
\pgfpathlineto{\pgfqpoint{1.412108in}{2.713850in}}%
\pgfpathlineto{\pgfqpoint{1.416874in}{2.717760in}}%
\pgfpathlineto{\pgfqpoint{1.437213in}{2.717265in}}%
\pgfpathlineto{\pgfqpoint{1.441809in}{2.704137in}}%
\pgfpathlineto{\pgfqpoint{1.437865in}{2.692613in}}%
\pgfpathlineto{\pgfqpoint{1.442347in}{2.688811in}}%
\pgfpathlineto{\pgfqpoint{1.444703in}{2.682307in}}%
\pgfpathlineto{\pgfqpoint{1.444189in}{2.670075in}}%
\pgfpathlineto{\pgfqpoint{1.450169in}{2.661305in}}%
\pgfpathlineto{\pgfqpoint{1.453448in}{2.647561in}}%
\pgfpathlineto{\pgfqpoint{1.451765in}{2.640138in}}%
\pgfpathlineto{\pgfqpoint{1.453795in}{2.632614in}}%
\pgfpathlineto{\pgfqpoint{1.456722in}{2.589045in}}%
\pgfpathlineto{\pgfqpoint{1.452528in}{2.585613in}}%
\pgfpathlineto{\pgfqpoint{1.458062in}{2.580871in}}%
\pgfpathlineto{\pgfqpoint{1.453713in}{2.575103in}}%
\pgfpathlineto{\pgfqpoint{1.457635in}{2.570400in}}%
\pgfpathlineto{\pgfqpoint{1.455109in}{2.562374in}}%
\pgfpathlineto{\pgfqpoint{1.460758in}{2.560197in}}%
\pgfpathlineto{\pgfqpoint{1.468024in}{2.547989in}}%
\pgfpathlineto{\pgfqpoint{1.473424in}{2.544311in}}%
\pgfpathlineto{\pgfqpoint{1.474837in}{2.538933in}}%
\pgfpathlineto{\pgfqpoint{1.477965in}{2.536604in}}%
\pgfpathlineto{\pgfqpoint{1.477313in}{2.532139in}}%
\pgfpathlineto{\pgfqpoint{1.484056in}{2.528904in}}%
\pgfpathlineto{\pgfqpoint{1.482391in}{2.520313in}}%
\pgfpathlineto{\pgfqpoint{1.476587in}{2.515813in}}%
\pgfpathlineto{\pgfqpoint{1.473950in}{2.507451in}}%
\pgfpathlineto{\pgfqpoint{1.473250in}{2.496254in}}%
\pgfpathlineto{\pgfqpoint{1.470025in}{2.495183in}}%
\pgfpathlineto{\pgfqpoint{1.459486in}{2.485368in}}%
\pgfpathlineto{\pgfqpoint{1.450119in}{2.482777in}}%
\pgfpathlineto{\pgfqpoint{1.445648in}{2.486735in}}%
\pgfpathlineto{\pgfqpoint{1.449968in}{2.497240in}}%
\pgfpathlineto{\pgfqpoint{1.448648in}{2.502397in}}%
\pgfpathlineto{\pgfqpoint{1.454837in}{2.505010in}}%
\pgfpathlineto{\pgfqpoint{1.460625in}{2.520208in}}%
\pgfpathlineto{\pgfqpoint{1.458180in}{2.533190in}}%
\pgfpathlineto{\pgfqpoint{1.454269in}{2.536255in}}%
\pgfpathlineto{\pgfqpoint{1.441186in}{2.535262in}}%
\pgfpathlineto{\pgfqpoint{1.443823in}{2.531871in}}%
\pgfpathlineto{\pgfqpoint{1.434354in}{2.522137in}}%
\pgfpathlineto{\pgfqpoint{1.431303in}{2.527549in}}%
\pgfpathlineto{\pgfqpoint{1.432097in}{2.534620in}}%
\pgfpathlineto{\pgfqpoint{1.437533in}{2.534490in}}%
\pgfpathlineto{\pgfqpoint{1.442372in}{2.539684in}}%
\pgfpathlineto{\pgfqpoint{1.445236in}{2.547269in}}%
\pgfpathlineto{\pgfqpoint{1.456026in}{2.545128in}}%
\pgfpathlineto{\pgfqpoint{1.447388in}{2.549180in}}%
\pgfpathlineto{\pgfqpoint{1.448148in}{2.554607in}}%
\pgfpathlineto{\pgfqpoint{1.444553in}{2.557058in}}%
\pgfpathlineto{\pgfqpoint{1.443082in}{2.564452in}}%
\pgfpathlineto{\pgfqpoint{1.445633in}{2.568303in}}%
\pgfpathlineto{\pgfqpoint{1.439548in}{2.580265in}}%
\pgfpathlineto{\pgfqpoint{1.440061in}{2.587875in}}%
\pgfpathlineto{\pgfqpoint{1.432930in}{2.594651in}}%
\pgfpathlineto{\pgfqpoint{1.429073in}{2.600133in}}%
\pgfpathlineto{\pgfqpoint{1.434400in}{2.605358in}}%
\pgfpathlineto{\pgfqpoint{1.436374in}{2.613167in}}%
\pgfpathlineto{\pgfqpoint{1.434612in}{2.618294in}}%
\pgfpathlineto{\pgfqpoint{1.436673in}{2.624409in}}%
\pgfpathlineto{\pgfqpoint{1.434046in}{2.644346in}}%
\pgfpathlineto{\pgfqpoint{1.430119in}{2.653391in}}%
\pgfpathlineto{\pgfqpoint{1.425627in}{2.656790in}}%
\pgfpathlineto{\pgfqpoint{1.427539in}{2.668556in}}%
\pgfpathlineto{\pgfqpoint{1.426190in}{2.677745in}}%
\pgfpathlineto{\pgfqpoint{1.428633in}{2.683640in}}%
\pgfpathlineto{\pgfqpoint{1.423578in}{2.684465in}}%
\pgfpathlineto{\pgfqpoint{1.422223in}{2.670149in}}%
\pgfpathlineto{\pgfqpoint{1.416543in}{2.654178in}}%
\pgfpathlineto{\pgfqpoint{1.411982in}{2.657815in}}%
\pgfpathlineto{\pgfqpoint{1.411372in}{2.663064in}}%
\pgfpathlineto{\pgfqpoint{1.405792in}{2.670234in}}%
\pgfpathlineto{\pgfqpoint{1.408627in}{2.674929in}}%
\pgfpathlineto{\pgfqpoint{1.407746in}{2.685497in}}%
\pgfpathlineto{\pgfqpoint{1.400524in}{2.681780in}}%
\pgfpathlineto{\pgfqpoint{1.401941in}{2.676048in}}%
\pgfpathlineto{\pgfqpoint{1.400581in}{2.668771in}}%
\pgfpathlineto{\pgfqpoint{1.392474in}{2.668784in}}%
\pgfpathlineto{\pgfqpoint{1.388840in}{2.673295in}}%
\pgfpathlineto{\pgfqpoint{1.383356in}{2.674156in}}%
\pgfpathlineto{\pgfqpoint{1.379665in}{2.679012in}}%
\pgfpathlineto{\pgfqpoint{1.373315in}{2.692347in}}%
\pgfpathlineto{\pgfqpoint{1.371257in}{2.702701in}}%
\pgfpathlineto{\pgfqpoint{1.372574in}{2.707102in}}%
\pgfpathlineto{\pgfqpoint{1.369501in}{2.716794in}}%
\pgfpathlineto{\pgfqpoint{1.364408in}{2.722405in}}%
\pgfpathlineto{\pgfqpoint{1.355600in}{2.736743in}}%
\pgfpathlineto{\pgfqpoint{1.349211in}{2.749198in}}%
\pgfpathlineto{\pgfqpoint{1.355730in}{2.749385in}}%
\pgfpathlineto{\pgfqpoint{1.359566in}{2.752000in}}%
\pgfpathlineto{\pgfqpoint{1.360320in}{2.760094in}}%
\pgfpathlineto{\pgfqpoint{1.341875in}{2.761071in}}%
\pgfpathlineto{\pgfqpoint{1.326187in}{2.778596in}}%
\pgfpathlineto{\pgfqpoint{1.330312in}{2.783205in}}%
\pgfpathlineto{\pgfqpoint{1.322726in}{2.783992in}}%
\pgfpathlineto{\pgfqpoint{1.307659in}{2.800285in}}%
\pgfpathlineto{\pgfqpoint{1.283789in}{2.808454in}}%
\pgfpathlineto{\pgfqpoint{1.280575in}{2.818029in}}%
\pgfpathlineto{\pgfqpoint{1.275622in}{2.823016in}}%
\pgfpathlineto{\pgfqpoint{1.275397in}{2.827699in}}%
\pgfpathlineto{\pgfqpoint{1.271671in}{2.831176in}}%
\pgfpathlineto{\pgfqpoint{1.276561in}{2.836988in}}%
\pgfpathlineto{\pgfqpoint{1.265409in}{2.837549in}}%
\pgfpathlineto{\pgfqpoint{1.262574in}{2.845070in}}%
\pgfpathlineto{\pgfqpoint{1.257499in}{2.848272in}}%
\pgfpathlineto{\pgfqpoint{1.267663in}{2.852137in}}%
\pgfpathlineto{\pgfqpoint{1.262377in}{2.859031in}}%
\pgfpathlineto{\pgfqpoint{1.252649in}{2.861924in}}%
\pgfpathlineto{\pgfqpoint{1.254162in}{2.874658in}}%
\pgfpathlineto{\pgfqpoint{1.251373in}{2.879166in}}%
\pgfpathlineto{\pgfqpoint{1.246350in}{2.882341in}}%
\pgfpathlineto{\pgfqpoint{1.242055in}{2.879156in}}%
\pgfpathlineto{\pgfqpoint{1.239517in}{2.882151in}}%
\pgfpathlineto{\pgfqpoint{1.232116in}{2.883009in}}%
\pgfpathlineto{\pgfqpoint{1.232712in}{2.894564in}}%
\pgfpathlineto{\pgfqpoint{1.221982in}{2.886760in}}%
\pgfpathlineto{\pgfqpoint{1.222208in}{2.870676in}}%
\pgfpathlineto{\pgfqpoint{1.210691in}{2.867592in}}%
\pgfpathlineto{\pgfqpoint{1.203169in}{2.861169in}}%
\pgfpathlineto{\pgfqpoint{1.197123in}{2.859816in}}%
\pgfpathlineto{\pgfqpoint{1.190581in}{2.866289in}}%
\pgfpathlineto{\pgfqpoint{1.184755in}{2.865431in}}%
\pgfpathlineto{\pgfqpoint{1.185835in}{2.871344in}}%
\pgfpathlineto{\pgfqpoint{1.177554in}{2.868160in}}%
\pgfpathlineto{\pgfqpoint{1.169881in}{2.867802in}}%
\pgfpathlineto{\pgfqpoint{1.160900in}{2.864574in}}%
\pgfpathlineto{\pgfqpoint{1.142675in}{2.866391in}}%
\pgfpathlineto{\pgfqpoint{1.136823in}{2.865196in}}%
\pgfpathlineto{\pgfqpoint{1.132601in}{2.870513in}}%
\pgfpathlineto{\pgfqpoint{1.124808in}{2.871044in}}%
\pgfpathlineto{\pgfqpoint{1.123187in}{2.877934in}}%
\pgfpathlineto{\pgfqpoint{1.127832in}{2.881688in}}%
\pgfpathlineto{\pgfqpoint{1.133281in}{2.881335in}}%
\pgfpathlineto{\pgfqpoint{1.137829in}{2.876217in}}%
\pgfpathlineto{\pgfqpoint{1.140824in}{2.884078in}}%
\pgfpathlineto{\pgfqpoint{1.136645in}{2.892787in}}%
\pgfpathlineto{\pgfqpoint{1.146104in}{2.900453in}}%
\pgfpathlineto{\pgfqpoint{1.155583in}{2.903278in}}%
\pgfpathlineto{\pgfqpoint{1.162183in}{2.907887in}}%
\pgfpathlineto{\pgfqpoint{1.166457in}{2.912851in}}%
\pgfpathlineto{\pgfqpoint{1.168843in}{2.920989in}}%
\pgfpathlineto{\pgfqpoint{1.176380in}{2.918774in}}%
\pgfpathlineto{\pgfqpoint{1.192721in}{2.919912in}}%
\pgfpathlineto{\pgfqpoint{1.196229in}{2.910989in}}%
\pgfpathlineto{\pgfqpoint{1.201960in}{2.910600in}}%
\pgfpathlineto{\pgfqpoint{1.212392in}{2.896942in}}%
\pgfpathlineto{\pgfqpoint{1.212411in}{2.901921in}}%
\pgfpathlineto{\pgfqpoint{1.208767in}{2.903540in}}%
\pgfpathlineto{\pgfqpoint{1.202098in}{2.919845in}}%
\pgfpathlineto{\pgfqpoint{1.205343in}{2.922060in}}%
\pgfpathlineto{\pgfqpoint{1.196436in}{2.930567in}}%
\pgfpathlineto{\pgfqpoint{1.188475in}{2.932223in}}%
\pgfpathlineto{\pgfqpoint{1.180890in}{2.929282in}}%
\pgfpathlineto{\pgfqpoint{1.167898in}{2.931573in}}%
\pgfpathlineto{\pgfqpoint{1.161737in}{2.927452in}}%
\pgfpathlineto{\pgfqpoint{1.147708in}{2.924268in}}%
\pgfpathlineto{\pgfqpoint{1.146425in}{2.919686in}}%
\pgfpathlineto{\pgfqpoint{1.135646in}{2.918437in}}%
\pgfpathlineto{\pgfqpoint{1.132880in}{2.913199in}}%
\pgfpathlineto{\pgfqpoint{1.126714in}{2.909361in}}%
\pgfpathlineto{\pgfqpoint{1.119266in}{2.910021in}}%
\pgfpathlineto{\pgfqpoint{1.115548in}{2.906310in}}%
\pgfpathlineto{\pgfqpoint{1.110856in}{2.906485in}}%
\pgfpathlineto{\pgfqpoint{1.106372in}{2.909824in}}%
\pgfpathlineto{\pgfqpoint{1.098533in}{2.909228in}}%
\pgfpathlineto{\pgfqpoint{1.096687in}{2.906150in}}%
\pgfpathlineto{\pgfqpoint{1.088453in}{2.909309in}}%
\pgfpathlineto{\pgfqpoint{1.081462in}{2.901155in}}%
\pgfpathlineto{\pgfqpoint{1.088166in}{2.893265in}}%
\pgfpathlineto{\pgfqpoint{1.090745in}{2.886777in}}%
\pgfpathlineto{\pgfqpoint{1.088367in}{2.881281in}}%
\pgfpathlineto{\pgfqpoint{1.078842in}{2.877488in}}%
\pgfpathlineto{\pgfqpoint{1.073324in}{2.880575in}}%
\pgfpathlineto{\pgfqpoint{1.066378in}{2.878904in}}%
\pgfpathlineto{\pgfqpoint{1.065541in}{2.874087in}}%
\pgfpathlineto{\pgfqpoint{1.057086in}{2.875791in}}%
\pgfpathlineto{\pgfqpoint{1.058958in}{2.870978in}}%
\pgfpathlineto{\pgfqpoint{1.046487in}{2.869963in}}%
\pgfpathlineto{\pgfqpoint{1.038414in}{2.876072in}}%
\pgfpathlineto{\pgfqpoint{1.034102in}{2.872187in}}%
\pgfpathlineto{\pgfqpoint{1.022628in}{2.876323in}}%
\pgfpathlineto{\pgfqpoint{1.019770in}{2.872134in}}%
\pgfpathlineto{\pgfqpoint{1.014082in}{2.870270in}}%
\pgfpathlineto{\pgfqpoint{1.009518in}{2.875182in}}%
\pgfpathlineto{\pgfqpoint{0.993915in}{2.873961in}}%
\pgfpathlineto{\pgfqpoint{0.993931in}{2.866717in}}%
\pgfpathlineto{\pgfqpoint{0.987663in}{2.864800in}}%
\pgfpathlineto{\pgfqpoint{0.982786in}{2.868523in}}%
\pgfpathlineto{\pgfqpoint{0.976340in}{2.866454in}}%
\pgfpathlineto{\pgfqpoint{0.967998in}{2.865961in}}%
\pgfpathlineto{\pgfqpoint{0.955569in}{2.871199in}}%
\pgfpathlineto{\pgfqpoint{0.948453in}{2.865798in}}%
\pgfpathlineto{\pgfqpoint{0.938273in}{2.873016in}}%
\pgfpathlineto{\pgfqpoint{0.934945in}{2.870281in}}%
\pgfpathlineto{\pgfqpoint{0.932608in}{2.863211in}}%
\pgfpathlineto{\pgfqpoint{0.926065in}{2.859066in}}%
\pgfpathlineto{\pgfqpoint{0.915395in}{2.863811in}}%
\pgfpathlineto{\pgfqpoint{0.897718in}{2.875532in}}%
\pgfpathlineto{\pgfqpoint{0.892566in}{2.876574in}}%
\pgfpathlineto{\pgfqpoint{0.889659in}{2.874131in}}%
\pgfpathlineto{\pgfqpoint{0.881884in}{2.875792in}}%
\pgfpathlineto{\pgfqpoint{0.878091in}{2.879440in}}%
\pgfpathlineto{\pgfqpoint{0.870326in}{2.881406in}}%
\pgfpathlineto{\pgfqpoint{0.859720in}{2.881520in}}%
\pgfpathlineto{\pgfqpoint{0.855509in}{2.885871in}}%
\pgfpathlineto{\pgfqpoint{0.863579in}{2.890374in}}%
\pgfpathlineto{\pgfqpoint{0.861696in}{2.894819in}}%
\pgfpathlineto{\pgfqpoint{0.856152in}{2.893821in}}%
\pgfpathlineto{\pgfqpoint{0.853382in}{2.890327in}}%
\pgfpathlineto{\pgfqpoint{0.840401in}{2.885316in}}%
\pgfpathlineto{\pgfqpoint{0.833837in}{2.887021in}}%
\pgfpathlineto{\pgfqpoint{0.831075in}{2.897193in}}%
\pgfpathlineto{\pgfqpoint{0.822887in}{2.892819in}}%
\pgfpathlineto{\pgfqpoint{0.820507in}{2.905712in}}%
\pgfpathlineto{\pgfqpoint{0.816627in}{2.902389in}}%
\pgfpathlineto{\pgfqpoint{0.817276in}{2.897428in}}%
\pgfpathlineto{\pgfqpoint{0.808354in}{2.898738in}}%
\pgfpathlineto{\pgfqpoint{0.817635in}{2.907383in}}%
\pgfpathlineto{\pgfqpoint{0.827041in}{2.902246in}}%
\pgfpathlineto{\pgfqpoint{0.829057in}{2.904825in}}%
\pgfpathlineto{\pgfqpoint{0.838952in}{2.901734in}}%
\pgfpathlineto{\pgfqpoint{0.838334in}{2.905253in}}%
\pgfpathlineto{\pgfqpoint{0.865329in}{2.906418in}}%
\pgfpathlineto{\pgfqpoint{0.878499in}{2.900929in}}%
\pgfpathlineto{\pgfqpoint{0.883595in}{2.895513in}}%
\pgfpathlineto{\pgfqpoint{0.882083in}{2.890785in}}%
\pgfpathlineto{\pgfqpoint{0.887200in}{2.888996in}}%
\pgfpathlineto{\pgfqpoint{0.899915in}{2.896458in}}%
\pgfpathlineto{\pgfqpoint{0.916364in}{2.896936in}}%
\pgfpathlineto{\pgfqpoint{0.931303in}{2.892438in}}%
\pgfpathlineto{\pgfqpoint{0.939764in}{2.893193in}}%
\pgfpathlineto{\pgfqpoint{0.942916in}{2.886215in}}%
\pgfpathlineto{\pgfqpoint{0.947648in}{2.893908in}}%
\pgfpathlineto{\pgfqpoint{0.950558in}{2.895574in}}%
\pgfpathlineto{\pgfqpoint{0.964023in}{2.898918in}}%
\pgfpathlineto{\pgfqpoint{0.969028in}{2.896874in}}%
\pgfpathlineto{\pgfqpoint{0.974439in}{2.899301in}}%
\pgfpathlineto{\pgfqpoint{0.980895in}{2.898032in}}%
\pgfpathlineto{\pgfqpoint{0.982418in}{2.900859in}}%
\pgfpathlineto{\pgfqpoint{0.995888in}{2.913831in}}%
\pgfpathlineto{\pgfqpoint{1.001592in}{2.915755in}}%
\pgfpathlineto{\pgfqpoint{1.004963in}{2.922756in}}%
\pgfpathlineto{\pgfqpoint{1.022854in}{2.928179in}}%
\pgfpathlineto{\pgfqpoint{1.025094in}{2.932315in}}%
\pgfpathlineto{\pgfqpoint{0.999738in}{2.938698in}}%
\pgfpathlineto{\pgfqpoint{1.000735in}{2.945564in}}%
\pgfpathlineto{\pgfqpoint{0.998128in}{2.954632in}}%
\pgfpathlineto{\pgfqpoint{0.992224in}{2.953878in}}%
\pgfpathlineto{\pgfqpoint{0.988207in}{2.942111in}}%
\pgfpathlineto{\pgfqpoint{0.983545in}{2.941141in}}%
\pgfpathlineto{\pgfqpoint{0.980510in}{2.944950in}}%
\pgfpathlineto{\pgfqpoint{0.984050in}{2.961614in}}%
\pgfpathlineto{\pgfqpoint{0.980525in}{2.968597in}}%
\pgfpathlineto{\pgfqpoint{0.973696in}{2.976829in}}%
\pgfpathlineto{\pgfqpoint{0.976978in}{2.983110in}}%
\pgfpathlineto{\pgfqpoint{0.962567in}{2.983336in}}%
\pgfpathlineto{\pgfqpoint{0.962896in}{2.985773in}}%
\pgfpathlineto{\pgfqpoint{0.954297in}{2.989187in}}%
\pgfpathclose%
\pgfusepath{stroke,fill}%
\end{pgfscope}%
\begin{pgfscope}%
\pgfpathrectangle{\pgfqpoint{0.100000in}{2.413063in}}{\pgfqpoint{5.037500in}{3.427208in}}%
\pgfusepath{clip}%
\pgfsetbuttcap%
\pgfsetmiterjoin%
\definecolor{currentfill}{rgb}{1.000000,1.000000,1.000000}%
\pgfsetfillcolor{currentfill}%
\pgfsetlinewidth{0.501875pt}%
\definecolor{currentstroke}{rgb}{0.827451,0.827451,0.827451}%
\pgfsetstrokecolor{currentstroke}%
\pgfsetdash{}{0pt}%
\pgfpathmoveto{\pgfqpoint{0.925909in}{3.108677in}}%
\pgfpathlineto{\pgfqpoint{0.924127in}{3.105546in}}%
\pgfpathlineto{\pgfqpoint{0.928841in}{3.098878in}}%
\pgfpathlineto{\pgfqpoint{0.921389in}{3.092640in}}%
\pgfpathlineto{\pgfqpoint{0.921597in}{3.087275in}}%
\pgfpathlineto{\pgfqpoint{0.918227in}{3.086009in}}%
\pgfpathlineto{\pgfqpoint{0.906646in}{3.094010in}}%
\pgfpathlineto{\pgfqpoint{0.898390in}{3.111524in}}%
\pgfpathlineto{\pgfqpoint{0.897838in}{3.116585in}}%
\pgfpathlineto{\pgfqpoint{0.899960in}{3.122642in}}%
\pgfpathlineto{\pgfqpoint{0.908905in}{3.113376in}}%
\pgfpathlineto{\pgfqpoint{0.911642in}{3.115164in}}%
\pgfpathlineto{\pgfqpoint{0.919230in}{3.113463in}}%
\pgfpathclose%
\pgfusepath{stroke,fill}%
\end{pgfscope}%
\begin{pgfscope}%
\pgfpathrectangle{\pgfqpoint{0.100000in}{2.413063in}}{\pgfqpoint{5.037500in}{3.427208in}}%
\pgfusepath{clip}%
\pgfsetbuttcap%
\pgfsetmiterjoin%
\definecolor{currentfill}{rgb}{1.000000,1.000000,1.000000}%
\pgfsetfillcolor{currentfill}%
\pgfsetlinewidth{0.501875pt}%
\definecolor{currentstroke}{rgb}{0.827451,0.827451,0.827451}%
\pgfsetstrokecolor{currentstroke}%
\pgfsetdash{}{0pt}%
\pgfpathmoveto{\pgfqpoint{0.788555in}{2.905777in}}%
\pgfpathlineto{\pgfqpoint{0.780416in}{2.904455in}}%
\pgfpathlineto{\pgfqpoint{0.774525in}{2.907178in}}%
\pgfpathlineto{\pgfqpoint{0.771950in}{2.911611in}}%
\pgfpathlineto{\pgfqpoint{0.774868in}{2.917970in}}%
\pgfpathlineto{\pgfqpoint{0.781368in}{2.916288in}}%
\pgfpathlineto{\pgfqpoint{0.792051in}{2.920061in}}%
\pgfpathlineto{\pgfqpoint{0.795793in}{2.914849in}}%
\pgfpathlineto{\pgfqpoint{0.799421in}{2.915812in}}%
\pgfpathlineto{\pgfqpoint{0.808203in}{2.912569in}}%
\pgfpathlineto{\pgfqpoint{0.811998in}{2.908449in}}%
\pgfpathlineto{\pgfqpoint{0.807367in}{2.897674in}}%
\pgfpathlineto{\pgfqpoint{0.802761in}{2.895390in}}%
\pgfpathlineto{\pgfqpoint{0.798621in}{2.896621in}}%
\pgfpathclose%
\pgfusepath{stroke,fill}%
\end{pgfscope}%
\begin{pgfscope}%
\pgfpathrectangle{\pgfqpoint{0.100000in}{2.413063in}}{\pgfqpoint{5.037500in}{3.427208in}}%
\pgfusepath{clip}%
\pgfsetbuttcap%
\pgfsetmiterjoin%
\definecolor{currentfill}{rgb}{1.000000,1.000000,1.000000}%
\pgfsetfillcolor{currentfill}%
\pgfsetlinewidth{0.501875pt}%
\definecolor{currentstroke}{rgb}{0.827451,0.827451,0.827451}%
\pgfsetstrokecolor{currentstroke}%
\pgfsetdash{}{0pt}%
\pgfpathmoveto{\pgfqpoint{0.715775in}{2.915567in}}%
\pgfpathlineto{\pgfqpoint{0.704241in}{2.921069in}}%
\pgfpathlineto{\pgfqpoint{0.693568in}{2.922705in}}%
\pgfpathlineto{\pgfqpoint{0.690204in}{2.925048in}}%
\pgfpathlineto{\pgfqpoint{0.694035in}{2.929660in}}%
\pgfpathlineto{\pgfqpoint{0.700833in}{2.923899in}}%
\pgfpathlineto{\pgfqpoint{0.706535in}{2.924693in}}%
\pgfpathlineto{\pgfqpoint{0.715807in}{2.928683in}}%
\pgfpathlineto{\pgfqpoint{0.716557in}{2.933822in}}%
\pgfpathlineto{\pgfqpoint{0.727946in}{2.931278in}}%
\pgfpathlineto{\pgfqpoint{0.725279in}{2.923541in}}%
\pgfpathlineto{\pgfqpoint{0.732535in}{2.922975in}}%
\pgfpathlineto{\pgfqpoint{0.726447in}{2.913772in}}%
\pgfpathlineto{\pgfqpoint{0.720552in}{2.916755in}}%
\pgfpathclose%
\pgfusepath{stroke,fill}%
\end{pgfscope}%
\begin{pgfscope}%
\pgfpathrectangle{\pgfqpoint{0.100000in}{2.413063in}}{\pgfqpoint{5.037500in}{3.427208in}}%
\pgfusepath{clip}%
\pgfsetbuttcap%
\pgfsetmiterjoin%
\definecolor{currentfill}{rgb}{1.000000,1.000000,1.000000}%
\pgfsetfillcolor{currentfill}%
\pgfsetlinewidth{0.501875pt}%
\definecolor{currentstroke}{rgb}{0.827451,0.827451,0.827451}%
\pgfsetstrokecolor{currentstroke}%
\pgfsetdash{}{0pt}%
\pgfpathmoveto{\pgfqpoint{0.695511in}{2.934587in}}%
\pgfpathlineto{\pgfqpoint{0.691272in}{2.932282in}}%
\pgfpathlineto{\pgfqpoint{0.679721in}{2.935825in}}%
\pgfpathlineto{\pgfqpoint{0.671090in}{2.933773in}}%
\pgfpathlineto{\pgfqpoint{0.666705in}{2.939602in}}%
\pgfpathlineto{\pgfqpoint{0.668618in}{2.942271in}}%
\pgfpathlineto{\pgfqpoint{0.675159in}{2.942717in}}%
\pgfpathlineto{\pgfqpoint{0.680162in}{2.941042in}}%
\pgfpathlineto{\pgfqpoint{0.685614in}{2.943693in}}%
\pgfpathlineto{\pgfqpoint{0.693972in}{2.940147in}}%
\pgfpathclose%
\pgfusepath{stroke,fill}%
\end{pgfscope}%
\begin{pgfscope}%
\pgfpathrectangle{\pgfqpoint{0.100000in}{2.413063in}}{\pgfqpoint{5.037500in}{3.427208in}}%
\pgfusepath{clip}%
\pgfsetbuttcap%
\pgfsetmiterjoin%
\definecolor{currentfill}{rgb}{1.000000,1.000000,1.000000}%
\pgfsetfillcolor{currentfill}%
\pgfsetlinewidth{0.501875pt}%
\definecolor{currentstroke}{rgb}{0.827451,0.827451,0.827451}%
\pgfsetstrokecolor{currentstroke}%
\pgfsetdash{}{0pt}%
\pgfpathmoveto{\pgfqpoint{0.559354in}{3.020376in}}%
\pgfpathlineto{\pgfqpoint{0.559047in}{3.014660in}}%
\pgfpathlineto{\pgfqpoint{0.550615in}{3.011068in}}%
\pgfpathlineto{\pgfqpoint{0.542915in}{3.020060in}}%
\pgfpathlineto{\pgfqpoint{0.551550in}{3.023056in}}%
\pgfpathclose%
\pgfusepath{stroke,fill}%
\end{pgfscope}%
\begin{pgfscope}%
\pgfpathrectangle{\pgfqpoint{0.100000in}{2.413063in}}{\pgfqpoint{5.037500in}{3.427208in}}%
\pgfusepath{clip}%
\pgfsetbuttcap%
\pgfsetmiterjoin%
\definecolor{currentfill}{rgb}{1.000000,1.000000,1.000000}%
\pgfsetfillcolor{currentfill}%
\pgfsetlinewidth{0.501875pt}%
\definecolor{currentstroke}{rgb}{0.827451,0.827451,0.827451}%
\pgfsetstrokecolor{currentstroke}%
\pgfsetdash{}{0pt}%
\pgfpathmoveto{\pgfqpoint{0.489807in}{3.052507in}}%
\pgfpathlineto{\pgfqpoint{0.491597in}{3.054992in}}%
\pgfpathlineto{\pgfqpoint{0.501941in}{3.054564in}}%
\pgfpathlineto{\pgfqpoint{0.503755in}{3.058261in}}%
\pgfpathlineto{\pgfqpoint{0.508293in}{3.055035in}}%
\pgfpathlineto{\pgfqpoint{0.505673in}{3.048828in}}%
\pgfpathlineto{\pgfqpoint{0.501818in}{3.048302in}}%
\pgfpathlineto{\pgfqpoint{0.492605in}{3.050250in}}%
\pgfpathclose%
\pgfusepath{stroke,fill}%
\end{pgfscope}%
\begin{pgfscope}%
\pgfpathrectangle{\pgfqpoint{0.100000in}{2.413063in}}{\pgfqpoint{5.037500in}{3.427208in}}%
\pgfusepath{clip}%
\pgfsetbuttcap%
\pgfsetmiterjoin%
\definecolor{currentfill}{rgb}{1.000000,1.000000,1.000000}%
\pgfsetfillcolor{currentfill}%
\pgfsetlinewidth{0.501875pt}%
\definecolor{currentstroke}{rgb}{0.827451,0.827451,0.827451}%
\pgfsetstrokecolor{currentstroke}%
\pgfsetdash{}{0pt}%
\pgfpathmoveto{\pgfqpoint{0.475139in}{3.071744in}}%
\pgfpathlineto{\pgfqpoint{0.474512in}{3.077342in}}%
\pgfpathlineto{\pgfqpoint{0.479908in}{3.076928in}}%
\pgfpathlineto{\pgfqpoint{0.479533in}{3.084790in}}%
\pgfpathlineto{\pgfqpoint{0.484961in}{3.080547in}}%
\pgfpathlineto{\pgfqpoint{0.483083in}{3.074414in}}%
\pgfpathclose%
\pgfusepath{stroke,fill}%
\end{pgfscope}%
\begin{pgfscope}%
\pgfpathrectangle{\pgfqpoint{0.100000in}{2.413063in}}{\pgfqpoint{5.037500in}{3.427208in}}%
\pgfusepath{clip}%
\pgfsetbuttcap%
\pgfsetmiterjoin%
\definecolor{currentfill}{rgb}{1.000000,1.000000,1.000000}%
\pgfsetfillcolor{currentfill}%
\pgfsetlinewidth{0.501875pt}%
\definecolor{currentstroke}{rgb}{0.827451,0.827451,0.827451}%
\pgfsetstrokecolor{currentstroke}%
\pgfsetdash{}{0pt}%
\pgfpathmoveto{\pgfqpoint{0.950612in}{3.271270in}}%
\pgfpathlineto{\pgfqpoint{0.942005in}{3.273128in}}%
\pgfpathlineto{\pgfqpoint{0.940155in}{3.278880in}}%
\pgfpathlineto{\pgfqpoint{0.943114in}{3.284630in}}%
\pgfpathlineto{\pgfqpoint{0.949739in}{3.287633in}}%
\pgfpathlineto{\pgfqpoint{0.957325in}{3.272874in}}%
\pgfpathlineto{\pgfqpoint{0.964205in}{3.272219in}}%
\pgfpathlineto{\pgfqpoint{0.969338in}{3.267664in}}%
\pgfpathlineto{\pgfqpoint{0.969238in}{3.261398in}}%
\pgfpathlineto{\pgfqpoint{0.965484in}{3.258394in}}%
\pgfpathlineto{\pgfqpoint{0.971588in}{3.239861in}}%
\pgfpathlineto{\pgfqpoint{0.978267in}{3.233060in}}%
\pgfpathlineto{\pgfqpoint{0.971438in}{3.230902in}}%
\pgfpathlineto{\pgfqpoint{0.965912in}{3.238533in}}%
\pgfpathlineto{\pgfqpoint{0.953900in}{3.236170in}}%
\pgfpathlineto{\pgfqpoint{0.957635in}{3.243721in}}%
\pgfpathlineto{\pgfqpoint{0.955896in}{3.246729in}}%
\pgfpathlineto{\pgfqpoint{0.956916in}{3.254600in}}%
\pgfpathlineto{\pgfqpoint{0.954536in}{3.268829in}}%
\pgfpathclose%
\pgfusepath{stroke,fill}%
\end{pgfscope}%
\begin{pgfscope}%
\pgfpathrectangle{\pgfqpoint{0.100000in}{2.413063in}}{\pgfqpoint{5.037500in}{3.427208in}}%
\pgfusepath{clip}%
\pgfsetbuttcap%
\pgfsetmiterjoin%
\definecolor{currentfill}{rgb}{1.000000,1.000000,1.000000}%
\pgfsetfillcolor{currentfill}%
\pgfsetlinewidth{0.501875pt}%
\definecolor{currentstroke}{rgb}{0.827451,0.827451,0.827451}%
\pgfsetstrokecolor{currentstroke}%
\pgfsetdash{}{0pt}%
\pgfpathmoveto{\pgfqpoint{1.253018in}{2.847315in}}%
\pgfpathlineto{\pgfqpoint{1.240541in}{2.847455in}}%
\pgfpathlineto{\pgfqpoint{1.241502in}{2.853719in}}%
\pgfpathlineto{\pgfqpoint{1.246496in}{2.855959in}}%
\pgfpathclose%
\pgfusepath{stroke,fill}%
\end{pgfscope}%
\begin{pgfscope}%
\pgfpathrectangle{\pgfqpoint{0.100000in}{2.413063in}}{\pgfqpoint{5.037500in}{3.427208in}}%
\pgfusepath{clip}%
\pgfsetbuttcap%
\pgfsetmiterjoin%
\definecolor{currentfill}{rgb}{1.000000,1.000000,1.000000}%
\pgfsetfillcolor{currentfill}%
\pgfsetlinewidth{0.501875pt}%
\definecolor{currentstroke}{rgb}{0.827451,0.827451,0.827451}%
\pgfsetstrokecolor{currentstroke}%
\pgfsetdash{}{0pt}%
\pgfpathmoveto{\pgfqpoint{1.235704in}{2.852987in}}%
\pgfpathlineto{\pgfqpoint{1.226764in}{2.850825in}}%
\pgfpathlineto{\pgfqpoint{1.218270in}{2.847120in}}%
\pgfpathlineto{\pgfqpoint{1.215647in}{2.851485in}}%
\pgfpathlineto{\pgfqpoint{1.230255in}{2.854439in}}%
\pgfpathclose%
\pgfusepath{stroke,fill}%
\end{pgfscope}%
\begin{pgfscope}%
\pgfpathrectangle{\pgfqpoint{0.100000in}{2.413063in}}{\pgfqpoint{5.037500in}{3.427208in}}%
\pgfusepath{clip}%
\pgfsetbuttcap%
\pgfsetmiterjoin%
\definecolor{currentfill}{rgb}{1.000000,1.000000,1.000000}%
\pgfsetfillcolor{currentfill}%
\pgfsetlinewidth{0.501875pt}%
\definecolor{currentstroke}{rgb}{0.827451,0.827451,0.827451}%
\pgfsetstrokecolor{currentstroke}%
\pgfsetdash{}{0pt}%
\pgfpathmoveto{\pgfqpoint{1.098678in}{2.848553in}}%
\pgfpathlineto{\pgfqpoint{1.099020in}{2.844222in}}%
\pgfpathlineto{\pgfqpoint{1.093868in}{2.841529in}}%
\pgfpathlineto{\pgfqpoint{1.079494in}{2.847704in}}%
\pgfpathlineto{\pgfqpoint{1.077265in}{2.845220in}}%
\pgfpathlineto{\pgfqpoint{1.073788in}{2.856971in}}%
\pgfpathlineto{\pgfqpoint{1.080989in}{2.858606in}}%
\pgfpathlineto{\pgfqpoint{1.084318in}{2.853777in}}%
\pgfpathlineto{\pgfqpoint{1.091547in}{2.859822in}}%
\pgfpathclose%
\pgfusepath{stroke,fill}%
\end{pgfscope}%
\begin{pgfscope}%
\pgfpathrectangle{\pgfqpoint{0.100000in}{2.413063in}}{\pgfqpoint{5.037500in}{3.427208in}}%
\pgfusepath{clip}%
\pgfsetbuttcap%
\pgfsetmiterjoin%
\definecolor{currentfill}{rgb}{1.000000,1.000000,1.000000}%
\pgfsetfillcolor{currentfill}%
\pgfsetlinewidth{0.501875pt}%
\definecolor{currentstroke}{rgb}{0.827451,0.827451,0.827451}%
\pgfsetstrokecolor{currentstroke}%
\pgfsetdash{}{0pt}%
\pgfpathmoveto{\pgfqpoint{1.422926in}{2.608438in}}%
\pgfpathlineto{\pgfqpoint{1.410872in}{2.603950in}}%
\pgfpathlineto{\pgfqpoint{1.404488in}{2.603681in}}%
\pgfpathlineto{\pgfqpoint{1.409684in}{2.613202in}}%
\pgfpathlineto{\pgfqpoint{1.414124in}{2.616752in}}%
\pgfpathlineto{\pgfqpoint{1.413981in}{2.626611in}}%
\pgfpathlineto{\pgfqpoint{1.418807in}{2.644409in}}%
\pgfpathlineto{\pgfqpoint{1.423252in}{2.647789in}}%
\pgfpathlineto{\pgfqpoint{1.433513in}{2.642895in}}%
\pgfpathlineto{\pgfqpoint{1.432015in}{2.640091in}}%
\pgfpathlineto{\pgfqpoint{1.432792in}{2.626619in}}%
\pgfpathlineto{\pgfqpoint{1.429465in}{2.622982in}}%
\pgfpathlineto{\pgfqpoint{1.428037in}{2.613413in}}%
\pgfpathclose%
\pgfusepath{stroke,fill}%
\end{pgfscope}%
\begin{pgfscope}%
\pgfpathrectangle{\pgfqpoint{0.100000in}{2.413063in}}{\pgfqpoint{5.037500in}{3.427208in}}%
\pgfusepath{clip}%
\pgfsetbuttcap%
\pgfsetmiterjoin%
\definecolor{currentfill}{rgb}{1.000000,1.000000,1.000000}%
\pgfsetfillcolor{currentfill}%
\pgfsetlinewidth{0.501875pt}%
\definecolor{currentstroke}{rgb}{0.827451,0.827451,0.827451}%
\pgfsetstrokecolor{currentstroke}%
\pgfsetdash{}{0pt}%
\pgfpathmoveto{\pgfqpoint{1.396233in}{2.652557in}}%
\pgfpathlineto{\pgfqpoint{1.403294in}{2.640364in}}%
\pgfpathlineto{\pgfqpoint{1.401985in}{2.637455in}}%
\pgfpathlineto{\pgfqpoint{1.411430in}{2.634498in}}%
\pgfpathlineto{\pgfqpoint{1.408718in}{2.623290in}}%
\pgfpathlineto{\pgfqpoint{1.404606in}{2.623808in}}%
\pgfpathlineto{\pgfqpoint{1.393800in}{2.643346in}}%
\pgfpathlineto{\pgfqpoint{1.395336in}{2.634553in}}%
\pgfpathlineto{\pgfqpoint{1.393496in}{2.629488in}}%
\pgfpathlineto{\pgfqpoint{1.388920in}{2.627235in}}%
\pgfpathlineto{\pgfqpoint{1.386318in}{2.629663in}}%
\pgfpathlineto{\pgfqpoint{1.384866in}{2.640922in}}%
\pgfpathlineto{\pgfqpoint{1.385723in}{2.648224in}}%
\pgfpathlineto{\pgfqpoint{1.382675in}{2.651411in}}%
\pgfpathlineto{\pgfqpoint{1.390726in}{2.660634in}}%
\pgfpathlineto{\pgfqpoint{1.396009in}{2.663371in}}%
\pgfpathlineto{\pgfqpoint{1.397768in}{2.659775in}}%
\pgfpathlineto{\pgfqpoint{1.403280in}{2.662558in}}%
\pgfpathlineto{\pgfqpoint{1.407273in}{2.655001in}}%
\pgfpathlineto{\pgfqpoint{1.415909in}{2.645386in}}%
\pgfpathlineto{\pgfqpoint{1.414730in}{2.641031in}}%
\pgfpathlineto{\pgfqpoint{1.409935in}{2.636286in}}%
\pgfpathlineto{\pgfqpoint{1.405501in}{2.638522in}}%
\pgfpathclose%
\pgfusepath{stroke,fill}%
\end{pgfscope}%
\begin{pgfscope}%
\pgfpathrectangle{\pgfqpoint{0.100000in}{2.413063in}}{\pgfqpoint{5.037500in}{3.427208in}}%
\pgfusepath{clip}%
\pgfsetbuttcap%
\pgfsetmiterjoin%
\definecolor{currentfill}{rgb}{1.000000,1.000000,1.000000}%
\pgfsetfillcolor{currentfill}%
\pgfsetlinewidth{0.501875pt}%
\definecolor{currentstroke}{rgb}{0.827451,0.827451,0.827451}%
\pgfsetstrokecolor{currentstroke}%
\pgfsetdash{}{0pt}%
\pgfpathmoveto{\pgfqpoint{1.045813in}{2.825442in}}%
\pgfpathlineto{\pgfqpoint{1.037066in}{2.824304in}}%
\pgfpathlineto{\pgfqpoint{1.024746in}{2.819568in}}%
\pgfpathlineto{\pgfqpoint{1.021873in}{2.822141in}}%
\pgfpathlineto{\pgfqpoint{1.034741in}{2.825049in}}%
\pgfpathlineto{\pgfqpoint{1.031292in}{2.827587in}}%
\pgfpathlineto{\pgfqpoint{1.023313in}{2.829445in}}%
\pgfpathlineto{\pgfqpoint{1.021183in}{2.835391in}}%
\pgfpathlineto{\pgfqpoint{1.025198in}{2.840514in}}%
\pgfpathlineto{\pgfqpoint{1.028578in}{2.852270in}}%
\pgfpathlineto{\pgfqpoint{1.040928in}{2.854460in}}%
\pgfpathlineto{\pgfqpoint{1.046907in}{2.849381in}}%
\pgfpathlineto{\pgfqpoint{1.052523in}{2.854297in}}%
\pgfpathlineto{\pgfqpoint{1.059567in}{2.853091in}}%
\pgfpathlineto{\pgfqpoint{1.064629in}{2.844012in}}%
\pgfpathlineto{\pgfqpoint{1.068916in}{2.852081in}}%
\pgfpathlineto{\pgfqpoint{1.081083in}{2.833205in}}%
\pgfpathlineto{\pgfqpoint{1.076688in}{2.831696in}}%
\pgfpathlineto{\pgfqpoint{1.073990in}{2.827604in}}%
\pgfpathlineto{\pgfqpoint{1.079087in}{2.823870in}}%
\pgfpathlineto{\pgfqpoint{1.071184in}{2.820369in}}%
\pgfpathlineto{\pgfqpoint{1.065454in}{2.822297in}}%
\pgfpathlineto{\pgfqpoint{1.054630in}{2.828686in}}%
\pgfpathclose%
\pgfusepath{stroke,fill}%
\end{pgfscope}%
\begin{pgfscope}%
\pgfpathrectangle{\pgfqpoint{0.100000in}{2.413063in}}{\pgfqpoint{5.037500in}{3.427208in}}%
\pgfusepath{clip}%
\pgfsetbuttcap%
\pgfsetmiterjoin%
\definecolor{currentfill}{rgb}{1.000000,1.000000,1.000000}%
\pgfsetfillcolor{currentfill}%
\pgfsetlinewidth{0.501875pt}%
\definecolor{currentstroke}{rgb}{0.827451,0.827451,0.827451}%
\pgfsetstrokecolor{currentstroke}%
\pgfsetdash{}{0pt}%
\pgfpathmoveto{\pgfqpoint{1.388888in}{2.572423in}}%
\pgfpathlineto{\pgfqpoint{1.387153in}{2.589856in}}%
\pgfpathlineto{\pgfqpoint{1.391828in}{2.596257in}}%
\pgfpathlineto{\pgfqpoint{1.386830in}{2.596702in}}%
\pgfpathlineto{\pgfqpoint{1.388657in}{2.609151in}}%
\pgfpathlineto{\pgfqpoint{1.390811in}{2.616542in}}%
\pgfpathlineto{\pgfqpoint{1.388939in}{2.626471in}}%
\pgfpathlineto{\pgfqpoint{1.393517in}{2.628394in}}%
\pgfpathlineto{\pgfqpoint{1.395011in}{2.630962in}}%
\pgfpathlineto{\pgfqpoint{1.399522in}{2.629988in}}%
\pgfpathlineto{\pgfqpoint{1.403426in}{2.622032in}}%
\pgfpathlineto{\pgfqpoint{1.404151in}{2.614744in}}%
\pgfpathlineto{\pgfqpoint{1.399375in}{2.592915in}}%
\pgfpathclose%
\pgfusepath{stroke,fill}%
\end{pgfscope}%
\begin{pgfscope}%
\pgfpathrectangle{\pgfqpoint{0.100000in}{2.413063in}}{\pgfqpoint{5.037500in}{3.427208in}}%
\pgfusepath{clip}%
\pgfsetbuttcap%
\pgfsetmiterjoin%
\definecolor{currentfill}{rgb}{1.000000,1.000000,1.000000}%
\pgfsetfillcolor{currentfill}%
\pgfsetlinewidth{0.501875pt}%
\definecolor{currentstroke}{rgb}{0.827451,0.827451,0.827451}%
\pgfsetstrokecolor{currentstroke}%
\pgfsetdash{}{0pt}%
\pgfpathmoveto{\pgfqpoint{1.387850in}{2.625640in}}%
\pgfpathlineto{\pgfqpoint{1.387611in}{2.617140in}}%
\pgfpathlineto{\pgfqpoint{1.383691in}{2.613281in}}%
\pgfpathlineto{\pgfqpoint{1.379160in}{2.614707in}}%
\pgfpathlineto{\pgfqpoint{1.384887in}{2.627054in}}%
\pgfpathclose%
\pgfusepath{stroke,fill}%
\end{pgfscope}%
\begin{pgfscope}%
\pgfpathrectangle{\pgfqpoint{0.100000in}{2.413063in}}{\pgfqpoint{5.037500in}{3.427208in}}%
\pgfusepath{clip}%
\pgfsetbuttcap%
\pgfsetmiterjoin%
\definecolor{currentfill}{rgb}{1.000000,1.000000,1.000000}%
\pgfsetfillcolor{currentfill}%
\pgfsetlinewidth{0.501875pt}%
\definecolor{currentstroke}{rgb}{0.827451,0.827451,0.827451}%
\pgfsetstrokecolor{currentstroke}%
\pgfsetdash{}{0pt}%
\pgfpathmoveto{\pgfqpoint{1.435950in}{2.587901in}}%
\pgfpathlineto{\pgfqpoint{1.432607in}{2.573676in}}%
\pgfpathlineto{\pgfqpoint{1.424819in}{2.569179in}}%
\pgfpathlineto{\pgfqpoint{1.414600in}{2.573322in}}%
\pgfpathlineto{\pgfqpoint{1.417416in}{2.578850in}}%
\pgfpathlineto{\pgfqpoint{1.417359in}{2.582678in}}%
\pgfpathlineto{\pgfqpoint{1.421158in}{2.589182in}}%
\pgfpathlineto{\pgfqpoint{1.418488in}{2.587873in}}%
\pgfpathlineto{\pgfqpoint{1.420434in}{2.591049in}}%
\pgfpathlineto{\pgfqpoint{1.416736in}{2.598265in}}%
\pgfpathlineto{\pgfqpoint{1.420933in}{2.599596in}}%
\pgfpathlineto{\pgfqpoint{1.430648in}{2.591152in}}%
\pgfpathclose%
\pgfusepath{stroke,fill}%
\end{pgfscope}%
\begin{pgfscope}%
\pgfpathrectangle{\pgfqpoint{0.100000in}{2.413063in}}{\pgfqpoint{5.037500in}{3.427208in}}%
\pgfusepath{clip}%
\pgfsetbuttcap%
\pgfsetmiterjoin%
\definecolor{currentfill}{rgb}{1.000000,1.000000,1.000000}%
\pgfsetfillcolor{currentfill}%
\pgfsetlinewidth{0.501875pt}%
\definecolor{currentstroke}{rgb}{0.827451,0.827451,0.827451}%
\pgfsetstrokecolor{currentstroke}%
\pgfsetdash{}{0pt}%
\pgfpathmoveto{\pgfqpoint{1.401856in}{2.562748in}}%
\pgfpathlineto{\pgfqpoint{1.397450in}{2.561035in}}%
\pgfpathlineto{\pgfqpoint{1.399966in}{2.576046in}}%
\pgfpathlineto{\pgfqpoint{1.405253in}{2.579549in}}%
\pgfpathlineto{\pgfqpoint{1.403964in}{2.590797in}}%
\pgfpathlineto{\pgfqpoint{1.406113in}{2.595784in}}%
\pgfpathlineto{\pgfqpoint{1.410363in}{2.597653in}}%
\pgfpathlineto{\pgfqpoint{1.414824in}{2.592761in}}%
\pgfpathlineto{\pgfqpoint{1.418653in}{2.586122in}}%
\pgfpathlineto{\pgfqpoint{1.406344in}{2.573117in}}%
\pgfpathclose%
\pgfusepath{stroke,fill}%
\end{pgfscope}%
\begin{pgfscope}%
\pgfpathrectangle{\pgfqpoint{0.100000in}{2.413063in}}{\pgfqpoint{5.037500in}{3.427208in}}%
\pgfusepath{clip}%
\pgfsetbuttcap%
\pgfsetmiterjoin%
\definecolor{currentfill}{rgb}{1.000000,1.000000,1.000000}%
\pgfsetfillcolor{currentfill}%
\pgfsetlinewidth{0.501875pt}%
\definecolor{currentstroke}{rgb}{0.827451,0.827451,0.827451}%
\pgfsetstrokecolor{currentstroke}%
\pgfsetdash{}{0pt}%
\pgfpathmoveto{\pgfqpoint{1.437904in}{2.578393in}}%
\pgfpathlineto{\pgfqpoint{1.440063in}{2.568035in}}%
\pgfpathlineto{\pgfqpoint{1.430417in}{2.569220in}}%
\pgfpathclose%
\pgfusepath{stroke,fill}%
\end{pgfscope}%
\begin{pgfscope}%
\pgfpathrectangle{\pgfqpoint{0.100000in}{2.413063in}}{\pgfqpoint{5.037500in}{3.427208in}}%
\pgfusepath{clip}%
\pgfsetbuttcap%
\pgfsetmiterjoin%
\definecolor{currentfill}{rgb}{1.000000,1.000000,1.000000}%
\pgfsetfillcolor{currentfill}%
\pgfsetlinewidth{0.501875pt}%
\definecolor{currentstroke}{rgb}{0.827451,0.827451,0.827451}%
\pgfsetstrokecolor{currentstroke}%
\pgfsetdash{}{0pt}%
\pgfpathmoveto{\pgfqpoint{1.442508in}{2.562566in}}%
\pgfpathlineto{\pgfqpoint{1.444088in}{2.556175in}}%
\pgfpathlineto{\pgfqpoint{1.447114in}{2.554131in}}%
\pgfpathlineto{\pgfqpoint{1.446563in}{2.548116in}}%
\pgfpathlineto{\pgfqpoint{1.442336in}{2.546093in}}%
\pgfpathlineto{\pgfqpoint{1.438904in}{2.555072in}}%
\pgfpathclose%
\pgfusepath{stroke,fill}%
\end{pgfscope}%
\begin{pgfscope}%
\pgfpathrectangle{\pgfqpoint{0.100000in}{2.413063in}}{\pgfqpoint{5.037500in}{3.427208in}}%
\pgfusepath{clip}%
\pgfsetbuttcap%
\pgfsetmiterjoin%
\definecolor{currentfill}{rgb}{1.000000,1.000000,1.000000}%
\pgfsetfillcolor{currentfill}%
\pgfsetlinewidth{0.501875pt}%
\definecolor{currentstroke}{rgb}{0.827451,0.827451,0.827451}%
\pgfsetstrokecolor{currentstroke}%
\pgfsetdash{}{0pt}%
\pgfpathmoveto{\pgfqpoint{1.434457in}{2.565017in}}%
\pgfpathlineto{\pgfqpoint{1.431834in}{2.557594in}}%
\pgfpathlineto{\pgfqpoint{1.427759in}{2.557867in}}%
\pgfpathlineto{\pgfqpoint{1.425189in}{2.563988in}}%
\pgfpathlineto{\pgfqpoint{1.429376in}{2.566887in}}%
\pgfpathclose%
\pgfusepath{stroke,fill}%
\end{pgfscope}%
\begin{pgfscope}%
\pgfpathrectangle{\pgfqpoint{0.100000in}{2.413063in}}{\pgfqpoint{5.037500in}{3.427208in}}%
\pgfusepath{clip}%
\pgfsetbuttcap%
\pgfsetmiterjoin%
\definecolor{currentfill}{rgb}{1.000000,1.000000,1.000000}%
\pgfsetfillcolor{currentfill}%
\pgfsetlinewidth{0.501875pt}%
\definecolor{currentstroke}{rgb}{0.827451,0.827451,0.827451}%
\pgfsetstrokecolor{currentstroke}%
\pgfsetdash{}{0pt}%
\pgfpathmoveto{\pgfqpoint{1.423464in}{2.526748in}}%
\pgfpathlineto{\pgfqpoint{1.427100in}{2.522771in}}%
\pgfpathlineto{\pgfqpoint{1.427901in}{2.514391in}}%
\pgfpathlineto{\pgfqpoint{1.429814in}{2.511956in}}%
\pgfpathlineto{\pgfqpoint{1.426689in}{2.504374in}}%
\pgfpathlineto{\pgfqpoint{1.422577in}{2.500104in}}%
\pgfpathlineto{\pgfqpoint{1.420388in}{2.492184in}}%
\pgfpathlineto{\pgfqpoint{1.415893in}{2.504527in}}%
\pgfpathlineto{\pgfqpoint{1.415148in}{2.515734in}}%
\pgfpathlineto{\pgfqpoint{1.412633in}{2.521313in}}%
\pgfpathlineto{\pgfqpoint{1.403970in}{2.525836in}}%
\pgfpathlineto{\pgfqpoint{1.412096in}{2.535268in}}%
\pgfpathlineto{\pgfqpoint{1.406679in}{2.539299in}}%
\pgfpathlineto{\pgfqpoint{1.411451in}{2.543316in}}%
\pgfpathlineto{\pgfqpoint{1.417787in}{2.558739in}}%
\pgfpathlineto{\pgfqpoint{1.411220in}{2.564177in}}%
\pgfpathlineto{\pgfqpoint{1.413750in}{2.569397in}}%
\pgfpathlineto{\pgfqpoint{1.422238in}{2.564766in}}%
\pgfpathlineto{\pgfqpoint{1.423098in}{2.557458in}}%
\pgfpathlineto{\pgfqpoint{1.419886in}{2.553296in}}%
\pgfpathlineto{\pgfqpoint{1.426250in}{2.547666in}}%
\pgfpathlineto{\pgfqpoint{1.428254in}{2.538896in}}%
\pgfpathlineto{\pgfqpoint{1.427982in}{2.526595in}}%
\pgfpathclose%
\pgfusepath{stroke,fill}%
\end{pgfscope}%
\begin{pgfscope}%
\pgfpathrectangle{\pgfqpoint{0.100000in}{2.413063in}}{\pgfqpoint{5.037500in}{3.427208in}}%
\pgfusepath{clip}%
\pgfsetbuttcap%
\pgfsetmiterjoin%
\definecolor{currentfill}{rgb}{1.000000,1.000000,1.000000}%
\pgfsetfillcolor{currentfill}%
\pgfsetlinewidth{0.501875pt}%
\definecolor{currentstroke}{rgb}{0.827451,0.827451,0.827451}%
\pgfsetstrokecolor{currentstroke}%
\pgfsetdash{}{0pt}%
\pgfpathmoveto{\pgfqpoint{1.439005in}{2.557624in}}%
\pgfpathlineto{\pgfqpoint{1.437026in}{2.553358in}}%
\pgfpathlineto{\pgfqpoint{1.438885in}{2.546429in}}%
\pgfpathlineto{\pgfqpoint{1.434125in}{2.545847in}}%
\pgfpathlineto{\pgfqpoint{1.429134in}{2.549054in}}%
\pgfpathlineto{\pgfqpoint{1.430754in}{2.556120in}}%
\pgfpathclose%
\pgfusepath{stroke,fill}%
\end{pgfscope}%
\begin{pgfscope}%
\pgfpathrectangle{\pgfqpoint{0.100000in}{2.413063in}}{\pgfqpoint{5.037500in}{3.427208in}}%
\pgfusepath{clip}%
\pgfsetbuttcap%
\pgfsetmiterjoin%
\definecolor{currentfill}{rgb}{1.000000,1.000000,1.000000}%
\pgfsetfillcolor{currentfill}%
\pgfsetlinewidth{0.501875pt}%
\definecolor{currentstroke}{rgb}{0.827451,0.827451,0.827451}%
\pgfsetstrokecolor{currentstroke}%
\pgfsetdash{}{0pt}%
\pgfpathmoveto{\pgfqpoint{1.416821in}{2.558107in}}%
\pgfpathlineto{\pgfqpoint{1.406314in}{2.553395in}}%
\pgfpathlineto{\pgfqpoint{1.402743in}{2.555090in}}%
\pgfpathlineto{\pgfqpoint{1.409908in}{2.561307in}}%
\pgfpathclose%
\pgfusepath{stroke,fill}%
\end{pgfscope}%
\begin{pgfscope}%
\pgfpathrectangle{\pgfqpoint{0.100000in}{2.413063in}}{\pgfqpoint{5.037500in}{3.427208in}}%
\pgfusepath{clip}%
\pgfsetbuttcap%
\pgfsetmiterjoin%
\definecolor{currentfill}{rgb}{1.000000,1.000000,1.000000}%
\pgfsetfillcolor{currentfill}%
\pgfsetlinewidth{0.501875pt}%
\definecolor{currentstroke}{rgb}{0.827451,0.827451,0.827451}%
\pgfsetstrokecolor{currentstroke}%
\pgfsetdash{}{0pt}%
\pgfpathmoveto{\pgfqpoint{1.439121in}{2.512137in}}%
\pgfpathlineto{\pgfqpoint{1.436616in}{2.519232in}}%
\pgfpathlineto{\pgfqpoint{1.441914in}{2.520977in}}%
\pgfpathlineto{\pgfqpoint{1.443518in}{2.528629in}}%
\pgfpathlineto{\pgfqpoint{1.449259in}{2.528637in}}%
\pgfpathlineto{\pgfqpoint{1.449235in}{2.533965in}}%
\pgfpathlineto{\pgfqpoint{1.457214in}{2.532705in}}%
\pgfpathlineto{\pgfqpoint{1.458574in}{2.518262in}}%
\pgfpathlineto{\pgfqpoint{1.453946in}{2.509054in}}%
\pgfpathlineto{\pgfqpoint{1.447079in}{2.503302in}}%
\pgfpathclose%
\pgfusepath{stroke,fill}%
\end{pgfscope}%
\begin{pgfscope}%
\pgfpathrectangle{\pgfqpoint{0.100000in}{2.413063in}}{\pgfqpoint{5.037500in}{3.427208in}}%
\pgfusepath{clip}%
\pgfsetbuttcap%
\pgfsetmiterjoin%
\definecolor{currentfill}{rgb}{1.000000,1.000000,1.000000}%
\pgfsetfillcolor{currentfill}%
\pgfsetlinewidth{0.501875pt}%
\definecolor{currentstroke}{rgb}{0.827451,0.827451,0.827451}%
\pgfsetstrokecolor{currentstroke}%
\pgfsetdash{}{0pt}%
\pgfpathmoveto{\pgfqpoint{1.436019in}{2.517594in}}%
\pgfpathlineto{\pgfqpoint{1.438281in}{2.510876in}}%
\pgfpathlineto{\pgfqpoint{1.432702in}{2.510059in}}%
\pgfpathclose%
\pgfusepath{stroke,fill}%
\end{pgfscope}%
\begin{pgfscope}%
\pgfpathrectangle{\pgfqpoint{0.100000in}{2.413063in}}{\pgfqpoint{5.037500in}{3.427208in}}%
\pgfusepath{clip}%
\pgfsetbuttcap%
\pgfsetmiterjoin%
\definecolor{currentfill}{rgb}{1.000000,1.000000,1.000000}%
\pgfsetfillcolor{currentfill}%
\pgfsetlinewidth{0.501875pt}%
\definecolor{currentstroke}{rgb}{0.827451,0.827451,0.827451}%
\pgfsetstrokecolor{currentstroke}%
\pgfsetdash{}{0pt}%
\pgfpathmoveto{\pgfqpoint{1.440722in}{2.507713in}}%
\pgfpathlineto{\pgfqpoint{1.439944in}{2.499149in}}%
\pgfpathlineto{\pgfqpoint{1.434058in}{2.499858in}}%
\pgfpathlineto{\pgfqpoint{1.435962in}{2.503996in}}%
\pgfpathclose%
\pgfusepath{stroke,fill}%
\end{pgfscope}%
\begin{pgfscope}%
\pgfpathrectangle{\pgfqpoint{0.100000in}{2.413063in}}{\pgfqpoint{5.037500in}{3.427208in}}%
\pgfusepath{clip}%
\pgfsetbuttcap%
\pgfsetroundjoin%
\pgfsetlinewidth{0.501875pt}%
\definecolor{currentstroke}{rgb}{0.827451,0.827451,0.827451}%
\pgfsetstrokecolor{currentstroke}%
\pgfsetdash{}{0pt}%
\pgfusepath{stroke}%
\end{pgfscope}%
\begin{pgfscope}%
\pgfpathrectangle{\pgfqpoint{0.100000in}{2.413063in}}{\pgfqpoint{5.037500in}{3.427208in}}%
\pgfusepath{clip}%
\pgfsetbuttcap%
\pgfsetroundjoin%
\pgfsetlinewidth{0.501875pt}%
\definecolor{currentstroke}{rgb}{0.827451,0.827451,0.827451}%
\pgfsetstrokecolor{currentstroke}%
\pgfsetdash{}{0pt}%
\pgfusepath{stroke}%
\end{pgfscope}%
\begin{pgfscope}%
\pgfpathrectangle{\pgfqpoint{0.100000in}{2.413063in}}{\pgfqpoint{5.037500in}{3.427208in}}%
\pgfusepath{clip}%
\pgfsetbuttcap%
\pgfsetroundjoin%
\pgfsetlinewidth{0.501875pt}%
\definecolor{currentstroke}{rgb}{0.827451,0.827451,0.827451}%
\pgfsetstrokecolor{currentstroke}%
\pgfsetdash{}{0pt}%
\pgfusepath{stroke}%
\end{pgfscope}%
\begin{pgfscope}%
\pgfpathrectangle{\pgfqpoint{0.100000in}{2.413063in}}{\pgfqpoint{5.037500in}{3.427208in}}%
\pgfusepath{clip}%
\pgfsetbuttcap%
\pgfsetroundjoin%
\pgfsetlinewidth{0.501875pt}%
\definecolor{currentstroke}{rgb}{0.827451,0.827451,0.827451}%
\pgfsetstrokecolor{currentstroke}%
\pgfsetdash{}{0pt}%
\pgfusepath{stroke}%
\end{pgfscope}%
\begin{pgfscope}%
\pgfpathrectangle{\pgfqpoint{0.100000in}{2.413063in}}{\pgfqpoint{5.037500in}{3.427208in}}%
\pgfusepath{clip}%
\pgfsetbuttcap%
\pgfsetroundjoin%
\pgfsetlinewidth{0.501875pt}%
\definecolor{currentstroke}{rgb}{0.827451,0.827451,0.827451}%
\pgfsetstrokecolor{currentstroke}%
\pgfsetdash{}{0pt}%
\pgfusepath{stroke}%
\end{pgfscope}%
\begin{pgfscope}%
\pgfpathrectangle{\pgfqpoint{0.100000in}{2.413063in}}{\pgfqpoint{5.037500in}{3.427208in}}%
\pgfusepath{clip}%
\pgfsetbuttcap%
\pgfsetroundjoin%
\pgfsetlinewidth{0.501875pt}%
\definecolor{currentstroke}{rgb}{0.827451,0.827451,0.827451}%
\pgfsetstrokecolor{currentstroke}%
\pgfsetdash{}{0pt}%
\pgfusepath{stroke}%
\end{pgfscope}%
\begin{pgfscope}%
\pgfpathrectangle{\pgfqpoint{0.100000in}{2.413063in}}{\pgfqpoint{5.037500in}{3.427208in}}%
\pgfusepath{clip}%
\pgfsetbuttcap%
\pgfsetroundjoin%
\pgfsetlinewidth{0.501875pt}%
\definecolor{currentstroke}{rgb}{0.827451,0.827451,0.827451}%
\pgfsetstrokecolor{currentstroke}%
\pgfsetdash{}{0pt}%
\pgfpathmoveto{\pgfqpoint{1.129685in}{5.642352in}}%
\pgfpathlineto{\pgfqpoint{1.102423in}{5.534332in}}%
\pgfpathlineto{\pgfqpoint{1.083039in}{5.456934in}}%
\pgfpathlineto{\pgfqpoint{1.059991in}{5.363826in}}%
\pgfpathlineto{\pgfqpoint{1.055466in}{5.340878in}}%
\pgfpathlineto{\pgfqpoint{1.058421in}{5.319829in}}%
\pgfpathlineto{\pgfqpoint{1.054504in}{5.300440in}}%
\pgfpathlineto{\pgfqpoint{0.893620in}{5.342423in}}%
\pgfpathlineto{\pgfqpoint{0.879086in}{5.337547in}}%
\pgfpathlineto{\pgfqpoint{0.866649in}{5.341778in}}%
\pgfpathlineto{\pgfqpoint{0.809066in}{5.343162in}}%
\pgfpathlineto{\pgfqpoint{0.789590in}{5.337671in}}%
\pgfpathlineto{\pgfqpoint{0.770250in}{5.339526in}}%
\pgfpathlineto{\pgfqpoint{0.762052in}{5.348120in}}%
\pgfpathlineto{\pgfqpoint{0.710499in}{5.346247in}}%
\pgfpathlineto{\pgfqpoint{0.702357in}{5.360292in}}%
\pgfpathlineto{\pgfqpoint{0.686901in}{5.367691in}}%
\pgfpathlineto{\pgfqpoint{0.664413in}{5.372200in}}%
\pgfpathlineto{\pgfqpoint{0.625598in}{5.365598in}}%
\pgfpathlineto{\pgfqpoint{0.611269in}{5.372066in}}%
\pgfpathlineto{\pgfqpoint{0.589177in}{5.389269in}}%
\pgfpathlineto{\pgfqpoint{0.595818in}{5.422986in}}%
\pgfpathlineto{\pgfqpoint{0.593490in}{5.440205in}}%
\pgfpathlineto{\pgfqpoint{0.575981in}{5.458485in}}%
\pgfpathlineto{\pgfqpoint{0.564765in}{5.457351in}}%
\pgfpathlineto{\pgfqpoint{0.556720in}{5.475841in}}%
\pgfpathlineto{\pgfqpoint{0.537662in}{5.483315in}}%
\pgfpathlineto{\pgfqpoint{0.523809in}{5.482363in}}%
\pgfpathlineto{\pgfqpoint{0.519327in}{5.501353in}}%
\pgfpathlineto{\pgfqpoint{0.533128in}{5.499332in}}%
\pgfpathlineto{\pgfqpoint{0.532031in}{5.525726in}}%
\pgfpathlineto{\pgfqpoint{0.526017in}{5.541398in}}%
\pgfpathlineto{\pgfqpoint{0.528668in}{5.561458in}}%
\pgfpathlineto{\pgfqpoint{0.535889in}{5.576545in}}%
\pgfpathlineto{\pgfqpoint{0.535482in}{5.605175in}}%
\pgfpathlineto{\pgfqpoint{0.531627in}{5.615562in}}%
\pgfpathlineto{\pgfqpoint{0.538284in}{5.648888in}}%
\pgfpathlineto{\pgfqpoint{0.536362in}{5.670367in}}%
\pgfpathlineto{\pgfqpoint{0.529707in}{5.680665in}}%
\pgfpathlineto{\pgfqpoint{0.530722in}{5.714337in}}%
\pgfpathlineto{\pgfqpoint{0.540403in}{5.739099in}}%
\pgfpathlineto{\pgfqpoint{0.550932in}{5.733039in}}%
\pgfpathlineto{\pgfqpoint{0.585796in}{5.697025in}}%
\pgfpathlineto{\pgfqpoint{0.628042in}{5.677317in}}%
\pgfpathlineto{\pgfqpoint{0.649673in}{5.674985in}}%
\pgfpathlineto{\pgfqpoint{0.670052in}{5.666594in}}%
\pgfpathlineto{\pgfqpoint{0.675837in}{5.638359in}}%
\pgfpathlineto{\pgfqpoint{0.657977in}{5.633010in}}%
\pgfpathlineto{\pgfqpoint{0.646142in}{5.610758in}}%
\pgfpathlineto{\pgfqpoint{0.660038in}{5.611994in}}%
\pgfpathlineto{\pgfqpoint{0.665641in}{5.621983in}}%
\pgfpathlineto{\pgfqpoint{0.685238in}{5.634473in}}%
\pgfpathlineto{\pgfqpoint{0.684226in}{5.615924in}}%
\pgfpathlineto{\pgfqpoint{0.671141in}{5.612934in}}%
\pgfpathlineto{\pgfqpoint{0.673314in}{5.589093in}}%
\pgfpathlineto{\pgfqpoint{0.657380in}{5.565210in}}%
\pgfpathlineto{\pgfqpoint{0.647182in}{5.576567in}}%
\pgfpathlineto{\pgfqpoint{0.616877in}{5.570392in}}%
\pgfpathlineto{\pgfqpoint{0.615283in}{5.556499in}}%
\pgfpathlineto{\pgfqpoint{0.625730in}{5.548035in}}%
\pgfpathlineto{\pgfqpoint{0.641668in}{5.547203in}}%
\pgfpathlineto{\pgfqpoint{0.663706in}{5.565290in}}%
\pgfpathlineto{\pgfqpoint{0.681148in}{5.566792in}}%
\pgfpathlineto{\pgfqpoint{0.681817in}{5.586812in}}%
\pgfpathlineto{\pgfqpoint{0.690807in}{5.616218in}}%
\pgfpathlineto{\pgfqpoint{0.712110in}{5.638088in}}%
\pgfpathlineto{\pgfqpoint{0.705007in}{5.655027in}}%
\pgfpathlineto{\pgfqpoint{0.709859in}{5.673264in}}%
\pgfpathlineto{\pgfqpoint{0.700407in}{5.691439in}}%
\pgfpathlineto{\pgfqpoint{0.713877in}{5.714682in}}%
\pgfpathlineto{\pgfqpoint{0.715895in}{5.728665in}}%
\pgfpathlineto{\pgfqpoint{0.704200in}{5.737856in}}%
\pgfpathlineto{\pgfqpoint{0.706147in}{5.761364in}}%
\pgfpathlineto{\pgfqpoint{0.846264in}{5.718794in}}%
\pgfpathlineto{\pgfqpoint{0.995055in}{5.677070in}}%
\pgfpathlineto{\pgfqpoint{1.129685in}{5.642352in}}%
\pgfusepath{stroke}%
\end{pgfscope}%
\begin{pgfscope}%
\pgfpathrectangle{\pgfqpoint{0.100000in}{2.413063in}}{\pgfqpoint{5.037500in}{3.427208in}}%
\pgfusepath{clip}%
\pgfsetbuttcap%
\pgfsetroundjoin%
\pgfsetlinewidth{0.501875pt}%
\definecolor{currentstroke}{rgb}{0.827451,0.827451,0.827451}%
\pgfsetstrokecolor{currentstroke}%
\pgfsetdash{}{0pt}%
\pgfpathmoveto{\pgfqpoint{0.683098in}{5.680037in}}%
\pgfpathlineto{\pgfqpoint{0.689997in}{5.653438in}}%
\pgfpathlineto{\pgfqpoint{0.700816in}{5.644603in}}%
\pgfpathlineto{\pgfqpoint{0.692183in}{5.633355in}}%
\pgfpathlineto{\pgfqpoint{0.683769in}{5.649836in}}%
\pgfpathlineto{\pgfqpoint{0.683098in}{5.680037in}}%
\pgfusepath{stroke}%
\end{pgfscope}%
\begin{pgfscope}%
\pgfpathrectangle{\pgfqpoint{0.100000in}{2.413063in}}{\pgfqpoint{5.037500in}{3.427208in}}%
\pgfusepath{clip}%
\pgfsetbuttcap%
\pgfsetroundjoin%
\pgfsetlinewidth{0.501875pt}%
\definecolor{currentstroke}{rgb}{0.827451,0.827451,0.827451}%
\pgfsetstrokecolor{currentstroke}%
\pgfsetdash{}{0pt}%
\pgfpathmoveto{\pgfqpoint{1.201939in}{5.624893in}}%
\pgfpathlineto{\pgfqpoint{1.351716in}{5.591281in}}%
\pgfpathlineto{\pgfqpoint{1.492772in}{5.562770in}}%
\pgfpathlineto{\pgfqpoint{1.601299in}{5.542885in}}%
\pgfpathlineto{\pgfqpoint{1.695915in}{5.526989in}}%
\pgfpathlineto{\pgfqpoint{1.790740in}{5.512398in}}%
\pgfpathlineto{\pgfqpoint{1.871493in}{5.501022in}}%
\pgfpathlineto{\pgfqpoint{1.952373in}{5.490591in}}%
\pgfpathlineto{\pgfqpoint{2.033369in}{5.481105in}}%
\pgfpathlineto{\pgfqpoint{2.109697in}{5.473044in}}%
\pgfpathlineto{\pgfqpoint{2.099200in}{5.356717in}}%
\pgfpathlineto{\pgfqpoint{2.083575in}{5.199518in}}%
\pgfpathlineto{\pgfqpoint{2.075380in}{5.118668in}}%
\pgfpathlineto{\pgfqpoint{2.063595in}{5.009707in}}%
\pgfpathlineto{\pgfqpoint{1.980159in}{5.018817in}}%
\pgfpathlineto{\pgfqpoint{1.884635in}{5.029661in}}%
\pgfpathlineto{\pgfqpoint{1.751998in}{5.047814in}}%
\pgfpathlineto{\pgfqpoint{1.692759in}{5.056264in}}%
\pgfpathlineto{\pgfqpoint{1.546810in}{5.079174in}}%
\pgfpathlineto{\pgfqpoint{1.496569in}{5.088365in}}%
\pgfpathlineto{\pgfqpoint{1.486097in}{5.028827in}}%
\pgfpathlineto{\pgfqpoint{1.480372in}{5.033075in}}%
\pgfpathlineto{\pgfqpoint{1.469613in}{5.061717in}}%
\pgfpathlineto{\pgfqpoint{1.455784in}{5.059953in}}%
\pgfpathlineto{\pgfqpoint{1.445885in}{5.045474in}}%
\pgfpathlineto{\pgfqpoint{1.425343in}{5.043537in}}%
\pgfpathlineto{\pgfqpoint{1.421216in}{5.050465in}}%
\pgfpathlineto{\pgfqpoint{1.401974in}{5.049626in}}%
\pgfpathlineto{\pgfqpoint{1.392301in}{5.056334in}}%
\pgfpathlineto{\pgfqpoint{1.378775in}{5.045956in}}%
\pgfpathlineto{\pgfqpoint{1.345921in}{5.055295in}}%
\pgfpathlineto{\pgfqpoint{1.331583in}{5.050280in}}%
\pgfpathlineto{\pgfqpoint{1.323870in}{5.072534in}}%
\pgfpathlineto{\pgfqpoint{1.323509in}{5.093709in}}%
\pgfpathlineto{\pgfqpoint{1.300862in}{5.108938in}}%
\pgfpathlineto{\pgfqpoint{1.304555in}{5.130885in}}%
\pgfpathlineto{\pgfqpoint{1.288430in}{5.167138in}}%
\pgfpathlineto{\pgfqpoint{1.289516in}{5.197955in}}%
\pgfpathlineto{\pgfqpoint{1.275657in}{5.213038in}}%
\pgfpathlineto{\pgfqpoint{1.262653in}{5.199694in}}%
\pgfpathlineto{\pgfqpoint{1.244953in}{5.192477in}}%
\pgfpathlineto{\pgfqpoint{1.231480in}{5.207352in}}%
\pgfpathlineto{\pgfqpoint{1.233671in}{5.231107in}}%
\pgfpathlineto{\pgfqpoint{1.249829in}{5.240466in}}%
\pgfpathlineto{\pgfqpoint{1.245534in}{5.255516in}}%
\pgfpathlineto{\pgfqpoint{1.273621in}{5.328361in}}%
\pgfpathlineto{\pgfqpoint{1.251474in}{5.330187in}}%
\pgfpathlineto{\pgfqpoint{1.249197in}{5.343259in}}%
\pgfpathlineto{\pgfqpoint{1.233064in}{5.354532in}}%
\pgfpathlineto{\pgfqpoint{1.234095in}{5.367134in}}%
\pgfpathlineto{\pgfqpoint{1.225585in}{5.376924in}}%
\pgfpathlineto{\pgfqpoint{1.211732in}{5.412469in}}%
\pgfpathlineto{\pgfqpoint{1.199072in}{5.419718in}}%
\pgfpathlineto{\pgfqpoint{1.190007in}{5.450394in}}%
\pgfpathlineto{\pgfqpoint{1.192146in}{5.470785in}}%
\pgfpathlineto{\pgfqpoint{1.175235in}{5.508340in}}%
\pgfpathlineto{\pgfqpoint{1.201939in}{5.624893in}}%
\pgfusepath{stroke}%
\end{pgfscope}%
\begin{pgfscope}%
\pgfpathrectangle{\pgfqpoint{0.100000in}{2.413063in}}{\pgfqpoint{5.037500in}{3.427208in}}%
\pgfusepath{clip}%
\pgfsetbuttcap%
\pgfsetroundjoin%
\pgfsetlinewidth{0.501875pt}%
\definecolor{currentstroke}{rgb}{0.827451,0.827451,0.827451}%
\pgfsetstrokecolor{currentstroke}%
\pgfsetdash{}{0pt}%
\pgfpathmoveto{\pgfqpoint{4.817670in}{5.025444in}}%
\pgfpathlineto{\pgfqpoint{4.814672in}{5.038191in}}%
\pgfpathlineto{\pgfqpoint{4.798011in}{5.049357in}}%
\pgfpathlineto{\pgfqpoint{4.753984in}{5.193372in}}%
\pgfpathlineto{\pgfqpoint{4.730253in}{5.262870in}}%
\pgfpathlineto{\pgfqpoint{4.749331in}{5.282892in}}%
\pgfpathlineto{\pgfqpoint{4.767877in}{5.331893in}}%
\pgfpathlineto{\pgfqpoint{4.777126in}{5.347553in}}%
\pgfpathlineto{\pgfqpoint{4.770574in}{5.354129in}}%
\pgfpathlineto{\pgfqpoint{4.768379in}{5.397626in}}%
\pgfpathlineto{\pgfqpoint{4.776654in}{5.410960in}}%
\pgfpathlineto{\pgfqpoint{4.773027in}{5.441953in}}%
\pgfpathlineto{\pgfqpoint{4.806052in}{5.543639in}}%
\pgfpathlineto{\pgfqpoint{4.820933in}{5.544194in}}%
\pgfpathlineto{\pgfqpoint{4.827110in}{5.526031in}}%
\pgfpathlineto{\pgfqpoint{4.840334in}{5.520815in}}%
\pgfpathlineto{\pgfqpoint{4.865315in}{5.542180in}}%
\pgfpathlineto{\pgfqpoint{4.884903in}{5.554799in}}%
\pgfpathlineto{\pgfqpoint{4.928062in}{5.532577in}}%
\pgfpathlineto{\pgfqpoint{4.967053in}{5.409195in}}%
\pgfpathlineto{\pgfqpoint{4.974465in}{5.378840in}}%
\pgfpathlineto{\pgfqpoint{4.991544in}{5.375280in}}%
\pgfpathlineto{\pgfqpoint{5.014707in}{5.354663in}}%
\pgfpathlineto{\pgfqpoint{5.013396in}{5.342651in}}%
\pgfpathlineto{\pgfqpoint{5.029175in}{5.328413in}}%
\pgfpathlineto{\pgfqpoint{5.041992in}{5.336582in}}%
\pgfpathlineto{\pgfqpoint{5.068814in}{5.305301in}}%
\pgfpathlineto{\pgfqpoint{5.056826in}{5.280267in}}%
\pgfpathlineto{\pgfqpoint{5.040752in}{5.279772in}}%
\pgfpathlineto{\pgfqpoint{5.027855in}{5.257424in}}%
\pgfpathlineto{\pgfqpoint{5.012232in}{5.254255in}}%
\pgfpathlineto{\pgfqpoint{5.000763in}{5.242355in}}%
\pgfpathlineto{\pgfqpoint{4.966507in}{5.229594in}}%
\pgfpathlineto{\pgfqpoint{4.945773in}{5.209029in}}%
\pgfpathlineto{\pgfqpoint{4.935241in}{5.223790in}}%
\pgfpathlineto{\pgfqpoint{4.925671in}{5.213211in}}%
\pgfpathlineto{\pgfqpoint{4.928292in}{5.170441in}}%
\pgfpathlineto{\pgfqpoint{4.920714in}{5.153500in}}%
\pgfpathlineto{\pgfqpoint{4.904185in}{5.158127in}}%
\pgfpathlineto{\pgfqpoint{4.901518in}{5.140759in}}%
\pgfpathlineto{\pgfqpoint{4.894375in}{5.133617in}}%
\pgfpathlineto{\pgfqpoint{4.880081in}{5.135357in}}%
\pgfpathlineto{\pgfqpoint{4.880982in}{5.119118in}}%
\pgfpathlineto{\pgfqpoint{4.859316in}{5.123730in}}%
\pgfpathlineto{\pgfqpoint{4.847484in}{5.101231in}}%
\pgfpathlineto{\pgfqpoint{4.851952in}{5.089459in}}%
\pgfpathlineto{\pgfqpoint{4.844914in}{5.069879in}}%
\pgfpathlineto{\pgfqpoint{4.833835in}{5.055478in}}%
\pgfpathlineto{\pgfqpoint{4.831046in}{5.025414in}}%
\pgfpathlineto{\pgfqpoint{4.817670in}{5.025444in}}%
\pgfusepath{stroke}%
\end{pgfscope}%
\begin{pgfscope}%
\pgfpathrectangle{\pgfqpoint{0.100000in}{2.413063in}}{\pgfqpoint{5.037500in}{3.427208in}}%
\pgfusepath{clip}%
\pgfsetbuttcap%
\pgfsetroundjoin%
\pgfsetlinewidth{0.501875pt}%
\definecolor{currentstroke}{rgb}{0.827451,0.827451,0.827451}%
\pgfsetstrokecolor{currentstroke}%
\pgfsetdash{}{0pt}%
\pgfpathmoveto{\pgfqpoint{4.972660in}{5.220812in}}%
\pgfpathlineto{\pgfqpoint{4.982442in}{5.231057in}}%
\pgfpathlineto{\pgfqpoint{4.991754in}{5.221427in}}%
\pgfpathlineto{\pgfqpoint{4.975027in}{5.208668in}}%
\pgfpathlineto{\pgfqpoint{4.972660in}{5.220812in}}%
\pgfusepath{stroke}%
\end{pgfscope}%
\begin{pgfscope}%
\pgfpathrectangle{\pgfqpoint{0.100000in}{2.413063in}}{\pgfqpoint{5.037500in}{3.427208in}}%
\pgfusepath{clip}%
\pgfsetbuttcap%
\pgfsetroundjoin%
\pgfsetlinewidth{0.501875pt}%
\definecolor{currentstroke}{rgb}{0.827451,0.827451,0.827451}%
\pgfsetstrokecolor{currentstroke}%
\pgfsetdash{}{0pt}%
\pgfpathmoveto{\pgfqpoint{2.075380in}{5.118668in}}%
\pgfpathlineto{\pgfqpoint{2.083575in}{5.199518in}}%
\pgfpathlineto{\pgfqpoint{2.099200in}{5.356717in}}%
\pgfpathlineto{\pgfqpoint{2.109697in}{5.473044in}}%
\pgfpathlineto{\pgfqpoint{2.195665in}{5.464979in}}%
\pgfpathlineto{\pgfqpoint{2.305646in}{5.456223in}}%
\pgfpathlineto{\pgfqpoint{2.406173in}{5.449747in}}%
\pgfpathlineto{\pgfqpoint{2.497197in}{5.445142in}}%
\pgfpathlineto{\pgfqpoint{2.633036in}{5.440482in}}%
\pgfpathlineto{\pgfqpoint{2.642307in}{5.402197in}}%
\pgfpathlineto{\pgfqpoint{2.638988in}{5.383901in}}%
\pgfpathlineto{\pgfqpoint{2.637896in}{5.346294in}}%
\pgfpathlineto{\pgfqpoint{2.644203in}{5.318130in}}%
\pgfpathlineto{\pgfqpoint{2.658653in}{5.276579in}}%
\pgfpathlineto{\pgfqpoint{2.658619in}{5.224267in}}%
\pgfpathlineto{\pgfqpoint{2.661293in}{5.163497in}}%
\pgfpathlineto{\pgfqpoint{2.664964in}{5.147119in}}%
\pgfpathlineto{\pgfqpoint{2.675704in}{5.129137in}}%
\pgfpathlineto{\pgfqpoint{2.679246in}{5.101121in}}%
\pgfpathlineto{\pgfqpoint{2.677717in}{5.082416in}}%
\pgfpathlineto{\pgfqpoint{2.563838in}{5.084883in}}%
\pgfpathlineto{\pgfqpoint{2.480994in}{5.089082in}}%
\pgfpathlineto{\pgfqpoint{2.359549in}{5.095547in}}%
\pgfpathlineto{\pgfqpoint{2.239785in}{5.104054in}}%
\pgfpathlineto{\pgfqpoint{2.160017in}{5.110489in}}%
\pgfpathlineto{\pgfqpoint{2.075380in}{5.118668in}}%
\pgfusepath{stroke}%
\end{pgfscope}%
\begin{pgfscope}%
\pgfpathrectangle{\pgfqpoint{0.100000in}{2.413063in}}{\pgfqpoint{5.037500in}{3.427208in}}%
\pgfusepath{clip}%
\pgfsetbuttcap%
\pgfsetroundjoin%
\pgfsetlinewidth{0.501875pt}%
\definecolor{currentstroke}{rgb}{0.827451,0.827451,0.827451}%
\pgfsetstrokecolor{currentstroke}%
\pgfsetdash{}{0pt}%
\pgfpathmoveto{\pgfqpoint{2.040958in}{4.780305in}}%
\pgfpathlineto{\pgfqpoint{2.054105in}{4.915834in}}%
\pgfpathlineto{\pgfqpoint{2.063595in}{5.009707in}}%
\pgfpathlineto{\pgfqpoint{2.075380in}{5.118668in}}%
\pgfpathlineto{\pgfqpoint{2.160017in}{5.110489in}}%
\pgfpathlineto{\pgfqpoint{2.239785in}{5.104054in}}%
\pgfpathlineto{\pgfqpoint{2.359549in}{5.095547in}}%
\pgfpathlineto{\pgfqpoint{2.480994in}{5.089082in}}%
\pgfpathlineto{\pgfqpoint{2.563838in}{5.084883in}}%
\pgfpathlineto{\pgfqpoint{2.677717in}{5.082416in}}%
\pgfpathlineto{\pgfqpoint{2.670006in}{5.059917in}}%
\pgfpathlineto{\pgfqpoint{2.654617in}{5.042254in}}%
\pgfpathlineto{\pgfqpoint{2.666402in}{5.021918in}}%
\pgfpathlineto{\pgfqpoint{2.679397in}{5.017573in}}%
\pgfpathlineto{\pgfqpoint{2.685563in}{5.005892in}}%
\pgfpathlineto{\pgfqpoint{2.684148in}{4.920604in}}%
\pgfpathlineto{\pgfqpoint{2.681778in}{4.800574in}}%
\pgfpathlineto{\pgfqpoint{2.673013in}{4.769048in}}%
\pgfpathlineto{\pgfqpoint{2.680853in}{4.751610in}}%
\pgfpathlineto{\pgfqpoint{2.665113in}{4.714137in}}%
\pgfpathlineto{\pgfqpoint{2.681695in}{4.683929in}}%
\pgfpathlineto{\pgfqpoint{2.667607in}{4.686241in}}%
\pgfpathlineto{\pgfqpoint{2.657995in}{4.705035in}}%
\pgfpathlineto{\pgfqpoint{2.623692in}{4.717918in}}%
\pgfpathlineto{\pgfqpoint{2.602058in}{4.729251in}}%
\pgfpathlineto{\pgfqpoint{2.565756in}{4.730193in}}%
\pgfpathlineto{\pgfqpoint{2.553155in}{4.719869in}}%
\pgfpathlineto{\pgfqpoint{2.512066in}{4.740198in}}%
\pgfpathlineto{\pgfqpoint{2.508914in}{4.746627in}}%
\pgfpathlineto{\pgfqpoint{2.365512in}{4.753414in}}%
\pgfpathlineto{\pgfqpoint{2.278390in}{4.758408in}}%
\pgfpathlineto{\pgfqpoint{2.206426in}{4.764043in}}%
\pgfpathlineto{\pgfqpoint{2.087517in}{4.775278in}}%
\pgfpathlineto{\pgfqpoint{2.040958in}{4.780305in}}%
\pgfusepath{stroke}%
\end{pgfscope}%
\begin{pgfscope}%
\pgfpathrectangle{\pgfqpoint{0.100000in}{2.413063in}}{\pgfqpoint{5.037500in}{3.427208in}}%
\pgfusepath{clip}%
\pgfsetbuttcap%
\pgfsetroundjoin%
\pgfsetlinewidth{0.501875pt}%
\definecolor{currentstroke}{rgb}{0.827451,0.827451,0.827451}%
\pgfsetstrokecolor{currentstroke}%
\pgfsetdash{}{0pt}%
\pgfpathmoveto{\pgfqpoint{2.018405in}{4.550818in}}%
\pgfpathlineto{\pgfqpoint{1.941955in}{4.557799in}}%
\pgfpathlineto{\pgfqpoint{1.775315in}{4.578367in}}%
\pgfpathlineto{\pgfqpoint{1.684663in}{4.591377in}}%
\pgfpathlineto{\pgfqpoint{1.587436in}{4.605334in}}%
\pgfpathlineto{\pgfqpoint{1.505537in}{4.618440in}}%
\pgfpathlineto{\pgfqpoint{1.415642in}{4.633831in}}%
\pgfpathlineto{\pgfqpoint{1.436081in}{4.747212in}}%
\pgfpathlineto{\pgfqpoint{1.456784in}{4.863472in}}%
\pgfpathlineto{\pgfqpoint{1.486097in}{5.028827in}}%
\pgfpathlineto{\pgfqpoint{1.496569in}{5.088365in}}%
\pgfpathlineto{\pgfqpoint{1.546810in}{5.079174in}}%
\pgfpathlineto{\pgfqpoint{1.692759in}{5.056264in}}%
\pgfpathlineto{\pgfqpoint{1.751998in}{5.047814in}}%
\pgfpathlineto{\pgfqpoint{1.884635in}{5.029661in}}%
\pgfpathlineto{\pgfqpoint{1.980159in}{5.018817in}}%
\pgfpathlineto{\pgfqpoint{2.063595in}{5.009707in}}%
\pgfpathlineto{\pgfqpoint{2.054105in}{4.915834in}}%
\pgfpathlineto{\pgfqpoint{2.040958in}{4.780305in}}%
\pgfpathlineto{\pgfqpoint{2.029673in}{4.665125in}}%
\pgfpathlineto{\pgfqpoint{2.018405in}{4.550818in}}%
\pgfusepath{stroke}%
\end{pgfscope}%
\begin{pgfscope}%
\pgfpathrectangle{\pgfqpoint{0.100000in}{2.413063in}}{\pgfqpoint{5.037500in}{3.427208in}}%
\pgfusepath{clip}%
\pgfsetbuttcap%
\pgfsetroundjoin%
\pgfsetlinewidth{0.501875pt}%
\definecolor{currentstroke}{rgb}{0.827451,0.827451,0.827451}%
\pgfsetstrokecolor{currentstroke}%
\pgfsetdash{}{0pt}%
\pgfpathmoveto{\pgfqpoint{3.416161in}{4.707215in}}%
\pgfpathlineto{\pgfqpoint{3.319282in}{4.700333in}}%
\pgfpathlineto{\pgfqpoint{3.174878in}{4.694179in}}%
\pgfpathlineto{\pgfqpoint{3.169391in}{4.708766in}}%
\pgfpathlineto{\pgfqpoint{3.137369in}{4.719670in}}%
\pgfpathlineto{\pgfqpoint{3.130306in}{4.740271in}}%
\pgfpathlineto{\pgfqpoint{3.127348in}{4.765756in}}%
\pgfpathlineto{\pgfqpoint{3.134564in}{4.778814in}}%
\pgfpathlineto{\pgfqpoint{3.123167in}{4.791344in}}%
\pgfpathlineto{\pgfqpoint{3.120418in}{4.806289in}}%
\pgfpathlineto{\pgfqpoint{3.116744in}{4.839351in}}%
\pgfpathlineto{\pgfqpoint{3.105822in}{4.857310in}}%
\pgfpathlineto{\pgfqpoint{3.086424in}{4.867390in}}%
\pgfpathlineto{\pgfqpoint{3.065315in}{4.884023in}}%
\pgfpathlineto{\pgfqpoint{3.054403in}{4.903703in}}%
\pgfpathlineto{\pgfqpoint{3.034824in}{4.911628in}}%
\pgfpathlineto{\pgfqpoint{3.023314in}{4.924526in}}%
\pgfpathlineto{\pgfqpoint{3.009371in}{4.926715in}}%
\pgfpathlineto{\pgfqpoint{2.984477in}{4.945866in}}%
\pgfpathlineto{\pgfqpoint{2.988521in}{4.967905in}}%
\pgfpathlineto{\pgfqpoint{2.987741in}{5.009815in}}%
\pgfpathlineto{\pgfqpoint{2.995419in}{5.021352in}}%
\pgfpathlineto{\pgfqpoint{2.988521in}{5.038771in}}%
\pgfpathlineto{\pgfqpoint{2.976370in}{5.042141in}}%
\pgfpathlineto{\pgfqpoint{2.977374in}{5.057440in}}%
\pgfpathlineto{\pgfqpoint{2.992450in}{5.081609in}}%
\pgfpathlineto{\pgfqpoint{3.022305in}{5.100721in}}%
\pgfpathlineto{\pgfqpoint{3.020445in}{5.168690in}}%
\pgfpathlineto{\pgfqpoint{3.035383in}{5.178900in}}%
\pgfpathlineto{\pgfqpoint{3.049520in}{5.172098in}}%
\pgfpathlineto{\pgfqpoint{3.078318in}{5.182055in}}%
\pgfpathlineto{\pgfqpoint{3.132500in}{5.207078in}}%
\pgfpathlineto{\pgfqpoint{3.139558in}{5.199330in}}%
\pgfpathlineto{\pgfqpoint{3.129296in}{5.164236in}}%
\pgfpathlineto{\pgfqpoint{3.144563in}{5.171938in}}%
\pgfpathlineto{\pgfqpoint{3.170712in}{5.164247in}}%
\pgfpathlineto{\pgfqpoint{3.186780in}{5.157828in}}%
\pgfpathlineto{\pgfqpoint{3.195789in}{5.139018in}}%
\pgfpathlineto{\pgfqpoint{3.278223in}{5.121266in}}%
\pgfpathlineto{\pgfqpoint{3.302863in}{5.109087in}}%
\pgfpathlineto{\pgfqpoint{3.327916in}{5.109231in}}%
\pgfpathlineto{\pgfqpoint{3.353658in}{5.104206in}}%
\pgfpathlineto{\pgfqpoint{3.370364in}{5.087046in}}%
\pgfpathlineto{\pgfqpoint{3.386328in}{5.078566in}}%
\pgfpathlineto{\pgfqpoint{3.389261in}{5.054109in}}%
\pgfpathlineto{\pgfqpoint{3.384545in}{5.038752in}}%
\pgfpathlineto{\pgfqpoint{3.402337in}{5.037617in}}%
\pgfpathlineto{\pgfqpoint{3.396343in}{5.019806in}}%
\pgfpathlineto{\pgfqpoint{3.402044in}{5.013479in}}%
\pgfpathlineto{\pgfqpoint{3.407714in}{4.996666in}}%
\pgfpathlineto{\pgfqpoint{3.390375in}{4.987764in}}%
\pgfpathlineto{\pgfqpoint{3.380307in}{4.962963in}}%
\pgfpathlineto{\pgfqpoint{3.377148in}{4.945425in}}%
\pgfpathlineto{\pgfqpoint{3.386807in}{4.942424in}}%
\pgfpathlineto{\pgfqpoint{3.399173in}{4.955574in}}%
\pgfpathlineto{\pgfqpoint{3.409687in}{4.978365in}}%
\pgfpathlineto{\pgfqpoint{3.423912in}{4.986280in}}%
\pgfpathlineto{\pgfqpoint{3.434602in}{4.975984in}}%
\pgfpathlineto{\pgfqpoint{3.424080in}{4.944821in}}%
\pgfpathlineto{\pgfqpoint{3.420778in}{4.920629in}}%
\pgfpathlineto{\pgfqpoint{3.423887in}{4.903239in}}%
\pgfpathlineto{\pgfqpoint{3.414116in}{4.893401in}}%
\pgfpathlineto{\pgfqpoint{3.409236in}{4.869361in}}%
\pgfpathlineto{\pgfqpoint{3.413223in}{4.844095in}}%
\pgfpathlineto{\pgfqpoint{3.401692in}{4.806729in}}%
\pgfpathlineto{\pgfqpoint{3.401934in}{4.788058in}}%
\pgfpathlineto{\pgfqpoint{3.411046in}{4.747597in}}%
\pgfpathlineto{\pgfqpoint{3.416952in}{4.740656in}}%
\pgfpathlineto{\pgfqpoint{3.416161in}{4.707215in}}%
\pgfusepath{stroke}%
\end{pgfscope}%
\begin{pgfscope}%
\pgfpathrectangle{\pgfqpoint{0.100000in}{2.413063in}}{\pgfqpoint{5.037500in}{3.427208in}}%
\pgfusepath{clip}%
\pgfsetbuttcap%
\pgfsetroundjoin%
\pgfsetlinewidth{0.501875pt}%
\definecolor{currentstroke}{rgb}{0.827451,0.827451,0.827451}%
\pgfsetstrokecolor{currentstroke}%
\pgfsetdash{}{0pt}%
\pgfpathmoveto{\pgfqpoint{3.452484in}{5.035138in}}%
\pgfpathlineto{\pgfqpoint{3.454646in}{5.018999in}}%
\pgfpathlineto{\pgfqpoint{3.434836in}{4.976475in}}%
\pgfpathlineto{\pgfqpoint{3.426032in}{4.988793in}}%
\pgfpathlineto{\pgfqpoint{3.452484in}{5.035138in}}%
\pgfusepath{stroke}%
\end{pgfscope}%
\begin{pgfscope}%
\pgfpathrectangle{\pgfqpoint{0.100000in}{2.413063in}}{\pgfqpoint{5.037500in}{3.427208in}}%
\pgfusepath{clip}%
\pgfsetbuttcap%
\pgfsetroundjoin%
\pgfsetlinewidth{0.501875pt}%
\definecolor{currentstroke}{rgb}{0.827451,0.827451,0.827451}%
\pgfsetstrokecolor{currentstroke}%
\pgfsetdash{}{0pt}%
\pgfpathmoveto{\pgfqpoint{1.054504in}{5.300440in}}%
\pgfpathlineto{\pgfqpoint{1.058421in}{5.319829in}}%
\pgfpathlineto{\pgfqpoint{1.055466in}{5.340878in}}%
\pgfpathlineto{\pgfqpoint{1.059991in}{5.363826in}}%
\pgfpathlineto{\pgfqpoint{1.083039in}{5.456934in}}%
\pgfpathlineto{\pgfqpoint{1.102423in}{5.534332in}}%
\pgfpathlineto{\pgfqpoint{1.129685in}{5.642352in}}%
\pgfpathlineto{\pgfqpoint{1.201939in}{5.624893in}}%
\pgfpathlineto{\pgfqpoint{1.175235in}{5.508340in}}%
\pgfpathlineto{\pgfqpoint{1.192146in}{5.470785in}}%
\pgfpathlineto{\pgfqpoint{1.190007in}{5.450394in}}%
\pgfpathlineto{\pgfqpoint{1.199072in}{5.419718in}}%
\pgfpathlineto{\pgfqpoint{1.211732in}{5.412469in}}%
\pgfpathlineto{\pgfqpoint{1.225585in}{5.376924in}}%
\pgfpathlineto{\pgfqpoint{1.234095in}{5.367134in}}%
\pgfpathlineto{\pgfqpoint{1.233064in}{5.354532in}}%
\pgfpathlineto{\pgfqpoint{1.249197in}{5.343259in}}%
\pgfpathlineto{\pgfqpoint{1.251474in}{5.330187in}}%
\pgfpathlineto{\pgfqpoint{1.273621in}{5.328361in}}%
\pgfpathlineto{\pgfqpoint{1.245534in}{5.255516in}}%
\pgfpathlineto{\pgfqpoint{1.249829in}{5.240466in}}%
\pgfpathlineto{\pgfqpoint{1.233671in}{5.231107in}}%
\pgfpathlineto{\pgfqpoint{1.231480in}{5.207352in}}%
\pgfpathlineto{\pgfqpoint{1.244953in}{5.192477in}}%
\pgfpathlineto{\pgfqpoint{1.262653in}{5.199694in}}%
\pgfpathlineto{\pgfqpoint{1.275657in}{5.213038in}}%
\pgfpathlineto{\pgfqpoint{1.289516in}{5.197955in}}%
\pgfpathlineto{\pgfqpoint{1.288430in}{5.167138in}}%
\pgfpathlineto{\pgfqpoint{1.304555in}{5.130885in}}%
\pgfpathlineto{\pgfqpoint{1.300862in}{5.108938in}}%
\pgfpathlineto{\pgfqpoint{1.323509in}{5.093709in}}%
\pgfpathlineto{\pgfqpoint{1.323870in}{5.072534in}}%
\pgfpathlineto{\pgfqpoint{1.331583in}{5.050280in}}%
\pgfpathlineto{\pgfqpoint{1.345921in}{5.055295in}}%
\pgfpathlineto{\pgfqpoint{1.378775in}{5.045956in}}%
\pgfpathlineto{\pgfqpoint{1.392301in}{5.056334in}}%
\pgfpathlineto{\pgfqpoint{1.401974in}{5.049626in}}%
\pgfpathlineto{\pgfqpoint{1.421216in}{5.050465in}}%
\pgfpathlineto{\pgfqpoint{1.425343in}{5.043537in}}%
\pgfpathlineto{\pgfqpoint{1.445885in}{5.045474in}}%
\pgfpathlineto{\pgfqpoint{1.455784in}{5.059953in}}%
\pgfpathlineto{\pgfqpoint{1.469613in}{5.061717in}}%
\pgfpathlineto{\pgfqpoint{1.480372in}{5.033075in}}%
\pgfpathlineto{\pgfqpoint{1.486097in}{5.028827in}}%
\pgfpathlineto{\pgfqpoint{1.456784in}{4.863472in}}%
\pgfpathlineto{\pgfqpoint{1.436081in}{4.747212in}}%
\pgfpathlineto{\pgfqpoint{1.272793in}{4.778734in}}%
\pgfpathlineto{\pgfqpoint{1.184593in}{4.796279in}}%
\pgfpathlineto{\pgfqpoint{1.102093in}{4.814420in}}%
\pgfpathlineto{\pgfqpoint{1.024981in}{4.831878in}}%
\pgfpathlineto{\pgfqpoint{0.935738in}{4.853400in}}%
\pgfpathlineto{\pgfqpoint{0.981770in}{5.042238in}}%
\pgfpathlineto{\pgfqpoint{0.984252in}{5.056039in}}%
\pgfpathlineto{\pgfqpoint{1.004722in}{5.092200in}}%
\pgfpathlineto{\pgfqpoint{0.983526in}{5.113934in}}%
\pgfpathlineto{\pgfqpoint{0.987927in}{5.135277in}}%
\pgfpathlineto{\pgfqpoint{0.995996in}{5.140619in}}%
\pgfpathlineto{\pgfqpoint{1.010406in}{5.162611in}}%
\pgfpathlineto{\pgfqpoint{1.031412in}{5.177827in}}%
\pgfpathlineto{\pgfqpoint{1.032559in}{5.189084in}}%
\pgfpathlineto{\pgfqpoint{1.045237in}{5.200333in}}%
\pgfpathlineto{\pgfqpoint{1.055763in}{5.221387in}}%
\pgfpathlineto{\pgfqpoint{1.077356in}{5.243662in}}%
\pgfpathlineto{\pgfqpoint{1.075823in}{5.265662in}}%
\pgfpathlineto{\pgfqpoint{1.061060in}{5.277883in}}%
\pgfpathlineto{\pgfqpoint{1.054504in}{5.300440in}}%
\pgfusepath{stroke}%
\end{pgfscope}%
\begin{pgfscope}%
\pgfpathrectangle{\pgfqpoint{0.100000in}{2.413063in}}{\pgfqpoint{5.037500in}{3.427208in}}%
\pgfusepath{clip}%
\pgfsetbuttcap%
\pgfsetroundjoin%
\pgfsetlinewidth{0.501875pt}%
\definecolor{currentstroke}{rgb}{0.827451,0.827451,0.827451}%
\pgfsetstrokecolor{currentstroke}%
\pgfsetdash{}{0pt}%
\pgfpathmoveto{\pgfqpoint{4.628939in}{4.931925in}}%
\pgfpathlineto{\pgfqpoint{4.622154in}{4.953342in}}%
\pgfpathlineto{\pgfqpoint{4.609564in}{5.018312in}}%
\pgfpathlineto{\pgfqpoint{4.593285in}{5.038300in}}%
\pgfpathlineto{\pgfqpoint{4.579007in}{5.074245in}}%
\pgfpathlineto{\pgfqpoint{4.583695in}{5.100896in}}%
\pgfpathlineto{\pgfqpoint{4.579935in}{5.120538in}}%
\pgfpathlineto{\pgfqpoint{4.567736in}{5.139830in}}%
\pgfpathlineto{\pgfqpoint{4.567234in}{5.161082in}}%
\pgfpathlineto{\pgfqpoint{4.560150in}{5.184002in}}%
\pgfpathlineto{\pgfqpoint{4.623544in}{5.199604in}}%
\pgfpathlineto{\pgfqpoint{4.705886in}{5.221830in}}%
\pgfpathlineto{\pgfqpoint{4.709172in}{5.209085in}}%
\pgfpathlineto{\pgfqpoint{4.703929in}{5.188795in}}%
\pgfpathlineto{\pgfqpoint{4.716237in}{5.172560in}}%
\pgfpathlineto{\pgfqpoint{4.709700in}{5.151988in}}%
\pgfpathlineto{\pgfqpoint{4.683724in}{5.126129in}}%
\pgfpathlineto{\pgfqpoint{4.690864in}{5.106703in}}%
\pgfpathlineto{\pgfqpoint{4.685572in}{5.065644in}}%
\pgfpathlineto{\pgfqpoint{4.678164in}{5.042494in}}%
\pgfpathlineto{\pgfqpoint{4.687616in}{4.976662in}}%
\pgfpathlineto{\pgfqpoint{4.686342in}{4.953500in}}%
\pgfpathlineto{\pgfqpoint{4.695486in}{4.946066in}}%
\pgfpathlineto{\pgfqpoint{4.628939in}{4.931925in}}%
\pgfusepath{stroke}%
\end{pgfscope}%
\begin{pgfscope}%
\pgfpathrectangle{\pgfqpoint{0.100000in}{2.413063in}}{\pgfqpoint{5.037500in}{3.427208in}}%
\pgfusepath{clip}%
\pgfsetbuttcap%
\pgfsetroundjoin%
\pgfsetlinewidth{0.501875pt}%
\definecolor{currentstroke}{rgb}{0.827451,0.827451,0.827451}%
\pgfsetstrokecolor{currentstroke}%
\pgfsetdash{}{0pt}%
\pgfpathmoveto{\pgfqpoint{2.681778in}{4.800574in}}%
\pgfpathlineto{\pgfqpoint{2.684148in}{4.920604in}}%
\pgfpathlineto{\pgfqpoint{2.685563in}{5.005892in}}%
\pgfpathlineto{\pgfqpoint{2.679397in}{5.017573in}}%
\pgfpathlineto{\pgfqpoint{2.666402in}{5.021918in}}%
\pgfpathlineto{\pgfqpoint{2.654617in}{5.042254in}}%
\pgfpathlineto{\pgfqpoint{2.670006in}{5.059917in}}%
\pgfpathlineto{\pgfqpoint{2.677717in}{5.082416in}}%
\pgfpathlineto{\pgfqpoint{2.679246in}{5.101121in}}%
\pgfpathlineto{\pgfqpoint{2.675704in}{5.129137in}}%
\pgfpathlineto{\pgfqpoint{2.664964in}{5.147119in}}%
\pgfpathlineto{\pgfqpoint{2.661293in}{5.163497in}}%
\pgfpathlineto{\pgfqpoint{2.658619in}{5.224267in}}%
\pgfpathlineto{\pgfqpoint{2.658653in}{5.276579in}}%
\pgfpathlineto{\pgfqpoint{2.644203in}{5.318130in}}%
\pgfpathlineto{\pgfqpoint{2.637896in}{5.346294in}}%
\pgfpathlineto{\pgfqpoint{2.638988in}{5.383901in}}%
\pgfpathlineto{\pgfqpoint{2.642307in}{5.402197in}}%
\pgfpathlineto{\pgfqpoint{2.633036in}{5.440482in}}%
\pgfpathlineto{\pgfqpoint{2.696152in}{5.439219in}}%
\pgfpathlineto{\pgfqpoint{2.792021in}{5.438395in}}%
\pgfpathlineto{\pgfqpoint{2.792545in}{5.481909in}}%
\pgfpathlineto{\pgfqpoint{2.816948in}{5.477118in}}%
\pgfpathlineto{\pgfqpoint{2.828643in}{5.424060in}}%
\pgfpathlineto{\pgfqpoint{2.837265in}{5.404983in}}%
\pgfpathlineto{\pgfqpoint{2.858707in}{5.404420in}}%
\pgfpathlineto{\pgfqpoint{2.863506in}{5.397946in}}%
\pgfpathlineto{\pgfqpoint{2.893414in}{5.395079in}}%
\pgfpathlineto{\pgfqpoint{2.898441in}{5.381925in}}%
\pgfpathlineto{\pgfqpoint{2.919053in}{5.384882in}}%
\pgfpathlineto{\pgfqpoint{2.935062in}{5.397185in}}%
\pgfpathlineto{\pgfqpoint{2.962670in}{5.396725in}}%
\pgfpathlineto{\pgfqpoint{2.979752in}{5.386831in}}%
\pgfpathlineto{\pgfqpoint{2.981716in}{5.377552in}}%
\pgfpathlineto{\pgfqpoint{2.997970in}{5.375609in}}%
\pgfpathlineto{\pgfqpoint{3.008578in}{5.350299in}}%
\pgfpathlineto{\pgfqpoint{3.015422in}{5.365861in}}%
\pgfpathlineto{\pgfqpoint{3.034094in}{5.366818in}}%
\pgfpathlineto{\pgfqpoint{3.038817in}{5.354698in}}%
\pgfpathlineto{\pgfqpoint{3.059826in}{5.349166in}}%
\pgfpathlineto{\pgfqpoint{3.071412in}{5.331709in}}%
\pgfpathlineto{\pgfqpoint{3.097108in}{5.337129in}}%
\pgfpathlineto{\pgfqpoint{3.125382in}{5.358552in}}%
\pgfpathlineto{\pgfqpoint{3.135691in}{5.339667in}}%
\pgfpathlineto{\pgfqpoint{3.182085in}{5.344836in}}%
\pgfpathlineto{\pgfqpoint{3.201917in}{5.331874in}}%
\pgfpathlineto{\pgfqpoint{3.213460in}{5.336502in}}%
\pgfpathlineto{\pgfqpoint{3.222716in}{5.329205in}}%
\pgfpathlineto{\pgfqpoint{3.195257in}{5.311893in}}%
\pgfpathlineto{\pgfqpoint{3.156062in}{5.296468in}}%
\pgfpathlineto{\pgfqpoint{3.117242in}{5.265636in}}%
\pgfpathlineto{\pgfqpoint{3.083587in}{5.225090in}}%
\pgfpathlineto{\pgfqpoint{3.058111in}{5.201119in}}%
\pgfpathlineto{\pgfqpoint{3.035795in}{5.184642in}}%
\pgfpathlineto{\pgfqpoint{3.020445in}{5.168690in}}%
\pgfpathlineto{\pgfqpoint{3.022305in}{5.100721in}}%
\pgfpathlineto{\pgfqpoint{2.992450in}{5.081609in}}%
\pgfpathlineto{\pgfqpoint{2.977374in}{5.057440in}}%
\pgfpathlineto{\pgfqpoint{2.976370in}{5.042141in}}%
\pgfpathlineto{\pgfqpoint{2.988521in}{5.038771in}}%
\pgfpathlineto{\pgfqpoint{2.995419in}{5.021352in}}%
\pgfpathlineto{\pgfqpoint{2.987741in}{5.009815in}}%
\pgfpathlineto{\pgfqpoint{2.988521in}{4.967905in}}%
\pgfpathlineto{\pgfqpoint{2.984477in}{4.945866in}}%
\pgfpathlineto{\pgfqpoint{3.009371in}{4.926715in}}%
\pgfpathlineto{\pgfqpoint{3.023314in}{4.924526in}}%
\pgfpathlineto{\pgfqpoint{3.034824in}{4.911628in}}%
\pgfpathlineto{\pgfqpoint{3.054403in}{4.903703in}}%
\pgfpathlineto{\pgfqpoint{3.065315in}{4.884023in}}%
\pgfpathlineto{\pgfqpoint{3.086424in}{4.867390in}}%
\pgfpathlineto{\pgfqpoint{3.105822in}{4.857310in}}%
\pgfpathlineto{\pgfqpoint{3.116744in}{4.839351in}}%
\pgfpathlineto{\pgfqpoint{3.120418in}{4.806289in}}%
\pgfpathlineto{\pgfqpoint{3.017472in}{4.802551in}}%
\pgfpathlineto{\pgfqpoint{2.929720in}{4.800716in}}%
\pgfpathlineto{\pgfqpoint{2.849769in}{4.799541in}}%
\pgfpathlineto{\pgfqpoint{2.765190in}{4.799670in}}%
\pgfpathlineto{\pgfqpoint{2.681778in}{4.800574in}}%
\pgfusepath{stroke}%
\end{pgfscope}%
\begin{pgfscope}%
\pgfpathrectangle{\pgfqpoint{0.100000in}{2.413063in}}{\pgfqpoint{5.037500in}{3.427208in}}%
\pgfusepath{clip}%
\pgfsetbuttcap%
\pgfsetroundjoin%
\pgfsetlinewidth{0.501875pt}%
\definecolor{currentstroke}{rgb}{0.827451,0.827451,0.827451}%
\pgfsetstrokecolor{currentstroke}%
\pgfsetdash{}{0pt}%
\pgfpathmoveto{\pgfqpoint{0.344393in}{5.024965in}}%
\pgfpathlineto{\pgfqpoint{0.336240in}{5.039967in}}%
\pgfpathlineto{\pgfqpoint{0.341446in}{5.078475in}}%
\pgfpathlineto{\pgfqpoint{0.351493in}{5.098599in}}%
\pgfpathlineto{\pgfqpoint{0.346470in}{5.125823in}}%
\pgfpathlineto{\pgfqpoint{0.356952in}{5.137270in}}%
\pgfpathlineto{\pgfqpoint{0.376075in}{5.168128in}}%
\pgfpathlineto{\pgfqpoint{0.392289in}{5.186755in}}%
\pgfpathlineto{\pgfqpoint{0.415990in}{5.227325in}}%
\pgfpathlineto{\pgfqpoint{0.434168in}{5.271519in}}%
\pgfpathlineto{\pgfqpoint{0.453465in}{5.312967in}}%
\pgfpathlineto{\pgfqpoint{0.457366in}{5.330279in}}%
\pgfpathlineto{\pgfqpoint{0.483987in}{5.379801in}}%
\pgfpathlineto{\pgfqpoint{0.489113in}{5.401551in}}%
\pgfpathlineto{\pgfqpoint{0.500447in}{5.424392in}}%
\pgfpathlineto{\pgfqpoint{0.504499in}{5.452254in}}%
\pgfpathlineto{\pgfqpoint{0.526137in}{5.465840in}}%
\pgfpathlineto{\pgfqpoint{0.551794in}{5.472687in}}%
\pgfpathlineto{\pgfqpoint{0.564765in}{5.457351in}}%
\pgfpathlineto{\pgfqpoint{0.575981in}{5.458485in}}%
\pgfpathlineto{\pgfqpoint{0.593490in}{5.440205in}}%
\pgfpathlineto{\pgfqpoint{0.595818in}{5.422986in}}%
\pgfpathlineto{\pgfqpoint{0.589177in}{5.389269in}}%
\pgfpathlineto{\pgfqpoint{0.611269in}{5.372066in}}%
\pgfpathlineto{\pgfqpoint{0.625598in}{5.365598in}}%
\pgfpathlineto{\pgfqpoint{0.664413in}{5.372200in}}%
\pgfpathlineto{\pgfqpoint{0.686901in}{5.367691in}}%
\pgfpathlineto{\pgfqpoint{0.702357in}{5.360292in}}%
\pgfpathlineto{\pgfqpoint{0.710499in}{5.346247in}}%
\pgfpathlineto{\pgfqpoint{0.762052in}{5.348120in}}%
\pgfpathlineto{\pgfqpoint{0.770250in}{5.339526in}}%
\pgfpathlineto{\pgfqpoint{0.789590in}{5.337671in}}%
\pgfpathlineto{\pgfqpoint{0.809066in}{5.343162in}}%
\pgfpathlineto{\pgfqpoint{0.866649in}{5.341778in}}%
\pgfpathlineto{\pgfqpoint{0.879086in}{5.337547in}}%
\pgfpathlineto{\pgfqpoint{0.893620in}{5.342423in}}%
\pgfpathlineto{\pgfqpoint{1.054504in}{5.300440in}}%
\pgfpathlineto{\pgfqpoint{1.061060in}{5.277883in}}%
\pgfpathlineto{\pgfqpoint{1.075823in}{5.265662in}}%
\pgfpathlineto{\pgfqpoint{1.077356in}{5.243662in}}%
\pgfpathlineto{\pgfqpoint{1.055763in}{5.221387in}}%
\pgfpathlineto{\pgfqpoint{1.045237in}{5.200333in}}%
\pgfpathlineto{\pgfqpoint{1.032559in}{5.189084in}}%
\pgfpathlineto{\pgfqpoint{1.031412in}{5.177827in}}%
\pgfpathlineto{\pgfqpoint{1.010406in}{5.162611in}}%
\pgfpathlineto{\pgfqpoint{0.995996in}{5.140619in}}%
\pgfpathlineto{\pgfqpoint{0.987927in}{5.135277in}}%
\pgfpathlineto{\pgfqpoint{0.983526in}{5.113934in}}%
\pgfpathlineto{\pgfqpoint{1.004722in}{5.092200in}}%
\pgfpathlineto{\pgfqpoint{0.984252in}{5.056039in}}%
\pgfpathlineto{\pgfqpoint{0.981770in}{5.042238in}}%
\pgfpathlineto{\pgfqpoint{0.935738in}{4.853400in}}%
\pgfpathlineto{\pgfqpoint{0.838915in}{4.878210in}}%
\pgfpathlineto{\pgfqpoint{0.745509in}{4.902295in}}%
\pgfpathlineto{\pgfqpoint{0.689161in}{4.917975in}}%
\pgfpathlineto{\pgfqpoint{0.616760in}{4.938599in}}%
\pgfpathlineto{\pgfqpoint{0.501273in}{4.974949in}}%
\pgfpathlineto{\pgfqpoint{0.375733in}{5.014034in}}%
\pgfpathlineto{\pgfqpoint{0.344393in}{5.024965in}}%
\pgfusepath{stroke}%
\end{pgfscope}%
\begin{pgfscope}%
\pgfpathrectangle{\pgfqpoint{0.100000in}{2.413063in}}{\pgfqpoint{5.037500in}{3.427208in}}%
\pgfusepath{clip}%
\pgfsetbuttcap%
\pgfsetroundjoin%
\pgfsetlinewidth{0.501875pt}%
\definecolor{currentstroke}{rgb}{0.827451,0.827451,0.827451}%
\pgfsetstrokecolor{currentstroke}%
\pgfsetdash{}{0pt}%
\pgfpathmoveto{\pgfqpoint{4.695486in}{4.946066in}}%
\pgfpathlineto{\pgfqpoint{4.686342in}{4.953500in}}%
\pgfpathlineto{\pgfqpoint{4.687616in}{4.976662in}}%
\pgfpathlineto{\pgfqpoint{4.678164in}{5.042494in}}%
\pgfpathlineto{\pgfqpoint{4.685572in}{5.065644in}}%
\pgfpathlineto{\pgfqpoint{4.690864in}{5.106703in}}%
\pgfpathlineto{\pgfqpoint{4.683724in}{5.126129in}}%
\pgfpathlineto{\pgfqpoint{4.709700in}{5.151988in}}%
\pgfpathlineto{\pgfqpoint{4.716237in}{5.172560in}}%
\pgfpathlineto{\pgfqpoint{4.703929in}{5.188795in}}%
\pgfpathlineto{\pgfqpoint{4.709172in}{5.209085in}}%
\pgfpathlineto{\pgfqpoint{4.705886in}{5.221830in}}%
\pgfpathlineto{\pgfqpoint{4.708708in}{5.249125in}}%
\pgfpathlineto{\pgfqpoint{4.713985in}{5.257552in}}%
\pgfpathlineto{\pgfqpoint{4.730253in}{5.262870in}}%
\pgfpathlineto{\pgfqpoint{4.753984in}{5.193372in}}%
\pgfpathlineto{\pgfqpoint{4.798011in}{5.049357in}}%
\pgfpathlineto{\pgfqpoint{4.814672in}{5.038191in}}%
\pgfpathlineto{\pgfqpoint{4.817670in}{5.025444in}}%
\pgfpathlineto{\pgfqpoint{4.826463in}{5.020294in}}%
\pgfpathlineto{\pgfqpoint{4.825794in}{4.997176in}}%
\pgfpathlineto{\pgfqpoint{4.816477in}{4.996805in}}%
\pgfpathlineto{\pgfqpoint{4.797601in}{4.982409in}}%
\pgfpathlineto{\pgfqpoint{4.792157in}{4.967966in}}%
\pgfpathlineto{\pgfqpoint{4.741634in}{4.955466in}}%
\pgfpathlineto{\pgfqpoint{4.695486in}{4.946066in}}%
\pgfusepath{stroke}%
\end{pgfscope}%
\begin{pgfscope}%
\pgfpathrectangle{\pgfqpoint{0.100000in}{2.413063in}}{\pgfqpoint{5.037500in}{3.427208in}}%
\pgfusepath{clip}%
\pgfsetbuttcap%
\pgfsetroundjoin%
\pgfsetlinewidth{0.501875pt}%
\definecolor{currentstroke}{rgb}{0.827451,0.827451,0.827451}%
\pgfsetstrokecolor{currentstroke}%
\pgfsetdash{}{0pt}%
\pgfpathmoveto{\pgfqpoint{3.115663in}{4.445025in}}%
\pgfpathlineto{\pgfqpoint{3.088992in}{4.471439in}}%
\pgfpathlineto{\pgfqpoint{3.003734in}{4.466521in}}%
\pgfpathlineto{\pgfqpoint{2.870765in}{4.462140in}}%
\pgfpathlineto{\pgfqpoint{2.736996in}{4.464227in}}%
\pgfpathlineto{\pgfqpoint{2.727607in}{4.480599in}}%
\pgfpathlineto{\pgfqpoint{2.731440in}{4.496676in}}%
\pgfpathlineto{\pgfqpoint{2.729620in}{4.524217in}}%
\pgfpathlineto{\pgfqpoint{2.722550in}{4.550893in}}%
\pgfpathlineto{\pgfqpoint{2.723372in}{4.564980in}}%
\pgfpathlineto{\pgfqpoint{2.708345in}{4.580897in}}%
\pgfpathlineto{\pgfqpoint{2.711614in}{4.603044in}}%
\pgfpathlineto{\pgfqpoint{2.688563in}{4.646790in}}%
\pgfpathlineto{\pgfqpoint{2.681695in}{4.683929in}}%
\pgfpathlineto{\pgfqpoint{2.665113in}{4.714137in}}%
\pgfpathlineto{\pgfqpoint{2.680853in}{4.751610in}}%
\pgfpathlineto{\pgfqpoint{2.673013in}{4.769048in}}%
\pgfpathlineto{\pgfqpoint{2.681778in}{4.800574in}}%
\pgfpathlineto{\pgfqpoint{2.765190in}{4.799670in}}%
\pgfpathlineto{\pgfqpoint{2.849769in}{4.799541in}}%
\pgfpathlineto{\pgfqpoint{2.929720in}{4.800716in}}%
\pgfpathlineto{\pgfqpoint{3.017472in}{4.802551in}}%
\pgfpathlineto{\pgfqpoint{3.120418in}{4.806289in}}%
\pgfpathlineto{\pgfqpoint{3.123167in}{4.791344in}}%
\pgfpathlineto{\pgfqpoint{3.134564in}{4.778814in}}%
\pgfpathlineto{\pgfqpoint{3.127348in}{4.765756in}}%
\pgfpathlineto{\pgfqpoint{3.130306in}{4.740271in}}%
\pgfpathlineto{\pgfqpoint{3.137369in}{4.719670in}}%
\pgfpathlineto{\pgfqpoint{3.169391in}{4.708766in}}%
\pgfpathlineto{\pgfqpoint{3.174878in}{4.694179in}}%
\pgfpathlineto{\pgfqpoint{3.192451in}{4.677802in}}%
\pgfpathlineto{\pgfqpoint{3.199619in}{4.660860in}}%
\pgfpathlineto{\pgfqpoint{3.217424in}{4.649495in}}%
\pgfpathlineto{\pgfqpoint{3.220210in}{4.635813in}}%
\pgfpathlineto{\pgfqpoint{3.216743in}{4.615099in}}%
\pgfpathlineto{\pgfqpoint{3.207681in}{4.608891in}}%
\pgfpathlineto{\pgfqpoint{3.204948in}{4.589174in}}%
\pgfpathlineto{\pgfqpoint{3.178904in}{4.573514in}}%
\pgfpathlineto{\pgfqpoint{3.144957in}{4.564935in}}%
\pgfpathlineto{\pgfqpoint{3.141846in}{4.545209in}}%
\pgfpathlineto{\pgfqpoint{3.154942in}{4.531116in}}%
\pgfpathlineto{\pgfqpoint{3.155476in}{4.513396in}}%
\pgfpathlineto{\pgfqpoint{3.144911in}{4.499468in}}%
\pgfpathlineto{\pgfqpoint{3.139365in}{4.478775in}}%
\pgfpathlineto{\pgfqpoint{3.120999in}{4.471925in}}%
\pgfpathlineto{\pgfqpoint{3.122170in}{4.448866in}}%
\pgfpathlineto{\pgfqpoint{3.115663in}{4.445025in}}%
\pgfusepath{stroke}%
\end{pgfscope}%
\begin{pgfscope}%
\pgfpathrectangle{\pgfqpoint{0.100000in}{2.413063in}}{\pgfqpoint{5.037500in}{3.427208in}}%
\pgfusepath{clip}%
\pgfsetbuttcap%
\pgfsetroundjoin%
\pgfsetlinewidth{0.501875pt}%
\definecolor{currentstroke}{rgb}{0.827451,0.827451,0.827451}%
\pgfsetstrokecolor{currentstroke}%
\pgfsetdash{}{0pt}%
\pgfpathmoveto{\pgfqpoint{4.770516in}{4.879426in}}%
\pgfpathlineto{\pgfqpoint{4.745254in}{4.875440in}}%
\pgfpathlineto{\pgfqpoint{4.629215in}{4.849072in}}%
\pgfpathlineto{\pgfqpoint{4.627187in}{4.852145in}}%
\pgfpathlineto{\pgfqpoint{4.628939in}{4.931925in}}%
\pgfpathlineto{\pgfqpoint{4.695486in}{4.946066in}}%
\pgfpathlineto{\pgfqpoint{4.741634in}{4.955466in}}%
\pgfpathlineto{\pgfqpoint{4.792157in}{4.967966in}}%
\pgfpathlineto{\pgfqpoint{4.797601in}{4.982409in}}%
\pgfpathlineto{\pgfqpoint{4.816477in}{4.996805in}}%
\pgfpathlineto{\pgfqpoint{4.825794in}{4.997176in}}%
\pgfpathlineto{\pgfqpoint{4.838043in}{4.976174in}}%
\pgfpathlineto{\pgfqpoint{4.826929in}{4.945516in}}%
\pgfpathlineto{\pgfqpoint{4.825298in}{4.927539in}}%
\pgfpathlineto{\pgfqpoint{4.847794in}{4.929227in}}%
\pgfpathlineto{\pgfqpoint{4.857997in}{4.920604in}}%
\pgfpathlineto{\pgfqpoint{4.880876in}{4.885354in}}%
\pgfpathlineto{\pgfqpoint{4.892247in}{4.881067in}}%
\pgfpathlineto{\pgfqpoint{4.911295in}{4.882657in}}%
\pgfpathlineto{\pgfqpoint{4.924550in}{4.894636in}}%
\pgfpathlineto{\pgfqpoint{4.933361in}{4.883945in}}%
\pgfpathlineto{\pgfqpoint{4.877952in}{4.854706in}}%
\pgfpathlineto{\pgfqpoint{4.876220in}{4.875687in}}%
\pgfpathlineto{\pgfqpoint{4.851050in}{4.843078in}}%
\pgfpathlineto{\pgfqpoint{4.842200in}{4.837460in}}%
\pgfpathlineto{\pgfqpoint{4.829857in}{4.856276in}}%
\pgfpathlineto{\pgfqpoint{4.826478in}{4.858855in}}%
\pgfpathlineto{\pgfqpoint{4.814987in}{4.864941in}}%
\pgfpathlineto{\pgfqpoint{4.804966in}{4.889628in}}%
\pgfpathlineto{\pgfqpoint{4.770516in}{4.879426in}}%
\pgfusepath{stroke}%
\end{pgfscope}%
\begin{pgfscope}%
\pgfpathrectangle{\pgfqpoint{0.100000in}{2.413063in}}{\pgfqpoint{5.037500in}{3.427208in}}%
\pgfusepath{clip}%
\pgfsetbuttcap%
\pgfsetroundjoin%
\pgfsetlinewidth{0.501875pt}%
\definecolor{currentstroke}{rgb}{0.827451,0.827451,0.827451}%
\pgfsetstrokecolor{currentstroke}%
\pgfsetdash{}{0pt}%
\pgfpathmoveto{\pgfqpoint{2.182662in}{4.420100in}}%
\pgfpathlineto{\pgfqpoint{2.191913in}{4.534814in}}%
\pgfpathlineto{\pgfqpoint{2.139519in}{4.539067in}}%
\pgfpathlineto{\pgfqpoint{2.018405in}{4.550818in}}%
\pgfpathlineto{\pgfqpoint{2.029673in}{4.665125in}}%
\pgfpathlineto{\pgfqpoint{2.040958in}{4.780305in}}%
\pgfpathlineto{\pgfqpoint{2.087517in}{4.775278in}}%
\pgfpathlineto{\pgfqpoint{2.206426in}{4.764043in}}%
\pgfpathlineto{\pgfqpoint{2.278390in}{4.758408in}}%
\pgfpathlineto{\pgfqpoint{2.365512in}{4.753414in}}%
\pgfpathlineto{\pgfqpoint{2.508914in}{4.746627in}}%
\pgfpathlineto{\pgfqpoint{2.512066in}{4.740198in}}%
\pgfpathlineto{\pgfqpoint{2.553155in}{4.719869in}}%
\pgfpathlineto{\pgfqpoint{2.565756in}{4.730193in}}%
\pgfpathlineto{\pgfqpoint{2.602058in}{4.729251in}}%
\pgfpathlineto{\pgfqpoint{2.623692in}{4.717918in}}%
\pgfpathlineto{\pgfqpoint{2.657995in}{4.705035in}}%
\pgfpathlineto{\pgfqpoint{2.667607in}{4.686241in}}%
\pgfpathlineto{\pgfqpoint{2.681695in}{4.683929in}}%
\pgfpathlineto{\pgfqpoint{2.688563in}{4.646790in}}%
\pgfpathlineto{\pgfqpoint{2.711614in}{4.603044in}}%
\pgfpathlineto{\pgfqpoint{2.708345in}{4.580897in}}%
\pgfpathlineto{\pgfqpoint{2.723372in}{4.564980in}}%
\pgfpathlineto{\pgfqpoint{2.722550in}{4.550893in}}%
\pgfpathlineto{\pgfqpoint{2.729620in}{4.524217in}}%
\pgfpathlineto{\pgfqpoint{2.731440in}{4.496676in}}%
\pgfpathlineto{\pgfqpoint{2.727607in}{4.480599in}}%
\pgfpathlineto{\pgfqpoint{2.736996in}{4.464227in}}%
\pgfpathlineto{\pgfqpoint{2.749875in}{4.434454in}}%
\pgfpathlineto{\pgfqpoint{2.762201in}{4.422337in}}%
\pgfpathlineto{\pgfqpoint{2.776895in}{4.396079in}}%
\pgfpathlineto{\pgfqpoint{2.735250in}{4.395647in}}%
\pgfpathlineto{\pgfqpoint{2.645206in}{4.397040in}}%
\pgfpathlineto{\pgfqpoint{2.545721in}{4.400088in}}%
\pgfpathlineto{\pgfqpoint{2.445651in}{4.403928in}}%
\pgfpathlineto{\pgfqpoint{2.298517in}{4.411965in}}%
\pgfpathlineto{\pgfqpoint{2.182662in}{4.420100in}}%
\pgfusepath{stroke}%
\end{pgfscope}%
\begin{pgfscope}%
\pgfpathrectangle{\pgfqpoint{0.100000in}{2.413063in}}{\pgfqpoint{5.037500in}{3.427208in}}%
\pgfusepath{clip}%
\pgfsetbuttcap%
\pgfsetroundjoin%
\pgfsetlinewidth{0.501875pt}%
\definecolor{currentstroke}{rgb}{0.827451,0.827451,0.827451}%
\pgfsetstrokecolor{currentstroke}%
\pgfsetdash{}{0pt}%
\pgfpathmoveto{\pgfqpoint{4.098414in}{4.766121in}}%
\pgfpathlineto{\pgfqpoint{4.144485in}{4.810050in}}%
\pgfpathlineto{\pgfqpoint{4.150212in}{4.825665in}}%
\pgfpathlineto{\pgfqpoint{4.163712in}{4.839033in}}%
\pgfpathlineto{\pgfqpoint{4.153582in}{4.858496in}}%
\pgfpathlineto{\pgfqpoint{4.140881in}{4.869880in}}%
\pgfpathlineto{\pgfqpoint{4.137210in}{4.890054in}}%
\pgfpathlineto{\pgfqpoint{4.184483in}{4.910768in}}%
\pgfpathlineto{\pgfqpoint{4.223622in}{4.917320in}}%
\pgfpathlineto{\pgfqpoint{4.244653in}{4.917758in}}%
\pgfpathlineto{\pgfqpoint{4.260673in}{4.909843in}}%
\pgfpathlineto{\pgfqpoint{4.276290in}{4.916911in}}%
\pgfpathlineto{\pgfqpoint{4.314380in}{4.924832in}}%
\pgfpathlineto{\pgfqpoint{4.327541in}{4.935056in}}%
\pgfpathlineto{\pgfqpoint{4.347056in}{4.957689in}}%
\pgfpathlineto{\pgfqpoint{4.364809in}{4.967690in}}%
\pgfpathlineto{\pgfqpoint{4.366061in}{4.977283in}}%
\pgfpathlineto{\pgfqpoint{4.356744in}{4.999166in}}%
\pgfpathlineto{\pgfqpoint{4.363480in}{5.012045in}}%
\pgfpathlineto{\pgfqpoint{4.354410in}{5.025892in}}%
\pgfpathlineto{\pgfqpoint{4.340522in}{5.026859in}}%
\pgfpathlineto{\pgfqpoint{4.375282in}{5.068688in}}%
\pgfpathlineto{\pgfqpoint{4.379411in}{5.084629in}}%
\pgfpathlineto{\pgfqpoint{4.406769in}{5.125284in}}%
\pgfpathlineto{\pgfqpoint{4.432165in}{5.147284in}}%
\pgfpathlineto{\pgfqpoint{4.449604in}{5.156471in}}%
\pgfpathlineto{\pgfqpoint{4.506667in}{5.169401in}}%
\pgfpathlineto{\pgfqpoint{4.560150in}{5.184002in}}%
\pgfpathlineto{\pgfqpoint{4.567234in}{5.161082in}}%
\pgfpathlineto{\pgfqpoint{4.567736in}{5.139830in}}%
\pgfpathlineto{\pgfqpoint{4.579935in}{5.120538in}}%
\pgfpathlineto{\pgfqpoint{4.583695in}{5.100896in}}%
\pgfpathlineto{\pgfqpoint{4.579007in}{5.074245in}}%
\pgfpathlineto{\pgfqpoint{4.593285in}{5.038300in}}%
\pgfpathlineto{\pgfqpoint{4.609564in}{5.018312in}}%
\pgfpathlineto{\pgfqpoint{4.622154in}{4.953342in}}%
\pgfpathlineto{\pgfqpoint{4.628939in}{4.931925in}}%
\pgfpathlineto{\pgfqpoint{4.627187in}{4.852145in}}%
\pgfpathlineto{\pgfqpoint{4.629215in}{4.849072in}}%
\pgfpathlineto{\pgfqpoint{4.644037in}{4.763268in}}%
\pgfpathlineto{\pgfqpoint{4.652350in}{4.755449in}}%
\pgfpathlineto{\pgfqpoint{4.634470in}{4.738078in}}%
\pgfpathlineto{\pgfqpoint{4.643292in}{4.728107in}}%
\pgfpathlineto{\pgfqpoint{4.635538in}{4.713027in}}%
\pgfpathlineto{\pgfqpoint{4.635604in}{4.706606in}}%
\pgfpathlineto{\pgfqpoint{4.625910in}{4.700813in}}%
\pgfpathlineto{\pgfqpoint{4.621189in}{4.688007in}}%
\pgfpathlineto{\pgfqpoint{4.622686in}{4.723221in}}%
\pgfpathlineto{\pgfqpoint{4.592639in}{4.730974in}}%
\pgfpathlineto{\pgfqpoint{4.545650in}{4.747002in}}%
\pgfpathlineto{\pgfqpoint{4.539004in}{4.754897in}}%
\pgfpathlineto{\pgfqpoint{4.519368in}{4.756789in}}%
\pgfpathlineto{\pgfqpoint{4.508073in}{4.768540in}}%
\pgfpathlineto{\pgfqpoint{4.501979in}{4.791943in}}%
\pgfpathlineto{\pgfqpoint{4.485967in}{4.794854in}}%
\pgfpathlineto{\pgfqpoint{4.475258in}{4.807131in}}%
\pgfpathlineto{\pgfqpoint{4.338937in}{4.779648in}}%
\pgfpathlineto{\pgfqpoint{4.273785in}{4.766174in}}%
\pgfpathlineto{\pgfqpoint{4.174866in}{4.748119in}}%
\pgfpathlineto{\pgfqpoint{4.103631in}{4.736102in}}%
\pgfpathlineto{\pgfqpoint{4.098414in}{4.766121in}}%
\pgfusepath{stroke}%
\end{pgfscope}%
\begin{pgfscope}%
\pgfpathrectangle{\pgfqpoint{0.100000in}{2.413063in}}{\pgfqpoint{5.037500in}{3.427208in}}%
\pgfusepath{clip}%
\pgfsetbuttcap%
\pgfsetroundjoin%
\pgfsetlinewidth{0.501875pt}%
\definecolor{currentstroke}{rgb}{0.827451,0.827451,0.827451}%
\pgfsetstrokecolor{currentstroke}%
\pgfsetdash{}{0pt}%
\pgfpathmoveto{\pgfqpoint{4.645692in}{4.680855in}}%
\pgfpathlineto{\pgfqpoint{4.624609in}{4.674287in}}%
\pgfpathlineto{\pgfqpoint{4.621062in}{4.680325in}}%
\pgfpathlineto{\pgfqpoint{4.627829in}{4.700594in}}%
\pgfpathlineto{\pgfqpoint{4.640362in}{4.702589in}}%
\pgfpathlineto{\pgfqpoint{4.639240in}{4.708971in}}%
\pgfpathlineto{\pgfqpoint{4.650493in}{4.718549in}}%
\pgfpathlineto{\pgfqpoint{4.683026in}{4.726179in}}%
\pgfpathlineto{\pgfqpoint{4.697509in}{4.737730in}}%
\pgfpathlineto{\pgfqpoint{4.730045in}{4.747355in}}%
\pgfpathlineto{\pgfqpoint{4.744884in}{4.743833in}}%
\pgfpathlineto{\pgfqpoint{4.757367in}{4.759352in}}%
\pgfpathlineto{\pgfqpoint{4.776234in}{4.761403in}}%
\pgfpathlineto{\pgfqpoint{4.744046in}{4.731132in}}%
\pgfpathlineto{\pgfqpoint{4.645692in}{4.680855in}}%
\pgfusepath{stroke}%
\end{pgfscope}%
\begin{pgfscope}%
\pgfpathrectangle{\pgfqpoint{0.100000in}{2.413063in}}{\pgfqpoint{5.037500in}{3.427208in}}%
\pgfusepath{clip}%
\pgfsetbuttcap%
\pgfsetroundjoin%
\pgfsetlinewidth{0.501875pt}%
\definecolor{currentstroke}{rgb}{0.827451,0.827451,0.827451}%
\pgfsetstrokecolor{currentstroke}%
\pgfsetdash{}{0pt}%
\pgfpathmoveto{\pgfqpoint{4.172033in}{4.480971in}}%
\pgfpathlineto{\pgfqpoint{4.080863in}{4.465897in}}%
\pgfpathlineto{\pgfqpoint{4.064326in}{4.570087in}}%
\pgfpathlineto{\pgfqpoint{4.039771in}{4.723670in}}%
\pgfpathlineto{\pgfqpoint{4.098414in}{4.766121in}}%
\pgfpathlineto{\pgfqpoint{4.103631in}{4.736102in}}%
\pgfpathlineto{\pgfqpoint{4.174866in}{4.748119in}}%
\pgfpathlineto{\pgfqpoint{4.273785in}{4.766174in}}%
\pgfpathlineto{\pgfqpoint{4.338937in}{4.779648in}}%
\pgfpathlineto{\pgfqpoint{4.475258in}{4.807131in}}%
\pgfpathlineto{\pgfqpoint{4.485967in}{4.794854in}}%
\pgfpathlineto{\pgfqpoint{4.501979in}{4.791943in}}%
\pgfpathlineto{\pgfqpoint{4.508073in}{4.768540in}}%
\pgfpathlineto{\pgfqpoint{4.519368in}{4.756789in}}%
\pgfpathlineto{\pgfqpoint{4.539004in}{4.754897in}}%
\pgfpathlineto{\pgfqpoint{4.545650in}{4.747002in}}%
\pgfpathlineto{\pgfqpoint{4.538903in}{4.740911in}}%
\pgfpathlineto{\pgfqpoint{4.532824in}{4.719359in}}%
\pgfpathlineto{\pgfqpoint{4.518581in}{4.695143in}}%
\pgfpathlineto{\pgfqpoint{4.528142in}{4.684611in}}%
\pgfpathlineto{\pgfqpoint{4.520318in}{4.673328in}}%
\pgfpathlineto{\pgfqpoint{4.524732in}{4.648584in}}%
\pgfpathlineto{\pgfqpoint{4.538821in}{4.635595in}}%
\pgfpathlineto{\pgfqpoint{4.572254in}{4.614493in}}%
\pgfpathlineto{\pgfqpoint{4.545330in}{4.584718in}}%
\pgfpathlineto{\pgfqpoint{4.544953in}{4.573417in}}%
\pgfpathlineto{\pgfqpoint{4.523034in}{4.558842in}}%
\pgfpathlineto{\pgfqpoint{4.498797in}{4.556107in}}%
\pgfpathlineto{\pgfqpoint{4.492802in}{4.543439in}}%
\pgfpathlineto{\pgfqpoint{4.425387in}{4.528852in}}%
\pgfpathlineto{\pgfqpoint{4.346722in}{4.513009in}}%
\pgfpathlineto{\pgfqpoint{4.292651in}{4.503327in}}%
\pgfpathlineto{\pgfqpoint{4.172033in}{4.480971in}}%
\pgfusepath{stroke}%
\end{pgfscope}%
\begin{pgfscope}%
\pgfpathrectangle{\pgfqpoint{0.100000in}{2.413063in}}{\pgfqpoint{5.037500in}{3.427208in}}%
\pgfusepath{clip}%
\pgfsetbuttcap%
\pgfsetroundjoin%
\pgfsetlinewidth{0.501875pt}%
\definecolor{currentstroke}{rgb}{0.827451,0.827451,0.827451}%
\pgfsetstrokecolor{currentstroke}%
\pgfsetdash{}{0pt}%
\pgfpathmoveto{\pgfqpoint{4.629215in}{4.849072in}}%
\pgfpathlineto{\pgfqpoint{4.745254in}{4.875440in}}%
\pgfpathlineto{\pgfqpoint{4.770516in}{4.879426in}}%
\pgfpathlineto{\pgfqpoint{4.787236in}{4.813688in}}%
\pgfpathlineto{\pgfqpoint{4.784590in}{4.801916in}}%
\pgfpathlineto{\pgfqpoint{4.730879in}{4.781122in}}%
\pgfpathlineto{\pgfqpoint{4.698815in}{4.773854in}}%
\pgfpathlineto{\pgfqpoint{4.685191in}{4.757551in}}%
\pgfpathlineto{\pgfqpoint{4.643292in}{4.728107in}}%
\pgfpathlineto{\pgfqpoint{4.634470in}{4.738078in}}%
\pgfpathlineto{\pgfqpoint{4.652350in}{4.755449in}}%
\pgfpathlineto{\pgfqpoint{4.644037in}{4.763268in}}%
\pgfpathlineto{\pgfqpoint{4.629215in}{4.849072in}}%
\pgfusepath{stroke}%
\end{pgfscope}%
\begin{pgfscope}%
\pgfpathrectangle{\pgfqpoint{0.100000in}{2.413063in}}{\pgfqpoint{5.037500in}{3.427208in}}%
\pgfusepath{clip}%
\pgfsetbuttcap%
\pgfsetroundjoin%
\pgfsetlinewidth{0.501875pt}%
\definecolor{currentstroke}{rgb}{0.827451,0.827451,0.827451}%
\pgfsetstrokecolor{currentstroke}%
\pgfsetdash{}{0pt}%
\pgfpathmoveto{\pgfqpoint{4.770516in}{4.879426in}}%
\pgfpathlineto{\pgfqpoint{4.804966in}{4.889628in}}%
\pgfpathlineto{\pgfqpoint{4.814987in}{4.864941in}}%
\pgfpathlineto{\pgfqpoint{4.826478in}{4.858855in}}%
\pgfpathlineto{\pgfqpoint{4.812298in}{4.848456in}}%
\pgfpathlineto{\pgfqpoint{4.814086in}{4.817915in}}%
\pgfpathlineto{\pgfqpoint{4.784590in}{4.801916in}}%
\pgfpathlineto{\pgfqpoint{4.787236in}{4.813688in}}%
\pgfpathlineto{\pgfqpoint{4.770516in}{4.879426in}}%
\pgfusepath{stroke}%
\end{pgfscope}%
\begin{pgfscope}%
\pgfpathrectangle{\pgfqpoint{0.100000in}{2.413063in}}{\pgfqpoint{5.037500in}{3.427208in}}%
\pgfusepath{clip}%
\pgfsetbuttcap%
\pgfsetroundjoin%
\pgfsetlinewidth{0.501875pt}%
\definecolor{currentstroke}{rgb}{0.827451,0.827451,0.827451}%
\pgfsetstrokecolor{currentstroke}%
\pgfsetdash{}{0pt}%
\pgfpathmoveto{\pgfqpoint{4.519150in}{4.548078in}}%
\pgfpathlineto{\pgfqpoint{4.523034in}{4.558842in}}%
\pgfpathlineto{\pgfqpoint{4.544953in}{4.573417in}}%
\pgfpathlineto{\pgfqpoint{4.545330in}{4.584718in}}%
\pgfpathlineto{\pgfqpoint{4.572254in}{4.614493in}}%
\pgfpathlineto{\pgfqpoint{4.538821in}{4.635595in}}%
\pgfpathlineto{\pgfqpoint{4.524732in}{4.648584in}}%
\pgfpathlineto{\pgfqpoint{4.520318in}{4.673328in}}%
\pgfpathlineto{\pgfqpoint{4.528142in}{4.684611in}}%
\pgfpathlineto{\pgfqpoint{4.518581in}{4.695143in}}%
\pgfpathlineto{\pgfqpoint{4.532824in}{4.719359in}}%
\pgfpathlineto{\pgfqpoint{4.538903in}{4.740911in}}%
\pgfpathlineto{\pgfqpoint{4.545650in}{4.747002in}}%
\pgfpathlineto{\pgfqpoint{4.592639in}{4.730974in}}%
\pgfpathlineto{\pgfqpoint{4.622686in}{4.723221in}}%
\pgfpathlineto{\pgfqpoint{4.621189in}{4.688007in}}%
\pgfpathlineto{\pgfqpoint{4.602937in}{4.661303in}}%
\pgfpathlineto{\pgfqpoint{4.617977in}{4.657374in}}%
\pgfpathlineto{\pgfqpoint{4.633592in}{4.645915in}}%
\pgfpathlineto{\pgfqpoint{4.634513in}{4.614571in}}%
\pgfpathlineto{\pgfqpoint{4.629787in}{4.592383in}}%
\pgfpathlineto{\pgfqpoint{4.632932in}{4.574157in}}%
\pgfpathlineto{\pgfqpoint{4.605787in}{4.512656in}}%
\pgfpathlineto{\pgfqpoint{4.596032in}{4.483939in}}%
\pgfpathlineto{\pgfqpoint{4.582411in}{4.497898in}}%
\pgfpathlineto{\pgfqpoint{4.564348in}{4.495519in}}%
\pgfpathlineto{\pgfqpoint{4.519141in}{4.521637in}}%
\pgfpathlineto{\pgfqpoint{4.514516in}{4.535624in}}%
\pgfpathlineto{\pgfqpoint{4.519150in}{4.548078in}}%
\pgfusepath{stroke}%
\end{pgfscope}%
\begin{pgfscope}%
\pgfpathrectangle{\pgfqpoint{0.100000in}{2.413063in}}{\pgfqpoint{5.037500in}{3.427208in}}%
\pgfusepath{clip}%
\pgfsetbuttcap%
\pgfsetroundjoin%
\pgfsetlinewidth{0.501875pt}%
\definecolor{currentstroke}{rgb}{0.827451,0.827451,0.827451}%
\pgfsetstrokecolor{currentstroke}%
\pgfsetdash{}{0pt}%
\pgfpathmoveto{\pgfqpoint{3.437265in}{4.167692in}}%
\pgfpathlineto{\pgfqpoint{3.435151in}{4.196223in}}%
\pgfpathlineto{\pgfqpoint{3.443366in}{4.210378in}}%
\pgfpathlineto{\pgfqpoint{3.437951in}{4.217314in}}%
\pgfpathlineto{\pgfqpoint{3.457695in}{4.240023in}}%
\pgfpathlineto{\pgfqpoint{3.456564in}{4.245496in}}%
\pgfpathlineto{\pgfqpoint{3.475746in}{4.285114in}}%
\pgfpathlineto{\pgfqpoint{3.471869in}{4.304221in}}%
\pgfpathlineto{\pgfqpoint{3.457959in}{4.324225in}}%
\pgfpathlineto{\pgfqpoint{3.467615in}{4.348563in}}%
\pgfpathlineto{\pgfqpoint{3.455099in}{4.508720in}}%
\pgfpathlineto{\pgfqpoint{3.446087in}{4.621075in}}%
\pgfpathlineto{\pgfqpoint{3.458537in}{4.611765in}}%
\pgfpathlineto{\pgfqpoint{3.472426in}{4.612017in}}%
\pgfpathlineto{\pgfqpoint{3.505233in}{4.631021in}}%
\pgfpathlineto{\pgfqpoint{3.605857in}{4.640409in}}%
\pgfpathlineto{\pgfqpoint{3.680336in}{4.648259in}}%
\pgfpathlineto{\pgfqpoint{3.680994in}{4.640971in}}%
\pgfpathlineto{\pgfqpoint{3.697996in}{4.487004in}}%
\pgfpathlineto{\pgfqpoint{3.712630in}{4.343743in}}%
\pgfpathlineto{\pgfqpoint{3.706322in}{4.337017in}}%
\pgfpathlineto{\pgfqpoint{3.716011in}{4.320323in}}%
\pgfpathlineto{\pgfqpoint{3.715962in}{4.308292in}}%
\pgfpathlineto{\pgfqpoint{3.702135in}{4.305269in}}%
\pgfpathlineto{\pgfqpoint{3.686663in}{4.293671in}}%
\pgfpathlineto{\pgfqpoint{3.676188in}{4.298237in}}%
\pgfpathlineto{\pgfqpoint{3.660509in}{4.290813in}}%
\pgfpathlineto{\pgfqpoint{3.665360in}{4.275902in}}%
\pgfpathlineto{\pgfqpoint{3.649244in}{4.260922in}}%
\pgfpathlineto{\pgfqpoint{3.644795in}{4.243592in}}%
\pgfpathlineto{\pgfqpoint{3.633714in}{4.240750in}}%
\pgfpathlineto{\pgfqpoint{3.625445in}{4.227626in}}%
\pgfpathlineto{\pgfqpoint{3.626527in}{4.214409in}}%
\pgfpathlineto{\pgfqpoint{3.616798in}{4.205115in}}%
\pgfpathlineto{\pgfqpoint{3.602160in}{4.206550in}}%
\pgfpathlineto{\pgfqpoint{3.591032in}{4.220810in}}%
\pgfpathlineto{\pgfqpoint{3.572178in}{4.207078in}}%
\pgfpathlineto{\pgfqpoint{3.573500in}{4.195078in}}%
\pgfpathlineto{\pgfqpoint{3.552531in}{4.188067in}}%
\pgfpathlineto{\pgfqpoint{3.547262in}{4.196892in}}%
\pgfpathlineto{\pgfqpoint{3.526847in}{4.186886in}}%
\pgfpathlineto{\pgfqpoint{3.519389in}{4.172561in}}%
\pgfpathlineto{\pgfqpoint{3.494803in}{4.187250in}}%
\pgfpathlineto{\pgfqpoint{3.455836in}{4.177519in}}%
\pgfpathlineto{\pgfqpoint{3.437265in}{4.167692in}}%
\pgfusepath{stroke}%
\end{pgfscope}%
\begin{pgfscope}%
\pgfpathrectangle{\pgfqpoint{0.100000in}{2.413063in}}{\pgfqpoint{5.037500in}{3.427208in}}%
\pgfusepath{clip}%
\pgfsetbuttcap%
\pgfsetroundjoin%
\pgfsetlinewidth{0.501875pt}%
\definecolor{currentstroke}{rgb}{0.827451,0.827451,0.827451}%
\pgfsetstrokecolor{currentstroke}%
\pgfsetdash{}{0pt}%
\pgfpathmoveto{\pgfqpoint{0.689161in}{4.917975in}}%
\pgfpathlineto{\pgfqpoint{0.745509in}{4.902295in}}%
\pgfpathlineto{\pgfqpoint{0.838915in}{4.878210in}}%
\pgfpathlineto{\pgfqpoint{0.935738in}{4.853400in}}%
\pgfpathlineto{\pgfqpoint{1.024981in}{4.831878in}}%
\pgfpathlineto{\pgfqpoint{1.102093in}{4.814420in}}%
\pgfpathlineto{\pgfqpoint{1.184593in}{4.796279in}}%
\pgfpathlineto{\pgfqpoint{1.160756in}{4.683789in}}%
\pgfpathlineto{\pgfqpoint{1.139521in}{4.583892in}}%
\pgfpathlineto{\pgfqpoint{1.104685in}{4.422710in}}%
\pgfpathlineto{\pgfqpoint{1.078531in}{4.301052in}}%
\pgfpathlineto{\pgfqpoint{1.064402in}{4.233160in}}%
\pgfpathlineto{\pgfqpoint{1.046284in}{4.145043in}}%
\pgfpathlineto{\pgfqpoint{1.033749in}{4.127168in}}%
\pgfpathlineto{\pgfqpoint{1.023660in}{4.126570in}}%
\pgfpathlineto{\pgfqpoint{1.016464in}{4.142175in}}%
\pgfpathlineto{\pgfqpoint{0.999941in}{4.147841in}}%
\pgfpathlineto{\pgfqpoint{0.982136in}{4.145855in}}%
\pgfpathlineto{\pgfqpoint{0.977083in}{4.133092in}}%
\pgfpathlineto{\pgfqpoint{0.976556in}{4.088769in}}%
\pgfpathlineto{\pgfqpoint{0.971224in}{4.078649in}}%
\pgfpathlineto{\pgfqpoint{0.974271in}{4.043066in}}%
\pgfpathlineto{\pgfqpoint{0.963137in}{4.019360in}}%
\pgfpathlineto{\pgfqpoint{0.872849in}{4.158390in}}%
\pgfpathlineto{\pgfqpoint{0.783931in}{4.293488in}}%
\pgfpathlineto{\pgfqpoint{0.737645in}{4.364331in}}%
\pgfpathlineto{\pgfqpoint{0.699033in}{4.425478in}}%
\pgfpathlineto{\pgfqpoint{0.650381in}{4.501117in}}%
\pgfpathlineto{\pgfqpoint{0.595308in}{4.585963in}}%
\pgfpathlineto{\pgfqpoint{0.617950in}{4.666497in}}%
\pgfpathlineto{\pgfqpoint{0.663516in}{4.828009in}}%
\pgfpathlineto{\pgfqpoint{0.689161in}{4.917975in}}%
\pgfusepath{stroke}%
\end{pgfscope}%
\begin{pgfscope}%
\pgfpathrectangle{\pgfqpoint{0.100000in}{2.413063in}}{\pgfqpoint{5.037500in}{3.427208in}}%
\pgfusepath{clip}%
\pgfsetbuttcap%
\pgfsetroundjoin%
\pgfsetlinewidth{0.501875pt}%
\definecolor{currentstroke}{rgb}{0.827451,0.827451,0.827451}%
\pgfsetstrokecolor{currentstroke}%
\pgfsetdash{}{0pt}%
\pgfpathmoveto{\pgfqpoint{1.184593in}{4.796279in}}%
\pgfpathlineto{\pgfqpoint{1.272793in}{4.778734in}}%
\pgfpathlineto{\pgfqpoint{1.436081in}{4.747212in}}%
\pgfpathlineto{\pgfqpoint{1.415642in}{4.633831in}}%
\pgfpathlineto{\pgfqpoint{1.505537in}{4.618440in}}%
\pgfpathlineto{\pgfqpoint{1.587436in}{4.605334in}}%
\pgfpathlineto{\pgfqpoint{1.573205in}{4.515697in}}%
\pgfpathlineto{\pgfqpoint{1.558125in}{4.419035in}}%
\pgfpathlineto{\pgfqpoint{1.537936in}{4.292117in}}%
\pgfpathlineto{\pgfqpoint{1.537414in}{4.281477in}}%
\pgfpathlineto{\pgfqpoint{1.516457in}{4.149943in}}%
\pgfpathlineto{\pgfqpoint{1.430188in}{4.163315in}}%
\pgfpathlineto{\pgfqpoint{1.386229in}{4.172143in}}%
\pgfpathlineto{\pgfqpoint{1.227336in}{4.200074in}}%
\pgfpathlineto{\pgfqpoint{1.167500in}{4.211874in}}%
\pgfpathlineto{\pgfqpoint{1.064402in}{4.233160in}}%
\pgfpathlineto{\pgfqpoint{1.078531in}{4.301052in}}%
\pgfpathlineto{\pgfqpoint{1.104685in}{4.422710in}}%
\pgfpathlineto{\pgfqpoint{1.139521in}{4.583892in}}%
\pgfpathlineto{\pgfqpoint{1.160756in}{4.683789in}}%
\pgfpathlineto{\pgfqpoint{1.184593in}{4.796279in}}%
\pgfusepath{stroke}%
\end{pgfscope}%
\begin{pgfscope}%
\pgfpathrectangle{\pgfqpoint{0.100000in}{2.413063in}}{\pgfqpoint{5.037500in}{3.427208in}}%
\pgfusepath{clip}%
\pgfsetbuttcap%
\pgfsetroundjoin%
\pgfsetlinewidth{0.501875pt}%
\definecolor{currentstroke}{rgb}{0.827451,0.827451,0.827451}%
\pgfsetstrokecolor{currentstroke}%
\pgfsetdash{}{0pt}%
\pgfpathmoveto{\pgfqpoint{0.344393in}{5.024965in}}%
\pgfpathlineto{\pgfqpoint{0.375733in}{5.014034in}}%
\pgfpathlineto{\pgfqpoint{0.501273in}{4.974949in}}%
\pgfpathlineto{\pgfqpoint{0.616760in}{4.938599in}}%
\pgfpathlineto{\pgfqpoint{0.689161in}{4.917975in}}%
\pgfpathlineto{\pgfqpoint{0.663516in}{4.828009in}}%
\pgfpathlineto{\pgfqpoint{0.617950in}{4.666497in}}%
\pgfpathlineto{\pgfqpoint{0.595308in}{4.585963in}}%
\pgfpathlineto{\pgfqpoint{0.650381in}{4.501117in}}%
\pgfpathlineto{\pgfqpoint{0.699033in}{4.425478in}}%
\pgfpathlineto{\pgfqpoint{0.737645in}{4.364331in}}%
\pgfpathlineto{\pgfqpoint{0.783931in}{4.293488in}}%
\pgfpathlineto{\pgfqpoint{0.872849in}{4.158390in}}%
\pgfpathlineto{\pgfqpoint{0.963137in}{4.019360in}}%
\pgfpathlineto{\pgfqpoint{0.959513in}{4.005573in}}%
\pgfpathlineto{\pgfqpoint{0.970403in}{3.983615in}}%
\pgfpathlineto{\pgfqpoint{0.972557in}{3.953579in}}%
\pgfpathlineto{\pgfqpoint{0.988373in}{3.939008in}}%
\pgfpathlineto{\pgfqpoint{0.988999in}{3.927268in}}%
\pgfpathlineto{\pgfqpoint{0.960679in}{3.913943in}}%
\pgfpathlineto{\pgfqpoint{0.947187in}{3.900602in}}%
\pgfpathlineto{\pgfqpoint{0.942979in}{3.871143in}}%
\pgfpathlineto{\pgfqpoint{0.936150in}{3.855078in}}%
\pgfpathlineto{\pgfqpoint{0.921760in}{3.841535in}}%
\pgfpathlineto{\pgfqpoint{0.907448in}{3.806322in}}%
\pgfpathlineto{\pgfqpoint{0.927572in}{3.787979in}}%
\pgfpathlineto{\pgfqpoint{0.924974in}{3.772858in}}%
\pgfpathlineto{\pgfqpoint{0.908473in}{3.762335in}}%
\pgfpathlineto{\pgfqpoint{0.897097in}{3.764192in}}%
\pgfpathlineto{\pgfqpoint{0.762813in}{3.782827in}}%
\pgfpathlineto{\pgfqpoint{0.663632in}{3.796898in}}%
\pgfpathlineto{\pgfqpoint{0.667969in}{3.812911in}}%
\pgfpathlineto{\pgfqpoint{0.661512in}{3.839496in}}%
\pgfpathlineto{\pgfqpoint{0.660861in}{3.866312in}}%
\pgfpathlineto{\pgfqpoint{0.656642in}{3.882005in}}%
\pgfpathlineto{\pgfqpoint{0.643643in}{3.904441in}}%
\pgfpathlineto{\pgfqpoint{0.606255in}{3.956207in}}%
\pgfpathlineto{\pgfqpoint{0.594010in}{3.962550in}}%
\pgfpathlineto{\pgfqpoint{0.578232in}{3.962450in}}%
\pgfpathlineto{\pgfqpoint{0.581837in}{3.978780in}}%
\pgfpathlineto{\pgfqpoint{0.574366in}{3.999201in}}%
\pgfpathlineto{\pgfqpoint{0.537669in}{4.009351in}}%
\pgfpathlineto{\pgfqpoint{0.515300in}{4.028145in}}%
\pgfpathlineto{\pgfqpoint{0.513445in}{4.039644in}}%
\pgfpathlineto{\pgfqpoint{0.487688in}{4.068115in}}%
\pgfpathlineto{\pgfqpoint{0.463170in}{4.073562in}}%
\pgfpathlineto{\pgfqpoint{0.440454in}{4.088031in}}%
\pgfpathlineto{\pgfqpoint{0.410577in}{4.093037in}}%
\pgfpathlineto{\pgfqpoint{0.397812in}{4.112333in}}%
\pgfpathlineto{\pgfqpoint{0.409935in}{4.142878in}}%
\pgfpathlineto{\pgfqpoint{0.406227in}{4.149746in}}%
\pgfpathlineto{\pgfqpoint{0.416322in}{4.175209in}}%
\pgfpathlineto{\pgfqpoint{0.398364in}{4.188760in}}%
\pgfpathlineto{\pgfqpoint{0.404191in}{4.213326in}}%
\pgfpathlineto{\pgfqpoint{0.394598in}{4.219606in}}%
\pgfpathlineto{\pgfqpoint{0.386316in}{4.242826in}}%
\pgfpathlineto{\pgfqpoint{0.376322in}{4.249907in}}%
\pgfpathlineto{\pgfqpoint{0.375570in}{4.266679in}}%
\pgfpathlineto{\pgfqpoint{0.367740in}{4.278505in}}%
\pgfpathlineto{\pgfqpoint{0.355940in}{4.318337in}}%
\pgfpathlineto{\pgfqpoint{0.343030in}{4.337414in}}%
\pgfpathlineto{\pgfqpoint{0.345836in}{4.369800in}}%
\pgfpathlineto{\pgfqpoint{0.361018in}{4.373059in}}%
\pgfpathlineto{\pgfqpoint{0.370904in}{4.390624in}}%
\pgfpathlineto{\pgfqpoint{0.364931in}{4.409676in}}%
\pgfpathlineto{\pgfqpoint{0.348784in}{4.412856in}}%
\pgfpathlineto{\pgfqpoint{0.340727in}{4.421767in}}%
\pgfpathlineto{\pgfqpoint{0.327682in}{4.454544in}}%
\pgfpathlineto{\pgfqpoint{0.333779in}{4.466316in}}%
\pgfpathlineto{\pgfqpoint{0.329454in}{4.488234in}}%
\pgfpathlineto{\pgfqpoint{0.339031in}{4.516622in}}%
\pgfpathlineto{\pgfqpoint{0.350249in}{4.506168in}}%
\pgfpathlineto{\pgfqpoint{0.345158in}{4.493826in}}%
\pgfpathlineto{\pgfqpoint{0.364600in}{4.474261in}}%
\pgfpathlineto{\pgfqpoint{0.363359in}{4.503314in}}%
\pgfpathlineto{\pgfqpoint{0.355028in}{4.511106in}}%
\pgfpathlineto{\pgfqpoint{0.364550in}{4.536651in}}%
\pgfpathlineto{\pgfqpoint{0.400566in}{4.532583in}}%
\pgfpathlineto{\pgfqpoint{0.395719in}{4.542082in}}%
\pgfpathlineto{\pgfqpoint{0.371948in}{4.541147in}}%
\pgfpathlineto{\pgfqpoint{0.360641in}{4.555535in}}%
\pgfpathlineto{\pgfqpoint{0.346407in}{4.542758in}}%
\pgfpathlineto{\pgfqpoint{0.338854in}{4.521403in}}%
\pgfpathlineto{\pgfqpoint{0.318696in}{4.550139in}}%
\pgfpathlineto{\pgfqpoint{0.310932in}{4.555371in}}%
\pgfpathlineto{\pgfqpoint{0.313889in}{4.586646in}}%
\pgfpathlineto{\pgfqpoint{0.307746in}{4.605089in}}%
\pgfpathlineto{\pgfqpoint{0.296607in}{4.622417in}}%
\pgfpathlineto{\pgfqpoint{0.273798in}{4.675509in}}%
\pgfpathlineto{\pgfqpoint{0.281255in}{4.687249in}}%
\pgfpathlineto{\pgfqpoint{0.281168in}{4.724352in}}%
\pgfpathlineto{\pgfqpoint{0.293478in}{4.745047in}}%
\pgfpathlineto{\pgfqpoint{0.296301in}{4.777389in}}%
\pgfpathlineto{\pgfqpoint{0.284693in}{4.814445in}}%
\pgfpathlineto{\pgfqpoint{0.269293in}{4.838032in}}%
\pgfpathlineto{\pgfqpoint{0.272040in}{4.859316in}}%
\pgfpathlineto{\pgfqpoint{0.315286in}{4.910828in}}%
\pgfpathlineto{\pgfqpoint{0.317434in}{4.928400in}}%
\pgfpathlineto{\pgfqpoint{0.336921in}{4.961920in}}%
\pgfpathlineto{\pgfqpoint{0.339624in}{4.993692in}}%
\pgfpathlineto{\pgfqpoint{0.333359in}{5.001799in}}%
\pgfpathlineto{\pgfqpoint{0.344393in}{5.024965in}}%
\pgfusepath{stroke}%
\end{pgfscope}%
\begin{pgfscope}%
\pgfpathrectangle{\pgfqpoint{0.100000in}{2.413063in}}{\pgfqpoint{5.037500in}{3.427208in}}%
\pgfusepath{clip}%
\pgfsetbuttcap%
\pgfsetroundjoin%
\pgfsetlinewidth{0.501875pt}%
\definecolor{currentstroke}{rgb}{0.827451,0.827451,0.827451}%
\pgfsetstrokecolor{currentstroke}%
\pgfsetdash{}{0pt}%
\pgfpathmoveto{\pgfqpoint{3.712630in}{4.343743in}}%
\pgfpathlineto{\pgfqpoint{3.697996in}{4.487004in}}%
\pgfpathlineto{\pgfqpoint{3.680994in}{4.640971in}}%
\pgfpathlineto{\pgfqpoint{3.792336in}{4.657553in}}%
\pgfpathlineto{\pgfqpoint{3.821886in}{4.649862in}}%
\pgfpathlineto{\pgfqpoint{3.836059in}{4.641513in}}%
\pgfpathlineto{\pgfqpoint{3.853819in}{4.643829in}}%
\pgfpathlineto{\pgfqpoint{3.877234in}{4.630006in}}%
\pgfpathlineto{\pgfqpoint{3.920850in}{4.650579in}}%
\pgfpathlineto{\pgfqpoint{3.944919in}{4.651266in}}%
\pgfpathlineto{\pgfqpoint{3.973027in}{4.682624in}}%
\pgfpathlineto{\pgfqpoint{4.001625in}{4.701697in}}%
\pgfpathlineto{\pgfqpoint{4.039771in}{4.723670in}}%
\pgfpathlineto{\pgfqpoint{4.064326in}{4.570087in}}%
\pgfpathlineto{\pgfqpoint{4.052917in}{4.560224in}}%
\pgfpathlineto{\pgfqpoint{4.060283in}{4.551164in}}%
\pgfpathlineto{\pgfqpoint{4.063217in}{4.531303in}}%
\pgfpathlineto{\pgfqpoint{4.057501in}{4.512649in}}%
\pgfpathlineto{\pgfqpoint{4.056746in}{4.485383in}}%
\pgfpathlineto{\pgfqpoint{4.051378in}{4.449895in}}%
\pgfpathlineto{\pgfqpoint{4.024221in}{4.418243in}}%
\pgfpathlineto{\pgfqpoint{4.012789in}{4.411525in}}%
\pgfpathlineto{\pgfqpoint{4.003899in}{4.417309in}}%
\pgfpathlineto{\pgfqpoint{3.981897in}{4.387141in}}%
\pgfpathlineto{\pgfqpoint{3.983958in}{4.358094in}}%
\pgfpathlineto{\pgfqpoint{3.959456in}{4.365080in}}%
\pgfpathlineto{\pgfqpoint{3.947853in}{4.336069in}}%
\pgfpathlineto{\pgfqpoint{3.953718in}{4.315512in}}%
\pgfpathlineto{\pgfqpoint{3.944537in}{4.312475in}}%
\pgfpathlineto{\pgfqpoint{3.943252in}{4.296232in}}%
\pgfpathlineto{\pgfqpoint{3.920717in}{4.289680in}}%
\pgfpathlineto{\pgfqpoint{3.909006in}{4.302795in}}%
\pgfpathlineto{\pgfqpoint{3.893938in}{4.307880in}}%
\pgfpathlineto{\pgfqpoint{3.890398in}{4.321178in}}%
\pgfpathlineto{\pgfqpoint{3.876773in}{4.318839in}}%
\pgfpathlineto{\pgfqpoint{3.867848in}{4.306608in}}%
\pgfpathlineto{\pgfqpoint{3.855079in}{4.302307in}}%
\pgfpathlineto{\pgfqpoint{3.832524in}{4.310962in}}%
\pgfpathlineto{\pgfqpoint{3.820049in}{4.300664in}}%
\pgfpathlineto{\pgfqpoint{3.802320in}{4.312848in}}%
\pgfpathlineto{\pgfqpoint{3.768306in}{4.316588in}}%
\pgfpathlineto{\pgfqpoint{3.758047in}{4.338723in}}%
\pgfpathlineto{\pgfqpoint{3.745048in}{4.348530in}}%
\pgfpathlineto{\pgfqpoint{3.732447in}{4.342252in}}%
\pgfpathlineto{\pgfqpoint{3.712630in}{4.343743in}}%
\pgfusepath{stroke}%
\end{pgfscope}%
\begin{pgfscope}%
\pgfpathrectangle{\pgfqpoint{0.100000in}{2.413063in}}{\pgfqpoint{5.037500in}{3.427208in}}%
\pgfusepath{clip}%
\pgfsetbuttcap%
\pgfsetroundjoin%
\pgfsetlinewidth{0.501875pt}%
\definecolor{currentstroke}{rgb}{0.827451,0.827451,0.827451}%
\pgfsetstrokecolor{currentstroke}%
\pgfsetdash{}{0pt}%
\pgfpathmoveto{\pgfqpoint{3.344010in}{4.066681in}}%
\pgfpathlineto{\pgfqpoint{3.330964in}{4.077278in}}%
\pgfpathlineto{\pgfqpoint{3.320328in}{4.072236in}}%
\pgfpathlineto{\pgfqpoint{3.306729in}{4.097628in}}%
\pgfpathlineto{\pgfqpoint{3.313676in}{4.113588in}}%
\pgfpathlineto{\pgfqpoint{3.303664in}{4.131552in}}%
\pgfpathlineto{\pgfqpoint{3.303126in}{4.145694in}}%
\pgfpathlineto{\pgfqpoint{3.289597in}{4.150734in}}%
\pgfpathlineto{\pgfqpoint{3.283323in}{4.161390in}}%
\pgfpathlineto{\pgfqpoint{3.256870in}{4.174698in}}%
\pgfpathlineto{\pgfqpoint{3.233897in}{4.191095in}}%
\pgfpathlineto{\pgfqpoint{3.223222in}{4.203474in}}%
\pgfpathlineto{\pgfqpoint{3.223001in}{4.218571in}}%
\pgfpathlineto{\pgfqpoint{3.237285in}{4.247559in}}%
\pgfpathlineto{\pgfqpoint{3.241679in}{4.269731in}}%
\pgfpathlineto{\pgfqpoint{3.230042in}{4.282258in}}%
\pgfpathlineto{\pgfqpoint{3.214629in}{4.286979in}}%
\pgfpathlineto{\pgfqpoint{3.195930in}{4.276657in}}%
\pgfpathlineto{\pgfqpoint{3.188016in}{4.294377in}}%
\pgfpathlineto{\pgfqpoint{3.184022in}{4.318399in}}%
\pgfpathlineto{\pgfqpoint{3.156465in}{4.339812in}}%
\pgfpathlineto{\pgfqpoint{3.150959in}{4.349315in}}%
\pgfpathlineto{\pgfqpoint{3.125789in}{4.370826in}}%
\pgfpathlineto{\pgfqpoint{3.117900in}{4.386465in}}%
\pgfpathlineto{\pgfqpoint{3.110789in}{4.417479in}}%
\pgfpathlineto{\pgfqpoint{3.115663in}{4.445025in}}%
\pgfpathlineto{\pgfqpoint{3.122170in}{4.448866in}}%
\pgfpathlineto{\pgfqpoint{3.120999in}{4.471925in}}%
\pgfpathlineto{\pgfqpoint{3.139365in}{4.478775in}}%
\pgfpathlineto{\pgfqpoint{3.144911in}{4.499468in}}%
\pgfpathlineto{\pgfqpoint{3.155476in}{4.513396in}}%
\pgfpathlineto{\pgfqpoint{3.154942in}{4.531116in}}%
\pgfpathlineto{\pgfqpoint{3.141846in}{4.545209in}}%
\pgfpathlineto{\pgfqpoint{3.144957in}{4.564935in}}%
\pgfpathlineto{\pgfqpoint{3.178904in}{4.573514in}}%
\pgfpathlineto{\pgfqpoint{3.204948in}{4.589174in}}%
\pgfpathlineto{\pgfqpoint{3.207681in}{4.608891in}}%
\pgfpathlineto{\pgfqpoint{3.216743in}{4.615099in}}%
\pgfpathlineto{\pgfqpoint{3.220210in}{4.635813in}}%
\pgfpathlineto{\pgfqpoint{3.217424in}{4.649495in}}%
\pgfpathlineto{\pgfqpoint{3.199619in}{4.660860in}}%
\pgfpathlineto{\pgfqpoint{3.192451in}{4.677802in}}%
\pgfpathlineto{\pgfqpoint{3.174878in}{4.694179in}}%
\pgfpathlineto{\pgfqpoint{3.319282in}{4.700333in}}%
\pgfpathlineto{\pgfqpoint{3.416161in}{4.707215in}}%
\pgfpathlineto{\pgfqpoint{3.414389in}{4.686836in}}%
\pgfpathlineto{\pgfqpoint{3.430899in}{4.658727in}}%
\pgfpathlineto{\pgfqpoint{3.437831in}{4.634711in}}%
\pgfpathlineto{\pgfqpoint{3.446087in}{4.621075in}}%
\pgfpathlineto{\pgfqpoint{3.455099in}{4.508720in}}%
\pgfpathlineto{\pgfqpoint{3.467615in}{4.348563in}}%
\pgfpathlineto{\pgfqpoint{3.457959in}{4.324225in}}%
\pgfpathlineto{\pgfqpoint{3.471869in}{4.304221in}}%
\pgfpathlineto{\pgfqpoint{3.475746in}{4.285114in}}%
\pgfpathlineto{\pgfqpoint{3.456564in}{4.245496in}}%
\pgfpathlineto{\pgfqpoint{3.457695in}{4.240023in}}%
\pgfpathlineto{\pgfqpoint{3.437951in}{4.217314in}}%
\pgfpathlineto{\pgfqpoint{3.443366in}{4.210378in}}%
\pgfpathlineto{\pgfqpoint{3.435151in}{4.196223in}}%
\pgfpathlineto{\pgfqpoint{3.437265in}{4.167692in}}%
\pgfpathlineto{\pgfqpoint{3.427288in}{4.150184in}}%
\pgfpathlineto{\pgfqpoint{3.435402in}{4.129497in}}%
\pgfpathlineto{\pgfqpoint{3.401404in}{4.118255in}}%
\pgfpathlineto{\pgfqpoint{3.398271in}{4.106030in}}%
\pgfpathlineto{\pgfqpoint{3.407555in}{4.090534in}}%
\pgfpathlineto{\pgfqpoint{3.403283in}{4.080434in}}%
\pgfpathlineto{\pgfqpoint{3.366827in}{4.092910in}}%
\pgfpathlineto{\pgfqpoint{3.354799in}{4.094166in}}%
\pgfpathlineto{\pgfqpoint{3.339802in}{4.075199in}}%
\pgfpathlineto{\pgfqpoint{3.344010in}{4.066681in}}%
\pgfusepath{stroke}%
\end{pgfscope}%
\begin{pgfscope}%
\pgfpathrectangle{\pgfqpoint{0.100000in}{2.413063in}}{\pgfqpoint{5.037500in}{3.427208in}}%
\pgfusepath{clip}%
\pgfsetbuttcap%
\pgfsetroundjoin%
\pgfsetlinewidth{0.501875pt}%
\definecolor{currentstroke}{rgb}{0.827451,0.827451,0.827451}%
\pgfsetstrokecolor{currentstroke}%
\pgfsetdash{}{0pt}%
\pgfpathmoveto{\pgfqpoint{4.405194in}{4.415246in}}%
\pgfpathlineto{\pgfqpoint{4.395161in}{4.430143in}}%
\pgfpathlineto{\pgfqpoint{4.400822in}{4.438480in}}%
\pgfpathlineto{\pgfqpoint{4.414687in}{4.429117in}}%
\pgfpathlineto{\pgfqpoint{4.405194in}{4.415246in}}%
\pgfusepath{stroke}%
\end{pgfscope}%
\begin{pgfscope}%
\pgfpathrectangle{\pgfqpoint{0.100000in}{2.413063in}}{\pgfqpoint{5.037500in}{3.427208in}}%
\pgfusepath{clip}%
\pgfsetbuttcap%
\pgfsetroundjoin%
\pgfsetlinewidth{0.501875pt}%
\definecolor{currentstroke}{rgb}{0.827451,0.827451,0.827451}%
\pgfsetstrokecolor{currentstroke}%
\pgfsetdash{}{0pt}%
\pgfpathmoveto{\pgfqpoint{4.492802in}{4.543439in}}%
\pgfpathlineto{\pgfqpoint{4.498797in}{4.556107in}}%
\pgfpathlineto{\pgfqpoint{4.523034in}{4.558842in}}%
\pgfpathlineto{\pgfqpoint{4.519150in}{4.548078in}}%
\pgfpathlineto{\pgfqpoint{4.511154in}{4.534325in}}%
\pgfpathlineto{\pgfqpoint{4.516572in}{4.517941in}}%
\pgfpathlineto{\pgfqpoint{4.537956in}{4.498309in}}%
\pgfpathlineto{\pgfqpoint{4.542919in}{4.477636in}}%
\pgfpathlineto{\pgfqpoint{4.567556in}{4.451893in}}%
\pgfpathlineto{\pgfqpoint{4.577221in}{4.452995in}}%
\pgfpathlineto{\pgfqpoint{4.589261in}{4.414349in}}%
\pgfpathlineto{\pgfqpoint{4.587286in}{4.413964in}}%
\pgfpathlineto{\pgfqpoint{4.585091in}{4.413533in}}%
\pgfpathlineto{\pgfqpoint{4.531394in}{4.403253in}}%
\pgfpathlineto{\pgfqpoint{4.502665in}{4.505425in}}%
\pgfpathlineto{\pgfqpoint{4.492802in}{4.543439in}}%
\pgfusepath{stroke}%
\end{pgfscope}%
\begin{pgfscope}%
\pgfpathrectangle{\pgfqpoint{0.100000in}{2.413063in}}{\pgfqpoint{5.037500in}{3.427208in}}%
\pgfusepath{clip}%
\pgfsetbuttcap%
\pgfsetroundjoin%
\pgfsetlinewidth{0.501875pt}%
\definecolor{currentstroke}{rgb}{0.827451,0.827451,0.827451}%
\pgfsetstrokecolor{currentstroke}%
\pgfsetdash{}{0pt}%
\pgfpathmoveto{\pgfqpoint{3.991255in}{4.197116in}}%
\pgfpathlineto{\pgfqpoint{3.974287in}{4.197740in}}%
\pgfpathlineto{\pgfqpoint{3.958652in}{4.208510in}}%
\pgfpathlineto{\pgfqpoint{3.937530in}{4.241244in}}%
\pgfpathlineto{\pgfqpoint{3.919550in}{4.258580in}}%
\pgfpathlineto{\pgfqpoint{3.924249in}{4.271969in}}%
\pgfpathlineto{\pgfqpoint{3.920717in}{4.289680in}}%
\pgfpathlineto{\pgfqpoint{3.943252in}{4.296232in}}%
\pgfpathlineto{\pgfqpoint{3.944537in}{4.312475in}}%
\pgfpathlineto{\pgfqpoint{3.953718in}{4.315512in}}%
\pgfpathlineto{\pgfqpoint{3.947853in}{4.336069in}}%
\pgfpathlineto{\pgfqpoint{3.959456in}{4.365080in}}%
\pgfpathlineto{\pgfqpoint{3.983958in}{4.358094in}}%
\pgfpathlineto{\pgfqpoint{3.981897in}{4.387141in}}%
\pgfpathlineto{\pgfqpoint{4.003899in}{4.417309in}}%
\pgfpathlineto{\pgfqpoint{4.012789in}{4.411525in}}%
\pgfpathlineto{\pgfqpoint{4.024221in}{4.418243in}}%
\pgfpathlineto{\pgfqpoint{4.051378in}{4.449895in}}%
\pgfpathlineto{\pgfqpoint{4.056746in}{4.485383in}}%
\pgfpathlineto{\pgfqpoint{4.057501in}{4.512649in}}%
\pgfpathlineto{\pgfqpoint{4.063217in}{4.531303in}}%
\pgfpathlineto{\pgfqpoint{4.060283in}{4.551164in}}%
\pgfpathlineto{\pgfqpoint{4.052917in}{4.560224in}}%
\pgfpathlineto{\pgfqpoint{4.064326in}{4.570087in}}%
\pgfpathlineto{\pgfqpoint{4.080863in}{4.465897in}}%
\pgfpathlineto{\pgfqpoint{4.172033in}{4.480971in}}%
\pgfpathlineto{\pgfqpoint{4.181483in}{4.421503in}}%
\pgfpathlineto{\pgfqpoint{4.214489in}{4.460755in}}%
\pgfpathlineto{\pgfqpoint{4.222266in}{4.456841in}}%
\pgfpathlineto{\pgfqpoint{4.233707in}{4.479585in}}%
\pgfpathlineto{\pgfqpoint{4.249557in}{4.472454in}}%
\pgfpathlineto{\pgfqpoint{4.263377in}{4.473810in}}%
\pgfpathlineto{\pgfqpoint{4.272494in}{4.489620in}}%
\pgfpathlineto{\pgfqpoint{4.285729in}{4.498416in}}%
\pgfpathlineto{\pgfqpoint{4.304128in}{4.490670in}}%
\pgfpathlineto{\pgfqpoint{4.316235in}{4.493347in}}%
\pgfpathlineto{\pgfqpoint{4.333579in}{4.463330in}}%
\pgfpathlineto{\pgfqpoint{4.328563in}{4.440605in}}%
\pgfpathlineto{\pgfqpoint{4.283263in}{4.466191in}}%
\pgfpathlineto{\pgfqpoint{4.274778in}{4.445187in}}%
\pgfpathlineto{\pgfqpoint{4.277577in}{4.435523in}}%
\pgfpathlineto{\pgfqpoint{4.258297in}{4.402854in}}%
\pgfpathlineto{\pgfqpoint{4.249198in}{4.396356in}}%
\pgfpathlineto{\pgfqpoint{4.245088in}{4.381950in}}%
\pgfpathlineto{\pgfqpoint{4.229575in}{4.383457in}}%
\pgfpathlineto{\pgfqpoint{4.218435in}{4.344138in}}%
\pgfpathlineto{\pgfqpoint{4.212199in}{4.335115in}}%
\pgfpathlineto{\pgfqpoint{4.196175in}{4.338124in}}%
\pgfpathlineto{\pgfqpoint{4.179784in}{4.350525in}}%
\pgfpathlineto{\pgfqpoint{4.179217in}{4.331491in}}%
\pgfpathlineto{\pgfqpoint{4.163350in}{4.299502in}}%
\pgfpathlineto{\pgfqpoint{4.159403in}{4.276729in}}%
\pgfpathlineto{\pgfqpoint{4.147229in}{4.261562in}}%
\pgfpathlineto{\pgfqpoint{4.137999in}{4.237341in}}%
\pgfpathlineto{\pgfqpoint{4.137545in}{4.214919in}}%
\pgfpathlineto{\pgfqpoint{4.123291in}{4.210732in}}%
\pgfpathlineto{\pgfqpoint{4.107121in}{4.198039in}}%
\pgfpathlineto{\pgfqpoint{4.091551in}{4.195621in}}%
\pgfpathlineto{\pgfqpoint{4.087963in}{4.184878in}}%
\pgfpathlineto{\pgfqpoint{4.062876in}{4.173876in}}%
\pgfpathlineto{\pgfqpoint{4.048856in}{4.183250in}}%
\pgfpathlineto{\pgfqpoint{4.033186in}{4.165450in}}%
\pgfpathlineto{\pgfqpoint{4.006250in}{4.170697in}}%
\pgfpathlineto{\pgfqpoint{3.989729in}{4.189379in}}%
\pgfpathlineto{\pgfqpoint{3.991255in}{4.197116in}}%
\pgfusepath{stroke}%
\end{pgfscope}%
\begin{pgfscope}%
\pgfpathrectangle{\pgfqpoint{0.100000in}{2.413063in}}{\pgfqpoint{5.037500in}{3.427208in}}%
\pgfusepath{clip}%
\pgfsetbuttcap%
\pgfsetroundjoin%
\pgfsetlinewidth{0.501875pt}%
\definecolor{currentstroke}{rgb}{0.827451,0.827451,0.827451}%
\pgfsetstrokecolor{currentstroke}%
\pgfsetdash{}{0pt}%
\pgfpathmoveto{\pgfqpoint{4.585091in}{4.413533in}}%
\pgfpathlineto{\pgfqpoint{4.584403in}{4.392537in}}%
\pgfpathlineto{\pgfqpoint{4.576322in}{4.382215in}}%
\pgfpathlineto{\pgfqpoint{4.571165in}{4.359301in}}%
\pgfpathlineto{\pgfqpoint{4.547880in}{4.348739in}}%
\pgfpathlineto{\pgfqpoint{4.528343in}{4.345661in}}%
\pgfpathlineto{\pgfqpoint{4.534036in}{4.360731in}}%
\pgfpathlineto{\pgfqpoint{4.503955in}{4.373358in}}%
\pgfpathlineto{\pgfqpoint{4.479516in}{4.389169in}}%
\pgfpathlineto{\pgfqpoint{4.494646in}{4.412802in}}%
\pgfpathlineto{\pgfqpoint{4.482520in}{4.431173in}}%
\pgfpathlineto{\pgfqpoint{4.483948in}{4.447018in}}%
\pgfpathlineto{\pgfqpoint{4.475180in}{4.451368in}}%
\pgfpathlineto{\pgfqpoint{4.468066in}{4.477111in}}%
\pgfpathlineto{\pgfqpoint{4.484890in}{4.512009in}}%
\pgfpathlineto{\pgfqpoint{4.472250in}{4.517708in}}%
\pgfpathlineto{\pgfqpoint{4.468974in}{4.500491in}}%
\pgfpathlineto{\pgfqpoint{4.450931in}{4.495698in}}%
\pgfpathlineto{\pgfqpoint{4.452828in}{4.439052in}}%
\pgfpathlineto{\pgfqpoint{4.449529in}{4.420822in}}%
\pgfpathlineto{\pgfqpoint{4.458579in}{4.394839in}}%
\pgfpathlineto{\pgfqpoint{4.472502in}{4.382338in}}%
\pgfpathlineto{\pgfqpoint{4.466188in}{4.374488in}}%
\pgfpathlineto{\pgfqpoint{4.480400in}{4.363018in}}%
\pgfpathlineto{\pgfqpoint{4.485537in}{4.344401in}}%
\pgfpathlineto{\pgfqpoint{4.459488in}{4.359818in}}%
\pgfpathlineto{\pgfqpoint{4.443002in}{4.357815in}}%
\pgfpathlineto{\pgfqpoint{4.430202in}{4.373674in}}%
\pgfpathlineto{\pgfqpoint{4.417168in}{4.375219in}}%
\pgfpathlineto{\pgfqpoint{4.398644in}{4.367275in}}%
\pgfpathlineto{\pgfqpoint{4.391460in}{4.377174in}}%
\pgfpathlineto{\pgfqpoint{4.400902in}{4.397950in}}%
\pgfpathlineto{\pgfqpoint{4.405194in}{4.415246in}}%
\pgfpathlineto{\pgfqpoint{4.414687in}{4.429117in}}%
\pgfpathlineto{\pgfqpoint{4.400822in}{4.438480in}}%
\pgfpathlineto{\pgfqpoint{4.395161in}{4.430143in}}%
\pgfpathlineto{\pgfqpoint{4.381310in}{4.438606in}}%
\pgfpathlineto{\pgfqpoint{4.356785in}{4.444244in}}%
\pgfpathlineto{\pgfqpoint{4.359009in}{4.456642in}}%
\pgfpathlineto{\pgfqpoint{4.347892in}{4.463833in}}%
\pgfpathlineto{\pgfqpoint{4.333579in}{4.463330in}}%
\pgfpathlineto{\pgfqpoint{4.316235in}{4.493347in}}%
\pgfpathlineto{\pgfqpoint{4.304128in}{4.490670in}}%
\pgfpathlineto{\pgfqpoint{4.285729in}{4.498416in}}%
\pgfpathlineto{\pgfqpoint{4.272494in}{4.489620in}}%
\pgfpathlineto{\pgfqpoint{4.263377in}{4.473810in}}%
\pgfpathlineto{\pgfqpoint{4.249557in}{4.472454in}}%
\pgfpathlineto{\pgfqpoint{4.233707in}{4.479585in}}%
\pgfpathlineto{\pgfqpoint{4.222266in}{4.456841in}}%
\pgfpathlineto{\pgfqpoint{4.214489in}{4.460755in}}%
\pgfpathlineto{\pgfqpoint{4.181483in}{4.421503in}}%
\pgfpathlineto{\pgfqpoint{4.172033in}{4.480971in}}%
\pgfpathlineto{\pgfqpoint{4.292651in}{4.503327in}}%
\pgfpathlineto{\pgfqpoint{4.346722in}{4.513009in}}%
\pgfpathlineto{\pgfqpoint{4.425387in}{4.528852in}}%
\pgfpathlineto{\pgfqpoint{4.492802in}{4.543439in}}%
\pgfpathlineto{\pgfqpoint{4.502665in}{4.505425in}}%
\pgfpathlineto{\pgfqpoint{4.531394in}{4.403253in}}%
\pgfpathlineto{\pgfqpoint{4.585091in}{4.413533in}}%
\pgfusepath{stroke}%
\end{pgfscope}%
\begin{pgfscope}%
\pgfpathrectangle{\pgfqpoint{0.100000in}{2.413063in}}{\pgfqpoint{5.037500in}{3.427208in}}%
\pgfusepath{clip}%
\pgfsetbuttcap%
\pgfsetroundjoin%
\pgfsetlinewidth{0.501875pt}%
\definecolor{currentstroke}{rgb}{0.827451,0.827451,0.827451}%
\pgfsetstrokecolor{currentstroke}%
\pgfsetdash{}{0pt}%
\pgfpathmoveto{\pgfqpoint{2.157107in}{4.074398in}}%
\pgfpathlineto{\pgfqpoint{2.069106in}{4.082796in}}%
\pgfpathlineto{\pgfqpoint{1.977813in}{4.090887in}}%
\pgfpathlineto{\pgfqpoint{1.872319in}{4.101891in}}%
\pgfpathlineto{\pgfqpoint{1.715584in}{4.120532in}}%
\pgfpathlineto{\pgfqpoint{1.659981in}{4.129100in}}%
\pgfpathlineto{\pgfqpoint{1.516457in}{4.149943in}}%
\pgfpathlineto{\pgfqpoint{1.537414in}{4.281477in}}%
\pgfpathlineto{\pgfqpoint{1.537936in}{4.292117in}}%
\pgfpathlineto{\pgfqpoint{1.558125in}{4.419035in}}%
\pgfpathlineto{\pgfqpoint{1.573205in}{4.515697in}}%
\pgfpathlineto{\pgfqpoint{1.587436in}{4.605334in}}%
\pgfpathlineto{\pgfqpoint{1.684663in}{4.591377in}}%
\pgfpathlineto{\pgfqpoint{1.775315in}{4.578367in}}%
\pgfpathlineto{\pgfqpoint{1.941955in}{4.557799in}}%
\pgfpathlineto{\pgfqpoint{2.018405in}{4.550818in}}%
\pgfpathlineto{\pgfqpoint{2.139519in}{4.539067in}}%
\pgfpathlineto{\pgfqpoint{2.191913in}{4.534814in}}%
\pgfpathlineto{\pgfqpoint{2.182662in}{4.420100in}}%
\pgfpathlineto{\pgfqpoint{2.174305in}{4.309649in}}%
\pgfpathlineto{\pgfqpoint{2.167579in}{4.219715in}}%
\pgfpathlineto{\pgfqpoint{2.157107in}{4.074398in}}%
\pgfusepath{stroke}%
\end{pgfscope}%
\begin{pgfscope}%
\pgfpathrectangle{\pgfqpoint{0.100000in}{2.413063in}}{\pgfqpoint{5.037500in}{3.427208in}}%
\pgfusepath{clip}%
\pgfsetbuttcap%
\pgfsetroundjoin%
\pgfsetlinewidth{0.501875pt}%
\definecolor{currentstroke}{rgb}{0.827451,0.827451,0.827451}%
\pgfsetstrokecolor{currentstroke}%
\pgfsetdash{}{0pt}%
\pgfpathmoveto{\pgfqpoint{3.321282in}{4.009187in}}%
\pgfpathlineto{\pgfqpoint{3.324248in}{4.022485in}}%
\pgfpathlineto{\pgfqpoint{3.339624in}{4.019492in}}%
\pgfpathlineto{\pgfqpoint{3.344010in}{4.066681in}}%
\pgfpathlineto{\pgfqpoint{3.339802in}{4.075199in}}%
\pgfpathlineto{\pgfqpoint{3.354799in}{4.094166in}}%
\pgfpathlineto{\pgfqpoint{3.366827in}{4.092910in}}%
\pgfpathlineto{\pgfqpoint{3.403283in}{4.080434in}}%
\pgfpathlineto{\pgfqpoint{3.407555in}{4.090534in}}%
\pgfpathlineto{\pgfqpoint{3.398271in}{4.106030in}}%
\pgfpathlineto{\pgfqpoint{3.401404in}{4.118255in}}%
\pgfpathlineto{\pgfqpoint{3.435402in}{4.129497in}}%
\pgfpathlineto{\pgfqpoint{3.427288in}{4.150184in}}%
\pgfpathlineto{\pgfqpoint{3.437265in}{4.167692in}}%
\pgfpathlineto{\pgfqpoint{3.455836in}{4.177519in}}%
\pgfpathlineto{\pgfqpoint{3.494803in}{4.187250in}}%
\pgfpathlineto{\pgfqpoint{3.519389in}{4.172561in}}%
\pgfpathlineto{\pgfqpoint{3.526847in}{4.186886in}}%
\pgfpathlineto{\pgfqpoint{3.547262in}{4.196892in}}%
\pgfpathlineto{\pgfqpoint{3.552531in}{4.188067in}}%
\pgfpathlineto{\pgfqpoint{3.573500in}{4.195078in}}%
\pgfpathlineto{\pgfqpoint{3.572178in}{4.207078in}}%
\pgfpathlineto{\pgfqpoint{3.591032in}{4.220810in}}%
\pgfpathlineto{\pgfqpoint{3.602160in}{4.206550in}}%
\pgfpathlineto{\pgfqpoint{3.616798in}{4.205115in}}%
\pgfpathlineto{\pgfqpoint{3.626527in}{4.214409in}}%
\pgfpathlineto{\pgfqpoint{3.625445in}{4.227626in}}%
\pgfpathlineto{\pgfqpoint{3.633714in}{4.240750in}}%
\pgfpathlineto{\pgfqpoint{3.644795in}{4.243592in}}%
\pgfpathlineto{\pgfqpoint{3.649244in}{4.260922in}}%
\pgfpathlineto{\pgfqpoint{3.665360in}{4.275902in}}%
\pgfpathlineto{\pgfqpoint{3.660509in}{4.290813in}}%
\pgfpathlineto{\pgfqpoint{3.676188in}{4.298237in}}%
\pgfpathlineto{\pgfqpoint{3.686663in}{4.293671in}}%
\pgfpathlineto{\pgfqpoint{3.702135in}{4.305269in}}%
\pgfpathlineto{\pgfqpoint{3.715962in}{4.308292in}}%
\pgfpathlineto{\pgfqpoint{3.716011in}{4.320323in}}%
\pgfpathlineto{\pgfqpoint{3.706322in}{4.337017in}}%
\pgfpathlineto{\pgfqpoint{3.712630in}{4.343743in}}%
\pgfpathlineto{\pgfqpoint{3.732447in}{4.342252in}}%
\pgfpathlineto{\pgfqpoint{3.745048in}{4.348530in}}%
\pgfpathlineto{\pgfqpoint{3.758047in}{4.338723in}}%
\pgfpathlineto{\pgfqpoint{3.768306in}{4.316588in}}%
\pgfpathlineto{\pgfqpoint{3.802320in}{4.312848in}}%
\pgfpathlineto{\pgfqpoint{3.820049in}{4.300664in}}%
\pgfpathlineto{\pgfqpoint{3.832524in}{4.310962in}}%
\pgfpathlineto{\pgfqpoint{3.855079in}{4.302307in}}%
\pgfpathlineto{\pgfqpoint{3.867848in}{4.306608in}}%
\pgfpathlineto{\pgfqpoint{3.876773in}{4.318839in}}%
\pgfpathlineto{\pgfqpoint{3.890398in}{4.321178in}}%
\pgfpathlineto{\pgfqpoint{3.893938in}{4.307880in}}%
\pgfpathlineto{\pgfqpoint{3.909006in}{4.302795in}}%
\pgfpathlineto{\pgfqpoint{3.920717in}{4.289680in}}%
\pgfpathlineto{\pgfqpoint{3.924249in}{4.271969in}}%
\pgfpathlineto{\pgfqpoint{3.919550in}{4.258580in}}%
\pgfpathlineto{\pgfqpoint{3.937530in}{4.241244in}}%
\pgfpathlineto{\pgfqpoint{3.958652in}{4.208510in}}%
\pgfpathlineto{\pgfqpoint{3.974287in}{4.197740in}}%
\pgfpathlineto{\pgfqpoint{3.991255in}{4.197116in}}%
\pgfpathlineto{\pgfqpoint{3.959924in}{4.161166in}}%
\pgfpathlineto{\pgfqpoint{3.929084in}{4.139406in}}%
\pgfpathlineto{\pgfqpoint{3.917734in}{4.122126in}}%
\pgfpathlineto{\pgfqpoint{3.917926in}{4.112748in}}%
\pgfpathlineto{\pgfqpoint{3.901222in}{4.105560in}}%
\pgfpathlineto{\pgfqpoint{3.896435in}{4.092036in}}%
\pgfpathlineto{\pgfqpoint{3.861637in}{4.078413in}}%
\pgfpathlineto{\pgfqpoint{3.849302in}{4.069566in}}%
\pgfpathlineto{\pgfqpoint{3.847633in}{4.067678in}}%
\pgfpathlineto{\pgfqpoint{3.747536in}{4.058186in}}%
\pgfpathlineto{\pgfqpoint{3.687088in}{4.053103in}}%
\pgfpathlineto{\pgfqpoint{3.587896in}{4.047489in}}%
\pgfpathlineto{\pgfqpoint{3.464320in}{4.035264in}}%
\pgfpathlineto{\pgfqpoint{3.443904in}{4.038092in}}%
\pgfpathlineto{\pgfqpoint{3.448152in}{4.017250in}}%
\pgfpathlineto{\pgfqpoint{3.321282in}{4.009187in}}%
\pgfusepath{stroke}%
\end{pgfscope}%
\begin{pgfscope}%
\pgfpathrectangle{\pgfqpoint{0.100000in}{2.413063in}}{\pgfqpoint{5.037500in}{3.427208in}}%
\pgfusepath{clip}%
\pgfsetbuttcap%
\pgfsetroundjoin%
\pgfsetlinewidth{0.501875pt}%
\definecolor{currentstroke}{rgb}{0.827451,0.827451,0.827451}%
\pgfsetstrokecolor{currentstroke}%
\pgfsetdash{}{0pt}%
\pgfpathmoveto{\pgfqpoint{2.157107in}{4.074398in}}%
\pgfpathlineto{\pgfqpoint{2.167579in}{4.219715in}}%
\pgfpathlineto{\pgfqpoint{2.174305in}{4.309649in}}%
\pgfpathlineto{\pgfqpoint{2.182662in}{4.420100in}}%
\pgfpathlineto{\pgfqpoint{2.298517in}{4.411965in}}%
\pgfpathlineto{\pgfqpoint{2.445651in}{4.403928in}}%
\pgfpathlineto{\pgfqpoint{2.545721in}{4.400088in}}%
\pgfpathlineto{\pgfqpoint{2.645206in}{4.397040in}}%
\pgfpathlineto{\pgfqpoint{2.735250in}{4.395647in}}%
\pgfpathlineto{\pgfqpoint{2.776895in}{4.396079in}}%
\pgfpathlineto{\pgfqpoint{2.795226in}{4.381120in}}%
\pgfpathlineto{\pgfqpoint{2.809584in}{4.384135in}}%
\pgfpathlineto{\pgfqpoint{2.810034in}{4.371087in}}%
\pgfpathlineto{\pgfqpoint{2.799370in}{4.348517in}}%
\pgfpathlineto{\pgfqpoint{2.800527in}{4.334257in}}%
\pgfpathlineto{\pgfqpoint{2.810737in}{4.324854in}}%
\pgfpathlineto{\pgfqpoint{2.820114in}{4.305262in}}%
\pgfpathlineto{\pgfqpoint{2.839138in}{4.294013in}}%
\pgfpathlineto{\pgfqpoint{2.838503in}{4.220162in}}%
\pgfpathlineto{\pgfqpoint{2.839063in}{4.050335in}}%
\pgfpathlineto{\pgfqpoint{2.711550in}{4.050923in}}%
\pgfpathlineto{\pgfqpoint{2.577276in}{4.053273in}}%
\pgfpathlineto{\pgfqpoint{2.478429in}{4.056760in}}%
\pgfpathlineto{\pgfqpoint{2.395919in}{4.059944in}}%
\pgfpathlineto{\pgfqpoint{2.256916in}{4.068140in}}%
\pgfpathlineto{\pgfqpoint{2.157107in}{4.074398in}}%
\pgfusepath{stroke}%
\end{pgfscope}%
\begin{pgfscope}%
\pgfpathrectangle{\pgfqpoint{0.100000in}{2.413063in}}{\pgfqpoint{5.037500in}{3.427208in}}%
\pgfusepath{clip}%
\pgfsetbuttcap%
\pgfsetroundjoin%
\pgfsetlinewidth{0.501875pt}%
\definecolor{currentstroke}{rgb}{0.827451,0.827451,0.827451}%
\pgfsetstrokecolor{currentstroke}%
\pgfsetdash{}{0pt}%
\pgfpathmoveto{\pgfqpoint{4.033298in}{4.093639in}}%
\pgfpathlineto{\pgfqpoint{3.892070in}{4.073801in}}%
\pgfpathlineto{\pgfqpoint{3.849302in}{4.069566in}}%
\pgfpathlineto{\pgfqpoint{3.861637in}{4.078413in}}%
\pgfpathlineto{\pgfqpoint{3.896435in}{4.092036in}}%
\pgfpathlineto{\pgfqpoint{3.901222in}{4.105560in}}%
\pgfpathlineto{\pgfqpoint{3.917926in}{4.112748in}}%
\pgfpathlineto{\pgfqpoint{3.917734in}{4.122126in}}%
\pgfpathlineto{\pgfqpoint{3.929084in}{4.139406in}}%
\pgfpathlineto{\pgfqpoint{3.959924in}{4.161166in}}%
\pgfpathlineto{\pgfqpoint{3.991255in}{4.197116in}}%
\pgfpathlineto{\pgfqpoint{3.989729in}{4.189379in}}%
\pgfpathlineto{\pgfqpoint{4.006250in}{4.170697in}}%
\pgfpathlineto{\pgfqpoint{4.033186in}{4.165450in}}%
\pgfpathlineto{\pgfqpoint{4.048856in}{4.183250in}}%
\pgfpathlineto{\pgfqpoint{4.062876in}{4.173876in}}%
\pgfpathlineto{\pgfqpoint{4.087963in}{4.184878in}}%
\pgfpathlineto{\pgfqpoint{4.091551in}{4.195621in}}%
\pgfpathlineto{\pgfqpoint{4.107121in}{4.198039in}}%
\pgfpathlineto{\pgfqpoint{4.123291in}{4.210732in}}%
\pgfpathlineto{\pgfqpoint{4.137545in}{4.214919in}}%
\pgfpathlineto{\pgfqpoint{4.137999in}{4.237341in}}%
\pgfpathlineto{\pgfqpoint{4.147229in}{4.261562in}}%
\pgfpathlineto{\pgfqpoint{4.159403in}{4.276729in}}%
\pgfpathlineto{\pgfqpoint{4.163350in}{4.299502in}}%
\pgfpathlineto{\pgfqpoint{4.179217in}{4.331491in}}%
\pgfpathlineto{\pgfqpoint{4.179784in}{4.350525in}}%
\pgfpathlineto{\pgfqpoint{4.196175in}{4.338124in}}%
\pgfpathlineto{\pgfqpoint{4.212199in}{4.335115in}}%
\pgfpathlineto{\pgfqpoint{4.218435in}{4.344138in}}%
\pgfpathlineto{\pgfqpoint{4.229575in}{4.383457in}}%
\pgfpathlineto{\pgfqpoint{4.245088in}{4.381950in}}%
\pgfpathlineto{\pgfqpoint{4.249198in}{4.396356in}}%
\pgfpathlineto{\pgfqpoint{4.258297in}{4.402854in}}%
\pgfpathlineto{\pgfqpoint{4.277577in}{4.435523in}}%
\pgfpathlineto{\pgfqpoint{4.274778in}{4.445187in}}%
\pgfpathlineto{\pgfqpoint{4.283263in}{4.466191in}}%
\pgfpathlineto{\pgfqpoint{4.328563in}{4.440605in}}%
\pgfpathlineto{\pgfqpoint{4.333579in}{4.463330in}}%
\pgfpathlineto{\pgfqpoint{4.347892in}{4.463833in}}%
\pgfpathlineto{\pgfqpoint{4.359009in}{4.456642in}}%
\pgfpathlineto{\pgfqpoint{4.356785in}{4.444244in}}%
\pgfpathlineto{\pgfqpoint{4.381310in}{4.438606in}}%
\pgfpathlineto{\pgfqpoint{4.395161in}{4.430143in}}%
\pgfpathlineto{\pgfqpoint{4.405194in}{4.415246in}}%
\pgfpathlineto{\pgfqpoint{4.400902in}{4.397950in}}%
\pgfpathlineto{\pgfqpoint{4.392234in}{4.396533in}}%
\pgfpathlineto{\pgfqpoint{4.387208in}{4.370435in}}%
\pgfpathlineto{\pgfqpoint{4.398219in}{4.360231in}}%
\pgfpathlineto{\pgfqpoint{4.413719in}{4.368479in}}%
\pgfpathlineto{\pgfqpoint{4.428102in}{4.351062in}}%
\pgfpathlineto{\pgfqpoint{4.460236in}{4.347931in}}%
\pgfpathlineto{\pgfqpoint{4.465772in}{4.337912in}}%
\pgfpathlineto{\pgfqpoint{4.495510in}{4.328168in}}%
\pgfpathlineto{\pgfqpoint{4.491840in}{4.316670in}}%
\pgfpathlineto{\pgfqpoint{4.495709in}{4.284173in}}%
\pgfpathlineto{\pgfqpoint{4.507712in}{4.271902in}}%
\pgfpathlineto{\pgfqpoint{4.490000in}{4.266722in}}%
\pgfpathlineto{\pgfqpoint{4.492348in}{4.252849in}}%
\pgfpathlineto{\pgfqpoint{4.511242in}{4.241087in}}%
\pgfpathlineto{\pgfqpoint{4.512946in}{4.229484in}}%
\pgfpathlineto{\pgfqpoint{4.502278in}{4.220765in}}%
\pgfpathlineto{\pgfqpoint{4.480755in}{4.241421in}}%
\pgfpathlineto{\pgfqpoint{4.478627in}{4.226350in}}%
\pgfpathlineto{\pgfqpoint{4.496645in}{4.219183in}}%
\pgfpathlineto{\pgfqpoint{4.498448in}{4.211772in}}%
\pgfpathlineto{\pgfqpoint{4.523162in}{4.221567in}}%
\pgfpathlineto{\pgfqpoint{4.542100in}{4.224158in}}%
\pgfpathlineto{\pgfqpoint{4.561500in}{4.185016in}}%
\pgfpathlineto{\pgfqpoint{4.559336in}{4.184593in}}%
\pgfpathlineto{\pgfqpoint{4.550569in}{4.182781in}}%
\pgfpathlineto{\pgfqpoint{4.547986in}{4.182241in}}%
\pgfpathlineto{\pgfqpoint{4.546278in}{4.181907in}}%
\pgfpathlineto{\pgfqpoint{4.430773in}{4.157948in}}%
\pgfpathlineto{\pgfqpoint{4.327461in}{4.136802in}}%
\pgfpathlineto{\pgfqpoint{4.184620in}{4.112195in}}%
\pgfpathlineto{\pgfqpoint{4.063306in}{4.096172in}}%
\pgfpathlineto{\pgfqpoint{4.033298in}{4.093639in}}%
\pgfusepath{stroke}%
\end{pgfscope}%
\begin{pgfscope}%
\pgfpathrectangle{\pgfqpoint{0.100000in}{2.413063in}}{\pgfqpoint{5.037500in}{3.427208in}}%
\pgfusepath{clip}%
\pgfsetbuttcap%
\pgfsetroundjoin%
\pgfsetlinewidth{0.501875pt}%
\definecolor{currentstroke}{rgb}{0.827451,0.827451,0.827451}%
\pgfsetstrokecolor{currentstroke}%
\pgfsetdash{}{0pt}%
\pgfpathmoveto{\pgfqpoint{4.547880in}{4.348739in}}%
\pgfpathlineto{\pgfqpoint{4.571165in}{4.359301in}}%
\pgfpathlineto{\pgfqpoint{4.552570in}{4.304904in}}%
\pgfpathlineto{\pgfqpoint{4.540273in}{4.276098in}}%
\pgfpathlineto{\pgfqpoint{4.531414in}{4.292400in}}%
\pgfpathlineto{\pgfqpoint{4.547168in}{4.331428in}}%
\pgfpathlineto{\pgfqpoint{4.547880in}{4.348739in}}%
\pgfusepath{stroke}%
\end{pgfscope}%
\begin{pgfscope}%
\pgfpathrectangle{\pgfqpoint{0.100000in}{2.413063in}}{\pgfqpoint{5.037500in}{3.427208in}}%
\pgfusepath{clip}%
\pgfsetbuttcap%
\pgfsetroundjoin%
\pgfsetlinewidth{0.501875pt}%
\definecolor{currentstroke}{rgb}{0.827451,0.827451,0.827451}%
\pgfsetstrokecolor{currentstroke}%
\pgfsetdash{}{0pt}%
\pgfpathmoveto{\pgfqpoint{3.321282in}{4.009187in}}%
\pgfpathlineto{\pgfqpoint{3.315658in}{4.008373in}}%
\pgfpathlineto{\pgfqpoint{3.310355in}{4.008001in}}%
\pgfpathlineto{\pgfqpoint{3.312625in}{3.991720in}}%
\pgfpathlineto{\pgfqpoint{3.298944in}{3.975308in}}%
\pgfpathlineto{\pgfqpoint{3.296248in}{3.949628in}}%
\pgfpathlineto{\pgfqpoint{3.235087in}{3.945104in}}%
\pgfpathlineto{\pgfqpoint{3.240418in}{3.957162in}}%
\pgfpathlineto{\pgfqpoint{3.262470in}{3.979183in}}%
\pgfpathlineto{\pgfqpoint{3.263080in}{3.991948in}}%
\pgfpathlineto{\pgfqpoint{3.253318in}{4.004033in}}%
\pgfpathlineto{\pgfqpoint{3.162287in}{3.999219in}}%
\pgfpathlineto{\pgfqpoint{3.003135in}{3.994167in}}%
\pgfpathlineto{\pgfqpoint{2.909997in}{3.992469in}}%
\pgfpathlineto{\pgfqpoint{2.839596in}{3.991836in}}%
\pgfpathlineto{\pgfqpoint{2.839063in}{4.050335in}}%
\pgfpathlineto{\pgfqpoint{2.838503in}{4.220162in}}%
\pgfpathlineto{\pgfqpoint{2.839138in}{4.294013in}}%
\pgfpathlineto{\pgfqpoint{2.820114in}{4.305262in}}%
\pgfpathlineto{\pgfqpoint{2.810737in}{4.324854in}}%
\pgfpathlineto{\pgfqpoint{2.800527in}{4.334257in}}%
\pgfpathlineto{\pgfqpoint{2.799370in}{4.348517in}}%
\pgfpathlineto{\pgfqpoint{2.810034in}{4.371087in}}%
\pgfpathlineto{\pgfqpoint{2.809584in}{4.384135in}}%
\pgfpathlineto{\pgfqpoint{2.795226in}{4.381120in}}%
\pgfpathlineto{\pgfqpoint{2.776895in}{4.396079in}}%
\pgfpathlineto{\pgfqpoint{2.762201in}{4.422337in}}%
\pgfpathlineto{\pgfqpoint{2.749875in}{4.434454in}}%
\pgfpathlineto{\pgfqpoint{2.736996in}{4.464227in}}%
\pgfpathlineto{\pgfqpoint{2.870765in}{4.462140in}}%
\pgfpathlineto{\pgfqpoint{3.003734in}{4.466521in}}%
\pgfpathlineto{\pgfqpoint{3.088992in}{4.471439in}}%
\pgfpathlineto{\pgfqpoint{3.115663in}{4.445025in}}%
\pgfpathlineto{\pgfqpoint{3.110789in}{4.417479in}}%
\pgfpathlineto{\pgfqpoint{3.117900in}{4.386465in}}%
\pgfpathlineto{\pgfqpoint{3.125789in}{4.370826in}}%
\pgfpathlineto{\pgfqpoint{3.150959in}{4.349315in}}%
\pgfpathlineto{\pgfqpoint{3.156465in}{4.339812in}}%
\pgfpathlineto{\pgfqpoint{3.184022in}{4.318399in}}%
\pgfpathlineto{\pgfqpoint{3.188016in}{4.294377in}}%
\pgfpathlineto{\pgfqpoint{3.195930in}{4.276657in}}%
\pgfpathlineto{\pgfqpoint{3.214629in}{4.286979in}}%
\pgfpathlineto{\pgfqpoint{3.230042in}{4.282258in}}%
\pgfpathlineto{\pgfqpoint{3.241679in}{4.269731in}}%
\pgfpathlineto{\pgfqpoint{3.237285in}{4.247559in}}%
\pgfpathlineto{\pgfqpoint{3.223001in}{4.218571in}}%
\pgfpathlineto{\pgfqpoint{3.223222in}{4.203474in}}%
\pgfpathlineto{\pgfqpoint{3.233897in}{4.191095in}}%
\pgfpathlineto{\pgfqpoint{3.256870in}{4.174698in}}%
\pgfpathlineto{\pgfqpoint{3.283323in}{4.161390in}}%
\pgfpathlineto{\pgfqpoint{3.289597in}{4.150734in}}%
\pgfpathlineto{\pgfqpoint{3.303126in}{4.145694in}}%
\pgfpathlineto{\pgfqpoint{3.303664in}{4.131552in}}%
\pgfpathlineto{\pgfqpoint{3.313676in}{4.113588in}}%
\pgfpathlineto{\pgfqpoint{3.306729in}{4.097628in}}%
\pgfpathlineto{\pgfqpoint{3.320328in}{4.072236in}}%
\pgfpathlineto{\pgfqpoint{3.330964in}{4.077278in}}%
\pgfpathlineto{\pgfqpoint{3.344010in}{4.066681in}}%
\pgfpathlineto{\pgfqpoint{3.339624in}{4.019492in}}%
\pgfpathlineto{\pgfqpoint{3.324248in}{4.022485in}}%
\pgfpathlineto{\pgfqpoint{3.321282in}{4.009187in}}%
\pgfusepath{stroke}%
\end{pgfscope}%
\begin{pgfscope}%
\pgfpathrectangle{\pgfqpoint{0.100000in}{2.413063in}}{\pgfqpoint{5.037500in}{3.427208in}}%
\pgfusepath{clip}%
\pgfsetbuttcap%
\pgfsetroundjoin%
\pgfsetlinewidth{0.501875pt}%
\definecolor{currentstroke}{rgb}{0.827451,0.827451,0.827451}%
\pgfsetstrokecolor{currentstroke}%
\pgfsetdash{}{0pt}%
\pgfpathmoveto{\pgfqpoint{0.963137in}{4.019360in}}%
\pgfpathlineto{\pgfqpoint{0.974271in}{4.043066in}}%
\pgfpathlineto{\pgfqpoint{0.971224in}{4.078649in}}%
\pgfpathlineto{\pgfqpoint{0.976556in}{4.088769in}}%
\pgfpathlineto{\pgfqpoint{0.977083in}{4.133092in}}%
\pgfpathlineto{\pgfqpoint{0.982136in}{4.145855in}}%
\pgfpathlineto{\pgfqpoint{0.999941in}{4.147841in}}%
\pgfpathlineto{\pgfqpoint{1.016464in}{4.142175in}}%
\pgfpathlineto{\pgfqpoint{1.023660in}{4.126570in}}%
\pgfpathlineto{\pgfqpoint{1.033749in}{4.127168in}}%
\pgfpathlineto{\pgfqpoint{1.046284in}{4.145043in}}%
\pgfpathlineto{\pgfqpoint{1.064402in}{4.233160in}}%
\pgfpathlineto{\pgfqpoint{1.167500in}{4.211874in}}%
\pgfpathlineto{\pgfqpoint{1.227336in}{4.200074in}}%
\pgfpathlineto{\pgfqpoint{1.386229in}{4.172143in}}%
\pgfpathlineto{\pgfqpoint{1.430188in}{4.163315in}}%
\pgfpathlineto{\pgfqpoint{1.516457in}{4.149943in}}%
\pgfpathlineto{\pgfqpoint{1.498768in}{4.036047in}}%
\pgfpathlineto{\pgfqpoint{1.480371in}{3.917241in}}%
\pgfpathlineto{\pgfqpoint{1.459184in}{3.783582in}}%
\pgfpathlineto{\pgfqpoint{1.435357in}{3.630155in}}%
\pgfpathlineto{\pgfqpoint{1.416109in}{3.504143in}}%
\pgfpathlineto{\pgfqpoint{1.278236in}{3.526036in}}%
\pgfpathlineto{\pgfqpoint{1.217659in}{3.536411in}}%
\pgfpathlineto{\pgfqpoint{1.190598in}{3.552612in}}%
\pgfpathlineto{\pgfqpoint{1.014159in}{3.659067in}}%
\pgfpathlineto{\pgfqpoint{0.881749in}{3.739855in}}%
\pgfpathlineto{\pgfqpoint{0.886159in}{3.754174in}}%
\pgfpathlineto{\pgfqpoint{0.897097in}{3.764192in}}%
\pgfpathlineto{\pgfqpoint{0.908473in}{3.762335in}}%
\pgfpathlineto{\pgfqpoint{0.924974in}{3.772858in}}%
\pgfpathlineto{\pgfqpoint{0.927572in}{3.787979in}}%
\pgfpathlineto{\pgfqpoint{0.907448in}{3.806322in}}%
\pgfpathlineto{\pgfqpoint{0.921760in}{3.841535in}}%
\pgfpathlineto{\pgfqpoint{0.936150in}{3.855078in}}%
\pgfpathlineto{\pgfqpoint{0.942979in}{3.871143in}}%
\pgfpathlineto{\pgfqpoint{0.947187in}{3.900602in}}%
\pgfpathlineto{\pgfqpoint{0.960679in}{3.913943in}}%
\pgfpathlineto{\pgfqpoint{0.988999in}{3.927268in}}%
\pgfpathlineto{\pgfqpoint{0.988373in}{3.939008in}}%
\pgfpathlineto{\pgfqpoint{0.972557in}{3.953579in}}%
\pgfpathlineto{\pgfqpoint{0.970403in}{3.983615in}}%
\pgfpathlineto{\pgfqpoint{0.959513in}{4.005573in}}%
\pgfpathlineto{\pgfqpoint{0.963137in}{4.019360in}}%
\pgfusepath{stroke}%
\end{pgfscope}%
\begin{pgfscope}%
\pgfpathrectangle{\pgfqpoint{0.100000in}{2.413063in}}{\pgfqpoint{5.037500in}{3.427208in}}%
\pgfusepath{clip}%
\pgfsetbuttcap%
\pgfsetroundjoin%
\pgfsetlinewidth{0.501875pt}%
\definecolor{currentstroke}{rgb}{0.827451,0.827451,0.827451}%
\pgfsetstrokecolor{currentstroke}%
\pgfsetdash{}{0pt}%
\pgfpathmoveto{\pgfqpoint{2.854521in}{3.662061in}}%
\pgfpathlineto{\pgfqpoint{2.828963in}{3.669966in}}%
\pgfpathlineto{\pgfqpoint{2.796214in}{3.695046in}}%
\pgfpathlineto{\pgfqpoint{2.781668in}{3.700436in}}%
\pgfpathlineto{\pgfqpoint{2.772424in}{3.689604in}}%
\pgfpathlineto{\pgfqpoint{2.760751in}{3.689057in}}%
\pgfpathlineto{\pgfqpoint{2.745983in}{3.698253in}}%
\pgfpathlineto{\pgfqpoint{2.722809in}{3.686470in}}%
\pgfpathlineto{\pgfqpoint{2.714559in}{3.692273in}}%
\pgfpathlineto{\pgfqpoint{2.691756in}{3.678526in}}%
\pgfpathlineto{\pgfqpoint{2.667699in}{3.681062in}}%
\pgfpathlineto{\pgfqpoint{2.649143in}{3.689996in}}%
\pgfpathlineto{\pgfqpoint{2.636148in}{3.686633in}}%
\pgfpathlineto{\pgfqpoint{2.615348in}{3.699267in}}%
\pgfpathlineto{\pgfqpoint{2.603777in}{3.676985in}}%
\pgfpathlineto{\pgfqpoint{2.591944in}{3.696148in}}%
\pgfpathlineto{\pgfqpoint{2.568572in}{3.688713in}}%
\pgfpathlineto{\pgfqpoint{2.548129in}{3.706915in}}%
\pgfpathlineto{\pgfqpoint{2.530266in}{3.692244in}}%
\pgfpathlineto{\pgfqpoint{2.521320in}{3.705726in}}%
\pgfpathlineto{\pgfqpoint{2.508425in}{3.710104in}}%
\pgfpathlineto{\pgfqpoint{2.500572in}{3.723102in}}%
\pgfpathlineto{\pgfqpoint{2.483707in}{3.726801in}}%
\pgfpathlineto{\pgfqpoint{2.473975in}{3.717018in}}%
\pgfpathlineto{\pgfqpoint{2.457440in}{3.729697in}}%
\pgfpathlineto{\pgfqpoint{2.449732in}{3.726802in}}%
\pgfpathlineto{\pgfqpoint{2.422328in}{3.737076in}}%
\pgfpathlineto{\pgfqpoint{2.405164in}{3.738222in}}%
\pgfpathlineto{\pgfqpoint{2.397482in}{3.760041in}}%
\pgfpathlineto{\pgfqpoint{2.383723in}{3.757323in}}%
\pgfpathlineto{\pgfqpoint{2.367963in}{3.762707in}}%
\pgfpathlineto{\pgfqpoint{2.357586in}{3.759596in}}%
\pgfpathlineto{\pgfqpoint{2.335330in}{3.784101in}}%
\pgfpathlineto{\pgfqpoint{2.329128in}{3.782498in}}%
\pgfpathlineto{\pgfqpoint{2.334761in}{3.881873in}}%
\pgfpathlineto{\pgfqpoint{2.340896in}{4.004874in}}%
\pgfpathlineto{\pgfqpoint{2.240180in}{4.010507in}}%
\pgfpathlineto{\pgfqpoint{2.140792in}{4.018021in}}%
\pgfpathlineto{\pgfqpoint{2.064008in}{4.024689in}}%
\pgfpathlineto{\pgfqpoint{2.069106in}{4.082796in}}%
\pgfpathlineto{\pgfqpoint{2.157107in}{4.074398in}}%
\pgfpathlineto{\pgfqpoint{2.256916in}{4.068140in}}%
\pgfpathlineto{\pgfqpoint{2.395919in}{4.059944in}}%
\pgfpathlineto{\pgfqpoint{2.478429in}{4.056760in}}%
\pgfpathlineto{\pgfqpoint{2.577276in}{4.053273in}}%
\pgfpathlineto{\pgfqpoint{2.711550in}{4.050923in}}%
\pgfpathlineto{\pgfqpoint{2.839063in}{4.050335in}}%
\pgfpathlineto{\pgfqpoint{2.839596in}{3.991836in}}%
\pgfpathlineto{\pgfqpoint{2.846754in}{3.947764in}}%
\pgfpathlineto{\pgfqpoint{2.857873in}{3.866362in}}%
\pgfpathlineto{\pgfqpoint{2.855579in}{3.727350in}}%
\pgfpathlineto{\pgfqpoint{2.854521in}{3.662061in}}%
\pgfusepath{stroke}%
\end{pgfscope}%
\begin{pgfscope}%
\pgfpathrectangle{\pgfqpoint{0.100000in}{2.413063in}}{\pgfqpoint{5.037500in}{3.427208in}}%
\pgfusepath{clip}%
\pgfsetbuttcap%
\pgfsetroundjoin%
\pgfsetlinewidth{0.501875pt}%
\definecolor{currentstroke}{rgb}{0.827451,0.827451,0.827451}%
\pgfsetstrokecolor{currentstroke}%
\pgfsetdash{}{0pt}%
\pgfpathmoveto{\pgfqpoint{3.811487in}{3.877948in}}%
\pgfpathlineto{\pgfqpoint{3.811562in}{3.903705in}}%
\pgfpathlineto{\pgfqpoint{3.833945in}{3.913596in}}%
\pgfpathlineto{\pgfqpoint{3.834884in}{3.929407in}}%
\pgfpathlineto{\pgfqpoint{3.854924in}{3.948722in}}%
\pgfpathlineto{\pgfqpoint{3.879975in}{3.952678in}}%
\pgfpathlineto{\pgfqpoint{3.901997in}{3.973964in}}%
\pgfpathlineto{\pgfqpoint{3.924997in}{3.983478in}}%
\pgfpathlineto{\pgfqpoint{3.942259in}{4.012040in}}%
\pgfpathlineto{\pgfqpoint{3.957980in}{4.009966in}}%
\pgfpathlineto{\pgfqpoint{3.991206in}{4.035991in}}%
\pgfpathlineto{\pgfqpoint{4.008752in}{4.036482in}}%
\pgfpathlineto{\pgfqpoint{4.016161in}{4.056311in}}%
\pgfpathlineto{\pgfqpoint{4.030107in}{4.070117in}}%
\pgfpathlineto{\pgfqpoint{4.033298in}{4.093639in}}%
\pgfpathlineto{\pgfqpoint{4.063306in}{4.096172in}}%
\pgfpathlineto{\pgfqpoint{4.184620in}{4.112195in}}%
\pgfpathlineto{\pgfqpoint{4.327461in}{4.136802in}}%
\pgfpathlineto{\pgfqpoint{4.430773in}{4.157948in}}%
\pgfpathlineto{\pgfqpoint{4.546278in}{4.181907in}}%
\pgfpathlineto{\pgfqpoint{4.579284in}{4.136592in}}%
\pgfpathlineto{\pgfqpoint{4.561406in}{4.139485in}}%
\pgfpathlineto{\pgfqpoint{4.538456in}{4.133894in}}%
\pgfpathlineto{\pgfqpoint{4.515842in}{4.110800in}}%
\pgfpathlineto{\pgfqpoint{4.499590in}{4.112461in}}%
\pgfpathlineto{\pgfqpoint{4.497534in}{4.098745in}}%
\pgfpathlineto{\pgfqpoint{4.526910in}{4.109592in}}%
\pgfpathlineto{\pgfqpoint{4.556471in}{4.114065in}}%
\pgfpathlineto{\pgfqpoint{4.567447in}{4.108113in}}%
\pgfpathlineto{\pgfqpoint{4.582218in}{4.114897in}}%
\pgfpathlineto{\pgfqpoint{4.589841in}{4.110144in}}%
\pgfpathlineto{\pgfqpoint{4.596587in}{4.087509in}}%
\pgfpathlineto{\pgfqpoint{4.582547in}{4.080499in}}%
\pgfpathlineto{\pgfqpoint{4.572911in}{4.052898in}}%
\pgfpathlineto{\pgfqpoint{4.562808in}{4.042151in}}%
\pgfpathlineto{\pgfqpoint{4.531837in}{4.044502in}}%
\pgfpathlineto{\pgfqpoint{4.533501in}{4.060709in}}%
\pgfpathlineto{\pgfqpoint{4.516587in}{4.053630in}}%
\pgfpathlineto{\pgfqpoint{4.512908in}{4.040112in}}%
\pgfpathlineto{\pgfqpoint{4.524502in}{4.026111in}}%
\pgfpathlineto{\pgfqpoint{4.528440in}{4.002346in}}%
\pgfpathlineto{\pgfqpoint{4.509469in}{3.988806in}}%
\pgfpathlineto{\pgfqpoint{4.529986in}{3.984043in}}%
\pgfpathlineto{\pgfqpoint{4.539271in}{3.994000in}}%
\pgfpathlineto{\pgfqpoint{4.559800in}{3.995213in}}%
\pgfpathlineto{\pgfqpoint{4.549250in}{3.973712in}}%
\pgfpathlineto{\pgfqpoint{4.536329in}{3.963352in}}%
\pgfpathlineto{\pgfqpoint{4.497243in}{3.952980in}}%
\pgfpathlineto{\pgfqpoint{4.457112in}{3.916583in}}%
\pgfpathlineto{\pgfqpoint{4.432367in}{3.880721in}}%
\pgfpathlineto{\pgfqpoint{4.432221in}{3.866156in}}%
\pgfpathlineto{\pgfqpoint{4.422337in}{3.846058in}}%
\pgfpathlineto{\pgfqpoint{4.371607in}{3.832835in}}%
\pgfpathlineto{\pgfqpoint{4.250965in}{3.919317in}}%
\pgfpathlineto{\pgfqpoint{4.144721in}{3.903707in}}%
\pgfpathlineto{\pgfqpoint{4.143831in}{3.918121in}}%
\pgfpathlineto{\pgfqpoint{4.127682in}{3.934353in}}%
\pgfpathlineto{\pgfqpoint{4.115457in}{3.938326in}}%
\pgfpathlineto{\pgfqpoint{4.000022in}{3.926332in}}%
\pgfpathlineto{\pgfqpoint{3.973490in}{3.917333in}}%
\pgfpathlineto{\pgfqpoint{3.925582in}{3.893502in}}%
\pgfpathlineto{\pgfqpoint{3.884181in}{3.886905in}}%
\pgfpathlineto{\pgfqpoint{3.811487in}{3.877948in}}%
\pgfusepath{stroke}%
\end{pgfscope}%
\begin{pgfscope}%
\pgfpathrectangle{\pgfqpoint{0.100000in}{2.413063in}}{\pgfqpoint{5.037500in}{3.427208in}}%
\pgfusepath{clip}%
\pgfsetbuttcap%
\pgfsetroundjoin%
\pgfsetlinewidth{0.501875pt}%
\definecolor{currentstroke}{rgb}{0.827451,0.827451,0.827451}%
\pgfsetstrokecolor{currentstroke}%
\pgfsetdash{}{0pt}%
\pgfpathmoveto{\pgfqpoint{3.811487in}{3.877948in}}%
\pgfpathlineto{\pgfqpoint{3.690671in}{3.864679in}}%
\pgfpathlineto{\pgfqpoint{3.580149in}{3.854740in}}%
\pgfpathlineto{\pgfqpoint{3.466555in}{3.847394in}}%
\pgfpathlineto{\pgfqpoint{3.447056in}{3.844568in}}%
\pgfpathlineto{\pgfqpoint{3.370483in}{3.838678in}}%
\pgfpathlineto{\pgfqpoint{3.247837in}{3.831521in}}%
\pgfpathlineto{\pgfqpoint{3.269658in}{3.849634in}}%
\pgfpathlineto{\pgfqpoint{3.264679in}{3.868699in}}%
\pgfpathlineto{\pgfqpoint{3.269502in}{3.891811in}}%
\pgfpathlineto{\pgfqpoint{3.276871in}{3.902691in}}%
\pgfpathlineto{\pgfqpoint{3.276593in}{3.917829in}}%
\pgfpathlineto{\pgfqpoint{3.296214in}{3.927350in}}%
\pgfpathlineto{\pgfqpoint{3.296248in}{3.949628in}}%
\pgfpathlineto{\pgfqpoint{3.298944in}{3.975308in}}%
\pgfpathlineto{\pgfqpoint{3.312625in}{3.991720in}}%
\pgfpathlineto{\pgfqpoint{3.310355in}{4.008001in}}%
\pgfpathlineto{\pgfqpoint{3.315658in}{4.008373in}}%
\pgfpathlineto{\pgfqpoint{3.321282in}{4.009187in}}%
\pgfpathlineto{\pgfqpoint{3.448152in}{4.017250in}}%
\pgfpathlineto{\pgfqpoint{3.443904in}{4.038092in}}%
\pgfpathlineto{\pgfqpoint{3.464320in}{4.035264in}}%
\pgfpathlineto{\pgfqpoint{3.587896in}{4.047489in}}%
\pgfpathlineto{\pgfqpoint{3.687088in}{4.053103in}}%
\pgfpathlineto{\pgfqpoint{3.747536in}{4.058186in}}%
\pgfpathlineto{\pgfqpoint{3.847633in}{4.067678in}}%
\pgfpathlineto{\pgfqpoint{3.849302in}{4.069566in}}%
\pgfpathlineto{\pgfqpoint{3.892070in}{4.073801in}}%
\pgfpathlineto{\pgfqpoint{4.033298in}{4.093639in}}%
\pgfpathlineto{\pgfqpoint{4.030107in}{4.070117in}}%
\pgfpathlineto{\pgfqpoint{4.016161in}{4.056311in}}%
\pgfpathlineto{\pgfqpoint{4.008752in}{4.036482in}}%
\pgfpathlineto{\pgfqpoint{3.991206in}{4.035991in}}%
\pgfpathlineto{\pgfqpoint{3.957980in}{4.009966in}}%
\pgfpathlineto{\pgfqpoint{3.942259in}{4.012040in}}%
\pgfpathlineto{\pgfqpoint{3.924997in}{3.983478in}}%
\pgfpathlineto{\pgfqpoint{3.901997in}{3.973964in}}%
\pgfpathlineto{\pgfqpoint{3.879975in}{3.952678in}}%
\pgfpathlineto{\pgfqpoint{3.854924in}{3.948722in}}%
\pgfpathlineto{\pgfqpoint{3.834884in}{3.929407in}}%
\pgfpathlineto{\pgfqpoint{3.833945in}{3.913596in}}%
\pgfpathlineto{\pgfqpoint{3.811562in}{3.903705in}}%
\pgfpathlineto{\pgfqpoint{3.811487in}{3.877948in}}%
\pgfusepath{stroke}%
\end{pgfscope}%
\begin{pgfscope}%
\pgfpathrectangle{\pgfqpoint{0.100000in}{2.413063in}}{\pgfqpoint{5.037500in}{3.427208in}}%
\pgfusepath{clip}%
\pgfsetbuttcap%
\pgfsetroundjoin%
\pgfsetlinewidth{0.501875pt}%
\definecolor{currentstroke}{rgb}{0.827451,0.827451,0.827451}%
\pgfsetstrokecolor{currentstroke}%
\pgfsetdash{}{0pt}%
\pgfpathmoveto{\pgfqpoint{1.668801in}{3.520265in}}%
\pgfpathlineto{\pgfqpoint{1.660184in}{3.534111in}}%
\pgfpathlineto{\pgfqpoint{1.663749in}{3.546038in}}%
\pgfpathlineto{\pgfqpoint{1.836936in}{3.525734in}}%
\pgfpathlineto{\pgfqpoint{2.012416in}{3.508241in}}%
\pgfpathlineto{\pgfqpoint{2.017532in}{3.567562in}}%
\pgfpathlineto{\pgfqpoint{2.043908in}{3.824143in}}%
\pgfpathlineto{\pgfqpoint{2.061241in}{4.024844in}}%
\pgfpathlineto{\pgfqpoint{2.064008in}{4.024689in}}%
\pgfpathlineto{\pgfqpoint{2.240180in}{4.010507in}}%
\pgfpathlineto{\pgfqpoint{2.340896in}{4.004874in}}%
\pgfpathlineto{\pgfqpoint{2.329128in}{3.782498in}}%
\pgfpathlineto{\pgfqpoint{2.335330in}{3.784101in}}%
\pgfpathlineto{\pgfqpoint{2.357586in}{3.759596in}}%
\pgfpathlineto{\pgfqpoint{2.367963in}{3.762707in}}%
\pgfpathlineto{\pgfqpoint{2.383723in}{3.757323in}}%
\pgfpathlineto{\pgfqpoint{2.397482in}{3.760041in}}%
\pgfpathlineto{\pgfqpoint{2.405164in}{3.738222in}}%
\pgfpathlineto{\pgfqpoint{2.422328in}{3.737076in}}%
\pgfpathlineto{\pgfqpoint{2.449732in}{3.726802in}}%
\pgfpathlineto{\pgfqpoint{2.457440in}{3.729697in}}%
\pgfpathlineto{\pgfqpoint{2.473975in}{3.717018in}}%
\pgfpathlineto{\pgfqpoint{2.483707in}{3.726801in}}%
\pgfpathlineto{\pgfqpoint{2.500572in}{3.723102in}}%
\pgfpathlineto{\pgfqpoint{2.508425in}{3.710104in}}%
\pgfpathlineto{\pgfqpoint{2.521320in}{3.705726in}}%
\pgfpathlineto{\pgfqpoint{2.530266in}{3.692244in}}%
\pgfpathlineto{\pgfqpoint{2.548129in}{3.706915in}}%
\pgfpathlineto{\pgfqpoint{2.568572in}{3.688713in}}%
\pgfpathlineto{\pgfqpoint{2.591944in}{3.696148in}}%
\pgfpathlineto{\pgfqpoint{2.603777in}{3.676985in}}%
\pgfpathlineto{\pgfqpoint{2.615348in}{3.699267in}}%
\pgfpathlineto{\pgfqpoint{2.636148in}{3.686633in}}%
\pgfpathlineto{\pgfqpoint{2.649143in}{3.689996in}}%
\pgfpathlineto{\pgfqpoint{2.667699in}{3.681062in}}%
\pgfpathlineto{\pgfqpoint{2.691756in}{3.678526in}}%
\pgfpathlineto{\pgfqpoint{2.714559in}{3.692273in}}%
\pgfpathlineto{\pgfqpoint{2.722809in}{3.686470in}}%
\pgfpathlineto{\pgfqpoint{2.745983in}{3.698253in}}%
\pgfpathlineto{\pgfqpoint{2.760751in}{3.689057in}}%
\pgfpathlineto{\pgfqpoint{2.772424in}{3.689604in}}%
\pgfpathlineto{\pgfqpoint{2.781668in}{3.700436in}}%
\pgfpathlineto{\pgfqpoint{2.796214in}{3.695046in}}%
\pgfpathlineto{\pgfqpoint{2.828963in}{3.669966in}}%
\pgfpathlineto{\pgfqpoint{2.854521in}{3.662061in}}%
\pgfpathlineto{\pgfqpoint{2.864771in}{3.652378in}}%
\pgfpathlineto{\pgfqpoint{2.877599in}{3.657651in}}%
\pgfpathlineto{\pgfqpoint{2.897040in}{3.653613in}}%
\pgfpathlineto{\pgfqpoint{2.899043in}{3.472768in}}%
\pgfpathlineto{\pgfqpoint{2.912544in}{3.461313in}}%
\pgfpathlineto{\pgfqpoint{2.923366in}{3.440204in}}%
\pgfpathlineto{\pgfqpoint{2.919562in}{3.426082in}}%
\pgfpathlineto{\pgfqpoint{2.928060in}{3.420103in}}%
\pgfpathlineto{\pgfqpoint{2.934966in}{3.394061in}}%
\pgfpathlineto{\pgfqpoint{2.948193in}{3.380114in}}%
\pgfpathlineto{\pgfqpoint{2.951156in}{3.350504in}}%
\pgfpathlineto{\pgfqpoint{2.948856in}{3.337967in}}%
\pgfpathlineto{\pgfqpoint{2.930876in}{3.304788in}}%
\pgfpathlineto{\pgfqpoint{2.928809in}{3.280933in}}%
\pgfpathlineto{\pgfqpoint{2.934908in}{3.275954in}}%
\pgfpathlineto{\pgfqpoint{2.935438in}{3.255055in}}%
\pgfpathlineto{\pgfqpoint{2.929239in}{3.241920in}}%
\pgfpathlineto{\pgfqpoint{2.919598in}{3.236972in}}%
\pgfpathlineto{\pgfqpoint{2.910158in}{3.219808in}}%
\pgfpathlineto{\pgfqpoint{2.922190in}{3.203183in}}%
\pgfpathlineto{\pgfqpoint{2.898834in}{3.202856in}}%
\pgfpathlineto{\pgfqpoint{2.836385in}{3.174284in}}%
\pgfpathlineto{\pgfqpoint{2.848308in}{3.191383in}}%
\pgfpathlineto{\pgfqpoint{2.833870in}{3.200602in}}%
\pgfpathlineto{\pgfqpoint{2.830853in}{3.216277in}}%
\pgfpathlineto{\pgfqpoint{2.802674in}{3.188975in}}%
\pgfpathlineto{\pgfqpoint{2.815180in}{3.170314in}}%
\pgfpathlineto{\pgfqpoint{2.797340in}{3.146553in}}%
\pgfpathlineto{\pgfqpoint{2.787739in}{3.147050in}}%
\pgfpathlineto{\pgfqpoint{2.778707in}{3.121174in}}%
\pgfpathlineto{\pgfqpoint{2.750144in}{3.100819in}}%
\pgfpathlineto{\pgfqpoint{2.723459in}{3.093484in}}%
\pgfpathlineto{\pgfqpoint{2.676917in}{3.074401in}}%
\pgfpathlineto{\pgfqpoint{2.672094in}{3.085046in}}%
\pgfpathlineto{\pgfqpoint{2.643101in}{3.075248in}}%
\pgfpathlineto{\pgfqpoint{2.660934in}{3.058565in}}%
\pgfpathlineto{\pgfqpoint{2.632807in}{3.044071in}}%
\pgfpathlineto{\pgfqpoint{2.602468in}{3.022188in}}%
\pgfpathlineto{\pgfqpoint{2.595182in}{3.032361in}}%
\pgfpathlineto{\pgfqpoint{2.570354in}{3.017114in}}%
\pgfpathlineto{\pgfqpoint{2.594689in}{3.011325in}}%
\pgfpathlineto{\pgfqpoint{2.576341in}{2.987353in}}%
\pgfpathlineto{\pgfqpoint{2.567385in}{2.994487in}}%
\pgfpathlineto{\pgfqpoint{2.555355in}{2.983028in}}%
\pgfpathlineto{\pgfqpoint{2.570347in}{2.973019in}}%
\pgfpathlineto{\pgfqpoint{2.561483in}{2.958390in}}%
\pgfpathlineto{\pgfqpoint{2.550589in}{2.923757in}}%
\pgfpathlineto{\pgfqpoint{2.534875in}{2.890402in}}%
\pgfpathlineto{\pgfqpoint{2.541929in}{2.868538in}}%
\pgfpathlineto{\pgfqpoint{2.553920in}{2.816991in}}%
\pgfpathlineto{\pgfqpoint{2.555001in}{2.796129in}}%
\pgfpathlineto{\pgfqpoint{2.573528in}{2.768767in}}%
\pgfpathlineto{\pgfqpoint{2.559241in}{2.770366in}}%
\pgfpathlineto{\pgfqpoint{2.545364in}{2.756562in}}%
\pgfpathlineto{\pgfqpoint{2.529111in}{2.767411in}}%
\pgfpathlineto{\pgfqpoint{2.523277in}{2.778234in}}%
\pgfpathlineto{\pgfqpoint{2.500146in}{2.783269in}}%
\pgfpathlineto{\pgfqpoint{2.464813in}{2.783886in}}%
\pgfpathlineto{\pgfqpoint{2.438779in}{2.804383in}}%
\pgfpathlineto{\pgfqpoint{2.415141in}{2.807806in}}%
\pgfpathlineto{\pgfqpoint{2.400800in}{2.824095in}}%
\pgfpathlineto{\pgfqpoint{2.370766in}{2.830651in}}%
\pgfpathlineto{\pgfqpoint{2.354344in}{2.883026in}}%
\pgfpathlineto{\pgfqpoint{2.337552in}{2.904007in}}%
\pgfpathlineto{\pgfqpoint{2.340405in}{2.923943in}}%
\pgfpathlineto{\pgfqpoint{2.329993in}{2.938513in}}%
\pgfpathlineto{\pgfqpoint{2.336527in}{2.958428in}}%
\pgfpathlineto{\pgfqpoint{2.331137in}{2.973021in}}%
\pgfpathlineto{\pgfqpoint{2.314263in}{2.979632in}}%
\pgfpathlineto{\pgfqpoint{2.298486in}{2.996460in}}%
\pgfpathlineto{\pgfqpoint{2.292736in}{3.018989in}}%
\pgfpathlineto{\pgfqpoint{2.277807in}{3.039464in}}%
\pgfpathlineto{\pgfqpoint{2.257935in}{3.055395in}}%
\pgfpathlineto{\pgfqpoint{2.240037in}{3.101094in}}%
\pgfpathlineto{\pgfqpoint{2.225584in}{3.151073in}}%
\pgfpathlineto{\pgfqpoint{2.213725in}{3.170834in}}%
\pgfpathlineto{\pgfqpoint{2.193167in}{3.187471in}}%
\pgfpathlineto{\pgfqpoint{2.188029in}{3.199547in}}%
\pgfpathlineto{\pgfqpoint{2.168780in}{3.207045in}}%
\pgfpathlineto{\pgfqpoint{2.149781in}{3.239041in}}%
\pgfpathlineto{\pgfqpoint{2.132430in}{3.236585in}}%
\pgfpathlineto{\pgfqpoint{2.114812in}{3.244576in}}%
\pgfpathlineto{\pgfqpoint{2.089838in}{3.243031in}}%
\pgfpathlineto{\pgfqpoint{2.064467in}{3.256219in}}%
\pgfpathlineto{\pgfqpoint{2.057313in}{3.243677in}}%
\pgfpathlineto{\pgfqpoint{2.027664in}{3.243363in}}%
\pgfpathlineto{\pgfqpoint{2.012569in}{3.219586in}}%
\pgfpathlineto{\pgfqpoint{1.999441in}{3.190148in}}%
\pgfpathlineto{\pgfqpoint{1.971522in}{3.158547in}}%
\pgfpathlineto{\pgfqpoint{1.939894in}{3.172415in}}%
\pgfpathlineto{\pgfqpoint{1.935405in}{3.181580in}}%
\pgfpathlineto{\pgfqpoint{1.916184in}{3.188571in}}%
\pgfpathlineto{\pgfqpoint{1.910279in}{3.198133in}}%
\pgfpathlineto{\pgfqpoint{1.884762in}{3.207785in}}%
\pgfpathlineto{\pgfqpoint{1.870487in}{3.227517in}}%
\pgfpathlineto{\pgfqpoint{1.853817in}{3.237037in}}%
\pgfpathlineto{\pgfqpoint{1.839474in}{3.253650in}}%
\pgfpathlineto{\pgfqpoint{1.828300in}{3.281784in}}%
\pgfpathlineto{\pgfqpoint{1.829550in}{3.320230in}}%
\pgfpathlineto{\pgfqpoint{1.816442in}{3.339658in}}%
\pgfpathlineto{\pgfqpoint{1.814934in}{3.360700in}}%
\pgfpathlineto{\pgfqpoint{1.785850in}{3.392204in}}%
\pgfpathlineto{\pgfqpoint{1.768928in}{3.398929in}}%
\pgfpathlineto{\pgfqpoint{1.750970in}{3.428326in}}%
\pgfpathlineto{\pgfqpoint{1.735668in}{3.440005in}}%
\pgfpathlineto{\pgfqpoint{1.716224in}{3.468446in}}%
\pgfpathlineto{\pgfqpoint{1.696330in}{3.480754in}}%
\pgfpathlineto{\pgfqpoint{1.683301in}{3.512292in}}%
\pgfpathlineto{\pgfqpoint{1.668801in}{3.520265in}}%
\pgfpathlineto{\pgfqpoint{1.668801in}{3.520265in}}%
\pgfusepath{stroke}%
\end{pgfscope}%
\begin{pgfscope}%
\pgfpathrectangle{\pgfqpoint{0.100000in}{2.413063in}}{\pgfqpoint{5.037500in}{3.427208in}}%
\pgfusepath{clip}%
\pgfsetbuttcap%
\pgfsetroundjoin%
\pgfsetlinewidth{0.501875pt}%
\definecolor{currentstroke}{rgb}{0.827451,0.827451,0.827451}%
\pgfsetstrokecolor{currentstroke}%
\pgfsetdash{}{0pt}%
\pgfpathmoveto{\pgfqpoint{1.516457in}{4.149943in}}%
\pgfpathlineto{\pgfqpoint{1.659981in}{4.129100in}}%
\pgfpathlineto{\pgfqpoint{1.715584in}{4.120532in}}%
\pgfpathlineto{\pgfqpoint{1.872319in}{4.101891in}}%
\pgfpathlineto{\pgfqpoint{1.977813in}{4.090887in}}%
\pgfpathlineto{\pgfqpoint{2.069106in}{4.082796in}}%
\pgfpathlineto{\pgfqpoint{2.064008in}{4.024689in}}%
\pgfpathlineto{\pgfqpoint{2.061241in}{4.024844in}}%
\pgfpathlineto{\pgfqpoint{2.052879in}{3.925113in}}%
\pgfpathlineto{\pgfqpoint{2.043908in}{3.824143in}}%
\pgfpathlineto{\pgfqpoint{2.033530in}{3.718444in}}%
\pgfpathlineto{\pgfqpoint{2.017532in}{3.567562in}}%
\pgfpathlineto{\pgfqpoint{2.012416in}{3.508241in}}%
\pgfpathlineto{\pgfqpoint{1.918368in}{3.517731in}}%
\pgfpathlineto{\pgfqpoint{1.836936in}{3.525734in}}%
\pgfpathlineto{\pgfqpoint{1.724316in}{3.538525in}}%
\pgfpathlineto{\pgfqpoint{1.663749in}{3.546038in}}%
\pgfpathlineto{\pgfqpoint{1.660184in}{3.534111in}}%
\pgfpathlineto{\pgfqpoint{1.668801in}{3.520265in}}%
\pgfpathlineto{\pgfqpoint{1.596010in}{3.529718in}}%
\pgfpathlineto{\pgfqpoint{1.506202in}{3.542608in}}%
\pgfpathlineto{\pgfqpoint{1.498028in}{3.491813in}}%
\pgfpathlineto{\pgfqpoint{1.416109in}{3.504143in}}%
\pgfpathlineto{\pgfqpoint{1.435357in}{3.630155in}}%
\pgfpathlineto{\pgfqpoint{1.459184in}{3.783582in}}%
\pgfpathlineto{\pgfqpoint{1.480371in}{3.917241in}}%
\pgfpathlineto{\pgfqpoint{1.498768in}{4.036047in}}%
\pgfpathlineto{\pgfqpoint{1.516457in}{4.149943in}}%
\pgfusepath{stroke}%
\end{pgfscope}%
\begin{pgfscope}%
\pgfpathrectangle{\pgfqpoint{0.100000in}{2.413063in}}{\pgfqpoint{5.037500in}{3.427208in}}%
\pgfusepath{clip}%
\pgfsetbuttcap%
\pgfsetroundjoin%
\pgfsetlinewidth{0.501875pt}%
\definecolor{currentstroke}{rgb}{0.827451,0.827451,0.827451}%
\pgfsetstrokecolor{currentstroke}%
\pgfsetdash{}{0pt}%
\pgfpathmoveto{\pgfqpoint{3.798503in}{3.411653in}}%
\pgfpathlineto{\pgfqpoint{3.661580in}{3.396445in}}%
\pgfpathlineto{\pgfqpoint{3.540884in}{3.387076in}}%
\pgfpathlineto{\pgfqpoint{3.539374in}{3.372285in}}%
\pgfpathlineto{\pgfqpoint{3.550461in}{3.358206in}}%
\pgfpathlineto{\pgfqpoint{3.563992in}{3.349936in}}%
\pgfpathlineto{\pgfqpoint{3.563787in}{3.328159in}}%
\pgfpathlineto{\pgfqpoint{3.560202in}{3.313610in}}%
\pgfpathlineto{\pgfqpoint{3.548259in}{3.303125in}}%
\pgfpathlineto{\pgfqpoint{3.531613in}{3.304235in}}%
\pgfpathlineto{\pgfqpoint{3.515874in}{3.317227in}}%
\pgfpathlineto{\pgfqpoint{3.513069in}{3.340345in}}%
\pgfpathlineto{\pgfqpoint{3.501347in}{3.353791in}}%
\pgfpathlineto{\pgfqpoint{3.493369in}{3.305669in}}%
\pgfpathlineto{\pgfqpoint{3.466258in}{3.310242in}}%
\pgfpathlineto{\pgfqpoint{3.456830in}{3.394390in}}%
\pgfpathlineto{\pgfqpoint{3.446620in}{3.483072in}}%
\pgfpathlineto{\pgfqpoint{3.450412in}{3.611019in}}%
\pgfpathlineto{\pgfqpoint{3.452766in}{3.698819in}}%
\pgfpathlineto{\pgfqpoint{3.457771in}{3.832779in}}%
\pgfpathlineto{\pgfqpoint{3.447056in}{3.844568in}}%
\pgfpathlineto{\pgfqpoint{3.466555in}{3.847394in}}%
\pgfpathlineto{\pgfqpoint{3.580149in}{3.854740in}}%
\pgfpathlineto{\pgfqpoint{3.690671in}{3.864679in}}%
\pgfpathlineto{\pgfqpoint{3.719749in}{3.762827in}}%
\pgfpathlineto{\pgfqpoint{3.739580in}{3.688066in}}%
\pgfpathlineto{\pgfqpoint{3.757355in}{3.625478in}}%
\pgfpathlineto{\pgfqpoint{3.767636in}{3.600276in}}%
\pgfpathlineto{\pgfqpoint{3.783823in}{3.576822in}}%
\pgfpathlineto{\pgfqpoint{3.781209in}{3.564887in}}%
\pgfpathlineto{\pgfqpoint{3.792772in}{3.559073in}}%
\pgfpathlineto{\pgfqpoint{3.779107in}{3.540981in}}%
\pgfpathlineto{\pgfqpoint{3.774492in}{3.508762in}}%
\pgfpathlineto{\pgfqpoint{3.787846in}{3.471078in}}%
\pgfpathlineto{\pgfqpoint{3.785992in}{3.433159in}}%
\pgfpathlineto{\pgfqpoint{3.798503in}{3.411653in}}%
\pgfusepath{stroke}%
\end{pgfscope}%
\begin{pgfscope}%
\pgfpathrectangle{\pgfqpoint{0.100000in}{2.413063in}}{\pgfqpoint{5.037500in}{3.427208in}}%
\pgfusepath{clip}%
\pgfsetbuttcap%
\pgfsetroundjoin%
\pgfsetlinewidth{0.501875pt}%
\definecolor{currentstroke}{rgb}{0.827451,0.827451,0.827451}%
\pgfsetstrokecolor{currentstroke}%
\pgfsetdash{}{0pt}%
\pgfpathmoveto{\pgfqpoint{3.466258in}{3.310242in}}%
\pgfpathlineto{\pgfqpoint{3.438437in}{3.302281in}}%
\pgfpathlineto{\pgfqpoint{3.417699in}{3.307393in}}%
\pgfpathlineto{\pgfqpoint{3.379188in}{3.295143in}}%
\pgfpathlineto{\pgfqpoint{3.350143in}{3.279367in}}%
\pgfpathlineto{\pgfqpoint{3.338225in}{3.307874in}}%
\pgfpathlineto{\pgfqpoint{3.325940in}{3.319823in}}%
\pgfpathlineto{\pgfqpoint{3.320014in}{3.334122in}}%
\pgfpathlineto{\pgfqpoint{3.328701in}{3.372823in}}%
\pgfpathlineto{\pgfqpoint{3.246390in}{3.367755in}}%
\pgfpathlineto{\pgfqpoint{3.139657in}{3.363127in}}%
\pgfpathlineto{\pgfqpoint{3.146007in}{3.372761in}}%
\pgfpathlineto{\pgfqpoint{3.138158in}{3.390955in}}%
\pgfpathlineto{\pgfqpoint{3.145979in}{3.394663in}}%
\pgfpathlineto{\pgfqpoint{3.144244in}{3.412144in}}%
\pgfpathlineto{\pgfqpoint{3.157876in}{3.429096in}}%
\pgfpathlineto{\pgfqpoint{3.165334in}{3.462006in}}%
\pgfpathlineto{\pgfqpoint{3.190283in}{3.483688in}}%
\pgfpathlineto{\pgfqpoint{3.181036in}{3.504745in}}%
\pgfpathlineto{\pgfqpoint{3.198727in}{3.516866in}}%
\pgfpathlineto{\pgfqpoint{3.183467in}{3.538809in}}%
\pgfpathlineto{\pgfqpoint{3.172421in}{3.589042in}}%
\pgfpathlineto{\pgfqpoint{3.176610in}{3.598163in}}%
\pgfpathlineto{\pgfqpoint{3.183224in}{3.615659in}}%
\pgfpathlineto{\pgfqpoint{3.177068in}{3.634020in}}%
\pgfpathlineto{\pgfqpoint{3.180079in}{3.642373in}}%
\pgfpathlineto{\pgfqpoint{3.169249in}{3.673932in}}%
\pgfpathlineto{\pgfqpoint{3.201219in}{3.730598in}}%
\pgfpathlineto{\pgfqpoint{3.202705in}{3.745680in}}%
\pgfpathlineto{\pgfqpoint{3.218105in}{3.756647in}}%
\pgfpathlineto{\pgfqpoint{3.234537in}{3.792851in}}%
\pgfpathlineto{\pgfqpoint{3.233756in}{3.807568in}}%
\pgfpathlineto{\pgfqpoint{3.254223in}{3.822611in}}%
\pgfpathlineto{\pgfqpoint{3.247837in}{3.831521in}}%
\pgfpathlineto{\pgfqpoint{3.370483in}{3.838678in}}%
\pgfpathlineto{\pgfqpoint{3.447056in}{3.844568in}}%
\pgfpathlineto{\pgfqpoint{3.457771in}{3.832779in}}%
\pgfpathlineto{\pgfqpoint{3.452766in}{3.698819in}}%
\pgfpathlineto{\pgfqpoint{3.450412in}{3.611019in}}%
\pgfpathlineto{\pgfqpoint{3.446620in}{3.483072in}}%
\pgfpathlineto{\pgfqpoint{3.456830in}{3.394390in}}%
\pgfpathlineto{\pgfqpoint{3.466258in}{3.310242in}}%
\pgfusepath{stroke}%
\end{pgfscope}%
\begin{pgfscope}%
\pgfpathrectangle{\pgfqpoint{0.100000in}{2.413063in}}{\pgfqpoint{5.037500in}{3.427208in}}%
\pgfusepath{clip}%
\pgfsetbuttcap%
\pgfsetroundjoin%
\pgfsetlinewidth{0.501875pt}%
\definecolor{currentstroke}{rgb}{0.827451,0.827451,0.827451}%
\pgfsetstrokecolor{currentstroke}%
\pgfsetdash{}{0pt}%
\pgfpathmoveto{\pgfqpoint{3.690671in}{3.864679in}}%
\pgfpathlineto{\pgfqpoint{3.811487in}{3.877948in}}%
\pgfpathlineto{\pgfqpoint{3.884181in}{3.886905in}}%
\pgfpathlineto{\pgfqpoint{3.925582in}{3.893502in}}%
\pgfpathlineto{\pgfqpoint{3.908432in}{3.866452in}}%
\pgfpathlineto{\pgfqpoint{3.908443in}{3.853698in}}%
\pgfpathlineto{\pgfqpoint{3.926096in}{3.846821in}}%
\pgfpathlineto{\pgfqpoint{3.938102in}{3.835717in}}%
\pgfpathlineto{\pgfqpoint{3.956245in}{3.834345in}}%
\pgfpathlineto{\pgfqpoint{3.973202in}{3.803087in}}%
\pgfpathlineto{\pgfqpoint{3.991596in}{3.781044in}}%
\pgfpathlineto{\pgfqpoint{4.024265in}{3.762518in}}%
\pgfpathlineto{\pgfqpoint{4.033550in}{3.747742in}}%
\pgfpathlineto{\pgfqpoint{4.062682in}{3.731761in}}%
\pgfpathlineto{\pgfqpoint{4.064872in}{3.720172in}}%
\pgfpathlineto{\pgfqpoint{4.088505in}{3.697128in}}%
\pgfpathlineto{\pgfqpoint{4.106607in}{3.690567in}}%
\pgfpathlineto{\pgfqpoint{4.119426in}{3.668808in}}%
\pgfpathlineto{\pgfqpoint{4.125101in}{3.644350in}}%
\pgfpathlineto{\pgfqpoint{4.143903in}{3.634896in}}%
\pgfpathlineto{\pgfqpoint{4.159040in}{3.608607in}}%
\pgfpathlineto{\pgfqpoint{4.163922in}{3.589272in}}%
\pgfpathlineto{\pgfqpoint{4.186104in}{3.581043in}}%
\pgfpathlineto{\pgfqpoint{4.167490in}{3.545404in}}%
\pgfpathlineto{\pgfqpoint{4.160485in}{3.524342in}}%
\pgfpathlineto{\pgfqpoint{4.165056in}{3.514470in}}%
\pgfpathlineto{\pgfqpoint{4.157076in}{3.498062in}}%
\pgfpathlineto{\pgfqpoint{4.153683in}{3.475378in}}%
\pgfpathlineto{\pgfqpoint{4.139492in}{3.471153in}}%
\pgfpathlineto{\pgfqpoint{4.146986in}{3.450384in}}%
\pgfpathlineto{\pgfqpoint{4.146508in}{3.424034in}}%
\pgfpathlineto{\pgfqpoint{4.121284in}{3.425022in}}%
\pgfpathlineto{\pgfqpoint{4.103711in}{3.429829in}}%
\pgfpathlineto{\pgfqpoint{4.096108in}{3.420892in}}%
\pgfpathlineto{\pgfqpoint{4.101781in}{3.399780in}}%
\pgfpathlineto{\pgfqpoint{4.100548in}{3.375235in}}%
\pgfpathlineto{\pgfqpoint{4.089462in}{3.373357in}}%
\pgfpathlineto{\pgfqpoint{4.080494in}{3.396281in}}%
\pgfpathlineto{\pgfqpoint{3.989179in}{3.390195in}}%
\pgfpathlineto{\pgfqpoint{3.874037in}{3.383913in}}%
\pgfpathlineto{\pgfqpoint{3.815967in}{3.379825in}}%
\pgfpathlineto{\pgfqpoint{3.798503in}{3.411653in}}%
\pgfpathlineto{\pgfqpoint{3.785992in}{3.433159in}}%
\pgfpathlineto{\pgfqpoint{3.787846in}{3.471078in}}%
\pgfpathlineto{\pgfqpoint{3.774492in}{3.508762in}}%
\pgfpathlineto{\pgfqpoint{3.779107in}{3.540981in}}%
\pgfpathlineto{\pgfqpoint{3.792772in}{3.559073in}}%
\pgfpathlineto{\pgfqpoint{3.781209in}{3.564887in}}%
\pgfpathlineto{\pgfqpoint{3.783823in}{3.576822in}}%
\pgfpathlineto{\pgfqpoint{3.767636in}{3.600276in}}%
\pgfpathlineto{\pgfqpoint{3.757355in}{3.625478in}}%
\pgfpathlineto{\pgfqpoint{3.739580in}{3.688066in}}%
\pgfpathlineto{\pgfqpoint{3.719749in}{3.762827in}}%
\pgfpathlineto{\pgfqpoint{3.690671in}{3.864679in}}%
\pgfusepath{stroke}%
\end{pgfscope}%
\begin{pgfscope}%
\pgfpathrectangle{\pgfqpoint{0.100000in}{2.413063in}}{\pgfqpoint{5.037500in}{3.427208in}}%
\pgfusepath{clip}%
\pgfsetbuttcap%
\pgfsetroundjoin%
\pgfsetlinewidth{0.501875pt}%
\definecolor{currentstroke}{rgb}{0.827451,0.827451,0.827451}%
\pgfsetstrokecolor{currentstroke}%
\pgfsetdash{}{0pt}%
\pgfpathmoveto{\pgfqpoint{3.925582in}{3.893502in}}%
\pgfpathlineto{\pgfqpoint{3.973490in}{3.917333in}}%
\pgfpathlineto{\pgfqpoint{4.000022in}{3.926332in}}%
\pgfpathlineto{\pgfqpoint{4.115457in}{3.938326in}}%
\pgfpathlineto{\pgfqpoint{4.127682in}{3.934353in}}%
\pgfpathlineto{\pgfqpoint{4.143831in}{3.918121in}}%
\pgfpathlineto{\pgfqpoint{4.144721in}{3.903707in}}%
\pgfpathlineto{\pgfqpoint{4.250965in}{3.919317in}}%
\pgfpathlineto{\pgfqpoint{4.371607in}{3.832835in}}%
\pgfpathlineto{\pgfqpoint{4.348983in}{3.809254in}}%
\pgfpathlineto{\pgfqpoint{4.317938in}{3.754490in}}%
\pgfpathlineto{\pgfqpoint{4.326775in}{3.742720in}}%
\pgfpathlineto{\pgfqpoint{4.310214in}{3.719876in}}%
\pgfpathlineto{\pgfqpoint{4.293741in}{3.717290in}}%
\pgfpathlineto{\pgfqpoint{4.293680in}{3.703562in}}%
\pgfpathlineto{\pgfqpoint{4.281764in}{3.689198in}}%
\pgfpathlineto{\pgfqpoint{4.266934in}{3.686285in}}%
\pgfpathlineto{\pgfqpoint{4.270170in}{3.673521in}}%
\pgfpathlineto{\pgfqpoint{4.261894in}{3.663723in}}%
\pgfpathlineto{\pgfqpoint{4.242099in}{3.655259in}}%
\pgfpathlineto{\pgfqpoint{4.217970in}{3.636005in}}%
\pgfpathlineto{\pgfqpoint{4.222525in}{3.623548in}}%
\pgfpathlineto{\pgfqpoint{4.207340in}{3.615727in}}%
\pgfpathlineto{\pgfqpoint{4.194533in}{3.623950in}}%
\pgfpathlineto{\pgfqpoint{4.185167in}{3.588197in}}%
\pgfpathlineto{\pgfqpoint{4.163922in}{3.589272in}}%
\pgfpathlineto{\pgfqpoint{4.159040in}{3.608607in}}%
\pgfpathlineto{\pgfqpoint{4.143903in}{3.634896in}}%
\pgfpathlineto{\pgfqpoint{4.125101in}{3.644350in}}%
\pgfpathlineto{\pgfqpoint{4.119426in}{3.668808in}}%
\pgfpathlineto{\pgfqpoint{4.106607in}{3.690567in}}%
\pgfpathlineto{\pgfqpoint{4.088505in}{3.697128in}}%
\pgfpathlineto{\pgfqpoint{4.064872in}{3.720172in}}%
\pgfpathlineto{\pgfqpoint{4.062682in}{3.731761in}}%
\pgfpathlineto{\pgfqpoint{4.033550in}{3.747742in}}%
\pgfpathlineto{\pgfqpoint{4.024265in}{3.762518in}}%
\pgfpathlineto{\pgfqpoint{3.991596in}{3.781044in}}%
\pgfpathlineto{\pgfqpoint{3.973202in}{3.803087in}}%
\pgfpathlineto{\pgfqpoint{3.956245in}{3.834345in}}%
\pgfpathlineto{\pgfqpoint{3.938102in}{3.835717in}}%
\pgfpathlineto{\pgfqpoint{3.926096in}{3.846821in}}%
\pgfpathlineto{\pgfqpoint{3.908443in}{3.853698in}}%
\pgfpathlineto{\pgfqpoint{3.908432in}{3.866452in}}%
\pgfpathlineto{\pgfqpoint{3.925582in}{3.893502in}}%
\pgfusepath{stroke}%
\end{pgfscope}%
\begin{pgfscope}%
\pgfpathrectangle{\pgfqpoint{0.100000in}{2.413063in}}{\pgfqpoint{5.037500in}{3.427208in}}%
\pgfusepath{clip}%
\pgfsetbuttcap%
\pgfsetroundjoin%
\pgfsetlinewidth{0.501875pt}%
\definecolor{currentstroke}{rgb}{0.827451,0.827451,0.827451}%
\pgfsetstrokecolor{currentstroke}%
\pgfsetdash{}{0pt}%
\pgfpathmoveto{\pgfqpoint{2.839596in}{3.991836in}}%
\pgfpathlineto{\pgfqpoint{2.909997in}{3.992469in}}%
\pgfpathlineto{\pgfqpoint{3.003135in}{3.994167in}}%
\pgfpathlineto{\pgfqpoint{3.162287in}{3.999219in}}%
\pgfpathlineto{\pgfqpoint{3.253318in}{4.004033in}}%
\pgfpathlineto{\pgfqpoint{3.263080in}{3.991948in}}%
\pgfpathlineto{\pgfqpoint{3.262470in}{3.979183in}}%
\pgfpathlineto{\pgfqpoint{3.240418in}{3.957162in}}%
\pgfpathlineto{\pgfqpoint{3.235087in}{3.945104in}}%
\pgfpathlineto{\pgfqpoint{3.296248in}{3.949628in}}%
\pgfpathlineto{\pgfqpoint{3.296214in}{3.927350in}}%
\pgfpathlineto{\pgfqpoint{3.276593in}{3.917829in}}%
\pgfpathlineto{\pgfqpoint{3.276871in}{3.902691in}}%
\pgfpathlineto{\pgfqpoint{3.269502in}{3.891811in}}%
\pgfpathlineto{\pgfqpoint{3.264679in}{3.868699in}}%
\pgfpathlineto{\pgfqpoint{3.269658in}{3.849634in}}%
\pgfpathlineto{\pgfqpoint{3.247837in}{3.831521in}}%
\pgfpathlineto{\pgfqpoint{3.254223in}{3.822611in}}%
\pgfpathlineto{\pgfqpoint{3.233756in}{3.807568in}}%
\pgfpathlineto{\pgfqpoint{3.234537in}{3.792851in}}%
\pgfpathlineto{\pgfqpoint{3.218105in}{3.756647in}}%
\pgfpathlineto{\pgfqpoint{3.202705in}{3.745680in}}%
\pgfpathlineto{\pgfqpoint{3.201219in}{3.730598in}}%
\pgfpathlineto{\pgfqpoint{3.169249in}{3.673932in}}%
\pgfpathlineto{\pgfqpoint{3.180079in}{3.642373in}}%
\pgfpathlineto{\pgfqpoint{3.177068in}{3.634020in}}%
\pgfpathlineto{\pgfqpoint{3.183224in}{3.615659in}}%
\pgfpathlineto{\pgfqpoint{3.176610in}{3.598163in}}%
\pgfpathlineto{\pgfqpoint{3.089189in}{3.594554in}}%
\pgfpathlineto{\pgfqpoint{2.975699in}{3.592684in}}%
\pgfpathlineto{\pgfqpoint{2.897420in}{3.591978in}}%
\pgfpathlineto{\pgfqpoint{2.897040in}{3.653613in}}%
\pgfpathlineto{\pgfqpoint{2.877599in}{3.657651in}}%
\pgfpathlineto{\pgfqpoint{2.864771in}{3.652378in}}%
\pgfpathlineto{\pgfqpoint{2.854521in}{3.662061in}}%
\pgfpathlineto{\pgfqpoint{2.855579in}{3.727350in}}%
\pgfpathlineto{\pgfqpoint{2.857873in}{3.866362in}}%
\pgfpathlineto{\pgfqpoint{2.846754in}{3.947764in}}%
\pgfpathlineto{\pgfqpoint{2.839596in}{3.991836in}}%
\pgfusepath{stroke}%
\end{pgfscope}%
\begin{pgfscope}%
\pgfpathrectangle{\pgfqpoint{0.100000in}{2.413063in}}{\pgfqpoint{5.037500in}{3.427208in}}%
\pgfusepath{clip}%
\pgfsetbuttcap%
\pgfsetroundjoin%
\pgfsetlinewidth{0.501875pt}%
\definecolor{currentstroke}{rgb}{0.827451,0.827451,0.827451}%
\pgfsetstrokecolor{currentstroke}%
\pgfsetdash{}{0pt}%
\pgfpathmoveto{\pgfqpoint{2.897420in}{3.591978in}}%
\pgfpathlineto{\pgfqpoint{2.975699in}{3.592684in}}%
\pgfpathlineto{\pgfqpoint{3.089189in}{3.594554in}}%
\pgfpathlineto{\pgfqpoint{3.176610in}{3.598163in}}%
\pgfpathlineto{\pgfqpoint{3.172421in}{3.589042in}}%
\pgfpathlineto{\pgfqpoint{3.183467in}{3.538809in}}%
\pgfpathlineto{\pgfqpoint{3.198727in}{3.516866in}}%
\pgfpathlineto{\pgfqpoint{3.181036in}{3.504745in}}%
\pgfpathlineto{\pgfqpoint{3.190283in}{3.483688in}}%
\pgfpathlineto{\pgfqpoint{3.165334in}{3.462006in}}%
\pgfpathlineto{\pgfqpoint{3.157876in}{3.429096in}}%
\pgfpathlineto{\pgfqpoint{3.144244in}{3.412144in}}%
\pgfpathlineto{\pgfqpoint{3.145979in}{3.394663in}}%
\pgfpathlineto{\pgfqpoint{3.138158in}{3.390955in}}%
\pgfpathlineto{\pgfqpoint{3.146007in}{3.372761in}}%
\pgfpathlineto{\pgfqpoint{3.139657in}{3.363127in}}%
\pgfpathlineto{\pgfqpoint{3.246390in}{3.367755in}}%
\pgfpathlineto{\pgfqpoint{3.328701in}{3.372823in}}%
\pgfpathlineto{\pgfqpoint{3.320014in}{3.334122in}}%
\pgfpathlineto{\pgfqpoint{3.325940in}{3.319823in}}%
\pgfpathlineto{\pgfqpoint{3.338225in}{3.307874in}}%
\pgfpathlineto{\pgfqpoint{3.350143in}{3.279367in}}%
\pgfpathlineto{\pgfqpoint{3.312500in}{3.285934in}}%
\pgfpathlineto{\pgfqpoint{3.298615in}{3.296732in}}%
\pgfpathlineto{\pgfqpoint{3.282070in}{3.297246in}}%
\pgfpathlineto{\pgfqpoint{3.264677in}{3.273602in}}%
\pgfpathlineto{\pgfqpoint{3.268143in}{3.262835in}}%
\pgfpathlineto{\pgfqpoint{3.323709in}{3.256321in}}%
\pgfpathlineto{\pgfqpoint{3.338256in}{3.243945in}}%
\pgfpathlineto{\pgfqpoint{3.361928in}{3.237578in}}%
\pgfpathlineto{\pgfqpoint{3.351589in}{3.222926in}}%
\pgfpathlineto{\pgfqpoint{3.334171in}{3.210148in}}%
\pgfpathlineto{\pgfqpoint{3.380146in}{3.181411in}}%
\pgfpathlineto{\pgfqpoint{3.401562in}{3.176921in}}%
\pgfpathlineto{\pgfqpoint{3.410468in}{3.153432in}}%
\pgfpathlineto{\pgfqpoint{3.390304in}{3.149111in}}%
\pgfpathlineto{\pgfqpoint{3.352407in}{3.178654in}}%
\pgfpathlineto{\pgfqpoint{3.337590in}{3.182760in}}%
\pgfpathlineto{\pgfqpoint{3.330420in}{3.194431in}}%
\pgfpathlineto{\pgfqpoint{3.308442in}{3.189641in}}%
\pgfpathlineto{\pgfqpoint{3.301561in}{3.174602in}}%
\pgfpathlineto{\pgfqpoint{3.305919in}{3.157845in}}%
\pgfpathlineto{\pgfqpoint{3.291123in}{3.147957in}}%
\pgfpathlineto{\pgfqpoint{3.277619in}{3.172318in}}%
\pgfpathlineto{\pgfqpoint{3.253967in}{3.165054in}}%
\pgfpathlineto{\pgfqpoint{3.245105in}{3.150519in}}%
\pgfpathlineto{\pgfqpoint{3.228308in}{3.154653in}}%
\pgfpathlineto{\pgfqpoint{3.217569in}{3.173009in}}%
\pgfpathlineto{\pgfqpoint{3.189078in}{3.179071in}}%
\pgfpathlineto{\pgfqpoint{3.183698in}{3.188622in}}%
\pgfpathlineto{\pgfqpoint{3.154014in}{3.205261in}}%
\pgfpathlineto{\pgfqpoint{3.146645in}{3.219809in}}%
\pgfpathlineto{\pgfqpoint{3.121782in}{3.213864in}}%
\pgfpathlineto{\pgfqpoint{3.124940in}{3.227183in}}%
\pgfpathlineto{\pgfqpoint{3.094036in}{3.213520in}}%
\pgfpathlineto{\pgfqpoint{3.102338in}{3.199344in}}%
\pgfpathlineto{\pgfqpoint{3.078472in}{3.190963in}}%
\pgfpathlineto{\pgfqpoint{3.046862in}{3.195569in}}%
\pgfpathlineto{\pgfqpoint{2.982884in}{3.217484in}}%
\pgfpathlineto{\pgfqpoint{2.933518in}{3.213134in}}%
\pgfpathlineto{\pgfqpoint{2.929239in}{3.241920in}}%
\pgfpathlineto{\pgfqpoint{2.935438in}{3.255055in}}%
\pgfpathlineto{\pgfqpoint{2.934908in}{3.275954in}}%
\pgfpathlineto{\pgfqpoint{2.928809in}{3.280933in}}%
\pgfpathlineto{\pgfqpoint{2.930876in}{3.304788in}}%
\pgfpathlineto{\pgfqpoint{2.948856in}{3.337967in}}%
\pgfpathlineto{\pgfqpoint{2.951156in}{3.350504in}}%
\pgfpathlineto{\pgfqpoint{2.948193in}{3.380114in}}%
\pgfpathlineto{\pgfqpoint{2.934966in}{3.394061in}}%
\pgfpathlineto{\pgfqpoint{2.928060in}{3.420103in}}%
\pgfpathlineto{\pgfqpoint{2.919562in}{3.426082in}}%
\pgfpathlineto{\pgfqpoint{2.923366in}{3.440204in}}%
\pgfpathlineto{\pgfqpoint{2.912544in}{3.461313in}}%
\pgfpathlineto{\pgfqpoint{2.899043in}{3.472768in}}%
\pgfpathlineto{\pgfqpoint{2.897420in}{3.591978in}}%
\pgfusepath{stroke}%
\end{pgfscope}%
\begin{pgfscope}%
\pgfpathrectangle{\pgfqpoint{0.100000in}{2.413063in}}{\pgfqpoint{5.037500in}{3.427208in}}%
\pgfusepath{clip}%
\pgfsetbuttcap%
\pgfsetroundjoin%
\pgfsetlinewidth{0.501875pt}%
\definecolor{currentstroke}{rgb}{0.827451,0.827451,0.827451}%
\pgfsetstrokecolor{currentstroke}%
\pgfsetdash{}{0pt}%
\pgfpathmoveto{\pgfqpoint{3.106777in}{3.198259in}}%
\pgfpathlineto{\pgfqpoint{3.118102in}{3.204994in}}%
\pgfpathlineto{\pgfqpoint{3.131847in}{3.197031in}}%
\pgfpathlineto{\pgfqpoint{3.124181in}{3.186073in}}%
\pgfpathlineto{\pgfqpoint{3.106777in}{3.198259in}}%
\pgfusepath{stroke}%
\end{pgfscope}%
\begin{pgfscope}%
\pgfpathrectangle{\pgfqpoint{0.100000in}{2.413063in}}{\pgfqpoint{5.037500in}{3.427208in}}%
\pgfusepath{clip}%
\pgfsetbuttcap%
\pgfsetroundjoin%
\pgfsetlinewidth{0.501875pt}%
\definecolor{currentstroke}{rgb}{0.827451,0.827451,0.827451}%
\pgfsetstrokecolor{currentstroke}%
\pgfsetdash{}{0pt}%
\pgfpathmoveto{\pgfqpoint{3.563787in}{3.328159in}}%
\pgfpathlineto{\pgfqpoint{3.563992in}{3.349936in}}%
\pgfpathlineto{\pgfqpoint{3.550461in}{3.358206in}}%
\pgfpathlineto{\pgfqpoint{3.539374in}{3.372285in}}%
\pgfpathlineto{\pgfqpoint{3.540884in}{3.387076in}}%
\pgfpathlineto{\pgfqpoint{3.661580in}{3.396445in}}%
\pgfpathlineto{\pgfqpoint{3.798503in}{3.411653in}}%
\pgfpathlineto{\pgfqpoint{3.815967in}{3.379825in}}%
\pgfpathlineto{\pgfqpoint{3.874037in}{3.383913in}}%
\pgfpathlineto{\pgfqpoint{3.989179in}{3.390195in}}%
\pgfpathlineto{\pgfqpoint{4.080494in}{3.396281in}}%
\pgfpathlineto{\pgfqpoint{4.089462in}{3.373357in}}%
\pgfpathlineto{\pgfqpoint{4.100548in}{3.375235in}}%
\pgfpathlineto{\pgfqpoint{4.101781in}{3.399780in}}%
\pgfpathlineto{\pgfqpoint{4.096108in}{3.420892in}}%
\pgfpathlineto{\pgfqpoint{4.103711in}{3.429829in}}%
\pgfpathlineto{\pgfqpoint{4.121284in}{3.425022in}}%
\pgfpathlineto{\pgfqpoint{4.146508in}{3.424034in}}%
\pgfpathlineto{\pgfqpoint{4.150386in}{3.405172in}}%
\pgfpathlineto{\pgfqpoint{4.158145in}{3.394394in}}%
\pgfpathlineto{\pgfqpoint{4.164205in}{3.370814in}}%
\pgfpathlineto{\pgfqpoint{4.182979in}{3.334321in}}%
\pgfpathlineto{\pgfqpoint{4.183073in}{3.324452in}}%
\pgfpathlineto{\pgfqpoint{4.210820in}{3.281602in}}%
\pgfpathlineto{\pgfqpoint{4.213504in}{3.272750in}}%
\pgfpathlineto{\pgfqpoint{4.255129in}{3.212232in}}%
\pgfpathlineto{\pgfqpoint{4.248535in}{3.211255in}}%
\pgfpathlineto{\pgfqpoint{4.266083in}{3.168210in}}%
\pgfpathlineto{\pgfqpoint{4.314290in}{3.093379in}}%
\pgfpathlineto{\pgfqpoint{4.342279in}{3.038358in}}%
\pgfpathlineto{\pgfqpoint{4.351634in}{3.028725in}}%
\pgfpathlineto{\pgfqpoint{4.367664in}{2.994190in}}%
\pgfpathlineto{\pgfqpoint{4.373225in}{2.938891in}}%
\pgfpathlineto{\pgfqpoint{4.375438in}{2.897542in}}%
\pgfpathlineto{\pgfqpoint{4.372765in}{2.871067in}}%
\pgfpathlineto{\pgfqpoint{4.364170in}{2.852141in}}%
\pgfpathlineto{\pgfqpoint{4.368164in}{2.827323in}}%
\pgfpathlineto{\pgfqpoint{4.358910in}{2.807678in}}%
\pgfpathlineto{\pgfqpoint{4.331435in}{2.791570in}}%
\pgfpathlineto{\pgfqpoint{4.313566in}{2.792746in}}%
\pgfpathlineto{\pgfqpoint{4.301962in}{2.784325in}}%
\pgfpathlineto{\pgfqpoint{4.298638in}{2.806808in}}%
\pgfpathlineto{\pgfqpoint{4.279430in}{2.812743in}}%
\pgfpathlineto{\pgfqpoint{4.262210in}{2.844115in}}%
\pgfpathlineto{\pgfqpoint{4.260267in}{2.858401in}}%
\pgfpathlineto{\pgfqpoint{4.229450in}{2.867076in}}%
\pgfpathlineto{\pgfqpoint{4.209592in}{2.865198in}}%
\pgfpathlineto{\pgfqpoint{4.198356in}{2.885927in}}%
\pgfpathlineto{\pgfqpoint{4.185492in}{2.923185in}}%
\pgfpathlineto{\pgfqpoint{4.167640in}{2.930738in}}%
\pgfpathlineto{\pgfqpoint{4.157962in}{2.952089in}}%
\pgfpathlineto{\pgfqpoint{4.134293in}{2.964672in}}%
\pgfpathlineto{\pgfqpoint{4.120612in}{2.980412in}}%
\pgfpathlineto{\pgfqpoint{4.097012in}{3.023203in}}%
\pgfpathlineto{\pgfqpoint{4.094236in}{3.053321in}}%
\pgfpathlineto{\pgfqpoint{4.107097in}{3.072591in}}%
\pgfpathlineto{\pgfqpoint{4.105811in}{3.085982in}}%
\pgfpathlineto{\pgfqpoint{4.090949in}{3.087413in}}%
\pgfpathlineto{\pgfqpoint{4.078532in}{3.096726in}}%
\pgfpathlineto{\pgfqpoint{4.071732in}{3.085353in}}%
\pgfpathlineto{\pgfqpoint{4.083649in}{3.076086in}}%
\pgfpathlineto{\pgfqpoint{4.074976in}{3.059422in}}%
\pgfpathlineto{\pgfqpoint{4.060944in}{3.073253in}}%
\pgfpathlineto{\pgfqpoint{4.062571in}{3.111608in}}%
\pgfpathlineto{\pgfqpoint{4.069340in}{3.142725in}}%
\pgfpathlineto{\pgfqpoint{4.065861in}{3.196118in}}%
\pgfpathlineto{\pgfqpoint{4.051874in}{3.208836in}}%
\pgfpathlineto{\pgfqpoint{4.044840in}{3.225155in}}%
\pgfpathlineto{\pgfqpoint{4.020746in}{3.224791in}}%
\pgfpathlineto{\pgfqpoint{3.996928in}{3.251644in}}%
\pgfpathlineto{\pgfqpoint{3.980959in}{3.259692in}}%
\pgfpathlineto{\pgfqpoint{3.976224in}{3.276697in}}%
\pgfpathlineto{\pgfqpoint{3.960604in}{3.282669in}}%
\pgfpathlineto{\pgfqpoint{3.947660in}{3.301445in}}%
\pgfpathlineto{\pgfqpoint{3.913464in}{3.316770in}}%
\pgfpathlineto{\pgfqpoint{3.886879in}{3.317162in}}%
\pgfpathlineto{\pgfqpoint{3.875306in}{3.311276in}}%
\pgfpathlineto{\pgfqpoint{3.878182in}{3.292901in}}%
\pgfpathlineto{\pgfqpoint{3.866114in}{3.293771in}}%
\pgfpathlineto{\pgfqpoint{3.828991in}{3.268080in}}%
\pgfpathlineto{\pgfqpoint{3.784400in}{3.257897in}}%
\pgfpathlineto{\pgfqpoint{3.783668in}{3.270496in}}%
\pgfpathlineto{\pgfqpoint{3.773786in}{3.282810in}}%
\pgfpathlineto{\pgfqpoint{3.747272in}{3.299810in}}%
\pgfpathlineto{\pgfqpoint{3.709215in}{3.317240in}}%
\pgfpathlineto{\pgfqpoint{3.667877in}{3.326466in}}%
\pgfpathlineto{\pgfqpoint{3.660118in}{3.339013in}}%
\pgfpathlineto{\pgfqpoint{3.645213in}{3.328549in}}%
\pgfpathlineto{\pgfqpoint{3.627266in}{3.326240in}}%
\pgfpathlineto{\pgfqpoint{3.564793in}{3.309780in}}%
\pgfpathlineto{\pgfqpoint{3.563787in}{3.328159in}}%
\pgfusepath{stroke}%
\end{pgfscope}%
\begin{pgfscope}%
\pgfpathrectangle{\pgfqpoint{0.100000in}{2.413063in}}{\pgfqpoint{5.037500in}{3.427208in}}%
\pgfusepath{clip}%
\pgfsetbuttcap%
\pgfsetroundjoin%
\pgfsetlinewidth{0.501875pt}%
\definecolor{currentstroke}{rgb}{0.827451,0.827451,0.827451}%
\pgfsetstrokecolor{currentstroke}%
\pgfsetdash{}{0pt}%
\pgfpathmoveto{\pgfqpoint{4.261195in}{3.213093in}}%
\pgfpathlineto{\pgfqpoint{4.277358in}{3.194291in}}%
\pgfpathlineto{\pgfqpoint{4.258688in}{3.193108in}}%
\pgfpathlineto{\pgfqpoint{4.261195in}{3.213093in}}%
\pgfusepath{stroke}%
\end{pgfscope}%
\begin{pgfscope}%
\pgfpathrectangle{\pgfqpoint{0.100000in}{2.413063in}}{\pgfqpoint{5.037500in}{3.427208in}}%
\pgfusepath{clip}%
\pgfsetbuttcap%
\pgfsetroundjoin%
\pgfsetlinewidth{0.501875pt}%
\definecolor{currentstroke}{rgb}{0.827451,0.827451,0.827451}%
\pgfsetstrokecolor{currentstroke}%
\pgfsetdash{}{0pt}%
\pgfpathmoveto{\pgfqpoint{3.309667in}{5.359952in}}%
\pgfpathlineto{\pgfqpoint{3.300898in}{5.342883in}}%
\pgfpathlineto{\pgfqpoint{3.279976in}{5.332921in}}%
\pgfpathlineto{\pgfqpoint{3.270914in}{5.319513in}}%
\pgfpathlineto{\pgfqpoint{3.260299in}{5.329170in}}%
\pgfpathlineto{\pgfqpoint{3.309667in}{5.359952in}}%
\pgfusepath{stroke}%
\end{pgfscope}%
\begin{pgfscope}%
\pgfpathrectangle{\pgfqpoint{0.100000in}{2.413063in}}{\pgfqpoint{5.037500in}{3.427208in}}%
\pgfusepath{clip}%
\pgfsetbuttcap%
\pgfsetroundjoin%
\pgfsetlinewidth{0.501875pt}%
\definecolor{currentstroke}{rgb}{0.827451,0.827451,0.827451}%
\pgfsetstrokecolor{currentstroke}%
\pgfsetdash{}{0pt}%
\pgfpathmoveto{\pgfqpoint{3.316781in}{5.257228in}}%
\pgfpathlineto{\pgfqpoint{3.338188in}{5.277198in}}%
\pgfpathlineto{\pgfqpoint{3.371207in}{5.282428in}}%
\pgfpathlineto{\pgfqpoint{3.362113in}{5.268539in}}%
\pgfpathlineto{\pgfqpoint{3.339476in}{5.248453in}}%
\pgfpathlineto{\pgfqpoint{3.326242in}{5.222673in}}%
\pgfpathlineto{\pgfqpoint{3.320637in}{5.236656in}}%
\pgfpathlineto{\pgfqpoint{3.310685in}{5.238663in}}%
\pgfpathlineto{\pgfqpoint{3.309685in}{5.251287in}}%
\pgfpathlineto{\pgfqpoint{3.316781in}{5.257228in}}%
\pgfusepath{stroke}%
\end{pgfscope}%
\begin{pgfscope}%
\pgfpathrectangle{\pgfqpoint{0.100000in}{2.413063in}}{\pgfqpoint{5.037500in}{3.427208in}}%
\pgfusepath{clip}%
\pgfsetbuttcap%
\pgfsetroundjoin%
\pgfsetlinewidth{0.501875pt}%
\definecolor{currentstroke}{rgb}{0.827451,0.827451,0.827451}%
\pgfsetstrokecolor{currentstroke}%
\pgfsetdash{}{0pt}%
\pgfpathmoveto{\pgfqpoint{3.402044in}{5.013479in}}%
\pgfpathlineto{\pgfqpoint{3.396343in}{5.019806in}}%
\pgfpathlineto{\pgfqpoint{3.402337in}{5.037617in}}%
\pgfpathlineto{\pgfqpoint{3.384545in}{5.038752in}}%
\pgfpathlineto{\pgfqpoint{3.389261in}{5.054109in}}%
\pgfpathlineto{\pgfqpoint{3.386328in}{5.078566in}}%
\pgfpathlineto{\pgfqpoint{3.370364in}{5.087046in}}%
\pgfpathlineto{\pgfqpoint{3.353658in}{5.104206in}}%
\pgfpathlineto{\pgfqpoint{3.327916in}{5.109231in}}%
\pgfpathlineto{\pgfqpoint{3.302863in}{5.109087in}}%
\pgfpathlineto{\pgfqpoint{3.278223in}{5.121266in}}%
\pgfpathlineto{\pgfqpoint{3.195789in}{5.139018in}}%
\pgfpathlineto{\pgfqpoint{3.186780in}{5.157828in}}%
\pgfpathlineto{\pgfqpoint{3.170712in}{5.164247in}}%
\pgfpathlineto{\pgfqpoint{3.201064in}{5.178648in}}%
\pgfpathlineto{\pgfqpoint{3.218211in}{5.196616in}}%
\pgfpathlineto{\pgfqpoint{3.250147in}{5.201489in}}%
\pgfpathlineto{\pgfqpoint{3.269800in}{5.219789in}}%
\pgfpathlineto{\pgfqpoint{3.280110in}{5.220522in}}%
\pgfpathlineto{\pgfqpoint{3.287979in}{5.233574in}}%
\pgfpathlineto{\pgfqpoint{3.307161in}{5.249035in}}%
\pgfpathlineto{\pgfqpoint{3.308833in}{5.238107in}}%
\pgfpathlineto{\pgfqpoint{3.317476in}{5.235850in}}%
\pgfpathlineto{\pgfqpoint{3.323979in}{5.222817in}}%
\pgfpathlineto{\pgfqpoint{3.325151in}{5.200597in}}%
\pgfpathlineto{\pgfqpoint{3.344696in}{5.213876in}}%
\pgfpathlineto{\pgfqpoint{3.367471in}{5.216649in}}%
\pgfpathlineto{\pgfqpoint{3.386912in}{5.209693in}}%
\pgfpathlineto{\pgfqpoint{3.413276in}{5.173524in}}%
\pgfpathlineto{\pgfqpoint{3.442071in}{5.179304in}}%
\pgfpathlineto{\pgfqpoint{3.453772in}{5.169624in}}%
\pgfpathlineto{\pgfqpoint{3.472590in}{5.168760in}}%
\pgfpathlineto{\pgfqpoint{3.485093in}{5.186120in}}%
\pgfpathlineto{\pgfqpoint{3.508817in}{5.201466in}}%
\pgfpathlineto{\pgfqpoint{3.531619in}{5.206272in}}%
\pgfpathlineto{\pgfqpoint{3.559912in}{5.206773in}}%
\pgfpathlineto{\pgfqpoint{3.580587in}{5.218627in}}%
\pgfpathlineto{\pgfqpoint{3.597460in}{5.213167in}}%
\pgfpathlineto{\pgfqpoint{3.601054in}{5.188095in}}%
\pgfpathlineto{\pgfqpoint{3.637253in}{5.184169in}}%
\pgfpathlineto{\pgfqpoint{3.656916in}{5.195905in}}%
\pgfpathlineto{\pgfqpoint{3.670546in}{5.169467in}}%
\pgfpathlineto{\pgfqpoint{3.696553in}{5.138895in}}%
\pgfpathlineto{\pgfqpoint{3.660152in}{5.139072in}}%
\pgfpathlineto{\pgfqpoint{3.648651in}{5.135292in}}%
\pgfpathlineto{\pgfqpoint{3.632886in}{5.140204in}}%
\pgfpathlineto{\pgfqpoint{3.631825in}{5.119039in}}%
\pgfpathlineto{\pgfqpoint{3.603200in}{5.135579in}}%
\pgfpathlineto{\pgfqpoint{3.566388in}{5.140633in}}%
\pgfpathlineto{\pgfqpoint{3.556253in}{5.124526in}}%
\pgfpathlineto{\pgfqpoint{3.508088in}{5.116693in}}%
\pgfpathlineto{\pgfqpoint{3.502546in}{5.103048in}}%
\pgfpathlineto{\pgfqpoint{3.469029in}{5.098953in}}%
\pgfpathlineto{\pgfqpoint{3.458896in}{5.085082in}}%
\pgfpathlineto{\pgfqpoint{3.441168in}{5.081363in}}%
\pgfpathlineto{\pgfqpoint{3.427007in}{5.048470in}}%
\pgfpathlineto{\pgfqpoint{3.409079in}{5.016614in}}%
\pgfpathlineto{\pgfqpoint{3.402044in}{5.013479in}}%
\pgfusepath{stroke}%
\end{pgfscope}%
\begin{pgfscope}%
\pgfpathrectangle{\pgfqpoint{0.100000in}{2.413063in}}{\pgfqpoint{5.037500in}{3.427208in}}%
\pgfusepath{clip}%
\pgfsetbuttcap%
\pgfsetroundjoin%
\pgfsetlinewidth{0.501875pt}%
\definecolor{currentstroke}{rgb}{0.827451,0.827451,0.827451}%
\pgfsetstrokecolor{currentstroke}%
\pgfsetdash{}{0pt}%
\pgfpathmoveto{\pgfqpoint{3.505233in}{4.631021in}}%
\pgfpathlineto{\pgfqpoint{3.522326in}{4.649013in}}%
\pgfpathlineto{\pgfqpoint{3.530113in}{4.675098in}}%
\pgfpathlineto{\pgfqpoint{3.539374in}{4.690218in}}%
\pgfpathlineto{\pgfqpoint{3.545053in}{4.710794in}}%
\pgfpathlineto{\pgfqpoint{3.546819in}{4.751821in}}%
\pgfpathlineto{\pgfqpoint{3.538268in}{4.791140in}}%
\pgfpathlineto{\pgfqpoint{3.510013in}{4.851348in}}%
\pgfpathlineto{\pgfqpoint{3.517665in}{4.870279in}}%
\pgfpathlineto{\pgfqpoint{3.507697in}{4.896446in}}%
\pgfpathlineto{\pgfqpoint{3.524796in}{4.932634in}}%
\pgfpathlineto{\pgfqpoint{3.522010in}{4.972984in}}%
\pgfpathlineto{\pgfqpoint{3.533920in}{4.978063in}}%
\pgfpathlineto{\pgfqpoint{3.535416in}{4.997262in}}%
\pgfpathlineto{\pgfqpoint{3.556558in}{5.009585in}}%
\pgfpathlineto{\pgfqpoint{3.568515in}{5.007596in}}%
\pgfpathlineto{\pgfqpoint{3.571876in}{4.987005in}}%
\pgfpathlineto{\pgfqpoint{3.581208in}{4.986198in}}%
\pgfpathlineto{\pgfqpoint{3.589723in}{5.015898in}}%
\pgfpathlineto{\pgfqpoint{3.586829in}{5.038997in}}%
\pgfpathlineto{\pgfqpoint{3.592365in}{5.052261in}}%
\pgfpathlineto{\pgfqpoint{3.622298in}{5.065964in}}%
\pgfpathlineto{\pgfqpoint{3.608658in}{5.070841in}}%
\pgfpathlineto{\pgfqpoint{3.604215in}{5.082612in}}%
\pgfpathlineto{\pgfqpoint{3.614041in}{5.103283in}}%
\pgfpathlineto{\pgfqpoint{3.633417in}{5.110422in}}%
\pgfpathlineto{\pgfqpoint{3.655918in}{5.098226in}}%
\pgfpathlineto{\pgfqpoint{3.677133in}{5.098074in}}%
\pgfpathlineto{\pgfqpoint{3.686985in}{5.083827in}}%
\pgfpathlineto{\pgfqpoint{3.701842in}{5.084817in}}%
\pgfpathlineto{\pgfqpoint{3.747705in}{5.065012in}}%
\pgfpathlineto{\pgfqpoint{3.756869in}{5.045840in}}%
\pgfpathlineto{\pgfqpoint{3.746861in}{5.039200in}}%
\pgfpathlineto{\pgfqpoint{3.749945in}{5.024832in}}%
\pgfpathlineto{\pgfqpoint{3.759831in}{5.018442in}}%
\pgfpathlineto{\pgfqpoint{3.765341in}{5.000779in}}%
\pgfpathlineto{\pgfqpoint{3.764570in}{4.957748in}}%
\pgfpathlineto{\pgfqpoint{3.751517in}{4.947469in}}%
\pgfpathlineto{\pgfqpoint{3.748587in}{4.924866in}}%
\pgfpathlineto{\pgfqpoint{3.724374in}{4.903977in}}%
\pgfpathlineto{\pgfqpoint{3.725811in}{4.878692in}}%
\pgfpathlineto{\pgfqpoint{3.747033in}{4.869835in}}%
\pgfpathlineto{\pgfqpoint{3.770974in}{4.901382in}}%
\pgfpathlineto{\pgfqpoint{3.772953in}{4.912824in}}%
\pgfpathlineto{\pgfqpoint{3.802811in}{4.931842in}}%
\pgfpathlineto{\pgfqpoint{3.821798in}{4.923033in}}%
\pgfpathlineto{\pgfqpoint{3.833713in}{4.903124in}}%
\pgfpathlineto{\pgfqpoint{3.852934in}{4.834001in}}%
\pgfpathlineto{\pgfqpoint{3.863121in}{4.812132in}}%
\pgfpathlineto{\pgfqpoint{3.859932in}{4.803091in}}%
\pgfpathlineto{\pgfqpoint{3.860241in}{4.772307in}}%
\pgfpathlineto{\pgfqpoint{3.841712in}{4.775262in}}%
\pgfpathlineto{\pgfqpoint{3.831247in}{4.752258in}}%
\pgfpathlineto{\pgfqpoint{3.829796in}{4.736619in}}%
\pgfpathlineto{\pgfqpoint{3.815800in}{4.726580in}}%
\pgfpathlineto{\pgfqpoint{3.812690in}{4.696081in}}%
\pgfpathlineto{\pgfqpoint{3.792336in}{4.657553in}}%
\pgfpathlineto{\pgfqpoint{3.680994in}{4.640971in}}%
\pgfpathlineto{\pgfqpoint{3.680336in}{4.648259in}}%
\pgfpathlineto{\pgfqpoint{3.605857in}{4.640409in}}%
\pgfpathlineto{\pgfqpoint{3.505233in}{4.631021in}}%
\pgfusepath{stroke}%
\end{pgfscope}%
\begin{pgfscope}%
\pgfpathrectangle{\pgfqpoint{0.100000in}{2.413063in}}{\pgfqpoint{5.037500in}{3.427208in}}%
\pgfusepath{clip}%
\pgfsetbuttcap%
\pgfsetroundjoin%
\pgfsetlinewidth{0.501875pt}%
\definecolor{currentstroke}{rgb}{0.827451,0.827451,0.827451}%
\pgfsetstrokecolor{currentstroke}%
\pgfsetdash{}{0pt}%
\pgfpathmoveto{\pgfqpoint{0.000000in}{0.000000in}}%
\pgfusepath{stroke}%
\end{pgfscope}%
\begin{pgfscope}%
\pgfpathrectangle{\pgfqpoint{0.100000in}{2.413063in}}{\pgfqpoint{5.037500in}{3.427208in}}%
\pgfusepath{clip}%
\pgfsetbuttcap%
\pgfsetroundjoin%
\pgfsetlinewidth{0.501875pt}%
\definecolor{currentstroke}{rgb}{0.827451,0.827451,0.827451}%
\pgfsetstrokecolor{currentstroke}%
\pgfsetdash{}{0pt}%
\pgfusepath{stroke}%
\end{pgfscope}%
\begin{pgfscope}%
\pgfpathrectangle{\pgfqpoint{0.100000in}{2.413063in}}{\pgfqpoint{5.037500in}{3.427208in}}%
\pgfusepath{clip}%
\pgfsetbuttcap%
\pgfsetroundjoin%
\pgfsetlinewidth{0.501875pt}%
\definecolor{currentstroke}{rgb}{0.827451,0.827451,0.827451}%
\pgfsetstrokecolor{currentstroke}%
\pgfsetdash{}{0pt}%
\pgfusepath{stroke}%
\end{pgfscope}%
\begin{pgfscope}%
\pgfpathrectangle{\pgfqpoint{0.100000in}{2.413063in}}{\pgfqpoint{5.037500in}{3.427208in}}%
\pgfusepath{clip}%
\pgfsetbuttcap%
\pgfsetroundjoin%
\pgfsetlinewidth{0.501875pt}%
\definecolor{currentstroke}{rgb}{0.827451,0.827451,0.827451}%
\pgfsetstrokecolor{currentstroke}%
\pgfsetdash{}{0pt}%
\pgfusepath{stroke}%
\end{pgfscope}%
\begin{pgfscope}%
\pgfpathrectangle{\pgfqpoint{0.100000in}{2.413063in}}{\pgfqpoint{5.037500in}{3.427208in}}%
\pgfusepath{clip}%
\pgfsetbuttcap%
\pgfsetroundjoin%
\pgfsetlinewidth{0.501875pt}%
\definecolor{currentstroke}{rgb}{0.827451,0.827451,0.827451}%
\pgfsetstrokecolor{currentstroke}%
\pgfsetdash{}{0pt}%
\pgfusepath{stroke}%
\end{pgfscope}%
\begin{pgfscope}%
\pgfpathrectangle{\pgfqpoint{0.100000in}{2.413063in}}{\pgfqpoint{5.037500in}{3.427208in}}%
\pgfusepath{clip}%
\pgfsetbuttcap%
\pgfsetroundjoin%
\pgfsetlinewidth{0.501875pt}%
\definecolor{currentstroke}{rgb}{0.827451,0.827451,0.827451}%
\pgfsetstrokecolor{currentstroke}%
\pgfsetdash{}{0pt}%
\pgfusepath{stroke}%
\end{pgfscope}%
\begin{pgfscope}%
\pgfpathrectangle{\pgfqpoint{0.100000in}{2.413063in}}{\pgfqpoint{5.037500in}{3.427208in}}%
\pgfusepath{clip}%
\pgfsetbuttcap%
\pgfsetroundjoin%
\pgfsetlinewidth{0.501875pt}%
\definecolor{currentstroke}{rgb}{0.827451,0.827451,0.827451}%
\pgfsetstrokecolor{currentstroke}%
\pgfsetdash{}{0pt}%
\pgfusepath{stroke}%
\end{pgfscope}%
\begin{pgfscope}%
\pgfpathrectangle{\pgfqpoint{0.100000in}{2.413063in}}{\pgfqpoint{5.037500in}{3.427208in}}%
\pgfusepath{clip}%
\pgfsetbuttcap%
\pgfsetroundjoin%
\pgfsetlinewidth{0.501875pt}%
\definecolor{currentstroke}{rgb}{0.827451,0.827451,0.827451}%
\pgfsetstrokecolor{currentstroke}%
\pgfsetdash{}{0pt}%
\pgfusepath{stroke}%
\end{pgfscope}%
\begin{pgfscope}%
\pgfpathrectangle{\pgfqpoint{0.100000in}{2.413063in}}{\pgfqpoint{5.037500in}{3.427208in}}%
\pgfusepath{clip}%
\pgfsetbuttcap%
\pgfsetroundjoin%
\pgfsetlinewidth{0.501875pt}%
\definecolor{currentstroke}{rgb}{0.827451,0.827451,0.827451}%
\pgfsetstrokecolor{currentstroke}%
\pgfsetdash{}{0pt}%
\pgfusepath{stroke}%
\end{pgfscope}%
\begin{pgfscope}%
\pgfpathrectangle{\pgfqpoint{0.100000in}{2.413063in}}{\pgfqpoint{5.037500in}{3.427208in}}%
\pgfusepath{clip}%
\pgfsetbuttcap%
\pgfsetroundjoin%
\pgfsetlinewidth{0.501875pt}%
\definecolor{currentstroke}{rgb}{0.827451,0.827451,0.827451}%
\pgfsetstrokecolor{currentstroke}%
\pgfsetdash{}{0pt}%
\pgfusepath{stroke}%
\end{pgfscope}%
\begin{pgfscope}%
\pgfpathrectangle{\pgfqpoint{0.100000in}{2.413063in}}{\pgfqpoint{5.037500in}{3.427208in}}%
\pgfusepath{clip}%
\pgfsetbuttcap%
\pgfsetroundjoin%
\pgfsetlinewidth{0.501875pt}%
\definecolor{currentstroke}{rgb}{0.827451,0.827451,0.827451}%
\pgfsetstrokecolor{currentstroke}%
\pgfsetdash{}{0pt}%
\pgfusepath{stroke}%
\end{pgfscope}%
\begin{pgfscope}%
\pgfpathrectangle{\pgfqpoint{0.100000in}{2.413063in}}{\pgfqpoint{5.037500in}{3.427208in}}%
\pgfusepath{clip}%
\pgfsetbuttcap%
\pgfsetroundjoin%
\pgfsetlinewidth{0.501875pt}%
\definecolor{currentstroke}{rgb}{0.827451,0.827451,0.827451}%
\pgfsetstrokecolor{currentstroke}%
\pgfsetdash{}{0pt}%
\pgfusepath{stroke}%
\end{pgfscope}%
\begin{pgfscope}%
\pgfpathrectangle{\pgfqpoint{0.100000in}{2.413063in}}{\pgfqpoint{5.037500in}{3.427208in}}%
\pgfusepath{clip}%
\pgfsetbuttcap%
\pgfsetroundjoin%
\pgfsetlinewidth{0.501875pt}%
\definecolor{currentstroke}{rgb}{0.827451,0.827451,0.827451}%
\pgfsetstrokecolor{currentstroke}%
\pgfsetdash{}{0pt}%
\pgfusepath{stroke}%
\end{pgfscope}%
\begin{pgfscope}%
\pgfpathrectangle{\pgfqpoint{0.100000in}{2.413063in}}{\pgfqpoint{5.037500in}{3.427208in}}%
\pgfusepath{clip}%
\pgfsetbuttcap%
\pgfsetroundjoin%
\pgfsetlinewidth{0.501875pt}%
\definecolor{currentstroke}{rgb}{0.827451,0.827451,0.827451}%
\pgfsetstrokecolor{currentstroke}%
\pgfsetdash{}{0pt}%
\pgfusepath{stroke}%
\end{pgfscope}%
\begin{pgfscope}%
\pgfpathrectangle{\pgfqpoint{0.100000in}{2.413063in}}{\pgfqpoint{5.037500in}{3.427208in}}%
\pgfusepath{clip}%
\pgfsetbuttcap%
\pgfsetroundjoin%
\pgfsetlinewidth{0.501875pt}%
\definecolor{currentstroke}{rgb}{0.827451,0.827451,0.827451}%
\pgfsetstrokecolor{currentstroke}%
\pgfsetdash{}{0pt}%
\pgfusepath{stroke}%
\end{pgfscope}%
\begin{pgfscope}%
\pgfpathrectangle{\pgfqpoint{0.100000in}{2.413063in}}{\pgfqpoint{5.037500in}{3.427208in}}%
\pgfusepath{clip}%
\pgfsetbuttcap%
\pgfsetroundjoin%
\pgfsetlinewidth{0.501875pt}%
\definecolor{currentstroke}{rgb}{0.827451,0.827451,0.827451}%
\pgfsetstrokecolor{currentstroke}%
\pgfsetdash{}{0pt}%
\pgfusepath{stroke}%
\end{pgfscope}%
\begin{pgfscope}%
\pgfpathrectangle{\pgfqpoint{0.100000in}{2.413063in}}{\pgfqpoint{5.037500in}{3.427208in}}%
\pgfusepath{clip}%
\pgfsetbuttcap%
\pgfsetroundjoin%
\pgfsetlinewidth{0.501875pt}%
\definecolor{currentstroke}{rgb}{0.827451,0.827451,0.827451}%
\pgfsetstrokecolor{currentstroke}%
\pgfsetdash{}{0pt}%
\pgfusepath{stroke}%
\end{pgfscope}%
\begin{pgfscope}%
\pgfpathrectangle{\pgfqpoint{0.100000in}{2.413063in}}{\pgfqpoint{5.037500in}{3.427208in}}%
\pgfusepath{clip}%
\pgfsetbuttcap%
\pgfsetroundjoin%
\pgfsetlinewidth{0.501875pt}%
\definecolor{currentstroke}{rgb}{0.827451,0.827451,0.827451}%
\pgfsetstrokecolor{currentstroke}%
\pgfsetdash{}{0pt}%
\pgfusepath{stroke}%
\end{pgfscope}%
\begin{pgfscope}%
\pgfpathrectangle{\pgfqpoint{0.100000in}{2.413063in}}{\pgfqpoint{5.037500in}{3.427208in}}%
\pgfusepath{clip}%
\pgfsetbuttcap%
\pgfsetroundjoin%
\pgfsetlinewidth{0.501875pt}%
\definecolor{currentstroke}{rgb}{0.827451,0.827451,0.827451}%
\pgfsetstrokecolor{currentstroke}%
\pgfsetdash{}{0pt}%
\pgfusepath{stroke}%
\end{pgfscope}%
\begin{pgfscope}%
\pgfpathrectangle{\pgfqpoint{0.100000in}{2.413063in}}{\pgfqpoint{5.037500in}{3.427208in}}%
\pgfusepath{clip}%
\pgfsetbuttcap%
\pgfsetroundjoin%
\pgfsetlinewidth{0.501875pt}%
\definecolor{currentstroke}{rgb}{0.827451,0.827451,0.827451}%
\pgfsetstrokecolor{currentstroke}%
\pgfsetdash{}{0pt}%
\pgfusepath{stroke}%
\end{pgfscope}%
\begin{pgfscope}%
\pgfpathrectangle{\pgfqpoint{0.100000in}{2.413063in}}{\pgfqpoint{5.037500in}{3.427208in}}%
\pgfusepath{clip}%
\pgfsetbuttcap%
\pgfsetroundjoin%
\pgfsetlinewidth{0.501875pt}%
\definecolor{currentstroke}{rgb}{0.827451,0.827451,0.827451}%
\pgfsetstrokecolor{currentstroke}%
\pgfsetdash{}{0pt}%
\pgfusepath{stroke}%
\end{pgfscope}%
\begin{pgfscope}%
\pgfpathrectangle{\pgfqpoint{0.100000in}{2.413063in}}{\pgfqpoint{5.037500in}{3.427208in}}%
\pgfusepath{clip}%
\pgfsetbuttcap%
\pgfsetroundjoin%
\pgfsetlinewidth{0.501875pt}%
\definecolor{currentstroke}{rgb}{0.827451,0.827451,0.827451}%
\pgfsetstrokecolor{currentstroke}%
\pgfsetdash{}{0pt}%
\pgfusepath{stroke}%
\end{pgfscope}%
\begin{pgfscope}%
\pgfpathrectangle{\pgfqpoint{0.100000in}{2.413063in}}{\pgfqpoint{5.037500in}{3.427208in}}%
\pgfusepath{clip}%
\pgfsetbuttcap%
\pgfsetroundjoin%
\pgfsetlinewidth{0.501875pt}%
\definecolor{currentstroke}{rgb}{0.827451,0.827451,0.827451}%
\pgfsetstrokecolor{currentstroke}%
\pgfsetdash{}{0pt}%
\pgfusepath{stroke}%
\end{pgfscope}%
\begin{pgfscope}%
\pgfpathrectangle{\pgfqpoint{0.100000in}{2.413063in}}{\pgfqpoint{5.037500in}{3.427208in}}%
\pgfusepath{clip}%
\pgfsetbuttcap%
\pgfsetroundjoin%
\pgfsetlinewidth{0.501875pt}%
\definecolor{currentstroke}{rgb}{0.827451,0.827451,0.827451}%
\pgfsetstrokecolor{currentstroke}%
\pgfsetdash{}{0pt}%
\pgfusepath{stroke}%
\end{pgfscope}%
\begin{pgfscope}%
\pgfpathrectangle{\pgfqpoint{0.100000in}{2.413063in}}{\pgfqpoint{5.037500in}{3.427208in}}%
\pgfusepath{clip}%
\pgfsetbuttcap%
\pgfsetroundjoin%
\pgfsetlinewidth{0.501875pt}%
\definecolor{currentstroke}{rgb}{0.827451,0.827451,0.827451}%
\pgfsetstrokecolor{currentstroke}%
\pgfsetdash{}{0pt}%
\pgfusepath{stroke}%
\end{pgfscope}%
\begin{pgfscope}%
\pgfpathrectangle{\pgfqpoint{0.100000in}{2.413063in}}{\pgfqpoint{5.037500in}{3.427208in}}%
\pgfusepath{clip}%
\pgfsetbuttcap%
\pgfsetroundjoin%
\pgfsetlinewidth{0.501875pt}%
\definecolor{currentstroke}{rgb}{0.827451,0.827451,0.827451}%
\pgfsetstrokecolor{currentstroke}%
\pgfsetdash{}{0pt}%
\pgfusepath{stroke}%
\end{pgfscope}%
\begin{pgfscope}%
\pgfpathrectangle{\pgfqpoint{0.100000in}{2.413063in}}{\pgfqpoint{5.037500in}{3.427208in}}%
\pgfusepath{clip}%
\pgfsetbuttcap%
\pgfsetroundjoin%
\pgfsetlinewidth{0.501875pt}%
\definecolor{currentstroke}{rgb}{0.827451,0.827451,0.827451}%
\pgfsetstrokecolor{currentstroke}%
\pgfsetdash{}{0pt}%
\pgfusepath{stroke}%
\end{pgfscope}%
\begin{pgfscope}%
\pgfpathrectangle{\pgfqpoint{0.100000in}{2.413063in}}{\pgfqpoint{5.037500in}{3.427208in}}%
\pgfusepath{clip}%
\pgfsetbuttcap%
\pgfsetroundjoin%
\pgfsetlinewidth{0.501875pt}%
\definecolor{currentstroke}{rgb}{0.827451,0.827451,0.827451}%
\pgfsetstrokecolor{currentstroke}%
\pgfsetdash{}{0pt}%
\pgfusepath{stroke}%
\end{pgfscope}%
\begin{pgfscope}%
\pgfpathrectangle{\pgfqpoint{0.100000in}{2.413063in}}{\pgfqpoint{5.037500in}{3.427208in}}%
\pgfusepath{clip}%
\pgfsetrectcap%
\pgfsetroundjoin%
\pgfsetlinewidth{1.505625pt}%
\definecolor{currentstroke}{rgb}{0.000000,0.000000,1.000000}%
\pgfsetstrokecolor{currentstroke}%
\pgfsetstrokeopacity{0.500000}%
\pgfsetdash{}{0pt}%
\pgfpathmoveto{\pgfqpoint{3.684003in}{3.713619in}}%
\pgfusepath{stroke}%
\end{pgfscope}%
\begin{pgfscope}%
\pgfpathrectangle{\pgfqpoint{0.100000in}{2.413063in}}{\pgfqpoint{5.037500in}{3.427208in}}%
\pgfusepath{clip}%
\pgfsetbuttcap%
\pgfsetroundjoin%
\definecolor{currentfill}{rgb}{0.000000,0.000000,1.000000}%
\pgfsetfillcolor{currentfill}%
\pgfsetfillopacity{0.500000}%
\pgfsetlinewidth{0.250937pt}%
\definecolor{currentstroke}{rgb}{0.000000,0.000000,0.000000}%
\pgfsetstrokecolor{currentstroke}%
\pgfsetstrokeopacity{0.500000}%
\pgfsetdash{}{0pt}%
\pgfsys@defobject{currentmarker}{\pgfqpoint{-0.013889in}{-0.013889in}}{\pgfqpoint{0.013889in}{0.013889in}}{%
\pgfpathmoveto{\pgfqpoint{0.000000in}{-0.013889in}}%
\pgfpathcurveto{\pgfqpoint{0.003683in}{-0.013889in}}{\pgfqpoint{0.007216in}{-0.012425in}}{\pgfqpoint{0.009821in}{-0.009821in}}%
\pgfpathcurveto{\pgfqpoint{0.012425in}{-0.007216in}}{\pgfqpoint{0.013889in}{-0.003683in}}{\pgfqpoint{0.013889in}{0.000000in}}%
\pgfpathcurveto{\pgfqpoint{0.013889in}{0.003683in}}{\pgfqpoint{0.012425in}{0.007216in}}{\pgfqpoint{0.009821in}{0.009821in}}%
\pgfpathcurveto{\pgfqpoint{0.007216in}{0.012425in}}{\pgfqpoint{0.003683in}{0.013889in}}{\pgfqpoint{0.000000in}{0.013889in}}%
\pgfpathcurveto{\pgfqpoint{-0.003683in}{0.013889in}}{\pgfqpoint{-0.007216in}{0.012425in}}{\pgfqpoint{-0.009821in}{0.009821in}}%
\pgfpathcurveto{\pgfqpoint{-0.012425in}{0.007216in}}{\pgfqpoint{-0.013889in}{0.003683in}}{\pgfqpoint{-0.013889in}{0.000000in}}%
\pgfpathcurveto{\pgfqpoint{-0.013889in}{-0.003683in}}{\pgfqpoint{-0.012425in}{-0.007216in}}{\pgfqpoint{-0.009821in}{-0.009821in}}%
\pgfpathcurveto{\pgfqpoint{-0.007216in}{-0.012425in}}{\pgfqpoint{-0.003683in}{-0.013889in}}{\pgfqpoint{0.000000in}{-0.013889in}}%
\pgfpathclose%
\pgfusepath{stroke,fill}%
}%
\begin{pgfscope}%
\pgfsys@transformshift{3.684003in}{3.713619in}%
\pgfsys@useobject{currentmarker}{}%
\end{pgfscope}%
\end{pgfscope}%
\begin{pgfscope}%
\pgfpathrectangle{\pgfqpoint{0.100000in}{2.413063in}}{\pgfqpoint{5.037500in}{3.427208in}}%
\pgfusepath{clip}%
\pgfsetrectcap%
\pgfsetroundjoin%
\pgfsetlinewidth{1.505625pt}%
\definecolor{currentstroke}{rgb}{0.501961,0.501961,0.501961}%
\pgfsetstrokecolor{currentstroke}%
\pgfsetstrokeopacity{0.500000}%
\pgfsetdash{}{0pt}%
\pgfpathmoveto{\pgfqpoint{3.728741in}{3.597179in}}%
\pgfusepath{stroke}%
\end{pgfscope}%
\begin{pgfscope}%
\pgfpathrectangle{\pgfqpoint{0.100000in}{2.413063in}}{\pgfqpoint{5.037500in}{3.427208in}}%
\pgfusepath{clip}%
\pgfsetbuttcap%
\pgfsetroundjoin%
\definecolor{currentfill}{rgb}{0.501961,0.501961,0.501961}%
\pgfsetfillcolor{currentfill}%
\pgfsetfillopacity{0.500000}%
\pgfsetlinewidth{0.250937pt}%
\definecolor{currentstroke}{rgb}{0.000000,0.000000,0.000000}%
\pgfsetstrokecolor{currentstroke}%
\pgfsetstrokeopacity{0.500000}%
\pgfsetdash{}{0pt}%
\pgfsys@defobject{currentmarker}{\pgfqpoint{-0.013889in}{-0.013889in}}{\pgfqpoint{0.013889in}{0.013889in}}{%
\pgfpathmoveto{\pgfqpoint{0.000000in}{-0.013889in}}%
\pgfpathcurveto{\pgfqpoint{0.003683in}{-0.013889in}}{\pgfqpoint{0.007216in}{-0.012425in}}{\pgfqpoint{0.009821in}{-0.009821in}}%
\pgfpathcurveto{\pgfqpoint{0.012425in}{-0.007216in}}{\pgfqpoint{0.013889in}{-0.003683in}}{\pgfqpoint{0.013889in}{0.000000in}}%
\pgfpathcurveto{\pgfqpoint{0.013889in}{0.003683in}}{\pgfqpoint{0.012425in}{0.007216in}}{\pgfqpoint{0.009821in}{0.009821in}}%
\pgfpathcurveto{\pgfqpoint{0.007216in}{0.012425in}}{\pgfqpoint{0.003683in}{0.013889in}}{\pgfqpoint{0.000000in}{0.013889in}}%
\pgfpathcurveto{\pgfqpoint{-0.003683in}{0.013889in}}{\pgfqpoint{-0.007216in}{0.012425in}}{\pgfqpoint{-0.009821in}{0.009821in}}%
\pgfpathcurveto{\pgfqpoint{-0.012425in}{0.007216in}}{\pgfqpoint{-0.013889in}{0.003683in}}{\pgfqpoint{-0.013889in}{0.000000in}}%
\pgfpathcurveto{\pgfqpoint{-0.013889in}{-0.003683in}}{\pgfqpoint{-0.012425in}{-0.007216in}}{\pgfqpoint{-0.009821in}{-0.009821in}}%
\pgfpathcurveto{\pgfqpoint{-0.007216in}{-0.012425in}}{\pgfqpoint{-0.003683in}{-0.013889in}}{\pgfqpoint{0.000000in}{-0.013889in}}%
\pgfpathclose%
\pgfusepath{stroke,fill}%
}%
\begin{pgfscope}%
\pgfsys@transformshift{3.728741in}{3.597179in}%
\pgfsys@useobject{currentmarker}{}%
\end{pgfscope}%
\end{pgfscope}%
\begin{pgfscope}%
\pgfpathrectangle{\pgfqpoint{0.100000in}{2.413063in}}{\pgfqpoint{5.037500in}{3.427208in}}%
\pgfusepath{clip}%
\pgfsetrectcap%
\pgfsetroundjoin%
\pgfsetlinewidth{1.505625pt}%
\definecolor{currentstroke}{rgb}{0.000000,0.000000,1.000000}%
\pgfsetstrokecolor{currentstroke}%
\pgfsetstrokeopacity{0.500000}%
\pgfsetdash{}{0pt}%
\pgfpathmoveto{\pgfqpoint{3.593628in}{3.684664in}}%
\pgfusepath{stroke}%
\end{pgfscope}%
\begin{pgfscope}%
\pgfpathrectangle{\pgfqpoint{0.100000in}{2.413063in}}{\pgfqpoint{5.037500in}{3.427208in}}%
\pgfusepath{clip}%
\pgfsetbuttcap%
\pgfsetroundjoin%
\definecolor{currentfill}{rgb}{0.000000,0.000000,1.000000}%
\pgfsetfillcolor{currentfill}%
\pgfsetfillopacity{0.500000}%
\pgfsetlinewidth{0.250937pt}%
\definecolor{currentstroke}{rgb}{0.000000,0.000000,0.000000}%
\pgfsetstrokecolor{currentstroke}%
\pgfsetstrokeopacity{0.500000}%
\pgfsetdash{}{0pt}%
\pgfsys@defobject{currentmarker}{\pgfqpoint{-0.011111in}{-0.011111in}}{\pgfqpoint{0.011111in}{0.011111in}}{%
\pgfpathmoveto{\pgfqpoint{0.000000in}{-0.011111in}}%
\pgfpathcurveto{\pgfqpoint{0.002947in}{-0.011111in}}{\pgfqpoint{0.005773in}{-0.009940in}}{\pgfqpoint{0.007857in}{-0.007857in}}%
\pgfpathcurveto{\pgfqpoint{0.009940in}{-0.005773in}}{\pgfqpoint{0.011111in}{-0.002947in}}{\pgfqpoint{0.011111in}{0.000000in}}%
\pgfpathcurveto{\pgfqpoint{0.011111in}{0.002947in}}{\pgfqpoint{0.009940in}{0.005773in}}{\pgfqpoint{0.007857in}{0.007857in}}%
\pgfpathcurveto{\pgfqpoint{0.005773in}{0.009940in}}{\pgfqpoint{0.002947in}{0.011111in}}{\pgfqpoint{0.000000in}{0.011111in}}%
\pgfpathcurveto{\pgfqpoint{-0.002947in}{0.011111in}}{\pgfqpoint{-0.005773in}{0.009940in}}{\pgfqpoint{-0.007857in}{0.007857in}}%
\pgfpathcurveto{\pgfqpoint{-0.009940in}{0.005773in}}{\pgfqpoint{-0.011111in}{0.002947in}}{\pgfqpoint{-0.011111in}{0.000000in}}%
\pgfpathcurveto{\pgfqpoint{-0.011111in}{-0.002947in}}{\pgfqpoint{-0.009940in}{-0.005773in}}{\pgfqpoint{-0.007857in}{-0.007857in}}%
\pgfpathcurveto{\pgfqpoint{-0.005773in}{-0.009940in}}{\pgfqpoint{-0.002947in}{-0.011111in}}{\pgfqpoint{0.000000in}{-0.011111in}}%
\pgfpathclose%
\pgfusepath{stroke,fill}%
}%
\begin{pgfscope}%
\pgfsys@transformshift{3.593628in}{3.684664in}%
\pgfsys@useobject{currentmarker}{}%
\end{pgfscope}%
\end{pgfscope}%
\begin{pgfscope}%
\pgfpathrectangle{\pgfqpoint{0.100000in}{2.413063in}}{\pgfqpoint{5.037500in}{3.427208in}}%
\pgfusepath{clip}%
\pgfsetrectcap%
\pgfsetroundjoin%
\pgfsetlinewidth{1.505625pt}%
\definecolor{currentstroke}{rgb}{0.501961,0.501961,0.501961}%
\pgfsetstrokecolor{currentstroke}%
\pgfsetstrokeopacity{0.500000}%
\pgfsetdash{}{0pt}%
\pgfpathmoveto{\pgfqpoint{3.519290in}{3.335074in}}%
\pgfusepath{stroke}%
\end{pgfscope}%
\begin{pgfscope}%
\pgfpathrectangle{\pgfqpoint{0.100000in}{2.413063in}}{\pgfqpoint{5.037500in}{3.427208in}}%
\pgfusepath{clip}%
\pgfsetbuttcap%
\pgfsetroundjoin%
\definecolor{currentfill}{rgb}{0.501961,0.501961,0.501961}%
\pgfsetfillcolor{currentfill}%
\pgfsetfillopacity{0.500000}%
\pgfsetlinewidth{0.250937pt}%
\definecolor{currentstroke}{rgb}{0.000000,0.000000,0.000000}%
\pgfsetstrokecolor{currentstroke}%
\pgfsetstrokeopacity{0.500000}%
\pgfsetdash{}{0pt}%
\pgfsys@defobject{currentmarker}{\pgfqpoint{-0.013889in}{-0.013889in}}{\pgfqpoint{0.013889in}{0.013889in}}{%
\pgfpathmoveto{\pgfqpoint{0.000000in}{-0.013889in}}%
\pgfpathcurveto{\pgfqpoint{0.003683in}{-0.013889in}}{\pgfqpoint{0.007216in}{-0.012425in}}{\pgfqpoint{0.009821in}{-0.009821in}}%
\pgfpathcurveto{\pgfqpoint{0.012425in}{-0.007216in}}{\pgfqpoint{0.013889in}{-0.003683in}}{\pgfqpoint{0.013889in}{0.000000in}}%
\pgfpathcurveto{\pgfqpoint{0.013889in}{0.003683in}}{\pgfqpoint{0.012425in}{0.007216in}}{\pgfqpoint{0.009821in}{0.009821in}}%
\pgfpathcurveto{\pgfqpoint{0.007216in}{0.012425in}}{\pgfqpoint{0.003683in}{0.013889in}}{\pgfqpoint{0.000000in}{0.013889in}}%
\pgfpathcurveto{\pgfqpoint{-0.003683in}{0.013889in}}{\pgfqpoint{-0.007216in}{0.012425in}}{\pgfqpoint{-0.009821in}{0.009821in}}%
\pgfpathcurveto{\pgfqpoint{-0.012425in}{0.007216in}}{\pgfqpoint{-0.013889in}{0.003683in}}{\pgfqpoint{-0.013889in}{0.000000in}}%
\pgfpathcurveto{\pgfqpoint{-0.013889in}{-0.003683in}}{\pgfqpoint{-0.012425in}{-0.007216in}}{\pgfqpoint{-0.009821in}{-0.009821in}}%
\pgfpathcurveto{\pgfqpoint{-0.007216in}{-0.012425in}}{\pgfqpoint{-0.003683in}{-0.013889in}}{\pgfqpoint{0.000000in}{-0.013889in}}%
\pgfpathclose%
\pgfusepath{stroke,fill}%
}%
\begin{pgfscope}%
\pgfsys@transformshift{3.519290in}{3.335074in}%
\pgfsys@useobject{currentmarker}{}%
\end{pgfscope}%
\end{pgfscope}%
\begin{pgfscope}%
\pgfpathrectangle{\pgfqpoint{0.100000in}{2.413063in}}{\pgfqpoint{5.037500in}{3.427208in}}%
\pgfusepath{clip}%
\pgfsetrectcap%
\pgfsetroundjoin%
\pgfsetlinewidth{1.505625pt}%
\definecolor{currentstroke}{rgb}{0.501961,0.501961,0.501961}%
\pgfsetstrokecolor{currentstroke}%
\pgfsetstrokeopacity{0.500000}%
\pgfsetdash{}{0pt}%
\pgfpathmoveto{\pgfqpoint{3.557956in}{3.809517in}}%
\pgfusepath{stroke}%
\end{pgfscope}%
\begin{pgfscope}%
\pgfpathrectangle{\pgfqpoint{0.100000in}{2.413063in}}{\pgfqpoint{5.037500in}{3.427208in}}%
\pgfusepath{clip}%
\pgfsetbuttcap%
\pgfsetroundjoin%
\definecolor{currentfill}{rgb}{0.501961,0.501961,0.501961}%
\pgfsetfillcolor{currentfill}%
\pgfsetfillopacity{0.500000}%
\pgfsetlinewidth{0.250937pt}%
\definecolor{currentstroke}{rgb}{0.000000,0.000000,0.000000}%
\pgfsetstrokecolor{currentstroke}%
\pgfsetstrokeopacity{0.500000}%
\pgfsetdash{}{0pt}%
\pgfsys@defobject{currentmarker}{\pgfqpoint{-0.013889in}{-0.013889in}}{\pgfqpoint{0.013889in}{0.013889in}}{%
\pgfpathmoveto{\pgfqpoint{0.000000in}{-0.013889in}}%
\pgfpathcurveto{\pgfqpoint{0.003683in}{-0.013889in}}{\pgfqpoint{0.007216in}{-0.012425in}}{\pgfqpoint{0.009821in}{-0.009821in}}%
\pgfpathcurveto{\pgfqpoint{0.012425in}{-0.007216in}}{\pgfqpoint{0.013889in}{-0.003683in}}{\pgfqpoint{0.013889in}{0.000000in}}%
\pgfpathcurveto{\pgfqpoint{0.013889in}{0.003683in}}{\pgfqpoint{0.012425in}{0.007216in}}{\pgfqpoint{0.009821in}{0.009821in}}%
\pgfpathcurveto{\pgfqpoint{0.007216in}{0.012425in}}{\pgfqpoint{0.003683in}{0.013889in}}{\pgfqpoint{0.000000in}{0.013889in}}%
\pgfpathcurveto{\pgfqpoint{-0.003683in}{0.013889in}}{\pgfqpoint{-0.007216in}{0.012425in}}{\pgfqpoint{-0.009821in}{0.009821in}}%
\pgfpathcurveto{\pgfqpoint{-0.012425in}{0.007216in}}{\pgfqpoint{-0.013889in}{0.003683in}}{\pgfqpoint{-0.013889in}{0.000000in}}%
\pgfpathcurveto{\pgfqpoint{-0.013889in}{-0.003683in}}{\pgfqpoint{-0.012425in}{-0.007216in}}{\pgfqpoint{-0.009821in}{-0.009821in}}%
\pgfpathcurveto{\pgfqpoint{-0.007216in}{-0.012425in}}{\pgfqpoint{-0.003683in}{-0.013889in}}{\pgfqpoint{0.000000in}{-0.013889in}}%
\pgfpathclose%
\pgfusepath{stroke,fill}%
}%
\begin{pgfscope}%
\pgfsys@transformshift{3.557956in}{3.809517in}%
\pgfsys@useobject{currentmarker}{}%
\end{pgfscope}%
\end{pgfscope}%
\begin{pgfscope}%
\pgfpathrectangle{\pgfqpoint{0.100000in}{2.413063in}}{\pgfqpoint{5.037500in}{3.427208in}}%
\pgfusepath{clip}%
\pgfsetrectcap%
\pgfsetroundjoin%
\pgfsetlinewidth{1.505625pt}%
\definecolor{currentstroke}{rgb}{0.501961,0.501961,0.501961}%
\pgfsetstrokecolor{currentstroke}%
\pgfsetstrokeopacity{0.500000}%
\pgfsetdash{}{0pt}%
\pgfpathmoveto{\pgfqpoint{3.757252in}{3.433137in}}%
\pgfusepath{stroke}%
\end{pgfscope}%
\begin{pgfscope}%
\pgfpathrectangle{\pgfqpoint{0.100000in}{2.413063in}}{\pgfqpoint{5.037500in}{3.427208in}}%
\pgfusepath{clip}%
\pgfsetbuttcap%
\pgfsetroundjoin%
\definecolor{currentfill}{rgb}{0.501961,0.501961,0.501961}%
\pgfsetfillcolor{currentfill}%
\pgfsetfillopacity{0.500000}%
\pgfsetlinewidth{0.250937pt}%
\definecolor{currentstroke}{rgb}{0.000000,0.000000,0.000000}%
\pgfsetstrokecolor{currentstroke}%
\pgfsetstrokeopacity{0.500000}%
\pgfsetdash{}{0pt}%
\pgfsys@defobject{currentmarker}{\pgfqpoint{-0.013889in}{-0.013889in}}{\pgfqpoint{0.013889in}{0.013889in}}{%
\pgfpathmoveto{\pgfqpoint{0.000000in}{-0.013889in}}%
\pgfpathcurveto{\pgfqpoint{0.003683in}{-0.013889in}}{\pgfqpoint{0.007216in}{-0.012425in}}{\pgfqpoint{0.009821in}{-0.009821in}}%
\pgfpathcurveto{\pgfqpoint{0.012425in}{-0.007216in}}{\pgfqpoint{0.013889in}{-0.003683in}}{\pgfqpoint{0.013889in}{0.000000in}}%
\pgfpathcurveto{\pgfqpoint{0.013889in}{0.003683in}}{\pgfqpoint{0.012425in}{0.007216in}}{\pgfqpoint{0.009821in}{0.009821in}}%
\pgfpathcurveto{\pgfqpoint{0.007216in}{0.012425in}}{\pgfqpoint{0.003683in}{0.013889in}}{\pgfqpoint{0.000000in}{0.013889in}}%
\pgfpathcurveto{\pgfqpoint{-0.003683in}{0.013889in}}{\pgfqpoint{-0.007216in}{0.012425in}}{\pgfqpoint{-0.009821in}{0.009821in}}%
\pgfpathcurveto{\pgfqpoint{-0.012425in}{0.007216in}}{\pgfqpoint{-0.013889in}{0.003683in}}{\pgfqpoint{-0.013889in}{0.000000in}}%
\pgfpathcurveto{\pgfqpoint{-0.013889in}{-0.003683in}}{\pgfqpoint{-0.012425in}{-0.007216in}}{\pgfqpoint{-0.009821in}{-0.009821in}}%
\pgfpathcurveto{\pgfqpoint{-0.007216in}{-0.012425in}}{\pgfqpoint{-0.003683in}{-0.013889in}}{\pgfqpoint{0.000000in}{-0.013889in}}%
\pgfpathclose%
\pgfusepath{stroke,fill}%
}%
\begin{pgfscope}%
\pgfsys@transformshift{3.757252in}{3.433137in}%
\pgfsys@useobject{currentmarker}{}%
\end{pgfscope}%
\end{pgfscope}%
\begin{pgfscope}%
\pgfpathrectangle{\pgfqpoint{0.100000in}{2.413063in}}{\pgfqpoint{5.037500in}{3.427208in}}%
\pgfusepath{clip}%
\pgfsetrectcap%
\pgfsetroundjoin%
\pgfsetlinewidth{1.505625pt}%
\definecolor{currentstroke}{rgb}{0.501961,0.501961,0.501961}%
\pgfsetstrokecolor{currentstroke}%
\pgfsetstrokeopacity{0.500000}%
\pgfsetdash{}{0pt}%
\pgfpathmoveto{\pgfqpoint{3.497670in}{3.824813in}}%
\pgfusepath{stroke}%
\end{pgfscope}%
\begin{pgfscope}%
\pgfpathrectangle{\pgfqpoint{0.100000in}{2.413063in}}{\pgfqpoint{5.037500in}{3.427208in}}%
\pgfusepath{clip}%
\pgfsetbuttcap%
\pgfsetroundjoin%
\definecolor{currentfill}{rgb}{0.501961,0.501961,0.501961}%
\pgfsetfillcolor{currentfill}%
\pgfsetfillopacity{0.500000}%
\pgfsetlinewidth{0.250937pt}%
\definecolor{currentstroke}{rgb}{0.000000,0.000000,0.000000}%
\pgfsetstrokecolor{currentstroke}%
\pgfsetstrokeopacity{0.500000}%
\pgfsetdash{}{0pt}%
\pgfsys@defobject{currentmarker}{\pgfqpoint{-0.013889in}{-0.013889in}}{\pgfqpoint{0.013889in}{0.013889in}}{%
\pgfpathmoveto{\pgfqpoint{0.000000in}{-0.013889in}}%
\pgfpathcurveto{\pgfqpoint{0.003683in}{-0.013889in}}{\pgfqpoint{0.007216in}{-0.012425in}}{\pgfqpoint{0.009821in}{-0.009821in}}%
\pgfpathcurveto{\pgfqpoint{0.012425in}{-0.007216in}}{\pgfqpoint{0.013889in}{-0.003683in}}{\pgfqpoint{0.013889in}{0.000000in}}%
\pgfpathcurveto{\pgfqpoint{0.013889in}{0.003683in}}{\pgfqpoint{0.012425in}{0.007216in}}{\pgfqpoint{0.009821in}{0.009821in}}%
\pgfpathcurveto{\pgfqpoint{0.007216in}{0.012425in}}{\pgfqpoint{0.003683in}{0.013889in}}{\pgfqpoint{0.000000in}{0.013889in}}%
\pgfpathcurveto{\pgfqpoint{-0.003683in}{0.013889in}}{\pgfqpoint{-0.007216in}{0.012425in}}{\pgfqpoint{-0.009821in}{0.009821in}}%
\pgfpathcurveto{\pgfqpoint{-0.012425in}{0.007216in}}{\pgfqpoint{-0.013889in}{0.003683in}}{\pgfqpoint{-0.013889in}{0.000000in}}%
\pgfpathcurveto{\pgfqpoint{-0.013889in}{-0.003683in}}{\pgfqpoint{-0.012425in}{-0.007216in}}{\pgfqpoint{-0.009821in}{-0.009821in}}%
\pgfpathcurveto{\pgfqpoint{-0.007216in}{-0.012425in}}{\pgfqpoint{-0.003683in}{-0.013889in}}{\pgfqpoint{0.000000in}{-0.013889in}}%
\pgfpathclose%
\pgfusepath{stroke,fill}%
}%
\begin{pgfscope}%
\pgfsys@transformshift{3.497670in}{3.824813in}%
\pgfsys@useobject{currentmarker}{}%
\end{pgfscope}%
\end{pgfscope}%
\begin{pgfscope}%
\pgfpathrectangle{\pgfqpoint{0.100000in}{2.413063in}}{\pgfqpoint{5.037500in}{3.427208in}}%
\pgfusepath{clip}%
\pgfsetrectcap%
\pgfsetroundjoin%
\pgfsetlinewidth{1.505625pt}%
\definecolor{currentstroke}{rgb}{0.000000,0.000000,1.000000}%
\pgfsetstrokecolor{currentstroke}%
\pgfsetstrokeopacity{0.500000}%
\pgfsetdash{}{0pt}%
\pgfpathmoveto{\pgfqpoint{3.664388in}{3.748720in}}%
\pgfusepath{stroke}%
\end{pgfscope}%
\begin{pgfscope}%
\pgfpathrectangle{\pgfqpoint{0.100000in}{2.413063in}}{\pgfqpoint{5.037500in}{3.427208in}}%
\pgfusepath{clip}%
\pgfsetbuttcap%
\pgfsetroundjoin%
\definecolor{currentfill}{rgb}{0.000000,0.000000,1.000000}%
\pgfsetfillcolor{currentfill}%
\pgfsetfillopacity{0.500000}%
\pgfsetlinewidth{0.250937pt}%
\definecolor{currentstroke}{rgb}{0.000000,0.000000,0.000000}%
\pgfsetstrokecolor{currentstroke}%
\pgfsetstrokeopacity{0.500000}%
\pgfsetdash{}{0pt}%
\pgfsys@defobject{currentmarker}{\pgfqpoint{-0.013889in}{-0.013889in}}{\pgfqpoint{0.013889in}{0.013889in}}{%
\pgfpathmoveto{\pgfqpoint{0.000000in}{-0.013889in}}%
\pgfpathcurveto{\pgfqpoint{0.003683in}{-0.013889in}}{\pgfqpoint{0.007216in}{-0.012425in}}{\pgfqpoint{0.009821in}{-0.009821in}}%
\pgfpathcurveto{\pgfqpoint{0.012425in}{-0.007216in}}{\pgfqpoint{0.013889in}{-0.003683in}}{\pgfqpoint{0.013889in}{0.000000in}}%
\pgfpathcurveto{\pgfqpoint{0.013889in}{0.003683in}}{\pgfqpoint{0.012425in}{0.007216in}}{\pgfqpoint{0.009821in}{0.009821in}}%
\pgfpathcurveto{\pgfqpoint{0.007216in}{0.012425in}}{\pgfqpoint{0.003683in}{0.013889in}}{\pgfqpoint{0.000000in}{0.013889in}}%
\pgfpathcurveto{\pgfqpoint{-0.003683in}{0.013889in}}{\pgfqpoint{-0.007216in}{0.012425in}}{\pgfqpoint{-0.009821in}{0.009821in}}%
\pgfpathcurveto{\pgfqpoint{-0.012425in}{0.007216in}}{\pgfqpoint{-0.013889in}{0.003683in}}{\pgfqpoint{-0.013889in}{0.000000in}}%
\pgfpathcurveto{\pgfqpoint{-0.013889in}{-0.003683in}}{\pgfqpoint{-0.012425in}{-0.007216in}}{\pgfqpoint{-0.009821in}{-0.009821in}}%
\pgfpathcurveto{\pgfqpoint{-0.007216in}{-0.012425in}}{\pgfqpoint{-0.003683in}{-0.013889in}}{\pgfqpoint{0.000000in}{-0.013889in}}%
\pgfpathclose%
\pgfusepath{stroke,fill}%
}%
\begin{pgfscope}%
\pgfsys@transformshift{3.664388in}{3.748720in}%
\pgfsys@useobject{currentmarker}{}%
\end{pgfscope}%
\end{pgfscope}%
\begin{pgfscope}%
\pgfpathrectangle{\pgfqpoint{0.100000in}{2.413063in}}{\pgfqpoint{5.037500in}{3.427208in}}%
\pgfusepath{clip}%
\pgfsetrectcap%
\pgfsetroundjoin%
\pgfsetlinewidth{1.505625pt}%
\definecolor{currentstroke}{rgb}{0.501961,0.501961,0.501961}%
\pgfsetstrokecolor{currentstroke}%
\pgfsetstrokeopacity{0.500000}%
\pgfsetdash{}{0pt}%
\pgfpathmoveto{\pgfqpoint{3.601503in}{3.825717in}}%
\pgfusepath{stroke}%
\end{pgfscope}%
\begin{pgfscope}%
\pgfpathrectangle{\pgfqpoint{0.100000in}{2.413063in}}{\pgfqpoint{5.037500in}{3.427208in}}%
\pgfusepath{clip}%
\pgfsetbuttcap%
\pgfsetroundjoin%
\definecolor{currentfill}{rgb}{0.501961,0.501961,0.501961}%
\pgfsetfillcolor{currentfill}%
\pgfsetfillopacity{0.500000}%
\pgfsetlinewidth{0.250937pt}%
\definecolor{currentstroke}{rgb}{0.000000,0.000000,0.000000}%
\pgfsetstrokecolor{currentstroke}%
\pgfsetstrokeopacity{0.500000}%
\pgfsetdash{}{0pt}%
\pgfsys@defobject{currentmarker}{\pgfqpoint{-0.013889in}{-0.013889in}}{\pgfqpoint{0.013889in}{0.013889in}}{%
\pgfpathmoveto{\pgfqpoint{0.000000in}{-0.013889in}}%
\pgfpathcurveto{\pgfqpoint{0.003683in}{-0.013889in}}{\pgfqpoint{0.007216in}{-0.012425in}}{\pgfqpoint{0.009821in}{-0.009821in}}%
\pgfpathcurveto{\pgfqpoint{0.012425in}{-0.007216in}}{\pgfqpoint{0.013889in}{-0.003683in}}{\pgfqpoint{0.013889in}{0.000000in}}%
\pgfpathcurveto{\pgfqpoint{0.013889in}{0.003683in}}{\pgfqpoint{0.012425in}{0.007216in}}{\pgfqpoint{0.009821in}{0.009821in}}%
\pgfpathcurveto{\pgfqpoint{0.007216in}{0.012425in}}{\pgfqpoint{0.003683in}{0.013889in}}{\pgfqpoint{0.000000in}{0.013889in}}%
\pgfpathcurveto{\pgfqpoint{-0.003683in}{0.013889in}}{\pgfqpoint{-0.007216in}{0.012425in}}{\pgfqpoint{-0.009821in}{0.009821in}}%
\pgfpathcurveto{\pgfqpoint{-0.012425in}{0.007216in}}{\pgfqpoint{-0.013889in}{0.003683in}}{\pgfqpoint{-0.013889in}{0.000000in}}%
\pgfpathcurveto{\pgfqpoint{-0.013889in}{-0.003683in}}{\pgfqpoint{-0.012425in}{-0.007216in}}{\pgfqpoint{-0.009821in}{-0.009821in}}%
\pgfpathcurveto{\pgfqpoint{-0.007216in}{-0.012425in}}{\pgfqpoint{-0.003683in}{-0.013889in}}{\pgfqpoint{0.000000in}{-0.013889in}}%
\pgfpathclose%
\pgfusepath{stroke,fill}%
}%
\begin{pgfscope}%
\pgfsys@transformshift{3.601503in}{3.825717in}%
\pgfsys@useobject{currentmarker}{}%
\end{pgfscope}%
\end{pgfscope}%
\begin{pgfscope}%
\pgfpathrectangle{\pgfqpoint{0.100000in}{2.413063in}}{\pgfqpoint{5.037500in}{3.427208in}}%
\pgfusepath{clip}%
\pgfsetrectcap%
\pgfsetroundjoin%
\pgfsetlinewidth{1.505625pt}%
\definecolor{currentstroke}{rgb}{0.000000,0.000000,1.000000}%
\pgfsetstrokecolor{currentstroke}%
\pgfsetstrokeopacity{0.500000}%
\pgfsetdash{}{0pt}%
\pgfpathmoveto{\pgfqpoint{3.499471in}{3.347832in}}%
\pgfusepath{stroke}%
\end{pgfscope}%
\begin{pgfscope}%
\pgfpathrectangle{\pgfqpoint{0.100000in}{2.413063in}}{\pgfqpoint{5.037500in}{3.427208in}}%
\pgfusepath{clip}%
\pgfsetbuttcap%
\pgfsetroundjoin%
\definecolor{currentfill}{rgb}{0.000000,0.000000,1.000000}%
\pgfsetfillcolor{currentfill}%
\pgfsetfillopacity{0.500000}%
\pgfsetlinewidth{0.250937pt}%
\definecolor{currentstroke}{rgb}{0.000000,0.000000,0.000000}%
\pgfsetstrokecolor{currentstroke}%
\pgfsetstrokeopacity{0.500000}%
\pgfsetdash{}{0pt}%
\pgfsys@defobject{currentmarker}{\pgfqpoint{-0.022222in}{-0.022222in}}{\pgfqpoint{0.022222in}{0.022222in}}{%
\pgfpathmoveto{\pgfqpoint{0.000000in}{-0.022222in}}%
\pgfpathcurveto{\pgfqpoint{0.005893in}{-0.022222in}}{\pgfqpoint{0.011546in}{-0.019881in}}{\pgfqpoint{0.015713in}{-0.015713in}}%
\pgfpathcurveto{\pgfqpoint{0.019881in}{-0.011546in}}{\pgfqpoint{0.022222in}{-0.005893in}}{\pgfqpoint{0.022222in}{0.000000in}}%
\pgfpathcurveto{\pgfqpoint{0.022222in}{0.005893in}}{\pgfqpoint{0.019881in}{0.011546in}}{\pgfqpoint{0.015713in}{0.015713in}}%
\pgfpathcurveto{\pgfqpoint{0.011546in}{0.019881in}}{\pgfqpoint{0.005893in}{0.022222in}}{\pgfqpoint{0.000000in}{0.022222in}}%
\pgfpathcurveto{\pgfqpoint{-0.005893in}{0.022222in}}{\pgfqpoint{-0.011546in}{0.019881in}}{\pgfqpoint{-0.015713in}{0.015713in}}%
\pgfpathcurveto{\pgfqpoint{-0.019881in}{0.011546in}}{\pgfqpoint{-0.022222in}{0.005893in}}{\pgfqpoint{-0.022222in}{0.000000in}}%
\pgfpathcurveto{\pgfqpoint{-0.022222in}{-0.005893in}}{\pgfqpoint{-0.019881in}{-0.011546in}}{\pgfqpoint{-0.015713in}{-0.015713in}}%
\pgfpathcurveto{\pgfqpoint{-0.011546in}{-0.019881in}}{\pgfqpoint{-0.005893in}{-0.022222in}}{\pgfqpoint{0.000000in}{-0.022222in}}%
\pgfpathclose%
\pgfusepath{stroke,fill}%
}%
\begin{pgfscope}%
\pgfsys@transformshift{3.499471in}{3.347832in}%
\pgfsys@useobject{currentmarker}{}%
\end{pgfscope}%
\end{pgfscope}%
\begin{pgfscope}%
\pgfpathrectangle{\pgfqpoint{0.100000in}{2.413063in}}{\pgfqpoint{5.037500in}{3.427208in}}%
\pgfusepath{clip}%
\pgfsetrectcap%
\pgfsetroundjoin%
\pgfsetlinewidth{1.505625pt}%
\definecolor{currentstroke}{rgb}{0.000000,0.000000,1.000000}%
\pgfsetstrokecolor{currentstroke}%
\pgfsetstrokeopacity{0.500000}%
\pgfsetdash{}{0pt}%
\pgfpathmoveto{\pgfqpoint{3.654580in}{3.556172in}}%
\pgfusepath{stroke}%
\end{pgfscope}%
\begin{pgfscope}%
\pgfpathrectangle{\pgfqpoint{0.100000in}{2.413063in}}{\pgfqpoint{5.037500in}{3.427208in}}%
\pgfusepath{clip}%
\pgfsetbuttcap%
\pgfsetroundjoin%
\definecolor{currentfill}{rgb}{0.000000,0.000000,1.000000}%
\pgfsetfillcolor{currentfill}%
\pgfsetfillopacity{0.500000}%
\pgfsetlinewidth{0.250937pt}%
\definecolor{currentstroke}{rgb}{0.000000,0.000000,0.000000}%
\pgfsetstrokecolor{currentstroke}%
\pgfsetstrokeopacity{0.500000}%
\pgfsetdash{}{0pt}%
\pgfsys@defobject{currentmarker}{\pgfqpoint{-0.027778in}{-0.027778in}}{\pgfqpoint{0.027778in}{0.027778in}}{%
\pgfpathmoveto{\pgfqpoint{0.000000in}{-0.027778in}}%
\pgfpathcurveto{\pgfqpoint{0.007367in}{-0.027778in}}{\pgfqpoint{0.014433in}{-0.024851in}}{\pgfqpoint{0.019642in}{-0.019642in}}%
\pgfpathcurveto{\pgfqpoint{0.024851in}{-0.014433in}}{\pgfqpoint{0.027778in}{-0.007367in}}{\pgfqpoint{0.027778in}{0.000000in}}%
\pgfpathcurveto{\pgfqpoint{0.027778in}{0.007367in}}{\pgfqpoint{0.024851in}{0.014433in}}{\pgfqpoint{0.019642in}{0.019642in}}%
\pgfpathcurveto{\pgfqpoint{0.014433in}{0.024851in}}{\pgfqpoint{0.007367in}{0.027778in}}{\pgfqpoint{0.000000in}{0.027778in}}%
\pgfpathcurveto{\pgfqpoint{-0.007367in}{0.027778in}}{\pgfqpoint{-0.014433in}{0.024851in}}{\pgfqpoint{-0.019642in}{0.019642in}}%
\pgfpathcurveto{\pgfqpoint{-0.024851in}{0.014433in}}{\pgfqpoint{-0.027778in}{0.007367in}}{\pgfqpoint{-0.027778in}{0.000000in}}%
\pgfpathcurveto{\pgfqpoint{-0.027778in}{-0.007367in}}{\pgfqpoint{-0.024851in}{-0.014433in}}{\pgfqpoint{-0.019642in}{-0.019642in}}%
\pgfpathcurveto{\pgfqpoint{-0.014433in}{-0.024851in}}{\pgfqpoint{-0.007367in}{-0.027778in}}{\pgfqpoint{0.000000in}{-0.027778in}}%
\pgfpathclose%
\pgfusepath{stroke,fill}%
}%
\begin{pgfscope}%
\pgfsys@transformshift{3.654580in}{3.556172in}%
\pgfsys@useobject{currentmarker}{}%
\end{pgfscope}%
\end{pgfscope}%
\begin{pgfscope}%
\pgfpathrectangle{\pgfqpoint{0.100000in}{2.413063in}}{\pgfqpoint{5.037500in}{3.427208in}}%
\pgfusepath{clip}%
\pgfsetrectcap%
\pgfsetroundjoin%
\pgfsetlinewidth{1.505625pt}%
\definecolor{currentstroke}{rgb}{0.000000,0.000000,1.000000}%
\pgfsetstrokecolor{currentstroke}%
\pgfsetstrokeopacity{0.500000}%
\pgfsetdash{}{0pt}%
\pgfpathmoveto{\pgfqpoint{3.530356in}{3.649700in}}%
\pgfusepath{stroke}%
\end{pgfscope}%
\begin{pgfscope}%
\pgfpathrectangle{\pgfqpoint{0.100000in}{2.413063in}}{\pgfqpoint{5.037500in}{3.427208in}}%
\pgfusepath{clip}%
\pgfsetbuttcap%
\pgfsetroundjoin%
\definecolor{currentfill}{rgb}{0.000000,0.000000,1.000000}%
\pgfsetfillcolor{currentfill}%
\pgfsetfillopacity{0.500000}%
\pgfsetlinewidth{0.250937pt}%
\definecolor{currentstroke}{rgb}{0.000000,0.000000,0.000000}%
\pgfsetstrokecolor{currentstroke}%
\pgfsetstrokeopacity{0.500000}%
\pgfsetdash{}{0pt}%
\pgfsys@defobject{currentmarker}{\pgfqpoint{-0.022222in}{-0.022222in}}{\pgfqpoint{0.022222in}{0.022222in}}{%
\pgfpathmoveto{\pgfqpoint{0.000000in}{-0.022222in}}%
\pgfpathcurveto{\pgfqpoint{0.005893in}{-0.022222in}}{\pgfqpoint{0.011546in}{-0.019881in}}{\pgfqpoint{0.015713in}{-0.015713in}}%
\pgfpathcurveto{\pgfqpoint{0.019881in}{-0.011546in}}{\pgfqpoint{0.022222in}{-0.005893in}}{\pgfqpoint{0.022222in}{0.000000in}}%
\pgfpathcurveto{\pgfqpoint{0.022222in}{0.005893in}}{\pgfqpoint{0.019881in}{0.011546in}}{\pgfqpoint{0.015713in}{0.015713in}}%
\pgfpathcurveto{\pgfqpoint{0.011546in}{0.019881in}}{\pgfqpoint{0.005893in}{0.022222in}}{\pgfqpoint{0.000000in}{0.022222in}}%
\pgfpathcurveto{\pgfqpoint{-0.005893in}{0.022222in}}{\pgfqpoint{-0.011546in}{0.019881in}}{\pgfqpoint{-0.015713in}{0.015713in}}%
\pgfpathcurveto{\pgfqpoint{-0.019881in}{0.011546in}}{\pgfqpoint{-0.022222in}{0.005893in}}{\pgfqpoint{-0.022222in}{0.000000in}}%
\pgfpathcurveto{\pgfqpoint{-0.022222in}{-0.005893in}}{\pgfqpoint{-0.019881in}{-0.011546in}}{\pgfqpoint{-0.015713in}{-0.015713in}}%
\pgfpathcurveto{\pgfqpoint{-0.011546in}{-0.019881in}}{\pgfqpoint{-0.005893in}{-0.022222in}}{\pgfqpoint{0.000000in}{-0.022222in}}%
\pgfpathclose%
\pgfusepath{stroke,fill}%
}%
\begin{pgfscope}%
\pgfsys@transformshift{3.530356in}{3.649700in}%
\pgfsys@useobject{currentmarker}{}%
\end{pgfscope}%
\end{pgfscope}%
\begin{pgfscope}%
\pgfpathrectangle{\pgfqpoint{0.100000in}{2.413063in}}{\pgfqpoint{5.037500in}{3.427208in}}%
\pgfusepath{clip}%
\pgfsetrectcap%
\pgfsetroundjoin%
\pgfsetlinewidth{1.505625pt}%
\definecolor{currentstroke}{rgb}{0.000000,0.000000,1.000000}%
\pgfsetstrokecolor{currentstroke}%
\pgfsetstrokeopacity{0.500000}%
\pgfsetdash{}{0pt}%
\pgfpathmoveto{\pgfqpoint{1.206521in}{2.920030in}}%
\pgfusepath{stroke}%
\end{pgfscope}%
\begin{pgfscope}%
\pgfpathrectangle{\pgfqpoint{0.100000in}{2.413063in}}{\pgfqpoint{5.037500in}{3.427208in}}%
\pgfusepath{clip}%
\pgfsetbuttcap%
\pgfsetroundjoin%
\definecolor{currentfill}{rgb}{0.000000,0.000000,1.000000}%
\pgfsetfillcolor{currentfill}%
\pgfsetfillopacity{0.500000}%
\pgfsetlinewidth{0.250937pt}%
\definecolor{currentstroke}{rgb}{0.000000,0.000000,0.000000}%
\pgfsetstrokecolor{currentstroke}%
\pgfsetstrokeopacity{0.500000}%
\pgfsetdash{}{0pt}%
\pgfsys@defobject{currentmarker}{\pgfqpoint{-0.011111in}{-0.011111in}}{\pgfqpoint{0.011111in}{0.011111in}}{%
\pgfpathmoveto{\pgfqpoint{0.000000in}{-0.011111in}}%
\pgfpathcurveto{\pgfqpoint{0.002947in}{-0.011111in}}{\pgfqpoint{0.005773in}{-0.009940in}}{\pgfqpoint{0.007857in}{-0.007857in}}%
\pgfpathcurveto{\pgfqpoint{0.009940in}{-0.005773in}}{\pgfqpoint{0.011111in}{-0.002947in}}{\pgfqpoint{0.011111in}{0.000000in}}%
\pgfpathcurveto{\pgfqpoint{0.011111in}{0.002947in}}{\pgfqpoint{0.009940in}{0.005773in}}{\pgfqpoint{0.007857in}{0.007857in}}%
\pgfpathcurveto{\pgfqpoint{0.005773in}{0.009940in}}{\pgfqpoint{0.002947in}{0.011111in}}{\pgfqpoint{0.000000in}{0.011111in}}%
\pgfpathcurveto{\pgfqpoint{-0.002947in}{0.011111in}}{\pgfqpoint{-0.005773in}{0.009940in}}{\pgfqpoint{-0.007857in}{0.007857in}}%
\pgfpathcurveto{\pgfqpoint{-0.009940in}{0.005773in}}{\pgfqpoint{-0.011111in}{0.002947in}}{\pgfqpoint{-0.011111in}{0.000000in}}%
\pgfpathcurveto{\pgfqpoint{-0.011111in}{-0.002947in}}{\pgfqpoint{-0.009940in}{-0.005773in}}{\pgfqpoint{-0.007857in}{-0.007857in}}%
\pgfpathcurveto{\pgfqpoint{-0.005773in}{-0.009940in}}{\pgfqpoint{-0.002947in}{-0.011111in}}{\pgfqpoint{0.000000in}{-0.011111in}}%
\pgfpathclose%
\pgfusepath{stroke,fill}%
}%
\begin{pgfscope}%
\pgfsys@transformshift{1.206521in}{2.920030in}%
\pgfsys@useobject{currentmarker}{}%
\end{pgfscope}%
\end{pgfscope}%
\begin{pgfscope}%
\pgfpathrectangle{\pgfqpoint{0.100000in}{2.413063in}}{\pgfqpoint{5.037500in}{3.427208in}}%
\pgfusepath{clip}%
\pgfsetrectcap%
\pgfsetroundjoin%
\pgfsetlinewidth{1.505625pt}%
\definecolor{currentstroke}{rgb}{0.000000,0.000000,1.000000}%
\pgfsetstrokecolor{currentstroke}%
\pgfsetstrokeopacity{0.500000}%
\pgfsetdash{}{0pt}%
\pgfpathmoveto{\pgfqpoint{1.330071in}{3.026783in}}%
\pgfusepath{stroke}%
\end{pgfscope}%
\begin{pgfscope}%
\pgfpathrectangle{\pgfqpoint{0.100000in}{2.413063in}}{\pgfqpoint{5.037500in}{3.427208in}}%
\pgfusepath{clip}%
\pgfsetbuttcap%
\pgfsetroundjoin%
\definecolor{currentfill}{rgb}{0.000000,0.000000,1.000000}%
\pgfsetfillcolor{currentfill}%
\pgfsetfillopacity{0.500000}%
\pgfsetlinewidth{0.250937pt}%
\definecolor{currentstroke}{rgb}{0.000000,0.000000,0.000000}%
\pgfsetstrokecolor{currentstroke}%
\pgfsetstrokeopacity{0.500000}%
\pgfsetdash{}{0pt}%
\pgfsys@defobject{currentmarker}{\pgfqpoint{-0.005556in}{-0.005556in}}{\pgfqpoint{0.005556in}{0.005556in}}{%
\pgfpathmoveto{\pgfqpoint{0.000000in}{-0.005556in}}%
\pgfpathcurveto{\pgfqpoint{0.001473in}{-0.005556in}}{\pgfqpoint{0.002887in}{-0.004970in}}{\pgfqpoint{0.003928in}{-0.003928in}}%
\pgfpathcurveto{\pgfqpoint{0.004970in}{-0.002887in}}{\pgfqpoint{0.005556in}{-0.001473in}}{\pgfqpoint{0.005556in}{0.000000in}}%
\pgfpathcurveto{\pgfqpoint{0.005556in}{0.001473in}}{\pgfqpoint{0.004970in}{0.002887in}}{\pgfqpoint{0.003928in}{0.003928in}}%
\pgfpathcurveto{\pgfqpoint{0.002887in}{0.004970in}}{\pgfqpoint{0.001473in}{0.005556in}}{\pgfqpoint{0.000000in}{0.005556in}}%
\pgfpathcurveto{\pgfqpoint{-0.001473in}{0.005556in}}{\pgfqpoint{-0.002887in}{0.004970in}}{\pgfqpoint{-0.003928in}{0.003928in}}%
\pgfpathcurveto{\pgfqpoint{-0.004970in}{0.002887in}}{\pgfqpoint{-0.005556in}{0.001473in}}{\pgfqpoint{-0.005556in}{0.000000in}}%
\pgfpathcurveto{\pgfqpoint{-0.005556in}{-0.001473in}}{\pgfqpoint{-0.004970in}{-0.002887in}}{\pgfqpoint{-0.003928in}{-0.003928in}}%
\pgfpathcurveto{\pgfqpoint{-0.002887in}{-0.004970in}}{\pgfqpoint{-0.001473in}{-0.005556in}}{\pgfqpoint{0.000000in}{-0.005556in}}%
\pgfpathclose%
\pgfusepath{stroke,fill}%
}%
\begin{pgfscope}%
\pgfsys@transformshift{1.330071in}{3.026783in}%
\pgfsys@useobject{currentmarker}{}%
\end{pgfscope}%
\end{pgfscope}%
\begin{pgfscope}%
\pgfpathrectangle{\pgfqpoint{0.100000in}{2.413063in}}{\pgfqpoint{5.037500in}{3.427208in}}%
\pgfusepath{clip}%
\pgfsetrectcap%
\pgfsetroundjoin%
\pgfsetlinewidth{1.505625pt}%
\definecolor{currentstroke}{rgb}{0.678431,1.000000,0.184314}%
\pgfsetstrokecolor{currentstroke}%
\pgfsetstrokeopacity{0.500000}%
\pgfsetdash{}{0pt}%
\pgfpathmoveto{\pgfqpoint{1.242579in}{3.986452in}}%
\pgfusepath{stroke}%
\end{pgfscope}%
\begin{pgfscope}%
\pgfpathrectangle{\pgfqpoint{0.100000in}{2.413063in}}{\pgfqpoint{5.037500in}{3.427208in}}%
\pgfusepath{clip}%
\pgfsetbuttcap%
\pgfsetroundjoin%
\definecolor{currentfill}{rgb}{0.678431,1.000000,0.184314}%
\pgfsetfillcolor{currentfill}%
\pgfsetfillopacity{0.500000}%
\pgfsetlinewidth{0.250937pt}%
\definecolor{currentstroke}{rgb}{0.000000,0.000000,0.000000}%
\pgfsetstrokecolor{currentstroke}%
\pgfsetstrokeopacity{0.500000}%
\pgfsetdash{}{0pt}%
\pgfsys@defobject{currentmarker}{\pgfqpoint{-0.058333in}{-0.058333in}}{\pgfqpoint{0.058333in}{0.058333in}}{%
\pgfpathmoveto{\pgfqpoint{0.000000in}{-0.058333in}}%
\pgfpathcurveto{\pgfqpoint{0.015470in}{-0.058333in}}{\pgfqpoint{0.030309in}{-0.052187in}}{\pgfqpoint{0.041248in}{-0.041248in}}%
\pgfpathcurveto{\pgfqpoint{0.052187in}{-0.030309in}}{\pgfqpoint{0.058333in}{-0.015470in}}{\pgfqpoint{0.058333in}{0.000000in}}%
\pgfpathcurveto{\pgfqpoint{0.058333in}{0.015470in}}{\pgfqpoint{0.052187in}{0.030309in}}{\pgfqpoint{0.041248in}{0.041248in}}%
\pgfpathcurveto{\pgfqpoint{0.030309in}{0.052187in}}{\pgfqpoint{0.015470in}{0.058333in}}{\pgfqpoint{0.000000in}{0.058333in}}%
\pgfpathcurveto{\pgfqpoint{-0.015470in}{0.058333in}}{\pgfqpoint{-0.030309in}{0.052187in}}{\pgfqpoint{-0.041248in}{0.041248in}}%
\pgfpathcurveto{\pgfqpoint{-0.052187in}{0.030309in}}{\pgfqpoint{-0.058333in}{0.015470in}}{\pgfqpoint{-0.058333in}{0.000000in}}%
\pgfpathcurveto{\pgfqpoint{-0.058333in}{-0.015470in}}{\pgfqpoint{-0.052187in}{-0.030309in}}{\pgfqpoint{-0.041248in}{-0.041248in}}%
\pgfpathcurveto{\pgfqpoint{-0.030309in}{-0.052187in}}{\pgfqpoint{-0.015470in}{-0.058333in}}{\pgfqpoint{0.000000in}{-0.058333in}}%
\pgfpathclose%
\pgfusepath{stroke,fill}%
}%
\begin{pgfscope}%
\pgfsys@transformshift{1.242579in}{3.986452in}%
\pgfsys@useobject{currentmarker}{}%
\end{pgfscope}%
\end{pgfscope}%
\begin{pgfscope}%
\pgfpathrectangle{\pgfqpoint{0.100000in}{2.413063in}}{\pgfqpoint{5.037500in}{3.427208in}}%
\pgfusepath{clip}%
\pgfsetrectcap%
\pgfsetroundjoin%
\pgfsetlinewidth{1.505625pt}%
\definecolor{currentstroke}{rgb}{0.678431,1.000000,0.184314}%
\pgfsetstrokecolor{currentstroke}%
\pgfsetstrokeopacity{0.500000}%
\pgfsetdash{}{0pt}%
\pgfpathmoveto{\pgfqpoint{0.975939in}{3.954572in}}%
\pgfusepath{stroke}%
\end{pgfscope}%
\begin{pgfscope}%
\pgfpathrectangle{\pgfqpoint{0.100000in}{2.413063in}}{\pgfqpoint{5.037500in}{3.427208in}}%
\pgfusepath{clip}%
\pgfsetbuttcap%
\pgfsetroundjoin%
\definecolor{currentfill}{rgb}{0.678431,1.000000,0.184314}%
\pgfsetfillcolor{currentfill}%
\pgfsetfillopacity{0.500000}%
\pgfsetlinewidth{0.250937pt}%
\definecolor{currentstroke}{rgb}{0.000000,0.000000,0.000000}%
\pgfsetstrokecolor{currentstroke}%
\pgfsetstrokeopacity{0.500000}%
\pgfsetdash{}{0pt}%
\pgfsys@defobject{currentmarker}{\pgfqpoint{-0.063889in}{-0.063889in}}{\pgfqpoint{0.063889in}{0.063889in}}{%
\pgfpathmoveto{\pgfqpoint{0.000000in}{-0.063889in}}%
\pgfpathcurveto{\pgfqpoint{0.016944in}{-0.063889in}}{\pgfqpoint{0.033195in}{-0.057157in}}{\pgfqpoint{0.045176in}{-0.045176in}}%
\pgfpathcurveto{\pgfqpoint{0.057157in}{-0.033195in}}{\pgfqpoint{0.063889in}{-0.016944in}}{\pgfqpoint{0.063889in}{0.000000in}}%
\pgfpathcurveto{\pgfqpoint{0.063889in}{0.016944in}}{\pgfqpoint{0.057157in}{0.033195in}}{\pgfqpoint{0.045176in}{0.045176in}}%
\pgfpathcurveto{\pgfqpoint{0.033195in}{0.057157in}}{\pgfqpoint{0.016944in}{0.063889in}}{\pgfqpoint{0.000000in}{0.063889in}}%
\pgfpathcurveto{\pgfqpoint{-0.016944in}{0.063889in}}{\pgfqpoint{-0.033195in}{0.057157in}}{\pgfqpoint{-0.045176in}{0.045176in}}%
\pgfpathcurveto{\pgfqpoint{-0.057157in}{0.033195in}}{\pgfqpoint{-0.063889in}{0.016944in}}{\pgfqpoint{-0.063889in}{0.000000in}}%
\pgfpathcurveto{\pgfqpoint{-0.063889in}{-0.016944in}}{\pgfqpoint{-0.057157in}{-0.033195in}}{\pgfqpoint{-0.045176in}{-0.045176in}}%
\pgfpathcurveto{\pgfqpoint{-0.033195in}{-0.057157in}}{\pgfqpoint{-0.016944in}{-0.063889in}}{\pgfqpoint{0.000000in}{-0.063889in}}%
\pgfpathclose%
\pgfusepath{stroke,fill}%
}%
\begin{pgfscope}%
\pgfsys@transformshift{0.975939in}{3.954572in}%
\pgfsys@useobject{currentmarker}{}%
\end{pgfscope}%
\end{pgfscope}%
\begin{pgfscope}%
\pgfpathrectangle{\pgfqpoint{0.100000in}{2.413063in}}{\pgfqpoint{5.037500in}{3.427208in}}%
\pgfusepath{clip}%
\pgfsetrectcap%
\pgfsetroundjoin%
\pgfsetlinewidth{1.505625pt}%
\definecolor{currentstroke}{rgb}{0.678431,1.000000,0.184314}%
\pgfsetstrokecolor{currentstroke}%
\pgfsetstrokeopacity{0.500000}%
\pgfsetdash{}{0pt}%
\pgfpathmoveto{\pgfqpoint{1.165310in}{3.795127in}}%
\pgfusepath{stroke}%
\end{pgfscope}%
\begin{pgfscope}%
\pgfpathrectangle{\pgfqpoint{0.100000in}{2.413063in}}{\pgfqpoint{5.037500in}{3.427208in}}%
\pgfusepath{clip}%
\pgfsetbuttcap%
\pgfsetroundjoin%
\definecolor{currentfill}{rgb}{0.678431,1.000000,0.184314}%
\pgfsetfillcolor{currentfill}%
\pgfsetfillopacity{0.500000}%
\pgfsetlinewidth{0.250937pt}%
\definecolor{currentstroke}{rgb}{0.000000,0.000000,0.000000}%
\pgfsetstrokecolor{currentstroke}%
\pgfsetstrokeopacity{0.500000}%
\pgfsetdash{}{0pt}%
\pgfsys@defobject{currentmarker}{\pgfqpoint{-0.041667in}{-0.041667in}}{\pgfqpoint{0.041667in}{0.041667in}}{%
\pgfpathmoveto{\pgfqpoint{0.000000in}{-0.041667in}}%
\pgfpathcurveto{\pgfqpoint{0.011050in}{-0.041667in}}{\pgfqpoint{0.021649in}{-0.037276in}}{\pgfqpoint{0.029463in}{-0.029463in}}%
\pgfpathcurveto{\pgfqpoint{0.037276in}{-0.021649in}}{\pgfqpoint{0.041667in}{-0.011050in}}{\pgfqpoint{0.041667in}{0.000000in}}%
\pgfpathcurveto{\pgfqpoint{0.041667in}{0.011050in}}{\pgfqpoint{0.037276in}{0.021649in}}{\pgfqpoint{0.029463in}{0.029463in}}%
\pgfpathcurveto{\pgfqpoint{0.021649in}{0.037276in}}{\pgfqpoint{0.011050in}{0.041667in}}{\pgfqpoint{0.000000in}{0.041667in}}%
\pgfpathcurveto{\pgfqpoint{-0.011050in}{0.041667in}}{\pgfqpoint{-0.021649in}{0.037276in}}{\pgfqpoint{-0.029463in}{0.029463in}}%
\pgfpathcurveto{\pgfqpoint{-0.037276in}{0.021649in}}{\pgfqpoint{-0.041667in}{0.011050in}}{\pgfqpoint{-0.041667in}{0.000000in}}%
\pgfpathcurveto{\pgfqpoint{-0.041667in}{-0.011050in}}{\pgfqpoint{-0.037276in}{-0.021649in}}{\pgfqpoint{-0.029463in}{-0.029463in}}%
\pgfpathcurveto{\pgfqpoint{-0.021649in}{-0.037276in}}{\pgfqpoint{-0.011050in}{-0.041667in}}{\pgfqpoint{0.000000in}{-0.041667in}}%
\pgfpathclose%
\pgfusepath{stroke,fill}%
}%
\begin{pgfscope}%
\pgfsys@transformshift{1.165310in}{3.795127in}%
\pgfsys@useobject{currentmarker}{}%
\end{pgfscope}%
\end{pgfscope}%
\begin{pgfscope}%
\pgfpathrectangle{\pgfqpoint{0.100000in}{2.413063in}}{\pgfqpoint{5.037500in}{3.427208in}}%
\pgfusepath{clip}%
\pgfsetrectcap%
\pgfsetroundjoin%
\pgfsetlinewidth{1.505625pt}%
\definecolor{currentstroke}{rgb}{0.678431,1.000000,0.184314}%
\pgfsetstrokecolor{currentstroke}%
\pgfsetstrokeopacity{0.500000}%
\pgfsetdash{}{0pt}%
\pgfpathmoveto{\pgfqpoint{1.152245in}{3.926148in}}%
\pgfusepath{stroke}%
\end{pgfscope}%
\begin{pgfscope}%
\pgfpathrectangle{\pgfqpoint{0.100000in}{2.413063in}}{\pgfqpoint{5.037500in}{3.427208in}}%
\pgfusepath{clip}%
\pgfsetbuttcap%
\pgfsetroundjoin%
\definecolor{currentfill}{rgb}{0.678431,1.000000,0.184314}%
\pgfsetfillcolor{currentfill}%
\pgfsetfillopacity{0.500000}%
\pgfsetlinewidth{0.250937pt}%
\definecolor{currentstroke}{rgb}{0.000000,0.000000,0.000000}%
\pgfsetstrokecolor{currentstroke}%
\pgfsetstrokeopacity{0.500000}%
\pgfsetdash{}{0pt}%
\pgfsys@defobject{currentmarker}{\pgfqpoint{-0.052778in}{-0.052778in}}{\pgfqpoint{0.052778in}{0.052778in}}{%
\pgfpathmoveto{\pgfqpoint{0.000000in}{-0.052778in}}%
\pgfpathcurveto{\pgfqpoint{0.013997in}{-0.052778in}}{\pgfqpoint{0.027422in}{-0.047217in}}{\pgfqpoint{0.037320in}{-0.037320in}}%
\pgfpathcurveto{\pgfqpoint{0.047217in}{-0.027422in}}{\pgfqpoint{0.052778in}{-0.013997in}}{\pgfqpoint{0.052778in}{0.000000in}}%
\pgfpathcurveto{\pgfqpoint{0.052778in}{0.013997in}}{\pgfqpoint{0.047217in}{0.027422in}}{\pgfqpoint{0.037320in}{0.037320in}}%
\pgfpathcurveto{\pgfqpoint{0.027422in}{0.047217in}}{\pgfqpoint{0.013997in}{0.052778in}}{\pgfqpoint{0.000000in}{0.052778in}}%
\pgfpathcurveto{\pgfqpoint{-0.013997in}{0.052778in}}{\pgfqpoint{-0.027422in}{0.047217in}}{\pgfqpoint{-0.037320in}{0.037320in}}%
\pgfpathcurveto{\pgfqpoint{-0.047217in}{0.027422in}}{\pgfqpoint{-0.052778in}{0.013997in}}{\pgfqpoint{-0.052778in}{0.000000in}}%
\pgfpathcurveto{\pgfqpoint{-0.052778in}{-0.013997in}}{\pgfqpoint{-0.047217in}{-0.027422in}}{\pgfqpoint{-0.037320in}{-0.037320in}}%
\pgfpathcurveto{\pgfqpoint{-0.027422in}{-0.047217in}}{\pgfqpoint{-0.013997in}{-0.052778in}}{\pgfqpoint{0.000000in}{-0.052778in}}%
\pgfpathclose%
\pgfusepath{stroke,fill}%
}%
\begin{pgfscope}%
\pgfsys@transformshift{1.152245in}{3.926148in}%
\pgfsys@useobject{currentmarker}{}%
\end{pgfscope}%
\end{pgfscope}%
\begin{pgfscope}%
\pgfpathrectangle{\pgfqpoint{0.100000in}{2.413063in}}{\pgfqpoint{5.037500in}{3.427208in}}%
\pgfusepath{clip}%
\pgfsetrectcap%
\pgfsetroundjoin%
\pgfsetlinewidth{1.505625pt}%
\definecolor{currentstroke}{rgb}{0.678431,1.000000,0.184314}%
\pgfsetstrokecolor{currentstroke}%
\pgfsetstrokeopacity{0.500000}%
\pgfsetdash{}{0pt}%
\pgfpathmoveto{\pgfqpoint{1.297020in}{3.548381in}}%
\pgfusepath{stroke}%
\end{pgfscope}%
\begin{pgfscope}%
\pgfpathrectangle{\pgfqpoint{0.100000in}{2.413063in}}{\pgfqpoint{5.037500in}{3.427208in}}%
\pgfusepath{clip}%
\pgfsetbuttcap%
\pgfsetroundjoin%
\definecolor{currentfill}{rgb}{0.678431,1.000000,0.184314}%
\pgfsetfillcolor{currentfill}%
\pgfsetfillopacity{0.500000}%
\pgfsetlinewidth{0.250937pt}%
\definecolor{currentstroke}{rgb}{0.000000,0.000000,0.000000}%
\pgfsetstrokecolor{currentstroke}%
\pgfsetstrokeopacity{0.500000}%
\pgfsetdash{}{0pt}%
\pgfsys@defobject{currentmarker}{\pgfqpoint{-0.072222in}{-0.072222in}}{\pgfqpoint{0.072222in}{0.072222in}}{%
\pgfpathmoveto{\pgfqpoint{0.000000in}{-0.072222in}}%
\pgfpathcurveto{\pgfqpoint{0.019154in}{-0.072222in}}{\pgfqpoint{0.037525in}{-0.064612in}}{\pgfqpoint{0.051069in}{-0.051069in}}%
\pgfpathcurveto{\pgfqpoint{0.064612in}{-0.037525in}}{\pgfqpoint{0.072222in}{-0.019154in}}{\pgfqpoint{0.072222in}{0.000000in}}%
\pgfpathcurveto{\pgfqpoint{0.072222in}{0.019154in}}{\pgfqpoint{0.064612in}{0.037525in}}{\pgfqpoint{0.051069in}{0.051069in}}%
\pgfpathcurveto{\pgfqpoint{0.037525in}{0.064612in}}{\pgfqpoint{0.019154in}{0.072222in}}{\pgfqpoint{0.000000in}{0.072222in}}%
\pgfpathcurveto{\pgfqpoint{-0.019154in}{0.072222in}}{\pgfqpoint{-0.037525in}{0.064612in}}{\pgfqpoint{-0.051069in}{0.051069in}}%
\pgfpathcurveto{\pgfqpoint{-0.064612in}{0.037525in}}{\pgfqpoint{-0.072222in}{0.019154in}}{\pgfqpoint{-0.072222in}{0.000000in}}%
\pgfpathcurveto{\pgfqpoint{-0.072222in}{-0.019154in}}{\pgfqpoint{-0.064612in}{-0.037525in}}{\pgfqpoint{-0.051069in}{-0.051069in}}%
\pgfpathcurveto{\pgfqpoint{-0.037525in}{-0.064612in}}{\pgfqpoint{-0.019154in}{-0.072222in}}{\pgfqpoint{0.000000in}{-0.072222in}}%
\pgfpathclose%
\pgfusepath{stroke,fill}%
}%
\begin{pgfscope}%
\pgfsys@transformshift{1.297020in}{3.548381in}%
\pgfsys@useobject{currentmarker}{}%
\end{pgfscope}%
\end{pgfscope}%
\begin{pgfscope}%
\pgfpathrectangle{\pgfqpoint{0.100000in}{2.413063in}}{\pgfqpoint{5.037500in}{3.427208in}}%
\pgfusepath{clip}%
\pgfsetrectcap%
\pgfsetroundjoin%
\pgfsetlinewidth{1.505625pt}%
\definecolor{currentstroke}{rgb}{0.678431,1.000000,0.184314}%
\pgfsetstrokecolor{currentstroke}%
\pgfsetstrokeopacity{0.500000}%
\pgfsetdash{}{0pt}%
\pgfpathmoveto{\pgfqpoint{1.249803in}{3.635149in}}%
\pgfusepath{stroke}%
\end{pgfscope}%
\begin{pgfscope}%
\pgfpathrectangle{\pgfqpoint{0.100000in}{2.413063in}}{\pgfqpoint{5.037500in}{3.427208in}}%
\pgfusepath{clip}%
\pgfsetbuttcap%
\pgfsetroundjoin%
\definecolor{currentfill}{rgb}{0.678431,1.000000,0.184314}%
\pgfsetfillcolor{currentfill}%
\pgfsetfillopacity{0.500000}%
\pgfsetlinewidth{0.250937pt}%
\definecolor{currentstroke}{rgb}{0.000000,0.000000,0.000000}%
\pgfsetstrokecolor{currentstroke}%
\pgfsetstrokeopacity{0.500000}%
\pgfsetdash{}{0pt}%
\pgfsys@defobject{currentmarker}{\pgfqpoint{-0.038889in}{-0.038889in}}{\pgfqpoint{0.038889in}{0.038889in}}{%
\pgfpathmoveto{\pgfqpoint{0.000000in}{-0.038889in}}%
\pgfpathcurveto{\pgfqpoint{0.010313in}{-0.038889in}}{\pgfqpoint{0.020206in}{-0.034791in}}{\pgfqpoint{0.027499in}{-0.027499in}}%
\pgfpathcurveto{\pgfqpoint{0.034791in}{-0.020206in}}{\pgfqpoint{0.038889in}{-0.010313in}}{\pgfqpoint{0.038889in}{0.000000in}}%
\pgfpathcurveto{\pgfqpoint{0.038889in}{0.010313in}}{\pgfqpoint{0.034791in}{0.020206in}}{\pgfqpoint{0.027499in}{0.027499in}}%
\pgfpathcurveto{\pgfqpoint{0.020206in}{0.034791in}}{\pgfqpoint{0.010313in}{0.038889in}}{\pgfqpoint{0.000000in}{0.038889in}}%
\pgfpathcurveto{\pgfqpoint{-0.010313in}{0.038889in}}{\pgfqpoint{-0.020206in}{0.034791in}}{\pgfqpoint{-0.027499in}{0.027499in}}%
\pgfpathcurveto{\pgfqpoint{-0.034791in}{0.020206in}}{\pgfqpoint{-0.038889in}{0.010313in}}{\pgfqpoint{-0.038889in}{0.000000in}}%
\pgfpathcurveto{\pgfqpoint{-0.038889in}{-0.010313in}}{\pgfqpoint{-0.034791in}{-0.020206in}}{\pgfqpoint{-0.027499in}{-0.027499in}}%
\pgfpathcurveto{\pgfqpoint{-0.020206in}{-0.034791in}}{\pgfqpoint{-0.010313in}{-0.038889in}}{\pgfqpoint{0.000000in}{-0.038889in}}%
\pgfpathclose%
\pgfusepath{stroke,fill}%
}%
\begin{pgfscope}%
\pgfsys@transformshift{1.249803in}{3.635149in}%
\pgfsys@useobject{currentmarker}{}%
\end{pgfscope}%
\end{pgfscope}%
\begin{pgfscope}%
\pgfpathrectangle{\pgfqpoint{0.100000in}{2.413063in}}{\pgfqpoint{5.037500in}{3.427208in}}%
\pgfusepath{clip}%
\pgfsetrectcap%
\pgfsetroundjoin%
\pgfsetlinewidth{1.505625pt}%
\definecolor{currentstroke}{rgb}{0.678431,1.000000,0.184314}%
\pgfsetstrokecolor{currentstroke}%
\pgfsetstrokeopacity{0.500000}%
\pgfsetdash{}{0pt}%
\pgfpathmoveto{\pgfqpoint{0.919145in}{3.752754in}}%
\pgfusepath{stroke}%
\end{pgfscope}%
\begin{pgfscope}%
\pgfpathrectangle{\pgfqpoint{0.100000in}{2.413063in}}{\pgfqpoint{5.037500in}{3.427208in}}%
\pgfusepath{clip}%
\pgfsetbuttcap%
\pgfsetroundjoin%
\definecolor{currentfill}{rgb}{0.678431,1.000000,0.184314}%
\pgfsetfillcolor{currentfill}%
\pgfsetfillopacity{0.500000}%
\pgfsetlinewidth{0.250937pt}%
\definecolor{currentstroke}{rgb}{0.000000,0.000000,0.000000}%
\pgfsetstrokecolor{currentstroke}%
\pgfsetstrokeopacity{0.500000}%
\pgfsetdash{}{0pt}%
\pgfsys@defobject{currentmarker}{\pgfqpoint{-0.152778in}{-0.152778in}}{\pgfqpoint{0.152778in}{0.152778in}}{%
\pgfpathmoveto{\pgfqpoint{0.000000in}{-0.152778in}}%
\pgfpathcurveto{\pgfqpoint{0.040517in}{-0.152778in}}{\pgfqpoint{0.079380in}{-0.136680in}}{\pgfqpoint{0.108030in}{-0.108030in}}%
\pgfpathcurveto{\pgfqpoint{0.136680in}{-0.079380in}}{\pgfqpoint{0.152778in}{-0.040517in}}{\pgfqpoint{0.152778in}{0.000000in}}%
\pgfpathcurveto{\pgfqpoint{0.152778in}{0.040517in}}{\pgfqpoint{0.136680in}{0.079380in}}{\pgfqpoint{0.108030in}{0.108030in}}%
\pgfpathcurveto{\pgfqpoint{0.079380in}{0.136680in}}{\pgfqpoint{0.040517in}{0.152778in}}{\pgfqpoint{0.000000in}{0.152778in}}%
\pgfpathcurveto{\pgfqpoint{-0.040517in}{0.152778in}}{\pgfqpoint{-0.079380in}{0.136680in}}{\pgfqpoint{-0.108030in}{0.108030in}}%
\pgfpathcurveto{\pgfqpoint{-0.136680in}{0.079380in}}{\pgfqpoint{-0.152778in}{0.040517in}}{\pgfqpoint{-0.152778in}{0.000000in}}%
\pgfpathcurveto{\pgfqpoint{-0.152778in}{-0.040517in}}{\pgfqpoint{-0.136680in}{-0.079380in}}{\pgfqpoint{-0.108030in}{-0.108030in}}%
\pgfpathcurveto{\pgfqpoint{-0.079380in}{-0.136680in}}{\pgfqpoint{-0.040517in}{-0.152778in}}{\pgfqpoint{0.000000in}{-0.152778in}}%
\pgfpathclose%
\pgfusepath{stroke,fill}%
}%
\begin{pgfscope}%
\pgfsys@transformshift{0.919145in}{3.752754in}%
\pgfsys@useobject{currentmarker}{}%
\end{pgfscope}%
\end{pgfscope}%
\begin{pgfscope}%
\pgfpathrectangle{\pgfqpoint{0.100000in}{2.413063in}}{\pgfqpoint{5.037500in}{3.427208in}}%
\pgfusepath{clip}%
\pgfsetrectcap%
\pgfsetroundjoin%
\pgfsetlinewidth{1.505625pt}%
\definecolor{currentstroke}{rgb}{0.678431,1.000000,0.184314}%
\pgfsetstrokecolor{currentstroke}%
\pgfsetstrokeopacity{0.500000}%
\pgfsetdash{}{0pt}%
\pgfpathmoveto{\pgfqpoint{2.882663in}{3.942918in}}%
\pgfusepath{stroke}%
\end{pgfscope}%
\begin{pgfscope}%
\pgfpathrectangle{\pgfqpoint{0.100000in}{2.413063in}}{\pgfqpoint{5.037500in}{3.427208in}}%
\pgfusepath{clip}%
\pgfsetbuttcap%
\pgfsetroundjoin%
\definecolor{currentfill}{rgb}{0.678431,1.000000,0.184314}%
\pgfsetfillcolor{currentfill}%
\pgfsetfillopacity{0.500000}%
\pgfsetlinewidth{0.250937pt}%
\definecolor{currentstroke}{rgb}{0.000000,0.000000,0.000000}%
\pgfsetstrokecolor{currentstroke}%
\pgfsetstrokeopacity{0.500000}%
\pgfsetdash{}{0pt}%
\pgfsys@defobject{currentmarker}{\pgfqpoint{-0.019444in}{-0.019444in}}{\pgfqpoint{0.019444in}{0.019444in}}{%
\pgfpathmoveto{\pgfqpoint{0.000000in}{-0.019444in}}%
\pgfpathcurveto{\pgfqpoint{0.005157in}{-0.019444in}}{\pgfqpoint{0.010103in}{-0.017396in}}{\pgfqpoint{0.013749in}{-0.013749in}}%
\pgfpathcurveto{\pgfqpoint{0.017396in}{-0.010103in}}{\pgfqpoint{0.019444in}{-0.005157in}}{\pgfqpoint{0.019444in}{0.000000in}}%
\pgfpathcurveto{\pgfqpoint{0.019444in}{0.005157in}}{\pgfqpoint{0.017396in}{0.010103in}}{\pgfqpoint{0.013749in}{0.013749in}}%
\pgfpathcurveto{\pgfqpoint{0.010103in}{0.017396in}}{\pgfqpoint{0.005157in}{0.019444in}}{\pgfqpoint{0.000000in}{0.019444in}}%
\pgfpathcurveto{\pgfqpoint{-0.005157in}{0.019444in}}{\pgfqpoint{-0.010103in}{0.017396in}}{\pgfqpoint{-0.013749in}{0.013749in}}%
\pgfpathcurveto{\pgfqpoint{-0.017396in}{0.010103in}}{\pgfqpoint{-0.019444in}{0.005157in}}{\pgfqpoint{-0.019444in}{0.000000in}}%
\pgfpathcurveto{\pgfqpoint{-0.019444in}{-0.005157in}}{\pgfqpoint{-0.017396in}{-0.010103in}}{\pgfqpoint{-0.013749in}{-0.013749in}}%
\pgfpathcurveto{\pgfqpoint{-0.010103in}{-0.017396in}}{\pgfqpoint{-0.005157in}{-0.019444in}}{\pgfqpoint{0.000000in}{-0.019444in}}%
\pgfpathclose%
\pgfusepath{stroke,fill}%
}%
\begin{pgfscope}%
\pgfsys@transformshift{2.882663in}{3.942918in}%
\pgfsys@useobject{currentmarker}{}%
\end{pgfscope}%
\end{pgfscope}%
\begin{pgfscope}%
\pgfpathrectangle{\pgfqpoint{0.100000in}{2.413063in}}{\pgfqpoint{5.037500in}{3.427208in}}%
\pgfusepath{clip}%
\pgfsetrectcap%
\pgfsetroundjoin%
\pgfsetlinewidth{1.505625pt}%
\definecolor{currentstroke}{rgb}{0.678431,1.000000,0.184314}%
\pgfsetstrokecolor{currentstroke}%
\pgfsetstrokeopacity{0.500000}%
\pgfsetdash{}{0pt}%
\pgfpathmoveto{\pgfqpoint{2.858226in}{3.864827in}}%
\pgfusepath{stroke}%
\end{pgfscope}%
\begin{pgfscope}%
\pgfpathrectangle{\pgfqpoint{0.100000in}{2.413063in}}{\pgfqpoint{5.037500in}{3.427208in}}%
\pgfusepath{clip}%
\pgfsetbuttcap%
\pgfsetroundjoin%
\definecolor{currentfill}{rgb}{0.678431,1.000000,0.184314}%
\pgfsetfillcolor{currentfill}%
\pgfsetfillopacity{0.500000}%
\pgfsetlinewidth{0.250937pt}%
\definecolor{currentstroke}{rgb}{0.000000,0.000000,0.000000}%
\pgfsetstrokecolor{currentstroke}%
\pgfsetstrokeopacity{0.500000}%
\pgfsetdash{}{0pt}%
\pgfsys@defobject{currentmarker}{\pgfqpoint{-0.033333in}{-0.033333in}}{\pgfqpoint{0.033333in}{0.033333in}}{%
\pgfpathmoveto{\pgfqpoint{0.000000in}{-0.033333in}}%
\pgfpathcurveto{\pgfqpoint{0.008840in}{-0.033333in}}{\pgfqpoint{0.017319in}{-0.029821in}}{\pgfqpoint{0.023570in}{-0.023570in}}%
\pgfpathcurveto{\pgfqpoint{0.029821in}{-0.017319in}}{\pgfqpoint{0.033333in}{-0.008840in}}{\pgfqpoint{0.033333in}{0.000000in}}%
\pgfpathcurveto{\pgfqpoint{0.033333in}{0.008840in}}{\pgfqpoint{0.029821in}{0.017319in}}{\pgfqpoint{0.023570in}{0.023570in}}%
\pgfpathcurveto{\pgfqpoint{0.017319in}{0.029821in}}{\pgfqpoint{0.008840in}{0.033333in}}{\pgfqpoint{0.000000in}{0.033333in}}%
\pgfpathcurveto{\pgfqpoint{-0.008840in}{0.033333in}}{\pgfqpoint{-0.017319in}{0.029821in}}{\pgfqpoint{-0.023570in}{0.023570in}}%
\pgfpathcurveto{\pgfqpoint{-0.029821in}{0.017319in}}{\pgfqpoint{-0.033333in}{0.008840in}}{\pgfqpoint{-0.033333in}{0.000000in}}%
\pgfpathcurveto{\pgfqpoint{-0.033333in}{-0.008840in}}{\pgfqpoint{-0.029821in}{-0.017319in}}{\pgfqpoint{-0.023570in}{-0.023570in}}%
\pgfpathcurveto{\pgfqpoint{-0.017319in}{-0.029821in}}{\pgfqpoint{-0.008840in}{-0.033333in}}{\pgfqpoint{0.000000in}{-0.033333in}}%
\pgfpathclose%
\pgfusepath{stroke,fill}%
}%
\begin{pgfscope}%
\pgfsys@transformshift{2.858226in}{3.864827in}%
\pgfsys@useobject{currentmarker}{}%
\end{pgfscope}%
\end{pgfscope}%
\begin{pgfscope}%
\pgfpathrectangle{\pgfqpoint{0.100000in}{2.413063in}}{\pgfqpoint{5.037500in}{3.427208in}}%
\pgfusepath{clip}%
\pgfsetrectcap%
\pgfsetroundjoin%
\pgfsetlinewidth{1.505625pt}%
\definecolor{currentstroke}{rgb}{0.678431,1.000000,0.184314}%
\pgfsetstrokecolor{currentstroke}%
\pgfsetstrokeopacity{0.500000}%
\pgfsetdash{}{0pt}%
\pgfpathmoveto{\pgfqpoint{2.990171in}{3.758232in}}%
\pgfusepath{stroke}%
\end{pgfscope}%
\begin{pgfscope}%
\pgfpathrectangle{\pgfqpoint{0.100000in}{2.413063in}}{\pgfqpoint{5.037500in}{3.427208in}}%
\pgfusepath{clip}%
\pgfsetbuttcap%
\pgfsetroundjoin%
\definecolor{currentfill}{rgb}{0.678431,1.000000,0.184314}%
\pgfsetfillcolor{currentfill}%
\pgfsetfillopacity{0.500000}%
\pgfsetlinewidth{0.250937pt}%
\definecolor{currentstroke}{rgb}{0.000000,0.000000,0.000000}%
\pgfsetstrokecolor{currentstroke}%
\pgfsetstrokeopacity{0.500000}%
\pgfsetdash{}{0pt}%
\pgfsys@defobject{currentmarker}{\pgfqpoint{-0.011111in}{-0.011111in}}{\pgfqpoint{0.011111in}{0.011111in}}{%
\pgfpathmoveto{\pgfqpoint{0.000000in}{-0.011111in}}%
\pgfpathcurveto{\pgfqpoint{0.002947in}{-0.011111in}}{\pgfqpoint{0.005773in}{-0.009940in}}{\pgfqpoint{0.007857in}{-0.007857in}}%
\pgfpathcurveto{\pgfqpoint{0.009940in}{-0.005773in}}{\pgfqpoint{0.011111in}{-0.002947in}}{\pgfqpoint{0.011111in}{0.000000in}}%
\pgfpathcurveto{\pgfqpoint{0.011111in}{0.002947in}}{\pgfqpoint{0.009940in}{0.005773in}}{\pgfqpoint{0.007857in}{0.007857in}}%
\pgfpathcurveto{\pgfqpoint{0.005773in}{0.009940in}}{\pgfqpoint{0.002947in}{0.011111in}}{\pgfqpoint{0.000000in}{0.011111in}}%
\pgfpathcurveto{\pgfqpoint{-0.002947in}{0.011111in}}{\pgfqpoint{-0.005773in}{0.009940in}}{\pgfqpoint{-0.007857in}{0.007857in}}%
\pgfpathcurveto{\pgfqpoint{-0.009940in}{0.005773in}}{\pgfqpoint{-0.011111in}{0.002947in}}{\pgfqpoint{-0.011111in}{0.000000in}}%
\pgfpathcurveto{\pgfqpoint{-0.011111in}{-0.002947in}}{\pgfqpoint{-0.009940in}{-0.005773in}}{\pgfqpoint{-0.007857in}{-0.007857in}}%
\pgfpathcurveto{\pgfqpoint{-0.005773in}{-0.009940in}}{\pgfqpoint{-0.002947in}{-0.011111in}}{\pgfqpoint{0.000000in}{-0.011111in}}%
\pgfpathclose%
\pgfusepath{stroke,fill}%
}%
\begin{pgfscope}%
\pgfsys@transformshift{2.990171in}{3.758232in}%
\pgfsys@useobject{currentmarker}{}%
\end{pgfscope}%
\end{pgfscope}%
\begin{pgfscope}%
\pgfpathrectangle{\pgfqpoint{0.100000in}{2.413063in}}{\pgfqpoint{5.037500in}{3.427208in}}%
\pgfusepath{clip}%
\pgfsetrectcap%
\pgfsetroundjoin%
\pgfsetlinewidth{1.505625pt}%
\definecolor{currentstroke}{rgb}{0.678431,1.000000,0.184314}%
\pgfsetstrokecolor{currentstroke}%
\pgfsetstrokeopacity{0.500000}%
\pgfsetdash{}{0pt}%
\pgfpathmoveto{\pgfqpoint{3.205657in}{3.926711in}}%
\pgfusepath{stroke}%
\end{pgfscope}%
\begin{pgfscope}%
\pgfpathrectangle{\pgfqpoint{0.100000in}{2.413063in}}{\pgfqpoint{5.037500in}{3.427208in}}%
\pgfusepath{clip}%
\pgfsetbuttcap%
\pgfsetroundjoin%
\definecolor{currentfill}{rgb}{0.678431,1.000000,0.184314}%
\pgfsetfillcolor{currentfill}%
\pgfsetfillopacity{0.500000}%
\pgfsetlinewidth{0.250937pt}%
\definecolor{currentstroke}{rgb}{0.000000,0.000000,0.000000}%
\pgfsetstrokecolor{currentstroke}%
\pgfsetstrokeopacity{0.500000}%
\pgfsetdash{}{0pt}%
\pgfsys@defobject{currentmarker}{\pgfqpoint{-0.019444in}{-0.019444in}}{\pgfqpoint{0.019444in}{0.019444in}}{%
\pgfpathmoveto{\pgfqpoint{0.000000in}{-0.019444in}}%
\pgfpathcurveto{\pgfqpoint{0.005157in}{-0.019444in}}{\pgfqpoint{0.010103in}{-0.017396in}}{\pgfqpoint{0.013749in}{-0.013749in}}%
\pgfpathcurveto{\pgfqpoint{0.017396in}{-0.010103in}}{\pgfqpoint{0.019444in}{-0.005157in}}{\pgfqpoint{0.019444in}{0.000000in}}%
\pgfpathcurveto{\pgfqpoint{0.019444in}{0.005157in}}{\pgfqpoint{0.017396in}{0.010103in}}{\pgfqpoint{0.013749in}{0.013749in}}%
\pgfpathcurveto{\pgfqpoint{0.010103in}{0.017396in}}{\pgfqpoint{0.005157in}{0.019444in}}{\pgfqpoint{0.000000in}{0.019444in}}%
\pgfpathcurveto{\pgfqpoint{-0.005157in}{0.019444in}}{\pgfqpoint{-0.010103in}{0.017396in}}{\pgfqpoint{-0.013749in}{0.013749in}}%
\pgfpathcurveto{\pgfqpoint{-0.017396in}{0.010103in}}{\pgfqpoint{-0.019444in}{0.005157in}}{\pgfqpoint{-0.019444in}{0.000000in}}%
\pgfpathcurveto{\pgfqpoint{-0.019444in}{-0.005157in}}{\pgfqpoint{-0.017396in}{-0.010103in}}{\pgfqpoint{-0.013749in}{-0.013749in}}%
\pgfpathcurveto{\pgfqpoint{-0.010103in}{-0.017396in}}{\pgfqpoint{-0.005157in}{-0.019444in}}{\pgfqpoint{0.000000in}{-0.019444in}}%
\pgfpathclose%
\pgfusepath{stroke,fill}%
}%
\begin{pgfscope}%
\pgfsys@transformshift{3.205657in}{3.926711in}%
\pgfsys@useobject{currentmarker}{}%
\end{pgfscope}%
\end{pgfscope}%
\begin{pgfscope}%
\pgfpathrectangle{\pgfqpoint{0.100000in}{2.413063in}}{\pgfqpoint{5.037500in}{3.427208in}}%
\pgfusepath{clip}%
\pgfsetrectcap%
\pgfsetroundjoin%
\pgfsetlinewidth{1.505625pt}%
\definecolor{currentstroke}{rgb}{0.678431,1.000000,0.184314}%
\pgfsetstrokecolor{currentstroke}%
\pgfsetstrokeopacity{0.500000}%
\pgfsetdash{}{0pt}%
\pgfpathmoveto{\pgfqpoint{3.061256in}{3.794532in}}%
\pgfusepath{stroke}%
\end{pgfscope}%
\begin{pgfscope}%
\pgfpathrectangle{\pgfqpoint{0.100000in}{2.413063in}}{\pgfqpoint{5.037500in}{3.427208in}}%
\pgfusepath{clip}%
\pgfsetbuttcap%
\pgfsetroundjoin%
\definecolor{currentfill}{rgb}{0.678431,1.000000,0.184314}%
\pgfsetfillcolor{currentfill}%
\pgfsetfillopacity{0.500000}%
\pgfsetlinewidth{0.250937pt}%
\definecolor{currentstroke}{rgb}{0.000000,0.000000,0.000000}%
\pgfsetstrokecolor{currentstroke}%
\pgfsetstrokeopacity{0.500000}%
\pgfsetdash{}{0pt}%
\pgfsys@defobject{currentmarker}{\pgfqpoint{-0.016667in}{-0.016667in}}{\pgfqpoint{0.016667in}{0.016667in}}{%
\pgfpathmoveto{\pgfqpoint{0.000000in}{-0.016667in}}%
\pgfpathcurveto{\pgfqpoint{0.004420in}{-0.016667in}}{\pgfqpoint{0.008660in}{-0.014911in}}{\pgfqpoint{0.011785in}{-0.011785in}}%
\pgfpathcurveto{\pgfqpoint{0.014911in}{-0.008660in}}{\pgfqpoint{0.016667in}{-0.004420in}}{\pgfqpoint{0.016667in}{0.000000in}}%
\pgfpathcurveto{\pgfqpoint{0.016667in}{0.004420in}}{\pgfqpoint{0.014911in}{0.008660in}}{\pgfqpoint{0.011785in}{0.011785in}}%
\pgfpathcurveto{\pgfqpoint{0.008660in}{0.014911in}}{\pgfqpoint{0.004420in}{0.016667in}}{\pgfqpoint{0.000000in}{0.016667in}}%
\pgfpathcurveto{\pgfqpoint{-0.004420in}{0.016667in}}{\pgfqpoint{-0.008660in}{0.014911in}}{\pgfqpoint{-0.011785in}{0.011785in}}%
\pgfpathcurveto{\pgfqpoint{-0.014911in}{0.008660in}}{\pgfqpoint{-0.016667in}{0.004420in}}{\pgfqpoint{-0.016667in}{0.000000in}}%
\pgfpathcurveto{\pgfqpoint{-0.016667in}{-0.004420in}}{\pgfqpoint{-0.014911in}{-0.008660in}}{\pgfqpoint{-0.011785in}{-0.011785in}}%
\pgfpathcurveto{\pgfqpoint{-0.008660in}{-0.014911in}}{\pgfqpoint{-0.004420in}{-0.016667in}}{\pgfqpoint{0.000000in}{-0.016667in}}%
\pgfpathclose%
\pgfusepath{stroke,fill}%
}%
\begin{pgfscope}%
\pgfsys@transformshift{3.061256in}{3.794532in}%
\pgfsys@useobject{currentmarker}{}%
\end{pgfscope}%
\end{pgfscope}%
\begin{pgfscope}%
\pgfpathrectangle{\pgfqpoint{0.100000in}{2.413063in}}{\pgfqpoint{5.037500in}{3.427208in}}%
\pgfusepath{clip}%
\pgfsetrectcap%
\pgfsetroundjoin%
\pgfsetlinewidth{1.505625pt}%
\definecolor{currentstroke}{rgb}{0.678431,1.000000,0.184314}%
\pgfsetstrokecolor{currentstroke}%
\pgfsetstrokeopacity{0.500000}%
\pgfsetdash{}{0pt}%
\pgfpathmoveto{\pgfqpoint{3.090389in}{3.735577in}}%
\pgfusepath{stroke}%
\end{pgfscope}%
\begin{pgfscope}%
\pgfpathrectangle{\pgfqpoint{0.100000in}{2.413063in}}{\pgfqpoint{5.037500in}{3.427208in}}%
\pgfusepath{clip}%
\pgfsetbuttcap%
\pgfsetroundjoin%
\definecolor{currentfill}{rgb}{0.678431,1.000000,0.184314}%
\pgfsetfillcolor{currentfill}%
\pgfsetfillopacity{0.500000}%
\pgfsetlinewidth{0.250937pt}%
\definecolor{currentstroke}{rgb}{0.000000,0.000000,0.000000}%
\pgfsetstrokecolor{currentstroke}%
\pgfsetstrokeopacity{0.500000}%
\pgfsetdash{}{0pt}%
\pgfsys@defobject{currentmarker}{\pgfqpoint{-0.030556in}{-0.030556in}}{\pgfqpoint{0.030556in}{0.030556in}}{%
\pgfpathmoveto{\pgfqpoint{0.000000in}{-0.030556in}}%
\pgfpathcurveto{\pgfqpoint{0.008103in}{-0.030556in}}{\pgfqpoint{0.015876in}{-0.027336in}}{\pgfqpoint{0.021606in}{-0.021606in}}%
\pgfpathcurveto{\pgfqpoint{0.027336in}{-0.015876in}}{\pgfqpoint{0.030556in}{-0.008103in}}{\pgfqpoint{0.030556in}{0.000000in}}%
\pgfpathcurveto{\pgfqpoint{0.030556in}{0.008103in}}{\pgfqpoint{0.027336in}{0.015876in}}{\pgfqpoint{0.021606in}{0.021606in}}%
\pgfpathcurveto{\pgfqpoint{0.015876in}{0.027336in}}{\pgfqpoint{0.008103in}{0.030556in}}{\pgfqpoint{0.000000in}{0.030556in}}%
\pgfpathcurveto{\pgfqpoint{-0.008103in}{0.030556in}}{\pgfqpoint{-0.015876in}{0.027336in}}{\pgfqpoint{-0.021606in}{0.021606in}}%
\pgfpathcurveto{\pgfqpoint{-0.027336in}{0.015876in}}{\pgfqpoint{-0.030556in}{0.008103in}}{\pgfqpoint{-0.030556in}{0.000000in}}%
\pgfpathcurveto{\pgfqpoint{-0.030556in}{-0.008103in}}{\pgfqpoint{-0.027336in}{-0.015876in}}{\pgfqpoint{-0.021606in}{-0.021606in}}%
\pgfpathcurveto{\pgfqpoint{-0.015876in}{-0.027336in}}{\pgfqpoint{-0.008103in}{-0.030556in}}{\pgfqpoint{0.000000in}{-0.030556in}}%
\pgfpathclose%
\pgfusepath{stroke,fill}%
}%
\begin{pgfscope}%
\pgfsys@transformshift{3.090389in}{3.735577in}%
\pgfsys@useobject{currentmarker}{}%
\end{pgfscope}%
\end{pgfscope}%
\begin{pgfscope}%
\pgfpathrectangle{\pgfqpoint{0.100000in}{2.413063in}}{\pgfqpoint{5.037500in}{3.427208in}}%
\pgfusepath{clip}%
\pgfsetrectcap%
\pgfsetroundjoin%
\pgfsetlinewidth{1.505625pt}%
\definecolor{currentstroke}{rgb}{0.501961,0.501961,0.501961}%
\pgfsetstrokecolor{currentstroke}%
\pgfsetstrokeopacity{0.500000}%
\pgfsetdash{}{0pt}%
\pgfpathmoveto{\pgfqpoint{0.570500in}{4.160029in}}%
\pgfusepath{stroke}%
\end{pgfscope}%
\begin{pgfscope}%
\pgfpathrectangle{\pgfqpoint{0.100000in}{2.413063in}}{\pgfqpoint{5.037500in}{3.427208in}}%
\pgfusepath{clip}%
\pgfsetbuttcap%
\pgfsetroundjoin%
\definecolor{currentfill}{rgb}{0.501961,0.501961,0.501961}%
\pgfsetfillcolor{currentfill}%
\pgfsetfillopacity{0.500000}%
\pgfsetlinewidth{0.250937pt}%
\definecolor{currentstroke}{rgb}{0.000000,0.000000,0.000000}%
\pgfsetstrokecolor{currentstroke}%
\pgfsetstrokeopacity{0.500000}%
\pgfsetdash{}{0pt}%
\pgfsys@defobject{currentmarker}{\pgfqpoint{-0.013889in}{-0.013889in}}{\pgfqpoint{0.013889in}{0.013889in}}{%
\pgfpathmoveto{\pgfqpoint{0.000000in}{-0.013889in}}%
\pgfpathcurveto{\pgfqpoint{0.003683in}{-0.013889in}}{\pgfqpoint{0.007216in}{-0.012425in}}{\pgfqpoint{0.009821in}{-0.009821in}}%
\pgfpathcurveto{\pgfqpoint{0.012425in}{-0.007216in}}{\pgfqpoint{0.013889in}{-0.003683in}}{\pgfqpoint{0.013889in}{0.000000in}}%
\pgfpathcurveto{\pgfqpoint{0.013889in}{0.003683in}}{\pgfqpoint{0.012425in}{0.007216in}}{\pgfqpoint{0.009821in}{0.009821in}}%
\pgfpathcurveto{\pgfqpoint{0.007216in}{0.012425in}}{\pgfqpoint{0.003683in}{0.013889in}}{\pgfqpoint{0.000000in}{0.013889in}}%
\pgfpathcurveto{\pgfqpoint{-0.003683in}{0.013889in}}{\pgfqpoint{-0.007216in}{0.012425in}}{\pgfqpoint{-0.009821in}{0.009821in}}%
\pgfpathcurveto{\pgfqpoint{-0.012425in}{0.007216in}}{\pgfqpoint{-0.013889in}{0.003683in}}{\pgfqpoint{-0.013889in}{0.000000in}}%
\pgfpathcurveto{\pgfqpoint{-0.013889in}{-0.003683in}}{\pgfqpoint{-0.012425in}{-0.007216in}}{\pgfqpoint{-0.009821in}{-0.009821in}}%
\pgfpathcurveto{\pgfqpoint{-0.007216in}{-0.012425in}}{\pgfqpoint{-0.003683in}{-0.013889in}}{\pgfqpoint{0.000000in}{-0.013889in}}%
\pgfpathclose%
\pgfusepath{stroke,fill}%
}%
\begin{pgfscope}%
\pgfsys@transformshift{0.570500in}{4.160029in}%
\pgfsys@useobject{currentmarker}{}%
\end{pgfscope}%
\end{pgfscope}%
\begin{pgfscope}%
\pgfpathrectangle{\pgfqpoint{0.100000in}{2.413063in}}{\pgfqpoint{5.037500in}{3.427208in}}%
\pgfusepath{clip}%
\pgfsetrectcap%
\pgfsetroundjoin%
\pgfsetlinewidth{1.505625pt}%
\definecolor{currentstroke}{rgb}{0.000000,0.000000,1.000000}%
\pgfsetstrokecolor{currentstroke}%
\pgfsetstrokeopacity{0.500000}%
\pgfsetdash{}{0pt}%
\pgfpathmoveto{\pgfqpoint{0.461714in}{4.713248in}}%
\pgfusepath{stroke}%
\end{pgfscope}%
\begin{pgfscope}%
\pgfpathrectangle{\pgfqpoint{0.100000in}{2.413063in}}{\pgfqpoint{5.037500in}{3.427208in}}%
\pgfusepath{clip}%
\pgfsetbuttcap%
\pgfsetroundjoin%
\definecolor{currentfill}{rgb}{0.000000,0.000000,1.000000}%
\pgfsetfillcolor{currentfill}%
\pgfsetfillopacity{0.500000}%
\pgfsetlinewidth{0.250937pt}%
\definecolor{currentstroke}{rgb}{0.000000,0.000000,0.000000}%
\pgfsetstrokecolor{currentstroke}%
\pgfsetstrokeopacity{0.500000}%
\pgfsetdash{}{0pt}%
\pgfsys@defobject{currentmarker}{\pgfqpoint{-0.005556in}{-0.005556in}}{\pgfqpoint{0.005556in}{0.005556in}}{%
\pgfpathmoveto{\pgfqpoint{0.000000in}{-0.005556in}}%
\pgfpathcurveto{\pgfqpoint{0.001473in}{-0.005556in}}{\pgfqpoint{0.002887in}{-0.004970in}}{\pgfqpoint{0.003928in}{-0.003928in}}%
\pgfpathcurveto{\pgfqpoint{0.004970in}{-0.002887in}}{\pgfqpoint{0.005556in}{-0.001473in}}{\pgfqpoint{0.005556in}{0.000000in}}%
\pgfpathcurveto{\pgfqpoint{0.005556in}{0.001473in}}{\pgfqpoint{0.004970in}{0.002887in}}{\pgfqpoint{0.003928in}{0.003928in}}%
\pgfpathcurveto{\pgfqpoint{0.002887in}{0.004970in}}{\pgfqpoint{0.001473in}{0.005556in}}{\pgfqpoint{0.000000in}{0.005556in}}%
\pgfpathcurveto{\pgfqpoint{-0.001473in}{0.005556in}}{\pgfqpoint{-0.002887in}{0.004970in}}{\pgfqpoint{-0.003928in}{0.003928in}}%
\pgfpathcurveto{\pgfqpoint{-0.004970in}{0.002887in}}{\pgfqpoint{-0.005556in}{0.001473in}}{\pgfqpoint{-0.005556in}{0.000000in}}%
\pgfpathcurveto{\pgfqpoint{-0.005556in}{-0.001473in}}{\pgfqpoint{-0.004970in}{-0.002887in}}{\pgfqpoint{-0.003928in}{-0.003928in}}%
\pgfpathcurveto{\pgfqpoint{-0.002887in}{-0.004970in}}{\pgfqpoint{-0.001473in}{-0.005556in}}{\pgfqpoint{0.000000in}{-0.005556in}}%
\pgfpathclose%
\pgfusepath{stroke,fill}%
}%
\begin{pgfscope}%
\pgfsys@transformshift{0.461714in}{4.713248in}%
\pgfsys@useobject{currentmarker}{}%
\end{pgfscope}%
\end{pgfscope}%
\begin{pgfscope}%
\pgfpathrectangle{\pgfqpoint{0.100000in}{2.413063in}}{\pgfqpoint{5.037500in}{3.427208in}}%
\pgfusepath{clip}%
\pgfsetrectcap%
\pgfsetroundjoin%
\pgfsetlinewidth{1.505625pt}%
\definecolor{currentstroke}{rgb}{0.678431,1.000000,0.184314}%
\pgfsetstrokecolor{currentstroke}%
\pgfsetstrokeopacity{0.500000}%
\pgfsetdash{}{0pt}%
\pgfpathmoveto{\pgfqpoint{0.818988in}{3.790241in}}%
\pgfusepath{stroke}%
\end{pgfscope}%
\begin{pgfscope}%
\pgfpathrectangle{\pgfqpoint{0.100000in}{2.413063in}}{\pgfqpoint{5.037500in}{3.427208in}}%
\pgfusepath{clip}%
\pgfsetbuttcap%
\pgfsetroundjoin%
\definecolor{currentfill}{rgb}{0.678431,1.000000,0.184314}%
\pgfsetfillcolor{currentfill}%
\pgfsetfillopacity{0.500000}%
\pgfsetlinewidth{0.250937pt}%
\definecolor{currentstroke}{rgb}{0.000000,0.000000,0.000000}%
\pgfsetstrokecolor{currentstroke}%
\pgfsetstrokeopacity{0.500000}%
\pgfsetdash{}{0pt}%
\pgfsys@defobject{currentmarker}{\pgfqpoint{-0.172222in}{-0.172222in}}{\pgfqpoint{0.172222in}{0.172222in}}{%
\pgfpathmoveto{\pgfqpoint{0.000000in}{-0.172222in}}%
\pgfpathcurveto{\pgfqpoint{0.045674in}{-0.172222in}}{\pgfqpoint{0.089483in}{-0.154076in}}{\pgfqpoint{0.121780in}{-0.121780in}}%
\pgfpathcurveto{\pgfqpoint{0.154076in}{-0.089483in}}{\pgfqpoint{0.172222in}{-0.045674in}}{\pgfqpoint{0.172222in}{0.000000in}}%
\pgfpathcurveto{\pgfqpoint{0.172222in}{0.045674in}}{\pgfqpoint{0.154076in}{0.089483in}}{\pgfqpoint{0.121780in}{0.121780in}}%
\pgfpathcurveto{\pgfqpoint{0.089483in}{0.154076in}}{\pgfqpoint{0.045674in}{0.172222in}}{\pgfqpoint{0.000000in}{0.172222in}}%
\pgfpathcurveto{\pgfqpoint{-0.045674in}{0.172222in}}{\pgfqpoint{-0.089483in}{0.154076in}}{\pgfqpoint{-0.121780in}{0.121780in}}%
\pgfpathcurveto{\pgfqpoint{-0.154076in}{0.089483in}}{\pgfqpoint{-0.172222in}{0.045674in}}{\pgfqpoint{-0.172222in}{0.000000in}}%
\pgfpathcurveto{\pgfqpoint{-0.172222in}{-0.045674in}}{\pgfqpoint{-0.154076in}{-0.089483in}}{\pgfqpoint{-0.121780in}{-0.121780in}}%
\pgfpathcurveto{\pgfqpoint{-0.089483in}{-0.154076in}}{\pgfqpoint{-0.045674in}{-0.172222in}}{\pgfqpoint{0.000000in}{-0.172222in}}%
\pgfpathclose%
\pgfusepath{stroke,fill}%
}%
\begin{pgfscope}%
\pgfsys@transformshift{0.818988in}{3.790241in}%
\pgfsys@useobject{currentmarker}{}%
\end{pgfscope}%
\end{pgfscope}%
\begin{pgfscope}%
\pgfpathrectangle{\pgfqpoint{0.100000in}{2.413063in}}{\pgfqpoint{5.037500in}{3.427208in}}%
\pgfusepath{clip}%
\pgfsetrectcap%
\pgfsetroundjoin%
\pgfsetlinewidth{1.505625pt}%
\definecolor{currentstroke}{rgb}{0.678431,1.000000,0.184314}%
\pgfsetstrokecolor{currentstroke}%
\pgfsetstrokeopacity{0.500000}%
\pgfsetdash{}{0pt}%
\pgfpathmoveto{\pgfqpoint{0.549937in}{4.327788in}}%
\pgfusepath{stroke}%
\end{pgfscope}%
\begin{pgfscope}%
\pgfpathrectangle{\pgfqpoint{0.100000in}{2.413063in}}{\pgfqpoint{5.037500in}{3.427208in}}%
\pgfusepath{clip}%
\pgfsetbuttcap%
\pgfsetroundjoin%
\definecolor{currentfill}{rgb}{0.678431,1.000000,0.184314}%
\pgfsetfillcolor{currentfill}%
\pgfsetfillopacity{0.500000}%
\pgfsetlinewidth{0.250937pt}%
\definecolor{currentstroke}{rgb}{0.000000,0.000000,0.000000}%
\pgfsetstrokecolor{currentstroke}%
\pgfsetstrokeopacity{0.500000}%
\pgfsetdash{}{0pt}%
\pgfsys@defobject{currentmarker}{\pgfqpoint{-0.011111in}{-0.011111in}}{\pgfqpoint{0.011111in}{0.011111in}}{%
\pgfpathmoveto{\pgfqpoint{0.000000in}{-0.011111in}}%
\pgfpathcurveto{\pgfqpoint{0.002947in}{-0.011111in}}{\pgfqpoint{0.005773in}{-0.009940in}}{\pgfqpoint{0.007857in}{-0.007857in}}%
\pgfpathcurveto{\pgfqpoint{0.009940in}{-0.005773in}}{\pgfqpoint{0.011111in}{-0.002947in}}{\pgfqpoint{0.011111in}{0.000000in}}%
\pgfpathcurveto{\pgfqpoint{0.011111in}{0.002947in}}{\pgfqpoint{0.009940in}{0.005773in}}{\pgfqpoint{0.007857in}{0.007857in}}%
\pgfpathcurveto{\pgfqpoint{0.005773in}{0.009940in}}{\pgfqpoint{0.002947in}{0.011111in}}{\pgfqpoint{0.000000in}{0.011111in}}%
\pgfpathcurveto{\pgfqpoint{-0.002947in}{0.011111in}}{\pgfqpoint{-0.005773in}{0.009940in}}{\pgfqpoint{-0.007857in}{0.007857in}}%
\pgfpathcurveto{\pgfqpoint{-0.009940in}{0.005773in}}{\pgfqpoint{-0.011111in}{0.002947in}}{\pgfqpoint{-0.011111in}{0.000000in}}%
\pgfpathcurveto{\pgfqpoint{-0.011111in}{-0.002947in}}{\pgfqpoint{-0.009940in}{-0.005773in}}{\pgfqpoint{-0.007857in}{-0.007857in}}%
\pgfpathcurveto{\pgfqpoint{-0.005773in}{-0.009940in}}{\pgfqpoint{-0.002947in}{-0.011111in}}{\pgfqpoint{0.000000in}{-0.011111in}}%
\pgfpathclose%
\pgfusepath{stroke,fill}%
}%
\begin{pgfscope}%
\pgfsys@transformshift{0.549937in}{4.327788in}%
\pgfsys@useobject{currentmarker}{}%
\end{pgfscope}%
\end{pgfscope}%
\begin{pgfscope}%
\pgfpathrectangle{\pgfqpoint{0.100000in}{2.413063in}}{\pgfqpoint{5.037500in}{3.427208in}}%
\pgfusepath{clip}%
\pgfsetrectcap%
\pgfsetroundjoin%
\pgfsetlinewidth{1.505625pt}%
\definecolor{currentstroke}{rgb}{0.678431,1.000000,0.184314}%
\pgfsetstrokecolor{currentstroke}%
\pgfsetstrokeopacity{0.500000}%
\pgfsetdash{}{0pt}%
\pgfpathmoveto{\pgfqpoint{0.543143in}{4.281573in}}%
\pgfusepath{stroke}%
\end{pgfscope}%
\begin{pgfscope}%
\pgfpathrectangle{\pgfqpoint{0.100000in}{2.413063in}}{\pgfqpoint{5.037500in}{3.427208in}}%
\pgfusepath{clip}%
\pgfsetbuttcap%
\pgfsetroundjoin%
\definecolor{currentfill}{rgb}{0.678431,1.000000,0.184314}%
\pgfsetfillcolor{currentfill}%
\pgfsetfillopacity{0.500000}%
\pgfsetlinewidth{0.250937pt}%
\definecolor{currentstroke}{rgb}{0.000000,0.000000,0.000000}%
\pgfsetstrokecolor{currentstroke}%
\pgfsetstrokeopacity{0.500000}%
\pgfsetdash{}{0pt}%
\pgfsys@defobject{currentmarker}{\pgfqpoint{-0.022222in}{-0.022222in}}{\pgfqpoint{0.022222in}{0.022222in}}{%
\pgfpathmoveto{\pgfqpoint{0.000000in}{-0.022222in}}%
\pgfpathcurveto{\pgfqpoint{0.005893in}{-0.022222in}}{\pgfqpoint{0.011546in}{-0.019881in}}{\pgfqpoint{0.015713in}{-0.015713in}}%
\pgfpathcurveto{\pgfqpoint{0.019881in}{-0.011546in}}{\pgfqpoint{0.022222in}{-0.005893in}}{\pgfqpoint{0.022222in}{0.000000in}}%
\pgfpathcurveto{\pgfqpoint{0.022222in}{0.005893in}}{\pgfqpoint{0.019881in}{0.011546in}}{\pgfqpoint{0.015713in}{0.015713in}}%
\pgfpathcurveto{\pgfqpoint{0.011546in}{0.019881in}}{\pgfqpoint{0.005893in}{0.022222in}}{\pgfqpoint{0.000000in}{0.022222in}}%
\pgfpathcurveto{\pgfqpoint{-0.005893in}{0.022222in}}{\pgfqpoint{-0.011546in}{0.019881in}}{\pgfqpoint{-0.015713in}{0.015713in}}%
\pgfpathcurveto{\pgfqpoint{-0.019881in}{0.011546in}}{\pgfqpoint{-0.022222in}{0.005893in}}{\pgfqpoint{-0.022222in}{0.000000in}}%
\pgfpathcurveto{\pgfqpoint{-0.022222in}{-0.005893in}}{\pgfqpoint{-0.019881in}{-0.011546in}}{\pgfqpoint{-0.015713in}{-0.015713in}}%
\pgfpathcurveto{\pgfqpoint{-0.011546in}{-0.019881in}}{\pgfqpoint{-0.005893in}{-0.022222in}}{\pgfqpoint{0.000000in}{-0.022222in}}%
\pgfpathclose%
\pgfusepath{stroke,fill}%
}%
\begin{pgfscope}%
\pgfsys@transformshift{0.543143in}{4.281573in}%
\pgfsys@useobject{currentmarker}{}%
\end{pgfscope}%
\end{pgfscope}%
\begin{pgfscope}%
\pgfpathrectangle{\pgfqpoint{0.100000in}{2.413063in}}{\pgfqpoint{5.037500in}{3.427208in}}%
\pgfusepath{clip}%
\pgfsetrectcap%
\pgfsetroundjoin%
\pgfsetlinewidth{1.505625pt}%
\definecolor{currentstroke}{rgb}{0.000000,0.000000,1.000000}%
\pgfsetstrokecolor{currentstroke}%
\pgfsetstrokeopacity{0.500000}%
\pgfsetdash{}{0pt}%
\pgfpathmoveto{\pgfqpoint{0.602428in}{3.993927in}}%
\pgfusepath{stroke}%
\end{pgfscope}%
\begin{pgfscope}%
\pgfpathrectangle{\pgfqpoint{0.100000in}{2.413063in}}{\pgfqpoint{5.037500in}{3.427208in}}%
\pgfusepath{clip}%
\pgfsetbuttcap%
\pgfsetroundjoin%
\definecolor{currentfill}{rgb}{0.000000,0.000000,1.000000}%
\pgfsetfillcolor{currentfill}%
\pgfsetfillopacity{0.500000}%
\pgfsetlinewidth{0.250937pt}%
\definecolor{currentstroke}{rgb}{0.000000,0.000000,0.000000}%
\pgfsetstrokecolor{currentstroke}%
\pgfsetstrokeopacity{0.500000}%
\pgfsetdash{}{0pt}%
\pgfsys@defobject{currentmarker}{\pgfqpoint{-0.047222in}{-0.047222in}}{\pgfqpoint{0.047222in}{0.047222in}}{%
\pgfpathmoveto{\pgfqpoint{0.000000in}{-0.047222in}}%
\pgfpathcurveto{\pgfqpoint{0.012523in}{-0.047222in}}{\pgfqpoint{0.024536in}{-0.042247in}}{\pgfqpoint{0.033391in}{-0.033391in}}%
\pgfpathcurveto{\pgfqpoint{0.042247in}{-0.024536in}}{\pgfqpoint{0.047222in}{-0.012523in}}{\pgfqpoint{0.047222in}{0.000000in}}%
\pgfpathcurveto{\pgfqpoint{0.047222in}{0.012523in}}{\pgfqpoint{0.042247in}{0.024536in}}{\pgfqpoint{0.033391in}{0.033391in}}%
\pgfpathcurveto{\pgfqpoint{0.024536in}{0.042247in}}{\pgfqpoint{0.012523in}{0.047222in}}{\pgfqpoint{0.000000in}{0.047222in}}%
\pgfpathcurveto{\pgfqpoint{-0.012523in}{0.047222in}}{\pgfqpoint{-0.024536in}{0.042247in}}{\pgfqpoint{-0.033391in}{0.033391in}}%
\pgfpathcurveto{\pgfqpoint{-0.042247in}{0.024536in}}{\pgfqpoint{-0.047222in}{0.012523in}}{\pgfqpoint{-0.047222in}{0.000000in}}%
\pgfpathcurveto{\pgfqpoint{-0.047222in}{-0.012523in}}{\pgfqpoint{-0.042247in}{-0.024536in}}{\pgfqpoint{-0.033391in}{-0.033391in}}%
\pgfpathcurveto{\pgfqpoint{-0.024536in}{-0.042247in}}{\pgfqpoint{-0.012523in}{-0.047222in}}{\pgfqpoint{0.000000in}{-0.047222in}}%
\pgfpathclose%
\pgfusepath{stroke,fill}%
}%
\begin{pgfscope}%
\pgfsys@transformshift{0.602428in}{3.993927in}%
\pgfsys@useobject{currentmarker}{}%
\end{pgfscope}%
\end{pgfscope}%
\begin{pgfscope}%
\pgfpathrectangle{\pgfqpoint{0.100000in}{2.413063in}}{\pgfqpoint{5.037500in}{3.427208in}}%
\pgfusepath{clip}%
\pgfsetrectcap%
\pgfsetroundjoin%
\pgfsetlinewidth{1.505625pt}%
\definecolor{currentstroke}{rgb}{0.678431,1.000000,0.184314}%
\pgfsetstrokecolor{currentstroke}%
\pgfsetstrokeopacity{0.500000}%
\pgfsetdash{}{0pt}%
\pgfpathmoveto{\pgfqpoint{0.557875in}{4.378461in}}%
\pgfusepath{stroke}%
\end{pgfscope}%
\begin{pgfscope}%
\pgfpathrectangle{\pgfqpoint{0.100000in}{2.413063in}}{\pgfqpoint{5.037500in}{3.427208in}}%
\pgfusepath{clip}%
\pgfsetbuttcap%
\pgfsetroundjoin%
\definecolor{currentfill}{rgb}{0.678431,1.000000,0.184314}%
\pgfsetfillcolor{currentfill}%
\pgfsetfillopacity{0.500000}%
\pgfsetlinewidth{0.250937pt}%
\definecolor{currentstroke}{rgb}{0.000000,0.000000,0.000000}%
\pgfsetstrokecolor{currentstroke}%
\pgfsetstrokeopacity{0.500000}%
\pgfsetdash{}{0pt}%
\pgfsys@defobject{currentmarker}{\pgfqpoint{-0.011111in}{-0.011111in}}{\pgfqpoint{0.011111in}{0.011111in}}{%
\pgfpathmoveto{\pgfqpoint{0.000000in}{-0.011111in}}%
\pgfpathcurveto{\pgfqpoint{0.002947in}{-0.011111in}}{\pgfqpoint{0.005773in}{-0.009940in}}{\pgfqpoint{0.007857in}{-0.007857in}}%
\pgfpathcurveto{\pgfqpoint{0.009940in}{-0.005773in}}{\pgfqpoint{0.011111in}{-0.002947in}}{\pgfqpoint{0.011111in}{0.000000in}}%
\pgfpathcurveto{\pgfqpoint{0.011111in}{0.002947in}}{\pgfqpoint{0.009940in}{0.005773in}}{\pgfqpoint{0.007857in}{0.007857in}}%
\pgfpathcurveto{\pgfqpoint{0.005773in}{0.009940in}}{\pgfqpoint{0.002947in}{0.011111in}}{\pgfqpoint{0.000000in}{0.011111in}}%
\pgfpathcurveto{\pgfqpoint{-0.002947in}{0.011111in}}{\pgfqpoint{-0.005773in}{0.009940in}}{\pgfqpoint{-0.007857in}{0.007857in}}%
\pgfpathcurveto{\pgfqpoint{-0.009940in}{0.005773in}}{\pgfqpoint{-0.011111in}{0.002947in}}{\pgfqpoint{-0.011111in}{0.000000in}}%
\pgfpathcurveto{\pgfqpoint{-0.011111in}{-0.002947in}}{\pgfqpoint{-0.009940in}{-0.005773in}}{\pgfqpoint{-0.007857in}{-0.007857in}}%
\pgfpathcurveto{\pgfqpoint{-0.005773in}{-0.009940in}}{\pgfqpoint{-0.002947in}{-0.011111in}}{\pgfqpoint{0.000000in}{-0.011111in}}%
\pgfpathclose%
\pgfusepath{stroke,fill}%
}%
\begin{pgfscope}%
\pgfsys@transformshift{0.557875in}{4.378461in}%
\pgfsys@useobject{currentmarker}{}%
\end{pgfscope}%
\end{pgfscope}%
\begin{pgfscope}%
\pgfpathrectangle{\pgfqpoint{0.100000in}{2.413063in}}{\pgfqpoint{5.037500in}{3.427208in}}%
\pgfusepath{clip}%
\pgfsetrectcap%
\pgfsetroundjoin%
\pgfsetlinewidth{1.505625pt}%
\definecolor{currentstroke}{rgb}{0.678431,1.000000,0.184314}%
\pgfsetstrokecolor{currentstroke}%
\pgfsetstrokeopacity{0.500000}%
\pgfsetdash{}{0pt}%
\pgfpathmoveto{\pgfqpoint{0.499312in}{4.410808in}}%
\pgfusepath{stroke}%
\end{pgfscope}%
\begin{pgfscope}%
\pgfpathrectangle{\pgfqpoint{0.100000in}{2.413063in}}{\pgfqpoint{5.037500in}{3.427208in}}%
\pgfusepath{clip}%
\pgfsetbuttcap%
\pgfsetroundjoin%
\definecolor{currentfill}{rgb}{0.678431,1.000000,0.184314}%
\pgfsetfillcolor{currentfill}%
\pgfsetfillopacity{0.500000}%
\pgfsetlinewidth{0.250937pt}%
\definecolor{currentstroke}{rgb}{0.000000,0.000000,0.000000}%
\pgfsetstrokecolor{currentstroke}%
\pgfsetstrokeopacity{0.500000}%
\pgfsetdash{}{0pt}%
\pgfsys@defobject{currentmarker}{\pgfqpoint{-0.008333in}{-0.008333in}}{\pgfqpoint{0.008333in}{0.008333in}}{%
\pgfpathmoveto{\pgfqpoint{0.000000in}{-0.008333in}}%
\pgfpathcurveto{\pgfqpoint{0.002210in}{-0.008333in}}{\pgfqpoint{0.004330in}{-0.007455in}}{\pgfqpoint{0.005893in}{-0.005893in}}%
\pgfpathcurveto{\pgfqpoint{0.007455in}{-0.004330in}}{\pgfqpoint{0.008333in}{-0.002210in}}{\pgfqpoint{0.008333in}{0.000000in}}%
\pgfpathcurveto{\pgfqpoint{0.008333in}{0.002210in}}{\pgfqpoint{0.007455in}{0.004330in}}{\pgfqpoint{0.005893in}{0.005893in}}%
\pgfpathcurveto{\pgfqpoint{0.004330in}{0.007455in}}{\pgfqpoint{0.002210in}{0.008333in}}{\pgfqpoint{0.000000in}{0.008333in}}%
\pgfpathcurveto{\pgfqpoint{-0.002210in}{0.008333in}}{\pgfqpoint{-0.004330in}{0.007455in}}{\pgfqpoint{-0.005893in}{0.005893in}}%
\pgfpathcurveto{\pgfqpoint{-0.007455in}{0.004330in}}{\pgfqpoint{-0.008333in}{0.002210in}}{\pgfqpoint{-0.008333in}{0.000000in}}%
\pgfpathcurveto{\pgfqpoint{-0.008333in}{-0.002210in}}{\pgfqpoint{-0.007455in}{-0.004330in}}{\pgfqpoint{-0.005893in}{-0.005893in}}%
\pgfpathcurveto{\pgfqpoint{-0.004330in}{-0.007455in}}{\pgfqpoint{-0.002210in}{-0.008333in}}{\pgfqpoint{0.000000in}{-0.008333in}}%
\pgfpathclose%
\pgfusepath{stroke,fill}%
}%
\begin{pgfscope}%
\pgfsys@transformshift{0.499312in}{4.410808in}%
\pgfsys@useobject{currentmarker}{}%
\end{pgfscope}%
\end{pgfscope}%
\begin{pgfscope}%
\pgfpathrectangle{\pgfqpoint{0.100000in}{2.413063in}}{\pgfqpoint{5.037500in}{3.427208in}}%
\pgfusepath{clip}%
\pgfsetrectcap%
\pgfsetroundjoin%
\pgfsetlinewidth{1.505625pt}%
\definecolor{currentstroke}{rgb}{0.501961,0.501961,0.501961}%
\pgfsetstrokecolor{currentstroke}%
\pgfsetstrokeopacity{0.500000}%
\pgfsetdash{}{0pt}%
\pgfpathmoveto{\pgfqpoint{0.465142in}{4.461082in}}%
\pgfusepath{stroke}%
\end{pgfscope}%
\begin{pgfscope}%
\pgfpathrectangle{\pgfqpoint{0.100000in}{2.413063in}}{\pgfqpoint{5.037500in}{3.427208in}}%
\pgfusepath{clip}%
\pgfsetbuttcap%
\pgfsetroundjoin%
\definecolor{currentfill}{rgb}{0.501961,0.501961,0.501961}%
\pgfsetfillcolor{currentfill}%
\pgfsetfillopacity{0.500000}%
\pgfsetlinewidth{0.250937pt}%
\definecolor{currentstroke}{rgb}{0.000000,0.000000,0.000000}%
\pgfsetstrokecolor{currentstroke}%
\pgfsetstrokeopacity{0.500000}%
\pgfsetdash{}{0pt}%
\pgfsys@defobject{currentmarker}{\pgfqpoint{-0.013889in}{-0.013889in}}{\pgfqpoint{0.013889in}{0.013889in}}{%
\pgfpathmoveto{\pgfqpoint{0.000000in}{-0.013889in}}%
\pgfpathcurveto{\pgfqpoint{0.003683in}{-0.013889in}}{\pgfqpoint{0.007216in}{-0.012425in}}{\pgfqpoint{0.009821in}{-0.009821in}}%
\pgfpathcurveto{\pgfqpoint{0.012425in}{-0.007216in}}{\pgfqpoint{0.013889in}{-0.003683in}}{\pgfqpoint{0.013889in}{0.000000in}}%
\pgfpathcurveto{\pgfqpoint{0.013889in}{0.003683in}}{\pgfqpoint{0.012425in}{0.007216in}}{\pgfqpoint{0.009821in}{0.009821in}}%
\pgfpathcurveto{\pgfqpoint{0.007216in}{0.012425in}}{\pgfqpoint{0.003683in}{0.013889in}}{\pgfqpoint{0.000000in}{0.013889in}}%
\pgfpathcurveto{\pgfqpoint{-0.003683in}{0.013889in}}{\pgfqpoint{-0.007216in}{0.012425in}}{\pgfqpoint{-0.009821in}{0.009821in}}%
\pgfpathcurveto{\pgfqpoint{-0.012425in}{0.007216in}}{\pgfqpoint{-0.013889in}{0.003683in}}{\pgfqpoint{-0.013889in}{0.000000in}}%
\pgfpathcurveto{\pgfqpoint{-0.013889in}{-0.003683in}}{\pgfqpoint{-0.012425in}{-0.007216in}}{\pgfqpoint{-0.009821in}{-0.009821in}}%
\pgfpathcurveto{\pgfqpoint{-0.007216in}{-0.012425in}}{\pgfqpoint{-0.003683in}{-0.013889in}}{\pgfqpoint{0.000000in}{-0.013889in}}%
\pgfpathclose%
\pgfusepath{stroke,fill}%
}%
\begin{pgfscope}%
\pgfsys@transformshift{0.465142in}{4.461082in}%
\pgfsys@useobject{currentmarker}{}%
\end{pgfscope}%
\end{pgfscope}%
\begin{pgfscope}%
\pgfpathrectangle{\pgfqpoint{0.100000in}{2.413063in}}{\pgfqpoint{5.037500in}{3.427208in}}%
\pgfusepath{clip}%
\pgfsetrectcap%
\pgfsetroundjoin%
\pgfsetlinewidth{1.505625pt}%
\definecolor{currentstroke}{rgb}{0.000000,0.000000,1.000000}%
\pgfsetstrokecolor{currentstroke}%
\pgfsetstrokeopacity{0.500000}%
\pgfsetdash{}{0pt}%
\pgfpathmoveto{\pgfqpoint{0.375033in}{4.567443in}}%
\pgfusepath{stroke}%
\end{pgfscope}%
\begin{pgfscope}%
\pgfpathrectangle{\pgfqpoint{0.100000in}{2.413063in}}{\pgfqpoint{5.037500in}{3.427208in}}%
\pgfusepath{clip}%
\pgfsetbuttcap%
\pgfsetroundjoin%
\definecolor{currentfill}{rgb}{0.000000,0.000000,1.000000}%
\pgfsetfillcolor{currentfill}%
\pgfsetfillopacity{0.500000}%
\pgfsetlinewidth{0.250937pt}%
\definecolor{currentstroke}{rgb}{0.000000,0.000000,0.000000}%
\pgfsetstrokecolor{currentstroke}%
\pgfsetstrokeopacity{0.500000}%
\pgfsetdash{}{0pt}%
\pgfsys@defobject{currentmarker}{\pgfqpoint{-0.027778in}{-0.027778in}}{\pgfqpoint{0.027778in}{0.027778in}}{%
\pgfpathmoveto{\pgfqpoint{0.000000in}{-0.027778in}}%
\pgfpathcurveto{\pgfqpoint{0.007367in}{-0.027778in}}{\pgfqpoint{0.014433in}{-0.024851in}}{\pgfqpoint{0.019642in}{-0.019642in}}%
\pgfpathcurveto{\pgfqpoint{0.024851in}{-0.014433in}}{\pgfqpoint{0.027778in}{-0.007367in}}{\pgfqpoint{0.027778in}{0.000000in}}%
\pgfpathcurveto{\pgfqpoint{0.027778in}{0.007367in}}{\pgfqpoint{0.024851in}{0.014433in}}{\pgfqpoint{0.019642in}{0.019642in}}%
\pgfpathcurveto{\pgfqpoint{0.014433in}{0.024851in}}{\pgfqpoint{0.007367in}{0.027778in}}{\pgfqpoint{0.000000in}{0.027778in}}%
\pgfpathcurveto{\pgfqpoint{-0.007367in}{0.027778in}}{\pgfqpoint{-0.014433in}{0.024851in}}{\pgfqpoint{-0.019642in}{0.019642in}}%
\pgfpathcurveto{\pgfqpoint{-0.024851in}{0.014433in}}{\pgfqpoint{-0.027778in}{0.007367in}}{\pgfqpoint{-0.027778in}{0.000000in}}%
\pgfpathcurveto{\pgfqpoint{-0.027778in}{-0.007367in}}{\pgfqpoint{-0.024851in}{-0.014433in}}{\pgfqpoint{-0.019642in}{-0.019642in}}%
\pgfpathcurveto{\pgfqpoint{-0.014433in}{-0.024851in}}{\pgfqpoint{-0.007367in}{-0.027778in}}{\pgfqpoint{0.000000in}{-0.027778in}}%
\pgfpathclose%
\pgfusepath{stroke,fill}%
}%
\begin{pgfscope}%
\pgfsys@transformshift{0.375033in}{4.567443in}%
\pgfsys@useobject{currentmarker}{}%
\end{pgfscope}%
\end{pgfscope}%
\begin{pgfscope}%
\pgfpathrectangle{\pgfqpoint{0.100000in}{2.413063in}}{\pgfqpoint{5.037500in}{3.427208in}}%
\pgfusepath{clip}%
\pgfsetrectcap%
\pgfsetroundjoin%
\pgfsetlinewidth{1.505625pt}%
\definecolor{currentstroke}{rgb}{0.000000,0.000000,1.000000}%
\pgfsetstrokecolor{currentstroke}%
\pgfsetstrokeopacity{0.500000}%
\pgfsetdash{}{0pt}%
\pgfpathmoveto{\pgfqpoint{0.520176in}{4.032959in}}%
\pgfusepath{stroke}%
\end{pgfscope}%
\begin{pgfscope}%
\pgfpathrectangle{\pgfqpoint{0.100000in}{2.413063in}}{\pgfqpoint{5.037500in}{3.427208in}}%
\pgfusepath{clip}%
\pgfsetbuttcap%
\pgfsetroundjoin%
\definecolor{currentfill}{rgb}{0.000000,0.000000,1.000000}%
\pgfsetfillcolor{currentfill}%
\pgfsetfillopacity{0.500000}%
\pgfsetlinewidth{0.250937pt}%
\definecolor{currentstroke}{rgb}{0.000000,0.000000,0.000000}%
\pgfsetstrokecolor{currentstroke}%
\pgfsetstrokeopacity{0.500000}%
\pgfsetdash{}{0pt}%
\pgfsys@defobject{currentmarker}{\pgfqpoint{-0.016667in}{-0.016667in}}{\pgfqpoint{0.016667in}{0.016667in}}{%
\pgfpathmoveto{\pgfqpoint{0.000000in}{-0.016667in}}%
\pgfpathcurveto{\pgfqpoint{0.004420in}{-0.016667in}}{\pgfqpoint{0.008660in}{-0.014911in}}{\pgfqpoint{0.011785in}{-0.011785in}}%
\pgfpathcurveto{\pgfqpoint{0.014911in}{-0.008660in}}{\pgfqpoint{0.016667in}{-0.004420in}}{\pgfqpoint{0.016667in}{0.000000in}}%
\pgfpathcurveto{\pgfqpoint{0.016667in}{0.004420in}}{\pgfqpoint{0.014911in}{0.008660in}}{\pgfqpoint{0.011785in}{0.011785in}}%
\pgfpathcurveto{\pgfqpoint{0.008660in}{0.014911in}}{\pgfqpoint{0.004420in}{0.016667in}}{\pgfqpoint{0.000000in}{0.016667in}}%
\pgfpathcurveto{\pgfqpoint{-0.004420in}{0.016667in}}{\pgfqpoint{-0.008660in}{0.014911in}}{\pgfqpoint{-0.011785in}{0.011785in}}%
\pgfpathcurveto{\pgfqpoint{-0.014911in}{0.008660in}}{\pgfqpoint{-0.016667in}{0.004420in}}{\pgfqpoint{-0.016667in}{0.000000in}}%
\pgfpathcurveto{\pgfqpoint{-0.016667in}{-0.004420in}}{\pgfqpoint{-0.014911in}{-0.008660in}}{\pgfqpoint{-0.011785in}{-0.011785in}}%
\pgfpathcurveto{\pgfqpoint{-0.008660in}{-0.014911in}}{\pgfqpoint{-0.004420in}{-0.016667in}}{\pgfqpoint{0.000000in}{-0.016667in}}%
\pgfpathclose%
\pgfusepath{stroke,fill}%
}%
\begin{pgfscope}%
\pgfsys@transformshift{0.520176in}{4.032959in}%
\pgfsys@useobject{currentmarker}{}%
\end{pgfscope}%
\end{pgfscope}%
\begin{pgfscope}%
\pgfpathrectangle{\pgfqpoint{0.100000in}{2.413063in}}{\pgfqpoint{5.037500in}{3.427208in}}%
\pgfusepath{clip}%
\pgfsetrectcap%
\pgfsetroundjoin%
\pgfsetlinewidth{1.505625pt}%
\definecolor{currentstroke}{rgb}{0.000000,0.000000,1.000000}%
\pgfsetstrokecolor{currentstroke}%
\pgfsetstrokeopacity{0.500000}%
\pgfsetdash{}{0pt}%
\pgfpathmoveto{\pgfqpoint{0.444132in}{4.822021in}}%
\pgfusepath{stroke}%
\end{pgfscope}%
\begin{pgfscope}%
\pgfpathrectangle{\pgfqpoint{0.100000in}{2.413063in}}{\pgfqpoint{5.037500in}{3.427208in}}%
\pgfusepath{clip}%
\pgfsetbuttcap%
\pgfsetroundjoin%
\definecolor{currentfill}{rgb}{0.000000,0.000000,1.000000}%
\pgfsetfillcolor{currentfill}%
\pgfsetfillopacity{0.500000}%
\pgfsetlinewidth{0.250937pt}%
\definecolor{currentstroke}{rgb}{0.000000,0.000000,0.000000}%
\pgfsetstrokecolor{currentstroke}%
\pgfsetstrokeopacity{0.500000}%
\pgfsetdash{}{0pt}%
\pgfsys@defobject{currentmarker}{\pgfqpoint{-0.008333in}{-0.008333in}}{\pgfqpoint{0.008333in}{0.008333in}}{%
\pgfpathmoveto{\pgfqpoint{0.000000in}{-0.008333in}}%
\pgfpathcurveto{\pgfqpoint{0.002210in}{-0.008333in}}{\pgfqpoint{0.004330in}{-0.007455in}}{\pgfqpoint{0.005893in}{-0.005893in}}%
\pgfpathcurveto{\pgfqpoint{0.007455in}{-0.004330in}}{\pgfqpoint{0.008333in}{-0.002210in}}{\pgfqpoint{0.008333in}{0.000000in}}%
\pgfpathcurveto{\pgfqpoint{0.008333in}{0.002210in}}{\pgfqpoint{0.007455in}{0.004330in}}{\pgfqpoint{0.005893in}{0.005893in}}%
\pgfpathcurveto{\pgfqpoint{0.004330in}{0.007455in}}{\pgfqpoint{0.002210in}{0.008333in}}{\pgfqpoint{0.000000in}{0.008333in}}%
\pgfpathcurveto{\pgfqpoint{-0.002210in}{0.008333in}}{\pgfqpoint{-0.004330in}{0.007455in}}{\pgfqpoint{-0.005893in}{0.005893in}}%
\pgfpathcurveto{\pgfqpoint{-0.007455in}{0.004330in}}{\pgfqpoint{-0.008333in}{0.002210in}}{\pgfqpoint{-0.008333in}{0.000000in}}%
\pgfpathcurveto{\pgfqpoint{-0.008333in}{-0.002210in}}{\pgfqpoint{-0.007455in}{-0.004330in}}{\pgfqpoint{-0.005893in}{-0.005893in}}%
\pgfpathcurveto{\pgfqpoint{-0.004330in}{-0.007455in}}{\pgfqpoint{-0.002210in}{-0.008333in}}{\pgfqpoint{0.000000in}{-0.008333in}}%
\pgfpathclose%
\pgfusepath{stroke,fill}%
}%
\begin{pgfscope}%
\pgfsys@transformshift{0.444132in}{4.822021in}%
\pgfsys@useobject{currentmarker}{}%
\end{pgfscope}%
\end{pgfscope}%
\begin{pgfscope}%
\pgfpathrectangle{\pgfqpoint{0.100000in}{2.413063in}}{\pgfqpoint{5.037500in}{3.427208in}}%
\pgfusepath{clip}%
\pgfsetrectcap%
\pgfsetroundjoin%
\pgfsetlinewidth{1.505625pt}%
\definecolor{currentstroke}{rgb}{0.000000,0.000000,1.000000}%
\pgfsetstrokecolor{currentstroke}%
\pgfsetstrokeopacity{0.500000}%
\pgfsetdash{}{0pt}%
\pgfpathmoveto{\pgfqpoint{0.678250in}{3.962355in}}%
\pgfusepath{stroke}%
\end{pgfscope}%
\begin{pgfscope}%
\pgfpathrectangle{\pgfqpoint{0.100000in}{2.413063in}}{\pgfqpoint{5.037500in}{3.427208in}}%
\pgfusepath{clip}%
\pgfsetbuttcap%
\pgfsetroundjoin%
\definecolor{currentfill}{rgb}{0.000000,0.000000,1.000000}%
\pgfsetfillcolor{currentfill}%
\pgfsetfillopacity{0.500000}%
\pgfsetlinewidth{0.250937pt}%
\definecolor{currentstroke}{rgb}{0.000000,0.000000,0.000000}%
\pgfsetstrokecolor{currentstroke}%
\pgfsetstrokeopacity{0.500000}%
\pgfsetdash{}{0pt}%
\pgfsys@defobject{currentmarker}{\pgfqpoint{-0.038889in}{-0.038889in}}{\pgfqpoint{0.038889in}{0.038889in}}{%
\pgfpathmoveto{\pgfqpoint{0.000000in}{-0.038889in}}%
\pgfpathcurveto{\pgfqpoint{0.010313in}{-0.038889in}}{\pgfqpoint{0.020206in}{-0.034791in}}{\pgfqpoint{0.027499in}{-0.027499in}}%
\pgfpathcurveto{\pgfqpoint{0.034791in}{-0.020206in}}{\pgfqpoint{0.038889in}{-0.010313in}}{\pgfqpoint{0.038889in}{0.000000in}}%
\pgfpathcurveto{\pgfqpoint{0.038889in}{0.010313in}}{\pgfqpoint{0.034791in}{0.020206in}}{\pgfqpoint{0.027499in}{0.027499in}}%
\pgfpathcurveto{\pgfqpoint{0.020206in}{0.034791in}}{\pgfqpoint{0.010313in}{0.038889in}}{\pgfqpoint{0.000000in}{0.038889in}}%
\pgfpathcurveto{\pgfqpoint{-0.010313in}{0.038889in}}{\pgfqpoint{-0.020206in}{0.034791in}}{\pgfqpoint{-0.027499in}{0.027499in}}%
\pgfpathcurveto{\pgfqpoint{-0.034791in}{0.020206in}}{\pgfqpoint{-0.038889in}{0.010313in}}{\pgfqpoint{-0.038889in}{0.000000in}}%
\pgfpathcurveto{\pgfqpoint{-0.038889in}{-0.010313in}}{\pgfqpoint{-0.034791in}{-0.020206in}}{\pgfqpoint{-0.027499in}{-0.027499in}}%
\pgfpathcurveto{\pgfqpoint{-0.020206in}{-0.034791in}}{\pgfqpoint{-0.010313in}{-0.038889in}}{\pgfqpoint{0.000000in}{-0.038889in}}%
\pgfpathclose%
\pgfusepath{stroke,fill}%
}%
\begin{pgfscope}%
\pgfsys@transformshift{0.678250in}{3.962355in}%
\pgfsys@useobject{currentmarker}{}%
\end{pgfscope}%
\end{pgfscope}%
\begin{pgfscope}%
\pgfpathrectangle{\pgfqpoint{0.100000in}{2.413063in}}{\pgfqpoint{5.037500in}{3.427208in}}%
\pgfusepath{clip}%
\pgfsetrectcap%
\pgfsetroundjoin%
\pgfsetlinewidth{1.505625pt}%
\definecolor{currentstroke}{rgb}{0.000000,0.000000,1.000000}%
\pgfsetstrokecolor{currentstroke}%
\pgfsetstrokeopacity{0.500000}%
\pgfsetdash{}{0pt}%
\pgfpathmoveto{\pgfqpoint{0.452839in}{4.577966in}}%
\pgfusepath{stroke}%
\end{pgfscope}%
\begin{pgfscope}%
\pgfpathrectangle{\pgfqpoint{0.100000in}{2.413063in}}{\pgfqpoint{5.037500in}{3.427208in}}%
\pgfusepath{clip}%
\pgfsetbuttcap%
\pgfsetroundjoin%
\definecolor{currentfill}{rgb}{0.000000,0.000000,1.000000}%
\pgfsetfillcolor{currentfill}%
\pgfsetfillopacity{0.500000}%
\pgfsetlinewidth{0.250937pt}%
\definecolor{currentstroke}{rgb}{0.000000,0.000000,0.000000}%
\pgfsetstrokecolor{currentstroke}%
\pgfsetstrokeopacity{0.500000}%
\pgfsetdash{}{0pt}%
\pgfsys@defobject{currentmarker}{\pgfqpoint{-0.027778in}{-0.027778in}}{\pgfqpoint{0.027778in}{0.027778in}}{%
\pgfpathmoveto{\pgfqpoint{0.000000in}{-0.027778in}}%
\pgfpathcurveto{\pgfqpoint{0.007367in}{-0.027778in}}{\pgfqpoint{0.014433in}{-0.024851in}}{\pgfqpoint{0.019642in}{-0.019642in}}%
\pgfpathcurveto{\pgfqpoint{0.024851in}{-0.014433in}}{\pgfqpoint{0.027778in}{-0.007367in}}{\pgfqpoint{0.027778in}{0.000000in}}%
\pgfpathcurveto{\pgfqpoint{0.027778in}{0.007367in}}{\pgfqpoint{0.024851in}{0.014433in}}{\pgfqpoint{0.019642in}{0.019642in}}%
\pgfpathcurveto{\pgfqpoint{0.014433in}{0.024851in}}{\pgfqpoint{0.007367in}{0.027778in}}{\pgfqpoint{0.000000in}{0.027778in}}%
\pgfpathcurveto{\pgfqpoint{-0.007367in}{0.027778in}}{\pgfqpoint{-0.014433in}{0.024851in}}{\pgfqpoint{-0.019642in}{0.019642in}}%
\pgfpathcurveto{\pgfqpoint{-0.024851in}{0.014433in}}{\pgfqpoint{-0.027778in}{0.007367in}}{\pgfqpoint{-0.027778in}{0.000000in}}%
\pgfpathcurveto{\pgfqpoint{-0.027778in}{-0.007367in}}{\pgfqpoint{-0.024851in}{-0.014433in}}{\pgfqpoint{-0.019642in}{-0.019642in}}%
\pgfpathcurveto{\pgfqpoint{-0.014433in}{-0.024851in}}{\pgfqpoint{-0.007367in}{-0.027778in}}{\pgfqpoint{0.000000in}{-0.027778in}}%
\pgfpathclose%
\pgfusepath{stroke,fill}%
}%
\begin{pgfscope}%
\pgfsys@transformshift{0.452839in}{4.577966in}%
\pgfsys@useobject{currentmarker}{}%
\end{pgfscope}%
\end{pgfscope}%
\begin{pgfscope}%
\pgfpathrectangle{\pgfqpoint{0.100000in}{2.413063in}}{\pgfqpoint{5.037500in}{3.427208in}}%
\pgfusepath{clip}%
\pgfsetrectcap%
\pgfsetroundjoin%
\pgfsetlinewidth{1.505625pt}%
\definecolor{currentstroke}{rgb}{0.678431,1.000000,0.184314}%
\pgfsetstrokecolor{currentstroke}%
\pgfsetstrokeopacity{0.500000}%
\pgfsetdash{}{0pt}%
\pgfpathmoveto{\pgfqpoint{0.375477in}{4.371832in}}%
\pgfusepath{stroke}%
\end{pgfscope}%
\begin{pgfscope}%
\pgfpathrectangle{\pgfqpoint{0.100000in}{2.413063in}}{\pgfqpoint{5.037500in}{3.427208in}}%
\pgfusepath{clip}%
\pgfsetbuttcap%
\pgfsetroundjoin%
\definecolor{currentfill}{rgb}{0.678431,1.000000,0.184314}%
\pgfsetfillcolor{currentfill}%
\pgfsetfillopacity{0.500000}%
\pgfsetlinewidth{0.250937pt}%
\definecolor{currentstroke}{rgb}{0.000000,0.000000,0.000000}%
\pgfsetstrokecolor{currentstroke}%
\pgfsetstrokeopacity{0.500000}%
\pgfsetdash{}{0pt}%
\pgfsys@defobject{currentmarker}{\pgfqpoint{-0.019444in}{-0.019444in}}{\pgfqpoint{0.019444in}{0.019444in}}{%
\pgfpathmoveto{\pgfqpoint{0.000000in}{-0.019444in}}%
\pgfpathcurveto{\pgfqpoint{0.005157in}{-0.019444in}}{\pgfqpoint{0.010103in}{-0.017396in}}{\pgfqpoint{0.013749in}{-0.013749in}}%
\pgfpathcurveto{\pgfqpoint{0.017396in}{-0.010103in}}{\pgfqpoint{0.019444in}{-0.005157in}}{\pgfqpoint{0.019444in}{0.000000in}}%
\pgfpathcurveto{\pgfqpoint{0.019444in}{0.005157in}}{\pgfqpoint{0.017396in}{0.010103in}}{\pgfqpoint{0.013749in}{0.013749in}}%
\pgfpathcurveto{\pgfqpoint{0.010103in}{0.017396in}}{\pgfqpoint{0.005157in}{0.019444in}}{\pgfqpoint{0.000000in}{0.019444in}}%
\pgfpathcurveto{\pgfqpoint{-0.005157in}{0.019444in}}{\pgfqpoint{-0.010103in}{0.017396in}}{\pgfqpoint{-0.013749in}{0.013749in}}%
\pgfpathcurveto{\pgfqpoint{-0.017396in}{0.010103in}}{\pgfqpoint{-0.019444in}{0.005157in}}{\pgfqpoint{-0.019444in}{0.000000in}}%
\pgfpathcurveto{\pgfqpoint{-0.019444in}{-0.005157in}}{\pgfqpoint{-0.017396in}{-0.010103in}}{\pgfqpoint{-0.013749in}{-0.013749in}}%
\pgfpathcurveto{\pgfqpoint{-0.010103in}{-0.017396in}}{\pgfqpoint{-0.005157in}{-0.019444in}}{\pgfqpoint{0.000000in}{-0.019444in}}%
\pgfpathclose%
\pgfusepath{stroke,fill}%
}%
\begin{pgfscope}%
\pgfsys@transformshift{0.375477in}{4.371832in}%
\pgfsys@useobject{currentmarker}{}%
\end{pgfscope}%
\end{pgfscope}%
\begin{pgfscope}%
\pgfpathrectangle{\pgfqpoint{0.100000in}{2.413063in}}{\pgfqpoint{5.037500in}{3.427208in}}%
\pgfusepath{clip}%
\pgfsetrectcap%
\pgfsetroundjoin%
\pgfsetlinewidth{1.505625pt}%
\definecolor{currentstroke}{rgb}{0.000000,0.000000,1.000000}%
\pgfsetstrokecolor{currentstroke}%
\pgfsetstrokeopacity{0.500000}%
\pgfsetdash{}{0pt}%
\pgfpathmoveto{\pgfqpoint{0.665426in}{3.818137in}}%
\pgfusepath{stroke}%
\end{pgfscope}%
\begin{pgfscope}%
\pgfpathrectangle{\pgfqpoint{0.100000in}{2.413063in}}{\pgfqpoint{5.037500in}{3.427208in}}%
\pgfusepath{clip}%
\pgfsetbuttcap%
\pgfsetroundjoin%
\definecolor{currentfill}{rgb}{0.000000,0.000000,1.000000}%
\pgfsetfillcolor{currentfill}%
\pgfsetfillopacity{0.500000}%
\pgfsetlinewidth{0.250937pt}%
\definecolor{currentstroke}{rgb}{0.000000,0.000000,0.000000}%
\pgfsetstrokecolor{currentstroke}%
\pgfsetstrokeopacity{0.500000}%
\pgfsetdash{}{0pt}%
\pgfsys@defobject{currentmarker}{\pgfqpoint{-0.033333in}{-0.033333in}}{\pgfqpoint{0.033333in}{0.033333in}}{%
\pgfpathmoveto{\pgfqpoint{0.000000in}{-0.033333in}}%
\pgfpathcurveto{\pgfqpoint{0.008840in}{-0.033333in}}{\pgfqpoint{0.017319in}{-0.029821in}}{\pgfqpoint{0.023570in}{-0.023570in}}%
\pgfpathcurveto{\pgfqpoint{0.029821in}{-0.017319in}}{\pgfqpoint{0.033333in}{-0.008840in}}{\pgfqpoint{0.033333in}{0.000000in}}%
\pgfpathcurveto{\pgfqpoint{0.033333in}{0.008840in}}{\pgfqpoint{0.029821in}{0.017319in}}{\pgfqpoint{0.023570in}{0.023570in}}%
\pgfpathcurveto{\pgfqpoint{0.017319in}{0.029821in}}{\pgfqpoint{0.008840in}{0.033333in}}{\pgfqpoint{0.000000in}{0.033333in}}%
\pgfpathcurveto{\pgfqpoint{-0.008840in}{0.033333in}}{\pgfqpoint{-0.017319in}{0.029821in}}{\pgfqpoint{-0.023570in}{0.023570in}}%
\pgfpathcurveto{\pgfqpoint{-0.029821in}{0.017319in}}{\pgfqpoint{-0.033333in}{0.008840in}}{\pgfqpoint{-0.033333in}{0.000000in}}%
\pgfpathcurveto{\pgfqpoint{-0.033333in}{-0.008840in}}{\pgfqpoint{-0.029821in}{-0.017319in}}{\pgfqpoint{-0.023570in}{-0.023570in}}%
\pgfpathcurveto{\pgfqpoint{-0.017319in}{-0.029821in}}{\pgfqpoint{-0.008840in}{-0.033333in}}{\pgfqpoint{0.000000in}{-0.033333in}}%
\pgfpathclose%
\pgfusepath{stroke,fill}%
}%
\begin{pgfscope}%
\pgfsys@transformshift{0.665426in}{3.818137in}%
\pgfsys@useobject{currentmarker}{}%
\end{pgfscope}%
\end{pgfscope}%
\begin{pgfscope}%
\pgfpathrectangle{\pgfqpoint{0.100000in}{2.413063in}}{\pgfqpoint{5.037500in}{3.427208in}}%
\pgfusepath{clip}%
\pgfsetrectcap%
\pgfsetroundjoin%
\pgfsetlinewidth{1.505625pt}%
\definecolor{currentstroke}{rgb}{0.000000,0.000000,1.000000}%
\pgfsetstrokecolor{currentstroke}%
\pgfsetstrokeopacity{0.500000}%
\pgfsetdash{}{0pt}%
\pgfpathmoveto{\pgfqpoint{0.345769in}{4.514080in}}%
\pgfusepath{stroke}%
\end{pgfscope}%
\begin{pgfscope}%
\pgfpathrectangle{\pgfqpoint{0.100000in}{2.413063in}}{\pgfqpoint{5.037500in}{3.427208in}}%
\pgfusepath{clip}%
\pgfsetbuttcap%
\pgfsetroundjoin%
\definecolor{currentfill}{rgb}{0.000000,0.000000,1.000000}%
\pgfsetfillcolor{currentfill}%
\pgfsetfillopacity{0.500000}%
\pgfsetlinewidth{0.250937pt}%
\definecolor{currentstroke}{rgb}{0.000000,0.000000,0.000000}%
\pgfsetstrokecolor{currentstroke}%
\pgfsetstrokeopacity{0.500000}%
\pgfsetdash{}{0pt}%
\pgfsys@defobject{currentmarker}{\pgfqpoint{-0.030556in}{-0.030556in}}{\pgfqpoint{0.030556in}{0.030556in}}{%
\pgfpathmoveto{\pgfqpoint{0.000000in}{-0.030556in}}%
\pgfpathcurveto{\pgfqpoint{0.008103in}{-0.030556in}}{\pgfqpoint{0.015876in}{-0.027336in}}{\pgfqpoint{0.021606in}{-0.021606in}}%
\pgfpathcurveto{\pgfqpoint{0.027336in}{-0.015876in}}{\pgfqpoint{0.030556in}{-0.008103in}}{\pgfqpoint{0.030556in}{0.000000in}}%
\pgfpathcurveto{\pgfqpoint{0.030556in}{0.008103in}}{\pgfqpoint{0.027336in}{0.015876in}}{\pgfqpoint{0.021606in}{0.021606in}}%
\pgfpathcurveto{\pgfqpoint{0.015876in}{0.027336in}}{\pgfqpoint{0.008103in}{0.030556in}}{\pgfqpoint{0.000000in}{0.030556in}}%
\pgfpathcurveto{\pgfqpoint{-0.008103in}{0.030556in}}{\pgfqpoint{-0.015876in}{0.027336in}}{\pgfqpoint{-0.021606in}{0.021606in}}%
\pgfpathcurveto{\pgfqpoint{-0.027336in}{0.015876in}}{\pgfqpoint{-0.030556in}{0.008103in}}{\pgfqpoint{-0.030556in}{0.000000in}}%
\pgfpathcurveto{\pgfqpoint{-0.030556in}{-0.008103in}}{\pgfqpoint{-0.027336in}{-0.015876in}}{\pgfqpoint{-0.021606in}{-0.021606in}}%
\pgfpathcurveto{\pgfqpoint{-0.015876in}{-0.027336in}}{\pgfqpoint{-0.008103in}{-0.030556in}}{\pgfqpoint{0.000000in}{-0.030556in}}%
\pgfpathclose%
\pgfusepath{stroke,fill}%
}%
\begin{pgfscope}%
\pgfsys@transformshift{0.345769in}{4.514080in}%
\pgfsys@useobject{currentmarker}{}%
\end{pgfscope}%
\end{pgfscope}%
\begin{pgfscope}%
\pgfpathrectangle{\pgfqpoint{0.100000in}{2.413063in}}{\pgfqpoint{5.037500in}{3.427208in}}%
\pgfusepath{clip}%
\pgfsetrectcap%
\pgfsetroundjoin%
\pgfsetlinewidth{1.505625pt}%
\definecolor{currentstroke}{rgb}{0.000000,0.000000,1.000000}%
\pgfsetstrokecolor{currentstroke}%
\pgfsetstrokeopacity{0.500000}%
\pgfsetdash{}{0pt}%
\pgfpathmoveto{\pgfqpoint{0.376878in}{4.451092in}}%
\pgfusepath{stroke}%
\end{pgfscope}%
\begin{pgfscope}%
\pgfpathrectangle{\pgfqpoint{0.100000in}{2.413063in}}{\pgfqpoint{5.037500in}{3.427208in}}%
\pgfusepath{clip}%
\pgfsetbuttcap%
\pgfsetroundjoin%
\definecolor{currentfill}{rgb}{0.000000,0.000000,1.000000}%
\pgfsetfillcolor{currentfill}%
\pgfsetfillopacity{0.500000}%
\pgfsetlinewidth{0.250937pt}%
\definecolor{currentstroke}{rgb}{0.000000,0.000000,0.000000}%
\pgfsetstrokecolor{currentstroke}%
\pgfsetstrokeopacity{0.500000}%
\pgfsetdash{}{0pt}%
\pgfsys@defobject{currentmarker}{\pgfqpoint{-0.016667in}{-0.016667in}}{\pgfqpoint{0.016667in}{0.016667in}}{%
\pgfpathmoveto{\pgfqpoint{0.000000in}{-0.016667in}}%
\pgfpathcurveto{\pgfqpoint{0.004420in}{-0.016667in}}{\pgfqpoint{0.008660in}{-0.014911in}}{\pgfqpoint{0.011785in}{-0.011785in}}%
\pgfpathcurveto{\pgfqpoint{0.014911in}{-0.008660in}}{\pgfqpoint{0.016667in}{-0.004420in}}{\pgfqpoint{0.016667in}{0.000000in}}%
\pgfpathcurveto{\pgfqpoint{0.016667in}{0.004420in}}{\pgfqpoint{0.014911in}{0.008660in}}{\pgfqpoint{0.011785in}{0.011785in}}%
\pgfpathcurveto{\pgfqpoint{0.008660in}{0.014911in}}{\pgfqpoint{0.004420in}{0.016667in}}{\pgfqpoint{0.000000in}{0.016667in}}%
\pgfpathcurveto{\pgfqpoint{-0.004420in}{0.016667in}}{\pgfqpoint{-0.008660in}{0.014911in}}{\pgfqpoint{-0.011785in}{0.011785in}}%
\pgfpathcurveto{\pgfqpoint{-0.014911in}{0.008660in}}{\pgfqpoint{-0.016667in}{0.004420in}}{\pgfqpoint{-0.016667in}{0.000000in}}%
\pgfpathcurveto{\pgfqpoint{-0.016667in}{-0.004420in}}{\pgfqpoint{-0.014911in}{-0.008660in}}{\pgfqpoint{-0.011785in}{-0.011785in}}%
\pgfpathcurveto{\pgfqpoint{-0.008660in}{-0.014911in}}{\pgfqpoint{-0.004420in}{-0.016667in}}{\pgfqpoint{0.000000in}{-0.016667in}}%
\pgfpathclose%
\pgfusepath{stroke,fill}%
}%
\begin{pgfscope}%
\pgfsys@transformshift{0.376878in}{4.451092in}%
\pgfsys@useobject{currentmarker}{}%
\end{pgfscope}%
\end{pgfscope}%
\begin{pgfscope}%
\pgfpathrectangle{\pgfqpoint{0.100000in}{2.413063in}}{\pgfqpoint{5.037500in}{3.427208in}}%
\pgfusepath{clip}%
\pgfsetrectcap%
\pgfsetroundjoin%
\pgfsetlinewidth{1.505625pt}%
\definecolor{currentstroke}{rgb}{0.000000,0.000000,1.000000}%
\pgfsetstrokecolor{currentstroke}%
\pgfsetstrokeopacity{0.500000}%
\pgfsetdash{}{0pt}%
\pgfpathmoveto{\pgfqpoint{0.418989in}{4.191586in}}%
\pgfusepath{stroke}%
\end{pgfscope}%
\begin{pgfscope}%
\pgfpathrectangle{\pgfqpoint{0.100000in}{2.413063in}}{\pgfqpoint{5.037500in}{3.427208in}}%
\pgfusepath{clip}%
\pgfsetbuttcap%
\pgfsetroundjoin%
\definecolor{currentfill}{rgb}{0.000000,0.000000,1.000000}%
\pgfsetfillcolor{currentfill}%
\pgfsetfillopacity{0.500000}%
\pgfsetlinewidth{0.250937pt}%
\definecolor{currentstroke}{rgb}{0.000000,0.000000,0.000000}%
\pgfsetstrokecolor{currentstroke}%
\pgfsetstrokeopacity{0.500000}%
\pgfsetdash{}{0pt}%
\pgfsys@defobject{currentmarker}{\pgfqpoint{-0.019444in}{-0.019444in}}{\pgfqpoint{0.019444in}{0.019444in}}{%
\pgfpathmoveto{\pgfqpoint{0.000000in}{-0.019444in}}%
\pgfpathcurveto{\pgfqpoint{0.005157in}{-0.019444in}}{\pgfqpoint{0.010103in}{-0.017396in}}{\pgfqpoint{0.013749in}{-0.013749in}}%
\pgfpathcurveto{\pgfqpoint{0.017396in}{-0.010103in}}{\pgfqpoint{0.019444in}{-0.005157in}}{\pgfqpoint{0.019444in}{0.000000in}}%
\pgfpathcurveto{\pgfqpoint{0.019444in}{0.005157in}}{\pgfqpoint{0.017396in}{0.010103in}}{\pgfqpoint{0.013749in}{0.013749in}}%
\pgfpathcurveto{\pgfqpoint{0.010103in}{0.017396in}}{\pgfqpoint{0.005157in}{0.019444in}}{\pgfqpoint{0.000000in}{0.019444in}}%
\pgfpathcurveto{\pgfqpoint{-0.005157in}{0.019444in}}{\pgfqpoint{-0.010103in}{0.017396in}}{\pgfqpoint{-0.013749in}{0.013749in}}%
\pgfpathcurveto{\pgfqpoint{-0.017396in}{0.010103in}}{\pgfqpoint{-0.019444in}{0.005157in}}{\pgfqpoint{-0.019444in}{0.000000in}}%
\pgfpathcurveto{\pgfqpoint{-0.019444in}{-0.005157in}}{\pgfqpoint{-0.017396in}{-0.010103in}}{\pgfqpoint{-0.013749in}{-0.013749in}}%
\pgfpathcurveto{\pgfqpoint{-0.010103in}{-0.017396in}}{\pgfqpoint{-0.005157in}{-0.019444in}}{\pgfqpoint{0.000000in}{-0.019444in}}%
\pgfpathclose%
\pgfusepath{stroke,fill}%
}%
\begin{pgfscope}%
\pgfsys@transformshift{0.418989in}{4.191586in}%
\pgfsys@useobject{currentmarker}{}%
\end{pgfscope}%
\end{pgfscope}%
\begin{pgfscope}%
\pgfpathrectangle{\pgfqpoint{0.100000in}{2.413063in}}{\pgfqpoint{5.037500in}{3.427208in}}%
\pgfusepath{clip}%
\pgfsetrectcap%
\pgfsetroundjoin%
\pgfsetlinewidth{1.505625pt}%
\definecolor{currentstroke}{rgb}{0.000000,0.000000,1.000000}%
\pgfsetstrokecolor{currentstroke}%
\pgfsetstrokeopacity{0.500000}%
\pgfsetdash{}{0pt}%
\pgfpathmoveto{\pgfqpoint{0.352770in}{4.414777in}}%
\pgfusepath{stroke}%
\end{pgfscope}%
\begin{pgfscope}%
\pgfpathrectangle{\pgfqpoint{0.100000in}{2.413063in}}{\pgfqpoint{5.037500in}{3.427208in}}%
\pgfusepath{clip}%
\pgfsetbuttcap%
\pgfsetroundjoin%
\definecolor{currentfill}{rgb}{0.000000,0.000000,1.000000}%
\pgfsetfillcolor{currentfill}%
\pgfsetfillopacity{0.500000}%
\pgfsetlinewidth{0.250937pt}%
\definecolor{currentstroke}{rgb}{0.000000,0.000000,0.000000}%
\pgfsetstrokecolor{currentstroke}%
\pgfsetstrokeopacity{0.500000}%
\pgfsetdash{}{0pt}%
\pgfsys@defobject{currentmarker}{\pgfqpoint{-0.005556in}{-0.005556in}}{\pgfqpoint{0.005556in}{0.005556in}}{%
\pgfpathmoveto{\pgfqpoint{0.000000in}{-0.005556in}}%
\pgfpathcurveto{\pgfqpoint{0.001473in}{-0.005556in}}{\pgfqpoint{0.002887in}{-0.004970in}}{\pgfqpoint{0.003928in}{-0.003928in}}%
\pgfpathcurveto{\pgfqpoint{0.004970in}{-0.002887in}}{\pgfqpoint{0.005556in}{-0.001473in}}{\pgfqpoint{0.005556in}{0.000000in}}%
\pgfpathcurveto{\pgfqpoint{0.005556in}{0.001473in}}{\pgfqpoint{0.004970in}{0.002887in}}{\pgfqpoint{0.003928in}{0.003928in}}%
\pgfpathcurveto{\pgfqpoint{0.002887in}{0.004970in}}{\pgfqpoint{0.001473in}{0.005556in}}{\pgfqpoint{0.000000in}{0.005556in}}%
\pgfpathcurveto{\pgfqpoint{-0.001473in}{0.005556in}}{\pgfqpoint{-0.002887in}{0.004970in}}{\pgfqpoint{-0.003928in}{0.003928in}}%
\pgfpathcurveto{\pgfqpoint{-0.004970in}{0.002887in}}{\pgfqpoint{-0.005556in}{0.001473in}}{\pgfqpoint{-0.005556in}{0.000000in}}%
\pgfpathcurveto{\pgfqpoint{-0.005556in}{-0.001473in}}{\pgfqpoint{-0.004970in}{-0.002887in}}{\pgfqpoint{-0.003928in}{-0.003928in}}%
\pgfpathcurveto{\pgfqpoint{-0.002887in}{-0.004970in}}{\pgfqpoint{-0.001473in}{-0.005556in}}{\pgfqpoint{0.000000in}{-0.005556in}}%
\pgfpathclose%
\pgfusepath{stroke,fill}%
}%
\begin{pgfscope}%
\pgfsys@transformshift{0.352770in}{4.414777in}%
\pgfsys@useobject{currentmarker}{}%
\end{pgfscope}%
\end{pgfscope}%
\begin{pgfscope}%
\pgfpathrectangle{\pgfqpoint{0.100000in}{2.413063in}}{\pgfqpoint{5.037500in}{3.427208in}}%
\pgfusepath{clip}%
\pgfsetrectcap%
\pgfsetroundjoin%
\pgfsetlinewidth{1.505625pt}%
\definecolor{currentstroke}{rgb}{0.000000,0.000000,1.000000}%
\pgfsetstrokecolor{currentstroke}%
\pgfsetstrokeopacity{0.500000}%
\pgfsetdash{}{0pt}%
\pgfpathmoveto{\pgfqpoint{0.428730in}{4.149174in}}%
\pgfusepath{stroke}%
\end{pgfscope}%
\begin{pgfscope}%
\pgfpathrectangle{\pgfqpoint{0.100000in}{2.413063in}}{\pgfqpoint{5.037500in}{3.427208in}}%
\pgfusepath{clip}%
\pgfsetbuttcap%
\pgfsetroundjoin%
\definecolor{currentfill}{rgb}{0.000000,0.000000,1.000000}%
\pgfsetfillcolor{currentfill}%
\pgfsetfillopacity{0.500000}%
\pgfsetlinewidth{0.250937pt}%
\definecolor{currentstroke}{rgb}{0.000000,0.000000,0.000000}%
\pgfsetstrokecolor{currentstroke}%
\pgfsetstrokeopacity{0.500000}%
\pgfsetdash{}{0pt}%
\pgfsys@defobject{currentmarker}{\pgfqpoint{-0.008333in}{-0.008333in}}{\pgfqpoint{0.008333in}{0.008333in}}{%
\pgfpathmoveto{\pgfqpoint{0.000000in}{-0.008333in}}%
\pgfpathcurveto{\pgfqpoint{0.002210in}{-0.008333in}}{\pgfqpoint{0.004330in}{-0.007455in}}{\pgfqpoint{0.005893in}{-0.005893in}}%
\pgfpathcurveto{\pgfqpoint{0.007455in}{-0.004330in}}{\pgfqpoint{0.008333in}{-0.002210in}}{\pgfqpoint{0.008333in}{0.000000in}}%
\pgfpathcurveto{\pgfqpoint{0.008333in}{0.002210in}}{\pgfqpoint{0.007455in}{0.004330in}}{\pgfqpoint{0.005893in}{0.005893in}}%
\pgfpathcurveto{\pgfqpoint{0.004330in}{0.007455in}}{\pgfqpoint{0.002210in}{0.008333in}}{\pgfqpoint{0.000000in}{0.008333in}}%
\pgfpathcurveto{\pgfqpoint{-0.002210in}{0.008333in}}{\pgfqpoint{-0.004330in}{0.007455in}}{\pgfqpoint{-0.005893in}{0.005893in}}%
\pgfpathcurveto{\pgfqpoint{-0.007455in}{0.004330in}}{\pgfqpoint{-0.008333in}{0.002210in}}{\pgfqpoint{-0.008333in}{0.000000in}}%
\pgfpathcurveto{\pgfqpoint{-0.008333in}{-0.002210in}}{\pgfqpoint{-0.007455in}{-0.004330in}}{\pgfqpoint{-0.005893in}{-0.005893in}}%
\pgfpathcurveto{\pgfqpoint{-0.004330in}{-0.007455in}}{\pgfqpoint{-0.002210in}{-0.008333in}}{\pgfqpoint{0.000000in}{-0.008333in}}%
\pgfpathclose%
\pgfusepath{stroke,fill}%
}%
\begin{pgfscope}%
\pgfsys@transformshift{0.428730in}{4.149174in}%
\pgfsys@useobject{currentmarker}{}%
\end{pgfscope}%
\end{pgfscope}%
\begin{pgfscope}%
\pgfpathrectangle{\pgfqpoint{0.100000in}{2.413063in}}{\pgfqpoint{5.037500in}{3.427208in}}%
\pgfusepath{clip}%
\pgfsetrectcap%
\pgfsetroundjoin%
\pgfsetlinewidth{1.505625pt}%
\definecolor{currentstroke}{rgb}{0.000000,0.000000,1.000000}%
\pgfsetstrokecolor{currentstroke}%
\pgfsetstrokeopacity{0.500000}%
\pgfsetdash{}{0pt}%
\pgfpathmoveto{\pgfqpoint{0.342987in}{4.594739in}}%
\pgfusepath{stroke}%
\end{pgfscope}%
\begin{pgfscope}%
\pgfpathrectangle{\pgfqpoint{0.100000in}{2.413063in}}{\pgfqpoint{5.037500in}{3.427208in}}%
\pgfusepath{clip}%
\pgfsetbuttcap%
\pgfsetroundjoin%
\definecolor{currentfill}{rgb}{0.000000,0.000000,1.000000}%
\pgfsetfillcolor{currentfill}%
\pgfsetfillopacity{0.500000}%
\pgfsetlinewidth{0.250937pt}%
\definecolor{currentstroke}{rgb}{0.000000,0.000000,0.000000}%
\pgfsetstrokecolor{currentstroke}%
\pgfsetstrokeopacity{0.500000}%
\pgfsetdash{}{0pt}%
\pgfsys@defobject{currentmarker}{\pgfqpoint{-0.025000in}{-0.025000in}}{\pgfqpoint{0.025000in}{0.025000in}}{%
\pgfpathmoveto{\pgfqpoint{0.000000in}{-0.025000in}}%
\pgfpathcurveto{\pgfqpoint{0.006630in}{-0.025000in}}{\pgfqpoint{0.012989in}{-0.022366in}}{\pgfqpoint{0.017678in}{-0.017678in}}%
\pgfpathcurveto{\pgfqpoint{0.022366in}{-0.012989in}}{\pgfqpoint{0.025000in}{-0.006630in}}{\pgfqpoint{0.025000in}{0.000000in}}%
\pgfpathcurveto{\pgfqpoint{0.025000in}{0.006630in}}{\pgfqpoint{0.022366in}{0.012989in}}{\pgfqpoint{0.017678in}{0.017678in}}%
\pgfpathcurveto{\pgfqpoint{0.012989in}{0.022366in}}{\pgfqpoint{0.006630in}{0.025000in}}{\pgfqpoint{0.000000in}{0.025000in}}%
\pgfpathcurveto{\pgfqpoint{-0.006630in}{0.025000in}}{\pgfqpoint{-0.012989in}{0.022366in}}{\pgfqpoint{-0.017678in}{0.017678in}}%
\pgfpathcurveto{\pgfqpoint{-0.022366in}{0.012989in}}{\pgfqpoint{-0.025000in}{0.006630in}}{\pgfqpoint{-0.025000in}{0.000000in}}%
\pgfpathcurveto{\pgfqpoint{-0.025000in}{-0.006630in}}{\pgfqpoint{-0.022366in}{-0.012989in}}{\pgfqpoint{-0.017678in}{-0.017678in}}%
\pgfpathcurveto{\pgfqpoint{-0.012989in}{-0.022366in}}{\pgfqpoint{-0.006630in}{-0.025000in}}{\pgfqpoint{0.000000in}{-0.025000in}}%
\pgfpathclose%
\pgfusepath{stroke,fill}%
}%
\begin{pgfscope}%
\pgfsys@transformshift{0.342987in}{4.594739in}%
\pgfsys@useobject{currentmarker}{}%
\end{pgfscope}%
\end{pgfscope}%
\begin{pgfscope}%
\pgfpathrectangle{\pgfqpoint{0.100000in}{2.413063in}}{\pgfqpoint{5.037500in}{3.427208in}}%
\pgfusepath{clip}%
\pgfsetrectcap%
\pgfsetroundjoin%
\pgfsetlinewidth{1.505625pt}%
\definecolor{currentstroke}{rgb}{0.000000,0.000000,1.000000}%
\pgfsetstrokecolor{currentstroke}%
\pgfsetstrokeopacity{0.500000}%
\pgfsetdash{}{0pt}%
\pgfpathmoveto{\pgfqpoint{0.449905in}{4.503858in}}%
\pgfusepath{stroke}%
\end{pgfscope}%
\begin{pgfscope}%
\pgfpathrectangle{\pgfqpoint{0.100000in}{2.413063in}}{\pgfqpoint{5.037500in}{3.427208in}}%
\pgfusepath{clip}%
\pgfsetbuttcap%
\pgfsetroundjoin%
\definecolor{currentfill}{rgb}{0.000000,0.000000,1.000000}%
\pgfsetfillcolor{currentfill}%
\pgfsetfillopacity{0.500000}%
\pgfsetlinewidth{0.250937pt}%
\definecolor{currentstroke}{rgb}{0.000000,0.000000,0.000000}%
\pgfsetstrokecolor{currentstroke}%
\pgfsetstrokeopacity{0.500000}%
\pgfsetdash{}{0pt}%
\pgfsys@defobject{currentmarker}{\pgfqpoint{-0.011111in}{-0.011111in}}{\pgfqpoint{0.011111in}{0.011111in}}{%
\pgfpathmoveto{\pgfqpoint{0.000000in}{-0.011111in}}%
\pgfpathcurveto{\pgfqpoint{0.002947in}{-0.011111in}}{\pgfqpoint{0.005773in}{-0.009940in}}{\pgfqpoint{0.007857in}{-0.007857in}}%
\pgfpathcurveto{\pgfqpoint{0.009940in}{-0.005773in}}{\pgfqpoint{0.011111in}{-0.002947in}}{\pgfqpoint{0.011111in}{0.000000in}}%
\pgfpathcurveto{\pgfqpoint{0.011111in}{0.002947in}}{\pgfqpoint{0.009940in}{0.005773in}}{\pgfqpoint{0.007857in}{0.007857in}}%
\pgfpathcurveto{\pgfqpoint{0.005773in}{0.009940in}}{\pgfqpoint{0.002947in}{0.011111in}}{\pgfqpoint{0.000000in}{0.011111in}}%
\pgfpathcurveto{\pgfqpoint{-0.002947in}{0.011111in}}{\pgfqpoint{-0.005773in}{0.009940in}}{\pgfqpoint{-0.007857in}{0.007857in}}%
\pgfpathcurveto{\pgfqpoint{-0.009940in}{0.005773in}}{\pgfqpoint{-0.011111in}{0.002947in}}{\pgfqpoint{-0.011111in}{0.000000in}}%
\pgfpathcurveto{\pgfqpoint{-0.011111in}{-0.002947in}}{\pgfqpoint{-0.009940in}{-0.005773in}}{\pgfqpoint{-0.007857in}{-0.007857in}}%
\pgfpathcurveto{\pgfqpoint{-0.005773in}{-0.009940in}}{\pgfqpoint{-0.002947in}{-0.011111in}}{\pgfqpoint{0.000000in}{-0.011111in}}%
\pgfpathclose%
\pgfusepath{stroke,fill}%
}%
\begin{pgfscope}%
\pgfsys@transformshift{0.449905in}{4.503858in}%
\pgfsys@useobject{currentmarker}{}%
\end{pgfscope}%
\end{pgfscope}%
\begin{pgfscope}%
\pgfpathrectangle{\pgfqpoint{0.100000in}{2.413063in}}{\pgfqpoint{5.037500in}{3.427208in}}%
\pgfusepath{clip}%
\pgfsetrectcap%
\pgfsetroundjoin%
\pgfsetlinewidth{1.505625pt}%
\definecolor{currentstroke}{rgb}{0.000000,0.000000,1.000000}%
\pgfsetstrokecolor{currentstroke}%
\pgfsetstrokeopacity{0.500000}%
\pgfsetdash{}{0pt}%
\pgfpathmoveto{\pgfqpoint{0.370959in}{4.545432in}}%
\pgfusepath{stroke}%
\end{pgfscope}%
\begin{pgfscope}%
\pgfpathrectangle{\pgfqpoint{0.100000in}{2.413063in}}{\pgfqpoint{5.037500in}{3.427208in}}%
\pgfusepath{clip}%
\pgfsetbuttcap%
\pgfsetroundjoin%
\definecolor{currentfill}{rgb}{0.000000,0.000000,1.000000}%
\pgfsetfillcolor{currentfill}%
\pgfsetfillopacity{0.500000}%
\pgfsetlinewidth{0.250937pt}%
\definecolor{currentstroke}{rgb}{0.000000,0.000000,0.000000}%
\pgfsetstrokecolor{currentstroke}%
\pgfsetstrokeopacity{0.500000}%
\pgfsetdash{}{0pt}%
\pgfsys@defobject{currentmarker}{\pgfqpoint{-0.044444in}{-0.044444in}}{\pgfqpoint{0.044444in}{0.044444in}}{%
\pgfpathmoveto{\pgfqpoint{0.000000in}{-0.044444in}}%
\pgfpathcurveto{\pgfqpoint{0.011787in}{-0.044444in}}{\pgfqpoint{0.023092in}{-0.039761in}}{\pgfqpoint{0.031427in}{-0.031427in}}%
\pgfpathcurveto{\pgfqpoint{0.039761in}{-0.023092in}}{\pgfqpoint{0.044444in}{-0.011787in}}{\pgfqpoint{0.044444in}{0.000000in}}%
\pgfpathcurveto{\pgfqpoint{0.044444in}{0.011787in}}{\pgfqpoint{0.039761in}{0.023092in}}{\pgfqpoint{0.031427in}{0.031427in}}%
\pgfpathcurveto{\pgfqpoint{0.023092in}{0.039761in}}{\pgfqpoint{0.011787in}{0.044444in}}{\pgfqpoint{0.000000in}{0.044444in}}%
\pgfpathcurveto{\pgfqpoint{-0.011787in}{0.044444in}}{\pgfqpoint{-0.023092in}{0.039761in}}{\pgfqpoint{-0.031427in}{0.031427in}}%
\pgfpathcurveto{\pgfqpoint{-0.039761in}{0.023092in}}{\pgfqpoint{-0.044444in}{0.011787in}}{\pgfqpoint{-0.044444in}{0.000000in}}%
\pgfpathcurveto{\pgfqpoint{-0.044444in}{-0.011787in}}{\pgfqpoint{-0.039761in}{-0.023092in}}{\pgfqpoint{-0.031427in}{-0.031427in}}%
\pgfpathcurveto{\pgfqpoint{-0.023092in}{-0.039761in}}{\pgfqpoint{-0.011787in}{-0.044444in}}{\pgfqpoint{0.000000in}{-0.044444in}}%
\pgfpathclose%
\pgfusepath{stroke,fill}%
}%
\begin{pgfscope}%
\pgfsys@transformshift{0.370959in}{4.545432in}%
\pgfsys@useobject{currentmarker}{}%
\end{pgfscope}%
\end{pgfscope}%
\begin{pgfscope}%
\pgfpathrectangle{\pgfqpoint{0.100000in}{2.413063in}}{\pgfqpoint{5.037500in}{3.427208in}}%
\pgfusepath{clip}%
\pgfsetrectcap%
\pgfsetroundjoin%
\pgfsetlinewidth{1.505625pt}%
\definecolor{currentstroke}{rgb}{0.678431,1.000000,0.184314}%
\pgfsetstrokecolor{currentstroke}%
\pgfsetstrokeopacity{0.500000}%
\pgfsetdash{}{0pt}%
\pgfpathmoveto{\pgfqpoint{0.574887in}{4.273145in}}%
\pgfusepath{stroke}%
\end{pgfscope}%
\begin{pgfscope}%
\pgfpathrectangle{\pgfqpoint{0.100000in}{2.413063in}}{\pgfqpoint{5.037500in}{3.427208in}}%
\pgfusepath{clip}%
\pgfsetbuttcap%
\pgfsetroundjoin%
\definecolor{currentfill}{rgb}{0.678431,1.000000,0.184314}%
\pgfsetfillcolor{currentfill}%
\pgfsetfillopacity{0.500000}%
\pgfsetlinewidth{0.250937pt}%
\definecolor{currentstroke}{rgb}{0.000000,0.000000,0.000000}%
\pgfsetstrokecolor{currentstroke}%
\pgfsetstrokeopacity{0.500000}%
\pgfsetdash{}{0pt}%
\pgfsys@defobject{currentmarker}{\pgfqpoint{-0.047222in}{-0.047222in}}{\pgfqpoint{0.047222in}{0.047222in}}{%
\pgfpathmoveto{\pgfqpoint{0.000000in}{-0.047222in}}%
\pgfpathcurveto{\pgfqpoint{0.012523in}{-0.047222in}}{\pgfqpoint{0.024536in}{-0.042247in}}{\pgfqpoint{0.033391in}{-0.033391in}}%
\pgfpathcurveto{\pgfqpoint{0.042247in}{-0.024536in}}{\pgfqpoint{0.047222in}{-0.012523in}}{\pgfqpoint{0.047222in}{0.000000in}}%
\pgfpathcurveto{\pgfqpoint{0.047222in}{0.012523in}}{\pgfqpoint{0.042247in}{0.024536in}}{\pgfqpoint{0.033391in}{0.033391in}}%
\pgfpathcurveto{\pgfqpoint{0.024536in}{0.042247in}}{\pgfqpoint{0.012523in}{0.047222in}}{\pgfqpoint{0.000000in}{0.047222in}}%
\pgfpathcurveto{\pgfqpoint{-0.012523in}{0.047222in}}{\pgfqpoint{-0.024536in}{0.042247in}}{\pgfqpoint{-0.033391in}{0.033391in}}%
\pgfpathcurveto{\pgfqpoint{-0.042247in}{0.024536in}}{\pgfqpoint{-0.047222in}{0.012523in}}{\pgfqpoint{-0.047222in}{0.000000in}}%
\pgfpathcurveto{\pgfqpoint{-0.047222in}{-0.012523in}}{\pgfqpoint{-0.042247in}{-0.024536in}}{\pgfqpoint{-0.033391in}{-0.033391in}}%
\pgfpathcurveto{\pgfqpoint{-0.024536in}{-0.042247in}}{\pgfqpoint{-0.012523in}{-0.047222in}}{\pgfqpoint{0.000000in}{-0.047222in}}%
\pgfpathclose%
\pgfusepath{stroke,fill}%
}%
\begin{pgfscope}%
\pgfsys@transformshift{0.574887in}{4.273145in}%
\pgfsys@useobject{currentmarker}{}%
\end{pgfscope}%
\end{pgfscope}%
\begin{pgfscope}%
\pgfpathrectangle{\pgfqpoint{0.100000in}{2.413063in}}{\pgfqpoint{5.037500in}{3.427208in}}%
\pgfusepath{clip}%
\pgfsetrectcap%
\pgfsetroundjoin%
\pgfsetlinewidth{1.505625pt}%
\definecolor{currentstroke}{rgb}{0.678431,1.000000,0.184314}%
\pgfsetstrokecolor{currentstroke}%
\pgfsetstrokeopacity{0.500000}%
\pgfsetdash{}{0pt}%
\pgfpathmoveto{\pgfqpoint{0.460859in}{4.642756in}}%
\pgfusepath{stroke}%
\end{pgfscope}%
\begin{pgfscope}%
\pgfpathrectangle{\pgfqpoint{0.100000in}{2.413063in}}{\pgfqpoint{5.037500in}{3.427208in}}%
\pgfusepath{clip}%
\pgfsetbuttcap%
\pgfsetroundjoin%
\definecolor{currentfill}{rgb}{0.678431,1.000000,0.184314}%
\pgfsetfillcolor{currentfill}%
\pgfsetfillopacity{0.500000}%
\pgfsetlinewidth{0.250937pt}%
\definecolor{currentstroke}{rgb}{0.000000,0.000000,0.000000}%
\pgfsetstrokecolor{currentstroke}%
\pgfsetstrokeopacity{0.500000}%
\pgfsetdash{}{0pt}%
\pgfsys@defobject{currentmarker}{\pgfqpoint{-0.011111in}{-0.011111in}}{\pgfqpoint{0.011111in}{0.011111in}}{%
\pgfpathmoveto{\pgfqpoint{0.000000in}{-0.011111in}}%
\pgfpathcurveto{\pgfqpoint{0.002947in}{-0.011111in}}{\pgfqpoint{0.005773in}{-0.009940in}}{\pgfqpoint{0.007857in}{-0.007857in}}%
\pgfpathcurveto{\pgfqpoint{0.009940in}{-0.005773in}}{\pgfqpoint{0.011111in}{-0.002947in}}{\pgfqpoint{0.011111in}{0.000000in}}%
\pgfpathcurveto{\pgfqpoint{0.011111in}{0.002947in}}{\pgfqpoint{0.009940in}{0.005773in}}{\pgfqpoint{0.007857in}{0.007857in}}%
\pgfpathcurveto{\pgfqpoint{0.005773in}{0.009940in}}{\pgfqpoint{0.002947in}{0.011111in}}{\pgfqpoint{0.000000in}{0.011111in}}%
\pgfpathcurveto{\pgfqpoint{-0.002947in}{0.011111in}}{\pgfqpoint{-0.005773in}{0.009940in}}{\pgfqpoint{-0.007857in}{0.007857in}}%
\pgfpathcurveto{\pgfqpoint{-0.009940in}{0.005773in}}{\pgfqpoint{-0.011111in}{0.002947in}}{\pgfqpoint{-0.011111in}{0.000000in}}%
\pgfpathcurveto{\pgfqpoint{-0.011111in}{-0.002947in}}{\pgfqpoint{-0.009940in}{-0.005773in}}{\pgfqpoint{-0.007857in}{-0.007857in}}%
\pgfpathcurveto{\pgfqpoint{-0.005773in}{-0.009940in}}{\pgfqpoint{-0.002947in}{-0.011111in}}{\pgfqpoint{0.000000in}{-0.011111in}}%
\pgfpathclose%
\pgfusepath{stroke,fill}%
}%
\begin{pgfscope}%
\pgfsys@transformshift{0.460859in}{4.642756in}%
\pgfsys@useobject{currentmarker}{}%
\end{pgfscope}%
\end{pgfscope}%
\begin{pgfscope}%
\pgfpathrectangle{\pgfqpoint{0.100000in}{2.413063in}}{\pgfqpoint{5.037500in}{3.427208in}}%
\pgfusepath{clip}%
\pgfsetrectcap%
\pgfsetroundjoin%
\pgfsetlinewidth{1.505625pt}%
\definecolor{currentstroke}{rgb}{0.000000,0.000000,1.000000}%
\pgfsetstrokecolor{currentstroke}%
\pgfsetstrokeopacity{0.500000}%
\pgfsetdash{}{0pt}%
\pgfpathmoveto{\pgfqpoint{1.900192in}{4.448961in}}%
\pgfusepath{stroke}%
\end{pgfscope}%
\begin{pgfscope}%
\pgfpathrectangle{\pgfqpoint{0.100000in}{2.413063in}}{\pgfqpoint{5.037500in}{3.427208in}}%
\pgfusepath{clip}%
\pgfsetbuttcap%
\pgfsetroundjoin%
\definecolor{currentfill}{rgb}{0.000000,0.000000,1.000000}%
\pgfsetfillcolor{currentfill}%
\pgfsetfillopacity{0.500000}%
\pgfsetlinewidth{0.250937pt}%
\definecolor{currentstroke}{rgb}{0.000000,0.000000,0.000000}%
\pgfsetstrokecolor{currentstroke}%
\pgfsetstrokeopacity{0.500000}%
\pgfsetdash{}{0pt}%
\pgfsys@defobject{currentmarker}{\pgfqpoint{-0.033333in}{-0.033333in}}{\pgfqpoint{0.033333in}{0.033333in}}{%
\pgfpathmoveto{\pgfqpoint{0.000000in}{-0.033333in}}%
\pgfpathcurveto{\pgfqpoint{0.008840in}{-0.033333in}}{\pgfqpoint{0.017319in}{-0.029821in}}{\pgfqpoint{0.023570in}{-0.023570in}}%
\pgfpathcurveto{\pgfqpoint{0.029821in}{-0.017319in}}{\pgfqpoint{0.033333in}{-0.008840in}}{\pgfqpoint{0.033333in}{0.000000in}}%
\pgfpathcurveto{\pgfqpoint{0.033333in}{0.008840in}}{\pgfqpoint{0.029821in}{0.017319in}}{\pgfqpoint{0.023570in}{0.023570in}}%
\pgfpathcurveto{\pgfqpoint{0.017319in}{0.029821in}}{\pgfqpoint{0.008840in}{0.033333in}}{\pgfqpoint{0.000000in}{0.033333in}}%
\pgfpathcurveto{\pgfqpoint{-0.008840in}{0.033333in}}{\pgfqpoint{-0.017319in}{0.029821in}}{\pgfqpoint{-0.023570in}{0.023570in}}%
\pgfpathcurveto{\pgfqpoint{-0.029821in}{0.017319in}}{\pgfqpoint{-0.033333in}{0.008840in}}{\pgfqpoint{-0.033333in}{0.000000in}}%
\pgfpathcurveto{\pgfqpoint{-0.033333in}{-0.008840in}}{\pgfqpoint{-0.029821in}{-0.017319in}}{\pgfqpoint{-0.023570in}{-0.023570in}}%
\pgfpathcurveto{\pgfqpoint{-0.017319in}{-0.029821in}}{\pgfqpoint{-0.008840in}{-0.033333in}}{\pgfqpoint{0.000000in}{-0.033333in}}%
\pgfpathclose%
\pgfusepath{stroke,fill}%
}%
\begin{pgfscope}%
\pgfsys@transformshift{1.900192in}{4.448961in}%
\pgfsys@useobject{currentmarker}{}%
\end{pgfscope}%
\end{pgfscope}%
\begin{pgfscope}%
\pgfpathrectangle{\pgfqpoint{0.100000in}{2.413063in}}{\pgfqpoint{5.037500in}{3.427208in}}%
\pgfusepath{clip}%
\pgfsetrectcap%
\pgfsetroundjoin%
\pgfsetlinewidth{1.505625pt}%
\definecolor{currentstroke}{rgb}{0.000000,0.000000,1.000000}%
\pgfsetstrokecolor{currentstroke}%
\pgfsetstrokeopacity{0.500000}%
\pgfsetdash{}{0pt}%
\pgfpathmoveto{\pgfqpoint{1.924544in}{4.309425in}}%
\pgfusepath{stroke}%
\end{pgfscope}%
\begin{pgfscope}%
\pgfpathrectangle{\pgfqpoint{0.100000in}{2.413063in}}{\pgfqpoint{5.037500in}{3.427208in}}%
\pgfusepath{clip}%
\pgfsetbuttcap%
\pgfsetroundjoin%
\definecolor{currentfill}{rgb}{0.000000,0.000000,1.000000}%
\pgfsetfillcolor{currentfill}%
\pgfsetfillopacity{0.500000}%
\pgfsetlinewidth{0.250937pt}%
\definecolor{currentstroke}{rgb}{0.000000,0.000000,0.000000}%
\pgfsetstrokecolor{currentstroke}%
\pgfsetstrokeopacity{0.500000}%
\pgfsetdash{}{0pt}%
\pgfsys@defobject{currentmarker}{\pgfqpoint{-0.041667in}{-0.041667in}}{\pgfqpoint{0.041667in}{0.041667in}}{%
\pgfpathmoveto{\pgfqpoint{0.000000in}{-0.041667in}}%
\pgfpathcurveto{\pgfqpoint{0.011050in}{-0.041667in}}{\pgfqpoint{0.021649in}{-0.037276in}}{\pgfqpoint{0.029463in}{-0.029463in}}%
\pgfpathcurveto{\pgfqpoint{0.037276in}{-0.021649in}}{\pgfqpoint{0.041667in}{-0.011050in}}{\pgfqpoint{0.041667in}{0.000000in}}%
\pgfpathcurveto{\pgfqpoint{0.041667in}{0.011050in}}{\pgfqpoint{0.037276in}{0.021649in}}{\pgfqpoint{0.029463in}{0.029463in}}%
\pgfpathcurveto{\pgfqpoint{0.021649in}{0.037276in}}{\pgfqpoint{0.011050in}{0.041667in}}{\pgfqpoint{0.000000in}{0.041667in}}%
\pgfpathcurveto{\pgfqpoint{-0.011050in}{0.041667in}}{\pgfqpoint{-0.021649in}{0.037276in}}{\pgfqpoint{-0.029463in}{0.029463in}}%
\pgfpathcurveto{\pgfqpoint{-0.037276in}{0.021649in}}{\pgfqpoint{-0.041667in}{0.011050in}}{\pgfqpoint{-0.041667in}{0.000000in}}%
\pgfpathcurveto{\pgfqpoint{-0.041667in}{-0.011050in}}{\pgfqpoint{-0.037276in}{-0.021649in}}{\pgfqpoint{-0.029463in}{-0.029463in}}%
\pgfpathcurveto{\pgfqpoint{-0.021649in}{-0.037276in}}{\pgfqpoint{-0.011050in}{-0.041667in}}{\pgfqpoint{0.000000in}{-0.041667in}}%
\pgfpathclose%
\pgfusepath{stroke,fill}%
}%
\begin{pgfscope}%
\pgfsys@transformshift{1.924544in}{4.309425in}%
\pgfsys@useobject{currentmarker}{}%
\end{pgfscope}%
\end{pgfscope}%
\begin{pgfscope}%
\pgfpathrectangle{\pgfqpoint{0.100000in}{2.413063in}}{\pgfqpoint{5.037500in}{3.427208in}}%
\pgfusepath{clip}%
\pgfsetrectcap%
\pgfsetroundjoin%
\pgfsetlinewidth{1.505625pt}%
\definecolor{currentstroke}{rgb}{0.000000,0.000000,1.000000}%
\pgfsetstrokecolor{currentstroke}%
\pgfsetstrokeopacity{0.500000}%
\pgfsetdash{}{0pt}%
\pgfpathmoveto{\pgfqpoint{1.921764in}{4.414584in}}%
\pgfusepath{stroke}%
\end{pgfscope}%
\begin{pgfscope}%
\pgfpathrectangle{\pgfqpoint{0.100000in}{2.413063in}}{\pgfqpoint{5.037500in}{3.427208in}}%
\pgfusepath{clip}%
\pgfsetbuttcap%
\pgfsetroundjoin%
\definecolor{currentfill}{rgb}{0.000000,0.000000,1.000000}%
\pgfsetfillcolor{currentfill}%
\pgfsetfillopacity{0.500000}%
\pgfsetlinewidth{0.250937pt}%
\definecolor{currentstroke}{rgb}{0.000000,0.000000,0.000000}%
\pgfsetstrokecolor{currentstroke}%
\pgfsetstrokeopacity{0.500000}%
\pgfsetdash{}{0pt}%
\pgfsys@defobject{currentmarker}{\pgfqpoint{-0.052778in}{-0.052778in}}{\pgfqpoint{0.052778in}{0.052778in}}{%
\pgfpathmoveto{\pgfqpoint{0.000000in}{-0.052778in}}%
\pgfpathcurveto{\pgfqpoint{0.013997in}{-0.052778in}}{\pgfqpoint{0.027422in}{-0.047217in}}{\pgfqpoint{0.037320in}{-0.037320in}}%
\pgfpathcurveto{\pgfqpoint{0.047217in}{-0.027422in}}{\pgfqpoint{0.052778in}{-0.013997in}}{\pgfqpoint{0.052778in}{0.000000in}}%
\pgfpathcurveto{\pgfqpoint{0.052778in}{0.013997in}}{\pgfqpoint{0.047217in}{0.027422in}}{\pgfqpoint{0.037320in}{0.037320in}}%
\pgfpathcurveto{\pgfqpoint{0.027422in}{0.047217in}}{\pgfqpoint{0.013997in}{0.052778in}}{\pgfqpoint{0.000000in}{0.052778in}}%
\pgfpathcurveto{\pgfqpoint{-0.013997in}{0.052778in}}{\pgfqpoint{-0.027422in}{0.047217in}}{\pgfqpoint{-0.037320in}{0.037320in}}%
\pgfpathcurveto{\pgfqpoint{-0.047217in}{0.027422in}}{\pgfqpoint{-0.052778in}{0.013997in}}{\pgfqpoint{-0.052778in}{0.000000in}}%
\pgfpathcurveto{\pgfqpoint{-0.052778in}{-0.013997in}}{\pgfqpoint{-0.047217in}{-0.027422in}}{\pgfqpoint{-0.037320in}{-0.037320in}}%
\pgfpathcurveto{\pgfqpoint{-0.027422in}{-0.047217in}}{\pgfqpoint{-0.013997in}{-0.052778in}}{\pgfqpoint{0.000000in}{-0.052778in}}%
\pgfpathclose%
\pgfusepath{stroke,fill}%
}%
\begin{pgfscope}%
\pgfsys@transformshift{1.921764in}{4.414584in}%
\pgfsys@useobject{currentmarker}{}%
\end{pgfscope}%
\end{pgfscope}%
\begin{pgfscope}%
\pgfpathrectangle{\pgfqpoint{0.100000in}{2.413063in}}{\pgfqpoint{5.037500in}{3.427208in}}%
\pgfusepath{clip}%
\pgfsetrectcap%
\pgfsetroundjoin%
\pgfsetlinewidth{1.505625pt}%
\definecolor{currentstroke}{rgb}{0.000000,0.000000,1.000000}%
\pgfsetstrokecolor{currentstroke}%
\pgfsetstrokeopacity{0.500000}%
\pgfsetdash{}{0pt}%
\pgfpathmoveto{\pgfqpoint{1.924872in}{4.508293in}}%
\pgfusepath{stroke}%
\end{pgfscope}%
\begin{pgfscope}%
\pgfpathrectangle{\pgfqpoint{0.100000in}{2.413063in}}{\pgfqpoint{5.037500in}{3.427208in}}%
\pgfusepath{clip}%
\pgfsetbuttcap%
\pgfsetroundjoin%
\definecolor{currentfill}{rgb}{0.000000,0.000000,1.000000}%
\pgfsetfillcolor{currentfill}%
\pgfsetfillopacity{0.500000}%
\pgfsetlinewidth{0.250937pt}%
\definecolor{currentstroke}{rgb}{0.000000,0.000000,0.000000}%
\pgfsetstrokecolor{currentstroke}%
\pgfsetstrokeopacity{0.500000}%
\pgfsetdash{}{0pt}%
\pgfsys@defobject{currentmarker}{\pgfqpoint{-0.038889in}{-0.038889in}}{\pgfqpoint{0.038889in}{0.038889in}}{%
\pgfpathmoveto{\pgfqpoint{0.000000in}{-0.038889in}}%
\pgfpathcurveto{\pgfqpoint{0.010313in}{-0.038889in}}{\pgfqpoint{0.020206in}{-0.034791in}}{\pgfqpoint{0.027499in}{-0.027499in}}%
\pgfpathcurveto{\pgfqpoint{0.034791in}{-0.020206in}}{\pgfqpoint{0.038889in}{-0.010313in}}{\pgfqpoint{0.038889in}{0.000000in}}%
\pgfpathcurveto{\pgfqpoint{0.038889in}{0.010313in}}{\pgfqpoint{0.034791in}{0.020206in}}{\pgfqpoint{0.027499in}{0.027499in}}%
\pgfpathcurveto{\pgfqpoint{0.020206in}{0.034791in}}{\pgfqpoint{0.010313in}{0.038889in}}{\pgfqpoint{0.000000in}{0.038889in}}%
\pgfpathcurveto{\pgfqpoint{-0.010313in}{0.038889in}}{\pgfqpoint{-0.020206in}{0.034791in}}{\pgfqpoint{-0.027499in}{0.027499in}}%
\pgfpathcurveto{\pgfqpoint{-0.034791in}{0.020206in}}{\pgfqpoint{-0.038889in}{0.010313in}}{\pgfqpoint{-0.038889in}{0.000000in}}%
\pgfpathcurveto{\pgfqpoint{-0.038889in}{-0.010313in}}{\pgfqpoint{-0.034791in}{-0.020206in}}{\pgfqpoint{-0.027499in}{-0.027499in}}%
\pgfpathcurveto{\pgfqpoint{-0.020206in}{-0.034791in}}{\pgfqpoint{-0.010313in}{-0.038889in}}{\pgfqpoint{0.000000in}{-0.038889in}}%
\pgfpathclose%
\pgfusepath{stroke,fill}%
}%
\begin{pgfscope}%
\pgfsys@transformshift{1.924872in}{4.508293in}%
\pgfsys@useobject{currentmarker}{}%
\end{pgfscope}%
\end{pgfscope}%
\begin{pgfscope}%
\pgfpathrectangle{\pgfqpoint{0.100000in}{2.413063in}}{\pgfqpoint{5.037500in}{3.427208in}}%
\pgfusepath{clip}%
\pgfsetrectcap%
\pgfsetroundjoin%
\pgfsetlinewidth{1.505625pt}%
\definecolor{currentstroke}{rgb}{0.000000,0.000000,1.000000}%
\pgfsetstrokecolor{currentstroke}%
\pgfsetstrokeopacity{0.500000}%
\pgfsetdash{}{0pt}%
\pgfpathmoveto{\pgfqpoint{1.597107in}{4.378471in}}%
\pgfusepath{stroke}%
\end{pgfscope}%
\begin{pgfscope}%
\pgfpathrectangle{\pgfqpoint{0.100000in}{2.413063in}}{\pgfqpoint{5.037500in}{3.427208in}}%
\pgfusepath{clip}%
\pgfsetbuttcap%
\pgfsetroundjoin%
\definecolor{currentfill}{rgb}{0.000000,0.000000,1.000000}%
\pgfsetfillcolor{currentfill}%
\pgfsetfillopacity{0.500000}%
\pgfsetlinewidth{0.250937pt}%
\definecolor{currentstroke}{rgb}{0.000000,0.000000,0.000000}%
\pgfsetstrokecolor{currentstroke}%
\pgfsetstrokeopacity{0.500000}%
\pgfsetdash{}{0pt}%
\pgfsys@defobject{currentmarker}{\pgfqpoint{-0.038889in}{-0.038889in}}{\pgfqpoint{0.038889in}{0.038889in}}{%
\pgfpathmoveto{\pgfqpoint{0.000000in}{-0.038889in}}%
\pgfpathcurveto{\pgfqpoint{0.010313in}{-0.038889in}}{\pgfqpoint{0.020206in}{-0.034791in}}{\pgfqpoint{0.027499in}{-0.027499in}}%
\pgfpathcurveto{\pgfqpoint{0.034791in}{-0.020206in}}{\pgfqpoint{0.038889in}{-0.010313in}}{\pgfqpoint{0.038889in}{0.000000in}}%
\pgfpathcurveto{\pgfqpoint{0.038889in}{0.010313in}}{\pgfqpoint{0.034791in}{0.020206in}}{\pgfqpoint{0.027499in}{0.027499in}}%
\pgfpathcurveto{\pgfqpoint{0.020206in}{0.034791in}}{\pgfqpoint{0.010313in}{0.038889in}}{\pgfqpoint{0.000000in}{0.038889in}}%
\pgfpathcurveto{\pgfqpoint{-0.010313in}{0.038889in}}{\pgfqpoint{-0.020206in}{0.034791in}}{\pgfqpoint{-0.027499in}{0.027499in}}%
\pgfpathcurveto{\pgfqpoint{-0.034791in}{0.020206in}}{\pgfqpoint{-0.038889in}{0.010313in}}{\pgfqpoint{-0.038889in}{0.000000in}}%
\pgfpathcurveto{\pgfqpoint{-0.038889in}{-0.010313in}}{\pgfqpoint{-0.034791in}{-0.020206in}}{\pgfqpoint{-0.027499in}{-0.027499in}}%
\pgfpathcurveto{\pgfqpoint{-0.020206in}{-0.034791in}}{\pgfqpoint{-0.010313in}{-0.038889in}}{\pgfqpoint{0.000000in}{-0.038889in}}%
\pgfpathclose%
\pgfusepath{stroke,fill}%
}%
\begin{pgfscope}%
\pgfsys@transformshift{1.597107in}{4.378471in}%
\pgfsys@useobject{currentmarker}{}%
\end{pgfscope}%
\end{pgfscope}%
\begin{pgfscope}%
\pgfpathrectangle{\pgfqpoint{0.100000in}{2.413063in}}{\pgfqpoint{5.037500in}{3.427208in}}%
\pgfusepath{clip}%
\pgfsetrectcap%
\pgfsetroundjoin%
\pgfsetlinewidth{1.505625pt}%
\definecolor{currentstroke}{rgb}{0.000000,0.000000,1.000000}%
\pgfsetstrokecolor{currentstroke}%
\pgfsetstrokeopacity{0.500000}%
\pgfsetdash{}{0pt}%
\pgfpathmoveto{\pgfqpoint{1.954425in}{4.490288in}}%
\pgfusepath{stroke}%
\end{pgfscope}%
\begin{pgfscope}%
\pgfpathrectangle{\pgfqpoint{0.100000in}{2.413063in}}{\pgfqpoint{5.037500in}{3.427208in}}%
\pgfusepath{clip}%
\pgfsetbuttcap%
\pgfsetroundjoin%
\definecolor{currentfill}{rgb}{0.000000,0.000000,1.000000}%
\pgfsetfillcolor{currentfill}%
\pgfsetfillopacity{0.500000}%
\pgfsetlinewidth{0.250937pt}%
\definecolor{currentstroke}{rgb}{0.000000,0.000000,0.000000}%
\pgfsetstrokecolor{currentstroke}%
\pgfsetstrokeopacity{0.500000}%
\pgfsetdash{}{0pt}%
\pgfsys@defobject{currentmarker}{\pgfqpoint{-0.050000in}{-0.050000in}}{\pgfqpoint{0.050000in}{0.050000in}}{%
\pgfpathmoveto{\pgfqpoint{0.000000in}{-0.050000in}}%
\pgfpathcurveto{\pgfqpoint{0.013260in}{-0.050000in}}{\pgfqpoint{0.025979in}{-0.044732in}}{\pgfqpoint{0.035355in}{-0.035355in}}%
\pgfpathcurveto{\pgfqpoint{0.044732in}{-0.025979in}}{\pgfqpoint{0.050000in}{-0.013260in}}{\pgfqpoint{0.050000in}{0.000000in}}%
\pgfpathcurveto{\pgfqpoint{0.050000in}{0.013260in}}{\pgfqpoint{0.044732in}{0.025979in}}{\pgfqpoint{0.035355in}{0.035355in}}%
\pgfpathcurveto{\pgfqpoint{0.025979in}{0.044732in}}{\pgfqpoint{0.013260in}{0.050000in}}{\pgfqpoint{0.000000in}{0.050000in}}%
\pgfpathcurveto{\pgfqpoint{-0.013260in}{0.050000in}}{\pgfqpoint{-0.025979in}{0.044732in}}{\pgfqpoint{-0.035355in}{0.035355in}}%
\pgfpathcurveto{\pgfqpoint{-0.044732in}{0.025979in}}{\pgfqpoint{-0.050000in}{0.013260in}}{\pgfqpoint{-0.050000in}{0.000000in}}%
\pgfpathcurveto{\pgfqpoint{-0.050000in}{-0.013260in}}{\pgfqpoint{-0.044732in}{-0.025979in}}{\pgfqpoint{-0.035355in}{-0.035355in}}%
\pgfpathcurveto{\pgfqpoint{-0.025979in}{-0.044732in}}{\pgfqpoint{-0.013260in}{-0.050000in}}{\pgfqpoint{0.000000in}{-0.050000in}}%
\pgfpathclose%
\pgfusepath{stroke,fill}%
}%
\begin{pgfscope}%
\pgfsys@transformshift{1.954425in}{4.490288in}%
\pgfsys@useobject{currentmarker}{}%
\end{pgfscope}%
\end{pgfscope}%
\begin{pgfscope}%
\pgfpathrectangle{\pgfqpoint{0.100000in}{2.413063in}}{\pgfqpoint{5.037500in}{3.427208in}}%
\pgfusepath{clip}%
\pgfsetrectcap%
\pgfsetroundjoin%
\pgfsetlinewidth{1.505625pt}%
\definecolor{currentstroke}{rgb}{0.000000,0.000000,1.000000}%
\pgfsetstrokecolor{currentstroke}%
\pgfsetstrokeopacity{0.500000}%
\pgfsetdash{}{0pt}%
\pgfpathmoveto{\pgfqpoint{1.936783in}{4.240996in}}%
\pgfusepath{stroke}%
\end{pgfscope}%
\begin{pgfscope}%
\pgfpathrectangle{\pgfqpoint{0.100000in}{2.413063in}}{\pgfqpoint{5.037500in}{3.427208in}}%
\pgfusepath{clip}%
\pgfsetbuttcap%
\pgfsetroundjoin%
\definecolor{currentfill}{rgb}{0.000000,0.000000,1.000000}%
\pgfsetfillcolor{currentfill}%
\pgfsetfillopacity{0.500000}%
\pgfsetlinewidth{0.250937pt}%
\definecolor{currentstroke}{rgb}{0.000000,0.000000,0.000000}%
\pgfsetstrokecolor{currentstroke}%
\pgfsetstrokeopacity{0.500000}%
\pgfsetdash{}{0pt}%
\pgfsys@defobject{currentmarker}{\pgfqpoint{-0.077778in}{-0.077778in}}{\pgfqpoint{0.077778in}{0.077778in}}{%
\pgfpathmoveto{\pgfqpoint{0.000000in}{-0.077778in}}%
\pgfpathcurveto{\pgfqpoint{0.020627in}{-0.077778in}}{\pgfqpoint{0.040412in}{-0.069583in}}{\pgfqpoint{0.054997in}{-0.054997in}}%
\pgfpathcurveto{\pgfqpoint{0.069583in}{-0.040412in}}{\pgfqpoint{0.077778in}{-0.020627in}}{\pgfqpoint{0.077778in}{0.000000in}}%
\pgfpathcurveto{\pgfqpoint{0.077778in}{0.020627in}}{\pgfqpoint{0.069583in}{0.040412in}}{\pgfqpoint{0.054997in}{0.054997in}}%
\pgfpathcurveto{\pgfqpoint{0.040412in}{0.069583in}}{\pgfqpoint{0.020627in}{0.077778in}}{\pgfqpoint{0.000000in}{0.077778in}}%
\pgfpathcurveto{\pgfqpoint{-0.020627in}{0.077778in}}{\pgfqpoint{-0.040412in}{0.069583in}}{\pgfqpoint{-0.054997in}{0.054997in}}%
\pgfpathcurveto{\pgfqpoint{-0.069583in}{0.040412in}}{\pgfqpoint{-0.077778in}{0.020627in}}{\pgfqpoint{-0.077778in}{0.000000in}}%
\pgfpathcurveto{\pgfqpoint{-0.077778in}{-0.020627in}}{\pgfqpoint{-0.069583in}{-0.040412in}}{\pgfqpoint{-0.054997in}{-0.054997in}}%
\pgfpathcurveto{\pgfqpoint{-0.040412in}{-0.069583in}}{\pgfqpoint{-0.020627in}{-0.077778in}}{\pgfqpoint{0.000000in}{-0.077778in}}%
\pgfpathclose%
\pgfusepath{stroke,fill}%
}%
\begin{pgfscope}%
\pgfsys@transformshift{1.936783in}{4.240996in}%
\pgfsys@useobject{currentmarker}{}%
\end{pgfscope}%
\end{pgfscope}%
\begin{pgfscope}%
\pgfpathrectangle{\pgfqpoint{0.100000in}{2.413063in}}{\pgfqpoint{5.037500in}{3.427208in}}%
\pgfusepath{clip}%
\pgfsetrectcap%
\pgfsetroundjoin%
\pgfsetlinewidth{1.505625pt}%
\definecolor{currentstroke}{rgb}{0.000000,0.000000,1.000000}%
\pgfsetstrokecolor{currentstroke}%
\pgfsetstrokeopacity{0.500000}%
\pgfsetdash{}{0pt}%
\pgfpathmoveto{\pgfqpoint{4.676566in}{4.756149in}}%
\pgfusepath{stroke}%
\end{pgfscope}%
\begin{pgfscope}%
\pgfpathrectangle{\pgfqpoint{0.100000in}{2.413063in}}{\pgfqpoint{5.037500in}{3.427208in}}%
\pgfusepath{clip}%
\pgfsetbuttcap%
\pgfsetroundjoin%
\definecolor{currentfill}{rgb}{0.000000,0.000000,1.000000}%
\pgfsetfillcolor{currentfill}%
\pgfsetfillopacity{0.500000}%
\pgfsetlinewidth{0.250937pt}%
\definecolor{currentstroke}{rgb}{0.000000,0.000000,0.000000}%
\pgfsetstrokecolor{currentstroke}%
\pgfsetstrokeopacity{0.500000}%
\pgfsetdash{}{0pt}%
\pgfsys@defobject{currentmarker}{\pgfqpoint{-0.041667in}{-0.041667in}}{\pgfqpoint{0.041667in}{0.041667in}}{%
\pgfpathmoveto{\pgfqpoint{0.000000in}{-0.041667in}}%
\pgfpathcurveto{\pgfqpoint{0.011050in}{-0.041667in}}{\pgfqpoint{0.021649in}{-0.037276in}}{\pgfqpoint{0.029463in}{-0.029463in}}%
\pgfpathcurveto{\pgfqpoint{0.037276in}{-0.021649in}}{\pgfqpoint{0.041667in}{-0.011050in}}{\pgfqpoint{0.041667in}{0.000000in}}%
\pgfpathcurveto{\pgfqpoint{0.041667in}{0.011050in}}{\pgfqpoint{0.037276in}{0.021649in}}{\pgfqpoint{0.029463in}{0.029463in}}%
\pgfpathcurveto{\pgfqpoint{0.021649in}{0.037276in}}{\pgfqpoint{0.011050in}{0.041667in}}{\pgfqpoint{0.000000in}{0.041667in}}%
\pgfpathcurveto{\pgfqpoint{-0.011050in}{0.041667in}}{\pgfqpoint{-0.021649in}{0.037276in}}{\pgfqpoint{-0.029463in}{0.029463in}}%
\pgfpathcurveto{\pgfqpoint{-0.037276in}{0.021649in}}{\pgfqpoint{-0.041667in}{0.011050in}}{\pgfqpoint{-0.041667in}{0.000000in}}%
\pgfpathcurveto{\pgfqpoint{-0.041667in}{-0.011050in}}{\pgfqpoint{-0.037276in}{-0.021649in}}{\pgfqpoint{-0.029463in}{-0.029463in}}%
\pgfpathcurveto{\pgfqpoint{-0.021649in}{-0.037276in}}{\pgfqpoint{-0.011050in}{-0.041667in}}{\pgfqpoint{0.000000in}{-0.041667in}}%
\pgfpathclose%
\pgfusepath{stroke,fill}%
}%
\begin{pgfscope}%
\pgfsys@transformshift{4.676566in}{4.756149in}%
\pgfsys@useobject{currentmarker}{}%
\end{pgfscope}%
\end{pgfscope}%
\begin{pgfscope}%
\pgfpathrectangle{\pgfqpoint{0.100000in}{2.413063in}}{\pgfqpoint{5.037500in}{3.427208in}}%
\pgfusepath{clip}%
\pgfsetrectcap%
\pgfsetroundjoin%
\pgfsetlinewidth{1.505625pt}%
\definecolor{currentstroke}{rgb}{0.000000,0.000000,1.000000}%
\pgfsetstrokecolor{currentstroke}%
\pgfsetstrokeopacity{0.500000}%
\pgfsetdash{}{0pt}%
\pgfpathmoveto{\pgfqpoint{4.649399in}{4.776556in}}%
\pgfusepath{stroke}%
\end{pgfscope}%
\begin{pgfscope}%
\pgfpathrectangle{\pgfqpoint{0.100000in}{2.413063in}}{\pgfqpoint{5.037500in}{3.427208in}}%
\pgfusepath{clip}%
\pgfsetbuttcap%
\pgfsetroundjoin%
\definecolor{currentfill}{rgb}{0.000000,0.000000,1.000000}%
\pgfsetfillcolor{currentfill}%
\pgfsetfillopacity{0.500000}%
\pgfsetlinewidth{0.250937pt}%
\definecolor{currentstroke}{rgb}{0.000000,0.000000,0.000000}%
\pgfsetstrokecolor{currentstroke}%
\pgfsetstrokeopacity{0.500000}%
\pgfsetdash{}{0pt}%
\pgfsys@defobject{currentmarker}{\pgfqpoint{-0.030556in}{-0.030556in}}{\pgfqpoint{0.030556in}{0.030556in}}{%
\pgfpathmoveto{\pgfqpoint{0.000000in}{-0.030556in}}%
\pgfpathcurveto{\pgfqpoint{0.008103in}{-0.030556in}}{\pgfqpoint{0.015876in}{-0.027336in}}{\pgfqpoint{0.021606in}{-0.021606in}}%
\pgfpathcurveto{\pgfqpoint{0.027336in}{-0.015876in}}{\pgfqpoint{0.030556in}{-0.008103in}}{\pgfqpoint{0.030556in}{0.000000in}}%
\pgfpathcurveto{\pgfqpoint{0.030556in}{0.008103in}}{\pgfqpoint{0.027336in}{0.015876in}}{\pgfqpoint{0.021606in}{0.021606in}}%
\pgfpathcurveto{\pgfqpoint{0.015876in}{0.027336in}}{\pgfqpoint{0.008103in}{0.030556in}}{\pgfqpoint{0.000000in}{0.030556in}}%
\pgfpathcurveto{\pgfqpoint{-0.008103in}{0.030556in}}{\pgfqpoint{-0.015876in}{0.027336in}}{\pgfqpoint{-0.021606in}{0.021606in}}%
\pgfpathcurveto{\pgfqpoint{-0.027336in}{0.015876in}}{\pgfqpoint{-0.030556in}{0.008103in}}{\pgfqpoint{-0.030556in}{0.000000in}}%
\pgfpathcurveto{\pgfqpoint{-0.030556in}{-0.008103in}}{\pgfqpoint{-0.027336in}{-0.015876in}}{\pgfqpoint{-0.021606in}{-0.021606in}}%
\pgfpathcurveto{\pgfqpoint{-0.015876in}{-0.027336in}}{\pgfqpoint{-0.008103in}{-0.030556in}}{\pgfqpoint{0.000000in}{-0.030556in}}%
\pgfpathclose%
\pgfusepath{stroke,fill}%
}%
\begin{pgfscope}%
\pgfsys@transformshift{4.649399in}{4.776556in}%
\pgfsys@useobject{currentmarker}{}%
\end{pgfscope}%
\end{pgfscope}%
\begin{pgfscope}%
\pgfpathrectangle{\pgfqpoint{0.100000in}{2.413063in}}{\pgfqpoint{5.037500in}{3.427208in}}%
\pgfusepath{clip}%
\pgfsetrectcap%
\pgfsetroundjoin%
\pgfsetlinewidth{1.505625pt}%
\definecolor{currentstroke}{rgb}{0.000000,0.000000,1.000000}%
\pgfsetstrokecolor{currentstroke}%
\pgfsetstrokeopacity{0.500000}%
\pgfsetdash{}{0pt}%
\pgfpathmoveto{\pgfqpoint{4.703787in}{4.833704in}}%
\pgfusepath{stroke}%
\end{pgfscope}%
\begin{pgfscope}%
\pgfpathrectangle{\pgfqpoint{0.100000in}{2.413063in}}{\pgfqpoint{5.037500in}{3.427208in}}%
\pgfusepath{clip}%
\pgfsetbuttcap%
\pgfsetroundjoin%
\definecolor{currentfill}{rgb}{0.000000,0.000000,1.000000}%
\pgfsetfillcolor{currentfill}%
\pgfsetfillopacity{0.500000}%
\pgfsetlinewidth{0.250937pt}%
\definecolor{currentstroke}{rgb}{0.000000,0.000000,0.000000}%
\pgfsetstrokecolor{currentstroke}%
\pgfsetstrokeopacity{0.500000}%
\pgfsetdash{}{0pt}%
\pgfsys@defobject{currentmarker}{\pgfqpoint{-0.038889in}{-0.038889in}}{\pgfqpoint{0.038889in}{0.038889in}}{%
\pgfpathmoveto{\pgfqpoint{0.000000in}{-0.038889in}}%
\pgfpathcurveto{\pgfqpoint{0.010313in}{-0.038889in}}{\pgfqpoint{0.020206in}{-0.034791in}}{\pgfqpoint{0.027499in}{-0.027499in}}%
\pgfpathcurveto{\pgfqpoint{0.034791in}{-0.020206in}}{\pgfqpoint{0.038889in}{-0.010313in}}{\pgfqpoint{0.038889in}{0.000000in}}%
\pgfpathcurveto{\pgfqpoint{0.038889in}{0.010313in}}{\pgfqpoint{0.034791in}{0.020206in}}{\pgfqpoint{0.027499in}{0.027499in}}%
\pgfpathcurveto{\pgfqpoint{0.020206in}{0.034791in}}{\pgfqpoint{0.010313in}{0.038889in}}{\pgfqpoint{0.000000in}{0.038889in}}%
\pgfpathcurveto{\pgfqpoint{-0.010313in}{0.038889in}}{\pgfqpoint{-0.020206in}{0.034791in}}{\pgfqpoint{-0.027499in}{0.027499in}}%
\pgfpathcurveto{\pgfqpoint{-0.034791in}{0.020206in}}{\pgfqpoint{-0.038889in}{0.010313in}}{\pgfqpoint{-0.038889in}{0.000000in}}%
\pgfpathcurveto{\pgfqpoint{-0.038889in}{-0.010313in}}{\pgfqpoint{-0.034791in}{-0.020206in}}{\pgfqpoint{-0.027499in}{-0.027499in}}%
\pgfpathcurveto{\pgfqpoint{-0.020206in}{-0.034791in}}{\pgfqpoint{-0.010313in}{-0.038889in}}{\pgfqpoint{0.000000in}{-0.038889in}}%
\pgfpathclose%
\pgfusepath{stroke,fill}%
}%
\begin{pgfscope}%
\pgfsys@transformshift{4.703787in}{4.833704in}%
\pgfsys@useobject{currentmarker}{}%
\end{pgfscope}%
\end{pgfscope}%
\begin{pgfscope}%
\pgfpathrectangle{\pgfqpoint{0.100000in}{2.413063in}}{\pgfqpoint{5.037500in}{3.427208in}}%
\pgfusepath{clip}%
\pgfsetrectcap%
\pgfsetroundjoin%
\pgfsetlinewidth{1.505625pt}%
\definecolor{currentstroke}{rgb}{0.000000,0.000000,1.000000}%
\pgfsetstrokecolor{currentstroke}%
\pgfsetstrokeopacity{0.500000}%
\pgfsetdash{}{0pt}%
\pgfpathmoveto{\pgfqpoint{4.696218in}{4.777739in}}%
\pgfusepath{stroke}%
\end{pgfscope}%
\begin{pgfscope}%
\pgfpathrectangle{\pgfqpoint{0.100000in}{2.413063in}}{\pgfqpoint{5.037500in}{3.427208in}}%
\pgfusepath{clip}%
\pgfsetbuttcap%
\pgfsetroundjoin%
\definecolor{currentfill}{rgb}{0.000000,0.000000,1.000000}%
\pgfsetfillcolor{currentfill}%
\pgfsetfillopacity{0.500000}%
\pgfsetlinewidth{0.250937pt}%
\definecolor{currentstroke}{rgb}{0.000000,0.000000,0.000000}%
\pgfsetstrokecolor{currentstroke}%
\pgfsetstrokeopacity{0.500000}%
\pgfsetdash{}{0pt}%
\pgfsys@defobject{currentmarker}{\pgfqpoint{-0.036111in}{-0.036111in}}{\pgfqpoint{0.036111in}{0.036111in}}{%
\pgfpathmoveto{\pgfqpoint{0.000000in}{-0.036111in}}%
\pgfpathcurveto{\pgfqpoint{0.009577in}{-0.036111in}}{\pgfqpoint{0.018763in}{-0.032306in}}{\pgfqpoint{0.025534in}{-0.025534in}}%
\pgfpathcurveto{\pgfqpoint{0.032306in}{-0.018763in}}{\pgfqpoint{0.036111in}{-0.009577in}}{\pgfqpoint{0.036111in}{0.000000in}}%
\pgfpathcurveto{\pgfqpoint{0.036111in}{0.009577in}}{\pgfqpoint{0.032306in}{0.018763in}}{\pgfqpoint{0.025534in}{0.025534in}}%
\pgfpathcurveto{\pgfqpoint{0.018763in}{0.032306in}}{\pgfqpoint{0.009577in}{0.036111in}}{\pgfqpoint{0.000000in}{0.036111in}}%
\pgfpathcurveto{\pgfqpoint{-0.009577in}{0.036111in}}{\pgfqpoint{-0.018763in}{0.032306in}}{\pgfqpoint{-0.025534in}{0.025534in}}%
\pgfpathcurveto{\pgfqpoint{-0.032306in}{0.018763in}}{\pgfqpoint{-0.036111in}{0.009577in}}{\pgfqpoint{-0.036111in}{0.000000in}}%
\pgfpathcurveto{\pgfqpoint{-0.036111in}{-0.009577in}}{\pgfqpoint{-0.032306in}{-0.018763in}}{\pgfqpoint{-0.025534in}{-0.025534in}}%
\pgfpathcurveto{\pgfqpoint{-0.018763in}{-0.032306in}}{\pgfqpoint{-0.009577in}{-0.036111in}}{\pgfqpoint{0.000000in}{-0.036111in}}%
\pgfpathclose%
\pgfusepath{stroke,fill}%
}%
\begin{pgfscope}%
\pgfsys@transformshift{4.696218in}{4.777739in}%
\pgfsys@useobject{currentmarker}{}%
\end{pgfscope}%
\end{pgfscope}%
\begin{pgfscope}%
\pgfpathrectangle{\pgfqpoint{0.100000in}{2.413063in}}{\pgfqpoint{5.037500in}{3.427208in}}%
\pgfusepath{clip}%
\pgfsetrectcap%
\pgfsetroundjoin%
\pgfsetlinewidth{1.505625pt}%
\definecolor{currentstroke}{rgb}{0.000000,0.000000,1.000000}%
\pgfsetstrokecolor{currentstroke}%
\pgfsetstrokeopacity{0.500000}%
\pgfsetdash{}{0pt}%
\pgfpathmoveto{\pgfqpoint{4.761265in}{4.819862in}}%
\pgfusepath{stroke}%
\end{pgfscope}%
\begin{pgfscope}%
\pgfpathrectangle{\pgfqpoint{0.100000in}{2.413063in}}{\pgfqpoint{5.037500in}{3.427208in}}%
\pgfusepath{clip}%
\pgfsetbuttcap%
\pgfsetroundjoin%
\definecolor{currentfill}{rgb}{0.000000,0.000000,1.000000}%
\pgfsetfillcolor{currentfill}%
\pgfsetfillopacity{0.500000}%
\pgfsetlinewidth{0.250937pt}%
\definecolor{currentstroke}{rgb}{0.000000,0.000000,0.000000}%
\pgfsetstrokecolor{currentstroke}%
\pgfsetstrokeopacity{0.500000}%
\pgfsetdash{}{0pt}%
\pgfsys@defobject{currentmarker}{\pgfqpoint{-0.050000in}{-0.050000in}}{\pgfqpoint{0.050000in}{0.050000in}}{%
\pgfpathmoveto{\pgfqpoint{0.000000in}{-0.050000in}}%
\pgfpathcurveto{\pgfqpoint{0.013260in}{-0.050000in}}{\pgfqpoint{0.025979in}{-0.044732in}}{\pgfqpoint{0.035355in}{-0.035355in}}%
\pgfpathcurveto{\pgfqpoint{0.044732in}{-0.025979in}}{\pgfqpoint{0.050000in}{-0.013260in}}{\pgfqpoint{0.050000in}{0.000000in}}%
\pgfpathcurveto{\pgfqpoint{0.050000in}{0.013260in}}{\pgfqpoint{0.044732in}{0.025979in}}{\pgfqpoint{0.035355in}{0.035355in}}%
\pgfpathcurveto{\pgfqpoint{0.025979in}{0.044732in}}{\pgfqpoint{0.013260in}{0.050000in}}{\pgfqpoint{0.000000in}{0.050000in}}%
\pgfpathcurveto{\pgfqpoint{-0.013260in}{0.050000in}}{\pgfqpoint{-0.025979in}{0.044732in}}{\pgfqpoint{-0.035355in}{0.035355in}}%
\pgfpathcurveto{\pgfqpoint{-0.044732in}{0.025979in}}{\pgfqpoint{-0.050000in}{0.013260in}}{\pgfqpoint{-0.050000in}{0.000000in}}%
\pgfpathcurveto{\pgfqpoint{-0.050000in}{-0.013260in}}{\pgfqpoint{-0.044732in}{-0.025979in}}{\pgfqpoint{-0.035355in}{-0.035355in}}%
\pgfpathcurveto{\pgfqpoint{-0.025979in}{-0.044732in}}{\pgfqpoint{-0.013260in}{-0.050000in}}{\pgfqpoint{0.000000in}{-0.050000in}}%
\pgfpathclose%
\pgfusepath{stroke,fill}%
}%
\begin{pgfscope}%
\pgfsys@transformshift{4.761265in}{4.819862in}%
\pgfsys@useobject{currentmarker}{}%
\end{pgfscope}%
\end{pgfscope}%
\begin{pgfscope}%
\pgfpathrectangle{\pgfqpoint{0.100000in}{2.413063in}}{\pgfqpoint{5.037500in}{3.427208in}}%
\pgfusepath{clip}%
\pgfsetrectcap%
\pgfsetroundjoin%
\pgfsetlinewidth{1.505625pt}%
\definecolor{currentstroke}{rgb}{0.000000,0.000000,1.000000}%
\pgfsetstrokecolor{currentstroke}%
\pgfsetstrokeopacity{0.500000}%
\pgfsetdash{}{0pt}%
\pgfpathmoveto{\pgfqpoint{4.678703in}{4.803082in}}%
\pgfusepath{stroke}%
\end{pgfscope}%
\begin{pgfscope}%
\pgfpathrectangle{\pgfqpoint{0.100000in}{2.413063in}}{\pgfqpoint{5.037500in}{3.427208in}}%
\pgfusepath{clip}%
\pgfsetbuttcap%
\pgfsetroundjoin%
\definecolor{currentfill}{rgb}{0.000000,0.000000,1.000000}%
\pgfsetfillcolor{currentfill}%
\pgfsetfillopacity{0.500000}%
\pgfsetlinewidth{0.250937pt}%
\definecolor{currentstroke}{rgb}{0.000000,0.000000,0.000000}%
\pgfsetstrokecolor{currentstroke}%
\pgfsetstrokeopacity{0.500000}%
\pgfsetdash{}{0pt}%
\pgfsys@defobject{currentmarker}{\pgfqpoint{-0.047222in}{-0.047222in}}{\pgfqpoint{0.047222in}{0.047222in}}{%
\pgfpathmoveto{\pgfqpoint{0.000000in}{-0.047222in}}%
\pgfpathcurveto{\pgfqpoint{0.012523in}{-0.047222in}}{\pgfqpoint{0.024536in}{-0.042247in}}{\pgfqpoint{0.033391in}{-0.033391in}}%
\pgfpathcurveto{\pgfqpoint{0.042247in}{-0.024536in}}{\pgfqpoint{0.047222in}{-0.012523in}}{\pgfqpoint{0.047222in}{0.000000in}}%
\pgfpathcurveto{\pgfqpoint{0.047222in}{0.012523in}}{\pgfqpoint{0.042247in}{0.024536in}}{\pgfqpoint{0.033391in}{0.033391in}}%
\pgfpathcurveto{\pgfqpoint{0.024536in}{0.042247in}}{\pgfqpoint{0.012523in}{0.047222in}}{\pgfqpoint{0.000000in}{0.047222in}}%
\pgfpathcurveto{\pgfqpoint{-0.012523in}{0.047222in}}{\pgfqpoint{-0.024536in}{0.042247in}}{\pgfqpoint{-0.033391in}{0.033391in}}%
\pgfpathcurveto{\pgfqpoint{-0.042247in}{0.024536in}}{\pgfqpoint{-0.047222in}{0.012523in}}{\pgfqpoint{-0.047222in}{0.000000in}}%
\pgfpathcurveto{\pgfqpoint{-0.047222in}{-0.012523in}}{\pgfqpoint{-0.042247in}{-0.024536in}}{\pgfqpoint{-0.033391in}{-0.033391in}}%
\pgfpathcurveto{\pgfqpoint{-0.024536in}{-0.042247in}}{\pgfqpoint{-0.012523in}{-0.047222in}}{\pgfqpoint{0.000000in}{-0.047222in}}%
\pgfpathclose%
\pgfusepath{stroke,fill}%
}%
\begin{pgfscope}%
\pgfsys@transformshift{4.678703in}{4.803082in}%
\pgfsys@useobject{currentmarker}{}%
\end{pgfscope}%
\end{pgfscope}%
\begin{pgfscope}%
\pgfpathrectangle{\pgfqpoint{0.100000in}{2.413063in}}{\pgfqpoint{5.037500in}{3.427208in}}%
\pgfusepath{clip}%
\pgfsetrectcap%
\pgfsetroundjoin%
\pgfsetlinewidth{1.505625pt}%
\definecolor{currentstroke}{rgb}{0.000000,0.000000,1.000000}%
\pgfsetstrokecolor{currentstroke}%
\pgfsetstrokeopacity{0.500000}%
\pgfsetdash{}{0pt}%
\pgfpathmoveto{\pgfqpoint{4.529746in}{4.484780in}}%
\pgfusepath{stroke}%
\end{pgfscope}%
\begin{pgfscope}%
\pgfpathrectangle{\pgfqpoint{0.100000in}{2.413063in}}{\pgfqpoint{5.037500in}{3.427208in}}%
\pgfusepath{clip}%
\pgfsetbuttcap%
\pgfsetroundjoin%
\definecolor{currentfill}{rgb}{0.000000,0.000000,1.000000}%
\pgfsetfillcolor{currentfill}%
\pgfsetfillopacity{0.500000}%
\pgfsetlinewidth{0.250937pt}%
\definecolor{currentstroke}{rgb}{0.000000,0.000000,0.000000}%
\pgfsetstrokecolor{currentstroke}%
\pgfsetstrokeopacity{0.500000}%
\pgfsetdash{}{0pt}%
\pgfsys@defobject{currentmarker}{\pgfqpoint{-0.027778in}{-0.027778in}}{\pgfqpoint{0.027778in}{0.027778in}}{%
\pgfpathmoveto{\pgfqpoint{0.000000in}{-0.027778in}}%
\pgfpathcurveto{\pgfqpoint{0.007367in}{-0.027778in}}{\pgfqpoint{0.014433in}{-0.024851in}}{\pgfqpoint{0.019642in}{-0.019642in}}%
\pgfpathcurveto{\pgfqpoint{0.024851in}{-0.014433in}}{\pgfqpoint{0.027778in}{-0.007367in}}{\pgfqpoint{0.027778in}{0.000000in}}%
\pgfpathcurveto{\pgfqpoint{0.027778in}{0.007367in}}{\pgfqpoint{0.024851in}{0.014433in}}{\pgfqpoint{0.019642in}{0.019642in}}%
\pgfpathcurveto{\pgfqpoint{0.014433in}{0.024851in}}{\pgfqpoint{0.007367in}{0.027778in}}{\pgfqpoint{0.000000in}{0.027778in}}%
\pgfpathcurveto{\pgfqpoint{-0.007367in}{0.027778in}}{\pgfqpoint{-0.014433in}{0.024851in}}{\pgfqpoint{-0.019642in}{0.019642in}}%
\pgfpathcurveto{\pgfqpoint{-0.024851in}{0.014433in}}{\pgfqpoint{-0.027778in}{0.007367in}}{\pgfqpoint{-0.027778in}{0.000000in}}%
\pgfpathcurveto{\pgfqpoint{-0.027778in}{-0.007367in}}{\pgfqpoint{-0.024851in}{-0.014433in}}{\pgfqpoint{-0.019642in}{-0.019642in}}%
\pgfpathcurveto{\pgfqpoint{-0.014433in}{-0.024851in}}{\pgfqpoint{-0.007367in}{-0.027778in}}{\pgfqpoint{0.000000in}{-0.027778in}}%
\pgfpathclose%
\pgfusepath{stroke,fill}%
}%
\begin{pgfscope}%
\pgfsys@transformshift{4.529746in}{4.484780in}%
\pgfsys@useobject{currentmarker}{}%
\end{pgfscope}%
\end{pgfscope}%
\begin{pgfscope}%
\pgfpathrectangle{\pgfqpoint{0.100000in}{2.413063in}}{\pgfqpoint{5.037500in}{3.427208in}}%
\pgfusepath{clip}%
\pgfsetrectcap%
\pgfsetroundjoin%
\pgfsetlinewidth{1.505625pt}%
\definecolor{currentstroke}{rgb}{0.678431,1.000000,0.184314}%
\pgfsetstrokecolor{currentstroke}%
\pgfsetstrokeopacity{0.500000}%
\pgfsetdash{}{0pt}%
\pgfpathmoveto{\pgfqpoint{4.542388in}{4.394206in}}%
\pgfusepath{stroke}%
\end{pgfscope}%
\begin{pgfscope}%
\pgfpathrectangle{\pgfqpoint{0.100000in}{2.413063in}}{\pgfqpoint{5.037500in}{3.427208in}}%
\pgfusepath{clip}%
\pgfsetbuttcap%
\pgfsetroundjoin%
\definecolor{currentfill}{rgb}{0.678431,1.000000,0.184314}%
\pgfsetfillcolor{currentfill}%
\pgfsetfillopacity{0.500000}%
\pgfsetlinewidth{0.250937pt}%
\definecolor{currentstroke}{rgb}{0.000000,0.000000,0.000000}%
\pgfsetstrokecolor{currentstroke}%
\pgfsetstrokeopacity{0.500000}%
\pgfsetdash{}{0pt}%
\pgfsys@defobject{currentmarker}{\pgfqpoint{-0.005556in}{-0.005556in}}{\pgfqpoint{0.005556in}{0.005556in}}{%
\pgfpathmoveto{\pgfqpoint{0.000000in}{-0.005556in}}%
\pgfpathcurveto{\pgfqpoint{0.001473in}{-0.005556in}}{\pgfqpoint{0.002887in}{-0.004970in}}{\pgfqpoint{0.003928in}{-0.003928in}}%
\pgfpathcurveto{\pgfqpoint{0.004970in}{-0.002887in}}{\pgfqpoint{0.005556in}{-0.001473in}}{\pgfqpoint{0.005556in}{0.000000in}}%
\pgfpathcurveto{\pgfqpoint{0.005556in}{0.001473in}}{\pgfqpoint{0.004970in}{0.002887in}}{\pgfqpoint{0.003928in}{0.003928in}}%
\pgfpathcurveto{\pgfqpoint{0.002887in}{0.004970in}}{\pgfqpoint{0.001473in}{0.005556in}}{\pgfqpoint{0.000000in}{0.005556in}}%
\pgfpathcurveto{\pgfqpoint{-0.001473in}{0.005556in}}{\pgfqpoint{-0.002887in}{0.004970in}}{\pgfqpoint{-0.003928in}{0.003928in}}%
\pgfpathcurveto{\pgfqpoint{-0.004970in}{0.002887in}}{\pgfqpoint{-0.005556in}{0.001473in}}{\pgfqpoint{-0.005556in}{0.000000in}}%
\pgfpathcurveto{\pgfqpoint{-0.005556in}{-0.001473in}}{\pgfqpoint{-0.004970in}{-0.002887in}}{\pgfqpoint{-0.003928in}{-0.003928in}}%
\pgfpathcurveto{\pgfqpoint{-0.002887in}{-0.004970in}}{\pgfqpoint{-0.001473in}{-0.005556in}}{\pgfqpoint{0.000000in}{-0.005556in}}%
\pgfpathclose%
\pgfusepath{stroke,fill}%
}%
\begin{pgfscope}%
\pgfsys@transformshift{4.542388in}{4.394206in}%
\pgfsys@useobject{currentmarker}{}%
\end{pgfscope}%
\end{pgfscope}%
\begin{pgfscope}%
\pgfpathrectangle{\pgfqpoint{0.100000in}{2.413063in}}{\pgfqpoint{5.037500in}{3.427208in}}%
\pgfusepath{clip}%
\pgfsetrectcap%
\pgfsetroundjoin%
\pgfsetlinewidth{1.505625pt}%
\definecolor{currentstroke}{rgb}{0.000000,0.000000,1.000000}%
\pgfsetstrokecolor{currentstroke}%
\pgfsetstrokeopacity{0.500000}%
\pgfsetdash{}{0pt}%
\pgfpathmoveto{\pgfqpoint{4.403529in}{4.427445in}}%
\pgfusepath{stroke}%
\end{pgfscope}%
\begin{pgfscope}%
\pgfpathrectangle{\pgfqpoint{0.100000in}{2.413063in}}{\pgfqpoint{5.037500in}{3.427208in}}%
\pgfusepath{clip}%
\pgfsetbuttcap%
\pgfsetroundjoin%
\definecolor{currentfill}{rgb}{0.000000,0.000000,1.000000}%
\pgfsetfillcolor{currentfill}%
\pgfsetfillopacity{0.500000}%
\pgfsetlinewidth{0.250937pt}%
\definecolor{currentstroke}{rgb}{0.000000,0.000000,0.000000}%
\pgfsetstrokecolor{currentstroke}%
\pgfsetstrokeopacity{0.500000}%
\pgfsetdash{}{0pt}%
\pgfsys@defobject{currentmarker}{\pgfqpoint{-0.022222in}{-0.022222in}}{\pgfqpoint{0.022222in}{0.022222in}}{%
\pgfpathmoveto{\pgfqpoint{0.000000in}{-0.022222in}}%
\pgfpathcurveto{\pgfqpoint{0.005893in}{-0.022222in}}{\pgfqpoint{0.011546in}{-0.019881in}}{\pgfqpoint{0.015713in}{-0.015713in}}%
\pgfpathcurveto{\pgfqpoint{0.019881in}{-0.011546in}}{\pgfqpoint{0.022222in}{-0.005893in}}{\pgfqpoint{0.022222in}{0.000000in}}%
\pgfpathcurveto{\pgfqpoint{0.022222in}{0.005893in}}{\pgfqpoint{0.019881in}{0.011546in}}{\pgfqpoint{0.015713in}{0.015713in}}%
\pgfpathcurveto{\pgfqpoint{0.011546in}{0.019881in}}{\pgfqpoint{0.005893in}{0.022222in}}{\pgfqpoint{0.000000in}{0.022222in}}%
\pgfpathcurveto{\pgfqpoint{-0.005893in}{0.022222in}}{\pgfqpoint{-0.011546in}{0.019881in}}{\pgfqpoint{-0.015713in}{0.015713in}}%
\pgfpathcurveto{\pgfqpoint{-0.019881in}{0.011546in}}{\pgfqpoint{-0.022222in}{0.005893in}}{\pgfqpoint{-0.022222in}{0.000000in}}%
\pgfpathcurveto{\pgfqpoint{-0.022222in}{-0.005893in}}{\pgfqpoint{-0.019881in}{-0.011546in}}{\pgfqpoint{-0.015713in}{-0.015713in}}%
\pgfpathcurveto{\pgfqpoint{-0.011546in}{-0.019881in}}{\pgfqpoint{-0.005893in}{-0.022222in}}{\pgfqpoint{0.000000in}{-0.022222in}}%
\pgfpathclose%
\pgfusepath{stroke,fill}%
}%
\begin{pgfscope}%
\pgfsys@transformshift{4.403529in}{4.427445in}%
\pgfsys@useobject{currentmarker}{}%
\end{pgfscope}%
\end{pgfscope}%
\begin{pgfscope}%
\pgfpathrectangle{\pgfqpoint{0.100000in}{2.413063in}}{\pgfqpoint{5.037500in}{3.427208in}}%
\pgfusepath{clip}%
\pgfsetrectcap%
\pgfsetroundjoin%
\pgfsetlinewidth{1.505625pt}%
\definecolor{currentstroke}{rgb}{0.000000,0.000000,1.000000}%
\pgfsetstrokecolor{currentstroke}%
\pgfsetstrokeopacity{0.500000}%
\pgfsetdash{}{0pt}%
\pgfpathmoveto{\pgfqpoint{4.170525in}{2.940820in}}%
\pgfusepath{stroke}%
\end{pgfscope}%
\begin{pgfscope}%
\pgfpathrectangle{\pgfqpoint{0.100000in}{2.413063in}}{\pgfqpoint{5.037500in}{3.427208in}}%
\pgfusepath{clip}%
\pgfsetbuttcap%
\pgfsetroundjoin%
\definecolor{currentfill}{rgb}{0.000000,0.000000,1.000000}%
\pgfsetfillcolor{currentfill}%
\pgfsetfillopacity{0.500000}%
\pgfsetlinewidth{0.250937pt}%
\definecolor{currentstroke}{rgb}{0.000000,0.000000,0.000000}%
\pgfsetstrokecolor{currentstroke}%
\pgfsetstrokeopacity{0.500000}%
\pgfsetdash{}{0pt}%
\pgfsys@defobject{currentmarker}{\pgfqpoint{-0.008333in}{-0.008333in}}{\pgfqpoint{0.008333in}{0.008333in}}{%
\pgfpathmoveto{\pgfqpoint{0.000000in}{-0.008333in}}%
\pgfpathcurveto{\pgfqpoint{0.002210in}{-0.008333in}}{\pgfqpoint{0.004330in}{-0.007455in}}{\pgfqpoint{0.005893in}{-0.005893in}}%
\pgfpathcurveto{\pgfqpoint{0.007455in}{-0.004330in}}{\pgfqpoint{0.008333in}{-0.002210in}}{\pgfqpoint{0.008333in}{0.000000in}}%
\pgfpathcurveto{\pgfqpoint{0.008333in}{0.002210in}}{\pgfqpoint{0.007455in}{0.004330in}}{\pgfqpoint{0.005893in}{0.005893in}}%
\pgfpathcurveto{\pgfqpoint{0.004330in}{0.007455in}}{\pgfqpoint{0.002210in}{0.008333in}}{\pgfqpoint{0.000000in}{0.008333in}}%
\pgfpathcurveto{\pgfqpoint{-0.002210in}{0.008333in}}{\pgfqpoint{-0.004330in}{0.007455in}}{\pgfqpoint{-0.005893in}{0.005893in}}%
\pgfpathcurveto{\pgfqpoint{-0.007455in}{0.004330in}}{\pgfqpoint{-0.008333in}{0.002210in}}{\pgfqpoint{-0.008333in}{0.000000in}}%
\pgfpathcurveto{\pgfqpoint{-0.008333in}{-0.002210in}}{\pgfqpoint{-0.007455in}{-0.004330in}}{\pgfqpoint{-0.005893in}{-0.005893in}}%
\pgfpathcurveto{\pgfqpoint{-0.004330in}{-0.007455in}}{\pgfqpoint{-0.002210in}{-0.008333in}}{\pgfqpoint{0.000000in}{-0.008333in}}%
\pgfpathclose%
\pgfusepath{stroke,fill}%
}%
\begin{pgfscope}%
\pgfsys@transformshift{4.170525in}{2.940820in}%
\pgfsys@useobject{currentmarker}{}%
\end{pgfscope}%
\end{pgfscope}%
\begin{pgfscope}%
\pgfpathrectangle{\pgfqpoint{0.100000in}{2.413063in}}{\pgfqpoint{5.037500in}{3.427208in}}%
\pgfusepath{clip}%
\pgfsetrectcap%
\pgfsetroundjoin%
\pgfsetlinewidth{1.505625pt}%
\definecolor{currentstroke}{rgb}{0.501961,0.501961,0.501961}%
\pgfsetstrokecolor{currentstroke}%
\pgfsetstrokeopacity{0.500000}%
\pgfsetdash{}{0pt}%
\pgfpathmoveto{\pgfqpoint{3.645536in}{3.368139in}}%
\pgfusepath{stroke}%
\end{pgfscope}%
\begin{pgfscope}%
\pgfpathrectangle{\pgfqpoint{0.100000in}{2.413063in}}{\pgfqpoint{5.037500in}{3.427208in}}%
\pgfusepath{clip}%
\pgfsetbuttcap%
\pgfsetroundjoin%
\definecolor{currentfill}{rgb}{0.501961,0.501961,0.501961}%
\pgfsetfillcolor{currentfill}%
\pgfsetfillopacity{0.500000}%
\pgfsetlinewidth{0.250937pt}%
\definecolor{currentstroke}{rgb}{0.000000,0.000000,0.000000}%
\pgfsetstrokecolor{currentstroke}%
\pgfsetstrokeopacity{0.500000}%
\pgfsetdash{}{0pt}%
\pgfsys@defobject{currentmarker}{\pgfqpoint{-0.013889in}{-0.013889in}}{\pgfqpoint{0.013889in}{0.013889in}}{%
\pgfpathmoveto{\pgfqpoint{0.000000in}{-0.013889in}}%
\pgfpathcurveto{\pgfqpoint{0.003683in}{-0.013889in}}{\pgfqpoint{0.007216in}{-0.012425in}}{\pgfqpoint{0.009821in}{-0.009821in}}%
\pgfpathcurveto{\pgfqpoint{0.012425in}{-0.007216in}}{\pgfqpoint{0.013889in}{-0.003683in}}{\pgfqpoint{0.013889in}{0.000000in}}%
\pgfpathcurveto{\pgfqpoint{0.013889in}{0.003683in}}{\pgfqpoint{0.012425in}{0.007216in}}{\pgfqpoint{0.009821in}{0.009821in}}%
\pgfpathcurveto{\pgfqpoint{0.007216in}{0.012425in}}{\pgfqpoint{0.003683in}{0.013889in}}{\pgfqpoint{0.000000in}{0.013889in}}%
\pgfpathcurveto{\pgfqpoint{-0.003683in}{0.013889in}}{\pgfqpoint{-0.007216in}{0.012425in}}{\pgfqpoint{-0.009821in}{0.009821in}}%
\pgfpathcurveto{\pgfqpoint{-0.012425in}{0.007216in}}{\pgfqpoint{-0.013889in}{0.003683in}}{\pgfqpoint{-0.013889in}{0.000000in}}%
\pgfpathcurveto{\pgfqpoint{-0.013889in}{-0.003683in}}{\pgfqpoint{-0.012425in}{-0.007216in}}{\pgfqpoint{-0.009821in}{-0.009821in}}%
\pgfpathcurveto{\pgfqpoint{-0.007216in}{-0.012425in}}{\pgfqpoint{-0.003683in}{-0.013889in}}{\pgfqpoint{0.000000in}{-0.013889in}}%
\pgfpathclose%
\pgfusepath{stroke,fill}%
}%
\begin{pgfscope}%
\pgfsys@transformshift{3.645536in}{3.368139in}%
\pgfsys@useobject{currentmarker}{}%
\end{pgfscope}%
\end{pgfscope}%
\begin{pgfscope}%
\pgfpathrectangle{\pgfqpoint{0.100000in}{2.413063in}}{\pgfqpoint{5.037500in}{3.427208in}}%
\pgfusepath{clip}%
\pgfsetrectcap%
\pgfsetroundjoin%
\pgfsetlinewidth{1.505625pt}%
\definecolor{currentstroke}{rgb}{0.501961,0.501961,0.501961}%
\pgfsetstrokecolor{currentstroke}%
\pgfsetstrokeopacity{0.500000}%
\pgfsetdash{}{0pt}%
\pgfpathmoveto{\pgfqpoint{3.644545in}{3.326395in}}%
\pgfusepath{stroke}%
\end{pgfscope}%
\begin{pgfscope}%
\pgfpathrectangle{\pgfqpoint{0.100000in}{2.413063in}}{\pgfqpoint{5.037500in}{3.427208in}}%
\pgfusepath{clip}%
\pgfsetbuttcap%
\pgfsetroundjoin%
\definecolor{currentfill}{rgb}{0.501961,0.501961,0.501961}%
\pgfsetfillcolor{currentfill}%
\pgfsetfillopacity{0.500000}%
\pgfsetlinewidth{0.250937pt}%
\definecolor{currentstroke}{rgb}{0.000000,0.000000,0.000000}%
\pgfsetstrokecolor{currentstroke}%
\pgfsetstrokeopacity{0.500000}%
\pgfsetdash{}{0pt}%
\pgfsys@defobject{currentmarker}{\pgfqpoint{-0.013889in}{-0.013889in}}{\pgfqpoint{0.013889in}{0.013889in}}{%
\pgfpathmoveto{\pgfqpoint{0.000000in}{-0.013889in}}%
\pgfpathcurveto{\pgfqpoint{0.003683in}{-0.013889in}}{\pgfqpoint{0.007216in}{-0.012425in}}{\pgfqpoint{0.009821in}{-0.009821in}}%
\pgfpathcurveto{\pgfqpoint{0.012425in}{-0.007216in}}{\pgfqpoint{0.013889in}{-0.003683in}}{\pgfqpoint{0.013889in}{0.000000in}}%
\pgfpathcurveto{\pgfqpoint{0.013889in}{0.003683in}}{\pgfqpoint{0.012425in}{0.007216in}}{\pgfqpoint{0.009821in}{0.009821in}}%
\pgfpathcurveto{\pgfqpoint{0.007216in}{0.012425in}}{\pgfqpoint{0.003683in}{0.013889in}}{\pgfqpoint{0.000000in}{0.013889in}}%
\pgfpathcurveto{\pgfqpoint{-0.003683in}{0.013889in}}{\pgfqpoint{-0.007216in}{0.012425in}}{\pgfqpoint{-0.009821in}{0.009821in}}%
\pgfpathcurveto{\pgfqpoint{-0.012425in}{0.007216in}}{\pgfqpoint{-0.013889in}{0.003683in}}{\pgfqpoint{-0.013889in}{0.000000in}}%
\pgfpathcurveto{\pgfqpoint{-0.013889in}{-0.003683in}}{\pgfqpoint{-0.012425in}{-0.007216in}}{\pgfqpoint{-0.009821in}{-0.009821in}}%
\pgfpathcurveto{\pgfqpoint{-0.007216in}{-0.012425in}}{\pgfqpoint{-0.003683in}{-0.013889in}}{\pgfqpoint{0.000000in}{-0.013889in}}%
\pgfpathclose%
\pgfusepath{stroke,fill}%
}%
\begin{pgfscope}%
\pgfsys@transformshift{3.644545in}{3.326395in}%
\pgfsys@useobject{currentmarker}{}%
\end{pgfscope}%
\end{pgfscope}%
\begin{pgfscope}%
\pgfpathrectangle{\pgfqpoint{0.100000in}{2.413063in}}{\pgfqpoint{5.037500in}{3.427208in}}%
\pgfusepath{clip}%
\pgfsetrectcap%
\pgfsetroundjoin%
\pgfsetlinewidth{1.505625pt}%
\definecolor{currentstroke}{rgb}{0.000000,0.000000,1.000000}%
\pgfsetstrokecolor{currentstroke}%
\pgfsetstrokeopacity{0.500000}%
\pgfsetdash{}{0pt}%
\pgfpathmoveto{\pgfqpoint{4.204682in}{3.218061in}}%
\pgfusepath{stroke}%
\end{pgfscope}%
\begin{pgfscope}%
\pgfpathrectangle{\pgfqpoint{0.100000in}{2.413063in}}{\pgfqpoint{5.037500in}{3.427208in}}%
\pgfusepath{clip}%
\pgfsetbuttcap%
\pgfsetroundjoin%
\definecolor{currentfill}{rgb}{0.000000,0.000000,1.000000}%
\pgfsetfillcolor{currentfill}%
\pgfsetfillopacity{0.500000}%
\pgfsetlinewidth{0.250937pt}%
\definecolor{currentstroke}{rgb}{0.000000,0.000000,0.000000}%
\pgfsetstrokecolor{currentstroke}%
\pgfsetstrokeopacity{0.500000}%
\pgfsetdash{}{0pt}%
\pgfsys@defobject{currentmarker}{\pgfqpoint{-0.008333in}{-0.008333in}}{\pgfqpoint{0.008333in}{0.008333in}}{%
\pgfpathmoveto{\pgfqpoint{0.000000in}{-0.008333in}}%
\pgfpathcurveto{\pgfqpoint{0.002210in}{-0.008333in}}{\pgfqpoint{0.004330in}{-0.007455in}}{\pgfqpoint{0.005893in}{-0.005893in}}%
\pgfpathcurveto{\pgfqpoint{0.007455in}{-0.004330in}}{\pgfqpoint{0.008333in}{-0.002210in}}{\pgfqpoint{0.008333in}{0.000000in}}%
\pgfpathcurveto{\pgfqpoint{0.008333in}{0.002210in}}{\pgfqpoint{0.007455in}{0.004330in}}{\pgfqpoint{0.005893in}{0.005893in}}%
\pgfpathcurveto{\pgfqpoint{0.004330in}{0.007455in}}{\pgfqpoint{0.002210in}{0.008333in}}{\pgfqpoint{0.000000in}{0.008333in}}%
\pgfpathcurveto{\pgfqpoint{-0.002210in}{0.008333in}}{\pgfqpoint{-0.004330in}{0.007455in}}{\pgfqpoint{-0.005893in}{0.005893in}}%
\pgfpathcurveto{\pgfqpoint{-0.007455in}{0.004330in}}{\pgfqpoint{-0.008333in}{0.002210in}}{\pgfqpoint{-0.008333in}{0.000000in}}%
\pgfpathcurveto{\pgfqpoint{-0.008333in}{-0.002210in}}{\pgfqpoint{-0.007455in}{-0.004330in}}{\pgfqpoint{-0.005893in}{-0.005893in}}%
\pgfpathcurveto{\pgfqpoint{-0.004330in}{-0.007455in}}{\pgfqpoint{-0.002210in}{-0.008333in}}{\pgfqpoint{0.000000in}{-0.008333in}}%
\pgfpathclose%
\pgfusepath{stroke,fill}%
}%
\begin{pgfscope}%
\pgfsys@transformshift{4.204682in}{3.218061in}%
\pgfsys@useobject{currentmarker}{}%
\end{pgfscope}%
\end{pgfscope}%
\begin{pgfscope}%
\pgfpathrectangle{\pgfqpoint{0.100000in}{2.413063in}}{\pgfqpoint{5.037500in}{3.427208in}}%
\pgfusepath{clip}%
\pgfsetrectcap%
\pgfsetroundjoin%
\pgfsetlinewidth{1.505625pt}%
\definecolor{currentstroke}{rgb}{0.000000,0.000000,1.000000}%
\pgfsetstrokecolor{currentstroke}%
\pgfsetstrokeopacity{0.500000}%
\pgfsetdash{}{0pt}%
\pgfpathmoveto{\pgfqpoint{4.085003in}{3.289230in}}%
\pgfusepath{stroke}%
\end{pgfscope}%
\begin{pgfscope}%
\pgfpathrectangle{\pgfqpoint{0.100000in}{2.413063in}}{\pgfqpoint{5.037500in}{3.427208in}}%
\pgfusepath{clip}%
\pgfsetbuttcap%
\pgfsetroundjoin%
\definecolor{currentfill}{rgb}{0.000000,0.000000,1.000000}%
\pgfsetfillcolor{currentfill}%
\pgfsetfillopacity{0.500000}%
\pgfsetlinewidth{0.250937pt}%
\definecolor{currentstroke}{rgb}{0.000000,0.000000,0.000000}%
\pgfsetstrokecolor{currentstroke}%
\pgfsetstrokeopacity{0.500000}%
\pgfsetdash{}{0pt}%
\pgfsys@defobject{currentmarker}{\pgfqpoint{-0.008333in}{-0.008333in}}{\pgfqpoint{0.008333in}{0.008333in}}{%
\pgfpathmoveto{\pgfqpoint{0.000000in}{-0.008333in}}%
\pgfpathcurveto{\pgfqpoint{0.002210in}{-0.008333in}}{\pgfqpoint{0.004330in}{-0.007455in}}{\pgfqpoint{0.005893in}{-0.005893in}}%
\pgfpathcurveto{\pgfqpoint{0.007455in}{-0.004330in}}{\pgfqpoint{0.008333in}{-0.002210in}}{\pgfqpoint{0.008333in}{0.000000in}}%
\pgfpathcurveto{\pgfqpoint{0.008333in}{0.002210in}}{\pgfqpoint{0.007455in}{0.004330in}}{\pgfqpoint{0.005893in}{0.005893in}}%
\pgfpathcurveto{\pgfqpoint{0.004330in}{0.007455in}}{\pgfqpoint{0.002210in}{0.008333in}}{\pgfqpoint{0.000000in}{0.008333in}}%
\pgfpathcurveto{\pgfqpoint{-0.002210in}{0.008333in}}{\pgfqpoint{-0.004330in}{0.007455in}}{\pgfqpoint{-0.005893in}{0.005893in}}%
\pgfpathcurveto{\pgfqpoint{-0.007455in}{0.004330in}}{\pgfqpoint{-0.008333in}{0.002210in}}{\pgfqpoint{-0.008333in}{0.000000in}}%
\pgfpathcurveto{\pgfqpoint{-0.008333in}{-0.002210in}}{\pgfqpoint{-0.007455in}{-0.004330in}}{\pgfqpoint{-0.005893in}{-0.005893in}}%
\pgfpathcurveto{\pgfqpoint{-0.004330in}{-0.007455in}}{\pgfqpoint{-0.002210in}{-0.008333in}}{\pgfqpoint{0.000000in}{-0.008333in}}%
\pgfpathclose%
\pgfusepath{stroke,fill}%
}%
\begin{pgfscope}%
\pgfsys@transformshift{4.085003in}{3.289230in}%
\pgfsys@useobject{currentmarker}{}%
\end{pgfscope}%
\end{pgfscope}%
\begin{pgfscope}%
\pgfpathrectangle{\pgfqpoint{0.100000in}{2.413063in}}{\pgfqpoint{5.037500in}{3.427208in}}%
\pgfusepath{clip}%
\pgfsetrectcap%
\pgfsetroundjoin%
\pgfsetlinewidth{1.505625pt}%
\definecolor{currentstroke}{rgb}{0.501961,0.501961,0.501961}%
\pgfsetstrokecolor{currentstroke}%
\pgfsetstrokeopacity{0.500000}%
\pgfsetdash{}{0pt}%
\pgfpathmoveto{\pgfqpoint{4.073308in}{3.187567in}}%
\pgfusepath{stroke}%
\end{pgfscope}%
\begin{pgfscope}%
\pgfpathrectangle{\pgfqpoint{0.100000in}{2.413063in}}{\pgfqpoint{5.037500in}{3.427208in}}%
\pgfusepath{clip}%
\pgfsetbuttcap%
\pgfsetroundjoin%
\definecolor{currentfill}{rgb}{0.501961,0.501961,0.501961}%
\pgfsetfillcolor{currentfill}%
\pgfsetfillopacity{0.500000}%
\pgfsetlinewidth{0.250937pt}%
\definecolor{currentstroke}{rgb}{0.000000,0.000000,0.000000}%
\pgfsetstrokecolor{currentstroke}%
\pgfsetstrokeopacity{0.500000}%
\pgfsetdash{}{0pt}%
\pgfsys@defobject{currentmarker}{\pgfqpoint{-0.013889in}{-0.013889in}}{\pgfqpoint{0.013889in}{0.013889in}}{%
\pgfpathmoveto{\pgfqpoint{0.000000in}{-0.013889in}}%
\pgfpathcurveto{\pgfqpoint{0.003683in}{-0.013889in}}{\pgfqpoint{0.007216in}{-0.012425in}}{\pgfqpoint{0.009821in}{-0.009821in}}%
\pgfpathcurveto{\pgfqpoint{0.012425in}{-0.007216in}}{\pgfqpoint{0.013889in}{-0.003683in}}{\pgfqpoint{0.013889in}{0.000000in}}%
\pgfpathcurveto{\pgfqpoint{0.013889in}{0.003683in}}{\pgfqpoint{0.012425in}{0.007216in}}{\pgfqpoint{0.009821in}{0.009821in}}%
\pgfpathcurveto{\pgfqpoint{0.007216in}{0.012425in}}{\pgfqpoint{0.003683in}{0.013889in}}{\pgfqpoint{0.000000in}{0.013889in}}%
\pgfpathcurveto{\pgfqpoint{-0.003683in}{0.013889in}}{\pgfqpoint{-0.007216in}{0.012425in}}{\pgfqpoint{-0.009821in}{0.009821in}}%
\pgfpathcurveto{\pgfqpoint{-0.012425in}{0.007216in}}{\pgfqpoint{-0.013889in}{0.003683in}}{\pgfqpoint{-0.013889in}{0.000000in}}%
\pgfpathcurveto{\pgfqpoint{-0.013889in}{-0.003683in}}{\pgfqpoint{-0.012425in}{-0.007216in}}{\pgfqpoint{-0.009821in}{-0.009821in}}%
\pgfpathcurveto{\pgfqpoint{-0.007216in}{-0.012425in}}{\pgfqpoint{-0.003683in}{-0.013889in}}{\pgfqpoint{0.000000in}{-0.013889in}}%
\pgfpathclose%
\pgfusepath{stroke,fill}%
}%
\begin{pgfscope}%
\pgfsys@transformshift{4.073308in}{3.187567in}%
\pgfsys@useobject{currentmarker}{}%
\end{pgfscope}%
\end{pgfscope}%
\begin{pgfscope}%
\pgfpathrectangle{\pgfqpoint{0.100000in}{2.413063in}}{\pgfqpoint{5.037500in}{3.427208in}}%
\pgfusepath{clip}%
\pgfsetrectcap%
\pgfsetroundjoin%
\pgfsetlinewidth{1.505625pt}%
\definecolor{currentstroke}{rgb}{0.000000,0.000000,1.000000}%
\pgfsetstrokecolor{currentstroke}%
\pgfsetstrokeopacity{0.500000}%
\pgfsetdash{}{0pt}%
\pgfpathmoveto{\pgfqpoint{4.140568in}{3.377345in}}%
\pgfusepath{stroke}%
\end{pgfscope}%
\begin{pgfscope}%
\pgfpathrectangle{\pgfqpoint{0.100000in}{2.413063in}}{\pgfqpoint{5.037500in}{3.427208in}}%
\pgfusepath{clip}%
\pgfsetbuttcap%
\pgfsetroundjoin%
\definecolor{currentfill}{rgb}{0.000000,0.000000,1.000000}%
\pgfsetfillcolor{currentfill}%
\pgfsetfillopacity{0.500000}%
\pgfsetlinewidth{0.250937pt}%
\definecolor{currentstroke}{rgb}{0.000000,0.000000,0.000000}%
\pgfsetstrokecolor{currentstroke}%
\pgfsetstrokeopacity{0.500000}%
\pgfsetdash{}{0pt}%
\pgfsys@defobject{currentmarker}{\pgfqpoint{-0.005556in}{-0.005556in}}{\pgfqpoint{0.005556in}{0.005556in}}{%
\pgfpathmoveto{\pgfqpoint{0.000000in}{-0.005556in}}%
\pgfpathcurveto{\pgfqpoint{0.001473in}{-0.005556in}}{\pgfqpoint{0.002887in}{-0.004970in}}{\pgfqpoint{0.003928in}{-0.003928in}}%
\pgfpathcurveto{\pgfqpoint{0.004970in}{-0.002887in}}{\pgfqpoint{0.005556in}{-0.001473in}}{\pgfqpoint{0.005556in}{0.000000in}}%
\pgfpathcurveto{\pgfqpoint{0.005556in}{0.001473in}}{\pgfqpoint{0.004970in}{0.002887in}}{\pgfqpoint{0.003928in}{0.003928in}}%
\pgfpathcurveto{\pgfqpoint{0.002887in}{0.004970in}}{\pgfqpoint{0.001473in}{0.005556in}}{\pgfqpoint{0.000000in}{0.005556in}}%
\pgfpathcurveto{\pgfqpoint{-0.001473in}{0.005556in}}{\pgfqpoint{-0.002887in}{0.004970in}}{\pgfqpoint{-0.003928in}{0.003928in}}%
\pgfpathcurveto{\pgfqpoint{-0.004970in}{0.002887in}}{\pgfqpoint{-0.005556in}{0.001473in}}{\pgfqpoint{-0.005556in}{0.000000in}}%
\pgfpathcurveto{\pgfqpoint{-0.005556in}{-0.001473in}}{\pgfqpoint{-0.004970in}{-0.002887in}}{\pgfqpoint{-0.003928in}{-0.003928in}}%
\pgfpathcurveto{\pgfqpoint{-0.002887in}{-0.004970in}}{\pgfqpoint{-0.001473in}{-0.005556in}}{\pgfqpoint{0.000000in}{-0.005556in}}%
\pgfpathclose%
\pgfusepath{stroke,fill}%
}%
\begin{pgfscope}%
\pgfsys@transformshift{4.140568in}{3.377345in}%
\pgfsys@useobject{currentmarker}{}%
\end{pgfscope}%
\end{pgfscope}%
\begin{pgfscope}%
\pgfpathrectangle{\pgfqpoint{0.100000in}{2.413063in}}{\pgfqpoint{5.037500in}{3.427208in}}%
\pgfusepath{clip}%
\pgfsetrectcap%
\pgfsetroundjoin%
\pgfsetlinewidth{1.505625pt}%
\definecolor{currentstroke}{rgb}{0.000000,0.000000,1.000000}%
\pgfsetstrokecolor{currentstroke}%
\pgfsetstrokeopacity{0.500000}%
\pgfsetdash{}{0pt}%
\pgfpathmoveto{\pgfqpoint{4.149644in}{3.107988in}}%
\pgfusepath{stroke}%
\end{pgfscope}%
\begin{pgfscope}%
\pgfpathrectangle{\pgfqpoint{0.100000in}{2.413063in}}{\pgfqpoint{5.037500in}{3.427208in}}%
\pgfusepath{clip}%
\pgfsetbuttcap%
\pgfsetroundjoin%
\definecolor{currentfill}{rgb}{0.000000,0.000000,1.000000}%
\pgfsetfillcolor{currentfill}%
\pgfsetfillopacity{0.500000}%
\pgfsetlinewidth{0.250937pt}%
\definecolor{currentstroke}{rgb}{0.000000,0.000000,0.000000}%
\pgfsetstrokecolor{currentstroke}%
\pgfsetstrokeopacity{0.500000}%
\pgfsetdash{}{0pt}%
\pgfsys@defobject{currentmarker}{\pgfqpoint{-0.016667in}{-0.016667in}}{\pgfqpoint{0.016667in}{0.016667in}}{%
\pgfpathmoveto{\pgfqpoint{0.000000in}{-0.016667in}}%
\pgfpathcurveto{\pgfqpoint{0.004420in}{-0.016667in}}{\pgfqpoint{0.008660in}{-0.014911in}}{\pgfqpoint{0.011785in}{-0.011785in}}%
\pgfpathcurveto{\pgfqpoint{0.014911in}{-0.008660in}}{\pgfqpoint{0.016667in}{-0.004420in}}{\pgfqpoint{0.016667in}{0.000000in}}%
\pgfpathcurveto{\pgfqpoint{0.016667in}{0.004420in}}{\pgfqpoint{0.014911in}{0.008660in}}{\pgfqpoint{0.011785in}{0.011785in}}%
\pgfpathcurveto{\pgfqpoint{0.008660in}{0.014911in}}{\pgfqpoint{0.004420in}{0.016667in}}{\pgfqpoint{0.000000in}{0.016667in}}%
\pgfpathcurveto{\pgfqpoint{-0.004420in}{0.016667in}}{\pgfqpoint{-0.008660in}{0.014911in}}{\pgfqpoint{-0.011785in}{0.011785in}}%
\pgfpathcurveto{\pgfqpoint{-0.014911in}{0.008660in}}{\pgfqpoint{-0.016667in}{0.004420in}}{\pgfqpoint{-0.016667in}{0.000000in}}%
\pgfpathcurveto{\pgfqpoint{-0.016667in}{-0.004420in}}{\pgfqpoint{-0.014911in}{-0.008660in}}{\pgfqpoint{-0.011785in}{-0.011785in}}%
\pgfpathcurveto{\pgfqpoint{-0.008660in}{-0.014911in}}{\pgfqpoint{-0.004420in}{-0.016667in}}{\pgfqpoint{0.000000in}{-0.016667in}}%
\pgfpathclose%
\pgfusepath{stroke,fill}%
}%
\begin{pgfscope}%
\pgfsys@transformshift{4.149644in}{3.107988in}%
\pgfsys@useobject{currentmarker}{}%
\end{pgfscope}%
\end{pgfscope}%
\begin{pgfscope}%
\pgfpathrectangle{\pgfqpoint{0.100000in}{2.413063in}}{\pgfqpoint{5.037500in}{3.427208in}}%
\pgfusepath{clip}%
\pgfsetrectcap%
\pgfsetroundjoin%
\pgfsetlinewidth{1.505625pt}%
\definecolor{currentstroke}{rgb}{0.000000,0.000000,1.000000}%
\pgfsetstrokecolor{currentstroke}%
\pgfsetstrokeopacity{0.500000}%
\pgfsetdash{}{0pt}%
\pgfpathmoveto{\pgfqpoint{4.372433in}{2.872717in}}%
\pgfusepath{stroke}%
\end{pgfscope}%
\begin{pgfscope}%
\pgfpathrectangle{\pgfqpoint{0.100000in}{2.413063in}}{\pgfqpoint{5.037500in}{3.427208in}}%
\pgfusepath{clip}%
\pgfsetbuttcap%
\pgfsetroundjoin%
\definecolor{currentfill}{rgb}{0.000000,0.000000,1.000000}%
\pgfsetfillcolor{currentfill}%
\pgfsetfillopacity{0.500000}%
\pgfsetlinewidth{0.250937pt}%
\definecolor{currentstroke}{rgb}{0.000000,0.000000,0.000000}%
\pgfsetstrokecolor{currentstroke}%
\pgfsetstrokeopacity{0.500000}%
\pgfsetdash{}{0pt}%
\pgfsys@defobject{currentmarker}{\pgfqpoint{-0.005556in}{-0.005556in}}{\pgfqpoint{0.005556in}{0.005556in}}{%
\pgfpathmoveto{\pgfqpoint{0.000000in}{-0.005556in}}%
\pgfpathcurveto{\pgfqpoint{0.001473in}{-0.005556in}}{\pgfqpoint{0.002887in}{-0.004970in}}{\pgfqpoint{0.003928in}{-0.003928in}}%
\pgfpathcurveto{\pgfqpoint{0.004970in}{-0.002887in}}{\pgfqpoint{0.005556in}{-0.001473in}}{\pgfqpoint{0.005556in}{0.000000in}}%
\pgfpathcurveto{\pgfqpoint{0.005556in}{0.001473in}}{\pgfqpoint{0.004970in}{0.002887in}}{\pgfqpoint{0.003928in}{0.003928in}}%
\pgfpathcurveto{\pgfqpoint{0.002887in}{0.004970in}}{\pgfqpoint{0.001473in}{0.005556in}}{\pgfqpoint{0.000000in}{0.005556in}}%
\pgfpathcurveto{\pgfqpoint{-0.001473in}{0.005556in}}{\pgfqpoint{-0.002887in}{0.004970in}}{\pgfqpoint{-0.003928in}{0.003928in}}%
\pgfpathcurveto{\pgfqpoint{-0.004970in}{0.002887in}}{\pgfqpoint{-0.005556in}{0.001473in}}{\pgfqpoint{-0.005556in}{0.000000in}}%
\pgfpathcurveto{\pgfqpoint{-0.005556in}{-0.001473in}}{\pgfqpoint{-0.004970in}{-0.002887in}}{\pgfqpoint{-0.003928in}{-0.003928in}}%
\pgfpathcurveto{\pgfqpoint{-0.002887in}{-0.004970in}}{\pgfqpoint{-0.001473in}{-0.005556in}}{\pgfqpoint{0.000000in}{-0.005556in}}%
\pgfpathclose%
\pgfusepath{stroke,fill}%
}%
\begin{pgfscope}%
\pgfsys@transformshift{4.372433in}{2.872717in}%
\pgfsys@useobject{currentmarker}{}%
\end{pgfscope}%
\end{pgfscope}%
\begin{pgfscope}%
\pgfpathrectangle{\pgfqpoint{0.100000in}{2.413063in}}{\pgfqpoint{5.037500in}{3.427208in}}%
\pgfusepath{clip}%
\pgfsetrectcap%
\pgfsetroundjoin%
\pgfsetlinewidth{1.505625pt}%
\definecolor{currentstroke}{rgb}{0.501961,0.501961,0.501961}%
\pgfsetstrokecolor{currentstroke}%
\pgfsetstrokeopacity{0.500000}%
\pgfsetdash{}{0pt}%
\pgfpathmoveto{\pgfqpoint{4.197819in}{2.889530in}}%
\pgfusepath{stroke}%
\end{pgfscope}%
\begin{pgfscope}%
\pgfpathrectangle{\pgfqpoint{0.100000in}{2.413063in}}{\pgfqpoint{5.037500in}{3.427208in}}%
\pgfusepath{clip}%
\pgfsetbuttcap%
\pgfsetroundjoin%
\definecolor{currentfill}{rgb}{0.501961,0.501961,0.501961}%
\pgfsetfillcolor{currentfill}%
\pgfsetfillopacity{0.500000}%
\pgfsetlinewidth{0.250937pt}%
\definecolor{currentstroke}{rgb}{0.000000,0.000000,0.000000}%
\pgfsetstrokecolor{currentstroke}%
\pgfsetstrokeopacity{0.500000}%
\pgfsetdash{}{0pt}%
\pgfsys@defobject{currentmarker}{\pgfqpoint{-0.013889in}{-0.013889in}}{\pgfqpoint{0.013889in}{0.013889in}}{%
\pgfpathmoveto{\pgfqpoint{0.000000in}{-0.013889in}}%
\pgfpathcurveto{\pgfqpoint{0.003683in}{-0.013889in}}{\pgfqpoint{0.007216in}{-0.012425in}}{\pgfqpoint{0.009821in}{-0.009821in}}%
\pgfpathcurveto{\pgfqpoint{0.012425in}{-0.007216in}}{\pgfqpoint{0.013889in}{-0.003683in}}{\pgfqpoint{0.013889in}{0.000000in}}%
\pgfpathcurveto{\pgfqpoint{0.013889in}{0.003683in}}{\pgfqpoint{0.012425in}{0.007216in}}{\pgfqpoint{0.009821in}{0.009821in}}%
\pgfpathcurveto{\pgfqpoint{0.007216in}{0.012425in}}{\pgfqpoint{0.003683in}{0.013889in}}{\pgfqpoint{0.000000in}{0.013889in}}%
\pgfpathcurveto{\pgfqpoint{-0.003683in}{0.013889in}}{\pgfqpoint{-0.007216in}{0.012425in}}{\pgfqpoint{-0.009821in}{0.009821in}}%
\pgfpathcurveto{\pgfqpoint{-0.012425in}{0.007216in}}{\pgfqpoint{-0.013889in}{0.003683in}}{\pgfqpoint{-0.013889in}{0.000000in}}%
\pgfpathcurveto{\pgfqpoint{-0.013889in}{-0.003683in}}{\pgfqpoint{-0.012425in}{-0.007216in}}{\pgfqpoint{-0.009821in}{-0.009821in}}%
\pgfpathcurveto{\pgfqpoint{-0.007216in}{-0.012425in}}{\pgfqpoint{-0.003683in}{-0.013889in}}{\pgfqpoint{0.000000in}{-0.013889in}}%
\pgfpathclose%
\pgfusepath{stroke,fill}%
}%
\begin{pgfscope}%
\pgfsys@transformshift{4.197819in}{2.889530in}%
\pgfsys@useobject{currentmarker}{}%
\end{pgfscope}%
\end{pgfscope}%
\begin{pgfscope}%
\pgfpathrectangle{\pgfqpoint{0.100000in}{2.413063in}}{\pgfqpoint{5.037500in}{3.427208in}}%
\pgfusepath{clip}%
\pgfsetrectcap%
\pgfsetroundjoin%
\pgfsetlinewidth{1.505625pt}%
\definecolor{currentstroke}{rgb}{0.501961,0.501961,0.501961}%
\pgfsetstrokecolor{currentstroke}%
\pgfsetstrokeopacity{0.500000}%
\pgfsetdash{}{0pt}%
\pgfpathmoveto{\pgfqpoint{4.136698in}{2.988100in}}%
\pgfusepath{stroke}%
\end{pgfscope}%
\begin{pgfscope}%
\pgfpathrectangle{\pgfqpoint{0.100000in}{2.413063in}}{\pgfqpoint{5.037500in}{3.427208in}}%
\pgfusepath{clip}%
\pgfsetbuttcap%
\pgfsetroundjoin%
\definecolor{currentfill}{rgb}{0.501961,0.501961,0.501961}%
\pgfsetfillcolor{currentfill}%
\pgfsetfillopacity{0.500000}%
\pgfsetlinewidth{0.250937pt}%
\definecolor{currentstroke}{rgb}{0.000000,0.000000,0.000000}%
\pgfsetstrokecolor{currentstroke}%
\pgfsetstrokeopacity{0.500000}%
\pgfsetdash{}{0pt}%
\pgfsys@defobject{currentmarker}{\pgfqpoint{-0.013889in}{-0.013889in}}{\pgfqpoint{0.013889in}{0.013889in}}{%
\pgfpathmoveto{\pgfqpoint{0.000000in}{-0.013889in}}%
\pgfpathcurveto{\pgfqpoint{0.003683in}{-0.013889in}}{\pgfqpoint{0.007216in}{-0.012425in}}{\pgfqpoint{0.009821in}{-0.009821in}}%
\pgfpathcurveto{\pgfqpoint{0.012425in}{-0.007216in}}{\pgfqpoint{0.013889in}{-0.003683in}}{\pgfqpoint{0.013889in}{0.000000in}}%
\pgfpathcurveto{\pgfqpoint{0.013889in}{0.003683in}}{\pgfqpoint{0.012425in}{0.007216in}}{\pgfqpoint{0.009821in}{0.009821in}}%
\pgfpathcurveto{\pgfqpoint{0.007216in}{0.012425in}}{\pgfqpoint{0.003683in}{0.013889in}}{\pgfqpoint{0.000000in}{0.013889in}}%
\pgfpathcurveto{\pgfqpoint{-0.003683in}{0.013889in}}{\pgfqpoint{-0.007216in}{0.012425in}}{\pgfqpoint{-0.009821in}{0.009821in}}%
\pgfpathcurveto{\pgfqpoint{-0.012425in}{0.007216in}}{\pgfqpoint{-0.013889in}{0.003683in}}{\pgfqpoint{-0.013889in}{0.000000in}}%
\pgfpathcurveto{\pgfqpoint{-0.013889in}{-0.003683in}}{\pgfqpoint{-0.012425in}{-0.007216in}}{\pgfqpoint{-0.009821in}{-0.009821in}}%
\pgfpathcurveto{\pgfqpoint{-0.007216in}{-0.012425in}}{\pgfqpoint{-0.003683in}{-0.013889in}}{\pgfqpoint{0.000000in}{-0.013889in}}%
\pgfpathclose%
\pgfusepath{stroke,fill}%
}%
\begin{pgfscope}%
\pgfsys@transformshift{4.136698in}{2.988100in}%
\pgfsys@useobject{currentmarker}{}%
\end{pgfscope}%
\end{pgfscope}%
\begin{pgfscope}%
\pgfpathrectangle{\pgfqpoint{0.100000in}{2.413063in}}{\pgfqpoint{5.037500in}{3.427208in}}%
\pgfusepath{clip}%
\pgfsetrectcap%
\pgfsetroundjoin%
\pgfsetlinewidth{1.505625pt}%
\definecolor{currentstroke}{rgb}{0.000000,0.000000,1.000000}%
\pgfsetstrokecolor{currentstroke}%
\pgfsetstrokeopacity{0.500000}%
\pgfsetdash{}{0pt}%
\pgfpathmoveto{\pgfqpoint{4.111216in}{3.238146in}}%
\pgfusepath{stroke}%
\end{pgfscope}%
\begin{pgfscope}%
\pgfpathrectangle{\pgfqpoint{0.100000in}{2.413063in}}{\pgfqpoint{5.037500in}{3.427208in}}%
\pgfusepath{clip}%
\pgfsetbuttcap%
\pgfsetroundjoin%
\definecolor{currentfill}{rgb}{0.000000,0.000000,1.000000}%
\pgfsetfillcolor{currentfill}%
\pgfsetfillopacity{0.500000}%
\pgfsetlinewidth{0.250937pt}%
\definecolor{currentstroke}{rgb}{0.000000,0.000000,0.000000}%
\pgfsetstrokecolor{currentstroke}%
\pgfsetstrokeopacity{0.500000}%
\pgfsetdash{}{0pt}%
\pgfsys@defobject{currentmarker}{\pgfqpoint{-0.008333in}{-0.008333in}}{\pgfqpoint{0.008333in}{0.008333in}}{%
\pgfpathmoveto{\pgfqpoint{0.000000in}{-0.008333in}}%
\pgfpathcurveto{\pgfqpoint{0.002210in}{-0.008333in}}{\pgfqpoint{0.004330in}{-0.007455in}}{\pgfqpoint{0.005893in}{-0.005893in}}%
\pgfpathcurveto{\pgfqpoint{0.007455in}{-0.004330in}}{\pgfqpoint{0.008333in}{-0.002210in}}{\pgfqpoint{0.008333in}{0.000000in}}%
\pgfpathcurveto{\pgfqpoint{0.008333in}{0.002210in}}{\pgfqpoint{0.007455in}{0.004330in}}{\pgfqpoint{0.005893in}{0.005893in}}%
\pgfpathcurveto{\pgfqpoint{0.004330in}{0.007455in}}{\pgfqpoint{0.002210in}{0.008333in}}{\pgfqpoint{0.000000in}{0.008333in}}%
\pgfpathcurveto{\pgfqpoint{-0.002210in}{0.008333in}}{\pgfqpoint{-0.004330in}{0.007455in}}{\pgfqpoint{-0.005893in}{0.005893in}}%
\pgfpathcurveto{\pgfqpoint{-0.007455in}{0.004330in}}{\pgfqpoint{-0.008333in}{0.002210in}}{\pgfqpoint{-0.008333in}{0.000000in}}%
\pgfpathcurveto{\pgfqpoint{-0.008333in}{-0.002210in}}{\pgfqpoint{-0.007455in}{-0.004330in}}{\pgfqpoint{-0.005893in}{-0.005893in}}%
\pgfpathcurveto{\pgfqpoint{-0.004330in}{-0.007455in}}{\pgfqpoint{-0.002210in}{-0.008333in}}{\pgfqpoint{0.000000in}{-0.008333in}}%
\pgfpathclose%
\pgfusepath{stroke,fill}%
}%
\begin{pgfscope}%
\pgfsys@transformshift{4.111216in}{3.238146in}%
\pgfsys@useobject{currentmarker}{}%
\end{pgfscope}%
\end{pgfscope}%
\begin{pgfscope}%
\pgfpathrectangle{\pgfqpoint{0.100000in}{2.413063in}}{\pgfqpoint{5.037500in}{3.427208in}}%
\pgfusepath{clip}%
\pgfsetrectcap%
\pgfsetroundjoin%
\pgfsetlinewidth{1.505625pt}%
\definecolor{currentstroke}{rgb}{0.000000,0.000000,1.000000}%
\pgfsetstrokecolor{currentstroke}%
\pgfsetstrokeopacity{0.500000}%
\pgfsetdash{}{0pt}%
\pgfpathmoveto{\pgfqpoint{4.199267in}{3.174835in}}%
\pgfusepath{stroke}%
\end{pgfscope}%
\begin{pgfscope}%
\pgfpathrectangle{\pgfqpoint{0.100000in}{2.413063in}}{\pgfqpoint{5.037500in}{3.427208in}}%
\pgfusepath{clip}%
\pgfsetbuttcap%
\pgfsetroundjoin%
\definecolor{currentfill}{rgb}{0.000000,0.000000,1.000000}%
\pgfsetfillcolor{currentfill}%
\pgfsetfillopacity{0.500000}%
\pgfsetlinewidth{0.250937pt}%
\definecolor{currentstroke}{rgb}{0.000000,0.000000,0.000000}%
\pgfsetstrokecolor{currentstroke}%
\pgfsetstrokeopacity{0.500000}%
\pgfsetdash{}{0pt}%
\pgfsys@defobject{currentmarker}{\pgfqpoint{-0.025000in}{-0.025000in}}{\pgfqpoint{0.025000in}{0.025000in}}{%
\pgfpathmoveto{\pgfqpoint{0.000000in}{-0.025000in}}%
\pgfpathcurveto{\pgfqpoint{0.006630in}{-0.025000in}}{\pgfqpoint{0.012989in}{-0.022366in}}{\pgfqpoint{0.017678in}{-0.017678in}}%
\pgfpathcurveto{\pgfqpoint{0.022366in}{-0.012989in}}{\pgfqpoint{0.025000in}{-0.006630in}}{\pgfqpoint{0.025000in}{0.000000in}}%
\pgfpathcurveto{\pgfqpoint{0.025000in}{0.006630in}}{\pgfqpoint{0.022366in}{0.012989in}}{\pgfqpoint{0.017678in}{0.017678in}}%
\pgfpathcurveto{\pgfqpoint{0.012989in}{0.022366in}}{\pgfqpoint{0.006630in}{0.025000in}}{\pgfqpoint{0.000000in}{0.025000in}}%
\pgfpathcurveto{\pgfqpoint{-0.006630in}{0.025000in}}{\pgfqpoint{-0.012989in}{0.022366in}}{\pgfqpoint{-0.017678in}{0.017678in}}%
\pgfpathcurveto{\pgfqpoint{-0.022366in}{0.012989in}}{\pgfqpoint{-0.025000in}{0.006630in}}{\pgfqpoint{-0.025000in}{0.000000in}}%
\pgfpathcurveto{\pgfqpoint{-0.025000in}{-0.006630in}}{\pgfqpoint{-0.022366in}{-0.012989in}}{\pgfqpoint{-0.017678in}{-0.017678in}}%
\pgfpathcurveto{\pgfqpoint{-0.012989in}{-0.022366in}}{\pgfqpoint{-0.006630in}{-0.025000in}}{\pgfqpoint{0.000000in}{-0.025000in}}%
\pgfpathclose%
\pgfusepath{stroke,fill}%
}%
\begin{pgfscope}%
\pgfsys@transformshift{4.199267in}{3.174835in}%
\pgfsys@useobject{currentmarker}{}%
\end{pgfscope}%
\end{pgfscope}%
\begin{pgfscope}%
\pgfpathrectangle{\pgfqpoint{0.100000in}{2.413063in}}{\pgfqpoint{5.037500in}{3.427208in}}%
\pgfusepath{clip}%
\pgfsetrectcap%
\pgfsetroundjoin%
\pgfsetlinewidth{1.505625pt}%
\definecolor{currentstroke}{rgb}{0.501961,0.501961,0.501961}%
\pgfsetstrokecolor{currentstroke}%
\pgfsetstrokeopacity{0.500000}%
\pgfsetdash{}{0pt}%
\pgfpathmoveto{\pgfqpoint{4.283427in}{3.127572in}}%
\pgfusepath{stroke}%
\end{pgfscope}%
\begin{pgfscope}%
\pgfpathrectangle{\pgfqpoint{0.100000in}{2.413063in}}{\pgfqpoint{5.037500in}{3.427208in}}%
\pgfusepath{clip}%
\pgfsetbuttcap%
\pgfsetroundjoin%
\definecolor{currentfill}{rgb}{0.501961,0.501961,0.501961}%
\pgfsetfillcolor{currentfill}%
\pgfsetfillopacity{0.500000}%
\pgfsetlinewidth{0.250937pt}%
\definecolor{currentstroke}{rgb}{0.000000,0.000000,0.000000}%
\pgfsetstrokecolor{currentstroke}%
\pgfsetstrokeopacity{0.500000}%
\pgfsetdash{}{0pt}%
\pgfsys@defobject{currentmarker}{\pgfqpoint{-0.013889in}{-0.013889in}}{\pgfqpoint{0.013889in}{0.013889in}}{%
\pgfpathmoveto{\pgfqpoint{0.000000in}{-0.013889in}}%
\pgfpathcurveto{\pgfqpoint{0.003683in}{-0.013889in}}{\pgfqpoint{0.007216in}{-0.012425in}}{\pgfqpoint{0.009821in}{-0.009821in}}%
\pgfpathcurveto{\pgfqpoint{0.012425in}{-0.007216in}}{\pgfqpoint{0.013889in}{-0.003683in}}{\pgfqpoint{0.013889in}{0.000000in}}%
\pgfpathcurveto{\pgfqpoint{0.013889in}{0.003683in}}{\pgfqpoint{0.012425in}{0.007216in}}{\pgfqpoint{0.009821in}{0.009821in}}%
\pgfpathcurveto{\pgfqpoint{0.007216in}{0.012425in}}{\pgfqpoint{0.003683in}{0.013889in}}{\pgfqpoint{0.000000in}{0.013889in}}%
\pgfpathcurveto{\pgfqpoint{-0.003683in}{0.013889in}}{\pgfqpoint{-0.007216in}{0.012425in}}{\pgfqpoint{-0.009821in}{0.009821in}}%
\pgfpathcurveto{\pgfqpoint{-0.012425in}{0.007216in}}{\pgfqpoint{-0.013889in}{0.003683in}}{\pgfqpoint{-0.013889in}{0.000000in}}%
\pgfpathcurveto{\pgfqpoint{-0.013889in}{-0.003683in}}{\pgfqpoint{-0.012425in}{-0.007216in}}{\pgfqpoint{-0.009821in}{-0.009821in}}%
\pgfpathcurveto{\pgfqpoint{-0.007216in}{-0.012425in}}{\pgfqpoint{-0.003683in}{-0.013889in}}{\pgfqpoint{0.000000in}{-0.013889in}}%
\pgfpathclose%
\pgfusepath{stroke,fill}%
}%
\begin{pgfscope}%
\pgfsys@transformshift{4.283427in}{3.127572in}%
\pgfsys@useobject{currentmarker}{}%
\end{pgfscope}%
\end{pgfscope}%
\begin{pgfscope}%
\pgfpathrectangle{\pgfqpoint{0.100000in}{2.413063in}}{\pgfqpoint{5.037500in}{3.427208in}}%
\pgfusepath{clip}%
\pgfsetrectcap%
\pgfsetroundjoin%
\pgfsetlinewidth{1.505625pt}%
\definecolor{currentstroke}{rgb}{0.501961,0.501961,0.501961}%
\pgfsetstrokecolor{currentstroke}%
\pgfsetstrokeopacity{0.500000}%
\pgfsetdash{}{0pt}%
\pgfpathmoveto{\pgfqpoint{3.743850in}{3.307362in}}%
\pgfusepath{stroke}%
\end{pgfscope}%
\begin{pgfscope}%
\pgfpathrectangle{\pgfqpoint{0.100000in}{2.413063in}}{\pgfqpoint{5.037500in}{3.427208in}}%
\pgfusepath{clip}%
\pgfsetbuttcap%
\pgfsetroundjoin%
\definecolor{currentfill}{rgb}{0.501961,0.501961,0.501961}%
\pgfsetfillcolor{currentfill}%
\pgfsetfillopacity{0.500000}%
\pgfsetlinewidth{0.250937pt}%
\definecolor{currentstroke}{rgb}{0.000000,0.000000,0.000000}%
\pgfsetstrokecolor{currentstroke}%
\pgfsetstrokeopacity{0.500000}%
\pgfsetdash{}{0pt}%
\pgfsys@defobject{currentmarker}{\pgfqpoint{-0.013889in}{-0.013889in}}{\pgfqpoint{0.013889in}{0.013889in}}{%
\pgfpathmoveto{\pgfqpoint{0.000000in}{-0.013889in}}%
\pgfpathcurveto{\pgfqpoint{0.003683in}{-0.013889in}}{\pgfqpoint{0.007216in}{-0.012425in}}{\pgfqpoint{0.009821in}{-0.009821in}}%
\pgfpathcurveto{\pgfqpoint{0.012425in}{-0.007216in}}{\pgfqpoint{0.013889in}{-0.003683in}}{\pgfqpoint{0.013889in}{0.000000in}}%
\pgfpathcurveto{\pgfqpoint{0.013889in}{0.003683in}}{\pgfqpoint{0.012425in}{0.007216in}}{\pgfqpoint{0.009821in}{0.009821in}}%
\pgfpathcurveto{\pgfqpoint{0.007216in}{0.012425in}}{\pgfqpoint{0.003683in}{0.013889in}}{\pgfqpoint{0.000000in}{0.013889in}}%
\pgfpathcurveto{\pgfqpoint{-0.003683in}{0.013889in}}{\pgfqpoint{-0.007216in}{0.012425in}}{\pgfqpoint{-0.009821in}{0.009821in}}%
\pgfpathcurveto{\pgfqpoint{-0.012425in}{0.007216in}}{\pgfqpoint{-0.013889in}{0.003683in}}{\pgfqpoint{-0.013889in}{0.000000in}}%
\pgfpathcurveto{\pgfqpoint{-0.013889in}{-0.003683in}}{\pgfqpoint{-0.012425in}{-0.007216in}}{\pgfqpoint{-0.009821in}{-0.009821in}}%
\pgfpathcurveto{\pgfqpoint{-0.007216in}{-0.012425in}}{\pgfqpoint{-0.003683in}{-0.013889in}}{\pgfqpoint{0.000000in}{-0.013889in}}%
\pgfpathclose%
\pgfusepath{stroke,fill}%
}%
\begin{pgfscope}%
\pgfsys@transformshift{3.743850in}{3.307362in}%
\pgfsys@useobject{currentmarker}{}%
\end{pgfscope}%
\end{pgfscope}%
\begin{pgfscope}%
\pgfpathrectangle{\pgfqpoint{0.100000in}{2.413063in}}{\pgfqpoint{5.037500in}{3.427208in}}%
\pgfusepath{clip}%
\pgfsetrectcap%
\pgfsetroundjoin%
\pgfsetlinewidth{1.505625pt}%
\definecolor{currentstroke}{rgb}{0.000000,0.000000,1.000000}%
\pgfsetstrokecolor{currentstroke}%
\pgfsetstrokeopacity{0.500000}%
\pgfsetdash{}{0pt}%
\pgfpathmoveto{\pgfqpoint{3.584571in}{3.322865in}}%
\pgfusepath{stroke}%
\end{pgfscope}%
\begin{pgfscope}%
\pgfpathrectangle{\pgfqpoint{0.100000in}{2.413063in}}{\pgfqpoint{5.037500in}{3.427208in}}%
\pgfusepath{clip}%
\pgfsetbuttcap%
\pgfsetroundjoin%
\definecolor{currentfill}{rgb}{0.000000,0.000000,1.000000}%
\pgfsetfillcolor{currentfill}%
\pgfsetfillopacity{0.500000}%
\pgfsetlinewidth{0.250937pt}%
\definecolor{currentstroke}{rgb}{0.000000,0.000000,0.000000}%
\pgfsetstrokecolor{currentstroke}%
\pgfsetstrokeopacity{0.500000}%
\pgfsetdash{}{0pt}%
\pgfsys@defobject{currentmarker}{\pgfqpoint{-0.008333in}{-0.008333in}}{\pgfqpoint{0.008333in}{0.008333in}}{%
\pgfpathmoveto{\pgfqpoint{0.000000in}{-0.008333in}}%
\pgfpathcurveto{\pgfqpoint{0.002210in}{-0.008333in}}{\pgfqpoint{0.004330in}{-0.007455in}}{\pgfqpoint{0.005893in}{-0.005893in}}%
\pgfpathcurveto{\pgfqpoint{0.007455in}{-0.004330in}}{\pgfqpoint{0.008333in}{-0.002210in}}{\pgfqpoint{0.008333in}{0.000000in}}%
\pgfpathcurveto{\pgfqpoint{0.008333in}{0.002210in}}{\pgfqpoint{0.007455in}{0.004330in}}{\pgfqpoint{0.005893in}{0.005893in}}%
\pgfpathcurveto{\pgfqpoint{0.004330in}{0.007455in}}{\pgfqpoint{0.002210in}{0.008333in}}{\pgfqpoint{0.000000in}{0.008333in}}%
\pgfpathcurveto{\pgfqpoint{-0.002210in}{0.008333in}}{\pgfqpoint{-0.004330in}{0.007455in}}{\pgfqpoint{-0.005893in}{0.005893in}}%
\pgfpathcurveto{\pgfqpoint{-0.007455in}{0.004330in}}{\pgfqpoint{-0.008333in}{0.002210in}}{\pgfqpoint{-0.008333in}{0.000000in}}%
\pgfpathcurveto{\pgfqpoint{-0.008333in}{-0.002210in}}{\pgfqpoint{-0.007455in}{-0.004330in}}{\pgfqpoint{-0.005893in}{-0.005893in}}%
\pgfpathcurveto{\pgfqpoint{-0.004330in}{-0.007455in}}{\pgfqpoint{-0.002210in}{-0.008333in}}{\pgfqpoint{0.000000in}{-0.008333in}}%
\pgfpathclose%
\pgfusepath{stroke,fill}%
}%
\begin{pgfscope}%
\pgfsys@transformshift{3.584571in}{3.322865in}%
\pgfsys@useobject{currentmarker}{}%
\end{pgfscope}%
\end{pgfscope}%
\begin{pgfscope}%
\pgfpathrectangle{\pgfqpoint{0.100000in}{2.413063in}}{\pgfqpoint{5.037500in}{3.427208in}}%
\pgfusepath{clip}%
\pgfsetrectcap%
\pgfsetroundjoin%
\pgfsetlinewidth{1.505625pt}%
\definecolor{currentstroke}{rgb}{0.501961,0.501961,0.501961}%
\pgfsetstrokecolor{currentstroke}%
\pgfsetstrokeopacity{0.500000}%
\pgfsetdash{}{0pt}%
\pgfpathmoveto{\pgfqpoint{4.327246in}{3.046750in}}%
\pgfusepath{stroke}%
\end{pgfscope}%
\begin{pgfscope}%
\pgfpathrectangle{\pgfqpoint{0.100000in}{2.413063in}}{\pgfqpoint{5.037500in}{3.427208in}}%
\pgfusepath{clip}%
\pgfsetbuttcap%
\pgfsetroundjoin%
\definecolor{currentfill}{rgb}{0.501961,0.501961,0.501961}%
\pgfsetfillcolor{currentfill}%
\pgfsetfillopacity{0.500000}%
\pgfsetlinewidth{0.250937pt}%
\definecolor{currentstroke}{rgb}{0.000000,0.000000,0.000000}%
\pgfsetstrokecolor{currentstroke}%
\pgfsetstrokeopacity{0.500000}%
\pgfsetdash{}{0pt}%
\pgfsys@defobject{currentmarker}{\pgfqpoint{-0.013889in}{-0.013889in}}{\pgfqpoint{0.013889in}{0.013889in}}{%
\pgfpathmoveto{\pgfqpoint{0.000000in}{-0.013889in}}%
\pgfpathcurveto{\pgfqpoint{0.003683in}{-0.013889in}}{\pgfqpoint{0.007216in}{-0.012425in}}{\pgfqpoint{0.009821in}{-0.009821in}}%
\pgfpathcurveto{\pgfqpoint{0.012425in}{-0.007216in}}{\pgfqpoint{0.013889in}{-0.003683in}}{\pgfqpoint{0.013889in}{0.000000in}}%
\pgfpathcurveto{\pgfqpoint{0.013889in}{0.003683in}}{\pgfqpoint{0.012425in}{0.007216in}}{\pgfqpoint{0.009821in}{0.009821in}}%
\pgfpathcurveto{\pgfqpoint{0.007216in}{0.012425in}}{\pgfqpoint{0.003683in}{0.013889in}}{\pgfqpoint{0.000000in}{0.013889in}}%
\pgfpathcurveto{\pgfqpoint{-0.003683in}{0.013889in}}{\pgfqpoint{-0.007216in}{0.012425in}}{\pgfqpoint{-0.009821in}{0.009821in}}%
\pgfpathcurveto{\pgfqpoint{-0.012425in}{0.007216in}}{\pgfqpoint{-0.013889in}{0.003683in}}{\pgfqpoint{-0.013889in}{0.000000in}}%
\pgfpathcurveto{\pgfqpoint{-0.013889in}{-0.003683in}}{\pgfqpoint{-0.012425in}{-0.007216in}}{\pgfqpoint{-0.009821in}{-0.009821in}}%
\pgfpathcurveto{\pgfqpoint{-0.007216in}{-0.012425in}}{\pgfqpoint{-0.003683in}{-0.013889in}}{\pgfqpoint{0.000000in}{-0.013889in}}%
\pgfpathclose%
\pgfusepath{stroke,fill}%
}%
\begin{pgfscope}%
\pgfsys@transformshift{4.327246in}{3.046750in}%
\pgfsys@useobject{currentmarker}{}%
\end{pgfscope}%
\end{pgfscope}%
\begin{pgfscope}%
\pgfpathrectangle{\pgfqpoint{0.100000in}{2.413063in}}{\pgfqpoint{5.037500in}{3.427208in}}%
\pgfusepath{clip}%
\pgfsetrectcap%
\pgfsetroundjoin%
\pgfsetlinewidth{1.505625pt}%
\definecolor{currentstroke}{rgb}{0.501961,0.501961,0.501961}%
\pgfsetstrokecolor{currentstroke}%
\pgfsetstrokeopacity{0.500000}%
\pgfsetdash{}{0pt}%
\pgfpathmoveto{\pgfqpoint{4.158371in}{2.977590in}}%
\pgfusepath{stroke}%
\end{pgfscope}%
\begin{pgfscope}%
\pgfpathrectangle{\pgfqpoint{0.100000in}{2.413063in}}{\pgfqpoint{5.037500in}{3.427208in}}%
\pgfusepath{clip}%
\pgfsetbuttcap%
\pgfsetroundjoin%
\definecolor{currentfill}{rgb}{0.501961,0.501961,0.501961}%
\pgfsetfillcolor{currentfill}%
\pgfsetfillopacity{0.500000}%
\pgfsetlinewidth{0.250937pt}%
\definecolor{currentstroke}{rgb}{0.000000,0.000000,0.000000}%
\pgfsetstrokecolor{currentstroke}%
\pgfsetstrokeopacity{0.500000}%
\pgfsetdash{}{0pt}%
\pgfsys@defobject{currentmarker}{\pgfqpoint{-0.013889in}{-0.013889in}}{\pgfqpoint{0.013889in}{0.013889in}}{%
\pgfpathmoveto{\pgfqpoint{0.000000in}{-0.013889in}}%
\pgfpathcurveto{\pgfqpoint{0.003683in}{-0.013889in}}{\pgfqpoint{0.007216in}{-0.012425in}}{\pgfqpoint{0.009821in}{-0.009821in}}%
\pgfpathcurveto{\pgfqpoint{0.012425in}{-0.007216in}}{\pgfqpoint{0.013889in}{-0.003683in}}{\pgfqpoint{0.013889in}{0.000000in}}%
\pgfpathcurveto{\pgfqpoint{0.013889in}{0.003683in}}{\pgfqpoint{0.012425in}{0.007216in}}{\pgfqpoint{0.009821in}{0.009821in}}%
\pgfpathcurveto{\pgfqpoint{0.007216in}{0.012425in}}{\pgfqpoint{0.003683in}{0.013889in}}{\pgfqpoint{0.000000in}{0.013889in}}%
\pgfpathcurveto{\pgfqpoint{-0.003683in}{0.013889in}}{\pgfqpoint{-0.007216in}{0.012425in}}{\pgfqpoint{-0.009821in}{0.009821in}}%
\pgfpathcurveto{\pgfqpoint{-0.012425in}{0.007216in}}{\pgfqpoint{-0.013889in}{0.003683in}}{\pgfqpoint{-0.013889in}{0.000000in}}%
\pgfpathcurveto{\pgfqpoint{-0.013889in}{-0.003683in}}{\pgfqpoint{-0.012425in}{-0.007216in}}{\pgfqpoint{-0.009821in}{-0.009821in}}%
\pgfpathcurveto{\pgfqpoint{-0.007216in}{-0.012425in}}{\pgfqpoint{-0.003683in}{-0.013889in}}{\pgfqpoint{0.000000in}{-0.013889in}}%
\pgfpathclose%
\pgfusepath{stroke,fill}%
}%
\begin{pgfscope}%
\pgfsys@transformshift{4.158371in}{2.977590in}%
\pgfsys@useobject{currentmarker}{}%
\end{pgfscope}%
\end{pgfscope}%
\begin{pgfscope}%
\pgfpathrectangle{\pgfqpoint{0.100000in}{2.413063in}}{\pgfqpoint{5.037500in}{3.427208in}}%
\pgfusepath{clip}%
\pgfsetrectcap%
\pgfsetroundjoin%
\pgfsetlinewidth{1.505625pt}%
\definecolor{currentstroke}{rgb}{0.501961,0.501961,0.501961}%
\pgfsetstrokecolor{currentstroke}%
\pgfsetstrokeopacity{0.500000}%
\pgfsetdash{}{0pt}%
\pgfpathmoveto{\pgfqpoint{4.305082in}{3.105300in}}%
\pgfusepath{stroke}%
\end{pgfscope}%
\begin{pgfscope}%
\pgfpathrectangle{\pgfqpoint{0.100000in}{2.413063in}}{\pgfqpoint{5.037500in}{3.427208in}}%
\pgfusepath{clip}%
\pgfsetbuttcap%
\pgfsetroundjoin%
\definecolor{currentfill}{rgb}{0.501961,0.501961,0.501961}%
\pgfsetfillcolor{currentfill}%
\pgfsetfillopacity{0.500000}%
\pgfsetlinewidth{0.250937pt}%
\definecolor{currentstroke}{rgb}{0.000000,0.000000,0.000000}%
\pgfsetstrokecolor{currentstroke}%
\pgfsetstrokeopacity{0.500000}%
\pgfsetdash{}{0pt}%
\pgfsys@defobject{currentmarker}{\pgfqpoint{-0.013889in}{-0.013889in}}{\pgfqpoint{0.013889in}{0.013889in}}{%
\pgfpathmoveto{\pgfqpoint{0.000000in}{-0.013889in}}%
\pgfpathcurveto{\pgfqpoint{0.003683in}{-0.013889in}}{\pgfqpoint{0.007216in}{-0.012425in}}{\pgfqpoint{0.009821in}{-0.009821in}}%
\pgfpathcurveto{\pgfqpoint{0.012425in}{-0.007216in}}{\pgfqpoint{0.013889in}{-0.003683in}}{\pgfqpoint{0.013889in}{0.000000in}}%
\pgfpathcurveto{\pgfqpoint{0.013889in}{0.003683in}}{\pgfqpoint{0.012425in}{0.007216in}}{\pgfqpoint{0.009821in}{0.009821in}}%
\pgfpathcurveto{\pgfqpoint{0.007216in}{0.012425in}}{\pgfqpoint{0.003683in}{0.013889in}}{\pgfqpoint{0.000000in}{0.013889in}}%
\pgfpathcurveto{\pgfqpoint{-0.003683in}{0.013889in}}{\pgfqpoint{-0.007216in}{0.012425in}}{\pgfqpoint{-0.009821in}{0.009821in}}%
\pgfpathcurveto{\pgfqpoint{-0.012425in}{0.007216in}}{\pgfqpoint{-0.013889in}{0.003683in}}{\pgfqpoint{-0.013889in}{0.000000in}}%
\pgfpathcurveto{\pgfqpoint{-0.013889in}{-0.003683in}}{\pgfqpoint{-0.012425in}{-0.007216in}}{\pgfqpoint{-0.009821in}{-0.009821in}}%
\pgfpathcurveto{\pgfqpoint{-0.007216in}{-0.012425in}}{\pgfqpoint{-0.003683in}{-0.013889in}}{\pgfqpoint{0.000000in}{-0.013889in}}%
\pgfpathclose%
\pgfusepath{stroke,fill}%
}%
\begin{pgfscope}%
\pgfsys@transformshift{4.305082in}{3.105300in}%
\pgfsys@useobject{currentmarker}{}%
\end{pgfscope}%
\end{pgfscope}%
\begin{pgfscope}%
\pgfpathrectangle{\pgfqpoint{0.100000in}{2.413063in}}{\pgfqpoint{5.037500in}{3.427208in}}%
\pgfusepath{clip}%
\pgfsetrectcap%
\pgfsetroundjoin%
\pgfsetlinewidth{1.505625pt}%
\definecolor{currentstroke}{rgb}{0.501961,0.501961,0.501961}%
\pgfsetstrokecolor{currentstroke}%
\pgfsetstrokeopacity{0.500000}%
\pgfsetdash{}{0pt}%
\pgfpathmoveto{\pgfqpoint{4.211198in}{3.052586in}}%
\pgfusepath{stroke}%
\end{pgfscope}%
\begin{pgfscope}%
\pgfpathrectangle{\pgfqpoint{0.100000in}{2.413063in}}{\pgfqpoint{5.037500in}{3.427208in}}%
\pgfusepath{clip}%
\pgfsetbuttcap%
\pgfsetroundjoin%
\definecolor{currentfill}{rgb}{0.501961,0.501961,0.501961}%
\pgfsetfillcolor{currentfill}%
\pgfsetfillopacity{0.500000}%
\pgfsetlinewidth{0.250937pt}%
\definecolor{currentstroke}{rgb}{0.000000,0.000000,0.000000}%
\pgfsetstrokecolor{currentstroke}%
\pgfsetstrokeopacity{0.500000}%
\pgfsetdash{}{0pt}%
\pgfsys@defobject{currentmarker}{\pgfqpoint{-0.013889in}{-0.013889in}}{\pgfqpoint{0.013889in}{0.013889in}}{%
\pgfpathmoveto{\pgfqpoint{0.000000in}{-0.013889in}}%
\pgfpathcurveto{\pgfqpoint{0.003683in}{-0.013889in}}{\pgfqpoint{0.007216in}{-0.012425in}}{\pgfqpoint{0.009821in}{-0.009821in}}%
\pgfpathcurveto{\pgfqpoint{0.012425in}{-0.007216in}}{\pgfqpoint{0.013889in}{-0.003683in}}{\pgfqpoint{0.013889in}{0.000000in}}%
\pgfpathcurveto{\pgfqpoint{0.013889in}{0.003683in}}{\pgfqpoint{0.012425in}{0.007216in}}{\pgfqpoint{0.009821in}{0.009821in}}%
\pgfpathcurveto{\pgfqpoint{0.007216in}{0.012425in}}{\pgfqpoint{0.003683in}{0.013889in}}{\pgfqpoint{0.000000in}{0.013889in}}%
\pgfpathcurveto{\pgfqpoint{-0.003683in}{0.013889in}}{\pgfqpoint{-0.007216in}{0.012425in}}{\pgfqpoint{-0.009821in}{0.009821in}}%
\pgfpathcurveto{\pgfqpoint{-0.012425in}{0.007216in}}{\pgfqpoint{-0.013889in}{0.003683in}}{\pgfqpoint{-0.013889in}{0.000000in}}%
\pgfpathcurveto{\pgfqpoint{-0.013889in}{-0.003683in}}{\pgfqpoint{-0.012425in}{-0.007216in}}{\pgfqpoint{-0.009821in}{-0.009821in}}%
\pgfpathcurveto{\pgfqpoint{-0.007216in}{-0.012425in}}{\pgfqpoint{-0.003683in}{-0.013889in}}{\pgfqpoint{0.000000in}{-0.013889in}}%
\pgfpathclose%
\pgfusepath{stroke,fill}%
}%
\begin{pgfscope}%
\pgfsys@transformshift{4.211198in}{3.052586in}%
\pgfsys@useobject{currentmarker}{}%
\end{pgfscope}%
\end{pgfscope}%
\begin{pgfscope}%
\pgfpathrectangle{\pgfqpoint{0.100000in}{2.413063in}}{\pgfqpoint{5.037500in}{3.427208in}}%
\pgfusepath{clip}%
\pgfsetrectcap%
\pgfsetroundjoin%
\pgfsetlinewidth{1.505625pt}%
\definecolor{currentstroke}{rgb}{0.000000,0.000000,1.000000}%
\pgfsetstrokecolor{currentstroke}%
\pgfsetstrokeopacity{0.500000}%
\pgfsetdash{}{0pt}%
\pgfpathmoveto{\pgfqpoint{3.877566in}{3.354757in}}%
\pgfusepath{stroke}%
\end{pgfscope}%
\begin{pgfscope}%
\pgfpathrectangle{\pgfqpoint{0.100000in}{2.413063in}}{\pgfqpoint{5.037500in}{3.427208in}}%
\pgfusepath{clip}%
\pgfsetbuttcap%
\pgfsetroundjoin%
\definecolor{currentfill}{rgb}{0.000000,0.000000,1.000000}%
\pgfsetfillcolor{currentfill}%
\pgfsetfillopacity{0.500000}%
\pgfsetlinewidth{0.250937pt}%
\definecolor{currentstroke}{rgb}{0.000000,0.000000,0.000000}%
\pgfsetstrokecolor{currentstroke}%
\pgfsetstrokeopacity{0.500000}%
\pgfsetdash{}{0pt}%
\pgfsys@defobject{currentmarker}{\pgfqpoint{-0.011111in}{-0.011111in}}{\pgfqpoint{0.011111in}{0.011111in}}{%
\pgfpathmoveto{\pgfqpoint{0.000000in}{-0.011111in}}%
\pgfpathcurveto{\pgfqpoint{0.002947in}{-0.011111in}}{\pgfqpoint{0.005773in}{-0.009940in}}{\pgfqpoint{0.007857in}{-0.007857in}}%
\pgfpathcurveto{\pgfqpoint{0.009940in}{-0.005773in}}{\pgfqpoint{0.011111in}{-0.002947in}}{\pgfqpoint{0.011111in}{0.000000in}}%
\pgfpathcurveto{\pgfqpoint{0.011111in}{0.002947in}}{\pgfqpoint{0.009940in}{0.005773in}}{\pgfqpoint{0.007857in}{0.007857in}}%
\pgfpathcurveto{\pgfqpoint{0.005773in}{0.009940in}}{\pgfqpoint{0.002947in}{0.011111in}}{\pgfqpoint{0.000000in}{0.011111in}}%
\pgfpathcurveto{\pgfqpoint{-0.002947in}{0.011111in}}{\pgfqpoint{-0.005773in}{0.009940in}}{\pgfqpoint{-0.007857in}{0.007857in}}%
\pgfpathcurveto{\pgfqpoint{-0.009940in}{0.005773in}}{\pgfqpoint{-0.011111in}{0.002947in}}{\pgfqpoint{-0.011111in}{0.000000in}}%
\pgfpathcurveto{\pgfqpoint{-0.011111in}{-0.002947in}}{\pgfqpoint{-0.009940in}{-0.005773in}}{\pgfqpoint{-0.007857in}{-0.007857in}}%
\pgfpathcurveto{\pgfqpoint{-0.005773in}{-0.009940in}}{\pgfqpoint{-0.002947in}{-0.011111in}}{\pgfqpoint{0.000000in}{-0.011111in}}%
\pgfpathclose%
\pgfusepath{stroke,fill}%
}%
\begin{pgfscope}%
\pgfsys@transformshift{3.877566in}{3.354757in}%
\pgfsys@useobject{currentmarker}{}%
\end{pgfscope}%
\end{pgfscope}%
\begin{pgfscope}%
\pgfpathrectangle{\pgfqpoint{0.100000in}{2.413063in}}{\pgfqpoint{5.037500in}{3.427208in}}%
\pgfusepath{clip}%
\pgfsetrectcap%
\pgfsetroundjoin%
\pgfsetlinewidth{1.505625pt}%
\definecolor{currentstroke}{rgb}{0.501961,0.501961,0.501961}%
\pgfsetstrokecolor{currentstroke}%
\pgfsetstrokeopacity{0.500000}%
\pgfsetdash{}{0pt}%
\pgfpathmoveto{\pgfqpoint{4.099020in}{3.089954in}}%
\pgfusepath{stroke}%
\end{pgfscope}%
\begin{pgfscope}%
\pgfpathrectangle{\pgfqpoint{0.100000in}{2.413063in}}{\pgfqpoint{5.037500in}{3.427208in}}%
\pgfusepath{clip}%
\pgfsetbuttcap%
\pgfsetroundjoin%
\definecolor{currentfill}{rgb}{0.501961,0.501961,0.501961}%
\pgfsetfillcolor{currentfill}%
\pgfsetfillopacity{0.500000}%
\pgfsetlinewidth{0.250937pt}%
\definecolor{currentstroke}{rgb}{0.000000,0.000000,0.000000}%
\pgfsetstrokecolor{currentstroke}%
\pgfsetstrokeopacity{0.500000}%
\pgfsetdash{}{0pt}%
\pgfsys@defobject{currentmarker}{\pgfqpoint{-0.013889in}{-0.013889in}}{\pgfqpoint{0.013889in}{0.013889in}}{%
\pgfpathmoveto{\pgfqpoint{0.000000in}{-0.013889in}}%
\pgfpathcurveto{\pgfqpoint{0.003683in}{-0.013889in}}{\pgfqpoint{0.007216in}{-0.012425in}}{\pgfqpoint{0.009821in}{-0.009821in}}%
\pgfpathcurveto{\pgfqpoint{0.012425in}{-0.007216in}}{\pgfqpoint{0.013889in}{-0.003683in}}{\pgfqpoint{0.013889in}{0.000000in}}%
\pgfpathcurveto{\pgfqpoint{0.013889in}{0.003683in}}{\pgfqpoint{0.012425in}{0.007216in}}{\pgfqpoint{0.009821in}{0.009821in}}%
\pgfpathcurveto{\pgfqpoint{0.007216in}{0.012425in}}{\pgfqpoint{0.003683in}{0.013889in}}{\pgfqpoint{0.000000in}{0.013889in}}%
\pgfpathcurveto{\pgfqpoint{-0.003683in}{0.013889in}}{\pgfqpoint{-0.007216in}{0.012425in}}{\pgfqpoint{-0.009821in}{0.009821in}}%
\pgfpathcurveto{\pgfqpoint{-0.012425in}{0.007216in}}{\pgfqpoint{-0.013889in}{0.003683in}}{\pgfqpoint{-0.013889in}{0.000000in}}%
\pgfpathcurveto{\pgfqpoint{-0.013889in}{-0.003683in}}{\pgfqpoint{-0.012425in}{-0.007216in}}{\pgfqpoint{-0.009821in}{-0.009821in}}%
\pgfpathcurveto{\pgfqpoint{-0.007216in}{-0.012425in}}{\pgfqpoint{-0.003683in}{-0.013889in}}{\pgfqpoint{0.000000in}{-0.013889in}}%
\pgfpathclose%
\pgfusepath{stroke,fill}%
}%
\begin{pgfscope}%
\pgfsys@transformshift{4.099020in}{3.089954in}%
\pgfsys@useobject{currentmarker}{}%
\end{pgfscope}%
\end{pgfscope}%
\begin{pgfscope}%
\pgfpathrectangle{\pgfqpoint{0.100000in}{2.413063in}}{\pgfqpoint{5.037500in}{3.427208in}}%
\pgfusepath{clip}%
\pgfsetrectcap%
\pgfsetroundjoin%
\pgfsetlinewidth{1.505625pt}%
\definecolor{currentstroke}{rgb}{0.501961,0.501961,0.501961}%
\pgfsetstrokecolor{currentstroke}%
\pgfsetstrokeopacity{0.500000}%
\pgfsetdash{}{0pt}%
\pgfpathmoveto{\pgfqpoint{4.134085in}{3.213055in}}%
\pgfusepath{stroke}%
\end{pgfscope}%
\begin{pgfscope}%
\pgfpathrectangle{\pgfqpoint{0.100000in}{2.413063in}}{\pgfqpoint{5.037500in}{3.427208in}}%
\pgfusepath{clip}%
\pgfsetbuttcap%
\pgfsetroundjoin%
\definecolor{currentfill}{rgb}{0.501961,0.501961,0.501961}%
\pgfsetfillcolor{currentfill}%
\pgfsetfillopacity{0.500000}%
\pgfsetlinewidth{0.250937pt}%
\definecolor{currentstroke}{rgb}{0.000000,0.000000,0.000000}%
\pgfsetstrokecolor{currentstroke}%
\pgfsetstrokeopacity{0.500000}%
\pgfsetdash{}{0pt}%
\pgfsys@defobject{currentmarker}{\pgfqpoint{-0.013889in}{-0.013889in}}{\pgfqpoint{0.013889in}{0.013889in}}{%
\pgfpathmoveto{\pgfqpoint{0.000000in}{-0.013889in}}%
\pgfpathcurveto{\pgfqpoint{0.003683in}{-0.013889in}}{\pgfqpoint{0.007216in}{-0.012425in}}{\pgfqpoint{0.009821in}{-0.009821in}}%
\pgfpathcurveto{\pgfqpoint{0.012425in}{-0.007216in}}{\pgfqpoint{0.013889in}{-0.003683in}}{\pgfqpoint{0.013889in}{0.000000in}}%
\pgfpathcurveto{\pgfqpoint{0.013889in}{0.003683in}}{\pgfqpoint{0.012425in}{0.007216in}}{\pgfqpoint{0.009821in}{0.009821in}}%
\pgfpathcurveto{\pgfqpoint{0.007216in}{0.012425in}}{\pgfqpoint{0.003683in}{0.013889in}}{\pgfqpoint{0.000000in}{0.013889in}}%
\pgfpathcurveto{\pgfqpoint{-0.003683in}{0.013889in}}{\pgfqpoint{-0.007216in}{0.012425in}}{\pgfqpoint{-0.009821in}{0.009821in}}%
\pgfpathcurveto{\pgfqpoint{-0.012425in}{0.007216in}}{\pgfqpoint{-0.013889in}{0.003683in}}{\pgfqpoint{-0.013889in}{0.000000in}}%
\pgfpathcurveto{\pgfqpoint{-0.013889in}{-0.003683in}}{\pgfqpoint{-0.012425in}{-0.007216in}}{\pgfqpoint{-0.009821in}{-0.009821in}}%
\pgfpathcurveto{\pgfqpoint{-0.007216in}{-0.012425in}}{\pgfqpoint{-0.003683in}{-0.013889in}}{\pgfqpoint{0.000000in}{-0.013889in}}%
\pgfpathclose%
\pgfusepath{stroke,fill}%
}%
\begin{pgfscope}%
\pgfsys@transformshift{4.134085in}{3.213055in}%
\pgfsys@useobject{currentmarker}{}%
\end{pgfscope}%
\end{pgfscope}%
\begin{pgfscope}%
\pgfpathrectangle{\pgfqpoint{0.100000in}{2.413063in}}{\pgfqpoint{5.037500in}{3.427208in}}%
\pgfusepath{clip}%
\pgfsetrectcap%
\pgfsetroundjoin%
\pgfsetlinewidth{1.505625pt}%
\definecolor{currentstroke}{rgb}{0.678431,1.000000,0.184314}%
\pgfsetstrokecolor{currentstroke}%
\pgfsetstrokeopacity{0.500000}%
\pgfsetdash{}{0pt}%
\pgfpathmoveto{\pgfqpoint{3.874269in}{3.487855in}}%
\pgfusepath{stroke}%
\end{pgfscope}%
\begin{pgfscope}%
\pgfpathrectangle{\pgfqpoint{0.100000in}{2.413063in}}{\pgfqpoint{5.037500in}{3.427208in}}%
\pgfusepath{clip}%
\pgfsetbuttcap%
\pgfsetroundjoin%
\definecolor{currentfill}{rgb}{0.678431,1.000000,0.184314}%
\pgfsetfillcolor{currentfill}%
\pgfsetfillopacity{0.500000}%
\pgfsetlinewidth{0.250937pt}%
\definecolor{currentstroke}{rgb}{0.000000,0.000000,0.000000}%
\pgfsetstrokecolor{currentstroke}%
\pgfsetstrokeopacity{0.500000}%
\pgfsetdash{}{0pt}%
\pgfsys@defobject{currentmarker}{\pgfqpoint{-0.019444in}{-0.019444in}}{\pgfqpoint{0.019444in}{0.019444in}}{%
\pgfpathmoveto{\pgfqpoint{0.000000in}{-0.019444in}}%
\pgfpathcurveto{\pgfqpoint{0.005157in}{-0.019444in}}{\pgfqpoint{0.010103in}{-0.017396in}}{\pgfqpoint{0.013749in}{-0.013749in}}%
\pgfpathcurveto{\pgfqpoint{0.017396in}{-0.010103in}}{\pgfqpoint{0.019444in}{-0.005157in}}{\pgfqpoint{0.019444in}{0.000000in}}%
\pgfpathcurveto{\pgfqpoint{0.019444in}{0.005157in}}{\pgfqpoint{0.017396in}{0.010103in}}{\pgfqpoint{0.013749in}{0.013749in}}%
\pgfpathcurveto{\pgfqpoint{0.010103in}{0.017396in}}{\pgfqpoint{0.005157in}{0.019444in}}{\pgfqpoint{0.000000in}{0.019444in}}%
\pgfpathcurveto{\pgfqpoint{-0.005157in}{0.019444in}}{\pgfqpoint{-0.010103in}{0.017396in}}{\pgfqpoint{-0.013749in}{0.013749in}}%
\pgfpathcurveto{\pgfqpoint{-0.017396in}{0.010103in}}{\pgfqpoint{-0.019444in}{0.005157in}}{\pgfqpoint{-0.019444in}{0.000000in}}%
\pgfpathcurveto{\pgfqpoint{-0.019444in}{-0.005157in}}{\pgfqpoint{-0.017396in}{-0.010103in}}{\pgfqpoint{-0.013749in}{-0.013749in}}%
\pgfpathcurveto{\pgfqpoint{-0.010103in}{-0.017396in}}{\pgfqpoint{-0.005157in}{-0.019444in}}{\pgfqpoint{0.000000in}{-0.019444in}}%
\pgfpathclose%
\pgfusepath{stroke,fill}%
}%
\begin{pgfscope}%
\pgfsys@transformshift{3.874269in}{3.487855in}%
\pgfsys@useobject{currentmarker}{}%
\end{pgfscope}%
\end{pgfscope}%
\begin{pgfscope}%
\pgfpathrectangle{\pgfqpoint{0.100000in}{2.413063in}}{\pgfqpoint{5.037500in}{3.427208in}}%
\pgfusepath{clip}%
\pgfsetrectcap%
\pgfsetroundjoin%
\pgfsetlinewidth{1.505625pt}%
\definecolor{currentstroke}{rgb}{0.678431,1.000000,0.184314}%
\pgfsetstrokecolor{currentstroke}%
\pgfsetstrokeopacity{0.500000}%
\pgfsetdash{}{0pt}%
\pgfpathmoveto{\pgfqpoint{3.914523in}{3.770303in}}%
\pgfusepath{stroke}%
\end{pgfscope}%
\begin{pgfscope}%
\pgfpathrectangle{\pgfqpoint{0.100000in}{2.413063in}}{\pgfqpoint{5.037500in}{3.427208in}}%
\pgfusepath{clip}%
\pgfsetbuttcap%
\pgfsetroundjoin%
\definecolor{currentfill}{rgb}{0.678431,1.000000,0.184314}%
\pgfsetfillcolor{currentfill}%
\pgfsetfillopacity{0.500000}%
\pgfsetlinewidth{0.250937pt}%
\definecolor{currentstroke}{rgb}{0.000000,0.000000,0.000000}%
\pgfsetstrokecolor{currentstroke}%
\pgfsetstrokeopacity{0.500000}%
\pgfsetdash{}{0pt}%
\pgfsys@defobject{currentmarker}{\pgfqpoint{-0.016667in}{-0.016667in}}{\pgfqpoint{0.016667in}{0.016667in}}{%
\pgfpathmoveto{\pgfqpoint{0.000000in}{-0.016667in}}%
\pgfpathcurveto{\pgfqpoint{0.004420in}{-0.016667in}}{\pgfqpoint{0.008660in}{-0.014911in}}{\pgfqpoint{0.011785in}{-0.011785in}}%
\pgfpathcurveto{\pgfqpoint{0.014911in}{-0.008660in}}{\pgfqpoint{0.016667in}{-0.004420in}}{\pgfqpoint{0.016667in}{0.000000in}}%
\pgfpathcurveto{\pgfqpoint{0.016667in}{0.004420in}}{\pgfqpoint{0.014911in}{0.008660in}}{\pgfqpoint{0.011785in}{0.011785in}}%
\pgfpathcurveto{\pgfqpoint{0.008660in}{0.014911in}}{\pgfqpoint{0.004420in}{0.016667in}}{\pgfqpoint{0.000000in}{0.016667in}}%
\pgfpathcurveto{\pgfqpoint{-0.004420in}{0.016667in}}{\pgfqpoint{-0.008660in}{0.014911in}}{\pgfqpoint{-0.011785in}{0.011785in}}%
\pgfpathcurveto{\pgfqpoint{-0.014911in}{0.008660in}}{\pgfqpoint{-0.016667in}{0.004420in}}{\pgfqpoint{-0.016667in}{0.000000in}}%
\pgfpathcurveto{\pgfqpoint{-0.016667in}{-0.004420in}}{\pgfqpoint{-0.014911in}{-0.008660in}}{\pgfqpoint{-0.011785in}{-0.011785in}}%
\pgfpathcurveto{\pgfqpoint{-0.008660in}{-0.014911in}}{\pgfqpoint{-0.004420in}{-0.016667in}}{\pgfqpoint{0.000000in}{-0.016667in}}%
\pgfpathclose%
\pgfusepath{stroke,fill}%
}%
\begin{pgfscope}%
\pgfsys@transformshift{3.914523in}{3.770303in}%
\pgfsys@useobject{currentmarker}{}%
\end{pgfscope}%
\end{pgfscope}%
\begin{pgfscope}%
\pgfpathrectangle{\pgfqpoint{0.100000in}{2.413063in}}{\pgfqpoint{5.037500in}{3.427208in}}%
\pgfusepath{clip}%
\pgfsetrectcap%
\pgfsetroundjoin%
\pgfsetlinewidth{1.505625pt}%
\definecolor{currentstroke}{rgb}{0.678431,1.000000,0.184314}%
\pgfsetstrokecolor{currentstroke}%
\pgfsetstrokeopacity{0.500000}%
\pgfsetdash{}{0pt}%
\pgfpathmoveto{\pgfqpoint{3.821936in}{3.734931in}}%
\pgfusepath{stroke}%
\end{pgfscope}%
\begin{pgfscope}%
\pgfpathrectangle{\pgfqpoint{0.100000in}{2.413063in}}{\pgfqpoint{5.037500in}{3.427208in}}%
\pgfusepath{clip}%
\pgfsetbuttcap%
\pgfsetroundjoin%
\definecolor{currentfill}{rgb}{0.678431,1.000000,0.184314}%
\pgfsetfillcolor{currentfill}%
\pgfsetfillopacity{0.500000}%
\pgfsetlinewidth{0.250937pt}%
\definecolor{currentstroke}{rgb}{0.000000,0.000000,0.000000}%
\pgfsetstrokecolor{currentstroke}%
\pgfsetstrokeopacity{0.500000}%
\pgfsetdash{}{0pt}%
\pgfsys@defobject{currentmarker}{\pgfqpoint{-0.016667in}{-0.016667in}}{\pgfqpoint{0.016667in}{0.016667in}}{%
\pgfpathmoveto{\pgfqpoint{0.000000in}{-0.016667in}}%
\pgfpathcurveto{\pgfqpoint{0.004420in}{-0.016667in}}{\pgfqpoint{0.008660in}{-0.014911in}}{\pgfqpoint{0.011785in}{-0.011785in}}%
\pgfpathcurveto{\pgfqpoint{0.014911in}{-0.008660in}}{\pgfqpoint{0.016667in}{-0.004420in}}{\pgfqpoint{0.016667in}{0.000000in}}%
\pgfpathcurveto{\pgfqpoint{0.016667in}{0.004420in}}{\pgfqpoint{0.014911in}{0.008660in}}{\pgfqpoint{0.011785in}{0.011785in}}%
\pgfpathcurveto{\pgfqpoint{0.008660in}{0.014911in}}{\pgfqpoint{0.004420in}{0.016667in}}{\pgfqpoint{0.000000in}{0.016667in}}%
\pgfpathcurveto{\pgfqpoint{-0.004420in}{0.016667in}}{\pgfqpoint{-0.008660in}{0.014911in}}{\pgfqpoint{-0.011785in}{0.011785in}}%
\pgfpathcurveto{\pgfqpoint{-0.014911in}{0.008660in}}{\pgfqpoint{-0.016667in}{0.004420in}}{\pgfqpoint{-0.016667in}{0.000000in}}%
\pgfpathcurveto{\pgfqpoint{-0.016667in}{-0.004420in}}{\pgfqpoint{-0.014911in}{-0.008660in}}{\pgfqpoint{-0.011785in}{-0.011785in}}%
\pgfpathcurveto{\pgfqpoint{-0.008660in}{-0.014911in}}{\pgfqpoint{-0.004420in}{-0.016667in}}{\pgfqpoint{0.000000in}{-0.016667in}}%
\pgfpathclose%
\pgfusepath{stroke,fill}%
}%
\begin{pgfscope}%
\pgfsys@transformshift{3.821936in}{3.734931in}%
\pgfsys@useobject{currentmarker}{}%
\end{pgfscope}%
\end{pgfscope}%
\begin{pgfscope}%
\pgfpathrectangle{\pgfqpoint{0.100000in}{2.413063in}}{\pgfqpoint{5.037500in}{3.427208in}}%
\pgfusepath{clip}%
\pgfsetrectcap%
\pgfsetroundjoin%
\pgfsetlinewidth{1.505625pt}%
\definecolor{currentstroke}{rgb}{0.678431,1.000000,0.184314}%
\pgfsetstrokecolor{currentstroke}%
\pgfsetstrokeopacity{0.500000}%
\pgfsetdash{}{0pt}%
\pgfpathmoveto{\pgfqpoint{4.056797in}{3.733205in}}%
\pgfusepath{stroke}%
\end{pgfscope}%
\begin{pgfscope}%
\pgfpathrectangle{\pgfqpoint{0.100000in}{2.413063in}}{\pgfqpoint{5.037500in}{3.427208in}}%
\pgfusepath{clip}%
\pgfsetbuttcap%
\pgfsetroundjoin%
\definecolor{currentfill}{rgb}{0.678431,1.000000,0.184314}%
\pgfsetfillcolor{currentfill}%
\pgfsetfillopacity{0.500000}%
\pgfsetlinewidth{0.250937pt}%
\definecolor{currentstroke}{rgb}{0.000000,0.000000,0.000000}%
\pgfsetstrokecolor{currentstroke}%
\pgfsetstrokeopacity{0.500000}%
\pgfsetdash{}{0pt}%
\pgfsys@defobject{currentmarker}{\pgfqpoint{-0.011111in}{-0.011111in}}{\pgfqpoint{0.011111in}{0.011111in}}{%
\pgfpathmoveto{\pgfqpoint{0.000000in}{-0.011111in}}%
\pgfpathcurveto{\pgfqpoint{0.002947in}{-0.011111in}}{\pgfqpoint{0.005773in}{-0.009940in}}{\pgfqpoint{0.007857in}{-0.007857in}}%
\pgfpathcurveto{\pgfqpoint{0.009940in}{-0.005773in}}{\pgfqpoint{0.011111in}{-0.002947in}}{\pgfqpoint{0.011111in}{0.000000in}}%
\pgfpathcurveto{\pgfqpoint{0.011111in}{0.002947in}}{\pgfqpoint{0.009940in}{0.005773in}}{\pgfqpoint{0.007857in}{0.007857in}}%
\pgfpathcurveto{\pgfqpoint{0.005773in}{0.009940in}}{\pgfqpoint{0.002947in}{0.011111in}}{\pgfqpoint{0.000000in}{0.011111in}}%
\pgfpathcurveto{\pgfqpoint{-0.002947in}{0.011111in}}{\pgfqpoint{-0.005773in}{0.009940in}}{\pgfqpoint{-0.007857in}{0.007857in}}%
\pgfpathcurveto{\pgfqpoint{-0.009940in}{0.005773in}}{\pgfqpoint{-0.011111in}{0.002947in}}{\pgfqpoint{-0.011111in}{0.000000in}}%
\pgfpathcurveto{\pgfqpoint{-0.011111in}{-0.002947in}}{\pgfqpoint{-0.009940in}{-0.005773in}}{\pgfqpoint{-0.007857in}{-0.007857in}}%
\pgfpathcurveto{\pgfqpoint{-0.005773in}{-0.009940in}}{\pgfqpoint{-0.002947in}{-0.011111in}}{\pgfqpoint{0.000000in}{-0.011111in}}%
\pgfpathclose%
\pgfusepath{stroke,fill}%
}%
\begin{pgfscope}%
\pgfsys@transformshift{4.056797in}{3.733205in}%
\pgfsys@useobject{currentmarker}{}%
\end{pgfscope}%
\end{pgfscope}%
\begin{pgfscope}%
\pgfpathrectangle{\pgfqpoint{0.100000in}{2.413063in}}{\pgfqpoint{5.037500in}{3.427208in}}%
\pgfusepath{clip}%
\pgfsetrectcap%
\pgfsetroundjoin%
\pgfsetlinewidth{1.505625pt}%
\definecolor{currentstroke}{rgb}{0.678431,1.000000,0.184314}%
\pgfsetstrokecolor{currentstroke}%
\pgfsetstrokeopacity{0.500000}%
\pgfsetdash{}{0pt}%
\pgfpathmoveto{\pgfqpoint{4.142799in}{3.473859in}}%
\pgfusepath{stroke}%
\end{pgfscope}%
\begin{pgfscope}%
\pgfpathrectangle{\pgfqpoint{0.100000in}{2.413063in}}{\pgfqpoint{5.037500in}{3.427208in}}%
\pgfusepath{clip}%
\pgfsetbuttcap%
\pgfsetroundjoin%
\definecolor{currentfill}{rgb}{0.678431,1.000000,0.184314}%
\pgfsetfillcolor{currentfill}%
\pgfsetfillopacity{0.500000}%
\pgfsetlinewidth{0.250937pt}%
\definecolor{currentstroke}{rgb}{0.000000,0.000000,0.000000}%
\pgfsetstrokecolor{currentstroke}%
\pgfsetstrokeopacity{0.500000}%
\pgfsetdash{}{0pt}%
\pgfsys@defobject{currentmarker}{\pgfqpoint{-0.022222in}{-0.022222in}}{\pgfqpoint{0.022222in}{0.022222in}}{%
\pgfpathmoveto{\pgfqpoint{0.000000in}{-0.022222in}}%
\pgfpathcurveto{\pgfqpoint{0.005893in}{-0.022222in}}{\pgfqpoint{0.011546in}{-0.019881in}}{\pgfqpoint{0.015713in}{-0.015713in}}%
\pgfpathcurveto{\pgfqpoint{0.019881in}{-0.011546in}}{\pgfqpoint{0.022222in}{-0.005893in}}{\pgfqpoint{0.022222in}{0.000000in}}%
\pgfpathcurveto{\pgfqpoint{0.022222in}{0.005893in}}{\pgfqpoint{0.019881in}{0.011546in}}{\pgfqpoint{0.015713in}{0.015713in}}%
\pgfpathcurveto{\pgfqpoint{0.011546in}{0.019881in}}{\pgfqpoint{0.005893in}{0.022222in}}{\pgfqpoint{0.000000in}{0.022222in}}%
\pgfpathcurveto{\pgfqpoint{-0.005893in}{0.022222in}}{\pgfqpoint{-0.011546in}{0.019881in}}{\pgfqpoint{-0.015713in}{0.015713in}}%
\pgfpathcurveto{\pgfqpoint{-0.019881in}{0.011546in}}{\pgfqpoint{-0.022222in}{0.005893in}}{\pgfqpoint{-0.022222in}{0.000000in}}%
\pgfpathcurveto{\pgfqpoint{-0.022222in}{-0.005893in}}{\pgfqpoint{-0.019881in}{-0.011546in}}{\pgfqpoint{-0.015713in}{-0.015713in}}%
\pgfpathcurveto{\pgfqpoint{-0.011546in}{-0.019881in}}{\pgfqpoint{-0.005893in}{-0.022222in}}{\pgfqpoint{0.000000in}{-0.022222in}}%
\pgfpathclose%
\pgfusepath{stroke,fill}%
}%
\begin{pgfscope}%
\pgfsys@transformshift{4.142799in}{3.473859in}%
\pgfsys@useobject{currentmarker}{}%
\end{pgfscope}%
\end{pgfscope}%
\begin{pgfscope}%
\pgfpathrectangle{\pgfqpoint{0.100000in}{2.413063in}}{\pgfqpoint{5.037500in}{3.427208in}}%
\pgfusepath{clip}%
\pgfsetrectcap%
\pgfsetroundjoin%
\pgfsetlinewidth{1.505625pt}%
\definecolor{currentstroke}{rgb}{0.678431,1.000000,0.184314}%
\pgfsetstrokecolor{currentstroke}%
\pgfsetstrokeopacity{0.500000}%
\pgfsetdash{}{0pt}%
\pgfpathmoveto{\pgfqpoint{3.781250in}{3.580214in}}%
\pgfusepath{stroke}%
\end{pgfscope}%
\begin{pgfscope}%
\pgfpathrectangle{\pgfqpoint{0.100000in}{2.413063in}}{\pgfqpoint{5.037500in}{3.427208in}}%
\pgfusepath{clip}%
\pgfsetbuttcap%
\pgfsetroundjoin%
\definecolor{currentfill}{rgb}{0.678431,1.000000,0.184314}%
\pgfsetfillcolor{currentfill}%
\pgfsetfillopacity{0.500000}%
\pgfsetlinewidth{0.250937pt}%
\definecolor{currentstroke}{rgb}{0.000000,0.000000,0.000000}%
\pgfsetstrokecolor{currentstroke}%
\pgfsetstrokeopacity{0.500000}%
\pgfsetdash{}{0pt}%
\pgfsys@defobject{currentmarker}{\pgfqpoint{-0.019444in}{-0.019444in}}{\pgfqpoint{0.019444in}{0.019444in}}{%
\pgfpathmoveto{\pgfqpoint{0.000000in}{-0.019444in}}%
\pgfpathcurveto{\pgfqpoint{0.005157in}{-0.019444in}}{\pgfqpoint{0.010103in}{-0.017396in}}{\pgfqpoint{0.013749in}{-0.013749in}}%
\pgfpathcurveto{\pgfqpoint{0.017396in}{-0.010103in}}{\pgfqpoint{0.019444in}{-0.005157in}}{\pgfqpoint{0.019444in}{0.000000in}}%
\pgfpathcurveto{\pgfqpoint{0.019444in}{0.005157in}}{\pgfqpoint{0.017396in}{0.010103in}}{\pgfqpoint{0.013749in}{0.013749in}}%
\pgfpathcurveto{\pgfqpoint{0.010103in}{0.017396in}}{\pgfqpoint{0.005157in}{0.019444in}}{\pgfqpoint{0.000000in}{0.019444in}}%
\pgfpathcurveto{\pgfqpoint{-0.005157in}{0.019444in}}{\pgfqpoint{-0.010103in}{0.017396in}}{\pgfqpoint{-0.013749in}{0.013749in}}%
\pgfpathcurveto{\pgfqpoint{-0.017396in}{0.010103in}}{\pgfqpoint{-0.019444in}{0.005157in}}{\pgfqpoint{-0.019444in}{0.000000in}}%
\pgfpathcurveto{\pgfqpoint{-0.019444in}{-0.005157in}}{\pgfqpoint{-0.017396in}{-0.010103in}}{\pgfqpoint{-0.013749in}{-0.013749in}}%
\pgfpathcurveto{\pgfqpoint{-0.010103in}{-0.017396in}}{\pgfqpoint{-0.005157in}{-0.019444in}}{\pgfqpoint{0.000000in}{-0.019444in}}%
\pgfpathclose%
\pgfusepath{stroke,fill}%
}%
\begin{pgfscope}%
\pgfsys@transformshift{3.781250in}{3.580214in}%
\pgfsys@useobject{currentmarker}{}%
\end{pgfscope}%
\end{pgfscope}%
\begin{pgfscope}%
\pgfpathrectangle{\pgfqpoint{0.100000in}{2.413063in}}{\pgfqpoint{5.037500in}{3.427208in}}%
\pgfusepath{clip}%
\pgfsetrectcap%
\pgfsetroundjoin%
\pgfsetlinewidth{1.505625pt}%
\definecolor{currentstroke}{rgb}{0.678431,1.000000,0.184314}%
\pgfsetstrokecolor{currentstroke}%
\pgfsetstrokeopacity{0.500000}%
\pgfsetdash{}{0pt}%
\pgfpathmoveto{\pgfqpoint{3.763715in}{3.841616in}}%
\pgfusepath{stroke}%
\end{pgfscope}%
\begin{pgfscope}%
\pgfpathrectangle{\pgfqpoint{0.100000in}{2.413063in}}{\pgfqpoint{5.037500in}{3.427208in}}%
\pgfusepath{clip}%
\pgfsetbuttcap%
\pgfsetroundjoin%
\definecolor{currentfill}{rgb}{0.678431,1.000000,0.184314}%
\pgfsetfillcolor{currentfill}%
\pgfsetfillopacity{0.500000}%
\pgfsetlinewidth{0.250937pt}%
\definecolor{currentstroke}{rgb}{0.000000,0.000000,0.000000}%
\pgfsetstrokecolor{currentstroke}%
\pgfsetstrokeopacity{0.500000}%
\pgfsetdash{}{0pt}%
\pgfsys@defobject{currentmarker}{\pgfqpoint{-0.075000in}{-0.075000in}}{\pgfqpoint{0.075000in}{0.075000in}}{%
\pgfpathmoveto{\pgfqpoint{0.000000in}{-0.075000in}}%
\pgfpathcurveto{\pgfqpoint{0.019890in}{-0.075000in}}{\pgfqpoint{0.038968in}{-0.067098in}}{\pgfqpoint{0.053033in}{-0.053033in}}%
\pgfpathcurveto{\pgfqpoint{0.067098in}{-0.038968in}}{\pgfqpoint{0.075000in}{-0.019890in}}{\pgfqpoint{0.075000in}{0.000000in}}%
\pgfpathcurveto{\pgfqpoint{0.075000in}{0.019890in}}{\pgfqpoint{0.067098in}{0.038968in}}{\pgfqpoint{0.053033in}{0.053033in}}%
\pgfpathcurveto{\pgfqpoint{0.038968in}{0.067098in}}{\pgfqpoint{0.019890in}{0.075000in}}{\pgfqpoint{0.000000in}{0.075000in}}%
\pgfpathcurveto{\pgfqpoint{-0.019890in}{0.075000in}}{\pgfqpoint{-0.038968in}{0.067098in}}{\pgfqpoint{-0.053033in}{0.053033in}}%
\pgfpathcurveto{\pgfqpoint{-0.067098in}{0.038968in}}{\pgfqpoint{-0.075000in}{0.019890in}}{\pgfqpoint{-0.075000in}{0.000000in}}%
\pgfpathcurveto{\pgfqpoint{-0.075000in}{-0.019890in}}{\pgfqpoint{-0.067098in}{-0.038968in}}{\pgfqpoint{-0.053033in}{-0.053033in}}%
\pgfpathcurveto{\pgfqpoint{-0.038968in}{-0.067098in}}{\pgfqpoint{-0.019890in}{-0.075000in}}{\pgfqpoint{0.000000in}{-0.075000in}}%
\pgfpathclose%
\pgfusepath{stroke,fill}%
}%
\begin{pgfscope}%
\pgfsys@transformshift{3.763715in}{3.841616in}%
\pgfsys@useobject{currentmarker}{}%
\end{pgfscope}%
\end{pgfscope}%
\begin{pgfscope}%
\pgfpathrectangle{\pgfqpoint{0.100000in}{2.413063in}}{\pgfqpoint{5.037500in}{3.427208in}}%
\pgfusepath{clip}%
\pgfsetrectcap%
\pgfsetroundjoin%
\pgfsetlinewidth{1.505625pt}%
\definecolor{currentstroke}{rgb}{0.678431,1.000000,0.184314}%
\pgfsetstrokecolor{currentstroke}%
\pgfsetstrokeopacity{0.500000}%
\pgfsetdash{}{0pt}%
\pgfpathmoveto{\pgfqpoint{3.868196in}{3.804393in}}%
\pgfusepath{stroke}%
\end{pgfscope}%
\begin{pgfscope}%
\pgfpathrectangle{\pgfqpoint{0.100000in}{2.413063in}}{\pgfqpoint{5.037500in}{3.427208in}}%
\pgfusepath{clip}%
\pgfsetbuttcap%
\pgfsetroundjoin%
\definecolor{currentfill}{rgb}{0.678431,1.000000,0.184314}%
\pgfsetfillcolor{currentfill}%
\pgfsetfillopacity{0.500000}%
\pgfsetlinewidth{0.250937pt}%
\definecolor{currentstroke}{rgb}{0.000000,0.000000,0.000000}%
\pgfsetstrokecolor{currentstroke}%
\pgfsetstrokeopacity{0.500000}%
\pgfsetdash{}{0pt}%
\pgfsys@defobject{currentmarker}{\pgfqpoint{-0.016667in}{-0.016667in}}{\pgfqpoint{0.016667in}{0.016667in}}{%
\pgfpathmoveto{\pgfqpoint{0.000000in}{-0.016667in}}%
\pgfpathcurveto{\pgfqpoint{0.004420in}{-0.016667in}}{\pgfqpoint{0.008660in}{-0.014911in}}{\pgfqpoint{0.011785in}{-0.011785in}}%
\pgfpathcurveto{\pgfqpoint{0.014911in}{-0.008660in}}{\pgfqpoint{0.016667in}{-0.004420in}}{\pgfqpoint{0.016667in}{0.000000in}}%
\pgfpathcurveto{\pgfqpoint{0.016667in}{0.004420in}}{\pgfqpoint{0.014911in}{0.008660in}}{\pgfqpoint{0.011785in}{0.011785in}}%
\pgfpathcurveto{\pgfqpoint{0.008660in}{0.014911in}}{\pgfqpoint{0.004420in}{0.016667in}}{\pgfqpoint{0.000000in}{0.016667in}}%
\pgfpathcurveto{\pgfqpoint{-0.004420in}{0.016667in}}{\pgfqpoint{-0.008660in}{0.014911in}}{\pgfqpoint{-0.011785in}{0.011785in}}%
\pgfpathcurveto{\pgfqpoint{-0.014911in}{0.008660in}}{\pgfqpoint{-0.016667in}{0.004420in}}{\pgfqpoint{-0.016667in}{0.000000in}}%
\pgfpathcurveto{\pgfqpoint{-0.016667in}{-0.004420in}}{\pgfqpoint{-0.014911in}{-0.008660in}}{\pgfqpoint{-0.011785in}{-0.011785in}}%
\pgfpathcurveto{\pgfqpoint{-0.008660in}{-0.014911in}}{\pgfqpoint{-0.004420in}{-0.016667in}}{\pgfqpoint{0.000000in}{-0.016667in}}%
\pgfpathclose%
\pgfusepath{stroke,fill}%
}%
\begin{pgfscope}%
\pgfsys@transformshift{3.868196in}{3.804393in}%
\pgfsys@useobject{currentmarker}{}%
\end{pgfscope}%
\end{pgfscope}%
\begin{pgfscope}%
\pgfpathrectangle{\pgfqpoint{0.100000in}{2.413063in}}{\pgfqpoint{5.037500in}{3.427208in}}%
\pgfusepath{clip}%
\pgfsetrectcap%
\pgfsetroundjoin%
\pgfsetlinewidth{1.505625pt}%
\definecolor{currentstroke}{rgb}{0.678431,1.000000,0.184314}%
\pgfsetstrokecolor{currentstroke}%
\pgfsetstrokeopacity{0.500000}%
\pgfsetdash{}{0pt}%
\pgfpathmoveto{\pgfqpoint{4.120605in}{3.552479in}}%
\pgfusepath{stroke}%
\end{pgfscope}%
\begin{pgfscope}%
\pgfpathrectangle{\pgfqpoint{0.100000in}{2.413063in}}{\pgfqpoint{5.037500in}{3.427208in}}%
\pgfusepath{clip}%
\pgfsetbuttcap%
\pgfsetroundjoin%
\definecolor{currentfill}{rgb}{0.678431,1.000000,0.184314}%
\pgfsetfillcolor{currentfill}%
\pgfsetfillopacity{0.500000}%
\pgfsetlinewidth{0.250937pt}%
\definecolor{currentstroke}{rgb}{0.000000,0.000000,0.000000}%
\pgfsetstrokecolor{currentstroke}%
\pgfsetstrokeopacity{0.500000}%
\pgfsetdash{}{0pt}%
\pgfsys@defobject{currentmarker}{\pgfqpoint{-0.027778in}{-0.027778in}}{\pgfqpoint{0.027778in}{0.027778in}}{%
\pgfpathmoveto{\pgfqpoint{0.000000in}{-0.027778in}}%
\pgfpathcurveto{\pgfqpoint{0.007367in}{-0.027778in}}{\pgfqpoint{0.014433in}{-0.024851in}}{\pgfqpoint{0.019642in}{-0.019642in}}%
\pgfpathcurveto{\pgfqpoint{0.024851in}{-0.014433in}}{\pgfqpoint{0.027778in}{-0.007367in}}{\pgfqpoint{0.027778in}{0.000000in}}%
\pgfpathcurveto{\pgfqpoint{0.027778in}{0.007367in}}{\pgfqpoint{0.024851in}{0.014433in}}{\pgfqpoint{0.019642in}{0.019642in}}%
\pgfpathcurveto{\pgfqpoint{0.014433in}{0.024851in}}{\pgfqpoint{0.007367in}{0.027778in}}{\pgfqpoint{0.000000in}{0.027778in}}%
\pgfpathcurveto{\pgfqpoint{-0.007367in}{0.027778in}}{\pgfqpoint{-0.014433in}{0.024851in}}{\pgfqpoint{-0.019642in}{0.019642in}}%
\pgfpathcurveto{\pgfqpoint{-0.024851in}{0.014433in}}{\pgfqpoint{-0.027778in}{0.007367in}}{\pgfqpoint{-0.027778in}{0.000000in}}%
\pgfpathcurveto{\pgfqpoint{-0.027778in}{-0.007367in}}{\pgfqpoint{-0.024851in}{-0.014433in}}{\pgfqpoint{-0.019642in}{-0.019642in}}%
\pgfpathcurveto{\pgfqpoint{-0.014433in}{-0.024851in}}{\pgfqpoint{-0.007367in}{-0.027778in}}{\pgfqpoint{0.000000in}{-0.027778in}}%
\pgfpathclose%
\pgfusepath{stroke,fill}%
}%
\begin{pgfscope}%
\pgfsys@transformshift{4.120605in}{3.552479in}%
\pgfsys@useobject{currentmarker}{}%
\end{pgfscope}%
\end{pgfscope}%
\begin{pgfscope}%
\pgfpathrectangle{\pgfqpoint{0.100000in}{2.413063in}}{\pgfqpoint{5.037500in}{3.427208in}}%
\pgfusepath{clip}%
\pgfsetrectcap%
\pgfsetroundjoin%
\pgfsetlinewidth{1.505625pt}%
\definecolor{currentstroke}{rgb}{0.678431,1.000000,0.184314}%
\pgfsetstrokecolor{currentstroke}%
\pgfsetstrokeopacity{0.500000}%
\pgfsetdash{}{0pt}%
\pgfpathmoveto{\pgfqpoint{3.907380in}{3.639417in}}%
\pgfusepath{stroke}%
\end{pgfscope}%
\begin{pgfscope}%
\pgfpathrectangle{\pgfqpoint{0.100000in}{2.413063in}}{\pgfqpoint{5.037500in}{3.427208in}}%
\pgfusepath{clip}%
\pgfsetbuttcap%
\pgfsetroundjoin%
\definecolor{currentfill}{rgb}{0.678431,1.000000,0.184314}%
\pgfsetfillcolor{currentfill}%
\pgfsetfillopacity{0.500000}%
\pgfsetlinewidth{0.250937pt}%
\definecolor{currentstroke}{rgb}{0.000000,0.000000,0.000000}%
\pgfsetstrokecolor{currentstroke}%
\pgfsetstrokeopacity{0.500000}%
\pgfsetdash{}{0pt}%
\pgfsys@defobject{currentmarker}{\pgfqpoint{-0.019444in}{-0.019444in}}{\pgfqpoint{0.019444in}{0.019444in}}{%
\pgfpathmoveto{\pgfqpoint{0.000000in}{-0.019444in}}%
\pgfpathcurveto{\pgfqpoint{0.005157in}{-0.019444in}}{\pgfqpoint{0.010103in}{-0.017396in}}{\pgfqpoint{0.013749in}{-0.013749in}}%
\pgfpathcurveto{\pgfqpoint{0.017396in}{-0.010103in}}{\pgfqpoint{0.019444in}{-0.005157in}}{\pgfqpoint{0.019444in}{0.000000in}}%
\pgfpathcurveto{\pgfqpoint{0.019444in}{0.005157in}}{\pgfqpoint{0.017396in}{0.010103in}}{\pgfqpoint{0.013749in}{0.013749in}}%
\pgfpathcurveto{\pgfqpoint{0.010103in}{0.017396in}}{\pgfqpoint{0.005157in}{0.019444in}}{\pgfqpoint{0.000000in}{0.019444in}}%
\pgfpathcurveto{\pgfqpoint{-0.005157in}{0.019444in}}{\pgfqpoint{-0.010103in}{0.017396in}}{\pgfqpoint{-0.013749in}{0.013749in}}%
\pgfpathcurveto{\pgfqpoint{-0.017396in}{0.010103in}}{\pgfqpoint{-0.019444in}{0.005157in}}{\pgfqpoint{-0.019444in}{0.000000in}}%
\pgfpathcurveto{\pgfqpoint{-0.019444in}{-0.005157in}}{\pgfqpoint{-0.017396in}{-0.010103in}}{\pgfqpoint{-0.013749in}{-0.013749in}}%
\pgfpathcurveto{\pgfqpoint{-0.010103in}{-0.017396in}}{\pgfqpoint{-0.005157in}{-0.019444in}}{\pgfqpoint{0.000000in}{-0.019444in}}%
\pgfpathclose%
\pgfusepath{stroke,fill}%
}%
\begin{pgfscope}%
\pgfsys@transformshift{3.907380in}{3.639417in}%
\pgfsys@useobject{currentmarker}{}%
\end{pgfscope}%
\end{pgfscope}%
\begin{pgfscope}%
\pgfpathrectangle{\pgfqpoint{0.100000in}{2.413063in}}{\pgfqpoint{5.037500in}{3.427208in}}%
\pgfusepath{clip}%
\pgfsetrectcap%
\pgfsetroundjoin%
\pgfsetlinewidth{1.505625pt}%
\definecolor{currentstroke}{rgb}{0.678431,1.000000,0.184314}%
\pgfsetstrokecolor{currentstroke}%
\pgfsetstrokeopacity{0.500000}%
\pgfsetdash{}{0pt}%
\pgfpathmoveto{\pgfqpoint{3.741599in}{3.784942in}}%
\pgfusepath{stroke}%
\end{pgfscope}%
\begin{pgfscope}%
\pgfpathrectangle{\pgfqpoint{0.100000in}{2.413063in}}{\pgfqpoint{5.037500in}{3.427208in}}%
\pgfusepath{clip}%
\pgfsetbuttcap%
\pgfsetroundjoin%
\definecolor{currentfill}{rgb}{0.678431,1.000000,0.184314}%
\pgfsetfillcolor{currentfill}%
\pgfsetfillopacity{0.500000}%
\pgfsetlinewidth{0.250937pt}%
\definecolor{currentstroke}{rgb}{0.000000,0.000000,0.000000}%
\pgfsetstrokecolor{currentstroke}%
\pgfsetstrokeopacity{0.500000}%
\pgfsetdash{}{0pt}%
\pgfsys@defobject{currentmarker}{\pgfqpoint{-0.050000in}{-0.050000in}}{\pgfqpoint{0.050000in}{0.050000in}}{%
\pgfpathmoveto{\pgfqpoint{0.000000in}{-0.050000in}}%
\pgfpathcurveto{\pgfqpoint{0.013260in}{-0.050000in}}{\pgfqpoint{0.025979in}{-0.044732in}}{\pgfqpoint{0.035355in}{-0.035355in}}%
\pgfpathcurveto{\pgfqpoint{0.044732in}{-0.025979in}}{\pgfqpoint{0.050000in}{-0.013260in}}{\pgfqpoint{0.050000in}{0.000000in}}%
\pgfpathcurveto{\pgfqpoint{0.050000in}{0.013260in}}{\pgfqpoint{0.044732in}{0.025979in}}{\pgfqpoint{0.035355in}{0.035355in}}%
\pgfpathcurveto{\pgfqpoint{0.025979in}{0.044732in}}{\pgfqpoint{0.013260in}{0.050000in}}{\pgfqpoint{0.000000in}{0.050000in}}%
\pgfpathcurveto{\pgfqpoint{-0.013260in}{0.050000in}}{\pgfqpoint{-0.025979in}{0.044732in}}{\pgfqpoint{-0.035355in}{0.035355in}}%
\pgfpathcurveto{\pgfqpoint{-0.044732in}{0.025979in}}{\pgfqpoint{-0.050000in}{0.013260in}}{\pgfqpoint{-0.050000in}{0.000000in}}%
\pgfpathcurveto{\pgfqpoint{-0.050000in}{-0.013260in}}{\pgfqpoint{-0.044732in}{-0.025979in}}{\pgfqpoint{-0.035355in}{-0.035355in}}%
\pgfpathcurveto{\pgfqpoint{-0.025979in}{-0.044732in}}{\pgfqpoint{-0.013260in}{-0.050000in}}{\pgfqpoint{0.000000in}{-0.050000in}}%
\pgfpathclose%
\pgfusepath{stroke,fill}%
}%
\begin{pgfscope}%
\pgfsys@transformshift{3.741599in}{3.784942in}%
\pgfsys@useobject{currentmarker}{}%
\end{pgfscope}%
\end{pgfscope}%
\begin{pgfscope}%
\pgfpathrectangle{\pgfqpoint{0.100000in}{2.413063in}}{\pgfqpoint{5.037500in}{3.427208in}}%
\pgfusepath{clip}%
\pgfsetrectcap%
\pgfsetroundjoin%
\pgfsetlinewidth{1.505625pt}%
\definecolor{currentstroke}{rgb}{0.678431,1.000000,0.184314}%
\pgfsetstrokecolor{currentstroke}%
\pgfsetstrokeopacity{0.500000}%
\pgfsetdash{}{0pt}%
\pgfpathmoveto{\pgfqpoint{4.164798in}{3.586837in}}%
\pgfusepath{stroke}%
\end{pgfscope}%
\begin{pgfscope}%
\pgfpathrectangle{\pgfqpoint{0.100000in}{2.413063in}}{\pgfqpoint{5.037500in}{3.427208in}}%
\pgfusepath{clip}%
\pgfsetbuttcap%
\pgfsetroundjoin%
\definecolor{currentfill}{rgb}{0.678431,1.000000,0.184314}%
\pgfsetfillcolor{currentfill}%
\pgfsetfillopacity{0.500000}%
\pgfsetlinewidth{0.250937pt}%
\definecolor{currentstroke}{rgb}{0.000000,0.000000,0.000000}%
\pgfsetstrokecolor{currentstroke}%
\pgfsetstrokeopacity{0.500000}%
\pgfsetdash{}{0pt}%
\pgfsys@defobject{currentmarker}{\pgfqpoint{-0.013889in}{-0.013889in}}{\pgfqpoint{0.013889in}{0.013889in}}{%
\pgfpathmoveto{\pgfqpoint{0.000000in}{-0.013889in}}%
\pgfpathcurveto{\pgfqpoint{0.003683in}{-0.013889in}}{\pgfqpoint{0.007216in}{-0.012425in}}{\pgfqpoint{0.009821in}{-0.009821in}}%
\pgfpathcurveto{\pgfqpoint{0.012425in}{-0.007216in}}{\pgfqpoint{0.013889in}{-0.003683in}}{\pgfqpoint{0.013889in}{0.000000in}}%
\pgfpathcurveto{\pgfqpoint{0.013889in}{0.003683in}}{\pgfqpoint{0.012425in}{0.007216in}}{\pgfqpoint{0.009821in}{0.009821in}}%
\pgfpathcurveto{\pgfqpoint{0.007216in}{0.012425in}}{\pgfqpoint{0.003683in}{0.013889in}}{\pgfqpoint{0.000000in}{0.013889in}}%
\pgfpathcurveto{\pgfqpoint{-0.003683in}{0.013889in}}{\pgfqpoint{-0.007216in}{0.012425in}}{\pgfqpoint{-0.009821in}{0.009821in}}%
\pgfpathcurveto{\pgfqpoint{-0.012425in}{0.007216in}}{\pgfqpoint{-0.013889in}{0.003683in}}{\pgfqpoint{-0.013889in}{0.000000in}}%
\pgfpathcurveto{\pgfqpoint{-0.013889in}{-0.003683in}}{\pgfqpoint{-0.012425in}{-0.007216in}}{\pgfqpoint{-0.009821in}{-0.009821in}}%
\pgfpathcurveto{\pgfqpoint{-0.007216in}{-0.012425in}}{\pgfqpoint{-0.003683in}{-0.013889in}}{\pgfqpoint{0.000000in}{-0.013889in}}%
\pgfpathclose%
\pgfusepath{stroke,fill}%
}%
\begin{pgfscope}%
\pgfsys@transformshift{4.164798in}{3.586837in}%
\pgfsys@useobject{currentmarker}{}%
\end{pgfscope}%
\end{pgfscope}%
\begin{pgfscope}%
\pgfpathrectangle{\pgfqpoint{0.100000in}{2.413063in}}{\pgfqpoint{5.037500in}{3.427208in}}%
\pgfusepath{clip}%
\pgfsetrectcap%
\pgfsetroundjoin%
\pgfsetlinewidth{1.505625pt}%
\definecolor{currentstroke}{rgb}{0.678431,1.000000,0.184314}%
\pgfsetstrokecolor{currentstroke}%
\pgfsetstrokeopacity{0.500000}%
\pgfsetdash{}{0pt}%
\pgfpathmoveto{\pgfqpoint{3.971514in}{3.412687in}}%
\pgfusepath{stroke}%
\end{pgfscope}%
\begin{pgfscope}%
\pgfpathrectangle{\pgfqpoint{0.100000in}{2.413063in}}{\pgfqpoint{5.037500in}{3.427208in}}%
\pgfusepath{clip}%
\pgfsetbuttcap%
\pgfsetroundjoin%
\definecolor{currentfill}{rgb}{0.678431,1.000000,0.184314}%
\pgfsetfillcolor{currentfill}%
\pgfsetfillopacity{0.500000}%
\pgfsetlinewidth{0.250937pt}%
\definecolor{currentstroke}{rgb}{0.000000,0.000000,0.000000}%
\pgfsetstrokecolor{currentstroke}%
\pgfsetstrokeopacity{0.500000}%
\pgfsetdash{}{0pt}%
\pgfsys@defobject{currentmarker}{\pgfqpoint{-0.025000in}{-0.025000in}}{\pgfqpoint{0.025000in}{0.025000in}}{%
\pgfpathmoveto{\pgfqpoint{0.000000in}{-0.025000in}}%
\pgfpathcurveto{\pgfqpoint{0.006630in}{-0.025000in}}{\pgfqpoint{0.012989in}{-0.022366in}}{\pgfqpoint{0.017678in}{-0.017678in}}%
\pgfpathcurveto{\pgfqpoint{0.022366in}{-0.012989in}}{\pgfqpoint{0.025000in}{-0.006630in}}{\pgfqpoint{0.025000in}{0.000000in}}%
\pgfpathcurveto{\pgfqpoint{0.025000in}{0.006630in}}{\pgfqpoint{0.022366in}{0.012989in}}{\pgfqpoint{0.017678in}{0.017678in}}%
\pgfpathcurveto{\pgfqpoint{0.012989in}{0.022366in}}{\pgfqpoint{0.006630in}{0.025000in}}{\pgfqpoint{0.000000in}{0.025000in}}%
\pgfpathcurveto{\pgfqpoint{-0.006630in}{0.025000in}}{\pgfqpoint{-0.012989in}{0.022366in}}{\pgfqpoint{-0.017678in}{0.017678in}}%
\pgfpathcurveto{\pgfqpoint{-0.022366in}{0.012989in}}{\pgfqpoint{-0.025000in}{0.006630in}}{\pgfqpoint{-0.025000in}{0.000000in}}%
\pgfpathcurveto{\pgfqpoint{-0.025000in}{-0.006630in}}{\pgfqpoint{-0.022366in}{-0.012989in}}{\pgfqpoint{-0.017678in}{-0.017678in}}%
\pgfpathcurveto{\pgfqpoint{-0.012989in}{-0.022366in}}{\pgfqpoint{-0.006630in}{-0.025000in}}{\pgfqpoint{0.000000in}{-0.025000in}}%
\pgfpathclose%
\pgfusepath{stroke,fill}%
}%
\begin{pgfscope}%
\pgfsys@transformshift{3.971514in}{3.412687in}%
\pgfsys@useobject{currentmarker}{}%
\end{pgfscope}%
\end{pgfscope}%
\begin{pgfscope}%
\pgfpathrectangle{\pgfqpoint{0.100000in}{2.413063in}}{\pgfqpoint{5.037500in}{3.427208in}}%
\pgfusepath{clip}%
\pgfsetrectcap%
\pgfsetroundjoin%
\pgfsetlinewidth{1.505625pt}%
\definecolor{currentstroke}{rgb}{0.678431,1.000000,0.184314}%
\pgfsetstrokecolor{currentstroke}%
\pgfsetstrokeopacity{0.500000}%
\pgfsetdash{}{0pt}%
\pgfpathmoveto{\pgfqpoint{3.911534in}{3.611641in}}%
\pgfusepath{stroke}%
\end{pgfscope}%
\begin{pgfscope}%
\pgfpathrectangle{\pgfqpoint{0.100000in}{2.413063in}}{\pgfqpoint{5.037500in}{3.427208in}}%
\pgfusepath{clip}%
\pgfsetbuttcap%
\pgfsetroundjoin%
\definecolor{currentfill}{rgb}{0.678431,1.000000,0.184314}%
\pgfsetfillcolor{currentfill}%
\pgfsetfillopacity{0.500000}%
\pgfsetlinewidth{0.250937pt}%
\definecolor{currentstroke}{rgb}{0.000000,0.000000,0.000000}%
\pgfsetstrokecolor{currentstroke}%
\pgfsetstrokeopacity{0.500000}%
\pgfsetdash{}{0pt}%
\pgfsys@defobject{currentmarker}{\pgfqpoint{-0.022222in}{-0.022222in}}{\pgfqpoint{0.022222in}{0.022222in}}{%
\pgfpathmoveto{\pgfqpoint{0.000000in}{-0.022222in}}%
\pgfpathcurveto{\pgfqpoint{0.005893in}{-0.022222in}}{\pgfqpoint{0.011546in}{-0.019881in}}{\pgfqpoint{0.015713in}{-0.015713in}}%
\pgfpathcurveto{\pgfqpoint{0.019881in}{-0.011546in}}{\pgfqpoint{0.022222in}{-0.005893in}}{\pgfqpoint{0.022222in}{0.000000in}}%
\pgfpathcurveto{\pgfqpoint{0.022222in}{0.005893in}}{\pgfqpoint{0.019881in}{0.011546in}}{\pgfqpoint{0.015713in}{0.015713in}}%
\pgfpathcurveto{\pgfqpoint{0.011546in}{0.019881in}}{\pgfqpoint{0.005893in}{0.022222in}}{\pgfqpoint{0.000000in}{0.022222in}}%
\pgfpathcurveto{\pgfqpoint{-0.005893in}{0.022222in}}{\pgfqpoint{-0.011546in}{0.019881in}}{\pgfqpoint{-0.015713in}{0.015713in}}%
\pgfpathcurveto{\pgfqpoint{-0.019881in}{0.011546in}}{\pgfqpoint{-0.022222in}{0.005893in}}{\pgfqpoint{-0.022222in}{0.000000in}}%
\pgfpathcurveto{\pgfqpoint{-0.022222in}{-0.005893in}}{\pgfqpoint{-0.019881in}{-0.011546in}}{\pgfqpoint{-0.015713in}{-0.015713in}}%
\pgfpathcurveto{\pgfqpoint{-0.011546in}{-0.019881in}}{\pgfqpoint{-0.005893in}{-0.022222in}}{\pgfqpoint{0.000000in}{-0.022222in}}%
\pgfpathclose%
\pgfusepath{stroke,fill}%
}%
\begin{pgfscope}%
\pgfsys@transformshift{3.911534in}{3.611641in}%
\pgfsys@useobject{currentmarker}{}%
\end{pgfscope}%
\end{pgfscope}%
\begin{pgfscope}%
\pgfpathrectangle{\pgfqpoint{0.100000in}{2.413063in}}{\pgfqpoint{5.037500in}{3.427208in}}%
\pgfusepath{clip}%
\pgfsetrectcap%
\pgfsetroundjoin%
\pgfsetlinewidth{1.505625pt}%
\definecolor{currentstroke}{rgb}{0.000000,0.000000,1.000000}%
\pgfsetstrokecolor{currentstroke}%
\pgfsetstrokeopacity{0.500000}%
\pgfsetdash{}{0pt}%
\pgfpathmoveto{\pgfqpoint{1.800904in}{2.961134in}}%
\pgfusepath{stroke}%
\end{pgfscope}%
\begin{pgfscope}%
\pgfpathrectangle{\pgfqpoint{0.100000in}{2.413063in}}{\pgfqpoint{5.037500in}{3.427208in}}%
\pgfusepath{clip}%
\pgfsetbuttcap%
\pgfsetroundjoin%
\definecolor{currentfill}{rgb}{0.000000,0.000000,1.000000}%
\pgfsetfillcolor{currentfill}%
\pgfsetfillopacity{0.500000}%
\pgfsetlinewidth{0.250937pt}%
\definecolor{currentstroke}{rgb}{0.000000,0.000000,0.000000}%
\pgfsetstrokecolor{currentstroke}%
\pgfsetstrokeopacity{0.500000}%
\pgfsetdash{}{0pt}%
\pgfsys@defobject{currentmarker}{\pgfqpoint{-0.125000in}{-0.125000in}}{\pgfqpoint{0.125000in}{0.125000in}}{%
\pgfpathmoveto{\pgfqpoint{0.000000in}{-0.125000in}}%
\pgfpathcurveto{\pgfqpoint{0.033150in}{-0.125000in}}{\pgfqpoint{0.064947in}{-0.111829in}}{\pgfqpoint{0.088388in}{-0.088388in}}%
\pgfpathcurveto{\pgfqpoint{0.111829in}{-0.064947in}}{\pgfqpoint{0.125000in}{-0.033150in}}{\pgfqpoint{0.125000in}{0.000000in}}%
\pgfpathcurveto{\pgfqpoint{0.125000in}{0.033150in}}{\pgfqpoint{0.111829in}{0.064947in}}{\pgfqpoint{0.088388in}{0.088388in}}%
\pgfpathcurveto{\pgfqpoint{0.064947in}{0.111829in}}{\pgfqpoint{0.033150in}{0.125000in}}{\pgfqpoint{0.000000in}{0.125000in}}%
\pgfpathcurveto{\pgfqpoint{-0.033150in}{0.125000in}}{\pgfqpoint{-0.064947in}{0.111829in}}{\pgfqpoint{-0.088388in}{0.088388in}}%
\pgfpathcurveto{\pgfqpoint{-0.111829in}{0.064947in}}{\pgfqpoint{-0.125000in}{0.033150in}}{\pgfqpoint{-0.125000in}{0.000000in}}%
\pgfpathcurveto{\pgfqpoint{-0.125000in}{-0.033150in}}{\pgfqpoint{-0.111829in}{-0.064947in}}{\pgfqpoint{-0.088388in}{-0.088388in}}%
\pgfpathcurveto{\pgfqpoint{-0.064947in}{-0.111829in}}{\pgfqpoint{-0.033150in}{-0.125000in}}{\pgfqpoint{0.000000in}{-0.125000in}}%
\pgfpathclose%
\pgfusepath{stroke,fill}%
}%
\begin{pgfscope}%
\pgfsys@transformshift{1.800904in}{2.961134in}%
\pgfsys@useobject{currentmarker}{}%
\end{pgfscope}%
\end{pgfscope}%
\begin{pgfscope}%
\pgfpathrectangle{\pgfqpoint{0.100000in}{2.413063in}}{\pgfqpoint{5.037500in}{3.427208in}}%
\pgfusepath{clip}%
\pgfsetrectcap%
\pgfsetroundjoin%
\pgfsetlinewidth{1.505625pt}%
\definecolor{currentstroke}{rgb}{0.000000,0.000000,1.000000}%
\pgfsetstrokecolor{currentstroke}%
\pgfsetstrokeopacity{0.500000}%
\pgfsetdash{}{0pt}%
\pgfpathmoveto{\pgfqpoint{1.714465in}{3.099522in}}%
\pgfusepath{stroke}%
\end{pgfscope}%
\begin{pgfscope}%
\pgfpathrectangle{\pgfqpoint{0.100000in}{2.413063in}}{\pgfqpoint{5.037500in}{3.427208in}}%
\pgfusepath{clip}%
\pgfsetbuttcap%
\pgfsetroundjoin%
\definecolor{currentfill}{rgb}{0.000000,0.000000,1.000000}%
\pgfsetfillcolor{currentfill}%
\pgfsetfillopacity{0.500000}%
\pgfsetlinewidth{0.250937pt}%
\definecolor{currentstroke}{rgb}{0.000000,0.000000,0.000000}%
\pgfsetstrokecolor{currentstroke}%
\pgfsetstrokeopacity{0.500000}%
\pgfsetdash{}{0pt}%
\pgfsys@defobject{currentmarker}{\pgfqpoint{-0.077778in}{-0.077778in}}{\pgfqpoint{0.077778in}{0.077778in}}{%
\pgfpathmoveto{\pgfqpoint{0.000000in}{-0.077778in}}%
\pgfpathcurveto{\pgfqpoint{0.020627in}{-0.077778in}}{\pgfqpoint{0.040412in}{-0.069583in}}{\pgfqpoint{0.054997in}{-0.054997in}}%
\pgfpathcurveto{\pgfqpoint{0.069583in}{-0.040412in}}{\pgfqpoint{0.077778in}{-0.020627in}}{\pgfqpoint{0.077778in}{0.000000in}}%
\pgfpathcurveto{\pgfqpoint{0.077778in}{0.020627in}}{\pgfqpoint{0.069583in}{0.040412in}}{\pgfqpoint{0.054997in}{0.054997in}}%
\pgfpathcurveto{\pgfqpoint{0.040412in}{0.069583in}}{\pgfqpoint{0.020627in}{0.077778in}}{\pgfqpoint{0.000000in}{0.077778in}}%
\pgfpathcurveto{\pgfqpoint{-0.020627in}{0.077778in}}{\pgfqpoint{-0.040412in}{0.069583in}}{\pgfqpoint{-0.054997in}{0.054997in}}%
\pgfpathcurveto{\pgfqpoint{-0.069583in}{0.040412in}}{\pgfqpoint{-0.077778in}{0.020627in}}{\pgfqpoint{-0.077778in}{0.000000in}}%
\pgfpathcurveto{\pgfqpoint{-0.077778in}{-0.020627in}}{\pgfqpoint{-0.069583in}{-0.040412in}}{\pgfqpoint{-0.054997in}{-0.054997in}}%
\pgfpathcurveto{\pgfqpoint{-0.040412in}{-0.069583in}}{\pgfqpoint{-0.020627in}{-0.077778in}}{\pgfqpoint{0.000000in}{-0.077778in}}%
\pgfpathclose%
\pgfusepath{stroke,fill}%
}%
\begin{pgfscope}%
\pgfsys@transformshift{1.714465in}{3.099522in}%
\pgfsys@useobject{currentmarker}{}%
\end{pgfscope}%
\end{pgfscope}%
\begin{pgfscope}%
\pgfpathrectangle{\pgfqpoint{0.100000in}{2.413063in}}{\pgfqpoint{5.037500in}{3.427208in}}%
\pgfusepath{clip}%
\pgfsetrectcap%
\pgfsetroundjoin%
\pgfsetlinewidth{1.505625pt}%
\definecolor{currentstroke}{rgb}{0.678431,1.000000,0.184314}%
\pgfsetstrokecolor{currentstroke}%
\pgfsetstrokeopacity{0.500000}%
\pgfsetdash{}{0pt}%
\pgfpathmoveto{\pgfqpoint{1.043203in}{5.017798in}}%
\pgfusepath{stroke}%
\end{pgfscope}%
\begin{pgfscope}%
\pgfpathrectangle{\pgfqpoint{0.100000in}{2.413063in}}{\pgfqpoint{5.037500in}{3.427208in}}%
\pgfusepath{clip}%
\pgfsetbuttcap%
\pgfsetroundjoin%
\definecolor{currentfill}{rgb}{0.678431,1.000000,0.184314}%
\pgfsetfillcolor{currentfill}%
\pgfsetfillopacity{0.500000}%
\pgfsetlinewidth{0.250937pt}%
\definecolor{currentstroke}{rgb}{0.000000,0.000000,0.000000}%
\pgfsetstrokecolor{currentstroke}%
\pgfsetstrokeopacity{0.500000}%
\pgfsetdash{}{0pt}%
\pgfsys@defobject{currentmarker}{\pgfqpoint{-0.013889in}{-0.013889in}}{\pgfqpoint{0.013889in}{0.013889in}}{%
\pgfpathmoveto{\pgfqpoint{0.000000in}{-0.013889in}}%
\pgfpathcurveto{\pgfqpoint{0.003683in}{-0.013889in}}{\pgfqpoint{0.007216in}{-0.012425in}}{\pgfqpoint{0.009821in}{-0.009821in}}%
\pgfpathcurveto{\pgfqpoint{0.012425in}{-0.007216in}}{\pgfqpoint{0.013889in}{-0.003683in}}{\pgfqpoint{0.013889in}{0.000000in}}%
\pgfpathcurveto{\pgfqpoint{0.013889in}{0.003683in}}{\pgfqpoint{0.012425in}{0.007216in}}{\pgfqpoint{0.009821in}{0.009821in}}%
\pgfpathcurveto{\pgfqpoint{0.007216in}{0.012425in}}{\pgfqpoint{0.003683in}{0.013889in}}{\pgfqpoint{0.000000in}{0.013889in}}%
\pgfpathcurveto{\pgfqpoint{-0.003683in}{0.013889in}}{\pgfqpoint{-0.007216in}{0.012425in}}{\pgfqpoint{-0.009821in}{0.009821in}}%
\pgfpathcurveto{\pgfqpoint{-0.012425in}{0.007216in}}{\pgfqpoint{-0.013889in}{0.003683in}}{\pgfqpoint{-0.013889in}{0.000000in}}%
\pgfpathcurveto{\pgfqpoint{-0.013889in}{-0.003683in}}{\pgfqpoint{-0.012425in}{-0.007216in}}{\pgfqpoint{-0.009821in}{-0.009821in}}%
\pgfpathcurveto{\pgfqpoint{-0.007216in}{-0.012425in}}{\pgfqpoint{-0.003683in}{-0.013889in}}{\pgfqpoint{0.000000in}{-0.013889in}}%
\pgfpathclose%
\pgfusepath{stroke,fill}%
}%
\begin{pgfscope}%
\pgfsys@transformshift{1.043203in}{5.017798in}%
\pgfsys@useobject{currentmarker}{}%
\end{pgfscope}%
\end{pgfscope}%
\begin{pgfscope}%
\pgfpathrectangle{\pgfqpoint{0.100000in}{2.413063in}}{\pgfqpoint{5.037500in}{3.427208in}}%
\pgfusepath{clip}%
\pgfsetrectcap%
\pgfsetroundjoin%
\pgfsetlinewidth{1.505625pt}%
\definecolor{currentstroke}{rgb}{0.678431,1.000000,0.184314}%
\pgfsetstrokecolor{currentstroke}%
\pgfsetstrokeopacity{0.500000}%
\pgfsetdash{}{0pt}%
\pgfpathmoveto{\pgfqpoint{1.111792in}{5.487791in}}%
\pgfusepath{stroke}%
\end{pgfscope}%
\begin{pgfscope}%
\pgfpathrectangle{\pgfqpoint{0.100000in}{2.413063in}}{\pgfqpoint{5.037500in}{3.427208in}}%
\pgfusepath{clip}%
\pgfsetbuttcap%
\pgfsetroundjoin%
\definecolor{currentfill}{rgb}{0.678431,1.000000,0.184314}%
\pgfsetfillcolor{currentfill}%
\pgfsetfillopacity{0.500000}%
\pgfsetlinewidth{0.250937pt}%
\definecolor{currentstroke}{rgb}{0.000000,0.000000,0.000000}%
\pgfsetstrokecolor{currentstroke}%
\pgfsetstrokeopacity{0.500000}%
\pgfsetdash{}{0pt}%
\pgfsys@defobject{currentmarker}{\pgfqpoint{-0.030556in}{-0.030556in}}{\pgfqpoint{0.030556in}{0.030556in}}{%
\pgfpathmoveto{\pgfqpoint{0.000000in}{-0.030556in}}%
\pgfpathcurveto{\pgfqpoint{0.008103in}{-0.030556in}}{\pgfqpoint{0.015876in}{-0.027336in}}{\pgfqpoint{0.021606in}{-0.021606in}}%
\pgfpathcurveto{\pgfqpoint{0.027336in}{-0.015876in}}{\pgfqpoint{0.030556in}{-0.008103in}}{\pgfqpoint{0.030556in}{0.000000in}}%
\pgfpathcurveto{\pgfqpoint{0.030556in}{0.008103in}}{\pgfqpoint{0.027336in}{0.015876in}}{\pgfqpoint{0.021606in}{0.021606in}}%
\pgfpathcurveto{\pgfqpoint{0.015876in}{0.027336in}}{\pgfqpoint{0.008103in}{0.030556in}}{\pgfqpoint{0.000000in}{0.030556in}}%
\pgfpathcurveto{\pgfqpoint{-0.008103in}{0.030556in}}{\pgfqpoint{-0.015876in}{0.027336in}}{\pgfqpoint{-0.021606in}{0.021606in}}%
\pgfpathcurveto{\pgfqpoint{-0.027336in}{0.015876in}}{\pgfqpoint{-0.030556in}{0.008103in}}{\pgfqpoint{-0.030556in}{0.000000in}}%
\pgfpathcurveto{\pgfqpoint{-0.030556in}{-0.008103in}}{\pgfqpoint{-0.027336in}{-0.015876in}}{\pgfqpoint{-0.021606in}{-0.021606in}}%
\pgfpathcurveto{\pgfqpoint{-0.015876in}{-0.027336in}}{\pgfqpoint{-0.008103in}{-0.030556in}}{\pgfqpoint{0.000000in}{-0.030556in}}%
\pgfpathclose%
\pgfusepath{stroke,fill}%
}%
\begin{pgfscope}%
\pgfsys@transformshift{1.111792in}{5.487791in}%
\pgfsys@useobject{currentmarker}{}%
\end{pgfscope}%
\end{pgfscope}%
\begin{pgfscope}%
\pgfpathrectangle{\pgfqpoint{0.100000in}{2.413063in}}{\pgfqpoint{5.037500in}{3.427208in}}%
\pgfusepath{clip}%
\pgfsetrectcap%
\pgfsetroundjoin%
\pgfsetlinewidth{1.505625pt}%
\definecolor{currentstroke}{rgb}{0.678431,1.000000,0.184314}%
\pgfsetstrokecolor{currentstroke}%
\pgfsetstrokeopacity{0.500000}%
\pgfsetdash{}{0pt}%
\pgfpathmoveto{\pgfqpoint{1.384741in}{4.932340in}}%
\pgfusepath{stroke}%
\end{pgfscope}%
\begin{pgfscope}%
\pgfpathrectangle{\pgfqpoint{0.100000in}{2.413063in}}{\pgfqpoint{5.037500in}{3.427208in}}%
\pgfusepath{clip}%
\pgfsetbuttcap%
\pgfsetroundjoin%
\definecolor{currentfill}{rgb}{0.678431,1.000000,0.184314}%
\pgfsetfillcolor{currentfill}%
\pgfsetfillopacity{0.500000}%
\pgfsetlinewidth{0.250937pt}%
\definecolor{currentstroke}{rgb}{0.000000,0.000000,0.000000}%
\pgfsetstrokecolor{currentstroke}%
\pgfsetstrokeopacity{0.500000}%
\pgfsetdash{}{0pt}%
\pgfsys@defobject{currentmarker}{\pgfqpoint{-0.013889in}{-0.013889in}}{\pgfqpoint{0.013889in}{0.013889in}}{%
\pgfpathmoveto{\pgfqpoint{0.000000in}{-0.013889in}}%
\pgfpathcurveto{\pgfqpoint{0.003683in}{-0.013889in}}{\pgfqpoint{0.007216in}{-0.012425in}}{\pgfqpoint{0.009821in}{-0.009821in}}%
\pgfpathcurveto{\pgfqpoint{0.012425in}{-0.007216in}}{\pgfqpoint{0.013889in}{-0.003683in}}{\pgfqpoint{0.013889in}{0.000000in}}%
\pgfpathcurveto{\pgfqpoint{0.013889in}{0.003683in}}{\pgfqpoint{0.012425in}{0.007216in}}{\pgfqpoint{0.009821in}{0.009821in}}%
\pgfpathcurveto{\pgfqpoint{0.007216in}{0.012425in}}{\pgfqpoint{0.003683in}{0.013889in}}{\pgfqpoint{0.000000in}{0.013889in}}%
\pgfpathcurveto{\pgfqpoint{-0.003683in}{0.013889in}}{\pgfqpoint{-0.007216in}{0.012425in}}{\pgfqpoint{-0.009821in}{0.009821in}}%
\pgfpathcurveto{\pgfqpoint{-0.012425in}{0.007216in}}{\pgfqpoint{-0.013889in}{0.003683in}}{\pgfqpoint{-0.013889in}{0.000000in}}%
\pgfpathcurveto{\pgfqpoint{-0.013889in}{-0.003683in}}{\pgfqpoint{-0.012425in}{-0.007216in}}{\pgfqpoint{-0.009821in}{-0.009821in}}%
\pgfpathcurveto{\pgfqpoint{-0.007216in}{-0.012425in}}{\pgfqpoint{-0.003683in}{-0.013889in}}{\pgfqpoint{0.000000in}{-0.013889in}}%
\pgfpathclose%
\pgfusepath{stroke,fill}%
}%
\begin{pgfscope}%
\pgfsys@transformshift{1.384741in}{4.932340in}%
\pgfsys@useobject{currentmarker}{}%
\end{pgfscope}%
\end{pgfscope}%
\begin{pgfscope}%
\pgfpathrectangle{\pgfqpoint{0.100000in}{2.413063in}}{\pgfqpoint{5.037500in}{3.427208in}}%
\pgfusepath{clip}%
\pgfsetrectcap%
\pgfsetroundjoin%
\pgfsetlinewidth{1.505625pt}%
\definecolor{currentstroke}{rgb}{0.678431,1.000000,0.184314}%
\pgfsetstrokecolor{currentstroke}%
\pgfsetstrokeopacity{0.500000}%
\pgfsetdash{}{0pt}%
\pgfpathmoveto{\pgfqpoint{1.058529in}{5.349824in}}%
\pgfusepath{stroke}%
\end{pgfscope}%
\begin{pgfscope}%
\pgfpathrectangle{\pgfqpoint{0.100000in}{2.413063in}}{\pgfqpoint{5.037500in}{3.427208in}}%
\pgfusepath{clip}%
\pgfsetbuttcap%
\pgfsetroundjoin%
\definecolor{currentfill}{rgb}{0.678431,1.000000,0.184314}%
\pgfsetfillcolor{currentfill}%
\pgfsetfillopacity{0.500000}%
\pgfsetlinewidth{0.250937pt}%
\definecolor{currentstroke}{rgb}{0.000000,0.000000,0.000000}%
\pgfsetstrokecolor{currentstroke}%
\pgfsetstrokeopacity{0.500000}%
\pgfsetdash{}{0pt}%
\pgfsys@defobject{currentmarker}{\pgfqpoint{-0.022222in}{-0.022222in}}{\pgfqpoint{0.022222in}{0.022222in}}{%
\pgfpathmoveto{\pgfqpoint{0.000000in}{-0.022222in}}%
\pgfpathcurveto{\pgfqpoint{0.005893in}{-0.022222in}}{\pgfqpoint{0.011546in}{-0.019881in}}{\pgfqpoint{0.015713in}{-0.015713in}}%
\pgfpathcurveto{\pgfqpoint{0.019881in}{-0.011546in}}{\pgfqpoint{0.022222in}{-0.005893in}}{\pgfqpoint{0.022222in}{0.000000in}}%
\pgfpathcurveto{\pgfqpoint{0.022222in}{0.005893in}}{\pgfqpoint{0.019881in}{0.011546in}}{\pgfqpoint{0.015713in}{0.015713in}}%
\pgfpathcurveto{\pgfqpoint{0.011546in}{0.019881in}}{\pgfqpoint{0.005893in}{0.022222in}}{\pgfqpoint{0.000000in}{0.022222in}}%
\pgfpathcurveto{\pgfqpoint{-0.005893in}{0.022222in}}{\pgfqpoint{-0.011546in}{0.019881in}}{\pgfqpoint{-0.015713in}{0.015713in}}%
\pgfpathcurveto{\pgfqpoint{-0.019881in}{0.011546in}}{\pgfqpoint{-0.022222in}{0.005893in}}{\pgfqpoint{-0.022222in}{0.000000in}}%
\pgfpathcurveto{\pgfqpoint{-0.022222in}{-0.005893in}}{\pgfqpoint{-0.019881in}{-0.011546in}}{\pgfqpoint{-0.015713in}{-0.015713in}}%
\pgfpathcurveto{\pgfqpoint{-0.011546in}{-0.019881in}}{\pgfqpoint{-0.005893in}{-0.022222in}}{\pgfqpoint{0.000000in}{-0.022222in}}%
\pgfpathclose%
\pgfusepath{stroke,fill}%
}%
\begin{pgfscope}%
\pgfsys@transformshift{1.058529in}{5.349824in}%
\pgfsys@useobject{currentmarker}{}%
\end{pgfscope}%
\end{pgfscope}%
\begin{pgfscope}%
\pgfpathrectangle{\pgfqpoint{0.100000in}{2.413063in}}{\pgfqpoint{5.037500in}{3.427208in}}%
\pgfusepath{clip}%
\pgfsetrectcap%
\pgfsetroundjoin%
\pgfsetlinewidth{1.505625pt}%
\definecolor{currentstroke}{rgb}{0.678431,1.000000,0.184314}%
\pgfsetstrokecolor{currentstroke}%
\pgfsetstrokeopacity{0.500000}%
\pgfsetdash{}{0pt}%
\pgfpathmoveto{\pgfqpoint{1.338071in}{4.867867in}}%
\pgfusepath{stroke}%
\end{pgfscope}%
\begin{pgfscope}%
\pgfpathrectangle{\pgfqpoint{0.100000in}{2.413063in}}{\pgfqpoint{5.037500in}{3.427208in}}%
\pgfusepath{clip}%
\pgfsetbuttcap%
\pgfsetroundjoin%
\definecolor{currentfill}{rgb}{0.678431,1.000000,0.184314}%
\pgfsetfillcolor{currentfill}%
\pgfsetfillopacity{0.500000}%
\pgfsetlinewidth{0.250937pt}%
\definecolor{currentstroke}{rgb}{0.000000,0.000000,0.000000}%
\pgfsetstrokecolor{currentstroke}%
\pgfsetstrokeopacity{0.500000}%
\pgfsetdash{}{0pt}%
\pgfsys@defobject{currentmarker}{\pgfqpoint{-0.019444in}{-0.019444in}}{\pgfqpoint{0.019444in}{0.019444in}}{%
\pgfpathmoveto{\pgfqpoint{0.000000in}{-0.019444in}}%
\pgfpathcurveto{\pgfqpoint{0.005157in}{-0.019444in}}{\pgfqpoint{0.010103in}{-0.017396in}}{\pgfqpoint{0.013749in}{-0.013749in}}%
\pgfpathcurveto{\pgfqpoint{0.017396in}{-0.010103in}}{\pgfqpoint{0.019444in}{-0.005157in}}{\pgfqpoint{0.019444in}{0.000000in}}%
\pgfpathcurveto{\pgfqpoint{0.019444in}{0.005157in}}{\pgfqpoint{0.017396in}{0.010103in}}{\pgfqpoint{0.013749in}{0.013749in}}%
\pgfpathcurveto{\pgfqpoint{0.010103in}{0.017396in}}{\pgfqpoint{0.005157in}{0.019444in}}{\pgfqpoint{0.000000in}{0.019444in}}%
\pgfpathcurveto{\pgfqpoint{-0.005157in}{0.019444in}}{\pgfqpoint{-0.010103in}{0.017396in}}{\pgfqpoint{-0.013749in}{0.013749in}}%
\pgfpathcurveto{\pgfqpoint{-0.017396in}{0.010103in}}{\pgfqpoint{-0.019444in}{0.005157in}}{\pgfqpoint{-0.019444in}{0.000000in}}%
\pgfpathcurveto{\pgfqpoint{-0.019444in}{-0.005157in}}{\pgfqpoint{-0.017396in}{-0.010103in}}{\pgfqpoint{-0.013749in}{-0.013749in}}%
\pgfpathcurveto{\pgfqpoint{-0.010103in}{-0.017396in}}{\pgfqpoint{-0.005157in}{-0.019444in}}{\pgfqpoint{0.000000in}{-0.019444in}}%
\pgfpathclose%
\pgfusepath{stroke,fill}%
}%
\begin{pgfscope}%
\pgfsys@transformshift{1.338071in}{4.867867in}%
\pgfsys@useobject{currentmarker}{}%
\end{pgfscope}%
\end{pgfscope}%
\begin{pgfscope}%
\pgfpathrectangle{\pgfqpoint{0.100000in}{2.413063in}}{\pgfqpoint{5.037500in}{3.427208in}}%
\pgfusepath{clip}%
\pgfsetrectcap%
\pgfsetroundjoin%
\pgfsetlinewidth{1.505625pt}%
\definecolor{currentstroke}{rgb}{0.000000,0.000000,1.000000}%
\pgfsetstrokecolor{currentstroke}%
\pgfsetstrokeopacity{0.500000}%
\pgfsetdash{}{0pt}%
\pgfpathmoveto{\pgfqpoint{3.329976in}{4.467971in}}%
\pgfusepath{stroke}%
\end{pgfscope}%
\begin{pgfscope}%
\pgfpathrectangle{\pgfqpoint{0.100000in}{2.413063in}}{\pgfqpoint{5.037500in}{3.427208in}}%
\pgfusepath{clip}%
\pgfsetbuttcap%
\pgfsetroundjoin%
\definecolor{currentfill}{rgb}{0.000000,0.000000,1.000000}%
\pgfsetfillcolor{currentfill}%
\pgfsetfillopacity{0.500000}%
\pgfsetlinewidth{0.250937pt}%
\definecolor{currentstroke}{rgb}{0.000000,0.000000,0.000000}%
\pgfsetstrokecolor{currentstroke}%
\pgfsetstrokeopacity{0.500000}%
\pgfsetdash{}{0pt}%
\pgfsys@defobject{currentmarker}{\pgfqpoint{-0.011111in}{-0.011111in}}{\pgfqpoint{0.011111in}{0.011111in}}{%
\pgfpathmoveto{\pgfqpoint{0.000000in}{-0.011111in}}%
\pgfpathcurveto{\pgfqpoint{0.002947in}{-0.011111in}}{\pgfqpoint{0.005773in}{-0.009940in}}{\pgfqpoint{0.007857in}{-0.007857in}}%
\pgfpathcurveto{\pgfqpoint{0.009940in}{-0.005773in}}{\pgfqpoint{0.011111in}{-0.002947in}}{\pgfqpoint{0.011111in}{0.000000in}}%
\pgfpathcurveto{\pgfqpoint{0.011111in}{0.002947in}}{\pgfqpoint{0.009940in}{0.005773in}}{\pgfqpoint{0.007857in}{0.007857in}}%
\pgfpathcurveto{\pgfqpoint{0.005773in}{0.009940in}}{\pgfqpoint{0.002947in}{0.011111in}}{\pgfqpoint{0.000000in}{0.011111in}}%
\pgfpathcurveto{\pgfqpoint{-0.002947in}{0.011111in}}{\pgfqpoint{-0.005773in}{0.009940in}}{\pgfqpoint{-0.007857in}{0.007857in}}%
\pgfpathcurveto{\pgfqpoint{-0.009940in}{0.005773in}}{\pgfqpoint{-0.011111in}{0.002947in}}{\pgfqpoint{-0.011111in}{0.000000in}}%
\pgfpathcurveto{\pgfqpoint{-0.011111in}{-0.002947in}}{\pgfqpoint{-0.009940in}{-0.005773in}}{\pgfqpoint{-0.007857in}{-0.007857in}}%
\pgfpathcurveto{\pgfqpoint{-0.005773in}{-0.009940in}}{\pgfqpoint{-0.002947in}{-0.011111in}}{\pgfqpoint{0.000000in}{-0.011111in}}%
\pgfpathclose%
\pgfusepath{stroke,fill}%
}%
\begin{pgfscope}%
\pgfsys@transformshift{3.329976in}{4.467971in}%
\pgfsys@useobject{currentmarker}{}%
\end{pgfscope}%
\end{pgfscope}%
\begin{pgfscope}%
\pgfpathrectangle{\pgfqpoint{0.100000in}{2.413063in}}{\pgfqpoint{5.037500in}{3.427208in}}%
\pgfusepath{clip}%
\pgfsetrectcap%
\pgfsetroundjoin%
\pgfsetlinewidth{1.505625pt}%
\definecolor{currentstroke}{rgb}{0.000000,0.000000,1.000000}%
\pgfsetstrokecolor{currentstroke}%
\pgfsetstrokeopacity{0.500000}%
\pgfsetdash{}{0pt}%
\pgfpathmoveto{\pgfqpoint{3.330628in}{4.151165in}}%
\pgfusepath{stroke}%
\end{pgfscope}%
\begin{pgfscope}%
\pgfpathrectangle{\pgfqpoint{0.100000in}{2.413063in}}{\pgfqpoint{5.037500in}{3.427208in}}%
\pgfusepath{clip}%
\pgfsetbuttcap%
\pgfsetroundjoin%
\definecolor{currentfill}{rgb}{0.000000,0.000000,1.000000}%
\pgfsetfillcolor{currentfill}%
\pgfsetfillopacity{0.500000}%
\pgfsetlinewidth{0.250937pt}%
\definecolor{currentstroke}{rgb}{0.000000,0.000000,0.000000}%
\pgfsetstrokecolor{currentstroke}%
\pgfsetstrokeopacity{0.500000}%
\pgfsetdash{}{0pt}%
\pgfsys@defobject{currentmarker}{\pgfqpoint{-0.016667in}{-0.016667in}}{\pgfqpoint{0.016667in}{0.016667in}}{%
\pgfpathmoveto{\pgfqpoint{0.000000in}{-0.016667in}}%
\pgfpathcurveto{\pgfqpoint{0.004420in}{-0.016667in}}{\pgfqpoint{0.008660in}{-0.014911in}}{\pgfqpoint{0.011785in}{-0.011785in}}%
\pgfpathcurveto{\pgfqpoint{0.014911in}{-0.008660in}}{\pgfqpoint{0.016667in}{-0.004420in}}{\pgfqpoint{0.016667in}{0.000000in}}%
\pgfpathcurveto{\pgfqpoint{0.016667in}{0.004420in}}{\pgfqpoint{0.014911in}{0.008660in}}{\pgfqpoint{0.011785in}{0.011785in}}%
\pgfpathcurveto{\pgfqpoint{0.008660in}{0.014911in}}{\pgfqpoint{0.004420in}{0.016667in}}{\pgfqpoint{0.000000in}{0.016667in}}%
\pgfpathcurveto{\pgfqpoint{-0.004420in}{0.016667in}}{\pgfqpoint{-0.008660in}{0.014911in}}{\pgfqpoint{-0.011785in}{0.011785in}}%
\pgfpathcurveto{\pgfqpoint{-0.014911in}{0.008660in}}{\pgfqpoint{-0.016667in}{0.004420in}}{\pgfqpoint{-0.016667in}{0.000000in}}%
\pgfpathcurveto{\pgfqpoint{-0.016667in}{-0.004420in}}{\pgfqpoint{-0.014911in}{-0.008660in}}{\pgfqpoint{-0.011785in}{-0.011785in}}%
\pgfpathcurveto{\pgfqpoint{-0.008660in}{-0.014911in}}{\pgfqpoint{-0.004420in}{-0.016667in}}{\pgfqpoint{0.000000in}{-0.016667in}}%
\pgfpathclose%
\pgfusepath{stroke,fill}%
}%
\begin{pgfscope}%
\pgfsys@transformshift{3.330628in}{4.151165in}%
\pgfsys@useobject{currentmarker}{}%
\end{pgfscope}%
\end{pgfscope}%
\begin{pgfscope}%
\pgfpathrectangle{\pgfqpoint{0.100000in}{2.413063in}}{\pgfqpoint{5.037500in}{3.427208in}}%
\pgfusepath{clip}%
\pgfsetrectcap%
\pgfsetroundjoin%
\pgfsetlinewidth{1.505625pt}%
\definecolor{currentstroke}{rgb}{0.000000,0.000000,1.000000}%
\pgfsetstrokecolor{currentstroke}%
\pgfsetstrokeopacity{0.500000}%
\pgfsetdash{}{0pt}%
\pgfpathmoveto{\pgfqpoint{3.398646in}{4.431619in}}%
\pgfusepath{stroke}%
\end{pgfscope}%
\begin{pgfscope}%
\pgfpathrectangle{\pgfqpoint{0.100000in}{2.413063in}}{\pgfqpoint{5.037500in}{3.427208in}}%
\pgfusepath{clip}%
\pgfsetbuttcap%
\pgfsetroundjoin%
\definecolor{currentfill}{rgb}{0.000000,0.000000,1.000000}%
\pgfsetfillcolor{currentfill}%
\pgfsetfillopacity{0.500000}%
\pgfsetlinewidth{0.250937pt}%
\definecolor{currentstroke}{rgb}{0.000000,0.000000,0.000000}%
\pgfsetstrokecolor{currentstroke}%
\pgfsetstrokeopacity{0.500000}%
\pgfsetdash{}{0pt}%
\pgfsys@defobject{currentmarker}{\pgfqpoint{-0.011111in}{-0.011111in}}{\pgfqpoint{0.011111in}{0.011111in}}{%
\pgfpathmoveto{\pgfqpoint{0.000000in}{-0.011111in}}%
\pgfpathcurveto{\pgfqpoint{0.002947in}{-0.011111in}}{\pgfqpoint{0.005773in}{-0.009940in}}{\pgfqpoint{0.007857in}{-0.007857in}}%
\pgfpathcurveto{\pgfqpoint{0.009940in}{-0.005773in}}{\pgfqpoint{0.011111in}{-0.002947in}}{\pgfqpoint{0.011111in}{0.000000in}}%
\pgfpathcurveto{\pgfqpoint{0.011111in}{0.002947in}}{\pgfqpoint{0.009940in}{0.005773in}}{\pgfqpoint{0.007857in}{0.007857in}}%
\pgfpathcurveto{\pgfqpoint{0.005773in}{0.009940in}}{\pgfqpoint{0.002947in}{0.011111in}}{\pgfqpoint{0.000000in}{0.011111in}}%
\pgfpathcurveto{\pgfqpoint{-0.002947in}{0.011111in}}{\pgfqpoint{-0.005773in}{0.009940in}}{\pgfqpoint{-0.007857in}{0.007857in}}%
\pgfpathcurveto{\pgfqpoint{-0.009940in}{0.005773in}}{\pgfqpoint{-0.011111in}{0.002947in}}{\pgfqpoint{-0.011111in}{0.000000in}}%
\pgfpathcurveto{\pgfqpoint{-0.011111in}{-0.002947in}}{\pgfqpoint{-0.009940in}{-0.005773in}}{\pgfqpoint{-0.007857in}{-0.007857in}}%
\pgfpathcurveto{\pgfqpoint{-0.005773in}{-0.009940in}}{\pgfqpoint{-0.002947in}{-0.011111in}}{\pgfqpoint{0.000000in}{-0.011111in}}%
\pgfpathclose%
\pgfusepath{stroke,fill}%
}%
\begin{pgfscope}%
\pgfsys@transformshift{3.398646in}{4.431619in}%
\pgfsys@useobject{currentmarker}{}%
\end{pgfscope}%
\end{pgfscope}%
\begin{pgfscope}%
\pgfpathrectangle{\pgfqpoint{0.100000in}{2.413063in}}{\pgfqpoint{5.037500in}{3.427208in}}%
\pgfusepath{clip}%
\pgfsetrectcap%
\pgfsetroundjoin%
\pgfsetlinewidth{1.505625pt}%
\definecolor{currentstroke}{rgb}{0.000000,0.000000,1.000000}%
\pgfsetstrokecolor{currentstroke}%
\pgfsetstrokeopacity{0.500000}%
\pgfsetdash{}{0pt}%
\pgfpathmoveto{\pgfqpoint{3.436563in}{4.637868in}}%
\pgfusepath{stroke}%
\end{pgfscope}%
\begin{pgfscope}%
\pgfpathrectangle{\pgfqpoint{0.100000in}{2.413063in}}{\pgfqpoint{5.037500in}{3.427208in}}%
\pgfusepath{clip}%
\pgfsetbuttcap%
\pgfsetroundjoin%
\definecolor{currentfill}{rgb}{0.000000,0.000000,1.000000}%
\pgfsetfillcolor{currentfill}%
\pgfsetfillopacity{0.500000}%
\pgfsetlinewidth{0.250937pt}%
\definecolor{currentstroke}{rgb}{0.000000,0.000000,0.000000}%
\pgfsetstrokecolor{currentstroke}%
\pgfsetstrokeopacity{0.500000}%
\pgfsetdash{}{0pt}%
\pgfsys@defobject{currentmarker}{\pgfqpoint{-0.030556in}{-0.030556in}}{\pgfqpoint{0.030556in}{0.030556in}}{%
\pgfpathmoveto{\pgfqpoint{0.000000in}{-0.030556in}}%
\pgfpathcurveto{\pgfqpoint{0.008103in}{-0.030556in}}{\pgfqpoint{0.015876in}{-0.027336in}}{\pgfqpoint{0.021606in}{-0.021606in}}%
\pgfpathcurveto{\pgfqpoint{0.027336in}{-0.015876in}}{\pgfqpoint{0.030556in}{-0.008103in}}{\pgfqpoint{0.030556in}{0.000000in}}%
\pgfpathcurveto{\pgfqpoint{0.030556in}{0.008103in}}{\pgfqpoint{0.027336in}{0.015876in}}{\pgfqpoint{0.021606in}{0.021606in}}%
\pgfpathcurveto{\pgfqpoint{0.015876in}{0.027336in}}{\pgfqpoint{0.008103in}{0.030556in}}{\pgfqpoint{0.000000in}{0.030556in}}%
\pgfpathcurveto{\pgfqpoint{-0.008103in}{0.030556in}}{\pgfqpoint{-0.015876in}{0.027336in}}{\pgfqpoint{-0.021606in}{0.021606in}}%
\pgfpathcurveto{\pgfqpoint{-0.027336in}{0.015876in}}{\pgfqpoint{-0.030556in}{0.008103in}}{\pgfqpoint{-0.030556in}{0.000000in}}%
\pgfpathcurveto{\pgfqpoint{-0.030556in}{-0.008103in}}{\pgfqpoint{-0.027336in}{-0.015876in}}{\pgfqpoint{-0.021606in}{-0.021606in}}%
\pgfpathcurveto{\pgfqpoint{-0.015876in}{-0.027336in}}{\pgfqpoint{-0.008103in}{-0.030556in}}{\pgfqpoint{0.000000in}{-0.030556in}}%
\pgfpathclose%
\pgfusepath{stroke,fill}%
}%
\begin{pgfscope}%
\pgfsys@transformshift{3.436563in}{4.637868in}%
\pgfsys@useobject{currentmarker}{}%
\end{pgfscope}%
\end{pgfscope}%
\begin{pgfscope}%
\pgfpathrectangle{\pgfqpoint{0.100000in}{2.413063in}}{\pgfqpoint{5.037500in}{3.427208in}}%
\pgfusepath{clip}%
\pgfsetrectcap%
\pgfsetroundjoin%
\pgfsetlinewidth{1.505625pt}%
\definecolor{currentstroke}{rgb}{0.501961,0.501961,0.501961}%
\pgfsetstrokecolor{currentstroke}%
\pgfsetstrokeopacity{0.500000}%
\pgfsetdash{}{0pt}%
\pgfpathmoveto{\pgfqpoint{3.452382in}{4.436821in}}%
\pgfusepath{stroke}%
\end{pgfscope}%
\begin{pgfscope}%
\pgfpathrectangle{\pgfqpoint{0.100000in}{2.413063in}}{\pgfqpoint{5.037500in}{3.427208in}}%
\pgfusepath{clip}%
\pgfsetbuttcap%
\pgfsetroundjoin%
\definecolor{currentfill}{rgb}{0.501961,0.501961,0.501961}%
\pgfsetfillcolor{currentfill}%
\pgfsetfillopacity{0.500000}%
\pgfsetlinewidth{0.250937pt}%
\definecolor{currentstroke}{rgb}{0.000000,0.000000,0.000000}%
\pgfsetstrokecolor{currentstroke}%
\pgfsetstrokeopacity{0.500000}%
\pgfsetdash{}{0pt}%
\pgfsys@defobject{currentmarker}{\pgfqpoint{-0.013889in}{-0.013889in}}{\pgfqpoint{0.013889in}{0.013889in}}{%
\pgfpathmoveto{\pgfqpoint{0.000000in}{-0.013889in}}%
\pgfpathcurveto{\pgfqpoint{0.003683in}{-0.013889in}}{\pgfqpoint{0.007216in}{-0.012425in}}{\pgfqpoint{0.009821in}{-0.009821in}}%
\pgfpathcurveto{\pgfqpoint{0.012425in}{-0.007216in}}{\pgfqpoint{0.013889in}{-0.003683in}}{\pgfqpoint{0.013889in}{0.000000in}}%
\pgfpathcurveto{\pgfqpoint{0.013889in}{0.003683in}}{\pgfqpoint{0.012425in}{0.007216in}}{\pgfqpoint{0.009821in}{0.009821in}}%
\pgfpathcurveto{\pgfqpoint{0.007216in}{0.012425in}}{\pgfqpoint{0.003683in}{0.013889in}}{\pgfqpoint{0.000000in}{0.013889in}}%
\pgfpathcurveto{\pgfqpoint{-0.003683in}{0.013889in}}{\pgfqpoint{-0.007216in}{0.012425in}}{\pgfqpoint{-0.009821in}{0.009821in}}%
\pgfpathcurveto{\pgfqpoint{-0.012425in}{0.007216in}}{\pgfqpoint{-0.013889in}{0.003683in}}{\pgfqpoint{-0.013889in}{0.000000in}}%
\pgfpathcurveto{\pgfqpoint{-0.013889in}{-0.003683in}}{\pgfqpoint{-0.012425in}{-0.007216in}}{\pgfqpoint{-0.009821in}{-0.009821in}}%
\pgfpathcurveto{\pgfqpoint{-0.007216in}{-0.012425in}}{\pgfqpoint{-0.003683in}{-0.013889in}}{\pgfqpoint{0.000000in}{-0.013889in}}%
\pgfpathclose%
\pgfusepath{stroke,fill}%
}%
\begin{pgfscope}%
\pgfsys@transformshift{3.452382in}{4.436821in}%
\pgfsys@useobject{currentmarker}{}%
\end{pgfscope}%
\end{pgfscope}%
\begin{pgfscope}%
\pgfpathrectangle{\pgfqpoint{0.100000in}{2.413063in}}{\pgfqpoint{5.037500in}{3.427208in}}%
\pgfusepath{clip}%
\pgfsetrectcap%
\pgfsetroundjoin%
\pgfsetlinewidth{1.505625pt}%
\definecolor{currentstroke}{rgb}{0.678431,1.000000,0.184314}%
\pgfsetstrokecolor{currentstroke}%
\pgfsetstrokeopacity{0.500000}%
\pgfsetdash{}{0pt}%
\pgfpathmoveto{\pgfqpoint{3.185573in}{4.580908in}}%
\pgfusepath{stroke}%
\end{pgfscope}%
\begin{pgfscope}%
\pgfpathrectangle{\pgfqpoint{0.100000in}{2.413063in}}{\pgfqpoint{5.037500in}{3.427208in}}%
\pgfusepath{clip}%
\pgfsetbuttcap%
\pgfsetroundjoin%
\definecolor{currentfill}{rgb}{0.678431,1.000000,0.184314}%
\pgfsetfillcolor{currentfill}%
\pgfsetfillopacity{0.500000}%
\pgfsetlinewidth{0.250937pt}%
\definecolor{currentstroke}{rgb}{0.000000,0.000000,0.000000}%
\pgfsetstrokecolor{currentstroke}%
\pgfsetstrokeopacity{0.500000}%
\pgfsetdash{}{0pt}%
\pgfsys@defobject{currentmarker}{\pgfqpoint{-0.011111in}{-0.011111in}}{\pgfqpoint{0.011111in}{0.011111in}}{%
\pgfpathmoveto{\pgfqpoint{0.000000in}{-0.011111in}}%
\pgfpathcurveto{\pgfqpoint{0.002947in}{-0.011111in}}{\pgfqpoint{0.005773in}{-0.009940in}}{\pgfqpoint{0.007857in}{-0.007857in}}%
\pgfpathcurveto{\pgfqpoint{0.009940in}{-0.005773in}}{\pgfqpoint{0.011111in}{-0.002947in}}{\pgfqpoint{0.011111in}{0.000000in}}%
\pgfpathcurveto{\pgfqpoint{0.011111in}{0.002947in}}{\pgfqpoint{0.009940in}{0.005773in}}{\pgfqpoint{0.007857in}{0.007857in}}%
\pgfpathcurveto{\pgfqpoint{0.005773in}{0.009940in}}{\pgfqpoint{0.002947in}{0.011111in}}{\pgfqpoint{0.000000in}{0.011111in}}%
\pgfpathcurveto{\pgfqpoint{-0.002947in}{0.011111in}}{\pgfqpoint{-0.005773in}{0.009940in}}{\pgfqpoint{-0.007857in}{0.007857in}}%
\pgfpathcurveto{\pgfqpoint{-0.009940in}{0.005773in}}{\pgfqpoint{-0.011111in}{0.002947in}}{\pgfqpoint{-0.011111in}{0.000000in}}%
\pgfpathcurveto{\pgfqpoint{-0.011111in}{-0.002947in}}{\pgfqpoint{-0.009940in}{-0.005773in}}{\pgfqpoint{-0.007857in}{-0.007857in}}%
\pgfpathcurveto{\pgfqpoint{-0.005773in}{-0.009940in}}{\pgfqpoint{-0.002947in}{-0.011111in}}{\pgfqpoint{0.000000in}{-0.011111in}}%
\pgfpathclose%
\pgfusepath{stroke,fill}%
}%
\begin{pgfscope}%
\pgfsys@transformshift{3.185573in}{4.580908in}%
\pgfsys@useobject{currentmarker}{}%
\end{pgfscope}%
\end{pgfscope}%
\begin{pgfscope}%
\pgfpathrectangle{\pgfqpoint{0.100000in}{2.413063in}}{\pgfqpoint{5.037500in}{3.427208in}}%
\pgfusepath{clip}%
\pgfsetrectcap%
\pgfsetroundjoin%
\pgfsetlinewidth{1.505625pt}%
\definecolor{currentstroke}{rgb}{0.000000,0.000000,1.000000}%
\pgfsetstrokecolor{currentstroke}%
\pgfsetstrokeopacity{0.500000}%
\pgfsetdash{}{0pt}%
\pgfpathmoveto{\pgfqpoint{3.338431in}{4.396084in}}%
\pgfusepath{stroke}%
\end{pgfscope}%
\begin{pgfscope}%
\pgfpathrectangle{\pgfqpoint{0.100000in}{2.413063in}}{\pgfqpoint{5.037500in}{3.427208in}}%
\pgfusepath{clip}%
\pgfsetbuttcap%
\pgfsetroundjoin%
\definecolor{currentfill}{rgb}{0.000000,0.000000,1.000000}%
\pgfsetfillcolor{currentfill}%
\pgfsetfillopacity{0.500000}%
\pgfsetlinewidth{0.250937pt}%
\definecolor{currentstroke}{rgb}{0.000000,0.000000,0.000000}%
\pgfsetstrokecolor{currentstroke}%
\pgfsetstrokeopacity{0.500000}%
\pgfsetdash{}{0pt}%
\pgfsys@defobject{currentmarker}{\pgfqpoint{-0.033333in}{-0.033333in}}{\pgfqpoint{0.033333in}{0.033333in}}{%
\pgfpathmoveto{\pgfqpoint{0.000000in}{-0.033333in}}%
\pgfpathcurveto{\pgfqpoint{0.008840in}{-0.033333in}}{\pgfqpoint{0.017319in}{-0.029821in}}{\pgfqpoint{0.023570in}{-0.023570in}}%
\pgfpathcurveto{\pgfqpoint{0.029821in}{-0.017319in}}{\pgfqpoint{0.033333in}{-0.008840in}}{\pgfqpoint{0.033333in}{0.000000in}}%
\pgfpathcurveto{\pgfqpoint{0.033333in}{0.008840in}}{\pgfqpoint{0.029821in}{0.017319in}}{\pgfqpoint{0.023570in}{0.023570in}}%
\pgfpathcurveto{\pgfqpoint{0.017319in}{0.029821in}}{\pgfqpoint{0.008840in}{0.033333in}}{\pgfqpoint{0.000000in}{0.033333in}}%
\pgfpathcurveto{\pgfqpoint{-0.008840in}{0.033333in}}{\pgfqpoint{-0.017319in}{0.029821in}}{\pgfqpoint{-0.023570in}{0.023570in}}%
\pgfpathcurveto{\pgfqpoint{-0.029821in}{0.017319in}}{\pgfqpoint{-0.033333in}{0.008840in}}{\pgfqpoint{-0.033333in}{0.000000in}}%
\pgfpathcurveto{\pgfqpoint{-0.033333in}{-0.008840in}}{\pgfqpoint{-0.029821in}{-0.017319in}}{\pgfqpoint{-0.023570in}{-0.023570in}}%
\pgfpathcurveto{\pgfqpoint{-0.017319in}{-0.029821in}}{\pgfqpoint{-0.008840in}{-0.033333in}}{\pgfqpoint{0.000000in}{-0.033333in}}%
\pgfpathclose%
\pgfusepath{stroke,fill}%
}%
\begin{pgfscope}%
\pgfsys@transformshift{3.338431in}{4.396084in}%
\pgfsys@useobject{currentmarker}{}%
\end{pgfscope}%
\end{pgfscope}%
\begin{pgfscope}%
\pgfpathrectangle{\pgfqpoint{0.100000in}{2.413063in}}{\pgfqpoint{5.037500in}{3.427208in}}%
\pgfusepath{clip}%
\pgfsetrectcap%
\pgfsetroundjoin%
\pgfsetlinewidth{1.505625pt}%
\definecolor{currentstroke}{rgb}{0.000000,0.000000,1.000000}%
\pgfsetstrokecolor{currentstroke}%
\pgfsetstrokeopacity{0.500000}%
\pgfsetdash{}{0pt}%
\pgfpathmoveto{\pgfqpoint{3.423139in}{4.549440in}}%
\pgfusepath{stroke}%
\end{pgfscope}%
\begin{pgfscope}%
\pgfpathrectangle{\pgfqpoint{0.100000in}{2.413063in}}{\pgfqpoint{5.037500in}{3.427208in}}%
\pgfusepath{clip}%
\pgfsetbuttcap%
\pgfsetroundjoin%
\definecolor{currentfill}{rgb}{0.000000,0.000000,1.000000}%
\pgfsetfillcolor{currentfill}%
\pgfsetfillopacity{0.500000}%
\pgfsetlinewidth{0.250937pt}%
\definecolor{currentstroke}{rgb}{0.000000,0.000000,0.000000}%
\pgfsetstrokecolor{currentstroke}%
\pgfsetstrokeopacity{0.500000}%
\pgfsetdash{}{0pt}%
\pgfsys@defobject{currentmarker}{\pgfqpoint{-0.005556in}{-0.005556in}}{\pgfqpoint{0.005556in}{0.005556in}}{%
\pgfpathmoveto{\pgfqpoint{0.000000in}{-0.005556in}}%
\pgfpathcurveto{\pgfqpoint{0.001473in}{-0.005556in}}{\pgfqpoint{0.002887in}{-0.004970in}}{\pgfqpoint{0.003928in}{-0.003928in}}%
\pgfpathcurveto{\pgfqpoint{0.004970in}{-0.002887in}}{\pgfqpoint{0.005556in}{-0.001473in}}{\pgfqpoint{0.005556in}{0.000000in}}%
\pgfpathcurveto{\pgfqpoint{0.005556in}{0.001473in}}{\pgfqpoint{0.004970in}{0.002887in}}{\pgfqpoint{0.003928in}{0.003928in}}%
\pgfpathcurveto{\pgfqpoint{0.002887in}{0.004970in}}{\pgfqpoint{0.001473in}{0.005556in}}{\pgfqpoint{0.000000in}{0.005556in}}%
\pgfpathcurveto{\pgfqpoint{-0.001473in}{0.005556in}}{\pgfqpoint{-0.002887in}{0.004970in}}{\pgfqpoint{-0.003928in}{0.003928in}}%
\pgfpathcurveto{\pgfqpoint{-0.004970in}{0.002887in}}{\pgfqpoint{-0.005556in}{0.001473in}}{\pgfqpoint{-0.005556in}{0.000000in}}%
\pgfpathcurveto{\pgfqpoint{-0.005556in}{-0.001473in}}{\pgfqpoint{-0.004970in}{-0.002887in}}{\pgfqpoint{-0.003928in}{-0.003928in}}%
\pgfpathcurveto{\pgfqpoint{-0.002887in}{-0.004970in}}{\pgfqpoint{-0.001473in}{-0.005556in}}{\pgfqpoint{0.000000in}{-0.005556in}}%
\pgfpathclose%
\pgfusepath{stroke,fill}%
}%
\begin{pgfscope}%
\pgfsys@transformshift{3.423139in}{4.549440in}%
\pgfsys@useobject{currentmarker}{}%
\end{pgfscope}%
\end{pgfscope}%
\begin{pgfscope}%
\pgfpathrectangle{\pgfqpoint{0.100000in}{2.413063in}}{\pgfqpoint{5.037500in}{3.427208in}}%
\pgfusepath{clip}%
\pgfsetrectcap%
\pgfsetroundjoin%
\pgfsetlinewidth{1.505625pt}%
\definecolor{currentstroke}{rgb}{0.000000,0.000000,1.000000}%
\pgfsetstrokecolor{currentstroke}%
\pgfsetstrokeopacity{0.500000}%
\pgfsetdash{}{0pt}%
\pgfpathmoveto{\pgfqpoint{3.276435in}{4.490082in}}%
\pgfusepath{stroke}%
\end{pgfscope}%
\begin{pgfscope}%
\pgfpathrectangle{\pgfqpoint{0.100000in}{2.413063in}}{\pgfqpoint{5.037500in}{3.427208in}}%
\pgfusepath{clip}%
\pgfsetbuttcap%
\pgfsetroundjoin%
\definecolor{currentfill}{rgb}{0.000000,0.000000,1.000000}%
\pgfsetfillcolor{currentfill}%
\pgfsetfillopacity{0.500000}%
\pgfsetlinewidth{0.250937pt}%
\definecolor{currentstroke}{rgb}{0.000000,0.000000,0.000000}%
\pgfsetstrokecolor{currentstroke}%
\pgfsetstrokeopacity{0.500000}%
\pgfsetdash{}{0pt}%
\pgfsys@defobject{currentmarker}{\pgfqpoint{-0.011111in}{-0.011111in}}{\pgfqpoint{0.011111in}{0.011111in}}{%
\pgfpathmoveto{\pgfqpoint{0.000000in}{-0.011111in}}%
\pgfpathcurveto{\pgfqpoint{0.002947in}{-0.011111in}}{\pgfqpoint{0.005773in}{-0.009940in}}{\pgfqpoint{0.007857in}{-0.007857in}}%
\pgfpathcurveto{\pgfqpoint{0.009940in}{-0.005773in}}{\pgfqpoint{0.011111in}{-0.002947in}}{\pgfqpoint{0.011111in}{0.000000in}}%
\pgfpathcurveto{\pgfqpoint{0.011111in}{0.002947in}}{\pgfqpoint{0.009940in}{0.005773in}}{\pgfqpoint{0.007857in}{0.007857in}}%
\pgfpathcurveto{\pgfqpoint{0.005773in}{0.009940in}}{\pgfqpoint{0.002947in}{0.011111in}}{\pgfqpoint{0.000000in}{0.011111in}}%
\pgfpathcurveto{\pgfqpoint{-0.002947in}{0.011111in}}{\pgfqpoint{-0.005773in}{0.009940in}}{\pgfqpoint{-0.007857in}{0.007857in}}%
\pgfpathcurveto{\pgfqpoint{-0.009940in}{0.005773in}}{\pgfqpoint{-0.011111in}{0.002947in}}{\pgfqpoint{-0.011111in}{0.000000in}}%
\pgfpathcurveto{\pgfqpoint{-0.011111in}{-0.002947in}}{\pgfqpoint{-0.009940in}{-0.005773in}}{\pgfqpoint{-0.007857in}{-0.007857in}}%
\pgfpathcurveto{\pgfqpoint{-0.005773in}{-0.009940in}}{\pgfqpoint{-0.002947in}{-0.011111in}}{\pgfqpoint{0.000000in}{-0.011111in}}%
\pgfpathclose%
\pgfusepath{stroke,fill}%
}%
\begin{pgfscope}%
\pgfsys@transformshift{3.276435in}{4.490082in}%
\pgfsys@useobject{currentmarker}{}%
\end{pgfscope}%
\end{pgfscope}%
\begin{pgfscope}%
\pgfpathrectangle{\pgfqpoint{0.100000in}{2.413063in}}{\pgfqpoint{5.037500in}{3.427208in}}%
\pgfusepath{clip}%
\pgfsetrectcap%
\pgfsetroundjoin%
\pgfsetlinewidth{1.505625pt}%
\definecolor{currentstroke}{rgb}{0.000000,0.000000,1.000000}%
\pgfsetstrokecolor{currentstroke}%
\pgfsetstrokeopacity{0.500000}%
\pgfsetdash{}{0pt}%
\pgfpathmoveto{\pgfqpoint{3.307789in}{4.674204in}}%
\pgfusepath{stroke}%
\end{pgfscope}%
\begin{pgfscope}%
\pgfpathrectangle{\pgfqpoint{0.100000in}{2.413063in}}{\pgfqpoint{5.037500in}{3.427208in}}%
\pgfusepath{clip}%
\pgfsetbuttcap%
\pgfsetroundjoin%
\definecolor{currentfill}{rgb}{0.000000,0.000000,1.000000}%
\pgfsetfillcolor{currentfill}%
\pgfsetfillopacity{0.500000}%
\pgfsetlinewidth{0.250937pt}%
\definecolor{currentstroke}{rgb}{0.000000,0.000000,0.000000}%
\pgfsetstrokecolor{currentstroke}%
\pgfsetstrokeopacity{0.500000}%
\pgfsetdash{}{0pt}%
\pgfsys@defobject{currentmarker}{\pgfqpoint{-0.036111in}{-0.036111in}}{\pgfqpoint{0.036111in}{0.036111in}}{%
\pgfpathmoveto{\pgfqpoint{0.000000in}{-0.036111in}}%
\pgfpathcurveto{\pgfqpoint{0.009577in}{-0.036111in}}{\pgfqpoint{0.018763in}{-0.032306in}}{\pgfqpoint{0.025534in}{-0.025534in}}%
\pgfpathcurveto{\pgfqpoint{0.032306in}{-0.018763in}}{\pgfqpoint{0.036111in}{-0.009577in}}{\pgfqpoint{0.036111in}{0.000000in}}%
\pgfpathcurveto{\pgfqpoint{0.036111in}{0.009577in}}{\pgfqpoint{0.032306in}{0.018763in}}{\pgfqpoint{0.025534in}{0.025534in}}%
\pgfpathcurveto{\pgfqpoint{0.018763in}{0.032306in}}{\pgfqpoint{0.009577in}{0.036111in}}{\pgfqpoint{0.000000in}{0.036111in}}%
\pgfpathcurveto{\pgfqpoint{-0.009577in}{0.036111in}}{\pgfqpoint{-0.018763in}{0.032306in}}{\pgfqpoint{-0.025534in}{0.025534in}}%
\pgfpathcurveto{\pgfqpoint{-0.032306in}{0.018763in}}{\pgfqpoint{-0.036111in}{0.009577in}}{\pgfqpoint{-0.036111in}{0.000000in}}%
\pgfpathcurveto{\pgfqpoint{-0.036111in}{-0.009577in}}{\pgfqpoint{-0.032306in}{-0.018763in}}{\pgfqpoint{-0.025534in}{-0.025534in}}%
\pgfpathcurveto{\pgfqpoint{-0.018763in}{-0.032306in}}{\pgfqpoint{-0.009577in}{-0.036111in}}{\pgfqpoint{0.000000in}{-0.036111in}}%
\pgfpathclose%
\pgfusepath{stroke,fill}%
}%
\begin{pgfscope}%
\pgfsys@transformshift{3.307789in}{4.674204in}%
\pgfsys@useobject{currentmarker}{}%
\end{pgfscope}%
\end{pgfscope}%
\begin{pgfscope}%
\pgfpathrectangle{\pgfqpoint{0.100000in}{2.413063in}}{\pgfqpoint{5.037500in}{3.427208in}}%
\pgfusepath{clip}%
\pgfsetrectcap%
\pgfsetroundjoin%
\pgfsetlinewidth{1.505625pt}%
\definecolor{currentstroke}{rgb}{0.000000,0.000000,1.000000}%
\pgfsetstrokecolor{currentstroke}%
\pgfsetstrokeopacity{0.500000}%
\pgfsetdash{}{0pt}%
\pgfpathmoveto{\pgfqpoint{3.277687in}{4.386909in}}%
\pgfusepath{stroke}%
\end{pgfscope}%
\begin{pgfscope}%
\pgfpathrectangle{\pgfqpoint{0.100000in}{2.413063in}}{\pgfqpoint{5.037500in}{3.427208in}}%
\pgfusepath{clip}%
\pgfsetbuttcap%
\pgfsetroundjoin%
\definecolor{currentfill}{rgb}{0.000000,0.000000,1.000000}%
\pgfsetfillcolor{currentfill}%
\pgfsetfillopacity{0.500000}%
\pgfsetlinewidth{0.250937pt}%
\definecolor{currentstroke}{rgb}{0.000000,0.000000,0.000000}%
\pgfsetstrokecolor{currentstroke}%
\pgfsetstrokeopacity{0.500000}%
\pgfsetdash{}{0pt}%
\pgfsys@defobject{currentmarker}{\pgfqpoint{-0.016667in}{-0.016667in}}{\pgfqpoint{0.016667in}{0.016667in}}{%
\pgfpathmoveto{\pgfqpoint{0.000000in}{-0.016667in}}%
\pgfpathcurveto{\pgfqpoint{0.004420in}{-0.016667in}}{\pgfqpoint{0.008660in}{-0.014911in}}{\pgfqpoint{0.011785in}{-0.011785in}}%
\pgfpathcurveto{\pgfqpoint{0.014911in}{-0.008660in}}{\pgfqpoint{0.016667in}{-0.004420in}}{\pgfqpoint{0.016667in}{0.000000in}}%
\pgfpathcurveto{\pgfqpoint{0.016667in}{0.004420in}}{\pgfqpoint{0.014911in}{0.008660in}}{\pgfqpoint{0.011785in}{0.011785in}}%
\pgfpathcurveto{\pgfqpoint{0.008660in}{0.014911in}}{\pgfqpoint{0.004420in}{0.016667in}}{\pgfqpoint{0.000000in}{0.016667in}}%
\pgfpathcurveto{\pgfqpoint{-0.004420in}{0.016667in}}{\pgfqpoint{-0.008660in}{0.014911in}}{\pgfqpoint{-0.011785in}{0.011785in}}%
\pgfpathcurveto{\pgfqpoint{-0.014911in}{0.008660in}}{\pgfqpoint{-0.016667in}{0.004420in}}{\pgfqpoint{-0.016667in}{0.000000in}}%
\pgfpathcurveto{\pgfqpoint{-0.016667in}{-0.004420in}}{\pgfqpoint{-0.014911in}{-0.008660in}}{\pgfqpoint{-0.011785in}{-0.011785in}}%
\pgfpathcurveto{\pgfqpoint{-0.008660in}{-0.014911in}}{\pgfqpoint{-0.004420in}{-0.016667in}}{\pgfqpoint{0.000000in}{-0.016667in}}%
\pgfpathclose%
\pgfusepath{stroke,fill}%
}%
\begin{pgfscope}%
\pgfsys@transformshift{3.277687in}{4.386909in}%
\pgfsys@useobject{currentmarker}{}%
\end{pgfscope}%
\end{pgfscope}%
\begin{pgfscope}%
\pgfpathrectangle{\pgfqpoint{0.100000in}{2.413063in}}{\pgfqpoint{5.037500in}{3.427208in}}%
\pgfusepath{clip}%
\pgfsetrectcap%
\pgfsetroundjoin%
\pgfsetlinewidth{1.505625pt}%
\definecolor{currentstroke}{rgb}{0.678431,1.000000,0.184314}%
\pgfsetstrokecolor{currentstroke}%
\pgfsetstrokeopacity{0.500000}%
\pgfsetdash{}{0pt}%
\pgfpathmoveto{\pgfqpoint{3.558808in}{4.335360in}}%
\pgfusepath{stroke}%
\end{pgfscope}%
\begin{pgfscope}%
\pgfpathrectangle{\pgfqpoint{0.100000in}{2.413063in}}{\pgfqpoint{5.037500in}{3.427208in}}%
\pgfusepath{clip}%
\pgfsetbuttcap%
\pgfsetroundjoin%
\definecolor{currentfill}{rgb}{0.678431,1.000000,0.184314}%
\pgfsetfillcolor{currentfill}%
\pgfsetfillopacity{0.500000}%
\pgfsetlinewidth{0.250937pt}%
\definecolor{currentstroke}{rgb}{0.000000,0.000000,0.000000}%
\pgfsetstrokecolor{currentstroke}%
\pgfsetstrokeopacity{0.500000}%
\pgfsetdash{}{0pt}%
\pgfsys@defobject{currentmarker}{\pgfqpoint{-0.044444in}{-0.044444in}}{\pgfqpoint{0.044444in}{0.044444in}}{%
\pgfpathmoveto{\pgfqpoint{0.000000in}{-0.044444in}}%
\pgfpathcurveto{\pgfqpoint{0.011787in}{-0.044444in}}{\pgfqpoint{0.023092in}{-0.039761in}}{\pgfqpoint{0.031427in}{-0.031427in}}%
\pgfpathcurveto{\pgfqpoint{0.039761in}{-0.023092in}}{\pgfqpoint{0.044444in}{-0.011787in}}{\pgfqpoint{0.044444in}{0.000000in}}%
\pgfpathcurveto{\pgfqpoint{0.044444in}{0.011787in}}{\pgfqpoint{0.039761in}{0.023092in}}{\pgfqpoint{0.031427in}{0.031427in}}%
\pgfpathcurveto{\pgfqpoint{0.023092in}{0.039761in}}{\pgfqpoint{0.011787in}{0.044444in}}{\pgfqpoint{0.000000in}{0.044444in}}%
\pgfpathcurveto{\pgfqpoint{-0.011787in}{0.044444in}}{\pgfqpoint{-0.023092in}{0.039761in}}{\pgfqpoint{-0.031427in}{0.031427in}}%
\pgfpathcurveto{\pgfqpoint{-0.039761in}{0.023092in}}{\pgfqpoint{-0.044444in}{0.011787in}}{\pgfqpoint{-0.044444in}{0.000000in}}%
\pgfpathcurveto{\pgfqpoint{-0.044444in}{-0.011787in}}{\pgfqpoint{-0.039761in}{-0.023092in}}{\pgfqpoint{-0.031427in}{-0.031427in}}%
\pgfpathcurveto{\pgfqpoint{-0.023092in}{-0.039761in}}{\pgfqpoint{-0.011787in}{-0.044444in}}{\pgfqpoint{0.000000in}{-0.044444in}}%
\pgfpathclose%
\pgfusepath{stroke,fill}%
}%
\begin{pgfscope}%
\pgfsys@transformshift{3.558808in}{4.335360in}%
\pgfsys@useobject{currentmarker}{}%
\end{pgfscope}%
\end{pgfscope}%
\begin{pgfscope}%
\pgfpathrectangle{\pgfqpoint{0.100000in}{2.413063in}}{\pgfqpoint{5.037500in}{3.427208in}}%
\pgfusepath{clip}%
\pgfsetrectcap%
\pgfsetroundjoin%
\pgfsetlinewidth{1.505625pt}%
\definecolor{currentstroke}{rgb}{0.678431,1.000000,0.184314}%
\pgfsetstrokecolor{currentstroke}%
\pgfsetstrokeopacity{0.500000}%
\pgfsetdash{}{0pt}%
\pgfpathmoveto{\pgfqpoint{3.612876in}{4.344575in}}%
\pgfusepath{stroke}%
\end{pgfscope}%
\begin{pgfscope}%
\pgfpathrectangle{\pgfqpoint{0.100000in}{2.413063in}}{\pgfqpoint{5.037500in}{3.427208in}}%
\pgfusepath{clip}%
\pgfsetbuttcap%
\pgfsetroundjoin%
\definecolor{currentfill}{rgb}{0.678431,1.000000,0.184314}%
\pgfsetfillcolor{currentfill}%
\pgfsetfillopacity{0.500000}%
\pgfsetlinewidth{0.250937pt}%
\definecolor{currentstroke}{rgb}{0.000000,0.000000,0.000000}%
\pgfsetstrokecolor{currentstroke}%
\pgfsetstrokeopacity{0.500000}%
\pgfsetdash{}{0pt}%
\pgfsys@defobject{currentmarker}{\pgfqpoint{-0.030556in}{-0.030556in}}{\pgfqpoint{0.030556in}{0.030556in}}{%
\pgfpathmoveto{\pgfqpoint{0.000000in}{-0.030556in}}%
\pgfpathcurveto{\pgfqpoint{0.008103in}{-0.030556in}}{\pgfqpoint{0.015876in}{-0.027336in}}{\pgfqpoint{0.021606in}{-0.021606in}}%
\pgfpathcurveto{\pgfqpoint{0.027336in}{-0.015876in}}{\pgfqpoint{0.030556in}{-0.008103in}}{\pgfqpoint{0.030556in}{0.000000in}}%
\pgfpathcurveto{\pgfqpoint{0.030556in}{0.008103in}}{\pgfqpoint{0.027336in}{0.015876in}}{\pgfqpoint{0.021606in}{0.021606in}}%
\pgfpathcurveto{\pgfqpoint{0.015876in}{0.027336in}}{\pgfqpoint{0.008103in}{0.030556in}}{\pgfqpoint{0.000000in}{0.030556in}}%
\pgfpathcurveto{\pgfqpoint{-0.008103in}{0.030556in}}{\pgfqpoint{-0.015876in}{0.027336in}}{\pgfqpoint{-0.021606in}{0.021606in}}%
\pgfpathcurveto{\pgfqpoint{-0.027336in}{0.015876in}}{\pgfqpoint{-0.030556in}{0.008103in}}{\pgfqpoint{-0.030556in}{0.000000in}}%
\pgfpathcurveto{\pgfqpoint{-0.030556in}{-0.008103in}}{\pgfqpoint{-0.027336in}{-0.015876in}}{\pgfqpoint{-0.021606in}{-0.021606in}}%
\pgfpathcurveto{\pgfqpoint{-0.015876in}{-0.027336in}}{\pgfqpoint{-0.008103in}{-0.030556in}}{\pgfqpoint{0.000000in}{-0.030556in}}%
\pgfpathclose%
\pgfusepath{stroke,fill}%
}%
\begin{pgfscope}%
\pgfsys@transformshift{3.612876in}{4.344575in}%
\pgfsys@useobject{currentmarker}{}%
\end{pgfscope}%
\end{pgfscope}%
\begin{pgfscope}%
\pgfpathrectangle{\pgfqpoint{0.100000in}{2.413063in}}{\pgfqpoint{5.037500in}{3.427208in}}%
\pgfusepath{clip}%
\pgfsetrectcap%
\pgfsetroundjoin%
\pgfsetlinewidth{1.505625pt}%
\definecolor{currentstroke}{rgb}{0.678431,1.000000,0.184314}%
\pgfsetstrokecolor{currentstroke}%
\pgfsetstrokeopacity{0.500000}%
\pgfsetdash{}{0pt}%
\pgfpathmoveto{\pgfqpoint{3.579642in}{4.628409in}}%
\pgfusepath{stroke}%
\end{pgfscope}%
\begin{pgfscope}%
\pgfpathrectangle{\pgfqpoint{0.100000in}{2.413063in}}{\pgfqpoint{5.037500in}{3.427208in}}%
\pgfusepath{clip}%
\pgfsetbuttcap%
\pgfsetroundjoin%
\definecolor{currentfill}{rgb}{0.678431,1.000000,0.184314}%
\pgfsetfillcolor{currentfill}%
\pgfsetfillopacity{0.500000}%
\pgfsetlinewidth{0.250937pt}%
\definecolor{currentstroke}{rgb}{0.000000,0.000000,0.000000}%
\pgfsetstrokecolor{currentstroke}%
\pgfsetstrokeopacity{0.500000}%
\pgfsetdash{}{0pt}%
\pgfsys@defobject{currentmarker}{\pgfqpoint{-0.050000in}{-0.050000in}}{\pgfqpoint{0.050000in}{0.050000in}}{%
\pgfpathmoveto{\pgfqpoint{0.000000in}{-0.050000in}}%
\pgfpathcurveto{\pgfqpoint{0.013260in}{-0.050000in}}{\pgfqpoint{0.025979in}{-0.044732in}}{\pgfqpoint{0.035355in}{-0.035355in}}%
\pgfpathcurveto{\pgfqpoint{0.044732in}{-0.025979in}}{\pgfqpoint{0.050000in}{-0.013260in}}{\pgfqpoint{0.050000in}{0.000000in}}%
\pgfpathcurveto{\pgfqpoint{0.050000in}{0.013260in}}{\pgfqpoint{0.044732in}{0.025979in}}{\pgfqpoint{0.035355in}{0.035355in}}%
\pgfpathcurveto{\pgfqpoint{0.025979in}{0.044732in}}{\pgfqpoint{0.013260in}{0.050000in}}{\pgfqpoint{0.000000in}{0.050000in}}%
\pgfpathcurveto{\pgfqpoint{-0.013260in}{0.050000in}}{\pgfqpoint{-0.025979in}{0.044732in}}{\pgfqpoint{-0.035355in}{0.035355in}}%
\pgfpathcurveto{\pgfqpoint{-0.044732in}{0.025979in}}{\pgfqpoint{-0.050000in}{0.013260in}}{\pgfqpoint{-0.050000in}{0.000000in}}%
\pgfpathcurveto{\pgfqpoint{-0.050000in}{-0.013260in}}{\pgfqpoint{-0.044732in}{-0.025979in}}{\pgfqpoint{-0.035355in}{-0.035355in}}%
\pgfpathcurveto{\pgfqpoint{-0.025979in}{-0.044732in}}{\pgfqpoint{-0.013260in}{-0.050000in}}{\pgfqpoint{0.000000in}{-0.050000in}}%
\pgfpathclose%
\pgfusepath{stroke,fill}%
}%
\begin{pgfscope}%
\pgfsys@transformshift{3.579642in}{4.628409in}%
\pgfsys@useobject{currentmarker}{}%
\end{pgfscope}%
\end{pgfscope}%
\begin{pgfscope}%
\pgfpathrectangle{\pgfqpoint{0.100000in}{2.413063in}}{\pgfqpoint{5.037500in}{3.427208in}}%
\pgfusepath{clip}%
\pgfsetrectcap%
\pgfsetroundjoin%
\pgfsetlinewidth{1.505625pt}%
\definecolor{currentstroke}{rgb}{0.678431,1.000000,0.184314}%
\pgfsetstrokecolor{currentstroke}%
\pgfsetstrokeopacity{0.500000}%
\pgfsetdash{}{0pt}%
\pgfpathmoveto{\pgfqpoint{3.479193in}{4.190553in}}%
\pgfusepath{stroke}%
\end{pgfscope}%
\begin{pgfscope}%
\pgfpathrectangle{\pgfqpoint{0.100000in}{2.413063in}}{\pgfqpoint{5.037500in}{3.427208in}}%
\pgfusepath{clip}%
\pgfsetbuttcap%
\pgfsetroundjoin%
\definecolor{currentfill}{rgb}{0.678431,1.000000,0.184314}%
\pgfsetfillcolor{currentfill}%
\pgfsetfillopacity{0.500000}%
\pgfsetlinewidth{0.250937pt}%
\definecolor{currentstroke}{rgb}{0.000000,0.000000,0.000000}%
\pgfsetstrokecolor{currentstroke}%
\pgfsetstrokeopacity{0.500000}%
\pgfsetdash{}{0pt}%
\pgfsys@defobject{currentmarker}{\pgfqpoint{-0.036111in}{-0.036111in}}{\pgfqpoint{0.036111in}{0.036111in}}{%
\pgfpathmoveto{\pgfqpoint{0.000000in}{-0.036111in}}%
\pgfpathcurveto{\pgfqpoint{0.009577in}{-0.036111in}}{\pgfqpoint{0.018763in}{-0.032306in}}{\pgfqpoint{0.025534in}{-0.025534in}}%
\pgfpathcurveto{\pgfqpoint{0.032306in}{-0.018763in}}{\pgfqpoint{0.036111in}{-0.009577in}}{\pgfqpoint{0.036111in}{0.000000in}}%
\pgfpathcurveto{\pgfqpoint{0.036111in}{0.009577in}}{\pgfqpoint{0.032306in}{0.018763in}}{\pgfqpoint{0.025534in}{0.025534in}}%
\pgfpathcurveto{\pgfqpoint{0.018763in}{0.032306in}}{\pgfqpoint{0.009577in}{0.036111in}}{\pgfqpoint{0.000000in}{0.036111in}}%
\pgfpathcurveto{\pgfqpoint{-0.009577in}{0.036111in}}{\pgfqpoint{-0.018763in}{0.032306in}}{\pgfqpoint{-0.025534in}{0.025534in}}%
\pgfpathcurveto{\pgfqpoint{-0.032306in}{0.018763in}}{\pgfqpoint{-0.036111in}{0.009577in}}{\pgfqpoint{-0.036111in}{0.000000in}}%
\pgfpathcurveto{\pgfqpoint{-0.036111in}{-0.009577in}}{\pgfqpoint{-0.032306in}{-0.018763in}}{\pgfqpoint{-0.025534in}{-0.025534in}}%
\pgfpathcurveto{\pgfqpoint{-0.018763in}{-0.032306in}}{\pgfqpoint{-0.009577in}{-0.036111in}}{\pgfqpoint{0.000000in}{-0.036111in}}%
\pgfpathclose%
\pgfusepath{stroke,fill}%
}%
\begin{pgfscope}%
\pgfsys@transformshift{3.479193in}{4.190553in}%
\pgfsys@useobject{currentmarker}{}%
\end{pgfscope}%
\end{pgfscope}%
\begin{pgfscope}%
\pgfpathrectangle{\pgfqpoint{0.100000in}{2.413063in}}{\pgfqpoint{5.037500in}{3.427208in}}%
\pgfusepath{clip}%
\pgfsetrectcap%
\pgfsetroundjoin%
\pgfsetlinewidth{1.505625pt}%
\definecolor{currentstroke}{rgb}{0.678431,1.000000,0.184314}%
\pgfsetstrokecolor{currentstroke}%
\pgfsetstrokeopacity{0.500000}%
\pgfsetdash{}{0pt}%
\pgfpathmoveto{\pgfqpoint{3.658913in}{4.566924in}}%
\pgfusepath{stroke}%
\end{pgfscope}%
\begin{pgfscope}%
\pgfpathrectangle{\pgfqpoint{0.100000in}{2.413063in}}{\pgfqpoint{5.037500in}{3.427208in}}%
\pgfusepath{clip}%
\pgfsetbuttcap%
\pgfsetroundjoin%
\definecolor{currentfill}{rgb}{0.678431,1.000000,0.184314}%
\pgfsetfillcolor{currentfill}%
\pgfsetfillopacity{0.500000}%
\pgfsetlinewidth{0.250937pt}%
\definecolor{currentstroke}{rgb}{0.000000,0.000000,0.000000}%
\pgfsetstrokecolor{currentstroke}%
\pgfsetstrokeopacity{0.500000}%
\pgfsetdash{}{0pt}%
\pgfsys@defobject{currentmarker}{\pgfqpoint{-0.041667in}{-0.041667in}}{\pgfqpoint{0.041667in}{0.041667in}}{%
\pgfpathmoveto{\pgfqpoint{0.000000in}{-0.041667in}}%
\pgfpathcurveto{\pgfqpoint{0.011050in}{-0.041667in}}{\pgfqpoint{0.021649in}{-0.037276in}}{\pgfqpoint{0.029463in}{-0.029463in}}%
\pgfpathcurveto{\pgfqpoint{0.037276in}{-0.021649in}}{\pgfqpoint{0.041667in}{-0.011050in}}{\pgfqpoint{0.041667in}{0.000000in}}%
\pgfpathcurveto{\pgfqpoint{0.041667in}{0.011050in}}{\pgfqpoint{0.037276in}{0.021649in}}{\pgfqpoint{0.029463in}{0.029463in}}%
\pgfpathcurveto{\pgfqpoint{0.021649in}{0.037276in}}{\pgfqpoint{0.011050in}{0.041667in}}{\pgfqpoint{0.000000in}{0.041667in}}%
\pgfpathcurveto{\pgfqpoint{-0.011050in}{0.041667in}}{\pgfqpoint{-0.021649in}{0.037276in}}{\pgfqpoint{-0.029463in}{0.029463in}}%
\pgfpathcurveto{\pgfqpoint{-0.037276in}{0.021649in}}{\pgfqpoint{-0.041667in}{0.011050in}}{\pgfqpoint{-0.041667in}{0.000000in}}%
\pgfpathcurveto{\pgfqpoint{-0.041667in}{-0.011050in}}{\pgfqpoint{-0.037276in}{-0.021649in}}{\pgfqpoint{-0.029463in}{-0.029463in}}%
\pgfpathcurveto{\pgfqpoint{-0.021649in}{-0.037276in}}{\pgfqpoint{-0.011050in}{-0.041667in}}{\pgfqpoint{0.000000in}{-0.041667in}}%
\pgfpathclose%
\pgfusepath{stroke,fill}%
}%
\begin{pgfscope}%
\pgfsys@transformshift{3.658913in}{4.566924in}%
\pgfsys@useobject{currentmarker}{}%
\end{pgfscope}%
\end{pgfscope}%
\begin{pgfscope}%
\pgfpathrectangle{\pgfqpoint{0.100000in}{2.413063in}}{\pgfqpoint{5.037500in}{3.427208in}}%
\pgfusepath{clip}%
\pgfsetrectcap%
\pgfsetroundjoin%
\pgfsetlinewidth{1.505625pt}%
\definecolor{currentstroke}{rgb}{0.678431,1.000000,0.184314}%
\pgfsetstrokecolor{currentstroke}%
\pgfsetstrokeopacity{0.500000}%
\pgfsetdash{}{0pt}%
\pgfpathmoveto{\pgfqpoint{3.623626in}{4.450288in}}%
\pgfusepath{stroke}%
\end{pgfscope}%
\begin{pgfscope}%
\pgfpathrectangle{\pgfqpoint{0.100000in}{2.413063in}}{\pgfqpoint{5.037500in}{3.427208in}}%
\pgfusepath{clip}%
\pgfsetbuttcap%
\pgfsetroundjoin%
\definecolor{currentfill}{rgb}{0.678431,1.000000,0.184314}%
\pgfsetfillcolor{currentfill}%
\pgfsetfillopacity{0.500000}%
\pgfsetlinewidth{0.250937pt}%
\definecolor{currentstroke}{rgb}{0.000000,0.000000,0.000000}%
\pgfsetstrokecolor{currentstroke}%
\pgfsetstrokeopacity{0.500000}%
\pgfsetdash{}{0pt}%
\pgfsys@defobject{currentmarker}{\pgfqpoint{-0.036111in}{-0.036111in}}{\pgfqpoint{0.036111in}{0.036111in}}{%
\pgfpathmoveto{\pgfqpoint{0.000000in}{-0.036111in}}%
\pgfpathcurveto{\pgfqpoint{0.009577in}{-0.036111in}}{\pgfqpoint{0.018763in}{-0.032306in}}{\pgfqpoint{0.025534in}{-0.025534in}}%
\pgfpathcurveto{\pgfqpoint{0.032306in}{-0.018763in}}{\pgfqpoint{0.036111in}{-0.009577in}}{\pgfqpoint{0.036111in}{0.000000in}}%
\pgfpathcurveto{\pgfqpoint{0.036111in}{0.009577in}}{\pgfqpoint{0.032306in}{0.018763in}}{\pgfqpoint{0.025534in}{0.025534in}}%
\pgfpathcurveto{\pgfqpoint{0.018763in}{0.032306in}}{\pgfqpoint{0.009577in}{0.036111in}}{\pgfqpoint{0.000000in}{0.036111in}}%
\pgfpathcurveto{\pgfqpoint{-0.009577in}{0.036111in}}{\pgfqpoint{-0.018763in}{0.032306in}}{\pgfqpoint{-0.025534in}{0.025534in}}%
\pgfpathcurveto{\pgfqpoint{-0.032306in}{0.018763in}}{\pgfqpoint{-0.036111in}{0.009577in}}{\pgfqpoint{-0.036111in}{0.000000in}}%
\pgfpathcurveto{\pgfqpoint{-0.036111in}{-0.009577in}}{\pgfqpoint{-0.032306in}{-0.018763in}}{\pgfqpoint{-0.025534in}{-0.025534in}}%
\pgfpathcurveto{\pgfqpoint{-0.018763in}{-0.032306in}}{\pgfqpoint{-0.009577in}{-0.036111in}}{\pgfqpoint{0.000000in}{-0.036111in}}%
\pgfpathclose%
\pgfusepath{stroke,fill}%
}%
\begin{pgfscope}%
\pgfsys@transformshift{3.623626in}{4.450288in}%
\pgfsys@useobject{currentmarker}{}%
\end{pgfscope}%
\end{pgfscope}%
\begin{pgfscope}%
\pgfpathrectangle{\pgfqpoint{0.100000in}{2.413063in}}{\pgfqpoint{5.037500in}{3.427208in}}%
\pgfusepath{clip}%
\pgfsetrectcap%
\pgfsetroundjoin%
\pgfsetlinewidth{1.505625pt}%
\definecolor{currentstroke}{rgb}{0.678431,1.000000,0.184314}%
\pgfsetstrokecolor{currentstroke}%
\pgfsetstrokeopacity{0.500000}%
\pgfsetdash{}{0pt}%
\pgfpathmoveto{\pgfqpoint{3.585492in}{4.407463in}}%
\pgfusepath{stroke}%
\end{pgfscope}%
\begin{pgfscope}%
\pgfpathrectangle{\pgfqpoint{0.100000in}{2.413063in}}{\pgfqpoint{5.037500in}{3.427208in}}%
\pgfusepath{clip}%
\pgfsetbuttcap%
\pgfsetroundjoin%
\definecolor{currentfill}{rgb}{0.678431,1.000000,0.184314}%
\pgfsetfillcolor{currentfill}%
\pgfsetfillopacity{0.500000}%
\pgfsetlinewidth{0.250937pt}%
\definecolor{currentstroke}{rgb}{0.000000,0.000000,0.000000}%
\pgfsetstrokecolor{currentstroke}%
\pgfsetstrokeopacity{0.500000}%
\pgfsetdash{}{0pt}%
\pgfsys@defobject{currentmarker}{\pgfqpoint{-0.036111in}{-0.036111in}}{\pgfqpoint{0.036111in}{0.036111in}}{%
\pgfpathmoveto{\pgfqpoint{0.000000in}{-0.036111in}}%
\pgfpathcurveto{\pgfqpoint{0.009577in}{-0.036111in}}{\pgfqpoint{0.018763in}{-0.032306in}}{\pgfqpoint{0.025534in}{-0.025534in}}%
\pgfpathcurveto{\pgfqpoint{0.032306in}{-0.018763in}}{\pgfqpoint{0.036111in}{-0.009577in}}{\pgfqpoint{0.036111in}{0.000000in}}%
\pgfpathcurveto{\pgfqpoint{0.036111in}{0.009577in}}{\pgfqpoint{0.032306in}{0.018763in}}{\pgfqpoint{0.025534in}{0.025534in}}%
\pgfpathcurveto{\pgfqpoint{0.018763in}{0.032306in}}{\pgfqpoint{0.009577in}{0.036111in}}{\pgfqpoint{0.000000in}{0.036111in}}%
\pgfpathcurveto{\pgfqpoint{-0.009577in}{0.036111in}}{\pgfqpoint{-0.018763in}{0.032306in}}{\pgfqpoint{-0.025534in}{0.025534in}}%
\pgfpathcurveto{\pgfqpoint{-0.032306in}{0.018763in}}{\pgfqpoint{-0.036111in}{0.009577in}}{\pgfqpoint{-0.036111in}{0.000000in}}%
\pgfpathcurveto{\pgfqpoint{-0.036111in}{-0.009577in}}{\pgfqpoint{-0.032306in}{-0.018763in}}{\pgfqpoint{-0.025534in}{-0.025534in}}%
\pgfpathcurveto{\pgfqpoint{-0.018763in}{-0.032306in}}{\pgfqpoint{-0.009577in}{-0.036111in}}{\pgfqpoint{0.000000in}{-0.036111in}}%
\pgfpathclose%
\pgfusepath{stroke,fill}%
}%
\begin{pgfscope}%
\pgfsys@transformshift{3.585492in}{4.407463in}%
\pgfsys@useobject{currentmarker}{}%
\end{pgfscope}%
\end{pgfscope}%
\begin{pgfscope}%
\pgfpathrectangle{\pgfqpoint{0.100000in}{2.413063in}}{\pgfqpoint{5.037500in}{3.427208in}}%
\pgfusepath{clip}%
\pgfsetrectcap%
\pgfsetroundjoin%
\pgfsetlinewidth{1.505625pt}%
\definecolor{currentstroke}{rgb}{0.678431,1.000000,0.184314}%
\pgfsetstrokecolor{currentstroke}%
\pgfsetstrokeopacity{0.500000}%
\pgfsetdash{}{0pt}%
\pgfpathmoveto{\pgfqpoint{3.579613in}{4.489985in}}%
\pgfusepath{stroke}%
\end{pgfscope}%
\begin{pgfscope}%
\pgfpathrectangle{\pgfqpoint{0.100000in}{2.413063in}}{\pgfqpoint{5.037500in}{3.427208in}}%
\pgfusepath{clip}%
\pgfsetbuttcap%
\pgfsetroundjoin%
\definecolor{currentfill}{rgb}{0.678431,1.000000,0.184314}%
\pgfsetfillcolor{currentfill}%
\pgfsetfillopacity{0.500000}%
\pgfsetlinewidth{0.250937pt}%
\definecolor{currentstroke}{rgb}{0.000000,0.000000,0.000000}%
\pgfsetstrokecolor{currentstroke}%
\pgfsetstrokeopacity{0.500000}%
\pgfsetdash{}{0pt}%
\pgfsys@defobject{currentmarker}{\pgfqpoint{-0.027778in}{-0.027778in}}{\pgfqpoint{0.027778in}{0.027778in}}{%
\pgfpathmoveto{\pgfqpoint{0.000000in}{-0.027778in}}%
\pgfpathcurveto{\pgfqpoint{0.007367in}{-0.027778in}}{\pgfqpoint{0.014433in}{-0.024851in}}{\pgfqpoint{0.019642in}{-0.019642in}}%
\pgfpathcurveto{\pgfqpoint{0.024851in}{-0.014433in}}{\pgfqpoint{0.027778in}{-0.007367in}}{\pgfqpoint{0.027778in}{0.000000in}}%
\pgfpathcurveto{\pgfqpoint{0.027778in}{0.007367in}}{\pgfqpoint{0.024851in}{0.014433in}}{\pgfqpoint{0.019642in}{0.019642in}}%
\pgfpathcurveto{\pgfqpoint{0.014433in}{0.024851in}}{\pgfqpoint{0.007367in}{0.027778in}}{\pgfqpoint{0.000000in}{0.027778in}}%
\pgfpathcurveto{\pgfqpoint{-0.007367in}{0.027778in}}{\pgfqpoint{-0.014433in}{0.024851in}}{\pgfqpoint{-0.019642in}{0.019642in}}%
\pgfpathcurveto{\pgfqpoint{-0.024851in}{0.014433in}}{\pgfqpoint{-0.027778in}{0.007367in}}{\pgfqpoint{-0.027778in}{0.000000in}}%
\pgfpathcurveto{\pgfqpoint{-0.027778in}{-0.007367in}}{\pgfqpoint{-0.024851in}{-0.014433in}}{\pgfqpoint{-0.019642in}{-0.019642in}}%
\pgfpathcurveto{\pgfqpoint{-0.014433in}{-0.024851in}}{\pgfqpoint{-0.007367in}{-0.027778in}}{\pgfqpoint{0.000000in}{-0.027778in}}%
\pgfpathclose%
\pgfusepath{stroke,fill}%
}%
\begin{pgfscope}%
\pgfsys@transformshift{3.579613in}{4.489985in}%
\pgfsys@useobject{currentmarker}{}%
\end{pgfscope}%
\end{pgfscope}%
\begin{pgfscope}%
\pgfpathrectangle{\pgfqpoint{0.100000in}{2.413063in}}{\pgfqpoint{5.037500in}{3.427208in}}%
\pgfusepath{clip}%
\pgfsetrectcap%
\pgfsetroundjoin%
\pgfsetlinewidth{1.505625pt}%
\definecolor{currentstroke}{rgb}{0.678431,1.000000,0.184314}%
\pgfsetstrokecolor{currentstroke}%
\pgfsetstrokeopacity{0.500000}%
\pgfsetdash{}{0pt}%
\pgfpathmoveto{\pgfqpoint{3.515644in}{4.475920in}}%
\pgfusepath{stroke}%
\end{pgfscope}%
\begin{pgfscope}%
\pgfpathrectangle{\pgfqpoint{0.100000in}{2.413063in}}{\pgfqpoint{5.037500in}{3.427208in}}%
\pgfusepath{clip}%
\pgfsetbuttcap%
\pgfsetroundjoin%
\definecolor{currentfill}{rgb}{0.678431,1.000000,0.184314}%
\pgfsetfillcolor{currentfill}%
\pgfsetfillopacity{0.500000}%
\pgfsetlinewidth{0.250937pt}%
\definecolor{currentstroke}{rgb}{0.000000,0.000000,0.000000}%
\pgfsetstrokecolor{currentstroke}%
\pgfsetstrokeopacity{0.500000}%
\pgfsetdash{}{0pt}%
\pgfsys@defobject{currentmarker}{\pgfqpoint{-0.036111in}{-0.036111in}}{\pgfqpoint{0.036111in}{0.036111in}}{%
\pgfpathmoveto{\pgfqpoint{0.000000in}{-0.036111in}}%
\pgfpathcurveto{\pgfqpoint{0.009577in}{-0.036111in}}{\pgfqpoint{0.018763in}{-0.032306in}}{\pgfqpoint{0.025534in}{-0.025534in}}%
\pgfpathcurveto{\pgfqpoint{0.032306in}{-0.018763in}}{\pgfqpoint{0.036111in}{-0.009577in}}{\pgfqpoint{0.036111in}{0.000000in}}%
\pgfpathcurveto{\pgfqpoint{0.036111in}{0.009577in}}{\pgfqpoint{0.032306in}{0.018763in}}{\pgfqpoint{0.025534in}{0.025534in}}%
\pgfpathcurveto{\pgfqpoint{0.018763in}{0.032306in}}{\pgfqpoint{0.009577in}{0.036111in}}{\pgfqpoint{0.000000in}{0.036111in}}%
\pgfpathcurveto{\pgfqpoint{-0.009577in}{0.036111in}}{\pgfqpoint{-0.018763in}{0.032306in}}{\pgfqpoint{-0.025534in}{0.025534in}}%
\pgfpathcurveto{\pgfqpoint{-0.032306in}{0.018763in}}{\pgfqpoint{-0.036111in}{0.009577in}}{\pgfqpoint{-0.036111in}{0.000000in}}%
\pgfpathcurveto{\pgfqpoint{-0.036111in}{-0.009577in}}{\pgfqpoint{-0.032306in}{-0.018763in}}{\pgfqpoint{-0.025534in}{-0.025534in}}%
\pgfpathcurveto{\pgfqpoint{-0.018763in}{-0.032306in}}{\pgfqpoint{-0.009577in}{-0.036111in}}{\pgfqpoint{0.000000in}{-0.036111in}}%
\pgfpathclose%
\pgfusepath{stroke,fill}%
}%
\begin{pgfscope}%
\pgfsys@transformshift{3.515644in}{4.475920in}%
\pgfsys@useobject{currentmarker}{}%
\end{pgfscope}%
\end{pgfscope}%
\begin{pgfscope}%
\pgfpathrectangle{\pgfqpoint{0.100000in}{2.413063in}}{\pgfqpoint{5.037500in}{3.427208in}}%
\pgfusepath{clip}%
\pgfsetrectcap%
\pgfsetroundjoin%
\pgfsetlinewidth{1.505625pt}%
\definecolor{currentstroke}{rgb}{0.678431,1.000000,0.184314}%
\pgfsetstrokecolor{currentstroke}%
\pgfsetstrokeopacity{0.500000}%
\pgfsetdash{}{0pt}%
\pgfpathmoveto{\pgfqpoint{3.500672in}{4.623903in}}%
\pgfusepath{stroke}%
\end{pgfscope}%
\begin{pgfscope}%
\pgfpathrectangle{\pgfqpoint{0.100000in}{2.413063in}}{\pgfqpoint{5.037500in}{3.427208in}}%
\pgfusepath{clip}%
\pgfsetbuttcap%
\pgfsetroundjoin%
\definecolor{currentfill}{rgb}{0.678431,1.000000,0.184314}%
\pgfsetfillcolor{currentfill}%
\pgfsetfillopacity{0.500000}%
\pgfsetlinewidth{0.250937pt}%
\definecolor{currentstroke}{rgb}{0.000000,0.000000,0.000000}%
\pgfsetstrokecolor{currentstroke}%
\pgfsetstrokeopacity{0.500000}%
\pgfsetdash{}{0pt}%
\pgfsys@defobject{currentmarker}{\pgfqpoint{-0.058333in}{-0.058333in}}{\pgfqpoint{0.058333in}{0.058333in}}{%
\pgfpathmoveto{\pgfqpoint{0.000000in}{-0.058333in}}%
\pgfpathcurveto{\pgfqpoint{0.015470in}{-0.058333in}}{\pgfqpoint{0.030309in}{-0.052187in}}{\pgfqpoint{0.041248in}{-0.041248in}}%
\pgfpathcurveto{\pgfqpoint{0.052187in}{-0.030309in}}{\pgfqpoint{0.058333in}{-0.015470in}}{\pgfqpoint{0.058333in}{0.000000in}}%
\pgfpathcurveto{\pgfqpoint{0.058333in}{0.015470in}}{\pgfqpoint{0.052187in}{0.030309in}}{\pgfqpoint{0.041248in}{0.041248in}}%
\pgfpathcurveto{\pgfqpoint{0.030309in}{0.052187in}}{\pgfqpoint{0.015470in}{0.058333in}}{\pgfqpoint{0.000000in}{0.058333in}}%
\pgfpathcurveto{\pgfqpoint{-0.015470in}{0.058333in}}{\pgfqpoint{-0.030309in}{0.052187in}}{\pgfqpoint{-0.041248in}{0.041248in}}%
\pgfpathcurveto{\pgfqpoint{-0.052187in}{0.030309in}}{\pgfqpoint{-0.058333in}{0.015470in}}{\pgfqpoint{-0.058333in}{0.000000in}}%
\pgfpathcurveto{\pgfqpoint{-0.058333in}{-0.015470in}}{\pgfqpoint{-0.052187in}{-0.030309in}}{\pgfqpoint{-0.041248in}{-0.041248in}}%
\pgfpathcurveto{\pgfqpoint{-0.030309in}{-0.052187in}}{\pgfqpoint{-0.015470in}{-0.058333in}}{\pgfqpoint{0.000000in}{-0.058333in}}%
\pgfpathclose%
\pgfusepath{stroke,fill}%
}%
\begin{pgfscope}%
\pgfsys@transformshift{3.500672in}{4.623903in}%
\pgfsys@useobject{currentmarker}{}%
\end{pgfscope}%
\end{pgfscope}%
\begin{pgfscope}%
\pgfpathrectangle{\pgfqpoint{0.100000in}{2.413063in}}{\pgfqpoint{5.037500in}{3.427208in}}%
\pgfusepath{clip}%
\pgfsetrectcap%
\pgfsetroundjoin%
\pgfsetlinewidth{1.505625pt}%
\definecolor{currentstroke}{rgb}{0.678431,1.000000,0.184314}%
\pgfsetstrokecolor{currentstroke}%
\pgfsetstrokeopacity{0.500000}%
\pgfsetdash{}{0pt}%
\pgfpathmoveto{\pgfqpoint{3.648286in}{4.463096in}}%
\pgfusepath{stroke}%
\end{pgfscope}%
\begin{pgfscope}%
\pgfpathrectangle{\pgfqpoint{0.100000in}{2.413063in}}{\pgfqpoint{5.037500in}{3.427208in}}%
\pgfusepath{clip}%
\pgfsetbuttcap%
\pgfsetroundjoin%
\definecolor{currentfill}{rgb}{0.678431,1.000000,0.184314}%
\pgfsetfillcolor{currentfill}%
\pgfsetfillopacity{0.500000}%
\pgfsetlinewidth{0.250937pt}%
\definecolor{currentstroke}{rgb}{0.000000,0.000000,0.000000}%
\pgfsetstrokecolor{currentstroke}%
\pgfsetstrokeopacity{0.500000}%
\pgfsetdash{}{0pt}%
\pgfsys@defobject{currentmarker}{\pgfqpoint{-0.050000in}{-0.050000in}}{\pgfqpoint{0.050000in}{0.050000in}}{%
\pgfpathmoveto{\pgfqpoint{0.000000in}{-0.050000in}}%
\pgfpathcurveto{\pgfqpoint{0.013260in}{-0.050000in}}{\pgfqpoint{0.025979in}{-0.044732in}}{\pgfqpoint{0.035355in}{-0.035355in}}%
\pgfpathcurveto{\pgfqpoint{0.044732in}{-0.025979in}}{\pgfqpoint{0.050000in}{-0.013260in}}{\pgfqpoint{0.050000in}{0.000000in}}%
\pgfpathcurveto{\pgfqpoint{0.050000in}{0.013260in}}{\pgfqpoint{0.044732in}{0.025979in}}{\pgfqpoint{0.035355in}{0.035355in}}%
\pgfpathcurveto{\pgfqpoint{0.025979in}{0.044732in}}{\pgfqpoint{0.013260in}{0.050000in}}{\pgfqpoint{0.000000in}{0.050000in}}%
\pgfpathcurveto{\pgfqpoint{-0.013260in}{0.050000in}}{\pgfqpoint{-0.025979in}{0.044732in}}{\pgfqpoint{-0.035355in}{0.035355in}}%
\pgfpathcurveto{\pgfqpoint{-0.044732in}{0.025979in}}{\pgfqpoint{-0.050000in}{0.013260in}}{\pgfqpoint{-0.050000in}{0.000000in}}%
\pgfpathcurveto{\pgfqpoint{-0.050000in}{-0.013260in}}{\pgfqpoint{-0.044732in}{-0.025979in}}{\pgfqpoint{-0.035355in}{-0.035355in}}%
\pgfpathcurveto{\pgfqpoint{-0.025979in}{-0.044732in}}{\pgfqpoint{-0.013260in}{-0.050000in}}{\pgfqpoint{0.000000in}{-0.050000in}}%
\pgfpathclose%
\pgfusepath{stroke,fill}%
}%
\begin{pgfscope}%
\pgfsys@transformshift{3.648286in}{4.463096in}%
\pgfsys@useobject{currentmarker}{}%
\end{pgfscope}%
\end{pgfscope}%
\begin{pgfscope}%
\pgfpathrectangle{\pgfqpoint{0.100000in}{2.413063in}}{\pgfqpoint{5.037500in}{3.427208in}}%
\pgfusepath{clip}%
\pgfsetrectcap%
\pgfsetroundjoin%
\pgfsetlinewidth{1.505625pt}%
\definecolor{currentstroke}{rgb}{0.678431,1.000000,0.184314}%
\pgfsetstrokecolor{currentstroke}%
\pgfsetstrokeopacity{0.500000}%
\pgfsetdash{}{0pt}%
\pgfpathmoveto{\pgfqpoint{3.556209in}{4.626271in}}%
\pgfusepath{stroke}%
\end{pgfscope}%
\begin{pgfscope}%
\pgfpathrectangle{\pgfqpoint{0.100000in}{2.413063in}}{\pgfqpoint{5.037500in}{3.427208in}}%
\pgfusepath{clip}%
\pgfsetbuttcap%
\pgfsetroundjoin%
\definecolor{currentfill}{rgb}{0.678431,1.000000,0.184314}%
\pgfsetfillcolor{currentfill}%
\pgfsetfillopacity{0.500000}%
\pgfsetlinewidth{0.250937pt}%
\definecolor{currentstroke}{rgb}{0.000000,0.000000,0.000000}%
\pgfsetstrokecolor{currentstroke}%
\pgfsetstrokeopacity{0.500000}%
\pgfsetdash{}{0pt}%
\pgfsys@defobject{currentmarker}{\pgfqpoint{-0.041667in}{-0.041667in}}{\pgfqpoint{0.041667in}{0.041667in}}{%
\pgfpathmoveto{\pgfqpoint{0.000000in}{-0.041667in}}%
\pgfpathcurveto{\pgfqpoint{0.011050in}{-0.041667in}}{\pgfqpoint{0.021649in}{-0.037276in}}{\pgfqpoint{0.029463in}{-0.029463in}}%
\pgfpathcurveto{\pgfqpoint{0.037276in}{-0.021649in}}{\pgfqpoint{0.041667in}{-0.011050in}}{\pgfqpoint{0.041667in}{0.000000in}}%
\pgfpathcurveto{\pgfqpoint{0.041667in}{0.011050in}}{\pgfqpoint{0.037276in}{0.021649in}}{\pgfqpoint{0.029463in}{0.029463in}}%
\pgfpathcurveto{\pgfqpoint{0.021649in}{0.037276in}}{\pgfqpoint{0.011050in}{0.041667in}}{\pgfqpoint{0.000000in}{0.041667in}}%
\pgfpathcurveto{\pgfqpoint{-0.011050in}{0.041667in}}{\pgfqpoint{-0.021649in}{0.037276in}}{\pgfqpoint{-0.029463in}{0.029463in}}%
\pgfpathcurveto{\pgfqpoint{-0.037276in}{0.021649in}}{\pgfqpoint{-0.041667in}{0.011050in}}{\pgfqpoint{-0.041667in}{0.000000in}}%
\pgfpathcurveto{\pgfqpoint{-0.041667in}{-0.011050in}}{\pgfqpoint{-0.037276in}{-0.021649in}}{\pgfqpoint{-0.029463in}{-0.029463in}}%
\pgfpathcurveto{\pgfqpoint{-0.021649in}{-0.037276in}}{\pgfqpoint{-0.011050in}{-0.041667in}}{\pgfqpoint{0.000000in}{-0.041667in}}%
\pgfpathclose%
\pgfusepath{stroke,fill}%
}%
\begin{pgfscope}%
\pgfsys@transformshift{3.556209in}{4.626271in}%
\pgfsys@useobject{currentmarker}{}%
\end{pgfscope}%
\end{pgfscope}%
\begin{pgfscope}%
\pgfpathrectangle{\pgfqpoint{0.100000in}{2.413063in}}{\pgfqpoint{5.037500in}{3.427208in}}%
\pgfusepath{clip}%
\pgfsetrectcap%
\pgfsetroundjoin%
\pgfsetlinewidth{1.505625pt}%
\definecolor{currentstroke}{rgb}{0.678431,1.000000,0.184314}%
\pgfsetstrokecolor{currentstroke}%
\pgfsetstrokeopacity{0.500000}%
\pgfsetdash{}{0pt}%
\pgfpathmoveto{\pgfqpoint{3.477710in}{4.362820in}}%
\pgfusepath{stroke}%
\end{pgfscope}%
\begin{pgfscope}%
\pgfpathrectangle{\pgfqpoint{0.100000in}{2.413063in}}{\pgfqpoint{5.037500in}{3.427208in}}%
\pgfusepath{clip}%
\pgfsetbuttcap%
\pgfsetroundjoin%
\definecolor{currentfill}{rgb}{0.678431,1.000000,0.184314}%
\pgfsetfillcolor{currentfill}%
\pgfsetfillopacity{0.500000}%
\pgfsetlinewidth{0.250937pt}%
\definecolor{currentstroke}{rgb}{0.000000,0.000000,0.000000}%
\pgfsetstrokecolor{currentstroke}%
\pgfsetstrokeopacity{0.500000}%
\pgfsetdash{}{0pt}%
\pgfsys@defobject{currentmarker}{\pgfqpoint{-0.063889in}{-0.063889in}}{\pgfqpoint{0.063889in}{0.063889in}}{%
\pgfpathmoveto{\pgfqpoint{0.000000in}{-0.063889in}}%
\pgfpathcurveto{\pgfqpoint{0.016944in}{-0.063889in}}{\pgfqpoint{0.033195in}{-0.057157in}}{\pgfqpoint{0.045176in}{-0.045176in}}%
\pgfpathcurveto{\pgfqpoint{0.057157in}{-0.033195in}}{\pgfqpoint{0.063889in}{-0.016944in}}{\pgfqpoint{0.063889in}{0.000000in}}%
\pgfpathcurveto{\pgfqpoint{0.063889in}{0.016944in}}{\pgfqpoint{0.057157in}{0.033195in}}{\pgfqpoint{0.045176in}{0.045176in}}%
\pgfpathcurveto{\pgfqpoint{0.033195in}{0.057157in}}{\pgfqpoint{0.016944in}{0.063889in}}{\pgfqpoint{0.000000in}{0.063889in}}%
\pgfpathcurveto{\pgfqpoint{-0.016944in}{0.063889in}}{\pgfqpoint{-0.033195in}{0.057157in}}{\pgfqpoint{-0.045176in}{0.045176in}}%
\pgfpathcurveto{\pgfqpoint{-0.057157in}{0.033195in}}{\pgfqpoint{-0.063889in}{0.016944in}}{\pgfqpoint{-0.063889in}{0.000000in}}%
\pgfpathcurveto{\pgfqpoint{-0.063889in}{-0.016944in}}{\pgfqpoint{-0.057157in}{-0.033195in}}{\pgfqpoint{-0.045176in}{-0.045176in}}%
\pgfpathcurveto{\pgfqpoint{-0.033195in}{-0.057157in}}{\pgfqpoint{-0.016944in}{-0.063889in}}{\pgfqpoint{0.000000in}{-0.063889in}}%
\pgfpathclose%
\pgfusepath{stroke,fill}%
}%
\begin{pgfscope}%
\pgfsys@transformshift{3.477710in}{4.362820in}%
\pgfsys@useobject{currentmarker}{}%
\end{pgfscope}%
\end{pgfscope}%
\begin{pgfscope}%
\pgfpathrectangle{\pgfqpoint{0.100000in}{2.413063in}}{\pgfqpoint{5.037500in}{3.427208in}}%
\pgfusepath{clip}%
\pgfsetrectcap%
\pgfsetroundjoin%
\pgfsetlinewidth{1.505625pt}%
\definecolor{currentstroke}{rgb}{0.501961,0.501961,0.501961}%
\pgfsetstrokecolor{currentstroke}%
\pgfsetstrokeopacity{0.500000}%
\pgfsetdash{}{0pt}%
\pgfpathmoveto{\pgfqpoint{2.922572in}{4.630525in}}%
\pgfusepath{stroke}%
\end{pgfscope}%
\begin{pgfscope}%
\pgfpathrectangle{\pgfqpoint{0.100000in}{2.413063in}}{\pgfqpoint{5.037500in}{3.427208in}}%
\pgfusepath{clip}%
\pgfsetbuttcap%
\pgfsetroundjoin%
\definecolor{currentfill}{rgb}{0.501961,0.501961,0.501961}%
\pgfsetfillcolor{currentfill}%
\pgfsetfillopacity{0.500000}%
\pgfsetlinewidth{0.250937pt}%
\definecolor{currentstroke}{rgb}{0.000000,0.000000,0.000000}%
\pgfsetstrokecolor{currentstroke}%
\pgfsetstrokeopacity{0.500000}%
\pgfsetdash{}{0pt}%
\pgfsys@defobject{currentmarker}{\pgfqpoint{-0.013889in}{-0.013889in}}{\pgfqpoint{0.013889in}{0.013889in}}{%
\pgfpathmoveto{\pgfqpoint{0.000000in}{-0.013889in}}%
\pgfpathcurveto{\pgfqpoint{0.003683in}{-0.013889in}}{\pgfqpoint{0.007216in}{-0.012425in}}{\pgfqpoint{0.009821in}{-0.009821in}}%
\pgfpathcurveto{\pgfqpoint{0.012425in}{-0.007216in}}{\pgfqpoint{0.013889in}{-0.003683in}}{\pgfqpoint{0.013889in}{0.000000in}}%
\pgfpathcurveto{\pgfqpoint{0.013889in}{0.003683in}}{\pgfqpoint{0.012425in}{0.007216in}}{\pgfqpoint{0.009821in}{0.009821in}}%
\pgfpathcurveto{\pgfqpoint{0.007216in}{0.012425in}}{\pgfqpoint{0.003683in}{0.013889in}}{\pgfqpoint{0.000000in}{0.013889in}}%
\pgfpathcurveto{\pgfqpoint{-0.003683in}{0.013889in}}{\pgfqpoint{-0.007216in}{0.012425in}}{\pgfqpoint{-0.009821in}{0.009821in}}%
\pgfpathcurveto{\pgfqpoint{-0.012425in}{0.007216in}}{\pgfqpoint{-0.013889in}{0.003683in}}{\pgfqpoint{-0.013889in}{0.000000in}}%
\pgfpathcurveto{\pgfqpoint{-0.013889in}{-0.003683in}}{\pgfqpoint{-0.012425in}{-0.007216in}}{\pgfqpoint{-0.009821in}{-0.009821in}}%
\pgfpathcurveto{\pgfqpoint{-0.007216in}{-0.012425in}}{\pgfqpoint{-0.003683in}{-0.013889in}}{\pgfqpoint{0.000000in}{-0.013889in}}%
\pgfpathclose%
\pgfusepath{stroke,fill}%
}%
\begin{pgfscope}%
\pgfsys@transformshift{2.922572in}{4.630525in}%
\pgfsys@useobject{currentmarker}{}%
\end{pgfscope}%
\end{pgfscope}%
\begin{pgfscope}%
\pgfpathrectangle{\pgfqpoint{0.100000in}{2.413063in}}{\pgfqpoint{5.037500in}{3.427208in}}%
\pgfusepath{clip}%
\pgfsetrectcap%
\pgfsetroundjoin%
\pgfsetlinewidth{1.505625pt}%
\definecolor{currentstroke}{rgb}{0.501961,0.501961,0.501961}%
\pgfsetstrokecolor{currentstroke}%
\pgfsetstrokeopacity{0.500000}%
\pgfsetdash{}{0pt}%
\pgfpathmoveto{\pgfqpoint{3.089457in}{4.628984in}}%
\pgfusepath{stroke}%
\end{pgfscope}%
\begin{pgfscope}%
\pgfpathrectangle{\pgfqpoint{0.100000in}{2.413063in}}{\pgfqpoint{5.037500in}{3.427208in}}%
\pgfusepath{clip}%
\pgfsetbuttcap%
\pgfsetroundjoin%
\definecolor{currentfill}{rgb}{0.501961,0.501961,0.501961}%
\pgfsetfillcolor{currentfill}%
\pgfsetfillopacity{0.500000}%
\pgfsetlinewidth{0.250937pt}%
\definecolor{currentstroke}{rgb}{0.000000,0.000000,0.000000}%
\pgfsetstrokecolor{currentstroke}%
\pgfsetstrokeopacity{0.500000}%
\pgfsetdash{}{0pt}%
\pgfsys@defobject{currentmarker}{\pgfqpoint{-0.013889in}{-0.013889in}}{\pgfqpoint{0.013889in}{0.013889in}}{%
\pgfpathmoveto{\pgfqpoint{0.000000in}{-0.013889in}}%
\pgfpathcurveto{\pgfqpoint{0.003683in}{-0.013889in}}{\pgfqpoint{0.007216in}{-0.012425in}}{\pgfqpoint{0.009821in}{-0.009821in}}%
\pgfpathcurveto{\pgfqpoint{0.012425in}{-0.007216in}}{\pgfqpoint{0.013889in}{-0.003683in}}{\pgfqpoint{0.013889in}{0.000000in}}%
\pgfpathcurveto{\pgfqpoint{0.013889in}{0.003683in}}{\pgfqpoint{0.012425in}{0.007216in}}{\pgfqpoint{0.009821in}{0.009821in}}%
\pgfpathcurveto{\pgfqpoint{0.007216in}{0.012425in}}{\pgfqpoint{0.003683in}{0.013889in}}{\pgfqpoint{0.000000in}{0.013889in}}%
\pgfpathcurveto{\pgfqpoint{-0.003683in}{0.013889in}}{\pgfqpoint{-0.007216in}{0.012425in}}{\pgfqpoint{-0.009821in}{0.009821in}}%
\pgfpathcurveto{\pgfqpoint{-0.012425in}{0.007216in}}{\pgfqpoint{-0.013889in}{0.003683in}}{\pgfqpoint{-0.013889in}{0.000000in}}%
\pgfpathcurveto{\pgfqpoint{-0.013889in}{-0.003683in}}{\pgfqpoint{-0.012425in}{-0.007216in}}{\pgfqpoint{-0.009821in}{-0.009821in}}%
\pgfpathcurveto{\pgfqpoint{-0.007216in}{-0.012425in}}{\pgfqpoint{-0.003683in}{-0.013889in}}{\pgfqpoint{0.000000in}{-0.013889in}}%
\pgfpathclose%
\pgfusepath{stroke,fill}%
}%
\begin{pgfscope}%
\pgfsys@transformshift{3.089457in}{4.628984in}%
\pgfsys@useobject{currentmarker}{}%
\end{pgfscope}%
\end{pgfscope}%
\begin{pgfscope}%
\pgfpathrectangle{\pgfqpoint{0.100000in}{2.413063in}}{\pgfqpoint{5.037500in}{3.427208in}}%
\pgfusepath{clip}%
\pgfsetrectcap%
\pgfsetroundjoin%
\pgfsetlinewidth{1.505625pt}%
\definecolor{currentstroke}{rgb}{0.501961,0.501961,0.501961}%
\pgfsetstrokecolor{currentstroke}%
\pgfsetstrokeopacity{0.500000}%
\pgfsetdash{}{0pt}%
\pgfpathmoveto{\pgfqpoint{2.924486in}{4.580320in}}%
\pgfusepath{stroke}%
\end{pgfscope}%
\begin{pgfscope}%
\pgfpathrectangle{\pgfqpoint{0.100000in}{2.413063in}}{\pgfqpoint{5.037500in}{3.427208in}}%
\pgfusepath{clip}%
\pgfsetbuttcap%
\pgfsetroundjoin%
\definecolor{currentfill}{rgb}{0.501961,0.501961,0.501961}%
\pgfsetfillcolor{currentfill}%
\pgfsetfillopacity{0.500000}%
\pgfsetlinewidth{0.250937pt}%
\definecolor{currentstroke}{rgb}{0.000000,0.000000,0.000000}%
\pgfsetstrokecolor{currentstroke}%
\pgfsetstrokeopacity{0.500000}%
\pgfsetdash{}{0pt}%
\pgfsys@defobject{currentmarker}{\pgfqpoint{-0.013889in}{-0.013889in}}{\pgfqpoint{0.013889in}{0.013889in}}{%
\pgfpathmoveto{\pgfqpoint{0.000000in}{-0.013889in}}%
\pgfpathcurveto{\pgfqpoint{0.003683in}{-0.013889in}}{\pgfqpoint{0.007216in}{-0.012425in}}{\pgfqpoint{0.009821in}{-0.009821in}}%
\pgfpathcurveto{\pgfqpoint{0.012425in}{-0.007216in}}{\pgfqpoint{0.013889in}{-0.003683in}}{\pgfqpoint{0.013889in}{0.000000in}}%
\pgfpathcurveto{\pgfqpoint{0.013889in}{0.003683in}}{\pgfqpoint{0.012425in}{0.007216in}}{\pgfqpoint{0.009821in}{0.009821in}}%
\pgfpathcurveto{\pgfqpoint{0.007216in}{0.012425in}}{\pgfqpoint{0.003683in}{0.013889in}}{\pgfqpoint{0.000000in}{0.013889in}}%
\pgfpathcurveto{\pgfqpoint{-0.003683in}{0.013889in}}{\pgfqpoint{-0.007216in}{0.012425in}}{\pgfqpoint{-0.009821in}{0.009821in}}%
\pgfpathcurveto{\pgfqpoint{-0.012425in}{0.007216in}}{\pgfqpoint{-0.013889in}{0.003683in}}{\pgfqpoint{-0.013889in}{0.000000in}}%
\pgfpathcurveto{\pgfqpoint{-0.013889in}{-0.003683in}}{\pgfqpoint{-0.012425in}{-0.007216in}}{\pgfqpoint{-0.009821in}{-0.009821in}}%
\pgfpathcurveto{\pgfqpoint{-0.007216in}{-0.012425in}}{\pgfqpoint{-0.003683in}{-0.013889in}}{\pgfqpoint{0.000000in}{-0.013889in}}%
\pgfpathclose%
\pgfusepath{stroke,fill}%
}%
\begin{pgfscope}%
\pgfsys@transformshift{2.924486in}{4.580320in}%
\pgfsys@useobject{currentmarker}{}%
\end{pgfscope}%
\end{pgfscope}%
\begin{pgfscope}%
\pgfpathrectangle{\pgfqpoint{0.100000in}{2.413063in}}{\pgfqpoint{5.037500in}{3.427208in}}%
\pgfusepath{clip}%
\pgfsetrectcap%
\pgfsetroundjoin%
\pgfsetlinewidth{1.505625pt}%
\definecolor{currentstroke}{rgb}{0.501961,0.501961,0.501961}%
\pgfsetstrokecolor{currentstroke}%
\pgfsetstrokeopacity{0.500000}%
\pgfsetdash{}{0pt}%
\pgfpathmoveto{\pgfqpoint{3.172689in}{4.693064in}}%
\pgfusepath{stroke}%
\end{pgfscope}%
\begin{pgfscope}%
\pgfpathrectangle{\pgfqpoint{0.100000in}{2.413063in}}{\pgfqpoint{5.037500in}{3.427208in}}%
\pgfusepath{clip}%
\pgfsetbuttcap%
\pgfsetroundjoin%
\definecolor{currentfill}{rgb}{0.501961,0.501961,0.501961}%
\pgfsetfillcolor{currentfill}%
\pgfsetfillopacity{0.500000}%
\pgfsetlinewidth{0.250937pt}%
\definecolor{currentstroke}{rgb}{0.000000,0.000000,0.000000}%
\pgfsetstrokecolor{currentstroke}%
\pgfsetstrokeopacity{0.500000}%
\pgfsetdash{}{0pt}%
\pgfsys@defobject{currentmarker}{\pgfqpoint{-0.013889in}{-0.013889in}}{\pgfqpoint{0.013889in}{0.013889in}}{%
\pgfpathmoveto{\pgfqpoint{0.000000in}{-0.013889in}}%
\pgfpathcurveto{\pgfqpoint{0.003683in}{-0.013889in}}{\pgfqpoint{0.007216in}{-0.012425in}}{\pgfqpoint{0.009821in}{-0.009821in}}%
\pgfpathcurveto{\pgfqpoint{0.012425in}{-0.007216in}}{\pgfqpoint{0.013889in}{-0.003683in}}{\pgfqpoint{0.013889in}{0.000000in}}%
\pgfpathcurveto{\pgfqpoint{0.013889in}{0.003683in}}{\pgfqpoint{0.012425in}{0.007216in}}{\pgfqpoint{0.009821in}{0.009821in}}%
\pgfpathcurveto{\pgfqpoint{0.007216in}{0.012425in}}{\pgfqpoint{0.003683in}{0.013889in}}{\pgfqpoint{0.000000in}{0.013889in}}%
\pgfpathcurveto{\pgfqpoint{-0.003683in}{0.013889in}}{\pgfqpoint{-0.007216in}{0.012425in}}{\pgfqpoint{-0.009821in}{0.009821in}}%
\pgfpathcurveto{\pgfqpoint{-0.012425in}{0.007216in}}{\pgfqpoint{-0.013889in}{0.003683in}}{\pgfqpoint{-0.013889in}{0.000000in}}%
\pgfpathcurveto{\pgfqpoint{-0.013889in}{-0.003683in}}{\pgfqpoint{-0.012425in}{-0.007216in}}{\pgfqpoint{-0.009821in}{-0.009821in}}%
\pgfpathcurveto{\pgfqpoint{-0.007216in}{-0.012425in}}{\pgfqpoint{-0.003683in}{-0.013889in}}{\pgfqpoint{0.000000in}{-0.013889in}}%
\pgfpathclose%
\pgfusepath{stroke,fill}%
}%
\begin{pgfscope}%
\pgfsys@transformshift{3.172689in}{4.693064in}%
\pgfsys@useobject{currentmarker}{}%
\end{pgfscope}%
\end{pgfscope}%
\begin{pgfscope}%
\pgfpathrectangle{\pgfqpoint{0.100000in}{2.413063in}}{\pgfqpoint{5.037500in}{3.427208in}}%
\pgfusepath{clip}%
\pgfsetrectcap%
\pgfsetroundjoin%
\pgfsetlinewidth{1.505625pt}%
\definecolor{currentstroke}{rgb}{0.000000,0.000000,1.000000}%
\pgfsetstrokecolor{currentstroke}%
\pgfsetstrokeopacity{0.500000}%
\pgfsetdash{}{0pt}%
\pgfpathmoveto{\pgfqpoint{3.102876in}{4.593191in}}%
\pgfusepath{stroke}%
\end{pgfscope}%
\begin{pgfscope}%
\pgfpathrectangle{\pgfqpoint{0.100000in}{2.413063in}}{\pgfqpoint{5.037500in}{3.427208in}}%
\pgfusepath{clip}%
\pgfsetbuttcap%
\pgfsetroundjoin%
\definecolor{currentfill}{rgb}{0.000000,0.000000,1.000000}%
\pgfsetfillcolor{currentfill}%
\pgfsetfillopacity{0.500000}%
\pgfsetlinewidth{0.250937pt}%
\definecolor{currentstroke}{rgb}{0.000000,0.000000,0.000000}%
\pgfsetstrokecolor{currentstroke}%
\pgfsetstrokeopacity{0.500000}%
\pgfsetdash{}{0pt}%
\pgfsys@defobject{currentmarker}{\pgfqpoint{-0.008333in}{-0.008333in}}{\pgfqpoint{0.008333in}{0.008333in}}{%
\pgfpathmoveto{\pgfqpoint{0.000000in}{-0.008333in}}%
\pgfpathcurveto{\pgfqpoint{0.002210in}{-0.008333in}}{\pgfqpoint{0.004330in}{-0.007455in}}{\pgfqpoint{0.005893in}{-0.005893in}}%
\pgfpathcurveto{\pgfqpoint{0.007455in}{-0.004330in}}{\pgfqpoint{0.008333in}{-0.002210in}}{\pgfqpoint{0.008333in}{0.000000in}}%
\pgfpathcurveto{\pgfqpoint{0.008333in}{0.002210in}}{\pgfqpoint{0.007455in}{0.004330in}}{\pgfqpoint{0.005893in}{0.005893in}}%
\pgfpathcurveto{\pgfqpoint{0.004330in}{0.007455in}}{\pgfqpoint{0.002210in}{0.008333in}}{\pgfqpoint{0.000000in}{0.008333in}}%
\pgfpathcurveto{\pgfqpoint{-0.002210in}{0.008333in}}{\pgfqpoint{-0.004330in}{0.007455in}}{\pgfqpoint{-0.005893in}{0.005893in}}%
\pgfpathcurveto{\pgfqpoint{-0.007455in}{0.004330in}}{\pgfqpoint{-0.008333in}{0.002210in}}{\pgfqpoint{-0.008333in}{0.000000in}}%
\pgfpathcurveto{\pgfqpoint{-0.008333in}{-0.002210in}}{\pgfqpoint{-0.007455in}{-0.004330in}}{\pgfqpoint{-0.005893in}{-0.005893in}}%
\pgfpathcurveto{\pgfqpoint{-0.004330in}{-0.007455in}}{\pgfqpoint{-0.002210in}{-0.008333in}}{\pgfqpoint{0.000000in}{-0.008333in}}%
\pgfpathclose%
\pgfusepath{stroke,fill}%
}%
\begin{pgfscope}%
\pgfsys@transformshift{3.102876in}{4.593191in}%
\pgfsys@useobject{currentmarker}{}%
\end{pgfscope}%
\end{pgfscope}%
\begin{pgfscope}%
\pgfpathrectangle{\pgfqpoint{0.100000in}{2.413063in}}{\pgfqpoint{5.037500in}{3.427208in}}%
\pgfusepath{clip}%
\pgfsetrectcap%
\pgfsetroundjoin%
\pgfsetlinewidth{1.505625pt}%
\definecolor{currentstroke}{rgb}{0.678431,1.000000,0.184314}%
\pgfsetstrokecolor{currentstroke}%
\pgfsetstrokeopacity{0.500000}%
\pgfsetdash{}{0pt}%
\pgfpathmoveto{\pgfqpoint{2.684559in}{4.684742in}}%
\pgfusepath{stroke}%
\end{pgfscope}%
\begin{pgfscope}%
\pgfpathrectangle{\pgfqpoint{0.100000in}{2.413063in}}{\pgfqpoint{5.037500in}{3.427208in}}%
\pgfusepath{clip}%
\pgfsetbuttcap%
\pgfsetroundjoin%
\definecolor{currentfill}{rgb}{0.678431,1.000000,0.184314}%
\pgfsetfillcolor{currentfill}%
\pgfsetfillopacity{0.500000}%
\pgfsetlinewidth{0.250937pt}%
\definecolor{currentstroke}{rgb}{0.000000,0.000000,0.000000}%
\pgfsetstrokecolor{currentstroke}%
\pgfsetstrokeopacity{0.500000}%
\pgfsetdash{}{0pt}%
\pgfsys@defobject{currentmarker}{\pgfqpoint{-0.008333in}{-0.008333in}}{\pgfqpoint{0.008333in}{0.008333in}}{%
\pgfpathmoveto{\pgfqpoint{0.000000in}{-0.008333in}}%
\pgfpathcurveto{\pgfqpoint{0.002210in}{-0.008333in}}{\pgfqpoint{0.004330in}{-0.007455in}}{\pgfqpoint{0.005893in}{-0.005893in}}%
\pgfpathcurveto{\pgfqpoint{0.007455in}{-0.004330in}}{\pgfqpoint{0.008333in}{-0.002210in}}{\pgfqpoint{0.008333in}{0.000000in}}%
\pgfpathcurveto{\pgfqpoint{0.008333in}{0.002210in}}{\pgfqpoint{0.007455in}{0.004330in}}{\pgfqpoint{0.005893in}{0.005893in}}%
\pgfpathcurveto{\pgfqpoint{0.004330in}{0.007455in}}{\pgfqpoint{0.002210in}{0.008333in}}{\pgfqpoint{0.000000in}{0.008333in}}%
\pgfpathcurveto{\pgfqpoint{-0.002210in}{0.008333in}}{\pgfqpoint{-0.004330in}{0.007455in}}{\pgfqpoint{-0.005893in}{0.005893in}}%
\pgfpathcurveto{\pgfqpoint{-0.007455in}{0.004330in}}{\pgfqpoint{-0.008333in}{0.002210in}}{\pgfqpoint{-0.008333in}{0.000000in}}%
\pgfpathcurveto{\pgfqpoint{-0.008333in}{-0.002210in}}{\pgfqpoint{-0.007455in}{-0.004330in}}{\pgfqpoint{-0.005893in}{-0.005893in}}%
\pgfpathcurveto{\pgfqpoint{-0.004330in}{-0.007455in}}{\pgfqpoint{-0.002210in}{-0.008333in}}{\pgfqpoint{0.000000in}{-0.008333in}}%
\pgfpathclose%
\pgfusepath{stroke,fill}%
}%
\begin{pgfscope}%
\pgfsys@transformshift{2.684559in}{4.684742in}%
\pgfsys@useobject{currentmarker}{}%
\end{pgfscope}%
\end{pgfscope}%
\begin{pgfscope}%
\pgfpathrectangle{\pgfqpoint{0.100000in}{2.413063in}}{\pgfqpoint{5.037500in}{3.427208in}}%
\pgfusepath{clip}%
\pgfsetrectcap%
\pgfsetroundjoin%
\pgfsetlinewidth{1.505625pt}%
\definecolor{currentstroke}{rgb}{0.501961,0.501961,0.501961}%
\pgfsetstrokecolor{currentstroke}%
\pgfsetstrokeopacity{0.500000}%
\pgfsetdash{}{0pt}%
\pgfpathmoveto{\pgfqpoint{3.030598in}{4.687321in}}%
\pgfusepath{stroke}%
\end{pgfscope}%
\begin{pgfscope}%
\pgfpathrectangle{\pgfqpoint{0.100000in}{2.413063in}}{\pgfqpoint{5.037500in}{3.427208in}}%
\pgfusepath{clip}%
\pgfsetbuttcap%
\pgfsetroundjoin%
\definecolor{currentfill}{rgb}{0.501961,0.501961,0.501961}%
\pgfsetfillcolor{currentfill}%
\pgfsetfillopacity{0.500000}%
\pgfsetlinewidth{0.250937pt}%
\definecolor{currentstroke}{rgb}{0.000000,0.000000,0.000000}%
\pgfsetstrokecolor{currentstroke}%
\pgfsetstrokeopacity{0.500000}%
\pgfsetdash{}{0pt}%
\pgfsys@defobject{currentmarker}{\pgfqpoint{-0.013889in}{-0.013889in}}{\pgfqpoint{0.013889in}{0.013889in}}{%
\pgfpathmoveto{\pgfqpoint{0.000000in}{-0.013889in}}%
\pgfpathcurveto{\pgfqpoint{0.003683in}{-0.013889in}}{\pgfqpoint{0.007216in}{-0.012425in}}{\pgfqpoint{0.009821in}{-0.009821in}}%
\pgfpathcurveto{\pgfqpoint{0.012425in}{-0.007216in}}{\pgfqpoint{0.013889in}{-0.003683in}}{\pgfqpoint{0.013889in}{0.000000in}}%
\pgfpathcurveto{\pgfqpoint{0.013889in}{0.003683in}}{\pgfqpoint{0.012425in}{0.007216in}}{\pgfqpoint{0.009821in}{0.009821in}}%
\pgfpathcurveto{\pgfqpoint{0.007216in}{0.012425in}}{\pgfqpoint{0.003683in}{0.013889in}}{\pgfqpoint{0.000000in}{0.013889in}}%
\pgfpathcurveto{\pgfqpoint{-0.003683in}{0.013889in}}{\pgfqpoint{-0.007216in}{0.012425in}}{\pgfqpoint{-0.009821in}{0.009821in}}%
\pgfpathcurveto{\pgfqpoint{-0.012425in}{0.007216in}}{\pgfqpoint{-0.013889in}{0.003683in}}{\pgfqpoint{-0.013889in}{0.000000in}}%
\pgfpathcurveto{\pgfqpoint{-0.013889in}{-0.003683in}}{\pgfqpoint{-0.012425in}{-0.007216in}}{\pgfqpoint{-0.009821in}{-0.009821in}}%
\pgfpathcurveto{\pgfqpoint{-0.007216in}{-0.012425in}}{\pgfqpoint{-0.003683in}{-0.013889in}}{\pgfqpoint{0.000000in}{-0.013889in}}%
\pgfpathclose%
\pgfusepath{stroke,fill}%
}%
\begin{pgfscope}%
\pgfsys@transformshift{3.030598in}{4.687321in}%
\pgfsys@useobject{currentmarker}{}%
\end{pgfscope}%
\end{pgfscope}%
\begin{pgfscope}%
\pgfpathrectangle{\pgfqpoint{0.100000in}{2.413063in}}{\pgfqpoint{5.037500in}{3.427208in}}%
\pgfusepath{clip}%
\pgfsetrectcap%
\pgfsetroundjoin%
\pgfsetlinewidth{1.505625pt}%
\definecolor{currentstroke}{rgb}{0.678431,1.000000,0.184314}%
\pgfsetstrokecolor{currentstroke}%
\pgfsetstrokeopacity{0.500000}%
\pgfsetdash{}{0pt}%
\pgfpathmoveto{\pgfqpoint{2.783006in}{4.277745in}}%
\pgfusepath{stroke}%
\end{pgfscope}%
\begin{pgfscope}%
\pgfpathrectangle{\pgfqpoint{0.100000in}{2.413063in}}{\pgfqpoint{5.037500in}{3.427208in}}%
\pgfusepath{clip}%
\pgfsetbuttcap%
\pgfsetroundjoin%
\definecolor{currentfill}{rgb}{0.678431,1.000000,0.184314}%
\pgfsetfillcolor{currentfill}%
\pgfsetfillopacity{0.500000}%
\pgfsetlinewidth{0.250937pt}%
\definecolor{currentstroke}{rgb}{0.000000,0.000000,0.000000}%
\pgfsetstrokecolor{currentstroke}%
\pgfsetstrokeopacity{0.500000}%
\pgfsetdash{}{0pt}%
\pgfsys@defobject{currentmarker}{\pgfqpoint{-0.011111in}{-0.011111in}}{\pgfqpoint{0.011111in}{0.011111in}}{%
\pgfpathmoveto{\pgfqpoint{0.000000in}{-0.011111in}}%
\pgfpathcurveto{\pgfqpoint{0.002947in}{-0.011111in}}{\pgfqpoint{0.005773in}{-0.009940in}}{\pgfqpoint{0.007857in}{-0.007857in}}%
\pgfpathcurveto{\pgfqpoint{0.009940in}{-0.005773in}}{\pgfqpoint{0.011111in}{-0.002947in}}{\pgfqpoint{0.011111in}{0.000000in}}%
\pgfpathcurveto{\pgfqpoint{0.011111in}{0.002947in}}{\pgfqpoint{0.009940in}{0.005773in}}{\pgfqpoint{0.007857in}{0.007857in}}%
\pgfpathcurveto{\pgfqpoint{0.005773in}{0.009940in}}{\pgfqpoint{0.002947in}{0.011111in}}{\pgfqpoint{0.000000in}{0.011111in}}%
\pgfpathcurveto{\pgfqpoint{-0.002947in}{0.011111in}}{\pgfqpoint{-0.005773in}{0.009940in}}{\pgfqpoint{-0.007857in}{0.007857in}}%
\pgfpathcurveto{\pgfqpoint{-0.009940in}{0.005773in}}{\pgfqpoint{-0.011111in}{0.002947in}}{\pgfqpoint{-0.011111in}{0.000000in}}%
\pgfpathcurveto{\pgfqpoint{-0.011111in}{-0.002947in}}{\pgfqpoint{-0.009940in}{-0.005773in}}{\pgfqpoint{-0.007857in}{-0.007857in}}%
\pgfpathcurveto{\pgfqpoint{-0.005773in}{-0.009940in}}{\pgfqpoint{-0.002947in}{-0.011111in}}{\pgfqpoint{0.000000in}{-0.011111in}}%
\pgfpathclose%
\pgfusepath{stroke,fill}%
}%
\begin{pgfscope}%
\pgfsys@transformshift{2.783006in}{4.277745in}%
\pgfsys@useobject{currentmarker}{}%
\end{pgfscope}%
\end{pgfscope}%
\begin{pgfscope}%
\pgfpathrectangle{\pgfqpoint{0.100000in}{2.413063in}}{\pgfqpoint{5.037500in}{3.427208in}}%
\pgfusepath{clip}%
\pgfsetrectcap%
\pgfsetroundjoin%
\pgfsetlinewidth{1.505625pt}%
\definecolor{currentstroke}{rgb}{0.678431,1.000000,0.184314}%
\pgfsetstrokecolor{currentstroke}%
\pgfsetstrokeopacity{0.500000}%
\pgfsetdash{}{0pt}%
\pgfpathmoveto{\pgfqpoint{2.663858in}{4.303304in}}%
\pgfusepath{stroke}%
\end{pgfscope}%
\begin{pgfscope}%
\pgfpathrectangle{\pgfqpoint{0.100000in}{2.413063in}}{\pgfqpoint{5.037500in}{3.427208in}}%
\pgfusepath{clip}%
\pgfsetbuttcap%
\pgfsetroundjoin%
\definecolor{currentfill}{rgb}{0.678431,1.000000,0.184314}%
\pgfsetfillcolor{currentfill}%
\pgfsetfillopacity{0.500000}%
\pgfsetlinewidth{0.250937pt}%
\definecolor{currentstroke}{rgb}{0.000000,0.000000,0.000000}%
\pgfsetstrokecolor{currentstroke}%
\pgfsetstrokeopacity{0.500000}%
\pgfsetdash{}{0pt}%
\pgfsys@defobject{currentmarker}{\pgfqpoint{-0.008333in}{-0.008333in}}{\pgfqpoint{0.008333in}{0.008333in}}{%
\pgfpathmoveto{\pgfqpoint{0.000000in}{-0.008333in}}%
\pgfpathcurveto{\pgfqpoint{0.002210in}{-0.008333in}}{\pgfqpoint{0.004330in}{-0.007455in}}{\pgfqpoint{0.005893in}{-0.005893in}}%
\pgfpathcurveto{\pgfqpoint{0.007455in}{-0.004330in}}{\pgfqpoint{0.008333in}{-0.002210in}}{\pgfqpoint{0.008333in}{0.000000in}}%
\pgfpathcurveto{\pgfqpoint{0.008333in}{0.002210in}}{\pgfqpoint{0.007455in}{0.004330in}}{\pgfqpoint{0.005893in}{0.005893in}}%
\pgfpathcurveto{\pgfqpoint{0.004330in}{0.007455in}}{\pgfqpoint{0.002210in}{0.008333in}}{\pgfqpoint{0.000000in}{0.008333in}}%
\pgfpathcurveto{\pgfqpoint{-0.002210in}{0.008333in}}{\pgfqpoint{-0.004330in}{0.007455in}}{\pgfqpoint{-0.005893in}{0.005893in}}%
\pgfpathcurveto{\pgfqpoint{-0.007455in}{0.004330in}}{\pgfqpoint{-0.008333in}{0.002210in}}{\pgfqpoint{-0.008333in}{0.000000in}}%
\pgfpathcurveto{\pgfqpoint{-0.008333in}{-0.002210in}}{\pgfqpoint{-0.007455in}{-0.004330in}}{\pgfqpoint{-0.005893in}{-0.005893in}}%
\pgfpathcurveto{\pgfqpoint{-0.004330in}{-0.007455in}}{\pgfqpoint{-0.002210in}{-0.008333in}}{\pgfqpoint{0.000000in}{-0.008333in}}%
\pgfpathclose%
\pgfusepath{stroke,fill}%
}%
\begin{pgfscope}%
\pgfsys@transformshift{2.663858in}{4.303304in}%
\pgfsys@useobject{currentmarker}{}%
\end{pgfscope}%
\end{pgfscope}%
\begin{pgfscope}%
\pgfpathrectangle{\pgfqpoint{0.100000in}{2.413063in}}{\pgfqpoint{5.037500in}{3.427208in}}%
\pgfusepath{clip}%
\pgfsetrectcap%
\pgfsetroundjoin%
\pgfsetlinewidth{1.505625pt}%
\definecolor{currentstroke}{rgb}{0.678431,1.000000,0.184314}%
\pgfsetstrokecolor{currentstroke}%
\pgfsetstrokeopacity{0.500000}%
\pgfsetdash{}{0pt}%
\pgfpathmoveto{\pgfqpoint{2.743539in}{4.286819in}}%
\pgfusepath{stroke}%
\end{pgfscope}%
\begin{pgfscope}%
\pgfpathrectangle{\pgfqpoint{0.100000in}{2.413063in}}{\pgfqpoint{5.037500in}{3.427208in}}%
\pgfusepath{clip}%
\pgfsetbuttcap%
\pgfsetroundjoin%
\definecolor{currentfill}{rgb}{0.678431,1.000000,0.184314}%
\pgfsetfillcolor{currentfill}%
\pgfsetfillopacity{0.500000}%
\pgfsetlinewidth{0.250937pt}%
\definecolor{currentstroke}{rgb}{0.000000,0.000000,0.000000}%
\pgfsetstrokecolor{currentstroke}%
\pgfsetstrokeopacity{0.500000}%
\pgfsetdash{}{0pt}%
\pgfsys@defobject{currentmarker}{\pgfqpoint{-0.019444in}{-0.019444in}}{\pgfqpoint{0.019444in}{0.019444in}}{%
\pgfpathmoveto{\pgfqpoint{0.000000in}{-0.019444in}}%
\pgfpathcurveto{\pgfqpoint{0.005157in}{-0.019444in}}{\pgfqpoint{0.010103in}{-0.017396in}}{\pgfqpoint{0.013749in}{-0.013749in}}%
\pgfpathcurveto{\pgfqpoint{0.017396in}{-0.010103in}}{\pgfqpoint{0.019444in}{-0.005157in}}{\pgfqpoint{0.019444in}{0.000000in}}%
\pgfpathcurveto{\pgfqpoint{0.019444in}{0.005157in}}{\pgfqpoint{0.017396in}{0.010103in}}{\pgfqpoint{0.013749in}{0.013749in}}%
\pgfpathcurveto{\pgfqpoint{0.010103in}{0.017396in}}{\pgfqpoint{0.005157in}{0.019444in}}{\pgfqpoint{0.000000in}{0.019444in}}%
\pgfpathcurveto{\pgfqpoint{-0.005157in}{0.019444in}}{\pgfqpoint{-0.010103in}{0.017396in}}{\pgfqpoint{-0.013749in}{0.013749in}}%
\pgfpathcurveto{\pgfqpoint{-0.017396in}{0.010103in}}{\pgfqpoint{-0.019444in}{0.005157in}}{\pgfqpoint{-0.019444in}{0.000000in}}%
\pgfpathcurveto{\pgfqpoint{-0.019444in}{-0.005157in}}{\pgfqpoint{-0.017396in}{-0.010103in}}{\pgfqpoint{-0.013749in}{-0.013749in}}%
\pgfpathcurveto{\pgfqpoint{-0.010103in}{-0.017396in}}{\pgfqpoint{-0.005157in}{-0.019444in}}{\pgfqpoint{0.000000in}{-0.019444in}}%
\pgfpathclose%
\pgfusepath{stroke,fill}%
}%
\begin{pgfscope}%
\pgfsys@transformshift{2.743539in}{4.286819in}%
\pgfsys@useobject{currentmarker}{}%
\end{pgfscope}%
\end{pgfscope}%
\begin{pgfscope}%
\pgfpathrectangle{\pgfqpoint{0.100000in}{2.413063in}}{\pgfqpoint{5.037500in}{3.427208in}}%
\pgfusepath{clip}%
\pgfsetrectcap%
\pgfsetroundjoin%
\pgfsetlinewidth{1.505625pt}%
\definecolor{currentstroke}{rgb}{0.501961,0.501961,0.501961}%
\pgfsetstrokecolor{currentstroke}%
\pgfsetstrokeopacity{0.500000}%
\pgfsetdash{}{0pt}%
\pgfpathmoveto{\pgfqpoint{2.591100in}{4.133088in}}%
\pgfusepath{stroke}%
\end{pgfscope}%
\begin{pgfscope}%
\pgfpathrectangle{\pgfqpoint{0.100000in}{2.413063in}}{\pgfqpoint{5.037500in}{3.427208in}}%
\pgfusepath{clip}%
\pgfsetbuttcap%
\pgfsetroundjoin%
\definecolor{currentfill}{rgb}{0.501961,0.501961,0.501961}%
\pgfsetfillcolor{currentfill}%
\pgfsetfillopacity{0.500000}%
\pgfsetlinewidth{0.250937pt}%
\definecolor{currentstroke}{rgb}{0.000000,0.000000,0.000000}%
\pgfsetstrokecolor{currentstroke}%
\pgfsetstrokeopacity{0.500000}%
\pgfsetdash{}{0pt}%
\pgfsys@defobject{currentmarker}{\pgfqpoint{-0.013889in}{-0.013889in}}{\pgfqpoint{0.013889in}{0.013889in}}{%
\pgfpathmoveto{\pgfqpoint{0.000000in}{-0.013889in}}%
\pgfpathcurveto{\pgfqpoint{0.003683in}{-0.013889in}}{\pgfqpoint{0.007216in}{-0.012425in}}{\pgfqpoint{0.009821in}{-0.009821in}}%
\pgfpathcurveto{\pgfqpoint{0.012425in}{-0.007216in}}{\pgfqpoint{0.013889in}{-0.003683in}}{\pgfqpoint{0.013889in}{0.000000in}}%
\pgfpathcurveto{\pgfqpoint{0.013889in}{0.003683in}}{\pgfqpoint{0.012425in}{0.007216in}}{\pgfqpoint{0.009821in}{0.009821in}}%
\pgfpathcurveto{\pgfqpoint{0.007216in}{0.012425in}}{\pgfqpoint{0.003683in}{0.013889in}}{\pgfqpoint{0.000000in}{0.013889in}}%
\pgfpathcurveto{\pgfqpoint{-0.003683in}{0.013889in}}{\pgfqpoint{-0.007216in}{0.012425in}}{\pgfqpoint{-0.009821in}{0.009821in}}%
\pgfpathcurveto{\pgfqpoint{-0.012425in}{0.007216in}}{\pgfqpoint{-0.013889in}{0.003683in}}{\pgfqpoint{-0.013889in}{0.000000in}}%
\pgfpathcurveto{\pgfqpoint{-0.013889in}{-0.003683in}}{\pgfqpoint{-0.012425in}{-0.007216in}}{\pgfqpoint{-0.009821in}{-0.009821in}}%
\pgfpathcurveto{\pgfqpoint{-0.007216in}{-0.012425in}}{\pgfqpoint{-0.003683in}{-0.013889in}}{\pgfqpoint{0.000000in}{-0.013889in}}%
\pgfpathclose%
\pgfusepath{stroke,fill}%
}%
\begin{pgfscope}%
\pgfsys@transformshift{2.591100in}{4.133088in}%
\pgfsys@useobject{currentmarker}{}%
\end{pgfscope}%
\end{pgfscope}%
\begin{pgfscope}%
\pgfpathrectangle{\pgfqpoint{0.100000in}{2.413063in}}{\pgfqpoint{5.037500in}{3.427208in}}%
\pgfusepath{clip}%
\pgfsetrectcap%
\pgfsetroundjoin%
\pgfsetlinewidth{1.505625pt}%
\definecolor{currentstroke}{rgb}{0.678431,1.000000,0.184314}%
\pgfsetstrokecolor{currentstroke}%
\pgfsetstrokeopacity{0.500000}%
\pgfsetdash{}{0pt}%
\pgfpathmoveto{\pgfqpoint{3.590438in}{4.086531in}}%
\pgfusepath{stroke}%
\end{pgfscope}%
\begin{pgfscope}%
\pgfpathrectangle{\pgfqpoint{0.100000in}{2.413063in}}{\pgfqpoint{5.037500in}{3.427208in}}%
\pgfusepath{clip}%
\pgfsetbuttcap%
\pgfsetroundjoin%
\definecolor{currentfill}{rgb}{0.678431,1.000000,0.184314}%
\pgfsetfillcolor{currentfill}%
\pgfsetfillopacity{0.500000}%
\pgfsetlinewidth{0.250937pt}%
\definecolor{currentstroke}{rgb}{0.000000,0.000000,0.000000}%
\pgfsetstrokecolor{currentstroke}%
\pgfsetstrokeopacity{0.500000}%
\pgfsetdash{}{0pt}%
\pgfsys@defobject{currentmarker}{\pgfqpoint{-0.025000in}{-0.025000in}}{\pgfqpoint{0.025000in}{0.025000in}}{%
\pgfpathmoveto{\pgfqpoint{0.000000in}{-0.025000in}}%
\pgfpathcurveto{\pgfqpoint{0.006630in}{-0.025000in}}{\pgfqpoint{0.012989in}{-0.022366in}}{\pgfqpoint{0.017678in}{-0.017678in}}%
\pgfpathcurveto{\pgfqpoint{0.022366in}{-0.012989in}}{\pgfqpoint{0.025000in}{-0.006630in}}{\pgfqpoint{0.025000in}{0.000000in}}%
\pgfpathcurveto{\pgfqpoint{0.025000in}{0.006630in}}{\pgfqpoint{0.022366in}{0.012989in}}{\pgfqpoint{0.017678in}{0.017678in}}%
\pgfpathcurveto{\pgfqpoint{0.012989in}{0.022366in}}{\pgfqpoint{0.006630in}{0.025000in}}{\pgfqpoint{0.000000in}{0.025000in}}%
\pgfpathcurveto{\pgfqpoint{-0.006630in}{0.025000in}}{\pgfqpoint{-0.012989in}{0.022366in}}{\pgfqpoint{-0.017678in}{0.017678in}}%
\pgfpathcurveto{\pgfqpoint{-0.022366in}{0.012989in}}{\pgfqpoint{-0.025000in}{0.006630in}}{\pgfqpoint{-0.025000in}{0.000000in}}%
\pgfpathcurveto{\pgfqpoint{-0.025000in}{-0.006630in}}{\pgfqpoint{-0.022366in}{-0.012989in}}{\pgfqpoint{-0.017678in}{-0.017678in}}%
\pgfpathcurveto{\pgfqpoint{-0.012989in}{-0.022366in}}{\pgfqpoint{-0.006630in}{-0.025000in}}{\pgfqpoint{0.000000in}{-0.025000in}}%
\pgfpathclose%
\pgfusepath{stroke,fill}%
}%
\begin{pgfscope}%
\pgfsys@transformshift{3.590438in}{4.086531in}%
\pgfsys@useobject{currentmarker}{}%
\end{pgfscope}%
\end{pgfscope}%
\begin{pgfscope}%
\pgfpathrectangle{\pgfqpoint{0.100000in}{2.413063in}}{\pgfqpoint{5.037500in}{3.427208in}}%
\pgfusepath{clip}%
\pgfsetrectcap%
\pgfsetroundjoin%
\pgfsetlinewidth{1.505625pt}%
\definecolor{currentstroke}{rgb}{0.678431,1.000000,0.184314}%
\pgfsetstrokecolor{currentstroke}%
\pgfsetstrokeopacity{0.500000}%
\pgfsetdash{}{0pt}%
\pgfpathmoveto{\pgfqpoint{3.635835in}{4.172410in}}%
\pgfusepath{stroke}%
\end{pgfscope}%
\begin{pgfscope}%
\pgfpathrectangle{\pgfqpoint{0.100000in}{2.413063in}}{\pgfqpoint{5.037500in}{3.427208in}}%
\pgfusepath{clip}%
\pgfsetbuttcap%
\pgfsetroundjoin%
\definecolor{currentfill}{rgb}{0.678431,1.000000,0.184314}%
\pgfsetfillcolor{currentfill}%
\pgfsetfillopacity{0.500000}%
\pgfsetlinewidth{0.250937pt}%
\definecolor{currentstroke}{rgb}{0.000000,0.000000,0.000000}%
\pgfsetstrokecolor{currentstroke}%
\pgfsetstrokeopacity{0.500000}%
\pgfsetdash{}{0pt}%
\pgfsys@defobject{currentmarker}{\pgfqpoint{-0.005556in}{-0.005556in}}{\pgfqpoint{0.005556in}{0.005556in}}{%
\pgfpathmoveto{\pgfqpoint{0.000000in}{-0.005556in}}%
\pgfpathcurveto{\pgfqpoint{0.001473in}{-0.005556in}}{\pgfqpoint{0.002887in}{-0.004970in}}{\pgfqpoint{0.003928in}{-0.003928in}}%
\pgfpathcurveto{\pgfqpoint{0.004970in}{-0.002887in}}{\pgfqpoint{0.005556in}{-0.001473in}}{\pgfqpoint{0.005556in}{0.000000in}}%
\pgfpathcurveto{\pgfqpoint{0.005556in}{0.001473in}}{\pgfqpoint{0.004970in}{0.002887in}}{\pgfqpoint{0.003928in}{0.003928in}}%
\pgfpathcurveto{\pgfqpoint{0.002887in}{0.004970in}}{\pgfqpoint{0.001473in}{0.005556in}}{\pgfqpoint{0.000000in}{0.005556in}}%
\pgfpathcurveto{\pgfqpoint{-0.001473in}{0.005556in}}{\pgfqpoint{-0.002887in}{0.004970in}}{\pgfqpoint{-0.003928in}{0.003928in}}%
\pgfpathcurveto{\pgfqpoint{-0.004970in}{0.002887in}}{\pgfqpoint{-0.005556in}{0.001473in}}{\pgfqpoint{-0.005556in}{0.000000in}}%
\pgfpathcurveto{\pgfqpoint{-0.005556in}{-0.001473in}}{\pgfqpoint{-0.004970in}{-0.002887in}}{\pgfqpoint{-0.003928in}{-0.003928in}}%
\pgfpathcurveto{\pgfqpoint{-0.002887in}{-0.004970in}}{\pgfqpoint{-0.001473in}{-0.005556in}}{\pgfqpoint{0.000000in}{-0.005556in}}%
\pgfpathclose%
\pgfusepath{stroke,fill}%
}%
\begin{pgfscope}%
\pgfsys@transformshift{3.635835in}{4.172410in}%
\pgfsys@useobject{currentmarker}{}%
\end{pgfscope}%
\end{pgfscope}%
\begin{pgfscope}%
\pgfpathrectangle{\pgfqpoint{0.100000in}{2.413063in}}{\pgfqpoint{5.037500in}{3.427208in}}%
\pgfusepath{clip}%
\pgfsetrectcap%
\pgfsetroundjoin%
\pgfsetlinewidth{1.505625pt}%
\definecolor{currentstroke}{rgb}{0.501961,0.501961,0.501961}%
\pgfsetstrokecolor{currentstroke}%
\pgfsetstrokeopacity{0.500000}%
\pgfsetdash{}{0pt}%
\pgfpathmoveto{\pgfqpoint{3.754570in}{4.226123in}}%
\pgfusepath{stroke}%
\end{pgfscope}%
\begin{pgfscope}%
\pgfpathrectangle{\pgfqpoint{0.100000in}{2.413063in}}{\pgfqpoint{5.037500in}{3.427208in}}%
\pgfusepath{clip}%
\pgfsetbuttcap%
\pgfsetroundjoin%
\definecolor{currentfill}{rgb}{0.501961,0.501961,0.501961}%
\pgfsetfillcolor{currentfill}%
\pgfsetfillopacity{0.500000}%
\pgfsetlinewidth{0.250937pt}%
\definecolor{currentstroke}{rgb}{0.000000,0.000000,0.000000}%
\pgfsetstrokecolor{currentstroke}%
\pgfsetstrokeopacity{0.500000}%
\pgfsetdash{}{0pt}%
\pgfsys@defobject{currentmarker}{\pgfqpoint{-0.013889in}{-0.013889in}}{\pgfqpoint{0.013889in}{0.013889in}}{%
\pgfpathmoveto{\pgfqpoint{0.000000in}{-0.013889in}}%
\pgfpathcurveto{\pgfqpoint{0.003683in}{-0.013889in}}{\pgfqpoint{0.007216in}{-0.012425in}}{\pgfqpoint{0.009821in}{-0.009821in}}%
\pgfpathcurveto{\pgfqpoint{0.012425in}{-0.007216in}}{\pgfqpoint{0.013889in}{-0.003683in}}{\pgfqpoint{0.013889in}{0.000000in}}%
\pgfpathcurveto{\pgfqpoint{0.013889in}{0.003683in}}{\pgfqpoint{0.012425in}{0.007216in}}{\pgfqpoint{0.009821in}{0.009821in}}%
\pgfpathcurveto{\pgfqpoint{0.007216in}{0.012425in}}{\pgfqpoint{0.003683in}{0.013889in}}{\pgfqpoint{0.000000in}{0.013889in}}%
\pgfpathcurveto{\pgfqpoint{-0.003683in}{0.013889in}}{\pgfqpoint{-0.007216in}{0.012425in}}{\pgfqpoint{-0.009821in}{0.009821in}}%
\pgfpathcurveto{\pgfqpoint{-0.012425in}{0.007216in}}{\pgfqpoint{-0.013889in}{0.003683in}}{\pgfqpoint{-0.013889in}{0.000000in}}%
\pgfpathcurveto{\pgfqpoint{-0.013889in}{-0.003683in}}{\pgfqpoint{-0.012425in}{-0.007216in}}{\pgfqpoint{-0.009821in}{-0.009821in}}%
\pgfpathcurveto{\pgfqpoint{-0.007216in}{-0.012425in}}{\pgfqpoint{-0.003683in}{-0.013889in}}{\pgfqpoint{0.000000in}{-0.013889in}}%
\pgfpathclose%
\pgfusepath{stroke,fill}%
}%
\begin{pgfscope}%
\pgfsys@transformshift{3.754570in}{4.226123in}%
\pgfsys@useobject{currentmarker}{}%
\end{pgfscope}%
\end{pgfscope}%
\begin{pgfscope}%
\pgfpathrectangle{\pgfqpoint{0.100000in}{2.413063in}}{\pgfqpoint{5.037500in}{3.427208in}}%
\pgfusepath{clip}%
\pgfsetrectcap%
\pgfsetroundjoin%
\pgfsetlinewidth{1.505625pt}%
\definecolor{currentstroke}{rgb}{0.678431,1.000000,0.184314}%
\pgfsetstrokecolor{currentstroke}%
\pgfsetstrokeopacity{0.500000}%
\pgfsetdash{}{0pt}%
\pgfpathmoveto{\pgfqpoint{3.638327in}{4.237522in}}%
\pgfusepath{stroke}%
\end{pgfscope}%
\begin{pgfscope}%
\pgfpathrectangle{\pgfqpoint{0.100000in}{2.413063in}}{\pgfqpoint{5.037500in}{3.427208in}}%
\pgfusepath{clip}%
\pgfsetbuttcap%
\pgfsetroundjoin%
\definecolor{currentfill}{rgb}{0.678431,1.000000,0.184314}%
\pgfsetfillcolor{currentfill}%
\pgfsetfillopacity{0.500000}%
\pgfsetlinewidth{0.250937pt}%
\definecolor{currentstroke}{rgb}{0.000000,0.000000,0.000000}%
\pgfsetstrokecolor{currentstroke}%
\pgfsetstrokeopacity{0.500000}%
\pgfsetdash{}{0pt}%
\pgfsys@defobject{currentmarker}{\pgfqpoint{-0.005556in}{-0.005556in}}{\pgfqpoint{0.005556in}{0.005556in}}{%
\pgfpathmoveto{\pgfqpoint{0.000000in}{-0.005556in}}%
\pgfpathcurveto{\pgfqpoint{0.001473in}{-0.005556in}}{\pgfqpoint{0.002887in}{-0.004970in}}{\pgfqpoint{0.003928in}{-0.003928in}}%
\pgfpathcurveto{\pgfqpoint{0.004970in}{-0.002887in}}{\pgfqpoint{0.005556in}{-0.001473in}}{\pgfqpoint{0.005556in}{0.000000in}}%
\pgfpathcurveto{\pgfqpoint{0.005556in}{0.001473in}}{\pgfqpoint{0.004970in}{0.002887in}}{\pgfqpoint{0.003928in}{0.003928in}}%
\pgfpathcurveto{\pgfqpoint{0.002887in}{0.004970in}}{\pgfqpoint{0.001473in}{0.005556in}}{\pgfqpoint{0.000000in}{0.005556in}}%
\pgfpathcurveto{\pgfqpoint{-0.001473in}{0.005556in}}{\pgfqpoint{-0.002887in}{0.004970in}}{\pgfqpoint{-0.003928in}{0.003928in}}%
\pgfpathcurveto{\pgfqpoint{-0.004970in}{0.002887in}}{\pgfqpoint{-0.005556in}{0.001473in}}{\pgfqpoint{-0.005556in}{0.000000in}}%
\pgfpathcurveto{\pgfqpoint{-0.005556in}{-0.001473in}}{\pgfqpoint{-0.004970in}{-0.002887in}}{\pgfqpoint{-0.003928in}{-0.003928in}}%
\pgfpathcurveto{\pgfqpoint{-0.002887in}{-0.004970in}}{\pgfqpoint{-0.001473in}{-0.005556in}}{\pgfqpoint{0.000000in}{-0.005556in}}%
\pgfpathclose%
\pgfusepath{stroke,fill}%
}%
\begin{pgfscope}%
\pgfsys@transformshift{3.638327in}{4.237522in}%
\pgfsys@useobject{currentmarker}{}%
\end{pgfscope}%
\end{pgfscope}%
\begin{pgfscope}%
\pgfpathrectangle{\pgfqpoint{0.100000in}{2.413063in}}{\pgfqpoint{5.037500in}{3.427208in}}%
\pgfusepath{clip}%
\pgfsetrectcap%
\pgfsetroundjoin%
\pgfsetlinewidth{1.505625pt}%
\definecolor{currentstroke}{rgb}{0.678431,1.000000,0.184314}%
\pgfsetstrokecolor{currentstroke}%
\pgfsetstrokeopacity{0.500000}%
\pgfsetdash{}{0pt}%
\pgfpathmoveto{\pgfqpoint{3.521226in}{4.170926in}}%
\pgfusepath{stroke}%
\end{pgfscope}%
\begin{pgfscope}%
\pgfpathrectangle{\pgfqpoint{0.100000in}{2.413063in}}{\pgfqpoint{5.037500in}{3.427208in}}%
\pgfusepath{clip}%
\pgfsetbuttcap%
\pgfsetroundjoin%
\definecolor{currentfill}{rgb}{0.678431,1.000000,0.184314}%
\pgfsetfillcolor{currentfill}%
\pgfsetfillopacity{0.500000}%
\pgfsetlinewidth{0.250937pt}%
\definecolor{currentstroke}{rgb}{0.000000,0.000000,0.000000}%
\pgfsetstrokecolor{currentstroke}%
\pgfsetstrokeopacity{0.500000}%
\pgfsetdash{}{0pt}%
\pgfsys@defobject{currentmarker}{\pgfqpoint{-0.011111in}{-0.011111in}}{\pgfqpoint{0.011111in}{0.011111in}}{%
\pgfpathmoveto{\pgfqpoint{0.000000in}{-0.011111in}}%
\pgfpathcurveto{\pgfqpoint{0.002947in}{-0.011111in}}{\pgfqpoint{0.005773in}{-0.009940in}}{\pgfqpoint{0.007857in}{-0.007857in}}%
\pgfpathcurveto{\pgfqpoint{0.009940in}{-0.005773in}}{\pgfqpoint{0.011111in}{-0.002947in}}{\pgfqpoint{0.011111in}{0.000000in}}%
\pgfpathcurveto{\pgfqpoint{0.011111in}{0.002947in}}{\pgfqpoint{0.009940in}{0.005773in}}{\pgfqpoint{0.007857in}{0.007857in}}%
\pgfpathcurveto{\pgfqpoint{0.005773in}{0.009940in}}{\pgfqpoint{0.002947in}{0.011111in}}{\pgfqpoint{0.000000in}{0.011111in}}%
\pgfpathcurveto{\pgfqpoint{-0.002947in}{0.011111in}}{\pgfqpoint{-0.005773in}{0.009940in}}{\pgfqpoint{-0.007857in}{0.007857in}}%
\pgfpathcurveto{\pgfqpoint{-0.009940in}{0.005773in}}{\pgfqpoint{-0.011111in}{0.002947in}}{\pgfqpoint{-0.011111in}{0.000000in}}%
\pgfpathcurveto{\pgfqpoint{-0.011111in}{-0.002947in}}{\pgfqpoint{-0.009940in}{-0.005773in}}{\pgfqpoint{-0.007857in}{-0.007857in}}%
\pgfpathcurveto{\pgfqpoint{-0.005773in}{-0.009940in}}{\pgfqpoint{-0.002947in}{-0.011111in}}{\pgfqpoint{0.000000in}{-0.011111in}}%
\pgfpathclose%
\pgfusepath{stroke,fill}%
}%
\begin{pgfscope}%
\pgfsys@transformshift{3.521226in}{4.170926in}%
\pgfsys@useobject{currentmarker}{}%
\end{pgfscope}%
\end{pgfscope}%
\begin{pgfscope}%
\pgfpathrectangle{\pgfqpoint{0.100000in}{2.413063in}}{\pgfqpoint{5.037500in}{3.427208in}}%
\pgfusepath{clip}%
\pgfsetrectcap%
\pgfsetroundjoin%
\pgfsetlinewidth{1.505625pt}%
\definecolor{currentstroke}{rgb}{0.678431,1.000000,0.184314}%
\pgfsetstrokecolor{currentstroke}%
\pgfsetstrokeopacity{0.500000}%
\pgfsetdash{}{0pt}%
\pgfpathmoveto{\pgfqpoint{3.058052in}{3.383452in}}%
\pgfusepath{stroke}%
\end{pgfscope}%
\begin{pgfscope}%
\pgfpathrectangle{\pgfqpoint{0.100000in}{2.413063in}}{\pgfqpoint{5.037500in}{3.427208in}}%
\pgfusepath{clip}%
\pgfsetbuttcap%
\pgfsetroundjoin%
\definecolor{currentfill}{rgb}{0.678431,1.000000,0.184314}%
\pgfsetfillcolor{currentfill}%
\pgfsetfillopacity{0.500000}%
\pgfsetlinewidth{0.250937pt}%
\definecolor{currentstroke}{rgb}{0.000000,0.000000,0.000000}%
\pgfsetstrokecolor{currentstroke}%
\pgfsetstrokeopacity{0.500000}%
\pgfsetdash{}{0pt}%
\pgfsys@defobject{currentmarker}{\pgfqpoint{-0.058333in}{-0.058333in}}{\pgfqpoint{0.058333in}{0.058333in}}{%
\pgfpathmoveto{\pgfqpoint{0.000000in}{-0.058333in}}%
\pgfpathcurveto{\pgfqpoint{0.015470in}{-0.058333in}}{\pgfqpoint{0.030309in}{-0.052187in}}{\pgfqpoint{0.041248in}{-0.041248in}}%
\pgfpathcurveto{\pgfqpoint{0.052187in}{-0.030309in}}{\pgfqpoint{0.058333in}{-0.015470in}}{\pgfqpoint{0.058333in}{0.000000in}}%
\pgfpathcurveto{\pgfqpoint{0.058333in}{0.015470in}}{\pgfqpoint{0.052187in}{0.030309in}}{\pgfqpoint{0.041248in}{0.041248in}}%
\pgfpathcurveto{\pgfqpoint{0.030309in}{0.052187in}}{\pgfqpoint{0.015470in}{0.058333in}}{\pgfqpoint{0.000000in}{0.058333in}}%
\pgfpathcurveto{\pgfqpoint{-0.015470in}{0.058333in}}{\pgfqpoint{-0.030309in}{0.052187in}}{\pgfqpoint{-0.041248in}{0.041248in}}%
\pgfpathcurveto{\pgfqpoint{-0.052187in}{0.030309in}}{\pgfqpoint{-0.058333in}{0.015470in}}{\pgfqpoint{-0.058333in}{0.000000in}}%
\pgfpathcurveto{\pgfqpoint{-0.058333in}{-0.015470in}}{\pgfqpoint{-0.052187in}{-0.030309in}}{\pgfqpoint{-0.041248in}{-0.041248in}}%
\pgfpathcurveto{\pgfqpoint{-0.030309in}{-0.052187in}}{\pgfqpoint{-0.015470in}{-0.058333in}}{\pgfqpoint{0.000000in}{-0.058333in}}%
\pgfpathclose%
\pgfusepath{stroke,fill}%
}%
\begin{pgfscope}%
\pgfsys@transformshift{3.058052in}{3.383452in}%
\pgfsys@useobject{currentmarker}{}%
\end{pgfscope}%
\end{pgfscope}%
\begin{pgfscope}%
\pgfpathrectangle{\pgfqpoint{0.100000in}{2.413063in}}{\pgfqpoint{5.037500in}{3.427208in}}%
\pgfusepath{clip}%
\pgfsetrectcap%
\pgfsetroundjoin%
\pgfsetlinewidth{1.505625pt}%
\definecolor{currentstroke}{rgb}{0.678431,1.000000,0.184314}%
\pgfsetstrokecolor{currentstroke}%
\pgfsetstrokeopacity{0.500000}%
\pgfsetdash{}{0pt}%
\pgfpathmoveto{\pgfqpoint{3.189939in}{3.301015in}}%
\pgfusepath{stroke}%
\end{pgfscope}%
\begin{pgfscope}%
\pgfpathrectangle{\pgfqpoint{0.100000in}{2.413063in}}{\pgfqpoint{5.037500in}{3.427208in}}%
\pgfusepath{clip}%
\pgfsetbuttcap%
\pgfsetroundjoin%
\definecolor{currentfill}{rgb}{0.678431,1.000000,0.184314}%
\pgfsetfillcolor{currentfill}%
\pgfsetfillopacity{0.500000}%
\pgfsetlinewidth{0.250937pt}%
\definecolor{currentstroke}{rgb}{0.000000,0.000000,0.000000}%
\pgfsetstrokecolor{currentstroke}%
\pgfsetstrokeopacity{0.500000}%
\pgfsetdash{}{0pt}%
\pgfsys@defobject{currentmarker}{\pgfqpoint{-0.025000in}{-0.025000in}}{\pgfqpoint{0.025000in}{0.025000in}}{%
\pgfpathmoveto{\pgfqpoint{0.000000in}{-0.025000in}}%
\pgfpathcurveto{\pgfqpoint{0.006630in}{-0.025000in}}{\pgfqpoint{0.012989in}{-0.022366in}}{\pgfqpoint{0.017678in}{-0.017678in}}%
\pgfpathcurveto{\pgfqpoint{0.022366in}{-0.012989in}}{\pgfqpoint{0.025000in}{-0.006630in}}{\pgfqpoint{0.025000in}{0.000000in}}%
\pgfpathcurveto{\pgfqpoint{0.025000in}{0.006630in}}{\pgfqpoint{0.022366in}{0.012989in}}{\pgfqpoint{0.017678in}{0.017678in}}%
\pgfpathcurveto{\pgfqpoint{0.012989in}{0.022366in}}{\pgfqpoint{0.006630in}{0.025000in}}{\pgfqpoint{0.000000in}{0.025000in}}%
\pgfpathcurveto{\pgfqpoint{-0.006630in}{0.025000in}}{\pgfqpoint{-0.012989in}{0.022366in}}{\pgfqpoint{-0.017678in}{0.017678in}}%
\pgfpathcurveto{\pgfqpoint{-0.022366in}{0.012989in}}{\pgfqpoint{-0.025000in}{0.006630in}}{\pgfqpoint{-0.025000in}{0.000000in}}%
\pgfpathcurveto{\pgfqpoint{-0.025000in}{-0.006630in}}{\pgfqpoint{-0.022366in}{-0.012989in}}{\pgfqpoint{-0.017678in}{-0.017678in}}%
\pgfpathcurveto{\pgfqpoint{-0.012989in}{-0.022366in}}{\pgfqpoint{-0.006630in}{-0.025000in}}{\pgfqpoint{0.000000in}{-0.025000in}}%
\pgfpathclose%
\pgfusepath{stroke,fill}%
}%
\begin{pgfscope}%
\pgfsys@transformshift{3.189939in}{3.301015in}%
\pgfsys@useobject{currentmarker}{}%
\end{pgfscope}%
\end{pgfscope}%
\begin{pgfscope}%
\pgfpathrectangle{\pgfqpoint{0.100000in}{2.413063in}}{\pgfqpoint{5.037500in}{3.427208in}}%
\pgfusepath{clip}%
\pgfsetrectcap%
\pgfsetroundjoin%
\pgfsetlinewidth{1.505625pt}%
\definecolor{currentstroke}{rgb}{0.678431,1.000000,0.184314}%
\pgfsetstrokecolor{currentstroke}%
\pgfsetstrokeopacity{0.500000}%
\pgfsetdash{}{0pt}%
\pgfpathmoveto{\pgfqpoint{3.449980in}{4.605174in}}%
\pgfusepath{stroke}%
\end{pgfscope}%
\begin{pgfscope}%
\pgfpathrectangle{\pgfqpoint{0.100000in}{2.413063in}}{\pgfqpoint{5.037500in}{3.427208in}}%
\pgfusepath{clip}%
\pgfsetbuttcap%
\pgfsetroundjoin%
\definecolor{currentfill}{rgb}{0.678431,1.000000,0.184314}%
\pgfsetfillcolor{currentfill}%
\pgfsetfillopacity{0.500000}%
\pgfsetlinewidth{0.250937pt}%
\definecolor{currentstroke}{rgb}{0.000000,0.000000,0.000000}%
\pgfsetstrokecolor{currentstroke}%
\pgfsetstrokeopacity{0.500000}%
\pgfsetdash{}{0pt}%
\pgfsys@defobject{currentmarker}{\pgfqpoint{-0.016667in}{-0.016667in}}{\pgfqpoint{0.016667in}{0.016667in}}{%
\pgfpathmoveto{\pgfqpoint{0.000000in}{-0.016667in}}%
\pgfpathcurveto{\pgfqpoint{0.004420in}{-0.016667in}}{\pgfqpoint{0.008660in}{-0.014911in}}{\pgfqpoint{0.011785in}{-0.011785in}}%
\pgfpathcurveto{\pgfqpoint{0.014911in}{-0.008660in}}{\pgfqpoint{0.016667in}{-0.004420in}}{\pgfqpoint{0.016667in}{0.000000in}}%
\pgfpathcurveto{\pgfqpoint{0.016667in}{0.004420in}}{\pgfqpoint{0.014911in}{0.008660in}}{\pgfqpoint{0.011785in}{0.011785in}}%
\pgfpathcurveto{\pgfqpoint{0.008660in}{0.014911in}}{\pgfqpoint{0.004420in}{0.016667in}}{\pgfqpoint{0.000000in}{0.016667in}}%
\pgfpathcurveto{\pgfqpoint{-0.004420in}{0.016667in}}{\pgfqpoint{-0.008660in}{0.014911in}}{\pgfqpoint{-0.011785in}{0.011785in}}%
\pgfpathcurveto{\pgfqpoint{-0.014911in}{0.008660in}}{\pgfqpoint{-0.016667in}{0.004420in}}{\pgfqpoint{-0.016667in}{0.000000in}}%
\pgfpathcurveto{\pgfqpoint{-0.016667in}{-0.004420in}}{\pgfqpoint{-0.014911in}{-0.008660in}}{\pgfqpoint{-0.011785in}{-0.011785in}}%
\pgfpathcurveto{\pgfqpoint{-0.008660in}{-0.014911in}}{\pgfqpoint{-0.004420in}{-0.016667in}}{\pgfqpoint{0.000000in}{-0.016667in}}%
\pgfpathclose%
\pgfusepath{stroke,fill}%
}%
\begin{pgfscope}%
\pgfsys@transformshift{3.449980in}{4.605174in}%
\pgfsys@useobject{currentmarker}{}%
\end{pgfscope}%
\end{pgfscope}%
\begin{pgfscope}%
\pgfpathrectangle{\pgfqpoint{0.100000in}{2.413063in}}{\pgfqpoint{5.037500in}{3.427208in}}%
\pgfusepath{clip}%
\pgfsetrectcap%
\pgfsetroundjoin%
\pgfsetlinewidth{1.505625pt}%
\definecolor{currentstroke}{rgb}{0.678431,1.000000,0.184314}%
\pgfsetstrokecolor{currentstroke}%
\pgfsetstrokeopacity{0.500000}%
\pgfsetdash{}{0pt}%
\pgfpathmoveto{\pgfqpoint{3.238245in}{3.203433in}}%
\pgfusepath{stroke}%
\end{pgfscope}%
\begin{pgfscope}%
\pgfpathrectangle{\pgfqpoint{0.100000in}{2.413063in}}{\pgfqpoint{5.037500in}{3.427208in}}%
\pgfusepath{clip}%
\pgfsetbuttcap%
\pgfsetroundjoin%
\definecolor{currentfill}{rgb}{0.678431,1.000000,0.184314}%
\pgfsetfillcolor{currentfill}%
\pgfsetfillopacity{0.500000}%
\pgfsetlinewidth{0.250937pt}%
\definecolor{currentstroke}{rgb}{0.000000,0.000000,0.000000}%
\pgfsetstrokecolor{currentstroke}%
\pgfsetstrokeopacity{0.500000}%
\pgfsetdash{}{0pt}%
\pgfsys@defobject{currentmarker}{\pgfqpoint{-0.013889in}{-0.013889in}}{\pgfqpoint{0.013889in}{0.013889in}}{%
\pgfpathmoveto{\pgfqpoint{0.000000in}{-0.013889in}}%
\pgfpathcurveto{\pgfqpoint{0.003683in}{-0.013889in}}{\pgfqpoint{0.007216in}{-0.012425in}}{\pgfqpoint{0.009821in}{-0.009821in}}%
\pgfpathcurveto{\pgfqpoint{0.012425in}{-0.007216in}}{\pgfqpoint{0.013889in}{-0.003683in}}{\pgfqpoint{0.013889in}{0.000000in}}%
\pgfpathcurveto{\pgfqpoint{0.013889in}{0.003683in}}{\pgfqpoint{0.012425in}{0.007216in}}{\pgfqpoint{0.009821in}{0.009821in}}%
\pgfpathcurveto{\pgfqpoint{0.007216in}{0.012425in}}{\pgfqpoint{0.003683in}{0.013889in}}{\pgfqpoint{0.000000in}{0.013889in}}%
\pgfpathcurveto{\pgfqpoint{-0.003683in}{0.013889in}}{\pgfqpoint{-0.007216in}{0.012425in}}{\pgfqpoint{-0.009821in}{0.009821in}}%
\pgfpathcurveto{\pgfqpoint{-0.012425in}{0.007216in}}{\pgfqpoint{-0.013889in}{0.003683in}}{\pgfqpoint{-0.013889in}{0.000000in}}%
\pgfpathcurveto{\pgfqpoint{-0.013889in}{-0.003683in}}{\pgfqpoint{-0.012425in}{-0.007216in}}{\pgfqpoint{-0.009821in}{-0.009821in}}%
\pgfpathcurveto{\pgfqpoint{-0.007216in}{-0.012425in}}{\pgfqpoint{-0.003683in}{-0.013889in}}{\pgfqpoint{0.000000in}{-0.013889in}}%
\pgfpathclose%
\pgfusepath{stroke,fill}%
}%
\begin{pgfscope}%
\pgfsys@transformshift{3.238245in}{3.203433in}%
\pgfsys@useobject{currentmarker}{}%
\end{pgfscope}%
\end{pgfscope}%
\begin{pgfscope}%
\pgfpathrectangle{\pgfqpoint{0.100000in}{2.413063in}}{\pgfqpoint{5.037500in}{3.427208in}}%
\pgfusepath{clip}%
\pgfsetrectcap%
\pgfsetroundjoin%
\pgfsetlinewidth{1.505625pt}%
\definecolor{currentstroke}{rgb}{0.678431,1.000000,0.184314}%
\pgfsetstrokecolor{currentstroke}%
\pgfsetstrokeopacity{0.500000}%
\pgfsetdash{}{0pt}%
\pgfpathmoveto{\pgfqpoint{3.104026in}{3.271375in}}%
\pgfusepath{stroke}%
\end{pgfscope}%
\begin{pgfscope}%
\pgfpathrectangle{\pgfqpoint{0.100000in}{2.413063in}}{\pgfqpoint{5.037500in}{3.427208in}}%
\pgfusepath{clip}%
\pgfsetbuttcap%
\pgfsetroundjoin%
\definecolor{currentfill}{rgb}{0.678431,1.000000,0.184314}%
\pgfsetfillcolor{currentfill}%
\pgfsetfillopacity{0.500000}%
\pgfsetlinewidth{0.250937pt}%
\definecolor{currentstroke}{rgb}{0.000000,0.000000,0.000000}%
\pgfsetstrokecolor{currentstroke}%
\pgfsetstrokeopacity{0.500000}%
\pgfsetdash{}{0pt}%
\pgfsys@defobject{currentmarker}{\pgfqpoint{-0.030556in}{-0.030556in}}{\pgfqpoint{0.030556in}{0.030556in}}{%
\pgfpathmoveto{\pgfqpoint{0.000000in}{-0.030556in}}%
\pgfpathcurveto{\pgfqpoint{0.008103in}{-0.030556in}}{\pgfqpoint{0.015876in}{-0.027336in}}{\pgfqpoint{0.021606in}{-0.021606in}}%
\pgfpathcurveto{\pgfqpoint{0.027336in}{-0.015876in}}{\pgfqpoint{0.030556in}{-0.008103in}}{\pgfqpoint{0.030556in}{0.000000in}}%
\pgfpathcurveto{\pgfqpoint{0.030556in}{0.008103in}}{\pgfqpoint{0.027336in}{0.015876in}}{\pgfqpoint{0.021606in}{0.021606in}}%
\pgfpathcurveto{\pgfqpoint{0.015876in}{0.027336in}}{\pgfqpoint{0.008103in}{0.030556in}}{\pgfqpoint{0.000000in}{0.030556in}}%
\pgfpathcurveto{\pgfqpoint{-0.008103in}{0.030556in}}{\pgfqpoint{-0.015876in}{0.027336in}}{\pgfqpoint{-0.021606in}{0.021606in}}%
\pgfpathcurveto{\pgfqpoint{-0.027336in}{0.015876in}}{\pgfqpoint{-0.030556in}{0.008103in}}{\pgfqpoint{-0.030556in}{0.000000in}}%
\pgfpathcurveto{\pgfqpoint{-0.030556in}{-0.008103in}}{\pgfqpoint{-0.027336in}{-0.015876in}}{\pgfqpoint{-0.021606in}{-0.021606in}}%
\pgfpathcurveto{\pgfqpoint{-0.015876in}{-0.027336in}}{\pgfqpoint{-0.008103in}{-0.030556in}}{\pgfqpoint{0.000000in}{-0.030556in}}%
\pgfpathclose%
\pgfusepath{stroke,fill}%
}%
\begin{pgfscope}%
\pgfsys@transformshift{3.104026in}{3.271375in}%
\pgfsys@useobject{currentmarker}{}%
\end{pgfscope}%
\end{pgfscope}%
\begin{pgfscope}%
\pgfpathrectangle{\pgfqpoint{0.100000in}{2.413063in}}{\pgfqpoint{5.037500in}{3.427208in}}%
\pgfusepath{clip}%
\pgfsetrectcap%
\pgfsetroundjoin%
\pgfsetlinewidth{1.505625pt}%
\definecolor{currentstroke}{rgb}{0.678431,1.000000,0.184314}%
\pgfsetstrokecolor{currentstroke}%
\pgfsetstrokeopacity{0.500000}%
\pgfsetdash{}{0pt}%
\pgfpathmoveto{\pgfqpoint{2.983531in}{3.268508in}}%
\pgfusepath{stroke}%
\end{pgfscope}%
\begin{pgfscope}%
\pgfpathrectangle{\pgfqpoint{0.100000in}{2.413063in}}{\pgfqpoint{5.037500in}{3.427208in}}%
\pgfusepath{clip}%
\pgfsetbuttcap%
\pgfsetroundjoin%
\definecolor{currentfill}{rgb}{0.678431,1.000000,0.184314}%
\pgfsetfillcolor{currentfill}%
\pgfsetfillopacity{0.500000}%
\pgfsetlinewidth{0.250937pt}%
\definecolor{currentstroke}{rgb}{0.000000,0.000000,0.000000}%
\pgfsetstrokecolor{currentstroke}%
\pgfsetstrokeopacity{0.500000}%
\pgfsetdash{}{0pt}%
\pgfsys@defobject{currentmarker}{\pgfqpoint{-0.008333in}{-0.008333in}}{\pgfqpoint{0.008333in}{0.008333in}}{%
\pgfpathmoveto{\pgfqpoint{0.000000in}{-0.008333in}}%
\pgfpathcurveto{\pgfqpoint{0.002210in}{-0.008333in}}{\pgfqpoint{0.004330in}{-0.007455in}}{\pgfqpoint{0.005893in}{-0.005893in}}%
\pgfpathcurveto{\pgfqpoint{0.007455in}{-0.004330in}}{\pgfqpoint{0.008333in}{-0.002210in}}{\pgfqpoint{0.008333in}{0.000000in}}%
\pgfpathcurveto{\pgfqpoint{0.008333in}{0.002210in}}{\pgfqpoint{0.007455in}{0.004330in}}{\pgfqpoint{0.005893in}{0.005893in}}%
\pgfpathcurveto{\pgfqpoint{0.004330in}{0.007455in}}{\pgfqpoint{0.002210in}{0.008333in}}{\pgfqpoint{0.000000in}{0.008333in}}%
\pgfpathcurveto{\pgfqpoint{-0.002210in}{0.008333in}}{\pgfqpoint{-0.004330in}{0.007455in}}{\pgfqpoint{-0.005893in}{0.005893in}}%
\pgfpathcurveto{\pgfqpoint{-0.007455in}{0.004330in}}{\pgfqpoint{-0.008333in}{0.002210in}}{\pgfqpoint{-0.008333in}{0.000000in}}%
\pgfpathcurveto{\pgfqpoint{-0.008333in}{-0.002210in}}{\pgfqpoint{-0.007455in}{-0.004330in}}{\pgfqpoint{-0.005893in}{-0.005893in}}%
\pgfpathcurveto{\pgfqpoint{-0.004330in}{-0.007455in}}{\pgfqpoint{-0.002210in}{-0.008333in}}{\pgfqpoint{0.000000in}{-0.008333in}}%
\pgfpathclose%
\pgfusepath{stroke,fill}%
}%
\begin{pgfscope}%
\pgfsys@transformshift{2.983531in}{3.268508in}%
\pgfsys@useobject{currentmarker}{}%
\end{pgfscope}%
\end{pgfscope}%
\begin{pgfscope}%
\pgfpathrectangle{\pgfqpoint{0.100000in}{2.413063in}}{\pgfqpoint{5.037500in}{3.427208in}}%
\pgfusepath{clip}%
\pgfsetrectcap%
\pgfsetroundjoin%
\pgfsetlinewidth{1.505625pt}%
\definecolor{currentstroke}{rgb}{0.678431,1.000000,0.184314}%
\pgfsetstrokecolor{currentstroke}%
\pgfsetstrokeopacity{0.500000}%
\pgfsetdash{}{0pt}%
\pgfpathmoveto{\pgfqpoint{3.239760in}{4.208703in}}%
\pgfusepath{stroke}%
\end{pgfscope}%
\begin{pgfscope}%
\pgfpathrectangle{\pgfqpoint{0.100000in}{2.413063in}}{\pgfqpoint{5.037500in}{3.427208in}}%
\pgfusepath{clip}%
\pgfsetbuttcap%
\pgfsetroundjoin%
\definecolor{currentfill}{rgb}{0.678431,1.000000,0.184314}%
\pgfsetfillcolor{currentfill}%
\pgfsetfillopacity{0.500000}%
\pgfsetlinewidth{0.250937pt}%
\definecolor{currentstroke}{rgb}{0.000000,0.000000,0.000000}%
\pgfsetstrokecolor{currentstroke}%
\pgfsetstrokeopacity{0.500000}%
\pgfsetdash{}{0pt}%
\pgfsys@defobject{currentmarker}{\pgfqpoint{-0.050000in}{-0.050000in}}{\pgfqpoint{0.050000in}{0.050000in}}{%
\pgfpathmoveto{\pgfqpoint{0.000000in}{-0.050000in}}%
\pgfpathcurveto{\pgfqpoint{0.013260in}{-0.050000in}}{\pgfqpoint{0.025979in}{-0.044732in}}{\pgfqpoint{0.035355in}{-0.035355in}}%
\pgfpathcurveto{\pgfqpoint{0.044732in}{-0.025979in}}{\pgfqpoint{0.050000in}{-0.013260in}}{\pgfqpoint{0.050000in}{0.000000in}}%
\pgfpathcurveto{\pgfqpoint{0.050000in}{0.013260in}}{\pgfqpoint{0.044732in}{0.025979in}}{\pgfqpoint{0.035355in}{0.035355in}}%
\pgfpathcurveto{\pgfqpoint{0.025979in}{0.044732in}}{\pgfqpoint{0.013260in}{0.050000in}}{\pgfqpoint{0.000000in}{0.050000in}}%
\pgfpathcurveto{\pgfqpoint{-0.013260in}{0.050000in}}{\pgfqpoint{-0.025979in}{0.044732in}}{\pgfqpoint{-0.035355in}{0.035355in}}%
\pgfpathcurveto{\pgfqpoint{-0.044732in}{0.025979in}}{\pgfqpoint{-0.050000in}{0.013260in}}{\pgfqpoint{-0.050000in}{0.000000in}}%
\pgfpathcurveto{\pgfqpoint{-0.050000in}{-0.013260in}}{\pgfqpoint{-0.044732in}{-0.025979in}}{\pgfqpoint{-0.035355in}{-0.035355in}}%
\pgfpathcurveto{\pgfqpoint{-0.025979in}{-0.044732in}}{\pgfqpoint{-0.013260in}{-0.050000in}}{\pgfqpoint{0.000000in}{-0.050000in}}%
\pgfpathclose%
\pgfusepath{stroke,fill}%
}%
\begin{pgfscope}%
\pgfsys@transformshift{3.239760in}{4.208703in}%
\pgfsys@useobject{currentmarker}{}%
\end{pgfscope}%
\end{pgfscope}%
\begin{pgfscope}%
\pgfpathrectangle{\pgfqpoint{0.100000in}{2.413063in}}{\pgfqpoint{5.037500in}{3.427208in}}%
\pgfusepath{clip}%
\pgfsetrectcap%
\pgfsetroundjoin%
\pgfsetlinewidth{1.505625pt}%
\definecolor{currentstroke}{rgb}{0.000000,0.000000,1.000000}%
\pgfsetstrokecolor{currentstroke}%
\pgfsetstrokeopacity{0.500000}%
\pgfsetdash{}{0pt}%
\pgfpathmoveto{\pgfqpoint{3.301808in}{3.248013in}}%
\pgfusepath{stroke}%
\end{pgfscope}%
\begin{pgfscope}%
\pgfpathrectangle{\pgfqpoint{0.100000in}{2.413063in}}{\pgfqpoint{5.037500in}{3.427208in}}%
\pgfusepath{clip}%
\pgfsetbuttcap%
\pgfsetroundjoin%
\definecolor{currentfill}{rgb}{0.000000,0.000000,1.000000}%
\pgfsetfillcolor{currentfill}%
\pgfsetfillopacity{0.500000}%
\pgfsetlinewidth{0.250937pt}%
\definecolor{currentstroke}{rgb}{0.000000,0.000000,0.000000}%
\pgfsetstrokecolor{currentstroke}%
\pgfsetstrokeopacity{0.500000}%
\pgfsetdash{}{0pt}%
\pgfsys@defobject{currentmarker}{\pgfqpoint{-0.008333in}{-0.008333in}}{\pgfqpoint{0.008333in}{0.008333in}}{%
\pgfpathmoveto{\pgfqpoint{0.000000in}{-0.008333in}}%
\pgfpathcurveto{\pgfqpoint{0.002210in}{-0.008333in}}{\pgfqpoint{0.004330in}{-0.007455in}}{\pgfqpoint{0.005893in}{-0.005893in}}%
\pgfpathcurveto{\pgfqpoint{0.007455in}{-0.004330in}}{\pgfqpoint{0.008333in}{-0.002210in}}{\pgfqpoint{0.008333in}{0.000000in}}%
\pgfpathcurveto{\pgfqpoint{0.008333in}{0.002210in}}{\pgfqpoint{0.007455in}{0.004330in}}{\pgfqpoint{0.005893in}{0.005893in}}%
\pgfpathcurveto{\pgfqpoint{0.004330in}{0.007455in}}{\pgfqpoint{0.002210in}{0.008333in}}{\pgfqpoint{0.000000in}{0.008333in}}%
\pgfpathcurveto{\pgfqpoint{-0.002210in}{0.008333in}}{\pgfqpoint{-0.004330in}{0.007455in}}{\pgfqpoint{-0.005893in}{0.005893in}}%
\pgfpathcurveto{\pgfqpoint{-0.007455in}{0.004330in}}{\pgfqpoint{-0.008333in}{0.002210in}}{\pgfqpoint{-0.008333in}{0.000000in}}%
\pgfpathcurveto{\pgfqpoint{-0.008333in}{-0.002210in}}{\pgfqpoint{-0.007455in}{-0.004330in}}{\pgfqpoint{-0.005893in}{-0.005893in}}%
\pgfpathcurveto{\pgfqpoint{-0.004330in}{-0.007455in}}{\pgfqpoint{-0.002210in}{-0.008333in}}{\pgfqpoint{0.000000in}{-0.008333in}}%
\pgfpathclose%
\pgfusepath{stroke,fill}%
}%
\begin{pgfscope}%
\pgfsys@transformshift{3.301808in}{3.248013in}%
\pgfsys@useobject{currentmarker}{}%
\end{pgfscope}%
\end{pgfscope}%
\begin{pgfscope}%
\pgfpathrectangle{\pgfqpoint{0.100000in}{2.413063in}}{\pgfqpoint{5.037500in}{3.427208in}}%
\pgfusepath{clip}%
\pgfsetrectcap%
\pgfsetroundjoin%
\pgfsetlinewidth{1.505625pt}%
\definecolor{currentstroke}{rgb}{0.678431,1.000000,0.184314}%
\pgfsetstrokecolor{currentstroke}%
\pgfsetstrokeopacity{0.500000}%
\pgfsetdash{}{0pt}%
\pgfpathmoveto{\pgfqpoint{2.924780in}{3.534245in}}%
\pgfusepath{stroke}%
\end{pgfscope}%
\begin{pgfscope}%
\pgfpathrectangle{\pgfqpoint{0.100000in}{2.413063in}}{\pgfqpoint{5.037500in}{3.427208in}}%
\pgfusepath{clip}%
\pgfsetbuttcap%
\pgfsetroundjoin%
\definecolor{currentfill}{rgb}{0.678431,1.000000,0.184314}%
\pgfsetfillcolor{currentfill}%
\pgfsetfillopacity{0.500000}%
\pgfsetlinewidth{0.250937pt}%
\definecolor{currentstroke}{rgb}{0.000000,0.000000,0.000000}%
\pgfsetstrokecolor{currentstroke}%
\pgfsetstrokeopacity{0.500000}%
\pgfsetdash{}{0pt}%
\pgfsys@defobject{currentmarker}{\pgfqpoint{-0.030556in}{-0.030556in}}{\pgfqpoint{0.030556in}{0.030556in}}{%
\pgfpathmoveto{\pgfqpoint{0.000000in}{-0.030556in}}%
\pgfpathcurveto{\pgfqpoint{0.008103in}{-0.030556in}}{\pgfqpoint{0.015876in}{-0.027336in}}{\pgfqpoint{0.021606in}{-0.021606in}}%
\pgfpathcurveto{\pgfqpoint{0.027336in}{-0.015876in}}{\pgfqpoint{0.030556in}{-0.008103in}}{\pgfqpoint{0.030556in}{0.000000in}}%
\pgfpathcurveto{\pgfqpoint{0.030556in}{0.008103in}}{\pgfqpoint{0.027336in}{0.015876in}}{\pgfqpoint{0.021606in}{0.021606in}}%
\pgfpathcurveto{\pgfqpoint{0.015876in}{0.027336in}}{\pgfqpoint{0.008103in}{0.030556in}}{\pgfqpoint{0.000000in}{0.030556in}}%
\pgfpathcurveto{\pgfqpoint{-0.008103in}{0.030556in}}{\pgfqpoint{-0.015876in}{0.027336in}}{\pgfqpoint{-0.021606in}{0.021606in}}%
\pgfpathcurveto{\pgfqpoint{-0.027336in}{0.015876in}}{\pgfqpoint{-0.030556in}{0.008103in}}{\pgfqpoint{-0.030556in}{0.000000in}}%
\pgfpathcurveto{\pgfqpoint{-0.030556in}{-0.008103in}}{\pgfqpoint{-0.027336in}{-0.015876in}}{\pgfqpoint{-0.021606in}{-0.021606in}}%
\pgfpathcurveto{\pgfqpoint{-0.015876in}{-0.027336in}}{\pgfqpoint{-0.008103in}{-0.030556in}}{\pgfqpoint{0.000000in}{-0.030556in}}%
\pgfpathclose%
\pgfusepath{stroke,fill}%
}%
\begin{pgfscope}%
\pgfsys@transformshift{2.924780in}{3.534245in}%
\pgfsys@useobject{currentmarker}{}%
\end{pgfscope}%
\end{pgfscope}%
\begin{pgfscope}%
\pgfpathrectangle{\pgfqpoint{0.100000in}{2.413063in}}{\pgfqpoint{5.037500in}{3.427208in}}%
\pgfusepath{clip}%
\pgfsetrectcap%
\pgfsetroundjoin%
\pgfsetlinewidth{1.505625pt}%
\definecolor{currentstroke}{rgb}{0.000000,0.000000,1.000000}%
\pgfsetstrokecolor{currentstroke}%
\pgfsetstrokeopacity{0.500000}%
\pgfsetdash{}{0pt}%
\pgfpathmoveto{\pgfqpoint{4.927826in}{5.258593in}}%
\pgfusepath{stroke}%
\end{pgfscope}%
\begin{pgfscope}%
\pgfpathrectangle{\pgfqpoint{0.100000in}{2.413063in}}{\pgfqpoint{5.037500in}{3.427208in}}%
\pgfusepath{clip}%
\pgfsetbuttcap%
\pgfsetroundjoin%
\definecolor{currentfill}{rgb}{0.000000,0.000000,1.000000}%
\pgfsetfillcolor{currentfill}%
\pgfsetfillopacity{0.500000}%
\pgfsetlinewidth{0.250937pt}%
\definecolor{currentstroke}{rgb}{0.000000,0.000000,0.000000}%
\pgfsetstrokecolor{currentstroke}%
\pgfsetstrokeopacity{0.500000}%
\pgfsetdash{}{0pt}%
\pgfsys@defobject{currentmarker}{\pgfqpoint{-0.041667in}{-0.041667in}}{\pgfqpoint{0.041667in}{0.041667in}}{%
\pgfpathmoveto{\pgfqpoint{0.000000in}{-0.041667in}}%
\pgfpathcurveto{\pgfqpoint{0.011050in}{-0.041667in}}{\pgfqpoint{0.021649in}{-0.037276in}}{\pgfqpoint{0.029463in}{-0.029463in}}%
\pgfpathcurveto{\pgfqpoint{0.037276in}{-0.021649in}}{\pgfqpoint{0.041667in}{-0.011050in}}{\pgfqpoint{0.041667in}{0.000000in}}%
\pgfpathcurveto{\pgfqpoint{0.041667in}{0.011050in}}{\pgfqpoint{0.037276in}{0.021649in}}{\pgfqpoint{0.029463in}{0.029463in}}%
\pgfpathcurveto{\pgfqpoint{0.021649in}{0.037276in}}{\pgfqpoint{0.011050in}{0.041667in}}{\pgfqpoint{0.000000in}{0.041667in}}%
\pgfpathcurveto{\pgfqpoint{-0.011050in}{0.041667in}}{\pgfqpoint{-0.021649in}{0.037276in}}{\pgfqpoint{-0.029463in}{0.029463in}}%
\pgfpathcurveto{\pgfqpoint{-0.037276in}{0.021649in}}{\pgfqpoint{-0.041667in}{0.011050in}}{\pgfqpoint{-0.041667in}{0.000000in}}%
\pgfpathcurveto{\pgfqpoint{-0.041667in}{-0.011050in}}{\pgfqpoint{-0.037276in}{-0.021649in}}{\pgfqpoint{-0.029463in}{-0.029463in}}%
\pgfpathcurveto{\pgfqpoint{-0.021649in}{-0.037276in}}{\pgfqpoint{-0.011050in}{-0.041667in}}{\pgfqpoint{0.000000in}{-0.041667in}}%
\pgfpathclose%
\pgfusepath{stroke,fill}%
}%
\begin{pgfscope}%
\pgfsys@transformshift{4.927826in}{5.258593in}%
\pgfsys@useobject{currentmarker}{}%
\end{pgfscope}%
\end{pgfscope}%
\begin{pgfscope}%
\pgfpathrectangle{\pgfqpoint{0.100000in}{2.413063in}}{\pgfqpoint{5.037500in}{3.427208in}}%
\pgfusepath{clip}%
\pgfsetrectcap%
\pgfsetroundjoin%
\pgfsetlinewidth{1.505625pt}%
\definecolor{currentstroke}{rgb}{0.000000,0.000000,1.000000}%
\pgfsetstrokecolor{currentstroke}%
\pgfsetstrokeopacity{0.500000}%
\pgfsetdash{}{0pt}%
\pgfpathmoveto{\pgfqpoint{4.836271in}{5.147841in}}%
\pgfusepath{stroke}%
\end{pgfscope}%
\begin{pgfscope}%
\pgfpathrectangle{\pgfqpoint{0.100000in}{2.413063in}}{\pgfqpoint{5.037500in}{3.427208in}}%
\pgfusepath{clip}%
\pgfsetbuttcap%
\pgfsetroundjoin%
\definecolor{currentfill}{rgb}{0.000000,0.000000,1.000000}%
\pgfsetfillcolor{currentfill}%
\pgfsetfillopacity{0.500000}%
\pgfsetlinewidth{0.250937pt}%
\definecolor{currentstroke}{rgb}{0.000000,0.000000,0.000000}%
\pgfsetstrokecolor{currentstroke}%
\pgfsetstrokeopacity{0.500000}%
\pgfsetdash{}{0pt}%
\pgfsys@defobject{currentmarker}{\pgfqpoint{-0.050000in}{-0.050000in}}{\pgfqpoint{0.050000in}{0.050000in}}{%
\pgfpathmoveto{\pgfqpoint{0.000000in}{-0.050000in}}%
\pgfpathcurveto{\pgfqpoint{0.013260in}{-0.050000in}}{\pgfqpoint{0.025979in}{-0.044732in}}{\pgfqpoint{0.035355in}{-0.035355in}}%
\pgfpathcurveto{\pgfqpoint{0.044732in}{-0.025979in}}{\pgfqpoint{0.050000in}{-0.013260in}}{\pgfqpoint{0.050000in}{0.000000in}}%
\pgfpathcurveto{\pgfqpoint{0.050000in}{0.013260in}}{\pgfqpoint{0.044732in}{0.025979in}}{\pgfqpoint{0.035355in}{0.035355in}}%
\pgfpathcurveto{\pgfqpoint{0.025979in}{0.044732in}}{\pgfqpoint{0.013260in}{0.050000in}}{\pgfqpoint{0.000000in}{0.050000in}}%
\pgfpathcurveto{\pgfqpoint{-0.013260in}{0.050000in}}{\pgfqpoint{-0.025979in}{0.044732in}}{\pgfqpoint{-0.035355in}{0.035355in}}%
\pgfpathcurveto{\pgfqpoint{-0.044732in}{0.025979in}}{\pgfqpoint{-0.050000in}{0.013260in}}{\pgfqpoint{-0.050000in}{0.000000in}}%
\pgfpathcurveto{\pgfqpoint{-0.050000in}{-0.013260in}}{\pgfqpoint{-0.044732in}{-0.025979in}}{\pgfqpoint{-0.035355in}{-0.035355in}}%
\pgfpathcurveto{\pgfqpoint{-0.025979in}{-0.044732in}}{\pgfqpoint{-0.013260in}{-0.050000in}}{\pgfqpoint{0.000000in}{-0.050000in}}%
\pgfpathclose%
\pgfusepath{stroke,fill}%
}%
\begin{pgfscope}%
\pgfsys@transformshift{4.836271in}{5.147841in}%
\pgfsys@useobject{currentmarker}{}%
\end{pgfscope}%
\end{pgfscope}%
\begin{pgfscope}%
\pgfpathrectangle{\pgfqpoint{0.100000in}{2.413063in}}{\pgfqpoint{5.037500in}{3.427208in}}%
\pgfusepath{clip}%
\pgfsetrectcap%
\pgfsetroundjoin%
\pgfsetlinewidth{1.505625pt}%
\definecolor{currentstroke}{rgb}{0.000000,0.000000,1.000000}%
\pgfsetstrokecolor{currentstroke}%
\pgfsetstrokeopacity{0.500000}%
\pgfsetdash{}{0pt}%
\pgfpathmoveto{\pgfqpoint{4.846720in}{5.098047in}}%
\pgfusepath{stroke}%
\end{pgfscope}%
\begin{pgfscope}%
\pgfpathrectangle{\pgfqpoint{0.100000in}{2.413063in}}{\pgfqpoint{5.037500in}{3.427208in}}%
\pgfusepath{clip}%
\pgfsetbuttcap%
\pgfsetroundjoin%
\definecolor{currentfill}{rgb}{0.000000,0.000000,1.000000}%
\pgfsetfillcolor{currentfill}%
\pgfsetfillopacity{0.500000}%
\pgfsetlinewidth{0.250937pt}%
\definecolor{currentstroke}{rgb}{0.000000,0.000000,0.000000}%
\pgfsetstrokecolor{currentstroke}%
\pgfsetstrokeopacity{0.500000}%
\pgfsetdash{}{0pt}%
\pgfsys@defobject{currentmarker}{\pgfqpoint{-0.041667in}{-0.041667in}}{\pgfqpoint{0.041667in}{0.041667in}}{%
\pgfpathmoveto{\pgfqpoint{0.000000in}{-0.041667in}}%
\pgfpathcurveto{\pgfqpoint{0.011050in}{-0.041667in}}{\pgfqpoint{0.021649in}{-0.037276in}}{\pgfqpoint{0.029463in}{-0.029463in}}%
\pgfpathcurveto{\pgfqpoint{0.037276in}{-0.021649in}}{\pgfqpoint{0.041667in}{-0.011050in}}{\pgfqpoint{0.041667in}{0.000000in}}%
\pgfpathcurveto{\pgfqpoint{0.041667in}{0.011050in}}{\pgfqpoint{0.037276in}{0.021649in}}{\pgfqpoint{0.029463in}{0.029463in}}%
\pgfpathcurveto{\pgfqpoint{0.021649in}{0.037276in}}{\pgfqpoint{0.011050in}{0.041667in}}{\pgfqpoint{0.000000in}{0.041667in}}%
\pgfpathcurveto{\pgfqpoint{-0.011050in}{0.041667in}}{\pgfqpoint{-0.021649in}{0.037276in}}{\pgfqpoint{-0.029463in}{0.029463in}}%
\pgfpathcurveto{\pgfqpoint{-0.037276in}{0.021649in}}{\pgfqpoint{-0.041667in}{0.011050in}}{\pgfqpoint{-0.041667in}{0.000000in}}%
\pgfpathcurveto{\pgfqpoint{-0.041667in}{-0.011050in}}{\pgfqpoint{-0.037276in}{-0.021649in}}{\pgfqpoint{-0.029463in}{-0.029463in}}%
\pgfpathcurveto{\pgfqpoint{-0.021649in}{-0.037276in}}{\pgfqpoint{-0.011050in}{-0.041667in}}{\pgfqpoint{0.000000in}{-0.041667in}}%
\pgfpathclose%
\pgfusepath{stroke,fill}%
}%
\begin{pgfscope}%
\pgfsys@transformshift{4.846720in}{5.098047in}%
\pgfsys@useobject{currentmarker}{}%
\end{pgfscope}%
\end{pgfscope}%
\begin{pgfscope}%
\pgfpathrectangle{\pgfqpoint{0.100000in}{2.413063in}}{\pgfqpoint{5.037500in}{3.427208in}}%
\pgfusepath{clip}%
\pgfsetrectcap%
\pgfsetroundjoin%
\pgfsetlinewidth{1.505625pt}%
\definecolor{currentstroke}{rgb}{0.000000,0.000000,1.000000}%
\pgfsetstrokecolor{currentstroke}%
\pgfsetstrokeopacity{0.500000}%
\pgfsetdash{}{0pt}%
\pgfpathmoveto{\pgfqpoint{4.431773in}{4.479688in}}%
\pgfusepath{stroke}%
\end{pgfscope}%
\begin{pgfscope}%
\pgfpathrectangle{\pgfqpoint{0.100000in}{2.413063in}}{\pgfqpoint{5.037500in}{3.427208in}}%
\pgfusepath{clip}%
\pgfsetbuttcap%
\pgfsetroundjoin%
\definecolor{currentfill}{rgb}{0.000000,0.000000,1.000000}%
\pgfsetfillcolor{currentfill}%
\pgfsetfillopacity{0.500000}%
\pgfsetlinewidth{0.250937pt}%
\definecolor{currentstroke}{rgb}{0.000000,0.000000,0.000000}%
\pgfsetstrokecolor{currentstroke}%
\pgfsetstrokeopacity{0.500000}%
\pgfsetdash{}{0pt}%
\pgfsys@defobject{currentmarker}{\pgfqpoint{-0.022222in}{-0.022222in}}{\pgfqpoint{0.022222in}{0.022222in}}{%
\pgfpathmoveto{\pgfqpoint{0.000000in}{-0.022222in}}%
\pgfpathcurveto{\pgfqpoint{0.005893in}{-0.022222in}}{\pgfqpoint{0.011546in}{-0.019881in}}{\pgfqpoint{0.015713in}{-0.015713in}}%
\pgfpathcurveto{\pgfqpoint{0.019881in}{-0.011546in}}{\pgfqpoint{0.022222in}{-0.005893in}}{\pgfqpoint{0.022222in}{0.000000in}}%
\pgfpathcurveto{\pgfqpoint{0.022222in}{0.005893in}}{\pgfqpoint{0.019881in}{0.011546in}}{\pgfqpoint{0.015713in}{0.015713in}}%
\pgfpathcurveto{\pgfqpoint{0.011546in}{0.019881in}}{\pgfqpoint{0.005893in}{0.022222in}}{\pgfqpoint{0.000000in}{0.022222in}}%
\pgfpathcurveto{\pgfqpoint{-0.005893in}{0.022222in}}{\pgfqpoint{-0.011546in}{0.019881in}}{\pgfqpoint{-0.015713in}{0.015713in}}%
\pgfpathcurveto{\pgfqpoint{-0.019881in}{0.011546in}}{\pgfqpoint{-0.022222in}{0.005893in}}{\pgfqpoint{-0.022222in}{0.000000in}}%
\pgfpathcurveto{\pgfqpoint{-0.022222in}{-0.005893in}}{\pgfqpoint{-0.019881in}{-0.011546in}}{\pgfqpoint{-0.015713in}{-0.015713in}}%
\pgfpathcurveto{\pgfqpoint{-0.011546in}{-0.019881in}}{\pgfqpoint{-0.005893in}{-0.022222in}}{\pgfqpoint{0.000000in}{-0.022222in}}%
\pgfpathclose%
\pgfusepath{stroke,fill}%
}%
\begin{pgfscope}%
\pgfsys@transformshift{4.431773in}{4.479688in}%
\pgfsys@useobject{currentmarker}{}%
\end{pgfscope}%
\end{pgfscope}%
\begin{pgfscope}%
\pgfpathrectangle{\pgfqpoint{0.100000in}{2.413063in}}{\pgfqpoint{5.037500in}{3.427208in}}%
\pgfusepath{clip}%
\pgfsetrectcap%
\pgfsetroundjoin%
\pgfsetlinewidth{1.505625pt}%
\definecolor{currentstroke}{rgb}{0.000000,0.000000,1.000000}%
\pgfsetstrokecolor{currentstroke}%
\pgfsetstrokeopacity{0.500000}%
\pgfsetdash{}{0pt}%
\pgfpathmoveto{\pgfqpoint{4.463839in}{4.369843in}}%
\pgfusepath{stroke}%
\end{pgfscope}%
\begin{pgfscope}%
\pgfpathrectangle{\pgfqpoint{0.100000in}{2.413063in}}{\pgfqpoint{5.037500in}{3.427208in}}%
\pgfusepath{clip}%
\pgfsetbuttcap%
\pgfsetroundjoin%
\definecolor{currentfill}{rgb}{0.000000,0.000000,1.000000}%
\pgfsetfillcolor{currentfill}%
\pgfsetfillopacity{0.500000}%
\pgfsetlinewidth{0.250937pt}%
\definecolor{currentstroke}{rgb}{0.000000,0.000000,0.000000}%
\pgfsetstrokecolor{currentstroke}%
\pgfsetstrokeopacity{0.500000}%
\pgfsetdash{}{0pt}%
\pgfsys@defobject{currentmarker}{\pgfqpoint{-0.019444in}{-0.019444in}}{\pgfqpoint{0.019444in}{0.019444in}}{%
\pgfpathmoveto{\pgfqpoint{0.000000in}{-0.019444in}}%
\pgfpathcurveto{\pgfqpoint{0.005157in}{-0.019444in}}{\pgfqpoint{0.010103in}{-0.017396in}}{\pgfqpoint{0.013749in}{-0.013749in}}%
\pgfpathcurveto{\pgfqpoint{0.017396in}{-0.010103in}}{\pgfqpoint{0.019444in}{-0.005157in}}{\pgfqpoint{0.019444in}{0.000000in}}%
\pgfpathcurveto{\pgfqpoint{0.019444in}{0.005157in}}{\pgfqpoint{0.017396in}{0.010103in}}{\pgfqpoint{0.013749in}{0.013749in}}%
\pgfpathcurveto{\pgfqpoint{0.010103in}{0.017396in}}{\pgfqpoint{0.005157in}{0.019444in}}{\pgfqpoint{0.000000in}{0.019444in}}%
\pgfpathcurveto{\pgfqpoint{-0.005157in}{0.019444in}}{\pgfqpoint{-0.010103in}{0.017396in}}{\pgfqpoint{-0.013749in}{0.013749in}}%
\pgfpathcurveto{\pgfqpoint{-0.017396in}{0.010103in}}{\pgfqpoint{-0.019444in}{0.005157in}}{\pgfqpoint{-0.019444in}{0.000000in}}%
\pgfpathcurveto{\pgfqpoint{-0.019444in}{-0.005157in}}{\pgfqpoint{-0.017396in}{-0.010103in}}{\pgfqpoint{-0.013749in}{-0.013749in}}%
\pgfpathcurveto{\pgfqpoint{-0.010103in}{-0.017396in}}{\pgfqpoint{-0.005157in}{-0.019444in}}{\pgfqpoint{0.000000in}{-0.019444in}}%
\pgfpathclose%
\pgfusepath{stroke,fill}%
}%
\begin{pgfscope}%
\pgfsys@transformshift{4.463839in}{4.369843in}%
\pgfsys@useobject{currentmarker}{}%
\end{pgfscope}%
\end{pgfscope}%
\begin{pgfscope}%
\pgfpathrectangle{\pgfqpoint{0.100000in}{2.413063in}}{\pgfqpoint{5.037500in}{3.427208in}}%
\pgfusepath{clip}%
\pgfsetrectcap%
\pgfsetroundjoin%
\pgfsetlinewidth{1.505625pt}%
\definecolor{currentstroke}{rgb}{0.678431,1.000000,0.184314}%
\pgfsetstrokecolor{currentstroke}%
\pgfsetstrokeopacity{0.500000}%
\pgfsetdash{}{0pt}%
\pgfpathmoveto{\pgfqpoint{4.236084in}{4.484394in}}%
\pgfusepath{stroke}%
\end{pgfscope}%
\begin{pgfscope}%
\pgfpathrectangle{\pgfqpoint{0.100000in}{2.413063in}}{\pgfqpoint{5.037500in}{3.427208in}}%
\pgfusepath{clip}%
\pgfsetbuttcap%
\pgfsetroundjoin%
\definecolor{currentfill}{rgb}{0.678431,1.000000,0.184314}%
\pgfsetfillcolor{currentfill}%
\pgfsetfillopacity{0.500000}%
\pgfsetlinewidth{0.250937pt}%
\definecolor{currentstroke}{rgb}{0.000000,0.000000,0.000000}%
\pgfsetstrokecolor{currentstroke}%
\pgfsetstrokeopacity{0.500000}%
\pgfsetdash{}{0pt}%
\pgfsys@defobject{currentmarker}{\pgfqpoint{-0.019444in}{-0.019444in}}{\pgfqpoint{0.019444in}{0.019444in}}{%
\pgfpathmoveto{\pgfqpoint{0.000000in}{-0.019444in}}%
\pgfpathcurveto{\pgfqpoint{0.005157in}{-0.019444in}}{\pgfqpoint{0.010103in}{-0.017396in}}{\pgfqpoint{0.013749in}{-0.013749in}}%
\pgfpathcurveto{\pgfqpoint{0.017396in}{-0.010103in}}{\pgfqpoint{0.019444in}{-0.005157in}}{\pgfqpoint{0.019444in}{0.000000in}}%
\pgfpathcurveto{\pgfqpoint{0.019444in}{0.005157in}}{\pgfqpoint{0.017396in}{0.010103in}}{\pgfqpoint{0.013749in}{0.013749in}}%
\pgfpathcurveto{\pgfqpoint{0.010103in}{0.017396in}}{\pgfqpoint{0.005157in}{0.019444in}}{\pgfqpoint{0.000000in}{0.019444in}}%
\pgfpathcurveto{\pgfqpoint{-0.005157in}{0.019444in}}{\pgfqpoint{-0.010103in}{0.017396in}}{\pgfqpoint{-0.013749in}{0.013749in}}%
\pgfpathcurveto{\pgfqpoint{-0.017396in}{0.010103in}}{\pgfqpoint{-0.019444in}{0.005157in}}{\pgfqpoint{-0.019444in}{0.000000in}}%
\pgfpathcurveto{\pgfqpoint{-0.019444in}{-0.005157in}}{\pgfqpoint{-0.017396in}{-0.010103in}}{\pgfqpoint{-0.013749in}{-0.013749in}}%
\pgfpathcurveto{\pgfqpoint{-0.010103in}{-0.017396in}}{\pgfqpoint{-0.005157in}{-0.019444in}}{\pgfqpoint{0.000000in}{-0.019444in}}%
\pgfpathclose%
\pgfusepath{stroke,fill}%
}%
\begin{pgfscope}%
\pgfsys@transformshift{4.236084in}{4.484394in}%
\pgfsys@useobject{currentmarker}{}%
\end{pgfscope}%
\end{pgfscope}%
\begin{pgfscope}%
\pgfpathrectangle{\pgfqpoint{0.100000in}{2.413063in}}{\pgfqpoint{5.037500in}{3.427208in}}%
\pgfusepath{clip}%
\pgfsetrectcap%
\pgfsetroundjoin%
\pgfsetlinewidth{1.505625pt}%
\definecolor{currentstroke}{rgb}{0.678431,1.000000,0.184314}%
\pgfsetstrokecolor{currentstroke}%
\pgfsetstrokeopacity{0.500000}%
\pgfsetdash{}{0pt}%
\pgfpathmoveto{\pgfqpoint{4.327155in}{4.500101in}}%
\pgfusepath{stroke}%
\end{pgfscope}%
\begin{pgfscope}%
\pgfpathrectangle{\pgfqpoint{0.100000in}{2.413063in}}{\pgfqpoint{5.037500in}{3.427208in}}%
\pgfusepath{clip}%
\pgfsetbuttcap%
\pgfsetroundjoin%
\definecolor{currentfill}{rgb}{0.678431,1.000000,0.184314}%
\pgfsetfillcolor{currentfill}%
\pgfsetfillopacity{0.500000}%
\pgfsetlinewidth{0.250937pt}%
\definecolor{currentstroke}{rgb}{0.000000,0.000000,0.000000}%
\pgfsetstrokecolor{currentstroke}%
\pgfsetstrokeopacity{0.500000}%
\pgfsetdash{}{0pt}%
\pgfsys@defobject{currentmarker}{\pgfqpoint{-0.005556in}{-0.005556in}}{\pgfqpoint{0.005556in}{0.005556in}}{%
\pgfpathmoveto{\pgfqpoint{0.000000in}{-0.005556in}}%
\pgfpathcurveto{\pgfqpoint{0.001473in}{-0.005556in}}{\pgfqpoint{0.002887in}{-0.004970in}}{\pgfqpoint{0.003928in}{-0.003928in}}%
\pgfpathcurveto{\pgfqpoint{0.004970in}{-0.002887in}}{\pgfqpoint{0.005556in}{-0.001473in}}{\pgfqpoint{0.005556in}{0.000000in}}%
\pgfpathcurveto{\pgfqpoint{0.005556in}{0.001473in}}{\pgfqpoint{0.004970in}{0.002887in}}{\pgfqpoint{0.003928in}{0.003928in}}%
\pgfpathcurveto{\pgfqpoint{0.002887in}{0.004970in}}{\pgfqpoint{0.001473in}{0.005556in}}{\pgfqpoint{0.000000in}{0.005556in}}%
\pgfpathcurveto{\pgfqpoint{-0.001473in}{0.005556in}}{\pgfqpoint{-0.002887in}{0.004970in}}{\pgfqpoint{-0.003928in}{0.003928in}}%
\pgfpathcurveto{\pgfqpoint{-0.004970in}{0.002887in}}{\pgfqpoint{-0.005556in}{0.001473in}}{\pgfqpoint{-0.005556in}{0.000000in}}%
\pgfpathcurveto{\pgfqpoint{-0.005556in}{-0.001473in}}{\pgfqpoint{-0.004970in}{-0.002887in}}{\pgfqpoint{-0.003928in}{-0.003928in}}%
\pgfpathcurveto{\pgfqpoint{-0.002887in}{-0.004970in}}{\pgfqpoint{-0.001473in}{-0.005556in}}{\pgfqpoint{0.000000in}{-0.005556in}}%
\pgfpathclose%
\pgfusepath{stroke,fill}%
}%
\begin{pgfscope}%
\pgfsys@transformshift{4.327155in}{4.500101in}%
\pgfsys@useobject{currentmarker}{}%
\end{pgfscope}%
\end{pgfscope}%
\begin{pgfscope}%
\pgfpathrectangle{\pgfqpoint{0.100000in}{2.413063in}}{\pgfqpoint{5.037500in}{3.427208in}}%
\pgfusepath{clip}%
\pgfsetrectcap%
\pgfsetroundjoin%
\pgfsetlinewidth{1.505625pt}%
\definecolor{currentstroke}{rgb}{0.000000,0.000000,1.000000}%
\pgfsetstrokecolor{currentstroke}%
\pgfsetstrokeopacity{0.500000}%
\pgfsetdash{}{0pt}%
\pgfpathmoveto{\pgfqpoint{4.906699in}{4.874236in}}%
\pgfusepath{stroke}%
\end{pgfscope}%
\begin{pgfscope}%
\pgfpathrectangle{\pgfqpoint{0.100000in}{2.413063in}}{\pgfqpoint{5.037500in}{3.427208in}}%
\pgfusepath{clip}%
\pgfsetbuttcap%
\pgfsetroundjoin%
\definecolor{currentfill}{rgb}{0.000000,0.000000,1.000000}%
\pgfsetfillcolor{currentfill}%
\pgfsetfillopacity{0.500000}%
\pgfsetlinewidth{0.250937pt}%
\definecolor{currentstroke}{rgb}{0.000000,0.000000,0.000000}%
\pgfsetstrokecolor{currentstroke}%
\pgfsetstrokeopacity{0.500000}%
\pgfsetdash{}{0pt}%
\pgfsys@defobject{currentmarker}{\pgfqpoint{-0.019444in}{-0.019444in}}{\pgfqpoint{0.019444in}{0.019444in}}{%
\pgfpathmoveto{\pgfqpoint{0.000000in}{-0.019444in}}%
\pgfpathcurveto{\pgfqpoint{0.005157in}{-0.019444in}}{\pgfqpoint{0.010103in}{-0.017396in}}{\pgfqpoint{0.013749in}{-0.013749in}}%
\pgfpathcurveto{\pgfqpoint{0.017396in}{-0.010103in}}{\pgfqpoint{0.019444in}{-0.005157in}}{\pgfqpoint{0.019444in}{0.000000in}}%
\pgfpathcurveto{\pgfqpoint{0.019444in}{0.005157in}}{\pgfqpoint{0.017396in}{0.010103in}}{\pgfqpoint{0.013749in}{0.013749in}}%
\pgfpathcurveto{\pgfqpoint{0.010103in}{0.017396in}}{\pgfqpoint{0.005157in}{0.019444in}}{\pgfqpoint{0.000000in}{0.019444in}}%
\pgfpathcurveto{\pgfqpoint{-0.005157in}{0.019444in}}{\pgfqpoint{-0.010103in}{0.017396in}}{\pgfqpoint{-0.013749in}{0.013749in}}%
\pgfpathcurveto{\pgfqpoint{-0.017396in}{0.010103in}}{\pgfqpoint{-0.019444in}{0.005157in}}{\pgfqpoint{-0.019444in}{0.000000in}}%
\pgfpathcurveto{\pgfqpoint{-0.019444in}{-0.005157in}}{\pgfqpoint{-0.017396in}{-0.010103in}}{\pgfqpoint{-0.013749in}{-0.013749in}}%
\pgfpathcurveto{\pgfqpoint{-0.010103in}{-0.017396in}}{\pgfqpoint{-0.005157in}{-0.019444in}}{\pgfqpoint{0.000000in}{-0.019444in}}%
\pgfpathclose%
\pgfusepath{stroke,fill}%
}%
\begin{pgfscope}%
\pgfsys@transformshift{4.906699in}{4.874236in}%
\pgfsys@useobject{currentmarker}{}%
\end{pgfscope}%
\end{pgfscope}%
\begin{pgfscope}%
\pgfpathrectangle{\pgfqpoint{0.100000in}{2.413063in}}{\pgfqpoint{5.037500in}{3.427208in}}%
\pgfusepath{clip}%
\pgfsetrectcap%
\pgfsetroundjoin%
\pgfsetlinewidth{1.505625pt}%
\definecolor{currentstroke}{rgb}{0.000000,0.000000,1.000000}%
\pgfsetstrokecolor{currentstroke}%
\pgfsetstrokeopacity{0.500000}%
\pgfsetdash{}{0pt}%
\pgfpathmoveto{\pgfqpoint{4.820957in}{4.935309in}}%
\pgfusepath{stroke}%
\end{pgfscope}%
\begin{pgfscope}%
\pgfpathrectangle{\pgfqpoint{0.100000in}{2.413063in}}{\pgfqpoint{5.037500in}{3.427208in}}%
\pgfusepath{clip}%
\pgfsetbuttcap%
\pgfsetroundjoin%
\definecolor{currentfill}{rgb}{0.000000,0.000000,1.000000}%
\pgfsetfillcolor{currentfill}%
\pgfsetfillopacity{0.500000}%
\pgfsetlinewidth{0.250937pt}%
\definecolor{currentstroke}{rgb}{0.000000,0.000000,0.000000}%
\pgfsetstrokecolor{currentstroke}%
\pgfsetstrokeopacity{0.500000}%
\pgfsetdash{}{0pt}%
\pgfsys@defobject{currentmarker}{\pgfqpoint{-0.025000in}{-0.025000in}}{\pgfqpoint{0.025000in}{0.025000in}}{%
\pgfpathmoveto{\pgfqpoint{0.000000in}{-0.025000in}}%
\pgfpathcurveto{\pgfqpoint{0.006630in}{-0.025000in}}{\pgfqpoint{0.012989in}{-0.022366in}}{\pgfqpoint{0.017678in}{-0.017678in}}%
\pgfpathcurveto{\pgfqpoint{0.022366in}{-0.012989in}}{\pgfqpoint{0.025000in}{-0.006630in}}{\pgfqpoint{0.025000in}{0.000000in}}%
\pgfpathcurveto{\pgfqpoint{0.025000in}{0.006630in}}{\pgfqpoint{0.022366in}{0.012989in}}{\pgfqpoint{0.017678in}{0.017678in}}%
\pgfpathcurveto{\pgfqpoint{0.012989in}{0.022366in}}{\pgfqpoint{0.006630in}{0.025000in}}{\pgfqpoint{0.000000in}{0.025000in}}%
\pgfpathcurveto{\pgfqpoint{-0.006630in}{0.025000in}}{\pgfqpoint{-0.012989in}{0.022366in}}{\pgfqpoint{-0.017678in}{0.017678in}}%
\pgfpathcurveto{\pgfqpoint{-0.022366in}{0.012989in}}{\pgfqpoint{-0.025000in}{0.006630in}}{\pgfqpoint{-0.025000in}{0.000000in}}%
\pgfpathcurveto{\pgfqpoint{-0.025000in}{-0.006630in}}{\pgfqpoint{-0.022366in}{-0.012989in}}{\pgfqpoint{-0.017678in}{-0.017678in}}%
\pgfpathcurveto{\pgfqpoint{-0.012989in}{-0.022366in}}{\pgfqpoint{-0.006630in}{-0.025000in}}{\pgfqpoint{0.000000in}{-0.025000in}}%
\pgfpathclose%
\pgfusepath{stroke,fill}%
}%
\begin{pgfscope}%
\pgfsys@transformshift{4.820957in}{4.935309in}%
\pgfsys@useobject{currentmarker}{}%
\end{pgfscope}%
\end{pgfscope}%
\begin{pgfscope}%
\pgfpathrectangle{\pgfqpoint{0.100000in}{2.413063in}}{\pgfqpoint{5.037500in}{3.427208in}}%
\pgfusepath{clip}%
\pgfsetrectcap%
\pgfsetroundjoin%
\pgfsetlinewidth{1.505625pt}%
\definecolor{currentstroke}{rgb}{0.000000,0.000000,1.000000}%
\pgfsetstrokecolor{currentstroke}%
\pgfsetstrokeopacity{0.500000}%
\pgfsetdash{}{0pt}%
\pgfpathmoveto{\pgfqpoint{4.758476in}{4.938361in}}%
\pgfusepath{stroke}%
\end{pgfscope}%
\begin{pgfscope}%
\pgfpathrectangle{\pgfqpoint{0.100000in}{2.413063in}}{\pgfqpoint{5.037500in}{3.427208in}}%
\pgfusepath{clip}%
\pgfsetbuttcap%
\pgfsetroundjoin%
\definecolor{currentfill}{rgb}{0.000000,0.000000,1.000000}%
\pgfsetfillcolor{currentfill}%
\pgfsetfillopacity{0.500000}%
\pgfsetlinewidth{0.250937pt}%
\definecolor{currentstroke}{rgb}{0.000000,0.000000,0.000000}%
\pgfsetstrokecolor{currentstroke}%
\pgfsetstrokeopacity{0.500000}%
\pgfsetdash{}{0pt}%
\pgfsys@defobject{currentmarker}{\pgfqpoint{-0.030556in}{-0.030556in}}{\pgfqpoint{0.030556in}{0.030556in}}{%
\pgfpathmoveto{\pgfqpoint{0.000000in}{-0.030556in}}%
\pgfpathcurveto{\pgfqpoint{0.008103in}{-0.030556in}}{\pgfqpoint{0.015876in}{-0.027336in}}{\pgfqpoint{0.021606in}{-0.021606in}}%
\pgfpathcurveto{\pgfqpoint{0.027336in}{-0.015876in}}{\pgfqpoint{0.030556in}{-0.008103in}}{\pgfqpoint{0.030556in}{0.000000in}}%
\pgfpathcurveto{\pgfqpoint{0.030556in}{0.008103in}}{\pgfqpoint{0.027336in}{0.015876in}}{\pgfqpoint{0.021606in}{0.021606in}}%
\pgfpathcurveto{\pgfqpoint{0.015876in}{0.027336in}}{\pgfqpoint{0.008103in}{0.030556in}}{\pgfqpoint{0.000000in}{0.030556in}}%
\pgfpathcurveto{\pgfqpoint{-0.008103in}{0.030556in}}{\pgfqpoint{-0.015876in}{0.027336in}}{\pgfqpoint{-0.021606in}{0.021606in}}%
\pgfpathcurveto{\pgfqpoint{-0.027336in}{0.015876in}}{\pgfqpoint{-0.030556in}{0.008103in}}{\pgfqpoint{-0.030556in}{0.000000in}}%
\pgfpathcurveto{\pgfqpoint{-0.030556in}{-0.008103in}}{\pgfqpoint{-0.027336in}{-0.015876in}}{\pgfqpoint{-0.021606in}{-0.021606in}}%
\pgfpathcurveto{\pgfqpoint{-0.015876in}{-0.027336in}}{\pgfqpoint{-0.008103in}{-0.030556in}}{\pgfqpoint{0.000000in}{-0.030556in}}%
\pgfpathclose%
\pgfusepath{stroke,fill}%
}%
\begin{pgfscope}%
\pgfsys@transformshift{4.758476in}{4.938361in}%
\pgfsys@useobject{currentmarker}{}%
\end{pgfscope}%
\end{pgfscope}%
\begin{pgfscope}%
\pgfpathrectangle{\pgfqpoint{0.100000in}{2.413063in}}{\pgfqpoint{5.037500in}{3.427208in}}%
\pgfusepath{clip}%
\pgfsetrectcap%
\pgfsetroundjoin%
\pgfsetlinewidth{1.505625pt}%
\definecolor{currentstroke}{rgb}{0.000000,0.000000,1.000000}%
\pgfsetstrokecolor{currentstroke}%
\pgfsetstrokeopacity{0.500000}%
\pgfsetdash{}{0pt}%
\pgfpathmoveto{\pgfqpoint{4.853127in}{4.857502in}}%
\pgfusepath{stroke}%
\end{pgfscope}%
\begin{pgfscope}%
\pgfpathrectangle{\pgfqpoint{0.100000in}{2.413063in}}{\pgfqpoint{5.037500in}{3.427208in}}%
\pgfusepath{clip}%
\pgfsetbuttcap%
\pgfsetroundjoin%
\definecolor{currentfill}{rgb}{0.000000,0.000000,1.000000}%
\pgfsetfillcolor{currentfill}%
\pgfsetfillopacity{0.500000}%
\pgfsetlinewidth{0.250937pt}%
\definecolor{currentstroke}{rgb}{0.000000,0.000000,0.000000}%
\pgfsetstrokecolor{currentstroke}%
\pgfsetstrokeopacity{0.500000}%
\pgfsetdash{}{0pt}%
\pgfsys@defobject{currentmarker}{\pgfqpoint{-0.030556in}{-0.030556in}}{\pgfqpoint{0.030556in}{0.030556in}}{%
\pgfpathmoveto{\pgfqpoint{0.000000in}{-0.030556in}}%
\pgfpathcurveto{\pgfqpoint{0.008103in}{-0.030556in}}{\pgfqpoint{0.015876in}{-0.027336in}}{\pgfqpoint{0.021606in}{-0.021606in}}%
\pgfpathcurveto{\pgfqpoint{0.027336in}{-0.015876in}}{\pgfqpoint{0.030556in}{-0.008103in}}{\pgfqpoint{0.030556in}{0.000000in}}%
\pgfpathcurveto{\pgfqpoint{0.030556in}{0.008103in}}{\pgfqpoint{0.027336in}{0.015876in}}{\pgfqpoint{0.021606in}{0.021606in}}%
\pgfpathcurveto{\pgfqpoint{0.015876in}{0.027336in}}{\pgfqpoint{0.008103in}{0.030556in}}{\pgfqpoint{0.000000in}{0.030556in}}%
\pgfpathcurveto{\pgfqpoint{-0.008103in}{0.030556in}}{\pgfqpoint{-0.015876in}{0.027336in}}{\pgfqpoint{-0.021606in}{0.021606in}}%
\pgfpathcurveto{\pgfqpoint{-0.027336in}{0.015876in}}{\pgfqpoint{-0.030556in}{0.008103in}}{\pgfqpoint{-0.030556in}{0.000000in}}%
\pgfpathcurveto{\pgfqpoint{-0.030556in}{-0.008103in}}{\pgfqpoint{-0.027336in}{-0.015876in}}{\pgfqpoint{-0.021606in}{-0.021606in}}%
\pgfpathcurveto{\pgfqpoint{-0.015876in}{-0.027336in}}{\pgfqpoint{-0.008103in}{-0.030556in}}{\pgfqpoint{0.000000in}{-0.030556in}}%
\pgfpathclose%
\pgfusepath{stroke,fill}%
}%
\begin{pgfscope}%
\pgfsys@transformshift{4.853127in}{4.857502in}%
\pgfsys@useobject{currentmarker}{}%
\end{pgfscope}%
\end{pgfscope}%
\begin{pgfscope}%
\pgfpathrectangle{\pgfqpoint{0.100000in}{2.413063in}}{\pgfqpoint{5.037500in}{3.427208in}}%
\pgfusepath{clip}%
\pgfsetrectcap%
\pgfsetroundjoin%
\pgfsetlinewidth{1.505625pt}%
\definecolor{currentstroke}{rgb}{0.000000,0.000000,1.000000}%
\pgfsetstrokecolor{currentstroke}%
\pgfsetstrokeopacity{0.500000}%
\pgfsetdash{}{0pt}%
\pgfpathmoveto{\pgfqpoint{4.638079in}{4.899009in}}%
\pgfusepath{stroke}%
\end{pgfscope}%
\begin{pgfscope}%
\pgfpathrectangle{\pgfqpoint{0.100000in}{2.413063in}}{\pgfqpoint{5.037500in}{3.427208in}}%
\pgfusepath{clip}%
\pgfsetbuttcap%
\pgfsetroundjoin%
\definecolor{currentfill}{rgb}{0.000000,0.000000,1.000000}%
\pgfsetfillcolor{currentfill}%
\pgfsetfillopacity{0.500000}%
\pgfsetlinewidth{0.250937pt}%
\definecolor{currentstroke}{rgb}{0.000000,0.000000,0.000000}%
\pgfsetstrokecolor{currentstroke}%
\pgfsetstrokeopacity{0.500000}%
\pgfsetdash{}{0pt}%
\pgfsys@defobject{currentmarker}{\pgfqpoint{-0.030556in}{-0.030556in}}{\pgfqpoint{0.030556in}{0.030556in}}{%
\pgfpathmoveto{\pgfqpoint{0.000000in}{-0.030556in}}%
\pgfpathcurveto{\pgfqpoint{0.008103in}{-0.030556in}}{\pgfqpoint{0.015876in}{-0.027336in}}{\pgfqpoint{0.021606in}{-0.021606in}}%
\pgfpathcurveto{\pgfqpoint{0.027336in}{-0.015876in}}{\pgfqpoint{0.030556in}{-0.008103in}}{\pgfqpoint{0.030556in}{0.000000in}}%
\pgfpathcurveto{\pgfqpoint{0.030556in}{0.008103in}}{\pgfqpoint{0.027336in}{0.015876in}}{\pgfqpoint{0.021606in}{0.021606in}}%
\pgfpathcurveto{\pgfqpoint{0.015876in}{0.027336in}}{\pgfqpoint{0.008103in}{0.030556in}}{\pgfqpoint{0.000000in}{0.030556in}}%
\pgfpathcurveto{\pgfqpoint{-0.008103in}{0.030556in}}{\pgfqpoint{-0.015876in}{0.027336in}}{\pgfqpoint{-0.021606in}{0.021606in}}%
\pgfpathcurveto{\pgfqpoint{-0.027336in}{0.015876in}}{\pgfqpoint{-0.030556in}{0.008103in}}{\pgfqpoint{-0.030556in}{0.000000in}}%
\pgfpathcurveto{\pgfqpoint{-0.030556in}{-0.008103in}}{\pgfqpoint{-0.027336in}{-0.015876in}}{\pgfqpoint{-0.021606in}{-0.021606in}}%
\pgfpathcurveto{\pgfqpoint{-0.015876in}{-0.027336in}}{\pgfqpoint{-0.008103in}{-0.030556in}}{\pgfqpoint{0.000000in}{-0.030556in}}%
\pgfpathclose%
\pgfusepath{stroke,fill}%
}%
\begin{pgfscope}%
\pgfsys@transformshift{4.638079in}{4.899009in}%
\pgfsys@useobject{currentmarker}{}%
\end{pgfscope}%
\end{pgfscope}%
\begin{pgfscope}%
\pgfpathrectangle{\pgfqpoint{0.100000in}{2.413063in}}{\pgfqpoint{5.037500in}{3.427208in}}%
\pgfusepath{clip}%
\pgfsetrectcap%
\pgfsetroundjoin%
\pgfsetlinewidth{1.505625pt}%
\definecolor{currentstroke}{rgb}{0.000000,0.000000,1.000000}%
\pgfsetstrokecolor{currentstroke}%
\pgfsetstrokeopacity{0.500000}%
\pgfsetdash{}{0pt}%
\pgfpathmoveto{\pgfqpoint{4.702051in}{4.873445in}}%
\pgfusepath{stroke}%
\end{pgfscope}%
\begin{pgfscope}%
\pgfpathrectangle{\pgfqpoint{0.100000in}{2.413063in}}{\pgfqpoint{5.037500in}{3.427208in}}%
\pgfusepath{clip}%
\pgfsetbuttcap%
\pgfsetroundjoin%
\definecolor{currentfill}{rgb}{0.000000,0.000000,1.000000}%
\pgfsetfillcolor{currentfill}%
\pgfsetfillopacity{0.500000}%
\pgfsetlinewidth{0.250937pt}%
\definecolor{currentstroke}{rgb}{0.000000,0.000000,0.000000}%
\pgfsetstrokecolor{currentstroke}%
\pgfsetstrokeopacity{0.500000}%
\pgfsetdash{}{0pt}%
\pgfsys@defobject{currentmarker}{\pgfqpoint{-0.030556in}{-0.030556in}}{\pgfqpoint{0.030556in}{0.030556in}}{%
\pgfpathmoveto{\pgfqpoint{0.000000in}{-0.030556in}}%
\pgfpathcurveto{\pgfqpoint{0.008103in}{-0.030556in}}{\pgfqpoint{0.015876in}{-0.027336in}}{\pgfqpoint{0.021606in}{-0.021606in}}%
\pgfpathcurveto{\pgfqpoint{0.027336in}{-0.015876in}}{\pgfqpoint{0.030556in}{-0.008103in}}{\pgfqpoint{0.030556in}{0.000000in}}%
\pgfpathcurveto{\pgfqpoint{0.030556in}{0.008103in}}{\pgfqpoint{0.027336in}{0.015876in}}{\pgfqpoint{0.021606in}{0.021606in}}%
\pgfpathcurveto{\pgfqpoint{0.015876in}{0.027336in}}{\pgfqpoint{0.008103in}{0.030556in}}{\pgfqpoint{0.000000in}{0.030556in}}%
\pgfpathcurveto{\pgfqpoint{-0.008103in}{0.030556in}}{\pgfqpoint{-0.015876in}{0.027336in}}{\pgfqpoint{-0.021606in}{0.021606in}}%
\pgfpathcurveto{\pgfqpoint{-0.027336in}{0.015876in}}{\pgfqpoint{-0.030556in}{0.008103in}}{\pgfqpoint{-0.030556in}{0.000000in}}%
\pgfpathcurveto{\pgfqpoint{-0.030556in}{-0.008103in}}{\pgfqpoint{-0.027336in}{-0.015876in}}{\pgfqpoint{-0.021606in}{-0.021606in}}%
\pgfpathcurveto{\pgfqpoint{-0.015876in}{-0.027336in}}{\pgfqpoint{-0.008103in}{-0.030556in}}{\pgfqpoint{0.000000in}{-0.030556in}}%
\pgfpathclose%
\pgfusepath{stroke,fill}%
}%
\begin{pgfscope}%
\pgfsys@transformshift{4.702051in}{4.873445in}%
\pgfsys@useobject{currentmarker}{}%
\end{pgfscope}%
\end{pgfscope}%
\begin{pgfscope}%
\pgfpathrectangle{\pgfqpoint{0.100000in}{2.413063in}}{\pgfqpoint{5.037500in}{3.427208in}}%
\pgfusepath{clip}%
\pgfsetrectcap%
\pgfsetroundjoin%
\pgfsetlinewidth{1.505625pt}%
\definecolor{currentstroke}{rgb}{0.000000,0.000000,1.000000}%
\pgfsetstrokecolor{currentstroke}%
\pgfsetstrokeopacity{0.500000}%
\pgfsetdash{}{0pt}%
\pgfpathmoveto{\pgfqpoint{4.762625in}{4.908152in}}%
\pgfusepath{stroke}%
\end{pgfscope}%
\begin{pgfscope}%
\pgfpathrectangle{\pgfqpoint{0.100000in}{2.413063in}}{\pgfqpoint{5.037500in}{3.427208in}}%
\pgfusepath{clip}%
\pgfsetbuttcap%
\pgfsetroundjoin%
\definecolor{currentfill}{rgb}{0.000000,0.000000,1.000000}%
\pgfsetfillcolor{currentfill}%
\pgfsetfillopacity{0.500000}%
\pgfsetlinewidth{0.250937pt}%
\definecolor{currentstroke}{rgb}{0.000000,0.000000,0.000000}%
\pgfsetstrokecolor{currentstroke}%
\pgfsetstrokeopacity{0.500000}%
\pgfsetdash{}{0pt}%
\pgfsys@defobject{currentmarker}{\pgfqpoint{-0.022222in}{-0.022222in}}{\pgfqpoint{0.022222in}{0.022222in}}{%
\pgfpathmoveto{\pgfqpoint{0.000000in}{-0.022222in}}%
\pgfpathcurveto{\pgfqpoint{0.005893in}{-0.022222in}}{\pgfqpoint{0.011546in}{-0.019881in}}{\pgfqpoint{0.015713in}{-0.015713in}}%
\pgfpathcurveto{\pgfqpoint{0.019881in}{-0.011546in}}{\pgfqpoint{0.022222in}{-0.005893in}}{\pgfqpoint{0.022222in}{0.000000in}}%
\pgfpathcurveto{\pgfqpoint{0.022222in}{0.005893in}}{\pgfqpoint{0.019881in}{0.011546in}}{\pgfqpoint{0.015713in}{0.015713in}}%
\pgfpathcurveto{\pgfqpoint{0.011546in}{0.019881in}}{\pgfqpoint{0.005893in}{0.022222in}}{\pgfqpoint{0.000000in}{0.022222in}}%
\pgfpathcurveto{\pgfqpoint{-0.005893in}{0.022222in}}{\pgfqpoint{-0.011546in}{0.019881in}}{\pgfqpoint{-0.015713in}{0.015713in}}%
\pgfpathcurveto{\pgfqpoint{-0.019881in}{0.011546in}}{\pgfqpoint{-0.022222in}{0.005893in}}{\pgfqpoint{-0.022222in}{0.000000in}}%
\pgfpathcurveto{\pgfqpoint{-0.022222in}{-0.005893in}}{\pgfqpoint{-0.019881in}{-0.011546in}}{\pgfqpoint{-0.015713in}{-0.015713in}}%
\pgfpathcurveto{\pgfqpoint{-0.011546in}{-0.019881in}}{\pgfqpoint{-0.005893in}{-0.022222in}}{\pgfqpoint{0.000000in}{-0.022222in}}%
\pgfpathclose%
\pgfusepath{stroke,fill}%
}%
\begin{pgfscope}%
\pgfsys@transformshift{4.762625in}{4.908152in}%
\pgfsys@useobject{currentmarker}{}%
\end{pgfscope}%
\end{pgfscope}%
\begin{pgfscope}%
\pgfpathrectangle{\pgfqpoint{0.100000in}{2.413063in}}{\pgfqpoint{5.037500in}{3.427208in}}%
\pgfusepath{clip}%
\pgfsetrectcap%
\pgfsetroundjoin%
\pgfsetlinewidth{1.505625pt}%
\definecolor{currentstroke}{rgb}{0.000000,0.000000,1.000000}%
\pgfsetstrokecolor{currentstroke}%
\pgfsetstrokeopacity{0.500000}%
\pgfsetdash{}{0pt}%
\pgfpathmoveto{\pgfqpoint{3.763378in}{4.717009in}}%
\pgfusepath{stroke}%
\end{pgfscope}%
\begin{pgfscope}%
\pgfpathrectangle{\pgfqpoint{0.100000in}{2.413063in}}{\pgfqpoint{5.037500in}{3.427208in}}%
\pgfusepath{clip}%
\pgfsetbuttcap%
\pgfsetroundjoin%
\definecolor{currentfill}{rgb}{0.000000,0.000000,1.000000}%
\pgfsetfillcolor{currentfill}%
\pgfsetfillopacity{0.500000}%
\pgfsetlinewidth{0.250937pt}%
\definecolor{currentstroke}{rgb}{0.000000,0.000000,0.000000}%
\pgfsetstrokecolor{currentstroke}%
\pgfsetstrokeopacity{0.500000}%
\pgfsetdash{}{0pt}%
\pgfsys@defobject{currentmarker}{\pgfqpoint{-0.036111in}{-0.036111in}}{\pgfqpoint{0.036111in}{0.036111in}}{%
\pgfpathmoveto{\pgfqpoint{0.000000in}{-0.036111in}}%
\pgfpathcurveto{\pgfqpoint{0.009577in}{-0.036111in}}{\pgfqpoint{0.018763in}{-0.032306in}}{\pgfqpoint{0.025534in}{-0.025534in}}%
\pgfpathcurveto{\pgfqpoint{0.032306in}{-0.018763in}}{\pgfqpoint{0.036111in}{-0.009577in}}{\pgfqpoint{0.036111in}{0.000000in}}%
\pgfpathcurveto{\pgfqpoint{0.036111in}{0.009577in}}{\pgfqpoint{0.032306in}{0.018763in}}{\pgfqpoint{0.025534in}{0.025534in}}%
\pgfpathcurveto{\pgfqpoint{0.018763in}{0.032306in}}{\pgfqpoint{0.009577in}{0.036111in}}{\pgfqpoint{0.000000in}{0.036111in}}%
\pgfpathcurveto{\pgfqpoint{-0.009577in}{0.036111in}}{\pgfqpoint{-0.018763in}{0.032306in}}{\pgfqpoint{-0.025534in}{0.025534in}}%
\pgfpathcurveto{\pgfqpoint{-0.032306in}{0.018763in}}{\pgfqpoint{-0.036111in}{0.009577in}}{\pgfqpoint{-0.036111in}{0.000000in}}%
\pgfpathcurveto{\pgfqpoint{-0.036111in}{-0.009577in}}{\pgfqpoint{-0.032306in}{-0.018763in}}{\pgfqpoint{-0.025534in}{-0.025534in}}%
\pgfpathcurveto{\pgfqpoint{-0.018763in}{-0.032306in}}{\pgfqpoint{-0.009577in}{-0.036111in}}{\pgfqpoint{0.000000in}{-0.036111in}}%
\pgfpathclose%
\pgfusepath{stroke,fill}%
}%
\begin{pgfscope}%
\pgfsys@transformshift{3.763378in}{4.717009in}%
\pgfsys@useobject{currentmarker}{}%
\end{pgfscope}%
\end{pgfscope}%
\begin{pgfscope}%
\pgfpathrectangle{\pgfqpoint{0.100000in}{2.413063in}}{\pgfqpoint{5.037500in}{3.427208in}}%
\pgfusepath{clip}%
\pgfsetrectcap%
\pgfsetroundjoin%
\pgfsetlinewidth{1.505625pt}%
\definecolor{currentstroke}{rgb}{0.000000,0.000000,1.000000}%
\pgfsetstrokecolor{currentstroke}%
\pgfsetstrokeopacity{0.500000}%
\pgfsetdash{}{0pt}%
\pgfpathmoveto{\pgfqpoint{3.639727in}{4.708532in}}%
\pgfusepath{stroke}%
\end{pgfscope}%
\begin{pgfscope}%
\pgfpathrectangle{\pgfqpoint{0.100000in}{2.413063in}}{\pgfqpoint{5.037500in}{3.427208in}}%
\pgfusepath{clip}%
\pgfsetbuttcap%
\pgfsetroundjoin%
\definecolor{currentfill}{rgb}{0.000000,0.000000,1.000000}%
\pgfsetfillcolor{currentfill}%
\pgfsetfillopacity{0.500000}%
\pgfsetlinewidth{0.250937pt}%
\definecolor{currentstroke}{rgb}{0.000000,0.000000,0.000000}%
\pgfsetstrokecolor{currentstroke}%
\pgfsetstrokeopacity{0.500000}%
\pgfsetdash{}{0pt}%
\pgfsys@defobject{currentmarker}{\pgfqpoint{-0.058333in}{-0.058333in}}{\pgfqpoint{0.058333in}{0.058333in}}{%
\pgfpathmoveto{\pgfqpoint{0.000000in}{-0.058333in}}%
\pgfpathcurveto{\pgfqpoint{0.015470in}{-0.058333in}}{\pgfqpoint{0.030309in}{-0.052187in}}{\pgfqpoint{0.041248in}{-0.041248in}}%
\pgfpathcurveto{\pgfqpoint{0.052187in}{-0.030309in}}{\pgfqpoint{0.058333in}{-0.015470in}}{\pgfqpoint{0.058333in}{0.000000in}}%
\pgfpathcurveto{\pgfqpoint{0.058333in}{0.015470in}}{\pgfqpoint{0.052187in}{0.030309in}}{\pgfqpoint{0.041248in}{0.041248in}}%
\pgfpathcurveto{\pgfqpoint{0.030309in}{0.052187in}}{\pgfqpoint{0.015470in}{0.058333in}}{\pgfqpoint{0.000000in}{0.058333in}}%
\pgfpathcurveto{\pgfqpoint{-0.015470in}{0.058333in}}{\pgfqpoint{-0.030309in}{0.052187in}}{\pgfqpoint{-0.041248in}{0.041248in}}%
\pgfpathcurveto{\pgfqpoint{-0.052187in}{0.030309in}}{\pgfqpoint{-0.058333in}{0.015470in}}{\pgfqpoint{-0.058333in}{0.000000in}}%
\pgfpathcurveto{\pgfqpoint{-0.058333in}{-0.015470in}}{\pgfqpoint{-0.052187in}{-0.030309in}}{\pgfqpoint{-0.041248in}{-0.041248in}}%
\pgfpathcurveto{\pgfqpoint{-0.030309in}{-0.052187in}}{\pgfqpoint{-0.015470in}{-0.058333in}}{\pgfqpoint{0.000000in}{-0.058333in}}%
\pgfpathclose%
\pgfusepath{stroke,fill}%
}%
\begin{pgfscope}%
\pgfsys@transformshift{3.639727in}{4.708532in}%
\pgfsys@useobject{currentmarker}{}%
\end{pgfscope}%
\end{pgfscope}%
\begin{pgfscope}%
\pgfpathrectangle{\pgfqpoint{0.100000in}{2.413063in}}{\pgfqpoint{5.037500in}{3.427208in}}%
\pgfusepath{clip}%
\pgfsetrectcap%
\pgfsetroundjoin%
\pgfsetlinewidth{1.505625pt}%
\definecolor{currentstroke}{rgb}{0.000000,0.000000,1.000000}%
\pgfsetstrokecolor{currentstroke}%
\pgfsetstrokeopacity{0.500000}%
\pgfsetdash{}{0pt}%
\pgfpathmoveto{\pgfqpoint{3.731354in}{4.867293in}}%
\pgfusepath{stroke}%
\end{pgfscope}%
\begin{pgfscope}%
\pgfpathrectangle{\pgfqpoint{0.100000in}{2.413063in}}{\pgfqpoint{5.037500in}{3.427208in}}%
\pgfusepath{clip}%
\pgfsetbuttcap%
\pgfsetroundjoin%
\definecolor{currentfill}{rgb}{0.000000,0.000000,1.000000}%
\pgfsetfillcolor{currentfill}%
\pgfsetfillopacity{0.500000}%
\pgfsetlinewidth{0.250937pt}%
\definecolor{currentstroke}{rgb}{0.000000,0.000000,0.000000}%
\pgfsetstrokecolor{currentstroke}%
\pgfsetstrokeopacity{0.500000}%
\pgfsetdash{}{0pt}%
\pgfsys@defobject{currentmarker}{\pgfqpoint{-0.038889in}{-0.038889in}}{\pgfqpoint{0.038889in}{0.038889in}}{%
\pgfpathmoveto{\pgfqpoint{0.000000in}{-0.038889in}}%
\pgfpathcurveto{\pgfqpoint{0.010313in}{-0.038889in}}{\pgfqpoint{0.020206in}{-0.034791in}}{\pgfqpoint{0.027499in}{-0.027499in}}%
\pgfpathcurveto{\pgfqpoint{0.034791in}{-0.020206in}}{\pgfqpoint{0.038889in}{-0.010313in}}{\pgfqpoint{0.038889in}{0.000000in}}%
\pgfpathcurveto{\pgfqpoint{0.038889in}{0.010313in}}{\pgfqpoint{0.034791in}{0.020206in}}{\pgfqpoint{0.027499in}{0.027499in}}%
\pgfpathcurveto{\pgfqpoint{0.020206in}{0.034791in}}{\pgfqpoint{0.010313in}{0.038889in}}{\pgfqpoint{0.000000in}{0.038889in}}%
\pgfpathcurveto{\pgfqpoint{-0.010313in}{0.038889in}}{\pgfqpoint{-0.020206in}{0.034791in}}{\pgfqpoint{-0.027499in}{0.027499in}}%
\pgfpathcurveto{\pgfqpoint{-0.034791in}{0.020206in}}{\pgfqpoint{-0.038889in}{0.010313in}}{\pgfqpoint{-0.038889in}{0.000000in}}%
\pgfpathcurveto{\pgfqpoint{-0.038889in}{-0.010313in}}{\pgfqpoint{-0.034791in}{-0.020206in}}{\pgfqpoint{-0.027499in}{-0.027499in}}%
\pgfpathcurveto{\pgfqpoint{-0.020206in}{-0.034791in}}{\pgfqpoint{-0.010313in}{-0.038889in}}{\pgfqpoint{0.000000in}{-0.038889in}}%
\pgfpathclose%
\pgfusepath{stroke,fill}%
}%
\begin{pgfscope}%
\pgfsys@transformshift{3.731354in}{4.867293in}%
\pgfsys@useobject{currentmarker}{}%
\end{pgfscope}%
\end{pgfscope}%
\begin{pgfscope}%
\pgfpathrectangle{\pgfqpoint{0.100000in}{2.413063in}}{\pgfqpoint{5.037500in}{3.427208in}}%
\pgfusepath{clip}%
\pgfsetrectcap%
\pgfsetroundjoin%
\pgfsetlinewidth{1.505625pt}%
\definecolor{currentstroke}{rgb}{0.000000,0.000000,1.000000}%
\pgfsetstrokecolor{currentstroke}%
\pgfsetstrokeopacity{0.500000}%
\pgfsetdash{}{0pt}%
\pgfpathmoveto{\pgfqpoint{3.819255in}{4.733545in}}%
\pgfusepath{stroke}%
\end{pgfscope}%
\begin{pgfscope}%
\pgfpathrectangle{\pgfqpoint{0.100000in}{2.413063in}}{\pgfqpoint{5.037500in}{3.427208in}}%
\pgfusepath{clip}%
\pgfsetbuttcap%
\pgfsetroundjoin%
\definecolor{currentfill}{rgb}{0.000000,0.000000,1.000000}%
\pgfsetfillcolor{currentfill}%
\pgfsetfillopacity{0.500000}%
\pgfsetlinewidth{0.250937pt}%
\definecolor{currentstroke}{rgb}{0.000000,0.000000,0.000000}%
\pgfsetstrokecolor{currentstroke}%
\pgfsetstrokeopacity{0.500000}%
\pgfsetdash{}{0pt}%
\pgfsys@defobject{currentmarker}{\pgfqpoint{-0.016667in}{-0.016667in}}{\pgfqpoint{0.016667in}{0.016667in}}{%
\pgfpathmoveto{\pgfqpoint{0.000000in}{-0.016667in}}%
\pgfpathcurveto{\pgfqpoint{0.004420in}{-0.016667in}}{\pgfqpoint{0.008660in}{-0.014911in}}{\pgfqpoint{0.011785in}{-0.011785in}}%
\pgfpathcurveto{\pgfqpoint{0.014911in}{-0.008660in}}{\pgfqpoint{0.016667in}{-0.004420in}}{\pgfqpoint{0.016667in}{0.000000in}}%
\pgfpathcurveto{\pgfqpoint{0.016667in}{0.004420in}}{\pgfqpoint{0.014911in}{0.008660in}}{\pgfqpoint{0.011785in}{0.011785in}}%
\pgfpathcurveto{\pgfqpoint{0.008660in}{0.014911in}}{\pgfqpoint{0.004420in}{0.016667in}}{\pgfqpoint{0.000000in}{0.016667in}}%
\pgfpathcurveto{\pgfqpoint{-0.004420in}{0.016667in}}{\pgfqpoint{-0.008660in}{0.014911in}}{\pgfqpoint{-0.011785in}{0.011785in}}%
\pgfpathcurveto{\pgfqpoint{-0.014911in}{0.008660in}}{\pgfqpoint{-0.016667in}{0.004420in}}{\pgfqpoint{-0.016667in}{0.000000in}}%
\pgfpathcurveto{\pgfqpoint{-0.016667in}{-0.004420in}}{\pgfqpoint{-0.014911in}{-0.008660in}}{\pgfqpoint{-0.011785in}{-0.011785in}}%
\pgfpathcurveto{\pgfqpoint{-0.008660in}{-0.014911in}}{\pgfqpoint{-0.004420in}{-0.016667in}}{\pgfqpoint{0.000000in}{-0.016667in}}%
\pgfpathclose%
\pgfusepath{stroke,fill}%
}%
\begin{pgfscope}%
\pgfsys@transformshift{3.819255in}{4.733545in}%
\pgfsys@useobject{currentmarker}{}%
\end{pgfscope}%
\end{pgfscope}%
\begin{pgfscope}%
\pgfpathrectangle{\pgfqpoint{0.100000in}{2.413063in}}{\pgfqpoint{5.037500in}{3.427208in}}%
\pgfusepath{clip}%
\pgfsetrectcap%
\pgfsetroundjoin%
\pgfsetlinewidth{1.505625pt}%
\definecolor{currentstroke}{rgb}{0.000000,0.000000,1.000000}%
\pgfsetstrokecolor{currentstroke}%
\pgfsetstrokeopacity{0.500000}%
\pgfsetdash{}{0pt}%
\pgfpathmoveto{\pgfqpoint{3.753328in}{4.802843in}}%
\pgfusepath{stroke}%
\end{pgfscope}%
\begin{pgfscope}%
\pgfpathrectangle{\pgfqpoint{0.100000in}{2.413063in}}{\pgfqpoint{5.037500in}{3.427208in}}%
\pgfusepath{clip}%
\pgfsetbuttcap%
\pgfsetroundjoin%
\definecolor{currentfill}{rgb}{0.000000,0.000000,1.000000}%
\pgfsetfillcolor{currentfill}%
\pgfsetfillopacity{0.500000}%
\pgfsetlinewidth{0.250937pt}%
\definecolor{currentstroke}{rgb}{0.000000,0.000000,0.000000}%
\pgfsetstrokecolor{currentstroke}%
\pgfsetstrokeopacity{0.500000}%
\pgfsetdash{}{0pt}%
\pgfsys@defobject{currentmarker}{\pgfqpoint{-0.066667in}{-0.066667in}}{\pgfqpoint{0.066667in}{0.066667in}}{%
\pgfpathmoveto{\pgfqpoint{0.000000in}{-0.066667in}}%
\pgfpathcurveto{\pgfqpoint{0.017680in}{-0.066667in}}{\pgfqpoint{0.034639in}{-0.059642in}}{\pgfqpoint{0.047140in}{-0.047140in}}%
\pgfpathcurveto{\pgfqpoint{0.059642in}{-0.034639in}}{\pgfqpoint{0.066667in}{-0.017680in}}{\pgfqpoint{0.066667in}{0.000000in}}%
\pgfpathcurveto{\pgfqpoint{0.066667in}{0.017680in}}{\pgfqpoint{0.059642in}{0.034639in}}{\pgfqpoint{0.047140in}{0.047140in}}%
\pgfpathcurveto{\pgfqpoint{0.034639in}{0.059642in}}{\pgfqpoint{0.017680in}{0.066667in}}{\pgfqpoint{0.000000in}{0.066667in}}%
\pgfpathcurveto{\pgfqpoint{-0.017680in}{0.066667in}}{\pgfqpoint{-0.034639in}{0.059642in}}{\pgfqpoint{-0.047140in}{0.047140in}}%
\pgfpathcurveto{\pgfqpoint{-0.059642in}{0.034639in}}{\pgfqpoint{-0.066667in}{0.017680in}}{\pgfqpoint{-0.066667in}{0.000000in}}%
\pgfpathcurveto{\pgfqpoint{-0.066667in}{-0.017680in}}{\pgfqpoint{-0.059642in}{-0.034639in}}{\pgfqpoint{-0.047140in}{-0.047140in}}%
\pgfpathcurveto{\pgfqpoint{-0.034639in}{-0.059642in}}{\pgfqpoint{-0.017680in}{-0.066667in}}{\pgfqpoint{0.000000in}{-0.066667in}}%
\pgfpathclose%
\pgfusepath{stroke,fill}%
}%
\begin{pgfscope}%
\pgfsys@transformshift{3.753328in}{4.802843in}%
\pgfsys@useobject{currentmarker}{}%
\end{pgfscope}%
\end{pgfscope}%
\begin{pgfscope}%
\pgfpathrectangle{\pgfqpoint{0.100000in}{2.413063in}}{\pgfqpoint{5.037500in}{3.427208in}}%
\pgfusepath{clip}%
\pgfsetrectcap%
\pgfsetroundjoin%
\pgfsetlinewidth{1.505625pt}%
\definecolor{currentstroke}{rgb}{0.000000,0.000000,1.000000}%
\pgfsetstrokecolor{currentstroke}%
\pgfsetstrokeopacity{0.500000}%
\pgfsetdash{}{0pt}%
\pgfpathmoveto{\pgfqpoint{3.590943in}{4.778086in}}%
\pgfusepath{stroke}%
\end{pgfscope}%
\begin{pgfscope}%
\pgfpathrectangle{\pgfqpoint{0.100000in}{2.413063in}}{\pgfqpoint{5.037500in}{3.427208in}}%
\pgfusepath{clip}%
\pgfsetbuttcap%
\pgfsetroundjoin%
\definecolor{currentfill}{rgb}{0.000000,0.000000,1.000000}%
\pgfsetfillcolor{currentfill}%
\pgfsetfillopacity{0.500000}%
\pgfsetlinewidth{0.250937pt}%
\definecolor{currentstroke}{rgb}{0.000000,0.000000,0.000000}%
\pgfsetstrokecolor{currentstroke}%
\pgfsetstrokeopacity{0.500000}%
\pgfsetdash{}{0pt}%
\pgfsys@defobject{currentmarker}{\pgfqpoint{-0.036111in}{-0.036111in}}{\pgfqpoint{0.036111in}{0.036111in}}{%
\pgfpathmoveto{\pgfqpoint{0.000000in}{-0.036111in}}%
\pgfpathcurveto{\pgfqpoint{0.009577in}{-0.036111in}}{\pgfqpoint{0.018763in}{-0.032306in}}{\pgfqpoint{0.025534in}{-0.025534in}}%
\pgfpathcurveto{\pgfqpoint{0.032306in}{-0.018763in}}{\pgfqpoint{0.036111in}{-0.009577in}}{\pgfqpoint{0.036111in}{0.000000in}}%
\pgfpathcurveto{\pgfqpoint{0.036111in}{0.009577in}}{\pgfqpoint{0.032306in}{0.018763in}}{\pgfqpoint{0.025534in}{0.025534in}}%
\pgfpathcurveto{\pgfqpoint{0.018763in}{0.032306in}}{\pgfqpoint{0.009577in}{0.036111in}}{\pgfqpoint{0.000000in}{0.036111in}}%
\pgfpathcurveto{\pgfqpoint{-0.009577in}{0.036111in}}{\pgfqpoint{-0.018763in}{0.032306in}}{\pgfqpoint{-0.025534in}{0.025534in}}%
\pgfpathcurveto{\pgfqpoint{-0.032306in}{0.018763in}}{\pgfqpoint{-0.036111in}{0.009577in}}{\pgfqpoint{-0.036111in}{0.000000in}}%
\pgfpathcurveto{\pgfqpoint{-0.036111in}{-0.009577in}}{\pgfqpoint{-0.032306in}{-0.018763in}}{\pgfqpoint{-0.025534in}{-0.025534in}}%
\pgfpathcurveto{\pgfqpoint{-0.018763in}{-0.032306in}}{\pgfqpoint{-0.009577in}{-0.036111in}}{\pgfqpoint{0.000000in}{-0.036111in}}%
\pgfpathclose%
\pgfusepath{stroke,fill}%
}%
\begin{pgfscope}%
\pgfsys@transformshift{3.590943in}{4.778086in}%
\pgfsys@useobject{currentmarker}{}%
\end{pgfscope}%
\end{pgfscope}%
\begin{pgfscope}%
\pgfpathrectangle{\pgfqpoint{0.100000in}{2.413063in}}{\pgfqpoint{5.037500in}{3.427208in}}%
\pgfusepath{clip}%
\pgfsetrectcap%
\pgfsetroundjoin%
\pgfsetlinewidth{1.505625pt}%
\definecolor{currentstroke}{rgb}{0.000000,0.000000,1.000000}%
\pgfsetstrokecolor{currentstroke}%
\pgfsetstrokeopacity{0.500000}%
\pgfsetdash{}{0pt}%
\pgfpathmoveto{\pgfqpoint{3.706510in}{4.707536in}}%
\pgfusepath{stroke}%
\end{pgfscope}%
\begin{pgfscope}%
\pgfpathrectangle{\pgfqpoint{0.100000in}{2.413063in}}{\pgfqpoint{5.037500in}{3.427208in}}%
\pgfusepath{clip}%
\pgfsetbuttcap%
\pgfsetroundjoin%
\definecolor{currentfill}{rgb}{0.000000,0.000000,1.000000}%
\pgfsetfillcolor{currentfill}%
\pgfsetfillopacity{0.500000}%
\pgfsetlinewidth{0.250937pt}%
\definecolor{currentstroke}{rgb}{0.000000,0.000000,0.000000}%
\pgfsetstrokecolor{currentstroke}%
\pgfsetstrokeopacity{0.500000}%
\pgfsetdash{}{0pt}%
\pgfsys@defobject{currentmarker}{\pgfqpoint{-0.047222in}{-0.047222in}}{\pgfqpoint{0.047222in}{0.047222in}}{%
\pgfpathmoveto{\pgfqpoint{0.000000in}{-0.047222in}}%
\pgfpathcurveto{\pgfqpoint{0.012523in}{-0.047222in}}{\pgfqpoint{0.024536in}{-0.042247in}}{\pgfqpoint{0.033391in}{-0.033391in}}%
\pgfpathcurveto{\pgfqpoint{0.042247in}{-0.024536in}}{\pgfqpoint{0.047222in}{-0.012523in}}{\pgfqpoint{0.047222in}{0.000000in}}%
\pgfpathcurveto{\pgfqpoint{0.047222in}{0.012523in}}{\pgfqpoint{0.042247in}{0.024536in}}{\pgfqpoint{0.033391in}{0.033391in}}%
\pgfpathcurveto{\pgfqpoint{0.024536in}{0.042247in}}{\pgfqpoint{0.012523in}{0.047222in}}{\pgfqpoint{0.000000in}{0.047222in}}%
\pgfpathcurveto{\pgfqpoint{-0.012523in}{0.047222in}}{\pgfqpoint{-0.024536in}{0.042247in}}{\pgfqpoint{-0.033391in}{0.033391in}}%
\pgfpathcurveto{\pgfqpoint{-0.042247in}{0.024536in}}{\pgfqpoint{-0.047222in}{0.012523in}}{\pgfqpoint{-0.047222in}{0.000000in}}%
\pgfpathcurveto{\pgfqpoint{-0.047222in}{-0.012523in}}{\pgfqpoint{-0.042247in}{-0.024536in}}{\pgfqpoint{-0.033391in}{-0.033391in}}%
\pgfpathcurveto{\pgfqpoint{-0.024536in}{-0.042247in}}{\pgfqpoint{-0.012523in}{-0.047222in}}{\pgfqpoint{0.000000in}{-0.047222in}}%
\pgfpathclose%
\pgfusepath{stroke,fill}%
}%
\begin{pgfscope}%
\pgfsys@transformshift{3.706510in}{4.707536in}%
\pgfsys@useobject{currentmarker}{}%
\end{pgfscope}%
\end{pgfscope}%
\begin{pgfscope}%
\pgfpathrectangle{\pgfqpoint{0.100000in}{2.413063in}}{\pgfqpoint{5.037500in}{3.427208in}}%
\pgfusepath{clip}%
\pgfsetrectcap%
\pgfsetroundjoin%
\pgfsetlinewidth{1.505625pt}%
\definecolor{currentstroke}{rgb}{0.000000,0.000000,1.000000}%
\pgfsetstrokecolor{currentstroke}%
\pgfsetstrokeopacity{0.500000}%
\pgfsetdash{}{0pt}%
\pgfpathmoveto{\pgfqpoint{3.605727in}{4.701726in}}%
\pgfusepath{stroke}%
\end{pgfscope}%
\begin{pgfscope}%
\pgfpathrectangle{\pgfqpoint{0.100000in}{2.413063in}}{\pgfqpoint{5.037500in}{3.427208in}}%
\pgfusepath{clip}%
\pgfsetbuttcap%
\pgfsetroundjoin%
\definecolor{currentfill}{rgb}{0.000000,0.000000,1.000000}%
\pgfsetfillcolor{currentfill}%
\pgfsetfillopacity{0.500000}%
\pgfsetlinewidth{0.250937pt}%
\definecolor{currentstroke}{rgb}{0.000000,0.000000,0.000000}%
\pgfsetstrokecolor{currentstroke}%
\pgfsetstrokeopacity{0.500000}%
\pgfsetdash{}{0pt}%
\pgfsys@defobject{currentmarker}{\pgfqpoint{-0.036111in}{-0.036111in}}{\pgfqpoint{0.036111in}{0.036111in}}{%
\pgfpathmoveto{\pgfqpoint{0.000000in}{-0.036111in}}%
\pgfpathcurveto{\pgfqpoint{0.009577in}{-0.036111in}}{\pgfqpoint{0.018763in}{-0.032306in}}{\pgfqpoint{0.025534in}{-0.025534in}}%
\pgfpathcurveto{\pgfqpoint{0.032306in}{-0.018763in}}{\pgfqpoint{0.036111in}{-0.009577in}}{\pgfqpoint{0.036111in}{0.000000in}}%
\pgfpathcurveto{\pgfqpoint{0.036111in}{0.009577in}}{\pgfqpoint{0.032306in}{0.018763in}}{\pgfqpoint{0.025534in}{0.025534in}}%
\pgfpathcurveto{\pgfqpoint{0.018763in}{0.032306in}}{\pgfqpoint{0.009577in}{0.036111in}}{\pgfqpoint{0.000000in}{0.036111in}}%
\pgfpathcurveto{\pgfqpoint{-0.009577in}{0.036111in}}{\pgfqpoint{-0.018763in}{0.032306in}}{\pgfqpoint{-0.025534in}{0.025534in}}%
\pgfpathcurveto{\pgfqpoint{-0.032306in}{0.018763in}}{\pgfqpoint{-0.036111in}{0.009577in}}{\pgfqpoint{-0.036111in}{0.000000in}}%
\pgfpathcurveto{\pgfqpoint{-0.036111in}{-0.009577in}}{\pgfqpoint{-0.032306in}{-0.018763in}}{\pgfqpoint{-0.025534in}{-0.025534in}}%
\pgfpathcurveto{\pgfqpoint{-0.018763in}{-0.032306in}}{\pgfqpoint{-0.009577in}{-0.036111in}}{\pgfqpoint{0.000000in}{-0.036111in}}%
\pgfpathclose%
\pgfusepath{stroke,fill}%
}%
\begin{pgfscope}%
\pgfsys@transformshift{3.605727in}{4.701726in}%
\pgfsys@useobject{currentmarker}{}%
\end{pgfscope}%
\end{pgfscope}%
\begin{pgfscope}%
\pgfpathrectangle{\pgfqpoint{0.100000in}{2.413063in}}{\pgfqpoint{5.037500in}{3.427208in}}%
\pgfusepath{clip}%
\pgfsetrectcap%
\pgfsetroundjoin%
\pgfsetlinewidth{1.505625pt}%
\definecolor{currentstroke}{rgb}{0.000000,0.000000,1.000000}%
\pgfsetstrokecolor{currentstroke}%
\pgfsetstrokeopacity{0.500000}%
\pgfsetdash{}{0pt}%
\pgfpathmoveto{\pgfqpoint{3.687384in}{4.761983in}}%
\pgfusepath{stroke}%
\end{pgfscope}%
\begin{pgfscope}%
\pgfpathrectangle{\pgfqpoint{0.100000in}{2.413063in}}{\pgfqpoint{5.037500in}{3.427208in}}%
\pgfusepath{clip}%
\pgfsetbuttcap%
\pgfsetroundjoin%
\definecolor{currentfill}{rgb}{0.000000,0.000000,1.000000}%
\pgfsetfillcolor{currentfill}%
\pgfsetfillopacity{0.500000}%
\pgfsetlinewidth{0.250937pt}%
\definecolor{currentstroke}{rgb}{0.000000,0.000000,0.000000}%
\pgfsetstrokecolor{currentstroke}%
\pgfsetstrokeopacity{0.500000}%
\pgfsetdash{}{0pt}%
\pgfsys@defobject{currentmarker}{\pgfqpoint{-0.036111in}{-0.036111in}}{\pgfqpoint{0.036111in}{0.036111in}}{%
\pgfpathmoveto{\pgfqpoint{0.000000in}{-0.036111in}}%
\pgfpathcurveto{\pgfqpoint{0.009577in}{-0.036111in}}{\pgfqpoint{0.018763in}{-0.032306in}}{\pgfqpoint{0.025534in}{-0.025534in}}%
\pgfpathcurveto{\pgfqpoint{0.032306in}{-0.018763in}}{\pgfqpoint{0.036111in}{-0.009577in}}{\pgfqpoint{0.036111in}{0.000000in}}%
\pgfpathcurveto{\pgfqpoint{0.036111in}{0.009577in}}{\pgfqpoint{0.032306in}{0.018763in}}{\pgfqpoint{0.025534in}{0.025534in}}%
\pgfpathcurveto{\pgfqpoint{0.018763in}{0.032306in}}{\pgfqpoint{0.009577in}{0.036111in}}{\pgfqpoint{0.000000in}{0.036111in}}%
\pgfpathcurveto{\pgfqpoint{-0.009577in}{0.036111in}}{\pgfqpoint{-0.018763in}{0.032306in}}{\pgfqpoint{-0.025534in}{0.025534in}}%
\pgfpathcurveto{\pgfqpoint{-0.032306in}{0.018763in}}{\pgfqpoint{-0.036111in}{0.009577in}}{\pgfqpoint{-0.036111in}{0.000000in}}%
\pgfpathcurveto{\pgfqpoint{-0.036111in}{-0.009577in}}{\pgfqpoint{-0.032306in}{-0.018763in}}{\pgfqpoint{-0.025534in}{-0.025534in}}%
\pgfpathcurveto{\pgfqpoint{-0.018763in}{-0.032306in}}{\pgfqpoint{-0.009577in}{-0.036111in}}{\pgfqpoint{0.000000in}{-0.036111in}}%
\pgfpathclose%
\pgfusepath{stroke,fill}%
}%
\begin{pgfscope}%
\pgfsys@transformshift{3.687384in}{4.761983in}%
\pgfsys@useobject{currentmarker}{}%
\end{pgfscope}%
\end{pgfscope}%
\begin{pgfscope}%
\pgfpathrectangle{\pgfqpoint{0.100000in}{2.413063in}}{\pgfqpoint{5.037500in}{3.427208in}}%
\pgfusepath{clip}%
\pgfsetrectcap%
\pgfsetroundjoin%
\pgfsetlinewidth{1.505625pt}%
\definecolor{currentstroke}{rgb}{0.000000,0.000000,1.000000}%
\pgfsetstrokecolor{currentstroke}%
\pgfsetstrokeopacity{0.500000}%
\pgfsetdash{}{0pt}%
\pgfpathmoveto{\pgfqpoint{3.701307in}{4.866114in}}%
\pgfusepath{stroke}%
\end{pgfscope}%
\begin{pgfscope}%
\pgfpathrectangle{\pgfqpoint{0.100000in}{2.413063in}}{\pgfqpoint{5.037500in}{3.427208in}}%
\pgfusepath{clip}%
\pgfsetbuttcap%
\pgfsetroundjoin%
\definecolor{currentfill}{rgb}{0.000000,0.000000,1.000000}%
\pgfsetfillcolor{currentfill}%
\pgfsetfillopacity{0.500000}%
\pgfsetlinewidth{0.250937pt}%
\definecolor{currentstroke}{rgb}{0.000000,0.000000,0.000000}%
\pgfsetstrokecolor{currentstroke}%
\pgfsetstrokeopacity{0.500000}%
\pgfsetdash{}{0pt}%
\pgfsys@defobject{currentmarker}{\pgfqpoint{-0.022222in}{-0.022222in}}{\pgfqpoint{0.022222in}{0.022222in}}{%
\pgfpathmoveto{\pgfqpoint{0.000000in}{-0.022222in}}%
\pgfpathcurveto{\pgfqpoint{0.005893in}{-0.022222in}}{\pgfqpoint{0.011546in}{-0.019881in}}{\pgfqpoint{0.015713in}{-0.015713in}}%
\pgfpathcurveto{\pgfqpoint{0.019881in}{-0.011546in}}{\pgfqpoint{0.022222in}{-0.005893in}}{\pgfqpoint{0.022222in}{0.000000in}}%
\pgfpathcurveto{\pgfqpoint{0.022222in}{0.005893in}}{\pgfqpoint{0.019881in}{0.011546in}}{\pgfqpoint{0.015713in}{0.015713in}}%
\pgfpathcurveto{\pgfqpoint{0.011546in}{0.019881in}}{\pgfqpoint{0.005893in}{0.022222in}}{\pgfqpoint{0.000000in}{0.022222in}}%
\pgfpathcurveto{\pgfqpoint{-0.005893in}{0.022222in}}{\pgfqpoint{-0.011546in}{0.019881in}}{\pgfqpoint{-0.015713in}{0.015713in}}%
\pgfpathcurveto{\pgfqpoint{-0.019881in}{0.011546in}}{\pgfqpoint{-0.022222in}{0.005893in}}{\pgfqpoint{-0.022222in}{0.000000in}}%
\pgfpathcurveto{\pgfqpoint{-0.022222in}{-0.005893in}}{\pgfqpoint{-0.019881in}{-0.011546in}}{\pgfqpoint{-0.015713in}{-0.015713in}}%
\pgfpathcurveto{\pgfqpoint{-0.011546in}{-0.019881in}}{\pgfqpoint{-0.005893in}{-0.022222in}}{\pgfqpoint{0.000000in}{-0.022222in}}%
\pgfpathclose%
\pgfusepath{stroke,fill}%
}%
\begin{pgfscope}%
\pgfsys@transformshift{3.701307in}{4.866114in}%
\pgfsys@useobject{currentmarker}{}%
\end{pgfscope}%
\end{pgfscope}%
\begin{pgfscope}%
\pgfpathrectangle{\pgfqpoint{0.100000in}{2.413063in}}{\pgfqpoint{5.037500in}{3.427208in}}%
\pgfusepath{clip}%
\pgfsetrectcap%
\pgfsetroundjoin%
\pgfsetlinewidth{1.505625pt}%
\definecolor{currentstroke}{rgb}{0.000000,0.000000,1.000000}%
\pgfsetstrokecolor{currentstroke}%
\pgfsetstrokeopacity{0.500000}%
\pgfsetdash{}{0pt}%
\pgfpathmoveto{\pgfqpoint{3.796884in}{4.680144in}}%
\pgfusepath{stroke}%
\end{pgfscope}%
\begin{pgfscope}%
\pgfpathrectangle{\pgfqpoint{0.100000in}{2.413063in}}{\pgfqpoint{5.037500in}{3.427208in}}%
\pgfusepath{clip}%
\pgfsetbuttcap%
\pgfsetroundjoin%
\definecolor{currentfill}{rgb}{0.000000,0.000000,1.000000}%
\pgfsetfillcolor{currentfill}%
\pgfsetfillopacity{0.500000}%
\pgfsetlinewidth{0.250937pt}%
\definecolor{currentstroke}{rgb}{0.000000,0.000000,0.000000}%
\pgfsetstrokecolor{currentstroke}%
\pgfsetstrokeopacity{0.500000}%
\pgfsetdash{}{0pt}%
\pgfsys@defobject{currentmarker}{\pgfqpoint{-0.041667in}{-0.041667in}}{\pgfqpoint{0.041667in}{0.041667in}}{%
\pgfpathmoveto{\pgfqpoint{0.000000in}{-0.041667in}}%
\pgfpathcurveto{\pgfqpoint{0.011050in}{-0.041667in}}{\pgfqpoint{0.021649in}{-0.037276in}}{\pgfqpoint{0.029463in}{-0.029463in}}%
\pgfpathcurveto{\pgfqpoint{0.037276in}{-0.021649in}}{\pgfqpoint{0.041667in}{-0.011050in}}{\pgfqpoint{0.041667in}{0.000000in}}%
\pgfpathcurveto{\pgfqpoint{0.041667in}{0.011050in}}{\pgfqpoint{0.037276in}{0.021649in}}{\pgfqpoint{0.029463in}{0.029463in}}%
\pgfpathcurveto{\pgfqpoint{0.021649in}{0.037276in}}{\pgfqpoint{0.011050in}{0.041667in}}{\pgfqpoint{0.000000in}{0.041667in}}%
\pgfpathcurveto{\pgfqpoint{-0.011050in}{0.041667in}}{\pgfqpoint{-0.021649in}{0.037276in}}{\pgfqpoint{-0.029463in}{0.029463in}}%
\pgfpathcurveto{\pgfqpoint{-0.037276in}{0.021649in}}{\pgfqpoint{-0.041667in}{0.011050in}}{\pgfqpoint{-0.041667in}{0.000000in}}%
\pgfpathcurveto{\pgfqpoint{-0.041667in}{-0.011050in}}{\pgfqpoint{-0.037276in}{-0.021649in}}{\pgfqpoint{-0.029463in}{-0.029463in}}%
\pgfpathcurveto{\pgfqpoint{-0.021649in}{-0.037276in}}{\pgfqpoint{-0.011050in}{-0.041667in}}{\pgfqpoint{0.000000in}{-0.041667in}}%
\pgfpathclose%
\pgfusepath{stroke,fill}%
}%
\begin{pgfscope}%
\pgfsys@transformshift{3.796884in}{4.680144in}%
\pgfsys@useobject{currentmarker}{}%
\end{pgfscope}%
\end{pgfscope}%
\begin{pgfscope}%
\pgfpathrectangle{\pgfqpoint{0.100000in}{2.413063in}}{\pgfqpoint{5.037500in}{3.427208in}}%
\pgfusepath{clip}%
\pgfsetrectcap%
\pgfsetroundjoin%
\pgfsetlinewidth{1.505625pt}%
\definecolor{currentstroke}{rgb}{0.000000,0.000000,1.000000}%
\pgfsetstrokecolor{currentstroke}%
\pgfsetstrokeopacity{0.500000}%
\pgfsetdash{}{0pt}%
\pgfpathmoveto{\pgfqpoint{3.539146in}{4.804347in}}%
\pgfusepath{stroke}%
\end{pgfscope}%
\begin{pgfscope}%
\pgfpathrectangle{\pgfqpoint{0.100000in}{2.413063in}}{\pgfqpoint{5.037500in}{3.427208in}}%
\pgfusepath{clip}%
\pgfsetbuttcap%
\pgfsetroundjoin%
\definecolor{currentfill}{rgb}{0.000000,0.000000,1.000000}%
\pgfsetfillcolor{currentfill}%
\pgfsetfillopacity{0.500000}%
\pgfsetlinewidth{0.250937pt}%
\definecolor{currentstroke}{rgb}{0.000000,0.000000,0.000000}%
\pgfsetstrokecolor{currentstroke}%
\pgfsetstrokeopacity{0.500000}%
\pgfsetdash{}{0pt}%
\pgfsys@defobject{currentmarker}{\pgfqpoint{-0.072222in}{-0.072222in}}{\pgfqpoint{0.072222in}{0.072222in}}{%
\pgfpathmoveto{\pgfqpoint{0.000000in}{-0.072222in}}%
\pgfpathcurveto{\pgfqpoint{0.019154in}{-0.072222in}}{\pgfqpoint{0.037525in}{-0.064612in}}{\pgfqpoint{0.051069in}{-0.051069in}}%
\pgfpathcurveto{\pgfqpoint{0.064612in}{-0.037525in}}{\pgfqpoint{0.072222in}{-0.019154in}}{\pgfqpoint{0.072222in}{0.000000in}}%
\pgfpathcurveto{\pgfqpoint{0.072222in}{0.019154in}}{\pgfqpoint{0.064612in}{0.037525in}}{\pgfqpoint{0.051069in}{0.051069in}}%
\pgfpathcurveto{\pgfqpoint{0.037525in}{0.064612in}}{\pgfqpoint{0.019154in}{0.072222in}}{\pgfqpoint{0.000000in}{0.072222in}}%
\pgfpathcurveto{\pgfqpoint{-0.019154in}{0.072222in}}{\pgfqpoint{-0.037525in}{0.064612in}}{\pgfqpoint{-0.051069in}{0.051069in}}%
\pgfpathcurveto{\pgfqpoint{-0.064612in}{0.037525in}}{\pgfqpoint{-0.072222in}{0.019154in}}{\pgfqpoint{-0.072222in}{0.000000in}}%
\pgfpathcurveto{\pgfqpoint{-0.072222in}{-0.019154in}}{\pgfqpoint{-0.064612in}{-0.037525in}}{\pgfqpoint{-0.051069in}{-0.051069in}}%
\pgfpathcurveto{\pgfqpoint{-0.037525in}{-0.064612in}}{\pgfqpoint{-0.019154in}{-0.072222in}}{\pgfqpoint{0.000000in}{-0.072222in}}%
\pgfpathclose%
\pgfusepath{stroke,fill}%
}%
\begin{pgfscope}%
\pgfsys@transformshift{3.539146in}{4.804347in}%
\pgfsys@useobject{currentmarker}{}%
\end{pgfscope}%
\end{pgfscope}%
\begin{pgfscope}%
\pgfpathrectangle{\pgfqpoint{0.100000in}{2.413063in}}{\pgfqpoint{5.037500in}{3.427208in}}%
\pgfusepath{clip}%
\pgfsetrectcap%
\pgfsetroundjoin%
\pgfsetlinewidth{1.505625pt}%
\definecolor{currentstroke}{rgb}{0.000000,0.000000,1.000000}%
\pgfsetstrokecolor{currentstroke}%
\pgfsetstrokeopacity{0.500000}%
\pgfsetdash{}{0pt}%
\pgfpathmoveto{\pgfqpoint{3.554230in}{4.643033in}}%
\pgfusepath{stroke}%
\end{pgfscope}%
\begin{pgfscope}%
\pgfpathrectangle{\pgfqpoint{0.100000in}{2.413063in}}{\pgfqpoint{5.037500in}{3.427208in}}%
\pgfusepath{clip}%
\pgfsetbuttcap%
\pgfsetroundjoin%
\definecolor{currentfill}{rgb}{0.000000,0.000000,1.000000}%
\pgfsetfillcolor{currentfill}%
\pgfsetfillopacity{0.500000}%
\pgfsetlinewidth{0.250937pt}%
\definecolor{currentstroke}{rgb}{0.000000,0.000000,0.000000}%
\pgfsetstrokecolor{currentstroke}%
\pgfsetstrokeopacity{0.500000}%
\pgfsetdash{}{0pt}%
\pgfsys@defobject{currentmarker}{\pgfqpoint{-0.036111in}{-0.036111in}}{\pgfqpoint{0.036111in}{0.036111in}}{%
\pgfpathmoveto{\pgfqpoint{0.000000in}{-0.036111in}}%
\pgfpathcurveto{\pgfqpoint{0.009577in}{-0.036111in}}{\pgfqpoint{0.018763in}{-0.032306in}}{\pgfqpoint{0.025534in}{-0.025534in}}%
\pgfpathcurveto{\pgfqpoint{0.032306in}{-0.018763in}}{\pgfqpoint{0.036111in}{-0.009577in}}{\pgfqpoint{0.036111in}{0.000000in}}%
\pgfpathcurveto{\pgfqpoint{0.036111in}{0.009577in}}{\pgfqpoint{0.032306in}{0.018763in}}{\pgfqpoint{0.025534in}{0.025534in}}%
\pgfpathcurveto{\pgfqpoint{0.018763in}{0.032306in}}{\pgfqpoint{0.009577in}{0.036111in}}{\pgfqpoint{0.000000in}{0.036111in}}%
\pgfpathcurveto{\pgfqpoint{-0.009577in}{0.036111in}}{\pgfqpoint{-0.018763in}{0.032306in}}{\pgfqpoint{-0.025534in}{0.025534in}}%
\pgfpathcurveto{\pgfqpoint{-0.032306in}{0.018763in}}{\pgfqpoint{-0.036111in}{0.009577in}}{\pgfqpoint{-0.036111in}{0.000000in}}%
\pgfpathcurveto{\pgfqpoint{-0.036111in}{-0.009577in}}{\pgfqpoint{-0.032306in}{-0.018763in}}{\pgfqpoint{-0.025534in}{-0.025534in}}%
\pgfpathcurveto{\pgfqpoint{-0.018763in}{-0.032306in}}{\pgfqpoint{-0.009577in}{-0.036111in}}{\pgfqpoint{0.000000in}{-0.036111in}}%
\pgfpathclose%
\pgfusepath{stroke,fill}%
}%
\begin{pgfscope}%
\pgfsys@transformshift{3.554230in}{4.643033in}%
\pgfsys@useobject{currentmarker}{}%
\end{pgfscope}%
\end{pgfscope}%
\begin{pgfscope}%
\pgfpathrectangle{\pgfqpoint{0.100000in}{2.413063in}}{\pgfqpoint{5.037500in}{3.427208in}}%
\pgfusepath{clip}%
\pgfsetrectcap%
\pgfsetroundjoin%
\pgfsetlinewidth{1.505625pt}%
\definecolor{currentstroke}{rgb}{0.000000,0.000000,1.000000}%
\pgfsetstrokecolor{currentstroke}%
\pgfsetstrokeopacity{0.500000}%
\pgfsetdash{}{0pt}%
\pgfpathmoveto{\pgfqpoint{3.728805in}{4.846680in}}%
\pgfusepath{stroke}%
\end{pgfscope}%
\begin{pgfscope}%
\pgfpathrectangle{\pgfqpoint{0.100000in}{2.413063in}}{\pgfqpoint{5.037500in}{3.427208in}}%
\pgfusepath{clip}%
\pgfsetbuttcap%
\pgfsetroundjoin%
\definecolor{currentfill}{rgb}{0.000000,0.000000,1.000000}%
\pgfsetfillcolor{currentfill}%
\pgfsetfillopacity{0.500000}%
\pgfsetlinewidth{0.250937pt}%
\definecolor{currentstroke}{rgb}{0.000000,0.000000,0.000000}%
\pgfsetstrokecolor{currentstroke}%
\pgfsetstrokeopacity{0.500000}%
\pgfsetdash{}{0pt}%
\pgfsys@defobject{currentmarker}{\pgfqpoint{-0.052778in}{-0.052778in}}{\pgfqpoint{0.052778in}{0.052778in}}{%
\pgfpathmoveto{\pgfqpoint{0.000000in}{-0.052778in}}%
\pgfpathcurveto{\pgfqpoint{0.013997in}{-0.052778in}}{\pgfqpoint{0.027422in}{-0.047217in}}{\pgfqpoint{0.037320in}{-0.037320in}}%
\pgfpathcurveto{\pgfqpoint{0.047217in}{-0.027422in}}{\pgfqpoint{0.052778in}{-0.013997in}}{\pgfqpoint{0.052778in}{0.000000in}}%
\pgfpathcurveto{\pgfqpoint{0.052778in}{0.013997in}}{\pgfqpoint{0.047217in}{0.027422in}}{\pgfqpoint{0.037320in}{0.037320in}}%
\pgfpathcurveto{\pgfqpoint{0.027422in}{0.047217in}}{\pgfqpoint{0.013997in}{0.052778in}}{\pgfqpoint{0.000000in}{0.052778in}}%
\pgfpathcurveto{\pgfqpoint{-0.013997in}{0.052778in}}{\pgfqpoint{-0.027422in}{0.047217in}}{\pgfqpoint{-0.037320in}{0.037320in}}%
\pgfpathcurveto{\pgfqpoint{-0.047217in}{0.027422in}}{\pgfqpoint{-0.052778in}{0.013997in}}{\pgfqpoint{-0.052778in}{0.000000in}}%
\pgfpathcurveto{\pgfqpoint{-0.052778in}{-0.013997in}}{\pgfqpoint{-0.047217in}{-0.027422in}}{\pgfqpoint{-0.037320in}{-0.037320in}}%
\pgfpathcurveto{\pgfqpoint{-0.027422in}{-0.047217in}}{\pgfqpoint{-0.013997in}{-0.052778in}}{\pgfqpoint{0.000000in}{-0.052778in}}%
\pgfpathclose%
\pgfusepath{stroke,fill}%
}%
\begin{pgfscope}%
\pgfsys@transformshift{3.728805in}{4.846680in}%
\pgfsys@useobject{currentmarker}{}%
\end{pgfscope}%
\end{pgfscope}%
\begin{pgfscope}%
\pgfpathrectangle{\pgfqpoint{0.100000in}{2.413063in}}{\pgfqpoint{5.037500in}{3.427208in}}%
\pgfusepath{clip}%
\pgfsetrectcap%
\pgfsetroundjoin%
\pgfsetlinewidth{1.505625pt}%
\definecolor{currentstroke}{rgb}{0.678431,1.000000,0.184314}%
\pgfsetstrokecolor{currentstroke}%
\pgfsetstrokeopacity{0.500000}%
\pgfsetdash{}{0pt}%
\pgfpathmoveto{\pgfqpoint{3.032951in}{5.182346in}}%
\pgfusepath{stroke}%
\end{pgfscope}%
\begin{pgfscope}%
\pgfpathrectangle{\pgfqpoint{0.100000in}{2.413063in}}{\pgfqpoint{5.037500in}{3.427208in}}%
\pgfusepath{clip}%
\pgfsetbuttcap%
\pgfsetroundjoin%
\definecolor{currentfill}{rgb}{0.678431,1.000000,0.184314}%
\pgfsetfillcolor{currentfill}%
\pgfsetfillopacity{0.500000}%
\pgfsetlinewidth{0.250937pt}%
\definecolor{currentstroke}{rgb}{0.000000,0.000000,0.000000}%
\pgfsetstrokecolor{currentstroke}%
\pgfsetstrokeopacity{0.500000}%
\pgfsetdash{}{0pt}%
\pgfsys@defobject{currentmarker}{\pgfqpoint{-0.022222in}{-0.022222in}}{\pgfqpoint{0.022222in}{0.022222in}}{%
\pgfpathmoveto{\pgfqpoint{0.000000in}{-0.022222in}}%
\pgfpathcurveto{\pgfqpoint{0.005893in}{-0.022222in}}{\pgfqpoint{0.011546in}{-0.019881in}}{\pgfqpoint{0.015713in}{-0.015713in}}%
\pgfpathcurveto{\pgfqpoint{0.019881in}{-0.011546in}}{\pgfqpoint{0.022222in}{-0.005893in}}{\pgfqpoint{0.022222in}{0.000000in}}%
\pgfpathcurveto{\pgfqpoint{0.022222in}{0.005893in}}{\pgfqpoint{0.019881in}{0.011546in}}{\pgfqpoint{0.015713in}{0.015713in}}%
\pgfpathcurveto{\pgfqpoint{0.011546in}{0.019881in}}{\pgfqpoint{0.005893in}{0.022222in}}{\pgfqpoint{0.000000in}{0.022222in}}%
\pgfpathcurveto{\pgfqpoint{-0.005893in}{0.022222in}}{\pgfqpoint{-0.011546in}{0.019881in}}{\pgfqpoint{-0.015713in}{0.015713in}}%
\pgfpathcurveto{\pgfqpoint{-0.019881in}{0.011546in}}{\pgfqpoint{-0.022222in}{0.005893in}}{\pgfqpoint{-0.022222in}{0.000000in}}%
\pgfpathcurveto{\pgfqpoint{-0.022222in}{-0.005893in}}{\pgfqpoint{-0.019881in}{-0.011546in}}{\pgfqpoint{-0.015713in}{-0.015713in}}%
\pgfpathcurveto{\pgfqpoint{-0.011546in}{-0.019881in}}{\pgfqpoint{-0.005893in}{-0.022222in}}{\pgfqpoint{0.000000in}{-0.022222in}}%
\pgfpathclose%
\pgfusepath{stroke,fill}%
}%
\begin{pgfscope}%
\pgfsys@transformshift{3.032951in}{5.182346in}%
\pgfsys@useobject{currentmarker}{}%
\end{pgfscope}%
\end{pgfscope}%
\begin{pgfscope}%
\pgfpathrectangle{\pgfqpoint{0.100000in}{2.413063in}}{\pgfqpoint{5.037500in}{3.427208in}}%
\pgfusepath{clip}%
\pgfsetrectcap%
\pgfsetroundjoin%
\pgfsetlinewidth{1.505625pt}%
\definecolor{currentstroke}{rgb}{0.501961,0.501961,0.501961}%
\pgfsetstrokecolor{currentstroke}%
\pgfsetstrokeopacity{0.500000}%
\pgfsetdash{}{0pt}%
\pgfpathmoveto{\pgfqpoint{2.887116in}{4.876746in}}%
\pgfusepath{stroke}%
\end{pgfscope}%
\begin{pgfscope}%
\pgfpathrectangle{\pgfqpoint{0.100000in}{2.413063in}}{\pgfqpoint{5.037500in}{3.427208in}}%
\pgfusepath{clip}%
\pgfsetbuttcap%
\pgfsetroundjoin%
\definecolor{currentfill}{rgb}{0.501961,0.501961,0.501961}%
\pgfsetfillcolor{currentfill}%
\pgfsetfillopacity{0.500000}%
\pgfsetlinewidth{0.250937pt}%
\definecolor{currentstroke}{rgb}{0.000000,0.000000,0.000000}%
\pgfsetstrokecolor{currentstroke}%
\pgfsetstrokeopacity{0.500000}%
\pgfsetdash{}{0pt}%
\pgfsys@defobject{currentmarker}{\pgfqpoint{-0.013889in}{-0.013889in}}{\pgfqpoint{0.013889in}{0.013889in}}{%
\pgfpathmoveto{\pgfqpoint{0.000000in}{-0.013889in}}%
\pgfpathcurveto{\pgfqpoint{0.003683in}{-0.013889in}}{\pgfqpoint{0.007216in}{-0.012425in}}{\pgfqpoint{0.009821in}{-0.009821in}}%
\pgfpathcurveto{\pgfqpoint{0.012425in}{-0.007216in}}{\pgfqpoint{0.013889in}{-0.003683in}}{\pgfqpoint{0.013889in}{0.000000in}}%
\pgfpathcurveto{\pgfqpoint{0.013889in}{0.003683in}}{\pgfqpoint{0.012425in}{0.007216in}}{\pgfqpoint{0.009821in}{0.009821in}}%
\pgfpathcurveto{\pgfqpoint{0.007216in}{0.012425in}}{\pgfqpoint{0.003683in}{0.013889in}}{\pgfqpoint{0.000000in}{0.013889in}}%
\pgfpathcurveto{\pgfqpoint{-0.003683in}{0.013889in}}{\pgfqpoint{-0.007216in}{0.012425in}}{\pgfqpoint{-0.009821in}{0.009821in}}%
\pgfpathcurveto{\pgfqpoint{-0.012425in}{0.007216in}}{\pgfqpoint{-0.013889in}{0.003683in}}{\pgfqpoint{-0.013889in}{0.000000in}}%
\pgfpathcurveto{\pgfqpoint{-0.013889in}{-0.003683in}}{\pgfqpoint{-0.012425in}{-0.007216in}}{\pgfqpoint{-0.009821in}{-0.009821in}}%
\pgfpathcurveto{\pgfqpoint{-0.007216in}{-0.012425in}}{\pgfqpoint{-0.003683in}{-0.013889in}}{\pgfqpoint{0.000000in}{-0.013889in}}%
\pgfpathclose%
\pgfusepath{stroke,fill}%
}%
\begin{pgfscope}%
\pgfsys@transformshift{2.887116in}{4.876746in}%
\pgfsys@useobject{currentmarker}{}%
\end{pgfscope}%
\end{pgfscope}%
\begin{pgfscope}%
\pgfpathrectangle{\pgfqpoint{0.100000in}{2.413063in}}{\pgfqpoint{5.037500in}{3.427208in}}%
\pgfusepath{clip}%
\pgfsetrectcap%
\pgfsetroundjoin%
\pgfsetlinewidth{1.505625pt}%
\definecolor{currentstroke}{rgb}{0.678431,1.000000,0.184314}%
\pgfsetstrokecolor{currentstroke}%
\pgfsetstrokeopacity{0.500000}%
\pgfsetdash{}{0pt}%
\pgfpathmoveto{\pgfqpoint{2.946176in}{4.971799in}}%
\pgfusepath{stroke}%
\end{pgfscope}%
\begin{pgfscope}%
\pgfpathrectangle{\pgfqpoint{0.100000in}{2.413063in}}{\pgfqpoint{5.037500in}{3.427208in}}%
\pgfusepath{clip}%
\pgfsetbuttcap%
\pgfsetroundjoin%
\definecolor{currentfill}{rgb}{0.678431,1.000000,0.184314}%
\pgfsetfillcolor{currentfill}%
\pgfsetfillopacity{0.500000}%
\pgfsetlinewidth{0.250937pt}%
\definecolor{currentstroke}{rgb}{0.000000,0.000000,0.000000}%
\pgfsetstrokecolor{currentstroke}%
\pgfsetstrokeopacity{0.500000}%
\pgfsetdash{}{0pt}%
\pgfsys@defobject{currentmarker}{\pgfqpoint{-0.008333in}{-0.008333in}}{\pgfqpoint{0.008333in}{0.008333in}}{%
\pgfpathmoveto{\pgfqpoint{0.000000in}{-0.008333in}}%
\pgfpathcurveto{\pgfqpoint{0.002210in}{-0.008333in}}{\pgfqpoint{0.004330in}{-0.007455in}}{\pgfqpoint{0.005893in}{-0.005893in}}%
\pgfpathcurveto{\pgfqpoint{0.007455in}{-0.004330in}}{\pgfqpoint{0.008333in}{-0.002210in}}{\pgfqpoint{0.008333in}{0.000000in}}%
\pgfpathcurveto{\pgfqpoint{0.008333in}{0.002210in}}{\pgfqpoint{0.007455in}{0.004330in}}{\pgfqpoint{0.005893in}{0.005893in}}%
\pgfpathcurveto{\pgfqpoint{0.004330in}{0.007455in}}{\pgfqpoint{0.002210in}{0.008333in}}{\pgfqpoint{0.000000in}{0.008333in}}%
\pgfpathcurveto{\pgfqpoint{-0.002210in}{0.008333in}}{\pgfqpoint{-0.004330in}{0.007455in}}{\pgfqpoint{-0.005893in}{0.005893in}}%
\pgfpathcurveto{\pgfqpoint{-0.007455in}{0.004330in}}{\pgfqpoint{-0.008333in}{0.002210in}}{\pgfqpoint{-0.008333in}{0.000000in}}%
\pgfpathcurveto{\pgfqpoint{-0.008333in}{-0.002210in}}{\pgfqpoint{-0.007455in}{-0.004330in}}{\pgfqpoint{-0.005893in}{-0.005893in}}%
\pgfpathcurveto{\pgfqpoint{-0.004330in}{-0.007455in}}{\pgfqpoint{-0.002210in}{-0.008333in}}{\pgfqpoint{0.000000in}{-0.008333in}}%
\pgfpathclose%
\pgfusepath{stroke,fill}%
}%
\begin{pgfscope}%
\pgfsys@transformshift{2.946176in}{4.971799in}%
\pgfsys@useobject{currentmarker}{}%
\end{pgfscope}%
\end{pgfscope}%
\begin{pgfscope}%
\pgfpathrectangle{\pgfqpoint{0.100000in}{2.413063in}}{\pgfqpoint{5.037500in}{3.427208in}}%
\pgfusepath{clip}%
\pgfsetrectcap%
\pgfsetroundjoin%
\pgfsetlinewidth{1.505625pt}%
\definecolor{currentstroke}{rgb}{0.678431,1.000000,0.184314}%
\pgfsetstrokecolor{currentstroke}%
\pgfsetstrokeopacity{0.500000}%
\pgfsetdash{}{0pt}%
\pgfpathmoveto{\pgfqpoint{3.014958in}{4.863032in}}%
\pgfusepath{stroke}%
\end{pgfscope}%
\begin{pgfscope}%
\pgfpathrectangle{\pgfqpoint{0.100000in}{2.413063in}}{\pgfqpoint{5.037500in}{3.427208in}}%
\pgfusepath{clip}%
\pgfsetbuttcap%
\pgfsetroundjoin%
\definecolor{currentfill}{rgb}{0.678431,1.000000,0.184314}%
\pgfsetfillcolor{currentfill}%
\pgfsetfillopacity{0.500000}%
\pgfsetlinewidth{0.250937pt}%
\definecolor{currentstroke}{rgb}{0.000000,0.000000,0.000000}%
\pgfsetstrokecolor{currentstroke}%
\pgfsetstrokeopacity{0.500000}%
\pgfsetdash{}{0pt}%
\pgfsys@defobject{currentmarker}{\pgfqpoint{-0.011111in}{-0.011111in}}{\pgfqpoint{0.011111in}{0.011111in}}{%
\pgfpathmoveto{\pgfqpoint{0.000000in}{-0.011111in}}%
\pgfpathcurveto{\pgfqpoint{0.002947in}{-0.011111in}}{\pgfqpoint{0.005773in}{-0.009940in}}{\pgfqpoint{0.007857in}{-0.007857in}}%
\pgfpathcurveto{\pgfqpoint{0.009940in}{-0.005773in}}{\pgfqpoint{0.011111in}{-0.002947in}}{\pgfqpoint{0.011111in}{0.000000in}}%
\pgfpathcurveto{\pgfqpoint{0.011111in}{0.002947in}}{\pgfqpoint{0.009940in}{0.005773in}}{\pgfqpoint{0.007857in}{0.007857in}}%
\pgfpathcurveto{\pgfqpoint{0.005773in}{0.009940in}}{\pgfqpoint{0.002947in}{0.011111in}}{\pgfqpoint{0.000000in}{0.011111in}}%
\pgfpathcurveto{\pgfqpoint{-0.002947in}{0.011111in}}{\pgfqpoint{-0.005773in}{0.009940in}}{\pgfqpoint{-0.007857in}{0.007857in}}%
\pgfpathcurveto{\pgfqpoint{-0.009940in}{0.005773in}}{\pgfqpoint{-0.011111in}{0.002947in}}{\pgfqpoint{-0.011111in}{0.000000in}}%
\pgfpathcurveto{\pgfqpoint{-0.011111in}{-0.002947in}}{\pgfqpoint{-0.009940in}{-0.005773in}}{\pgfqpoint{-0.007857in}{-0.007857in}}%
\pgfpathcurveto{\pgfqpoint{-0.005773in}{-0.009940in}}{\pgfqpoint{-0.002947in}{-0.011111in}}{\pgfqpoint{0.000000in}{-0.011111in}}%
\pgfpathclose%
\pgfusepath{stroke,fill}%
}%
\begin{pgfscope}%
\pgfsys@transformshift{3.014958in}{4.863032in}%
\pgfsys@useobject{currentmarker}{}%
\end{pgfscope}%
\end{pgfscope}%
\begin{pgfscope}%
\pgfpathrectangle{\pgfqpoint{0.100000in}{2.413063in}}{\pgfqpoint{5.037500in}{3.427208in}}%
\pgfusepath{clip}%
\pgfsetrectcap%
\pgfsetroundjoin%
\pgfsetlinewidth{1.505625pt}%
\definecolor{currentstroke}{rgb}{0.678431,1.000000,0.184314}%
\pgfsetstrokecolor{currentstroke}%
\pgfsetstrokeopacity{0.500000}%
\pgfsetdash{}{0pt}%
\pgfpathmoveto{\pgfqpoint{2.871956in}{5.038443in}}%
\pgfusepath{stroke}%
\end{pgfscope}%
\begin{pgfscope}%
\pgfpathrectangle{\pgfqpoint{0.100000in}{2.413063in}}{\pgfqpoint{5.037500in}{3.427208in}}%
\pgfusepath{clip}%
\pgfsetbuttcap%
\pgfsetroundjoin%
\definecolor{currentfill}{rgb}{0.678431,1.000000,0.184314}%
\pgfsetfillcolor{currentfill}%
\pgfsetfillopacity{0.500000}%
\pgfsetlinewidth{0.250937pt}%
\definecolor{currentstroke}{rgb}{0.000000,0.000000,0.000000}%
\pgfsetstrokecolor{currentstroke}%
\pgfsetstrokeopacity{0.500000}%
\pgfsetdash{}{0pt}%
\pgfsys@defobject{currentmarker}{\pgfqpoint{-0.011111in}{-0.011111in}}{\pgfqpoint{0.011111in}{0.011111in}}{%
\pgfpathmoveto{\pgfqpoint{0.000000in}{-0.011111in}}%
\pgfpathcurveto{\pgfqpoint{0.002947in}{-0.011111in}}{\pgfqpoint{0.005773in}{-0.009940in}}{\pgfqpoint{0.007857in}{-0.007857in}}%
\pgfpathcurveto{\pgfqpoint{0.009940in}{-0.005773in}}{\pgfqpoint{0.011111in}{-0.002947in}}{\pgfqpoint{0.011111in}{0.000000in}}%
\pgfpathcurveto{\pgfqpoint{0.011111in}{0.002947in}}{\pgfqpoint{0.009940in}{0.005773in}}{\pgfqpoint{0.007857in}{0.007857in}}%
\pgfpathcurveto{\pgfqpoint{0.005773in}{0.009940in}}{\pgfqpoint{0.002947in}{0.011111in}}{\pgfqpoint{0.000000in}{0.011111in}}%
\pgfpathcurveto{\pgfqpoint{-0.002947in}{0.011111in}}{\pgfqpoint{-0.005773in}{0.009940in}}{\pgfqpoint{-0.007857in}{0.007857in}}%
\pgfpathcurveto{\pgfqpoint{-0.009940in}{0.005773in}}{\pgfqpoint{-0.011111in}{0.002947in}}{\pgfqpoint{-0.011111in}{0.000000in}}%
\pgfpathcurveto{\pgfqpoint{-0.011111in}{-0.002947in}}{\pgfqpoint{-0.009940in}{-0.005773in}}{\pgfqpoint{-0.007857in}{-0.007857in}}%
\pgfpathcurveto{\pgfqpoint{-0.005773in}{-0.009940in}}{\pgfqpoint{-0.002947in}{-0.011111in}}{\pgfqpoint{0.000000in}{-0.011111in}}%
\pgfpathclose%
\pgfusepath{stroke,fill}%
}%
\begin{pgfscope}%
\pgfsys@transformshift{2.871956in}{5.038443in}%
\pgfsys@useobject{currentmarker}{}%
\end{pgfscope}%
\end{pgfscope}%
\begin{pgfscope}%
\pgfpathrectangle{\pgfqpoint{0.100000in}{2.413063in}}{\pgfqpoint{5.037500in}{3.427208in}}%
\pgfusepath{clip}%
\pgfsetrectcap%
\pgfsetroundjoin%
\pgfsetlinewidth{1.505625pt}%
\definecolor{currentstroke}{rgb}{0.678431,1.000000,0.184314}%
\pgfsetstrokecolor{currentstroke}%
\pgfsetstrokeopacity{0.500000}%
\pgfsetdash{}{0pt}%
\pgfpathmoveto{\pgfqpoint{3.397181in}{3.302431in}}%
\pgfusepath{stroke}%
\end{pgfscope}%
\begin{pgfscope}%
\pgfpathrectangle{\pgfqpoint{0.100000in}{2.413063in}}{\pgfqpoint{5.037500in}{3.427208in}}%
\pgfusepath{clip}%
\pgfsetbuttcap%
\pgfsetroundjoin%
\definecolor{currentfill}{rgb}{0.678431,1.000000,0.184314}%
\pgfsetfillcolor{currentfill}%
\pgfsetfillopacity{0.500000}%
\pgfsetlinewidth{0.250937pt}%
\definecolor{currentstroke}{rgb}{0.000000,0.000000,0.000000}%
\pgfsetstrokecolor{currentstroke}%
\pgfsetstrokeopacity{0.500000}%
\pgfsetdash{}{0pt}%
\pgfsys@defobject{currentmarker}{\pgfqpoint{-0.055556in}{-0.055556in}}{\pgfqpoint{0.055556in}{0.055556in}}{%
\pgfpathmoveto{\pgfqpoint{0.000000in}{-0.055556in}}%
\pgfpathcurveto{\pgfqpoint{0.014734in}{-0.055556in}}{\pgfqpoint{0.028866in}{-0.049702in}}{\pgfqpoint{0.039284in}{-0.039284in}}%
\pgfpathcurveto{\pgfqpoint{0.049702in}{-0.028866in}}{\pgfqpoint{0.055556in}{-0.014734in}}{\pgfqpoint{0.055556in}{0.000000in}}%
\pgfpathcurveto{\pgfqpoint{0.055556in}{0.014734in}}{\pgfqpoint{0.049702in}{0.028866in}}{\pgfqpoint{0.039284in}{0.039284in}}%
\pgfpathcurveto{\pgfqpoint{0.028866in}{0.049702in}}{\pgfqpoint{0.014734in}{0.055556in}}{\pgfqpoint{0.000000in}{0.055556in}}%
\pgfpathcurveto{\pgfqpoint{-0.014734in}{0.055556in}}{\pgfqpoint{-0.028866in}{0.049702in}}{\pgfqpoint{-0.039284in}{0.039284in}}%
\pgfpathcurveto{\pgfqpoint{-0.049702in}{0.028866in}}{\pgfqpoint{-0.055556in}{0.014734in}}{\pgfqpoint{-0.055556in}{0.000000in}}%
\pgfpathcurveto{\pgfqpoint{-0.055556in}{-0.014734in}}{\pgfqpoint{-0.049702in}{-0.028866in}}{\pgfqpoint{-0.039284in}{-0.039284in}}%
\pgfpathcurveto{\pgfqpoint{-0.028866in}{-0.049702in}}{\pgfqpoint{-0.014734in}{-0.055556in}}{\pgfqpoint{0.000000in}{-0.055556in}}%
\pgfpathclose%
\pgfusepath{stroke,fill}%
}%
\begin{pgfscope}%
\pgfsys@transformshift{3.397181in}{3.302431in}%
\pgfsys@useobject{currentmarker}{}%
\end{pgfscope}%
\end{pgfscope}%
\begin{pgfscope}%
\pgfpathrectangle{\pgfqpoint{0.100000in}{2.413063in}}{\pgfqpoint{5.037500in}{3.427208in}}%
\pgfusepath{clip}%
\pgfsetrectcap%
\pgfsetroundjoin%
\pgfsetlinewidth{1.505625pt}%
\definecolor{currentstroke}{rgb}{0.678431,1.000000,0.184314}%
\pgfsetstrokecolor{currentstroke}%
\pgfsetstrokeopacity{0.500000}%
\pgfsetdash{}{0pt}%
\pgfpathmoveto{\pgfqpoint{3.370369in}{3.412562in}}%
\pgfusepath{stroke}%
\end{pgfscope}%
\begin{pgfscope}%
\pgfpathrectangle{\pgfqpoint{0.100000in}{2.413063in}}{\pgfqpoint{5.037500in}{3.427208in}}%
\pgfusepath{clip}%
\pgfsetbuttcap%
\pgfsetroundjoin%
\definecolor{currentfill}{rgb}{0.678431,1.000000,0.184314}%
\pgfsetfillcolor{currentfill}%
\pgfsetfillopacity{0.500000}%
\pgfsetlinewidth{0.250937pt}%
\definecolor{currentstroke}{rgb}{0.000000,0.000000,0.000000}%
\pgfsetstrokecolor{currentstroke}%
\pgfsetstrokeopacity{0.500000}%
\pgfsetdash{}{0pt}%
\pgfsys@defobject{currentmarker}{\pgfqpoint{-0.044444in}{-0.044444in}}{\pgfqpoint{0.044444in}{0.044444in}}{%
\pgfpathmoveto{\pgfqpoint{0.000000in}{-0.044444in}}%
\pgfpathcurveto{\pgfqpoint{0.011787in}{-0.044444in}}{\pgfqpoint{0.023092in}{-0.039761in}}{\pgfqpoint{0.031427in}{-0.031427in}}%
\pgfpathcurveto{\pgfqpoint{0.039761in}{-0.023092in}}{\pgfqpoint{0.044444in}{-0.011787in}}{\pgfqpoint{0.044444in}{0.000000in}}%
\pgfpathcurveto{\pgfqpoint{0.044444in}{0.011787in}}{\pgfqpoint{0.039761in}{0.023092in}}{\pgfqpoint{0.031427in}{0.031427in}}%
\pgfpathcurveto{\pgfqpoint{0.023092in}{0.039761in}}{\pgfqpoint{0.011787in}{0.044444in}}{\pgfqpoint{0.000000in}{0.044444in}}%
\pgfpathcurveto{\pgfqpoint{-0.011787in}{0.044444in}}{\pgfqpoint{-0.023092in}{0.039761in}}{\pgfqpoint{-0.031427in}{0.031427in}}%
\pgfpathcurveto{\pgfqpoint{-0.039761in}{0.023092in}}{\pgfqpoint{-0.044444in}{0.011787in}}{\pgfqpoint{-0.044444in}{0.000000in}}%
\pgfpathcurveto{\pgfqpoint{-0.044444in}{-0.011787in}}{\pgfqpoint{-0.039761in}{-0.023092in}}{\pgfqpoint{-0.031427in}{-0.031427in}}%
\pgfpathcurveto{\pgfqpoint{-0.023092in}{-0.039761in}}{\pgfqpoint{-0.011787in}{-0.044444in}}{\pgfqpoint{0.000000in}{-0.044444in}}%
\pgfpathclose%
\pgfusepath{stroke,fill}%
}%
\begin{pgfscope}%
\pgfsys@transformshift{3.370369in}{3.412562in}%
\pgfsys@useobject{currentmarker}{}%
\end{pgfscope}%
\end{pgfscope}%
\begin{pgfscope}%
\pgfpathrectangle{\pgfqpoint{0.100000in}{2.413063in}}{\pgfqpoint{5.037500in}{3.427208in}}%
\pgfusepath{clip}%
\pgfsetrectcap%
\pgfsetroundjoin%
\pgfsetlinewidth{1.505625pt}%
\definecolor{currentstroke}{rgb}{0.678431,1.000000,0.184314}%
\pgfsetstrokecolor{currentstroke}%
\pgfsetstrokeopacity{0.500000}%
\pgfsetdash{}{0pt}%
\pgfpathmoveto{\pgfqpoint{3.277318in}{3.533073in}}%
\pgfusepath{stroke}%
\end{pgfscope}%
\begin{pgfscope}%
\pgfpathrectangle{\pgfqpoint{0.100000in}{2.413063in}}{\pgfqpoint{5.037500in}{3.427208in}}%
\pgfusepath{clip}%
\pgfsetbuttcap%
\pgfsetroundjoin%
\definecolor{currentfill}{rgb}{0.678431,1.000000,0.184314}%
\pgfsetfillcolor{currentfill}%
\pgfsetfillopacity{0.500000}%
\pgfsetlinewidth{0.250937pt}%
\definecolor{currentstroke}{rgb}{0.000000,0.000000,0.000000}%
\pgfsetstrokecolor{currentstroke}%
\pgfsetstrokeopacity{0.500000}%
\pgfsetdash{}{0pt}%
\pgfsys@defobject{currentmarker}{\pgfqpoint{-0.041667in}{-0.041667in}}{\pgfqpoint{0.041667in}{0.041667in}}{%
\pgfpathmoveto{\pgfqpoint{0.000000in}{-0.041667in}}%
\pgfpathcurveto{\pgfqpoint{0.011050in}{-0.041667in}}{\pgfqpoint{0.021649in}{-0.037276in}}{\pgfqpoint{0.029463in}{-0.029463in}}%
\pgfpathcurveto{\pgfqpoint{0.037276in}{-0.021649in}}{\pgfqpoint{0.041667in}{-0.011050in}}{\pgfqpoint{0.041667in}{0.000000in}}%
\pgfpathcurveto{\pgfqpoint{0.041667in}{0.011050in}}{\pgfqpoint{0.037276in}{0.021649in}}{\pgfqpoint{0.029463in}{0.029463in}}%
\pgfpathcurveto{\pgfqpoint{0.021649in}{0.037276in}}{\pgfqpoint{0.011050in}{0.041667in}}{\pgfqpoint{0.000000in}{0.041667in}}%
\pgfpathcurveto{\pgfqpoint{-0.011050in}{0.041667in}}{\pgfqpoint{-0.021649in}{0.037276in}}{\pgfqpoint{-0.029463in}{0.029463in}}%
\pgfpathcurveto{\pgfqpoint{-0.037276in}{0.021649in}}{\pgfqpoint{-0.041667in}{0.011050in}}{\pgfqpoint{-0.041667in}{0.000000in}}%
\pgfpathcurveto{\pgfqpoint{-0.041667in}{-0.011050in}}{\pgfqpoint{-0.037276in}{-0.021649in}}{\pgfqpoint{-0.029463in}{-0.029463in}}%
\pgfpathcurveto{\pgfqpoint{-0.021649in}{-0.037276in}}{\pgfqpoint{-0.011050in}{-0.041667in}}{\pgfqpoint{0.000000in}{-0.041667in}}%
\pgfpathclose%
\pgfusepath{stroke,fill}%
}%
\begin{pgfscope}%
\pgfsys@transformshift{3.277318in}{3.533073in}%
\pgfsys@useobject{currentmarker}{}%
\end{pgfscope}%
\end{pgfscope}%
\begin{pgfscope}%
\pgfpathrectangle{\pgfqpoint{0.100000in}{2.413063in}}{\pgfqpoint{5.037500in}{3.427208in}}%
\pgfusepath{clip}%
\pgfsetrectcap%
\pgfsetroundjoin%
\pgfsetlinewidth{1.505625pt}%
\definecolor{currentstroke}{rgb}{0.678431,1.000000,0.184314}%
\pgfsetstrokecolor{currentstroke}%
\pgfsetstrokeopacity{0.500000}%
\pgfsetdash{}{0pt}%
\pgfpathmoveto{\pgfqpoint{3.306143in}{4.100990in}}%
\pgfusepath{stroke}%
\end{pgfscope}%
\begin{pgfscope}%
\pgfpathrectangle{\pgfqpoint{0.100000in}{2.413063in}}{\pgfqpoint{5.037500in}{3.427208in}}%
\pgfusepath{clip}%
\pgfsetbuttcap%
\pgfsetroundjoin%
\definecolor{currentfill}{rgb}{0.678431,1.000000,0.184314}%
\pgfsetfillcolor{currentfill}%
\pgfsetfillopacity{0.500000}%
\pgfsetlinewidth{0.250937pt}%
\definecolor{currentstroke}{rgb}{0.000000,0.000000,0.000000}%
\pgfsetstrokecolor{currentstroke}%
\pgfsetstrokeopacity{0.500000}%
\pgfsetdash{}{0pt}%
\pgfsys@defobject{currentmarker}{\pgfqpoint{-0.019444in}{-0.019444in}}{\pgfqpoint{0.019444in}{0.019444in}}{%
\pgfpathmoveto{\pgfqpoint{0.000000in}{-0.019444in}}%
\pgfpathcurveto{\pgfqpoint{0.005157in}{-0.019444in}}{\pgfqpoint{0.010103in}{-0.017396in}}{\pgfqpoint{0.013749in}{-0.013749in}}%
\pgfpathcurveto{\pgfqpoint{0.017396in}{-0.010103in}}{\pgfqpoint{0.019444in}{-0.005157in}}{\pgfqpoint{0.019444in}{0.000000in}}%
\pgfpathcurveto{\pgfqpoint{0.019444in}{0.005157in}}{\pgfqpoint{0.017396in}{0.010103in}}{\pgfqpoint{0.013749in}{0.013749in}}%
\pgfpathcurveto{\pgfqpoint{0.010103in}{0.017396in}}{\pgfqpoint{0.005157in}{0.019444in}}{\pgfqpoint{0.000000in}{0.019444in}}%
\pgfpathcurveto{\pgfqpoint{-0.005157in}{0.019444in}}{\pgfqpoint{-0.010103in}{0.017396in}}{\pgfqpoint{-0.013749in}{0.013749in}}%
\pgfpathcurveto{\pgfqpoint{-0.017396in}{0.010103in}}{\pgfqpoint{-0.019444in}{0.005157in}}{\pgfqpoint{-0.019444in}{0.000000in}}%
\pgfpathcurveto{\pgfqpoint{-0.019444in}{-0.005157in}}{\pgfqpoint{-0.017396in}{-0.010103in}}{\pgfqpoint{-0.013749in}{-0.013749in}}%
\pgfpathcurveto{\pgfqpoint{-0.010103in}{-0.017396in}}{\pgfqpoint{-0.005157in}{-0.019444in}}{\pgfqpoint{0.000000in}{-0.019444in}}%
\pgfpathclose%
\pgfusepath{stroke,fill}%
}%
\begin{pgfscope}%
\pgfsys@transformshift{3.306143in}{4.100990in}%
\pgfsys@useobject{currentmarker}{}%
\end{pgfscope}%
\end{pgfscope}%
\begin{pgfscope}%
\pgfpathrectangle{\pgfqpoint{0.100000in}{2.413063in}}{\pgfqpoint{5.037500in}{3.427208in}}%
\pgfusepath{clip}%
\pgfsetrectcap%
\pgfsetroundjoin%
\pgfsetlinewidth{1.505625pt}%
\definecolor{currentstroke}{rgb}{0.678431,1.000000,0.184314}%
\pgfsetstrokecolor{currentstroke}%
\pgfsetstrokeopacity{0.500000}%
\pgfsetdash{}{0pt}%
\pgfpathmoveto{\pgfqpoint{3.042850in}{4.278911in}}%
\pgfusepath{stroke}%
\end{pgfscope}%
\begin{pgfscope}%
\pgfpathrectangle{\pgfqpoint{0.100000in}{2.413063in}}{\pgfqpoint{5.037500in}{3.427208in}}%
\pgfusepath{clip}%
\pgfsetbuttcap%
\pgfsetroundjoin%
\definecolor{currentfill}{rgb}{0.678431,1.000000,0.184314}%
\pgfsetfillcolor{currentfill}%
\pgfsetfillopacity{0.500000}%
\pgfsetlinewidth{0.250937pt}%
\definecolor{currentstroke}{rgb}{0.000000,0.000000,0.000000}%
\pgfsetstrokecolor{currentstroke}%
\pgfsetstrokeopacity{0.500000}%
\pgfsetdash{}{0pt}%
\pgfsys@defobject{currentmarker}{\pgfqpoint{-0.013889in}{-0.013889in}}{\pgfqpoint{0.013889in}{0.013889in}}{%
\pgfpathmoveto{\pgfqpoint{0.000000in}{-0.013889in}}%
\pgfpathcurveto{\pgfqpoint{0.003683in}{-0.013889in}}{\pgfqpoint{0.007216in}{-0.012425in}}{\pgfqpoint{0.009821in}{-0.009821in}}%
\pgfpathcurveto{\pgfqpoint{0.012425in}{-0.007216in}}{\pgfqpoint{0.013889in}{-0.003683in}}{\pgfqpoint{0.013889in}{0.000000in}}%
\pgfpathcurveto{\pgfqpoint{0.013889in}{0.003683in}}{\pgfqpoint{0.012425in}{0.007216in}}{\pgfqpoint{0.009821in}{0.009821in}}%
\pgfpathcurveto{\pgfqpoint{0.007216in}{0.012425in}}{\pgfqpoint{0.003683in}{0.013889in}}{\pgfqpoint{0.000000in}{0.013889in}}%
\pgfpathcurveto{\pgfqpoint{-0.003683in}{0.013889in}}{\pgfqpoint{-0.007216in}{0.012425in}}{\pgfqpoint{-0.009821in}{0.009821in}}%
\pgfpathcurveto{\pgfqpoint{-0.012425in}{0.007216in}}{\pgfqpoint{-0.013889in}{0.003683in}}{\pgfqpoint{-0.013889in}{0.000000in}}%
\pgfpathcurveto{\pgfqpoint{-0.013889in}{-0.003683in}}{\pgfqpoint{-0.012425in}{-0.007216in}}{\pgfqpoint{-0.009821in}{-0.009821in}}%
\pgfpathcurveto{\pgfqpoint{-0.007216in}{-0.012425in}}{\pgfqpoint{-0.003683in}{-0.013889in}}{\pgfqpoint{0.000000in}{-0.013889in}}%
\pgfpathclose%
\pgfusepath{stroke,fill}%
}%
\begin{pgfscope}%
\pgfsys@transformshift{3.042850in}{4.278911in}%
\pgfsys@useobject{currentmarker}{}%
\end{pgfscope}%
\end{pgfscope}%
\begin{pgfscope}%
\pgfpathrectangle{\pgfqpoint{0.100000in}{2.413063in}}{\pgfqpoint{5.037500in}{3.427208in}}%
\pgfusepath{clip}%
\pgfsetrectcap%
\pgfsetroundjoin%
\pgfsetlinewidth{1.505625pt}%
\definecolor{currentstroke}{rgb}{0.678431,1.000000,0.184314}%
\pgfsetstrokecolor{currentstroke}%
\pgfsetstrokeopacity{0.500000}%
\pgfsetdash{}{0pt}%
\pgfpathmoveto{\pgfqpoint{3.058630in}{4.236242in}}%
\pgfusepath{stroke}%
\end{pgfscope}%
\begin{pgfscope}%
\pgfpathrectangle{\pgfqpoint{0.100000in}{2.413063in}}{\pgfqpoint{5.037500in}{3.427208in}}%
\pgfusepath{clip}%
\pgfsetbuttcap%
\pgfsetroundjoin%
\definecolor{currentfill}{rgb}{0.678431,1.000000,0.184314}%
\pgfsetfillcolor{currentfill}%
\pgfsetfillopacity{0.500000}%
\pgfsetlinewidth{0.250937pt}%
\definecolor{currentstroke}{rgb}{0.000000,0.000000,0.000000}%
\pgfsetstrokecolor{currentstroke}%
\pgfsetstrokeopacity{0.500000}%
\pgfsetdash{}{0pt}%
\pgfsys@defobject{currentmarker}{\pgfqpoint{-0.019444in}{-0.019444in}}{\pgfqpoint{0.019444in}{0.019444in}}{%
\pgfpathmoveto{\pgfqpoint{0.000000in}{-0.019444in}}%
\pgfpathcurveto{\pgfqpoint{0.005157in}{-0.019444in}}{\pgfqpoint{0.010103in}{-0.017396in}}{\pgfqpoint{0.013749in}{-0.013749in}}%
\pgfpathcurveto{\pgfqpoint{0.017396in}{-0.010103in}}{\pgfqpoint{0.019444in}{-0.005157in}}{\pgfqpoint{0.019444in}{0.000000in}}%
\pgfpathcurveto{\pgfqpoint{0.019444in}{0.005157in}}{\pgfqpoint{0.017396in}{0.010103in}}{\pgfqpoint{0.013749in}{0.013749in}}%
\pgfpathcurveto{\pgfqpoint{0.010103in}{0.017396in}}{\pgfqpoint{0.005157in}{0.019444in}}{\pgfqpoint{0.000000in}{0.019444in}}%
\pgfpathcurveto{\pgfqpoint{-0.005157in}{0.019444in}}{\pgfqpoint{-0.010103in}{0.017396in}}{\pgfqpoint{-0.013749in}{0.013749in}}%
\pgfpathcurveto{\pgfqpoint{-0.017396in}{0.010103in}}{\pgfqpoint{-0.019444in}{0.005157in}}{\pgfqpoint{-0.019444in}{0.000000in}}%
\pgfpathcurveto{\pgfqpoint{-0.019444in}{-0.005157in}}{\pgfqpoint{-0.017396in}{-0.010103in}}{\pgfqpoint{-0.013749in}{-0.013749in}}%
\pgfpathcurveto{\pgfqpoint{-0.010103in}{-0.017396in}}{\pgfqpoint{-0.005157in}{-0.019444in}}{\pgfqpoint{0.000000in}{-0.019444in}}%
\pgfpathclose%
\pgfusepath{stroke,fill}%
}%
\begin{pgfscope}%
\pgfsys@transformshift{3.058630in}{4.236242in}%
\pgfsys@useobject{currentmarker}{}%
\end{pgfscope}%
\end{pgfscope}%
\begin{pgfscope}%
\pgfpathrectangle{\pgfqpoint{0.100000in}{2.413063in}}{\pgfqpoint{5.037500in}{3.427208in}}%
\pgfusepath{clip}%
\pgfsetrectcap%
\pgfsetroundjoin%
\pgfsetlinewidth{1.505625pt}%
\definecolor{currentstroke}{rgb}{0.678431,1.000000,0.184314}%
\pgfsetstrokecolor{currentstroke}%
\pgfsetstrokeopacity{0.500000}%
\pgfsetdash{}{0pt}%
\pgfpathmoveto{\pgfqpoint{2.848864in}{4.060427in}}%
\pgfusepath{stroke}%
\end{pgfscope}%
\begin{pgfscope}%
\pgfpathrectangle{\pgfqpoint{0.100000in}{2.413063in}}{\pgfqpoint{5.037500in}{3.427208in}}%
\pgfusepath{clip}%
\pgfsetbuttcap%
\pgfsetroundjoin%
\definecolor{currentfill}{rgb}{0.678431,1.000000,0.184314}%
\pgfsetfillcolor{currentfill}%
\pgfsetfillopacity{0.500000}%
\pgfsetlinewidth{0.250937pt}%
\definecolor{currentstroke}{rgb}{0.000000,0.000000,0.000000}%
\pgfsetstrokecolor{currentstroke}%
\pgfsetstrokeopacity{0.500000}%
\pgfsetdash{}{0pt}%
\pgfsys@defobject{currentmarker}{\pgfqpoint{-0.019444in}{-0.019444in}}{\pgfqpoint{0.019444in}{0.019444in}}{%
\pgfpathmoveto{\pgfqpoint{0.000000in}{-0.019444in}}%
\pgfpathcurveto{\pgfqpoint{0.005157in}{-0.019444in}}{\pgfqpoint{0.010103in}{-0.017396in}}{\pgfqpoint{0.013749in}{-0.013749in}}%
\pgfpathcurveto{\pgfqpoint{0.017396in}{-0.010103in}}{\pgfqpoint{0.019444in}{-0.005157in}}{\pgfqpoint{0.019444in}{0.000000in}}%
\pgfpathcurveto{\pgfqpoint{0.019444in}{0.005157in}}{\pgfqpoint{0.017396in}{0.010103in}}{\pgfqpoint{0.013749in}{0.013749in}}%
\pgfpathcurveto{\pgfqpoint{0.010103in}{0.017396in}}{\pgfqpoint{0.005157in}{0.019444in}}{\pgfqpoint{0.000000in}{0.019444in}}%
\pgfpathcurveto{\pgfqpoint{-0.005157in}{0.019444in}}{\pgfqpoint{-0.010103in}{0.017396in}}{\pgfqpoint{-0.013749in}{0.013749in}}%
\pgfpathcurveto{\pgfqpoint{-0.017396in}{0.010103in}}{\pgfqpoint{-0.019444in}{0.005157in}}{\pgfqpoint{-0.019444in}{0.000000in}}%
\pgfpathcurveto{\pgfqpoint{-0.019444in}{-0.005157in}}{\pgfqpoint{-0.017396in}{-0.010103in}}{\pgfqpoint{-0.013749in}{-0.013749in}}%
\pgfpathcurveto{\pgfqpoint{-0.010103in}{-0.017396in}}{\pgfqpoint{-0.005157in}{-0.019444in}}{\pgfqpoint{0.000000in}{-0.019444in}}%
\pgfpathclose%
\pgfusepath{stroke,fill}%
}%
\begin{pgfscope}%
\pgfsys@transformshift{2.848864in}{4.060427in}%
\pgfsys@useobject{currentmarker}{}%
\end{pgfscope}%
\end{pgfscope}%
\begin{pgfscope}%
\pgfpathrectangle{\pgfqpoint{0.100000in}{2.413063in}}{\pgfqpoint{5.037500in}{3.427208in}}%
\pgfusepath{clip}%
\pgfsetrectcap%
\pgfsetroundjoin%
\pgfsetlinewidth{1.505625pt}%
\definecolor{currentstroke}{rgb}{0.678431,1.000000,0.184314}%
\pgfsetstrokecolor{currentstroke}%
\pgfsetstrokeopacity{0.500000}%
\pgfsetdash{}{0pt}%
\pgfpathmoveto{\pgfqpoint{2.843184in}{4.290769in}}%
\pgfusepath{stroke}%
\end{pgfscope}%
\begin{pgfscope}%
\pgfpathrectangle{\pgfqpoint{0.100000in}{2.413063in}}{\pgfqpoint{5.037500in}{3.427208in}}%
\pgfusepath{clip}%
\pgfsetbuttcap%
\pgfsetroundjoin%
\definecolor{currentfill}{rgb}{0.678431,1.000000,0.184314}%
\pgfsetfillcolor{currentfill}%
\pgfsetfillopacity{0.500000}%
\pgfsetlinewidth{0.250937pt}%
\definecolor{currentstroke}{rgb}{0.000000,0.000000,0.000000}%
\pgfsetstrokecolor{currentstroke}%
\pgfsetstrokeopacity{0.500000}%
\pgfsetdash{}{0pt}%
\pgfsys@defobject{currentmarker}{\pgfqpoint{-0.013889in}{-0.013889in}}{\pgfqpoint{0.013889in}{0.013889in}}{%
\pgfpathmoveto{\pgfqpoint{0.000000in}{-0.013889in}}%
\pgfpathcurveto{\pgfqpoint{0.003683in}{-0.013889in}}{\pgfqpoint{0.007216in}{-0.012425in}}{\pgfqpoint{0.009821in}{-0.009821in}}%
\pgfpathcurveto{\pgfqpoint{0.012425in}{-0.007216in}}{\pgfqpoint{0.013889in}{-0.003683in}}{\pgfqpoint{0.013889in}{0.000000in}}%
\pgfpathcurveto{\pgfqpoint{0.013889in}{0.003683in}}{\pgfqpoint{0.012425in}{0.007216in}}{\pgfqpoint{0.009821in}{0.009821in}}%
\pgfpathcurveto{\pgfqpoint{0.007216in}{0.012425in}}{\pgfqpoint{0.003683in}{0.013889in}}{\pgfqpoint{0.000000in}{0.013889in}}%
\pgfpathcurveto{\pgfqpoint{-0.003683in}{0.013889in}}{\pgfqpoint{-0.007216in}{0.012425in}}{\pgfqpoint{-0.009821in}{0.009821in}}%
\pgfpathcurveto{\pgfqpoint{-0.012425in}{0.007216in}}{\pgfqpoint{-0.013889in}{0.003683in}}{\pgfqpoint{-0.013889in}{0.000000in}}%
\pgfpathcurveto{\pgfqpoint{-0.013889in}{-0.003683in}}{\pgfqpoint{-0.012425in}{-0.007216in}}{\pgfqpoint{-0.009821in}{-0.009821in}}%
\pgfpathcurveto{\pgfqpoint{-0.007216in}{-0.012425in}}{\pgfqpoint{-0.003683in}{-0.013889in}}{\pgfqpoint{0.000000in}{-0.013889in}}%
\pgfpathclose%
\pgfusepath{stroke,fill}%
}%
\begin{pgfscope}%
\pgfsys@transformshift{2.843184in}{4.290769in}%
\pgfsys@useobject{currentmarker}{}%
\end{pgfscope}%
\end{pgfscope}%
\begin{pgfscope}%
\pgfpathrectangle{\pgfqpoint{0.100000in}{2.413063in}}{\pgfqpoint{5.037500in}{3.427208in}}%
\pgfusepath{clip}%
\pgfsetrectcap%
\pgfsetroundjoin%
\pgfsetlinewidth{1.505625pt}%
\definecolor{currentstroke}{rgb}{0.678431,1.000000,0.184314}%
\pgfsetstrokecolor{currentstroke}%
\pgfsetstrokeopacity{0.500000}%
\pgfsetdash{}{0pt}%
\pgfpathmoveto{\pgfqpoint{2.817703in}{4.369465in}}%
\pgfusepath{stroke}%
\end{pgfscope}%
\begin{pgfscope}%
\pgfpathrectangle{\pgfqpoint{0.100000in}{2.413063in}}{\pgfqpoint{5.037500in}{3.427208in}}%
\pgfusepath{clip}%
\pgfsetbuttcap%
\pgfsetroundjoin%
\definecolor{currentfill}{rgb}{0.678431,1.000000,0.184314}%
\pgfsetfillcolor{currentfill}%
\pgfsetfillopacity{0.500000}%
\pgfsetlinewidth{0.250937pt}%
\definecolor{currentstroke}{rgb}{0.000000,0.000000,0.000000}%
\pgfsetstrokecolor{currentstroke}%
\pgfsetstrokeopacity{0.500000}%
\pgfsetdash{}{0pt}%
\pgfsys@defobject{currentmarker}{\pgfqpoint{-0.019444in}{-0.019444in}}{\pgfqpoint{0.019444in}{0.019444in}}{%
\pgfpathmoveto{\pgfqpoint{0.000000in}{-0.019444in}}%
\pgfpathcurveto{\pgfqpoint{0.005157in}{-0.019444in}}{\pgfqpoint{0.010103in}{-0.017396in}}{\pgfqpoint{0.013749in}{-0.013749in}}%
\pgfpathcurveto{\pgfqpoint{0.017396in}{-0.010103in}}{\pgfqpoint{0.019444in}{-0.005157in}}{\pgfqpoint{0.019444in}{0.000000in}}%
\pgfpathcurveto{\pgfqpoint{0.019444in}{0.005157in}}{\pgfqpoint{0.017396in}{0.010103in}}{\pgfqpoint{0.013749in}{0.013749in}}%
\pgfpathcurveto{\pgfqpoint{0.010103in}{0.017396in}}{\pgfqpoint{0.005157in}{0.019444in}}{\pgfqpoint{0.000000in}{0.019444in}}%
\pgfpathcurveto{\pgfqpoint{-0.005157in}{0.019444in}}{\pgfqpoint{-0.010103in}{0.017396in}}{\pgfqpoint{-0.013749in}{0.013749in}}%
\pgfpathcurveto{\pgfqpoint{-0.017396in}{0.010103in}}{\pgfqpoint{-0.019444in}{0.005157in}}{\pgfqpoint{-0.019444in}{0.000000in}}%
\pgfpathcurveto{\pgfqpoint{-0.019444in}{-0.005157in}}{\pgfqpoint{-0.017396in}{-0.010103in}}{\pgfqpoint{-0.013749in}{-0.013749in}}%
\pgfpathcurveto{\pgfqpoint{-0.010103in}{-0.017396in}}{\pgfqpoint{-0.005157in}{-0.019444in}}{\pgfqpoint{0.000000in}{-0.019444in}}%
\pgfpathclose%
\pgfusepath{stroke,fill}%
}%
\begin{pgfscope}%
\pgfsys@transformshift{2.817703in}{4.369465in}%
\pgfsys@useobject{currentmarker}{}%
\end{pgfscope}%
\end{pgfscope}%
\begin{pgfscope}%
\pgfpathrectangle{\pgfqpoint{0.100000in}{2.413063in}}{\pgfqpoint{5.037500in}{3.427208in}}%
\pgfusepath{clip}%
\pgfsetrectcap%
\pgfsetroundjoin%
\pgfsetlinewidth{1.505625pt}%
\definecolor{currentstroke}{rgb}{0.501961,0.501961,0.501961}%
\pgfsetstrokecolor{currentstroke}%
\pgfsetstrokeopacity{0.500000}%
\pgfsetdash{}{0pt}%
\pgfpathmoveto{\pgfqpoint{3.235918in}{4.249441in}}%
\pgfusepath{stroke}%
\end{pgfscope}%
\begin{pgfscope}%
\pgfpathrectangle{\pgfqpoint{0.100000in}{2.413063in}}{\pgfqpoint{5.037500in}{3.427208in}}%
\pgfusepath{clip}%
\pgfsetbuttcap%
\pgfsetroundjoin%
\definecolor{currentfill}{rgb}{0.501961,0.501961,0.501961}%
\pgfsetfillcolor{currentfill}%
\pgfsetfillopacity{0.500000}%
\pgfsetlinewidth{0.250937pt}%
\definecolor{currentstroke}{rgb}{0.000000,0.000000,0.000000}%
\pgfsetstrokecolor{currentstroke}%
\pgfsetstrokeopacity{0.500000}%
\pgfsetdash{}{0pt}%
\pgfsys@defobject{currentmarker}{\pgfqpoint{-0.013889in}{-0.013889in}}{\pgfqpoint{0.013889in}{0.013889in}}{%
\pgfpathmoveto{\pgfqpoint{0.000000in}{-0.013889in}}%
\pgfpathcurveto{\pgfqpoint{0.003683in}{-0.013889in}}{\pgfqpoint{0.007216in}{-0.012425in}}{\pgfqpoint{0.009821in}{-0.009821in}}%
\pgfpathcurveto{\pgfqpoint{0.012425in}{-0.007216in}}{\pgfqpoint{0.013889in}{-0.003683in}}{\pgfqpoint{0.013889in}{0.000000in}}%
\pgfpathcurveto{\pgfqpoint{0.013889in}{0.003683in}}{\pgfqpoint{0.012425in}{0.007216in}}{\pgfqpoint{0.009821in}{0.009821in}}%
\pgfpathcurveto{\pgfqpoint{0.007216in}{0.012425in}}{\pgfqpoint{0.003683in}{0.013889in}}{\pgfqpoint{0.000000in}{0.013889in}}%
\pgfpathcurveto{\pgfqpoint{-0.003683in}{0.013889in}}{\pgfqpoint{-0.007216in}{0.012425in}}{\pgfqpoint{-0.009821in}{0.009821in}}%
\pgfpathcurveto{\pgfqpoint{-0.012425in}{0.007216in}}{\pgfqpoint{-0.013889in}{0.003683in}}{\pgfqpoint{-0.013889in}{0.000000in}}%
\pgfpathcurveto{\pgfqpoint{-0.013889in}{-0.003683in}}{\pgfqpoint{-0.012425in}{-0.007216in}}{\pgfqpoint{-0.009821in}{-0.009821in}}%
\pgfpathcurveto{\pgfqpoint{-0.007216in}{-0.012425in}}{\pgfqpoint{-0.003683in}{-0.013889in}}{\pgfqpoint{0.000000in}{-0.013889in}}%
\pgfpathclose%
\pgfusepath{stroke,fill}%
}%
\begin{pgfscope}%
\pgfsys@transformshift{3.235918in}{4.249441in}%
\pgfsys@useobject{currentmarker}{}%
\end{pgfscope}%
\end{pgfscope}%
\begin{pgfscope}%
\pgfpathrectangle{\pgfqpoint{0.100000in}{2.413063in}}{\pgfqpoint{5.037500in}{3.427208in}}%
\pgfusepath{clip}%
\pgfsetrectcap%
\pgfsetroundjoin%
\pgfsetlinewidth{1.505625pt}%
\definecolor{currentstroke}{rgb}{0.678431,1.000000,0.184314}%
\pgfsetstrokecolor{currentstroke}%
\pgfsetstrokeopacity{0.500000}%
\pgfsetdash{}{0pt}%
\pgfpathmoveto{\pgfqpoint{2.960251in}{4.076874in}}%
\pgfusepath{stroke}%
\end{pgfscope}%
\begin{pgfscope}%
\pgfpathrectangle{\pgfqpoint{0.100000in}{2.413063in}}{\pgfqpoint{5.037500in}{3.427208in}}%
\pgfusepath{clip}%
\pgfsetbuttcap%
\pgfsetroundjoin%
\definecolor{currentfill}{rgb}{0.678431,1.000000,0.184314}%
\pgfsetfillcolor{currentfill}%
\pgfsetfillopacity{0.500000}%
\pgfsetlinewidth{0.250937pt}%
\definecolor{currentstroke}{rgb}{0.000000,0.000000,0.000000}%
\pgfsetstrokecolor{currentstroke}%
\pgfsetstrokeopacity{0.500000}%
\pgfsetdash{}{0pt}%
\pgfsys@defobject{currentmarker}{\pgfqpoint{-0.027778in}{-0.027778in}}{\pgfqpoint{0.027778in}{0.027778in}}{%
\pgfpathmoveto{\pgfqpoint{0.000000in}{-0.027778in}}%
\pgfpathcurveto{\pgfqpoint{0.007367in}{-0.027778in}}{\pgfqpoint{0.014433in}{-0.024851in}}{\pgfqpoint{0.019642in}{-0.019642in}}%
\pgfpathcurveto{\pgfqpoint{0.024851in}{-0.014433in}}{\pgfqpoint{0.027778in}{-0.007367in}}{\pgfqpoint{0.027778in}{0.000000in}}%
\pgfpathcurveto{\pgfqpoint{0.027778in}{0.007367in}}{\pgfqpoint{0.024851in}{0.014433in}}{\pgfqpoint{0.019642in}{0.019642in}}%
\pgfpathcurveto{\pgfqpoint{0.014433in}{0.024851in}}{\pgfqpoint{0.007367in}{0.027778in}}{\pgfqpoint{0.000000in}{0.027778in}}%
\pgfpathcurveto{\pgfqpoint{-0.007367in}{0.027778in}}{\pgfqpoint{-0.014433in}{0.024851in}}{\pgfqpoint{-0.019642in}{0.019642in}}%
\pgfpathcurveto{\pgfqpoint{-0.024851in}{0.014433in}}{\pgfqpoint{-0.027778in}{0.007367in}}{\pgfqpoint{-0.027778in}{0.000000in}}%
\pgfpathcurveto{\pgfqpoint{-0.027778in}{-0.007367in}}{\pgfqpoint{-0.024851in}{-0.014433in}}{\pgfqpoint{-0.019642in}{-0.019642in}}%
\pgfpathcurveto{\pgfqpoint{-0.014433in}{-0.024851in}}{\pgfqpoint{-0.007367in}{-0.027778in}}{\pgfqpoint{0.000000in}{-0.027778in}}%
\pgfpathclose%
\pgfusepath{stroke,fill}%
}%
\begin{pgfscope}%
\pgfsys@transformshift{2.960251in}{4.076874in}%
\pgfsys@useobject{currentmarker}{}%
\end{pgfscope}%
\end{pgfscope}%
\begin{pgfscope}%
\pgfpathrectangle{\pgfqpoint{0.100000in}{2.413063in}}{\pgfqpoint{5.037500in}{3.427208in}}%
\pgfusepath{clip}%
\pgfsetrectcap%
\pgfsetroundjoin%
\pgfsetlinewidth{1.505625pt}%
\definecolor{currentstroke}{rgb}{0.678431,1.000000,0.184314}%
\pgfsetstrokecolor{currentstroke}%
\pgfsetstrokeopacity{0.500000}%
\pgfsetdash{}{0pt}%
\pgfpathmoveto{\pgfqpoint{1.716761in}{5.145212in}}%
\pgfusepath{stroke}%
\end{pgfscope}%
\begin{pgfscope}%
\pgfpathrectangle{\pgfqpoint{0.100000in}{2.413063in}}{\pgfqpoint{5.037500in}{3.427208in}}%
\pgfusepath{clip}%
\pgfsetbuttcap%
\pgfsetroundjoin%
\definecolor{currentfill}{rgb}{0.678431,1.000000,0.184314}%
\pgfsetfillcolor{currentfill}%
\pgfsetfillopacity{0.500000}%
\pgfsetlinewidth{0.250937pt}%
\definecolor{currentstroke}{rgb}{0.000000,0.000000,0.000000}%
\pgfsetstrokecolor{currentstroke}%
\pgfsetstrokeopacity{0.500000}%
\pgfsetdash{}{0pt}%
\pgfsys@defobject{currentmarker}{\pgfqpoint{-0.041667in}{-0.041667in}}{\pgfqpoint{0.041667in}{0.041667in}}{%
\pgfpathmoveto{\pgfqpoint{0.000000in}{-0.041667in}}%
\pgfpathcurveto{\pgfqpoint{0.011050in}{-0.041667in}}{\pgfqpoint{0.021649in}{-0.037276in}}{\pgfqpoint{0.029463in}{-0.029463in}}%
\pgfpathcurveto{\pgfqpoint{0.037276in}{-0.021649in}}{\pgfqpoint{0.041667in}{-0.011050in}}{\pgfqpoint{0.041667in}{0.000000in}}%
\pgfpathcurveto{\pgfqpoint{0.041667in}{0.011050in}}{\pgfqpoint{0.037276in}{0.021649in}}{\pgfqpoint{0.029463in}{0.029463in}}%
\pgfpathcurveto{\pgfqpoint{0.021649in}{0.037276in}}{\pgfqpoint{0.011050in}{0.041667in}}{\pgfqpoint{0.000000in}{0.041667in}}%
\pgfpathcurveto{\pgfqpoint{-0.011050in}{0.041667in}}{\pgfqpoint{-0.021649in}{0.037276in}}{\pgfqpoint{-0.029463in}{0.029463in}}%
\pgfpathcurveto{\pgfqpoint{-0.037276in}{0.021649in}}{\pgfqpoint{-0.041667in}{0.011050in}}{\pgfqpoint{-0.041667in}{0.000000in}}%
\pgfpathcurveto{\pgfqpoint{-0.041667in}{-0.011050in}}{\pgfqpoint{-0.037276in}{-0.021649in}}{\pgfqpoint{-0.029463in}{-0.029463in}}%
\pgfpathcurveto{\pgfqpoint{-0.021649in}{-0.037276in}}{\pgfqpoint{-0.011050in}{-0.041667in}}{\pgfqpoint{0.000000in}{-0.041667in}}%
\pgfpathclose%
\pgfusepath{stroke,fill}%
}%
\begin{pgfscope}%
\pgfsys@transformshift{1.716761in}{5.145212in}%
\pgfsys@useobject{currentmarker}{}%
\end{pgfscope}%
\end{pgfscope}%
\begin{pgfscope}%
\pgfpathrectangle{\pgfqpoint{0.100000in}{2.413063in}}{\pgfqpoint{5.037500in}{3.427208in}}%
\pgfusepath{clip}%
\pgfsetrectcap%
\pgfsetroundjoin%
\pgfsetlinewidth{1.505625pt}%
\definecolor{currentstroke}{rgb}{0.678431,1.000000,0.184314}%
\pgfsetstrokecolor{currentstroke}%
\pgfsetstrokeopacity{0.500000}%
\pgfsetdash{}{0pt}%
\pgfpathmoveto{\pgfqpoint{1.529298in}{5.378464in}}%
\pgfusepath{stroke}%
\end{pgfscope}%
\begin{pgfscope}%
\pgfpathrectangle{\pgfqpoint{0.100000in}{2.413063in}}{\pgfqpoint{5.037500in}{3.427208in}}%
\pgfusepath{clip}%
\pgfsetbuttcap%
\pgfsetroundjoin%
\definecolor{currentfill}{rgb}{0.678431,1.000000,0.184314}%
\pgfsetfillcolor{currentfill}%
\pgfsetfillopacity{0.500000}%
\pgfsetlinewidth{0.250937pt}%
\definecolor{currentstroke}{rgb}{0.000000,0.000000,0.000000}%
\pgfsetstrokecolor{currentstroke}%
\pgfsetstrokeopacity{0.500000}%
\pgfsetdash{}{0pt}%
\pgfsys@defobject{currentmarker}{\pgfqpoint{-0.044444in}{-0.044444in}}{\pgfqpoint{0.044444in}{0.044444in}}{%
\pgfpathmoveto{\pgfqpoint{0.000000in}{-0.044444in}}%
\pgfpathcurveto{\pgfqpoint{0.011787in}{-0.044444in}}{\pgfqpoint{0.023092in}{-0.039761in}}{\pgfqpoint{0.031427in}{-0.031427in}}%
\pgfpathcurveto{\pgfqpoint{0.039761in}{-0.023092in}}{\pgfqpoint{0.044444in}{-0.011787in}}{\pgfqpoint{0.044444in}{0.000000in}}%
\pgfpathcurveto{\pgfqpoint{0.044444in}{0.011787in}}{\pgfqpoint{0.039761in}{0.023092in}}{\pgfqpoint{0.031427in}{0.031427in}}%
\pgfpathcurveto{\pgfqpoint{0.023092in}{0.039761in}}{\pgfqpoint{0.011787in}{0.044444in}}{\pgfqpoint{0.000000in}{0.044444in}}%
\pgfpathcurveto{\pgfqpoint{-0.011787in}{0.044444in}}{\pgfqpoint{-0.023092in}{0.039761in}}{\pgfqpoint{-0.031427in}{0.031427in}}%
\pgfpathcurveto{\pgfqpoint{-0.039761in}{0.023092in}}{\pgfqpoint{-0.044444in}{0.011787in}}{\pgfqpoint{-0.044444in}{0.000000in}}%
\pgfpathcurveto{\pgfqpoint{-0.044444in}{-0.011787in}}{\pgfqpoint{-0.039761in}{-0.023092in}}{\pgfqpoint{-0.031427in}{-0.031427in}}%
\pgfpathcurveto{\pgfqpoint{-0.023092in}{-0.039761in}}{\pgfqpoint{-0.011787in}{-0.044444in}}{\pgfqpoint{0.000000in}{-0.044444in}}%
\pgfpathclose%
\pgfusepath{stroke,fill}%
}%
\begin{pgfscope}%
\pgfsys@transformshift{1.529298in}{5.378464in}%
\pgfsys@useobject{currentmarker}{}%
\end{pgfscope}%
\end{pgfscope}%
\begin{pgfscope}%
\pgfpathrectangle{\pgfqpoint{0.100000in}{2.413063in}}{\pgfqpoint{5.037500in}{3.427208in}}%
\pgfusepath{clip}%
\pgfsetrectcap%
\pgfsetroundjoin%
\pgfsetlinewidth{1.505625pt}%
\definecolor{currentstroke}{rgb}{0.678431,1.000000,0.184314}%
\pgfsetstrokecolor{currentstroke}%
\pgfsetstrokeopacity{0.500000}%
\pgfsetdash{}{0pt}%
\pgfpathmoveto{\pgfqpoint{1.305298in}{5.347237in}}%
\pgfusepath{stroke}%
\end{pgfscope}%
\begin{pgfscope}%
\pgfpathrectangle{\pgfqpoint{0.100000in}{2.413063in}}{\pgfqpoint{5.037500in}{3.427208in}}%
\pgfusepath{clip}%
\pgfsetbuttcap%
\pgfsetroundjoin%
\definecolor{currentfill}{rgb}{0.678431,1.000000,0.184314}%
\pgfsetfillcolor{currentfill}%
\pgfsetfillopacity{0.500000}%
\pgfsetlinewidth{0.250937pt}%
\definecolor{currentstroke}{rgb}{0.000000,0.000000,0.000000}%
\pgfsetstrokecolor{currentstroke}%
\pgfsetstrokeopacity{0.500000}%
\pgfsetdash{}{0pt}%
\pgfsys@defobject{currentmarker}{\pgfqpoint{-0.044444in}{-0.044444in}}{\pgfqpoint{0.044444in}{0.044444in}}{%
\pgfpathmoveto{\pgfqpoint{0.000000in}{-0.044444in}}%
\pgfpathcurveto{\pgfqpoint{0.011787in}{-0.044444in}}{\pgfqpoint{0.023092in}{-0.039761in}}{\pgfqpoint{0.031427in}{-0.031427in}}%
\pgfpathcurveto{\pgfqpoint{0.039761in}{-0.023092in}}{\pgfqpoint{0.044444in}{-0.011787in}}{\pgfqpoint{0.044444in}{0.000000in}}%
\pgfpathcurveto{\pgfqpoint{0.044444in}{0.011787in}}{\pgfqpoint{0.039761in}{0.023092in}}{\pgfqpoint{0.031427in}{0.031427in}}%
\pgfpathcurveto{\pgfqpoint{0.023092in}{0.039761in}}{\pgfqpoint{0.011787in}{0.044444in}}{\pgfqpoint{0.000000in}{0.044444in}}%
\pgfpathcurveto{\pgfqpoint{-0.011787in}{0.044444in}}{\pgfqpoint{-0.023092in}{0.039761in}}{\pgfqpoint{-0.031427in}{0.031427in}}%
\pgfpathcurveto{\pgfqpoint{-0.039761in}{0.023092in}}{\pgfqpoint{-0.044444in}{0.011787in}}{\pgfqpoint{-0.044444in}{0.000000in}}%
\pgfpathcurveto{\pgfqpoint{-0.044444in}{-0.011787in}}{\pgfqpoint{-0.039761in}{-0.023092in}}{\pgfqpoint{-0.031427in}{-0.031427in}}%
\pgfpathcurveto{\pgfqpoint{-0.023092in}{-0.039761in}}{\pgfqpoint{-0.011787in}{-0.044444in}}{\pgfqpoint{0.000000in}{-0.044444in}}%
\pgfpathclose%
\pgfusepath{stroke,fill}%
}%
\begin{pgfscope}%
\pgfsys@transformshift{1.305298in}{5.347237in}%
\pgfsys@useobject{currentmarker}{}%
\end{pgfscope}%
\end{pgfscope}%
\begin{pgfscope}%
\pgfpathrectangle{\pgfqpoint{0.100000in}{2.413063in}}{\pgfqpoint{5.037500in}{3.427208in}}%
\pgfusepath{clip}%
\pgfsetrectcap%
\pgfsetroundjoin%
\pgfsetlinewidth{1.505625pt}%
\definecolor{currentstroke}{rgb}{0.678431,1.000000,0.184314}%
\pgfsetstrokecolor{currentstroke}%
\pgfsetstrokeopacity{0.500000}%
\pgfsetdash{}{0pt}%
\pgfpathmoveto{\pgfqpoint{2.513569in}{4.507898in}}%
\pgfusepath{stroke}%
\end{pgfscope}%
\begin{pgfscope}%
\pgfpathrectangle{\pgfqpoint{0.100000in}{2.413063in}}{\pgfqpoint{5.037500in}{3.427208in}}%
\pgfusepath{clip}%
\pgfsetbuttcap%
\pgfsetroundjoin%
\definecolor{currentfill}{rgb}{0.678431,1.000000,0.184314}%
\pgfsetfillcolor{currentfill}%
\pgfsetfillopacity{0.500000}%
\pgfsetlinewidth{0.250937pt}%
\definecolor{currentstroke}{rgb}{0.000000,0.000000,0.000000}%
\pgfsetstrokecolor{currentstroke}%
\pgfsetstrokeopacity{0.500000}%
\pgfsetdash{}{0pt}%
\pgfsys@defobject{currentmarker}{\pgfqpoint{-0.047222in}{-0.047222in}}{\pgfqpoint{0.047222in}{0.047222in}}{%
\pgfpathmoveto{\pgfqpoint{0.000000in}{-0.047222in}}%
\pgfpathcurveto{\pgfqpoint{0.012523in}{-0.047222in}}{\pgfqpoint{0.024536in}{-0.042247in}}{\pgfqpoint{0.033391in}{-0.033391in}}%
\pgfpathcurveto{\pgfqpoint{0.042247in}{-0.024536in}}{\pgfqpoint{0.047222in}{-0.012523in}}{\pgfqpoint{0.047222in}{0.000000in}}%
\pgfpathcurveto{\pgfqpoint{0.047222in}{0.012523in}}{\pgfqpoint{0.042247in}{0.024536in}}{\pgfqpoint{0.033391in}{0.033391in}}%
\pgfpathcurveto{\pgfqpoint{0.024536in}{0.042247in}}{\pgfqpoint{0.012523in}{0.047222in}}{\pgfqpoint{0.000000in}{0.047222in}}%
\pgfpathcurveto{\pgfqpoint{-0.012523in}{0.047222in}}{\pgfqpoint{-0.024536in}{0.042247in}}{\pgfqpoint{-0.033391in}{0.033391in}}%
\pgfpathcurveto{\pgfqpoint{-0.042247in}{0.024536in}}{\pgfqpoint{-0.047222in}{0.012523in}}{\pgfqpoint{-0.047222in}{0.000000in}}%
\pgfpathcurveto{\pgfqpoint{-0.047222in}{-0.012523in}}{\pgfqpoint{-0.042247in}{-0.024536in}}{\pgfqpoint{-0.033391in}{-0.033391in}}%
\pgfpathcurveto{\pgfqpoint{-0.024536in}{-0.042247in}}{\pgfqpoint{-0.012523in}{-0.047222in}}{\pgfqpoint{0.000000in}{-0.047222in}}%
\pgfpathclose%
\pgfusepath{stroke,fill}%
}%
\begin{pgfscope}%
\pgfsys@transformshift{2.513569in}{4.507898in}%
\pgfsys@useobject{currentmarker}{}%
\end{pgfscope}%
\end{pgfscope}%
\begin{pgfscope}%
\pgfpathrectangle{\pgfqpoint{0.100000in}{2.413063in}}{\pgfqpoint{5.037500in}{3.427208in}}%
\pgfusepath{clip}%
\pgfsetrectcap%
\pgfsetroundjoin%
\pgfsetlinewidth{1.505625pt}%
\definecolor{currentstroke}{rgb}{0.678431,1.000000,0.184314}%
\pgfsetstrokecolor{currentstroke}%
\pgfsetstrokeopacity{0.500000}%
\pgfsetdash{}{0pt}%
\pgfpathmoveto{\pgfqpoint{2.658694in}{4.489578in}}%
\pgfusepath{stroke}%
\end{pgfscope}%
\begin{pgfscope}%
\pgfpathrectangle{\pgfqpoint{0.100000in}{2.413063in}}{\pgfqpoint{5.037500in}{3.427208in}}%
\pgfusepath{clip}%
\pgfsetbuttcap%
\pgfsetroundjoin%
\definecolor{currentfill}{rgb}{0.678431,1.000000,0.184314}%
\pgfsetfillcolor{currentfill}%
\pgfsetfillopacity{0.500000}%
\pgfsetlinewidth{0.250937pt}%
\definecolor{currentstroke}{rgb}{0.000000,0.000000,0.000000}%
\pgfsetstrokecolor{currentstroke}%
\pgfsetstrokeopacity{0.500000}%
\pgfsetdash{}{0pt}%
\pgfsys@defobject{currentmarker}{\pgfqpoint{-0.033333in}{-0.033333in}}{\pgfqpoint{0.033333in}{0.033333in}}{%
\pgfpathmoveto{\pgfqpoint{0.000000in}{-0.033333in}}%
\pgfpathcurveto{\pgfqpoint{0.008840in}{-0.033333in}}{\pgfqpoint{0.017319in}{-0.029821in}}{\pgfqpoint{0.023570in}{-0.023570in}}%
\pgfpathcurveto{\pgfqpoint{0.029821in}{-0.017319in}}{\pgfqpoint{0.033333in}{-0.008840in}}{\pgfqpoint{0.033333in}{0.000000in}}%
\pgfpathcurveto{\pgfqpoint{0.033333in}{0.008840in}}{\pgfqpoint{0.029821in}{0.017319in}}{\pgfqpoint{0.023570in}{0.023570in}}%
\pgfpathcurveto{\pgfqpoint{0.017319in}{0.029821in}}{\pgfqpoint{0.008840in}{0.033333in}}{\pgfqpoint{0.000000in}{0.033333in}}%
\pgfpathcurveto{\pgfqpoint{-0.008840in}{0.033333in}}{\pgfqpoint{-0.017319in}{0.029821in}}{\pgfqpoint{-0.023570in}{0.023570in}}%
\pgfpathcurveto{\pgfqpoint{-0.029821in}{0.017319in}}{\pgfqpoint{-0.033333in}{0.008840in}}{\pgfqpoint{-0.033333in}{0.000000in}}%
\pgfpathcurveto{\pgfqpoint{-0.033333in}{-0.008840in}}{\pgfqpoint{-0.029821in}{-0.017319in}}{\pgfqpoint{-0.023570in}{-0.023570in}}%
\pgfpathcurveto{\pgfqpoint{-0.017319in}{-0.029821in}}{\pgfqpoint{-0.008840in}{-0.033333in}}{\pgfqpoint{0.000000in}{-0.033333in}}%
\pgfpathclose%
\pgfusepath{stroke,fill}%
}%
\begin{pgfscope}%
\pgfsys@transformshift{2.658694in}{4.489578in}%
\pgfsys@useobject{currentmarker}{}%
\end{pgfscope}%
\end{pgfscope}%
\begin{pgfscope}%
\pgfpathrectangle{\pgfqpoint{0.100000in}{2.413063in}}{\pgfqpoint{5.037500in}{3.427208in}}%
\pgfusepath{clip}%
\pgfsetrectcap%
\pgfsetroundjoin%
\pgfsetlinewidth{1.505625pt}%
\definecolor{currentstroke}{rgb}{0.678431,1.000000,0.184314}%
\pgfsetstrokecolor{currentstroke}%
\pgfsetstrokeopacity{0.500000}%
\pgfsetdash{}{0pt}%
\pgfpathmoveto{\pgfqpoint{2.722888in}{4.541515in}}%
\pgfusepath{stroke}%
\end{pgfscope}%
\begin{pgfscope}%
\pgfpathrectangle{\pgfqpoint{0.100000in}{2.413063in}}{\pgfqpoint{5.037500in}{3.427208in}}%
\pgfusepath{clip}%
\pgfsetbuttcap%
\pgfsetroundjoin%
\definecolor{currentfill}{rgb}{0.678431,1.000000,0.184314}%
\pgfsetfillcolor{currentfill}%
\pgfsetfillopacity{0.500000}%
\pgfsetlinewidth{0.250937pt}%
\definecolor{currentstroke}{rgb}{0.000000,0.000000,0.000000}%
\pgfsetstrokecolor{currentstroke}%
\pgfsetstrokeopacity{0.500000}%
\pgfsetdash{}{0pt}%
\pgfsys@defobject{currentmarker}{\pgfqpoint{-0.030556in}{-0.030556in}}{\pgfqpoint{0.030556in}{0.030556in}}{%
\pgfpathmoveto{\pgfqpoint{0.000000in}{-0.030556in}}%
\pgfpathcurveto{\pgfqpoint{0.008103in}{-0.030556in}}{\pgfqpoint{0.015876in}{-0.027336in}}{\pgfqpoint{0.021606in}{-0.021606in}}%
\pgfpathcurveto{\pgfqpoint{0.027336in}{-0.015876in}}{\pgfqpoint{0.030556in}{-0.008103in}}{\pgfqpoint{0.030556in}{0.000000in}}%
\pgfpathcurveto{\pgfqpoint{0.030556in}{0.008103in}}{\pgfqpoint{0.027336in}{0.015876in}}{\pgfqpoint{0.021606in}{0.021606in}}%
\pgfpathcurveto{\pgfqpoint{0.015876in}{0.027336in}}{\pgfqpoint{0.008103in}{0.030556in}}{\pgfqpoint{0.000000in}{0.030556in}}%
\pgfpathcurveto{\pgfqpoint{-0.008103in}{0.030556in}}{\pgfqpoint{-0.015876in}{0.027336in}}{\pgfqpoint{-0.021606in}{0.021606in}}%
\pgfpathcurveto{\pgfqpoint{-0.027336in}{0.015876in}}{\pgfqpoint{-0.030556in}{0.008103in}}{\pgfqpoint{-0.030556in}{0.000000in}}%
\pgfpathcurveto{\pgfqpoint{-0.030556in}{-0.008103in}}{\pgfqpoint{-0.027336in}{-0.015876in}}{\pgfqpoint{-0.021606in}{-0.021606in}}%
\pgfpathcurveto{\pgfqpoint{-0.015876in}{-0.027336in}}{\pgfqpoint{-0.008103in}{-0.030556in}}{\pgfqpoint{0.000000in}{-0.030556in}}%
\pgfpathclose%
\pgfusepath{stroke,fill}%
}%
\begin{pgfscope}%
\pgfsys@transformshift{2.722888in}{4.541515in}%
\pgfsys@useobject{currentmarker}{}%
\end{pgfscope}%
\end{pgfscope}%
\begin{pgfscope}%
\pgfpathrectangle{\pgfqpoint{0.100000in}{2.413063in}}{\pgfqpoint{5.037500in}{3.427208in}}%
\pgfusepath{clip}%
\pgfsetrectcap%
\pgfsetroundjoin%
\pgfsetlinewidth{1.505625pt}%
\definecolor{currentstroke}{rgb}{0.678431,1.000000,0.184314}%
\pgfsetstrokecolor{currentstroke}%
\pgfsetstrokeopacity{0.500000}%
\pgfsetdash{}{0pt}%
\pgfpathmoveto{\pgfqpoint{0.620199in}{4.599268in}}%
\pgfusepath{stroke}%
\end{pgfscope}%
\begin{pgfscope}%
\pgfpathrectangle{\pgfqpoint{0.100000in}{2.413063in}}{\pgfqpoint{5.037500in}{3.427208in}}%
\pgfusepath{clip}%
\pgfsetbuttcap%
\pgfsetroundjoin%
\definecolor{currentfill}{rgb}{0.678431,1.000000,0.184314}%
\pgfsetfillcolor{currentfill}%
\pgfsetfillopacity{0.500000}%
\pgfsetlinewidth{0.250937pt}%
\definecolor{currentstroke}{rgb}{0.000000,0.000000,0.000000}%
\pgfsetstrokecolor{currentstroke}%
\pgfsetstrokeopacity{0.500000}%
\pgfsetdash{}{0pt}%
\pgfsys@defobject{currentmarker}{\pgfqpoint{-0.011111in}{-0.011111in}}{\pgfqpoint{0.011111in}{0.011111in}}{%
\pgfpathmoveto{\pgfqpoint{0.000000in}{-0.011111in}}%
\pgfpathcurveto{\pgfqpoint{0.002947in}{-0.011111in}}{\pgfqpoint{0.005773in}{-0.009940in}}{\pgfqpoint{0.007857in}{-0.007857in}}%
\pgfpathcurveto{\pgfqpoint{0.009940in}{-0.005773in}}{\pgfqpoint{0.011111in}{-0.002947in}}{\pgfqpoint{0.011111in}{0.000000in}}%
\pgfpathcurveto{\pgfqpoint{0.011111in}{0.002947in}}{\pgfqpoint{0.009940in}{0.005773in}}{\pgfqpoint{0.007857in}{0.007857in}}%
\pgfpathcurveto{\pgfqpoint{0.005773in}{0.009940in}}{\pgfqpoint{0.002947in}{0.011111in}}{\pgfqpoint{0.000000in}{0.011111in}}%
\pgfpathcurveto{\pgfqpoint{-0.002947in}{0.011111in}}{\pgfqpoint{-0.005773in}{0.009940in}}{\pgfqpoint{-0.007857in}{0.007857in}}%
\pgfpathcurveto{\pgfqpoint{-0.009940in}{0.005773in}}{\pgfqpoint{-0.011111in}{0.002947in}}{\pgfqpoint{-0.011111in}{0.000000in}}%
\pgfpathcurveto{\pgfqpoint{-0.011111in}{-0.002947in}}{\pgfqpoint{-0.009940in}{-0.005773in}}{\pgfqpoint{-0.007857in}{-0.007857in}}%
\pgfpathcurveto{\pgfqpoint{-0.005773in}{-0.009940in}}{\pgfqpoint{-0.002947in}{-0.011111in}}{\pgfqpoint{0.000000in}{-0.011111in}}%
\pgfpathclose%
\pgfusepath{stroke,fill}%
}%
\begin{pgfscope}%
\pgfsys@transformshift{0.620199in}{4.599268in}%
\pgfsys@useobject{currentmarker}{}%
\end{pgfscope}%
\end{pgfscope}%
\begin{pgfscope}%
\pgfpathrectangle{\pgfqpoint{0.100000in}{2.413063in}}{\pgfqpoint{5.037500in}{3.427208in}}%
\pgfusepath{clip}%
\pgfsetrectcap%
\pgfsetroundjoin%
\pgfsetlinewidth{1.505625pt}%
\definecolor{currentstroke}{rgb}{0.000000,0.000000,1.000000}%
\pgfsetstrokecolor{currentstroke}%
\pgfsetstrokeopacity{0.500000}%
\pgfsetdash{}{0pt}%
\pgfpathmoveto{\pgfqpoint{0.944001in}{4.161574in}}%
\pgfusepath{stroke}%
\end{pgfscope}%
\begin{pgfscope}%
\pgfpathrectangle{\pgfqpoint{0.100000in}{2.413063in}}{\pgfqpoint{5.037500in}{3.427208in}}%
\pgfusepath{clip}%
\pgfsetbuttcap%
\pgfsetroundjoin%
\definecolor{currentfill}{rgb}{0.000000,0.000000,1.000000}%
\pgfsetfillcolor{currentfill}%
\pgfsetfillopacity{0.500000}%
\pgfsetlinewidth{0.250937pt}%
\definecolor{currentstroke}{rgb}{0.000000,0.000000,0.000000}%
\pgfsetstrokecolor{currentstroke}%
\pgfsetstrokeopacity{0.500000}%
\pgfsetdash{}{0pt}%
\pgfsys@defobject{currentmarker}{\pgfqpoint{-0.066667in}{-0.066667in}}{\pgfqpoint{0.066667in}{0.066667in}}{%
\pgfpathmoveto{\pgfqpoint{0.000000in}{-0.066667in}}%
\pgfpathcurveto{\pgfqpoint{0.017680in}{-0.066667in}}{\pgfqpoint{0.034639in}{-0.059642in}}{\pgfqpoint{0.047140in}{-0.047140in}}%
\pgfpathcurveto{\pgfqpoint{0.059642in}{-0.034639in}}{\pgfqpoint{0.066667in}{-0.017680in}}{\pgfqpoint{0.066667in}{0.000000in}}%
\pgfpathcurveto{\pgfqpoint{0.066667in}{0.017680in}}{\pgfqpoint{0.059642in}{0.034639in}}{\pgfqpoint{0.047140in}{0.047140in}}%
\pgfpathcurveto{\pgfqpoint{0.034639in}{0.059642in}}{\pgfqpoint{0.017680in}{0.066667in}}{\pgfqpoint{0.000000in}{0.066667in}}%
\pgfpathcurveto{\pgfqpoint{-0.017680in}{0.066667in}}{\pgfqpoint{-0.034639in}{0.059642in}}{\pgfqpoint{-0.047140in}{0.047140in}}%
\pgfpathcurveto{\pgfqpoint{-0.059642in}{0.034639in}}{\pgfqpoint{-0.066667in}{0.017680in}}{\pgfqpoint{-0.066667in}{0.000000in}}%
\pgfpathcurveto{\pgfqpoint{-0.066667in}{-0.017680in}}{\pgfqpoint{-0.059642in}{-0.034639in}}{\pgfqpoint{-0.047140in}{-0.047140in}}%
\pgfpathcurveto{\pgfqpoint{-0.034639in}{-0.059642in}}{\pgfqpoint{-0.017680in}{-0.066667in}}{\pgfqpoint{0.000000in}{-0.066667in}}%
\pgfpathclose%
\pgfusepath{stroke,fill}%
}%
\begin{pgfscope}%
\pgfsys@transformshift{0.944001in}{4.161574in}%
\pgfsys@useobject{currentmarker}{}%
\end{pgfscope}%
\end{pgfscope}%
\begin{pgfscope}%
\pgfpathrectangle{\pgfqpoint{0.100000in}{2.413063in}}{\pgfqpoint{5.037500in}{3.427208in}}%
\pgfusepath{clip}%
\pgfsetrectcap%
\pgfsetroundjoin%
\pgfsetlinewidth{1.505625pt}%
\definecolor{currentstroke}{rgb}{0.501961,0.501961,0.501961}%
\pgfsetstrokecolor{currentstroke}%
\pgfsetstrokeopacity{0.500000}%
\pgfsetdash{}{0pt}%
\pgfpathmoveto{\pgfqpoint{0.627569in}{4.640906in}}%
\pgfusepath{stroke}%
\end{pgfscope}%
\begin{pgfscope}%
\pgfpathrectangle{\pgfqpoint{0.100000in}{2.413063in}}{\pgfqpoint{5.037500in}{3.427208in}}%
\pgfusepath{clip}%
\pgfsetbuttcap%
\pgfsetroundjoin%
\definecolor{currentfill}{rgb}{0.501961,0.501961,0.501961}%
\pgfsetfillcolor{currentfill}%
\pgfsetfillopacity{0.500000}%
\pgfsetlinewidth{0.250937pt}%
\definecolor{currentstroke}{rgb}{0.000000,0.000000,0.000000}%
\pgfsetstrokecolor{currentstroke}%
\pgfsetstrokeopacity{0.500000}%
\pgfsetdash{}{0pt}%
\pgfsys@defobject{currentmarker}{\pgfqpoint{-0.013889in}{-0.013889in}}{\pgfqpoint{0.013889in}{0.013889in}}{%
\pgfpathmoveto{\pgfqpoint{0.000000in}{-0.013889in}}%
\pgfpathcurveto{\pgfqpoint{0.003683in}{-0.013889in}}{\pgfqpoint{0.007216in}{-0.012425in}}{\pgfqpoint{0.009821in}{-0.009821in}}%
\pgfpathcurveto{\pgfqpoint{0.012425in}{-0.007216in}}{\pgfqpoint{0.013889in}{-0.003683in}}{\pgfqpoint{0.013889in}{0.000000in}}%
\pgfpathcurveto{\pgfqpoint{0.013889in}{0.003683in}}{\pgfqpoint{0.012425in}{0.007216in}}{\pgfqpoint{0.009821in}{0.009821in}}%
\pgfpathcurveto{\pgfqpoint{0.007216in}{0.012425in}}{\pgfqpoint{0.003683in}{0.013889in}}{\pgfqpoint{0.000000in}{0.013889in}}%
\pgfpathcurveto{\pgfqpoint{-0.003683in}{0.013889in}}{\pgfqpoint{-0.007216in}{0.012425in}}{\pgfqpoint{-0.009821in}{0.009821in}}%
\pgfpathcurveto{\pgfqpoint{-0.012425in}{0.007216in}}{\pgfqpoint{-0.013889in}{0.003683in}}{\pgfqpoint{-0.013889in}{0.000000in}}%
\pgfpathcurveto{\pgfqpoint{-0.013889in}{-0.003683in}}{\pgfqpoint{-0.012425in}{-0.007216in}}{\pgfqpoint{-0.009821in}{-0.009821in}}%
\pgfpathcurveto{\pgfqpoint{-0.007216in}{-0.012425in}}{\pgfqpoint{-0.003683in}{-0.013889in}}{\pgfqpoint{0.000000in}{-0.013889in}}%
\pgfpathclose%
\pgfusepath{stroke,fill}%
}%
\begin{pgfscope}%
\pgfsys@transformshift{0.627569in}{4.640906in}%
\pgfsys@useobject{currentmarker}{}%
\end{pgfscope}%
\end{pgfscope}%
\begin{pgfscope}%
\pgfpathrectangle{\pgfqpoint{0.100000in}{2.413063in}}{\pgfqpoint{5.037500in}{3.427208in}}%
\pgfusepath{clip}%
\pgfsetrectcap%
\pgfsetroundjoin%
\pgfsetlinewidth{1.505625pt}%
\definecolor{currentstroke}{rgb}{0.501961,0.501961,0.501961}%
\pgfsetstrokecolor{currentstroke}%
\pgfsetstrokeopacity{0.500000}%
\pgfsetdash{}{0pt}%
\pgfpathmoveto{\pgfqpoint{4.810896in}{5.032692in}}%
\pgfusepath{stroke}%
\end{pgfscope}%
\begin{pgfscope}%
\pgfpathrectangle{\pgfqpoint{0.100000in}{2.413063in}}{\pgfqpoint{5.037500in}{3.427208in}}%
\pgfusepath{clip}%
\pgfsetbuttcap%
\pgfsetroundjoin%
\definecolor{currentfill}{rgb}{0.501961,0.501961,0.501961}%
\pgfsetfillcolor{currentfill}%
\pgfsetfillopacity{0.500000}%
\pgfsetlinewidth{0.250937pt}%
\definecolor{currentstroke}{rgb}{0.000000,0.000000,0.000000}%
\pgfsetstrokecolor{currentstroke}%
\pgfsetstrokeopacity{0.500000}%
\pgfsetdash{}{0pt}%
\pgfsys@defobject{currentmarker}{\pgfqpoint{-0.013889in}{-0.013889in}}{\pgfqpoint{0.013889in}{0.013889in}}{%
\pgfpathmoveto{\pgfqpoint{0.000000in}{-0.013889in}}%
\pgfpathcurveto{\pgfqpoint{0.003683in}{-0.013889in}}{\pgfqpoint{0.007216in}{-0.012425in}}{\pgfqpoint{0.009821in}{-0.009821in}}%
\pgfpathcurveto{\pgfqpoint{0.012425in}{-0.007216in}}{\pgfqpoint{0.013889in}{-0.003683in}}{\pgfqpoint{0.013889in}{0.000000in}}%
\pgfpathcurveto{\pgfqpoint{0.013889in}{0.003683in}}{\pgfqpoint{0.012425in}{0.007216in}}{\pgfqpoint{0.009821in}{0.009821in}}%
\pgfpathcurveto{\pgfqpoint{0.007216in}{0.012425in}}{\pgfqpoint{0.003683in}{0.013889in}}{\pgfqpoint{0.000000in}{0.013889in}}%
\pgfpathcurveto{\pgfqpoint{-0.003683in}{0.013889in}}{\pgfqpoint{-0.007216in}{0.012425in}}{\pgfqpoint{-0.009821in}{0.009821in}}%
\pgfpathcurveto{\pgfqpoint{-0.012425in}{0.007216in}}{\pgfqpoint{-0.013889in}{0.003683in}}{\pgfqpoint{-0.013889in}{0.000000in}}%
\pgfpathcurveto{\pgfqpoint{-0.013889in}{-0.003683in}}{\pgfqpoint{-0.012425in}{-0.007216in}}{\pgfqpoint{-0.009821in}{-0.009821in}}%
\pgfpathcurveto{\pgfqpoint{-0.007216in}{-0.012425in}}{\pgfqpoint{-0.003683in}{-0.013889in}}{\pgfqpoint{0.000000in}{-0.013889in}}%
\pgfpathclose%
\pgfusepath{stroke,fill}%
}%
\begin{pgfscope}%
\pgfsys@transformshift{4.810896in}{5.032692in}%
\pgfsys@useobject{currentmarker}{}%
\end{pgfscope}%
\end{pgfscope}%
\begin{pgfscope}%
\pgfpathrectangle{\pgfqpoint{0.100000in}{2.413063in}}{\pgfqpoint{5.037500in}{3.427208in}}%
\pgfusepath{clip}%
\pgfsetrectcap%
\pgfsetroundjoin%
\pgfsetlinewidth{1.505625pt}%
\definecolor{currentstroke}{rgb}{0.501961,0.501961,0.501961}%
\pgfsetstrokecolor{currentstroke}%
\pgfsetstrokeopacity{0.500000}%
\pgfsetdash{}{0pt}%
\pgfpathmoveto{\pgfqpoint{4.769641in}{4.997446in}}%
\pgfusepath{stroke}%
\end{pgfscope}%
\begin{pgfscope}%
\pgfpathrectangle{\pgfqpoint{0.100000in}{2.413063in}}{\pgfqpoint{5.037500in}{3.427208in}}%
\pgfusepath{clip}%
\pgfsetbuttcap%
\pgfsetroundjoin%
\definecolor{currentfill}{rgb}{0.501961,0.501961,0.501961}%
\pgfsetfillcolor{currentfill}%
\pgfsetfillopacity{0.500000}%
\pgfsetlinewidth{0.250937pt}%
\definecolor{currentstroke}{rgb}{0.000000,0.000000,0.000000}%
\pgfsetstrokecolor{currentstroke}%
\pgfsetstrokeopacity{0.500000}%
\pgfsetdash{}{0pt}%
\pgfsys@defobject{currentmarker}{\pgfqpoint{-0.013889in}{-0.013889in}}{\pgfqpoint{0.013889in}{0.013889in}}{%
\pgfpathmoveto{\pgfqpoint{0.000000in}{-0.013889in}}%
\pgfpathcurveto{\pgfqpoint{0.003683in}{-0.013889in}}{\pgfqpoint{0.007216in}{-0.012425in}}{\pgfqpoint{0.009821in}{-0.009821in}}%
\pgfpathcurveto{\pgfqpoint{0.012425in}{-0.007216in}}{\pgfqpoint{0.013889in}{-0.003683in}}{\pgfqpoint{0.013889in}{0.000000in}}%
\pgfpathcurveto{\pgfqpoint{0.013889in}{0.003683in}}{\pgfqpoint{0.012425in}{0.007216in}}{\pgfqpoint{0.009821in}{0.009821in}}%
\pgfpathcurveto{\pgfqpoint{0.007216in}{0.012425in}}{\pgfqpoint{0.003683in}{0.013889in}}{\pgfqpoint{0.000000in}{0.013889in}}%
\pgfpathcurveto{\pgfqpoint{-0.003683in}{0.013889in}}{\pgfqpoint{-0.007216in}{0.012425in}}{\pgfqpoint{-0.009821in}{0.009821in}}%
\pgfpathcurveto{\pgfqpoint{-0.012425in}{0.007216in}}{\pgfqpoint{-0.013889in}{0.003683in}}{\pgfqpoint{-0.013889in}{0.000000in}}%
\pgfpathcurveto{\pgfqpoint{-0.013889in}{-0.003683in}}{\pgfqpoint{-0.012425in}{-0.007216in}}{\pgfqpoint{-0.009821in}{-0.009821in}}%
\pgfpathcurveto{\pgfqpoint{-0.007216in}{-0.012425in}}{\pgfqpoint{-0.003683in}{-0.013889in}}{\pgfqpoint{0.000000in}{-0.013889in}}%
\pgfpathclose%
\pgfusepath{stroke,fill}%
}%
\begin{pgfscope}%
\pgfsys@transformshift{4.769641in}{4.997446in}%
\pgfsys@useobject{currentmarker}{}%
\end{pgfscope}%
\end{pgfscope}%
\begin{pgfscope}%
\pgfpathrectangle{\pgfqpoint{0.100000in}{2.413063in}}{\pgfqpoint{5.037500in}{3.427208in}}%
\pgfusepath{clip}%
\pgfsetrectcap%
\pgfsetroundjoin%
\pgfsetlinewidth{1.505625pt}%
\definecolor{currentstroke}{rgb}{0.000000,0.000000,1.000000}%
\pgfsetstrokecolor{currentstroke}%
\pgfsetstrokeopacity{0.500000}%
\pgfsetdash{}{0pt}%
\pgfpathmoveto{\pgfqpoint{4.823756in}{5.021108in}}%
\pgfusepath{stroke}%
\end{pgfscope}%
\begin{pgfscope}%
\pgfpathrectangle{\pgfqpoint{0.100000in}{2.413063in}}{\pgfqpoint{5.037500in}{3.427208in}}%
\pgfusepath{clip}%
\pgfsetbuttcap%
\pgfsetroundjoin%
\definecolor{currentfill}{rgb}{0.000000,0.000000,1.000000}%
\pgfsetfillcolor{currentfill}%
\pgfsetfillopacity{0.500000}%
\pgfsetlinewidth{0.250937pt}%
\definecolor{currentstroke}{rgb}{0.000000,0.000000,0.000000}%
\pgfsetstrokecolor{currentstroke}%
\pgfsetstrokeopacity{0.500000}%
\pgfsetdash{}{0pt}%
\pgfsys@defobject{currentmarker}{\pgfqpoint{-0.005556in}{-0.005556in}}{\pgfqpoint{0.005556in}{0.005556in}}{%
\pgfpathmoveto{\pgfqpoint{0.000000in}{-0.005556in}}%
\pgfpathcurveto{\pgfqpoint{0.001473in}{-0.005556in}}{\pgfqpoint{0.002887in}{-0.004970in}}{\pgfqpoint{0.003928in}{-0.003928in}}%
\pgfpathcurveto{\pgfqpoint{0.004970in}{-0.002887in}}{\pgfqpoint{0.005556in}{-0.001473in}}{\pgfqpoint{0.005556in}{0.000000in}}%
\pgfpathcurveto{\pgfqpoint{0.005556in}{0.001473in}}{\pgfqpoint{0.004970in}{0.002887in}}{\pgfqpoint{0.003928in}{0.003928in}}%
\pgfpathcurveto{\pgfqpoint{0.002887in}{0.004970in}}{\pgfqpoint{0.001473in}{0.005556in}}{\pgfqpoint{0.000000in}{0.005556in}}%
\pgfpathcurveto{\pgfqpoint{-0.001473in}{0.005556in}}{\pgfqpoint{-0.002887in}{0.004970in}}{\pgfqpoint{-0.003928in}{0.003928in}}%
\pgfpathcurveto{\pgfqpoint{-0.004970in}{0.002887in}}{\pgfqpoint{-0.005556in}{0.001473in}}{\pgfqpoint{-0.005556in}{0.000000in}}%
\pgfpathcurveto{\pgfqpoint{-0.005556in}{-0.001473in}}{\pgfqpoint{-0.004970in}{-0.002887in}}{\pgfqpoint{-0.003928in}{-0.003928in}}%
\pgfpathcurveto{\pgfqpoint{-0.002887in}{-0.004970in}}{\pgfqpoint{-0.001473in}{-0.005556in}}{\pgfqpoint{0.000000in}{-0.005556in}}%
\pgfpathclose%
\pgfusepath{stroke,fill}%
}%
\begin{pgfscope}%
\pgfsys@transformshift{4.823756in}{5.021108in}%
\pgfsys@useobject{currentmarker}{}%
\end{pgfscope}%
\end{pgfscope}%
\begin{pgfscope}%
\pgfpathrectangle{\pgfqpoint{0.100000in}{2.413063in}}{\pgfqpoint{5.037500in}{3.427208in}}%
\pgfusepath{clip}%
\pgfsetrectcap%
\pgfsetroundjoin%
\pgfsetlinewidth{1.505625pt}%
\definecolor{currentstroke}{rgb}{0.000000,0.000000,1.000000}%
\pgfsetstrokecolor{currentstroke}%
\pgfsetstrokeopacity{0.500000}%
\pgfsetdash{}{0pt}%
\pgfpathmoveto{\pgfqpoint{4.620381in}{4.529404in}}%
\pgfusepath{stroke}%
\end{pgfscope}%
\begin{pgfscope}%
\pgfpathrectangle{\pgfqpoint{0.100000in}{2.413063in}}{\pgfqpoint{5.037500in}{3.427208in}}%
\pgfusepath{clip}%
\pgfsetbuttcap%
\pgfsetroundjoin%
\definecolor{currentfill}{rgb}{0.000000,0.000000,1.000000}%
\pgfsetfillcolor{currentfill}%
\pgfsetfillopacity{0.500000}%
\pgfsetlinewidth{0.250937pt}%
\definecolor{currentstroke}{rgb}{0.000000,0.000000,0.000000}%
\pgfsetstrokecolor{currentstroke}%
\pgfsetstrokeopacity{0.500000}%
\pgfsetdash{}{0pt}%
\pgfsys@defobject{currentmarker}{\pgfqpoint{-0.069444in}{-0.069444in}}{\pgfqpoint{0.069444in}{0.069444in}}{%
\pgfpathmoveto{\pgfqpoint{0.000000in}{-0.069444in}}%
\pgfpathcurveto{\pgfqpoint{0.018417in}{-0.069444in}}{\pgfqpoint{0.036082in}{-0.062127in}}{\pgfqpoint{0.049105in}{-0.049105in}}%
\pgfpathcurveto{\pgfqpoint{0.062127in}{-0.036082in}}{\pgfqpoint{0.069444in}{-0.018417in}}{\pgfqpoint{0.069444in}{0.000000in}}%
\pgfpathcurveto{\pgfqpoint{0.069444in}{0.018417in}}{\pgfqpoint{0.062127in}{0.036082in}}{\pgfqpoint{0.049105in}{0.049105in}}%
\pgfpathcurveto{\pgfqpoint{0.036082in}{0.062127in}}{\pgfqpoint{0.018417in}{0.069444in}}{\pgfqpoint{0.000000in}{0.069444in}}%
\pgfpathcurveto{\pgfqpoint{-0.018417in}{0.069444in}}{\pgfqpoint{-0.036082in}{0.062127in}}{\pgfqpoint{-0.049105in}{0.049105in}}%
\pgfpathcurveto{\pgfqpoint{-0.062127in}{0.036082in}}{\pgfqpoint{-0.069444in}{0.018417in}}{\pgfqpoint{-0.069444in}{0.000000in}}%
\pgfpathcurveto{\pgfqpoint{-0.069444in}{-0.018417in}}{\pgfqpoint{-0.062127in}{-0.036082in}}{\pgfqpoint{-0.049105in}{-0.049105in}}%
\pgfpathcurveto{\pgfqpoint{-0.036082in}{-0.062127in}}{\pgfqpoint{-0.018417in}{-0.069444in}}{\pgfqpoint{0.000000in}{-0.069444in}}%
\pgfpathclose%
\pgfusepath{stroke,fill}%
}%
\begin{pgfscope}%
\pgfsys@transformshift{4.620381in}{4.529404in}%
\pgfsys@useobject{currentmarker}{}%
\end{pgfscope}%
\end{pgfscope}%
\begin{pgfscope}%
\pgfpathrectangle{\pgfqpoint{0.100000in}{2.413063in}}{\pgfqpoint{5.037500in}{3.427208in}}%
\pgfusepath{clip}%
\pgfsetrectcap%
\pgfsetroundjoin%
\pgfsetlinewidth{1.505625pt}%
\definecolor{currentstroke}{rgb}{0.678431,1.000000,0.184314}%
\pgfsetstrokecolor{currentstroke}%
\pgfsetstrokeopacity{0.500000}%
\pgfsetdash{}{0pt}%
\pgfpathmoveto{\pgfqpoint{4.609445in}{4.516655in}}%
\pgfusepath{stroke}%
\end{pgfscope}%
\begin{pgfscope}%
\pgfpathrectangle{\pgfqpoint{0.100000in}{2.413063in}}{\pgfqpoint{5.037500in}{3.427208in}}%
\pgfusepath{clip}%
\pgfsetbuttcap%
\pgfsetroundjoin%
\definecolor{currentfill}{rgb}{0.678431,1.000000,0.184314}%
\pgfsetfillcolor{currentfill}%
\pgfsetfillopacity{0.500000}%
\pgfsetlinewidth{0.250937pt}%
\definecolor{currentstroke}{rgb}{0.000000,0.000000,0.000000}%
\pgfsetstrokecolor{currentstroke}%
\pgfsetstrokeopacity{0.500000}%
\pgfsetdash{}{0pt}%
\pgfsys@defobject{currentmarker}{\pgfqpoint{-0.025000in}{-0.025000in}}{\pgfqpoint{0.025000in}{0.025000in}}{%
\pgfpathmoveto{\pgfqpoint{0.000000in}{-0.025000in}}%
\pgfpathcurveto{\pgfqpoint{0.006630in}{-0.025000in}}{\pgfqpoint{0.012989in}{-0.022366in}}{\pgfqpoint{0.017678in}{-0.017678in}}%
\pgfpathcurveto{\pgfqpoint{0.022366in}{-0.012989in}}{\pgfqpoint{0.025000in}{-0.006630in}}{\pgfqpoint{0.025000in}{0.000000in}}%
\pgfpathcurveto{\pgfqpoint{0.025000in}{0.006630in}}{\pgfqpoint{0.022366in}{0.012989in}}{\pgfqpoint{0.017678in}{0.017678in}}%
\pgfpathcurveto{\pgfqpoint{0.012989in}{0.022366in}}{\pgfqpoint{0.006630in}{0.025000in}}{\pgfqpoint{0.000000in}{0.025000in}}%
\pgfpathcurveto{\pgfqpoint{-0.006630in}{0.025000in}}{\pgfqpoint{-0.012989in}{0.022366in}}{\pgfqpoint{-0.017678in}{0.017678in}}%
\pgfpathcurveto{\pgfqpoint{-0.022366in}{0.012989in}}{\pgfqpoint{-0.025000in}{0.006630in}}{\pgfqpoint{-0.025000in}{0.000000in}}%
\pgfpathcurveto{\pgfqpoint{-0.025000in}{-0.006630in}}{\pgfqpoint{-0.022366in}{-0.012989in}}{\pgfqpoint{-0.017678in}{-0.017678in}}%
\pgfpathcurveto{\pgfqpoint{-0.012989in}{-0.022366in}}{\pgfqpoint{-0.006630in}{-0.025000in}}{\pgfqpoint{0.000000in}{-0.025000in}}%
\pgfpathclose%
\pgfusepath{stroke,fill}%
}%
\begin{pgfscope}%
\pgfsys@transformshift{4.609445in}{4.516655in}%
\pgfsys@useobject{currentmarker}{}%
\end{pgfscope}%
\end{pgfscope}%
\begin{pgfscope}%
\pgfpathrectangle{\pgfqpoint{0.100000in}{2.413063in}}{\pgfqpoint{5.037500in}{3.427208in}}%
\pgfusepath{clip}%
\pgfsetrectcap%
\pgfsetroundjoin%
\pgfsetlinewidth{1.505625pt}%
\definecolor{currentstroke}{rgb}{0.000000,0.000000,1.000000}%
\pgfsetstrokecolor{currentstroke}%
\pgfsetstrokeopacity{0.500000}%
\pgfsetdash{}{0pt}%
\pgfpathmoveto{\pgfqpoint{4.570899in}{4.618831in}}%
\pgfusepath{stroke}%
\end{pgfscope}%
\begin{pgfscope}%
\pgfpathrectangle{\pgfqpoint{0.100000in}{2.413063in}}{\pgfqpoint{5.037500in}{3.427208in}}%
\pgfusepath{clip}%
\pgfsetbuttcap%
\pgfsetroundjoin%
\definecolor{currentfill}{rgb}{0.000000,0.000000,1.000000}%
\pgfsetfillcolor{currentfill}%
\pgfsetfillopacity{0.500000}%
\pgfsetlinewidth{0.250937pt}%
\definecolor{currentstroke}{rgb}{0.000000,0.000000,0.000000}%
\pgfsetstrokecolor{currentstroke}%
\pgfsetstrokeopacity{0.500000}%
\pgfsetdash{}{0pt}%
\pgfsys@defobject{currentmarker}{\pgfqpoint{-0.027778in}{-0.027778in}}{\pgfqpoint{0.027778in}{0.027778in}}{%
\pgfpathmoveto{\pgfqpoint{0.000000in}{-0.027778in}}%
\pgfpathcurveto{\pgfqpoint{0.007367in}{-0.027778in}}{\pgfqpoint{0.014433in}{-0.024851in}}{\pgfqpoint{0.019642in}{-0.019642in}}%
\pgfpathcurveto{\pgfqpoint{0.024851in}{-0.014433in}}{\pgfqpoint{0.027778in}{-0.007367in}}{\pgfqpoint{0.027778in}{0.000000in}}%
\pgfpathcurveto{\pgfqpoint{0.027778in}{0.007367in}}{\pgfqpoint{0.024851in}{0.014433in}}{\pgfqpoint{0.019642in}{0.019642in}}%
\pgfpathcurveto{\pgfqpoint{0.014433in}{0.024851in}}{\pgfqpoint{0.007367in}{0.027778in}}{\pgfqpoint{0.000000in}{0.027778in}}%
\pgfpathcurveto{\pgfqpoint{-0.007367in}{0.027778in}}{\pgfqpoint{-0.014433in}{0.024851in}}{\pgfqpoint{-0.019642in}{0.019642in}}%
\pgfpathcurveto{\pgfqpoint{-0.024851in}{0.014433in}}{\pgfqpoint{-0.027778in}{0.007367in}}{\pgfqpoint{-0.027778in}{0.000000in}}%
\pgfpathcurveto{\pgfqpoint{-0.027778in}{-0.007367in}}{\pgfqpoint{-0.024851in}{-0.014433in}}{\pgfqpoint{-0.019642in}{-0.019642in}}%
\pgfpathcurveto{\pgfqpoint{-0.014433in}{-0.024851in}}{\pgfqpoint{-0.007367in}{-0.027778in}}{\pgfqpoint{0.000000in}{-0.027778in}}%
\pgfpathclose%
\pgfusepath{stroke,fill}%
}%
\begin{pgfscope}%
\pgfsys@transformshift{4.570899in}{4.618831in}%
\pgfsys@useobject{currentmarker}{}%
\end{pgfscope}%
\end{pgfscope}%
\begin{pgfscope}%
\pgfpathrectangle{\pgfqpoint{0.100000in}{2.413063in}}{\pgfqpoint{5.037500in}{3.427208in}}%
\pgfusepath{clip}%
\pgfsetrectcap%
\pgfsetroundjoin%
\pgfsetlinewidth{1.505625pt}%
\definecolor{currentstroke}{rgb}{0.000000,0.000000,1.000000}%
\pgfsetstrokecolor{currentstroke}%
\pgfsetstrokeopacity{0.500000}%
\pgfsetdash{}{0pt}%
\pgfpathmoveto{\pgfqpoint{4.567371in}{4.530224in}}%
\pgfusepath{stroke}%
\end{pgfscope}%
\begin{pgfscope}%
\pgfpathrectangle{\pgfqpoint{0.100000in}{2.413063in}}{\pgfqpoint{5.037500in}{3.427208in}}%
\pgfusepath{clip}%
\pgfsetbuttcap%
\pgfsetroundjoin%
\definecolor{currentfill}{rgb}{0.000000,0.000000,1.000000}%
\pgfsetfillcolor{currentfill}%
\pgfsetfillopacity{0.500000}%
\pgfsetlinewidth{0.250937pt}%
\definecolor{currentstroke}{rgb}{0.000000,0.000000,0.000000}%
\pgfsetstrokecolor{currentstroke}%
\pgfsetstrokeopacity{0.500000}%
\pgfsetdash{}{0pt}%
\pgfsys@defobject{currentmarker}{\pgfqpoint{-0.033333in}{-0.033333in}}{\pgfqpoint{0.033333in}{0.033333in}}{%
\pgfpathmoveto{\pgfqpoint{0.000000in}{-0.033333in}}%
\pgfpathcurveto{\pgfqpoint{0.008840in}{-0.033333in}}{\pgfqpoint{0.017319in}{-0.029821in}}{\pgfqpoint{0.023570in}{-0.023570in}}%
\pgfpathcurveto{\pgfqpoint{0.029821in}{-0.017319in}}{\pgfqpoint{0.033333in}{-0.008840in}}{\pgfqpoint{0.033333in}{0.000000in}}%
\pgfpathcurveto{\pgfqpoint{0.033333in}{0.008840in}}{\pgfqpoint{0.029821in}{0.017319in}}{\pgfqpoint{0.023570in}{0.023570in}}%
\pgfpathcurveto{\pgfqpoint{0.017319in}{0.029821in}}{\pgfqpoint{0.008840in}{0.033333in}}{\pgfqpoint{0.000000in}{0.033333in}}%
\pgfpathcurveto{\pgfqpoint{-0.008840in}{0.033333in}}{\pgfqpoint{-0.017319in}{0.029821in}}{\pgfqpoint{-0.023570in}{0.023570in}}%
\pgfpathcurveto{\pgfqpoint{-0.029821in}{0.017319in}}{\pgfqpoint{-0.033333in}{0.008840in}}{\pgfqpoint{-0.033333in}{0.000000in}}%
\pgfpathcurveto{\pgfqpoint{-0.033333in}{-0.008840in}}{\pgfqpoint{-0.029821in}{-0.017319in}}{\pgfqpoint{-0.023570in}{-0.023570in}}%
\pgfpathcurveto{\pgfqpoint{-0.017319in}{-0.029821in}}{\pgfqpoint{-0.008840in}{-0.033333in}}{\pgfqpoint{0.000000in}{-0.033333in}}%
\pgfpathclose%
\pgfusepath{stroke,fill}%
}%
\begin{pgfscope}%
\pgfsys@transformshift{4.567371in}{4.530224in}%
\pgfsys@useobject{currentmarker}{}%
\end{pgfscope}%
\end{pgfscope}%
\begin{pgfscope}%
\pgfpathrectangle{\pgfqpoint{0.100000in}{2.413063in}}{\pgfqpoint{5.037500in}{3.427208in}}%
\pgfusepath{clip}%
\pgfsetrectcap%
\pgfsetroundjoin%
\pgfsetlinewidth{1.505625pt}%
\definecolor{currentstroke}{rgb}{0.000000,0.000000,1.000000}%
\pgfsetstrokecolor{currentstroke}%
\pgfsetstrokeopacity{0.500000}%
\pgfsetdash{}{0pt}%
\pgfpathmoveto{\pgfqpoint{1.706681in}{3.900122in}}%
\pgfusepath{stroke}%
\end{pgfscope}%
\begin{pgfscope}%
\pgfpathrectangle{\pgfqpoint{0.100000in}{2.413063in}}{\pgfqpoint{5.037500in}{3.427208in}}%
\pgfusepath{clip}%
\pgfsetbuttcap%
\pgfsetroundjoin%
\definecolor{currentfill}{rgb}{0.000000,0.000000,1.000000}%
\pgfsetfillcolor{currentfill}%
\pgfsetfillopacity{0.500000}%
\pgfsetlinewidth{0.250937pt}%
\definecolor{currentstroke}{rgb}{0.000000,0.000000,0.000000}%
\pgfsetstrokecolor{currentstroke}%
\pgfsetstrokeopacity{0.500000}%
\pgfsetdash{}{0pt}%
\pgfsys@defobject{currentmarker}{\pgfqpoint{-0.008333in}{-0.008333in}}{\pgfqpoint{0.008333in}{0.008333in}}{%
\pgfpathmoveto{\pgfqpoint{0.000000in}{-0.008333in}}%
\pgfpathcurveto{\pgfqpoint{0.002210in}{-0.008333in}}{\pgfqpoint{0.004330in}{-0.007455in}}{\pgfqpoint{0.005893in}{-0.005893in}}%
\pgfpathcurveto{\pgfqpoint{0.007455in}{-0.004330in}}{\pgfqpoint{0.008333in}{-0.002210in}}{\pgfqpoint{0.008333in}{0.000000in}}%
\pgfpathcurveto{\pgfqpoint{0.008333in}{0.002210in}}{\pgfqpoint{0.007455in}{0.004330in}}{\pgfqpoint{0.005893in}{0.005893in}}%
\pgfpathcurveto{\pgfqpoint{0.004330in}{0.007455in}}{\pgfqpoint{0.002210in}{0.008333in}}{\pgfqpoint{0.000000in}{0.008333in}}%
\pgfpathcurveto{\pgfqpoint{-0.002210in}{0.008333in}}{\pgfqpoint{-0.004330in}{0.007455in}}{\pgfqpoint{-0.005893in}{0.005893in}}%
\pgfpathcurveto{\pgfqpoint{-0.007455in}{0.004330in}}{\pgfqpoint{-0.008333in}{0.002210in}}{\pgfqpoint{-0.008333in}{0.000000in}}%
\pgfpathcurveto{\pgfqpoint{-0.008333in}{-0.002210in}}{\pgfqpoint{-0.007455in}{-0.004330in}}{\pgfqpoint{-0.005893in}{-0.005893in}}%
\pgfpathcurveto{\pgfqpoint{-0.004330in}{-0.007455in}}{\pgfqpoint{-0.002210in}{-0.008333in}}{\pgfqpoint{0.000000in}{-0.008333in}}%
\pgfpathclose%
\pgfusepath{stroke,fill}%
}%
\begin{pgfscope}%
\pgfsys@transformshift{1.706681in}{3.900122in}%
\pgfsys@useobject{currentmarker}{}%
\end{pgfscope}%
\end{pgfscope}%
\begin{pgfscope}%
\pgfpathrectangle{\pgfqpoint{0.100000in}{2.413063in}}{\pgfqpoint{5.037500in}{3.427208in}}%
\pgfusepath{clip}%
\pgfsetrectcap%
\pgfsetroundjoin%
\pgfsetlinewidth{1.505625pt}%
\definecolor{currentstroke}{rgb}{0.501961,0.501961,0.501961}%
\pgfsetstrokecolor{currentstroke}%
\pgfsetstrokeopacity{0.500000}%
\pgfsetdash{}{0pt}%
\pgfpathmoveto{\pgfqpoint{1.588407in}{4.108047in}}%
\pgfusepath{stroke}%
\end{pgfscope}%
\begin{pgfscope}%
\pgfpathrectangle{\pgfqpoint{0.100000in}{2.413063in}}{\pgfqpoint{5.037500in}{3.427208in}}%
\pgfusepath{clip}%
\pgfsetbuttcap%
\pgfsetroundjoin%
\definecolor{currentfill}{rgb}{0.501961,0.501961,0.501961}%
\pgfsetfillcolor{currentfill}%
\pgfsetfillopacity{0.500000}%
\pgfsetlinewidth{0.250937pt}%
\definecolor{currentstroke}{rgb}{0.000000,0.000000,0.000000}%
\pgfsetstrokecolor{currentstroke}%
\pgfsetstrokeopacity{0.500000}%
\pgfsetdash{}{0pt}%
\pgfsys@defobject{currentmarker}{\pgfqpoint{-0.013889in}{-0.013889in}}{\pgfqpoint{0.013889in}{0.013889in}}{%
\pgfpathmoveto{\pgfqpoint{0.000000in}{-0.013889in}}%
\pgfpathcurveto{\pgfqpoint{0.003683in}{-0.013889in}}{\pgfqpoint{0.007216in}{-0.012425in}}{\pgfqpoint{0.009821in}{-0.009821in}}%
\pgfpathcurveto{\pgfqpoint{0.012425in}{-0.007216in}}{\pgfqpoint{0.013889in}{-0.003683in}}{\pgfqpoint{0.013889in}{0.000000in}}%
\pgfpathcurveto{\pgfqpoint{0.013889in}{0.003683in}}{\pgfqpoint{0.012425in}{0.007216in}}{\pgfqpoint{0.009821in}{0.009821in}}%
\pgfpathcurveto{\pgfqpoint{0.007216in}{0.012425in}}{\pgfqpoint{0.003683in}{0.013889in}}{\pgfqpoint{0.000000in}{0.013889in}}%
\pgfpathcurveto{\pgfqpoint{-0.003683in}{0.013889in}}{\pgfqpoint{-0.007216in}{0.012425in}}{\pgfqpoint{-0.009821in}{0.009821in}}%
\pgfpathcurveto{\pgfqpoint{-0.012425in}{0.007216in}}{\pgfqpoint{-0.013889in}{0.003683in}}{\pgfqpoint{-0.013889in}{0.000000in}}%
\pgfpathcurveto{\pgfqpoint{-0.013889in}{-0.003683in}}{\pgfqpoint{-0.012425in}{-0.007216in}}{\pgfqpoint{-0.009821in}{-0.009821in}}%
\pgfpathcurveto{\pgfqpoint{-0.007216in}{-0.012425in}}{\pgfqpoint{-0.003683in}{-0.013889in}}{\pgfqpoint{0.000000in}{-0.013889in}}%
\pgfpathclose%
\pgfusepath{stroke,fill}%
}%
\begin{pgfscope}%
\pgfsys@transformshift{1.588407in}{4.108047in}%
\pgfsys@useobject{currentmarker}{}%
\end{pgfscope}%
\end{pgfscope}%
\begin{pgfscope}%
\pgfpathrectangle{\pgfqpoint{0.100000in}{2.413063in}}{\pgfqpoint{5.037500in}{3.427208in}}%
\pgfusepath{clip}%
\pgfsetrectcap%
\pgfsetroundjoin%
\pgfsetlinewidth{1.505625pt}%
\definecolor{currentstroke}{rgb}{0.678431,1.000000,0.184314}%
\pgfsetstrokecolor{currentstroke}%
\pgfsetstrokeopacity{0.500000}%
\pgfsetdash{}{0pt}%
\pgfpathmoveto{\pgfqpoint{1.653203in}{3.583998in}}%
\pgfusepath{stroke}%
\end{pgfscope}%
\begin{pgfscope}%
\pgfpathrectangle{\pgfqpoint{0.100000in}{2.413063in}}{\pgfqpoint{5.037500in}{3.427208in}}%
\pgfusepath{clip}%
\pgfsetbuttcap%
\pgfsetroundjoin%
\definecolor{currentfill}{rgb}{0.678431,1.000000,0.184314}%
\pgfsetfillcolor{currentfill}%
\pgfsetfillopacity{0.500000}%
\pgfsetlinewidth{0.250937pt}%
\definecolor{currentstroke}{rgb}{0.000000,0.000000,0.000000}%
\pgfsetstrokecolor{currentstroke}%
\pgfsetstrokeopacity{0.500000}%
\pgfsetdash{}{0pt}%
\pgfsys@defobject{currentmarker}{\pgfqpoint{-0.016667in}{-0.016667in}}{\pgfqpoint{0.016667in}{0.016667in}}{%
\pgfpathmoveto{\pgfqpoint{0.000000in}{-0.016667in}}%
\pgfpathcurveto{\pgfqpoint{0.004420in}{-0.016667in}}{\pgfqpoint{0.008660in}{-0.014911in}}{\pgfqpoint{0.011785in}{-0.011785in}}%
\pgfpathcurveto{\pgfqpoint{0.014911in}{-0.008660in}}{\pgfqpoint{0.016667in}{-0.004420in}}{\pgfqpoint{0.016667in}{0.000000in}}%
\pgfpathcurveto{\pgfqpoint{0.016667in}{0.004420in}}{\pgfqpoint{0.014911in}{0.008660in}}{\pgfqpoint{0.011785in}{0.011785in}}%
\pgfpathcurveto{\pgfqpoint{0.008660in}{0.014911in}}{\pgfqpoint{0.004420in}{0.016667in}}{\pgfqpoint{0.000000in}{0.016667in}}%
\pgfpathcurveto{\pgfqpoint{-0.004420in}{0.016667in}}{\pgfqpoint{-0.008660in}{0.014911in}}{\pgfqpoint{-0.011785in}{0.011785in}}%
\pgfpathcurveto{\pgfqpoint{-0.014911in}{0.008660in}}{\pgfqpoint{-0.016667in}{0.004420in}}{\pgfqpoint{-0.016667in}{0.000000in}}%
\pgfpathcurveto{\pgfqpoint{-0.016667in}{-0.004420in}}{\pgfqpoint{-0.014911in}{-0.008660in}}{\pgfqpoint{-0.011785in}{-0.011785in}}%
\pgfpathcurveto{\pgfqpoint{-0.008660in}{-0.014911in}}{\pgfqpoint{-0.004420in}{-0.016667in}}{\pgfqpoint{0.000000in}{-0.016667in}}%
\pgfpathclose%
\pgfusepath{stroke,fill}%
}%
\begin{pgfscope}%
\pgfsys@transformshift{1.653203in}{3.583998in}%
\pgfsys@useobject{currentmarker}{}%
\end{pgfscope}%
\end{pgfscope}%
\begin{pgfscope}%
\pgfpathrectangle{\pgfqpoint{0.100000in}{2.413063in}}{\pgfqpoint{5.037500in}{3.427208in}}%
\pgfusepath{clip}%
\pgfsetrectcap%
\pgfsetroundjoin%
\pgfsetlinewidth{1.505625pt}%
\definecolor{currentstroke}{rgb}{0.000000,0.000000,1.000000}%
\pgfsetstrokecolor{currentstroke}%
\pgfsetstrokeopacity{0.500000}%
\pgfsetdash{}{0pt}%
\pgfpathmoveto{\pgfqpoint{1.781876in}{3.960833in}}%
\pgfusepath{stroke}%
\end{pgfscope}%
\begin{pgfscope}%
\pgfpathrectangle{\pgfqpoint{0.100000in}{2.413063in}}{\pgfqpoint{5.037500in}{3.427208in}}%
\pgfusepath{clip}%
\pgfsetbuttcap%
\pgfsetroundjoin%
\definecolor{currentfill}{rgb}{0.000000,0.000000,1.000000}%
\pgfsetfillcolor{currentfill}%
\pgfsetfillopacity{0.500000}%
\pgfsetlinewidth{0.250937pt}%
\definecolor{currentstroke}{rgb}{0.000000,0.000000,0.000000}%
\pgfsetstrokecolor{currentstroke}%
\pgfsetstrokeopacity{0.500000}%
\pgfsetdash{}{0pt}%
\pgfsys@defobject{currentmarker}{\pgfqpoint{-0.022222in}{-0.022222in}}{\pgfqpoint{0.022222in}{0.022222in}}{%
\pgfpathmoveto{\pgfqpoint{0.000000in}{-0.022222in}}%
\pgfpathcurveto{\pgfqpoint{0.005893in}{-0.022222in}}{\pgfqpoint{0.011546in}{-0.019881in}}{\pgfqpoint{0.015713in}{-0.015713in}}%
\pgfpathcurveto{\pgfqpoint{0.019881in}{-0.011546in}}{\pgfqpoint{0.022222in}{-0.005893in}}{\pgfqpoint{0.022222in}{0.000000in}}%
\pgfpathcurveto{\pgfqpoint{0.022222in}{0.005893in}}{\pgfqpoint{0.019881in}{0.011546in}}{\pgfqpoint{0.015713in}{0.015713in}}%
\pgfpathcurveto{\pgfqpoint{0.011546in}{0.019881in}}{\pgfqpoint{0.005893in}{0.022222in}}{\pgfqpoint{0.000000in}{0.022222in}}%
\pgfpathcurveto{\pgfqpoint{-0.005893in}{0.022222in}}{\pgfqpoint{-0.011546in}{0.019881in}}{\pgfqpoint{-0.015713in}{0.015713in}}%
\pgfpathcurveto{\pgfqpoint{-0.019881in}{0.011546in}}{\pgfqpoint{-0.022222in}{0.005893in}}{\pgfqpoint{-0.022222in}{0.000000in}}%
\pgfpathcurveto{\pgfqpoint{-0.022222in}{-0.005893in}}{\pgfqpoint{-0.019881in}{-0.011546in}}{\pgfqpoint{-0.015713in}{-0.015713in}}%
\pgfpathcurveto{\pgfqpoint{-0.011546in}{-0.019881in}}{\pgfqpoint{-0.005893in}{-0.022222in}}{\pgfqpoint{0.000000in}{-0.022222in}}%
\pgfpathclose%
\pgfusepath{stroke,fill}%
}%
\begin{pgfscope}%
\pgfsys@transformshift{1.781876in}{3.960833in}%
\pgfsys@useobject{currentmarker}{}%
\end{pgfscope}%
\end{pgfscope}%
\begin{pgfscope}%
\pgfpathrectangle{\pgfqpoint{0.100000in}{2.413063in}}{\pgfqpoint{5.037500in}{3.427208in}}%
\pgfusepath{clip}%
\pgfsetrectcap%
\pgfsetroundjoin%
\pgfsetlinewidth{1.505625pt}%
\definecolor{currentstroke}{rgb}{0.678431,1.000000,0.184314}%
\pgfsetstrokecolor{currentstroke}%
\pgfsetstrokeopacity{0.500000}%
\pgfsetdash{}{0pt}%
\pgfpathmoveto{\pgfqpoint{4.590541in}{4.911408in}}%
\pgfusepath{stroke}%
\end{pgfscope}%
\begin{pgfscope}%
\pgfpathrectangle{\pgfqpoint{0.100000in}{2.413063in}}{\pgfqpoint{5.037500in}{3.427208in}}%
\pgfusepath{clip}%
\pgfsetbuttcap%
\pgfsetroundjoin%
\definecolor{currentfill}{rgb}{0.678431,1.000000,0.184314}%
\pgfsetfillcolor{currentfill}%
\pgfsetfillopacity{0.500000}%
\pgfsetlinewidth{0.250937pt}%
\definecolor{currentstroke}{rgb}{0.000000,0.000000,0.000000}%
\pgfsetstrokecolor{currentstroke}%
\pgfsetstrokeopacity{0.500000}%
\pgfsetdash{}{0pt}%
\pgfsys@defobject{currentmarker}{\pgfqpoint{-0.027778in}{-0.027778in}}{\pgfqpoint{0.027778in}{0.027778in}}{%
\pgfpathmoveto{\pgfqpoint{0.000000in}{-0.027778in}}%
\pgfpathcurveto{\pgfqpoint{0.007367in}{-0.027778in}}{\pgfqpoint{0.014433in}{-0.024851in}}{\pgfqpoint{0.019642in}{-0.019642in}}%
\pgfpathcurveto{\pgfqpoint{0.024851in}{-0.014433in}}{\pgfqpoint{0.027778in}{-0.007367in}}{\pgfqpoint{0.027778in}{0.000000in}}%
\pgfpathcurveto{\pgfqpoint{0.027778in}{0.007367in}}{\pgfqpoint{0.024851in}{0.014433in}}{\pgfqpoint{0.019642in}{0.019642in}}%
\pgfpathcurveto{\pgfqpoint{0.014433in}{0.024851in}}{\pgfqpoint{0.007367in}{0.027778in}}{\pgfqpoint{0.000000in}{0.027778in}}%
\pgfpathcurveto{\pgfqpoint{-0.007367in}{0.027778in}}{\pgfqpoint{-0.014433in}{0.024851in}}{\pgfqpoint{-0.019642in}{0.019642in}}%
\pgfpathcurveto{\pgfqpoint{-0.024851in}{0.014433in}}{\pgfqpoint{-0.027778in}{0.007367in}}{\pgfqpoint{-0.027778in}{0.000000in}}%
\pgfpathcurveto{\pgfqpoint{-0.027778in}{-0.007367in}}{\pgfqpoint{-0.024851in}{-0.014433in}}{\pgfqpoint{-0.019642in}{-0.019642in}}%
\pgfpathcurveto{\pgfqpoint{-0.014433in}{-0.024851in}}{\pgfqpoint{-0.007367in}{-0.027778in}}{\pgfqpoint{0.000000in}{-0.027778in}}%
\pgfpathclose%
\pgfusepath{stroke,fill}%
}%
\begin{pgfscope}%
\pgfsys@transformshift{4.590541in}{4.911408in}%
\pgfsys@useobject{currentmarker}{}%
\end{pgfscope}%
\end{pgfscope}%
\begin{pgfscope}%
\pgfpathrectangle{\pgfqpoint{0.100000in}{2.413063in}}{\pgfqpoint{5.037500in}{3.427208in}}%
\pgfusepath{clip}%
\pgfsetrectcap%
\pgfsetroundjoin%
\pgfsetlinewidth{1.505625pt}%
\definecolor{currentstroke}{rgb}{0.678431,1.000000,0.184314}%
\pgfsetstrokecolor{currentstroke}%
\pgfsetstrokeopacity{0.500000}%
\pgfsetdash{}{0pt}%
\pgfpathmoveto{\pgfqpoint{4.425126in}{4.808565in}}%
\pgfusepath{stroke}%
\end{pgfscope}%
\begin{pgfscope}%
\pgfpathrectangle{\pgfqpoint{0.100000in}{2.413063in}}{\pgfqpoint{5.037500in}{3.427208in}}%
\pgfusepath{clip}%
\pgfsetbuttcap%
\pgfsetroundjoin%
\definecolor{currentfill}{rgb}{0.678431,1.000000,0.184314}%
\pgfsetfillcolor{currentfill}%
\pgfsetfillopacity{0.500000}%
\pgfsetlinewidth{0.250937pt}%
\definecolor{currentstroke}{rgb}{0.000000,0.000000,0.000000}%
\pgfsetstrokecolor{currentstroke}%
\pgfsetstrokeopacity{0.500000}%
\pgfsetdash{}{0pt}%
\pgfsys@defobject{currentmarker}{\pgfqpoint{-0.041667in}{-0.041667in}}{\pgfqpoint{0.041667in}{0.041667in}}{%
\pgfpathmoveto{\pgfqpoint{0.000000in}{-0.041667in}}%
\pgfpathcurveto{\pgfqpoint{0.011050in}{-0.041667in}}{\pgfqpoint{0.021649in}{-0.037276in}}{\pgfqpoint{0.029463in}{-0.029463in}}%
\pgfpathcurveto{\pgfqpoint{0.037276in}{-0.021649in}}{\pgfqpoint{0.041667in}{-0.011050in}}{\pgfqpoint{0.041667in}{0.000000in}}%
\pgfpathcurveto{\pgfqpoint{0.041667in}{0.011050in}}{\pgfqpoint{0.037276in}{0.021649in}}{\pgfqpoint{0.029463in}{0.029463in}}%
\pgfpathcurveto{\pgfqpoint{0.021649in}{0.037276in}}{\pgfqpoint{0.011050in}{0.041667in}}{\pgfqpoint{0.000000in}{0.041667in}}%
\pgfpathcurveto{\pgfqpoint{-0.011050in}{0.041667in}}{\pgfqpoint{-0.021649in}{0.037276in}}{\pgfqpoint{-0.029463in}{0.029463in}}%
\pgfpathcurveto{\pgfqpoint{-0.037276in}{0.021649in}}{\pgfqpoint{-0.041667in}{0.011050in}}{\pgfqpoint{-0.041667in}{0.000000in}}%
\pgfpathcurveto{\pgfqpoint{-0.041667in}{-0.011050in}}{\pgfqpoint{-0.037276in}{-0.021649in}}{\pgfqpoint{-0.029463in}{-0.029463in}}%
\pgfpathcurveto{\pgfqpoint{-0.021649in}{-0.037276in}}{\pgfqpoint{-0.011050in}{-0.041667in}}{\pgfqpoint{0.000000in}{-0.041667in}}%
\pgfpathclose%
\pgfusepath{stroke,fill}%
}%
\begin{pgfscope}%
\pgfsys@transformshift{4.425126in}{4.808565in}%
\pgfsys@useobject{currentmarker}{}%
\end{pgfscope}%
\end{pgfscope}%
\begin{pgfscope}%
\pgfpathrectangle{\pgfqpoint{0.100000in}{2.413063in}}{\pgfqpoint{5.037500in}{3.427208in}}%
\pgfusepath{clip}%
\pgfsetrectcap%
\pgfsetroundjoin%
\pgfsetlinewidth{1.505625pt}%
\definecolor{currentstroke}{rgb}{0.678431,1.000000,0.184314}%
\pgfsetstrokecolor{currentstroke}%
\pgfsetstrokeopacity{0.500000}%
\pgfsetdash{}{0pt}%
\pgfpathmoveto{\pgfqpoint{4.160198in}{4.849420in}}%
\pgfusepath{stroke}%
\end{pgfscope}%
\begin{pgfscope}%
\pgfpathrectangle{\pgfqpoint{0.100000in}{2.413063in}}{\pgfqpoint{5.037500in}{3.427208in}}%
\pgfusepath{clip}%
\pgfsetbuttcap%
\pgfsetroundjoin%
\definecolor{currentfill}{rgb}{0.678431,1.000000,0.184314}%
\pgfsetfillcolor{currentfill}%
\pgfsetfillopacity{0.500000}%
\pgfsetlinewidth{0.250937pt}%
\definecolor{currentstroke}{rgb}{0.000000,0.000000,0.000000}%
\pgfsetstrokecolor{currentstroke}%
\pgfsetstrokeopacity{0.500000}%
\pgfsetdash{}{0pt}%
\pgfsys@defobject{currentmarker}{\pgfqpoint{-0.030556in}{-0.030556in}}{\pgfqpoint{0.030556in}{0.030556in}}{%
\pgfpathmoveto{\pgfqpoint{0.000000in}{-0.030556in}}%
\pgfpathcurveto{\pgfqpoint{0.008103in}{-0.030556in}}{\pgfqpoint{0.015876in}{-0.027336in}}{\pgfqpoint{0.021606in}{-0.021606in}}%
\pgfpathcurveto{\pgfqpoint{0.027336in}{-0.015876in}}{\pgfqpoint{0.030556in}{-0.008103in}}{\pgfqpoint{0.030556in}{0.000000in}}%
\pgfpathcurveto{\pgfqpoint{0.030556in}{0.008103in}}{\pgfqpoint{0.027336in}{0.015876in}}{\pgfqpoint{0.021606in}{0.021606in}}%
\pgfpathcurveto{\pgfqpoint{0.015876in}{0.027336in}}{\pgfqpoint{0.008103in}{0.030556in}}{\pgfqpoint{0.000000in}{0.030556in}}%
\pgfpathcurveto{\pgfqpoint{-0.008103in}{0.030556in}}{\pgfqpoint{-0.015876in}{0.027336in}}{\pgfqpoint{-0.021606in}{0.021606in}}%
\pgfpathcurveto{\pgfqpoint{-0.027336in}{0.015876in}}{\pgfqpoint{-0.030556in}{0.008103in}}{\pgfqpoint{-0.030556in}{0.000000in}}%
\pgfpathcurveto{\pgfqpoint{-0.030556in}{-0.008103in}}{\pgfqpoint{-0.027336in}{-0.015876in}}{\pgfqpoint{-0.021606in}{-0.021606in}}%
\pgfpathcurveto{\pgfqpoint{-0.015876in}{-0.027336in}}{\pgfqpoint{-0.008103in}{-0.030556in}}{\pgfqpoint{0.000000in}{-0.030556in}}%
\pgfpathclose%
\pgfusepath{stroke,fill}%
}%
\begin{pgfscope}%
\pgfsys@transformshift{4.160198in}{4.849420in}%
\pgfsys@useobject{currentmarker}{}%
\end{pgfscope}%
\end{pgfscope}%
\begin{pgfscope}%
\pgfpathrectangle{\pgfqpoint{0.100000in}{2.413063in}}{\pgfqpoint{5.037500in}{3.427208in}}%
\pgfusepath{clip}%
\pgfsetrectcap%
\pgfsetroundjoin%
\pgfsetlinewidth{1.505625pt}%
\definecolor{currentstroke}{rgb}{0.678431,1.000000,0.184314}%
\pgfsetstrokecolor{currentstroke}%
\pgfsetstrokeopacity{0.500000}%
\pgfsetdash{}{0pt}%
\pgfpathmoveto{\pgfqpoint{4.350454in}{4.792186in}}%
\pgfusepath{stroke}%
\end{pgfscope}%
\begin{pgfscope}%
\pgfpathrectangle{\pgfqpoint{0.100000in}{2.413063in}}{\pgfqpoint{5.037500in}{3.427208in}}%
\pgfusepath{clip}%
\pgfsetbuttcap%
\pgfsetroundjoin%
\definecolor{currentfill}{rgb}{0.678431,1.000000,0.184314}%
\pgfsetfillcolor{currentfill}%
\pgfsetfillopacity{0.500000}%
\pgfsetlinewidth{0.250937pt}%
\definecolor{currentstroke}{rgb}{0.000000,0.000000,0.000000}%
\pgfsetstrokecolor{currentstroke}%
\pgfsetstrokeopacity{0.500000}%
\pgfsetdash{}{0pt}%
\pgfsys@defobject{currentmarker}{\pgfqpoint{-0.027778in}{-0.027778in}}{\pgfqpoint{0.027778in}{0.027778in}}{%
\pgfpathmoveto{\pgfqpoint{0.000000in}{-0.027778in}}%
\pgfpathcurveto{\pgfqpoint{0.007367in}{-0.027778in}}{\pgfqpoint{0.014433in}{-0.024851in}}{\pgfqpoint{0.019642in}{-0.019642in}}%
\pgfpathcurveto{\pgfqpoint{0.024851in}{-0.014433in}}{\pgfqpoint{0.027778in}{-0.007367in}}{\pgfqpoint{0.027778in}{0.000000in}}%
\pgfpathcurveto{\pgfqpoint{0.027778in}{0.007367in}}{\pgfqpoint{0.024851in}{0.014433in}}{\pgfqpoint{0.019642in}{0.019642in}}%
\pgfpathcurveto{\pgfqpoint{0.014433in}{0.024851in}}{\pgfqpoint{0.007367in}{0.027778in}}{\pgfqpoint{0.000000in}{0.027778in}}%
\pgfpathcurveto{\pgfqpoint{-0.007367in}{0.027778in}}{\pgfqpoint{-0.014433in}{0.024851in}}{\pgfqpoint{-0.019642in}{0.019642in}}%
\pgfpathcurveto{\pgfqpoint{-0.024851in}{0.014433in}}{\pgfqpoint{-0.027778in}{0.007367in}}{\pgfqpoint{-0.027778in}{0.000000in}}%
\pgfpathcurveto{\pgfqpoint{-0.027778in}{-0.007367in}}{\pgfqpoint{-0.024851in}{-0.014433in}}{\pgfqpoint{-0.019642in}{-0.019642in}}%
\pgfpathcurveto{\pgfqpoint{-0.014433in}{-0.024851in}}{\pgfqpoint{-0.007367in}{-0.027778in}}{\pgfqpoint{0.000000in}{-0.027778in}}%
\pgfpathclose%
\pgfusepath{stroke,fill}%
}%
\begin{pgfscope}%
\pgfsys@transformshift{4.350454in}{4.792186in}%
\pgfsys@useobject{currentmarker}{}%
\end{pgfscope}%
\end{pgfscope}%
\begin{pgfscope}%
\pgfpathrectangle{\pgfqpoint{0.100000in}{2.413063in}}{\pgfqpoint{5.037500in}{3.427208in}}%
\pgfusepath{clip}%
\pgfsetrectcap%
\pgfsetroundjoin%
\pgfsetlinewidth{1.505625pt}%
\definecolor{currentstroke}{rgb}{0.678431,1.000000,0.184314}%
\pgfsetstrokecolor{currentstroke}%
\pgfsetstrokeopacity{0.500000}%
\pgfsetdash{}{0pt}%
\pgfpathmoveto{\pgfqpoint{4.581959in}{4.987532in}}%
\pgfusepath{stroke}%
\end{pgfscope}%
\begin{pgfscope}%
\pgfpathrectangle{\pgfqpoint{0.100000in}{2.413063in}}{\pgfqpoint{5.037500in}{3.427208in}}%
\pgfusepath{clip}%
\pgfsetbuttcap%
\pgfsetroundjoin%
\definecolor{currentfill}{rgb}{0.678431,1.000000,0.184314}%
\pgfsetfillcolor{currentfill}%
\pgfsetfillopacity{0.500000}%
\pgfsetlinewidth{0.250937pt}%
\definecolor{currentstroke}{rgb}{0.000000,0.000000,0.000000}%
\pgfsetstrokecolor{currentstroke}%
\pgfsetstrokeopacity{0.500000}%
\pgfsetdash{}{0pt}%
\pgfsys@defobject{currentmarker}{\pgfqpoint{-0.047222in}{-0.047222in}}{\pgfqpoint{0.047222in}{0.047222in}}{%
\pgfpathmoveto{\pgfqpoint{0.000000in}{-0.047222in}}%
\pgfpathcurveto{\pgfqpoint{0.012523in}{-0.047222in}}{\pgfqpoint{0.024536in}{-0.042247in}}{\pgfqpoint{0.033391in}{-0.033391in}}%
\pgfpathcurveto{\pgfqpoint{0.042247in}{-0.024536in}}{\pgfqpoint{0.047222in}{-0.012523in}}{\pgfqpoint{0.047222in}{0.000000in}}%
\pgfpathcurveto{\pgfqpoint{0.047222in}{0.012523in}}{\pgfqpoint{0.042247in}{0.024536in}}{\pgfqpoint{0.033391in}{0.033391in}}%
\pgfpathcurveto{\pgfqpoint{0.024536in}{0.042247in}}{\pgfqpoint{0.012523in}{0.047222in}}{\pgfqpoint{0.000000in}{0.047222in}}%
\pgfpathcurveto{\pgfqpoint{-0.012523in}{0.047222in}}{\pgfqpoint{-0.024536in}{0.042247in}}{\pgfqpoint{-0.033391in}{0.033391in}}%
\pgfpathcurveto{\pgfqpoint{-0.042247in}{0.024536in}}{\pgfqpoint{-0.047222in}{0.012523in}}{\pgfqpoint{-0.047222in}{0.000000in}}%
\pgfpathcurveto{\pgfqpoint{-0.047222in}{-0.012523in}}{\pgfqpoint{-0.042247in}{-0.024536in}}{\pgfqpoint{-0.033391in}{-0.033391in}}%
\pgfpathcurveto{\pgfqpoint{-0.024536in}{-0.042247in}}{\pgfqpoint{-0.012523in}{-0.047222in}}{\pgfqpoint{0.000000in}{-0.047222in}}%
\pgfpathclose%
\pgfusepath{stroke,fill}%
}%
\begin{pgfscope}%
\pgfsys@transformshift{4.581959in}{4.987532in}%
\pgfsys@useobject{currentmarker}{}%
\end{pgfscope}%
\end{pgfscope}%
\begin{pgfscope}%
\pgfpathrectangle{\pgfqpoint{0.100000in}{2.413063in}}{\pgfqpoint{5.037500in}{3.427208in}}%
\pgfusepath{clip}%
\pgfsetrectcap%
\pgfsetroundjoin%
\pgfsetlinewidth{1.505625pt}%
\definecolor{currentstroke}{rgb}{0.678431,1.000000,0.184314}%
\pgfsetstrokecolor{currentstroke}%
\pgfsetstrokeopacity{0.500000}%
\pgfsetdash{}{0pt}%
\pgfpathmoveto{\pgfqpoint{4.368363in}{4.837031in}}%
\pgfusepath{stroke}%
\end{pgfscope}%
\begin{pgfscope}%
\pgfpathrectangle{\pgfqpoint{0.100000in}{2.413063in}}{\pgfqpoint{5.037500in}{3.427208in}}%
\pgfusepath{clip}%
\pgfsetbuttcap%
\pgfsetroundjoin%
\definecolor{currentfill}{rgb}{0.678431,1.000000,0.184314}%
\pgfsetfillcolor{currentfill}%
\pgfsetfillopacity{0.500000}%
\pgfsetlinewidth{0.250937pt}%
\definecolor{currentstroke}{rgb}{0.000000,0.000000,0.000000}%
\pgfsetstrokecolor{currentstroke}%
\pgfsetstrokeopacity{0.500000}%
\pgfsetdash{}{0pt}%
\pgfsys@defobject{currentmarker}{\pgfqpoint{-0.025000in}{-0.025000in}}{\pgfqpoint{0.025000in}{0.025000in}}{%
\pgfpathmoveto{\pgfqpoint{0.000000in}{-0.025000in}}%
\pgfpathcurveto{\pgfqpoint{0.006630in}{-0.025000in}}{\pgfqpoint{0.012989in}{-0.022366in}}{\pgfqpoint{0.017678in}{-0.017678in}}%
\pgfpathcurveto{\pgfqpoint{0.022366in}{-0.012989in}}{\pgfqpoint{0.025000in}{-0.006630in}}{\pgfqpoint{0.025000in}{0.000000in}}%
\pgfpathcurveto{\pgfqpoint{0.025000in}{0.006630in}}{\pgfqpoint{0.022366in}{0.012989in}}{\pgfqpoint{0.017678in}{0.017678in}}%
\pgfpathcurveto{\pgfqpoint{0.012989in}{0.022366in}}{\pgfqpoint{0.006630in}{0.025000in}}{\pgfqpoint{0.000000in}{0.025000in}}%
\pgfpathcurveto{\pgfqpoint{-0.006630in}{0.025000in}}{\pgfqpoint{-0.012989in}{0.022366in}}{\pgfqpoint{-0.017678in}{0.017678in}}%
\pgfpathcurveto{\pgfqpoint{-0.022366in}{0.012989in}}{\pgfqpoint{-0.025000in}{0.006630in}}{\pgfqpoint{-0.025000in}{0.000000in}}%
\pgfpathcurveto{\pgfqpoint{-0.025000in}{-0.006630in}}{\pgfqpoint{-0.022366in}{-0.012989in}}{\pgfqpoint{-0.017678in}{-0.017678in}}%
\pgfpathcurveto{\pgfqpoint{-0.012989in}{-0.022366in}}{\pgfqpoint{-0.006630in}{-0.025000in}}{\pgfqpoint{0.000000in}{-0.025000in}}%
\pgfpathclose%
\pgfusepath{stroke,fill}%
}%
\begin{pgfscope}%
\pgfsys@transformshift{4.368363in}{4.837031in}%
\pgfsys@useobject{currentmarker}{}%
\end{pgfscope}%
\end{pgfscope}%
\begin{pgfscope}%
\pgfpathrectangle{\pgfqpoint{0.100000in}{2.413063in}}{\pgfqpoint{5.037500in}{3.427208in}}%
\pgfusepath{clip}%
\pgfsetrectcap%
\pgfsetroundjoin%
\pgfsetlinewidth{1.505625pt}%
\definecolor{currentstroke}{rgb}{0.678431,1.000000,0.184314}%
\pgfsetstrokecolor{currentstroke}%
\pgfsetstrokeopacity{0.500000}%
\pgfsetdash{}{0pt}%
\pgfpathmoveto{\pgfqpoint{4.589191in}{4.825473in}}%
\pgfusepath{stroke}%
\end{pgfscope}%
\begin{pgfscope}%
\pgfpathrectangle{\pgfqpoint{0.100000in}{2.413063in}}{\pgfqpoint{5.037500in}{3.427208in}}%
\pgfusepath{clip}%
\pgfsetbuttcap%
\pgfsetroundjoin%
\definecolor{currentfill}{rgb}{0.678431,1.000000,0.184314}%
\pgfsetfillcolor{currentfill}%
\pgfsetfillopacity{0.500000}%
\pgfsetlinewidth{0.250937pt}%
\definecolor{currentstroke}{rgb}{0.000000,0.000000,0.000000}%
\pgfsetstrokecolor{currentstroke}%
\pgfsetstrokeopacity{0.500000}%
\pgfsetdash{}{0pt}%
\pgfsys@defobject{currentmarker}{\pgfqpoint{-0.022222in}{-0.022222in}}{\pgfqpoint{0.022222in}{0.022222in}}{%
\pgfpathmoveto{\pgfqpoint{0.000000in}{-0.022222in}}%
\pgfpathcurveto{\pgfqpoint{0.005893in}{-0.022222in}}{\pgfqpoint{0.011546in}{-0.019881in}}{\pgfqpoint{0.015713in}{-0.015713in}}%
\pgfpathcurveto{\pgfqpoint{0.019881in}{-0.011546in}}{\pgfqpoint{0.022222in}{-0.005893in}}{\pgfqpoint{0.022222in}{0.000000in}}%
\pgfpathcurveto{\pgfqpoint{0.022222in}{0.005893in}}{\pgfqpoint{0.019881in}{0.011546in}}{\pgfqpoint{0.015713in}{0.015713in}}%
\pgfpathcurveto{\pgfqpoint{0.011546in}{0.019881in}}{\pgfqpoint{0.005893in}{0.022222in}}{\pgfqpoint{0.000000in}{0.022222in}}%
\pgfpathcurveto{\pgfqpoint{-0.005893in}{0.022222in}}{\pgfqpoint{-0.011546in}{0.019881in}}{\pgfqpoint{-0.015713in}{0.015713in}}%
\pgfpathcurveto{\pgfqpoint{-0.019881in}{0.011546in}}{\pgfqpoint{-0.022222in}{0.005893in}}{\pgfqpoint{-0.022222in}{0.000000in}}%
\pgfpathcurveto{\pgfqpoint{-0.022222in}{-0.005893in}}{\pgfqpoint{-0.019881in}{-0.011546in}}{\pgfqpoint{-0.015713in}{-0.015713in}}%
\pgfpathcurveto{\pgfqpoint{-0.011546in}{-0.019881in}}{\pgfqpoint{-0.005893in}{-0.022222in}}{\pgfqpoint{0.000000in}{-0.022222in}}%
\pgfpathclose%
\pgfusepath{stroke,fill}%
}%
\begin{pgfscope}%
\pgfsys@transformshift{4.589191in}{4.825473in}%
\pgfsys@useobject{currentmarker}{}%
\end{pgfscope}%
\end{pgfscope}%
\begin{pgfscope}%
\pgfpathrectangle{\pgfqpoint{0.100000in}{2.413063in}}{\pgfqpoint{5.037500in}{3.427208in}}%
\pgfusepath{clip}%
\pgfsetrectcap%
\pgfsetroundjoin%
\pgfsetlinewidth{1.505625pt}%
\definecolor{currentstroke}{rgb}{0.000000,0.000000,1.000000}%
\pgfsetstrokecolor{currentstroke}%
\pgfsetstrokeopacity{0.500000}%
\pgfsetdash{}{0pt}%
\pgfpathmoveto{\pgfqpoint{4.622153in}{4.691381in}}%
\pgfusepath{stroke}%
\end{pgfscope}%
\begin{pgfscope}%
\pgfpathrectangle{\pgfqpoint{0.100000in}{2.413063in}}{\pgfqpoint{5.037500in}{3.427208in}}%
\pgfusepath{clip}%
\pgfsetbuttcap%
\pgfsetroundjoin%
\definecolor{currentfill}{rgb}{0.000000,0.000000,1.000000}%
\pgfsetfillcolor{currentfill}%
\pgfsetfillopacity{0.500000}%
\pgfsetlinewidth{0.250937pt}%
\definecolor{currentstroke}{rgb}{0.000000,0.000000,0.000000}%
\pgfsetstrokecolor{currentstroke}%
\pgfsetstrokeopacity{0.500000}%
\pgfsetdash{}{0pt}%
\pgfsys@defobject{currentmarker}{\pgfqpoint{-0.066667in}{-0.066667in}}{\pgfqpoint{0.066667in}{0.066667in}}{%
\pgfpathmoveto{\pgfqpoint{0.000000in}{-0.066667in}}%
\pgfpathcurveto{\pgfqpoint{0.017680in}{-0.066667in}}{\pgfqpoint{0.034639in}{-0.059642in}}{\pgfqpoint{0.047140in}{-0.047140in}}%
\pgfpathcurveto{\pgfqpoint{0.059642in}{-0.034639in}}{\pgfqpoint{0.066667in}{-0.017680in}}{\pgfqpoint{0.066667in}{0.000000in}}%
\pgfpathcurveto{\pgfqpoint{0.066667in}{0.017680in}}{\pgfqpoint{0.059642in}{0.034639in}}{\pgfqpoint{0.047140in}{0.047140in}}%
\pgfpathcurveto{\pgfqpoint{0.034639in}{0.059642in}}{\pgfqpoint{0.017680in}{0.066667in}}{\pgfqpoint{0.000000in}{0.066667in}}%
\pgfpathcurveto{\pgfqpoint{-0.017680in}{0.066667in}}{\pgfqpoint{-0.034639in}{0.059642in}}{\pgfqpoint{-0.047140in}{0.047140in}}%
\pgfpathcurveto{\pgfqpoint{-0.059642in}{0.034639in}}{\pgfqpoint{-0.066667in}{0.017680in}}{\pgfqpoint{-0.066667in}{0.000000in}}%
\pgfpathcurveto{\pgfqpoint{-0.066667in}{-0.017680in}}{\pgfqpoint{-0.059642in}{-0.034639in}}{\pgfqpoint{-0.047140in}{-0.047140in}}%
\pgfpathcurveto{\pgfqpoint{-0.034639in}{-0.059642in}}{\pgfqpoint{-0.017680in}{-0.066667in}}{\pgfqpoint{0.000000in}{-0.066667in}}%
\pgfpathclose%
\pgfusepath{stroke,fill}%
}%
\begin{pgfscope}%
\pgfsys@transformshift{4.622153in}{4.691381in}%
\pgfsys@useobject{currentmarker}{}%
\end{pgfscope}%
\end{pgfscope}%
\begin{pgfscope}%
\pgfpathrectangle{\pgfqpoint{0.100000in}{2.413063in}}{\pgfqpoint{5.037500in}{3.427208in}}%
\pgfusepath{clip}%
\pgfsetrectcap%
\pgfsetroundjoin%
\pgfsetlinewidth{1.505625pt}%
\definecolor{currentstroke}{rgb}{0.678431,1.000000,0.184314}%
\pgfsetstrokecolor{currentstroke}%
\pgfsetstrokeopacity{0.500000}%
\pgfsetdash{}{0pt}%
\pgfpathmoveto{\pgfqpoint{4.259261in}{4.899660in}}%
\pgfusepath{stroke}%
\end{pgfscope}%
\begin{pgfscope}%
\pgfpathrectangle{\pgfqpoint{0.100000in}{2.413063in}}{\pgfqpoint{5.037500in}{3.427208in}}%
\pgfusepath{clip}%
\pgfsetbuttcap%
\pgfsetroundjoin%
\definecolor{currentfill}{rgb}{0.678431,1.000000,0.184314}%
\pgfsetfillcolor{currentfill}%
\pgfsetfillopacity{0.500000}%
\pgfsetlinewidth{0.250937pt}%
\definecolor{currentstroke}{rgb}{0.000000,0.000000,0.000000}%
\pgfsetstrokecolor{currentstroke}%
\pgfsetstrokeopacity{0.500000}%
\pgfsetdash{}{0pt}%
\pgfsys@defobject{currentmarker}{\pgfqpoint{-0.033333in}{-0.033333in}}{\pgfqpoint{0.033333in}{0.033333in}}{%
\pgfpathmoveto{\pgfqpoint{0.000000in}{-0.033333in}}%
\pgfpathcurveto{\pgfqpoint{0.008840in}{-0.033333in}}{\pgfqpoint{0.017319in}{-0.029821in}}{\pgfqpoint{0.023570in}{-0.023570in}}%
\pgfpathcurveto{\pgfqpoint{0.029821in}{-0.017319in}}{\pgfqpoint{0.033333in}{-0.008840in}}{\pgfqpoint{0.033333in}{0.000000in}}%
\pgfpathcurveto{\pgfqpoint{0.033333in}{0.008840in}}{\pgfqpoint{0.029821in}{0.017319in}}{\pgfqpoint{0.023570in}{0.023570in}}%
\pgfpathcurveto{\pgfqpoint{0.017319in}{0.029821in}}{\pgfqpoint{0.008840in}{0.033333in}}{\pgfqpoint{0.000000in}{0.033333in}}%
\pgfpathcurveto{\pgfqpoint{-0.008840in}{0.033333in}}{\pgfqpoint{-0.017319in}{0.029821in}}{\pgfqpoint{-0.023570in}{0.023570in}}%
\pgfpathcurveto{\pgfqpoint{-0.029821in}{0.017319in}}{\pgfqpoint{-0.033333in}{0.008840in}}{\pgfqpoint{-0.033333in}{0.000000in}}%
\pgfpathcurveto{\pgfqpoint{-0.033333in}{-0.008840in}}{\pgfqpoint{-0.029821in}{-0.017319in}}{\pgfqpoint{-0.023570in}{-0.023570in}}%
\pgfpathcurveto{\pgfqpoint{-0.017319in}{-0.029821in}}{\pgfqpoint{-0.008840in}{-0.033333in}}{\pgfqpoint{0.000000in}{-0.033333in}}%
\pgfpathclose%
\pgfusepath{stroke,fill}%
}%
\begin{pgfscope}%
\pgfsys@transformshift{4.259261in}{4.899660in}%
\pgfsys@useobject{currentmarker}{}%
\end{pgfscope}%
\end{pgfscope}%
\begin{pgfscope}%
\pgfpathrectangle{\pgfqpoint{0.100000in}{2.413063in}}{\pgfqpoint{5.037500in}{3.427208in}}%
\pgfusepath{clip}%
\pgfsetrectcap%
\pgfsetroundjoin%
\pgfsetlinewidth{1.505625pt}%
\definecolor{currentstroke}{rgb}{0.678431,1.000000,0.184314}%
\pgfsetstrokecolor{currentstroke}%
\pgfsetstrokeopacity{0.500000}%
\pgfsetdash{}{0pt}%
\pgfpathmoveto{\pgfqpoint{4.383016in}{4.911810in}}%
\pgfusepath{stroke}%
\end{pgfscope}%
\begin{pgfscope}%
\pgfpathrectangle{\pgfqpoint{0.100000in}{2.413063in}}{\pgfqpoint{5.037500in}{3.427208in}}%
\pgfusepath{clip}%
\pgfsetbuttcap%
\pgfsetroundjoin%
\definecolor{currentfill}{rgb}{0.678431,1.000000,0.184314}%
\pgfsetfillcolor{currentfill}%
\pgfsetfillopacity{0.500000}%
\pgfsetlinewidth{0.250937pt}%
\definecolor{currentstroke}{rgb}{0.000000,0.000000,0.000000}%
\pgfsetstrokecolor{currentstroke}%
\pgfsetstrokeopacity{0.500000}%
\pgfsetdash{}{0pt}%
\pgfsys@defobject{currentmarker}{\pgfqpoint{-0.033333in}{-0.033333in}}{\pgfqpoint{0.033333in}{0.033333in}}{%
\pgfpathmoveto{\pgfqpoint{0.000000in}{-0.033333in}}%
\pgfpathcurveto{\pgfqpoint{0.008840in}{-0.033333in}}{\pgfqpoint{0.017319in}{-0.029821in}}{\pgfqpoint{0.023570in}{-0.023570in}}%
\pgfpathcurveto{\pgfqpoint{0.029821in}{-0.017319in}}{\pgfqpoint{0.033333in}{-0.008840in}}{\pgfqpoint{0.033333in}{0.000000in}}%
\pgfpathcurveto{\pgfqpoint{0.033333in}{0.008840in}}{\pgfqpoint{0.029821in}{0.017319in}}{\pgfqpoint{0.023570in}{0.023570in}}%
\pgfpathcurveto{\pgfqpoint{0.017319in}{0.029821in}}{\pgfqpoint{0.008840in}{0.033333in}}{\pgfqpoint{0.000000in}{0.033333in}}%
\pgfpathcurveto{\pgfqpoint{-0.008840in}{0.033333in}}{\pgfqpoint{-0.017319in}{0.029821in}}{\pgfqpoint{-0.023570in}{0.023570in}}%
\pgfpathcurveto{\pgfqpoint{-0.029821in}{0.017319in}}{\pgfqpoint{-0.033333in}{0.008840in}}{\pgfqpoint{-0.033333in}{0.000000in}}%
\pgfpathcurveto{\pgfqpoint{-0.033333in}{-0.008840in}}{\pgfqpoint{-0.029821in}{-0.017319in}}{\pgfqpoint{-0.023570in}{-0.023570in}}%
\pgfpathcurveto{\pgfqpoint{-0.017319in}{-0.029821in}}{\pgfqpoint{-0.008840in}{-0.033333in}}{\pgfqpoint{0.000000in}{-0.033333in}}%
\pgfpathclose%
\pgfusepath{stroke,fill}%
}%
\begin{pgfscope}%
\pgfsys@transformshift{4.383016in}{4.911810in}%
\pgfsys@useobject{currentmarker}{}%
\end{pgfscope}%
\end{pgfscope}%
\begin{pgfscope}%
\pgfpathrectangle{\pgfqpoint{0.100000in}{2.413063in}}{\pgfqpoint{5.037500in}{3.427208in}}%
\pgfusepath{clip}%
\pgfsetrectcap%
\pgfsetroundjoin%
\pgfsetlinewidth{1.505625pt}%
\definecolor{currentstroke}{rgb}{0.678431,1.000000,0.184314}%
\pgfsetstrokecolor{currentstroke}%
\pgfsetstrokeopacity{0.500000}%
\pgfsetdash{}{0pt}%
\pgfpathmoveto{\pgfqpoint{4.457128in}{4.934032in}}%
\pgfusepath{stroke}%
\end{pgfscope}%
\begin{pgfscope}%
\pgfpathrectangle{\pgfqpoint{0.100000in}{2.413063in}}{\pgfqpoint{5.037500in}{3.427208in}}%
\pgfusepath{clip}%
\pgfsetbuttcap%
\pgfsetroundjoin%
\definecolor{currentfill}{rgb}{0.678431,1.000000,0.184314}%
\pgfsetfillcolor{currentfill}%
\pgfsetfillopacity{0.500000}%
\pgfsetlinewidth{0.250937pt}%
\definecolor{currentstroke}{rgb}{0.000000,0.000000,0.000000}%
\pgfsetstrokecolor{currentstroke}%
\pgfsetstrokeopacity{0.500000}%
\pgfsetdash{}{0pt}%
\pgfsys@defobject{currentmarker}{\pgfqpoint{-0.033333in}{-0.033333in}}{\pgfqpoint{0.033333in}{0.033333in}}{%
\pgfpathmoveto{\pgfqpoint{0.000000in}{-0.033333in}}%
\pgfpathcurveto{\pgfqpoint{0.008840in}{-0.033333in}}{\pgfqpoint{0.017319in}{-0.029821in}}{\pgfqpoint{0.023570in}{-0.023570in}}%
\pgfpathcurveto{\pgfqpoint{0.029821in}{-0.017319in}}{\pgfqpoint{0.033333in}{-0.008840in}}{\pgfqpoint{0.033333in}{0.000000in}}%
\pgfpathcurveto{\pgfqpoint{0.033333in}{0.008840in}}{\pgfqpoint{0.029821in}{0.017319in}}{\pgfqpoint{0.023570in}{0.023570in}}%
\pgfpathcurveto{\pgfqpoint{0.017319in}{0.029821in}}{\pgfqpoint{0.008840in}{0.033333in}}{\pgfqpoint{0.000000in}{0.033333in}}%
\pgfpathcurveto{\pgfqpoint{-0.008840in}{0.033333in}}{\pgfqpoint{-0.017319in}{0.029821in}}{\pgfqpoint{-0.023570in}{0.023570in}}%
\pgfpathcurveto{\pgfqpoint{-0.029821in}{0.017319in}}{\pgfqpoint{-0.033333in}{0.008840in}}{\pgfqpoint{-0.033333in}{0.000000in}}%
\pgfpathcurveto{\pgfqpoint{-0.033333in}{-0.008840in}}{\pgfqpoint{-0.029821in}{-0.017319in}}{\pgfqpoint{-0.023570in}{-0.023570in}}%
\pgfpathcurveto{\pgfqpoint{-0.017319in}{-0.029821in}}{\pgfqpoint{-0.008840in}{-0.033333in}}{\pgfqpoint{0.000000in}{-0.033333in}}%
\pgfpathclose%
\pgfusepath{stroke,fill}%
}%
\begin{pgfscope}%
\pgfsys@transformshift{4.457128in}{4.934032in}%
\pgfsys@useobject{currentmarker}{}%
\end{pgfscope}%
\end{pgfscope}%
\begin{pgfscope}%
\pgfpathrectangle{\pgfqpoint{0.100000in}{2.413063in}}{\pgfqpoint{5.037500in}{3.427208in}}%
\pgfusepath{clip}%
\pgfsetrectcap%
\pgfsetroundjoin%
\pgfsetlinewidth{1.505625pt}%
\definecolor{currentstroke}{rgb}{0.678431,1.000000,0.184314}%
\pgfsetstrokecolor{currentstroke}%
\pgfsetstrokeopacity{0.500000}%
\pgfsetdash{}{0pt}%
\pgfpathmoveto{\pgfqpoint{4.380229in}{5.020625in}}%
\pgfusepath{stroke}%
\end{pgfscope}%
\begin{pgfscope}%
\pgfpathrectangle{\pgfqpoint{0.100000in}{2.413063in}}{\pgfqpoint{5.037500in}{3.427208in}}%
\pgfusepath{clip}%
\pgfsetbuttcap%
\pgfsetroundjoin%
\definecolor{currentfill}{rgb}{0.678431,1.000000,0.184314}%
\pgfsetfillcolor{currentfill}%
\pgfsetfillopacity{0.500000}%
\pgfsetlinewidth{0.250937pt}%
\definecolor{currentstroke}{rgb}{0.000000,0.000000,0.000000}%
\pgfsetstrokecolor{currentstroke}%
\pgfsetstrokeopacity{0.500000}%
\pgfsetdash{}{0pt}%
\pgfsys@defobject{currentmarker}{\pgfqpoint{-0.086111in}{-0.086111in}}{\pgfqpoint{0.086111in}{0.086111in}}{%
\pgfpathmoveto{\pgfqpoint{0.000000in}{-0.086111in}}%
\pgfpathcurveto{\pgfqpoint{0.022837in}{-0.086111in}}{\pgfqpoint{0.044742in}{-0.077038in}}{\pgfqpoint{0.060890in}{-0.060890in}}%
\pgfpathcurveto{\pgfqpoint{0.077038in}{-0.044742in}}{\pgfqpoint{0.086111in}{-0.022837in}}{\pgfqpoint{0.086111in}{0.000000in}}%
\pgfpathcurveto{\pgfqpoint{0.086111in}{0.022837in}}{\pgfqpoint{0.077038in}{0.044742in}}{\pgfqpoint{0.060890in}{0.060890in}}%
\pgfpathcurveto{\pgfqpoint{0.044742in}{0.077038in}}{\pgfqpoint{0.022837in}{0.086111in}}{\pgfqpoint{0.000000in}{0.086111in}}%
\pgfpathcurveto{\pgfqpoint{-0.022837in}{0.086111in}}{\pgfqpoint{-0.044742in}{0.077038in}}{\pgfqpoint{-0.060890in}{0.060890in}}%
\pgfpathcurveto{\pgfqpoint{-0.077038in}{0.044742in}}{\pgfqpoint{-0.086111in}{0.022837in}}{\pgfqpoint{-0.086111in}{0.000000in}}%
\pgfpathcurveto{\pgfqpoint{-0.086111in}{-0.022837in}}{\pgfqpoint{-0.077038in}{-0.044742in}}{\pgfqpoint{-0.060890in}{-0.060890in}}%
\pgfpathcurveto{\pgfqpoint{-0.044742in}{-0.077038in}}{\pgfqpoint{-0.022837in}{-0.086111in}}{\pgfqpoint{0.000000in}{-0.086111in}}%
\pgfpathclose%
\pgfusepath{stroke,fill}%
}%
\begin{pgfscope}%
\pgfsys@transformshift{4.380229in}{5.020625in}%
\pgfsys@useobject{currentmarker}{}%
\end{pgfscope}%
\end{pgfscope}%
\begin{pgfscope}%
\pgfpathrectangle{\pgfqpoint{0.100000in}{2.413063in}}{\pgfqpoint{5.037500in}{3.427208in}}%
\pgfusepath{clip}%
\pgfsetrectcap%
\pgfsetroundjoin%
\pgfsetlinewidth{1.505625pt}%
\definecolor{currentstroke}{rgb}{0.501961,0.501961,0.501961}%
\pgfsetstrokecolor{currentstroke}%
\pgfsetstrokeopacity{0.500000}%
\pgfsetdash{}{0pt}%
\pgfpathmoveto{\pgfqpoint{3.967863in}{3.969128in}}%
\pgfusepath{stroke}%
\end{pgfscope}%
\begin{pgfscope}%
\pgfpathrectangle{\pgfqpoint{0.100000in}{2.413063in}}{\pgfqpoint{5.037500in}{3.427208in}}%
\pgfusepath{clip}%
\pgfsetbuttcap%
\pgfsetroundjoin%
\definecolor{currentfill}{rgb}{0.501961,0.501961,0.501961}%
\pgfsetfillcolor{currentfill}%
\pgfsetfillopacity{0.500000}%
\pgfsetlinewidth{0.250937pt}%
\definecolor{currentstroke}{rgb}{0.000000,0.000000,0.000000}%
\pgfsetstrokecolor{currentstroke}%
\pgfsetstrokeopacity{0.500000}%
\pgfsetdash{}{0pt}%
\pgfsys@defobject{currentmarker}{\pgfqpoint{-0.013889in}{-0.013889in}}{\pgfqpoint{0.013889in}{0.013889in}}{%
\pgfpathmoveto{\pgfqpoint{0.000000in}{-0.013889in}}%
\pgfpathcurveto{\pgfqpoint{0.003683in}{-0.013889in}}{\pgfqpoint{0.007216in}{-0.012425in}}{\pgfqpoint{0.009821in}{-0.009821in}}%
\pgfpathcurveto{\pgfqpoint{0.012425in}{-0.007216in}}{\pgfqpoint{0.013889in}{-0.003683in}}{\pgfqpoint{0.013889in}{0.000000in}}%
\pgfpathcurveto{\pgfqpoint{0.013889in}{0.003683in}}{\pgfqpoint{0.012425in}{0.007216in}}{\pgfqpoint{0.009821in}{0.009821in}}%
\pgfpathcurveto{\pgfqpoint{0.007216in}{0.012425in}}{\pgfqpoint{0.003683in}{0.013889in}}{\pgfqpoint{0.000000in}{0.013889in}}%
\pgfpathcurveto{\pgfqpoint{-0.003683in}{0.013889in}}{\pgfqpoint{-0.007216in}{0.012425in}}{\pgfqpoint{-0.009821in}{0.009821in}}%
\pgfpathcurveto{\pgfqpoint{-0.012425in}{0.007216in}}{\pgfqpoint{-0.013889in}{0.003683in}}{\pgfqpoint{-0.013889in}{0.000000in}}%
\pgfpathcurveto{\pgfqpoint{-0.013889in}{-0.003683in}}{\pgfqpoint{-0.012425in}{-0.007216in}}{\pgfqpoint{-0.009821in}{-0.009821in}}%
\pgfpathcurveto{\pgfqpoint{-0.007216in}{-0.012425in}}{\pgfqpoint{-0.003683in}{-0.013889in}}{\pgfqpoint{0.000000in}{-0.013889in}}%
\pgfpathclose%
\pgfusepath{stroke,fill}%
}%
\begin{pgfscope}%
\pgfsys@transformshift{3.967863in}{3.969128in}%
\pgfsys@useobject{currentmarker}{}%
\end{pgfscope}%
\end{pgfscope}%
\begin{pgfscope}%
\pgfpathrectangle{\pgfqpoint{0.100000in}{2.413063in}}{\pgfqpoint{5.037500in}{3.427208in}}%
\pgfusepath{clip}%
\pgfsetrectcap%
\pgfsetroundjoin%
\pgfsetlinewidth{1.505625pt}%
\definecolor{currentstroke}{rgb}{0.501961,0.501961,0.501961}%
\pgfsetstrokecolor{currentstroke}%
\pgfsetstrokeopacity{0.500000}%
\pgfsetdash{}{0pt}%
\pgfpathmoveto{\pgfqpoint{4.246950in}{4.070213in}}%
\pgfusepath{stroke}%
\end{pgfscope}%
\begin{pgfscope}%
\pgfpathrectangle{\pgfqpoint{0.100000in}{2.413063in}}{\pgfqpoint{5.037500in}{3.427208in}}%
\pgfusepath{clip}%
\pgfsetbuttcap%
\pgfsetroundjoin%
\definecolor{currentfill}{rgb}{0.501961,0.501961,0.501961}%
\pgfsetfillcolor{currentfill}%
\pgfsetfillopacity{0.500000}%
\pgfsetlinewidth{0.250937pt}%
\definecolor{currentstroke}{rgb}{0.000000,0.000000,0.000000}%
\pgfsetstrokecolor{currentstroke}%
\pgfsetstrokeopacity{0.500000}%
\pgfsetdash{}{0pt}%
\pgfsys@defobject{currentmarker}{\pgfqpoint{-0.013889in}{-0.013889in}}{\pgfqpoint{0.013889in}{0.013889in}}{%
\pgfpathmoveto{\pgfqpoint{0.000000in}{-0.013889in}}%
\pgfpathcurveto{\pgfqpoint{0.003683in}{-0.013889in}}{\pgfqpoint{0.007216in}{-0.012425in}}{\pgfqpoint{0.009821in}{-0.009821in}}%
\pgfpathcurveto{\pgfqpoint{0.012425in}{-0.007216in}}{\pgfqpoint{0.013889in}{-0.003683in}}{\pgfqpoint{0.013889in}{0.000000in}}%
\pgfpathcurveto{\pgfqpoint{0.013889in}{0.003683in}}{\pgfqpoint{0.012425in}{0.007216in}}{\pgfqpoint{0.009821in}{0.009821in}}%
\pgfpathcurveto{\pgfqpoint{0.007216in}{0.012425in}}{\pgfqpoint{0.003683in}{0.013889in}}{\pgfqpoint{0.000000in}{0.013889in}}%
\pgfpathcurveto{\pgfqpoint{-0.003683in}{0.013889in}}{\pgfqpoint{-0.007216in}{0.012425in}}{\pgfqpoint{-0.009821in}{0.009821in}}%
\pgfpathcurveto{\pgfqpoint{-0.012425in}{0.007216in}}{\pgfqpoint{-0.013889in}{0.003683in}}{\pgfqpoint{-0.013889in}{0.000000in}}%
\pgfpathcurveto{\pgfqpoint{-0.013889in}{-0.003683in}}{\pgfqpoint{-0.012425in}{-0.007216in}}{\pgfqpoint{-0.009821in}{-0.009821in}}%
\pgfpathcurveto{\pgfqpoint{-0.007216in}{-0.012425in}}{\pgfqpoint{-0.003683in}{-0.013889in}}{\pgfqpoint{0.000000in}{-0.013889in}}%
\pgfpathclose%
\pgfusepath{stroke,fill}%
}%
\begin{pgfscope}%
\pgfsys@transformshift{4.246950in}{4.070213in}%
\pgfsys@useobject{currentmarker}{}%
\end{pgfscope}%
\end{pgfscope}%
\begin{pgfscope}%
\pgfpathrectangle{\pgfqpoint{0.100000in}{2.413063in}}{\pgfqpoint{5.037500in}{3.427208in}}%
\pgfusepath{clip}%
\pgfsetrectcap%
\pgfsetroundjoin%
\pgfsetlinewidth{1.505625pt}%
\definecolor{currentstroke}{rgb}{0.000000,0.000000,1.000000}%
\pgfsetstrokecolor{currentstroke}%
\pgfsetstrokeopacity{0.500000}%
\pgfsetdash{}{0pt}%
\pgfpathmoveto{\pgfqpoint{4.133311in}{3.949918in}}%
\pgfusepath{stroke}%
\end{pgfscope}%
\begin{pgfscope}%
\pgfpathrectangle{\pgfqpoint{0.100000in}{2.413063in}}{\pgfqpoint{5.037500in}{3.427208in}}%
\pgfusepath{clip}%
\pgfsetbuttcap%
\pgfsetroundjoin%
\definecolor{currentfill}{rgb}{0.000000,0.000000,1.000000}%
\pgfsetfillcolor{currentfill}%
\pgfsetfillopacity{0.500000}%
\pgfsetlinewidth{0.250937pt}%
\definecolor{currentstroke}{rgb}{0.000000,0.000000,0.000000}%
\pgfsetstrokecolor{currentstroke}%
\pgfsetstrokeopacity{0.500000}%
\pgfsetdash{}{0pt}%
\pgfsys@defobject{currentmarker}{\pgfqpoint{-0.005556in}{-0.005556in}}{\pgfqpoint{0.005556in}{0.005556in}}{%
\pgfpathmoveto{\pgfqpoint{0.000000in}{-0.005556in}}%
\pgfpathcurveto{\pgfqpoint{0.001473in}{-0.005556in}}{\pgfqpoint{0.002887in}{-0.004970in}}{\pgfqpoint{0.003928in}{-0.003928in}}%
\pgfpathcurveto{\pgfqpoint{0.004970in}{-0.002887in}}{\pgfqpoint{0.005556in}{-0.001473in}}{\pgfqpoint{0.005556in}{0.000000in}}%
\pgfpathcurveto{\pgfqpoint{0.005556in}{0.001473in}}{\pgfqpoint{0.004970in}{0.002887in}}{\pgfqpoint{0.003928in}{0.003928in}}%
\pgfpathcurveto{\pgfqpoint{0.002887in}{0.004970in}}{\pgfqpoint{0.001473in}{0.005556in}}{\pgfqpoint{0.000000in}{0.005556in}}%
\pgfpathcurveto{\pgfqpoint{-0.001473in}{0.005556in}}{\pgfqpoint{-0.002887in}{0.004970in}}{\pgfqpoint{-0.003928in}{0.003928in}}%
\pgfpathcurveto{\pgfqpoint{-0.004970in}{0.002887in}}{\pgfqpoint{-0.005556in}{0.001473in}}{\pgfqpoint{-0.005556in}{0.000000in}}%
\pgfpathcurveto{\pgfqpoint{-0.005556in}{-0.001473in}}{\pgfqpoint{-0.004970in}{-0.002887in}}{\pgfqpoint{-0.003928in}{-0.003928in}}%
\pgfpathcurveto{\pgfqpoint{-0.002887in}{-0.004970in}}{\pgfqpoint{-0.001473in}{-0.005556in}}{\pgfqpoint{0.000000in}{-0.005556in}}%
\pgfpathclose%
\pgfusepath{stroke,fill}%
}%
\begin{pgfscope}%
\pgfsys@transformshift{4.133311in}{3.949918in}%
\pgfsys@useobject{currentmarker}{}%
\end{pgfscope}%
\end{pgfscope}%
\begin{pgfscope}%
\pgfpathrectangle{\pgfqpoint{0.100000in}{2.413063in}}{\pgfqpoint{5.037500in}{3.427208in}}%
\pgfusepath{clip}%
\pgfsetrectcap%
\pgfsetroundjoin%
\pgfsetlinewidth{1.505625pt}%
\definecolor{currentstroke}{rgb}{0.678431,1.000000,0.184314}%
\pgfsetstrokecolor{currentstroke}%
\pgfsetstrokeopacity{0.500000}%
\pgfsetdash{}{0pt}%
\pgfpathmoveto{\pgfqpoint{4.298489in}{4.067381in}}%
\pgfusepath{stroke}%
\end{pgfscope}%
\begin{pgfscope}%
\pgfpathrectangle{\pgfqpoint{0.100000in}{2.413063in}}{\pgfqpoint{5.037500in}{3.427208in}}%
\pgfusepath{clip}%
\pgfsetbuttcap%
\pgfsetroundjoin%
\definecolor{currentfill}{rgb}{0.678431,1.000000,0.184314}%
\pgfsetfillcolor{currentfill}%
\pgfsetfillopacity{0.500000}%
\pgfsetlinewidth{0.250937pt}%
\definecolor{currentstroke}{rgb}{0.000000,0.000000,0.000000}%
\pgfsetstrokecolor{currentstroke}%
\pgfsetstrokeopacity{0.500000}%
\pgfsetdash{}{0pt}%
\pgfsys@defobject{currentmarker}{\pgfqpoint{-0.008333in}{-0.008333in}}{\pgfqpoint{0.008333in}{0.008333in}}{%
\pgfpathmoveto{\pgfqpoint{0.000000in}{-0.008333in}}%
\pgfpathcurveto{\pgfqpoint{0.002210in}{-0.008333in}}{\pgfqpoint{0.004330in}{-0.007455in}}{\pgfqpoint{0.005893in}{-0.005893in}}%
\pgfpathcurveto{\pgfqpoint{0.007455in}{-0.004330in}}{\pgfqpoint{0.008333in}{-0.002210in}}{\pgfqpoint{0.008333in}{0.000000in}}%
\pgfpathcurveto{\pgfqpoint{0.008333in}{0.002210in}}{\pgfqpoint{0.007455in}{0.004330in}}{\pgfqpoint{0.005893in}{0.005893in}}%
\pgfpathcurveto{\pgfqpoint{0.004330in}{0.007455in}}{\pgfqpoint{0.002210in}{0.008333in}}{\pgfqpoint{0.000000in}{0.008333in}}%
\pgfpathcurveto{\pgfqpoint{-0.002210in}{0.008333in}}{\pgfqpoint{-0.004330in}{0.007455in}}{\pgfqpoint{-0.005893in}{0.005893in}}%
\pgfpathcurveto{\pgfqpoint{-0.007455in}{0.004330in}}{\pgfqpoint{-0.008333in}{0.002210in}}{\pgfqpoint{-0.008333in}{0.000000in}}%
\pgfpathcurveto{\pgfqpoint{-0.008333in}{-0.002210in}}{\pgfqpoint{-0.007455in}{-0.004330in}}{\pgfqpoint{-0.005893in}{-0.005893in}}%
\pgfpathcurveto{\pgfqpoint{-0.004330in}{-0.007455in}}{\pgfqpoint{-0.002210in}{-0.008333in}}{\pgfqpoint{0.000000in}{-0.008333in}}%
\pgfpathclose%
\pgfusepath{stroke,fill}%
}%
\begin{pgfscope}%
\pgfsys@transformshift{4.298489in}{4.067381in}%
\pgfsys@useobject{currentmarker}{}%
\end{pgfscope}%
\end{pgfscope}%
\begin{pgfscope}%
\pgfpathrectangle{\pgfqpoint{0.100000in}{2.413063in}}{\pgfqpoint{5.037500in}{3.427208in}}%
\pgfusepath{clip}%
\pgfsetrectcap%
\pgfsetroundjoin%
\pgfsetlinewidth{1.505625pt}%
\definecolor{currentstroke}{rgb}{0.000000,0.000000,1.000000}%
\pgfsetstrokecolor{currentstroke}%
\pgfsetstrokeopacity{0.500000}%
\pgfsetdash{}{0pt}%
\pgfpathmoveto{\pgfqpoint{4.319477in}{3.960847in}}%
\pgfusepath{stroke}%
\end{pgfscope}%
\begin{pgfscope}%
\pgfpathrectangle{\pgfqpoint{0.100000in}{2.413063in}}{\pgfqpoint{5.037500in}{3.427208in}}%
\pgfusepath{clip}%
\pgfsetbuttcap%
\pgfsetroundjoin%
\definecolor{currentfill}{rgb}{0.000000,0.000000,1.000000}%
\pgfsetfillcolor{currentfill}%
\pgfsetfillopacity{0.500000}%
\pgfsetlinewidth{0.250937pt}%
\definecolor{currentstroke}{rgb}{0.000000,0.000000,0.000000}%
\pgfsetstrokecolor{currentstroke}%
\pgfsetstrokeopacity{0.500000}%
\pgfsetdash{}{0pt}%
\pgfsys@defobject{currentmarker}{\pgfqpoint{-0.013889in}{-0.013889in}}{\pgfqpoint{0.013889in}{0.013889in}}{%
\pgfpathmoveto{\pgfqpoint{0.000000in}{-0.013889in}}%
\pgfpathcurveto{\pgfqpoint{0.003683in}{-0.013889in}}{\pgfqpoint{0.007216in}{-0.012425in}}{\pgfqpoint{0.009821in}{-0.009821in}}%
\pgfpathcurveto{\pgfqpoint{0.012425in}{-0.007216in}}{\pgfqpoint{0.013889in}{-0.003683in}}{\pgfqpoint{0.013889in}{0.000000in}}%
\pgfpathcurveto{\pgfqpoint{0.013889in}{0.003683in}}{\pgfqpoint{0.012425in}{0.007216in}}{\pgfqpoint{0.009821in}{0.009821in}}%
\pgfpathcurveto{\pgfqpoint{0.007216in}{0.012425in}}{\pgfqpoint{0.003683in}{0.013889in}}{\pgfqpoint{0.000000in}{0.013889in}}%
\pgfpathcurveto{\pgfqpoint{-0.003683in}{0.013889in}}{\pgfqpoint{-0.007216in}{0.012425in}}{\pgfqpoint{-0.009821in}{0.009821in}}%
\pgfpathcurveto{\pgfqpoint{-0.012425in}{0.007216in}}{\pgfqpoint{-0.013889in}{0.003683in}}{\pgfqpoint{-0.013889in}{0.000000in}}%
\pgfpathcurveto{\pgfqpoint{-0.013889in}{-0.003683in}}{\pgfqpoint{-0.012425in}{-0.007216in}}{\pgfqpoint{-0.009821in}{-0.009821in}}%
\pgfpathcurveto{\pgfqpoint{-0.007216in}{-0.012425in}}{\pgfqpoint{-0.003683in}{-0.013889in}}{\pgfqpoint{0.000000in}{-0.013889in}}%
\pgfpathclose%
\pgfusepath{stroke,fill}%
}%
\begin{pgfscope}%
\pgfsys@transformshift{4.319477in}{3.960847in}%
\pgfsys@useobject{currentmarker}{}%
\end{pgfscope}%
\end{pgfscope}%
\begin{pgfscope}%
\pgfpathrectangle{\pgfqpoint{0.100000in}{2.413063in}}{\pgfqpoint{5.037500in}{3.427208in}}%
\pgfusepath{clip}%
\pgfsetrectcap%
\pgfsetroundjoin%
\pgfsetlinewidth{1.505625pt}%
\definecolor{currentstroke}{rgb}{0.501961,0.501961,0.501961}%
\pgfsetstrokecolor{currentstroke}%
\pgfsetstrokeopacity{0.500000}%
\pgfsetdash{}{0pt}%
\pgfpathmoveto{\pgfqpoint{4.394677in}{4.013655in}}%
\pgfusepath{stroke}%
\end{pgfscope}%
\begin{pgfscope}%
\pgfpathrectangle{\pgfqpoint{0.100000in}{2.413063in}}{\pgfqpoint{5.037500in}{3.427208in}}%
\pgfusepath{clip}%
\pgfsetbuttcap%
\pgfsetroundjoin%
\definecolor{currentfill}{rgb}{0.501961,0.501961,0.501961}%
\pgfsetfillcolor{currentfill}%
\pgfsetfillopacity{0.500000}%
\pgfsetlinewidth{0.250937pt}%
\definecolor{currentstroke}{rgb}{0.000000,0.000000,0.000000}%
\pgfsetstrokecolor{currentstroke}%
\pgfsetstrokeopacity{0.500000}%
\pgfsetdash{}{0pt}%
\pgfsys@defobject{currentmarker}{\pgfqpoint{-0.013889in}{-0.013889in}}{\pgfqpoint{0.013889in}{0.013889in}}{%
\pgfpathmoveto{\pgfqpoint{0.000000in}{-0.013889in}}%
\pgfpathcurveto{\pgfqpoint{0.003683in}{-0.013889in}}{\pgfqpoint{0.007216in}{-0.012425in}}{\pgfqpoint{0.009821in}{-0.009821in}}%
\pgfpathcurveto{\pgfqpoint{0.012425in}{-0.007216in}}{\pgfqpoint{0.013889in}{-0.003683in}}{\pgfqpoint{0.013889in}{0.000000in}}%
\pgfpathcurveto{\pgfqpoint{0.013889in}{0.003683in}}{\pgfqpoint{0.012425in}{0.007216in}}{\pgfqpoint{0.009821in}{0.009821in}}%
\pgfpathcurveto{\pgfqpoint{0.007216in}{0.012425in}}{\pgfqpoint{0.003683in}{0.013889in}}{\pgfqpoint{0.000000in}{0.013889in}}%
\pgfpathcurveto{\pgfqpoint{-0.003683in}{0.013889in}}{\pgfqpoint{-0.007216in}{0.012425in}}{\pgfqpoint{-0.009821in}{0.009821in}}%
\pgfpathcurveto{\pgfqpoint{-0.012425in}{0.007216in}}{\pgfqpoint{-0.013889in}{0.003683in}}{\pgfqpoint{-0.013889in}{0.000000in}}%
\pgfpathcurveto{\pgfqpoint{-0.013889in}{-0.003683in}}{\pgfqpoint{-0.012425in}{-0.007216in}}{\pgfqpoint{-0.009821in}{-0.009821in}}%
\pgfpathcurveto{\pgfqpoint{-0.007216in}{-0.012425in}}{\pgfqpoint{-0.003683in}{-0.013889in}}{\pgfqpoint{0.000000in}{-0.013889in}}%
\pgfpathclose%
\pgfusepath{stroke,fill}%
}%
\begin{pgfscope}%
\pgfsys@transformshift{4.394677in}{4.013655in}%
\pgfsys@useobject{currentmarker}{}%
\end{pgfscope}%
\end{pgfscope}%
\begin{pgfscope}%
\pgfpathrectangle{\pgfqpoint{0.100000in}{2.413063in}}{\pgfqpoint{5.037500in}{3.427208in}}%
\pgfusepath{clip}%
\pgfsetrectcap%
\pgfsetroundjoin%
\pgfsetlinewidth{1.505625pt}%
\definecolor{currentstroke}{rgb}{0.000000,0.000000,1.000000}%
\pgfsetstrokecolor{currentstroke}%
\pgfsetstrokeopacity{0.500000}%
\pgfsetdash{}{0pt}%
\pgfpathmoveto{\pgfqpoint{4.214866in}{4.062032in}}%
\pgfusepath{stroke}%
\end{pgfscope}%
\begin{pgfscope}%
\pgfpathrectangle{\pgfqpoint{0.100000in}{2.413063in}}{\pgfqpoint{5.037500in}{3.427208in}}%
\pgfusepath{clip}%
\pgfsetbuttcap%
\pgfsetroundjoin%
\definecolor{currentfill}{rgb}{0.000000,0.000000,1.000000}%
\pgfsetfillcolor{currentfill}%
\pgfsetfillopacity{0.500000}%
\pgfsetlinewidth{0.250937pt}%
\definecolor{currentstroke}{rgb}{0.000000,0.000000,0.000000}%
\pgfsetstrokecolor{currentstroke}%
\pgfsetstrokeopacity{0.500000}%
\pgfsetdash{}{0pt}%
\pgfsys@defobject{currentmarker}{\pgfqpoint{-0.011111in}{-0.011111in}}{\pgfqpoint{0.011111in}{0.011111in}}{%
\pgfpathmoveto{\pgfqpoint{0.000000in}{-0.011111in}}%
\pgfpathcurveto{\pgfqpoint{0.002947in}{-0.011111in}}{\pgfqpoint{0.005773in}{-0.009940in}}{\pgfqpoint{0.007857in}{-0.007857in}}%
\pgfpathcurveto{\pgfqpoint{0.009940in}{-0.005773in}}{\pgfqpoint{0.011111in}{-0.002947in}}{\pgfqpoint{0.011111in}{0.000000in}}%
\pgfpathcurveto{\pgfqpoint{0.011111in}{0.002947in}}{\pgfqpoint{0.009940in}{0.005773in}}{\pgfqpoint{0.007857in}{0.007857in}}%
\pgfpathcurveto{\pgfqpoint{0.005773in}{0.009940in}}{\pgfqpoint{0.002947in}{0.011111in}}{\pgfqpoint{0.000000in}{0.011111in}}%
\pgfpathcurveto{\pgfqpoint{-0.002947in}{0.011111in}}{\pgfqpoint{-0.005773in}{0.009940in}}{\pgfqpoint{-0.007857in}{0.007857in}}%
\pgfpathcurveto{\pgfqpoint{-0.009940in}{0.005773in}}{\pgfqpoint{-0.011111in}{0.002947in}}{\pgfqpoint{-0.011111in}{0.000000in}}%
\pgfpathcurveto{\pgfqpoint{-0.011111in}{-0.002947in}}{\pgfqpoint{-0.009940in}{-0.005773in}}{\pgfqpoint{-0.007857in}{-0.007857in}}%
\pgfpathcurveto{\pgfqpoint{-0.005773in}{-0.009940in}}{\pgfqpoint{-0.002947in}{-0.011111in}}{\pgfqpoint{0.000000in}{-0.011111in}}%
\pgfpathclose%
\pgfusepath{stroke,fill}%
}%
\begin{pgfscope}%
\pgfsys@transformshift{4.214866in}{4.062032in}%
\pgfsys@useobject{currentmarker}{}%
\end{pgfscope}%
\end{pgfscope}%
\begin{pgfscope}%
\pgfpathrectangle{\pgfqpoint{0.100000in}{2.413063in}}{\pgfqpoint{5.037500in}{3.427208in}}%
\pgfusepath{clip}%
\pgfsetrectcap%
\pgfsetroundjoin%
\pgfsetlinewidth{1.505625pt}%
\definecolor{currentstroke}{rgb}{0.501961,0.501961,0.501961}%
\pgfsetstrokecolor{currentstroke}%
\pgfsetstrokeopacity{0.500000}%
\pgfsetdash{}{0pt}%
\pgfpathmoveto{\pgfqpoint{4.447469in}{4.050603in}}%
\pgfusepath{stroke}%
\end{pgfscope}%
\begin{pgfscope}%
\pgfpathrectangle{\pgfqpoint{0.100000in}{2.413063in}}{\pgfqpoint{5.037500in}{3.427208in}}%
\pgfusepath{clip}%
\pgfsetbuttcap%
\pgfsetroundjoin%
\definecolor{currentfill}{rgb}{0.501961,0.501961,0.501961}%
\pgfsetfillcolor{currentfill}%
\pgfsetfillopacity{0.500000}%
\pgfsetlinewidth{0.250937pt}%
\definecolor{currentstroke}{rgb}{0.000000,0.000000,0.000000}%
\pgfsetstrokecolor{currentstroke}%
\pgfsetstrokeopacity{0.500000}%
\pgfsetdash{}{0pt}%
\pgfsys@defobject{currentmarker}{\pgfqpoint{-0.013889in}{-0.013889in}}{\pgfqpoint{0.013889in}{0.013889in}}{%
\pgfpathmoveto{\pgfqpoint{0.000000in}{-0.013889in}}%
\pgfpathcurveto{\pgfqpoint{0.003683in}{-0.013889in}}{\pgfqpoint{0.007216in}{-0.012425in}}{\pgfqpoint{0.009821in}{-0.009821in}}%
\pgfpathcurveto{\pgfqpoint{0.012425in}{-0.007216in}}{\pgfqpoint{0.013889in}{-0.003683in}}{\pgfqpoint{0.013889in}{0.000000in}}%
\pgfpathcurveto{\pgfqpoint{0.013889in}{0.003683in}}{\pgfqpoint{0.012425in}{0.007216in}}{\pgfqpoint{0.009821in}{0.009821in}}%
\pgfpathcurveto{\pgfqpoint{0.007216in}{0.012425in}}{\pgfqpoint{0.003683in}{0.013889in}}{\pgfqpoint{0.000000in}{0.013889in}}%
\pgfpathcurveto{\pgfqpoint{-0.003683in}{0.013889in}}{\pgfqpoint{-0.007216in}{0.012425in}}{\pgfqpoint{-0.009821in}{0.009821in}}%
\pgfpathcurveto{\pgfqpoint{-0.012425in}{0.007216in}}{\pgfqpoint{-0.013889in}{0.003683in}}{\pgfqpoint{-0.013889in}{0.000000in}}%
\pgfpathcurveto{\pgfqpoint{-0.013889in}{-0.003683in}}{\pgfqpoint{-0.012425in}{-0.007216in}}{\pgfqpoint{-0.009821in}{-0.009821in}}%
\pgfpathcurveto{\pgfqpoint{-0.007216in}{-0.012425in}}{\pgfqpoint{-0.003683in}{-0.013889in}}{\pgfqpoint{0.000000in}{-0.013889in}}%
\pgfpathclose%
\pgfusepath{stroke,fill}%
}%
\begin{pgfscope}%
\pgfsys@transformshift{4.447469in}{4.050603in}%
\pgfsys@useobject{currentmarker}{}%
\end{pgfscope}%
\end{pgfscope}%
\begin{pgfscope}%
\pgfpathrectangle{\pgfqpoint{0.100000in}{2.413063in}}{\pgfqpoint{5.037500in}{3.427208in}}%
\pgfusepath{clip}%
\pgfsetrectcap%
\pgfsetroundjoin%
\pgfsetlinewidth{1.505625pt}%
\definecolor{currentstroke}{rgb}{0.678431,1.000000,0.184314}%
\pgfsetstrokecolor{currentstroke}%
\pgfsetstrokeopacity{0.500000}%
\pgfsetdash{}{0pt}%
\pgfpathmoveto{\pgfqpoint{4.078163in}{4.000492in}}%
\pgfusepath{stroke}%
\end{pgfscope}%
\begin{pgfscope}%
\pgfpathrectangle{\pgfqpoint{0.100000in}{2.413063in}}{\pgfqpoint{5.037500in}{3.427208in}}%
\pgfusepath{clip}%
\pgfsetbuttcap%
\pgfsetroundjoin%
\definecolor{currentfill}{rgb}{0.678431,1.000000,0.184314}%
\pgfsetfillcolor{currentfill}%
\pgfsetfillopacity{0.500000}%
\pgfsetlinewidth{0.250937pt}%
\definecolor{currentstroke}{rgb}{0.000000,0.000000,0.000000}%
\pgfsetstrokecolor{currentstroke}%
\pgfsetstrokeopacity{0.500000}%
\pgfsetdash{}{0pt}%
\pgfsys@defobject{currentmarker}{\pgfqpoint{-0.005556in}{-0.005556in}}{\pgfqpoint{0.005556in}{0.005556in}}{%
\pgfpathmoveto{\pgfqpoint{0.000000in}{-0.005556in}}%
\pgfpathcurveto{\pgfqpoint{0.001473in}{-0.005556in}}{\pgfqpoint{0.002887in}{-0.004970in}}{\pgfqpoint{0.003928in}{-0.003928in}}%
\pgfpathcurveto{\pgfqpoint{0.004970in}{-0.002887in}}{\pgfqpoint{0.005556in}{-0.001473in}}{\pgfqpoint{0.005556in}{0.000000in}}%
\pgfpathcurveto{\pgfqpoint{0.005556in}{0.001473in}}{\pgfqpoint{0.004970in}{0.002887in}}{\pgfqpoint{0.003928in}{0.003928in}}%
\pgfpathcurveto{\pgfqpoint{0.002887in}{0.004970in}}{\pgfqpoint{0.001473in}{0.005556in}}{\pgfqpoint{0.000000in}{0.005556in}}%
\pgfpathcurveto{\pgfqpoint{-0.001473in}{0.005556in}}{\pgfqpoint{-0.002887in}{0.004970in}}{\pgfqpoint{-0.003928in}{0.003928in}}%
\pgfpathcurveto{\pgfqpoint{-0.004970in}{0.002887in}}{\pgfqpoint{-0.005556in}{0.001473in}}{\pgfqpoint{-0.005556in}{0.000000in}}%
\pgfpathcurveto{\pgfqpoint{-0.005556in}{-0.001473in}}{\pgfqpoint{-0.004970in}{-0.002887in}}{\pgfqpoint{-0.003928in}{-0.003928in}}%
\pgfpathcurveto{\pgfqpoint{-0.002887in}{-0.004970in}}{\pgfqpoint{-0.001473in}{-0.005556in}}{\pgfqpoint{0.000000in}{-0.005556in}}%
\pgfpathclose%
\pgfusepath{stroke,fill}%
}%
\begin{pgfscope}%
\pgfsys@transformshift{4.078163in}{4.000492in}%
\pgfsys@useobject{currentmarker}{}%
\end{pgfscope}%
\end{pgfscope}%
\begin{pgfscope}%
\pgfpathrectangle{\pgfqpoint{0.100000in}{2.413063in}}{\pgfqpoint{5.037500in}{3.427208in}}%
\pgfusepath{clip}%
\pgfsetrectcap%
\pgfsetroundjoin%
\pgfsetlinewidth{1.505625pt}%
\definecolor{currentstroke}{rgb}{0.678431,1.000000,0.184314}%
\pgfsetstrokecolor{currentstroke}%
\pgfsetstrokeopacity{0.500000}%
\pgfsetdash{}{0pt}%
\pgfpathmoveto{\pgfqpoint{4.460629in}{3.952203in}}%
\pgfusepath{stroke}%
\end{pgfscope}%
\begin{pgfscope}%
\pgfpathrectangle{\pgfqpoint{0.100000in}{2.413063in}}{\pgfqpoint{5.037500in}{3.427208in}}%
\pgfusepath{clip}%
\pgfsetbuttcap%
\pgfsetroundjoin%
\definecolor{currentfill}{rgb}{0.678431,1.000000,0.184314}%
\pgfsetfillcolor{currentfill}%
\pgfsetfillopacity{0.500000}%
\pgfsetlinewidth{0.250937pt}%
\definecolor{currentstroke}{rgb}{0.000000,0.000000,0.000000}%
\pgfsetstrokecolor{currentstroke}%
\pgfsetstrokeopacity{0.500000}%
\pgfsetdash{}{0pt}%
\pgfsys@defobject{currentmarker}{\pgfqpoint{-0.016667in}{-0.016667in}}{\pgfqpoint{0.016667in}{0.016667in}}{%
\pgfpathmoveto{\pgfqpoint{0.000000in}{-0.016667in}}%
\pgfpathcurveto{\pgfqpoint{0.004420in}{-0.016667in}}{\pgfqpoint{0.008660in}{-0.014911in}}{\pgfqpoint{0.011785in}{-0.011785in}}%
\pgfpathcurveto{\pgfqpoint{0.014911in}{-0.008660in}}{\pgfqpoint{0.016667in}{-0.004420in}}{\pgfqpoint{0.016667in}{0.000000in}}%
\pgfpathcurveto{\pgfqpoint{0.016667in}{0.004420in}}{\pgfqpoint{0.014911in}{0.008660in}}{\pgfqpoint{0.011785in}{0.011785in}}%
\pgfpathcurveto{\pgfqpoint{0.008660in}{0.014911in}}{\pgfqpoint{0.004420in}{0.016667in}}{\pgfqpoint{0.000000in}{0.016667in}}%
\pgfpathcurveto{\pgfqpoint{-0.004420in}{0.016667in}}{\pgfqpoint{-0.008660in}{0.014911in}}{\pgfqpoint{-0.011785in}{0.011785in}}%
\pgfpathcurveto{\pgfqpoint{-0.014911in}{0.008660in}}{\pgfqpoint{-0.016667in}{0.004420in}}{\pgfqpoint{-0.016667in}{0.000000in}}%
\pgfpathcurveto{\pgfqpoint{-0.016667in}{-0.004420in}}{\pgfqpoint{-0.014911in}{-0.008660in}}{\pgfqpoint{-0.011785in}{-0.011785in}}%
\pgfpathcurveto{\pgfqpoint{-0.008660in}{-0.014911in}}{\pgfqpoint{-0.004420in}{-0.016667in}}{\pgfqpoint{0.000000in}{-0.016667in}}%
\pgfpathclose%
\pgfusepath{stroke,fill}%
}%
\begin{pgfscope}%
\pgfsys@transformshift{4.460629in}{3.952203in}%
\pgfsys@useobject{currentmarker}{}%
\end{pgfscope}%
\end{pgfscope}%
\begin{pgfscope}%
\pgfpathrectangle{\pgfqpoint{0.100000in}{2.413063in}}{\pgfqpoint{5.037500in}{3.427208in}}%
\pgfusepath{clip}%
\pgfsetrectcap%
\pgfsetroundjoin%
\pgfsetlinewidth{1.505625pt}%
\definecolor{currentstroke}{rgb}{0.678431,1.000000,0.184314}%
\pgfsetstrokecolor{currentstroke}%
\pgfsetstrokeopacity{0.500000}%
\pgfsetdash{}{0pt}%
\pgfpathmoveto{\pgfqpoint{4.488531in}{3.999440in}}%
\pgfusepath{stroke}%
\end{pgfscope}%
\begin{pgfscope}%
\pgfpathrectangle{\pgfqpoint{0.100000in}{2.413063in}}{\pgfqpoint{5.037500in}{3.427208in}}%
\pgfusepath{clip}%
\pgfsetbuttcap%
\pgfsetroundjoin%
\definecolor{currentfill}{rgb}{0.678431,1.000000,0.184314}%
\pgfsetfillcolor{currentfill}%
\pgfsetfillopacity{0.500000}%
\pgfsetlinewidth{0.250937pt}%
\definecolor{currentstroke}{rgb}{0.000000,0.000000,0.000000}%
\pgfsetstrokecolor{currentstroke}%
\pgfsetstrokeopacity{0.500000}%
\pgfsetdash{}{0pt}%
\pgfsys@defobject{currentmarker}{\pgfqpoint{-0.008333in}{-0.008333in}}{\pgfqpoint{0.008333in}{0.008333in}}{%
\pgfpathmoveto{\pgfqpoint{0.000000in}{-0.008333in}}%
\pgfpathcurveto{\pgfqpoint{0.002210in}{-0.008333in}}{\pgfqpoint{0.004330in}{-0.007455in}}{\pgfqpoint{0.005893in}{-0.005893in}}%
\pgfpathcurveto{\pgfqpoint{0.007455in}{-0.004330in}}{\pgfqpoint{0.008333in}{-0.002210in}}{\pgfqpoint{0.008333in}{0.000000in}}%
\pgfpathcurveto{\pgfqpoint{0.008333in}{0.002210in}}{\pgfqpoint{0.007455in}{0.004330in}}{\pgfqpoint{0.005893in}{0.005893in}}%
\pgfpathcurveto{\pgfqpoint{0.004330in}{0.007455in}}{\pgfqpoint{0.002210in}{0.008333in}}{\pgfqpoint{0.000000in}{0.008333in}}%
\pgfpathcurveto{\pgfqpoint{-0.002210in}{0.008333in}}{\pgfqpoint{-0.004330in}{0.007455in}}{\pgfqpoint{-0.005893in}{0.005893in}}%
\pgfpathcurveto{\pgfqpoint{-0.007455in}{0.004330in}}{\pgfqpoint{-0.008333in}{0.002210in}}{\pgfqpoint{-0.008333in}{0.000000in}}%
\pgfpathcurveto{\pgfqpoint{-0.008333in}{-0.002210in}}{\pgfqpoint{-0.007455in}{-0.004330in}}{\pgfqpoint{-0.005893in}{-0.005893in}}%
\pgfpathcurveto{\pgfqpoint{-0.004330in}{-0.007455in}}{\pgfqpoint{-0.002210in}{-0.008333in}}{\pgfqpoint{0.000000in}{-0.008333in}}%
\pgfpathclose%
\pgfusepath{stroke,fill}%
}%
\begin{pgfscope}%
\pgfsys@transformshift{4.488531in}{3.999440in}%
\pgfsys@useobject{currentmarker}{}%
\end{pgfscope}%
\end{pgfscope}%
\begin{pgfscope}%
\pgfpathrectangle{\pgfqpoint{0.100000in}{2.413063in}}{\pgfqpoint{5.037500in}{3.427208in}}%
\pgfusepath{clip}%
\pgfsetrectcap%
\pgfsetroundjoin%
\pgfsetlinewidth{1.505625pt}%
\definecolor{currentstroke}{rgb}{0.501961,0.501961,0.501961}%
\pgfsetstrokecolor{currentstroke}%
\pgfsetstrokeopacity{0.500000}%
\pgfsetdash{}{0pt}%
\pgfpathmoveto{\pgfqpoint{4.326728in}{4.047444in}}%
\pgfusepath{stroke}%
\end{pgfscope}%
\begin{pgfscope}%
\pgfpathrectangle{\pgfqpoint{0.100000in}{2.413063in}}{\pgfqpoint{5.037500in}{3.427208in}}%
\pgfusepath{clip}%
\pgfsetbuttcap%
\pgfsetroundjoin%
\definecolor{currentfill}{rgb}{0.501961,0.501961,0.501961}%
\pgfsetfillcolor{currentfill}%
\pgfsetfillopacity{0.500000}%
\pgfsetlinewidth{0.250937pt}%
\definecolor{currentstroke}{rgb}{0.000000,0.000000,0.000000}%
\pgfsetstrokecolor{currentstroke}%
\pgfsetstrokeopacity{0.500000}%
\pgfsetdash{}{0pt}%
\pgfsys@defobject{currentmarker}{\pgfqpoint{-0.013889in}{-0.013889in}}{\pgfqpoint{0.013889in}{0.013889in}}{%
\pgfpathmoveto{\pgfqpoint{0.000000in}{-0.013889in}}%
\pgfpathcurveto{\pgfqpoint{0.003683in}{-0.013889in}}{\pgfqpoint{0.007216in}{-0.012425in}}{\pgfqpoint{0.009821in}{-0.009821in}}%
\pgfpathcurveto{\pgfqpoint{0.012425in}{-0.007216in}}{\pgfqpoint{0.013889in}{-0.003683in}}{\pgfqpoint{0.013889in}{0.000000in}}%
\pgfpathcurveto{\pgfqpoint{0.013889in}{0.003683in}}{\pgfqpoint{0.012425in}{0.007216in}}{\pgfqpoint{0.009821in}{0.009821in}}%
\pgfpathcurveto{\pgfqpoint{0.007216in}{0.012425in}}{\pgfqpoint{0.003683in}{0.013889in}}{\pgfqpoint{0.000000in}{0.013889in}}%
\pgfpathcurveto{\pgfqpoint{-0.003683in}{0.013889in}}{\pgfqpoint{-0.007216in}{0.012425in}}{\pgfqpoint{-0.009821in}{0.009821in}}%
\pgfpathcurveto{\pgfqpoint{-0.012425in}{0.007216in}}{\pgfqpoint{-0.013889in}{0.003683in}}{\pgfqpoint{-0.013889in}{0.000000in}}%
\pgfpathcurveto{\pgfqpoint{-0.013889in}{-0.003683in}}{\pgfqpoint{-0.012425in}{-0.007216in}}{\pgfqpoint{-0.009821in}{-0.009821in}}%
\pgfpathcurveto{\pgfqpoint{-0.007216in}{-0.012425in}}{\pgfqpoint{-0.003683in}{-0.013889in}}{\pgfqpoint{0.000000in}{-0.013889in}}%
\pgfpathclose%
\pgfusepath{stroke,fill}%
}%
\begin{pgfscope}%
\pgfsys@transformshift{4.326728in}{4.047444in}%
\pgfsys@useobject{currentmarker}{}%
\end{pgfscope}%
\end{pgfscope}%
\begin{pgfscope}%
\pgfpathrectangle{\pgfqpoint{0.100000in}{2.413063in}}{\pgfqpoint{5.037500in}{3.427208in}}%
\pgfusepath{clip}%
\pgfsetrectcap%
\pgfsetroundjoin%
\pgfsetlinewidth{1.505625pt}%
\definecolor{currentstroke}{rgb}{0.000000,0.000000,1.000000}%
\pgfsetstrokecolor{currentstroke}%
\pgfsetstrokeopacity{0.500000}%
\pgfsetdash{}{0pt}%
\pgfpathmoveto{\pgfqpoint{4.400786in}{4.080282in}}%
\pgfusepath{stroke}%
\end{pgfscope}%
\begin{pgfscope}%
\pgfpathrectangle{\pgfqpoint{0.100000in}{2.413063in}}{\pgfqpoint{5.037500in}{3.427208in}}%
\pgfusepath{clip}%
\pgfsetbuttcap%
\pgfsetroundjoin%
\definecolor{currentfill}{rgb}{0.000000,0.000000,1.000000}%
\pgfsetfillcolor{currentfill}%
\pgfsetfillopacity{0.500000}%
\pgfsetlinewidth{0.250937pt}%
\definecolor{currentstroke}{rgb}{0.000000,0.000000,0.000000}%
\pgfsetstrokecolor{currentstroke}%
\pgfsetstrokeopacity{0.500000}%
\pgfsetdash{}{0pt}%
\pgfsys@defobject{currentmarker}{\pgfqpoint{-0.025000in}{-0.025000in}}{\pgfqpoint{0.025000in}{0.025000in}}{%
\pgfpathmoveto{\pgfqpoint{0.000000in}{-0.025000in}}%
\pgfpathcurveto{\pgfqpoint{0.006630in}{-0.025000in}}{\pgfqpoint{0.012989in}{-0.022366in}}{\pgfqpoint{0.017678in}{-0.017678in}}%
\pgfpathcurveto{\pgfqpoint{0.022366in}{-0.012989in}}{\pgfqpoint{0.025000in}{-0.006630in}}{\pgfqpoint{0.025000in}{0.000000in}}%
\pgfpathcurveto{\pgfqpoint{0.025000in}{0.006630in}}{\pgfqpoint{0.022366in}{0.012989in}}{\pgfqpoint{0.017678in}{0.017678in}}%
\pgfpathcurveto{\pgfqpoint{0.012989in}{0.022366in}}{\pgfqpoint{0.006630in}{0.025000in}}{\pgfqpoint{0.000000in}{0.025000in}}%
\pgfpathcurveto{\pgfqpoint{-0.006630in}{0.025000in}}{\pgfqpoint{-0.012989in}{0.022366in}}{\pgfqpoint{-0.017678in}{0.017678in}}%
\pgfpathcurveto{\pgfqpoint{-0.022366in}{0.012989in}}{\pgfqpoint{-0.025000in}{0.006630in}}{\pgfqpoint{-0.025000in}{0.000000in}}%
\pgfpathcurveto{\pgfqpoint{-0.025000in}{-0.006630in}}{\pgfqpoint{-0.022366in}{-0.012989in}}{\pgfqpoint{-0.017678in}{-0.017678in}}%
\pgfpathcurveto{\pgfqpoint{-0.012989in}{-0.022366in}}{\pgfqpoint{-0.006630in}{-0.025000in}}{\pgfqpoint{0.000000in}{-0.025000in}}%
\pgfpathclose%
\pgfusepath{stroke,fill}%
}%
\begin{pgfscope}%
\pgfsys@transformshift{4.400786in}{4.080282in}%
\pgfsys@useobject{currentmarker}{}%
\end{pgfscope}%
\end{pgfscope}%
\begin{pgfscope}%
\pgfpathrectangle{\pgfqpoint{0.100000in}{2.413063in}}{\pgfqpoint{5.037500in}{3.427208in}}%
\pgfusepath{clip}%
\pgfsetrectcap%
\pgfsetroundjoin%
\pgfsetlinewidth{1.505625pt}%
\definecolor{currentstroke}{rgb}{0.678431,1.000000,0.184314}%
\pgfsetstrokecolor{currentstroke}%
\pgfsetstrokeopacity{0.500000}%
\pgfsetdash{}{0pt}%
\pgfpathmoveto{\pgfqpoint{4.424090in}{3.883005in}}%
\pgfusepath{stroke}%
\end{pgfscope}%
\begin{pgfscope}%
\pgfpathrectangle{\pgfqpoint{0.100000in}{2.413063in}}{\pgfqpoint{5.037500in}{3.427208in}}%
\pgfusepath{clip}%
\pgfsetbuttcap%
\pgfsetroundjoin%
\definecolor{currentfill}{rgb}{0.678431,1.000000,0.184314}%
\pgfsetfillcolor{currentfill}%
\pgfsetfillopacity{0.500000}%
\pgfsetlinewidth{0.250937pt}%
\definecolor{currentstroke}{rgb}{0.000000,0.000000,0.000000}%
\pgfsetstrokecolor{currentstroke}%
\pgfsetstrokeopacity{0.500000}%
\pgfsetdash{}{0pt}%
\pgfsys@defobject{currentmarker}{\pgfqpoint{-0.008333in}{-0.008333in}}{\pgfqpoint{0.008333in}{0.008333in}}{%
\pgfpathmoveto{\pgfqpoint{0.000000in}{-0.008333in}}%
\pgfpathcurveto{\pgfqpoint{0.002210in}{-0.008333in}}{\pgfqpoint{0.004330in}{-0.007455in}}{\pgfqpoint{0.005893in}{-0.005893in}}%
\pgfpathcurveto{\pgfqpoint{0.007455in}{-0.004330in}}{\pgfqpoint{0.008333in}{-0.002210in}}{\pgfqpoint{0.008333in}{0.000000in}}%
\pgfpathcurveto{\pgfqpoint{0.008333in}{0.002210in}}{\pgfqpoint{0.007455in}{0.004330in}}{\pgfqpoint{0.005893in}{0.005893in}}%
\pgfpathcurveto{\pgfqpoint{0.004330in}{0.007455in}}{\pgfqpoint{0.002210in}{0.008333in}}{\pgfqpoint{0.000000in}{0.008333in}}%
\pgfpathcurveto{\pgfqpoint{-0.002210in}{0.008333in}}{\pgfqpoint{-0.004330in}{0.007455in}}{\pgfqpoint{-0.005893in}{0.005893in}}%
\pgfpathcurveto{\pgfqpoint{-0.007455in}{0.004330in}}{\pgfqpoint{-0.008333in}{0.002210in}}{\pgfqpoint{-0.008333in}{0.000000in}}%
\pgfpathcurveto{\pgfqpoint{-0.008333in}{-0.002210in}}{\pgfqpoint{-0.007455in}{-0.004330in}}{\pgfqpoint{-0.005893in}{-0.005893in}}%
\pgfpathcurveto{\pgfqpoint{-0.004330in}{-0.007455in}}{\pgfqpoint{-0.002210in}{-0.008333in}}{\pgfqpoint{0.000000in}{-0.008333in}}%
\pgfpathclose%
\pgfusepath{stroke,fill}%
}%
\begin{pgfscope}%
\pgfsys@transformshift{4.424090in}{3.883005in}%
\pgfsys@useobject{currentmarker}{}%
\end{pgfscope}%
\end{pgfscope}%
\begin{pgfscope}%
\pgfpathrectangle{\pgfqpoint{0.100000in}{2.413063in}}{\pgfqpoint{5.037500in}{3.427208in}}%
\pgfusepath{clip}%
\pgfsetrectcap%
\pgfsetroundjoin%
\pgfsetlinewidth{1.505625pt}%
\definecolor{currentstroke}{rgb}{0.501961,0.501961,0.501961}%
\pgfsetstrokecolor{currentstroke}%
\pgfsetstrokeopacity{0.500000}%
\pgfsetdash{}{0pt}%
\pgfpathmoveto{\pgfqpoint{4.172799in}{4.058209in}}%
\pgfusepath{stroke}%
\end{pgfscope}%
\begin{pgfscope}%
\pgfpathrectangle{\pgfqpoint{0.100000in}{2.413063in}}{\pgfqpoint{5.037500in}{3.427208in}}%
\pgfusepath{clip}%
\pgfsetbuttcap%
\pgfsetroundjoin%
\definecolor{currentfill}{rgb}{0.501961,0.501961,0.501961}%
\pgfsetfillcolor{currentfill}%
\pgfsetfillopacity{0.500000}%
\pgfsetlinewidth{0.250937pt}%
\definecolor{currentstroke}{rgb}{0.000000,0.000000,0.000000}%
\pgfsetstrokecolor{currentstroke}%
\pgfsetstrokeopacity{0.500000}%
\pgfsetdash{}{0pt}%
\pgfsys@defobject{currentmarker}{\pgfqpoint{-0.013889in}{-0.013889in}}{\pgfqpoint{0.013889in}{0.013889in}}{%
\pgfpathmoveto{\pgfqpoint{0.000000in}{-0.013889in}}%
\pgfpathcurveto{\pgfqpoint{0.003683in}{-0.013889in}}{\pgfqpoint{0.007216in}{-0.012425in}}{\pgfqpoint{0.009821in}{-0.009821in}}%
\pgfpathcurveto{\pgfqpoint{0.012425in}{-0.007216in}}{\pgfqpoint{0.013889in}{-0.003683in}}{\pgfqpoint{0.013889in}{0.000000in}}%
\pgfpathcurveto{\pgfqpoint{0.013889in}{0.003683in}}{\pgfqpoint{0.012425in}{0.007216in}}{\pgfqpoint{0.009821in}{0.009821in}}%
\pgfpathcurveto{\pgfqpoint{0.007216in}{0.012425in}}{\pgfqpoint{0.003683in}{0.013889in}}{\pgfqpoint{0.000000in}{0.013889in}}%
\pgfpathcurveto{\pgfqpoint{-0.003683in}{0.013889in}}{\pgfqpoint{-0.007216in}{0.012425in}}{\pgfqpoint{-0.009821in}{0.009821in}}%
\pgfpathcurveto{\pgfqpoint{-0.012425in}{0.007216in}}{\pgfqpoint{-0.013889in}{0.003683in}}{\pgfqpoint{-0.013889in}{0.000000in}}%
\pgfpathcurveto{\pgfqpoint{-0.013889in}{-0.003683in}}{\pgfqpoint{-0.012425in}{-0.007216in}}{\pgfqpoint{-0.009821in}{-0.009821in}}%
\pgfpathcurveto{\pgfqpoint{-0.007216in}{-0.012425in}}{\pgfqpoint{-0.003683in}{-0.013889in}}{\pgfqpoint{0.000000in}{-0.013889in}}%
\pgfpathclose%
\pgfusepath{stroke,fill}%
}%
\begin{pgfscope}%
\pgfsys@transformshift{4.172799in}{4.058209in}%
\pgfsys@useobject{currentmarker}{}%
\end{pgfscope}%
\end{pgfscope}%
\begin{pgfscope}%
\pgfpathrectangle{\pgfqpoint{0.100000in}{2.413063in}}{\pgfqpoint{5.037500in}{3.427208in}}%
\pgfusepath{clip}%
\pgfsetrectcap%
\pgfsetroundjoin%
\pgfsetlinewidth{1.505625pt}%
\definecolor{currentstroke}{rgb}{0.000000,0.000000,1.000000}%
\pgfsetstrokecolor{currentstroke}%
\pgfsetstrokeopacity{0.500000}%
\pgfsetdash{}{0pt}%
\pgfpathmoveto{\pgfqpoint{2.344274in}{5.197479in}}%
\pgfusepath{stroke}%
\end{pgfscope}%
\begin{pgfscope}%
\pgfpathrectangle{\pgfqpoint{0.100000in}{2.413063in}}{\pgfqpoint{5.037500in}{3.427208in}}%
\pgfusepath{clip}%
\pgfsetbuttcap%
\pgfsetroundjoin%
\definecolor{currentfill}{rgb}{0.000000,0.000000,1.000000}%
\pgfsetfillcolor{currentfill}%
\pgfsetfillopacity{0.500000}%
\pgfsetlinewidth{0.250937pt}%
\definecolor{currentstroke}{rgb}{0.000000,0.000000,0.000000}%
\pgfsetstrokecolor{currentstroke}%
\pgfsetstrokeopacity{0.500000}%
\pgfsetdash{}{0pt}%
\pgfsys@defobject{currentmarker}{\pgfqpoint{-0.008333in}{-0.008333in}}{\pgfqpoint{0.008333in}{0.008333in}}{%
\pgfpathmoveto{\pgfqpoint{0.000000in}{-0.008333in}}%
\pgfpathcurveto{\pgfqpoint{0.002210in}{-0.008333in}}{\pgfqpoint{0.004330in}{-0.007455in}}{\pgfqpoint{0.005893in}{-0.005893in}}%
\pgfpathcurveto{\pgfqpoint{0.007455in}{-0.004330in}}{\pgfqpoint{0.008333in}{-0.002210in}}{\pgfqpoint{0.008333in}{0.000000in}}%
\pgfpathcurveto{\pgfqpoint{0.008333in}{0.002210in}}{\pgfqpoint{0.007455in}{0.004330in}}{\pgfqpoint{0.005893in}{0.005893in}}%
\pgfpathcurveto{\pgfqpoint{0.004330in}{0.007455in}}{\pgfqpoint{0.002210in}{0.008333in}}{\pgfqpoint{0.000000in}{0.008333in}}%
\pgfpathcurveto{\pgfqpoint{-0.002210in}{0.008333in}}{\pgfqpoint{-0.004330in}{0.007455in}}{\pgfqpoint{-0.005893in}{0.005893in}}%
\pgfpathcurveto{\pgfqpoint{-0.007455in}{0.004330in}}{\pgfqpoint{-0.008333in}{0.002210in}}{\pgfqpoint{-0.008333in}{0.000000in}}%
\pgfpathcurveto{\pgfqpoint{-0.008333in}{-0.002210in}}{\pgfqpoint{-0.007455in}{-0.004330in}}{\pgfqpoint{-0.005893in}{-0.005893in}}%
\pgfpathcurveto{\pgfqpoint{-0.004330in}{-0.007455in}}{\pgfqpoint{-0.002210in}{-0.008333in}}{\pgfqpoint{0.000000in}{-0.008333in}}%
\pgfpathclose%
\pgfusepath{stroke,fill}%
}%
\begin{pgfscope}%
\pgfsys@transformshift{2.344274in}{5.197479in}%
\pgfsys@useobject{currentmarker}{}%
\end{pgfscope}%
\end{pgfscope}%
\begin{pgfscope}%
\pgfpathrectangle{\pgfqpoint{0.100000in}{2.413063in}}{\pgfqpoint{5.037500in}{3.427208in}}%
\pgfusepath{clip}%
\pgfsetrectcap%
\pgfsetroundjoin%
\pgfsetlinewidth{1.505625pt}%
\definecolor{currentstroke}{rgb}{0.501961,0.501961,0.501961}%
\pgfsetstrokecolor{currentstroke}%
\pgfsetstrokeopacity{0.500000}%
\pgfsetdash{}{0pt}%
\pgfpathmoveto{\pgfqpoint{2.661894in}{5.192252in}}%
\pgfusepath{stroke}%
\end{pgfscope}%
\begin{pgfscope}%
\pgfpathrectangle{\pgfqpoint{0.100000in}{2.413063in}}{\pgfqpoint{5.037500in}{3.427208in}}%
\pgfusepath{clip}%
\pgfsetbuttcap%
\pgfsetroundjoin%
\definecolor{currentfill}{rgb}{0.501961,0.501961,0.501961}%
\pgfsetfillcolor{currentfill}%
\pgfsetfillopacity{0.500000}%
\pgfsetlinewidth{0.250937pt}%
\definecolor{currentstroke}{rgb}{0.000000,0.000000,0.000000}%
\pgfsetstrokecolor{currentstroke}%
\pgfsetstrokeopacity{0.500000}%
\pgfsetdash{}{0pt}%
\pgfsys@defobject{currentmarker}{\pgfqpoint{-0.013889in}{-0.013889in}}{\pgfqpoint{0.013889in}{0.013889in}}{%
\pgfpathmoveto{\pgfqpoint{0.000000in}{-0.013889in}}%
\pgfpathcurveto{\pgfqpoint{0.003683in}{-0.013889in}}{\pgfqpoint{0.007216in}{-0.012425in}}{\pgfqpoint{0.009821in}{-0.009821in}}%
\pgfpathcurveto{\pgfqpoint{0.012425in}{-0.007216in}}{\pgfqpoint{0.013889in}{-0.003683in}}{\pgfqpoint{0.013889in}{0.000000in}}%
\pgfpathcurveto{\pgfqpoint{0.013889in}{0.003683in}}{\pgfqpoint{0.012425in}{0.007216in}}{\pgfqpoint{0.009821in}{0.009821in}}%
\pgfpathcurveto{\pgfqpoint{0.007216in}{0.012425in}}{\pgfqpoint{0.003683in}{0.013889in}}{\pgfqpoint{0.000000in}{0.013889in}}%
\pgfpathcurveto{\pgfqpoint{-0.003683in}{0.013889in}}{\pgfqpoint{-0.007216in}{0.012425in}}{\pgfqpoint{-0.009821in}{0.009821in}}%
\pgfpathcurveto{\pgfqpoint{-0.012425in}{0.007216in}}{\pgfqpoint{-0.013889in}{0.003683in}}{\pgfqpoint{-0.013889in}{0.000000in}}%
\pgfpathcurveto{\pgfqpoint{-0.013889in}{-0.003683in}}{\pgfqpoint{-0.012425in}{-0.007216in}}{\pgfqpoint{-0.009821in}{-0.009821in}}%
\pgfpathcurveto{\pgfqpoint{-0.007216in}{-0.012425in}}{\pgfqpoint{-0.003683in}{-0.013889in}}{\pgfqpoint{0.000000in}{-0.013889in}}%
\pgfpathclose%
\pgfusepath{stroke,fill}%
}%
\begin{pgfscope}%
\pgfsys@transformshift{2.661894in}{5.192252in}%
\pgfsys@useobject{currentmarker}{}%
\end{pgfscope}%
\end{pgfscope}%
\begin{pgfscope}%
\pgfpathrectangle{\pgfqpoint{0.100000in}{2.413063in}}{\pgfqpoint{5.037500in}{3.427208in}}%
\pgfusepath{clip}%
\pgfsetrectcap%
\pgfsetroundjoin%
\pgfsetlinewidth{1.505625pt}%
\definecolor{currentstroke}{rgb}{0.000000,0.000000,1.000000}%
\pgfsetstrokecolor{currentstroke}%
\pgfsetstrokeopacity{0.500000}%
\pgfsetdash{}{0pt}%
\pgfpathmoveto{\pgfqpoint{2.643208in}{5.312659in}}%
\pgfusepath{stroke}%
\end{pgfscope}%
\begin{pgfscope}%
\pgfpathrectangle{\pgfqpoint{0.100000in}{2.413063in}}{\pgfqpoint{5.037500in}{3.427208in}}%
\pgfusepath{clip}%
\pgfsetbuttcap%
\pgfsetroundjoin%
\definecolor{currentfill}{rgb}{0.000000,0.000000,1.000000}%
\pgfsetfillcolor{currentfill}%
\pgfsetfillopacity{0.500000}%
\pgfsetlinewidth{0.250937pt}%
\definecolor{currentstroke}{rgb}{0.000000,0.000000,0.000000}%
\pgfsetstrokecolor{currentstroke}%
\pgfsetstrokeopacity{0.500000}%
\pgfsetdash{}{0pt}%
\pgfsys@defobject{currentmarker}{\pgfqpoint{-0.005556in}{-0.005556in}}{\pgfqpoint{0.005556in}{0.005556in}}{%
\pgfpathmoveto{\pgfqpoint{0.000000in}{-0.005556in}}%
\pgfpathcurveto{\pgfqpoint{0.001473in}{-0.005556in}}{\pgfqpoint{0.002887in}{-0.004970in}}{\pgfqpoint{0.003928in}{-0.003928in}}%
\pgfpathcurveto{\pgfqpoint{0.004970in}{-0.002887in}}{\pgfqpoint{0.005556in}{-0.001473in}}{\pgfqpoint{0.005556in}{0.000000in}}%
\pgfpathcurveto{\pgfqpoint{0.005556in}{0.001473in}}{\pgfqpoint{0.004970in}{0.002887in}}{\pgfqpoint{0.003928in}{0.003928in}}%
\pgfpathcurveto{\pgfqpoint{0.002887in}{0.004970in}}{\pgfqpoint{0.001473in}{0.005556in}}{\pgfqpoint{0.000000in}{0.005556in}}%
\pgfpathcurveto{\pgfqpoint{-0.001473in}{0.005556in}}{\pgfqpoint{-0.002887in}{0.004970in}}{\pgfqpoint{-0.003928in}{0.003928in}}%
\pgfpathcurveto{\pgfqpoint{-0.004970in}{0.002887in}}{\pgfqpoint{-0.005556in}{0.001473in}}{\pgfqpoint{-0.005556in}{0.000000in}}%
\pgfpathcurveto{\pgfqpoint{-0.005556in}{-0.001473in}}{\pgfqpoint{-0.004970in}{-0.002887in}}{\pgfqpoint{-0.003928in}{-0.003928in}}%
\pgfpathcurveto{\pgfqpoint{-0.002887in}{-0.004970in}}{\pgfqpoint{-0.001473in}{-0.005556in}}{\pgfqpoint{0.000000in}{-0.005556in}}%
\pgfpathclose%
\pgfusepath{stroke,fill}%
}%
\begin{pgfscope}%
\pgfsys@transformshift{2.643208in}{5.312659in}%
\pgfsys@useobject{currentmarker}{}%
\end{pgfscope}%
\end{pgfscope}%
\begin{pgfscope}%
\pgfpathrectangle{\pgfqpoint{0.100000in}{2.413063in}}{\pgfqpoint{5.037500in}{3.427208in}}%
\pgfusepath{clip}%
\pgfsetrectcap%
\pgfsetroundjoin%
\pgfsetlinewidth{1.505625pt}%
\definecolor{currentstroke}{rgb}{0.678431,1.000000,0.184314}%
\pgfsetstrokecolor{currentstroke}%
\pgfsetstrokeopacity{0.500000}%
\pgfsetdash{}{0pt}%
\pgfpathmoveto{\pgfqpoint{3.970658in}{4.607541in}}%
\pgfusepath{stroke}%
\end{pgfscope}%
\begin{pgfscope}%
\pgfpathrectangle{\pgfqpoint{0.100000in}{2.413063in}}{\pgfqpoint{5.037500in}{3.427208in}}%
\pgfusepath{clip}%
\pgfsetbuttcap%
\pgfsetroundjoin%
\definecolor{currentfill}{rgb}{0.678431,1.000000,0.184314}%
\pgfsetfillcolor{currentfill}%
\pgfsetfillopacity{0.500000}%
\pgfsetlinewidth{0.250937pt}%
\definecolor{currentstroke}{rgb}{0.000000,0.000000,0.000000}%
\pgfsetstrokecolor{currentstroke}%
\pgfsetstrokeopacity{0.500000}%
\pgfsetdash{}{0pt}%
\pgfsys@defobject{currentmarker}{\pgfqpoint{-0.019444in}{-0.019444in}}{\pgfqpoint{0.019444in}{0.019444in}}{%
\pgfpathmoveto{\pgfqpoint{0.000000in}{-0.019444in}}%
\pgfpathcurveto{\pgfqpoint{0.005157in}{-0.019444in}}{\pgfqpoint{0.010103in}{-0.017396in}}{\pgfqpoint{0.013749in}{-0.013749in}}%
\pgfpathcurveto{\pgfqpoint{0.017396in}{-0.010103in}}{\pgfqpoint{0.019444in}{-0.005157in}}{\pgfqpoint{0.019444in}{0.000000in}}%
\pgfpathcurveto{\pgfqpoint{0.019444in}{0.005157in}}{\pgfqpoint{0.017396in}{0.010103in}}{\pgfqpoint{0.013749in}{0.013749in}}%
\pgfpathcurveto{\pgfqpoint{0.010103in}{0.017396in}}{\pgfqpoint{0.005157in}{0.019444in}}{\pgfqpoint{0.000000in}{0.019444in}}%
\pgfpathcurveto{\pgfqpoint{-0.005157in}{0.019444in}}{\pgfqpoint{-0.010103in}{0.017396in}}{\pgfqpoint{-0.013749in}{0.013749in}}%
\pgfpathcurveto{\pgfqpoint{-0.017396in}{0.010103in}}{\pgfqpoint{-0.019444in}{0.005157in}}{\pgfqpoint{-0.019444in}{0.000000in}}%
\pgfpathcurveto{\pgfqpoint{-0.019444in}{-0.005157in}}{\pgfqpoint{-0.017396in}{-0.010103in}}{\pgfqpoint{-0.013749in}{-0.013749in}}%
\pgfpathcurveto{\pgfqpoint{-0.010103in}{-0.017396in}}{\pgfqpoint{-0.005157in}{-0.019444in}}{\pgfqpoint{0.000000in}{-0.019444in}}%
\pgfpathclose%
\pgfusepath{stroke,fill}%
}%
\begin{pgfscope}%
\pgfsys@transformshift{3.970658in}{4.607541in}%
\pgfsys@useobject{currentmarker}{}%
\end{pgfscope}%
\end{pgfscope}%
\begin{pgfscope}%
\pgfpathrectangle{\pgfqpoint{0.100000in}{2.413063in}}{\pgfqpoint{5.037500in}{3.427208in}}%
\pgfusepath{clip}%
\pgfsetrectcap%
\pgfsetroundjoin%
\pgfsetlinewidth{1.505625pt}%
\definecolor{currentstroke}{rgb}{0.678431,1.000000,0.184314}%
\pgfsetstrokecolor{currentstroke}%
\pgfsetstrokeopacity{0.500000}%
\pgfsetdash{}{0pt}%
\pgfpathmoveto{\pgfqpoint{3.987573in}{4.576983in}}%
\pgfusepath{stroke}%
\end{pgfscope}%
\begin{pgfscope}%
\pgfpathrectangle{\pgfqpoint{0.100000in}{2.413063in}}{\pgfqpoint{5.037500in}{3.427208in}}%
\pgfusepath{clip}%
\pgfsetbuttcap%
\pgfsetroundjoin%
\definecolor{currentfill}{rgb}{0.678431,1.000000,0.184314}%
\pgfsetfillcolor{currentfill}%
\pgfsetfillopacity{0.500000}%
\pgfsetlinewidth{0.250937pt}%
\definecolor{currentstroke}{rgb}{0.000000,0.000000,0.000000}%
\pgfsetstrokecolor{currentstroke}%
\pgfsetstrokeopacity{0.500000}%
\pgfsetdash{}{0pt}%
\pgfsys@defobject{currentmarker}{\pgfqpoint{-0.030556in}{-0.030556in}}{\pgfqpoint{0.030556in}{0.030556in}}{%
\pgfpathmoveto{\pgfqpoint{0.000000in}{-0.030556in}}%
\pgfpathcurveto{\pgfqpoint{0.008103in}{-0.030556in}}{\pgfqpoint{0.015876in}{-0.027336in}}{\pgfqpoint{0.021606in}{-0.021606in}}%
\pgfpathcurveto{\pgfqpoint{0.027336in}{-0.015876in}}{\pgfqpoint{0.030556in}{-0.008103in}}{\pgfqpoint{0.030556in}{0.000000in}}%
\pgfpathcurveto{\pgfqpoint{0.030556in}{0.008103in}}{\pgfqpoint{0.027336in}{0.015876in}}{\pgfqpoint{0.021606in}{0.021606in}}%
\pgfpathcurveto{\pgfqpoint{0.015876in}{0.027336in}}{\pgfqpoint{0.008103in}{0.030556in}}{\pgfqpoint{0.000000in}{0.030556in}}%
\pgfpathcurveto{\pgfqpoint{-0.008103in}{0.030556in}}{\pgfqpoint{-0.015876in}{0.027336in}}{\pgfqpoint{-0.021606in}{0.021606in}}%
\pgfpathcurveto{\pgfqpoint{-0.027336in}{0.015876in}}{\pgfqpoint{-0.030556in}{0.008103in}}{\pgfqpoint{-0.030556in}{0.000000in}}%
\pgfpathcurveto{\pgfqpoint{-0.030556in}{-0.008103in}}{\pgfqpoint{-0.027336in}{-0.015876in}}{\pgfqpoint{-0.021606in}{-0.021606in}}%
\pgfpathcurveto{\pgfqpoint{-0.015876in}{-0.027336in}}{\pgfqpoint{-0.008103in}{-0.030556in}}{\pgfqpoint{0.000000in}{-0.030556in}}%
\pgfpathclose%
\pgfusepath{stroke,fill}%
}%
\begin{pgfscope}%
\pgfsys@transformshift{3.987573in}{4.576983in}%
\pgfsys@useobject{currentmarker}{}%
\end{pgfscope}%
\end{pgfscope}%
\begin{pgfscope}%
\pgfpathrectangle{\pgfqpoint{0.100000in}{2.413063in}}{\pgfqpoint{5.037500in}{3.427208in}}%
\pgfusepath{clip}%
\pgfsetrectcap%
\pgfsetroundjoin%
\pgfsetlinewidth{1.505625pt}%
\definecolor{currentstroke}{rgb}{0.678431,1.000000,0.184314}%
\pgfsetstrokecolor{currentstroke}%
\pgfsetstrokeopacity{0.500000}%
\pgfsetdash{}{0pt}%
\pgfpathmoveto{\pgfqpoint{3.739192in}{4.346646in}}%
\pgfusepath{stroke}%
\end{pgfscope}%
\begin{pgfscope}%
\pgfpathrectangle{\pgfqpoint{0.100000in}{2.413063in}}{\pgfqpoint{5.037500in}{3.427208in}}%
\pgfusepath{clip}%
\pgfsetbuttcap%
\pgfsetroundjoin%
\definecolor{currentfill}{rgb}{0.678431,1.000000,0.184314}%
\pgfsetfillcolor{currentfill}%
\pgfsetfillopacity{0.500000}%
\pgfsetlinewidth{0.250937pt}%
\definecolor{currentstroke}{rgb}{0.000000,0.000000,0.000000}%
\pgfsetstrokecolor{currentstroke}%
\pgfsetstrokeopacity{0.500000}%
\pgfsetdash{}{0pt}%
\pgfsys@defobject{currentmarker}{\pgfqpoint{-0.016667in}{-0.016667in}}{\pgfqpoint{0.016667in}{0.016667in}}{%
\pgfpathmoveto{\pgfqpoint{0.000000in}{-0.016667in}}%
\pgfpathcurveto{\pgfqpoint{0.004420in}{-0.016667in}}{\pgfqpoint{0.008660in}{-0.014911in}}{\pgfqpoint{0.011785in}{-0.011785in}}%
\pgfpathcurveto{\pgfqpoint{0.014911in}{-0.008660in}}{\pgfqpoint{0.016667in}{-0.004420in}}{\pgfqpoint{0.016667in}{0.000000in}}%
\pgfpathcurveto{\pgfqpoint{0.016667in}{0.004420in}}{\pgfqpoint{0.014911in}{0.008660in}}{\pgfqpoint{0.011785in}{0.011785in}}%
\pgfpathcurveto{\pgfqpoint{0.008660in}{0.014911in}}{\pgfqpoint{0.004420in}{0.016667in}}{\pgfqpoint{0.000000in}{0.016667in}}%
\pgfpathcurveto{\pgfqpoint{-0.004420in}{0.016667in}}{\pgfqpoint{-0.008660in}{0.014911in}}{\pgfqpoint{-0.011785in}{0.011785in}}%
\pgfpathcurveto{\pgfqpoint{-0.014911in}{0.008660in}}{\pgfqpoint{-0.016667in}{0.004420in}}{\pgfqpoint{-0.016667in}{0.000000in}}%
\pgfpathcurveto{\pgfqpoint{-0.016667in}{-0.004420in}}{\pgfqpoint{-0.014911in}{-0.008660in}}{\pgfqpoint{-0.011785in}{-0.011785in}}%
\pgfpathcurveto{\pgfqpoint{-0.008660in}{-0.014911in}}{\pgfqpoint{-0.004420in}{-0.016667in}}{\pgfqpoint{0.000000in}{-0.016667in}}%
\pgfpathclose%
\pgfusepath{stroke,fill}%
}%
\begin{pgfscope}%
\pgfsys@transformshift{3.739192in}{4.346646in}%
\pgfsys@useobject{currentmarker}{}%
\end{pgfscope}%
\end{pgfscope}%
\begin{pgfscope}%
\pgfpathrectangle{\pgfqpoint{0.100000in}{2.413063in}}{\pgfqpoint{5.037500in}{3.427208in}}%
\pgfusepath{clip}%
\pgfsetrectcap%
\pgfsetroundjoin%
\pgfsetlinewidth{1.505625pt}%
\definecolor{currentstroke}{rgb}{0.000000,0.000000,1.000000}%
\pgfsetstrokecolor{currentstroke}%
\pgfsetstrokeopacity{0.500000}%
\pgfsetdash{}{0pt}%
\pgfpathmoveto{\pgfqpoint{3.948532in}{4.653429in}}%
\pgfusepath{stroke}%
\end{pgfscope}%
\begin{pgfscope}%
\pgfpathrectangle{\pgfqpoint{0.100000in}{2.413063in}}{\pgfqpoint{5.037500in}{3.427208in}}%
\pgfusepath{clip}%
\pgfsetbuttcap%
\pgfsetroundjoin%
\definecolor{currentfill}{rgb}{0.000000,0.000000,1.000000}%
\pgfsetfillcolor{currentfill}%
\pgfsetfillopacity{0.500000}%
\pgfsetlinewidth{0.250937pt}%
\definecolor{currentstroke}{rgb}{0.000000,0.000000,0.000000}%
\pgfsetstrokecolor{currentstroke}%
\pgfsetstrokeopacity{0.500000}%
\pgfsetdash{}{0pt}%
\pgfsys@defobject{currentmarker}{\pgfqpoint{-0.019444in}{-0.019444in}}{\pgfqpoint{0.019444in}{0.019444in}}{%
\pgfpathmoveto{\pgfqpoint{0.000000in}{-0.019444in}}%
\pgfpathcurveto{\pgfqpoint{0.005157in}{-0.019444in}}{\pgfqpoint{0.010103in}{-0.017396in}}{\pgfqpoint{0.013749in}{-0.013749in}}%
\pgfpathcurveto{\pgfqpoint{0.017396in}{-0.010103in}}{\pgfqpoint{0.019444in}{-0.005157in}}{\pgfqpoint{0.019444in}{0.000000in}}%
\pgfpathcurveto{\pgfqpoint{0.019444in}{0.005157in}}{\pgfqpoint{0.017396in}{0.010103in}}{\pgfqpoint{0.013749in}{0.013749in}}%
\pgfpathcurveto{\pgfqpoint{0.010103in}{0.017396in}}{\pgfqpoint{0.005157in}{0.019444in}}{\pgfqpoint{0.000000in}{0.019444in}}%
\pgfpathcurveto{\pgfqpoint{-0.005157in}{0.019444in}}{\pgfqpoint{-0.010103in}{0.017396in}}{\pgfqpoint{-0.013749in}{0.013749in}}%
\pgfpathcurveto{\pgfqpoint{-0.017396in}{0.010103in}}{\pgfqpoint{-0.019444in}{0.005157in}}{\pgfqpoint{-0.019444in}{0.000000in}}%
\pgfpathcurveto{\pgfqpoint{-0.019444in}{-0.005157in}}{\pgfqpoint{-0.017396in}{-0.010103in}}{\pgfqpoint{-0.013749in}{-0.013749in}}%
\pgfpathcurveto{\pgfqpoint{-0.010103in}{-0.017396in}}{\pgfqpoint{-0.005157in}{-0.019444in}}{\pgfqpoint{0.000000in}{-0.019444in}}%
\pgfpathclose%
\pgfusepath{stroke,fill}%
}%
\begin{pgfscope}%
\pgfsys@transformshift{3.948532in}{4.653429in}%
\pgfsys@useobject{currentmarker}{}%
\end{pgfscope}%
\end{pgfscope}%
\begin{pgfscope}%
\pgfpathrectangle{\pgfqpoint{0.100000in}{2.413063in}}{\pgfqpoint{5.037500in}{3.427208in}}%
\pgfusepath{clip}%
\pgfsetrectcap%
\pgfsetroundjoin%
\pgfsetlinewidth{1.505625pt}%
\definecolor{currentstroke}{rgb}{0.678431,1.000000,0.184314}%
\pgfsetstrokecolor{currentstroke}%
\pgfsetstrokeopacity{0.500000}%
\pgfsetdash{}{0pt}%
\pgfpathmoveto{\pgfqpoint{3.860194in}{4.461577in}}%
\pgfusepath{stroke}%
\end{pgfscope}%
\begin{pgfscope}%
\pgfpathrectangle{\pgfqpoint{0.100000in}{2.413063in}}{\pgfqpoint{5.037500in}{3.427208in}}%
\pgfusepath{clip}%
\pgfsetbuttcap%
\pgfsetroundjoin%
\definecolor{currentfill}{rgb}{0.678431,1.000000,0.184314}%
\pgfsetfillcolor{currentfill}%
\pgfsetfillopacity{0.500000}%
\pgfsetlinewidth{0.250937pt}%
\definecolor{currentstroke}{rgb}{0.000000,0.000000,0.000000}%
\pgfsetstrokecolor{currentstroke}%
\pgfsetstrokeopacity{0.500000}%
\pgfsetdash{}{0pt}%
\pgfsys@defobject{currentmarker}{\pgfqpoint{-0.016667in}{-0.016667in}}{\pgfqpoint{0.016667in}{0.016667in}}{%
\pgfpathmoveto{\pgfqpoint{0.000000in}{-0.016667in}}%
\pgfpathcurveto{\pgfqpoint{0.004420in}{-0.016667in}}{\pgfqpoint{0.008660in}{-0.014911in}}{\pgfqpoint{0.011785in}{-0.011785in}}%
\pgfpathcurveto{\pgfqpoint{0.014911in}{-0.008660in}}{\pgfqpoint{0.016667in}{-0.004420in}}{\pgfqpoint{0.016667in}{0.000000in}}%
\pgfpathcurveto{\pgfqpoint{0.016667in}{0.004420in}}{\pgfqpoint{0.014911in}{0.008660in}}{\pgfqpoint{0.011785in}{0.011785in}}%
\pgfpathcurveto{\pgfqpoint{0.008660in}{0.014911in}}{\pgfqpoint{0.004420in}{0.016667in}}{\pgfqpoint{0.000000in}{0.016667in}}%
\pgfpathcurveto{\pgfqpoint{-0.004420in}{0.016667in}}{\pgfqpoint{-0.008660in}{0.014911in}}{\pgfqpoint{-0.011785in}{0.011785in}}%
\pgfpathcurveto{\pgfqpoint{-0.014911in}{0.008660in}}{\pgfqpoint{-0.016667in}{0.004420in}}{\pgfqpoint{-0.016667in}{0.000000in}}%
\pgfpathcurveto{\pgfqpoint{-0.016667in}{-0.004420in}}{\pgfqpoint{-0.014911in}{-0.008660in}}{\pgfqpoint{-0.011785in}{-0.011785in}}%
\pgfpathcurveto{\pgfqpoint{-0.008660in}{-0.014911in}}{\pgfqpoint{-0.004420in}{-0.016667in}}{\pgfqpoint{0.000000in}{-0.016667in}}%
\pgfpathclose%
\pgfusepath{stroke,fill}%
}%
\begin{pgfscope}%
\pgfsys@transformshift{3.860194in}{4.461577in}%
\pgfsys@useobject{currentmarker}{}%
\end{pgfscope}%
\end{pgfscope}%
\begin{pgfscope}%
\pgfpathrectangle{\pgfqpoint{0.100000in}{2.413063in}}{\pgfqpoint{5.037500in}{3.427208in}}%
\pgfusepath{clip}%
\pgfsetrectcap%
\pgfsetroundjoin%
\pgfsetlinewidth{1.505625pt}%
\definecolor{currentstroke}{rgb}{0.678431,1.000000,0.184314}%
\pgfsetstrokecolor{currentstroke}%
\pgfsetstrokeopacity{0.500000}%
\pgfsetdash{}{0pt}%
\pgfpathmoveto{\pgfqpoint{3.758682in}{4.425173in}}%
\pgfusepath{stroke}%
\end{pgfscope}%
\begin{pgfscope}%
\pgfpathrectangle{\pgfqpoint{0.100000in}{2.413063in}}{\pgfqpoint{5.037500in}{3.427208in}}%
\pgfusepath{clip}%
\pgfsetbuttcap%
\pgfsetroundjoin%
\definecolor{currentfill}{rgb}{0.678431,1.000000,0.184314}%
\pgfsetfillcolor{currentfill}%
\pgfsetfillopacity{0.500000}%
\pgfsetlinewidth{0.250937pt}%
\definecolor{currentstroke}{rgb}{0.000000,0.000000,0.000000}%
\pgfsetstrokecolor{currentstroke}%
\pgfsetstrokeopacity{0.500000}%
\pgfsetdash{}{0pt}%
\pgfsys@defobject{currentmarker}{\pgfqpoint{-0.016667in}{-0.016667in}}{\pgfqpoint{0.016667in}{0.016667in}}{%
\pgfpathmoveto{\pgfqpoint{0.000000in}{-0.016667in}}%
\pgfpathcurveto{\pgfqpoint{0.004420in}{-0.016667in}}{\pgfqpoint{0.008660in}{-0.014911in}}{\pgfqpoint{0.011785in}{-0.011785in}}%
\pgfpathcurveto{\pgfqpoint{0.014911in}{-0.008660in}}{\pgfqpoint{0.016667in}{-0.004420in}}{\pgfqpoint{0.016667in}{0.000000in}}%
\pgfpathcurveto{\pgfqpoint{0.016667in}{0.004420in}}{\pgfqpoint{0.014911in}{0.008660in}}{\pgfqpoint{0.011785in}{0.011785in}}%
\pgfpathcurveto{\pgfqpoint{0.008660in}{0.014911in}}{\pgfqpoint{0.004420in}{0.016667in}}{\pgfqpoint{0.000000in}{0.016667in}}%
\pgfpathcurveto{\pgfqpoint{-0.004420in}{0.016667in}}{\pgfqpoint{-0.008660in}{0.014911in}}{\pgfqpoint{-0.011785in}{0.011785in}}%
\pgfpathcurveto{\pgfqpoint{-0.014911in}{0.008660in}}{\pgfqpoint{-0.016667in}{0.004420in}}{\pgfqpoint{-0.016667in}{0.000000in}}%
\pgfpathcurveto{\pgfqpoint{-0.016667in}{-0.004420in}}{\pgfqpoint{-0.014911in}{-0.008660in}}{\pgfqpoint{-0.011785in}{-0.011785in}}%
\pgfpathcurveto{\pgfqpoint{-0.008660in}{-0.014911in}}{\pgfqpoint{-0.004420in}{-0.016667in}}{\pgfqpoint{0.000000in}{-0.016667in}}%
\pgfpathclose%
\pgfusepath{stroke,fill}%
}%
\begin{pgfscope}%
\pgfsys@transformshift{3.758682in}{4.425173in}%
\pgfsys@useobject{currentmarker}{}%
\end{pgfscope}%
\end{pgfscope}%
\begin{pgfscope}%
\pgfpathrectangle{\pgfqpoint{0.100000in}{2.413063in}}{\pgfqpoint{5.037500in}{3.427208in}}%
\pgfusepath{clip}%
\pgfsetrectcap%
\pgfsetroundjoin%
\pgfsetlinewidth{1.505625pt}%
\definecolor{currentstroke}{rgb}{0.678431,1.000000,0.184314}%
\pgfsetstrokecolor{currentstroke}%
\pgfsetstrokeopacity{0.500000}%
\pgfsetdash{}{0pt}%
\pgfpathmoveto{\pgfqpoint{3.752726in}{4.538555in}}%
\pgfusepath{stroke}%
\end{pgfscope}%
\begin{pgfscope}%
\pgfpathrectangle{\pgfqpoint{0.100000in}{2.413063in}}{\pgfqpoint{5.037500in}{3.427208in}}%
\pgfusepath{clip}%
\pgfsetbuttcap%
\pgfsetroundjoin%
\definecolor{currentfill}{rgb}{0.678431,1.000000,0.184314}%
\pgfsetfillcolor{currentfill}%
\pgfsetfillopacity{0.500000}%
\pgfsetlinewidth{0.250937pt}%
\definecolor{currentstroke}{rgb}{0.000000,0.000000,0.000000}%
\pgfsetstrokecolor{currentstroke}%
\pgfsetstrokeopacity{0.500000}%
\pgfsetdash{}{0pt}%
\pgfsys@defobject{currentmarker}{\pgfqpoint{-0.016667in}{-0.016667in}}{\pgfqpoint{0.016667in}{0.016667in}}{%
\pgfpathmoveto{\pgfqpoint{0.000000in}{-0.016667in}}%
\pgfpathcurveto{\pgfqpoint{0.004420in}{-0.016667in}}{\pgfqpoint{0.008660in}{-0.014911in}}{\pgfqpoint{0.011785in}{-0.011785in}}%
\pgfpathcurveto{\pgfqpoint{0.014911in}{-0.008660in}}{\pgfqpoint{0.016667in}{-0.004420in}}{\pgfqpoint{0.016667in}{0.000000in}}%
\pgfpathcurveto{\pgfqpoint{0.016667in}{0.004420in}}{\pgfqpoint{0.014911in}{0.008660in}}{\pgfqpoint{0.011785in}{0.011785in}}%
\pgfpathcurveto{\pgfqpoint{0.008660in}{0.014911in}}{\pgfqpoint{0.004420in}{0.016667in}}{\pgfqpoint{0.000000in}{0.016667in}}%
\pgfpathcurveto{\pgfqpoint{-0.004420in}{0.016667in}}{\pgfqpoint{-0.008660in}{0.014911in}}{\pgfqpoint{-0.011785in}{0.011785in}}%
\pgfpathcurveto{\pgfqpoint{-0.014911in}{0.008660in}}{\pgfqpoint{-0.016667in}{0.004420in}}{\pgfqpoint{-0.016667in}{0.000000in}}%
\pgfpathcurveto{\pgfqpoint{-0.016667in}{-0.004420in}}{\pgfqpoint{-0.014911in}{-0.008660in}}{\pgfqpoint{-0.011785in}{-0.011785in}}%
\pgfpathcurveto{\pgfqpoint{-0.008660in}{-0.014911in}}{\pgfqpoint{-0.004420in}{-0.016667in}}{\pgfqpoint{0.000000in}{-0.016667in}}%
\pgfpathclose%
\pgfusepath{stroke,fill}%
}%
\begin{pgfscope}%
\pgfsys@transformshift{3.752726in}{4.538555in}%
\pgfsys@useobject{currentmarker}{}%
\end{pgfscope}%
\end{pgfscope}%
\begin{pgfscope}%
\pgfpathrectangle{\pgfqpoint{0.100000in}{2.413063in}}{\pgfqpoint{5.037500in}{3.427208in}}%
\pgfusepath{clip}%
\pgfsetrectcap%
\pgfsetroundjoin%
\pgfsetlinewidth{1.505625pt}%
\definecolor{currentstroke}{rgb}{0.678431,1.000000,0.184314}%
\pgfsetstrokecolor{currentstroke}%
\pgfsetstrokeopacity{0.500000}%
\pgfsetdash{}{0pt}%
\pgfpathmoveto{\pgfqpoint{3.890081in}{4.558161in}}%
\pgfusepath{stroke}%
\end{pgfscope}%
\begin{pgfscope}%
\pgfpathrectangle{\pgfqpoint{0.100000in}{2.413063in}}{\pgfqpoint{5.037500in}{3.427208in}}%
\pgfusepath{clip}%
\pgfsetbuttcap%
\pgfsetroundjoin%
\definecolor{currentfill}{rgb}{0.678431,1.000000,0.184314}%
\pgfsetfillcolor{currentfill}%
\pgfsetfillopacity{0.500000}%
\pgfsetlinewidth{0.250937pt}%
\definecolor{currentstroke}{rgb}{0.000000,0.000000,0.000000}%
\pgfsetstrokecolor{currentstroke}%
\pgfsetstrokeopacity{0.500000}%
\pgfsetdash{}{0pt}%
\pgfsys@defobject{currentmarker}{\pgfqpoint{-0.019444in}{-0.019444in}}{\pgfqpoint{0.019444in}{0.019444in}}{%
\pgfpathmoveto{\pgfqpoint{0.000000in}{-0.019444in}}%
\pgfpathcurveto{\pgfqpoint{0.005157in}{-0.019444in}}{\pgfqpoint{0.010103in}{-0.017396in}}{\pgfqpoint{0.013749in}{-0.013749in}}%
\pgfpathcurveto{\pgfqpoint{0.017396in}{-0.010103in}}{\pgfqpoint{0.019444in}{-0.005157in}}{\pgfqpoint{0.019444in}{0.000000in}}%
\pgfpathcurveto{\pgfqpoint{0.019444in}{0.005157in}}{\pgfqpoint{0.017396in}{0.010103in}}{\pgfqpoint{0.013749in}{0.013749in}}%
\pgfpathcurveto{\pgfqpoint{0.010103in}{0.017396in}}{\pgfqpoint{0.005157in}{0.019444in}}{\pgfqpoint{0.000000in}{0.019444in}}%
\pgfpathcurveto{\pgfqpoint{-0.005157in}{0.019444in}}{\pgfqpoint{-0.010103in}{0.017396in}}{\pgfqpoint{-0.013749in}{0.013749in}}%
\pgfpathcurveto{\pgfqpoint{-0.017396in}{0.010103in}}{\pgfqpoint{-0.019444in}{0.005157in}}{\pgfqpoint{-0.019444in}{0.000000in}}%
\pgfpathcurveto{\pgfqpoint{-0.019444in}{-0.005157in}}{\pgfqpoint{-0.017396in}{-0.010103in}}{\pgfqpoint{-0.013749in}{-0.013749in}}%
\pgfpathcurveto{\pgfqpoint{-0.010103in}{-0.017396in}}{\pgfqpoint{-0.005157in}{-0.019444in}}{\pgfqpoint{0.000000in}{-0.019444in}}%
\pgfpathclose%
\pgfusepath{stroke,fill}%
}%
\begin{pgfscope}%
\pgfsys@transformshift{3.890081in}{4.558161in}%
\pgfsys@useobject{currentmarker}{}%
\end{pgfscope}%
\end{pgfscope}%
\begin{pgfscope}%
\pgfpathrectangle{\pgfqpoint{0.100000in}{2.413063in}}{\pgfqpoint{5.037500in}{3.427208in}}%
\pgfusepath{clip}%
\pgfsetrectcap%
\pgfsetroundjoin%
\pgfsetlinewidth{1.505625pt}%
\definecolor{currentstroke}{rgb}{0.678431,1.000000,0.184314}%
\pgfsetstrokecolor{currentstroke}%
\pgfsetstrokeopacity{0.500000}%
\pgfsetdash{}{0pt}%
\pgfpathmoveto{\pgfqpoint{3.789983in}{4.448154in}}%
\pgfusepath{stroke}%
\end{pgfscope}%
\begin{pgfscope}%
\pgfpathrectangle{\pgfqpoint{0.100000in}{2.413063in}}{\pgfqpoint{5.037500in}{3.427208in}}%
\pgfusepath{clip}%
\pgfsetbuttcap%
\pgfsetroundjoin%
\definecolor{currentfill}{rgb}{0.678431,1.000000,0.184314}%
\pgfsetfillcolor{currentfill}%
\pgfsetfillopacity{0.500000}%
\pgfsetlinewidth{0.250937pt}%
\definecolor{currentstroke}{rgb}{0.000000,0.000000,0.000000}%
\pgfsetstrokecolor{currentstroke}%
\pgfsetstrokeopacity{0.500000}%
\pgfsetdash{}{0pt}%
\pgfsys@defobject{currentmarker}{\pgfqpoint{-0.025000in}{-0.025000in}}{\pgfqpoint{0.025000in}{0.025000in}}{%
\pgfpathmoveto{\pgfqpoint{0.000000in}{-0.025000in}}%
\pgfpathcurveto{\pgfqpoint{0.006630in}{-0.025000in}}{\pgfqpoint{0.012989in}{-0.022366in}}{\pgfqpoint{0.017678in}{-0.017678in}}%
\pgfpathcurveto{\pgfqpoint{0.022366in}{-0.012989in}}{\pgfqpoint{0.025000in}{-0.006630in}}{\pgfqpoint{0.025000in}{0.000000in}}%
\pgfpathcurveto{\pgfqpoint{0.025000in}{0.006630in}}{\pgfqpoint{0.022366in}{0.012989in}}{\pgfqpoint{0.017678in}{0.017678in}}%
\pgfpathcurveto{\pgfqpoint{0.012989in}{0.022366in}}{\pgfqpoint{0.006630in}{0.025000in}}{\pgfqpoint{0.000000in}{0.025000in}}%
\pgfpathcurveto{\pgfqpoint{-0.006630in}{0.025000in}}{\pgfqpoint{-0.012989in}{0.022366in}}{\pgfqpoint{-0.017678in}{0.017678in}}%
\pgfpathcurveto{\pgfqpoint{-0.022366in}{0.012989in}}{\pgfqpoint{-0.025000in}{0.006630in}}{\pgfqpoint{-0.025000in}{0.000000in}}%
\pgfpathcurveto{\pgfqpoint{-0.025000in}{-0.006630in}}{\pgfqpoint{-0.022366in}{-0.012989in}}{\pgfqpoint{-0.017678in}{-0.017678in}}%
\pgfpathcurveto{\pgfqpoint{-0.012989in}{-0.022366in}}{\pgfqpoint{-0.006630in}{-0.025000in}}{\pgfqpoint{0.000000in}{-0.025000in}}%
\pgfpathclose%
\pgfusepath{stroke,fill}%
}%
\begin{pgfscope}%
\pgfsys@transformshift{3.789983in}{4.448154in}%
\pgfsys@useobject{currentmarker}{}%
\end{pgfscope}%
\end{pgfscope}%
\begin{pgfscope}%
\pgfpathrectangle{\pgfqpoint{0.100000in}{2.413063in}}{\pgfqpoint{5.037500in}{3.427208in}}%
\pgfusepath{clip}%
\pgfsetrectcap%
\pgfsetroundjoin%
\pgfsetlinewidth{1.505625pt}%
\definecolor{currentstroke}{rgb}{0.678431,1.000000,0.184314}%
\pgfsetstrokecolor{currentstroke}%
\pgfsetstrokeopacity{0.500000}%
\pgfsetdash{}{0pt}%
\pgfpathmoveto{\pgfqpoint{3.790448in}{4.651915in}}%
\pgfusepath{stroke}%
\end{pgfscope}%
\begin{pgfscope}%
\pgfpathrectangle{\pgfqpoint{0.100000in}{2.413063in}}{\pgfqpoint{5.037500in}{3.427208in}}%
\pgfusepath{clip}%
\pgfsetbuttcap%
\pgfsetroundjoin%
\definecolor{currentfill}{rgb}{0.678431,1.000000,0.184314}%
\pgfsetfillcolor{currentfill}%
\pgfsetfillopacity{0.500000}%
\pgfsetlinewidth{0.250937pt}%
\definecolor{currentstroke}{rgb}{0.000000,0.000000,0.000000}%
\pgfsetstrokecolor{currentstroke}%
\pgfsetstrokeopacity{0.500000}%
\pgfsetdash{}{0pt}%
\pgfsys@defobject{currentmarker}{\pgfqpoint{-0.013889in}{-0.013889in}}{\pgfqpoint{0.013889in}{0.013889in}}{%
\pgfpathmoveto{\pgfqpoint{0.000000in}{-0.013889in}}%
\pgfpathcurveto{\pgfqpoint{0.003683in}{-0.013889in}}{\pgfqpoint{0.007216in}{-0.012425in}}{\pgfqpoint{0.009821in}{-0.009821in}}%
\pgfpathcurveto{\pgfqpoint{0.012425in}{-0.007216in}}{\pgfqpoint{0.013889in}{-0.003683in}}{\pgfqpoint{0.013889in}{0.000000in}}%
\pgfpathcurveto{\pgfqpoint{0.013889in}{0.003683in}}{\pgfqpoint{0.012425in}{0.007216in}}{\pgfqpoint{0.009821in}{0.009821in}}%
\pgfpathcurveto{\pgfqpoint{0.007216in}{0.012425in}}{\pgfqpoint{0.003683in}{0.013889in}}{\pgfqpoint{0.000000in}{0.013889in}}%
\pgfpathcurveto{\pgfqpoint{-0.003683in}{0.013889in}}{\pgfqpoint{-0.007216in}{0.012425in}}{\pgfqpoint{-0.009821in}{0.009821in}}%
\pgfpathcurveto{\pgfqpoint{-0.012425in}{0.007216in}}{\pgfqpoint{-0.013889in}{0.003683in}}{\pgfqpoint{-0.013889in}{0.000000in}}%
\pgfpathcurveto{\pgfqpoint{-0.013889in}{-0.003683in}}{\pgfqpoint{-0.012425in}{-0.007216in}}{\pgfqpoint{-0.009821in}{-0.009821in}}%
\pgfpathcurveto{\pgfqpoint{-0.007216in}{-0.012425in}}{\pgfqpoint{-0.003683in}{-0.013889in}}{\pgfqpoint{0.000000in}{-0.013889in}}%
\pgfpathclose%
\pgfusepath{stroke,fill}%
}%
\begin{pgfscope}%
\pgfsys@transformshift{3.790448in}{4.651915in}%
\pgfsys@useobject{currentmarker}{}%
\end{pgfscope}%
\end{pgfscope}%
\begin{pgfscope}%
\pgfpathrectangle{\pgfqpoint{0.100000in}{2.413063in}}{\pgfqpoint{5.037500in}{3.427208in}}%
\pgfusepath{clip}%
\pgfsetrectcap%
\pgfsetroundjoin%
\pgfsetlinewidth{1.505625pt}%
\definecolor{currentstroke}{rgb}{0.678431,1.000000,0.184314}%
\pgfsetstrokecolor{currentstroke}%
\pgfsetstrokeopacity{0.500000}%
\pgfsetdash{}{0pt}%
\pgfpathmoveto{\pgfqpoint{4.062468in}{4.544330in}}%
\pgfusepath{stroke}%
\end{pgfscope}%
\begin{pgfscope}%
\pgfpathrectangle{\pgfqpoint{0.100000in}{2.413063in}}{\pgfqpoint{5.037500in}{3.427208in}}%
\pgfusepath{clip}%
\pgfsetbuttcap%
\pgfsetroundjoin%
\definecolor{currentfill}{rgb}{0.678431,1.000000,0.184314}%
\pgfsetfillcolor{currentfill}%
\pgfsetfillopacity{0.500000}%
\pgfsetlinewidth{0.250937pt}%
\definecolor{currentstroke}{rgb}{0.000000,0.000000,0.000000}%
\pgfsetstrokecolor{currentstroke}%
\pgfsetstrokeopacity{0.500000}%
\pgfsetdash{}{0pt}%
\pgfsys@defobject{currentmarker}{\pgfqpoint{-0.033333in}{-0.033333in}}{\pgfqpoint{0.033333in}{0.033333in}}{%
\pgfpathmoveto{\pgfqpoint{0.000000in}{-0.033333in}}%
\pgfpathcurveto{\pgfqpoint{0.008840in}{-0.033333in}}{\pgfqpoint{0.017319in}{-0.029821in}}{\pgfqpoint{0.023570in}{-0.023570in}}%
\pgfpathcurveto{\pgfqpoint{0.029821in}{-0.017319in}}{\pgfqpoint{0.033333in}{-0.008840in}}{\pgfqpoint{0.033333in}{0.000000in}}%
\pgfpathcurveto{\pgfqpoint{0.033333in}{0.008840in}}{\pgfqpoint{0.029821in}{0.017319in}}{\pgfqpoint{0.023570in}{0.023570in}}%
\pgfpathcurveto{\pgfqpoint{0.017319in}{0.029821in}}{\pgfqpoint{0.008840in}{0.033333in}}{\pgfqpoint{0.000000in}{0.033333in}}%
\pgfpathcurveto{\pgfqpoint{-0.008840in}{0.033333in}}{\pgfqpoint{-0.017319in}{0.029821in}}{\pgfqpoint{-0.023570in}{0.023570in}}%
\pgfpathcurveto{\pgfqpoint{-0.029821in}{0.017319in}}{\pgfqpoint{-0.033333in}{0.008840in}}{\pgfqpoint{-0.033333in}{0.000000in}}%
\pgfpathcurveto{\pgfqpoint{-0.033333in}{-0.008840in}}{\pgfqpoint{-0.029821in}{-0.017319in}}{\pgfqpoint{-0.023570in}{-0.023570in}}%
\pgfpathcurveto{\pgfqpoint{-0.017319in}{-0.029821in}}{\pgfqpoint{-0.008840in}{-0.033333in}}{\pgfqpoint{0.000000in}{-0.033333in}}%
\pgfpathclose%
\pgfusepath{stroke,fill}%
}%
\begin{pgfscope}%
\pgfsys@transformshift{4.062468in}{4.544330in}%
\pgfsys@useobject{currentmarker}{}%
\end{pgfscope}%
\end{pgfscope}%
\begin{pgfscope}%
\pgfpathrectangle{\pgfqpoint{0.100000in}{2.413063in}}{\pgfqpoint{5.037500in}{3.427208in}}%
\pgfusepath{clip}%
\pgfsetrectcap%
\pgfsetroundjoin%
\pgfsetlinewidth{1.505625pt}%
\definecolor{currentstroke}{rgb}{0.678431,1.000000,0.184314}%
\pgfsetstrokecolor{currentstroke}%
\pgfsetstrokeopacity{0.500000}%
\pgfsetdash{}{0pt}%
\pgfpathmoveto{\pgfqpoint{4.044939in}{4.620954in}}%
\pgfusepath{stroke}%
\end{pgfscope}%
\begin{pgfscope}%
\pgfpathrectangle{\pgfqpoint{0.100000in}{2.413063in}}{\pgfqpoint{5.037500in}{3.427208in}}%
\pgfusepath{clip}%
\pgfsetbuttcap%
\pgfsetroundjoin%
\definecolor{currentfill}{rgb}{0.678431,1.000000,0.184314}%
\pgfsetfillcolor{currentfill}%
\pgfsetfillopacity{0.500000}%
\pgfsetlinewidth{0.250937pt}%
\definecolor{currentstroke}{rgb}{0.000000,0.000000,0.000000}%
\pgfsetstrokecolor{currentstroke}%
\pgfsetstrokeopacity{0.500000}%
\pgfsetdash{}{0pt}%
\pgfsys@defobject{currentmarker}{\pgfqpoint{-0.033333in}{-0.033333in}}{\pgfqpoint{0.033333in}{0.033333in}}{%
\pgfpathmoveto{\pgfqpoint{0.000000in}{-0.033333in}}%
\pgfpathcurveto{\pgfqpoint{0.008840in}{-0.033333in}}{\pgfqpoint{0.017319in}{-0.029821in}}{\pgfqpoint{0.023570in}{-0.023570in}}%
\pgfpathcurveto{\pgfqpoint{0.029821in}{-0.017319in}}{\pgfqpoint{0.033333in}{-0.008840in}}{\pgfqpoint{0.033333in}{0.000000in}}%
\pgfpathcurveto{\pgfqpoint{0.033333in}{0.008840in}}{\pgfqpoint{0.029821in}{0.017319in}}{\pgfqpoint{0.023570in}{0.023570in}}%
\pgfpathcurveto{\pgfqpoint{0.017319in}{0.029821in}}{\pgfqpoint{0.008840in}{0.033333in}}{\pgfqpoint{0.000000in}{0.033333in}}%
\pgfpathcurveto{\pgfqpoint{-0.008840in}{0.033333in}}{\pgfqpoint{-0.017319in}{0.029821in}}{\pgfqpoint{-0.023570in}{0.023570in}}%
\pgfpathcurveto{\pgfqpoint{-0.029821in}{0.017319in}}{\pgfqpoint{-0.033333in}{0.008840in}}{\pgfqpoint{-0.033333in}{0.000000in}}%
\pgfpathcurveto{\pgfqpoint{-0.033333in}{-0.008840in}}{\pgfqpoint{-0.029821in}{-0.017319in}}{\pgfqpoint{-0.023570in}{-0.023570in}}%
\pgfpathcurveto{\pgfqpoint{-0.017319in}{-0.029821in}}{\pgfqpoint{-0.008840in}{-0.033333in}}{\pgfqpoint{0.000000in}{-0.033333in}}%
\pgfpathclose%
\pgfusepath{stroke,fill}%
}%
\begin{pgfscope}%
\pgfsys@transformshift{4.044939in}{4.620954in}%
\pgfsys@useobject{currentmarker}{}%
\end{pgfscope}%
\end{pgfscope}%
\begin{pgfscope}%
\pgfpathrectangle{\pgfqpoint{0.100000in}{2.413063in}}{\pgfqpoint{5.037500in}{3.427208in}}%
\pgfusepath{clip}%
\pgfsetrectcap%
\pgfsetroundjoin%
\pgfsetlinewidth{1.505625pt}%
\definecolor{currentstroke}{rgb}{0.678431,1.000000,0.184314}%
\pgfsetstrokecolor{currentstroke}%
\pgfsetstrokeopacity{0.500000}%
\pgfsetdash{}{0pt}%
\pgfpathmoveto{\pgfqpoint{2.481932in}{3.780796in}}%
\pgfusepath{stroke}%
\end{pgfscope}%
\begin{pgfscope}%
\pgfpathrectangle{\pgfqpoint{0.100000in}{2.413063in}}{\pgfqpoint{5.037500in}{3.427208in}}%
\pgfusepath{clip}%
\pgfsetbuttcap%
\pgfsetroundjoin%
\definecolor{currentfill}{rgb}{0.678431,1.000000,0.184314}%
\pgfsetfillcolor{currentfill}%
\pgfsetfillopacity{0.500000}%
\pgfsetlinewidth{0.250937pt}%
\definecolor{currentstroke}{rgb}{0.000000,0.000000,0.000000}%
\pgfsetstrokecolor{currentstroke}%
\pgfsetstrokeopacity{0.500000}%
\pgfsetdash{}{0pt}%
\pgfsys@defobject{currentmarker}{\pgfqpoint{-0.027778in}{-0.027778in}}{\pgfqpoint{0.027778in}{0.027778in}}{%
\pgfpathmoveto{\pgfqpoint{0.000000in}{-0.027778in}}%
\pgfpathcurveto{\pgfqpoint{0.007367in}{-0.027778in}}{\pgfqpoint{0.014433in}{-0.024851in}}{\pgfqpoint{0.019642in}{-0.019642in}}%
\pgfpathcurveto{\pgfqpoint{0.024851in}{-0.014433in}}{\pgfqpoint{0.027778in}{-0.007367in}}{\pgfqpoint{0.027778in}{0.000000in}}%
\pgfpathcurveto{\pgfqpoint{0.027778in}{0.007367in}}{\pgfqpoint{0.024851in}{0.014433in}}{\pgfqpoint{0.019642in}{0.019642in}}%
\pgfpathcurveto{\pgfqpoint{0.014433in}{0.024851in}}{\pgfqpoint{0.007367in}{0.027778in}}{\pgfqpoint{0.000000in}{0.027778in}}%
\pgfpathcurveto{\pgfqpoint{-0.007367in}{0.027778in}}{\pgfqpoint{-0.014433in}{0.024851in}}{\pgfqpoint{-0.019642in}{0.019642in}}%
\pgfpathcurveto{\pgfqpoint{-0.024851in}{0.014433in}}{\pgfqpoint{-0.027778in}{0.007367in}}{\pgfqpoint{-0.027778in}{0.000000in}}%
\pgfpathcurveto{\pgfqpoint{-0.027778in}{-0.007367in}}{\pgfqpoint{-0.024851in}{-0.014433in}}{\pgfqpoint{-0.019642in}{-0.019642in}}%
\pgfpathcurveto{\pgfqpoint{-0.014433in}{-0.024851in}}{\pgfqpoint{-0.007367in}{-0.027778in}}{\pgfqpoint{0.000000in}{-0.027778in}}%
\pgfpathclose%
\pgfusepath{stroke,fill}%
}%
\begin{pgfscope}%
\pgfsys@transformshift{2.481932in}{3.780796in}%
\pgfsys@useobject{currentmarker}{}%
\end{pgfscope}%
\end{pgfscope}%
\begin{pgfscope}%
\pgfpathrectangle{\pgfqpoint{0.100000in}{2.413063in}}{\pgfqpoint{5.037500in}{3.427208in}}%
\pgfusepath{clip}%
\pgfsetrectcap%
\pgfsetroundjoin%
\pgfsetlinewidth{1.505625pt}%
\definecolor{currentstroke}{rgb}{0.678431,1.000000,0.184314}%
\pgfsetstrokecolor{currentstroke}%
\pgfsetstrokeopacity{0.500000}%
\pgfsetdash{}{0pt}%
\pgfpathmoveto{\pgfqpoint{2.567661in}{3.877827in}}%
\pgfusepath{stroke}%
\end{pgfscope}%
\begin{pgfscope}%
\pgfpathrectangle{\pgfqpoint{0.100000in}{2.413063in}}{\pgfqpoint{5.037500in}{3.427208in}}%
\pgfusepath{clip}%
\pgfsetbuttcap%
\pgfsetroundjoin%
\definecolor{currentfill}{rgb}{0.678431,1.000000,0.184314}%
\pgfsetfillcolor{currentfill}%
\pgfsetfillopacity{0.500000}%
\pgfsetlinewidth{0.250937pt}%
\definecolor{currentstroke}{rgb}{0.000000,0.000000,0.000000}%
\pgfsetstrokecolor{currentstroke}%
\pgfsetstrokeopacity{0.500000}%
\pgfsetdash{}{0pt}%
\pgfsys@defobject{currentmarker}{\pgfqpoint{-0.025000in}{-0.025000in}}{\pgfqpoint{0.025000in}{0.025000in}}{%
\pgfpathmoveto{\pgfqpoint{0.000000in}{-0.025000in}}%
\pgfpathcurveto{\pgfqpoint{0.006630in}{-0.025000in}}{\pgfqpoint{0.012989in}{-0.022366in}}{\pgfqpoint{0.017678in}{-0.017678in}}%
\pgfpathcurveto{\pgfqpoint{0.022366in}{-0.012989in}}{\pgfqpoint{0.025000in}{-0.006630in}}{\pgfqpoint{0.025000in}{0.000000in}}%
\pgfpathcurveto{\pgfqpoint{0.025000in}{0.006630in}}{\pgfqpoint{0.022366in}{0.012989in}}{\pgfqpoint{0.017678in}{0.017678in}}%
\pgfpathcurveto{\pgfqpoint{0.012989in}{0.022366in}}{\pgfqpoint{0.006630in}{0.025000in}}{\pgfqpoint{0.000000in}{0.025000in}}%
\pgfpathcurveto{\pgfqpoint{-0.006630in}{0.025000in}}{\pgfqpoint{-0.012989in}{0.022366in}}{\pgfqpoint{-0.017678in}{0.017678in}}%
\pgfpathcurveto{\pgfqpoint{-0.022366in}{0.012989in}}{\pgfqpoint{-0.025000in}{0.006630in}}{\pgfqpoint{-0.025000in}{0.000000in}}%
\pgfpathcurveto{\pgfqpoint{-0.025000in}{-0.006630in}}{\pgfqpoint{-0.022366in}{-0.012989in}}{\pgfqpoint{-0.017678in}{-0.017678in}}%
\pgfpathcurveto{\pgfqpoint{-0.012989in}{-0.022366in}}{\pgfqpoint{-0.006630in}{-0.025000in}}{\pgfqpoint{0.000000in}{-0.025000in}}%
\pgfpathclose%
\pgfusepath{stroke,fill}%
}%
\begin{pgfscope}%
\pgfsys@transformshift{2.567661in}{3.877827in}%
\pgfsys@useobject{currentmarker}{}%
\end{pgfscope}%
\end{pgfscope}%
\begin{pgfscope}%
\pgfpathrectangle{\pgfqpoint{0.100000in}{2.413063in}}{\pgfqpoint{5.037500in}{3.427208in}}%
\pgfusepath{clip}%
\pgfsetrectcap%
\pgfsetroundjoin%
\pgfsetlinewidth{1.505625pt}%
\definecolor{currentstroke}{rgb}{0.678431,1.000000,0.184314}%
\pgfsetstrokecolor{currentstroke}%
\pgfsetstrokeopacity{0.500000}%
\pgfsetdash{}{0pt}%
\pgfpathmoveto{\pgfqpoint{2.711698in}{3.953793in}}%
\pgfusepath{stroke}%
\end{pgfscope}%
\begin{pgfscope}%
\pgfpathrectangle{\pgfqpoint{0.100000in}{2.413063in}}{\pgfqpoint{5.037500in}{3.427208in}}%
\pgfusepath{clip}%
\pgfsetbuttcap%
\pgfsetroundjoin%
\definecolor{currentfill}{rgb}{0.678431,1.000000,0.184314}%
\pgfsetfillcolor{currentfill}%
\pgfsetfillopacity{0.500000}%
\pgfsetlinewidth{0.250937pt}%
\definecolor{currentstroke}{rgb}{0.000000,0.000000,0.000000}%
\pgfsetstrokecolor{currentstroke}%
\pgfsetstrokeopacity{0.500000}%
\pgfsetdash{}{0pt}%
\pgfsys@defobject{currentmarker}{\pgfqpoint{-0.025000in}{-0.025000in}}{\pgfqpoint{0.025000in}{0.025000in}}{%
\pgfpathmoveto{\pgfqpoint{0.000000in}{-0.025000in}}%
\pgfpathcurveto{\pgfqpoint{0.006630in}{-0.025000in}}{\pgfqpoint{0.012989in}{-0.022366in}}{\pgfqpoint{0.017678in}{-0.017678in}}%
\pgfpathcurveto{\pgfqpoint{0.022366in}{-0.012989in}}{\pgfqpoint{0.025000in}{-0.006630in}}{\pgfqpoint{0.025000in}{0.000000in}}%
\pgfpathcurveto{\pgfqpoint{0.025000in}{0.006630in}}{\pgfqpoint{0.022366in}{0.012989in}}{\pgfqpoint{0.017678in}{0.017678in}}%
\pgfpathcurveto{\pgfqpoint{0.012989in}{0.022366in}}{\pgfqpoint{0.006630in}{0.025000in}}{\pgfqpoint{0.000000in}{0.025000in}}%
\pgfpathcurveto{\pgfqpoint{-0.006630in}{0.025000in}}{\pgfqpoint{-0.012989in}{0.022366in}}{\pgfqpoint{-0.017678in}{0.017678in}}%
\pgfpathcurveto{\pgfqpoint{-0.022366in}{0.012989in}}{\pgfqpoint{-0.025000in}{0.006630in}}{\pgfqpoint{-0.025000in}{0.000000in}}%
\pgfpathcurveto{\pgfqpoint{-0.025000in}{-0.006630in}}{\pgfqpoint{-0.022366in}{-0.012989in}}{\pgfqpoint{-0.017678in}{-0.017678in}}%
\pgfpathcurveto{\pgfqpoint{-0.012989in}{-0.022366in}}{\pgfqpoint{-0.006630in}{-0.025000in}}{\pgfqpoint{0.000000in}{-0.025000in}}%
\pgfpathclose%
\pgfusepath{stroke,fill}%
}%
\begin{pgfscope}%
\pgfsys@transformshift{2.711698in}{3.953793in}%
\pgfsys@useobject{currentmarker}{}%
\end{pgfscope}%
\end{pgfscope}%
\begin{pgfscope}%
\pgfpathrectangle{\pgfqpoint{0.100000in}{2.413063in}}{\pgfqpoint{5.037500in}{3.427208in}}%
\pgfusepath{clip}%
\pgfsetrectcap%
\pgfsetroundjoin%
\pgfsetlinewidth{1.505625pt}%
\definecolor{currentstroke}{rgb}{0.000000,0.000000,1.000000}%
\pgfsetstrokecolor{currentstroke}%
\pgfsetstrokeopacity{0.500000}%
\pgfsetdash{}{0pt}%
\pgfpathmoveto{\pgfqpoint{0.526793in}{5.285996in}}%
\pgfusepath{stroke}%
\end{pgfscope}%
\begin{pgfscope}%
\pgfpathrectangle{\pgfqpoint{0.100000in}{2.413063in}}{\pgfqpoint{5.037500in}{3.427208in}}%
\pgfusepath{clip}%
\pgfsetbuttcap%
\pgfsetroundjoin%
\definecolor{currentfill}{rgb}{0.000000,0.000000,1.000000}%
\pgfsetfillcolor{currentfill}%
\pgfsetfillopacity{0.500000}%
\pgfsetlinewidth{0.250937pt}%
\definecolor{currentstroke}{rgb}{0.000000,0.000000,0.000000}%
\pgfsetstrokecolor{currentstroke}%
\pgfsetstrokeopacity{0.500000}%
\pgfsetdash{}{0pt}%
\pgfsys@defobject{currentmarker}{\pgfqpoint{-0.016667in}{-0.016667in}}{\pgfqpoint{0.016667in}{0.016667in}}{%
\pgfpathmoveto{\pgfqpoint{0.000000in}{-0.016667in}}%
\pgfpathcurveto{\pgfqpoint{0.004420in}{-0.016667in}}{\pgfqpoint{0.008660in}{-0.014911in}}{\pgfqpoint{0.011785in}{-0.011785in}}%
\pgfpathcurveto{\pgfqpoint{0.014911in}{-0.008660in}}{\pgfqpoint{0.016667in}{-0.004420in}}{\pgfqpoint{0.016667in}{0.000000in}}%
\pgfpathcurveto{\pgfqpoint{0.016667in}{0.004420in}}{\pgfqpoint{0.014911in}{0.008660in}}{\pgfqpoint{0.011785in}{0.011785in}}%
\pgfpathcurveto{\pgfqpoint{0.008660in}{0.014911in}}{\pgfqpoint{0.004420in}{0.016667in}}{\pgfqpoint{0.000000in}{0.016667in}}%
\pgfpathcurveto{\pgfqpoint{-0.004420in}{0.016667in}}{\pgfqpoint{-0.008660in}{0.014911in}}{\pgfqpoint{-0.011785in}{0.011785in}}%
\pgfpathcurveto{\pgfqpoint{-0.014911in}{0.008660in}}{\pgfqpoint{-0.016667in}{0.004420in}}{\pgfqpoint{-0.016667in}{0.000000in}}%
\pgfpathcurveto{\pgfqpoint{-0.016667in}{-0.004420in}}{\pgfqpoint{-0.014911in}{-0.008660in}}{\pgfqpoint{-0.011785in}{-0.011785in}}%
\pgfpathcurveto{\pgfqpoint{-0.008660in}{-0.014911in}}{\pgfqpoint{-0.004420in}{-0.016667in}}{\pgfqpoint{0.000000in}{-0.016667in}}%
\pgfpathclose%
\pgfusepath{stroke,fill}%
}%
\begin{pgfscope}%
\pgfsys@transformshift{0.526793in}{5.285996in}%
\pgfsys@useobject{currentmarker}{}%
\end{pgfscope}%
\end{pgfscope}%
\begin{pgfscope}%
\pgfpathrectangle{\pgfqpoint{0.100000in}{2.413063in}}{\pgfqpoint{5.037500in}{3.427208in}}%
\pgfusepath{clip}%
\pgfsetrectcap%
\pgfsetroundjoin%
\pgfsetlinewidth{1.505625pt}%
\definecolor{currentstroke}{rgb}{0.000000,0.000000,1.000000}%
\pgfsetstrokecolor{currentstroke}%
\pgfsetstrokeopacity{0.500000}%
\pgfsetdash{}{0pt}%
\pgfpathmoveto{\pgfqpoint{0.648561in}{5.178411in}}%
\pgfusepath{stroke}%
\end{pgfscope}%
\begin{pgfscope}%
\pgfpathrectangle{\pgfqpoint{0.100000in}{2.413063in}}{\pgfqpoint{5.037500in}{3.427208in}}%
\pgfusepath{clip}%
\pgfsetbuttcap%
\pgfsetroundjoin%
\definecolor{currentfill}{rgb}{0.000000,0.000000,1.000000}%
\pgfsetfillcolor{currentfill}%
\pgfsetfillopacity{0.500000}%
\pgfsetlinewidth{0.250937pt}%
\definecolor{currentstroke}{rgb}{0.000000,0.000000,0.000000}%
\pgfsetstrokecolor{currentstroke}%
\pgfsetstrokeopacity{0.500000}%
\pgfsetdash{}{0pt}%
\pgfsys@defobject{currentmarker}{\pgfqpoint{-0.016667in}{-0.016667in}}{\pgfqpoint{0.016667in}{0.016667in}}{%
\pgfpathmoveto{\pgfqpoint{0.000000in}{-0.016667in}}%
\pgfpathcurveto{\pgfqpoint{0.004420in}{-0.016667in}}{\pgfqpoint{0.008660in}{-0.014911in}}{\pgfqpoint{0.011785in}{-0.011785in}}%
\pgfpathcurveto{\pgfqpoint{0.014911in}{-0.008660in}}{\pgfqpoint{0.016667in}{-0.004420in}}{\pgfqpoint{0.016667in}{0.000000in}}%
\pgfpathcurveto{\pgfqpoint{0.016667in}{0.004420in}}{\pgfqpoint{0.014911in}{0.008660in}}{\pgfqpoint{0.011785in}{0.011785in}}%
\pgfpathcurveto{\pgfqpoint{0.008660in}{0.014911in}}{\pgfqpoint{0.004420in}{0.016667in}}{\pgfqpoint{0.000000in}{0.016667in}}%
\pgfpathcurveto{\pgfqpoint{-0.004420in}{0.016667in}}{\pgfqpoint{-0.008660in}{0.014911in}}{\pgfqpoint{-0.011785in}{0.011785in}}%
\pgfpathcurveto{\pgfqpoint{-0.014911in}{0.008660in}}{\pgfqpoint{-0.016667in}{0.004420in}}{\pgfqpoint{-0.016667in}{0.000000in}}%
\pgfpathcurveto{\pgfqpoint{-0.016667in}{-0.004420in}}{\pgfqpoint{-0.014911in}{-0.008660in}}{\pgfqpoint{-0.011785in}{-0.011785in}}%
\pgfpathcurveto{\pgfqpoint{-0.008660in}{-0.014911in}}{\pgfqpoint{-0.004420in}{-0.016667in}}{\pgfqpoint{0.000000in}{-0.016667in}}%
\pgfpathclose%
\pgfusepath{stroke,fill}%
}%
\begin{pgfscope}%
\pgfsys@transformshift{0.648561in}{5.178411in}%
\pgfsys@useobject{currentmarker}{}%
\end{pgfscope}%
\end{pgfscope}%
\begin{pgfscope}%
\pgfpathrectangle{\pgfqpoint{0.100000in}{2.413063in}}{\pgfqpoint{5.037500in}{3.427208in}}%
\pgfusepath{clip}%
\pgfsetrectcap%
\pgfsetroundjoin%
\pgfsetlinewidth{1.505625pt}%
\definecolor{currentstroke}{rgb}{0.000000,0.000000,1.000000}%
\pgfsetstrokecolor{currentstroke}%
\pgfsetstrokeopacity{0.500000}%
\pgfsetdash{}{0pt}%
\pgfpathmoveto{\pgfqpoint{0.512006in}{5.281994in}}%
\pgfusepath{stroke}%
\end{pgfscope}%
\begin{pgfscope}%
\pgfpathrectangle{\pgfqpoint{0.100000in}{2.413063in}}{\pgfqpoint{5.037500in}{3.427208in}}%
\pgfusepath{clip}%
\pgfsetbuttcap%
\pgfsetroundjoin%
\definecolor{currentfill}{rgb}{0.000000,0.000000,1.000000}%
\pgfsetfillcolor{currentfill}%
\pgfsetfillopacity{0.500000}%
\pgfsetlinewidth{0.250937pt}%
\definecolor{currentstroke}{rgb}{0.000000,0.000000,0.000000}%
\pgfsetstrokecolor{currentstroke}%
\pgfsetstrokeopacity{0.500000}%
\pgfsetdash{}{0pt}%
\pgfsys@defobject{currentmarker}{\pgfqpoint{-0.013889in}{-0.013889in}}{\pgfqpoint{0.013889in}{0.013889in}}{%
\pgfpathmoveto{\pgfqpoint{0.000000in}{-0.013889in}}%
\pgfpathcurveto{\pgfqpoint{0.003683in}{-0.013889in}}{\pgfqpoint{0.007216in}{-0.012425in}}{\pgfqpoint{0.009821in}{-0.009821in}}%
\pgfpathcurveto{\pgfqpoint{0.012425in}{-0.007216in}}{\pgfqpoint{0.013889in}{-0.003683in}}{\pgfqpoint{0.013889in}{0.000000in}}%
\pgfpathcurveto{\pgfqpoint{0.013889in}{0.003683in}}{\pgfqpoint{0.012425in}{0.007216in}}{\pgfqpoint{0.009821in}{0.009821in}}%
\pgfpathcurveto{\pgfqpoint{0.007216in}{0.012425in}}{\pgfqpoint{0.003683in}{0.013889in}}{\pgfqpoint{0.000000in}{0.013889in}}%
\pgfpathcurveto{\pgfqpoint{-0.003683in}{0.013889in}}{\pgfqpoint{-0.007216in}{0.012425in}}{\pgfqpoint{-0.009821in}{0.009821in}}%
\pgfpathcurveto{\pgfqpoint{-0.012425in}{0.007216in}}{\pgfqpoint{-0.013889in}{0.003683in}}{\pgfqpoint{-0.013889in}{0.000000in}}%
\pgfpathcurveto{\pgfqpoint{-0.013889in}{-0.003683in}}{\pgfqpoint{-0.012425in}{-0.007216in}}{\pgfqpoint{-0.009821in}{-0.009821in}}%
\pgfpathcurveto{\pgfqpoint{-0.007216in}{-0.012425in}}{\pgfqpoint{-0.003683in}{-0.013889in}}{\pgfqpoint{0.000000in}{-0.013889in}}%
\pgfpathclose%
\pgfusepath{stroke,fill}%
}%
\begin{pgfscope}%
\pgfsys@transformshift{0.512006in}{5.281994in}%
\pgfsys@useobject{currentmarker}{}%
\end{pgfscope}%
\end{pgfscope}%
\begin{pgfscope}%
\pgfpathrectangle{\pgfqpoint{0.100000in}{2.413063in}}{\pgfqpoint{5.037500in}{3.427208in}}%
\pgfusepath{clip}%
\pgfsetrectcap%
\pgfsetroundjoin%
\pgfsetlinewidth{1.505625pt}%
\definecolor{currentstroke}{rgb}{0.000000,0.000000,1.000000}%
\pgfsetstrokecolor{currentstroke}%
\pgfsetstrokeopacity{0.500000}%
\pgfsetdash{}{0pt}%
\pgfpathmoveto{\pgfqpoint{0.504276in}{5.228196in}}%
\pgfusepath{stroke}%
\end{pgfscope}%
\begin{pgfscope}%
\pgfpathrectangle{\pgfqpoint{0.100000in}{2.413063in}}{\pgfqpoint{5.037500in}{3.427208in}}%
\pgfusepath{clip}%
\pgfsetbuttcap%
\pgfsetroundjoin%
\definecolor{currentfill}{rgb}{0.000000,0.000000,1.000000}%
\pgfsetfillcolor{currentfill}%
\pgfsetfillopacity{0.500000}%
\pgfsetlinewidth{0.250937pt}%
\definecolor{currentstroke}{rgb}{0.000000,0.000000,0.000000}%
\pgfsetstrokecolor{currentstroke}%
\pgfsetstrokeopacity{0.500000}%
\pgfsetdash{}{0pt}%
\pgfsys@defobject{currentmarker}{\pgfqpoint{-0.025000in}{-0.025000in}}{\pgfqpoint{0.025000in}{0.025000in}}{%
\pgfpathmoveto{\pgfqpoint{0.000000in}{-0.025000in}}%
\pgfpathcurveto{\pgfqpoint{0.006630in}{-0.025000in}}{\pgfqpoint{0.012989in}{-0.022366in}}{\pgfqpoint{0.017678in}{-0.017678in}}%
\pgfpathcurveto{\pgfqpoint{0.022366in}{-0.012989in}}{\pgfqpoint{0.025000in}{-0.006630in}}{\pgfqpoint{0.025000in}{0.000000in}}%
\pgfpathcurveto{\pgfqpoint{0.025000in}{0.006630in}}{\pgfqpoint{0.022366in}{0.012989in}}{\pgfqpoint{0.017678in}{0.017678in}}%
\pgfpathcurveto{\pgfqpoint{0.012989in}{0.022366in}}{\pgfqpoint{0.006630in}{0.025000in}}{\pgfqpoint{0.000000in}{0.025000in}}%
\pgfpathcurveto{\pgfqpoint{-0.006630in}{0.025000in}}{\pgfqpoint{-0.012989in}{0.022366in}}{\pgfqpoint{-0.017678in}{0.017678in}}%
\pgfpathcurveto{\pgfqpoint{-0.022366in}{0.012989in}}{\pgfqpoint{-0.025000in}{0.006630in}}{\pgfqpoint{-0.025000in}{0.000000in}}%
\pgfpathcurveto{\pgfqpoint{-0.025000in}{-0.006630in}}{\pgfqpoint{-0.022366in}{-0.012989in}}{\pgfqpoint{-0.017678in}{-0.017678in}}%
\pgfpathcurveto{\pgfqpoint{-0.012989in}{-0.022366in}}{\pgfqpoint{-0.006630in}{-0.025000in}}{\pgfqpoint{0.000000in}{-0.025000in}}%
\pgfpathclose%
\pgfusepath{stroke,fill}%
}%
\begin{pgfscope}%
\pgfsys@transformshift{0.504276in}{5.228196in}%
\pgfsys@useobject{currentmarker}{}%
\end{pgfscope}%
\end{pgfscope}%
\begin{pgfscope}%
\pgfpathrectangle{\pgfqpoint{0.100000in}{2.413063in}}{\pgfqpoint{5.037500in}{3.427208in}}%
\pgfusepath{clip}%
\pgfsetrectcap%
\pgfsetroundjoin%
\pgfsetlinewidth{1.505625pt}%
\definecolor{currentstroke}{rgb}{0.000000,0.000000,1.000000}%
\pgfsetstrokecolor{currentstroke}%
\pgfsetstrokeopacity{0.500000}%
\pgfsetdash{}{0pt}%
\pgfpathmoveto{\pgfqpoint{0.431607in}{5.050000in}}%
\pgfusepath{stroke}%
\end{pgfscope}%
\begin{pgfscope}%
\pgfpathrectangle{\pgfqpoint{0.100000in}{2.413063in}}{\pgfqpoint{5.037500in}{3.427208in}}%
\pgfusepath{clip}%
\pgfsetbuttcap%
\pgfsetroundjoin%
\definecolor{currentfill}{rgb}{0.000000,0.000000,1.000000}%
\pgfsetfillcolor{currentfill}%
\pgfsetfillopacity{0.500000}%
\pgfsetlinewidth{0.250937pt}%
\definecolor{currentstroke}{rgb}{0.000000,0.000000,0.000000}%
\pgfsetstrokecolor{currentstroke}%
\pgfsetstrokeopacity{0.500000}%
\pgfsetdash{}{0pt}%
\pgfsys@defobject{currentmarker}{\pgfqpoint{-0.022222in}{-0.022222in}}{\pgfqpoint{0.022222in}{0.022222in}}{%
\pgfpathmoveto{\pgfqpoint{0.000000in}{-0.022222in}}%
\pgfpathcurveto{\pgfqpoint{0.005893in}{-0.022222in}}{\pgfqpoint{0.011546in}{-0.019881in}}{\pgfqpoint{0.015713in}{-0.015713in}}%
\pgfpathcurveto{\pgfqpoint{0.019881in}{-0.011546in}}{\pgfqpoint{0.022222in}{-0.005893in}}{\pgfqpoint{0.022222in}{0.000000in}}%
\pgfpathcurveto{\pgfqpoint{0.022222in}{0.005893in}}{\pgfqpoint{0.019881in}{0.011546in}}{\pgfqpoint{0.015713in}{0.015713in}}%
\pgfpathcurveto{\pgfqpoint{0.011546in}{0.019881in}}{\pgfqpoint{0.005893in}{0.022222in}}{\pgfqpoint{0.000000in}{0.022222in}}%
\pgfpathcurveto{\pgfqpoint{-0.005893in}{0.022222in}}{\pgfqpoint{-0.011546in}{0.019881in}}{\pgfqpoint{-0.015713in}{0.015713in}}%
\pgfpathcurveto{\pgfqpoint{-0.019881in}{0.011546in}}{\pgfqpoint{-0.022222in}{0.005893in}}{\pgfqpoint{-0.022222in}{0.000000in}}%
\pgfpathcurveto{\pgfqpoint{-0.022222in}{-0.005893in}}{\pgfqpoint{-0.019881in}{-0.011546in}}{\pgfqpoint{-0.015713in}{-0.015713in}}%
\pgfpathcurveto{\pgfqpoint{-0.011546in}{-0.019881in}}{\pgfqpoint{-0.005893in}{-0.022222in}}{\pgfqpoint{0.000000in}{-0.022222in}}%
\pgfpathclose%
\pgfusepath{stroke,fill}%
}%
\begin{pgfscope}%
\pgfsys@transformshift{0.431607in}{5.050000in}%
\pgfsys@useobject{currentmarker}{}%
\end{pgfscope}%
\end{pgfscope}%
\begin{pgfscope}%
\pgfpathrectangle{\pgfqpoint{0.100000in}{2.413063in}}{\pgfqpoint{5.037500in}{3.427208in}}%
\pgfusepath{clip}%
\pgfsetrectcap%
\pgfsetroundjoin%
\pgfsetlinewidth{1.505625pt}%
\definecolor{currentstroke}{rgb}{0.000000,0.000000,1.000000}%
\pgfsetstrokecolor{currentstroke}%
\pgfsetstrokeopacity{0.500000}%
\pgfsetdash{}{0pt}%
\pgfpathmoveto{\pgfqpoint{0.464601in}{5.025783in}}%
\pgfusepath{stroke}%
\end{pgfscope}%
\begin{pgfscope}%
\pgfpathrectangle{\pgfqpoint{0.100000in}{2.413063in}}{\pgfqpoint{5.037500in}{3.427208in}}%
\pgfusepath{clip}%
\pgfsetbuttcap%
\pgfsetroundjoin%
\definecolor{currentfill}{rgb}{0.000000,0.000000,1.000000}%
\pgfsetfillcolor{currentfill}%
\pgfsetfillopacity{0.500000}%
\pgfsetlinewidth{0.250937pt}%
\definecolor{currentstroke}{rgb}{0.000000,0.000000,0.000000}%
\pgfsetstrokecolor{currentstroke}%
\pgfsetstrokeopacity{0.500000}%
\pgfsetdash{}{0pt}%
\pgfsys@defobject{currentmarker}{\pgfqpoint{-0.016667in}{-0.016667in}}{\pgfqpoint{0.016667in}{0.016667in}}{%
\pgfpathmoveto{\pgfqpoint{0.000000in}{-0.016667in}}%
\pgfpathcurveto{\pgfqpoint{0.004420in}{-0.016667in}}{\pgfqpoint{0.008660in}{-0.014911in}}{\pgfqpoint{0.011785in}{-0.011785in}}%
\pgfpathcurveto{\pgfqpoint{0.014911in}{-0.008660in}}{\pgfqpoint{0.016667in}{-0.004420in}}{\pgfqpoint{0.016667in}{0.000000in}}%
\pgfpathcurveto{\pgfqpoint{0.016667in}{0.004420in}}{\pgfqpoint{0.014911in}{0.008660in}}{\pgfqpoint{0.011785in}{0.011785in}}%
\pgfpathcurveto{\pgfqpoint{0.008660in}{0.014911in}}{\pgfqpoint{0.004420in}{0.016667in}}{\pgfqpoint{0.000000in}{0.016667in}}%
\pgfpathcurveto{\pgfqpoint{-0.004420in}{0.016667in}}{\pgfqpoint{-0.008660in}{0.014911in}}{\pgfqpoint{-0.011785in}{0.011785in}}%
\pgfpathcurveto{\pgfqpoint{-0.014911in}{0.008660in}}{\pgfqpoint{-0.016667in}{0.004420in}}{\pgfqpoint{-0.016667in}{0.000000in}}%
\pgfpathcurveto{\pgfqpoint{-0.016667in}{-0.004420in}}{\pgfqpoint{-0.014911in}{-0.008660in}}{\pgfqpoint{-0.011785in}{-0.011785in}}%
\pgfpathcurveto{\pgfqpoint{-0.008660in}{-0.014911in}}{\pgfqpoint{-0.004420in}{-0.016667in}}{\pgfqpoint{0.000000in}{-0.016667in}}%
\pgfpathclose%
\pgfusepath{stroke,fill}%
}%
\begin{pgfscope}%
\pgfsys@transformshift{0.464601in}{5.025783in}%
\pgfsys@useobject{currentmarker}{}%
\end{pgfscope}%
\end{pgfscope}%
\begin{pgfscope}%
\pgfpathrectangle{\pgfqpoint{0.100000in}{2.413063in}}{\pgfqpoint{5.037500in}{3.427208in}}%
\pgfusepath{clip}%
\pgfsetrectcap%
\pgfsetroundjoin%
\pgfsetlinewidth{1.505625pt}%
\definecolor{currentstroke}{rgb}{0.000000,0.000000,1.000000}%
\pgfsetstrokecolor{currentstroke}%
\pgfsetstrokeopacity{0.500000}%
\pgfsetdash{}{0pt}%
\pgfpathmoveto{\pgfqpoint{0.591363in}{5.372886in}}%
\pgfusepath{stroke}%
\end{pgfscope}%
\begin{pgfscope}%
\pgfpathrectangle{\pgfqpoint{0.100000in}{2.413063in}}{\pgfqpoint{5.037500in}{3.427208in}}%
\pgfusepath{clip}%
\pgfsetbuttcap%
\pgfsetroundjoin%
\definecolor{currentfill}{rgb}{0.000000,0.000000,1.000000}%
\pgfsetfillcolor{currentfill}%
\pgfsetfillopacity{0.500000}%
\pgfsetlinewidth{0.250937pt}%
\definecolor{currentstroke}{rgb}{0.000000,0.000000,0.000000}%
\pgfsetstrokecolor{currentstroke}%
\pgfsetstrokeopacity{0.500000}%
\pgfsetdash{}{0pt}%
\pgfsys@defobject{currentmarker}{\pgfqpoint{-0.016667in}{-0.016667in}}{\pgfqpoint{0.016667in}{0.016667in}}{%
\pgfpathmoveto{\pgfqpoint{0.000000in}{-0.016667in}}%
\pgfpathcurveto{\pgfqpoint{0.004420in}{-0.016667in}}{\pgfqpoint{0.008660in}{-0.014911in}}{\pgfqpoint{0.011785in}{-0.011785in}}%
\pgfpathcurveto{\pgfqpoint{0.014911in}{-0.008660in}}{\pgfqpoint{0.016667in}{-0.004420in}}{\pgfqpoint{0.016667in}{0.000000in}}%
\pgfpathcurveto{\pgfqpoint{0.016667in}{0.004420in}}{\pgfqpoint{0.014911in}{0.008660in}}{\pgfqpoint{0.011785in}{0.011785in}}%
\pgfpathcurveto{\pgfqpoint{0.008660in}{0.014911in}}{\pgfqpoint{0.004420in}{0.016667in}}{\pgfqpoint{0.000000in}{0.016667in}}%
\pgfpathcurveto{\pgfqpoint{-0.004420in}{0.016667in}}{\pgfqpoint{-0.008660in}{0.014911in}}{\pgfqpoint{-0.011785in}{0.011785in}}%
\pgfpathcurveto{\pgfqpoint{-0.014911in}{0.008660in}}{\pgfqpoint{-0.016667in}{0.004420in}}{\pgfqpoint{-0.016667in}{0.000000in}}%
\pgfpathcurveto{\pgfqpoint{-0.016667in}{-0.004420in}}{\pgfqpoint{-0.014911in}{-0.008660in}}{\pgfqpoint{-0.011785in}{-0.011785in}}%
\pgfpathcurveto{\pgfqpoint{-0.008660in}{-0.014911in}}{\pgfqpoint{-0.004420in}{-0.016667in}}{\pgfqpoint{0.000000in}{-0.016667in}}%
\pgfpathclose%
\pgfusepath{stroke,fill}%
}%
\begin{pgfscope}%
\pgfsys@transformshift{0.591363in}{5.372886in}%
\pgfsys@useobject{currentmarker}{}%
\end{pgfscope}%
\end{pgfscope}%
\begin{pgfscope}%
\pgfpathrectangle{\pgfqpoint{0.100000in}{2.413063in}}{\pgfqpoint{5.037500in}{3.427208in}}%
\pgfusepath{clip}%
\pgfsetrectcap%
\pgfsetroundjoin%
\pgfsetlinewidth{1.505625pt}%
\definecolor{currentstroke}{rgb}{0.000000,0.000000,1.000000}%
\pgfsetstrokecolor{currentstroke}%
\pgfsetstrokeopacity{0.500000}%
\pgfsetdash{}{0pt}%
\pgfpathmoveto{\pgfqpoint{0.543139in}{5.317548in}}%
\pgfusepath{stroke}%
\end{pgfscope}%
\begin{pgfscope}%
\pgfpathrectangle{\pgfqpoint{0.100000in}{2.413063in}}{\pgfqpoint{5.037500in}{3.427208in}}%
\pgfusepath{clip}%
\pgfsetbuttcap%
\pgfsetroundjoin%
\definecolor{currentfill}{rgb}{0.000000,0.000000,1.000000}%
\pgfsetfillcolor{currentfill}%
\pgfsetfillopacity{0.500000}%
\pgfsetlinewidth{0.250937pt}%
\definecolor{currentstroke}{rgb}{0.000000,0.000000,0.000000}%
\pgfsetstrokecolor{currentstroke}%
\pgfsetstrokeopacity{0.500000}%
\pgfsetdash{}{0pt}%
\pgfsys@defobject{currentmarker}{\pgfqpoint{-0.013889in}{-0.013889in}}{\pgfqpoint{0.013889in}{0.013889in}}{%
\pgfpathmoveto{\pgfqpoint{0.000000in}{-0.013889in}}%
\pgfpathcurveto{\pgfqpoint{0.003683in}{-0.013889in}}{\pgfqpoint{0.007216in}{-0.012425in}}{\pgfqpoint{0.009821in}{-0.009821in}}%
\pgfpathcurveto{\pgfqpoint{0.012425in}{-0.007216in}}{\pgfqpoint{0.013889in}{-0.003683in}}{\pgfqpoint{0.013889in}{0.000000in}}%
\pgfpathcurveto{\pgfqpoint{0.013889in}{0.003683in}}{\pgfqpoint{0.012425in}{0.007216in}}{\pgfqpoint{0.009821in}{0.009821in}}%
\pgfpathcurveto{\pgfqpoint{0.007216in}{0.012425in}}{\pgfqpoint{0.003683in}{0.013889in}}{\pgfqpoint{0.000000in}{0.013889in}}%
\pgfpathcurveto{\pgfqpoint{-0.003683in}{0.013889in}}{\pgfqpoint{-0.007216in}{0.012425in}}{\pgfqpoint{-0.009821in}{0.009821in}}%
\pgfpathcurveto{\pgfqpoint{-0.012425in}{0.007216in}}{\pgfqpoint{-0.013889in}{0.003683in}}{\pgfqpoint{-0.013889in}{0.000000in}}%
\pgfpathcurveto{\pgfqpoint{-0.013889in}{-0.003683in}}{\pgfqpoint{-0.012425in}{-0.007216in}}{\pgfqpoint{-0.009821in}{-0.009821in}}%
\pgfpathcurveto{\pgfqpoint{-0.007216in}{-0.012425in}}{\pgfqpoint{-0.003683in}{-0.013889in}}{\pgfqpoint{0.000000in}{-0.013889in}}%
\pgfpathclose%
\pgfusepath{stroke,fill}%
}%
\begin{pgfscope}%
\pgfsys@transformshift{0.543139in}{5.317548in}%
\pgfsys@useobject{currentmarker}{}%
\end{pgfscope}%
\end{pgfscope}%
\begin{pgfscope}%
\pgfpathrectangle{\pgfqpoint{0.100000in}{2.413063in}}{\pgfqpoint{5.037500in}{3.427208in}}%
\pgfusepath{clip}%
\pgfsetrectcap%
\pgfsetroundjoin%
\pgfsetlinewidth{1.505625pt}%
\definecolor{currentstroke}{rgb}{0.678431,1.000000,0.184314}%
\pgfsetstrokecolor{currentstroke}%
\pgfsetstrokeopacity{0.500000}%
\pgfsetdash{}{0pt}%
\pgfpathmoveto{\pgfqpoint{4.498915in}{4.648262in}}%
\pgfusepath{stroke}%
\end{pgfscope}%
\begin{pgfscope}%
\pgfpathrectangle{\pgfqpoint{0.100000in}{2.413063in}}{\pgfqpoint{5.037500in}{3.427208in}}%
\pgfusepath{clip}%
\pgfsetbuttcap%
\pgfsetroundjoin%
\definecolor{currentfill}{rgb}{0.678431,1.000000,0.184314}%
\pgfsetfillcolor{currentfill}%
\pgfsetfillopacity{0.500000}%
\pgfsetlinewidth{0.250937pt}%
\definecolor{currentstroke}{rgb}{0.000000,0.000000,0.000000}%
\pgfsetstrokecolor{currentstroke}%
\pgfsetstrokeopacity{0.500000}%
\pgfsetdash{}{0pt}%
\pgfsys@defobject{currentmarker}{\pgfqpoint{-0.011111in}{-0.011111in}}{\pgfqpoint{0.011111in}{0.011111in}}{%
\pgfpathmoveto{\pgfqpoint{0.000000in}{-0.011111in}}%
\pgfpathcurveto{\pgfqpoint{0.002947in}{-0.011111in}}{\pgfqpoint{0.005773in}{-0.009940in}}{\pgfqpoint{0.007857in}{-0.007857in}}%
\pgfpathcurveto{\pgfqpoint{0.009940in}{-0.005773in}}{\pgfqpoint{0.011111in}{-0.002947in}}{\pgfqpoint{0.011111in}{0.000000in}}%
\pgfpathcurveto{\pgfqpoint{0.011111in}{0.002947in}}{\pgfqpoint{0.009940in}{0.005773in}}{\pgfqpoint{0.007857in}{0.007857in}}%
\pgfpathcurveto{\pgfqpoint{0.005773in}{0.009940in}}{\pgfqpoint{0.002947in}{0.011111in}}{\pgfqpoint{0.000000in}{0.011111in}}%
\pgfpathcurveto{\pgfqpoint{-0.002947in}{0.011111in}}{\pgfqpoint{-0.005773in}{0.009940in}}{\pgfqpoint{-0.007857in}{0.007857in}}%
\pgfpathcurveto{\pgfqpoint{-0.009940in}{0.005773in}}{\pgfqpoint{-0.011111in}{0.002947in}}{\pgfqpoint{-0.011111in}{0.000000in}}%
\pgfpathcurveto{\pgfqpoint{-0.011111in}{-0.002947in}}{\pgfqpoint{-0.009940in}{-0.005773in}}{\pgfqpoint{-0.007857in}{-0.007857in}}%
\pgfpathcurveto{\pgfqpoint{-0.005773in}{-0.009940in}}{\pgfqpoint{-0.002947in}{-0.011111in}}{\pgfqpoint{0.000000in}{-0.011111in}}%
\pgfpathclose%
\pgfusepath{stroke,fill}%
}%
\begin{pgfscope}%
\pgfsys@transformshift{4.498915in}{4.648262in}%
\pgfsys@useobject{currentmarker}{}%
\end{pgfscope}%
\end{pgfscope}%
\begin{pgfscope}%
\pgfpathrectangle{\pgfqpoint{0.100000in}{2.413063in}}{\pgfqpoint{5.037500in}{3.427208in}}%
\pgfusepath{clip}%
\pgfsetrectcap%
\pgfsetroundjoin%
\pgfsetlinewidth{1.505625pt}%
\definecolor{currentstroke}{rgb}{0.678431,1.000000,0.184314}%
\pgfsetstrokecolor{currentstroke}%
\pgfsetstrokeopacity{0.500000}%
\pgfsetdash{}{0pt}%
\pgfpathmoveto{\pgfqpoint{4.250037in}{4.588307in}}%
\pgfusepath{stroke}%
\end{pgfscope}%
\begin{pgfscope}%
\pgfpathrectangle{\pgfqpoint{0.100000in}{2.413063in}}{\pgfqpoint{5.037500in}{3.427208in}}%
\pgfusepath{clip}%
\pgfsetbuttcap%
\pgfsetroundjoin%
\definecolor{currentfill}{rgb}{0.678431,1.000000,0.184314}%
\pgfsetfillcolor{currentfill}%
\pgfsetfillopacity{0.500000}%
\pgfsetlinewidth{0.250937pt}%
\definecolor{currentstroke}{rgb}{0.000000,0.000000,0.000000}%
\pgfsetstrokecolor{currentstroke}%
\pgfsetstrokeopacity{0.500000}%
\pgfsetdash{}{0pt}%
\pgfsys@defobject{currentmarker}{\pgfqpoint{-0.027778in}{-0.027778in}}{\pgfqpoint{0.027778in}{0.027778in}}{%
\pgfpathmoveto{\pgfqpoint{0.000000in}{-0.027778in}}%
\pgfpathcurveto{\pgfqpoint{0.007367in}{-0.027778in}}{\pgfqpoint{0.014433in}{-0.024851in}}{\pgfqpoint{0.019642in}{-0.019642in}}%
\pgfpathcurveto{\pgfqpoint{0.024851in}{-0.014433in}}{\pgfqpoint{0.027778in}{-0.007367in}}{\pgfqpoint{0.027778in}{0.000000in}}%
\pgfpathcurveto{\pgfqpoint{0.027778in}{0.007367in}}{\pgfqpoint{0.024851in}{0.014433in}}{\pgfqpoint{0.019642in}{0.019642in}}%
\pgfpathcurveto{\pgfqpoint{0.014433in}{0.024851in}}{\pgfqpoint{0.007367in}{0.027778in}}{\pgfqpoint{0.000000in}{0.027778in}}%
\pgfpathcurveto{\pgfqpoint{-0.007367in}{0.027778in}}{\pgfqpoint{-0.014433in}{0.024851in}}{\pgfqpoint{-0.019642in}{0.019642in}}%
\pgfpathcurveto{\pgfqpoint{-0.024851in}{0.014433in}}{\pgfqpoint{-0.027778in}{0.007367in}}{\pgfqpoint{-0.027778in}{0.000000in}}%
\pgfpathcurveto{\pgfqpoint{-0.027778in}{-0.007367in}}{\pgfqpoint{-0.024851in}{-0.014433in}}{\pgfqpoint{-0.019642in}{-0.019642in}}%
\pgfpathcurveto{\pgfqpoint{-0.014433in}{-0.024851in}}{\pgfqpoint{-0.007367in}{-0.027778in}}{\pgfqpoint{0.000000in}{-0.027778in}}%
\pgfpathclose%
\pgfusepath{stroke,fill}%
}%
\begin{pgfscope}%
\pgfsys@transformshift{4.250037in}{4.588307in}%
\pgfsys@useobject{currentmarker}{}%
\end{pgfscope}%
\end{pgfscope}%
\begin{pgfscope}%
\pgfpathrectangle{\pgfqpoint{0.100000in}{2.413063in}}{\pgfqpoint{5.037500in}{3.427208in}}%
\pgfusepath{clip}%
\pgfsetrectcap%
\pgfsetroundjoin%
\pgfsetlinewidth{1.505625pt}%
\definecolor{currentstroke}{rgb}{0.678431,1.000000,0.184314}%
\pgfsetstrokecolor{currentstroke}%
\pgfsetstrokeopacity{0.500000}%
\pgfsetdash{}{0pt}%
\pgfpathmoveto{\pgfqpoint{4.405482in}{4.675732in}}%
\pgfusepath{stroke}%
\end{pgfscope}%
\begin{pgfscope}%
\pgfpathrectangle{\pgfqpoint{0.100000in}{2.413063in}}{\pgfqpoint{5.037500in}{3.427208in}}%
\pgfusepath{clip}%
\pgfsetbuttcap%
\pgfsetroundjoin%
\definecolor{currentfill}{rgb}{0.678431,1.000000,0.184314}%
\pgfsetfillcolor{currentfill}%
\pgfsetfillopacity{0.500000}%
\pgfsetlinewidth{0.250937pt}%
\definecolor{currentstroke}{rgb}{0.000000,0.000000,0.000000}%
\pgfsetstrokecolor{currentstroke}%
\pgfsetstrokeopacity{0.500000}%
\pgfsetdash{}{0pt}%
\pgfsys@defobject{currentmarker}{\pgfqpoint{-0.038889in}{-0.038889in}}{\pgfqpoint{0.038889in}{0.038889in}}{%
\pgfpathmoveto{\pgfqpoint{0.000000in}{-0.038889in}}%
\pgfpathcurveto{\pgfqpoint{0.010313in}{-0.038889in}}{\pgfqpoint{0.020206in}{-0.034791in}}{\pgfqpoint{0.027499in}{-0.027499in}}%
\pgfpathcurveto{\pgfqpoint{0.034791in}{-0.020206in}}{\pgfqpoint{0.038889in}{-0.010313in}}{\pgfqpoint{0.038889in}{0.000000in}}%
\pgfpathcurveto{\pgfqpoint{0.038889in}{0.010313in}}{\pgfqpoint{0.034791in}{0.020206in}}{\pgfqpoint{0.027499in}{0.027499in}}%
\pgfpathcurveto{\pgfqpoint{0.020206in}{0.034791in}}{\pgfqpoint{0.010313in}{0.038889in}}{\pgfqpoint{0.000000in}{0.038889in}}%
\pgfpathcurveto{\pgfqpoint{-0.010313in}{0.038889in}}{\pgfqpoint{-0.020206in}{0.034791in}}{\pgfqpoint{-0.027499in}{0.027499in}}%
\pgfpathcurveto{\pgfqpoint{-0.034791in}{0.020206in}}{\pgfqpoint{-0.038889in}{0.010313in}}{\pgfqpoint{-0.038889in}{0.000000in}}%
\pgfpathcurveto{\pgfqpoint{-0.038889in}{-0.010313in}}{\pgfqpoint{-0.034791in}{-0.020206in}}{\pgfqpoint{-0.027499in}{-0.027499in}}%
\pgfpathcurveto{\pgfqpoint{-0.020206in}{-0.034791in}}{\pgfqpoint{-0.010313in}{-0.038889in}}{\pgfqpoint{0.000000in}{-0.038889in}}%
\pgfpathclose%
\pgfusepath{stroke,fill}%
}%
\begin{pgfscope}%
\pgfsys@transformshift{4.405482in}{4.675732in}%
\pgfsys@useobject{currentmarker}{}%
\end{pgfscope}%
\end{pgfscope}%
\begin{pgfscope}%
\pgfpathrectangle{\pgfqpoint{0.100000in}{2.413063in}}{\pgfqpoint{5.037500in}{3.427208in}}%
\pgfusepath{clip}%
\pgfsetrectcap%
\pgfsetroundjoin%
\pgfsetlinewidth{1.505625pt}%
\definecolor{currentstroke}{rgb}{0.678431,1.000000,0.184314}%
\pgfsetstrokecolor{currentstroke}%
\pgfsetstrokeopacity{0.500000}%
\pgfsetdash{}{0pt}%
\pgfpathmoveto{\pgfqpoint{4.325892in}{4.534885in}}%
\pgfusepath{stroke}%
\end{pgfscope}%
\begin{pgfscope}%
\pgfpathrectangle{\pgfqpoint{0.100000in}{2.413063in}}{\pgfqpoint{5.037500in}{3.427208in}}%
\pgfusepath{clip}%
\pgfsetbuttcap%
\pgfsetroundjoin%
\definecolor{currentfill}{rgb}{0.678431,1.000000,0.184314}%
\pgfsetfillcolor{currentfill}%
\pgfsetfillopacity{0.500000}%
\pgfsetlinewidth{0.250937pt}%
\definecolor{currentstroke}{rgb}{0.000000,0.000000,0.000000}%
\pgfsetstrokecolor{currentstroke}%
\pgfsetstrokeopacity{0.500000}%
\pgfsetdash{}{0pt}%
\pgfsys@defobject{currentmarker}{\pgfqpoint{-0.030556in}{-0.030556in}}{\pgfqpoint{0.030556in}{0.030556in}}{%
\pgfpathmoveto{\pgfqpoint{0.000000in}{-0.030556in}}%
\pgfpathcurveto{\pgfqpoint{0.008103in}{-0.030556in}}{\pgfqpoint{0.015876in}{-0.027336in}}{\pgfqpoint{0.021606in}{-0.021606in}}%
\pgfpathcurveto{\pgfqpoint{0.027336in}{-0.015876in}}{\pgfqpoint{0.030556in}{-0.008103in}}{\pgfqpoint{0.030556in}{0.000000in}}%
\pgfpathcurveto{\pgfqpoint{0.030556in}{0.008103in}}{\pgfqpoint{0.027336in}{0.015876in}}{\pgfqpoint{0.021606in}{0.021606in}}%
\pgfpathcurveto{\pgfqpoint{0.015876in}{0.027336in}}{\pgfqpoint{0.008103in}{0.030556in}}{\pgfqpoint{0.000000in}{0.030556in}}%
\pgfpathcurveto{\pgfqpoint{-0.008103in}{0.030556in}}{\pgfqpoint{-0.015876in}{0.027336in}}{\pgfqpoint{-0.021606in}{0.021606in}}%
\pgfpathcurveto{\pgfqpoint{-0.027336in}{0.015876in}}{\pgfqpoint{-0.030556in}{0.008103in}}{\pgfqpoint{-0.030556in}{0.000000in}}%
\pgfpathcurveto{\pgfqpoint{-0.030556in}{-0.008103in}}{\pgfqpoint{-0.027336in}{-0.015876in}}{\pgfqpoint{-0.021606in}{-0.021606in}}%
\pgfpathcurveto{\pgfqpoint{-0.015876in}{-0.027336in}}{\pgfqpoint{-0.008103in}{-0.030556in}}{\pgfqpoint{0.000000in}{-0.030556in}}%
\pgfpathclose%
\pgfusepath{stroke,fill}%
}%
\begin{pgfscope}%
\pgfsys@transformshift{4.325892in}{4.534885in}%
\pgfsys@useobject{currentmarker}{}%
\end{pgfscope}%
\end{pgfscope}%
\begin{pgfscope}%
\pgfpathrectangle{\pgfqpoint{0.100000in}{2.413063in}}{\pgfqpoint{5.037500in}{3.427208in}}%
\pgfusepath{clip}%
\pgfsetrectcap%
\pgfsetroundjoin%
\pgfsetlinewidth{1.505625pt}%
\definecolor{currentstroke}{rgb}{0.678431,1.000000,0.184314}%
\pgfsetstrokecolor{currentstroke}%
\pgfsetstrokeopacity{0.500000}%
\pgfsetdash{}{0pt}%
\pgfpathmoveto{\pgfqpoint{4.513781in}{4.698398in}}%
\pgfusepath{stroke}%
\end{pgfscope}%
\begin{pgfscope}%
\pgfpathrectangle{\pgfqpoint{0.100000in}{2.413063in}}{\pgfqpoint{5.037500in}{3.427208in}}%
\pgfusepath{clip}%
\pgfsetbuttcap%
\pgfsetroundjoin%
\definecolor{currentfill}{rgb}{0.678431,1.000000,0.184314}%
\pgfsetfillcolor{currentfill}%
\pgfsetfillopacity{0.500000}%
\pgfsetlinewidth{0.250937pt}%
\definecolor{currentstroke}{rgb}{0.000000,0.000000,0.000000}%
\pgfsetstrokecolor{currentstroke}%
\pgfsetstrokeopacity{0.500000}%
\pgfsetdash{}{0pt}%
\pgfsys@defobject{currentmarker}{\pgfqpoint{-0.008333in}{-0.008333in}}{\pgfqpoint{0.008333in}{0.008333in}}{%
\pgfpathmoveto{\pgfqpoint{0.000000in}{-0.008333in}}%
\pgfpathcurveto{\pgfqpoint{0.002210in}{-0.008333in}}{\pgfqpoint{0.004330in}{-0.007455in}}{\pgfqpoint{0.005893in}{-0.005893in}}%
\pgfpathcurveto{\pgfqpoint{0.007455in}{-0.004330in}}{\pgfqpoint{0.008333in}{-0.002210in}}{\pgfqpoint{0.008333in}{0.000000in}}%
\pgfpathcurveto{\pgfqpoint{0.008333in}{0.002210in}}{\pgfqpoint{0.007455in}{0.004330in}}{\pgfqpoint{0.005893in}{0.005893in}}%
\pgfpathcurveto{\pgfqpoint{0.004330in}{0.007455in}}{\pgfqpoint{0.002210in}{0.008333in}}{\pgfqpoint{0.000000in}{0.008333in}}%
\pgfpathcurveto{\pgfqpoint{-0.002210in}{0.008333in}}{\pgfqpoint{-0.004330in}{0.007455in}}{\pgfqpoint{-0.005893in}{0.005893in}}%
\pgfpathcurveto{\pgfqpoint{-0.007455in}{0.004330in}}{\pgfqpoint{-0.008333in}{0.002210in}}{\pgfqpoint{-0.008333in}{0.000000in}}%
\pgfpathcurveto{\pgfqpoint{-0.008333in}{-0.002210in}}{\pgfqpoint{-0.007455in}{-0.004330in}}{\pgfqpoint{-0.005893in}{-0.005893in}}%
\pgfpathcurveto{\pgfqpoint{-0.004330in}{-0.007455in}}{\pgfqpoint{-0.002210in}{-0.008333in}}{\pgfqpoint{0.000000in}{-0.008333in}}%
\pgfpathclose%
\pgfusepath{stroke,fill}%
}%
\begin{pgfscope}%
\pgfsys@transformshift{4.513781in}{4.698398in}%
\pgfsys@useobject{currentmarker}{}%
\end{pgfscope}%
\end{pgfscope}%
\begin{pgfscope}%
\pgfpathrectangle{\pgfqpoint{0.100000in}{2.413063in}}{\pgfqpoint{5.037500in}{3.427208in}}%
\pgfusepath{clip}%
\pgfsetrectcap%
\pgfsetroundjoin%
\pgfsetlinewidth{1.505625pt}%
\definecolor{currentstroke}{rgb}{0.501961,0.501961,0.501961}%
\pgfsetstrokecolor{currentstroke}%
\pgfsetstrokeopacity{0.500000}%
\pgfsetdash{}{0pt}%
\pgfpathmoveto{\pgfqpoint{4.073915in}{4.745853in}}%
\pgfusepath{stroke}%
\end{pgfscope}%
\begin{pgfscope}%
\pgfpathrectangle{\pgfqpoint{0.100000in}{2.413063in}}{\pgfqpoint{5.037500in}{3.427208in}}%
\pgfusepath{clip}%
\pgfsetbuttcap%
\pgfsetroundjoin%
\definecolor{currentfill}{rgb}{0.501961,0.501961,0.501961}%
\pgfsetfillcolor{currentfill}%
\pgfsetfillopacity{0.500000}%
\pgfsetlinewidth{0.250937pt}%
\definecolor{currentstroke}{rgb}{0.000000,0.000000,0.000000}%
\pgfsetstrokecolor{currentstroke}%
\pgfsetstrokeopacity{0.500000}%
\pgfsetdash{}{0pt}%
\pgfsys@defobject{currentmarker}{\pgfqpoint{-0.013889in}{-0.013889in}}{\pgfqpoint{0.013889in}{0.013889in}}{%
\pgfpathmoveto{\pgfqpoint{0.000000in}{-0.013889in}}%
\pgfpathcurveto{\pgfqpoint{0.003683in}{-0.013889in}}{\pgfqpoint{0.007216in}{-0.012425in}}{\pgfqpoint{0.009821in}{-0.009821in}}%
\pgfpathcurveto{\pgfqpoint{0.012425in}{-0.007216in}}{\pgfqpoint{0.013889in}{-0.003683in}}{\pgfqpoint{0.013889in}{0.000000in}}%
\pgfpathcurveto{\pgfqpoint{0.013889in}{0.003683in}}{\pgfqpoint{0.012425in}{0.007216in}}{\pgfqpoint{0.009821in}{0.009821in}}%
\pgfpathcurveto{\pgfqpoint{0.007216in}{0.012425in}}{\pgfqpoint{0.003683in}{0.013889in}}{\pgfqpoint{0.000000in}{0.013889in}}%
\pgfpathcurveto{\pgfqpoint{-0.003683in}{0.013889in}}{\pgfqpoint{-0.007216in}{0.012425in}}{\pgfqpoint{-0.009821in}{0.009821in}}%
\pgfpathcurveto{\pgfqpoint{-0.012425in}{0.007216in}}{\pgfqpoint{-0.013889in}{0.003683in}}{\pgfqpoint{-0.013889in}{0.000000in}}%
\pgfpathcurveto{\pgfqpoint{-0.013889in}{-0.003683in}}{\pgfqpoint{-0.012425in}{-0.007216in}}{\pgfqpoint{-0.009821in}{-0.009821in}}%
\pgfpathcurveto{\pgfqpoint{-0.007216in}{-0.012425in}}{\pgfqpoint{-0.003683in}{-0.013889in}}{\pgfqpoint{0.000000in}{-0.013889in}}%
\pgfpathclose%
\pgfusepath{stroke,fill}%
}%
\begin{pgfscope}%
\pgfsys@transformshift{4.073915in}{4.745853in}%
\pgfsys@useobject{currentmarker}{}%
\end{pgfscope}%
\end{pgfscope}%
\begin{pgfscope}%
\pgfpathrectangle{\pgfqpoint{0.100000in}{2.413063in}}{\pgfqpoint{5.037500in}{3.427208in}}%
\pgfusepath{clip}%
\pgfsetrectcap%
\pgfsetroundjoin%
\pgfsetlinewidth{1.505625pt}%
\definecolor{currentstroke}{rgb}{0.678431,1.000000,0.184314}%
\pgfsetstrokecolor{currentstroke}%
\pgfsetstrokeopacity{0.500000}%
\pgfsetdash{}{0pt}%
\pgfpathmoveto{\pgfqpoint{4.365499in}{4.529766in}}%
\pgfusepath{stroke}%
\end{pgfscope}%
\begin{pgfscope}%
\pgfpathrectangle{\pgfqpoint{0.100000in}{2.413063in}}{\pgfqpoint{5.037500in}{3.427208in}}%
\pgfusepath{clip}%
\pgfsetbuttcap%
\pgfsetroundjoin%
\definecolor{currentfill}{rgb}{0.678431,1.000000,0.184314}%
\pgfsetfillcolor{currentfill}%
\pgfsetfillopacity{0.500000}%
\pgfsetlinewidth{0.250937pt}%
\definecolor{currentstroke}{rgb}{0.000000,0.000000,0.000000}%
\pgfsetstrokecolor{currentstroke}%
\pgfsetstrokeopacity{0.500000}%
\pgfsetdash{}{0pt}%
\pgfsys@defobject{currentmarker}{\pgfqpoint{-0.016667in}{-0.016667in}}{\pgfqpoint{0.016667in}{0.016667in}}{%
\pgfpathmoveto{\pgfqpoint{0.000000in}{-0.016667in}}%
\pgfpathcurveto{\pgfqpoint{0.004420in}{-0.016667in}}{\pgfqpoint{0.008660in}{-0.014911in}}{\pgfqpoint{0.011785in}{-0.011785in}}%
\pgfpathcurveto{\pgfqpoint{0.014911in}{-0.008660in}}{\pgfqpoint{0.016667in}{-0.004420in}}{\pgfqpoint{0.016667in}{0.000000in}}%
\pgfpathcurveto{\pgfqpoint{0.016667in}{0.004420in}}{\pgfqpoint{0.014911in}{0.008660in}}{\pgfqpoint{0.011785in}{0.011785in}}%
\pgfpathcurveto{\pgfqpoint{0.008660in}{0.014911in}}{\pgfqpoint{0.004420in}{0.016667in}}{\pgfqpoint{0.000000in}{0.016667in}}%
\pgfpathcurveto{\pgfqpoint{-0.004420in}{0.016667in}}{\pgfqpoint{-0.008660in}{0.014911in}}{\pgfqpoint{-0.011785in}{0.011785in}}%
\pgfpathcurveto{\pgfqpoint{-0.014911in}{0.008660in}}{\pgfqpoint{-0.016667in}{0.004420in}}{\pgfqpoint{-0.016667in}{0.000000in}}%
\pgfpathcurveto{\pgfqpoint{-0.016667in}{-0.004420in}}{\pgfqpoint{-0.014911in}{-0.008660in}}{\pgfqpoint{-0.011785in}{-0.011785in}}%
\pgfpathcurveto{\pgfqpoint{-0.008660in}{-0.014911in}}{\pgfqpoint{-0.004420in}{-0.016667in}}{\pgfqpoint{0.000000in}{-0.016667in}}%
\pgfpathclose%
\pgfusepath{stroke,fill}%
}%
\begin{pgfscope}%
\pgfsys@transformshift{4.365499in}{4.529766in}%
\pgfsys@useobject{currentmarker}{}%
\end{pgfscope}%
\end{pgfscope}%
\begin{pgfscope}%
\pgfpathrectangle{\pgfqpoint{0.100000in}{2.413063in}}{\pgfqpoint{5.037500in}{3.427208in}}%
\pgfusepath{clip}%
\pgfsetrectcap%
\pgfsetroundjoin%
\pgfsetlinewidth{1.505625pt}%
\definecolor{currentstroke}{rgb}{0.678431,1.000000,0.184314}%
\pgfsetstrokecolor{currentstroke}%
\pgfsetstrokeopacity{0.500000}%
\pgfsetdash{}{0pt}%
\pgfpathmoveto{\pgfqpoint{4.385486in}{4.584896in}}%
\pgfusepath{stroke}%
\end{pgfscope}%
\begin{pgfscope}%
\pgfpathrectangle{\pgfqpoint{0.100000in}{2.413063in}}{\pgfqpoint{5.037500in}{3.427208in}}%
\pgfusepath{clip}%
\pgfsetbuttcap%
\pgfsetroundjoin%
\definecolor{currentfill}{rgb}{0.678431,1.000000,0.184314}%
\pgfsetfillcolor{currentfill}%
\pgfsetfillopacity{0.500000}%
\pgfsetlinewidth{0.250937pt}%
\definecolor{currentstroke}{rgb}{0.000000,0.000000,0.000000}%
\pgfsetstrokecolor{currentstroke}%
\pgfsetstrokeopacity{0.500000}%
\pgfsetdash{}{0pt}%
\pgfsys@defobject{currentmarker}{\pgfqpoint{-0.011111in}{-0.011111in}}{\pgfqpoint{0.011111in}{0.011111in}}{%
\pgfpathmoveto{\pgfqpoint{0.000000in}{-0.011111in}}%
\pgfpathcurveto{\pgfqpoint{0.002947in}{-0.011111in}}{\pgfqpoint{0.005773in}{-0.009940in}}{\pgfqpoint{0.007857in}{-0.007857in}}%
\pgfpathcurveto{\pgfqpoint{0.009940in}{-0.005773in}}{\pgfqpoint{0.011111in}{-0.002947in}}{\pgfqpoint{0.011111in}{0.000000in}}%
\pgfpathcurveto{\pgfqpoint{0.011111in}{0.002947in}}{\pgfqpoint{0.009940in}{0.005773in}}{\pgfqpoint{0.007857in}{0.007857in}}%
\pgfpathcurveto{\pgfqpoint{0.005773in}{0.009940in}}{\pgfqpoint{0.002947in}{0.011111in}}{\pgfqpoint{0.000000in}{0.011111in}}%
\pgfpathcurveto{\pgfqpoint{-0.002947in}{0.011111in}}{\pgfqpoint{-0.005773in}{0.009940in}}{\pgfqpoint{-0.007857in}{0.007857in}}%
\pgfpathcurveto{\pgfqpoint{-0.009940in}{0.005773in}}{\pgfqpoint{-0.011111in}{0.002947in}}{\pgfqpoint{-0.011111in}{0.000000in}}%
\pgfpathcurveto{\pgfqpoint{-0.011111in}{-0.002947in}}{\pgfqpoint{-0.009940in}{-0.005773in}}{\pgfqpoint{-0.007857in}{-0.007857in}}%
\pgfpathcurveto{\pgfqpoint{-0.005773in}{-0.009940in}}{\pgfqpoint{-0.002947in}{-0.011111in}}{\pgfqpoint{0.000000in}{-0.011111in}}%
\pgfpathclose%
\pgfusepath{stroke,fill}%
}%
\begin{pgfscope}%
\pgfsys@transformshift{4.385486in}{4.584896in}%
\pgfsys@useobject{currentmarker}{}%
\end{pgfscope}%
\end{pgfscope}%
\begin{pgfscope}%
\pgfpathrectangle{\pgfqpoint{0.100000in}{2.413063in}}{\pgfqpoint{5.037500in}{3.427208in}}%
\pgfusepath{clip}%
\pgfsetrectcap%
\pgfsetroundjoin%
\pgfsetlinewidth{1.505625pt}%
\definecolor{currentstroke}{rgb}{0.678431,1.000000,0.184314}%
\pgfsetstrokecolor{currentstroke}%
\pgfsetstrokeopacity{0.500000}%
\pgfsetdash{}{0pt}%
\pgfpathmoveto{\pgfqpoint{4.208507in}{4.558288in}}%
\pgfusepath{stroke}%
\end{pgfscope}%
\begin{pgfscope}%
\pgfpathrectangle{\pgfqpoint{0.100000in}{2.413063in}}{\pgfqpoint{5.037500in}{3.427208in}}%
\pgfusepath{clip}%
\pgfsetbuttcap%
\pgfsetroundjoin%
\definecolor{currentfill}{rgb}{0.678431,1.000000,0.184314}%
\pgfsetfillcolor{currentfill}%
\pgfsetfillopacity{0.500000}%
\pgfsetlinewidth{0.250937pt}%
\definecolor{currentstroke}{rgb}{0.000000,0.000000,0.000000}%
\pgfsetstrokecolor{currentstroke}%
\pgfsetstrokeopacity{0.500000}%
\pgfsetdash{}{0pt}%
\pgfsys@defobject{currentmarker}{\pgfqpoint{-0.025000in}{-0.025000in}}{\pgfqpoint{0.025000in}{0.025000in}}{%
\pgfpathmoveto{\pgfqpoint{0.000000in}{-0.025000in}}%
\pgfpathcurveto{\pgfqpoint{0.006630in}{-0.025000in}}{\pgfqpoint{0.012989in}{-0.022366in}}{\pgfqpoint{0.017678in}{-0.017678in}}%
\pgfpathcurveto{\pgfqpoint{0.022366in}{-0.012989in}}{\pgfqpoint{0.025000in}{-0.006630in}}{\pgfqpoint{0.025000in}{0.000000in}}%
\pgfpathcurveto{\pgfqpoint{0.025000in}{0.006630in}}{\pgfqpoint{0.022366in}{0.012989in}}{\pgfqpoint{0.017678in}{0.017678in}}%
\pgfpathcurveto{\pgfqpoint{0.012989in}{0.022366in}}{\pgfqpoint{0.006630in}{0.025000in}}{\pgfqpoint{0.000000in}{0.025000in}}%
\pgfpathcurveto{\pgfqpoint{-0.006630in}{0.025000in}}{\pgfqpoint{-0.012989in}{0.022366in}}{\pgfqpoint{-0.017678in}{0.017678in}}%
\pgfpathcurveto{\pgfqpoint{-0.022366in}{0.012989in}}{\pgfqpoint{-0.025000in}{0.006630in}}{\pgfqpoint{-0.025000in}{0.000000in}}%
\pgfpathcurveto{\pgfqpoint{-0.025000in}{-0.006630in}}{\pgfqpoint{-0.022366in}{-0.012989in}}{\pgfqpoint{-0.017678in}{-0.017678in}}%
\pgfpathcurveto{\pgfqpoint{-0.012989in}{-0.022366in}}{\pgfqpoint{-0.006630in}{-0.025000in}}{\pgfqpoint{0.000000in}{-0.025000in}}%
\pgfpathclose%
\pgfusepath{stroke,fill}%
}%
\begin{pgfscope}%
\pgfsys@transformshift{4.208507in}{4.558288in}%
\pgfsys@useobject{currentmarker}{}%
\end{pgfscope}%
\end{pgfscope}%
\begin{pgfscope}%
\pgfpathrectangle{\pgfqpoint{0.100000in}{2.413063in}}{\pgfqpoint{5.037500in}{3.427208in}}%
\pgfusepath{clip}%
\pgfsetrectcap%
\pgfsetroundjoin%
\pgfsetlinewidth{1.505625pt}%
\definecolor{currentstroke}{rgb}{0.678431,1.000000,0.184314}%
\pgfsetstrokecolor{currentstroke}%
\pgfsetstrokeopacity{0.500000}%
\pgfsetdash{}{0pt}%
\pgfpathmoveto{\pgfqpoint{4.440822in}{4.569431in}}%
\pgfusepath{stroke}%
\end{pgfscope}%
\begin{pgfscope}%
\pgfpathrectangle{\pgfqpoint{0.100000in}{2.413063in}}{\pgfqpoint{5.037500in}{3.427208in}}%
\pgfusepath{clip}%
\pgfsetbuttcap%
\pgfsetroundjoin%
\definecolor{currentfill}{rgb}{0.678431,1.000000,0.184314}%
\pgfsetfillcolor{currentfill}%
\pgfsetfillopacity{0.500000}%
\pgfsetlinewidth{0.250937pt}%
\definecolor{currentstroke}{rgb}{0.000000,0.000000,0.000000}%
\pgfsetstrokecolor{currentstroke}%
\pgfsetstrokeopacity{0.500000}%
\pgfsetdash{}{0pt}%
\pgfsys@defobject{currentmarker}{\pgfqpoint{-0.005556in}{-0.005556in}}{\pgfqpoint{0.005556in}{0.005556in}}{%
\pgfpathmoveto{\pgfqpoint{0.000000in}{-0.005556in}}%
\pgfpathcurveto{\pgfqpoint{0.001473in}{-0.005556in}}{\pgfqpoint{0.002887in}{-0.004970in}}{\pgfqpoint{0.003928in}{-0.003928in}}%
\pgfpathcurveto{\pgfqpoint{0.004970in}{-0.002887in}}{\pgfqpoint{0.005556in}{-0.001473in}}{\pgfqpoint{0.005556in}{0.000000in}}%
\pgfpathcurveto{\pgfqpoint{0.005556in}{0.001473in}}{\pgfqpoint{0.004970in}{0.002887in}}{\pgfqpoint{0.003928in}{0.003928in}}%
\pgfpathcurveto{\pgfqpoint{0.002887in}{0.004970in}}{\pgfqpoint{0.001473in}{0.005556in}}{\pgfqpoint{0.000000in}{0.005556in}}%
\pgfpathcurveto{\pgfqpoint{-0.001473in}{0.005556in}}{\pgfqpoint{-0.002887in}{0.004970in}}{\pgfqpoint{-0.003928in}{0.003928in}}%
\pgfpathcurveto{\pgfqpoint{-0.004970in}{0.002887in}}{\pgfqpoint{-0.005556in}{0.001473in}}{\pgfqpoint{-0.005556in}{0.000000in}}%
\pgfpathcurveto{\pgfqpoint{-0.005556in}{-0.001473in}}{\pgfqpoint{-0.004970in}{-0.002887in}}{\pgfqpoint{-0.003928in}{-0.003928in}}%
\pgfpathcurveto{\pgfqpoint{-0.002887in}{-0.004970in}}{\pgfqpoint{-0.001473in}{-0.005556in}}{\pgfqpoint{0.000000in}{-0.005556in}}%
\pgfpathclose%
\pgfusepath{stroke,fill}%
}%
\begin{pgfscope}%
\pgfsys@transformshift{4.440822in}{4.569431in}%
\pgfsys@useobject{currentmarker}{}%
\end{pgfscope}%
\end{pgfscope}%
\begin{pgfscope}%
\pgfpathrectangle{\pgfqpoint{0.100000in}{2.413063in}}{\pgfqpoint{5.037500in}{3.427208in}}%
\pgfusepath{clip}%
\pgfsetrectcap%
\pgfsetroundjoin%
\pgfsetlinewidth{1.505625pt}%
\definecolor{currentstroke}{rgb}{0.678431,1.000000,0.184314}%
\pgfsetstrokecolor{currentstroke}%
\pgfsetstrokeopacity{0.500000}%
\pgfsetdash{}{0pt}%
\pgfpathmoveto{\pgfqpoint{4.423020in}{4.602009in}}%
\pgfusepath{stroke}%
\end{pgfscope}%
\begin{pgfscope}%
\pgfpathrectangle{\pgfqpoint{0.100000in}{2.413063in}}{\pgfqpoint{5.037500in}{3.427208in}}%
\pgfusepath{clip}%
\pgfsetbuttcap%
\pgfsetroundjoin%
\definecolor{currentfill}{rgb}{0.678431,1.000000,0.184314}%
\pgfsetfillcolor{currentfill}%
\pgfsetfillopacity{0.500000}%
\pgfsetlinewidth{0.250937pt}%
\definecolor{currentstroke}{rgb}{0.000000,0.000000,0.000000}%
\pgfsetstrokecolor{currentstroke}%
\pgfsetstrokeopacity{0.500000}%
\pgfsetdash{}{0pt}%
\pgfsys@defobject{currentmarker}{\pgfqpoint{-0.019444in}{-0.019444in}}{\pgfqpoint{0.019444in}{0.019444in}}{%
\pgfpathmoveto{\pgfqpoint{0.000000in}{-0.019444in}}%
\pgfpathcurveto{\pgfqpoint{0.005157in}{-0.019444in}}{\pgfqpoint{0.010103in}{-0.017396in}}{\pgfqpoint{0.013749in}{-0.013749in}}%
\pgfpathcurveto{\pgfqpoint{0.017396in}{-0.010103in}}{\pgfqpoint{0.019444in}{-0.005157in}}{\pgfqpoint{0.019444in}{0.000000in}}%
\pgfpathcurveto{\pgfqpoint{0.019444in}{0.005157in}}{\pgfqpoint{0.017396in}{0.010103in}}{\pgfqpoint{0.013749in}{0.013749in}}%
\pgfpathcurveto{\pgfqpoint{0.010103in}{0.017396in}}{\pgfqpoint{0.005157in}{0.019444in}}{\pgfqpoint{0.000000in}{0.019444in}}%
\pgfpathcurveto{\pgfqpoint{-0.005157in}{0.019444in}}{\pgfqpoint{-0.010103in}{0.017396in}}{\pgfqpoint{-0.013749in}{0.013749in}}%
\pgfpathcurveto{\pgfqpoint{-0.017396in}{0.010103in}}{\pgfqpoint{-0.019444in}{0.005157in}}{\pgfqpoint{-0.019444in}{0.000000in}}%
\pgfpathcurveto{\pgfqpoint{-0.019444in}{-0.005157in}}{\pgfqpoint{-0.017396in}{-0.010103in}}{\pgfqpoint{-0.013749in}{-0.013749in}}%
\pgfpathcurveto{\pgfqpoint{-0.010103in}{-0.017396in}}{\pgfqpoint{-0.005157in}{-0.019444in}}{\pgfqpoint{0.000000in}{-0.019444in}}%
\pgfpathclose%
\pgfusepath{stroke,fill}%
}%
\begin{pgfscope}%
\pgfsys@transformshift{4.423020in}{4.602009in}%
\pgfsys@useobject{currentmarker}{}%
\end{pgfscope}%
\end{pgfscope}%
\begin{pgfscope}%
\pgfpathrectangle{\pgfqpoint{0.100000in}{2.413063in}}{\pgfqpoint{5.037500in}{3.427208in}}%
\pgfusepath{clip}%
\pgfsetrectcap%
\pgfsetroundjoin%
\pgfsetlinewidth{1.505625pt}%
\definecolor{currentstroke}{rgb}{0.000000,0.000000,1.000000}%
\pgfsetstrokecolor{currentstroke}%
\pgfsetstrokeopacity{0.500000}%
\pgfsetdash{}{0pt}%
\pgfpathmoveto{\pgfqpoint{4.541403in}{4.580980in}}%
\pgfusepath{stroke}%
\end{pgfscope}%
\begin{pgfscope}%
\pgfpathrectangle{\pgfqpoint{0.100000in}{2.413063in}}{\pgfqpoint{5.037500in}{3.427208in}}%
\pgfusepath{clip}%
\pgfsetbuttcap%
\pgfsetroundjoin%
\definecolor{currentfill}{rgb}{0.000000,0.000000,1.000000}%
\pgfsetfillcolor{currentfill}%
\pgfsetfillopacity{0.500000}%
\pgfsetlinewidth{0.250937pt}%
\definecolor{currentstroke}{rgb}{0.000000,0.000000,0.000000}%
\pgfsetstrokecolor{currentstroke}%
\pgfsetstrokeopacity{0.500000}%
\pgfsetdash{}{0pt}%
\pgfsys@defobject{currentmarker}{\pgfqpoint{-0.011111in}{-0.011111in}}{\pgfqpoint{0.011111in}{0.011111in}}{%
\pgfpathmoveto{\pgfqpoint{0.000000in}{-0.011111in}}%
\pgfpathcurveto{\pgfqpoint{0.002947in}{-0.011111in}}{\pgfqpoint{0.005773in}{-0.009940in}}{\pgfqpoint{0.007857in}{-0.007857in}}%
\pgfpathcurveto{\pgfqpoint{0.009940in}{-0.005773in}}{\pgfqpoint{0.011111in}{-0.002947in}}{\pgfqpoint{0.011111in}{0.000000in}}%
\pgfpathcurveto{\pgfqpoint{0.011111in}{0.002947in}}{\pgfqpoint{0.009940in}{0.005773in}}{\pgfqpoint{0.007857in}{0.007857in}}%
\pgfpathcurveto{\pgfqpoint{0.005773in}{0.009940in}}{\pgfqpoint{0.002947in}{0.011111in}}{\pgfqpoint{0.000000in}{0.011111in}}%
\pgfpathcurveto{\pgfqpoint{-0.002947in}{0.011111in}}{\pgfqpoint{-0.005773in}{0.009940in}}{\pgfqpoint{-0.007857in}{0.007857in}}%
\pgfpathcurveto{\pgfqpoint{-0.009940in}{0.005773in}}{\pgfqpoint{-0.011111in}{0.002947in}}{\pgfqpoint{-0.011111in}{0.000000in}}%
\pgfpathcurveto{\pgfqpoint{-0.011111in}{-0.002947in}}{\pgfqpoint{-0.009940in}{-0.005773in}}{\pgfqpoint{-0.007857in}{-0.007857in}}%
\pgfpathcurveto{\pgfqpoint{-0.005773in}{-0.009940in}}{\pgfqpoint{-0.002947in}{-0.011111in}}{\pgfqpoint{0.000000in}{-0.011111in}}%
\pgfpathclose%
\pgfusepath{stroke,fill}%
}%
\begin{pgfscope}%
\pgfsys@transformshift{4.541403in}{4.580980in}%
\pgfsys@useobject{currentmarker}{}%
\end{pgfscope}%
\end{pgfscope}%
\begin{pgfscope}%
\pgfpathrectangle{\pgfqpoint{0.100000in}{2.413063in}}{\pgfqpoint{5.037500in}{3.427208in}}%
\pgfusepath{clip}%
\pgfsetrectcap%
\pgfsetroundjoin%
\pgfsetlinewidth{1.505625pt}%
\definecolor{currentstroke}{rgb}{0.678431,1.000000,0.184314}%
\pgfsetstrokecolor{currentstroke}%
\pgfsetstrokeopacity{0.500000}%
\pgfsetdash{}{0pt}%
\pgfpathmoveto{\pgfqpoint{4.113913in}{4.555385in}}%
\pgfusepath{stroke}%
\end{pgfscope}%
\begin{pgfscope}%
\pgfpathrectangle{\pgfqpoint{0.100000in}{2.413063in}}{\pgfqpoint{5.037500in}{3.427208in}}%
\pgfusepath{clip}%
\pgfsetbuttcap%
\pgfsetroundjoin%
\definecolor{currentfill}{rgb}{0.678431,1.000000,0.184314}%
\pgfsetfillcolor{currentfill}%
\pgfsetfillopacity{0.500000}%
\pgfsetlinewidth{0.250937pt}%
\definecolor{currentstroke}{rgb}{0.000000,0.000000,0.000000}%
\pgfsetstrokecolor{currentstroke}%
\pgfsetstrokeopacity{0.500000}%
\pgfsetdash{}{0pt}%
\pgfsys@defobject{currentmarker}{\pgfqpoint{-0.011111in}{-0.011111in}}{\pgfqpoint{0.011111in}{0.011111in}}{%
\pgfpathmoveto{\pgfqpoint{0.000000in}{-0.011111in}}%
\pgfpathcurveto{\pgfqpoint{0.002947in}{-0.011111in}}{\pgfqpoint{0.005773in}{-0.009940in}}{\pgfqpoint{0.007857in}{-0.007857in}}%
\pgfpathcurveto{\pgfqpoint{0.009940in}{-0.005773in}}{\pgfqpoint{0.011111in}{-0.002947in}}{\pgfqpoint{0.011111in}{0.000000in}}%
\pgfpathcurveto{\pgfqpoint{0.011111in}{0.002947in}}{\pgfqpoint{0.009940in}{0.005773in}}{\pgfqpoint{0.007857in}{0.007857in}}%
\pgfpathcurveto{\pgfqpoint{0.005773in}{0.009940in}}{\pgfqpoint{0.002947in}{0.011111in}}{\pgfqpoint{0.000000in}{0.011111in}}%
\pgfpathcurveto{\pgfqpoint{-0.002947in}{0.011111in}}{\pgfqpoint{-0.005773in}{0.009940in}}{\pgfqpoint{-0.007857in}{0.007857in}}%
\pgfpathcurveto{\pgfqpoint{-0.009940in}{0.005773in}}{\pgfqpoint{-0.011111in}{0.002947in}}{\pgfqpoint{-0.011111in}{0.000000in}}%
\pgfpathcurveto{\pgfqpoint{-0.011111in}{-0.002947in}}{\pgfqpoint{-0.009940in}{-0.005773in}}{\pgfqpoint{-0.007857in}{-0.007857in}}%
\pgfpathcurveto{\pgfqpoint{-0.005773in}{-0.009940in}}{\pgfqpoint{-0.002947in}{-0.011111in}}{\pgfqpoint{0.000000in}{-0.011111in}}%
\pgfpathclose%
\pgfusepath{stroke,fill}%
}%
\begin{pgfscope}%
\pgfsys@transformshift{4.113913in}{4.555385in}%
\pgfsys@useobject{currentmarker}{}%
\end{pgfscope}%
\end{pgfscope}%
\begin{pgfscope}%
\pgfpathrectangle{\pgfqpoint{0.100000in}{2.413063in}}{\pgfqpoint{5.037500in}{3.427208in}}%
\pgfusepath{clip}%
\pgfsetrectcap%
\pgfsetroundjoin%
\pgfsetlinewidth{1.505625pt}%
\definecolor{currentstroke}{rgb}{0.678431,1.000000,0.184314}%
\pgfsetstrokecolor{currentstroke}%
\pgfsetstrokeopacity{0.500000}%
\pgfsetdash{}{0pt}%
\pgfpathmoveto{\pgfqpoint{4.466278in}{4.609780in}}%
\pgfusepath{stroke}%
\end{pgfscope}%
\begin{pgfscope}%
\pgfpathrectangle{\pgfqpoint{0.100000in}{2.413063in}}{\pgfqpoint{5.037500in}{3.427208in}}%
\pgfusepath{clip}%
\pgfsetbuttcap%
\pgfsetroundjoin%
\definecolor{currentfill}{rgb}{0.678431,1.000000,0.184314}%
\pgfsetfillcolor{currentfill}%
\pgfsetfillopacity{0.500000}%
\pgfsetlinewidth{0.250937pt}%
\definecolor{currentstroke}{rgb}{0.000000,0.000000,0.000000}%
\pgfsetstrokecolor{currentstroke}%
\pgfsetstrokeopacity{0.500000}%
\pgfsetdash{}{0pt}%
\pgfsys@defobject{currentmarker}{\pgfqpoint{-0.005556in}{-0.005556in}}{\pgfqpoint{0.005556in}{0.005556in}}{%
\pgfpathmoveto{\pgfqpoint{0.000000in}{-0.005556in}}%
\pgfpathcurveto{\pgfqpoint{0.001473in}{-0.005556in}}{\pgfqpoint{0.002887in}{-0.004970in}}{\pgfqpoint{0.003928in}{-0.003928in}}%
\pgfpathcurveto{\pgfqpoint{0.004970in}{-0.002887in}}{\pgfqpoint{0.005556in}{-0.001473in}}{\pgfqpoint{0.005556in}{0.000000in}}%
\pgfpathcurveto{\pgfqpoint{0.005556in}{0.001473in}}{\pgfqpoint{0.004970in}{0.002887in}}{\pgfqpoint{0.003928in}{0.003928in}}%
\pgfpathcurveto{\pgfqpoint{0.002887in}{0.004970in}}{\pgfqpoint{0.001473in}{0.005556in}}{\pgfqpoint{0.000000in}{0.005556in}}%
\pgfpathcurveto{\pgfqpoint{-0.001473in}{0.005556in}}{\pgfqpoint{-0.002887in}{0.004970in}}{\pgfqpoint{-0.003928in}{0.003928in}}%
\pgfpathcurveto{\pgfqpoint{-0.004970in}{0.002887in}}{\pgfqpoint{-0.005556in}{0.001473in}}{\pgfqpoint{-0.005556in}{0.000000in}}%
\pgfpathcurveto{\pgfqpoint{-0.005556in}{-0.001473in}}{\pgfqpoint{-0.004970in}{-0.002887in}}{\pgfqpoint{-0.003928in}{-0.003928in}}%
\pgfpathcurveto{\pgfqpoint{-0.002887in}{-0.004970in}}{\pgfqpoint{-0.001473in}{-0.005556in}}{\pgfqpoint{0.000000in}{-0.005556in}}%
\pgfpathclose%
\pgfusepath{stroke,fill}%
}%
\begin{pgfscope}%
\pgfsys@transformshift{4.466278in}{4.609780in}%
\pgfsys@useobject{currentmarker}{}%
\end{pgfscope}%
\end{pgfscope}%
\begin{pgfscope}%
\pgfpathrectangle{\pgfqpoint{0.100000in}{2.413063in}}{\pgfqpoint{5.037500in}{3.427208in}}%
\pgfusepath{clip}%
\pgfsetrectcap%
\pgfsetroundjoin%
\pgfsetlinewidth{1.505625pt}%
\definecolor{currentstroke}{rgb}{0.678431,1.000000,0.184314}%
\pgfsetstrokecolor{currentstroke}%
\pgfsetstrokeopacity{0.500000}%
\pgfsetdash{}{0pt}%
\pgfpathmoveto{\pgfqpoint{4.462981in}{4.735541in}}%
\pgfusepath{stroke}%
\end{pgfscope}%
\begin{pgfscope}%
\pgfpathrectangle{\pgfqpoint{0.100000in}{2.413063in}}{\pgfqpoint{5.037500in}{3.427208in}}%
\pgfusepath{clip}%
\pgfsetbuttcap%
\pgfsetroundjoin%
\definecolor{currentfill}{rgb}{0.678431,1.000000,0.184314}%
\pgfsetfillcolor{currentfill}%
\pgfsetfillopacity{0.500000}%
\pgfsetlinewidth{0.250937pt}%
\definecolor{currentstroke}{rgb}{0.000000,0.000000,0.000000}%
\pgfsetstrokecolor{currentstroke}%
\pgfsetstrokeopacity{0.500000}%
\pgfsetdash{}{0pt}%
\pgfsys@defobject{currentmarker}{\pgfqpoint{-0.016667in}{-0.016667in}}{\pgfqpoint{0.016667in}{0.016667in}}{%
\pgfpathmoveto{\pgfqpoint{0.000000in}{-0.016667in}}%
\pgfpathcurveto{\pgfqpoint{0.004420in}{-0.016667in}}{\pgfqpoint{0.008660in}{-0.014911in}}{\pgfqpoint{0.011785in}{-0.011785in}}%
\pgfpathcurveto{\pgfqpoint{0.014911in}{-0.008660in}}{\pgfqpoint{0.016667in}{-0.004420in}}{\pgfqpoint{0.016667in}{0.000000in}}%
\pgfpathcurveto{\pgfqpoint{0.016667in}{0.004420in}}{\pgfqpoint{0.014911in}{0.008660in}}{\pgfqpoint{0.011785in}{0.011785in}}%
\pgfpathcurveto{\pgfqpoint{0.008660in}{0.014911in}}{\pgfqpoint{0.004420in}{0.016667in}}{\pgfqpoint{0.000000in}{0.016667in}}%
\pgfpathcurveto{\pgfqpoint{-0.004420in}{0.016667in}}{\pgfqpoint{-0.008660in}{0.014911in}}{\pgfqpoint{-0.011785in}{0.011785in}}%
\pgfpathcurveto{\pgfqpoint{-0.014911in}{0.008660in}}{\pgfqpoint{-0.016667in}{0.004420in}}{\pgfqpoint{-0.016667in}{0.000000in}}%
\pgfpathcurveto{\pgfqpoint{-0.016667in}{-0.004420in}}{\pgfqpoint{-0.014911in}{-0.008660in}}{\pgfqpoint{-0.011785in}{-0.011785in}}%
\pgfpathcurveto{\pgfqpoint{-0.008660in}{-0.014911in}}{\pgfqpoint{-0.004420in}{-0.016667in}}{\pgfqpoint{0.000000in}{-0.016667in}}%
\pgfpathclose%
\pgfusepath{stroke,fill}%
}%
\begin{pgfscope}%
\pgfsys@transformshift{4.462981in}{4.735541in}%
\pgfsys@useobject{currentmarker}{}%
\end{pgfscope}%
\end{pgfscope}%
\begin{pgfscope}%
\pgfpathrectangle{\pgfqpoint{0.100000in}{2.413063in}}{\pgfqpoint{5.037500in}{3.427208in}}%
\pgfusepath{clip}%
\pgfsetrectcap%
\pgfsetroundjoin%
\pgfsetlinewidth{1.505625pt}%
\definecolor{currentstroke}{rgb}{0.678431,1.000000,0.184314}%
\pgfsetstrokecolor{currentstroke}%
\pgfsetstrokeopacity{0.500000}%
\pgfsetdash{}{0pt}%
\pgfpathmoveto{\pgfqpoint{4.290048in}{4.628094in}}%
\pgfusepath{stroke}%
\end{pgfscope}%
\begin{pgfscope}%
\pgfpathrectangle{\pgfqpoint{0.100000in}{2.413063in}}{\pgfqpoint{5.037500in}{3.427208in}}%
\pgfusepath{clip}%
\pgfsetbuttcap%
\pgfsetroundjoin%
\definecolor{currentfill}{rgb}{0.678431,1.000000,0.184314}%
\pgfsetfillcolor{currentfill}%
\pgfsetfillopacity{0.500000}%
\pgfsetlinewidth{0.250937pt}%
\definecolor{currentstroke}{rgb}{0.000000,0.000000,0.000000}%
\pgfsetstrokecolor{currentstroke}%
\pgfsetstrokeopacity{0.500000}%
\pgfsetdash{}{0pt}%
\pgfsys@defobject{currentmarker}{\pgfqpoint{-0.013889in}{-0.013889in}}{\pgfqpoint{0.013889in}{0.013889in}}{%
\pgfpathmoveto{\pgfqpoint{0.000000in}{-0.013889in}}%
\pgfpathcurveto{\pgfqpoint{0.003683in}{-0.013889in}}{\pgfqpoint{0.007216in}{-0.012425in}}{\pgfqpoint{0.009821in}{-0.009821in}}%
\pgfpathcurveto{\pgfqpoint{0.012425in}{-0.007216in}}{\pgfqpoint{0.013889in}{-0.003683in}}{\pgfqpoint{0.013889in}{0.000000in}}%
\pgfpathcurveto{\pgfqpoint{0.013889in}{0.003683in}}{\pgfqpoint{0.012425in}{0.007216in}}{\pgfqpoint{0.009821in}{0.009821in}}%
\pgfpathcurveto{\pgfqpoint{0.007216in}{0.012425in}}{\pgfqpoint{0.003683in}{0.013889in}}{\pgfqpoint{0.000000in}{0.013889in}}%
\pgfpathcurveto{\pgfqpoint{-0.003683in}{0.013889in}}{\pgfqpoint{-0.007216in}{0.012425in}}{\pgfqpoint{-0.009821in}{0.009821in}}%
\pgfpathcurveto{\pgfqpoint{-0.012425in}{0.007216in}}{\pgfqpoint{-0.013889in}{0.003683in}}{\pgfqpoint{-0.013889in}{0.000000in}}%
\pgfpathcurveto{\pgfqpoint{-0.013889in}{-0.003683in}}{\pgfqpoint{-0.012425in}{-0.007216in}}{\pgfqpoint{-0.009821in}{-0.009821in}}%
\pgfpathcurveto{\pgfqpoint{-0.007216in}{-0.012425in}}{\pgfqpoint{-0.003683in}{-0.013889in}}{\pgfqpoint{0.000000in}{-0.013889in}}%
\pgfpathclose%
\pgfusepath{stroke,fill}%
}%
\begin{pgfscope}%
\pgfsys@transformshift{4.290048in}{4.628094in}%
\pgfsys@useobject{currentmarker}{}%
\end{pgfscope}%
\end{pgfscope}%
\begin{pgfscope}%
\pgfpathrectangle{\pgfqpoint{0.100000in}{2.413063in}}{\pgfqpoint{5.037500in}{3.427208in}}%
\pgfusepath{clip}%
\pgfsetrectcap%
\pgfsetroundjoin%
\pgfsetlinewidth{1.505625pt}%
\definecolor{currentstroke}{rgb}{0.678431,1.000000,0.184314}%
\pgfsetstrokecolor{currentstroke}%
\pgfsetstrokeopacity{0.500000}%
\pgfsetdash{}{0pt}%
\pgfpathmoveto{\pgfqpoint{4.353156in}{4.693901in}}%
\pgfusepath{stroke}%
\end{pgfscope}%
\begin{pgfscope}%
\pgfpathrectangle{\pgfqpoint{0.100000in}{2.413063in}}{\pgfqpoint{5.037500in}{3.427208in}}%
\pgfusepath{clip}%
\pgfsetbuttcap%
\pgfsetroundjoin%
\definecolor{currentfill}{rgb}{0.678431,1.000000,0.184314}%
\pgfsetfillcolor{currentfill}%
\pgfsetfillopacity{0.500000}%
\pgfsetlinewidth{0.250937pt}%
\definecolor{currentstroke}{rgb}{0.000000,0.000000,0.000000}%
\pgfsetstrokecolor{currentstroke}%
\pgfsetstrokeopacity{0.500000}%
\pgfsetdash{}{0pt}%
\pgfsys@defobject{currentmarker}{\pgfqpoint{-0.030556in}{-0.030556in}}{\pgfqpoint{0.030556in}{0.030556in}}{%
\pgfpathmoveto{\pgfqpoint{0.000000in}{-0.030556in}}%
\pgfpathcurveto{\pgfqpoint{0.008103in}{-0.030556in}}{\pgfqpoint{0.015876in}{-0.027336in}}{\pgfqpoint{0.021606in}{-0.021606in}}%
\pgfpathcurveto{\pgfqpoint{0.027336in}{-0.015876in}}{\pgfqpoint{0.030556in}{-0.008103in}}{\pgfqpoint{0.030556in}{0.000000in}}%
\pgfpathcurveto{\pgfqpoint{0.030556in}{0.008103in}}{\pgfqpoint{0.027336in}{0.015876in}}{\pgfqpoint{0.021606in}{0.021606in}}%
\pgfpathcurveto{\pgfqpoint{0.015876in}{0.027336in}}{\pgfqpoint{0.008103in}{0.030556in}}{\pgfqpoint{0.000000in}{0.030556in}}%
\pgfpathcurveto{\pgfqpoint{-0.008103in}{0.030556in}}{\pgfqpoint{-0.015876in}{0.027336in}}{\pgfqpoint{-0.021606in}{0.021606in}}%
\pgfpathcurveto{\pgfqpoint{-0.027336in}{0.015876in}}{\pgfqpoint{-0.030556in}{0.008103in}}{\pgfqpoint{-0.030556in}{0.000000in}}%
\pgfpathcurveto{\pgfqpoint{-0.030556in}{-0.008103in}}{\pgfqpoint{-0.027336in}{-0.015876in}}{\pgfqpoint{-0.021606in}{-0.021606in}}%
\pgfpathcurveto{\pgfqpoint{-0.015876in}{-0.027336in}}{\pgfqpoint{-0.008103in}{-0.030556in}}{\pgfqpoint{0.000000in}{-0.030556in}}%
\pgfpathclose%
\pgfusepath{stroke,fill}%
}%
\begin{pgfscope}%
\pgfsys@transformshift{4.353156in}{4.693901in}%
\pgfsys@useobject{currentmarker}{}%
\end{pgfscope}%
\end{pgfscope}%
\begin{pgfscope}%
\pgfpathrectangle{\pgfqpoint{0.100000in}{2.413063in}}{\pgfqpoint{5.037500in}{3.427208in}}%
\pgfusepath{clip}%
\pgfsetrectcap%
\pgfsetroundjoin%
\pgfsetlinewidth{1.505625pt}%
\definecolor{currentstroke}{rgb}{0.678431,1.000000,0.184314}%
\pgfsetstrokecolor{currentstroke}%
\pgfsetstrokeopacity{0.500000}%
\pgfsetdash{}{0pt}%
\pgfpathmoveto{\pgfqpoint{4.406146in}{4.553374in}}%
\pgfusepath{stroke}%
\end{pgfscope}%
\begin{pgfscope}%
\pgfpathrectangle{\pgfqpoint{0.100000in}{2.413063in}}{\pgfqpoint{5.037500in}{3.427208in}}%
\pgfusepath{clip}%
\pgfsetbuttcap%
\pgfsetroundjoin%
\definecolor{currentfill}{rgb}{0.678431,1.000000,0.184314}%
\pgfsetfillcolor{currentfill}%
\pgfsetfillopacity{0.500000}%
\pgfsetlinewidth{0.250937pt}%
\definecolor{currentstroke}{rgb}{0.000000,0.000000,0.000000}%
\pgfsetstrokecolor{currentstroke}%
\pgfsetstrokeopacity{0.500000}%
\pgfsetdash{}{0pt}%
\pgfsys@defobject{currentmarker}{\pgfqpoint{-0.013889in}{-0.013889in}}{\pgfqpoint{0.013889in}{0.013889in}}{%
\pgfpathmoveto{\pgfqpoint{0.000000in}{-0.013889in}}%
\pgfpathcurveto{\pgfqpoint{0.003683in}{-0.013889in}}{\pgfqpoint{0.007216in}{-0.012425in}}{\pgfqpoint{0.009821in}{-0.009821in}}%
\pgfpathcurveto{\pgfqpoint{0.012425in}{-0.007216in}}{\pgfqpoint{0.013889in}{-0.003683in}}{\pgfqpoint{0.013889in}{0.000000in}}%
\pgfpathcurveto{\pgfqpoint{0.013889in}{0.003683in}}{\pgfqpoint{0.012425in}{0.007216in}}{\pgfqpoint{0.009821in}{0.009821in}}%
\pgfpathcurveto{\pgfqpoint{0.007216in}{0.012425in}}{\pgfqpoint{0.003683in}{0.013889in}}{\pgfqpoint{0.000000in}{0.013889in}}%
\pgfpathcurveto{\pgfqpoint{-0.003683in}{0.013889in}}{\pgfqpoint{-0.007216in}{0.012425in}}{\pgfqpoint{-0.009821in}{0.009821in}}%
\pgfpathcurveto{\pgfqpoint{-0.012425in}{0.007216in}}{\pgfqpoint{-0.013889in}{0.003683in}}{\pgfqpoint{-0.013889in}{0.000000in}}%
\pgfpathcurveto{\pgfqpoint{-0.013889in}{-0.003683in}}{\pgfqpoint{-0.012425in}{-0.007216in}}{\pgfqpoint{-0.009821in}{-0.009821in}}%
\pgfpathcurveto{\pgfqpoint{-0.007216in}{-0.012425in}}{\pgfqpoint{-0.003683in}{-0.013889in}}{\pgfqpoint{0.000000in}{-0.013889in}}%
\pgfpathclose%
\pgfusepath{stroke,fill}%
}%
\begin{pgfscope}%
\pgfsys@transformshift{4.406146in}{4.553374in}%
\pgfsys@useobject{currentmarker}{}%
\end{pgfscope}%
\end{pgfscope}%
\begin{pgfscope}%
\pgfpathrectangle{\pgfqpoint{0.100000in}{2.413063in}}{\pgfqpoint{5.037500in}{3.427208in}}%
\pgfusepath{clip}%
\pgfsetrectcap%
\pgfsetroundjoin%
\pgfsetlinewidth{1.505625pt}%
\definecolor{currentstroke}{rgb}{0.000000,0.000000,1.000000}%
\pgfsetstrokecolor{currentstroke}%
\pgfsetstrokeopacity{0.500000}%
\pgfsetdash{}{0pt}%
\pgfpathmoveto{\pgfqpoint{4.807749in}{4.867735in}}%
\pgfusepath{stroke}%
\end{pgfscope}%
\begin{pgfscope}%
\pgfpathrectangle{\pgfqpoint{0.100000in}{2.413063in}}{\pgfqpoint{5.037500in}{3.427208in}}%
\pgfusepath{clip}%
\pgfsetbuttcap%
\pgfsetroundjoin%
\definecolor{currentfill}{rgb}{0.000000,0.000000,1.000000}%
\pgfsetfillcolor{currentfill}%
\pgfsetfillopacity{0.500000}%
\pgfsetlinewidth{0.250937pt}%
\definecolor{currentstroke}{rgb}{0.000000,0.000000,0.000000}%
\pgfsetstrokecolor{currentstroke}%
\pgfsetstrokeopacity{0.500000}%
\pgfsetdash{}{0pt}%
\pgfsys@defobject{currentmarker}{\pgfqpoint{-0.013889in}{-0.013889in}}{\pgfqpoint{0.013889in}{0.013889in}}{%
\pgfpathmoveto{\pgfqpoint{0.000000in}{-0.013889in}}%
\pgfpathcurveto{\pgfqpoint{0.003683in}{-0.013889in}}{\pgfqpoint{0.007216in}{-0.012425in}}{\pgfqpoint{0.009821in}{-0.009821in}}%
\pgfpathcurveto{\pgfqpoint{0.012425in}{-0.007216in}}{\pgfqpoint{0.013889in}{-0.003683in}}{\pgfqpoint{0.013889in}{0.000000in}}%
\pgfpathcurveto{\pgfqpoint{0.013889in}{0.003683in}}{\pgfqpoint{0.012425in}{0.007216in}}{\pgfqpoint{0.009821in}{0.009821in}}%
\pgfpathcurveto{\pgfqpoint{0.007216in}{0.012425in}}{\pgfqpoint{0.003683in}{0.013889in}}{\pgfqpoint{0.000000in}{0.013889in}}%
\pgfpathcurveto{\pgfqpoint{-0.003683in}{0.013889in}}{\pgfqpoint{-0.007216in}{0.012425in}}{\pgfqpoint{-0.009821in}{0.009821in}}%
\pgfpathcurveto{\pgfqpoint{-0.012425in}{0.007216in}}{\pgfqpoint{-0.013889in}{0.003683in}}{\pgfqpoint{-0.013889in}{0.000000in}}%
\pgfpathcurveto{\pgfqpoint{-0.013889in}{-0.003683in}}{\pgfqpoint{-0.012425in}{-0.007216in}}{\pgfqpoint{-0.009821in}{-0.009821in}}%
\pgfpathcurveto{\pgfqpoint{-0.007216in}{-0.012425in}}{\pgfqpoint{-0.003683in}{-0.013889in}}{\pgfqpoint{0.000000in}{-0.013889in}}%
\pgfpathclose%
\pgfusepath{stroke,fill}%
}%
\begin{pgfscope}%
\pgfsys@transformshift{4.807749in}{4.867735in}%
\pgfsys@useobject{currentmarker}{}%
\end{pgfscope}%
\end{pgfscope}%
\begin{pgfscope}%
\pgfpathrectangle{\pgfqpoint{0.100000in}{2.413063in}}{\pgfqpoint{5.037500in}{3.427208in}}%
\pgfusepath{clip}%
\pgfsetrectcap%
\pgfsetroundjoin%
\pgfsetlinewidth{1.505625pt}%
\definecolor{currentstroke}{rgb}{0.000000,0.000000,1.000000}%
\pgfsetstrokecolor{currentstroke}%
\pgfsetstrokeopacity{0.500000}%
\pgfsetdash{}{0pt}%
\pgfpathmoveto{\pgfqpoint{4.263845in}{3.685403in}}%
\pgfusepath{stroke}%
\end{pgfscope}%
\begin{pgfscope}%
\pgfpathrectangle{\pgfqpoint{0.100000in}{2.413063in}}{\pgfqpoint{5.037500in}{3.427208in}}%
\pgfusepath{clip}%
\pgfsetbuttcap%
\pgfsetroundjoin%
\definecolor{currentfill}{rgb}{0.000000,0.000000,1.000000}%
\pgfsetfillcolor{currentfill}%
\pgfsetfillopacity{0.500000}%
\pgfsetlinewidth{0.250937pt}%
\definecolor{currentstroke}{rgb}{0.000000,0.000000,0.000000}%
\pgfsetstrokecolor{currentstroke}%
\pgfsetstrokeopacity{0.500000}%
\pgfsetdash{}{0pt}%
\pgfsys@defobject{currentmarker}{\pgfqpoint{-0.025000in}{-0.025000in}}{\pgfqpoint{0.025000in}{0.025000in}}{%
\pgfpathmoveto{\pgfqpoint{0.000000in}{-0.025000in}}%
\pgfpathcurveto{\pgfqpoint{0.006630in}{-0.025000in}}{\pgfqpoint{0.012989in}{-0.022366in}}{\pgfqpoint{0.017678in}{-0.017678in}}%
\pgfpathcurveto{\pgfqpoint{0.022366in}{-0.012989in}}{\pgfqpoint{0.025000in}{-0.006630in}}{\pgfqpoint{0.025000in}{0.000000in}}%
\pgfpathcurveto{\pgfqpoint{0.025000in}{0.006630in}}{\pgfqpoint{0.022366in}{0.012989in}}{\pgfqpoint{0.017678in}{0.017678in}}%
\pgfpathcurveto{\pgfqpoint{0.012989in}{0.022366in}}{\pgfqpoint{0.006630in}{0.025000in}}{\pgfqpoint{0.000000in}{0.025000in}}%
\pgfpathcurveto{\pgfqpoint{-0.006630in}{0.025000in}}{\pgfqpoint{-0.012989in}{0.022366in}}{\pgfqpoint{-0.017678in}{0.017678in}}%
\pgfpathcurveto{\pgfqpoint{-0.022366in}{0.012989in}}{\pgfqpoint{-0.025000in}{0.006630in}}{\pgfqpoint{-0.025000in}{0.000000in}}%
\pgfpathcurveto{\pgfqpoint{-0.025000in}{-0.006630in}}{\pgfqpoint{-0.022366in}{-0.012989in}}{\pgfqpoint{-0.017678in}{-0.017678in}}%
\pgfpathcurveto{\pgfqpoint{-0.012989in}{-0.022366in}}{\pgfqpoint{-0.006630in}{-0.025000in}}{\pgfqpoint{0.000000in}{-0.025000in}}%
\pgfpathclose%
\pgfusepath{stroke,fill}%
}%
\begin{pgfscope}%
\pgfsys@transformshift{4.263845in}{3.685403in}%
\pgfsys@useobject{currentmarker}{}%
\end{pgfscope}%
\end{pgfscope}%
\begin{pgfscope}%
\pgfpathrectangle{\pgfqpoint{0.100000in}{2.413063in}}{\pgfqpoint{5.037500in}{3.427208in}}%
\pgfusepath{clip}%
\pgfsetrectcap%
\pgfsetroundjoin%
\pgfsetlinewidth{1.505625pt}%
\definecolor{currentstroke}{rgb}{0.000000,0.000000,1.000000}%
\pgfsetstrokecolor{currentstroke}%
\pgfsetstrokeopacity{0.500000}%
\pgfsetdash{}{0pt}%
\pgfpathmoveto{\pgfqpoint{4.137182in}{3.807175in}}%
\pgfusepath{stroke}%
\end{pgfscope}%
\begin{pgfscope}%
\pgfpathrectangle{\pgfqpoint{0.100000in}{2.413063in}}{\pgfqpoint{5.037500in}{3.427208in}}%
\pgfusepath{clip}%
\pgfsetbuttcap%
\pgfsetroundjoin%
\definecolor{currentfill}{rgb}{0.000000,0.000000,1.000000}%
\pgfsetfillcolor{currentfill}%
\pgfsetfillopacity{0.500000}%
\pgfsetlinewidth{0.250937pt}%
\definecolor{currentstroke}{rgb}{0.000000,0.000000,0.000000}%
\pgfsetstrokecolor{currentstroke}%
\pgfsetstrokeopacity{0.500000}%
\pgfsetdash{}{0pt}%
\pgfsys@defobject{currentmarker}{\pgfqpoint{-0.025000in}{-0.025000in}}{\pgfqpoint{0.025000in}{0.025000in}}{%
\pgfpathmoveto{\pgfqpoint{0.000000in}{-0.025000in}}%
\pgfpathcurveto{\pgfqpoint{0.006630in}{-0.025000in}}{\pgfqpoint{0.012989in}{-0.022366in}}{\pgfqpoint{0.017678in}{-0.017678in}}%
\pgfpathcurveto{\pgfqpoint{0.022366in}{-0.012989in}}{\pgfqpoint{0.025000in}{-0.006630in}}{\pgfqpoint{0.025000in}{0.000000in}}%
\pgfpathcurveto{\pgfqpoint{0.025000in}{0.006630in}}{\pgfqpoint{0.022366in}{0.012989in}}{\pgfqpoint{0.017678in}{0.017678in}}%
\pgfpathcurveto{\pgfqpoint{0.012989in}{0.022366in}}{\pgfqpoint{0.006630in}{0.025000in}}{\pgfqpoint{0.000000in}{0.025000in}}%
\pgfpathcurveto{\pgfqpoint{-0.006630in}{0.025000in}}{\pgfqpoint{-0.012989in}{0.022366in}}{\pgfqpoint{-0.017678in}{0.017678in}}%
\pgfpathcurveto{\pgfqpoint{-0.022366in}{0.012989in}}{\pgfqpoint{-0.025000in}{0.006630in}}{\pgfqpoint{-0.025000in}{0.000000in}}%
\pgfpathcurveto{\pgfqpoint{-0.025000in}{-0.006630in}}{\pgfqpoint{-0.022366in}{-0.012989in}}{\pgfqpoint{-0.017678in}{-0.017678in}}%
\pgfpathcurveto{\pgfqpoint{-0.012989in}{-0.022366in}}{\pgfqpoint{-0.006630in}{-0.025000in}}{\pgfqpoint{0.000000in}{-0.025000in}}%
\pgfpathclose%
\pgfusepath{stroke,fill}%
}%
\begin{pgfscope}%
\pgfsys@transformshift{4.137182in}{3.807175in}%
\pgfsys@useobject{currentmarker}{}%
\end{pgfscope}%
\end{pgfscope}%
\begin{pgfscope}%
\pgfpathrectangle{\pgfqpoint{0.100000in}{2.413063in}}{\pgfqpoint{5.037500in}{3.427208in}}%
\pgfusepath{clip}%
\pgfsetrectcap%
\pgfsetroundjoin%
\pgfsetlinewidth{1.505625pt}%
\definecolor{currentstroke}{rgb}{0.000000,0.000000,1.000000}%
\pgfsetstrokecolor{currentstroke}%
\pgfsetstrokeopacity{0.500000}%
\pgfsetdash{}{0pt}%
\pgfpathmoveto{\pgfqpoint{4.253237in}{3.849127in}}%
\pgfusepath{stroke}%
\end{pgfscope}%
\begin{pgfscope}%
\pgfpathrectangle{\pgfqpoint{0.100000in}{2.413063in}}{\pgfqpoint{5.037500in}{3.427208in}}%
\pgfusepath{clip}%
\pgfsetbuttcap%
\pgfsetroundjoin%
\definecolor{currentfill}{rgb}{0.000000,0.000000,1.000000}%
\pgfsetfillcolor{currentfill}%
\pgfsetfillopacity{0.500000}%
\pgfsetlinewidth{0.250937pt}%
\definecolor{currentstroke}{rgb}{0.000000,0.000000,0.000000}%
\pgfsetstrokecolor{currentstroke}%
\pgfsetstrokeopacity{0.500000}%
\pgfsetdash{}{0pt}%
\pgfsys@defobject{currentmarker}{\pgfqpoint{-0.022222in}{-0.022222in}}{\pgfqpoint{0.022222in}{0.022222in}}{%
\pgfpathmoveto{\pgfqpoint{0.000000in}{-0.022222in}}%
\pgfpathcurveto{\pgfqpoint{0.005893in}{-0.022222in}}{\pgfqpoint{0.011546in}{-0.019881in}}{\pgfqpoint{0.015713in}{-0.015713in}}%
\pgfpathcurveto{\pgfqpoint{0.019881in}{-0.011546in}}{\pgfqpoint{0.022222in}{-0.005893in}}{\pgfqpoint{0.022222in}{0.000000in}}%
\pgfpathcurveto{\pgfqpoint{0.022222in}{0.005893in}}{\pgfqpoint{0.019881in}{0.011546in}}{\pgfqpoint{0.015713in}{0.015713in}}%
\pgfpathcurveto{\pgfqpoint{0.011546in}{0.019881in}}{\pgfqpoint{0.005893in}{0.022222in}}{\pgfqpoint{0.000000in}{0.022222in}}%
\pgfpathcurveto{\pgfqpoint{-0.005893in}{0.022222in}}{\pgfqpoint{-0.011546in}{0.019881in}}{\pgfqpoint{-0.015713in}{0.015713in}}%
\pgfpathcurveto{\pgfqpoint{-0.019881in}{0.011546in}}{\pgfqpoint{-0.022222in}{0.005893in}}{\pgfqpoint{-0.022222in}{0.000000in}}%
\pgfpathcurveto{\pgfqpoint{-0.022222in}{-0.005893in}}{\pgfqpoint{-0.019881in}{-0.011546in}}{\pgfqpoint{-0.015713in}{-0.015713in}}%
\pgfpathcurveto{\pgfqpoint{-0.011546in}{-0.019881in}}{\pgfqpoint{-0.005893in}{-0.022222in}}{\pgfqpoint{0.000000in}{-0.022222in}}%
\pgfpathclose%
\pgfusepath{stroke,fill}%
}%
\begin{pgfscope}%
\pgfsys@transformshift{4.253237in}{3.849127in}%
\pgfsys@useobject{currentmarker}{}%
\end{pgfscope}%
\end{pgfscope}%
\begin{pgfscope}%
\pgfpathrectangle{\pgfqpoint{0.100000in}{2.413063in}}{\pgfqpoint{5.037500in}{3.427208in}}%
\pgfusepath{clip}%
\pgfsetrectcap%
\pgfsetroundjoin%
\pgfsetlinewidth{1.505625pt}%
\definecolor{currentstroke}{rgb}{0.000000,0.000000,1.000000}%
\pgfsetstrokecolor{currentstroke}%
\pgfsetstrokeopacity{0.500000}%
\pgfsetdash{}{0pt}%
\pgfpathmoveto{\pgfqpoint{3.975994in}{3.842770in}}%
\pgfusepath{stroke}%
\end{pgfscope}%
\begin{pgfscope}%
\pgfpathrectangle{\pgfqpoint{0.100000in}{2.413063in}}{\pgfqpoint{5.037500in}{3.427208in}}%
\pgfusepath{clip}%
\pgfsetbuttcap%
\pgfsetroundjoin%
\definecolor{currentfill}{rgb}{0.000000,0.000000,1.000000}%
\pgfsetfillcolor{currentfill}%
\pgfsetfillopacity{0.500000}%
\pgfsetlinewidth{0.250937pt}%
\definecolor{currentstroke}{rgb}{0.000000,0.000000,0.000000}%
\pgfsetstrokecolor{currentstroke}%
\pgfsetstrokeopacity{0.500000}%
\pgfsetdash{}{0pt}%
\pgfsys@defobject{currentmarker}{\pgfqpoint{-0.019444in}{-0.019444in}}{\pgfqpoint{0.019444in}{0.019444in}}{%
\pgfpathmoveto{\pgfqpoint{0.000000in}{-0.019444in}}%
\pgfpathcurveto{\pgfqpoint{0.005157in}{-0.019444in}}{\pgfqpoint{0.010103in}{-0.017396in}}{\pgfqpoint{0.013749in}{-0.013749in}}%
\pgfpathcurveto{\pgfqpoint{0.017396in}{-0.010103in}}{\pgfqpoint{0.019444in}{-0.005157in}}{\pgfqpoint{0.019444in}{0.000000in}}%
\pgfpathcurveto{\pgfqpoint{0.019444in}{0.005157in}}{\pgfqpoint{0.017396in}{0.010103in}}{\pgfqpoint{0.013749in}{0.013749in}}%
\pgfpathcurveto{\pgfqpoint{0.010103in}{0.017396in}}{\pgfqpoint{0.005157in}{0.019444in}}{\pgfqpoint{0.000000in}{0.019444in}}%
\pgfpathcurveto{\pgfqpoint{-0.005157in}{0.019444in}}{\pgfqpoint{-0.010103in}{0.017396in}}{\pgfqpoint{-0.013749in}{0.013749in}}%
\pgfpathcurveto{\pgfqpoint{-0.017396in}{0.010103in}}{\pgfqpoint{-0.019444in}{0.005157in}}{\pgfqpoint{-0.019444in}{0.000000in}}%
\pgfpathcurveto{\pgfqpoint{-0.019444in}{-0.005157in}}{\pgfqpoint{-0.017396in}{-0.010103in}}{\pgfqpoint{-0.013749in}{-0.013749in}}%
\pgfpathcurveto{\pgfqpoint{-0.010103in}{-0.017396in}}{\pgfqpoint{-0.005157in}{-0.019444in}}{\pgfqpoint{0.000000in}{-0.019444in}}%
\pgfpathclose%
\pgfusepath{stroke,fill}%
}%
\begin{pgfscope}%
\pgfsys@transformshift{3.975994in}{3.842770in}%
\pgfsys@useobject{currentmarker}{}%
\end{pgfscope}%
\end{pgfscope}%
\begin{pgfscope}%
\pgfpathrectangle{\pgfqpoint{0.100000in}{2.413063in}}{\pgfqpoint{5.037500in}{3.427208in}}%
\pgfusepath{clip}%
\pgfsetrectcap%
\pgfsetroundjoin%
\pgfsetlinewidth{1.505625pt}%
\definecolor{currentstroke}{rgb}{0.000000,0.000000,1.000000}%
\pgfsetstrokecolor{currentstroke}%
\pgfsetstrokeopacity{0.500000}%
\pgfsetdash{}{0pt}%
\pgfpathmoveto{\pgfqpoint{3.994269in}{3.885455in}}%
\pgfusepath{stroke}%
\end{pgfscope}%
\begin{pgfscope}%
\pgfpathrectangle{\pgfqpoint{0.100000in}{2.413063in}}{\pgfqpoint{5.037500in}{3.427208in}}%
\pgfusepath{clip}%
\pgfsetbuttcap%
\pgfsetroundjoin%
\definecolor{currentfill}{rgb}{0.000000,0.000000,1.000000}%
\pgfsetfillcolor{currentfill}%
\pgfsetfillopacity{0.500000}%
\pgfsetlinewidth{0.250937pt}%
\definecolor{currentstroke}{rgb}{0.000000,0.000000,0.000000}%
\pgfsetstrokecolor{currentstroke}%
\pgfsetstrokeopacity{0.500000}%
\pgfsetdash{}{0pt}%
\pgfsys@defobject{currentmarker}{\pgfqpoint{-0.019444in}{-0.019444in}}{\pgfqpoint{0.019444in}{0.019444in}}{%
\pgfpathmoveto{\pgfqpoint{0.000000in}{-0.019444in}}%
\pgfpathcurveto{\pgfqpoint{0.005157in}{-0.019444in}}{\pgfqpoint{0.010103in}{-0.017396in}}{\pgfqpoint{0.013749in}{-0.013749in}}%
\pgfpathcurveto{\pgfqpoint{0.017396in}{-0.010103in}}{\pgfqpoint{0.019444in}{-0.005157in}}{\pgfqpoint{0.019444in}{0.000000in}}%
\pgfpathcurveto{\pgfqpoint{0.019444in}{0.005157in}}{\pgfqpoint{0.017396in}{0.010103in}}{\pgfqpoint{0.013749in}{0.013749in}}%
\pgfpathcurveto{\pgfqpoint{0.010103in}{0.017396in}}{\pgfqpoint{0.005157in}{0.019444in}}{\pgfqpoint{0.000000in}{0.019444in}}%
\pgfpathcurveto{\pgfqpoint{-0.005157in}{0.019444in}}{\pgfqpoint{-0.010103in}{0.017396in}}{\pgfqpoint{-0.013749in}{0.013749in}}%
\pgfpathcurveto{\pgfqpoint{-0.017396in}{0.010103in}}{\pgfqpoint{-0.019444in}{0.005157in}}{\pgfqpoint{-0.019444in}{0.000000in}}%
\pgfpathcurveto{\pgfqpoint{-0.019444in}{-0.005157in}}{\pgfqpoint{-0.017396in}{-0.010103in}}{\pgfqpoint{-0.013749in}{-0.013749in}}%
\pgfpathcurveto{\pgfqpoint{-0.010103in}{-0.017396in}}{\pgfqpoint{-0.005157in}{-0.019444in}}{\pgfqpoint{0.000000in}{-0.019444in}}%
\pgfpathclose%
\pgfusepath{stroke,fill}%
}%
\begin{pgfscope}%
\pgfsys@transformshift{3.994269in}{3.885455in}%
\pgfsys@useobject{currentmarker}{}%
\end{pgfscope}%
\end{pgfscope}%
\begin{pgfscope}%
\pgfpathrectangle{\pgfqpoint{0.100000in}{2.413063in}}{\pgfqpoint{5.037500in}{3.427208in}}%
\pgfusepath{clip}%
\pgfsetrectcap%
\pgfsetroundjoin%
\pgfsetlinewidth{1.505625pt}%
\definecolor{currentstroke}{rgb}{0.000000,0.000000,1.000000}%
\pgfsetstrokecolor{currentstroke}%
\pgfsetstrokeopacity{0.500000}%
\pgfsetdash{}{0pt}%
\pgfpathmoveto{\pgfqpoint{4.196108in}{3.607309in}}%
\pgfusepath{stroke}%
\end{pgfscope}%
\begin{pgfscope}%
\pgfpathrectangle{\pgfqpoint{0.100000in}{2.413063in}}{\pgfqpoint{5.037500in}{3.427208in}}%
\pgfusepath{clip}%
\pgfsetbuttcap%
\pgfsetroundjoin%
\definecolor{currentfill}{rgb}{0.000000,0.000000,1.000000}%
\pgfsetfillcolor{currentfill}%
\pgfsetfillopacity{0.500000}%
\pgfsetlinewidth{0.250937pt}%
\definecolor{currentstroke}{rgb}{0.000000,0.000000,0.000000}%
\pgfsetstrokecolor{currentstroke}%
\pgfsetstrokeopacity{0.500000}%
\pgfsetdash{}{0pt}%
\pgfsys@defobject{currentmarker}{\pgfqpoint{-0.019444in}{-0.019444in}}{\pgfqpoint{0.019444in}{0.019444in}}{%
\pgfpathmoveto{\pgfqpoint{0.000000in}{-0.019444in}}%
\pgfpathcurveto{\pgfqpoint{0.005157in}{-0.019444in}}{\pgfqpoint{0.010103in}{-0.017396in}}{\pgfqpoint{0.013749in}{-0.013749in}}%
\pgfpathcurveto{\pgfqpoint{0.017396in}{-0.010103in}}{\pgfqpoint{0.019444in}{-0.005157in}}{\pgfqpoint{0.019444in}{0.000000in}}%
\pgfpathcurveto{\pgfqpoint{0.019444in}{0.005157in}}{\pgfqpoint{0.017396in}{0.010103in}}{\pgfqpoint{0.013749in}{0.013749in}}%
\pgfpathcurveto{\pgfqpoint{0.010103in}{0.017396in}}{\pgfqpoint{0.005157in}{0.019444in}}{\pgfqpoint{0.000000in}{0.019444in}}%
\pgfpathcurveto{\pgfqpoint{-0.005157in}{0.019444in}}{\pgfqpoint{-0.010103in}{0.017396in}}{\pgfqpoint{-0.013749in}{0.013749in}}%
\pgfpathcurveto{\pgfqpoint{-0.017396in}{0.010103in}}{\pgfqpoint{-0.019444in}{0.005157in}}{\pgfqpoint{-0.019444in}{0.000000in}}%
\pgfpathcurveto{\pgfqpoint{-0.019444in}{-0.005157in}}{\pgfqpoint{-0.017396in}{-0.010103in}}{\pgfqpoint{-0.013749in}{-0.013749in}}%
\pgfpathcurveto{\pgfqpoint{-0.010103in}{-0.017396in}}{\pgfqpoint{-0.005157in}{-0.019444in}}{\pgfqpoint{0.000000in}{-0.019444in}}%
\pgfpathclose%
\pgfusepath{stroke,fill}%
}%
\begin{pgfscope}%
\pgfsys@transformshift{4.196108in}{3.607309in}%
\pgfsys@useobject{currentmarker}{}%
\end{pgfscope}%
\end{pgfscope}%
\begin{pgfscope}%
\pgfpathrectangle{\pgfqpoint{0.100000in}{2.413063in}}{\pgfqpoint{5.037500in}{3.427208in}}%
\pgfusepath{clip}%
\pgfsetrectcap%
\pgfsetroundjoin%
\pgfsetlinewidth{1.505625pt}%
\definecolor{currentstroke}{rgb}{0.000000,0.000000,1.000000}%
\pgfsetstrokecolor{currentstroke}%
\pgfsetstrokeopacity{0.500000}%
\pgfsetdash{}{0pt}%
\pgfpathmoveto{\pgfqpoint{4.346076in}{3.806299in}}%
\pgfusepath{stroke}%
\end{pgfscope}%
\begin{pgfscope}%
\pgfpathrectangle{\pgfqpoint{0.100000in}{2.413063in}}{\pgfqpoint{5.037500in}{3.427208in}}%
\pgfusepath{clip}%
\pgfsetbuttcap%
\pgfsetroundjoin%
\definecolor{currentfill}{rgb}{0.000000,0.000000,1.000000}%
\pgfsetfillcolor{currentfill}%
\pgfsetfillopacity{0.500000}%
\pgfsetlinewidth{0.250937pt}%
\definecolor{currentstroke}{rgb}{0.000000,0.000000,0.000000}%
\pgfsetstrokecolor{currentstroke}%
\pgfsetstrokeopacity{0.500000}%
\pgfsetdash{}{0pt}%
\pgfsys@defobject{currentmarker}{\pgfqpoint{-0.016667in}{-0.016667in}}{\pgfqpoint{0.016667in}{0.016667in}}{%
\pgfpathmoveto{\pgfqpoint{0.000000in}{-0.016667in}}%
\pgfpathcurveto{\pgfqpoint{0.004420in}{-0.016667in}}{\pgfqpoint{0.008660in}{-0.014911in}}{\pgfqpoint{0.011785in}{-0.011785in}}%
\pgfpathcurveto{\pgfqpoint{0.014911in}{-0.008660in}}{\pgfqpoint{0.016667in}{-0.004420in}}{\pgfqpoint{0.016667in}{0.000000in}}%
\pgfpathcurveto{\pgfqpoint{0.016667in}{0.004420in}}{\pgfqpoint{0.014911in}{0.008660in}}{\pgfqpoint{0.011785in}{0.011785in}}%
\pgfpathcurveto{\pgfqpoint{0.008660in}{0.014911in}}{\pgfqpoint{0.004420in}{0.016667in}}{\pgfqpoint{0.000000in}{0.016667in}}%
\pgfpathcurveto{\pgfqpoint{-0.004420in}{0.016667in}}{\pgfqpoint{-0.008660in}{0.014911in}}{\pgfqpoint{-0.011785in}{0.011785in}}%
\pgfpathcurveto{\pgfqpoint{-0.014911in}{0.008660in}}{\pgfqpoint{-0.016667in}{0.004420in}}{\pgfqpoint{-0.016667in}{0.000000in}}%
\pgfpathcurveto{\pgfqpoint{-0.016667in}{-0.004420in}}{\pgfqpoint{-0.014911in}{-0.008660in}}{\pgfqpoint{-0.011785in}{-0.011785in}}%
\pgfpathcurveto{\pgfqpoint{-0.008660in}{-0.014911in}}{\pgfqpoint{-0.004420in}{-0.016667in}}{\pgfqpoint{0.000000in}{-0.016667in}}%
\pgfpathclose%
\pgfusepath{stroke,fill}%
}%
\begin{pgfscope}%
\pgfsys@transformshift{4.346076in}{3.806299in}%
\pgfsys@useobject{currentmarker}{}%
\end{pgfscope}%
\end{pgfscope}%
\begin{pgfscope}%
\pgfpathrectangle{\pgfqpoint{0.100000in}{2.413063in}}{\pgfqpoint{5.037500in}{3.427208in}}%
\pgfusepath{clip}%
\pgfsetrectcap%
\pgfsetroundjoin%
\pgfsetlinewidth{1.505625pt}%
\definecolor{currentstroke}{rgb}{0.000000,0.000000,1.000000}%
\pgfsetstrokecolor{currentstroke}%
\pgfsetstrokeopacity{0.500000}%
\pgfsetdash{}{0pt}%
\pgfpathmoveto{\pgfqpoint{4.036398in}{3.902925in}}%
\pgfusepath{stroke}%
\end{pgfscope}%
\begin{pgfscope}%
\pgfpathrectangle{\pgfqpoint{0.100000in}{2.413063in}}{\pgfqpoint{5.037500in}{3.427208in}}%
\pgfusepath{clip}%
\pgfsetbuttcap%
\pgfsetroundjoin%
\definecolor{currentfill}{rgb}{0.000000,0.000000,1.000000}%
\pgfsetfillcolor{currentfill}%
\pgfsetfillopacity{0.500000}%
\pgfsetlinewidth{0.250937pt}%
\definecolor{currentstroke}{rgb}{0.000000,0.000000,0.000000}%
\pgfsetstrokecolor{currentstroke}%
\pgfsetstrokeopacity{0.500000}%
\pgfsetdash{}{0pt}%
\pgfsys@defobject{currentmarker}{\pgfqpoint{-0.030556in}{-0.030556in}}{\pgfqpoint{0.030556in}{0.030556in}}{%
\pgfpathmoveto{\pgfqpoint{0.000000in}{-0.030556in}}%
\pgfpathcurveto{\pgfqpoint{0.008103in}{-0.030556in}}{\pgfqpoint{0.015876in}{-0.027336in}}{\pgfqpoint{0.021606in}{-0.021606in}}%
\pgfpathcurveto{\pgfqpoint{0.027336in}{-0.015876in}}{\pgfqpoint{0.030556in}{-0.008103in}}{\pgfqpoint{0.030556in}{0.000000in}}%
\pgfpathcurveto{\pgfqpoint{0.030556in}{0.008103in}}{\pgfqpoint{0.027336in}{0.015876in}}{\pgfqpoint{0.021606in}{0.021606in}}%
\pgfpathcurveto{\pgfqpoint{0.015876in}{0.027336in}}{\pgfqpoint{0.008103in}{0.030556in}}{\pgfqpoint{0.000000in}{0.030556in}}%
\pgfpathcurveto{\pgfqpoint{-0.008103in}{0.030556in}}{\pgfqpoint{-0.015876in}{0.027336in}}{\pgfqpoint{-0.021606in}{0.021606in}}%
\pgfpathcurveto{\pgfqpoint{-0.027336in}{0.015876in}}{\pgfqpoint{-0.030556in}{0.008103in}}{\pgfqpoint{-0.030556in}{0.000000in}}%
\pgfpathcurveto{\pgfqpoint{-0.030556in}{-0.008103in}}{\pgfqpoint{-0.027336in}{-0.015876in}}{\pgfqpoint{-0.021606in}{-0.021606in}}%
\pgfpathcurveto{\pgfqpoint{-0.015876in}{-0.027336in}}{\pgfqpoint{-0.008103in}{-0.030556in}}{\pgfqpoint{0.000000in}{-0.030556in}}%
\pgfpathclose%
\pgfusepath{stroke,fill}%
}%
\begin{pgfscope}%
\pgfsys@transformshift{4.036398in}{3.902925in}%
\pgfsys@useobject{currentmarker}{}%
\end{pgfscope}%
\end{pgfscope}%
\begin{pgfscope}%
\pgfpathrectangle{\pgfqpoint{0.100000in}{2.413063in}}{\pgfqpoint{5.037500in}{3.427208in}}%
\pgfusepath{clip}%
\pgfsetrectcap%
\pgfsetroundjoin%
\pgfsetlinewidth{1.505625pt}%
\definecolor{currentstroke}{rgb}{0.000000,0.000000,1.000000}%
\pgfsetstrokecolor{currentstroke}%
\pgfsetstrokeopacity{0.500000}%
\pgfsetdash{}{0pt}%
\pgfpathmoveto{\pgfqpoint{4.204247in}{3.808425in}}%
\pgfusepath{stroke}%
\end{pgfscope}%
\begin{pgfscope}%
\pgfpathrectangle{\pgfqpoint{0.100000in}{2.413063in}}{\pgfqpoint{5.037500in}{3.427208in}}%
\pgfusepath{clip}%
\pgfsetbuttcap%
\pgfsetroundjoin%
\definecolor{currentfill}{rgb}{0.000000,0.000000,1.000000}%
\pgfsetfillcolor{currentfill}%
\pgfsetfillopacity{0.500000}%
\pgfsetlinewidth{0.250937pt}%
\definecolor{currentstroke}{rgb}{0.000000,0.000000,0.000000}%
\pgfsetstrokecolor{currentstroke}%
\pgfsetstrokeopacity{0.500000}%
\pgfsetdash{}{0pt}%
\pgfsys@defobject{currentmarker}{\pgfqpoint{-0.016667in}{-0.016667in}}{\pgfqpoint{0.016667in}{0.016667in}}{%
\pgfpathmoveto{\pgfqpoint{0.000000in}{-0.016667in}}%
\pgfpathcurveto{\pgfqpoint{0.004420in}{-0.016667in}}{\pgfqpoint{0.008660in}{-0.014911in}}{\pgfqpoint{0.011785in}{-0.011785in}}%
\pgfpathcurveto{\pgfqpoint{0.014911in}{-0.008660in}}{\pgfqpoint{0.016667in}{-0.004420in}}{\pgfqpoint{0.016667in}{0.000000in}}%
\pgfpathcurveto{\pgfqpoint{0.016667in}{0.004420in}}{\pgfqpoint{0.014911in}{0.008660in}}{\pgfqpoint{0.011785in}{0.011785in}}%
\pgfpathcurveto{\pgfqpoint{0.008660in}{0.014911in}}{\pgfqpoint{0.004420in}{0.016667in}}{\pgfqpoint{0.000000in}{0.016667in}}%
\pgfpathcurveto{\pgfqpoint{-0.004420in}{0.016667in}}{\pgfqpoint{-0.008660in}{0.014911in}}{\pgfqpoint{-0.011785in}{0.011785in}}%
\pgfpathcurveto{\pgfqpoint{-0.014911in}{0.008660in}}{\pgfqpoint{-0.016667in}{0.004420in}}{\pgfqpoint{-0.016667in}{0.000000in}}%
\pgfpathcurveto{\pgfqpoint{-0.016667in}{-0.004420in}}{\pgfqpoint{-0.014911in}{-0.008660in}}{\pgfqpoint{-0.011785in}{-0.011785in}}%
\pgfpathcurveto{\pgfqpoint{-0.008660in}{-0.014911in}}{\pgfqpoint{-0.004420in}{-0.016667in}}{\pgfqpoint{0.000000in}{-0.016667in}}%
\pgfpathclose%
\pgfusepath{stroke,fill}%
}%
\begin{pgfscope}%
\pgfsys@transformshift{4.204247in}{3.808425in}%
\pgfsys@useobject{currentmarker}{}%
\end{pgfscope}%
\end{pgfscope}%
\begin{pgfscope}%
\pgfpathrectangle{\pgfqpoint{0.100000in}{2.413063in}}{\pgfqpoint{5.037500in}{3.427208in}}%
\pgfusepath{clip}%
\pgfsetrectcap%
\pgfsetroundjoin%
\pgfsetlinewidth{1.505625pt}%
\definecolor{currentstroke}{rgb}{0.678431,1.000000,0.184314}%
\pgfsetstrokecolor{currentstroke}%
\pgfsetstrokeopacity{0.500000}%
\pgfsetdash{}{0pt}%
\pgfpathmoveto{\pgfqpoint{2.121809in}{4.897686in}}%
\pgfusepath{stroke}%
\end{pgfscope}%
\begin{pgfscope}%
\pgfpathrectangle{\pgfqpoint{0.100000in}{2.413063in}}{\pgfqpoint{5.037500in}{3.427208in}}%
\pgfusepath{clip}%
\pgfsetbuttcap%
\pgfsetroundjoin%
\definecolor{currentfill}{rgb}{0.678431,1.000000,0.184314}%
\pgfsetfillcolor{currentfill}%
\pgfsetfillopacity{0.500000}%
\pgfsetlinewidth{0.250937pt}%
\definecolor{currentstroke}{rgb}{0.000000,0.000000,0.000000}%
\pgfsetstrokecolor{currentstroke}%
\pgfsetstrokeopacity{0.500000}%
\pgfsetdash{}{0pt}%
\pgfsys@defobject{currentmarker}{\pgfqpoint{-0.022222in}{-0.022222in}}{\pgfqpoint{0.022222in}{0.022222in}}{%
\pgfpathmoveto{\pgfqpoint{0.000000in}{-0.022222in}}%
\pgfpathcurveto{\pgfqpoint{0.005893in}{-0.022222in}}{\pgfqpoint{0.011546in}{-0.019881in}}{\pgfqpoint{0.015713in}{-0.015713in}}%
\pgfpathcurveto{\pgfqpoint{0.019881in}{-0.011546in}}{\pgfqpoint{0.022222in}{-0.005893in}}{\pgfqpoint{0.022222in}{0.000000in}}%
\pgfpathcurveto{\pgfqpoint{0.022222in}{0.005893in}}{\pgfqpoint{0.019881in}{0.011546in}}{\pgfqpoint{0.015713in}{0.015713in}}%
\pgfpathcurveto{\pgfqpoint{0.011546in}{0.019881in}}{\pgfqpoint{0.005893in}{0.022222in}}{\pgfqpoint{0.000000in}{0.022222in}}%
\pgfpathcurveto{\pgfqpoint{-0.005893in}{0.022222in}}{\pgfqpoint{-0.011546in}{0.019881in}}{\pgfqpoint{-0.015713in}{0.015713in}}%
\pgfpathcurveto{\pgfqpoint{-0.019881in}{0.011546in}}{\pgfqpoint{-0.022222in}{0.005893in}}{\pgfqpoint{-0.022222in}{0.000000in}}%
\pgfpathcurveto{\pgfqpoint{-0.022222in}{-0.005893in}}{\pgfqpoint{-0.019881in}{-0.011546in}}{\pgfqpoint{-0.015713in}{-0.015713in}}%
\pgfpathcurveto{\pgfqpoint{-0.011546in}{-0.019881in}}{\pgfqpoint{-0.005893in}{-0.022222in}}{\pgfqpoint{0.000000in}{-0.022222in}}%
\pgfpathclose%
\pgfusepath{stroke,fill}%
}%
\begin{pgfscope}%
\pgfsys@transformshift{2.121809in}{4.897686in}%
\pgfsys@useobject{currentmarker}{}%
\end{pgfscope}%
\end{pgfscope}%
\begin{pgfscope}%
\pgfpathrectangle{\pgfqpoint{0.100000in}{2.413063in}}{\pgfqpoint{5.037500in}{3.427208in}}%
\pgfusepath{clip}%
\pgfsetrectcap%
\pgfsetroundjoin%
\pgfsetlinewidth{1.505625pt}%
\definecolor{currentstroke}{rgb}{0.678431,1.000000,0.184314}%
\pgfsetstrokecolor{currentstroke}%
\pgfsetstrokeopacity{0.500000}%
\pgfsetdash{}{0pt}%
\pgfpathmoveto{\pgfqpoint{2.661793in}{4.806727in}}%
\pgfusepath{stroke}%
\end{pgfscope}%
\begin{pgfscope}%
\pgfpathrectangle{\pgfqpoint{0.100000in}{2.413063in}}{\pgfqpoint{5.037500in}{3.427208in}}%
\pgfusepath{clip}%
\pgfsetbuttcap%
\pgfsetroundjoin%
\definecolor{currentfill}{rgb}{0.678431,1.000000,0.184314}%
\pgfsetfillcolor{currentfill}%
\pgfsetfillopacity{0.500000}%
\pgfsetlinewidth{0.250937pt}%
\definecolor{currentstroke}{rgb}{0.000000,0.000000,0.000000}%
\pgfsetstrokecolor{currentstroke}%
\pgfsetstrokeopacity{0.500000}%
\pgfsetdash{}{0pt}%
\pgfsys@defobject{currentmarker}{\pgfqpoint{-0.016667in}{-0.016667in}}{\pgfqpoint{0.016667in}{0.016667in}}{%
\pgfpathmoveto{\pgfqpoint{0.000000in}{-0.016667in}}%
\pgfpathcurveto{\pgfqpoint{0.004420in}{-0.016667in}}{\pgfqpoint{0.008660in}{-0.014911in}}{\pgfqpoint{0.011785in}{-0.011785in}}%
\pgfpathcurveto{\pgfqpoint{0.014911in}{-0.008660in}}{\pgfqpoint{0.016667in}{-0.004420in}}{\pgfqpoint{0.016667in}{0.000000in}}%
\pgfpathcurveto{\pgfqpoint{0.016667in}{0.004420in}}{\pgfqpoint{0.014911in}{0.008660in}}{\pgfqpoint{0.011785in}{0.011785in}}%
\pgfpathcurveto{\pgfqpoint{0.008660in}{0.014911in}}{\pgfqpoint{0.004420in}{0.016667in}}{\pgfqpoint{0.000000in}{0.016667in}}%
\pgfpathcurveto{\pgfqpoint{-0.004420in}{0.016667in}}{\pgfqpoint{-0.008660in}{0.014911in}}{\pgfqpoint{-0.011785in}{0.011785in}}%
\pgfpathcurveto{\pgfqpoint{-0.014911in}{0.008660in}}{\pgfqpoint{-0.016667in}{0.004420in}}{\pgfqpoint{-0.016667in}{0.000000in}}%
\pgfpathcurveto{\pgfqpoint{-0.016667in}{-0.004420in}}{\pgfqpoint{-0.014911in}{-0.008660in}}{\pgfqpoint{-0.011785in}{-0.011785in}}%
\pgfpathcurveto{\pgfqpoint{-0.008660in}{-0.014911in}}{\pgfqpoint{-0.004420in}{-0.016667in}}{\pgfqpoint{0.000000in}{-0.016667in}}%
\pgfpathclose%
\pgfusepath{stroke,fill}%
}%
\begin{pgfscope}%
\pgfsys@transformshift{2.661793in}{4.806727in}%
\pgfsys@useobject{currentmarker}{}%
\end{pgfscope}%
\end{pgfscope}%
\begin{pgfscope}%
\pgfpathrectangle{\pgfqpoint{0.100000in}{2.413063in}}{\pgfqpoint{5.037500in}{3.427208in}}%
\pgfusepath{clip}%
\pgfsetrectcap%
\pgfsetroundjoin%
\pgfsetlinewidth{1.505625pt}%
\definecolor{currentstroke}{rgb}{0.678431,1.000000,0.184314}%
\pgfsetstrokecolor{currentstroke}%
\pgfsetstrokeopacity{0.500000}%
\pgfsetdash{}{0pt}%
\pgfpathmoveto{\pgfqpoint{3.718137in}{3.874012in}}%
\pgfusepath{stroke}%
\end{pgfscope}%
\begin{pgfscope}%
\pgfpathrectangle{\pgfqpoint{0.100000in}{2.413063in}}{\pgfqpoint{5.037500in}{3.427208in}}%
\pgfusepath{clip}%
\pgfsetbuttcap%
\pgfsetroundjoin%
\definecolor{currentfill}{rgb}{0.678431,1.000000,0.184314}%
\pgfsetfillcolor{currentfill}%
\pgfsetfillopacity{0.500000}%
\pgfsetlinewidth{0.250937pt}%
\definecolor{currentstroke}{rgb}{0.000000,0.000000,0.000000}%
\pgfsetstrokecolor{currentstroke}%
\pgfsetstrokeopacity{0.500000}%
\pgfsetdash{}{0pt}%
\pgfsys@defobject{currentmarker}{\pgfqpoint{-0.013889in}{-0.013889in}}{\pgfqpoint{0.013889in}{0.013889in}}{%
\pgfpathmoveto{\pgfqpoint{0.000000in}{-0.013889in}}%
\pgfpathcurveto{\pgfqpoint{0.003683in}{-0.013889in}}{\pgfqpoint{0.007216in}{-0.012425in}}{\pgfqpoint{0.009821in}{-0.009821in}}%
\pgfpathcurveto{\pgfqpoint{0.012425in}{-0.007216in}}{\pgfqpoint{0.013889in}{-0.003683in}}{\pgfqpoint{0.013889in}{0.000000in}}%
\pgfpathcurveto{\pgfqpoint{0.013889in}{0.003683in}}{\pgfqpoint{0.012425in}{0.007216in}}{\pgfqpoint{0.009821in}{0.009821in}}%
\pgfpathcurveto{\pgfqpoint{0.007216in}{0.012425in}}{\pgfqpoint{0.003683in}{0.013889in}}{\pgfqpoint{0.000000in}{0.013889in}}%
\pgfpathcurveto{\pgfqpoint{-0.003683in}{0.013889in}}{\pgfqpoint{-0.007216in}{0.012425in}}{\pgfqpoint{-0.009821in}{0.009821in}}%
\pgfpathcurveto{\pgfqpoint{-0.012425in}{0.007216in}}{\pgfqpoint{-0.013889in}{0.003683in}}{\pgfqpoint{-0.013889in}{0.000000in}}%
\pgfpathcurveto{\pgfqpoint{-0.013889in}{-0.003683in}}{\pgfqpoint{-0.012425in}{-0.007216in}}{\pgfqpoint{-0.009821in}{-0.009821in}}%
\pgfpathcurveto{\pgfqpoint{-0.007216in}{-0.012425in}}{\pgfqpoint{-0.003683in}{-0.013889in}}{\pgfqpoint{0.000000in}{-0.013889in}}%
\pgfpathclose%
\pgfusepath{stroke,fill}%
}%
\begin{pgfscope}%
\pgfsys@transformshift{3.718137in}{3.874012in}%
\pgfsys@useobject{currentmarker}{}%
\end{pgfscope}%
\end{pgfscope}%
\begin{pgfscope}%
\pgfpathrectangle{\pgfqpoint{0.100000in}{2.413063in}}{\pgfqpoint{5.037500in}{3.427208in}}%
\pgfusepath{clip}%
\pgfsetrectcap%
\pgfsetroundjoin%
\pgfsetlinewidth{1.505625pt}%
\definecolor{currentstroke}{rgb}{0.501961,0.501961,0.501961}%
\pgfsetstrokecolor{currentstroke}%
\pgfsetstrokeopacity{0.500000}%
\pgfsetdash{}{0pt}%
\pgfpathmoveto{\pgfqpoint{3.511014in}{4.025917in}}%
\pgfusepath{stroke}%
\end{pgfscope}%
\begin{pgfscope}%
\pgfpathrectangle{\pgfqpoint{0.100000in}{2.413063in}}{\pgfqpoint{5.037500in}{3.427208in}}%
\pgfusepath{clip}%
\pgfsetbuttcap%
\pgfsetroundjoin%
\definecolor{currentfill}{rgb}{0.501961,0.501961,0.501961}%
\pgfsetfillcolor{currentfill}%
\pgfsetfillopacity{0.500000}%
\pgfsetlinewidth{0.250937pt}%
\definecolor{currentstroke}{rgb}{0.000000,0.000000,0.000000}%
\pgfsetstrokecolor{currentstroke}%
\pgfsetstrokeopacity{0.500000}%
\pgfsetdash{}{0pt}%
\pgfsys@defobject{currentmarker}{\pgfqpoint{-0.013889in}{-0.013889in}}{\pgfqpoint{0.013889in}{0.013889in}}{%
\pgfpathmoveto{\pgfqpoint{0.000000in}{-0.013889in}}%
\pgfpathcurveto{\pgfqpoint{0.003683in}{-0.013889in}}{\pgfqpoint{0.007216in}{-0.012425in}}{\pgfqpoint{0.009821in}{-0.009821in}}%
\pgfpathcurveto{\pgfqpoint{0.012425in}{-0.007216in}}{\pgfqpoint{0.013889in}{-0.003683in}}{\pgfqpoint{0.013889in}{0.000000in}}%
\pgfpathcurveto{\pgfqpoint{0.013889in}{0.003683in}}{\pgfqpoint{0.012425in}{0.007216in}}{\pgfqpoint{0.009821in}{0.009821in}}%
\pgfpathcurveto{\pgfqpoint{0.007216in}{0.012425in}}{\pgfqpoint{0.003683in}{0.013889in}}{\pgfqpoint{0.000000in}{0.013889in}}%
\pgfpathcurveto{\pgfqpoint{-0.003683in}{0.013889in}}{\pgfqpoint{-0.007216in}{0.012425in}}{\pgfqpoint{-0.009821in}{0.009821in}}%
\pgfpathcurveto{\pgfqpoint{-0.012425in}{0.007216in}}{\pgfqpoint{-0.013889in}{0.003683in}}{\pgfqpoint{-0.013889in}{0.000000in}}%
\pgfpathcurveto{\pgfqpoint{-0.013889in}{-0.003683in}}{\pgfqpoint{-0.012425in}{-0.007216in}}{\pgfqpoint{-0.009821in}{-0.009821in}}%
\pgfpathcurveto{\pgfqpoint{-0.007216in}{-0.012425in}}{\pgfqpoint{-0.003683in}{-0.013889in}}{\pgfqpoint{0.000000in}{-0.013889in}}%
\pgfpathclose%
\pgfusepath{stroke,fill}%
}%
\begin{pgfscope}%
\pgfsys@transformshift{3.511014in}{4.025917in}%
\pgfsys@useobject{currentmarker}{}%
\end{pgfscope}%
\end{pgfscope}%
\begin{pgfscope}%
\pgfpathrectangle{\pgfqpoint{0.100000in}{2.413063in}}{\pgfqpoint{5.037500in}{3.427208in}}%
\pgfusepath{clip}%
\pgfsetrectcap%
\pgfsetroundjoin%
\pgfsetlinewidth{1.505625pt}%
\definecolor{currentstroke}{rgb}{0.501961,0.501961,0.501961}%
\pgfsetstrokecolor{currentstroke}%
\pgfsetstrokeopacity{0.500000}%
\pgfsetdash{}{0pt}%
\pgfpathmoveto{\pgfqpoint{3.757359in}{3.891528in}}%
\pgfusepath{stroke}%
\end{pgfscope}%
\begin{pgfscope}%
\pgfpathrectangle{\pgfqpoint{0.100000in}{2.413063in}}{\pgfqpoint{5.037500in}{3.427208in}}%
\pgfusepath{clip}%
\pgfsetbuttcap%
\pgfsetroundjoin%
\definecolor{currentfill}{rgb}{0.501961,0.501961,0.501961}%
\pgfsetfillcolor{currentfill}%
\pgfsetfillopacity{0.500000}%
\pgfsetlinewidth{0.250937pt}%
\definecolor{currentstroke}{rgb}{0.000000,0.000000,0.000000}%
\pgfsetstrokecolor{currentstroke}%
\pgfsetstrokeopacity{0.500000}%
\pgfsetdash{}{0pt}%
\pgfsys@defobject{currentmarker}{\pgfqpoint{-0.013889in}{-0.013889in}}{\pgfqpoint{0.013889in}{0.013889in}}{%
\pgfpathmoveto{\pgfqpoint{0.000000in}{-0.013889in}}%
\pgfpathcurveto{\pgfqpoint{0.003683in}{-0.013889in}}{\pgfqpoint{0.007216in}{-0.012425in}}{\pgfqpoint{0.009821in}{-0.009821in}}%
\pgfpathcurveto{\pgfqpoint{0.012425in}{-0.007216in}}{\pgfqpoint{0.013889in}{-0.003683in}}{\pgfqpoint{0.013889in}{0.000000in}}%
\pgfpathcurveto{\pgfqpoint{0.013889in}{0.003683in}}{\pgfqpoint{0.012425in}{0.007216in}}{\pgfqpoint{0.009821in}{0.009821in}}%
\pgfpathcurveto{\pgfqpoint{0.007216in}{0.012425in}}{\pgfqpoint{0.003683in}{0.013889in}}{\pgfqpoint{0.000000in}{0.013889in}}%
\pgfpathcurveto{\pgfqpoint{-0.003683in}{0.013889in}}{\pgfqpoint{-0.007216in}{0.012425in}}{\pgfqpoint{-0.009821in}{0.009821in}}%
\pgfpathcurveto{\pgfqpoint{-0.012425in}{0.007216in}}{\pgfqpoint{-0.013889in}{0.003683in}}{\pgfqpoint{-0.013889in}{0.000000in}}%
\pgfpathcurveto{\pgfqpoint{-0.013889in}{-0.003683in}}{\pgfqpoint{-0.012425in}{-0.007216in}}{\pgfqpoint{-0.009821in}{-0.009821in}}%
\pgfpathcurveto{\pgfqpoint{-0.007216in}{-0.012425in}}{\pgfqpoint{-0.003683in}{-0.013889in}}{\pgfqpoint{0.000000in}{-0.013889in}}%
\pgfpathclose%
\pgfusepath{stroke,fill}%
}%
\begin{pgfscope}%
\pgfsys@transformshift{3.757359in}{3.891528in}%
\pgfsys@useobject{currentmarker}{}%
\end{pgfscope}%
\end{pgfscope}%
\begin{pgfscope}%
\pgfpathrectangle{\pgfqpoint{0.100000in}{2.413063in}}{\pgfqpoint{5.037500in}{3.427208in}}%
\pgfusepath{clip}%
\pgfsetrectcap%
\pgfsetroundjoin%
\pgfsetlinewidth{1.505625pt}%
\definecolor{currentstroke}{rgb}{0.501961,0.501961,0.501961}%
\pgfsetstrokecolor{currentstroke}%
\pgfsetstrokeopacity{0.500000}%
\pgfsetdash{}{0pt}%
\pgfpathmoveto{\pgfqpoint{3.383445in}{3.910583in}}%
\pgfusepath{stroke}%
\end{pgfscope}%
\begin{pgfscope}%
\pgfpathrectangle{\pgfqpoint{0.100000in}{2.413063in}}{\pgfqpoint{5.037500in}{3.427208in}}%
\pgfusepath{clip}%
\pgfsetbuttcap%
\pgfsetroundjoin%
\definecolor{currentfill}{rgb}{0.501961,0.501961,0.501961}%
\pgfsetfillcolor{currentfill}%
\pgfsetfillopacity{0.500000}%
\pgfsetlinewidth{0.250937pt}%
\definecolor{currentstroke}{rgb}{0.000000,0.000000,0.000000}%
\pgfsetstrokecolor{currentstroke}%
\pgfsetstrokeopacity{0.500000}%
\pgfsetdash{}{0pt}%
\pgfsys@defobject{currentmarker}{\pgfqpoint{-0.013889in}{-0.013889in}}{\pgfqpoint{0.013889in}{0.013889in}}{%
\pgfpathmoveto{\pgfqpoint{0.000000in}{-0.013889in}}%
\pgfpathcurveto{\pgfqpoint{0.003683in}{-0.013889in}}{\pgfqpoint{0.007216in}{-0.012425in}}{\pgfqpoint{0.009821in}{-0.009821in}}%
\pgfpathcurveto{\pgfqpoint{0.012425in}{-0.007216in}}{\pgfqpoint{0.013889in}{-0.003683in}}{\pgfqpoint{0.013889in}{0.000000in}}%
\pgfpathcurveto{\pgfqpoint{0.013889in}{0.003683in}}{\pgfqpoint{0.012425in}{0.007216in}}{\pgfqpoint{0.009821in}{0.009821in}}%
\pgfpathcurveto{\pgfqpoint{0.007216in}{0.012425in}}{\pgfqpoint{0.003683in}{0.013889in}}{\pgfqpoint{0.000000in}{0.013889in}}%
\pgfpathcurveto{\pgfqpoint{-0.003683in}{0.013889in}}{\pgfqpoint{-0.007216in}{0.012425in}}{\pgfqpoint{-0.009821in}{0.009821in}}%
\pgfpathcurveto{\pgfqpoint{-0.012425in}{0.007216in}}{\pgfqpoint{-0.013889in}{0.003683in}}{\pgfqpoint{-0.013889in}{0.000000in}}%
\pgfpathcurveto{\pgfqpoint{-0.013889in}{-0.003683in}}{\pgfqpoint{-0.012425in}{-0.007216in}}{\pgfqpoint{-0.009821in}{-0.009821in}}%
\pgfpathcurveto{\pgfqpoint{-0.007216in}{-0.012425in}}{\pgfqpoint{-0.003683in}{-0.013889in}}{\pgfqpoint{0.000000in}{-0.013889in}}%
\pgfpathclose%
\pgfusepath{stroke,fill}%
}%
\begin{pgfscope}%
\pgfsys@transformshift{3.383445in}{3.910583in}%
\pgfsys@useobject{currentmarker}{}%
\end{pgfscope}%
\end{pgfscope}%
\begin{pgfscope}%
\pgfpathrectangle{\pgfqpoint{0.100000in}{2.413063in}}{\pgfqpoint{5.037500in}{3.427208in}}%
\pgfusepath{clip}%
\pgfsetrectcap%
\pgfsetroundjoin%
\pgfsetlinewidth{1.505625pt}%
\definecolor{currentstroke}{rgb}{0.678431,1.000000,0.184314}%
\pgfsetstrokecolor{currentstroke}%
\pgfsetstrokeopacity{0.500000}%
\pgfsetdash{}{0pt}%
\pgfpathmoveto{\pgfqpoint{3.975097in}{4.053085in}}%
\pgfusepath{stroke}%
\end{pgfscope}%
\begin{pgfscope}%
\pgfpathrectangle{\pgfqpoint{0.100000in}{2.413063in}}{\pgfqpoint{5.037500in}{3.427208in}}%
\pgfusepath{clip}%
\pgfsetbuttcap%
\pgfsetroundjoin%
\definecolor{currentfill}{rgb}{0.678431,1.000000,0.184314}%
\pgfsetfillcolor{currentfill}%
\pgfsetfillopacity{0.500000}%
\pgfsetlinewidth{0.250937pt}%
\definecolor{currentstroke}{rgb}{0.000000,0.000000,0.000000}%
\pgfsetstrokecolor{currentstroke}%
\pgfsetstrokeopacity{0.500000}%
\pgfsetdash{}{0pt}%
\pgfsys@defobject{currentmarker}{\pgfqpoint{-0.005556in}{-0.005556in}}{\pgfqpoint{0.005556in}{0.005556in}}{%
\pgfpathmoveto{\pgfqpoint{0.000000in}{-0.005556in}}%
\pgfpathcurveto{\pgfqpoint{0.001473in}{-0.005556in}}{\pgfqpoint{0.002887in}{-0.004970in}}{\pgfqpoint{0.003928in}{-0.003928in}}%
\pgfpathcurveto{\pgfqpoint{0.004970in}{-0.002887in}}{\pgfqpoint{0.005556in}{-0.001473in}}{\pgfqpoint{0.005556in}{0.000000in}}%
\pgfpathcurveto{\pgfqpoint{0.005556in}{0.001473in}}{\pgfqpoint{0.004970in}{0.002887in}}{\pgfqpoint{0.003928in}{0.003928in}}%
\pgfpathcurveto{\pgfqpoint{0.002887in}{0.004970in}}{\pgfqpoint{0.001473in}{0.005556in}}{\pgfqpoint{0.000000in}{0.005556in}}%
\pgfpathcurveto{\pgfqpoint{-0.001473in}{0.005556in}}{\pgfqpoint{-0.002887in}{0.004970in}}{\pgfqpoint{-0.003928in}{0.003928in}}%
\pgfpathcurveto{\pgfqpoint{-0.004970in}{0.002887in}}{\pgfqpoint{-0.005556in}{0.001473in}}{\pgfqpoint{-0.005556in}{0.000000in}}%
\pgfpathcurveto{\pgfqpoint{-0.005556in}{-0.001473in}}{\pgfqpoint{-0.004970in}{-0.002887in}}{\pgfqpoint{-0.003928in}{-0.003928in}}%
\pgfpathcurveto{\pgfqpoint{-0.002887in}{-0.004970in}}{\pgfqpoint{-0.001473in}{-0.005556in}}{\pgfqpoint{0.000000in}{-0.005556in}}%
\pgfpathclose%
\pgfusepath{stroke,fill}%
}%
\begin{pgfscope}%
\pgfsys@transformshift{3.975097in}{4.053085in}%
\pgfsys@useobject{currentmarker}{}%
\end{pgfscope}%
\end{pgfscope}%
\begin{pgfscope}%
\pgfpathrectangle{\pgfqpoint{0.100000in}{2.413063in}}{\pgfqpoint{5.037500in}{3.427208in}}%
\pgfusepath{clip}%
\pgfsetrectcap%
\pgfsetroundjoin%
\pgfsetlinewidth{1.505625pt}%
\definecolor{currentstroke}{rgb}{0.678431,1.000000,0.184314}%
\pgfsetstrokecolor{currentstroke}%
\pgfsetstrokeopacity{0.500000}%
\pgfsetdash{}{0pt}%
\pgfpathmoveto{\pgfqpoint{3.985695in}{4.087467in}}%
\pgfusepath{stroke}%
\end{pgfscope}%
\begin{pgfscope}%
\pgfpathrectangle{\pgfqpoint{0.100000in}{2.413063in}}{\pgfqpoint{5.037500in}{3.427208in}}%
\pgfusepath{clip}%
\pgfsetbuttcap%
\pgfsetroundjoin%
\definecolor{currentfill}{rgb}{0.678431,1.000000,0.184314}%
\pgfsetfillcolor{currentfill}%
\pgfsetfillopacity{0.500000}%
\pgfsetlinewidth{0.250937pt}%
\definecolor{currentstroke}{rgb}{0.000000,0.000000,0.000000}%
\pgfsetstrokecolor{currentstroke}%
\pgfsetstrokeopacity{0.500000}%
\pgfsetdash{}{0pt}%
\pgfsys@defobject{currentmarker}{\pgfqpoint{-0.008333in}{-0.008333in}}{\pgfqpoint{0.008333in}{0.008333in}}{%
\pgfpathmoveto{\pgfqpoint{0.000000in}{-0.008333in}}%
\pgfpathcurveto{\pgfqpoint{0.002210in}{-0.008333in}}{\pgfqpoint{0.004330in}{-0.007455in}}{\pgfqpoint{0.005893in}{-0.005893in}}%
\pgfpathcurveto{\pgfqpoint{0.007455in}{-0.004330in}}{\pgfqpoint{0.008333in}{-0.002210in}}{\pgfqpoint{0.008333in}{0.000000in}}%
\pgfpathcurveto{\pgfqpoint{0.008333in}{0.002210in}}{\pgfqpoint{0.007455in}{0.004330in}}{\pgfqpoint{0.005893in}{0.005893in}}%
\pgfpathcurveto{\pgfqpoint{0.004330in}{0.007455in}}{\pgfqpoint{0.002210in}{0.008333in}}{\pgfqpoint{0.000000in}{0.008333in}}%
\pgfpathcurveto{\pgfqpoint{-0.002210in}{0.008333in}}{\pgfqpoint{-0.004330in}{0.007455in}}{\pgfqpoint{-0.005893in}{0.005893in}}%
\pgfpathcurveto{\pgfqpoint{-0.007455in}{0.004330in}}{\pgfqpoint{-0.008333in}{0.002210in}}{\pgfqpoint{-0.008333in}{0.000000in}}%
\pgfpathcurveto{\pgfqpoint{-0.008333in}{-0.002210in}}{\pgfqpoint{-0.007455in}{-0.004330in}}{\pgfqpoint{-0.005893in}{-0.005893in}}%
\pgfpathcurveto{\pgfqpoint{-0.004330in}{-0.007455in}}{\pgfqpoint{-0.002210in}{-0.008333in}}{\pgfqpoint{0.000000in}{-0.008333in}}%
\pgfpathclose%
\pgfusepath{stroke,fill}%
}%
\begin{pgfscope}%
\pgfsys@transformshift{3.985695in}{4.087467in}%
\pgfsys@useobject{currentmarker}{}%
\end{pgfscope}%
\end{pgfscope}%
\begin{pgfscope}%
\pgfpathrectangle{\pgfqpoint{0.100000in}{2.413063in}}{\pgfqpoint{5.037500in}{3.427208in}}%
\pgfusepath{clip}%
\pgfsetrectcap%
\pgfsetroundjoin%
\pgfsetlinewidth{1.505625pt}%
\definecolor{currentstroke}{rgb}{0.678431,1.000000,0.184314}%
\pgfsetstrokecolor{currentstroke}%
\pgfsetstrokeopacity{0.500000}%
\pgfsetdash{}{0pt}%
\pgfpathmoveto{\pgfqpoint{3.952422in}{4.077493in}}%
\pgfusepath{stroke}%
\end{pgfscope}%
\begin{pgfscope}%
\pgfpathrectangle{\pgfqpoint{0.100000in}{2.413063in}}{\pgfqpoint{5.037500in}{3.427208in}}%
\pgfusepath{clip}%
\pgfsetbuttcap%
\pgfsetroundjoin%
\definecolor{currentfill}{rgb}{0.678431,1.000000,0.184314}%
\pgfsetfillcolor{currentfill}%
\pgfsetfillopacity{0.500000}%
\pgfsetlinewidth{0.250937pt}%
\definecolor{currentstroke}{rgb}{0.000000,0.000000,0.000000}%
\pgfsetstrokecolor{currentstroke}%
\pgfsetstrokeopacity{0.500000}%
\pgfsetdash{}{0pt}%
\pgfsys@defobject{currentmarker}{\pgfqpoint{-0.008333in}{-0.008333in}}{\pgfqpoint{0.008333in}{0.008333in}}{%
\pgfpathmoveto{\pgfqpoint{0.000000in}{-0.008333in}}%
\pgfpathcurveto{\pgfqpoint{0.002210in}{-0.008333in}}{\pgfqpoint{0.004330in}{-0.007455in}}{\pgfqpoint{0.005893in}{-0.005893in}}%
\pgfpathcurveto{\pgfqpoint{0.007455in}{-0.004330in}}{\pgfqpoint{0.008333in}{-0.002210in}}{\pgfqpoint{0.008333in}{0.000000in}}%
\pgfpathcurveto{\pgfqpoint{0.008333in}{0.002210in}}{\pgfqpoint{0.007455in}{0.004330in}}{\pgfqpoint{0.005893in}{0.005893in}}%
\pgfpathcurveto{\pgfqpoint{0.004330in}{0.007455in}}{\pgfqpoint{0.002210in}{0.008333in}}{\pgfqpoint{0.000000in}{0.008333in}}%
\pgfpathcurveto{\pgfqpoint{-0.002210in}{0.008333in}}{\pgfqpoint{-0.004330in}{0.007455in}}{\pgfqpoint{-0.005893in}{0.005893in}}%
\pgfpathcurveto{\pgfqpoint{-0.007455in}{0.004330in}}{\pgfqpoint{-0.008333in}{0.002210in}}{\pgfqpoint{-0.008333in}{0.000000in}}%
\pgfpathcurveto{\pgfqpoint{-0.008333in}{-0.002210in}}{\pgfqpoint{-0.007455in}{-0.004330in}}{\pgfqpoint{-0.005893in}{-0.005893in}}%
\pgfpathcurveto{\pgfqpoint{-0.004330in}{-0.007455in}}{\pgfqpoint{-0.002210in}{-0.008333in}}{\pgfqpoint{0.000000in}{-0.008333in}}%
\pgfpathclose%
\pgfusepath{stroke,fill}%
}%
\begin{pgfscope}%
\pgfsys@transformshift{3.952422in}{4.077493in}%
\pgfsys@useobject{currentmarker}{}%
\end{pgfscope}%
\end{pgfscope}%
\begin{pgfscope}%
\pgfpathrectangle{\pgfqpoint{0.100000in}{2.413063in}}{\pgfqpoint{5.037500in}{3.427208in}}%
\pgfusepath{clip}%
\pgfsetrectcap%
\pgfsetroundjoin%
\pgfsetlinewidth{1.505625pt}%
\definecolor{currentstroke}{rgb}{0.501961,0.501961,0.501961}%
\pgfsetstrokecolor{currentstroke}%
\pgfsetstrokeopacity{0.500000}%
\pgfsetdash{}{0pt}%
\pgfpathmoveto{\pgfqpoint{3.835675in}{3.993728in}}%
\pgfusepath{stroke}%
\end{pgfscope}%
\begin{pgfscope}%
\pgfpathrectangle{\pgfqpoint{0.100000in}{2.413063in}}{\pgfqpoint{5.037500in}{3.427208in}}%
\pgfusepath{clip}%
\pgfsetbuttcap%
\pgfsetroundjoin%
\definecolor{currentfill}{rgb}{0.501961,0.501961,0.501961}%
\pgfsetfillcolor{currentfill}%
\pgfsetfillopacity{0.500000}%
\pgfsetlinewidth{0.250937pt}%
\definecolor{currentstroke}{rgb}{0.000000,0.000000,0.000000}%
\pgfsetstrokecolor{currentstroke}%
\pgfsetstrokeopacity{0.500000}%
\pgfsetdash{}{0pt}%
\pgfsys@defobject{currentmarker}{\pgfqpoint{-0.013889in}{-0.013889in}}{\pgfqpoint{0.013889in}{0.013889in}}{%
\pgfpathmoveto{\pgfqpoint{0.000000in}{-0.013889in}}%
\pgfpathcurveto{\pgfqpoint{0.003683in}{-0.013889in}}{\pgfqpoint{0.007216in}{-0.012425in}}{\pgfqpoint{0.009821in}{-0.009821in}}%
\pgfpathcurveto{\pgfqpoint{0.012425in}{-0.007216in}}{\pgfqpoint{0.013889in}{-0.003683in}}{\pgfqpoint{0.013889in}{0.000000in}}%
\pgfpathcurveto{\pgfqpoint{0.013889in}{0.003683in}}{\pgfqpoint{0.012425in}{0.007216in}}{\pgfqpoint{0.009821in}{0.009821in}}%
\pgfpathcurveto{\pgfqpoint{0.007216in}{0.012425in}}{\pgfqpoint{0.003683in}{0.013889in}}{\pgfqpoint{0.000000in}{0.013889in}}%
\pgfpathcurveto{\pgfqpoint{-0.003683in}{0.013889in}}{\pgfqpoint{-0.007216in}{0.012425in}}{\pgfqpoint{-0.009821in}{0.009821in}}%
\pgfpathcurveto{\pgfqpoint{-0.012425in}{0.007216in}}{\pgfqpoint{-0.013889in}{0.003683in}}{\pgfqpoint{-0.013889in}{0.000000in}}%
\pgfpathcurveto{\pgfqpoint{-0.013889in}{-0.003683in}}{\pgfqpoint{-0.012425in}{-0.007216in}}{\pgfqpoint{-0.009821in}{-0.009821in}}%
\pgfpathcurveto{\pgfqpoint{-0.007216in}{-0.012425in}}{\pgfqpoint{-0.003683in}{-0.013889in}}{\pgfqpoint{0.000000in}{-0.013889in}}%
\pgfpathclose%
\pgfusepath{stroke,fill}%
}%
\begin{pgfscope}%
\pgfsys@transformshift{3.835675in}{3.993728in}%
\pgfsys@useobject{currentmarker}{}%
\end{pgfscope}%
\end{pgfscope}%
\begin{pgfscope}%
\pgfpathrectangle{\pgfqpoint{0.100000in}{2.413063in}}{\pgfqpoint{5.037500in}{3.427208in}}%
\pgfusepath{clip}%
\pgfsetrectcap%
\pgfsetroundjoin%
\pgfsetlinewidth{1.505625pt}%
\definecolor{currentstroke}{rgb}{0.501961,0.501961,0.501961}%
\pgfsetstrokecolor{currentstroke}%
\pgfsetstrokeopacity{0.500000}%
\pgfsetdash{}{0pt}%
\pgfpathmoveto{\pgfqpoint{3.270873in}{3.849879in}}%
\pgfusepath{stroke}%
\end{pgfscope}%
\begin{pgfscope}%
\pgfpathrectangle{\pgfqpoint{0.100000in}{2.413063in}}{\pgfqpoint{5.037500in}{3.427208in}}%
\pgfusepath{clip}%
\pgfsetbuttcap%
\pgfsetroundjoin%
\definecolor{currentfill}{rgb}{0.501961,0.501961,0.501961}%
\pgfsetfillcolor{currentfill}%
\pgfsetfillopacity{0.500000}%
\pgfsetlinewidth{0.250937pt}%
\definecolor{currentstroke}{rgb}{0.000000,0.000000,0.000000}%
\pgfsetstrokecolor{currentstroke}%
\pgfsetstrokeopacity{0.500000}%
\pgfsetdash{}{0pt}%
\pgfsys@defobject{currentmarker}{\pgfqpoint{-0.013889in}{-0.013889in}}{\pgfqpoint{0.013889in}{0.013889in}}{%
\pgfpathmoveto{\pgfqpoint{0.000000in}{-0.013889in}}%
\pgfpathcurveto{\pgfqpoint{0.003683in}{-0.013889in}}{\pgfqpoint{0.007216in}{-0.012425in}}{\pgfqpoint{0.009821in}{-0.009821in}}%
\pgfpathcurveto{\pgfqpoint{0.012425in}{-0.007216in}}{\pgfqpoint{0.013889in}{-0.003683in}}{\pgfqpoint{0.013889in}{0.000000in}}%
\pgfpathcurveto{\pgfqpoint{0.013889in}{0.003683in}}{\pgfqpoint{0.012425in}{0.007216in}}{\pgfqpoint{0.009821in}{0.009821in}}%
\pgfpathcurveto{\pgfqpoint{0.007216in}{0.012425in}}{\pgfqpoint{0.003683in}{0.013889in}}{\pgfqpoint{0.000000in}{0.013889in}}%
\pgfpathcurveto{\pgfqpoint{-0.003683in}{0.013889in}}{\pgfqpoint{-0.007216in}{0.012425in}}{\pgfqpoint{-0.009821in}{0.009821in}}%
\pgfpathcurveto{\pgfqpoint{-0.012425in}{0.007216in}}{\pgfqpoint{-0.013889in}{0.003683in}}{\pgfqpoint{-0.013889in}{0.000000in}}%
\pgfpathcurveto{\pgfqpoint{-0.013889in}{-0.003683in}}{\pgfqpoint{-0.012425in}{-0.007216in}}{\pgfqpoint{-0.009821in}{-0.009821in}}%
\pgfpathcurveto{\pgfqpoint{-0.007216in}{-0.012425in}}{\pgfqpoint{-0.003683in}{-0.013889in}}{\pgfqpoint{0.000000in}{-0.013889in}}%
\pgfpathclose%
\pgfusepath{stroke,fill}%
}%
\begin{pgfscope}%
\pgfsys@transformshift{3.270873in}{3.849879in}%
\pgfsys@useobject{currentmarker}{}%
\end{pgfscope}%
\end{pgfscope}%
\begin{pgfscope}%
\pgfpathrectangle{\pgfqpoint{0.100000in}{2.413063in}}{\pgfqpoint{5.037500in}{3.427208in}}%
\pgfusepath{clip}%
\pgfsetrectcap%
\pgfsetroundjoin%
\pgfsetlinewidth{1.505625pt}%
\definecolor{currentstroke}{rgb}{0.678431,1.000000,0.184314}%
\pgfsetstrokecolor{currentstroke}%
\pgfsetstrokeopacity{0.500000}%
\pgfsetdash{}{0pt}%
\pgfpathmoveto{\pgfqpoint{3.889904in}{4.030032in}}%
\pgfusepath{stroke}%
\end{pgfscope}%
\begin{pgfscope}%
\pgfpathrectangle{\pgfqpoint{0.100000in}{2.413063in}}{\pgfqpoint{5.037500in}{3.427208in}}%
\pgfusepath{clip}%
\pgfsetbuttcap%
\pgfsetroundjoin%
\definecolor{currentfill}{rgb}{0.678431,1.000000,0.184314}%
\pgfsetfillcolor{currentfill}%
\pgfsetfillopacity{0.500000}%
\pgfsetlinewidth{0.250937pt}%
\definecolor{currentstroke}{rgb}{0.000000,0.000000,0.000000}%
\pgfsetstrokecolor{currentstroke}%
\pgfsetstrokeopacity{0.500000}%
\pgfsetdash{}{0pt}%
\pgfsys@defobject{currentmarker}{\pgfqpoint{-0.005556in}{-0.005556in}}{\pgfqpoint{0.005556in}{0.005556in}}{%
\pgfpathmoveto{\pgfqpoint{0.000000in}{-0.005556in}}%
\pgfpathcurveto{\pgfqpoint{0.001473in}{-0.005556in}}{\pgfqpoint{0.002887in}{-0.004970in}}{\pgfqpoint{0.003928in}{-0.003928in}}%
\pgfpathcurveto{\pgfqpoint{0.004970in}{-0.002887in}}{\pgfqpoint{0.005556in}{-0.001473in}}{\pgfqpoint{0.005556in}{0.000000in}}%
\pgfpathcurveto{\pgfqpoint{0.005556in}{0.001473in}}{\pgfqpoint{0.004970in}{0.002887in}}{\pgfqpoint{0.003928in}{0.003928in}}%
\pgfpathcurveto{\pgfqpoint{0.002887in}{0.004970in}}{\pgfqpoint{0.001473in}{0.005556in}}{\pgfqpoint{0.000000in}{0.005556in}}%
\pgfpathcurveto{\pgfqpoint{-0.001473in}{0.005556in}}{\pgfqpoint{-0.002887in}{0.004970in}}{\pgfqpoint{-0.003928in}{0.003928in}}%
\pgfpathcurveto{\pgfqpoint{-0.004970in}{0.002887in}}{\pgfqpoint{-0.005556in}{0.001473in}}{\pgfqpoint{-0.005556in}{0.000000in}}%
\pgfpathcurveto{\pgfqpoint{-0.005556in}{-0.001473in}}{\pgfqpoint{-0.004970in}{-0.002887in}}{\pgfqpoint{-0.003928in}{-0.003928in}}%
\pgfpathcurveto{\pgfqpoint{-0.002887in}{-0.004970in}}{\pgfqpoint{-0.001473in}{-0.005556in}}{\pgfqpoint{0.000000in}{-0.005556in}}%
\pgfpathclose%
\pgfusepath{stroke,fill}%
}%
\begin{pgfscope}%
\pgfsys@transformshift{3.889904in}{4.030032in}%
\pgfsys@useobject{currentmarker}{}%
\end{pgfscope}%
\end{pgfscope}%
\begin{pgfscope}%
\pgfpathrectangle{\pgfqpoint{0.100000in}{2.413063in}}{\pgfqpoint{5.037500in}{3.427208in}}%
\pgfusepath{clip}%
\pgfsetrectcap%
\pgfsetroundjoin%
\pgfsetlinewidth{1.505625pt}%
\definecolor{currentstroke}{rgb}{0.000000,0.000000,1.000000}%
\pgfsetstrokecolor{currentstroke}%
\pgfsetstrokeopacity{0.500000}%
\pgfsetdash{}{0pt}%
\pgfpathmoveto{\pgfqpoint{3.568755in}{3.988672in}}%
\pgfusepath{stroke}%
\end{pgfscope}%
\begin{pgfscope}%
\pgfpathrectangle{\pgfqpoint{0.100000in}{2.413063in}}{\pgfqpoint{5.037500in}{3.427208in}}%
\pgfusepath{clip}%
\pgfsetbuttcap%
\pgfsetroundjoin%
\definecolor{currentfill}{rgb}{0.000000,0.000000,1.000000}%
\pgfsetfillcolor{currentfill}%
\pgfsetfillopacity{0.500000}%
\pgfsetlinewidth{0.250937pt}%
\definecolor{currentstroke}{rgb}{0.000000,0.000000,0.000000}%
\pgfsetstrokecolor{currentstroke}%
\pgfsetstrokeopacity{0.500000}%
\pgfsetdash{}{0pt}%
\pgfsys@defobject{currentmarker}{\pgfqpoint{-0.005556in}{-0.005556in}}{\pgfqpoint{0.005556in}{0.005556in}}{%
\pgfpathmoveto{\pgfqpoint{0.000000in}{-0.005556in}}%
\pgfpathcurveto{\pgfqpoint{0.001473in}{-0.005556in}}{\pgfqpoint{0.002887in}{-0.004970in}}{\pgfqpoint{0.003928in}{-0.003928in}}%
\pgfpathcurveto{\pgfqpoint{0.004970in}{-0.002887in}}{\pgfqpoint{0.005556in}{-0.001473in}}{\pgfqpoint{0.005556in}{0.000000in}}%
\pgfpathcurveto{\pgfqpoint{0.005556in}{0.001473in}}{\pgfqpoint{0.004970in}{0.002887in}}{\pgfqpoint{0.003928in}{0.003928in}}%
\pgfpathcurveto{\pgfqpoint{0.002887in}{0.004970in}}{\pgfqpoint{0.001473in}{0.005556in}}{\pgfqpoint{0.000000in}{0.005556in}}%
\pgfpathcurveto{\pgfqpoint{-0.001473in}{0.005556in}}{\pgfqpoint{-0.002887in}{0.004970in}}{\pgfqpoint{-0.003928in}{0.003928in}}%
\pgfpathcurveto{\pgfqpoint{-0.004970in}{0.002887in}}{\pgfqpoint{-0.005556in}{0.001473in}}{\pgfqpoint{-0.005556in}{0.000000in}}%
\pgfpathcurveto{\pgfqpoint{-0.005556in}{-0.001473in}}{\pgfqpoint{-0.004970in}{-0.002887in}}{\pgfqpoint{-0.003928in}{-0.003928in}}%
\pgfpathcurveto{\pgfqpoint{-0.002887in}{-0.004970in}}{\pgfqpoint{-0.001473in}{-0.005556in}}{\pgfqpoint{0.000000in}{-0.005556in}}%
\pgfpathclose%
\pgfusepath{stroke,fill}%
}%
\begin{pgfscope}%
\pgfsys@transformshift{3.568755in}{3.988672in}%
\pgfsys@useobject{currentmarker}{}%
\end{pgfscope}%
\end{pgfscope}%
\begin{pgfscope}%
\pgfpathrectangle{\pgfqpoint{0.100000in}{2.413063in}}{\pgfqpoint{5.037500in}{3.427208in}}%
\pgfusepath{clip}%
\pgfsetrectcap%
\pgfsetroundjoin%
\pgfsetlinewidth{1.505625pt}%
\definecolor{currentstroke}{rgb}{0.000000,0.000000,1.000000}%
\pgfsetstrokecolor{currentstroke}%
\pgfsetstrokeopacity{0.500000}%
\pgfsetdash{}{0pt}%
\pgfpathmoveto{\pgfqpoint{2.341376in}{3.536703in}}%
\pgfusepath{stroke}%
\end{pgfscope}%
\begin{pgfscope}%
\pgfpathrectangle{\pgfqpoint{0.100000in}{2.413063in}}{\pgfqpoint{5.037500in}{3.427208in}}%
\pgfusepath{clip}%
\pgfsetbuttcap%
\pgfsetroundjoin%
\definecolor{currentfill}{rgb}{0.000000,0.000000,1.000000}%
\pgfsetfillcolor{currentfill}%
\pgfsetfillopacity{0.500000}%
\pgfsetlinewidth{0.250937pt}%
\definecolor{currentstroke}{rgb}{0.000000,0.000000,0.000000}%
\pgfsetstrokecolor{currentstroke}%
\pgfsetstrokeopacity{0.500000}%
\pgfsetdash{}{0pt}%
\pgfsys@defobject{currentmarker}{\pgfqpoint{-0.019444in}{-0.019444in}}{\pgfqpoint{0.019444in}{0.019444in}}{%
\pgfpathmoveto{\pgfqpoint{0.000000in}{-0.019444in}}%
\pgfpathcurveto{\pgfqpoint{0.005157in}{-0.019444in}}{\pgfqpoint{0.010103in}{-0.017396in}}{\pgfqpoint{0.013749in}{-0.013749in}}%
\pgfpathcurveto{\pgfqpoint{0.017396in}{-0.010103in}}{\pgfqpoint{0.019444in}{-0.005157in}}{\pgfqpoint{0.019444in}{0.000000in}}%
\pgfpathcurveto{\pgfqpoint{0.019444in}{0.005157in}}{\pgfqpoint{0.017396in}{0.010103in}}{\pgfqpoint{0.013749in}{0.013749in}}%
\pgfpathcurveto{\pgfqpoint{0.010103in}{0.017396in}}{\pgfqpoint{0.005157in}{0.019444in}}{\pgfqpoint{0.000000in}{0.019444in}}%
\pgfpathcurveto{\pgfqpoint{-0.005157in}{0.019444in}}{\pgfqpoint{-0.010103in}{0.017396in}}{\pgfqpoint{-0.013749in}{0.013749in}}%
\pgfpathcurveto{\pgfqpoint{-0.017396in}{0.010103in}}{\pgfqpoint{-0.019444in}{0.005157in}}{\pgfqpoint{-0.019444in}{0.000000in}}%
\pgfpathcurveto{\pgfqpoint{-0.019444in}{-0.005157in}}{\pgfqpoint{-0.017396in}{-0.010103in}}{\pgfqpoint{-0.013749in}{-0.013749in}}%
\pgfpathcurveto{\pgfqpoint{-0.010103in}{-0.017396in}}{\pgfqpoint{-0.005157in}{-0.019444in}}{\pgfqpoint{0.000000in}{-0.019444in}}%
\pgfpathclose%
\pgfusepath{stroke,fill}%
}%
\begin{pgfscope}%
\pgfsys@transformshift{2.341376in}{3.536703in}%
\pgfsys@useobject{currentmarker}{}%
\end{pgfscope}%
\end{pgfscope}%
\begin{pgfscope}%
\pgfpathrectangle{\pgfqpoint{0.100000in}{2.413063in}}{\pgfqpoint{5.037500in}{3.427208in}}%
\pgfusepath{clip}%
\pgfsetrectcap%
\pgfsetroundjoin%
\pgfsetlinewidth{1.505625pt}%
\definecolor{currentstroke}{rgb}{0.000000,0.000000,1.000000}%
\pgfsetstrokecolor{currentstroke}%
\pgfsetstrokeopacity{0.500000}%
\pgfsetdash{}{0pt}%
\pgfpathmoveto{\pgfqpoint{2.160335in}{3.868102in}}%
\pgfusepath{stroke}%
\end{pgfscope}%
\begin{pgfscope}%
\pgfpathrectangle{\pgfqpoint{0.100000in}{2.413063in}}{\pgfqpoint{5.037500in}{3.427208in}}%
\pgfusepath{clip}%
\pgfsetbuttcap%
\pgfsetroundjoin%
\definecolor{currentfill}{rgb}{0.000000,0.000000,1.000000}%
\pgfsetfillcolor{currentfill}%
\pgfsetfillopacity{0.500000}%
\pgfsetlinewidth{0.250937pt}%
\definecolor{currentstroke}{rgb}{0.000000,0.000000,0.000000}%
\pgfsetstrokecolor{currentstroke}%
\pgfsetstrokeopacity{0.500000}%
\pgfsetdash{}{0pt}%
\pgfsys@defobject{currentmarker}{\pgfqpoint{-0.016667in}{-0.016667in}}{\pgfqpoint{0.016667in}{0.016667in}}{%
\pgfpathmoveto{\pgfqpoint{0.000000in}{-0.016667in}}%
\pgfpathcurveto{\pgfqpoint{0.004420in}{-0.016667in}}{\pgfqpoint{0.008660in}{-0.014911in}}{\pgfqpoint{0.011785in}{-0.011785in}}%
\pgfpathcurveto{\pgfqpoint{0.014911in}{-0.008660in}}{\pgfqpoint{0.016667in}{-0.004420in}}{\pgfqpoint{0.016667in}{0.000000in}}%
\pgfpathcurveto{\pgfqpoint{0.016667in}{0.004420in}}{\pgfqpoint{0.014911in}{0.008660in}}{\pgfqpoint{0.011785in}{0.011785in}}%
\pgfpathcurveto{\pgfqpoint{0.008660in}{0.014911in}}{\pgfqpoint{0.004420in}{0.016667in}}{\pgfqpoint{0.000000in}{0.016667in}}%
\pgfpathcurveto{\pgfqpoint{-0.004420in}{0.016667in}}{\pgfqpoint{-0.008660in}{0.014911in}}{\pgfqpoint{-0.011785in}{0.011785in}}%
\pgfpathcurveto{\pgfqpoint{-0.014911in}{0.008660in}}{\pgfqpoint{-0.016667in}{0.004420in}}{\pgfqpoint{-0.016667in}{0.000000in}}%
\pgfpathcurveto{\pgfqpoint{-0.016667in}{-0.004420in}}{\pgfqpoint{-0.014911in}{-0.008660in}}{\pgfqpoint{-0.011785in}{-0.011785in}}%
\pgfpathcurveto{\pgfqpoint{-0.008660in}{-0.014911in}}{\pgfqpoint{-0.004420in}{-0.016667in}}{\pgfqpoint{0.000000in}{-0.016667in}}%
\pgfpathclose%
\pgfusepath{stroke,fill}%
}%
\begin{pgfscope}%
\pgfsys@transformshift{2.160335in}{3.868102in}%
\pgfsys@useobject{currentmarker}{}%
\end{pgfscope}%
\end{pgfscope}%
\begin{pgfscope}%
\pgfpathrectangle{\pgfqpoint{0.100000in}{2.413063in}}{\pgfqpoint{5.037500in}{3.427208in}}%
\pgfusepath{clip}%
\pgfsetrectcap%
\pgfsetroundjoin%
\pgfsetlinewidth{1.505625pt}%
\definecolor{currentstroke}{rgb}{0.000000,0.000000,1.000000}%
\pgfsetstrokecolor{currentstroke}%
\pgfsetstrokeopacity{0.500000}%
\pgfsetdash{}{0pt}%
\pgfpathmoveto{\pgfqpoint{2.528183in}{3.276102in}}%
\pgfusepath{stroke}%
\end{pgfscope}%
\begin{pgfscope}%
\pgfpathrectangle{\pgfqpoint{0.100000in}{2.413063in}}{\pgfqpoint{5.037500in}{3.427208in}}%
\pgfusepath{clip}%
\pgfsetbuttcap%
\pgfsetroundjoin%
\definecolor{currentfill}{rgb}{0.000000,0.000000,1.000000}%
\pgfsetfillcolor{currentfill}%
\pgfsetfillopacity{0.500000}%
\pgfsetlinewidth{0.250937pt}%
\definecolor{currentstroke}{rgb}{0.000000,0.000000,0.000000}%
\pgfsetstrokecolor{currentstroke}%
\pgfsetstrokeopacity{0.500000}%
\pgfsetdash{}{0pt}%
\pgfsys@defobject{currentmarker}{\pgfqpoint{-0.013889in}{-0.013889in}}{\pgfqpoint{0.013889in}{0.013889in}}{%
\pgfpathmoveto{\pgfqpoint{0.000000in}{-0.013889in}}%
\pgfpathcurveto{\pgfqpoint{0.003683in}{-0.013889in}}{\pgfqpoint{0.007216in}{-0.012425in}}{\pgfqpoint{0.009821in}{-0.009821in}}%
\pgfpathcurveto{\pgfqpoint{0.012425in}{-0.007216in}}{\pgfqpoint{0.013889in}{-0.003683in}}{\pgfqpoint{0.013889in}{0.000000in}}%
\pgfpathcurveto{\pgfqpoint{0.013889in}{0.003683in}}{\pgfqpoint{0.012425in}{0.007216in}}{\pgfqpoint{0.009821in}{0.009821in}}%
\pgfpathcurveto{\pgfqpoint{0.007216in}{0.012425in}}{\pgfqpoint{0.003683in}{0.013889in}}{\pgfqpoint{0.000000in}{0.013889in}}%
\pgfpathcurveto{\pgfqpoint{-0.003683in}{0.013889in}}{\pgfqpoint{-0.007216in}{0.012425in}}{\pgfqpoint{-0.009821in}{0.009821in}}%
\pgfpathcurveto{\pgfqpoint{-0.012425in}{0.007216in}}{\pgfqpoint{-0.013889in}{0.003683in}}{\pgfqpoint{-0.013889in}{0.000000in}}%
\pgfpathcurveto{\pgfqpoint{-0.013889in}{-0.003683in}}{\pgfqpoint{-0.012425in}{-0.007216in}}{\pgfqpoint{-0.009821in}{-0.009821in}}%
\pgfpathcurveto{\pgfqpoint{-0.007216in}{-0.012425in}}{\pgfqpoint{-0.003683in}{-0.013889in}}{\pgfqpoint{0.000000in}{-0.013889in}}%
\pgfpathclose%
\pgfusepath{stroke,fill}%
}%
\begin{pgfscope}%
\pgfsys@transformshift{2.528183in}{3.276102in}%
\pgfsys@useobject{currentmarker}{}%
\end{pgfscope}%
\end{pgfscope}%
\begin{pgfscope}%
\pgfpathrectangle{\pgfqpoint{0.100000in}{2.413063in}}{\pgfqpoint{5.037500in}{3.427208in}}%
\pgfusepath{clip}%
\pgfsetrectcap%
\pgfsetroundjoin%
\pgfsetlinewidth{1.505625pt}%
\definecolor{currentstroke}{rgb}{0.000000,0.000000,1.000000}%
\pgfsetstrokecolor{currentstroke}%
\pgfsetstrokeopacity{0.500000}%
\pgfsetdash{}{0pt}%
\pgfpathmoveto{\pgfqpoint{2.894684in}{3.250822in}}%
\pgfusepath{stroke}%
\end{pgfscope}%
\begin{pgfscope}%
\pgfpathrectangle{\pgfqpoint{0.100000in}{2.413063in}}{\pgfqpoint{5.037500in}{3.427208in}}%
\pgfusepath{clip}%
\pgfsetbuttcap%
\pgfsetroundjoin%
\definecolor{currentfill}{rgb}{0.000000,0.000000,1.000000}%
\pgfsetfillcolor{currentfill}%
\pgfsetfillopacity{0.500000}%
\pgfsetlinewidth{0.250937pt}%
\definecolor{currentstroke}{rgb}{0.000000,0.000000,0.000000}%
\pgfsetstrokecolor{currentstroke}%
\pgfsetstrokeopacity{0.500000}%
\pgfsetdash{}{0pt}%
\pgfsys@defobject{currentmarker}{\pgfqpoint{-0.050000in}{-0.050000in}}{\pgfqpoint{0.050000in}{0.050000in}}{%
\pgfpathmoveto{\pgfqpoint{0.000000in}{-0.050000in}}%
\pgfpathcurveto{\pgfqpoint{0.013260in}{-0.050000in}}{\pgfqpoint{0.025979in}{-0.044732in}}{\pgfqpoint{0.035355in}{-0.035355in}}%
\pgfpathcurveto{\pgfqpoint{0.044732in}{-0.025979in}}{\pgfqpoint{0.050000in}{-0.013260in}}{\pgfqpoint{0.050000in}{0.000000in}}%
\pgfpathcurveto{\pgfqpoint{0.050000in}{0.013260in}}{\pgfqpoint{0.044732in}{0.025979in}}{\pgfqpoint{0.035355in}{0.035355in}}%
\pgfpathcurveto{\pgfqpoint{0.025979in}{0.044732in}}{\pgfqpoint{0.013260in}{0.050000in}}{\pgfqpoint{0.000000in}{0.050000in}}%
\pgfpathcurveto{\pgfqpoint{-0.013260in}{0.050000in}}{\pgfqpoint{-0.025979in}{0.044732in}}{\pgfqpoint{-0.035355in}{0.035355in}}%
\pgfpathcurveto{\pgfqpoint{-0.044732in}{0.025979in}}{\pgfqpoint{-0.050000in}{0.013260in}}{\pgfqpoint{-0.050000in}{0.000000in}}%
\pgfpathcurveto{\pgfqpoint{-0.050000in}{-0.013260in}}{\pgfqpoint{-0.044732in}{-0.025979in}}{\pgfqpoint{-0.035355in}{-0.035355in}}%
\pgfpathcurveto{\pgfqpoint{-0.025979in}{-0.044732in}}{\pgfqpoint{-0.013260in}{-0.050000in}}{\pgfqpoint{0.000000in}{-0.050000in}}%
\pgfpathclose%
\pgfusepath{stroke,fill}%
}%
\begin{pgfscope}%
\pgfsys@transformshift{2.894684in}{3.250822in}%
\pgfsys@useobject{currentmarker}{}%
\end{pgfscope}%
\end{pgfscope}%
\begin{pgfscope}%
\pgfpathrectangle{\pgfqpoint{0.100000in}{2.413063in}}{\pgfqpoint{5.037500in}{3.427208in}}%
\pgfusepath{clip}%
\pgfsetrectcap%
\pgfsetroundjoin%
\pgfsetlinewidth{1.505625pt}%
\definecolor{currentstroke}{rgb}{0.000000,0.000000,1.000000}%
\pgfsetstrokecolor{currentstroke}%
\pgfsetstrokeopacity{0.500000}%
\pgfsetdash{}{0pt}%
\pgfpathmoveto{\pgfqpoint{2.539796in}{2.764801in}}%
\pgfusepath{stroke}%
\end{pgfscope}%
\begin{pgfscope}%
\pgfpathrectangle{\pgfqpoint{0.100000in}{2.413063in}}{\pgfqpoint{5.037500in}{3.427208in}}%
\pgfusepath{clip}%
\pgfsetbuttcap%
\pgfsetroundjoin%
\definecolor{currentfill}{rgb}{0.000000,0.000000,1.000000}%
\pgfsetfillcolor{currentfill}%
\pgfsetfillopacity{0.500000}%
\pgfsetlinewidth{0.250937pt}%
\definecolor{currentstroke}{rgb}{0.000000,0.000000,0.000000}%
\pgfsetstrokecolor{currentstroke}%
\pgfsetstrokeopacity{0.500000}%
\pgfsetdash{}{0pt}%
\pgfsys@defobject{currentmarker}{\pgfqpoint{-0.036111in}{-0.036111in}}{\pgfqpoint{0.036111in}{0.036111in}}{%
\pgfpathmoveto{\pgfqpoint{0.000000in}{-0.036111in}}%
\pgfpathcurveto{\pgfqpoint{0.009577in}{-0.036111in}}{\pgfqpoint{0.018763in}{-0.032306in}}{\pgfqpoint{0.025534in}{-0.025534in}}%
\pgfpathcurveto{\pgfqpoint{0.032306in}{-0.018763in}}{\pgfqpoint{0.036111in}{-0.009577in}}{\pgfqpoint{0.036111in}{0.000000in}}%
\pgfpathcurveto{\pgfqpoint{0.036111in}{0.009577in}}{\pgfqpoint{0.032306in}{0.018763in}}{\pgfqpoint{0.025534in}{0.025534in}}%
\pgfpathcurveto{\pgfqpoint{0.018763in}{0.032306in}}{\pgfqpoint{0.009577in}{0.036111in}}{\pgfqpoint{0.000000in}{0.036111in}}%
\pgfpathcurveto{\pgfqpoint{-0.009577in}{0.036111in}}{\pgfqpoint{-0.018763in}{0.032306in}}{\pgfqpoint{-0.025534in}{0.025534in}}%
\pgfpathcurveto{\pgfqpoint{-0.032306in}{0.018763in}}{\pgfqpoint{-0.036111in}{0.009577in}}{\pgfqpoint{-0.036111in}{0.000000in}}%
\pgfpathcurveto{\pgfqpoint{-0.036111in}{-0.009577in}}{\pgfqpoint{-0.032306in}{-0.018763in}}{\pgfqpoint{-0.025534in}{-0.025534in}}%
\pgfpathcurveto{\pgfqpoint{-0.018763in}{-0.032306in}}{\pgfqpoint{-0.009577in}{-0.036111in}}{\pgfqpoint{0.000000in}{-0.036111in}}%
\pgfpathclose%
\pgfusepath{stroke,fill}%
}%
\begin{pgfscope}%
\pgfsys@transformshift{2.539796in}{2.764801in}%
\pgfsys@useobject{currentmarker}{}%
\end{pgfscope}%
\end{pgfscope}%
\begin{pgfscope}%
\pgfpathrectangle{\pgfqpoint{0.100000in}{2.413063in}}{\pgfqpoint{5.037500in}{3.427208in}}%
\pgfusepath{clip}%
\pgfsetrectcap%
\pgfsetroundjoin%
\pgfsetlinewidth{1.505625pt}%
\definecolor{currentstroke}{rgb}{0.000000,0.000000,1.000000}%
\pgfsetstrokecolor{currentstroke}%
\pgfsetstrokeopacity{0.500000}%
\pgfsetdash{}{0pt}%
\pgfpathmoveto{\pgfqpoint{2.671238in}{3.314144in}}%
\pgfusepath{stroke}%
\end{pgfscope}%
\begin{pgfscope}%
\pgfpathrectangle{\pgfqpoint{0.100000in}{2.413063in}}{\pgfqpoint{5.037500in}{3.427208in}}%
\pgfusepath{clip}%
\pgfsetbuttcap%
\pgfsetroundjoin%
\definecolor{currentfill}{rgb}{0.000000,0.000000,1.000000}%
\pgfsetfillcolor{currentfill}%
\pgfsetfillopacity{0.500000}%
\pgfsetlinewidth{0.250937pt}%
\definecolor{currentstroke}{rgb}{0.000000,0.000000,0.000000}%
\pgfsetstrokecolor{currentstroke}%
\pgfsetstrokeopacity{0.500000}%
\pgfsetdash{}{0pt}%
\pgfsys@defobject{currentmarker}{\pgfqpoint{-0.022222in}{-0.022222in}}{\pgfqpoint{0.022222in}{0.022222in}}{%
\pgfpathmoveto{\pgfqpoint{0.000000in}{-0.022222in}}%
\pgfpathcurveto{\pgfqpoint{0.005893in}{-0.022222in}}{\pgfqpoint{0.011546in}{-0.019881in}}{\pgfqpoint{0.015713in}{-0.015713in}}%
\pgfpathcurveto{\pgfqpoint{0.019881in}{-0.011546in}}{\pgfqpoint{0.022222in}{-0.005893in}}{\pgfqpoint{0.022222in}{0.000000in}}%
\pgfpathcurveto{\pgfqpoint{0.022222in}{0.005893in}}{\pgfqpoint{0.019881in}{0.011546in}}{\pgfqpoint{0.015713in}{0.015713in}}%
\pgfpathcurveto{\pgfqpoint{0.011546in}{0.019881in}}{\pgfqpoint{0.005893in}{0.022222in}}{\pgfqpoint{0.000000in}{0.022222in}}%
\pgfpathcurveto{\pgfqpoint{-0.005893in}{0.022222in}}{\pgfqpoint{-0.011546in}{0.019881in}}{\pgfqpoint{-0.015713in}{0.015713in}}%
\pgfpathcurveto{\pgfqpoint{-0.019881in}{0.011546in}}{\pgfqpoint{-0.022222in}{0.005893in}}{\pgfqpoint{-0.022222in}{0.000000in}}%
\pgfpathcurveto{\pgfqpoint{-0.022222in}{-0.005893in}}{\pgfqpoint{-0.019881in}{-0.011546in}}{\pgfqpoint{-0.015713in}{-0.015713in}}%
\pgfpathcurveto{\pgfqpoint{-0.011546in}{-0.019881in}}{\pgfqpoint{-0.005893in}{-0.022222in}}{\pgfqpoint{0.000000in}{-0.022222in}}%
\pgfpathclose%
\pgfusepath{stroke,fill}%
}%
\begin{pgfscope}%
\pgfsys@transformshift{2.671238in}{3.314144in}%
\pgfsys@useobject{currentmarker}{}%
\end{pgfscope}%
\end{pgfscope}%
\begin{pgfscope}%
\pgfpathrectangle{\pgfqpoint{0.100000in}{2.413063in}}{\pgfqpoint{5.037500in}{3.427208in}}%
\pgfusepath{clip}%
\pgfsetrectcap%
\pgfsetroundjoin%
\pgfsetlinewidth{1.505625pt}%
\definecolor{currentstroke}{rgb}{0.000000,0.000000,1.000000}%
\pgfsetstrokecolor{currentstroke}%
\pgfsetstrokeopacity{0.500000}%
\pgfsetdash{}{0pt}%
\pgfpathmoveto{\pgfqpoint{2.555575in}{2.986438in}}%
\pgfusepath{stroke}%
\end{pgfscope}%
\begin{pgfscope}%
\pgfpathrectangle{\pgfqpoint{0.100000in}{2.413063in}}{\pgfqpoint{5.037500in}{3.427208in}}%
\pgfusepath{clip}%
\pgfsetbuttcap%
\pgfsetroundjoin%
\definecolor{currentfill}{rgb}{0.000000,0.000000,1.000000}%
\pgfsetfillcolor{currentfill}%
\pgfsetfillopacity{0.500000}%
\pgfsetlinewidth{0.250937pt}%
\definecolor{currentstroke}{rgb}{0.000000,0.000000,0.000000}%
\pgfsetstrokecolor{currentstroke}%
\pgfsetstrokeopacity{0.500000}%
\pgfsetdash{}{0pt}%
\pgfsys@defobject{currentmarker}{\pgfqpoint{-0.044444in}{-0.044444in}}{\pgfqpoint{0.044444in}{0.044444in}}{%
\pgfpathmoveto{\pgfqpoint{0.000000in}{-0.044444in}}%
\pgfpathcurveto{\pgfqpoint{0.011787in}{-0.044444in}}{\pgfqpoint{0.023092in}{-0.039761in}}{\pgfqpoint{0.031427in}{-0.031427in}}%
\pgfpathcurveto{\pgfqpoint{0.039761in}{-0.023092in}}{\pgfqpoint{0.044444in}{-0.011787in}}{\pgfqpoint{0.044444in}{0.000000in}}%
\pgfpathcurveto{\pgfqpoint{0.044444in}{0.011787in}}{\pgfqpoint{0.039761in}{0.023092in}}{\pgfqpoint{0.031427in}{0.031427in}}%
\pgfpathcurveto{\pgfqpoint{0.023092in}{0.039761in}}{\pgfqpoint{0.011787in}{0.044444in}}{\pgfqpoint{0.000000in}{0.044444in}}%
\pgfpathcurveto{\pgfqpoint{-0.011787in}{0.044444in}}{\pgfqpoint{-0.023092in}{0.039761in}}{\pgfqpoint{-0.031427in}{0.031427in}}%
\pgfpathcurveto{\pgfqpoint{-0.039761in}{0.023092in}}{\pgfqpoint{-0.044444in}{0.011787in}}{\pgfqpoint{-0.044444in}{0.000000in}}%
\pgfpathcurveto{\pgfqpoint{-0.044444in}{-0.011787in}}{\pgfqpoint{-0.039761in}{-0.023092in}}{\pgfqpoint{-0.031427in}{-0.031427in}}%
\pgfpathcurveto{\pgfqpoint{-0.023092in}{-0.039761in}}{\pgfqpoint{-0.011787in}{-0.044444in}}{\pgfqpoint{0.000000in}{-0.044444in}}%
\pgfpathclose%
\pgfusepath{stroke,fill}%
}%
\begin{pgfscope}%
\pgfsys@transformshift{2.555575in}{2.986438in}%
\pgfsys@useobject{currentmarker}{}%
\end{pgfscope}%
\end{pgfscope}%
\begin{pgfscope}%
\pgfpathrectangle{\pgfqpoint{0.100000in}{2.413063in}}{\pgfqpoint{5.037500in}{3.427208in}}%
\pgfusepath{clip}%
\pgfsetrectcap%
\pgfsetroundjoin%
\pgfsetlinewidth{1.505625pt}%
\definecolor{currentstroke}{rgb}{0.000000,0.000000,1.000000}%
\pgfsetstrokecolor{currentstroke}%
\pgfsetstrokeopacity{0.500000}%
\pgfsetdash{}{0pt}%
\pgfpathmoveto{\pgfqpoint{2.629147in}{3.564594in}}%
\pgfusepath{stroke}%
\end{pgfscope}%
\begin{pgfscope}%
\pgfpathrectangle{\pgfqpoint{0.100000in}{2.413063in}}{\pgfqpoint{5.037500in}{3.427208in}}%
\pgfusepath{clip}%
\pgfsetbuttcap%
\pgfsetroundjoin%
\definecolor{currentfill}{rgb}{0.000000,0.000000,1.000000}%
\pgfsetfillcolor{currentfill}%
\pgfsetfillopacity{0.500000}%
\pgfsetlinewidth{0.250937pt}%
\definecolor{currentstroke}{rgb}{0.000000,0.000000,0.000000}%
\pgfsetstrokecolor{currentstroke}%
\pgfsetstrokeopacity{0.500000}%
\pgfsetdash{}{0pt}%
\pgfsys@defobject{currentmarker}{\pgfqpoint{-0.019444in}{-0.019444in}}{\pgfqpoint{0.019444in}{0.019444in}}{%
\pgfpathmoveto{\pgfqpoint{0.000000in}{-0.019444in}}%
\pgfpathcurveto{\pgfqpoint{0.005157in}{-0.019444in}}{\pgfqpoint{0.010103in}{-0.017396in}}{\pgfqpoint{0.013749in}{-0.013749in}}%
\pgfpathcurveto{\pgfqpoint{0.017396in}{-0.010103in}}{\pgfqpoint{0.019444in}{-0.005157in}}{\pgfqpoint{0.019444in}{0.000000in}}%
\pgfpathcurveto{\pgfqpoint{0.019444in}{0.005157in}}{\pgfqpoint{0.017396in}{0.010103in}}{\pgfqpoint{0.013749in}{0.013749in}}%
\pgfpathcurveto{\pgfqpoint{0.010103in}{0.017396in}}{\pgfqpoint{0.005157in}{0.019444in}}{\pgfqpoint{0.000000in}{0.019444in}}%
\pgfpathcurveto{\pgfqpoint{-0.005157in}{0.019444in}}{\pgfqpoint{-0.010103in}{0.017396in}}{\pgfqpoint{-0.013749in}{0.013749in}}%
\pgfpathcurveto{\pgfqpoint{-0.017396in}{0.010103in}}{\pgfqpoint{-0.019444in}{0.005157in}}{\pgfqpoint{-0.019444in}{0.000000in}}%
\pgfpathcurveto{\pgfqpoint{-0.019444in}{-0.005157in}}{\pgfqpoint{-0.017396in}{-0.010103in}}{\pgfqpoint{-0.013749in}{-0.013749in}}%
\pgfpathcurveto{\pgfqpoint{-0.010103in}{-0.017396in}}{\pgfqpoint{-0.005157in}{-0.019444in}}{\pgfqpoint{0.000000in}{-0.019444in}}%
\pgfpathclose%
\pgfusepath{stroke,fill}%
}%
\begin{pgfscope}%
\pgfsys@transformshift{2.629147in}{3.564594in}%
\pgfsys@useobject{currentmarker}{}%
\end{pgfscope}%
\end{pgfscope}%
\begin{pgfscope}%
\pgfpathrectangle{\pgfqpoint{0.100000in}{2.413063in}}{\pgfqpoint{5.037500in}{3.427208in}}%
\pgfusepath{clip}%
\pgfsetrectcap%
\pgfsetroundjoin%
\pgfsetlinewidth{1.505625pt}%
\definecolor{currentstroke}{rgb}{0.000000,0.000000,1.000000}%
\pgfsetstrokecolor{currentstroke}%
\pgfsetstrokeopacity{0.500000}%
\pgfsetdash{}{0pt}%
\pgfpathmoveto{\pgfqpoint{1.673527in}{3.516590in}}%
\pgfusepath{stroke}%
\end{pgfscope}%
\begin{pgfscope}%
\pgfpathrectangle{\pgfqpoint{0.100000in}{2.413063in}}{\pgfqpoint{5.037500in}{3.427208in}}%
\pgfusepath{clip}%
\pgfsetbuttcap%
\pgfsetroundjoin%
\definecolor{currentfill}{rgb}{0.000000,0.000000,1.000000}%
\pgfsetfillcolor{currentfill}%
\pgfsetfillopacity{0.500000}%
\pgfsetlinewidth{0.250937pt}%
\definecolor{currentstroke}{rgb}{0.000000,0.000000,0.000000}%
\pgfsetstrokecolor{currentstroke}%
\pgfsetstrokeopacity{0.500000}%
\pgfsetdash{}{0pt}%
\pgfsys@defobject{currentmarker}{\pgfqpoint{-0.033333in}{-0.033333in}}{\pgfqpoint{0.033333in}{0.033333in}}{%
\pgfpathmoveto{\pgfqpoint{0.000000in}{-0.033333in}}%
\pgfpathcurveto{\pgfqpoint{0.008840in}{-0.033333in}}{\pgfqpoint{0.017319in}{-0.029821in}}{\pgfqpoint{0.023570in}{-0.023570in}}%
\pgfpathcurveto{\pgfqpoint{0.029821in}{-0.017319in}}{\pgfqpoint{0.033333in}{-0.008840in}}{\pgfqpoint{0.033333in}{0.000000in}}%
\pgfpathcurveto{\pgfqpoint{0.033333in}{0.008840in}}{\pgfqpoint{0.029821in}{0.017319in}}{\pgfqpoint{0.023570in}{0.023570in}}%
\pgfpathcurveto{\pgfqpoint{0.017319in}{0.029821in}}{\pgfqpoint{0.008840in}{0.033333in}}{\pgfqpoint{0.000000in}{0.033333in}}%
\pgfpathcurveto{\pgfqpoint{-0.008840in}{0.033333in}}{\pgfqpoint{-0.017319in}{0.029821in}}{\pgfqpoint{-0.023570in}{0.023570in}}%
\pgfpathcurveto{\pgfqpoint{-0.029821in}{0.017319in}}{\pgfqpoint{-0.033333in}{0.008840in}}{\pgfqpoint{-0.033333in}{0.000000in}}%
\pgfpathcurveto{\pgfqpoint{-0.033333in}{-0.008840in}}{\pgfqpoint{-0.029821in}{-0.017319in}}{\pgfqpoint{-0.023570in}{-0.023570in}}%
\pgfpathcurveto{\pgfqpoint{-0.017319in}{-0.029821in}}{\pgfqpoint{-0.008840in}{-0.033333in}}{\pgfqpoint{0.000000in}{-0.033333in}}%
\pgfpathclose%
\pgfusepath{stroke,fill}%
}%
\begin{pgfscope}%
\pgfsys@transformshift{1.673527in}{3.516590in}%
\pgfsys@useobject{currentmarker}{}%
\end{pgfscope}%
\end{pgfscope}%
\begin{pgfscope}%
\pgfpathrectangle{\pgfqpoint{0.100000in}{2.413063in}}{\pgfqpoint{5.037500in}{3.427208in}}%
\pgfusepath{clip}%
\pgfsetrectcap%
\pgfsetroundjoin%
\pgfsetlinewidth{1.505625pt}%
\definecolor{currentstroke}{rgb}{0.000000,0.000000,1.000000}%
\pgfsetstrokecolor{currentstroke}%
\pgfsetstrokeopacity{0.500000}%
\pgfsetdash{}{0pt}%
\pgfpathmoveto{\pgfqpoint{2.766900in}{3.212317in}}%
\pgfusepath{stroke}%
\end{pgfscope}%
\begin{pgfscope}%
\pgfpathrectangle{\pgfqpoint{0.100000in}{2.413063in}}{\pgfqpoint{5.037500in}{3.427208in}}%
\pgfusepath{clip}%
\pgfsetbuttcap%
\pgfsetroundjoin%
\definecolor{currentfill}{rgb}{0.000000,0.000000,1.000000}%
\pgfsetfillcolor{currentfill}%
\pgfsetfillopacity{0.500000}%
\pgfsetlinewidth{0.250937pt}%
\definecolor{currentstroke}{rgb}{0.000000,0.000000,0.000000}%
\pgfsetstrokecolor{currentstroke}%
\pgfsetstrokeopacity{0.500000}%
\pgfsetdash{}{0pt}%
\pgfsys@defobject{currentmarker}{\pgfqpoint{-0.033333in}{-0.033333in}}{\pgfqpoint{0.033333in}{0.033333in}}{%
\pgfpathmoveto{\pgfqpoint{0.000000in}{-0.033333in}}%
\pgfpathcurveto{\pgfqpoint{0.008840in}{-0.033333in}}{\pgfqpoint{0.017319in}{-0.029821in}}{\pgfqpoint{0.023570in}{-0.023570in}}%
\pgfpathcurveto{\pgfqpoint{0.029821in}{-0.017319in}}{\pgfqpoint{0.033333in}{-0.008840in}}{\pgfqpoint{0.033333in}{0.000000in}}%
\pgfpathcurveto{\pgfqpoint{0.033333in}{0.008840in}}{\pgfqpoint{0.029821in}{0.017319in}}{\pgfqpoint{0.023570in}{0.023570in}}%
\pgfpathcurveto{\pgfqpoint{0.017319in}{0.029821in}}{\pgfqpoint{0.008840in}{0.033333in}}{\pgfqpoint{0.000000in}{0.033333in}}%
\pgfpathcurveto{\pgfqpoint{-0.008840in}{0.033333in}}{\pgfqpoint{-0.017319in}{0.029821in}}{\pgfqpoint{-0.023570in}{0.023570in}}%
\pgfpathcurveto{\pgfqpoint{-0.029821in}{0.017319in}}{\pgfqpoint{-0.033333in}{0.008840in}}{\pgfqpoint{-0.033333in}{0.000000in}}%
\pgfpathcurveto{\pgfqpoint{-0.033333in}{-0.008840in}}{\pgfqpoint{-0.029821in}{-0.017319in}}{\pgfqpoint{-0.023570in}{-0.023570in}}%
\pgfpathcurveto{\pgfqpoint{-0.017319in}{-0.029821in}}{\pgfqpoint{-0.008840in}{-0.033333in}}{\pgfqpoint{0.000000in}{-0.033333in}}%
\pgfpathclose%
\pgfusepath{stroke,fill}%
}%
\begin{pgfscope}%
\pgfsys@transformshift{2.766900in}{3.212317in}%
\pgfsys@useobject{currentmarker}{}%
\end{pgfscope}%
\end{pgfscope}%
\begin{pgfscope}%
\pgfpathrectangle{\pgfqpoint{0.100000in}{2.413063in}}{\pgfqpoint{5.037500in}{3.427208in}}%
\pgfusepath{clip}%
\pgfsetrectcap%
\pgfsetroundjoin%
\pgfsetlinewidth{1.505625pt}%
\definecolor{currentstroke}{rgb}{0.000000,0.000000,1.000000}%
\pgfsetstrokecolor{currentstroke}%
\pgfsetstrokeopacity{0.500000}%
\pgfsetdash{}{0pt}%
\pgfpathmoveto{\pgfqpoint{2.532736in}{3.374442in}}%
\pgfusepath{stroke}%
\end{pgfscope}%
\begin{pgfscope}%
\pgfpathrectangle{\pgfqpoint{0.100000in}{2.413063in}}{\pgfqpoint{5.037500in}{3.427208in}}%
\pgfusepath{clip}%
\pgfsetbuttcap%
\pgfsetroundjoin%
\definecolor{currentfill}{rgb}{0.000000,0.000000,1.000000}%
\pgfsetfillcolor{currentfill}%
\pgfsetfillopacity{0.500000}%
\pgfsetlinewidth{0.250937pt}%
\definecolor{currentstroke}{rgb}{0.000000,0.000000,0.000000}%
\pgfsetstrokecolor{currentstroke}%
\pgfsetstrokeopacity{0.500000}%
\pgfsetdash{}{0pt}%
\pgfsys@defobject{currentmarker}{\pgfqpoint{-0.027778in}{-0.027778in}}{\pgfqpoint{0.027778in}{0.027778in}}{%
\pgfpathmoveto{\pgfqpoint{0.000000in}{-0.027778in}}%
\pgfpathcurveto{\pgfqpoint{0.007367in}{-0.027778in}}{\pgfqpoint{0.014433in}{-0.024851in}}{\pgfqpoint{0.019642in}{-0.019642in}}%
\pgfpathcurveto{\pgfqpoint{0.024851in}{-0.014433in}}{\pgfqpoint{0.027778in}{-0.007367in}}{\pgfqpoint{0.027778in}{0.000000in}}%
\pgfpathcurveto{\pgfqpoint{0.027778in}{0.007367in}}{\pgfqpoint{0.024851in}{0.014433in}}{\pgfqpoint{0.019642in}{0.019642in}}%
\pgfpathcurveto{\pgfqpoint{0.014433in}{0.024851in}}{\pgfqpoint{0.007367in}{0.027778in}}{\pgfqpoint{0.000000in}{0.027778in}}%
\pgfpathcurveto{\pgfqpoint{-0.007367in}{0.027778in}}{\pgfqpoint{-0.014433in}{0.024851in}}{\pgfqpoint{-0.019642in}{0.019642in}}%
\pgfpathcurveto{\pgfqpoint{-0.024851in}{0.014433in}}{\pgfqpoint{-0.027778in}{0.007367in}}{\pgfqpoint{-0.027778in}{0.000000in}}%
\pgfpathcurveto{\pgfqpoint{-0.027778in}{-0.007367in}}{\pgfqpoint{-0.024851in}{-0.014433in}}{\pgfqpoint{-0.019642in}{-0.019642in}}%
\pgfpathcurveto{\pgfqpoint{-0.014433in}{-0.024851in}}{\pgfqpoint{-0.007367in}{-0.027778in}}{\pgfqpoint{0.000000in}{-0.027778in}}%
\pgfpathclose%
\pgfusepath{stroke,fill}%
}%
\begin{pgfscope}%
\pgfsys@transformshift{2.532736in}{3.374442in}%
\pgfsys@useobject{currentmarker}{}%
\end{pgfscope}%
\end{pgfscope}%
\begin{pgfscope}%
\pgfpathrectangle{\pgfqpoint{0.100000in}{2.413063in}}{\pgfqpoint{5.037500in}{3.427208in}}%
\pgfusepath{clip}%
\pgfsetrectcap%
\pgfsetroundjoin%
\pgfsetlinewidth{1.505625pt}%
\definecolor{currentstroke}{rgb}{0.000000,0.000000,1.000000}%
\pgfsetstrokecolor{currentstroke}%
\pgfsetstrokeopacity{0.500000}%
\pgfsetdash{}{0pt}%
\pgfpathmoveto{\pgfqpoint{2.334733in}{2.960286in}}%
\pgfusepath{stroke}%
\end{pgfscope}%
\begin{pgfscope}%
\pgfpathrectangle{\pgfqpoint{0.100000in}{2.413063in}}{\pgfqpoint{5.037500in}{3.427208in}}%
\pgfusepath{clip}%
\pgfsetbuttcap%
\pgfsetroundjoin%
\definecolor{currentfill}{rgb}{0.000000,0.000000,1.000000}%
\pgfsetfillcolor{currentfill}%
\pgfsetfillopacity{0.500000}%
\pgfsetlinewidth{0.250937pt}%
\definecolor{currentstroke}{rgb}{0.000000,0.000000,0.000000}%
\pgfsetstrokecolor{currentstroke}%
\pgfsetstrokeopacity{0.500000}%
\pgfsetdash{}{0pt}%
\pgfsys@defobject{currentmarker}{\pgfqpoint{-0.036111in}{-0.036111in}}{\pgfqpoint{0.036111in}{0.036111in}}{%
\pgfpathmoveto{\pgfqpoint{0.000000in}{-0.036111in}}%
\pgfpathcurveto{\pgfqpoint{0.009577in}{-0.036111in}}{\pgfqpoint{0.018763in}{-0.032306in}}{\pgfqpoint{0.025534in}{-0.025534in}}%
\pgfpathcurveto{\pgfqpoint{0.032306in}{-0.018763in}}{\pgfqpoint{0.036111in}{-0.009577in}}{\pgfqpoint{0.036111in}{0.000000in}}%
\pgfpathcurveto{\pgfqpoint{0.036111in}{0.009577in}}{\pgfqpoint{0.032306in}{0.018763in}}{\pgfqpoint{0.025534in}{0.025534in}}%
\pgfpathcurveto{\pgfqpoint{0.018763in}{0.032306in}}{\pgfqpoint{0.009577in}{0.036111in}}{\pgfqpoint{0.000000in}{0.036111in}}%
\pgfpathcurveto{\pgfqpoint{-0.009577in}{0.036111in}}{\pgfqpoint{-0.018763in}{0.032306in}}{\pgfqpoint{-0.025534in}{0.025534in}}%
\pgfpathcurveto{\pgfqpoint{-0.032306in}{0.018763in}}{\pgfqpoint{-0.036111in}{0.009577in}}{\pgfqpoint{-0.036111in}{0.000000in}}%
\pgfpathcurveto{\pgfqpoint{-0.036111in}{-0.009577in}}{\pgfqpoint{-0.032306in}{-0.018763in}}{\pgfqpoint{-0.025534in}{-0.025534in}}%
\pgfpathcurveto{\pgfqpoint{-0.018763in}{-0.032306in}}{\pgfqpoint{-0.009577in}{-0.036111in}}{\pgfqpoint{0.000000in}{-0.036111in}}%
\pgfpathclose%
\pgfusepath{stroke,fill}%
}%
\begin{pgfscope}%
\pgfsys@transformshift{2.334733in}{2.960286in}%
\pgfsys@useobject{currentmarker}{}%
\end{pgfscope}%
\end{pgfscope}%
\begin{pgfscope}%
\pgfpathrectangle{\pgfqpoint{0.100000in}{2.413063in}}{\pgfqpoint{5.037500in}{3.427208in}}%
\pgfusepath{clip}%
\pgfsetrectcap%
\pgfsetroundjoin%
\pgfsetlinewidth{1.505625pt}%
\definecolor{currentstroke}{rgb}{0.000000,0.000000,1.000000}%
\pgfsetstrokecolor{currentstroke}%
\pgfsetstrokeopacity{0.500000}%
\pgfsetdash{}{0pt}%
\pgfpathmoveto{\pgfqpoint{2.829493in}{3.530973in}}%
\pgfusepath{stroke}%
\end{pgfscope}%
\begin{pgfscope}%
\pgfpathrectangle{\pgfqpoint{0.100000in}{2.413063in}}{\pgfqpoint{5.037500in}{3.427208in}}%
\pgfusepath{clip}%
\pgfsetbuttcap%
\pgfsetroundjoin%
\definecolor{currentfill}{rgb}{0.000000,0.000000,1.000000}%
\pgfsetfillcolor{currentfill}%
\pgfsetfillopacity{0.500000}%
\pgfsetlinewidth{0.250937pt}%
\definecolor{currentstroke}{rgb}{0.000000,0.000000,0.000000}%
\pgfsetstrokecolor{currentstroke}%
\pgfsetstrokeopacity{0.500000}%
\pgfsetdash{}{0pt}%
\pgfsys@defobject{currentmarker}{\pgfqpoint{-0.033333in}{-0.033333in}}{\pgfqpoint{0.033333in}{0.033333in}}{%
\pgfpathmoveto{\pgfqpoint{0.000000in}{-0.033333in}}%
\pgfpathcurveto{\pgfqpoint{0.008840in}{-0.033333in}}{\pgfqpoint{0.017319in}{-0.029821in}}{\pgfqpoint{0.023570in}{-0.023570in}}%
\pgfpathcurveto{\pgfqpoint{0.029821in}{-0.017319in}}{\pgfqpoint{0.033333in}{-0.008840in}}{\pgfqpoint{0.033333in}{0.000000in}}%
\pgfpathcurveto{\pgfqpoint{0.033333in}{0.008840in}}{\pgfqpoint{0.029821in}{0.017319in}}{\pgfqpoint{0.023570in}{0.023570in}}%
\pgfpathcurveto{\pgfqpoint{0.017319in}{0.029821in}}{\pgfqpoint{0.008840in}{0.033333in}}{\pgfqpoint{0.000000in}{0.033333in}}%
\pgfpathcurveto{\pgfqpoint{-0.008840in}{0.033333in}}{\pgfqpoint{-0.017319in}{0.029821in}}{\pgfqpoint{-0.023570in}{0.023570in}}%
\pgfpathcurveto{\pgfqpoint{-0.029821in}{0.017319in}}{\pgfqpoint{-0.033333in}{0.008840in}}{\pgfqpoint{-0.033333in}{0.000000in}}%
\pgfpathcurveto{\pgfqpoint{-0.033333in}{-0.008840in}}{\pgfqpoint{-0.029821in}{-0.017319in}}{\pgfqpoint{-0.023570in}{-0.023570in}}%
\pgfpathcurveto{\pgfqpoint{-0.017319in}{-0.029821in}}{\pgfqpoint{-0.008840in}{-0.033333in}}{\pgfqpoint{0.000000in}{-0.033333in}}%
\pgfpathclose%
\pgfusepath{stroke,fill}%
}%
\begin{pgfscope}%
\pgfsys@transformshift{2.829493in}{3.530973in}%
\pgfsys@useobject{currentmarker}{}%
\end{pgfscope}%
\end{pgfscope}%
\begin{pgfscope}%
\pgfpathrectangle{\pgfqpoint{0.100000in}{2.413063in}}{\pgfqpoint{5.037500in}{3.427208in}}%
\pgfusepath{clip}%
\pgfsetrectcap%
\pgfsetroundjoin%
\pgfsetlinewidth{1.505625pt}%
\definecolor{currentstroke}{rgb}{0.000000,0.000000,1.000000}%
\pgfsetstrokecolor{currentstroke}%
\pgfsetstrokeopacity{0.500000}%
\pgfsetdash{}{0pt}%
\pgfpathmoveto{\pgfqpoint{2.144143in}{3.680550in}}%
\pgfusepath{stroke}%
\end{pgfscope}%
\begin{pgfscope}%
\pgfpathrectangle{\pgfqpoint{0.100000in}{2.413063in}}{\pgfqpoint{5.037500in}{3.427208in}}%
\pgfusepath{clip}%
\pgfsetbuttcap%
\pgfsetroundjoin%
\definecolor{currentfill}{rgb}{0.000000,0.000000,1.000000}%
\pgfsetfillcolor{currentfill}%
\pgfsetfillopacity{0.500000}%
\pgfsetlinewidth{0.250937pt}%
\definecolor{currentstroke}{rgb}{0.000000,0.000000,0.000000}%
\pgfsetstrokecolor{currentstroke}%
\pgfsetstrokeopacity{0.500000}%
\pgfsetdash{}{0pt}%
\pgfsys@defobject{currentmarker}{\pgfqpoint{-0.025000in}{-0.025000in}}{\pgfqpoint{0.025000in}{0.025000in}}{%
\pgfpathmoveto{\pgfqpoint{0.000000in}{-0.025000in}}%
\pgfpathcurveto{\pgfqpoint{0.006630in}{-0.025000in}}{\pgfqpoint{0.012989in}{-0.022366in}}{\pgfqpoint{0.017678in}{-0.017678in}}%
\pgfpathcurveto{\pgfqpoint{0.022366in}{-0.012989in}}{\pgfqpoint{0.025000in}{-0.006630in}}{\pgfqpoint{0.025000in}{0.000000in}}%
\pgfpathcurveto{\pgfqpoint{0.025000in}{0.006630in}}{\pgfqpoint{0.022366in}{0.012989in}}{\pgfqpoint{0.017678in}{0.017678in}}%
\pgfpathcurveto{\pgfqpoint{0.012989in}{0.022366in}}{\pgfqpoint{0.006630in}{0.025000in}}{\pgfqpoint{0.000000in}{0.025000in}}%
\pgfpathcurveto{\pgfqpoint{-0.006630in}{0.025000in}}{\pgfqpoint{-0.012989in}{0.022366in}}{\pgfqpoint{-0.017678in}{0.017678in}}%
\pgfpathcurveto{\pgfqpoint{-0.022366in}{0.012989in}}{\pgfqpoint{-0.025000in}{0.006630in}}{\pgfqpoint{-0.025000in}{0.000000in}}%
\pgfpathcurveto{\pgfqpoint{-0.025000in}{-0.006630in}}{\pgfqpoint{-0.022366in}{-0.012989in}}{\pgfqpoint{-0.017678in}{-0.017678in}}%
\pgfpathcurveto{\pgfqpoint{-0.012989in}{-0.022366in}}{\pgfqpoint{-0.006630in}{-0.025000in}}{\pgfqpoint{0.000000in}{-0.025000in}}%
\pgfpathclose%
\pgfusepath{stroke,fill}%
}%
\begin{pgfscope}%
\pgfsys@transformshift{2.144143in}{3.680550in}%
\pgfsys@useobject{currentmarker}{}%
\end{pgfscope}%
\end{pgfscope}%
\begin{pgfscope}%
\pgfpathrectangle{\pgfqpoint{0.100000in}{2.413063in}}{\pgfqpoint{5.037500in}{3.427208in}}%
\pgfusepath{clip}%
\pgfsetrectcap%
\pgfsetroundjoin%
\pgfsetlinewidth{1.505625pt}%
\definecolor{currentstroke}{rgb}{0.000000,0.000000,1.000000}%
\pgfsetstrokecolor{currentstroke}%
\pgfsetstrokeopacity{0.500000}%
\pgfsetdash{}{0pt}%
\pgfpathmoveto{\pgfqpoint{2.462352in}{2.801522in}}%
\pgfusepath{stroke}%
\end{pgfscope}%
\begin{pgfscope}%
\pgfpathrectangle{\pgfqpoint{0.100000in}{2.413063in}}{\pgfqpoint{5.037500in}{3.427208in}}%
\pgfusepath{clip}%
\pgfsetbuttcap%
\pgfsetroundjoin%
\definecolor{currentfill}{rgb}{0.000000,0.000000,1.000000}%
\pgfsetfillcolor{currentfill}%
\pgfsetfillopacity{0.500000}%
\pgfsetlinewidth{0.250937pt}%
\definecolor{currentstroke}{rgb}{0.000000,0.000000,0.000000}%
\pgfsetstrokecolor{currentstroke}%
\pgfsetstrokeopacity{0.500000}%
\pgfsetdash{}{0pt}%
\pgfsys@defobject{currentmarker}{\pgfqpoint{-0.025000in}{-0.025000in}}{\pgfqpoint{0.025000in}{0.025000in}}{%
\pgfpathmoveto{\pgfqpoint{0.000000in}{-0.025000in}}%
\pgfpathcurveto{\pgfqpoint{0.006630in}{-0.025000in}}{\pgfqpoint{0.012989in}{-0.022366in}}{\pgfqpoint{0.017678in}{-0.017678in}}%
\pgfpathcurveto{\pgfqpoint{0.022366in}{-0.012989in}}{\pgfqpoint{0.025000in}{-0.006630in}}{\pgfqpoint{0.025000in}{0.000000in}}%
\pgfpathcurveto{\pgfqpoint{0.025000in}{0.006630in}}{\pgfqpoint{0.022366in}{0.012989in}}{\pgfqpoint{0.017678in}{0.017678in}}%
\pgfpathcurveto{\pgfqpoint{0.012989in}{0.022366in}}{\pgfqpoint{0.006630in}{0.025000in}}{\pgfqpoint{0.000000in}{0.025000in}}%
\pgfpathcurveto{\pgfqpoint{-0.006630in}{0.025000in}}{\pgfqpoint{-0.012989in}{0.022366in}}{\pgfqpoint{-0.017678in}{0.017678in}}%
\pgfpathcurveto{\pgfqpoint{-0.022366in}{0.012989in}}{\pgfqpoint{-0.025000in}{0.006630in}}{\pgfqpoint{-0.025000in}{0.000000in}}%
\pgfpathcurveto{\pgfqpoint{-0.025000in}{-0.006630in}}{\pgfqpoint{-0.022366in}{-0.012989in}}{\pgfqpoint{-0.017678in}{-0.017678in}}%
\pgfpathcurveto{\pgfqpoint{-0.012989in}{-0.022366in}}{\pgfqpoint{-0.006630in}{-0.025000in}}{\pgfqpoint{0.000000in}{-0.025000in}}%
\pgfpathclose%
\pgfusepath{stroke,fill}%
}%
\begin{pgfscope}%
\pgfsys@transformshift{2.462352in}{2.801522in}%
\pgfsys@useobject{currentmarker}{}%
\end{pgfscope}%
\end{pgfscope}%
\begin{pgfscope}%
\pgfpathrectangle{\pgfqpoint{0.100000in}{2.413063in}}{\pgfqpoint{5.037500in}{3.427208in}}%
\pgfusepath{clip}%
\pgfsetrectcap%
\pgfsetroundjoin%
\pgfsetlinewidth{1.505625pt}%
\definecolor{currentstroke}{rgb}{0.000000,0.000000,1.000000}%
\pgfsetstrokecolor{currentstroke}%
\pgfsetstrokeopacity{0.500000}%
\pgfsetdash{}{0pt}%
\pgfpathmoveto{\pgfqpoint{2.108493in}{3.499680in}}%
\pgfusepath{stroke}%
\end{pgfscope}%
\begin{pgfscope}%
\pgfpathrectangle{\pgfqpoint{0.100000in}{2.413063in}}{\pgfqpoint{5.037500in}{3.427208in}}%
\pgfusepath{clip}%
\pgfsetbuttcap%
\pgfsetroundjoin%
\definecolor{currentfill}{rgb}{0.000000,0.000000,1.000000}%
\pgfsetfillcolor{currentfill}%
\pgfsetfillopacity{0.500000}%
\pgfsetlinewidth{0.250937pt}%
\definecolor{currentstroke}{rgb}{0.000000,0.000000,0.000000}%
\pgfsetstrokecolor{currentstroke}%
\pgfsetstrokeopacity{0.500000}%
\pgfsetdash{}{0pt}%
\pgfsys@defobject{currentmarker}{\pgfqpoint{-0.055556in}{-0.055556in}}{\pgfqpoint{0.055556in}{0.055556in}}{%
\pgfpathmoveto{\pgfqpoint{0.000000in}{-0.055556in}}%
\pgfpathcurveto{\pgfqpoint{0.014734in}{-0.055556in}}{\pgfqpoint{0.028866in}{-0.049702in}}{\pgfqpoint{0.039284in}{-0.039284in}}%
\pgfpathcurveto{\pgfqpoint{0.049702in}{-0.028866in}}{\pgfqpoint{0.055556in}{-0.014734in}}{\pgfqpoint{0.055556in}{0.000000in}}%
\pgfpathcurveto{\pgfqpoint{0.055556in}{0.014734in}}{\pgfqpoint{0.049702in}{0.028866in}}{\pgfqpoint{0.039284in}{0.039284in}}%
\pgfpathcurveto{\pgfqpoint{0.028866in}{0.049702in}}{\pgfqpoint{0.014734in}{0.055556in}}{\pgfqpoint{0.000000in}{0.055556in}}%
\pgfpathcurveto{\pgfqpoint{-0.014734in}{0.055556in}}{\pgfqpoint{-0.028866in}{0.049702in}}{\pgfqpoint{-0.039284in}{0.039284in}}%
\pgfpathcurveto{\pgfqpoint{-0.049702in}{0.028866in}}{\pgfqpoint{-0.055556in}{0.014734in}}{\pgfqpoint{-0.055556in}{0.000000in}}%
\pgfpathcurveto{\pgfqpoint{-0.055556in}{-0.014734in}}{\pgfqpoint{-0.049702in}{-0.028866in}}{\pgfqpoint{-0.039284in}{-0.039284in}}%
\pgfpathcurveto{\pgfqpoint{-0.028866in}{-0.049702in}}{\pgfqpoint{-0.014734in}{-0.055556in}}{\pgfqpoint{0.000000in}{-0.055556in}}%
\pgfpathclose%
\pgfusepath{stroke,fill}%
}%
\begin{pgfscope}%
\pgfsys@transformshift{2.108493in}{3.499680in}%
\pgfsys@useobject{currentmarker}{}%
\end{pgfscope}%
\end{pgfscope}%
\begin{pgfscope}%
\pgfpathrectangle{\pgfqpoint{0.100000in}{2.413063in}}{\pgfqpoint{5.037500in}{3.427208in}}%
\pgfusepath{clip}%
\pgfsetrectcap%
\pgfsetroundjoin%
\pgfsetlinewidth{1.505625pt}%
\definecolor{currentstroke}{rgb}{0.000000,0.000000,1.000000}%
\pgfsetstrokecolor{currentstroke}%
\pgfsetstrokeopacity{0.500000}%
\pgfsetdash{}{0pt}%
\pgfpathmoveto{\pgfqpoint{2.078653in}{3.484405in}}%
\pgfusepath{stroke}%
\end{pgfscope}%
\begin{pgfscope}%
\pgfpathrectangle{\pgfqpoint{0.100000in}{2.413063in}}{\pgfqpoint{5.037500in}{3.427208in}}%
\pgfusepath{clip}%
\pgfsetbuttcap%
\pgfsetroundjoin%
\definecolor{currentfill}{rgb}{0.000000,0.000000,1.000000}%
\pgfsetfillcolor{currentfill}%
\pgfsetfillopacity{0.500000}%
\pgfsetlinewidth{0.250937pt}%
\definecolor{currentstroke}{rgb}{0.000000,0.000000,0.000000}%
\pgfsetstrokecolor{currentstroke}%
\pgfsetstrokeopacity{0.500000}%
\pgfsetdash{}{0pt}%
\pgfsys@defobject{currentmarker}{\pgfqpoint{-0.083333in}{-0.083333in}}{\pgfqpoint{0.083333in}{0.083333in}}{%
\pgfpathmoveto{\pgfqpoint{0.000000in}{-0.083333in}}%
\pgfpathcurveto{\pgfqpoint{0.022100in}{-0.083333in}}{\pgfqpoint{0.043298in}{-0.074553in}}{\pgfqpoint{0.058926in}{-0.058926in}}%
\pgfpathcurveto{\pgfqpoint{0.074553in}{-0.043298in}}{\pgfqpoint{0.083333in}{-0.022100in}}{\pgfqpoint{0.083333in}{0.000000in}}%
\pgfpathcurveto{\pgfqpoint{0.083333in}{0.022100in}}{\pgfqpoint{0.074553in}{0.043298in}}{\pgfqpoint{0.058926in}{0.058926in}}%
\pgfpathcurveto{\pgfqpoint{0.043298in}{0.074553in}}{\pgfqpoint{0.022100in}{0.083333in}}{\pgfqpoint{0.000000in}{0.083333in}}%
\pgfpathcurveto{\pgfqpoint{-0.022100in}{0.083333in}}{\pgfqpoint{-0.043298in}{0.074553in}}{\pgfqpoint{-0.058926in}{0.058926in}}%
\pgfpathcurveto{\pgfqpoint{-0.074553in}{0.043298in}}{\pgfqpoint{-0.083333in}{0.022100in}}{\pgfqpoint{-0.083333in}{0.000000in}}%
\pgfpathcurveto{\pgfqpoint{-0.083333in}{-0.022100in}}{\pgfqpoint{-0.074553in}{-0.043298in}}{\pgfqpoint{-0.058926in}{-0.058926in}}%
\pgfpathcurveto{\pgfqpoint{-0.043298in}{-0.074553in}}{\pgfqpoint{-0.022100in}{-0.083333in}}{\pgfqpoint{0.000000in}{-0.083333in}}%
\pgfpathclose%
\pgfusepath{stroke,fill}%
}%
\begin{pgfscope}%
\pgfsys@transformshift{2.078653in}{3.484405in}%
\pgfsys@useobject{currentmarker}{}%
\end{pgfscope}%
\end{pgfscope}%
\begin{pgfscope}%
\pgfpathrectangle{\pgfqpoint{0.100000in}{2.413063in}}{\pgfqpoint{5.037500in}{3.427208in}}%
\pgfusepath{clip}%
\pgfsetrectcap%
\pgfsetroundjoin%
\pgfsetlinewidth{1.505625pt}%
\definecolor{currentstroke}{rgb}{0.000000,0.000000,1.000000}%
\pgfsetstrokecolor{currentstroke}%
\pgfsetstrokeopacity{0.500000}%
\pgfsetdash{}{0pt}%
\pgfpathmoveto{\pgfqpoint{2.265570in}{3.426888in}}%
\pgfusepath{stroke}%
\end{pgfscope}%
\begin{pgfscope}%
\pgfpathrectangle{\pgfqpoint{0.100000in}{2.413063in}}{\pgfqpoint{5.037500in}{3.427208in}}%
\pgfusepath{clip}%
\pgfsetbuttcap%
\pgfsetroundjoin%
\definecolor{currentfill}{rgb}{0.000000,0.000000,1.000000}%
\pgfsetfillcolor{currentfill}%
\pgfsetfillopacity{0.500000}%
\pgfsetlinewidth{0.250937pt}%
\definecolor{currentstroke}{rgb}{0.000000,0.000000,0.000000}%
\pgfsetstrokecolor{currentstroke}%
\pgfsetstrokeopacity{0.500000}%
\pgfsetdash{}{0pt}%
\pgfsys@defobject{currentmarker}{\pgfqpoint{-0.022222in}{-0.022222in}}{\pgfqpoint{0.022222in}{0.022222in}}{%
\pgfpathmoveto{\pgfqpoint{0.000000in}{-0.022222in}}%
\pgfpathcurveto{\pgfqpoint{0.005893in}{-0.022222in}}{\pgfqpoint{0.011546in}{-0.019881in}}{\pgfqpoint{0.015713in}{-0.015713in}}%
\pgfpathcurveto{\pgfqpoint{0.019881in}{-0.011546in}}{\pgfqpoint{0.022222in}{-0.005893in}}{\pgfqpoint{0.022222in}{0.000000in}}%
\pgfpathcurveto{\pgfqpoint{0.022222in}{0.005893in}}{\pgfqpoint{0.019881in}{0.011546in}}{\pgfqpoint{0.015713in}{0.015713in}}%
\pgfpathcurveto{\pgfqpoint{0.011546in}{0.019881in}}{\pgfqpoint{0.005893in}{0.022222in}}{\pgfqpoint{0.000000in}{0.022222in}}%
\pgfpathcurveto{\pgfqpoint{-0.005893in}{0.022222in}}{\pgfqpoint{-0.011546in}{0.019881in}}{\pgfqpoint{-0.015713in}{0.015713in}}%
\pgfpathcurveto{\pgfqpoint{-0.019881in}{0.011546in}}{\pgfqpoint{-0.022222in}{0.005893in}}{\pgfqpoint{-0.022222in}{0.000000in}}%
\pgfpathcurveto{\pgfqpoint{-0.022222in}{-0.005893in}}{\pgfqpoint{-0.019881in}{-0.011546in}}{\pgfqpoint{-0.015713in}{-0.015713in}}%
\pgfpathcurveto{\pgfqpoint{-0.011546in}{-0.019881in}}{\pgfqpoint{-0.005893in}{-0.022222in}}{\pgfqpoint{0.000000in}{-0.022222in}}%
\pgfpathclose%
\pgfusepath{stroke,fill}%
}%
\begin{pgfscope}%
\pgfsys@transformshift{2.265570in}{3.426888in}%
\pgfsys@useobject{currentmarker}{}%
\end{pgfscope}%
\end{pgfscope}%
\begin{pgfscope}%
\pgfpathrectangle{\pgfqpoint{0.100000in}{2.413063in}}{\pgfqpoint{5.037500in}{3.427208in}}%
\pgfusepath{clip}%
\pgfsetrectcap%
\pgfsetroundjoin%
\pgfsetlinewidth{1.505625pt}%
\definecolor{currentstroke}{rgb}{0.000000,0.000000,1.000000}%
\pgfsetstrokecolor{currentstroke}%
\pgfsetstrokeopacity{0.500000}%
\pgfsetdash{}{0pt}%
\pgfpathmoveto{\pgfqpoint{2.448846in}{3.180064in}}%
\pgfusepath{stroke}%
\end{pgfscope}%
\begin{pgfscope}%
\pgfpathrectangle{\pgfqpoint{0.100000in}{2.413063in}}{\pgfqpoint{5.037500in}{3.427208in}}%
\pgfusepath{clip}%
\pgfsetbuttcap%
\pgfsetroundjoin%
\definecolor{currentfill}{rgb}{0.000000,0.000000,1.000000}%
\pgfsetfillcolor{currentfill}%
\pgfsetfillopacity{0.500000}%
\pgfsetlinewidth{0.250937pt}%
\definecolor{currentstroke}{rgb}{0.000000,0.000000,0.000000}%
\pgfsetstrokecolor{currentstroke}%
\pgfsetstrokeopacity{0.500000}%
\pgfsetdash{}{0pt}%
\pgfsys@defobject{currentmarker}{\pgfqpoint{-0.025000in}{-0.025000in}}{\pgfqpoint{0.025000in}{0.025000in}}{%
\pgfpathmoveto{\pgfqpoint{0.000000in}{-0.025000in}}%
\pgfpathcurveto{\pgfqpoint{0.006630in}{-0.025000in}}{\pgfqpoint{0.012989in}{-0.022366in}}{\pgfqpoint{0.017678in}{-0.017678in}}%
\pgfpathcurveto{\pgfqpoint{0.022366in}{-0.012989in}}{\pgfqpoint{0.025000in}{-0.006630in}}{\pgfqpoint{0.025000in}{0.000000in}}%
\pgfpathcurveto{\pgfqpoint{0.025000in}{0.006630in}}{\pgfqpoint{0.022366in}{0.012989in}}{\pgfqpoint{0.017678in}{0.017678in}}%
\pgfpathcurveto{\pgfqpoint{0.012989in}{0.022366in}}{\pgfqpoint{0.006630in}{0.025000in}}{\pgfqpoint{0.000000in}{0.025000in}}%
\pgfpathcurveto{\pgfqpoint{-0.006630in}{0.025000in}}{\pgfqpoint{-0.012989in}{0.022366in}}{\pgfqpoint{-0.017678in}{0.017678in}}%
\pgfpathcurveto{\pgfqpoint{-0.022366in}{0.012989in}}{\pgfqpoint{-0.025000in}{0.006630in}}{\pgfqpoint{-0.025000in}{0.000000in}}%
\pgfpathcurveto{\pgfqpoint{-0.025000in}{-0.006630in}}{\pgfqpoint{-0.022366in}{-0.012989in}}{\pgfqpoint{-0.017678in}{-0.017678in}}%
\pgfpathcurveto{\pgfqpoint{-0.012989in}{-0.022366in}}{\pgfqpoint{-0.006630in}{-0.025000in}}{\pgfqpoint{0.000000in}{-0.025000in}}%
\pgfpathclose%
\pgfusepath{stroke,fill}%
}%
\begin{pgfscope}%
\pgfsys@transformshift{2.448846in}{3.180064in}%
\pgfsys@useobject{currentmarker}{}%
\end{pgfscope}%
\end{pgfscope}%
\begin{pgfscope}%
\pgfpathrectangle{\pgfqpoint{0.100000in}{2.413063in}}{\pgfqpoint{5.037500in}{3.427208in}}%
\pgfusepath{clip}%
\pgfsetrectcap%
\pgfsetroundjoin%
\pgfsetlinewidth{1.505625pt}%
\definecolor{currentstroke}{rgb}{0.000000,0.000000,1.000000}%
\pgfsetstrokecolor{currentstroke}%
\pgfsetstrokeopacity{0.500000}%
\pgfsetdash{}{0pt}%
\pgfpathmoveto{\pgfqpoint{2.649125in}{3.663744in}}%
\pgfusepath{stroke}%
\end{pgfscope}%
\begin{pgfscope}%
\pgfpathrectangle{\pgfqpoint{0.100000in}{2.413063in}}{\pgfqpoint{5.037500in}{3.427208in}}%
\pgfusepath{clip}%
\pgfsetbuttcap%
\pgfsetroundjoin%
\definecolor{currentfill}{rgb}{0.000000,0.000000,1.000000}%
\pgfsetfillcolor{currentfill}%
\pgfsetfillopacity{0.500000}%
\pgfsetlinewidth{0.250937pt}%
\definecolor{currentstroke}{rgb}{0.000000,0.000000,0.000000}%
\pgfsetstrokecolor{currentstroke}%
\pgfsetstrokeopacity{0.500000}%
\pgfsetdash{}{0pt}%
\pgfsys@defobject{currentmarker}{\pgfqpoint{-0.016667in}{-0.016667in}}{\pgfqpoint{0.016667in}{0.016667in}}{%
\pgfpathmoveto{\pgfqpoint{0.000000in}{-0.016667in}}%
\pgfpathcurveto{\pgfqpoint{0.004420in}{-0.016667in}}{\pgfqpoint{0.008660in}{-0.014911in}}{\pgfqpoint{0.011785in}{-0.011785in}}%
\pgfpathcurveto{\pgfqpoint{0.014911in}{-0.008660in}}{\pgfqpoint{0.016667in}{-0.004420in}}{\pgfqpoint{0.016667in}{0.000000in}}%
\pgfpathcurveto{\pgfqpoint{0.016667in}{0.004420in}}{\pgfqpoint{0.014911in}{0.008660in}}{\pgfqpoint{0.011785in}{0.011785in}}%
\pgfpathcurveto{\pgfqpoint{0.008660in}{0.014911in}}{\pgfqpoint{0.004420in}{0.016667in}}{\pgfqpoint{0.000000in}{0.016667in}}%
\pgfpathcurveto{\pgfqpoint{-0.004420in}{0.016667in}}{\pgfqpoint{-0.008660in}{0.014911in}}{\pgfqpoint{-0.011785in}{0.011785in}}%
\pgfpathcurveto{\pgfqpoint{-0.014911in}{0.008660in}}{\pgfqpoint{-0.016667in}{0.004420in}}{\pgfqpoint{-0.016667in}{0.000000in}}%
\pgfpathcurveto{\pgfqpoint{-0.016667in}{-0.004420in}}{\pgfqpoint{-0.014911in}{-0.008660in}}{\pgfqpoint{-0.011785in}{-0.011785in}}%
\pgfpathcurveto{\pgfqpoint{-0.008660in}{-0.014911in}}{\pgfqpoint{-0.004420in}{-0.016667in}}{\pgfqpoint{0.000000in}{-0.016667in}}%
\pgfpathclose%
\pgfusepath{stroke,fill}%
}%
\begin{pgfscope}%
\pgfsys@transformshift{2.649125in}{3.663744in}%
\pgfsys@useobject{currentmarker}{}%
\end{pgfscope}%
\end{pgfscope}%
\begin{pgfscope}%
\pgfpathrectangle{\pgfqpoint{0.100000in}{2.413063in}}{\pgfqpoint{5.037500in}{3.427208in}}%
\pgfusepath{clip}%
\pgfsetrectcap%
\pgfsetroundjoin%
\pgfsetlinewidth{1.505625pt}%
\definecolor{currentstroke}{rgb}{0.000000,0.000000,1.000000}%
\pgfsetstrokecolor{currentstroke}%
\pgfsetstrokeopacity{0.500000}%
\pgfsetdash{}{0pt}%
\pgfpathmoveto{\pgfqpoint{2.896083in}{3.638458in}}%
\pgfusepath{stroke}%
\end{pgfscope}%
\begin{pgfscope}%
\pgfpathrectangle{\pgfqpoint{0.100000in}{2.413063in}}{\pgfqpoint{5.037500in}{3.427208in}}%
\pgfusepath{clip}%
\pgfsetbuttcap%
\pgfsetroundjoin%
\definecolor{currentfill}{rgb}{0.000000,0.000000,1.000000}%
\pgfsetfillcolor{currentfill}%
\pgfsetfillopacity{0.500000}%
\pgfsetlinewidth{0.250937pt}%
\definecolor{currentstroke}{rgb}{0.000000,0.000000,0.000000}%
\pgfsetstrokecolor{currentstroke}%
\pgfsetstrokeopacity{0.500000}%
\pgfsetdash{}{0pt}%
\pgfsys@defobject{currentmarker}{\pgfqpoint{-0.008333in}{-0.008333in}}{\pgfqpoint{0.008333in}{0.008333in}}{%
\pgfpathmoveto{\pgfqpoint{0.000000in}{-0.008333in}}%
\pgfpathcurveto{\pgfqpoint{0.002210in}{-0.008333in}}{\pgfqpoint{0.004330in}{-0.007455in}}{\pgfqpoint{0.005893in}{-0.005893in}}%
\pgfpathcurveto{\pgfqpoint{0.007455in}{-0.004330in}}{\pgfqpoint{0.008333in}{-0.002210in}}{\pgfqpoint{0.008333in}{0.000000in}}%
\pgfpathcurveto{\pgfqpoint{0.008333in}{0.002210in}}{\pgfqpoint{0.007455in}{0.004330in}}{\pgfqpoint{0.005893in}{0.005893in}}%
\pgfpathcurveto{\pgfqpoint{0.004330in}{0.007455in}}{\pgfqpoint{0.002210in}{0.008333in}}{\pgfqpoint{0.000000in}{0.008333in}}%
\pgfpathcurveto{\pgfqpoint{-0.002210in}{0.008333in}}{\pgfqpoint{-0.004330in}{0.007455in}}{\pgfqpoint{-0.005893in}{0.005893in}}%
\pgfpathcurveto{\pgfqpoint{-0.007455in}{0.004330in}}{\pgfqpoint{-0.008333in}{0.002210in}}{\pgfqpoint{-0.008333in}{0.000000in}}%
\pgfpathcurveto{\pgfqpoint{-0.008333in}{-0.002210in}}{\pgfqpoint{-0.007455in}{-0.004330in}}{\pgfqpoint{-0.005893in}{-0.005893in}}%
\pgfpathcurveto{\pgfqpoint{-0.004330in}{-0.007455in}}{\pgfqpoint{-0.002210in}{-0.008333in}}{\pgfqpoint{0.000000in}{-0.008333in}}%
\pgfpathclose%
\pgfusepath{stroke,fill}%
}%
\begin{pgfscope}%
\pgfsys@transformshift{2.896083in}{3.638458in}%
\pgfsys@useobject{currentmarker}{}%
\end{pgfscope}%
\end{pgfscope}%
\begin{pgfscope}%
\pgfpathrectangle{\pgfqpoint{0.100000in}{2.413063in}}{\pgfqpoint{5.037500in}{3.427208in}}%
\pgfusepath{clip}%
\pgfsetrectcap%
\pgfsetroundjoin%
\pgfsetlinewidth{1.505625pt}%
\definecolor{currentstroke}{rgb}{0.000000,0.000000,1.000000}%
\pgfsetstrokecolor{currentstroke}%
\pgfsetstrokeopacity{0.500000}%
\pgfsetdash{}{0pt}%
\pgfpathmoveto{\pgfqpoint{2.774645in}{3.513664in}}%
\pgfusepath{stroke}%
\end{pgfscope}%
\begin{pgfscope}%
\pgfpathrectangle{\pgfqpoint{0.100000in}{2.413063in}}{\pgfqpoint{5.037500in}{3.427208in}}%
\pgfusepath{clip}%
\pgfsetbuttcap%
\pgfsetroundjoin%
\definecolor{currentfill}{rgb}{0.000000,0.000000,1.000000}%
\pgfsetfillcolor{currentfill}%
\pgfsetfillopacity{0.500000}%
\pgfsetlinewidth{0.250937pt}%
\definecolor{currentstroke}{rgb}{0.000000,0.000000,0.000000}%
\pgfsetstrokecolor{currentstroke}%
\pgfsetstrokeopacity{0.500000}%
\pgfsetdash{}{0pt}%
\pgfsys@defobject{currentmarker}{\pgfqpoint{-0.016667in}{-0.016667in}}{\pgfqpoint{0.016667in}{0.016667in}}{%
\pgfpathmoveto{\pgfqpoint{0.000000in}{-0.016667in}}%
\pgfpathcurveto{\pgfqpoint{0.004420in}{-0.016667in}}{\pgfqpoint{0.008660in}{-0.014911in}}{\pgfqpoint{0.011785in}{-0.011785in}}%
\pgfpathcurveto{\pgfqpoint{0.014911in}{-0.008660in}}{\pgfqpoint{0.016667in}{-0.004420in}}{\pgfqpoint{0.016667in}{0.000000in}}%
\pgfpathcurveto{\pgfqpoint{0.016667in}{0.004420in}}{\pgfqpoint{0.014911in}{0.008660in}}{\pgfqpoint{0.011785in}{0.011785in}}%
\pgfpathcurveto{\pgfqpoint{0.008660in}{0.014911in}}{\pgfqpoint{0.004420in}{0.016667in}}{\pgfqpoint{0.000000in}{0.016667in}}%
\pgfpathcurveto{\pgfqpoint{-0.004420in}{0.016667in}}{\pgfqpoint{-0.008660in}{0.014911in}}{\pgfqpoint{-0.011785in}{0.011785in}}%
\pgfpathcurveto{\pgfqpoint{-0.014911in}{0.008660in}}{\pgfqpoint{-0.016667in}{0.004420in}}{\pgfqpoint{-0.016667in}{0.000000in}}%
\pgfpathcurveto{\pgfqpoint{-0.016667in}{-0.004420in}}{\pgfqpoint{-0.014911in}{-0.008660in}}{\pgfqpoint{-0.011785in}{-0.011785in}}%
\pgfpathcurveto{\pgfqpoint{-0.008660in}{-0.014911in}}{\pgfqpoint{-0.004420in}{-0.016667in}}{\pgfqpoint{0.000000in}{-0.016667in}}%
\pgfpathclose%
\pgfusepath{stroke,fill}%
}%
\begin{pgfscope}%
\pgfsys@transformshift{2.774645in}{3.513664in}%
\pgfsys@useobject{currentmarker}{}%
\end{pgfscope}%
\end{pgfscope}%
\begin{pgfscope}%
\pgfpathrectangle{\pgfqpoint{0.100000in}{2.413063in}}{\pgfqpoint{5.037500in}{3.427208in}}%
\pgfusepath{clip}%
\pgfsetrectcap%
\pgfsetroundjoin%
\pgfsetlinewidth{1.505625pt}%
\definecolor{currentstroke}{rgb}{0.000000,0.000000,1.000000}%
\pgfsetstrokecolor{currentstroke}%
\pgfsetstrokeopacity{0.500000}%
\pgfsetdash{}{0pt}%
\pgfpathmoveto{\pgfqpoint{2.598833in}{3.103133in}}%
\pgfusepath{stroke}%
\end{pgfscope}%
\begin{pgfscope}%
\pgfpathrectangle{\pgfqpoint{0.100000in}{2.413063in}}{\pgfqpoint{5.037500in}{3.427208in}}%
\pgfusepath{clip}%
\pgfsetbuttcap%
\pgfsetroundjoin%
\definecolor{currentfill}{rgb}{0.000000,0.000000,1.000000}%
\pgfsetfillcolor{currentfill}%
\pgfsetfillopacity{0.500000}%
\pgfsetlinewidth{0.250937pt}%
\definecolor{currentstroke}{rgb}{0.000000,0.000000,0.000000}%
\pgfsetstrokecolor{currentstroke}%
\pgfsetstrokeopacity{0.500000}%
\pgfsetdash{}{0pt}%
\pgfsys@defobject{currentmarker}{\pgfqpoint{-0.044444in}{-0.044444in}}{\pgfqpoint{0.044444in}{0.044444in}}{%
\pgfpathmoveto{\pgfqpoint{0.000000in}{-0.044444in}}%
\pgfpathcurveto{\pgfqpoint{0.011787in}{-0.044444in}}{\pgfqpoint{0.023092in}{-0.039761in}}{\pgfqpoint{0.031427in}{-0.031427in}}%
\pgfpathcurveto{\pgfqpoint{0.039761in}{-0.023092in}}{\pgfqpoint{0.044444in}{-0.011787in}}{\pgfqpoint{0.044444in}{0.000000in}}%
\pgfpathcurveto{\pgfqpoint{0.044444in}{0.011787in}}{\pgfqpoint{0.039761in}{0.023092in}}{\pgfqpoint{0.031427in}{0.031427in}}%
\pgfpathcurveto{\pgfqpoint{0.023092in}{0.039761in}}{\pgfqpoint{0.011787in}{0.044444in}}{\pgfqpoint{0.000000in}{0.044444in}}%
\pgfpathcurveto{\pgfqpoint{-0.011787in}{0.044444in}}{\pgfqpoint{-0.023092in}{0.039761in}}{\pgfqpoint{-0.031427in}{0.031427in}}%
\pgfpathcurveto{\pgfqpoint{-0.039761in}{0.023092in}}{\pgfqpoint{-0.044444in}{0.011787in}}{\pgfqpoint{-0.044444in}{0.000000in}}%
\pgfpathcurveto{\pgfqpoint{-0.044444in}{-0.011787in}}{\pgfqpoint{-0.039761in}{-0.023092in}}{\pgfqpoint{-0.031427in}{-0.031427in}}%
\pgfpathcurveto{\pgfqpoint{-0.023092in}{-0.039761in}}{\pgfqpoint{-0.011787in}{-0.044444in}}{\pgfqpoint{0.000000in}{-0.044444in}}%
\pgfpathclose%
\pgfusepath{stroke,fill}%
}%
\begin{pgfscope}%
\pgfsys@transformshift{2.598833in}{3.103133in}%
\pgfsys@useobject{currentmarker}{}%
\end{pgfscope}%
\end{pgfscope}%
\begin{pgfscope}%
\pgfpathrectangle{\pgfqpoint{0.100000in}{2.413063in}}{\pgfqpoint{5.037500in}{3.427208in}}%
\pgfusepath{clip}%
\pgfsetrectcap%
\pgfsetroundjoin%
\pgfsetlinewidth{1.505625pt}%
\definecolor{currentstroke}{rgb}{0.000000,0.000000,1.000000}%
\pgfsetstrokecolor{currentstroke}%
\pgfsetstrokeopacity{0.500000}%
\pgfsetdash{}{0pt}%
\pgfpathmoveto{\pgfqpoint{2.591727in}{3.423086in}}%
\pgfusepath{stroke}%
\end{pgfscope}%
\begin{pgfscope}%
\pgfpathrectangle{\pgfqpoint{0.100000in}{2.413063in}}{\pgfqpoint{5.037500in}{3.427208in}}%
\pgfusepath{clip}%
\pgfsetbuttcap%
\pgfsetroundjoin%
\definecolor{currentfill}{rgb}{0.000000,0.000000,1.000000}%
\pgfsetfillcolor{currentfill}%
\pgfsetfillopacity{0.500000}%
\pgfsetlinewidth{0.250937pt}%
\definecolor{currentstroke}{rgb}{0.000000,0.000000,0.000000}%
\pgfsetstrokecolor{currentstroke}%
\pgfsetstrokeopacity{0.500000}%
\pgfsetdash{}{0pt}%
\pgfsys@defobject{currentmarker}{\pgfqpoint{-0.013889in}{-0.013889in}}{\pgfqpoint{0.013889in}{0.013889in}}{%
\pgfpathmoveto{\pgfqpoint{0.000000in}{-0.013889in}}%
\pgfpathcurveto{\pgfqpoint{0.003683in}{-0.013889in}}{\pgfqpoint{0.007216in}{-0.012425in}}{\pgfqpoint{0.009821in}{-0.009821in}}%
\pgfpathcurveto{\pgfqpoint{0.012425in}{-0.007216in}}{\pgfqpoint{0.013889in}{-0.003683in}}{\pgfqpoint{0.013889in}{0.000000in}}%
\pgfpathcurveto{\pgfqpoint{0.013889in}{0.003683in}}{\pgfqpoint{0.012425in}{0.007216in}}{\pgfqpoint{0.009821in}{0.009821in}}%
\pgfpathcurveto{\pgfqpoint{0.007216in}{0.012425in}}{\pgfqpoint{0.003683in}{0.013889in}}{\pgfqpoint{0.000000in}{0.013889in}}%
\pgfpathcurveto{\pgfqpoint{-0.003683in}{0.013889in}}{\pgfqpoint{-0.007216in}{0.012425in}}{\pgfqpoint{-0.009821in}{0.009821in}}%
\pgfpathcurveto{\pgfqpoint{-0.012425in}{0.007216in}}{\pgfqpoint{-0.013889in}{0.003683in}}{\pgfqpoint{-0.013889in}{0.000000in}}%
\pgfpathcurveto{\pgfqpoint{-0.013889in}{-0.003683in}}{\pgfqpoint{-0.012425in}{-0.007216in}}{\pgfqpoint{-0.009821in}{-0.009821in}}%
\pgfpathcurveto{\pgfqpoint{-0.007216in}{-0.012425in}}{\pgfqpoint{-0.003683in}{-0.013889in}}{\pgfqpoint{0.000000in}{-0.013889in}}%
\pgfpathclose%
\pgfusepath{stroke,fill}%
}%
\begin{pgfscope}%
\pgfsys@transformshift{2.591727in}{3.423086in}%
\pgfsys@useobject{currentmarker}{}%
\end{pgfscope}%
\end{pgfscope}%
\begin{pgfscope}%
\pgfpathrectangle{\pgfqpoint{0.100000in}{2.413063in}}{\pgfqpoint{5.037500in}{3.427208in}}%
\pgfusepath{clip}%
\pgfsetrectcap%
\pgfsetroundjoin%
\pgfsetlinewidth{1.505625pt}%
\definecolor{currentstroke}{rgb}{0.000000,0.000000,1.000000}%
\pgfsetstrokecolor{currentstroke}%
\pgfsetstrokeopacity{0.500000}%
\pgfsetdash{}{0pt}%
\pgfpathmoveto{\pgfqpoint{2.469056in}{3.700922in}}%
\pgfusepath{stroke}%
\end{pgfscope}%
\begin{pgfscope}%
\pgfpathrectangle{\pgfqpoint{0.100000in}{2.413063in}}{\pgfqpoint{5.037500in}{3.427208in}}%
\pgfusepath{clip}%
\pgfsetbuttcap%
\pgfsetroundjoin%
\definecolor{currentfill}{rgb}{0.000000,0.000000,1.000000}%
\pgfsetfillcolor{currentfill}%
\pgfsetfillopacity{0.500000}%
\pgfsetlinewidth{0.250937pt}%
\definecolor{currentstroke}{rgb}{0.000000,0.000000,0.000000}%
\pgfsetstrokecolor{currentstroke}%
\pgfsetstrokeopacity{0.500000}%
\pgfsetdash{}{0pt}%
\pgfsys@defobject{currentmarker}{\pgfqpoint{-0.027778in}{-0.027778in}}{\pgfqpoint{0.027778in}{0.027778in}}{%
\pgfpathmoveto{\pgfqpoint{0.000000in}{-0.027778in}}%
\pgfpathcurveto{\pgfqpoint{0.007367in}{-0.027778in}}{\pgfqpoint{0.014433in}{-0.024851in}}{\pgfqpoint{0.019642in}{-0.019642in}}%
\pgfpathcurveto{\pgfqpoint{0.024851in}{-0.014433in}}{\pgfqpoint{0.027778in}{-0.007367in}}{\pgfqpoint{0.027778in}{0.000000in}}%
\pgfpathcurveto{\pgfqpoint{0.027778in}{0.007367in}}{\pgfqpoint{0.024851in}{0.014433in}}{\pgfqpoint{0.019642in}{0.019642in}}%
\pgfpathcurveto{\pgfqpoint{0.014433in}{0.024851in}}{\pgfqpoint{0.007367in}{0.027778in}}{\pgfqpoint{0.000000in}{0.027778in}}%
\pgfpathcurveto{\pgfqpoint{-0.007367in}{0.027778in}}{\pgfqpoint{-0.014433in}{0.024851in}}{\pgfqpoint{-0.019642in}{0.019642in}}%
\pgfpathcurveto{\pgfqpoint{-0.024851in}{0.014433in}}{\pgfqpoint{-0.027778in}{0.007367in}}{\pgfqpoint{-0.027778in}{0.000000in}}%
\pgfpathcurveto{\pgfqpoint{-0.027778in}{-0.007367in}}{\pgfqpoint{-0.024851in}{-0.014433in}}{\pgfqpoint{-0.019642in}{-0.019642in}}%
\pgfpathcurveto{\pgfqpoint{-0.014433in}{-0.024851in}}{\pgfqpoint{-0.007367in}{-0.027778in}}{\pgfqpoint{0.000000in}{-0.027778in}}%
\pgfpathclose%
\pgfusepath{stroke,fill}%
}%
\begin{pgfscope}%
\pgfsys@transformshift{2.469056in}{3.700922in}%
\pgfsys@useobject{currentmarker}{}%
\end{pgfscope}%
\end{pgfscope}%
\begin{pgfscope}%
\pgfpathrectangle{\pgfqpoint{0.100000in}{2.413063in}}{\pgfqpoint{5.037500in}{3.427208in}}%
\pgfusepath{clip}%
\pgfsetrectcap%
\pgfsetroundjoin%
\pgfsetlinewidth{1.505625pt}%
\definecolor{currentstroke}{rgb}{0.678431,1.000000,0.184314}%
\pgfsetstrokecolor{currentstroke}%
\pgfsetstrokeopacity{0.500000}%
\pgfsetdash{}{0pt}%
\pgfpathmoveto{\pgfqpoint{1.364191in}{4.729322in}}%
\pgfusepath{stroke}%
\end{pgfscope}%
\begin{pgfscope}%
\pgfpathrectangle{\pgfqpoint{0.100000in}{2.413063in}}{\pgfqpoint{5.037500in}{3.427208in}}%
\pgfusepath{clip}%
\pgfsetbuttcap%
\pgfsetroundjoin%
\definecolor{currentfill}{rgb}{0.678431,1.000000,0.184314}%
\pgfsetfillcolor{currentfill}%
\pgfsetfillopacity{0.500000}%
\pgfsetlinewidth{0.250937pt}%
\definecolor{currentstroke}{rgb}{0.000000,0.000000,0.000000}%
\pgfsetstrokecolor{currentstroke}%
\pgfsetstrokeopacity{0.500000}%
\pgfsetdash{}{0pt}%
\pgfsys@defobject{currentmarker}{\pgfqpoint{-0.016667in}{-0.016667in}}{\pgfqpoint{0.016667in}{0.016667in}}{%
\pgfpathmoveto{\pgfqpoint{0.000000in}{-0.016667in}}%
\pgfpathcurveto{\pgfqpoint{0.004420in}{-0.016667in}}{\pgfqpoint{0.008660in}{-0.014911in}}{\pgfqpoint{0.011785in}{-0.011785in}}%
\pgfpathcurveto{\pgfqpoint{0.014911in}{-0.008660in}}{\pgfqpoint{0.016667in}{-0.004420in}}{\pgfqpoint{0.016667in}{0.000000in}}%
\pgfpathcurveto{\pgfqpoint{0.016667in}{0.004420in}}{\pgfqpoint{0.014911in}{0.008660in}}{\pgfqpoint{0.011785in}{0.011785in}}%
\pgfpathcurveto{\pgfqpoint{0.008660in}{0.014911in}}{\pgfqpoint{0.004420in}{0.016667in}}{\pgfqpoint{0.000000in}{0.016667in}}%
\pgfpathcurveto{\pgfqpoint{-0.004420in}{0.016667in}}{\pgfqpoint{-0.008660in}{0.014911in}}{\pgfqpoint{-0.011785in}{0.011785in}}%
\pgfpathcurveto{\pgfqpoint{-0.014911in}{0.008660in}}{\pgfqpoint{-0.016667in}{0.004420in}}{\pgfqpoint{-0.016667in}{0.000000in}}%
\pgfpathcurveto{\pgfqpoint{-0.016667in}{-0.004420in}}{\pgfqpoint{-0.014911in}{-0.008660in}}{\pgfqpoint{-0.011785in}{-0.011785in}}%
\pgfpathcurveto{\pgfqpoint{-0.008660in}{-0.014911in}}{\pgfqpoint{-0.004420in}{-0.016667in}}{\pgfqpoint{0.000000in}{-0.016667in}}%
\pgfpathclose%
\pgfusepath{stroke,fill}%
}%
\begin{pgfscope}%
\pgfsys@transformshift{1.364191in}{4.729322in}%
\pgfsys@useobject{currentmarker}{}%
\end{pgfscope}%
\end{pgfscope}%
\begin{pgfscope}%
\pgfpathrectangle{\pgfqpoint{0.100000in}{2.413063in}}{\pgfqpoint{5.037500in}{3.427208in}}%
\pgfusepath{clip}%
\pgfsetrectcap%
\pgfsetroundjoin%
\pgfsetlinewidth{1.505625pt}%
\definecolor{currentstroke}{rgb}{0.678431,1.000000,0.184314}%
\pgfsetstrokecolor{currentstroke}%
\pgfsetstrokeopacity{0.500000}%
\pgfsetdash{}{0pt}%
\pgfpathmoveto{\pgfqpoint{1.341563in}{4.673966in}}%
\pgfusepath{stroke}%
\end{pgfscope}%
\begin{pgfscope}%
\pgfpathrectangle{\pgfqpoint{0.100000in}{2.413063in}}{\pgfqpoint{5.037500in}{3.427208in}}%
\pgfusepath{clip}%
\pgfsetbuttcap%
\pgfsetroundjoin%
\definecolor{currentfill}{rgb}{0.678431,1.000000,0.184314}%
\pgfsetfillcolor{currentfill}%
\pgfsetfillopacity{0.500000}%
\pgfsetlinewidth{0.250937pt}%
\definecolor{currentstroke}{rgb}{0.000000,0.000000,0.000000}%
\pgfsetstrokecolor{currentstroke}%
\pgfsetstrokeopacity{0.500000}%
\pgfsetdash{}{0pt}%
\pgfsys@defobject{currentmarker}{\pgfqpoint{-0.025000in}{-0.025000in}}{\pgfqpoint{0.025000in}{0.025000in}}{%
\pgfpathmoveto{\pgfqpoint{0.000000in}{-0.025000in}}%
\pgfpathcurveto{\pgfqpoint{0.006630in}{-0.025000in}}{\pgfqpoint{0.012989in}{-0.022366in}}{\pgfqpoint{0.017678in}{-0.017678in}}%
\pgfpathcurveto{\pgfqpoint{0.022366in}{-0.012989in}}{\pgfqpoint{0.025000in}{-0.006630in}}{\pgfqpoint{0.025000in}{0.000000in}}%
\pgfpathcurveto{\pgfqpoint{0.025000in}{0.006630in}}{\pgfqpoint{0.022366in}{0.012989in}}{\pgfqpoint{0.017678in}{0.017678in}}%
\pgfpathcurveto{\pgfqpoint{0.012989in}{0.022366in}}{\pgfqpoint{0.006630in}{0.025000in}}{\pgfqpoint{0.000000in}{0.025000in}}%
\pgfpathcurveto{\pgfqpoint{-0.006630in}{0.025000in}}{\pgfqpoint{-0.012989in}{0.022366in}}{\pgfqpoint{-0.017678in}{0.017678in}}%
\pgfpathcurveto{\pgfqpoint{-0.022366in}{0.012989in}}{\pgfqpoint{-0.025000in}{0.006630in}}{\pgfqpoint{-0.025000in}{0.000000in}}%
\pgfpathcurveto{\pgfqpoint{-0.025000in}{-0.006630in}}{\pgfqpoint{-0.022366in}{-0.012989in}}{\pgfqpoint{-0.017678in}{-0.017678in}}%
\pgfpathcurveto{\pgfqpoint{-0.012989in}{-0.022366in}}{\pgfqpoint{-0.006630in}{-0.025000in}}{\pgfqpoint{0.000000in}{-0.025000in}}%
\pgfpathclose%
\pgfusepath{stroke,fill}%
}%
\begin{pgfscope}%
\pgfsys@transformshift{1.341563in}{4.673966in}%
\pgfsys@useobject{currentmarker}{}%
\end{pgfscope}%
\end{pgfscope}%
\begin{pgfscope}%
\pgfpathrectangle{\pgfqpoint{0.100000in}{2.413063in}}{\pgfqpoint{5.037500in}{3.427208in}}%
\pgfusepath{clip}%
\pgfsetrectcap%
\pgfsetroundjoin%
\pgfsetlinewidth{1.505625pt}%
\definecolor{currentstroke}{rgb}{0.678431,1.000000,0.184314}%
\pgfsetstrokecolor{currentstroke}%
\pgfsetstrokeopacity{0.500000}%
\pgfsetdash{}{0pt}%
\pgfpathmoveto{\pgfqpoint{1.347653in}{4.556899in}}%
\pgfusepath{stroke}%
\end{pgfscope}%
\begin{pgfscope}%
\pgfpathrectangle{\pgfqpoint{0.100000in}{2.413063in}}{\pgfqpoint{5.037500in}{3.427208in}}%
\pgfusepath{clip}%
\pgfsetbuttcap%
\pgfsetroundjoin%
\definecolor{currentfill}{rgb}{0.678431,1.000000,0.184314}%
\pgfsetfillcolor{currentfill}%
\pgfsetfillopacity{0.500000}%
\pgfsetlinewidth{0.250937pt}%
\definecolor{currentstroke}{rgb}{0.000000,0.000000,0.000000}%
\pgfsetstrokecolor{currentstroke}%
\pgfsetstrokeopacity{0.500000}%
\pgfsetdash{}{0pt}%
\pgfsys@defobject{currentmarker}{\pgfqpoint{-0.022222in}{-0.022222in}}{\pgfqpoint{0.022222in}{0.022222in}}{%
\pgfpathmoveto{\pgfqpoint{0.000000in}{-0.022222in}}%
\pgfpathcurveto{\pgfqpoint{0.005893in}{-0.022222in}}{\pgfqpoint{0.011546in}{-0.019881in}}{\pgfqpoint{0.015713in}{-0.015713in}}%
\pgfpathcurveto{\pgfqpoint{0.019881in}{-0.011546in}}{\pgfqpoint{0.022222in}{-0.005893in}}{\pgfqpoint{0.022222in}{0.000000in}}%
\pgfpathcurveto{\pgfqpoint{0.022222in}{0.005893in}}{\pgfqpoint{0.019881in}{0.011546in}}{\pgfqpoint{0.015713in}{0.015713in}}%
\pgfpathcurveto{\pgfqpoint{0.011546in}{0.019881in}}{\pgfqpoint{0.005893in}{0.022222in}}{\pgfqpoint{0.000000in}{0.022222in}}%
\pgfpathcurveto{\pgfqpoint{-0.005893in}{0.022222in}}{\pgfqpoint{-0.011546in}{0.019881in}}{\pgfqpoint{-0.015713in}{0.015713in}}%
\pgfpathcurveto{\pgfqpoint{-0.019881in}{0.011546in}}{\pgfqpoint{-0.022222in}{0.005893in}}{\pgfqpoint{-0.022222in}{0.000000in}}%
\pgfpathcurveto{\pgfqpoint{-0.022222in}{-0.005893in}}{\pgfqpoint{-0.019881in}{-0.011546in}}{\pgfqpoint{-0.015713in}{-0.015713in}}%
\pgfpathcurveto{\pgfqpoint{-0.011546in}{-0.019881in}}{\pgfqpoint{-0.005893in}{-0.022222in}}{\pgfqpoint{0.000000in}{-0.022222in}}%
\pgfpathclose%
\pgfusepath{stroke,fill}%
}%
\begin{pgfscope}%
\pgfsys@transformshift{1.347653in}{4.556899in}%
\pgfsys@useobject{currentmarker}{}%
\end{pgfscope}%
\end{pgfscope}%
\begin{pgfscope}%
\pgfpathrectangle{\pgfqpoint{0.100000in}{2.413063in}}{\pgfqpoint{5.037500in}{3.427208in}}%
\pgfusepath{clip}%
\pgfsetrectcap%
\pgfsetroundjoin%
\pgfsetlinewidth{1.505625pt}%
\definecolor{currentstroke}{rgb}{0.678431,1.000000,0.184314}%
\pgfsetstrokecolor{currentstroke}%
\pgfsetstrokeopacity{0.500000}%
\pgfsetdash{}{0pt}%
\pgfpathmoveto{\pgfqpoint{1.108330in}{4.236603in}}%
\pgfusepath{stroke}%
\end{pgfscope}%
\begin{pgfscope}%
\pgfpathrectangle{\pgfqpoint{0.100000in}{2.413063in}}{\pgfqpoint{5.037500in}{3.427208in}}%
\pgfusepath{clip}%
\pgfsetbuttcap%
\pgfsetroundjoin%
\definecolor{currentfill}{rgb}{0.678431,1.000000,0.184314}%
\pgfsetfillcolor{currentfill}%
\pgfsetfillopacity{0.500000}%
\pgfsetlinewidth{0.250937pt}%
\definecolor{currentstroke}{rgb}{0.000000,0.000000,0.000000}%
\pgfsetstrokecolor{currentstroke}%
\pgfsetstrokeopacity{0.500000}%
\pgfsetdash{}{0pt}%
\pgfsys@defobject{currentmarker}{\pgfqpoint{-0.027778in}{-0.027778in}}{\pgfqpoint{0.027778in}{0.027778in}}{%
\pgfpathmoveto{\pgfqpoint{0.000000in}{-0.027778in}}%
\pgfpathcurveto{\pgfqpoint{0.007367in}{-0.027778in}}{\pgfqpoint{0.014433in}{-0.024851in}}{\pgfqpoint{0.019642in}{-0.019642in}}%
\pgfpathcurveto{\pgfqpoint{0.024851in}{-0.014433in}}{\pgfqpoint{0.027778in}{-0.007367in}}{\pgfqpoint{0.027778in}{0.000000in}}%
\pgfpathcurveto{\pgfqpoint{0.027778in}{0.007367in}}{\pgfqpoint{0.024851in}{0.014433in}}{\pgfqpoint{0.019642in}{0.019642in}}%
\pgfpathcurveto{\pgfqpoint{0.014433in}{0.024851in}}{\pgfqpoint{0.007367in}{0.027778in}}{\pgfqpoint{0.000000in}{0.027778in}}%
\pgfpathcurveto{\pgfqpoint{-0.007367in}{0.027778in}}{\pgfqpoint{-0.014433in}{0.024851in}}{\pgfqpoint{-0.019642in}{0.019642in}}%
\pgfpathcurveto{\pgfqpoint{-0.024851in}{0.014433in}}{\pgfqpoint{-0.027778in}{0.007367in}}{\pgfqpoint{-0.027778in}{0.000000in}}%
\pgfpathcurveto{\pgfqpoint{-0.027778in}{-0.007367in}}{\pgfqpoint{-0.024851in}{-0.014433in}}{\pgfqpoint{-0.019642in}{-0.019642in}}%
\pgfpathcurveto{\pgfqpoint{-0.014433in}{-0.024851in}}{\pgfqpoint{-0.007367in}{-0.027778in}}{\pgfqpoint{0.000000in}{-0.027778in}}%
\pgfpathclose%
\pgfusepath{stroke,fill}%
}%
\begin{pgfscope}%
\pgfsys@transformshift{1.108330in}{4.236603in}%
\pgfsys@useobject{currentmarker}{}%
\end{pgfscope}%
\end{pgfscope}%
\begin{pgfscope}%
\pgfpathrectangle{\pgfqpoint{0.100000in}{2.413063in}}{\pgfqpoint{5.037500in}{3.427208in}}%
\pgfusepath{clip}%
\pgfsetrectcap%
\pgfsetroundjoin%
\pgfsetlinewidth{1.505625pt}%
\definecolor{currentstroke}{rgb}{0.678431,1.000000,0.184314}%
\pgfsetstrokecolor{currentstroke}%
\pgfsetstrokeopacity{0.500000}%
\pgfsetdash{}{0pt}%
\pgfpathmoveto{\pgfqpoint{1.338959in}{4.620996in}}%
\pgfusepath{stroke}%
\end{pgfscope}%
\begin{pgfscope}%
\pgfpathrectangle{\pgfqpoint{0.100000in}{2.413063in}}{\pgfqpoint{5.037500in}{3.427208in}}%
\pgfusepath{clip}%
\pgfsetbuttcap%
\pgfsetroundjoin%
\definecolor{currentfill}{rgb}{0.678431,1.000000,0.184314}%
\pgfsetfillcolor{currentfill}%
\pgfsetfillopacity{0.500000}%
\pgfsetlinewidth{0.250937pt}%
\definecolor{currentstroke}{rgb}{0.000000,0.000000,0.000000}%
\pgfsetstrokecolor{currentstroke}%
\pgfsetstrokeopacity{0.500000}%
\pgfsetdash{}{0pt}%
\pgfsys@defobject{currentmarker}{\pgfqpoint{-0.022222in}{-0.022222in}}{\pgfqpoint{0.022222in}{0.022222in}}{%
\pgfpathmoveto{\pgfqpoint{0.000000in}{-0.022222in}}%
\pgfpathcurveto{\pgfqpoint{0.005893in}{-0.022222in}}{\pgfqpoint{0.011546in}{-0.019881in}}{\pgfqpoint{0.015713in}{-0.015713in}}%
\pgfpathcurveto{\pgfqpoint{0.019881in}{-0.011546in}}{\pgfqpoint{0.022222in}{-0.005893in}}{\pgfqpoint{0.022222in}{0.000000in}}%
\pgfpathcurveto{\pgfqpoint{0.022222in}{0.005893in}}{\pgfqpoint{0.019881in}{0.011546in}}{\pgfqpoint{0.015713in}{0.015713in}}%
\pgfpathcurveto{\pgfqpoint{0.011546in}{0.019881in}}{\pgfqpoint{0.005893in}{0.022222in}}{\pgfqpoint{0.000000in}{0.022222in}}%
\pgfpathcurveto{\pgfqpoint{-0.005893in}{0.022222in}}{\pgfqpoint{-0.011546in}{0.019881in}}{\pgfqpoint{-0.015713in}{0.015713in}}%
\pgfpathcurveto{\pgfqpoint{-0.019881in}{0.011546in}}{\pgfqpoint{-0.022222in}{0.005893in}}{\pgfqpoint{-0.022222in}{0.000000in}}%
\pgfpathcurveto{\pgfqpoint{-0.022222in}{-0.005893in}}{\pgfqpoint{-0.019881in}{-0.011546in}}{\pgfqpoint{-0.015713in}{-0.015713in}}%
\pgfpathcurveto{\pgfqpoint{-0.011546in}{-0.019881in}}{\pgfqpoint{-0.005893in}{-0.022222in}}{\pgfqpoint{0.000000in}{-0.022222in}}%
\pgfpathclose%
\pgfusepath{stroke,fill}%
}%
\begin{pgfscope}%
\pgfsys@transformshift{1.338959in}{4.620996in}%
\pgfsys@useobject{currentmarker}{}%
\end{pgfscope}%
\end{pgfscope}%
\begin{pgfscope}%
\pgfpathrectangle{\pgfqpoint{0.100000in}{2.413063in}}{\pgfqpoint{5.037500in}{3.427208in}}%
\pgfusepath{clip}%
\pgfsetrectcap%
\pgfsetroundjoin%
\pgfsetlinewidth{1.505625pt}%
\definecolor{currentstroke}{rgb}{0.501961,0.501961,0.501961}%
\pgfsetstrokecolor{currentstroke}%
\pgfsetstrokeopacity{0.500000}%
\pgfsetdash{}{0pt}%
\pgfpathmoveto{\pgfqpoint{4.585415in}{5.126583in}}%
\pgfusepath{stroke}%
\end{pgfscope}%
\begin{pgfscope}%
\pgfpathrectangle{\pgfqpoint{0.100000in}{2.413063in}}{\pgfqpoint{5.037500in}{3.427208in}}%
\pgfusepath{clip}%
\pgfsetbuttcap%
\pgfsetroundjoin%
\definecolor{currentfill}{rgb}{0.501961,0.501961,0.501961}%
\pgfsetfillcolor{currentfill}%
\pgfsetfillopacity{0.500000}%
\pgfsetlinewidth{0.250937pt}%
\definecolor{currentstroke}{rgb}{0.000000,0.000000,0.000000}%
\pgfsetstrokecolor{currentstroke}%
\pgfsetstrokeopacity{0.500000}%
\pgfsetdash{}{0pt}%
\pgfsys@defobject{currentmarker}{\pgfqpoint{-0.013889in}{-0.013889in}}{\pgfqpoint{0.013889in}{0.013889in}}{%
\pgfpathmoveto{\pgfqpoint{0.000000in}{-0.013889in}}%
\pgfpathcurveto{\pgfqpoint{0.003683in}{-0.013889in}}{\pgfqpoint{0.007216in}{-0.012425in}}{\pgfqpoint{0.009821in}{-0.009821in}}%
\pgfpathcurveto{\pgfqpoint{0.012425in}{-0.007216in}}{\pgfqpoint{0.013889in}{-0.003683in}}{\pgfqpoint{0.013889in}{0.000000in}}%
\pgfpathcurveto{\pgfqpoint{0.013889in}{0.003683in}}{\pgfqpoint{0.012425in}{0.007216in}}{\pgfqpoint{0.009821in}{0.009821in}}%
\pgfpathcurveto{\pgfqpoint{0.007216in}{0.012425in}}{\pgfqpoint{0.003683in}{0.013889in}}{\pgfqpoint{0.000000in}{0.013889in}}%
\pgfpathcurveto{\pgfqpoint{-0.003683in}{0.013889in}}{\pgfqpoint{-0.007216in}{0.012425in}}{\pgfqpoint{-0.009821in}{0.009821in}}%
\pgfpathcurveto{\pgfqpoint{-0.012425in}{0.007216in}}{\pgfqpoint{-0.013889in}{0.003683in}}{\pgfqpoint{-0.013889in}{0.000000in}}%
\pgfpathcurveto{\pgfqpoint{-0.013889in}{-0.003683in}}{\pgfqpoint{-0.012425in}{-0.007216in}}{\pgfqpoint{-0.009821in}{-0.009821in}}%
\pgfpathcurveto{\pgfqpoint{-0.007216in}{-0.012425in}}{\pgfqpoint{-0.003683in}{-0.013889in}}{\pgfqpoint{0.000000in}{-0.013889in}}%
\pgfpathclose%
\pgfusepath{stroke,fill}%
}%
\begin{pgfscope}%
\pgfsys@transformshift{4.585415in}{5.126583in}%
\pgfsys@useobject{currentmarker}{}%
\end{pgfscope}%
\end{pgfscope}%
\begin{pgfscope}%
\pgfpathrectangle{\pgfqpoint{0.100000in}{2.413063in}}{\pgfqpoint{5.037500in}{3.427208in}}%
\pgfusepath{clip}%
\pgfsetrectcap%
\pgfsetroundjoin%
\pgfsetlinewidth{1.505625pt}%
\definecolor{currentstroke}{rgb}{0.678431,1.000000,0.184314}%
\pgfsetstrokecolor{currentstroke}%
\pgfsetstrokeopacity{0.500000}%
\pgfsetdash{}{0pt}%
\pgfpathmoveto{\pgfqpoint{4.136391in}{4.184211in}}%
\pgfusepath{stroke}%
\end{pgfscope}%
\begin{pgfscope}%
\pgfpathrectangle{\pgfqpoint{0.100000in}{2.413063in}}{\pgfqpoint{5.037500in}{3.427208in}}%
\pgfusepath{clip}%
\pgfsetbuttcap%
\pgfsetroundjoin%
\definecolor{currentfill}{rgb}{0.678431,1.000000,0.184314}%
\pgfsetfillcolor{currentfill}%
\pgfsetfillopacity{0.500000}%
\pgfsetlinewidth{0.250937pt}%
\definecolor{currentstroke}{rgb}{0.000000,0.000000,0.000000}%
\pgfsetstrokecolor{currentstroke}%
\pgfsetstrokeopacity{0.500000}%
\pgfsetdash{}{0pt}%
\pgfsys@defobject{currentmarker}{\pgfqpoint{-0.027778in}{-0.027778in}}{\pgfqpoint{0.027778in}{0.027778in}}{%
\pgfpathmoveto{\pgfqpoint{0.000000in}{-0.027778in}}%
\pgfpathcurveto{\pgfqpoint{0.007367in}{-0.027778in}}{\pgfqpoint{0.014433in}{-0.024851in}}{\pgfqpoint{0.019642in}{-0.019642in}}%
\pgfpathcurveto{\pgfqpoint{0.024851in}{-0.014433in}}{\pgfqpoint{0.027778in}{-0.007367in}}{\pgfqpoint{0.027778in}{0.000000in}}%
\pgfpathcurveto{\pgfqpoint{0.027778in}{0.007367in}}{\pgfqpoint{0.024851in}{0.014433in}}{\pgfqpoint{0.019642in}{0.019642in}}%
\pgfpathcurveto{\pgfqpoint{0.014433in}{0.024851in}}{\pgfqpoint{0.007367in}{0.027778in}}{\pgfqpoint{0.000000in}{0.027778in}}%
\pgfpathcurveto{\pgfqpoint{-0.007367in}{0.027778in}}{\pgfqpoint{-0.014433in}{0.024851in}}{\pgfqpoint{-0.019642in}{0.019642in}}%
\pgfpathcurveto{\pgfqpoint{-0.024851in}{0.014433in}}{\pgfqpoint{-0.027778in}{0.007367in}}{\pgfqpoint{-0.027778in}{0.000000in}}%
\pgfpathcurveto{\pgfqpoint{-0.027778in}{-0.007367in}}{\pgfqpoint{-0.024851in}{-0.014433in}}{\pgfqpoint{-0.019642in}{-0.019642in}}%
\pgfpathcurveto{\pgfqpoint{-0.014433in}{-0.024851in}}{\pgfqpoint{-0.007367in}{-0.027778in}}{\pgfqpoint{0.000000in}{-0.027778in}}%
\pgfpathclose%
\pgfusepath{stroke,fill}%
}%
\begin{pgfscope}%
\pgfsys@transformshift{4.136391in}{4.184211in}%
\pgfsys@useobject{currentmarker}{}%
\end{pgfscope}%
\end{pgfscope}%
\begin{pgfscope}%
\pgfpathrectangle{\pgfqpoint{0.100000in}{2.413063in}}{\pgfqpoint{5.037500in}{3.427208in}}%
\pgfusepath{clip}%
\pgfsetrectcap%
\pgfsetroundjoin%
\pgfsetlinewidth{1.505625pt}%
\definecolor{currentstroke}{rgb}{0.000000,0.000000,1.000000}%
\pgfsetstrokecolor{currentstroke}%
\pgfsetstrokeopacity{0.500000}%
\pgfsetdash{}{0pt}%
\pgfpathmoveto{\pgfqpoint{4.294817in}{4.305064in}}%
\pgfusepath{stroke}%
\end{pgfscope}%
\begin{pgfscope}%
\pgfpathrectangle{\pgfqpoint{0.100000in}{2.413063in}}{\pgfqpoint{5.037500in}{3.427208in}}%
\pgfusepath{clip}%
\pgfsetbuttcap%
\pgfsetroundjoin%
\definecolor{currentfill}{rgb}{0.000000,0.000000,1.000000}%
\pgfsetfillcolor{currentfill}%
\pgfsetfillopacity{0.500000}%
\pgfsetlinewidth{0.250937pt}%
\definecolor{currentstroke}{rgb}{0.000000,0.000000,0.000000}%
\pgfsetstrokecolor{currentstroke}%
\pgfsetstrokeopacity{0.500000}%
\pgfsetdash{}{0pt}%
\pgfsys@defobject{currentmarker}{\pgfqpoint{-0.011111in}{-0.011111in}}{\pgfqpoint{0.011111in}{0.011111in}}{%
\pgfpathmoveto{\pgfqpoint{0.000000in}{-0.011111in}}%
\pgfpathcurveto{\pgfqpoint{0.002947in}{-0.011111in}}{\pgfqpoint{0.005773in}{-0.009940in}}{\pgfqpoint{0.007857in}{-0.007857in}}%
\pgfpathcurveto{\pgfqpoint{0.009940in}{-0.005773in}}{\pgfqpoint{0.011111in}{-0.002947in}}{\pgfqpoint{0.011111in}{0.000000in}}%
\pgfpathcurveto{\pgfqpoint{0.011111in}{0.002947in}}{\pgfqpoint{0.009940in}{0.005773in}}{\pgfqpoint{0.007857in}{0.007857in}}%
\pgfpathcurveto{\pgfqpoint{0.005773in}{0.009940in}}{\pgfqpoint{0.002947in}{0.011111in}}{\pgfqpoint{0.000000in}{0.011111in}}%
\pgfpathcurveto{\pgfqpoint{-0.002947in}{0.011111in}}{\pgfqpoint{-0.005773in}{0.009940in}}{\pgfqpoint{-0.007857in}{0.007857in}}%
\pgfpathcurveto{\pgfqpoint{-0.009940in}{0.005773in}}{\pgfqpoint{-0.011111in}{0.002947in}}{\pgfqpoint{-0.011111in}{0.000000in}}%
\pgfpathcurveto{\pgfqpoint{-0.011111in}{-0.002947in}}{\pgfqpoint{-0.009940in}{-0.005773in}}{\pgfqpoint{-0.007857in}{-0.007857in}}%
\pgfpathcurveto{\pgfqpoint{-0.005773in}{-0.009940in}}{\pgfqpoint{-0.002947in}{-0.011111in}}{\pgfqpoint{0.000000in}{-0.011111in}}%
\pgfpathclose%
\pgfusepath{stroke,fill}%
}%
\begin{pgfscope}%
\pgfsys@transformshift{4.294817in}{4.305064in}%
\pgfsys@useobject{currentmarker}{}%
\end{pgfscope}%
\end{pgfscope}%
\begin{pgfscope}%
\pgfpathrectangle{\pgfqpoint{0.100000in}{2.413063in}}{\pgfqpoint{5.037500in}{3.427208in}}%
\pgfusepath{clip}%
\pgfsetrectcap%
\pgfsetroundjoin%
\pgfsetlinewidth{1.505625pt}%
\definecolor{currentstroke}{rgb}{0.000000,0.000000,1.000000}%
\pgfsetstrokecolor{currentstroke}%
\pgfsetstrokeopacity{0.500000}%
\pgfsetdash{}{0pt}%
\pgfpathmoveto{\pgfqpoint{4.251269in}{4.346316in}}%
\pgfusepath{stroke}%
\end{pgfscope}%
\begin{pgfscope}%
\pgfpathrectangle{\pgfqpoint{0.100000in}{2.413063in}}{\pgfqpoint{5.037500in}{3.427208in}}%
\pgfusepath{clip}%
\pgfsetbuttcap%
\pgfsetroundjoin%
\definecolor{currentfill}{rgb}{0.000000,0.000000,1.000000}%
\pgfsetfillcolor{currentfill}%
\pgfsetfillopacity{0.500000}%
\pgfsetlinewidth{0.250937pt}%
\definecolor{currentstroke}{rgb}{0.000000,0.000000,0.000000}%
\pgfsetstrokecolor{currentstroke}%
\pgfsetstrokeopacity{0.500000}%
\pgfsetdash{}{0pt}%
\pgfsys@defobject{currentmarker}{\pgfqpoint{-0.008333in}{-0.008333in}}{\pgfqpoint{0.008333in}{0.008333in}}{%
\pgfpathmoveto{\pgfqpoint{0.000000in}{-0.008333in}}%
\pgfpathcurveto{\pgfqpoint{0.002210in}{-0.008333in}}{\pgfqpoint{0.004330in}{-0.007455in}}{\pgfqpoint{0.005893in}{-0.005893in}}%
\pgfpathcurveto{\pgfqpoint{0.007455in}{-0.004330in}}{\pgfqpoint{0.008333in}{-0.002210in}}{\pgfqpoint{0.008333in}{0.000000in}}%
\pgfpathcurveto{\pgfqpoint{0.008333in}{0.002210in}}{\pgfqpoint{0.007455in}{0.004330in}}{\pgfqpoint{0.005893in}{0.005893in}}%
\pgfpathcurveto{\pgfqpoint{0.004330in}{0.007455in}}{\pgfqpoint{0.002210in}{0.008333in}}{\pgfqpoint{0.000000in}{0.008333in}}%
\pgfpathcurveto{\pgfqpoint{-0.002210in}{0.008333in}}{\pgfqpoint{-0.004330in}{0.007455in}}{\pgfqpoint{-0.005893in}{0.005893in}}%
\pgfpathcurveto{\pgfqpoint{-0.007455in}{0.004330in}}{\pgfqpoint{-0.008333in}{0.002210in}}{\pgfqpoint{-0.008333in}{0.000000in}}%
\pgfpathcurveto{\pgfqpoint{-0.008333in}{-0.002210in}}{\pgfqpoint{-0.007455in}{-0.004330in}}{\pgfqpoint{-0.005893in}{-0.005893in}}%
\pgfpathcurveto{\pgfqpoint{-0.004330in}{-0.007455in}}{\pgfqpoint{-0.002210in}{-0.008333in}}{\pgfqpoint{0.000000in}{-0.008333in}}%
\pgfpathclose%
\pgfusepath{stroke,fill}%
}%
\begin{pgfscope}%
\pgfsys@transformshift{4.251269in}{4.346316in}%
\pgfsys@useobject{currentmarker}{}%
\end{pgfscope}%
\end{pgfscope}%
\begin{pgfscope}%
\pgfpathrectangle{\pgfqpoint{0.100000in}{2.413063in}}{\pgfqpoint{5.037500in}{3.427208in}}%
\pgfusepath{clip}%
\pgfsetrectcap%
\pgfsetroundjoin%
\pgfsetlinewidth{1.505625pt}%
\definecolor{currentstroke}{rgb}{0.501961,0.501961,0.501961}%
\pgfsetstrokecolor{currentstroke}%
\pgfsetstrokeopacity{0.500000}%
\pgfsetdash{}{0pt}%
\pgfpathmoveto{\pgfqpoint{4.244154in}{4.222410in}}%
\pgfusepath{stroke}%
\end{pgfscope}%
\begin{pgfscope}%
\pgfpathrectangle{\pgfqpoint{0.100000in}{2.413063in}}{\pgfqpoint{5.037500in}{3.427208in}}%
\pgfusepath{clip}%
\pgfsetbuttcap%
\pgfsetroundjoin%
\definecolor{currentfill}{rgb}{0.501961,0.501961,0.501961}%
\pgfsetfillcolor{currentfill}%
\pgfsetfillopacity{0.500000}%
\pgfsetlinewidth{0.250937pt}%
\definecolor{currentstroke}{rgb}{0.000000,0.000000,0.000000}%
\pgfsetstrokecolor{currentstroke}%
\pgfsetstrokeopacity{0.500000}%
\pgfsetdash{}{0pt}%
\pgfsys@defobject{currentmarker}{\pgfqpoint{-0.013889in}{-0.013889in}}{\pgfqpoint{0.013889in}{0.013889in}}{%
\pgfpathmoveto{\pgfqpoint{0.000000in}{-0.013889in}}%
\pgfpathcurveto{\pgfqpoint{0.003683in}{-0.013889in}}{\pgfqpoint{0.007216in}{-0.012425in}}{\pgfqpoint{0.009821in}{-0.009821in}}%
\pgfpathcurveto{\pgfqpoint{0.012425in}{-0.007216in}}{\pgfqpoint{0.013889in}{-0.003683in}}{\pgfqpoint{0.013889in}{0.000000in}}%
\pgfpathcurveto{\pgfqpoint{0.013889in}{0.003683in}}{\pgfqpoint{0.012425in}{0.007216in}}{\pgfqpoint{0.009821in}{0.009821in}}%
\pgfpathcurveto{\pgfqpoint{0.007216in}{0.012425in}}{\pgfqpoint{0.003683in}{0.013889in}}{\pgfqpoint{0.000000in}{0.013889in}}%
\pgfpathcurveto{\pgfqpoint{-0.003683in}{0.013889in}}{\pgfqpoint{-0.007216in}{0.012425in}}{\pgfqpoint{-0.009821in}{0.009821in}}%
\pgfpathcurveto{\pgfqpoint{-0.012425in}{0.007216in}}{\pgfqpoint{-0.013889in}{0.003683in}}{\pgfqpoint{-0.013889in}{0.000000in}}%
\pgfpathcurveto{\pgfqpoint{-0.013889in}{-0.003683in}}{\pgfqpoint{-0.012425in}{-0.007216in}}{\pgfqpoint{-0.009821in}{-0.009821in}}%
\pgfpathcurveto{\pgfqpoint{-0.007216in}{-0.012425in}}{\pgfqpoint{-0.003683in}{-0.013889in}}{\pgfqpoint{0.000000in}{-0.013889in}}%
\pgfpathclose%
\pgfusepath{stroke,fill}%
}%
\begin{pgfscope}%
\pgfsys@transformshift{4.244154in}{4.222410in}%
\pgfsys@useobject{currentmarker}{}%
\end{pgfscope}%
\end{pgfscope}%
\begin{pgfscope}%
\pgfpathrectangle{\pgfqpoint{0.100000in}{2.413063in}}{\pgfqpoint{5.037500in}{3.427208in}}%
\pgfusepath{clip}%
\pgfsetrectcap%
\pgfsetroundjoin%
\pgfsetlinewidth{1.505625pt}%
\definecolor{currentstroke}{rgb}{0.000000,0.000000,1.000000}%
\pgfsetstrokecolor{currentstroke}%
\pgfsetstrokeopacity{0.500000}%
\pgfsetdash{}{0pt}%
\pgfpathmoveto{\pgfqpoint{4.398579in}{4.267220in}}%
\pgfusepath{stroke}%
\end{pgfscope}%
\begin{pgfscope}%
\pgfpathrectangle{\pgfqpoint{0.100000in}{2.413063in}}{\pgfqpoint{5.037500in}{3.427208in}}%
\pgfusepath{clip}%
\pgfsetbuttcap%
\pgfsetroundjoin%
\definecolor{currentfill}{rgb}{0.000000,0.000000,1.000000}%
\pgfsetfillcolor{currentfill}%
\pgfsetfillopacity{0.500000}%
\pgfsetlinewidth{0.250937pt}%
\definecolor{currentstroke}{rgb}{0.000000,0.000000,0.000000}%
\pgfsetstrokecolor{currentstroke}%
\pgfsetstrokeopacity{0.500000}%
\pgfsetdash{}{0pt}%
\pgfsys@defobject{currentmarker}{\pgfqpoint{-0.022222in}{-0.022222in}}{\pgfqpoint{0.022222in}{0.022222in}}{%
\pgfpathmoveto{\pgfqpoint{0.000000in}{-0.022222in}}%
\pgfpathcurveto{\pgfqpoint{0.005893in}{-0.022222in}}{\pgfqpoint{0.011546in}{-0.019881in}}{\pgfqpoint{0.015713in}{-0.015713in}}%
\pgfpathcurveto{\pgfqpoint{0.019881in}{-0.011546in}}{\pgfqpoint{0.022222in}{-0.005893in}}{\pgfqpoint{0.022222in}{0.000000in}}%
\pgfpathcurveto{\pgfqpoint{0.022222in}{0.005893in}}{\pgfqpoint{0.019881in}{0.011546in}}{\pgfqpoint{0.015713in}{0.015713in}}%
\pgfpathcurveto{\pgfqpoint{0.011546in}{0.019881in}}{\pgfqpoint{0.005893in}{0.022222in}}{\pgfqpoint{0.000000in}{0.022222in}}%
\pgfpathcurveto{\pgfqpoint{-0.005893in}{0.022222in}}{\pgfqpoint{-0.011546in}{0.019881in}}{\pgfqpoint{-0.015713in}{0.015713in}}%
\pgfpathcurveto{\pgfqpoint{-0.019881in}{0.011546in}}{\pgfqpoint{-0.022222in}{0.005893in}}{\pgfqpoint{-0.022222in}{0.000000in}}%
\pgfpathcurveto{\pgfqpoint{-0.022222in}{-0.005893in}}{\pgfqpoint{-0.019881in}{-0.011546in}}{\pgfqpoint{-0.015713in}{-0.015713in}}%
\pgfpathcurveto{\pgfqpoint{-0.011546in}{-0.019881in}}{\pgfqpoint{-0.005893in}{-0.022222in}}{\pgfqpoint{0.000000in}{-0.022222in}}%
\pgfpathclose%
\pgfusepath{stroke,fill}%
}%
\begin{pgfscope}%
\pgfsys@transformshift{4.398579in}{4.267220in}%
\pgfsys@useobject{currentmarker}{}%
\end{pgfscope}%
\end{pgfscope}%
\begin{pgfscope}%
\pgfpathrectangle{\pgfqpoint{0.100000in}{2.413063in}}{\pgfqpoint{5.037500in}{3.427208in}}%
\pgfusepath{clip}%
\pgfsetrectcap%
\pgfsetroundjoin%
\pgfsetlinewidth{1.505625pt}%
\definecolor{currentstroke}{rgb}{0.000000,0.000000,1.000000}%
\pgfsetstrokecolor{currentstroke}%
\pgfsetstrokeopacity{0.500000}%
\pgfsetdash{}{0pt}%
\pgfpathmoveto{\pgfqpoint{4.178350in}{4.195940in}}%
\pgfusepath{stroke}%
\end{pgfscope}%
\begin{pgfscope}%
\pgfpathrectangle{\pgfqpoint{0.100000in}{2.413063in}}{\pgfqpoint{5.037500in}{3.427208in}}%
\pgfusepath{clip}%
\pgfsetbuttcap%
\pgfsetroundjoin%
\definecolor{currentfill}{rgb}{0.000000,0.000000,1.000000}%
\pgfsetfillcolor{currentfill}%
\pgfsetfillopacity{0.500000}%
\pgfsetlinewidth{0.250937pt}%
\definecolor{currentstroke}{rgb}{0.000000,0.000000,0.000000}%
\pgfsetstrokecolor{currentstroke}%
\pgfsetstrokeopacity{0.500000}%
\pgfsetdash{}{0pt}%
\pgfsys@defobject{currentmarker}{\pgfqpoint{-0.005556in}{-0.005556in}}{\pgfqpoint{0.005556in}{0.005556in}}{%
\pgfpathmoveto{\pgfqpoint{0.000000in}{-0.005556in}}%
\pgfpathcurveto{\pgfqpoint{0.001473in}{-0.005556in}}{\pgfqpoint{0.002887in}{-0.004970in}}{\pgfqpoint{0.003928in}{-0.003928in}}%
\pgfpathcurveto{\pgfqpoint{0.004970in}{-0.002887in}}{\pgfqpoint{0.005556in}{-0.001473in}}{\pgfqpoint{0.005556in}{0.000000in}}%
\pgfpathcurveto{\pgfqpoint{0.005556in}{0.001473in}}{\pgfqpoint{0.004970in}{0.002887in}}{\pgfqpoint{0.003928in}{0.003928in}}%
\pgfpathcurveto{\pgfqpoint{0.002887in}{0.004970in}}{\pgfqpoint{0.001473in}{0.005556in}}{\pgfqpoint{0.000000in}{0.005556in}}%
\pgfpathcurveto{\pgfqpoint{-0.001473in}{0.005556in}}{\pgfqpoint{-0.002887in}{0.004970in}}{\pgfqpoint{-0.003928in}{0.003928in}}%
\pgfpathcurveto{\pgfqpoint{-0.004970in}{0.002887in}}{\pgfqpoint{-0.005556in}{0.001473in}}{\pgfqpoint{-0.005556in}{0.000000in}}%
\pgfpathcurveto{\pgfqpoint{-0.005556in}{-0.001473in}}{\pgfqpoint{-0.004970in}{-0.002887in}}{\pgfqpoint{-0.003928in}{-0.003928in}}%
\pgfpathcurveto{\pgfqpoint{-0.002887in}{-0.004970in}}{\pgfqpoint{-0.001473in}{-0.005556in}}{\pgfqpoint{0.000000in}{-0.005556in}}%
\pgfpathclose%
\pgfusepath{stroke,fill}%
}%
\begin{pgfscope}%
\pgfsys@transformshift{4.178350in}{4.195940in}%
\pgfsys@useobject{currentmarker}{}%
\end{pgfscope}%
\end{pgfscope}%
\begin{pgfscope}%
\pgfpathrectangle{\pgfqpoint{0.100000in}{2.413063in}}{\pgfqpoint{5.037500in}{3.427208in}}%
\pgfusepath{clip}%
\pgfsetrectcap%
\pgfsetroundjoin%
\pgfsetlinewidth{1.505625pt}%
\definecolor{currentstroke}{rgb}{0.000000,0.000000,1.000000}%
\pgfsetstrokecolor{currentstroke}%
\pgfsetstrokeopacity{0.500000}%
\pgfsetdash{}{0pt}%
\pgfpathmoveto{\pgfqpoint{4.239205in}{4.309104in}}%
\pgfusepath{stroke}%
\end{pgfscope}%
\begin{pgfscope}%
\pgfpathrectangle{\pgfqpoint{0.100000in}{2.413063in}}{\pgfqpoint{5.037500in}{3.427208in}}%
\pgfusepath{clip}%
\pgfsetbuttcap%
\pgfsetroundjoin%
\definecolor{currentfill}{rgb}{0.000000,0.000000,1.000000}%
\pgfsetfillcolor{currentfill}%
\pgfsetfillopacity{0.500000}%
\pgfsetlinewidth{0.250937pt}%
\definecolor{currentstroke}{rgb}{0.000000,0.000000,0.000000}%
\pgfsetstrokecolor{currentstroke}%
\pgfsetstrokeopacity{0.500000}%
\pgfsetdash{}{0pt}%
\pgfsys@defobject{currentmarker}{\pgfqpoint{-0.008333in}{-0.008333in}}{\pgfqpoint{0.008333in}{0.008333in}}{%
\pgfpathmoveto{\pgfqpoint{0.000000in}{-0.008333in}}%
\pgfpathcurveto{\pgfqpoint{0.002210in}{-0.008333in}}{\pgfqpoint{0.004330in}{-0.007455in}}{\pgfqpoint{0.005893in}{-0.005893in}}%
\pgfpathcurveto{\pgfqpoint{0.007455in}{-0.004330in}}{\pgfqpoint{0.008333in}{-0.002210in}}{\pgfqpoint{0.008333in}{0.000000in}}%
\pgfpathcurveto{\pgfqpoint{0.008333in}{0.002210in}}{\pgfqpoint{0.007455in}{0.004330in}}{\pgfqpoint{0.005893in}{0.005893in}}%
\pgfpathcurveto{\pgfqpoint{0.004330in}{0.007455in}}{\pgfqpoint{0.002210in}{0.008333in}}{\pgfqpoint{0.000000in}{0.008333in}}%
\pgfpathcurveto{\pgfqpoint{-0.002210in}{0.008333in}}{\pgfqpoint{-0.004330in}{0.007455in}}{\pgfqpoint{-0.005893in}{0.005893in}}%
\pgfpathcurveto{\pgfqpoint{-0.007455in}{0.004330in}}{\pgfqpoint{-0.008333in}{0.002210in}}{\pgfqpoint{-0.008333in}{0.000000in}}%
\pgfpathcurveto{\pgfqpoint{-0.008333in}{-0.002210in}}{\pgfqpoint{-0.007455in}{-0.004330in}}{\pgfqpoint{-0.005893in}{-0.005893in}}%
\pgfpathcurveto{\pgfqpoint{-0.004330in}{-0.007455in}}{\pgfqpoint{-0.002210in}{-0.008333in}}{\pgfqpoint{0.000000in}{-0.008333in}}%
\pgfpathclose%
\pgfusepath{stroke,fill}%
}%
\begin{pgfscope}%
\pgfsys@transformshift{4.239205in}{4.309104in}%
\pgfsys@useobject{currentmarker}{}%
\end{pgfscope}%
\end{pgfscope}%
\begin{pgfscope}%
\pgfpathrectangle{\pgfqpoint{0.100000in}{2.413063in}}{\pgfqpoint{5.037500in}{3.427208in}}%
\pgfusepath{clip}%
\pgfsetrectcap%
\pgfsetroundjoin%
\pgfsetlinewidth{1.505625pt}%
\definecolor{currentstroke}{rgb}{0.000000,0.000000,1.000000}%
\pgfsetstrokecolor{currentstroke}%
\pgfsetstrokeopacity{0.500000}%
\pgfsetdash{}{0pt}%
\pgfpathmoveto{\pgfqpoint{4.545369in}{4.216571in}}%
\pgfusepath{stroke}%
\end{pgfscope}%
\begin{pgfscope}%
\pgfpathrectangle{\pgfqpoint{0.100000in}{2.413063in}}{\pgfqpoint{5.037500in}{3.427208in}}%
\pgfusepath{clip}%
\pgfsetbuttcap%
\pgfsetroundjoin%
\definecolor{currentfill}{rgb}{0.000000,0.000000,1.000000}%
\pgfsetfillcolor{currentfill}%
\pgfsetfillopacity{0.500000}%
\pgfsetlinewidth{0.250937pt}%
\definecolor{currentstroke}{rgb}{0.000000,0.000000,0.000000}%
\pgfsetstrokecolor{currentstroke}%
\pgfsetstrokeopacity{0.500000}%
\pgfsetdash{}{0pt}%
\pgfsys@defobject{currentmarker}{\pgfqpoint{-0.016667in}{-0.016667in}}{\pgfqpoint{0.016667in}{0.016667in}}{%
\pgfpathmoveto{\pgfqpoint{0.000000in}{-0.016667in}}%
\pgfpathcurveto{\pgfqpoint{0.004420in}{-0.016667in}}{\pgfqpoint{0.008660in}{-0.014911in}}{\pgfqpoint{0.011785in}{-0.011785in}}%
\pgfpathcurveto{\pgfqpoint{0.014911in}{-0.008660in}}{\pgfqpoint{0.016667in}{-0.004420in}}{\pgfqpoint{0.016667in}{0.000000in}}%
\pgfpathcurveto{\pgfqpoint{0.016667in}{0.004420in}}{\pgfqpoint{0.014911in}{0.008660in}}{\pgfqpoint{0.011785in}{0.011785in}}%
\pgfpathcurveto{\pgfqpoint{0.008660in}{0.014911in}}{\pgfqpoint{0.004420in}{0.016667in}}{\pgfqpoint{0.000000in}{0.016667in}}%
\pgfpathcurveto{\pgfqpoint{-0.004420in}{0.016667in}}{\pgfqpoint{-0.008660in}{0.014911in}}{\pgfqpoint{-0.011785in}{0.011785in}}%
\pgfpathcurveto{\pgfqpoint{-0.014911in}{0.008660in}}{\pgfqpoint{-0.016667in}{0.004420in}}{\pgfqpoint{-0.016667in}{0.000000in}}%
\pgfpathcurveto{\pgfqpoint{-0.016667in}{-0.004420in}}{\pgfqpoint{-0.014911in}{-0.008660in}}{\pgfqpoint{-0.011785in}{-0.011785in}}%
\pgfpathcurveto{\pgfqpoint{-0.008660in}{-0.014911in}}{\pgfqpoint{-0.004420in}{-0.016667in}}{\pgfqpoint{0.000000in}{-0.016667in}}%
\pgfpathclose%
\pgfusepath{stroke,fill}%
}%
\begin{pgfscope}%
\pgfsys@transformshift{4.545369in}{4.216571in}%
\pgfsys@useobject{currentmarker}{}%
\end{pgfscope}%
\end{pgfscope}%
\begin{pgfscope}%
\pgfpathrectangle{\pgfqpoint{0.100000in}{2.413063in}}{\pgfqpoint{5.037500in}{3.427208in}}%
\pgfusepath{clip}%
\pgfsetrectcap%
\pgfsetroundjoin%
\pgfsetlinewidth{1.505625pt}%
\definecolor{currentstroke}{rgb}{0.000000,0.000000,1.000000}%
\pgfsetstrokecolor{currentstroke}%
\pgfsetstrokeopacity{0.500000}%
\pgfsetdash{}{0pt}%
\pgfpathmoveto{\pgfqpoint{4.298077in}{4.441049in}}%
\pgfusepath{stroke}%
\end{pgfscope}%
\begin{pgfscope}%
\pgfpathrectangle{\pgfqpoint{0.100000in}{2.413063in}}{\pgfqpoint{5.037500in}{3.427208in}}%
\pgfusepath{clip}%
\pgfsetbuttcap%
\pgfsetroundjoin%
\definecolor{currentfill}{rgb}{0.000000,0.000000,1.000000}%
\pgfsetfillcolor{currentfill}%
\pgfsetfillopacity{0.500000}%
\pgfsetlinewidth{0.250937pt}%
\definecolor{currentstroke}{rgb}{0.000000,0.000000,0.000000}%
\pgfsetstrokecolor{currentstroke}%
\pgfsetstrokeopacity{0.500000}%
\pgfsetdash{}{0pt}%
\pgfsys@defobject{currentmarker}{\pgfqpoint{-0.008333in}{-0.008333in}}{\pgfqpoint{0.008333in}{0.008333in}}{%
\pgfpathmoveto{\pgfqpoint{0.000000in}{-0.008333in}}%
\pgfpathcurveto{\pgfqpoint{0.002210in}{-0.008333in}}{\pgfqpoint{0.004330in}{-0.007455in}}{\pgfqpoint{0.005893in}{-0.005893in}}%
\pgfpathcurveto{\pgfqpoint{0.007455in}{-0.004330in}}{\pgfqpoint{0.008333in}{-0.002210in}}{\pgfqpoint{0.008333in}{0.000000in}}%
\pgfpathcurveto{\pgfqpoint{0.008333in}{0.002210in}}{\pgfqpoint{0.007455in}{0.004330in}}{\pgfqpoint{0.005893in}{0.005893in}}%
\pgfpathcurveto{\pgfqpoint{0.004330in}{0.007455in}}{\pgfqpoint{0.002210in}{0.008333in}}{\pgfqpoint{0.000000in}{0.008333in}}%
\pgfpathcurveto{\pgfqpoint{-0.002210in}{0.008333in}}{\pgfqpoint{-0.004330in}{0.007455in}}{\pgfqpoint{-0.005893in}{0.005893in}}%
\pgfpathcurveto{\pgfqpoint{-0.007455in}{0.004330in}}{\pgfqpoint{-0.008333in}{0.002210in}}{\pgfqpoint{-0.008333in}{0.000000in}}%
\pgfpathcurveto{\pgfqpoint{-0.008333in}{-0.002210in}}{\pgfqpoint{-0.007455in}{-0.004330in}}{\pgfqpoint{-0.005893in}{-0.005893in}}%
\pgfpathcurveto{\pgfqpoint{-0.004330in}{-0.007455in}}{\pgfqpoint{-0.002210in}{-0.008333in}}{\pgfqpoint{0.000000in}{-0.008333in}}%
\pgfpathclose%
\pgfusepath{stroke,fill}%
}%
\begin{pgfscope}%
\pgfsys@transformshift{4.298077in}{4.441049in}%
\pgfsys@useobject{currentmarker}{}%
\end{pgfscope}%
\end{pgfscope}%
\begin{pgfscope}%
\pgfpathrectangle{\pgfqpoint{0.100000in}{2.413063in}}{\pgfqpoint{5.037500in}{3.427208in}}%
\pgfusepath{clip}%
\pgfsetrectcap%
\pgfsetroundjoin%
\pgfsetlinewidth{1.505625pt}%
\definecolor{currentstroke}{rgb}{0.678431,1.000000,0.184314}%
\pgfsetstrokecolor{currentstroke}%
\pgfsetstrokeopacity{0.500000}%
\pgfsetdash{}{0pt}%
\pgfpathmoveto{\pgfqpoint{0.718567in}{5.727400in}}%
\pgfusepath{stroke}%
\end{pgfscope}%
\begin{pgfscope}%
\pgfpathrectangle{\pgfqpoint{0.100000in}{2.413063in}}{\pgfqpoint{5.037500in}{3.427208in}}%
\pgfusepath{clip}%
\pgfsetbuttcap%
\pgfsetroundjoin%
\definecolor{currentfill}{rgb}{0.678431,1.000000,0.184314}%
\pgfsetfillcolor{currentfill}%
\pgfsetfillopacity{0.500000}%
\pgfsetlinewidth{0.250937pt}%
\definecolor{currentstroke}{rgb}{0.000000,0.000000,0.000000}%
\pgfsetstrokecolor{currentstroke}%
\pgfsetstrokeopacity{0.500000}%
\pgfsetdash{}{0pt}%
\pgfsys@defobject{currentmarker}{\pgfqpoint{-0.016667in}{-0.016667in}}{\pgfqpoint{0.016667in}{0.016667in}}{%
\pgfpathmoveto{\pgfqpoint{0.000000in}{-0.016667in}}%
\pgfpathcurveto{\pgfqpoint{0.004420in}{-0.016667in}}{\pgfqpoint{0.008660in}{-0.014911in}}{\pgfqpoint{0.011785in}{-0.011785in}}%
\pgfpathcurveto{\pgfqpoint{0.014911in}{-0.008660in}}{\pgfqpoint{0.016667in}{-0.004420in}}{\pgfqpoint{0.016667in}{0.000000in}}%
\pgfpathcurveto{\pgfqpoint{0.016667in}{0.004420in}}{\pgfqpoint{0.014911in}{0.008660in}}{\pgfqpoint{0.011785in}{0.011785in}}%
\pgfpathcurveto{\pgfqpoint{0.008660in}{0.014911in}}{\pgfqpoint{0.004420in}{0.016667in}}{\pgfqpoint{0.000000in}{0.016667in}}%
\pgfpathcurveto{\pgfqpoint{-0.004420in}{0.016667in}}{\pgfqpoint{-0.008660in}{0.014911in}}{\pgfqpoint{-0.011785in}{0.011785in}}%
\pgfpathcurveto{\pgfqpoint{-0.014911in}{0.008660in}}{\pgfqpoint{-0.016667in}{0.004420in}}{\pgfqpoint{-0.016667in}{0.000000in}}%
\pgfpathcurveto{\pgfqpoint{-0.016667in}{-0.004420in}}{\pgfqpoint{-0.014911in}{-0.008660in}}{\pgfqpoint{-0.011785in}{-0.011785in}}%
\pgfpathcurveto{\pgfqpoint{-0.008660in}{-0.014911in}}{\pgfqpoint{-0.004420in}{-0.016667in}}{\pgfqpoint{0.000000in}{-0.016667in}}%
\pgfpathclose%
\pgfusepath{stroke,fill}%
}%
\begin{pgfscope}%
\pgfsys@transformshift{0.718567in}{5.727400in}%
\pgfsys@useobject{currentmarker}{}%
\end{pgfscope}%
\end{pgfscope}%
\begin{pgfscope}%
\pgfpathrectangle{\pgfqpoint{0.100000in}{2.413063in}}{\pgfqpoint{5.037500in}{3.427208in}}%
\pgfusepath{clip}%
\pgfsetrectcap%
\pgfsetroundjoin%
\pgfsetlinewidth{1.505625pt}%
\definecolor{currentstroke}{rgb}{0.678431,1.000000,0.184314}%
\pgfsetstrokecolor{currentstroke}%
\pgfsetstrokeopacity{0.500000}%
\pgfsetdash{}{0pt}%
\pgfpathmoveto{\pgfqpoint{0.655467in}{5.597815in}}%
\pgfusepath{stroke}%
\end{pgfscope}%
\begin{pgfscope}%
\pgfpathrectangle{\pgfqpoint{0.100000in}{2.413063in}}{\pgfqpoint{5.037500in}{3.427208in}}%
\pgfusepath{clip}%
\pgfsetbuttcap%
\pgfsetroundjoin%
\definecolor{currentfill}{rgb}{0.678431,1.000000,0.184314}%
\pgfsetfillcolor{currentfill}%
\pgfsetfillopacity{0.500000}%
\pgfsetlinewidth{0.250937pt}%
\definecolor{currentstroke}{rgb}{0.000000,0.000000,0.000000}%
\pgfsetstrokecolor{currentstroke}%
\pgfsetstrokeopacity{0.500000}%
\pgfsetdash{}{0pt}%
\pgfsys@defobject{currentmarker}{\pgfqpoint{-0.016667in}{-0.016667in}}{\pgfqpoint{0.016667in}{0.016667in}}{%
\pgfpathmoveto{\pgfqpoint{0.000000in}{-0.016667in}}%
\pgfpathcurveto{\pgfqpoint{0.004420in}{-0.016667in}}{\pgfqpoint{0.008660in}{-0.014911in}}{\pgfqpoint{0.011785in}{-0.011785in}}%
\pgfpathcurveto{\pgfqpoint{0.014911in}{-0.008660in}}{\pgfqpoint{0.016667in}{-0.004420in}}{\pgfqpoint{0.016667in}{0.000000in}}%
\pgfpathcurveto{\pgfqpoint{0.016667in}{0.004420in}}{\pgfqpoint{0.014911in}{0.008660in}}{\pgfqpoint{0.011785in}{0.011785in}}%
\pgfpathcurveto{\pgfqpoint{0.008660in}{0.014911in}}{\pgfqpoint{0.004420in}{0.016667in}}{\pgfqpoint{0.000000in}{0.016667in}}%
\pgfpathcurveto{\pgfqpoint{-0.004420in}{0.016667in}}{\pgfqpoint{-0.008660in}{0.014911in}}{\pgfqpoint{-0.011785in}{0.011785in}}%
\pgfpathcurveto{\pgfqpoint{-0.014911in}{0.008660in}}{\pgfqpoint{-0.016667in}{0.004420in}}{\pgfqpoint{-0.016667in}{0.000000in}}%
\pgfpathcurveto{\pgfqpoint{-0.016667in}{-0.004420in}}{\pgfqpoint{-0.014911in}{-0.008660in}}{\pgfqpoint{-0.011785in}{-0.011785in}}%
\pgfpathcurveto{\pgfqpoint{-0.008660in}{-0.014911in}}{\pgfqpoint{-0.004420in}{-0.016667in}}{\pgfqpoint{0.000000in}{-0.016667in}}%
\pgfpathclose%
\pgfusepath{stroke,fill}%
}%
\begin{pgfscope}%
\pgfsys@transformshift{0.655467in}{5.597815in}%
\pgfsys@useobject{currentmarker}{}%
\end{pgfscope}%
\end{pgfscope}%
\begin{pgfscope}%
\pgfpathrectangle{\pgfqpoint{0.100000in}{2.413063in}}{\pgfqpoint{5.037500in}{3.427208in}}%
\pgfusepath{clip}%
\pgfsetrectcap%
\pgfsetroundjoin%
\pgfsetlinewidth{1.505625pt}%
\definecolor{currentstroke}{rgb}{0.678431,1.000000,0.184314}%
\pgfsetstrokecolor{currentstroke}%
\pgfsetstrokeopacity{0.500000}%
\pgfsetdash{}{0pt}%
\pgfpathmoveto{\pgfqpoint{0.888019in}{5.369454in}}%
\pgfusepath{stroke}%
\end{pgfscope}%
\begin{pgfscope}%
\pgfpathrectangle{\pgfqpoint{0.100000in}{2.413063in}}{\pgfqpoint{5.037500in}{3.427208in}}%
\pgfusepath{clip}%
\pgfsetbuttcap%
\pgfsetroundjoin%
\definecolor{currentfill}{rgb}{0.678431,1.000000,0.184314}%
\pgfsetfillcolor{currentfill}%
\pgfsetfillopacity{0.500000}%
\pgfsetlinewidth{0.250937pt}%
\definecolor{currentstroke}{rgb}{0.000000,0.000000,0.000000}%
\pgfsetstrokecolor{currentstroke}%
\pgfsetstrokeopacity{0.500000}%
\pgfsetdash{}{0pt}%
\pgfsys@defobject{currentmarker}{\pgfqpoint{-0.033333in}{-0.033333in}}{\pgfqpoint{0.033333in}{0.033333in}}{%
\pgfpathmoveto{\pgfqpoint{0.000000in}{-0.033333in}}%
\pgfpathcurveto{\pgfqpoint{0.008840in}{-0.033333in}}{\pgfqpoint{0.017319in}{-0.029821in}}{\pgfqpoint{0.023570in}{-0.023570in}}%
\pgfpathcurveto{\pgfqpoint{0.029821in}{-0.017319in}}{\pgfqpoint{0.033333in}{-0.008840in}}{\pgfqpoint{0.033333in}{0.000000in}}%
\pgfpathcurveto{\pgfqpoint{0.033333in}{0.008840in}}{\pgfqpoint{0.029821in}{0.017319in}}{\pgfqpoint{0.023570in}{0.023570in}}%
\pgfpathcurveto{\pgfqpoint{0.017319in}{0.029821in}}{\pgfqpoint{0.008840in}{0.033333in}}{\pgfqpoint{0.000000in}{0.033333in}}%
\pgfpathcurveto{\pgfqpoint{-0.008840in}{0.033333in}}{\pgfqpoint{-0.017319in}{0.029821in}}{\pgfqpoint{-0.023570in}{0.023570in}}%
\pgfpathcurveto{\pgfqpoint{-0.029821in}{0.017319in}}{\pgfqpoint{-0.033333in}{0.008840in}}{\pgfqpoint{-0.033333in}{0.000000in}}%
\pgfpathcurveto{\pgfqpoint{-0.033333in}{-0.008840in}}{\pgfqpoint{-0.029821in}{-0.017319in}}{\pgfqpoint{-0.023570in}{-0.023570in}}%
\pgfpathcurveto{\pgfqpoint{-0.017319in}{-0.029821in}}{\pgfqpoint{-0.008840in}{-0.033333in}}{\pgfqpoint{0.000000in}{-0.033333in}}%
\pgfpathclose%
\pgfusepath{stroke,fill}%
}%
\begin{pgfscope}%
\pgfsys@transformshift{0.888019in}{5.369454in}%
\pgfsys@useobject{currentmarker}{}%
\end{pgfscope}%
\end{pgfscope}%
\begin{pgfscope}%
\pgfpathrectangle{\pgfqpoint{0.100000in}{2.413063in}}{\pgfqpoint{5.037500in}{3.427208in}}%
\pgfusepath{clip}%
\pgfsetrectcap%
\pgfsetroundjoin%
\pgfsetlinewidth{1.505625pt}%
\definecolor{currentstroke}{rgb}{0.678431,1.000000,0.184314}%
\pgfsetstrokecolor{currentstroke}%
\pgfsetstrokeopacity{0.500000}%
\pgfsetdash{}{0pt}%
\pgfpathmoveto{\pgfqpoint{0.591430in}{5.448578in}}%
\pgfusepath{stroke}%
\end{pgfscope}%
\begin{pgfscope}%
\pgfpathrectangle{\pgfqpoint{0.100000in}{2.413063in}}{\pgfqpoint{5.037500in}{3.427208in}}%
\pgfusepath{clip}%
\pgfsetbuttcap%
\pgfsetroundjoin%
\definecolor{currentfill}{rgb}{0.678431,1.000000,0.184314}%
\pgfsetfillcolor{currentfill}%
\pgfsetfillopacity{0.500000}%
\pgfsetlinewidth{0.250937pt}%
\definecolor{currentstroke}{rgb}{0.000000,0.000000,0.000000}%
\pgfsetstrokecolor{currentstroke}%
\pgfsetstrokeopacity{0.500000}%
\pgfsetdash{}{0pt}%
\pgfsys@defobject{currentmarker}{\pgfqpoint{-0.022222in}{-0.022222in}}{\pgfqpoint{0.022222in}{0.022222in}}{%
\pgfpathmoveto{\pgfqpoint{0.000000in}{-0.022222in}}%
\pgfpathcurveto{\pgfqpoint{0.005893in}{-0.022222in}}{\pgfqpoint{0.011546in}{-0.019881in}}{\pgfqpoint{0.015713in}{-0.015713in}}%
\pgfpathcurveto{\pgfqpoint{0.019881in}{-0.011546in}}{\pgfqpoint{0.022222in}{-0.005893in}}{\pgfqpoint{0.022222in}{0.000000in}}%
\pgfpathcurveto{\pgfqpoint{0.022222in}{0.005893in}}{\pgfqpoint{0.019881in}{0.011546in}}{\pgfqpoint{0.015713in}{0.015713in}}%
\pgfpathcurveto{\pgfqpoint{0.011546in}{0.019881in}}{\pgfqpoint{0.005893in}{0.022222in}}{\pgfqpoint{0.000000in}{0.022222in}}%
\pgfpathcurveto{\pgfqpoint{-0.005893in}{0.022222in}}{\pgfqpoint{-0.011546in}{0.019881in}}{\pgfqpoint{-0.015713in}{0.015713in}}%
\pgfpathcurveto{\pgfqpoint{-0.019881in}{0.011546in}}{\pgfqpoint{-0.022222in}{0.005893in}}{\pgfqpoint{-0.022222in}{0.000000in}}%
\pgfpathcurveto{\pgfqpoint{-0.022222in}{-0.005893in}}{\pgfqpoint{-0.019881in}{-0.011546in}}{\pgfqpoint{-0.015713in}{-0.015713in}}%
\pgfpathcurveto{\pgfqpoint{-0.011546in}{-0.019881in}}{\pgfqpoint{-0.005893in}{-0.022222in}}{\pgfqpoint{0.000000in}{-0.022222in}}%
\pgfpathclose%
\pgfusepath{stroke,fill}%
}%
\begin{pgfscope}%
\pgfsys@transformshift{0.591430in}{5.448578in}%
\pgfsys@useobject{currentmarker}{}%
\end{pgfscope}%
\end{pgfscope}%
\begin{pgfscope}%
\pgfpathrectangle{\pgfqpoint{0.100000in}{2.413063in}}{\pgfqpoint{5.037500in}{3.427208in}}%
\pgfusepath{clip}%
\pgfsetrectcap%
\pgfsetroundjoin%
\pgfsetlinewidth{1.505625pt}%
\definecolor{currentstroke}{rgb}{0.678431,1.000000,0.184314}%
\pgfsetstrokecolor{currentstroke}%
\pgfsetstrokeopacity{0.500000}%
\pgfsetdash{}{0pt}%
\pgfpathmoveto{\pgfqpoint{0.718215in}{5.686601in}}%
\pgfusepath{stroke}%
\end{pgfscope}%
\begin{pgfscope}%
\pgfpathrectangle{\pgfqpoint{0.100000in}{2.413063in}}{\pgfqpoint{5.037500in}{3.427208in}}%
\pgfusepath{clip}%
\pgfsetbuttcap%
\pgfsetroundjoin%
\definecolor{currentfill}{rgb}{0.678431,1.000000,0.184314}%
\pgfsetfillcolor{currentfill}%
\pgfsetfillopacity{0.500000}%
\pgfsetlinewidth{0.250937pt}%
\definecolor{currentstroke}{rgb}{0.000000,0.000000,0.000000}%
\pgfsetstrokecolor{currentstroke}%
\pgfsetstrokeopacity{0.500000}%
\pgfsetdash{}{0pt}%
\pgfsys@defobject{currentmarker}{\pgfqpoint{-0.022222in}{-0.022222in}}{\pgfqpoint{0.022222in}{0.022222in}}{%
\pgfpathmoveto{\pgfqpoint{0.000000in}{-0.022222in}}%
\pgfpathcurveto{\pgfqpoint{0.005893in}{-0.022222in}}{\pgfqpoint{0.011546in}{-0.019881in}}{\pgfqpoint{0.015713in}{-0.015713in}}%
\pgfpathcurveto{\pgfqpoint{0.019881in}{-0.011546in}}{\pgfqpoint{0.022222in}{-0.005893in}}{\pgfqpoint{0.022222in}{0.000000in}}%
\pgfpathcurveto{\pgfqpoint{0.022222in}{0.005893in}}{\pgfqpoint{0.019881in}{0.011546in}}{\pgfqpoint{0.015713in}{0.015713in}}%
\pgfpathcurveto{\pgfqpoint{0.011546in}{0.019881in}}{\pgfqpoint{0.005893in}{0.022222in}}{\pgfqpoint{0.000000in}{0.022222in}}%
\pgfpathcurveto{\pgfqpoint{-0.005893in}{0.022222in}}{\pgfqpoint{-0.011546in}{0.019881in}}{\pgfqpoint{-0.015713in}{0.015713in}}%
\pgfpathcurveto{\pgfqpoint{-0.019881in}{0.011546in}}{\pgfqpoint{-0.022222in}{0.005893in}}{\pgfqpoint{-0.022222in}{0.000000in}}%
\pgfpathcurveto{\pgfqpoint{-0.022222in}{-0.005893in}}{\pgfqpoint{-0.019881in}{-0.011546in}}{\pgfqpoint{-0.015713in}{-0.015713in}}%
\pgfpathcurveto{\pgfqpoint{-0.011546in}{-0.019881in}}{\pgfqpoint{-0.005893in}{-0.022222in}}{\pgfqpoint{0.000000in}{-0.022222in}}%
\pgfpathclose%
\pgfusepath{stroke,fill}%
}%
\begin{pgfscope}%
\pgfsys@transformshift{0.718215in}{5.686601in}%
\pgfsys@useobject{currentmarker}{}%
\end{pgfscope}%
\end{pgfscope}%
\begin{pgfscope}%
\pgfpathrectangle{\pgfqpoint{0.100000in}{2.413063in}}{\pgfqpoint{5.037500in}{3.427208in}}%
\pgfusepath{clip}%
\pgfsetrectcap%
\pgfsetroundjoin%
\pgfsetlinewidth{1.505625pt}%
\definecolor{currentstroke}{rgb}{0.678431,1.000000,0.184314}%
\pgfsetstrokecolor{currentstroke}%
\pgfsetstrokeopacity{0.500000}%
\pgfsetdash{}{0pt}%
\pgfpathmoveto{\pgfqpoint{0.627763in}{5.547031in}}%
\pgfusepath{stroke}%
\end{pgfscope}%
\begin{pgfscope}%
\pgfpathrectangle{\pgfqpoint{0.100000in}{2.413063in}}{\pgfqpoint{5.037500in}{3.427208in}}%
\pgfusepath{clip}%
\pgfsetbuttcap%
\pgfsetroundjoin%
\definecolor{currentfill}{rgb}{0.678431,1.000000,0.184314}%
\pgfsetfillcolor{currentfill}%
\pgfsetfillopacity{0.500000}%
\pgfsetlinewidth{0.250937pt}%
\definecolor{currentstroke}{rgb}{0.000000,0.000000,0.000000}%
\pgfsetstrokecolor{currentstroke}%
\pgfsetstrokeopacity{0.500000}%
\pgfsetdash{}{0pt}%
\pgfsys@defobject{currentmarker}{\pgfqpoint{-0.019444in}{-0.019444in}}{\pgfqpoint{0.019444in}{0.019444in}}{%
\pgfpathmoveto{\pgfqpoint{0.000000in}{-0.019444in}}%
\pgfpathcurveto{\pgfqpoint{0.005157in}{-0.019444in}}{\pgfqpoint{0.010103in}{-0.017396in}}{\pgfqpoint{0.013749in}{-0.013749in}}%
\pgfpathcurveto{\pgfqpoint{0.017396in}{-0.010103in}}{\pgfqpoint{0.019444in}{-0.005157in}}{\pgfqpoint{0.019444in}{0.000000in}}%
\pgfpathcurveto{\pgfqpoint{0.019444in}{0.005157in}}{\pgfqpoint{0.017396in}{0.010103in}}{\pgfqpoint{0.013749in}{0.013749in}}%
\pgfpathcurveto{\pgfqpoint{0.010103in}{0.017396in}}{\pgfqpoint{0.005157in}{0.019444in}}{\pgfqpoint{0.000000in}{0.019444in}}%
\pgfpathcurveto{\pgfqpoint{-0.005157in}{0.019444in}}{\pgfqpoint{-0.010103in}{0.017396in}}{\pgfqpoint{-0.013749in}{0.013749in}}%
\pgfpathcurveto{\pgfqpoint{-0.017396in}{0.010103in}}{\pgfqpoint{-0.019444in}{0.005157in}}{\pgfqpoint{-0.019444in}{0.000000in}}%
\pgfpathcurveto{\pgfqpoint{-0.019444in}{-0.005157in}}{\pgfqpoint{-0.017396in}{-0.010103in}}{\pgfqpoint{-0.013749in}{-0.013749in}}%
\pgfpathcurveto{\pgfqpoint{-0.010103in}{-0.017396in}}{\pgfqpoint{-0.005157in}{-0.019444in}}{\pgfqpoint{0.000000in}{-0.019444in}}%
\pgfpathclose%
\pgfusepath{stroke,fill}%
}%
\begin{pgfscope}%
\pgfsys@transformshift{0.627763in}{5.547031in}%
\pgfsys@useobject{currentmarker}{}%
\end{pgfscope}%
\end{pgfscope}%
\begin{pgfscope}%
\pgfpathrectangle{\pgfqpoint{0.100000in}{2.413063in}}{\pgfqpoint{5.037500in}{3.427208in}}%
\pgfusepath{clip}%
\pgfsetrectcap%
\pgfsetroundjoin%
\pgfsetlinewidth{1.505625pt}%
\definecolor{currentstroke}{rgb}{0.000000,0.000000,1.000000}%
\pgfsetstrokecolor{currentstroke}%
\pgfsetstrokeopacity{0.500000}%
\pgfsetdash{}{0pt}%
\pgfpathmoveto{\pgfqpoint{0.689747in}{5.595764in}}%
\pgfusepath{stroke}%
\end{pgfscope}%
\begin{pgfscope}%
\pgfpathrectangle{\pgfqpoint{0.100000in}{2.413063in}}{\pgfqpoint{5.037500in}{3.427208in}}%
\pgfusepath{clip}%
\pgfsetbuttcap%
\pgfsetroundjoin%
\definecolor{currentfill}{rgb}{0.000000,0.000000,1.000000}%
\pgfsetfillcolor{currentfill}%
\pgfsetfillopacity{0.500000}%
\pgfsetlinewidth{0.250937pt}%
\definecolor{currentstroke}{rgb}{0.000000,0.000000,0.000000}%
\pgfsetstrokecolor{currentstroke}%
\pgfsetstrokeopacity{0.500000}%
\pgfsetdash{}{0pt}%
\pgfsys@defobject{currentmarker}{\pgfqpoint{-0.019444in}{-0.019444in}}{\pgfqpoint{0.019444in}{0.019444in}}{%
\pgfpathmoveto{\pgfqpoint{0.000000in}{-0.019444in}}%
\pgfpathcurveto{\pgfqpoint{0.005157in}{-0.019444in}}{\pgfqpoint{0.010103in}{-0.017396in}}{\pgfqpoint{0.013749in}{-0.013749in}}%
\pgfpathcurveto{\pgfqpoint{0.017396in}{-0.010103in}}{\pgfqpoint{0.019444in}{-0.005157in}}{\pgfqpoint{0.019444in}{0.000000in}}%
\pgfpathcurveto{\pgfqpoint{0.019444in}{0.005157in}}{\pgfqpoint{0.017396in}{0.010103in}}{\pgfqpoint{0.013749in}{0.013749in}}%
\pgfpathcurveto{\pgfqpoint{0.010103in}{0.017396in}}{\pgfqpoint{0.005157in}{0.019444in}}{\pgfqpoint{0.000000in}{0.019444in}}%
\pgfpathcurveto{\pgfqpoint{-0.005157in}{0.019444in}}{\pgfqpoint{-0.010103in}{0.017396in}}{\pgfqpoint{-0.013749in}{0.013749in}}%
\pgfpathcurveto{\pgfqpoint{-0.017396in}{0.010103in}}{\pgfqpoint{-0.019444in}{0.005157in}}{\pgfqpoint{-0.019444in}{0.000000in}}%
\pgfpathcurveto{\pgfqpoint{-0.019444in}{-0.005157in}}{\pgfqpoint{-0.017396in}{-0.010103in}}{\pgfqpoint{-0.013749in}{-0.013749in}}%
\pgfpathcurveto{\pgfqpoint{-0.010103in}{-0.017396in}}{\pgfqpoint{-0.005157in}{-0.019444in}}{\pgfqpoint{0.000000in}{-0.019444in}}%
\pgfpathclose%
\pgfusepath{stroke,fill}%
}%
\begin{pgfscope}%
\pgfsys@transformshift{0.689747in}{5.595764in}%
\pgfsys@useobject{currentmarker}{}%
\end{pgfscope}%
\end{pgfscope}%
\begin{pgfscope}%
\pgfpathrectangle{\pgfqpoint{0.100000in}{2.413063in}}{\pgfqpoint{5.037500in}{3.427208in}}%
\pgfusepath{clip}%
\pgfsetrectcap%
\pgfsetroundjoin%
\pgfsetlinewidth{1.505625pt}%
\definecolor{currentstroke}{rgb}{0.678431,1.000000,0.184314}%
\pgfsetstrokecolor{currentstroke}%
\pgfsetstrokeopacity{0.500000}%
\pgfsetdash{}{0pt}%
\pgfpathmoveto{\pgfqpoint{1.062204in}{5.497835in}}%
\pgfusepath{stroke}%
\end{pgfscope}%
\begin{pgfscope}%
\pgfpathrectangle{\pgfqpoint{0.100000in}{2.413063in}}{\pgfqpoint{5.037500in}{3.427208in}}%
\pgfusepath{clip}%
\pgfsetbuttcap%
\pgfsetroundjoin%
\definecolor{currentfill}{rgb}{0.678431,1.000000,0.184314}%
\pgfsetfillcolor{currentfill}%
\pgfsetfillopacity{0.500000}%
\pgfsetlinewidth{0.250937pt}%
\definecolor{currentstroke}{rgb}{0.000000,0.000000,0.000000}%
\pgfsetstrokecolor{currentstroke}%
\pgfsetstrokeopacity{0.500000}%
\pgfsetdash{}{0pt}%
\pgfsys@defobject{currentmarker}{\pgfqpoint{-0.030556in}{-0.030556in}}{\pgfqpoint{0.030556in}{0.030556in}}{%
\pgfpathmoveto{\pgfqpoint{0.000000in}{-0.030556in}}%
\pgfpathcurveto{\pgfqpoint{0.008103in}{-0.030556in}}{\pgfqpoint{0.015876in}{-0.027336in}}{\pgfqpoint{0.021606in}{-0.021606in}}%
\pgfpathcurveto{\pgfqpoint{0.027336in}{-0.015876in}}{\pgfqpoint{0.030556in}{-0.008103in}}{\pgfqpoint{0.030556in}{0.000000in}}%
\pgfpathcurveto{\pgfqpoint{0.030556in}{0.008103in}}{\pgfqpoint{0.027336in}{0.015876in}}{\pgfqpoint{0.021606in}{0.021606in}}%
\pgfpathcurveto{\pgfqpoint{0.015876in}{0.027336in}}{\pgfqpoint{0.008103in}{0.030556in}}{\pgfqpoint{0.000000in}{0.030556in}}%
\pgfpathcurveto{\pgfqpoint{-0.008103in}{0.030556in}}{\pgfqpoint{-0.015876in}{0.027336in}}{\pgfqpoint{-0.021606in}{0.021606in}}%
\pgfpathcurveto{\pgfqpoint{-0.027336in}{0.015876in}}{\pgfqpoint{-0.030556in}{0.008103in}}{\pgfqpoint{-0.030556in}{0.000000in}}%
\pgfpathcurveto{\pgfqpoint{-0.030556in}{-0.008103in}}{\pgfqpoint{-0.027336in}{-0.015876in}}{\pgfqpoint{-0.021606in}{-0.021606in}}%
\pgfpathcurveto{\pgfqpoint{-0.015876in}{-0.027336in}}{\pgfqpoint{-0.008103in}{-0.030556in}}{\pgfqpoint{0.000000in}{-0.030556in}}%
\pgfpathclose%
\pgfusepath{stroke,fill}%
}%
\begin{pgfscope}%
\pgfsys@transformshift{1.062204in}{5.497835in}%
\pgfsys@useobject{currentmarker}{}%
\end{pgfscope}%
\end{pgfscope}%
\begin{pgfscope}%
\pgfpathrectangle{\pgfqpoint{0.100000in}{2.413063in}}{\pgfqpoint{5.037500in}{3.427208in}}%
\pgfusepath{clip}%
\pgfsetrectcap%
\pgfsetroundjoin%
\pgfsetlinewidth{1.505625pt}%
\definecolor{currentstroke}{rgb}{0.678431,1.000000,0.184314}%
\pgfsetstrokecolor{currentstroke}%
\pgfsetstrokeopacity{0.500000}%
\pgfsetdash{}{0pt}%
\pgfpathmoveto{\pgfqpoint{0.946329in}{5.336301in}}%
\pgfusepath{stroke}%
\end{pgfscope}%
\begin{pgfscope}%
\pgfpathrectangle{\pgfqpoint{0.100000in}{2.413063in}}{\pgfqpoint{5.037500in}{3.427208in}}%
\pgfusepath{clip}%
\pgfsetbuttcap%
\pgfsetroundjoin%
\definecolor{currentfill}{rgb}{0.678431,1.000000,0.184314}%
\pgfsetfillcolor{currentfill}%
\pgfsetfillopacity{0.500000}%
\pgfsetlinewidth{0.250937pt}%
\definecolor{currentstroke}{rgb}{0.000000,0.000000,0.000000}%
\pgfsetstrokecolor{currentstroke}%
\pgfsetstrokeopacity{0.500000}%
\pgfsetdash{}{0pt}%
\pgfsys@defobject{currentmarker}{\pgfqpoint{-0.030556in}{-0.030556in}}{\pgfqpoint{0.030556in}{0.030556in}}{%
\pgfpathmoveto{\pgfqpoint{0.000000in}{-0.030556in}}%
\pgfpathcurveto{\pgfqpoint{0.008103in}{-0.030556in}}{\pgfqpoint{0.015876in}{-0.027336in}}{\pgfqpoint{0.021606in}{-0.021606in}}%
\pgfpathcurveto{\pgfqpoint{0.027336in}{-0.015876in}}{\pgfqpoint{0.030556in}{-0.008103in}}{\pgfqpoint{0.030556in}{0.000000in}}%
\pgfpathcurveto{\pgfqpoint{0.030556in}{0.008103in}}{\pgfqpoint{0.027336in}{0.015876in}}{\pgfqpoint{0.021606in}{0.021606in}}%
\pgfpathcurveto{\pgfqpoint{0.015876in}{0.027336in}}{\pgfqpoint{0.008103in}{0.030556in}}{\pgfqpoint{0.000000in}{0.030556in}}%
\pgfpathcurveto{\pgfqpoint{-0.008103in}{0.030556in}}{\pgfqpoint{-0.015876in}{0.027336in}}{\pgfqpoint{-0.021606in}{0.021606in}}%
\pgfpathcurveto{\pgfqpoint{-0.027336in}{0.015876in}}{\pgfqpoint{-0.030556in}{0.008103in}}{\pgfqpoint{-0.030556in}{0.000000in}}%
\pgfpathcurveto{\pgfqpoint{-0.030556in}{-0.008103in}}{\pgfqpoint{-0.027336in}{-0.015876in}}{\pgfqpoint{-0.021606in}{-0.021606in}}%
\pgfpathcurveto{\pgfqpoint{-0.015876in}{-0.027336in}}{\pgfqpoint{-0.008103in}{-0.030556in}}{\pgfqpoint{0.000000in}{-0.030556in}}%
\pgfpathclose%
\pgfusepath{stroke,fill}%
}%
\begin{pgfscope}%
\pgfsys@transformshift{0.946329in}{5.336301in}%
\pgfsys@useobject{currentmarker}{}%
\end{pgfscope}%
\end{pgfscope}%
\begin{pgfscope}%
\pgfpathrectangle{\pgfqpoint{0.100000in}{2.413063in}}{\pgfqpoint{5.037500in}{3.427208in}}%
\pgfusepath{clip}%
\pgfsetrectcap%
\pgfsetroundjoin%
\pgfsetlinewidth{1.505625pt}%
\definecolor{currentstroke}{rgb}{0.678431,1.000000,0.184314}%
\pgfsetstrokecolor{currentstroke}%
\pgfsetstrokeopacity{0.500000}%
\pgfsetdash{}{0pt}%
\pgfpathmoveto{\pgfqpoint{0.835996in}{5.530263in}}%
\pgfusepath{stroke}%
\end{pgfscope}%
\begin{pgfscope}%
\pgfpathrectangle{\pgfqpoint{0.100000in}{2.413063in}}{\pgfqpoint{5.037500in}{3.427208in}}%
\pgfusepath{clip}%
\pgfsetbuttcap%
\pgfsetroundjoin%
\definecolor{currentfill}{rgb}{0.678431,1.000000,0.184314}%
\pgfsetfillcolor{currentfill}%
\pgfsetfillopacity{0.500000}%
\pgfsetlinewidth{0.250937pt}%
\definecolor{currentstroke}{rgb}{0.000000,0.000000,0.000000}%
\pgfsetstrokecolor{currentstroke}%
\pgfsetstrokeopacity{0.500000}%
\pgfsetdash{}{0pt}%
\pgfsys@defobject{currentmarker}{\pgfqpoint{-0.030556in}{-0.030556in}}{\pgfqpoint{0.030556in}{0.030556in}}{%
\pgfpathmoveto{\pgfqpoint{0.000000in}{-0.030556in}}%
\pgfpathcurveto{\pgfqpoint{0.008103in}{-0.030556in}}{\pgfqpoint{0.015876in}{-0.027336in}}{\pgfqpoint{0.021606in}{-0.021606in}}%
\pgfpathcurveto{\pgfqpoint{0.027336in}{-0.015876in}}{\pgfqpoint{0.030556in}{-0.008103in}}{\pgfqpoint{0.030556in}{0.000000in}}%
\pgfpathcurveto{\pgfqpoint{0.030556in}{0.008103in}}{\pgfqpoint{0.027336in}{0.015876in}}{\pgfqpoint{0.021606in}{0.021606in}}%
\pgfpathcurveto{\pgfqpoint{0.015876in}{0.027336in}}{\pgfqpoint{0.008103in}{0.030556in}}{\pgfqpoint{0.000000in}{0.030556in}}%
\pgfpathcurveto{\pgfqpoint{-0.008103in}{0.030556in}}{\pgfqpoint{-0.015876in}{0.027336in}}{\pgfqpoint{-0.021606in}{0.021606in}}%
\pgfpathcurveto{\pgfqpoint{-0.027336in}{0.015876in}}{\pgfqpoint{-0.030556in}{0.008103in}}{\pgfqpoint{-0.030556in}{0.000000in}}%
\pgfpathcurveto{\pgfqpoint{-0.030556in}{-0.008103in}}{\pgfqpoint{-0.027336in}{-0.015876in}}{\pgfqpoint{-0.021606in}{-0.021606in}}%
\pgfpathcurveto{\pgfqpoint{-0.015876in}{-0.027336in}}{\pgfqpoint{-0.008103in}{-0.030556in}}{\pgfqpoint{0.000000in}{-0.030556in}}%
\pgfpathclose%
\pgfusepath{stroke,fill}%
}%
\begin{pgfscope}%
\pgfsys@transformshift{0.835996in}{5.530263in}%
\pgfsys@useobject{currentmarker}{}%
\end{pgfscope}%
\end{pgfscope}%
\begin{pgfscope}%
\pgfpathrectangle{\pgfqpoint{0.100000in}{2.413063in}}{\pgfqpoint{5.037500in}{3.427208in}}%
\pgfusepath{clip}%
\pgfsetrectcap%
\pgfsetroundjoin%
\pgfsetlinewidth{1.505625pt}%
\definecolor{currentstroke}{rgb}{0.678431,1.000000,0.184314}%
\pgfsetstrokecolor{currentstroke}%
\pgfsetstrokeopacity{0.500000}%
\pgfsetdash{}{0pt}%
\pgfpathmoveto{\pgfqpoint{0.794333in}{5.442772in}}%
\pgfusepath{stroke}%
\end{pgfscope}%
\begin{pgfscope}%
\pgfpathrectangle{\pgfqpoint{0.100000in}{2.413063in}}{\pgfqpoint{5.037500in}{3.427208in}}%
\pgfusepath{clip}%
\pgfsetbuttcap%
\pgfsetroundjoin%
\definecolor{currentfill}{rgb}{0.678431,1.000000,0.184314}%
\pgfsetfillcolor{currentfill}%
\pgfsetfillopacity{0.500000}%
\pgfsetlinewidth{0.250937pt}%
\definecolor{currentstroke}{rgb}{0.000000,0.000000,0.000000}%
\pgfsetstrokecolor{currentstroke}%
\pgfsetstrokeopacity{0.500000}%
\pgfsetdash{}{0pt}%
\pgfsys@defobject{currentmarker}{\pgfqpoint{-0.052778in}{-0.052778in}}{\pgfqpoint{0.052778in}{0.052778in}}{%
\pgfpathmoveto{\pgfqpoint{0.000000in}{-0.052778in}}%
\pgfpathcurveto{\pgfqpoint{0.013997in}{-0.052778in}}{\pgfqpoint{0.027422in}{-0.047217in}}{\pgfqpoint{0.037320in}{-0.037320in}}%
\pgfpathcurveto{\pgfqpoint{0.047217in}{-0.027422in}}{\pgfqpoint{0.052778in}{-0.013997in}}{\pgfqpoint{0.052778in}{0.000000in}}%
\pgfpathcurveto{\pgfqpoint{0.052778in}{0.013997in}}{\pgfqpoint{0.047217in}{0.027422in}}{\pgfqpoint{0.037320in}{0.037320in}}%
\pgfpathcurveto{\pgfqpoint{0.027422in}{0.047217in}}{\pgfqpoint{0.013997in}{0.052778in}}{\pgfqpoint{0.000000in}{0.052778in}}%
\pgfpathcurveto{\pgfqpoint{-0.013997in}{0.052778in}}{\pgfqpoint{-0.027422in}{0.047217in}}{\pgfqpoint{-0.037320in}{0.037320in}}%
\pgfpathcurveto{\pgfqpoint{-0.047217in}{0.027422in}}{\pgfqpoint{-0.052778in}{0.013997in}}{\pgfqpoint{-0.052778in}{0.000000in}}%
\pgfpathcurveto{\pgfqpoint{-0.052778in}{-0.013997in}}{\pgfqpoint{-0.047217in}{-0.027422in}}{\pgfqpoint{-0.037320in}{-0.037320in}}%
\pgfpathcurveto{\pgfqpoint{-0.027422in}{-0.047217in}}{\pgfqpoint{-0.013997in}{-0.052778in}}{\pgfqpoint{0.000000in}{-0.052778in}}%
\pgfpathclose%
\pgfusepath{stroke,fill}%
}%
\begin{pgfscope}%
\pgfsys@transformshift{0.794333in}{5.442772in}%
\pgfsys@useobject{currentmarker}{}%
\end{pgfscope}%
\end{pgfscope}%
\begin{pgfscope}%
\pgfpathrectangle{\pgfqpoint{0.100000in}{2.413063in}}{\pgfqpoint{5.037500in}{3.427208in}}%
\pgfusepath{clip}%
\pgfsetrectcap%
\pgfsetroundjoin%
\pgfsetlinewidth{1.505625pt}%
\definecolor{currentstroke}{rgb}{0.678431,1.000000,0.184314}%
\pgfsetstrokecolor{currentstroke}%
\pgfsetstrokeopacity{0.500000}%
\pgfsetdash{}{0pt}%
\pgfpathmoveto{\pgfqpoint{4.056648in}{4.235611in}}%
\pgfusepath{stroke}%
\end{pgfscope}%
\begin{pgfscope}%
\pgfpathrectangle{\pgfqpoint{0.100000in}{2.413063in}}{\pgfqpoint{5.037500in}{3.427208in}}%
\pgfusepath{clip}%
\pgfsetbuttcap%
\pgfsetroundjoin%
\definecolor{currentfill}{rgb}{0.678431,1.000000,0.184314}%
\pgfsetfillcolor{currentfill}%
\pgfsetfillopacity{0.500000}%
\pgfsetlinewidth{0.250937pt}%
\definecolor{currentstroke}{rgb}{0.000000,0.000000,0.000000}%
\pgfsetstrokecolor{currentstroke}%
\pgfsetstrokeopacity{0.500000}%
\pgfsetdash{}{0pt}%
\pgfsys@defobject{currentmarker}{\pgfqpoint{-0.047222in}{-0.047222in}}{\pgfqpoint{0.047222in}{0.047222in}}{%
\pgfpathmoveto{\pgfqpoint{0.000000in}{-0.047222in}}%
\pgfpathcurveto{\pgfqpoint{0.012523in}{-0.047222in}}{\pgfqpoint{0.024536in}{-0.042247in}}{\pgfqpoint{0.033391in}{-0.033391in}}%
\pgfpathcurveto{\pgfqpoint{0.042247in}{-0.024536in}}{\pgfqpoint{0.047222in}{-0.012523in}}{\pgfqpoint{0.047222in}{0.000000in}}%
\pgfpathcurveto{\pgfqpoint{0.047222in}{0.012523in}}{\pgfqpoint{0.042247in}{0.024536in}}{\pgfqpoint{0.033391in}{0.033391in}}%
\pgfpathcurveto{\pgfqpoint{0.024536in}{0.042247in}}{\pgfqpoint{0.012523in}{0.047222in}}{\pgfqpoint{0.000000in}{0.047222in}}%
\pgfpathcurveto{\pgfqpoint{-0.012523in}{0.047222in}}{\pgfqpoint{-0.024536in}{0.042247in}}{\pgfqpoint{-0.033391in}{0.033391in}}%
\pgfpathcurveto{\pgfqpoint{-0.042247in}{0.024536in}}{\pgfqpoint{-0.047222in}{0.012523in}}{\pgfqpoint{-0.047222in}{0.000000in}}%
\pgfpathcurveto{\pgfqpoint{-0.047222in}{-0.012523in}}{\pgfqpoint{-0.042247in}{-0.024536in}}{\pgfqpoint{-0.033391in}{-0.033391in}}%
\pgfpathcurveto{\pgfqpoint{-0.024536in}{-0.042247in}}{\pgfqpoint{-0.012523in}{-0.047222in}}{\pgfqpoint{0.000000in}{-0.047222in}}%
\pgfpathclose%
\pgfusepath{stroke,fill}%
}%
\begin{pgfscope}%
\pgfsys@transformshift{4.056648in}{4.235611in}%
\pgfsys@useobject{currentmarker}{}%
\end{pgfscope}%
\end{pgfscope}%
\begin{pgfscope}%
\pgfpathrectangle{\pgfqpoint{0.100000in}{2.413063in}}{\pgfqpoint{5.037500in}{3.427208in}}%
\pgfusepath{clip}%
\pgfsetrectcap%
\pgfsetroundjoin%
\pgfsetlinewidth{1.505625pt}%
\definecolor{currentstroke}{rgb}{0.678431,1.000000,0.184314}%
\pgfsetstrokecolor{currentstroke}%
\pgfsetstrokeopacity{0.500000}%
\pgfsetdash{}{0pt}%
\pgfpathmoveto{\pgfqpoint{4.006916in}{4.294787in}}%
\pgfusepath{stroke}%
\end{pgfscope}%
\begin{pgfscope}%
\pgfpathrectangle{\pgfqpoint{0.100000in}{2.413063in}}{\pgfqpoint{5.037500in}{3.427208in}}%
\pgfusepath{clip}%
\pgfsetbuttcap%
\pgfsetroundjoin%
\definecolor{currentfill}{rgb}{0.678431,1.000000,0.184314}%
\pgfsetfillcolor{currentfill}%
\pgfsetfillopacity{0.500000}%
\pgfsetlinewidth{0.250937pt}%
\definecolor{currentstroke}{rgb}{0.000000,0.000000,0.000000}%
\pgfsetstrokecolor{currentstroke}%
\pgfsetstrokeopacity{0.500000}%
\pgfsetdash{}{0pt}%
\pgfsys@defobject{currentmarker}{\pgfqpoint{-0.038889in}{-0.038889in}}{\pgfqpoint{0.038889in}{0.038889in}}{%
\pgfpathmoveto{\pgfqpoint{0.000000in}{-0.038889in}}%
\pgfpathcurveto{\pgfqpoint{0.010313in}{-0.038889in}}{\pgfqpoint{0.020206in}{-0.034791in}}{\pgfqpoint{0.027499in}{-0.027499in}}%
\pgfpathcurveto{\pgfqpoint{0.034791in}{-0.020206in}}{\pgfqpoint{0.038889in}{-0.010313in}}{\pgfqpoint{0.038889in}{0.000000in}}%
\pgfpathcurveto{\pgfqpoint{0.038889in}{0.010313in}}{\pgfqpoint{0.034791in}{0.020206in}}{\pgfqpoint{0.027499in}{0.027499in}}%
\pgfpathcurveto{\pgfqpoint{0.020206in}{0.034791in}}{\pgfqpoint{0.010313in}{0.038889in}}{\pgfqpoint{0.000000in}{0.038889in}}%
\pgfpathcurveto{\pgfqpoint{-0.010313in}{0.038889in}}{\pgfqpoint{-0.020206in}{0.034791in}}{\pgfqpoint{-0.027499in}{0.027499in}}%
\pgfpathcurveto{\pgfqpoint{-0.034791in}{0.020206in}}{\pgfqpoint{-0.038889in}{0.010313in}}{\pgfqpoint{-0.038889in}{0.000000in}}%
\pgfpathcurveto{\pgfqpoint{-0.038889in}{-0.010313in}}{\pgfqpoint{-0.034791in}{-0.020206in}}{\pgfqpoint{-0.027499in}{-0.027499in}}%
\pgfpathcurveto{\pgfqpoint{-0.020206in}{-0.034791in}}{\pgfqpoint{-0.010313in}{-0.038889in}}{\pgfqpoint{0.000000in}{-0.038889in}}%
\pgfpathclose%
\pgfusepath{stroke,fill}%
}%
\begin{pgfscope}%
\pgfsys@transformshift{4.006916in}{4.294787in}%
\pgfsys@useobject{currentmarker}{}%
\end{pgfscope}%
\end{pgfscope}%
\begin{pgfscope}%
\pgfpathrectangle{\pgfqpoint{0.100000in}{2.413063in}}{\pgfqpoint{5.037500in}{3.427208in}}%
\pgfusepath{clip}%
\pgfsetrectcap%
\pgfsetroundjoin%
\pgfsetlinewidth{1.505625pt}%
\definecolor{currentstroke}{rgb}{0.678431,1.000000,0.184314}%
\pgfsetstrokecolor{currentstroke}%
\pgfsetstrokeopacity{0.500000}%
\pgfsetdash{}{0pt}%
\pgfpathmoveto{\pgfqpoint{3.933253in}{4.292196in}}%
\pgfusepath{stroke}%
\end{pgfscope}%
\begin{pgfscope}%
\pgfpathrectangle{\pgfqpoint{0.100000in}{2.413063in}}{\pgfqpoint{5.037500in}{3.427208in}}%
\pgfusepath{clip}%
\pgfsetbuttcap%
\pgfsetroundjoin%
\definecolor{currentfill}{rgb}{0.678431,1.000000,0.184314}%
\pgfsetfillcolor{currentfill}%
\pgfsetfillopacity{0.500000}%
\pgfsetlinewidth{0.250937pt}%
\definecolor{currentstroke}{rgb}{0.000000,0.000000,0.000000}%
\pgfsetstrokecolor{currentstroke}%
\pgfsetstrokeopacity{0.500000}%
\pgfsetdash{}{0pt}%
\pgfsys@defobject{currentmarker}{\pgfqpoint{-0.033333in}{-0.033333in}}{\pgfqpoint{0.033333in}{0.033333in}}{%
\pgfpathmoveto{\pgfqpoint{0.000000in}{-0.033333in}}%
\pgfpathcurveto{\pgfqpoint{0.008840in}{-0.033333in}}{\pgfqpoint{0.017319in}{-0.029821in}}{\pgfqpoint{0.023570in}{-0.023570in}}%
\pgfpathcurveto{\pgfqpoint{0.029821in}{-0.017319in}}{\pgfqpoint{0.033333in}{-0.008840in}}{\pgfqpoint{0.033333in}{0.000000in}}%
\pgfpathcurveto{\pgfqpoint{0.033333in}{0.008840in}}{\pgfqpoint{0.029821in}{0.017319in}}{\pgfqpoint{0.023570in}{0.023570in}}%
\pgfpathcurveto{\pgfqpoint{0.017319in}{0.029821in}}{\pgfqpoint{0.008840in}{0.033333in}}{\pgfqpoint{0.000000in}{0.033333in}}%
\pgfpathcurveto{\pgfqpoint{-0.008840in}{0.033333in}}{\pgfqpoint{-0.017319in}{0.029821in}}{\pgfqpoint{-0.023570in}{0.023570in}}%
\pgfpathcurveto{\pgfqpoint{-0.029821in}{0.017319in}}{\pgfqpoint{-0.033333in}{0.008840in}}{\pgfqpoint{-0.033333in}{0.000000in}}%
\pgfpathcurveto{\pgfqpoint{-0.033333in}{-0.008840in}}{\pgfqpoint{-0.029821in}{-0.017319in}}{\pgfqpoint{-0.023570in}{-0.023570in}}%
\pgfpathcurveto{\pgfqpoint{-0.017319in}{-0.029821in}}{\pgfqpoint{-0.008840in}{-0.033333in}}{\pgfqpoint{0.000000in}{-0.033333in}}%
\pgfpathclose%
\pgfusepath{stroke,fill}%
}%
\begin{pgfscope}%
\pgfsys@transformshift{3.933253in}{4.292196in}%
\pgfsys@useobject{currentmarker}{}%
\end{pgfscope}%
\end{pgfscope}%
\begin{pgfscope}%
\pgfpathrectangle{\pgfqpoint{0.100000in}{2.413063in}}{\pgfqpoint{5.037500in}{3.427208in}}%
\pgfusepath{clip}%
\pgfsetrectcap%
\pgfsetroundjoin%
\pgfsetlinewidth{1.505625pt}%
\definecolor{currentstroke}{rgb}{0.678431,1.000000,0.184314}%
\pgfsetstrokecolor{currentstroke}%
\pgfsetstrokeopacity{0.500000}%
\pgfsetdash{}{0pt}%
\pgfpathmoveto{\pgfqpoint{4.132275in}{4.463641in}}%
\pgfusepath{stroke}%
\end{pgfscope}%
\begin{pgfscope}%
\pgfpathrectangle{\pgfqpoint{0.100000in}{2.413063in}}{\pgfqpoint{5.037500in}{3.427208in}}%
\pgfusepath{clip}%
\pgfsetbuttcap%
\pgfsetroundjoin%
\definecolor{currentfill}{rgb}{0.678431,1.000000,0.184314}%
\pgfsetfillcolor{currentfill}%
\pgfsetfillopacity{0.500000}%
\pgfsetlinewidth{0.250937pt}%
\definecolor{currentstroke}{rgb}{0.000000,0.000000,0.000000}%
\pgfsetstrokecolor{currentstroke}%
\pgfsetstrokeopacity{0.500000}%
\pgfsetdash{}{0pt}%
\pgfsys@defobject{currentmarker}{\pgfqpoint{-0.025000in}{-0.025000in}}{\pgfqpoint{0.025000in}{0.025000in}}{%
\pgfpathmoveto{\pgfqpoint{0.000000in}{-0.025000in}}%
\pgfpathcurveto{\pgfqpoint{0.006630in}{-0.025000in}}{\pgfqpoint{0.012989in}{-0.022366in}}{\pgfqpoint{0.017678in}{-0.017678in}}%
\pgfpathcurveto{\pgfqpoint{0.022366in}{-0.012989in}}{\pgfqpoint{0.025000in}{-0.006630in}}{\pgfqpoint{0.025000in}{0.000000in}}%
\pgfpathcurveto{\pgfqpoint{0.025000in}{0.006630in}}{\pgfqpoint{0.022366in}{0.012989in}}{\pgfqpoint{0.017678in}{0.017678in}}%
\pgfpathcurveto{\pgfqpoint{0.012989in}{0.022366in}}{\pgfqpoint{0.006630in}{0.025000in}}{\pgfqpoint{0.000000in}{0.025000in}}%
\pgfpathcurveto{\pgfqpoint{-0.006630in}{0.025000in}}{\pgfqpoint{-0.012989in}{0.022366in}}{\pgfqpoint{-0.017678in}{0.017678in}}%
\pgfpathcurveto{\pgfqpoint{-0.022366in}{0.012989in}}{\pgfqpoint{-0.025000in}{0.006630in}}{\pgfqpoint{-0.025000in}{0.000000in}}%
\pgfpathcurveto{\pgfqpoint{-0.025000in}{-0.006630in}}{\pgfqpoint{-0.022366in}{-0.012989in}}{\pgfqpoint{-0.017678in}{-0.017678in}}%
\pgfpathcurveto{\pgfqpoint{-0.012989in}{-0.022366in}}{\pgfqpoint{-0.006630in}{-0.025000in}}{\pgfqpoint{0.000000in}{-0.025000in}}%
\pgfpathclose%
\pgfusepath{stroke,fill}%
}%
\begin{pgfscope}%
\pgfsys@transformshift{4.132275in}{4.463641in}%
\pgfsys@useobject{currentmarker}{}%
\end{pgfscope}%
\end{pgfscope}%
\begin{pgfscope}%
\pgfpathrectangle{\pgfqpoint{0.100000in}{2.413063in}}{\pgfqpoint{5.037500in}{3.427208in}}%
\pgfusepath{clip}%
\pgfsetrectcap%
\pgfsetroundjoin%
\pgfsetlinewidth{1.505625pt}%
\definecolor{currentstroke}{rgb}{0.678431,1.000000,0.184314}%
\pgfsetstrokecolor{currentstroke}%
\pgfsetstrokeopacity{0.500000}%
\pgfsetdash{}{0pt}%
\pgfpathmoveto{\pgfqpoint{3.998684in}{4.400590in}}%
\pgfusepath{stroke}%
\end{pgfscope}%
\begin{pgfscope}%
\pgfpathrectangle{\pgfqpoint{0.100000in}{2.413063in}}{\pgfqpoint{5.037500in}{3.427208in}}%
\pgfusepath{clip}%
\pgfsetbuttcap%
\pgfsetroundjoin%
\definecolor{currentfill}{rgb}{0.678431,1.000000,0.184314}%
\pgfsetfillcolor{currentfill}%
\pgfsetfillopacity{0.500000}%
\pgfsetlinewidth{0.250937pt}%
\definecolor{currentstroke}{rgb}{0.000000,0.000000,0.000000}%
\pgfsetstrokecolor{currentstroke}%
\pgfsetstrokeopacity{0.500000}%
\pgfsetdash{}{0pt}%
\pgfsys@defobject{currentmarker}{\pgfqpoint{-0.044444in}{-0.044444in}}{\pgfqpoint{0.044444in}{0.044444in}}{%
\pgfpathmoveto{\pgfqpoint{0.000000in}{-0.044444in}}%
\pgfpathcurveto{\pgfqpoint{0.011787in}{-0.044444in}}{\pgfqpoint{0.023092in}{-0.039761in}}{\pgfqpoint{0.031427in}{-0.031427in}}%
\pgfpathcurveto{\pgfqpoint{0.039761in}{-0.023092in}}{\pgfqpoint{0.044444in}{-0.011787in}}{\pgfqpoint{0.044444in}{0.000000in}}%
\pgfpathcurveto{\pgfqpoint{0.044444in}{0.011787in}}{\pgfqpoint{0.039761in}{0.023092in}}{\pgfqpoint{0.031427in}{0.031427in}}%
\pgfpathcurveto{\pgfqpoint{0.023092in}{0.039761in}}{\pgfqpoint{0.011787in}{0.044444in}}{\pgfqpoint{0.000000in}{0.044444in}}%
\pgfpathcurveto{\pgfqpoint{-0.011787in}{0.044444in}}{\pgfqpoint{-0.023092in}{0.039761in}}{\pgfqpoint{-0.031427in}{0.031427in}}%
\pgfpathcurveto{\pgfqpoint{-0.039761in}{0.023092in}}{\pgfqpoint{-0.044444in}{0.011787in}}{\pgfqpoint{-0.044444in}{0.000000in}}%
\pgfpathcurveto{\pgfqpoint{-0.044444in}{-0.011787in}}{\pgfqpoint{-0.039761in}{-0.023092in}}{\pgfqpoint{-0.031427in}{-0.031427in}}%
\pgfpathcurveto{\pgfqpoint{-0.023092in}{-0.039761in}}{\pgfqpoint{-0.011787in}{-0.044444in}}{\pgfqpoint{0.000000in}{-0.044444in}}%
\pgfpathclose%
\pgfusepath{stroke,fill}%
}%
\begin{pgfscope}%
\pgfsys@transformshift{3.998684in}{4.400590in}%
\pgfsys@useobject{currentmarker}{}%
\end{pgfscope}%
\end{pgfscope}%
\begin{pgfscope}%
\pgfpathrectangle{\pgfqpoint{0.100000in}{2.413063in}}{\pgfqpoint{5.037500in}{3.427208in}}%
\pgfusepath{clip}%
\pgfsetrectcap%
\pgfsetroundjoin%
\pgfsetlinewidth{1.505625pt}%
\definecolor{currentstroke}{rgb}{0.678431,1.000000,0.184314}%
\pgfsetstrokecolor{currentstroke}%
\pgfsetstrokeopacity{0.500000}%
\pgfsetdash{}{0pt}%
\pgfpathmoveto{\pgfqpoint{4.057487in}{4.502137in}}%
\pgfusepath{stroke}%
\end{pgfscope}%
\begin{pgfscope}%
\pgfpathrectangle{\pgfqpoint{0.100000in}{2.413063in}}{\pgfqpoint{5.037500in}{3.427208in}}%
\pgfusepath{clip}%
\pgfsetbuttcap%
\pgfsetroundjoin%
\definecolor{currentfill}{rgb}{0.678431,1.000000,0.184314}%
\pgfsetfillcolor{currentfill}%
\pgfsetfillopacity{0.500000}%
\pgfsetlinewidth{0.250937pt}%
\definecolor{currentstroke}{rgb}{0.000000,0.000000,0.000000}%
\pgfsetstrokecolor{currentstroke}%
\pgfsetstrokeopacity{0.500000}%
\pgfsetdash{}{0pt}%
\pgfsys@defobject{currentmarker}{\pgfqpoint{-0.061111in}{-0.061111in}}{\pgfqpoint{0.061111in}{0.061111in}}{%
\pgfpathmoveto{\pgfqpoint{0.000000in}{-0.061111in}}%
\pgfpathcurveto{\pgfqpoint{0.016207in}{-0.061111in}}{\pgfqpoint{0.031752in}{-0.054672in}}{\pgfqpoint{0.043212in}{-0.043212in}}%
\pgfpathcurveto{\pgfqpoint{0.054672in}{-0.031752in}}{\pgfqpoint{0.061111in}{-0.016207in}}{\pgfqpoint{0.061111in}{0.000000in}}%
\pgfpathcurveto{\pgfqpoint{0.061111in}{0.016207in}}{\pgfqpoint{0.054672in}{0.031752in}}{\pgfqpoint{0.043212in}{0.043212in}}%
\pgfpathcurveto{\pgfqpoint{0.031752in}{0.054672in}}{\pgfqpoint{0.016207in}{0.061111in}}{\pgfqpoint{0.000000in}{0.061111in}}%
\pgfpathcurveto{\pgfqpoint{-0.016207in}{0.061111in}}{\pgfqpoint{-0.031752in}{0.054672in}}{\pgfqpoint{-0.043212in}{0.043212in}}%
\pgfpathcurveto{\pgfqpoint{-0.054672in}{0.031752in}}{\pgfqpoint{-0.061111in}{0.016207in}}{\pgfqpoint{-0.061111in}{0.000000in}}%
\pgfpathcurveto{\pgfqpoint{-0.061111in}{-0.016207in}}{\pgfqpoint{-0.054672in}{-0.031752in}}{\pgfqpoint{-0.043212in}{-0.043212in}}%
\pgfpathcurveto{\pgfqpoint{-0.031752in}{-0.054672in}}{\pgfqpoint{-0.016207in}{-0.061111in}}{\pgfqpoint{0.000000in}{-0.061111in}}%
\pgfpathclose%
\pgfusepath{stroke,fill}%
}%
\begin{pgfscope}%
\pgfsys@transformshift{4.057487in}{4.502137in}%
\pgfsys@useobject{currentmarker}{}%
\end{pgfscope}%
\end{pgfscope}%
\begin{pgfscope}%
\pgfpathrectangle{\pgfqpoint{0.100000in}{2.413063in}}{\pgfqpoint{5.037500in}{3.427208in}}%
\pgfusepath{clip}%
\pgfsetrectcap%
\pgfsetroundjoin%
\pgfsetlinewidth{1.505625pt}%
\definecolor{currentstroke}{rgb}{0.678431,1.000000,0.184314}%
\pgfsetstrokecolor{currentstroke}%
\pgfsetstrokeopacity{0.500000}%
\pgfsetdash{}{0pt}%
\pgfpathmoveto{\pgfqpoint{3.349641in}{4.907379in}}%
\pgfusepath{stroke}%
\end{pgfscope}%
\begin{pgfscope}%
\pgfpathrectangle{\pgfqpoint{0.100000in}{2.413063in}}{\pgfqpoint{5.037500in}{3.427208in}}%
\pgfusepath{clip}%
\pgfsetbuttcap%
\pgfsetroundjoin%
\definecolor{currentfill}{rgb}{0.678431,1.000000,0.184314}%
\pgfsetfillcolor{currentfill}%
\pgfsetfillopacity{0.500000}%
\pgfsetlinewidth{0.250937pt}%
\definecolor{currentstroke}{rgb}{0.000000,0.000000,0.000000}%
\pgfsetstrokecolor{currentstroke}%
\pgfsetstrokeopacity{0.500000}%
\pgfsetdash{}{0pt}%
\pgfsys@defobject{currentmarker}{\pgfqpoint{-0.025000in}{-0.025000in}}{\pgfqpoint{0.025000in}{0.025000in}}{%
\pgfpathmoveto{\pgfqpoint{0.000000in}{-0.025000in}}%
\pgfpathcurveto{\pgfqpoint{0.006630in}{-0.025000in}}{\pgfqpoint{0.012989in}{-0.022366in}}{\pgfqpoint{0.017678in}{-0.017678in}}%
\pgfpathcurveto{\pgfqpoint{0.022366in}{-0.012989in}}{\pgfqpoint{0.025000in}{-0.006630in}}{\pgfqpoint{0.025000in}{0.000000in}}%
\pgfpathcurveto{\pgfqpoint{0.025000in}{0.006630in}}{\pgfqpoint{0.022366in}{0.012989in}}{\pgfqpoint{0.017678in}{0.017678in}}%
\pgfpathcurveto{\pgfqpoint{0.012989in}{0.022366in}}{\pgfqpoint{0.006630in}{0.025000in}}{\pgfqpoint{0.000000in}{0.025000in}}%
\pgfpathcurveto{\pgfqpoint{-0.006630in}{0.025000in}}{\pgfqpoint{-0.012989in}{0.022366in}}{\pgfqpoint{-0.017678in}{0.017678in}}%
\pgfpathcurveto{\pgfqpoint{-0.022366in}{0.012989in}}{\pgfqpoint{-0.025000in}{0.006630in}}{\pgfqpoint{-0.025000in}{0.000000in}}%
\pgfpathcurveto{\pgfqpoint{-0.025000in}{-0.006630in}}{\pgfqpoint{-0.022366in}{-0.012989in}}{\pgfqpoint{-0.017678in}{-0.017678in}}%
\pgfpathcurveto{\pgfqpoint{-0.012989in}{-0.022366in}}{\pgfqpoint{-0.006630in}{-0.025000in}}{\pgfqpoint{0.000000in}{-0.025000in}}%
\pgfpathclose%
\pgfusepath{stroke,fill}%
}%
\begin{pgfscope}%
\pgfsys@transformshift{3.349641in}{4.907379in}%
\pgfsys@useobject{currentmarker}{}%
\end{pgfscope}%
\end{pgfscope}%
\begin{pgfscope}%
\pgfpathrectangle{\pgfqpoint{0.100000in}{2.413063in}}{\pgfqpoint{5.037500in}{3.427208in}}%
\pgfusepath{clip}%
\pgfsetrectcap%
\pgfsetroundjoin%
\pgfsetlinewidth{1.505625pt}%
\definecolor{currentstroke}{rgb}{0.678431,1.000000,0.184314}%
\pgfsetstrokecolor{currentstroke}%
\pgfsetstrokeopacity{0.500000}%
\pgfsetdash{}{0pt}%
\pgfpathmoveto{\pgfqpoint{3.091567in}{4.956771in}}%
\pgfusepath{stroke}%
\end{pgfscope}%
\begin{pgfscope}%
\pgfpathrectangle{\pgfqpoint{0.100000in}{2.413063in}}{\pgfqpoint{5.037500in}{3.427208in}}%
\pgfusepath{clip}%
\pgfsetbuttcap%
\pgfsetroundjoin%
\definecolor{currentfill}{rgb}{0.678431,1.000000,0.184314}%
\pgfsetfillcolor{currentfill}%
\pgfsetfillopacity{0.500000}%
\pgfsetlinewidth{0.250937pt}%
\definecolor{currentstroke}{rgb}{0.000000,0.000000,0.000000}%
\pgfsetstrokecolor{currentstroke}%
\pgfsetstrokeopacity{0.500000}%
\pgfsetdash{}{0pt}%
\pgfsys@defobject{currentmarker}{\pgfqpoint{-0.036111in}{-0.036111in}}{\pgfqpoint{0.036111in}{0.036111in}}{%
\pgfpathmoveto{\pgfqpoint{0.000000in}{-0.036111in}}%
\pgfpathcurveto{\pgfqpoint{0.009577in}{-0.036111in}}{\pgfqpoint{0.018763in}{-0.032306in}}{\pgfqpoint{0.025534in}{-0.025534in}}%
\pgfpathcurveto{\pgfqpoint{0.032306in}{-0.018763in}}{\pgfqpoint{0.036111in}{-0.009577in}}{\pgfqpoint{0.036111in}{0.000000in}}%
\pgfpathcurveto{\pgfqpoint{0.036111in}{0.009577in}}{\pgfqpoint{0.032306in}{0.018763in}}{\pgfqpoint{0.025534in}{0.025534in}}%
\pgfpathcurveto{\pgfqpoint{0.018763in}{0.032306in}}{\pgfqpoint{0.009577in}{0.036111in}}{\pgfqpoint{0.000000in}{0.036111in}}%
\pgfpathcurveto{\pgfqpoint{-0.009577in}{0.036111in}}{\pgfqpoint{-0.018763in}{0.032306in}}{\pgfqpoint{-0.025534in}{0.025534in}}%
\pgfpathcurveto{\pgfqpoint{-0.032306in}{0.018763in}}{\pgfqpoint{-0.036111in}{0.009577in}}{\pgfqpoint{-0.036111in}{0.000000in}}%
\pgfpathcurveto{\pgfqpoint{-0.036111in}{-0.009577in}}{\pgfqpoint{-0.032306in}{-0.018763in}}{\pgfqpoint{-0.025534in}{-0.025534in}}%
\pgfpathcurveto{\pgfqpoint{-0.018763in}{-0.032306in}}{\pgfqpoint{-0.009577in}{-0.036111in}}{\pgfqpoint{0.000000in}{-0.036111in}}%
\pgfpathclose%
\pgfusepath{stroke,fill}%
}%
\begin{pgfscope}%
\pgfsys@transformshift{3.091567in}{4.956771in}%
\pgfsys@useobject{currentmarker}{}%
\end{pgfscope}%
\end{pgfscope}%
\begin{pgfscope}%
\pgfpathrectangle{\pgfqpoint{0.100000in}{2.413063in}}{\pgfqpoint{5.037500in}{3.427208in}}%
\pgfusepath{clip}%
\pgfsetrectcap%
\pgfsetroundjoin%
\pgfsetlinewidth{1.505625pt}%
\definecolor{currentstroke}{rgb}{0.678431,1.000000,0.184314}%
\pgfsetstrokecolor{currentstroke}%
\pgfsetstrokeopacity{0.500000}%
\pgfsetdash{}{0pt}%
\pgfpathmoveto{\pgfqpoint{3.350467in}{4.851087in}}%
\pgfusepath{stroke}%
\end{pgfscope}%
\begin{pgfscope}%
\pgfpathrectangle{\pgfqpoint{0.100000in}{2.413063in}}{\pgfqpoint{5.037500in}{3.427208in}}%
\pgfusepath{clip}%
\pgfsetbuttcap%
\pgfsetroundjoin%
\definecolor{currentfill}{rgb}{0.678431,1.000000,0.184314}%
\pgfsetfillcolor{currentfill}%
\pgfsetfillopacity{0.500000}%
\pgfsetlinewidth{0.250937pt}%
\definecolor{currentstroke}{rgb}{0.000000,0.000000,0.000000}%
\pgfsetstrokecolor{currentstroke}%
\pgfsetstrokeopacity{0.500000}%
\pgfsetdash{}{0pt}%
\pgfsys@defobject{currentmarker}{\pgfqpoint{-0.027778in}{-0.027778in}}{\pgfqpoint{0.027778in}{0.027778in}}{%
\pgfpathmoveto{\pgfqpoint{0.000000in}{-0.027778in}}%
\pgfpathcurveto{\pgfqpoint{0.007367in}{-0.027778in}}{\pgfqpoint{0.014433in}{-0.024851in}}{\pgfqpoint{0.019642in}{-0.019642in}}%
\pgfpathcurveto{\pgfqpoint{0.024851in}{-0.014433in}}{\pgfqpoint{0.027778in}{-0.007367in}}{\pgfqpoint{0.027778in}{0.000000in}}%
\pgfpathcurveto{\pgfqpoint{0.027778in}{0.007367in}}{\pgfqpoint{0.024851in}{0.014433in}}{\pgfqpoint{0.019642in}{0.019642in}}%
\pgfpathcurveto{\pgfqpoint{0.014433in}{0.024851in}}{\pgfqpoint{0.007367in}{0.027778in}}{\pgfqpoint{0.000000in}{0.027778in}}%
\pgfpathcurveto{\pgfqpoint{-0.007367in}{0.027778in}}{\pgfqpoint{-0.014433in}{0.024851in}}{\pgfqpoint{-0.019642in}{0.019642in}}%
\pgfpathcurveto{\pgfqpoint{-0.024851in}{0.014433in}}{\pgfqpoint{-0.027778in}{0.007367in}}{\pgfqpoint{-0.027778in}{0.000000in}}%
\pgfpathcurveto{\pgfqpoint{-0.027778in}{-0.007367in}}{\pgfqpoint{-0.024851in}{-0.014433in}}{\pgfqpoint{-0.019642in}{-0.019642in}}%
\pgfpathcurveto{\pgfqpoint{-0.014433in}{-0.024851in}}{\pgfqpoint{-0.007367in}{-0.027778in}}{\pgfqpoint{0.000000in}{-0.027778in}}%
\pgfpathclose%
\pgfusepath{stroke,fill}%
}%
\begin{pgfscope}%
\pgfsys@transformshift{3.350467in}{4.851087in}%
\pgfsys@useobject{currentmarker}{}%
\end{pgfscope}%
\end{pgfscope}%
\begin{pgfscope}%
\pgfpathrectangle{\pgfqpoint{0.100000in}{2.413063in}}{\pgfqpoint{5.037500in}{3.427208in}}%
\pgfusepath{clip}%
\pgfsetrectcap%
\pgfsetroundjoin%
\pgfsetlinewidth{1.505625pt}%
\definecolor{currentstroke}{rgb}{0.678431,1.000000,0.184314}%
\pgfsetstrokecolor{currentstroke}%
\pgfsetstrokeopacity{0.500000}%
\pgfsetdash{}{0pt}%
\pgfpathmoveto{\pgfqpoint{3.378797in}{4.940736in}}%
\pgfusepath{stroke}%
\end{pgfscope}%
\begin{pgfscope}%
\pgfpathrectangle{\pgfqpoint{0.100000in}{2.413063in}}{\pgfqpoint{5.037500in}{3.427208in}}%
\pgfusepath{clip}%
\pgfsetbuttcap%
\pgfsetroundjoin%
\definecolor{currentfill}{rgb}{0.678431,1.000000,0.184314}%
\pgfsetfillcolor{currentfill}%
\pgfsetfillopacity{0.500000}%
\pgfsetlinewidth{0.250937pt}%
\definecolor{currentstroke}{rgb}{0.000000,0.000000,0.000000}%
\pgfsetstrokecolor{currentstroke}%
\pgfsetstrokeopacity{0.500000}%
\pgfsetdash{}{0pt}%
\pgfsys@defobject{currentmarker}{\pgfqpoint{-0.027778in}{-0.027778in}}{\pgfqpoint{0.027778in}{0.027778in}}{%
\pgfpathmoveto{\pgfqpoint{0.000000in}{-0.027778in}}%
\pgfpathcurveto{\pgfqpoint{0.007367in}{-0.027778in}}{\pgfqpoint{0.014433in}{-0.024851in}}{\pgfqpoint{0.019642in}{-0.019642in}}%
\pgfpathcurveto{\pgfqpoint{0.024851in}{-0.014433in}}{\pgfqpoint{0.027778in}{-0.007367in}}{\pgfqpoint{0.027778in}{0.000000in}}%
\pgfpathcurveto{\pgfqpoint{0.027778in}{0.007367in}}{\pgfqpoint{0.024851in}{0.014433in}}{\pgfqpoint{0.019642in}{0.019642in}}%
\pgfpathcurveto{\pgfqpoint{0.014433in}{0.024851in}}{\pgfqpoint{0.007367in}{0.027778in}}{\pgfqpoint{0.000000in}{0.027778in}}%
\pgfpathcurveto{\pgfqpoint{-0.007367in}{0.027778in}}{\pgfqpoint{-0.014433in}{0.024851in}}{\pgfqpoint{-0.019642in}{0.019642in}}%
\pgfpathcurveto{\pgfqpoint{-0.024851in}{0.014433in}}{\pgfqpoint{-0.027778in}{0.007367in}}{\pgfqpoint{-0.027778in}{0.000000in}}%
\pgfpathcurveto{\pgfqpoint{-0.027778in}{-0.007367in}}{\pgfqpoint{-0.024851in}{-0.014433in}}{\pgfqpoint{-0.019642in}{-0.019642in}}%
\pgfpathcurveto{\pgfqpoint{-0.014433in}{-0.024851in}}{\pgfqpoint{-0.007367in}{-0.027778in}}{\pgfqpoint{0.000000in}{-0.027778in}}%
\pgfpathclose%
\pgfusepath{stroke,fill}%
}%
\begin{pgfscope}%
\pgfsys@transformshift{3.378797in}{4.940736in}%
\pgfsys@useobject{currentmarker}{}%
\end{pgfscope}%
\end{pgfscope}%
\begin{pgfscope}%
\pgfpathrectangle{\pgfqpoint{0.100000in}{2.413063in}}{\pgfqpoint{5.037500in}{3.427208in}}%
\pgfusepath{clip}%
\pgfsetrectcap%
\pgfsetroundjoin%
\pgfsetlinewidth{1.505625pt}%
\definecolor{currentstroke}{rgb}{0.678431,1.000000,0.184314}%
\pgfsetstrokecolor{currentstroke}%
\pgfsetstrokeopacity{0.500000}%
\pgfsetdash{}{0pt}%
\pgfpathmoveto{\pgfqpoint{3.313193in}{4.725863in}}%
\pgfusepath{stroke}%
\end{pgfscope}%
\begin{pgfscope}%
\pgfpathrectangle{\pgfqpoint{0.100000in}{2.413063in}}{\pgfqpoint{5.037500in}{3.427208in}}%
\pgfusepath{clip}%
\pgfsetbuttcap%
\pgfsetroundjoin%
\definecolor{currentfill}{rgb}{0.678431,1.000000,0.184314}%
\pgfsetfillcolor{currentfill}%
\pgfsetfillopacity{0.500000}%
\pgfsetlinewidth{0.250937pt}%
\definecolor{currentstroke}{rgb}{0.000000,0.000000,0.000000}%
\pgfsetstrokecolor{currentstroke}%
\pgfsetstrokeopacity{0.500000}%
\pgfsetdash{}{0pt}%
\pgfsys@defobject{currentmarker}{\pgfqpoint{-0.027778in}{-0.027778in}}{\pgfqpoint{0.027778in}{0.027778in}}{%
\pgfpathmoveto{\pgfqpoint{0.000000in}{-0.027778in}}%
\pgfpathcurveto{\pgfqpoint{0.007367in}{-0.027778in}}{\pgfqpoint{0.014433in}{-0.024851in}}{\pgfqpoint{0.019642in}{-0.019642in}}%
\pgfpathcurveto{\pgfqpoint{0.024851in}{-0.014433in}}{\pgfqpoint{0.027778in}{-0.007367in}}{\pgfqpoint{0.027778in}{0.000000in}}%
\pgfpathcurveto{\pgfqpoint{0.027778in}{0.007367in}}{\pgfqpoint{0.024851in}{0.014433in}}{\pgfqpoint{0.019642in}{0.019642in}}%
\pgfpathcurveto{\pgfqpoint{0.014433in}{0.024851in}}{\pgfqpoint{0.007367in}{0.027778in}}{\pgfqpoint{0.000000in}{0.027778in}}%
\pgfpathcurveto{\pgfqpoint{-0.007367in}{0.027778in}}{\pgfqpoint{-0.014433in}{0.024851in}}{\pgfqpoint{-0.019642in}{0.019642in}}%
\pgfpathcurveto{\pgfqpoint{-0.024851in}{0.014433in}}{\pgfqpoint{-0.027778in}{0.007367in}}{\pgfqpoint{-0.027778in}{0.000000in}}%
\pgfpathcurveto{\pgfqpoint{-0.027778in}{-0.007367in}}{\pgfqpoint{-0.024851in}{-0.014433in}}{\pgfqpoint{-0.019642in}{-0.019642in}}%
\pgfpathcurveto{\pgfqpoint{-0.014433in}{-0.024851in}}{\pgfqpoint{-0.007367in}{-0.027778in}}{\pgfqpoint{0.000000in}{-0.027778in}}%
\pgfpathclose%
\pgfusepath{stroke,fill}%
}%
\begin{pgfscope}%
\pgfsys@transformshift{3.313193in}{4.725863in}%
\pgfsys@useobject{currentmarker}{}%
\end{pgfscope}%
\end{pgfscope}%
\begin{pgfscope}%
\pgfpathrectangle{\pgfqpoint{0.100000in}{2.413063in}}{\pgfqpoint{5.037500in}{3.427208in}}%
\pgfusepath{clip}%
\pgfsetrectcap%
\pgfsetroundjoin%
\pgfsetlinewidth{1.505625pt}%
\definecolor{currentstroke}{rgb}{0.678431,1.000000,0.184314}%
\pgfsetstrokecolor{currentstroke}%
\pgfsetstrokeopacity{0.500000}%
\pgfsetdash{}{0pt}%
\pgfpathmoveto{\pgfqpoint{3.117641in}{4.840924in}}%
\pgfusepath{stroke}%
\end{pgfscope}%
\begin{pgfscope}%
\pgfpathrectangle{\pgfqpoint{0.100000in}{2.413063in}}{\pgfqpoint{5.037500in}{3.427208in}}%
\pgfusepath{clip}%
\pgfsetbuttcap%
\pgfsetroundjoin%
\definecolor{currentfill}{rgb}{0.678431,1.000000,0.184314}%
\pgfsetfillcolor{currentfill}%
\pgfsetfillopacity{0.500000}%
\pgfsetlinewidth{0.250937pt}%
\definecolor{currentstroke}{rgb}{0.000000,0.000000,0.000000}%
\pgfsetstrokecolor{currentstroke}%
\pgfsetstrokeopacity{0.500000}%
\pgfsetdash{}{0pt}%
\pgfsys@defobject{currentmarker}{\pgfqpoint{-0.027778in}{-0.027778in}}{\pgfqpoint{0.027778in}{0.027778in}}{%
\pgfpathmoveto{\pgfqpoint{0.000000in}{-0.027778in}}%
\pgfpathcurveto{\pgfqpoint{0.007367in}{-0.027778in}}{\pgfqpoint{0.014433in}{-0.024851in}}{\pgfqpoint{0.019642in}{-0.019642in}}%
\pgfpathcurveto{\pgfqpoint{0.024851in}{-0.014433in}}{\pgfqpoint{0.027778in}{-0.007367in}}{\pgfqpoint{0.027778in}{0.000000in}}%
\pgfpathcurveto{\pgfqpoint{0.027778in}{0.007367in}}{\pgfqpoint{0.024851in}{0.014433in}}{\pgfqpoint{0.019642in}{0.019642in}}%
\pgfpathcurveto{\pgfqpoint{0.014433in}{0.024851in}}{\pgfqpoint{0.007367in}{0.027778in}}{\pgfqpoint{0.000000in}{0.027778in}}%
\pgfpathcurveto{\pgfqpoint{-0.007367in}{0.027778in}}{\pgfqpoint{-0.014433in}{0.024851in}}{\pgfqpoint{-0.019642in}{0.019642in}}%
\pgfpathcurveto{\pgfqpoint{-0.024851in}{0.014433in}}{\pgfqpoint{-0.027778in}{0.007367in}}{\pgfqpoint{-0.027778in}{0.000000in}}%
\pgfpathcurveto{\pgfqpoint{-0.027778in}{-0.007367in}}{\pgfqpoint{-0.024851in}{-0.014433in}}{\pgfqpoint{-0.019642in}{-0.019642in}}%
\pgfpathcurveto{\pgfqpoint{-0.014433in}{-0.024851in}}{\pgfqpoint{-0.007367in}{-0.027778in}}{\pgfqpoint{0.000000in}{-0.027778in}}%
\pgfpathclose%
\pgfusepath{stroke,fill}%
}%
\begin{pgfscope}%
\pgfsys@transformshift{3.117641in}{4.840924in}%
\pgfsys@useobject{currentmarker}{}%
\end{pgfscope}%
\end{pgfscope}%
\begin{pgfscope}%
\pgfpathrectangle{\pgfqpoint{0.100000in}{2.413063in}}{\pgfqpoint{5.037500in}{3.427208in}}%
\pgfusepath{clip}%
\pgfsetrectcap%
\pgfsetroundjoin%
\pgfsetlinewidth{1.505625pt}%
\definecolor{currentstroke}{rgb}{0.678431,1.000000,0.184314}%
\pgfsetstrokecolor{currentstroke}%
\pgfsetstrokeopacity{0.500000}%
\pgfsetdash{}{0pt}%
\pgfpathmoveto{\pgfqpoint{3.277383in}{4.765152in}}%
\pgfusepath{stroke}%
\end{pgfscope}%
\begin{pgfscope}%
\pgfpathrectangle{\pgfqpoint{0.100000in}{2.413063in}}{\pgfqpoint{5.037500in}{3.427208in}}%
\pgfusepath{clip}%
\pgfsetbuttcap%
\pgfsetroundjoin%
\definecolor{currentfill}{rgb}{0.678431,1.000000,0.184314}%
\pgfsetfillcolor{currentfill}%
\pgfsetfillopacity{0.500000}%
\pgfsetlinewidth{0.250937pt}%
\definecolor{currentstroke}{rgb}{0.000000,0.000000,0.000000}%
\pgfsetstrokecolor{currentstroke}%
\pgfsetstrokeopacity{0.500000}%
\pgfsetdash{}{0pt}%
\pgfsys@defobject{currentmarker}{\pgfqpoint{-0.019444in}{-0.019444in}}{\pgfqpoint{0.019444in}{0.019444in}}{%
\pgfpathmoveto{\pgfqpoint{0.000000in}{-0.019444in}}%
\pgfpathcurveto{\pgfqpoint{0.005157in}{-0.019444in}}{\pgfqpoint{0.010103in}{-0.017396in}}{\pgfqpoint{0.013749in}{-0.013749in}}%
\pgfpathcurveto{\pgfqpoint{0.017396in}{-0.010103in}}{\pgfqpoint{0.019444in}{-0.005157in}}{\pgfqpoint{0.019444in}{0.000000in}}%
\pgfpathcurveto{\pgfqpoint{0.019444in}{0.005157in}}{\pgfqpoint{0.017396in}{0.010103in}}{\pgfqpoint{0.013749in}{0.013749in}}%
\pgfpathcurveto{\pgfqpoint{0.010103in}{0.017396in}}{\pgfqpoint{0.005157in}{0.019444in}}{\pgfqpoint{0.000000in}{0.019444in}}%
\pgfpathcurveto{\pgfqpoint{-0.005157in}{0.019444in}}{\pgfqpoint{-0.010103in}{0.017396in}}{\pgfqpoint{-0.013749in}{0.013749in}}%
\pgfpathcurveto{\pgfqpoint{-0.017396in}{0.010103in}}{\pgfqpoint{-0.019444in}{0.005157in}}{\pgfqpoint{-0.019444in}{0.000000in}}%
\pgfpathcurveto{\pgfqpoint{-0.019444in}{-0.005157in}}{\pgfqpoint{-0.017396in}{-0.010103in}}{\pgfqpoint{-0.013749in}{-0.013749in}}%
\pgfpathcurveto{\pgfqpoint{-0.010103in}{-0.017396in}}{\pgfqpoint{-0.005157in}{-0.019444in}}{\pgfqpoint{0.000000in}{-0.019444in}}%
\pgfpathclose%
\pgfusepath{stroke,fill}%
}%
\begin{pgfscope}%
\pgfsys@transformshift{3.277383in}{4.765152in}%
\pgfsys@useobject{currentmarker}{}%
\end{pgfscope}%
\end{pgfscope}%
\begin{pgfscope}%
\pgfpathrectangle{\pgfqpoint{0.100000in}{2.413063in}}{\pgfqpoint{5.037500in}{3.427208in}}%
\pgfusepath{clip}%
\pgfsetrectcap%
\pgfsetroundjoin%
\pgfsetlinewidth{1.505625pt}%
\definecolor{currentstroke}{rgb}{0.678431,1.000000,0.184314}%
\pgfsetstrokecolor{currentstroke}%
\pgfsetstrokeopacity{0.500000}%
\pgfsetdash{}{0pt}%
\pgfpathmoveto{\pgfqpoint{3.400651in}{4.769173in}}%
\pgfusepath{stroke}%
\end{pgfscope}%
\begin{pgfscope}%
\pgfpathrectangle{\pgfqpoint{0.100000in}{2.413063in}}{\pgfqpoint{5.037500in}{3.427208in}}%
\pgfusepath{clip}%
\pgfsetbuttcap%
\pgfsetroundjoin%
\definecolor{currentfill}{rgb}{0.678431,1.000000,0.184314}%
\pgfsetfillcolor{currentfill}%
\pgfsetfillopacity{0.500000}%
\pgfsetlinewidth{0.250937pt}%
\definecolor{currentstroke}{rgb}{0.000000,0.000000,0.000000}%
\pgfsetstrokecolor{currentstroke}%
\pgfsetstrokeopacity{0.500000}%
\pgfsetdash{}{0pt}%
\pgfsys@defobject{currentmarker}{\pgfqpoint{-0.019444in}{-0.019444in}}{\pgfqpoint{0.019444in}{0.019444in}}{%
\pgfpathmoveto{\pgfqpoint{0.000000in}{-0.019444in}}%
\pgfpathcurveto{\pgfqpoint{0.005157in}{-0.019444in}}{\pgfqpoint{0.010103in}{-0.017396in}}{\pgfqpoint{0.013749in}{-0.013749in}}%
\pgfpathcurveto{\pgfqpoint{0.017396in}{-0.010103in}}{\pgfqpoint{0.019444in}{-0.005157in}}{\pgfqpoint{0.019444in}{0.000000in}}%
\pgfpathcurveto{\pgfqpoint{0.019444in}{0.005157in}}{\pgfqpoint{0.017396in}{0.010103in}}{\pgfqpoint{0.013749in}{0.013749in}}%
\pgfpathcurveto{\pgfqpoint{0.010103in}{0.017396in}}{\pgfqpoint{0.005157in}{0.019444in}}{\pgfqpoint{0.000000in}{0.019444in}}%
\pgfpathcurveto{\pgfqpoint{-0.005157in}{0.019444in}}{\pgfqpoint{-0.010103in}{0.017396in}}{\pgfqpoint{-0.013749in}{0.013749in}}%
\pgfpathcurveto{\pgfqpoint{-0.017396in}{0.010103in}}{\pgfqpoint{-0.019444in}{0.005157in}}{\pgfqpoint{-0.019444in}{0.000000in}}%
\pgfpathcurveto{\pgfqpoint{-0.019444in}{-0.005157in}}{\pgfqpoint{-0.017396in}{-0.010103in}}{\pgfqpoint{-0.013749in}{-0.013749in}}%
\pgfpathcurveto{\pgfqpoint{-0.010103in}{-0.017396in}}{\pgfqpoint{-0.005157in}{-0.019444in}}{\pgfqpoint{0.000000in}{-0.019444in}}%
\pgfpathclose%
\pgfusepath{stroke,fill}%
}%
\begin{pgfscope}%
\pgfsys@transformshift{3.400651in}{4.769173in}%
\pgfsys@useobject{currentmarker}{}%
\end{pgfscope}%
\end{pgfscope}%
\begin{pgfscope}%
\pgfpathrectangle{\pgfqpoint{0.100000in}{2.413063in}}{\pgfqpoint{5.037500in}{3.427208in}}%
\pgfusepath{clip}%
\pgfsetrectcap%
\pgfsetroundjoin%
\pgfsetlinewidth{1.505625pt}%
\definecolor{currentstroke}{rgb}{0.678431,1.000000,0.184314}%
\pgfsetstrokecolor{currentstroke}%
\pgfsetstrokeopacity{0.500000}%
\pgfsetdash{}{0pt}%
\pgfpathmoveto{\pgfqpoint{3.340539in}{4.878855in}}%
\pgfusepath{stroke}%
\end{pgfscope}%
\begin{pgfscope}%
\pgfpathrectangle{\pgfqpoint{0.100000in}{2.413063in}}{\pgfqpoint{5.037500in}{3.427208in}}%
\pgfusepath{clip}%
\pgfsetbuttcap%
\pgfsetroundjoin%
\definecolor{currentfill}{rgb}{0.678431,1.000000,0.184314}%
\pgfsetfillcolor{currentfill}%
\pgfsetfillopacity{0.500000}%
\pgfsetlinewidth{0.250937pt}%
\definecolor{currentstroke}{rgb}{0.000000,0.000000,0.000000}%
\pgfsetstrokecolor{currentstroke}%
\pgfsetstrokeopacity{0.500000}%
\pgfsetdash{}{0pt}%
\pgfsys@defobject{currentmarker}{\pgfqpoint{-0.027778in}{-0.027778in}}{\pgfqpoint{0.027778in}{0.027778in}}{%
\pgfpathmoveto{\pgfqpoint{0.000000in}{-0.027778in}}%
\pgfpathcurveto{\pgfqpoint{0.007367in}{-0.027778in}}{\pgfqpoint{0.014433in}{-0.024851in}}{\pgfqpoint{0.019642in}{-0.019642in}}%
\pgfpathcurveto{\pgfqpoint{0.024851in}{-0.014433in}}{\pgfqpoint{0.027778in}{-0.007367in}}{\pgfqpoint{0.027778in}{0.000000in}}%
\pgfpathcurveto{\pgfqpoint{0.027778in}{0.007367in}}{\pgfqpoint{0.024851in}{0.014433in}}{\pgfqpoint{0.019642in}{0.019642in}}%
\pgfpathcurveto{\pgfqpoint{0.014433in}{0.024851in}}{\pgfqpoint{0.007367in}{0.027778in}}{\pgfqpoint{0.000000in}{0.027778in}}%
\pgfpathcurveto{\pgfqpoint{-0.007367in}{0.027778in}}{\pgfqpoint{-0.014433in}{0.024851in}}{\pgfqpoint{-0.019642in}{0.019642in}}%
\pgfpathcurveto{\pgfqpoint{-0.024851in}{0.014433in}}{\pgfqpoint{-0.027778in}{0.007367in}}{\pgfqpoint{-0.027778in}{0.000000in}}%
\pgfpathcurveto{\pgfqpoint{-0.027778in}{-0.007367in}}{\pgfqpoint{-0.024851in}{-0.014433in}}{\pgfqpoint{-0.019642in}{-0.019642in}}%
\pgfpathcurveto{\pgfqpoint{-0.014433in}{-0.024851in}}{\pgfqpoint{-0.007367in}{-0.027778in}}{\pgfqpoint{0.000000in}{-0.027778in}}%
\pgfpathclose%
\pgfusepath{stroke,fill}%
}%
\begin{pgfscope}%
\pgfsys@transformshift{3.340539in}{4.878855in}%
\pgfsys@useobject{currentmarker}{}%
\end{pgfscope}%
\end{pgfscope}%
\begin{pgfscope}%
\pgfpathrectangle{\pgfqpoint{0.100000in}{2.413063in}}{\pgfqpoint{5.037500in}{3.427208in}}%
\pgfusepath{clip}%
\pgfsetrectcap%
\pgfsetroundjoin%
\pgfsetlinewidth{1.505625pt}%
\definecolor{currentstroke}{rgb}{0.678431,1.000000,0.184314}%
\pgfsetstrokecolor{currentstroke}%
\pgfsetstrokeopacity{0.500000}%
\pgfsetdash{}{0pt}%
\pgfpathmoveto{\pgfqpoint{3.415252in}{4.734560in}}%
\pgfusepath{stroke}%
\end{pgfscope}%
\begin{pgfscope}%
\pgfpathrectangle{\pgfqpoint{0.100000in}{2.413063in}}{\pgfqpoint{5.037500in}{3.427208in}}%
\pgfusepath{clip}%
\pgfsetbuttcap%
\pgfsetroundjoin%
\definecolor{currentfill}{rgb}{0.678431,1.000000,0.184314}%
\pgfsetfillcolor{currentfill}%
\pgfsetfillopacity{0.500000}%
\pgfsetlinewidth{0.250937pt}%
\definecolor{currentstroke}{rgb}{0.000000,0.000000,0.000000}%
\pgfsetstrokecolor{currentstroke}%
\pgfsetstrokeopacity{0.500000}%
\pgfsetdash{}{0pt}%
\pgfsys@defobject{currentmarker}{\pgfqpoint{-0.030556in}{-0.030556in}}{\pgfqpoint{0.030556in}{0.030556in}}{%
\pgfpathmoveto{\pgfqpoint{0.000000in}{-0.030556in}}%
\pgfpathcurveto{\pgfqpoint{0.008103in}{-0.030556in}}{\pgfqpoint{0.015876in}{-0.027336in}}{\pgfqpoint{0.021606in}{-0.021606in}}%
\pgfpathcurveto{\pgfqpoint{0.027336in}{-0.015876in}}{\pgfqpoint{0.030556in}{-0.008103in}}{\pgfqpoint{0.030556in}{0.000000in}}%
\pgfpathcurveto{\pgfqpoint{0.030556in}{0.008103in}}{\pgfqpoint{0.027336in}{0.015876in}}{\pgfqpoint{0.021606in}{0.021606in}}%
\pgfpathcurveto{\pgfqpoint{0.015876in}{0.027336in}}{\pgfqpoint{0.008103in}{0.030556in}}{\pgfqpoint{0.000000in}{0.030556in}}%
\pgfpathcurveto{\pgfqpoint{-0.008103in}{0.030556in}}{\pgfqpoint{-0.015876in}{0.027336in}}{\pgfqpoint{-0.021606in}{0.021606in}}%
\pgfpathcurveto{\pgfqpoint{-0.027336in}{0.015876in}}{\pgfqpoint{-0.030556in}{0.008103in}}{\pgfqpoint{-0.030556in}{0.000000in}}%
\pgfpathcurveto{\pgfqpoint{-0.030556in}{-0.008103in}}{\pgfqpoint{-0.027336in}{-0.015876in}}{\pgfqpoint{-0.021606in}{-0.021606in}}%
\pgfpathcurveto{\pgfqpoint{-0.015876in}{-0.027336in}}{\pgfqpoint{-0.008103in}{-0.030556in}}{\pgfqpoint{0.000000in}{-0.030556in}}%
\pgfpathclose%
\pgfusepath{stroke,fill}%
}%
\begin{pgfscope}%
\pgfsys@transformshift{3.415252in}{4.734560in}%
\pgfsys@useobject{currentmarker}{}%
\end{pgfscope}%
\end{pgfscope}%
\begin{pgfscope}%
\pgfpathrectangle{\pgfqpoint{0.100000in}{2.413063in}}{\pgfqpoint{5.037500in}{3.427208in}}%
\pgfusepath{clip}%
\pgfsetrectcap%
\pgfsetroundjoin%
\pgfsetlinewidth{1.505625pt}%
\definecolor{currentstroke}{rgb}{0.678431,1.000000,0.184314}%
\pgfsetstrokecolor{currentstroke}%
\pgfsetstrokeopacity{0.500000}%
\pgfsetdash{}{0pt}%
\pgfpathmoveto{\pgfqpoint{3.411522in}{4.852958in}}%
\pgfusepath{stroke}%
\end{pgfscope}%
\begin{pgfscope}%
\pgfpathrectangle{\pgfqpoint{0.100000in}{2.413063in}}{\pgfqpoint{5.037500in}{3.427208in}}%
\pgfusepath{clip}%
\pgfsetbuttcap%
\pgfsetroundjoin%
\definecolor{currentfill}{rgb}{0.678431,1.000000,0.184314}%
\pgfsetfillcolor{currentfill}%
\pgfsetfillopacity{0.500000}%
\pgfsetlinewidth{0.250937pt}%
\definecolor{currentstroke}{rgb}{0.000000,0.000000,0.000000}%
\pgfsetstrokecolor{currentstroke}%
\pgfsetstrokeopacity{0.500000}%
\pgfsetdash{}{0pt}%
\pgfsys@defobject{currentmarker}{\pgfqpoint{-0.027778in}{-0.027778in}}{\pgfqpoint{0.027778in}{0.027778in}}{%
\pgfpathmoveto{\pgfqpoint{0.000000in}{-0.027778in}}%
\pgfpathcurveto{\pgfqpoint{0.007367in}{-0.027778in}}{\pgfqpoint{0.014433in}{-0.024851in}}{\pgfqpoint{0.019642in}{-0.019642in}}%
\pgfpathcurveto{\pgfqpoint{0.024851in}{-0.014433in}}{\pgfqpoint{0.027778in}{-0.007367in}}{\pgfqpoint{0.027778in}{0.000000in}}%
\pgfpathcurveto{\pgfqpoint{0.027778in}{0.007367in}}{\pgfqpoint{0.024851in}{0.014433in}}{\pgfqpoint{0.019642in}{0.019642in}}%
\pgfpathcurveto{\pgfqpoint{0.014433in}{0.024851in}}{\pgfqpoint{0.007367in}{0.027778in}}{\pgfqpoint{0.000000in}{0.027778in}}%
\pgfpathcurveto{\pgfqpoint{-0.007367in}{0.027778in}}{\pgfqpoint{-0.014433in}{0.024851in}}{\pgfqpoint{-0.019642in}{0.019642in}}%
\pgfpathcurveto{\pgfqpoint{-0.024851in}{0.014433in}}{\pgfqpoint{-0.027778in}{0.007367in}}{\pgfqpoint{-0.027778in}{0.000000in}}%
\pgfpathcurveto{\pgfqpoint{-0.027778in}{-0.007367in}}{\pgfqpoint{-0.024851in}{-0.014433in}}{\pgfqpoint{-0.019642in}{-0.019642in}}%
\pgfpathcurveto{\pgfqpoint{-0.014433in}{-0.024851in}}{\pgfqpoint{-0.007367in}{-0.027778in}}{\pgfqpoint{0.000000in}{-0.027778in}}%
\pgfpathclose%
\pgfusepath{stroke,fill}%
}%
\begin{pgfscope}%
\pgfsys@transformshift{3.411522in}{4.852958in}%
\pgfsys@useobject{currentmarker}{}%
\end{pgfscope}%
\end{pgfscope}%
\begin{pgfscope}%
\pgfpathrectangle{\pgfqpoint{0.100000in}{2.413063in}}{\pgfqpoint{5.037500in}{3.427208in}}%
\pgfusepath{clip}%
\pgfsetrectcap%
\pgfsetroundjoin%
\pgfsetlinewidth{1.505625pt}%
\definecolor{currentstroke}{rgb}{0.678431,1.000000,0.184314}%
\pgfsetstrokecolor{currentstroke}%
\pgfsetstrokeopacity{0.500000}%
\pgfsetdash{}{0pt}%
\pgfpathmoveto{\pgfqpoint{3.243806in}{4.981389in}}%
\pgfusepath{stroke}%
\end{pgfscope}%
\begin{pgfscope}%
\pgfpathrectangle{\pgfqpoint{0.100000in}{2.413063in}}{\pgfqpoint{5.037500in}{3.427208in}}%
\pgfusepath{clip}%
\pgfsetbuttcap%
\pgfsetroundjoin%
\definecolor{currentfill}{rgb}{0.678431,1.000000,0.184314}%
\pgfsetfillcolor{currentfill}%
\pgfsetfillopacity{0.500000}%
\pgfsetlinewidth{0.250937pt}%
\definecolor{currentstroke}{rgb}{0.000000,0.000000,0.000000}%
\pgfsetstrokecolor{currentstroke}%
\pgfsetstrokeopacity{0.500000}%
\pgfsetdash{}{0pt}%
\pgfsys@defobject{currentmarker}{\pgfqpoint{-0.030556in}{-0.030556in}}{\pgfqpoint{0.030556in}{0.030556in}}{%
\pgfpathmoveto{\pgfqpoint{0.000000in}{-0.030556in}}%
\pgfpathcurveto{\pgfqpoint{0.008103in}{-0.030556in}}{\pgfqpoint{0.015876in}{-0.027336in}}{\pgfqpoint{0.021606in}{-0.021606in}}%
\pgfpathcurveto{\pgfqpoint{0.027336in}{-0.015876in}}{\pgfqpoint{0.030556in}{-0.008103in}}{\pgfqpoint{0.030556in}{0.000000in}}%
\pgfpathcurveto{\pgfqpoint{0.030556in}{0.008103in}}{\pgfqpoint{0.027336in}{0.015876in}}{\pgfqpoint{0.021606in}{0.021606in}}%
\pgfpathcurveto{\pgfqpoint{0.015876in}{0.027336in}}{\pgfqpoint{0.008103in}{0.030556in}}{\pgfqpoint{0.000000in}{0.030556in}}%
\pgfpathcurveto{\pgfqpoint{-0.008103in}{0.030556in}}{\pgfqpoint{-0.015876in}{0.027336in}}{\pgfqpoint{-0.021606in}{0.021606in}}%
\pgfpathcurveto{\pgfqpoint{-0.027336in}{0.015876in}}{\pgfqpoint{-0.030556in}{0.008103in}}{\pgfqpoint{-0.030556in}{0.000000in}}%
\pgfpathcurveto{\pgfqpoint{-0.030556in}{-0.008103in}}{\pgfqpoint{-0.027336in}{-0.015876in}}{\pgfqpoint{-0.021606in}{-0.021606in}}%
\pgfpathcurveto{\pgfqpoint{-0.015876in}{-0.027336in}}{\pgfqpoint{-0.008103in}{-0.030556in}}{\pgfqpoint{0.000000in}{-0.030556in}}%
\pgfpathclose%
\pgfusepath{stroke,fill}%
}%
\begin{pgfscope}%
\pgfsys@transformshift{3.243806in}{4.981389in}%
\pgfsys@useobject{currentmarker}{}%
\end{pgfscope}%
\end{pgfscope}%
\begin{pgfscope}%
\pgfpathrectangle{\pgfqpoint{0.100000in}{2.413063in}}{\pgfqpoint{5.037500in}{3.427208in}}%
\pgfusepath{clip}%
\pgfsetrectcap%
\pgfsetroundjoin%
\pgfsetlinewidth{1.505625pt}%
\definecolor{currentstroke}{rgb}{0.678431,1.000000,0.184314}%
\pgfsetstrokecolor{currentstroke}%
\pgfsetstrokeopacity{0.500000}%
\pgfsetdash{}{0pt}%
\pgfpathmoveto{\pgfqpoint{1.849699in}{4.785986in}}%
\pgfusepath{stroke}%
\end{pgfscope}%
\begin{pgfscope}%
\pgfpathrectangle{\pgfqpoint{0.100000in}{2.413063in}}{\pgfqpoint{5.037500in}{3.427208in}}%
\pgfusepath{clip}%
\pgfsetbuttcap%
\pgfsetroundjoin%
\definecolor{currentfill}{rgb}{0.678431,1.000000,0.184314}%
\pgfsetfillcolor{currentfill}%
\pgfsetfillopacity{0.500000}%
\pgfsetlinewidth{0.250937pt}%
\definecolor{currentstroke}{rgb}{0.000000,0.000000,0.000000}%
\pgfsetstrokecolor{currentstroke}%
\pgfsetstrokeopacity{0.500000}%
\pgfsetdash{}{0pt}%
\pgfsys@defobject{currentmarker}{\pgfqpoint{-0.038889in}{-0.038889in}}{\pgfqpoint{0.038889in}{0.038889in}}{%
\pgfpathmoveto{\pgfqpoint{0.000000in}{-0.038889in}}%
\pgfpathcurveto{\pgfqpoint{0.010313in}{-0.038889in}}{\pgfqpoint{0.020206in}{-0.034791in}}{\pgfqpoint{0.027499in}{-0.027499in}}%
\pgfpathcurveto{\pgfqpoint{0.034791in}{-0.020206in}}{\pgfqpoint{0.038889in}{-0.010313in}}{\pgfqpoint{0.038889in}{0.000000in}}%
\pgfpathcurveto{\pgfqpoint{0.038889in}{0.010313in}}{\pgfqpoint{0.034791in}{0.020206in}}{\pgfqpoint{0.027499in}{0.027499in}}%
\pgfpathcurveto{\pgfqpoint{0.020206in}{0.034791in}}{\pgfqpoint{0.010313in}{0.038889in}}{\pgfqpoint{0.000000in}{0.038889in}}%
\pgfpathcurveto{\pgfqpoint{-0.010313in}{0.038889in}}{\pgfqpoint{-0.020206in}{0.034791in}}{\pgfqpoint{-0.027499in}{0.027499in}}%
\pgfpathcurveto{\pgfqpoint{-0.034791in}{0.020206in}}{\pgfqpoint{-0.038889in}{0.010313in}}{\pgfqpoint{-0.038889in}{0.000000in}}%
\pgfpathcurveto{\pgfqpoint{-0.038889in}{-0.010313in}}{\pgfqpoint{-0.034791in}{-0.020206in}}{\pgfqpoint{-0.027499in}{-0.027499in}}%
\pgfpathcurveto{\pgfqpoint{-0.020206in}{-0.034791in}}{\pgfqpoint{-0.010313in}{-0.038889in}}{\pgfqpoint{0.000000in}{-0.038889in}}%
\pgfpathclose%
\pgfusepath{stroke,fill}%
}%
\begin{pgfscope}%
\pgfsys@transformshift{1.849699in}{4.785986in}%
\pgfsys@useobject{currentmarker}{}%
\end{pgfscope}%
\end{pgfscope}%
\begin{pgfscope}%
\pgfpathrectangle{\pgfqpoint{0.100000in}{2.413063in}}{\pgfqpoint{5.037500in}{3.427208in}}%
\pgfusepath{clip}%
\pgfsetrectcap%
\pgfsetroundjoin%
\pgfsetlinewidth{1.505625pt}%
\definecolor{currentstroke}{rgb}{0.678431,1.000000,0.184314}%
\pgfsetstrokecolor{currentstroke}%
\pgfsetstrokeopacity{0.500000}%
\pgfsetdash{}{0pt}%
\pgfpathmoveto{\pgfqpoint{1.953642in}{4.573410in}}%
\pgfusepath{stroke}%
\end{pgfscope}%
\begin{pgfscope}%
\pgfpathrectangle{\pgfqpoint{0.100000in}{2.413063in}}{\pgfqpoint{5.037500in}{3.427208in}}%
\pgfusepath{clip}%
\pgfsetbuttcap%
\pgfsetroundjoin%
\definecolor{currentfill}{rgb}{0.678431,1.000000,0.184314}%
\pgfsetfillcolor{currentfill}%
\pgfsetfillopacity{0.500000}%
\pgfsetlinewidth{0.250937pt}%
\definecolor{currentstroke}{rgb}{0.000000,0.000000,0.000000}%
\pgfsetstrokecolor{currentstroke}%
\pgfsetstrokeopacity{0.500000}%
\pgfsetdash{}{0pt}%
\pgfsys@defobject{currentmarker}{\pgfqpoint{-0.044444in}{-0.044444in}}{\pgfqpoint{0.044444in}{0.044444in}}{%
\pgfpathmoveto{\pgfqpoint{0.000000in}{-0.044444in}}%
\pgfpathcurveto{\pgfqpoint{0.011787in}{-0.044444in}}{\pgfqpoint{0.023092in}{-0.039761in}}{\pgfqpoint{0.031427in}{-0.031427in}}%
\pgfpathcurveto{\pgfqpoint{0.039761in}{-0.023092in}}{\pgfqpoint{0.044444in}{-0.011787in}}{\pgfqpoint{0.044444in}{0.000000in}}%
\pgfpathcurveto{\pgfqpoint{0.044444in}{0.011787in}}{\pgfqpoint{0.039761in}{0.023092in}}{\pgfqpoint{0.031427in}{0.031427in}}%
\pgfpathcurveto{\pgfqpoint{0.023092in}{0.039761in}}{\pgfqpoint{0.011787in}{0.044444in}}{\pgfqpoint{0.000000in}{0.044444in}}%
\pgfpathcurveto{\pgfqpoint{-0.011787in}{0.044444in}}{\pgfqpoint{-0.023092in}{0.039761in}}{\pgfqpoint{-0.031427in}{0.031427in}}%
\pgfpathcurveto{\pgfqpoint{-0.039761in}{0.023092in}}{\pgfqpoint{-0.044444in}{0.011787in}}{\pgfqpoint{-0.044444in}{0.000000in}}%
\pgfpathcurveto{\pgfqpoint{-0.044444in}{-0.011787in}}{\pgfqpoint{-0.039761in}{-0.023092in}}{\pgfqpoint{-0.031427in}{-0.031427in}}%
\pgfpathcurveto{\pgfqpoint{-0.023092in}{-0.039761in}}{\pgfqpoint{-0.011787in}{-0.044444in}}{\pgfqpoint{0.000000in}{-0.044444in}}%
\pgfpathclose%
\pgfusepath{stroke,fill}%
}%
\begin{pgfscope}%
\pgfsys@transformshift{1.953642in}{4.573410in}%
\pgfsys@useobject{currentmarker}{}%
\end{pgfscope}%
\end{pgfscope}%
\begin{pgfscope}%
\pgfpathrectangle{\pgfqpoint{0.100000in}{2.413063in}}{\pgfqpoint{5.037500in}{3.427208in}}%
\pgfusepath{clip}%
\pgfsetrectcap%
\pgfsetroundjoin%
\pgfsetlinewidth{1.505625pt}%
\definecolor{currentstroke}{rgb}{0.501961,0.501961,0.501961}%
\pgfsetstrokecolor{currentstroke}%
\pgfsetstrokeopacity{0.500000}%
\pgfsetdash{}{0pt}%
\pgfpathmoveto{\pgfqpoint{2.769875in}{5.668910in}}%
\pgfusepath{stroke}%
\end{pgfscope}%
\begin{pgfscope}%
\pgfpathrectangle{\pgfqpoint{0.100000in}{2.413063in}}{\pgfqpoint{5.037500in}{3.427208in}}%
\pgfusepath{clip}%
\pgfsetbuttcap%
\pgfsetroundjoin%
\definecolor{currentfill}{rgb}{0.501961,0.501961,0.501961}%
\pgfsetfillcolor{currentfill}%
\pgfsetfillopacity{0.500000}%
\pgfsetlinewidth{0.250937pt}%
\definecolor{currentstroke}{rgb}{0.000000,0.000000,0.000000}%
\pgfsetstrokecolor{currentstroke}%
\pgfsetstrokeopacity{0.500000}%
\pgfsetdash{}{0pt}%
\pgfsys@defobject{currentmarker}{\pgfqpoint{-0.013889in}{-0.013889in}}{\pgfqpoint{0.013889in}{0.013889in}}{%
\pgfpathmoveto{\pgfqpoint{0.000000in}{-0.013889in}}%
\pgfpathcurveto{\pgfqpoint{0.003683in}{-0.013889in}}{\pgfqpoint{0.007216in}{-0.012425in}}{\pgfqpoint{0.009821in}{-0.009821in}}%
\pgfpathcurveto{\pgfqpoint{0.012425in}{-0.007216in}}{\pgfqpoint{0.013889in}{-0.003683in}}{\pgfqpoint{0.013889in}{0.000000in}}%
\pgfpathcurveto{\pgfqpoint{0.013889in}{0.003683in}}{\pgfqpoint{0.012425in}{0.007216in}}{\pgfqpoint{0.009821in}{0.009821in}}%
\pgfpathcurveto{\pgfqpoint{0.007216in}{0.012425in}}{\pgfqpoint{0.003683in}{0.013889in}}{\pgfqpoint{0.000000in}{0.013889in}}%
\pgfpathcurveto{\pgfqpoint{-0.003683in}{0.013889in}}{\pgfqpoint{-0.007216in}{0.012425in}}{\pgfqpoint{-0.009821in}{0.009821in}}%
\pgfpathcurveto{\pgfqpoint{-0.012425in}{0.007216in}}{\pgfqpoint{-0.013889in}{0.003683in}}{\pgfqpoint{-0.013889in}{0.000000in}}%
\pgfpathcurveto{\pgfqpoint{-0.013889in}{-0.003683in}}{\pgfqpoint{-0.012425in}{-0.007216in}}{\pgfqpoint{-0.009821in}{-0.009821in}}%
\pgfpathcurveto{\pgfqpoint{-0.007216in}{-0.012425in}}{\pgfqpoint{-0.003683in}{-0.013889in}}{\pgfqpoint{0.000000in}{-0.013889in}}%
\pgfpathclose%
\pgfusepath{stroke,fill}%
}%
\begin{pgfscope}%
\pgfsys@transformshift{2.769875in}{5.668910in}%
\pgfsys@useobject{currentmarker}{}%
\end{pgfscope}%
\end{pgfscope}%
\begin{pgfscope}%
\pgfpathrectangle{\pgfqpoint{0.100000in}{2.413063in}}{\pgfqpoint{5.037500in}{3.427208in}}%
\pgfusepath{clip}%
\pgfsetrectcap%
\pgfsetroundjoin%
\pgfsetlinewidth{1.505625pt}%
\definecolor{currentstroke}{rgb}{0.000000,0.000000,1.000000}%
\pgfsetstrokecolor{currentstroke}%
\pgfsetstrokeopacity{0.500000}%
\pgfsetdash{}{0pt}%
\pgfpathmoveto{\pgfqpoint{3.550687in}{5.668910in}}%
\pgfusepath{stroke}%
\end{pgfscope}%
\begin{pgfscope}%
\pgfpathrectangle{\pgfqpoint{0.100000in}{2.413063in}}{\pgfqpoint{5.037500in}{3.427208in}}%
\pgfusepath{clip}%
\pgfsetbuttcap%
\pgfsetroundjoin%
\definecolor{currentfill}{rgb}{0.000000,0.000000,1.000000}%
\pgfsetfillcolor{currentfill}%
\pgfsetfillopacity{0.500000}%
\pgfsetlinewidth{0.250937pt}%
\definecolor{currentstroke}{rgb}{0.000000,0.000000,0.000000}%
\pgfsetstrokecolor{currentstroke}%
\pgfsetstrokeopacity{0.500000}%
\pgfsetdash{}{0pt}%
\pgfsys@defobject{currentmarker}{\pgfqpoint{-0.083333in}{-0.083333in}}{\pgfqpoint{0.083333in}{0.083333in}}{%
\pgfpathmoveto{\pgfqpoint{0.000000in}{-0.083333in}}%
\pgfpathcurveto{\pgfqpoint{0.022100in}{-0.083333in}}{\pgfqpoint{0.043298in}{-0.074553in}}{\pgfqpoint{0.058926in}{-0.058926in}}%
\pgfpathcurveto{\pgfqpoint{0.074553in}{-0.043298in}}{\pgfqpoint{0.083333in}{-0.022100in}}{\pgfqpoint{0.083333in}{0.000000in}}%
\pgfpathcurveto{\pgfqpoint{0.083333in}{0.022100in}}{\pgfqpoint{0.074553in}{0.043298in}}{\pgfqpoint{0.058926in}{0.058926in}}%
\pgfpathcurveto{\pgfqpoint{0.043298in}{0.074553in}}{\pgfqpoint{0.022100in}{0.083333in}}{\pgfqpoint{0.000000in}{0.083333in}}%
\pgfpathcurveto{\pgfqpoint{-0.022100in}{0.083333in}}{\pgfqpoint{-0.043298in}{0.074553in}}{\pgfqpoint{-0.058926in}{0.058926in}}%
\pgfpathcurveto{\pgfqpoint{-0.074553in}{0.043298in}}{\pgfqpoint{-0.083333in}{0.022100in}}{\pgfqpoint{-0.083333in}{0.000000in}}%
\pgfpathcurveto{\pgfqpoint{-0.083333in}{-0.022100in}}{\pgfqpoint{-0.074553in}{-0.043298in}}{\pgfqpoint{-0.058926in}{-0.058926in}}%
\pgfpathcurveto{\pgfqpoint{-0.043298in}{-0.074553in}}{\pgfqpoint{-0.022100in}{-0.083333in}}{\pgfqpoint{0.000000in}{-0.083333in}}%
\pgfpathclose%
\pgfusepath{stroke,fill}%
}%
\begin{pgfscope}%
\pgfsys@transformshift{3.550687in}{5.668910in}%
\pgfsys@useobject{currentmarker}{}%
\end{pgfscope}%
\end{pgfscope}%
\begin{pgfscope}%
\pgfpathrectangle{\pgfqpoint{0.100000in}{2.413063in}}{\pgfqpoint{5.037500in}{3.427208in}}%
\pgfusepath{clip}%
\pgfsetrectcap%
\pgfsetroundjoin%
\pgfsetlinewidth{1.505625pt}%
\definecolor{currentstroke}{rgb}{0.000000,0.000000,1.000000}%
\pgfsetstrokecolor{currentstroke}%
\pgfsetstrokeopacity{0.500000}%
\pgfsetdash{}{0pt}%
\pgfpathmoveto{\pgfqpoint{3.550687in}{5.497550in}}%
\pgfusepath{stroke}%
\end{pgfscope}%
\begin{pgfscope}%
\pgfpathrectangle{\pgfqpoint{0.100000in}{2.413063in}}{\pgfqpoint{5.037500in}{3.427208in}}%
\pgfusepath{clip}%
\pgfsetbuttcap%
\pgfsetroundjoin%
\definecolor{currentfill}{rgb}{0.000000,0.000000,1.000000}%
\pgfsetfillcolor{currentfill}%
\pgfsetfillopacity{0.500000}%
\pgfsetlinewidth{0.250937pt}%
\definecolor{currentstroke}{rgb}{0.000000,0.000000,0.000000}%
\pgfsetstrokecolor{currentstroke}%
\pgfsetstrokeopacity{0.500000}%
\pgfsetdash{}{0pt}%
\pgfsys@defobject{currentmarker}{\pgfqpoint{-0.055556in}{-0.055556in}}{\pgfqpoint{0.055556in}{0.055556in}}{%
\pgfpathmoveto{\pgfqpoint{0.000000in}{-0.055556in}}%
\pgfpathcurveto{\pgfqpoint{0.014734in}{-0.055556in}}{\pgfqpoint{0.028866in}{-0.049702in}}{\pgfqpoint{0.039284in}{-0.039284in}}%
\pgfpathcurveto{\pgfqpoint{0.049702in}{-0.028866in}}{\pgfqpoint{0.055556in}{-0.014734in}}{\pgfqpoint{0.055556in}{0.000000in}}%
\pgfpathcurveto{\pgfqpoint{0.055556in}{0.014734in}}{\pgfqpoint{0.049702in}{0.028866in}}{\pgfqpoint{0.039284in}{0.039284in}}%
\pgfpathcurveto{\pgfqpoint{0.028866in}{0.049702in}}{\pgfqpoint{0.014734in}{0.055556in}}{\pgfqpoint{0.000000in}{0.055556in}}%
\pgfpathcurveto{\pgfqpoint{-0.014734in}{0.055556in}}{\pgfqpoint{-0.028866in}{0.049702in}}{\pgfqpoint{-0.039284in}{0.039284in}}%
\pgfpathcurveto{\pgfqpoint{-0.049702in}{0.028866in}}{\pgfqpoint{-0.055556in}{0.014734in}}{\pgfqpoint{-0.055556in}{0.000000in}}%
\pgfpathcurveto{\pgfqpoint{-0.055556in}{-0.014734in}}{\pgfqpoint{-0.049702in}{-0.028866in}}{\pgfqpoint{-0.039284in}{-0.039284in}}%
\pgfpathcurveto{\pgfqpoint{-0.028866in}{-0.049702in}}{\pgfqpoint{-0.014734in}{-0.055556in}}{\pgfqpoint{0.000000in}{-0.055556in}}%
\pgfpathclose%
\pgfusepath{stroke,fill}%
}%
\begin{pgfscope}%
\pgfsys@transformshift{3.550687in}{5.497550in}%
\pgfsys@useobject{currentmarker}{}%
\end{pgfscope}%
\end{pgfscope}%
\begin{pgfscope}%
\pgfpathrectangle{\pgfqpoint{0.100000in}{2.413063in}}{\pgfqpoint{5.037500in}{3.427208in}}%
\pgfusepath{clip}%
\pgfsetrectcap%
\pgfsetroundjoin%
\pgfsetlinewidth{1.505625pt}%
\definecolor{currentstroke}{rgb}{0.000000,0.000000,1.000000}%
\pgfsetstrokecolor{currentstroke}%
\pgfsetstrokeopacity{0.500000}%
\pgfsetdash{}{0pt}%
\pgfpathmoveto{\pgfqpoint{3.550687in}{5.360462in}}%
\pgfusepath{stroke}%
\end{pgfscope}%
\begin{pgfscope}%
\pgfpathrectangle{\pgfqpoint{0.100000in}{2.413063in}}{\pgfqpoint{5.037500in}{3.427208in}}%
\pgfusepath{clip}%
\pgfsetbuttcap%
\pgfsetroundjoin%
\definecolor{currentfill}{rgb}{0.000000,0.000000,1.000000}%
\pgfsetfillcolor{currentfill}%
\pgfsetfillopacity{0.500000}%
\pgfsetlinewidth{0.250937pt}%
\definecolor{currentstroke}{rgb}{0.000000,0.000000,0.000000}%
\pgfsetstrokecolor{currentstroke}%
\pgfsetstrokeopacity{0.500000}%
\pgfsetdash{}{0pt}%
\pgfsys@defobject{currentmarker}{\pgfqpoint{-0.027778in}{-0.027778in}}{\pgfqpoint{0.027778in}{0.027778in}}{%
\pgfpathmoveto{\pgfqpoint{0.000000in}{-0.027778in}}%
\pgfpathcurveto{\pgfqpoint{0.007367in}{-0.027778in}}{\pgfqpoint{0.014433in}{-0.024851in}}{\pgfqpoint{0.019642in}{-0.019642in}}%
\pgfpathcurveto{\pgfqpoint{0.024851in}{-0.014433in}}{\pgfqpoint{0.027778in}{-0.007367in}}{\pgfqpoint{0.027778in}{0.000000in}}%
\pgfpathcurveto{\pgfqpoint{0.027778in}{0.007367in}}{\pgfqpoint{0.024851in}{0.014433in}}{\pgfqpoint{0.019642in}{0.019642in}}%
\pgfpathcurveto{\pgfqpoint{0.014433in}{0.024851in}}{\pgfqpoint{0.007367in}{0.027778in}}{\pgfqpoint{0.000000in}{0.027778in}}%
\pgfpathcurveto{\pgfqpoint{-0.007367in}{0.027778in}}{\pgfqpoint{-0.014433in}{0.024851in}}{\pgfqpoint{-0.019642in}{0.019642in}}%
\pgfpathcurveto{\pgfqpoint{-0.024851in}{0.014433in}}{\pgfqpoint{-0.027778in}{0.007367in}}{\pgfqpoint{-0.027778in}{0.000000in}}%
\pgfpathcurveto{\pgfqpoint{-0.027778in}{-0.007367in}}{\pgfqpoint{-0.024851in}{-0.014433in}}{\pgfqpoint{-0.019642in}{-0.019642in}}%
\pgfpathcurveto{\pgfqpoint{-0.014433in}{-0.024851in}}{\pgfqpoint{-0.007367in}{-0.027778in}}{\pgfqpoint{0.000000in}{-0.027778in}}%
\pgfpathclose%
\pgfusepath{stroke,fill}%
}%
\begin{pgfscope}%
\pgfsys@transformshift{3.550687in}{5.360462in}%
\pgfsys@useobject{currentmarker}{}%
\end{pgfscope}%
\end{pgfscope}%
\begin{pgfscope}%
\pgfpathrectangle{\pgfqpoint{0.100000in}{2.413063in}}{\pgfqpoint{5.037500in}{3.427208in}}%
\pgfusepath{clip}%
\pgfsetrectcap%
\pgfsetroundjoin%
\pgfsetlinewidth{1.505625pt}%
\definecolor{currentstroke}{rgb}{0.678431,1.000000,0.184314}%
\pgfsetstrokecolor{currentstroke}%
\pgfsetstrokeopacity{0.500000}%
\pgfsetdash{}{0pt}%
\pgfpathmoveto{\pgfqpoint{4.205562in}{5.668910in}}%
\pgfusepath{stroke}%
\end{pgfscope}%
\begin{pgfscope}%
\pgfpathrectangle{\pgfqpoint{0.100000in}{2.413063in}}{\pgfqpoint{5.037500in}{3.427208in}}%
\pgfusepath{clip}%
\pgfsetbuttcap%
\pgfsetroundjoin%
\definecolor{currentfill}{rgb}{0.678431,1.000000,0.184314}%
\pgfsetfillcolor{currentfill}%
\pgfsetfillopacity{0.500000}%
\pgfsetlinewidth{0.250937pt}%
\definecolor{currentstroke}{rgb}{0.000000,0.000000,0.000000}%
\pgfsetstrokecolor{currentstroke}%
\pgfsetstrokeopacity{0.500000}%
\pgfsetdash{}{0pt}%
\pgfsys@defobject{currentmarker}{\pgfqpoint{-0.083333in}{-0.083333in}}{\pgfqpoint{0.083333in}{0.083333in}}{%
\pgfpathmoveto{\pgfqpoint{0.000000in}{-0.083333in}}%
\pgfpathcurveto{\pgfqpoint{0.022100in}{-0.083333in}}{\pgfqpoint{0.043298in}{-0.074553in}}{\pgfqpoint{0.058926in}{-0.058926in}}%
\pgfpathcurveto{\pgfqpoint{0.074553in}{-0.043298in}}{\pgfqpoint{0.083333in}{-0.022100in}}{\pgfqpoint{0.083333in}{0.000000in}}%
\pgfpathcurveto{\pgfqpoint{0.083333in}{0.022100in}}{\pgfqpoint{0.074553in}{0.043298in}}{\pgfqpoint{0.058926in}{0.058926in}}%
\pgfpathcurveto{\pgfqpoint{0.043298in}{0.074553in}}{\pgfqpoint{0.022100in}{0.083333in}}{\pgfqpoint{0.000000in}{0.083333in}}%
\pgfpathcurveto{\pgfqpoint{-0.022100in}{0.083333in}}{\pgfqpoint{-0.043298in}{0.074553in}}{\pgfqpoint{-0.058926in}{0.058926in}}%
\pgfpathcurveto{\pgfqpoint{-0.074553in}{0.043298in}}{\pgfqpoint{-0.083333in}{0.022100in}}{\pgfqpoint{-0.083333in}{0.000000in}}%
\pgfpathcurveto{\pgfqpoint{-0.083333in}{-0.022100in}}{\pgfqpoint{-0.074553in}{-0.043298in}}{\pgfqpoint{-0.058926in}{-0.058926in}}%
\pgfpathcurveto{\pgfqpoint{-0.043298in}{-0.074553in}}{\pgfqpoint{-0.022100in}{-0.083333in}}{\pgfqpoint{0.000000in}{-0.083333in}}%
\pgfpathclose%
\pgfusepath{stroke,fill}%
}%
\begin{pgfscope}%
\pgfsys@transformshift{4.205562in}{5.668910in}%
\pgfsys@useobject{currentmarker}{}%
\end{pgfscope}%
\end{pgfscope}%
\begin{pgfscope}%
\pgfpathrectangle{\pgfqpoint{0.100000in}{2.413063in}}{\pgfqpoint{5.037500in}{3.427208in}}%
\pgfusepath{clip}%
\pgfsetrectcap%
\pgfsetroundjoin%
\pgfsetlinewidth{1.505625pt}%
\definecolor{currentstroke}{rgb}{0.678431,1.000000,0.184314}%
\pgfsetstrokecolor{currentstroke}%
\pgfsetstrokeopacity{0.500000}%
\pgfsetdash{}{0pt}%
\pgfpathmoveto{\pgfqpoint{4.205562in}{5.497550in}}%
\pgfusepath{stroke}%
\end{pgfscope}%
\begin{pgfscope}%
\pgfpathrectangle{\pgfqpoint{0.100000in}{2.413063in}}{\pgfqpoint{5.037500in}{3.427208in}}%
\pgfusepath{clip}%
\pgfsetbuttcap%
\pgfsetroundjoin%
\definecolor{currentfill}{rgb}{0.678431,1.000000,0.184314}%
\pgfsetfillcolor{currentfill}%
\pgfsetfillopacity{0.500000}%
\pgfsetlinewidth{0.250937pt}%
\definecolor{currentstroke}{rgb}{0.000000,0.000000,0.000000}%
\pgfsetstrokecolor{currentstroke}%
\pgfsetstrokeopacity{0.500000}%
\pgfsetdash{}{0pt}%
\pgfsys@defobject{currentmarker}{\pgfqpoint{-0.055556in}{-0.055556in}}{\pgfqpoint{0.055556in}{0.055556in}}{%
\pgfpathmoveto{\pgfqpoint{0.000000in}{-0.055556in}}%
\pgfpathcurveto{\pgfqpoint{0.014734in}{-0.055556in}}{\pgfqpoint{0.028866in}{-0.049702in}}{\pgfqpoint{0.039284in}{-0.039284in}}%
\pgfpathcurveto{\pgfqpoint{0.049702in}{-0.028866in}}{\pgfqpoint{0.055556in}{-0.014734in}}{\pgfqpoint{0.055556in}{0.000000in}}%
\pgfpathcurveto{\pgfqpoint{0.055556in}{0.014734in}}{\pgfqpoint{0.049702in}{0.028866in}}{\pgfqpoint{0.039284in}{0.039284in}}%
\pgfpathcurveto{\pgfqpoint{0.028866in}{0.049702in}}{\pgfqpoint{0.014734in}{0.055556in}}{\pgfqpoint{0.000000in}{0.055556in}}%
\pgfpathcurveto{\pgfqpoint{-0.014734in}{0.055556in}}{\pgfqpoint{-0.028866in}{0.049702in}}{\pgfqpoint{-0.039284in}{0.039284in}}%
\pgfpathcurveto{\pgfqpoint{-0.049702in}{0.028866in}}{\pgfqpoint{-0.055556in}{0.014734in}}{\pgfqpoint{-0.055556in}{0.000000in}}%
\pgfpathcurveto{\pgfqpoint{-0.055556in}{-0.014734in}}{\pgfqpoint{-0.049702in}{-0.028866in}}{\pgfqpoint{-0.039284in}{-0.039284in}}%
\pgfpathcurveto{\pgfqpoint{-0.028866in}{-0.049702in}}{\pgfqpoint{-0.014734in}{-0.055556in}}{\pgfqpoint{0.000000in}{-0.055556in}}%
\pgfpathclose%
\pgfusepath{stroke,fill}%
}%
\begin{pgfscope}%
\pgfsys@transformshift{4.205562in}{5.497550in}%
\pgfsys@useobject{currentmarker}{}%
\end{pgfscope}%
\end{pgfscope}%
\begin{pgfscope}%
\pgfpathrectangle{\pgfqpoint{0.100000in}{2.413063in}}{\pgfqpoint{5.037500in}{3.427208in}}%
\pgfusepath{clip}%
\pgfsetrectcap%
\pgfsetroundjoin%
\pgfsetlinewidth{1.505625pt}%
\definecolor{currentstroke}{rgb}{0.678431,1.000000,0.184314}%
\pgfsetstrokecolor{currentstroke}%
\pgfsetstrokeopacity{0.500000}%
\pgfsetdash{}{0pt}%
\pgfpathmoveto{\pgfqpoint{4.205562in}{5.360462in}}%
\pgfusepath{stroke}%
\end{pgfscope}%
\begin{pgfscope}%
\pgfpathrectangle{\pgfqpoint{0.100000in}{2.413063in}}{\pgfqpoint{5.037500in}{3.427208in}}%
\pgfusepath{clip}%
\pgfsetbuttcap%
\pgfsetroundjoin%
\definecolor{currentfill}{rgb}{0.678431,1.000000,0.184314}%
\pgfsetfillcolor{currentfill}%
\pgfsetfillopacity{0.500000}%
\pgfsetlinewidth{0.250937pt}%
\definecolor{currentstroke}{rgb}{0.000000,0.000000,0.000000}%
\pgfsetstrokecolor{currentstroke}%
\pgfsetstrokeopacity{0.500000}%
\pgfsetdash{}{0pt}%
\pgfsys@defobject{currentmarker}{\pgfqpoint{-0.027778in}{-0.027778in}}{\pgfqpoint{0.027778in}{0.027778in}}{%
\pgfpathmoveto{\pgfqpoint{0.000000in}{-0.027778in}}%
\pgfpathcurveto{\pgfqpoint{0.007367in}{-0.027778in}}{\pgfqpoint{0.014433in}{-0.024851in}}{\pgfqpoint{0.019642in}{-0.019642in}}%
\pgfpathcurveto{\pgfqpoint{0.024851in}{-0.014433in}}{\pgfqpoint{0.027778in}{-0.007367in}}{\pgfqpoint{0.027778in}{0.000000in}}%
\pgfpathcurveto{\pgfqpoint{0.027778in}{0.007367in}}{\pgfqpoint{0.024851in}{0.014433in}}{\pgfqpoint{0.019642in}{0.019642in}}%
\pgfpathcurveto{\pgfqpoint{0.014433in}{0.024851in}}{\pgfqpoint{0.007367in}{0.027778in}}{\pgfqpoint{0.000000in}{0.027778in}}%
\pgfpathcurveto{\pgfqpoint{-0.007367in}{0.027778in}}{\pgfqpoint{-0.014433in}{0.024851in}}{\pgfqpoint{-0.019642in}{0.019642in}}%
\pgfpathcurveto{\pgfqpoint{-0.024851in}{0.014433in}}{\pgfqpoint{-0.027778in}{0.007367in}}{\pgfqpoint{-0.027778in}{0.000000in}}%
\pgfpathcurveto{\pgfqpoint{-0.027778in}{-0.007367in}}{\pgfqpoint{-0.024851in}{-0.014433in}}{\pgfqpoint{-0.019642in}{-0.019642in}}%
\pgfpathcurveto{\pgfqpoint{-0.014433in}{-0.024851in}}{\pgfqpoint{-0.007367in}{-0.027778in}}{\pgfqpoint{0.000000in}{-0.027778in}}%
\pgfpathclose%
\pgfusepath{stroke,fill}%
}%
\begin{pgfscope}%
\pgfsys@transformshift{4.205562in}{5.360462in}%
\pgfsys@useobject{currentmarker}{}%
\end{pgfscope}%
\end{pgfscope}%
\begin{pgfscope}%
\definecolor{textcolor}{rgb}{0.000000,0.000000,0.000000}%
\pgfsetstrokecolor{textcolor}%
\pgfsetfillcolor{textcolor}%
\pgftext[x=2.895812in,y=5.636352in,left,base]{\color{textcolor}\setmainfont{Lato}\rmfamily\fontsize{7.000000}{8.400000}\selectfont +/- 0.2pp}%
\end{pgfscope}%
\begin{pgfscope}%
\definecolor{textcolor}{rgb}{0.000000,0.000000,0.000000}%
\pgfsetstrokecolor{textcolor}%
\pgfsetfillcolor{textcolor}%
\pgftext[x=3.676625in,y=5.636352in,left,base]{\color{textcolor}\setmainfont{Lato}\rmfamily\fontsize{7.000000}{8.400000}\selectfont +3.0pp}%
\end{pgfscope}%
\begin{pgfscope}%
\definecolor{textcolor}{rgb}{0.000000,0.000000,0.000000}%
\pgfsetstrokecolor{textcolor}%
\pgfsetfillcolor{textcolor}%
\pgftext[x=3.676625in,y=5.483670in,left,base]{\color{textcolor}\setmainfont{Lato}\rmfamily\fontsize{7.000000}{8.400000}\selectfont +2.0pp}%
\end{pgfscope}%
\begin{pgfscope}%
\definecolor{textcolor}{rgb}{0.000000,0.000000,0.000000}%
\pgfsetstrokecolor{textcolor}%
\pgfsetfillcolor{textcolor}%
\pgftext[x=3.676625in,y=5.330988in,left,base]{\color{textcolor}\setmainfont{Lato}\rmfamily\fontsize{7.000000}{8.400000}\selectfont +1.0pp}%
\end{pgfscope}%
\begin{pgfscope}%
\definecolor{textcolor}{rgb}{0.000000,0.000000,0.000000}%
\pgfsetstrokecolor{textcolor}%
\pgfsetfillcolor{textcolor}%
\pgftext[x=4.331500in,y=5.636352in,left,base]{\color{textcolor}\setmainfont{Lato}\rmfamily\fontsize{7.000000}{8.400000}\selectfont -3.0pp}%
\end{pgfscope}%
\begin{pgfscope}%
\definecolor{textcolor}{rgb}{0.000000,0.000000,0.000000}%
\pgfsetstrokecolor{textcolor}%
\pgfsetfillcolor{textcolor}%
\pgftext[x=4.331500in,y=5.483670in,left,base]{\color{textcolor}\setmainfont{Lato}\rmfamily\fontsize{7.000000}{8.400000}\selectfont -2.0pp}%
\end{pgfscope}%
\begin{pgfscope}%
\definecolor{textcolor}{rgb}{0.000000,0.000000,0.000000}%
\pgfsetstrokecolor{textcolor}%
\pgfsetfillcolor{textcolor}%
\pgftext[x=4.331500in,y=5.330988in,left,base]{\color{textcolor}\setmainfont{Lato}\rmfamily\fontsize{7.000000}{8.400000}\selectfont -1.0pp}%
\end{pgfscope}%
\begin{pgfscope}%
\pgfpathrectangle{\pgfqpoint{0.100000in}{0.100000in}}{\pgfqpoint{5.037500in}{2.013333in}}%
\pgfusepath{clip}%
\pgfsetrectcap%
\pgfsetroundjoin%
\pgfsetlinewidth{1.505625pt}%
\definecolor{currentstroke}{rgb}{0.000000,0.000000,1.000000}%
\pgfsetstrokecolor{currentstroke}%
\pgfsetstrokeopacity{0.500000}%
\pgfsetdash{}{0pt}%
\pgfpathmoveto{\pgfqpoint{0.780062in}{2.012667in}}%
\pgfusepath{stroke}%
\end{pgfscope}%
\begin{pgfscope}%
\pgfpathrectangle{\pgfqpoint{0.100000in}{0.100000in}}{\pgfqpoint{5.037500in}{2.013333in}}%
\pgfusepath{clip}%
\pgfsetbuttcap%
\pgfsetroundjoin%
\definecolor{currentfill}{rgb}{0.000000,0.000000,1.000000}%
\pgfsetfillcolor{currentfill}%
\pgfsetfillopacity{0.500000}%
\pgfsetlinewidth{0.250937pt}%
\definecolor{currentstroke}{rgb}{0.000000,0.000000,0.000000}%
\pgfsetstrokecolor{currentstroke}%
\pgfsetstrokeopacity{0.500000}%
\pgfsetdash{}{0pt}%
\pgfsys@defobject{currentmarker}{\pgfqpoint{-0.066667in}{-0.066667in}}{\pgfqpoint{0.066667in}{0.066667in}}{%
\pgfpathmoveto{\pgfqpoint{0.000000in}{-0.066667in}}%
\pgfpathcurveto{\pgfqpoint{0.017680in}{-0.066667in}}{\pgfqpoint{0.034639in}{-0.059642in}}{\pgfqpoint{0.047140in}{-0.047140in}}%
\pgfpathcurveto{\pgfqpoint{0.059642in}{-0.034639in}}{\pgfqpoint{0.066667in}{-0.017680in}}{\pgfqpoint{0.066667in}{0.000000in}}%
\pgfpathcurveto{\pgfqpoint{0.066667in}{0.017680in}}{\pgfqpoint{0.059642in}{0.034639in}}{\pgfqpoint{0.047140in}{0.047140in}}%
\pgfpathcurveto{\pgfqpoint{0.034639in}{0.059642in}}{\pgfqpoint{0.017680in}{0.066667in}}{\pgfqpoint{0.000000in}{0.066667in}}%
\pgfpathcurveto{\pgfqpoint{-0.017680in}{0.066667in}}{\pgfqpoint{-0.034639in}{0.059642in}}{\pgfqpoint{-0.047140in}{0.047140in}}%
\pgfpathcurveto{\pgfqpoint{-0.059642in}{0.034639in}}{\pgfqpoint{-0.066667in}{0.017680in}}{\pgfqpoint{-0.066667in}{0.000000in}}%
\pgfpathcurveto{\pgfqpoint{-0.066667in}{-0.017680in}}{\pgfqpoint{-0.059642in}{-0.034639in}}{\pgfqpoint{-0.047140in}{-0.047140in}}%
\pgfpathcurveto{\pgfqpoint{-0.034639in}{-0.059642in}}{\pgfqpoint{-0.017680in}{-0.066667in}}{\pgfqpoint{0.000000in}{-0.066667in}}%
\pgfpathclose%
\pgfusepath{stroke,fill}%
}%
\begin{pgfscope}%
\pgfsys@transformshift{0.780062in}{2.012667in}%
\pgfsys@useobject{currentmarker}{}%
\end{pgfscope}%
\end{pgfscope}%
\begin{pgfscope}%
\pgfpathrectangle{\pgfqpoint{0.100000in}{0.100000in}}{\pgfqpoint{5.037500in}{2.013333in}}%
\pgfusepath{clip}%
\pgfsetrectcap%
\pgfsetroundjoin%
\pgfsetlinewidth{0.250937pt}%
\definecolor{currentstroke}{rgb}{0.862745,0.862745,0.862745}%
\pgfsetstrokecolor{currentstroke}%
\pgfsetdash{}{0pt}%
\pgfpathmoveto{\pgfqpoint{0.906000in}{1.932133in}}%
\pgfpathlineto{\pgfqpoint{4.180375in}{1.932133in}}%
\pgfusepath{stroke}%
\end{pgfscope}%
\begin{pgfscope}%
\pgfpathrectangle{\pgfqpoint{0.100000in}{0.100000in}}{\pgfqpoint{5.037500in}{2.013333in}}%
\pgfusepath{clip}%
\pgfsetrectcap%
\pgfsetroundjoin%
\pgfsetlinewidth{1.505625pt}%
\definecolor{currentstroke}{rgb}{0.000000,0.000000,1.000000}%
\pgfsetstrokecolor{currentstroke}%
\pgfsetstrokeopacity{0.500000}%
\pgfsetdash{}{0pt}%
\pgfpathmoveto{\pgfqpoint{0.780062in}{1.829636in}}%
\pgfusepath{stroke}%
\end{pgfscope}%
\begin{pgfscope}%
\pgfpathrectangle{\pgfqpoint{0.100000in}{0.100000in}}{\pgfqpoint{5.037500in}{2.013333in}}%
\pgfusepath{clip}%
\pgfsetbuttcap%
\pgfsetroundjoin%
\definecolor{currentfill}{rgb}{0.000000,0.000000,1.000000}%
\pgfsetfillcolor{currentfill}%
\pgfsetfillopacity{0.500000}%
\pgfsetlinewidth{0.250937pt}%
\definecolor{currentstroke}{rgb}{0.000000,0.000000,0.000000}%
\pgfsetstrokecolor{currentstroke}%
\pgfsetstrokeopacity{0.500000}%
\pgfsetdash{}{0pt}%
\pgfsys@defobject{currentmarker}{\pgfqpoint{-0.047222in}{-0.047222in}}{\pgfqpoint{0.047222in}{0.047222in}}{%
\pgfpathmoveto{\pgfqpoint{0.000000in}{-0.047222in}}%
\pgfpathcurveto{\pgfqpoint{0.012523in}{-0.047222in}}{\pgfqpoint{0.024536in}{-0.042247in}}{\pgfqpoint{0.033391in}{-0.033391in}}%
\pgfpathcurveto{\pgfqpoint{0.042247in}{-0.024536in}}{\pgfqpoint{0.047222in}{-0.012523in}}{\pgfqpoint{0.047222in}{0.000000in}}%
\pgfpathcurveto{\pgfqpoint{0.047222in}{0.012523in}}{\pgfqpoint{0.042247in}{0.024536in}}{\pgfqpoint{0.033391in}{0.033391in}}%
\pgfpathcurveto{\pgfqpoint{0.024536in}{0.042247in}}{\pgfqpoint{0.012523in}{0.047222in}}{\pgfqpoint{0.000000in}{0.047222in}}%
\pgfpathcurveto{\pgfqpoint{-0.012523in}{0.047222in}}{\pgfqpoint{-0.024536in}{0.042247in}}{\pgfqpoint{-0.033391in}{0.033391in}}%
\pgfpathcurveto{\pgfqpoint{-0.042247in}{0.024536in}}{\pgfqpoint{-0.047222in}{0.012523in}}{\pgfqpoint{-0.047222in}{0.000000in}}%
\pgfpathcurveto{\pgfqpoint{-0.047222in}{-0.012523in}}{\pgfqpoint{-0.042247in}{-0.024536in}}{\pgfqpoint{-0.033391in}{-0.033391in}}%
\pgfpathcurveto{\pgfqpoint{-0.024536in}{-0.042247in}}{\pgfqpoint{-0.012523in}{-0.047222in}}{\pgfqpoint{0.000000in}{-0.047222in}}%
\pgfpathclose%
\pgfusepath{stroke,fill}%
}%
\begin{pgfscope}%
\pgfsys@transformshift{0.780062in}{1.829636in}%
\pgfsys@useobject{currentmarker}{}%
\end{pgfscope}%
\end{pgfscope}%
\begin{pgfscope}%
\pgfpathrectangle{\pgfqpoint{0.100000in}{0.100000in}}{\pgfqpoint{5.037500in}{2.013333in}}%
\pgfusepath{clip}%
\pgfsetrectcap%
\pgfsetroundjoin%
\pgfsetlinewidth{0.250937pt}%
\definecolor{currentstroke}{rgb}{0.862745,0.862745,0.862745}%
\pgfsetstrokecolor{currentstroke}%
\pgfsetdash{}{0pt}%
\pgfpathmoveto{\pgfqpoint{0.906000in}{1.749103in}}%
\pgfpathlineto{\pgfqpoint{4.180375in}{1.749103in}}%
\pgfusepath{stroke}%
\end{pgfscope}%
\begin{pgfscope}%
\pgfpathrectangle{\pgfqpoint{0.100000in}{0.100000in}}{\pgfqpoint{5.037500in}{2.013333in}}%
\pgfusepath{clip}%
\pgfsetrectcap%
\pgfsetroundjoin%
\pgfsetlinewidth{1.505625pt}%
\definecolor{currentstroke}{rgb}{0.000000,0.000000,1.000000}%
\pgfsetstrokecolor{currentstroke}%
\pgfsetstrokeopacity{0.500000}%
\pgfsetdash{}{0pt}%
\pgfpathmoveto{\pgfqpoint{0.780062in}{1.646606in}}%
\pgfusepath{stroke}%
\end{pgfscope}%
\begin{pgfscope}%
\pgfpathrectangle{\pgfqpoint{0.100000in}{0.100000in}}{\pgfqpoint{5.037500in}{2.013333in}}%
\pgfusepath{clip}%
\pgfsetbuttcap%
\pgfsetroundjoin%
\definecolor{currentfill}{rgb}{0.000000,0.000000,1.000000}%
\pgfsetfillcolor{currentfill}%
\pgfsetfillopacity{0.500000}%
\pgfsetlinewidth{0.250937pt}%
\definecolor{currentstroke}{rgb}{0.000000,0.000000,0.000000}%
\pgfsetstrokecolor{currentstroke}%
\pgfsetstrokeopacity{0.500000}%
\pgfsetdash{}{0pt}%
\pgfsys@defobject{currentmarker}{\pgfqpoint{-0.030556in}{-0.030556in}}{\pgfqpoint{0.030556in}{0.030556in}}{%
\pgfpathmoveto{\pgfqpoint{0.000000in}{-0.030556in}}%
\pgfpathcurveto{\pgfqpoint{0.008103in}{-0.030556in}}{\pgfqpoint{0.015876in}{-0.027336in}}{\pgfqpoint{0.021606in}{-0.021606in}}%
\pgfpathcurveto{\pgfqpoint{0.027336in}{-0.015876in}}{\pgfqpoint{0.030556in}{-0.008103in}}{\pgfqpoint{0.030556in}{0.000000in}}%
\pgfpathcurveto{\pgfqpoint{0.030556in}{0.008103in}}{\pgfqpoint{0.027336in}{0.015876in}}{\pgfqpoint{0.021606in}{0.021606in}}%
\pgfpathcurveto{\pgfqpoint{0.015876in}{0.027336in}}{\pgfqpoint{0.008103in}{0.030556in}}{\pgfqpoint{0.000000in}{0.030556in}}%
\pgfpathcurveto{\pgfqpoint{-0.008103in}{0.030556in}}{\pgfqpoint{-0.015876in}{0.027336in}}{\pgfqpoint{-0.021606in}{0.021606in}}%
\pgfpathcurveto{\pgfqpoint{-0.027336in}{0.015876in}}{\pgfqpoint{-0.030556in}{0.008103in}}{\pgfqpoint{-0.030556in}{0.000000in}}%
\pgfpathcurveto{\pgfqpoint{-0.030556in}{-0.008103in}}{\pgfqpoint{-0.027336in}{-0.015876in}}{\pgfqpoint{-0.021606in}{-0.021606in}}%
\pgfpathcurveto{\pgfqpoint{-0.015876in}{-0.027336in}}{\pgfqpoint{-0.008103in}{-0.030556in}}{\pgfqpoint{0.000000in}{-0.030556in}}%
\pgfpathclose%
\pgfusepath{stroke,fill}%
}%
\begin{pgfscope}%
\pgfsys@transformshift{0.780062in}{1.646606in}%
\pgfsys@useobject{currentmarker}{}%
\end{pgfscope}%
\end{pgfscope}%
\begin{pgfscope}%
\pgfpathrectangle{\pgfqpoint{0.100000in}{0.100000in}}{\pgfqpoint{5.037500in}{2.013333in}}%
\pgfusepath{clip}%
\pgfsetrectcap%
\pgfsetroundjoin%
\pgfsetlinewidth{0.250937pt}%
\definecolor{currentstroke}{rgb}{0.862745,0.862745,0.862745}%
\pgfsetstrokecolor{currentstroke}%
\pgfsetdash{}{0pt}%
\pgfpathmoveto{\pgfqpoint{0.906000in}{1.566073in}}%
\pgfpathlineto{\pgfqpoint{4.180375in}{1.566073in}}%
\pgfusepath{stroke}%
\end{pgfscope}%
\begin{pgfscope}%
\pgfpathrectangle{\pgfqpoint{0.100000in}{0.100000in}}{\pgfqpoint{5.037500in}{2.013333in}}%
\pgfusepath{clip}%
\pgfsetrectcap%
\pgfsetroundjoin%
\pgfsetlinewidth{1.505625pt}%
\definecolor{currentstroke}{rgb}{0.000000,0.000000,1.000000}%
\pgfsetstrokecolor{currentstroke}%
\pgfsetstrokeopacity{0.500000}%
\pgfsetdash{}{0pt}%
\pgfpathmoveto{\pgfqpoint{0.780062in}{1.463576in}}%
\pgfusepath{stroke}%
\end{pgfscope}%
\begin{pgfscope}%
\pgfpathrectangle{\pgfqpoint{0.100000in}{0.100000in}}{\pgfqpoint{5.037500in}{2.013333in}}%
\pgfusepath{clip}%
\pgfsetbuttcap%
\pgfsetroundjoin%
\definecolor{currentfill}{rgb}{0.000000,0.000000,1.000000}%
\pgfsetfillcolor{currentfill}%
\pgfsetfillopacity{0.500000}%
\pgfsetlinewidth{0.250937pt}%
\definecolor{currentstroke}{rgb}{0.000000,0.000000,0.000000}%
\pgfsetstrokecolor{currentstroke}%
\pgfsetstrokeopacity{0.500000}%
\pgfsetdash{}{0pt}%
\pgfsys@defobject{currentmarker}{\pgfqpoint{-0.019444in}{-0.019444in}}{\pgfqpoint{0.019444in}{0.019444in}}{%
\pgfpathmoveto{\pgfqpoint{0.000000in}{-0.019444in}}%
\pgfpathcurveto{\pgfqpoint{0.005157in}{-0.019444in}}{\pgfqpoint{0.010103in}{-0.017396in}}{\pgfqpoint{0.013749in}{-0.013749in}}%
\pgfpathcurveto{\pgfqpoint{0.017396in}{-0.010103in}}{\pgfqpoint{0.019444in}{-0.005157in}}{\pgfqpoint{0.019444in}{0.000000in}}%
\pgfpathcurveto{\pgfqpoint{0.019444in}{0.005157in}}{\pgfqpoint{0.017396in}{0.010103in}}{\pgfqpoint{0.013749in}{0.013749in}}%
\pgfpathcurveto{\pgfqpoint{0.010103in}{0.017396in}}{\pgfqpoint{0.005157in}{0.019444in}}{\pgfqpoint{0.000000in}{0.019444in}}%
\pgfpathcurveto{\pgfqpoint{-0.005157in}{0.019444in}}{\pgfqpoint{-0.010103in}{0.017396in}}{\pgfqpoint{-0.013749in}{0.013749in}}%
\pgfpathcurveto{\pgfqpoint{-0.017396in}{0.010103in}}{\pgfqpoint{-0.019444in}{0.005157in}}{\pgfqpoint{-0.019444in}{0.000000in}}%
\pgfpathcurveto{\pgfqpoint{-0.019444in}{-0.005157in}}{\pgfqpoint{-0.017396in}{-0.010103in}}{\pgfqpoint{-0.013749in}{-0.013749in}}%
\pgfpathcurveto{\pgfqpoint{-0.010103in}{-0.017396in}}{\pgfqpoint{-0.005157in}{-0.019444in}}{\pgfqpoint{0.000000in}{-0.019444in}}%
\pgfpathclose%
\pgfusepath{stroke,fill}%
}%
\begin{pgfscope}%
\pgfsys@transformshift{0.780062in}{1.463576in}%
\pgfsys@useobject{currentmarker}{}%
\end{pgfscope}%
\end{pgfscope}%
\begin{pgfscope}%
\pgfpathrectangle{\pgfqpoint{0.100000in}{0.100000in}}{\pgfqpoint{5.037500in}{2.013333in}}%
\pgfusepath{clip}%
\pgfsetrectcap%
\pgfsetroundjoin%
\pgfsetlinewidth{0.250937pt}%
\definecolor{currentstroke}{rgb}{0.862745,0.862745,0.862745}%
\pgfsetstrokecolor{currentstroke}%
\pgfsetdash{}{0pt}%
\pgfpathmoveto{\pgfqpoint{0.906000in}{1.383042in}}%
\pgfpathlineto{\pgfqpoint{4.180375in}{1.383042in}}%
\pgfusepath{stroke}%
\end{pgfscope}%
\begin{pgfscope}%
\pgfpathrectangle{\pgfqpoint{0.100000in}{0.100000in}}{\pgfqpoint{5.037500in}{2.013333in}}%
\pgfusepath{clip}%
\pgfsetrectcap%
\pgfsetroundjoin%
\pgfsetlinewidth{1.505625pt}%
\definecolor{currentstroke}{rgb}{0.000000,0.000000,1.000000}%
\pgfsetstrokecolor{currentstroke}%
\pgfsetstrokeopacity{0.500000}%
\pgfsetdash{}{0pt}%
\pgfpathmoveto{\pgfqpoint{0.780062in}{1.280545in}}%
\pgfusepath{stroke}%
\end{pgfscope}%
\begin{pgfscope}%
\pgfpathrectangle{\pgfqpoint{0.100000in}{0.100000in}}{\pgfqpoint{5.037500in}{2.013333in}}%
\pgfusepath{clip}%
\pgfsetbuttcap%
\pgfsetroundjoin%
\definecolor{currentfill}{rgb}{0.000000,0.000000,1.000000}%
\pgfsetfillcolor{currentfill}%
\pgfsetfillopacity{0.500000}%
\pgfsetlinewidth{0.250937pt}%
\definecolor{currentstroke}{rgb}{0.000000,0.000000,0.000000}%
\pgfsetstrokecolor{currentstroke}%
\pgfsetstrokeopacity{0.500000}%
\pgfsetdash{}{0pt}%
\pgfsys@defobject{currentmarker}{\pgfqpoint{-0.033333in}{-0.033333in}}{\pgfqpoint{0.033333in}{0.033333in}}{%
\pgfpathmoveto{\pgfqpoint{0.000000in}{-0.033333in}}%
\pgfpathcurveto{\pgfqpoint{0.008840in}{-0.033333in}}{\pgfqpoint{0.017319in}{-0.029821in}}{\pgfqpoint{0.023570in}{-0.023570in}}%
\pgfpathcurveto{\pgfqpoint{0.029821in}{-0.017319in}}{\pgfqpoint{0.033333in}{-0.008840in}}{\pgfqpoint{0.033333in}{0.000000in}}%
\pgfpathcurveto{\pgfqpoint{0.033333in}{0.008840in}}{\pgfqpoint{0.029821in}{0.017319in}}{\pgfqpoint{0.023570in}{0.023570in}}%
\pgfpathcurveto{\pgfqpoint{0.017319in}{0.029821in}}{\pgfqpoint{0.008840in}{0.033333in}}{\pgfqpoint{0.000000in}{0.033333in}}%
\pgfpathcurveto{\pgfqpoint{-0.008840in}{0.033333in}}{\pgfqpoint{-0.017319in}{0.029821in}}{\pgfqpoint{-0.023570in}{0.023570in}}%
\pgfpathcurveto{\pgfqpoint{-0.029821in}{0.017319in}}{\pgfqpoint{-0.033333in}{0.008840in}}{\pgfqpoint{-0.033333in}{0.000000in}}%
\pgfpathcurveto{\pgfqpoint{-0.033333in}{-0.008840in}}{\pgfqpoint{-0.029821in}{-0.017319in}}{\pgfqpoint{-0.023570in}{-0.023570in}}%
\pgfpathcurveto{\pgfqpoint{-0.017319in}{-0.029821in}}{\pgfqpoint{-0.008840in}{-0.033333in}}{\pgfqpoint{0.000000in}{-0.033333in}}%
\pgfpathclose%
\pgfusepath{stroke,fill}%
}%
\begin{pgfscope}%
\pgfsys@transformshift{0.780062in}{1.280545in}%
\pgfsys@useobject{currentmarker}{}%
\end{pgfscope}%
\end{pgfscope}%
\begin{pgfscope}%
\pgfpathrectangle{\pgfqpoint{0.100000in}{0.100000in}}{\pgfqpoint{5.037500in}{2.013333in}}%
\pgfusepath{clip}%
\pgfsetrectcap%
\pgfsetroundjoin%
\pgfsetlinewidth{0.250937pt}%
\definecolor{currentstroke}{rgb}{0.862745,0.862745,0.862745}%
\pgfsetstrokecolor{currentstroke}%
\pgfsetdash{}{0pt}%
\pgfpathmoveto{\pgfqpoint{0.906000in}{1.200012in}}%
\pgfpathlineto{\pgfqpoint{4.180375in}{1.200012in}}%
\pgfusepath{stroke}%
\end{pgfscope}%
\begin{pgfscope}%
\pgfpathrectangle{\pgfqpoint{0.100000in}{0.100000in}}{\pgfqpoint{5.037500in}{2.013333in}}%
\pgfusepath{clip}%
\pgfsetrectcap%
\pgfsetroundjoin%
\pgfsetlinewidth{1.505625pt}%
\definecolor{currentstroke}{rgb}{0.000000,0.000000,1.000000}%
\pgfsetstrokecolor{currentstroke}%
\pgfsetstrokeopacity{0.500000}%
\pgfsetdash{}{0pt}%
\pgfpathmoveto{\pgfqpoint{0.780062in}{1.097515in}}%
\pgfusepath{stroke}%
\end{pgfscope}%
\begin{pgfscope}%
\pgfpathrectangle{\pgfqpoint{0.100000in}{0.100000in}}{\pgfqpoint{5.037500in}{2.013333in}}%
\pgfusepath{clip}%
\pgfsetbuttcap%
\pgfsetroundjoin%
\definecolor{currentfill}{rgb}{0.000000,0.000000,1.000000}%
\pgfsetfillcolor{currentfill}%
\pgfsetfillopacity{0.500000}%
\pgfsetlinewidth{0.250937pt}%
\definecolor{currentstroke}{rgb}{0.000000,0.000000,0.000000}%
\pgfsetstrokecolor{currentstroke}%
\pgfsetstrokeopacity{0.500000}%
\pgfsetdash{}{0pt}%
\pgfsys@defobject{currentmarker}{\pgfqpoint{-0.022222in}{-0.022222in}}{\pgfqpoint{0.022222in}{0.022222in}}{%
\pgfpathmoveto{\pgfqpoint{0.000000in}{-0.022222in}}%
\pgfpathcurveto{\pgfqpoint{0.005893in}{-0.022222in}}{\pgfqpoint{0.011546in}{-0.019881in}}{\pgfqpoint{0.015713in}{-0.015713in}}%
\pgfpathcurveto{\pgfqpoint{0.019881in}{-0.011546in}}{\pgfqpoint{0.022222in}{-0.005893in}}{\pgfqpoint{0.022222in}{0.000000in}}%
\pgfpathcurveto{\pgfqpoint{0.022222in}{0.005893in}}{\pgfqpoint{0.019881in}{0.011546in}}{\pgfqpoint{0.015713in}{0.015713in}}%
\pgfpathcurveto{\pgfqpoint{0.011546in}{0.019881in}}{\pgfqpoint{0.005893in}{0.022222in}}{\pgfqpoint{0.000000in}{0.022222in}}%
\pgfpathcurveto{\pgfqpoint{-0.005893in}{0.022222in}}{\pgfqpoint{-0.011546in}{0.019881in}}{\pgfqpoint{-0.015713in}{0.015713in}}%
\pgfpathcurveto{\pgfqpoint{-0.019881in}{0.011546in}}{\pgfqpoint{-0.022222in}{0.005893in}}{\pgfqpoint{-0.022222in}{0.000000in}}%
\pgfpathcurveto{\pgfqpoint{-0.022222in}{-0.005893in}}{\pgfqpoint{-0.019881in}{-0.011546in}}{\pgfqpoint{-0.015713in}{-0.015713in}}%
\pgfpathcurveto{\pgfqpoint{-0.011546in}{-0.019881in}}{\pgfqpoint{-0.005893in}{-0.022222in}}{\pgfqpoint{0.000000in}{-0.022222in}}%
\pgfpathclose%
\pgfusepath{stroke,fill}%
}%
\begin{pgfscope}%
\pgfsys@transformshift{0.780062in}{1.097515in}%
\pgfsys@useobject{currentmarker}{}%
\end{pgfscope}%
\end{pgfscope}%
\begin{pgfscope}%
\pgfpathrectangle{\pgfqpoint{0.100000in}{0.100000in}}{\pgfqpoint{5.037500in}{2.013333in}}%
\pgfusepath{clip}%
\pgfsetrectcap%
\pgfsetroundjoin%
\pgfsetlinewidth{0.250937pt}%
\definecolor{currentstroke}{rgb}{0.862745,0.862745,0.862745}%
\pgfsetstrokecolor{currentstroke}%
\pgfsetdash{}{0pt}%
\pgfpathmoveto{\pgfqpoint{0.906000in}{1.016982in}}%
\pgfpathlineto{\pgfqpoint{4.180375in}{1.016982in}}%
\pgfusepath{stroke}%
\end{pgfscope}%
\begin{pgfscope}%
\pgfpathrectangle{\pgfqpoint{0.100000in}{0.100000in}}{\pgfqpoint{5.037500in}{2.013333in}}%
\pgfusepath{clip}%
\pgfsetrectcap%
\pgfsetroundjoin%
\pgfsetlinewidth{1.505625pt}%
\definecolor{currentstroke}{rgb}{0.678431,1.000000,0.184314}%
\pgfsetstrokecolor{currentstroke}%
\pgfsetstrokeopacity{0.500000}%
\pgfsetdash{}{0pt}%
\pgfpathmoveto{\pgfqpoint{0.780062in}{0.914485in}}%
\pgfusepath{stroke}%
\end{pgfscope}%
\begin{pgfscope}%
\pgfpathrectangle{\pgfqpoint{0.100000in}{0.100000in}}{\pgfqpoint{5.037500in}{2.013333in}}%
\pgfusepath{clip}%
\pgfsetbuttcap%
\pgfsetroundjoin%
\definecolor{currentfill}{rgb}{0.678431,1.000000,0.184314}%
\pgfsetfillcolor{currentfill}%
\pgfsetfillopacity{0.500000}%
\pgfsetlinewidth{0.250937pt}%
\definecolor{currentstroke}{rgb}{0.000000,0.000000,0.000000}%
\pgfsetstrokecolor{currentstroke}%
\pgfsetstrokeopacity{0.500000}%
\pgfsetdash{}{0pt}%
\pgfsys@defobject{currentmarker}{\pgfqpoint{-0.016667in}{-0.016667in}}{\pgfqpoint{0.016667in}{0.016667in}}{%
\pgfpathmoveto{\pgfqpoint{0.000000in}{-0.016667in}}%
\pgfpathcurveto{\pgfqpoint{0.004420in}{-0.016667in}}{\pgfqpoint{0.008660in}{-0.014911in}}{\pgfqpoint{0.011785in}{-0.011785in}}%
\pgfpathcurveto{\pgfqpoint{0.014911in}{-0.008660in}}{\pgfqpoint{0.016667in}{-0.004420in}}{\pgfqpoint{0.016667in}{0.000000in}}%
\pgfpathcurveto{\pgfqpoint{0.016667in}{0.004420in}}{\pgfqpoint{0.014911in}{0.008660in}}{\pgfqpoint{0.011785in}{0.011785in}}%
\pgfpathcurveto{\pgfqpoint{0.008660in}{0.014911in}}{\pgfqpoint{0.004420in}{0.016667in}}{\pgfqpoint{0.000000in}{0.016667in}}%
\pgfpathcurveto{\pgfqpoint{-0.004420in}{0.016667in}}{\pgfqpoint{-0.008660in}{0.014911in}}{\pgfqpoint{-0.011785in}{0.011785in}}%
\pgfpathcurveto{\pgfqpoint{-0.014911in}{0.008660in}}{\pgfqpoint{-0.016667in}{0.004420in}}{\pgfqpoint{-0.016667in}{0.000000in}}%
\pgfpathcurveto{\pgfqpoint{-0.016667in}{-0.004420in}}{\pgfqpoint{-0.014911in}{-0.008660in}}{\pgfqpoint{-0.011785in}{-0.011785in}}%
\pgfpathcurveto{\pgfqpoint{-0.008660in}{-0.014911in}}{\pgfqpoint{-0.004420in}{-0.016667in}}{\pgfqpoint{0.000000in}{-0.016667in}}%
\pgfpathclose%
\pgfusepath{stroke,fill}%
}%
\begin{pgfscope}%
\pgfsys@transformshift{0.780062in}{0.914485in}%
\pgfsys@useobject{currentmarker}{}%
\end{pgfscope}%
\end{pgfscope}%
\begin{pgfscope}%
\pgfpathrectangle{\pgfqpoint{0.100000in}{0.100000in}}{\pgfqpoint{5.037500in}{2.013333in}}%
\pgfusepath{clip}%
\pgfsetrectcap%
\pgfsetroundjoin%
\pgfsetlinewidth{0.250937pt}%
\definecolor{currentstroke}{rgb}{0.862745,0.862745,0.862745}%
\pgfsetstrokecolor{currentstroke}%
\pgfsetdash{}{0pt}%
\pgfpathmoveto{\pgfqpoint{0.906000in}{0.833952in}}%
\pgfpathlineto{\pgfqpoint{4.180375in}{0.833952in}}%
\pgfusepath{stroke}%
\end{pgfscope}%
\begin{pgfscope}%
\pgfpathrectangle{\pgfqpoint{0.100000in}{0.100000in}}{\pgfqpoint{5.037500in}{2.013333in}}%
\pgfusepath{clip}%
\pgfsetrectcap%
\pgfsetroundjoin%
\pgfsetlinewidth{1.505625pt}%
\definecolor{currentstroke}{rgb}{0.000000,0.000000,1.000000}%
\pgfsetstrokecolor{currentstroke}%
\pgfsetstrokeopacity{0.500000}%
\pgfsetdash{}{0pt}%
\pgfpathmoveto{\pgfqpoint{0.780062in}{0.731455in}}%
\pgfusepath{stroke}%
\end{pgfscope}%
\begin{pgfscope}%
\pgfpathrectangle{\pgfqpoint{0.100000in}{0.100000in}}{\pgfqpoint{5.037500in}{2.013333in}}%
\pgfusepath{clip}%
\pgfsetbuttcap%
\pgfsetroundjoin%
\definecolor{currentfill}{rgb}{0.000000,0.000000,1.000000}%
\pgfsetfillcolor{currentfill}%
\pgfsetfillopacity{0.500000}%
\pgfsetlinewidth{0.250937pt}%
\definecolor{currentstroke}{rgb}{0.000000,0.000000,0.000000}%
\pgfsetstrokecolor{currentstroke}%
\pgfsetstrokeopacity{0.500000}%
\pgfsetdash{}{0pt}%
\pgfsys@defobject{currentmarker}{\pgfqpoint{-0.005556in}{-0.005556in}}{\pgfqpoint{0.005556in}{0.005556in}}{%
\pgfpathmoveto{\pgfqpoint{0.000000in}{-0.005556in}}%
\pgfpathcurveto{\pgfqpoint{0.001473in}{-0.005556in}}{\pgfqpoint{0.002887in}{-0.004970in}}{\pgfqpoint{0.003928in}{-0.003928in}}%
\pgfpathcurveto{\pgfqpoint{0.004970in}{-0.002887in}}{\pgfqpoint{0.005556in}{-0.001473in}}{\pgfqpoint{0.005556in}{0.000000in}}%
\pgfpathcurveto{\pgfqpoint{0.005556in}{0.001473in}}{\pgfqpoint{0.004970in}{0.002887in}}{\pgfqpoint{0.003928in}{0.003928in}}%
\pgfpathcurveto{\pgfqpoint{0.002887in}{0.004970in}}{\pgfqpoint{0.001473in}{0.005556in}}{\pgfqpoint{0.000000in}{0.005556in}}%
\pgfpathcurveto{\pgfqpoint{-0.001473in}{0.005556in}}{\pgfqpoint{-0.002887in}{0.004970in}}{\pgfqpoint{-0.003928in}{0.003928in}}%
\pgfpathcurveto{\pgfqpoint{-0.004970in}{0.002887in}}{\pgfqpoint{-0.005556in}{0.001473in}}{\pgfqpoint{-0.005556in}{0.000000in}}%
\pgfpathcurveto{\pgfqpoint{-0.005556in}{-0.001473in}}{\pgfqpoint{-0.004970in}{-0.002887in}}{\pgfqpoint{-0.003928in}{-0.003928in}}%
\pgfpathcurveto{\pgfqpoint{-0.002887in}{-0.004970in}}{\pgfqpoint{-0.001473in}{-0.005556in}}{\pgfqpoint{0.000000in}{-0.005556in}}%
\pgfpathclose%
\pgfusepath{stroke,fill}%
}%
\begin{pgfscope}%
\pgfsys@transformshift{0.780062in}{0.731455in}%
\pgfsys@useobject{currentmarker}{}%
\end{pgfscope}%
\end{pgfscope}%
\begin{pgfscope}%
\pgfpathrectangle{\pgfqpoint{0.100000in}{0.100000in}}{\pgfqpoint{5.037500in}{2.013333in}}%
\pgfusepath{clip}%
\pgfsetrectcap%
\pgfsetroundjoin%
\pgfsetlinewidth{0.250937pt}%
\definecolor{currentstroke}{rgb}{0.862745,0.862745,0.862745}%
\pgfsetstrokecolor{currentstroke}%
\pgfsetdash{}{0pt}%
\pgfpathmoveto{\pgfqpoint{0.906000in}{0.650921in}}%
\pgfpathlineto{\pgfqpoint{4.180375in}{0.650921in}}%
\pgfusepath{stroke}%
\end{pgfscope}%
\begin{pgfscope}%
\pgfpathrectangle{\pgfqpoint{0.100000in}{0.100000in}}{\pgfqpoint{5.037500in}{2.013333in}}%
\pgfusepath{clip}%
\pgfsetrectcap%
\pgfsetroundjoin%
\pgfsetlinewidth{1.505625pt}%
\definecolor{currentstroke}{rgb}{0.000000,0.000000,1.000000}%
\pgfsetstrokecolor{currentstroke}%
\pgfsetstrokeopacity{0.500000}%
\pgfsetdash{}{0pt}%
\pgfpathmoveto{\pgfqpoint{0.780062in}{0.548424in}}%
\pgfusepath{stroke}%
\end{pgfscope}%
\begin{pgfscope}%
\pgfpathrectangle{\pgfqpoint{0.100000in}{0.100000in}}{\pgfqpoint{5.037500in}{2.013333in}}%
\pgfusepath{clip}%
\pgfsetbuttcap%
\pgfsetroundjoin%
\definecolor{currentfill}{rgb}{0.000000,0.000000,1.000000}%
\pgfsetfillcolor{currentfill}%
\pgfsetfillopacity{0.500000}%
\pgfsetlinewidth{0.250937pt}%
\definecolor{currentstroke}{rgb}{0.000000,0.000000,0.000000}%
\pgfsetstrokecolor{currentstroke}%
\pgfsetstrokeopacity{0.500000}%
\pgfsetdash{}{0pt}%
\pgfsys@defobject{currentmarker}{\pgfqpoint{-0.011111in}{-0.011111in}}{\pgfqpoint{0.011111in}{0.011111in}}{%
\pgfpathmoveto{\pgfqpoint{0.000000in}{-0.011111in}}%
\pgfpathcurveto{\pgfqpoint{0.002947in}{-0.011111in}}{\pgfqpoint{0.005773in}{-0.009940in}}{\pgfqpoint{0.007857in}{-0.007857in}}%
\pgfpathcurveto{\pgfqpoint{0.009940in}{-0.005773in}}{\pgfqpoint{0.011111in}{-0.002947in}}{\pgfqpoint{0.011111in}{0.000000in}}%
\pgfpathcurveto{\pgfqpoint{0.011111in}{0.002947in}}{\pgfqpoint{0.009940in}{0.005773in}}{\pgfqpoint{0.007857in}{0.007857in}}%
\pgfpathcurveto{\pgfqpoint{0.005773in}{0.009940in}}{\pgfqpoint{0.002947in}{0.011111in}}{\pgfqpoint{0.000000in}{0.011111in}}%
\pgfpathcurveto{\pgfqpoint{-0.002947in}{0.011111in}}{\pgfqpoint{-0.005773in}{0.009940in}}{\pgfqpoint{-0.007857in}{0.007857in}}%
\pgfpathcurveto{\pgfqpoint{-0.009940in}{0.005773in}}{\pgfqpoint{-0.011111in}{0.002947in}}{\pgfqpoint{-0.011111in}{0.000000in}}%
\pgfpathcurveto{\pgfqpoint{-0.011111in}{-0.002947in}}{\pgfqpoint{-0.009940in}{-0.005773in}}{\pgfqpoint{-0.007857in}{-0.007857in}}%
\pgfpathcurveto{\pgfqpoint{-0.005773in}{-0.009940in}}{\pgfqpoint{-0.002947in}{-0.011111in}}{\pgfqpoint{0.000000in}{-0.011111in}}%
\pgfpathclose%
\pgfusepath{stroke,fill}%
}%
\begin{pgfscope}%
\pgfsys@transformshift{0.780062in}{0.548424in}%
\pgfsys@useobject{currentmarker}{}%
\end{pgfscope}%
\end{pgfscope}%
\begin{pgfscope}%
\pgfpathrectangle{\pgfqpoint{0.100000in}{0.100000in}}{\pgfqpoint{5.037500in}{2.013333in}}%
\pgfusepath{clip}%
\pgfsetrectcap%
\pgfsetroundjoin%
\pgfsetlinewidth{0.250937pt}%
\definecolor{currentstroke}{rgb}{0.862745,0.862745,0.862745}%
\pgfsetstrokecolor{currentstroke}%
\pgfsetdash{}{0pt}%
\pgfpathmoveto{\pgfqpoint{0.906000in}{0.467891in}}%
\pgfpathlineto{\pgfqpoint{4.180375in}{0.467891in}}%
\pgfusepath{stroke}%
\end{pgfscope}%
\begin{pgfscope}%
\pgfpathrectangle{\pgfqpoint{0.100000in}{0.100000in}}{\pgfqpoint{5.037500in}{2.013333in}}%
\pgfusepath{clip}%
\pgfsetrectcap%
\pgfsetroundjoin%
\pgfsetlinewidth{1.505625pt}%
\definecolor{currentstroke}{rgb}{0.000000,0.000000,1.000000}%
\pgfsetstrokecolor{currentstroke}%
\pgfsetstrokeopacity{0.500000}%
\pgfsetdash{}{0pt}%
\pgfpathmoveto{\pgfqpoint{0.780062in}{0.365394in}}%
\pgfusepath{stroke}%
\end{pgfscope}%
\begin{pgfscope}%
\pgfpathrectangle{\pgfqpoint{0.100000in}{0.100000in}}{\pgfqpoint{5.037500in}{2.013333in}}%
\pgfusepath{clip}%
\pgfsetbuttcap%
\pgfsetroundjoin%
\definecolor{currentfill}{rgb}{0.000000,0.000000,1.000000}%
\pgfsetfillcolor{currentfill}%
\pgfsetfillopacity{0.500000}%
\pgfsetlinewidth{0.250937pt}%
\definecolor{currentstroke}{rgb}{0.000000,0.000000,0.000000}%
\pgfsetstrokecolor{currentstroke}%
\pgfsetstrokeopacity{0.500000}%
\pgfsetdash{}{0pt}%
\pgfsys@defobject{currentmarker}{\pgfqpoint{-0.025000in}{-0.025000in}}{\pgfqpoint{0.025000in}{0.025000in}}{%
\pgfpathmoveto{\pgfqpoint{0.000000in}{-0.025000in}}%
\pgfpathcurveto{\pgfqpoint{0.006630in}{-0.025000in}}{\pgfqpoint{0.012989in}{-0.022366in}}{\pgfqpoint{0.017678in}{-0.017678in}}%
\pgfpathcurveto{\pgfqpoint{0.022366in}{-0.012989in}}{\pgfqpoint{0.025000in}{-0.006630in}}{\pgfqpoint{0.025000in}{0.000000in}}%
\pgfpathcurveto{\pgfqpoint{0.025000in}{0.006630in}}{\pgfqpoint{0.022366in}{0.012989in}}{\pgfqpoint{0.017678in}{0.017678in}}%
\pgfpathcurveto{\pgfqpoint{0.012989in}{0.022366in}}{\pgfqpoint{0.006630in}{0.025000in}}{\pgfqpoint{0.000000in}{0.025000in}}%
\pgfpathcurveto{\pgfqpoint{-0.006630in}{0.025000in}}{\pgfqpoint{-0.012989in}{0.022366in}}{\pgfqpoint{-0.017678in}{0.017678in}}%
\pgfpathcurveto{\pgfqpoint{-0.022366in}{0.012989in}}{\pgfqpoint{-0.025000in}{0.006630in}}{\pgfqpoint{-0.025000in}{0.000000in}}%
\pgfpathcurveto{\pgfqpoint{-0.025000in}{-0.006630in}}{\pgfqpoint{-0.022366in}{-0.012989in}}{\pgfqpoint{-0.017678in}{-0.017678in}}%
\pgfpathcurveto{\pgfqpoint{-0.012989in}{-0.022366in}}{\pgfqpoint{-0.006630in}{-0.025000in}}{\pgfqpoint{0.000000in}{-0.025000in}}%
\pgfpathclose%
\pgfusepath{stroke,fill}%
}%
\begin{pgfscope}%
\pgfsys@transformshift{0.780062in}{0.365394in}%
\pgfsys@useobject{currentmarker}{}%
\end{pgfscope}%
\end{pgfscope}%
\begin{pgfscope}%
\pgfpathrectangle{\pgfqpoint{0.100000in}{0.100000in}}{\pgfqpoint{5.037500in}{2.013333in}}%
\pgfusepath{clip}%
\pgfsetrectcap%
\pgfsetroundjoin%
\pgfsetlinewidth{0.250937pt}%
\definecolor{currentstroke}{rgb}{0.862745,0.862745,0.862745}%
\pgfsetstrokecolor{currentstroke}%
\pgfsetdash{}{0pt}%
\pgfpathmoveto{\pgfqpoint{0.906000in}{0.284861in}}%
\pgfpathlineto{\pgfqpoint{4.180375in}{0.284861in}}%
\pgfusepath{stroke}%
\end{pgfscope}%
\begin{pgfscope}%
\pgfpathrectangle{\pgfqpoint{0.100000in}{0.100000in}}{\pgfqpoint{5.037500in}{2.013333in}}%
\pgfusepath{clip}%
\pgfsetrectcap%
\pgfsetroundjoin%
\pgfsetlinewidth{1.505625pt}%
\definecolor{currentstroke}{rgb}{0.678431,1.000000,0.184314}%
\pgfsetstrokecolor{currentstroke}%
\pgfsetstrokeopacity{0.500000}%
\pgfsetdash{}{0pt}%
\pgfpathmoveto{\pgfqpoint{0.780062in}{0.182364in}}%
\pgfusepath{stroke}%
\end{pgfscope}%
\begin{pgfscope}%
\pgfpathrectangle{\pgfqpoint{0.100000in}{0.100000in}}{\pgfqpoint{5.037500in}{2.013333in}}%
\pgfusepath{clip}%
\pgfsetbuttcap%
\pgfsetroundjoin%
\definecolor{currentfill}{rgb}{0.678431,1.000000,0.184314}%
\pgfsetfillcolor{currentfill}%
\pgfsetfillopacity{0.500000}%
\pgfsetlinewidth{0.250937pt}%
\definecolor{currentstroke}{rgb}{0.000000,0.000000,0.000000}%
\pgfsetstrokecolor{currentstroke}%
\pgfsetstrokeopacity{0.500000}%
\pgfsetdash{}{0pt}%
\pgfsys@defobject{currentmarker}{\pgfqpoint{-0.041667in}{-0.041667in}}{\pgfqpoint{0.041667in}{0.041667in}}{%
\pgfpathmoveto{\pgfqpoint{0.000000in}{-0.041667in}}%
\pgfpathcurveto{\pgfqpoint{0.011050in}{-0.041667in}}{\pgfqpoint{0.021649in}{-0.037276in}}{\pgfqpoint{0.029463in}{-0.029463in}}%
\pgfpathcurveto{\pgfqpoint{0.037276in}{-0.021649in}}{\pgfqpoint{0.041667in}{-0.011050in}}{\pgfqpoint{0.041667in}{0.000000in}}%
\pgfpathcurveto{\pgfqpoint{0.041667in}{0.011050in}}{\pgfqpoint{0.037276in}{0.021649in}}{\pgfqpoint{0.029463in}{0.029463in}}%
\pgfpathcurveto{\pgfqpoint{0.021649in}{0.037276in}}{\pgfqpoint{0.011050in}{0.041667in}}{\pgfqpoint{0.000000in}{0.041667in}}%
\pgfpathcurveto{\pgfqpoint{-0.011050in}{0.041667in}}{\pgfqpoint{-0.021649in}{0.037276in}}{\pgfqpoint{-0.029463in}{0.029463in}}%
\pgfpathcurveto{\pgfqpoint{-0.037276in}{0.021649in}}{\pgfqpoint{-0.041667in}{0.011050in}}{\pgfqpoint{-0.041667in}{0.000000in}}%
\pgfpathcurveto{\pgfqpoint{-0.041667in}{-0.011050in}}{\pgfqpoint{-0.037276in}{-0.021649in}}{\pgfqpoint{-0.029463in}{-0.029463in}}%
\pgfpathcurveto{\pgfqpoint{-0.021649in}{-0.037276in}}{\pgfqpoint{-0.011050in}{-0.041667in}}{\pgfqpoint{0.000000in}{-0.041667in}}%
\pgfpathclose%
\pgfusepath{stroke,fill}%
}%
\begin{pgfscope}%
\pgfsys@transformshift{0.780062in}{0.182364in}%
\pgfsys@useobject{currentmarker}{}%
\end{pgfscope}%
\end{pgfscope}%
\begin{pgfscope}%
\pgfpathrectangle{\pgfqpoint{0.100000in}{0.100000in}}{\pgfqpoint{5.037500in}{2.013333in}}%
\pgfusepath{clip}%
\pgfsetrectcap%
\pgfsetroundjoin%
\pgfsetlinewidth{0.250937pt}%
\definecolor{currentstroke}{rgb}{0.862745,0.862745,0.862745}%
\pgfsetstrokecolor{currentstroke}%
\pgfsetdash{}{0pt}%
\pgfpathmoveto{\pgfqpoint{0.906000in}{0.101830in}}%
\pgfpathlineto{\pgfqpoint{4.180375in}{0.101830in}}%
\pgfusepath{stroke}%
\end{pgfscope}%
\begin{pgfscope}%
\definecolor{textcolor}{rgb}{0.000000,0.000000,0.000000}%
\pgfsetstrokecolor{textcolor}%
\pgfsetfillcolor{textcolor}%
\pgftext[x=2.266125in,y=5.634638in,left,base]{\color{textcolor}\setmainfont{Lato}\rmfamily\fontsize{8.000000}{9.600000}\bfseries\selectfont Legend:}%
\end{pgfscope}%
\begin{pgfscope}%
\definecolor{textcolor}{rgb}{0.000000,0.000000,0.000000}%
\pgfsetstrokecolor{textcolor}%
\pgfsetfillcolor{textcolor}%
\pgftext[x=0.906000in,y=1.972400in,left,base]{\color{textcolor}\setmainfont{Lato}\rmfamily\fontsize{8.000000}{9.600000}\selectfont New York, NY}%
\end{pgfscope}%
\begin{pgfscope}%
\definecolor{textcolor}{rgb}{0.000000,0.000000,0.000000}%
\pgfsetstrokecolor{textcolor}%
\pgfsetfillcolor{textcolor}%
\pgftext[x=1.963875in,y=1.972400in,left,base]{\color{textcolor}\setmainfont{Lato}\rmfamily\fontsize{8.000000}{9.600000}\selectfont 5.5}%
\end{pgfscope}%
\begin{pgfscope}%
\definecolor{textcolor}{rgb}{0.000000,0.000000,0.000000}%
\pgfsetstrokecolor{textcolor}%
\pgfsetfillcolor{textcolor}%
\pgftext[x=2.568375in,y=1.972400in,left,base]{\color{textcolor}\setmainfont{Lato}\rmfamily\fontsize{8.000000}{9.600000}\selectfont 3.1}%
\end{pgfscope}%
\begin{pgfscope}%
\definecolor{textcolor}{rgb}{0.000000,0.000000,0.000000}%
\pgfsetstrokecolor{textcolor}%
\pgfsetfillcolor{textcolor}%
\pgftext[x=0.654125in,y=1.972400in,right,base]{\color{textcolor}\setmainfont{Lato}\rmfamily\fontsize{8.000000}{9.600000}\selectfont +2.4}%
\end{pgfscope}%
\begin{pgfscope}%
\definecolor{textcolor}{rgb}{0.000000,0.000000,0.000000}%
\pgfsetstrokecolor{textcolor}%
\pgfsetfillcolor{textcolor}%
\pgftext[x=3.097312in,y=1.972400in,left,base]{\color{textcolor}\setmainfont{Lato}\rmfamily\fontsize{8.000000}{9.600000}\selectfont 9,626,000}%
\end{pgfscope}%
\begin{pgfscope}%
\definecolor{textcolor}{rgb}{0.000000,0.000000,0.000000}%
\pgfsetstrokecolor{textcolor}%
\pgfsetfillcolor{textcolor}%
\pgftext[x=4.180375in,y=1.972400in,right,base]{\color{textcolor}\setmainfont{Lato}\rmfamily\fontsize{8.000000}{9.600000}\selectfont -3.6}%
\end{pgfscope}%
\begin{pgfscope}%
\definecolor{textcolor}{rgb}{0.000000,0.000000,0.000000}%
\pgfsetstrokecolor{textcolor}%
\pgfsetfillcolor{textcolor}%
\pgftext[x=0.906000in,y=1.789370in,left,base]{\color{textcolor}\setmainfont{Lato}\rmfamily\fontsize{8.000000}{9.600000}\selectfont Los Angeles, CA}%
\end{pgfscope}%
\begin{pgfscope}%
\definecolor{textcolor}{rgb}{0.000000,0.000000,0.000000}%
\pgfsetstrokecolor{textcolor}%
\pgfsetfillcolor{textcolor}%
\pgftext[x=1.963875in,y=1.789370in,left,base]{\color{textcolor}\setmainfont{Lato}\rmfamily\fontsize{8.000000}{9.600000}\selectfont 5.6}%
\end{pgfscope}%
\begin{pgfscope}%
\definecolor{textcolor}{rgb}{0.000000,0.000000,0.000000}%
\pgfsetstrokecolor{textcolor}%
\pgfsetfillcolor{textcolor}%
\pgftext[x=2.568375in,y=1.789370in,left,base]{\color{textcolor}\setmainfont{Lato}\rmfamily\fontsize{8.000000}{9.600000}\selectfont 3.9}%
\end{pgfscope}%
\begin{pgfscope}%
\definecolor{textcolor}{rgb}{0.000000,0.000000,0.000000}%
\pgfsetstrokecolor{textcolor}%
\pgfsetfillcolor{textcolor}%
\pgftext[x=0.654125in,y=1.789370in,right,base]{\color{textcolor}\setmainfont{Lato}\rmfamily\fontsize{8.000000}{9.600000}\selectfont +1.7}%
\end{pgfscope}%
\begin{pgfscope}%
\definecolor{textcolor}{rgb}{0.000000,0.000000,0.000000}%
\pgfsetstrokecolor{textcolor}%
\pgfsetfillcolor{textcolor}%
\pgftext[x=3.097312in,y=1.789370in,left,base]{\color{textcolor}\setmainfont{Lato}\rmfamily\fontsize{8.000000}{9.600000}\selectfont 6,610,100}%
\end{pgfscope}%
\begin{pgfscope}%
\definecolor{textcolor}{rgb}{0.000000,0.000000,0.000000}%
\pgfsetstrokecolor{textcolor}%
\pgfsetfillcolor{textcolor}%
\pgftext[x=4.180375in,y=1.789370in,right,base]{\color{textcolor}\setmainfont{Lato}\rmfamily\fontsize{8.000000}{9.600000}\selectfont -3.2}%
\end{pgfscope}%
\begin{pgfscope}%
\definecolor{textcolor}{rgb}{0.000000,0.000000,0.000000}%
\pgfsetstrokecolor{textcolor}%
\pgfsetfillcolor{textcolor}%
\pgftext[x=0.906000in,y=1.606339in,left,base]{\color{textcolor}\setmainfont{Lato}\rmfamily\fontsize{8.000000}{9.600000}\selectfont Chicago, IL}%
\end{pgfscope}%
\begin{pgfscope}%
\definecolor{textcolor}{rgb}{0.000000,0.000000,0.000000}%
\pgfsetstrokecolor{textcolor}%
\pgfsetfillcolor{textcolor}%
\pgftext[x=1.963875in,y=1.606339in,left,base]{\color{textcolor}\setmainfont{Lato}\rmfamily\fontsize{8.000000}{9.600000}\selectfont 4.3}%
\end{pgfscope}%
\begin{pgfscope}%
\definecolor{textcolor}{rgb}{0.000000,0.000000,0.000000}%
\pgfsetstrokecolor{textcolor}%
\pgfsetfillcolor{textcolor}%
\pgftext[x=2.568375in,y=1.606339in,left,base]{\color{textcolor}\setmainfont{Lato}\rmfamily\fontsize{8.000000}{9.600000}\selectfont 3.2}%
\end{pgfscope}%
\begin{pgfscope}%
\definecolor{textcolor}{rgb}{0.000000,0.000000,0.000000}%
\pgfsetstrokecolor{textcolor}%
\pgfsetfillcolor{textcolor}%
\pgftext[x=0.654125in,y=1.606339in,right,base]{\color{textcolor}\setmainfont{Lato}\rmfamily\fontsize{8.000000}{9.600000}\selectfont +1.1}%
\end{pgfscope}%
\begin{pgfscope}%
\definecolor{textcolor}{rgb}{0.000000,0.000000,0.000000}%
\pgfsetstrokecolor{textcolor}%
\pgfsetfillcolor{textcolor}%
\pgftext[x=3.097312in,y=1.606339in,left,base]{\color{textcolor}\setmainfont{Lato}\rmfamily\fontsize{8.000000}{9.600000}\selectfont 4,832,400}%
\end{pgfscope}%
\begin{pgfscope}%
\definecolor{textcolor}{rgb}{0.000000,0.000000,0.000000}%
\pgfsetstrokecolor{textcolor}%
\pgfsetfillcolor{textcolor}%
\pgftext[x=4.180375in,y=1.606339in,right,base]{\color{textcolor}\setmainfont{Lato}\rmfamily\fontsize{8.000000}{9.600000}\selectfont 0.1}%
\end{pgfscope}%
\begin{pgfscope}%
\definecolor{textcolor}{rgb}{0.000000,0.000000,0.000000}%
\pgfsetstrokecolor{textcolor}%
\pgfsetfillcolor{textcolor}%
\pgftext[x=0.906000in,y=1.423309in,left,base]{\color{textcolor}\setmainfont{Lato}\rmfamily\fontsize{8.000000}{9.600000}\selectfont Dallas, TX}%
\end{pgfscope}%
\begin{pgfscope}%
\definecolor{textcolor}{rgb}{0.000000,0.000000,0.000000}%
\pgfsetstrokecolor{textcolor}%
\pgfsetfillcolor{textcolor}%
\pgftext[x=1.963875in,y=1.423309in,left,base]{\color{textcolor}\setmainfont{Lato}\rmfamily\fontsize{8.000000}{9.600000}\selectfont 3.6}%
\end{pgfscope}%
\begin{pgfscope}%
\definecolor{textcolor}{rgb}{0.000000,0.000000,0.000000}%
\pgfsetstrokecolor{textcolor}%
\pgfsetfillcolor{textcolor}%
\pgftext[x=2.568375in,y=1.423309in,left,base]{\color{textcolor}\setmainfont{Lato}\rmfamily\fontsize{8.000000}{9.600000}\selectfont 2.9}%
\end{pgfscope}%
\begin{pgfscope}%
\definecolor{textcolor}{rgb}{0.000000,0.000000,0.000000}%
\pgfsetstrokecolor{textcolor}%
\pgfsetfillcolor{textcolor}%
\pgftext[x=0.654125in,y=1.423309in,right,base]{\color{textcolor}\setmainfont{Lato}\rmfamily\fontsize{8.000000}{9.600000}\selectfont +0.7}%
\end{pgfscope}%
\begin{pgfscope}%
\definecolor{textcolor}{rgb}{0.000000,0.000000,0.000000}%
\pgfsetstrokecolor{textcolor}%
\pgfsetfillcolor{textcolor}%
\pgftext[x=3.097312in,y=1.423309in,left,base]{\color{textcolor}\setmainfont{Lato}\rmfamily\fontsize{8.000000}{9.600000}\selectfont 4,182,200}%
\end{pgfscope}%
\begin{pgfscope}%
\definecolor{textcolor}{rgb}{0.000000,0.000000,0.000000}%
\pgfsetstrokecolor{textcolor}%
\pgfsetfillcolor{textcolor}%
\pgftext[x=4.180375in,y=1.423309in,right,base]{\color{textcolor}\setmainfont{Lato}\rmfamily\fontsize{8.000000}{9.600000}\selectfont 3.4}%
\end{pgfscope}%
\begin{pgfscope}%
\definecolor{textcolor}{rgb}{0.000000,0.000000,0.000000}%
\pgfsetstrokecolor{textcolor}%
\pgfsetfillcolor{textcolor}%
\pgftext[x=0.906000in,y=1.240279in,left,base]{\color{textcolor}\setmainfont{Lato}\rmfamily\fontsize{8.000000}{9.600000}\selectfont Houston, TX}%
\end{pgfscope}%
\begin{pgfscope}%
\definecolor{textcolor}{rgb}{0.000000,0.000000,0.000000}%
\pgfsetstrokecolor{textcolor}%
\pgfsetfillcolor{textcolor}%
\pgftext[x=1.963875in,y=1.240279in,left,base]{\color{textcolor}\setmainfont{Lato}\rmfamily\fontsize{8.000000}{9.600000}\selectfont 4.8}%
\end{pgfscope}%
\begin{pgfscope}%
\definecolor{textcolor}{rgb}{0.000000,0.000000,0.000000}%
\pgfsetstrokecolor{textcolor}%
\pgfsetfillcolor{textcolor}%
\pgftext[x=2.568375in,y=1.240279in,left,base]{\color{textcolor}\setmainfont{Lato}\rmfamily\fontsize{8.000000}{9.600000}\selectfont 3.6}%
\end{pgfscope}%
\begin{pgfscope}%
\definecolor{textcolor}{rgb}{0.000000,0.000000,0.000000}%
\pgfsetstrokecolor{textcolor}%
\pgfsetfillcolor{textcolor}%
\pgftext[x=0.654125in,y=1.240279in,right,base]{\color{textcolor}\setmainfont{Lato}\rmfamily\fontsize{8.000000}{9.600000}\selectfont +1.2}%
\end{pgfscope}%
\begin{pgfscope}%
\definecolor{textcolor}{rgb}{0.000000,0.000000,0.000000}%
\pgfsetstrokecolor{textcolor}%
\pgfsetfillcolor{textcolor}%
\pgftext[x=3.097312in,y=1.240279in,left,base]{\color{textcolor}\setmainfont{Lato}\rmfamily\fontsize{8.000000}{9.600000}\selectfont 3,482,700}%
\end{pgfscope}%
\begin{pgfscope}%
\definecolor{textcolor}{rgb}{0.000000,0.000000,0.000000}%
\pgfsetstrokecolor{textcolor}%
\pgfsetfillcolor{textcolor}%
\pgftext[x=4.180375in,y=1.240279in,right,base]{\color{textcolor}\setmainfont{Lato}\rmfamily\fontsize{8.000000}{9.600000}\selectfont 0.6}%
\end{pgfscope}%
\begin{pgfscope}%
\definecolor{textcolor}{rgb}{0.000000,0.000000,0.000000}%
\pgfsetstrokecolor{textcolor}%
\pgfsetfillcolor{textcolor}%
\pgftext[x=0.906000in,y=1.057248in,left,base]{\color{textcolor}\setmainfont{Lato}\rmfamily\fontsize{8.000000}{9.600000}\selectfont Washington, DC}%
\end{pgfscope}%
\begin{pgfscope}%
\definecolor{textcolor}{rgb}{0.000000,0.000000,0.000000}%
\pgfsetstrokecolor{textcolor}%
\pgfsetfillcolor{textcolor}%
\pgftext[x=1.963875in,y=1.057248in,left,base]{\color{textcolor}\setmainfont{Lato}\rmfamily\fontsize{8.000000}{9.600000}\selectfont 3.3}%
\end{pgfscope}%
\begin{pgfscope}%
\definecolor{textcolor}{rgb}{0.000000,0.000000,0.000000}%
\pgfsetstrokecolor{textcolor}%
\pgfsetfillcolor{textcolor}%
\pgftext[x=2.568375in,y=1.057248in,left,base]{\color{textcolor}\setmainfont{Lato}\rmfamily\fontsize{8.000000}{9.600000}\selectfont 2.5}%
\end{pgfscope}%
\begin{pgfscope}%
\definecolor{textcolor}{rgb}{0.000000,0.000000,0.000000}%
\pgfsetstrokecolor{textcolor}%
\pgfsetfillcolor{textcolor}%
\pgftext[x=0.654125in,y=1.057248in,right,base]{\color{textcolor}\setmainfont{Lato}\rmfamily\fontsize{8.000000}{9.600000}\selectfont +0.8}%
\end{pgfscope}%
\begin{pgfscope}%
\definecolor{textcolor}{rgb}{0.000000,0.000000,0.000000}%
\pgfsetstrokecolor{textcolor}%
\pgfsetfillcolor{textcolor}%
\pgftext[x=3.097312in,y=1.057248in,left,base]{\color{textcolor}\setmainfont{Lato}\rmfamily\fontsize{8.000000}{9.600000}\selectfont 3,351,300}%
\end{pgfscope}%
\begin{pgfscope}%
\definecolor{textcolor}{rgb}{0.000000,0.000000,0.000000}%
\pgfsetstrokecolor{textcolor}%
\pgfsetfillcolor{textcolor}%
\pgftext[x=4.180375in,y=1.057248in,right,base]{\color{textcolor}\setmainfont{Lato}\rmfamily\fontsize{8.000000}{9.600000}\selectfont -4.3}%
\end{pgfscope}%
\begin{pgfscope}%
\definecolor{textcolor}{rgb}{0.000000,0.000000,0.000000}%
\pgfsetstrokecolor{textcolor}%
\pgfsetfillcolor{textcolor}%
\pgftext[x=0.906000in,y=0.874218in,left,base]{\color{textcolor}\setmainfont{Lato}\rmfamily\fontsize{8.000000}{9.600000}\selectfont Atlanta, GA}%
\end{pgfscope}%
\begin{pgfscope}%
\definecolor{textcolor}{rgb}{0.000000,0.000000,0.000000}%
\pgfsetstrokecolor{textcolor}%
\pgfsetfillcolor{textcolor}%
\pgftext[x=1.963875in,y=0.874218in,left,base]{\color{textcolor}\setmainfont{Lato}\rmfamily\fontsize{8.000000}{9.600000}\selectfont 2.3}%
\end{pgfscope}%
\begin{pgfscope}%
\definecolor{textcolor}{rgb}{0.000000,0.000000,0.000000}%
\pgfsetstrokecolor{textcolor}%
\pgfsetfillcolor{textcolor}%
\pgftext[x=2.568375in,y=0.874218in,left,base]{\color{textcolor}\setmainfont{Lato}\rmfamily\fontsize{8.000000}{9.600000}\selectfont 2.9}%
\end{pgfscope}%
\begin{pgfscope}%
\definecolor{textcolor}{rgb}{0.000000,0.000000,0.000000}%
\pgfsetstrokecolor{textcolor}%
\pgfsetfillcolor{textcolor}%
\pgftext[x=0.654125in,y=0.874218in,right,base]{\color{textcolor}\setmainfont{Lato}\rmfamily\fontsize{8.000000}{9.600000}\selectfont -0.6}%
\end{pgfscope}%
\begin{pgfscope}%
\definecolor{textcolor}{rgb}{0.000000,0.000000,0.000000}%
\pgfsetstrokecolor{textcolor}%
\pgfsetfillcolor{textcolor}%
\pgftext[x=3.097312in,y=0.874218in,left,base]{\color{textcolor}\setmainfont{Lato}\rmfamily\fontsize{8.000000}{9.600000}\selectfont 3,157,300}%
\end{pgfscope}%
\begin{pgfscope}%
\definecolor{textcolor}{rgb}{0.000000,0.000000,0.000000}%
\pgfsetstrokecolor{textcolor}%
\pgfsetfillcolor{textcolor}%
\pgftext[x=4.180375in,y=0.874218in,right,base]{\color{textcolor}\setmainfont{Lato}\rmfamily\fontsize{8.000000}{9.600000}\selectfont 0.9}%
\end{pgfscope}%
\begin{pgfscope}%
\definecolor{textcolor}{rgb}{0.000000,0.000000,0.000000}%
\pgfsetstrokecolor{textcolor}%
\pgfsetfillcolor{textcolor}%
\pgftext[x=0.906000in,y=0.691188in,left,base]{\color{textcolor}\setmainfont{Lato}\rmfamily\fontsize{8.000000}{9.600000}\selectfont Miami, FL}%
\end{pgfscope}%
\begin{pgfscope}%
\definecolor{textcolor}{rgb}{0.000000,0.000000,0.000000}%
\pgfsetstrokecolor{textcolor}%
\pgfsetfillcolor{textcolor}%
\pgftext[x=1.963875in,y=0.691188in,left,base]{\color{textcolor}\setmainfont{Lato}\rmfamily\fontsize{8.000000}{9.600000}\selectfont 2.6}%
\end{pgfscope}%
\begin{pgfscope}%
\definecolor{textcolor}{rgb}{0.000000,0.000000,0.000000}%
\pgfsetstrokecolor{textcolor}%
\pgfsetfillcolor{textcolor}%
\pgftext[x=2.568375in,y=0.691188in,left,base]{\color{textcolor}\setmainfont{Lato}\rmfamily\fontsize{8.000000}{9.600000}\selectfont 2.4}%
\end{pgfscope}%
\begin{pgfscope}%
\definecolor{textcolor}{rgb}{0.000000,0.000000,0.000000}%
\pgfsetstrokecolor{textcolor}%
\pgfsetfillcolor{textcolor}%
\pgftext[x=0.654125in,y=0.691188in,right,base]{\color{textcolor}\setmainfont{Lato}\rmfamily\fontsize{8.000000}{9.600000}\selectfont unch.}%
\end{pgfscope}%
\begin{pgfscope}%
\definecolor{textcolor}{rgb}{0.000000,0.000000,0.000000}%
\pgfsetstrokecolor{textcolor}%
\pgfsetfillcolor{textcolor}%
\pgftext[x=3.097312in,y=0.691188in,left,base]{\color{textcolor}\setmainfont{Lato}\rmfamily\fontsize{8.000000}{9.600000}\selectfont 3,132,800}%
\end{pgfscope}%
\begin{pgfscope}%
\definecolor{textcolor}{rgb}{0.000000,0.000000,0.000000}%
\pgfsetstrokecolor{textcolor}%
\pgfsetfillcolor{textcolor}%
\pgftext[x=4.180375in,y=0.691188in,right,base]{\color{textcolor}\setmainfont{Lato}\rmfamily\fontsize{8.000000}{9.600000}\selectfont -0.8}%
\end{pgfscope}%
\begin{pgfscope}%
\definecolor{textcolor}{rgb}{0.000000,0.000000,0.000000}%
\pgfsetstrokecolor{textcolor}%
\pgfsetfillcolor{textcolor}%
\pgftext[x=0.906000in,y=0.508158in,left,base]{\color{textcolor}\setmainfont{Lato}\rmfamily\fontsize{8.000000}{9.600000}\selectfont Philadelphia, PA}%
\end{pgfscope}%
\begin{pgfscope}%
\definecolor{textcolor}{rgb}{0.000000,0.000000,0.000000}%
\pgfsetstrokecolor{textcolor}%
\pgfsetfillcolor{textcolor}%
\pgftext[x=1.963875in,y=0.508158in,left,base]{\color{textcolor}\setmainfont{Lato}\rmfamily\fontsize{8.000000}{9.600000}\selectfont 4.2}%
\end{pgfscope}%
\begin{pgfscope}%
\definecolor{textcolor}{rgb}{0.000000,0.000000,0.000000}%
\pgfsetstrokecolor{textcolor}%
\pgfsetfillcolor{textcolor}%
\pgftext[x=2.568375in,y=0.508158in,left,base]{\color{textcolor}\setmainfont{Lato}\rmfamily\fontsize{8.000000}{9.600000}\selectfont 3.8}%
\end{pgfscope}%
\begin{pgfscope}%
\definecolor{textcolor}{rgb}{0.000000,0.000000,0.000000}%
\pgfsetstrokecolor{textcolor}%
\pgfsetfillcolor{textcolor}%
\pgftext[x=0.654125in,y=0.508158in,right,base]{\color{textcolor}\setmainfont{Lato}\rmfamily\fontsize{8.000000}{9.600000}\selectfont +0.4}%
\end{pgfscope}%
\begin{pgfscope}%
\definecolor{textcolor}{rgb}{0.000000,0.000000,0.000000}%
\pgfsetstrokecolor{textcolor}%
\pgfsetfillcolor{textcolor}%
\pgftext[x=3.097312in,y=0.508158in,left,base]{\color{textcolor}\setmainfont{Lato}\rmfamily\fontsize{8.000000}{9.600000}\selectfont 3,036,500}%
\end{pgfscope}%
\begin{pgfscope}%
\definecolor{textcolor}{rgb}{0.000000,0.000000,0.000000}%
\pgfsetstrokecolor{textcolor}%
\pgfsetfillcolor{textcolor}%
\pgftext[x=4.180375in,y=0.508158in,right,base]{\color{textcolor}\setmainfont{Lato}\rmfamily\fontsize{8.000000}{9.600000}\selectfont -4.1}%
\end{pgfscope}%
\begin{pgfscope}%
\definecolor{textcolor}{rgb}{0.000000,0.000000,0.000000}%
\pgfsetstrokecolor{textcolor}%
\pgfsetfillcolor{textcolor}%
\pgftext[x=0.906000in,y=0.325127in,left,base]{\color{textcolor}\setmainfont{Lato}\rmfamily\fontsize{8.000000}{9.600000}\selectfont Boston, MA}%
\end{pgfscope}%
\begin{pgfscope}%
\definecolor{textcolor}{rgb}{0.000000,0.000000,0.000000}%
\pgfsetstrokecolor{textcolor}%
\pgfsetfillcolor{textcolor}%
\pgftext[x=1.963875in,y=0.325127in,left,base]{\color{textcolor}\setmainfont{Lato}\rmfamily\fontsize{8.000000}{9.600000}\selectfont 3.1}%
\end{pgfscope}%
\begin{pgfscope}%
\definecolor{textcolor}{rgb}{0.000000,0.000000,0.000000}%
\pgfsetstrokecolor{textcolor}%
\pgfsetfillcolor{textcolor}%
\pgftext[x=2.568375in,y=0.325127in,left,base]{\color{textcolor}\setmainfont{Lato}\rmfamily\fontsize{8.000000}{9.600000}\selectfont 2.2}%
\end{pgfscope}%
\begin{pgfscope}%
\definecolor{textcolor}{rgb}{0.000000,0.000000,0.000000}%
\pgfsetstrokecolor{textcolor}%
\pgfsetfillcolor{textcolor}%
\pgftext[x=0.654125in,y=0.325127in,right,base]{\color{textcolor}\setmainfont{Lato}\rmfamily\fontsize{8.000000}{9.600000}\selectfont +0.9}%
\end{pgfscope}%
\begin{pgfscope}%
\definecolor{textcolor}{rgb}{0.000000,0.000000,0.000000}%
\pgfsetstrokecolor{textcolor}%
\pgfsetfillcolor{textcolor}%
\pgftext[x=3.097312in,y=0.325127in,left,base]{\color{textcolor}\setmainfont{Lato}\rmfamily\fontsize{8.000000}{9.600000}\selectfont 2,749,000}%
\end{pgfscope}%
\begin{pgfscope}%
\definecolor{textcolor}{rgb}{0.000000,0.000000,0.000000}%
\pgfsetstrokecolor{textcolor}%
\pgfsetfillcolor{textcolor}%
\pgftext[x=4.180375in,y=0.325127in,right,base]{\color{textcolor}\setmainfont{Lato}\rmfamily\fontsize{8.000000}{9.600000}\selectfont -2.2}%
\end{pgfscope}%
\begin{pgfscope}%
\definecolor{textcolor}{rgb}{0.000000,0.000000,0.000000}%
\pgfsetstrokecolor{textcolor}%
\pgfsetfillcolor{textcolor}%
\pgftext[x=0.906000in,y=0.142097in,left,base]{\color{textcolor}\setmainfont{Lato}\rmfamily\fontsize{8.000000}{9.600000}\selectfont Phoenix, AZ}%
\end{pgfscope}%
\begin{pgfscope}%
\definecolor{textcolor}{rgb}{0.000000,0.000000,0.000000}%
\pgfsetstrokecolor{textcolor}%
\pgfsetfillcolor{textcolor}%
\pgftext[x=1.963875in,y=0.142097in,left,base]{\color{textcolor}\setmainfont{Lato}\rmfamily\fontsize{8.000000}{9.600000}\selectfont 2.4}%
\end{pgfscope}%
\begin{pgfscope}%
\definecolor{textcolor}{rgb}{0.000000,0.000000,0.000000}%
\pgfsetstrokecolor{textcolor}%
\pgfsetfillcolor{textcolor}%
\pgftext[x=2.568375in,y=0.142097in,left,base]{\color{textcolor}\setmainfont{Lato}\rmfamily\fontsize{8.000000}{9.600000}\selectfont 3.9}%
\end{pgfscope}%
\begin{pgfscope}%
\definecolor{textcolor}{rgb}{0.000000,0.000000,0.000000}%
\pgfsetstrokecolor{textcolor}%
\pgfsetfillcolor{textcolor}%
\pgftext[x=0.654125in,y=0.142097in,right,base]{\color{textcolor}\setmainfont{Lato}\rmfamily\fontsize{8.000000}{9.600000}\selectfont -1.5}%
\end{pgfscope}%
\begin{pgfscope}%
\definecolor{textcolor}{rgb}{0.000000,0.000000,0.000000}%
\pgfsetstrokecolor{textcolor}%
\pgfsetfillcolor{textcolor}%
\pgftext[x=3.097312in,y=0.142097in,left,base]{\color{textcolor}\setmainfont{Lato}\rmfamily\fontsize{8.000000}{9.600000}\selectfont 2,601,900}%
\end{pgfscope}%
\begin{pgfscope}%
\definecolor{textcolor}{rgb}{0.000000,0.000000,0.000000}%
\pgfsetstrokecolor{textcolor}%
\pgfsetfillcolor{textcolor}%
\pgftext[x=4.180375in,y=0.142097in,right,base]{\color{textcolor}\setmainfont{Lato}\rmfamily\fontsize{8.000000}{9.600000}\selectfont 3.1}%
\end{pgfscope}%
\begin{pgfscope}%
\definecolor{textcolor}{rgb}{0.000000,0.000000,0.000000}%
\pgfsetstrokecolor{textcolor}%
\pgfsetfillcolor{textcolor}%
\pgftext[x=0.283182in,y=2.351273in,left,base]{\color{textcolor}\setmainfont{Lato}\rmfamily\fontsize{10.000000}{12.000000}\bfseries\selectfont Largest MSAs:}%
\end{pgfscope}%
\begin{pgfscope}%
\definecolor{textcolor}{rgb}{0.411765,0.411765,0.411765}%
\pgfsetstrokecolor{textcolor}%
\pgfsetfillcolor{textcolor}%
\pgftext[x=0.924318in,y=2.149939in,left,base]{\color{textcolor}\setmainfont{Lato}\rmfamily\fontsize{9.000000}{10.800000}\selectfont Core City \ \ \ \ \ \ \ \ Dec 21  \ \ \ \ \ \  Dec 19 \ \ \ \ \ \ Labor Force \ \ \ \ \ \ Pct Ch*}%
\end{pgfscope}%
\end{pgfpicture}%
\makeatother%
\endgroup%

\vspace{-3mm}

\footnotesize{Source: Bureau of Labor Statistics; \ Full Table: \tbllink{msa_unemp_rate.csv} \\ \**Pct Ch is percent change in labor force from \input{text/unemp_map_date.txt}}
\end{minipage}
\end{document}
